\documentclass{report}
\usepackage[utf8]{inputenc}
\usepackage[spanish,english]{babel}
\usepackage{amsmath,amssymb,amsthm,amsfonts}
\usepackage{geometry}
\usepackage{hyperref}
\usepackage{fancyhdr}
\usepackage{titlesec}
\usepackage{listings}
\usepackage{graphicx,graphics}
\usepackage{multicol}
\usepackage{multirow}
\usepackage{color}
\usepackage{float} 
\usepackage{subfig}
\usepackage[figuresright]{rotating}
\usepackage{enumerate}
\usepackage{anysize} 
\usepackage{url}

\title{Temas Selectos de Probabilidad}
\author{Carlos E. Martínez-Rodríguez}
\date{Julio 2024}

\geometry{
  a4paper,
  left=25mm,
  right=25mm,
  top=30mm,
  bottom=30mm,
}

% Configuración de encabezados y pies de página
\pagestyle{fancy}
\fancyhf{}
\fancyhead[L]{\leftmark}
\fancyfoot[C]{\thepage}
\fancyfoot[R]{\rightmark}
\fancyfoot[L]{Carlos E. Martínez-Rodríguez}

% Redefinir el nombre de los capítulos
\titleformat{\chapter}[display]
  {\normalfont\huge\bfseries}
  {CAPÍTULO \thechapter}
  {20pt}
  {\Huge}

% Configuración para la inclusión de código fuente en R
\lstset{
    language=R,
    basicstyle=\ttfamily\small,
    numbers=left,
    numberstyle=\tiny,
    stepnumber=1,
    numbersep=5pt,
    showspaces=false,
    showstringspaces=false,
    showtabs=false,
    frame=single,
    tabsize=2,
    captionpos=b,
    breaklines=true,
    breakatwhitespace=false,
    title=\lstname
}

% Definiciones de nuevos entornos
\newtheorem{Algthm}{Algoritmo}[section]
\newtheorem{Def}{Definición}[section]
\newtheorem{Ejem}{Ejemplo}[section]
\newtheorem{Teo}{Teorema}[section]
\newtheorem{Dem}{Demostración}[section]
\newtheorem{Note}{Nota}[section]
\newtheorem{Sol}{Solución}[section]
\newtheorem{Prop}{Proposición}[section]
\newtheorem{Cor}{Corolario}[section]
\newtheorem{Col}{Corolario}[section]
\newtheorem{Coro}{Corolario}[section]
\newtheorem{Lemma}{Lema}[section]
\newtheorem{Lem}{Lema}[section]
\newtheorem{Lema}{Lema}[section]
\newtheorem{Sup}{Supuestos}[section]
\newtheorem{Assumption}{Supuestos}[section]
\newtheorem{Remark}{Observación}[section]
\newtheorem{Condition}{Condición}[section]
\newtheorem{Theorem}{Teorema}[section]
%\newtheorem{proof}{Demostración}[section]
\newtheorem{Corollary}{Corolario}[section]
\newtheorem{Ejemplo}{Ejemplo}[section]
\newtheorem{Example}{Ejemplo}[section]
\newtheorem{Obs}{Observación}[section]

% Nuevos comandos
\def\RR{\mathbb{R}}
\def\ZZ{\mathbb{Z}}
\newcommand{\nat}{\mathbb{N}}
\newcommand{\ent}{\mathbb{Z}}
\newcommand{\rea}{\mathbb{R}}
\newcommand{\Eb}{\mathbf{E}}
\newcommand{\esp}{\mathbb{E}}
\newcommand{\prob}{\mathbb{P}}
\newcommand{\indora}{\mbox{$1$\hspace{-0.8ex}$1$}}
\newcommand{\ER}{\left(E,\mathcal{E}\right)}
\newcommand{\KM}{\left(P_{s,t}\right)}
\newcommand{\Xt}{\left(X_{t}\right)_{t\in I}}
\newcommand{\PE}{\left(X_{t}\right)_{t\in I}}
\newcommand{\SG}{\left(P_{t}\right)}
\newcommand{\CM}{\mathbf{P}^{x}}
\newcommand\mypar{\par\vspace{\baselineskip}}

%______________________________________________________________________

\begin{document}

\maketitle

\tableofcontents

%==<>====<>====<>====<>====<>====<>====<>====<>====<>====<>====
\part{TERCERA PARTE: TEMAS SELECTOS}
%==<>====<>====<>====<>====<>====<>====<>====<>====<>====<>====

%\chapter{Probabilidad Avanzada}
%\input{ProbabilidadAvanzada}
%---- ESTE CAPITULO YA ESTA CORREGIDO ---
%-- INCLUYE VARIAS SECCIONES----
%\documentclass{article}
%_-_-_-_-_-_-_-_-_-_-_-_-_-_-_-_-_-_-_-_-_-_-_-_-_-_-_-
% PAQUETES A UTILIZAR
%_-_-_-_-_-_-_-_-_-_-_-_-_-_-_-_-_-_-_-_-_-_-_-_-_-_-_-
\usepackage[utf8]{inputenc}
\usepackage[spanish,english]{babel}
\usepackage{amsmath,amssymb,amsthm,amsfonts}
\usepackage{geometry}
\usepackage{hyperref}
\usepackage{fancyhdr}
\usepackage{titlesec}
\usepackage{listings}
\usepackage{graphicx,graphics}
\usepackage{multicol}
\usepackage{multirow}
\usepackage{color}
\usepackage{float} 
\usepackage{subfig}
\usepackage[figuresright]{rotating}
\usepackage{enumerate}
\usepackage{anysize} 
\usepackage{url}
\usepackage{imakeidx}
%_-_-_-_-_-_-_-_-_-_-_-_-_-_-_-_-_-_-_-_-_-_-_-_-_-_-_-
% TITULO DEL DOCUMENTO
%_-_-_-_-_-_-_-_-_-_-_-_-_-_-_-_-_-_-_-_-_-_-_-_-_-_-_-
\title{Notas sobre Sistemas de Espera\\
\small{Notes about Queueting Systems}}
\author{Carlos E. Martínez-Rodríguez}
\date{}
%_-_-_-_-_-_-_-_-_-_-_-_-_-_-_-_-_-_-_-_-_-_-_-_-_-_-_-
% MODIFICACION DE LOS MARGENES
%_-_-_-_-_-_-_-_-_-_-_-_-_-_-_-_-_-_-_-_-_-_-_-_-_-_-_-
\geometry{
  a4paper,
  left=15mm,
  right=15mm,
  left=14mm,
  right=14mm,
  top=30mm,
  bottom=30mm,
}
%_-_-_-_-_-_-_-_-_-_-_-_-_-_-_-_-_-_-_-_-_-_-_-_-_-_-_-
% CONFIGURACION DE ENCABEZADOS Y PIES DE PAG
%_-_-_-_-_-_-_-_-_-_-_-_-_-_-_-_-_-_-_-_-_-_-_-_-_-_-_-
\pagestyle{fancy}
\fancyhf{}
\fancyhead[L]{\nouppercase{\leftmark}} % Sección en el encabezado izquierdo
\fancyfoot[C]{\thepage} % Número de página centrado en el pie
\fancyfoot[L]{\tiny Carlos E. Martínez-Rodríguez} % Autor en el pie izquierdo
\fancyfoot[R]{\tiny \nouppercase{\rightmark}} % Subsection actual en el pie derecho

%_-_-_-_-_-_-_-_-_-_-_-_-_-_-_-_-_-_-_-_-_-_-_-_-_-_-_-
%_-_-_-_-_-_-_-_-_-_-_-_-_-_-_-_-_-_-_-_-_-_-_-_-_-_-_-
% Definiciones de nuevos entornos
%_-_-_-_-_-_-_-_-_-_-_-_-_-_-_-_-_-_-_-_-_-_-_-_-_-_-_-
\newtheorem{Def}{Definición}[section]
\newtheorem{Ejem}{Ejemplo}[section]
\newtheorem{Teo}{Teorema}[section]
\newtheorem{Note}{Nota}[section]
\newtheorem{Prop}{Proposición}[section]
\newtheorem{Cor}{Corolario}[section]
\newtheorem{Coro}{Corolario}[section]
\newtheorem{Lema}{Lema}[section]
\newtheorem{Lemma}{Lema}[section]
\newtheorem{Lem}{Lema}[section]
\newtheorem{Sup}{Supuestos}[section]
\newtheorem{Obs}{Observación}[section]
%_-_-_-_-_-_-_-_-_-_-_-_-_-_-_-_-_-_-_-_-_-_-_-_-_-_-_-
%NUEVOS COMANDOS
%_-_-_-_-_-_-_-_-_-_-_-_-_-_-_-_-_-_-_-_-_-_-_-_-_-_-_-
\newcommand{\nat}{\mathbb{N}}
\newcommand{\ent}{\mathbb{Z}}
\newcommand{\rea}{\mathbb{R}}
\newcommand{\Eb}{\mathbf{E}}
\newcommand{\esp}{\mathbb{E}}
\newcommand{\prob}{\mathbb{P}}
\newcommand{\indora}{\mbox{$1$\hspace{-0.8ex}$1$}}
\newcommand{\ER}{\left(E,\mathcal{E}\right)}
\newcommand{\KM}{\left(P_{s,t}\right)}
\newcommand{\PE}{\left(X_{t}\right)_{t\in I}}
\newcommand{\CM}{\mathbf{P}^{x}}
\renewcommand{\abstractname}{Resumen}
\numberwithin{equation}{section}
\newcommand{\acmclass}[1]{\noindent\textbf{ACM Class:} #1\\}
\newcommand{\mscclass}[1]{\noindent\textbf{MSC Class:} #1\\}
%_-_-_-_-_-_-_-_-_-_-_-_-_-_-_-_-_-_-_-_-_-_-_-_-_-_-_-
\makeindex

%_-_-_-_-_-_-_-_-_-_-_-_-_-_-_-_-_-_-_-_-_-_-_-_-_-_-_-
\begin{document}
%_-_-_-_-_-_-_-_-_-_-_-_-_-_-_-_-_-_-_-_-_-_-_-_-_-_-_-
\maketitle

%<<>><<>><<>><<>><<>><<>><<>><<>><<>><<>><<>><<>>
\begin{abstract}
%<<>><<>><<>><<>><<>><<>><<>><<>><<>><<>><<>><<>>
%_-_-_-_-_-_-_-_-_-_-_-_-_-_-_-_-_-_-_-_-_-_-_-_-_-_-_-

\end{abstract}

\begin{otherlanguage}{english}
\renewcommand{\abstractname}{Abstract} % Cambia "Resumen" a "Abstract"
\begin{abstract}

\end{abstract}
\end{otherlanguage}
%<<>><<>><<>><<>><<>><<>><<>><<>><<>><<>><<>><<>>

\tableofcontents
%\newpage

%<====>====<><====>====<><====>====<><====>====<><====>
%\part{Introducci\'on a Procesos Regenerativos}
%<====>====<><====>====<><====>====<><====>====<><====>
%_-_-_-_-_-_-_-_-_-_-_-_-_-_-_-_-_-_-_-_-_-_-_-_-_-_-_-
\section*{Introducción}
%_-_-_-_-_-_-_-_-_-_-_-_-_-_-_-_-_-_-_-_-_-_-_-_-_-_-_-



\begin{otherlanguage}{english}
\renewcommand{\abstractname}{Abstract} % Cambia "Resumen" a "Abstract"
\section*{Introduction}


\end{otherlanguage}

%______________________________________________________________________
\section{Ec

\begin{Teo}[Teorema de Continuidad]
Sup\'ongase que $\left\{X_{n},n=1,2,3,\ldots\right\}$ son variables aleatorias finitas, no negativas con valores enteros tales que $P\left(X_{n}=k\right)=p_{k}^{(n)}$, para $n=1,2,3,\ldots$, $k=0,1,2,\ldots$, con $\sum_{k=0}^{\infty}p_{k}^{(n)}=1$, para $n=1,2,3,\ldots$. Sea $g_{n}$ la PGF para la variable aleatoria $X_{n}$. Entonces existe una sucesi\'on $\left\{p_{k}\right\}$ tal que \begin{eqnarray*}
lim_{n\rightarrow\infty}p_{k}^{(n)}=p_{k}\textrm{ para }0<s<1.
\end{eqnarray*}
En este caso, $g\left(s\right)=\sum_{k=0}^{\infty}s^{k}p_{k}$. Adem\'as
\begin{eqnarray*}
\sum_{k=0}^{\infty}p_{k}=1\textrm{ si y s\'olo si
}lim_{s\uparrow1}g\left(s\right)=1
\end{eqnarray*}
\end{Teo}

\begin{Teo}
Sea $N$ una variable aleatoria con valores enteros no negativos finita tal que $P\left(N=k\right)=p_{k}$, para $k=0,1,2,\ldots$, y $\sum_{k=0}^{\infty}p_{k}=P\left(N<\infty\right)=1$. Sea $\Phi$ la PGF de $N$ tal que $g\left(s\right)=\esp\left[s^{N}\right]=\sum_{k=0}^{\infty}s^{k}p_{k}$ con $g\left(1\right)=1$. Si $0\leq p_{1}\leq1$ y $\esp\left[N\right]=g^{'}\left(1\right)\leq1$, entonces no existe soluci\'on  de la ecuaci\'on $g\left(s\right)=s$ en el intervalo $\left[0,1\right)$. Si $\esp\left[N\right]=g^{'}\left(1\right)>1$, lo cual implica que $0\leq p_{1}<1$, entonces existe una \'unica soluci\'on de la ecuaci\'on $g\left(s\right)=s$ en el intervalo
$\left[0,1\right)$.
\end{Teo}

\begin{Teo}
Si $X$ y $Y$ tienen PGF $G_{X}$ y $G_{Y}$ respectivamente, entonces,\[G_{X}\left(s\right)=G_{Y}\left(s\right)\] para toda $s$, si y s\'olo si \[P\left(X=k\right))=P\left(Y=k\right)\] para toda $k=0,1,\ldots,$., es decir, si y s\'olo si $X$ y $Y$ tienen la misma distribuci\'on de probabilidad.
\end{Teo}


\begin{Teo}
Para cada $n$ fijo, sea la sucesi\'oin de probabilidades $\left\{a_{0,n},a_{1,n},\ldots,\right\}$, tales que $a_{k,n}\geq0$ para toda $k=0,1,2,\ldots,$ y $\sum_{k\geq0}a_{k,n}=1$, y sea $G_{n}\left(s\right)$ la correspondiente funci\'on generadora, $G_{n}\left(s\right)=\sum_{k\geq0}a_{k,n}s^{k}$. De modo que para cada valor fijo de $k$
\begin{eqnarray*}
lim_{n\rightarrow\infty}a_{k,n}=a_{k},
\end{eqnarray*}
es decir converge en distribuci\'on, es necesario y suficiente que para cada valor fijo $s\in\left[0,\right)$,
\begin{eqnarray*}
lim_{n\rightarrow\infty}G_{n}\left(s\right)=G\left(s\right),
\end{eqnarray*}
donde $G\left(s\right)=\sum_{k\geq0}p_{k}s^{k}$, para cualquier la funci\'on generadora del l\'imite de la sucesi\'on.
\end{Teo}

\begin{Teo}[Teorema de Abel]
Sea $G\left(s\right)=\sum_{k\geq0}a_{k}s^{k}$ para cualquier $\left\{p_{0},p_{1},\ldots,\right\}$, tales que $p_{k}\geq0$ para toda $k=0,1,2,\ldots,$. Entonces $G\left(s\right)$ es continua por la derecha en $s=1$, es decir
\begin{eqnarray*}
lim_{s\uparrow1}G\left(s\right)=\sum_{k\geq0}p_{k}=G\left(\right),
\end{eqnarray*}
sin importar si la suma es finita o no.
\end{Teo}
\begin{Note}
El radio de Convergencia para cualquier PGF es $R\geq1$, entonces, el Teorema de Abel nos dice que a\'un en el peor escenario, cuando $R=1$, a\'un se puede confiar en que la PGF ser\'a continua en $s=1$, en contraste, no se puede asegurar que la PGF ser\'a continua en el l\'imite inferior $-R$, puesto que la PGF es sim\'etrica alrededor del cero: la PGF converge para todo $s\in\left(-R,R\right)$, y no lo hace para $s<-R$ o $s>R$. Adem\'as nos dice que podemos escribir $G_{X}\left(1\right)$ como una abreviaci\'on de $lim_{s\uparrow1}G_{X}\left(s\right)$.
\end{Note}

Entonces si suponemos que la diferenciaci\'on t\'ermino a t\'ermino est\'a permitida, entonces

\begin{eqnarray*}
G_{X}^{'}\left(s\right)&=&\sum_{x=1}^{\infty}xs^{x-1}p_{x}
\end{eqnarray*}

el Teorema de Abel nos dice que
\begin{eqnarray*}
\esp\left(X\right]&=&\lim_{s\uparrow1}G_{X}^{'}\left(s\right):\\
\esp\left[X\right]&=&=\sum_{x=1}^{\infty}xp_{x}=G_{X}^{'}\left(1\right)\\
&=&\lim_{s\uparrow1}G_{X}^{'}\left(s\right),
\end{eqnarray*}
dado que el Teorema de Abel se aplica a
\begin{eqnarray*}
G_{X}^{'}\left(s\right)&=&\sum_{x=1}^{\infty}xs^{x-1}p_{x},
\end{eqnarray*}
estableciendo as\'i que $G_{X}^{'}\left(s\right)$ es continua en $s=1$. Sin el Teorema de Abel no se podr\'ia asegurar que el l\'imite de $G_{X}^{'}\left(s\right)$ conforme $s\uparrow1$ sea la respuesta correcta para $\esp\left[X\right]$.

\begin{Note}
La PGF converge para todo $|s|<R$, para alg\'un $R$. De hecho la PGF converge absolutamente si $|s|<R$. La PGF adem\'as converge uniformemente en conjuntos de la forma $\left\{s:|s|<R^{'}\right\}$, donde $R^{'}<R$, es decir, $\forall\epsilon>0, \exists n_{0}\in\ent$ tal que $\forall s$, con $|s|<R^{'}$, y $\forall n\geq n_{0}$,
\begin{eqnarray*}
|\sum_{x=0}^{n}s^{x}\prob\left(X=x\right)-G_{X}\left(s\right)|<\epsilon.
\end{eqnarray*}
De hecho, la convergencia uniforme es la que nos permite diferenciar t\'ermino a t\'ermino:
\begin{eqnarray*}
G_{X}\left(s\right)=\esp\left[s^{X}\right]=\sum_{x=0}^{\infty}s^{x}\prob\left(X=x\right),
\end{eqnarray*}
y sea $s<R$.
\begin{enumerate}
\item
\begin{eqnarray*}
G_{X}^{'}\left(s\right)&=&\frac{d}{ds}\left(\sum_{x=0}^{\infty}s^{x}\prob\left(X=x\right)\right)=\sum_{x=0}^{\infty}\frac{d}{ds}\left(s^{x}\prob\left(X=x\right)\right)\\
&=&\sum_{x=0}^{n}xs^{x-1}\prob\left(X=x\right).
\end{eqnarray*}

\item\begin{eqnarray*}
\int_{a}^{b}G_{X}\left(s\right)ds&=&\int_{a}^{b}\left(\sum_{x=0}^{\infty}s^{x}\prob\left(X=x\right)\right)ds=\sum_{x=0}^{\infty}\left(\int_{a}^{b}s^{x}\prob\left(X=x\right)ds\right)\\
&=&\sum_{x=0}^{\infty}\frac{s^{x+1}}{x+1}\prob\left(X=x\right),
\end{eqnarray*}
para $-R<a<b<R$.
\end{enumerate}
\end{Note}

\begin{Teo}[Teorema de Convergencia Mon\'otona para PGF]
Sean $X$ y $X_{n}$ variables aleatorias no negativas, con valores en los enteros, finitas, tales que
\begin{eqnarray*}
lim_{n\rightarrow\infty}G_{X_{n}}\left(s\right)&=&G_{X}\left(s\right)
\end{eqnarray*}
para $0\leq s\leq1$, entonces
\begin{eqnarray*}
lim_{n\rightarrow\infty}P\left(X_{n}=k\right)=P\left(X=k\right),
\end{eqnarray*}
para $k=0,1,2,\ldots.$
\end{Teo}

El teorema anterior requiere del siguiente lema

\begin{Lemma}
Sean $a_{n,k}\in\ent^{+}$, $n\in\nat$ constantes no negativas con $\sum_{k\geq0}a_{k,n}\leq1$. Sup\'ongase que para $0\leq s\leq1$,
se tiene
\begin{eqnarray*}
a_{n}\left(s\right)&=&\sum_{k=0}^{\infty}a_{k,n}s^{k}\rightarrow
a\left(s\right)=\sum_{k=0}^{\infty}a_{k}s^{k}.
\end{eqnarray*}
Entonces
\begin{eqnarray*}
a_{0,n}\rightarrow a_{0}.
\end{eqnarray*}
\end{Lemma}


%_________________________________________________________________________
\section{Redes de Jackson}
%_________________________________________________________________________
Cuando se considera la cantidad de
usuarios que llegan a cada uno de los nodos desde fuera del
sistema m\'as los que provienen del resto de los nodos, se dice
que la red es abierta y recibe el nombre de {\em Red de Jackson Abierta}.\\

Si denotamos por $Q_{1}\left(t\right),Q_{2}\left(t\right),\ldots,Q_{K}\left(t\right)$ el n\'umero de usuarios presentes en la cola $1,2,\ldots,K$ respectivamente al tiempo $t$, entonces se tiene la colecci\'on de colas $\left\{Q_{1},Q_{2},\ldots,Q_{K}\right\}$, donde despu\'es de que el usuario es atendido en la cola $i$, se traslada a la cola $j$ con probabilidad $p_{ij}$. En caso de que un usuario decida volver a ser atendido en $i$, este permanecer\'a en la misma cola con probabilidad $p_{ii}$. Para considerar a los usuarios que entran al sistema por primera vez por $i$, m\'as aquellos que provienen de otra cola, es necesario considerar un estado adicional $0$, con probabilidad de transici\'on $p_{00}=0$, $p_{0j}\geq0$ y $p_{j0}\geq0$, para $j=1,2,\ldots,K$, entonces en general la probabilidad de transici\'on de una cola a otra puede representarse por $P=\left(p_{ij}\right)_{i,j=0}^{K}$.\\

Para el caso espec\'ifico en el que en cada una de las colas los tiempos entre arribos y los tiempos de servicio sean exponenciales con par\'ametro de intensidad $\lambda$ y media $\mu$, respectivamente, con $m$ servidores y sin restricciones en la capacidad de almacenamiento en cada una de las colas, en Chee-Hook y Boon-Hee \cite{HookHee}, cap. 6, se muestra que el n\'umero de
usuarios en las $K$ colas, en el caso estacionario, puede determinarse por la ecuaci\'on (\ref{Eq.7.5.1})  que a
continuaci\'on se presenta, adem\'as de que la distribuci\'on l\'imite de la misma es (\ref{Eq.7.5.2}).\\

El n\'umero de usuarios en las $K$ colas en su estado estacionario, ver \cite{Bhat}, se define como
\begin{equation}\label{Eq.7.5.1}
p_{q_{1}q_{2}\cdots
q_{K}}=P\left[Q_{1}=q_{1},Q_{2}=q_{2},\ldots,Q_{K}=q_{K}\right].
\end{equation}

Jackson (1957), demostr\'o que la distribuci\'on l\'imite
$p_{q_{1}q_{2}\cdots q_{K}}$ de (\ref{Eq.7.5.1}) es

\begin{equation}\label{Eq.7.5.2}
p_{q_{1}q_{2}\cdots
q_{K}}=P_{1}\left(q_{1}\right)P_{2}\left(q_{2}\right)\cdots
P_{K}\left(q_{K}\right),
\end{equation}

donde
\begin{equation}\label{Eq.7.5.3}
p_{i}\left(r\right)=\left\{\begin{array}{cc}
 p_{i}\left(0\right)\frac{\left(\gamma_{i}/\mu_{i}\right)^{r}}{r!},  & r=0,1,2,\ldots,m, \\
 p_{i}\left(0\right)\frac{\left(\gamma_{i}/\mu_{i}\right)^{r}}{m!m^{r-m}}, & r=m,m+1,\ldots .\\
\end{array}\right.
\end{equation}

y

\begin{equation}\label{Eq.7.5.4}
\gamma_{i}=\lambda_{i}+\sum p_{ji}\gamma_{j},\textrm{
}i=1,2,\ldots,K.
\end{equation}

La relaci\'on (\ref{Eq.7.5.4}) es importante puesto que considera no solamente los arribos externos si no que adem\'as permite considerar intercambio de clientes entre las distintas colas que conforman el sistema.\\

Dados $\lambda_{i}$ y $p_{ij}$, la cantidad $\gamma_{i}$ puede determinarse a partir de la ecuaci\'on (\ref{Eq.7.5.4}) de manera recursiva. Adem\'as $p_{i}\left(0\right)$ puede determinarse utilizando la condici\'on de normalidad
\[\sum_{q_{1}}\sum_{q_{2}}\cdots\sum_{q_{K}}p_{q_{1}q_{2}\cdots q_{K}}=1.\]

Sin embargo las Redes de Jackson tienen el inconveniente de que no consideran el caso en que existan tiempos de traslado entre las colas. 



\section{Resultados Adicionales}


%_______________________________________________________________________________________
\subsection{Procesos de Renovaci\'on y Regenerativos}
%_______________________________________________________________________________________




En Sigman, Thorison y Wolff \cite{Sigman2} prueban que para la existencia de un una sucesi\'on infinita no decreciente de tiempos de regeneraci\'on $\tau_{1}\leq\tau_{2}\leq\cdots$ en los cuales el proceso se regenera, basta un tiempo de regeneraci\'on $R_{1}$, donde $R_{j}=\tau_{j}-\tau_{j-1}$. Para tal efecto se requiere la existencia de un espacio de probabilidad $\left(\Omega,\mathcal{F},\prob\right)$, y proceso estoc\'astico $\textit{X}=\left\{X\left(t\right):t\geq0\right\}$ con espacio de estados $\left(S,\mathcal{R}\right)$, con $\mathcal{R}$ $\sigma$-\'algebra.

\begin{Prop}
Si existe una variable aleatoria no negativa $R_{1}$ tal que $\theta_{R1}X=_{D}X$, entonces $\left(\Omega,\mathcal{F},\prob\right)$ puede extenderse para soportar una sucesi\'on estacionaria de variables aleatorias $R=\left\{R_{k}:k\geq1\right\}$, tal que para $k\geq1$,
\begin{eqnarray*}
\theta_{k}\left(X,R\right)=_{D}\left(X,R\right).
\end{eqnarray*}

Adem\'as, para $k\geq1$, $\theta_{k}R$ es condicionalmente independiente de $\left(X,R_{1},\ldots,R_{k}\right)$, dado $\theta_{\tau k}X$.

\end{Prop}


\begin{itemize}
\item Doob en 1953 demostr\'o que el estado estacionario de un proceso de partida en un sistema de espera $M/G/\infty$, es Poisson con la misma tasa que el proceso de arribos.

\item Burke en 1968, fue el primero en demostrar que el estado estacionario de un proceso de salida de una cola $M/M/s$ es un proceso Poisson.

\item Disney en 1973 obtuvo el siguiente resultado:

\begin{Teo}
Para el sistema de espera $M/G/1/L$ con disciplina FIFO, el proceso $\textbf{I}$ es un proceso de renovaci\'on si y s\'olo si el proceso denominado longitud de la cola es estacionario y se cumple cualquiera de los siguientes casos:

\begin{itemize}
\item[a)] Los tiempos de servicio son identicamente cero;
\item[b)] $L=0$, para cualquier proceso de servicio $S$;
\item[c)] $L=1$ y $G=D$;
\item[d)] $L=\infty$ y $G=M$.
\end{itemize}
En estos casos, respectivamente, las distribuciones de interpartida $P\left\{T_{n+1}-T_{n}\leq t\right\}$ son


\begin{itemize}
\item[a)] $1-e^{-\lambda t}$, $t\geq0$;
\item[b)] $1-e^{-\lambda t}*F\left(t\right)$, $t\geq0$;
\item[c)] $1-e^{-\lambda t}*\indora_{d}\left(t\right)$, $t\geq0$;
\item[d)] $1-e^{-\lambda t}*F\left(t\right)$, $t\geq0$.
\end{itemize}
\end{Teo}


\item Finch (1959) mostr\'o que para los sistemas $M/G/1/L$, con $1\leq L\leq \infty$ con distribuciones de servicio dos veces diferenciable, solamente el sistema $M/M/1/\infty$ tiene proceso de salida de renovaci\'on estacionario.

\item King (1971) demostro que un sistema de colas estacionario $M/G/1/1$ tiene sus tiempos de interpartida sucesivas $D_{n}$ y $D_{n+1}$ son independientes, si y s\'olo si, $G=D$, en cuyo caso le proceso de salida es de renovaci\'on.

\item Disney (1973) demostr\'o que el \'unico sistema estacionario $M/G/1/L$, que tiene proceso de salida de renovaci\'on  son los sistemas $M/M/1$ y $M/D/1/1$.



\item El siguiente resultado es de Disney y Koning (1985)
\begin{Teo}
En un sistema de espera $M/G/s$, el estado estacionario del proceso de salida es un proceso Poisson para cualquier distribuci\'on de los tiempos de servicio si el sistema tiene cualquiera de las siguientes cuatro propiedades.

\begin{itemize}
\item[a)] $s=\infty$
\item[b)] La disciplina de servicio es de procesador compartido.
\item[c)] La disciplina de servicio es LCFS y preemptive resume, esto se cumple para $L<\infty$
\item[d)] $G=M$.
\end{itemize}

\end{Teo}

\item El siguiente resultado es de Alamatsaz (1983)

\begin{Teo}
En cualquier sistema de colas $GI/G/1/L$ con $1\leq L<\infty$ y distribuci\'on de interarribos $A$ y distribuci\'on de los tiempos de servicio $B$, tal que $A\left(0\right)=0$, $A\left(t\right)\left(1-B\left(t\right)\right)>0$ para alguna $t>0$ y $B\left(t\right)$ para toda $t>0$, es imposible que el proceso de salida estacionario sea de renovaci\'on.
\end{Teo}

\end{itemize}



%________________________________________________________________________
%\subsection{Procesos Regenerativos Sigman, Thorisson y Wolff \cite{Sigman1}}
%________________________________________________________________________


\begin{Def}[Definici\'on Cl\'asica]
Un proceso estoc\'astico $X=\left\{X\left(t\right):t\geq0\right\}$ es llamado regenerativo is existe una variable aleatoria $R_{1}>0$ tal que
\begin{itemize}
\item[i)] $\left\{X\left(t+R_{1}\right):t\geq0\right\}$ es independiente de $\left\{\left\{X\left(t\right):t<R_{1}\right\},\right\}$
\item[ii)] $\left\{X\left(t+R_{1}\right):t\geq0\right\}$ es estoc\'asticamente equivalente a $\left\{X\left(t\right):t>0\right\}$
\end{itemize}

Llamamos a $R_{1}$ tiempo de regeneraci\'on, y decimos que $X$ se regenera en este punto.
\end{Def}

$\left\{X\left(t+R_{1}\right)\right\}$ es regenerativo con tiempo de regeneraci\'on $R_{2}$, independiente de $R_{1}$ pero con la misma distribuci\'on que $R_{1}$. Procediendo de esta manera se obtiene una secuencia de variables aleatorias independientes e id\'enticamente distribuidas $\left\{R_{n}\right\}$ llamados longitudes de ciclo. Si definimos a $Z_{k}\equiv R_{1}+R_{2}+\cdots+R_{k}$, se tiene un proceso de renovaci\'on llamado proceso de renovaci\'on encajado para $X$.


\begin{Note}
La existencia de un primer tiempo de regeneraci\'on, $R_{1}$, implica la existencia de una sucesi\'on completa de estos tiempos $R_{1},R_{2}\ldots,$ que satisfacen la propiedad deseada \cite{Sigman2}.
\end{Note}


\begin{Note} Para la cola $GI/GI/1$ los usuarios arriban con tiempos $t_{n}$ y son atendidos con tiempos de servicio $S_{n}$, los tiempos de arribo forman un proceso de renovaci\'on  con tiempos entre arribos independientes e identicamente distribuidos (\texttt{i.i.d.})$T_{n}=t_{n}-t_{n-1}$, adem\'as los tiempos de servicio son \texttt{i.i.d.} e independientes de los procesos de arribo. Por \textit{estable} se entiende que $\esp S_{n}<\esp T_{n}<\infty$.
\end{Note}
 


\begin{Def}
Para $x$ fijo y para cada $t\geq0$, sea $I_{x}\left(t\right)=1$ si $X\left(t\right)\leq x$,  $I_{x}\left(t\right)=0$ en caso contrario, y def\'inanse los tiempos promedio
\begin{eqnarray*}
\overline{X}&=&lim_{t\rightarrow\infty}\frac{1}{t}\int_{0}^{\infty}X\left(u\right)du\\
\prob\left(X_{\infty}\leq x\right)&=&lim_{t\rightarrow\infty}\frac{1}{t}\int_{0}^{\infty}I_{x}\left(u\right)du,
\end{eqnarray*}
cuando estos l\'imites existan.
\end{Def}

Como consecuencia del teorema de Renovaci\'on-Recompensa, se tiene que el primer l\'imite  existe y es igual a la constante
\begin{eqnarray*}
\overline{X}&=&\frac{\esp\left[\int_{0}^{R_{1}}X\left(t\right)dt\right]}{\esp\left[R_{1}\right]},
\end{eqnarray*}
suponiendo que ambas esperanzas son finitas.
 
\begin{Note}
Funciones de procesos regenerativos son regenerativas, es decir, si $X\left(t\right)$ es regenerativo y se define el proceso $Y\left(t\right)$ por $Y\left(t\right)=f\left(X\left(t\right)\right)$ para alguna funci\'on Borel medible $f\left(\cdot\right)$. Adem\'as $Y$ es regenerativo con los mismos tiempos de renovaci\'on que $X$. 

En general, los tiempos de renovaci\'on, $Z_{k}$ de un proceso regenerativo no requieren ser tiempos de paro con respecto a la evoluci\'on de $X\left(t\right)$.
\end{Note} 

\begin{Note}
Una funci\'on de un proceso de Markov, usualmente no ser\'a un proceso de Markov, sin embargo ser\'a regenerativo si el proceso de Markov lo es.
\end{Note}

 
\begin{Note}
Un proceso regenerativo con media de la longitud de ciclo finita es llamado positivo recurrente.
\end{Note}


\begin{Note}
\begin{itemize}
\item[a)] Si el proceso regenerativo $X$ es positivo recurrente y tiene trayectorias muestrales no negativas, entonces la ecuaci\'on anterior es v\'alida.
\item[b)] Si $X$ es positivo recurrente regenerativo, podemos construir una \'unica versi\'on estacionaria de este proceso, $X_{e}=\left\{X_{e}\left(t\right)\right\}$, donde $X_{e}$ es un proceso estoc\'astico regenerativo y estrictamente estacionario, con distribuci\'on marginal distribuida como $X_{\infty}$
\end{itemize}
\end{Note}


%__________________________________________________________________________________________
%\subsection{Procesos Regenerativos Estacionarios - Stidham \cite{Stidham}}
%__________________________________________________________________________________________


Un proceso estoc\'astico a tiempo continuo $\left\{V\left(t\right),t\geq0\right\}$ es un proceso regenerativo si existe una sucesi\'on de variables aleatorias independientes e id\'enticamente distribuidas $\left\{X_{1},X_{2},\ldots\right\}$, sucesi\'on de renovaci\'on, tal que para cualquier conjunto de Borel $A$, 

\begin{eqnarray*}
\prob\left\{V\left(t\right)\in A|X_{1}+X_{2}+\cdots+X_{R\left(t\right)}=s,\left\{V\left(\tau\right),\tau<s\right\}\right\}=\prob\left\{V\left(t-s\right)\in A|X_{1}>t-s\right\},
\end{eqnarray*}
para todo $0\leq s\leq t$, donde $R\left(t\right)=\max\left\{X_{1}+X_{2}+\cdots+X_{j}\leq t\right\}=$n\'umero de renovaciones ({\emph{puntos de regeneraci\'on}}) que ocurren en $\left[0,t\right]$. El intervalo $\left[0,X_{1}\right)$ es llamado {\emph{primer ciclo de regeneraci\'on}} de $\left\{V\left(t \right),t\geq0\right\}$, $\left[X_{1},X_{1}+X_{2}\right)$ el {\emph{segundo ciclo de regeneraci\'on}}, y as\'i sucesivamente.

Sea $X=X_{1}$ y sea $F$ la funci\'on de distrbuci\'on de $X$


\begin{Def}
Se define el proceso estacionario, $\left\{V^{*}\left(t\right),t\geq0\right\}$, para $\left\{V\left(t\right),t\geq0\right\}$ por

\begin{eqnarray*}
\prob\left\{V\left(t\right)\in A\right\}=\frac{1}{\esp\left[X\right]}\int_{0}^{\infty}\prob\left\{V\left(t+x\right)\in A|X>x\right\}\left(1-F\left(x\right)\right)dx,
\end{eqnarray*} 
para todo $t\geq0$ y todo conjunto de Borel $A$.
\end{Def}

\begin{Def}
Una distribuci\'on se dice que es {\emph{aritm\'etica}} si todos sus puntos de incremento son m\'ultiplos de la forma $0,\lambda, 2\lambda,\ldots$ para alguna $\lambda>0$ entera.
\end{Def}


\begin{Def}
Una modificaci\'on medible de un proceso $\left\{V\left(t\right),t\geq0\right\}$, es una versi\'on de este, $\left\{V\left(t,w\right)\right\}$ conjuntamente medible para $t\geq0$ y para $w\in S$, $S$ espacio de estados para $\left\{V\left(t\right),t\geq0\right\}$.
\end{Def}

\begin{Teo}
Sea $\left\{V\left(t\right),t\geq\right\}$ un proceso regenerativo no negativo con modificaci\'on medible. Sea $\esp\left[X\right]<\infty$. Entonces el proceso estacionario dado por la ecuaci\'on anterior est\'a bien definido y tiene funci\'on de distribuci\'on independiente de $t$, adem\'as
\begin{itemize}
\item[i)] \begin{eqnarray*}
\esp\left[V^{*}\left(0\right)\right]&=&\frac{\esp\left[\int_{0}^{X}V\left(s\right)ds\right]}{\esp\left[X\right]}\end{eqnarray*}
\item[ii)] Si $\esp\left[V^{*}\left(0\right)\right]<\infty$, equivalentemente, si $\esp\left[\int_{0}^{X}V\left(s\right)ds\right]<\infty$,entonces
\begin{eqnarray*}
\frac{\int_{0}^{t}V\left(s\right)ds}{t}\rightarrow\frac{\esp\left[\int_{0}^{X}V\left(s\right)ds\right]}{\esp\left[X\right]}
\end{eqnarray*}
con probabilidad 1 y en media, cuando $t\rightarrow\infty$.
\end{itemize}
\end{Teo}

\begin{Coro}
Sea $\left\{V\left(t\right),t\geq0\right\}$ un proceso regenerativo no negativo, con modificaci\'on medible. Si $\esp <\infty$, $F$ es no-aritm\'etica, y para todo $x\geq0$, $P\left\{V\left(t\right)\leq x,C>x\right\}$ es de variaci\'on acotada como funci\'on de $t$ en cada intervalo finito $\left[0,\tau\right]$, entonces $V\left(t\right)$ converge en distribuci\'on  cuando $t\rightarrow\infty$ y $$\esp V=\frac{\esp \int_{0}^{X}V\left(s\right)ds}{\esp X}$$
Donde $V$ tiene la distribuci\'on l\'imite de $V\left(t\right)$ cuando $t\rightarrow\infty$.

\end{Coro}

Para el caso discreto se tienen resultados similares.



%______________________________________________________________________
%\subsection{Procesos de Renovaci\'on}
%______________________________________________________________________

\begin{Def}%\label{Def.Tn}
Sean $0\leq T_{1}\leq T_{2}\leq \ldots$ son tiempos aleatorios infinitos en los cuales ocurren ciertos eventos. El n\'umero de tiempos $T_{n}$ en el intervalo $\left[0,t\right)$ es

\begin{eqnarray}
N\left(t\right)=\sum_{n=1}^{\infty}\indora\left(T_{n}\leq t\right),
\end{eqnarray}
para $t\geq0$.
\end{Def}

Si se consideran los puntos $T_{n}$ como elementos de $\rea_{+}$, y $N\left(t\right)$ es el n\'umero de puntos en $\rea$. El proceso denotado por $\left\{N\left(t\right):t\geq0\right\}$, denotado por $N\left(t\right)$, es un proceso puntual en $\rea_{+}$. Los $T_{n}$ son los tiempos de ocurrencia, el proceso puntual $N\left(t\right)$ es simple si su n\'umero de ocurrencias son distintas: $0<T_{1}<T_{2}<\ldots$ casi seguramente.

\begin{Def}
Un proceso puntual $N\left(t\right)$ es un proceso de renovaci\'on si los tiempos de interocurrencia $\xi_{n}=T_{n}-T_{n-1}$, para $n\geq1$, son independientes e identicamente distribuidos con distribuci\'on $F$, donde $F\left(0\right)=0$ y $T_{0}=0$. Los $T_{n}$ son llamados tiempos de renovaci\'on, referente a la independencia o renovaci\'on de la informaci\'on estoc\'astica en estos tiempos. Los $\xi_{n}$ son los tiempos de inter-renovaci\'on, y $N\left(t\right)$ es el n\'umero de renovaciones en el intervalo $\left[0,t\right)$
\end{Def}


\begin{Note}
Para definir un proceso de renovaci\'on para cualquier contexto, solamente hay que especificar una distribuci\'on $F$, con $F\left(0\right)=0$, para los tiempos de inter-renovaci\'on. La funci\'on $F$ en turno degune las otra variables aleatorias. De manera formal, existe un espacio de probabilidad y una sucesi\'on de variables aleatorias $\xi_{1},\xi_{2},\ldots$ definidas en este con distribuci\'on $F$. Entonces las otras cantidades son $T_{n}=\sum_{k=1}^{n}\xi_{k}$ y $N\left(t\right)=\sum_{n=1}^{\infty}\indora\left(T_{n}\leq t\right)$, donde $T_{n}\rightarrow\infty$ casi seguramente por la Ley Fuerte de los Grandes Números.
\end{Note}

%___________________________________________________________________________________________
%
%\subsection{Teorema Principal de Renovaci\'on}
%___________________________________________________________________________________________
%

\begin{Note} Una funci\'on $h:\rea_{+}\rightarrow\rea$ es Directamente Riemann Integrable en los siguientes casos:
\begin{itemize}
\item[a)] $h\left(t\right)\geq0$ es decreciente y Riemann Integrable.
\item[b)] $h$ es continua excepto posiblemente en un conjunto de Lebesgue de medida 0, y $|h\left(t\right)|\leq b\left(t\right)$, donde $b$ es DRI.
\end{itemize}
\end{Note}

\begin{Teo}[Teorema Principal de Renovaci\'on]
Si $F$ es no aritm\'etica y $h\left(t\right)$ es Directamente Riemann Integrable (DRI), entonces

\begin{eqnarray*}
lim_{t\rightarrow\infty}U\star h=\frac{1}{\mu}\int_{\rea_{+}}h\left(s\right)ds.
\end{eqnarray*}
\end{Teo}

\begin{Prop}
Cualquier funci\'on $H\left(t\right)$ acotada en intervalos finitos y que es 0 para $t<0$ puede expresarse como
\begin{eqnarray*}
H\left(t\right)=U\star h\left(t\right)\textrm{,  donde }h\left(t\right)=H\left(t\right)-F\star H\left(t\right)
\end{eqnarray*}
\end{Prop}

\begin{Def}
Un proceso estoc\'astico $X\left(t\right)$ es crudamente regenerativo en un tiempo aleatorio positivo $T$ si
\begin{eqnarray*}
\esp\left[X\left(T+t\right)|T\right]=\esp\left[X\left(t\right)\right]\textrm{, para }t\geq0,\end{eqnarray*}
y con las esperanzas anteriores finitas.
\end{Def}

\begin{Prop}
Sup\'ongase que $X\left(t\right)$ es un proceso crudamente regenerativo en $T$, que tiene distribuci\'on $F$. Si $\esp\left[X\left(t\right)\right]$ es acotado en intervalos finitos, entonces
\begin{eqnarray*}
\esp\left[X\left(t\right)\right]=U\star h\left(t\right)\textrm{,  donde }h\left(t\right)=\esp\left[X\left(t\right)\indora\left(T>t\right)\right].
\end{eqnarray*}
\end{Prop}

\begin{Teo}[Regeneraci\'on Cruda]
Sup\'ongase que $X\left(t\right)$ es un proceso con valores positivo crudamente regenerativo en $T$, y def\'inase $M=\sup\left\{|X\left(t\right)|:t\leq T\right\}$. Si $T$ es no aritm\'etico y $M$ y $MT$ tienen media finita, entonces
\begin{eqnarray*}
lim_{t\rightarrow\infty}\esp\left[X\left(t\right)\right]=\frac{1}{\mu}\int_{\rea_{+}}h\left(s\right)ds,
\end{eqnarray*}
donde $h\left(t\right)=\esp\left[X\left(t\right)\indora\left(T>t\right)\right]$.
\end{Teo}

%___________________________________________________________________________________________
%
%\subsection{Propiedades de los Procesos de Renovaci\'on}
%___________________________________________________________________________________________
%

Los tiempos $T_{n}$ est\'an relacionados con los conteos de $N\left(t\right)$ por

\begin{eqnarray*}
\left\{N\left(t\right)\geq n\right\}&=&\left\{T_{n}\leq t\right\}\\
T_{N\left(t\right)}\leq &t&<T_{N\left(t\right)+1},
\end{eqnarray*}

adem\'as $N\left(T_{n}\right)=n$, y 

\begin{eqnarray*}
N\left(t\right)=\max\left\{n:T_{n}\leq t\right\}=\min\left\{n:T_{n+1}>t\right\}
\end{eqnarray*}

Por propiedades de la convoluci\'on se sabe que

\begin{eqnarray*}
P\left\{T_{n}\leq t\right\}=F^{n\star}\left(t\right)
\end{eqnarray*}
que es la $n$-\'esima convoluci\'on de $F$. Entonces 

\begin{eqnarray*}
\left\{N\left(t\right)\geq n\right\}&=&\left\{T_{n}\leq t\right\}\\
P\left\{N\left(t\right)\leq n\right\}&=&1-F^{\left(n+1\right)\star}\left(t\right)
\end{eqnarray*}

Adem\'as usando el hecho de que $\esp\left[N\left(t\right)\right]=\sum_{n=1}^{\infty}P\left\{N\left(t\right)\geq n\right\}$
se tiene que

\begin{eqnarray*}
\esp\left[N\left(t\right)\right]=\sum_{n=1}^{\infty}F^{n\star}\left(t\right)
\end{eqnarray*}

\begin{Prop}
Para cada $t\geq0$, la funci\'on generadora de momentos $\esp\left[e^{\alpha N\left(t\right)}\right]$ existe para alguna $\alpha$ en una vecindad del 0, y de aqu\'i que $\esp\left[N\left(t\right)^{m}\right]<\infty$, para $m\geq1$.
\end{Prop}


\begin{Note}
Si el primer tiempo de renovaci\'on $\xi_{1}$ no tiene la misma distribuci\'on que el resto de las $\xi_{n}$, para $n\geq2$, a $N\left(t\right)$ se le llama Proceso de Renovaci\'on retardado, donde si $\xi$ tiene distribuci\'on $G$, entonces el tiempo $T_{n}$ de la $n$-\'esima renovaci\'on tiene distribuci\'on $G\star F^{\left(n-1\right)\star}\left(t\right)$
\end{Note}


\begin{Teo}
Para una constante $\mu\leq\infty$ ( o variable aleatoria), las siguientes expresiones son equivalentes:

\begin{eqnarray}
lim_{n\rightarrow\infty}n^{-1}T_{n}&=&\mu,\textrm{ c.s.}\\
lim_{t\rightarrow\infty}t^{-1}N\left(t\right)&=&1/\mu,\textrm{ c.s.}
\end{eqnarray}
\end{Teo}


Es decir, $T_{n}$ satisface la Ley Fuerte de los Grandes N\'umeros s\'i y s\'olo s\'i $N\left/t\right)$ la cumple.


\begin{Coro}[Ley Fuerte de los Grandes N\'umeros para Procesos de Renovaci\'on]
Si $N\left(t\right)$ es un proceso de renovaci\'on cuyos tiempos de inter-renovaci\'on tienen media $\mu\leq\infty$, entonces
\begin{eqnarray}
t^{-1}N\left(t\right)\rightarrow 1/\mu,\textrm{ c.s. cuando }t\rightarrow\infty.
\end{eqnarray}

\end{Coro}


Considerar el proceso estoc\'astico de valores reales $\left\{Z\left(t\right):t\geq0\right\}$ en el mismo espacio de probabilidad que $N\left(t\right)$

\begin{Def}
Para el proceso $\left\{Z\left(t\right):t\geq0\right\}$ se define la fluctuaci\'on m\'axima de $Z\left(t\right)$ en el intervalo $\left(T_{n-1},T_{n}\right]$:
\begin{eqnarray*}
M_{n}=\sup_{T_{n-1}<t\leq T_{n}}|Z\left(t\right)-Z\left(T_{n-1}\right)|
\end{eqnarray*}
\end{Def}

\begin{Teo}
Sup\'ongase que $n^{-1}T_{n}\rightarrow\mu$ c.s. cuando $n\rightarrow\infty$, donde $\mu\leq\infty$ es una constante o variable aleatoria. Sea $a$ una constante o variable aleatoria que puede ser infinita cuando $\mu$ es finita, y considere las expresiones l\'imite:
\begin{eqnarray}
lim_{n\rightarrow\infty}n^{-1}Z\left(T_{n}\right)&=&a,\textrm{ c.s.}\\
lim_{t\rightarrow\infty}t^{-1}Z\left(t\right)&=&a/\mu,\textrm{ c.s.}
\end{eqnarray}
La segunda expresi\'on implica la primera. Conversamente, la primera implica la segunda si el proceso $Z\left(t\right)$ es creciente, o si $lim_{n\rightarrow\infty}n^{-1}M_{n}=0$ c.s.
\end{Teo}

\begin{Coro}
Si $N\left(t\right)$ es un proceso de renovaci\'on, y $\left(Z\left(T_{n}\right)-Z\left(T_{n-1}\right),M_{n}\right)$, para $n\geq1$, son variables aleatorias independientes e id\'enticamente distribuidas con media finita, entonces,
\begin{eqnarray}
lim_{t\rightarrow\infty}t^{-1}Z\left(t\right)\rightarrow\frac{\esp\left[Z\left(T_{1}\right)-Z\left(T_{0}\right)\right]}{\esp\left[T_{1}\right]},\textrm{ c.s. cuando  }t\rightarrow\infty.
\end{eqnarray}
\end{Coro}



%___________________________________________________________________________________________
%
%\subsection{Propiedades de los Procesos de Renovaci\'on}
%___________________________________________________________________________________________
%

Los tiempos $T_{n}$ est\'an relacionados con los conteos de $N\left(t\right)$ por

\begin{eqnarray*}
\left\{N\left(t\right)\geq n\right\}&=&\left\{T_{n}\leq t\right\}\\
T_{N\left(t\right)}\leq &t&<T_{N\left(t\right)+1},
\end{eqnarray*}

adem\'as $N\left(T_{n}\right)=n$, y 

\begin{eqnarray*}
N\left(t\right)=\max\left\{n:T_{n}\leq t\right\}=\min\left\{n:T_{n+1}>t\right\}
\end{eqnarray*}

Por propiedades de la convoluci\'on se sabe que

\begin{eqnarray*}
P\left\{T_{n}\leq t\right\}=F^{n\star}\left(t\right)
\end{eqnarray*}
que es la $n$-\'esima convoluci\'on de $F$. Entonces 

\begin{eqnarray*}
\left\{N\left(t\right)\geq n\right\}&=&\left\{T_{n}\leq t\right\}\\
P\left\{N\left(t\right)\leq n\right\}&=&1-F^{\left(n+1\right)\star}\left(t\right)
\end{eqnarray*}

Adem\'as usando el hecho de que $\esp\left[N\left(t\right)\right]=\sum_{n=1}^{\infty}P\left\{N\left(t\right)\geq n\right\}$
se tiene que

\begin{eqnarray*}
\esp\left[N\left(t\right)\right]=\sum_{n=1}^{\infty}F^{n\star}\left(t\right)
\end{eqnarray*}

\begin{Prop}
Para cada $t\geq0$, la funci\'on generadora de momentos $\esp\left[e^{\alpha N\left(t\right)}\right]$ existe para alguna $\alpha$ en una vecindad del 0, y de aqu\'i que $\esp\left[N\left(t\right)^{m}\right]<\infty$, para $m\geq1$.
\end{Prop}


\begin{Note}
Si el primer tiempo de renovaci\'on $\xi_{1}$ no tiene la misma distribuci\'on que el resto de las $\xi_{n}$, para $n\geq2$, a $N\left(t\right)$ se le llama Proceso de Renovaci\'on retardado, donde si $\xi$ tiene distribuci\'on $G$, entonces el tiempo $T_{n}$ de la $n$-\'esima renovaci\'on tiene distribuci\'on $G\star F^{\left(n-1\right)\star}\left(t\right)$
\end{Note}


\begin{Teo}
Para una constante $\mu\leq\infty$ ( o variable aleatoria), las siguientes expresiones son equivalentes:

\begin{eqnarray}
lim_{n\rightarrow\infty}n^{-1}T_{n}&=&\mu,\textrm{ c.s.}\\
lim_{t\rightarrow\infty}t^{-1}N\left(t\right)&=&1/\mu,\textrm{ c.s.}
\end{eqnarray}
\end{Teo}


Es decir, $T_{n}$ satisface la Ley Fuerte de los Grandes N\'umeros s\'i y s\'olo s\'i $N\left/t\right)$ la cumple.


\begin{Coro}[Ley Fuerte de los Grandes N\'umeros para Procesos de Renovaci\'on]
Si $N\left(t\right)$ es un proceso de renovaci\'on cuyos tiempos de inter-renovaci\'on tienen media $\mu\leq\infty$, entonces
\begin{eqnarray}
t^{-1}N\left(t\right)\rightarrow 1/\mu,\textrm{ c.s. cuando }t\rightarrow\infty.
\end{eqnarray}

\end{Coro}


Considerar el proceso estoc\'astico de valores reales $\left\{Z\left(t\right):t\geq0\right\}$ en el mismo espacio de probabilidad que $N\left(t\right)$

\begin{Def}
Para el proceso $\left\{Z\left(t\right):t\geq0\right\}$ se define la fluctuaci\'on m\'axima de $Z\left(t\right)$ en el intervalo $\left(T_{n-1},T_{n}\right]$:
\begin{eqnarray*}
M_{n}=\sup_{T_{n-1}<t\leq T_{n}}|Z\left(t\right)-Z\left(T_{n-1}\right)|
\end{eqnarray*}
\end{Def}

\begin{Teo}
Sup\'ongase que $n^{-1}T_{n}\rightarrow\mu$ c.s. cuando $n\rightarrow\infty$, donde $\mu\leq\infty$ es una constante o variable aleatoria. Sea $a$ una constante o variable aleatoria que puede ser infinita cuando $\mu$ es finita, y considere las expresiones l\'imite:
\begin{eqnarray}
lim_{n\rightarrow\infty}n^{-1}Z\left(T_{n}\right)&=&a,\textrm{ c.s.}\\
lim_{t\rightarrow\infty}t^{-1}Z\left(t\right)&=&a/\mu,\textrm{ c.s.}
\end{eqnarray}
La segunda expresi\'on implica la primera. Conversamente, la primera implica la segunda si el proceso $Z\left(t\right)$ es creciente, o si $lim_{n\rightarrow\infty}n^{-1}M_{n}=0$ c.s.
\end{Teo}

\begin{Coro}
Si $N\left(t\right)$ es un proceso de renovaci\'on, y $\left(Z\left(T_{n}\right)-Z\left(T_{n-1}\right),M_{n}\right)$, para $n\geq1$, son variables aleatorias independientes e id\'enticamente distribuidas con media finita, entonces,
\begin{eqnarray}
lim_{t\rightarrow\infty}t^{-1}Z\left(t\right)\rightarrow\frac{\esp\left[Z\left(T_{1}\right)-Z\left(T_{0}\right)\right]}{\esp\left[T_{1}\right]},\textrm{ c.s. cuando  }t\rightarrow\infty.
\end{eqnarray}
\end{Coro}


%___________________________________________________________________________________________
%
%\subsection{Propiedades de los Procesos de Renovaci\'on}
%___________________________________________________________________________________________
%

Los tiempos $T_{n}$ est\'an relacionados con los conteos de $N\left(t\right)$ por

\begin{eqnarray*}
\left\{N\left(t\right)\geq n\right\}&=&\left\{T_{n}\leq t\right\}\\
T_{N\left(t\right)}\leq &t&<T_{N\left(t\right)+1},
\end{eqnarray*}

adem\'as $N\left(T_{n}\right)=n$, y 

\begin{eqnarray*}
N\left(t\right)=\max\left\{n:T_{n}\leq t\right\}=\min\left\{n:T_{n+1}>t\right\}
\end{eqnarray*}

Por propiedades de la convoluci\'on se sabe que

\begin{eqnarray*}
P\left\{T_{n}\leq t\right\}=F^{n\star}\left(t\right)
\end{eqnarray*}
que es la $n$-\'esima convoluci\'on de $F$. Entonces 

\begin{eqnarray*}
\left\{N\left(t\right)\geq n\right\}&=&\left\{T_{n}\leq t\right\}\\
P\left\{N\left(t\right)\leq n\right\}&=&1-F^{\left(n+1\right)\star}\left(t\right)
\end{eqnarray*}

Adem\'as usando el hecho de que $\esp\left[N\left(t\right)\right]=\sum_{n=1}^{\infty}P\left\{N\left(t\right)\geq n\right\}$
se tiene que

\begin{eqnarray*}
\esp\left[N\left(t\right)\right]=\sum_{n=1}^{\infty}F^{n\star}\left(t\right)
\end{eqnarray*}

\begin{Prop}
Para cada $t\geq0$, la funci\'on generadora de momentos $\esp\left[e^{\alpha N\left(t\right)}\right]$ existe para alguna $\alpha$ en una vecindad del 0, y de aqu\'i que $\esp\left[N\left(t\right)^{m}\right]<\infty$, para $m\geq1$.
\end{Prop}


\begin{Note}
Si el primer tiempo de renovaci\'on $\xi_{1}$ no tiene la misma distribuci\'on que el resto de las $\xi_{n}$, para $n\geq2$, a $N\left(t\right)$ se le llama Proceso de Renovaci\'on retardado, donde si $\xi$ tiene distribuci\'on $G$, entonces el tiempo $T_{n}$ de la $n$-\'esima renovaci\'on tiene distribuci\'on $G\star F^{\left(n-1\right)\star}\left(t\right)$
\end{Note}


\begin{Teo}
Para una constante $\mu\leq\infty$ ( o variable aleatoria), las siguientes expresiones son equivalentes:

\begin{eqnarray}
lim_{n\rightarrow\infty}n^{-1}T_{n}&=&\mu,\textrm{ c.s.}\\
lim_{t\rightarrow\infty}t^{-1}N\left(t\right)&=&1/\mu,\textrm{ c.s.}
\end{eqnarray}
\end{Teo}


Es decir, $T_{n}$ satisface la Ley Fuerte de los Grandes N\'umeros s\'i y s\'olo s\'i $N\left/t\right)$ la cumple.


\begin{Coro}[Ley Fuerte de los Grandes N\'umeros para Procesos de Renovaci\'on]
Si $N\left(t\right)$ es un proceso de renovaci\'on cuyos tiempos de inter-renovaci\'on tienen media $\mu\leq\infty$, entonces
\begin{eqnarray}
t^{-1}N\left(t\right)\rightarrow 1/\mu,\textrm{ c.s. cuando }t\rightarrow\infty.
\end{eqnarray}

\end{Coro}


Considerar el proceso estoc\'astico de valores reales $\left\{Z\left(t\right):t\geq0\right\}$ en el mismo espacio de probabilidad que $N\left(t\right)$

\begin{Def}
Para el proceso $\left\{Z\left(t\right):t\geq0\right\}$ se define la fluctuaci\'on m\'axima de $Z\left(t\right)$ en el intervalo $\left(T_{n-1},T_{n}\right]$:
\begin{eqnarray*}
M_{n}=\sup_{T_{n-1}<t\leq T_{n}}|Z\left(t\right)-Z\left(T_{n-1}\right)|
\end{eqnarray*}
\end{Def}

\begin{Teo}
Sup\'ongase que $n^{-1}T_{n}\rightarrow\mu$ c.s. cuando $n\rightarrow\infty$, donde $\mu\leq\infty$ es una constante o variable aleatoria. Sea $a$ una constante o variable aleatoria que puede ser infinita cuando $\mu$ es finita, y considere las expresiones l\'imite:
\begin{eqnarray}
lim_{n\rightarrow\infty}n^{-1}Z\left(T_{n}\right)&=&a,\textrm{ c.s.}\\
lim_{t\rightarrow\infty}t^{-1}Z\left(t\right)&=&a/\mu,\textrm{ c.s.}
\end{eqnarray}
La segunda expresi\'on implica la primera. Conversamente, la primera implica la segunda si el proceso $Z\left(t\right)$ es creciente, o si $lim_{n\rightarrow\infty}n^{-1}M_{n}=0$ c.s.
\end{Teo}

\begin{Coro}
Si $N\left(t\right)$ es un proceso de renovaci\'on, y $\left(Z\left(T_{n}\right)-Z\left(T_{n-1}\right),M_{n}\right)$, para $n\geq1$, son variables aleatorias independientes e id\'enticamente distribuidas con media finita, entonces,
\begin{eqnarray}
lim_{t\rightarrow\infty}t^{-1}Z\left(t\right)\rightarrow\frac{\esp\left[Z\left(T_{1}\right)-Z\left(T_{0}\right)\right]}{\esp\left[T_{1}\right]},\textrm{ c.s. cuando  }t\rightarrow\infty.
\end{eqnarray}
\end{Coro}

%___________________________________________________________________________________________
%
%\subsection{Propiedades de los Procesos de Renovaci\'on}
%___________________________________________________________________________________________
%

Los tiempos $T_{n}$ est\'an relacionados con los conteos de $N\left(t\right)$ por

\begin{eqnarray*}
\left\{N\left(t\right)\geq n\right\}&=&\left\{T_{n}\leq t\right\}\\
T_{N\left(t\right)}\leq &t&<T_{N\left(t\right)+1},
\end{eqnarray*}

adem\'as $N\left(T_{n}\right)=n$, y 

\begin{eqnarray*}
N\left(t\right)=\max\left\{n:T_{n}\leq t\right\}=\min\left\{n:T_{n+1}>t\right\}
\end{eqnarray*}

Por propiedades de la convoluci\'on se sabe que

\begin{eqnarray*}
P\left\{T_{n}\leq t\right\}=F^{n\star}\left(t\right)
\end{eqnarray*}
que es la $n$-\'esima convoluci\'on de $F$. Entonces 

\begin{eqnarray*}
\left\{N\left(t\right)\geq n\right\}&=&\left\{T_{n}\leq t\right\}\\
P\left\{N\left(t\right)\leq n\right\}&=&1-F^{\left(n+1\right)\star}\left(t\right)
\end{eqnarray*}

Adem\'as usando el hecho de que $\esp\left[N\left(t\right)\right]=\sum_{n=1}^{\infty}P\left\{N\left(t\right)\geq n\right\}$
se tiene que

\begin{eqnarray*}
\esp\left[N\left(t\right)\right]=\sum_{n=1}^{\infty}F^{n\star}\left(t\right)
\end{eqnarray*}

\begin{Prop}
Para cada $t\geq0$, la funci\'on generadora de momentos $\esp\left[e^{\alpha N\left(t\right)}\right]$ existe para alguna $\alpha$ en una vecindad del 0, y de aqu\'i que $\esp\left[N\left(t\right)^{m}\right]<\infty$, para $m\geq1$.
\end{Prop}


\begin{Note}
Si el primer tiempo de renovaci\'on $\xi_{1}$ no tiene la misma distribuci\'on que el resto de las $\xi_{n}$, para $n\geq2$, a $N\left(t\right)$ se le llama Proceso de Renovaci\'on retardado, donde si $\xi$ tiene distribuci\'on $G$, entonces el tiempo $T_{n}$ de la $n$-\'esima renovaci\'on tiene distribuci\'on $G\star F^{\left(n-1\right)\star}\left(t\right)$
\end{Note}


\begin{Teo}
Para una constante $\mu\leq\infty$ ( o variable aleatoria), las siguientes expresiones son equivalentes:

\begin{eqnarray}
lim_{n\rightarrow\infty}n^{-1}T_{n}&=&\mu,\textrm{ c.s.}\\
lim_{t\rightarrow\infty}t^{-1}N\left(t\right)&=&1/\mu,\textrm{ c.s.}
\end{eqnarray}
\end{Teo}


Es decir, $T_{n}$ satisface la Ley Fuerte de los Grandes N\'umeros s\'i y s\'olo s\'i $N\left/t\right)$ la cumple.


\begin{Coro}[Ley Fuerte de los Grandes N\'umeros para Procesos de Renovaci\'on]
Si $N\left(t\right)$ es un proceso de renovaci\'on cuyos tiempos de inter-renovaci\'on tienen media $\mu\leq\infty$, entonces
\begin{eqnarray}
t^{-1}N\left(t\right)\rightarrow 1/\mu,\textrm{ c.s. cuando }t\rightarrow\infty.
\end{eqnarray}

\end{Coro}


Considerar el proceso estoc\'astico de valores reales $\left\{Z\left(t\right):t\geq0\right\}$ en el mismo espacio de probabilidad que $N\left(t\right)$

\begin{Def}
Para el proceso $\left\{Z\left(t\right):t\geq0\right\}$ se define la fluctuaci\'on m\'axima de $Z\left(t\right)$ en el intervalo $\left(T_{n-1},T_{n}\right]$:
\begin{eqnarray*}
M_{n}=\sup_{T_{n-1}<t\leq T_{n}}|Z\left(t\right)-Z\left(T_{n-1}\right)|
\end{eqnarray*}
\end{Def}

\begin{Teo}
Sup\'ongase que $n^{-1}T_{n}\rightarrow\mu$ c.s. cuando $n\rightarrow\infty$, donde $\mu\leq\infty$ es una constante o variable aleatoria. Sea $a$ una constante o variable aleatoria que puede ser infinita cuando $\mu$ es finita, y considere las expresiones l\'imite:
\begin{eqnarray}
lim_{n\rightarrow\infty}n^{-1}Z\left(T_{n}\right)&=&a,\textrm{ c.s.}\\
lim_{t\rightarrow\infty}t^{-1}Z\left(t\right)&=&a/\mu,\textrm{ c.s.}
\end{eqnarray}
La segunda expresi\'on implica la primera. Conversamente, la primera implica la segunda si el proceso $Z\left(t\right)$ es creciente, o si $lim_{n\rightarrow\infty}n^{-1}M_{n}=0$ c.s.
\end{Teo}

\begin{Coro}
Si $N\left(t\right)$ es un proceso de renovaci\'on, y $\left(Z\left(T_{n}\right)-Z\left(T_{n-1}\right),M_{n}\right)$, para $n\geq1$, son variables aleatorias independientes e id\'enticamente distribuidas con media finita, entonces,
\begin{eqnarray}
lim_{t\rightarrow\infty}t^{-1}Z\left(t\right)\rightarrow\frac{\esp\left[Z\left(T_{1}\right)-Z\left(T_{0}\right)\right]}{\esp\left[T_{1}\right]},\textrm{ c.s. cuando  }t\rightarrow\infty.
\end{eqnarray}
\end{Coro}
%___________________________________________________________________________________________
%
\subsection{Propiedades de los Procesos de Renovaci\'on}
%___________________________________________________________________________________________
%

Los tiempos $T_{n}$ est\'an relacionados con los conteos de $N\left(t\right)$ por

\begin{eqnarray*}
\left\{N\left(t\right)\geq n\right\}&=&\left\{T_{n}\leq t\right\}\\
T_{N\left(t\right)}\leq &t&<T_{N\left(t\right)+1},
\end{eqnarray*}

adem\'as $N\left(T_{n}\right)=n$, y 

\begin{eqnarray*}
N\left(t\right)=\max\left\{n:T_{n}\leq t\right\}=\min\left\{n:T_{n+1}>t\right\}
\end{eqnarray*}

Por propiedades de la convoluci\'on se sabe que

\begin{eqnarray*}
P\left\{T_{n}\leq t\right\}=F^{n\star}\left(t\right)
\end{eqnarray*}
que es la $n$-\'esima convoluci\'on de $F$. Entonces 

\begin{eqnarray*}
\left\{N\left(t\right)\geq n\right\}&=&\left\{T_{n}\leq t\right\}\\
P\left\{N\left(t\right)\leq n\right\}&=&1-F^{\left(n+1\right)\star}\left(t\right)
\end{eqnarray*}

Adem\'as usando el hecho de que $\esp\left[N\left(t\right)\right]=\sum_{n=1}^{\infty}P\left\{N\left(t\right)\geq n\right\}$
se tiene que

\begin{eqnarray*}
\esp\left[N\left(t\right)\right]=\sum_{n=1}^{\infty}F^{n\star}\left(t\right)
\end{eqnarray*}

\begin{Prop}
Para cada $t\geq0$, la funci\'on generadora de momentos $\esp\left[e^{\alpha N\left(t\right)}\right]$ existe para alguna $\alpha$ en una vecindad del 0, y de aqu\'i que $\esp\left[N\left(t\right)^{m}\right]<\infty$, para $m\geq1$.
\end{Prop}


\begin{Note}
Si el primer tiempo de renovaci\'on $\xi_{1}$ no tiene la misma distribuci\'on que el resto de las $\xi_{n}$, para $n\geq2$, a $N\left(t\right)$ se le llama Proceso de Renovaci\'on retardado, donde si $\xi$ tiene distribuci\'on $G$, entonces el tiempo $T_{n}$ de la $n$-\'esima renovaci\'on tiene distribuci\'on $G\star F^{\left(n-1\right)\star}\left(t\right)$
\end{Note}


\begin{Teo}
Para una constante $\mu\leq\infty$ ( o variable aleatoria), las siguientes expresiones son equivalentes:

\begin{eqnarray}
lim_{n\rightarrow\infty}n^{-1}T_{n}&=&\mu,\textrm{ c.s.}\\
lim_{t\rightarrow\infty}t^{-1}N\left(t\right)&=&1/\mu,\textrm{ c.s.}
\end{eqnarray}
\end{Teo}


Es decir, $T_{n}$ satisface la Ley Fuerte de los Grandes N\'umeros s\'i y s\'olo s\'i $N\left/t\right)$ la cumple.


\begin{Coro}[Ley Fuerte de los Grandes N\'umeros para Procesos de Renovaci\'on]
Si $N\left(t\right)$ es un proceso de renovaci\'on cuyos tiempos de inter-renovaci\'on tienen media $\mu\leq\infty$, entonces
\begin{eqnarray}
t^{-1}N\left(t\right)\rightarrow 1/\mu,\textrm{ c.s. cuando }t\rightarrow\infty.
\end{eqnarray}

\end{Coro}


Considerar el proceso estoc\'astico de valores reales $\left\{Z\left(t\right):t\geq0\right\}$ en el mismo espacio de probabilidad que $N\left(t\right)$

\begin{Def}
Para el proceso $\left\{Z\left(t\right):t\geq0\right\}$ se define la fluctuaci\'on m\'axima de $Z\left(t\right)$ en el intervalo $\left(T_{n-1},T_{n}\right]$:
\begin{eqnarray*}
M_{n}=\sup_{T_{n-1}<t\leq T_{n}}|Z\left(t\right)-Z\left(T_{n-1}\right)|
\end{eqnarray*}
\end{Def}

\begin{Teo}
Sup\'ongase que $n^{-1}T_{n}\rightarrow\mu$ c.s. cuando $n\rightarrow\infty$, donde $\mu\leq\infty$ es una constante o variable aleatoria. Sea $a$ una constante o variable aleatoria que puede ser infinita cuando $\mu$ es finita, y considere las expresiones l\'imite:
\begin{eqnarray}
lim_{n\rightarrow\infty}n^{-1}Z\left(T_{n}\right)&=&a,\textrm{ c.s.}\\
lim_{t\rightarrow\infty}t^{-1}Z\left(t\right)&=&a/\mu,\textrm{ c.s.}
\end{eqnarray}
La segunda expresi\'on implica la primera. Conversamente, la primera implica la segunda si el proceso $Z\left(t\right)$ es creciente, o si $lim_{n\rightarrow\infty}n^{-1}M_{n}=0$ c.s.
\end{Teo}

\begin{Coro}
Si $N\left(t\right)$ es un proceso de renovaci\'on, y $\left(Z\left(T_{n}\right)-Z\left(T_{n-1}\right),M_{n}\right)$, para $n\geq1$, son variables aleatorias independientes e id\'enticamente distribuidas con media finita, entonces,
\begin{eqnarray}
lim_{t\rightarrow\infty}t^{-1}Z\left(t\right)\rightarrow\frac{\esp\left[Z\left(T_{1}\right)-Z\left(T_{0}\right)\right]}{\esp\left[T_{1}\right]},\textrm{ c.s. cuando  }t\rightarrow\infty.
\end{eqnarray}
\end{Coro}


%___________________________________________________________________________________________
%
%\subsection{Funci\'on de Renovaci\'on}
%___________________________________________________________________________________________
%


\begin{Def}
Sea $h\left(t\right)$ funci\'on de valores reales en $\rea$ acotada en intervalos finitos e igual a cero para $t<0$ La ecuaci\'on de renovaci\'on para $h\left(t\right)$ y la distribuci\'on $F$ es

\begin{eqnarray}%\label{Ec.Renovacion}
H\left(t\right)=h\left(t\right)+\int_{\left[0,t\right]}H\left(t-s\right)dF\left(s\right)\textrm{,    }t\geq0,
\end{eqnarray}
donde $H\left(t\right)$ es una funci\'on de valores reales. Esto es $H=h+F\star H$. Decimos que $H\left(t\right)$ es soluci\'on de esta ecuaci\'on si satisface la ecuaci\'on, y es acotada en intervalos finitos e iguales a cero para $t<0$.
\end{Def}

\begin{Prop}
La funci\'on $U\star h\left(t\right)$ es la \'unica soluci\'on de la ecuaci\'on de renovaci\'on (\ref{Ec.Renovacion}).
\end{Prop}

\begin{Teo}[Teorema Renovaci\'on Elemental]
\begin{eqnarray*}
t^{-1}U\left(t\right)\rightarrow 1/\mu\textrm{,    cuando }t\rightarrow\infty.
\end{eqnarray*}
\end{Teo}

%___________________________________________________________________________________________
%
%\subsection{Funci\'on de Renovaci\'on}
%___________________________________________________________________________________________
%


Sup\'ongase que $N\left(t\right)$ es un proceso de renovaci\'on con distribuci\'on $F$ con media finita $\mu$.

\begin{Def}
La funci\'on de renovaci\'on asociada con la distribuci\'on $F$, del proceso $N\left(t\right)$, es
\begin{eqnarray*}
U\left(t\right)=\sum_{n=1}^{\infty}F^{n\star}\left(t\right),\textrm{   }t\geq0,
\end{eqnarray*}
donde $F^{0\star}\left(t\right)=\indora\left(t\geq0\right)$.
\end{Def}


\begin{Prop}
Sup\'ongase que la distribuci\'on de inter-renovaci\'on $F$ tiene densidad $f$. Entonces $U\left(t\right)$ tambi\'en tiene densidad, para $t>0$, y es $U^{'}\left(t\right)=\sum_{n=0}^{\infty}f^{n\star}\left(t\right)$. Adem\'as
\begin{eqnarray*}
\prob\left\{N\left(t\right)>N\left(t-\right)\right\}=0\textrm{,   }t\geq0.
\end{eqnarray*}
\end{Prop}

\begin{Def}
La Transformada de Laplace-Stieljes de $F$ est\'a dada por

\begin{eqnarray*}
\hat{F}\left(\alpha\right)=\int_{\rea_{+}}e^{-\alpha t}dF\left(t\right)\textrm{,  }\alpha\geq0.
\end{eqnarray*}
\end{Def}

Entonces

\begin{eqnarray*}
\hat{U}\left(\alpha\right)=\sum_{n=0}^{\infty}\hat{F^{n\star}}\left(\alpha\right)=\sum_{n=0}^{\infty}\hat{F}\left(\alpha\right)^{n}=\frac{1}{1-\hat{F}\left(\alpha\right)}.
\end{eqnarray*}


\begin{Prop}
La Transformada de Laplace $\hat{U}\left(\alpha\right)$ y $\hat{F}\left(\alpha\right)$ determina una a la otra de manera \'unica por la relaci\'on $\hat{U}\left(\alpha\right)=\frac{1}{1-\hat{F}\left(\alpha\right)}$.
\end{Prop}


\begin{Note}
Un proceso de renovaci\'on $N\left(t\right)$ cuyos tiempos de inter-renovaci\'on tienen media finita, es un proceso Poisson con tasa $\lambda$ si y s\'olo s\'i $\esp\left[U\left(t\right)\right]=\lambda t$, para $t\geq0$.
\end{Note}


\begin{Teo}
Sea $N\left(t\right)$ un proceso puntual simple con puntos de localizaci\'on $T_{n}$ tal que $\eta\left(t\right)=\esp\left[N\left(\right)\right]$ es finita para cada $t$. Entonces para cualquier funci\'on $f:\rea_{+}\rightarrow\rea$,
\begin{eqnarray*}
\esp\left[\sum_{n=1}^{N\left(\right)}f\left(T_{n}\right)\right]=\int_{\left(0,t\right]}f\left(s\right)d\eta\left(s\right)\textrm{,  }t\geq0,
\end{eqnarray*}
suponiendo que la integral exista. Adem\'as si $X_{1},X_{2},\ldots$ son variables aleatorias definidas en el mismo espacio de probabilidad que el proceso $N\left(t\right)$ tal que $\esp\left[X_{n}|T_{n}=s\right]=f\left(s\right)$, independiente de $n$. Entonces
\begin{eqnarray*}
\esp\left[\sum_{n=1}^{N\left(t\right)}X_{n}\right]=\int_{\left(0,t\right]}f\left(s\right)d\eta\left(s\right)\textrm{,  }t\geq0,
\end{eqnarray*} 
suponiendo que la integral exista. 
\end{Teo}

\begin{Coro}[Identidad de Wald para Renovaciones]
Para el proceso de renovaci\'on $N\left(t\right)$,
\begin{eqnarray*}
\esp\left[T_{N\left(t\right)+1}\right]=\mu\esp\left[N\left(t\right)+1\right]\textrm{,  }t\geq0,
\end{eqnarray*}  
\end{Coro}

%______________________________________________________________________
%\subsection{Procesos de Renovaci\'on}
%______________________________________________________________________

\begin{Def}%\label{Def.Tn}
Sean $0\leq T_{1}\leq T_{2}\leq \ldots$ son tiempos aleatorios infinitos en los cuales ocurren ciertos eventos. El n\'umero de tiempos $T_{n}$ en el intervalo $\left[0,t\right)$ es

\begin{eqnarray}
N\left(t\right)=\sum_{n=1}^{\infty}\indora\left(T_{n}\leq t\right),
\end{eqnarray}
para $t\geq0$.
\end{Def}

Si se consideran los puntos $T_{n}$ como elementos de $\rea_{+}$, y $N\left(t\right)$ es el n\'umero de puntos en $\rea$. El proceso denotado por $\left\{N\left(t\right):t\geq0\right\}$, denotado por $N\left(t\right)$, es un proceso puntual en $\rea_{+}$. Los $T_{n}$ son los tiempos de ocurrencia, el proceso puntual $N\left(t\right)$ es simple si su n\'umero de ocurrencias son distintas: $0<T_{1}<T_{2}<\ldots$ casi seguramente.

\begin{Def}
Un proceso puntual $N\left(t\right)$ es un proceso de renovaci\'on si los tiempos de interocurrencia $\xi_{n}=T_{n}-T_{n-1}$, para $n\geq1$, son independientes e identicamente distribuidos con distribuci\'on $F$, donde $F\left(0\right)=0$ y $T_{0}=0$. Los $T_{n}$ son llamados tiempos de renovaci\'on, referente a la independencia o renovaci\'on de la informaci\'on estoc\'astica en estos tiempos. Los $\xi_{n}$ son los tiempos de inter-renovaci\'on, y $N\left(t\right)$ es el n\'umero de renovaciones en el intervalo $\left[0,t\right)$
\end{Def}


\begin{Note}
Para definir un proceso de renovaci\'on para cualquier contexto, solamente hay que especificar una distribuci\'on $F$, con $F\left(0\right)=0$, para los tiempos de inter-renovaci\'on. La funci\'on $F$ en turno degune las otra variables aleatorias. De manera formal, existe un espacio de probabilidad y una sucesi\'on de variables aleatorias $\xi_{1},\xi_{2},\ldots$ definidas en este con distribuci\'on $F$. Entonces las otras cantidades son $T_{n}=\sum_{k=1}^{n}\xi_{k}$ y $N\left(t\right)=\sum_{n=1}^{\infty}\indora\left(T_{n}\leq t\right)$, donde $T_{n}\rightarrow\infty$ casi seguramente por la Ley Fuerte de los Grandes Números.
\end{Note}

%___________________________________________________________________________________________
%
%\subsection{Renewal and Regenerative Processes: Serfozo\cite{Serfozo}}
%___________________________________________________________________________________________
%
\begin{Def}%\label{Def.Tn}
Sean $0\leq T_{1}\leq T_{2}\leq \ldots$ son tiempos aleatorios infinitos en los cuales ocurren ciertos eventos. El n\'umero de tiempos $T_{n}$ en el intervalo $\left[0,t\right)$ es

\begin{eqnarray}
N\left(t\right)=\sum_{n=1}^{\infty}\indora\left(T_{n}\leq t\right),
\end{eqnarray}
para $t\geq0$.
\end{Def}

Si se consideran los puntos $T_{n}$ como elementos de $\rea_{+}$, y $N\left(t\right)$ es el n\'umero de puntos en $\rea$. El proceso denotado por $\left\{N\left(t\right):t\geq0\right\}$, denotado por $N\left(t\right)$, es un proceso puntual en $\rea_{+}$. Los $T_{n}$ son los tiempos de ocurrencia, el proceso puntual $N\left(t\right)$ es simple si su n\'umero de ocurrencias son distintas: $0<T_{1}<T_{2}<\ldots$ casi seguramente.

\begin{Def}
Un proceso puntual $N\left(t\right)$ es un proceso de renovaci\'on si los tiempos de interocurrencia $\xi_{n}=T_{n}-T_{n-1}$, para $n\geq1$, son independientes e identicamente distribuidos con distribuci\'on $F$, donde $F\left(0\right)=0$ y $T_{0}=0$. Los $T_{n}$ son llamados tiempos de renovaci\'on, referente a la independencia o renovaci\'on de la informaci\'on estoc\'astica en estos tiempos. Los $\xi_{n}$ son los tiempos de inter-renovaci\'on, y $N\left(t\right)$ es el n\'umero de renovaciones en el intervalo $\left[0,t\right)$
\end{Def}


\begin{Note}
Para definir un proceso de renovaci\'on para cualquier contexto, solamente hay que especificar una distribuci\'on $F$, con $F\left(0\right)=0$, para los tiempos de inter-renovaci\'on. La funci\'on $F$ en turno degune las otra variables aleatorias. De manera formal, existe un espacio de probabilidad y una sucesi\'on de variables aleatorias $\xi_{1},\xi_{2},\ldots$ definidas en este con distribuci\'on $F$. Entonces las otras cantidades son $T_{n}=\sum_{k=1}^{n}\xi_{k}$ y $N\left(t\right)=\sum_{n=1}^{\infty}\indora\left(T_{n}\leq t\right)$, donde $T_{n}\rightarrow\infty$ casi seguramente por la Ley Fuerte de los Grandes N\'umeros.
\end{Note}







Los tiempos $T_{n}$ est\'an relacionados con los conteos de $N\left(t\right)$ por

\begin{eqnarray*}
\left\{N\left(t\right)\geq n\right\}&=&\left\{T_{n}\leq t\right\}\\
T_{N\left(t\right)}\leq &t&<T_{N\left(t\right)+1},
\end{eqnarray*}

adem\'as $N\left(T_{n}\right)=n$, y 

\begin{eqnarray*}
N\left(t\right)=\max\left\{n:T_{n}\leq t\right\}=\min\left\{n:T_{n+1}>t\right\}
\end{eqnarray*}

Por propiedades de la convoluci\'on se sabe que

\begin{eqnarray*}
P\left\{T_{n}\leq t\right\}=F^{n\star}\left(t\right)
\end{eqnarray*}
que es la $n$-\'esima convoluci\'on de $F$. Entonces 

\begin{eqnarray*}
\left\{N\left(t\right)\geq n\right\}&=&\left\{T_{n}\leq t\right\}\\
P\left\{N\left(t\right)\leq n\right\}&=&1-F^{\left(n+1\right)\star}\left(t\right)
\end{eqnarray*}

Adem\'as usando el hecho de que $\esp\left[N\left(t\right)\right]=\sum_{n=1}^{\infty}P\left\{N\left(t\right)\geq n\right\}$
se tiene que

\begin{eqnarray*}
\esp\left[N\left(t\right)\right]=\sum_{n=1}^{\infty}F^{n\star}\left(t\right)
\end{eqnarray*}

\begin{Prop}
Para cada $t\geq0$, la funci\'on generadora de momentos $\esp\left[e^{\alpha N\left(t\right)}\right]$ existe para alguna $\alpha$ en una vecindad del 0, y de aqu\'i que $\esp\left[N\left(t\right)^{m}\right]<\infty$, para $m\geq1$.
\end{Prop}

\begin{Ejem}[\textbf{Proceso Poisson}]

Suponga que se tienen tiempos de inter-renovaci\'on \textit{i.i.d.} del proceso de renovaci\'on $N\left(t\right)$ tienen distribuci\'on exponencial $F\left(t\right)=q-e^{-\lambda t}$ con tasa $\lambda$. Entonces $N\left(t\right)$ es un proceso Poisson con tasa $\lambda$.

\end{Ejem}


\begin{Note}
Si el primer tiempo de renovaci\'on $\xi_{1}$ no tiene la misma distribuci\'on que el resto de las $\xi_{n}$, para $n\geq2$, a $N\left(t\right)$ se le llama Proceso de Renovaci\'on retardado, donde si $\xi$ tiene distribuci\'on $G$, entonces el tiempo $T_{n}$ de la $n$-\'esima renovaci\'on tiene distribuci\'on $G\star F^{\left(n-1\right)\star}\left(t\right)$
\end{Note}


\begin{Teo}
Para una constante $\mu\leq\infty$ ( o variable aleatoria), las siguientes expresiones son equivalentes:

\begin{eqnarray}
lim_{n\rightarrow\infty}n^{-1}T_{n}&=&\mu,\textrm{ c.s.}\\
lim_{t\rightarrow\infty}t^{-1}N\left(t\right)&=&1/\mu,\textrm{ c.s.}
\end{eqnarray}
\end{Teo}


Es decir, $T_{n}$ satisface la Ley Fuerte de los Grandes N\'umeros s\'i y s\'olo s\'i $N\left/t\right)$ la cumple.


\begin{Coro}[Ley Fuerte de los Grandes N\'umeros para Procesos de Renovaci\'on]
Si $N\left(t\right)$ es un proceso de renovaci\'on cuyos tiempos de inter-renovaci\'on tienen media $\mu\leq\infty$, entonces
\begin{eqnarray}
t^{-1}N\left(t\right)\rightarrow 1/\mu,\textrm{ c.s. cuando }t\rightarrow\infty.
\end{eqnarray}

\end{Coro}


Considerar el proceso estoc\'astico de valores reales $\left\{Z\left(t\right):t\geq0\right\}$ en el mismo espacio de probabilidad que $N\left(t\right)$

\begin{Def}
Para el proceso $\left\{Z\left(t\right):t\geq0\right\}$ se define la fluctuaci\'on m\'axima de $Z\left(t\right)$ en el intervalo $\left(T_{n-1},T_{n}\right]$:
\begin{eqnarray*}
M_{n}=\sup_{T_{n-1}<t\leq T_{n}}|Z\left(t\right)-Z\left(T_{n-1}\right)|
\end{eqnarray*}
\end{Def}

\begin{Teo}
Sup\'ongase que $n^{-1}T_{n}\rightarrow\mu$ c.s. cuando $n\rightarrow\infty$, donde $\mu\leq\infty$ es una constante o variable aleatoria. Sea $a$ una constante o variable aleatoria que puede ser infinita cuando $\mu$ es finita, y considere las expresiones l\'imite:
\begin{eqnarray}
lim_{n\rightarrow\infty}n^{-1}Z\left(T_{n}\right)&=&a,\textrm{ c.s.}\\
lim_{t\rightarrow\infty}t^{-1}Z\left(t\right)&=&a/\mu,\textrm{ c.s.}
\end{eqnarray}
La segunda expresi\'on implica la primera. Conversamente, la primera implica la segunda si el proceso $Z\left(t\right)$ es creciente, o si $lim_{n\rightarrow\infty}n^{-1}M_{n}=0$ c.s.
\end{Teo}

\begin{Coro}
Si $N\left(t\right)$ es un proceso de renovaci\'on, y $\left(Z\left(T_{n}\right)-Z\left(T_{n-1}\right),M_{n}\right)$, para $n\geq1$, son variables aleatorias independientes e id\'enticamente distribuidas con media finita, entonces,
\begin{eqnarray}
lim_{t\rightarrow\infty}t^{-1}Z\left(t\right)\rightarrow\frac{\esp\left[Z\left(T_{1}\right)-Z\left(T_{0}\right)\right]}{\esp\left[T_{1}\right]},\textrm{ c.s. cuando  }t\rightarrow\infty.
\end{eqnarray}
\end{Coro}


Sup\'ongase que $N\left(t\right)$ es un proceso de renovaci\'on con distribuci\'on $F$ con media finita $\mu$.

\begin{Def}
La funci\'on de renovaci\'on asociada con la distribuci\'on $F$, del proceso $N\left(t\right)$, es
\begin{eqnarray*}
U\left(t\right)=\sum_{n=1}^{\infty}F^{n\star}\left(t\right),\textrm{   }t\geq0,
\end{eqnarray*}
donde $F^{0\star}\left(t\right)=\indora\left(t\geq0\right)$.
\end{Def}


\begin{Prop}
Sup\'ongase que la distribuci\'on de inter-renovaci\'on $F$ tiene densidad $f$. Entonces $U\left(t\right)$ tambi\'en tiene densidad, para $t>0$, y es $U^{'}\left(t\right)=\sum_{n=0}^{\infty}f^{n\star}\left(t\right)$. Adem\'as
\begin{eqnarray*}
\prob\left\{N\left(t\right)>N\left(t-\right)\right\}=0\textrm{,   }t\geq0.
\end{eqnarray*}
\end{Prop}

\begin{Def}
La Transformada de Laplace-Stieljes de $F$ est\'a dada por

\begin{eqnarray*}
\hat{F}\left(\alpha\right)=\int_{\rea_{+}}e^{-\alpha t}dF\left(t\right)\textrm{,  }\alpha\geq0.
\end{eqnarray*}
\end{Def}

Entonces

\begin{eqnarray*}
\hat{U}\left(\alpha\right)=\sum_{n=0}^{\infty}\hat{F^{n\star}}\left(\alpha\right)=\sum_{n=0}^{\infty}\hat{F}\left(\alpha\right)^{n}=\frac{1}{1-\hat{F}\left(\alpha\right)}.
\end{eqnarray*}


\begin{Prop}
La Transformada de Laplace $\hat{U}\left(\alpha\right)$ y $\hat{F}\left(\alpha\right)$ determina una a la otra de manera \'unica por la relaci\'on $\hat{U}\left(\alpha\right)=\frac{1}{1-\hat{F}\left(\alpha\right)}$.
\end{Prop}


\begin{Note}
Un proceso de renovaci\'on $N\left(t\right)$ cuyos tiempos de inter-renovaci\'on tienen media finita, es un proceso Poisson con tasa $\lambda$ si y s\'olo s\'i $\esp\left[U\left(t\right)\right]=\lambda t$, para $t\geq0$.
\end{Note}


\begin{Teo}
Sea $N\left(t\right)$ un proceso puntual simple con puntos de localizaci\'on $T_{n}$ tal que $\eta\left(t\right)=\esp\left[N\left(\right)\right]$ es finita para cada $t$. Entonces para cualquier funci\'on $f:\rea_{+}\rightarrow\rea$,
\begin{eqnarray*}
\esp\left[\sum_{n=1}^{N\left(\right)}f\left(T_{n}\right)\right]=\int_{\left(0,t\right]}f\left(s\right)d\eta\left(s\right)\textrm{,  }t\geq0,
\end{eqnarray*}
suponiendo que la integral exista. Adem\'as si $X_{1},X_{2},\ldots$ son variables aleatorias definidas en el mismo espacio de probabilidad que el proceso $N\left(t\right)$ tal que $\esp\left[X_{n}|T_{n}=s\right]=f\left(s\right)$, independiente de $n$. Entonces
\begin{eqnarray*}
\esp\left[\sum_{n=1}^{N\left(t\right)}X_{n}\right]=\int_{\left(0,t\right]}f\left(s\right)d\eta\left(s\right)\textrm{,  }t\geq0,
\end{eqnarray*} 
suponiendo que la integral exista. 
\end{Teo}

\begin{Coro}[Identidad de Wald para Renovaciones]
Para el proceso de renovaci\'on $N\left(t\right)$,
\begin{eqnarray*}
\esp\left[T_{N\left(t\right)+1}\right]=\mu\esp\left[N\left(t\right)+1\right]\textrm{,  }t\geq0,
\end{eqnarray*}  
\end{Coro}


\begin{Def}
Sea $h\left(t\right)$ funci\'on de valores reales en $\rea$ acotada en intervalos finitos e igual a cero para $t<0$ La ecuaci\'on de renovaci\'on para $h\left(t\right)$ y la distribuci\'on $F$ es

\begin{eqnarray}%\label{Ec.Renovacion}
H\left(t\right)=h\left(t\right)+\int_{\left[0,t\right]}H\left(t-s\right)dF\left(s\right)\textrm{,    }t\geq0,
\end{eqnarray}
donde $H\left(t\right)$ es una funci\'on de valores reales. Esto es $H=h+F\star H$. Decimos que $H\left(t\right)$ es soluci\'on de esta ecuaci\'on si satisface la ecuaci\'on, y es acotada en intervalos finitos e iguales a cero para $t<0$.
\end{Def}

\begin{Prop}
La funci\'on $U\star h\left(t\right)$ es la \'unica soluci\'on de la ecuaci\'on de renovaci\'on (\ref{Ec.Renovacion}).
\end{Prop}

\begin{Teo}[Teorema Renovaci\'on Elemental]
\begin{eqnarray*}
t^{-1}U\left(t\right)\rightarrow 1/\mu\textrm{,    cuando }t\rightarrow\infty.
\end{eqnarray*}
\end{Teo}



Sup\'ongase que $N\left(t\right)$ es un proceso de renovaci\'on con distribuci\'on $F$ con media finita $\mu$.

\begin{Def}
La funci\'on de renovaci\'on asociada con la distribuci\'on $F$, del proceso $N\left(t\right)$, es
\begin{eqnarray*}
U\left(t\right)=\sum_{n=1}^{\infty}F^{n\star}\left(t\right),\textrm{   }t\geq0,
\end{eqnarray*}
donde $F^{0\star}\left(t\right)=\indora\left(t\geq0\right)$.
\end{Def}


\begin{Prop}
Sup\'ongase que la distribuci\'on de inter-renovaci\'on $F$ tiene densidad $f$. Entonces $U\left(t\right)$ tambi\'en tiene densidad, para $t>0$, y es $U^{'}\left(t\right)=\sum_{n=0}^{\infty}f^{n\star}\left(t\right)$. Adem\'as
\begin{eqnarray*}
\prob\left\{N\left(t\right)>N\left(t-\right)\right\}=0\textrm{,   }t\geq0.
\end{eqnarray*}
\end{Prop}

\begin{Def}
La Transformada de Laplace-Stieljes de $F$ est\'a dada por

\begin{eqnarray*}
\hat{F}\left(\alpha\right)=\int_{\rea_{+}}e^{-\alpha t}dF\left(t\right)\textrm{,  }\alpha\geq0.
\end{eqnarray*}
\end{Def}

Entonces

\begin{eqnarray*}
\hat{U}\left(\alpha\right)=\sum_{n=0}^{\infty}\hat{F^{n\star}}\left(\alpha\right)=\sum_{n=0}^{\infty}\hat{F}\left(\alpha\right)^{n}=\frac{1}{1-\hat{F}\left(\alpha\right)}.
\end{eqnarray*}


\begin{Prop}
La Transformada de Laplace $\hat{U}\left(\alpha\right)$ y $\hat{F}\left(\alpha\right)$ determina una a la otra de manera \'unica por la relaci\'on $\hat{U}\left(\alpha\right)=\frac{1}{1-\hat{F}\left(\alpha\right)}$.
\end{Prop}


\begin{Note}
Un proceso de renovaci\'on $N\left(t\right)$ cuyos tiempos de inter-renovaci\'on tienen media finita, es un proceso Poisson con tasa $\lambda$ si y s\'olo s\'i $\esp\left[U\left(t\right)\right]=\lambda t$, para $t\geq0$.
\end{Note}


\begin{Teo}
Sea $N\left(t\right)$ un proceso puntual simple con puntos de localizaci\'on $T_{n}$ tal que $\eta\left(t\right)=\esp\left[N\left(\right)\right]$ es finita para cada $t$. Entonces para cualquier funci\'on $f:\rea_{+}\rightarrow\rea$,
\begin{eqnarray*}
\esp\left[\sum_{n=1}^{N\left(\right)}f\left(T_{n}\right)\right]=\int_{\left(0,t\right]}f\left(s\right)d\eta\left(s\right)\textrm{,  }t\geq0,
\end{eqnarray*}
suponiendo que la integral exista. Adem\'as si $X_{1},X_{2},\ldots$ son variables aleatorias definidas en el mismo espacio de probabilidad que el proceso $N\left(t\right)$ tal que $\esp\left[X_{n}|T_{n}=s\right]=f\left(s\right)$, independiente de $n$. Entonces
\begin{eqnarray*}
\esp\left[\sum_{n=1}^{N\left(t\right)}X_{n}\right]=\int_{\left(0,t\right]}f\left(s\right)d\eta\left(s\right)\textrm{,  }t\geq0,
\end{eqnarray*} 
suponiendo que la integral exista. 
\end{Teo}

\begin{Coro}[Identidad de Wald para Renovaciones]
Para el proceso de renovaci\'on $N\left(t\right)$,
\begin{eqnarray*}
\esp\left[T_{N\left(t\right)+1}\right]=\mu\esp\left[N\left(t\right)+1\right]\textrm{,  }t\geq0,
\end{eqnarray*}  
\end{Coro}


\begin{Def}
Sea $h\left(t\right)$ funci\'on de valores reales en $\rea$ acotada en intervalos finitos e igual a cero para $t<0$ La ecuaci\'on de renovaci\'on para $h\left(t\right)$ y la distribuci\'on $F$ es

\begin{eqnarray}%\label{Ec.Renovacion}
H\left(t\right)=h\left(t\right)+\int_{\left[0,t\right]}H\left(t-s\right)dF\left(s\right)\textrm{,    }t\geq0,
\end{eqnarray}
donde $H\left(t\right)$ es una funci\'on de valores reales. Esto es $H=h+F\star H$. Decimos que $H\left(t\right)$ es soluci\'on de esta ecuaci\'on si satisface la ecuaci\'on, y es acotada en intervalos finitos e iguales a cero para $t<0$.
\end{Def}

\begin{Prop}
La funci\'on $U\star h\left(t\right)$ es la \'unica soluci\'on de la ecuaci\'on de renovaci\'on (\ref{Ec.Renovacion}).
\end{Prop}

\begin{Teo}[Teorema Renovaci\'on Elemental]
\begin{eqnarray*}
t^{-1}U\left(t\right)\rightarrow 1/\mu\textrm{,    cuando }t\rightarrow\infty.
\end{eqnarray*}
\end{Teo}


\begin{Note} Una funci\'on $h:\rea_{+}\rightarrow\rea$ es Directamente Riemann Integrable en los siguientes casos:
\begin{itemize}
\item[a)] $h\left(t\right)\geq0$ es decreciente y Riemann Integrable.
\item[b)] $h$ es continua excepto posiblemente en un conjunto de Lebesgue de medida 0, y $|h\left(t\right)|\leq b\left(t\right)$, donde $b$ es DRI.
\end{itemize}
\end{Note}

\begin{Teo}[Teorema Principal de Renovaci\'on]
Si $F$ es no aritm\'etica y $h\left(t\right)$ es Directamente Riemann Integrable (DRI), entonces

\begin{eqnarray*}
lim_{t\rightarrow\infty}U\star h=\frac{1}{\mu}\int_{\rea_{+}}h\left(s\right)ds.
\end{eqnarray*}
\end{Teo}

\begin{Prop}
Cualquier funci\'on $H\left(t\right)$ acotada en intervalos finitos y que es 0 para $t<0$ puede expresarse como
\begin{eqnarray*}
H\left(t\right)=U\star h\left(t\right)\textrm{,  donde }h\left(t\right)=H\left(t\right)-F\star H\left(t\right)
\end{eqnarray*}
\end{Prop}

\begin{Def}
Un proceso estoc\'astico $X\left(t\right)$ es crudamente regenerativo en un tiempo aleatorio positivo $T$ si
\begin{eqnarray*}
\esp\left[X\left(T+t\right)|T\right]=\esp\left[X\left(t\right)\right]\textrm{, para }t\geq0,\end{eqnarray*}
y con las esperanzas anteriores finitas.
\end{Def}

\begin{Prop}
Sup\'ongase que $X\left(t\right)$ es un proceso crudamente regenerativo en $T$, que tiene distribuci\'on $F$. Si $\esp\left[X\left(t\right)\right]$ es acotado en intervalos finitos, entonces
\begin{eqnarray*}
\esp\left[X\left(t\right)\right]=U\star h\left(t\right)\textrm{,  donde }h\left(t\right)=\esp\left[X\left(t\right)\indora\left(T>t\right)\right].
\end{eqnarray*}
\end{Prop}

\begin{Teo}[Regeneraci\'on Cruda]
Sup\'ongase que $X\left(t\right)$ es un proceso con valores positivo crudamente regenerativo en $T$, y def\'inase $M=\sup\left\{|X\left(t\right)|:t\leq T\right\}$. Si $T$ es no aritm\'etico y $M$ y $MT$ tienen media finita, entonces
\begin{eqnarray*}
lim_{t\rightarrow\infty}\esp\left[X\left(t\right)\right]=\frac{1}{\mu}\int_{\rea_{+}}h\left(s\right)ds,
\end{eqnarray*}
donde $h\left(t\right)=\esp\left[X\left(t\right)\indora\left(T>t\right)\right]$.
\end{Teo}


\begin{Note} Una funci\'on $h:\rea_{+}\rightarrow\rea$ es Directamente Riemann Integrable en los siguientes casos:
\begin{itemize}
\item[a)] $h\left(t\right)\geq0$ es decreciente y Riemann Integrable.
\item[b)] $h$ es continua excepto posiblemente en un conjunto de Lebesgue de medida 0, y $|h\left(t\right)|\leq b\left(t\right)$, donde $b$ es DRI.
\end{itemize}
\end{Note}

\begin{Teo}[Teorema Principal de Renovaci\'on]
Si $F$ es no aritm\'etica y $h\left(t\right)$ es Directamente Riemann Integrable (DRI), entonces

\begin{eqnarray*}
lim_{t\rightarrow\infty}U\star h=\frac{1}{\mu}\int_{\rea_{+}}h\left(s\right)ds.
\end{eqnarray*}
\end{Teo}

\begin{Prop}
Cualquier funci\'on $H\left(t\right)$ acotada en intervalos finitos y que es 0 para $t<0$ puede expresarse como
\begin{eqnarray*}
H\left(t\right)=U\star h\left(t\right)\textrm{,  donde }h\left(t\right)=H\left(t\right)-F\star H\left(t\right)
\end{eqnarray*}
\end{Prop}

\begin{Def}
Un proceso estoc\'astico $X\left(t\right)$ es crudamente regenerativo en un tiempo aleatorio positivo $T$ si
\begin{eqnarray*}
\esp\left[X\left(T+t\right)|T\right]=\esp\left[X\left(t\right)\right]\textrm{, para }t\geq0,\end{eqnarray*}
y con las esperanzas anteriores finitas.
\end{Def}

\begin{Prop}
Sup\'ongase que $X\left(t\right)$ es un proceso crudamente regenerativo en $T$, que tiene distribuci\'on $F$. Si $\esp\left[X\left(t\right)\right]$ es acotado en intervalos finitos, entonces
\begin{eqnarray*}
\esp\left[X\left(t\right)\right]=U\star h\left(t\right)\textrm{,  donde }h\left(t\right)=\esp\left[X\left(t\right)\indora\left(T>t\right)\right].
\end{eqnarray*}
\end{Prop}

\begin{Teo}[Regeneraci\'on Cruda]
Sup\'ongase que $X\left(t\right)$ es un proceso con valores positivo crudamente regenerativo en $T$, y def\'inase $M=\sup\left\{|X\left(t\right)|:t\leq T\right\}$. Si $T$ es no aritm\'etico y $M$ y $MT$ tienen media finita, entonces
\begin{eqnarray*}
lim_{t\rightarrow\infty}\esp\left[X\left(t\right)\right]=\frac{1}{\mu}\int_{\rea_{+}}h\left(s\right)ds,
\end{eqnarray*}
donde $h\left(t\right)=\esp\left[X\left(t\right)\indora\left(T>t\right)\right]$.
\end{Teo}

\begin{Def}
Para el proceso $\left\{\left(N\left(t\right),X\left(t\right)\right):t\geq0\right\}$, sus trayectoria muestrales en el intervalo de tiempo $\left[T_{n-1},T_{n}\right)$ est\'an descritas por
\begin{eqnarray*}
\zeta_{n}=\left(\xi_{n},\left\{X\left(T_{n-1}+t\right):0\leq t<\xi_{n}\right\}\right)
\end{eqnarray*}
Este $\zeta_{n}$ es el $n$-\'esimo segmento del proceso. El proceso es regenerativo sobre los tiempos $T_{n}$ si sus segmentos $\zeta_{n}$ son independientes e id\'enticamennte distribuidos.
\end{Def}


\begin{Note}
Si $\tilde{X}\left(t\right)$ con espacio de estados $\tilde{S}$ es regenerativo sobre $T_{n}$, entonces $X\left(t\right)=f\left(\tilde{X}\left(t\right)\right)$ tambi\'en es regenerativo sobre $T_{n}$, para cualquier funci\'on $f:\tilde{S}\rightarrow S$.
\end{Note}

\begin{Note}
Los procesos regenerativos son crudamente regenerativos, pero no al rev\'es.
\end{Note}


\begin{Note}
Un proceso estoc\'astico a tiempo continuo o discreto es regenerativo si existe un proceso de renovaci\'on  tal que los segmentos del proceso entre tiempos de renovaci\'on sucesivos son i.i.d., es decir, para $\left\{X\left(t\right):t\geq0\right\}$ proceso estoc\'astico a tiempo continuo con espacio de estados $S$, espacio m\'etrico.
\end{Note}

Para $\left\{X\left(t\right):t\geq0\right\}$ Proceso Estoc\'astico a tiempo continuo con estado de espacios $S$, que es un espacio m\'etrico, con trayectorias continuas por la derecha y con l\'imites por la izquierda c.s. Sea $N\left(t\right)$ un proceso de renovaci\'on en $\rea_{+}$ definido en el mismo espacio de probabilidad que $X\left(t\right)$, con tiempos de renovaci\'on $T$ y tiempos de inter-renovaci\'on $\xi_{n}=T_{n}-T_{n-1}$, con misma distribuci\'on $F$ de media finita $\mu$.



\begin{Def}
Para el proceso $\left\{\left(N\left(t\right),X\left(t\right)\right):t\geq0\right\}$, sus trayectoria muestrales en el intervalo de tiempo $\left[T_{n-1},T_{n}\right)$ est\'an descritas por
\begin{eqnarray*}
\zeta_{n}=\left(\xi_{n},\left\{X\left(T_{n-1}+t\right):0\leq t<\xi_{n}\right\}\right)
\end{eqnarray*}
Este $\zeta_{n}$ es el $n$-\'esimo segmento del proceso. El proceso es regenerativo sobre los tiempos $T_{n}$ si sus segmentos $\zeta_{n}$ son independientes e id\'enticamennte distribuidos.
\end{Def}

\begin{Note}
Un proceso regenerativo con media de la longitud de ciclo finita es llamado positivo recurrente.
\end{Note}

\begin{Teo}[Procesos Regenerativos]
Suponga que el proceso
\end{Teo}


\begin{Def}[Renewal Process Trinity]
Para un proceso de renovaci\'on $N\left(t\right)$, los siguientes procesos proveen de informaci\'on sobre los tiempos de renovaci\'on.
\begin{itemize}
\item $A\left(t\right)=t-T_{N\left(t\right)}$, el tiempo de recurrencia hacia atr\'as al tiempo $t$, que es el tiempo desde la \'ultima renovaci\'on para $t$.

\item $B\left(t\right)=T_{N\left(t\right)+1}-t$, el tiempo de recurrencia hacia adelante al tiempo $t$, residual del tiempo de renovaci\'on, que es el tiempo para la pr\'oxima renovaci\'on despu\'es de $t$.

\item $L\left(t\right)=\xi_{N\left(t\right)+1}=A\left(t\right)+B\left(t\right)$, la longitud del intervalo de renovaci\'on que contiene a $t$.
\end{itemize}
\end{Def}

\begin{Note}
El proceso tridimensional $\left(A\left(t\right),B\left(t\right),L\left(t\right)\right)$ es regenerativo sobre $T_{n}$, y por ende cada proceso lo es. Cada proceso $A\left(t\right)$ y $B\left(t\right)$ son procesos de MArkov a tiempo continuo con trayectorias continuas por partes en el espacio de estados $\rea_{+}$. Una expresi\'on conveniente para su distribuci\'on conjunta es, para $0\leq x<t,y\geq0$
\begin{equation}\label{NoRenovacion}
P\left\{A\left(t\right)>x,B\left(t\right)>y\right\}=
P\left\{N\left(t+y\right)-N\left((t-x)\right)=0\right\}
\end{equation}
\end{Note}

\begin{Ejem}[Tiempos de recurrencia Poisson]
Si $N\left(t\right)$ es un proceso Poisson con tasa $\lambda$, entonces de la expresi\'on (\ref{NoRenovacion}) se tiene que

\begin{eqnarray*}
\begin{array}{lc}
P\left\{A\left(t\right)>x,B\left(t\right)>y\right\}=e^{-\lambda\left(x+y\right)},&0\leq x<t,y\geq0,
\end{array}
\end{eqnarray*}
que es la probabilidad Poisson de no renovaciones en un intervalo de longitud $x+y$.

\end{Ejem}

\begin{Note}
Una cadena de Markov erg\'odica tiene la propiedad de ser estacionaria si la distribuci\'on de su estado al tiempo $0$ es su distribuci\'on estacionaria.
\end{Note}


\begin{Def}
Un proceso estoc\'astico a tiempo continuo $\left\{X\left(t\right):t\geq0\right\}$ en un espacio general es estacionario si sus distribuciones finito dimensionales son invariantes bajo cualquier  traslado: para cada $0\leq s_{1}<s_{2}<\cdots<s_{k}$ y $t\geq0$,
\begin{eqnarray*}
\left(X\left(s_{1}+t\right),\ldots,X\left(s_{k}+t\right)\right)=_{d}\left(X\left(s_{1}\right),\ldots,X\left(s_{k}\right)\right).
\end{eqnarray*}
\end{Def}

\begin{Note}
Un proceso de Markov es estacionario si $X\left(t\right)=_{d}X\left(0\right)$, $t\geq0$.
\end{Note}

Considerese el proceso $N\left(t\right)=\sum_{n}\indora\left(\tau_{n}\leq t\right)$ en $\rea_{+}$, con puntos $0<\tau_{1}<\tau_{2}<\cdots$.

\begin{Prop}
Si $N$ es un proceso puntual estacionario y $\esp\left[N\left(1\right)\right]<\infty$, entonces $\esp\left[N\left(t\right)\right]=t\esp\left[N\left(1\right)\right]$, $t\geq0$

\end{Prop}

\begin{Teo}
Los siguientes enunciados son equivalentes
\begin{itemize}
\item[i)] El proceso retardado de renovaci\'on $N$ es estacionario.

\item[ii)] EL proceso de tiempos de recurrencia hacia adelante $B\left(t\right)$ es estacionario.


\item[iii)] $\esp\left[N\left(t\right)\right]=t/\mu$,


\item[iv)] $G\left(t\right)=F_{e}\left(t\right)=\frac{1}{\mu}\int_{0}^{t}\left[1-F\left(s\right)\right]ds$
\end{itemize}
Cuando estos enunciados son ciertos, $P\left\{B\left(t\right)\leq x\right\}=F_{e}\left(x\right)$, para $t,x\geq0$.

\end{Teo}

\begin{Note}
Una consecuencia del teorema anterior es que el Proceso Poisson es el \'unico proceso sin retardo que es estacionario.
\end{Note}

\begin{Coro}
El proceso de renovaci\'on $N\left(t\right)$ sin retardo, y cuyos tiempos de inter renonaci\'on tienen media finita, es estacionario si y s\'olo si es un proceso Poisson.

\end{Coro}


%________________________________________________________________________
%\subsection{Procesos Regenerativos}
%________________________________________________________________________



\begin{Note}
Si $\tilde{X}\left(t\right)$ con espacio de estados $\tilde{S}$ es regenerativo sobre $T_{n}$, entonces $X\left(t\right)=f\left(\tilde{X}\left(t\right)\right)$ tambi\'en es regenerativo sobre $T_{n}$, para cualquier funci\'on $f:\tilde{S}\rightarrow S$.
\end{Note}

\begin{Note}
Los procesos regenerativos son crudamente regenerativos, pero no al rev\'es.
\end{Note}
%\subsection*{Procesos Regenerativos: Sigman\cite{Sigman1}}
\begin{Def}[Definici\'on Cl\'asica]
Un proceso estoc\'astico $X=\left\{X\left(t\right):t\geq0\right\}$ es llamado regenerativo is existe una variable aleatoria $R_{1}>0$ tal que
\begin{itemize}
\item[i)] $\left\{X\left(t+R_{1}\right):t\geq0\right\}$ es independiente de $\left\{\left\{X\left(t\right):t<R_{1}\right\},\right\}$
\item[ii)] $\left\{X\left(t+R_{1}\right):t\geq0\right\}$ es estoc\'asticamente equivalente a $\left\{X\left(t\right):t>0\right\}$
\end{itemize}

Llamamos a $R_{1}$ tiempo de regeneraci\'on, y decimos que $X$ se regenera en este punto.
\end{Def}

$\left\{X\left(t+R_{1}\right)\right\}$ es regenerativo con tiempo de regeneraci\'on $R_{2}$, independiente de $R_{1}$ pero con la misma distribuci\'on que $R_{1}$. Procediendo de esta manera se obtiene una secuencia de variables aleatorias independientes e id\'enticamente distribuidas $\left\{R_{n}\right\}$ llamados longitudes de ciclo. Si definimos a $Z_{k}\equiv R_{1}+R_{2}+\cdots+R_{k}$, se tiene un proceso de renovaci\'on llamado proceso de renovaci\'on encajado para $X$.




\begin{Def}
Para $x$ fijo y para cada $t\geq0$, sea $I_{x}\left(t\right)=1$ si $X\left(t\right)\leq x$,  $I_{x}\left(t\right)=0$ en caso contrario, y def\'inanse los tiempos promedio
\begin{eqnarray*}
\overline{X}&=&lim_{t\rightarrow\infty}\frac{1}{t}\int_{0}^{\infty}X\left(u\right)du\\
\prob\left(X_{\infty}\leq x\right)&=&lim_{t\rightarrow\infty}\frac{1}{t}\int_{0}^{\infty}I_{x}\left(u\right)du,
\end{eqnarray*}
cuando estos l\'imites existan.
\end{Def}

Como consecuencia del teorema de Renovaci\'on-Recompensa, se tiene que el primer l\'imite  existe y es igual a la constante
\begin{eqnarray*}
\overline{X}&=&\frac{\esp\left[\int_{0}^{R_{1}}X\left(t\right)dt\right]}{\esp\left[R_{1}\right]},
\end{eqnarray*}
suponiendo que ambas esperanzas son finitas.

\begin{Note}
\begin{itemize}
\item[a)] Si el proceso regenerativo $X$ es positivo recurrente y tiene trayectorias muestrales no negativas, entonces la ecuaci\'on anterior es v\'alida.
\item[b)] Si $X$ es positivo recurrente regenerativo, podemos construir una \'unica versi\'on estacionaria de este proceso, $X_{e}=\left\{X_{e}\left(t\right)\right\}$, donde $X_{e}$ es un proceso estoc\'astico regenerativo y estrictamente estacionario, con distribuci\'on marginal distribuida como $X_{\infty}$
\end{itemize}
\end{Note}

%________________________________________________________________________
%\subsection{Procesos Regenerativos}
%________________________________________________________________________

Para $\left\{X\left(t\right):t\geq0\right\}$ Proceso Estoc\'astico a tiempo continuo con estado de espacios $S$, que es un espacio m\'etrico, con trayectorias continuas por la derecha y con l\'imites por la izquierda c.s. Sea $N\left(t\right)$ un proceso de renovaci\'on en $\rea_{+}$ definido en el mismo espacio de probabilidad que $X\left(t\right)$, con tiempos de renovaci\'on $T$ y tiempos de inter-renovaci\'on $\xi_{n}=T_{n}-T_{n-1}$, con misma distribuci\'on $F$ de media finita $\mu$.



\begin{Def}
Para el proceso $\left\{\left(N\left(t\right),X\left(t\right)\right):t\geq0\right\}$, sus trayectoria muestrales en el intervalo de tiempo $\left[T_{n-1},T_{n}\right)$ est\'an descritas por
\begin{eqnarray*}
\zeta_{n}=\left(\xi_{n},\left\{X\left(T_{n-1}+t\right):0\leq t<\xi_{n}\right\}\right)
\end{eqnarray*}
Este $\zeta_{n}$ es el $n$-\'esimo segmento del proceso. El proceso es regenerativo sobre los tiempos $T_{n}$ si sus segmentos $\zeta_{n}$ son independientes e id\'enticamennte distribuidos.
\end{Def}


\begin{Note}
Si $\tilde{X}\left(t\right)$ con espacio de estados $\tilde{S}$ es regenerativo sobre $T_{n}$, entonces $X\left(t\right)=f\left(\tilde{X}\left(t\right)\right)$ tambi\'en es regenerativo sobre $T_{n}$, para cualquier funci\'on $f:\tilde{S}\rightarrow S$.
\end{Note}

\begin{Note}
Los procesos regenerativos son crudamente regenerativos, pero no al rev\'es.
\end{Note}

\begin{Def}[Definici\'on Cl\'asica]
Un proceso estoc\'astico $X=\left\{X\left(t\right):t\geq0\right\}$ es llamado regenerativo is existe una variable aleatoria $R_{1}>0$ tal que
\begin{itemize}
\item[i)] $\left\{X\left(t+R_{1}\right):t\geq0\right\}$ es independiente de $\left\{\left\{X\left(t\right):t<R_{1}\right\},\right\}$
\item[ii)] $\left\{X\left(t+R_{1}\right):t\geq0\right\}$ es estoc\'asticamente equivalente a $\left\{X\left(t\right):t>0\right\}$
\end{itemize}

Llamamos a $R_{1}$ tiempo de regeneraci\'on, y decimos que $X$ se regenera en este punto.
\end{Def}

$\left\{X\left(t+R_{1}\right)\right\}$ es regenerativo con tiempo de regeneraci\'on $R_{2}$, independiente de $R_{1}$ pero con la misma distribuci\'on que $R_{1}$. Procediendo de esta manera se obtiene una secuencia de variables aleatorias independientes e id\'enticamente distribuidas $\left\{R_{n}\right\}$ llamados longitudes de ciclo. Si definimos a $Z_{k}\equiv R_{1}+R_{2}+\cdots+R_{k}$, se tiene un proceso de renovaci\'on llamado proceso de renovaci\'on encajado para $X$.

\begin{Note}
Un proceso regenerativo con media de la longitud de ciclo finita es llamado positivo recurrente.
\end{Note}


\begin{Def}
Para $x$ fijo y para cada $t\geq0$, sea $I_{x}\left(t\right)=1$ si $X\left(t\right)\leq x$,  $I_{x}\left(t\right)=0$ en caso contrario, y def\'inanse los tiempos promedio
\begin{eqnarray*}
\overline{X}&=&lim_{t\rightarrow\infty}\frac{1}{t}\int_{0}^{\infty}X\left(u\right)du\\
\prob\left(X_{\infty}\leq x\right)&=&lim_{t\rightarrow\infty}\frac{1}{t}\int_{0}^{\infty}I_{x}\left(u\right)du,
\end{eqnarray*}
cuando estos l\'imites existan.
\end{Def}

Como consecuencia del teorema de Renovaci\'on-Recompensa, se tiene que el primer l\'imite  existe y es igual a la constante
\begin{eqnarray*}
\overline{X}&=&\frac{\esp\left[\int_{0}^{R_{1}}X\left(t\right)dt\right]}{\esp\left[R_{1}\right]},
\end{eqnarray*}
suponiendo que ambas esperanzas son finitas.

\begin{Note}
\begin{itemize}
\item[a)] Si el proceso regenerativo $X$ es positivo recurrente y tiene trayectorias muestrales no negativas, entonces la ecuaci\'on anterior es v\'alida.
\item[b)] Si $X$ es positivo recurrente regenerativo, podemos construir una \'unica versi\'on estacionaria de este proceso, $X_{e}=\left\{X_{e}\left(t\right)\right\}$, donde $X_{e}$ es un proceso estoc\'astico regenerativo y estrictamente estacionario, con distribuci\'on marginal distribuida como $X_{\infty}$
\end{itemize}
\end{Note}

%__________________________________________________________________________________________
%\subsection{Procesos Regenerativos Estacionarios - Stidham \cite{Stidham}}
%__________________________________________________________________________________________


Un proceso estoc\'astico a tiempo continuo $\left\{V\left(t\right),t\geq0\right\}$ es un proceso regenerativo si existe una sucesi\'on de variables aleatorias independientes e id\'enticamente distribuidas $\left\{X_{1},X_{2},\ldots\right\}$, sucesi\'on de renovaci\'on, tal que para cualquier conjunto de Borel $A$, 

\begin{eqnarray*}
\prob\left\{V\left(t\right)\in A|X_{1}+X_{2}+\cdots+X_{R\left(t\right)}=s,\left\{V\left(\tau\right),\tau<s\right\}\right\}=\prob\left\{V\left(t-s\right)\in A|X_{1}>t-s\right\},
\end{eqnarray*}
para todo $0\leq s\leq t$, donde $R\left(t\right)=\max\left\{X_{1}+X_{2}+\cdots+X_{j}\leq t\right\}=$n\'umero de renovaciones ({\emph{puntos de regeneraci\'on}}) que ocurren en $\left[0,t\right]$. El intervalo $\left[0,X_{1}\right)$ es llamado {\emph{primer ciclo de regeneraci\'on}} de $\left\{V\left(t \right),t\geq0\right\}$, $\left[X_{1},X_{1}+X_{2}\right)$ el {\emph{segundo ciclo de regeneraci\'on}}, y as\'i sucesivamente.

Sea $X=X_{1}$ y sea $F$ la funci\'on de distrbuci\'on de $X$


\begin{Def}
Se define el proceso estacionario, $\left\{V^{*}\left(t\right),t\geq0\right\}$, para $\left\{V\left(t\right),t\geq0\right\}$ por

\begin{eqnarray*}
\prob\left\{V\left(t\right)\in A\right\}=\frac{1}{\esp\left[X\right]}\int_{0}^{\infty}\prob\left\{V\left(t+x\right)\in A|X>x\right\}\left(1-F\left(x\right)\right)dx,
\end{eqnarray*} 
para todo $t\geq0$ y todo conjunto de Borel $A$.
\end{Def}

\begin{Def}
Una distribuci\'on se dice que es {\emph{aritm\'etica}} si todos sus puntos de incremento son m\'ultiplos de la forma $0,\lambda, 2\lambda,\ldots$ para alguna $\lambda>0$ entera.
\end{Def}


\begin{Def}
Una modificaci\'on medible de un proceso $\left\{V\left(t\right),t\geq0\right\}$, es una versi\'on de este, $\left\{V\left(t,w\right)\right\}$ conjuntamente medible para $t\geq0$ y para $w\in S$, $S$ espacio de estados para $\left\{V\left(t\right),t\geq0\right\}$.
\end{Def}

\begin{Teo}
Sea $\left\{V\left(t\right),t\geq\right\}$ un proceso regenerativo no negativo con modificaci\'on medible. Sea $\esp\left[X\right]<\infty$. Entonces el proceso estacionario dado por la ecuaci\'on anterior est\'a bien definido y tiene funci\'on de distribuci\'on independiente de $t$, adem\'as
\begin{itemize}
\item[i)] \begin{eqnarray*}
\esp\left[V^{*}\left(0\right)\right]&=&\frac{\esp\left[\int_{0}^{X}V\left(s\right)ds\right]}{\esp\left[X\right]}\end{eqnarray*}
\item[ii)] Si $\esp\left[V^{*}\left(0\right)\right]<\infty$, equivalentemente, si $\esp\left[\int_{0}^{X}V\left(s\right)ds\right]<\infty$,entonces
\begin{eqnarray*}
\frac{\int_{0}^{t}V\left(s\right)ds}{t}\rightarrow\frac{\esp\left[\int_{0}^{X}V\left(s\right)ds\right]}{\esp\left[X\right]}
\end{eqnarray*}
con probabilidad 1 y en media, cuando $t\rightarrow\infty$.
\end{itemize}
\end{Teo}


%__________________________________________________________________________________________
%\subsection{Procesos Regenerativos Estacionarios - Stidham \cite{Stidham}}
%__________________________________________________________________________________________


Un proceso estoc\'astico a tiempo continuo $\left\{V\left(t\right),t\geq0\right\}$ es un proceso regenerativo si existe una sucesi\'on de variables aleatorias independientes e id\'enticamente distribuidas $\left\{X_{1},X_{2},\ldots\right\}$, sucesi\'on de renovaci\'on, tal que para cualquier conjunto de Borel $A$, 

\begin{eqnarray*}
\prob\left\{V\left(t\right)\in A|X_{1}+X_{2}+\cdots+X_{R\left(t\right)}=s,\left\{V\left(\tau\right),\tau<s\right\}\right\}=\prob\left\{V\left(t-s\right)\in A|X_{1}>t-s\right\},
\end{eqnarray*}
para todo $0\leq s\leq t$, donde $R\left(t\right)=\max\left\{X_{1}+X_{2}+\cdots+X_{j}\leq t\right\}=$n\'umero de renovaciones ({\emph{puntos de regeneraci\'on}}) que ocurren en $\left[0,t\right]$. El intervalo $\left[0,X_{1}\right)$ es llamado {\emph{primer ciclo de regeneraci\'on}} de $\left\{V\left(t \right),t\geq0\right\}$, $\left[X_{1},X_{1}+X_{2}\right)$ el {\emph{segundo ciclo de regeneraci\'on}}, y as\'i sucesivamente.

Sea $X=X_{1}$ y sea $F$ la funci\'on de distrbuci\'on de $X$


\begin{Def}
Se define el proceso estacionario, $\left\{V^{*}\left(t\right),t\geq0\right\}$, para $\left\{V\left(t\right),t\geq0\right\}$ por

\begin{eqnarray*}
\prob\left\{V\left(t\right)\in A\right\}=\frac{1}{\esp\left[X\right]}\int_{0}^{\infty}\prob\left\{V\left(t+x\right)\in A|X>x\right\}\left(1-F\left(x\right)\right)dx,
\end{eqnarray*} 
para todo $t\geq0$ y todo conjunto de Borel $A$.
\end{Def}

\begin{Def}
Una distribuci\'on se dice que es {\emph{aritm\'etica}} si todos sus puntos de incremento son m\'ultiplos de la forma $0,\lambda, 2\lambda,\ldots$ para alguna $\lambda>0$ entera.
\end{Def}


\begin{Def}
Una modificaci\'on medible de un proceso $\left\{V\left(t\right),t\geq0\right\}$, es una versi\'on de este, $\left\{V\left(t,w\right)\right\}$ conjuntamente medible para $t\geq0$ y para $w\in S$, $S$ espacio de estados para $\left\{V\left(t\right),t\geq0\right\}$.
\end{Def}

\begin{Teo}
Sea $\left\{V\left(t\right),t\geq\right\}$ un proceso regenerativo no negativo con modificaci\'on medible. Sea $\esp\left[X\right]<\infty$. Entonces el proceso estacionario dado por la ecuaci\'on anterior est\'a bien definido y tiene funci\'on de distribuci\'on independiente de $t$, adem\'as
\begin{itemize}
\item[i)] \begin{eqnarray*}
\esp\left[V^{*}\left(0\right)\right]&=&\frac{\esp\left[\int_{0}^{X}V\left(s\right)ds\right]}{\esp\left[X\right]}\end{eqnarray*}
\item[ii)] Si $\esp\left[V^{*}\left(0\right)\right]<\infty$, equivalentemente, si $\esp\left[\int_{0}^{X}V\left(s\right)ds\right]<\infty$,entonces
\begin{eqnarray*}
\frac{\int_{0}^{t}V\left(s\right)ds}{t}\rightarrow\frac{\esp\left[\int_{0}^{X}V\left(s\right)ds\right]}{\esp\left[X\right]}
\end{eqnarray*}
con probabilidad 1 y en media, cuando $t\rightarrow\infty$.
\end{itemize}
\end{Teo}

Para $\left\{X\left(t\right):t\geq0\right\}$ Proceso Estoc\'astico a tiempo continuo con estado de espacios $S$, que es un espacio m\'etrico, con trayectorias continuas por la derecha y con l\'imites por la izquierda c.s. Sea $N\left(t\right)$ un proceso de renovaci\'on en $\rea_{+}$ definido en el mismo espacio de probabilidad que $X\left(t\right)$, con tiempos de renovaci\'on $T$ y tiempos de inter-renovaci\'on $\xi_{n}=T_{n}-T_{n-1}$, con misma distribuci\'on $F$ de media finita $\mu$.



Sean $T_{1},T_{2},\ldots$ los puntos donde las longitudes de las colas de la red de sistemas de visitas c\'iclicas son cero simult\'aneamente, cuando la cola $Q_{j}$ es visitada por el servidor para dar servicio, es decir, $L_{1}\left(T_{i}\right)=0,L_{2}\left(T_{i}\right)=0,\hat{L}_{1}\left(T_{i}\right)=0$ y $\hat{L}_{2}\left(T_{i}\right)=0$, a estos puntos se les denominar\'a puntos regenerativos. Sea la funci\'on generadora de momentos para $L_{i}$, el n\'umero de usuarios en la cola $Q_{i}\left(z\right)$ en cualquier momento, est\'a dada por el tiempo promedio de $z^{L_{i}\left(t\right)}$ sobre el ciclo regenerativo definido anteriormente:

\begin{eqnarray*}
Q_{i}\left(z\right)&=&\esp\left[z^{L_{i}\left(t\right)}\right]=\frac{\esp\left[\sum_{m=1}^{M_{i}}\sum_{t=\tau_{i}\left(m\right)}^{\tau_{i}\left(m+1\right)-1}z^{L_{i}\left(t\right)}\right]}{\esp\left[\sum_{m=1}^{M_{i}}\tau_{i}\left(m+1\right)-\tau_{i}\left(m\right)\right]}
\end{eqnarray*}

$M_{i}$ es un tiempo de paro en el proceso regenerativo con $\esp\left[M_{i}\right]<\infty$\footnote{En Stidham\cite{Stidham} y Heyman  se muestra que una condici\'on suficiente para que el proceso regenerativo 
estacionario sea un procesoo estacionario es que el valor esperado del tiempo del ciclo regenerativo sea finito, es decir: $\esp\left[\sum_{m=1}^{M_{i}}C_{i}^{(m)}\right]<\infty$, como cada $C_{i}^{(m)}$ contiene intervalos de r\'eplica positivos, se tiene que $\esp\left[M_{i}\right]<\infty$, adem\'as, como $M_{i}>0$, se tiene que la condici\'on anterior es equivalente a tener que $\esp\left[C_{i}\right]<\infty$,
por lo tanto una condici\'on suficiente para la existencia del proceso regenerativo est\'a dada por $\sum_{k=1}^{N}\mu_{k}<1.$}, se sigue del lema de Wald que:


\begin{eqnarray*}
\esp\left[\sum_{m=1}^{M_{i}}\sum_{t=\tau_{i}\left(m\right)}^{\tau_{i}\left(m+1\right)-1}z^{L_{i}\left(t\right)}\right]&=&\esp\left[M_{i}\right]\esp\left[\sum_{t=\tau_{i}\left(m\right)}^{\tau_{i}\left(m+1\right)-1}z^{L_{i}\left(t\right)}\right]\\
\esp\left[\sum_{m=1}^{M_{i}}\tau_{i}\left(m+1\right)-\tau_{i}\left(m\right)\right]&=&\esp\left[M_{i}\right]\esp\left[\tau_{i}\left(m+1\right)-\tau_{i}\left(m\right)\right]
\end{eqnarray*}

por tanto se tiene que


\begin{eqnarray*}
Q_{i}\left(z\right)&=&\frac{\esp\left[\sum_{t=\tau_{i}\left(m\right)}^{\tau_{i}\left(m+1\right)-1}z^{L_{i}\left(t\right)}\right]}{\esp\left[\tau_{i}\left(m+1\right)-\tau_{i}\left(m\right)\right]}
\end{eqnarray*}

observar que el denominador es simplemente la duraci\'on promedio del tiempo del ciclo.


Haciendo las siguientes sustituciones en la ecuaci\'on (\ref{Corolario2}): $n\rightarrow t-\tau_{i}\left(m\right)$, $T \rightarrow \overline{\tau}_{i}\left(m\right)-\tau_{i}\left(m\right)$, $L_{n}\rightarrow L_{i}\left(t\right)$ y $F\left(z\right)=\esp\left[z^{L_{0}}\right]\rightarrow F_{i}\left(z\right)=\esp\left[z^{L_{i}\tau_{i}\left(m\right)}\right]$, se puede ver que

\begin{eqnarray}\label{Eq.Arribos.Primera}
\esp\left[\sum_{n=0}^{T-1}z^{L_{n}}\right]=
\esp\left[\sum_{t=\tau_{i}\left(m\right)}^{\overline{\tau}_{i}\left(m\right)-1}z^{L_{i}\left(t\right)}\right]
=z\frac{F_{i}\left(z\right)-1}{z-P_{i}\left(z\right)}
\end{eqnarray}

Por otra parte durante el tiempo de intervisita para la cola $i$, $L_{i}\left(t\right)$ solamente se incrementa de manera que el incremento por intervalo de tiempo est\'a dado por la funci\'on generadora de probabilidades de $P_{i}\left(z\right)$, por tanto la suma sobre el tiempo de intervisita puede evaluarse como:

\begin{eqnarray*}
\esp\left[\sum_{t=\tau_{i}\left(m\right)}^{\tau_{i}\left(m+1\right)-1}z^{L_{i}\left(t\right)}\right]&=&\esp\left[\sum_{t=\tau_{i}\left(m\right)}^{\tau_{i}\left(m+1\right)-1}\left\{P_{i}\left(z\right)\right\}^{t-\overline{\tau}_{i}\left(m\right)}\right]=\frac{1-\esp\left[\left\{P_{i}\left(z\right)\right\}^{\tau_{i}\left(m+1\right)-\overline{\tau}_{i}\left(m\right)}\right]}{1-P_{i}\left(z\right)}\\
&=&\frac{1-I_{i}\left[P_{i}\left(z\right)\right]}{1-P_{i}\left(z\right)}
\end{eqnarray*}
por tanto

\begin{eqnarray*}
\esp\left[\sum_{t=\tau_{i}\left(m\right)}^{\tau_{i}\left(m+1\right)-1}z^{L_{i}\left(t\right)}\right]&=&
\frac{1-F_{i}\left(z\right)}{1-P_{i}\left(z\right)}
\end{eqnarray*}

Por lo tanto

\begin{eqnarray*}
Q_{i}\left(z\right)&=&\frac{\esp\left[\sum_{t=\tau_{i}\left(m\right)}^{\tau_{i}
\left(m+1\right)-1}z^{L_{i}\left(t\right)}\right]}{\esp\left[\tau_{i}\left(m+1\right)-\tau_{i}\left(m\right)\right]}\\
&=&\frac{1}{\esp\left[\tau_{i}\left(m+1\right)-\tau_{i}\left(m\right)\right]}
\left\{
\esp\left[\sum_{t=\tau_{i}\left(m\right)}^{\overline{\tau}_{i}\left(m\right)-1}
z^{L_{i}\left(t\right)}\right]
+\esp\left[\sum_{t=\overline{\tau}_{i}\left(m\right)}^{\tau_{i}\left(m+1\right)-1}
z^{L_{i}\left(t\right)}\right]\right\}\\
&=&\frac{1}{\esp\left[\tau_{i}\left(m+1\right)-\tau_{i}\left(m\right)\right]}
\left\{
z\frac{F_{i}\left(z\right)-1}{z-P_{i}\left(z\right)}+\frac{1-F_{i}\left(z\right)}
{1-P_{i}\left(z\right)}
\right\}
\end{eqnarray*}

es decir

\begin{equation}
Q_{i}\left(z\right)=\frac{1}{\esp\left[C_{i}\right]}\cdot\frac{1-F_{i}\left(z\right)}{P_{i}\left(z\right)-z}\cdot\frac{\left(1-z\right)P_{i}\left(z\right)}{1-P_{i}\left(z\right)}
\end{equation}

\begin{Teo}
Dada una Red de Sistemas de Visitas C\'iclicas (RSVC), conformada por dos Sistemas de Visitas C\'iclicas (SVC), donde cada uno de ellos consta de dos colas tipo $M/M/1$. Los dos sistemas est\'an comunicados entre s\'i por medio de la transferencia de usuarios entre las colas $Q_{1}\leftrightarrow Q_{3}$ y $Q_{2}\leftrightarrow Q_{4}$. Se definen los eventos para los procesos de arribos al tiempo $t$, $A_{j}\left(t\right)=\left\{0 \textrm{ arribos en }Q_{j}\textrm{ al tiempo }t\right\}$ para alg\'un tiempo $t\geq0$ y $Q_{j}$ la cola $j$-\'esima en la RSVC, para $j=1,2,3,4$.  Existe un intervalo $I\neq\emptyset$ tal que para $T^{*}\in I$, tal que $\prob\left\{A_{1}\left(T^{*}\right),A_{2}\left(Tt^{*}\right),
A_{3}\left(T^{*}\right),A_{4}\left(T^{*}\right)|T^{*}\in I\right\}>0$.
\end{Teo}

\begin{proof}
Sin p\'erdida de generalidad podemos considerar como base del an\'alisis a la cola $Q_{1}$ del primer sistema que conforma la RSVC.

Sea $n>0$, ciclo en el primer sistema en el que se sabe que $L_{j}\left(\overline{\tau}_{1}\left(n\right)\right)=0$, pues la pol\'itica de servicio con que atienden los servidores es la exhaustiva. Como es sabido, para trasladarse a la siguiente cola, el servidor incurre en un tiempo de traslado $r_{1}\left(n\right)>0$, entonces tenemos que $\tau_{2}\left(n\right)=\overline{\tau}_{1}\left(n\right)+r_{1}\left(n\right)$.


Definamos el intervalo $I_{1}\equiv\left[\overline{\tau}_{1}\left(n\right),\tau_{2}\left(n\right)\right]$ de longitud $\xi_{1}=r_{1}\left(n\right)$. Dado que los tiempos entre arribo son exponenciales con tasa $\tilde{\mu}_{1}=\mu_{1}+\hat{\mu}_{1}$ ($\mu_{1}$ son los arribos a $Q_{1}$ por primera vez al sistema, mientras que $\hat{\mu}_{1}$ son los arribos de traslado procedentes de $Q_{3}$) se tiene que la probabilidad del evento $A_{1}\left(t\right)$ est\'a dada por 

\begin{equation}
\prob\left\{A_{1}\left(t\right)|t\in I_{1}\left(n\right)\right\}=e^{-\tilde{\mu}_{1}\xi_{1}\left(n\right)}.
\end{equation} 

Por otra parte, para la cola $Q_{2}$, el tiempo $\overline{\tau}_{2}\left(n-1\right)$ es tal que $L_{2}\left(\overline{\tau}_{2}\left(n-1\right)\right)=0$, es decir, es el tiempo en que la cola queda totalmente vac\'ia en el ciclo anterior a $n$. Entonces tenemos un sgundo intervalo $I_{2}\equiv\left[\overline{\tau}_{2}\left(n-1\right),\tau_{2}\left(n\right)\right]$. Por lo tanto la probabilidad del evento $A_{2}\left(t\right)$ tiene probabilidad dada por

\begin{equation}
\prob\left\{A_{2}\left(t\right)|t\in I_{2}\left(n\right)\right\}=e^{-\tilde{\mu}_{2}\xi_{2}\left(n\right)},
\end{equation} 

donde $\xi_{2}\left(n\right)=\tau_{2}\left(n\right)-\overline{\tau}_{2}\left(n-1\right)$.



Entonces, se tiene que

\begin{eqnarray*}
\prob\left\{A_{1}\left(t\right),A_{2}\left(t\right)|t\in I_{1}\left(n\right)\right\}&=&
\prob\left\{A_{1}\left(t\right)|t\in I_{1}\left(n\right)\right\}
\prob\left\{A_{2}\left(t\right)|t\in I_{1}\left(n\right)\right\}\\
&\geq&
\prob\left\{A_{1}\left(t\right)|t\in I_{1}\left(n\right)\right\}
\prob\left\{A_{2}\left(t\right)|t\in I_{2}\left(n\right)\right\}\\
&=&e^{-\tilde{\mu}_{1}\xi_{1}\left(n\right)}e^{-\tilde{\mu}_{2}\xi_{2}\left(n\right)}
=e^{-\left[\tilde{\mu}_{1}\xi_{1}\left(n\right)+\tilde{\mu}_{2}\xi_{2}\left(n\right)\right]}.
\end{eqnarray*}


es decir, 

\begin{equation}
\prob\left\{A_{1}\left(t\right),A_{2}\left(t\right)|t\in I_{1}\left(n\right)\right\}
=e^{-\left[\tilde{\mu}_{1}\xi_{1}\left(n\right)+\tilde{\mu}_{2}\xi_{2}
\left(n\right)\right]}>0.
\end{equation}

En lo que respecta a la relaci\'on entre los dos SVC que conforman la RSVC, se afirma que existe $m>0$ tal que $\overline{\tau}_{3}\left(m\right)<\tau_{2}\left(n\right)<\tau_{4}\left(m\right)$.

Para $Q_{3}$ sea $I_{3}=\left[\overline{\tau}_{3}\left(m\right),\tau_{4}\left(m\right)\right]$ con longitud  $\xi_{3}\left(m\right)=r_{3}\left(m\right)$, entonces 

\begin{equation}
\prob\left\{A_{3}\left(t\right)|t\in I_{3}\left(n\right)\right\}=e^{-\tilde{\mu}_{3}\xi_{3}\left(n\right)}.
\end{equation} 

An\'alogamente que como se hizo para $Q_{2}$, tenemos que para $Q_{4}$ se tiene el intervalo $I_{4}=\left[\overline{\tau}_{4}\left(m-1\right),\tau_{4}\left(m\right)\right]$ con longitud $\xi_{4}\left(m\right)=\tau_{4}\left(m\right)-\overline{\tau}_{4}\left(m-1\right)$, entonces


\begin{equation}
\prob\left\{A_{4}\left(t\right)|t\in I_{4}\left(m\right)\right\}=e^{-\tilde{\mu}_{4}\xi_{4}\left(n\right)}.
\end{equation} 

Al igual que para el primer sistema, dado que $I_{3}\left(m\right)\subset I_{4}\left(m\right)$, se tiene que

\begin{eqnarray*}
\xi_{3}\left(m\right)\leq\xi_{4}\left(m\right)&\Leftrightarrow& -\xi_{3}\left(m\right)\geq-\xi_{4}\left(m\right)
\\
-\tilde{\mu}_{4}\xi_{3}\left(m\right)\geq-\tilde{\mu}_{4}\xi_{4}\left(m\right)&\Leftrightarrow&
e^{-\tilde{\mu}_{4}\xi_{3}\left(m\right)}\geq e^{-\tilde{\mu}_{4}\xi_{4}\left(m\right)}\\
\prob\left\{A_{4}\left(t\right)|t\in I_{3}\left(m\right)\right\}&\geq&
\prob\left\{A_{4}\left(t\right)|t\in I_{4}\left(m\right)\right\}
\end{eqnarray*}

Entonces, dado que los eventos $A_{3}$ y $A_{4}$ son independientes, se tiene que

\begin{eqnarray*}
\prob\left\{A_{3}\left(t\right),A_{4}\left(t\right)|t\in I_{3}\left(m\right)\right\}&=&
\prob\left\{A_{3}\left(t\right)|t\in I_{3}\left(m\right)\right\}
\prob\left\{A_{4}\left(t\right)|t\in I_{3}\left(m\right)\right\}\\
&\geq&
\prob\left\{A_{3}\left(t\right)|t\in I_{3}\left(n\right)\right\}
\prob\left\{A_{4}\left(t\right)|t\in I_{4}\left(n\right)\right\}\\
&=&e^{-\tilde{\mu}_{3}\xi_{3}\left(m\right)}e^{-\tilde{\mu}_{4}\xi_{4}
\left(m\right)}
=e^{-\left[\tilde{\mu}_{3}\xi_{3}\left(m\right)+\tilde{\mu}_{4}\xi_{4}
\left(m\right)\right]}.
\end{eqnarray*}


es decir, 

\begin{equation}
\prob\left\{A_{3}\left(t\right),A_{4}\left(t\right)|t\in I_{3}\left(m\right)\right\}
=e^{-\left[\tilde{\mu}_{3}\xi_{3}\left(m\right)+\tilde{\mu}_{4}\xi_{4}
\left(m\right)\right]}>0.
\end{equation}

Por construcci\'on se tiene que $I\left(n,m\right)\equiv I_{1}\left(n\right)\cap I_{3}\left(m\right)\neq\emptyset$,entonces en particular se tienen las contenciones $I\left(n,m\right)\subseteq I_{1}\left(n\right)$ y $I\left(n,m\right)\subseteq I_{3}\left(m\right)$, por lo tanto si definimos $\xi_{n,m}\equiv\ell\left(I\left(n,m\right)\right)$ tenemos que

\begin{eqnarray*}
\xi_{n,m}\leq\xi_{1}\left(n\right)\textrm{ y }\xi_{n,m}\leq\xi_{3}\left(m\right)\textrm{ entonces }
-\xi_{n,m}\geq-\xi_{1}\left(n\right)\textrm{ y }-\xi_{n,m}\leq-\xi_{3}\left(m\right)\\
\end{eqnarray*}
por lo tanto tenemos las desigualdades 



\begin{eqnarray*}
\begin{array}{ll}
-\tilde{\mu}_{1}\xi_{n,m}\geq-\tilde{\mu}_{1}\xi_{1}\left(n\right),&
-\tilde{\mu}_{2}\xi_{n,m}\geq-\tilde{\mu}_{2}\xi_{1}\left(n\right)
\geq-\tilde{\mu}_{2}\xi_{2}\left(n\right),\\
-\tilde{\mu}_{3}\xi_{n,m}\geq-\tilde{\mu}_{3}\xi_{3}\left(m\right),&
-\tilde{\mu}_{4}\xi_{n,m}\geq-\tilde{\mu}_{4}\xi_{3}\left(m\right)
\geq-\tilde{\mu}_{4}\xi_{4}\left(m\right).
\end{array}
\end{eqnarray*}

Sea $T^{*}\in I_{n,m}$, entonces dado que en particular $T^{*}\in I_{1}\left(n\right)$ se cumple con probabilidad positiva que no hay arribos a las colas $Q_{1}$ y $Q_{2}$, en consecuencia, tampoco hay usuarios de transferencia para $Q_{3}$ y $Q_{4}$, es decir, $\tilde{\mu}_{1}=\mu_{1}$, $\tilde{\mu}_{2}=\mu_{2}$, $\tilde{\mu}_{3}=\mu_{3}$, $\tilde{\mu}_{4}=\mu_{4}$, es decir, los eventos $Q_{1}$ y $Q_{3}$ son condicionalmente independientes en el intervalo $I_{n,m}$; lo mismo ocurre para las colas $Q_{2}$ y $Q_{4}$, por lo tanto tenemos que


\begin{eqnarray}
\begin{array}{l}
\prob\left\{A_{1}\left(T^{*}\right),A_{2}\left(T^{*}\right),
A_{3}\left(T^{*}\right),A_{4}\left(T^{*}\right)|T^{*}\in I_{n,m}\right\}
=\prod_{j=1}^{4}\prob\left\{A_{j}\left(T^{*}\right)|T^{*}\in I_{n,m}\right\}\\
\geq\prob\left\{A_{1}\left(T^{*}\right)|T^{*}\in I_{1}\left(n\right)\right\}
\prob\left\{A_{2}\left(T^{*}\right)|T^{*}\in I_{2}\left(n\right)\right\}
\prob\left\{A_{3}\left(T^{*}\right)|T^{*}\in I_{3}\left(m\right)\right\}
\prob\left\{A_{4}\left(T^{*}\right)|T^{*}\in I_{4}\left(m\right)\right\}\\
=e^{-\mu_{1}\xi_{1}\left(n\right)}
e^{-\mu_{2}\xi_{2}\left(n\right)}
e^{-\mu_{3}\xi_{3}\left(m\right)}
e^{-\mu_{4}\xi_{4}\left(m\right)}
=e^{-\left[\tilde{\mu}_{1}\xi_{1}\left(n\right)
+\tilde{\mu}_{2}\xi_{2}\left(n\right)
+\tilde{\mu}_{3}\xi_{3}\left(m\right)
+\tilde{\mu}_{4}\xi_{4}
\left(m\right)\right]}>0.
\end{array}
\end{eqnarray}
\end{proof}


Estos resultados aparecen en Daley (1968) \cite{Daley68} para $\left\{T_{n}\right\}$ intervalos de inter-arribo, $\left\{D_{n}\right\}$ intervalos de inter-salida y $\left\{S_{n}\right\}$ tiempos de servicio.

\begin{itemize}
\item Si el proceso $\left\{T_{n}\right\}$ es Poisson, el proceso $\left\{D_{n}\right\}$ es no correlacionado si y s\'olo si es un proceso Poisso, lo cual ocurre si y s\'olo si $\left\{S_{n}\right\}$ son exponenciales negativas.

\item Si $\left\{S_{n}\right\}$ son exponenciales negativas, $\left\{D_{n}\right\}$ es un proceso de renovaci\'on  si y s\'olo si es un proceso Poisson, lo cual ocurre si y s\'olo si $\left\{T_{n}\right\}$ es un proceso Poisson.

\item $\esp\left(D_{n}\right)=\esp\left(T_{n}\right)$.

\item Para un sistema de visitas $GI/M/1$ se tiene el siguiente teorema:

\begin{Teo}
En un sistema estacionario $GI/M/1$ los intervalos de interpartida tienen
\begin{eqnarray*}
\esp\left(e^{-\theta D_{n}}\right)&=&\mu\left(\mu+\theta\right)^{-1}\left[\delta\theta
-\mu\left(1-\delta\right)\alpha\left(\theta\right)\right]
\left[\theta-\mu\left(1-\delta\right)^{-1}\right]\\
\alpha\left(\theta\right)&=&\esp\left[e^{-\theta T_{0}}\right]\\
var\left(D_{n}\right)&=&var\left(T_{0}\right)-\left(\tau^{-1}-\delta^{-1}\right)
2\delta\left(\esp\left(S_{0}\right)\right)^{2}\left(1-\delta\right)^{-1}.
\end{eqnarray*}
\end{Teo}



\begin{Teo}
El proceso de salida de un sistema de colas estacionario $GI/M/1$ es un proceso de renovaci\'on si y s\'olo si el proceso de entrada es un proceso Poisson, en cuyo caso el proceso de salida es un proceso Poisson.
\end{Teo}


\begin{Teo}
Los intervalos de interpartida $\left\{D_{n}\right\}$ de un sistema $M/G/1$ estacionario son no correlacionados si y s\'olo si la distribuci\'on de los tiempos de servicio es exponencial negativa, es decir, el sistema es de tipo  $M/M/1$.

\end{Teo}



\end{itemize}


\subsection{Resultados para Procesos de Salida}

En Sigman, Thorison y Wolff \cite{Sigman2} prueban que para la existencia de un una sucesi\'on infinita no decreciente de tiempos de regeneraci\'on $\tau_{1}\leq\tau_{2}\leq\cdots$ en los cuales el proceso se regenera, basta un tiempo de regeneraci\'on $R_{1}$, donde $R_{j}=\tau_{j}-\tau_{j-1}$. Para tal efecto se requiere la existencia de un espacio de probabilidad $\left(\Omega,\mathcal{F},\prob\right)$, y proceso estoc\'astico $\textit{X}=\left\{X\left(t\right):t\geq0\right\}$ con espacio de estados $\left(S,\mathcal{R}\right)$, con $\mathcal{R}$ $\sigma$-\'algebra.

\begin{Prop}
Si existe una variable aleatoria no negativa $R_{1}$ tal que $\theta_{R\footnotesize{1}}X=_{D}X$, entonces $\left(\Omega,\mathcal{F},\prob\right)$ puede extenderse para soportar una sucesi\'on estacionaria de variables aleatorias $R=\left\{R_{k}:k\geq1\right\}$, tal que para $k\geq1$,
\begin{eqnarray*}
\theta_{k}\left(X,R\right)=_{D}\left(X,R\right).
\end{eqnarray*}

Adem\'as, para $k\geq1$, $\theta_{k}R$ es condicionalmente independiente de $\left(X,R_{1},\ldots,R_{k}\right)$, dado $\theta_{\tau k}X$.

\end{Prop}


\begin{itemize}
\item Doob en 1953 demostr\'o que el estado estacionario de un proceso de partida en un sistema de espera $M/G/\infty$, es Poisson con la misma tasa que el proceso de arribos.

\item Burke en 1968, fue el primero en demostrar que el estado estacionario de un proceso de salida de una cola $M/M/s$ es un proceso Poisson.

\item Disney en 1973 obtuvo el siguiente resultado:

\begin{Teo}
Para el sistema de espera $M/G/1/L$ con disciplina FIFO, el proceso $\textbf{I}$ es un proceso de renovaci\'on si y s\'olo si el proceso denominado longitud de la cola es estacionario y se cumple cualquiera de los siguientes casos:

\begin{itemize}
\item[a)] Los tiempos de servicio son identicamente cero;
\item[b)] $L=0$, para cualquier proceso de servicio $S$;
\item[c)] $L=1$ y $G=D$;
\item[d)] $L=\infty$ y $G=M$.
\end{itemize}
En estos casos, respectivamente, las distribuciones de interpartida $P\left\{T_{n+1}-T_{n}\leq t\right\}$ son


\begin{itemize}
\item[a)] $1-e^{-\lambda t}$, $t\geq0$;
\item[b)] $1-e^{-\lambda t}*F\left(t\right)$, $t\geq0$;
\item[c)] $1-e^{-\lambda t}*\indora_{d}\left(t\right)$, $t\geq0$;
\item[d)] $1-e^{-\lambda t}*F\left(t\right)$, $t\geq0$.
\end{itemize}
\end{Teo}


\item Finch (1959) mostr\'o que para los sistemas $M/G/1/L$, con $1\leq L\leq \infty$ con distribuciones de servicio dos veces diferenciable, solamente el sistema $M/M/1/\infty$ tiene proceso de salida de renovaci\'on estacionario.

\item King (1971) demostro que un sistema de colas estacionario $M/G/1/1$ tiene sus tiempos de interpartida sucesivas $D_{n}$ y $D_{n+1}$ son independientes, si y s\'olo si, $G=D$, en cuyo caso le proceso de salida es de renovaci\'on.

\item Disney (1973) demostr\'o que el \'unico sistema estacionario $M/G/1/L$, que tiene proceso de salida de renovaci\'on  son los sistemas $M/M/1$ y $M/D/1/1$.



\item El siguiente resultado es de Disney y Koning (1985)
\begin{Teo}
En un sistema de espera $M/G/s$, el estado estacionario del proceso de salida es un proceso Poisson para cualquier distribuci\'on de los tiempos de servicio si el sistema tiene cualquiera de las siguientes cuatro propiedades.

\begin{itemize}
\item[a)] $s=\infty$
\item[b)] La disciplina de servicio es de procesador compartido.
\item[c)] La disciplina de servicio es LCFS y preemptive resume, esto se cumple para $L<\infty$
\item[d)] $G=M$.
\end{itemize}

\end{Teo}

\item El siguiente resultado es de Alamatsaz (1983)

\begin{Teo}
En cualquier sistema de colas $GI/G/1/L$ con $1\leq L<\infty$ y distribuci\'on de interarribos $A$ y distribuci\'on de los tiempos de servicio $B$, tal que $A\left(0\right)=0$, $A\left(t\right)\left(1-B\left(t\right)\right)>0$ para alguna $t>0$ y $B\left(t\right)$ para toda $t>0$, es imposible que el proceso de salida estacionario sea de renovaci\'on.
\end{Teo}

\end{itemize}

Estos resultados aparecen en Daley (1968) \cite{Daley68} para $\left\{T_{n}\right\}$ intervalos de inter-arribo, $\left\{D_{n}\right\}$ intervalos de inter-salida y $\left\{S_{n}\right\}$ tiempos de servicio.

\begin{itemize}
\item Si el proceso $\left\{T_{n}\right\}$ es Poisson, el proceso $\left\{D_{n}\right\}$ es no correlacionado si y s\'olo si es un proceso Poisso, lo cual ocurre si y s\'olo si $\left\{S_{n}\right\}$ son exponenciales negativas.

\item Si $\left\{S_{n}\right\}$ son exponenciales negativas, $\left\{D_{n}\right\}$ es un proceso de renovaci\'on  si y s\'olo si es un proceso Poisson, lo cual ocurre si y s\'olo si $\left\{T_{n}\right\}$ es un proceso Poisson.

\item $\esp\left(D_{n}\right)=\esp\left(T_{n}\right)$.

\item Para un sistema de visitas $GI/M/1$ se tiene el siguiente teorema:

\begin{Teo}
En un sistema estacionario $GI/M/1$ los intervalos de interpartida tienen
\begin{eqnarray*}
\esp\left(e^{-\theta D_{n}}\right)&=&\mu\left(\mu+\theta\right)^{-1}\left[\delta\theta
-\mu\left(1-\delta\right)\alpha\left(\theta\right)\right]
\left[\theta-\mu\left(1-\delta\right)^{-1}\right]\\
\alpha\left(\theta\right)&=&\esp\left[e^{-\theta T_{0}}\right]\\
var\left(D_{n}\right)&=&var\left(T_{0}\right)-\left(\tau^{-1}-\delta^{-1}\right)
2\delta\left(\esp\left(S_{0}\right)\right)^{2}\left(1-\delta\right)^{-1}.
\end{eqnarray*}
\end{Teo}



\begin{Teo}
El proceso de salida de un sistema de colas estacionario $GI/M/1$ es un proceso de renovaci\'on si y s\'olo si el proceso de entrada es un proceso Poisson, en cuyo caso el proceso de salida es un proceso Poisson.
\end{Teo}


\begin{Teo}
Los intervalos de interpartida $\left\{D_{n}\right\}$ de un sistema $M/G/1$ estacionario son no correlacionados si y s\'olo si la distribuci\'on de los tiempos de servicio es exponencial negativa, es decir, el sistema es de tipo  $M/M/1$.

\end{Teo}



\end{itemize}
%\newpage

\section{Aplicaci\'on a Teor\'ia de Colas}



Def\'inanse los puntos de regenaraci\'on  en el proceso $\left[L_{1}\left(t\right),L_{2}\left(t\right),\ldots,L_{N}\left(t\right)\right]$. Los puntos cuando la cola $i$ es visitada y todos los $L_{j}\left(\tau_{i}\left(m\right)\right)=0$ para $i=1,2$  son puntos de regeneraci\'on. Se llama ciclo regenerativo al intervalo entre dos puntos regenerativos sucesivos.

Sea $M_{i}$  el n\'umero de ciclos de visita en un ciclo regenerativo, y sea $C_{i}^{(m)}$, para $m=1,2,\ldots,M_{i}$ la duraci\'on del $m$-\'esimo ciclo de visita en un ciclo regenerativo. Se define el ciclo del tiempo de visita promedio $\esp\left[C_{i}\right]$ como

\begin{eqnarray*}
\esp\left[C_{i}\right]&=&\frac{\esp\left[\sum_{m=1}^{M_{i}}C_{i}^{(m)}\right]}{\esp\left[M_{i}\right]}
\end{eqnarray*}




Sea la funci\'on generadora de momentos para $L_{i}$, el n\'umero de usuarios en la cola $Q_{i}\left(z\right)$ en cualquier momento, est\'a dada por el tiempo promedio de $z^{L_{i}\left(t\right)}$ sobre el ciclo regenerativo definido anteriormente:

\begin{eqnarray*}
Q_{i}\left(z\right)&=&\esp\left[z^{L_{i}\left(t\right)}\right]=\frac{\esp\left[\sum_{m=1}^{M_{i}}\sum_{t=\tau_{i}\left(m\right)}^{\tau_{i}\left(m+1\right)-1}z^{L_{i}\left(t\right)}\right]}{\esp\left[\sum_{m=1}^{M_{i}}\tau_{i}\left(m+1\right)-\tau_{i}\left(m\right)\right]}
\end{eqnarray*}

$M_{i}$ es un tiempo de paro en el proceso regenerativo con $\esp\left[M_{i}\right]<\infty$, se sigue del lema de Wald que:


\begin{eqnarray*}
\esp\left[\sum_{m=1}^{M_{i}}\sum_{t=\tau_{i}\left(m\right)}^{\tau_{i}\left(m+1\right)-1}z^{L_{i}\left(t\right)}\right]&=&\esp\left[M_{i}\right]\esp\left[\sum_{t=\tau_{i}\left(m\right)}^{\tau_{i}\left(m+1\right)-1}z^{L_{i}\left(t\right)}\right]\\
\esp\left[\sum_{m=1}^{M_{i}}\tau_{i}\left(m+1\right)-\tau_{i}\left(m\right)\right]&=&\esp\left[M_{i}\right]\esp\left[\tau_{i}\left(m+1\right)-\tau_{i}\left(m\right)\right]
\end{eqnarray*}

por tanto se tiene que


\begin{eqnarray*}
Q_{i}\left(z\right)&=&\frac{\esp\left[\sum_{t=\tau_{i}\left(m\right)}^{\tau_{i}\left(m+1\right)-1}z^{L_{i}\left(t\right)}\right]}{\esp\left[\tau_{i}\left(m+1\right)-\tau_{i}\left(m\right)\right]}
\end{eqnarray*}

observar que el denominador es simplemente la duraci\'on promedio del tiempo del ciclo.


Se puede demostrar (ver Hideaki Takagi 1986) que

\begin{eqnarray*}
\esp\left[\sum_{t=\tau_{i}\left(m\right)}^{\tau_{i}\left(m+1\right)-1}z^{L_{i}\left(t\right)}\right]=z\frac{F_{i}\left(z\right)-1}{z-P_{i}\left(z\right)}
\end{eqnarray*}

Durante el tiempo de intervisita para la cola $i$, $L_{i}\left(t\right)$ solamente se incrementa de manera que el incremento por intervalo de tiempo est\'a dado por la funci\'on generadora de probabilidades de $P_{i}\left(z\right)$, por tanto la suma sobre el tiempo de intervisita puede evaluarse como:

\begin{eqnarray*}
\esp\left[\sum_{t=\tau_{i}\left(m\right)}^{\tau_{i}\left(m+1\right)-1}z^{L_{i}\left(t\right)}\right]&=&\esp\left[\sum_{t=\tau_{i}\left(m\right)}^{\tau_{i}\left(m+1\right)-1}\left\{P_{i}\left(z\right)\right\}^{t-\overline{\tau}_{i}\left(m\right)}\right]=\frac{1-\esp\left[\left\{P_{i}\left(z\right)\right\}^{\tau_{i}\left(m+1\right)-\overline{\tau}_{i}\left(m\right)}\right]}{1-P_{i}\left(z\right)}\\
&=&\frac{1-I_{i}\left[P_{i}\left(z\right)\right]}{1-P_{i}\left(z\right)}
\end{eqnarray*}
por tanto

\begin{eqnarray*}
\esp\left[\sum_{t=\tau_{i}\left(m\right)}^{\tau_{i}\left(m+1\right)-1}z^{L_{i}\left(t\right)}\right]&=&\frac{1-F_{i}\left(z\right)}{1-P_{i}\left(z\right)}
\end{eqnarray*}

Haciendo uso de lo hasta ahora desarrollado se tiene que

\begin{eqnarray*}
Q_{i}\left(z\right)&=&\frac{1}{\esp\left[C_{i}\right]}\cdot\frac{1-F_{i}\left(z\right)}{P_{i}\left(z\right)-z}\cdot\frac{\left(1-z\right)P_{i}\left(z\right)}{1-P_{i}\left(z\right)}\\
&=&\frac{\mu_{i}\left(1-\mu_{i}\right)}{f_{i}\left(i\right)}\cdot\frac{1-F_{i}\left(z\right)}{P_{i}\left(z\right)-z}\cdot\frac{\left(1-z\right)P_{i}\left(z\right)}{1-P_{i}\left(z\right)}
\end{eqnarray*}

\begin{Def}
Sea $L_{i}^{*}$el n\'umero de usuarios en la cola $Q_{i}$ cuando es visitada por el servidor para dar servicio, entonces

\begin{eqnarray}
\esp\left[L_{i}^{*}\right]&=&f_{i}\left(i\right)\\
Var\left[L_{i}^{*}\right]&=&f_{i}\left(i,i\right)+\esp\left[L_{i}^{*}\right]-\esp\left[L_{i}^{*}\right]^{2}.
\end{eqnarray}

\end{Def}


\begin{Def}
El tiempo de intervisita $I_{i}$ es el periodo de tiempo que comienza cuando se ha completado el servicio en un ciclo y termina cuando es visitada nuevamente en el pr\'oximo ciclo. Su  duraci\'on del mismo est\'a dada por $\tau_{i}\left(m+1\right)-\overline{\tau}_{i}\left(m\right)$.
\end{Def}


Recordemos las siguientes expresiones:

\begin{eqnarray*}
S_{i}\left(z\right)&=&\esp\left[z^{\overline{\tau}_{i}\left(m\right)-\tau_{i}\left(m\right)}\right]=F_{i}\left(\theta\left(z\right)\right),\\
F\left(z\right)&=&\esp\left[z^{L_{0}}\right],\\
P\left(z\right)&=&\esp\left[z^{X_{n}}\right],\\
F_{i}\left(z\right)&=&\esp\left[z^{L_{i}\left(\tau_{i}\left(m\right)\right)}\right],
\theta_{i}\left(z\right)-zP_{i}
\end{eqnarray*}

entonces 

\begin{eqnarray*}
\esp\left[S_{i}\right]&=&\frac{\esp\left[L_{i}^{*}\right]}{1-\mu_{i}}=\frac{f_{i}\left(i\right)}{1-\mu_{i}},\\
Var\left[S_{i}\right]&=&\frac{Var\left[L_{i}^{*}\right]}{\left(1-\mu_{i}\right)^{2}}+\frac{\sigma^{2}\esp\left[L_{i}^{*}\right]}{\left(1-\mu_{i}\right)^{3}}
\end{eqnarray*}

donde recordemos que

\begin{eqnarray*}
Var\left[L_{i}^{*}\right]&=&f_{i}\left(i,i\right)+f_{i}\left(i\right)-f_{i}\left(i\right)^{2}.
\end{eqnarray*}

La duraci\'on del tiempo de intervisita es $\tau_{i}\left(m+1\right)-\overline{\tau}\left(m\right)$. Dado que el n\'umero de usuarios presentes en $Q_{i}$ al tiempo $t=\tau_{i}\left(m+1\right)$ es igual al n\'umero de arribos durante el intervalo de tiempo $\left[\overline{\tau}\left(m\right),\tau_{i}\left(m+1\right)\right]$ se tiene que


\begin{eqnarray*}
\esp\left[z_{i}^{L_{i}\left(\tau_{i}\left(m+1\right)\right)}\right]=\esp\left[\left\{P_{i}\left(z_{i}\right)\right\}^{\tau_{i}\left(m+1\right)-\overline{\tau}\left(m\right)}\right]
\end{eqnarray*}

entonces, si \begin{eqnarray*}I_{i}\left(z\right)&=&\esp\left[z^{\tau_{i}\left(m+1\right)-\overline{\tau}\left(m\right)}\right]\end{eqnarray*} se tienen que

\begin{eqnarray*}
F_{i}\left(z\right)=I_{i}\left[P_{i}\left(z\right)\right]
\end{eqnarray*}
para $i=1,2$, por tanto



\begin{eqnarray*}
\esp\left[L_{i}^{*}\right]&=&\mu_{i}\esp\left[I_{i}\right]\\
Var\left[L_{i}^{*}\right]&=&\mu_{i}^{2}Var\left[I_{i}\right]+\sigma^{2}\esp\left[I_{i}\right]
\end{eqnarray*}
para $i=1,2$, por tanto


\begin{eqnarray*}
\esp\left[I_{i}\right]&=&\frac{f_{i}\left(i\right)}{\mu_{i}},
\end{eqnarray*}
adem\'as

\begin{eqnarray*}
Var\left[I_{i}\right]&=&\frac{Var\left[L_{i}^{*}\right]}{\mu_{i}^{2}}-\frac{\sigma_{i}^{2}}{\mu_{i}^{2}}f_{i}\left(i\right).
\end{eqnarray*}


Si  $C_{i}\left(z\right)=\esp\left[z^{\overline{\tau}\left(m+1\right)-\overline{\tau}_{i}\left(m\right)}\right]$el tiempo de duraci\'on del ciclo, entonces, por lo hasta ahora establecido, se tiene que

\begin{eqnarray*}
C_{i}\left(z\right)=I_{i}\left[\theta_{i}\left(z\right)\right],
\end{eqnarray*}
entonces

\begin{eqnarray*}
\esp\left[C_{i}\right]&=&\esp\left[I_{i}\right]\esp\left[\theta_{i}\left(z\right)\right]=\frac{\esp\left[L_{i}^{*}\right]}{\mu_{i}}\frac{1}{1-\mu_{i}}=\frac{f_{i}\left(i\right)}{\mu_{i}\left(1-\mu_{i}\right)}\\
Var\left[C_{i}\right]&=&\frac{Var\left[L_{i}^{*}\right]}{\mu_{i}^{2}\left(1-\mu_{i}\right)^{2}}.
\end{eqnarray*}

Por tanto se tienen las siguientes igualdades


\begin{eqnarray*}
\esp\left[S_{i}\right]&=&\mu_{i}\esp\left[C_{i}\right],\\
\esp\left[I_{i}\right]&=&\left(1-\mu_{i}\right)\esp\left[C_{i}\right]\\
\end{eqnarray*}

derivando con respecto a $z$



\begin{eqnarray*}
\frac{d Q_{i}\left(z\right)}{d z}&=&\frac{\left(1-F_{i}\left(z\right)\right)P_{i}\left(z\right)}{\esp\left[C_{i}\right]\left(1-P_{i}\left(z\right)\right)\left(P_{i}\left(z\right)-z\right)}\\
&-&\frac{\left(1-z\right)P_{i}\left(z\right)F_{i}^{'}\left(z\right)}{\esp\left[C_{i}\right]\left(1-P_{i}\left(z\right)\right)\left(P_{i}\left(z\right)-z\right)}\\
&-&\frac{\left(1-z\right)\left(1-F_{i}\left(z\right)\right)P_{i}\left(z\right)\left(P_{i}^{'}\left(z\right)-1\right)}{\esp\left[C_{i}\right]\left(1-P_{i}\left(z\right)\right)\left(P_{i}\left(z\right)-z\right)^{2}}\\
&+&\frac{\left(1-z\right)\left(1-F_{i}\left(z\right)\right)P_{i}^{'}\left(z\right)}{\esp\left[C_{i}\right]\left(1-P_{i}\left(z\right)\right)\left(P_{i}\left(z\right)-z\right)}\\
&+&\frac{\left(1-z\right)\left(1-F_{i}\left(z\right)\right)P_{i}\left(z\right)P_{i}^{'}\left(z\right)}{\esp\left[C_{i}\right]\left(1-P_{i}\left(z\right)\right)^{2}\left(P_{i}\left(z\right)-z\right)}
\end{eqnarray*}

Calculando el l\'imite cuando $z\rightarrow1^{+}$:
\begin{eqnarray}
Q_{i}^{(1)}\left(z\right)=\lim_{z\rightarrow1^{+}}\frac{d Q_{i}\left(z\right)}{dz}&=&\lim_{z\rightarrow1}\frac{\left(1-F_{i}\left(z\right)\right)P_{i}\left(z\right)}{\esp\left[C_{i}\right]\left(1-P_{i}\left(z\right)\right)\left(P_{i}\left(z\right)-z\right)}\\
&-&\lim_{z\rightarrow1^{+}}\frac{\left(1-z\right)P_{i}\left(z\right)F_{i}^{'}\left(z\right)}{\esp\left[C_{i}\right]\left(1-P_{i}\left(z\right)\right)\left(P_{i}\left(z\right)-z\right)}\\
&-&\lim_{z\rightarrow1^{+}}\frac{\left(1-z\right)\left(1-F_{i}\left(z\right)\right)P_{i}\left(z\right)\left(P_{i}^{'}\left(z\right)-1\right)}{\esp\left[C_{i}\right]\left(1-P_{i}\left(z\right)\right)\left(P_{i}\left(z\right)-z\right)^{2}}\\
&+&\lim_{z\rightarrow1^{+}}\frac{\left(1-z\right)\left(1-F_{i}\left(z\right)\right)P_{i}^{'}\left(z\right)}{\esp\left[C_{i}\right]\left(1-P_{i}\left(z\right)\right)\left(P_{i}\left(z\right)-z\right)}\\
&+&\lim_{z\rightarrow1^{+}}\frac{\left(1-z\right)\left(1-F_{i}\left(z\right)\right)P_{i}\left(z\right)P_{i}^{'}\left(z\right)}{\esp\left[C_{i}\right]\left(1-P_{i}\left(z\right)\right)^{2}\left(P_{i}\left(z\right)-z\right)}
\end{eqnarray}

Entonces:
%______________________________________________________

\begin{eqnarray*}
\lim_{z\rightarrow1^{+}}\frac{\left(1-F_{i}\left(z\right)\right)P_{i}\left(z\right)}{\left(1-P_{i}\left(z\right)\right)\left(P_{i}\left(z\right)-z\right)}&=&\lim_{z\rightarrow1^{+}}\frac{\frac{d}{dz}\left[\left(1-F_{i}\left(z\right)\right)P_{i}\left(z\right)\right]}{\frac{d}{dz}\left[\left(1-P_{i}\left(z\right)\right)\left(-z+P_{i}\left(z\right)\right)\right]}\\
&=&\lim_{z\rightarrow1^{+}}\frac{-P_{i}\left(z\right)F_{i}^{'}\left(z\right)+\left(1-F_{i}\left(z\right)\right)P_{i}^{'}\left(z\right)}{\left(1-P_{i}\left(z\right)\right)\left(-1+P_{i}^{'}\left(z\right)\right)-\left(-z+P_{i}\left(z\right)\right)P_{i}^{'}\left(z\right)}
\end{eqnarray*}


%______________________________________________________


\begin{eqnarray*}
\lim_{z\rightarrow1^{+}}\frac{\left(1-z\right)P_{i}\left(z\right)F_{i}^{'}\left(z\right)}{\left(1-P_{i}\left(z\right)\right)\left(P_{i}\left(z\right)-z\right)}&=&\lim_{z\rightarrow1^{+}}\frac{\frac{d}{dz}\left[\left(1-z\right)P_{i}\left(z\right)F_{i}^{'}\left(z\right)\right]}{\frac{d}{dz}\left[\left(1-P_{i}\left(z\right)\right)\left(P_{i}\left(z\right)-z\right)\right]}\\
&=&\lim_{z\rightarrow1^{+}}\frac{-P_{i}\left(z\right) F_{i}^{'}\left(z\right)+(1-z) F_{i}^{'}\left(z\right) P_{i}^{'}\left(z\right)+(1-z) P_{i}\left(z\right)F_{i}^{''}\left(z\right)}{\left(1-P_{i}\left(z\right)\right)\left(-1+P_{i}^{'}\left(z\right)\right)-\left(-z+P_{i}\left(z\right)\right)P_{i}^{'}\left(z\right)}
\end{eqnarray*}


%______________________________________________________

\begin{eqnarray*}
&&\lim_{z\rightarrow1^{+}}\frac{\left(1-z\right)\left(1-F_{i}\left(z\right)\right)P_{i}\left(z\right)\left(P_{i}^{'}\left(z\right)-1\right)}{\left(1-P_{i}\left(z\right)\right)\left(P_{i}\left(z\right)-z\right)^{2}}=\lim_{z\rightarrow1^{+}}\frac{\frac{d}{dz}\left[\left(1-z\right)\left(1-F_{i}\left(z\right)\right)P_{i}\left(z\right)\left(P_{i}^{'}\left(z\right)-1\right)\right]}{\frac{d}{dz}\left[\left(1-P_{i}\left(z\right)\right)\left(P_{i}\left(z\right)-z\right)^{2}\right]}\\
&=&\lim_{z\rightarrow1^{+}}\frac{-\left(1-F_{i}\left(z\right)\right) P_{i}\left(z\right)\left(-1+P_{i}^{'}\left(z\right)\right)-(1-z) P_{i}\left(z\right)F_{i}^{'}\left(z\right)\left(-1+P_{i}^{'}\left(z\right)\right)}{2\left(1-P_{i}\left(z\right)\right)\left(-z+P_{i}\left(z\right)\right) \left(-1+P_{i}^{'}\left(z\right)\right)-\left(-z+P_{i}\left(z\right)\right)^2 P_{i}^{'}\left(z\right)}\\
&+&\lim_{z\rightarrow1^{+}}\frac{+(1-z) \left(1-F_{i}\left(z\right)\right) \left(-1+P_{i}^{'}\left(z\right)\right) P_{i}^{'}\left(z\right)}{{2\left(1-P_{i}\left(z\right)\right)\left(-z+P_{i}\left(z\right)\right) \left(-1+P_{i}^{'}\left(z\right)\right)-\left(-z+P_{i}\left(z\right)\right)^2 P_{i}^{'}\left(z\right)}}\\
&+&\lim_{z\rightarrow1^{+}}\frac{+(1-z) \left(1-F_{i}\left(z\right)\right) P_{i}\left(z\right)P_{i}^{''}\left(z\right)}{{2\left(1-P_{i}\left(z\right)\right)\left(-z+P_{i}\left(z\right)\right) \left(-1+P_{i}^{'}\left(z\right)\right)-\left(-z+P_{i}\left(z\right)\right)^2 P_{i}^{'}\left(z\right)}}
\end{eqnarray*}











%______________________________________________________
\begin{eqnarray*}
&&\lim_{z\rightarrow1^{+}}\frac{\left(1-z\right)\left(1-F_{i}\left(z\right)\right)P_{i}^{'}\left(z\right)}{\left(1-P_{i}\left(z\right)\right)\left(P_{i}\left(z\right)-z\right)}=\lim_{z\rightarrow1^{+}}\frac{\frac{d}{dz}\left[\left(1-z\right)\left(1-F_{i}\left(z\right)\right)P_{i}^{'}\left(z\right)\right]}{\frac{d}{dz}\left[\left(1-P_{i}\left(z\right)\right)\left(P_{i}\left(z\right)-z\right)\right]}\\
&=&\lim_{z\rightarrow1^{+}}\frac{-\left(1-F_{i}\left(z\right)\right) P_{i}^{'}\left(z\right)-(1-z) F_{i}^{'}\left(z\right) P_{i}^{'}\left(z\right)+(1-z) \left(1-F_{i}\left(z\right)\right) P_{i}^{''}\left(z\right)}{\left(1-P_{i}\left(z\right)\right) \left(-1+P_{i}^{'}\left(z\right)\right)-\left(-z+P_{i}\left(z\right)\right) P_{i}^{'}\left(z\right)}\frac{}{}
\end{eqnarray*}

%______________________________________________________
\begin{eqnarray*}
&&\lim_{z\rightarrow1^{+}}\frac{\left(1-z\right)\left(1-F_{i}\left(z\right)\right)P_{i}\left(z\right)P_{i}^{'}\left(z\right)}{\left(1-P_{i}\left(z\right)\right)^{2}\left(P_{i}\left(z\right)-z\right)}=\lim_{z\rightarrow1^{+}}\frac{\frac{d}{dz}\left[\left(1-z\right)\left(1-F_{i}\left(z\right)\right)P_{i}\left(z\right)P_{i}^{'}\left(z\right)\right]}{\frac{d}{dz}\left[\left(1-P_{i}\left(z\right)\right)^{2}\left(P_{i}\left(z\right)-z\right)\right]}\\
&=&\lim_{z\rightarrow1^{+}}\frac{-\left(1-F_{i}\left(z\right)\right) P_{i}\left(z\right) P_{i}^{'}\left(z\right)-(1-z) P_{i}\left(z\right) F_{i}^{'}\left(z\right)P_i'[z]}{\left(1-P_{i}\left(z\right)\right)^2 \left(-1+P_{i}^{'}\left(z\right)\right)-2 \left(1-P_{i}\left(z\right)\right) \left(-z+P_{i}\left(z\right)\right) P_{i}^{'}\left(z\right)}\\
&+&\lim_{z\rightarrow1^{+}}\frac{(1-z) \left(1-F_{i}\left(z\right)\right) P_{i}^{'}\left(z\right)^2+(1-z) \left(1-F_{i}\left(z\right)\right) P_{i}\left(z\right) P_{i}^{''}\left(z\right)}{\left(1-P_{i}\left(z\right)\right)^2 \left(-1+P_{i}^{'}\left(z\right)\right)-2 \left(1-P_{i}\left(z\right)\right) \left(-z+P_{i}\left(z\right)\right) P_{i}^{'}\left(z\right)}\\
\end{eqnarray*}



En nuestra notaci\'on $V\left(t\right)\equiv C_{i}$ y $X_{i}=C_{i}^{(m)}$ para nuestra segunda definici\'on, mientras que para la primera la notaci\'on es: $X\left(t\right)\equiv C_{i}$ y $R_{i}\equiv C_{i}^{(m)}$.


%___________________________________________________________________________________________
%\section{Tiempos de Ciclo e Intervisita}
%___________________________________________________________________________________________


\begin{Def}
Sea $L_{i}^{*}$el n\'umero de usuarios en la cola $Q_{i}$ cuando es visitada por el servidor para dar servicio, entonces

\begin{eqnarray}
\esp\left[L_{i}^{*}\right]&=&f_{i}\left(i\right)\\
Var\left[L_{i}^{*}\right]&=&f_{i}\left(i,i\right)+\esp\left[L_{i}^{*}\right]-\esp\left[L_{i}^{*}\right]^{2}.
\end{eqnarray}

\end{Def}

\begin{Def}
El tiempo de Ciclo $C_{i}$ es e periodo de tiempo que comienza cuando la cola $i$ es visitada por primera vez en un ciclo, y termina cuando es visitado nuevamente en el pr\'oximo ciclo. La duraci\'on del mismo est\'a dada por $\tau_{i}\left(m+1\right)-\tau_{i}\left(m\right)$, o equivalentemente $\overline{\tau}_{i}\left(m+1\right)-\overline{\tau}_{i}\left(m\right)$ bajo condiciones de estabilidad.
\end{Def}

\begin{Def}
El tiempo de intervisita $I_{i}$ es el periodo de tiempo que comienza cuando se ha completado el servicio en un ciclo y termina cuando es visitada nuevamente en el pr\'oximo ciclo. Su  duraci\'on del mismo est\'a dada por $\tau_{i}\left(m+1\right)-\overline{\tau}_{i}\left(m\right)$.
\end{Def}


Recordemos las siguientes expresiones:

\begin{eqnarray*}
S_{i}\left(z\right)&=&\esp\left[z^{\overline{\tau}_{i}\left(m\right)-\tau_{i}\left(m\right)}\right]=F_{i}\left(\theta\left(z\right)\right),\\
F\left(z\right)&=&\esp\left[z^{L_{0}}\right],\\
P\left(z\right)&=&\esp\left[z^{X_{n}}\right],\\
F_{i}\left(z\right)&=&\esp\left[z^{L_{i}\left(\tau_{i}\left(m\right)\right)}\right],
\theta_{i}\left(z\right)-zP_{i}
\end{eqnarray*}

entonces 

\begin{eqnarray*}
\esp\left[S_{i}\right]&=&\frac{\esp\left[L_{i}^{*}\right]}{1-\mu_{i}}=\frac{f_{i}\left(i\right)}{1-\mu_{i}},\\
Var\left[S_{i}\right]&=&\frac{Var\left[L_{i}^{*}\right]}{\left(1-\mu_{i}\right)^{2}}+\frac{\sigma^{2}\esp\left[L_{i}^{*}\right]}{\left(1-\mu_{i}\right)^{3}}
\end{eqnarray*}

donde recordemos que

\begin{eqnarray*}
Var\left[L_{i}^{*}\right]&=&f_{i}\left(i,i\right)+f_{i}\left(i\right)-f_{i}\left(i\right)^{2}.
\end{eqnarray*}

La duraci\'on del tiempo de intervisita es $\tau_{i}\left(m+1\right)-\overline{\tau}\left(m\right)$. Dado que el n\'umero de usuarios presentes en $Q_{i}$ al tiempo $t=\tau_{i}\left(m+1\right)$ es igual al n\'umero de arribos durante el intervalo de tiempo $\left[\overline{\tau}\left(m\right),\tau_{i}\left(m+1\right)\right]$ se tiene que


\begin{eqnarray*}
\esp\left[z_{i}^{L_{i}\left(\tau_{i}\left(m+1\right)\right)}\right]=\esp\left[\left\{P_{i}\left(z_{i}\right)\right\}^{\tau_{i}\left(m+1\right)-\overline{\tau}\left(m\right)}\right]
\end{eqnarray*}

entonces, si \begin{eqnarray*}I_{i}\left(z\right)&=&\esp\left[z^{\tau_{i}\left(m+1\right)-\overline{\tau}\left(m\right)}\right]\end{eqnarray*} se tienen que

\begin{eqnarray*}
F_{i}\left(z\right)=I_{i}\left[P_{i}\left(z\right)\right]
\end{eqnarray*}
para $i=1,2$, por tanto



\begin{eqnarray*}
\esp\left[L_{i}^{*}\right]&=&\mu_{i}\esp\left[I_{i}\right]\\
Var\left[L_{i}^{*}\right]&=&\mu_{i}^{2}Var\left[I_{i}\right]+\sigma^{2}\esp\left[I_{i}\right]
\end{eqnarray*}
para $i=1,2$, por tanto


\begin{eqnarray*}
\esp\left[I_{i}\right]&=&\frac{f_{i}\left(i\right)}{\mu_{i}},
\end{eqnarray*}
adem\'as

\begin{eqnarray*}
Var\left[I_{i}\right]&=&\frac{Var\left[L_{i}^{*}\right]}{\mu_{i}^{2}}-\frac{\sigma_{i}^{2}}{\mu_{i}^{2}}f_{i}\left(i\right).
\end{eqnarray*}


Si  $C_{i}\left(z\right)=\esp\left[z^{\overline{\tau}\left(m+1\right)-\overline{\tau}_{i}\left(m\right)}\right]$el tiempo de duraci\'on del ciclo, entonces, por lo hasta ahora establecido, se tiene que

\begin{eqnarray*}
C_{i}\left(z\right)=I_{i}\left[\theta_{i}\left(z\right)\right],
\end{eqnarray*}
entonces

\begin{eqnarray*}
\esp\left[C_{i}\right]&=&\esp\left[I_{i}\right]\esp\left[\theta_{i}\left(z\right)\right]=\frac{\esp\left[L_{i}^{*}\right]}{\mu_{i}}\frac{1}{1-\mu_{i}}=\frac{f_{i}\left(i\right)}{\mu_{i}\left(1-\mu_{i}\right)}\\
Var\left[C_{i}\right]&=&\frac{Var\left[L_{i}^{*}\right]}{\mu_{i}^{2}\left(1-\mu_{i}\right)^{2}}.
\end{eqnarray*}

Por tanto se tienen las siguientes igualdades


\begin{eqnarray*}
\esp\left[S_{i}\right]&=&\mu_{i}\esp\left[C_{i}\right],\\
\esp\left[I_{i}\right]&=&\left(1-\mu_{i}\right)\esp\left[C_{i}\right]\\
\end{eqnarray*}

Def\'inanse los puntos de regenaraci\'on  en el proceso $\left[L_{1}\left(t\right),L_{2}\left(t\right),\ldots,L_{N}\left(t\right)\right]$. Los puntos cuando la cola $i$ es visitada y todos los $L_{j}\left(\tau_{i}\left(m\right)\right)=0$ para $i=1,2$  son puntos de regeneraci\'on. Se llama ciclo regenerativo al intervalo entre dos puntos regenerativos sucesivos.

Sea $M_{i}$  el n\'umero de ciclos de visita en un ciclo regenerativo, y sea $C_{i}^{(m)}$, para $m=1,2,\ldots,M_{i}$ la duraci\'on del $m$-\'esimo ciclo de visita en un ciclo regenerativo. Se define el ciclo del tiempo de visita promedio $\esp\left[C_{i}\right]$ como

\begin{eqnarray*}
\esp\left[C_{i}\right]&=&\frac{\esp\left[\sum_{m=1}^{M_{i}}C_{i}^{(m)}\right]}{\esp\left[M_{i}\right]}
\end{eqnarray*}


En Stid72 y Heym82 se muestra que una condici\'on suficiente para que el proceso regenerativo 
estacionario sea un procesoo estacionario es que el valor esperado del tiempo del ciclo regenerativo sea finito:

\begin{eqnarray*}
\esp\left[\sum_{m=1}^{M_{i}}C_{i}^{(m)}\right]<\infty.
\end{eqnarray*}

como cada $C_{i}^{(m)}$ contiene intervalos de r\'eplica positivos, se tiene que $\esp\left[M_{i}\right]<\infty$, adem\'as, como $M_{i}>0$, se tiene que la condici\'on anterior es equivalente a tener que 

\begin{eqnarray*}
\esp\left[C_{i}\right]<\infty,
\end{eqnarray*}
por lo tanto una condici\'on suficiente para la existencia del proceso regenerativo est\'a dada por

\begin{eqnarray*}
\sum_{k=1}^{N}\mu_{k}<1.
\end{eqnarray*}



\begin{Note}\label{Cita1.Stidham}
En Stidham\cite{Stidham} y Heyman  se muestra que una condici\'on suficiente para que el proceso regenerativo 
estacionario sea un procesoo estacionario es que el valor esperado del tiempo del ciclo regenerativo sea finito:

\begin{eqnarray*}
\esp\left[\sum_{m=1}^{M_{i}}C_{i}^{(m)}\right]<\infty.
\end{eqnarray*}

como cada $C_{i}^{(m)}$ contiene intervalos de r\'eplica positivos, se tiene que $\esp\left[M_{i}\right]<\infty$, adem\'as, como $M_{i}>0$, se tiene que la condici\'on anterior es equivalente a tener que 

\begin{eqnarray*}
\esp\left[C_{i}\right]<\infty,
\end{eqnarray*}
por lo tanto una condici\'on suficiente para la existencia del proceso regenerativo est\'a dada por

\begin{eqnarray*}
\sum_{k=1}^{N}\mu_{k}<1.
\end{eqnarray*}
\end{Note}





\begin{thebibliography}{99}

\bibitem{ISL}
James, G., Witten, D., Hastie, T., and Tibshirani, R. (2013). \textit{An Introduction to Statistical Learning: with Applications in R}. Springer.

\bibitem{Logistic}
Hosmer, D. W., Lemeshow, S., and Sturdivant, R. X. (2013). \textit{Applied Logistic Regression} (3rd ed.). Wiley.

\bibitem{PatternRecognition}
Bishop, C. M. (2006). \textit{Pattern Recognition and Machine Learning}. Springer.

\bibitem{Harrell}
Harrell, F. E. (2015). \textit{Regression Modeling Strategies: With Applications to Linear Models, Logistic and Ordinal Regression, and Survival Analysis}. Springer.

\bibitem{RDocumentation}
R Documentation and Tutorials: \url{https://cran.r-project.org/manuals.html}

\bibitem{RBlogger}
Tutorials on R-bloggers: \url{https://www.r-bloggers.com/}

\bibitem{CourseraML}
Coursera: \textit{Machine Learning} by Andrew Ng.

\bibitem{edXDS}
edX: \textit{Data Science and Machine Learning Essentials} by Microsoft.

% Libros adicionales
\bibitem{Ross}
Ross, S. M. (2014). \textit{Introduction to Probability and Statistics for Engineers and Scientists}. Academic Press.

\bibitem{DeGroot}
DeGroot, M. H., and Schervish, M. J. (2012). \textit{Probability and Statistics} (4th ed.). Pearson.

\bibitem{Hogg}
Hogg, R. V., McKean, J., and Craig, A. T. (2019). \textit{Introduction to Mathematical Statistics} (8th ed.). Pearson.

\bibitem{Kleinbaum}
Kleinbaum, D. G., and Klein, M. (2010). \textit{Logistic Regression: A Self-Learning Text} (3rd ed.). Springer.

% Artículos y tutoriales adicionales
\bibitem{Wasserman}
Wasserman, L. (2004). \textit{All of Statistics: A Concise Course in Statistical Inference}. Springer.

\bibitem{KhanAcademy}
Probability and Statistics Tutorials on Khan Academy: \url{https://www.khanacademy.org/math/statistics-probability}

\bibitem{OnlineStatBook}
Online Statistics Education: \url{http://onlinestatbook.com/}

\bibitem{Peng}
Peng, C. Y. J., Lee, K. L., and Ingersoll, G. M. (2002). \textit{An Introduction to Logistic Regression Analysis and Reporting}. The Journal of Educational Research.

\bibitem{Agresti}
Agresti, A. (2007). \textit{An Introduction to Categorical Data Analysis} (2nd ed.). Wiley.

\bibitem{Han}
Han, J., Pei, J., and Kamber, M. (2011). \textit{Data Mining: Concepts and Techniques}. Morgan Kaufmann.

\bibitem{TowardsDataScience}
Data Cleaning and Preprocessing on Towards Data Science: \url{https://towardsdatascience.com/data-cleaning-and-preprocessing}

\bibitem{Molinaro}
Molinaro, A. M., Simon, R., and Pfeiffer, R. M. (2005). \textit{Prediction error estimation: a comparison of resampling methods}. Bioinformatics.

\bibitem{EvaluatingModels}
Evaluating Machine Learning Models on Towards Data Science: \url{https://towardsdatascience.com/evaluating-machine-learning-models}

\bibitem{LogisticRegressionGuide}
Practical Guide to Logistic Regression in R on Towards Data Science: \url{https://towardsdatascience.com/practical-guide-to-logistic-regression-in-r}

% Cursos en línea adicionales
\bibitem{CourseraStatistics}
Coursera: \textit{Statistics with R} by Duke University.

\bibitem{edXProbability}
edX: \textit{Data Science: Probability} by Harvard University.

\bibitem{CourseraLogistic}
Coursera: \textit{Logistic Regression} by Stanford University.

\bibitem{edXInference}
edX: \textit{Data Science: Inference and Modeling} by Harvard University.

\bibitem{CourseraWrangling}
Coursera: \textit{Data Science: Wrangling and Cleaning} by Johns Hopkins University.

\bibitem{edXRBasics}
edX: \textit{Data Science: R Basics} by Harvard University.

\bibitem{CourseraRegression}
Coursera: \textit{Regression Models} by Johns Hopkins University.

\bibitem{edXStatInference}
edX: \textit{Data Science: Statistical Inference} by Harvard University.

\bibitem{SurvivalAnalysis}
An Introduction to Survival Analysis on Towards Data Science: \url{https://towardsdatascience.com/an-introduction-to-survival-analysis}

\bibitem{MultinomialLogistic}
Multinomial Logistic Regression on DataCamp: \url{https://www.datacamp.com/community/tutorials/multinomial-logistic-regression-R}

\bibitem{CourseraSurvival}
Coursera: \textit{Survival Analysis} by Johns Hopkins University.

\bibitem{edXHighthroughput}
edX: \textit{Data Science: Statistical Inference and Modeling for High-throughput Experiments} by Harvard University.

\end{thebibliography}


\printindex
\end{document}

% -------- QUE A SU VEZ INCLUYE ----
%\chapter{PROCESOS DE MARKOV DE SALTOS}
%%_____________________________________________________________________________________
%
\section{Procesos de Markov de Saltos}
%_____________________________________________________________________________________
%


Consideremos un estado que comienza en el estado $x_{0}$ al tiempo $0$, supongamos que el sistema permanece en $x_{0}$ hasta alg\'un tiempo positivo $\tau_{1}$, tiempo en el que el sistema salta a un nuevo estado $x_{1}\neq x_{0}$. Puede ocurrir que el sistema permanezca en $x_{0}$ de manera indefinida, en este caso hacemos $\tau_{1}=\infty$. Si $\tau_{1}$ es finito, el sistema permanecer\'a en $x_{1}$ hasta $\tau_{2}$, y as\'i sucesivamente.
Sea
\begin{equation}
X\left(t\right)=\left\{\begin{array}{cc}
x_{0} & 0\leq t<\tau_{1}\\
x_{1} & \tau_{1}\leq t<\tau_{2}\\
x_{2} & \tau_{2}\leq t<\tau_{3}\\
\vdots &\\
\end{array}\right.
\end{equation}

A este proceso  se le llama {\em proceso de salto}. Si
\begin{equation}
lim_{n\rightarrow\infty}\tau_{n}=\left\{\begin{array}{cc}
<\infty & X_{t}\textrm{ explota}\\
=\infty & X_{t}\textrm{ no explota}\\
\end{array}\right.
\end{equation}

Un proceso puro de saltos es un proceso de saltos que satisface la propiedad de Markov.

\begin{Prop}
Un proceso de saltos es Markoviano si y s\'olo si todos los estados no absorbentes $x$ son tales que
\begin{eqnarray*}
P_{x}\left(\tau_{1}>t+s|\tau_{1}>s\right)=P_{x}\left(\tau_{1}>t\right)
\end{eqnarray*}
para $s,t\geq0$, equivalentemente

\begin{equation}\label{Eq.5}
\frac{1-F_{x}\left(t+s\right)}{1-F_{x}\left(s\right)}=1-F_{x}\left(t\right).
\end{equation}
\end{Prop}

\begin{Note}
Una distribuci\'on $F_{x}$ satisface la ecuaci\'on (\ref{Eq.5}) si y s\'olo si es una funci\'on de distribuci\'on exponencial para todos los estados no absorbentes $x$.
\end{Note}

Por un proceso de nacimiento y muerte se entiende un proceso de Markov de Saltos, $\left\{X_{t}\right\}_{t\geq0}$ en $E=\nat$ tal que del estado $n$ s\'olo se puede mover a $n-1$ o $n+1$, es decir, la matriz intensidad es de la forma:

\begin{equation}
\Lambda=\left(\begin{array}{ccccc}
-\beta_{0}&\beta_{0} & 0 & 0 & \ldots\\
\delta_{1}&-\beta_{1}-\delta_{1} & \beta_{1}&0&\ldots\\
0&\delta_{2}&-\beta_{2}-\delta_{2} & \beta_{2}&\ldots\\
\vdots & & & \ddots &
\end{array}\right)
\end{equation}

donde $\beta_{n}$ son las probabilidades de nacimiento y
$\delta_{n}$ las probabilidades de muerte.

La matriz de transici\'on es
\begin{equation}
Q=\left(\begin{array}{ccccc}
0& 1 & 0 & 0 & \ldots\\
q_{1}&0 & p_{1}&0&\ldots\\
0&q_{2}&0& p_{2}&\ldots\\
\vdots & & & \ddots &
\end{array}\right)
\end{equation}
con $p_{n}=\frac{\beta_{n}}{\beta_{n}+\delta_{n}}$ y
$q_{n}=\frac{\delta_{n}}{\beta_{n}+\delta_{n}}$

\begin{Prop}
La recurrencia de un Proceso Markoviano de Saltos
$\left\{X_{t}\right\}_{t\geq0}$ con espacio de estados numerable, o equivalentemente de la cadena encajada $\left\{Y_{n}\right\}$ es equivalente a
\begin{equation}\label{Eq.2.1}
\sum_{n=1}^{\infty}\frac{\delta_{1}\cdots\delta_{n}}{\beta_{1}\cdots\beta_{n}}=\sum_{n=1}^{\infty}\frac{q_{1}\cdots
q_{n}}{p_{1}\cdots p_{n}}=\infty
\end{equation}
\end{Prop}

\begin{Lem}
Independientemente de la recurrencia o transitoriedad de la cadena, hay una y s\'olo una, salvo m\'ultiplos, soluci\'on $\nu$
a $\nu\Lambda=0$, dada por
\begin{equation}\label{Eq.2.2}
\nu_{n}=\frac{\beta_{0}\cdots\beta_{n-1}}{\delta_{1}\cdots\delta_{n}}\nu_{0}
\end{equation}
\end{Lem}

\begin{Cor}\label{Corolario2.3}
En el caso recurrente, la medida estacionaria $\mu$ para
$\left\{Y_{n}\right\}$ est\'a dada por
\begin{equation}\label{Eq.2.3}
\mu_{n}=\frac{p_{1}\cdots p_{n-1}}{q_{1}\cdots q_{n}}\mu_{0}
\end{equation}
para $n=1,2,\ldots$
\end{Cor}

\begin{Def}
Una medida $\nu$ es estacionaria si $0\leq\nu_{j}<\infty$ y para toda $t$ se cumple que $\nu P^{t}=nu$.
\end{Def}


\begin{Def}
Un proceso irreducible recurrente con medida estacionaria con masa finita es llamado erg\'odico.
\end{Def}

\begin{Teo}\label{Teo4.3}
Un Proceso de Saltos de Markov irreducible no explosivo es erg\'odico si y s\'olo si uno puede encontrar una soluci\'on $\pi$ de probabilidad, $|\pi|=1$, $0\leq\pi_{j}\leq1$ para $\nu\Lambda=0$. En este caso $\pi$ es la distribuci\'on estacionaria.
\end{Teo}
\begin{Cor}\label{Corolario2.4}
$\left\{X_{t}\right\}_{t\geq0}$ es erg\'odica si y s\'olo si (\ref{Eq.2.1}) se cumple y $S<\infty$, en cuyo caso la distribuci\'on estacionaria $\pi$ est\'a dada por

\begin{equation}\label{Eq.2.4}
\pi_{0}=\frac{1}{S}\textrm{,
}\pi_{n}=\frac{1}{S}\frac{\beta_{0}\cdots\beta_{n-1}}{\delta_{1}\cdots\delta_{n}}\textrm{,
}n=1,2,\ldots
\end{equation}
\end{Cor}


Sea $E$ espacio discreto de estados, finito o numerable, y sea $\left\{X_{t}\right\}$ un proceso de Markov con espacio de estados $E$. Una medida $\mu$ en $E$ definida por sus probabilidades puntuales $\mu_{i}$, escribimos $p_{ij}^{t}=P^{t}\left(i,\left\{j\right\}\right)=P_{i}\left(X_{t}=j\right)$.

El monto del tiempo gastado en cada estado es positivo, de modo tal que las trayectorias muestrales son constantes por partes. Para un proceso de saltos denotamos por los tiempos de saltos a $S_{0}=0<S_{1}<S_{2}\cdots$, los tiempos entre saltos consecutivos $T_{n}=S_{n+1}-S_{n}$ y la secuencia de estados visitados por $Y_{0},Y_{1},\ldots$, as\'i las trayectorias muestrales son constantes entre $S_{n}$ consecutivos, continua por la derecha, es decir, $X_{S_{n}}=Y_{n}$. 

La descripci\'on de un modelo pr\'actico est\'a dado usualmente en t\'erminos de las intensidades $\lambda\left(i\right)$ y las probabilidades de salto $q_{ij}$ m\'as que en t\'erminos de la matriz de transici\'on $P^{t}$. Sup\'ongase de ahora en adelante que $q_{ii}=0$ cuando $\lambda\left(i\right)>0$

\begin{Teo}
Cualquier Proceso de Markov de Saltos satisface la Propiedad
Fuerte de Markov
\end{Teo}

\begin{Def}
Una medida $v\neq0$ es estacionaria si $0\leq v_{j}<\infty$, $vP^{t}=v$ para toda $t$.
\end{Def}

\begin{Teo}\label{Teo.4.2}
Supongamos que $\left\{X_{t}\right\}$ es irreducible recurrente en $E$. Entonces existe una y s\'olo una, salvo m\'ultiplos, medida estacionaria $v$. Esta $v$ tiene la propiedad de que $0<v_{j}<\infty$ para todo $j$ y puede encontrarse en cualquiera de las siguientes formas

\begin{itemize}
\item[i)] Para alg\'un estado $i$, fijo pero arbitrario, $v_{j}$ es el tiempo esperado utilizado en $j$ entre dos llegadas consecutivas al estado $i$;
\begin{equation}\label{Eq.4.2}
v_{j}=\esp_{i}\int_{0}^{w\left(i\right)}\indora\left(X_{t}=j\right)dt
\end{equation}
con $w\left(i\right)=\inf\left\{t>0:X_{t}=i,X_{t^{-}}=\lim_{s\uparrow t}X_{s}\neq i\right\}$. 
\item[ii)]
$v_{j}=\frac{\mu_{j}}{\lambda\left(j\right)}$, donde $\mu$ es estacionaria para $\left\{Y_{n}\right\}$. \item[iii)] como
soluci\'on de $v\Lambda=0$.
\end{itemize}
\end{Teo}

\begin{Def}
Un proceso irreducible recurrente con medida estacionaria de masa
finita es llamado erg\'odico.
\end{Def}

\begin{Teo}\label{Teo.4.3}
Un proceso de Markov de saltos irreducible no explosivo es erg\'odico si y s\'olo si se puede encontrar una soluci\'on, de probabilidad, $\pi$, con $|\pi|=1$ y $0\leq\pi_{j}\leq1$, a $\pi\Lambda=0$. En este caso $\pi$ es la distribuci\'on estacionaria.
\end{Teo}

\begin{Cor}\label{Cor.4.4}
Una condici\'on suficiente para la ergodicidad de un proceso irreducible es la existencia de una probabilidad $\pi$ que resuelva el sistema $\pi\Lambda=0$ y que adem\'as tenga la propiedad de que $\sum\pi_{j}\lambda\left(j\right)$.
\end{Cor}

%_____________________________________________________________________________________
%
\section{Matriz Intensidad}
%_____________________________________________________________________________________
%


\begin{Def}
La matriz intensidad
$\Lambda=\left(\lambda\left(i,j\right)\right)_{i,j\in E}$ del proceso de saltos $\left\{X_{t}\right\}_{t\geq0}$ est\'a dada por
\begin{eqnarray*}
\lambda\left(i,j\right)&=&\lambda\left(i\right)q_{i,j}\textrm{,    }j\neq i\\
\lambda\left(i,i\right)&=&-\lambda\left(i\right)
\end{eqnarray*}
\end{Def}


\begin{Prop}\label{Prop.3.1}
Una matriz $E\times E$, $\Lambda$ es la matriz de intensidad de un proceso markoviano de saltos $\left\{X_{t}\right\}_{t\geq0}$ si y s\'olo si
\begin{eqnarray*}
\lambda\left(i,i\right)\leq0\textrm{, }\lambda\left(i,j\right)\textrm{,   }i\neq j\textrm{,  }\sum_{j\in E}\lambda\left(i,j\right)=0.
\end{eqnarray*}
Adem\'as, $\Lambda$ est\'a en correspondencia uno a uno con la
distribuci\'on del proceso minimal dado por el teorema 3.1.
\end{Prop}


Para el caso particular de la Cola $M/M/1$, la matr\'iz de itensidad est\'a dada por
\begin{eqnarray*}
\Lambda=\left[\begin{array}{cccccc}
-\beta & \beta & 0 &0 &0& \cdots\\
\delta & -\beta-\delta & \beta & 0 & 0 &\cdots\\
0 & \delta & -\beta-\delta & \beta & 0 &\cdots\\
\vdots & & & & & \ddots\\
\end{array}\right]
\end{eqnarray*}


%____________________________________________________________________________
\section{Medidas Estacionarias}
%____________________________________________________________________________
%


\begin{Def}
Una medida $v\neq0$ es estacionaria si $0\leq v_{j}<\infty$, $vP^{t}=v$ para toda $t$.
\end{Def}

\begin{Teo}\label{Teo.4.2}
Supongamos que $\left\{X_{t}\right\}$ es irreducible recurrente en $E$. Entonces existe una y s\'olo una, salvo m\'ultiplos, medida estacionaria $v$. Esta $v$ tiene la propiedad de que $0<v_{j}<\infty$ para todo $j$ y puede encontrarse en cualquiera de las siguientes formas

\begin{itemize}
\item[i)] Para alg\'un estado $i$, fijo pero arbitrario, $v_{j}$ es el tiempo esperado utilizado en $j$ entre dos llegadas consecutivas al estado $i$;
\begin{equation}\label{Eq.4.2}
v_{j}=\esp_{i}\int_{0}^{w\left(i\right)}\indora\left(X_{t}=j\right)dt
\end{equation}
con $w\left(i\right)=\inf\left\{t>0:X_{t}=i,X_{t^{-}}=\lim_{s\uparrow t}X_{s}\neq i\right\}$. 
\item[ii)]
$v_{j}=\frac{\mu_{j}}{\lambda\left(j\right)}$, donde $\mu$ es estacionaria para $\left\{Y_{n}\right\}$. 
\item[iii)] como soluci\'on de $v\Lambda=0$.
\end{itemize}
\end{Teo}


%____________________________________________________________________________
\section{Criterios de Ergodicidad}
%____________________________________________________________________________
%

\begin{Def}
Un proceso irreducible recurrente con medida estacionaria de masa finita es llamado erg\'odico.
\end{Def}

\begin{Teo}\label{Teo.4.3}
Un proceso de Markov de saltos irreducible no explosivo es erg\'odico si y s\'olo si se puede encontrar una soluci\'on, de probabilidad, $\pi$, con $|\pi|=1$ y $0\leq\pi_{j}\leq1$, a $\pi\Lambda=0$. En este caso $\pi$ es la distribuci\'on estacionaria.
\end{Teo}

\begin{Cor}\label{Cor.4.4}
Una condici\'on suficiente para la ergodicidad de un proceso irreducible es la existencia de una probabilidad $\pi$ que resuelva el sistema $\pi\Lambda=0$ y que adem\'as tenga la propiedad de que $\sum\pi_{j}\lambda\left(j\right)<\infty$.
\end{Cor}

\begin{Prop}
Si el proceso es erg\'odico, entonces existe una versi\'on estrictamente estacionaria
$\left\{X_{t}\right\}_{-\infty<t<\infty}$con doble tiempo
infinito.
\end{Prop}

\begin{Teo}
Si $\left\{X_{t}\right\}$ es erg\'odico y $\pi$ es la distribuci\'on estacionaria, entonces para todo $i,j$, $p_{ij}^{t}\rightarrow\pi_{j}$ cuando $t\rightarrow\infty$.
\end{Teo}

\begin{Cor}
Si $\left\{X_{t}\right\}$ es irreducible recurente pero no erg\'odica, es decir $|v|=\infty$, entonces $p_{ij}^{t}\rightarrow0$ para todo $i,j\in E$.
\end{Cor}

\begin{Cor}
Para cualquier proceso Markoviano de Saltos minimal, irreducible o
no, los l\'imites $li_{t\rightarrow\infty}p_{ij}^{t}$ existe.
\end{Cor}




%\chapter{PROCESOS NACIMIENTO Y MUERTE}
%
%_____________________________________________________________________________________
%
\section{Procesos de Nacimiento y Muerte}
%_____________________________________________________________________________________
%

\begin{Prop}\label{Prop.2.1}
La recurrencia de $\left\{X_{t}\right\}$, o equivalentemente de
$\left\{Y_{n}\right\}$ es equivalente a
\begin{equation}\label{Eq.2.1}
\sum_{n=1}^{\infty}\frac{\delta_{1}\cdots\delta_{n}}{\beta_{1}\cdots\beta_{n}}=\sum_{n=1}^{\infty}\frac{q_{1}\cdots
q_{n}}{p_{1}\cdots p_{n}}=\infty
\end{equation}
\end{Prop}

\begin{Lema}\label{Lema.2.2}
Independientemente de la recurrencia o transitorieadad, existe una
y s\'olo una, salvo m\'ultiplos, soluci\'on a $v\Lambda=0$, dada por
\begin{equation}\label{Eq.2.2}
v_{n}=\frac{\beta_{0}\cdots\beta_{n-1}}{\delta_{1}\cdots\delta_{n}}v_{0}
\end{equation}
para $n=1,2,\ldots$.
\end{Lema}

\begin{Cor}\label{Cor.2.3}
En el caso recurrente, la medida estacionaria $\mu$ para
$\left\{Y_{n}\right\}$ est\'a dada por
\begin{equation}
\mu_{n}=\frac{p_{1}\cdots p_{n-1}}{q_{1}\cdots q_{n}}\mu_{0}
\end{equation}
para $n=1,2,\ldots$.
\end{Cor}

Se define a
$S=1+\sum_{n=1}^{\infty}\frac{\beta_{0}\beta_{1}\cdots\beta_{n-1}}{\delta_{1}\delta_{2}\cdots\delta_{n}}$

\begin{Cor}\label{Cor.2.4}
$\left\{X_{t}\right\}$ es erg\'odica si y s\'olo si la ecuaci\'on
(\ref{Eq.2.1}) se cumple y adem\'as $S<\infty$, en cuyo caso la
distribuci\'on erg\'odica, $\pi$, est\'a dada por
\begin{equation}\label{Eq.2.4}
\pi_{0}=\frac{1}{S}\textrm{,
}\pi_{n}=\frac{1}{S}\frac{\beta_{0}\cdots\beta_{n-1}}{\delta_{1}\cdots\delta_{n}}
\end{equation}
para $n=1,2,\ldots$.
\end{Cor}
%_____________________________________________________________________________________
\section{Procesos de Nacimiento y Muerte Generales}
%_____________________________________________________________________________________

Por un proceso de nacimiento y muerte se entiende un proceso de saltos de markov $\left\{X_{t}\right\}_{t\geq0}$ con espacio de estados a lo m\'as numerable, con la propiedad de que s\'olo puede ir al estado $n+1$ o al estado $n-1$, es decir, su matriz de intensidad es de la forma
\begin{eqnarray*}
\Lambda=\left[\begin{array}{cccccc}
-\beta_{0} & \beta_{0} & 0 &0 &0& \cdots\\
\delta_{1} & -\beta_{1}-\delta_{1} & \beta_{1} & 0 & 0 &\cdots\\
0 & \delta_{2} & -\beta_{2}-\delta_{2} & \beta_{2} & 0 &\cdots\\
\vdots & & & & & \ddots\\
\end{array}\right]
\end{eqnarray*}
donde $\beta_{n}$ son las intensidades de nacimiento y $\delta_{n}$ las intensidades de muerte, o tambi\'en se puede ver como a $X_{t}$ el n\'umero de usuarios en una cola al tiempo $t$, un salto hacia arriba corresponde a la llegada de un nuevo usuario y un salto hacia abajo como al abandono de un usuario despu\'es de haber recibido su servicio.

La cadena de saltos $\left\{Y_{n}\right\}$ tiene matriz de transici\'on dada por
\begin{eqnarray*}
Q=\left[\begin{array}{cccccc}
0 & 1 & 0 &0 &0& \cdots\\
q_{1} & 0 & p_{1} & 0 & 0 &\cdots\\
0 & q_{2} & 0 & p_{2} & 0 &\cdots\\
\vdots & & & & & \ddots\\
\end{array}\right]
\end{eqnarray*}
donde $p_{n}=\frac{\beta_{n}}{\beta_{n}+\delta_{n}}$ y $q_{n}=1-p_{n}=\frac{\delta_{n}}{\beta_{n}+\delta_{n}}$, donde adem\'as se asumne por el momento que $p_{n}$ no puede tomar el valor $0$ \'o $1$ para cualquier valor de $n$.

\begin{Prop}\label{Prop.2.1}
La recurrencia de $\left\{X_{t}\right\}$, o equivalentemente de $\left\{Y_{n}\right\}$ es equivalente a
\begin{equation}\label{Eq.2.1}
\sum_{n=1}^{\infty}\frac{\delta_{1}\cdots\delta_{n}}{\beta_{1}\cdots\beta_{n}}=\sum_{n=1}^{\infty}\frac{q_{1}\cdots q_{n}}{p_{1}\cdots p_{n}}=\infty
\end{equation}
\end{Prop}

\begin{Lema}\label{Lema.2.2}
Independientemente de la recurrencia o transitorieadad, existe una y s\'olo una, salvo m\'ultiplos, soluci\'on a $v\Lambda=0$, dada por
\begin{equation}\label{Eq.2.2}
v_{n}=\frac{\beta_{0}\cdots\beta_{n-1}}{\delta_{1}\cdots\delta_{n}}v_{0}
\end{equation}
para $n=1,2,\ldots$.
\end{Lema}

\begin{Cor}\label{Cor.2.3}
En el caso recurrente, la medida estacionaria $\mu$ para $\left\{Y_{n}\right\}$ est\'a dada por
\begin{equation}\label{Eq.}
\mu_{n}=\frac{p_{1}\cdots p_{n-1}}{q_{1}\cdots q_{n}}\mu_{0}
\end{equation}
para $n=1,2,\ldots$.
\end{Cor}

Se define a $S=1+\sum_{n=1}^{\infty}\frac{\beta_{0}\beta_{1}\cdots\beta_{n-1}}{\delta_{1}\delta_{2}\cdots\delta_{n}}$.

\begin{Cor}\label{Cor.2.4}
$\left\{X_{t}\right\}$ es erg\'odica si y s\'olo si la ecuaci\'on (\ref{Eq.2.1}) se cumple y adem\'as $S<\infty$, en cuyo caso la distribuci\'on erg\'odica, $\pi$, est\'a dada por
\begin{equation}\label{Eq.2.4}
\pi_{0}=\frac{1}{S}\textrm{,     }\pi_{n}=\frac{1}{S}\frac{\beta_{0}\cdots\beta_{n-1}}{\delta_{1}\cdots\delta_{n}}
\end{equation}
para $n=1,2,\ldots$.
\end{Cor}





%_____________________________________________________________________________________
%
\section{Notaci\'on Kendall-Lee}
%_____________________________________________________________________________________
%

A partir de este momento se har\'an las siguientes consideraciones:
\begin{itemize}
\item[a) ]Si $t_{n}$ es el tiempo aleatorio en el que llega al sistema el $n$-\'esimo cliente, para $n=1,2,\ldots$, $t_{0}=0$ y $t_{0}<t_{1}<\cdots$ se definen los tiempos entre arribos $\tau_{n}=t_{n}-t_{n-1}$ para $n=1,2,\ldots$, variables aleatorias independientes e id\'enticamente distribuidas.

\item[b) ]Los tiempos entre arribos tienen un valor medio $E\left(\tau\right)$ finito y positivo $\frac{1}{\beta}$, es decir, $\beta$ se puede ver como la tasa o intensidad promedio de arribos al sistema por unidad de tiempo.
\item[c) ]  Adem\'as se supondr\'a que los servidores son identicos y si $s$ denota la variable aleatoria que describe el tiempo de servicio, entonces $E\left(s\right)=\frac{1}{\delta}$, $\delta$ es la tasa promedio de servicio por servidor.
\end{itemize}


La notaci\'on de Kendall-Lee es una forma abreviada de describir un sistema de espera con las siguientes componentes:
\begin{itemize}
\item[a)] {\em\bf Fuente}: Poblaci\'on de clientes potenciales del sistema, esta puede ser finita o infinita. 
\item[b)] {\em\bf Proceso de Arribos}: Proceso determinado por la funci\'on de distribuci\'on $A\left(t\right)=P\left\{\tau\leq t\right\}$ de los tiempos entre arribos.
\end{itemize}

Adem\'as tenemos las siguientes igualdades
\begin{equation}\label{Eq.0.1}
N\left(t\right)=N_{q}\left(t\right)+N_{s}\left(s\right)
\end{equation}
donde
\begin{itemize}
\item $N\left(t\right)$ es el n\'umero de clientes en el sistema al tiempo $t$. 
\item $N_{q}\left(t\right)$ es el n\'umero de cliente en la cola al tiempo $t$.
\item $N_{s}\left(t\right)$ es el n\'umero de clientes recibiendo servicio en el tiempo $t$.
\end{itemize}

Bajo la hip\'otesis de estacionareidad, es decir, las caracter\'isticas de funcionamiento del sistema se han estabilizado en valores independientes del tiempo, entonces
\begin{equation}
N=N_{q}+N_{s}.
\end{equation}

Los valores medios de las cantidades anteriores se escriben como $L=E\left(N\right)$, $L_{q}=E\left(N_{q}\right)$ y $L_{s}=E\left(N_{s}\right)$, entonces de la ecuaci\'on \ref{Eq.0.1} se obtiene

\begin{equation}
L=L_{q}+L_{s}
\end{equation}
Si $q$ es el tiempo que pasa un cliente en la cola antes de recibir servicio, y W es el tiempo total que un cliente pasa en el sistema, entonces \[w=q+s\] por lo tanto \[W=W_{q}+W_{s},\] donde $W=E\left(w\right)$, $W_{q}=E\left(q\right)$ y $W_{s}=E\left(s\right)=\frac{1}{\delta}$.

La intensidad de tr\'afico se define como
\begin{equation}
\rho=\frac{E\left(s\right)}{E\left(\tau\right)}=\frac{\beta}{\delta}.
\end{equation}

La utilizaci\'on por servidor es
\begin{equation}
u=\frac{\rho}{c}=\frac{\beta}{c\delta}.
\end{equation}
donde $c$ es el n\'umero de servidores.

Esta notaci\'on es una forma abreviada de describir un sistema de espera con componentes dados a continuaci\'on, la notaci\'on es

\begin{equation}\label{Notacion.K.L.}
A/S/c/K/F/d
\end{equation}

Cada una de las letras describe:

\begin{itemize}
\item $A$ es la distribuci\'on de los tiempos entre arribos.
\item $S$ es la distribuci\'on del tiempo de servicio.
\item $c$ es el n\'umero de servidores.
\item $K$ es la capacidad del sistema.
\item $F$ es el n\'umero de individuos en la fuente.
\item $d$ es la disciplina del servicio
\end{itemize}

Usualmente se acostumbra suponer que $K=\infty$, $F=\infty$ y $d=FIFO$, es decir, First In First Out. Las distribuciones usuales para $A$ y $B$ son:

\begin{itemize}
\item $GI$ para la distribuci\'on general de los tiempos entre arribos.
\item $G$ distribuci\'on general del tiempo de servicio.
\item $M$ Distribuci\'on exponencial para $A$ o $S$.
\item $E_{K}$ Distribuci\'on Erlang-$K$, para $A$ o $S$.
\item $D$ tiempos entre arribos o de servicio constantes, es decir, deterministicos.
\end{itemize}


%_____________________________________________________________________________________
%
\subsection{Cola $M/M/1$}
%_____________________________________________________________________________________
%
Este modelo corresponde a un proceso de nacimiento y muerte con $\beta_{n}=\beta$ y $\delta_{n}=\delta$ independiente del valor de $n$. La intensidad de tr\'afico $\rho=\frac{\beta}{\delta}$, implica que el criterio de recurrencia (ecuaci\'on \ref{Eq.2.1}) quede de la forma:
\begin{eqnarray*}
1+\sum_{n=1}^{\infty}\rho^{-n}=\infty.
\end{eqnarray*}
Equivalentemente el proceso es recurrente si y s\'olo si
\begin{eqnarray*}
\sum_{n\geq1}\left(\frac{\beta}{\delta}\right)^{n}<\infty\Leftrightarrow \frac{\beta}{\delta}<1.
\end{eqnarray*}
Entonces
$S=\frac{\delta}{\delta-\beta}$, luego por la ecuaci\'on \ref{Eq.2.4} se tiene que
\begin{eqnarray*}
\pi_{0}&=&\frac{\delta-\beta}{\delta}=1-\frac{\beta}{\delta},\\
\pi_{n}&=&\pi_{0}\left(\frac{\beta}{\delta}\right)^{n}=\left(1-\frac{\beta}{\delta}\right)\left(\frac{\beta}{\delta}\right)^{n}=\left(1-\rho\right)\rho^{n}.
\end{eqnarray*}


Lo cual nos lleva a la siguiente proposici\'on:

\begin{Prop}
La cola $M/M/1$ con intensidad de tr\'afico $\rho$, es recurrente si y s\'olo si $\rho\leq1$.
\end{Prop}

Entonces por el corolario \ref{Cor.2.3}

\begin{Prop}
La cola $M/M/1$ con intensidad de tr\'afico $\rho$ es erg\'odica si y s\'olo si $\rho<1$. En cuyo caso, la distribuci\'on de equilibrio $\pi$ de la longitud de la cola es geom\'etrica, $\pi_{n}=\left(1-\rho\right)\rho^{n}$, para $n=1,2,\ldots$.
\end{Prop}

De la proposici\'on anterior se desprenden varios hechos importantes.
\begin{itemize}
\item[a) ] $\prob\left[X_{t}=0\right]=\pi_{0}=1-\rho$, es decir, la probabilidad de que el sistema se encuentre ocupado.
\item[b) ] De las propiedades de la distribuci\'on Geom\'etrica se desprende que
\begin{itemize}
\item[i) ] $\esp\left[X_{t}\right]=\frac{\rho}{1-\rho}$,
\item[ii) ] $Var\left[X_{t}\right]=\frac{\rho}{\left(1-\rho\right)^{2}}$.
\end{itemize}
\end{itemize}

Si $L$ es el n\'umero esperado de clientes en el sistema, incluyendo los que est\'an siendo atendidos, entonces
\begin{eqnarray}
L=\frac{\rho}{1-\rho}.
\end{eqnarray}
Si adem\'as $W$ es el tiempo total del cliente en la cola: $W=W_{q}+W_{s}$, $\rho=\frac{\esp\left[s\right]}{\esp\left[\tau\right]}=\beta W_{s}$, puesto que $W_{s}=\esp\left[s\right]$ y $\esp\left[\tau\right]=\frac{1}{\delta}$. Por la f\'ormula de Little $L=\lambda W$
\begin{eqnarray*}
W&=&\frac{L}{\beta}=\frac{\frac{\rho}{1-\rho}}{\beta}=\frac{\rho}{\delta}\frac{1}{1-\rho}=\frac{W_{s}}{1-\rho}\\
&=&\frac{1}{\delta\left(1-\rho\right)}=\frac{1}{\delta-\beta},
\end{eqnarray*}

luego entonces

\begin{eqnarray*}
W_{q}&=&W-W_{s}=\frac{1}{\delta-\beta}-\frac{1}{\delta}=\frac{\beta}{\delta(\delta-\beta)}\\
&=&\frac{\rho}{1-\rho}\frac{1}{\delta}=\esp\left[s\right]\frac{\rho}{1-\rho}.
\end{eqnarray*}

Entonces

\begin{eqnarray*}
L_{q}=\beta W_{q}=\frac{\rho^{2}}{1-\rho}.
\end{eqnarray*}

Finalmente, tenemos las siguientes proposiciones:

\begin{Prop}
\begin{enumerate}
\item $W\left(t\right)=1-e^{-\frac{t}{W}}$.
\item $W_{q}\left(t\right)=1-\rho\exp^{-\frac{t}{W}}$.
\end{enumerate}
donde $W=\esp(w)$.
\end{Prop}

\begin{Prop}
La cola M/M/1 con intensidad de tr\'afico $\rho$ es recurrente si
y s\'olo si $\rho\leq1$
\end{Prop}

\begin{Prop}
La cola M/M/1 con intensidad de tr\'afica $\rho$ es ergodica si y
s\'olo si $\rho<1$. En este caso, la distribuci\'on de equilibrio
$\pi$ de la longitud de la cola es geom\'etrica,
$\pi_{n}=\left(1-\rho\right)\rho^{n}$, para $n=0,1,2,\ldots$.
\end{Prop}
%_____________________________________________________________________________________
%
\subsection{Cola $M/M/\infty$}
%_____________________________________________________________________________________
%

Este tipo de modelos se utilizan para estimar el n\'umero de l\'ineas en uso en una gran red comunicaci\'on o para estimar valores en los sistemas $M/M/c$ o $M/M/c/c$, en el se puede pensar que siempre hay un servidor disponible para cada cliente que llega.

Se puede considerar como un proceso de nacimiento y muerte con par\'ametros $\beta_{n}=\beta$ y $\mu_{n}=n\mu$ para $n=0,1,2,\ldots$. Este modelo corresponde al caso en que $\beta_{n}=\beta$ y $\delta_{n}=n\delta$, en este caso el par\'ametro de inter\'es $\eta=\frac{\beta}{\delta}$, luego, la ecuaci\'on \ref{Eq.2.1} queda de la forma:

\begin{eqnarray*}
\sum_{n=1}^{\infty}\frac{\delta_{1}\cdots\delta_{n}}{\beta_{1}\cdots\beta_{n}}=\sum_{n=1}^{\infty}n!\eta^{-n}=\infty\\
\end{eqnarray*}
con $S=1+\sum_{n=1}^{\infty}\frac{\eta^{n}}{n!}=e$, entonces por la ecuaci\'on \ref{Eq.2.4} se tiene que

\begin{eqnarray}\label{MMinf.pi}
\pi_{0}=e^{\rho},\\
\pi_{n}=e^{-\rho}\frac{\rho^{n}}{n!}.
\end{eqnarray}
Entonces, el n\'umero promedio de servidores ocupados es equivalente a considerar el n\'umero de clientes en el  sistema, es decir,
\begin{eqnarray}
L=\esp\left[N\right]=\rho.\\
Var\left[N\right]=\rho.
\end{eqnarray}
Adem\'as se tiene que $W_{q}=0$ y $L_{q}=0$. El tiempo promedio en el sistema es el tiempo promedio de servicio, es decir, $W=\esp\left[s\right]=\frac{1}{\delta}$.Resumiendo, tenemos la sisuguiente proposici\'on:

\begin{Prop}
La cola $M/M/\infty$ es erg\'odica para todos los valores de $\eta$. La distribuci\'on de equilibrio $\pi$ es Poisson con media $\eta$,
\begin{eqnarray}
\pi_{n}=\frac{e^{-n}\eta^{n}}{n!}.
\end{eqnarray}
\end{Prop}
%_____________________________________________________________________________________
%
\subsection{Cola $M/M/m$}
%_____________________________________________________________________________________
%

Este sistema considera $m$ servidores id\'enticos, con tiempos entre arribos y de servicio exponenciales con medias $\esp\left[\tau\right]=\frac{1}{\beta}$ y
$\esp\left[s\right]=\frac{1}{\delta}$. definimos ahora la utilizaci\'on por servidor como $u=\frac{\rho}{m}$ que tambi\'en se puede interpretar como la fracci\'on de tiempo promedio que cada servidor est\'a ocupado.

La cola $M/M/m$ se puede considerar como un proceso de nacimiento y muerte con par\'ametros: $\beta_{n}=\beta$ para $n=0,1,2,\ldots$ y
\begin{eqnarray}
\delta_{n}=\left\{\begin{array}{cc}
n\delta & n=0,1,\ldots,m-1\\
c\delta & n=m,\ldots\\
\end{array}\right.
\end{eqnarray}

entonces  la condici\'on de recurrencia se va a cumplir s\'i y s\'olo si $\sum_{n\geq1}\frac{\beta_{0}\cdots\beta_{n-1}}{\delta_{1}\cdots\delta_{n}}<\infty$,
equivalentemente se debe de cumplir que
\begin{eqnarray*}
S&=&1+\sum_{n\geq1}\frac{\beta_{0}\cdots\beta_{n-1}}{\delta_{1}\cdots\delta_{n}}=\sum_{n=0}^{m-1}\frac{\beta_{0}\cdots\beta_{n-1}}{\delta_{1}\cdots\delta_{n}}+\sum_{n=0}^{\infty}\frac{\beta_{0}\cdots\beta_{n-1}}{\delta_{1}\cdots\delta_{n}}\\
&=&\sum_{n=0}^{m-1}\frac{\beta^{n}}{n!\delta^{n}}+\sum_{n=0}^{\infty}\frac{\rho^{m}}{m!}u^{n}
\end{eqnarray*}
converja, lo cual ocurre si $u<1$, en este caso

\begin{eqnarray}
S=\sum_{n=0}^{m-1}\frac{\rho^{n}}{n!}+\frac{\rho^{m}}{m!}\left(1-u\right)
\end{eqnarray}
luego, para este caso se tiene que

\begin{eqnarray}
\pi_{0}&=&\frac{1}{S}\\
\pi_{n}&=&\left\{\begin{array}{cc}
\pi_{0}\frac{\rho^{n}}{n!} & n=0,1,\ldots,m-1\\
\pi_{0}\frac{\rho^{n}}{m!m^{n-m}}& n=m,\ldots\\
\end{array}\right.
\end{eqnarray}
Al igual que se hizo antes, determinaremos los valores de
$L_{q},W_{q},W$ y $L$:
\begin{eqnarray*}
L_{q}&=&\esp\left[N_{q}\right]=\sum_{n=0}^{\infty}\left(n-m\right)\pi_{n}=\sum_{n=0}^{\infty}n\pi_{n+m}\\
&=&\sum_{n=0}^{\infty}n\pi_{0}\frac{\rho^{n+m}}{m!m^{n+m}}=\pi_{0}\frac{\rho^{m}}{m!}\sum_{n=0}^{\infty}nu^{n}=\pi_{0}\frac{u\rho^{m}}{m!}\sum_{n=0}^{\infty}\frac{d}{du}u^{n}\\
&=&\pi_{0}\frac{u\rho^{m}}{m!}\frac{d}{du}\sum_{n=0}^{\infty}u^{n}=\pi_{0}\frac{u\rho^{m}}{m!}\frac{d}{du}\left(\frac{1}{1-u}\right)=\pi_{0}\frac{u\rho^{m}}{m!}\frac{1}{\left(1-u\right)^{2}},
\end{eqnarray*}

es decir
\begin{equation}
L_{q}=\frac{u\pi_{0}\rho^{m}}{m!\left(1-u\right)^{2}},
\end{equation}
luego
\begin{equation}
W_{q}=\frac{L_{q}}{\beta}.
\end{equation}
Adem\'as
\begin{equation}
W=W_{q}+\frac{1}{\delta}
\end{equation}

Si definimos
\begin{eqnarray}
C\left(m,\rho\right)=\frac{\pi_{0}\rho^{m}}{m!\left(1-u\right)}=\frac{\pi_{m}}{1-u},
\end{eqnarray}
que es la probabilidad de que un cliente que llegue al sistema
tenga que esperar en la cola. Entonces podemos reescribir las
ecuaciones reci\'en enunciadas:

\begin{eqnarray}
L_{q}&=&\frac{C\left(m,\rho\right)u}{1-u},\\
W_{q}&=&\frac{C\left(m,\rho\right)\esp\left[s\right]}{m\left(1-u\right)}\\
\end{eqnarray}
Por tanto tenemos las siguientes proposiciones:

\begin{Prop}
La cola $M/M/m$ con intensidad de tr\'afico $\rho$ es erg\'odica si y s\'olo si $\rho<1$. En este caso la distribuci\'on erg\'odica $\pi$ est\'a dada por
\begin{eqnarray}
\pi_{n}=\left\{\begin{array}{cc}
\frac{1}{S}\frac{\eta^{n}}{n!} & 0\leq n\leq m\\
\frac{1}{S}\frac{\eta^{m}}{m!}\rho^{n-m} & m\leq n<\infty\\
\end{array}\right.
\end{eqnarray}
\end{Prop}

\begin{Prop}
Para $t\geq0$
\begin{itemize}
\item[a)]
\begin{eqnarray}
W_{q}\left(t\right)=1-C\left(m,\rho\right)e^{-c\delta
t\left(1-u\right)}.
\end{eqnarray} 
\item[b)]\begin{eqnarray}
W\left(t\right)=\left\{\begin{array}{cc}
1+e^{-\delta t}\frac{\rho-m+W_{q}\left(0\right)}{m-1-\rho}+e^{-m\delta t\left(1-u\right)}\frac{C\left(m,\rho\right)}{m-1-\rho} & \rho\neq m-1\\
1-\left(1+C\left(m,\rho\right)\delta t\right)e^{-\delta t} & \rho=m-1\\
\end{array}\right.
\end{eqnarray}
\end{itemize}
\end{Prop}

Resumiendo, para este caso $\beta_{n}=\beta$ y
$\delta_{n}=m\left(n\right)\delta$, donde $m\left(n\right)$ es el n\'umero de servidores ocupados en el estado $n$, es decir,
$m\left(n\right)=m$, para $n\geq m$ y $m\left(n\right)=m$ para
$1\leq n\leq m$. La intensidad de tr\'afico es
$\rho=\frac{\beta}{m\delta}$ y $\frac{\beta_{n}}{\delta_{n}}=\rho$
para $n\geq m$. As\'i, al igual que en el caso $m=1$, la ecuaci\'on
\ref{Eq.2.1} y la recurrencia se cumplen si y s\'olo si
$\sum_{n=1}^{\infty}\rho^{-n}=\infty$, es decir, cuando
$\rho\leq1$. 


%_____________________________________________________________________________________
%
\subsection{Cola $M/M/m/m$}
%_____________________________________________________________________________________
%

Consideremos un sistema con $m$ servidores id\'enticos, pero ahora cada uno es de capacidad finita $m$. Si todos los servidores se encuentran ocupados, el siguiente usuario en llegar se pierde pues no se le deja esperar a que reciba servicio. Este tipo de sistemas pueden verse como un proceso de nacimiento y muerte con
\begin{eqnarray}
\beta_{n}=\left\{\begin{array}{cc}
\beta & n=0,1,2,\ldots,m-1\\
0 & n\geq m\\
\end{array}
\right.
\end{eqnarray}

\begin{eqnarray}
\delta_{n}=\left\{\begin{array}{cc}
n\delta & n=0,1,2,\ldots,m-1\\
0 & n\geq m\\
\end{array}
\right.
\end{eqnarray}
El proceso tiene epacio de estados finitos, $S=\left\{0,1,\ldots,m\right\}$, entonces de las ecuaciones que determinan la distribuci\'on estacionaria se tiene que
\begin{equation}\label{Eq.13.1}
\pi_{n}=\left\{\begin{array}{cc}
\pi_{0}\frac{\rho^{n}}{n!} & n=0,1,2,\ldots,m\\
0 & n\geq m\\
\end{array}
\right.
\end{equation}
y adem\'as
\begin{equation}
\pi_{0}=\left(\sum_{n=0}^{m}\frac{\rho^{n}}{n!}\right)^{-1}.
\end{equation}
A la ecuaci\'on \ref{Eq.13.1} se le llama {\em distribuci\'on truncada}. Si definimos
$\pi_{m}=B\left(m,\rho\right)=\pi_{0}\frac{\rho^{m}}{m!}$, $\pi_{m}$ representa la probabilidad de que todos los servidores se encuentren ocupados, y tambi\'en se le conoce como {\em f\'ormula de p\'erdida de Erlang}. Necesariamente en este caso el tiempo de espera en la cola $W_{q}$ y el n\'umero promedio de clientes en la cola $L_{q}$ deben de ser cero puesto que no se permite esperar para recibir servicio, m\'as a\'un, el tiempo de espera en el sistema y el tiempo de serivcio tienen la misma distribuci\'on, es decir,
\[W\left(t\right)=\prob\left\{w\leq t\right\}=1-e^{-\mu t},\] en particular
\[W=\esp\left[w\right]=\esp\left[s\right]=\frac{1}{\delta}.\]
Por otra parte, el n\'umero esperado de clientes en el sistema es
\begin{eqnarray*}
L&=&\esp\left[N\right]=\sum_{n=0}^{m}n\pi_{n}=\pi_{0}\rho\sum_{n=0}^{m}\frac{\rho^{n-1}}{\left(n-1\right)!}\\
&=&\pi_{0}\rho\sum_{n=0}^{m-1}\frac{\rho^{n}}{n!}
\end{eqnarray*}
entonces, se tiene que
\begin{equation}
L=\rho\left(1-B\left(m,\rho\right)\right)=\esp\left[s\right]\left(1-B\left(m,\rho\right)\right).
\end{equation}
Adem\'as
\begin{equation}
\delta_{q}=\delta\left(1-B\left(m,\rho\right)\right)
\end{equation}
representa la tasa promedio efectiva de arribos al sistema.
%_____________________________________________________________________________________
%
\subsection{Cola M/G/1}
%_____________________________________________________________________________________
%
Consideremos un sistema de espera con un servidor, en el que los tiempos entre arribos son exponenciales, y los tiempos de servicio tienen una distribuci\'on general $G$. Sea $N\left(t\right)_{t\geq0}$ el n\'umero de clientes en el sistema al tiempo $t$, y sean $t_{1}<t_{2}<\dots$ los tiempos sucesivos en los que los clientes completan su servicio y salen del sistema.

La sucesi\'on $\left\{X_{n}\right\}$ definida por
$X_{n}=N\left(t_{n}\right)$ es una cadena de Markov, en espec\'ifico es la Cadena encajada del proceso de llegadas de usuarios. Sea $U_{n}$ el n\'umero de clientes que llegan al sistema durante el tiempo de servicio del $n$-\'esimo cliente, entonces se tiene que

\begin{eqnarray*}
X_{n+1}=\left\{\begin{array}{cc}
X_{n}-1+U_{n+1} & \textrm{si }X_{n}\geq1,\\
U_{n+1} & \textrm{si }X_{n}=0\\
\end{array}\right.
\end{eqnarray*}

Dado que los procesos de arribos de los usuarios es Poisson con par\'ametro $\lambda$, la probabilidad condicional de que lleguen $j$ clientes al sistema dado que el tiempo de servicio es $s=t$, resulta:
\begin{eqnarray*}
\prob\left\{U=j|s=t\right\}=e^{-\lambda t}\frac{\left(\lambda
t\right)^{j}}{j!}\textrm{,   }j=0,1,\ldots
\end{eqnarray*}
por el teorema de la probabilidad total se tiene que
\begin{equation}
a_{j}=\prob\left\{U=j\right\}=\int_{0}^{\infty}\prob\left\{U=j|s=t\right\}dG\left(t\right)=\int_{0}^{\infty}e^{-\lambda
t}\frac{\left(\lambda t\right)^{j}}{j!}dG\left(t\right)
\end{equation}
donde $G$ es la distribuci\'on de los tiempos de servicio. Las probabilidades de transici\'on de la cadena est\'an dadas por
\begin{equation}
p_{0j}=\prob\left\{U_{n+1}=j\right\}=a_{j}\textrm{, para
}j=0,1,\ldots
\end{equation}
y para $i\geq1$
\begin{equation}
p_{ij}=\left\{\begin{array}{cc}
\prob\left\{U_{n+1}=j-i+1\right\}=a_{j-i+1}&\textrm{, para }j\geq i-1\\
0 & j<i-1\\
\end{array}
\right.
\end{equation}
Entonces la matriz de transici\'on es:
\begin{eqnarray*}
P=\left[\begin{array}{ccccc}
a_{0} & a_{1} & a_{2} & a_{3} & \cdots\\
a_{0} & a_{1} & a_{2} & a_{3} & \cdots\\
0 & a_{0} & a_{1} & a_{2} & \cdots\\
0 & 0 & a_{0} & a_{1} & \cdots\\
\vdots & \vdots & \cdots & \ddots &\vdots\\
\end{array}
\right].
\end{eqnarray*}
Sea $\rho=\sum_{n=0}na_{n}$, entonces se tiene el siguiente teorema:
\begin{Teo}
La cadena encajada $\left\{X_{n}\right\}$ es
\begin{itemize}
\item[a)] Recurrente positiva si $\rho<1$,
\item[b)] Transitoria
si $\rho>1$, 
\item[c)] Recurrente nula si $\rho=1$.
\end{itemize}
\end{Teo}

Recordemos que si la cadena de Markov $\left\{X_{n}\right\}$ tiene una distribuci\'on estacionaria entonces existe una distribuci\'on de probabilidad $\pi=\left(\pi_{0},\pi_{1},\ldots,\right)$, con $\pi_{i}\geq0$ y $\sum_{i\geq1}\pi_{i}=1$ tal que satisface la
ecuaci\'on $\pi=\pi P$, equivalentemente
\begin{equation}\label{Eq.18.9}
\pi_{j}=\sum_{i=0}^{\infty}\pi_{k}p_{ij},\textrm{ para
}j=0,1,2,\ldots
\end{equation}
que se puede ver como
\begin{equation}\label{Eq.19.6}
\pi_{j}=\pi_{0}a_{j}+\sum_{i=1}^{j+1}\pi_{i}a_{j-i+1}\textrm{,
para }j=0,1,\ldots
\end{equation}
si definimos
\begin{eqnarray}
\pi\left(z\right)=\sum_{j=0}^{\infty}\pi_{j}z^{j}
\end{eqnarray}
y 
\begin{equation}
A\left(z\right)=\sum_{j=0}^{\infty}a_{j}z^{j}
\end{equation}
con $|z_{j}|\leq1$. Si la ecuaci\'on \ref{Eq.19.6} la multiplicamos por $z^{j}$ y sumando sobre $j$, se tiene que
\begin{eqnarray*}
\sum_{j=0}^{\infty}\pi_{j}z^{j}&=&\sum_{j=0}^{\infty}\pi_{0}a_{j}z^{j}+\sum_{j=0}^{\infty}\sum_{i=1}^{j+1}\pi_{i}a_{j-i+1}z^{j}\\
&=&\pi_{0}\sum_{j=0}^{\infty}a_{j}z^{j}+\sum_{j=0}^{\infty}a_{j}z^{j}\sum_{i=1}^{\infty}\pi_{i}a_{i-1}\\
&=&\pi_{0}A\left(z\right)+A\left(z\right)\left(\frac{\pi\left(z\right)-\pi_{0}}{z}\right)\\
\end{eqnarray*}
es decir,

\begin{equation}
\pi\left(z\right)=\pi_{0}A\left(z\right)+A\left(z\right)\left(\frac{\pi\left(z\right)-\pi_{0}}{z}\right)\Leftrightarrow\pi\left(z\right)=\frac{\pi_{0}A\left(z\right)\left(z-1\right)}{z-A\left(z\right)}
\end{equation}

Si $z\rightarrow1$, entonces $A\left(z\right)\rightarrow A\left(1\right)=1$, y adem\'as $A^{'}\left(z\right)\rightarrow A^{'}\left(1\right)=\rho$. Si aplicamos la Regla de L'Hospital se tiene que
\begin{eqnarray*}
\sum_{j=0}^{\infty}\pi_{j}=lim_{z\rightarrow1^{-}}\pi\left(z\right)=\pi_{0}lim_{z\rightarrow1^{-}}\frac{z-1}{z-A\left(z\right)}=\frac{\pi_{0}}{1-\rho}
\end{eqnarray*}
Retomando,
\begin{eqnarray*}
a_{j}=\prob\left\{U=j\right\}=\int_{0}^{\infty}e^{-\lambda
t}\frac{\left(\lambda t\right)^{n}}{n!}dG\left(t\right)\textrm{,
para }n=0,1,2,\ldots
\end{eqnarray*}
entonces
\begin{eqnarray*}
\rho&=&\sum_{n=0}^{\infty}na_{n}=\sum_{n=0}^{\infty}n\int_{0}^{\infty}e^{-\lambda t}\frac{\left(\lambda t\right)^{n}}{n!}dG\left(t\right)\\
&=&\int_{0}^{\infty}\sum_{n=0}^{\infty}ne^{-\lambda
t}\frac{\left(\lambda
t\right)^{n}}{n!}dG\left(t\right)=\int_{0}^{\infty}\lambda
tdG\left(t\right)=\lambda\esp\left[s\right]
\end{eqnarray*}

Adem\'as, se tiene que $\rho=\beta\esp\left[s\right]=\frac{\beta}{\delta}$ y la distribuci\'on estacionaria est\'a dada por
\begin{eqnarray}
\pi_{j}&=&\pi_{0}a_{j}+\sum_{i=1}^{j+1}\pi_{i}a_{j-i+1}\textrm{, para }j=0,1,\ldots\\
\pi_{0}&=&1-\rho.
\end{eqnarray}
Por otra parte se tiene que\begin{equation}
L=\pi^{'}\left(1\right)=\rho+\frac{A^{''}\left(1\right)}{2\left(1-\rho\right)}
\end{equation}

pero $A^{''}\left(1\right)=\sum_{n=1}n\left(n-1\right)a_{n}= \esp\left[U^{2}\right]-\esp\left[U\right]$, $\esp\left[U\right]=\rho$ y
$\esp\left[U^{2}\right]=\lambda^{2}\esp\left[s^{2}\right]+\rho$.
Por lo tanto $L=\rho+\frac{\beta^{2}\esp\left[s^{2}\right]}{2\left(1-\rho\right)}$.

De las f\'ormulas de Little, se tiene que $W=E\left(w\right)=\frac{L}{\beta}$, tambi\'en el tiempo de espera en la cola
\begin{equation}
W_{q}=\esp\left(q\right)=\esp\left(w\right)-\esp\left(s\right)=\frac{L}{\beta}-\frac{1}{\delta},
\end{equation}
adem\'as el n\'umero promedio de clientes en la cola es
\begin{equation}
L_{q}=\esp\left(N_{q}\right)=\beta W_{q}=L-\rho
\end{equation}


%____________________________________________________________________________
\subsection{Cola con Infinidad de Servidores}

Este caso corresponde a $\beta_{n}=\beta$ y $\delta_{n}=n\delta$. El par\'ametro de inter\'es es $\eta=\frac{\beta}{\delta}$, de donde se obtiene:
\begin{eqnarray*}
\sum_{n\geq0}\frac{\delta_{1}\cdots\delta_{n}}{\beta_{1}\cdots\beta_{n}}=\sum_{n=1}^{\infty}n!\eta^{n}=\infty,\\
S=1+\sum_{n=1}^{\infty}\frac{\eta^{n}}{n!}=e^{n}.
\end{eqnarray*}

\begin{Prop}
La cola $M/M/\infty$ es ergodica para todos los valores de $\eta$. La distribuci\'on de equilibrio $\pi$ es Poisson con media $\eta$, $\pi_{n}=\frac{e^{-n}\eta}{n!}$
\end{Prop}

%\chapter{TEORIA DE COLAS}
%%____________________________________________________________________________
%%%\begin{frame}
\section{Queueing Theory at Markovian Level}


\subsection{General Death Birth Processes}




\subsection{General Death Birth Processes}


Consideremos un estado que comienza en el estado $x_{0}$ al tiempo
$0$, supongamos que el sistema permanece en $x_{0}$ hasta alg\'un
tiempo positivo $\tau_{1}$, tiempo en el que el sistema salta a un
nuevo estado $x_{1}\neq x_{0}$. Puede ocurrir que el sistema
permanezca en $x_{0}$ de manera indefinida, en este caso hacemos
$\tau_{1}=\infty$. Si $\tau_{1}$ es finito, el sistema
permanecer\'a en $x_{1}$ hasta $\tau_{2}$, y as\'i sucesivamente.
Sea
\begin{equation}
X\left(t\right)=\left\{\begin{array}{cc}
x_{0} & 0\leq t<\tau_{1}\\
x_{1} & \tau_{1}\leq t<\tau_{2}\\
x_{2} & \tau_{2}\leq t<\tau_{3}\\
\vdots &\\
\end{array}\right.
\end{equation}

A este proceso  se le llama {\em proceso de salto}. Si
\begin{equation}
lim_{n\rightarrow\infty}\tau_{n}=\left\{\begin{array}{cc}
<\infty & X_{t}\textrm{ explota}\\
=\infty & X_{t}\textrm{ no explota}\\
\end{array}\right.
\end{equation}

Un proceso puro de saltos es un proceso de saltos que satisface la
propiedad de Markov.
%\end{frame}
%____________________________________________________________________________
%%%\begin{frame}
\begin{Prop}
Un proceso de saltos es Markoviano si y s\'olo si todos los
estados no absorbentes $x$ son tales que
\begin{eqnarray*}
P_{x}\left(\tau_{1}>t+s|\tau_{1}>s\right)=P_{x}\left(\tau_{1}>t\right)
\end{eqnarray*}
para $s,t\geq0$, equivalentemente

\begin{equation}\label{Eq.5}
\frac{1-F_{x}\left(t+s\right)}{1-F_{x}\left(s\right)}=1-F_{x}\left(t\right).
\end{equation}

\end{Prop}

\begin{Note}
Una distribuci\'on $F_{x}$ satisface la ecuaci\'on (\ref{Eq.5}) si
y s\'olo si es una funci\'on de distribuci\'on exponencial para
todos los estados no absorbentes $x$.
\end{Note}

%\end{frame}
%____________________________________________________________________________
%%%\begin{frame}
Por un proceso de nacimiento y muerte se entiende un proceso de
Markov de Saltos, $\left\{X_{t}\right\}_{t\geq0}$ en $E=\nat$ tal
que del estado $n$ s\'olo se puede mover a $n-1$ o $n+1$, es
decir, la matriz intensidad es de la forma:

\begin{equation}
\Lambda=\left(\begin{array}{ccccc}
-\beta_{0}&\beta_{0} & 0 & 0 & \ldots\\
\delta_{1}&-\beta_{1}-\delta_{1} & \beta_{1}&0&\ldots\\
0&\delta_{2}&-\beta_{2}-\delta_{2} & \beta_{2}&\ldots\\
\vdots & & & \ddots &
\end{array}\right)
\end{equation}

donde $\beta_{n}$ son las probabilidades de nacimiento y
$\delta_{n}$ las probabilidades de muerte.

La matriz de transici\'on es
\begin{equation}
Q=\left(\begin{array}{ccccc}
0& 1 & 0 & 0 & \ldots\\
q_{1}&0 & p_{1}&0&\ldots\\
0&q_{2}&0& p_{2}&\ldots\\
\vdots & & & \ddots &
\end{array}\right)
\end{equation}
con $p_{n}=\frac{\beta_{n}}{\beta_{n}+\delta_{n}}$ y
$q_{n}=\frac{\delta_{n}}{\beta_{n}+\delta_{n}}$
%\end{frame}
%____________________________________________________________________________
%%%\begin{frame}
\begin{Prop}
La recurrencia de un Proceso Markoviano de Saltos
$\left\{X_{t}\right\}_{t\geq0}$ con espacio de estados numerable,
o equivalentemente de la cadena encajada $\left\{Y_{n}\right\}$ es
equivalente a
\begin{equation}\label{Eq.2.1}
\sum_{n=1}^{\infty}\frac{\delta_{1}\cdots\delta_{n}}{\beta_{1}\cdots\beta_{n}}=\sum_{n=1}^{\infty}\frac{q_{1}\cdots
q_{n}}{p_{1}\cdots p_{n}}=\infty
\end{equation}
\end{Prop}

\begin{Lem}
Independientemente de la recurrencia o transitoriedad de la
cadena, hay una y s\'olo una, salvo m\'ultiplos, soluci\'on $\nu$
a $\nu\Lambda=0$, dada por
\begin{equation}\label{Eq.2.2}
\nu_{n}=\frac{\beta_{0}\cdots\beta_{n-1}}{\delta_{1}\cdots\delta_{n}}\nu_{0}
\end{equation}
\end{Lem}

%\end{frame}
%____________________________________________________________________________
%%%\begin{frame}
\begin{Col}\label{Corolario2.3}
En el caso recurrente, la medida estacionaria $\mu$ para
$\left\{Y_{n}\right\}$ est\'a dada por
\begin{equation}\label{Eq.2.3}
\mu_{n}=\frac{p_{1}\cdots p_{n-1}}{q_{1}\cdots q_{n}}\mu_{0}
\end{equation}
para $n=1,2,\ldots$
\end{Col}

\begin{Def}
Una medida $\nu$ es estacionaria si $0\leq\nu_{j}<\infty$ y para
toda $t$ se cumple que $\nu P^{t}=nu$.
\end{Def}


\begin{Def}
Un proceso irreducible recurrente con medida estacionaria con masa
finita es llamado erg\'odico.
\end{Def}

\begin{Teo}\label{Teo4.3}
Un Proceso de Saltos de Markov irreducible no explosivo es
erg\'odico si y s\'olo si uno puede encontrar una soluci\'on
$\pi$ de probabilidad, $|\pi|=1$, $0\leq\pi_{j}\leq1$ para
$\nu\Lambda=0$. En este caso $\pi$ es la distribuci\'on
estacionaria.
\end{Teo}
%\end{frame}
%____________________________________________________________________________
%%%\begin{frame}

\begin{Col}\label{Corolario2.4}
$\left\{X_{t}\right\}_{t\geq0}$ es erg\'odica si y s\'olo si
(\ref{Eq.2.1}) se cumple y $S<\infty$, en cuyo caso la
distribuci\'on estacionaria $\pi$ est\'a dada por

\begin{equation}\label{Eq.2.4}
\pi_{0}=\frac{1}{S}\textrm{,
}\pi_{n}=\frac{1}{S}\frac{\beta_{0}\cdots\beta_{n-1}}{\delta_{1}\cdots\delta_{n}}\textrm{,
}n=1,2,\ldots
\end{equation}
\end{Col}



%\chapter{NOTACION DE KENDALL LEE}
%



%_____________________________________________________________________________________
%
\section{Notaci\'on Kendall-Lee}
%_____________________________________________________________________________________
%

A partir de este momento se har\'an las siguientes consideraciones:
\begin{itemize}
\item[a) ]Si $t_{n}$ es el tiempo aleatorio en el que llega al sistema el $n$-\'esimo cliente, para $n=1,2,\ldots$, $t_{0}=0$ y $t_{0}<t_{1}<\cdots$ se definen los tiempos entre arribos $\tau_{n}=t_{n}-t_{n-1}$ para $n=1,2,\ldots$, variables aleatorias independientes e id\'enticamente distribuidas.

\item[b) ]Los tiempos entre arribos tienen un valor medio $E\left(\tau\right)$ finito y positivo $\frac{1}{\beta}$, es decir, $\beta$ se puede ver como la tasa o intensidad promedio de arribos al sistema por unidad de tiempo.
\item[c) ]  Adem\'as se supondr\'a que los servidores son identicos y si $s$ denota la variable aleatoria que describe el tiempo de servicio, entonces $E\left(s\right)=\frac{1}{\delta}$, $\delta$ es la tasa promedio de servicio por servidor.
\end{itemize}


La notaci\'on de Kendall-Lee es una forma abreviada de describir un sistema de espera con las siguientes componentes:
\begin{itemize}
\item[a)] {\em\bf Fuente}: Poblaci\'on de clientes potenciales del sistema, esta puede ser finita o infinita. 
\item[b)] {\em\bf Proceso de Arribos}: Proceso determinado por la funci\'on de distribuci\'on $A\left(t\right)=P\left\{\tau\leq t\right\}$ de los tiempos entre arribos.
\end{itemize}

Adem\'as tenemos las siguientes igualdades
\begin{equation}\label{Eq.0.1}
N\left(t\right)=N_{q}\left(t\right)+N_{s}\left(s\right)
\end{equation}
donde
\begin{itemize}
\item $N\left(t\right)$ es el n\'umero de clientes en el sistema al tiempo $t$. 
\item $N_{q}\left(t\right)$ es el n\'umero de cliente en la cola al tiempo $t$.
\item $N_{s}\left(t\right)$ es el n\'umero de clientes recibiendo servicio en el tiempo $t$.
\end{itemize}

Bajo la hip\'otesis de estacionareidad, es decir, las caracter\'isticas de funcionamiento del sistema se han estabilizado en valores independientes del tiempo, entonces
\begin{equation}
N=N_{q}+N_{s}.
\end{equation}

Los valores medios de las cantidades anteriores se escriben como $L=E\left(N\right)$, $L_{q}=E\left(N_{q}\right)$ y $L_{s}=E\left(N_{s}\right)$, entonces de la ecuaci\'on \ref{Eq.0.1} se obtiene

\begin{equation}
L=L_{q}+L_{s}
\end{equation}
Si $q$ es el tiempo que pasa un cliente en la cola antes de recibir servicio, y W es el tiempo total que un cliente pasa en el sistema, entonces \[w=q+s\] por lo tanto \[W=W_{q}+W_{s},\] donde $W=E\left(w\right)$, $W_{q}=E\left(q\right)$ y $W_{s}=E\left(s\right)=\frac{1}{\delta}$.

La intensidad de tr\'afico se define como
\begin{equation}
\rho=\frac{E\left(s\right)}{E\left(\tau\right)}=\frac{\beta}{\delta}.
\end{equation}

La utilizaci\'on por servidor es
\begin{equation}
u=\frac{\rho}{c}=\frac{\beta}{c\delta}.
\end{equation}
donde $c$ es el n\'umero de servidores.

Esta notaci\'on es una forma abreviada de describir un sistema de espera con componentes dados a continuaci\'on, la notaci\'on es

\begin{equation}\label{Notacion.K.L.}
A/S/c/K/F/d
\end{equation}

Cada una de las letras describe:

\begin{itemize}
\item $A$ es la distribuci\'on de los tiempos entre arribos.
\item $S$ es la distribuci\'on del tiempo de servicio.
\item $c$ es el n\'umero de servidores.
\item $K$ es la capacidad del sistema.
\item $F$ es el n\'umero de individuos en la fuente.
\item $d$ es la disciplina del servicio
\end{itemize}

Usualmente se acostumbra suponer que $K=\infty$, $F=\infty$ y $d=FIFO$, es decir, First In First Out. Las distribuciones usuales para $A$ y $B$ son:

\begin{itemize}
\item $GI$ para la distribuci\'on general de los tiempos entre arribos.
\item $G$ distribuci\'on general del tiempo de servicio.
\item $M$ Distribuci\'on exponencial para $A$ o $S$.
\item $E_{K}$ Distribuci\'on Erlang-$K$, para $A$ o $S$.
\item $D$ tiempos entre arribos o de servicio constantes, es decir, deterministicos.
\end{itemize}


%\chapter{REDES DE COLAS}
%
%_____________________________________________________________________________________
%
\section{Redes de Colas}
%_____________________________________________________________________________________

%_____________________________________________________________________________________
%
\subsection{Sistemas Abiertos}
%_____________________________________________________________________________________
%

Considerese un sistema con dos servidores, en los cuales los usuarios llegan de acuerdo a un proceso poisson con intensidad $\lambda_{1}$ al primer servidor, despu\'es de ser atendido se pasa a la siguiente cola en el segundo servidor. Cada servidor atiende a un usuario a la vez con tiempo exponencial con raz\'on $\mu_{i}$, para $i=1,2$. A este tipo de sistemas se les conoce como sistemas secuenciales.

Def\'inase el par $\left(n,m\right)$ como el n\'umero de usuarios en el servidor 1 y 2 respectivamente. Las ecuaciones de balance son
\begin{eqnarray}\label{Eq.Balance}
\lambda P_{0,0}&=&\mu_{2}P_{0,1}\\
\left(\lambda+\mu_{1}\right)P_{n,0}&=&\mu_{2}P_{n,1}+\lambda P_{n-1,0}\\
\left(\lambda+\mu_{2}\right)P_{0,m}&=&\mu_{2}P_{0,m+1}+\mu_{1}P_{1,m-1}\\
\left(\lambda+\mu_{1}+\mu_{2}\right)P_{n,m}&=&\mu_{2}P_{n,m+1}+\mu_{1}P_{n+1,m-1}+\lambda
P_{n-1,m}
\end{eqnarray}

Cada servidor puede ser visto como un modelo de tipo $M/M/1$, de igual manera el proceso de salida de una cola $M/M/1$ con raz\'on $\lambda$, nos permite asumir que el servidor 2 tambi\'en es una cola $M/M/1$. Adem\'as la probabilidad de que haya $n$ usuarios en el servidor 1 es
\begin{eqnarray*}
P\left\{n\textrm{ en el servidor }1\right\}&=&\left(\frac{\lambda}{\mu_{1}}\right)^{n}\left(1-\frac{\lambda}{\mu_{1}}\right)=\rho_{1}^{n}\left(1-\rho_{1}\right)\\
P\left\{m\textrm{ en el servidor }2\right\}&=&\left(\frac{\lambda}{\mu_{2}}\right)^{n}\left(1-\frac{\lambda}{\mu_{2}}\right)=\rho_{2}^{m}\left(1-\rho_{2}\right)\\
\end{eqnarray*}
Si el n\'umero de usuarios en los servidores 1 y 2 son variables aleatorias independientes, se sigue que:
\begin{equation}\label{Eq.8.16}
P_{n,m}=\rho_{1}^{n}\left(1-\rho_{1}\right)\rho_{2}^{m}\left(1-\rho_{2}\right)
\end{equation}
Verifiquemos que $P_{n,m}$ satisface las ecuaciones de balance (\ref{Eq.Balance}) Antes de eso, enunciemos unas igualdades que nos ser\'an de utilidad:
\begin{eqnarray*}
\mu_{i}\rho_{i}&=&\lambda\textrm{ para }i=1,2.\\
\lambda P_{0,0}&=&\lambda\left(1-\rho_{1}\right)\left(1-\rho_{2}\right)\\
\textrm{ y }\mu_{2} P_{0,1}&=&\mu_{2}\left(1-\rho_{1}\right)\rho_{2}\left(1-\rho_{2}\right)\Rightarrow\\
\lambda P_{0,0}&=&\mu_{2} P_{0,1}\\
\left(\lambda+\mu_{2}\right)P_{0,m}&=&\left(\lambda+\mu_{2}\right)\left(1-\rho_{1}\right)\rho_{2}^{m}\left(1-\rho_{2}\right)\\
\mu_{2}P_{0,m+1}&=&\lambda\left(1-\rho_{1}\right)\rho_{2}^{m}\left(1-\rho_{2}\right)\\
&=&\mu_{2}\left(1-\rho_{1}\right)\rho_{2}^{m}\left(1-\rho_{2}\right)\\
\mu_{1}P_{1,m-1}&=&\frac{\lambda}{\rho_{2}}\left(1-\rho_{1}\right)\rho_{2}^{m}\left(1-\rho_{2}\right)\Rightarrow\\
\left(\lambda+\mu_{2}\right)P_{0,m}&=&\mu_{2}P_{0,m+1}+\mu_{1}P_{1,m-1}\\
\left(\lambda+\mu_{1}+\mu_{2}\right)P_{n,m}&=&\left(\lambda+\mu_{1}+\mu_{2}\right)\rho^{n}\left(1-\rho_{1}\right)\rho_{2}^{m}\left(1-\rho_{2}\right)\\
\mu_{2}P_{n,m+1}&=&\mu_{2}\rho_{2}\rho_{1}^{n}\left(1-\rho_{1}\right)\rho_{2}^{m}\left(1-\rho_{2}\right)\\
\mu_{1} P_{n-1,m-1}&=&\mu_{1}\frac{\rho_{1}}{\rho_{2}}\rho_{1}^{n}\left(1-\rho_{1}\right)\rho_{2}^{m}\left(1-\rho_{2}\right)\\
\lambda P_{n-1,m}&=&\frac{\lambda}{\rho_{1}}\rho_{1}^{n}\left(1-\rho_{1}\right)\rho_{2}^{m}\left(1-\rho_{2}\right)\\
\Rightarrow\left(\lambda+\mu_{1}+\mu_{2}\right)P_{n,m}&=&\mu_{2}P_{n,m+1}+\mu_{1} P_{n-1,m-1}+\lambda P_{n-1,m}\\
\end{eqnarray*}
entonces efectivamente la ecuaci\'on (\ref{Eq.8.16}) satisface las ecuaciones de balance (\ref{Eq.Balance}). El n\'umero promedio  de usuarios en el sistema, est\'a dado por
\begin{eqnarray*}
L&=&\sum_{n,m}\left(n+m\right)P_{n,m}=\sum_{n,m}nP_{n,m}+\sum_{n,m}mP_{n,m}\\
&=&\sum_{n}\sum_{m}nP_{n,m}+\sum_{m}\sum_{n}mP_{n,m}=\sum_{n}n\sum_{m}P_{n,m}+\sum_{m}m\sum_{n}P_{n,m}\\
&=&\sum_{n}n\sum_{m}\rho_{1}^{n}\left(1-\rho_{1}\right)\rho_{2}^{m}\left(1-\rho_{2}\right)+\sum_{m}m\sum_{n}\rho_{1}^{n}\left(1-\rho_{1}\right)\rho_{2}^{m}\left(1-\rho_{2}\right)\\
&=&\sum_{n}n\rho_{1}^{n}\left(1-\rho_{1}\right)\sum_{m}\rho_{2}^{m}\left(1-\rho_{2}\right)+\sum_{m}m\rho_{2}^{m}\left(1-\rho_{2}\right)\sum_{n}\rho_{1}^{n}\left(1-\rho_{1}\right)\\
&=&\sum_{n}n\rho_{1}^{n}\left(1-\rho_{1}\right)+\sum_{m}m\rho_{2}^{m}\left(1-\rho_{2}\right)\\
&=&\frac{\lambda}{\mu_{1}-\lambda}+\frac{\lambda}{\mu_{2}-\lambda}
\end{eqnarray*}


%-------------- CAPITULO SE GENERO COMO DOCUMENTO INDEPENDIENTE ----

%-------------- CAPITULO SE GENERO COMO DOCUMENTO INDEPENDIENTE ----
%\chapter{CADENAS DE MARKOV}
%%_____________________________________________________________________________________
%
\section{Estacionareidad}
%_____________________________________________________________________________________
%}

Sea $v=\left(v_{i}\right)_{i\in E}$ medida no negativa en $E$, podemos definir una nueva medida $v\prob$ que asigna masa $\sum_{i\in E}v_{i}p_{ij}$ a cada estado $j$.

\begin{Def}
La medida $v$ es estacionaria si $v_{i}<\infty$ para toda $i$ y adem\'as $v\prob=v$.
\end{Def}
En el caso de que $v$ sea distribuci\'on, independientemente de que sea estacionaria o no, se cumple con

\begin{eqnarray*}
\prob_{v}\left[X_{1}=j\right]=\sum_{i\in E}\prob_{v}\left[X_{0}=i\right]p_{ij}=\sum_{i\in E}v_{i}p_{ij}=\left(vP\right)_{j}
\end{eqnarray*}

\begin{Teo}
Supongamos que $v$ es una distribuci\'on estacionaria. Entonces
\begin{itemize}
\item[i)] La cadena es estrictamente estacionaria con respecto a
$\prob_{v}$, es decir, $\prob_{v}$-distribuci\'on de $\left\{X_{n},X_{n+1},\ldots\right\}$ no depende de $n$;
\item[ii)] Existe un aversi\'on estrictamente estacionaria $\left\{X_{n}\right\}_{n\in Z}$ de la cadena con doble tiempo infinito y $\prob\left(X_{n}=i\right)=v_{i}$ para toda $n\in Z$.
\end{itemize}
\end{Teo}

\begin{Teo}
Sea $i$ estado fijo, recurrente. Entonces una medida estacionaria $v$ puede definirse haciendo que $v_{j}$ sea el n\'umero esperado de visitas a $j$ entre dos visitas consecutivas $i$,

\begin{equation}\label{Eq.3.1}
v_{j}=\esp_{i}\sum_{n=0}^{\tau(i)-1}\indora\left(X_{n}=i\right)=\sum_{n=0}^{\infty}\prob_{i}\left[X_{n}=j,\tau(i)>n\right]
\end{equation}
\end{Teo}

\begin{Teo}\label{Teo.3.3}
Si la cadena es irreducible y recurrente, entonces existe una medida estacionaria $v$, tal que satisface $0<v_{j}<\infty$ para toda $j$, y es \'unica salvo factores multiplicativos, es decir, si $v,v^{*}$ son estacionarias, entonces $c=cv^{*}$ para alguna $c\in\left(0,\infty\right)$.
\end{Teo}

\begin{Cor}\label{Cor.3.5}
Si la cadena es irreducible y positiva recurrente, existe una
\'unica distribuci\'on estacionaria $\pi$ dada por
\begin{equation}
\pi_{j}=\frac{1}{\esp_{i}\tau_{i}}\esp_{i}\sum_{n=0}^{\tau\left(i\right)-1}\indora\left(X_{n}=j\right)=\frac{1}{\esp_{j}\tau\left(j\right)}.
\end{equation}
\end{Cor}

\begin{Cor}\label{Cor.3.6}
Cualquier cadena de Markov irreducible con un espacio de estados finito es positiva recurrente.
\end{Cor}
%_____________________________________________________________________________________
%
\section{Funciones Arm\'onicas, Recurrencia y Transitoriedad}
%_____________________________________________________________________________________
%
\begin{Def}\label{Def.Armonica}
Una funci\'on Arm\'onica es el eigenvector derecho $h$ de $P$ correspondiente al eigenvalor 1.
\end{Def}
\begin{eqnarray*}
Ph=h\Leftrightarrow h\left(i\right)=\sum_{j\in E}p_{ij}h\left(j\right)=\esp_{i}h\left(X_{1}\right)=\esp\left[h\left(X_{n+1}\right)|X_{n}=i\right].
\end{eqnarray*}
es decir, $\left\{h\left(X_{n}\right)\right\}$ es martingala.
\begin{Prop}\label{Prop.5.2}
Sea $\left\{X_{n}\right\}$ cadena irreducible  y sea $i$ estado fijo arbitrario. Entonces la cadena es transitoria s\'i y s\'olo si existe una funci\'on no cero, acotada $h:E-\left\{i\right\}\rightarrow\rea$ que satisface
\begin{equation}\label{Eq.5.1}
h\left(j\right)=\sum_{k\neq i}p_{jk}h\left(k\right)\textrm{   para }j\neq i.
\end{equation}
\end{Prop}

\begin{Prop}\label{Prop.5.4}
Suponga que la cadena es irreducible y sea $E_{0}$ un subconjunto finito de $E$ tal que se cumple la ecuaci\'on 5.2 para alguna funci\'on $h$ acotada que satisface $h\left(i\right)<h\left(j\right)$ para alg\'un estado $i\notin E_{0}$ y todo $j\in E_{0}$. Entonces la cadena es transitoria.
\end{Prop}

%_____________________________________________________________________________________
%
\section{Teor\'ia Erg\'odica}
%_____________________________________________________________________________________
%
\begin{Lema}
Sea $\left\{X_{n}\right\}$ cadena irreducible y se $F$ subconjunto finito del espacio de estados. Entonces la cadena es positiva recurrente si $\esp_{i}\tau\left(F\right)<\infty$ para todo $i\in F$.
\end{Lema}

\begin{Prop}
Sea $\left\{X_{n}\right\}$ cadena irreducible y transiente o cero recurrente, entonces $p_{ij}^{n}\rightarrow0$ conforme $n\rightarrow\infty$ para cualquier $i,j\in E$, $E$ espacio de estados.
\end{Prop}
Utilizando el teorema (2.2) y el corolario \ref{Cor.3.5}, se demuestra el siguiente resultado importante.

\begin{Teo}
Sea $\left\{X_{n}\right\}$ cadena irreducible y aperi\'odica positiva recurrente, y sea $\pi=\left\{\pi_{j}\right\}_{j\in E}$ la distribuci\'on estacionaria. Entonces $p_{ij}^{n}\rightarrow\pi_{j}$ para todo $i,j$.
\end{Teo}
\begin{Def}\label{Def.Ergodicidad}
Una cadena irreducible aperiodica, positiva recurrente con medida estacionaria $v$, es llamada {\em erg\'odica}.
\end{Def}

\begin{Prop}\label{Prop.4.4}
Sea $\left\{X_{n}\right\}$ cadena irreducible y recurrente con medida estacionaria $v$, entocnes para todo $i,j,k,l\in E$
\begin{equation}
\frac{\sum_{n=0}^{m}p_{ij}^{n}}{\sum_{n=0}^{m}p_{lk}^{n}}\rightarrow\frac{v_{j}}{v_{k}}\textrm{,    }m\rightarrow\infty
\end{equation}
\end{Prop}
\begin{Lema}\label{Lema.4.5}
La matriz $\widetilde{P}$ con elementos $\widetilde{p}_{ij}=\frac{v_{ji}p_{ji}}{v_{i}}$ es una matriz de transici\'on. Adem\'s, el $i$-\'esimo elementos $\widetilde{p}_{ij}^{m}$ de la matriz potencia $\widetilde{P}^{m}$ est? dada por $\widetilde{p}_{ij}^{m}=\frac{v_{ji}p_{ji}^{m}}{v_{i}}$.
\end{Lema}

\begin{Lema}
Def\'inase $N_{i}^{m}=\sum_{n=0}^{m}\indora\left(X_{n}=i\right)$ como el n\'umero de visitas a $i$ antes del tiempo $m$. Entonces si la cadena es reducible y recurrente, $lim_{m\rightarrow\infty}\frac{\esp_{j}N_{i}^{m}}{\esp_{k}N_{i}^{m}}=1$ para todo $j,k\in E$.
\end{Lema}


%_____________________________________________________________________________________
%
\subsection{Ejemplo de Cadena de Markov para dos Estados}
%_____________________________________________________________________________________
%

Supongamos que se tiene la siguiente cadena:
\begin{equation}
\left(\begin{array}{cc}
1-q & q\\
p & 1-p\\
\end{array}
\right).
\end{equation}
Si $P\left[X_{0}=0\right]=\pi_{0}(0)=a$ y $P\left[X_{0}=1\right]=\pi_{0}(1)=b=1-\pi_{0}(0)$, con $a+b=1$, entonces despu\'es de un procedimiento m\'as o menos corto se tiene que:

\begin{eqnarray*}
P\left[X_{n}=0\right]=\frac{p}{p+q}+\left(1-p-q\right)^{n}\left(a-\frac{p}{p+q}\right).\\
P\left[X_{n}=1\right]=\frac{q}{p+q}+\left(1-p-q\right)^{n}\left(b-\frac{q}{p+q}\right).\\
\end{eqnarray*}
donde, como $0<p,q<1$, se tiene que $|1-p-q|<1$, entonces $\left(1-p-q\right)^{n}\rightarrow 0$ cuando $n\rightarrow\infty$. Por lo tanto
\begin{eqnarray*}
lim_{n\rightarrow\infty}P\left[X_{n}=0\right]=\frac{p}{p+q}.\\
lim_{n\rightarrow\infty}P\left[X_{n}=1\right]=\frac{q}{p+q}.
\end{eqnarray*}
Si hacemos $v=\left(\frac{p}{p+q},\frac{q}{p+q}\right)$, entonces
\begin{eqnarray*}
\left(\frac{p}{p+q},\frac{q}{p+q}\right)\left(\begin{array}{cc}
1-q & q\\
p & 1-p\\
\end{array}\right).
\end{eqnarray*}


\begin{Prop}\label{Prop.5.4}
Suponga que la cadena es irreducible y sea $E_{0}$ un subconjunto finito de $E$ tal que se cumple la ecuaci\'on 5.2 para alguna funci\'on $h$ acotada que satisface $h\left(i\right)<h\left(j\right)$ para alg\'un estado $i\notin E_{0}$ y todo $j\in E_{0}$. Entonces la cadena es transitoria.
\end{Prop}

%_____________________________________________________________________________________
%
\section{Procesos de Markov de Saltos}
%_____________________________________________________________________________________
%


Consideremos un estado que comienza en el estado $x_{0}$ al tiempo $0$, supongamos que el sistema permanece en $x_{0}$ hasta alg\'un tiempo positivo $\tau_{1}$, tiempo en el que el sistema salta a un nuevo estado $x_{1}\neq x_{0}$. Puede ocurrir que el sistema permanezca en $x_{0}$ de manera indefinida, en este caso hacemos $\tau_{1}=\infty$. Si $\tau_{1}$ es finito, el sistema permanecer\'a en $x_{1}$ hasta $\tau_{2}$, y as\'i sucesivamente.
Sea
\begin{equation}
X\left(t\right)=\left\{\begin{array}{cc}
x_{0} & 0\leq t<\tau_{1}\\
x_{1} & \tau_{1}\leq t<\tau_{2}\\
x_{2} & \tau_{2}\leq t<\tau_{3}\\
\vdots &\\
\end{array}\right.
\end{equation}

A este proceso  se le llama {\em proceso de salto}. Si
\begin{equation}
lim_{n\rightarrow\infty}\tau_{n}=\left\{\begin{array}{cc}
<\infty & X_{t}\textrm{ explota}\\
=\infty & X_{t}\textrm{ no explota}\\
\end{array}\right.
\end{equation}

Un proceso puro de saltos es un proceso de saltos que satisface la propiedad de Markov.

\begin{Prop}
Un proceso de saltos es Markoviano si y s\'olo si todos los estados no absorbentes $x$ son tales que
\begin{eqnarray*}
P_{x}\left(\tau_{1}>t+s|\tau_{1}>s\right)=P_{x}\left(\tau_{1}>t\right)
\end{eqnarray*}
para $s,t\geq0$, equivalentemente

\begin{equation}\label{Eq.5}
\frac{1-F_{x}\left(t+s\right)}{1-F_{x}\left(s\right)}=1-F_{x}\left(t\right).
\end{equation}
\end{Prop}

\begin{Note}
Una distribuci\'on $F_{x}$ satisface la ecuaci\'on (\ref{Eq.5}) si y s\'olo si es una funci\'on de distribuci\'on exponencial para todos los estados no absorbentes $x$.
\end{Note}

Por un proceso de nacimiento y muerte se entiende un proceso de Markov de Saltos, $\left\{X_{t}\right\}_{t\geq0}$ en $E=\nat$ tal que del estado $n$ s\'olo se puede mover a $n-1$ o $n+1$, es decir, la matriz intensidad es de la forma:

\begin{equation}
\Lambda=\left(\begin{array}{ccccc}
-\beta_{0}&\beta_{0} & 0 & 0 & \ldots\\
\delta_{1}&-\beta_{1}-\delta_{1} & \beta_{1}&0&\ldots\\
0&\delta_{2}&-\beta_{2}-\delta_{2} & \beta_{2}&\ldots\\
\vdots & & & \ddots &
\end{array}\right)
\end{equation}

donde $\beta_{n}$ son las probabilidades de nacimiento y
$\delta_{n}$ las probabilidades de muerte.

La matriz de transici\'on es
\begin{equation}
Q=\left(\begin{array}{ccccc}
0& 1 & 0 & 0 & \ldots\\
q_{1}&0 & p_{1}&0&\ldots\\
0&q_{2}&0& p_{2}&\ldots\\
\vdots & & & \ddots &
\end{array}\right)
\end{equation}
con $p_{n}=\frac{\beta_{n}}{\beta_{n}+\delta_{n}}$ y
$q_{n}=\frac{\delta_{n}}{\beta_{n}+\delta_{n}}$

\begin{Prop}
La recurrencia de un Proceso Markoviano de Saltos
$\left\{X_{t}\right\}_{t\geq0}$ con espacio de estados numerable, o equivalentemente de la cadena encajada $\left\{Y_{n}\right\}$ es equivalente a
\begin{equation}\label{Eq.2.1}
\sum_{n=1}^{\infty}\frac{\delta_{1}\cdots\delta_{n}}{\beta_{1}\cdots\beta_{n}}=\sum_{n=1}^{\infty}\frac{q_{1}\cdots
q_{n}}{p_{1}\cdots p_{n}}=\infty
\end{equation}
\end{Prop}

\begin{Lem}
Independientemente de la recurrencia o transitoriedad de la cadena, hay una y s\'olo una, salvo m\'ultiplos, soluci\'on $\nu$
a $\nu\Lambda=0$, dada por
\begin{equation}\label{Eq.2.2}
\nu_{n}=\frac{\beta_{0}\cdots\beta_{n-1}}{\delta_{1}\cdots\delta_{n}}\nu_{0}
\end{equation}
\end{Lem}

\begin{Cor}\label{Corolario2.3}
En el caso recurrente, la medida estacionaria $\mu$ para
$\left\{Y_{n}\right\}$ est\'a dada por
\begin{equation}\label{Eq.2.3}
\mu_{n}=\frac{p_{1}\cdots p_{n-1}}{q_{1}\cdots q_{n}}\mu_{0}
\end{equation}
para $n=1,2,\ldots$
\end{Cor}

\begin{Def}
Una medida $\nu$ es estacionaria si $0\leq\nu_{j}<\infty$ y para toda $t$ se cumple que $\nu P^{t}=nu$.
\end{Def}


\begin{Def}
Un proceso irreducible recurrente con medida estacionaria con masa finita es llamado erg\'odico.
\end{Def}

\begin{Teo}\label{Teo4.3}
Un Proceso de Saltos de Markov irreducible no explosivo es erg\'odico si y s\'olo si uno puede encontrar una soluci\'on $\pi$ de probabilidad, $|\pi|=1$, $0\leq\pi_{j}\leq1$ para $\nu\Lambda=0$. En este caso $\pi$ es la distribuci\'on estacionaria.
\end{Teo}
\begin{Cor}\label{Corolario2.4}
$\left\{X_{t}\right\}_{t\geq0}$ es erg\'odica si y s\'olo si (\ref{Eq.2.1}) se cumple y $S<\infty$, en cuyo caso la distribuci\'on estacionaria $\pi$ est\'a dada por

\begin{equation}\label{Eq.2.4}
\pi_{0}=\frac{1}{S}\textrm{,
}\pi_{n}=\frac{1}{S}\frac{\beta_{0}\cdots\beta_{n-1}}{\delta_{1}\cdots\delta_{n}}\textrm{,
}n=1,2,\ldots
\end{equation}
\end{Cor}


Sea $E$ espacio discreto de estados, finito o numerable, y sea $\left\{X_{t}\right\}$ un proceso de Markov con espacio de estados $E$. Una medida $\mu$ en $E$ definida por sus probabilidades puntuales $\mu_{i}$, escribimos $p_{ij}^{t}=P^{t}\left(i,\left\{j\right\}\right)=P_{i}\left(X_{t}=j\right)$.

El monto del tiempo gastado en cada estado es positivo, de modo tal que las trayectorias muestrales son constantes por partes. Para un proceso de saltos denotamos por los tiempos de saltos a $S_{0}=0<S_{1}<S_{2}\cdots$, los tiempos entre saltos consecutivos $T_{n}=S_{n+1}-S_{n}$ y la secuencia de estados visitados por $Y_{0},Y_{1},\ldots$, as\'i las trayectorias muestrales son constantes entre $S_{n}$ consecutivos, continua por la derecha, es decir, $X_{S_{n}}=Y_{n}$. 

La descripci\'on de un modelo pr\'actico est\'a dado usualmente en t\'erminos de las intensidades $\lambda\left(i\right)$ y las probabilidades de salto $q_{ij}$ m\'as que en t\'erminos de la matriz de transici\'on $P^{t}$. Sup\'ongase de ahora en adelante que $q_{ii}=0$ cuando $\lambda\left(i\right)>0$

\begin{Teo}
Cualquier Proceso de Markov de Saltos satisface la Propiedad
Fuerte de Markov
\end{Teo}

\begin{Def}
Una medida $v\neq0$ es estacionaria si $0\leq v_{j}<\infty$, $vP^{t}=v$ para toda $t$.
\end{Def}

\begin{Teo}\label{Teo.4.2}
Supongamos que $\left\{X_{t}\right\}$ es irreducible recurrente en $E$. Entonces existe una y s\'olo una, salvo m\'ultiplos, medida estacionaria $v$. Esta $v$ tiene la propiedad de que $0<v_{j}<\infty$ para todo $j$ y puede encontrarse en cualquiera de las siguientes formas

\begin{itemize}
\item[i)] Para alg\'un estado $i$, fijo pero arbitrario, $v_{j}$ es el tiempo esperado utilizado en $j$ entre dos llegadas consecutivas al estado $i$;
\begin{equation}\label{Eq.4.2}
v_{j}=\esp_{i}\int_{0}^{w\left(i\right)}\indora\left(X_{t}=j\right)dt
\end{equation}
con $w\left(i\right)=\inf\left\{t>0:X_{t}=i,X_{t^{-}}=\lim_{s\uparrow t}X_{s}\neq i\right\}$. 
\item[ii)]
$v_{j}=\frac{\mu_{j}}{\lambda\left(j\right)}$, donde $\mu$ es estacionaria para $\left\{Y_{n}\right\}$. \item[iii)] como
soluci\'on de $v\Lambda=0$.
\end{itemize}
\end{Teo}

\begin{Def}
Un proceso irreducible recurrente con medida estacionaria de masa
finita es llamado erg\'odico.
\end{Def}

\begin{Teo}\label{Teo.4.3}
Un proceso de Markov de saltos irreducible no explosivo es erg\'odico si y s\'olo si se puede encontrar una soluci\'on, de probabilidad, $\pi$, con $|\pi|=1$ y $0\leq\pi_{j}\leq1$, a $\pi\Lambda=0$. En este caso $\pi$ es la distribuci\'on estacionaria.
\end{Teo}

\begin{Cor}\label{Cor.4.4}
Una condici\'on suficiente para la ergodicidad de un proceso irreducible es la existencia de una probabilidad $\pi$ que resuelva el sistema $\pi\Lambda=0$ y que adem\'as tenga la propiedad de que $\sum\pi_{j}\lambda\left(j\right)$.
\end{Cor}

%_____________________________________________________________________________________
%
\section{Matriz Intensidad}
%_____________________________________________________________________________________
%


\begin{Def}
La matriz intensidad
$\Lambda=\left(\lambda\left(i,j\right)\right)_{i,j\in E}$ del proceso de saltos $\left\{X_{t}\right\}_{t\geq0}$ est\'a dada por
\begin{eqnarray*}
\lambda\left(i,j\right)&=&\lambda\left(i\right)q_{i,j}\textrm{,    }j\neq i\\
\lambda\left(i,i\right)&=&-\lambda\left(i\right)
\end{eqnarray*}
\end{Def}


\begin{Prop}\label{Prop.3.1}
Una matriz $E\times E$, $\Lambda$ es la matriz de intensidad de un proceso markoviano de saltos $\left\{X_{t}\right\}_{t\geq0}$ si y s\'olo si
\begin{eqnarray*}
\lambda\left(i,i\right)\leq0\textrm{, }\lambda\left(i,j\right)\textrm{,   }i\neq j\textrm{,  }\sum_{j\in E}\lambda\left(i,j\right)=0.
\end{eqnarray*}
Adem\'as, $\Lambda$ est\'a en correspondencia uno a uno con la
distribuci\'on del proceso minimal dado por el teorema 3.1.
\end{Prop}


Para el caso particular de la Cola $M/M/1$, la matr\'iz de itensidad est\'a dada por
\begin{eqnarray*}
\Lambda=\left[\begin{array}{cccccc}
-\beta & \beta & 0 &0 &0& \cdots\\
\delta & -\beta-\delta & \beta & 0 & 0 &\cdots\\
0 & \delta & -\beta-\delta & \beta & 0 &\cdots\\
\vdots & & & & & \ddots\\
\end{array}\right]
\end{eqnarray*}


%____________________________________________________________________________
\section{Medidas Estacionarias}
%____________________________________________________________________________
%


\begin{Def}
Una medida $v\neq0$ es estacionaria si $0\leq v_{j}<\infty$, $vP^{t}=v$ para toda $t$.
\end{Def}

\begin{Teo}\label{Teo.4.2}
Supongamos que $\left\{X_{t}\right\}$ es irreducible recurrente en $E$. Entonces existe una y s\'olo una, salvo m\'ultiplos, medida estacionaria $v$. Esta $v$ tiene la propiedad de que $0<v_{j}<\infty$ para todo $j$ y puede encontrarse en cualquiera de las siguientes formas

\begin{itemize}
\item[i)] Para alg\'un estado $i$, fijo pero arbitrario, $v_{j}$ es el tiempo esperado utilizado en $j$ entre dos llegadas consecutivas al estado $i$;
\begin{equation}\label{Eq.4.2}
v_{j}=\esp_{i}\int_{0}^{w\left(i\right)}\indora\left(X_{t}=j\right)dt
\end{equation}
con $w\left(i\right)=\inf\left\{t>0:X_{t}=i,X_{t^{-}}=\lim_{s\uparrow t}X_{s}\neq i\right\}$. 
\item[ii)]
$v_{j}=\frac{\mu_{j}}{\lambda\left(j\right)}$, donde $\mu$ es estacionaria para $\left\{Y_{n}\right\}$. 
\item[iii)] como soluci\'on de $v\Lambda=0$.
\end{itemize}
\end{Teo}


%____________________________________________________________________________
\section{Criterios de Ergodicidad}
%____________________________________________________________________________
%

\begin{Def}
Un proceso irreducible recurrente con medida estacionaria de masa finita es llamado erg\'odico.
\end{Def}

\begin{Teo}\label{Teo.4.3}
Un proceso de Markov de saltos irreducible no explosivo es erg\'odico si y s\'olo si se puede encontrar una soluci\'on, de probabilidad, $\pi$, con $|\pi|=1$ y $0\leq\pi_{j}\leq1$, a $\pi\Lambda=0$. En este caso $\pi$ es la distribuci\'on estacionaria.
\end{Teo}

\begin{Cor}\label{Cor.4.4}
Una condici\'on suficiente para la ergodicidad de un proceso irreducible es la existencia de una probabilidad $\pi$ que resuelva el sistema $\pi\Lambda=0$ y que adem\'as tenga la propiedad de que $\sum\pi_{j}\lambda\left(j\right)<\infty$.
\end{Cor}

\begin{Prop}
Si el proceso es erg\'odico, entonces existe una versi\'on estrictamente estacionaria
$\left\{X_{t}\right\}_{-\infty<t<\infty}$con doble tiempo
infinito.
\end{Prop}

\begin{Teo}
Si $\left\{X_{t}\right\}$ es erg\'odico y $\pi$ es la distribuci\'on estacionaria, entonces para todo $i,j$, $p_{ij}^{t}\rightarrow\pi_{j}$ cuando $t\rightarrow\infty$.
\end{Teo}

\begin{Cor}
Si $\left\{X_{t}\right\}$ es irreducible recurente pero no erg\'odica, es decir $|v|=\infty$, entonces $p_{ij}^{t}\rightarrow0$ para todo $i,j\in E$.
\end{Cor}

\begin{Cor}
Para cualquier proceso Markoviano de Saltos minimal, irreducible o
no, los l\'imites $li_{t\rightarrow\infty}p_{ij}^{t}$ existe.
\end{Cor}


%_____________________________________________________________________________________
%
\section{Procesos de Nacimiento y Muerte}
%_____________________________________________________________________________________
%

\begin{Prop}\label{Prop.2.1}
La recurrencia de $\left\{X_{t}\right\}$, o equivalentemente de
$\left\{Y_{n}\right\}$ es equivalente a
\begin{equation}\label{Eq.2.1}
\sum_{n=1}^{\infty}\frac{\delta_{1}\cdots\delta_{n}}{\beta_{1}\cdots\beta_{n}}=\sum_{n=1}^{\infty}\frac{q_{1}\cdots
q_{n}}{p_{1}\cdots p_{n}}=\infty
\end{equation}
\end{Prop}

\begin{Lema}\label{Lema.2.2}
Independientemente de la recurrencia o transitorieadad, existe una
y s\'olo una, salvo m\'ultiplos, soluci\'on a $v\Lambda=0$, dada por
\begin{equation}\label{Eq.2.2}
v_{n}=\frac{\beta_{0}\cdots\beta_{n-1}}{\delta_{1}\cdots\delta_{n}}v_{0}
\end{equation}
para $n=1,2,\ldots$.
\end{Lema}

\begin{Cor}\label{Cor.2.3}
En el caso recurrente, la medida estacionaria $\mu$ para
$\left\{Y_{n}\right\}$ est\'a dada por
\begin{equation}
\mu_{n}=\frac{p_{1}\cdots p_{n-1}}{q_{1}\cdots q_{n}}\mu_{0}
\end{equation}
para $n=1,2,\ldots$.
\end{Cor}

Se define a
$S=1+\sum_{n=1}^{\infty}\frac{\beta_{0}\beta_{1}\cdots\beta_{n-1}}{\delta_{1}\delta_{2}\cdots\delta_{n}}$

\begin{Cor}\label{Cor.2.4}
$\left\{X_{t}\right\}$ es erg\'odica si y s\'olo si la ecuaci\'on
(\ref{Eq.2.1}) se cumple y adem\'as $S<\infty$, en cuyo caso la
distribuci\'on erg\'odica, $\pi$, est\'a dada por
\begin{equation}\label{Eq.2.4}
\pi_{0}=\frac{1}{S}\textrm{,
}\pi_{n}=\frac{1}{S}\frac{\beta_{0}\cdots\beta_{n-1}}{\delta_{1}\cdots\delta_{n}}
\end{equation}
para $n=1,2,\ldots$.
\end{Cor}
%_____________________________________________________________________________________
\section{Procesos de Nacimiento y Muerte Generales}
%_____________________________________________________________________________________

Por un proceso de nacimiento y muerte se entiende un proceso de saltos de markov $\left\{X_{t}\right\}_{t\geq0}$ con espacio de estados a lo m\'as numerable, con la propiedad de que s\'olo puede ir al estado $n+1$ o al estado $n-1$, es decir, su matriz de intensidad es de la forma
\begin{eqnarray*}
\Lambda=\left[\begin{array}{cccccc}
-\beta_{0} & \beta_{0} & 0 &0 &0& \cdots\\
\delta_{1} & -\beta_{1}-\delta_{1} & \beta_{1} & 0 & 0 &\cdots\\
0 & \delta_{2} & -\beta_{2}-\delta_{2} & \beta_{2} & 0 &\cdots\\
\vdots & & & & & \ddots\\
\end{array}\right]
\end{eqnarray*}
donde $\beta_{n}$ son las intensidades de nacimiento y $\delta_{n}$ las intensidades de muerte, o tambi\'en se puede ver como a $X_{t}$ el n\'umero de usuarios en una cola al tiempo $t$, un salto hacia arriba corresponde a la llegada de un nuevo usuario y un salto hacia abajo como al abandono de un usuario despu\'es de haber recibido su servicio.

La cadena de saltos $\left\{Y_{n}\right\}$ tiene matriz de transici\'on dada por
\begin{eqnarray*}
Q=\left[\begin{array}{cccccc}
0 & 1 & 0 &0 &0& \cdots\\
q_{1} & 0 & p_{1} & 0 & 0 &\cdots\\
0 & q_{2} & 0 & p_{2} & 0 &\cdots\\
\vdots & & & & & \ddots\\
\end{array}\right]
\end{eqnarray*}
donde $p_{n}=\frac{\beta_{n}}{\beta_{n}+\delta_{n}}$ y $q_{n}=1-p_{n}=\frac{\delta_{n}}{\beta_{n}+\delta_{n}}$, donde adem\'as se asumne por el momento que $p_{n}$ no puede tomar el valor $0$ \'o $1$ para cualquier valor de $n$.

\begin{Prop}\label{Prop.2.1}
La recurrencia de $\left\{X_{t}\right\}$, o equivalentemente de $\left\{Y_{n}\right\}$ es equivalente a
\begin{equation}\label{Eq.2.1}
\sum_{n=1}^{\infty}\frac{\delta_{1}\cdots\delta_{n}}{\beta_{1}\cdots\beta_{n}}=\sum_{n=1}^{\infty}\frac{q_{1}\cdots q_{n}}{p_{1}\cdots p_{n}}=\infty
\end{equation}
\end{Prop}

\begin{Lema}\label{Lema.2.2}
Independientemente de la recurrencia o transitorieadad, existe una y s\'olo una, salvo m\'ultiplos, soluci\'on a $v\Lambda=0$, dada por
\begin{equation}\label{Eq.2.2}
v_{n}=\frac{\beta_{0}\cdots\beta_{n-1}}{\delta_{1}\cdots\delta_{n}}v_{0}
\end{equation}
para $n=1,2,\ldots$.
\end{Lema}

\begin{Cor}\label{Cor.2.3}
En el caso recurrente, la medida estacionaria $\mu$ para $\left\{Y_{n}\right\}$ est\'a dada por
\begin{equation}\label{Eq.}
\mu_{n}=\frac{p_{1}\cdots p_{n-1}}{q_{1}\cdots q_{n}}\mu_{0}
\end{equation}
para $n=1,2,\ldots$.
\end{Cor}

Se define a $S=1+\sum_{n=1}^{\infty}\frac{\beta_{0}\beta_{1}\cdots\beta_{n-1}}{\delta_{1}\delta_{2}\cdots\delta_{n}}$.

\begin{Cor}\label{Cor.2.4}
$\left\{X_{t}\right\}$ es erg\'odica si y s\'olo si la ecuaci\'on (\ref{Eq.2.1}) se cumple y adem\'as $S<\infty$, en cuyo caso la distribuci\'on erg\'odica, $\pi$, est\'a dada por
\begin{equation}\label{Eq.2.4}
\pi_{0}=\frac{1}{S}\textrm{,     }\pi_{n}=\frac{1}{S}\frac{\beta_{0}\cdots\beta_{n-1}}{\delta_{1}\cdots\delta_{n}}
\end{equation}
para $n=1,2,\ldots$.
\end{Cor}





%_____________________________________________________________________________________
%
\section{Notaci\'on Kendall-Lee}
%_____________________________________________________________________________________
%

A partir de este momento se har\'an las siguientes consideraciones:
\begin{itemize}
\item[a) ]Si $t_{n}$ es el tiempo aleatorio en el que llega al sistema el $n$-\'esimo cliente, para $n=1,2,\ldots$, $t_{0}=0$ y $t_{0}<t_{1}<\cdots$ se definen los tiempos entre arribos $\tau_{n}=t_{n}-t_{n-1}$ para $n=1,2,\ldots$, variables aleatorias independientes e id\'enticamente distribuidas.

\item[b) ]Los tiempos entre arribos tienen un valor medio $E\left(\tau\right)$ finito y positivo $\frac{1}{\beta}$, es decir, $\beta$ se puede ver como la tasa o intensidad promedio de arribos al sistema por unidad de tiempo.
\item[c) ]  Adem\'as se supondr\'a que los servidores son identicos y si $s$ denota la variable aleatoria que describe el tiempo de servicio, entonces $E\left(s\right)=\frac{1}{\delta}$, $\delta$ es la tasa promedio de servicio por servidor.
\end{itemize}


La notaci\'on de Kendall-Lee es una forma abreviada de describir un sistema de espera con las siguientes componentes:
\begin{itemize}
\item[a)] {\em\bf Fuente}: Poblaci\'on de clientes potenciales del sistema, esta puede ser finita o infinita. 
\item[b)] {\em\bf Proceso de Arribos}: Proceso determinado por la funci\'on de distribuci\'on $A\left(t\right)=P\left\{\tau\leq t\right\}$ de los tiempos entre arribos.
\end{itemize}

Adem\'as tenemos las siguientes igualdades
\begin{equation}\label{Eq.0.1}
N\left(t\right)=N_{q}\left(t\right)+N_{s}\left(s\right)
\end{equation}
donde
\begin{itemize}
\item $N\left(t\right)$ es el n\'umero de clientes en el sistema al tiempo $t$. 
\item $N_{q}\left(t\right)$ es el n\'umero de cliente en la cola al tiempo $t$.
\item $N_{s}\left(t\right)$ es el n\'umero de clientes recibiendo servicio en el tiempo $t$.
\end{itemize}

Bajo la hip\'otesis de estacionareidad, es decir, las caracter\'isticas de funcionamiento del sistema se han estabilizado en valores independientes del tiempo, entonces
\begin{equation}
N=N_{q}+N_{s}.
\end{equation}

Los valores medios de las cantidades anteriores se escriben como $L=E\left(N\right)$, $L_{q}=E\left(N_{q}\right)$ y $L_{s}=E\left(N_{s}\right)$, entonces de la ecuaci\'on \ref{Eq.0.1} se obtiene

\begin{equation}
L=L_{q}+L_{s}
\end{equation}
Si $q$ es el tiempo que pasa un cliente en la cola antes de recibir servicio, y W es el tiempo total que un cliente pasa en el sistema, entonces \[w=q+s\] por lo tanto \[W=W_{q}+W_{s},\] donde $W=E\left(w\right)$, $W_{q}=E\left(q\right)$ y $W_{s}=E\left(s\right)=\frac{1}{\delta}$.

La intensidad de tr\'afico se define como
\begin{equation}
\rho=\frac{E\left(s\right)}{E\left(\tau\right)}=\frac{\beta}{\delta}.
\end{equation}

La utilizaci\'on por servidor es
\begin{equation}
u=\frac{\rho}{c}=\frac{\beta}{c\delta}.
\end{equation}
donde $c$ es el n\'umero de servidores.

Esta notaci\'on es una forma abreviada de describir un sistema de espera con componentes dados a continuaci\'on, la notaci\'on es

\begin{equation}\label{Notacion.K.L.}
A/S/c/K/F/d
\end{equation}

Cada una de las letras describe:

\begin{itemize}
\item $A$ es la distribuci\'on de los tiempos entre arribos.
\item $S$ es la distribuci\'on del tiempo de servicio.
\item $c$ es el n\'umero de servidores.
\item $K$ es la capacidad del sistema.
\item $F$ es el n\'umero de individuos en la fuente.
\item $d$ es la disciplina del servicio
\end{itemize}

Usualmente se acostumbra suponer que $K=\infty$, $F=\infty$ y $d=FIFO$, es decir, First In First Out. Las distribuciones usuales para $A$ y $B$ son:

\begin{itemize}
\item $GI$ para la distribuci\'on general de los tiempos entre arribos.
\item $G$ distribuci\'on general del tiempo de servicio.
\item $M$ Distribuci\'on exponencial para $A$ o $S$.
\item $E_{K}$ Distribuci\'on Erlang-$K$, para $A$ o $S$.
\item $D$ tiempos entre arribos o de servicio constantes, es decir, deterministicos.
\end{itemize}


%_____________________________________________________________________________________
%
\subsection{Cola $M/M/1$}
%_____________________________________________________________________________________
%
Este modelo corresponde a un proceso de nacimiento y muerte con $\beta_{n}=\beta$ y $\delta_{n}=\delta$ independiente del valor de $n$. La intensidad de tr\'afico $\rho=\frac{\beta}{\delta}$, implica que el criterio de recurrencia (ecuaci\'on \ref{Eq.2.1}) quede de la forma:
\begin{eqnarray*}
1+\sum_{n=1}^{\infty}\rho^{-n}=\infty.
\end{eqnarray*}
Equivalentemente el proceso es recurrente si y s\'olo si
\begin{eqnarray*}
\sum_{n\geq1}\left(\frac{\beta}{\delta}\right)^{n}<\infty\Leftrightarrow \frac{\beta}{\delta}<1.
\end{eqnarray*}
Entonces
$S=\frac{\delta}{\delta-\beta}$, luego por la ecuaci\'on \ref{Eq.2.4} se tiene que
\begin{eqnarray*}
\pi_{0}&=&\frac{\delta-\beta}{\delta}=1-\frac{\beta}{\delta},\\
\pi_{n}&=&\pi_{0}\left(\frac{\beta}{\delta}\right)^{n}=\left(1-\frac{\beta}{\delta}\right)\left(\frac{\beta}{\delta}\right)^{n}=\left(1-\rho\right)\rho^{n}.
\end{eqnarray*}


Lo cual nos lleva a la siguiente proposici\'on:

\begin{Prop}
La cola $M/M/1$ con intensidad de tr\'afico $\rho$, es recurrente si y s\'olo si $\rho\leq1$.
\end{Prop}

Entonces por el corolario \ref{Cor.2.3}

\begin{Prop}
La cola $M/M/1$ con intensidad de tr\'afico $\rho$ es erg\'odica si y s\'olo si $\rho<1$. En cuyo caso, la distribuci\'on de equilibrio $\pi$ de la longitud de la cola es geom\'etrica, $\pi_{n}=\left(1-\rho\right)\rho^{n}$, para $n=1,2,\ldots$.
\end{Prop}

De la proposici\'on anterior se desprenden varios hechos importantes.
\begin{itemize}
\item[a) ] $\prob\left[X_{t}=0\right]=\pi_{0}=1-\rho$, es decir, la probabilidad de que el sistema se encuentre ocupado.
\item[b) ] De las propiedades de la distribuci\'on Geom\'etrica se desprende que
\begin{itemize}
\item[i) ] $\esp\left[X_{t}\right]=\frac{\rho}{1-\rho}$,
\item[ii) ] $Var\left[X_{t}\right]=\frac{\rho}{\left(1-\rho\right)^{2}}$.
\end{itemize}
\end{itemize}

Si $L$ es el n\'umero esperado de clientes en el sistema, incluyendo los que est\'an siendo atendidos, entonces
\begin{eqnarray}
L=\frac{\rho}{1-\rho}.
\end{eqnarray}
Si adem\'as $W$ es el tiempo total del cliente en la cola: $W=W_{q}+W_{s}$, $\rho=\frac{\esp\left[s\right]}{\esp\left[\tau\right]}=\beta W_{s}$, puesto que $W_{s}=\esp\left[s\right]$ y $\esp\left[\tau\right]=\frac{1}{\delta}$. Por la f\'ormula de Little $L=\lambda W$
\begin{eqnarray*}
W&=&\frac{L}{\beta}=\frac{\frac{\rho}{1-\rho}}{\beta}=\frac{\rho}{\delta}\frac{1}{1-\rho}=\frac{W_{s}}{1-\rho}\\
&=&\frac{1}{\delta\left(1-\rho\right)}=\frac{1}{\delta-\beta},
\end{eqnarray*}

luego entonces

\begin{eqnarray*}
W_{q}&=&W-W_{s}=\frac{1}{\delta-\beta}-\frac{1}{\delta}=\frac{\beta}{\delta(\delta-\beta)}\\
&=&\frac{\rho}{1-\rho}\frac{1}{\delta}=\esp\left[s\right]\frac{\rho}{1-\rho}.
\end{eqnarray*}

Entonces

\begin{eqnarray*}
L_{q}=\beta W_{q}=\frac{\rho^{2}}{1-\rho}.
\end{eqnarray*}

Finalmente, tenemos las siguientes proposiciones:

\begin{Prop}
\begin{enumerate}
\item $W\left(t\right)=1-e^{-\frac{t}{W}}$.
\item $W_{q}\left(t\right)=1-\rho\exp^{-\frac{t}{W}}$.
\end{enumerate}
donde $W=\esp(w)$.
\end{Prop}

\begin{Prop}
La cola M/M/1 con intensidad de tr\'afico $\rho$ es recurrente si
y s\'olo si $\rho\leq1$
\end{Prop}

\begin{Prop}
La cola M/M/1 con intensidad de tr\'afica $\rho$ es ergodica si y
s\'olo si $\rho<1$. En este caso, la distribuci\'on de equilibrio
$\pi$ de la longitud de la cola es geom\'etrica,
$\pi_{n}=\left(1-\rho\right)\rho^{n}$, para $n=0,1,2,\ldots$.
\end{Prop}
%_____________________________________________________________________________________
%
\subsection{Cola $M/M/\infty$}
%_____________________________________________________________________________________
%

Este tipo de modelos se utilizan para estimar el n\'umero de l\'ineas en uso en una gran red comunicaci\'on o para estimar valores en los sistemas $M/M/c$ o $M/M/c/c$, en el se puede pensar que siempre hay un servidor disponible para cada cliente que llega.

Se puede considerar como un proceso de nacimiento y muerte con par\'ametros $\beta_{n}=\beta$ y $\mu_{n}=n\mu$ para $n=0,1,2,\ldots$. Este modelo corresponde al caso en que $\beta_{n}=\beta$ y $\delta_{n}=n\delta$, en este caso el par\'ametro de inter\'es $\eta=\frac{\beta}{\delta}$, luego, la ecuaci\'on \ref{Eq.2.1} queda de la forma:

\begin{eqnarray*}
\sum_{n=1}^{\infty}\frac{\delta_{1}\cdots\delta_{n}}{\beta_{1}\cdots\beta_{n}}=\sum_{n=1}^{\infty}n!\eta^{-n}=\infty\\
\end{eqnarray*}
con $S=1+\sum_{n=1}^{\infty}\frac{\eta^{n}}{n!}=e$, entonces por la ecuaci\'on \ref{Eq.2.4} se tiene que

\begin{eqnarray}\label{MMinf.pi}
\pi_{0}=e^{\rho},\\
\pi_{n}=e^{-\rho}\frac{\rho^{n}}{n!}.
\end{eqnarray}
Entonces, el n\'umero promedio de servidores ocupados es equivalente a considerar el n\'umero de clientes en el  sistema, es decir,
\begin{eqnarray}
L=\esp\left[N\right]=\rho.\\
Var\left[N\right]=\rho.
\end{eqnarray}
Adem\'as se tiene que $W_{q}=0$ y $L_{q}=0$. El tiempo promedio en el sistema es el tiempo promedio de servicio, es decir, $W=\esp\left[s\right]=\frac{1}{\delta}$.Resumiendo, tenemos la sisuguiente proposici\'on:

\begin{Prop}
La cola $M/M/\infty$ es erg\'odica para todos los valores de $\eta$. La distribuci\'on de equilibrio $\pi$ es Poisson con media $\eta$,
\begin{eqnarray}
\pi_{n}=\frac{e^{-n}\eta^{n}}{n!}.
\end{eqnarray}
\end{Prop}
%_____________________________________________________________________________________
%
\subsection{Cola $M/M/m$}
%_____________________________________________________________________________________
%

Este sistema considera $m$ servidores id\'enticos, con tiempos entre arribos y de servicio exponenciales con medias $\esp\left[\tau\right]=\frac{1}{\beta}$ y
$\esp\left[s\right]=\frac{1}{\delta}$. definimos ahora la utilizaci\'on por servidor como $u=\frac{\rho}{m}$ que tambi\'en se puede interpretar como la fracci\'on de tiempo promedio que cada servidor est\'a ocupado.

La cola $M/M/m$ se puede considerar como un proceso de nacimiento y muerte con par\'ametros: $\beta_{n}=\beta$ para $n=0,1,2,\ldots$ y
\begin{eqnarray}
\delta_{n}=\left\{\begin{array}{cc}
n\delta & n=0,1,\ldots,m-1\\
c\delta & n=m,\ldots\\
\end{array}\right.
\end{eqnarray}

entonces  la condici\'on de recurrencia se va a cumplir s\'i y s\'olo si $\sum_{n\geq1}\frac{\beta_{0}\cdots\beta_{n-1}}{\delta_{1}\cdots\delta_{n}}<\infty$,
equivalentemente se debe de cumplir que
\begin{eqnarray*}
S&=&1+\sum_{n\geq1}\frac{\beta_{0}\cdots\beta_{n-1}}{\delta_{1}\cdots\delta_{n}}=\sum_{n=0}^{m-1}\frac{\beta_{0}\cdots\beta_{n-1}}{\delta_{1}\cdots\delta_{n}}+\sum_{n=0}^{\infty}\frac{\beta_{0}\cdots\beta_{n-1}}{\delta_{1}\cdots\delta_{n}}\\
&=&\sum_{n=0}^{m-1}\frac{\beta^{n}}{n!\delta^{n}}+\sum_{n=0}^{\infty}\frac{\rho^{m}}{m!}u^{n}
\end{eqnarray*}
converja, lo cual ocurre si $u<1$, en este caso

\begin{eqnarray}
S=\sum_{n=0}^{m-1}\frac{\rho^{n}}{n!}+\frac{\rho^{m}}{m!}\left(1-u\right)
\end{eqnarray}
luego, para este caso se tiene que

\begin{eqnarray}
\pi_{0}&=&\frac{1}{S}\\
\pi_{n}&=&\left\{\begin{array}{cc}
\pi_{0}\frac{\rho^{n}}{n!} & n=0,1,\ldots,m-1\\
\pi_{0}\frac{\rho^{n}}{m!m^{n-m}}& n=m,\ldots\\
\end{array}\right.
\end{eqnarray}
Al igual que se hizo antes, determinaremos los valores de
$L_{q},W_{q},W$ y $L$:
\begin{eqnarray*}
L_{q}&=&\esp\left[N_{q}\right]=\sum_{n=0}^{\infty}\left(n-m\right)\pi_{n}=\sum_{n=0}^{\infty}n\pi_{n+m}\\
&=&\sum_{n=0}^{\infty}n\pi_{0}\frac{\rho^{n+m}}{m!m^{n+m}}=\pi_{0}\frac{\rho^{m}}{m!}\sum_{n=0}^{\infty}nu^{n}=\pi_{0}\frac{u\rho^{m}}{m!}\sum_{n=0}^{\infty}\frac{d}{du}u^{n}\\
&=&\pi_{0}\frac{u\rho^{m}}{m!}\frac{d}{du}\sum_{n=0}^{\infty}u^{n}=\pi_{0}\frac{u\rho^{m}}{m!}\frac{d}{du}\left(\frac{1}{1-u}\right)=\pi_{0}\frac{u\rho^{m}}{m!}\frac{1}{\left(1-u\right)^{2}},
\end{eqnarray*}

es decir
\begin{equation}
L_{q}=\frac{u\pi_{0}\rho^{m}}{m!\left(1-u\right)^{2}},
\end{equation}
luego
\begin{equation}
W_{q}=\frac{L_{q}}{\beta}.
\end{equation}
Adem\'as
\begin{equation}
W=W_{q}+\frac{1}{\delta}
\end{equation}

Si definimos
\begin{eqnarray}
C\left(m,\rho\right)=\frac{\pi_{0}\rho^{m}}{m!\left(1-u\right)}=\frac{\pi_{m}}{1-u},
\end{eqnarray}
que es la probabilidad de que un cliente que llegue al sistema
tenga que esperar en la cola. Entonces podemos reescribir las
ecuaciones reci\'en enunciadas:

\begin{eqnarray}
L_{q}&=&\frac{C\left(m,\rho\right)u}{1-u},\\
W_{q}&=&\frac{C\left(m,\rho\right)\esp\left[s\right]}{m\left(1-u\right)}\\
\end{eqnarray}
Por tanto tenemos las siguientes proposiciones:

\begin{Prop}
La cola $M/M/m$ con intensidad de tr\'afico $\rho$ es erg\'odica si y s\'olo si $\rho<1$. En este caso la distribuci\'on erg\'odica $\pi$ est\'a dada por
\begin{eqnarray}
\pi_{n}=\left\{\begin{array}{cc}
\frac{1}{S}\frac{\eta^{n}}{n!} & 0\leq n\leq m\\
\frac{1}{S}\frac{\eta^{m}}{m!}\rho^{n-m} & m\leq n<\infty\\
\end{array}\right.
\end{eqnarray}
\end{Prop}

\begin{Prop}
Para $t\geq0$
\begin{itemize}
\item[a)]
\begin{eqnarray}
W_{q}\left(t\right)=1-C\left(m,\rho\right)e^{-c\delta
t\left(1-u\right)}.
\end{eqnarray} 
\item[b)]\begin{eqnarray}
W\left(t\right)=\left\{\begin{array}{cc}
1+e^{-\delta t}\frac{\rho-m+W_{q}\left(0\right)}{m-1-\rho}+e^{-m\delta t\left(1-u\right)}\frac{C\left(m,\rho\right)}{m-1-\rho} & \rho\neq m-1\\
1-\left(1+C\left(m,\rho\right)\delta t\right)e^{-\delta t} & \rho=m-1\\
\end{array}\right.
\end{eqnarray}
\end{itemize}
\end{Prop}

Resumiendo, para este caso $\beta_{n}=\beta$ y
$\delta_{n}=m\left(n\right)\delta$, donde $m\left(n\right)$ es el n\'umero de servidores ocupados en el estado $n$, es decir,
$m\left(n\right)=m$, para $n\geq m$ y $m\left(n\right)=m$ para
$1\leq n\leq m$. La intensidad de tr\'afico es
$\rho=\frac{\beta}{m\delta}$ y $\frac{\beta_{n}}{\delta_{n}}=\rho$
para $n\geq m$. As\'i, al igual que en el caso $m=1$, la ecuaci\'on
\ref{Eq.2.1} y la recurrencia se cumplen si y s\'olo si
$\sum_{n=1}^{\infty}\rho^{-n}=\infty$, es decir, cuando
$\rho\leq1$. 


%_____________________________________________________________________________________
%
\subsection{Cola $M/M/m/m$}
%_____________________________________________________________________________________
%

Consideremos un sistema con $m$ servidores id\'enticos, pero ahora cada uno es de capacidad finita $m$. Si todos los servidores se encuentran ocupados, el siguiente usuario en llegar se pierde pues no se le deja esperar a que reciba servicio. Este tipo de sistemas pueden verse como un proceso de nacimiento y muerte con
\begin{eqnarray}
\beta_{n}=\left\{\begin{array}{cc}
\beta & n=0,1,2,\ldots,m-1\\
0 & n\geq m\\
\end{array}
\right.
\end{eqnarray}

\begin{eqnarray}
\delta_{n}=\left\{\begin{array}{cc}
n\delta & n=0,1,2,\ldots,m-1\\
0 & n\geq m\\
\end{array}
\right.
\end{eqnarray}
El proceso tiene epacio de estados finitos, $S=\left\{0,1,\ldots,m\right\}$, entonces de las ecuaciones que determinan la distribuci\'on estacionaria se tiene que
\begin{equation}\label{Eq.13.1}
\pi_{n}=\left\{\begin{array}{cc}
\pi_{0}\frac{\rho^{n}}{n!} & n=0,1,2,\ldots,m\\
0 & n\geq m\\
\end{array}
\right.
\end{equation}
y adem\'as
\begin{equation}
\pi_{0}=\left(\sum_{n=0}^{m}\frac{\rho^{n}}{n!}\right)^{-1}.
\end{equation}
A la ecuaci\'on \ref{Eq.13.1} se le llama {\em distribuci\'on truncada}. Si definimos
$\pi_{m}=B\left(m,\rho\right)=\pi_{0}\frac{\rho^{m}}{m!}$, $\pi_{m}$ representa la probabilidad de que todos los servidores se encuentren ocupados, y tambi\'en se le conoce como {\em f\'ormula de p\'erdida de Erlang}. Necesariamente en este caso el tiempo de espera en la cola $W_{q}$ y el n\'umero promedio de clientes en la cola $L_{q}$ deben de ser cero puesto que no se permite esperar para recibir servicio, m\'as a\'un, el tiempo de espera en el sistema y el tiempo de serivcio tienen la misma distribuci\'on, es decir,
\[W\left(t\right)=\prob\left\{w\leq t\right\}=1-e^{-\mu t},\] en particular
\[W=\esp\left[w\right]=\esp\left[s\right]=\frac{1}{\delta}.\]
Por otra parte, el n\'umero esperado de clientes en el sistema es
\begin{eqnarray*}
L&=&\esp\left[N\right]=\sum_{n=0}^{m}n\pi_{n}=\pi_{0}\rho\sum_{n=0}^{m}\frac{\rho^{n-1}}{\left(n-1\right)!}\\
&=&\pi_{0}\rho\sum_{n=0}^{m-1}\frac{\rho^{n}}{n!}
\end{eqnarray*}
entonces, se tiene que
\begin{equation}
L=\rho\left(1-B\left(m,\rho\right)\right)=\esp\left[s\right]\left(1-B\left(m,\rho\right)\right).
\end{equation}
Adem\'as
\begin{equation}
\delta_{q}=\delta\left(1-B\left(m,\rho\right)\right)
\end{equation}
representa la tasa promedio efectiva de arribos al sistema.
%_____________________________________________________________________________________
%
\subsection{Cola M/G/1}
%_____________________________________________________________________________________
%
Consideremos un sistema de espera con un servidor, en el que los tiempos entre arribos son exponenciales, y los tiempos de servicio tienen una distribuci\'on general $G$. Sea $N\left(t\right)_{t\geq0}$ el n\'umero de clientes en el sistema al tiempo $t$, y sean $t_{1}<t_{2}<\dots$ los tiempos sucesivos en los que los clientes completan su servicio y salen del sistema.

La sucesi\'on $\left\{X_{n}\right\}$ definida por
$X_{n}=N\left(t_{n}\right)$ es una cadena de Markov, en espec\'ifico es la Cadena encajada del proceso de llegadas de usuarios. Sea $U_{n}$ el n\'umero de clientes que llegan al sistema durante el tiempo de servicio del $n$-\'esimo cliente, entonces se tiene que

\begin{eqnarray*}
X_{n+1}=\left\{\begin{array}{cc}
X_{n}-1+U_{n+1} & \textrm{si }X_{n}\geq1,\\
U_{n+1} & \textrm{si }X_{n}=0\\
\end{array}\right.
\end{eqnarray*}

Dado que los procesos de arribos de los usuarios es Poisson con par\'ametro $\lambda$, la probabilidad condicional de que lleguen $j$ clientes al sistema dado que el tiempo de servicio es $s=t$, resulta:
\begin{eqnarray*}
\prob\left\{U=j|s=t\right\}=e^{-\lambda t}\frac{\left(\lambda
t\right)^{j}}{j!}\textrm{,   }j=0,1,\ldots
\end{eqnarray*}
por el teorema de la probabilidad total se tiene que
\begin{equation}
a_{j}=\prob\left\{U=j\right\}=\int_{0}^{\infty}\prob\left\{U=j|s=t\right\}dG\left(t\right)=\int_{0}^{\infty}e^{-\lambda
t}\frac{\left(\lambda t\right)^{j}}{j!}dG\left(t\right)
\end{equation}
donde $G$ es la distribuci\'on de los tiempos de servicio. Las probabilidades de transici\'on de la cadena est\'an dadas por
\begin{equation}
p_{0j}=\prob\left\{U_{n+1}=j\right\}=a_{j}\textrm{, para
}j=0,1,\ldots
\end{equation}
y para $i\geq1$
\begin{equation}
p_{ij}=\left\{\begin{array}{cc}
\prob\left\{U_{n+1}=j-i+1\right\}=a_{j-i+1}&\textrm{, para }j\geq i-1\\
0 & j<i-1\\
\end{array}
\right.
\end{equation}
Entonces la matriz de transici\'on es:
\begin{eqnarray*}
P=\left[\begin{array}{ccccc}
a_{0} & a_{1} & a_{2} & a_{3} & \cdots\\
a_{0} & a_{1} & a_{2} & a_{3} & \cdots\\
0 & a_{0} & a_{1} & a_{2} & \cdots\\
0 & 0 & a_{0} & a_{1} & \cdots\\
\vdots & \vdots & \cdots & \ddots &\vdots\\
\end{array}
\right].
\end{eqnarray*}
Sea $\rho=\sum_{n=0}na_{n}$, entonces se tiene el siguiente teorema:
\begin{Teo}
La cadena encajada $\left\{X_{n}\right\}$ es
\begin{itemize}
\item[a)] Recurrente positiva si $\rho<1$,
\item[b)] Transitoria
si $\rho>1$, 
\item[c)] Recurrente nula si $\rho=1$.
\end{itemize}
\end{Teo}

Recordemos que si la cadena de Markov $\left\{X_{n}\right\}$ tiene una distribuci\'on estacionaria entonces existe una distribuci\'on de probabilidad $\pi=\left(\pi_{0},\pi_{1},\ldots,\right)$, con $\pi_{i}\geq0$ y $\sum_{i\geq1}\pi_{i}=1$ tal que satisface la
ecuaci\'on $\pi=\pi P$, equivalentemente
\begin{equation}\label{Eq.18.9}
\pi_{j}=\sum_{i=0}^{\infty}\pi_{k}p_{ij},\textrm{ para
}j=0,1,2,\ldots
\end{equation}
que se puede ver como
\begin{equation}\label{Eq.19.6}
\pi_{j}=\pi_{0}a_{j}+\sum_{i=1}^{j+1}\pi_{i}a_{j-i+1}\textrm{,
para }j=0,1,\ldots
\end{equation}
si definimos
\begin{eqnarray}
\pi\left(z\right)=\sum_{j=0}^{\infty}\pi_{j}z^{j}
\end{eqnarray}
y 
\begin{equation}
A\left(z\right)=\sum_{j=0}^{\infty}a_{j}z^{j}
\end{equation}
con $|z_{j}|\leq1$. Si la ecuaci\'on \ref{Eq.19.6} la multiplicamos por $z^{j}$ y sumando sobre $j$, se tiene que
\begin{eqnarray*}
\sum_{j=0}^{\infty}\pi_{j}z^{j}&=&\sum_{j=0}^{\infty}\pi_{0}a_{j}z^{j}+\sum_{j=0}^{\infty}\sum_{i=1}^{j+1}\pi_{i}a_{j-i+1}z^{j}\\
&=&\pi_{0}\sum_{j=0}^{\infty}a_{j}z^{j}+\sum_{j=0}^{\infty}a_{j}z^{j}\sum_{i=1}^{\infty}\pi_{i}a_{i-1}\\
&=&\pi_{0}A\left(z\right)+A\left(z\right)\left(\frac{\pi\left(z\right)-\pi_{0}}{z}\right)\\
\end{eqnarray*}
es decir,

\begin{equation}
\pi\left(z\right)=\pi_{0}A\left(z\right)+A\left(z\right)\left(\frac{\pi\left(z\right)-\pi_{0}}{z}\right)\Leftrightarrow\pi\left(z\right)=\frac{\pi_{0}A\left(z\right)\left(z-1\right)}{z-A\left(z\right)}
\end{equation}

Si $z\rightarrow1$, entonces $A\left(z\right)\rightarrow A\left(1\right)=1$, y adem\'as $A^{'}\left(z\right)\rightarrow A^{'}\left(1\right)=\rho$. Si aplicamos la Regla de L'Hospital se tiene que
\begin{eqnarray*}
\sum_{j=0}^{\infty}\pi_{j}=lim_{z\rightarrow1^{-}}\pi\left(z\right)=\pi_{0}lim_{z\rightarrow1^{-}}\frac{z-1}{z-A\left(z\right)}=\frac{\pi_{0}}{1-\rho}
\end{eqnarray*}
Retomando,
\begin{eqnarray*}
a_{j}=\prob\left\{U=j\right\}=\int_{0}^{\infty}e^{-\lambda
t}\frac{\left(\lambda t\right)^{n}}{n!}dG\left(t\right)\textrm{,
para }n=0,1,2,\ldots
\end{eqnarray*}
entonces
\begin{eqnarray*}
\rho&=&\sum_{n=0}^{\infty}na_{n}=\sum_{n=0}^{\infty}n\int_{0}^{\infty}e^{-\lambda t}\frac{\left(\lambda t\right)^{n}}{n!}dG\left(t\right)\\
&=&\int_{0}^{\infty}\sum_{n=0}^{\infty}ne^{-\lambda
t}\frac{\left(\lambda
t\right)^{n}}{n!}dG\left(t\right)=\int_{0}^{\infty}\lambda
tdG\left(t\right)=\lambda\esp\left[s\right]
\end{eqnarray*}

Adem\'as, se tiene que $\rho=\beta\esp\left[s\right]=\frac{\beta}{\delta}$ y la distribuci\'on estacionaria est\'a dada por
\begin{eqnarray}
\pi_{j}&=&\pi_{0}a_{j}+\sum_{i=1}^{j+1}\pi_{i}a_{j-i+1}\textrm{, para }j=0,1,\ldots\\
\pi_{0}&=&1-\rho.
\end{eqnarray}
Por otra parte se tiene que\begin{equation}
L=\pi^{'}\left(1\right)=\rho+\frac{A^{''}\left(1\right)}{2\left(1-\rho\right)}
\end{equation}

pero $A^{''}\left(1\right)=\sum_{n=1}n\left(n-1\right)a_{n}= \esp\left[U^{2}\right]-\esp\left[U\right]$, $\esp\left[U\right]=\rho$ y
$\esp\left[U^{2}\right]=\lambda^{2}\esp\left[s^{2}\right]+\rho$.
Por lo tanto $L=\rho+\frac{\beta^{2}\esp\left[s^{2}\right]}{2\left(1-\rho\right)}$.

De las f\'ormulas de Little, se tiene que $W=E\left(w\right)=\frac{L}{\beta}$, tambi\'en el tiempo de espera en la cola
\begin{equation}
W_{q}=\esp\left(q\right)=\esp\left(w\right)-\esp\left(s\right)=\frac{L}{\beta}-\frac{1}{\delta},
\end{equation}
adem\'as el n\'umero promedio de clientes en la cola es
\begin{equation}
L_{q}=\esp\left(N_{q}\right)=\beta W_{q}=L-\rho
\end{equation}


%____________________________________________________________________________
\subsection{Cola con Infinidad de Servidores}

Este caso corresponde a $\beta_{n}=\beta$ y $\delta_{n}=n\delta$. El par\'ametro de inter\'es es $\eta=\frac{\beta}{\delta}$, de donde se obtiene:
\begin{eqnarray*}
\sum_{n\geq0}\frac{\delta_{1}\cdots\delta_{n}}{\beta_{1}\cdots\beta_{n}}=\sum_{n=1}^{\infty}n!\eta^{n}=\infty,\\
S=1+\sum_{n=1}^{\infty}\frac{\eta^{n}}{n!}=e^{n}.
\end{eqnarray*}

\begin{Prop}
La cola $M/M/\infty$ es ergodica para todos los valores de $\eta$. La distribuci\'on de equilibrio $\pi$ es Poisson con media $\eta$, $\pi_{n}=\frac{e^{-n}\eta}{n!}$
\end{Prop}



%_____________________________________________________________________________________
%
\section{Redes de Colas:Sistemas Abiertos}
%_____________________________________________________________________________________

Considerese un sistema con dos servidores, en los cuales los usuarios llegan de acuerdo a un proceso poisson con intensidad $\lambda_{1}$ al primer servidor, despu\'es de ser atendido se pasa a la siguiente cola en el segundo servidor. Cada servidor atiende a un usuario a la vez con tiempo exponencial con raz\'on $\mu_{i}$, para $i=1,2$. A este tipo de sistemas se les conoce como sistemas secuenciales.

Def\'inase el par $\left(n,m\right)$ como el n\'umero de usuarios en el servidor 1 y 2 respectivamente. Las ecuaciones de balance son
\begin{eqnarray}\label{Eq.Balance}
\lambda P_{0,0}&=&\mu_{2}P_{0,1}\\
\left(\lambda+\mu_{1}\right)P_{n,0}&=&\mu_{2}P_{n,1}+\lambda P_{n-1,0}\\
\left(\lambda+\mu_{2}\right)P_{0,m}&=&\mu_{2}P_{0,m+1}+\mu_{1}P_{1,m-1}\\
\left(\lambda+\mu_{1}+\mu_{2}\right)P_{n,m}&=&\mu_{2}P_{n,m+1}+\mu_{1}P_{n+1,m-1}+\lambda
P_{n-1,m}
\end{eqnarray}

Cada servidor puede ser visto como un modelo de tipo $M/M/1$, de igual manera el proceso de salida de una cola $M/M/1$ con raz\'on $\lambda$, nos permite asumir que el servidor 2 tambi\'en es una cola $M/M/1$. Adem\'as la probabilidad de que haya $n$ usuarios en el servidor 1 es
\begin{eqnarray*}
P\left\{n\textrm{ en el servidor }1\right\}&=&\left(\frac{\lambda}{\mu_{1}}\right)^{n}\left(1-\frac{\lambda}{\mu_{1}}\right)=\rho_{1}^{n}\left(1-\rho_{1}\right)\\
P\left\{m\textrm{ en el servidor }2\right\}&=&\left(\frac{\lambda}{\mu_{2}}\right)^{n}\left(1-\frac{\lambda}{\mu_{2}}\right)=\rho_{2}^{m}\left(1-\rho_{2}\right)\\
\end{eqnarray*}
Si el n\'umero de usuarios en los servidores 1 y 2 son variables aleatorias independientes, se sigue que:
\begin{equation}\label{Eq.8.16}
P_{n,m}=\rho_{1}^{n}\left(1-\rho_{1}\right)\rho_{2}^{m}\left(1-\rho_{2}\right)
\end{equation}
Verifiquemos que $P_{n,m}$ satisface las ecuaciones de balance (\ref{Eq.Balance}) Antes de eso, enunciemos unas igualdades que nos ser\'an de utilidad:
\begin{eqnarray*}
\mu_{i}\rho_{i}&=&\lambda\textrm{ para }i=1,2.\\
\lambda P_{0,0}&=&\lambda\left(1-\rho_{1}\right)\left(1-\rho_{2}\right)\\
\textrm{ y }\mu_{2} P_{0,1}&=&\mu_{2}\left(1-\rho_{1}\right)\rho_{2}\left(1-\rho_{2}\right)\Rightarrow\\
\lambda P_{0,0}&=&\mu_{2} P_{0,1}\\
\left(\lambda+\mu_{2}\right)P_{0,m}&=&\left(\lambda+\mu_{2}\right)\left(1-\rho_{1}\right)\rho_{2}^{m}\left(1-\rho_{2}\right)\\
\mu_{2}P_{0,m+1}&=&\lambda\left(1-\rho_{1}\right)\rho_{2}^{m}\left(1-\rho_{2}\right)\\
&=&\mu_{2}\left(1-\rho_{1}\right)\rho_{2}^{m}\left(1-\rho_{2}\right)\\
\mu_{1}P_{1,m-1}&=&\frac{\lambda}{\rho_{2}}\left(1-\rho_{1}\right)\rho_{2}^{m}\left(1-\rho_{2}\right)\Rightarrow\\
\left(\lambda+\mu_{2}\right)P_{0,m}&=&\mu_{2}P_{0,m+1}+\mu_{1}P_{1,m-1}\\
\left(\lambda+\mu_{1}+\mu_{2}\right)P_{n,m}&=&\left(\lambda+\mu_{1}+\mu_{2}\right)\rho^{n}\left(1-\rho_{1}\right)\rho_{2}^{m}\left(1-\rho_{2}\right)\\
\mu_{2}P_{n,m+1}&=&\mu_{2}\rho_{2}\rho_{1}^{n}\left(1-\rho_{1}\right)\rho_{2}^{m}\left(1-\rho_{2}\right)\\
\mu_{1} P_{n-1,m-1}&=&\mu_{1}\frac{\rho_{1}}{\rho_{2}}\rho_{1}^{n}\left(1-\rho_{1}\right)\rho_{2}^{m}\left(1-\rho_{2}\right)\\
\lambda P_{n-1,m}&=&\frac{\lambda}{\rho_{1}}\rho_{1}^{n}\left(1-\rho_{1}\right)\rho_{2}^{m}\left(1-\rho_{2}\right)\\
\Rightarrow\left(\lambda+\mu_{1}+\mu_{2}\right)P_{n,m}&=&\mu_{2}P_{n,m+1}+\mu_{1} P_{n-1,m-1}+\lambda P_{n-1,m}\\
\end{eqnarray*}
entonces efectivamente la ecuaci\'on (\ref{Eq.8.16}) satisface las ecuaciones de balance (\ref{Eq.Balance}). El n\'umero promedio  de usuarios en el sistema, est\'a dado por
\begin{eqnarray*}
L&=&\sum_{n,m}\left(n+m\right)P_{n,m}=\sum_{n,m}nP_{n,m}+\sum_{n,m}mP_{n,m}\\
&=&\sum_{n}\sum_{m}nP_{n,m}+\sum_{m}\sum_{n}mP_{n,m}=\sum_{n}n\sum_{m}P_{n,m}+\sum_{m}m\sum_{n}P_{n,m}\\
&=&\sum_{n}n\sum_{m}\rho_{1}^{n}\left(1-\rho_{1}\right)\rho_{2}^{m}\left(1-\rho_{2}\right)+\sum_{m}m\sum_{n}\rho_{1}^{n}\left(1-\rho_{1}\right)\rho_{2}^{m}\left(1-\rho_{2}\right)\\
&=&\sum_{n}n\rho_{1}^{n}\left(1-\rho_{1}\right)\sum_{m}\rho_{2}^{m}\left(1-\rho_{2}\right)+\sum_{m}m\rho_{2}^{m}\left(1-\rho_{2}\right)\sum_{n}\rho_{1}^{n}\left(1-\rho_{1}\right)\\
&=&\sum_{n}n\rho_{1}^{n}\left(1-\rho_{1}\right)+\sum_{m}m\rho_{2}^{m}\left(1-\rho_{2}\right)\\
&=&\frac{\lambda}{\mu_{1}-\lambda}+\frac{\lambda}{\mu_{2}-\lambda}
\end{eqnarray*}



% ==__--__-- ==__--__-- ==__--__-- ==__--__--
%-------------- CAPITULO SE GENERO COMO DOCUMENTO INDEPENDIENTE ----
%\chapter{Sistemas de Visita}
%\section{Sistemas de Visitas}
%_________________________________________________________________________
%\subsection{Historia}
%_________________________________________________________________________
Los {\emph{Sistemas de Visitas}} fueron introducidos a principios de los a\~nos 50, ver \cite{Boxma,BoonMeiWinands,LevySidi,TesisRoubos,Takagi,Semenova}, con un problema relacionado con las personas encargadas de la revisi\'on y reparaci\'on de m\'aquinas; m\'as adelante fueron utilizados para estudiar problemas de control de se\~nales de tr\'afico. A partir de ese momento el campo de aplicaci\'on ha crecido considerablemente, por ejemplo en: comunicaci\'on en redes de computadoras, rob\'otica, tr\'afico y transporte, manufactura, producci\'on, distribuci\'on de correo, sistema de saludp\'ublica, etc.
%_________________________________________________________________________
%\section{Descripci\'on}
%_________________________________________________________________________

Un modelo de colas es un modelo matem\'atico que describe la situaci\'on en la que uno o varios usuarios solicitan de un servicio a una instancia, computadora o persona. Aquellos usuarios que no son atendidos inmediatamente toman un lugar en una cola en espera de servicio. Un sistema de visitas consiste en modelos de colas conformadas por varias colas y un solo servidor que las visita en alg\'un orden para atender a los usuarios que se encuentran esperando por servicio.

%_________________________________________________________________________
%\section{Objetivos}
%_________________________________________________________________________

Uno de los principales objetivos de este tipo de sistemas es tratar de mejorar el desempe\~no del sistema de visitas. Una de medida de desempe\~no importante es el tiempo de respuesta del sistema, as\'i como los tiempos promedios de espera en una fila y el tiempo promedio total que tarda en ser realizada una operaci\'on completa a lo largo de todo el sistema.\\

Algunas medidas de desempe\~no para los usuarios son los valores promedio de espera para ser atendidos, de servicio, de permanencia total en el sistema; mientras que para el servidor son los valores promedio de permanencia en una cola atendiendo, de traslado entre las colas, de duraci\'on del ciclo entre dos visitas consecutivas a la misma cola, entre otras medidas de desempe\~no estudiadas en la literatura.

%_________________________________________________________________________
%\section{Caracter\'isticas}
%_________________________________________________________________________

En la mayor\'ia de los modelos de colas c\'iclicas, la capacidad de almacenamiento es infinita, es decir la cola puede acomodar a una cantidad infinita de usuarios a la vez.

%_________________________________________________________________________
%\subsection{Clasificaci\'on}
%_________________________________________________________________________
Los sistemas de visitas pueden dividirse en dos clases:
\begin{itemize}
\item[i)] hay varios servidores y los usuarios que llegan al sistema eligen un servidor de entre los que est\'an presentes.

\item[ii)] hay uno o varios servidores que son comunes a todas las colas, estos visitan a cada una de las colas y atienden a los usuarios que est\'an presentes al momento de la visita del
servidor.
\end{itemize}
%_________________________________________________________________________
%\subsection{Tiempos de arribo a las colas}
%_________________________________________________________________________

La manera en que los usuarios llegan a las colas. Los usuarios llegan a las colas de manera tal que los tiempos entre arribos son independientes e id\'enticamente distribuidos. En la mayor\'ia de los modelos de visitas c\'iclicas, la capacidad de almacenamiento es infinita, es decir la cola puede acomodar a una cantidad infinita de usuarios a la vez.

%________________________________________________________
%\subsection{Tiempos de servicio}
%________________________________________________________
Los tiempos de servicio en una cola son usualmente considerados como muestra de una distribuci\'on de probabilidad que caracteriza a la cola, adem\'as se acostumbra considerarlos mutuamente independientes e independientes del estado actual del sistema. 

%________________________________________________________
%\subsection{Traslados del Servidor}
%________________________________________________________

La ruta de atenci\'on del servidor, es el orden en el cual el servidor visita las colas determinado por un mecanismo que puede depender del estado actual del sistema (din\'amico) o puede ser independiente del estado del sistema (est\'atico). 

El mecanismo m\'as utilizado es el c\'iclico. Para modelar sistemas en los cuales ciertas colas son visitadas con mayor frecuencia que otras, las colas c\'iclicas se han extendido a colas peri\'odicas, en las cuales el servidor visita la cola conforme a una orden de servicio de longitud finita. 

El {\em orden de visita} se entiende como la regla utilizada por el servidor para elegir la pr\'oxima cola. Este servicio puede ser din\'amico o est\'atico:

\begin{itemize}
\item[i)] Para el caso {\em est\'atico} la regla permanece invariante a lo largo del curso de la operaci\'on del sistema.

\item[ii)] Para el caso {\em din\'amico} la cola que se elige para servicio en el momento depende de un conocimiento total o parcial del estado del sistema.
\end{itemize}

Dentro de los ordenes de tipo est\'atico hay varios, los m\'as comunes son:

\begin{itemize}
\item[i)] {\em c\'iclico}: Si denotamos por $\left\{Q_{i}\right\}_{i=1}^{N}$ al conjunto de colas a las cuales el servidor visita en el orden \[Q_{1},Q_{2},\ldots,Q_{N},Q_{1},Q_{2},\ldots,Q_{N}.\]

\item[ii)] {\em peri\'odico}: el servidor visita las colas en el orden:
\[Q_{T\left(1\right)},Q_{T\left(2\right)},\ldots,Q_{T\left(M\right)},Q_{T\left(1\right)},\ldots,Q_{T\left(M\right)}\]
caracterizada por una tabla de visitas
\[\left(T\left(1\right),T\left(2\right),\ldots,T\left(M\right)\right),\]
con $M\geq N$, $T\left(i\right)\in\left\{1,2,\ldots,N\right\}$ e $i=\overline{1,M}$. Hay un caso especial, {\em colas tipo elevador} donde las colas son atendidas en el orden \[Q_{1},Q_{2},\ldots,Q_{N},Q_{1},Q_{2},\ldots,Q_{N-1},Q_{N},Q_{N-1},\ldots,Q_{1}\].

\item[iii)] {\em aleatorio}: la cola $Q_{i}$ es elegida para serbatendida con probabilidad $p_{i}$, $i=\overline{1,N}$, $\sum_{i=1}^{N}p_{i}=1$. Una posible variaci\'on es que despu\'es de atender $Q_{i}$ el servidor se desplaza a $Q_{j}$ con probabilidad $p_{ij}$, con $i,j=\overline{1,N}$, $\sum_{j=1}^{N}p_{ij}=1$, para $i=\overline{1,N}$.
\end{itemize}

El servidor usualmente incurrir\'a en tiempos de traslado para ir de una cola a otra. Un sistema de visitas puede expresarse en un par de par\'ametros: el n\'umero de colas, que usualmente se denotar\'a por $N$, y el tr\'afico caracter\'istico de las colas, que consiste de los procesos de arribo y los procesos de servicio, la figura (\ref{Sistema.de.Visitas}) caracteriza a estos sistemas.\\

%________________________________________________________
%\subsection{Disciplina de servicio}
%________________________________________________________

La disciplina de servicio especifica el n\'umero de usuarios que son atendidos durante la visita del servidor a la cola; estas pueden ser clasificadas en l\'imite de usuarios atendidos y en usuarios atendidos en un tiempo l\'imite, poniendo restricciones en la cantidad de tiempo utilizado por el servidor en una visita a la cola. Alternativamente pueden ser clasificadas en pol\'iticas exhaustivas y pol\'iticas cerradas, dependiendo en si los usuarios que llegaron a la cola mientras el servidor estaba dando servicio son candidatos para ser atendidos por el servidor que se encuentra en la cola dando servicio. En la pol\'itica exhaustiva estos usuarios son candidatos para ser atendidos mientras que en la cerrada no lo son. De estas dos pol\'iticas se han creado h\'ibridos los cuales pueden revisarse en \cite{BoonMeiWinands}.

La disciplina de la cola especifica el orden en el cual los usuarios presentes en la cola son atendidos. La m\'as com\'un es la {\em First-In-First-Served}.

%________________________________________________________
%\subsection{Pol\'itica de Servicio}
%________________________________________________________

Las pol\'iticas m\'as comunes son las de tipo exhaustivo que consiste en que el servidor continuar\'a trabajando hasta que la cola quede vac\'ia; y la pol\'itica cerrada, bajo la cual ser\'an atendidos exactamente aquellos que estaban presentes al momento en que lleg\'o el servidor a la cola. 

Las pol\'iticas de servicio deben de satisfacer las siguientes propiedades:
\begin{itemize}
\item[i)] No dependen de los procesos de servicio anteriores.
\item[ii)] La selecci\'on de los usuarios para ser atendidos es independiente del tiempo de servicio requerido  y de los posibles arribos futuros.
\item[iii)] las pol{\'\i}ticas de servicio que son aplicadas, es decir, el n\'umero de usuarios en la cola que ser{\'a}n atendidos durante la visita del servidor a la misma; \'estas pueden ser clasificadas por la cantidad de usuarios atendidos y por el n\'umero de usuarios atendidos en un intervalo de tiempo determinado. Las principales pol\'iticas de servicio para las cuales se han desarrollado aplicaciones son: la exhaustiva, la cerrada y la $k$-l\'imite, ver \cite{LevySidi, Takagi, Semenova}. De estas pol\'iticas se han creado h\'ibridos los cuales pueden revisarse en Boon and Van der Mei \cite{BoonMeiWinands}.

\item[iv)] Una pol{\'\i}tica de servicio es asignada a cada etapa independiente de la cola que se est{\'a} atendiendo, no necesariamente es la misma para todas las etapas.
\item[v)] El servidor da servicio de manera constante.

\item[vi)] La pol\'itica de servicio se asume mon\'otona (ver
\cite{Stability}).

\end{itemize}

Las principales pol\'iticas deterministas de servicio son:
\begin{itemize}

\item[i)] {\em Cerrada} donde solamente los usuarios presentes al comienzo de la etapa son considerados para ser atendidos.

\item[ii)] {\em Exhaustiva} en la que tanto los usuarios presentes al comienzo de la etapa como los que arriban   mientras se est\'a dando servicio son considerados para ser atendidos.

\item[iii)] $k_{i}$-limited: el n\'umero de usuarios por atender en la cola $i$ est\' acotado por $k_{i}$.

\item[iv)] {\em tiempo limitado} la cola es atendida solo por un periodo de tiempo fijo.
\end{itemize}

%________________________________________________________
%\subsection*{Extras}
%________________________________________________________
\begin{itemize}
\item Una etapa es el periodo de tiempo durante el cual el
servidor atiende de manera continua en una sola cola.

\item Un ciclo  es el periodo necesario para terminar $l$ etapas.

\end{itemize}


Boxma y Groenendijk \cite{Boxma2} enuncian la Ley de
Pseudo-Conservaci\'on para la pol\'itica exhaustiva como

\begin{equation}\label{LPCPE}
\sum_{i=1}^{N}\rho_{i}\esp
W_{i}=\rho\frac{\sum_{i=1}^{N}\lambda_{i}\esp\left[\delta_{i}^{(2)}\left(1\right)\right]}{2\left(1-\rho\right)}+\rho\frac{\delta^{(2)}}{2\delta}+\frac{\delta}{2\left(1-\rho\right)}\left[\rho^{2}-\sum_{i=1}^{N}\rho_{i}^{2}\right],
\end{equation}

donde $\delta=\sum_{i=1}^{N}\delta_{i}\left(1\right)$ y
$\delta_{i}^{(2)}$ denota el segundo momento de los tiempos de traslado entre colas del servidor, $\delta^{(2)}$ es el segundo momento de los tiempos de traslado entre las colas de todo el sistema, finalmente sea $\rho=\sum_{i=1}^{N}\rho_{i}$. Por otro lado, se tiene que

\begin{equation}\label{Eq.Tiempo.Espera}
\esp W_{i}=\frac{\esp I_{i}^{2}}{2\esp
I_{i}}+\frac{\lambda_{i}\esp\left[\eta_{i}^{(2)}\left(1\right)\right]}{2\left(1-\rho_{i}\right)},
\end{equation}

con $I_{i}$ definido como el peri\'odo de intervisita, es decir el tiempo entre una salida y el pr\'oximo arribo del servidor a la cola $Q_{i}$, dado por $I_{i}=C_{i}-V_{i}$, donde $C_{i}$ es la longitud del ciclo, definido como el tiempo entre dos instantes de
visita consecutivos a la cola $Q_{i}$ y $V_{i}$ es el periodo de visita, definido como el tiempo que el servidor utiliza en atender a los usuarios de la cola $Q_{i}$.
\begin{equation}\label{Eq.Periodo.Intervisita}
\esp
I_{i}=\frac{\left(1-\rho_{i}\right)}{1-\rho}\sum_{i=1}^{N}\esp\left[\delta_{i}\left(1\right)\right],
\end{equation}

con

\begin{equation}\label{Eq.Periodo.Intervisita}
\esp
I_{i}^{2}=\esp\left[\delta_{i-1}^{(2)}\left(1\right)\right]-\left(\esp\left[\delta_{i-1}\left(1\right)\right]\right)^{2}+
\frac{1-\rho_{i}}{\rho_{i}}\sum_{j=1,j\neq i}^{N}r_{ij}+\left(\esp
I_{i}\right)^{2},
\end{equation}

donde el conjunto de valores $\left\{r_{ij}:i,j=1,2,\ldots,N\right\}$ representan la covarianza del tiempo para las colas $i$ y $j$; para sistemas con servicio
exhaustivo, el tiempo de estaci\'on para la cola $i$ se define como el intervalo de tiempo entre instantes sucesivos cuando el servidor abandona la cola $i-1$ y la cola $i$. Hideaki Takagi \cite{Takagi} proporciona expresiones cerradas para calcular $r_{ij}$, \'estas implican resolver un sistema de $N^{2}$
Ecuaciones lineales;

%{\footnotesize{
\begin{eqnarray}\label{Eq.Cov.TT}
r_{ij}&=&\frac{\rho_{i}}{1-\rho_{i}}\left(\sum_{m=i+1}^{N}r_{jm}+\sum_{m=1}^{j-1}r_{jm}+\sum_{m=j}^{i-1}r_{jm}\right),\textrm{
}j<i,\\
r_{ij}&=&\frac{\rho_{i}}{1-\rho_{i}}\left(\sum_{m=i+1}^{j-1}r_{jm}+\sum_{m=j}^{N}r_{jm}+\sum_{m=1}^{i-1}r_{jm}\right),\textrm{
}j>i,\\
r_{ij}&=&\frac{\esp\left[\delta_{i-1}^{(2)}\left(1\right)\right]-\left(\esp\left[\delta_{i-1}\left(1\right)\right]\right)^{2}}
{\left(1-\rho_{i}\right)^{2}}+\frac{\lambda_{i}\esp\left[\eta_{i}\left(1\right)^{(2)}\right]}{\left(1-\rho_{i}\right)^{3}}+\frac{\rho_{i}}{1-\rho_{i}}\sum_{j=i,j=1}^{N}r_{ij}.
\end{eqnarray}%}}

Para el caso de la Pol\'itica Cerrada la Ley de Pseudo-Conservaci\'on se expresa en los siguientes t\'erminos.
\begin{equation}\label{LPCPG}
\sum_{i=1}^{N}\rho_{i}\esp
W_{i}=\rho\frac{\sum_{i=1}^{N}\lambda_{i}\esp\left[\delta_{i}\left(1\right)^{(2)}\right]}{2\left(1-\rho\right)}+\rho\frac{\delta^{(2)}}{2\delta}+\frac{\delta}{2\left(1-\rho\right)}\left[\rho^{2}+\sum_{i=1}^{N}\rho_{i}^{2}\right],
\end{equation}
el tiempo de espera promedio para los usuarios en la cola $Q_{1}$ se puede determinar por medio de
\begin{equation}\label{Eq.Tiempo.Espera.Gated}
\esp W_{i}=\frac{\left(1+\rho_{i}\right)\esp C_{i}^{2}}{2\esp
C_{i}},
\end{equation}
donde $C_{i}$ denota la longitud del ciclo para la cola $Q_{i}$, definida como el tiempo entre dos instantes consecutivos de visita en $Q_{i}$, cuyo segundo momento est\'a dado por

\begin{equation}\label{Eq.Periodo.Intervisita.Gated}
\esp C_{i}^{2}=\frac{1}{\rho_{i}}\sum_{j=1,j\neq
i}^{N}r_{ij}+\sum_{j=1}^{N}r_{ij}+\left(\esp C\right)^{2},
\end{equation}
con
\begin{eqnarray*}
\esp C=\frac{\delta}{1-\rho},
\end{eqnarray*}

donde $r_{ij}$ representa la covarianza del tiempo de estaci\'on para las colas $i$ y $j$, pero el tiempo de estaci\'on para la cola $i$ para la pol\'itica cerrada se define como el intervalo de tiempo entre instantes sucesivos cuando el servidor visita la cola $i$ y la cola $i+1$. El conjunto $\left\{r_{ij}:i,j=1,2,\ldots,N\right\}$ se calcula resolviendo un
sistema de $N^{2}$ ecuaciones lineales

\begin{eqnarray}\label{Eq.Cov.TT.Gated}
r_{ij}&=&\rho_{i}\left(\sum_{m=i}^{N}r_{jm}+\sum_{m=1}^{j-1}r_{jm}+\sum_{m=j}^{i-1}r_{mj}\right),\textrm{
}j<i,\\
r_{ij}&=&\rho_{i}\left(\sum_{m=i}^{j-1}r_{jm}+\sum_{m=j}^{N}r_{jm}+\sum_{m=1}^{i-1}r_{mj}\right),\textrm{
}j>i,\\
r_{ij}&=&r_{i-1}^{(2)}-\left(r_{i-1}^{(1)}\right)^{2}+\lambda_{i}b_{i}^{(2)}\esp
C_{i}+\rho_{i}\sum_{j=1,j\neq
i}^{N}r_{ij}+\rho_{i}^{2}\sum_{i=j,j=1}^{N}r_{ij}.
\end{eqnarray}

Finalmente, Takagi \cite{Takagi} proponen una aproximaci\'on para los tiempos de espera de los usuarios en cada una de las colas:
\begin{eqnarray*}
\sum_{i=1}^{N}\frac{\rho_{i}}{\rho}\left(1-\frac{\lambda_{i}\delta}{1-\rho}\right)\esp\left[W_{i}\right]&=&\sum_{i=1}^{N}\frac{\lambda_{i}\esp\left[\eta_{i}\left(1\right)^{(2)}\right]}{2\left(1-\rho\right)}\\
+\frac{\sum_{i=1}^{N}\esp\left[\delta_{i}^{2}\right]-\left(\esp\left[\delta_{i}\left(1\right)\right]\right)^{2}}{2\delta}&+&\frac{\delta\left(\rho-\sum_{i=1}^{N}\rho_{i}^{2}\right)}{2\rho\left(1-\rho\right)}+\frac{\delta\sum_{i=1}^{N}\rho_{i}^{2}}{\rho\left(1-\rho\right)},
\end{eqnarray*}
entonces
\begin{eqnarray*}\label{LPCPKL}
\esp
W_{i}&\cong&\frac{1-\rho+\rho_{i}}{1-\rho-\lambda_{i}\delta}\times\frac{1-\rho}{\rho\left(1-\rho\right)+\sum_{i=1}^{N}\rho_{i}^{2}}\\
&\times&\left[\frac{\rho}{2\left(1-\rho\right)}\sum_{i=1}^{N}\lambda_{i}\esp\left[\eta_{i}\left(1\right)^{(2)}\right]+\frac{\rho\Delta^{2}}{2\delta}+\frac{\delta}{2\left(1-\rho\right)}\sum_{i=1}^{N}\rho_{i}\left(1+\rho_{i}\right).\right]
\end{eqnarray*}
donde $\Delta^{2}=\sum_{i=1}^{N}\delta_{i}^{2}$. 

El modelo est\'a compuesto por $c$ colas de capacidad infinita, etiquetadas de $1$ a $c$ las cuales son atendidas por $s$
servidores. Los servidores atienden de acuerdo a una cadena de Markov independiente $\left(X^{i}_{n}\right)_{n}$ con $1\leq i\leq s$ y $n\in\left\{1,2,\ldots,c\right\}$ con la misma matriz de transici\'on $r_{k,l}$ y \'unica medida invariante $\left(p_{k}\right)$. Cada servidor permanece atendiendo en la cola un periodo llamado de visita y determinada por la pol\'itica de
servicio asignada a la cola.

Los usuarios llegan a la cola $k$ con una tasa $\lambda_{k}$ y son atendidos a una raz\'on $\mu_{k}$. Las sucesiones de tiempos de interarribo $\left(\tau_{k}\left(n\right)\right)_{n}$, la de
tiempos de servicio $\left(\sigma_{k}^{i}\left(n\right)\right)_{n}$ y la de tiempos de cambio $\left(\sigma_{k,l}^{0,i}\left(n\right)\right)_{n}$ requeridas en la cola $k$ para el servidor $i$ son sucesiones independientes e id\'enticamente distribuidas con distribuci\'on general independiente de $i$, con media $\sigma_{k}=\frac{1}{\mu_{k}}$, respectivamente $\sigma_{k,l}^{0}=\frac{1}{\mu_{k,l}^{0}}$, e independiente de las cadenas de Markov $\left(X^{i}_{n}\right)_{n}$. Adem\'as se supone que los tiempos de interarribo se asume son acotados, para cada $\rho_{k}=\lambda_{k}\sigma_{k}<s$ para asegurar la estabilidad de la cola $k$ cuando opera como una cola $M/GM/1$.

Una pol\'itica de servicio determina el n\'umero de usuarios que ser\'an atendidos sin interrupci\'on en periodo de servicio por los servidores que atienden a la cola. Para un solo servidor esta se define a trav\'es de una funci\'on $f$ donde $f\left(x,a\right)$ es el n\'umero de usuarios que son atendidos sin interrupci\'on cuando el servidor llega a la cola y encuentra $x$ usuarios esperando dado el tiempo transcurrido de interarribo $a$. Sea $v\left(x,a\right)$ la du raci\'on del periodo de servicio para una sola condici\'on inicial $\left(x,a\right)$.

Las pol\'iticas de servicio consideradas satisfacen las siguientes
propiedades:

\begin{itemize}
\item[i)] Hay conservaci\'on del trabajo, es decir
\[v\left(x,a\right)=\sum_{l=1}^{f\left(x,a\right)}\sigma\left(l\right)\]
con $f\left(0,a\right)=v\left(0,a\right)=0$, donde
$\left(\sigma\left(l\right)\right)_{l}$ es una sucesi\'on independiente e id\'enticamente distribuida de los tiempos de servicio solicitados. 
\item[ii)] La selecci\'on de usuarios para se atendidos es independiente de sus correspondientes tiempos de servicio y del pasado hasta el inicio del periodo de servicio. As\'i las distribuci\'on $\left(f,v\right)$ no depende del orden en el cu\'al son atendidos los usuarios. \item[iii)] La pol\'itica de servicio es mon\'otona en el sentido de que para cada $a\geq0$ los n\'umeros $f\left(x,a\right)$ son mon\'otonos en distribuci\'on en $x$ y su l\'imite en distribuci\'on cuando $x\rightarrow\infty$ es una variable aleatoria $F^{*0}$ que no depende de $a$. \item[iv)] El n\'umero de usuarios atendidos por cada servidor es acotado por
$f^{min}\left(x\right)$ de la longitud de la cola $x$ que adem\'as converge mon\'otonamente en distribuci\'on a $F^{*}$ cuando $x\rightarrow\infty$
\end{itemize}



Un sistema de visitas o sistema de colas consiste en un cierto n\'umero de filas o colas atendidas por un solo servidor en un orden determinado, estos se puede aplicar en situaciones en las cuales varios tipos de usuarios intentan tener acceso a una fuente en com\'un que est\'a disponible para un solo tipo de usuario a la vez. 

Un sistema de visitas consiste en varias colas a las cuales los usuarios llegan conforme a un proceso poisson con tasa $\lambda_{i}$; la capacidad de las mismas, es decir, el n\'umero de lugares disponibles; el n\'umero de servidores que llegan a la cola correspondiente para dar servicio a los usuarios; la manera en que los servidores dan servicio; el tiempo que tarda el servidor en ir de una a otra cola, as\'i como el orden y la disciplina de servicio de la cola; la pol\'itica de servicio determina cuales y cuantos usuarios ser\'an atendidos durante la visita del servidor a la cola.\\

Las pol\'iticas m\'as comunes son las de tipo exhaustivo que consiste en que el servidor continuar\'a trabajando hasta que la cola quede vac\'ia; y la pol\'itica cerrada, bajo la cual ser\'an atendidos exactamente aquellos que estaban presentes al momento en que lleg\'o el servidor a la cola. El esquema de ruta determina en que orden el servidor visitara a las colas. La decisi\'on del servidor sobre la pr\'oxima cola que visitar\'a puede depender de la informaci\'on disponible para el servidor, por ejemplo las longitudes de las
colas.\\

El servidor usualmente incurrir\'a en tiempos de traslado para ir de una cola a otra. Un sistema de visitas puede expresarse en un par de par\'ametros: el n\'umero de colas, que usualmente se denotar\'a por $N$, y el tr\'afico caracter\'istico de las colas, que consiste de los procesos de arribo y los procesos de servicio, la figura (\ref{GRafSistColasCiclicas}) caracteriza a estos sistemas.\\

Algunas medidas de desempe\~no para los usuarios son los valores promedio de espera para ser atendidos, de servicio, de permanencia total en el sistema; mientras que para el servidor son los valores promedio de permanencia en una cola atendiendo, de traslado entre las colas, de duraci\'on del ciclo entre dos visitas consecutivas a la misma cola, entre otras medidas de desempe\~no estudiaddas en la literatura.


%___________________________________________________________________________________________
%
 \section{Funci\'on Generadora de Probabilidades}
%___________________________________________________________________________________________

\begin{Teo}[Teorema de Continuidad]
Sup\'ongase que $\left\{X_{n},n=1,2,3,\ldots\right\}$ son variables aleatorias finitas, no negativas con valores enteros tales que $P\left(X_{n}=k\right)=p_{k}^{(n)}$, para $n=1,2,3,\ldots$, $k=0,1,2,\ldots$, con $\sum_{k=0}^{\infty}p_{k}^{(n)}=1$, para $n=1,2,3,\ldots$. Sea $g_{n}$ la PGF para la variable aleatoria $X_{n}$. Entonces existe una sucesi\'on $\left\{p_{k}\right\}$ tal que \begin{eqnarray*}
lim_{n\rightarrow\infty}p_{k}^{(n)}=p_{k}\textrm{ para }0<s<1.
\end{eqnarray*}

En este caso, $g\left(s\right)=\sum_{k=0}^{\infty}s^{k}p_{k}$. Adem\'as
\begin{eqnarray*}
\sum_{k=0}^{\infty}p_{k}=1\textrm{ si y s\'olo si
}lim_{s\uparrow1}g\left(s\right)=1
\end{eqnarray*}
\end{Teo}

\begin{Teo}
Sea $N$ una variable aleatoria con valores enteros no negativos finita tal que $P\left(N=k\right)=p_{k}$, para $k=0,1,2,\ldots$, y $\sum_{k=0}^{\infty}p_{k}=P\left(N<\infty\right)=1$. Sea $\Phi$ la PGF de $N$ tal que
$g\left(s\right)=\esp\left[s^{N}\right]=\sum_{k=0}^{\infty}s^{k}p_{k}$ con $g\left(1\right)=1$. Si $0\leq p_{1}\leq1$ y $\esp\left[N\right]=g^{'}\left(1\right)\leq1$, entonces no existe soluci\'on  de la ecuaci\'on $g\left(s\right)=s$ en el intervalo $\left[0,1\right)$. Si $\esp\left[N\right]=g^{'}\left(1\right)>1$, lo cual implica que $0\leq p_{1}<1$, entonces existe una \'unica soluci\'on de la ecuaci\'on $g\left(s\right)=s$ en el intervalo $\left[0,1\right)$.
\end{Teo}


\begin{Teo}
Si $X$ y $Y$ tienen PGF $G_{X}$ y $G_{Y}$ respectivamente, entonces,\[G_{X}\left(s\right)=G_{Y}\left(s\right)\] para tod $s$, si y s\'olo si \[P\left(X=k\right))=P\left(Y=k\right)\] para toda $k=0,1,\ldots,$., es decir, si y s\'olo si $X$ y $Y$ tienen la misma distribuci\'on de probabilidad.
\end{Teo}


\begin{Teo}
Para cada $n$ fijo, sea la sucesi\'oin de probabilidades $\left\{a_{0,n},a_{1,n},\ldots,\right\}$, tales que $a_{k,n}\geq0$ para toda $k=0,1,2,\ldots,$ y $\sum_{k\geq0}a_{k,n}=1$, y sea $G_{n}\left(s\right)$ la correspondiente funci\'on generadora, $G_{n}\left(s\right)=\sum_{k\geq0}a_{k,n}s^{k}$. De modo que para cada valor fijo de $k$
\begin{eqnarray*}
lim_{n\rightarrow\infty}a_{k,n}=a_{k},
\end{eqnarray*}
es decir converge en distribuci\'on, es necesario y suficiente que para cada valor fijo $s\in\left[0,\right)$,

\begin{eqnarray*}
lim_{n\rightarrow\infty}G_{n}\left(s\right)=G\left(s\right),
\end{eqnarray*}
donde $G\left(s\right)=\sum_{k\geq0}p_{k}s^{k}$, para cualquier la funci\'on generadora del l\'imite de la sucesi\'on.
\end{Teo}

\begin{Teo}[Teorema de Abel]
Sea $G\left(s\right)=\sum_{k\geq0}a_{k}s^{k}$ para cualquier $\left\{p_{0},p_{1},\ldots,\right\}$, tales que $p_{k}\geq0$ para toda $k=0,1,2,\ldots,$. Entonces $G\left(s\right)$ es continua por la derecha en $s=1$, es decir
\begin{eqnarray*}
lim_{s\uparrow1}G\left(s\right)=\sum_{k\geq0}p_{k}=G\left(\right),
\end{eqnarray*}
sin importar si la suma es finita o no.
\end{Teo}

\begin{Note}
El radio de Convergencia para cualquier PGF es $R\geq1$, entonces, el Teorema de Abel nos dice que a\'un en el peor escenario, cuando $R=1$, a\'un se puede confiar en que la PGF ser\'a continua en $s=1$, en contraste, no se puede asegurar que la PGF ser\'a continua en el l\'imite inferior $-R$, puesto que la PGF es sim\'etrica alrededor del cero: la PGF converge para todo $s\in\left(-R,R\right)$, y no lo hace para $s<-R$ o $s>R$. Adem\'as nos dice que podemos escribir $G_{X}\left(1\right)$ como una abreviaci\'on de $lim_{s\uparrow1}G_{X}\left(s\right)$.
\end{Note}

Entonces si suponemos que la diferenciaci\'on t\'ermino a t\'ermino est\'a permitida, entonces

\begin{eqnarray*}
G_{X}^{'}\left(s\right)&=&\sum_{x=1}^{\infty}xs^{x-1}p_{x}
\end{eqnarray*}

el Teorema de Abel nos dice que
\begin{eqnarray*}
\esp\left(X\right]&=&\lim_{s\uparrow1}G_{X}^{'}\left(s\right):\\
\esp\left[X\right]&=&=\sum_{x=1}^{\infty}xp_{x}=G_{X}^{'}\left(1\right)\\
&=&\lim_{s\uparrow1}G_{X}^{'}\left(s\right),
\end{eqnarray*}
dado que el Teorema de Abel se aplica a
\begin{eqnarray*}
G_{X}^{'}\left(s\right)&=&\sum_{x=1}^{\infty}xs^{x-1}p_{x},
\end{eqnarray*}
estableciendo as\'i que $G_{X}^{'}\left(s\right)$ es continua en $s=1$. Sin el Teorema de Abel no se podr\'ia asegurar que el l\'imite de $G_{X}^{'}\left(s\right)$ conforme $s\uparrow1$ sea la respuesta correcta para $\esp\left[X\right]$.

\begin{Note}
La PGF converge para todo $|s|<R$, para alg\'un $R$. De hecho la PGF converge absolutamente si $|s|<R$. La PGF adem\'as converge uniformemente en conjuntos de la forma $\left\{s:|s|<R^{'}\right\}$, donde $R^{'}<R$, es decir, $\forall\epsilon>0, \exists n_{0}\in\ent$ tal que $\forall s$, con $|s|<R^{'}$, y $\forall n\geq n_{0}$,
\begin{eqnarray*}
|\sum_{x=0}^{n}s^{x}\prob\left(X=x\right)-G_{X}\left(s\right)|<\epsilon.
\end{eqnarray*}
De hecho, la convergencia uniforme es la que nos permite diferenciar t\'ermino a t\'ermino:
\begin{eqnarray*}
G_{X}\left(s\right)=\esp\left[s^{X}\right]=\sum_{x=0}^{\infty}s^{x}\prob\left(X=x\right),
\end{eqnarray*}
y sea $s<R$.
\begin{enumerate}
\item
\begin{eqnarray*}
G_{X}^{'}\left(s\right)&=&\frac{d}{ds}\left(\sum_{x=0}^{\infty}s^{x}\prob\left(X=x\right)\right)=\sum_{x=0}^{\infty}\frac{d}{ds}\left(s^{x}\prob\left(X=x\right)\right)\\
&=&\sum_{x=0}^{n}xs^{x-1}\prob\left(X=x\right).
\end{eqnarray*}

\item\begin{eqnarray*}
\int_{a}^{b}G_{X}\left(s\right)ds&=&\int_{a}^{b}\left(\sum_{x=0}^{\infty}s^{x}\prob\left(X=x\right)\right)ds=\sum_{x=0}^{\infty}\left(\int_{a}^{b}s^{x}\prob\left(X=x\right)ds\right)\\
&=&\sum_{x=0}^{\infty}\frac{s^{x+1}}{x+1}\prob\left(X=x\right),
\end{eqnarray*}
para $-R<a<b<R$.
\end{enumerate}
\end{Note}

\begin{Teo}[Teorema de Convergencia Mon\'otona para PGF] Sean $X$ y $X_{n}$ variables aleatorias no negativas, con valores en los enteros, finitas, tales que
\begin{eqnarray*}
lim_{n\rightarrow\infty}G_{X_{n}}\left(s\right)&=&G_{X}\left(s\right)
\end{eqnarray*}
para $0\leq s\leq1$, entonces
\begin{eqnarray*}
lim_{n\rightarrow\infty}P\left(X_{n}=k\right)=P\left(X=k\right),
\end{eqnarray*}
para $k=0,1,2,\ldots.$
\end{Teo}

El teorema anterior requiere del siguiente lema

\begin{Lemma}
Sean $a_{n,k}\in\ent^{+}$, $n\in\nat$ constantes no negativas con
$\sum_{k\geq0}a_{k,n}\leq1$. Sup\'ongase que para $0\leq s\leq1$,
se tiene
\begin{eqnarray*}
a_{n}\left(s\right)&=&\sum_{k=0}^{\infty}a_{k,n}s^{k}\rightarrow
a\left(s\right)=\sum_{k=0}^{\infty}a_{k}s^{k}.
\end{eqnarray*}
Entonces
\begin{eqnarray*}
a_{0,n}\rightarrow a_{0}.
\end{eqnarray*}
\end{Lemma}

Consideremos un sistema que consta de \'unicamente un servidor y una sola cola, a la cual los usuarios arriban conforme a un proceso poisson cuya tasa promedio de llegada es $1/\lambda$; la tasa promedio con la cual el servidor da servicio es $1/\mu$, adem\'as los tiempos entre arribos y los tiempos de servicio son independientes entre s\'i.

Se define la carga de tr\'afico $\rho:=\frac{\lambda}{\mu}$, para este modelo existe un teorema que nos dice la relaci\'on que hay entre el valor de $\rho$ y la estabilidad de la cola:

\begin{Prop}
La cola $M/M/1$ con carga de tr\'afico $\rho$, es estable si y s\'olo si $\rho<1$.
\end{Prop}

Este teorema nos permite determinar las principales medidas de desempe\~no: Tiempo de espera en el sistema, $W$, el n\'umero esperado de clientes en el sistema, $L$, adem\'as de los tiempos promedio e espera tanto en la cola como de servicio, $s$ representa el tiempo de servicio para un cliente:

\begin{eqnarray}
 L&=&\frac{\rho}{1-\rho},\\
W&=&\frac{1}{\mu-\lambda},\\
W_{q}&=&\esp\left[s\right]\frac{\rho}{1-\rho}\textrm{,  y }\\
L_{q}&=&\frac{\rho^{2}}{1-\rho}.
\end{eqnarray}

Esta es la idea general, poder determinar la principales medidas de desempe\~no para un sistema de colas o sistema de visitas, para este fin es necesario realizar los siguientes supuestos. En teor\'ia de colas hay casos particulares, para los cuales es posible determinar espec\'ificamente medidas de desempe\~no del sistema bajo condiciones de estabilidad, tales como los tiempos promedio de espera y de servicio, tanto en el sistema como en cada
una de las colas.


En teor\'ia de colas hay casos particulares, para los cuales es posible determinar espec\'ificamente medidas de desempe\~no del sistema bajo condiciones de estabilidad, tales como los tiempos promedio de espera y de servicio, tanto en el sistema como en cada
una de las colas. Se considerar\'an intervalos de tiempo de la forma $\left[t,t+1\right]$. Los usuarios arriban por paquetes de manera independiente del resto de las colas. Se define el grupo de usuarios que llegan a cada una de las colas del sistema 1, caracterizadas por $Q_{1}$ y $Q_{2}$ respectivamente, en el intervalo de tiempo $\left[t,t+1\right]$ por $X_{1}\left(t\right),X_{2}\left(t\right)$.

Para cada uno de los procesos anteriores se define su Funci\'on Generadora de Probabilidades (PGF):

\begin{eqnarray*}
\begin{array}{cc}
P_{1}\left(z_{1}\right)=\esp\left[z_{1}^{X_{1}\left(t\right)}\right], & P_{2}\left(z_{2}\right)=\esp\left[z_{2}^{X_{2}\left(t\right)}\right].\\
\end{array}
\end{eqnarray*}

Con primer momento definidos por

\begin{eqnarray*}
\mu_{1}&=&\esp\left[X_{1}\left(t\right)\right]=P_{1}^{(1)}\left(1\right),\\
\mu_{2}&=&\esp\left[X_{2}\left(t\right)\right]=P_{2}^{(1)}\left(1\right).
\end{eqnarray*}


En lo que respecta al servidor, en t\'erminos de los tiempos de visita a cada una de las colas, se denotar\'an por $\tau_{1},\tau_{2}$ para $Q_{1},Q_{2}$ respectivamente; y a los
tiempos en que el servidor termina de atender en las colas $Q_{1},Q_{2}$, se les denotar\'a por $\overline{\tau}_{1},\overline{\tau}_{2}$ respectivamente. Entonces, los tiempos de servicio est\'an dados por las diferencias $\overline{\tau}_{1}-\tau_{1},\overline{\tau}_{2}-\tau_{2}$ para $Q_{1},Q_{2}$. An\'alogamente los tiempos de traslado del servidor desde el momento en que termina de atender a una cola y llega a la siguiente para comenzar a dar servicio est\'an dados por $\tau_{2}-\overline{\tau}_{1},\tau_{1}-\overline{\tau}_{2}$.


La FGP para estos tiempos de traslado est\'an dados por

\begin{eqnarray*}
\begin{array}{cc}
R_{1}\left(z_{1}\right)=\esp\left[z_{1}^{\tau_{2}-\overline{\tau}_{1}}\right],
&
R_{2}\left(z_{2}\right)=\esp\left[z_{2}^{\tau_{1}-\overline{\tau}_{2}}\right],
\end{array}
\end{eqnarray*}

y al igual que como se hizo con anterioridad

\begin{eqnarray*}
\begin{array}{cc}
r_{1}=R_{1}^{(1)}\left(1\right)=\esp\left[\tau_{2}-\overline{\tau}_{1}\right],
&
r_{2}=R_{2}^{(1)}\left(1\right)=\esp\left[\tau_{1}-\overline{\tau}_{2}\right],\\
\end{array}
\end{eqnarray*}
Sean $\alpha_{1},\alpha_{2}$ el n\'umero de usuarios que arriban
en grupo a la cola $Q_{1}$ y $Q_{2}$ respectivamente. Sus PGF's
est\'an definidas como

\begin{eqnarray*}
\begin{array}{cc}
A_{1}\left(z\right)=\esp\left[z^{\alpha_{1}\left(t\right)}\right],&
A_{2}\left(z\right)=\esp\left[z^{\alpha_{2}\left(t\right)}\right].\\
\end{array}
\end{eqnarray*}

Su primer momento est\'a dado por

\begin{eqnarray*}
\begin{array}{cc}
\lambda_{1}=\esp\left[\alpha_{1}\left(t\right)\right]=A_{1}^{(1)}\left(1\right),&
\lambda_{2}=\esp\left[\alpha_{2}\left(t\right)\right]=A_{2}^{(1)}\left(1\right).\\
\end{array}
\end{eqnarray*}

Sean $\beta_{1},\beta_{2}$ el n\'umero de usuarios que arriban en el grupo $\alpha_{1},\alpha_{2}$ a la cola $Q_{1}$ y $Q_{2}$, respectivamente, de igual manera se definen sus PGF's

\begin{eqnarray*}
\begin{array}{cc}
B_{1}\left(z\right)=\esp\left[z^{\beta_{1}\left(t\right)}\right],&
B_{2}\left(z\right)=\esp\left[z^{\beta_{2}\left(t\right)}\right],\\
\end{array}
\end{eqnarray*}

con

\begin{eqnarray*}
\begin{array}{cc}
b_{1}=\esp\left[\beta_{1}\left(t\right)\right]=B_{1}^{(1)}\left(1\right),&
b_{2}=\esp\left[\beta_{2}\left(t\right)\right]=B_{2}^{(1)}\left(1\right).\\
\end{array}
\end{eqnarray*}

La distribuci\'on para el n\'umero de grupos que arriban al sistema en cada una de las colas se definen por:

\begin{eqnarray*}
\begin{array}{cc}
P_{1}\left(z_{1}\right)=A_{1}\left[B_{1}\left(z_{1}\right)\right]=\esp\left[B_{1}\left(z_{1}\right)^{\alpha_{1}\left(t\right)}\right],& P_{2}\left(z_{1}\right)=A_{1}\left[B_{1}\left(z_{1}\right)\right]=\esp\left[B_{1}\left(z_{1}\right)^{\alpha_{1}\left(t\right)}\right],\\
\end{array}
\end{eqnarray*}

entonces

\begin{eqnarray*}
P_{1}^{(1)}\left(1\right)&=&\esp\left[\alpha_{1}\left(t\right)B_{1}^{(1)}\left(1\right)\right]=B_{1}^{(1)}\left(1\right)\esp\left[\alpha_{1}\left(t\right)\right]=\lambda_{1}b_{1}\\
P_{2}^{(1)}\left(1\right)&=&\esp\left[\alpha_{2}\left(t\right)B_{2}^{(1)}\left(1\right)\right]=B_{2}^{(1)}\left(1\right)\esp\left[\alpha_{2}\left(t\right)\right]=\lambda_{2}b_{2}.\\
\end{eqnarray*}

De lo desarrollado hasta ahora se tiene lo siguiente

\begin{eqnarray*}
&&\esp\left[z_{1}^{L_{1}\left(\overline{\tau}_{1}\right)}z_{2}^{L_{2}\left(\overline{\tau}_{1}\right)}\right]=\esp\left[z_{2}^{L_{2}\left(\overline{\tau}_{1}\right)}\right]=\esp\left[z_{2}^{L_{2}\left(\tau_{1}\right)+X_{2}\left(\overline{\tau}_{1}-\tau_{1}\right)}\right]\\
&=&\esp\left[\left\{z_{2}^{L_{2}\left(\tau_{1}\right)}\right\}\left\{z_{2}^{X_{2}\left(\overline{\tau}_{1}-\tau_{1}\right)}\right\}\right]=\esp\left[\left\{z_{2}^{L_{2}\left(\tau_{1}\right)}\right\}\left\{P_{2}\left(z_{2}\right)\right\}^{\overline{\tau}_{1}-\tau_{1}}\right]\\
&=&\esp\left[\left\{z_{2}^{L_{2}\left(\tau_{1}\right)}\right\}\left\{\theta_{1}\left(P_{2}\left(z_{2}\right)\right)\right\}^{L_{1}\left(\tau_{1}\right)}\right]=F_{1}\left(\theta_{1}\left(P_{2}\left(z_{2}\right)\right),z_{2}\right)
\end{eqnarray*}

es decir 
\begin{equation}\label{Eq.base.F1}
\esp\left[z_{1}^{L_{1}\left(\overline{\tau}_{1}\right)}z_{2}^{L_{2}\left(\overline{\tau}_{1}\right)}\right]=F_{1}\left(\theta_{1}\left(P_{2}\left(z_{2}\right)\right),z_{2}\right).
\end{equation}

Procediendo de manera an\'aloga para $\overline{\tau}_{2}$:

\begin{eqnarray*}
\esp\left[z_{1}^{L_{1}\left(\overline{\tau}_{2}\right)}z_{2}^{L_{2}\left(\overline{\tau}_{2}\right)}\right]&=&\esp\left[z_{1}^{L_{1}\left(\overline{\tau}_{2}\right)}\right]=\esp\left[z_{1}^{L_{1}\left(\tau_{2}\right)+X_{1}\left(\overline{\tau}_{2}-\tau_{2}\right)}\right]=\esp\left[\left\{z_{1}^{L_{1}\left(\tau_{2}\right)}\right\}\left\{z_{1}^{X_{1}\left(\overline{\tau}_{2}-\tau_{2}\right)}\right\}\right]\\
&=&\esp\left[\left\{z_{1}^{L_{1}\left(\tau_{2}\right)}\right\}\left\{P_{1}\left(z_{1}\right)\right\}^{\overline{\tau}_{2}-\tau_{2}}\right]=\esp\left[\left\{z_{1}^{L_{1}\left(\tau_{2}\right)}\right\}\left\{\theta_{2}\left(P_{1}\left(z_{1}\right)\right)\right\}^{L_{2}\left(\tau_{2}\right)}\right]\\
&=&F_{2}\left(z_{1},\theta_{2}\left(P_{1}\left(z_{1}\right)\right)\right)
\end{eqnarray*}
por tanto
\begin{equation}\label{Eq.PGF.Conjunta.Tau2}
\esp\left[z_{1}^{L_{1}\left(\overline{\tau}_{2}\right)}z_{2}^{L_{2}\left(\overline{\tau}_{2}\right)}\right]=F_{2}\left(z_{1},\theta_{2}\left(P_{1}\left(z_{1}\right)\right)\right)
\end{equation}

Ahora, para el intervalo de tiempo
$\left[\overline{\tau}_{1},\tau_{2}\right]$ y $\left[\overline{\tau}_{2},\tau_{1}\right]$, los arribos de los usuarios modifican el n\'umero de usuarios que llegan a las colas, es decir, los procesos
$L_{1}\left(t\right)$ y $L_{2}\left(t\right)$. La PGF para el n\'umero de arribos a todas las estaciones durante el intervalo $\left[\overline{\tau}_{1},\tau_{2}\right]$  cuya distribuci\'on est\'a especificada por la distribuci\'on compuesta $R_{1}\left(\mathbf{z}\right),R_{2}\left(\mathbf{z}\right)$:

\begin{eqnarray*}
R_{1}\left(\mathbf{z}\right)=R_{1}\left(\prod_{i=1}^{2}P\left(z_{i}\right)\right)=\esp\left[\left\{\prod_{i=1}^{2}P\left(z_{i}\right)\right\}^{\tau_{2}-\overline{\tau}_{1}}\right]\\
R_{2}\left(\mathbf{z}\right)=R_{2}\left(\prod_{i=1}^{2}P\left(z_{i}\right)\right)=\esp\left[\left\{\prod_{i=1}^{2}P\left(z_{i}\right)\right\}^{\tau_{1}-\overline{\tau}_{2}}\right]\\
\end{eqnarray*}

Dado que los eventos en
$\left[\tau_{1},\overline{\tau}_{1}\right]$ y $\left[\overline{\tau}_{1},\tau_{2}\right]$ son independientes, la
PGF conjunta para el n\'umero de usuarios en el sistema al tiempo $t=\tau_{2}$ la PGF conjunta para el n\'umero de usuarios en el sistema est\'an dadas por

\begin{eqnarray*}
F_{1}\left(\mathbf{z}\right)&=&R_{2}\left(\prod_{i=1}^{2}P\left(z_{i}\right)\right)F_{2}\left(z_{1},\theta_{2}\left(P_{1}\left(z_{1}\right)\right)\right)\\
F_{2}\left(\mathbf{z}\right)&=&R_{1}\left(\prod_{i=1}^{2}P\left(z_{i}\right)\right)F_{1}\left(\theta_{1}\left(P_{2}\left(z_{2}\right)\right),z_{2}\right)\\
\end{eqnarray*}

Entonces debemos de determinar las siguientes expresiones:

\begin{eqnarray*}
\begin{array}{cc}
f_{1}\left(1\right)=\frac{\partial F_{1}\left(\mathbf{z}\right)}{\partial z_{1}}|_{\mathbf{z}=1}, & f_{1}\left(2\right)=\frac{\partial F_{1}\left(\mathbf{z}\right)}{\partial z_{2}}|_{\mathbf{z}=1},\\
f_{2}\left(1\right)=\frac{\partial F_{2}\left(\mathbf{z}\right)}{\partial z_{1}}|_{\mathbf{z}=1}, & f_{2}\left(2\right)=\frac{\partial F_{2}\left(\mathbf{z}\right)}{\partial z_{2}}|_{\mathbf{z}=1},\\
\end{array}
\end{eqnarray*}

calculando las derivadas parciales 
\begin{eqnarray*}
\frac{\partial R_{1}\left(\mathbf{z}\right)}{\partial
z_{1}}|_{\mathbf{z}=1}&=&R_{1}^{(1)}\left(1\right)P_{1}^{(1)}\left(1\right)\\
\frac{\partial R_{1}\left(\mathbf{z}\right)}{\partial
z_{2}}|_{\mathbf{z}=1}&=&R_{1}^{(1)}\left(1\right)P_{2}^{(1)}\left(1\right)\\
\frac{\partial R_{2}\left(\mathbf{z}\right)}{\partial
z_{1}}|_{\mathbf{z}=1}&=&R_{2}^{(1)}\left(1\right)P_{1}^{(1)}\left(1\right)\\
\frac{\partial R_{2}\left(\mathbf{z}\right)}{\partial
z_{2}}|_{\mathbf{z}=1}&=&R_{2}^{(1)}\left(1\right)P_{2}^{(1)}\left(1\right)\\
\end{eqnarray*}

igualando a cero

\begin{eqnarray*}
\frac{\partial}{\partial
z_{1}}F_{1}\left(\theta_{1}\left(P_{2}\left(z_{2}\right)\right),z_{2}\right)&=&0\\
\frac{\partial}{\partial
z_{2}}F_{1}\left(\theta_{1}\left(P_{2}\left(z_{2}\right)\right),z_{2}\right)&=&\frac{\partial
F_{1}}{\partial z_{2}}+\frac{\partial F_{1}}{\partial
z_{1}}\theta_{1}^{(1)}P_{2}^{(1)}\left(1\right)\\
\frac{\partial}{\partial
z_{1}}F_{2}\left(z_{1},\theta_{2}\left(P_{1}\left(z_{1}\right)\right)\right)&=&\frac{\partial
F_{2}}{\partial z_{1}}+\frac{\partial F_{2}}{\partial
z_{2}}\theta_{2}^{(1)}P_{1}^{(1)}\left(1\right)\\
\frac{\partial}{\partial
z_{2}}F_{2}\left(z_{1},\theta_{2}\left(P_{1}\left(z_{1}\right)\right)\right)&=&0.
\end{eqnarray*}


Por lo tanto de las dos secciones anteriores se tiene que:


\begin{eqnarray*}
\frac{\partial F_{1}}{\partial z_{1}}&=&\frac{\partial
R_{2}}{\partial z_{1}}|_{\mathbf{z}=1}+\frac{\partial F_{2}}{\partial z_{1}}|_{\mathbf{z}=1}=R_{2}^{(1)}\left(1\right)P_{1}^{(1)}\left(1\right)+f_{2}\left(1\right)+f_{2}\left(2\right)\theta_{2}^{(1)}\left(1\right)P_{1}^{(1)}\left(1\right)\\
\frac{\partial F_{1}}{\partial z_{2}}&=&\frac{\partial
R_{2}}{\partial z_{2}}|_{\mathbf{z}=1}+\frac{\partial F_{2}}{\partial z_{2}}|_{\mathbf{z}=1}=R_{2}^{(1)}\left(1\right)P_{2}^{(1)}\left(1\right)\\
\frac{\partial F_{2}}{\partial z_{1}}&=&\frac{\partial
R_{1}}{\partial z_{1}}|_{\mathbf{z}=1}+\frac{\partial F_{1}}{\partial z_{1}}|_{\mathbf{z}=1}=R_{1}^{(1)}\left(1\right)P_{1}^{(1)}\left(1\right)\\
\frac{\partial F_{2}}{\partial z_{2}}&=&\frac{\partial
R_{1}}{\partial z_{2}}|_{\mathbf{z}=1}+\frac{\partial F_{1}}{\partial z_{2}}|_{\mathbf{z}=1}
=R_{1}^{(1)}\left(1\right)P_{2}^{(1)}\left(1\right)+f_{1}\left(1\right)\theta_{1}^{(1)}\left(1\right)P_{2}^{(1)}\left(1\right)\\
\end{eqnarray*}


El cual se puede escribir en forma equivalente:
\begin{eqnarray*}
f_{1}\left(1\right)&=&r_{2}\mu_{1}+f_{2}\left(1\right)+f_{2}\left(2\right)\frac{\mu_{1}}{1-\mu_{2}}\\
f_{1}\left(2\right)&=&r_{2}\mu_{2}\\
f_{2}\left(1\right)&=&r_{1}\mu_{1}\\
f_{2}\left(2\right)&=&r_{1}\mu_{2}+f_{1}\left(2\right)+f_{1}\left(1\right)\frac{\mu_{2}}{1-\mu_{1}}\\
\end{eqnarray*}

De donde:
\begin{eqnarray*}
f_{1}\left(1\right)&=&\mu_{1}\left[r_{2}+\frac{f_{2}\left(2\right)}{1-\mu_{2}}\right]+f_{2}\left(1\right)\\
f_{2}\left(2\right)&=&\mu_{2}\left[r_{1}+\frac{f_{1}\left(1\right)}{1-\mu_{1}}\right]+f_{1}\left(2\right)\\
\end{eqnarray*}

Resolviendo para $f_{1}\left(1\right)$:
\begin{eqnarray*}
f_{1}\left(1\right)&=&r_{2}\mu_{1}+f_{2}\left(1\right)+f_{2}\left(2\right)\frac{\mu_{1}}{1-\mu_{2}}=r_{2}\mu_{1}+r_{1}\mu_{1}+f_{2}\left(2\right)\frac{\mu_{1}}{1-\mu_{2}}\\
&=&\mu_{1}\left(r_{2}+r_{1}\right)+f_{2}\left(2\right)\frac{\mu_{1}}{1-\mu_{2}}=\mu_{1}\left(r+\frac{f_{2}\left(2\right)}{1-\mu_{2}}\right),\\
\end{eqnarray*}

entonces

\begin{eqnarray*}
f_{2}\left(2\right)&=&\mu_{2}\left(r_{1}+\frac{f_{1}\left(1\right)}{1-\mu_{1}}\right)+f_{1}\left(2\right)=\mu_{2}\left(r_{1}+\frac{f_{1}\left(1\right)}{1-\mu_{1}}\right)+r_{2}\mu_{2}\\
&=&\mu_{2}\left[r_{1}+r_{2}+\frac{f_{1}\left(1\right)}{1-\mu_{1}}\right]=\mu_{2}\left[r+\frac{f_{1}\left(1\right)}{1-\mu_{1}}\right]\\
&=&\mu_{2}r+\mu_{1}\left(r+\frac{f_{2}\left(2\right)}{1-\mu_{2}}\right)\frac{\mu_{2}}{1-\mu_{1}}\\
&=&\mu_{2}r+\mu_{2}\frac{r\mu_{1}}{1-\mu_{1}}+f_{2}\left(2\right)\frac{\mu_{1}\mu_{2}}{\left(1-\mu_{1}\right)\left(1-\mu_{2}\right)}\\
&=&\mu_{2}\left(r+\frac{r\mu_{1}}{1-\mu_{1}}\right)+f_{2}\left(2\right)\frac{\mu_{1}\mu_{2}}{\left(1-\mu_{1}\right)\left(1-\mu_{2}\right)}\\
&=&\mu_{2}\left(\frac{r}{1-\mu_{1}}\right)+f_{2}\left(2\right)\frac{\mu_{1}\mu_{2}}{\left(1-\mu_{1}\right)\left(1-\mu_{2}\right)}\\
\end{eqnarray*}
entonces
\begin{eqnarray*}
f_{2}\left(2\right)-f_{2}\left(2\right)\frac{\mu_{1}\mu_{2}}{\left(1-\mu_{1}\right)\left(1-\mu_{2}\right)}&=&\mu_{2}\left(\frac{r}{1-\mu_{1}}\right)\\
f_{2}\left(2\right)\left(1-\frac{\mu_{1}\mu_{2}}{\left(1-\mu_{1}\right)\left(1-\mu_{2}\right)}\right)&=&\mu_{2}\left(\frac{r}{1-\mu_{1}}\right)\\
f_{2}\left(2\right)\left(\frac{1-\mu_{1}-\mu_{2}+\mu_{1}\mu_{2}-\mu_{1}\mu_{2}}{\left(1-\mu_{1}\right)\left(1-\mu_{2}\right)}\right)&=&\mu_{2}\left(\frac{r}{1-\mu_{1}}\right)\\
f_{2}\left(2\right)\left(\frac{1-\mu}{\left(1-\mu_{1}\right)\left(1-\mu_{2}\right)}\right)&=&\mu_{2}\left(\frac{r}{1-\mu_{1}}\right)\\
\end{eqnarray*}
por tanto
\begin{eqnarray*}
f_{2}\left(2\right)&=&\frac{r\frac{\mu_{2}}{1-\mu_{1}}}{\frac{1-\mu}{\left(1-\mu_{1}\right)\left(1-\mu_{2}\right)}}=\frac{r\mu_{2}\left(1-\mu_{1}\right)\left(1-\mu_{2}\right)}{\left(1-\mu_{1}\right)\left(1-\mu\right)}\\
&=&\frac{\mu_{2}\left(1-\mu_{2}\right)}{1-\mu}r=r\mu_{2}\frac{1-\mu_{2}}{1-\mu}.
\end{eqnarray*}
es decir

\begin{eqnarray}
f_{2}\left(2\right)&=&r\mu_{2}\frac{1-\mu_{2}}{1-\mu}.
\end{eqnarray}

Entonces

\begin{eqnarray*}
f_{1}\left(1\right)&=&\mu_{1}r+f_{2}\left(2\right)\frac{\mu_{1}}{1-\mu_{2}}=\mu_{1}r+\left(\frac{\mu_{2}\left(1-\mu_{2}\right)}{1-\mu}r\right)\frac{\mu_{1}}{1-\mu_{2}}\\
&=&\mu_{1}r+\mu_{1}r\left(\frac{\mu_{2}}{1-\mu}\right)=\mu_{1}r\left[1+\frac{\mu_{2}}{1-\mu}\right]\\
&=&r\mu_{1}\frac{1-\mu_{1}}{1-\mu}\\
\end{eqnarray*}

%_________________________________________________________________________
\section{El problema de la ruina del jugador}
%_________________________________________________________________________

Supongamos que se tiene un jugador que cuenta con un capital inicial de $\tilde{L}_{0}\geq0$ unidades, esta persona realiza una serie de dos juegos simult\'aneos e independientes de manera sucesiva, dichos eventos son independientes e id\'enticos entre s\'i para cada realizaci\'on. Para $n\geq0$ fijo, la ganancia en el $n$-\'esimo juego es $\tilde{X}_{n}=X_{n}+Y_{n}$ unidades de las cuales se resta una cuota de 1 unidad por cada juego simult\'aneo, es decir, se restan dos unidades por cada juego realizado. En t\'erminos de la teor\'ia de colas puede pensarse como el n\'umero de usuarios que llegan a una cola v\'ia dos procesos de arribo distintos e independientes entre s\'i. Su Funci\'on Generadora de Probabilidades (FGP) est\'a dada por $F\left(z\right)=\esp\left[z^{\tilde{L}_{0}}\right]$ para $z\in\mathbb{C}$, adem\'as
$$\tilde{P}\left(z\right)=\esp\left[z^{\tilde{X}_{n}}\right]=\esp\left[z^{X_{n}+Y_{n}}\right]=\esp\left[z^{X_{n}}z^{Y_{n}}\right]=\esp\left[z^{X_{n}}\right]\esp\left[z^{Y_{n}}\right]=P\left(z\right)\check{P}\left(z\right),$$ con $\tilde{\mu}=\esp\left[\tilde{X}_{n}\right]=\tilde{P}\left[z\right]<1$. 

Sea $\tilde{L}_{n}$ el capital remanente despu\'es del $n$-\'esimo
juego. Entonces
$$\tilde{L}_{n}=\tilde{L}_{0}+\tilde{X}_{1}+\tilde{X}_{2}+\cdots+\tilde{X}_{n}-2n.$$

La ruina del jugador ocurre despu\'es del $n$-\'esimo juego, es decir, la cola se vac\'ia despu\'es del $n$-\'esimo juego, entonces sea $T$ definida como $T=min\left\{\tilde{L}_{n}=0\right\}$. Si $\tilde{L}_{0}=0$, entonces claramente $T=0$. En este sentido $T$ puede interpretarse como la longitud del periodo de tiempo que el servidor ocupa para dar servicio en la cola, comenzando con $\tilde{L}_{0}$ grupos de usuarios presentes en la cola, quienes arribaron conforme a un proceso dado por $\tilde{P}\left(z\right)$.

Sea $g_{n,k}$ la probabilidad del evento de que el jugador no caiga en ruina antes del $n$-\'esimo juego, y que adem\'as tenga un capital de $k$ unidades antes del $n$-\'esimo juego, es decir, dada $n\in\left\{1,2,\ldots\right\}$ y $k\in\left\{0,1,2,\ldots\right\}$
\begin{eqnarray}
g_{n,k}:=P\left\{\tilde{L}_{j}>0, j=1,\ldots,n,
\tilde{L}_{n}=k\right\},
\end{eqnarray}
la cual adem\'as se puede escribir como:
\begin{eqnarray*}
g_{n,k}&=&P\left\{\tilde{L}_{j}>0, j=1,\ldots,n,
\tilde{L}_{n}=k\right\}=\sum_{j=1}^{k+1}g_{n-1,j}P\left\{\tilde{X}_{n}=k-j+1\right\}\\
&=&\sum_{j=1}^{k+1}g_{n-1,j}P\left\{X_{n}+Y_{n}=k-j+1\right\}=\sum_{j=1}^{k+1}\sum_{l=1}^{j}g_{n-1,j}P\left\{X_{n}+Y_{n}=k-j+1,Y_{n}=l\right\}\\
&=&\sum_{j=1}^{k+1}\sum_{l=1}^{j}g_{n-1,j}P\left\{X_{n}+Y_{n}=k-j+1|Y_{n}=l\right\}P\left\{Y_{n}=l\right\}\\
&=&\sum_{j=1}^{k+1}\sum_{l=1}^{j}g_{n-1,j}P\left\{X_{n}=k-j-l+1\right\}P\left\{Y_{n}=l\right\},
\end{eqnarray*}

es decir
\begin{eqnarray}\label{Eq.Gnk.2S}
g_{n,k}=\sum_{j=1}^{k+1}\sum_{l=1}^{j}g_{n-1,j}P\left\{X_{n}=k-j-l+1\right\}P\left\{Y_{n}=l\right\}.
\end{eqnarray}
Adem\'as
\begin{equation}\label{Eq.L02S}
g_{0,k}=P\left\{\tilde{L}_{0}=k\right\}.
\end{equation}
Se definen las siguientes FGP:
\begin{equation}\label{Eq.3.16.a.2S}
G_{n}\left(z\right)=\sum_{k=0}^{\infty}g_{n,k}z^{k},\textrm{ para
}n=0,1,\ldots,
\end{equation}
y 
\begin{equation}\label{Eq.3.16.b.2S}
G\left(z,w\right)=\sum_{n=0}^{\infty}G_{n}\left(z\right)w^{n}, z,w\in\mathbb{C}.
\end{equation}
En particular para $k=0$,
\begin{eqnarray*}
g_{n,0}=G_{n}\left(0\right)=P\left\{\tilde{L}_{j}>0,\textrm{ para
}j<n,\textrm{ y }\tilde{L}_{n}=0\right\}=P\left\{T=n\right\},
\end{eqnarray*}
adem\'as
\begin{eqnarray*}%\label{Eq.G0w.2S}
G\left(0,w\right)=\sum_{n=0}^{\infty}G_{n}\left(0\right)w^{n}=\sum_{n=0}^{\infty}P\left\{T=n\right\}w^{n}
=\esp\left[w^{T}\right]
\end{eqnarray*}
la cu\'al resulta ser la FGP del tiempo de ruina $T$.

\begin{Prop}\label{Prop.1.1.2S}
Sean $z,w\in\mathbb{C}$ y sea $n\geq0$ fijo. Para $G_{n}\left(z\right)$ y $G\left(z,w\right)$ definidas como en (\ref{Eq.3.16.a.2S}) y (\ref{Eq.3.16.b.2S}) respectivamente, se tiene que
\begin{equation}\label{Eq.Pag.45}
G_{n}\left(z\right)=\frac{1}{z}\left[G_{n-1}\left(z\right)-G_{n-1}\left(0\right)\right]\tilde{P}\left(z\right).
\end{equation}

Adem\'as

\begin{equation}\label{Eq.Pag.46}
G\left(z,w\right)=\frac{zF\left(z\right)-wP\left(z\right)G\left(0,w\right)}{z-wR\left(z\right)},
\end{equation}

con un \'unico polo en el c\'irculo unitario, adem\'as, el polo es
de la forma $z=\theta\left(w\right)$ y satisface que

\begin{enumerate}
\item[i)]$\tilde{\theta}\left(1\right)=1$,

\item[ii)] $\tilde{\theta}^{(1)}\left(1\right)=\frac{1}{1-\tilde{\mu}}$,

\item[iii)]
$\tilde{\theta}^{(2)}\left(1\right)=\frac{\tilde{\mu}}{\left(1-\tilde{\mu}\right)^{2}}+\frac{\tilde{\sigma}}{\left(1-\tilde{\mu}\right)^{3}}$.
\end{enumerate}

Finalmente, adem\'as se cumple que
\begin{equation}
\esp\left[w^{T}\right]=G\left(0,w\right)=F\left[\tilde{\theta}\left(w\right)\right].
\end{equation}
\end{Prop}
\begin{proof}

Multiplicando las ecuaciones (\ref{Eq.Gnk.2S}) y (\ref{Eq.L02S})
por el t\'ermino $z^{k}$:

\begin{eqnarray*}
g_{n,k}z^{k}&=&\sum_{j=1}^{k+1}\sum_{l=1}^{j}g_{n-1,j}P\left\{X_{n}=k-j-l+1\right\}P\left\{Y_{n}=l\right\}z^{k},\\
g_{0,k}z^{k}&=&P\left\{\tilde{L}_{0}=k\right\}z^{k},
\end{eqnarray*}

ahora sumamos sobre $k$
\begin{eqnarray*}
\sum_{k=0}^{\infty}g_{n,k}z^{k}&=&\sum_{k=0}^{\infty}\sum_{j=1}^{k+1}\sum_{l=1}^{j}g_{n-1,j}P\left\{X_{n}=k-j-l+1\right\}P\left\{Y_{n}=l\right\}z^{k}\\
&=&\sum_{k=0}^{\infty}z^{k}\sum_{j=1}^{k+1}\sum_{l=1}^{j}g_{n-1,j}P\left\{X_{n}=k-\left(j+l-1\right)\right\}P\left\{Y_{n}=l\right\}\\
&=&\sum_{k=0}^{\infty}z^{k+\left(j+l-1\right)-\left(j+l-1\right)}\sum_{j=1}^{k+1}\sum_{l=1}^{j}g_{n-1,j}P\left\{X_{n}=k-\left(j+l-1\right)\right\}P\left\{Y_{n}=l\right\}\\
&=&\sum_{k=0}^{\infty}\sum_{j=1}^{k+1}\sum_{l=1}^{j}g_{n-1,j}z^{j-1}P\left\{X_{n}=k-\left(j+l-1\right)\right\}z^{k-\left(j+l-1\right)}P\left\{Y_{n}=l\right\}z^{l}\\
&=&\sum_{j=1}^{\infty}\sum_{l=1}^{j}g_{n-1,j}z^{j-1}\sum_{k=j+l-1}^{\infty}P\left\{X_{n}=k-\left(j+l-1\right)\right\}z^{k-\left(j+l-1\right)}P\left\{Y_{n}=l\right\}z^{l}\\
&=&\sum_{j=1}^{\infty}g_{n-1,j}z^{j-1}\sum_{l=1}^{j}\sum_{k=j+l-1}^{\infty}P\left\{X_{n}=k-\left(j+l-1\right)\right\}z^{k-\left(j+l-1\right)}P\left\{Y_{n}=l\right\}z^{l}\\
&=&\sum_{j=1}^{\infty}g_{n-1,j}z^{j-1}\sum_{k=j+l-1}^{\infty}\sum_{l=1}^{j}P\left\{X_{n}=k-\left(j+l-1\right)\right\}z^{k-\left(j+l-1\right)}P\left\{Y_{n}=l\right\}z^{l}\\
&=&\sum_{j=1}^{\infty}g_{n-1,j}z^{j-1}\sum_{k=j+l-1}^{\infty}\sum_{l=1}^{j}P\left\{X_{n}=k-\left(j+l-1\right)\right\}z^{k-\left(j+l-1\right)}\sum_{l=1}^{j}P\left\{Y_{n}=l\right\}z^{l}\\
&=&\sum_{j=1}^{\infty}g_{n-1,j}z^{j-1}\sum_{l=1}^{\infty}P\left\{Y_{n}=l\right\}z^{l}\sum_{k=j+l-1}^{\infty}\sum_{l=1}^{j}P\left\{X_{n}=k-\left(j+l-1\right)\right\}z^{k-\left(j+l-1\right)}\\
&=&\frac{1}{z}\left[G_{n-1}\left(z\right)-G_{n-1}\left(0\right)\right]\check{P}\left(z\right)\sum_{k=j+l-1}^{\infty}\sum_{l=1}^{j}P\left\{X_{n}=k-\left(j+l-1\right)\right\}z^{k-\left(j+l-1\right)}\\
&=&\frac{1}{z}\left[G_{n-1}\left(z\right)-G_{n-1}\left(0\right)\right]\check{P}\left(z\right)P\left(z\right)=\frac{1}{z}\left[G_{n-1}\left(z\right)-G_{n-1}\left(0\right)\right]\tilde{P}\left(z\right),
\end{eqnarray*}
es decir la ecuaci\'on (\ref{Eq.3.16.a.2S}) se puede reescribir como
\begin{equation}\label{Eq.3.16.a.2Sbis}
G_{n}\left(z\right)=\frac{1}{z}\left[G_{n-1}\left(z\right)-G_{n-1}\left(0\right)\right]\tilde{P}\left(z\right).
\end{equation}

Por otra parte recordemos la ecuaci\'on (\ref{Eq.3.16.a.2S})
\begin{eqnarray*}
G_{n}\left(z\right)&=&\sum_{k=0}^{\infty}g_{n,k}z^{k},\textrm{ entonces }\frac{G_{n}\left(z\right)}{z}=\sum_{k=1}^{\infty}g_{n,k}z^{k-1},
\end{eqnarray*}

por lo tanto utilizando la ecuaci\'on (\ref{Eq.3.16.a.2Sbis}):

\begin{eqnarray*}
G\left(z,w\right)&=&\sum_{n=0}^{\infty}G_{n}\left(z\right)w^{n}=G_{0}\left(z\right)+\sum_{n=1}^{\infty}G_{n}\left(z\right)w^{n}=F\left(z\right)+\sum_{n=0}^{\infty}\left[G_{n}\left(z\right)-G_{n}\left(0\right)\right]w^{n}\frac{\tilde{P}\left(z\right)}{z}\\
&=&F\left(z\right)+\frac{w}{z}\sum_{n=0}^{\infty}\left[G_{n}\left(z\right)-G_{n}\left(0\right)\right]w^{n-1}\tilde{P}\left(z\right)
\end{eqnarray*}
es decir
\begin{eqnarray*}
G\left(z,w\right)&=&F\left(z\right)+\frac{w}{z}\left[G\left(z,w\right)-G\left(0,w\right)\right]\tilde{P}\left(z\right),
\end{eqnarray*}
entonces
\begin{eqnarray*}
G\left(z,w\right)=F\left(z\right)+\frac{w}{z}\left[G\left(z,w\right)-G\left(0,w\right)\right]\tilde{P}\left(z\right)&=&F\left(z\right)+\frac{w}{z}\tilde{P}\left(z\right)G\left(z,w\right)-\frac{w}{z}\tilde{P}\left(z\right)G\left(0,w\right)\\
&\Leftrightarrow&\\
G\left(z,w\right)\left\{1-\frac{w}{z}\tilde{P}\left(z\right)\right\}&=&F\left(z\right)-\frac{w}{z}\tilde{P}\left(z\right)G\left(0,w\right),
\end{eqnarray*}
por lo tanto,
\begin{equation}
G\left(z,w\right)=\frac{zF\left(z\right)-w\tilde{P}\left(z\right)G\left(0,w\right)}{1-w\tilde{P}\left(z\right)}.
\end{equation}
Ahora $G\left(z,w\right)$ es anal\'itica en $|z|=1$. Sean $z,w$ tales que $|z|=1$ y $|w|\leq1$, como $\tilde{P}\left(z\right)$ es FGP
\begin{eqnarray*}
|z-\left(z-w\tilde{P}\left(z\right)\right)|<|z|\Leftrightarrow|w\tilde{P}\left(z\right)|<|z|
\end{eqnarray*}
es decir, se cumplen las condiciones del Teorema de Rouch\'e y por tanto, $z$ y $z-w\tilde{P}\left(z\right)$ tienen el mismo n\'umero de ceros en $|z|=1$. Sea $z=\tilde{\theta}\left(w\right)$ la soluci\'on \'unica de $z-w\tilde{P}\left(z\right)$, es decir
\begin{equation}\label{Eq.Theta.w}
\tilde{\theta}\left(w\right)-w\tilde{P}\left(\tilde{\theta}\left(w\right)\right)=0,
\end{equation}
con $|\tilde{\theta}\left(w\right)|<1$. Cabe hacer menci\'on que $\tilde{\theta}\left(w\right)$ es la FGP para el tiempo de ruina cuando $\tilde{L}_{0}=1$. Considerando la ecuaci\'on (\ref{Eq.Theta.w})
\begin{eqnarray*}
0&=&\frac{\partial}{\partial w}\tilde{\theta}\left(w\right)|_{w=1}-\frac{\partial}{\partial w}\left\{w\tilde{P}\left(\tilde{\theta}\left(w\right)\right)\right\}|_{w=1}=\tilde{\theta}^{(1)}\left(w\right)|_{w=1}-\frac{\partial}{\partial w}w\left\{\tilde{P}\left(\tilde{\theta}\left(w\right)\right)\right\}|_{w=1}\\
&-&w\frac{\partial}{\partial w}\tilde{P}\left(\tilde{\theta}\left(w\right)\right)|_{w=1}=\tilde{\theta}^{(1)}\left(1\right)-\tilde{P}\left(\tilde{\theta}\left(1\right)\right)-w\left\{\frac{\partial \tilde{P}\left(\tilde{\theta}\left(w\right)\right)}{\partial \tilde{\theta}\left(w\right)}\cdot\frac{\partial\tilde{\theta}\left(w\right)}{\partial w}|_{w=1}\right\}\\
&=&\tilde{\theta}^{(1)}\left(1\right)-\tilde{P}\left(\tilde{\theta}\left(1\right)
\right)-\tilde{P}^{(1)}\left(\tilde{\theta}\left(1\right)\right)\cdot\tilde{\theta}^{(1)}\left(1\right),
\end{eqnarray*}
luego
$$\tilde{P}\left(\tilde{\theta}\left(1\right)\right)=\tilde{\theta}^{(1)}\left(1\right)-\tilde{P}^{(1)}\left(\tilde{\theta}\left(1\right)\right)\cdot\tilde{\theta}^{(1)}\left(1\right)=\tilde{\theta}^{(1)}\left(1\right)\left(1-\tilde{P}^{(1)}\left(\tilde{\theta}\left(1\right)\right)\right),$$
por tanto $$\tilde{\theta}^{(1)}\left(1\right)=\frac{\tilde{P}\left(\tilde{\theta}\left(1\right)\right)}{\left(1-\tilde{P}^{(1)}\left(\tilde{\theta}\left(1\right)\right)\right)}=\frac{1}{1-\tilde{\mu}}.$$
Ahora determinemos el segundo momento de $\tilde{\theta}\left(w\right)$,
nuevamente consideremos la ecuaci\'on (\ref{Eq.Theta.w}):
\begin{eqnarray*}
0&=&\tilde{\theta}\left(w\right)-w\tilde{P}\left(\tilde{\theta}\left(w\right)\right)\Rightarrow 0=\frac{\partial}{\partial w}\left\{\tilde{\theta}\left(w\right)-w\tilde{P}\left(\tilde{\theta}\left(w\right)\right)\right\}\Rightarrow 0=\frac{\partial}{\partial w}\left\{\frac{\partial}{\partial w}\left\{\tilde{\theta}\left(w\right)-w\tilde{P}\left(\tilde{\theta}\left(w\right)\right)\right\}\right\}
\end{eqnarray*}
luego se tiene
\begin{eqnarray*}
&&\frac{\partial}{\partial w}\left\{\frac{\partial}{\partial w}\tilde{\theta}\left(w\right)-\frac{\partial}{\partial w}\left[w\tilde{P}\left(\tilde{\theta}\left(w\right)\right)\right]\right\}
=\frac{\partial}{\partial w}\left\{\frac{\partial}{\partial w}\tilde{\theta}\left(w\right)-\frac{\partial}{\partial w}\left[w\tilde{P}\left(\tilde{\theta}\left(w\right)\right)\right]\right\}\\
&=&\frac{\partial}{\partial w}\left\{\frac{\partial \tilde{\theta}\left(w\right)}{\partial w}-\left[\tilde{P}\left(\tilde{\theta}\left(w\right)\right)+w\frac{\partial}{\partial w}P\left(\tilde{\theta}\left(w\right)\right)\right]\right\}\\
&=&\frac{\partial}{\partial w}\left\{\frac{\partial \tilde{\theta}\left(w\right)}{\partial w}-\left(\tilde{P}\left(\tilde{\theta}\left(w\right)\right)+w\frac{\partial \tilde{P}\left(\tilde{\theta}\left(w\right)\right)}{\partial w}\frac{\partial \tilde{\theta}\left(w\right)}{\partial w}\right]\right\}\\
&=&\frac{\partial}{\partial w}\left\{\tilde{\theta}^{(1)}\left(w\right)-\tilde{P}\left(\tilde{\theta}\left(w\right)\right)-w\tilde{P}^{(1)}\left(\tilde{\theta}\left(w\right)\right)\tilde{\theta}^{(1)}\left(w\right)\right\}\\
&=&\frac{\partial}{\partial w}\tilde{\theta}^{(1)}\left(w\right)-\frac{\partial}{\partial w}\tilde{P}\left(\tilde{\theta}\left(w\right)\right)-\frac{\partial}{\partial w}\left[w\tilde{P}^{(1)}\left(\tilde{\theta}\left(w\right)\right)\tilde{\theta}^{(1)}\left(w\right)\right]\\
&=&\frac{\partial}{\partial w}\tilde{\theta}^{(1)}\left(w\right)-\frac{\partial\tilde{P}\left(\tilde{\theta}\left(w\right)\right)}{\partial\tilde{\theta}\left(w\right)}\frac{\partial \tilde{\theta}\left(w\right)}{\partial w}-\tilde{P}^{(1)}\left(\tilde{\theta}\left(w\right)\right)\tilde{\theta}^{(1)}\left(w\right)-w\frac{\partial\tilde{P}^{(1)}\left(\tilde{\theta}\left(w\right)\right)}{\partial w}\tilde{\theta}^{(1)}\left(w\right)\\
&-&w\tilde{P}^{(1)}\left(\tilde{\theta}\left(w\right)\right)\frac{\partial \tilde{\theta}^{(1)}\left(w\right)}{\partial w}\\
&=&\tilde{\theta}^{(2)}\left(w\right)-\tilde{P}^{(1)}\left(\tilde{\theta}\left(w\right)\right)\tilde{\theta}^{(1)}\left(w\right)-\tilde{P}^{(1)}\left(\tilde{\theta}\left(w\right)\right)\tilde{\theta}^{(1)}\left(w\right)-w\tilde{P}^{(2)}\left(\tilde{\theta}\left(w\right)\right)\left(\tilde{\theta}^{(1)}\left(w\right)\right)^{2}\\
&-&w\tilde{P}^{(1)}\left(\tilde{\theta}\left(w\right)\right)\tilde{\theta}^{(2)}\left(w\right)\\
&=&\tilde{\theta}^{(2)}\left(w\right)-2\tilde{P}^{(1)}\left(\tilde{\theta}\left(w\right)\right)\tilde{\theta}^{(1)}\left(w\right)-w\tilde{P}^{(2)}\left(\tilde{\theta}\left(w\right)\right)\left(\tilde{\theta}^{(1)}\left(w\right)\right)^{2}-w\tilde{P}^{(1)}\left(\tilde{\theta}\left(w\right)\right)\tilde{\theta}^{(2)}\left(w\right)\\
&=&\tilde{\theta}^{(2)}\left(w\right)\left[1-w\tilde{P}^{(1)}\left(\tilde{\theta}\left(w\right)\right)\right]-
\tilde{\theta}^{(1)}\left(w\right)\left[w\tilde{\theta}^{(1)}\left(w\right)\tilde{P}^{(2)}\left(\tilde{\theta}\left(w\right)\right)+2\tilde{P}^{(1)}\left(\tilde{\theta}\left(w\right)\right)\right]
\end{eqnarray*}
luego
\begin{eqnarray*}
\tilde{\theta}^{(2)}\left(w\right)&&\left[1-w\tilde{P}^{(1)}\left(\tilde{\theta}\left(w\right)\right)\right]-\tilde{\theta}^{(1)}\left(w\right)\left[w\tilde{\theta}^{(1)}\left(w\right)\tilde{P}^{(2)}\left(\tilde{\theta}\left(w\right)\right)+2\tilde{P}^{(1)}\left(\tilde{\theta}\left(w\right)\right)\right]=0\\
\tilde{\theta}^{(2)}\left(w\right)&=&\frac{\tilde{\theta}^{(1)}\left(w\right)\left[w\tilde{\theta}^{(1)}\left(w\right)\tilde{P}^{(2)}\left(\tilde{\theta}\left(w\right)\right)+2P^{(1)}\left(\tilde{\theta}\left(w\right)\right)\right]}{1-w\tilde{P}^{(1)}\left(\tilde{\theta}\left(w\right)\right)}\\
&=&\frac{\tilde{\theta}^{(1)}\left(w\right)w\tilde{\theta}^{(1)}\left(w\right)\tilde{P}^{(2)}\left(\tilde{\theta}\left(w\right)\right)}{1-w\tilde{P}^{(1)}\left(\tilde{\theta}\left(w\right)\right)}+\frac{2\tilde{\theta}^{(1)}\left(w\right)\tilde{P}^{(1)}\left(\tilde{\theta}\left(w\right)\right)}{1-w\tilde{P}^{(1)}\left(\tilde{\theta}\left(w\right)\right)}
\end{eqnarray*}
si evaluamos la expresi\'on anterior en $w=1$:
\begin{eqnarray*}
\tilde{\theta}^{(2)}\left(1\right)&=&\frac{\left(\tilde{\theta}^{(1)}\left(1\right)\right)^{2}\tilde{P}^{(2)}\left(\tilde{\theta}\left(1\right)\right)}{1-\tilde{P}^{(1)}\left(\tilde{\theta}\left(1\right)\right)}+\frac{2\tilde{\theta}^{(1)}\left(1\right)\tilde{P}^{(1)}\left(\tilde{\theta}\left(1\right)\right)}{1-\tilde{P}^{(1)}\left(\tilde{\theta}\left(1\right)\right)}=\frac{\left(\tilde{\theta}^{(1)}\left(1\right)\right)^{2}\tilde{P}^{(2)}\left(1\right)}{1-\tilde{P}^{(1)}\left(1\right)}+\frac{2\tilde{\theta}^{(1)}\left(1\right)\tilde{P}^{(1)}\left(1\right)}{1-\tilde{P}^{(1)}\left(1\right)}\\
&=&\frac{\left(\frac{1}{1-\tilde{\mu}}\right)^{2}\tilde{P}^{(2)}\left(1\right)}{1-\tilde{\mu}}+\frac{2\left(\frac{1}{1-\tilde{\mu}}\right)\tilde{\mu}}{1-\tilde{\mu}}=\frac{\tilde{P}^{(2)}\left(1\right)}{\left(1-\tilde{\mu}\right)^{3}}+\frac{2\tilde{\mu}}{\left(1-\tilde{\mu}\right)^{2}}=\frac{\sigma^{2}-\tilde{\mu}+\tilde{\mu}^{2}}{\left(1-\tilde{\mu}\right)^{3}}+\frac{2\tilde{\mu}}{\left(1-\tilde{\mu}\right)^{2}}\\
&=&\frac{\sigma^{2}-\tilde{\mu}+\tilde{\mu}^{2}+2\tilde{\mu}\left(1-\tilde{\mu}\right)}{\left(1-\tilde{\mu}\right)^{3}}
\end{eqnarray*}
es decir
\begin{eqnarray*}
\tilde{\theta}^{(2)}\left(1\right)&=&\frac{\sigma^{2}}{\left(1-\tilde{\mu}\right)^{3}}+\frac{\tilde{\mu}}{\left(1-\tilde{\mu}\right)^{2}}.
\end{eqnarray*}
\end{proof}

\begin{Coro}
El tiempo de ruina del jugador tiene primer y segundo momento dados por
\begin{eqnarray}
\esp\left[T\right]&=&\frac{\esp\left[\tilde{L}_{0}\right]}{1-\tilde{\mu}}\\
Var\left[T\right]&=&\frac{Var\left[\tilde{L}_{0}\right]}{\left(1-\tilde{\mu}\right)^{2}}+\frac{\sigma^{2}\esp\left[\tilde{L}_{0}\right]}{\left(1-\tilde{\mu}\right)^{3}}.
\end{eqnarray}
\end{Coro}

Se considerar\'an intervalos de tiempo de la forma
$\left[t,t+1\right]$. Los usuarios arriban por paquetes de manera
independiente del resto de las colas. Se define el grupo de
usuarios que llegan a cada una de las colas del sistema 1,
caracterizadas por $Q_{1}$ y $Q_{2}$ respectivamente, en el
intervalo de tiempo $\left[t,t+1\right]$ por
$X_{1}\left(t\right),X_{2}\left(t\right)$.


%______________________________________________________________________
\section{Ecuaciones Centrales}
%______________________________________________________________________

\begin{Prop}
Supongamos

\begin{equation}\label{Eq.1}
f_{i}\left(i\right)-f_{j}\left(i\right)=\mu_{i}\left[\sum_{k=j}^{i-1}r_{k}+\sum_{k=j}^{i-1}\frac{f_{k}\left(k\right)}{1-\mu_{k}}\right]
\end{equation}

\begin{equation}\label{Eq.2}
f_{i+1}\left(i\right)=r_{i}\mu_{i},
\end{equation}

Demostrar que

\begin{eqnarray*}
f_{i}\left(i\right)&=&\mu_{i}\left[\sum_{k=1}^{N}r_{k}+\sum_{k=1,k\neq i}^{N}\frac{f_{k}\left(k\right)}{1-\mu_{k}}\right].
\end{eqnarray*}

En la Ecuaci\'on (\ref{Eq.2}) hagamos $j=i+1$, entonces se tiene $f_{j}=r_{i}\mu_{i}$, lo mismo para (\ref{Eq.1})

\begin{eqnarray*}
f_{i}\left(i\right)&=&r_{i}\mu_{i}+\mu_{i}\left[\sum_{k=j}^{i-1}r_{k}+\sum_{k=j}^{i-1}\frac{f_{k}\left(k\right)}{1-\mu_{k}}\right]\\
&=&\mu_{i}\left[\sum_{k=j}^{i}r_{k}+\sum_{k=j}^{i-1}\frac{f_{k}\left(k\right)}{1-\mu_{k}}\right]\\
\end{eqnarray*}

entonces, tomando sobre todo valor de $1,\ldots,N$, tanto para antes de $i$ como para despu\'es de $i$, entonces

\begin{eqnarray*}
f_{i}\left(i\right)&=&\mu_{i}\left[\sum_{k=1}^{N}r_{k}+\sum_{k=1,k\neq i}^{N}\frac{f_{k}\left(k\right)}{1-\mu_{k}}\right].
\end{eqnarray*}
\end{Prop}

Ahora, supongamos nuevamente la ecuaci\'on (\ref{Eq.1})

\begin{eqnarray*}
f_{i}\left(i\right)-f_{j}\left(i\right)&=&\mu_{i}\left[\sum_{k=j}^{i-1}r_{k}+\sum_{k=j}^{i-1}\frac{f_{k}\left(k\right)}{1-\mu_{k}}\right]\\
&\Leftrightarrow&\\
f_{j}\left(j\right)-f_{i}\left(j\right)&=&\mu_{j}\left[\sum_{k=i}^{j-1}r_{k}+\sum_{k=i}^{j-1}\frac{f_{k}\left(k\right)}{1-\mu_{k}}\right]\\
f_{i}\left(j\right)&=&f_{j}\left(j\right)-\mu_{j}\left[\sum_{k=i}^{j-1}r_{k}+\sum_{k=i}^{j-1}\frac{f_{k}\left(k\right)}{1-\mu_{k}}\right]\\
&=&\mu_{j}\left(1-\mu_{j}\right)\frac{r}{1-\mu}-\mu_{j}\left[\sum_{k=i}^{j-1}r_{k}+\sum_{k=i}^{j-1}\frac{f_{k}\left(k\right)}{1-\mu_{k}}\right]\\
&=&\mu_{j}\left[\left(1-\mu_{j}\right)\frac{r}{1-\mu}-\sum_{k=i}^{j-1}r_{k}-\sum_{k=i}^{j-1}\frac{f_{k}\left(k\right)}{1-\mu_{k}}\right]\\
&=&\mu_{j}\left[\left(1-\mu_{j}\right)\frac{r}{1-\mu}-\sum_{k=i}^{j-1}r_{k}-\frac{r}{1-\mu}\sum_{k=i}^{j-1}\mu_{k}\right]\\
&=&\mu_{j}\left[\frac{r}{1-\mu}\left(1-\mu_{j}-\sum_{k=i}^{j-1}\mu_{k}\right)-\sum_{k=i}^{j-1}r_{k}\right]\\
&=&\mu_{j}\left[\frac{r}{1-\mu}\left(1-\sum_{k=i}^{j}\mu_{k}\right)-\sum_{k=i}^{j-1}r_{k}\right].\\
\end{eqnarray*}

Ahora,

\begin{eqnarray*}
1-\sum_{k=i}^{j}\mu_{k}&=&1-\sum_{k=1}^{N}\mu_{k}+\sum_{k=j+1}^{i-1}\mu_{k}\\
&\Leftrightarrow&\\
\sum_{k=i}^{j}\mu_{k}&=&\sum_{k=1}^{N}\mu_{k}-\sum_{k=j+1}^{i-1}\mu_{k}\\
&\Leftrightarrow&\\
\sum_{k=1}^{N}\mu_{k}&=&\sum_{k=i}^{j}\mu_{k}+\sum_{k=j+1}^{i-1}\mu_{k}\\
\end{eqnarray*}

Por tanto
\begin{eqnarray*}
f_{i}\left(j\right)&=&\mu_{j}\left[\frac{r}{1-\mu}\sum_{k=j+1}^{i-1}\mu_{k}+\sum_{k=j}^{i-1}r_{k}\right].
\end{eqnarray*}

\begin{Teo}[Teorema de Continuidad]
Sup\'ongase que $\left\{X_{n},n=1,2,3,\ldots\right\}$ son variables aleatorias finitas, no negativas con valores enteros tales que $P\left(X_{n}=k\right)=p_{k}^{(n)}$, para $n=1,2,3,\ldots$, $k=0,1,2,\ldots$, con $\sum_{k=0}^{\infty}p_{k}^{(n)}=1$, para $n=1,2,3,\ldots$. Sea $g_{n}$ la PGF para la variable aleatoria $X_{n}$. Entonces existe una sucesi\'on $\left\{p_{k}\right\}$ tal que \begin{eqnarray*}
lim_{n\rightarrow\infty}p_{k}^{(n)}=p_{k}\textrm{ para }0<s<1.
\end{eqnarray*}
En este caso, $g\left(s\right)=\sum_{k=0}^{\infty}s^{k}p_{k}$. Adem\'as
\begin{eqnarray*}
\sum_{k=0}^{\infty}p_{k}=1\textrm{ si y s\'olo si
}lim_{s\uparrow1}g\left(s\right)=1
\end{eqnarray*}
\end{Teo}

\begin{Teo}
Sea $N$ una variable aleatoria con valores enteros no negativos finita tal que $P\left(N=k\right)=p_{k}$, para $k=0,1,2,\ldots$, y $\sum_{k=0}^{\infty}p_{k}=P\left(N<\infty\right)=1$. Sea $\Phi$ la PGF de $N$ tal que $g\left(s\right)=\esp\left[s^{N}\right]=\sum_{k=0}^{\infty}s^{k}p_{k}$ con $g\left(1\right)=1$. Si $0\leq p_{1}\leq1$ y $\esp\left[N\right]=g^{'}\left(1\right)\leq1$, entonces no existe soluci\'on  de la ecuaci\'on $g\left(s\right)=s$ en el intervalo $\left[0,1\right)$. Si $\esp\left[N\right]=g^{'}\left(1\right)>1$, lo cual implica que $0\leq p_{1}<1$, entonces existe una \'unica soluci\'on de la ecuaci\'on $g\left(s\right)=s$ en el intervalo
$\left[0,1\right)$.
\end{Teo}

\begin{Teo}
Si $X$ y $Y$ tienen PGF $G_{X}$ y $G_{Y}$ respectivamente, entonces,\[G_{X}\left(s\right)=G_{Y}\left(s\right)\] para toda $s$, si y s\'olo si \[P\left(X=k\right))=P\left(Y=k\right)\] para toda $k=0,1,\ldots,$., es decir, si y s\'olo si $X$ y $Y$ tienen la misma distribuci\'on de probabilidad.
\end{Teo}


\begin{Teo}
Para cada $n$ fijo, sea la sucesi\'oin de probabilidades $\left\{a_{0,n},a_{1,n},\ldots,\right\}$, tales que $a_{k,n}\geq0$ para toda $k=0,1,2,\ldots,$ y $\sum_{k\geq0}a_{k,n}=1$, y sea $G_{n}\left(s\right)$ la correspondiente funci\'on generadora, $G_{n}\left(s\right)=\sum_{k\geq0}a_{k,n}s^{k}$. De modo que para cada valor fijo de $k$
\begin{eqnarray*}
lim_{n\rightarrow\infty}a_{k,n}=a_{k},
\end{eqnarray*}
es decir converge en distribuci\'on, es necesario y suficiente que para cada valor fijo $s\in\left[0,\right)$,
\begin{eqnarray*}
lim_{n\rightarrow\infty}G_{n}\left(s\right)=G\left(s\right),
\end{eqnarray*}
donde $G\left(s\right)=\sum_{k\geq0}p_{k}s^{k}$, para cualquier la funci\'on generadora del l\'imite de la sucesi\'on.
\end{Teo}

\begin{Teo}[Teorema de Abel]
Sea $G\left(s\right)=\sum_{k\geq0}a_{k}s^{k}$ para cualquier $\left\{p_{0},p_{1},\ldots,\right\}$, tales que $p_{k}\geq0$ para toda $k=0,1,2,\ldots,$. Entonces $G\left(s\right)$ es continua por la derecha en $s=1$, es decir
\begin{eqnarray*}
lim_{s\uparrow1}G\left(s\right)=\sum_{k\geq0}p_{k}=G\left(\right),
\end{eqnarray*}
sin importar si la suma es finita o no.
\end{Teo}
\begin{Note}
El radio de Convergencia para cualquier PGF es $R\geq1$, entonces, el Teorema de Abel nos dice que a\'un en el peor escenario, cuando $R=1$, a\'un se puede confiar en que la PGF ser\'a continua en $s=1$, en contraste, no se puede asegurar que la PGF ser\'a continua en el l\'imite inferior $-R$, puesto que la PGF es sim\'etrica alrededor del cero: la PGF converge para todo $s\in\left(-R,R\right)$, y no lo hace para $s<-R$ o $s>R$. Adem\'as nos dice que podemos escribir $G_{X}\left(1\right)$ como una abreviaci\'on de $lim_{s\uparrow1}G_{X}\left(s\right)$.
\end{Note}

Entonces si suponemos que la diferenciaci\'on t\'ermino a t\'ermino est\'a permitida, entonces

\begin{eqnarray*}
G_{X}^{'}\left(s\right)&=&\sum_{x=1}^{\infty}xs^{x-1}p_{x}
\end{eqnarray*}

el Teorema de Abel nos dice que
\begin{eqnarray*}
\esp\left(X\right]&=&\lim_{s\uparrow1}G_{X}^{'}\left(s\right):\\
\esp\left[X\right]&=&=\sum_{x=1}^{\infty}xp_{x}=G_{X}^{'}\left(1\right)\\
&=&\lim_{s\uparrow1}G_{X}^{'}\left(s\right),
\end{eqnarray*}
dado que el Teorema de Abel se aplica a
\begin{eqnarray*}
G_{X}^{'}\left(s\right)&=&\sum_{x=1}^{\infty}xs^{x-1}p_{x},
\end{eqnarray*}
estableciendo as\'i que $G_{X}^{'}\left(s\right)$ es continua en $s=1$. Sin el Teorema de Abel no se podr\'ia asegurar que el l\'imite de $G_{X}^{'}\left(s\right)$ conforme $s\uparrow1$ sea la respuesta correcta para $\esp\left[X\right]$.

\begin{Note}
La PGF converge para todo $|s|<R$, para alg\'un $R$. De hecho la PGF converge absolutamente si $|s|<R$. La PGF adem\'as converge uniformemente en conjuntos de la forma $\left\{s:|s|<R^{'}\right\}$, donde $R^{'}<R$, es decir, $\forall\epsilon>0, \exists n_{0}\in\ent$ tal que $\forall s$, con $|s|<R^{'}$, y $\forall n\geq n_{0}$,
\begin{eqnarray*}
|\sum_{x=0}^{n}s^{x}\prob\left(X=x\right)-G_{X}\left(s\right)|<\epsilon.
\end{eqnarray*}
De hecho, la convergencia uniforme es la que nos permite diferenciar t\'ermino a t\'ermino:
\begin{eqnarray*}
G_{X}\left(s\right)=\esp\left[s^{X}\right]=\sum_{x=0}^{\infty}s^{x}\prob\left(X=x\right),
\end{eqnarray*}
y sea $s<R$.
\begin{enumerate}
\item
\begin{eqnarray*}
G_{X}^{'}\left(s\right)&=&\frac{d}{ds}\left(\sum_{x=0}^{\infty}s^{x}\prob\left(X=x\right)\right)=\sum_{x=0}^{\infty}\frac{d}{ds}\left(s^{x}\prob\left(X=x\right)\right)\\
&=&\sum_{x=0}^{n}xs^{x-1}\prob\left(X=x\right).
\end{eqnarray*}

\item\begin{eqnarray*}
\int_{a}^{b}G_{X}\left(s\right)ds&=&\int_{a}^{b}\left(\sum_{x=0}^{\infty}s^{x}\prob\left(X=x\right)\right)ds=\sum_{x=0}^{\infty}\left(\int_{a}^{b}s^{x}\prob\left(X=x\right)ds\right)\\
&=&\sum_{x=0}^{\infty}\frac{s^{x+1}}{x+1}\prob\left(X=x\right),
\end{eqnarray*}
para $-R<a<b<R$.
\end{enumerate}
\end{Note}

\begin{Teo}[Teorema de Convergencia Mon\'otona para PGF]
Sean $X$ y $X_{n}$ variables aleatorias no negativas, con valores en los enteros, finitas, tales que
\begin{eqnarray*}
lim_{n\rightarrow\infty}G_{X_{n}}\left(s\right)&=&G_{X}\left(s\right)
\end{eqnarray*}
para $0\leq s\leq1$, entonces
\begin{eqnarray*}
lim_{n\rightarrow\infty}P\left(X_{n}=k\right)=P\left(X=k\right),
\end{eqnarray*}
para $k=0,1,2,\ldots.$
\end{Teo}

El teorema anterior requiere del siguiente lema

\begin{Lemma}
Sean $a_{n,k}\in\ent^{+}$, $n\in\nat$ constantes no negativas con $\sum_{k\geq0}a_{k,n}\leq1$. Sup\'ongase que para $0\leq s\leq1$,
se tiene
\begin{eqnarray*}
a_{n}\left(s\right)&=&\sum_{k=0}^{\infty}a_{k,n}s^{k}\rightarrow
a\left(s\right)=\sum_{k=0}^{\infty}a_{k}s^{k}.
\end{eqnarray*}
Entonces
\begin{eqnarray*}
a_{0,n}\rightarrow a_{0}.
\end{eqnarray*}
\end{Lemma}

%_________________________________________________________________________
\section{Redes de Jackson}
%_________________________________________________________________________
Cuando se considera la cantidad de
usuarios que llegan a cada uno de los nodos desde fuera del
sistema m\'as los que provienen del resto de los nodos, se dice
que la red es abierta y recibe el nombre de {\em Red de Jackson Abierta}.\\

Si denotamos por $Q_{1}\left(t\right),Q_{2}\left(t\right),\ldots,Q_{K}\left(t\right)$ el n\'umero de usuarios presentes en la cola $1,2,\ldots,K$ respectivamente al tiempo $t$, entonces se tiene la colecci\'on de colas $\left\{Q_{1},Q_{2},\ldots,Q_{K}\right\}$, donde despu\'es de que el usuario es atendido en la cola $i$, se traslada a la cola $j$ con probabilidad $p_{ij}$. En caso de que un usuario decida volver a ser atendido en $i$, este permanecer\'a en la misma cola con probabilidad $p_{ii}$. Para considerar a los usuarios que entran al sistema por primera vez por $i$, m\'as aquellos que provienen de otra cola, es necesario considerar un estado adicional $0$, con probabilidad de transici\'on $p_{00}=0$, $p_{0j}\geq0$ y $p_{j0}\geq0$, para $j=1,2,\ldots,K$, entonces en general la probabilidad de transici\'on de una cola a otra puede representarse por $P=\left(p_{ij}\right)_{i,j=0}^{K}$.\\

Para el caso espec\'ifico en el que en cada una de las colas los tiempos entre arribos y los tiempos de servicio sean exponenciales con par\'ametro de intensidad $\lambda$ y media $\mu$, respectivamente, con $m$ servidores y sin restricciones en la capacidad de almacenamiento en cada una de las colas, en Chee-Hook y Boon-Hee \cite{HookHee}, cap. 6, se muestra que el n\'umero de
usuarios en las $K$ colas, en el caso estacionario, puede determinarse por la ecuaci\'on (\ref{Eq.7.5.1})  que a
continuaci\'on se presenta, adem\'as de que la distribuci\'on l\'imite de la misma es (\ref{Eq.7.5.2}).\\

El n\'umero de usuarios en las $K$ colas en su estado estacionario, ver \cite{Bhat}, se define como
\begin{equation}\label{Eq.7.5.1}
p_{q_{1}q_{2}\cdots
q_{K}}=P\left[Q_{1}=q_{1},Q_{2}=q_{2},\ldots,Q_{K}=q_{K}\right].
\end{equation}

Jackson (1957), demostr\'o que la distribuci\'on l\'imite
$p_{q_{1}q_{2}\cdots q_{K}}$ de (\ref{Eq.7.5.1}) es

\begin{equation}\label{Eq.7.5.2}
p_{q_{1}q_{2}\cdots
q_{K}}=P_{1}\left(q_{1}\right)P_{2}\left(q_{2}\right)\cdots
P_{K}\left(q_{K}\right),
\end{equation}

donde
\begin{equation}\label{Eq.7.5.3}
p_{i}\left(r\right)=\left\{\begin{array}{cc}
 p_{i}\left(0\right)\frac{\left(\gamma_{i}/\mu_{i}\right)^{r}}{r!},  & r=0,1,2,\ldots,m, \\
 p_{i}\left(0\right)\frac{\left(\gamma_{i}/\mu_{i}\right)^{r}}{m!m^{r-m}}, & r=m,m+1,\ldots .\\
\end{array}\right.
\end{equation}

y

\begin{equation}\label{Eq.7.5.4}
\gamma_{i}=\lambda_{i}+\sum p_{ji}\gamma_{j},\textrm{
}i=1,2,\ldots,K.
\end{equation}

La relaci\'on (\ref{Eq.7.5.4}) es importante puesto que considera no solamente los arribos externos si no que adem\'as permite considerar intercambio de clientes entre las distintas colas que conforman el sistema.\\

Dados $\lambda_{i}$ y $p_{ij}$, la cantidad $\gamma_{i}$ puede determinarse a partir de la ecuaci\'on (\ref{Eq.7.5.4}) de manera recursiva. Adem\'as $p_{i}\left(0\right)$ puede determinarse utilizando la condici\'on de normalidad
\[\sum_{q_{1}}\sum_{q_{2}}\cdots\sum_{q_{K}}p_{q_{1}q_{2}\cdots q_{K}}=1.\]

Sin embargo las Redes de Jackson tienen el inconveniente de que no consideran el caso en que existan tiempos de traslado entre las colas. 




%--------------
%---- CAPITULO YA INCLUIDO EN OTRO ----
%\chapter{Polling Systems: Introducci\'on}
%\input{intPS}
%-------------- CAPITULO YA INCLUIDO EN OTRO ----
%\chapter{Polling Systems}
%

%______________________________________________________________________
\section{Ecuaciones Centrales}
%______________________________________________________________________



\begin{Prop}
Supongamos

\begin{equation}\label{Eq.1}
f_{i}\left(i\right)-f_{j}\left(i\right)=\mu_{i}\left[\sum_{k=j}^{i-1}r_{k}+\sum_{k=j}^{i-1}\frac{f_{k}\left(k\right)}{1-\mu_{k}}\right]
\end{equation}

\begin{equation}\label{Eq.2}
f_{i+1}\left(i\right)=r_{i}\mu_{i},
\end{equation}

Demostrar que

\begin{eqnarray*}
f_{i}\left(i\right)&=&\mu_{i}\left[\sum_{k=1}^{N}r_{k}+\sum_{k=1,k\neq
i}^{N}\frac{f_{k}\left(k\right)}{1-\mu_{k}}\right].
\end{eqnarray*}


En la Ecuaci\'on (\ref{Eq.2}) hagamos $j=i+1$, entonces se tiene
$f_{j}=r_{i}\mu_{i}$, lo mismo para (\ref{Eq.1})

\begin{eqnarray*}
f_{i}\left(i\right)&=&r_{i}\mu_{i}+\mu_{i}\left[\sum_{k=j}^{i-1}r_{k}+\sum_{k=j}^{i-1}\frac{f_{k}\left(k\right)}{1-\mu_{k}}\right]\\
&=&\mu_{i}\left[\sum_{k=j}^{i}r_{k}+\sum_{k=j}^{i-1}\frac{f_{k}\left(k\right)}{1-\mu_{k}}\right]\\
\end{eqnarray*}

entonces, tomando sobre todo valor de $1,\ldots,N$, tanto para
antes de $i$ como para despu\'es de $i$, entonces

\begin{eqnarray*}
f_{i}\left(i\right)&=&\mu_{i}\left[\sum_{k=1}^{N}r_{k}+\sum_{k=1,k\neq
i}^{N}\frac{f_{k}\left(k\right)}{1-\mu_{k}}\right].
\end{eqnarray*}
\end{Prop}


Ahora, supongamos nuevamente la ecuaci\'on (\ref{Eq.1})

\begin{eqnarray*}
f_{i}\left(i\right)-f_{j}\left(i\right)&=&\mu_{i}\left[\sum_{k=j}^{i-1}r_{k}+\sum_{k=j}^{i-1}\frac{f_{k}\left(k\right)}{1-\mu_{k}}\right]\\
&\Leftrightarrow&\\
f_{j}\left(j\right)-f_{i}\left(j\right)&=&\mu_{j}\left[\sum_{k=i}^{j-1}r_{k}+\sum_{k=i}^{j-1}\frac{f_{k}\left(k\right)}{1-\mu_{k}}\right]\\
f_{i}\left(j\right)&=&f_{j}\left(j\right)-\mu_{j}\left[\sum_{k=i}^{j-1}r_{k}+\sum_{k=i}^{j-1}\frac{f_{k}\left(k\right)}{1-\mu_{k}}\right]\\
&=&\mu_{j}\left(1-\mu_{j}\right)\frac{r}{1-\mu}-\mu_{j}\left[\sum_{k=i}^{j-1}r_{k}+\sum_{k=i}^{j-1}\frac{f_{k}\left(k\right)}{1-\mu_{k}}\right]\\
&=&\mu_{j}\left[\left(1-\mu_{j}\right)\frac{r}{1-\mu}-\sum_{k=i}^{j-1}r_{k}-\sum_{k=i}^{j-1}\frac{f_{k}\left(k\right)}{1-\mu_{k}}\right]\\
&=&\mu_{j}\left[\left(1-\mu_{j}\right)\frac{r}{1-\mu}-\sum_{k=i}^{j-1}r_{k}-\frac{r}{1-\mu}\sum_{k=i}^{j-1}\mu_{k}\right]\\
&=&\mu_{j}\left[\frac{r}{1-\mu}\left(1-\mu_{j}-\sum_{k=i}^{j-1}\mu_{k}\right)-\sum_{k=i}^{j-1}r_{k}\right]\\
&=&\mu_{j}\left[\frac{r}{1-\mu}\left(1-\sum_{k=i}^{j}\mu_{k}\right)-\sum_{k=i}^{j-1}r_{k}\right].\\
\end{eqnarray*}

Ahora,

\begin{eqnarray*}
1-\sum_{k=i}^{j}\mu_{k}&=&1-\sum_{k=1}^{N}\mu_{k}+\sum_{k=j+1}^{i-1}\mu_{k}\\
&\Leftrightarrow&\\
\sum_{k=i}^{j}\mu_{k}&=&\sum_{k=1}^{N}\mu_{k}-\sum_{k=j+1}^{i-1}\mu_{k}\\
&\Leftrightarrow&\\
\sum_{k=1}^{N}\mu_{k}&=&\sum_{k=i}^{j}\mu_{k}+\sum_{k=j+1}^{i-1}\mu_{k}\\
\end{eqnarray*}

Por tanto
\begin{eqnarray*}
f_{i}\left(j\right)&=&\mu_{j}\left[\frac{r}{1-\mu}\sum_{k=j+1}^{i-1}\mu_{k}+\sum_{k=j}^{i-1}r_{k}\right].
\end{eqnarray*}





\begin{Teo}[Teorema de Continuidad]
Sup\'ongase que $\left\{X_{n},n=1,2,3,\ldots\right\}$ son
variables aleatorias finitas, no negativas con valores enteros
tales que $P\left(X_{n}=k\right)=p_{k}^{(n)}$, para
$n=1,2,3,\ldots$, $k=0,1,2,\ldots$, con
$\sum_{k=0}^{\infty}p_{k}^{(n)}=1$, para $n=1,2,3,\ldots$. Sea
$g_{n}$ la PGF para la variable aleatoria $X_{n}$. Entonces existe
una sucesi\'on $\left\{p_{k}\right\}$ tal que \begin{eqnarray*}
lim_{n\rightarrow\infty}p_{k}^{(n)}=p_{k}\textrm{ para }0<s<1.
\end{eqnarray*}
En este caso, $g\left(s\right)=\sum_{k=0}^{\infty}s^{k}p_{k}$.
Adem\'as
\begin{eqnarray*}
\sum_{k=0}^{\infty}p_{k}=1\textrm{ si y s\'olo si
}lim_{s\uparrow1}g\left(s\right)=1
\end{eqnarray*}
\end{Teo}

\begin{Teo}
Sea $N$ una variable aleatoria con valores enteros no negativos
finita tal que $P\left(N=k\right)=p_{k}$, para $k=0,1,2,\ldots$, y
$\sum_{k=0}^{\infty}p_{k}=P\left(N<\infty\right)=1$. Sea $\Phi$ la
PGF de $N$ tal que
$g\left(s\right)=\esp\left[s^{N}\right]=\sum_{k=0}^{\infty}s^{k}p_{k}$
con $g\left(1\right)=1$. Si $0\leq p_{1}\leq1$ y
$\esp\left[N\right]=g^{'}\left(1\right)\leq1$, entonces no existe
soluci\'on  de la ecuaci\'on $g\left(s\right)=s$ en el intervalo
$\left[0,1\right)$. Si $\esp\left[N\right]=g^{'}\left(1\right)>1$,
lo cual implica que $0\leq p_{1}<1$, entonces existe una \'unica
soluci\'on de la ecuaci\'on $g\left(s\right)=s$ en el intervalo
$\left[0,1\right)$.
\end{Teo}


\begin{Teo}
Si $X$ y $Y$ tienen PGF $G_{X}$ y $G_{Y}$ respectivamente,
entonces,\[G_{X}\left(s\right)=G_{Y}\left(s\right)\] para toda
$s$, si y s\'olo si \[P\left(X=k\right))=P\left(Y=k\right)\] para
toda $k=0,1,\ldots,$., es decir, si y s\'olo si $X$ y $Y$ tienen
la misma distribuci\'on de probabilidad.
\end{Teo}


\begin{Teo}
Para cada $n$ fijo, sea la sucesi\'oin de probabilidades
$\left\{a_{0,n},a_{1,n},\ldots,\right\}$, tales que $a_{k,n}\geq0$
para toda $k=0,1,2,\ldots,$ y $\sum_{k\geq0}a_{k,n}=1$, y sea
$G_{n}\left(s\right)$ la correspondiente funci\'on generadora,
$G_{n}\left(s\right)=\sum_{k\geq0}a_{k,n}s^{k}$. De modo que para
cada valor fijo de $k$
\begin{eqnarray*}
lim_{n\rightarrow\infty}a_{k,n}=a_{k},
\end{eqnarray*}
es decir converge en distribuci\'on, es necesario y suficiente que
para cada valor fijo $s\in\left[0,\right)$,
\begin{eqnarray*}
lim_{n\rightarrow\infty}G_{n}\left(s\right)=G\left(s\right),
\end{eqnarray*}
donde $G\left(s\right)=\sum_{k\geq0}p_{k}s^{k}$, para cualquier

la funci\'on generadora del l\'imite de la sucesi\'on.
\end{Teo}

\begin{Teo}[Teorema de Abel]
Sea $G\left(s\right)=\sum_{k\geq0}a_{k}s^{k}$ para cualquier
$\left\{p_{0},p_{1},\ldots,\right\}$, tales que $p_{k}\geq0$ para
toda $k=0,1,2,\ldots,$. Entonces $G\left(s\right)$ es continua por
la derecha en $s=1$, es decir
\begin{eqnarray*}
lim_{s\uparrow1}G\left(s\right)=\sum_{k\geq0}p_{k}=G\left(\right),
\end{eqnarray*}
sin importar si la suma es finita o no.
\end{Teo}
\begin{Note}
El radio de Convergencia para cualquier PGF es $R\geq1$, entonces,
el Teorema de Abel nos dice que a\'un en el peor escenario, cuando
$R=1$, a\'un se puede confiar en que la PGF ser\'a continua en
$s=1$, en contraste, no se puede asegurar que la PGF ser\'a
continua en el l\'imite inferior $-R$, puesto que la PGF es
sim\'etrica alrededor del cero: la PGF converge para todo
$s\in\left(-R,R\right)$, y no lo hace para $s<-R$ o $s>R$.
Adem\'as nos dice que podemos escribir $G_{X}\left(1\right)$ como
una abreviaci\'on de $lim_{s\uparrow1}G_{X}\left(s\right)$.
\end{Note}

Entonces si suponemos que la diferenciaci\'on t\'ermino a
t\'ermino est\'a permitida, entonces

\begin{eqnarray*}
G_{X}^{'}\left(s\right)&=&\sum_{x=1}^{\infty}xs^{x-1}p_{x}
\end{eqnarray*}

el Teorema de Abel nos dice que
\begin{eqnarray*}
\esp\left(X\right]&=&\lim_{s\uparrow1}G_{X}^{'}\left(s\right):\\
\esp\left[X\right]&=&=\sum_{x=1}^{\infty}xp_{x}=G_{X}^{'}\left(1\right)\\
&=&\lim_{s\uparrow1}G_{X}^{'}\left(s\right),
\end{eqnarray*}
dado que el Teorema de Abel se aplica a
\begin{eqnarray*}
G_{X}^{'}\left(s\right)&=&\sum_{x=1}^{\infty}xs^{x-1}p_{x},
\end{eqnarray*}
estableciendo as\'i que $G_{X}^{'}\left(s\right)$ es continua en
$s=1$. Sin el Teorema de Abel no se podr\'ia asegurar que el
l\'imite de $G_{X}^{'}\left(s\right)$ conforme $s\uparrow1$ sea la
respuesta correcta para $\esp\left[X\right]$.

\begin{Note}
La PGF converge para todo $|s|<R$, para alg\'un $R$. De hecho la
PGF converge absolutamente si $|s|<R$. La PGF adem\'as converge
uniformemente en conjuntos de la forma
$\left\{s:|s|<R^{'}\right\}$, donde $R^{'}<R$, es decir,
$\forall\epsilon>0, \exists n_{0}\in\ent$ tal que $\forall s$, con
$|s|<R^{'}$, y $\forall n\geq n_{0}$,
\begin{eqnarray*}
|\sum_{x=0}^{n}s^{x}\prob\left(X=x\right)-G_{X}\left(s\right)|<\epsilon.
\end{eqnarray*}
De hecho, la convergencia uniforme es la que nos permite
diferenciar t\'ermino a t\'ermino:
\begin{eqnarray*}
G_{X}\left(s\right)=\esp\left[s^{X}\right]=\sum_{x=0}^{\infty}s^{x}\prob\left(X=x\right),
\end{eqnarray*}
y sea $s<R$.
\begin{enumerate}
\item
\begin{eqnarray*}
G_{X}^{'}\left(s\right)&=&\frac{d}{ds}\left(\sum_{x=0}^{\infty}s^{x}\prob\left(X=x\right)\right)=\sum_{x=0}^{\infty}\frac{d}{ds}\left(s^{x}\prob\left(X=x\right)\right)\\
&=&\sum_{x=0}^{n}xs^{x-1}\prob\left(X=x\right).
\end{eqnarray*}

\item\begin{eqnarray*}
\int_{a}^{b}G_{X}\left(s\right)ds&=&\int_{a}^{b}\left(\sum_{x=0}^{\infty}s^{x}\prob\left(X=x\right)\right)ds=\sum_{x=0}^{\infty}\left(\int_{a}^{b}s^{x}\prob\left(X=x\right)ds\right)\\
&=&\sum_{x=0}^{\infty}\frac{s^{x+1}}{x+1}\prob\left(X=x\right),
\end{eqnarray*}
para $-R<a<b<R$.
\end{enumerate}
\end{Note}

\begin{Teo}[Teorema de Convergencia Mon\'otona para PGF]
Sean $X$ y $X_{n}$ variables aleatorias no negativas, con valores
en los enteros, finitas, tales que
\begin{eqnarray*}
lim_{n\rightarrow\infty}G_{X_{n}}\left(s\right)&=&G_{X}\left(s\right)
\end{eqnarray*}
para $0\leq s\leq1$, entonces
\begin{eqnarray*}
lim_{n\rightarrow\infty}P\left(X_{n}=k\right)=P\left(X=k\right),
\end{eqnarray*}
para $k=0,1,2,\ldots.$
\end{Teo}

El teorema anterior requiere del siguiente lema

\begin{Lemma}
Sean $a_{n,k}\in\ent^{+}$, $n\in\nat$ constantes no negativas con
$\sum_{k\geq0}a_{k,n}\leq1$. Sup\'ongase que para $0\leq s\leq1$,
se tiene

\begin{eqnarray*}
a_{n}\left(s\right)&=&\sum_{k=0}^{\infty}a_{k,n}s^{k}\rightarrow
a\left(s\right)=\sum_{k=0}^{\infty}a_{k}s^{k}.
\end{eqnarray*}
Entonces
\begin{eqnarray*}
a_{0,n}\rightarrow a_{0}.
\end{eqnarray*}
\end{Lemma}


%________________________________________________________
\section{Funciones Generadoras de Probabilidad Conjunta}
%________________________________________________________


De lo desarrollado hasta ahora se tiene lo siguiente

\begin{eqnarray*}
&&\esp\left[z_{1}^{L_{1}\left(\overline{\tau}_{1}\right)}z_{2}^{L_{2}\left(\overline{\tau}_{1}\right)}\right]=\esp\left[z_{2}^{L_{2}\left(\overline{\tau}_{1}\right)}\right]=\esp\left[z_{2}^{L_{2}\left(\tau_{1}\right)+X_{2}\left(\overline{\tau}_{1}-\tau_{1}\right)}\right]\\
&=&\esp\left[\left\{z_{2}^{L_{2}\left(\tau_{1}\right)}\right\}\left\{z_{2}^{X_{2}\left(\overline{\tau}_{1}-\tau_{1}\right)}\right\}\right]=\esp\left[\left\{z_{2}^{L_{2}\left(\tau_{1}\right)}\right\}\left\{P_{2}\left(z_{2}\right)\right\}^{\overline{\tau}_{1}-\tau_{1}}\right]\\
&=&\esp\left[\left\{z_{2}^{L_{2}\left(\tau_{1}\right)}\right\}\left\{\theta_{1}\left(P_{2}\left(z_{2}\right)\right)\right\}^{L_{1}\left(\tau_{1}\right)}\right]=F_{1}\left(\theta_{1}\left(P_{2}\left(z_{2}\right)\right),z_{2}\right)
\end{eqnarray*}

es decir %{{\tiny
\begin{equation}\label{Eq.base.F1}
\esp\left[z_{1}^{L_{1}\left(\overline{\tau}_{1}\right)}z_{2}^{L_{2}\left(\overline{\tau}_{1}\right)}\right]=F_{1}\left(\theta_{1}\left(P_{2}\left(z_{2}\right)\right),z_{2}\right).
\end{equation}

Procediendo de manera an\'aloga para $\overline{\tau}_{2}$:

\begin{eqnarray*}
\esp\left[z_{1}^{L_{1}\left(\overline{\tau}_{2}\right)}z_{2}^{L_{2}\left(\overline{\tau}_{2}\right)}\right]&=&\esp\left[z_{1}^{L_{1}\left(\overline{\tau}_{2}\right)}\right]=\esp\left[z_{1}^{L_{1}\left(\tau_{2}\right)+X_{1}\left(\overline{\tau}_{2}-\tau_{2}\right)}\right]=\esp\left[\left\{z_{1}^{L_{1}\left(\tau_{2}\right)}\right\}\left\{z_{1}^{X_{1}\left(\overline{\tau}_{2}-\tau_{2}\right)}\right\}\right]\\
&=&\esp\left[\left\{z_{1}^{L_{1}\left(\tau_{2}\right)}\right\}\left\{P_{1}\left(z_{1}\right)\right\}^{\overline{\tau}_{2}-\tau_{2}}\right]=\esp\left[\left\{z_{1}^{L_{1}\left(\tau_{2}\right)}\right\}\left\{\theta_{2}\left(P_{1}\left(z_{1}\right)\right)\right\}^{L_{2}\left(\tau_{2}\right)}\right]\\
&=&F_{2}\left(z_{1},\theta_{2}\left(P_{1}\left(z_{1}\right)\right)\right)
\end{eqnarray*}%}}


\begin{equation}\label{Eq.PGF.Conjunta.Tau2}
\esp\left[z_{1}^{L_{1}\left(\overline{\tau}_{2}\right)}z_{2}^{L_{2}\left(\overline{\tau}_{2}\right)}\right]=F_{2}\left(z_{1},\theta_{2}\left(P_{1}\left(z_{1}\right)\right)\right)
\end{equation}%}

Ahora, para el intervalo de tiempo
$\left[\overline{\tau}_{1},\tau_{2}\right]$ y
$\left[\overline{\tau}_{2},\tau_{1}\right]$, los arribos de los
usuarios modifican el n\'umero de usuarios que llegan a las colas,
es decir, los procesos
$L_{1}\left(t\right)$
y $L_{2}\left(t\right)$. La PGF para el n\'umero de arribos
a todas las estaciones durante el intervalo
$\left[\overline{\tau}_{1},\tau_{2}\right]$  cuya distribuci\'on
est\'a especificada por la distribuci\'on compuesta
$R_{1}\left(\mathbf{z}\right),R_{2}\left(\mathbf{z}\right)$:

\begin{eqnarray*}
R_{1}\left(\mathbf{z}\right)=R_{1}\left(\prod_{i=1}^{2}P\left(z_{i}\right)\right)=\esp\left[\left\{\prod_{i=1}^{2}P\left(z_{i}\right)\right\}^{\tau_{2}-\overline{\tau}_{1}}\right]\\
R_{2}\left(\mathbf{z}\right)=R_{2}\left(\prod_{i=1}^{2}P\left(z_{i}\right)\right)=\esp\left[\left\{\prod_{i=1}^{2}P\left(z_{i}\right)\right\}^{\tau_{1}-\overline{\tau}_{2}}\right]\\
\end{eqnarray*}


Dado que los eventos en
$\left[\tau_{1},\overline{\tau}_{1}\right]$ y
$\left[\overline{\tau}_{1},\tau_{2}\right]$ son independientes, la
PGF conjunta para el n\'umero de usuarios en el sistema al tiempo
$t=\tau_{2}$ la PGF conjunta para el n\'umero de usuarios en el sistema est\'an dadas por

{\footnotesize{
\begin{eqnarray*}
F_{1}\left(\mathbf{z}\right)&=&R_{2}\left(\prod_{i=1}^{2}P\left(z_{i}\right)\right)F_{2}\left(z_{1},\theta_{2}\left(P_{1}\left(z_{1}\right)\right)\right)\\
F_{2}\left(\mathbf{z}\right)&=&R_{1}\left(\prod_{i=1}^{2}P\left(z_{i}\right)\right)F_{1}\left(\theta_{1}\left(P_{2}\left(z_{2}\right)\right),z_{2}\right)\\
\end{eqnarray*}}}


Entonces debemos de determinar las siguientes expresiones:


\begin{eqnarray*}
\begin{array}{cc}
f_{1}\left(1\right)=\frac{\partial F_{1}\left(\mathbf{z}\right)}{\partial z_{1}}|_{\mathbf{z}=1}, & f_{1}\left(2\right)=\frac{\partial F_{1}\left(\mathbf{z}\right)}{\partial z_{2}}|_{\mathbf{z}=1},\\
f_{2}\left(1\right)=\frac{\partial F_{2}\left(\mathbf{z}\right)}{\partial z_{1}}|_{\mathbf{z}=1}, & f_{2}\left(2\right)=\frac{\partial F_{2}\left(\mathbf{z}\right)}{\partial z_{2}}|_{\mathbf{z}=1},\\
\end{array}
\end{eqnarray*}


\begin{eqnarray*}
\frac{\partial R_{1}\left(\mathbf{z}\right)}{\partial
z_{1}}|_{\mathbf{z}=1}&=&R_{1}^{(1)}\left(1\right)P_{1}^{(1)}\left(1\right)\\
\frac{\partial R_{1}\left(\mathbf{z}\right)}{\partial
z_{2}}|_{\mathbf{z}=1}&=&R_{1}^{(1)}\left(1\right)P_{2}^{(1)}\left(1\right)\\
\frac{\partial R_{2}\left(\mathbf{z}\right)}{\partial
z_{1}}|_{\mathbf{z}=1}&=&R_{2}^{(1)}\left(1\right)P_{1}^{(1)}\left(1\right)\\
\frac{\partial R_{2}\left(\mathbf{z}\right)}{\partial
z_{2}}|_{\mathbf{z}=1}&=&R_{2}^{(1)}\left(1\right)P_{2}^{(1)}\left(1\right)\\
\end{eqnarray*}



\begin{eqnarray*}
\frac{\partial}{\partial
z_{1}}F_{1}\left(\theta_{1}\left(P_{2}\left(z_{2}\right)\right),z_{2}\right)&=&0\\
\frac{\partial}{\partial
z_{2}}F_{1}\left(\theta_{1}\left(P_{2}\left(z_{2}\right)\right),z_{2}\right)&=&\frac{\partial
F_{1}}{\partial z_{2}}+\frac{\partial F_{1}}{\partial
z_{1}}\theta_{1}^{(1)}P_{2}^{(1)}\left(1\right)\\
\frac{\partial}{\partial
z_{1}}F_{2}\left(z_{1},\theta_{2}\left(P_{1}\left(z_{1}\right)\right)\right)&=&\frac{\partial
F_{2}}{\partial z_{1}}+\frac{\partial F_{2}}{\partial
z_{2}}\theta_{2}^{(1)}P_{1}^{(1)}\left(1\right)\\
\frac{\partial}{\partial
z_{2}}F_{2}\left(z_{1},\theta_{2}\left(P_{1}\left(z_{1}\right)\right)\right)&=&0\\
\end{eqnarray*}


Por lo tanto de las dos secciones anteriores se tiene que:


\begin{eqnarray*}
\frac{\partial F_{1}}{\partial z_{1}}&=&\frac{\partial
R_{2}}{\partial z_{1}}|_{\mathbf{z}=1}+\frac{\partial F_{2}}{\partial z_{1}}|_{\mathbf{z}=1}=R_{2}^{(1)}\left(1\right)P_{1}^{(1)}\left(1\right)+f_{2}\left(1\right)+f_{2}\left(2\right)\theta_{2}^{(1)}\left(1\right)P_{1}^{(1)}\left(1\right)\\
\frac{\partial F_{1}}{\partial z_{2}}&=&\frac{\partial
R_{2}}{\partial z_{2}}|_{\mathbf{z}=1}+\frac{\partial F_{2}}{\partial z_{2}}|_{\mathbf{z}=1}=R_{2}^{(1)}\left(1\right)P_{2}^{(1)}\left(1\right)\\
\frac{\partial F_{2}}{\partial z_{1}}&=&\frac{\partial
R_{1}}{\partial z_{1}}|_{\mathbf{z}=1}+\frac{\partial F_{1}}{\partial z_{1}}|_{\mathbf{z}=1}=R_{1}^{(1)}\left(1\right)P_{1}^{(1)}\left(1\right)\\
\frac{\partial F_{2}}{\partial z_{2}}&=&\frac{\partial
R_{1}}{\partial z_{2}}|_{\mathbf{z}=1}+\frac{\partial F_{1}}{\partial z_{2}}|_{\mathbf{z}=1}
=R_{1}^{(1)}\left(1\right)P_{2}^{(1)}\left(1\right)+f_{1}\left(1\right)\theta_{1}^{(1)}\left(1\right)P_{2}^{(1)}\left(1\right)\\
\end{eqnarray*}


El cual se puede escribir en forma equivalente:
\begin{eqnarray*}
f_{1}\left(1\right)&=&r_{2}\mu_{1}+f_{2}\left(1\right)+f_{2}\left(2\right)\frac{\mu_{1}}{1-\mu_{2}}\\
f_{1}\left(2\right)&=&r_{2}\mu_{2}\\
f_{2}\left(1\right)&=&r_{1}\mu_{1}\\
f_{2}\left(2\right)&=&r_{1}\mu_{2}+f_{1}\left(2\right)+f_{1}\left(1\right)\frac{\mu_{2}}{1-\mu_{1}}\\
\end{eqnarray*}

De donde:
\begin{eqnarray*}
f_{1}\left(1\right)&=&\mu_{1}\left[r_{2}+\frac{f_{2}\left(2\right)}{1-\mu_{2}}\right]+f_{2}\left(1\right)\\
f_{2}\left(2\right)&=&\mu_{2}\left[r_{1}+\frac{f_{1}\left(1\right)}{1-\mu_{1}}\right]+f_{1}\left(2\right)\\
\end{eqnarray*}

Resolviendo para $f_{1}\left(1\right)$:
\begin{eqnarray*}
f_{1}\left(1\right)&=&r_{2}\mu_{1}+f_{2}\left(1\right)+f_{2}\left(2\right)\frac{\mu_{1}}{1-\mu_{2}}=r_{2}\mu_{1}+r_{1}\mu_{1}+f_{2}\left(2\right)\frac{\mu_{1}}{1-\mu_{2}}\\
&=&\mu_{1}\left(r_{2}+r_{1}\right)+f_{2}\left(2\right)\frac{\mu_{1}}{1-\mu_{2}}=\mu_{1}\left(r+\frac{f_{2}\left(2\right)}{1-\mu_{2}}\right),\\
\end{eqnarray*}

entonces

\begin{eqnarray*}
f_{2}\left(2\right)&=&\mu_{2}\left(r_{1}+\frac{f_{1}\left(1\right)}{1-\mu_{1}}\right)+f_{1}\left(2\right)=\mu_{2}\left(r_{1}+\frac{f_{1}\left(1\right)}{1-\mu_{1}}\right)+r_{2}\mu_{2}\\
&=&\mu_{2}\left[r_{1}+r_{2}+\frac{f_{1}\left(1\right)}{1-\mu_{1}}\right]=\mu_{2}\left[r+\frac{f_{1}\left(1\right)}{1-\mu_{1}}\right]\\
&=&\mu_{2}r+\mu_{1}\left(r+\frac{f_{2}\left(2\right)}{1-\mu_{2}}\right)\frac{\mu_{2}}{1-\mu_{1}}\\
&=&\mu_{2}r+\mu_{2}\frac{r\mu_{1}}{1-\mu_{1}}+f_{2}\left(2\right)\frac{\mu_{1}\mu_{2}}{\left(1-\mu_{1}\right)\left(1-\mu_{2}\right)}\\
&=&\mu_{2}\left(r+\frac{r\mu_{1}}{1-\mu_{1}}\right)+f_{2}\left(2\right)\frac{\mu_{1}\mu_{2}}{\left(1-\mu_{1}\right)\left(1-\mu_{2}\right)}\\
&=&\mu_{2}\left(\frac{r}{1-\mu_{1}}\right)+f_{2}\left(2\right)\frac{\mu_{1}\mu_{2}}{\left(1-\mu_{1}\right)\left(1-\mu_{2}\right)}\\
\end{eqnarray*}
entonces
\begin{eqnarray*}
f_{2}\left(2\right)-f_{2}\left(2\right)\frac{\mu_{1}\mu_{2}}{\left(1-\mu_{1}\right)\left(1-\mu_{2}\right)}&=&\mu_{2}\left(\frac{r}{1-\mu_{1}}\right)\\
f_{2}\left(2\right)\left(1-\frac{\mu_{1}\mu_{2}}{\left(1-\mu_{1}\right)\left(1-\mu_{2}\right)}\right)&=&\mu_{2}\left(\frac{r}{1-\mu_{1}}\right)\\
f_{2}\left(2\right)\left(\frac{1-\mu_{1}-\mu_{2}+\mu_{1}\mu_{2}-\mu_{1}\mu_{2}}{\left(1-\mu_{1}\right)\left(1-\mu_{2}\right)}\right)&=&\mu_{2}\left(\frac{r}{1-\mu_{1}}\right)\\
f_{2}\left(2\right)\left(\frac{1-\mu}{\left(1-\mu_{1}\right)\left(1-\mu_{2}\right)}\right)&=&\mu_{2}\left(\frac{r}{1-\mu_{1}}\right)\\
\end{eqnarray*}
por tanto
\begin{eqnarray*}
f_{2}\left(2\right)&=&\frac{r\frac{\mu_{2}}{1-\mu_{1}}}{\frac{1-\mu}{\left(1-\mu_{1}\right)\left(1-\mu_{2}\right)}}=\frac{r\mu_{2}\left(1-\mu_{1}\right)\left(1-\mu_{2}\right)}{\left(1-\mu_{1}\right)\left(1-\mu\right)}\\
&=&\frac{\mu_{2}\left(1-\mu_{2}\right)}{1-\mu}r=r\mu_{2}\frac{1-\mu_{2}}{1-\mu}.
\end{eqnarray*}
es decir

\begin{eqnarray}
f_{2}\left(2\right)&=&r\mu_{2}\frac{1-\mu_{2}}{1-\mu}.
\end{eqnarray}

Entonces

\begin{eqnarray*}
f_{1}\left(1\right)&=&\mu_{1}r+f_{2}\left(2\right)\frac{\mu_{1}}{1-\mu_{2}}=\mu_{1}r+\left(\frac{\mu_{2}\left(1-\mu_{2}\right)}{1-\mu}r\right)\frac{\mu_{1}}{1-\mu_{2}}\\
&=&\mu_{1}r+\mu_{1}r\left(\frac{\mu_{2}}{1-\mu}\right)=\mu_{1}r\left[1+\frac{\mu_{2}}{1-\mu}\right]\\
&=&r\mu_{1}\frac{1-\mu_{1}}{1-\mu}\\
\end{eqnarray*}




%-------------- CAPITULO YA INCLUIDO EN OTRO ----
%\chapter{Funciones Generadoras de Probabilidades}
%
%___________________________________________________________________________________________
%
\section{Funciones Generadoras de Probabilidades}
%___________________________________________________________________________________________

\begin{Teo}[Teorema de Continuidad]
Sup\'ongase que $\left\{X_{n},n=1,2,3,\ldots\right\}$ son
variables aleatorias finitas, no negativas con valores enteros
tales que $P\left(X_{n}=k\right)=p_{k}^{(n)}$, para
$n=1,2,3,\ldots$, $k=0,1,2,\ldots$, con
$\sum_{k=0}^{\infty}p_{k}^{(n)}=1$, para $n=1,2,3,\ldots$. Sea
$g_{n}$ la PGF para la variable aleatoria $X_{n}$. Entonces existe
una sucesi\'on $\left\{p_{k}\right\}$ tal que
\begin{eqnarray*}
lim_{n\rightarrow\infty}p_{k}^{(n)}=p_{k}\textrm{ para }0<s<1.
\end{eqnarray*}
En este caso, $g\left(s\right)=\sum_{k=0}^{\infty}s^{k}p_{k}$.
Adem\'as
\begin{eqnarray*}
\sum_{k=0}^{\infty}p_{k}=1\textrm{ si y s\'olo si
}lim_{s\uparrow1}g\left(s\right)=1
\end{eqnarray*}
\end{Teo}

\begin{Teo}
Sea $N$ una variable aleatoria con valores enteros no negativos
finita tal que $P\left(N=k\right)=p_{k}$, para $k=0,1,2,\ldots$, y
$\sum_{k=0}^{\infty}p_{k}=P\left(N<\infty\right)=1$. Sea $\Phi$ la
PGF de $N$ tal que
$g\left(s\right)=\esp\left[s^{N}\right]=\sum_{k=0}^{\infty}s^{k}p_{k}$
con $g\left(1\right)=1$. Si $0\leq p_{1}\leq1$ y
$\esp\left[N\right]=g^{'}\left(1\right)\leq1$, entonces no existe
soluci\'on  de la ecuaci\'on $g\left(s\right)=s$ en el intervalo
$\left[0,1\right)$. Si $\esp\left[N\right]=g^{'}\left(1\right)>1$,
lo cual implica que $0\leq p_{1}<1$, entonces existe una \'unica
soluci\'on de la ecuaci\'on $g\left(s\right)=s$ en el intervalo
$\left[0,1\right)$.
\end{Teo}


\begin{Teo}
Si $X$ y $Y$ tienen PGF $G_{X}$ y $G_{Y}$ respectivamente,
entonces,\[G_{X}\left(s\right)=G_{Y}\left(s\right)\] para toda
$s$, si y s\'olo si \[P\left(X=k\right))=P\left(Y=k\right)\] para
toda $k=0,1,\ldots,$., es decir, si y s\'olo si $X$ y $Y$ tienen
la misma distribuci\'on de probabilidad.
\end{Teo}


\begin{Teo}
Para cada $n$ fijo, sea la sucesi\'oin de probabilidades
$\left\{a_{0,n},a_{1,n},\ldots,\right\}$, tales que $a_{k,n}\geq0$
para toda $k=0,1,2,\ldots,$ y $\sum_{k\geq0}a_{k,n}=1$, y sea
$G_{n}\left(s\right)$ la correspondiente funci\'on generadora,
$G_{n}\left(s\right)=\sum_{k\geq0}a_{k,n}s^{k}$. De modo que para
cada valor fijo de $k$
\begin{eqnarray*}
lim_{n\rightarrow\infty}a_{k,n}=a_{k},
\end{eqnarray*}
es decir converge en distribuci\'on, es necesario y suficiente que
para cada valor fijo $s\in\left[0,\right)$,
\begin{eqnarray*}
lim_{n\rightarrow\infty}G_{n}\left(s\right)=G\left(s\right),
\end{eqnarray*}
donde $G\left(s\right)=\sum_{k\geq0}p_{k}s^{k}$, para cualquier

la funci\'on generadora del l\'imite de la sucesi\'on.
\end{Teo}

\begin{Teo}[Teorema de Abel]
Sea $G\left(s\right)=\sum_{k\geq0}a_{k}s^{k}$ para cualquier
$\left\{p_{0},p_{1},\ldots,\right\}$, tales que $p_{k}\geq0$ para
toda $k=0,1,2,\ldots,$. Entonces $G\left(s\right)$ es continua por
la derecha en $s=1$, es decir
\begin{eqnarray*}
lim_{s\uparrow1}G\left(s\right)=\sum_{k\geq0}p_{k}=G\left(\right),
\end{eqnarray*}
sin importar si la suma es finita o no.
\end{Teo}
\begin{Note}
El radio de Convergencia para cualquier PGF es $R\geq1$, entonces,
el Teorema de Abel nos dice que a\'un en el peor escenario, cuando
$R=1$, a\'un se puede confiar en que la PGF ser\'a continua en
$s=1$, en contraste, no se puede asegurar que la PGF ser\'a
continua en el l\'imite inferior $-R$, puesto que la PGF es
sim\'etrica alrededor del cero: la PGF converge para todo
$s\in\left(-R,R\right)$, y no lo hace para $s<-R$ o $s>R$.
Adem\'as nos dice que podemos escribir $G_{X}\left(1\right)$ como
una abreviaci\'on de $lim_{s\uparrow1}G_{X}\left(s\right)$.
\end{Note}

Entonces si suponemos que la diferenciaci\'on t\'ermino a
t\'ermino est\'a permitida, entonces

\begin{eqnarray*}
G_{X}^{'}\left(s\right)&=&\sum_{x=1}^{\infty}xs^{x-1}p_{x}
\end{eqnarray*}

el Teorema de Abel nos dice que
\begin{eqnarray*}
\esp\left(X\right]&=&\lim_{s\uparrow1}G_{X}^{'}\left(s\right):\\
\esp\left[X\right]&=&=\sum_{x=1}^{\infty}xp_{x}=G_{X}^{'}\left(1\right)\\
&=&\lim_{s\uparrow1}G_{X}^{'}\left(s\right),
\end{eqnarray*}
dado que el Teorema de Abel se aplica a
\begin{eqnarray*}
G_{X}^{'}\left(s\right)&=&\sum_{x=1}^{\infty}xs^{x-1}p_{x},
\end{eqnarray*}
estableciendo as\'i que $G_{X}^{'}\left(s\right)$ es continua en
$s=1$. Sin el Teorema de Abel no se podr\'ia asegurar que el
l\'imite de $G_{X}^{'}\left(s\right)$ conforme $s\uparrow1$ sea la
respuesta correcta para $\esp\left[X\right]$.

\begin{Note}
La PGF converge para todo $|s|<R$, para alg\'un $R$. De hecho la
PGF converge absolutamente si $|s|<R$. La PGF adem\'as converge
uniformemente en conjuntos de la forma
$\left\{s:|s|<R^{'}\right\}$, donde $R^{'}<R$, es decir,
$\forall\epsilon>0, \exists n_{0}\in\ent$ tal que $\forall s$, con
$|s|<R^{'}$, y $\forall n\geq n_{0}$,
\begin{eqnarray*}
|\sum_{x=0}^{n}s^{x}\prob\left(X=x\right)-G_{X}\left(s\right)|<\epsilon.
\end{eqnarray*}
De hecho, la convergencia uniforme es la que nos permite
diferenciar t\'ermino a t\'ermino:
\begin{eqnarray*}
G_{X}\left(s\right)=\esp\left[s^{X}\right]=\sum_{x=0}^{\infty}s^{x}\prob\left(X=x\right),
\end{eqnarray*}
y sea $s<R$.
\begin{enumerate}
\item
\begin{eqnarray*}
G_{X}^{'}\left(s\right)&=&\frac{d}{ds}\left(\sum_{x=0}^{\infty}s^{x}\prob\left(X=x\right)\right)=\sum_{x=0}^{\infty}\frac{d}{ds}\left(s^{x}\prob\left(X=x\right)\right)\\
&=&\sum_{x=0}^{n}xs^{x-1}\prob\left(X=x\right).
\end{eqnarray*}

\item\begin{eqnarray*}
\int_{a}^{b}G_{X}\left(s\right)ds&=&\int_{a}^{b}\left(\sum_{x=0}^{\infty}s^{x}\prob\left(X=x\right)\right)ds=\sum_{x=0}^{\infty}\left(\int_{a}^{b}s^{x}\prob\left(X=x\right)ds\right)\\
&=&\sum_{x=0}^{\infty}\frac{s^{x+1}}{x+1}\prob\left(X=x\right),
\end{eqnarray*}
para $-R<a<b<R$.
\end{enumerate}
\end{Note}

\begin{Teo}[Teorema de Convergencia Mon\'otona para PGF]
Sean $X$ y $X_{n}$ variables aleatorias no negativas, con valores
en los enteros, finitas, tales que
\begin{eqnarray*}
lim_{n\rightarrow\infty}G_{X_{n}}\left(s\right)&=&G_{X}\left(s\right)
\end{eqnarray*}
para $0\leq s\leq1$, entonces
\begin{eqnarray*}
lim_{n\rightarrow\infty}P\left(X_{n}=k\right)=P\left(X=k\right),
\end{eqnarray*}
para $k=0,1,2,\ldots.$
\end{Teo}

El teorema anterior requiere del siguiente lema

\begin{Lemma}
Sean $a_{n,k}\in\ent^{+}$, $n\in\nat$ constantes no negativas con
$\sum_{k\geq0}a_{k,n}\leq1$. Sup\'ongase que para $0\leq s\leq1$,
se tiene

\begin{eqnarray*}
a_{n}\left(s\right)&=&\sum_{k=0}^{\infty}a_{k,n}s^{k}\rightarrow
a\left(s\right)=\sum_{k=0}^{\infty}a_{k}s^{k}.
\end{eqnarray*}
Entonces
\begin{eqnarray*}
a_{0,n}\rightarrow a_{0}.
\end{eqnarray*}
\end{Lemma}
%_________________________________________________________________________
%\section{El teorema de Rouche y las FGP}
%_________________________________________________________________________



%_________________________________________________________________________
\section{El problema de la ruina del jugador}
%_________________________________________________________________________

Supongamos que se tiene un jugador que cuenta con un capital inicial de $\tilde{L}_{0}\geq0$ unidades, esta persona realiza una serie de dos juegos simult\'aneos e independientes de manera sucesiva, dichos eventos son independientes e id\'enticos entre s\'i para cada realizaci\'on. Para $n\geq0$ fijo, la ganancia en el $n$-\'esimo juego es $\tilde{X}_{n}=X_{n}+Y_{n}$ unidades de las cuales se resta una cuota de 1 unidad por cada juego simult\'aneo, es decir, se restan dos unidades por cada juego realizado. En t\'erminos de la teor\'ia de colas puede pensarse como el n\'umero de usuarios que llegan a una cola v\'ia dos procesos de arribo distintos e independientes entre s\'i. Su Funci\'on Generadora de Probabilidades (FGP) est\'a dada por $F\left(z\right)=\esp\left[z^{\tilde{L}_{0}}\right]$ para $z\in\mathbb{C}$, adem\'as
$$\tilde{P}\left(z\right)=\esp\left[z^{\tilde{X}_{n}}\right]=\esp\left[z^{X_{n}+Y_{n}}\right]=\esp\left[z^{X_{n}}z^{Y_{n}}\right]=\esp\left[z^{X_{n}}\right]\esp\left[z^{Y_{n}}\right]=P\left(z\right)\check{P}\left(z\right),$$ con $\tilde{\mu}=\esp\left[\tilde{X}_{n}\right]=\tilde{P}\left[z\right]<1$. Sea $\tilde{L}_{n}$ el capital remanente despu\'es del $n$-\'esimo
juego. Entonces

$$\tilde{L}_{n}=\tilde{L}_{0}+\tilde{X}_{1}+\tilde{X}_{2}+\cdots+\tilde{X}_{n}-2n.$$

La ruina del jugador ocurre despu\'es del $n$-\'esimo juego, es decir, la cola se vac\'ia despu\'es del $n$-\'esimo juego, entonces sea $T$ definida como $T=min\left\{\tilde{L}_{n}=0\right\}$. Si $\tilde{L}_{0}=0$, entonces claramente $T=0$. En este sentido $T$ puede interpretarse como la longitud del periodo de tiempo que el servidor ocupa para dar servicio en la cola, comenzando con $\tilde{L}_{0}$ grupos de usuarios presentes en la cola, quienes arribaron conforme a un proceso dado por $\tilde{P}\left(z\right)$.

Sea $g_{n,k}$ la probabilidad del evento de que el jugador no caiga en ruina antes del $n$-\'esimo juego, y que adem\'as tenga un capital de $k$ unidades antes del $n$-\'esimo juego, es decir, dada $n\in\left\{1,2,\ldots\right\}$ y $k\in\left\{0,1,2,\ldots\right\}$
\begin{eqnarray*}
g_{n,k}:=P\left\{\tilde{L}_{j}>0, j=1,\ldots,n,
\tilde{L}_{n}=k\right\},
\end{eqnarray*}
la cual adem\'as se puede escribir como:
\begin{eqnarray*}
g_{n,k}&=&P\left\{\tilde{L}_{j}>0, j=1,\ldots,n,
\tilde{L}_{n}=k\right\}=\sum_{j=1}^{k+1}g_{n-1,j}P\left\{\tilde{X}_{n}=k-j+1\right\}\\
&=&\sum_{j=1}^{k+1}g_{n-1,j}P\left\{X_{n}+Y_{n}=k-j+1\right\}=\sum_{j=1}^{k+1}\sum_{l=1}^{j}g_{n-1,j}P\left\{X_{n}+Y_{n}=k-j+1,Y_{n}=l\right\}\\
&=&\sum_{j=1}^{k+1}\sum_{l=1}^{j}g_{n-1,j}P\left\{X_{n}+Y_{n}=k-j+1|Y_{n}=l\right\}P\left\{Y_{n}=l\right\}\\
&=&\sum_{j=1}^{k+1}\sum_{l=1}^{j}g_{n-1,j}P\left\{X_{n}=k-j-l+1\right\}P\left\{Y_{n}=l\right\},
\end{eqnarray*}

es decir
\begin{eqnarray}\label{Eq.Gnk.2S}
g_{n,k}=\sum_{j=1}^{k+1}\sum_{l=1}^{j}g_{n-1,j}P\left\{X_{n}=k-j-l+1\right\}P\left\{Y_{n}=l\right\}.
\end{eqnarray}
Adem\'as
\begin{equation}\label{Eq.L02S}
g_{0,k}=P\left\{\tilde{L}_{0}=k\right\}.
\end{equation}
Se definen las siguientes FGP:
\begin{equation}\label{Eq.3.16.a.2S}
G_{n}\left(z\right)=\sum_{k=0}^{\infty}g_{n,k}z^{k},\textrm{ para
}n=0,1,\ldots,
\end{equation}
y 
\begin{equation}\label{Eq.3.16.b.2S}
G\left(z,w\right)=\sum_{n=0}^{\infty}G_{n}\left(z\right)w^{n}, z,w\in\mathbb{C}.
\end{equation}
En particular para $k=0$,
\begin{eqnarray*}
g_{n,0}=G_{n}\left(0\right)=P\left\{\tilde{L}_{j}>0,\textrm{ para
}j<n,\textrm{ y }\tilde{L}_{n}=0\right\}=P\left\{T=n\right\},
\end{eqnarray*}
adem\'as
\begin{eqnarray*}%\label{Eq.G0w.2S}
G\left(0,w\right)=\sum_{n=0}^{\infty}G_{n}\left(0\right)w^{n}=\sum_{n=0}^{\infty}P\left\{T=n\right\}w^{n}
=\esp\left[w^{T}\right]
\end{eqnarray*}
la cu\'al resulta ser la FGP del tiempo de ruina $T$.


\begin{Prop}\label{Prop.1.1.2S}
Sean $z,w\in\mathbb{C}$ y sea $n\geq0$ fijo. Para $G_{n}\left(z\right)$ y $G\left(z,w\right)$ definidas como en
(\ref{Eq.3.16.a.2S}) y (\ref{Eq.3.16.b.2S}) respectivamente, se tiene que
\begin{equation}\label{Eq.Pag.45}
G_{n}\left(z\right)=\frac{1}{z}\left[G_{n-1}\left(z\right)-G_{n-1}\left(0\right)\right]\tilde{P}\left(z\right).
\end{equation}

Adem\'as


\begin{equation}\label{Eq.Pag.46}
G\left(z,w\right)=\frac{zF\left(z\right)-wP\left(z\right)G\left(0,w\right)}{z-wR\left(z\right)},
\end{equation}

con un \'unico polo en el c\'irculo unitario, adem\'as, el polo es
de la forma $z=\theta\left(w\right)$ y satisface que

\begin{enumerate}
\item[i)]$\tilde{\theta}\left(1\right)=1$,

\item[ii)] $\tilde{\theta}^{(1)}\left(1\right)=\frac{1}{1-\tilde{\mu}}$,

\item[iii)]
$\tilde{\theta}^{(2)}\left(1\right)=\frac{\tilde{\mu}}{\left(1-\tilde{\mu}\right)^{2}}+\frac{\tilde{\sigma}}{\left(1-\tilde{\mu}\right)^{3}}$.
\end{enumerate}

Finalmente, adem\'as se cumple que
\begin{equation}
\esp\left[w^{T}\right]=G\left(0,w\right)=F\left[\tilde{\theta}\left(w\right)\right].
\end{equation}
\end{Prop}
\begin{proof}

Multiplicando las ecuaciones (\ref{Eq.Gnk.2S}) y (\ref{Eq.L02S})
por el t\'ermino $z^{k}$:

\begin{eqnarray*}
g_{n,k}z^{k}&=&\sum_{j=1}^{k+1}\sum_{l=1}^{j}g_{n-1,j}P\left\{X_{n}=k-j-l+1\right\}P\left\{Y_{n}=l\right\}z^{k},\\
g_{0,k}z^{k}&=&P\left\{\tilde{L}_{0}=k\right\}z^{k},
\end{eqnarray*}

ahora sumamos sobre $k$
\begin{eqnarray*}
\sum_{k=0}^{\infty}g_{n,k}z^{k}&=&\sum_{k=0}^{\infty}\sum_{j=1}^{k+1}\sum_{l=1}^{j}g_{n-1,j}P\left\{X_{n}=k-j-l+1\right\}P\left\{Y_{n}=l\right\}z^{k}\\
&=&\sum_{k=0}^{\infty}z^{k}\sum_{j=1}^{k+1}\sum_{l=1}^{j}g_{n-1,j}P\left\{X_{n}=k-\left(j+l
-1\right)\right\}P\left\{Y_{n}=l\right\}\\
&=&\sum_{k=0}^{\infty}z^{k+\left(j+l-1\right)-\left(j+l-1\right)}\sum_{j=1}^{k+1}\sum_{l=1}^{j}g_{n-1,j}P\left\{X_{n}=k-
\left(j+l-1\right)\right\}P\left\{Y_{n}=l\right\}\\
&=&\sum_{k=0}^{\infty}\sum_{j=1}^{k+1}\sum_{l=1}^{j}g_{n-1,j}z^{j-1}P\left\{X_{n}=k-
\left(j+l-1\right)\right\}z^{k-\left(j+l-1\right)}P\left\{Y_{n}=l\right\}z^{l}\\
&=&\sum_{j=1}^{\infty}\sum_{l=1}^{j}g_{n-1,j}z^{j-1}\sum_{k=j+l-1}^{\infty}P\left\{X_{n}=k-
\left(j+l-1\right)\right\}z^{k-\left(j+l-1\right)}P\left\{Y_{n}=l\right\}z^{l}\\
&=&\sum_{j=1}^{\infty}g_{n-1,j}z^{j-1}\sum_{l=1}^{j}\sum_{k=j+l-1}^{\infty}P\left\{X_{n}=k-
\left(j+l-1\right)\right\}z^{k-\left(j+l-1\right)}P\left\{Y_{n}=l\right\}z^{l}\\
&=&\sum_{j=1}^{\infty}g_{n-1,j}z^{j-1}\sum_{k=j+l-1}^{\infty}\sum_{l=1}^{j}P\left\{X_{n}=k-
\left(j+l-1\right)\right\}z^{k-\left(j+l-1\right)}P\left\{Y_{n}=l\right\}z^{l}\\
&=&\sum_{j=1}^{\infty}g_{n-1,j}z^{j-1}\sum_{k=j+l-1}^{\infty}\sum_{l=1}^{j}P\left\{X_{n}=k-
\left(j+l-1\right)\right\}z^{k-\left(j+l-1\right)}\sum_{l=1}^{j}P
\left\{Y_{n}=l\right\}z^{l}\\
\end{eqnarray*}
\begin{eqnarray*}
&=&\sum_{j=1}^{\infty}g_{n-1,j}z^{j-1}\sum_{l=1}^{\infty}P\left\{Y_{n}=l\right\}z^{l}
\sum_{k=j+l-1}^{\infty}\sum_{l=1}^{j}
P\left\{X_{n}=k-\left(j+l-1\right)\right\}z^{k-\left(j+l-1\right)}\\
&=&\frac{1}{z}\left[G_{n-1}\left(z\right)-G_{n-1}\left(0\right)\right]\check{P}\left(z\right)
\sum_{k=j+l-1}^{\infty}\sum_{l=1}^{j}
P\left\{X_{n}=k-\left(j+l-1\right)\right\}z^{k-\left(j+l-1\right)}\\
&=&\frac{1}{z}\left[G_{n-1}\left(z\right)-G_{n-1}\left(0\right)\right]\check{P}\left(z\right)P\left(z\right)=\frac{1}{z}\left[G_{n-1}\left(z\right)-G_{n-1}\left(0\right)\right]\tilde{P}\left(z\right),
\end{eqnarray*}
es decir la ecuaci\'on (\ref{Eq.3.16.a.2S}) se puede reescribir como
\begin{equation}\label{Eq.3.16.a.2Sbis}
G_{n}\left(z\right)=\frac{1}{z}\left[G_{n-1}\left(z\right)-G_{n-1}\left(0\right)\right]\tilde{P}\left(z\right).
\end{equation}

Por otra parte recordemos la ecuaci\'on (\ref{Eq.3.16.a.2S})
\begin{eqnarray*}
G_{n}\left(z\right)&=&\sum_{k=0}^{\infty}g_{n,k}z^{k},\textrm{ entonces }\frac{G_{n}\left(z\right)}{z}=\sum_{k=1}^{\infty}g_{n,k}z^{k-1},
\end{eqnarray*}

por lo tanto utilizando la ecuaci\'on (\ref{Eq.3.16.a.2Sbis}):

\begin{eqnarray*}
G\left(z,w\right)&=&\sum_{n=0}^{\infty}G_{n}\left(z\right)w^{n}=G_{0}\left(z\right)+
\sum_{n=1}^{\infty}G_{n}\left(z\right)w^{n}=F\left(z\right)+\sum_{n=0}^{\infty}\left[G_{n}\left(z\right)-G_{n}\left(0\right)\right]w^{n}\frac{\tilde{P}\left(z\right)}{z}\\
&=&F\left(z\right)+\frac{w}{z}\sum_{n=0}^{\infty}\left[G_{n}\left(z\right)-G_{n}\left(0\right)\right]w^{n-1}\tilde{P}\left(z\right)
\end{eqnarray*}
es decir
\begin{eqnarray*}
G\left(z,w\right)&=&F\left(z\right)+\frac{w}{z}\left[G\left(z,w\right)-G\left(0,w\right)\right]\tilde{P}\left(z\right),
\end{eqnarray*}
entonces
\begin{eqnarray*}
G\left(z,w\right)=F\left(z\right)+\frac{w}{z}\left[G\left(z,w\right)-G\left(0,w\right)\right]\tilde{P}\left(z\right)&=&F\left(z\right)+\frac{w}{z}\tilde{P}\left(z\right)G\left(z,w\right)-\frac{w}{z}\tilde{P}\left(z\right)G\left(0,w\right)\\
&\Leftrightarrow&\\
G\left(z,w\right)\left\{1-\frac{w}{z}\tilde{P}\left(z\right)\right\}&=&F\left(z\right)-\frac{w}{z}\tilde{P}\left(z\right)G\left(0,w\right),
\end{eqnarray*}
por lo tanto,
\begin{equation}
G\left(z,w\right)=\frac{zF\left(z\right)-w\tilde{P}\left(z\right)G\left(0,w\right)}{1-w\tilde{P}\left(z\right)}.
\end{equation}
Ahora $G\left(z,w\right)$ es anal\'itica en $|z|=1$. Sean $z,w$ tales que $|z|=1$ y $|w|\leq1$, como $\tilde{P}\left(z\right)$ es FGP
\begin{eqnarray*}
|z-\left(z-w\tilde{P}\left(z\right)\right)|<|z|\Leftrightarrow|w\tilde{P}\left(z\right)|<|z|
\end{eqnarray*}
es decir, se cumplen las condiciones del Teorema de Rouch\'e y por
tanto, $z$ y $z-w\tilde{P}\left(z\right)$ tienen el mismo n\'umero de
ceros en $|z|=1$. Sea $z=\tilde{\theta}\left(w\right)$ la soluci\'on
\'unica de $z-w\tilde{P}\left(z\right)$, es decir
\begin{equation}\label{Eq.Theta.w}
\tilde{\theta}\left(w\right)-w\tilde{P}\left(\tilde{\theta}\left(w\right)\right)=0,
\end{equation}
 con $|\tilde{\theta}\left(w\right)|<1$. Cabe hacer menci\'on que $\tilde{\theta}\left(w\right)$ es la FGP para el tiempo de ruina cuando $\tilde{L}_{0}=1$. Considerando la ecuaci\'on (\ref{Eq.Theta.w})
\begin{eqnarray*}
0&=&\frac{\partial}{\partial w}\tilde{\theta}\left(w\right)|_{w=1}-\frac{\partial}{\partial w}\left\{w\tilde{P}\left(\tilde{\theta}\left(w\right)\right)\right\}|_{w=1}=\tilde{\theta}^{(1)}\left(w\right)|_{w=1}-\frac{\partial}{\partial w}w\left\{\tilde{P}\left(\tilde{\theta}\left(w\right)\right)\right\}|_{w=1}\\
&-&w\frac{\partial}{\partial w}\tilde{P}\left(\tilde{\theta}\left(w\right)\right)|_{w=1}=\tilde{\theta}^{(1)}\left(1\right)-\tilde{P}\left(\tilde{\theta}\left(1\right)\right)-w\left\{\frac{\partial \tilde{P}\left(\tilde{\theta}\left(w\right)\right)}{\partial \tilde{\theta}\left(w\right)}\cdot\frac{\partial\tilde{\theta}\left(w\right)}{\partial w}|_{w=1}\right\}\\
&=&\tilde{\theta}^{(1)}\left(1\right)-\tilde{P}\left(\tilde{\theta}\left(1\right)
\right)-\tilde{P}^{(1)}\left(\tilde{\theta}\left(1\right)\right)\cdot\tilde{\theta}^{(1)}\left(1\right),
\end{eqnarray*}
luego
$$\tilde{P}\left(\tilde{\theta}\left(1\right)\right)=\tilde{\theta}^{(1)}\left(1\right)-\tilde{P}^{(1)}\left(\tilde{\theta}\left(1\right)\right)\cdot
\tilde{\theta}^{(1)}\left(1\right)=\tilde{\theta}^{(1)}\left(1\right)\left(1-\tilde{P}^{(1)}\left(\tilde{\theta}\left(1\right)\right)\right),$$
por tanto $$\tilde{\theta}^{(1)}\left(1\right)=\frac{\tilde{P}\left(\tilde{\theta}\left(1\right)\right)}{\left(1-\tilde{P}^{(1)}\left(\tilde{\theta}\left(1\right)\right)\right)}=\frac{1}{1-\tilde{\mu}}.$$
Ahora determinemos el segundo momento de $\tilde{\theta}\left(w\right)$,
nuevamente consideremos la ecuaci\'on (\ref{Eq.Theta.w}):
\begin{eqnarray*}
0&=&\tilde{\theta}\left(w\right)-w\tilde{P}\left(\tilde{\theta}\left(w\right)\right)\Rightarrow 0=\frac{\partial}{\partial w}\left\{\tilde{\theta}\left(w\right)-w\tilde{P}\left(\tilde{\theta}\left(w\right)\right)\right\}\Rightarrow 0=\frac{\partial}{\partial w}\left\{\frac{\partial}{\partial w}\left\{\tilde{\theta}\left(w\right)-w\tilde{P}\left(\tilde{\theta}\left(w\right)\right)\right\}\right\}
\end{eqnarray*}
luego se tiene
\begin{eqnarray*}
&&\frac{\partial}{\partial w}\left\{\frac{\partial}{\partial w}\tilde{\theta}\left(w\right)-\frac{\partial}{\partial w}\left[w\tilde{P}\left(\tilde{\theta}\left(w\right)\right)\right]\right\}
=\frac{\partial}{\partial w}\left\{\frac{\partial}{\partial w}\tilde{\theta}\left(w\right)-\frac{\partial}{\partial w}\left[w\tilde{P}\left(\tilde{\theta}\left(w\right)\right)\right]\right\}\\
&=&\frac{\partial}{\partial w}\left\{\frac{\partial \tilde{\theta}\left(w\right)}{\partial w}-\left[\tilde{P}\left(\tilde{\theta}\left(w\right)\right)+w\frac{\partial}{\partial w}P\left(\tilde{\theta}\left(w\right)\right)\right]\right\}\\
&=&\frac{\partial}{\partial w}\left\{\frac{\partial \tilde{\theta}\left(w\right)}{\partial w}-\left(\tilde{P}\left(\tilde{\theta}\left(w\right)\right)+w\frac{\partial \tilde{P}\left(\tilde{\theta}\left(w\right)\right)}{\partial w}\frac{\partial \tilde{\theta}\left(w\right)}{\partial w}\right]\right\}\\
&=&\frac{\partial}{\partial w}\left\{\tilde{\theta}^{(1)}\left(w\right)-\tilde{P}\left(\tilde{\theta}\left(w\right)\right)-w\tilde{P}^{(1)}\left(\tilde{\theta}\left(w\right)\right)\tilde{\theta}^{(1)}\left(w\right)\right\}\\
&=&\frac{\partial}{\partial w}\tilde{\theta}^{(1)}\left(w\right)-\frac{\partial}{\partial w}\tilde{P}\left(\tilde{\theta}\left(w\right)\right)-\frac{\partial}{\partial w}\left[w\tilde{P}^{(1)}\left(\tilde{\theta}\left(w\right)\right)\tilde{\theta}^{(1)}\left(w\right)\right]\\
&=&\frac{\partial}{\partial
w}\tilde{\theta}^{(1)}\left(w\right)-\frac{\partial
\tilde{P}\left(\tilde{\theta}\left(w\right)\right)}{\partial
\tilde{\theta}\left(w\right)}\frac{\partial \tilde{\theta}\left(w\right)}{\partial
w}-\tilde{P}^{(1)}\left(\tilde{\theta}\left(w\right)\right)\tilde{\theta}^{(1)}\left(w\right)
-w\frac{\partial\tilde{P}^{(1)}\left(\tilde{\theta}\left(w\right)\right)}{\partial
w}\tilde{\theta}^{(1)}\left(w\right)\\
&-&w\tilde{P}^{(1)}\left(\tilde{\theta}\left(w\right)\right)\frac{\partial
\tilde{\theta}^{(1)}\left(w\right)}{\partial w}\\
&=&\tilde{\theta}^{(2)}\left(w\right)-\tilde{P}^{(1)}\left(\tilde{\theta}\left(w\right)\right)\tilde{\theta}^{(1)}\left(w\right)
-\tilde{P}^{(1)}\left(\tilde{\theta}\left(w\right)\right)\tilde{\theta}^{(1)}\left(w\right)
-w\tilde{P}^{(2)}\left(\tilde{\theta}\left(w\right)\right)\left(\tilde{\theta}^{(1)}\left(w\right)\right)^{2}
\end{eqnarray*}
\begin{eqnarray*}
&-&w\tilde{P}^{(1)}\left(\tilde{\theta}\left(w\right)\right)\tilde{\theta}^{(2)}\left(w\right)\\
&=&\tilde{\theta}^{(2)}\left(w\right)-2\tilde{P}^{(1)}\left(\tilde{\theta}\left(w\right)\right)\tilde{\theta}^{(1)}\left(w\right)-w\tilde{P}^{(2)}\left(\tilde{\theta}\left(w\right)\right)\left(\tilde{\theta}^{(1)}\left(w\right)\right)^{2}-w\tilde{P}^{(1)}\left(\tilde{\theta}\left(w\right)\right)\tilde{\theta}^{(2)}\left(w\right)\\
&=&\tilde{\theta}^{(2)}\left(w\right)\left[1-w\tilde{P}^{(1)}\left(\tilde{\theta}\left(w\right)\right)\right]-
\tilde{\theta}^{(1)}\left(w\right)\left[w\tilde{\theta}^{(1)}\left(w\right)\tilde{P}^{(2)}\left(\tilde{\theta}\left(w\right)\right)+2\tilde{P}^{(1)}\left(\tilde{\theta}\left(w\right)\right)\right]
\end{eqnarray*}
luego
\begin{eqnarray*}
\tilde{\theta}^{(2)}\left(w\right)&&\left[1-w\tilde{P}^{(1)}\left(\tilde{\theta}\left(w\right)\right)\right]-\tilde{\theta}^{(1)}\left(w\right)\left[w\tilde{\theta}^{(1)}\left(w\right)\tilde{P}^{(2)}\left(\tilde{\theta}\left(w\right)\right)
+2\tilde{P}^{(1)}\left(\tilde{\theta}\left(w\right)\right)\right]=0\\
\tilde{\theta}^{(2)}\left(w\right)&=&\frac{\tilde{\theta}^{(1)}\left(w\right)\left[w\tilde{\theta}^{(1)}\left(w\right)\tilde{P}^{(2)}\left(\tilde{\theta}\left(w\right)\right)+2P^{(1)}\left(\tilde{\theta}\left(w\right)\right)\right]}{1-w\tilde{P}^{(1)}\left(\tilde{\theta}\left(w\right)\right)}\\
&=&\frac{\tilde{\theta}^{(1)}\left(w\right)w\tilde{\theta}^{(1)}\left(w\right)\tilde{P}^{(2)}\left(\tilde{\theta}\left(w\right)\right)}{1-w\tilde{P}^{(1)}\left(\tilde{\theta}\left(w\right)\right)}+\frac{2\tilde{\theta}^{(1)}\left(w\right)\tilde{P}^{(1)}\left(\tilde{\theta}\left(w\right)\right)}{1-w\tilde{P}^{(1)}\left(\tilde{\theta}\left(w\right)\right)}
\end{eqnarray*}
si evaluamos la expresi\'on anterior en $w=1$:
\begin{eqnarray*}
\tilde{\theta}^{(2)}\left(1\right)&=&\frac{\left(\tilde{\theta}^{(1)}\left(1\right)\right)^{2}\tilde{P}^{(2)}\left(\tilde{\theta}\left(1\right)\right)}{1-\tilde{P}^{(1)}\left(\tilde{\theta}\left(1\right)\right)}+\frac{2\tilde{\theta}^{(1)}\left(1\right)\tilde{P}^{(1)}\left(\tilde{\theta}\left(1\right)\right)}{1-\tilde{P}^{(1)}\left(\tilde{\theta}\left(1\right)\right)}=\frac{\left(\tilde{\theta}^{(1)}\left(1\right)\right)^{2}\tilde{P}^{(2)}\left(1\right)}{1-\tilde{P}^{(1)}\left(1\right)}+\frac{2\tilde{\theta}^{(1)}\left(1\right)\tilde{P}^{(1)}\left(1\right)}{1-\tilde{P}^{(1)}\left(1\right)}\\
&=&\frac{\left(\frac{1}{1-\tilde{\mu}}\right)^{2}\tilde{P}^{(2)}\left(1\right)}{1-\tilde{\mu}}+\frac{2\left(\frac{1}{1-\tilde{\mu}}\right)\tilde{\mu}}{1-\tilde{\mu}}=\frac{\tilde{P}^{(2)}\left(1\right)}{\left(1-\tilde{\mu}\right)^{3}}+\frac{2\tilde{\mu}}{\left(1-\tilde{\mu}\right)^{2}}=\frac{\sigma^{2}-\tilde{\mu}+\tilde{\mu}^{2}}{\left(1-\tilde{\mu}\right)^{3}}+\frac{2\tilde{\mu}}{\left(1-\tilde{\mu}\right)^{2}}\\
&=&\frac{\sigma^{2}-\tilde{\mu}+\tilde{\mu}^{2}+2\tilde{\mu}\left(1-\tilde{\mu}\right)}{\left(1-\tilde{\mu}\right)^{3}}
\end{eqnarray*}
es decir
\begin{eqnarray*}
\tilde{\theta}^{(2)}\left(1\right)&=&\frac{\sigma^{2}}{\left(1-\tilde{\mu}\right)^{3}}+\frac{\tilde{\mu}}{\left(1-\tilde{\mu}\right)^{2}}.
\end{eqnarray*}
\end{proof}

\begin{Coro}
El tiempo de ruina del jugador tiene primer y segundo momento dados por
\begin{eqnarray}
\esp\left[T\right]&=&\frac{\esp\left[\tilde{L}_{0}\right]}{1-\tilde{\mu}}\\
Var\left[T\right]&=&\frac{Var\left[\tilde{L}_{0}\right]}{\left(1-\tilde{\mu}\right)^{2}}+\frac{\sigma^{2}\esp\left[\tilde{L}_{0}\right]}{\left(1-\tilde{\mu}\right)^{3}}.
\end{eqnarray}
\end{Coro}
%_________________________________________________________________________________________________
\section{Sistemas de visitas: Ecuaciones Recursivas}
%__________________________________________________________________________________________________



%__________________________________________________________________________
\subsection{Definiciones}
%__________________________________________________________________________

Se considerar\'an intervalos de tiempo de la forma
$\left[t,t+1\right]$. Los usuarios arriban por paquetes de manera
independiente del resto de las colas. Se define el grupo de
usuarios que llegan a cada una de las colas del sistema 1,
caracterizadas por $Q_{1}$ y $Q_{2}$ respectivamente, en el
intervalo de tiempo $\left[t,t+1\right]$ por
$X_{1}\left(t\right),X_{2}\left(t\right)$.



Para cada uno de los procesos anteriores se define su Funci\'on
Generadora de Probabilidades (PGF):

\begin{eqnarray*}
\begin{array}{cc}
P_{1}\left(z_{1}\right)=\esp\left[z_{1}^{X_{1}\left(t\right)}\right], & P_{2}\left(z_{2}\right)=\esp\left[z_{2}^{X_{2}\left(t\right)}\right].\\
\end{array}
\end{eqnarray*}

Con primer momento definidos por

\begin{eqnarray*}
%\begin{array}{cc}
\mu_{1}&=&\esp\left[X_{1}\left(t\right)\right]=P_{1}^{(1)}\left(1\right),\\
\mu_{2}&=&\esp\left[X_{2}\left(t\right)\right]=P_{2}^{(1)}\left(1\right).\\
%\end{array}
\end{eqnarray*}


En lo que respecta al servidor, en t\'erminos de los tiempos de
visita a cada una de las colas, se denotar\'an por
$\tau_{1},\tau_{2}$ para $Q_{1},Q_{2}$ respectivamente; y a los
tiempos en que el servidor termina de atender en las colas
$Q_{1},Q_{2}$, se les denotar\'a por
$\overline{\tau}_{1},\overline{\tau}_{2}$ respectivamente.
Entonces, los tiempos de servicio est\'an dados por las
diferencias
$\overline{\tau}_{1}-\tau_{1},\overline{\tau}_{2}-\tau_{2}$ para
$Q_{1},Q_{2}$. An\'alogamente los tiempos de traslado del servidor
desde el momento en que termina de atender a una cola y llega a la
siguiente para comenzar a dar servicio est\'an dados por
$\tau_{2}-\overline{\tau}_{1},\tau_{1}-\overline{\tau}_{2}$.


La FGP para estos tiempos de traslado est\'an dados por

\begin{eqnarray*}
\begin{array}{cc}
R_{1}\left(z_{1}\right)=\esp\left[z_{1}^{\tau_{2}-\overline{\tau}_{1}}\right],
&
R_{2}\left(z_{2}\right)=\esp\left[z_{2}^{\tau_{1}-\overline{\tau}_{2}}\right],
\end{array}
\end{eqnarray*}

y al igual que como se hizo con anterioridad

\begin{eqnarray*}
\begin{array}{cc}
r_{1}=R_{1}^{(1)}\left(1\right)=\esp\left[\tau_{2}-\overline{\tau}_{1}\right],
&
r_{2}=R_{2}^{(1)}\left(1\right)=\esp\left[\tau_{1}-\overline{\tau}_{2}\right],\\
\end{array}
\end{eqnarray*}


Sean $\alpha_{1},\alpha_{2}$ el n\'umero de usuarios que arriban
en grupo a la cola $Q_{1}$ y $Q_{2}$ respectivamente. Sus PGF's
est\'an definidas como

\begin{eqnarray*}
\begin{array}{cc}
A_{1}\left(z\right)=\esp\left[z^{\alpha_{1}\left(t\right)}\right],&
A_{2}\left(z\right)=\esp\left[z^{\alpha_{2}\left(t\right)}\right].\\
\end{array}
\end{eqnarray*}

Su primer momento est\'a dado por

\begin{eqnarray*}
\begin{array}{cc}
\lambda_{1}=\esp\left[\alpha_{1}\left(t\right)\right]=A_{1}^{(1)}\left(1\right),&
\lambda_{2}=\esp\left[\alpha_{2}\left(t\right)\right]=A_{2}^{(1)}\left(1\right).\\
\end{array}
\end{eqnarray*}


Sean $\beta_{1},\beta_{2}$ el n\'umero de usuarios que arriban en
el grupo $\alpha_{1},\alpha_{2}$ a la cola $Q_{1}$ y $Q_{2}$,
respectivamente, de igual manera se definen sus PGF's

\begin{eqnarray*}
\begin{array}{cc}
B_{1}\left(z\right)=\esp\left[z^{\beta_{1}\left(t\right)}\right],&
B_{2}\left(z\right)=\esp\left[z^{\beta_{2}\left(t\right)}\right],\\
\end{array}
\end{eqnarray*}

con

\begin{eqnarray*}
\begin{array}{cc}
b_{1}=\esp\left[\beta_{1}\left(t\right)\right]=B_{1}^{(1)}\left(1\right),&
b_{2}=\esp\left[\beta_{2}\left(t\right)\right]=B_{2}^{(1)}\left(1\right).\\
\end{array}
\end{eqnarray*}

La distribuci\'on para el n\'umero de grupos que arriban al
sistema en cada una de las colas se definen por:

\begin{eqnarray*}
\begin{array}{cc}
P_{1}\left(z_{1}\right)=A_{1}\left[B_{1}\left(z_{1}\right)\right]=\esp\left[B_{1}\left(z_{1}\right)^{\alpha_{1}\left(t\right)}\right],&
P_{2}\left(z_{1}\right)=A_{1}\left[B_{1}\left(z_{1}\right)\right]=\esp\left[B_{1}\left(z_{1}\right)^{\alpha_{1}\left(t\right)}\right],\\
\end{array}
\end{eqnarray*}

entonces

\begin{eqnarray*}
P_{1}^{(1)}\left(1\right)&=&\esp\left[\alpha_{1}\left(t\right)B_{1}^{(1)}\left(1\right)\right]=B_{1}^{(1)}\left(1\right)\esp\left[\alpha_{1}\left(t\right)\right]=\lambda_{1}b_{1}\\
P_{2}^{(1)}\left(1\right)&=&\esp\left[\alpha_{2}\left(t\right)B_{2}^{(1)}\left(1\right)\right]=B_{2}^{(1)}\left(1\right)\esp\left[\alpha_{2}\left(t\right)\right]=\lambda_{2}b_{2}.\\
\end{eqnarray*}




%\end{Def}

%________________________________________________________
\subsection{Funciones Generadoras de Probabilidad Conjunta}
%________________________________________________________


De lo desarrollado hasta ahora se tiene lo siguiente

\begin{eqnarray*}
&&\esp\left[z_{1}^{L_{1}\left(\overline{\tau}_{1}\right)}z_{2}^{L_{2}\left(\overline{\tau}_{1}\right)}\right]=\esp\left[z_{2}^{L_{2}\left(\overline{\tau}_{1}\right)}\right]=\esp\left[z_{2}^{L_{2}\left(\tau_{1}\right)+X_{2}\left(\overline{\tau}_{1}-\tau_{1}\right)}\right]\\
&=&\esp\left[\left\{z_{2}^{L_{2}\left(\tau_{1}\right)}\right\}\left\{z_{2}^{X_{2}\left(\overline{\tau}_{1}-\tau_{1}\right)}\right\}\right]=\esp\left[\left\{z_{2}^{L_{2}\left(\tau_{1}\right)}\right\}\left\{P_{2}\left(z_{2}\right)\right\}^{\overline{\tau}_{1}-\tau_{1}}\right]\\
&=&\esp\left[\left\{z_{2}^{L_{2}\left(\tau_{1}\right)}\right\}\left\{\theta_{1}\left(P_{2}\left(z_{2}\right)\right)\right\}^{L_{1}\left(\tau_{1}\right)}\right]=F_{1}\left(\theta_{1}\left(P_{2}\left(z_{2}\right)\right),z_{2}\right)
\end{eqnarray*}

es decir %{{\tiny
\begin{equation}\label{Eq.base.F1}
\esp\left[z_{1}^{L_{1}\left(\overline{\tau}_{1}\right)}z_{2}^{L_{2}\left(\overline{\tau}_{1}\right)}\right]=F_{1}\left(\theta_{1}\left(P_{2}\left(z_{2}\right)\right),z_{2}\right).
\end{equation}

Procediendo de manera an\'aloga para $\overline{\tau}_{2}$:

\begin{eqnarray*}
\esp\left[z_{1}^{L_{1}\left(\overline{\tau}_{2}\right)}z_{2}^{L_{2}\left(\overline{\tau}_{2}\right)}\right]&=&\esp\left[z_{1}^{L_{1}\left(\overline{\tau}_{2}\right)}\right]=\esp\left[z_{1}^{L_{1}\left(\tau_{2}\right)+X_{1}\left(\overline{\tau}_{2}-\tau_{2}\right)}\right]=\esp\left[\left\{z_{1}^{L_{1}\left(\tau_{2}\right)}\right\}\left\{z_{1}^{X_{1}\left(\overline{\tau}_{2}-\tau_{2}\right)}\right\}\right]\\
&=&\esp\left[\left\{z_{1}^{L_{1}\left(\tau_{2}\right)}\right\}\left\{P_{1}\left(z_{1}\right)\right\}^{\overline{\tau}_{2}-\tau_{2}}\right]=\esp\left[\left\{z_{1}^{L_{1}\left(\tau_{2}\right)}\right\}\left\{\theta_{2}\left(P_{1}\left(z_{1}\right)\right)\right\}^{L_{2}\left(\tau_{2}\right)}\right]\\
&=&F_{2}\left(z_{1},\theta_{2}\left(P_{1}\left(z_{1}\right)\right)\right)
\end{eqnarray*}%}}


\begin{equation}\label{Eq.PGF.Conjunta.Tau2}
\esp\left[z_{1}^{L_{1}\left(\overline{\tau}_{2}\right)}z_{2}^{L_{2}\left(\overline{\tau}_{2}\right)}\right]=F_{2}\left(z_{1},\theta_{2}\left(P_{1}\left(z_{1}\right)\right)\right)
\end{equation}%}

Ahora, para el intervalo de tiempo
$\left[\overline{\tau}_{1},\tau_{2}\right]$ y
$\left[\overline{\tau}_{2},\tau_{1}\right]$, los arribos de los
usuarios modifican el n\'umero de usuarios que llegan a las colas,
es decir, los procesos
$L_{1}\left(t\right)$
y $L_{2}\left(t\right)$. La PGF para el n\'umero de arribos
a todas las estaciones durante el intervalo
$\left[\overline{\tau}_{1},\tau_{2}\right]$  cuya distribuci\'on
est\'a especificada por la distribuci\'on compuesta
$R_{1}\left(\mathbf{z}\right),R_{2}\left(\mathbf{z}\right)$:

\begin{eqnarray*}
R_{1}\left(\mathbf{z}\right)=R_{1}\left(\prod_{i=1}^{2}P\left(z_{i}\right)\right)=\esp\left[\left\{\prod_{i=1}^{2}P\left(z_{i}\right)\right\}^{\tau_{2}-\overline{\tau}_{1}}\right]\\
R_{2}\left(\mathbf{z}\right)=R_{2}\left(\prod_{i=1}^{2}P\left(z_{i}\right)\right)=\esp\left[\left\{\prod_{i=1}^{2}P\left(z_{i}\right)\right\}^{\tau_{1}-\overline{\tau}_{2}}\right]\\
\end{eqnarray*}


Dado que los eventos en
$\left[\tau_{1},\overline{\tau}_{1}\right]$ y
$\left[\overline{\tau}_{1},\tau_{2}\right]$ son independientes, la
PGF conjunta para el n\'umero de usuarios en el sistema al tiempo
$t=\tau_{2}$ la PGF conjunta para el n\'umero de usuarios en el sistema est\'an dadas por

{\footnotesize{
\begin{eqnarray*}
F_{1}\left(\mathbf{z}\right)&=&R_{2}\left(\prod_{i=1}^{2}P\left(z_{i}\right)\right)F_{2}\left(z_{1},\theta_{2}\left(P_{1}\left(z_{1}\right)\right)\right)\\
F_{2}\left(\mathbf{z}\right)&=&R_{1}\left(\prod_{i=1}^{2}P\left(z_{i}\right)\right)F_{1}\left(\theta_{1}\left(P_{2}\left(z_{2}\right)\right),z_{2}\right)\\
\end{eqnarray*}}}


Entonces debemos de determinar las siguientes expresiones:


\begin{eqnarray*}
\begin{array}{cc}
f_{1}\left(1\right)=\frac{\partial F_{1}\left(\mathbf{z}\right)}{\partial z_{1}}|_{\mathbf{z}=1}, & f_{1}\left(2\right)=\frac{\partial F_{1}\left(\mathbf{z}\right)}{\partial z_{2}}|_{\mathbf{z}=1},\\
f_{2}\left(1\right)=\frac{\partial F_{2}\left(\mathbf{z}\right)}{\partial z_{1}}|_{\mathbf{z}=1}, & f_{2}\left(2\right)=\frac{\partial F_{2}\left(\mathbf{z}\right)}{\partial z_{2}}|_{\mathbf{z}=1},\\
\end{array}
\end{eqnarray*}


\begin{eqnarray*}
\frac{\partial R_{1}\left(\mathbf{z}\right)}{\partial
z_{1}}|_{\mathbf{z}=1}&=&R_{1}^{(1)}\left(1\right)P_{1}^{(1)}\left(1\right)\\
\frac{\partial R_{1}\left(\mathbf{z}\right)}{\partial
z_{2}}|_{\mathbf{z}=1}&=&R_{1}^{(1)}\left(1\right)P_{2}^{(1)}\left(1\right)\\
\frac{\partial R_{2}\left(\mathbf{z}\right)}{\partial
z_{1}}|_{\mathbf{z}=1}&=&R_{2}^{(1)}\left(1\right)P_{1}^{(1)}\left(1\right)\\
\frac{\partial R_{2}\left(\mathbf{z}\right)}{\partial
z_{2}}|_{\mathbf{z}=1}&=&R_{2}^{(1)}\left(1\right)P_{2}^{(1)}\left(1\right)\\
\end{eqnarray*}



\begin{eqnarray*}
\frac{\partial}{\partial
z_{1}}F_{1}\left(\theta_{1}\left(P_{2}\left(z_{2}\right)\right),z_{2}\right)&=&0\\
\frac{\partial}{\partial
z_{2}}F_{1}\left(\theta_{1}\left(P_{2}\left(z_{2}\right)\right),z_{2}\right)&=&\frac{\partial
F_{1}}{\partial z_{2}}+\frac{\partial F_{1}}{\partial
z_{1}}\theta_{1}^{(1)}P_{2}^{(1)}\left(1\right)\\
\frac{\partial}{\partial
z_{1}}F_{2}\left(z_{1},\theta_{2}\left(P_{1}\left(z_{1}\right)\right)\right)&=&\frac{\partial
F_{2}}{\partial z_{1}}+\frac{\partial F_{2}}{\partial
z_{2}}\theta_{2}^{(1)}P_{1}^{(1)}\left(1\right)\\
\frac{\partial}{\partial
z_{2}}F_{2}\left(z_{1},\theta_{2}\left(P_{1}\left(z_{1}\right)\right)\right)&=&0\\
\end{eqnarray*}


Por lo tanto de las dos secciones anteriores se tiene que:


\begin{eqnarray*}
\frac{\partial F_{1}}{\partial z_{1}}&=&\frac{\partial
R_{2}}{\partial z_{1}}|_{\mathbf{z}=1}+\frac{\partial F_{2}}{\partial z_{1}}|_{\mathbf{z}=1}=R_{2}^{(1)}\left(1\right)P_{1}^{(1)}\left(1\right)+f_{2}\left(1\right)+f_{2}\left(2\right)\theta_{2}^{(1)}\left(1\right)P_{1}^{(1)}\left(1\right)\\
\frac{\partial F_{1}}{\partial z_{2}}&=&\frac{\partial
R_{2}}{\partial z_{2}}|_{\mathbf{z}=1}+\frac{\partial F_{2}}{\partial z_{2}}|_{\mathbf{z}=1}=R_{2}^{(1)}\left(1\right)P_{2}^{(1)}\left(1\right)\\
\frac{\partial F_{2}}{\partial z_{1}}&=&\frac{\partial
R_{1}}{\partial z_{1}}|_{\mathbf{z}=1}+\frac{\partial F_{1}}{\partial z_{1}}|_{\mathbf{z}=1}=R_{1}^{(1)}\left(1\right)P_{1}^{(1)}\left(1\right)\\
\frac{\partial F_{2}}{\partial z_{2}}&=&\frac{\partial
R_{1}}{\partial z_{2}}|_{\mathbf{z}=1}+\frac{\partial F_{1}}{\partial z_{2}}|_{\mathbf{z}=1}
=R_{1}^{(1)}\left(1\right)P_{2}^{(1)}\left(1\right)+f_{1}\left(1\right)\theta_{1}^{(1)}\left(1\right)P_{2}^{(1)}\left(1\right)\\
\end{eqnarray*}


El cual se puede escribir en forma equivalente:
\begin{eqnarray*}
f_{1}\left(1\right)&=&r_{2}\mu_{1}+f_{2}\left(1\right)+f_{2}\left(2\right)\frac{\mu_{1}}{1-\mu_{2}}\\
f_{1}\left(2\right)&=&r_{2}\mu_{2}\\
f_{2}\left(1\right)&=&r_{1}\mu_{1}\\
f_{2}\left(2\right)&=&r_{1}\mu_{2}+f_{1}\left(2\right)+f_{1}\left(1\right)\frac{\mu_{2}}{1-\mu_{1}}\\
\end{eqnarray*}

De donde:
\begin{eqnarray*}
f_{1}\left(1\right)&=&\mu_{1}\left[r_{2}+\frac{f_{2}\left(2\right)}{1-\mu_{2}}\right]+f_{2}\left(1\right)\\
f_{2}\left(2\right)&=&\mu_{2}\left[r_{1}+\frac{f_{1}\left(1\right)}{1-\mu_{1}}\right]+f_{1}\left(2\right)\\
\end{eqnarray*}

Resolviendo para $f_{1}\left(1\right)$:
\begin{eqnarray*}
f_{1}\left(1\right)&=&r_{2}\mu_{1}+f_{2}\left(1\right)+f_{2}\left(2\right)\frac{\mu_{1}}{1-\mu_{2}}=r_{2}\mu_{1}+r_{1}\mu_{1}+f_{2}\left(2\right)\frac{\mu_{1}}{1-\mu_{2}}\\
&=&\mu_{1}\left(r_{2}+r_{1}\right)+f_{2}\left(2\right)\frac{\mu_{1}}{1-\mu_{2}}=\mu_{1}\left(r+\frac{f_{2}\left(2\right)}{1-\mu_{2}}\right),\\
\end{eqnarray*}

entonces

\begin{eqnarray*}
f_{2}\left(2\right)&=&\mu_{2}\left(r_{1}+\frac{f_{1}\left(1\right)}{1-\mu_{1}}\right)+f_{1}\left(2\right)=\mu_{2}\left(r_{1}+\frac{f_{1}\left(1\right)}{1-\mu_{1}}\right)+r_{2}\mu_{2}\\
&=&\mu_{2}\left[r_{1}+r_{2}+\frac{f_{1}\left(1\right)}{1-\mu_{1}}\right]=\mu_{2}\left[r+\frac{f_{1}\left(1\right)}{1-\mu_{1}}\right]\\
&=&\mu_{2}r+\mu_{1}\left(r+\frac{f_{2}\left(2\right)}{1-\mu_{2}}\right)\frac{\mu_{2}}{1-\mu_{1}}\\
&=&\mu_{2}r+\mu_{2}\frac{r\mu_{1}}{1-\mu_{1}}+f_{2}\left(2\right)\frac{\mu_{1}\mu_{2}}{\left(1-\mu_{1}\right)\left(1-\mu_{2}\right)}\\
&=&\mu_{2}\left(r+\frac{r\mu_{1}}{1-\mu_{1}}\right)+f_{2}\left(2\right)\frac{\mu_{1}\mu_{2}}{\left(1-\mu_{1}\right)\left(1-\mu_{2}\right)}\\
&=&\mu_{2}\left(\frac{r}{1-\mu_{1}}\right)+f_{2}\left(2\right)\frac{\mu_{1}\mu_{2}}{\left(1-\mu_{1}\right)\left(1-\mu_{2}\right)}\\
\end{eqnarray*}
entonces
\begin{eqnarray*}
f_{2}\left(2\right)-f_{2}\left(2\right)\frac{\mu_{1}\mu_{2}}{\left(1-\mu_{1}\right)\left(1-\mu_{2}\right)}&=&\mu_{2}\left(\frac{r}{1-\mu_{1}}\right)\\
f_{2}\left(2\right)\left(1-\frac{\mu_{1}\mu_{2}}{\left(1-\mu_{1}\right)\left(1-\mu_{2}\right)}\right)&=&\mu_{2}\left(\frac{r}{1-\mu_{1}}\right)\\
f_{2}\left(2\right)\left(\frac{1-\mu_{1}-\mu_{2}+\mu_{1}\mu_{2}-\mu_{1}\mu_{2}}{\left(1-\mu_{1}\right)\left(1-\mu_{2}\right)}\right)&=&\mu_{2}\left(\frac{r}{1-\mu_{1}}\right)\\
f_{2}\left(2\right)\left(\frac{1-\mu}{\left(1-\mu_{1}\right)\left(1-\mu_{2}\right)}\right)&=&\mu_{2}\left(\frac{r}{1-\mu_{1}}\right)\\
\end{eqnarray*}
por tanto
\begin{eqnarray*}
f_{2}\left(2\right)&=&\frac{r\frac{\mu_{2}}{1-\mu_{1}}}{\frac{1-\mu}{\left(1-\mu_{1}\right)\left(1-\mu_{2}\right)}}=\frac{r\mu_{2}\left(1-\mu_{1}\right)\left(1-\mu_{2}\right)}{\left(1-\mu_{1}\right)\left(1-\mu\right)}\\
&=&\frac{\mu_{2}\left(1-\mu_{2}\right)}{1-\mu}r=r\mu_{2}\frac{1-\mu_{2}}{1-\mu}.
\end{eqnarray*}
es decir

\begin{eqnarray}
f_{2}\left(2\right)&=&r\mu_{2}\frac{1-\mu_{2}}{1-\mu}.
\end{eqnarray}

Entonces

\begin{eqnarray*}
f_{1}\left(1\right)&=&\mu_{1}r+f_{2}\left(2\right)\frac{\mu_{1}}{1-\mu_{2}}=\mu_{1}r+\left(\frac{\mu_{2}\left(1-\mu_{2}\right)}{1-\mu}r\right)\frac{\mu_{1}}{1-\mu_{2}}\\
&=&\mu_{1}r+\mu_{1}r\left(\frac{\mu_{2}}{1-\mu}\right)=\mu_{1}r\left[1+\frac{\mu_{2}}{1-\mu}\right]\\
&=&r\mu_{1}\frac{1-\mu_{1}}{1-\mu}\\
\end{eqnarray*}


%-------------- CAPITULO YA INCLUIDO EN OTRO ----
%\chapter{Modelos de Flujo}
%%___________________________________________________________________________________________
%
\section{Procesos Regenerativos}
%_____________________________________________________

Si $x$ es el n{\'u}mero de usuarios en la cola al comienzo del
periodo de servicio y $N_{s}\left(x\right)=N\left(x\right)$ es el
n{\'u}mero de usuarios que son atendidos con la pol{\'\i}tica $s$,
{\'u}nica en nuestro caso, durante un periodo de servicio,
entonces se asume que:
\begin{itemize}
\item[(S1.)]
\begin{equation}\label{S1}
lim_{x\rightarrow\infty}\esp\left[N\left(x\right)\right]=\overline{N}>0.
\end{equation}
\item[(S2.)]
\begin{equation}\label{S2}
\esp\left[N\left(x\right)\right]\leq \overline{N}, \end{equation}
para cualquier valor de $x$. \item La $n$-{\'e}sima ocurrencia va
acompa{\~n}ada con el tiempo de cambio de longitud
$\delta_{j,j+1}\left(n\right)$, independientes e id{\'e}nticamente
distribuidas, con
$\esp\left[\delta_{j,j+1}\left(1\right)\right]\geq0$. \item Se
define
\begin{equation}
\delta^{*}:=\sum_{j,j+1}\esp\left[\delta_{j,j+1}\left(1\right)\right].
\end{equation}

\item Los tiempos de inter-arribo a la cola $k$,son de la forma
$\left\{\xi_{k}\left(n\right)\right\}_{n\geq1}$, con la propiedad
de que son independientes e id{\'e}nticamente distribuidos.

\item Los tiempos de servicio
$\left\{\eta_{k}\left(n\right)\right\}_{n\geq1}$ tienen la
propiedad de ser independientes e id{\'e}nticamente distribuidos.

\item Se define la tasa de arribo a la $k$-{\'e}sima cola como
$\lambda_{k}=1/\esp\left[\xi_{k}\left(1\right)\right]$ y
adem{\'a}s se define

\item la tasa de servicio para la $k$-{\'e}sima cola como
$\mu_{k}=1/\esp\left[\eta_{k}\left(1\right)\right]$

\item tambi{\'e}n se define $\rho_{k}=\lambda_{k}/\mu_{k}$, donde
es necesario que $\rho<1$ para cuestiones de estabilidad.

\item De las pol{\'\i}ticas posibles solamente consideraremos la
pol{\'\i}tica cerrada (Gated).
\end{itemize}

Las Colas C\'iclicas se pueden describir por medio de un proceso
de Markov $\left(X\left(t\right)\right)_{t\in\rea}$, donde el
estado del sistema al tiempo $t\geq0$ est\'a dado por
\begin{equation}
X\left(t\right)=\left(Q\left(t\right),A\left(t\right),H\left(t\right),B\left(t\right),B^{0}\left(t\right),C\left(t\right)\right)
\end{equation}
definido en el espacio producto:
\begin{equation}
\mathcal{X}=\mathbb{Z}^{K}\times\rea_{+}^{K}\times\left(\left\{1,2,\ldots,K\right\}\times\left\{1,2,\ldots,S\right\}\right)^{M}\times\rea_{+}^{K}\times\rea_{+}^{K}\times\mathbb{Z}^{K},
\end{equation}

\begin{itemize}
\item $Q\left(t\right)=\left(Q_{k}\left(t\right),1\leq k\leq
K\right)$, es el n\'umero de usuarios en la cola $k$, incluyendo
aquellos que est\'an siendo atendidos provenientes de la
$k$-\'esima cola.

\item $A\left(t\right)=\left(A_{k}\left(t\right),1\leq k\leq
K\right)$, son los residuales de los tiempos de arribo en la cola
$k$. \item $H\left(t\right)$ es el par ordenado que consiste en la
cola que esta siendo atendida y la pol\'itica de servicio que se
utilizar\'a.

\item $B\left(t\right)$ es el tiempo de servicio residual.

\item $B^{0}\left(t\right)$ es el tiempo residual del cambio de
cola.

\item $C\left(t\right)$ indica el n\'umero de usuarios atendidos
durante la visita del servidor a la cola dada en
$H\left(t\right)$.
\end{itemize}

$A_{k}\left(t\right),B_{m}\left(t\right)$ y
$B_{m}^{0}\left(t\right)$ se suponen continuas por la derecha y
que satisfacen la propiedad fuerte de Markov, (\cite{Dai})

\begin{itemize}
\item Los tiempos de interarribo a la cola $k$,son de la forma
$\left\{\xi_{k}\left(n\right)\right\}_{n\geq1}$, con la propiedad
de que son independientes e id{\'e}nticamente distribuidos.

\item Los tiempos de servicio
$\left\{\eta_{k}\left(n\right)\right\}_{n\geq1}$ tienen la
propiedad de ser independientes e id{\'e}nticamente distribuidos.

\item Se define la tasa de arribo a la $k$-{\'e}sima cola como
$\lambda_{k}=1/\esp\left[\xi_{k}\left(1\right)\right]$ y
adem{\'a}s se define

\item la tasa de servicio para la $k$-{\'e}sima cola como
$\mu_{k}=1/\esp\left[\eta_{k}\left(1\right)\right]$

\item tambi{\'e}n se define $\rho_{k}=\lambda_{k}/\mu_{k}$, donde
es necesario que $\rho<1$ para cuestiones de estabilidad.

\item De las pol{\'\i}ticas posibles solamente consideraremos la
pol{\'\i}tica cerrada (Gated).
\end{itemize}

%\section{Preliminares}



Sup\'ongase que el sistema consta de varias colas a los cuales
llegan uno o varios servidores a dar servicio a los usuarios
esperando en la cola.\\


Si $x$ es el n\'umero de usuarios en la cola al comienzo del
periodo de servicio y $N_{s}\left(x\right)=N\left(x\right)$ es el
n\'umero de usuarios que son atendidos con la pol\'itica $s$,
\'unica en nuestro caso, durante un periodo de servicio, entonces
se asume que:
\begin{itemize}
\item[1)]\label{S1}$lim_{x\rightarrow\infty}\esp\left[N\left(x\right)\right]=\overline{N}>0$
\item[2)]\label{S2}$\esp\left[N\left(x\right)\right]\leq\overline{N}$para
cualquier valor de $x$.
\end{itemize}
La manera en que atiende el servidor $m$-\'esimo, en este caso en
espec\'ifico solo lo ilustraremos con un s\'olo servidor, es la
siguiente:
\begin{itemize}
\item Al t\'ermino de la visita a la cola $j$, el servidor se
cambia a la cola $j^{'}$ con probabilidad
$r_{j,j^{'}}^{m}=r_{j,j^{'}}$

\item La $n$-\'esima ocurrencia va acompa\~nada con el tiempo de
cambio de longitud $\delta_{j,j^{'}}\left(n\right)$,
independientes e id\'enticamente distribuidas, con
$\esp\left[\delta_{j,j^{'}}\left(1\right)\right]\geq0$.

\item Sea $\left\{p_{j}\right\}$ la distribuci\'on invariante
estacionaria \'unica para la Cadena de Markov con matriz de
transici\'on $\left(r_{j,j^{'}}\right)$.

\item Finalmente, se define
\begin{equation}
\delta^{*}:=\sum_{j,j^{'}}p_{j}r_{j,j^{'}}\esp\left[\delta_{j,j^{'}}\left(i\right)\right].
\end{equation}
\end{itemize}

Veamos un caso muy espec\'ifico en el cual los tiempos de arribo a cada una de las colas se comportan de acuerdo a un proceso Poisson de la forma
$\left\{\xi_{k}\left(n\right)\right\}_{n\geq1}$, y los tiempos de servicio en cada una de las colas son variables aleatorias distribuidas exponencialmente e id\'enticamente distribuidas
$\left\{\eta_{k}\left(n\right)\right\}_{n\geq1}$, donde ambos procesos adem\'as cumplen la condici\'on de ser independientes entre si. Para la $k$-\'esima cola se define la tasa de arribo a la como
$\lambda_{k}=1/\esp\left[\xi_{k}\left(1\right)\right]$ y la tasa
de servicio como
$\mu_{k}=1/\esp\left[\eta_{k}\left(1\right)\right]$, finalmente se
define la carga de la cola como $\rho_{k}=\lambda_{k}/\mu_{k}$,
donde se pide que $\rho<1$, para garantizar la estabilidad del sistema.\\

Se denotar\'a por $Q_{k}\left(t\right)$ el n\'umero de usuarios en la cola $k$,
$A_{k}\left(t\right)$ los residuales de los tiempos entre arribos a la cola $k$;
para cada servidor $m$, se denota por $B_{m}\left(t\right)$ los residuales de los tiempos de servicio al tiempo $t$; $B_{m}^{0}\left(t\right)$ son los residuales de los tiempos de traslado de la cola $k$ a la pr\'oxima por atender, al tiempo $t$, finalmente sea $C_{m}\left(t\right)$ el n\'umero de usuarios atendidos durante la visita del servidor a la cola $k$ al tiempo $t$.\\


En este sentido el proceso para el sistema de visitas se puede definir como:

\begin{equation}\label{Esp.Edos.Down}
X\left(t\right)^{T}=\left(Q_{k}\left(t\right),A_{k}\left(t\right),B_{m}\left(t\right),B_{m}^{0}\left(t\right),C_{m}\left(t\right)\right)
\end{equation}
para $k=1,\ldots,K$ y $m=1,2,\ldots,M$. $X$ evoluciona en el
espacio de estados:
$X=\ent_{+}^{K}\times\rea_{+}^{K}\times\left(\left\{1,2,\ldots,K\right\}\times\left\{1,2,\ldots,S\right\}\right)^{M}\times\rea_{+}^{K}\times\ent_{+}^{K}$.\\

El sistema aqu\'i descrito debe de cumplir con los siguientes supuestos b\'asicos de un sistema de visitas:

Antes enunciemos los supuestos que regir\'an en la red.

\begin{itemize}
\item[A1)] $\xi_{1},\ldots,\xi_{K},\eta_{1},\ldots,\eta_{K}$ son
mutuamente independientes y son sucesiones independientes e
id\'enticamente distribuidas.

\item[A2)] Para alg\'un entero $p\geq1$
\begin{eqnarray*}
\esp\left[\xi_{l}\left(1\right)^{p+1}\right]<\infty\textrm{ para }l\in\mathcal{A}\textrm{ y }\\
\esp\left[\eta_{k}\left(1\right)^{p+1}\right]<\infty\textrm{ para
}k=1,\ldots,K.
\end{eqnarray*}
donde $\mathcal{A}$ es la clase de posibles arribos.

\item[A3)] Para $k=1,2,\ldots,K$ existe una funci\'on positiva
$q_{k}\left(x\right)$ definida en $\rea_{+}$, y un entero $j_{k}$,
tal que
\begin{eqnarray}
P\left(\xi_{k}\left(1\right)\geq x\right)>0\textrm{, para todo }x>0\\
P\left\{a\leq\sum_{i=1}^{j_{k}}\xi_{k}\left(i\right)\leq
b\right\}\geq\int_{a}^{b}q_{k}\left(x\right)dx, \textrm{ }0\leq
a<b.
\end{eqnarray}
\end{itemize}

En particular los procesos de tiempo entre arribos y de servicio
considerados con fines de ilustraci\'on de la metodolog\'ia
cumplen con el supuesto $A2)$ para $p=1$, es decir, ambos procesos
tienen primer y segundo momento finito.

En lo que respecta al supuesto (A3), en Dai y Meyn \cite{DaiSean}
hacen ver que este se puede sustituir por

\begin{itemize}
\item[A3')] Para el Proceso de Markov $X$, cada subconjunto
compacto de $X$ es un conjunto peque\~no, ver definici\'on
\ref{Def.Cto.Peq.}.
\end{itemize}

Es por esta raz\'on que con la finalidad de poder hacer uso de
$A3^{'})$ es necesario recurrir a los Procesos de Harris y en
particular a los Procesos Harris Recurrente:
%_______________________________________________________________________
\subsection{Procesos Harris Recurrente}
%_______________________________________________________________________

Por el supuesto (A1) conforme a Davis \cite{Davis}, se puede
definir el proceso de saltos correspondiente de manera tal que
satisfaga el supuesto (\ref{Sup3.1.Davis}), de hecho la
demostraci\'on est\'a basada en la l\'inea de argumentaci\'on de
Davis, (\cite{Davis}, p\'aginas 362-364).

Entonces se tiene un espacio de estados Markoviano. El espacio de
Markov descrito en Dai y Meyn \cite{DaiSean}

\[\left(\Omega,\mathcal{F},\mathcal{F}_{t},X\left(t\right),\theta_{t},P_{x}\right)\]
es un proceso de Borel Derecho (Sharpe \cite{Sharpe}) en el
espacio de estados medible $\left(X,\mathcal{B}_{X}\right)$. El
Proceso $X=\left\{X\left(t\right),t\geq0\right\}$ tiene
trayectorias continuas por la derecha, est\'a definida en
$\left(\Omega,\mathcal{F}\right)$ y est\'a adaptado a
$\left\{\mathcal{F}_{t},t\geq0\right\}$; la colecci\'on
$\left\{P_{x},x\in \mathbb{X}\right\}$ son medidas de probabilidad
en $\left(\Omega,\mathcal{F}\right)$ tales que para todo $x\in
\mathbb{X}$
\[P_{x}\left\{X\left(0\right)=x\right\}=1\] y
\[E_{x}\left\{f\left(X\circ\theta_{t}\right)|\mathcal{F}_{t}\right\}=E_{X}\left(\tau\right)f\left(X\right)\]
en $\left\{\tau<\infty\right\}$, $P_{x}$-c.s. Donde $\tau$ es un
$\mathcal{F}_{t}$-tiempo de paro
\[\left(X\circ\theta_{\tau}\right)\left(w\right)=\left\{X\left(\tau\left(w\right)+t,w\right),t\geq0\right\}\]
y $f$ es una funci\'on de valores reales acotada y medible con la
$\sigma$-algebra de Kolmogorov generada por los cilindros.\\

Sea $P^{t}\left(x,D\right)$, $D\in\mathcal{B}_{\mathbb{X}}$,
$t\geq0$ probabilidad de transici\'on de $X$ definida como
\[P^{t}\left(x,D\right)=P_{x}\left(X\left(t\right)\in
D\right)\]


\begin{Def}
Una medida no cero $\pi$ en
$\left(\mathbf{X},\mathcal{B}_{\mathbf{X}}\right)$ es {\bf
invariante} para $X$ si $\pi$ es $\sigma$-finita y
\[\pi\left(D\right)=\int_{\mathbf{X}}P^{t}\left(x,D\right)\pi\left(dx\right)\]
para todo $D\in \mathcal{B}_{\mathbf{X}}$, con $t\geq0$.
\end{Def}

\begin{Def}
El proceso de Markov $X$ es llamado Harris recurrente si existe
una medida de probabilidad $\nu$ en
$\left(\mathbf{X},\mathcal{B}_{\mathbf{X}}\right)$, tal que si
$\nu\left(D\right)>0$ y $D\in\mathcal{B}_{\mathbf{X}}$
\[P_{x}\left\{\tau_{D}<\infty\right\}\equiv1\] cuando
$\tau_{D}=inf\left\{t\geq0:X_{t}\in D\right\}$.
\end{Def}

\begin{Note}
\begin{itemize}
\item[i)] Si $X$ es Harris recurrente, entonces existe una \'unica
medida invariante $\pi$ (Getoor \cite{Getoor}).

\item[ii)] Si la medida invariante es finita, entonces puede
normalizarse a una medida de probabilidad, en este caso se le
llama Proceso {\em Harris recurrente positivo}.


\item[iii)] Cuando $X$ es Harris recurrente positivo se dice que
la disciplina de servicio es estable. En este caso $\pi$ denota la
distribuci\'on estacionaria y hacemos
\[P_{\pi}\left(\cdot\right)=\int_{\mathbf{X}}P_{x}\left(\cdot\right)\pi\left(dx\right)\]
y se utiliza $E_{\pi}$ para denotar el operador esperanza
correspondiente.
\end{itemize}
\end{Note}

\begin{Def}\label{Def.Cto.Peq.}
Un conjunto $D\in\mathcal{B_{\mathbf{X}}}$ es llamado peque\~no si
existe un $t>0$, una medida de probabilidad $\nu$ en
$\mathcal{B_{\mathbf{X}}}$, y un $\delta>0$ tal que
\[P^{t}\left(x,A\right)\geq\delta\nu\left(A\right)\] para $x\in
D,A\in\mathcal{B_{X}}$.
\end{Def}

La siguiente serie de resultados vienen enunciados y demostrados
en Dai \cite{Dai}:
\begin{Lema}[Lema 3.1, Dai\cite{Dai}]
Sea $B$ conjunto peque\~no cerrado, supongamos que
$P_{x}\left(\tau_{B}<\infty\right)\equiv1$ y que para alg\'un
$\delta>0$ se cumple que
\begin{equation}\label{Eq.3.1}
\sup\esp_{x}\left[\tau_{B}\left(\delta\right)\right]<\infty,
\end{equation}
donde
$\tau_{B}\left(\delta\right)=inf\left\{t\geq\delta:X\left(t\right)\in
B\right\}$. Entonces, $X$ es un proceso Harris Recurrente
Positivo.
\end{Lema}

\begin{Lema}[Lema 3.1, Dai \cite{Dai}]\label{Lema.3.}
Bajo el supuesto (A3), el conjunto $B=\left\{|x|\leq k\right\}$ es
un conjunto peque\~no cerrado para cualquier $k>0$.
\end{Lema}

\begin{Teo}[Teorema 3.1, Dai\cite{Dai}]\label{Tma.3.1}
Si existe un $\delta>0$ tal que
\begin{equation}
lim_{|x|\rightarrow\infty}\frac{1}{|x|}\esp|X^{x}\left(|x|\delta\right)|=0,
\end{equation}
entonces la ecuaci\'on (\ref{Eq.3.1}) se cumple para
$B=\left\{|x|\leq k\right\}$ con alg\'un $k>0$. En particular, $X$
es Harris Recurrente Positivo.
\end{Teo}

\begin{Note}
En Meyn and Tweedie \cite{MeynTweedie} muestran que si
$P_{x}\left\{\tau_{D}<\infty\right\}\equiv1$ incluso para solo un
conjunto peque\~no, entonces el proceso es Harris Recurrente.
\end{Note}

Entonces, tenemos que el proceso $X$ es un proceso de Markov que
cumple con los supuestos $A1)$-$A3)$, lo que falta de hacer es
construir el Modelo de Flujo bas\'andonos en lo hasta ahora
presentado.
%_______________________________________________________________________
\subsection{Modelo de Flujo}
%_______________________________________________________________________

Dada una condici\'on inicial $x\in\textrm{X}$, sea
$Q_{k}^{x}\left(t\right)$ la longitud de la cola al tiempo $t$,
$T_{m,k}^{x}\left(t\right)$ el tiempo acumulado, al tiempo $t$,
que tarda el servidor $m$ en atender a los usuarios de la cola
$k$. Finalmente sea $T_{m,k}^{x,0}\left(t\right)$ el tiempo
acumulado, al tiempo $t$, que tarda el servidor $m$ en trasladarse
a otra cola a partir de la $k$-\'esima.\\

Sup\'ongase que la funci\'on
$\left(\overline{Q}\left(\cdot\right),\overline{T}_{m}
\left(\cdot\right),\overline{T}_{m}^{0} \left(\cdot\right)\right)$
para $m=1,2,\ldots,M$ es un punto l\'imite de
\begin{equation}\label{Eq.Punto.Limite}
\left(\frac{1}{|x|}Q^{x}\left(|x|t\right),\frac{1}{|x|}T_{m}^{x}\left(|x|t\right),\frac{1}{|x|}T_{m}^{x,0}\left(|x|t\right)\right)
\end{equation}
para $m=1,2,\ldots,M$, cuando $x\rightarrow\infty$. Entonces
$\left(\overline{Q}\left(t\right),\overline{T}_{m}
\left(t\right),\overline{T}_{m}^{0} \left(t\right)\right)$ es un
flujo l\'imite del sistema. Al conjunto de todos las posibles
flujos l\'imite se le llama \textbf{Modelo de Flujo}.\\

El modelo de flujo satisface el siguiente conjunto de ecuaciones:

\begin{equation}\label{Eq.MF.1}
\overline{Q}_{k}\left(t\right)=\overline{Q}_{k}\left(0\right)+\lambda_{k}t-\sum_{m=1}^{M}\mu_{k}\overline{T}_{m,k}\left(t\right)\\
\end{equation}
para $k=1,2,\ldots,K$.\\
\begin{equation}\label{Eq.MF.2}
\overline{Q}_{k}\left(t\right)\geq0\textrm{ para
}k=1,2,\ldots,K,\\
\end{equation}

\begin{equation}\label{Eq.MF.3}
\overline{T}_{m,k}\left(0\right)=0,\textrm{ y }\overline{T}_{m,k}\left(\cdot\right)\textrm{ es no decreciente},\\
\end{equation}
para $k=1,2,\ldots,K$ y $m=1,2,\ldots,M$,\\
\begin{equation}\label{Eq.MF.4}
\sum_{k=1}^{K}\overline{T}_{m,k}^{0}\left(t\right)+\overline{T}_{m,k}\left(t\right)=t\textrm{
para }m=1,2,\ldots,M.\\
\end{equation}

De acuerdo a Dai \cite{Dai}, se tiene que el conjunto de posibles
l\'imites
$\left(\overline{Q}\left(\cdot\right),\overline{T}\left(\cdot\right),\overline{T}^{0}\left(\cdot\right)\right)$,
en el sentido de que deben de satisfacer las ecuaciones
(\ref{Eq.MF.1})-(\ref{Eq.MF.4}), se le llama {\em Modelo de
Flujo}.


\begin{Def}[Definici\'on 4.1, , Dai \cite{Dai}]\label{Def.Modelo.Flujo}
Sea una disciplina de servicio espec\'ifica. Cualquier l\'imite
$\left(\overline{Q}\left(\cdot\right),\overline{T}\left(\cdot\right)\right)$
en (\ref{Eq.Punto.Limite}) es un {\em flujo l\'imite} de la
disciplina. Cualquier soluci\'on (\ref{Eq.MF.1})-(\ref{Eq.MF.4})
es llamado flujo soluci\'on de la disciplina. Se dice que el
modelo de flujo l\'imite, modelo de flujo, de la disciplina de la
cola es estable si existe una constante $\delta>0$ que depende de
$\mu,\lambda$ y $P$ solamente, tal que cualquier flujo l\'imite
con
$|\overline{Q}\left(0\right)|+|\overline{U}|+|\overline{V}|=1$, se
tiene que $\overline{Q}\left(\cdot+\delta\right)\equiv0$.
\end{Def}

Al conjunto de ecuaciones dadas en \ref{Eq.MF.1}-\ref{Eq.MF.4} se
le llama {\em Modelo de flujo} y al conjunto de todas las
soluciones del modelo de flujo
$\left(\overline{Q}\left(\cdot\right),\overline{T}
\left(\cdot\right)\right)$ se le denotar\'a por $\mathcal{Q}$.

Si se hace $|x|\rightarrow\infty$ sin restringir ninguna de las
componentes, tambi\'en se obtienen un modelo de flujo, pero en
este caso el residual de los procesos de arribo y servicio
introducen un retraso:
\begin{Teo}[Teorema 4.2, Dai\cite{Dai}]\label{Tma.4.2.Dai}
Sea una disciplina fija para la cola, suponga que se cumplen las
condiciones (A1))-(A3)). Si el modelo de flujo l\'imite de la
disciplina de la cola es estable, entonces la cadena de Markov $X$
que describe la din\'amica de la red bajo la disciplina es Harris
recurrente positiva.
\end{Teo}

Ahora se procede a escalar el espacio y el tiempo para reducir la
aparente fluctuaci\'on del modelo. Consid\'erese el proceso
\begin{equation}\label{Eq.3.7}
\overline{Q}^{x}\left(t\right)=\frac{1}{|x|}Q^{x}\left(|x|t\right)
\end{equation}
A este proceso se le conoce como el fluido escalado, y cualquier
l\'imite $\overline{Q}^{x}\left(t\right)$ es llamado flujo
l\'imite del proceso de longitud de la cola. Haciendo
$|q|\rightarrow\infty$ mientras se mantiene el resto de las
componentes fijas, cualquier punto l\'imite del proceso de
longitud de la cola normalizado $\overline{Q}^{x}$ es soluci\'on
del siguiente modelo de flujo.


\begin{Def}[Definici\'on 3.3, Dai y Meyn \cite{DaiSean}]
El modelo de flujo es estable si existe un tiempo fijo $t_{0}$ tal
que $\overline{Q}\left(t\right)=0$, con $t\geq t_{0}$, para
cualquier $\overline{Q}\left(\cdot\right)\in\mathcal{Q}$ que
cumple con $|\overline{Q}\left(0\right)|=1$.
\end{Def}

El siguiente resultado se encuentra en Chen \cite{Chen}.
\begin{Lemma}[Lema 3.1, Dai y Meyn \cite{DaiSean}]
Si el modelo de flujo definido por \ref{Eq.MF.1}-\ref{Eq.MF.4} es
estable, entonces el modelo de flujo retrasado es tambi\'en
estable, es decir, existe $t_{0}>0$ tal que
$\overline{Q}\left(t\right)=0$ para cualquier $t\geq t_{0}$, para
cualquier soluci\'on del modelo de flujo retrasado cuya
condici\'on inicial $\overline{x}$ satisface que
$|\overline{x}|=|\overline{Q}\left(0\right)|+|\overline{A}\left(0\right)|+|\overline{B}\left(0\right)|\leq1$.
\end{Lemma}


Ahora ya estamos en condiciones de enunciar los resultados principales:


\begin{Teo}[Teorema 2.1, Down \cite{Down}]\label{Tma2.1.Down}
Suponga que el modelo de flujo es estable, y que se cumplen los supuestos (A1) y (A2), entonces
\begin{itemize}
\item[i)] Para alguna constante $\kappa_{p}$, y para cada
condici\'on inicial $x\in X$
\begin{equation}\label{Estability.Eq1}
limsup_{t\rightarrow\infty}\frac{1}{t}\int_{0}^{t}\esp_{x}\left[|Q\left(s\right)|^{p}\right]ds\leq\kappa_{p},
\end{equation}
donde $p$ es el entero dado en (A2).
\end{itemize}
Si adem\'as se cumple la condici\'on (A3), entonces para cada
condici\'on inicial:
\begin{itemize}
\item[ii)] Los momentos transitorios convergen a su estado
estacionario:
 \begin{equation}\label{Estability.Eq2}
lim_{t\rightarrow\infty}\esp_{x}\left[Q_{k}\left(t\right)^{r}\right]=\esp_{\pi}\left[Q_{k}\left(0\right)^{r}\right]\leq\kappa_{r},
\end{equation}
para $r=1,2,\ldots,p$ y $k=1,2,\ldots,K$. Donde $\pi$ es la
probabilidad invariante para $\mathbf{X}$.

\item[iii)]  El primer momento converge con raz\'on $t^{p-1}$:
\begin{equation}\label{Estability.Eq3}
lim_{t\rightarrow\infty}t^{p-1}|\esp_{x}\left[Q_{k}\left(t\right)\right]-\esp_{\pi}\left[Q_{k}\left(0\right)\right]=0.
\end{equation}

\item[iv)] La {\em Ley Fuerte de los grandes n\'umeros} se cumple:
\begin{equation}\label{Estability.Eq4}
lim_{t\rightarrow\infty}\frac{1}{t}\int_{0}^{t}Q_{k}^{r}\left(s\right)ds=\esp_{\pi}\left[Q_{k}\left(0\right)^{r}\right],\textrm{
}\prob_{x}\textrm{-c.s.}
\end{equation}
para $r=1,2,\ldots,p$ y $k=1,2,\ldots,K$.
\end{itemize}
\end{Teo}

La contribuci\'on de Down a la teor\'ia de los Sistemas de Visitas
C\'iclicas, es la relaci\'on que hay entre la estabilidad del
sistema con el comportamiento de las medidas de desempe\~no, es
decir, la condici\'on suficiente para poder garantizar la
convergencia del proceso de la longitud de la cola as\'i como de
por los menos los dos primeros momentos adem\'as de una versi\'on
de la Ley Fuerte de los Grandes N\'umeros para los sistemas de
visitas.


\begin{Teo}[Teorema 2.3, Down \cite{Down}]\label{Tma2.3.Down}
Considere el siguiente valor:
\begin{equation}\label{Eq.Rho.1serv}
\rho=\sum_{k=1}^{K}\rho_{k}+max_{1\leq j\leq K}\left(\frac{\lambda_{j}}{\sum_{s=1}^{S}p_{js}\overline{N}_{s}}\right)\delta^{*}
\end{equation}
\begin{itemize}
\item[i)] Si $\rho<1$ entonces la red es estable, es decir, se cumple el teorema \ref{Tma2.1.Down}.

\item[ii)] Si $\rho<1$ entonces la red es inestable, es decir, se cumple el teorema \ref{Tma2.2.Down}
\end{itemize}
\end{Teo}

\begin{Teo}
Sea $\left(X_{n},\mathcal{F}_{n},n=0,1,\ldots,\right\}$ Proceso de
Markov con espacio de estados $\left(S_{0},\chi_{0}\right)$
generado por una distribuici\'on inicial $P_{o}$ y probabilidad de
transici\'on $p_{mn}$, para $m,n=0,1,\ldots,$ $m<n$, que por
notaci\'on se escribir\'a como $p\left(m,n,x,B\right)\rightarrow
p_{mn}\left(x,B\right)$. Sea $S$ tiempo de paro relativo a la
$\sigma$-\'algebra $\mathcal{F}_{n}$. Sea $T$ funci\'on medible,
$T:\Omega\rightarrow\left\{0,1,\ldots,\right\}$. Sup\'ongase que
$T\geq S$, entonces $T$ es tiempo de paro. Si $B\in\chi_{0}$,
entonces
\begin{equation}\label{Prop.Fuerte.Markov}
P\left\{X\left(T\right)\in
B,T<\infty|\mathcal{F}\left(S\right)\right\} =
p\left(S,T,X\left(s\right),B\right)
\end{equation}
en $\left\{T<\infty\right\}$.
\end{Teo}


Sea $K$ conjunto numerable y sea $d:K\rightarrow\nat$ funci\'on.
Para $v\in K$, $M_{v}$ es un conjunto abierto de
$\rea^{d\left(v\right)}$. Entonces \[E=\cup_{v\in
K}M_{v}=\left\{\left(v,\zeta\right):v\in K,\zeta\in
M_{v}\right\}.\]

Sea $\mathcal{E}$ la clase de conjuntos medibles en $E$:
\[\mathcal{E}=\left\{\cup_{v\in K}A_{v}:A_{v}\in \mathcal{M}_{v}\right\}.\]

donde $\mathcal{M}$ son los conjuntos de Borel de $M_{v}$.
Entonces $\left(E,\mathcal{E}\right)$ es un espacio de Borel. El
estado del proceso se denotar\'a por
$\mathbf{x}_{t}=\left(v_{t},\zeta_{t}\right)$. La distribuci\'on
de $\left(\mathbf{x}_{t}\right)$ est\'a determinada por por los
siguientes objetos:

\begin{itemize}
\item[i)] Los campos vectoriales $\left(\mathcal{H}_{v},v\in
K\right)$. \item[ii)] Una funci\'on medible $\lambda:E\rightarrow
\rea_{+}$. \item[iii)] Una medida de transici\'on
$Q:\mathcal{E}\times\left(E\cup\Gamma^{*}\right)\rightarrow\left[0,1\right]$
donde
\begin{equation}
\Gamma^{*}=\cup_{v\in K}\partial^{*}M_{v}.
\end{equation}
y
\begin{equation}
\partial^{*}M_{v}=\left\{z\in\partial M_{v}:\mathbf{\mathbf{\phi}_{v}\left(t,\zeta\right)=\mathbf{z}}\textrm{ para alguna }\left(t,\zeta\right)\in\rea_{+}\times M_{v}\right\}.
\end{equation}
$\partial M_{v}$ denota  la frontera de $M_{v}$.
\end{itemize}

El campo vectorial $\left(\mathcal{H}_{v},v\in K\right)$ se supone
tal que para cada $\mathbf{z}\in M_{v}$ existe una \'unica curva
integral $\mathbf{\phi}_{v}\left(t,\zeta\right)$ que satisface la
ecuaci\'on

\begin{equation}
\frac{d}{dt}f\left(\zeta_{t}\right)=\mathcal{H}f\left(\zeta_{t}\right),
\end{equation}
con $\zeta_{0}=\mathbf{z}$, para cualquier funci\'on suave
$f:\rea^{d}\rightarrow\rea$ y $\mathcal{H}$ denota el operador
diferencial de primer orden, con $\mathcal{H}=\mathcal{H}_{v}$ y
$\zeta_{t}=\mathbf{\phi}\left(t,\mathbf{z}\right)$. Adem\'as se
supone que $\mathcal{H}_{v}$ es conservativo, es decir, las curvas
integrales est\'an definidas para todo $t>0$.

Para $\mathbf{x}=\left(v,\zeta\right)\in E$ se denota
\[t^{*}\mathbf{x}=inf\left\{t>0:\mathbf{\phi}_{v}\left(t,\zeta\right)\in\partial^{*}M_{v}\right\}\]

En lo que respecta a la funci\'on $\lambda$, se supondr\'a que
para cada $\left(v,\zeta\right)\in E$ existe un $\epsilon>0$ tal
que la funci\'on
$s\rightarrow\lambda\left(v,\phi_{v}\left(s,\zeta\right)\right)\in
E$ es integrable para $s\in\left[0,\epsilon\right)$. La medida de
transici\'on $Q\left(A;\mathbf{x}\right)$ es una funci\'on medible
de $\mathbf{x}$ para cada $A\in\mathcal{E}$, definida para
$\mathbf{x}\in E\cup\Gamma^{*}$ y es una medida de probabilidad en
$\left(E,\mathcal{E}\right)$ para cada $\mathbf{x}\in E$.

El movimiento del proceso $\left(\mathbf{x}_{t}\right)$ comenzando
en $\mathbf{x}=\left(n,\mathbf{z}\right)\in E$ se puede construir
de la siguiente manera, def\'inase la funci\'on $F$ por

\begin{equation}
F\left(t\right)=\left\{\begin{array}{ll}\\
exp\left(-\int_{0}^{t}\lambda\left(n,\phi_{n}\left(s,\mathbf{z}\right)\right)ds\right), & t<t^{*}\left(\mathbf{x}\right),\\
0, & t\geq t^{*}\left(\mathbf{x}\right)
\end{array}\right.
\end{equation}

Sea $T_{1}$ una variable aleatoria tal que
$\prob\left[T_{1}>t\right]=F\left(t\right)$, ahora sea la variable
aleatoria $\left(N,Z\right)$ con distribuici\'on
$Q\left(\cdot;\phi_{n}\left(T_{1},\mathbf{z}\right)\right)$. La
trayectoria de $\left(\mathbf{x}_{t}\right)$ para $t\leq T_{1}$
es\footnote{Revisar p\'agina 362, y 364 de Davis \cite{Davis}.}
\begin{eqnarray*}
\mathbf{x}_{t}=\left(v_{t},\zeta_{t}\right)=\left\{\begin{array}{ll}
\left(n,\phi_{n}\left(t,\mathbf{z}\right)\right), & t<T_{1},\\
\left(N,\mathbf{Z}\right), & t=t_{1}.
\end{array}\right.
\end{eqnarray*}

Comenzando en $\mathbf{x}_{T_{1}}$ se selecciona el siguiente
tiempo de intersalto $T_{2}-T_{1}$ lugar del post-salto
$\mathbf{x}_{T_{2}}$ de manera similar y as\'i sucesivamente. Este
procedimiento nos da una trayectoria determinista por partes
$\mathbf{x}_{t}$ con tiempos de salto $T_{1},T_{2},\ldots$. Bajo
las condiciones enunciadas para $\lambda,T_{1}>0$  y
$T_{1}-T_{2}>0$ para cada $i$, con probabilidad 1. Se supone que
se cumple la siguiente condici\'on.

\begin{Sup}[Supuesto 3.1, Davis \cite{Davis}]\label{Sup3.1.Davis}
Sea $N_{t}:=\sum_{t}\indora_{\left(t\geq t\right)}$ el n\'umero de
saltos en $\left[0,t\right]$. Entonces
\begin{equation}
\esp\left[N_{t}\right]<\infty\textrm{ para toda }t.
\end{equation}
\end{Sup}

es un proceso de Markov, m\'as a\'un, es un Proceso Fuerte de
Markov, es decir, la Propiedad Fuerte de Markov se cumple para
cualquier tiempo de paro.


Sea $E$ es un espacio m\'etrico separable y la m\'etrica $d$ es
compatible con la topolog\'ia.


\begin{Def}
Un espacio topol\'ogico $E$ es llamado de {\em Rad\'on} si es
homeomorfo a un subconjunto universalmente medible de un espacio
m\'etrico compacto.
\end{Def}

Equivalentemente, la definici\'on de un espacio de Rad\'on puede
encontrarse en los siguientes t\'erminos:


\begin{Def}
$E$ es un espacio de Rad\'on si cada medida finita en
$\left(E,\mathcal{B}\left(E\right)\right)$ es regular interior o
cerrada, {\em tight}.
\end{Def}

\begin{Def}
Una medida finita, $\lambda$ en la $\sigma$-\'algebra de Borel de
un espacio metrizable $E$ se dice cerrada si
\begin{equation}\label{Eq.A2.3}
\lambda\left(E\right)=sup\left\{\lambda\left(K\right):K\textrm{ es
compacto en }E\right\}.
\end{equation}
\end{Def}

El siguiente teorema nos permite tener una mejor caracterizaci\'on
de los espacios de Rad\'on:
\begin{Teo}\label{Tma.A2.2}
Sea $E$ espacio separable metrizable. Entonces $E$ es Radoniano si
y s\'olo s\'i cada medida finita en
$\left(E,\mathcal{B}\left(E\right)\right)$ es cerrada.
\end{Teo}

Sea $E$ espacio de estados, tal que $E$ es un espacio de Rad\'on,
$\mathcal{B}\left(E\right)$ $\sigma$-\'algebra de Borel en $E$,
que se denotar\'a por $\mathcal{E}$.

Sea $\left(X,\mathcal{G},\prob\right)$ espacio de probabilidad,
$I\subset\rea$ conjunto de \'indices. Sea $\mathcal{F}_{\leq t}$
la $\sigma$-\'algebra natural definida como
$\sigma\left\{f\left(X_{r}\right):r\in I, r\leq
t,f\in\mathcal{E}\right\}$. Se considerar\'a una
$\sigma$-\'algebra m\'as general, $ \left(\mathcal{G}_{t}\right)$
tal que $\left(X_{t}\right)$ sea $\mathcal{E}$-adaptado.

\begin{Def}
Una familia $\left(P_{s,t}\right)$ de kernels de Markov en
$\left(E,\mathcal{E}\right)$ indexada por pares $s,t\in I$, con
$s\leq t$ es una funci\'on de transici\'on en $\ER$, si  para todo
$r\leq s< t$ en $I$ y todo $x\in E$,
$B\in\mathcal{E}$\footnote{Ecuaci\'on de Chapman-Kolmogorov}
\begin{equation}\label{Eq.Kernels}
P_{r,t}\left(x,B\right)=\int_{E}P_{r,s}\left(x,dy\right)P_{s,t}\left(y,B\right).
\end{equation}
\end{Def}

Se dice que la funci\'on de transici\'on $\KM$ en $\ER$ es la
funci\'on de transici\'on para un proceso $\PE$  con valores en
$E$ y que satisface la propiedad de
Markov\footnote{\begin{equation}\label{Eq.1.4.S}
\prob\left\{H|\mathcal{G}_{t}\right\}=\prob\left\{H|X_{t}\right\}\textrm{
}H\in p\mathcal{F}_{\geq t}.
\end{equation}} (\ref{Eq.1.4.S}) relativa a $\left(\mathcal{G}_{t}\right)$ si

\begin{equation}\label{Eq.1.6.S}
\prob\left\{f\left(X_{t}\right)|\mathcal{G}_{s}\right\}=P_{s,t}f\left(X_{t}\right)\textrm{
}s\leq t\in I,\textrm{ }f\in b\mathcal{E}.
\end{equation}

\begin{Def}
Una familia $\left(P_{t}\right)_{t\geq0}$ de kernels de Markov en
$\ER$ es llamada {\em Semigrupo de Transici\'on de Markov} o {\em
Semigrupo de Transici\'on} si
\[P_{t+s}f\left(x\right)=P_{t}\left(P_{s}f\right)\left(x\right),\textrm{ }t,s\geq0,\textrm{ }x\in E\textrm{ }f\in b\mathcal{E}.\]
\end{Def}
\begin{Note}
Si la funci\'on de transici\'on $\KM$ es llamada homog\'enea si
$P_{s,t}=P_{t-s}$.
\end{Note}

Un proceso de Markov que satisface la ecuaci\'on (\ref{Eq.1.6.S})
con funci\'on de transici\'on homog\'enea $\left(P_{t}\right)$
tiene la propiedad caracter\'istica
\begin{equation}\label{Eq.1.8.S}
\prob\left\{f\left(X_{t+s}\right)|\mathcal{G}_{t}\right\}=P_{s}f\left(X_{t}\right)\textrm{
}t,s\geq0,\textrm{ }f\in b\mathcal{E}.
\end{equation}
La ecuaci\'on anterior es la {\em Propiedad Simple de Markov} de
$X$ relativa a $\left(P_{t}\right)$.

En este sentido el proceso $\PE$ cumple con la propiedad de Markov
(\ref{Eq.1.8.S}) relativa a
$\left(\Omega,\mathcal{G},\mathcal{G}_{t},\prob\right)$ con
semigrupo de transici\'on $\left(P_{t}\right)$.

\begin{Def}
Un proceso estoc\'astico $\PE$ definido en
$\left(\Omega,\mathcal{G},\prob\right)$ con valores en el espacio
topol\'ogico $E$ es continuo por la derecha si cada trayectoria
muestral $t\rightarrow X_{t}\left(w\right)$ es un mapeo continuo
por la derecha de $I$ en $E$.
\end{Def}

\begin{Def}[HD1]\label{Eq.2.1.S}
Un semigrupo de Markov $\left(P_{t}\right)$ en un espacio de
Rad\'on $E$ se dice que satisface la condici\'on {\em HD1} si,
dada una medida de probabilidad $\mu$ en $E$, existe una
$\sigma$-\'algebra $\mathcal{E^{*}}$ con
$\mathcal{E}\subset\mathcal{E}^{*}$ y
$P_{t}\left(b\mathcal{E}^{*}\right)\subset b\mathcal{E}^{*}$, y un
$\mathcal{E}^{*}$-proceso $E$-valuado continuo por la derecha
$\PE$ en alg\'un espacio de probabilidad filtrado
$\left(\Omega,\mathcal{G},\mathcal{G}_{t},\prob\right)$ tal que
$X=\left(\Omega,\mathcal{G},\mathcal{G}_{t},\prob\right)$ es de
Markov (Homog\'eneo) con semigrupo de transici\'on $(P_{t})$ y
distribuci\'on inicial $\mu$.
\end{Def}

Consid\'erese la colecci\'on de variables aleatorias $X_{t}$
definidas en alg\'un espacio de probabilidad, y una colecci\'on de
medidas $\mathbf{P}^{x}$ tales que
$\mathbf{P}^{x}\left\{X_{0}=x\right\}$, y bajo cualquier
$\mathbf{P}^{x}$, $X_{t}$ es de Markov con semigrupo
$\left(P_{t}\right)$. $\mathbf{P}^{x}$ puede considerarse como la
distribuci\'on condicional de $\mathbf{P}$ dado $X_{0}=x$.

\begin{Def}\label{Def.2.2.S}
Sea $E$ espacio de Rad\'on, $\SG$ semigrupo de Markov en $\ER$. La
colecci\'on
$\mathbf{X}=\left(\Omega,\mathcal{G},\mathcal{G}_{t},X_{t},\theta_{t},\CM\right)$
es un proceso $\mathcal{E}$-Markov continuo por la derecha simple,
con espacio de estados $E$ y semigrupo de transici\'on $\SG$ en
caso de que $\mathbf{X}$ satisfaga las siguientes condiciones:
\begin{itemize}
\item[i)] $\left(\Omega,\mathcal{G},\mathcal{G}_{t}\right)$ es un
espacio de medida filtrado, y $X_{t}$ es un proceso $E$-valuado
continuo por la derecha $\mathcal{E}^{*}$-adaptado a
$\left(\mathcal{G}_{t}\right)$;

\item[ii)] $\left(\theta_{t}\right)_{t\geq0}$ es una colecci\'on
de operadores {\em shift} para $X$, es decir, mapea $\Omega$ en
s\'i mismo satisfaciendo para $t,s\geq0$,

\begin{equation}\label{Eq.Shift}
\theta_{t}\circ\theta_{s}=\theta_{t+s}\textrm{ y
}X_{t}\circ\theta_{t}=X_{t+s};
\end{equation}

\item[iii)] Para cualquier $x\in E$,$\CM\left\{X_{0}=x\right\}=1$,
y el proceso $\PE$ tiene la propiedad de Markov (\ref{Eq.1.8.S})
con semigrupo de transici\'on $\SG$ relativo a
$\left(\Omega,\mathcal{G},\mathcal{G}_{t},\CM\right)$.
\end{itemize}
\end{Def}

\begin{Def}[HD2]\label{Eq.2.2.S}
Para cualquier $\alpha>0$ y cualquier $f\in S^{\alpha}$, el
proceso $t\rightarrow f\left(X_{t}\right)$ es continuo por la
derecha casi seguramente.
\end{Def}

\begin{Def}\label{Def.PD}
Un sistema
$\mathbf{X}=\left(\Omega,\mathcal{G},\mathcal{G}_{t},X_{t},\theta_{t},\CM\right)$
es un proceso derecho en el espacio de Rad\'on $E$ con semigrupo
de transici\'on $\SG$ provisto de:
\begin{itemize}
\item[i)] $\mathbf{X}$ es una realizaci\'on  continua por la
derecha, \ref{Def.2.2.S}, de $\SG$.

\item[ii)] $\mathbf{X}$ satisface la condicion HD2,
\ref{Eq.2.2.S}, relativa a $\mathcal{G}_{t}$.

\item[iii)] $\mathcal{G}_{t}$ es aumentado y continuo por la
derecha.
\end{itemize}
\end{Def}

\begin{Lema}[Lema 4.2, Dai\cite{Dai}]\label{Lema4.2}
Sea $\left\{x_{n}\right\}\subset \mathbf{X}$ con
$|x_{n}|\rightarrow\infty$, conforme $n\rightarrow\infty$. Suponga
que
\[lim_{n\rightarrow\infty}\frac{1}{|x_{n}|}U\left(0\right)=\overline{U}\]
y
\[lim_{n\rightarrow\infty}\frac{1}{|x_{n}|}V\left(0\right)=\overline{V}.\]

Entonces, conforme $n\rightarrow\infty$, casi seguramente

\begin{equation}\label{E1.4.2}
\frac{1}{|x_{n}|}\Phi^{k}\left(\left[|x_{n}|t\right]\right)\rightarrow
P_{k}^{'}t\textrm{, u.o.c.,}
\end{equation}

\begin{equation}\label{E1.4.3}
\frac{1}{|x_{n}|}E^{x_{n}}_{k}\left(|x_{n}|t\right)\rightarrow
\alpha_{k}\left(t-\overline{U}_{k}\right)^{+}\textrm{, u.o.c.,}
\end{equation}

\begin{equation}\label{E1.4.4}
\frac{1}{|x_{n}|}S^{x_{n}}_{k}\left(|x_{n}|t\right)\rightarrow
\mu_{k}\left(t-\overline{V}_{k}\right)^{+}\textrm{, u.o.c.,}
\end{equation}

donde $\left[t\right]$ es la parte entera de $t$ y
$\mu_{k}=1/m_{k}=1/\esp\left[\eta_{k}\left(1\right)\right]$.
\end{Lema}

\begin{Lema}[Lema 4.3, Dai\cite{Dai}]\label{Lema.4.3}
Sea $\left\{x_{n}\right\}\subset \mathbf{X}$ con
$|x_{n}|\rightarrow\infty$, conforme $n\rightarrow\infty$. Suponga
que
\[lim_{n\rightarrow\infty}\frac{1}{|x_{n}|}U\left(0\right)=\overline{U}_{k}\]
y
\[lim_{n\rightarrow\infty}\frac{1}{|x_{n}|}V\left(0\right)=\overline{V}_{k}.\]
\begin{itemize}
\item[a)] Conforme $n\rightarrow\infty$ casi seguramente,
\[lim_{n\rightarrow\infty}\frac{1}{|x_{n}|}U^{x_{n}}_{k}\left(|x_{n}|t\right)=\left(\overline{U}_{k}-t\right)^{+}\textrm{, u.o.c.}\]
y
\[lim_{n\rightarrow\infty}\frac{1}{|x_{n}|}V^{x_{n}}_{k}\left(|x_{n}|t\right)=\left(\overline{V}_{k}-t\right)^{+}.\]

\item[b)] Para cada $t\geq0$ fijo,
\[\left\{\frac{1}{|x_{n}|}U^{x_{n}}_{k}\left(|x_{n}|t\right),|x_{n}|\geq1\right\}\]
y
\[\left\{\frac{1}{|x_{n}|}V^{x_{n}}_{k}\left(|x_{n}|t\right),|x_{n}|\geq1\right\}\]
\end{itemize}
son uniformemente convergentes.
\end{Lema}

$S_{l}^{x}\left(t\right)$ es el n\'umero total de servicios
completados de la clase $l$, si la clase $l$ est\'a dando $t$
unidades de tiempo de servicio. Sea $T_{l}^{x}\left(x\right)$ el
monto acumulado del tiempo de servicio que el servidor
$s\left(l\right)$ gasta en los usuarios de la clase $l$ al tiempo
$t$. Entonces $S_{l}^{x}\left(T_{l}^{x}\left(t\right)\right)$ es
el n\'umero total de servicios completados para la clase $l$ al
tiempo $t$. Una fracci\'on de estos usuarios,
$\Phi_{l}^{x}\left(S_{l}^{x}\left(T_{l}^{x}\left(t\right)\right)\right)$,
se convierte en usuarios de la clase $k$.\\

Entonces, dado lo anterior, se tiene la siguiente representaci\'on
para el proceso de la longitud de la cola:\\

\begin{equation}
Q_{k}^{x}\left(t\right)=_{k}^{x}\left(0\right)+E_{k}^{x}\left(t\right)+\sum_{l=1}^{K}\Phi_{k}^{l}\left(S_{l}^{x}\left(T_{l}^{x}\left(t\right)\right)\right)-S_{k}^{x}\left(T_{k}^{x}\left(t\right)\right)
\end{equation}
para $k=1,\ldots,K$. Para $i=1,\ldots,d$, sea
\[I_{i}^{x}\left(t\right)=t-\sum_{j\in C_{i}}T_{k}^{x}\left(t\right).\]

Entonces $I_{i}^{x}\left(t\right)$ es el monto acumulado del
tiempo que el servidor $i$ ha estado desocupado al tiempo $t$. Se
est\'a asumiendo que las disciplinas satisfacen la ley de
conservaci\'on del trabajo, es decir, el servidor $i$ est\'a en
pausa solamente cuando no hay usuarios en la estaci\'on $i$.
Entonces, se tiene que

\begin{equation}
\int_{0}^{\infty}\left(\sum_{k\in
C_{i}}Q_{k}^{x}\left(t\right)\right)dI_{i}^{x}\left(t\right)=0,
\end{equation}
para $i=1,\ldots,d$.\\

Hacer
\[T^{x}\left(t\right)=\left(T_{1}^{x}\left(t\right),\ldots,T_{K}^{x}\left(t\right)\right)^{'},\]
\[I^{x}\left(t\right)=\left(I_{1}^{x}\left(t\right),\ldots,I_{K}^{x}\left(t\right)\right)^{'}\]
y
\[S^{x}\left(T^{x}\left(t\right)\right)=\left(S_{1}^{x}\left(T_{1}^{x}\left(t\right)\right),\ldots,S_{K}^{x}\left(T_{K}^{x}\left(t\right)\right)\right)^{'}.\]

Para una disciplina que cumple con la ley de conservaci\'on del
trabajo, en forma vectorial, se tiene el siguiente conjunto de
ecuaciones

\begin{equation}\label{Eq.MF.1.3}
Q^{x}\left(t\right)=Q^{x}\left(0\right)+E^{x}\left(t\right)+\sum_{l=1}^{K}\Phi^{l}\left(S_{l}^{x}\left(T_{l}^{x}\left(t\right)\right)\right)-S^{x}\left(T^{x}\left(t\right)\right),\\
\end{equation}

\begin{equation}\label{Eq.MF.2.3}
Q^{x}\left(t\right)\geq0,\\
\end{equation}

\begin{equation}\label{Eq.MF.3.3}
T^{x}\left(0\right)=0,\textrm{ y }\overline{T}^{x}\left(t\right)\textrm{ es no decreciente},\\
\end{equation}

\begin{equation}\label{Eq.MF.4.3}
I^{x}\left(t\right)=et-CT^{x}\left(t\right)\textrm{ es no
decreciente}\\
\end{equation}

\begin{equation}\label{Eq.MF.5.3}
\int_{0}^{\infty}\left(CQ^{x}\left(t\right)\right)dI_{i}^{x}\left(t\right)=0,\\
\end{equation}

\begin{equation}\label{Eq.MF.6.3}
\textrm{Condiciones adicionales en
}\left(\overline{Q}^{x}\left(\cdot\right),\overline{T}^{x}\left(\cdot\right)\right)\textrm{
espec\'ificas de la disciplina de la cola,}
\end{equation}

donde $e$ es un vector de unos de dimensi\'on $d$, $C$ es la
matriz definida por
\[C_{ik}=\left\{\begin{array}{cc}
1,& S\left(k\right)=i,\\
0,& \textrm{ en otro caso}.\\
\end{array}\right.
\]
Es necesario enunciar el siguiente Teorema que se utilizar\'a para
el Teorema \ref{Tma.4.2.Dai}:
\begin{Teo}[Teorema 4.1, Dai \cite{Dai}]
Considere una disciplina que cumpla la ley de conservaci\'on del
trabajo, para casi todas las trayectorias muestrales $\omega$ y
cualquier sucesi\'on de estados iniciales
$\left\{x_{n}\right\}\subset \mathbf{X}$, con
$|x_{n}|\rightarrow\infty$, existe una subsucesi\'on
$\left\{x_{n_{j}}\right\}$ con $|x_{n_{j}}|\rightarrow\infty$ tal
que
\begin{equation}\label{Eq.4.15}
\frac{1}{|x_{n_{j}}|}\left(Q^{x_{n_{j}}}\left(0\right),U^{x_{n_{j}}}\left(0\right),V^{x_{n_{j}}}\left(0\right)\right)\rightarrow\left(\overline{Q}\left(0\right),\overline{U},\overline{V}\right),
\end{equation}

\begin{equation}\label{Eq.4.16}
\frac{1}{|x_{n_{j}}|}\left(Q^{x_{n_{j}}}\left(|x_{n_{j}}|t\right),T^{x_{n_{j}}}\left(|x_{n_{j}}|t\right)\right)\rightarrow\left(\overline{Q}\left(t\right),\overline{T}\left(t\right)\right)\textrm{
u.o.c.}
\end{equation}

Adem\'as,
$\left(\overline{Q}\left(t\right),\overline{T}\left(t\right)\right)$
satisface las siguientes ecuaciones:
\begin{equation}\label{Eq.MF.1.3a}
\overline{Q}\left(t\right)=Q\left(0\right)+\left(\alpha
t-\overline{U}\right)^{+}-\left(I-P\right)^{'}M^{-1}\left(\overline{T}\left(t\right)-\overline{V}\right)^{+},
\end{equation}

\begin{equation}\label{Eq.MF.2.3a}
\overline{Q}\left(t\right)\geq0,\\
\end{equation}

\begin{equation}\label{Eq.MF.3.3a}
\overline{T}\left(t\right)\textrm{ es no decreciente y comienza en cero},\\
\end{equation}

\begin{equation}\label{Eq.MF.4.3a}
\overline{I}\left(t\right)=et-C\overline{T}\left(t\right)\textrm{
es no decreciente,}\\
\end{equation}

\begin{equation}\label{Eq.MF.5.3a}
\int_{0}^{\infty}\left(C\overline{Q}\left(t\right)\right)d\overline{I}\left(t\right)=0,\\
\end{equation}

\begin{equation}\label{Eq.MF.6.3a}
\textrm{Condiciones adicionales en
}\left(\overline{Q}\left(\cdot\right),\overline{T}\left(\cdot\right)\right)\textrm{
especficas de la disciplina de la cola,}
\end{equation}
\end{Teo}


Propiedades importantes para el modelo de flujo retrasado:

\begin{Prop}
 Sea $\left(\overline{Q},\overline{T},\overline{T}^{0}\right)$ un flujo l\'imite de \ref{Eq.4.4} y suponga que cuando $x\rightarrow\infty$ a lo largo de
una subsucesi\'on
\[\left(\frac{1}{|x|}Q_{k}^{x}\left(0\right),\frac{1}{|x|}A_{k}^{x}\left(0\right),\frac{1}{|x|}B_{k}^{x}\left(0\right),\frac{1}{|x|}B_{k}^{x,0}\left(0\right)\right)\rightarrow\left(\overline{Q}_{k}\left(0\right),0,0,0\right)\]
para $k=1,\ldots,K$. EL flujo l\'imite tiene las siguientes
propiedades, donde las propiedades de la derivada se cumplen donde
la derivada exista:
\begin{itemize}
 \item[i)] Los vectores de tiempo ocupado $\overline{T}\left(t\right)$ y $\overline{T}^{0}\left(t\right)$ son crecientes y continuas con
$\overline{T}\left(0\right)=\overline{T}^{0}\left(0\right)=0$.
\item[ii)] Para todo $t\geq0$
\[\sum_{k=1}^{K}\left[\overline{T}_{k}\left(t\right)+\overline{T}_{k}^{0}\left(t\right)\right]=t\]
\item[iii)] Para todo $1\leq k\leq K$
\[\overline{Q}_{k}\left(t\right)=\overline{Q}_{k}\left(0\right)+\alpha_{k}t-\mu_{k}\overline{T}_{k}\left(t\right)\]
\item[iv)]  Para todo $1\leq k\leq K$
\[\dot{{\overline{T}}}_{k}\left(t\right)=\beta_{k}\] para $\overline{Q}_{k}\left(t\right)=0$.
\item[v)] Para todo $k,j$
\[\mu_{k}^{0}\overline{T}_{k}^{0}\left(t\right)=\mu_{j}^{0}\overline{T}_{j}^{0}\left(t\right)\]
\item[vi)]  Para todo $1\leq k\leq K$
\[\mu_{k}\dot{{\overline{T}}}_{k}\left(t\right)=l_{k}\mu_{k}^{0}\dot{{\overline{T}}}_{k}^{0}\left(t\right)\] para $\overline{Q}_{k}\left(t\right)>0$.
\end{itemize}
\end{Prop}

\begin{Lema}[Lema 3.1 \cite{Chen}]\label{Lema3.1}
Si el modelo de flujo es estable, definido por las ecuaciones
(3.8)-(3.13), entonces el modelo de flujo retrasado tambi\'en es
estable.
\end{Lema}

\begin{Teo}[Teorema 5.1 \cite{Chen}]\label{Tma.5.1.Chen}
La red de colas es estable si existe una constante $t_{0}$ que
depende de $\left(\alpha,\mu,T,U\right)$ y $V$ que satisfagan las
ecuaciones (5.1)-(5.5), $Z\left(t\right)=0$, para toda $t\geq
t_{0}$.
\end{Teo}



\begin{Lema}[Lema 5.2 \cite{Gut}]\label{Lema.5.2.Gut}
Sea $\left\{\xi\left(k\right):k\in\ent\right\}$ sucesi\'on de
variables aleatorias i.i.d. con valores en
$\left(0,\infty\right)$, y sea $E\left(t\right)$ el proceso de
conteo
\[E\left(t\right)=max\left\{n\geq1:\xi\left(1\right)+\cdots+\xi\left(n-1\right)\leq t\right\}.\]
Si $E\left[\xi\left(1\right)\right]<\infty$, entonces para
cualquier entero $r\geq1$
\begin{equation}
lim_{t\rightarrow\infty}\esp\left[\left(\frac{E\left(t\right)}{t}\right)^{r}\right]=\left(\frac{1}{E\left[\xi_{1}\right]}\right)^{r}
\end{equation}
de aqu\'i, bajo estas condiciones
\begin{itemize}
\item[a)] Para cualquier $t>0$,
$sup_{t\geq\delta}\esp\left[\left(\frac{E\left(t\right)}{t}\right)^{r}\right]$

\item[b)] Las variables aleatorias
$\left\{\left(\frac{E\left(t\right)}{t}\right)^{r}:t\geq1\right\}$
son uniformemente integrables.
\end{itemize}
\end{Lema}

\begin{Teo}[Teorema 5.1: Ley Fuerte para Procesos de Conteo
\cite{Gut}]\label{Tma.5.1.Gut} Sea
$0<\mu<\esp\left(X_{1}\right]\leq\infty$. entonces

\begin{itemize}
\item[a)] $\frac{N\left(t\right)}{t}\rightarrow\frac{1}{\mu}$
a.s., cuando $t\rightarrow\infty$.


\item[b)]$\esp\left[\frac{N\left(t\right)}{t}\right]^{r}\rightarrow\frac{1}{\mu^{r}}$,
cuando $t\rightarrow\infty$ para todo $r>0$..
\end{itemize}
\end{Teo}


\begin{Prop}[Proposici\'on 5.1 \cite{DaiSean}]\label{Prop.5.1}
Suponga que los supuestos (A1) y (A2) se cumplen, adem\'as suponga
que el modelo de flujo es estable. Entonces existe $t_{0}>0$ tal
que
\begin{equation}\label{Eq.Prop.5.1}
lim_{|x|\rightarrow\infty}\frac{1}{|x|^{p+1}}\esp_{x}\left[|X\left(t_{0}|x|\right)|^{p+1}\right]=0.
\end{equation}

\end{Prop}


\begin{Prop}[Proposici\'on 5.3 \cite{DaiSean}]
Sea $X$ proceso de estados para la red de colas, y suponga que se
cumplen los supuestos (A1) y (A2), entonces para alguna constante
positiva $C_{p+1}<\infty$, $\delta>0$ y un conjunto compacto
$C\subset X$.

\begin{equation}\label{Eq.5.4}
\esp_{x}\left[\int_{0}^{\tau_{C}\left(\delta\right)}\left(1+|X\left(t\right)|^{p}\right)dt\right]\leq
C_{p+1}\left(1+|x|^{p+1}\right)
\end{equation}
\end{Prop}

\begin{Prop}[Proposici\'on 5.4 \cite{DaiSean}]
Sea $X$ un proceso de Markov Borel Derecho en $X$, sea
$f:X\leftarrow\rea_{+}$ y defina para alguna $\delta>0$, y un
conjunto cerrado $C\subset X$
\[V\left(x\right):=\esp_{x}\left[\int_{0}^{\tau_{C}\left(\delta\right)}f\left(X\left(t\right)\right)dt\right]\]
para $x\in X$. Si $V$ es finito en todas partes y uniformemente
acotada en $C$, entonces existe $k<\infty$ tal que
\begin{equation}\label{Eq.5.11}
\frac{1}{t}\esp_{x}\left[V\left(x\right)\right]+\frac{1}{t}\int_{0}^{t}\esp_{x}\left[f\left(X\left(s\right)\right)ds\right]\leq\frac{1}{t}V\left(x\right)+k,
\end{equation}
para $x\in X$ y $t>0$.
\end{Prop}


\begin{Teo}[Teorema 5.5 \cite{DaiSean}]
Suponga que se cumplen (A1) y (A2), adem\'as suponga que el modelo
de flujo es estable. Entonces existe una constante $k_{p}<\infty$
tal que
\begin{equation}\label{Eq.5.13}
\frac{1}{t}\int_{0}^{t}\esp_{x}\left[|Q\left(s\right)|^{p}\right]ds\leq
k_{p}\left\{\frac{1}{t}|x|^{p+1}+1\right\}
\end{equation}
para $t\geq0$, $x\in X$. En particular para cada condici\'on
inicial
\begin{equation}\label{Eq.5.14}
Limsup_{t\rightarrow\infty}\frac{1}{t}\int_{0}^{t}\esp_{x}\left[|Q\left(s\right)|^{p}\right]ds\leq
k_{p}
\end{equation}
\end{Teo}

\begin{Teo}[Teorema 6.2 \cite{DaiSean}]\label{Tma.6.2}
Suponga que se cumplen los supuestos (A1)-(A3) y que el modelo de
flujo es estable, entonces se tiene que
\[\parallel P^{t}\left(c,\cdot\right)-\pi\left(\cdot\right)\parallel_{f_{p}}\rightarrow0\]
para $t\rightarrow\infty$ y $x\in X$. En particular para cada
condici\'on inicial
\[lim_{t\rightarrow\infty}\esp_{x}\left[\left|Q_{t}\right|^{p}\right]=\esp_{\pi}\left[\left|Q_{0}\right|^{p}\right]<\infty\]
\end{Teo}


\begin{Teo}[Teorema 6.3 \cite{DaiSean}]\label{Tma.6.3}
Suponga que se cumplen los supuestos (A1)-(A3) y que el modelo de
flujo es estable, entonces con
$f\left(x\right)=f_{1}\left(x\right)$, se tiene que
\[lim_{t\rightarrow\infty}t^{(p-1)\left|P^{t}\left(c,\cdot\right)-\pi\left(\cdot\right)\right|_{f}=0},\]
para $x\in X$. En particular, para cada condici\'on inicial
\[lim_{t\rightarrow\infty}t^{(p-1)}\left|\esp_{x}\left[Q_{t}\right]-\esp_{\pi}\left[Q_{0}\right]\right|=0.\]
\end{Teo}



\begin{Prop}[Proposici\'on 5.1, Dai y Meyn \cite{DaiSean}]\label{Prop.5.1.DaiSean}
Suponga que los supuestos A1) y A2) son ciertos y que el modelo de
flujo es estable. Entonces existe $t_{0}>0$ tal que
\begin{equation}
lim_{|x|\rightarrow\infty}\frac{1}{|x|^{p+1}}\esp_{x}\left[|X\left(t_{0}|x|\right)|^{p+1}\right]=0
\end{equation}
\end{Prop}

\begin{Lemma}[Lema 5.2, Dai y Meyn, \cite{DaiSean}]\label{Lema.5.2.DaiSean}
 Sea $\left\{\zeta\left(k\right):k\in \mathbb{z}\right\}$ una sucesi\'on independiente e id\'enticamente distribuida que toma valores en $\left(0,\infty\right)$,
y sea
$E\left(t\right)=max\left(n\geq1:\zeta\left(1\right)+\cdots+\zeta\left(n-1\right)\leq
t\right)$. Si $\esp\left[\zeta\left(1\right)\right]<\infty$,
entonces para cualquier entero $r\geq1$
\begin{equation}
 lim_{t\rightarrow\infty}\esp\left[\left(\frac{E\left(t\right)}{t}\right)^{r}\right]=\left(\frac{1}{\esp\left[\zeta_{1}\right]}\right)^{r}.
\end{equation}
Luego, bajo estas condiciones:
\begin{itemize}
 \item[a)] para cualquier $\delta>0$, $\sup_{t\geq\delta}\esp\left[\left(\frac{E\left(t\right)}{t}\right)^{r}\right]<\infty$
\item[b)] las variables aleatorias
$\left\{\left(\frac{E\left(t\right)}{t}\right)^{r}:t\geq1\right\}$
son uniformemente integrables.
\end{itemize}
\end{Lemma}

\begin{Teo}[Teorema 5.5, Dai y Meyn \cite{DaiSean}]\label{Tma.5.5.DaiSean}
Suponga que los supuestos A1) y A2) se cumplen y que el modelo de
flujo es estable. Entonces existe una constante $\kappa_{p}$ tal
que
\begin{equation}
\frac{1}{t}\int_{0}^{t}\esp_{x}\left[|Q\left(s\right)|^{p}\right]ds\leq\kappa_{p}\left\{\frac{1}{t}|x|^{p+1}+1\right\}
\end{equation}
para $t>0$ y $x\in X$. En particular, para cada condici\'on
inicial
\begin{eqnarray*}
\limsup_{t\rightarrow\infty}\frac{1}{t}\int_{0}^{t}\esp_{x}\left[|Q\left(s\right)|^{p}\right]ds\leq\kappa_{p}.
\end{eqnarray*}
\end{Teo}

\begin{Teo}[Teorema 6.2, Dai y Meyn \cite{DaiSean}]\label{Tma.6.2.DaiSean}
Suponga que se cumplen los supuestos A1), A2) y A3) y que el
modelo de flujo es estable. Entonces se tiene que
\begin{equation}
\left\|P^{t}\left(x,\cdot\right)-\pi\left(\cdot\right)\right\|_{f_{p}}\textrm{,
}t\rightarrow\infty,x\in X.
\end{equation}
En particular para cada condici\'on inicial
\begin{eqnarray*}
\lim_{t\rightarrow\infty}\esp_{x}\left[|Q\left(t\right)|^{p}\right]=\esp_{\pi}\left[|Q\left(0\right)|^{p}\right]\leq\kappa_{r}
\end{eqnarray*}
\end{Teo}
\begin{Teo}[Teorema 6.3, Dai y Meyn \cite{DaiSean}]\label{Tma.6.3.DaiSean}
Suponga que se cumplen los supuestos A1), A2) y A3) y que el
modelo de flujo es estable. Entonces con
$f\left(x\right)=f_{1}\left(x\right)$ se tiene
\begin{equation}
\lim_{t\rightarrow\infty}t^{p-1}\left\|P^{t}\left(x,\cdot\right)-\pi\left(\cdot\right)\right\|_{f}=0.
\end{equation}
En particular para cada condici\'on inicial
\begin{eqnarray*}
\lim_{t\rightarrow\infty}t^{p-1}|\esp_{x}\left[Q\left(t\right)\right]-\esp_{\pi}\left[Q\left(0\right)\right]|=0.
\end{eqnarray*}
\end{Teo}

\begin{Teo}[Teorema 6.4, Dai y Meyn, \cite{DaiSean}]\label{Tma.6.4.DaiSean}
Suponga que se cumplen los supuestos A1), A2) y A3) y que el
modelo de flujo es estable. Sea $\nu$ cualquier distribuci\'on de
probabilidad en $\left(X,\mathcal{B}_{X}\right)$, y $\pi$ la
distribuci\'on estacionaria de $X$.
\begin{itemize}
\item[i)] Para cualquier $f:X\leftarrow\rea_{+}$
\begin{equation}
\lim_{t\rightarrow\infty}\frac{1}{t}\int_{o}^{t}f\left(X\left(s\right)\right)ds=\pi\left(f\right):=\int
f\left(x\right)\pi\left(dx\right)
\end{equation}
$\prob$-c.s.

\item[ii)] Para cualquier $f:X\leftarrow\rea_{+}$ con
$\pi\left(|f|\right)<\infty$, la ecuaci\'on anterior se cumple.
\end{itemize}
\end{Teo}

\begin{Teo}[Teorema 2.2, Down \cite{Down}]\label{Tma2.2.Down}
Suponga que el fluido modelo es inestable en el sentido de que
para alguna $\epsilon_{0},c_{0}\geq0$,
\begin{equation}\label{Eq.Inestability}
|Q\left(T\right)|\geq\epsilon_{0}T-c_{0}\textrm{,   }T\geq0,
\end{equation}
para cualquier condici\'on inicial $Q\left(0\right)$, con
$|Q\left(0\right)|=1$. Entonces para cualquier $0<q\leq1$, existe
$B<0$ tal que para cualquier $|x|\geq B$,
\begin{equation}
\prob_{x}\left\{\mathbb{X}\rightarrow\infty\right\}\geq q.
\end{equation}
\end{Teo}



\begin{Def}
Sea $X$ un conjunto y $\mathcal{F}$ una $\sigma$-\'algebra de
subconjuntos de $X$, la pareja $\left(X,\mathcal{F}\right)$ es
llamado espacio medible. Un subconjunto $A$ de $X$ es llamado
medible, o medible con respecto a $\mathcal{F}$, si
$A\in\mathcal{F}$.
\end{Def}

\begin{Def}
Sea $\left(X,\mathcal{F},\mu\right)$ espacio de medida. Se dice
que la medida $\mu$ es $\sigma$-finita si se puede escribir
$X=\bigcup_{n\geq1}X_{n}$ con $X_{n}\in\mathcal{F}$ y
$\mu\left(X_{n}\right)<\infty$.
\end{Def}

\begin{Def}\label{Cto.Borel}
Sea $X$ el conjunto de los n\'umeros reales $\rea$. El \'algebra
de Borel es la $\sigma$-\'algebra $B$ generada por los intervalos
abiertos $\left(a,b\right)\in\rea$. Cualquier conjunto en $B$ es
llamado {\em Conjunto de Borel}.
\end{Def}

\begin{Def}\label{Funcion.Medible}
Una funci\'on $f:X\rightarrow\rea$, es medible si para cualquier
n\'umero real $\alpha$ el conjunto
\[\left\{x\in X:f\left(x\right)>\alpha\right\}\]
pertenece a $\mathcal{F}$. Equivalentemente, se dice que $f$ es
medible si
\[f^{-1}\left(\left(\alpha,\infty\right)\right)=\left\{x\in X:f\left(x\right)>\alpha\right\}\in\mathcal{F}.\]
\end{Def}


\begin{Def}\label{Def.Cilindros}
Sean $\left(\Omega_{i},\mathcal{F}_{i}\right)$, $i=1,2,\ldots,$
espacios medibles y $\Omega=\prod_{i=1}^{\infty}\Omega_{i}$ el
conjunto de todas las sucesiones
$\left(\omega_{1},\omega_{2},\ldots,\right)$ tales que
$\omega_{i}\in\Omega_{i}$, $i=1,2,\ldots,$. Si
$B^{n}\subset\prod_{i=1}^{\infty}\Omega_{i}$, definimos
$B_{n}=\left\{\omega\in\Omega:\left(\omega_{1},\omega_{2},\ldots,\omega_{n}\right)\in
B^{n}\right\}$. Al conjunto $B_{n}$ se le llama {\em cilindro} con
base $B^{n}$, el cilindro es llamado medible si
$B^{n}\in\prod_{i=1}^{\infty}\mathcal{F}_{i}$.
\end{Def}


\begin{Def}\label{Def.Proc.Adaptado}[TSP, Ash \cite{RBA}]
Sea $X\left(t\right),t\geq0$ proceso estoc\'astico, el proceso es
adaptado a la familia de $\sigma$-\'algebras $\mathcal{F}_{t}$,
para $t\geq0$, si para $s<t$ implica que
$\mathcal{F}_{s}\subset\mathcal{F}_{t}$, y $X\left(t\right)$ es
$\mathcal{F}_{t}$-medible para cada $t$. Si no se especifica
$\mathcal{F}_{t}$ entonces se toma $\mathcal{F}_{t}$ como
$\mathcal{F}\left(X\left(s\right),s\leq t\right)$, la m\'as
peque\~na $\sigma$-\'algebra de subconjuntos de $\Omega$ que hace
que cada $X\left(s\right)$, con $s\leq t$ sea Borel medible.
\end{Def}


\begin{Def}\label{Def.Tiempo.Paro}[TSP, Ash \cite{RBA}]
Sea $\left\{\mathcal{F}\left(t\right),t\geq0\right\}$ familia
creciente de sub $\sigma$-\'algebras. es decir,
$\mathcal{F}\left(s\right)\subset\mathcal{F}\left(t\right)$ para
$s\leq t$. Un tiempo de paro para $\mathcal{F}\left(t\right)$ es
una funci\'on $T:\Omega\rightarrow\left[0,\infty\right]$ tal que
$\left\{T\leq t\right\}\in\mathcal{F}\left(t\right)$ para cada
$t\geq0$. Un tiempo de paro para el proceso estoc\'astico
$X\left(t\right),t\geq0$ es un tiempo de paro para las
$\sigma$-\'algebras
$\mathcal{F}\left(t\right)=\mathcal{F}\left(X\left(s\right)\right)$.
\end{Def}

\begin{Def}
Sea $X\left(t\right),t\geq0$ proceso estoc\'astico, con
$\left(S,\chi\right)$ espacio de estados. Se dice que el proceso
es adaptado a $\left\{\mathcal{F}\left(t\right)\right\}$, es
decir, si para cualquier $s,t\in I$, $I$ conjunto de \'indices,
$s<t$, se tiene que
$\mathcal{F}\left(s\right)\subset\mathcal{F}\left(t\right)$ y
$X\left(t\right)$ es $\mathcal{F}\left(t\right)$-medible,
\end{Def}

\begin{Def}
Sea $X\left(t\right),t\geq0$ proceso estoc\'astico, se dice que es
un Proceso de Markov relativo a $\mathcal{F}\left(t\right)$ o que
$\left\{X\left(t\right),\mathcal{F}\left(t\right)\right\}$ es de
Markov si y s\'olo si para cualquier conjunto $B\in\chi$,  y
$s,t\in I$, $s<t$ se cumple que
\begin{equation}\label{Prop.Markov}
P\left\{X\left(t\right)\in
B|\mathcal{F}\left(s\right)\right\}=P\left\{X\left(t\right)\in
B|X\left(s\right)\right\}.
\end{equation}
\end{Def}
\begin{Note}
Si se dice que $\left\{X\left(t\right)\right\}$ es un Proceso de
Markov sin mencionar $\mathcal{F}\left(t\right)$, se asumir\'a que
\begin{eqnarray*}
\mathcal{F}\left(t\right)=\mathcal{F}_{0}\left(t\right)=\mathcal{F}\left(X\left(r\right),r\leq
t\right),
\end{eqnarray*}
entonces la ecuaci\'on (\ref{Prop.Markov}) se puede escribir como
\begin{equation}
P\left\{X\left(t\right)\in B|X\left(r\right),r\leq s\right\} =
P\left\{X\left(t\right)\in B|X\left(s\right)\right\}
\end{equation}
\end{Note}
%_______________________________________________________________
\subsection{Procesos de Estados de Markov}
%_______________________________________________________________

\begin{Teo}
Sea $\left(X_{n},\mathcal{F}_{n},n=0,1,\ldots,\right\}$ Proceso de
Markov con espacio de estados $\left(S_{0},\chi_{0}\right)$
generado por una distribuici\'on inicial $P_{o}$ y probabilidad de
transici\'on $p_{mn}$, para $m,n=0,1,\ldots,$ $m<n$, que por
notaci\'on se escribir\'a como $p\left(m,n,x,B\right)\rightarrow
p_{mn}\left(x,B\right)$. Sea $S$ tiempo de paro relativo a la
$\sigma$-\'algebra $\mathcal{F}_{n}$. Sea $T$ funci\'on medible,
$T:\Omega\rightarrow\left\{0,1,\ldots,\right\}$. Sup\'ongase que
$T\geq S$, entonces $T$ es tiempo de paro. Si $B\in\chi_{0}$,
entonces
\begin{equation}\label{Prop.Fuerte.Markov}
P\left\{X\left(T\right)\in
B,T<\infty|\mathcal{F}\left(S\right)\right\} =
p\left(S,T,X\left(s\right),B\right)
\end{equation}
en $\left\{T<\infty\right\}$.
\end{Teo}


Sea $K$ conjunto numerable y sea $d:K\rightarrow\nat$ funci\'on.
Para $v\in K$, $M_{v}$ es un conjunto abierto de
$\rea^{d\left(v\right)}$. Entonces \[E=\bigcup_{v\in
K}M_{v}=\left\{\left(v,\zeta\right):v\in K,\zeta\in
M_{v}\right\}.\]

Sea $\mathcal{E}$ la clase de conjuntos medibles en $E$:
\[\mathcal{E}=\left\{\bigcup_{v\in K}A_{v}:A_{v}\in \mathcal{M}_{v}\right\}.\]

donde $\mathcal{M}$ son los conjuntos de Borel de $M_{v}$.
Entonces $\left(E,\mathcal{E}\right)$ es un espacio de Borel. El
estado del proceso se denotar\'a por
$\mathbf{x}_{t}=\left(v_{t},\zeta_{t}\right)$. La distribuci\'on
de $\left(\mathbf{x}_{t}\right)$ est\'a determinada por por los
siguientes objetos:

\begin{itemize}
\item[i)] Los campos vectoriales $\left(\mathcal{H}_{v},v\in
K\right)$. \item[ii)] Una funci\'on medible $\lambda:E\rightarrow
\rea_{+}$. \item[iii)] Una medida de transici\'on
$Q:\mathcal{E}\times\left(E\cup\Gamma^{*}\right)\rightarrow\left[0,1\right]$
donde
\begin{equation}
\Gamma^{*}=\bigcup_{v\in K}\partial^{*}M_{v}.
\end{equation}
y
\begin{equation}
\partial^{*}M_{v}=\left\{z\in\partial M_{v}:\mathbf{\mathbf{\phi}_{v}\left(t,\zeta\right)=\mathbf{z}}\textrm{ para alguna }\left(t,\zeta\right)\in\rea_{+}\times M_{v}\right\}.
\end{equation}
$\partial M_{v}$ denota  la frontera de $M_{v}$.
\end{itemize}

El campo vectorial $\left(\mathcal{H}_{v},v\in K\right)$ se supone
tal que para cada $\mathbf{z}\in M_{v}$ existe una \'unica curva
integral $\mathbf{\phi}_{v}\left(t,\zeta\right)$ que satisface la
ecuaci\'on

\begin{equation}
\frac{d}{dt}f\left(\zeta_{t}\right)=\mathcal{H}f\left(\zeta_{t}\right),
\end{equation}
con $\zeta_{0}=\mathbf{z}$, para cualquier funci\'on suave
$f:\rea^{d}\rightarrow\rea$ y $\mathcal{H}$ denota el operador
diferencial de primer orden, con $\mathcal{H}=\mathcal{H}_{v}$ y
$\zeta_{t}=\mathbf{\phi}\left(t,\mathbf{z}\right)$. Adem\'as se
supone que $\mathcal{H}_{v}$ es conservativo, es decir, las curvas
integrales est\'an definidas para todo $t>0$.

Para $\mathbf{x}=\left(v,\zeta\right)\in E$ se denota
\[t^{*}\mathbf{x}=inf\left\{t>0:\mathbf{\phi}_{v}\left(t,\zeta\right)\in\partial^{*}M_{v}\right\}\]

En lo que respecta a la funci\'on $\lambda$, se supondr\'a que
para cada $\left(v,\zeta\right)\in E$ existe un $\epsilon>0$ tal
que la funci\'on
$s\rightarrow\lambda\left(v,\phi_{v}\left(s,\zeta\right)\right)\in
E$ es integrable para $s\in\left[0,\epsilon\right)$. La medida de
transici\'on $Q\left(A;\mathbf{x}\right)$ es una funci\'on medible
de $\mathbf{x}$ para cada $A\in\mathcal{E}$, definida para
$\mathbf{x}\in E\cup\Gamma^{*}$ y es una medida de probabilidad en
$\left(E,\mathcal{E}\right)$ para cada $\mathbf{x}\in E$.

El movimiento del proceso $\left(\mathbf{x}_{t}\right)$ comenzando
en $\mathbf{x}=\left(n,\mathbf{z}\right)\in E$ se puede construir
de la siguiente manera, def\'inase la funci\'on $F$ por

\begin{equation}
F\left(t\right)=\left\{\begin{array}{ll}\\
exp\left(-\int_{0}^{t}\lambda\left(n,\phi_{n}\left(s,\mathbf{z}\right)\right)ds\right), & t<t^{*}\left(\mathbf{x}\right),\\
0, & t\geq t^{*}\left(\mathbf{x}\right)
\end{array}\right.
\end{equation}

Sea $T_{1}$ una variable aleatoria tal que
$\prob\left[T_{1}>t\right]=F\left(t\right)$, ahora sea la variable
aleatoria $\left(N,Z\right)$ con distribuici\'on
$Q\left(\cdot;\phi_{n}\left(T_{1},\mathbf{z}\right)\right)$. La
trayectoria de $\left(\mathbf{x}_{t}\right)$ para $t\leq T_{1}$ es
\begin{eqnarray*}
\mathbf{x}_{t}=\left(v_{t},\zeta_{t}\right)=\left\{\begin{array}{ll}
\left(n,\phi_{n}\left(t,\mathbf{z}\right)\right), & t<T_{1},\\
\left(N,\mathbf{Z}\right), & t=t_{1}.
\end{array}\right.
\end{eqnarray*}

Comenzando en $\mathbf{x}_{T_{1}}$ se selecciona el siguiente
tiempo de intersalto $T_{2}-T_{1}$ lugar del post-salto
$\mathbf{x}_{T_{2}}$ de manera similar y as\'i sucesivamente. Este
procedimiento nos da una trayectoria determinista por partes
$\mathbf{x}_{t}$ con tiempos de salto $T_{1},T_{2},\ldots$. Bajo
las condiciones enunciadas para $\lambda,T_{1}>0$  y
$T_{1}-T_{2}>0$ para cada $i$, con probabilidad 1. Se supone que
se cumple la siguiente condici\'on.

\begin{Sup}[Supuesto 3.1, Davis \cite{Davis}]\label{Sup3.1.Davis}
Sea $N_{t}:=\sum_{t}\indora_{\left(t\geq t\right)}$ el n\'umero de
saltos en $\left[0,t\right]$. Entonces
\begin{equation}
\esp\left[N_{t}\right]<\infty\textrm{ para toda }t.
\end{equation}
\end{Sup}

es un proceso de Markov, m\'as a\'un, es un Proceso Fuerte de
Markov, es decir, la Propiedad Fuerte de Markov\footnote{Revisar
p\'agina 362, y 364 de Davis \cite{Davis}.} se cumple para
cualquier tiempo de paro.
%_________________________________________________________________________
%\renewcommand{\refname}{PROCESOS ESTOC\'ASTICOS}
%\renewcommand{\appendixname}{PROCESOS ESTOC\'ASTICOS}
%\renewcommand{\appendixtocname}{PROCESOS ESTOC\'ASTICOS}
%\renewcommand{\appendixpagename}{PROCESOS ESTOC\'ASTICOS}
%\appendix
%\clearpage % o \cleardoublepage
%\addappheadtotoc
%\appendixpage
%_________________________________________________________________________
\subsection{Teor\'ia General de Procesos Estoc\'asticos}
%_________________________________________________________________________
En esta secci\'on se har\'an las siguientes consideraciones: $E$
es un espacio m\'etrico separable y la m\'etrica $d$ es compatible
con la topolog\'ia.

\begin{Def}
Una medida finita, $\lambda$ en la $\sigma$-\'algebra de Borel de
un espacio metrizable $E$ se dice cerrada si
\begin{equation}\label{Eq.A2.3}
\lambda\left(E\right)=sup\left\{\lambda\left(K\right):K\textrm{ es
compacto en }E\right\}.
\end{equation}
\end{Def}

\begin{Def}
$E$ es un espacio de Rad\'on si cada medida finita en
$\left(E,\mathcal{B}\left(E\right)\right)$ es regular interior o cerrada,
{\em tight}.
\end{Def}


El siguiente teorema nos permite tener una mejor caracterizaci\'on de los espacios de Rad\'on:
\begin{Teo}\label{Tma.A2.2}
Sea $E$ espacio separable metrizable. Entonces $E$ es de Rad\'on
si y s\'olo s\'i cada medida finita en
$\left(E,\mathcal{B}\left(E\right)\right)$ es cerrada.
\end{Teo}

%_________________________________________________________________________________________
\subsection{Propiedades de Markov}
%_________________________________________________________________________________________

Sea $E$ espacio de estados, tal que $E$ es un espacio de Rad\'on, $\mathcal{B}\left(E\right)$ $\sigma$-\'algebra de Borel en $E$, que se denotar\'a por $\mathcal{E}$.

Sea $\left(X,\mathcal{G},\prob\right)$ espacio de probabilidad,
$I\subset\rea$ conjunto de índices. Sea $\mathcal{F}_{\leq t}$ la
$\sigma$-\'algebra natural definida como
$\sigma\left\{f\left(X_{r}\right):r\in I, r\leq
t,f\in\mathcal{E}\right\}$. Se considerar\'a una
$\sigma$-\'algebra m\'as general\footnote{qu\'e se quiere decir
con el t\'ermino: m\'as general?}, $ \left(\mathcal{G}_{t}\right)$
tal que $\left(X_{t}\right)$ sea $\mathcal{E}$-adaptado.

\begin{Def}
Una familia $\left(P_{s,t}\right)$ de kernels de Markov en $\left(E,\mathcal{E}\right)$ indexada por pares $s,t\in I$, con $s\leq t$ es una funci\'on de transici\'on en $\ER$, si  para todo $r\leq s< t$ en $I$ y todo $x\in E$, $B\in\mathcal{E}$
\begin{equation}\label{Eq.Kernels}
P_{r,t}\left(x,B\right)=\int_{E}P_{r,s}\left(x,dy\right)P_{s,t}\left(y,B\right)\footnote{Ecuaci\'on de Chapman-Kolmogorov}.
\end{equation}
\end{Def}

Se dice que la funci\'on de transici\'on $\KM$ en $\ER$ es la funci\'on de transici\'on para un proceso $\PE$  con valores en $E$ y que satisface la propiedad de Markov\footnote{\begin{equation}\label{Eq.1.4.S}
\prob\left\{H|\mathcal{G}_{t}\right\}=\prob\left\{H|X_{t}\right\}\textrm{ }H\in p\mathcal{F}_{\geq t}.
\end{equation}} (\ref{Eq.1.4.S}) relativa a $\left(\mathcal{G}_{t}\right)$ si

\begin{equation}\label{Eq.1.6.S}
\prob\left\{f\left(X_{t}\right)|\mathcal{G}_{s}\right\}=P_{s,t}f\left(X_{t}\right)\textrm{ }s\leq t\in I,\textrm{ }f\in b\mathcal{E}.
\end{equation}

\begin{Def}
Una familia $\left(P_{t}\right)_{t\geq0}$ de kernels de Markov en $\ER$ es llamada {\em Semigrupo de Transici\'on de Markov} o {\em Semigrupo de Transici\'on} si
\[P_{t+s}f\left(x\right)=P_{t}\left(P_{s}f\right)\left(x\right),\textrm{ }t,s\geq0,\textrm{ }x\in E\textrm{ }f\in b\mathcal{E}\footnote{Definir los t\'ermino $b\mathcal{E}$ y $p\mathcal{E}$}.\]
\end{Def}
\begin{Note}
Si la funci\'on de transici\'on $\KM$ es llamada homog\'enea si $P_{s,t}=P_{t-s}$.
\end{Note}

Un proceso de Markov que satisface la ecuaci\'on (\ref{Eq.1.6.S}) con funci\'on de transici\'on homog\'enea $\left(P_{t}\right)$ tiene la propiedad caracter\'istica
\begin{equation}\label{Eq.1.8.S}
\prob\left\{f\left(X_{t+s}\right)|\mathcal{G}_{t}\right\}=P_{s}f\left(X_{t}\right)\textrm{ }t,s\geq0,\textrm{ }f\in b\mathcal{E}.
\end{equation}
La ecuaci\'on anterior es la {\em Propiedad Simple de Markov} de $X$ relativa a $\left(P_{t}\right)$.

En este sentido el proceso $\PE$ cumple con la propiedad de Markov (\ref{Eq.1.8.S}) relativa a $\left(\Omega,\mathcal{G},\mathcal{G}_{t},\prob\right)$ con semigrupo de transici\'on $\left(P_{t}\right)$.
%_________________________________________________________________________________________
\subsection{Primer Condici\'on de Regularidad}
%_________________________________________________________________________________________
%\newcommand{\EM}{\left(\Omega,\mathcal{G},\prob\right)}
%\newcommand{\E4}{\left(\Omega,\mathcal{G},\mathcal{G}_{t},\prob\right)}
\begin{Def}
Un proceso estoc\'astico $\PE$ definido en
$\left(\Omega,\mathcal{G},\prob\right)$ con valores en el espacio
topol\'ogico $E$ es continuo por la derecha si cada trayectoria
muestral $t\rightarrow X_{t}\left(w\right)$ es un mapeo continuo
por la derecha de $I$ en $E$.
\end{Def}

\begin{Def}[HD1]\label{Eq.2.1.S}
Un semigrupo de Markov $\left(P_{t}\right)$ en un espacio de
Rad\'on $E$ se dice que satisface la condici\'on {\em HD1} si,
dada una medida de probabilidad $\mu$ en $E$, existe una
$\sigma$-\'algebra $\mathcal{E^{*}}$ con
$\mathcal{E}\subset\mathcal{E}^{*}$ y
$P_{t}\left(b\mathcal{E}^{*}\right)\subset b\mathcal{E}^{*}$, y un
$\mathcal{E}^{*}$-proceso $E$-valuado continuo por la derecha
$\PE$ en alg\'un espacio de probabilidad filtrado
$\left(\Omega,\mathcal{G},\mathcal{G}_{t},\prob\right)$ tal que
$X=\left(\Omega,\mathcal{G},\mathcal{G}_{t},\prob\right)$ es de
Markov (Homog\'eneo) con semigrupo de transici\'on $(P_{t})$ y
distribuci\'on inicial $\mu$.
\end{Def}

Consid\'erese la colecci\'on de variables aleatorias $X_{t}$
definidas en alg\'un espacio de probabilidad, y una colecci\'on de
medidas $\mathbf{P}^{x}$ tales que
$\mathbf{P}^{x}\left\{X_{0}=x\right\}$, y bajo cualquier
$\mathbf{P}^{x}$, $X_{t}$ es de Markov con semigrupo
$\left(P_{t}\right)$. $\mathbf{P}^{x}$ puede considerarse como la
distribuci\'on condicional de $\mathbf{P}$ dado $X_{0}=x$.

\begin{Def}\label{Def.2.2.S}
Sea $E$ espacio de Rad\'on, $\SG$ semigrupo de Markov en $\ER$. La colecci\'on $\mathbf{X}=\left(\Omega,\mathcal{G},\mathcal{G}_{t},X_{t},\theta_{t},\CM\right)$ es un proceso $\mathcal{E}$-Markov continuo por la derecha simple, con espacio de estados $E$ y semigrupo de transici\'on $\SG$ en caso de que $\mathbf{X}$ satisfaga las siguientes condiciones:
\begin{itemize}
\item[i)] $\left(\Omega,\mathcal{G},\mathcal{G}_{t}\right)$ es un espacio de medida filtrado, y $X_{t}$ es un proceso $E$-valuado continuo por la derecha $\mathcal{E}^{*}$-adaptado a $\left(\mathcal{G}_{t}\right)$;

\item[ii)] $\left(\theta_{t}\right)_{t\geq0}$ es una colecci\'on de operadores {\em shift} para $X$, es decir, mapea $\Omega$ en s\'i mismo satisfaciendo para $t,s\geq0$,

\begin{equation}\label{Eq.Shift}
\theta_{t}\circ\theta_{s}=\theta_{t+s}\textrm{ y }X_{t}\circ\theta_{t}=X_{t+s};
\end{equation}

\item[iii)] Para cualquier $x\in E$,$\CM\left\{X_{0}=x\right\}=1$, y el proceso $\PE$ tiene la propiedad de Markov (\ref{Eq.1.8.S}) con semigrupo de transici\'on $\SG$ relativo a $\left(\Omega,\mathcal{G},\mathcal{G}_{t},\CM\right)$.
\end{itemize}
\end{Def}

\begin{Def}[HD2]\label{Eq.2.2.S}
Para cualquier $\alpha>0$ y cualquier $f\in S^{\alpha}$, el proceso $t\rightarrow f\left(X_{t}\right)$ es continuo por la derecha casi seguramente.
\end{Def}

\begin{Def}\label{Def.PD}
Un sistema $\mathbf{X}=\left(\Omega,\mathcal{G},\mathcal{G}_{t},X_{t},\theta_{t},\CM\right)$ es un proceso derecho en el espacio de Rad\'on $E$ con semigrupo de transici\'on $\SG$ provisto de:
\begin{itemize}
\item[i)] $\mathbf{X}$ es una realizaci\'on  continua por la derecha, \ref{Def.2.2.S}, de $\SG$.

\item[ii)] $\mathbf{X}$ satisface la condicion HD2, \ref{Eq.2.2.S}, relativa a $\mathcal{G}_{t}$.

\item[iii)] $\mathcal{G}_{t}$ es aumentado y continuo por la derecha.
\end{itemize}
\end{Def}


%_________________________________________________________________________
%\renewcommand{\refname}{MODELO DE FLUJO}
%\renewcommand{\appendixname}{MODELO DE FLUJO}
%\renewcommand{\appendixtocname}{MODELO DE FLUJO}
%\renewcommand{\appendixpagename}{MODELO DE FLUJO}
%\appendix
%\clearpage % o \cleardoublepage
%\addappheadtotoc
%\appendixpage

\subsection{Construcci\'on del Modelo de Flujo}


\begin{Lema}[Lema 4.2, Dai\cite{Dai}]\label{Lema4.2}
Sea $\left\{x_{n}\right\}\subset \mathbf{X}$ con
$|x_{n}|\rightarrow\infty$, conforme $n\rightarrow\infty$. Suponga
que
\[lim_{n\rightarrow\infty}\frac{1}{|x_{n}|}U\left(0\right)=\overline{U}\]
y
\[lim_{n\rightarrow\infty}\frac{1}{|x_{n}|}V\left(0\right)=\overline{V}.\]

Entonces, conforme $n\rightarrow\infty$, casi seguramente

\begin{equation}\label{E1.4.2}
\frac{1}{|x_{n}|}\Phi^{k}\left(\left[|x_{n}|t\right]\right)\rightarrow
P_{k}^{'}t\textrm{, u.o.c.,}
\end{equation}

\begin{equation}\label{E1.4.3}
\frac{1}{|x_{n}|}E^{x_{n}}_{k}\left(|x_{n}|t\right)\rightarrow
\alpha_{k}\left(t-\overline{U}_{k}\right)^{+}\textrm{, u.o.c.,}
\end{equation}

\begin{equation}\label{E1.4.4}
\frac{1}{|x_{n}|}S^{x_{n}}_{k}\left(|x_{n}|t\right)\rightarrow
\mu_{k}\left(t-\overline{V}_{k}\right)^{+}\textrm{, u.o.c.,}
\end{equation}

donde $\left[t\right]$ es la parte entera de $t$ y
$\mu_{k}=1/m_{k}=1/\esp\left[\eta_{k}\left(1\right)\right]$.
\end{Lema}

\begin{Lema}[Lema 4.3, Dai\cite{Dai}]\label{Lema.4.3}
Sea $\left\{x_{n}\right\}\subset \mathbf{X}$ con
$|x_{n}|\rightarrow\infty$, conforme $n\rightarrow\infty$. Suponga
que
\[lim_{n\rightarrow\infty}\frac{1}{|x_{n}|}U_{k}\left(0\right)=\overline{U}_{k}\]
y
\[lim_{n\rightarrow\infty}\frac{1}{|x_{n}|}V_{k}\left(0\right)=\overline{V}_{k}.\]
\begin{itemize}
\item[a)] Conforme $n\rightarrow\infty$ casi seguramente,
\[lim_{n\rightarrow\infty}\frac{1}{|x_{n}|}U^{x_{n}}_{k}\left(|x_{n}|t\right)=\left(\overline{U}_{k}-t\right)^{+}\textrm{, u.o.c.}\]
y
\[lim_{n\rightarrow\infty}\frac{1}{|x_{n}|}V^{x_{n}}_{k}\left(|x_{n}|t\right)=\left(\overline{V}_{k}-t\right)^{+}.\]

\item[b)] Para cada $t\geq0$ fijo,
\[\left\{\frac{1}{|x_{n}|}U^{x_{n}}_{k}\left(|x_{n}|t\right),|x_{n}|\geq1\right\}\]
y
\[\left\{\frac{1}{|x_{n}|}V^{x_{n}}_{k}\left(|x_{n}|t\right),|x_{n}|\geq1\right\}\]
\end{itemize}
son uniformemente convergentes.
\end{Lema}

Sea $S_{l}^{x}\left(t\right)$ el n\'umero total de servicios
completados de la clase $l$, si la clase $l$ est\'a dando $t$
unidades de tiempo de servicio. Sea $T_{l}^{x}\left(x\right)$ el
monto acumulado del tiempo de servicio que el servidor
$s\left(l\right)$ gasta en los usuarios de la clase $l$ al tiempo
$t$. Entonces $S_{l}^{x}\left(T_{l}^{x}\left(t\right)\right)$ es
el n\'umero total de servicios completados para la clase $l$ al
tiempo $t$. Una fracci\'on de estos usuarios,
$\Phi_{k}^{x}\left(S_{l}^{x}\left(T_{l}^{x}\left(t\right)\right)\right)$,
se convierte en usuarios de la clase $k$.\\

Entonces, dado lo anterior, se tiene la siguiente representaci\'on
para el proceso de la longitud de la cola:\\

\begin{equation}
Q_{k}^{x}\left(t\right)=Q_{k}^{x}\left(0\right)+E_{k}^{x}\left(t\right)+\sum_{l=1}^{K}\Phi_{k}^{l}\left(S_{l}^{x}\left(T_{l}^{x}\left(t\right)\right)\right)-S_{k}^{x}\left(T_{k}^{x}\left(t\right)\right)
\end{equation}
para $k=1,\ldots,K$. Para $i=1,\ldots,d$, sea
\[I_{i}^{x}\left(t\right)=t-\sum_{j\in C_{i}}T_{k}^{x}\left(t\right).\]

Entonces $I_{i}^{x}\left(t\right)$ es el monto acumulado del
tiempo que el servidor $i$ ha estado desocupado al tiempo $t$. Se
est\'a asumiendo que las disciplinas satisfacen la ley de
conservaci\'on del trabajo, es decir, el servidor $i$ est\'a en
pausa solamente cuando no hay usuarios en la estaci\'on $i$.
Entonces, se tiene que

\begin{equation}
\int_{0}^{\infty}\left(\sum_{k\in
C_{i}}Q_{k}^{x}\left(t\right)\right)dI_{i}^{x}\left(t\right)=0,
\end{equation}
para $i=1,\ldots,d$.\\

Hacer
\[T^{x}\left(t\right)=\left(T_{1}^{x}\left(t\right),\ldots,T_{K}^{x}\left(t\right)\right)^{'},\]
\[I^{x}\left(t\right)=\left(I_{1}^{x}\left(t\right),\ldots,I_{K}^{x}\left(t\right)\right)^{'}\]
y
\[S^{x}\left(T^{x}\left(t\right)\right)=\left(S_{1}^{x}\left(T_{1}^{x}\left(t\right)\right),\ldots,S_{K}^{x}\left(T_{K}^{x}\left(t\right)\right)\right)^{'}.\]

Para una disciplina que cumple con la ley de conservaci\'on del
trabajo, en forma vectorial, se tiene el siguiente conjunto de
ecuaciones

\begin{equation}\label{Eq.MF.1.3}
Q^{x}\left(t\right)=Q^{x}\left(0\right)+E^{x}\left(t\right)+\sum_{l=1}^{K}\Phi^{l}\left(S_{l}^{x}\left(T_{l}^{x}\left(t\right)\right)\right)-S^{x}\left(T^{x}\left(t\right)\right),\\
\end{equation}

\begin{equation}\label{Eq.MF.2.3}
Q^{x}\left(t\right)\geq0,\\
\end{equation}

\begin{equation}\label{Eq.MF.3.3}
T^{x}\left(0\right)=0,\textrm{ y }\overline{T}^{x}\left(t\right)\textrm{ es no decreciente},\\
\end{equation}

\begin{equation}\label{Eq.MF.4.3}
I^{x}\left(t\right)=et-CT^{x}\left(t\right)\textrm{ es no
decreciente}\\
\end{equation}

\begin{equation}\label{Eq.MF.5.3}
\int_{0}^{\infty}\left(CQ^{x}\left(t\right)\right)dI_{i}^{x}\left(t\right)=0,\\
\end{equation}

\begin{equation}\label{Eq.MF.6.3}
\textrm{Condiciones adicionales en
}\left(\overline{Q}^{x}\left(\cdot\right),\overline{T}^{x}\left(\cdot\right)\right)\textrm{
espec\'ificas de la disciplina de la cola,}
\end{equation}

donde $e$ es un vector de unos de dimensi\'on $d$, $C$ es la
matriz definida por
\[C_{ik}=\left\{\begin{array}{cc}
1,& S\left(k\right)=i,\\
0,& \textrm{ en otro caso}.\\
\end{array}\right.
\]
Es necesario enunciar el siguiente Teorema que se utilizar\'a para
el Teorema \ref{Tma.4.2.Dai}:
\begin{Teo}[Teorema 4.1, Dai \cite{Dai}]
Considere una disciplina que cumpla la ley de conservaci\'on del
trabajo, para casi todas las trayectorias muestrales $\omega$ y
cualquier sucesi\'on de estados iniciales
$\left\{x_{n}\right\}\subset \mathbf{X}$, con
$|x_{n}|\rightarrow\infty$, existe una subsucesi\'on
$\left\{x_{n_{j}}\right\}$ con $|x_{n_{j}}|\rightarrow\infty$ tal
que
\begin{equation}\label{Eq.4.15}
\frac{1}{|x_{n_{j}}|}\left(Q^{x_{n_{j}}}\left(0\right),U^{x_{n_{j}}}\left(0\right),V^{x_{n_{j}}}\left(0\right)\right)\rightarrow\left(\overline{Q}\left(0\right),\overline{U},\overline{V}\right),
\end{equation}

\begin{equation}\label{Eq.4.16}
\frac{1}{|x_{n_{j}}|}\left(Q^{x_{n_{j}}}\left(|x_{n_{j}}|t\right),T^{x_{n_{j}}}\left(|x_{n_{j}}|t\right)\right)\rightarrow\left(\overline{Q}\left(t\right),\overline{T}\left(t\right)\right)\textrm{
u.o.c.}
\end{equation}

Adem\'as,
$\left(\overline{Q}\left(t\right),\overline{T}\left(t\right)\right)$
satisface las siguientes ecuaciones:
\begin{equation}\label{Eq.MF.1.3a}
\overline{Q}\left(t\right)=Q\left(0\right)+\left(\alpha
t-\overline{U}\right)^{+}-\left(I-P\right)^{'}M^{-1}\left(\overline{T}\left(t\right)-\overline{V}\right)^{+},
\end{equation}

\begin{equation}\label{Eq.MF.2.3a}
\overline{Q}\left(t\right)\geq0,\\
\end{equation}

\begin{equation}\label{Eq.MF.3.3a}
\overline{T}\left(t\right)\textrm{ es no decreciente y comienza en cero},\\
\end{equation}

\begin{equation}\label{Eq.MF.4.3a}
\overline{I}\left(t\right)=et-C\overline{T}\left(t\right)\textrm{
es no decreciente,}\\
\end{equation}

\begin{equation}\label{Eq.MF.5.3a}
\int_{0}^{\infty}\left(C\overline{Q}\left(t\right)\right)d\overline{I}\left(t\right)=0,\\
\end{equation}

\begin{equation}\label{Eq.MF.6.3a}
\textrm{Condiciones adicionales en
}\left(\overline{Q}\left(\cdot\right),\overline{T}\left(\cdot\right)\right)\textrm{
especficas de la disciplina de la cola,}
\end{equation}
\end{Teo}


Propiedades importantes para el modelo de flujo retrasado:

\begin{Prop}
 Sea $\left(\overline{Q},\overline{T},\overline{T}^{0}\right)$ un flujo l\'imite de \ref{Eq.4.4} y suponga que cuando $x\rightarrow\infty$ a lo largo de
una subsucesi\'on
\[\left(\frac{1}{|x|}Q_{k}^{x}\left(0\right),\frac{1}{|x|}A_{k}^{x}\left(0\right),\frac{1}{|x|}B_{k}^{x}\left(0\right),\frac{1}{|x|}B_{k}^{x,0}\left(0\right)\right)\rightarrow\left(\overline{Q}_{k}\left(0\right),0,0,0\right)\]
para $k=1,\ldots,K$. EL flujo l\'imite tiene las siguientes
propiedades, donde las propiedades de la derivada se cumplen donde
la derivada exista:
\begin{itemize}
 \item[i)] Los vectores de tiempo ocupado $\overline{T}\left(t\right)$ y $\overline{T}^{0}\left(t\right)$ son crecientes y continuas con
$\overline{T}\left(0\right)=\overline{T}^{0}\left(0\right)=0$.
\item[ii)] Para todo $t\geq0$
\[\sum_{k=1}^{K}\left[\overline{T}_{k}\left(t\right)+\overline{T}_{k}^{0}\left(t\right)\right]=t\]
\item[iii)] Para todo $1\leq k\leq K$
\[\overline{Q}_{k}\left(t\right)=\overline{Q}_{k}\left(0\right)+\alpha_{k}t-\mu_{k}\overline{T}_{k}\left(t\right)\]
\item[iv)]  Para todo $1\leq k\leq K$
\[\dot{{\overline{T}}}_{k}\left(t\right)=\beta_{k}\] para $\overline{Q}_{k}\left(t\right)=0$.
\item[v)] Para todo $k,j$
\[\mu_{k}^{0}\overline{T}_{k}^{0}\left(t\right)=\mu_{j}^{0}\overline{T}_{j}^{0}\left(t\right)\]
\item[vi)]  Para todo $1\leq k\leq K$
\[\mu_{k}\dot{{\overline{T}}}_{k}\left(t\right)=l_{k}\mu_{k}^{0}\dot{{\overline{T}}}_{k}^{0}\left(t\right)\] para $\overline{Q}_{k}\left(t\right)>0$.
\end{itemize}
\end{Prop}

\begin{Teo}[Teorema 5.1: Ley Fuerte para Procesos de Conteo
\cite{Gut}]\label{Tma.5.1.Gut} Sea
$0<\mu<\esp\left(X_{1}\right]\leq\infty$. entonces

\begin{itemize}
\item[a)] $\frac{N\left(t\right)}{t}\rightarrow\frac{1}{\mu}$
a.s., cuando $t\rightarrow\infty$.


\item[b)]$\esp\left[\frac{N\left(t\right)}{t}\right]^{r}\rightarrow\frac{1}{\mu^{r}}$,
cuando $t\rightarrow\infty$ para todo $r>0$..
\end{itemize}
\end{Teo}


\begin{Prop}[Proposici\'on 5.3 \cite{DaiSean}]
Sea $X$ proceso de estados para la red de colas, y suponga que se
cumplen los supuestos (A1) y (A2), entonces para alguna constante
positiva $C_{p+1}<\infty$, $\delta>0$ y un conjunto compacto
$C\subset X$.

\begin{equation}\label{Eq.5.4}
\esp_{x}\left[\int_{0}^{\tau_{C}\left(\delta\right)}\left(1+|X\left(t\right)|^{p}\right)dt\right]\leq
C_{p+1}\left(1+|x|^{p+1}\right)
\end{equation}
\end{Prop}

\begin{Prop}[Proposici\'on 5.4 \cite{DaiSean}]
Sea $X$ un proceso de Markov Borel Derecho en $X$, sea
$f:X\leftarrow\rea_{+}$ y defina para alguna $\delta>0$, y un
conjunto cerrado $C\subset X$
\[V\left(x\right):=\esp_{x}\left[\int_{0}^{\tau_{C}\left(\delta\right)}f\left(X\left(t\right)\right)dt\right]\]
para $x\in X$. Si $V$ es finito en todas partes y uniformemente
acotada en $C$, entonces existe $k<\infty$ tal que
\begin{equation}\label{Eq.5.11}
\frac{1}{t}\esp_{x}\left[V\left(x\right)\right]+\frac{1}{t}\int_{0}^{t}\esp_{x}\left[f\left(X\left(s\right)\right)ds\right]\leq\frac{1}{t}V\left(x\right)+k,
\end{equation}
para $x\in X$ y $t>0$.
\end{Prop}


%_________________________________________________________________________
%\renewcommand{\refname}{Ap\'endice D}
%\renewcommand{\appendixname}{ESTABILIDAD}
%\renewcommand{\appendixtocname}{ESTABILIDAD}
%\renewcommand{\appendixpagename}{ESTABILIDAD}
%\appendix
%\clearpage % o \cleardoublepage
%\addappheadtotoc
%\appendixpage

\subsection{Estabilidad}

\begin{Def}[Definici\'on 3.2, Dai y Meyn \cite{DaiSean}]
El modelo de flujo retrasado de una disciplina de servicio en una
red con retraso
$\left(\overline{A}\left(0\right),\overline{B}\left(0\right)\right)\in\rea_{+}^{K+|A|}$
se define como el conjunto de ecuaciones dadas en
\ref{Eq.3.8}-\ref{Eq.3.13}, junto con la condici\'on:
\begin{equation}\label{CondAd.FluidModel}
\overline{Q}\left(t\right)=\overline{Q}\left(0\right)+\left(\alpha
t-\overline{A}\left(0\right)\right)^{+}-\left(I-P^{'}\right)M\left(\overline{T}\left(t\right)-\overline{B}\left(0\right)\right)^{+}
\end{equation}
\end{Def}

entonces si el modelo de flujo retrasado tambi\'en es estable:


\begin{Def}[Definici\'on 3.1, Dai y Meyn \cite{DaiSean}]
Un flujo l\'imite (retrasado) para una red bajo una disciplina de
servicio espec\'ifica se define como cualquier soluci\'on
 $\left(\overline{Q}\left(\cdot\right),\overline{T}\left(\cdot\right)\right)$ de las siguientes ecuaciones, donde
$\overline{Q}\left(t\right)=\left(\overline{Q}_{1}\left(t\right),\ldots,\overline{Q}_{K}\left(t\right)\right)^{'}$
y
$\overline{T}\left(t\right)=\left(\overline{T}_{1}\left(t\right),\ldots,\overline{T}_{K}\left(t\right)\right)^{'}$
\begin{equation}\label{Eq.3.8}
\overline{Q}_{k}\left(t\right)=\overline{Q}_{k}\left(0\right)+\alpha_{k}t-\mu_{k}\overline{T}_{k}\left(t\right)+\sum_{l=1}^{k}P_{lk}\mu_{l}\overline{T}_{l}\left(t\right)\\
\end{equation}
\begin{equation}\label{Eq.3.9}
\overline{Q}_{k}\left(t\right)\geq0\textrm{ para }k=1,2,\ldots,K,\\
\end{equation}
\begin{equation}\label{Eq.3.10}
\overline{T}_{k}\left(0\right)=0,\textrm{ y }\overline{T}_{k}\left(\cdot\right)\textrm{ es no decreciente},\\
\end{equation}
\begin{equation}\label{Eq.3.11}
\overline{I}_{i}\left(t\right)=t-\sum_{k\in C_{i}}\overline{T}_{k}\left(t\right)\textrm{ es no decreciente}\\
\end{equation}
\begin{equation}\label{Eq.3.12}
\overline{I}_{i}\left(\cdot\right)\textrm{ se incrementa al tiempo }t\textrm{ cuando }\sum_{k\in C_{i}}Q_{k}^{x}\left(t\right)dI_{i}^{x}\left(t\right)=0\\
\end{equation}
\begin{equation}\label{Eq.3.13}
\textrm{condiciones adicionales sobre
}\left(Q^{x}\left(\cdot\right),T^{x}\left(\cdot\right)\right)\textrm{
referentes a la disciplina de servicio}
\end{equation}
\end{Def}

\begin{Lema}[Lema 3.1 \cite{Chen}]\label{Lema3.1}
Si el modelo de flujo es estable, definido por las ecuaciones
(3.8)-(3.13), entonces el modelo de flujo retrasado tambin es
estable.
\end{Lema}

\begin{Teo}[Teorema 5.1 \cite{Chen}]\label{Tma.5.1.Chen}
La red de colas es estable si existe una constante $t_{0}$ que
depende de $\left(\alpha,\mu,T,U\right)$ y $V$ que satisfagan las
ecuaciones (5.1)-(5.5), $Z\left(t\right)=0$, para toda $t\geq
t_{0}$.
\end{Teo}

\begin{Prop}[Proposici\'on 5.1, Dai y Meyn \cite{DaiSean}]\label{Prop.5.1.DaiSean}
Suponga que los supuestos A1) y A2) son ciertos y que el modelo de flujo es estable. Entonces existe $t_{0}>0$ tal que
\begin{equation}
lim_{|x|\rightarrow\infty}\frac{1}{|x|^{p+1}}\esp_{x}\left[|X\left(t_{0}|x|\right)|^{p+1}\right]=0
\end{equation}
\end{Prop}

\begin{Lemma}[Lema 5.2, Dai y Meyn \cite{DaiSean}]\label{Lema.5.2.DaiSean}
 Sea $\left\{\zeta\left(k\right):k\in \mathbb{z}\right\}$ una sucesi\'on independiente e id\'enticamente distribuida que toma valores en $\left(0,\infty\right)$,
y sea
$E\left(t\right)=max\left(n\geq1:\zeta\left(1\right)+\cdots+\zeta\left(n-1\right)\leq
t\right)$. Si $\esp\left[\zeta\left(1\right)\right]<\infty$,
entonces para cualquier entero $r\geq1$
\begin{equation}
 lim_{t\rightarrow\infty}\esp\left[\left(\frac{E\left(t\right)}{t}\right)^{r}\right]=\left(\frac{1}{\esp\left[\zeta_{1}\right]}\right)^{r}.
\end{equation}
Luego, bajo estas condiciones:
\begin{itemize}
 \item[a)] para cualquier $\delta>0$, $\sup_{t\geq\delta}\esp\left[\left(\frac{E\left(t\right)}{t}\right)^{r}\right]<\infty$
\item[b)] las variables aleatorias
$\left\{\left(\frac{E\left(t\right)}{t}\right)^{r}:t\geq1\right\}$
son uniformemente integrables.
\end{itemize}
\end{Lemma}

\begin{Teo}[Teorema 5.5, Dai y Meyn \cite{DaiSean}]\label{Tma.5.5.DaiSean}
Suponga que los supuestos A1) y A2) se cumplen y que el modelo de
flujo es estable. Entonces existe una constante $\kappa_{p}$ tal
que
\begin{equation}
\frac{1}{t}\int_{0}^{t}\esp_{x}\left[|Q\left(s\right)|^{p}\right]ds\leq\kappa_{p}\left\{\frac{1}{t}|x|^{p+1}+1\right\}
\end{equation}
para $t>0$ y $x\in X$. En particular, para cada condici\'on
inicial
\begin{eqnarray*}
\limsup_{t\rightarrow\infty}\frac{1}{t}\int_{0}^{t}\esp_{x}\left[|Q\left(s\right)|^{p}\right]ds\leq\kappa_{p}.
\end{eqnarray*}
\end{Teo}

\begin{Teo}[Teorema 6.2, Dai y Meyn \cite{DaiSean}]\label{Tma.6.2.DaiSean}
Suponga que se cumplen los supuestos A1), A2) y A3) y que el
modelo de flujo es estable. Entonces se tiene que
\begin{equation}
\left\|P^{t}\left(x,\cdot\right)-\pi\left(\cdot\right)\right\|_{f_{p}}\textrm{,
}t\rightarrow\infty,x\in X.
\end{equation}
En particular para cada condici\'on inicial
\begin{eqnarray*}
\lim_{t\rightarrow\infty}\esp_{x}\left[|Q\left(t\right)|^{p}\right]=\esp_{\pi}\left[|Q\left(0\right)|^{p}\right]\leq\kappa_{r}
\end{eqnarray*}
\end{Teo}
\begin{Teo}[Teorema 6.3, Dai y Meyn \cite{DaiSean}]\label{Tma.6.3.DaiSean}
Suponga que se cumplen los supuestos A1), A2) y A3) y que el
modelo de flujo es estable. Entonces con
$f\left(x\right)=f_{1}\left(x\right)$ se tiene
\begin{equation}
\lim_{t\rightarrow\infty}t^{p-1}\left\|P^{t}\left(x,\cdot\right)-\pi\left(\cdot\right)\right\|_{f}=0.
\end{equation}
En particular para cada condici\'on inicial
\begin{eqnarray*}
\lim_{t\rightarrow\infty}t^{p-1}|\esp_{x}\left[Q\left(t\right)\right]-\esp_{\pi}\left[Q\left(0\right)\right]|=0.
\end{eqnarray*}
\end{Teo}

\begin{Teo}[Teorema 6.4, Dai y Meyn \cite{DaiSean}]\label{Tma.6.4.DaiSean}
Suponga que se cumplen los supuestos A1), A2) y A3) y que el
modelo de flujo es estable. Sea $\nu$ cualquier distribuci\'on de
probabilidad en $\left(X,\mathcal{B}_{X}\right)$, y $\pi$ la
distribuci\'on estacionaria de $X$.
\begin{itemize}
\item[i)] Para cualquier $f:X\leftarrow\rea_{+}$
\begin{equation}
\lim_{t\rightarrow\infty}\frac{1}{t}\int_{o}^{t}f\left(X\left(s\right)\right)ds=\pi\left(f\right):=\int
f\left(x\right)\pi\left(dx\right)
\end{equation}
$\prob$-c.s.

\item[ii)] Para cualquier $f:X\leftarrow\rea_{+}$ con
$\pi\left(|f|\right)<\infty$, la ecuaci\'on anterior se cumple.
\end{itemize}
\end{Teo}

\begin{Teo}[Teorema 2.2, Down \cite{Down}]\label{Tma2.2.Down}
Suponga que el fluido modelo es inestable en el sentido de que
para alguna $\epsilon_{0},c_{0}\geq0$,
\begin{equation}\label{Eq.Inestability}
|Q\left(T\right)|\geq\epsilon_{0}T-c_{0}\textrm{,   }T\geq0,
\end{equation}
para cualquier condici\'on inicial $Q\left(0\right)$, con
$|Q\left(0\right)|=1$. Entonces para cualquier $0<q\leq1$, existe
$B<0$ tal que para cualquier $|x|\geq B$,
\begin{equation}
\prob_{x}\left\{\mathbb{X}\rightarrow\infty\right\}\geq q.
\end{equation}
\end{Teo}


\begin{Def}
Sea $X$ un conjunto y $\mathcal{F}$ una $\sigma$-\'algebra de
subconjuntos de $X$, la pareja $\left(X,\mathcal{F}\right)$ es
llamado espacio medible. Un subconjunto $A$ de $X$ es llamado
medible, o medible con respecto a $\mathcal{F}$, si
$A\in\mathcal{F}$.
\end{Def}

\begin{Def}
Sea $\left(X,\mathcal{F},\mu\right)$ espacio de medida. Se dice
que la medida $\mu$ es $\sigma$-finita si se puede escribir
$X=\bigcup_{n\geq1}X_{n}$ con $X_{n}\in\mathcal{F}$ y
$\mu\left(X_{n}\right)<\infty$.
\end{Def}

\begin{Def}\label{Cto.Borel}
Sea $X$ el conjunto de los \'umeros reales $\rea$. El \'algebra de
Borel es la $\sigma$-\'algebra $B$ generada por los intervalos
abiertos $\left(a,b\right)\in\rea$. Cualquier conjunto en $B$ es
llamado {\em Conjunto de Borel}.
\end{Def}

\begin{Def}\label{Funcion.Medible}
Una funci\'on $f:X\rightarrow\rea$, es medible si para cualquier
n\'umero real $\alpha$ el conjunto
\[\left\{x\in X:f\left(x\right)>\alpha\right\}\]
pertenece a $X$. Equivalentemente, se dice que $f$ es medible si
\[f^{-1}\left(\left(\alpha,\infty\right)\right)=\left\{x\in X:f\left(x\right)>\alpha\right\}\in\mathcal{F}.\]
\end{Def}


\begin{Def}\label{Def.Cilindros}
Sean $\left(\Omega_{i},\mathcal{F}_{i}\right)$, $i=1,2,\ldots,$
espacios medibles y $\Omega=\prod_{i=1}^{\infty}\Omega_{i}$ el
conjunto de todas las sucesiones
$\left(\omega_{1},\omega_{2},\ldots,\right)$ tales que
$\omega_{i}\in\Omega_{i}$, $i=1,2,\ldots,$. Si
$B^{n}\subset\prod_{i=1}^{\infty}\Omega_{i}$, definimos
$B_{n}=\left\{\omega\in\Omega:\left(\omega_{1},\omega_{2},\ldots,\omega_{n}\right)\in
B^{n}\right\}$. Al conjunto $B_{n}$ se le llama {\em cilindro} con
base $B^{n}$, el cilindro es llamado medible si
$B^{n}\in\prod_{i=1}^{\infty}\mathcal{F}_{i}$.
\end{Def}


\begin{Def}\label{Def.Proc.Adaptado}[TSP, Ash \cite{RBA}]
Sea $X\left(t\right),t\geq0$ proceso estoc\'astico, el proceso es
adaptado a la familia de $\sigma$-\'algebras $\mathcal{F}_{t}$,
para $t\geq0$, si para $s<t$ implica que
$\mathcal{F}_{s}\subset\mathcal{F}_{t}$, y $X\left(t\right)$ es
$\mathcal{F}_{t}$-medible para cada $t$. Si no se especifica
$\mathcal{F}_{t}$ entonces se toma $\mathcal{F}_{t}$ como
$\mathcal{F}\left(X\left(s\right),s\leq t\right)$, la m\'as
peque\~na $\sigma$-\'algebra de subconjuntos de $\Omega$ que hace
que cada $X\left(s\right)$, con $s\leq t$ sea Borel medible.
\end{Def}


\begin{Def}\label{Def.Tiempo.Paro}[TSP, Ash \cite{RBA}]
Sea $\left\{\mathcal{F}\left(t\right),t\geq0\right\}$ familia
creciente de sub $\sigma$-\'algebras. es decir,
$\mathcal{F}\left(s\right)\subset\mathcal{F}\left(t\right)$ para
$s\leq t$. Un tiempo de paro para $\mathcal{F}\left(t\right)$ es
una funci\'on $T:\Omega\rightarrow\left[0,\infty\right]$ tal que
$\left\{T\leq t\right\}\in\mathcal{F}\left(t\right)$ para cada
$t\geq0$. Un tiempo de paro para el proceso estoc\'astico
$X\left(t\right),t\geq0$ es un tiempo de paro para las
$\sigma$-\'algebras
$\mathcal{F}\left(t\right)=\mathcal{F}\left(X\left(s\right)\right)$.
\end{Def}

\begin{Def}
Sea $X\left(t\right),t\geq0$ proceso estoc\'astico, con
$\left(S,\chi\right)$ espacio de estados. Se dice que el proceso
es adaptado a $\left\{\mathcal{F}\left(t\right)\right\}$, es
decir, si para cualquier $s,t\in I$, $I$ conjunto de \'indices,
$s<t$, se tiene que
$\mathcal{F}\left(s\right)\subset\mathcal{F}\left(t\right)$ y
$X\left(t\right)$ es $\mathcal{F}\left(t\right)$-medible,
\end{Def}

\begin{Def}
Sea $X\left(t\right),t\geq0$ proceso estoc\'astico, se dice que es
un Proceso de Markov relativo a $\mathcal{F}\left(t\right)$ o que
$\left\{X\left(t\right),\mathcal{F}\left(t\right)\right\}$ es de
Markov si y s\'olo si para cualquier conjunto $B\in\chi$,  y
$s,t\in I$, $s<t$ se cumple que
\begin{equation}\label{Prop.Markov}
P\left\{X\left(t\right)\in
B|\mathcal{F}\left(s\right)\right\}=P\left\{X\left(t\right)\in
B|X\left(s\right)\right\}.
\end{equation}
\end{Def}
\begin{Note}
Si se dice que $\left\{X\left(t\right)\right\}$ es un Proceso de
Markov sin mencionar $\mathcal{F}\left(t\right)$, se asumir\'a que
\begin{eqnarray*}
\mathcal{F}\left(t\right)=\mathcal{F}_{0}\left(t\right)=\mathcal{F}\left(X\left(r\right),r\leq
t\right),
\end{eqnarray*}
entonces la ecuaci\'on (\ref{Prop.Markov}) se puede escribir como
\begin{equation}
P\left\{X\left(t\right)\in B|X\left(r\right),r\leq s\right\} =
P\left\{X\left(t\right)\in B|X\left(s\right)\right\}
\end{equation}
\end{Note}

\begin{Teo}
Sea $\left(X_{n},\mathcal{F}_{n},n=0,1,\ldots,\right\}$ Proceso de
Markov con espacio de estados $\left(S_{0},\chi_{0}\right)$
generado por una distribuici\'on inicial $P_{o}$ y probabilidad de
transici\'on $p_{mn}$, para $m,n=0,1,\ldots,$ $m<n$, que por
notaci\'on se escribir\'a como $p\left(m,n,x,B\right)\rightarrow
p_{mn}\left(x,B\right)$. Sea $S$ tiempo de paro relativo a la
$\sigma$-\'algebra $\mathcal{F}_{n}$. Sea $T$ funci\'on medible,
$T:\Omega\rightarrow\left\{0,1,\ldots,\right\}$. Sup\'ongase que
$T\geq S$, entonces $T$ es tiempo de paro. Si $B\in\chi_{0}$,
entonces
\begin{equation}\label{Prop.Fuerte.Markov}
P\left\{X\left(T\right)\in
B,T<\infty|\mathcal{F}\left(S\right)\right\} =
p\left(S,T,X\left(s\right),B\right)
\end{equation}
en $\left\{T<\infty\right\}$.
\end{Teo}


Sea $K$ conjunto numerable y sea $d:K\rightarrow\nat$ funci\'on.
Para $v\in K$, $M_{v}$ es un conjunto abierto de
$\rea^{d\left(v\right)}$. Entonces \[E=\cup_{v\in
K}M_{v}=\left\{\left(v,\zeta\right):v\in K,\zeta\in
M_{v}\right\}.\]

Sea $\mathcal{E}$ la clase de conjuntos medibles en $E$:
\[\mathcal{E}=\left\{\cup_{v\in K}A_{v}:A_{v}\in \mathcal{M}_{v}\right\}.\]

donde $\mathcal{M}$ son los conjuntos de Borel de $M_{v}$.
Entonces $\left(E,\mathcal{E}\right)$ es un espacio de Borel. El
estado del proceso se denotar\'a por
$\mathbf{x}_{t}=\left(v_{t},\zeta_{t}\right)$. La distribuci\'on
de $\left(\mathbf{x}_{t}\right)$ est\'a determinada por por los
siguientes objetos:

\begin{itemize}
\item[i)] Los campos vectoriales $\left(\mathcal{H}_{v},v\in
K\right)$. \item[ii)] Una funci\'on medible $\lambda:E\rightarrow
\rea_{+}$. \item[iii)] Una medida de transici\'on
$Q:\mathcal{E}\times\left(E\cup\Gamma^{*}\right)\rightarrow\left[0,1\right]$
donde
\begin{equation}
\Gamma^{*}=\cup_{v\in K}\partial^{*}M_{v}.
\end{equation}
y
\begin{equation}
\partial^{*}M_{v}=\left\{z\in\partial M_{v}:\mathbf{\mathbf{\phi}_{v}\left(t,\zeta\right)=\mathbf{z}}\textrm{ para alguna }\left(t,\zeta\right)\in\rea_{+}\times M_{v}\right\}.
\end{equation}
$\partial M_{v}$ denota  la frontera de $M_{v}$.
\end{itemize}

El campo vectorial $\left(\mathcal{H}_{v},v\in K\right)$ se supone
tal que para cada $\mathbf{z}\in M_{v}$ existe una \'unica curva
integral $\mathbf{\phi}_{v}\left(t,\zeta\right)$ que satisface la
ecuaci\'on

\begin{equation}
\frac{d}{dt}f\left(\zeta_{t}\right)=\mathcal{H}f\left(\zeta_{t}\right),
\end{equation}
con $\zeta_{0}=\mathbf{z}$, para cualquier funci\'on suave
$f:\rea^{d}\rightarrow\rea$ y $\mathcal{H}$ denota el operador
diferencial de primer orden, con $\mathcal{H}=\mathcal{H}_{v}$ y
$\zeta_{t}=\mathbf{\phi}\left(t,\mathbf{z}\right)$. Adem\'as se
supone que $\mathcal{H}_{v}$ es conservativo, es decir, las curvas
integrales est\'an definidas para todo $t>0$.

Para $\mathbf{x}=\left(v,\zeta\right)\in E$ se denota
\[t^{*}\mathbf{x}=inf\left\{t>0:\mathbf{\phi}_{v}\left(t,\zeta\right)\in\partial^{*}M_{v}\right\}\]

En lo que respecta a la funci\'on $\lambda$, se supondr\'a que
para cada $\left(v,\zeta\right)\in E$ existe un $\epsilon>0$ tal
que la funci\'on
$s\rightarrow\lambda\left(v,\phi_{v}\left(s,\zeta\right)\right)\in
E$ es integrable para $s\in\left[0,\epsilon\right)$. La medida de
transici\'on $Q\left(A;\mathbf{x}\right)$ es una funci\'on medible
de $\mathbf{x}$ para cada $A\in\mathcal{E}$, definida para
$\mathbf{x}\in E\cup\Gamma^{*}$ y es una medida de probabilidad en
$\left(E,\mathcal{E}\right)$ para cada $\mathbf{x}\in E$.

El movimiento del proceso $\left(\mathbf{x}_{t}\right)$ comenzando
en $\mathbf{x}=\left(n,\mathbf{z}\right)\in E$ se puede construir
de la siguiente manera, def\'inase la funci\'on $F$ por

\begin{equation}
F\left(t\right)=\left\{\begin{array}{ll}\\
exp\left(-\int_{0}^{t}\lambda\left(n,\phi_{n}\left(s,\mathbf{z}\right)\right)ds\right), & t<t^{*}\left(\mathbf{x}\right),\\
0, & t\geq t^{*}\left(\mathbf{x}\right)
\end{array}\right.
\end{equation}

Sea $T_{1}$ una variable aleatoria tal que
$\prob\left[T_{1}>t\right]=F\left(t\right)$, ahora sea la variable
aleatoria $\left(N,Z\right)$ con distribuici\'on
$Q\left(\cdot;\phi_{n}\left(T_{1},\mathbf{z}\right)\right)$. La
trayectoria de $\left(\mathbf{x}_{t}\right)$ para $t\leq T_{1}$
es\footnote{Revisar p\'agina 362, y 364 de Davis \cite{Davis}.}
\begin{eqnarray*}
\mathbf{x}_{t}=\left(v_{t},\zeta_{t}\right)=\left\{\begin{array}{ll}
\left(n,\phi_{n}\left(t,\mathbf{z}\right)\right), & t<T_{1},\\
\left(N,\mathbf{Z}\right), & t=t_{1}.
\end{array}\right.
\end{eqnarray*}

Comenzando en $\mathbf{x}_{T_{1}}$ se selecciona el siguiente
tiempo de intersalto $T_{2}-T_{1}$ lugar del post-salto
$\mathbf{x}_{T_{2}}$ de manera similar y as\'i sucesivamente. Este
procedimiento nos da una trayectoria determinista por partes
$\mathbf{x}_{t}$ con tiempos de salto $T_{1},T_{2},\ldots$. Bajo
las condiciones enunciadas para $\lambda,T_{1}>0$  y
$T_{1}-T_{2}>0$ para cada $i$, con probabilidad 1. Se supone que
se cumple la siquiente condici\'on.

\begin{Sup}[Supuesto 3.1, Davis \cite{Davis}]\label{Sup3.1.Davis}
Sea $N_{t}:=\sum_{t}\indora_{\left(t\geq t\right)}$ el n\'umero de
saltos en $\left[0,t\right]$. Entonces
\begin{equation}
\esp\left[N_{t}\right]<\infty\textrm{ para toda }t.
\end{equation}
\end{Sup}

es un proceso de Markov, m\'as a\'un, es un Proceso Fuerte de
Markov, es decir, la Propiedad Fuerte de Markov se cumple para
cualquier tiempo de paro.
%_________________________________________________________________________

En esta secci\'on se har\'an las siguientes consideraciones: $E$
es un espacio m\'etrico separable y la m\'etrica $d$ es compatible
con la topolog\'ia.


\begin{Def}
Un espacio topol\'ogico $E$ es llamado {\em Luisin} si es
homeomorfo a un subconjunto de Borel de un espacio m\'etrico
compacto.
\end{Def}

\begin{Def}
Un espacio topol\'ogico $E$ es llamado de {\em Rad\'on} si es
homeomorfo a un subconjunto universalmente medible de un espacio
m\'etrico compacto.
\end{Def}

Equivalentemente, la definici\'on de un espacio de Rad\'on puede
encontrarse en los siguientes t\'erminos:


\begin{Def}
$E$ es un espacio de Rad\'on si cada medida finita en
$\left(E,\mathcal{B}\left(E\right)\right)$ es regular interior o cerrada,
{\em tight}.
\end{Def}

\begin{Def}
Una medida finita, $\lambda$ en la $\sigma$-\'algebra de Borel de
un espacio metrizable $E$ se dice cerrada si
\begin{equation}\label{Eq.A2.3}
\lambda\left(E\right)=sup\left\{\lambda\left(K\right):K\textrm{ es
compacto en }E\right\}.
\end{equation}
\end{Def}

El siguiente teorema nos permite tener una mejor caracterizaci\'on de los espacios de Rad\'on:
\begin{Teo}\label{Tma.A2.2}
Sea $E$ espacio separable metrizable. Entonces $E$ es Radoniano si y s\'olo s\'i cada medida finita en $\left(E,\mathcal{B}\left(E\right)\right)$ es cerrada.
\end{Teo}

%_________________________________________________________________________________________
\subsection{Propiedades de Markov}
%_________________________________________________________________________________________

Sea $E$ espacio de estados, tal que $E$ es un espacio de Rad\'on, $\mathcal{B}\left(E\right)$ $\sigma$-\'algebra de Borel en $E$, que se denotar\'a por $\mathcal{E}$.

Sea $\left(X,\mathcal{G},\prob\right)$ espacio de probabilidad, $I\subset\rea$ conjunto de índices. Sea $\mathcal{F}_{\leq t}$ la $\sigma$-\'algebra natural definida como $\sigma\left\{f\left(X_{r}\right):r\in I, rleq t,f\in\mathcal{E}\right\}$. Se considerar\'a una $\sigma$-\'algebra m\'as general, $ \left(\mathcal{G}_{t}\right)$ tal que $\left(X_{t}\right)$ sea $\mathcal{E}$-adaptado.

\begin{Def}
Una familia $\left(P_{s,t}\right)$ de kernels de Markov en $\left(E,\mathcal{E}\right)$ indexada por pares $s,t\in I$, con $s\leq t$ es una funci\'on de transici\'on en $\ER$, si  para todo $r\leq s< t$ en $I$ y todo $x\in E$, $B\in\mathcal{E}$
\begin{equation}\label{Eq.Kernels}
P_{r,t}\left(x,B\right)=\int_{E}P_{r,s}\left(x,dy\right)P_{s,t}\left(y,B\right)\footnote{Ecuaci\'on de Chapman-Kolmogorov}.
\end{equation}
\end{Def}

Se dice que la funci\'on de transici\'on $\KM$ en $\ER$ es la funci\'on de transici\'on para un proceso $\PE$  con valores en $E$ y que satisface la propiedad de Markov\footnote{\begin{equation}\label{Eq.1.4.S}
\prob\left\{H|\mathcal{G}_{t}\right\}=\prob\left\{H|X_{t}\right\}\textrm{ }H\in p\mathcal{F}_{\geq t}.
\end{equation}} (\ref{Eq.1.4.S}) relativa a $\left(\mathcal{G}_{t}\right)$ si 

\begin{equation}\label{Eq.1.6.S}
\prob\left\{f\left(X_{t}\right)|\mathcal{G}_{s}\right\}=P_{s,t}f\left(X_{t}\right)\textrm{ }s\leq t\in I,\textrm{ }f\in b\mathcal{E}.
\end{equation}

\begin{Def}
Una familia $\left(P_{t}\right)_{t\geq0}$ de kernels de Markov en $\ER$ es llamada {\em Semigrupo de Transici\'on de Markov} o {\em Semigrupo de Transici\'on} si
\[P_{t+s}f\left(x\right)=P_{t}\left(P_{s}f\right)\left(x\right),\textrm{ }t,s\geq0,\textrm{ }x\in E\textrm{ }f\in b\mathcal{E}.\]
\end{Def}
\begin{Note}
Si la funci\'on de transici\'on $\KM$ es llamada homog\'enea si $P_{s,t}=P_{t-s}$.
\end{Note}

Un proceso de Markov que satisface la ecuaci\'on (\ref{Eq.1.6.S}) con funci\'on de transici\'on homog\'enea $\left(P_{t}\right)$ tiene la propiedad caracter\'istica
\begin{equation}\label{Eq.1.8.S}
\prob\left\{f\left(X_{t+s}\right)|\mathcal{G}_{t}\right\}=P_{s}f\left(X_{t}\right)\textrm{ }t,s\geq0,\textrm{ }f\in b\mathcal{E}.
\end{equation}
La ecuaci\'on anterior es la {\em Propiedad Simple de Markov} de $X$ relativa a $\left(P_{t}\right)$.

En este sentido el proceso $\PE$ cumple con la propiedad de Markov (\ref{Eq.1.8.S}) relativa a $\left(\Omega,\mathcal{G},\mathcal{G}_{t},\prob\right)$ con semigrupo de transici\'on $\left(P_{t}\right)$.
%_________________________________________________________________________________________
\subsection{Primer Condici\'on de Regularidad}
%_________________________________________________________________________________________
%\newcommand{\EM}{\left(\Omega,\mathcal{G},\prob\right)}
%\newcommand{\E4}{\left(\Omega,\mathcal{G},\mathcal{G}_{t},\prob\right)}
\begin{Def}
Un proceso estoc\'astico $\PE$ definido en $\left(\Omega,\mathcal{G},\prob\right)$ con valores en el espacio topol\'ogico $E$ es continuo por la derecha si cada trayectoria muestral $t\rightarrow X_{t}\left(w\right)$ es un mapeo continuo por la derecha de $I$ en $E$.
\end{Def}

\begin{Def}[HD1]\label{Eq.2.1.S}
Un semigrupo de Markov $\left/P_{t}\right)$ en un espacio de Rad\'on $E$ se dice que satisface la condici\'on {\em HD1} si, dada una medida de probabilidad $\mu$ en $E$, existe una $\sigma$-\'algebra $\mathcal{E^{*}}$ con $\mathcal{E}\subset\mathcal{E}$ y $P_{t}\left(b\mathcal{E}^{*}\right)\subset b\mathcal{E}^{*}$, y un $\mathcal{E}^{*}$-proceso $E$-valuado continuo por la derecha $\PE$ en alg\'un espacio de probabilidad filtrado $\left(\Omega,\mathcal{G},\mathcal{G}_{t},\prob\right)$ tal que $X=\left(\Omega,\mathcal{G},\mathcal{G}_{t},\prob\right)$ es de Markov (Homog\'eneo) con semigrupo de transici\'on $(P_{t})$ y distribuci\'on inicial $\mu$.
\end{Def}

Considerese la colecci\'on de variables aleatorias $X_{t}$ definidas en alg\'un espacio de probabilidad, y una colecci\'on de medidas $\mathbf{P}^{x}$ tales que $\mathbf{P}^{x}\left\{X_{0}=x\right\}$, y bajo cualquier $\mathbf{P}^{x}$, $X_{t}$ es de Markov con semigrupo $\left(P_{t}\right)$. $\mathbf{P}^{x}$ puede considerarse como la distribuci\'on condicional de $\mathbf{P}$ dado $X_{0}=x$.

\begin{Def}\label{Def.2.2.S}
Sea $E$ espacio de Rad\'on, $\SG$ semigrupo de Markov en $\ER$. La colecci\'on $\mathbf{X}=\left(\Omega,\mathcal{G},\mathcal{G}_{t},X_{t},\theta_{t},\CM\right)$ es un proceso $\mathcal{E}$-Markov continuo por la derecha simple, con espacio de estados $E$ y semigrupo de transici\'on $\SG$ en caso de que $\mathbf{X}$ satisfaga las siguientes condiciones:
\begin{itemize}
\item[i)] $\left(\Omega,\mathcal{G},\mathcal{G}_{t}\right)$ es un espacio de medida filtrado, y $X_{t}$ es un proceso $E$-valuado continuo por la derecha $\mathcal{E}^{*}$-adaptado a $\left(\mathcal{G}_{t}\right)$;

\item[ii)] $\left(\theta_{t}\right)_{t\geq0}$ es una colecci\'on de operadores {\em shift} para $X$, es decir, mapea $\Omega$ en s\'i mismo satisfaciendo para $t,s\geq0$,

\begin{equation}\label{Eq.Shift}
\theta_{t}\circ\theta_{s}=\theta_{t+s}\textrm{ y }X_{t}\circ\theta_{t}=X_{t+s};
\end{equation}

\item[iii)] Para cualquier $x\in E$,$\CM\left\{X_{0}=x\right\}=1$, y el proceso $\PE$ tiene la propiedad de Markov (\ref{Eq.1.8.S}) con semigrupo de transici\'on $\SG$ relativo a $\left(\Omega,\mathcal{G},\mathcal{G}_{t},\CM\right)$.
\end{itemize}
\end{Def}

\begin{Def}[HD2]\label{Eq.2.2.S}
Para cualquier $\alpha>0$ y cualquier $f\in S^{\alpha}$, el proceso $t\rightarrow f\left(X_{t}\right)$ es continuo por la derecha casi seguramente.
\end{Def}

\begin{Def}\label{Def.PD}
Un sistema $\mathbf{X}=\left(\Omega,\mathcal{G},\mathcal{G}_{t},X_{t},\theta_{t},\CM\right)$ es un proceso derecho en el espacio de Rad\'on $E$ con semigrupo de transici\'on $\SG$ provisto de:
\begin{itemize}
\item[i)] $\mathbf{X}$ es una realizaci\'on  continua por la derecha, \ref{Def.2.2.S}, de $\SG$.

\item[ii)] $\mathbf{X}$ satisface la condicion HD2, \ref{Eq.2.2.S}, relativa a $\mathcal{G}_{t}$.

\item[iii)] $\mathcal{G}_{t}$ es aumentado y continuo por la derecha.
\end{itemize}
\end{Def}




\begin{Lema}[Lema 4.2, Dai\cite{Dai}]\label{Lema4.2}
Sea $\left\{x_{n}\right\}\subset \mathbf{X}$ con
$|x_{n}|\rightarrow\infty$, conforme $n\rightarrow\infty$. Suponga
que
\[lim_{n\rightarrow\infty}\frac{1}{|x_{n}|}U\left(0\right)=\overline{U}\]
y
\[lim_{n\rightarrow\infty}\frac{1}{|x_{n}|}V\left(0\right)=\overline{V}.\]

Entonces, conforme $n\rightarrow\infty$, casi seguramente

\begin{equation}\label{E1.4.2}
\frac{1}{|x_{n}|}\Phi^{k}\left(\left[|x_{n}|t\right]\right)\rightarrow
P_{k}^{'}t\textrm{, u.o.c.,}
\end{equation}

\begin{equation}\label{E1.4.3}
\frac{1}{|x_{n}|}E^{x_{n}}_{k}\left(|x_{n}|t\right)\rightarrow
\alpha_{k}\left(t-\overline{U}_{k}\right)^{+}\textrm{, u.o.c.,}
\end{equation}

\begin{equation}\label{E1.4.4}
\frac{1}{|x_{n}|}S^{x_{n}}_{k}\left(|x_{n}|t\right)\rightarrow
\mu_{k}\left(t-\overline{V}_{k}\right)^{+}\textrm{, u.o.c.,}
\end{equation}

donde $\left[t\right]$ es la parte entera de $t$ y
$\mu_{k}=1/m_{k}=1/\esp\left[\eta_{k}\left(1\right)\right]$.
\end{Lema}

\begin{Lema}[Lema 4.3, Dai\cite{Dai}]\label{Lema.4.3}
Sea $\left\{x_{n}\right\}\subset \mathbf{X}$ con
$|x_{n}|\rightarrow\infty$, conforme $n\rightarrow\infty$. Suponga
que
\[lim_{n\rightarrow\infty}\frac{1}{|x_{n}|}U\left(0\right)=\overline{U}_{k}\]
y
\[lim_{n\rightarrow\infty}\frac{1}{|x_{n}|}V\left(0\right)=\overline{V}_{k}.\]
\begin{itemize}
\item[a)] Conforme $n\rightarrow\infty$ casi seguramente,
\[lim_{n\rightarrow\infty}\frac{1}{|x_{n}|}U^{x_{n}}_{k}\left(|x_{n}|t\right)=\left(\overline{U}_{k}-t\right)^{+}\textrm{, u.o.c.}\]
y
\[lim_{n\rightarrow\infty}\frac{1}{|x_{n}|}V^{x_{n}}_{k}\left(|x_{n}|t\right)=\left(\overline{V}_{k}-t\right)^{+}.\]

\item[b)] Para cada $t\geq0$ fijo,
\[\left\{\frac{1}{|x_{n}|}U^{x_{n}}_{k}\left(|x_{n}|t\right),|x_{n}|\geq1\right\}\]
y
\[\left\{\frac{1}{|x_{n}|}V^{x_{n}}_{k}\left(|x_{n}|t\right),|x_{n}|\geq1\right\}\]
\end{itemize}
son uniformemente convergentes.
\end{Lema}

$S_{l}^{x}\left(t\right)$ es el n\'umero total de servicios
completados de la clase $l$, si la clase $l$ est\'a dando $t$
unidades de tiempo de servicio. Sea $T_{l}^{x}\left(x\right)$ el
monto acumulado del tiempo de servicio que el servidor
$s\left(l\right)$ gasta en los usuarios de la clase $l$ al tiempo
$t$. Entonces $S_{l}^{x}\left(T_{l}^{x}\left(t\right)\right)$ es
el n\'umero total de servicios completados para la clase $l$ al
tiempo $t$. Una fracci\'on de estos usuarios,
$\Phi_{l}^{x}\left(S_{l}^{x}\left(T_{l}^{x}\left(t\right)\right)\right)$,
se convierte en usuarios de la clase $k$.\\

Entonces, dado lo anterior, se tiene la siguiente representaci\'on
para el proceso de la longitud de la cola:\\

\begin{equation}
Q_{k}^{x}\left(t\right)=_{k}^{x}\left(0\right)+E_{k}^{x}\left(t\right)+\sum_{l=1}^{K}\Phi_{k}^{l}\left(S_{l}^{x}\left(T_{l}^{x}\left(t\right)\right)\right)-S_{k}^{x}\left(T_{k}^{x}\left(t\right)\right)
\end{equation}
para $k=1,\ldots,K$. Para $i=1,\ldots,d$, sea
\[I_{i}^{x}\left(t\right)=t-\sum_{j\in C_{i}}T_{k}^{x}\left(t\right).\]

Entonces $I_{i}^{x}\left(t\right)$ es el monto acumulado del
tiempo que el servidor $i$ ha estado desocupado al tiempo $t$. Se
est\'a asumiendo que las disciplinas satisfacen la ley de
conservaci\'on del trabajo, es decir, el servidor $i$ est\'a en
pausa solamente cuando no hay usuarios en la estaci\'on $i$.
Entonces, se tiene que

\begin{equation}
\int_{0}^{\infty}\left(\sum_{k\in
C_{i}}Q_{k}^{x}\left(t\right)\right)dI_{i}^{x}\left(t\right)=0,
\end{equation}
para $i=1,\ldots,d$.\\

Hacer
\[T^{x}\left(t\right)=\left(T_{1}^{x}\left(t\right),\ldots,T_{K}^{x}\left(t\right)\right)^{'},\]
\[I^{x}\left(t\right)=\left(I_{1}^{x}\left(t\right),\ldots,I_{K}^{x}\left(t\right)\right)^{'}\]
y
\[S^{x}\left(T^{x}\left(t\right)\right)=\left(S_{1}^{x}\left(T_{1}^{x}\left(t\right)\right),\ldots,S_{K}^{x}\left(T_{K}^{x}\left(t\right)\right)\right)^{'}.\]

Para una disciplina que cumple con la ley de conservaci\'on del
trabajo, en forma vectorial, se tiene el siguiente conjunto de
ecuaciones

\begin{equation}\label{Eq.MF.1.3}
Q^{x}\left(t\right)=Q^{x}\left(0\right)+E^{x}\left(t\right)+\sum_{l=1}^{K}\Phi^{l}\left(S_{l}^{x}\left(T_{l}^{x}\left(t\right)\right)\right)-S^{x}\left(T^{x}\left(t\right)\right),\\
\end{equation}

\begin{equation}\label{Eq.MF.2.3}
Q^{x}\left(t\right)\geq0,\\
\end{equation}

\begin{equation}\label{Eq.MF.3.3}
T^{x}\left(0\right)=0,\textrm{ y }\overline{T}^{x}\left(t\right)\textrm{ es no decreciente},\\
\end{equation}

\begin{equation}\label{Eq.MF.4.3}
I^{x}\left(t\right)=et-CT^{x}\left(t\right)\textrm{ es no
decreciente}\\
\end{equation}

\begin{equation}\label{Eq.MF.5.3}
\int_{0}^{\infty}\left(CQ^{x}\left(t\right)\right)dI_{i}^{x}\left(t\right)=0,\\
\end{equation}

\begin{equation}\label{Eq.MF.6.3}
\textrm{Condiciones adicionales en
}\left(\overline{Q}^{x}\left(\cdot\right),\overline{T}^{x}\left(\cdot\right)\right)\textrm{
espec\'ificas de la disciplina de la cola,}
\end{equation}

donde $e$ es un vector de unos de dimensi\'on $d$, $C$ es la
matriz definida por
\[C_{ik}=\left\{\begin{array}{cc}
1,& S\left(k\right)=i,\\
0,& \textrm{ en otro caso}.\\
\end{array}\right.
\]
Es necesario enunciar el siguiente Teorema que se utilizar\'a para
el Teorema \ref{Tma.4.2.Dai}:
\begin{Teo}[Teorema 4.1, Dai \cite{Dai}]
Considere una disciplina que cumpla la ley de conservaci\'on del
trabajo, para casi todas las trayectorias muestrales $\omega$ y
cualquier sucesi\'on de estados iniciales
$\left\{x_{n}\right\}\subset \mathbf{X}$, con
$|x_{n}|\rightarrow\infty$, existe una subsucesi\'on
$\left\{x_{n_{j}}\right\}$ con $|x_{n_{j}}|\rightarrow\infty$ tal
que
\begin{equation}\label{Eq.4.15}
\frac{1}{|x_{n_{j}}|}\left(Q^{x_{n_{j}}}\left(0\right),U^{x_{n_{j}}}\left(0\right),V^{x_{n_{j}}}\left(0\right)\right)\rightarrow\left(\overline{Q}\left(0\right),\overline{U},\overline{V}\right),
\end{equation}

\begin{equation}\label{Eq.4.16}
\frac{1}{|x_{n_{j}}|}\left(Q^{x_{n_{j}}}\left(|x_{n_{j}}|t\right),T^{x_{n_{j}}}\left(|x_{n_{j}}|t\right)\right)\rightarrow\left(\overline{Q}\left(t\right),\overline{T}\left(t\right)\right)\textrm{
u.o.c.}
\end{equation}

Adem\'as,
$\left(\overline{Q}\left(t\right),\overline{T}\left(t\right)\right)$
satisface las siguientes ecuaciones:
\begin{equation}\label{Eq.MF.1.3a}
\overline{Q}\left(t\right)=Q\left(0\right)+\left(\alpha
t-\overline{U}\right)^{+}-\left(I-P\right)^{'}M^{-1}\left(\overline{T}\left(t\right)-\overline{V}\right)^{+},
\end{equation}

\begin{equation}\label{Eq.MF.2.3a}
\overline{Q}\left(t\right)\geq0,\\
\end{equation}

\begin{equation}\label{Eq.MF.3.3a}
\overline{T}\left(t\right)\textrm{ es no decreciente y comienza en cero},\\
\end{equation}

\begin{equation}\label{Eq.MF.4.3a}
\overline{I}\left(t\right)=et-C\overline{T}\left(t\right)\textrm{
es no decreciente,}\\
\end{equation}

\begin{equation}\label{Eq.MF.5.3a}
\int_{0}^{\infty}\left(C\overline{Q}\left(t\right)\right)d\overline{I}\left(t\right)=0,\\
\end{equation}

\begin{equation}\label{Eq.MF.6.3a}
\textrm{Condiciones adicionales en
}\left(\overline{Q}\left(\cdot\right),\overline{T}\left(\cdot\right)\right)\textrm{
especficas de la disciplina de la cola,}
\end{equation}
\end{Teo}

\begin{Def}[Definici\'on 4.1, , Dai \cite{Dai}]
Sea una disciplina de servicio espec\'ifica. Cualquier l\'imite
$\left(\overline{Q}\left(\cdot\right),\overline{T}\left(\cdot\right)\right)$
en \ref{Eq.4.16} es un {\em flujo l\'imite} de la disciplina.
Cualquier soluci\'on (\ref{Eq.MF.1.3a})-(\ref{Eq.MF.6.3a}) es
llamado flujo soluci\'on de la disciplina. Se dice que el modelo de flujo l\'imite, modelo de flujo, de la disciplina de la cola es estable si existe una constante
$\delta>0$ que depende de $\mu,\alpha$ y $P$ solamente, tal que
cualquier flujo l\'imite con
$|\overline{Q}\left(0\right)|+|\overline{U}|+|\overline{V}|=1$, se
tiene que $\overline{Q}\left(\cdot+\delta\right)\equiv0$.
\end{Def}

\begin{Teo}[Teorema 4.2, Dai\cite{Dai}]\label{Tma.4.2.Dai}
Sea una disciplina fija para la cola, suponga que se cumplen las
condiciones (1.2)-(1.5). Si el modelo de flujo l\'imite de la
disciplina de la cola es estable, entonces la cadena de Markov $X$
que describe la din\'amica de la red bajo la disciplina es Harris
recurrente positiva.
\end{Teo}

Ahora se procede a escalar el espacio y el tiempo para reducir la
aparente fluctuaci\'on del modelo. Consid\'erese el proceso
\begin{equation}\label{Eq.3.7}
\overline{Q}^{x}\left(t\right)=\frac{1}{|x|}Q^{x}\left(|x|t\right)
\end{equation}
A este proceso se le conoce como el fluido escalado, y cualquier l\'imite $\overline{Q}^{x}\left(t\right)$ es llamado flujo l\'imite del proceso de longitud de la cola. Haciendo $|q|\rightarrow\infty$ mientras se mantiene el resto de las componentes fijas, cualquier punto l\'imite del proceso de longitud de la cola normalizado $\overline{Q}^{x}$ es soluci\'on del siguiente modelo de flujo.

Al conjunto de ecuaciones dadas en \ref{Eq.3.8}-\ref{Eq.3.13} se
le llama {\em Modelo de flujo} y al conjunto de todas las
soluciones del modelo de flujo
$\left(\overline{Q}\left(\cdot\right),\overline{T}
\left(\cdot\right)\right)$ se le denotar\'a por $\mathcal{Q}$.

Si se hace $|x|\rightarrow\infty$ sin restringir ninguna de las
componentes, tambi\'en se obtienen un modelo de flujo, pero en
este caso el residual de los procesos de arribo y servicio
introducen un retraso:

\begin{Def}[Definici\'on 3.3, Dai y Meyn \cite{DaiSean}]
El modelo de flujo es estable si existe un tiempo fijo $t_{0}$ tal
que $\overline{Q}\left(t\right)=0$, con $t\geq t_{0}$, para
cualquier $\overline{Q}\left(\cdot\right)\in\mathcal{Q}$ que
cumple con $|\overline{Q}\left(0\right)|=1$.
\end{Def}

El siguiente resultado se encuentra en Chen \cite{Chen}.
\begin{Lemma}[Lema 3.1, Dai y Meyn \cite{DaiSean}]
Si el modelo de flujo definido por \ref{Eq.3.8}-\ref{Eq.3.13} es
estable, entonces el modelo de flujo retrasado es tambi\'en
estable, es decir, existe $t_{0}>0$ tal que
$\overline{Q}\left(t\right)=0$ para cualquier $t\geq t_{0}$, para
cualquier soluci\'on del modelo de flujo retrasado cuya
condici\'on inicial $\overline{x}$ satisface que
$|\overline{x}|=|\overline{Q}\left(0\right)|+|\overline{A}\left(0\right)|+|\overline{B}\left(0\right)|\leq1$.
\end{Lemma}


Propiedades importantes para el modelo de flujo retrasado:

\begin{Prop}
 Sea $\left(\overline{Q},\overline{T},\overline{T}^{0}\right)$ un flujo l\'imite de \ref{Eq.4.4} y suponga que cuando $x\rightarrow\infty$ a lo largo de
una subsucesi\'on
\[\left(\frac{1}{|x|}Q_{k}^{x}\left(0\right),\frac{1}{|x|}A_{k}^{x}\left(0\right),\frac{1}{|x|}B_{k}^{x}\left(0\right),\frac{1}{|x|}B_{k}^{x,0}\left(0\right)\right)\rightarrow\left(\overline{Q}_{k}\left(0\right),0,0,0\right)\]
para $k=1,\ldots,K$. EL flujo l\'imite tiene las siguientes
propiedades, donde las propiedades de la derivada se cumplen donde
la derivada exista:
\begin{itemize}
 \item[i)] Los vectores de tiempo ocupado $\overline{T}\left(t\right)$ y $\overline{T}^{0}\left(t\right)$ son crecientes y continuas con
$\overline{T}\left(0\right)=\overline{T}^{0}\left(0\right)=0$.
\item[ii)] Para todo $t\geq0$
\[\sum_{k=1}^{K}\left[\overline{T}_{k}\left(t\right)+\overline{T}_{k}^{0}\left(t\right)\right]=t\]
\item[iii)] Para todo $1\leq k\leq K$
\[\overline{Q}_{k}\left(t\right)=\overline{Q}_{k}\left(0\right)+\alpha_{k}t-\mu_{k}\overline{T}_{k}\left(t\right)\]
\item[iv)]  Para todo $1\leq k\leq K$
\[\dot{{\overline{T}}}_{k}\left(t\right)=\beta_{k}\] para $\overline{Q}_{k}\left(t\right)=0$.
\item[v)] Para todo $k,j$
\[\mu_{k}^{0}\overline{T}_{k}^{0}\left(t\right)=\mu_{j}^{0}\overline{T}_{j}^{0}\left(t\right)\]
\item[vi)]  Para todo $1\leq k\leq K$
\[\mu_{k}\dot{{\overline{T}}}_{k}\left(t\right)=l_{k}\mu_{k}^{0}\dot{{\overline{T}}}_{k}^{0}\left(t\right)\] para $\overline{Q}_{k}\left(t\right)>0$.
\end{itemize}
\end{Prop}

\begin{Lema}[Lema 3.1 \cite{Chen}]\label{Lema3.1}
Si el modelo de flujo es estable, definido por las ecuaciones
(3.8)-(3.13), entonces el modelo de flujo retrasado tambin es
estable.
\end{Lema}

\begin{Teo}[Teorema 5.2 \cite{Chen}]\label{Tma.5.2}
Si el modelo de flujo lineal correspondiente a la red de cola es
estable, entonces la red de colas es estable.
\end{Teo}

\begin{Teo}[Teorema 5.1 \cite{Chen}]\label{Tma.5.1.Chen}
La red de colas es estable si existe una constante $t_{0}$ que
depende de $\left(\alpha,\mu,T,U\right)$ y $V$ que satisfagan las
ecuaciones (5.1)-(5.5), $Z\left(t\right)=0$, para toda $t\geq
t_{0}$.
\end{Teo}



\begin{Lema}[Lema 5.2 \cite{Gut}]\label{Lema.5.2.Gut}
Sea $\left\{\xi\left(k\right):k\in\ent\right\}$ sucesin de
variables aleatorias i.i.d. con valores en
$\left(0,\infty\right)$, y sea $E\left(t\right)$ el proceso de
conteo
\[E\left(t\right)=max\left\{n\geq1:\xi\left(1\right)+\cdots+\xi\left(n-1\right)\leq t\right\}.\]
Si $E\left[\xi\left(1\right)\right]<\infty$, entonces para
cualquier entero $r\geq1$
\begin{equation}
lim_{t\rightarrow\infty}\esp\left[\left(\frac{E\left(t\right)}{t}\right)^{r}\right]=\left(\frac{1}{E\left[\xi_{1}\right]}\right)^{r}
\end{equation}
de aqu, bajo estas condiciones
\begin{itemize}
\item[a)] Para cualquier $t>0$,
$sup_{t\geq\delta}\esp\left[\left(\frac{E\left(t\right)}{t}\right)^{r}\right]$

\item[b)] Las variables aleatorias
$\left\{\left(\frac{E\left(t\right)}{t}\right)^{r}:t\geq1\right\}$
son uniformemente integrables.
\end{itemize}
\end{Lema}

\begin{Teo}[Teorema 5.1: Ley Fuerte para Procesos de Conteo
\cite{Gut}]\label{Tma.5.1.Gut} Sea
$0<\mu<\esp\left(X_{1}\right]\leq\infty$. entonces

\begin{itemize}
\item[a)] $\frac{N\left(t\right)}{t}\rightarrow\frac{1}{\mu}$
a.s., cuando $t\rightarrow\infty$.


\item[b)]$\esp\left[\frac{N\left(t\right)}{t}\right]^{r}\rightarrow\frac{1}{\mu^{r}}$,
cuando $t\rightarrow\infty$ para todo $r>0$..
\end{itemize}
\end{Teo}


\begin{Prop}[Proposicin 5.1 \cite{DaiSean}]\label{Prop.5.1}
Suponga que los supuestos (A1) y (A2) se cumplen, adems suponga
que el modelo de flujo es estable. Entonces existe $t_{0}>0$ tal
que
\begin{equation}\label{Eq.Prop.5.1}
lim_{|x|\rightarrow\infty}\frac{1}{|x|^{p+1}}\esp_{x}\left[|X\left(t_{0}|x|\right)|^{p+1}\right]=0.
\end{equation}

\end{Prop}


\begin{Prop}[Proposici\'on 5.3 \cite{DaiSean}]
Sea $X$ proceso de estados para la red de colas, y suponga que se
cumplen los supuestos (A1) y (A2), entonces para alguna constante
positiva $C_{p+1}<\infty$, $\delta>0$ y un conjunto compacto
$C\subset X$.

\begin{equation}\label{Eq.5.4}
\esp_{x}\left[\int_{0}^{\tau_{C}\left(\delta\right)}\left(1+|X\left(t\right)|^{p}\right)dt\right]\leq
C_{p+1}\left(1+|x|^{p+1}\right)
\end{equation}
\end{Prop}

\begin{Prop}[Proposici\'on 5.4 \cite{DaiSean}]
Sea $X$ un proceso de Markov Borel Derecho en $X$, sea
$f:X\leftarrow\rea_{+}$ y defina para alguna $\delta>0$, y un
conjunto cerrado $C\subset X$
\[V\left(x\right):=\esp_{x}\left[\int_{0}^{\tau_{C}\left(\delta\right)}f\left(X\left(t\right)\right)dt\right]\]
para $x\in X$. Si $V$ es finito en todas partes y uniformemente
acotada en $C$, entonces existe $k<\infty$ tal que
\begin{equation}\label{Eq.5.11}
\frac{1}{t}\esp_{x}\left[V\left(x\right)\right]+\frac{1}{t}\int_{0}^{t}\esp_{x}\left[f\left(X\left(s\right)\right)ds\right]\leq\frac{1}{t}V\left(x\right)+k,
\end{equation}
para $x\in X$ y $t>0$.
\end{Prop}


\begin{Teo}[Teorema 5.5 \cite{DaiSean}]
Suponga que se cumplen (A1) y (A2), adems suponga que el modelo
de flujo es estable. Entonces existe una constante $k_{p}<\infty$
tal que
\begin{equation}\label{Eq.5.13}
\frac{1}{t}\int_{0}^{t}\esp_{x}\left[|Q\left(s\right)|^{p}\right]ds\leq
k_{p}\left\{\frac{1}{t}|x|^{p+1}+1\right\}
\end{equation}
para $t\geq0$, $x\in X$. En particular para cada condici\'on inicial
\begin{equation}\label{Eq.5.14}
Limsup_{t\rightarrow\infty}\frac{1}{t}\int_{0}^{t}\esp_{x}\left[|Q\left(s\right)|^{p}\right]ds\leq
k_{p}
\end{equation}
\end{Teo}

\begin{Teo}[Teorema 6.2\cite{DaiSean}]\label{Tma.6.2}
Suponga que se cumplen los supuestos (A1)-(A3) y que el modelo de
flujo es estable, entonces se tiene que
\[\parallel P^{t}\left(c,\cdot\right)-\pi\left(\cdot\right)\parallel_{f_{p}}\rightarrow0\]
para $t\rightarrow\infty$ y $x\in X$. En particular para cada
condicin inicial
\[lim_{t\rightarrow\infty}\esp_{x}\left[\left|Q_{t}\right|^{p}\right]=\esp_{\pi}\left[\left|Q_{0}\right|^{p}\right]<\infty\]
\end{Teo}


\begin{Teo}[Teorema 6.3\cite{DaiSean}]\label{Tma.6.3}
Suponga que se cumplen los supuestos (A1)-(A3) y que el modelo de
flujo es estable, entonces con
$f\left(x\right)=f_{1}\left(x\right)$, se tiene que
\[lim_{t\rightarrow\infty}t^{(p-1)\left|P^{t}\left(c,\cdot\right)-\pi\left(\cdot\right)\right|_{f}=0},\]
para $x\in X$. En particular, para cada condicin inicial
\[lim_{t\rightarrow\infty}t^{(p-1)\left|\esp_{x}\left[Q_{t}\right]-\esp_{\pi}\left[Q_{0}\right]\right|=0}.\]
\end{Teo}


\begin{Prop}[Proposici\'on 5.1, Dai y Meyn \cite{DaiSean}]\label{Prop.5.1.DaiSean}
Suponga que los supuestos A1) y A2) son ciertos y que el modelo de flujo es estable. Entonces existe $t_{0}>0$ tal que
\begin{equation}
lim_{|x|\rightarrow\infty}\frac{1}{|x|^{p+1}}\esp_{x}\left[|X\left(t_{0}|x|\right)|^{p+1}\right]=0
\end{equation}
\end{Prop}

\begin{Lemma}[Lema 5.2, Dai y Meyn \cite{DaiSean}]\label{Lema.5.2.DaiSean}
 Sea $\left\{\zeta\left(k\right):k\in \mathbb{z}\right\}$ una sucesi\'on independiente e id\'enticamente distribuida que toma valores en $\left(0,\infty\right)$,
y sea
$E\left(t\right)=max\left(n\geq1:\zeta\left(1\right)+\cdots+\zeta\left(n-1\right)\leq
t\right)$. Si $\esp\left[\zeta\left(1\right)\right]<\infty$,
entonces para cualquier entero $r\geq1$
\begin{equation}
 lim_{t\rightarrow\infty}\esp\left[\left(\frac{E\left(t\right)}{t}\right)^{r}\right]=\left(\frac{1}{\esp\left[\zeta_{1}\right]}\right)^{r}.
\end{equation}
Luego, bajo estas condiciones:
\begin{itemize}
 \item[a)] para cualquier $\delta>0$, $\sup_{t\geq\delta}\esp\left[\left(\frac{E\left(t\right)}{t}\right)^{r}\right]<\infty$
\item[b)] las variables aleatorias
$\left\{\left(\frac{E\left(t\right)}{t}\right)^{r}:t\geq1\right\}$
son uniformemente integrables.
\end{itemize}
\end{Lemma}

\begin{Teo}[Teorema 5.5, Dai y Meyn \cite{DaiSean}]\label{Tma.5.5.DaiSean}
Suponga que los supuestos A1) y A2) se cumplen y que el modelo de
flujo es estable. Entonces existe una constante $\kappa_{p}$ tal
que
\begin{equation}
\frac{1}{t}\int_{0}^{t}\esp_{x}\left[|Q\left(s\right)|^{p}\right]ds\leq\kappa_{p}\left\{\frac{1}{t}|x|^{p+1}+1\right\}
\end{equation}
para $t>0$ y $x\in X$. En particular, para cada condici\'on
inicial
\begin{eqnarray*}
\limsup_{t\rightarrow\infty}\frac{1}{t}\int_{0}^{t}\esp_{x}\left[|Q\left(s\right)|^{p}\right]ds\leq\kappa_{p}.
\end{eqnarray*}
\end{Teo}

\begin{Teo}[Teorema 6.2, Dai y Meyn \cite{DaiSean}]\label{Tma.6.2.DaiSean}
Suponga que se cumplen los supuestos A1), A2) y A3) y que el
modelo de flujo es estable. Entonces se tiene que
\begin{equation}
\left\|P^{t}\left(x,\cdot\right)-\pi\left(\cdot\right)\right\|_{f_{p}}\textrm{,
}t\rightarrow\infty,x\in X.
\end{equation}
En particular para cada condici\'on inicial
\begin{eqnarray*}
\lim_{t\rightarrow\infty}\esp_{x}\left[|Q\left(t\right)|^{p}\right]=\esp_{\pi}\left[|Q\left(0\right)|^{p}\right]\leq\kappa_{r}
\end{eqnarray*}
\end{Teo}
\begin{Teo}[Teorema 6.3, Dai y Meyn \cite{DaiSean}]\label{Tma.6.3.DaiSean}
Suponga que se cumplen los supuestos A1), A2) y A3) y que el
modelo de flujo es estable. Entonces con
$f\left(x\right)=f_{1}\left(x\right)$ se tiene
\begin{equation}
\lim_{t\rightarrow\infty}t^{p-1}\left\|P^{t}\left(x,\cdot\right)-\pi\left(\cdot\right)\right\|_{f}=0.
\end{equation}
En particular para cada condici\'on inicial
\begin{eqnarray*}
\lim_{t\rightarrow\infty}t^{p-1}|\esp_{x}\left[Q\left(t\right)\right]-\esp_{\pi}\left[Q\left(0\right)\right]|=0.
\end{eqnarray*}
\end{Teo}

\begin{Teo}[Teorema 6.4, Dai y Meyn \cite{DaiSean}]\label{Tma.6.4.DaiSean}
Suponga que se cumplen los supuestos A1), A2) y A3) y que el
modelo de flujo es estable. Sea $\nu$ cualquier distribuci\'on de
probabilidad en $\left(X,\mathcal{B}_{X}\right)$, y $\pi$ la
distribuci\'on estacionaria de $X$.
\begin{itemize}
\item[i)] Para cualquier $f:X\leftarrow\rea_{+}$
\begin{equation}
\lim_{t\rightarrow\infty}\frac{1}{t}\int_{o}^{t}f\left(X\left(s\right)\right)ds=\pi\left(f\right):=\int
f\left(x\right)\pi\left(dx\right)
\end{equation}
$\prob$-c.s.

\item[ii)] Para cualquier $f:X\leftarrow\rea_{+}$ con
$\pi\left(|f|\right)<\infty$, la ecuaci\'on anterior se cumple.
\end{itemize}
\end{Teo}

\begin{Teo}[Teorema 2.2, Down \cite{Down}]\label{Tma2.2.Down}
Suponga que el fluido modelo es inestable en el sentido de que
para alguna $\epsilon_{0},c_{0}\geq0$,
\begin{equation}\label{Eq.Inestability}
|Q\left(T\right)|\geq\epsilon_{0}T-c_{0}\textrm{,   }T\geq0,
\end{equation}
para cualquier condici\'on inicial $Q\left(0\right)$, con
$|Q\left(0\right)|=1$. Entonces para cualquier $0<q\leq1$, existe
$B<0$ tal que para cualquier $|x|\geq B$,
\begin{equation}
\prob_{x}\left\{\mathbb{X}\rightarrow\infty\right\}\geq q.
\end{equation}
\end{Teo}



Es necesario hacer los siguientes supuestos sobre el
comportamiento del sistema de visitas c\'iclicas:
\begin{itemize}
\item Los tiempos de interarribo a la $k$-\'esima cola, son de la
forma $\left\{\xi_{k}\left(n\right)\right\}_{n\geq1}$, con la
propiedad de que son independientes e id{\'e}nticamente
distribuidos,
\item Los tiempos de servicio
$\left\{\eta_{k}\left(n\right)\right\}_{n\geq1}$ tienen la
propiedad de ser independientes e id{\'e}nticamente distribuidos,
\item Se define la tasa de arribo a la $k$-{\'e}sima cola como
$\lambda_{k}=1/\esp\left[\xi_{k}\left(1\right)\right]$,
\item la tasa de servicio para la $k$-{\'e}sima cola se define
como $\mu_{k}=1/\esp\left[\eta_{k}\left(1\right)\right]$,
\item tambi{\'e}n se define $\rho_{k}:=\lambda_{k}/\mu_{k}$, la
intensidad de tr\'afico del sistema o carga de la red, donde es
necesario que $\rho<1$ para cuestiones de estabilidad.
\end{itemize}



%_________________________________________________________________________
\subsection{Procesos Fuerte de Markov}
%_________________________________________________________________________
En Dai \cite{Dai} se muestra que para una amplia serie de disciplinas
de servicio el proceso $X$ es un Proceso Fuerte de
Markov, y por tanto se puede asumir que


Para establecer que $X=\left\{X\left(t\right),t\geq0\right\}$ es
un Proceso Fuerte de Markov, se siguen las secciones 2.3 y 2.4 de Kaspi and Mandelbaum \cite{KaspiMandelbaum}. \\

%______________________________________________________________
\subsubsection{Construcci\'on de un Proceso Determinista por partes, Davis
\cite{Davis}}.
%______________________________________________________________

%_________________________________________________________________________
\subsection{Procesos Harris Recurrentes Positivos}
%_________________________________________________________________________
Sea el proceso de Markov $X=\left\{X\left(t\right),t\geq0\right\}$
que describe la din\'amica de la red de colas. En lo que respecta
al supuesto (A3), en Dai y Meyn \cite{DaiSean} y Meyn y Down
\cite{MeynDown} hacen ver que este se puede sustituir por

\begin{itemize}
\item[A3')] Para el Proceso de Markov $X$, cada subconjunto
compacto de $X$ es un conjunto peque\~no.
\end{itemize}

Este supuesto es importante pues es un requisito para deducir la ergodicidad de la red.

%_________________________________________________________________________
\subsection{Construcci\'on de un Modelo de Flujo L\'imite}
%_________________________________________________________________________

Consideremos un caso m\'as simple para poner en contexto lo
anterior: para un sistema de visitas c\'iclicas se tiene que el
estado al tiempo $t$ es
\begin{equation}
X\left(t\right)=\left(Q\left(t\right),U\left(t\right),V\left(t\right)\right),
\end{equation}

donde $Q\left(t\right)$ es el n\'umero de usuarios formados en
cada estaci\'on. $U\left(t\right)$ es el tiempo restante antes de
que la siguiente clase $k$ de usuarios lleguen desde fuera del
sistema, $V\left(t\right)$ es el tiempo restante de servicio para
la clase $k$ de usuarios que est\'an siendo atendidos. Tanto
$U\left(t\right)$ como $V\left(t\right)$ se puede asumir que son
continuas por la derecha.

Sea
$x=\left(Q\left(0\right),U\left(0\right),V\left(0\right)\right)=\left(q,a,b\right)$,
el estado inicial de la red bajo una disciplina espec\'ifica para
la cola. Para $l\in\mathcal{E}$, donde $\mathcal{E}$ es el conjunto de clases de arribos externos, y $k=1,\ldots,K$ se define\\
\begin{eqnarray*}
E_{l}^{x}\left(t\right)&=&max\left\{r:U_{l}\left(0\right)+\xi_{l}\left(1\right)+\cdots+\xi_{l}\left(r-1\right)\leq
t\right\}\textrm{   }t\geq0,\\
S_{k}^{x}\left(t\right)&=&max\left\{r:V_{k}\left(0\right)+\eta_{k}\left(1\right)+\cdots+\eta_{k}\left(r-1\right)\leq
t\right\}\textrm{   }t\geq0.
\end{eqnarray*}

Para cada $k$ y cada $n$ se define

\begin{eqnarray*}\label{Eq.phi}
\Phi^{k}\left(n\right):=\sum_{i=1}^{n}\phi^{k}\left(i\right).
\end{eqnarray*}

donde $\phi^{k}\left(n\right)$ se define como el vector de ruta
para el $n$-\'esimo usuario de la clase $k$ que termina en la
estaci\'on $s\left(k\right)$, la $s$-\'eima componente de
$\phi^{k}\left(n\right)$ es uno si estos usuarios se convierten en
usuarios de la clase $l$ y cero en otro caso, por lo tanto
$\phi^{k}\left(n\right)$ es un vector {\em Bernoulli} de
dimensi\'on $K$ con par\'ametro $P_{k}^{'}$, donde $P_{k}$ denota
el $k$-\'esimo rengl\'on de $P=\left(P_{kl}\right)$.

Se asume que cada para cada $k$ la sucesi\'on $\phi^{k}\left(n\right)=\left\{\phi^{k}\left(n\right),n\geq1\right\}$
es independiente e id\'enticamente distribuida y que las
$\phi^{1}\left(n\right),\ldots,\phi^{K}\left(n\right)$ son
mutuamente independientes, adem\'as de independientes de los
procesos de arribo y de servicio.\\

\begin{Lema}[Lema 4.2, Dai\cite{Dai}]\label{Lema4.2}
Sea $\left\{x_{n}\right\}\subset \mathbf{X}$ con
$|x_{n}|\rightarrow\infty$, conforme $n\rightarrow\infty$. Suponga
que
\[lim_{n\rightarrow\infty}\frac{1}{|x_{n}|}U\left(0\right)=\overline{U}\]
y
\[lim_{n\rightarrow\infty}\frac{1}{|x_{n}|}V\left(0\right)=\overline{V}.\]

Entonces, conforme $n\rightarrow\infty$, casi seguramente

\begin{equation}\label{E1.4.2}
\frac{1}{|x_{n}|}\Phi^{k}\left(\left[|x_{n}|t\right]\right)\rightarrow
P_{k}^{'}t\textrm{, u.o.c.,}
\end{equation}

\begin{equation}\label{E1.4.3}
\frac{1}{|x_{n}|}E^{x_{n}}_{k}\left(|x_{n}|t\right)\rightarrow
\alpha_{k}\left(t-\overline{U}_{k}\right)^{+}\textrm{, u.o.c.,}
\end{equation}

\begin{equation}\label{E1.4.4}
\frac{1}{|x_{n}|}S^{x_{n}}_{k}\left(|x_{n}|t\right)\rightarrow
\mu_{k}\left(t-\overline{V}_{k}\right)^{+}\textrm{, u.o.c.,}
\end{equation}

donde $\left[t\right]$ es la parte entera de $t$ y
$\mu_{k}=1/m_{k}=1/\esp\left[\eta_{k}\left(1\right)\right]$.
\end{Lema}

\begin{Lema}[Lema 4.3, Dai\cite{Dai}]\label{Lema.4.3}
Sea $\left\{x_{n}\right\}\subset \mathbf{X}$ con
$|x_{n}|\rightarrow\infty$, conforme $n\rightarrow\infty$. Suponga
que
\[lim_{n\rightarrow\infty}\frac{1}{|x_{n}|}U\left(0\right)=\overline{U}_{k}\]
y
\[lim_{n\rightarrow\infty}\frac{1}{|x_{n}|}V\left(0\right)=\overline{V}_{k}.\]
\begin{itemize}
\item[a)] Conforme $n\rightarrow\infty$ casi seguramente,
\[lim_{n\rightarrow\infty}\frac{1}{|x_{n}|}U^{x_{n}}_{k}\left(|x_{n}|t\right)=\left(\overline{U}_{k}-t\right)^{+}\textrm{, u.o.c.}\]
y
\[lim_{n\rightarrow\infty}\frac{1}{|x_{n}|}V^{x_{n}}_{k}\left(|x_{n}|t\right)=\left(\overline{V}_{k}-t\right)^{+}.\]

\item[b)] Para cada $t\geq0$ fijo,
\[\left\{\frac{1}{|x_{n}|}U^{x_{n}}_{k}\left(|x_{n}|t\right),|x_{n}|\geq1\right\}\]
y
\[\left\{\frac{1}{|x_{n}|}V^{x_{n}}_{k}\left(|x_{n}|t\right),|x_{n}|\geq1\right\}\]
\end{itemize}
son uniformemente convergentes.
\end{Lema}

$S_{l}^{x}\left(t\right)$ es el n\'umero total de servicios
completados de la clase $l$, si la clase $l$ est\'a dando $t$
unidades de tiempo de servicio. Sea $T_{l}^{x}\left(x\right)$ el
monto acumulado del tiempo de servicio que el servidor
$s\left(l\right)$ gasta en los usuarios de la clase $l$ al tiempo
$t$. Entonces $S_{l}^{x}\left(T_{l}^{x}\left(t\right)\right)$ es
el n\'umero total de servicios completados para la clase $l$ al
tiempo $t$. Una fracci\'on de estos usuarios,
$\Phi_{l}^{x}\left(S_{l}^{x}\left(T_{l}^{x}\left(t\right)\right)\right)$,
se convierte en usuarios de la clase $k$.\\

Entonces, dado lo anterior, se tiene la siguiente representaci\'on
para el proceso de la longitud de la cola:\\

\begin{equation}
Q_{k}^{x}\left(t\right)=_{k}^{x}\left(0\right)+E_{k}^{x}\left(t\right)+\sum_{l=1}^{K}\Phi_{k}^{l}\left(S_{l}^{x}\left(T_{l}^{x}\left(t\right)\right)\right)-S_{k}^{x}\left(T_{k}^{x}\left(t\right)\right)
\end{equation}
para $k=1,\ldots,K$. Para $i=1,\ldots,d$, sea
\[I_{i}^{x}\left(t\right)=t-\sum_{j\in C_{i}}T_{k}^{x}\left(t\right).\]

Entonces $I_{i}^{x}\left(t\right)$ es el monto acumulado del
tiempo que el servidor $i$ ha estado desocupado al tiempo $t$. Se
est\'a asumiendo que las disciplinas satisfacen la ley de
conservaci\'on del trabajo, es decir, el servidor $i$ est\'a en
pausa solamente cuando no hay usuarios en la estaci\'on $i$.
Entonces, se tiene que

\begin{equation}
\int_{0}^{\infty}\left(\sum_{k\in
C_{i}}Q_{k}^{x}\left(t\right)\right)dI_{i}^{x}\left(t\right)=0,
\end{equation}
para $i=1,\ldots,d$.\\

Hacer
\[T^{x}\left(t\right)=\left(T_{1}^{x}\left(t\right),\ldots,T_{K}^{x}\left(t\right)\right)^{'},\]
\[I^{x}\left(t\right)=\left(I_{1}^{x}\left(t\right),\ldots,I_{K}^{x}\left(t\right)\right)^{'}\]
y
\[S^{x}\left(T^{x}\left(t\right)\right)=\left(S_{1}^{x}\left(T_{1}^{x}\left(t\right)\right),\ldots,S_{K}^{x}\left(T_{K}^{x}\left(t\right)\right)\right)^{'}.\]

Para una disciplina que cumple con la ley de conservaci\'on del
trabajo, en forma vectorial, se tiene el siguiente conjunto de
ecuaciones

\begin{equation}\label{Eq.MF.1.3}
Q^{x}\left(t\right)=Q^{x}\left(0\right)+E^{x}\left(t\right)+\sum_{l=1}^{K}\Phi^{l}\left(S_{l}^{x}\left(T_{l}^{x}\left(t\right)\right)\right)-S^{x}\left(T^{x}\left(t\right)\right),\\
\end{equation}

\begin{equation}\label{Eq.MF.2.3}
Q^{x}\left(t\right)\geq0,\\
\end{equation}

\begin{equation}\label{Eq.MF.3.3}
T^{x}\left(0\right)=0,\textrm{ y }\overline{T}^{x}\left(t\right)\textrm{ es no decreciente},\\
\end{equation}

\begin{equation}\label{Eq.MF.4.3}
I^{x}\left(t\right)=et-CT^{x}\left(t\right)\textrm{ es no
decreciente}\\
\end{equation}

\begin{equation}\label{Eq.MF.5.3}
\int_{0}^{\infty}\left(CQ^{x}\left(t\right)\right)dI_{i}^{x}\left(t\right)=0,\\
\end{equation}

\begin{equation}\label{Eq.MF.6.3}
\textrm{Condiciones adicionales en
}\left(\overline{Q}^{x}\left(\cdot\right),\overline{T}^{x}\left(\cdot\right)\right)\textrm{
espec\'ificas de la disciplina de la cola,}
\end{equation}

donde $e$ es un vector de unos de dimensi\'on $d$, $C$ es la
matriz definida por
\[C_{ik}=\left\{\begin{array}{cc}
1,& S\left(k\right)=i,\\
0,& \textrm{ en otro caso}.\\
\end{array}\right.
\]
Es necesario enunciar el siguiente Teorema que se utilizar\'a para
el Teorema \ref{Tma.4.2.Dai}:
\begin{Teo}[Teorema 4.1, Dai \cite{Dai}]
Considere una disciplina que cumpla la ley de conservaci\'on del
trabajo, para casi todas las trayectorias muestrales $\omega$ y
cualquier sucesi\'on de estados iniciales
$\left\{x_{n}\right\}\subset \mathbf{X}$, con
$|x_{n}|\rightarrow\infty$, existe una subsucesi\'on
$\left\{x_{n_{j}}\right\}$ con $|x_{n_{j}}|\rightarrow\infty$ tal
que
\begin{equation}\label{Eq.4.15}
\frac{1}{|x_{n_{j}}|}\left(Q^{x_{n_{j}}}\left(0\right),U^{x_{n_{j}}}\left(0\right),V^{x_{n_{j}}}\left(0\right)\right)\rightarrow\left(\overline{Q}\left(0\right),\overline{U},\overline{V}\right),
\end{equation}

\begin{equation}\label{Eq.4.16}
\frac{1}{|x_{n_{j}}|}\left(Q^{x_{n_{j}}}\left(|x_{n_{j}}|t\right),T^{x_{n_{j}}}\left(|x_{n_{j}}|t\right)\right)\rightarrow\left(\overline{Q}\left(t\right),\overline{T}\left(t\right)\right)\textrm{
u.o.c.}
\end{equation}

Adem\'as,
$\left(\overline{Q}\left(t\right),\overline{T}\left(t\right)\right)$
satisface las siguientes ecuaciones:
\begin{equation}\label{Eq.MF.1.3a}
\overline{Q}\left(t\right)=Q\left(0\right)+\left(\alpha
t-\overline{U}\right)^{+}-\left(I-P\right)^{'}M^{-1}\left(\overline{T}\left(t\right)-\overline{V}\right)^{+},
\end{equation}

\begin{equation}\label{Eq.MF.2.3a}
\overline{Q}\left(t\right)\geq0,\\
\end{equation}

\begin{equation}\label{Eq.MF.3.3a}
\overline{T}\left(t\right)\textrm{ es no decreciente y comienza en cero},\\
\end{equation}

\begin{equation}\label{Eq.MF.4.3a}
\overline{I}\left(t\right)=et-C\overline{T}\left(t\right)\textrm{
es no decreciente,}\\
\end{equation}

\begin{equation}\label{Eq.MF.5.3a}
\int_{0}^{\infty}\left(C\overline{Q}\left(t\right)\right)d\overline{I}\left(t\right)=0,\\
\end{equation}

\begin{equation}\label{Eq.MF.6.3a}
\textrm{Condiciones adicionales en
}\left(\overline{Q}\left(\cdot\right),\overline{T}\left(\cdot\right)\right)\textrm{
especficas de la disciplina de la cola,}
\end{equation}
\end{Teo}

\begin{Def}[Definici\'on 4.1, , Dai \cite{Dai}]
Sea una disciplina de servicio espec\'ifica. Cualquier l\'imite
$\left(\overline{Q}\left(\cdot\right),\overline{T}\left(\cdot\right)\right)$
en \ref{Eq.4.16} es un {\em flujo l\'imite} de la disciplina.
Cualquier soluci\'on (\ref{Eq.MF.1.3a})-(\ref{Eq.MF.6.3a}) es
llamado flujo soluci\'on de la disciplina. Se dice que el modelo de flujo l\'imite, modelo de flujo, de la disciplina de la cola es estable si existe una constante
$\delta>0$ que depende de $\mu,\alpha$ y $P$ solamente, tal que
cualquier flujo l\'imite con
$|\overline{Q}\left(0\right)|+|\overline{U}|+|\overline{V}|=1$, se
tiene que $\overline{Q}\left(\cdot+\delta\right)\equiv0$.
\end{Def}

\begin{Teo}[Teorema 4.2, Dai\cite{Dai}]\label{Tma.4.2.Dai}
Sea una disciplina fija para la cola, suponga que se cumplen las
condiciones (1.2)-(1.5). Si el modelo de flujo l\'imite de la
disciplina de la cola es estable, entonces la cadena de Markov $X$
que describe la din\'amica de la red bajo la disciplina es Harris
recurrente positiva.
\end{Teo}

Ahora se procede a escalar el espacio y el tiempo para reducir la
aparente fluctuaci\'on del modelo. Consid\'erese el proceso
\begin{equation}\label{Eq.3.7}
\overline{Q}^{x}\left(t\right)=\frac{1}{|x|}Q^{x}\left(|x|t\right)
\end{equation}
A este proceso se le conoce como el fluido escalado, y cualquier l\'imite $\overline{Q}^{x}\left(t\right)$ es llamado flujo l\'imite del proceso de longitud de la cola. Haciendo $|q|\rightarrow\infty$ mientras se mantiene el resto de las componentes fijas, cualquier punto l\'imite del proceso de longitud de la cola normalizado $\overline{Q}^{x}$ es soluci\'on del siguiente modelo de flujo.

\begin{Def}[Definici\'on 3.1, Dai y Meyn \cite{DaiSean}]
Un flujo l\'imite (retrasado) para una red bajo una disciplina de
servicio espec\'ifica se define como cualquier soluci\'on
 $\left(\overline{Q}\left(\cdot\right),\overline{T}\left(\cdot\right)\right)$ de las siguientes ecuaciones, donde
$\overline{Q}\left(t\right)=\left(\overline{Q}_{1}\left(t\right),\ldots,\overline{Q}_{K}\left(t\right)\right)^{'}$
y
$\overline{T}\left(t\right)=\left(\overline{T}_{1}\left(t\right),\ldots,\overline{T}_{K}\left(t\right)\right)^{'}$
\begin{equation}\label{Eq.3.8}
\overline{Q}_{k}\left(t\right)=\overline{Q}_{k}\left(0\right)+\alpha_{k}t-\mu_{k}\overline{T}_{k}\left(t\right)+\sum_{l=1}^{k}P_{lk}\mu_{l}\overline{T}_{l}\left(t\right)\\
\end{equation}
\begin{equation}\label{Eq.3.9}
\overline{Q}_{k}\left(t\right)\geq0\textrm{ para }k=1,2,\ldots,K,\\
\end{equation}
\begin{equation}\label{Eq.3.10}
\overline{T}_{k}\left(0\right)=0,\textrm{ y }\overline{T}_{k}\left(\cdot\right)\textrm{ es no decreciente},\\
\end{equation}
\begin{equation}\label{Eq.3.11}
\overline{I}_{i}\left(t\right)=t-\sum_{k\in C_{i}}\overline{T}_{k}\left(t\right)\textrm{ es no decreciente}\\
\end{equation}
\begin{equation}\label{Eq.3.12}
\overline{I}_{i}\left(\cdot\right)\textrm{ se incrementa al tiempo }t\textrm{ cuando }\sum_{k\in C_{i}}Q_{k}^{x}\left(t\right)dI_{i}^{x}\left(t\right)=0\\
\end{equation}
\begin{equation}\label{Eq.3.13}
\textrm{condiciones adicionales sobre
}\left(Q^{x}\left(\cdot\right),T^{x}\left(\cdot\right)\right)\textrm{
referentes a la disciplina de servicio}
\end{equation}
\end{Def}

Al conjunto de ecuaciones dadas en \ref{Eq.3.8}-\ref{Eq.3.13} se
le llama {\em Modelo de flujo} y al conjunto de todas las
soluciones del modelo de flujo
$\left(\overline{Q}\left(\cdot\right),\overline{T}
\left(\cdot\right)\right)$ se le denotar\'a por $\mathcal{Q}$.

Si se hace $|x|\rightarrow\infty$ sin restringir ninguna de las
componentes, tambi\'en se obtienen un modelo de flujo, pero en
este caso el residual de los procesos de arribo y servicio
introducen un retraso:

\begin{Def}[Definici\'on 3.2, Dai y Meyn \cite{DaiSean}]
El modelo de flujo retrasado de una disciplina de servicio en una
red con retraso
$\left(\overline{A}\left(0\right),\overline{B}\left(0\right)\right)\in\rea_{+}^{K+|A|}$
se define como el conjunto de ecuaciones dadas en
\ref{Eq.3.8}-\ref{Eq.3.13}, junto con la condici\'on:
\begin{equation}\label{CondAd.FluidModel}
\overline{Q}\left(t\right)=\overline{Q}\left(0\right)+\left(\alpha
t-\overline{A}\left(0\right)\right)^{+}-\left(I-P^{'}\right)M\left(\overline{T}\left(t\right)-\overline{B}\left(0\right)\right)^{+}
\end{equation}
\end{Def}

\begin{Def}[Definici\'on 3.3, Dai y Meyn \cite{DaiSean}]
El modelo de flujo es estable si existe un tiempo fijo $t_{0}$ tal
que $\overline{Q}\left(t\right)=0$, con $t\geq t_{0}$, para
cualquier $\overline{Q}\left(\cdot\right)\in\mathcal{Q}$ que
cumple con $|\overline{Q}\left(0\right)|=1$.
\end{Def}

El siguiente resultado se encuentra en Chen \cite{Chen}.
\begin{Lemma}[Lema 3.1, Dai y Meyn \cite{DaiSean}]
Si el modelo de flujo definido por \ref{Eq.3.8}-\ref{Eq.3.13} es
estable, entonces el modelo de flujo retrasado es tambi\'en
estable, es decir, existe $t_{0}>0$ tal que
$\overline{Q}\left(t\right)=0$ para cualquier $t\geq t_{0}$, para
cualquier soluci\'on del modelo de flujo retrasado cuya
condici\'on inicial $\overline{x}$ satisface que
$|\overline{x}|=|\overline{Q}\left(0\right)|+|\overline{A}\left(0\right)|+|\overline{B}\left(0\right)|\leq1$.
\end{Lemma}

%_________________________________________________________________________
\subsection{Modelo de Visitas C\'iclicas con un Servidor: Estabilidad}
%_________________________________________________________________________

%_________________________________________________________________________
\subsection{Teorema 2.1}
%_________________________________________________________________________



El resultado principal de Down \cite{Down} que relaciona la estabilidad del modelo de flujo con la estabilidad del sistema original

\begin{Teo}[Teorema 2.1, Down \cite{Down}]\label{Tma.2.1.Down}
Suponga que el modelo de flujo es estable, y que se cumplen los supuestos (A1) y (A2), entonces
\begin{itemize}
\item[i)] Para alguna constante $\kappa_{p}$, y para cada
condici\'on inicial $x\in X$
\begin{equation}\label{Estability.Eq1}
lim_{t\rightarrow\infty}\sup\frac{1}{t}\int_{0}^{t}\esp_{x}\left[|Q\left(s\right)|^{p}\right]ds\leq\kappa_{p},
\end{equation}
donde $p$ es el entero dado en (A2). Si adem\'as se cumple
la condici\'on (A3), entonces para cada condici\'on inicial:

\item[ii)] Los momentos transitorios convergen a su estado estacionario:
 \begin{equation}\label{Estability.Eq2}
lim_{t\rightarrow\infty}\esp_{x}\left[Q_{k}\left(t\right)^{r}\right]=\esp_{\pi}\left[Q_{k}\left(0\right)^{r}\right]\leq\kappa_{r},
\end{equation}
para $r=1,2,\ldots,p$ y $k=1,2,\ldots,K$. Donde $\pi$ es la
probabilidad invariante para $\mathbf{X}$.

\item[iii)]  El primer momento converge con raz\'on $t^{p-1}$:
\begin{equation}\label{Estability.Eq3}
lim_{t\rightarrow\infty}t^{p-1}|\esp_{x}\left[Q_{k}\left(t\right)\right]-\esp_{\pi}\left[Q\left(0\right)\right]=0.
\end{equation}

\item[iv)] La {\em Ley Fuerte de los grandes n\'umeros} se cumple:
\begin{equation}\label{Estability.Eq4}
lim_{t\rightarrow\infty}\frac{1}{t}\int_{0}^{t}Q_{k}^{r}\left(s\right)ds=\esp_{\pi}\left[Q_{k}\left(0\right)^{r}\right],\textrm{
}\prob_{x}\textrm{-c.s.}
\end{equation}
para $r=1,2,\ldots,p$ y $k=1,2,\ldots,K$.
\end{itemize}
\end{Teo}


\begin{Prop}[Proposici\'on 5.1, Dai y Meyn \cite{DaiSean}]\label{Prop.5.1.DaiSean}
Suponga que los supuestos A1) y A2) son ciertos y que el modelo de flujo es estable. Entonces existe $t_{0}>0$ tal que
\begin{equation}
lim_{|x|\rightarrow\infty}\frac{1}{|x|^{p+1}}\esp_{x}\left[|X\left(t_{0}|x|\right)|^{p+1}\right]=0
\end{equation}
\end{Prop}

\begin{Lemma}[Lema 5.2, Dai y Meyn \cite{DaiSean}]\label{Lema.5.2.DaiSean}
 Sea $\left\{\zeta\left(k\right):k\in \mathbb{z}\right\}$ una sucesi\'on independiente e id\'enticamente distribuida que toma valores en $\left(0,\infty\right)$,
y sea
$E\left(t\right)=max\left(n\geq1:\zeta\left(1\right)+\cdots+\zeta\left(n-1\right)\leq
t\right)$. Si $\esp\left[\zeta\left(1\right)\right]<\infty$,
entonces para cualquier entero $r\geq1$
\begin{equation}
 lim_{t\rightarrow\infty}\esp\left[\left(\frac{E\left(t\right)}{t}\right)^{r}\right]=\left(\frac{1}{\esp\left[\zeta_{1}\right]}\right)^{r}.
\end{equation}
Luego, bajo estas condiciones:
\begin{itemize}
 \item[a)] para cualquier $\delta>0$, $\sup_{t\geq\delta}\esp\left[\left(\frac{E\left(t\right)}{t}\right)^{r}\right]<\infty$
\item[b)] las variables aleatorias
$\left\{\left(\frac{E\left(t\right)}{t}\right)^{r}:t\geq1\right\}$
son uniformemente integrables.
\end{itemize}
\end{Lemma}

\begin{Teo}[Teorema 5.5, Dai y Meyn \cite{DaiSean}]\label{Tma.5.5.DaiSean}
Suponga que los supuestos A1) y A2) se cumplen y que el modelo de
flujo es estable. Entonces existe una constante $\kappa_{p}$ tal
que
\begin{equation}
\frac{1}{t}\int_{0}^{t}\esp_{x}\left[|Q\left(s\right)|^{p}\right]ds\leq\kappa_{p}\left\{\frac{1}{t}|x|^{p+1}+1\right\}
\end{equation}
para $t>0$ y $x\in X$. En particular, para cada condici\'on
inicial
\begin{eqnarray*}
\limsup_{t\rightarrow\infty}\frac{1}{t}\int_{0}^{t}\esp_{x}\left[|Q\left(s\right)|^{p}\right]ds\leq\kappa_{p}.
\end{eqnarray*}
\end{Teo}

\begin{Teo}[Teorema 6.2, Dai y Meyn \cite{DaiSean}]\label{Tma.6.2.DaiSean}
Suponga que se cumplen los supuestos A1), A2) y A3) y que el
modelo de flujo es estable. Entonces se tiene que
\begin{equation}
\left\|P^{t}\left(x,\cdot\right)-\pi\left(\cdot\right)\right\|_{f_{p}}\textrm{,
}t\rightarrow\infty,x\in X.
\end{equation}
En particular para cada condici\'on inicial
\begin{eqnarray*}
\lim_{t\rightarrow\infty}\esp_{x}\left[|Q\left(t\right)|^{p}\right]=\esp_{\pi}\left[|Q\left(0\right)|^{p}\right]\leq\kappa_{r}
\end{eqnarray*}
\end{Teo}
\begin{Teo}[Teorema 6.3, Dai y Meyn \cite{DaiSean}]\label{Tma.6.3.DaiSean}
Suponga que se cumplen los supuestos A1), A2) y A3) y que el
modelo de flujo es estable. Entonces con
$f\left(x\right)=f_{1}\left(x\right)$ se tiene
\begin{equation}
\lim_{t\rightarrow\infty}t^{p-1}\left\|P^{t}\left(x,\cdot\right)-\pi\left(\cdot\right)\right\|_{f}=0.
\end{equation}
En particular para cada condici\'on inicial
\begin{eqnarray*}
\lim_{t\rightarrow\infty}t^{p-1}|\esp_{x}\left[Q\left(t\right)\right]-\esp_{\pi}\left[Q\left(0\right)\right]|=0.
\end{eqnarray*}
\end{Teo}

\begin{Teo}[Teorema 6.4, Dai y Meyn \cite{DaiSean}]\label{Tma.6.4.DaiSean}
Suponga que se cumplen los supuestos A1), A2) y A3) y que el
modelo de flujo es estable. Sea $\nu$ cualquier distribuci\'on de
probabilidad en $\left(X,\mathcal{B}_{X}\right)$, y $\pi$ la
distribuci\'on estacionaria de $X$.
\begin{itemize}
\item[i)] Para cualquier $f:X\leftarrow\rea_{+}$
\begin{equation}
\lim_{t\rightarrow\infty}\frac{1}{t}\int_{o}^{t}f\left(X\left(s\right)\right)ds=\pi\left(f\right):=\int
f\left(x\right)\pi\left(dx\right)
\end{equation}
$\prob$-c.s.

\item[ii)] Para cualquier $f:X\leftarrow\rea_{+}$ con
$\pi\left(|f|\right)<\infty$, la ecuaci\'on anterior se cumple.
\end{itemize}
\end{Teo}

%_________________________________________________________________________
\subsection{Teorema 2.2}
%_________________________________________________________________________

\begin{Teo}[Teorema 2.2, Down \cite{Down}]\label{Tma2.2.Down}
Suponga que el fluido modelo es inestable en el sentido de que
para alguna $\epsilon_{0},c_{0}\geq0$,
\begin{equation}\label{Eq.Inestability}
|Q\left(T\right)|\geq\epsilon_{0}T-c_{0}\textrm{,   }T\geq0,
\end{equation}
para cualquier condici\'on inicial $Q\left(0\right)$, con
$|Q\left(0\right)|=1$. Entonces para cualquier $0<q\leq1$, existe
$B<0$ tal que para cualquier $|x|\geq B$,
\begin{equation}
\prob_{x}\left\{\mathbb{X}\rightarrow\infty\right\}\geq q.
\end{equation}
\end{Teo}

%_________________________________________________________________________
\subsection{Teorema 2.3}
%_________________________________________________________________________
\begin{Teo}[Teorema 2.3, Down \cite{Down}]\label{Tma2.3.Down}
Considere el siguiente valor:
\begin{equation}\label{Eq.Rho.1serv}
\rho=\sum_{k=1}^{K}\rho_{k}+max_{1\leq j\leq K}\left(\frac{\lambda_{j}}{\sum_{s=1}^{S}p_{js}\overline{N}_{s}}\right)\delta^{*}
\end{equation}
\begin{itemize}
\item[i)] Si $\rho<1$ entonces la red es estable, es decir, se cumple el teorema \ref{Tma.2.1.Down}.

\item[ii)] Si $\rho<1$ entonces la red es inestable, es decir, se cumple el teorema \ref{Tma2.2.Down}
\end{itemize}
\end{Teo}
%_____________________________________________________________________
\subsection{Definiciones  B\'asicas}
%_____________________________________________________________________
\begin{Def}
Sea $X$ un conjunto y $\mathcal{F}$ una $\sigma$-\'algebra de
subconjuntos de $X$, la pareja $\left(X,\mathcal{F}\right)$ es
llamado espacio medible. Un subconjunto $A$ de $X$ es llamado
medible, o medible con respecto a $\mathcal{F}$, si
$A\in\mathcal{F}$.
\end{Def}

\begin{Def}
Sea $\left(X,\mathcal{F},\mu\right)$ espacio de medida. Se dice
que la medida $\mu$ es $\sigma$-finita si se puede escribir
$X=\bigcup_{n\geq1}X_{n}$ con $X_{n}\in\mathcal{F}$ y
$\mu\left(X_{n}\right)<\infty$.
\end{Def}

\begin{Def}\label{Cto.Borel}
Sea $X$ el conjunto de los \'umeros reales $\rea$. El \'algebra de
Borel es la $\sigma$-\'algebra $B$ generada por los intervalos
abiertos $\left(a,b\right)\in\rea$. Cualquier conjunto en $B$ es
llamado {\em Conjunto de Borel}.
\end{Def}

\begin{Def}\label{Funcion.Medible}
Una funci\'on $f:X\rightarrow\rea$, es medible si para cualquier
n\'umero real $\alpha$ el conjunto
\[\left\{x\in X:f\left(x\right)>\alpha\right\}\]
pertenece a $X$. Equivalentemente, se dice que $f$ es medible si
\[f^{-1}\left(\left(\alpha,\infty\right)\right)=\left\{x\in X:f\left(x\right)>\alpha\right\}\in\mathcal{F}.\]
\end{Def}


\begin{Def}\label{Def.Cilindros}
Sean $\left(\Omega_{i},\mathcal{F}_{i}\right)$, $i=1,2,\ldots,$
espacios medibles y $\Omega=\prod_{i=1}^{\infty}\Omega_{i}$ el
conjunto de todas las sucesiones
$\left(\omega_{1},\omega_{2},\ldots,\right)$ tales que
$\omega_{i}\in\Omega_{i}$, $i=1,2,\ldots,$. Si
$B^{n}\subset\prod_{i=1}^{\infty}\Omega_{i}$, definimos
$B_{n}=\left\{\omega\in\Omega:\left(\omega_{1},\omega_{2},\ldots,\omega_{n}\right)\in
B^{n}\right\}$. Al conjunto $B_{n}$ se le llama {\em cilindro} con
base $B^{n}$, el cilindro es llamado medible si
$B^{n}\in\prod_{i=1}^{\infty}\mathcal{F}_{i}$.
\end{Def}


\begin{Def}\label{Def.Proc.Adaptado}[TSP, Ash \cite{RBA}]
Sea $X\left(t\right),t\geq0$ proceso estoc\'astico, el proceso es
adaptado a la familia de $\sigma$-\'algebras $\mathcal{F}_{t}$,
para $t\geq0$, si para $s<t$ implica que
$\mathcal{F}_{s}\subset\mathcal{F}_{t}$, y $X\left(t\right)$ es
$\mathcal{F}_{t}$-medible para cada $t$. Si no se especifica
$\mathcal{F}_{t}$ entonces se toma $\mathcal{F}_{t}$ como
$\mathcal{F}\left(X\left(s\right),s\leq t\right)$, la m\'as
peque\~na $\sigma$-\'algebra de subconjuntos de $\Omega$ que hace
que cada $X\left(s\right)$, con $s\leq t$ sea Borel medible.
\end{Def}


\begin{Def}\label{Def.Tiempo.Paro}[TSP, Ash \cite{RBA}]
Sea $\left\{\mathcal{F}\left(t\right),t\geq0\right\}$ familia
creciente de sub $\sigma$-\'algebras. es decir,
$\mathcal{F}\left(s\right)\subset\mathcal{F}\left(t\right)$ para
$s\leq t$. Un tiempo de paro para $\mathcal{F}\left(t\right)$ es
una funci\'on $T:\Omega\rightarrow\left[0,\infty\right]$ tal que
$\left\{T\leq t\right\}\in\mathcal{F}\left(t\right)$ para cada
$t\geq0$. Un tiempo de paro para el proceso estoc\'astico
$X\left(t\right),t\geq0$ es un tiempo de paro para las
$\sigma$-\'algebras
$\mathcal{F}\left(t\right)=\mathcal{F}\left(X\left(s\right)\right)$.
\end{Def}

\begin{Def}
Sea $X\left(t\right),t\geq0$ proceso estoc\'astico, con
$\left(S,\chi\right)$ espacio de estados. Se dice que el proceso
es adaptado a $\left\{\mathcal{F}\left(t\right)\right\}$, es
decir, si para cualquier $s,t\in I$, $I$ conjunto de \'indices,
$s<t$, se tiene que
$\mathcal{F}\left(s\right)\subset\mathcal{F}\left(t\right)$ y
$X\left(t\right)$ es $\mathcal{F}\left(t\right)$-medible,
\end{Def}

\begin{Def}
Sea $X\left(t\right),t\geq0$ proceso estoc\'astico, se dice que es
un Proceso de Markov relativo a $\mathcal{F}\left(t\right)$ o que
$\left\{X\left(t\right),\mathcal{F}\left(t\right)\right\}$ es de
Markov si y s\'olo si para cualquier conjunto $B\in\chi$,  y
$s,t\in I$, $s<t$ se cumple que
\begin{equation}\label{Prop.Markov}
P\left\{X\left(t\right)\in
B|\mathcal{F}\left(s\right)\right\}=P\left\{X\left(t\right)\in
B|X\left(s\right)\right\}.
\end{equation}
\end{Def}
\begin{Note}
Si se dice que $\left\{X\left(t\right)\right\}$ es un Proceso de
Markov sin mencionar $\mathcal{F}\left(t\right)$, se asumir\'a que
\begin{eqnarray*}
\mathcal{F}\left(t\right)=\mathcal{F}_{0}\left(t\right)=\mathcal{F}\left(X\left(r\right),r\leq
t\right),
\end{eqnarray*}
entonces la ecuaci\'on (\ref{Prop.Markov}) se puede escribir como
\begin{equation}
P\left\{X\left(t\right)\in B|X\left(r\right),r\leq s\right\} =
P\left\{X\left(t\right)\in B|X\left(s\right)\right\}
\end{equation}
\end{Note}

\begin{Teo}
Sea $\left(X_{n},\mathcal{F}_{n},n=0,1,\ldots,\right\}$ Proceso de
Markov con espacio de estados $\left(S_{0},\chi_{0}\right)$
generado por una distribuici\'on inicial $P_{o}$ y probabilidad de
transici\'on $p_{mn}$, para $m,n=0,1,\ldots,$ $m<n$, que por
notaci\'on se escribir\'a como $p\left(m,n,x,B\right)\rightarrow
p_{mn}\left(x,B\right)$. Sea $S$ tiempo de paro relativo a la
$\sigma$-\'algebra $\mathcal{F}_{n}$. Sea $T$ funci\'on medible,
$T:\Omega\rightarrow\left\{0,1,\ldots,\right\}$. Sup\'ongase que
$T\geq S$, entonces $T$ es tiempo de paro. Si $B\in\chi_{0}$,
entonces
\begin{equation}\label{Prop.Fuerte.Markov}
P\left\{X\left(T\right)\in
B,T<\infty|\mathcal{F}\left(S\right)\right\} =
p\left(S,T,X\left(s\right),B\right)
\end{equation}
en $\left\{T<\infty\right\}$.
\end{Teo}

Propiedades importantes para el modelo de flujo retrasado:

\begin{Prop}
 Sea $\left(\overline{Q},\overline{T},\overline{T}^{0}\right)$ un flujo l\'imite de \ref{Equation.4.4} y suponga que cuando $x\rightarrow\infty$ a lo largo de
una subsucesi\'on
\[\left(\frac{1}{|x|}Q_{k}^{x}\left(0\right),\frac{1}{|x|}A_{k}^{x}\left(0\right),\frac{1}{|x|}B_{k}^{x}\left(0\right),\frac{1}{|x|}B_{k}^{x,0}\left(0\right)\right)\rightarrow\left(\overline{Q}_{k}\left(0\right),0,0,0\right)\]
para $k=1,\ldots,K$. EL flujo l\'imite tiene las siguientes
propiedades, donde las propiedades de la derivada se cumplen donde
la derivada exista:
\begin{itemize}
 \item[i)] Los vectores de tiempo ocupado $\overline{T}\left(t\right)$ y $\overline{T}^{0}\left(t\right)$ son crecientes y continuas con
$\overline{T}\left(0\right)=\overline{T}^{0}\left(0\right)=0$.
\item[ii)] Para todo $t\geq0$
\[\sum_{k=1}^{K}\left[\overline{T}_{k}\left(t\right)+\overline{T}_{k}^{0}\left(t\right)\right]=t\]
\item[iii)] Para todo $1\leq k\leq K$
\[\overline{Q}_{k}\left(t\right)=\overline{Q}_{k}\left(0\right)+\alpha_{k}t-\mu_{k}\overline{T}_{k}\left(t\right)\]
\item[iv)]  Para todo $1\leq k\leq K$
\[\dot{{\overline{T}}}_{k}\left(t\right)=\beta_{k}\] para $\overline{Q}_{k}\left(t\right)=0$.
\item[v)] Para todo $k,j$
\[\mu_{k}^{0}\overline{T}_{k}^{0}\left(t\right)=\mu_{j}^{0}\overline{T}_{j}^{0}\left(t\right)\]
\item[vi)]  Para todo $1\leq k\leq K$
\[\mu_{k}\dot{{\overline{T}}}_{k}\left(t\right)=l_{k}\mu_{k}^{0}\dot{{\overline{T}}}_{k}^{0}\left(t\right)\] para $\overline{Q}_{k}\left(t\right)>0$.
\end{itemize}
\end{Prop}

\begin{Lema}[Lema 3.1 \cite{Chen}]\label{Lema3.1}
Si el modelo de flujo es estable, definido por las ecuaciones
(3.8)-(3.13), entonces el modelo de flujo retrasado tambin es
estable.
\end{Lema}

\begin{Teo}[Teorema 5.2 \cite{Chen}]\label{Tma.5.2}
Si el modelo de flujo lineal correspondiente a la red de cola es
estable, entonces la red de colas es estable.
\end{Teo}

\begin{Teo}[Teorema 5.1 \cite{Chen}]\label{Tma.5.1.Chen}
La red de colas es estable si existe una constante $t_{0}$ que
depende de $\left(\alpha,\mu,T,U\right)$ y $V$ que satisfagan las
ecuaciones (5.1)-(5.5), $Z\left(t\right)=0$, para toda $t\geq
t_{0}$.
\end{Teo}



\begin{Lema}[Lema 5.2 \cite{Gut}]\label{Lema.5.2.Gut}
Sea $\left\{\xi\left(k\right):k\in\ent\right\}$ sucesin de
variables aleatorias i.i.d. con valores en
$\left(0,\infty\right)$, y sea $E\left(t\right)$ el proceso de
conteo
\[E\left(t\right)=max\left\{n\geq1:\xi\left(1\right)+\cdots+\xi\left(n-1\right)\leq t\right\}.\]
Si $E\left[\xi\left(1\right)\right]<\infty$, entonces para
cualquier entero $r\geq1$
\begin{equation}
lim_{t\rightarrow\infty}\esp\left[\left(\frac{E\left(t\right)}{t}\right)^{r}\right]=\left(\frac{1}{E\left[\xi_{1}\right]}\right)^{r}
\end{equation}
de aqu, bajo estas condiciones
\begin{itemize}
\item[a)] Para cualquier $t>0$,
$sup_{t\geq\delta}\esp\left[\left(\frac{E\left(t\right)}{t}\right)^{r}\right]$

\item[b)] Las variables aleatorias
$\left\{\left(\frac{E\left(t\right)}{t}\right)^{r}:t\geq1\right\}$
son uniformemente integrables.
\end{itemize}
\end{Lema}

\begin{Teo}[Teorema 5.1: Ley Fuerte para Procesos de Conteo
\cite{Gut}]\label{Tma.5.1.Gut} Sea
$0<\mu<\esp\left(X_{1}\right]\leq\infty$. entonces

\begin{itemize}
\item[a)] $\frac{N\left(t\right)}{t}\rightarrow\frac{1}{\mu}$
a.s., cuando $t\rightarrow\infty$.


\item[b)]$\esp\left[\frac{N\left(t\right)}{t}\right]^{r}\rightarrow\frac{1}{\mu^{r}}$,
cuando $t\rightarrow\infty$ para todo $r>0$..
\end{itemize}
\end{Teo}


\begin{Prop}[Proposicin 5.1 \cite{DaiSean}]\label{Prop.5.1}
Suponga que los supuestos (A1) y (A2) se cumplen, adems suponga
que el modelo de flujo es estable. Entonces existe $t_{0}>0$ tal
que
\begin{equation}\label{Eq.Prop.5.1}
lim_{|x|\rightarrow\infty}\frac{1}{|x|^{p+1}}\esp_{x}\left[|X\left(t_{0}|x|\right)|^{p+1}\right]=0.
\end{equation}

\end{Prop}


\begin{Prop}[Proposici\'on 5.3 \cite{DaiSean}]
Sea $X$ proceso de estados para la red de colas, y suponga que se
cumplen los supuestos (A1) y (A2), entonces para alguna constante
positiva $C_{p+1}<\infty$, $\delta>0$ y un conjunto compacto
$C\subset X$.

\begin{equation}\label{Eq.5.4}
\esp_{x}\left[\int_{0}^{\tau_{C}\left(\delta\right)}\left(1+|X\left(t\right)|^{p}\right)dt\right]\leq
C_{p+1}\left(1+|x|^{p+1}\right)
\end{equation}
\end{Prop}

\begin{Prop}[Proposici\'on 5.4 \cite{DaiSean}]
Sea $X$ un proceso de Markov Borel Derecho en $X$, sea
$f:X\leftarrow\rea_{+}$ y defina para alguna $\delta>0$, y un
conjunto cerrado $C\subset X$
\[V\left(x\right):=\esp_{x}\left[\int_{0}^{\tau_{C}\left(\delta\right)}f\left(X\left(t\right)\right)dt\right]\]
para $x\in X$. Si $V$ es finito en todas partes y uniformemente
acotada en $C$, entonces existe $k<\infty$ tal que
\begin{equation}\label{Eq.5.11}
\frac{1}{t}\esp_{x}\left[V\left(x\right)\right]+\frac{1}{t}\int_{0}^{t}\esp_{x}\left[f\left(X\left(s\right)\right)ds\right]\leq\frac{1}{t}V\left(x\right)+k,
\end{equation}
para $x\in X$ y $t>0$.
\end{Prop}


\begin{Teo}[Teorema 5.5 \cite{DaiSean}]
Suponga que se cumplen (A1) y (A2), adems suponga que el modelo
de flujo es estable. Entonces existe una constante $k_{p}<\infty$
tal que
\begin{equation}\label{Eq.5.13}
\frac{1}{t}\int_{0}^{t}\esp_{x}\left[|Q\left(s\right)|^{p}\right]ds\leq
k_{p}\left\{\frac{1}{t}|x|^{p+1}+1\right\}
\end{equation}
para $t\geq0$, $x\in X$. En particular para cada condicin inicial
\begin{equation}\label{Eq.5.14}
Limsup_{t\rightarrow\infty}\frac{1}{t}\int_{0}^{t}\esp_{x}\left[|Q\left(s\right)|^{p}\right]ds\leq
k_{p}
\end{equation}
\end{Teo}

\begin{Teo}[Teorema 6.2\cite{DaiSean}]\label{Tma.6.2}
Suponga que se cumplen los supuestos (A1)-(A3) y que el modelo de
flujo es estable, entonces se tiene que
\[\parallel P^{t}\left(c,\cdot\right)-\pi\left(\cdot\right)\parallel_{f_{p}}\rightarrow0\]
para $t\rightarrow\infty$ y $x\in X$. En particular para cada
condicin inicial
\[lim_{t\rightarrow\infty}\esp_{x}\left[\left|Q_{t}\right|^{p}\right]=\esp_{\pi}\left[\left|Q_{0}\right|^{p}\right]<\infty\]
\end{Teo}


\begin{Teo}[Teorema 6.3\cite{DaiSean}]\label{Tma.6.3}
Suponga que se cumplen los supuestos (A1)-(A3) y que el modelo de
flujo es estable, entonces con
$f\left(x\right)=f_{1}\left(x\right)$, se tiene que
\[lim_{t\rightarrow\infty}t^{(p-1)\left|P^{t}\left(c,\cdot\right)-\pi\left(\cdot\right)\right|_{f}=0},\]
para $x\in X$. En particular, para cada condicin inicial
\[lim_{t\rightarrow\infty}t^{(p-1)\left|\esp_{x}\left[Q_{t}\right]-\esp_{\pi}\left[Q_{0}\right]\right|=0}.\]
\end{Teo}



Si $x$ es el n{\'u}mero de usuarios en la cola al comienzo del
periodo de servicio y $N_{s}\left(x\right)=N\left(x\right)$ es el
n{\'u}mero de usuarios que son atendidos con la pol{\'\i}tica $s$,
{\'u}nica en nuestro caso, durante un periodo de servicio,
entonces se asume que:
\begin{itemize}
\item[(S1.)]
\begin{equation}\label{S1}
lim_{x\rightarrow\infty}\esp\left[N\left(x\right)\right]=\overline{N}>0.
\end{equation}
\item[(S2.)]
\begin{equation}\label{S2}
\esp\left[N\left(x\right)\right]\leq \overline{N}, \end{equation}
para cualquier valor de $x$. \item La $n$-{\'e}sima ocurrencia va
acompa{\~n}ada con el tiempo de cambio de longitud
$\delta_{j,j+1}\left(n\right)$, independientes e id{\'e}nticamente
distribuidas, con
$\esp\left[\delta_{j,j+1}\left(1\right)\right]\geq0$. \item Se
define
\begin{equation}
\delta^{*}:=\sum_{j,j+1}\esp\left[\delta_{j,j+1}\left(1\right)\right].
\end{equation}

\item Los tiempos de inter-arribo a la cola $k$,son de la forma
$\left\{\xi_{k}\left(n\right)\right\}_{n\geq1}$, con la propiedad
de que son independientes e id{\'e}nticamente distribuidos.

\item Los tiempos de servicio
$\left\{\eta_{k}\left(n\right)\right\}_{n\geq1}$ tienen la
propiedad de ser independientes e id{\'e}nticamente distribuidos.

\item Se define la tasa de arribo a la $k$-{\'e}sima cola como
$\lambda_{k}=1/\esp\left[\xi_{k}\left(1\right)\right]$ y
adem{\'a}s se define

\item la tasa de servicio para la $k$-{\'e}sima cola como
$\mu_{k}=1/\esp\left[\eta_{k}\left(1\right)\right]$

\item tambi{\'e}n se define $\rho_{k}=\lambda_{k}/\mu_{k}$, donde
es necesario que $\rho<1$ para cuestiones de estabilidad.

\item De las pol{\'\i}ticas posibles solamente consideraremos la
pol{\'\i}tica cerrada (Gated).
\end{itemize}

Las Colas C\'iclicas se pueden describir por medio de un proceso
de Markov $\left(X\left(t\right)\right)_{t\in\rea}$, donde el
estado del sistema al tiempo $t\geq0$ est\'a dado por
\begin{equation}
X\left(t\right)=\left(Q\left(t\right),A\left(t\right),H\left(t\right),B\left(t\right),B^{0}\left(t\right),C\left(t\right)\right)
\end{equation}
definido en el espacio producto:
\begin{equation}
\mathcal{X}=\mathbb{Z}^{K}\times\rea_{+}^{K}\times\left(\left\{1,2,\ldots,K\right\}\times\left\{1,2,\ldots,S\right\}\right)^{M}\times\rea_{+}^{K}\times\rea_{+}^{K}\times\mathbb{Z}^{K},
\end{equation}

\begin{itemize}
\item $Q\left(t\right)=\left(Q_{k}\left(t\right),1\leq k\leq
K\right)$, es el n\'umero de usuarios en la cola $k$, incluyendo
aquellos que est\'an siendo atendidos provenientes de la
$k$-\'esima cola.

\item $A\left(t\right)=\left(A_{k}\left(t\right),1\leq k\leq
K\right)$, son los residuales de los tiempos de arribo en la cola
$k$. \item $H\left(t\right)$ es el par ordenado que consiste en la
cola que esta siendo atendida y la pol\'itica de servicio que se
utilizar\'a.

\item $B\left(t\right)$ es el tiempo de servicio residual.

\item $B^{0}\left(t\right)$ es el tiempo residual del cambio de
cola.

\item $C\left(t\right)$ indica el n\'umero de usuarios atendidos
durante la visita del servidor a la cola dada en
$H\left(t\right)$.
\end{itemize}

$A_{k}\left(t\right),B_{m}\left(t\right)$ y
$B_{m}^{0}\left(t\right)$ se suponen continuas por la derecha y
que satisfacen la propiedad fuerte de Markov, (\cite{Dai})

\begin{itemize}
\item Los tiempos de interarribo a la cola $k$,son de la forma
$\left\{\xi_{k}\left(n\right)\right\}_{n\geq1}$, con la propiedad
de que son independientes e id{\'e}nticamente distribuidos.

\item Los tiempos de servicio
$\left\{\eta_{k}\left(n\right)\right\}_{n\geq1}$ tienen la
propiedad de ser independientes e id{\'e}nticamente distribuidos.

\item Se define la tasa de arribo a la $k$-{\'e}sima cola como
$\lambda_{k}=1/\esp\left[\xi_{k}\left(1\right)\right]$ y
adem{\'a}s se define

\item la tasa de servicio para la $k$-{\'e}sima cola como
$\mu_{k}=1/\esp\left[\eta_{k}\left(1\right)\right]$

\item tambi{\'e}n se define $\rho_{k}=\lambda_{k}/\mu_{k}$, donde
es necesario que $\rho<1$ para cuestiones de estabilidad.

\item De las pol{\'\i}ticas posibles solamente consideraremos la
pol{\'\i}tica cerrada (Gated).
\end{itemize}


%_____________________________________________________


\subsection{Preliminares}



Sup\'ongase que el sistema consta de varias colas a los cuales
llegan uno o varios servidores a dar servicio a los usuarios
esperando en la cola.\\


Si $x$ es el n\'umero de usuarios en la cola al comienzo del
periodo de servicio y $N_{s}\left(x\right)=N\left(x\right)$ es el
n\'umero de usuarios que son atendidos con la pol\'itica $s$,
\'unica en nuestro caso, durante un periodo de servicio, entonces
se asume que:
\begin{itemize}
\item[1)]\label{S1}$lim_{x\rightarrow\infty}\esp\left[N\left(x\right)\right]=\overline{N}>0$
\item[2)]\label{S2}$\esp\left[N\left(x\right)\right]\leq\overline{N}$para
cualquier valor de $x$.
\end{itemize}
La manera en que atiende el servidor $m$-\'esimo, en este caso en
espec\'ifico solo lo ilustraremos con un s\'olo servidor, es la
siguiente:
\begin{itemize}
\item Al t\'ermino de la visita a la cola $j$, el servidor se
cambia a la cola $j^{'}$ con probabilidad
$r_{j,j^{'}}^{m}=r_{j,j^{'}}$

\item La $n$-\'esima ocurrencia va acompa\~nada con el tiempo de
cambio de longitud $\delta_{j,j^{'}}\left(n\right)$,
independientes e id\'enticamente distribuidas, con
$\esp\left[\delta_{j,j^{'}}\left(1\right)\right]\geq0$.

\item Sea $\left\{p_{j}\right\}$ la distribuci\'on invariante
estacionaria \'unica para la Cadena de Markov con matriz de
transici\'on $\left(r_{j,j^{'}}\right)$.

\item Finalmente, se define
\begin{equation}
\delta^{*}:=\sum_{j,j^{'}}p_{j}r_{j,j^{'}}\esp\left[\delta_{j,j^{'}}\left(i\right)\right].
\end{equation}
\end{itemize}

Veamos un caso muy espec\'ifico en el cual los tiempos de arribo a cada una de las colas se comportan de acuerdo a un proceso Poisson de la forma
$\left\{\xi_{k}\left(n\right)\right\}_{n\geq1}$, y los tiempos de servicio en cada una de las colas son variables aleatorias distribuidas exponencialmente e id\'enticamente distribuidas
$\left\{\eta_{k}\left(n\right)\right\}_{n\geq1}$, donde ambos procesos adem\'as cumplen la condici\'on de ser independientes entre si. Para la $k$-\'esima cola se define la tasa de arribo a la como
$\lambda_{k}=1/\esp\left[\xi_{k}\left(1\right)\right]$ y la tasa
de servicio como
$\mu_{k}=1/\esp\left[\eta_{k}\left(1\right)\right]$, finalmente se
define la carga de la cola como $\rho_{k}=\lambda_{k}/\mu_{k}$,
donde se pide que $\rho<1$, para garantizar la estabilidad del sistema.\\

Se denotar\'a por $Q_{k}\left(t\right)$ el n\'umero de usuarios en la cola $k$,
$A_{k}\left(t\right)$ los residuales de los tiempos entre arribos a la cola $k$;
para cada servidor $m$, se denota por $B_{m}\left(t\right)$ los residuales de los tiempos de servicio al tiempo $t$; $B_{m}^{0}\left(t\right)$ son los residuales de los tiempos de traslado de la cola $k$ a la pr\'oxima por atender, al tiempo $t$, finalmente sea $C_{m}\left(t\right)$ el n\'umero de usuarios atendidos durante la visita del servidor a la cola $k$ al tiempo $t$.\\


En este sentido el proceso para el sistema de visitas se puede definir como:

\begin{equation}\label{Esp.Edos.Down}
X\left(t\right)^{T}=\left(Q_{k}\left(t\right),A_{k}\left(t\right),B_{m}\left(t\right),B_{m}^{0}\left(t\right),C_{m}\left(t\right)\right)
\end{equation}
para $k=1,\ldots,K$ y $m=1,2,\ldots,M$. $X$ evoluciona en el
espacio de estados:
$X=\ent_{+}^{K}\times\rea_{+}^{K}\times\left(\left\{1,2,\ldots,K\right\}\times\left\{1,2,\ldots,S\right\}\right)^{M}\times\rea_{+}^{K}\times\ent_{+}^{K}$.\\

El sistema aqu\'i descrito debe de cumplir con los siguientes supuestos b\'asicos de un sistema de visitas:

Antes enunciemos los supuestos que regir\'an en la red.

\begin{itemize}
\item[A1)] $\xi_{1},\ldots,\xi_{K},\eta_{1},\ldots,\eta_{K}$ son
mutuamente independientes y son sucesiones independientes e
id\'enticamente distribuidas.

\item[A2)] Para alg\'un entero $p\geq1$
\begin{eqnarray*}
\esp\left[\xi_{l}\left(1\right)^{p+1}\right]<\infty\textrm{ para }l\in\mathcal{A}\textrm{ y }\\
\esp\left[\eta_{k}\left(1\right)^{p+1}\right]<\infty\textrm{ para
}k=1,\ldots,K.
\end{eqnarray*}
donde $\mathcal{A}$ es la clase de posibles arribos.

\item[A3)] Para $k=1,2,\ldots,K$ existe una funci\'on positiva
$q_{k}\left(x\right)$ definida en $\rea_{+}$, y un entero $j_{k}$,
tal que
\begin{eqnarray}
P\left(\xi_{k}\left(1\right)\geq x\right)>0\textrm{, para todo }x>0\\
P\left\{a\leq\sum_{i=1}^{j_{k}}\xi_{k}\left(i\right)\leq
b\right\}\geq\int_{a}^{b}q_{k}\left(x\right)dx, \textrm{ }0\leq
a<b.
\end{eqnarray}
\end{itemize}

En particular los procesos de tiempo entre arribos y de servicio
considerados con fines de ilustraci\'on de la metodolog\'ia
cumplen con el supuesto $A2)$ para $p=1$, es decir, ambos procesos
tienen primer y segundo momento finito.

En lo que respecta al supuesto (A3), en Dai y Meyn \cite{DaiSean}
hacen ver que este se puede sustituir por

\begin{itemize}
\item[A3')] Para el Proceso de Markov $X$, cada subconjunto
compacto de $X$ es un conjunto peque\~no, ver definici\'on
\ref{Def.Cto.Peq.}.
\end{itemize}

Es por esta raz\'on que con la finalidad de poder hacer uso de
$A3^{'})$ es necesario recurrir a los Procesos de Harris y en
particular a los Procesos Harris Recurrente:
%_______________________________________________________________________
\subsection{Procesos Harris Recurrente}
%_______________________________________________________________________

Por el supuesto (A1) conforme a Davis \cite{Davis}, se puede
definir el proceso de saltos correspondiente de manera tal que
satisfaga el supuesto (\ref{Sup3.1.Davis}), de hecho la
demostraci\'on est\'a basada en la l\'inea de argumentaci\'on de
Davis, (\cite{Davis}, p\'aginas 362-364).

Entonces se tiene un espacio de estados Markoviano. El espacio de
Markov descrito en Dai y Meyn \cite{DaiSean}

\[\left(\Omega,\mathcal{F},\mathcal{F}_{t},X\left(t\right),\theta_{t},P_{x}\right)\]
es un proceso de Borel Derecho (Sharpe \cite{Sharpe}) en el
espacio de estados medible $\left(X,\mathcal{B}_{X}\right)$. El
Proceso $X=\left\{X\left(t\right),t\geq0\right\}$ tiene
trayectorias continuas por la derecha, est\'a definida en
$\left(\Omega,\mathcal{F}\right)$ y est\'a adaptado a
$\left\{\mathcal{F}_{t},t\geq0\right\}$; la colecci\'on
$\left\{P_{x},x\in \mathbb{X}\right\}$ son medidas de probabilidad
en $\left(\Omega,\mathcal{F}\right)$ tales que para todo $x\in
\mathbb{X}$
\[P_{x}\left\{X\left(0\right)=x\right\}=1\] y
\[E_{x}\left\{f\left(X\circ\theta_{t}\right)|\mathcal{F}_{t}\right\}=E_{X}\left(\tau\right)f\left(X\right)\]
en $\left\{\tau<\infty\right\}$, $P_{x}$-c.s. Donde $\tau$ es un
$\mathcal{F}_{t}$-tiempo de paro
\[\left(X\circ\theta_{\tau}\right)\left(w\right)=\left\{X\left(\tau\left(w\right)+t,w\right),t\geq0\right\}\]
y $f$ es una funci\'on de valores reales acotada y medible con la
$\sigma$-algebra de Kolmogorov generada por los cilindros.\\

Sea $P^{t}\left(x,D\right)$, $D\in\mathcal{B}_{\mathbb{X}}$,
$t\geq0$ probabilidad de transici\'on de $X$ definida como
\[P^{t}\left(x,D\right)=P_{x}\left(X\left(t\right)\in
D\right)\]


\begin{Def}
Una medida no cero $\pi$ en
$\left(\mathbf{X},\mathcal{B}_{\mathbf{X}}\right)$ es {\bf
invariante} para $X$ si $\pi$ es $\sigma$-finita y
\[\pi\left(D\right)=\int_{\mathbf{X}}P^{t}\left(x,D\right)\pi\left(dx\right)\]
para todo $D\in \mathcal{B}_{\mathbf{X}}$, con $t\geq0$.
\end{Def}

\begin{Def}
El proceso de Markov $X$ es llamado Harris recurrente si existe
una medida de probabilidad $\nu$ en
$\left(\mathbf{X},\mathcal{B}_{\mathbf{X}}\right)$, tal que si
$\nu\left(D\right)>0$ y $D\in\mathcal{B}_{\mathbf{X}}$
\[P_{x}\left\{\tau_{D}<\infty\right\}\equiv1\] cuando
$\tau_{D}=inf\left\{t\geq0:X_{t}\in D\right\}$.
\end{Def}

\begin{Note}
\begin{itemize}
\item[i)] Si $X$ es Harris recurrente, entonces existe una \'unica
medida invariante $\pi$ (Getoor \cite{Getoor}).

\item[ii)] Si la medida invariante es finita, entonces puede
normalizarse a una medida de probabilidad, en este caso se le
llama Proceso {\em Harris recurrente positivo}.


\item[iii)] Cuando $X$ es Harris recurrente positivo se dice que
la disciplina de servicio es estable. En este caso $\pi$ denota la
distribuci\'on estacionaria y hacemos
\[P_{\pi}\left(\cdot\right)=\int_{\mathbf{X}}P_{x}\left(\cdot\right)\pi\left(dx\right)\]
y se utiliza $E_{\pi}$ para denotar el operador esperanza
correspondiente.
\end{itemize}
\end{Note}

\begin{Def}\label{Def.Cto.Peq.}
Un conjunto $D\in\mathcal{B_{\mathbf{X}}}$ es llamado peque\~no si
existe un $t>0$, una medida de probabilidad $\nu$ en
$\mathcal{B_{\mathbf{X}}}$, y un $\delta>0$ tal que
\[P^{t}\left(x,A\right)\geq\delta\nu\left(A\right)\] para $x\in
D,A\in\mathcal{B_{X}}$.
\end{Def}

La siguiente serie de resultados vienen enunciados y demostrados
en Dai \cite{Dai}:
\begin{Lema}[Lema 3.1, Dai\cite{Dai}]
Sea $B$ conjunto peque\~no cerrado, supongamos que
$P_{x}\left(\tau_{B}<\infty\right)\equiv1$ y que para alg\'un
$\delta>0$ se cumple que
\begin{equation}\label{Eq.3.1}
\sup\esp_{x}\left[\tau_{B}\left(\delta\right)\right]<\infty,
\end{equation}
donde
$\tau_{B}\left(\delta\right)=inf\left\{t\geq\delta:X\left(t\right)\in
B\right\}$. Entonces, $X$ es un proceso Harris Recurrente
Positivo.
\end{Lema}

\begin{Lema}[Lema 3.1, Dai \cite{Dai}]\label{Lema.3.}
Bajo el supuesto (A3), el conjunto $B=\left\{|x|\leq k\right\}$ es
un conjunto peque\~no cerrado para cualquier $k>0$.
\end{Lema}

\begin{Teo}[Teorema 3.1, Dai\cite{Dai}]\label{Tma.3.1}
Si existe un $\delta>0$ tal que
\begin{equation}
lim_{|x|\rightarrow\infty}\frac{1}{|x|}\esp|X^{x}\left(|x|\delta\right)|=0,
\end{equation}
entonces la ecuaci\'on (\ref{Eq.3.1}) se cumple para
$B=\left\{|x|\leq k\right\}$ con alg\'un $k>0$. En particular, $X$
es Harris Recurrente Positivo.
\end{Teo}

\begin{Note}
En Meyn and Tweedie \cite{MeynTweedie} muestran que si
$P_{x}\left\{\tau_{D}<\infty\right\}\equiv1$ incluso para solo un
conjunto peque\~no, entonces el proceso es Harris Recurrente.
\end{Note}

Entonces, tenemos que el proceso $X$ es un proceso de Markov que
cumple con los supuestos $A1)$-$A3)$, lo que falta de hacer es
construir el Modelo de Flujo bas\'andonos en lo hasta ahora
presentado.
%_______________________________________________________________________
\subsection{Modelo de Flujo}
%_______________________________________________________________________

Dada una condici\'on inicial $x\in\textrm{X}$, sea
$Q_{k}^{x}\left(t\right)$ la longitud de la cola al tiempo $t$,
$T_{m,k}^{x}\left(t\right)$ el tiempo acumulado, al tiempo $t$,
que tarda el servidor $m$ en atender a los usuarios de la cola
$k$. Finalmente sea $T_{m,k}^{x,0}\left(t\right)$ el tiempo
acumulado, al tiempo $t$, que tarda el servidor $m$ en trasladarse
a otra cola a partir de la $k$-\'esima.\\

Sup\'ongase que la funci\'on
$\left(\overline{Q}\left(\cdot\right),\overline{T}_{m}
\left(\cdot\right),\overline{T}_{m}^{0} \left(\cdot\right)\right)$
para $m=1,2,\ldots,M$ es un punto l\'imite de
\begin{equation}\label{Eq.Punto.Limite}
\left(\frac{1}{|x|}Q^{x}\left(|x|t\right),\frac{1}{|x|}T_{m}^{x}\left(|x|t\right),\frac{1}{|x|}T_{m}^{x,0}\left(|x|t\right)\right)
\end{equation}
para $m=1,2,\ldots,M$, cuando $x\rightarrow\infty$. Entonces
$\left(\overline{Q}\left(t\right),\overline{T}_{m}
\left(t\right),\overline{T}_{m}^{0} \left(t\right)\right)$ es un
flujo l\'imite del sistema. Al conjunto de todos las posibles
flujos l\'imite se le llama \textbf{Modelo de Flujo}.\\

El modelo de flujo satisface el siguiente conjunto de ecuaciones:

\begin{equation}\label{Eq.MF.1}
\overline{Q}_{k}\left(t\right)=\overline{Q}_{k}\left(0\right)+\lambda_{k}t-\sum_{m=1}^{M}\mu_{k}\overline{T}_{m,k}\left(t\right)\\
\end{equation}
para $k=1,2,\ldots,K$.\\
\begin{equation}\label{Eq.MF.2}
\overline{Q}_{k}\left(t\right)\geq0\textrm{ para
}k=1,2,\ldots,K,\\
\end{equation}

\begin{equation}\label{Eq.MF.3}
\overline{T}_{m,k}\left(0\right)=0,\textrm{ y }\overline{T}_{m,k}\left(\cdot\right)\textrm{ es no decreciente},\\
\end{equation}
para $k=1,2,\ldots,K$ y $m=1,2,\ldots,M$,\\
\begin{equation}\label{Eq.MF.4}
\sum_{k=1}^{K}\overline{T}_{m,k}^{0}\left(t\right)+\overline{T}_{m,k}\left(t\right)=t\textrm{
para }m=1,2,\ldots,M.\\
\end{equation}

De acuerdo a Dai \cite{Dai}, se tiene que el conjunto de posibles
l\'imites
$\left(\overline{Q}\left(\cdot\right),\overline{T}\left(\cdot\right),\overline{T}^{0}\left(\cdot\right)\right)$,
en el sentido de que deben de satisfacer las ecuaciones
(\ref{Eq.MF.1})-(\ref{Eq.MF.4}), se le llama {\em Modelo de
Flujo}.


\begin{Def}[Definici\'on 4.1, , Dai \cite{Dai}]\label{Def.Modelo.Flujo}
Sea una disciplina de servicio espec\'ifica. Cualquier l\'imite
$\left(\overline{Q}\left(\cdot\right),\overline{T}\left(\cdot\right)\right)$
en (\ref{Eq.Punto.Limite}) es un {\em flujo l\'imite} de la
disciplina. Cualquier soluci\'on (\ref{Eq.MF.1})-(\ref{Eq.MF.4})
es llamado flujo soluci\'on de la disciplina. Se dice que el
modelo de flujo l\'imite, modelo de flujo, de la disciplina de la
cola es estable si existe una constante $\delta>0$ que depende de
$\mu,\lambda$ y $P$ solamente, tal que cualquier flujo l\'imite
con
$|\overline{Q}\left(0\right)|+|\overline{U}|+|\overline{V}|=1$, se
tiene que $\overline{Q}\left(\cdot+\delta\right)\equiv0$.
\end{Def}

Al conjunto de ecuaciones dadas en \ref{Eq.MF.1}-\ref{Eq.MF.4} se
le llama {\em Modelo de flujo} y al conjunto de todas las
soluciones del modelo de flujo
$\left(\overline{Q}\left(\cdot\right),\overline{T}
\left(\cdot\right)\right)$ se le denotar\'a por $\mathcal{Q}$.

Si se hace $|x|\rightarrow\infty$ sin restringir ninguna de las
componentes, tambi\'en se obtienen un modelo de flujo, pero en
este caso el residual de los procesos de arribo y servicio
introducen un retraso:
\begin{Teo}[Teorema 4.2, Dai\cite{Dai}]\label{Tma.4.2.Dai}
Sea una disciplina fija para la cola, suponga que se cumplen las
condiciones (A1))-(A3)). Si el modelo de flujo l\'imite de la
disciplina de la cola es estable, entonces la cadena de Markov $X$
que describe la din\'amica de la red bajo la disciplina es Harris
recurrente positiva.
\end{Teo}

Ahora se procede a escalar el espacio y el tiempo para reducir la
aparente fluctuaci\'on del modelo. Consid\'erese el proceso
\begin{equation}\label{Eq.3.7}
\overline{Q}^{x}\left(t\right)=\frac{1}{|x|}Q^{x}\left(|x|t\right)
\end{equation}
A este proceso se le conoce como el fluido escalado, y cualquier
l\'imite $\overline{Q}^{x}\left(t\right)$ es llamado flujo
l\'imite del proceso de longitud de la cola. Haciendo
$|q|\rightarrow\infty$ mientras se mantiene el resto de las
componentes fijas, cualquier punto l\'imite del proceso de
longitud de la cola normalizado $\overline{Q}^{x}$ es soluci\'on
del siguiente modelo de flujo.


\begin{Def}[Definici\'on 3.3, Dai y Meyn \cite{DaiSean}]
El modelo de flujo es estable si existe un tiempo fijo $t_{0}$ tal
que $\overline{Q}\left(t\right)=0$, con $t\geq t_{0}$, para
cualquier $\overline{Q}\left(\cdot\right)\in\mathcal{Q}$ que
cumple con $|\overline{Q}\left(0\right)|=1$.
\end{Def}

El siguiente resultado se encuentra en Chen \cite{Chen}.
\begin{Lemma}[Lema 3.1, Dai y Meyn \cite{DaiSean}]
Si el modelo de flujo definido por \ref{Eq.MF.1}-\ref{Eq.MF.4} es
estable, entonces el modelo de flujo retrasado es tambi\'en
estable, es decir, existe $t_{0}>0$ tal que
$\overline{Q}\left(t\right)=0$ para cualquier $t\geq t_{0}$, para
cualquier soluci\'on del modelo de flujo retrasado cuya
condici\'on inicial $\overline{x}$ satisface que
$|\overline{x}|=|\overline{Q}\left(0\right)|+|\overline{A}\left(0\right)|+|\overline{B}\left(0\right)|\leq1$.
\end{Lemma}


Ahora ya estamos en condiciones de enunciar los resultados principales:


\begin{Teo}[Teorema 2.1, Down \cite{Down}]\label{Tma2.1.Down}
Suponga que el modelo de flujo es estable, y que se cumplen los supuestos (A1) y (A2), entonces
\begin{itemize}
\item[i)] Para alguna constante $\kappa_{p}$, y para cada
condici\'on inicial $x\in X$
\begin{equation}\label{Estability.Eq1}
limsup_{t\rightarrow\infty}\frac{1}{t}\int_{0}^{t}\esp_{x}\left[|Q\left(s\right)|^{p}\right]ds\leq\kappa_{p},
\end{equation}
donde $p$ es el entero dado en (A2).
\end{itemize}
Si adem\'as se cumple la condici\'on (A3), entonces para cada
condici\'on inicial:
\begin{itemize}
\item[ii)] Los momentos transitorios convergen a su estado
estacionario:
 \begin{equation}\label{Estability.Eq2}
lim_{t\rightarrow\infty}\esp_{x}\left[Q_{k}\left(t\right)^{r}\right]=\esp_{\pi}\left[Q_{k}\left(0\right)^{r}\right]\leq\kappa_{r},
\end{equation}
para $r=1,2,\ldots,p$ y $k=1,2,\ldots,K$. Donde $\pi$ es la
probabilidad invariante para $\mathbf{X}$.

\item[iii)]  El primer momento converge con raz\'on $t^{p-1}$:
\begin{equation}\label{Estability.Eq3}
lim_{t\rightarrow\infty}t^{p-1}|\esp_{x}\left[Q_{k}\left(t\right)\right]-\esp_{\pi}\left[Q_{k}\left(0\right)\right]=0.
\end{equation}

\item[iv)] La {\em Ley Fuerte de los grandes n\'umeros} se cumple:
\begin{equation}\label{Estability.Eq4}
lim_{t\rightarrow\infty}\frac{1}{t}\int_{0}^{t}Q_{k}^{r}\left(s\right)ds=\esp_{\pi}\left[Q_{k}\left(0\right)^{r}\right],\textrm{
}\prob_{x}\textrm{-c.s.}
\end{equation}
para $r=1,2,\ldots,p$ y $k=1,2,\ldots,K$.
\end{itemize}
\end{Teo}

La contribuci\'on de Down a la teor\'ia de los Sistemas de Visitas
C\'iclicas, es la relaci\'on que hay entre la estabilidad del
sistema con el comportamiento de las medidas de desempe\~no, es
decir, la condici\'on suficiente para poder garantizar la
convergencia del proceso de la longitud de la cola as\'i como de
por los menos los dos primeros momentos adem\'as de una versi\'on
de la Ley Fuerte de los Grandes N\'umeros para los sistemas de
visitas.


\begin{Teo}[Teorema 2.3, Down \cite{Down}]\label{Tma2.3.Down}
Considere el siguiente valor:
\begin{equation}\label{Eq.Rho.1serv}
\rho=\sum_{k=1}^{K}\rho_{k}+max_{1\leq j\leq K}\left(\frac{\lambda_{j}}{\sum_{s=1}^{S}p_{js}\overline{N}_{s}}\right)\delta^{*}
\end{equation}
\begin{itemize}
\item[i)] Si $\rho<1$ entonces la red es estable, es decir, se cumple el teorema \ref{Tma2.1.Down}.

\item[ii)] Si $\rho<1$ entonces la red es inestable, es decir, se cumple el teorema \ref{Tma2.2.Down}
\end{itemize}
\end{Teo}

\begin{Teo}
Sea $\left(X_{n},\mathcal{F}_{n},n=0,1,\ldots,\right\}$ Proceso de
Markov con espacio de estados $\left(S_{0},\chi_{0}\right)$
generado por una distribuici\'on inicial $P_{o}$ y probabilidad de
transici\'on $p_{mn}$, para $m,n=0,1,\ldots,$ $m<n$, que por
notaci\'on se escribir\'a como $p\left(m,n,x,B\right)\rightarrow
p_{mn}\left(x,B\right)$. Sea $S$ tiempo de paro relativo a la
$\sigma$-\'algebra $\mathcal{F}_{n}$. Sea $T$ funci\'on medible,
$T:\Omega\rightarrow\left\{0,1,\ldots,\right\}$. Sup\'ongase que
$T\geq S$, entonces $T$ es tiempo de paro. Si $B\in\chi_{0}$,
entonces
\begin{equation}\label{Prop.Fuerte.Markov}
P\left\{X\left(T\right)\in
B,T<\infty|\mathcal{F}\left(S\right)\right\} =
p\left(S,T,X\left(s\right),B\right)
\end{equation}
en $\left\{T<\infty\right\}$.
\end{Teo}


Sea $K$ conjunto numerable y sea $d:K\rightarrow\nat$ funci\'on.
Para $v\in K$, $M_{v}$ es un conjunto abierto de
$\rea^{d\left(v\right)}$. Entonces \[E=\cup_{v\in
K}M_{v}=\left\{\left(v,\zeta\right):v\in K,\zeta\in
M_{v}\right\}.\]

Sea $\mathcal{E}$ la clase de conjuntos medibles en $E$:
\[\mathcal{E}=\left\{\cup_{v\in K}A_{v}:A_{v}\in \mathcal{M}_{v}\right\}.\]

donde $\mathcal{M}$ son los conjuntos de Borel de $M_{v}$.
Entonces $\left(E,\mathcal{E}\right)$ es un espacio de Borel. El
estado del proceso se denotar\'a por
$\mathbf{x}_{t}=\left(v_{t},\zeta_{t}\right)$. La distribuci\'on
de $\left(\mathbf{x}_{t}\right)$ est\'a determinada por por los
siguientes objetos:

\begin{itemize}
\item[i)] Los campos vectoriales $\left(\mathcal{H}_{v},v\in
K\right)$. \item[ii)] Una funci\'on medible $\lambda:E\rightarrow
\rea_{+}$. \item[iii)] Una medida de transici\'on
$Q:\mathcal{E}\times\left(E\cup\Gamma^{*}\right)\rightarrow\left[0,1\right]$
donde
\begin{equation}
\Gamma^{*}=\cup_{v\in K}\partial^{*}M_{v}.
\end{equation}
y
\begin{equation}
\partial^{*}M_{v}=\left\{z\in\partial M_{v}:\mathbf{\mathbf{\phi}_{v}\left(t,\zeta\right)=\mathbf{z}}\textrm{ para alguna }\left(t,\zeta\right)\in\rea_{+}\times M_{v}\right\}.
\end{equation}
$\partial M_{v}$ denota  la frontera de $M_{v}$.
\end{itemize}

El campo vectorial $\left(\mathcal{H}_{v},v\in K\right)$ se supone
tal que para cada $\mathbf{z}\in M_{v}$ existe una \'unica curva
integral $\mathbf{\phi}_{v}\left(t,\zeta\right)$ que satisface la
ecuaci\'on

\begin{equation}
\frac{d}{dt}f\left(\zeta_{t}\right)=\mathcal{H}f\left(\zeta_{t}\right),
\end{equation}
con $\zeta_{0}=\mathbf{z}$, para cualquier funci\'on suave
$f:\rea^{d}\rightarrow\rea$ y $\mathcal{H}$ denota el operador
diferencial de primer orden, con $\mathcal{H}=\mathcal{H}_{v}$ y
$\zeta_{t}=\mathbf{\phi}\left(t,\mathbf{z}\right)$. Adem\'as se
supone que $\mathcal{H}_{v}$ es conservativo, es decir, las curvas
integrales est\'an definidas para todo $t>0$.

Para $\mathbf{x}=\left(v,\zeta\right)\in E$ se denota
\[t^{*}\mathbf{x}=inf\left\{t>0:\mathbf{\phi}_{v}\left(t,\zeta\right)\in\partial^{*}M_{v}\right\}\]

En lo que respecta a la funci\'on $\lambda$, se supondr\'a que
para cada $\left(v,\zeta\right)\in E$ existe un $\epsilon>0$ tal
que la funci\'on
$s\rightarrow\lambda\left(v,\phi_{v}\left(s,\zeta\right)\right)\in
E$ es integrable para $s\in\left[0,\epsilon\right)$. La medida de
transici\'on $Q\left(A;\mathbf{x}\right)$ es una funci\'on medible
de $\mathbf{x}$ para cada $A\in\mathcal{E}$, definida para
$\mathbf{x}\in E\cup\Gamma^{*}$ y es una medida de probabilidad en
$\left(E,\mathcal{E}\right)$ para cada $\mathbf{x}\in E$.

El movimiento del proceso $\left(\mathbf{x}_{t}\right)$ comenzando
en $\mathbf{x}=\left(n,\mathbf{z}\right)\in E$ se puede construir
de la siguiente manera, def\'inase la funci\'on $F$ por

\begin{equation}
F\left(t\right)=\left\{\begin{array}{ll}\\
exp\left(-\int_{0}^{t}\lambda\left(n,\phi_{n}\left(s,\mathbf{z}\right)\right)ds\right), & t<t^{*}\left(\mathbf{x}\right),\\
0, & t\geq t^{*}\left(\mathbf{x}\right)
\end{array}\right.
\end{equation}

Sea $T_{1}$ una variable aleatoria tal que
$\prob\left[T_{1}>t\right]=F\left(t\right)$, ahora sea la variable
aleatoria $\left(N,Z\right)$ con distribuici\'on
$Q\left(\cdot;\phi_{n}\left(T_{1},\mathbf{z}\right)\right)$. La
trayectoria de $\left(\mathbf{x}_{t}\right)$ para $t\leq T_{1}$
es\footnote{Revisar p\'agina 362, y 364 de Davis \cite{Davis}.}
\begin{eqnarray*}
\mathbf{x}_{t}=\left(v_{t},\zeta_{t}\right)=\left\{\begin{array}{ll}
\left(n,\phi_{n}\left(t,\mathbf{z}\right)\right), & t<T_{1},\\
\left(N,\mathbf{Z}\right), & t=t_{1}.
\end{array}\right.
\end{eqnarray*}

Comenzando en $\mathbf{x}_{T_{1}}$ se selecciona el siguiente
tiempo de intersalto $T_{2}-T_{1}$ lugar del post-salto
$\mathbf{x}_{T_{2}}$ de manera similar y as\'i sucesivamente. Este
procedimiento nos da una trayectoria determinista por partes
$\mathbf{x}_{t}$ con tiempos de salto $T_{1},T_{2},\ldots$. Bajo
las condiciones enunciadas para $\lambda,T_{1}>0$  y
$T_{1}-T_{2}>0$ para cada $i$, con probabilidad 1. Se supone que
se cumple la siguiente condici\'on.

\begin{Sup}[Supuesto 3.1, Davis \cite{Davis}]\label{Sup3.1.Davis}
Sea $N_{t}:=\sum_{t}\indora_{\left(t\geq t\right)}$ el n\'umero de
saltos en $\left[0,t\right]$. Entonces
\begin{equation}
\esp\left[N_{t}\right]<\infty\textrm{ para toda }t.
\end{equation}
\end{Sup}

es un proceso de Markov, m\'as a\'un, es un Proceso Fuerte de
Markov, es decir, la Propiedad Fuerte de Markov se cumple para
cualquier tiempo de paro.


Sea $E$ es un espacio m\'etrico separable y la m\'etrica $d$ es
compatible con la topolog\'ia.


\begin{Def}
Un espacio topol\'ogico $E$ es llamado de {\em Rad\'on} si es
homeomorfo a un subconjunto universalmente medible de un espacio
m\'etrico compacto.
\end{Def}

Equivalentemente, la definici\'on de un espacio de Rad\'on puede
encontrarse en los siguientes t\'erminos:


\begin{Def}
$E$ es un espacio de Rad\'on si cada medida finita en
$\left(E,\mathcal{B}\left(E\right)\right)$ es regular interior o
cerrada, {\em tight}.
\end{Def}

\begin{Def}
Una medida finita, $\lambda$ en la $\sigma$-\'algebra de Borel de
un espacio metrizable $E$ se dice cerrada si
\begin{equation}\label{Eq.A2.3}
\lambda\left(E\right)=sup\left\{\lambda\left(K\right):K\textrm{ es
compacto en }E\right\}.
\end{equation}
\end{Def}

El siguiente teorema nos permite tener una mejor caracterizaci\'on
de los espacios de Rad\'on:
\begin{Teo}\label{Tma.A2.2}
Sea $E$ espacio separable metrizable. Entonces $E$ es Radoniano si
y s\'olo s\'i cada medida finita en
$\left(E,\mathcal{B}\left(E\right)\right)$ es cerrada.
\end{Teo}

Sea $E$ espacio de estados, tal que $E$ es un espacio de Rad\'on,
$\mathcal{B}\left(E\right)$ $\sigma$-\'algebra de Borel en $E$,
que se denotar\'a por $\mathcal{E}$.

Sea $\left(X,\mathcal{G},\prob\right)$ espacio de probabilidad,
$I\subset\rea$ conjunto de \'indices. Sea $\mathcal{F}_{\leq t}$
la $\sigma$-\'algebra natural definida como
$\sigma\left\{f\left(X_{r}\right):r\in I, r\leq
t,f\in\mathcal{E}\right\}$. Se considerar\'a una
$\sigma$-\'algebra m\'as general, $ \left(\mathcal{G}_{t}\right)$
tal que $\left(X_{t}\right)$ sea $\mathcal{E}$-adaptado.

\begin{Def}
Una familia $\left(P_{s,t}\right)$ de kernels de Markov en
$\left(E,\mathcal{E}\right)$ indexada por pares $s,t\in I$, con
$s\leq t$ es una funci\'on de transici\'on en $\ER$, si  para todo
$r\leq s< t$ en $I$ y todo $x\in E$,
$B\in\mathcal{E}$\footnote{Ecuaci\'on de Chapman-Kolmogorov}
\begin{equation}\label{Eq.Kernels}
P_{r,t}\left(x,B\right)=\int_{E}P_{r,s}\left(x,dy\right)P_{s,t}\left(y,B\right).
\end{equation}
\end{Def}

Se dice que la funci\'on de transici\'on $\KM$ en $\ER$ es la
funci\'on de transici\'on para un proceso $\PE$  con valores en
$E$ y que satisface la propiedad de
Markov\footnote{\begin{equation}\label{Eq.1.4.S}
\prob\left\{H|\mathcal{G}_{t}\right\}=\prob\left\{H|X_{t}\right\}\textrm{
}H\in p\mathcal{F}_{\geq t}.
\end{equation}} (\ref{Eq.1.4.S}) relativa a $\left(\mathcal{G}_{t}\right)$ si

\begin{equation}\label{Eq.1.6.S}
\prob\left\{f\left(X_{t}\right)|\mathcal{G}_{s}\right\}=P_{s,t}f\left(X_{t}\right)\textrm{
}s\leq t\in I,\textrm{ }f\in b\mathcal{E}.
\end{equation}

\begin{Def}
Una familia $\left(P_{t}\right)_{t\geq0}$ de kernels de Markov en
$\ER$ es llamada {\em Semigrupo de Transici\'on de Markov} o {\em
Semigrupo de Transici\'on} si
\[P_{t+s}f\left(x\right)=P_{t}\left(P_{s}f\right)\left(x\right),\textrm{ }t,s\geq0,\textrm{ }x\in E\textrm{ }f\in b\mathcal{E}.\]
\end{Def}
\begin{Note}
Si la funci\'on de transici\'on $\KM$ es llamada homog\'enea si
$P_{s,t}=P_{t-s}$.
\end{Note}

Un proceso de Markov que satisface la ecuaci\'on (\ref{Eq.1.6.S})
con funci\'on de transici\'on homog\'enea $\left(P_{t}\right)$
tiene la propiedad caracter\'istica
\begin{equation}\label{Eq.1.8.S}
\prob\left\{f\left(X_{t+s}\right)|\mathcal{G}_{t}\right\}=P_{s}f\left(X_{t}\right)\textrm{
}t,s\geq0,\textrm{ }f\in b\mathcal{E}.
\end{equation}
La ecuaci\'on anterior es la {\em Propiedad Simple de Markov} de
$X$ relativa a $\left(P_{t}\right)$.

En este sentido el proceso $\PE$ cumple con la propiedad de Markov
(\ref{Eq.1.8.S}) relativa a
$\left(\Omega,\mathcal{G},\mathcal{G}_{t},\prob\right)$ con
semigrupo de transici\'on $\left(P_{t}\right)$.

\begin{Def}
Un proceso estoc\'astico $\PE$ definido en
$\left(\Omega,\mathcal{G},\prob\right)$ con valores en el espacio
topol\'ogico $E$ es continuo por la derecha si cada trayectoria
muestral $t\rightarrow X_{t}\left(w\right)$ es un mapeo continuo
por la derecha de $I$ en $E$.
\end{Def}

\begin{Def}[HD1]\label{Eq.2.1.S}
Un semigrupo de Markov $\left(P_{t}\right)$ en un espacio de
Rad\'on $E$ se dice que satisface la condici\'on {\em HD1} si,
dada una medida de probabilidad $\mu$ en $E$, existe una
$\sigma$-\'algebra $\mathcal{E^{*}}$ con
$\mathcal{E}\subset\mathcal{E}^{*}$ y
$P_{t}\left(b\mathcal{E}^{*}\right)\subset b\mathcal{E}^{*}$, y un
$\mathcal{E}^{*}$-proceso $E$-valuado continuo por la derecha
$\PE$ en alg\'un espacio de probabilidad filtrado
$\left(\Omega,\mathcal{G},\mathcal{G}_{t},\prob\right)$ tal que
$X=\left(\Omega,\mathcal{G},\mathcal{G}_{t},\prob\right)$ es de
Markov (Homog\'eneo) con semigrupo de transici\'on $(P_{t})$ y
distribuci\'on inicial $\mu$.
\end{Def}

Consid\'erese la colecci\'on de variables aleatorias $X_{t}$
definidas en alg\'un espacio de probabilidad, y una colecci\'on de
medidas $\mathbf{P}^{x}$ tales que
$\mathbf{P}^{x}\left\{X_{0}=x\right\}$, y bajo cualquier
$\mathbf{P}^{x}$, $X_{t}$ es de Markov con semigrupo
$\left(P_{t}\right)$. $\mathbf{P}^{x}$ puede considerarse como la
distribuci\'on condicional de $\mathbf{P}$ dado $X_{0}=x$.

\begin{Def}\label{Def.2.2.S}
Sea $E$ espacio de Rad\'on, $\SG$ semigrupo de Markov en $\ER$. La
colecci\'on
$\mathbf{X}=\left(\Omega,\mathcal{G},\mathcal{G}_{t},X_{t},\theta_{t},\CM\right)$
es un proceso $\mathcal{E}$-Markov continuo por la derecha simple,
con espacio de estados $E$ y semigrupo de transici\'on $\SG$ en
caso de que $\mathbf{X}$ satisfaga las siguientes condiciones:
\begin{itemize}
\item[i)] $\left(\Omega,\mathcal{G},\mathcal{G}_{t}\right)$ es un
espacio de medida filtrado, y $X_{t}$ es un proceso $E$-valuado
continuo por la derecha $\mathcal{E}^{*}$-adaptado a
$\left(\mathcal{G}_{t}\right)$;

\item[ii)] $\left(\theta_{t}\right)_{t\geq0}$ es una colecci\'on
de operadores {\em shift} para $X$, es decir, mapea $\Omega$ en
s\'i mismo satisfaciendo para $t,s\geq0$,

\begin{equation}\label{Eq.Shift}
\theta_{t}\circ\theta_{s}=\theta_{t+s}\textrm{ y
}X_{t}\circ\theta_{t}=X_{t+s};
\end{equation}

\item[iii)] Para cualquier $x\in E$,$\CM\left\{X_{0}=x\right\}=1$,
y el proceso $\PE$ tiene la propiedad de Markov (\ref{Eq.1.8.S})
con semigrupo de transici\'on $\SG$ relativo a
$\left(\Omega,\mathcal{G},\mathcal{G}_{t},\CM\right)$.
\end{itemize}
\end{Def}

\begin{Def}[HD2]\label{Eq.2.2.S}
Para cualquier $\alpha>0$ y cualquier $f\in S^{\alpha}$, el
proceso $t\rightarrow f\left(X_{t}\right)$ es continuo por la
derecha casi seguramente.
\end{Def}

\begin{Def}\label{Def.PD}
Un sistema
$\mathbf{X}=\left(\Omega,\mathcal{G},\mathcal{G}_{t},X_{t},\theta_{t},\CM\right)$
es un proceso derecho en el espacio de Rad\'on $E$ con semigrupo
de transici\'on $\SG$ provisto de:
\begin{itemize}
\item[i)] $\mathbf{X}$ es una realizaci\'on  continua por la
derecha, \ref{Def.2.2.S}, de $\SG$.

\item[ii)] $\mathbf{X}$ satisface la condicion HD2,
\ref{Eq.2.2.S}, relativa a $\mathcal{G}_{t}$.

\item[iii)] $\mathcal{G}_{t}$ es aumentado y continuo por la
derecha.
\end{itemize}
\end{Def}

\begin{Lema}[Lema 4.2, Dai\cite{Dai}]\label{Lema4.2}
Sea $\left\{x_{n}\right\}\subset \mathbf{X}$ con
$|x_{n}|\rightarrow\infty$, conforme $n\rightarrow\infty$. Suponga
que
\[lim_{n\rightarrow\infty}\frac{1}{|x_{n}|}U\left(0\right)=\overline{U}\]
y
\[lim_{n\rightarrow\infty}\frac{1}{|x_{n}|}V\left(0\right)=\overline{V}.\]

Entonces, conforme $n\rightarrow\infty$, casi seguramente

\begin{equation}\label{E1.4.2}
\frac{1}{|x_{n}|}\Phi^{k}\left(\left[|x_{n}|t\right]\right)\rightarrow
P_{k}^{'}t\textrm{, u.o.c.,}
\end{equation}

\begin{equation}\label{E1.4.3}
\frac{1}{|x_{n}|}E^{x_{n}}_{k}\left(|x_{n}|t\right)\rightarrow
\alpha_{k}\left(t-\overline{U}_{k}\right)^{+}\textrm{, u.o.c.,}
\end{equation}

\begin{equation}\label{E1.4.4}
\frac{1}{|x_{n}|}S^{x_{n}}_{k}\left(|x_{n}|t\right)\rightarrow
\mu_{k}\left(t-\overline{V}_{k}\right)^{+}\textrm{, u.o.c.,}
\end{equation}

donde $\left[t\right]$ es la parte entera de $t$ y
$\mu_{k}=1/m_{k}=1/\esp\left[\eta_{k}\left(1\right)\right]$.
\end{Lema}

\begin{Lema}[Lema 4.3, Dai\cite{Dai}]\label{Lema.4.3}
Sea $\left\{x_{n}\right\}\subset \mathbf{X}$ con
$|x_{n}|\rightarrow\infty$, conforme $n\rightarrow\infty$. Suponga
que
\[lim_{n\rightarrow\infty}\frac{1}{|x_{n}|}U\left(0\right)=\overline{U}_{k}\]
y
\[lim_{n\rightarrow\infty}\frac{1}{|x_{n}|}V\left(0\right)=\overline{V}_{k}.\]
\begin{itemize}
\item[a)] Conforme $n\rightarrow\infty$ casi seguramente,
\[lim_{n\rightarrow\infty}\frac{1}{|x_{n}|}U^{x_{n}}_{k}\left(|x_{n}|t\right)=\left(\overline{U}_{k}-t\right)^{+}\textrm{, u.o.c.}\]
y
\[lim_{n\rightarrow\infty}\frac{1}{|x_{n}|}V^{x_{n}}_{k}\left(|x_{n}|t\right)=\left(\overline{V}_{k}-t\right)^{+}.\]

\item[b)] Para cada $t\geq0$ fijo,
\[\left\{\frac{1}{|x_{n}|}U^{x_{n}}_{k}\left(|x_{n}|t\right),|x_{n}|\geq1\right\}\]
y
\[\left\{\frac{1}{|x_{n}|}V^{x_{n}}_{k}\left(|x_{n}|t\right),|x_{n}|\geq1\right\}\]
\end{itemize}
son uniformemente convergentes.
\end{Lema}

$S_{l}^{x}\left(t\right)$ es el n\'umero total de servicios
completados de la clase $l$, si la clase $l$ est\'a dando $t$
unidades de tiempo de servicio. Sea $T_{l}^{x}\left(x\right)$ el
monto acumulado del tiempo de servicio que el servidor
$s\left(l\right)$ gasta en los usuarios de la clase $l$ al tiempo
$t$. Entonces $S_{l}^{x}\left(T_{l}^{x}\left(t\right)\right)$ es
el n\'umero total de servicios completados para la clase $l$ al
tiempo $t$. Una fracci\'on de estos usuarios,
$\Phi_{l}^{x}\left(S_{l}^{x}\left(T_{l}^{x}\left(t\right)\right)\right)$,
se convierte en usuarios de la clase $k$.\\

Entonces, dado lo anterior, se tiene la siguiente representaci\'on
para el proceso de la longitud de la cola:\\

\begin{equation}
Q_{k}^{x}\left(t\right)=_{k}^{x}\left(0\right)+E_{k}^{x}\left(t\right)+\sum_{l=1}^{K}\Phi_{k}^{l}\left(S_{l}^{x}\left(T_{l}^{x}\left(t\right)\right)\right)-S_{k}^{x}\left(T_{k}^{x}\left(t\right)\right)
\end{equation}
para $k=1,\ldots,K$. Para $i=1,\ldots,d$, sea
\[I_{i}^{x}\left(t\right)=t-\sum_{j\in C_{i}}T_{k}^{x}\left(t\right).\]

Entonces $I_{i}^{x}\left(t\right)$ es el monto acumulado del
tiempo que el servidor $i$ ha estado desocupado al tiempo $t$. Se
est\'a asumiendo que las disciplinas satisfacen la ley de
conservaci\'on del trabajo, es decir, el servidor $i$ est\'a en
pausa solamente cuando no hay usuarios en la estaci\'on $i$.
Entonces, se tiene que

\begin{equation}
\int_{0}^{\infty}\left(\sum_{k\in
C_{i}}Q_{k}^{x}\left(t\right)\right)dI_{i}^{x}\left(t\right)=0,
\end{equation}
para $i=1,\ldots,d$.\\

Hacer
\[T^{x}\left(t\right)=\left(T_{1}^{x}\left(t\right),\ldots,T_{K}^{x}\left(t\right)\right)^{'},\]
\[I^{x}\left(t\right)=\left(I_{1}^{x}\left(t\right),\ldots,I_{K}^{x}\left(t\right)\right)^{'}\]
y
\[S^{x}\left(T^{x}\left(t\right)\right)=\left(S_{1}^{x}\left(T_{1}^{x}\left(t\right)\right),\ldots,S_{K}^{x}\left(T_{K}^{x}\left(t\right)\right)\right)^{'}.\]

Para una disciplina que cumple con la ley de conservaci\'on del
trabajo, en forma vectorial, se tiene el siguiente conjunto de
ecuaciones

\begin{equation}\label{Eq.MF.1.3}
Q^{x}\left(t\right)=Q^{x}\left(0\right)+E^{x}\left(t\right)+\sum_{l=1}^{K}\Phi^{l}\left(S_{l}^{x}\left(T_{l}^{x}\left(t\right)\right)\right)-S^{x}\left(T^{x}\left(t\right)\right),\\
\end{equation}

\begin{equation}\label{Eq.MF.2.3}
Q^{x}\left(t\right)\geq0,\\
\end{equation}

\begin{equation}\label{Eq.MF.3.3}
T^{x}\left(0\right)=0,\textrm{ y }\overline{T}^{x}\left(t\right)\textrm{ es no decreciente},\\
\end{equation}

\begin{equation}\label{Eq.MF.4.3}
I^{x}\left(t\right)=et-CT^{x}\left(t\right)\textrm{ es no
decreciente}\\
\end{equation}

\begin{equation}\label{Eq.MF.5.3}
\int_{0}^{\infty}\left(CQ^{x}\left(t\right)\right)dI_{i}^{x}\left(t\right)=0,\\
\end{equation}

\begin{equation}\label{Eq.MF.6.3}
\textrm{Condiciones adicionales en
}\left(\overline{Q}^{x}\left(\cdot\right),\overline{T}^{x}\left(\cdot\right)\right)\textrm{
espec\'ificas de la disciplina de la cola,}
\end{equation}

donde $e$ es un vector de unos de dimensi\'on $d$, $C$ es la
matriz definida por
\[C_{ik}=\left\{\begin{array}{cc}
1,& S\left(k\right)=i,\\
0,& \textrm{ en otro caso}.\\
\end{array}\right.
\]
Es necesario enunciar el siguiente Teorema que se utilizar\'a para
el Teorema \ref{Tma.4.2.Dai}:
\begin{Teo}[Teorema 4.1, Dai \cite{Dai}]
Considere una disciplina que cumpla la ley de conservaci\'on del
trabajo, para casi todas las trayectorias muestrales $\omega$ y
cualquier sucesi\'on de estados iniciales
$\left\{x_{n}\right\}\subset \mathbf{X}$, con
$|x_{n}|\rightarrow\infty$, existe una subsucesi\'on
$\left\{x_{n_{j}}\right\}$ con $|x_{n_{j}}|\rightarrow\infty$ tal
que
\begin{equation}\label{Eq.4.15}
\frac{1}{|x_{n_{j}}|}\left(Q^{x_{n_{j}}}\left(0\right),U^{x_{n_{j}}}\left(0\right),V^{x_{n_{j}}}\left(0\right)\right)\rightarrow\left(\overline{Q}\left(0\right),\overline{U},\overline{V}\right),
\end{equation}

\begin{equation}\label{Eq.4.16}
\frac{1}{|x_{n_{j}}|}\left(Q^{x_{n_{j}}}\left(|x_{n_{j}}|t\right),T^{x_{n_{j}}}\left(|x_{n_{j}}|t\right)\right)\rightarrow\left(\overline{Q}\left(t\right),\overline{T}\left(t\right)\right)\textrm{
u.o.c.}
\end{equation}

Adem\'as,
$\left(\overline{Q}\left(t\right),\overline{T}\left(t\right)\right)$
satisface las siguientes ecuaciones:
\begin{equation}\label{Eq.MF.1.3a}
\overline{Q}\left(t\right)=Q\left(0\right)+\left(\alpha
t-\overline{U}\right)^{+}-\left(I-P\right)^{'}M^{-1}\left(\overline{T}\left(t\right)-\overline{V}\right)^{+},
\end{equation}

\begin{equation}\label{Eq.MF.2.3a}
\overline{Q}\left(t\right)\geq0,\\
\end{equation}

\begin{equation}\label{Eq.MF.3.3a}
\overline{T}\left(t\right)\textrm{ es no decreciente y comienza en cero},\\
\end{equation}

\begin{equation}\label{Eq.MF.4.3a}
\overline{I}\left(t\right)=et-C\overline{T}\left(t\right)\textrm{
es no decreciente,}\\
\end{equation}

\begin{equation}\label{Eq.MF.5.3a}
\int_{0}^{\infty}\left(C\overline{Q}\left(t\right)\right)d\overline{I}\left(t\right)=0,\\
\end{equation}

\begin{equation}\label{Eq.MF.6.3a}
\textrm{Condiciones adicionales en
}\left(\overline{Q}\left(\cdot\right),\overline{T}\left(\cdot\right)\right)\textrm{
especficas de la disciplina de la cola,}
\end{equation}
\end{Teo}


Propiedades importantes para el modelo de flujo retrasado:

\begin{Prop}
 Sea $\left(\overline{Q},\overline{T},\overline{T}^{0}\right)$ un flujo l\'imite de \ref{Eq.4.4} y suponga que cuando $x\rightarrow\infty$ a lo largo de
una subsucesi\'on
\[\left(\frac{1}{|x|}Q_{k}^{x}\left(0\right),\frac{1}{|x|}A_{k}^{x}\left(0\right),\frac{1}{|x|}B_{k}^{x}\left(0\right),\frac{1}{|x|}B_{k}^{x,0}\left(0\right)\right)\rightarrow\left(\overline{Q}_{k}\left(0\right),0,0,0\right)\]
para $k=1,\ldots,K$. EL flujo l\'imite tiene las siguientes
propiedades, donde las propiedades de la derivada se cumplen donde
la derivada exista:
\begin{itemize}
 \item[i)] Los vectores de tiempo ocupado $\overline{T}\left(t\right)$ y $\overline{T}^{0}\left(t\right)$ son crecientes y continuas con
$\overline{T}\left(0\right)=\overline{T}^{0}\left(0\right)=0$.
\item[ii)] Para todo $t\geq0$
\[\sum_{k=1}^{K}\left[\overline{T}_{k}\left(t\right)+\overline{T}_{k}^{0}\left(t\right)\right]=t\]
\item[iii)] Para todo $1\leq k\leq K$
\[\overline{Q}_{k}\left(t\right)=\overline{Q}_{k}\left(0\right)+\alpha_{k}t-\mu_{k}\overline{T}_{k}\left(t\right)\]
\item[iv)]  Para todo $1\leq k\leq K$
\[\dot{{\overline{T}}}_{k}\left(t\right)=\beta_{k}\] para $\overline{Q}_{k}\left(t\right)=0$.
\item[v)] Para todo $k,j$
\[\mu_{k}^{0}\overline{T}_{k}^{0}\left(t\right)=\mu_{j}^{0}\overline{T}_{j}^{0}\left(t\right)\]
\item[vi)]  Para todo $1\leq k\leq K$
\[\mu_{k}\dot{{\overline{T}}}_{k}\left(t\right)=l_{k}\mu_{k}^{0}\dot{{\overline{T}}}_{k}^{0}\left(t\right)\] para $\overline{Q}_{k}\left(t\right)>0$.
\end{itemize}
\end{Prop}

\begin{Lema}[Lema 3.1 \cite{Chen}]\label{Lema3.1}
Si el modelo de flujo es estable, definido por las ecuaciones
(3.8)-(3.13), entonces el modelo de flujo retrasado tambi\'en es
estable.
\end{Lema}

\begin{Teo}[Teorema 5.1 \cite{Chen}]\label{Tma.5.1.Chen}
La red de colas es estable si existe una constante $t_{0}$ que
depende de $\left(\alpha,\mu,T,U\right)$ y $V$ que satisfagan las
ecuaciones (5.1)-(5.5), $Z\left(t\right)=0$, para toda $t\geq
t_{0}$.
\end{Teo}



\begin{Lema}[Lema 5.2 \cite{Gut}]\label{Lema.5.2.Gut}
Sea $\left\{\xi\left(k\right):k\in\ent\right\}$ sucesi\'on de
variables aleatorias i.i.d. con valores en
$\left(0,\infty\right)$, y sea $E\left(t\right)$ el proceso de
conteo
\[E\left(t\right)=max\left\{n\geq1:\xi\left(1\right)+\cdots+\xi\left(n-1\right)\leq t\right\}.\]
Si $E\left[\xi\left(1\right)\right]<\infty$, entonces para
cualquier entero $r\geq1$
\begin{equation}
lim_{t\rightarrow\infty}\esp\left[\left(\frac{E\left(t\right)}{t}\right)^{r}\right]=\left(\frac{1}{E\left[\xi_{1}\right]}\right)^{r}
\end{equation}
de aqu\'i, bajo estas condiciones
\begin{itemize}
\item[a)] Para cualquier $t>0$,
$sup_{t\geq\delta}\esp\left[\left(\frac{E\left(t\right)}{t}\right)^{r}\right]$

\item[b)] Las variables aleatorias
$\left\{\left(\frac{E\left(t\right)}{t}\right)^{r}:t\geq1\right\}$
son uniformemente integrables.
\end{itemize}
\end{Lema}

\begin{Teo}[Teorema 5.1: Ley Fuerte para Procesos de Conteo
\cite{Gut}]\label{Tma.5.1.Gut} Sea
$0<\mu<\esp\left(X_{1}\right]\leq\infty$. entonces

\begin{itemize}
\item[a)] $\frac{N\left(t\right)}{t}\rightarrow\frac{1}{\mu}$
a.s., cuando $t\rightarrow\infty$.


\item[b)]$\esp\left[\frac{N\left(t\right)}{t}\right]^{r}\rightarrow\frac{1}{\mu^{r}}$,
cuando $t\rightarrow\infty$ para todo $r>0$..
\end{itemize}
\end{Teo}


\begin{Prop}[Proposici\'on 5.1 \cite{DaiSean}]\label{Prop.5.1}
Suponga que los supuestos (A1) y (A2) se cumplen, adem\'as suponga
que el modelo de flujo es estable. Entonces existe $t_{0}>0$ tal
que
\begin{equation}\label{Eq.Prop.5.1}
lim_{|x|\rightarrow\infty}\frac{1}{|x|^{p+1}}\esp_{x}\left[|X\left(t_{0}|x|\right)|^{p+1}\right]=0.
\end{equation}

\end{Prop}


\begin{Prop}[Proposici\'on 5.3 \cite{DaiSean}]
Sea $X$ proceso de estados para la red de colas, y suponga que se
cumplen los supuestos (A1) y (A2), entonces para alguna constante
positiva $C_{p+1}<\infty$, $\delta>0$ y un conjunto compacto
$C\subset X$.

\begin{equation}\label{Eq.5.4}
\esp_{x}\left[\int_{0}^{\tau_{C}\left(\delta\right)}\left(1+|X\left(t\right)|^{p}\right)dt\right]\leq
C_{p+1}\left(1+|x|^{p+1}\right)
\end{equation}
\end{Prop}

\begin{Prop}[Proposici\'on 5.4 \cite{DaiSean}]
Sea $X$ un proceso de Markov Borel Derecho en $X$, sea
$f:X\leftarrow\rea_{+}$ y defina para alguna $\delta>0$, y un
conjunto cerrado $C\subset X$
\[V\left(x\right):=\esp_{x}\left[\int_{0}^{\tau_{C}\left(\delta\right)}f\left(X\left(t\right)\right)dt\right]\]
para $x\in X$. Si $V$ es finito en todas partes y uniformemente
acotada en $C$, entonces existe $k<\infty$ tal que
\begin{equation}\label{Eq.5.11}
\frac{1}{t}\esp_{x}\left[V\left(x\right)\right]+\frac{1}{t}\int_{0}^{t}\esp_{x}\left[f\left(X\left(s\right)\right)ds\right]\leq\frac{1}{t}V\left(x\right)+k,
\end{equation}
para $x\in X$ y $t>0$.
\end{Prop}


\begin{Teo}[Teorema 5.5 \cite{DaiSean}]
Suponga que se cumplen (A1) y (A2), adem\'as suponga que el modelo
de flujo es estable. Entonces existe una constante $k_{p}<\infty$
tal que
\begin{equation}\label{Eq.5.13}
\frac{1}{t}\int_{0}^{t}\esp_{x}\left[|Q\left(s\right)|^{p}\right]ds\leq
k_{p}\left\{\frac{1}{t}|x|^{p+1}+1\right\}
\end{equation}
para $t\geq0$, $x\in X$. En particular para cada condici\'on
inicial
\begin{equation}\label{Eq.5.14}
Limsup_{t\rightarrow\infty}\frac{1}{t}\int_{0}^{t}\esp_{x}\left[|Q\left(s\right)|^{p}\right]ds\leq
k_{p}
\end{equation}
\end{Teo}

\begin{Teo}[Teorema 6.2 \cite{DaiSean}]\label{Tma.6.2}
Suponga que se cumplen los supuestos (A1)-(A3) y que el modelo de
flujo es estable, entonces se tiene que
\[\parallel P^{t}\left(c,\cdot\right)-\pi\left(\cdot\right)\parallel_{f_{p}}\rightarrow0\]
para $t\rightarrow\infty$ y $x\in X$. En particular para cada
condici\'on inicial
\[lim_{t\rightarrow\infty}\esp_{x}\left[\left|Q_{t}\right|^{p}\right]=\esp_{\pi}\left[\left|Q_{0}\right|^{p}\right]<\infty\]
\end{Teo}


\begin{Teo}[Teorema 6.3 \cite{DaiSean}]\label{Tma.6.3}
Suponga que se cumplen los supuestos (A1)-(A3) y que el modelo de
flujo es estable, entonces con
$f\left(x\right)=f_{1}\left(x\right)$, se tiene que
\[lim_{t\rightarrow\infty}t^{(p-1)\left|P^{t}\left(c,\cdot\right)-\pi\left(\cdot\right)\right|_{f}=0},\]
para $x\in X$. En particular, para cada condici\'on inicial
\[lim_{t\rightarrow\infty}t^{(p-1)}\left|\esp_{x}\left[Q_{t}\right]-\esp_{\pi}\left[Q_{0}\right]\right|=0.\]
\end{Teo}



\begin{Prop}[Proposici\'on 5.1, Dai y Meyn \cite{DaiSean}]\label{Prop.5.1.DaiSean}
Suponga que los supuestos A1) y A2) son ciertos y que el modelo de
flujo es estable. Entonces existe $t_{0}>0$ tal que
\begin{equation}
lim_{|x|\rightarrow\infty}\frac{1}{|x|^{p+1}}\esp_{x}\left[|X\left(t_{0}|x|\right)|^{p+1}\right]=0
\end{equation}
\end{Prop}

\begin{Lemma}[Lema 5.2, Dai y Meyn, \cite{DaiSean}]\label{Lema.5.2.DaiSean}
 Sea $\left\{\zeta\left(k\right):k\in \mathbb{z}\right\}$ una sucesi\'on independiente e id\'enticamente distribuida que toma valores en $\left(0,\infty\right)$,
y sea
$E\left(t\right)=max\left(n\geq1:\zeta\left(1\right)+\cdots+\zeta\left(n-1\right)\leq
t\right)$. Si $\esp\left[\zeta\left(1\right)\right]<\infty$,
entonces para cualquier entero $r\geq1$
\begin{equation}
 lim_{t\rightarrow\infty}\esp\left[\left(\frac{E\left(t\right)}{t}\right)^{r}\right]=\left(\frac{1}{\esp\left[\zeta_{1}\right]}\right)^{r}.
\end{equation}
Luego, bajo estas condiciones:
\begin{itemize}
 \item[a)] para cualquier $\delta>0$, $\sup_{t\geq\delta}\esp\left[\left(\frac{E\left(t\right)}{t}\right)^{r}\right]<\infty$
\item[b)] las variables aleatorias
$\left\{\left(\frac{E\left(t\right)}{t}\right)^{r}:t\geq1\right\}$
son uniformemente integrables.
\end{itemize}
\end{Lemma}

\begin{Teo}[Teorema 5.5, Dai y Meyn \cite{DaiSean}]\label{Tma.5.5.DaiSean}
Suponga que los supuestos A1) y A2) se cumplen y que el modelo de
flujo es estable. Entonces existe una constante $\kappa_{p}$ tal
que
\begin{equation}
\frac{1}{t}\int_{0}^{t}\esp_{x}\left[|Q\left(s\right)|^{p}\right]ds\leq\kappa_{p}\left\{\frac{1}{t}|x|^{p+1}+1\right\}
\end{equation}
para $t>0$ y $x\in X$. En particular, para cada condici\'on
inicial
\begin{eqnarray*}
\limsup_{t\rightarrow\infty}\frac{1}{t}\int_{0}^{t}\esp_{x}\left[|Q\left(s\right)|^{p}\right]ds\leq\kappa_{p}.
\end{eqnarray*}
\end{Teo}

\begin{Teo}[Teorema 6.2, Dai y Meyn \cite{DaiSean}]\label{Tma.6.2.DaiSean}
Suponga que se cumplen los supuestos A1), A2) y A3) y que el
modelo de flujo es estable. Entonces se tiene que
\begin{equation}
\left\|P^{t}\left(x,\cdot\right)-\pi\left(\cdot\right)\right\|_{f_{p}}\textrm{,
}t\rightarrow\infty,x\in X.
\end{equation}
En particular para cada condici\'on inicial
\begin{eqnarray*}
\lim_{t\rightarrow\infty}\esp_{x}\left[|Q\left(t\right)|^{p}\right]=\esp_{\pi}\left[|Q\left(0\right)|^{p}\right]\leq\kappa_{r}
\end{eqnarray*}
\end{Teo}
\begin{Teo}[Teorema 6.3, Dai y Meyn \cite{DaiSean}]\label{Tma.6.3.DaiSean}
Suponga que se cumplen los supuestos A1), A2) y A3) y que el
modelo de flujo es estable. Entonces con
$f\left(x\right)=f_{1}\left(x\right)$ se tiene
\begin{equation}
\lim_{t\rightarrow\infty}t^{p-1}\left\|P^{t}\left(x,\cdot\right)-\pi\left(\cdot\right)\right\|_{f}=0.
\end{equation}
En particular para cada condici\'on inicial
\begin{eqnarray*}
\lim_{t\rightarrow\infty}t^{p-1}|\esp_{x}\left[Q\left(t\right)\right]-\esp_{\pi}\left[Q\left(0\right)\right]|=0.
\end{eqnarray*}
\end{Teo}

\begin{Teo}[Teorema 6.4, Dai y Meyn, \cite{DaiSean}]\label{Tma.6.4.DaiSean}
Suponga que se cumplen los supuestos A1), A2) y A3) y que el
modelo de flujo es estable. Sea $\nu$ cualquier distribuci\'on de
probabilidad en $\left(X,\mathcal{B}_{X}\right)$, y $\pi$ la
distribuci\'on estacionaria de $X$.
\begin{itemize}
\item[i)] Para cualquier $f:X\leftarrow\rea_{+}$
\begin{equation}
\lim_{t\rightarrow\infty}\frac{1}{t}\int_{o}^{t}f\left(X\left(s\right)\right)ds=\pi\left(f\right):=\int
f\left(x\right)\pi\left(dx\right)
\end{equation}
$\prob$-c.s.

\item[ii)] Para cualquier $f:X\leftarrow\rea_{+}$ con
$\pi\left(|f|\right)<\infty$, la ecuaci\'on anterior se cumple.
\end{itemize}
\end{Teo}

\begin{Teo}[Teorema 2.2, Down \cite{Down}]\label{Tma2.2.Down}
Suponga que el fluido modelo es inestable en el sentido de que
para alguna $\epsilon_{0},c_{0}\geq0$,
\begin{equation}\label{Eq.Inestability}
|Q\left(T\right)|\geq\epsilon_{0}T-c_{0}\textrm{,   }T\geq0,
\end{equation}
para cualquier condici\'on inicial $Q\left(0\right)$, con
$|Q\left(0\right)|=1$. Entonces para cualquier $0<q\leq1$, existe
$B<0$ tal que para cualquier $|x|\geq B$,
\begin{equation}
\prob_{x}\left\{\mathbb{X}\rightarrow\infty\right\}\geq q.
\end{equation}
\end{Teo}



\begin{Def}
Sea $X$ un conjunto y $\mathcal{F}$ una $\sigma$-\'algebra de
subconjuntos de $X$, la pareja $\left(X,\mathcal{F}\right)$ es
llamado espacio medible. Un subconjunto $A$ de $X$ es llamado
medible, o medible con respecto a $\mathcal{F}$, si
$A\in\mathcal{F}$.
\end{Def}

\begin{Def}
Sea $\left(X,\mathcal{F},\mu\right)$ espacio de medida. Se dice
que la medida $\mu$ es $\sigma$-finita si se puede escribir
$X=\bigcup_{n\geq1}X_{n}$ con $X_{n}\in\mathcal{F}$ y
$\mu\left(X_{n}\right)<\infty$.
\end{Def}

\begin{Def}\label{Cto.Borel}
Sea $X$ el conjunto de los n\'umeros reales $\rea$. El \'algebra
de Borel es la $\sigma$-\'algebra $B$ generada por los intervalos
abiertos $\left(a,b\right)\in\rea$. Cualquier conjunto en $B$ es
llamado {\em Conjunto de Borel}.
\end{Def}

\begin{Def}\label{Funcion.Medible}
Una funci\'on $f:X\rightarrow\rea$, es medible si para cualquier
n\'umero real $\alpha$ el conjunto
\[\left\{x\in X:f\left(x\right)>\alpha\right\}\]
pertenece a $\mathcal{F}$. Equivalentemente, se dice que $f$ es
medible si
\[f^{-1}\left(\left(\alpha,\infty\right)\right)=\left\{x\in X:f\left(x\right)>\alpha\right\}\in\mathcal{F}.\]
\end{Def}


\begin{Def}\label{Def.Cilindros}
Sean $\left(\Omega_{i},\mathcal{F}_{i}\right)$, $i=1,2,\ldots,$
espacios medibles y $\Omega=\prod_{i=1}^{\infty}\Omega_{i}$ el
conjunto de todas las sucesiones
$\left(\omega_{1},\omega_{2},\ldots,\right)$ tales que
$\omega_{i}\in\Omega_{i}$, $i=1,2,\ldots,$. Si
$B^{n}\subset\prod_{i=1}^{\infty}\Omega_{i}$, definimos
$B_{n}=\left\{\omega\in\Omega:\left(\omega_{1},\omega_{2},\ldots,\omega_{n}\right)\in
B^{n}\right\}$. Al conjunto $B_{n}$ se le llama {\em cilindro} con
base $B^{n}$, el cilindro es llamado medible si
$B^{n}\in\prod_{i=1}^{\infty}\mathcal{F}_{i}$.
\end{Def}


\begin{Def}\label{Def.Proc.Adaptado}[TSP, Ash \cite{RBA}]
Sea $X\left(t\right),t\geq0$ proceso estoc\'astico, el proceso es
adaptado a la familia de $\sigma$-\'algebras $\mathcal{F}_{t}$,
para $t\geq0$, si para $s<t$ implica que
$\mathcal{F}_{s}\subset\mathcal{F}_{t}$, y $X\left(t\right)$ es
$\mathcal{F}_{t}$-medible para cada $t$. Si no se especifica
$\mathcal{F}_{t}$ entonces se toma $\mathcal{F}_{t}$ como
$\mathcal{F}\left(X\left(s\right),s\leq t\right)$, la m\'as
peque\~na $\sigma$-\'algebra de subconjuntos de $\Omega$ que hace
que cada $X\left(s\right)$, con $s\leq t$ sea Borel medible.
\end{Def}


\begin{Def}\label{Def.Tiempo.Paro}[TSP, Ash \cite{RBA}]
Sea $\left\{\mathcal{F}\left(t\right),t\geq0\right\}$ familia
creciente de sub $\sigma$-\'algebras. es decir,
$\mathcal{F}\left(s\right)\subset\mathcal{F}\left(t\right)$ para
$s\leq t$. Un tiempo de paro para $\mathcal{F}\left(t\right)$ es
una funci\'on $T:\Omega\rightarrow\left[0,\infty\right]$ tal que
$\left\{T\leq t\right\}\in\mathcal{F}\left(t\right)$ para cada
$t\geq0$. Un tiempo de paro para el proceso estoc\'astico
$X\left(t\right),t\geq0$ es un tiempo de paro para las
$\sigma$-\'algebras
$\mathcal{F}\left(t\right)=\mathcal{F}\left(X\left(s\right)\right)$.
\end{Def}

\begin{Def}
Sea $X\left(t\right),t\geq0$ proceso estoc\'astico, con
$\left(S,\chi\right)$ espacio de estados. Se dice que el proceso
es adaptado a $\left\{\mathcal{F}\left(t\right)\right\}$, es
decir, si para cualquier $s,t\in I$, $I$ conjunto de \'indices,
$s<t$, se tiene que
$\mathcal{F}\left(s\right)\subset\mathcal{F}\left(t\right)$ y
$X\left(t\right)$ es $\mathcal{F}\left(t\right)$-medible,
\end{Def}

\begin{Def}
Sea $X\left(t\right),t\geq0$ proceso estoc\'astico, se dice que es
un Proceso de Markov relativo a $\mathcal{F}\left(t\right)$ o que
$\left\{X\left(t\right),\mathcal{F}\left(t\right)\right\}$ es de
Markov si y s\'olo si para cualquier conjunto $B\in\chi$,  y
$s,t\in I$, $s<t$ se cumple que
\begin{equation}\label{Prop.Markov}
P\left\{X\left(t\right)\in
B|\mathcal{F}\left(s\right)\right\}=P\left\{X\left(t\right)\in
B|X\left(s\right)\right\}.
\end{equation}
\end{Def}
\begin{Note}
Si se dice que $\left\{X\left(t\right)\right\}$ es un Proceso de
Markov sin mencionar $\mathcal{F}\left(t\right)$, se asumir\'a que
\begin{eqnarray*}
\mathcal{F}\left(t\right)=\mathcal{F}_{0}\left(t\right)=\mathcal{F}\left(X\left(r\right),r\leq
t\right),
\end{eqnarray*}
entonces la ecuaci\'on (\ref{Prop.Markov}) se puede escribir como
\begin{equation}
P\left\{X\left(t\right)\in B|X\left(r\right),r\leq s\right\} =
P\left\{X\left(t\right)\in B|X\left(s\right)\right\}
\end{equation}
\end{Note}
%_______________________________________________________________
\subsection{Procesos de Estados de Markov}
%_______________________________________________________________

\begin{Teo}
Sea $\left(X_{n},\mathcal{F}_{n},n=0,1,\ldots,\right\}$ Proceso de
Markov con espacio de estados $\left(S_{0},\chi_{0}\right)$
generado por una distribuici\'on inicial $P_{o}$ y probabilidad de
transici\'on $p_{mn}$, para $m,n=0,1,\ldots,$ $m<n$, que por
notaci\'on se escribir\'a como $p\left(m,n,x,B\right)\rightarrow
p_{mn}\left(x,B\right)$. Sea $S$ tiempo de paro relativo a la
$\sigma$-\'algebra $\mathcal{F}_{n}$. Sea $T$ funci\'on medible,
$T:\Omega\rightarrow\left\{0,1,\ldots,\right\}$. Sup\'ongase que
$T\geq S$, entonces $T$ es tiempo de paro. Si $B\in\chi_{0}$,
entonces
\begin{equation}\label{Prop.Fuerte.Markov}
P\left\{X\left(T\right)\in
B,T<\infty|\mathcal{F}\left(S\right)\right\} =
p\left(S,T,X\left(s\right),B\right)
\end{equation}
en $\left\{T<\infty\right\}$.
\end{Teo}


Sea $K$ conjunto numerable y sea $d:K\rightarrow\nat$ funci\'on.
Para $v\in K$, $M_{v}$ es un conjunto abierto de
$\rea^{d\left(v\right)}$. Entonces \[E=\bigcup_{v\in
K}M_{v}=\left\{\left(v,\zeta\right):v\in K,\zeta\in
M_{v}\right\}.\]

Sea $\mathcal{E}$ la clase de conjuntos medibles en $E$:
\[\mathcal{E}=\left\{\bigcup_{v\in K}A_{v}:A_{v}\in \mathcal{M}_{v}\right\}.\]

donde $\mathcal{M}$ son los conjuntos de Borel de $M_{v}$.
Entonces $\left(E,\mathcal{E}\right)$ es un espacio de Borel. El
estado del proceso se denotar\'a por
$\mathbf{x}_{t}=\left(v_{t},\zeta_{t}\right)$. La distribuci\'on
de $\left(\mathbf{x}_{t}\right)$ est\'a determinada por por los
siguientes objetos:

\begin{itemize}
\item[i)] Los campos vectoriales $\left(\mathcal{H}_{v},v\in
K\right)$. \item[ii)] Una funci\'on medible $\lambda:E\rightarrow
\rea_{+}$. \item[iii)] Una medida de transici\'on
$Q:\mathcal{E}\times\left(E\cup\Gamma^{*}\right)\rightarrow\left[0,1\right]$
donde
\begin{equation}
\Gamma^{*}=\bigcup_{v\in K}\partial^{*}M_{v}.
\end{equation}
y
\begin{equation}
\partial^{*}M_{v}=\left\{z\in\partial M_{v}:\mathbf{\mathbf{\phi}_{v}\left(t,\zeta\right)=\mathbf{z}}\textrm{ para alguna }\left(t,\zeta\right)\in\rea_{+}\times M_{v}\right\}.
\end{equation}
$\partial M_{v}$ denota  la frontera de $M_{v}$.
\end{itemize}

El campo vectorial $\left(\mathcal{H}_{v},v\in K\right)$ se supone
tal que para cada $\mathbf{z}\in M_{v}$ existe una \'unica curva
integral $\mathbf{\phi}_{v}\left(t,\zeta\right)$ que satisface la
ecuaci\'on

\begin{equation}
\frac{d}{dt}f\left(\zeta_{t}\right)=\mathcal{H}f\left(\zeta_{t}\right),
\end{equation}
con $\zeta_{0}=\mathbf{z}$, para cualquier funci\'on suave
$f:\rea^{d}\rightarrow\rea$ y $\mathcal{H}$ denota el operador
diferencial de primer orden, con $\mathcal{H}=\mathcal{H}_{v}$ y
$\zeta_{t}=\mathbf{\phi}\left(t,\mathbf{z}\right)$. Adem\'as se
supone que $\mathcal{H}_{v}$ es conservativo, es decir, las curvas
integrales est\'an definidas para todo $t>0$.

Para $\mathbf{x}=\left(v,\zeta\right)\in E$ se denota
\[t^{*}\mathbf{x}=inf\left\{t>0:\mathbf{\phi}_{v}\left(t,\zeta\right)\in\partial^{*}M_{v}\right\}\]

En lo que respecta a la funci\'on $\lambda$, se supondr\'a que
para cada $\left(v,\zeta\right)\in E$ existe un $\epsilon>0$ tal
que la funci\'on
$s\rightarrow\lambda\left(v,\phi_{v}\left(s,\zeta\right)\right)\in
E$ es integrable para $s\in\left[0,\epsilon\right)$. La medida de
transici\'on $Q\left(A;\mathbf{x}\right)$ es una funci\'on medible
de $\mathbf{x}$ para cada $A\in\mathcal{E}$, definida para
$\mathbf{x}\in E\cup\Gamma^{*}$ y es una medida de probabilidad en
$\left(E,\mathcal{E}\right)$ para cada $\mathbf{x}\in E$.

El movimiento del proceso $\left(\mathbf{x}_{t}\right)$ comenzando
en $\mathbf{x}=\left(n,\mathbf{z}\right)\in E$ se puede construir
de la siguiente manera, def\'inase la funci\'on $F$ por

\begin{equation}
F\left(t\right)=\left\{\begin{array}{ll}\\
exp\left(-\int_{0}^{t}\lambda\left(n,\phi_{n}\left(s,\mathbf{z}\right)\right)ds\right), & t<t^{*}\left(\mathbf{x}\right),\\
0, & t\geq t^{*}\left(\mathbf{x}\right)
\end{array}\right.
\end{equation}

Sea $T_{1}$ una variable aleatoria tal que
$\prob\left[T_{1}>t\right]=F\left(t\right)$, ahora sea la variable
aleatoria $\left(N,Z\right)$ con distribuici\'on
$Q\left(\cdot;\phi_{n}\left(T_{1},\mathbf{z}\right)\right)$. La
trayectoria de $\left(\mathbf{x}_{t}\right)$ para $t\leq T_{1}$ es
\begin{eqnarray*}
\mathbf{x}_{t}=\left(v_{t},\zeta_{t}\right)=\left\{\begin{array}{ll}
\left(n,\phi_{n}\left(t,\mathbf{z}\right)\right), & t<T_{1},\\
\left(N,\mathbf{Z}\right), & t=t_{1}.
\end{array}\right.
\end{eqnarray*}

Comenzando en $\mathbf{x}_{T_{1}}$ se selecciona el siguiente
tiempo de intersalto $T_{2}-T_{1}$ lugar del post-salto
$\mathbf{x}_{T_{2}}$ de manera similar y as\'i sucesivamente. Este
procedimiento nos da una trayectoria determinista por partes
$\mathbf{x}_{t}$ con tiempos de salto $T_{1},T_{2},\ldots$. Bajo
las condiciones enunciadas para $\lambda,T_{1}>0$  y
$T_{1}-T_{2}>0$ para cada $i$, con probabilidad 1. Se supone que
se cumple la siguiente condici\'on.

\begin{Sup}[Supuesto 3.1, Davis \cite{Davis}]\label{Sup3.1.Davis}
Sea $N_{t}:=\sum_{t}\indora_{\left(t\geq t\right)}$ el n\'umero de
saltos en $\left[0,t\right]$. Entonces
\begin{equation}
\esp\left[N_{t}\right]<\infty\textrm{ para toda }t.
\end{equation}
\end{Sup}

es un proceso de Markov, m\'as a\'un, es un Proceso Fuerte de
Markov, es decir, la Propiedad Fuerte de Markov\footnote{Revisar
p\'agina 362, y 364 de Davis \cite{Davis}.} se cumple para
cualquier tiempo de paro.
%_________________________________________________________________________
%\renewcommand{\refname}{PROCESOS ESTOC\'ASTICOS}
%\renewcommand{\appendixname}{PROCESOS ESTOC\'ASTICOS}
%\renewcommand{\appendixtocname}{PROCESOS ESTOC\'ASTICOS}
%\renewcommand{\appendixpagename}{PROCESOS ESTOC\'ASTICOS}
%\appendix
%\clearpage % o \cleardoublepage
%\addappheadtotoc
%\appendixpage
%_________________________________________________________________________
\subsection{Teor\'ia General de Procesos Estoc\'asticos}
%_________________________________________________________________________
En esta secci\'on se har\'an las siguientes consideraciones: $E$
es un espacio m\'etrico separable y la m\'etrica $d$ es compatible
con la topolog\'ia.

\begin{Def}
Una medida finita, $\lambda$ en la $\sigma$-\'algebra de Borel de
un espacio metrizable $E$ se dice cerrada si
\begin{equation}\label{Eq.A2.3}
\lambda\left(E\right)=sup\left\{\lambda\left(K\right):K\textrm{ es
compacto en }E\right\}.
\end{equation}
\end{Def}

\begin{Def}
$E$ es un espacio de Rad\'on si cada medida finita en
$\left(E,\mathcal{B}\left(E\right)\right)$ es regular interior o cerrada,
{\em tight}.
\end{Def}


El siguiente teorema nos permite tener una mejor caracterizaci\'on de los espacios de Rad\'on:
\begin{Teo}\label{Tma.A2.2}
Sea $E$ espacio separable metrizable. Entonces $E$ es de Rad\'on
si y s\'olo s\'i cada medida finita en
$\left(E,\mathcal{B}\left(E\right)\right)$ es cerrada.
\end{Teo}

%_________________________________________________________________________________________
\subsection{Propiedades de Markov}
%_________________________________________________________________________________________

Sea $E$ espacio de estados, tal que $E$ es un espacio de Rad\'on, $\mathcal{B}\left(E\right)$ $\sigma$-\'algebra de Borel en $E$, que se denotar\'a por $\mathcal{E}$.

Sea $\left(X,\mathcal{G},\prob\right)$ espacio de probabilidad,
$I\subset\rea$ conjunto de índices. Sea $\mathcal{F}_{\leq t}$ la
$\sigma$-\'algebra natural definida como
$\sigma\left\{f\left(X_{r}\right):r\in I, r\leq
t,f\in\mathcal{E}\right\}$. Se considerar\'a una
$\sigma$-\'algebra m\'as general\footnote{qu\'e se quiere decir
con el t\'ermino: m\'as general?}, $ \left(\mathcal{G}_{t}\right)$
tal que $\left(X_{t}\right)$ sea $\mathcal{E}$-adaptado.

\begin{Def}
Una familia $\left(P_{s,t}\right)$ de kernels de Markov en $\left(E,\mathcal{E}\right)$ indexada por pares $s,t\in I$, con $s\leq t$ es una funci\'on de transici\'on en $\ER$, si  para todo $r\leq s< t$ en $I$ y todo $x\in E$, $B\in\mathcal{E}$
\begin{equation}\label{Eq.Kernels}
P_{r,t}\left(x,B\right)=\int_{E}P_{r,s}\left(x,dy\right)P_{s,t}\left(y,B\right)\footnote{Ecuaci\'on de Chapman-Kolmogorov}.
\end{equation}
\end{Def}

Se dice que la funci\'on de transici\'on $\KM$ en $\ER$ es la funci\'on de transici\'on para un proceso $\PE$  con valores en $E$ y que satisface la propiedad de Markov\footnote{\begin{equation}\label{Eq.1.4.S}
\prob\left\{H|\mathcal{G}_{t}\right\}=\prob\left\{H|X_{t}\right\}\textrm{ }H\in p\mathcal{F}_{\geq t}.
\end{equation}} (\ref{Eq.1.4.S}) relativa a $\left(\mathcal{G}_{t}\right)$ si

\begin{equation}\label{Eq.1.6.S}
\prob\left\{f\left(X_{t}\right)|\mathcal{G}_{s}\right\}=P_{s,t}f\left(X_{t}\right)\textrm{ }s\leq t\in I,\textrm{ }f\in b\mathcal{E}.
\end{equation}

\begin{Def}
Una familia $\left(P_{t}\right)_{t\geq0}$ de kernels de Markov en $\ER$ es llamada {\em Semigrupo de Transici\'on de Markov} o {\em Semigrupo de Transici\'on} si
\[P_{t+s}f\left(x\right)=P_{t}\left(P_{s}f\right)\left(x\right),\textrm{ }t,s\geq0,\textrm{ }x\in E\textrm{ }f\in b\mathcal{E}\footnote{Definir los t\'ermino $b\mathcal{E}$ y $p\mathcal{E}$}.\]
\end{Def}
\begin{Note}
Si la funci\'on de transici\'on $\KM$ es llamada homog\'enea si $P_{s,t}=P_{t-s}$.
\end{Note}

Un proceso de Markov que satisface la ecuaci\'on (\ref{Eq.1.6.S}) con funci\'on de transici\'on homog\'enea $\left(P_{t}\right)$ tiene la propiedad caracter\'istica
\begin{equation}\label{Eq.1.8.S}
\prob\left\{f\left(X_{t+s}\right)|\mathcal{G}_{t}\right\}=P_{s}f\left(X_{t}\right)\textrm{ }t,s\geq0,\textrm{ }f\in b\mathcal{E}.
\end{equation}
La ecuaci\'on anterior es la {\em Propiedad Simple de Markov} de $X$ relativa a $\left(P_{t}\right)$.

En este sentido el proceso $\PE$ cumple con la propiedad de Markov (\ref{Eq.1.8.S}) relativa a $\left(\Omega,\mathcal{G},\mathcal{G}_{t},\prob\right)$ con semigrupo de transici\'on $\left(P_{t}\right)$.
%_________________________________________________________________________________________
\subsection{Primer Condici\'on de Regularidad}
%_________________________________________________________________________________________
%\newcommand{\EM}{\left(\Omega,\mathcal{G},\prob\right)}
%\newcommand{\E4}{\left(\Omega,\mathcal{G},\mathcal{G}_{t},\prob\right)}
\begin{Def}
Un proceso estoc\'astico $\PE$ definido en
$\left(\Omega,\mathcal{G},\prob\right)$ con valores en el espacio
topol\'ogico $E$ es continuo por la derecha si cada trayectoria
muestral $t\rightarrow X_{t}\left(w\right)$ es un mapeo continuo
por la derecha de $I$ en $E$.
\end{Def}

\begin{Def}[HD1]\label{Eq.2.1.S}
Un semigrupo de Markov $\left(P_{t}\right)$ en un espacio de
Rad\'on $E$ se dice que satisface la condici\'on {\em HD1} si,
dada una medida de probabilidad $\mu$ en $E$, existe una
$\sigma$-\'algebra $\mathcal{E^{*}}$ con
$\mathcal{E}\subset\mathcal{E}^{*}$ y
$P_{t}\left(b\mathcal{E}^{*}\right)\subset b\mathcal{E}^{*}$, y un
$\mathcal{E}^{*}$-proceso $E$-valuado continuo por la derecha
$\PE$ en alg\'un espacio de probabilidad filtrado
$\left(\Omega,\mathcal{G},\mathcal{G}_{t},\prob\right)$ tal que
$X=\left(\Omega,\mathcal{G},\mathcal{G}_{t},\prob\right)$ es de
Markov (Homog\'eneo) con semigrupo de transici\'on $(P_{t})$ y
distribuci\'on inicial $\mu$.
\end{Def}

Consid\'erese la colecci\'on de variables aleatorias $X_{t}$
definidas en alg\'un espacio de probabilidad, y una colecci\'on de
medidas $\mathbf{P}^{x}$ tales que
$\mathbf{P}^{x}\left\{X_{0}=x\right\}$, y bajo cualquier
$\mathbf{P}^{x}$, $X_{t}$ es de Markov con semigrupo
$\left(P_{t}\right)$. $\mathbf{P}^{x}$ puede considerarse como la
distribuci\'on condicional de $\mathbf{P}$ dado $X_{0}=x$.

\begin{Def}\label{Def.2.2.S}
Sea $E$ espacio de Rad\'on, $\SG$ semigrupo de Markov en $\ER$. La colecci\'on $\mathbf{X}=\left(\Omega,\mathcal{G},\mathcal{G}_{t},X_{t},\theta_{t},\CM\right)$ es un proceso $\mathcal{E}$-Markov continuo por la derecha simple, con espacio de estados $E$ y semigrupo de transici\'on $\SG$ en caso de que $\mathbf{X}$ satisfaga las siguientes condiciones:
\begin{itemize}
\item[i)] $\left(\Omega,\mathcal{G},\mathcal{G}_{t}\right)$ es un espacio de medida filtrado, y $X_{t}$ es un proceso $E$-valuado continuo por la derecha $\mathcal{E}^{*}$-adaptado a $\left(\mathcal{G}_{t}\right)$;

\item[ii)] $\left(\theta_{t}\right)_{t\geq0}$ es una colecci\'on de operadores {\em shift} para $X$, es decir, mapea $\Omega$ en s\'i mismo satisfaciendo para $t,s\geq0$,

\begin{equation}\label{Eq.Shift}
\theta_{t}\circ\theta_{s}=\theta_{t+s}\textrm{ y }X_{t}\circ\theta_{t}=X_{t+s};
\end{equation}

\item[iii)] Para cualquier $x\in E$,$\CM\left\{X_{0}=x\right\}=1$, y el proceso $\PE$ tiene la propiedad de Markov (\ref{Eq.1.8.S}) con semigrupo de transici\'on $\SG$ relativo a $\left(\Omega,\mathcal{G},\mathcal{G}_{t},\CM\right)$.
\end{itemize}
\end{Def}

\begin{Def}[HD2]\label{Eq.2.2.S}
Para cualquier $\alpha>0$ y cualquier $f\in S^{\alpha}$, el proceso $t\rightarrow f\left(X_{t}\right)$ es continuo por la derecha casi seguramente.
\end{Def}

\begin{Def}\label{Def.PD}
Un sistema $\mathbf{X}=\left(\Omega,\mathcal{G},\mathcal{G}_{t},X_{t},\theta_{t},\CM\right)$ es un proceso derecho en el espacio de Rad\'on $E$ con semigrupo de transici\'on $\SG$ provisto de:
\begin{itemize}
\item[i)] $\mathbf{X}$ es una realizaci\'on  continua por la derecha, \ref{Def.2.2.S}, de $\SG$.

\item[ii)] $\mathbf{X}$ satisface la condicion HD2, \ref{Eq.2.2.S}, relativa a $\mathcal{G}_{t}$.

\item[iii)] $\mathcal{G}_{t}$ es aumentado y continuo por la derecha.
\end{itemize}
\end{Def}


%_________________________________________________________________________
%\renewcommand{\refname}{MODELO DE FLUJO}
%\renewcommand{\appendixname}{MODELO DE FLUJO}
%\renewcommand{\appendixtocname}{MODELO DE FLUJO}
%\renewcommand{\appendixpagename}{MODELO DE FLUJO}
%\appendix
%\clearpage % o \cleardoublepage
%\addappheadtotoc
%\appendixpage

\subsection{Construcci\'on del Modelo de Flujo}


\begin{Lema}[Lema 4.2, Dai\cite{Dai}]\label{Lema4.2}
Sea $\left\{x_{n}\right\}\subset \mathbf{X}$ con
$|x_{n}|\rightarrow\infty$, conforme $n\rightarrow\infty$. Suponga
que
\[lim_{n\rightarrow\infty}\frac{1}{|x_{n}|}U\left(0\right)=\overline{U}\]
y
\[lim_{n\rightarrow\infty}\frac{1}{|x_{n}|}V\left(0\right)=\overline{V}.\]

Entonces, conforme $n\rightarrow\infty$, casi seguramente

\begin{equation}\label{E1.4.2}
\frac{1}{|x_{n}|}\Phi^{k}\left(\left[|x_{n}|t\right]\right)\rightarrow
P_{k}^{'}t\textrm{, u.o.c.,}
\end{equation}

\begin{equation}\label{E1.4.3}
\frac{1}{|x_{n}|}E^{x_{n}}_{k}\left(|x_{n}|t\right)\rightarrow
\alpha_{k}\left(t-\overline{U}_{k}\right)^{+}\textrm{, u.o.c.,}
\end{equation}

\begin{equation}\label{E1.4.4}
\frac{1}{|x_{n}|}S^{x_{n}}_{k}\left(|x_{n}|t\right)\rightarrow
\mu_{k}\left(t-\overline{V}_{k}\right)^{+}\textrm{, u.o.c.,}
\end{equation}

donde $\left[t\right]$ es la parte entera de $t$ y
$\mu_{k}=1/m_{k}=1/\esp\left[\eta_{k}\left(1\right)\right]$.
\end{Lema}

\begin{Lema}[Lema 4.3, Dai\cite{Dai}]\label{Lema.4.3}
Sea $\left\{x_{n}\right\}\subset \mathbf{X}$ con
$|x_{n}|\rightarrow\infty$, conforme $n\rightarrow\infty$. Suponga
que
\[lim_{n\rightarrow\infty}\frac{1}{|x_{n}|}U_{k}\left(0\right)=\overline{U}_{k}\]
y
\[lim_{n\rightarrow\infty}\frac{1}{|x_{n}|}V_{k}\left(0\right)=\overline{V}_{k}.\]
\begin{itemize}
\item[a)] Conforme $n\rightarrow\infty$ casi seguramente,
\[lim_{n\rightarrow\infty}\frac{1}{|x_{n}|}U^{x_{n}}_{k}\left(|x_{n}|t\right)=\left(\overline{U}_{k}-t\right)^{+}\textrm{, u.o.c.}\]
y
\[lim_{n\rightarrow\infty}\frac{1}{|x_{n}|}V^{x_{n}}_{k}\left(|x_{n}|t\right)=\left(\overline{V}_{k}-t\right)^{+}.\]

\item[b)] Para cada $t\geq0$ fijo,
\[\left\{\frac{1}{|x_{n}|}U^{x_{n}}_{k}\left(|x_{n}|t\right),|x_{n}|\geq1\right\}\]
y
\[\left\{\frac{1}{|x_{n}|}V^{x_{n}}_{k}\left(|x_{n}|t\right),|x_{n}|\geq1\right\}\]
\end{itemize}
son uniformemente convergentes.
\end{Lema}

Sea $S_{l}^{x}\left(t\right)$ el n\'umero total de servicios
completados de la clase $l$, si la clase $l$ est\'a dando $t$
unidades de tiempo de servicio. Sea $T_{l}^{x}\left(x\right)$ el
monto acumulado del tiempo de servicio que el servidor
$s\left(l\right)$ gasta en los usuarios de la clase $l$ al tiempo
$t$. Entonces $S_{l}^{x}\left(T_{l}^{x}\left(t\right)\right)$ es
el n\'umero total de servicios completados para la clase $l$ al
tiempo $t$. Una fracci\'on de estos usuarios,
$\Phi_{k}^{x}\left(S_{l}^{x}\left(T_{l}^{x}\left(t\right)\right)\right)$,
se convierte en usuarios de la clase $k$.\\

Entonces, dado lo anterior, se tiene la siguiente representaci\'on
para el proceso de la longitud de la cola:\\

\begin{equation}
Q_{k}^{x}\left(t\right)=Q_{k}^{x}\left(0\right)+E_{k}^{x}\left(t\right)+\sum_{l=1}^{K}\Phi_{k}^{l}\left(S_{l}^{x}\left(T_{l}^{x}\left(t\right)\right)\right)-S_{k}^{x}\left(T_{k}^{x}\left(t\right)\right)
\end{equation}
para $k=1,\ldots,K$. Para $i=1,\ldots,d$, sea
\[I_{i}^{x}\left(t\right)=t-\sum_{j\in C_{i}}T_{k}^{x}\left(t\right).\]

Entonces $I_{i}^{x}\left(t\right)$ es el monto acumulado del
tiempo que el servidor $i$ ha estado desocupado al tiempo $t$. Se
est\'a asumiendo que las disciplinas satisfacen la ley de
conservaci\'on del trabajo, es decir, el servidor $i$ est\'a en
pausa solamente cuando no hay usuarios en la estaci\'on $i$.
Entonces, se tiene que

\begin{equation}
\int_{0}^{\infty}\left(\sum_{k\in
C_{i}}Q_{k}^{x}\left(t\right)\right)dI_{i}^{x}\left(t\right)=0,
\end{equation}
para $i=1,\ldots,d$.\\

Hacer
\[T^{x}\left(t\right)=\left(T_{1}^{x}\left(t\right),\ldots,T_{K}^{x}\left(t\right)\right)^{'},\]
\[I^{x}\left(t\right)=\left(I_{1}^{x}\left(t\right),\ldots,I_{K}^{x}\left(t\right)\right)^{'}\]
y
\[S^{x}\left(T^{x}\left(t\right)\right)=\left(S_{1}^{x}\left(T_{1}^{x}\left(t\right)\right),\ldots,S_{K}^{x}\left(T_{K}^{x}\left(t\right)\right)\right)^{'}.\]

Para una disciplina que cumple con la ley de conservaci\'on del
trabajo, en forma vectorial, se tiene el siguiente conjunto de
ecuaciones

\begin{equation}\label{Eq.MF.1.3}
Q^{x}\left(t\right)=Q^{x}\left(0\right)+E^{x}\left(t\right)+\sum_{l=1}^{K}\Phi^{l}\left(S_{l}^{x}\left(T_{l}^{x}\left(t\right)\right)\right)-S^{x}\left(T^{x}\left(t\right)\right),\\
\end{equation}

\begin{equation}\label{Eq.MF.2.3}
Q^{x}\left(t\right)\geq0,\\
\end{equation}

\begin{equation}\label{Eq.MF.3.3}
T^{x}\left(0\right)=0,\textrm{ y }\overline{T}^{x}\left(t\right)\textrm{ es no decreciente},\\
\end{equation}

\begin{equation}\label{Eq.MF.4.3}
I^{x}\left(t\right)=et-CT^{x}\left(t\right)\textrm{ es no
decreciente}\\
\end{equation}

\begin{equation}\label{Eq.MF.5.3}
\int_{0}^{\infty}\left(CQ^{x}\left(t\right)\right)dI_{i}^{x}\left(t\right)=0,\\
\end{equation}

\begin{equation}\label{Eq.MF.6.3}
\textrm{Condiciones adicionales en
}\left(\overline{Q}^{x}\left(\cdot\right),\overline{T}^{x}\left(\cdot\right)\right)\textrm{
espec\'ificas de la disciplina de la cola,}
\end{equation}

donde $e$ es un vector de unos de dimensi\'on $d$, $C$ es la
matriz definida por
\[C_{ik}=\left\{\begin{array}{cc}
1,& S\left(k\right)=i,\\
0,& \textrm{ en otro caso}.\\
\end{array}\right.
\]
Es necesario enunciar el siguiente Teorema que se utilizar\'a para
el Teorema \ref{Tma.4.2.Dai}:
\begin{Teo}[Teorema 4.1, Dai \cite{Dai}]
Considere una disciplina que cumpla la ley de conservaci\'on del
trabajo, para casi todas las trayectorias muestrales $\omega$ y
cualquier sucesi\'on de estados iniciales
$\left\{x_{n}\right\}\subset \mathbf{X}$, con
$|x_{n}|\rightarrow\infty$, existe una subsucesi\'on
$\left\{x_{n_{j}}\right\}$ con $|x_{n_{j}}|\rightarrow\infty$ tal
que
\begin{equation}\label{Eq.4.15}
\frac{1}{|x_{n_{j}}|}\left(Q^{x_{n_{j}}}\left(0\right),U^{x_{n_{j}}}\left(0\right),V^{x_{n_{j}}}\left(0\right)\right)\rightarrow\left(\overline{Q}\left(0\right),\overline{U},\overline{V}\right),
\end{equation}

\begin{equation}\label{Eq.4.16}
\frac{1}{|x_{n_{j}}|}\left(Q^{x_{n_{j}}}\left(|x_{n_{j}}|t\right),T^{x_{n_{j}}}\left(|x_{n_{j}}|t\right)\right)\rightarrow\left(\overline{Q}\left(t\right),\overline{T}\left(t\right)\right)\textrm{
u.o.c.}
\end{equation}

Adem\'as,
$\left(\overline{Q}\left(t\right),\overline{T}\left(t\right)\right)$
satisface las siguientes ecuaciones:
\begin{equation}\label{Eq.MF.1.3a}
\overline{Q}\left(t\right)=Q\left(0\right)+\left(\alpha
t-\overline{U}\right)^{+}-\left(I-P\right)^{'}M^{-1}\left(\overline{T}\left(t\right)-\overline{V}\right)^{+},
\end{equation}

\begin{equation}\label{Eq.MF.2.3a}
\overline{Q}\left(t\right)\geq0,\\
\end{equation}

\begin{equation}\label{Eq.MF.3.3a}
\overline{T}\left(t\right)\textrm{ es no decreciente y comienza en cero},\\
\end{equation}

\begin{equation}\label{Eq.MF.4.3a}
\overline{I}\left(t\right)=et-C\overline{T}\left(t\right)\textrm{
es no decreciente,}\\
\end{equation}

\begin{equation}\label{Eq.MF.5.3a}
\int_{0}^{\infty}\left(C\overline{Q}\left(t\right)\right)d\overline{I}\left(t\right)=0,\\
\end{equation}

\begin{equation}\label{Eq.MF.6.3a}
\textrm{Condiciones adicionales en
}\left(\overline{Q}\left(\cdot\right),\overline{T}\left(\cdot\right)\right)\textrm{
especficas de la disciplina de la cola,}
\end{equation}
\end{Teo}


Propiedades importantes para el modelo de flujo retrasado:

\begin{Prop}
 Sea $\left(\overline{Q},\overline{T},\overline{T}^{0}\right)$ un flujo l\'imite de \ref{Eq.4.4} y suponga que cuando $x\rightarrow\infty$ a lo largo de
una subsucesi\'on
\[\left(\frac{1}{|x|}Q_{k}^{x}\left(0\right),\frac{1}{|x|}A_{k}^{x}\left(0\right),\frac{1}{|x|}B_{k}^{x}\left(0\right),\frac{1}{|x|}B_{k}^{x,0}\left(0\right)\right)\rightarrow\left(\overline{Q}_{k}\left(0\right),0,0,0\right)\]
para $k=1,\ldots,K$. EL flujo l\'imite tiene las siguientes
propiedades, donde las propiedades de la derivada se cumplen donde
la derivada exista:
\begin{itemize}
 \item[i)] Los vectores de tiempo ocupado $\overline{T}\left(t\right)$ y $\overline{T}^{0}\left(t\right)$ son crecientes y continuas con
$\overline{T}\left(0\right)=\overline{T}^{0}\left(0\right)=0$.
\item[ii)] Para todo $t\geq0$
\[\sum_{k=1}^{K}\left[\overline{T}_{k}\left(t\right)+\overline{T}_{k}^{0}\left(t\right)\right]=t\]
\item[iii)] Para todo $1\leq k\leq K$
\[\overline{Q}_{k}\left(t\right)=\overline{Q}_{k}\left(0\right)+\alpha_{k}t-\mu_{k}\overline{T}_{k}\left(t\right)\]
\item[iv)]  Para todo $1\leq k\leq K$
\[\dot{{\overline{T}}}_{k}\left(t\right)=\beta_{k}\] para $\overline{Q}_{k}\left(t\right)=0$.
\item[v)] Para todo $k,j$
\[\mu_{k}^{0}\overline{T}_{k}^{0}\left(t\right)=\mu_{j}^{0}\overline{T}_{j}^{0}\left(t\right)\]
\item[vi)]  Para todo $1\leq k\leq K$
\[\mu_{k}\dot{{\overline{T}}}_{k}\left(t\right)=l_{k}\mu_{k}^{0}\dot{{\overline{T}}}_{k}^{0}\left(t\right)\] para $\overline{Q}_{k}\left(t\right)>0$.
\end{itemize}
\end{Prop}

\begin{Teo}[Teorema 5.1: Ley Fuerte para Procesos de Conteo
\cite{Gut}]\label{Tma.5.1.Gut} Sea
$0<\mu<\esp\left(X_{1}\right]\leq\infty$. entonces

\begin{itemize}
\item[a)] $\frac{N\left(t\right)}{t}\rightarrow\frac{1}{\mu}$
a.s., cuando $t\rightarrow\infty$.


\item[b)]$\esp\left[\frac{N\left(t\right)}{t}\right]^{r}\rightarrow\frac{1}{\mu^{r}}$,
cuando $t\rightarrow\infty$ para todo $r>0$..
\end{itemize}
\end{Teo}


\begin{Prop}[Proposici\'on 5.3 \cite{DaiSean}]
Sea $X$ proceso de estados para la red de colas, y suponga que se
cumplen los supuestos (A1) y (A2), entonces para alguna constante
positiva $C_{p+1}<\infty$, $\delta>0$ y un conjunto compacto
$C\subset X$.

\begin{equation}\label{Eq.5.4}
\esp_{x}\left[\int_{0}^{\tau_{C}\left(\delta\right)}\left(1+|X\left(t\right)|^{p}\right)dt\right]\leq
C_{p+1}\left(1+|x|^{p+1}\right)
\end{equation}
\end{Prop}

\begin{Prop}[Proposici\'on 5.4 \cite{DaiSean}]
Sea $X$ un proceso de Markov Borel Derecho en $X$, sea
$f:X\leftarrow\rea_{+}$ y defina para alguna $\delta>0$, y un
conjunto cerrado $C\subset X$
\[V\left(x\right):=\esp_{x}\left[\int_{0}^{\tau_{C}\left(\delta\right)}f\left(X\left(t\right)\right)dt\right]\]
para $x\in X$. Si $V$ es finito en todas partes y uniformemente
acotada en $C$, entonces existe $k<\infty$ tal que
\begin{equation}\label{Eq.5.11}
\frac{1}{t}\esp_{x}\left[V\left(x\right)\right]+\frac{1}{t}\int_{0}^{t}\esp_{x}\left[f\left(X\left(s\right)\right)ds\right]\leq\frac{1}{t}V\left(x\right)+k,
\end{equation}
para $x\in X$ y $t>0$.
\end{Prop}


%_________________________________________________________________________
%\renewcommand{\refname}{Ap\'endice D}
%\renewcommand{\appendixname}{ESTABILIDAD}
%\renewcommand{\appendixtocname}{ESTABILIDAD}
%\renewcommand{\appendixpagename}{ESTABILIDAD}
%\appendix
%\clearpage % o \cleardoublepage
%\addappheadtotoc
%\appendixpage

\subsection{Estabilidad}

\begin{Def}[Definici\'on 3.2, Dai y Meyn \cite{DaiSean}]
El modelo de flujo retrasado de una disciplina de servicio en una
red con retraso
$\left(\overline{A}\left(0\right),\overline{B}\left(0\right)\right)\in\rea_{+}^{K+|A|}$
se define como el conjunto de ecuaciones dadas en
\ref{Eq.3.8}-\ref{Eq.3.13}, junto con la condici\'on:
\begin{equation}\label{CondAd.FluidModel}
\overline{Q}\left(t\right)=\overline{Q}\left(0\right)+\left(\alpha
t-\overline{A}\left(0\right)\right)^{+}-\left(I-P^{'}\right)M\left(\overline{T}\left(t\right)-\overline{B}\left(0\right)\right)^{+}
\end{equation}
\end{Def}

entonces si el modelo de flujo retrasado tambi\'en es estable:


\begin{Def}[Definici\'on 3.1, Dai y Meyn \cite{DaiSean}]
Un flujo l\'imite (retrasado) para una red bajo una disciplina de
servicio espec\'ifica se define como cualquier soluci\'on
 $\left(\overline{Q}\left(\cdot\right),\overline{T}\left(\cdot\right)\right)$ de las siguientes ecuaciones, donde
$\overline{Q}\left(t\right)=\left(\overline{Q}_{1}\left(t\right),\ldots,\overline{Q}_{K}\left(t\right)\right)^{'}$
y
$\overline{T}\left(t\right)=\left(\overline{T}_{1}\left(t\right),\ldots,\overline{T}_{K}\left(t\right)\right)^{'}$
\begin{equation}\label{Eq.3.8}
\overline{Q}_{k}\left(t\right)=\overline{Q}_{k}\left(0\right)+\alpha_{k}t-\mu_{k}\overline{T}_{k}\left(t\right)+\sum_{l=1}^{k}P_{lk}\mu_{l}\overline{T}_{l}\left(t\right)\\
\end{equation}
\begin{equation}\label{Eq.3.9}
\overline{Q}_{k}\left(t\right)\geq0\textrm{ para }k=1,2,\ldots,K,\\
\end{equation}
\begin{equation}\label{Eq.3.10}
\overline{T}_{k}\left(0\right)=0,\textrm{ y }\overline{T}_{k}\left(\cdot\right)\textrm{ es no decreciente},\\
\end{equation}
\begin{equation}\label{Eq.3.11}
\overline{I}_{i}\left(t\right)=t-\sum_{k\in C_{i}}\overline{T}_{k}\left(t\right)\textrm{ es no decreciente}\\
\end{equation}
\begin{equation}\label{Eq.3.12}
\overline{I}_{i}\left(\cdot\right)\textrm{ se incrementa al tiempo }t\textrm{ cuando }\sum_{k\in C_{i}}Q_{k}^{x}\left(t\right)dI_{i}^{x}\left(t\right)=0\\
\end{equation}
\begin{equation}\label{Eq.3.13}
\textrm{condiciones adicionales sobre
}\left(Q^{x}\left(\cdot\right),T^{x}\left(\cdot\right)\right)\textrm{
referentes a la disciplina de servicio}
\end{equation}
\end{Def}

\begin{Lema}[Lema 3.1 \cite{Chen}]\label{Lema3.1}
Si el modelo de flujo es estable, definido por las ecuaciones
(3.8)-(3.13), entonces el modelo de flujo retrasado tambin es
estable.
\end{Lema}

\begin{Teo}[Teorema 5.1 \cite{Chen}]\label{Tma.5.1.Chen}
La red de colas es estable si existe una constante $t_{0}$ que
depende de $\left(\alpha,\mu,T,U\right)$ y $V$ que satisfagan las
ecuaciones (5.1)-(5.5), $Z\left(t\right)=0$, para toda $t\geq
t_{0}$.
\end{Teo}

\begin{Prop}[Proposici\'on 5.1, Dai y Meyn \cite{DaiSean}]\label{Prop.5.1.DaiSean}
Suponga que los supuestos A1) y A2) son ciertos y que el modelo de flujo es estable. Entonces existe $t_{0}>0$ tal que
\begin{equation}
lim_{|x|\rightarrow\infty}\frac{1}{|x|^{p+1}}\esp_{x}\left[|X\left(t_{0}|x|\right)|^{p+1}\right]=0
\end{equation}
\end{Prop}

\begin{Lemma}[Lema 5.2, Dai y Meyn \cite{DaiSean}]\label{Lema.5.2.DaiSean}
 Sea $\left\{\zeta\left(k\right):k\in \mathbb{z}\right\}$ una sucesi\'on independiente e id\'enticamente distribuida que toma valores en $\left(0,\infty\right)$,
y sea
$E\left(t\right)=max\left(n\geq1:\zeta\left(1\right)+\cdots+\zeta\left(n-1\right)\leq
t\right)$. Si $\esp\left[\zeta\left(1\right)\right]<\infty$,
entonces para cualquier entero $r\geq1$
\begin{equation}
 lim_{t\rightarrow\infty}\esp\left[\left(\frac{E\left(t\right)}{t}\right)^{r}\right]=\left(\frac{1}{\esp\left[\zeta_{1}\right]}\right)^{r}.
\end{equation}
Luego, bajo estas condiciones:
\begin{itemize}
 \item[a)] para cualquier $\delta>0$, $\sup_{t\geq\delta}\esp\left[\left(\frac{E\left(t\right)}{t}\right)^{r}\right]<\infty$
\item[b)] las variables aleatorias
$\left\{\left(\frac{E\left(t\right)}{t}\right)^{r}:t\geq1\right\}$
son uniformemente integrables.
\end{itemize}
\end{Lemma}

\begin{Teo}[Teorema 5.5, Dai y Meyn \cite{DaiSean}]\label{Tma.5.5.DaiSean}
Suponga que los supuestos A1) y A2) se cumplen y que el modelo de
flujo es estable. Entonces existe una constante $\kappa_{p}$ tal
que
\begin{equation}
\frac{1}{t}\int_{0}^{t}\esp_{x}\left[|Q\left(s\right)|^{p}\right]ds\leq\kappa_{p}\left\{\frac{1}{t}|x|^{p+1}+1\right\}
\end{equation}
para $t>0$ y $x\in X$. En particular, para cada condici\'on
inicial
\begin{eqnarray*}
\limsup_{t\rightarrow\infty}\frac{1}{t}\int_{0}^{t}\esp_{x}\left[|Q\left(s\right)|^{p}\right]ds\leq\kappa_{p}.
\end{eqnarray*}
\end{Teo}

\begin{Teo}[Teorema 6.2, Dai y Meyn \cite{DaiSean}]\label{Tma.6.2.DaiSean}
Suponga que se cumplen los supuestos A1), A2) y A3) y que el
modelo de flujo es estable. Entonces se tiene que
\begin{equation}
\left\|P^{t}\left(x,\cdot\right)-\pi\left(\cdot\right)\right\|_{f_{p}}\textrm{,
}t\rightarrow\infty,x\in X.
\end{equation}
En particular para cada condici\'on inicial
\begin{eqnarray*}
\lim_{t\rightarrow\infty}\esp_{x}\left[|Q\left(t\right)|^{p}\right]=\esp_{\pi}\left[|Q\left(0\right)|^{p}\right]\leq\kappa_{r}
\end{eqnarray*}
\end{Teo}
\begin{Teo}[Teorema 6.3, Dai y Meyn \cite{DaiSean}]\label{Tma.6.3.DaiSean}
Suponga que se cumplen los supuestos A1), A2) y A3) y que el
modelo de flujo es estable. Entonces con
$f\left(x\right)=f_{1}\left(x\right)$ se tiene
\begin{equation}
\lim_{t\rightarrow\infty}t^{p-1}\left\|P^{t}\left(x,\cdot\right)-\pi\left(\cdot\right)\right\|_{f}=0.
\end{equation}
En particular para cada condici\'on inicial
\begin{eqnarray*}
\lim_{t\rightarrow\infty}t^{p-1}|\esp_{x}\left[Q\left(t\right)\right]-\esp_{\pi}\left[Q\left(0\right)\right]|=0.
\end{eqnarray*}
\end{Teo}

\begin{Teo}[Teorema 6.4, Dai y Meyn \cite{DaiSean}]\label{Tma.6.4.DaiSean}
Suponga que se cumplen los supuestos A1), A2) y A3) y que el
modelo de flujo es estable. Sea $\nu$ cualquier distribuci\'on de
probabilidad en $\left(X,\mathcal{B}_{X}\right)$, y $\pi$ la
distribuci\'on estacionaria de $X$.
\begin{itemize}
\item[i)] Para cualquier $f:X\leftarrow\rea_{+}$
\begin{equation}
\lim_{t\rightarrow\infty}\frac{1}{t}\int_{o}^{t}f\left(X\left(s\right)\right)ds=\pi\left(f\right):=\int
f\left(x\right)\pi\left(dx\right)
\end{equation}
$\prob$-c.s.

\item[ii)] Para cualquier $f:X\leftarrow\rea_{+}$ con
$\pi\left(|f|\right)<\infty$, la ecuaci\'on anterior se cumple.
\end{itemize}
\end{Teo}

\begin{Teo}[Teorema 2.2, Down \cite{Down}]\label{Tma2.2.Down}
Suponga que el fluido modelo es inestable en el sentido de que
para alguna $\epsilon_{0},c_{0}\geq0$,
\begin{equation}\label{Eq.Inestability}
|Q\left(T\right)|\geq\epsilon_{0}T-c_{0}\textrm{,   }T\geq0,
\end{equation}
para cualquier condici\'on inicial $Q\left(0\right)$, con
$|Q\left(0\right)|=1$. Entonces para cualquier $0<q\leq1$, existe
$B<0$ tal que para cualquier $|x|\geq B$,
\begin{equation}
\prob_{x}\left\{\mathbb{X}\rightarrow\infty\right\}\geq q.
\end{equation}
\end{Teo}


\begin{Def}
Sea $X$ un conjunto y $\mathcal{F}$ una $\sigma$-\'algebra de
subconjuntos de $X$, la pareja $\left(X,\mathcal{F}\right)$ es
llamado espacio medible. Un subconjunto $A$ de $X$ es llamado
medible, o medible con respecto a $\mathcal{F}$, si
$A\in\mathcal{F}$.
\end{Def}

\begin{Def}
Sea $\left(X,\mathcal{F},\mu\right)$ espacio de medida. Se dice
que la medida $\mu$ es $\sigma$-finita si se puede escribir
$X=\bigcup_{n\geq1}X_{n}$ con $X_{n}\in\mathcal{F}$ y
$\mu\left(X_{n}\right)<\infty$.
\end{Def}

\begin{Def}\label{Cto.Borel}
Sea $X$ el conjunto de los \'umeros reales $\rea$. El \'algebra de
Borel es la $\sigma$-\'algebra $B$ generada por los intervalos
abiertos $\left(a,b\right)\in\rea$. Cualquier conjunto en $B$ es
llamado {\em Conjunto de Borel}.
\end{Def}

\begin{Def}\label{Funcion.Medible}
Una funci\'on $f:X\rightarrow\rea$, es medible si para cualquier
n\'umero real $\alpha$ el conjunto
\[\left\{x\in X:f\left(x\right)>\alpha\right\}\]
pertenece a $X$. Equivalentemente, se dice que $f$ es medible si
\[f^{-1}\left(\left(\alpha,\infty\right)\right)=\left\{x\in X:f\left(x\right)>\alpha\right\}\in\mathcal{F}.\]
\end{Def}


\begin{Def}\label{Def.Cilindros}
Sean $\left(\Omega_{i},\mathcal{F}_{i}\right)$, $i=1,2,\ldots,$
espacios medibles y $\Omega=\prod_{i=1}^{\infty}\Omega_{i}$ el
conjunto de todas las sucesiones
$\left(\omega_{1},\omega_{2},\ldots,\right)$ tales que
$\omega_{i}\in\Omega_{i}$, $i=1,2,\ldots,$. Si
$B^{n}\subset\prod_{i=1}^{\infty}\Omega_{i}$, definimos
$B_{n}=\left\{\omega\in\Omega:\left(\omega_{1},\omega_{2},\ldots,\omega_{n}\right)\in
B^{n}\right\}$. Al conjunto $B_{n}$ se le llama {\em cilindro} con
base $B^{n}$, el cilindro es llamado medible si
$B^{n}\in\prod_{i=1}^{\infty}\mathcal{F}_{i}$.
\end{Def}


\begin{Def}\label{Def.Proc.Adaptado}[TSP, Ash \cite{RBA}]
Sea $X\left(t\right),t\geq0$ proceso estoc\'astico, el proceso es
adaptado a la familia de $\sigma$-\'algebras $\mathcal{F}_{t}$,
para $t\geq0$, si para $s<t$ implica que
$\mathcal{F}_{s}\subset\mathcal{F}_{t}$, y $X\left(t\right)$ es
$\mathcal{F}_{t}$-medible para cada $t$. Si no se especifica
$\mathcal{F}_{t}$ entonces se toma $\mathcal{F}_{t}$ como
$\mathcal{F}\left(X\left(s\right),s\leq t\right)$, la m\'as
peque\~na $\sigma$-\'algebra de subconjuntos de $\Omega$ que hace
que cada $X\left(s\right)$, con $s\leq t$ sea Borel medible.
\end{Def}


\begin{Def}\label{Def.Tiempo.Paro}[TSP, Ash \cite{RBA}]
Sea $\left\{\mathcal{F}\left(t\right),t\geq0\right\}$ familia
creciente de sub $\sigma$-\'algebras. es decir,
$\mathcal{F}\left(s\right)\subset\mathcal{F}\left(t\right)$ para
$s\leq t$. Un tiempo de paro para $\mathcal{F}\left(t\right)$ es
una funci\'on $T:\Omega\rightarrow\left[0,\infty\right]$ tal que
$\left\{T\leq t\right\}\in\mathcal{F}\left(t\right)$ para cada
$t\geq0$. Un tiempo de paro para el proceso estoc\'astico
$X\left(t\right),t\geq0$ es un tiempo de paro para las
$\sigma$-\'algebras
$\mathcal{F}\left(t\right)=\mathcal{F}\left(X\left(s\right)\right)$.
\end{Def}

\begin{Def}
Sea $X\left(t\right),t\geq0$ proceso estoc\'astico, con
$\left(S,\chi\right)$ espacio de estados. Se dice que el proceso
es adaptado a $\left\{\mathcal{F}\left(t\right)\right\}$, es
decir, si para cualquier $s,t\in I$, $I$ conjunto de \'indices,
$s<t$, se tiene que
$\mathcal{F}\left(s\right)\subset\mathcal{F}\left(t\right)$ y
$X\left(t\right)$ es $\mathcal{F}\left(t\right)$-medible,
\end{Def}

\begin{Def}
Sea $X\left(t\right),t\geq0$ proceso estoc\'astico, se dice que es
un Proceso de Markov relativo a $\mathcal{F}\left(t\right)$ o que
$\left\{X\left(t\right),\mathcal{F}\left(t\right)\right\}$ es de
Markov si y s\'olo si para cualquier conjunto $B\in\chi$,  y
$s,t\in I$, $s<t$ se cumple que
\begin{equation}\label{Prop.Markov}
P\left\{X\left(t\right)\in
B|\mathcal{F}\left(s\right)\right\}=P\left\{X\left(t\right)\in
B|X\left(s\right)\right\}.
\end{equation}
\end{Def}
\begin{Note}
Si se dice que $\left\{X\left(t\right)\right\}$ es un Proceso de
Markov sin mencionar $\mathcal{F}\left(t\right)$, se asumir\'a que
\begin{eqnarray*}
\mathcal{F}\left(t\right)=\mathcal{F}_{0}\left(t\right)=\mathcal{F}\left(X\left(r\right),r\leq
t\right),
\end{eqnarray*}
entonces la ecuaci\'on (\ref{Prop.Markov}) se puede escribir como
\begin{equation}
P\left\{X\left(t\right)\in B|X\left(r\right),r\leq s\right\} =
P\left\{X\left(t\right)\in B|X\left(s\right)\right\}
\end{equation}
\end{Note}

\begin{Teo}
Sea $\left(X_{n},\mathcal{F}_{n},n=0,1,\ldots,\right\}$ Proceso de
Markov con espacio de estados $\left(S_{0},\chi_{0}\right)$
generado por una distribuici\'on inicial $P_{o}$ y probabilidad de
transici\'on $p_{mn}$, para $m,n=0,1,\ldots,$ $m<n$, que por
notaci\'on se escribir\'a como $p\left(m,n,x,B\right)\rightarrow
p_{mn}\left(x,B\right)$. Sea $S$ tiempo de paro relativo a la
$\sigma$-\'algebra $\mathcal{F}_{n}$. Sea $T$ funci\'on medible,
$T:\Omega\rightarrow\left\{0,1,\ldots,\right\}$. Sup\'ongase que
$T\geq S$, entonces $T$ es tiempo de paro. Si $B\in\chi_{0}$,
entonces
\begin{equation}\label{Prop.Fuerte.Markov}
P\left\{X\left(T\right)\in
B,T<\infty|\mathcal{F}\left(S\right)\right\} =
p\left(S,T,X\left(s\right),B\right)
\end{equation}
en $\left\{T<\infty\right\}$.
\end{Teo}


Sea $K$ conjunto numerable y sea $d:K\rightarrow\nat$ funci\'on.
Para $v\in K$, $M_{v}$ es un conjunto abierto de
$\rea^{d\left(v\right)}$. Entonces \[E=\cup_{v\in
K}M_{v}=\left\{\left(v,\zeta\right):v\in K,\zeta\in
M_{v}\right\}.\]

Sea $\mathcal{E}$ la clase de conjuntos medibles en $E$:
\[\mathcal{E}=\left\{\cup_{v\in K}A_{v}:A_{v}\in \mathcal{M}_{v}\right\}.\]

donde $\mathcal{M}$ son los conjuntos de Borel de $M_{v}$.
Entonces $\left(E,\mathcal{E}\right)$ es un espacio de Borel. El
estado del proceso se denotar\'a por
$\mathbf{x}_{t}=\left(v_{t},\zeta_{t}\right)$. La distribuci\'on
de $\left(\mathbf{x}_{t}\right)$ est\'a determinada por por los
siguientes objetos:

\begin{itemize}
\item[i)] Los campos vectoriales $\left(\mathcal{H}_{v},v\in
K\right)$. \item[ii)] Una funci\'on medible $\lambda:E\rightarrow
\rea_{+}$. \item[iii)] Una medida de transici\'on
$Q:\mathcal{E}\times\left(E\cup\Gamma^{*}\right)\rightarrow\left[0,1\right]$
donde
\begin{equation}
\Gamma^{*}=\cup_{v\in K}\partial^{*}M_{v}.
\end{equation}
y
\begin{equation}
\partial^{*}M_{v}=\left\{z\in\partial M_{v}:\mathbf{\mathbf{\phi}_{v}\left(t,\zeta\right)=\mathbf{z}}\textrm{ para alguna }\left(t,\zeta\right)\in\rea_{+}\times M_{v}\right\}.
\end{equation}
$\partial M_{v}$ denota  la frontera de $M_{v}$.
\end{itemize}

El campo vectorial $\left(\mathcal{H}_{v},v\in K\right)$ se supone
tal que para cada $\mathbf{z}\in M_{v}$ existe una \'unica curva
integral $\mathbf{\phi}_{v}\left(t,\zeta\right)$ que satisface la
ecuaci\'on

\begin{equation}
\frac{d}{dt}f\left(\zeta_{t}\right)=\mathcal{H}f\left(\zeta_{t}\right),
\end{equation}
con $\zeta_{0}=\mathbf{z}$, para cualquier funci\'on suave
$f:\rea^{d}\rightarrow\rea$ y $\mathcal{H}$ denota el operador
diferencial de primer orden, con $\mathcal{H}=\mathcal{H}_{v}$ y
$\zeta_{t}=\mathbf{\phi}\left(t,\mathbf{z}\right)$. Adem\'as se
supone que $\mathcal{H}_{v}$ es conservativo, es decir, las curvas
integrales est\'an definidas para todo $t>0$.

Para $\mathbf{x}=\left(v,\zeta\right)\in E$ se denota
\[t^{*}\mathbf{x}=inf\left\{t>0:\mathbf{\phi}_{v}\left(t,\zeta\right)\in\partial^{*}M_{v}\right\}\]

En lo que respecta a la funci\'on $\lambda$, se supondr\'a que
para cada $\left(v,\zeta\right)\in E$ existe un $\epsilon>0$ tal
que la funci\'on
$s\rightarrow\lambda\left(v,\phi_{v}\left(s,\zeta\right)\right)\in
E$ es integrable para $s\in\left[0,\epsilon\right)$. La medida de
transici\'on $Q\left(A;\mathbf{x}\right)$ es una funci\'on medible
de $\mathbf{x}$ para cada $A\in\mathcal{E}$, definida para
$\mathbf{x}\in E\cup\Gamma^{*}$ y es una medida de probabilidad en
$\left(E,\mathcal{E}\right)$ para cada $\mathbf{x}\in E$.

El movimiento del proceso $\left(\mathbf{x}_{t}\right)$ comenzando
en $\mathbf{x}=\left(n,\mathbf{z}\right)\in E$ se puede construir
de la siguiente manera, def\'inase la funci\'on $F$ por

\begin{equation}
F\left(t\right)=\left\{\begin{array}{ll}\\
exp\left(-\int_{0}^{t}\lambda\left(n,\phi_{n}\left(s,\mathbf{z}\right)\right)ds\right), & t<t^{*}\left(\mathbf{x}\right),\\
0, & t\geq t^{*}\left(\mathbf{x}\right)
\end{array}\right.
\end{equation}

Sea $T_{1}$ una variable aleatoria tal que
$\prob\left[T_{1}>t\right]=F\left(t\right)$, ahora sea la variable
aleatoria $\left(N,Z\right)$ con distribuici\'on
$Q\left(\cdot;\phi_{n}\left(T_{1},\mathbf{z}\right)\right)$. La
trayectoria de $\left(\mathbf{x}_{t}\right)$ para $t\leq T_{1}$
es\footnote{Revisar p\'agina 362, y 364 de Davis \cite{Davis}.}
\begin{eqnarray*}
\mathbf{x}_{t}=\left(v_{t},\zeta_{t}\right)=\left\{\begin{array}{ll}
\left(n,\phi_{n}\left(t,\mathbf{z}\right)\right), & t<T_{1},\\
\left(N,\mathbf{Z}\right), & t=t_{1}.
\end{array}\right.
\end{eqnarray*}

Comenzando en $\mathbf{x}_{T_{1}}$ se selecciona el siguiente
tiempo de intersalto $T_{2}-T_{1}$ lugar del post-salto
$\mathbf{x}_{T_{2}}$ de manera similar y as\'i sucesivamente. Este
procedimiento nos da una trayectoria determinista por partes
$\mathbf{x}_{t}$ con tiempos de salto $T_{1},T_{2},\ldots$. Bajo
las condiciones enunciadas para $\lambda,T_{1}>0$  y
$T_{1}-T_{2}>0$ para cada $i$, con probabilidad 1. Se supone que
se cumple la siquiente condici\'on.

\begin{Sup}[Supuesto 3.1, Davis \cite{Davis}]\label{Sup3.1.Davis}
Sea $N_{t}:=\sum_{t}\indora_{\left(t\geq t\right)}$ el n\'umero de
saltos en $\left[0,t\right]$. Entonces
\begin{equation}
\esp\left[N_{t}\right]<\infty\textrm{ para toda }t.
\end{equation}
\end{Sup}

es un proceso de Markov, m\'as a\'un, es un Proceso Fuerte de
Markov, es decir, la Propiedad Fuerte de Markov se cumple para
cualquier tiempo de paro.
%_________________________________________________________________________

En esta secci\'on se har\'an las siguientes consideraciones: $E$
es un espacio m\'etrico separable y la m\'etrica $d$ es compatible
con la topolog\'ia.


\begin{Def}
Un espacio topol\'ogico $E$ es llamado {\em Luisin} si es
homeomorfo a un subconjunto de Borel de un espacio m\'etrico
compacto.
\end{Def}

\begin{Def}
Un espacio topol\'ogico $E$ es llamado de {\em Rad\'on} si es
homeomorfo a un subconjunto universalmente medible de un espacio
m\'etrico compacto.
\end{Def}

Equivalentemente, la definici\'on de un espacio de Rad\'on puede
encontrarse en los siguientes t\'erminos:


\begin{Def}
$E$ es un espacio de Rad\'on si cada medida finita en
$\left(E,\mathcal{B}\left(E\right)\right)$ es regular interior o cerrada,
{\em tight}.
\end{Def}

\begin{Def}
Una medida finita, $\lambda$ en la $\sigma$-\'algebra de Borel de
un espacio metrizable $E$ se dice cerrada si
\begin{equation}\label{Eq.A2.3}
\lambda\left(E\right)=sup\left\{\lambda\left(K\right):K\textrm{ es
compacto en }E\right\}.
\end{equation}
\end{Def}

El siguiente teorema nos permite tener una mejor caracterizaci\'on de los espacios de Rad\'on:
\begin{Teo}\label{Tma.A2.2}
Sea $E$ espacio separable metrizable. Entonces $E$ es Radoniano si y s\'olo s\'i cada medida finita en $\left(E,\mathcal{B}\left(E\right)\right)$ es cerrada.
\end{Teo}

%_________________________________________________________________________________________
\subsection{Propiedades de Markov}
%_________________________________________________________________________________________

Sea $E$ espacio de estados, tal que $E$ es un espacio de Rad\'on, $\mathcal{B}\left(E\right)$ $\sigma$-\'algebra de Borel en $E$, que se denotar\'a por $\mathcal{E}$.

Sea $\left(X,\mathcal{G},\prob\right)$ espacio de probabilidad, $I\subset\rea$ conjunto de índices. Sea $\mathcal{F}_{\leq t}$ la $\sigma$-\'algebra natural definida como $\sigma\left\{f\left(X_{r}\right):r\in I, rleq t,f\in\mathcal{E}\right\}$. Se considerar\'a una $\sigma$-\'algebra m\'as general, $ \left(\mathcal{G}_{t}\right)$ tal que $\left(X_{t}\right)$ sea $\mathcal{E}$-adaptado.

\begin{Def}
Una familia $\left(P_{s,t}\right)$ de kernels de Markov en $\left(E,\mathcal{E}\right)$ indexada por pares $s,t\in I$, con $s\leq t$ es una funci\'on de transici\'on en $\ER$, si  para todo $r\leq s< t$ en $I$ y todo $x\in E$, $B\in\mathcal{E}$
\begin{equation}\label{Eq.Kernels}
P_{r,t}\left(x,B\right)=\int_{E}P_{r,s}\left(x,dy\right)P_{s,t}\left(y,B\right)\footnote{Ecuaci\'on de Chapman-Kolmogorov}.
\end{equation}
\end{Def}

Se dice que la funci\'on de transici\'on $\KM$ en $\ER$ es la funci\'on de transici\'on para un proceso $\PE$  con valores en $E$ y que satisface la propiedad de Markov\footnote{\begin{equation}\label{Eq.1.4.S}
\prob\left\{H|\mathcal{G}_{t}\right\}=\prob\left\{H|X_{t}\right\}\textrm{ }H\in p\mathcal{F}_{\geq t}.
\end{equation}} (\ref{Eq.1.4.S}) relativa a $\left(\mathcal{G}_{t}\right)$ si 

\begin{equation}\label{Eq.1.6.S}
\prob\left\{f\left(X_{t}\right)|\mathcal{G}_{s}\right\}=P_{s,t}f\left(X_{t}\right)\textrm{ }s\leq t\in I,\textrm{ }f\in b\mathcal{E}.
\end{equation}

\begin{Def}
Una familia $\left(P_{t}\right)_{t\geq0}$ de kernels de Markov en $\ER$ es llamada {\em Semigrupo de Transici\'on de Markov} o {\em Semigrupo de Transici\'on} si
\[P_{t+s}f\left(x\right)=P_{t}\left(P_{s}f\right)\left(x\right),\textrm{ }t,s\geq0,\textrm{ }x\in E\textrm{ }f\in b\mathcal{E}.\]
\end{Def}
\begin{Note}
Si la funci\'on de transici\'on $\KM$ es llamada homog\'enea si $P_{s,t}=P_{t-s}$.
\end{Note}

Un proceso de Markov que satisface la ecuaci\'on (\ref{Eq.1.6.S}) con funci\'on de transici\'on homog\'enea $\left(P_{t}\right)$ tiene la propiedad caracter\'istica
\begin{equation}\label{Eq.1.8.S}
\prob\left\{f\left(X_{t+s}\right)|\mathcal{G}_{t}\right\}=P_{s}f\left(X_{t}\right)\textrm{ }t,s\geq0,\textrm{ }f\in b\mathcal{E}.
\end{equation}
La ecuaci\'on anterior es la {\em Propiedad Simple de Markov} de $X$ relativa a $\left(P_{t}\right)$.

En este sentido el proceso $\PE$ cumple con la propiedad de Markov (\ref{Eq.1.8.S}) relativa a $\left(\Omega,\mathcal{G},\mathcal{G}_{t},\prob\right)$ con semigrupo de transici\'on $\left(P_{t}\right)$.
%_________________________________________________________________________________________
\subsection{Primer Condici\'on de Regularidad}
%_________________________________________________________________________________________
%\newcommand{\EM}{\left(\Omega,\mathcal{G},\prob\right)}
%\newcommand{\E4}{\left(\Omega,\mathcal{G},\mathcal{G}_{t},\prob\right)}
\begin{Def}
Un proceso estoc\'astico $\PE$ definido en $\left(\Omega,\mathcal{G},\prob\right)$ con valores en el espacio topol\'ogico $E$ es continuo por la derecha si cada trayectoria muestral $t\rightarrow X_{t}\left(w\right)$ es un mapeo continuo por la derecha de $I$ en $E$.
\end{Def}

\begin{Def}[HD1]\label{Eq.2.1.S}
Un semigrupo de Markov $\left/P_{t}\right)$ en un espacio de Rad\'on $E$ se dice que satisface la condici\'on {\em HD1} si, dada una medida de probabilidad $\mu$ en $E$, existe una $\sigma$-\'algebra $\mathcal{E^{*}}$ con $\mathcal{E}\subset\mathcal{E}$ y $P_{t}\left(b\mathcal{E}^{*}\right)\subset b\mathcal{E}^{*}$, y un $\mathcal{E}^{*}$-proceso $E$-valuado continuo por la derecha $\PE$ en alg\'un espacio de probabilidad filtrado $\left(\Omega,\mathcal{G},\mathcal{G}_{t},\prob\right)$ tal que $X=\left(\Omega,\mathcal{G},\mathcal{G}_{t},\prob\right)$ es de Markov (Homog\'eneo) con semigrupo de transici\'on $(P_{t})$ y distribuci\'on inicial $\mu$.
\end{Def}

Considerese la colecci\'on de variables aleatorias $X_{t}$ definidas en alg\'un espacio de probabilidad, y una colecci\'on de medidas $\mathbf{P}^{x}$ tales que $\mathbf{P}^{x}\left\{X_{0}=x\right\}$, y bajo cualquier $\mathbf{P}^{x}$, $X_{t}$ es de Markov con semigrupo $\left(P_{t}\right)$. $\mathbf{P}^{x}$ puede considerarse como la distribuci\'on condicional de $\mathbf{P}$ dado $X_{0}=x$.

\begin{Def}\label{Def.2.2.S}
Sea $E$ espacio de Rad\'on, $\SG$ semigrupo de Markov en $\ER$. La colecci\'on $\mathbf{X}=\left(\Omega,\mathcal{G},\mathcal{G}_{t},X_{t},\theta_{t},\CM\right)$ es un proceso $\mathcal{E}$-Markov continuo por la derecha simple, con espacio de estados $E$ y semigrupo de transici\'on $\SG$ en caso de que $\mathbf{X}$ satisfaga las siguientes condiciones:
\begin{itemize}
\item[i)] $\left(\Omega,\mathcal{G},\mathcal{G}_{t}\right)$ es un espacio de medida filtrado, y $X_{t}$ es un proceso $E$-valuado continuo por la derecha $\mathcal{E}^{*}$-adaptado a $\left(\mathcal{G}_{t}\right)$;

\item[ii)] $\left(\theta_{t}\right)_{t\geq0}$ es una colecci\'on de operadores {\em shift} para $X$, es decir, mapea $\Omega$ en s\'i mismo satisfaciendo para $t,s\geq0$,

\begin{equation}\label{Eq.Shift}
\theta_{t}\circ\theta_{s}=\theta_{t+s}\textrm{ y }X_{t}\circ\theta_{t}=X_{t+s};
\end{equation}

\item[iii)] Para cualquier $x\in E$,$\CM\left\{X_{0}=x\right\}=1$, y el proceso $\PE$ tiene la propiedad de Markov (\ref{Eq.1.8.S}) con semigrupo de transici\'on $\SG$ relativo a $\left(\Omega,\mathcal{G},\mathcal{G}_{t},\CM\right)$.
\end{itemize}
\end{Def}

\begin{Def}[HD2]\label{Eq.2.2.S}
Para cualquier $\alpha>0$ y cualquier $f\in S^{\alpha}$, el proceso $t\rightarrow f\left(X_{t}\right)$ es continuo por la derecha casi seguramente.
\end{Def}

\begin{Def}\label{Def.PD}
Un sistema $\mathbf{X}=\left(\Omega,\mathcal{G},\mathcal{G}_{t},X_{t},\theta_{t},\CM\right)$ es un proceso derecho en el espacio de Rad\'on $E$ con semigrupo de transici\'on $\SG$ provisto de:
\begin{itemize}
\item[i)] $\mathbf{X}$ es una realizaci\'on  continua por la derecha, \ref{Def.2.2.S}, de $\SG$.

\item[ii)] $\mathbf{X}$ satisface la condicion HD2, \ref{Eq.2.2.S}, relativa a $\mathcal{G}_{t}$.

\item[iii)] $\mathcal{G}_{t}$ es aumentado y continuo por la derecha.
\end{itemize}
\end{Def}




\begin{Lema}[Lema 4.2, Dai\cite{Dai}]\label{Lema4.2}
Sea $\left\{x_{n}\right\}\subset \mathbf{X}$ con
$|x_{n}|\rightarrow\infty$, conforme $n\rightarrow\infty$. Suponga
que
\[lim_{n\rightarrow\infty}\frac{1}{|x_{n}|}U\left(0\right)=\overline{U}\]
y
\[lim_{n\rightarrow\infty}\frac{1}{|x_{n}|}V\left(0\right)=\overline{V}.\]

Entonces, conforme $n\rightarrow\infty$, casi seguramente

\begin{equation}\label{E1.4.2}
\frac{1}{|x_{n}|}\Phi^{k}\left(\left[|x_{n}|t\right]\right)\rightarrow
P_{k}^{'}t\textrm{, u.o.c.,}
\end{equation}

\begin{equation}\label{E1.4.3}
\frac{1}{|x_{n}|}E^{x_{n}}_{k}\left(|x_{n}|t\right)\rightarrow
\alpha_{k}\left(t-\overline{U}_{k}\right)^{+}\textrm{, u.o.c.,}
\end{equation}

\begin{equation}\label{E1.4.4}
\frac{1}{|x_{n}|}S^{x_{n}}_{k}\left(|x_{n}|t\right)\rightarrow
\mu_{k}\left(t-\overline{V}_{k}\right)^{+}\textrm{, u.o.c.,}
\end{equation}

donde $\left[t\right]$ es la parte entera de $t$ y
$\mu_{k}=1/m_{k}=1/\esp\left[\eta_{k}\left(1\right)\right]$.
\end{Lema}

\begin{Lema}[Lema 4.3, Dai\cite{Dai}]\label{Lema.4.3}
Sea $\left\{x_{n}\right\}\subset \mathbf{X}$ con
$|x_{n}|\rightarrow\infty$, conforme $n\rightarrow\infty$. Suponga
que
\[lim_{n\rightarrow\infty}\frac{1}{|x_{n}|}U\left(0\right)=\overline{U}_{k}\]
y
\[lim_{n\rightarrow\infty}\frac{1}{|x_{n}|}V\left(0\right)=\overline{V}_{k}.\]
\begin{itemize}
\item[a)] Conforme $n\rightarrow\infty$ casi seguramente,
\[lim_{n\rightarrow\infty}\frac{1}{|x_{n}|}U^{x_{n}}_{k}\left(|x_{n}|t\right)=\left(\overline{U}_{k}-t\right)^{+}\textrm{, u.o.c.}\]
y
\[lim_{n\rightarrow\infty}\frac{1}{|x_{n}|}V^{x_{n}}_{k}\left(|x_{n}|t\right)=\left(\overline{V}_{k}-t\right)^{+}.\]

\item[b)] Para cada $t\geq0$ fijo,
\[\left\{\frac{1}{|x_{n}|}U^{x_{n}}_{k}\left(|x_{n}|t\right),|x_{n}|\geq1\right\}\]
y
\[\left\{\frac{1}{|x_{n}|}V^{x_{n}}_{k}\left(|x_{n}|t\right),|x_{n}|\geq1\right\}\]
\end{itemize}
son uniformemente convergentes.
\end{Lema}

$S_{l}^{x}\left(t\right)$ es el n\'umero total de servicios
completados de la clase $l$, si la clase $l$ est\'a dando $t$
unidades de tiempo de servicio. Sea $T_{l}^{x}\left(x\right)$ el
monto acumulado del tiempo de servicio que el servidor
$s\left(l\right)$ gasta en los usuarios de la clase $l$ al tiempo
$t$. Entonces $S_{l}^{x}\left(T_{l}^{x}\left(t\right)\right)$ es
el n\'umero total de servicios completados para la clase $l$ al
tiempo $t$. Una fracci\'on de estos usuarios,
$\Phi_{l}^{x}\left(S_{l}^{x}\left(T_{l}^{x}\left(t\right)\right)\right)$,
se convierte en usuarios de la clase $k$.\\

Entonces, dado lo anterior, se tiene la siguiente representaci\'on
para el proceso de la longitud de la cola:\\

\begin{equation}
Q_{k}^{x}\left(t\right)=_{k}^{x}\left(0\right)+E_{k}^{x}\left(t\right)+\sum_{l=1}^{K}\Phi_{k}^{l}\left(S_{l}^{x}\left(T_{l}^{x}\left(t\right)\right)\right)-S_{k}^{x}\left(T_{k}^{x}\left(t\right)\right)
\end{equation}
para $k=1,\ldots,K$. Para $i=1,\ldots,d$, sea
\[I_{i}^{x}\left(t\right)=t-\sum_{j\in C_{i}}T_{k}^{x}\left(t\right).\]

Entonces $I_{i}^{x}\left(t\right)$ es el monto acumulado del
tiempo que el servidor $i$ ha estado desocupado al tiempo $t$. Se
est\'a asumiendo que las disciplinas satisfacen la ley de
conservaci\'on del trabajo, es decir, el servidor $i$ est\'a en
pausa solamente cuando no hay usuarios en la estaci\'on $i$.
Entonces, se tiene que

\begin{equation}
\int_{0}^{\infty}\left(\sum_{k\in
C_{i}}Q_{k}^{x}\left(t\right)\right)dI_{i}^{x}\left(t\right)=0,
\end{equation}
para $i=1,\ldots,d$.\\

Hacer
\[T^{x}\left(t\right)=\left(T_{1}^{x}\left(t\right),\ldots,T_{K}^{x}\left(t\right)\right)^{'},\]
\[I^{x}\left(t\right)=\left(I_{1}^{x}\left(t\right),\ldots,I_{K}^{x}\left(t\right)\right)^{'}\]
y
\[S^{x}\left(T^{x}\left(t\right)\right)=\left(S_{1}^{x}\left(T_{1}^{x}\left(t\right)\right),\ldots,S_{K}^{x}\left(T_{K}^{x}\left(t\right)\right)\right)^{'}.\]

Para una disciplina que cumple con la ley de conservaci\'on del
trabajo, en forma vectorial, se tiene el siguiente conjunto de
ecuaciones

\begin{equation}\label{Eq.MF.1.3}
Q^{x}\left(t\right)=Q^{x}\left(0\right)+E^{x}\left(t\right)+\sum_{l=1}^{K}\Phi^{l}\left(S_{l}^{x}\left(T_{l}^{x}\left(t\right)\right)\right)-S^{x}\left(T^{x}\left(t\right)\right),\\
\end{equation}

\begin{equation}\label{Eq.MF.2.3}
Q^{x}\left(t\right)\geq0,\\
\end{equation}

\begin{equation}\label{Eq.MF.3.3}
T^{x}\left(0\right)=0,\textrm{ y }\overline{T}^{x}\left(t\right)\textrm{ es no decreciente},\\
\end{equation}

\begin{equation}\label{Eq.MF.4.3}
I^{x}\left(t\right)=et-CT^{x}\left(t\right)\textrm{ es no
decreciente}\\
\end{equation}

\begin{equation}\label{Eq.MF.5.3}
\int_{0}^{\infty}\left(CQ^{x}\left(t\right)\right)dI_{i}^{x}\left(t\right)=0,\\
\end{equation}

\begin{equation}\label{Eq.MF.6.3}
\textrm{Condiciones adicionales en
}\left(\overline{Q}^{x}\left(\cdot\right),\overline{T}^{x}\left(\cdot\right)\right)\textrm{
espec\'ificas de la disciplina de la cola,}
\end{equation}

donde $e$ es un vector de unos de dimensi\'on $d$, $C$ es la
matriz definida por
\[C_{ik}=\left\{\begin{array}{cc}
1,& S\left(k\right)=i,\\
0,& \textrm{ en otro caso}.\\
\end{array}\right.
\]
Es necesario enunciar el siguiente Teorema que se utilizar\'a para
el Teorema \ref{Tma.4.2.Dai}:
\begin{Teo}[Teorema 4.1, Dai \cite{Dai}]
Considere una disciplina que cumpla la ley de conservaci\'on del
trabajo, para casi todas las trayectorias muestrales $\omega$ y
cualquier sucesi\'on de estados iniciales
$\left\{x_{n}\right\}\subset \mathbf{X}$, con
$|x_{n}|\rightarrow\infty$, existe una subsucesi\'on
$\left\{x_{n_{j}}\right\}$ con $|x_{n_{j}}|\rightarrow\infty$ tal
que
\begin{equation}\label{Eq.4.15}
\frac{1}{|x_{n_{j}}|}\left(Q^{x_{n_{j}}}\left(0\right),U^{x_{n_{j}}}\left(0\right),V^{x_{n_{j}}}\left(0\right)\right)\rightarrow\left(\overline{Q}\left(0\right),\overline{U},\overline{V}\right),
\end{equation}

\begin{equation}\label{Eq.4.16}
\frac{1}{|x_{n_{j}}|}\left(Q^{x_{n_{j}}}\left(|x_{n_{j}}|t\right),T^{x_{n_{j}}}\left(|x_{n_{j}}|t\right)\right)\rightarrow\left(\overline{Q}\left(t\right),\overline{T}\left(t\right)\right)\textrm{
u.o.c.}
\end{equation}

Adem\'as,
$\left(\overline{Q}\left(t\right),\overline{T}\left(t\right)\right)$
satisface las siguientes ecuaciones:
\begin{equation}\label{Eq.MF.1.3a}
\overline{Q}\left(t\right)=Q\left(0\right)+\left(\alpha
t-\overline{U}\right)^{+}-\left(I-P\right)^{'}M^{-1}\left(\overline{T}\left(t\right)-\overline{V}\right)^{+},
\end{equation}

\begin{equation}\label{Eq.MF.2.3a}
\overline{Q}\left(t\right)\geq0,\\
\end{equation}

\begin{equation}\label{Eq.MF.3.3a}
\overline{T}\left(t\right)\textrm{ es no decreciente y comienza en cero},\\
\end{equation}

\begin{equation}\label{Eq.MF.4.3a}
\overline{I}\left(t\right)=et-C\overline{T}\left(t\right)\textrm{
es no decreciente,}\\
\end{equation}

\begin{equation}\label{Eq.MF.5.3a}
\int_{0}^{\infty}\left(C\overline{Q}\left(t\right)\right)d\overline{I}\left(t\right)=0,\\
\end{equation}

\begin{equation}\label{Eq.MF.6.3a}
\textrm{Condiciones adicionales en
}\left(\overline{Q}\left(\cdot\right),\overline{T}\left(\cdot\right)\right)\textrm{
especficas de la disciplina de la cola,}
\end{equation}
\end{Teo}

\begin{Def}[Definici\'on 4.1, , Dai \cite{Dai}]
Sea una disciplina de servicio espec\'ifica. Cualquier l\'imite
$\left(\overline{Q}\left(\cdot\right),\overline{T}\left(\cdot\right)\right)$
en \ref{Eq.4.16} es un {\em flujo l\'imite} de la disciplina.
Cualquier soluci\'on (\ref{Eq.MF.1.3a})-(\ref{Eq.MF.6.3a}) es
llamado flujo soluci\'on de la disciplina. Se dice que el modelo de flujo l\'imite, modelo de flujo, de la disciplina de la cola es estable si existe una constante
$\delta>0$ que depende de $\mu,\alpha$ y $P$ solamente, tal que
cualquier flujo l\'imite con
$|\overline{Q}\left(0\right)|+|\overline{U}|+|\overline{V}|=1$, se
tiene que $\overline{Q}\left(\cdot+\delta\right)\equiv0$.
\end{Def}

\begin{Teo}[Teorema 4.2, Dai\cite{Dai}]\label{Tma.4.2.Dai}
Sea una disciplina fija para la cola, suponga que se cumplen las
condiciones (1.2)-(1.5). Si el modelo de flujo l\'imite de la
disciplina de la cola es estable, entonces la cadena de Markov $X$
que describe la din\'amica de la red bajo la disciplina es Harris
recurrente positiva.
\end{Teo}

Ahora se procede a escalar el espacio y el tiempo para reducir la
aparente fluctuaci\'on del modelo. Consid\'erese el proceso
\begin{equation}\label{Eq.3.7}
\overline{Q}^{x}\left(t\right)=\frac{1}{|x|}Q^{x}\left(|x|t\right)
\end{equation}
A este proceso se le conoce como el fluido escalado, y cualquier l\'imite $\overline{Q}^{x}\left(t\right)$ es llamado flujo l\'imite del proceso de longitud de la cola. Haciendo $|q|\rightarrow\infty$ mientras se mantiene el resto de las componentes fijas, cualquier punto l\'imite del proceso de longitud de la cola normalizado $\overline{Q}^{x}$ es soluci\'on del siguiente modelo de flujo.

Al conjunto de ecuaciones dadas en \ref{Eq.3.8}-\ref{Eq.3.13} se
le llama {\em Modelo de flujo} y al conjunto de todas las
soluciones del modelo de flujo
$\left(\overline{Q}\left(\cdot\right),\overline{T}
\left(\cdot\right)\right)$ se le denotar\'a por $\mathcal{Q}$.

Si se hace $|x|\rightarrow\infty$ sin restringir ninguna de las
componentes, tambi\'en se obtienen un modelo de flujo, pero en
este caso el residual de los procesos de arribo y servicio
introducen un retraso:

\begin{Def}[Definici\'on 3.3, Dai y Meyn \cite{DaiSean}]
El modelo de flujo es estable si existe un tiempo fijo $t_{0}$ tal
que $\overline{Q}\left(t\right)=0$, con $t\geq t_{0}$, para
cualquier $\overline{Q}\left(\cdot\right)\in\mathcal{Q}$ que
cumple con $|\overline{Q}\left(0\right)|=1$.
\end{Def}

El siguiente resultado se encuentra en Chen \cite{Chen}.
\begin{Lemma}[Lema 3.1, Dai y Meyn \cite{DaiSean}]
Si el modelo de flujo definido por \ref{Eq.3.8}-\ref{Eq.3.13} es
estable, entonces el modelo de flujo retrasado es tambi\'en
estable, es decir, existe $t_{0}>0$ tal que
$\overline{Q}\left(t\right)=0$ para cualquier $t\geq t_{0}$, para
cualquier soluci\'on del modelo de flujo retrasado cuya
condici\'on inicial $\overline{x}$ satisface que
$|\overline{x}|=|\overline{Q}\left(0\right)|+|\overline{A}\left(0\right)|+|\overline{B}\left(0\right)|\leq1$.
\end{Lemma}


Propiedades importantes para el modelo de flujo retrasado:

\begin{Prop}
 Sea $\left(\overline{Q},\overline{T},\overline{T}^{0}\right)$ un flujo l\'imite de \ref{Eq.4.4} y suponga que cuando $x\rightarrow\infty$ a lo largo de
una subsucesi\'on
\[\left(\frac{1}{|x|}Q_{k}^{x}\left(0\right),\frac{1}{|x|}A_{k}^{x}\left(0\right),\frac{1}{|x|}B_{k}^{x}\left(0\right),\frac{1}{|x|}B_{k}^{x,0}\left(0\right)\right)\rightarrow\left(\overline{Q}_{k}\left(0\right),0,0,0\right)\]
para $k=1,\ldots,K$. EL flujo l\'imite tiene las siguientes
propiedades, donde las propiedades de la derivada se cumplen donde
la derivada exista:
\begin{itemize}
 \item[i)] Los vectores de tiempo ocupado $\overline{T}\left(t\right)$ y $\overline{T}^{0}\left(t\right)$ son crecientes y continuas con
$\overline{T}\left(0\right)=\overline{T}^{0}\left(0\right)=0$.
\item[ii)] Para todo $t\geq0$
\[\sum_{k=1}^{K}\left[\overline{T}_{k}\left(t\right)+\overline{T}_{k}^{0}\left(t\right)\right]=t\]
\item[iii)] Para todo $1\leq k\leq K$
\[\overline{Q}_{k}\left(t\right)=\overline{Q}_{k}\left(0\right)+\alpha_{k}t-\mu_{k}\overline{T}_{k}\left(t\right)\]
\item[iv)]  Para todo $1\leq k\leq K$
\[\dot{{\overline{T}}}_{k}\left(t\right)=\beta_{k}\] para $\overline{Q}_{k}\left(t\right)=0$.
\item[v)] Para todo $k,j$
\[\mu_{k}^{0}\overline{T}_{k}^{0}\left(t\right)=\mu_{j}^{0}\overline{T}_{j}^{0}\left(t\right)\]
\item[vi)]  Para todo $1\leq k\leq K$
\[\mu_{k}\dot{{\overline{T}}}_{k}\left(t\right)=l_{k}\mu_{k}^{0}\dot{{\overline{T}}}_{k}^{0}\left(t\right)\] para $\overline{Q}_{k}\left(t\right)>0$.
\end{itemize}
\end{Prop}

\begin{Lema}[Lema 3.1 \cite{Chen}]\label{Lema3.1}
Si el modelo de flujo es estable, definido por las ecuaciones
(3.8)-(3.13), entonces el modelo de flujo retrasado tambin es
estable.
\end{Lema}

\begin{Teo}[Teorema 5.2 \cite{Chen}]\label{Tma.5.2}
Si el modelo de flujo lineal correspondiente a la red de cola es
estable, entonces la red de colas es estable.
\end{Teo}

\begin{Teo}[Teorema 5.1 \cite{Chen}]\label{Tma.5.1.Chen}
La red de colas es estable si existe una constante $t_{0}$ que
depende de $\left(\alpha,\mu,T,U\right)$ y $V$ que satisfagan las
ecuaciones (5.1)-(5.5), $Z\left(t\right)=0$, para toda $t\geq
t_{0}$.
\end{Teo}



\begin{Lema}[Lema 5.2 \cite{Gut}]\label{Lema.5.2.Gut}
Sea $\left\{\xi\left(k\right):k\in\ent\right\}$ sucesin de
variables aleatorias i.i.d. con valores en
$\left(0,\infty\right)$, y sea $E\left(t\right)$ el proceso de
conteo
\[E\left(t\right)=max\left\{n\geq1:\xi\left(1\right)+\cdots+\xi\left(n-1\right)\leq t\right\}.\]
Si $E\left[\xi\left(1\right)\right]<\infty$, entonces para
cualquier entero $r\geq1$
\begin{equation}
lim_{t\rightarrow\infty}\esp\left[\left(\frac{E\left(t\right)}{t}\right)^{r}\right]=\left(\frac{1}{E\left[\xi_{1}\right]}\right)^{r}
\end{equation}
de aqu, bajo estas condiciones
\begin{itemize}
\item[a)] Para cualquier $t>0$,
$sup_{t\geq\delta}\esp\left[\left(\frac{E\left(t\right)}{t}\right)^{r}\right]$

\item[b)] Las variables aleatorias
$\left\{\left(\frac{E\left(t\right)}{t}\right)^{r}:t\geq1\right\}$
son uniformemente integrables.
\end{itemize}
\end{Lema}

\begin{Teo}[Teorema 5.1: Ley Fuerte para Procesos de Conteo
\cite{Gut}]\label{Tma.5.1.Gut} Sea
$0<\mu<\esp\left(X_{1}\right]\leq\infty$. entonces

\begin{itemize}
\item[a)] $\frac{N\left(t\right)}{t}\rightarrow\frac{1}{\mu}$
a.s., cuando $t\rightarrow\infty$.


\item[b)]$\esp\left[\frac{N\left(t\right)}{t}\right]^{r}\rightarrow\frac{1}{\mu^{r}}$,
cuando $t\rightarrow\infty$ para todo $r>0$..
\end{itemize}
\end{Teo}


\begin{Prop}[Proposicin 5.1 \cite{DaiSean}]\label{Prop.5.1}
Suponga que los supuestos (A1) y (A2) se cumplen, adems suponga
que el modelo de flujo es estable. Entonces existe $t_{0}>0$ tal
que
\begin{equation}\label{Eq.Prop.5.1}
lim_{|x|\rightarrow\infty}\frac{1}{|x|^{p+1}}\esp_{x}\left[|X\left(t_{0}|x|\right)|^{p+1}\right]=0.
\end{equation}

\end{Prop}


\begin{Prop}[Proposici\'on 5.3 \cite{DaiSean}]
Sea $X$ proceso de estados para la red de colas, y suponga que se
cumplen los supuestos (A1) y (A2), entonces para alguna constante
positiva $C_{p+1}<\infty$, $\delta>0$ y un conjunto compacto
$C\subset X$.

\begin{equation}\label{Eq.5.4}
\esp_{x}\left[\int_{0}^{\tau_{C}\left(\delta\right)}\left(1+|X\left(t\right)|^{p}\right)dt\right]\leq
C_{p+1}\left(1+|x|^{p+1}\right)
\end{equation}
\end{Prop}

\begin{Prop}[Proposici\'on 5.4 \cite{DaiSean}]
Sea $X$ un proceso de Markov Borel Derecho en $X$, sea
$f:X\leftarrow\rea_{+}$ y defina para alguna $\delta>0$, y un
conjunto cerrado $C\subset X$
\[V\left(x\right):=\esp_{x}\left[\int_{0}^{\tau_{C}\left(\delta\right)}f\left(X\left(t\right)\right)dt\right]\]
para $x\in X$. Si $V$ es finito en todas partes y uniformemente
acotada en $C$, entonces existe $k<\infty$ tal que
\begin{equation}\label{Eq.5.11}
\frac{1}{t}\esp_{x}\left[V\left(x\right)\right]+\frac{1}{t}\int_{0}^{t}\esp_{x}\left[f\left(X\left(s\right)\right)ds\right]\leq\frac{1}{t}V\left(x\right)+k,
\end{equation}
para $x\in X$ y $t>0$.
\end{Prop}


\begin{Teo}[Teorema 5.5 \cite{DaiSean}]
Suponga que se cumplen (A1) y (A2), adems suponga que el modelo
de flujo es estable. Entonces existe una constante $k_{p}<\infty$
tal que
\begin{equation}\label{Eq.5.13}
\frac{1}{t}\int_{0}^{t}\esp_{x}\left[|Q\left(s\right)|^{p}\right]ds\leq
k_{p}\left\{\frac{1}{t}|x|^{p+1}+1\right\}
\end{equation}
para $t\geq0$, $x\in X$. En particular para cada condici\'on inicial
\begin{equation}\label{Eq.5.14}
Limsup_{t\rightarrow\infty}\frac{1}{t}\int_{0}^{t}\esp_{x}\left[|Q\left(s\right)|^{p}\right]ds\leq
k_{p}
\end{equation}
\end{Teo}

\begin{Teo}[Teorema 6.2\cite{DaiSean}]\label{Tma.6.2}
Suponga que se cumplen los supuestos (A1)-(A3) y que el modelo de
flujo es estable, entonces se tiene que
\[\parallel P^{t}\left(c,\cdot\right)-\pi\left(\cdot\right)\parallel_{f_{p}}\rightarrow0\]
para $t\rightarrow\infty$ y $x\in X$. En particular para cada
condicin inicial
\[lim_{t\rightarrow\infty}\esp_{x}\left[\left|Q_{t}\right|^{p}\right]=\esp_{\pi}\left[\left|Q_{0}\right|^{p}\right]<\infty\]
\end{Teo}


\begin{Teo}[Teorema 6.3\cite{DaiSean}]\label{Tma.6.3}
Suponga que se cumplen los supuestos (A1)-(A3) y que el modelo de
flujo es estable, entonces con
$f\left(x\right)=f_{1}\left(x\right)$, se tiene que
\[lim_{t\rightarrow\infty}t^{(p-1)\left|P^{t}\left(c,\cdot\right)-\pi\left(\cdot\right)\right|_{f}=0},\]
para $x\in X$. En particular, para cada condicin inicial
\[lim_{t\rightarrow\infty}t^{(p-1)\left|\esp_{x}\left[Q_{t}\right]-\esp_{\pi}\left[Q_{0}\right]\right|=0}.\]
\end{Teo}


\begin{Prop}[Proposici\'on 5.1, Dai y Meyn \cite{DaiSean}]\label{Prop.5.1.DaiSean}
Suponga que los supuestos A1) y A2) son ciertos y que el modelo de flujo es estable. Entonces existe $t_{0}>0$ tal que
\begin{equation}
lim_{|x|\rightarrow\infty}\frac{1}{|x|^{p+1}}\esp_{x}\left[|X\left(t_{0}|x|\right)|^{p+1}\right]=0
\end{equation}
\end{Prop}

\begin{Lemma}[Lema 5.2, Dai y Meyn \cite{DaiSean}]\label{Lema.5.2.DaiSean}
 Sea $\left\{\zeta\left(k\right):k\in \mathbb{z}\right\}$ una sucesi\'on independiente e id\'enticamente distribuida que toma valores en $\left(0,\infty\right)$,
y sea
$E\left(t\right)=max\left(n\geq1:\zeta\left(1\right)+\cdots+\zeta\left(n-1\right)\leq
t\right)$. Si $\esp\left[\zeta\left(1\right)\right]<\infty$,
entonces para cualquier entero $r\geq1$
\begin{equation}
 lim_{t\rightarrow\infty}\esp\left[\left(\frac{E\left(t\right)}{t}\right)^{r}\right]=\left(\frac{1}{\esp\left[\zeta_{1}\right]}\right)^{r}.
\end{equation}
Luego, bajo estas condiciones:
\begin{itemize}
 \item[a)] para cualquier $\delta>0$, $\sup_{t\geq\delta}\esp\left[\left(\frac{E\left(t\right)}{t}\right)^{r}\right]<\infty$
\item[b)] las variables aleatorias
$\left\{\left(\frac{E\left(t\right)}{t}\right)^{r}:t\geq1\right\}$
son uniformemente integrables.
\end{itemize}
\end{Lemma}

\begin{Teo}[Teorema 5.5, Dai y Meyn \cite{DaiSean}]\label{Tma.5.5.DaiSean}
Suponga que los supuestos A1) y A2) se cumplen y que el modelo de
flujo es estable. Entonces existe una constante $\kappa_{p}$ tal
que
\begin{equation}
\frac{1}{t}\int_{0}^{t}\esp_{x}\left[|Q\left(s\right)|^{p}\right]ds\leq\kappa_{p}\left\{\frac{1}{t}|x|^{p+1}+1\right\}
\end{equation}
para $t>0$ y $x\in X$. En particular, para cada condici\'on
inicial
\begin{eqnarray*}
\limsup_{t\rightarrow\infty}\frac{1}{t}\int_{0}^{t}\esp_{x}\left[|Q\left(s\right)|^{p}\right]ds\leq\kappa_{p}.
\end{eqnarray*}
\end{Teo}

\begin{Teo}[Teorema 6.2, Dai y Meyn \cite{DaiSean}]\label{Tma.6.2.DaiSean}
Suponga que se cumplen los supuestos A1), A2) y A3) y que el
modelo de flujo es estable. Entonces se tiene que
\begin{equation}
\left\|P^{t}\left(x,\cdot\right)-\pi\left(\cdot\right)\right\|_{f_{p}}\textrm{,
}t\rightarrow\infty,x\in X.
\end{equation}
En particular para cada condici\'on inicial
\begin{eqnarray*}
\lim_{t\rightarrow\infty}\esp_{x}\left[|Q\left(t\right)|^{p}\right]=\esp_{\pi}\left[|Q\left(0\right)|^{p}\right]\leq\kappa_{r}
\end{eqnarray*}
\end{Teo}
\begin{Teo}[Teorema 6.3, Dai y Meyn \cite{DaiSean}]\label{Tma.6.3.DaiSean}
Suponga que se cumplen los supuestos A1), A2) y A3) y que el
modelo de flujo es estable. Entonces con
$f\left(x\right)=f_{1}\left(x\right)$ se tiene
\begin{equation}
\lim_{t\rightarrow\infty}t^{p-1}\left\|P^{t}\left(x,\cdot\right)-\pi\left(\cdot\right)\right\|_{f}=0.
\end{equation}
En particular para cada condici\'on inicial
\begin{eqnarray*}
\lim_{t\rightarrow\infty}t^{p-1}|\esp_{x}\left[Q\left(t\right)\right]-\esp_{\pi}\left[Q\left(0\right)\right]|=0.
\end{eqnarray*}
\end{Teo}

\begin{Teo}[Teorema 6.4, Dai y Meyn \cite{DaiSean}]\label{Tma.6.4.DaiSean}
Suponga que se cumplen los supuestos A1), A2) y A3) y que el
modelo de flujo es estable. Sea $\nu$ cualquier distribuci\'on de
probabilidad en $\left(X,\mathcal{B}_{X}\right)$, y $\pi$ la
distribuci\'on estacionaria de $X$.
\begin{itemize}
\item[i)] Para cualquier $f:X\leftarrow\rea_{+}$
\begin{equation}
\lim_{t\rightarrow\infty}\frac{1}{t}\int_{o}^{t}f\left(X\left(s\right)\right)ds=\pi\left(f\right):=\int
f\left(x\right)\pi\left(dx\right)
\end{equation}
$\prob$-c.s.

\item[ii)] Para cualquier $f:X\leftarrow\rea_{+}$ con
$\pi\left(|f|\right)<\infty$, la ecuaci\'on anterior se cumple.
\end{itemize}
\end{Teo}

\begin{Teo}[Teorema 2.2, Down \cite{Down}]\label{Tma2.2.Down}
Suponga que el fluido modelo es inestable en el sentido de que
para alguna $\epsilon_{0},c_{0}\geq0$,
\begin{equation}\label{Eq.Inestability}
|Q\left(T\right)|\geq\epsilon_{0}T-c_{0}\textrm{,   }T\geq0,
\end{equation}
para cualquier condici\'on inicial $Q\left(0\right)$, con
$|Q\left(0\right)|=1$. Entonces para cualquier $0<q\leq1$, existe
$B<0$ tal que para cualquier $|x|\geq B$,
\begin{equation}
\prob_{x}\left\{\mathbb{X}\rightarrow\infty\right\}\geq q.
\end{equation}
\end{Teo}



Es necesario hacer los siguientes supuestos sobre el
comportamiento del sistema de visitas c\'iclicas:
\begin{itemize}
\item Los tiempos de interarribo a la $k$-\'esima cola, son de la
forma $\left\{\xi_{k}\left(n\right)\right\}_{n\geq1}$, con la
propiedad de que son independientes e id{\'e}nticamente
distribuidos,
\item Los tiempos de servicio
$\left\{\eta_{k}\left(n\right)\right\}_{n\geq1}$ tienen la
propiedad de ser independientes e id{\'e}nticamente distribuidos,
\item Se define la tasa de arribo a la $k$-{\'e}sima cola como
$\lambda_{k}=1/\esp\left[\xi_{k}\left(1\right)\right]$,
\item la tasa de servicio para la $k$-{\'e}sima cola se define
como $\mu_{k}=1/\esp\left[\eta_{k}\left(1\right)\right]$,
\item tambi{\'e}n se define $\rho_{k}:=\lambda_{k}/\mu_{k}$, la
intensidad de tr\'afico del sistema o carga de la red, donde es
necesario que $\rho<1$ para cuestiones de estabilidad.
\end{itemize}



%_________________________________________________________________________
\subsection{Procesos de Estados Markoviano para el Sistema}
%_________________________________________________________________________

%_________________________________________________________________________
\subsection{Procesos Fuerte de Markov}
%_________________________________________________________________________
En Dai \cite{Dai} se muestra que para una amplia serie de disciplinas
de servicio el proceso $X$ es un Proceso Fuerte de
Markov, y por tanto se puede asumir que


Para establecer que $X=\left\{X\left(t\right),t\geq0\right\}$ es
un Proceso Fuerte de Markov, se siguen las secciones 2.3 y 2.4 de Kaspi and Mandelbaum \cite{KaspiMandelbaum}. \\

%______________________________________________________________
\subsubsection{Construcci\'on de un Proceso Determinista por partes, Davis
\cite{Davis}}.
%______________________________________________________________

%_________________________________________________________________________
\subsection{Procesos Harris Recurrentes Positivos}
%_________________________________________________________________________
Sea el proceso de Markov $X=\left\{X\left(t\right),t\geq0\right\}$
que describe la din\'amica de la red de colas. En lo que respecta
al supuesto (A3), en Dai y Meyn \cite{DaiSean} y Meyn y Down
\cite{MeynDown} hacen ver que este se puede sustituir por

\begin{itemize}
\item[A3')] Para el Proceso de Markov $X$, cada subconjunto
compacto de $X$ es un conjunto peque\~no.
\end{itemize}

Este supuesto es importante pues es un requisito para deducir la ergodicidad de la red.

%_________________________________________________________________________
\subsection{Construcci\'on de un Modelo de Flujo L\'imite}
%_________________________________________________________________________

Consideremos un caso m\'as simple para poner en contexto lo
anterior: para un sistema de visitas c\'iclicas se tiene que el
estado al tiempo $t$ es
\begin{equation}
X\left(t\right)=\left(Q\left(t\right),U\left(t\right),V\left(t\right)\right),
\end{equation}

donde $Q\left(t\right)$ es el n\'umero de usuarios formados en
cada estaci\'on. $U\left(t\right)$ es el tiempo restante antes de
que la siguiente clase $k$ de usuarios lleguen desde fuera del
sistema, $V\left(t\right)$ es el tiempo restante de servicio para
la clase $k$ de usuarios que est\'an siendo atendidos. Tanto
$U\left(t\right)$ como $V\left(t\right)$ se puede asumir que son
continuas por la derecha.

Sea
$x=\left(Q\left(0\right),U\left(0\right),V\left(0\right)\right)=\left(q,a,b\right)$,
el estado inicial de la red bajo una disciplina espec\'ifica para
la cola. Para $l\in\mathcal{E}$, donde $\mathcal{E}$ es el conjunto de clases de arribos externos, y $k=1,\ldots,K$ se define\\
\begin{eqnarray*}
E_{l}^{x}\left(t\right)&=&max\left\{r:U_{l}\left(0\right)+\xi_{l}\left(1\right)+\cdots+\xi_{l}\left(r-1\right)\leq
t\right\}\textrm{   }t\geq0,\\
S_{k}^{x}\left(t\right)&=&max\left\{r:V_{k}\left(0\right)+\eta_{k}\left(1\right)+\cdots+\eta_{k}\left(r-1\right)\leq
t\right\}\textrm{   }t\geq0.
\end{eqnarray*}

Para cada $k$ y cada $n$ se define

\begin{eqnarray*}\label{Eq.phi}
\Phi^{k}\left(n\right):=\sum_{i=1}^{n}\phi^{k}\left(i\right).
\end{eqnarray*}

donde $\phi^{k}\left(n\right)$ se define como el vector de ruta
para el $n$-\'esimo usuario de la clase $k$ que termina en la
estaci\'on $s\left(k\right)$, la $s$-\'eima componente de
$\phi^{k}\left(n\right)$ es uno si estos usuarios se convierten en
usuarios de la clase $l$ y cero en otro caso, por lo tanto
$\phi^{k}\left(n\right)$ es un vector {\em Bernoulli} de
dimensi\'on $K$ con par\'ametro $P_{k}^{'}$, donde $P_{k}$ denota
el $k$-\'esimo rengl\'on de $P=\left(P_{kl}\right)$.

Se asume que cada para cada $k$ la sucesi\'on $\phi^{k}\left(n\right)=\left\{\phi^{k}\left(n\right),n\geq1\right\}$
es independiente e id\'enticamente distribuida y que las
$\phi^{1}\left(n\right),\ldots,\phi^{K}\left(n\right)$ son
mutuamente independientes, adem\'as de independientes de los
procesos de arribo y de servicio.\\

\begin{Lema}[Lema 4.2, Dai\cite{Dai}]\label{Lema4.2}
Sea $\left\{x_{n}\right\}\subset \mathbf{X}$ con
$|x_{n}|\rightarrow\infty$, conforme $n\rightarrow\infty$. Suponga
que
\[lim_{n\rightarrow\infty}\frac{1}{|x_{n}|}U\left(0\right)=\overline{U}\]
y
\[lim_{n\rightarrow\infty}\frac{1}{|x_{n}|}V\left(0\right)=\overline{V}.\]

Entonces, conforme $n\rightarrow\infty$, casi seguramente

\begin{equation}\label{E1.4.2}
\frac{1}{|x_{n}|}\Phi^{k}\left(\left[|x_{n}|t\right]\right)\rightarrow
P_{k}^{'}t\textrm{, u.o.c.,}
\end{equation}

\begin{equation}\label{E1.4.3}
\frac{1}{|x_{n}|}E^{x_{n}}_{k}\left(|x_{n}|t\right)\rightarrow
\alpha_{k}\left(t-\overline{U}_{k}\right)^{+}\textrm{, u.o.c.,}
\end{equation}

\begin{equation}\label{E1.4.4}
\frac{1}{|x_{n}|}S^{x_{n}}_{k}\left(|x_{n}|t\right)\rightarrow
\mu_{k}\left(t-\overline{V}_{k}\right)^{+}\textrm{, u.o.c.,}
\end{equation}

donde $\left[t\right]$ es la parte entera de $t$ y
$\mu_{k}=1/m_{k}=1/\esp\left[\eta_{k}\left(1\right)\right]$.
\end{Lema}

\begin{Lema}[Lema 4.3, Dai\cite{Dai}]\label{Lema.4.3}
Sea $\left\{x_{n}\right\}\subset \mathbf{X}$ con
$|x_{n}|\rightarrow\infty$, conforme $n\rightarrow\infty$. Suponga
que
\[lim_{n\rightarrow\infty}\frac{1}{|x_{n}|}U\left(0\right)=\overline{U}_{k}\]
y
\[lim_{n\rightarrow\infty}\frac{1}{|x_{n}|}V\left(0\right)=\overline{V}_{k}.\]
\begin{itemize}
\item[a)] Conforme $n\rightarrow\infty$ casi seguramente,
\[lim_{n\rightarrow\infty}\frac{1}{|x_{n}|}U^{x_{n}}_{k}\left(|x_{n}|t\right)=\left(\overline{U}_{k}-t\right)^{+}\textrm{, u.o.c.}\]
y
\[lim_{n\rightarrow\infty}\frac{1}{|x_{n}|}V^{x_{n}}_{k}\left(|x_{n}|t\right)=\left(\overline{V}_{k}-t\right)^{+}.\]

\item[b)] Para cada $t\geq0$ fijo,
\[\left\{\frac{1}{|x_{n}|}U^{x_{n}}_{k}\left(|x_{n}|t\right),|x_{n}|\geq1\right\}\]
y
\[\left\{\frac{1}{|x_{n}|}V^{x_{n}}_{k}\left(|x_{n}|t\right),|x_{n}|\geq1\right\}\]
\end{itemize}
son uniformemente convergentes.
\end{Lema}

$S_{l}^{x}\left(t\right)$ es el n\'umero total de servicios
completados de la clase $l$, si la clase $l$ est\'a dando $t$
unidades de tiempo de servicio. Sea $T_{l}^{x}\left(x\right)$ el
monto acumulado del tiempo de servicio que el servidor
$s\left(l\right)$ gasta en los usuarios de la clase $l$ al tiempo
$t$. Entonces $S_{l}^{x}\left(T_{l}^{x}\left(t\right)\right)$ es
el n\'umero total de servicios completados para la clase $l$ al
tiempo $t$. Una fracci\'on de estos usuarios,
$\Phi_{l}^{x}\left(S_{l}^{x}\left(T_{l}^{x}\left(t\right)\right)\right)$,
se convierte en usuarios de la clase $k$.\\

Entonces, dado lo anterior, se tiene la siguiente representaci\'on
para el proceso de la longitud de la cola:\\

\begin{equation}
Q_{k}^{x}\left(t\right)=_{k}^{x}\left(0\right)+E_{k}^{x}\left(t\right)+\sum_{l=1}^{K}\Phi_{k}^{l}\left(S_{l}^{x}\left(T_{l}^{x}\left(t\right)\right)\right)-S_{k}^{x}\left(T_{k}^{x}\left(t\right)\right)
\end{equation}
para $k=1,\ldots,K$. Para $i=1,\ldots,d$, sea
\[I_{i}^{x}\left(t\right)=t-\sum_{j\in C_{i}}T_{k}^{x}\left(t\right).\]

Entonces $I_{i}^{x}\left(t\right)$ es el monto acumulado del
tiempo que el servidor $i$ ha estado desocupado al tiempo $t$. Se
est\'a asumiendo que las disciplinas satisfacen la ley de
conservaci\'on del trabajo, es decir, el servidor $i$ est\'a en
pausa solamente cuando no hay usuarios en la estaci\'on $i$.
Entonces, se tiene que

\begin{equation}
\int_{0}^{\infty}\left(\sum_{k\in
C_{i}}Q_{k}^{x}\left(t\right)\right)dI_{i}^{x}\left(t\right)=0,
\end{equation}
para $i=1,\ldots,d$.\\

Hacer
\[T^{x}\left(t\right)=\left(T_{1}^{x}\left(t\right),\ldots,T_{K}^{x}\left(t\right)\right)^{'},\]
\[I^{x}\left(t\right)=\left(I_{1}^{x}\left(t\right),\ldots,I_{K}^{x}\left(t\right)\right)^{'}\]
y
\[S^{x}\left(T^{x}\left(t\right)\right)=\left(S_{1}^{x}\left(T_{1}^{x}\left(t\right)\right),\ldots,S_{K}^{x}\left(T_{K}^{x}\left(t\right)\right)\right)^{'}.\]

Para una disciplina que cumple con la ley de conservaci\'on del
trabajo, en forma vectorial, se tiene el siguiente conjunto de
ecuaciones

\begin{equation}\label{Eq.MF.1.3}
Q^{x}\left(t\right)=Q^{x}\left(0\right)+E^{x}\left(t\right)+\sum_{l=1}^{K}\Phi^{l}\left(S_{l}^{x}\left(T_{l}^{x}\left(t\right)\right)\right)-S^{x}\left(T^{x}\left(t\right)\right),\\
\end{equation}

\begin{equation}\label{Eq.MF.2.3}
Q^{x}\left(t\right)\geq0,\\
\end{equation}

\begin{equation}\label{Eq.MF.3.3}
T^{x}\left(0\right)=0,\textrm{ y }\overline{T}^{x}\left(t\right)\textrm{ es no decreciente},\\
\end{equation}

\begin{equation}\label{Eq.MF.4.3}
I^{x}\left(t\right)=et-CT^{x}\left(t\right)\textrm{ es no
decreciente}\\
\end{equation}

\begin{equation}\label{Eq.MF.5.3}
\int_{0}^{\infty}\left(CQ^{x}\left(t\right)\right)dI_{i}^{x}\left(t\right)=0,\\
\end{equation}

\begin{equation}\label{Eq.MF.6.3}
\textrm{Condiciones adicionales en
}\left(\overline{Q}^{x}\left(\cdot\right),\overline{T}^{x}\left(\cdot\right)\right)\textrm{
espec\'ificas de la disciplina de la cola,}
\end{equation}

donde $e$ es un vector de unos de dimensi\'on $d$, $C$ es la
matriz definida por
\[C_{ik}=\left\{\begin{array}{cc}
1,& S\left(k\right)=i,\\
0,& \textrm{ en otro caso}.\\
\end{array}\right.
\]
Es necesario enunciar el siguiente Teorema que se utilizar\'a para
el Teorema \ref{Tma.4.2.Dai}:
\begin{Teo}[Teorema 4.1, Dai \cite{Dai}]
Considere una disciplina que cumpla la ley de conservaci\'on del
trabajo, para casi todas las trayectorias muestrales $\omega$ y
cualquier sucesi\'on de estados iniciales
$\left\{x_{n}\right\}\subset \mathbf{X}$, con
$|x_{n}|\rightarrow\infty$, existe una subsucesi\'on
$\left\{x_{n_{j}}\right\}$ con $|x_{n_{j}}|\rightarrow\infty$ tal
que
\begin{equation}\label{Eq.4.15}
\frac{1}{|x_{n_{j}}|}\left(Q^{x_{n_{j}}}\left(0\right),U^{x_{n_{j}}}\left(0\right),V^{x_{n_{j}}}\left(0\right)\right)\rightarrow\left(\overline{Q}\left(0\right),\overline{U},\overline{V}\right),
\end{equation}

\begin{equation}\label{Eq.4.16}
\frac{1}{|x_{n_{j}}|}\left(Q^{x_{n_{j}}}\left(|x_{n_{j}}|t\right),T^{x_{n_{j}}}\left(|x_{n_{j}}|t\right)\right)\rightarrow\left(\overline{Q}\left(t\right),\overline{T}\left(t\right)\right)\textrm{
u.o.c.}
\end{equation}

Adem\'as,
$\left(\overline{Q}\left(t\right),\overline{T}\left(t\right)\right)$
satisface las siguientes ecuaciones:
\begin{equation}\label{Eq.MF.1.3a}
\overline{Q}\left(t\right)=Q\left(0\right)+\left(\alpha
t-\overline{U}\right)^{+}-\left(I-P\right)^{'}M^{-1}\left(\overline{T}\left(t\right)-\overline{V}\right)^{+},
\end{equation}

\begin{equation}\label{Eq.MF.2.3a}
\overline{Q}\left(t\right)\geq0,\\
\end{equation}

\begin{equation}\label{Eq.MF.3.3a}
\overline{T}\left(t\right)\textrm{ es no decreciente y comienza en cero},\\
\end{equation}

\begin{equation}\label{Eq.MF.4.3a}
\overline{I}\left(t\right)=et-C\overline{T}\left(t\right)\textrm{
es no decreciente,}\\
\end{equation}

\begin{equation}\label{Eq.MF.5.3a}
\int_{0}^{\infty}\left(C\overline{Q}\left(t\right)\right)d\overline{I}\left(t\right)=0,\\
\end{equation}

\begin{equation}\label{Eq.MF.6.3a}
\textrm{Condiciones adicionales en
}\left(\overline{Q}\left(\cdot\right),\overline{T}\left(\cdot\right)\right)\textrm{
especficas de la disciplina de la cola,}
\end{equation}
\end{Teo}

\begin{Def}[Definici\'on 4.1, , Dai \cite{Dai}]
Sea una disciplina de servicio espec\'ifica. Cualquier l\'imite
$\left(\overline{Q}\left(\cdot\right),\overline{T}\left(\cdot\right)\right)$
en \ref{Eq.4.16} es un {\em flujo l\'imite} de la disciplina.
Cualquier soluci\'on (\ref{Eq.MF.1.3a})-(\ref{Eq.MF.6.3a}) es
llamado flujo soluci\'on de la disciplina. Se dice que el modelo de flujo l\'imite, modelo de flujo, de la disciplina de la cola es estable si existe una constante
$\delta>0$ que depende de $\mu,\alpha$ y $P$ solamente, tal que
cualquier flujo l\'imite con
$|\overline{Q}\left(0\right)|+|\overline{U}|+|\overline{V}|=1$, se
tiene que $\overline{Q}\left(\cdot+\delta\right)\equiv0$.
\end{Def}

\begin{Teo}[Teorema 4.2, Dai\cite{Dai}]\label{Tma.4.2.Dai}
Sea una disciplina fija para la cola, suponga que se cumplen las
condiciones (1.2)-(1.5). Si el modelo de flujo l\'imite de la
disciplina de la cola es estable, entonces la cadena de Markov $X$
que describe la din\'amica de la red bajo la disciplina es Harris
recurrente positiva.
\end{Teo}

Ahora se procede a escalar el espacio y el tiempo para reducir la
aparente fluctuaci\'on del modelo. Consid\'erese el proceso
\begin{equation}\label{Eq.3.7}
\overline{Q}^{x}\left(t\right)=\frac{1}{|x|}Q^{x}\left(|x|t\right)
\end{equation}
A este proceso se le conoce como el fluido escalado, y cualquier l\'imite $\overline{Q}^{x}\left(t\right)$ es llamado flujo l\'imite del proceso de longitud de la cola. Haciendo $|q|\rightarrow\infty$ mientras se mantiene el resto de las componentes fijas, cualquier punto l\'imite del proceso de longitud de la cola normalizado $\overline{Q}^{x}$ es soluci\'on del siguiente modelo de flujo.

\begin{Def}[Definici\'on 3.1, Dai y Meyn \cite{DaiSean}]
Un flujo l\'imite (retrasado) para una red bajo una disciplina de
servicio espec\'ifica se define como cualquier soluci\'on
 $\left(\overline{Q}\left(\cdot\right),\overline{T}\left(\cdot\right)\right)$ de las siguientes ecuaciones, donde
$\overline{Q}\left(t\right)=\left(\overline{Q}_{1}\left(t\right),\ldots,\overline{Q}_{K}\left(t\right)\right)^{'}$
y
$\overline{T}\left(t\right)=\left(\overline{T}_{1}\left(t\right),\ldots,\overline{T}_{K}\left(t\right)\right)^{'}$
\begin{equation}\label{Eq.3.8}
\overline{Q}_{k}\left(t\right)=\overline{Q}_{k}\left(0\right)+\alpha_{k}t-\mu_{k}\overline{T}_{k}\left(t\right)+\sum_{l=1}^{k}P_{lk}\mu_{l}\overline{T}_{l}\left(t\right)\\
\end{equation}
\begin{equation}\label{Eq.3.9}
\overline{Q}_{k}\left(t\right)\geq0\textrm{ para }k=1,2,\ldots,K,\\
\end{equation}
\begin{equation}\label{Eq.3.10}
\overline{T}_{k}\left(0\right)=0,\textrm{ y }\overline{T}_{k}\left(\cdot\right)\textrm{ es no decreciente},\\
\end{equation}
\begin{equation}\label{Eq.3.11}
\overline{I}_{i}\left(t\right)=t-\sum_{k\in C_{i}}\overline{T}_{k}\left(t\right)\textrm{ es no decreciente}\\
\end{equation}
\begin{equation}\label{Eq.3.12}
\overline{I}_{i}\left(\cdot\right)\textrm{ se incrementa al tiempo }t\textrm{ cuando }\sum_{k\in C_{i}}Q_{k}^{x}\left(t\right)dI_{i}^{x}\left(t\right)=0\\
\end{equation}
\begin{equation}\label{Eq.3.13}
\textrm{condiciones adicionales sobre
}\left(Q^{x}\left(\cdot\right),T^{x}\left(\cdot\right)\right)\textrm{
referentes a la disciplina de servicio}
\end{equation}
\end{Def}

Al conjunto de ecuaciones dadas en \ref{Eq.3.8}-\ref{Eq.3.13} se
le llama {\em Modelo de flujo} y al conjunto de todas las
soluciones del modelo de flujo
$\left(\overline{Q}\left(\cdot\right),\overline{T}
\left(\cdot\right)\right)$ se le denotar\'a por $\mathcal{Q}$.

Si se hace $|x|\rightarrow\infty$ sin restringir ninguna de las
componentes, tambi\'en se obtienen un modelo de flujo, pero en
este caso el residual de los procesos de arribo y servicio
introducen un retraso:

\begin{Def}[Definici\'on 3.2, Dai y Meyn \cite{DaiSean}]
El modelo de flujo retrasado de una disciplina de servicio en una
red con retraso
$\left(\overline{A}\left(0\right),\overline{B}\left(0\right)\right)\in\rea_{+}^{K+|A|}$
se define como el conjunto de ecuaciones dadas en
\ref{Eq.3.8}-\ref{Eq.3.13}, junto con la condici\'on:
\begin{equation}\label{CondAd.FluidModel}
\overline{Q}\left(t\right)=\overline{Q}\left(0\right)+\left(\alpha
t-\overline{A}\left(0\right)\right)^{+}-\left(I-P^{'}\right)M\left(\overline{T}\left(t\right)-\overline{B}\left(0\right)\right)^{+}
\end{equation}
\end{Def}

\begin{Def}[Definici\'on 3.3, Dai y Meyn \cite{DaiSean}]
El modelo de flujo es estable si existe un tiempo fijo $t_{0}$ tal
que $\overline{Q}\left(t\right)=0$, con $t\geq t_{0}$, para
cualquier $\overline{Q}\left(\cdot\right)\in\mathcal{Q}$ que
cumple con $|\overline{Q}\left(0\right)|=1$.
\end{Def}

El siguiente resultado se encuentra en Chen \cite{Chen}.
\begin{Lemma}[Lema 3.1, Dai y Meyn \cite{DaiSean}]
Si el modelo de flujo definido por \ref{Eq.3.8}-\ref{Eq.3.13} es
estable, entonces el modelo de flujo retrasado es tambi\'en
estable, es decir, existe $t_{0}>0$ tal que
$\overline{Q}\left(t\right)=0$ para cualquier $t\geq t_{0}$, para
cualquier soluci\'on del modelo de flujo retrasado cuya
condici\'on inicial $\overline{x}$ satisface que
$|\overline{x}|=|\overline{Q}\left(0\right)|+|\overline{A}\left(0\right)|+|\overline{B}\left(0\right)|\leq1$.
\end{Lemma}

%_________________________________________________________________________
\subsection{Modelo de Visitas C\'iclicas con un Servidor: Estabilidad}
%_________________________________________________________________________

%_________________________________________________________________________
\subsection{Teorema 2.1}
%_________________________________________________________________________



El resultado principal de Down \cite{Down} que relaciona la estabilidad del modelo de flujo con la estabilidad del sistema original

\begin{Teo}[Teorema 2.1, Down \cite{Down}]\label{Tma.2.1.Down}
Suponga que el modelo de flujo es estable, y que se cumplen los supuestos (A1) y (A2), entonces
\begin{itemize}
\item[i)] Para alguna constante $\kappa_{p}$, y para cada
condici\'on inicial $x\in X$
\begin{equation}\label{Estability.Eq1}
lim_{t\rightarrow\infty}\sup\frac{1}{t}\int_{0}^{t}\esp_{x}\left[|Q\left(s\right)|^{p}\right]ds\leq\kappa_{p},
\end{equation}
donde $p$ es el entero dado en (A2). Si adem\'as se cumple
la condici\'on (A3), entonces para cada condici\'on inicial:

\item[ii)] Los momentos transitorios convergen a su estado estacionario:
 \begin{equation}\label{Estability.Eq2}
lim_{t\rightarrow\infty}\esp_{x}\left[Q_{k}\left(t\right)^{r}\right]=\esp_{\pi}\left[Q_{k}\left(0\right)^{r}\right]\leq\kappa_{r},
\end{equation}
para $r=1,2,\ldots,p$ y $k=1,2,\ldots,K$. Donde $\pi$ es la
probabilidad invariante para $\mathbf{X}$.

\item[iii)]  El primer momento converge con raz\'on $t^{p-1}$:
\begin{equation}\label{Estability.Eq3}
lim_{t\rightarrow\infty}t^{p-1}|\esp_{x}\left[Q_{k}\left(t\right)\right]-\esp_{\pi}\left[Q\left(0\right)\right]=0.
\end{equation}

\item[iv)] La {\em Ley Fuerte de los grandes n\'umeros} se cumple:
\begin{equation}\label{Estability.Eq4}
lim_{t\rightarrow\infty}\frac{1}{t}\int_{0}^{t}Q_{k}^{r}\left(s\right)ds=\esp_{\pi}\left[Q_{k}\left(0\right)^{r}\right],\textrm{
}\prob_{x}\textrm{-c.s.}
\end{equation}
para $r=1,2,\ldots,p$ y $k=1,2,\ldots,K$.
\end{itemize}
\end{Teo}


\begin{Prop}[Proposici\'on 5.1, Dai y Meyn \cite{DaiSean}]\label{Prop.5.1.DaiSean}
Suponga que los supuestos A1) y A2) son ciertos y que el modelo de flujo es estable. Entonces existe $t_{0}>0$ tal que
\begin{equation}
lim_{|x|\rightarrow\infty}\frac{1}{|x|^{p+1}}\esp_{x}\left[|X\left(t_{0}|x|\right)|^{p+1}\right]=0
\end{equation}
\end{Prop}

\begin{Lemma}[Lema 5.2, Dai y Meyn \cite{DaiSean}]\label{Lema.5.2.DaiSean}
 Sea $\left\{\zeta\left(k\right):k\in \mathbb{z}\right\}$ una sucesi\'on independiente e id\'enticamente distribuida que toma valores en $\left(0,\infty\right)$,
y sea
$E\left(t\right)=max\left(n\geq1:\zeta\left(1\right)+\cdots+\zeta\left(n-1\right)\leq
t\right)$. Si $\esp\left[\zeta\left(1\right)\right]<\infty$,
entonces para cualquier entero $r\geq1$
\begin{equation}
 lim_{t\rightarrow\infty}\esp\left[\left(\frac{E\left(t\right)}{t}\right)^{r}\right]=\left(\frac{1}{\esp\left[\zeta_{1}\right]}\right)^{r}.
\end{equation}
Luego, bajo estas condiciones:
\begin{itemize}
 \item[a)] para cualquier $\delta>0$, $\sup_{t\geq\delta}\esp\left[\left(\frac{E\left(t\right)}{t}\right)^{r}\right]<\infty$
\item[b)] las variables aleatorias
$\left\{\left(\frac{E\left(t\right)}{t}\right)^{r}:t\geq1\right\}$
son uniformemente integrables.
\end{itemize}
\end{Lemma}

\begin{Teo}[Teorema 5.5, Dai y Meyn \cite{DaiSean}]\label{Tma.5.5.DaiSean}
Suponga que los supuestos A1) y A2) se cumplen y que el modelo de
flujo es estable. Entonces existe una constante $\kappa_{p}$ tal
que
\begin{equation}
\frac{1}{t}\int_{0}^{t}\esp_{x}\left[|Q\left(s\right)|^{p}\right]ds\leq\kappa_{p}\left\{\frac{1}{t}|x|^{p+1}+1\right\}
\end{equation}
para $t>0$ y $x\in X$. En particular, para cada condici\'on
inicial
\begin{eqnarray*}
\limsup_{t\rightarrow\infty}\frac{1}{t}\int_{0}^{t}\esp_{x}\left[|Q\left(s\right)|^{p}\right]ds\leq\kappa_{p}.
\end{eqnarray*}
\end{Teo}

\begin{Teo}[Teorema 6.2, Dai y Meyn \cite{DaiSean}]\label{Tma.6.2.DaiSean}
Suponga que se cumplen los supuestos A1), A2) y A3) y que el
modelo de flujo es estable. Entonces se tiene que
\begin{equation}
\left\|P^{t}\left(x,\cdot\right)-\pi\left(\cdot\right)\right\|_{f_{p}}\textrm{,
}t\rightarrow\infty,x\in X.
\end{equation}
En particular para cada condici\'on inicial
\begin{eqnarray*}
\lim_{t\rightarrow\infty}\esp_{x}\left[|Q\left(t\right)|^{p}\right]=\esp_{\pi}\left[|Q\left(0\right)|^{p}\right]\leq\kappa_{r}
\end{eqnarray*}
\end{Teo}
\begin{Teo}[Teorema 6.3, Dai y Meyn \cite{DaiSean}]\label{Tma.6.3.DaiSean}
Suponga que se cumplen los supuestos A1), A2) y A3) y que el
modelo de flujo es estable. Entonces con
$f\left(x\right)=f_{1}\left(x\right)$ se tiene
\begin{equation}
\lim_{t\rightarrow\infty}t^{p-1}\left\|P^{t}\left(x,\cdot\right)-\pi\left(\cdot\right)\right\|_{f}=0.
\end{equation}
En particular para cada condici\'on inicial
\begin{eqnarray*}
\lim_{t\rightarrow\infty}t^{p-1}|\esp_{x}\left[Q\left(t\right)\right]-\esp_{\pi}\left[Q\left(0\right)\right]|=0.
\end{eqnarray*}
\end{Teo}

\begin{Teo}[Teorema 6.4, Dai y Meyn \cite{DaiSean}]\label{Tma.6.4.DaiSean}
Suponga que se cumplen los supuestos A1), A2) y A3) y que el
modelo de flujo es estable. Sea $\nu$ cualquier distribuci\'on de
probabilidad en $\left(X,\mathcal{B}_{X}\right)$, y $\pi$ la
distribuci\'on estacionaria de $X$.
\begin{itemize}
\item[i)] Para cualquier $f:X\leftarrow\rea_{+}$
\begin{equation}
\lim_{t\rightarrow\infty}\frac{1}{t}\int_{o}^{t}f\left(X\left(s\right)\right)ds=\pi\left(f\right):=\int
f\left(x\right)\pi\left(dx\right)
\end{equation}
$\prob$-c.s.

\item[ii)] Para cualquier $f:X\leftarrow\rea_{+}$ con
$\pi\left(|f|\right)<\infty$, la ecuaci\'on anterior se cumple.
\end{itemize}
\end{Teo}

%_________________________________________________________________________
\subsection{Teorema 2.2}
%_________________________________________________________________________

\begin{Teo}[Teorema 2.2, Down \cite{Down}]\label{Tma2.2.Down}
Suponga que el fluido modelo es inestable en el sentido de que
para alguna $\epsilon_{0},c_{0}\geq0$,
\begin{equation}\label{Eq.Inestability}
|Q\left(T\right)|\geq\epsilon_{0}T-c_{0}\textrm{,   }T\geq0,
\end{equation}
para cualquier condici\'on inicial $Q\left(0\right)$, con
$|Q\left(0\right)|=1$. Entonces para cualquier $0<q\leq1$, existe
$B<0$ tal que para cualquier $|x|\geq B$,
\begin{equation}
\prob_{x}\left\{\mathbb{X}\rightarrow\infty\right\}\geq q.
\end{equation}
\end{Teo}

%_________________________________________________________________________
\subsection{Teorema 2.3}
%_________________________________________________________________________
\begin{Teo}[Teorema 2.3, Down \cite{Down}]\label{Tma2.3.Down}
Considere el siguiente valor:
\begin{equation}\label{Eq.Rho.1serv}
\rho=\sum_{k=1}^{K}\rho_{k}+max_{1\leq j\leq K}\left(\frac{\lambda_{j}}{\sum_{s=1}^{S}p_{js}\overline{N}_{s}}\right)\delta^{*}
\end{equation}
\begin{itemize}
\item[i)] Si $\rho<1$ entonces la red es estable, es decir, se cumple el teorema \ref{Tma.2.1.Down}.

\item[ii)] Si $\rho<1$ entonces la red es inestable, es decir, se cumple el teorema \ref{Tma2.2.Down}
\end{itemize}
\end{Teo}
\newpage
%_____________________________________________________________________
\subsection{Definiciones  B\'asicas}
%_____________________________________________________________________
\begin{Def}
Sea $X$ un conjunto y $\mathcal{F}$ una $\sigma$-\'algebra de
subconjuntos de $X$, la pareja $\left(X,\mathcal{F}\right)$ es
llamado espacio medible. Un subconjunto $A$ de $X$ es llamado
medible, o medible con respecto a $\mathcal{F}$, si
$A\in\mathcal{F}$.
\end{Def}

\begin{Def}
Sea $\left(X,\mathcal{F},\mu\right)$ espacio de medida. Se dice
que la medida $\mu$ es $\sigma$-finita si se puede escribir
$X=\bigcup_{n\geq1}X_{n}$ con $X_{n}\in\mathcal{F}$ y
$\mu\left(X_{n}\right)<\infty$.
\end{Def}

\begin{Def}\label{Cto.Borel}
Sea $X$ el conjunto de los \'umeros reales $\rea$. El \'algebra de
Borel es la $\sigma$-\'algebra $B$ generada por los intervalos
abiertos $\left(a,b\right)\in\rea$. Cualquier conjunto en $B$ es
llamado {\em Conjunto de Borel}.
\end{Def}

\begin{Def}\label{Funcion.Medible}
Una funci\'on $f:X\rightarrow\rea$, es medible si para cualquier
n\'umero real $\alpha$ el conjunto
\[\left\{x\in X:f\left(x\right)>\alpha\right\}\]
pertenece a $X$. Equivalentemente, se dice que $f$ es medible si
\[f^{-1}\left(\left(\alpha,\infty\right)\right)=\left\{x\in X:f\left(x\right)>\alpha\right\}\in\mathcal{F}.\]
\end{Def}


\begin{Def}\label{Def.Cilindros}
Sean $\left(\Omega_{i},\mathcal{F}_{i}\right)$, $i=1,2,\ldots,$
espacios medibles y $\Omega=\prod_{i=1}^{\infty}\Omega_{i}$ el
conjunto de todas las sucesiones
$\left(\omega_{1},\omega_{2},\ldots,\right)$ tales que
$\omega_{i}\in\Omega_{i}$, $i=1,2,\ldots,$. Si
$B^{n}\subset\prod_{i=1}^{\infty}\Omega_{i}$, definimos
$B_{n}=\left\{\omega\in\Omega:\left(\omega_{1},\omega_{2},\ldots,\omega_{n}\right)\in
B^{n}\right\}$. Al conjunto $B_{n}$ se le llama {\em cilindro} con
base $B^{n}$, el cilindro es llamado medible si
$B^{n}\in\prod_{i=1}^{\infty}\mathcal{F}_{i}$.
\end{Def}


\begin{Def}\label{Def.Proc.Adaptado}[TSP, Ash \cite{RBA}]
Sea $X\left(t\right),t\geq0$ proceso estoc\'astico, el proceso es
adaptado a la familia de $\sigma$-\'algebras $\mathcal{F}_{t}$,
para $t\geq0$, si para $s<t$ implica que
$\mathcal{F}_{s}\subset\mathcal{F}_{t}$, y $X\left(t\right)$ es
$\mathcal{F}_{t}$-medible para cada $t$. Si no se especifica
$\mathcal{F}_{t}$ entonces se toma $\mathcal{F}_{t}$ como
$\mathcal{F}\left(X\left(s\right),s\leq t\right)$, la m\'as
peque\~na $\sigma$-\'algebra de subconjuntos de $\Omega$ que hace
que cada $X\left(s\right)$, con $s\leq t$ sea Borel medible.
\end{Def}


\begin{Def}\label{Def.Tiempo.Paro}[TSP, Ash \cite{RBA}]
Sea $\left\{\mathcal{F}\left(t\right),t\geq0\right\}$ familia
creciente de sub $\sigma$-\'algebras. es decir,
$\mathcal{F}\left(s\right)\subset\mathcal{F}\left(t\right)$ para
$s\leq t$. Un tiempo de paro para $\mathcal{F}\left(t\right)$ es
una funci\'on $T:\Omega\rightarrow\left[0,\infty\right]$ tal que
$\left\{T\leq t\right\}\in\mathcal{F}\left(t\right)$ para cada
$t\geq0$. Un tiempo de paro para el proceso estoc\'astico
$X\left(t\right),t\geq0$ es un tiempo de paro para las
$\sigma$-\'algebras
$\mathcal{F}\left(t\right)=\mathcal{F}\left(X\left(s\right)\right)$.
\end{Def}

\begin{Def}
Sea $X\left(t\right),t\geq0$ proceso estoc\'astico, con
$\left(S,\chi\right)$ espacio de estados. Se dice que el proceso
es adaptado a $\left\{\mathcal{F}\left(t\right)\right\}$, es
decir, si para cualquier $s,t\in I$, $I$ conjunto de \'indices,
$s<t$, se tiene que
$\mathcal{F}\left(s\right)\subset\mathcal{F}\left(t\right)$ y
$X\left(t\right)$ es $\mathcal{F}\left(t\right)$-medible,
\end{Def}

\begin{Def}
Sea $X\left(t\right),t\geq0$ proceso estoc\'astico, se dice que es
un Proceso de Markov relativo a $\mathcal{F}\left(t\right)$ o que
$\left\{X\left(t\right),\mathcal{F}\left(t\right)\right\}$ es de
Markov si y s\'olo si para cualquier conjunto $B\in\chi$,  y
$s,t\in I$, $s<t$ se cumple que
\begin{equation}\label{Prop.Markov}
P\left\{X\left(t\right)\in
B|\mathcal{F}\left(s\right)\right\}=P\left\{X\left(t\right)\in
B|X\left(s\right)\right\}.
\end{equation}
\end{Def}
\begin{Note}
Si se dice que $\left\{X\left(t\right)\right\}$ es un Proceso de
Markov sin mencionar $\mathcal{F}\left(t\right)$, se asumir\'a que
\begin{eqnarray*}
\mathcal{F}\left(t\right)=\mathcal{F}_{0}\left(t\right)=\mathcal{F}\left(X\left(r\right),r\leq
t\right),
\end{eqnarray*}
entonces la ecuaci\'on (\ref{Prop.Markov}) se puede escribir como
\begin{equation}
P\left\{X\left(t\right)\in B|X\left(r\right),r\leq s\right\} =
P\left\{X\left(t\right)\in B|X\left(s\right)\right\}
\end{equation}
\end{Note}

\begin{Teo}
Sea $\left(X_{n},\mathcal{F}_{n},n=0,1,\ldots,\right\}$ Proceso de
Markov con espacio de estados $\left(S_{0},\chi_{0}\right)$
generado por una distribuici\'on inicial $P_{o}$ y probabilidad de
transici\'on $p_{mn}$, para $m,n=0,1,\ldots,$ $m<n$, que por
notaci\'on se escribir\'a como $p\left(m,n,x,B\right)\rightarrow
p_{mn}\left(x,B\right)$. Sea $S$ tiempo de paro relativo a la
$\sigma$-\'algebra $\mathcal{F}_{n}$. Sea $T$ funci\'on medible,
$T:\Omega\rightarrow\left\{0,1,\ldots,\right\}$. Sup\'ongase que
$T\geq S$, entonces $T$ es tiempo de paro. Si $B\in\chi_{0}$,
entonces
\begin{equation}\label{Prop.Fuerte.Markov}
P\left\{X\left(T\right)\in
B,T<\infty|\mathcal{F}\left(S\right)\right\} =
p\left(S,T,X\left(s\right),B\right)
\end{equation}
en $\left\{T<\infty\right\}$.
\end{Teo}

Propiedades importantes para el modelo de flujo retrasado:

\begin{Prop}
 Sea $\left(\overline{Q},\overline{T},\overline{T}^{0}\right)$ un flujo l\'imite de \ref{Equation.4.4} y suponga que cuando $x\rightarrow\infty$ a lo largo de
una subsucesi\'on
\[\left(\frac{1}{|x|}Q_{k}^{x}\left(0\right),\frac{1}{|x|}A_{k}^{x}\left(0\right),\frac{1}{|x|}B_{k}^{x}\left(0\right),\frac{1}{|x|}B_{k}^{x,0}\left(0\right)\right)\rightarrow\left(\overline{Q}_{k}\left(0\right),0,0,0\right)\]
para $k=1,\ldots,K$. EL flujo l\'imite tiene las siguientes
propiedades, donde las propiedades de la derivada se cumplen donde
la derivada exista:
\begin{itemize}
 \item[i)] Los vectores de tiempo ocupado $\overline{T}\left(t\right)$ y $\overline{T}^{0}\left(t\right)$ son crecientes y continuas con
$\overline{T}\left(0\right)=\overline{T}^{0}\left(0\right)=0$.
\item[ii)] Para todo $t\geq0$
\[\sum_{k=1}^{K}\left[\overline{T}_{k}\left(t\right)+\overline{T}_{k}^{0}\left(t\right)\right]=t\]
\item[iii)] Para todo $1\leq k\leq K$
\[\overline{Q}_{k}\left(t\right)=\overline{Q}_{k}\left(0\right)+\alpha_{k}t-\mu_{k}\overline{T}_{k}\left(t\right)\]
\item[iv)]  Para todo $1\leq k\leq K$
\[\dot{{\overline{T}}}_{k}\left(t\right)=\beta_{k}\] para $\overline{Q}_{k}\left(t\right)=0$.
\item[v)] Para todo $k,j$
\[\mu_{k}^{0}\overline{T}_{k}^{0}\left(t\right)=\mu_{j}^{0}\overline{T}_{j}^{0}\left(t\right)\]
\item[vi)]  Para todo $1\leq k\leq K$
\[\mu_{k}\dot{{\overline{T}}}_{k}\left(t\right)=l_{k}\mu_{k}^{0}\dot{{\overline{T}}}_{k}^{0}\left(t\right)\] para $\overline{Q}_{k}\left(t\right)>0$.
\end{itemize}
\end{Prop}

\begin{Lema}[Lema 3.1 \cite{Chen}]\label{Lema3.1}
Si el modelo de flujo es estable, definido por las ecuaciones
(3.8)-(3.13), entonces el modelo de flujo retrasado tambin es
estable.
\end{Lema}

\begin{Teo}[Teorema 5.2 \cite{Chen}]\label{Tma.5.2}
Si el modelo de flujo lineal correspondiente a la red de cola es
estable, entonces la red de colas es estable.
\end{Teo}

\begin{Teo}[Teorema 5.1 \cite{Chen}]\label{Tma.5.1.Chen}
La red de colas es estable si existe una constante $t_{0}$ que
depende de $\left(\alpha,\mu,T,U\right)$ y $V$ que satisfagan las
ecuaciones (5.1)-(5.5), $Z\left(t\right)=0$, para toda $t\geq
t_{0}$.
\end{Teo}



\begin{Lema}[Lema 5.2 \cite{Gut}]\label{Lema.5.2.Gut}
Sea $\left\{\xi\left(k\right):k\in\ent\right\}$ sucesin de
variables aleatorias i.i.d. con valores en
$\left(0,\infty\right)$, y sea $E\left(t\right)$ el proceso de
conteo
\[E\left(t\right)=max\left\{n\geq1:\xi\left(1\right)+\cdots+\xi\left(n-1\right)\leq t\right\}.\]
Si $E\left[\xi\left(1\right)\right]<\infty$, entonces para
cualquier entero $r\geq1$
\begin{equation}
lim_{t\rightarrow\infty}\esp\left[\left(\frac{E\left(t\right)}{t}\right)^{r}\right]=\left(\frac{1}{E\left[\xi_{1}\right]}\right)^{r}
\end{equation}
de aqu, bajo estas condiciones
\begin{itemize}
\item[a)] Para cualquier $t>0$,
$sup_{t\geq\delta}\esp\left[\left(\frac{E\left(t\right)}{t}\right)^{r}\right]$

\item[b)] Las variables aleatorias
$\left\{\left(\frac{E\left(t\right)}{t}\right)^{r}:t\geq1\right\}$
son uniformemente integrables.
\end{itemize}
\end{Lema}

\begin{Teo}[Teorema 5.1: Ley Fuerte para Procesos de Conteo
\cite{Gut}]\label{Tma.5.1.Gut} Sea
$0<\mu<\esp\left(X_{1}\right]\leq\infty$. entonces

\begin{itemize}
\item[a)] $\frac{N\left(t\right)}{t}\rightarrow\frac{1}{\mu}$
a.s., cuando $t\rightarrow\infty$.


\item[b)]$\esp\left[\frac{N\left(t\right)}{t}\right]^{r}\rightarrow\frac{1}{\mu^{r}}$,
cuando $t\rightarrow\infty$ para todo $r>0$..
\end{itemize}
\end{Teo}


\begin{Prop}[Proposicin 5.1 \cite{DaiSean}]\label{Prop.5.1}
Suponga que los supuestos (A1) y (A2) se cumplen, adems suponga
que el modelo de flujo es estable. Entonces existe $t_{0}>0$ tal
que
\begin{equation}\label{Eq.Prop.5.1}
lim_{|x|\rightarrow\infty}\frac{1}{|x|^{p+1}}\esp_{x}\left[|X\left(t_{0}|x|\right)|^{p+1}\right]=0.
\end{equation}

\end{Prop}


\begin{Prop}[Proposici\'on 5.3 \cite{DaiSean}]
Sea $X$ proceso de estados para la red de colas, y suponga que se
cumplen los supuestos (A1) y (A2), entonces para alguna constante
positiva $C_{p+1}<\infty$, $\delta>0$ y un conjunto compacto
$C\subset X$.

\begin{equation}\label{Eq.5.4}
\esp_{x}\left[\int_{0}^{\tau_{C}\left(\delta\right)}\left(1+|X\left(t\right)|^{p}\right)dt\right]\leq
C_{p+1}\left(1+|x|^{p+1}\right)
\end{equation}
\end{Prop}

\begin{Prop}[Proposici\'on 5.4 \cite{DaiSean}]
Sea $X$ un proceso de Markov Borel Derecho en $X$, sea
$f:X\leftarrow\rea_{+}$ y defina para alguna $\delta>0$, y un
conjunto cerrado $C\subset X$
\[V\left(x\right):=\esp_{x}\left[\int_{0}^{\tau_{C}\left(\delta\right)}f\left(X\left(t\right)\right)dt\right]\]
para $x\in X$. Si $V$ es finito en todas partes y uniformemente
acotada en $C$, entonces existe $k<\infty$ tal que
\begin{equation}\label{Eq.5.11}
\frac{1}{t}\esp_{x}\left[V\left(x\right)\right]+\frac{1}{t}\int_{0}^{t}\esp_{x}\left[f\left(X\left(s\right)\right)ds\right]\leq\frac{1}{t}V\left(x\right)+k,
\end{equation}
para $x\in X$ y $t>0$.
\end{Prop}


\begin{Teo}[Teorema 5.5 \cite{DaiSean}]
Suponga que se cumplen (A1) y (A2), adems suponga que el modelo
de flujo es estable. Entonces existe una constante $k_{p}<\infty$
tal que
\begin{equation}\label{Eq.5.13}
\frac{1}{t}\int_{0}^{t}\esp_{x}\left[|Q\left(s\right)|^{p}\right]ds\leq
k_{p}\left\{\frac{1}{t}|x|^{p+1}+1\right\}
\end{equation}
para $t\geq0$, $x\in X$. En particular para cada condicin inicial
\begin{equation}\label{Eq.5.14}
Limsup_{t\rightarrow\infty}\frac{1}{t}\int_{0}^{t}\esp_{x}\left[|Q\left(s\right)|^{p}\right]ds\leq
k_{p}
\end{equation}
\end{Teo}

\begin{Teo}[Teorema 6.2\cite{DaiSean}]\label{Tma.6.2}
Suponga que se cumplen los supuestos (A1)-(A3) y que el modelo de
flujo es estable, entonces se tiene que
\[\parallel P^{t}\left(c,\cdot\right)-\pi\left(\cdot\right)\parallel_{f_{p}}\rightarrow0\]
para $t\rightarrow\infty$ y $x\in X$. En particular para cada
condicin inicial
\[lim_{t\rightarrow\infty}\esp_{x}\left[\left|Q_{t}\right|^{p}\right]=\esp_{\pi}\left[\left|Q_{0}\right|^{p}\right]<\infty\]
\end{Teo}


\begin{Teo}[Teorema 6.3\cite{DaiSean}]\label{Tma.6.3}
Suponga que se cumplen los supuestos (A1)-(A3) y que el modelo de
flujo es estable, entonces con
$f\left(x\right)=f_{1}\left(x\right)$, se tiene que
\[lim_{t\rightarrow\infty}t^{(p-1)\left|P^{t}\left(c,\cdot\right)-\pi\left(\cdot\right)\right|_{f}=0},\]
para $x\in X$. En particular, para cada condicin inicial
\[lim_{t\rightarrow\infty}t^{(p-1)\left|\esp_{x}\left[Q_{t}\right]-\esp_{\pi}\left[Q_{0}\right]\right|=0}.\]
\end{Teo}



%_____________________________________________________________________________________
\subsection{Proceso de Estados Markoviano para el Sistema}
%_________________________________________________________________________


Sean $Q_{k}\left(t\right)$ el n\'umero de usuarios en la cola $k$,
$A_{k}\left(t\right)$ el tiempo residual de arribos a la cola $k$,
para cada servidor $m$, sea $H_{m}\left(t\right)$ par ordenado que
consiste en la cola que est\'a siendo atendida y la pol\'itica de
servicio que se est\'a utilizando. $B_{m}\left(t\right)$ los
tiempos de servicio residuales, $B_{m}^{0}\left(t\right)$ el
tiempo residual de traslado, $C_{m}\left(t\right)$ el n\'umero de
usuarios atendidos durante la visita del servidor a la cola dada
en $H_{m}\left(t\right)$.

El proceso para el sistema de visitas se puede definir como:

\begin{equation}\label{Esp.Edos.Down}
X\left(t\right)^{T}=\left(Q_{k}\left(t\right),A_{k}\left(t\right),B_{m}\left(t\right),B_{m}^{0}\left(t\right),C_{m}\left(t\right)\right)
\end{equation}
para $k=1,\ldots,K$ y $m=1,2,\ldots,M$. $X$ evoluciona en el
espacio de estados:
$X=\ent_{+}^{K}\times\rea_{+}^{K}\times\left(\left\{1,2,\ldots,K\right\}\times\left\{1,2,\ldots,S\right\}\right)^{M}\times\rea_{+}^{K}\times\rea_{+}^{K}\times\ent_{+}^{K}$.\\

Antes enunciemos los supuestos que regir\'an en la red.


\begin{itemize}
\item[A1)] $\xi_{1},\ldots,\xi_{K},\eta_{1},\ldots,\eta_{K}$ son
mutuamente independientes y son sucesiones independientes e
id\'enticamente distribuidas.

\item[A2)] Para alg\'un entero $p\geq1$
\begin{eqnarray*}
\esp\left[\xi_{l}\left(1\right)^{p+1}\right]<\infty\textrm{ para }l\in\mathcal{A}\textrm{ y }\\
\esp\left[\eta_{k}\left(1\right)^{p+1}\right]<\infty\textrm{ para
}k=1,\ldots,K.
\end{eqnarray*}
donde $\mathcal{A}$ es la clase de posibles arribos.

\item[A3)] Para $k=1,2,\ldots,K$ existe una funci\'on positiva
$q_{k}\left(x\right)$ definida en $\rea_{+}$, y un entero $j_{k}$,
tal que
\begin{eqnarray}
P\left(\xi_{k}\left(1\right)\geq x\right)>0\textrm{, para todo }x>0\\
P\left(\xi_{k}\left(1\right)+\ldots\xi_{k}\left(j_{k}\right)\in dx\right)\geq q_{k}\left(x\right)dx0\textrm{ y }\\
\int_{0}^{\infty}q_{k}\left(x\right)dx>0
\end{eqnarray}
\end{itemize}
%_________________________________________________________________________
\subsection{Procesos Fuerte de Markov}
%_________________________________________________________________________

En Dai \cite{Dai} se muestra que para una amplia serie de
disciplinas de servicio el proceso $X$ es un Proceso Fuerte de
Markov, y por tanto se puede asumir que
\[\left(\left(\Omega,\mathcal{F}\right),\mathcal{F}_{t},X\left(t\right),\theta_{t},P_{x}\right)\]
es un proceso de Borel Derecho, Sharpe \cite{Sharpe}, en el
espacio de estados medible
$\left(X,\mathcal{B}_{X}\right)$.


Se har\'an las siguientes consideraciones: $E$ es un espacio
m\'etrico separable.


\begin{Def}
Un espacio topol\'ogico $E$ es llamado {\em Luisin} si es
homeomorfo a un subconjunto de Borel de un espacio m\'etrico
compacto.
\end{Def}

\begin{Def}
Un espacio topol\'ogico $E$ es llamado de {\em Rad\'on} si es
homeomorfo a un subconjunto universalmente medible de un espacio
m\'etrico compacto.
\end{Def}

Equivalentemente, la definici\'on de un espacio de Rad\'on puede
encontrarse en los siguientes t\'erminos:

\begin{Def}
$E$ es un espacio de Rad\'on si cada medida finita en
$\left(E,\mathcal{B}\left(E\right)\right)$ es regular interior o
cerrada, {\em tight}.
\end{Def}

\begin{Def}
Una medida finita, $\lambda$ en la $\sigma$-\'algebra de Borel de
un espacio metrizable $E$ se dice cerrada si
\begin{equation}\label{Eq.A2.3}
\lambda\left(E\right)=sup\left\{\lambda\left(K\right):K\textrm{ es
compacto en }E\right\}.
\end{equation}
\end{Def}

El siguiente teorema nos permite tener una mejor caracterizaci\'on
de los espacios de Rad\'on:
\begin{Teo}\label{Tma.A2.2}
Sea $E$ espacio separable metrizable. Entonces $E$ es Radoniano si
y s\'olo s\'i cada medida finita en
$\left(E,\mathcal{B}\left(E\right)\right)$ es cerrada.
\end{Teo}

%_________________________________________________________________________________________
\subsection{Propiedades de Markov}
%_________________________________________________________________________________________

Sea $E$ espacio de estados, tal que $E$ es un espacio de Rad\'on,
$\mathcal{B}\left(E\right)$ $\sigma$-\'algebra de Borel en $E$,
que se denotar\'a por $\mathcal{E}$.

Sea $\left(X,\mathcal{G},\prob\right)$ espacio de probabilidad,
$I\subset\rea$ conjunto de índices. Sea $\mathcal{F}_{\leq
t}$ la $\sigma$-\'algebra natural definida como
$\sigma\left\{f\left(X_{r}\right):r\in I, r\leq
t,f\in\mathcal{E}\right\}$. Se considerar\'a una
$\sigma$-\'algebra m\'as general, $ \left(\mathcal{G}_{t}\right)$
tal que $\left(X_{t}\right)$ sea $\mathcal{E}$-adaptado.

\begin{Def}
Una familia $\left(P_{s,t}\right)$ de kernels de Markov en
$\left(E,\mathcal{E}\right)$ indexada por pares $s,t\in I$, con
$s\leq t$ es una funci\'on de transici\'on en $\ER$, si  para todo
$r\leq s< t$ en $I$ y todo $x\in E$, $B\in\mathcal{E}$
\begin{equation}\label{Eq.Kernels}
P_{r,t}\left(x,B\right)=\int_{E}P_{r,s}\left(x,dy\right)P_{s,t}\left(y,B\right)\footnote{Ecuaci\'on
de Chapman-Kolmogorov}.
\end{equation}
\end{Def}

Se dice que la funci\'on de transici\'on $\KM$ en $\ER$ es la
funci\'on de transici\'on para un proceso $\PE$  con valores en
$E$ y que satisface la propiedad de
Markov \footnote{\begin{equation}\label{Eq.1.4.S}
\prob\left\{H|\mathcal{G}_{t}\right\}=\prob\left\{H|X_{t}\right\}\textrm{
}H\in p\mathcal{F}_{\geq t}.
\end{equation}} (\ref{Eq.1.4.S}) relativa a $\left(\mathcal{G}_{t}\right)$ si

\begin{equation}\label{Eq.1.6.S}
\prob\left\{f\left(X_{t}\right)|\mathcal{G}_{s}\right\}=P_{s,t}f\left(X_{t}\right)\textrm{
}s\leq t\in I,\textrm{ }f\in b\mathcal{E}.
\end{equation}

\begin{Def}
Una familia $\left(P_{t}\right)_{t\geq0}$ de kernels de Markov en
$\ER$ es llamada {\em Semigrupo de Transici\'on de Markov} o {\em
Semigrupo de Transici\'on} si
\[P_{t+s}f\left(x\right)=P_{t}\left(P_{s}f\right)\left(x\right),\textrm{ }t,s\geq0,\textrm{ }x\in E\textrm{ }f\in b\mathcal{E}.\]
\end{Def}

\begin{Note}
Si la funci\'on de transici\'on $\KM$ es llamada homog\'enea si
$P_{s,t}=P_{t-s}$.
\end{Note}


Un proceso de Markov que satisface la ecuaci\'on (\ref{Eq.1.6.S})
con funci\'on de transici\'on homog\'enea $\left(P_{t}\right)$
tiene la propiedad caracter\'istica
\begin{equation}\label{Eq.1.8.S}
\prob\left\{f\left(X_{t+s}\right)|\mathcal{G}_{t}\right\}=P_{s}f\left(X_{t}\right)\textrm{
}t,s\geq0,\textrm{ }f\in b\mathcal{E}.
\end{equation}
La ecuaci\'on anterior es la {\em Propiedad Simple de Markov} de
$X$ relativa a $\left(P_{t}\right)$.

En este sentido el proceso $\PE$ cumple con la propiedad de Markov
(\ref{Eq.1.8.S}) relativa a
$\left(\Omega,\mathcal{G},\mathcal{G}_{t},\prob\right)$ con
semigrupo de transici\'on $\left(P_{t}\right)$.

%_________________________________________________________________________________________
\subsection{Primer Condici\'on de Regularidad}
%_________________________________________________________________________________________


\begin{Def}
Un proceso estoc\'astico $\PE$ definido en
$\left(\Omega,\mathcal{G},\prob\right)$ con valores en el espacio
topol\'ogico $E$ es continuo por la derecha si cada trayectoria
muestral $t\rightarrow X_{t}\left(w\right)$ es un mapeo continuo
por la derecha de $I$ en $E$.
\end{Def}

\begin{Def}[HD1]\label{Eq.2.1.S}
Un semigrupo de Markov $\left/P_{t}\right)$ en un espacio de
Rad\'on $E$ se dice que satisface la condici\'on {\em HD1} si,
dada una medida de probabilidad $\mu$ en $E$, existe una
$\sigma$-\'algebra $\mathcal{E^{*}}$ con
$\mathcal{E}\subset\mathcal{E}$ y
$P_{t}\left(b\mathcal{E}^{*}\right)\subset b\mathcal{E}^{*}$, y un
$\mathcal{E}^{*}$-proceso $E$-valuado continuo por la derecha
$\PE$ en alg\'un espacio de probabilidad filtrado
$\left(\Omega,\mathcal{G},\mathcal{G}_{t},\prob\right)$ tal que
$X=\left(\Omega,\mathcal{G},\mathcal{G}_{t},\prob\right)$ es de
Markov (Homog\'eneo) con semigrupo de transici\'on $(P_{t})$ y
distribuci\'on inicial $\mu$.
\end{Def}
Consid\'erese la colecci\'on de variables aleatorias $X_{t}$
definidas en alg\'un espacio de probabilidad, y una colecci\'on de
medidas $\mathbf{P}^{x}$ tales que
$\mathbf{P}^{x}\left\{X_{0}=x\right\}$, y bajo cualquier
$\mathbf{P}^{x}$, $X_{t}$ es de Markov con semigrupo
$\left(P_{t}\right)$. $\mathbf{P}^{x}$ puede considerarse como la
distribuci\'on condicional de $\mathbf{P}$ dado $X_{0}=x$.

\begin{Def}\label{Def.2.2.S}
Sea $E$ espacio de Rad\'on, $\SG$ semigrupo de Markov en $\ER$. La
colecci\'on
$\mathbf{X}=\left(\Omega,\mathcal{G},\mathcal{G}_{t},X_{t},\theta_{t},\CM\right)$
es un proceso $\mathcal{E}$-Markov continuo por la derecha simple,
con espacio de estados $E$ y semigrupo de transici\'on $\SG$ en
caso de que $\mathbf{X}$ satisfaga las siguientes
condiciones:
\begin{itemize}
\item[i)] $\left(\Omega,\mathcal{G},\mathcal{G}_{t}\right)$ es un
espacio de medida filtrado, y $X_{t}$ es un proceso $E$-valuado
continuo por la derecha $\mathcal{E}^{*}$-adaptado a
$\left(\mathcal{G}_{t}\right)$;

\item[ii)] $\left(\theta_{t}\right)_{t\geq0}$ es una colecci\'on
de operadores {\em shift} para $X$, es decir, mapea $\Omega$ en
s\'i mismo satisfaciendo para $t,s\geq0$,

\begin{equation}\label{Eq.Shift}
\theta_{t}\circ\theta_{s}=\theta_{t+s}\textrm{ y
}X_{t}\circ\theta_{t}=X_{t+s};
\end{equation}

\item[iii)] Para cualquier $x\in E$,$\CM\left\{X_{0}=x\right\}=1$,
y el proceso $\PE$ tiene la propiedad de Markov (\ref{Eq.1.8.S})
con semigrupo de transici\'on $\SG$ relativo a
$\left(\Omega,\mathcal{G},\mathcal{G}_{t},\CM\right)$.
\end{itemize}
\end{Def}


\begin{Def}[HD2]\label{Eq.2.2.S}
Para cualquier $\alpha>0$ y cualquier $f\in S^{\alpha}$, el
proceso $t\rightarrow f\left(X_{t}\right)$ es continuo por la
derecha casi seguramente.
\end{Def}

\begin{Def}\label{Def.PD}
Un sistema
$\mathbf{X}=\left(\Omega,\mathcal{G},\mathcal{G}_{t},X_{t},\theta_{t},\CM\right)$
es un proceso derecho en el espacio de Rad\'on $E$ con semigrupo
de transici\'on $\SG$ provisto de:
\begin{itemize}
\item[i)] $\mathbf{X}$ es una realizaci\'on  continua por la
derecha, \ref{Def.2.2.S}, de $\SG$.

\item[ii)] $\mathbf{X}$ satisface la condicion HD2,
\ref{Eq.2.2.S}, relativa a $\mathcal{G}_{t}$.

\item[iii)] $\mathcal{G}_{t}$ es aumentado y continuo por la
derecha.
\end{itemize}
\end{Def}


\begin{Def}
Sea $X$ un conjunto y $\mathcal{F}$ una $\sigma$-\'algebra de
subconjuntos de $X$, la pareja $\left(X,\mathcal{F}\right)$ es
llamado espacio medible. Un subconjunto $A$ de $X$ es llamado
medible, o medible con respecto a $\mathcal{F}$, si
$A\in\mathcal{F}$.
\end{Def}

\begin{Def}
Sea $\left(X,\mathcal{F},\mu\right)$ espacio de medida. Se dice
que la medida $\mu$ es $\sigma$-finita si se puede escribir
$X=\bigcup_{n\geq1}X_{n}$ con $X_{n}\in\mathcal{F}$ y
$\mu\left(X_{n}\right)<\infty$.
\end{Def}

\begin{Def}\label{Cto.Borel}
Sea $X$ el conjunto de los \'umeros reales $\rea$. El \'algebra de
Borel es la $\sigma$-\'algebra $B$ generada por los intervalos
abiertos $\left(a,b\right)\in\rea$. Cualquier conjunto en $B$ es
llamado {\em Conjunto de Borel}.
\end{Def}

\begin{Def}\label{Funcion.Medible}
Una funci\'on $f:X\rightarrow\rea$, es medible si para cualquier
n\'umero real $\alpha$ el conjunto
\[\left\{x\in X:f\left(x\right)>\alpha\right\}\]
pertenece a $X$. Equivalentemente, se dice que $f$ es medible si
\[f^{-1}\left(\left(\alpha,\infty\right)\right)=\left\{x\in X:f\left(x\right)>\alpha\right\}\in\mathcal{F}.\]
\end{Def}


\begin{Def}\label{Def.Cilindros}
Sean $\left(\Omega_{i},\mathcal{F}_{i}\right)$, $i=1,2,\ldots,$
espacios medibles y $\Omega=\prod_{i=1}^{\infty}\Omega_{i}$ el
conjunto de todas las sucesiones
$\left(\omega_{1},\omega_{2},\ldots,\right)$ tales que
$\omega_{i}\in\Omega_{i}$, $i=1,2,\ldots,$. Si
$B^{n}\subset\prod_{i=1}^{\infty}\Omega_{i}$, definimos
$B_{n}=\left\{\omega\in\Omega:\left(\omega_{1},\omega_{2},\ldots,\omega_{n}\right)\in
B^{n}\right\}$. Al conjunto $B_{n}$ se le llama {\em cilindro} con
base $B^{n}$, el cilindro es llamado medible si
$B^{n}\in\prod_{i=1}^{\infty}\mathcal{F}_{i}$.
\end{Def}


\begin{Def}\label{Def.Proc.Adaptado}[TSP, Ash \cite{RBA}]
Sea $X\left(t\right),t\geq0$ proceso estoc\'astico, el proceso es
adaptado a la familia de $\sigma$-\'algebras $\mathcal{F}_{t}$,
para $t\geq0$, si para $s<t$ implica que
$\mathcal{F}_{s}\subset\mathcal{F}_{t}$, y $X\left(t\right)$ es
$\mathcal{F}_{t}$-medible para cada $t$. Si no se especifica
$\mathcal{F}_{t}$ entonces se toma $\mathcal{F}_{t}$ como
$\mathcal{F}\left(X\left(s\right),s\leq t\right)$, la m\'as
peque\~na $\sigma$-\'algebra de subconjuntos de $\Omega$ que hace
que cada $X\left(s\right)$, con $s\leq t$ sea Borel medible.
\end{Def}


\begin{Def}\label{Def.Tiempo.Paro}[TSP, Ash \cite{RBA}]
Sea $\left\{\mathcal{F}\left(t\right),t\geq0\right\}$ familia
creciente de sub $\sigma$-\'algebras. es decir,
$\mathcal{F}\left(s\right)\subset\mathcal{F}\left(t\right)$ para
$s\leq t$. Un tiempo de paro para $\mathcal{F}\left(t\right)$ es
una funci\'on $T:\Omega\rightarrow\left[0,\infty\right]$ tal que
$\left\{T\leq t\right\}\in\mathcal{F}\left(t\right)$ para cada
$t\geq0$. Un tiempo de paro para el proceso estoc\'astico
$X\left(t\right),t\geq0$ es un tiempo de paro para las
$\sigma$-\'algebras
$\mathcal{F}\left(t\right)=\mathcal{F}\left(X\left(s\right)\right)$.
\end{Def}

\begin{Def}
Sea $X\left(t\right),t\geq0$ proceso estoc\'astico, con
$\left(S,\chi\right)$ espacio de estados. Se dice que el proceso
es adaptado a $\left\{\mathcal{F}\left(t\right)\right\}$, es
decir, si para cualquier $s,t\in I$, $I$ conjunto de \'indices,
$s<t$, se tiene que
$\mathcal{F}\left(s\right)\subset\mathcal{F}\left(t\right)$ y
$X\left(t\right)$ es $\mathcal{F}\left(t\right)$-medible,
\end{Def}

\begin{Def}
Sea $X\left(t\right),t\geq0$ proceso estoc\'astico, se dice que es
un Proceso de Markov relativo a $\mathcal{F}\left(t\right)$ o que
$\left\{X\left(t\right),\mathcal{F}\left(t\right)\right\}$ es de
Markov si y s\'olo si para cualquier conjunto $B\in\chi$,  y
$s,t\in I$, $s<t$ se cumple que
\begin{equation}\label{Prop.Markov}
P\left\{X\left(t\right)\in
B|\mathcal{F}\left(s\right)\right\}=P\left\{X\left(t\right)\in
B|X\left(s\right)\right\}.
\end{equation}
\end{Def}
\begin{Note}
Si se dice que $\left\{X\left(t\right)\right\}$ es un Proceso de
Markov sin mencionar $\mathcal{F}\left(t\right)$, se asumir\'a que
\begin{eqnarray*}
\mathcal{F}\left(t\right)=\mathcal{F}_{0}\left(t\right)=\mathcal{F}\left(X\left(r\right),r\leq
t\right),
\end{eqnarray*}
entonces la ecuaci\'on (\ref{Prop.Markov}) se puede escribir como
\begin{equation}
P\left\{X\left(t\right)\in B|X\left(r\right),r\leq s\right\} =
P\left\{X\left(t\right)\in B|X\left(s\right)\right\}
\end{equation}
\end{Note}

\begin{Teo}
Sea $\left(X_{n},\mathcal{F}_{n},n=0,1,\ldots,\right\}$ Proceso de
Markov con espacio de estados $\left(S_{0},\chi_{0}\right)$
generado por una distribuici\'on inicial $P_{o}$ y probabilidad de
transici\'on $p_{mn}$, para $m,n=0,1,\ldots,$ $m<n$, que por
notaci\'on se escribir\'a como $p\left(m,n,x,B\right)\rightarrow
p_{mn}\left(x,B\right)$. Sea $S$ tiempo de paro relativo a la
$\sigma$-\'algebra $\mathcal{F}_{n}$. Sea $T$ funci\'on medible,
$T:\Omega\rightarrow\left\{0,1,\ldots,\right\}$. Sup\'ongase que
$T\geq S$, entonces $T$ es tiempo de paro. Si $B\in\chi_{0}$,
entonces
\begin{equation}\label{Prop.Fuerte.Markov}
P\left\{X\left(T\right)\in
B,T<\infty|\mathcal{F}\left(S\right)\right\} =
p\left(S,T,X\left(s\right),B\right)
\end{equation}
en $\left\{T<\infty\right\}$.
\end{Teo}


Sea $K$ conjunto numerable y sea $d:K\rightarrow\nat$ funci\'on.
Para $v\in K$, $M_{v}$ es un conjunto abierto de
$\rea^{d\left(v\right)}$. Entonces \[E=\cup_{v\in
K}M_{v}=\left\{\left(v,\zeta\right):v\in K,\zeta\in
M_{v}\right\}.\]

Sea $\mathcal{E}$ la clase de conjuntos medibles en $E$:
\[\mathcal{E}=\left\{\cup_{v\in K}A_{v}:A_{v}\in \mathcal{M}_{v}\right\}.\]

donde $\mathcal{M}$ son los conjuntos de Borel de $M_{v}$.
Entonces $\left(E,\mathcal{E}\right)$ es un espacio de Borel. El
estado del proceso se denotar\'a por
$\mathbf{x}_{t}=\left(v_{t},\zeta_{t}\right)$. La distribuci\'on
de $\left(\mathbf{x}_{t}\right)$ est\'a determinada por por los
siguientes objetos:

\begin{itemize}
\item[i)] Los campos vectoriales $\left(\mathcal{H}_{v},v\in
K\right)$. \item[ii)] Una funci\'on medible $\lambda:E\rightarrow
\rea_{+}$. \item[iii)] Una medida de transici\'on
$Q:\mathcal{E}\times\left(E\cup\Gamma^{*}\right)\rightarrow\left[0,1\right]$
donde
\begin{equation}
\Gamma^{*}=\cup_{v\in K}\partial^{*}M_{v}.
\end{equation}
y
\begin{equation}
\partial^{*}M_{v}=\left\{z\in\partial M_{v}:\mathbf{\mathbf{\phi}_{v}\left(t,\zeta\right)=\mathbf{z}}\textrm{ para alguna }\left(t,\zeta\right)\in\rea_{+}\times M_{v}\right\}.
\end{equation}
$\partial M_{v}$ denota  la frontera de $M_{v}$.
\end{itemize}

El campo vectorial $\left(\mathcal{H}_{v},v\in K\right)$ se supone
tal que para cada $\mathbf{z}\in M_{v}$ existe una \'unica curva
integral $\mathbf{\phi}_{v}\left(t,\zeta\right)$ que satisface la
ecuaci\'on

\begin{equation}
\frac{d}{dt}f\left(\zeta_{t}\right)=\mathcal{H}f\left(\zeta_{t}\right),
\end{equation}
con $\zeta_{0}=\mathbf{z}$, para cualquier funci\'on suave
$f:\rea^{d}\rightarrow\rea$ y $\mathcal{H}$ denota el operador
diferencial de primer orden, con $\mathcal{H}=\mathcal{H}_{v}$ y
$\zeta_{t}=\mathbf{\phi}\left(t,\mathbf{z}\right)$. Adem\'as se
supone que $\mathcal{H}_{v}$ es conservativo, es decir, las curvas
integrales est\'an definidas para todo $t>0$.

Para $\mathbf{x}=\left(v,\zeta\right)\in E$ se denota
\[t^{*}\mathbf{x}=inf\left\{t>0:\mathbf{\phi}_{v}\left(t,\zeta\right)\in\partial^{*}M_{v}\right\}\]

En lo que respecta a la funci\'on $\lambda$, se supondr\'a que
para cada $\left(v,\zeta\right)\in E$ existe un $\epsilon>0$ tal
que la funci\'on
$s\rightarrow\lambda\left(v,\phi_{v}\left(s,\zeta\right)\right)\in
E$ es integrable para $s\in\left[0,\epsilon\right)$. La medida de
transici\'on $Q\left(A;\mathbf{x}\right)$ es una funci\'on medible
de $\mathbf{x}$ para cada $A\in\mathcal{E}$, definida para
$\mathbf{x}\in E\cup\Gamma^{*}$ y es una medida de probabilidad en
$\left(E,\mathcal{E}\right)$ para cada $\mathbf{x}\in E$.

El movimiento del proceso $\left(\mathbf{x}_{t}\right)$ comenzando
en $\mathbf{x}=\left(n,\mathbf{z}\right)\in E$ se puede construir
de la siguiente manera, def\'inase la funci\'on $F$ por

\begin{equation}
F\left(t\right)=\left\{\begin{array}{ll}\\
exp\left(-\int_{0}^{t}\lambda\left(n,\phi_{n}\left(s,\mathbf{z}\right)\right)ds\right), & t<t^{*}\left(\mathbf{x}\right),\\
0, & t\geq t^{*}\left(\mathbf{x}\right)
\end{array}\right.
\end{equation}

Sea $T_{1}$ una variable aleatoria tal que
$\prob\left[T_{1}>t\right]=F\left(t\right)$, ahora sea la variable
aleatoria $\left(N,Z\right)$ con distribuici\'on
$Q\left(\cdot;\phi_{n}\left(T_{1},\mathbf{z}\right)\right)$. La
trayectoria de $\left(\mathbf{x}_{t}\right)$ para $t\leq T_{1}$
es\footnote{Revisar p\'agina 362, y 364 de Davis \cite{Davis}.}
\begin{eqnarray*}
\mathbf{x}_{t}=\left(v_{t},\zeta_{t}\right)=\left\{\begin{array}{ll}
\left(n,\phi_{n}\left(t,\mathbf{z}\right)\right), & t<T_{1},\\
\left(N,\mathbf{Z}\right), & t=t_{1}.
\end{array}\right.
\end{eqnarray*}

Comenzando en $\mathbf{x}_{T_{1}}$ se selecciona el siguiente
tiempo de intersalto $T_{2}-T_{1}$ lugar del post-salto
$\mathbf{x}_{T_{2}}$ de manera similar y as\'i sucesivamente. Este
procedimiento nos da una trayectoria determinista por partes
$\mathbf{x}_{t}$ con tiempos de salto $T_{1},T_{2},\ldots$. Bajo
las condiciones enunciadas para $\lambda,T_{1}>0$  y
$T_{1}-T_{2}>0$ para cada $i$, con probabilidad 1. Se supone que
se cumple la siquiente condici\'on.

\begin{Sup}[Supuesto 3.1, Davis \cite{Davis}]\label{Sup3.1.Davis}
Sea $N_{t}:=\sum_{t}\indora_{\left(t\geq t\right)}$ el n\'umero de
saltos en $\left[0,t\right]$. Entonces
\begin{equation}
\esp\left[N_{t}\right]<\infty\textrm{ para toda }t.
\end{equation}
\end{Sup}

es un proceso de Markov, m\'as a\'un, es un Proceso Fuerte de
Markov, es decir, la Propiedad Fuerte de Markov se cumple para
cualquier tiempo de paro.
%_________________________________________________________________________

En esta secci\'on se har\'an las siguientes consideraciones: $E$
es un espacio m\'etrico separable y la m\'etrica $d$ es compatible
con la topolog\'ia.


\begin{Def}
Un espacio topol\'ogico $E$ es llamado {\em Luisin} si es
homeomorfo a un subconjunto de Borel de un espacio m\'etrico
compacto.
\end{Def}

\begin{Def}
Un espacio topol\'ogico $E$ es llamado de {\em Rad\'on} si es
homeomorfo a un subconjunto universalmente medible de un espacio
m\'etrico compacto.
\end{Def}

Equivalentemente, la definici\'on de un espacio de Rad\'on puede
encontrarse en los siguientes t\'erminos:


\begin{Def}
$E$ es un espacio de Rad\'on si cada medida finita en
$\left(E,\mathcal{B}\left(E\right)\right)$ es regular interior o cerrada,
{\em tight}.
\end{Def}

\begin{Def}
Una medida finita, $\lambda$ en la $\sigma$-\'algebra de Borel de
un espacio metrizable $E$ se dice cerrada si
\begin{equation}\label{Eq.A2.3}
\lambda\left(E\right)=sup\left\{\lambda\left(K\right):K\textrm{ es
compacto en }E\right\}.
\end{equation}
\end{Def}

El siguiente teorema nos permite tener una mejor caracterizaci\'on de los espacios de Rad\'on:
\begin{Teo}\label{Tma.A2.2}
Sea $E$ espacio separable metrizable. Entonces $E$ es Radoniano si y s\'olo s\'i cada medida finita en $\left(E,\mathcal{B}\left(E\right)\right)$ es cerrada.
\end{Teo}

%_________________________________________________________________________________________
\subsection{Propiedades de Markov}
%_________________________________________________________________________________________

Sea $E$ espacio de estados, tal que $E$ es un espacio de Rad\'on, $\mathcal{B}\left(E\right)$ $\sigma$-\'algebra de Borel en $E$, que se denotar\'a por $\mathcal{E}$.

Sea $\left(X,\mathcal{G},\prob\right)$ espacio de probabilidad, $I\subset\rea$ conjunto de índices. Sea $\mathcal{F}_{\leq t}$ la $\sigma$-\'algebra natural definida como $\sigma\left\{f\left(X_{r}\right):r\in I, rleq t,f\in\mathcal{E}\right\}$. Se considerar\'a una $\sigma$-\'algebra m\'as general, $ \left(\mathcal{G}_{t}\right)$ tal que $\left(X_{t}\right)$ sea $\mathcal{E}$-adaptado.

\begin{Def}
Una familia $\left(P_{s,t}\right)$ de kernels de Markov en $\left(E,\mathcal{E}\right)$ indexada por pares $s,t\in I$, con $s\leq t$ es una funci\'on de transici\'on en $\ER$, si  para todo $r\leq s< t$ en $I$ y todo $x\in E$, $B\in\mathcal{E}$
\begin{equation}\label{Eq.Kernels}
P_{r,t}\left(x,B\right)=\int_{E}P_{r,s}\left(x,dy\right)P_{s,t}\left(y,B\right)\footnote{Ecuaci\'on de Chapman-Kolmogorov}.
\end{equation}
\end{Def}

Se dice que la funci\'on de transici\'on $\KM$ en $\ER$ es la funci\'on de transici\'on para un proceso $\PE$  con valores en $E$ y que satisface la propiedad de Markov\footnote{\begin{equation}\label{Eq.1.4.S}
\prob\left\{H|\mathcal{G}_{t}\right\}=\prob\left\{H|X_{t}\right\}\textrm{ }H\in p\mathcal{F}_{\geq t}.
\end{equation}} (\ref{Eq.1.4.S}) relativa a $\left(\mathcal{G}_{t}\right)$ si 

\begin{equation}\label{Eq.1.6.S}
\prob\left\{f\left(X_{t}\right)|\mathcal{G}_{s}\right\}=P_{s,t}f\left(X_{t}\right)\textrm{ }s\leq t\in I,\textrm{ }f\in b\mathcal{E}.
\end{equation}

\begin{Def}
Una familia $\left(P_{t}\right)_{t\geq0}$ de kernels de Markov en $\ER$ es llamada {\em Semigrupo de Transici\'on de Markov} o {\em Semigrupo de Transici\'on} si
\[P_{t+s}f\left(x\right)=P_{t}\left(P_{s}f\right)\left(x\right),\textrm{ }t,s\geq0,\textrm{ }x\in E\textrm{ }f\in b\mathcal{E}.\]
\end{Def}
\begin{Note}
Si la funci\'on de transici\'on $\KM$ es llamada homog\'enea si $P_{s,t}=P_{t-s}$.
\end{Note}

Un proceso de Markov que satisface la ecuaci\'on (\ref{Eq.1.6.S}) con funci\'on de transici\'on homog\'enea $\left(P_{t}\right)$ tiene la propiedad caracter\'istica
\begin{equation}\label{Eq.1.8.S}
\prob\left\{f\left(X_{t+s}\right)|\mathcal{G}_{t}\right\}=P_{s}f\left(X_{t}\right)\textrm{ }t,s\geq0,\textrm{ }f\in b\mathcal{E}.
\end{equation}
La ecuaci\'on anterior es la {\em Propiedad Simple de Markov} de $X$ relativa a $\left(P_{t}\right)$.

En este sentido el proceso $\PE$ cumple con la propiedad de Markov (\ref{Eq.1.8.S}) relativa a $\left(\Omega,\mathcal{G},\mathcal{G}_{t},\prob\right)$ con semigrupo de transici\'on $\left(P_{t}\right)$.
%_________________________________________________________________________________________
\subsection{Primer Condici\'on de Regularidad}
%_________________________________________________________________________________________
%\newcommand{\EM}{\left(\Omega,\mathcal{G},\prob\right)}
%\newcommand{\E4}{\left(\Omega,\mathcal{G},\mathcal{G}_{t},\prob\right)}
\begin{Def}
Un proceso estoc\'astico $\PE$ definido en $\left(\Omega,\mathcal{G},\prob\right)$ con valores en el espacio topol\'ogico $E$ es continuo por la derecha si cada trayectoria muestral $t\rightarrow X_{t}\left(w\right)$ es un mapeo continuo por la derecha de $I$ en $E$.
\end{Def}

\begin{Def}[HD1]\label{Eq.2.1.S}
Un semigrupo de Markov $\left/P_{t}\right)$ en un espacio de Rad\'on $E$ se dice que satisface la condici\'on {\em HD1} si, dada una medida de probabilidad $\mu$ en $E$, existe una $\sigma$-\'algebra $\mathcal{E^{*}}$ con $\mathcal{E}\subset\mathcal{E}$ y $P_{t}\left(b\mathcal{E}^{*}\right)\subset b\mathcal{E}^{*}$, y un $\mathcal{E}^{*}$-proceso $E$-valuado continuo por la derecha $\PE$ en alg\'un espacio de probabilidad filtrado $\left(\Omega,\mathcal{G},\mathcal{G}_{t},\prob\right)$ tal que $X=\left(\Omega,\mathcal{G},\mathcal{G}_{t},\prob\right)$ es de Markov (Homog\'eneo) con semigrupo de transici\'on $(P_{t})$ y distribuci\'on inicial $\mu$.
\end{Def}

Considerese la colecci\'on de variables aleatorias $X_{t}$ definidas en alg\'un espacio de probabilidad, y una colecci\'on de medidas $\mathbf{P}^{x}$ tales que $\mathbf{P}^{x}\left\{X_{0}=x\right\}$, y bajo cualquier $\mathbf{P}^{x}$, $X_{t}$ es de Markov con semigrupo $\left(P_{t}\right)$. $\mathbf{P}^{x}$ puede considerarse como la distribuci\'on condicional de $\mathbf{P}$ dado $X_{0}=x$.

\begin{Def}\label{Def.2.2.S}
Sea $E$ espacio de Rad\'on, $\SG$ semigrupo de Markov en $\ER$. La colecci\'on $\mathbf{X}=\left(\Omega,\mathcal{G},\mathcal{G}_{t},X_{t},\theta_{t},\CM\right)$ es un proceso $\mathcal{E}$-Markov continuo por la derecha simple, con espacio de estados $E$ y semigrupo de transici\'on $\SG$ en caso de que $\mathbf{X}$ satisfaga las siguientes condiciones:
\begin{itemize}
\item[i)] $\left(\Omega,\mathcal{G},\mathcal{G}_{t}\right)$ es un espacio de medida filtrado, y $X_{t}$ es un proceso $E$-valuado continuo por la derecha $\mathcal{E}^{*}$-adaptado a $\left(\mathcal{G}_{t}\right)$;

\item[ii)] $\left(\theta_{t}\right)_{t\geq0}$ es una colecci\'on de operadores {\em shift} para $X$, es decir, mapea $\Omega$ en s\'i mismo satisfaciendo para $t,s\geq0$,

\begin{equation}\label{Eq.Shift}
\theta_{t}\circ\theta_{s}=\theta_{t+s}\textrm{ y }X_{t}\circ\theta_{t}=X_{t+s};
\end{equation}

\item[iii)] Para cualquier $x\in E$,$\CM\left\{X_{0}=x\right\}=1$, y el proceso $\PE$ tiene la propiedad de Markov (\ref{Eq.1.8.S}) con semigrupo de transici\'on $\SG$ relativo a $\left(\Omega,\mathcal{G},\mathcal{G}_{t},\CM\right)$.
\end{itemize}
\end{Def}

\begin{Def}[HD2]\label{Eq.2.2.S}
Para cualquier $\alpha>0$ y cualquier $f\in S^{\alpha}$, el proceso $t\rightarrow f\left(X_{t}\right)$ es continuo por la derecha casi seguramente.
\end{Def}

\begin{Def}\label{Def.PD}
Un sistema $\mathbf{X}=\left(\Omega,\mathcal{G},\mathcal{G}_{t},X_{t},\theta_{t},\CM\right)$ es un proceso derecho en el espacio de Rad\'on $E$ con semigrupo de transici\'on $\SG$ provisto de:
\begin{itemize}
\item[i)] $\mathbf{X}$ es una realizaci\'on  continua por la derecha, \ref{Def.2.2.S}, de $\SG$.

\item[ii)] $\mathbf{X}$ satisface la condicion HD2, \ref{Eq.2.2.S}, relativa a $\mathcal{G}_{t}$.

\item[iii)] $\mathcal{G}_{t}$ es aumentado y continuo por la derecha.
\end{itemize}
\end{Def}



\begin{Lema}[Lema 4.2, Dai\cite{Dai}]\label{Lema4.2}
Sea $\left\{x_{n}\right\}\subset \mathbf{X}$ con
$|x_{n}|\rightarrow\infty$, conforme $n\rightarrow\infty$. Suponga
que
\[lim_{n\rightarrow\infty}\frac{1}{|x_{n}|}U\left(0\right)=\overline{U}\]
y
\[lim_{n\rightarrow\infty}\frac{1}{|x_{n}|}V\left(0\right)=\overline{V}.\]

Entonces, conforme $n\rightarrow\infty$, casi seguramente

\begin{equation}\label{E1.4.2}
\frac{1}{|x_{n}|}\Phi^{k}\left(\left[|x_{n}|t\right]\right)\rightarrow
P_{k}^{'}t\textrm{, u.o.c.,}
\end{equation}

\begin{equation}\label{E1.4.3}
\frac{1}{|x_{n}|}E^{x_{n}}_{k}\left(|x_{n}|t\right)\rightarrow
\alpha_{k}\left(t-\overline{U}_{k}\right)^{+}\textrm{, u.o.c.,}
\end{equation}

\begin{equation}\label{E1.4.4}
\frac{1}{|x_{n}|}S^{x_{n}}_{k}\left(|x_{n}|t\right)\rightarrow
\mu_{k}\left(t-\overline{V}_{k}\right)^{+}\textrm{, u.o.c.,}
\end{equation}

donde $\left[t\right]$ es la parte entera de $t$ y
$\mu_{k}=1/m_{k}=1/\esp\left[\eta_{k}\left(1\right)\right]$.
\end{Lema}

\begin{Lema}[Lema 4.3, Dai\cite{Dai}]\label{Lema.4.3}
Sea $\left\{x_{n}\right\}\subset \mathbf{X}$ con
$|x_{n}|\rightarrow\infty$, conforme $n\rightarrow\infty$. Suponga
que
\[lim_{n\rightarrow\infty}\frac{1}{|x_{n}|}U\left(0\right)=\overline{U}_{k}\]
y
\[lim_{n\rightarrow\infty}\frac{1}{|x_{n}|}V\left(0\right)=\overline{V}_{k}.\]
\begin{itemize}
\item[a)] Conforme $n\rightarrow\infty$ casi seguramente,
\[lim_{n\rightarrow\infty}\frac{1}{|x_{n}|}U^{x_{n}}_{k}\left(|x_{n}|t\right)=\left(\overline{U}_{k}-t\right)^{+}\textrm{, u.o.c.}\]
y
\[lim_{n\rightarrow\infty}\frac{1}{|x_{n}|}V^{x_{n}}_{k}\left(|x_{n}|t\right)=\left(\overline{V}_{k}-t\right)^{+}.\]

\item[b)] Para cada $t\geq0$ fijo,
\[\left\{\frac{1}{|x_{n}|}U^{x_{n}}_{k}\left(|x_{n}|t\right),|x_{n}|\geq1\right\}\]
y
\[\left\{\frac{1}{|x_{n}|}V^{x_{n}}_{k}\left(|x_{n}|t\right),|x_{n}|\geq1\right\}\]
\end{itemize}
son uniformemente convergentes.
\end{Lema}

$S_{l}^{x}\left(t\right)$ es el n\'umero total de servicios
completados de la clase $l$, si la clase $l$ est\'a dando $t$
unidades de tiempo de servicio. Sea $T_{l}^{x}\left(x\right)$ el
monto acumulado del tiempo de servicio que el servidor
$s\left(l\right)$ gasta en los usuarios de la clase $l$ al tiempo
$t$. Entonces $S_{l}^{x}\left(T_{l}^{x}\left(t\right)\right)$ es
el n\'umero total de servicios completados para la clase $l$ al
tiempo $t$. Una fracci\'on de estos usuarios,
$\Phi_{l}^{x}\left(S_{l}^{x}\left(T_{l}^{x}\left(t\right)\right)\right)$,
se convierte en usuarios de la clase $k$.\\

Entonces, dado lo anterior, se tiene la siguiente representaci\'on
para el proceso de la longitud de la cola:\\

\begin{equation}
Q_{k}^{x}\left(t\right)=_{k}^{x}\left(0\right)+E_{k}^{x}\left(t\right)+\sum_{l=1}^{K}\Phi_{k}^{l}\left(S_{l}^{x}\left(T_{l}^{x}\left(t\right)\right)\right)-S_{k}^{x}\left(T_{k}^{x}\left(t\right)\right)
\end{equation}
para $k=1,\ldots,K$. Para $i=1,\ldots,d$, sea
\[I_{i}^{x}\left(t\right)=t-\sum_{j\in C_{i}}T_{k}^{x}\left(t\right).\]

Entonces $I_{i}^{x}\left(t\right)$ es el monto acumulado del
tiempo que el servidor $i$ ha estado desocupado al tiempo $t$. Se
est\'a asumiendo que las disciplinas satisfacen la ley de
conservaci\'on del trabajo, es decir, el servidor $i$ est\'a en
pausa solamente cuando no hay usuarios en la estaci\'on $i$.
Entonces, se tiene que

\begin{equation}
\int_{0}^{\infty}\left(\sum_{k\in
C_{i}}Q_{k}^{x}\left(t\right)\right)dI_{i}^{x}\left(t\right)=0,
\end{equation}
para $i=1,\ldots,d$.\\

Hacer
\[T^{x}\left(t\right)=\left(T_{1}^{x}\left(t\right),\ldots,T_{K}^{x}\left(t\right)\right)^{'},\]
\[I^{x}\left(t\right)=\left(I_{1}^{x}\left(t\right),\ldots,I_{K}^{x}\left(t\right)\right)^{'}\]
y
\[S^{x}\left(T^{x}\left(t\right)\right)=\left(S_{1}^{x}\left(T_{1}^{x}\left(t\right)\right),\ldots,S_{K}^{x}\left(T_{K}^{x}\left(t\right)\right)\right)^{'}.\]

Para una disciplina que cumple con la ley de conservaci\'on del
trabajo, en forma vectorial, se tiene el siguiente conjunto de
ecuaciones

\begin{equation}\label{Eq.MF.1.3}
Q^{x}\left(t\right)=Q^{x}\left(0\right)+E^{x}\left(t\right)+\sum_{l=1}^{K}\Phi^{l}\left(S_{l}^{x}\left(T_{l}^{x}\left(t\right)\right)\right)-S^{x}\left(T^{x}\left(t\right)\right),\\
\end{equation}

\begin{equation}\label{Eq.MF.2.3}
Q^{x}\left(t\right)\geq0,\\
\end{equation}

\begin{equation}\label{Eq.MF.3.3}
T^{x}\left(0\right)=0,\textrm{ y }\overline{T}^{x}\left(t\right)\textrm{ es no decreciente},\\
\end{equation}

\begin{equation}\label{Eq.MF.4.3}
I^{x}\left(t\right)=et-CT^{x}\left(t\right)\textrm{ es no
decreciente}\\
\end{equation}

\begin{equation}\label{Eq.MF.5.3}
\int_{0}^{\infty}\left(CQ^{x}\left(t\right)\right)dI_{i}^{x}\left(t\right)=0,\\
\end{equation}

\begin{equation}\label{Eq.MF.6.3}
\textrm{Condiciones adicionales en
}\left(\overline{Q}^{x}\left(\cdot\right),\overline{T}^{x}\left(\cdot\right)\right)\textrm{
espec\'ificas de la disciplina de la cola,}
\end{equation}

donde $e$ es un vector de unos de dimensi\'on $d$, $C$ es la
matriz definida por
\[C_{ik}=\left\{\begin{array}{cc}
1,& S\left(k\right)=i,\\
0,& \textrm{ en otro caso}.\\
\end{array}\right.
\]
Es necesario enunciar el siguiente Teorema que se utilizar\'a para
el Teorema \ref{Tma.4.2.Dai}:
\begin{Teo}[Teorema 4.1, Dai \cite{Dai}]
Considere una disciplina que cumpla la ley de conservaci\'on del
trabajo, para casi todas las trayectorias muestrales $\omega$ y
cualquier sucesi\'on de estados iniciales
$\left\{x_{n}\right\}\subset \mathbf{X}$, con
$|x_{n}|\rightarrow\infty$, existe una subsucesi\'on
$\left\{x_{n_{j}}\right\}$ con $|x_{n_{j}}|\rightarrow\infty$ tal
que
\begin{equation}\label{Eq.4.15}
\frac{1}{|x_{n_{j}}|}\left(Q^{x_{n_{j}}}\left(0\right),U^{x_{n_{j}}}\left(0\right),V^{x_{n_{j}}}\left(0\right)\right)\rightarrow\left(\overline{Q}\left(0\right),\overline{U},\overline{V}\right),
\end{equation}

\begin{equation}\label{Eq.4.16}
\frac{1}{|x_{n_{j}}|}\left(Q^{x_{n_{j}}}\left(|x_{n_{j}}|t\right),T^{x_{n_{j}}}\left(|x_{n_{j}}|t\right)\right)\rightarrow\left(\overline{Q}\left(t\right),\overline{T}\left(t\right)\right)\textrm{
u.o.c.}
\end{equation}

Adem\'as,
$\left(\overline{Q}\left(t\right),\overline{T}\left(t\right)\right)$
satisface las siguientes ecuaciones:
\begin{equation}\label{Eq.MF.1.3a}
\overline{Q}\left(t\right)=Q\left(0\right)+\left(\alpha
t-\overline{U}\right)^{+}-\left(I-P\right)^{'}M^{-1}\left(\overline{T}\left(t\right)-\overline{V}\right)^{+},
\end{equation}

\begin{equation}\label{Eq.MF.2.3a}
\overline{Q}\left(t\right)\geq0,\\
\end{equation}

\begin{equation}\label{Eq.MF.3.3a}
\overline{T}\left(t\right)\textrm{ es no decreciente y comienza en cero},\\
\end{equation}

\begin{equation}\label{Eq.MF.4.3a}
\overline{I}\left(t\right)=et-C\overline{T}\left(t\right)\textrm{
es no decreciente,}\\
\end{equation}

\begin{equation}\label{Eq.MF.5.3a}
\int_{0}^{\infty}\left(C\overline{Q}\left(t\right)\right)d\overline{I}\left(t\right)=0,\\
\end{equation}

\begin{equation}\label{Eq.MF.6.3a}
\textrm{Condiciones adicionales en
}\left(\overline{Q}\left(\cdot\right),\overline{T}\left(\cdot\right)\right)\textrm{
especficas de la disciplina de la cola,}
\end{equation}
\end{Teo}

\begin{Def}[Definici\'on 4.1, , Dai \cite{Dai}]
Sea una disciplina de servicio espec\'ifica. Cualquier l\'imite
$\left(\overline{Q}\left(\cdot\right),\overline{T}\left(\cdot\right)\right)$
en \ref{Eq.4.16} es un {\em flujo l\'imite} de la disciplina.
Cualquier soluci\'on (\ref{Eq.MF.1.3a})-(\ref{Eq.MF.6.3a}) es
llamado flujo soluci\'on de la disciplina. Se dice que el modelo de flujo l\'imite, modelo de flujo, de la disciplina de la cola es estable si existe una constante
$\delta>0$ que depende de $\mu,\alpha$ y $P$ solamente, tal que
cualquier flujo l\'imite con
$|\overline{Q}\left(0\right)|+|\overline{U}|+|\overline{V}|=1$, se
tiene que $\overline{Q}\left(\cdot+\delta\right)\equiv0$.
\end{Def}

\begin{Teo}[Teorema 4.2, Dai\cite{Dai}]\label{Tma.4.2.Dai}
Sea una disciplina fija para la cola, suponga que se cumplen las
condiciones (1.2)-(1.5). Si el modelo de flujo l\'imite de la
disciplina de la cola es estable, entonces la cadena de Markov $X$
que describe la din\'amica de la red bajo la disciplina es Harris
recurrente positiva.
\end{Teo}

Ahora se procede a escalar el espacio y el tiempo para reducir la
aparente fluctuaci\'on del modelo. Consid\'erese el proceso
\begin{equation}\label{Eq.3.7}
\overline{Q}^{x}\left(t\right)=\frac{1}{|x|}Q^{x}\left(|x|t\right)
\end{equation}
A este proceso se le conoce como el fluido escalado, y cualquier l\'imite $\overline{Q}^{x}\left(t\right)$ es llamado flujo l\'imite del proceso de longitud de la cola. Haciendo $|q|\rightarrow\infty$ mientras se mantiene el resto de las componentes fijas, cualquier punto l\'imite del proceso de longitud de la cola normalizado $\overline{Q}^{x}$ es soluci\'on del siguiente modelo de flujo.

Al conjunto de ecuaciones dadas en \ref{Eq.3.8}-\ref{Eq.3.13} se
le llama {\em Modelo de flujo} y al conjunto de todas las
soluciones del modelo de flujo
$\left(\overline{Q}\left(\cdot\right),\overline{T}
\left(\cdot\right)\right)$ se le denotar\'a por $\mathcal{Q}$.

Si se hace $|x|\rightarrow\infty$ sin restringir ninguna de las
componentes, tambi\'en se obtienen un modelo de flujo, pero en
este caso el residual de los procesos de arribo y servicio
introducen un retraso:

\begin{Def}[Definici\'on 3.3, Dai y Meyn \cite{DaiSean}]
El modelo de flujo es estable si existe un tiempo fijo $t_{0}$ tal
que $\overline{Q}\left(t\right)=0$, con $t\geq t_{0}$, para
cualquier $\overline{Q}\left(\cdot\right)\in\mathcal{Q}$ que
cumple con $|\overline{Q}\left(0\right)|=1$.
\end{Def}

El siguiente resultado se encuentra en Chen \cite{Chen}.
\begin{Lemma}[Lema 3.1, Dai y Meyn \cite{DaiSean}]
Si el modelo de flujo definido por \ref{Eq.3.8}-\ref{Eq.3.13} es
estable, entonces el modelo de flujo retrasado es tambi\'en
estable, es decir, existe $t_{0}>0$ tal que
$\overline{Q}\left(t\right)=0$ para cualquier $t\geq t_{0}$, para
cualquier soluci\'on del modelo de flujo retrasado cuya
condici\'on inicial $\overline{x}$ satisface que
$|\overline{x}|=|\overline{Q}\left(0\right)|+|\overline{A}\left(0\right)|+|\overline{B}\left(0\right)|\leq1$.
\end{Lemma}


Propiedades importantes para el modelo de flujo retrasado:

\begin{Prop}
 Sea $\left(\overline{Q},\overline{T},\overline{T}^{0}\right)$ un flujo l\'imite de \ref{Eq.4.4} y suponga que cuando $x\rightarrow\infty$ a lo largo de
una subsucesi\'on
\[\left(\frac{1}{|x|}Q_{k}^{x}\left(0\right),\frac{1}{|x|}A_{k}^{x}\left(0\right),\frac{1}{|x|}B_{k}^{x}\left(0\right),\frac{1}{|x|}B_{k}^{x,0}\left(0\right)\right)\rightarrow\left(\overline{Q}_{k}\left(0\right),0,0,0\right)\]
para $k=1,\ldots,K$. EL flujo l\'imite tiene las siguientes
propiedades, donde las propiedades de la derivada se cumplen donde
la derivada exista:
\begin{itemize}
 \item[i)] Los vectores de tiempo ocupado $\overline{T}\left(t\right)$ y $\overline{T}^{0}\left(t\right)$ son crecientes y continuas con
$\overline{T}\left(0\right)=\overline{T}^{0}\left(0\right)=0$.
\item[ii)] Para todo $t\geq0$
\[\sum_{k=1}^{K}\left[\overline{T}_{k}\left(t\right)+\overline{T}_{k}^{0}\left(t\right)\right]=t\]
\item[iii)] Para todo $1\leq k\leq K$
\[\overline{Q}_{k}\left(t\right)=\overline{Q}_{k}\left(0\right)+\alpha_{k}t-\mu_{k}\overline{T}_{k}\left(t\right)\]
\item[iv)]  Para todo $1\leq k\leq K$
\[\dot{{\overline{T}}}_{k}\left(t\right)=\beta_{k}\] para $\overline{Q}_{k}\left(t\right)=0$.
\item[v)] Para todo $k,j$
\[\mu_{k}^{0}\overline{T}_{k}^{0}\left(t\right)=\mu_{j}^{0}\overline{T}_{j}^{0}\left(t\right)\]
\item[vi)]  Para todo $1\leq k\leq K$
\[\mu_{k}\dot{{\overline{T}}}_{k}\left(t\right)=l_{k}\mu_{k}^{0}\dot{{\overline{T}}}_{k}^{0}\left(t\right)\] para $\overline{Q}_{k}\left(t\right)>0$.
\end{itemize}
\end{Prop}

\begin{Lema}[Lema 3.1 \cite{Chen}]\label{Lema3.1}
Si el modelo de flujo es estable, definido por las ecuaciones
(3.8)-(3.13), entonces el modelo de flujo retrasado tambin es
estable.
\end{Lema}

\begin{Teo}[Teorema 5.2 \cite{Chen}]\label{Tma.5.2}
Si el modelo de flujo lineal correspondiente a la red de cola es
estable, entonces la red de colas es estable.
\end{Teo}

\begin{Teo}[Teorema 5.1 \cite{Chen}]\label{Tma.5.1.Chen}
La red de colas es estable si existe una constante $t_{0}$ que
depende de $\left(\alpha,\mu,T,U\right)$ y $V$ que satisfagan las
ecuaciones (5.1)-(5.5), $Z\left(t\right)=0$, para toda $t\geq
t_{0}$.
\end{Teo}



\begin{Lema}[Lema 5.2 \cite{Gut}]\label{Lema.5.2.Gut}
Sea $\left\{\xi\left(k\right):k\in\ent\right\}$ sucesin de
variables aleatorias i.i.d. con valores en
$\left(0,\infty\right)$, y sea $E\left(t\right)$ el proceso de
conteo
\[E\left(t\right)=max\left\{n\geq1:\xi\left(1\right)+\cdots+\xi\left(n-1\right)\leq t\right\}.\]
Si $E\left[\xi\left(1\right)\right]<\infty$, entonces para
cualquier entero $r\geq1$
\begin{equation}
lim_{t\rightarrow\infty}\esp\left[\left(\frac{E\left(t\right)}{t}\right)^{r}\right]=\left(\frac{1}{E\left[\xi_{1}\right]}\right)^{r}
\end{equation}
de aqu, bajo estas condiciones
\begin{itemize}
\item[a)] Para cualquier $t>0$,
$sup_{t\geq\delta}\esp\left[\left(\frac{E\left(t\right)}{t}\right)^{r}\right]$

\item[b)] Las variables aleatorias
$\left\{\left(\frac{E\left(t\right)}{t}\right)^{r}:t\geq1\right\}$
son uniformemente integrables.
\end{itemize}
\end{Lema}

\begin{Teo}[Teorema 5.1: Ley Fuerte para Procesos de Conteo
\cite{Gut}]\label{Tma.5.1.Gut} Sea
$0<\mu<\esp\left(X_{1}\right]\leq\infty$. entonces

\begin{itemize}
\item[a)] $\frac{N\left(t\right)}{t}\rightarrow\frac{1}{\mu}$
a.s., cuando $t\rightarrow\infty$.


\item[b)]$\esp\left[\frac{N\left(t\right)}{t}\right]^{r}\rightarrow\frac{1}{\mu^{r}}$,
cuando $t\rightarrow\infty$ para todo $r>0$..
\end{itemize}
\end{Teo}


\begin{Prop}[Proposicin 5.1 \cite{DaiSean}]\label{Prop.5.1}
Suponga que los supuestos (A1) y (A2) se cumplen, adems suponga
que el modelo de flujo es estable. Entonces existe $t_{0}>0$ tal
que
\begin{equation}\label{Eq.Prop.5.1}
lim_{|x|\rightarrow\infty}\frac{1}{|x|^{p+1}}\esp_{x}\left[|X\left(t_{0}|x|\right)|^{p+1}\right]=0.
\end{equation}

\end{Prop}


\begin{Prop}[Proposici\'on 5.3 \cite{DaiSean}]
Sea $X$ proceso de estados para la red de colas, y suponga que se
cumplen los supuestos (A1) y (A2), entonces para alguna constante
positiva $C_{p+1}<\infty$, $\delta>0$ y un conjunto compacto
$C\subset X$.

\begin{equation}\label{Eq.5.4}
\esp_{x}\left[\int_{0}^{\tau_{C}\left(\delta\right)}\left(1+|X\left(t\right)|^{p}\right)dt\right]\leq
C_{p+1}\left(1+|x|^{p+1}\right)
\end{equation}
\end{Prop}

\begin{Prop}[Proposici\'on 5.4 \cite{DaiSean}]
Sea $X$ un proceso de Markov Borel Derecho en $X$, sea
$f:X\leftarrow\rea_{+}$ y defina para alguna $\delta>0$, y un
conjunto cerrado $C\subset X$
\[V\left(x\right):=\esp_{x}\left[\int_{0}^{\tau_{C}\left(\delta\right)}f\left(X\left(t\right)\right)dt\right]\]
para $x\in X$. Si $V$ es finito en todas partes y uniformemente
acotada en $C$, entonces existe $k<\infty$ tal que
\begin{equation}\label{Eq.5.11}
\frac{1}{t}\esp_{x}\left[V\left(x\right)\right]+\frac{1}{t}\int_{0}^{t}\esp_{x}\left[f\left(X\left(s\right)\right)ds\right]\leq\frac{1}{t}V\left(x\right)+k,
\end{equation}
para $x\in X$ y $t>0$.
\end{Prop}


\begin{Teo}[Teorema 5.5 \cite{DaiSean}]
Suponga que se cumplen (A1) y (A2), adems suponga que el modelo
de flujo es estable. Entonces existe una constante $k_{p}<\infty$
tal que
\begin{equation}\label{Eq.5.13}
\frac{1}{t}\int_{0}^{t}\esp_{x}\left[|Q\left(s\right)|^{p}\right]ds\leq
k_{p}\left\{\frac{1}{t}|x|^{p+1}+1\right\}
\end{equation}
para $t\geq0$, $x\in X$. En particular para cada condici\'on inicial
\begin{equation}\label{Eq.5.14}
Limsup_{t\rightarrow\infty}\frac{1}{t}\int_{0}^{t}\esp_{x}\left[|Q\left(s\right)|^{p}\right]ds\leq
k_{p}
\end{equation}
\end{Teo}

\begin{Teo}[Teorema 6.2\cite{DaiSean}]\label{Tma.6.2}
Suponga que se cumplen los supuestos (A1)-(A3) y que el modelo de
flujo es estable, entonces se tiene que
\[\parallel P^{t}\left(c,\cdot\right)-\pi\left(\cdot\right)\parallel_{f_{p}}\rightarrow0\]
para $t\rightarrow\infty$ y $x\in X$. En particular para cada
condicin inicial
\[lim_{t\rightarrow\infty}\esp_{x}\left[\left|Q_{t}\right|^{p}\right]=\esp_{\pi}\left[\left|Q_{0}\right|^{p}\right]<\infty\]
\end{Teo}


\begin{Teo}[Teorema 6.3\cite{DaiSean}]\label{Tma.6.3}
Suponga que se cumplen los supuestos (A1)-(A3) y que el modelo de
flujo es estable, entonces con
$f\left(x\right)=f_{1}\left(x\right)$, se tiene que
\[lim_{t\rightarrow\infty}t^{(p-1)\left|P^{t}\left(c,\cdot\right)-\pi\left(\cdot\right)\right|_{f}=0},\]
para $x\in X$. En particular, para cada condicin inicial
\[lim_{t\rightarrow\infty}t^{(p-1)\left|\esp_{x}\left[Q_{t}\right]-\esp_{\pi}\left[Q_{0}\right]\right|=0}.\]
\end{Teo}



\begin{Prop}[Proposici\'on 5.1, Dai y Meyn \cite{DaiSean}]\label{Prop.5.1.DaiSean}
Suponga que los supuestos A1) y A2) son ciertos y que el modelo de flujo es estable. Entonces existe $t_{0}>0$ tal que
\begin{equation}
lim_{|x|\rightarrow\infty}\frac{1}{|x|^{p+1}}\esp_{x}\left[|X\left(t_{0}|x|\right)|^{p+1}\right]=0
\end{equation}
\end{Prop}

\begin{Lemma}[Lema 5.2, Dai y Meyn \cite{DaiSean}]\label{Lema.5.2.DaiSean}
 Sea $\left\{\zeta\left(k\right):k\in \mathbb{z}\right\}$ una sucesi\'on independiente e id\'enticamente distribuida que toma valores en $\left(0,\infty\right)$,
y sea
$E\left(t\right)=max\left(n\geq1:\zeta\left(1\right)+\cdots+\zeta\left(n-1\right)\leq
t\right)$. Si $\esp\left[\zeta\left(1\right)\right]<\infty$,
entonces para cualquier entero $r\geq1$
\begin{equation}
 lim_{t\rightarrow\infty}\esp\left[\left(\frac{E\left(t\right)}{t}\right)^{r}\right]=\left(\frac{1}{\esp\left[\zeta_{1}\right]}\right)^{r}.
\end{equation}
Luego, bajo estas condiciones:
\begin{itemize}
 \item[a)] para cualquier $\delta>0$, $\sup_{t\geq\delta}\esp\left[\left(\frac{E\left(t\right)}{t}\right)^{r}\right]<\infty$
\item[b)] las variables aleatorias
$\left\{\left(\frac{E\left(t\right)}{t}\right)^{r}:t\geq1\right\}$
son uniformemente integrables.
\end{itemize}
\end{Lemma}

\begin{Teo}[Teorema 5.5, Dai y Meyn \cite{DaiSean}]\label{Tma.5.5.DaiSean}
Suponga que los supuestos A1) y A2) se cumplen y que el modelo de
flujo es estable. Entonces existe una constante $\kappa_{p}$ tal
que
\begin{equation}
\frac{1}{t}\int_{0}^{t}\esp_{x}\left[|Q\left(s\right)|^{p}\right]ds\leq\kappa_{p}\left\{\frac{1}{t}|x|^{p+1}+1\right\}
\end{equation}
para $t>0$ y $x\in X$. En particular, para cada condici\'on
inicial
\begin{eqnarray*}
\limsup_{t\rightarrow\infty}\frac{1}{t}\int_{0}^{t}\esp_{x}\left[|Q\left(s\right)|^{p}\right]ds\leq\kappa_{p}.
\end{eqnarray*}
\end{Teo}

\begin{Teo}[Teorema 6.2, Dai y Meyn \cite{DaiSean}]\label{Tma.6.2.DaiSean}
Suponga que se cumplen los supuestos A1), A2) y A3) y que el
modelo de flujo es estable. Entonces se tiene que
\begin{equation}
\left\|P^{t}\left(x,\cdot\right)-\pi\left(\cdot\right)\right\|_{f_{p}}\textrm{,
}t\rightarrow\infty,x\in X.
\end{equation}
En particular para cada condici\'on inicial
\begin{eqnarray*}
\lim_{t\rightarrow\infty}\esp_{x}\left[|Q\left(t\right)|^{p}\right]=\esp_{\pi}\left[|Q\left(0\right)|^{p}\right]\leq\kappa_{r}
\end{eqnarray*}
\end{Teo}
\begin{Teo}[Teorema 6.3, Dai y Meyn \cite{DaiSean}]\label{Tma.6.3.DaiSean}
Suponga que se cumplen los supuestos A1), A2) y A3) y que el
modelo de flujo es estable. Entonces con
$f\left(x\right)=f_{1}\left(x\right)$ se tiene
\begin{equation}
\lim_{t\rightarrow\infty}t^{p-1}\left\|P^{t}\left(x,\cdot\right)-\pi\left(\cdot\right)\right\|_{f}=0.
\end{equation}
En particular para cada condici\'on inicial
\begin{eqnarray*}
\lim_{t\rightarrow\infty}t^{p-1}|\esp_{x}\left[Q\left(t\right)\right]-\esp_{\pi}\left[Q\left(0\right)\right]|=0.
\end{eqnarray*}
\end{Teo}

\begin{Teo}[Teorema 6.4, Dai y Meyn \cite{DaiSean}]\label{Tma.6.4.DaiSean}
Suponga que se cumplen los supuestos A1), A2) y A3) y que el
modelo de flujo es estable. Sea $\nu$ cualquier distribuci\'on de
probabilidad en $\left(X,\mathcal{B}_{X}\right)$, y $\pi$ la
distribuci\'on estacionaria de $X$.
\begin{itemize}
\item[i)] Para cualquier $f:X\leftarrow\rea_{+}$
\begin{equation}
\lim_{t\rightarrow\infty}\frac{1}{t}\int_{o}^{t}f\left(X\left(s\right)\right)ds=\pi\left(f\right):=\int
f\left(x\right)\pi\left(dx\right)
\end{equation}
$\prob$-c.s.

\item[ii)] Para cualquier $f:X\leftarrow\rea_{+}$ con
$\pi\left(|f|\right)<\infty$, la ecuaci\'on anterior se cumple.
\end{itemize}
\end{Teo}

\begin{Teo}[Teorema 2.2, Down \cite{Down}]\label{Tma2.2.Down}
Suponga que el fluido modelo es inestable en el sentido de que
para alguna $\epsilon_{0},c_{0}\geq0$,
\begin{equation}\label{Eq.Inestability}
|Q\left(T\right)|\geq\epsilon_{0}T-c_{0}\textrm{,   }T\geq0,
\end{equation}
para cualquier condici\'on inicial $Q\left(0\right)$, con
$|Q\left(0\right)|=1$. Entonces para cualquier $0<q\leq1$, existe
$B<0$ tal que para cualquier $|x|\geq B$,
\begin{equation}
\prob_{x}\left\{\mathbb{X}\rightarrow\infty\right\}\geq q.
\end{equation}
\end{Teo}


%_________________________________________________________________________
\subsection{Supuestos}
%_________________________________________________________________________
Consideremos el caso en el que se tienen varias colas a las cuales
llegan uno o varios servidores para dar servicio a los usuarios
que se encuentran presentes en la cola, como ya se mencion\'o hay
varios tipos de pol\'iticas de servicio, incluso podr\'ia ocurrir
que la manera en que atiende al resto de las colas sea distinta a
como lo hizo en las anteriores.\\

Para ejemplificar los sistemas de visitas c\'iclicas se
considerar\'a el caso en que a las colas los usuarios son atendidos con
una s\'ola pol\'itica de servicio.\\


Si $\omega$ es el n\'umero de usuarios en la cola al comienzo del
periodo de servicio y $N\left(\omega\right)$ es el n\'umero de
usuarios que son atendidos con una pol\'itica en espec\'ifico
durante el periodo de servicio, entonces se asume que:
\begin{itemize}
\item[1)]\label{S1}$lim_{\omega\rightarrow\infty}\esp\left[N\left(\omega\right)\right]=\overline{N}>0$;
\item[2)]\label{S2}$\esp\left[N\left(\omega\right)\right]\leq\overline{N}$
para cualquier valor de $\omega$.
\end{itemize}
La manera en que atiende el servidor $m$-\'esimo, es la siguiente:
\begin{itemize}
\item Al t\'ermino de la visita a la cola $j$, el servidor cambia
a la cola $j^{'}$ con probabilidad $r_{j,j^{'}}^{m}$

\item La $n$-\'esima vez que el servidor cambia de la cola $j$ a
$j'$, va acompa\~nada con el tiempo de cambio de longitud
$\delta_{j,j^{'}}^{m}\left(n\right)$, con
$\delta_{j,j^{'}}^{m}\left(n\right)$, $n\geq1$, variables
aleatorias independientes e id\'enticamente distribuidas, tales
que $\esp\left[\delta_{j,j^{'}}^{m}\left(1\right)\right]\geq0$.

\item Sea $\left\{p_{j}^{m}\right\}$ la distribuci\'on invariante
estacionaria \'unica para la Cadena de Markov con matriz de
transici\'on $\left(r_{j,j^{'}}^{m}\right)$, se supone que \'esta
existe.

\item Finalmente, se define el tiempo promedio total de traslado
entre las colas como
\begin{equation}
\delta^{*}:=\sum_{j,j^{'}}p_{j}^{m}r_{j,j^{'}}^{m}\esp\left[\delta_{j,j^{'}}^{m}\left(i\right)\right].
\end{equation}
\end{itemize}

Consideremos el caso donde los tiempos entre arribo a cada una de
las colas, $\left\{\xi_{k}\left(n\right)\right\}_{n\geq1}$ son
variables aleatorias independientes a id\'enticamente
distribuidas, y los tiempos de servicio en cada una de las colas
se distribuyen de manera independiente e id\'enticamente
distribuidas $\left\{\eta_{k}\left(n\right)\right\}_{n\geq1}$;
adem\'as ambos procesos cumplen la condici\'on de ser
independientes entre s\'i. Para la $k$-\'esima cola se define la
tasa de arribo por
$\lambda_{k}=1/\esp\left[\xi_{k}\left(1\right)\right]$ y la tasa
de servicio como
$\mu_{k}=1/\esp\left[\eta_{k}\left(1\right)\right]$, finalmente se
define la carga de la cola como $\rho_{k}=\lambda_{k}/\mu_{k}$,
donde se pide que $\rho=\sum_{k=1}^{K}\rho_{k}<1$, para garantizar
la estabilidad del sistema, esto es cierto para las pol\'iticas de
servicio exhaustiva y cerrada, ver Geetor \cite{Getoor}.\\

Si denotamos por
\begin{itemize}
\item $Q_{k}\left(t\right)$ el n\'umero de usuarios presentes en
la cola $k$ al tiempo $t$; \item $A_{k}\left(t\right)$ los
residuales de los tiempos entre arribos a la cola $k$; para cada
servidor $m$; \item $B_{m}\left(t\right)$ denota a los residuales
de los tiempos de servicio al tiempo $t$; \item
$B_{m}^{0}\left(t\right)$ los residuales de los tiempos de
traslado de la cola $k$ a la pr\'oxima por atender al tiempo $t$,

\item sea
$C_{m}\left(t\right)$ el n\'umero de usuarios atendidos durante la
visita del servidor a la cola $k$ al tiempo $t$.
\end{itemize}


En este sentido, el proceso para el sistema de visitas se puede
definir como:

\begin{equation}\label{Esp.Edos.Down}
X\left(t\right)^{T}=\left(Q_{k}\left(t\right),A_{k}\left(t\right),B_{m}\left(t\right),B_{m}^{0}\left(t\right),C_{m}\left(t\right)\right),
\end{equation}
para $k=1,\ldots,K$ y $m=1,2,\ldots,M$, donde $T$ indica que es el
transpuesto del vector que se est\'a definiendo. El proceso $X$
evoluciona en el espacio de estados:
$\mathbb{X}=\ent_{+}^{K}\times\rea_{+}^{K}\times\left(\left\{1,2,\ldots,K\right\}\times\left\{1,2,\ldots,S\right\}\right)^{M}\times\rea_{+}^{K}\times\ent_{+}^{K}$.\\

El sistema aqu\'i descrito debe de cumplir con los siguientes supuestos b\'asicos de un sistema de visitas:
%__________________________________________________________________________
\subsubsection{Supuestos B\'asicos}
%__________________________________________________________________________
\begin{itemize}
\item[A1)] Los procesos
$\xi_{1},\ldots,\xi_{K},\eta_{1},\ldots,\eta_{K}$ son mutuamente
independientes y son sucesiones independientes e id\'enticamente
distribuidas.

\item[A2)] Para alg\'un entero $p\geq1$
\begin{eqnarray*}
\esp\left[\xi_{l}\left(1\right)^{p+1}\right]&<&\infty\textrm{ para }l=1,\ldots,K\textrm{ y }\\
\esp\left[\eta_{k}\left(1\right)^{p+1}\right]&<&\infty\textrm{
para }k=1,\ldots,K.
\end{eqnarray*}
donde $\mathcal{A}$ es la clase de posibles arribos.

\item[A3)] Para cada $k=1,2,\ldots,K$ existe una funci\'on
positiva $q_{k}\left(\cdot\right)$ definida en $\rea_{+}$, y un
entero $j_{k}$, tal que
\begin{eqnarray}
P\left(\xi_{k}\left(1\right)\geq x\right)&>&0\textrm{, para todo }x>0,\\
P\left\{a\leq\sum_{i=1}^{j_{k}}\xi_{k}\left(i\right)\leq
b\right\}&\geq&\int_{a}^{b}q_{k}\left(x\right)dx, \textrm{ }0\leq
a<b.
\end{eqnarray}
\end{itemize}

En lo que respecta al supuesto (A3), en Dai y Meyn \cite{DaiSean}
hacen ver que este se puede sustituir por

\begin{itemize}
\item[A3')] Para el Proceso de Markov $X$, cada subconjunto
compacto del espacio de estados de $X$ es un conjunto peque\~no,
ver definici\'on \ref{Def.Cto.Peq.}.
\end{itemize}

Es por esta raz\'on que con la finalidad de poder hacer uso de
$A3^{'})$ es necesario recurrir a los Procesos de Harris y en
particular a los Procesos Harris Recurrente, ver \cite{Dai,
DaiSean}.
%_______________________________________________________________________
\subsection{Procesos Harris Recurrente}
%_______________________________________________________________________

Por el supuesto (A1) conforme a Davis \cite{Davis}, se puede
definir el proceso de saltos correspondiente de manera tal que
satisfaga el supuesto (A3'), de hecho la demostraci\'on est\'a
basada en la l\'inea de argumentaci\'on de Davis, \cite{Davis},
p\'aginas 362-364.\\

Entonces se tiene un espacio de estados en el cual el proceso $X$
satisface la Propiedad Fuerte de Markov, ver Dai y Meyn
\cite{DaiSean}, dado por

\[\left(\Omega,\mathcal{F},\mathcal{F}_{t},X\left(t\right),\theta_{t},P_{x}\right),\]
adem\'as de ser un proceso de Borel Derecho (Sharpe \cite{Sharpe})
en el espacio de estados medible
$\left(\mathbb{X},\mathcal{B}_\mathbb{X}\right)$. El Proceso
$X=\left\{X\left(t\right),t\geq0\right\}$ tiene trayectorias
continuas por la derecha, est\'a definido en
$\left(\Omega,\mathcal{F}\right)$ y est\'a adaptado a
$\left\{\mathcal{F}_{t},t\geq0\right\}$; la colecci\'on
$\left\{P_{x},x\in \mathbb{X}\right\}$ son medidas de probabilidad
en $\left(\Omega,\mathcal{F}\right)$ tales que para todo $x\in
\mathbb{X}$
\[P_{x}\left\{X\left(0\right)=x\right\}=1,\] y
\[E_{x}\left\{f\left(X\circ\theta_{t}\right)|\mathcal{F}_{t}\right\}=E_{X}\left(\tau\right)f\left(X\right),\]
en $\left\{\tau<\infty\right\}$, $P_{x}$-c.s., con $\theta_{t}$
definido como el operador shift.


Donde $\tau$ es un $\mathcal{F}_{t}$-tiempo de paro
\[\left(X\circ\theta_{\tau}\right)\left(w\right)=\left\{X\left(\tau\left(w\right)+t,w\right),t\geq0\right\},\]
y $f$ es una funci\'on de valores reales acotada y medible, ver \cite{Dai, KaspiMandelbaum}.\\

Sea $P^{t}\left(x,D\right)$, $D\in\mathcal{B}_{\mathbb{X}}$,
$t\geq0$ la probabilidad de transici\'on de $X$ queda definida
como:
\[P^{t}\left(x,D\right)=P_{x}\left(X\left(t\right)\in
D\right).\]


\begin{Def}
Una medida no cero $\pi$ en
$\left(\mathbb{X},\mathcal{B}_{\mathbb{X}}\right)$ es invariante
para $X$ si $\pi$ es $\sigma$-finita y
\[\pi\left(D\right)=\int_{\mathbb{X}}P^{t}\left(x,D\right)\pi\left(dx\right),\]
para todo $D\in \mathcal{B}_{\mathbb{X}}$, con $t\geq0$.
\end{Def}

\begin{Def}
El proceso de Markov $X$ es llamado Harris Recurrente si existe
una medida de probabilidad $\nu$ en
$\left(\mathbb{X},\mathcal{B}_{\mathbb{X}}\right)$, tal que si
$\nu\left(D\right)>0$ y $D\in\mathcal{B}_{\mathbb{X}}$
\[P_{x}\left\{\tau_{D}<\infty\right\}\equiv1,\] cuando
$\tau_{D}=inf\left\{t\geq0:X_{t}\in D\right\}$.
\end{Def}

\begin{Note}
\begin{itemize}
\item[i)] Si $X$ es Harris recurrente, entonces existe una \'unica
medida invariante $\pi$ (Getoor \cite{Getoor}).

\item[ii)] Si la medida invariante es finita, entonces puede
normalizarse a una medida de probabilidad, en este caso al proceso
$X$ se le llama Harris recurrente positivo.


\item[iii)] Cuando $X$ es Harris recurrente positivo se dice que
la disciplina de servicio es estable. En este caso $\pi$ denota la
distribuci\'on estacionaria y hacemos
\[P_{\pi}\left(\cdot\right)=\int_{\mathbf{X}}P_{x}\left(\cdot\right)\pi\left(dx\right),\]
y se utiliza $E_{\pi}$ para denotar el operador esperanza
correspondiente, ver \cite{DaiSean}.
\end{itemize}
\end{Note}

\begin{Def}\label{Def.Cto.Peq.}
Un conjunto $D\in\mathcal{B_{\mathbb{X}}}$ es llamado peque\~no si
existe un $t>0$, una medida de probabilidad $\nu$ en
$\mathcal{B_{\mathbb{X}}}$, y un $\delta>0$ tal que
\[P^{t}\left(x,A\right)\geq\delta\nu\left(A\right),\] para $x\in
D,A\in\mathcal{B_{\mathbb{X}}}$.
\end{Def}

La siguiente serie de resultados vienen enunciados y demostrados
en Dai \cite{Dai}:
\begin{Lema}[Lema 3.1, Dai \cite{Dai}]
Sea $B$ conjunto peque\~no cerrado, supongamos que
$P_{x}\left(\tau_{B}<\infty\right)\equiv1$ y que para alg\'un
$\delta>0$ se cumple que
\begin{equation}\label{Eq.3.1}
\sup\esp_{x}\left[\tau_{B}\left(\delta\right)\right]<\infty,
\end{equation}
donde
$\tau_{B}\left(\delta\right)=inf\left\{t\geq\delta:X\left(t\right)\in
B\right\}$. Entonces, $X$ es un proceso Harris recurrente
positivo.
\end{Lema}

\begin{Lema}[Lema 3.1, Dai \cite{Dai}]\label{Lema.3.}
Bajo el supuesto (A3), el conjunto
$B=\left\{x\in\mathbb{X}:|x|\leq k\right\}$ es un conjunto
peque\~no cerrado para cualquier $k>0$.
\end{Lema}

\begin{Teo}[Teorema 3.1, Dai \cite{Dai}]\label{Tma.3.1}
Si existe un $\delta>0$ tal que
\begin{equation}
lim_{|x|\rightarrow\infty}\frac{1}{|x|}\esp|X^{x}\left(|x|\delta\right)|=0,
\end{equation}
donde $X^{x}$ se utiliza para denotar que el proceso $X$ comienza
a partir de $x$, entonces la ecuaci\'on (\ref{Eq.3.1}) se cumple
para $B=\left\{x\in\mathbb{X}:|x|\leq k\right\}$ con alg\'un
$k>0$. En particular, $X$ es Harris recurrente positivo.
\end{Teo}

Entonces, tenemos que el proceso $X$ es un proceso de Markov que
cumple con los supuestos $A1)$-$A3)$, lo que falta de hacer es
construir el Modelo de Flujo bas\'andonos en lo hasta ahora
presentado.
%_______________________________________________________________________
\subsection{Modelo de Flujo}
%_______________________________________________________________________

Dada una condici\'on inicial $x\in\mathbb{X}$, sea

\begin{itemize}
\item $Q_{k}^{x}\left(t\right)$ la longitud de la cola al tiempo
$t$,

\item $T_{m,k}^{x}\left(t\right)$ el tiempo acumulado, al tiempo
$t$, que tarda el servidor $m$ en atender a los usuarios de la
cola $k$.

\item $T_{m,k}^{x,0}\left(t\right)$ el tiempo acumulado, al tiempo
$t$, que tarda el servidor $m$ en trasladarse a otra cola a partir de la $k$-\'esima.\\
\end{itemize}

Sup\'ongase que la funci\'on
$\left(\overline{Q}\left(\cdot\right),\overline{T}_{m}
\left(\cdot\right),\overline{T}_{m}^{0} \left(\cdot\right)\right)$
para $m=1,2,\ldots,M$ es un punto l\'imite de
\begin{equation}\label{Eq.Punto.Limite}
\left(\frac{1}{|x|}Q^{x}\left(|x|t\right),\frac{1}{|x|}T_{m}^{x}\left(|x|t\right),\frac{1}{|x|}T_{m}^{x,0}\left(|x|t\right)\right)
\end{equation}
para $m=1,2,\ldots,M$, cuando $x\rightarrow\infty$, ver
\cite{Down}. Entonces
$\left(\overline{Q}\left(t\right),\overline{T}_{m}
\left(t\right),\overline{T}_{m}^{0} \left(t\right)\right)$ es un
flujo l\'imite del sistema. Al conjunto de todos las posibles
flujos l\'imite se le llama {\emph{Modelo de Flujo}} y se le
denotar\'a por $\mathcal{Q}$, ver \cite{Down, Dai, DaiSean}.\\

El modelo de flujo satisface el siguiente conjunto de ecuaciones:

\begin{equation}\label{Eq.MF.1}
\overline{Q}_{k}\left(t\right)=\overline{Q}_{k}\left(0\right)+\lambda_{k}t-\sum_{m=1}^{M}\mu_{k}\overline{T}_{m,k}\left(t\right),\\
\end{equation}
para $k=1,2,\ldots,K$.\\
\begin{equation}\label{Eq.MF.2}
\overline{Q}_{k}\left(t\right)\geq0\textrm{ para
}k=1,2,\ldots,K.\\
\end{equation}

\begin{equation}\label{Eq.MF.3}
\overline{T}_{m,k}\left(0\right)=0,\textrm{ y }\overline{T}_{m,k}\left(\cdot\right)\textrm{ es no decreciente},\\
\end{equation}
para $k=1,2,\ldots,K$ y $m=1,2,\ldots,M$.\\
\begin{equation}\label{Eq.MF.4}
\sum_{k=1}^{K}\overline{T}_{m,k}^{0}\left(t\right)+\overline{T}_{m,k}\left(t\right)=t\textrm{
para }m=1,2,\ldots,M.\\
\end{equation}


\begin{Def}[Definici\'on 4.1, Dai \cite{Dai}]\label{Def.Modelo.Flujo}
Sea una disciplina de servicio espec\'ifica. Cualquier l\'imite
$\left(\overline{Q}\left(\cdot\right),\overline{T}\left(\cdot\right),\overline{T}^{0}\left(\cdot\right)\right)$
en (\ref{Eq.Punto.Limite}) es un {\em flujo l\'imite} de la
disciplina. Cualquier soluci\'on (\ref{Eq.MF.1})-(\ref{Eq.MF.4})
es llamado flujo soluci\'on de la disciplina.
\end{Def}

\begin{Def}
Se dice que el modelo de flujo l\'imite, modelo de flujo, de la
disciplina de la cola es estable si existe una constante
$\delta>0$ que depende de $\mu,\lambda$ y $P$ solamente, tal que
cualquier flujo l\'imite con
$|\overline{Q}\left(0\right)|+|\overline{U}|+|\overline{V}|=1$, se
tiene que $\overline{Q}\left(\cdot+\delta\right)\equiv0$.
\end{Def}

Si se hace $|x|\rightarrow\infty$ sin restringir ninguna de las
componentes, tambi\'en se obtienen un modelo de flujo, pero en
este caso el residual de los procesos de arribo y servicio
introducen un retraso:
\begin{Teo}[Teorema 4.2, Dai \cite{Dai}]\label{Tma.4.2.Dai}
Sea una disciplina fija para la cola, suponga que se cumplen las
condiciones (A1)-(A3). Si el modelo de flujo l\'imite de la
disciplina de la cola es estable, entonces la cadena de Markov $X$
que describe la din\'amica de la red bajo la disciplina es Harris
recurrente positiva.
\end{Teo}

Ahora se procede a escalar el espacio y el tiempo para reducir la
aparente fluctuaci\'on del modelo. Consid\'erese el proceso
\begin{equation}\label{Eq.3.7}
\overline{Q}^{x}\left(t\right)=\frac{1}{|x|}Q^{x}\left(|x|t\right).
\end{equation}
A este proceso se le conoce como el flujo escalado, y cualquier
l\'imite $\overline{Q}^{x}\left(t\right)$ es llamado flujo
l\'imite del proceso de longitud de la cola. Haciendo
$|q|\rightarrow\infty$ mientras se mantiene el resto de las
componentes fijas, cualquier punto l\'imite del proceso de
longitud de la cola normalizado $\overline{Q}^{x}$ es soluci\'on
del siguiente modelo de flujo.


\begin{Def}[Definici\'on 3.3, Dai y Meyn \cite{DaiSean}]
El modelo de flujo es estable si existe un tiempo fijo $t_{0}$ tal
que $\overline{Q}\left(t\right)=0$, con $t\geq t_{0}$, para
cualquier $\overline{Q}\left(\cdot\right)\in\mathcal{Q}$ que
cumple con $|\overline{Q}\left(0\right)|=1$.
\end{Def}

\begin{Lemma}[Lema 3.1, Dai y Meyn \cite{DaiSean}]
Si el modelo de flujo definido por (\ref{Eq.MF.1})-(\ref{Eq.MF.4})
es estable, entonces el modelo de flujo retrasado es tambi\'en
estable, es decir, existe $t_{0}>0$ tal que
$\overline{Q}\left(t\right)=0$ para cualquier $t\geq t_{0}$, para
cualquier soluci\'on del modelo de flujo retrasado cuya
condici\'on inicial $\overline{x}$ satisface que
$|\overline{x}|=|\overline{Q}\left(0\right)|+|\overline{A}\left(0\right)|+|\overline{B}\left(0\right)|\leq1$.
\end{Lemma}


Ahora ya estamos en condiciones de enunciar los resultados principales:


\begin{Teo}[Teorema 2.1, Down \cite{Down}]\label{Tma2.1.Down}
Suponga que el modelo de flujo es estable, y que se cumplen los supuestos (A1) y (A2), entonces
\begin{itemize}
\item[i)] Para alguna constante $\kappa_{p}$, y para cada
condici\'on inicial $x\in X$
\begin{equation}\label{Estability.Eq1}
\limsup_{t\rightarrow\infty}\frac{1}{t}\int_{0}^{t}\esp_{x}\left[|Q\left(s\right)|^{p}\right]ds\leq\kappa_{p},
\end{equation}
donde $p$ es el entero dado en (A2).
\end{itemize}
Si adem\'as se cumple la condici\'on (A3), entonces para cada
condici\'on inicial:
\begin{itemize}
\item[ii)] Los momentos transitorios convergen a su estado
estacionario:
 \begin{equation}\label{Estability.Eq2}
lim_{t\rightarrow\infty}\esp_{x}\left[Q_{k}\left(t\right)^{r}\right]=\esp_{\pi}\left[Q_{k}\left(0\right)^{r}\right]\leq\kappa_{r},
\end{equation}
para $r=1,2,\ldots,p$ y $k=1,2,\ldots,K$. Donde $\pi$ es la
probabilidad invariante para $X$.

\item[iii)]  El primer momento converge con raz\'on $t^{p-1}$:
\begin{equation}\label{Estability.Eq3}
lim_{t\rightarrow\infty}t^{p-1}|\esp_{x}\left[Q_{k}\left(t\right)\right]-\esp_{\pi}\left[Q_{k}\left(0\right)\right]|=0.
\end{equation}

\item[iv)] La {\em Ley Fuerte de los grandes n\'umeros} se cumple:
\begin{equation}\label{Estability.Eq4}
lim_{t\rightarrow\infty}\frac{1}{t}\int_{0}^{t}Q_{k}^{r}\left(s\right)ds=\esp_{\pi}\left[Q_{k}\left(0\right)^{r}\right],\textrm{
}\prob_{x}\textrm{-c.s.}
\end{equation}
para $r=1,2,\ldots,p$ y $k=1,2,\ldots,K$.
\end{itemize}
\end{Teo}

La contribuci\'on de Down a la teor\'ia de los {\emph {sistemas de
visitas c\'iclicas}}, es la relaci\'on que hay entre la
estabilidad del sistema con el comportamiento de las medidas de
desempe\~no, es decir, la condici\'on suficiente para poder
garantizar la convergencia del proceso de la longitud de la cola
as\'i como de por los menos los dos primeros momentos adem\'as de
una versi\'on de la Ley Fuerte de los Grandes N\'umeros para los
sistemas de visitas.


\begin{Teo}[Teorema 2.3, Down \cite{Down}]\label{Tma2.3.Down}
Considere el siguiente valor:
\begin{equation}\label{Eq.Rho.1serv}
\rho=\sum_{k=1}^{K}\rho_{k}+max_{1\leq j\leq K}\left(\frac{\lambda_{j}}{\sum_{s=1}^{S}p_{js}\overline{N}_{s}}\right)\delta^{*}
\end{equation}
\begin{itemize}
\item[i)] Si $\rho<1$ entonces la red es estable, es decir, se
cumple el Teorema \ref{Tma2.1.Down}.

\item[ii)] Si $\rho>1$ entonces la red es inestable, es decir, se
cumple el Teorema \ref{Tma2.2.Down}
\end{itemize}
\end{Teo}




Dado el proceso $X=\left\{X\left(t\right),t\geq0\right\}$ definido
en (\ref{Esp.Edos.Down}) que describe la din\'amica del sistema de
visitas c\'iclicas, si $U\left(t\right)$ es el residual de los
tiempos de llegada al tiempo $t$ entre dos usuarios consecutivos y
$V\left(t\right)$ es el residual de los tiempos de servicio al
tiempo $t$ para el usuario que est\'as siendo atendido por el
servidor. Sea $\mathbb{X}$ el espacio de estados que puede tomar
el proceso $X$.


\begin{Lema}[Lema 4.3, Dai\cite{Dai}]\label{Lema.4.3}
Sea $\left\{x_{n}\right\}\subset \mathbf{X}$ con
$|x_{n}|\rightarrow\infty$, conforme $n\rightarrow\infty$. Suponga
que
\[lim_{n\rightarrow\infty}\frac{1}{|x_{n}|}U\left(0\right)=\overline{U}_{k},\]
y
\[lim_{n\rightarrow\infty}\frac{1}{|x_{n}|}V\left(0\right)=\overline{V}_{k}.\]
\begin{itemize}
\item[a)] Conforme $n\rightarrow\infty$ casi seguramente,
\[lim_{n\rightarrow\infty}\frac{1}{|x_{n}|}U^{x_{n}}_{k}\left(|x_{n}|t\right)=\left(\overline{U}_{k}-t\right)^{+}\textrm{, u.o.c.}\]
y
\[lim_{n\rightarrow\infty}\frac{1}{|x_{n}|}V^{x_{n}}_{k}\left(|x_{n}|t\right)=\left(\overline{V}_{k}-t\right)^{+}.\]

\item[b)] Para cada $t\geq0$ fijo,
\[\left\{\frac{1}{|x_{n}|}U^{x_{n}}_{k}\left(|x_{n}|t\right),|x_{n}|\geq1\right\}\]
y
\[\left\{\frac{1}{|x_{n}|}V^{x_{n}}_{k}\left(|x_{n}|t\right),|x_{n}|\geq1\right\}\]
\end{itemize}
son uniformemente convergentes.
\end{Lema}

Sea $e$ es un vector de unos, $C$ es la matriz definida por
\[C_{ik}=\left\{\begin{array}{cc}
1,& S\left(k\right)=i,\\
0,& \textrm{ en otro caso}.\\
\end{array}\right.
\]
Es necesario enunciar el siguiente Teorema que se utilizar\'a para
el Teorema (\ref{Tma.4.2.Dai}):
\begin{Teo}[Teorema 4.1, Dai \cite{Dai}]
Considere una disciplina que cumpla la ley de conservaci\'on, para
casi todas las trayectorias muestrales $\omega$ y cualquier
sucesi\'on de estados iniciales $\left\{x_{n}\right\}\subset
\mathbf{X}$, con $|x_{n}|\rightarrow\infty$, existe una
subsucesi\'on $\left\{x_{n_{j}}\right\}$ con
$|x_{n_{j}}|\rightarrow\infty$ tal que
\begin{equation}\label{Eq.4.15}
\frac{1}{|x_{n_{j}}|}\left(Q^{x_{n_{j}}}\left(0\right),U^{x_{n_{j}}}\left(0\right),V^{x_{n_{j}}}\left(0\right)\right)\rightarrow\left(\overline{Q}\left(0\right),\overline{U},\overline{V}\right),
\end{equation}

\begin{equation}\label{Eq.4.16}
\frac{1}{|x_{n_{j}}|}\left(Q^{x_{n_{j}}}\left(|x_{n_{j}}|t\right),T^{x_{n_{j}}}\left(|x_{n_{j}}|t\right)\right)\rightarrow\left(\overline{Q}\left(t\right),\overline{T}\left(t\right)\right)\textrm{
u.o.c.}
\end{equation}

Adem\'as,
$\left(\overline{Q}\left(t\right),\overline{T}\left(t\right)\right)$
satisface las siguientes ecuaciones:
\begin{equation}\label{Eq.MF.1.3a}
\overline{Q}\left(t\right)=Q\left(0\right)+\left(\alpha
t-\overline{U}\right)^{+}-\left(I-P\right)^{'}M^{-1}\left(\overline{T}\left(t\right)-\overline{V}\right)^{+},
\end{equation}

\begin{equation}\label{Eq.MF.2.3a}
\overline{Q}\left(t\right)\geq0,\\
\end{equation}

\begin{equation}\label{Eq.MF.3.3a}
\overline{T}\left(t\right)\textrm{ es no decreciente y comienza en cero},\\
\end{equation}

\begin{equation}\label{Eq.MF.4.3a}
\overline{I}\left(t\right)=et-C\overline{T}\left(t\right)\textrm{
es no decreciente,}\\
\end{equation}

\begin{equation}\label{Eq.MF.5.3a}
\int_{0}^{\infty}\left(C\overline{Q}\left(t\right)\right)d\overline{I}\left(t\right)=0,\\
\end{equation}

\begin{equation}\label{Eq.MF.6.3a}
\textrm{Condiciones en
}\left(\overline{Q}\left(\cdot\right),\overline{T}\left(\cdot\right)\right)\textrm{
espec\'ificas de la disciplina de la cola,}
\end{equation}
\end{Teo}


Propiedades importantes para el modelo de flujo retrasado:

\begin{Prop}[Proposici\'on 4.2, Dai \cite{Dai}]
 Sea $\left(\overline{Q},\overline{T},\overline{T}^{0}\right)$ un flujo l\'imite de \ref{Eq.Punto.Limite}
 y suponga que cuando $x\rightarrow\infty$ a lo largo de una subsucesi\'on
\[\left(\frac{1}{|x|}Q_{k}^{x}\left(0\right),\frac{1}{|x|}A_{k}^{x}\left(0\right),\frac{1}{|x|}B_{k}^{x}\left(0\right),\frac{1}{|x|}B_{k}^{x,0}\left(0\right)\right)\rightarrow\left(\overline{Q}_{k}\left(0\right),0,0,0\right)\]
para $k=1,\ldots,K$. El flujo l\'imite tiene las siguientes
propiedades, donde las propiedades de la derivada se cumplen donde
la derivada exista:
\begin{itemize}
 \item[i)] Los vectores de tiempo ocupado $\overline{T}\left(t\right)$ y $\overline{T}^{0}\left(t\right)$ son crecientes y continuas con
$\overline{T}\left(0\right)=\overline{T}^{0}\left(0\right)=0$.
\item[ii)] Para todo $t\geq0$
\[\sum_{k=1}^{K}\left[\overline{T}_{k}\left(t\right)+\overline{T}_{k}^{0}\left(t\right)\right]=t.\]
\item[iii)] Para todo $1\leq k\leq K$
\[\overline{Q}_{k}\left(t\right)=\overline{Q}_{k}\left(0\right)+\alpha_{k}t-\mu_{k}\overline{T}_{k}\left(t\right).\]
\item[iv)]  Para todo $1\leq k\leq K$
\[\dot{{\overline{T}}}_{k}\left(t\right)=\rho_{k}\] para $\overline{Q}_{k}\left(t\right)=0$.
\item[v)] Para todo $k,j$
\[\mu_{k}^{0}\overline{T}_{k}^{0}\left(t\right)=\mu_{j}^{0}\overline{T}_{j}^{0}\left(t\right).\]
\item[vi)]  Para todo $1\leq k\leq K$
\[\mu_{k}\dot{{\overline{T}}}_{k}\left(t\right)=l_{k}\mu_{k}^{0}\dot{{\overline{T}}}_{k}^{0}\left(t\right),\] para $\overline{Q}_{k}\left(t\right)>0$.
\end{itemize}
\end{Prop}

\begin{Lema}[Lema 3.1, Chen \cite{Chen}]\label{Lema3.1}
Si el modelo de flujo es estable, definido por las ecuaciones
(3.8)-(3.13), entonces el modelo de flujo retrasado tambi\'en es
estable.
\end{Lema}

\begin{Lema}[Lema 5.2, Gut \cite{Gut}]\label{Lema.5.2.Gut}
Sea $\left\{\xi\left(k\right):k\in\ent\right\}$ sucesi\'on de
variables aleatorias i.i.d. con valores en
$\left(0,\infty\right)$, y sea $E\left(t\right)$ el proceso de
conteo
\[E\left(t\right)=max\left\{n\geq1:\xi\left(1\right)+\cdots+\xi\left(n-1\right)\leq t\right\}.\]
Si $E\left[\xi\left(1\right)\right]<\infty$, entonces para
cualquier entero $r\geq1$
\begin{equation}
lim_{t\rightarrow\infty}\esp\left[\left(\frac{E\left(t\right)}{t}\right)^{r}\right]=\left(\frac{1}{E\left[\xi_{1}\right]}\right)^{r},
\end{equation}
de aqu\'i, bajo estas condiciones
\begin{itemize}
\item[a)] Para cualquier $t>0$,
$sup_{t\geq\delta}\esp\left[\left(\frac{E\left(t\right)}{t}\right)^{r}\right]<\infty$.

\item[b)] Las variables aleatorias
$\left\{\left(\frac{E\left(t\right)}{t}\right)^{r}:t\geq1\right\}$
son uniformemente integrables.
\end{itemize}
\end{Lema}

\begin{Teo}[Teorema 5.1: Ley Fuerte para Procesos de Conteo, Gut
\cite{Gut}]\label{Tma.5.1.Gut} Sea
$0<\mu<\esp\left(X_{1}\right]\leq\infty$. entonces

\begin{itemize}
\item[a)] $\frac{N\left(t\right)}{t}\rightarrow\frac{1}{\mu}$
a.s., cuando $t\rightarrow\infty$.


\item[b)]$\esp\left[\frac{N\left(t\right)}{t}\right]^{r}\rightarrow\frac{1}{\mu^{r}}$,
cuando $t\rightarrow\infty$ para todo $r>0$.
\end{itemize}
\end{Teo}


\begin{Prop}[Proposici\'on 5.1, Dai y Sean \cite{DaiSean}]\label{Prop.5.1}
Suponga que los supuestos (A1) y (A2) se cumplen, adem\'as suponga
que el modelo de flujo es estable. Entonces existe $t_{0}>0$ tal
que
\begin{equation}\label{Eq.Prop.5.1}
lim_{|x|\rightarrow\infty}\frac{1}{|x|^{p+1}}\esp_{x}\left[|X\left(t_{0}|x|\right)|^{p+1}\right]=0.
\end{equation}

\end{Prop}


\begin{Prop}[Proposici\'on 5.3, Dai y Sean \cite{DaiSean}]\label{Prop.5.3.DaiSean}
Sea $X$ proceso de estados para la red de colas, y suponga que se
cumplen los supuestos (A1) y (A2), entonces para alguna constante
positiva $C_{p+1}<\infty$, $\delta>0$ y un conjunto compacto
$C\subset X$.

\begin{equation}\label{Eq.5.4}
\esp_{x}\left[\int_{0}^{\tau_{C}\left(\delta\right)}\left(1+|X\left(t\right)|^{p}\right)dt\right]\leq
C_{p+1}\left(1+|x|^{p+1}\right).
\end{equation}
\end{Prop}

\begin{Prop}[Proposici\'on 5.4, Dai y Sean \cite{DaiSean}]\label{Prop.5.4.DaiSean}
Sea $X$ un proceso de Markov Borel Derecho en $X$, sea
$f:X\leftarrow\rea_{+}$ y defina para alguna $\delta>0$, y un
conjunto cerrado $C\subset X$
\[V\left(x\right):=\esp_{x}\left[\int_{0}^{\tau_{C}\left(\delta\right)}f\left(X\left(t\right)\right)dt\right],\]
para $x\in X$. Si $V$ es finito en todas partes y uniformemente
acotada en $C$, entonces existe $k<\infty$ tal que
\begin{equation}\label{Eq.5.11}
\frac{1}{t}\esp_{x}\left[V\left(x\right)\right]+\frac{1}{t}\int_{0}^{t}\esp_{x}\left[f\left(X\left(s\right)\right)ds\right]\leq\frac{1}{t}V\left(x\right)+k,
\end{equation}
para $x\in X$ y $t>0$.
\end{Prop}


\begin{Teo}[Teorema 5.5, Dai y Sean  \cite{DaiSean}]
Suponga que se cumplen (A1) y (A2), adem\'as suponga que el modelo
de flujo es estable. Entonces existe una constante $k_{p}<\infty$
tal que
\begin{equation}\label{Eq.5.13}
\frac{1}{t}\int_{0}^{t}\esp_{x}\left[|Q\left(s\right)|^{p}\right]ds\leq
k_{p}\left\{\frac{1}{t}|x|^{p+1}+1\right\},
\end{equation}
para $t\geq0$, $x\in X$. En particular para cada condici\'on
inicial
\begin{equation}\label{Eq.5.14}
\limsup_{t\rightarrow\infty}\frac{1}{t}\int_{0}^{t}\esp_{x}\left[|Q\left(s\right)|^{p}\right]ds\leq
k_{p}.
\end{equation}
\end{Teo}

\begin{Teo}[Teorema 6.2 Dai y Sean \cite{DaiSean}]\label{Tma.6.2}
Suponga que se cumplen los supuestos (A1)-(A3) y que el modelo de
flujo es estable, entonces se tiene que
\[\parallel P^{t}\left(x,\cdot\right)-\pi\left(\cdot\right)\parallel_{f_{p}}\rightarrow0,\]
para $t\rightarrow\infty$ y $x\in X$. En particular para cada
condici\'on inicial
\[lim_{t\rightarrow\infty}\esp_{x}\left[\left|Q_{t}\right|^{p}\right]=\esp_{\pi}\left[\left|Q_{0}\right|^{p}\right]<\infty,\]
\end{Teo}

donde

\begin{eqnarray*}
\parallel
P^{t}\left(c,\cdot\right)-\pi\left(\cdot\right)\parallel_{f}=sup_{|g\leq
f|}|\int\pi\left(dy\right)g\left(y\right)-\int
P^{t}\left(x,dy\right)g\left(y\right)|,
\end{eqnarray*}
para $x\in\mathbb{X}$.

\begin{Teo}[Teorema 6.3, Dai y Sean \cite{DaiSean}]\label{Tma.6.3}
Suponga que se cumplen los supuestos (A1)-(A3) y que el modelo de
flujo es estable, entonces con
$f\left(x\right)=f_{1}\left(x\right)$, se tiene que
\[lim_{t\rightarrow\infty}t^{(p-1)}\left|P^{t}\left(c,\cdot\right)-\pi\left(\cdot\right)\right|_{f}=0,\]
para $x\in X$. En particular, para cada condici\'on inicial
\[lim_{t\rightarrow\infty}t^{(p-1)}\left|\esp_{x}\left[Q_{t}\right]-\esp_{\pi}\left[Q_{0}\right]\right|=0.\]
\end{Teo}



\begin{Prop}[Proposici\'on 5.1, Dai y Meyn \cite{DaiSean}]\label{Prop.5.1.DaiSean}
Suponga que los supuestos A1) y A2) son ciertos y que el modelo de
flujo es estable. Entonces existe $t_{0}>0$ tal que
\begin{equation}
lim_{|x|\rightarrow\infty}\frac{1}{|x|^{p+1}}\esp_{x}\left[|X\left(t_{0}|x|\right)|^{p+1}\right]=0.
\end{equation}
\end{Prop}


\begin{Teo}[Teorema 5.5, Dai y Meyn \cite{DaiSean}]\label{Tma.5.5.DaiSean}
Suponga que los supuestos A1) y A2) se cumplen y que el modelo de
flujo es estable. Entonces existe una constante $\kappa_{p}$ tal
que
\begin{equation}
\frac{1}{t}\int_{0}^{t}\esp_{x}\left[|Q\left(s\right)|^{p}\right]ds\leq\kappa_{p}\left\{\frac{1}{t}|x|^{p+1}+1\right\},
\end{equation}
para $t>0$ y $x\in X$. En particular, para cada condici\'on
inicial
\begin{eqnarray*}
\limsup_{t\rightarrow\infty}\frac{1}{t}\int_{0}^{t}\esp_{x}\left[|Q\left(s\right)|^{p}\right]ds\leq\kappa_{p}.
\end{eqnarray*}
\end{Teo}


\begin{Teo}[Teorema 6.4, Dai y Meyn \cite{DaiSean}]\label{Tma.6.4.DaiSean}
Suponga que se cumplen los supuestos A1), A2) y A3) y que el
modelo de flujo es estable. Sea $\nu$ cualquier distribuci\'on de
probabilidad en
$\left(\mathbb{X},\mathcal{B}_{\mathbb{X}}\right)$, y $\pi$ la
distribuci\'on estacionaria de $X$.
\begin{itemize}
\item[i)] Para cualquier $f:X\leftarrow\rea_{+}$
\begin{equation}
\lim_{t\rightarrow\infty}\frac{1}{t}\int_{o}^{t}f\left(X\left(s\right)\right)ds=\pi\left(f\right):=\int
f\left(x\right)\pi\left(dx\right),
\end{equation}
$\prob$-c.s.

\item[ii)] Para cualquier $f:X\leftarrow\rea_{+}$ con
$\pi\left(|f|\right)<\infty$, la ecuaci\'on anterior se cumple.
\end{itemize}
\end{Teo}

\begin{Teo}[Teorema 2.2, Down \cite{Down}]\label{Tma2.2.Down}
Suponga que el fluido modelo es inestable en el sentido de que
para alguna $\epsilon_{0},c_{0}\geq0$,
\begin{equation}\label{Eq.Inestability}
|Q\left(T\right)|\geq\epsilon_{0}T-c_{0}\textrm{,   }T\geq0,
\end{equation}
para cualquier condici\'on inicial $Q\left(0\right)$, con
$|Q\left(0\right)|=1$. Entonces para cualquier $0<q\leq1$, existe
$B<0$ tal que para cualquier $|x|\geq B$,
\begin{equation}
\prob_{x}\left\{\mathbb{X}\rightarrow\infty\right\}\geq q.
\end{equation}
\end{Teo}

\begin{Dem}[Teorema \ref{Tma2.1.Down}] La demostraci\'on de este
teorema se da a continuaci\'on:\\
\begin{itemize}
\item[i)] Utilizando la proposici\'on \ref{Prop.5.3.DaiSean} se
tiene que la proposici\'on \ref{Prop.5.4.DaiSean} es cierta para
$f\left(x\right)=1+|x|^{p}$.

\item[i)] es consecuencia directa del Teorema \ref{Tma.6.2}.

\item[iii)] ver la demostraci\'on dada en Dai y Sean
\cite{DaiSean} p\'aginas 1901-1902.

\item[iv)] ver Dai y Sean \cite{DaiSean} p\'aginas 1902-1903 \'o
\cite{MeynTweedie2}.
\end{itemize}
\end{Dem}

%_____________________________________________________________________
\subsubsection{Modelo de Flujo y Estabilidad}
%_____________________________________________________________________

Para cada $k$ y cada $n$ se define

\numberwithin{equation}{section}
\begin{equation}
\Phi^{k}\left(n\right):=\sum_{i=1}^{n}\phi^{k}\left(i\right).
\end{equation}

suponiendo que el estado inicial de la red es
$x=\left(q,a,b\right)\in X$, entonces para cada $k$

\begin{eqnarray}
E_{k}^{x}\left(t\right):=\max\left\{n\geq0:A_{k}^{x}\left(0\right)+\psi_{k}\left(1\right)+\cdots+\psi_{k}\left(n-1\right)\leq t\right\}\\
S_{k}^{x}\left(t\right):=\max\left\{n\geq0:B_{k}^{x}\left(0\right)+\eta_{k}\left(1\right)+\cdots+\eta_{k}\left(n-1\right)\leq
t\right\}
\end{eqnarray}

Sea $T_{k}^{x}\left(t\right)$ el tiempo acumulado que el servidor
$s\left(k\right)$ ha utilizado en los usuarios de la clase $k$ en
el intervalo $\left[0,t\right]$. Entonces se tienen las siguientes
ecuaciones:

\begin{equation}
Q_{k}^{x}\left(t\right)=Q_{k}^{x}\left(0\right)+E_{k}^{x}\left(t\right)+\sum_{l=1}^{k}\Phi_{k}^{l}S_{l}^{x}\left(T_{l}^{x}\right)-S_{k}^{x}\left(T_{k}^{x}\right)\\
\end{equation}
\begin{equation}
Q^{x}\left(t\right)=\left(Q^{x}_{1}\left(t\right),\ldots,Q^{x}_{K}\left(t\right)\right)^{'}\geq0,\\
\end{equation}
\begin{equation}
T^{x}\left(t\right)=\left(T^{x}_{1}\left(t\right),\ldots,T^{x}_{K}\left(t\right)\right)^{'}\geq0,\textrm{ es no decreciente}\\
\end{equation}
\begin{equation}
I_{i}^{x}\left(t\right)=t-\sum_{k\in C_{i}}T_{k}^{x}\left(t\right)\textrm{ es no decreciente}\\
\end{equation}
\begin{equation}
\int_{0}^{\infty}\sum_{k\in C_{i}}Q_{k}^{x}\left(t\right)dI_{i}^{x}\left(t\right)=0\\
\end{equation}
\begin{equation}
\textrm{condiciones adicionales sobre
}\left(Q^{x}\left(\cdot\right),T^{x}\left(\cdot\right)\right)\textrm{
referentes a la disciplina de servicio}
\end{equation}

Para reducir la fluctuaci\'on del modelo se escala tanto el
espacio como el tiempo, entonces se tiene el proceso:

\begin{equation}
\overline{Q}^{x}\left(t\right)=\frac{1}{|x|}Q^{x}\left(|x|t\right)
\end{equation}
Cualquier l\'imite $\overline{Q}\left(t\right)$ es llamado un
flujo l\'imite del proceso longitud de la cola. Si se hace
$|q|\rightarrow\infty$ y se mantienen las componentes restantes
fijas, de la condici\'on inicial $x$, cualquier punto l\'imite del
proceso normalizado $\overline{Q}^{x}$ es una soluci\'on del
siguiente modelo de flujo, ver \cite{Dai}.

\begin{Def}
Un flujo l\'imite (retrasado) para una red bajo una disciplina de
servicio espec\'ifica se define como cualquier soluci\'on
 $\left(Q^{x}\left(\cdot\right),T^{x}\left(\cdot\right)\right)$ de las siguientes ecuaciones, donde
$\overline{Q}\left(t\right)=\left(\overline{Q}_{1}\left(t\right),\ldots,\overline{Q}_{K}\left(t\right)\right)^{'}$
y
$\overline{T}\left(t\right)=\left(\overline{T}_{1}\left(t\right),\ldots,\overline{T}_{K}\left(t\right)\right)^{'}$
\begin{equation}\label{Eq.3.8}
\overline{Q}_{k}\left(t\right)=\overline{Q}_{k}\left(0\right)+\alpha_{k}t-\mu_{k}\overline{T}_{k}\left(t\right)+\sum_{l=1}^{k}P_{lk}\mu_{l}\overline{T}_{l}\left(t\right)\\
\end{equation}
\begin{equation}\label{Eq.3.9}
\overline{Q}_{k}\left(t\right)\geq0\textrm{ para }k=1,2,\ldots,K,\\
\end{equation}
\begin{equation}\label{Eq.3.10}
\overline{T}_{k}\left(0\right)=0,\textrm{ y }\overline{T}_{k}\left(\cdot\right)\textrm{ es no decreciente},\\
\end{equation}
\begin{equation}\label{Eq.3.11}
\overline{I}_{i}\left(t\right)=t-\sum_{k\in C_{i}}\overline{T}_{k}\left(t\right)\textrm{ es no decreciente}\\
\end{equation}
\begin{equation}\label{Eq.3.12}
\overline{I}_{i}\left(\cdot\right)\textrm{ se incrementa al tiempo}t\textrm{ cuando }\sum_{k\in C_{i}}Q_{k}^{x}\left(t\right)dI_{i}^{x}\left(t\right)=0\\
\end{equation}
\begin{equation}\label{Eq.3.13}
\textrm{condiciones adicionales sobre
}\left(Q^{x}\left(\cdot\right),T^{x}\left(\cdot\right)\right)\textrm{
referentes a la disciplina de servicio}
\end{equation}
\end{Def}

Al conjunto de ecuaciones dadas en \ref{Eq.3.8}-\ref{Eq.3.13} se
le llama {\em Modelo de flujo} y al conjunto de todas las
soluciones del modelo de flujo
$\left(\overline{Q}\left(\cdot\right),\overline{T}
\left(\cdot\right)\right)$ se le denotar\'a por $\mathcal{Q}$.

Si se hace $|x|\rightarrow\infty$ sin restringir ninguna de las
componentes, tambi\'en se obtienen un modelo de flujo, pero en
este caso el residual de los procesos de arribo y servicio
introducen un retraso:

\begin{Def}
El modelo de flujo retrasado de una disciplina de servicio en una
red con retraso
$\left(\overline{A}\left(0\right),\overline{B}\left(0\right)\right)\in\rea_{+}^{K+|A|}$
se define como el conjunto de ecuaciones dadas en
\ref{Eq.3.8}-\ref{Eq.3.13}, junto con la condici\'on:
\begin{equation}\label{CondAd.FluidModel}
\overline{Q}\left(t\right)=\overline{Q}\left(0\right)+\left(\alpha
t-\overline{A}\left(0\right)\right)^{+}-\left(I-P^{'}\right)M\left(\overline{T}\left(t\right)-\overline{B}\left(0\right)\right)^{+}
\end{equation}
\end{Def}

\begin{Def}
El modelo de flujo es estable si existe un tiempo fijo $t_{0}$ tal
que $\overline{Q}\left(t\right)=0$, con $t\geq t_{0}$, para
cualquier $\overline{Q}\left(\cdot\right)\in\mathcal{Q}$ que
cumple con $|\overline{Q}\left(0\right)|=1$.
\end{Def}

El siguiente resultado se encuentra en \cite{Chen}.
\begin{Lemma}
Si el modelo de flujo definido por \ref{Eq.3.8}-\ref{Eq.3.13} es
estable, entonces el modelo de flujo retrasado es tambi\'en
estable, es decir, existe $t_{0}>0$ tal que
$\overline{Q}\left(t\right)=0$ para cualquier $t\geq t_{0}$, para
cualquier soluci\'on del modelo de flujo retrasado cuya
condici\'on inicial $\overline{x}$ satisface que
$|overline{x}|=|\overline{Q}\left(0\right)|+|\overline{A}\left(0\right)|+|\overline{B}\left(0\right)|\leq1$.
\end{Lemma}

%_____________________________________________________________________
\subsubsection{Resultados principales}
%_____________________________________________________________________
Supuestos necesarios sobre la red

\begin{Sup}
\begin{itemize}
\item[A1)] $\psi_{1},\ldots,\psi_{K},\eta_{1},\ldots,\eta_{K}$ son
mutuamente independientes y son sucesiones independientes e
id\'enticamente distribuidas.

\item[A2)] Para alg\'un entero $p\geq1$
\begin{eqnarray*}
\esp\left[\psi_{l}\left(1\right)^{p+1}\right]<\infty\textrm{ para }l\in\mathcal{A}\textrm{ y }\\
\esp\left[\eta_{k}\left(1\right)^{p+1}\right]<\infty\textrm{ para
}k=1,\ldots,K.
\end{eqnarray*}
\item[A3)] El conjunto $\left\{x\in X:|x|=0\right\}$ es un
singleton, y para cada $k\in\mathcal{A}$, existe una funci\'on
positiva $q_{k}\left(x\right)$ definida en $\rea_{+}$, y un entero
$j_{k}$, tal que
\begin{eqnarray}
P\left(\psi_{k}\left(1\right)\geq x\right)>0\textrm{, para todo }x>0\\
P\left(\psi_{k}\left(1\right)+\ldots\psi_{k}\left(j_{k}\right)\in dx\right)\geq q_{k}\left(x\right)dx0\textrm{ y }\\
\int_{0}^{\infty}q_{k}\left(x\right)dx>0
\end{eqnarray}
\end{itemize}
\end{Sup}

El argumento dado en \cite{MaynDown} en el lema
\ref{Lema.34.MeynDown} se puede aplicar para deducir que todos los
subconjuntos compactos de $X$ son peque\~nos.Entonces la
condici\'on $A3)$ se puede generalizar a
\begin{itemize}
\item[A3')] Para el proceso de Markov $X$, cada subconjunto
compacto de $X$ es peque\~no.
\end{itemize}

\begin{Teo}\label{Tma.4.1}
Suponga que el modelo de flujo para una disciplina de servicio es
estable, y suponga adem\'as que las condiciones A1) y A2) se
satisfacen. Entonces:
\begin{itemize}
\item[i)] Para alguna constante $\kappa_{p}$, y para cada
condici\'on inicial $x\in X$
\begin{equation}
\limsup_{t\rightarrow\infty}\frac{1}{t}\int_{0}^{t}\esp_{x}\left[|Q\left(t\right)|^{p}\right]ds\leq\kappa_{p}
\end{equation}
donde $p$ es el entero dado por A2). Suponga adem\'as que A3) o A3')
se cumple, entonces la disciplina de servicio es estable y adem\'as
para cada condici\'on inicial se tiene lo siguiente: \item[ii)] Los
momentos transitorios convergen a sus valores en estado
estacionario:
\begin{equation}
\lim_{t\rightarrow\infty}\esp_{x}\left[Q_{k}\left(t\right)^{r}\right]=\esp_{\pi}\left[Q_{k}\left(0\right)^{r}\right]\leq\kappa_{r}
\end{equation}
para $r=1,\ldots,p$ y $k=1,\ldots,K$. \item[iii)] EL primer
momento converge con raz\'on $t^{p-1}$:
\begin{equation}
\lim_{t\rightarrow\infty}t^{p-1}|\esp_{x}\left[Q\left(t\right)\right]-\esp_{\pi}\left[Q\left(0\right)\right]|=0.
\end{equation}
\item[iv)] Se cumple la Ley Fuerte de los Grandes N\'umeros:
\begin{equation}
\lim_{t\rightarrow\infty}\frac{1}{t}\int_{0}^{t}Q_{k}^{r}\left(s\right)ds=\esp_{\pi}\left[Q_{k}\left(0\right)^{r}\right]
\end{equation}
$\prob$-c.s., para $r=1,\ldots,p$ y $k=1,\ldots,K$.
\end{itemize}
\end{Teo}
\begin{Dem}
La demostraci\'on de este resultado se da aplicando los teoremas
\ref{Tma.5.5}, \ref{Tma.6.2}, \ref{Tma.6.3} y \ref{Tma.6.4}
\end{Dem}

%_____________________________________________________________________
\subsubsection{Definiciones Generales}
%_____________________________________________________________________
Definimos un proceso de estados para la red que depende de la
pol\'itica de servicio utilizada. Bajo cualquier {\em preemptive
buffer priority} disciplina de servicio, el estado
$\mathbb{X}\left(t\right)$ a cualquier tiempo $t$ puede definirse
como
\begin{equation}\label{Eq.Esp.Estados}
\mathbb{X}\left(t\right)=\left(Q_{k}\left(t\right),A_{l}\left(t\right),B_{k}\left(t\right):k=1,2,\ldots,K,l\in\mathcal{A}\right)
\end{equation}
donde $Q_{k}\left(t\right)$ es la longitud de la cola para los
usuarios de la clase $k$, incluyendo aquellos que est\'an siendo
atendidos, $B_{k}\left(t\right)$ son los tiempos de servicio
residuales para los usuarios de la clase $k$ que est\'an en
servicio. Los tiempos de arribo residuales, que son iguales al
tiempo que queda hasta que el pr\'oximo usuario de la clase $k$
llega, se denotan por $A_{k}\left(t\right)$. Tanto
$B_{k}\left(t\right)$ como $A_{k}\left(t\right)$ se suponen
continuos por la derecha.

Sea $\mathbb{X}$ el espacio de estados para el proceso de estados
que por definici\'on es igual  al conjunto de posibles valores
para el estado $\mathbb{X}\left(t\right)$, y sea
$x=\left(q,a,b\right)$ un estado gen\'erico en $\mathbb{X}$, la
componente $q$ determina la posici\'on del usuario en la red,
$|q|$ denota la longitud total de la cola en la red.

Para un estado $x=\left(q,a,b\right)\in\mathbb{X}$ definimos la
{\em norma} de $x$ como $\left\|x\right|=|q|+|a|+|b|$. En
\cite{Dai} se muestra que para una amplia serie de disciplinas de
servicio el proceso $\mathbb{X}$ es un Proceso Fuerte de Markov, y
por tanto se puede asumir que
\[\left(\left(\Omega,\mathcal{F}\right),\mathcal{F}_{t},\mathbb{X}\left(t\right),\theta_{t},P_{x}\right)\]
es un proceso de Borel Derecho en el espacio de estadio medible
$\left(\mathbb{X},\mathcal{B}_{\mathbb{X}}\right)$. El Proceso
$X=\left\{\mathbb{X}\left(t\right),t\geq0\right\}$ tiene
trayectorias continuas por la derecha, est definida en
$\left(\Omega,\mathcal{F}\right)$ y est adaptado a
$\left\{\mathcal{F}_{t},t\geq0\right\}$; $\left\{P_{x},x\in
X\right\}$ son medidas de probabilidad en
$\left(\Omega,\mathcal{F}\right)$ tales que para todo $x\in X$
\[P_{x}\left\{\mathbb{X}\left(0\right)=x\right\}=1\] y
\[E_{x}\left\{f\left(X\circ\theta_{t}\right)|\mathcal{F}_{t}\right\}=E_{X}\left(\tau\right)f\left(X\right)\]
en $\left\{\tau<\infty\right\}$, $P_{x}$-c.s. Donde $\tau$ es un
$\mathcal{F}_{t}$-tiempo de paro
\[\left(X\circ\theta_{\tau}\right)\left(w\right)=\left\{\mathbb{X}\left(\tau\left(w\right)+t,w\right),t\geq0\right\}\]
y $f$ es una funci\'on de valores reales acotada y medible con la
sigma algebra de Kolmogorov generada por los cilindros.

Sea $P^{t}\left(x,D\right)$, $D\in\mathcal{B}_{\mathbb{X}}$,
$t\geq0$ probabilidad de transici\'on de $X$ definida como
\[P^{t}\left(x,D\right)=P_{x}\left(\mathbb{X}\left(t\right)\in
D\right)\]

\begin{Def}
Una medida no cero $\pi$ en
$\left(\mathbb{X},\mathcal{B}_{\mathbb{X}}\right)$ es {\em
invariante} para $X$ si $\pi$ es $\sigma$-finita y
\[\pi\left(D\right)=\int_{X}P^{t}\left(x,D\right)\pi\left(dx\right)\]
para todo $D\in \mathcal{B}_{\mathbb{X}}$, con $t\geq0$.
\end{Def}

\begin{Def}
El proceso de Markov $X$ es llamado {\em Harris recurrente} si
existe una medida de probabilidad $\nu$ en
$\left(\mathbb{X},\mathcal{B}_{\mathbb{X}}\right)$, tal que si
$\nu\left(D\right)>0$ y $D\in\mathcal{B}_{\mathbb{X}}$
\[P_{x}\left\{\tau_{D}<\infty\right\}\equiv1\] cuando
$\tau_{D}=\inf\left\{t\geq0:\mathbb{X}_{t}\in D\right\}$.
\end{Def}

\begin{itemize}
\item Si $X$ es Harris recurrente, entonces una \'unica medida
invariante $\pi$ existe (\cite{Getoor}). \item Si la medida
invariante es finita, entonces puede normalizarse a una medida de
probabilidad, en este caso se le llama {\em Harris recurrente
positiva}. \item Cuando $X$ es Harris recurrente positivo se dice
que la disciplina de servicio es estable. En este caso $\pi$
denota la ditribuci\'on estacionaria y hacemos
\[P_{\pi}\left(\cdot\right)[=\int_{X}P_{x}\left(\cdot\right)\pi\left(dx\right)\]
y se utiliza $E_{\pi}$ para denotar el operador esperanza
correspondiente, as, el proceso
$X=\left\{\mathbb{X}\left(t\right),t\geq0\right\}$ es un proceso
estrictamente estacionario bajo $P_{\pi}$
\end{itemize}

\begin{Def}
Un conjunto $D\in\mathcal{B}_\mathbb{X}$ es llamado peque\~no si
existe un $t>0$, una medida de probabilidad $\nu$ en
$\mathcal{B}_\mathbb{X}$, y un $\delta>0$ tal que
\[P^{t}\left(x,A\right)\geq\delta\nu\left(A\right)\] para $x\in
D,A\in\mathcal{B}_\mathbb{X}$.\footnote{En \cite{MeynTweedie}
muestran que si $P_{x}\left\{\tau_{D}<\infty\right\}\equiv1$
solamente para uno conjunto peque\~no, entonces el proceso es
Harris recurrente}
\end{Def}

%_____________________________________________________________________
\subsubsection{Definiciones y Descripci\'on del Modelo}
%________________________________________________________________________
El modelo est\'a compuesto por $c$ colas de capacidad infinita,
etiquetadas de $1$ a $c$ las cuales son atendidas por $s$
servidores. Los servidores atienden de acuerdo a una cadena de
Markov independiente $\left(X^{i}_{n}\right)_{n}$ con $1\leq i\leq
s$ y $n\in\left\{1,2,\ldots,c\right\}$ con la misma matriz de
transici\'on $r_{k,l}$ y \'unica medida invariante
$\left(p_{k}\right)$. Cada servidor permanece atendiendo en la
cola un periodo llamado de visita y determinada por la pol\'itica de
servicio asignada a la cola.

Los usuarios llegan a la cola $k$ con una tasa $\lambda_{k}$ y son
atendidos a una raz\'on $\mu_{k}$. Las sucesiones de tiempos de
interarribo $\left(\tau_{k}\left(n\right)\right)_{n}$, la de
tiempos de servicio
$\left(\sigma_{k}^{i}\left(n\right)\right)_{n}$ y la de tiempos de
cambio $\left(\sigma_{k,l}^{0,i}\left(n\right)\right)_{n}$
requeridas en la cola $k$ para el servidor $i$ son sucesiones
independientes e id\'enticamente distribuidas con distribuci\'on
general independiente de $i$, con media
$\sigma_{k}=\frac{1}{\mu_{k}}$, respectivamente
$\sigma_{k,l}^{0}=\frac{1}{\mu_{k,l}^{0}}$, e independiente de las
cadenas de Markov $\left(X^{i}_{n}\right)_{n}$. Adem\'as se supone
que los tiempos de interarribo se asume son acotados, para cada
$\rho_{k}=\lambda_{k}\sigma_{k}<s$ para asegurar la estabilidad de
la cola $k$ cuando opera como una cola $M/GM/1$.
%________________________________________________________________________
\subsubsection{Pol\'iticas de Servicio}
%_____________________________________________________________________
Una pol\'itica de servicio determina el n\'umero de usuarios que ser\'an
atendidos sin interrupci\'on en periodo de servicio por los
servidores que atienden a la cola. Para un solo servidor esta se
define a trav\'es de una funci\'on $f$ donde $f\left(x,a\right)$ es el
n\'umero de usuarios que son atendidos sin interrupci\'on cuando el
servidor llega a la cola y encuentra $x$ usuarios esperando dado
el tiempo transcurrido de interarribo $a$. Sea $v\left(x,a\right)$
la duraci\'on del periodo de servicio para una sola condici\'on
inicial $\left(x,a\right)$.

Las pol\'iticas de servicio consideradas satisfacen las siguientes
propiedades:

\begin{itemize}
\item[i)] Hay conservaci\'on del trabajo, es decir
\[v\left(x,a\right)=\sum_{l=1}^{f\left(x,a\right)}\sigma\left(l\right)\]
con $f\left(0,a\right)=v\left(0,a\right)=0$, donde
$\left(\sigma\left(l\right)\right)_{l}$ es una sucesi\'on
independiente e id\'enticamente distribuida de los tiempos de
servicio solicitados. \item[ii)] La selecci\'on de usuarios para se
atendidos es independiente de sus correspondientes tiempos de
servicio y del pasado hasta el inicio del periodo de servicio. As\'i
las distribuci\'on $\left(f,v\right)$ no depende del orden en el
cu\'al son atendidos los usuarios. \item[iii)] La pol\'itica de
servicio es mon\'otona en el sentido de que para cada $a\geq0$ los
n\'umeros $f\left(x,a\right)$ son mon\'otonos en distribuci\'on en $x$ y
su l\'imite en distribuci\'on cuando $x\rightarrow\infty$ es una
variable aleatoria $F^{*0}$ que no depende de $a$. \item[iv)] El
n\'umero de usuarios atendidos por cada servidor es acotado por
$f^{min}\left(x\right)$ de la longitud de la cola $x$ que adem\'as
converge mon\'otonamente en distribuci\'on a $F^{*}$ cuando
$x\rightarrow\infty$
\end{itemize}
%________________________________________________________________________
\subsubsection{Proceso de Estados}
%_____________________________________________________________________
El sistema de colas se describe por medio del proceso de Markov
$\left(X\left(t\right)\right)_{t\in\rea}$ como se define a
continuaci\'on. El estado del sistema al tiempo $t\geq0$ est\'a dado
por
\[X\left(t\right)=\left(Q\left(t\right),P\left(t\right),A\left(t\right),R\left(t\right),C\left(t\right)\right)\]
donde
\begin{itemize}
\item
$Q\left(t\right)=\left(Q_{k}\left(t\right)\right)_{k=1}^{c}$,
n\'umero de usuarios en la cola $k$ al tiempo $t$. \item
$P\left(t\right)=\left(P^{i}\left(t\right)\right)_{i=1}^{s}$, es
la posici\'on del servidor $i$. \item
$A\left(t\right)=\left(A_{k}\left(t\right)\right)_{k=1}^{c}$, es
el residual del tiempo de arribo en la cola $k$ al tiempo $t$.
\item
$R\left(t\right)=\left(R_{k}^{i}\left(t\right),R_{k,l}^{0,i}\left(t\right)\right)_{k,l,i=1}^{c,c,s}$,
el primero es el residual del tiempo de servicio del usuario
atendido por servidor $i$ en la cola $k$ al tiempo $t$, la segunda
componente es el residual del tiempo de cambio del servidor $i$ de
la cola $k$ a la cola $l$ al tiempo $t$. \item
$C\left(t\right)=\left(C_{k}^{i}\left(t\right)\right)_{k,i=1}^{c,s}$,
es la componente correspondiente a la cola $k$ y al servidor $i$
que est\'a determinada por la pol\'itica de servicio en la cola $k$
y que hace al proceso $X\left(t\right)$ un proceso de Markov.
\end{itemize}
Todos los procesos definidos arriba se suponen continuos por la
derecha.

El proceso $X$ tiene la propiedad fuerte de Markov y su espacio de
estados es el espacio producto
\[\mathcal{X}=\nat^{c}\times E^{s}\times \rea_{+}^{c}\times\rea_{+}^{cs}\times\rea_{+}^{c^{2}s}\times \mathcal{C}\] donde $E=\left\{1,2,\ldots,c\right\}^{2}\cup\left\{1,2,\ldots,c\right\}$ y $\mathcal{C}$  depende de las pol\'iticas de servicio.

%_____________________________________________________________________________________
\subsubsection{Introducci{\'o}n}
%_____________________________________________________________________________________
%


Si $x$ es el n{\'u}mero de usuarios en la cola al comienzo del
periodo de servicio y $N_{s}\left(x\right)=N\left(x\right)$ es el
n{\'u}mero de usuarios que son atendidos con la pol{\'\i}tica $s$,
{\'u}nica en nuestro caso durante un periodo de servicio, entonces
se asume que:
\begin{enumerate}
\item
\begin{equation}\label{S1}
lim_{x\rightarrow\infty}\esp\left[N\left(x\right)\right]=\overline{N}>0
\end{equation}
\item
\begin{equation}\label{S2}
\esp\left[N\left(x\right)\right]\leq \overline{N} \end{equation}
para cualquier valor de $x$.
\end{enumerate}
La manera en que atiende el servidor $m$-{\'e}simo, en este caso
en espec{\'\i}fico solo lo ilustraremos con un s{\'o}lo servidor,
es la siguiente:
\begin{itemize}
\item Al t{\'e}rmino de la visita a la cola $j$, el servidor se
cambia a la cola $j^{'}$ con probabilidad
$r_{j,j^{'}}^{m}=r_{j,j^{'}}$

\item La $n$-{\'e}sima ocurencia va acompa{\~n}ada con el tiempo
de cambio de longitud $\delta_{j,j^{'}}\left(n\right)$,
independientes e id{\'e}nticamente distribuidas, con
$\esp\left[\delta_{j,j^{'}}\left(1\right)\right]\geq0$.

\item Sea $\left\{p_{j}\right\}$ la {\'u}nica distribuci{\'o}n
invariante estacionaria para la Cadena de Markov con matriz de
transici{\'o}n $\left(r_{j,j^{'}}\right)$.

\item Finalmente, se define
\begin{equation}
\delta^{*}:=\sum_{j,j^{'}}p_{j}r_{j,j^{'}}\esp\left[\delta_{j,j^{'}}\left(1\right)\right].
\end{equation}
\end{itemize}
%_____________________________________________________________________
\subsubsection{Colas C\'iclicas}
%_____________________________________________________________________
El {\em token passing ring} es una estaci\'on de un solo servidor
con $K$ clases de usuarios. Cada clase tiene su propio regulador
en la estaci\'on. Los usuarios llegan al regulador con raz\'on
$\alpha_{k}$ y son atendidos con taza $\mu_{k}$.

La red se puede modelar como un Proceso de Markov con espacio de
estados continuo, continuo en el tiempo:
\begin{equation}
 X\left(t\right)^{T}=\left(Q_{k}\left(t\right),A_{l}\left(t\right),B_{k}\left(t\right),B_{k}^{0}\left(t\right),C\left(t\right):k=1,\ldots,K,l\in\mathcal{A}\right)
\end{equation}
donde $Q_{k}\left(t\right), B_{k}\left(t\right)$ y
$A_{k}\left(t\right)$ se define como en \ref{Eq.Esp.Estados},
$B_{k}^{0}\left(t\right)$ es el tiempo residual de cambio de la
clase $k$ a la clase $k+1\left(mod K\right)$; $C\left(t\right)$
indica el n\'umero de servicios que han sido comenzados y/o
completados durante la sesi\'on activa del buffer.

Los par\'ametros cruciales son la carga nominal de la cola $k$:
$\beta_{k}=\alpha_{k}/\mu_{k}$ y la carga total es
$\rho_{0}=\sum\beta_{k}$, la media total del tiempo de cambio en
un ciclo del token est\'a definido por
\begin{equation}
 u^{0}=\sum_{k=1}^{K}\esp\left[\eta_{k}^{0}\left(1\right)\right]=\sum_{k=1}^{K}\frac{1}{\mu_{k}^{0}}
\end{equation}

El proceso de la longitud de la cola $Q_{k}^{x}\left(t\right)$ y
el proceso de acumulaci\'on del tiempo de servicio
$T_{k}^{x}\left(t\right)$ para el buffer $k$ y para el estado
inicial $x$ se definen como antes. Sea $T_{k}^{x,0}\left(t\right)$
el tiempo acumulado al tiempo $t$ que el token tarda en cambiar
del buffer $k$ al $k+1\mod K$. Suponga que la funci\'on
$\left(\overline{Q}\left(\cdot\right),\overline{T}\left(\cdot\right),\overline{T}^{0}\left(\cdot\right)\right)$
es un punto l\'imite de
\begin{equation}\label{Eq.4.4}
\left(\frac{1}{|x|}Q^{x}\left(|x|t\right),\frac{1}{|x|}T^{x}\left(|x|t\right),\frac{1}{|x|}T^{x,0}\left(|x|t\right)\right)
\end{equation}
cuando $|x|\rightarrow\infty$. Entonces
$\left(\overline{Q}\left(t\right),\overline{T}\left(t\right),\overline{T}^{0}\left(t\right)\right)$
es un flujo l\'imite retrasado del token ring.

Propiedades importantes para el modelo de flujo retrasado

\begin{Prop}
 Sea $\left(\overline{Q},\overline{T},\overline{T}^{0}\right)$ un flujo l\'imite de \ref{Eq.4.4} y suponga que cuando $x\rightarrow\infty$ a lo largo de
una subsucesi\'on
\[\left(\frac{1}{|x|}Q_{k}^{x}\left(0\right),\frac{1}{|x|}A_{k}^{x}\left(0\right),\frac{1}{|x|}B_{k}^{x}\left(0\right),\frac{1}{|x|}B_{k}^{x,0}\left(0\right)\right)\rightarrow\left(\overline{Q}_{k}\left(0\right),0,0,0\right)\]
para $k=1,\ldots,K$. EL flujo l\'imite tiene las siguientes
propiedades, donde las propiedades de la derivada se cumplen donde
la derivada exista:
\begin{itemize}
 \item[i)] Los vectores de tiempo ocupado $\overline{T}\left(t\right)$ y $\overline{T}^{0}\left(t\right)$ son crecientes y continuas con
$\overline{T}\left(0\right)=\overline{T}^{0}\left(0\right)=0$.
\item[ii)] Para todo $t\geq0$
\[\sum_{k=1}^{K}\left[\overline{T}_{k}\left(t\right)+\overline{T}_{k}^{0}\left(t\right)\right]=t\]
\item[iii)] Para todo $1\leq k\leq K$
\[\overline{Q}_{k}\left(t\right)=\overline{Q}_{k}\left(0\right)+\alpha_{k}t-\mu_{k}\overline{T}_{k}\left(t\right)\]
\item[iv)]  Para todo $1\leq k\leq K$
\[\dot{{\overline{T}}}_{k}\left(t\right)=\beta_{k}\] para $\overline{Q}_{k}\left(t\right)=0$.
\item[v)] Para todo $k,j$
\[\mu_{k}^{0}\overline{T}_{k}^{0}\left(t\right)=\mu_{j}^{0}\overline{T}_{j}^{0}\left(t\right)\]
\item[vi)]  Para todo $1\leq k\leq K$
\[\mu_{k}\dot{{\overline{T}}}_{k}\left(t\right)=l_{k}\mu_{k}^{0}\dot{{\overline{T}}}_{k}^{0}\left(t\right)\] para $\overline{Q}_{k}\left(t\right)>0$.
\end{itemize}
\end{Prop}

%_____________________________________________________________________
\subsubsection{Resultados Previos}
%_____________________________________________________________________

\begin{Lemma}\label{Lema.34.MeynDown}
El proceso estoc\'astico $\Phi$ es un proceso de markov fuerte,
temporalmente homog\'eneo, con trayectorias muestrales continuas
por la derecha, cuyo espacio de estados $Y$ es igual a
$X\times\rea$
\end{Lemma}
\begin{Prop}
 Suponga que los supuestos A1) y A2) son ciertos y que el modelo de flujo es estable. Entonces existe $t_{0}>0$ tal que
\begin{equation}
 lim_{|x|\rightarrow\infty}\frac{1}{|x|^{p+1}}\esp_{x}\left[|X\left(t_{0}|x|\right)|^{p+1}\right]=0
\end{equation}
\end{Prop}

\begin{Lemma}\label{Lema.5.2}
 Sea $\left\{\zeta\left(k\right):k\in \mathbb{z}\right\}$ una sucesi\'on independiente e id\'enticamente distribuida que toma valores en $\left(0,\infty\right)$,
y sea
$E\left(t\right)=max\left(n\geq1:\zeta\left(1\right)+\cdots+\zeta\left(n-1\right)\leq
t\right)$. Si $\esp\left[\zeta\left(1\right)\right]<\infty$,
entonces para cualquier entero $r\geq1$
\begin{equation}
 lim_{t\rightarrow\infty}\esp\left[\left(\frac{E\left(t\right)}{t}\right)^{r}\right]=\left(\frac{1}{\esp\left[\zeta_{1}\right]}\right)^{r}.
\end{equation}
Luego, bajo estas condiciones:
\begin{itemize}
 \item[a)] para cualquier $\delta>0$, $\sup_{t\geq\delta}\esp\left[\left(\frac{E\left(t\right)}{t}\right)^{r}\right]<\infty$
\item[b)] las variables aleatorias
$\left\{\left(\frac{E\left(t\right)}{t}\right)^{r}:t\geq1\right\}$
son uniformemente integrables.
\end{itemize}
\end{Lemma}

\begin{Teo}\label{Tma.5.5}
Suponga que los supuestos A1) y A2) se cumplen y que el modelo de
flujo es estable. Entonces existe una constante $\kappa_{p}$ tal
que
\begin{equation}
\frac{1}{t}\int_{0}^{t}\esp_{x}\left[|Q\left(s\right)|^{p}\right]ds\leq\kappa_{p}\left\{\frac{1}{t}|x|^{p+1}+1\right\}
\end{equation}
para $t>0$ y $x\in X$. En particular, para cada condici\'on inicial
\begin{eqnarray*}
\limsup_{t\rightarrow\infty}\frac{1}{t}\int_{0}^{t}\esp_{x}\left[|Q\left(s\right)|^{p}\right]ds\leq\kappa_{p}.
\end{eqnarray*}
\end{Teo}

\begin{Teo}\label{Tma.6.2}
Suponga que se cumplen los supuestos A1), A2) y A3) y que el
modelo de flujo es estable. Entonces se tiene que
\begin{equation}
|\left|P^{t}\left(x,\cdot\right)-\pi\left(\cdot\right)\right||_{f_{p}}\textrm{,
}t\rightarrow\infty,x\in X.
\end{equation}
En particular para cada condici\'on inicial
\begin{eqnarray*}
\lim_{t\rightarrow\infty}\esp_{x}\left[|Q\left(t\right)|^{p}\right]=\esp_{\pi}\left[|Q\left(0\right)|^{p}\right]\leq\kappa_{r}
\end{eqnarray*}
\end{Teo}
\begin{Teo}\label{Tma.6.3}
Suponga que se cumplen los supuestos A1), A2) y A3) y que el
modelo de flujo es estable. Entonces con
$f\left(x\right)=f_{1}\left(x\right)$ se tiene
\begin{equation}
\lim_{t\rightarrow\infty}t^{p-1}|\left|P^{t}\left(x,\cdot\right)-\pi\left(\cdot\right)\right||_{f}=0.
\end{equation}
En particular para cada condici\'on inicial
\begin{eqnarray*}
\lim_{t\rightarrow\infty}t^{p-1}|\esp_{x}\left[Q\left(t\right)\right]-\esp_{\pi}\left[Q\left(0\right)\right]|=0.
\end{eqnarray*}
\end{Teo}

\begin{Teo}\label{Tma.6.4}
Suponga que se cumplen los supuestos A1), A2) y A3) y que el
modelo de flujo es estable. Sea $\nu$ cualquier distribuci\'on de
probabilidad en $\left(X,\mathcal{B}_{X}\right)$, y $\pi$ la
distribuci\'on estacionaria de $X$.
\begin{itemize}
\item[i)] Para cualquier $f:X\leftarrow\rea_{+}$
\begin{equation}
\lim_{t\rightarrow\infty}\frac{1}{t}\int_{o}^{t}f\left(X\left(s\right)\right)ds=\pi\left(f\right):=\int
f\left(x\right)\pi\left(dx\right)
\end{equation}
$\prob$-c.s. \item[ii)] Para cualquier $f:X\leftarrow\rea_{+}$ con
$\pi\left(|f|\right)<\infty$, la ecuaci\'on anterior se cumple.
\end{itemize}
\end{Teo}

%_____________________________________________________________________________________
%
\subsubsection{Teorema de Estabilidad: Descripci{\'o}n}
%_____________________________________________________________________________________
%


Si $x$ es el n{\'u}mero de usuarios en la cola al comienzo del
periodo de servicio y $N_{s}\left(x\right)=N\left(x\right)$ es el
n{\'u}mero de usuarios que son atendidos con la pol{\'\i}tica $s$,
{\'u}nica en nuestro caso durante un periodo de servicio, entonces
se asume que:
\begin{enumerate}
\item
\begin{equation}\label{S1}
lim_{x\rightarrow\infty}\esp\left[N\left(x\right)\right]=\overline{N}>0
\end{equation}
\item
\begin{equation}\label{S2}
\esp\left[N\left(x\right)\right]\leq \overline{N} \end{equation}
para cualquier valor de $x$.
\end{enumerate}
La manera en que atiende el servidor $m$-{\'e}simo, en este caso
en espec{\'\i}fico solo lo ilustraremos con un s{\'o}lo servidor,
es la siguiente:
\begin{itemize}
\item Al t{\'e}rmino de la visita a la cola $j$, el servidor se
cambia a la cola $j^{'}$ con probabilidad
$r_{j,j^{'}}^{m}=r_{j,j^{'}}$

\item La $n$-{\'e}sima ocurencia va acompa{\~n}ada con el tiempo
de cambio de longitud $\delta_{j,j^{'}}\left(n\right)$,
independientes e id{\'e}nticamente distribuidas, con
$\esp\left[\delta_{j,j^{'}}\left(1\right)\right]\geq0$.

\item Sea $\left\{p_{j}\right\}$ la {\'u}nica distribuci{\'o}n
invariante estacionaria para la Cadena de Markov con matriz de
transici{\'o}n $\left(r_{j,j^{'}}\right)$.

\item Finalmente, se define
\begin{equation}
\delta^{*}:=\sum_{j,j^{'}}p_{j}r_{j,j^{'}}\esp\left[\delta_{j,j^{'}}\left(1\right)\right].
\end{equation}
\end{itemize}

%_________________________________________________________________________
\subsection{Supuestos}
%_________________________________________________________________________
Consideremos el caso en el que se tienen varias colas a las cuales
llegan uno o varios servidores para dar servicio a los usuarios
que se encuentran presentes en la cola, como ya se mencion\'o hay
varios tipos de pol\'iticas de servicio, incluso podr\'ia ocurrir
que la manera en que atiende al resto de las colas sea distinta a
como lo hizo en las anteriores.\\

Para ejemplificar los sistemas de visitas c\'iclicas se
considerar\'a el caso en que a las colas los usuarios son atendidos con
una s\'ola pol\'itica de servicio.\\



Si $\omega$ es el n\'umero de usuarios en la cola al comienzo del
periodo de servicio y $N\left(\omega\right)$ es el n\'umero de
usuarios que son atendidos con una pol\'itica en espec\'ifico
durante el periodo de servicio, entonces se asume que:
\begin{itemize}
\item[1)]\label{S1}$lim_{\omega\rightarrow\infty}\esp\left[N\left(\omega\right)\right]=\overline{N}>0$;
\item[2)]\label{S2}$\esp\left[N\left(\omega\right)\right]\leq\overline{N}$
para cualquier valor de $\omega$.
\end{itemize}
La manera en que atiende el servidor $m$-\'esimo, es la siguiente:
\begin{itemize}
\item Al t\'ermino de la visita a la cola $j$, el servidor cambia
a la cola $j^{'}$ con probabilidad $r_{j,j^{'}}^{m}$

\item La $n$-\'esima vez que el servidor cambia de la cola $j$ a
$j'$, va acompa\~nada con el tiempo de cambio de longitud
$\delta_{j,j^{'}}^{m}\left(n\right)$, con
$\delta_{j,j^{'}}^{m}\left(n\right)$, $n\geq1$, variables
aleatorias independientes e id\'enticamente distribuidas, tales
que $\esp\left[\delta_{j,j^{'}}^{m}\left(1\right)\right]\geq0$.

\item Sea $\left\{p_{j}^{m}\right\}$ la distribuci\'on invariante
estacionaria \'unica para la Cadena de Markov con matriz de
transici\'on $\left(r_{j,j^{'}}^{m}\right)$, se supone que \'esta
existe.

\item Finalmente, se define el tiempo promedio total de traslado
entre las colas como
\begin{equation}
\delta^{*}:=\sum_{j,j^{'}}p_{j}^{m}r_{j,j^{'}}^{m}\esp\left[\delta_{j,j^{'}}^{m}\left(i\right)\right].
\end{equation}
\end{itemize}

Consideremos el caso donde los tiempos entre arribo a cada una de
las colas, $\left\{\xi_{k}\left(n\right)\right\}_{n\geq1}$ son
variables aleatorias independientes a id\'enticamente
distribuidas, y los tiempos de servicio en cada una de las colas
se distribuyen de manera independiente e id\'enticamente
distribuidas $\left\{\eta_{k}\left(n\right)\right\}_{n\geq1}$;
adem\'as ambos procesos cumplen la condici\'on de ser
independientes entre s\'i. Para la $k$-\'esima cola se define la
tasa de arribo por
$\lambda_{k}=1/\esp\left[\xi_{k}\left(1\right)\right]$ y la tasa
de servicio como
$\mu_{k}=1/\esp\left[\eta_{k}\left(1\right)\right]$, finalmente se
define la carga de la cola como $\rho_{k}=\lambda_{k}/\mu_{k}$,
donde se pide que $\rho=\sum_{k=1}^{K}\rho_{k}<1$, para garantizar
la estabilidad del sistema, esto es cierto para las pol\'iticas de
servicio exhaustiva y cerrada, ver Geetor \cite{Getoor}.\\

Si denotamos por
\begin{itemize}
\item $Q_{k}\left(t\right)$ el n\'umero de usuarios presentes en
la cola $k$ al tiempo $t$; \item $A_{k}\left(t\right)$ los
residuales de los tiempos entre arribos a la cola $k$; para cada
servidor $m$; \item $B_{m}\left(t\right)$ denota a los residuales
de los tiempos de servicio al tiempo $t$; \item
$B_{m}^{0}\left(t\right)$ los residuales de los tiempos de
traslado de la cola $k$ a la pr\'oxima por atender al tiempo $t$,

\item sea
$C_{m}\left(t\right)$ el n\'umero de usuarios atendidos durante la
visita del servidor a la cola $k$ al tiempo $t$.
\end{itemize}


En este sentido, el proceso para el sistema de visitas se puede
definir como:

\begin{equation}\label{Esp.Edos.Down}
X\left(t\right)^{T}=\left(Q_{k}\left(t\right),A_{k}\left(t\right),B_{m}\left(t\right),B_{m}^{0}\left(t\right),C_{m}\left(t\right)\right),
\end{equation}
para $k=1,\ldots,K$ y $m=1,2,\ldots,M$, donde $T$ indica que es el
transpuesto del vector que se est\'a definiendo. El proceso $X$
evoluciona en el espacio de estados:
$\mathbb{X}=\ent_{+}^{K}\times\rea_{+}^{K}\times\left(\left\{1,2,\ldots,K\right\}\times\left\{1,2,\ldots,S\right\}\right)^{M}\times\rea_{+}^{K}\times\ent_{+}^{K}$.\\

El sistema aqu\'i descrito debe de cumplir con los siguientes supuestos b\'asicos de un sistema de visitas:
%__________________________________________________________________________
\subsubsection{Supuestos B\'asicos}
%__________________________________________________________________________
\begin{itemize}
\item[A1)] Los procesos
$\xi_{1},\ldots,\xi_{K},\eta_{1},\ldots,\eta_{K}$ son mutuamente
independientes y son sucesiones independientes e id\'enticamente
distribuidas.

\item[A2)] Para alg\'un entero $p\geq1$
\begin{eqnarray*}
\esp\left[\xi_{l}\left(1\right)^{p+1}\right]&<&\infty\textrm{ para }l=1,\ldots,K\textrm{ y }\\
\esp\left[\eta_{k}\left(1\right)^{p+1}\right]&<&\infty\textrm{
para }k=1,\ldots,K.
\end{eqnarray*}
donde $\mathcal{A}$ es la clase de posibles arribos.

\item[A3)] Para cada $k=1,2,\ldots,K$ existe una funci\'on
positiva $q_{k}\left(\cdot\right)$ definida en $\rea_{+}$, y un
entero $j_{k}$, tal que
\begin{eqnarray}
P\left(\xi_{k}\left(1\right)\geq x\right)&>&0\textrm{, para todo }x>0,\\
P\left\{a\leq\sum_{i=1}^{j_{k}}\xi_{k}\left(i\right)\leq
b\right\}&\geq&\int_{a}^{b}q_{k}\left(x\right)dx, \textrm{ }0\leq
a<b.
\end{eqnarray}
\end{itemize}

En lo que respecta al supuesto (A3), en Dai y Meyn \cite{DaiSean}
hacen ver que este se puede sustituir por

\begin{itemize}
\item[A3')] Para el Proceso de Markov $X$, cada subconjunto
compacto del espacio de estados de $X$ es un conjunto peque\~no,
ver definici\'on \ref{Def.Cto.Peq.}.
\end{itemize}

Es por esta raz\'on que con la finalidad de poder hacer uso de
$A3^{'})$ es necesario recurrir a los Procesos de Harris y en
particular a los Procesos Harris Recurrente, ver \cite{Dai,
DaiSean}.
%_______________________________________________________________________
\subsection{Procesos Harris Recurrente}
%_______________________________________________________________________

Por el supuesto (A1) conforme a Davis \cite{Davis}, se puede
definir el proceso de saltos correspondiente de manera tal que
satisfaga el supuesto (A3'), de hecho la demostraci\'on est\'a
basada en la l\'inea de argumentaci\'on de Davis, \cite{Davis},
p\'aginas 362-364.\\

Entonces se tiene un espacio de estados en el cual el proceso $X$
satisface la Propiedad Fuerte de Markov, ver Dai y Meyn
\cite{DaiSean}, dado por

\[\left(\Omega,\mathcal{F},\mathcal{F}_{t},X\left(t\right),\theta_{t},P_{x}\right),\]
adem\'as de ser un proceso de Borel Derecho (Sharpe \cite{Sharpe})
en el espacio de estados medible
$\left(\mathbb{X},\mathcal{B}_\mathbb{X}\right)$. El Proceso
$X=\left\{X\left(t\right),t\geq0\right\}$ tiene trayectorias
continuas por la derecha, est\'a definido en
$\left(\Omega,\mathcal{F}\right)$ y est\'a adaptado a
$\left\{\mathcal{F}_{t},t\geq0\right\}$; la colecci\'on
$\left\{P_{x},x\in \mathbb{X}\right\}$ son medidas de probabilidad
en $\left(\Omega,\mathcal{F}\right)$ tales que para todo $x\in
\mathbb{X}$
\[P_{x}\left\{X\left(0\right)=x\right\}=1,\] y
\[E_{x}\left\{f\left(X\circ\theta_{t}\right)|\mathcal{F}_{t}\right\}=E_{X}\left(\tau\right)f\left(X\right),\]
en $\left\{\tau<\infty\right\}$, $P_{x}$-c.s., con $\theta_{t}$
definido como el operador shift.


Donde $\tau$ es un $\mathcal{F}_{t}$-tiempo de paro
\[\left(X\circ\theta_{\tau}\right)\left(w\right)=\left\{X\left(\tau\left(w\right)+t,w\right),t\geq0\right\},\]
y $f$ es una funci\'on de valores reales acotada y medible, ver \cite{Dai, KaspiMandelbaum}.\\

Sea $P^{t}\left(x,D\right)$, $D\in\mathcal{B}_{\mathbb{X}}$,
$t\geq0$ la probabilidad de transici\'on de $X$ queda definida
como:
\[P^{t}\left(x,D\right)=P_{x}\left(X\left(t\right)\in
D\right).\]


\begin{Def}
Una medida no cero $\pi$ en
$\left(\mathbb{X},\mathcal{B}_{\mathbb{X}}\right)$ es invariante
para $X$ si $\pi$ es $\sigma$-finita y
\[\pi\left(D\right)=\int_{\mathbb{X}}P^{t}\left(x,D\right)\pi\left(dx\right),\]
para todo $D\in \mathcal{B}_{\mathbb{X}}$, con $t\geq0$.
\end{Def}

\begin{Def}
El proceso de Markov $X$ es llamado Harris Recurrente si existe
una medida de probabilidad $\nu$ en
$\left(\mathbb{X},\mathcal{B}_{\mathbb{X}}\right)$, tal que si
$\nu\left(D\right)>0$ y $D\in\mathcal{B}_{\mathbb{X}}$
\[P_{x}\left\{\tau_{D}<\infty\right\}\equiv1,\] cuando
$\tau_{D}=inf\left\{t\geq0:X_{t}\in D\right\}$.
\end{Def}

\begin{Note}
\begin{itemize}
\item[i)] Si $X$ es Harris recurrente, entonces existe una \'unica
medida invariante $\pi$ (Getoor \cite{Getoor}).

\item[ii)] Si la medida invariante es finita, entonces puede
normalizarse a una medida de probabilidad, en este caso al proceso
$X$ se le llama Harris recurrente positivo.


\item[iii)] Cuando $X$ es Harris recurrente positivo se dice que
la disciplina de servicio es estable. En este caso $\pi$ denota la
distribuci\'on estacionaria y hacemos
\[P_{\pi}\left(\cdot\right)=\int_{\mathbf{X}}P_{x}\left(\cdot\right)\pi\left(dx\right),\]
y se utiliza $E_{\pi}$ para denotar el operador esperanza
correspondiente, ver \cite{DaiSean}.
\end{itemize}
\end{Note}

\begin{Def}\label{Def.Cto.Peq.}
Un conjunto $D\in\mathcal{B_{\mathbb{X}}}$ es llamado peque\~no si
existe un $t>0$, una medida de probabilidad $\nu$ en
$\mathcal{B_{\mathbb{X}}}$, y un $\delta>0$ tal que
\[P^{t}\left(x,A\right)\geq\delta\nu\left(A\right),\] para $x\in
D,A\in\mathcal{B_{\mathbb{X}}}$.
\end{Def}

La siguiente serie de resultados vienen enunciados y demostrados
en Dai \cite{Dai}:
\begin{Lema}[Lema 3.1, Dai \cite{Dai}]
Sea $B$ conjunto peque\~no cerrado, supongamos que
$P_{x}\left(\tau_{B}<\infty\right)\equiv1$ y que para alg\'un
$\delta>0$ se cumple que
\begin{equation}\label{Eq.3.1}
\sup\esp_{x}\left[\tau_{B}\left(\delta\right)\right]<\infty,
\end{equation}
donde
$\tau_{B}\left(\delta\right)=inf\left\{t\geq\delta:X\left(t\right)\in
B\right\}$. Entonces, $X$ es un proceso Harris recurrente
positivo.
\end{Lema}

\begin{Lema}[Lema 3.1, Dai \cite{Dai}]\label{Lema.3.}
Bajo el supuesto (A3), el conjunto
$B=\left\{x\in\mathbb{X}:|x|\leq k\right\}$ es un conjunto
peque\~no cerrado para cualquier $k>0$.
\end{Lema}

\begin{Teo}[Teorema 3.1, Dai \cite{Dai}]\label{Tma.3.1}
Si existe un $\delta>0$ tal que
\begin{equation}
lim_{|x|\rightarrow\infty}\frac{1}{|x|}\esp|X^{x}\left(|x|\delta\right)|=0,
\end{equation}
donde $X^{x}$ se utiliza para denotar que el proceso $X$ comienza
a partir de $x$, entonces la ecuaci\'on (\ref{Eq.3.1}) se cumple
para $B=\left\{x\in\mathbb{X}:|x|\leq k\right\}$ con alg\'un
$k>0$. En particular, $X$ es Harris recurrente positivo.
\end{Teo}

Entonces, tenemos que el proceso $X$ es un proceso de Markov que
cumple con los supuestos $A1)$-$A3)$, lo que falta de hacer es
construir el Modelo de Flujo bas\'andonos en lo hasta ahora
presentado.
%_______________________________________________________________________
\subsection{Modelo de Flujo}
%_______________________________________________________________________

Dada una condici\'on inicial $x\in\mathbb{X}$, sea

\begin{itemize}
\item $Q_{k}^{x}\left(t\right)$ la longitud de la cola al tiempo
$t$,

\item $T_{m,k}^{x}\left(t\right)$ el tiempo acumulado, al tiempo
$t$, que tarda el servidor $m$ en atender a los usuarios de la
cola $k$.

\item $T_{m,k}^{x,0}\left(t\right)$ el tiempo acumulado, al tiempo
$t$, que tarda el servidor $m$ en trasladarse a otra cola a partir de la $k$-\'esima.\\
\end{itemize}

Sup\'ongase que la funci\'on
$\left(\overline{Q}\left(\cdot\right),\overline{T}_{m}
\left(\cdot\right),\overline{T}_{m}^{0} \left(\cdot\right)\right)$
para $m=1,2,\ldots,M$ es un punto l\'imite de
\begin{equation}\label{Eq.Punto.Limite}
\left(\frac{1}{|x|}Q^{x}\left(|x|t\right),\frac{1}{|x|}T_{m}^{x}\left(|x|t\right),\frac{1}{|x|}T_{m}^{x,0}\left(|x|t\right)\right)
\end{equation}
para $m=1,2,\ldots,M$, cuando $x\rightarrow\infty$, ver
\cite{Down}. Entonces
$\left(\overline{Q}\left(t\right),\overline{T}_{m}
\left(t\right),\overline{T}_{m}^{0} \left(t\right)\right)$ es un
flujo l\'imite del sistema. Al conjunto de todos las posibles
flujos l\'imite se le llama {\emph{Modelo de Flujo}} y se le
denotar\'a por $\mathcal{Q}$, ver \cite{Down, Dai, DaiSean}.\\

El modelo de flujo satisface el siguiente conjunto de ecuaciones:

\begin{equation}\label{Eq.MF.1}
\overline{Q}_{k}\left(t\right)=\overline{Q}_{k}\left(0\right)+\lambda_{k}t-\sum_{m=1}^{M}\mu_{k}\overline{T}_{m,k}\left(t\right),\\
\end{equation}
para $k=1,2,\ldots,K$.\\
\begin{equation}\label{Eq.MF.2}
\overline{Q}_{k}\left(t\right)\geq0\textrm{ para
}k=1,2,\ldots,K.\\
\end{equation}

\begin{equation}\label{Eq.MF.3}
\overline{T}_{m,k}\left(0\right)=0,\textrm{ y }\overline{T}_{m,k}\left(\cdot\right)\textrm{ es no decreciente},\\
\end{equation}
para $k=1,2,\ldots,K$ y $m=1,2,\ldots,M$.\\
\begin{equation}\label{Eq.MF.4}
\sum_{k=1}^{K}\overline{T}_{m,k}^{0}\left(t\right)+\overline{T}_{m,k}\left(t\right)=t\textrm{
para }m=1,2,\ldots,M.\\
\end{equation}


\begin{Def}[Definici\'on 4.1, Dai \cite{Dai}]\label{Def.Modelo.Flujo}
Sea una disciplina de servicio espec\'ifica. Cualquier l\'imite
$\left(\overline{Q}\left(\cdot\right),\overline{T}\left(\cdot\right),\overline{T}^{0}\left(\cdot\right)\right)$
en (\ref{Eq.Punto.Limite}) es un {\em flujo l\'imite} de la
disciplina. Cualquier soluci\'on (\ref{Eq.MF.1})-(\ref{Eq.MF.4})
es llamado flujo soluci\'on de la disciplina.
\end{Def}

\begin{Def}
Se dice que el modelo de flujo l\'imite, modelo de flujo, de la
disciplina de la cola es estable si existe una constante
$\delta>0$ que depende de $\mu,\lambda$ y $P$ solamente, tal que
cualquier flujo l\'imite con
$|\overline{Q}\left(0\right)|+|\overline{U}|+|\overline{V}|=1$, se
tiene que $\overline{Q}\left(\cdot+\delta\right)\equiv0$.
\end{Def}

Si se hace $|x|\rightarrow\infty$ sin restringir ninguna de las
componentes, tambi\'en se obtienen un modelo de flujo, pero en
este caso el residual de los procesos de arribo y servicio
introducen un retraso:
\begin{Teo}[Teorema 4.2, Dai \cite{Dai}]\label{Tma.4.2.Dai}
Sea una disciplina fija para la cola, suponga que se cumplen las
condiciones (A1)-(A3). Si el modelo de flujo l\'imite de la
disciplina de la cola es estable, entonces la cadena de Markov $X$
que describe la din\'amica de la red bajo la disciplina es Harris
recurrente positiva.
\end{Teo}

Ahora se procede a escalar el espacio y el tiempo para reducir la
aparente fluctuaci\'on del modelo. Consid\'erese el proceso
\begin{equation}\label{Eq.3.7}
\overline{Q}^{x}\left(t\right)=\frac{1}{|x|}Q^{x}\left(|x|t\right).
\end{equation}
A este proceso se le conoce como el flujo escalado, y cualquier
l\'imite $\overline{Q}^{x}\left(t\right)$ es llamado flujo
l\'imite del proceso de longitud de la cola. Haciendo
$|q|\rightarrow\infty$ mientras se mantiene el resto de las
componentes fijas, cualquier punto l\'imite del proceso de
longitud de la cola normalizado $\overline{Q}^{x}$ es soluci\'on
del siguiente modelo de flujo.


\begin{Def}[Definici\'on 3.3, Dai y Meyn \cite{DaiSean}]
El modelo de flujo es estable si existe un tiempo fijo $t_{0}$ tal
que $\overline{Q}\left(t\right)=0$, con $t\geq t_{0}$, para
cualquier $\overline{Q}\left(\cdot\right)\in\mathcal{Q}$ que
cumple con $|\overline{Q}\left(0\right)|=1$.
\end{Def}

\begin{Lemma}[Lema 3.1, Dai y Meyn \cite{DaiSean}]
Si el modelo de flujo definido por (\ref{Eq.MF.1})-(\ref{Eq.MF.4})
es estable, entonces el modelo de flujo retrasado es tambi\'en
estable, es decir, existe $t_{0}>0$ tal que
$\overline{Q}\left(t\right)=0$ para cualquier $t\geq t_{0}$, para
cualquier soluci\'on del modelo de flujo retrasado cuya
condici\'on inicial $\overline{x}$ satisface que
$|\overline{x}|=|\overline{Q}\left(0\right)|+|\overline{A}\left(0\right)|+|\overline{B}\left(0\right)|\leq1$.
\end{Lemma}


Ahora ya estamos en condiciones de enunciar los resultados principales:


\begin{Teo}[Teorema 2.1, Down \cite{Down}]\label{Tma2.1.Down}
Suponga que el modelo de flujo es estable, y que se cumplen los supuestos (A1) y (A2), entonces
\begin{itemize}
\item[i)] Para alguna constante $\kappa_{p}$, y para cada
condici\'on inicial $x\in X$
\begin{equation}\label{Estability.Eq1}
\limsup_{t\rightarrow\infty}\frac{1}{t}\int_{0}^{t}\esp_{x}\left[|Q\left(s\right)|^{p}\right]ds\leq\kappa_{p},
\end{equation}
donde $p$ es el entero dado en (A2).
\end{itemize}
Si adem\'as se cumple la condici\'on (A3), entonces para cada
condici\'on inicial:
\begin{itemize}
\item[ii)] Los momentos transitorios convergen a su estado
estacionario:
 \begin{equation}\label{Estability.Eq2}
lim_{t\rightarrow\infty}\esp_{x}\left[Q_{k}\left(t\right)^{r}\right]=\esp_{\pi}\left[Q_{k}\left(0\right)^{r}\right]\leq\kappa_{r},
\end{equation}
para $r=1,2,\ldots,p$ y $k=1,2,\ldots,K$. Donde $\pi$ es la
probabilidad invariante para $X$.

\item[iii)]  El primer momento converge con raz\'on $t^{p-1}$:
\begin{equation}\label{Estability.Eq3}
lim_{t\rightarrow\infty}t^{p-1}|\esp_{x}\left[Q_{k}\left(t\right)\right]-\esp_{\pi}\left[Q_{k}\left(0\right)\right]|=0.
\end{equation}

\item[iv)] La {\em Ley Fuerte de los grandes n\'umeros} se cumple:
\begin{equation}\label{Estability.Eq4}
lim_{t\rightarrow\infty}\frac{1}{t}\int_{0}^{t}Q_{k}^{r}\left(s\right)ds=\esp_{\pi}\left[Q_{k}\left(0\right)^{r}\right],\textrm{
}\prob_{x}\textrm{-c.s.}
\end{equation}
para $r=1,2,\ldots,p$ y $k=1,2,\ldots,K$.
\end{itemize}
\end{Teo}

La contribuci\'on de Down a la teor\'ia de los {\emph {sistemas de
visitas c\'iclicas}}, es la relaci\'on que hay entre la
estabilidad del sistema con el comportamiento de las medidas de
desempe\~no, es decir, la condici\'on suficiente para poder
garantizar la convergencia del proceso de la longitud de la cola
as\'i como de por los menos los dos primeros momentos adem\'as de
una versi\'on de la Ley Fuerte de los Grandes N\'umeros para los
sistemas de visitas.


\begin{Teo}[Teorema 2.3, Down \cite{Down}]\label{Tma2.3.Down}
Considere el siguiente valor:
\begin{equation}\label{Eq.Rho.1serv}
\rho=\sum_{k=1}^{K}\rho_{k}+max_{1\leq j\leq K}\left(\frac{\lambda_{j}}{\sum_{s=1}^{S}p_{js}\overline{N}_{s}}\right)\delta^{*}
\end{equation}
\begin{itemize}
\item[i)] Si $\rho<1$ entonces la red es estable, es decir, se
cumple el Teorema \ref{Tma2.1.Down}.

\item[ii)] Si $\rho>1$ entonces la red es inestable, es decir, se
cumple el Teorema \ref{Tma2.2.Down}
\end{itemize}
\end{Teo}



%_________________________________________________________________________
\subsection{Modelo de Flujo}
%_________________________________________________________________________
Sup\'ongase que el sistema consta de varias colas a los cuales
llegan uno o varios servidores a dar servicio a los usuarios
esperando en la cola.\\


Sea $x$ el n\'umero de usuarios en la cola esperando por servicio
y $N\left(x\right)$ es el n\'umero de usuarios que son atendidos
con una pol\'itica dada y fija mientras el servidor permanece
dando servicio, entonces se asume que:
\begin{itemize}
\item[(S1.)]
\begin{equation}\label{S1}
lim_{x\rightarrow\infty}\esp\left[N\left(x\right)\right]=\overline{N}>0.
\end{equation}
\item[(S2.)]
\begin{equation}\label{S2}
\esp\left[N\left(x\right)\right]\leq \overline{N},
\end{equation}

para cualquier valor de $x$.
\end{itemize}

El tiempo que tarda un servidor en volver a dar servicio despu\'es
de abandonar la cola inmediata anterior y llegar a la pr\'oxima se
llama tiempo de traslado o de cambio  de cola, al momento de la
$n$-\'esima visita del servidor a la cola $j$ se genera una
sucesi\'on de variables aleatorias $\delta_{j,j+1}\left(n\right)$,
independientes e id\'enticamente distribuidas, con la propiedad de
que $\esp\left[\delta_{j,j+1}\left(1\right)\right]\geq0$.\\


Se define
\begin{equation}
\delta^{*}:=\sum_{j,j+1}\esp\left[\delta_{j,j+1}\left(1\right)\right].
\end{equation}
%\begin{figure}[H]
%\centering
%\includegraphics[width=7cm]{switchovertime.jpg}
%\caption{Sistema de Visitas C\'iclicas}
%\end{figure}

Los tiempos entre arribos a la cola $k$, son de la forma
$\left\{\xi_{k}\left(n\right)\right\}_{n\geq1}$, con la propiedad
de que son independientes e id\'enticamente distribuidos. Los
tiempos de servicio
$\left\{\eta_{k}\left(n\right)\right\}_{n\geq1}$ tienen la
propiedad de ser independientes e id\'enticamente distribuidos.
Para la $k$-\'esima cola se define la tasa de arribo a la como
$\lambda_{k}=1/\esp\left[\xi_{k}\left(1\right)\right]$ y la tasa
de servicio como
$\mu_{k}=1/\esp\left[\eta_{k}\left(1\right)\right]$, finalmente se
define la carga de la cola como $\rho_{k}=\lambda_{k}/\mu_{k}$,
donde se pide que $\rho<1$, para garantizar la estabilidad del sistema.\\

%_____________________________________________________________________
%\subsubsection{Proceso de Estados}
%_____________________________________________________________________

Para el caso m\'as sencillo podemos definir un proceso de estados
para la red que depende de la pol\'itica de servicio utilizada, el
estado $\mathbb{X}\left(t\right)$ a cualquier tiempo $t$ puede
definirse como
\begin{equation}\label{Eq.Esp.Estados}
\mathbb{X}\left(t\right)=\left(Q_{k}\left(t\right),A_{l}\left(t\right),B_{k}\left(t\right):k=1,2,\ldots,K,l\in\mathcal{A}\right),
\end{equation}

donde $Q_{k}\left(t\right)$ es la longitud de la cola $k$ para los
usuarios esperando servicio, incluyendo aquellos que est\'an
siendo atendidos, $B_{k}\left(t\right)$ son los tiempos de
servicio residuales para los usuarios de la clase $k$ que est\'an
en servicio.\\

Los tiempos entre arribos residuales, que son el tiempo que queda
hasta que el pr\'oximo usuario llega a la cola para recibir
servicio, se denotan por $A_{k}\left(t\right)$. Tanto
$B_{k}\left(t\right)$ como $A_{k}\left(t\right)$ se suponen
continuos por la derecha \cite{Dai2}.\\

Sea $\mathcal{X}$ el espacio de estados para el proceso de estados
que por definici\'on es igual  al conjunto de posibles valores
para el estado $\mathbb{X}\left(t\right)$, y sea
$x=\left(q,a,b\right)$ un estado gen\'erico en $\mathbb{X}$, la
componente $q$ determina la posici\'on del usuario en la red,
$|q|$ denota la longitud total de la cola en la red.\\

Para un estado $x=\left(q,a,b\right)\in\mathbb{X}$ definimos la
{\em norma} de $x$ como $\left\|x\right\|=|q|+|a|+|b|$. En
\cite{Dai} se muestra que para una amplia serie de disciplinas de
servicio el proceso $\mathbb{X}$ es un Proceso Fuerte de Markov, y
por tanto se puede asumir que
\[\left(\left(\Omega,\mathcal{F}\right),\mathcal{F}_{t},\mathbb{X}\left(t\right),\theta_{t},P_{x}\right)\]
es un proceso de {\em Borel Derecho} en el espacio de estados
medible $\left(\mathcal{X},\mathcal{B}_{\mathcal{X}}\right)$.\\

Sea $P^{t}\left(x,D\right)$, $D\in\mathcal{B}_{\mathbb{X}}$,
$t\geq0$ probabilidad de transici\'on de $X$ definida como
\[P^{t}\left(x,D\right)=P_{x}\left(\mathbb{X}\left(t\right)\in
D\right).\]

\begin{Def}
Una medida no cero $\pi$ en
$\left(\mathbb{X},\mathcal{B}_{\mathbb{X}}\right)$ es {\em
invariante} para $X$ si $\pi$ es $\sigma$-finita y
\[\pi\left(D\right)=\int_{X}P^{t}\left(x,D\right)\pi\left(dx\right),\]
para todo $D\in \mathcal{B}_{\mathbb{X}}$, con $t\geq0$.
\end{Def}

\begin{Def}
El proceso de Markov $X$ es llamado {\em Harris recurrente} si
existe una medida de probabilidad $\nu$ en
$\left(\mathbb{X},\mathcal{B}_{\mathbb{X}}\right)$, tal que si
$\nu\left(D\right)>0$ y $D\in\mathcal{B}_{\mathbb{X}}$
\[P_{x}\left\{\tau_{D}<\infty\right\}\equiv1,\] cuando
$\tau_{D}=inf\left\{t\geq0:\mathbb{X}_{t}\in D\right\}$.
\end{Def}

\begin{Def}
Un conjunto $D\in\mathcal{B}_\mathbb{X}$ es llamado peque\~no si
existe un $t>0$, una medida de probabilidad $\nu$ en
$\mathcal{B}_\mathbb{X}$, y un $\delta>0$ tal que
\[P^{t}\left(x,A\right)\geq\delta\nu\left(A\right),\] para $x\in
D,A\in\mathcal{B}_\mathbb{X}$.
\end{Def}
\begin{Note}
\begin{itemize}

\item[i)] Si $X$ es Harris recurrente, entonces existe una \'unica medida
invariante $\pi$ (\cite{Getoor}).

\item[ii)] Si la medida invariante es finita, entonces puede
normalizarse a una medida de probabilidad, en este caso a la
medida se le llama \textbf{Harris recurrente positiva}.

\item[iii)] Cuando $X$ es Harris recurrente positivo se dice que
la disciplina de servicio es estable. En este caso $\pi$ denota la
ditribuci\'on estacionaria; se define
\[P_{\pi}\left(\cdot\right)=\int_{X}P_{x}\left(\cdot\right)\pi\left(dx\right).\]
Se utiliza $E_{\pi}$ para denotar el operador esperanza
correspondiente, as\'i, el proceso
$X=\left\{\mathbb{X}\left(t\right),t\geq0\right\}$ es un proceso
estrictamente estacionario bajo $P_{\pi}$.

\item[iv)] En \cite{MeynTweedie} se muestra que si
$P_{x}\left\{\tau_{D}<\infty\right\}=1$ incluso para solamente un
conjunto peque\~no, entonces el proceso de Harris es recurrente.
\end{itemize}
\end{Note}


%_________________________________________________________________________
%\newpage
%_________________________________________________________________________
%\subsection{Modelo de Flujo}
%_____________________________________________________________________
Las Colas C\'iclicas se pueden describir por medio de un proceso
de Markov $\left(X\left(t\right)\right)_{t\in\rea}$, donde el
estado del sistema al tiempo $t\geq0$ est\'a dado por
\begin{equation}
X\left(t\right)=\left(Q\left(t\right),A\left(t\right),H\left(t\right),B\left(t\right),B^{0}\left(t\right),C\left(t\right)\right)
\end{equation}
definido en el espacio producto:
\begin{equation}
\mathcal{X}=\mathbb{Z}^{K}\times\rea_{+}^{K}\times\left(\left\{1,2,\ldots,K\right\}\times\left\{1,2,\ldots,S\right\}\right)^{M}\times\rea_{+}^{K}\times\rea_{+}^{K}\times\mathbb{Z}^{K},
\end{equation}

\begin{itemize}
\item $Q\left(t\right)=\left(Q_{k}\left(t\right),1\leq k\leq
K\right)$, es el n\'umero de usuarios en la cola $k$, incluyendo
aquellos que est\'an siendo atendidos provenientes de la
$k$-\'esima cola.

\item $A\left(t\right)=\left(A_{k}\left(t\right),1\leq k\leq
K\right)$, son los residuales de los tiempos de arribo en la cola
$k$. \item $H\left(t\right)$ es el par ordenado que consiste en la
cola que esta siendo atendida y la pol\'itica de servicio que se
utilizar\'a.

\item $B\left(t\right)$ es el tiempo de servicio residual.

\item $B^{0}\left(t\right)$ es el tiempo residual del cambio de
cola.

\item $C\left(t\right)$ indica el n\'umero de usuarios atendidos
durante la visita del servidor a la cola dada en
$H\left(t\right)$.
\end{itemize}

$A_{k}\left(t\right),B_{m}\left(t\right)$ y
$B_{m}^{0}\left(t\right)$ se suponen continuas por la derecha y
que satisfacen la propiedad fuerte de Markov, (\cite{Dai}).

Dada una condici\'on inicial $x\in\mathcal{X}$, $Q_{k}^{x}\left(t\right)$ es la longitud de la cola $k$ al tiempo $t$
y $T_{m,k}^{x}\left(t\right)$  el tiempo acumulado al tiempo $t$ que el servidor tarda en atender a los usuarios de la cola $k$.
De igual manera se define $T_{m,k}^{x,0}\left(t\right)$ el tiempo acumulado al tiempo $t$ que el servidor tarda en
cambiar de cola para volver a atender a los usuarios.

Para reducir la fluctuaci\'on del modelo se escala tanto el espacio como el tiempo, entonces se
tiene el proceso:

\begin{eqnarray}
\overline{Q}^{x}\left(t\right)=\frac{1}{|x|}Q^{x}\left(|x|t\right),\\
\overline{T}_{m}^{x}\left(t\right)=\frac{1}{|x|}T_{m}^{x}\left(|x|t\right),\\
\overline{T}_{m}^{x,0}\left(t\right)=\frac{1}{|x|}T_{m}^{x,0}\left(|x|t\right).
\end{eqnarray}
Cualquier l\'imite $\overline{Q}\left(t\right)$ es llamado un
flujo l\'imite del proceso longitud de la cola, al conjunto de todos los posibles flujos l\'imite
se le llamar\'a \textbf{modelo de flujo}, (\cite{MaynDown}).

\begin{Def}
Un flujo l\'imite para un sistema de visitas bajo una disciplina de
servicio espec\'ifica se define como cualquier soluci\'on
 $\left(\overline{Q}\left(\cdot\right),\overline{T}_{m}\left(\cdot\right),\overline{T}_{m}^{0}\left(\cdot\right)\right)$
 de las siguientes ecuaciones, donde
$\overline{Q}\left(t\right)=\left(\overline{Q}_{1}\left(t\right),\ldots,\overline{Q}_{K}\left(t\right)\right)$
y
$\overline{T}\left(t\right)=\left(\overline{T}_{1}\left(t\right),\ldots,\overline{T}_{K}\left(t\right)\right)$
\begin{equation}\label{Eq.3.8}
\overline{Q}_{k}\left(t\right)=\overline{Q}_{k}\left(0\right)+\lambda_{k}t-\sum_{m=1}^{M}\mu_{k}\overline{T}_{m,k}\left(t\right)\\
\end{equation}
\begin{equation}\label{Eq.3.9}
\overline{Q}_{k}\left(t\right)\geq0\textrm{ para }k=1,2,\ldots,K,\\
\end{equation}
\begin{equation}\label{Eq.3.10}
\overline{T}_{m,k}\left(0\right)=0,\textrm{ y }\overline{T}_{m,k}\left(\cdot\right)\textrm{ es no decreciente},\\
\end{equation}
\begin{equation}\label{Eq.3.11}
\sum_{k=1}^{K}\overline{T}_{m,k}^{0}\left(t\right)+\overline{T}_{m,k}\left(t\right)=t\textrm{ para}m=1,2,\ldots,M\\
\end{equation}
\end{Def}

Al conjunto de ecuaciones dadas en (\ref{Eq.3.8})-(\ref{Eq.3.11}) se
le llama {\em Modelo de flujo} y al conjunto de todas las
soluciones del modelo de flujo
$\left(\overline{Q}\left(\cdot\right),\overline{T}
\left(\cdot\right)\right)$ se le denotar\'a por $\mathcal{Q}$.


\begin{Def}
El modelo de flujo es estable si existe un tiempo fijo $t_{0}$ tal
que $\overline{Q}\left(t\right)=0$, con $t\geq t_{0}$, para
cualquier $\overline{Q}\left(\cdot\right)\in\mathcal{Q}$ que
cumple con $|\overline{Q}\left(0\right)|=1$.
\end{Def}


%_____________________________________________________________________________________
%
\subsection{Estabilidad de los Sistemas de Visitas C\'iclicas}
%_________________________________________________________________________

Es necesario realizar los siguientes supuestos, ver (\cite{Dai2}) y (\cite{DaiSean}):

\begin{itemize}
\item[A1)] $\xi_{1},\ldots,\xi_{K},\eta_{1},\ldots,\eta_{K}$ son
mutuamente independientes y son sucesiones independientes e
id\'enticamente distribuidas.

\item[A2)] Para alg\'un entero $p\geq1$
\begin{eqnarray*}
\esp\left[\xi_{k}\left(1\right)^{p+1}\right]&<&\infty\textrm{ para }l\in\mathcal{A}\textrm{ y }\\
\esp\left[\eta_{k}\left(1\right)^{p+1}\right]&<&\infty\textrm{ para
}k=1,\ldots,K.
\end{eqnarray*}
\item[A3)] El conjunto $\left\{x\in X:|x|=0\right\}$ es un
singleton, y para cada $k\in\mathcal{A}$, existe una funci\'on
positiva $q_{k}\left(x\right)$ definida en $\rea_{+}$, y un entero
$j_{k}$, tal que
\begin{eqnarray}
P\left(\xi_{k}\left(1\right)\geq x\right)&>&0\textrm{, para todo }x>0\\
P\left(\xi_{k}\left(1\right)+\ldots\xi_{k}\left(j_{k}\right)\in dx\right)&\geq& q_{k}\left(x\right)dx0\textrm{ y }\\
\int_{0}^{\infty}q_{k}\left(x\right)dx>0
\end{eqnarray}
\end{itemize}


En \cite{MaynDown} ser da un argumento para deducir que todos los
subconjuntos compactos de $X$ son peque\~nos. Entonces la
condici\'on A3) se puede generalizar a
\begin{itemize}
\item[A3')] Para el proceso de Markov $X$, cada subconjunto
compacto de $X$ es peque\~no.
\end{itemize}

\begin{Teo}\label{Tma2.1}
Suponga que el modelo de flujo para una disciplina de servicio es
estable, y suponga adem\'as que las condiciones A1) y A2) se
satisfacen. Entonces:
\begin{itemize}
\item[i)] Para alguna constante $\kappa_{p}$, y para cada
condici\'on inicial $x\in X$
\begin{equation}
\limsup_{t\rightarrow\infty}\frac{1}{t}\int_{0}^{t}\esp_{x}\left[|Q\left(t\right)|^{p}\right]ds\leq\kappa_{p}
\end{equation}
donde $p$ es el entero dado por A2).
\end{itemize}

Suponga adem\'as que A3) o A3')
se cumple, entonces la disciplina de servicio es estable y adem\'as
para cada condici\'on inicial se tiene lo siguiente:
\begin{itemize}

\item[ii)] Los momentos transitorios convergen a sus valores en estado
estacionario:
\begin{equation}
\lim_{t\rightarrow\infty}\esp_{x}\left[Q_{k}\left(t\right)^{r}\right]=\esp_{\pi}\left[Q_{k}\left(0\right)^{r}\right]\leq\kappa_{r}
\end{equation}
para $r=1,\ldots,p$ y $k=1,\ldots,K$. \item[iii)] El primer
momento converge con raz\'on $t^{p-1}$:
\begin{equation}
\lim_{t\rightarrow\infty}t^{p-1}|\esp_{x}\left[Q\left(t\right)\right]-\esp_{\pi}\left[Q\left(0\right)\right]|=0.
\end{equation}
\item[iv)] Se cumple la Ley Fuerte de los Grandes N\'umeros:
\begin{equation}
\lim_{t\rightarrow\infty}\frac{1}{t}\int_{0}^{t}Q_{k}^{r}\left(s\right)ds=\esp_{\pi}\left[Q_{k}\left(0\right)^{r}\right]
\end{equation}
$\prob$-c.s., para $r=1,\ldots,p$ y $k=1,\ldots,K$.
\end{itemize}
\end{Teo}


\begin{Teo}\label{Tma2.2}
Suponga que el fluido modelo es inestable en el sentido de que
para alguna $\epsilon_{0},c_{0}\geq0$,
\begin{equation}\label{Eq.Inestability}
|Q\left(T\right)|\geq\epsilon_{0}T-c_{0}\textrm{, con }T\geq0,
\end{equation}
para cualquier condici\'on inicial $Q\left(0\right)$, con
$|Q\left(0\right)|=1$. Entonces para cualquier $0<q\leq1$, existe
$B<0$ tal que para cualquier $|x|\geq B$,
\begin{equation}
\prob_{x}\left\{\mathbb{X}\rightarrow\infty\right\}\geq q.
\end{equation}
\end{Teo}

%_____________________________________________________________________________________
%

%_____________________________________________________________________
\subsection{Resultados principales}
%_____________________________________________________________________
En el caso particular de un modelo con un solo servidor, $M=1$, se
tiene que si se define
\begin{Def}\label{Def.Ro}
\begin{equation}\label{RoM1}
\rho=\sum_{k=1}^{K}\rho_{k}+\max_{1\leq j\leq
K}\left(\frac{\lambda_{j}}{\overline{N}}\right)\delta^{*}.
\end{equation}
\end{Def}
entonces

\begin{Teo}\label{Teo.Down}
\begin{itemize}
\item[i)] Si $\rho<1$, entonces la red es estable, es decir el teorema
(\ref{Tma2.1}) se cumple. \item[ii)] De lo contrario, es decir, si
$\rho>1$ entonces la red es inestable, es decir, el teorema
(\ref{Tma2.2}).
\end{itemize}
\end{Teo}



%_________________________________________________________________________
\subsection{Supuestos}
%_________________________________________________________________________
Consideremos el caso en el que se tienen varias colas a las cuales
llegan uno o varios servidores para dar servicio a los usuarios
que se encuentran presentes en la cola, como ya se mencion\'o hay
varios tipos de pol\'iticas de servicio, incluso podr\'ia ocurrir
que la manera en que atiende al resto de las colas sea distinta a
como lo hizo en las anteriores.\\

Para ejemplificar los sistemas de visitas c\'iclicas se
considerar\'a el caso en que a las colas los usuarios son atendidos con
una s\'ola pol\'itica de servicio.\\


Si $\omega$ es el n\'umero de usuarios en la cola al comienzo del
periodo de servicio y $N\left(\omega\right)$ es el n\'umero de
usuarios que son atendidos con una pol\'itica en espec\'ifico
durante el periodo de servicio, entonces se asume que:
\begin{itemize}
\item[1)]\label{S1}$lim_{\omega\rightarrow\infty}\esp\left[N\left(\omega\right)\right]=\overline{N}>0$;
\item[2)]\label{S2}$\esp\left[N\left(\omega\right)\right]\leq\overline{N}$
para cualquier valor de $\omega$.
\end{itemize}
La manera en que atiende el servidor $m$-\'esimo, es la siguiente:
\begin{itemize}
\item Al t\'ermino de la visita a la cola $j$, el servidor cambia
a la cola $j^{'}$ con probabilidad $r_{j,j^{'}}^{m}$

\item La $n$-\'esima vez que el servidor cambia de la cola $j$ a
$j'$, va acompa\~nada con el tiempo de cambio de longitud
$\delta_{j,j^{'}}^{m}\left(n\right)$, con
$\delta_{j,j^{'}}^{m}\left(n\right)$, $n\geq1$, variables
aleatorias independientes e id\'enticamente distribuidas, tales
que $\esp\left[\delta_{j,j^{'}}^{m}\left(1\right)\right]\geq0$.

\item Sea $\left\{p_{j}^{m}\right\}$ la distribuci\'on invariante
estacionaria \'unica para la Cadena de Markov con matriz de
transici\'on $\left(r_{j,j^{'}}^{m}\right)$, se supone que \'esta
existe.

\item Finalmente, se define el tiempo promedio total de traslado
entre las colas como
\begin{equation}
\delta^{*}:=\sum_{j,j^{'}}p_{j}^{m}r_{j,j^{'}}^{m}\esp\left[\delta_{j,j^{'}}^{m}\left(i\right)\right].
\end{equation}
\end{itemize}

Consideremos el caso donde los tiempos entre arribo a cada una de
las colas, $\left\{\xi_{k}\left(n\right)\right\}_{n\geq1}$ son
variables aleatorias independientes a id\'enticamente
distribuidas, y los tiempos de servicio en cada una de las colas
se distribuyen de manera independiente e id\'enticamente
distribuidas $\left\{\eta_{k}\left(n\right)\right\}_{n\geq1}$;
adem\'as ambos procesos cumplen la condici\'on de ser
independientes entre s\'i. Para la $k$-\'esima cola se define la
tasa de arribo por
$\lambda_{k}=1/\esp\left[\xi_{k}\left(1\right)\right]$ y la tasa
de servicio como
$\mu_{k}=1/\esp\left[\eta_{k}\left(1\right)\right]$, finalmente se
define la carga de la cola como $\rho_{k}=\lambda_{k}/\mu_{k}$,
donde se pide que $\rho=\sum_{k=1}^{K}\rho_{k}<1$, para garantizar
la estabilidad del sistema, esto es cierto para las pol\'iticas de
servicio exhaustiva y cerrada, ver Geetor \cite{Getoor}.\\

Si denotamos por
\begin{itemize}
\item $Q_{k}\left(t\right)$ el n\'umero de usuarios presentes en
la cola $k$ al tiempo $t$; \item $A_{k}\left(t\right)$ los
residuales de los tiempos entre arribos a la cola $k$; para cada
servidor $m$; \item $B_{m}\left(t\right)$ denota a los residuales
de los tiempos de servicio al tiempo $t$; \item
$B_{m}^{0}\left(t\right)$ los residuales de los tiempos de
traslado de la cola $k$ a la pr\'oxima por atender al tiempo $t$,

\item sea
$C_{m}\left(t\right)$ el n\'umero de usuarios atendidos durante la
visita del servidor a la cola $k$ al tiempo $t$.
\end{itemize}


En este sentido, el proceso para el sistema de visitas se puede
definir como:

\begin{equation}\label{Esp.Edos.Down}
X\left(t\right)^{T}=\left(Q_{k}\left(t\right),A_{k}\left(t\right),B_{m}\left(t\right),B_{m}^{0}\left(t\right),C_{m}\left(t\right)\right),
\end{equation}
para $k=1,\ldots,K$ y $m=1,2,\ldots,M$, donde $T$ indica que es el
transpuesto del vector que se est\'a definiendo. El proceso $X$
evoluciona en el espacio de estados:
$\mathbb{X}=\ent_{+}^{K}\times\rea_{+}^{K}\times\left(\left\{1,2,\ldots,K\right\}\times\left\{1,2,\ldots,S\right\}\right)^{M}\times\rea_{+}^{K}\times\ent_{+}^{K}$.\\

El sistema aqu\'i descrito debe de cumplir con los siguientes supuestos b\'asicos de un sistema de visitas:
%__________________________________________________________________________
\subsubsection{Supuestos B\'asicos}
%__________________________________________________________________________
\begin{itemize}
\item[A1)] Los procesos
$\xi_{1},\ldots,\xi_{K},\eta_{1},\ldots,\eta_{K}$ son mutuamente
independientes y son sucesiones independientes e id\'enticamente
distribuidas.

\item[A2)] Para alg\'un entero $p\geq1$
\begin{eqnarray*}
\esp\left[\xi_{l}\left(1\right)^{p+1}\right]&<&\infty\textrm{ para }l=1,\ldots,K\textrm{ y }\\
\esp\left[\eta_{k}\left(1\right)^{p+1}\right]&<&\infty\textrm{
para }k=1,\ldots,K.
\end{eqnarray*}
donde $\mathcal{A}$ es la clase de posibles arribos.

\item[A3)] Para cada $k=1,2,\ldots,K$ existe una funci\'on
positiva $q_{k}\left(\cdot\right)$ definida en $\rea_{+}$, y un
entero $j_{k}$, tal que
\begin{eqnarray}
P\left(\xi_{k}\left(1\right)\geq x\right)&>&0\textrm{, para todo }x>0,\\
P\left\{a\leq\sum_{i=1}^{j_{k}}\xi_{k}\left(i\right)\leq
b\right\}&\geq&\int_{a}^{b}q_{k}\left(x\right)dx, \textrm{ }0\leq
a<b.
\end{eqnarray}
\end{itemize}

En lo que respecta al supuesto (A3), en Dai y Meyn \cite{DaiSean}
hacen ver que este se puede sustituir por

\begin{itemize}
\item[A3')] Para el Proceso de Markov $X$, cada subconjunto
compacto del espacio de estados de $X$ es un conjunto peque\~no,
ver definici\'on \ref{Def.Cto.Peq.}.
\end{itemize}

Es por esta raz\'on que con la finalidad de poder hacer uso de
$A3^{'})$ es necesario recurrir a los Procesos de Harris y en
particular a los Procesos Harris Recurrente, ver \cite{Dai,
DaiSean}.
%_______________________________________________________________________
\subsection{Procesos Harris Recurrente}
%_______________________________________________________________________

Por el supuesto (A1) conforme a Davis \cite{Davis}, se puede
definir el proceso de saltos correspondiente de manera tal que
satisfaga el supuesto (A3'), de hecho la demostraci\'on est\'a
basada en la l\'inea de argumentaci\'on de Davis, \cite{Davis},
p\'aginas 362-364.\\

Entonces se tiene un espacio de estados en el cual el proceso $X$
satisface la Propiedad Fuerte de Markov, ver Dai y Meyn
\cite{DaiSean}, dado por

\[\left(\Omega,\mathcal{F},\mathcal{F}_{t},X\left(t\right),\theta_{t},P_{x}\right),\]
adem\'as de ser un proceso de Borel Derecho (Sharpe \cite{Sharpe})
en el espacio de estados medible
$\left(\mathbb{X},\mathcal{B}_\mathbb{X}\right)$. El Proceso
$X=\left\{X\left(t\right),t\geq0\right\}$ tiene trayectorias
continuas por la derecha, est\'a definido en
$\left(\Omega,\mathcal{F}\right)$ y est\'a adaptado a
$\left\{\mathcal{F}_{t},t\geq0\right\}$; la colecci\'on
$\left\{P_{x},x\in \mathbb{X}\right\}$ son medidas de probabilidad
en $\left(\Omega,\mathcal{F}\right)$ tales que para todo $x\in
\mathbb{X}$
\[P_{x}\left\{X\left(0\right)=x\right\}=1,\] y
\[E_{x}\left\{f\left(X\circ\theta_{t}\right)|\mathcal{F}_{t}\right\}=E_{X}\left(\tau\right)f\left(X\right),\]
en $\left\{\tau<\infty\right\}$, $P_{x}$-c.s., con $\theta_{t}$
definido como el operador shift.


Donde $\tau$ es un $\mathcal{F}_{t}$-tiempo de paro
\[\left(X\circ\theta_{\tau}\right)\left(w\right)=\left\{X\left(\tau\left(w\right)+t,w\right),t\geq0\right\},\]
y $f$ es una funci\'on de valores reales acotada y medible, ver \cite{Dai, KaspiMandelbaum}.\\

Sea $P^{t}\left(x,D\right)$, $D\in\mathcal{B}_{\mathbb{X}}$,
$t\geq0$ la probabilidad de transici\'on de $X$ queda definida
como:
\[P^{t}\left(x,D\right)=P_{x}\left(X\left(t\right)\in
D\right).\]


\begin{Def}
Una medida no cero $\pi$ en
$\left(\mathbb{X},\mathcal{B}_{\mathbb{X}}\right)$ es invariante
para $X$ si $\pi$ es $\sigma$-finita y
\[\pi\left(D\right)=\int_{\mathbb{X}}P^{t}\left(x,D\right)\pi\left(dx\right),\]
para todo $D\in \mathcal{B}_{\mathbb{X}}$, con $t\geq0$.
\end{Def}

\begin{Def}
El proceso de Markov $X$ es llamado Harris Recurrente si existe
una medida de probabilidad $\nu$ en
$\left(\mathbb{X},\mathcal{B}_{\mathbb{X}}\right)$, tal que si
$\nu\left(D\right)>0$ y $D\in\mathcal{B}_{\mathbb{X}}$
\[P_{x}\left\{\tau_{D}<\infty\right\}\equiv1,\] cuando
$\tau_{D}=inf\left\{t\geq0:X_{t}\in D\right\}$.
\end{Def}

\begin{Note}
\begin{itemize}
\item[i)] Si $X$ es Harris recurrente, entonces existe una \'unica
medida invariante $\pi$ (Getoor \cite{Getoor}).

\item[ii)] Si la medida invariante es finita, entonces puede
normalizarse a una medida de probabilidad, en este caso al proceso
$X$ se le llama Harris recurrente positivo.


\item[iii)] Cuando $X$ es Harris recurrente positivo se dice que
la disciplina de servicio es estable. En este caso $\pi$ denota la
distribuci\'on estacionaria y hacemos
\[P_{\pi}\left(\cdot\right)=\int_{\mathbf{X}}P_{x}\left(\cdot\right)\pi\left(dx\right),\]
y se utiliza $E_{\pi}$ para denotar el operador esperanza
correspondiente, ver \cite{DaiSean}.
\end{itemize}
\end{Note}

\begin{Def}\label{Def.Cto.Peq.}
Un conjunto $D\in\mathcal{B_{\mathbb{X}}}$ es llamado peque\~no si
existe un $t>0$, una medida de probabilidad $\nu$ en
$\mathcal{B_{\mathbb{X}}}$, y un $\delta>0$ tal que
\[P^{t}\left(x,A\right)\geq\delta\nu\left(A\right),\] para $x\in
D,A\in\mathcal{B_{\mathbb{X}}}$.
\end{Def}

La siguiente serie de resultados vienen enunciados y demostrados
en Dai \cite{Dai}:
\begin{Lema}[Lema 3.1, Dai \cite{Dai}]
Sea $B$ conjunto peque\~no cerrado, supongamos que
$P_{x}\left(\tau_{B}<\infty\right)\equiv1$ y que para alg\'un
$\delta>0$ se cumple que
\begin{equation}\label{Eq.3.1}
\sup\esp_{x}\left[\tau_{B}\left(\delta\right)\right]<\infty,
\end{equation}
donde
$\tau_{B}\left(\delta\right)=inf\left\{t\geq\delta:X\left(t\right)\in
B\right\}$. Entonces, $X$ es un proceso Harris recurrente
positivo.
\end{Lema}

\begin{Lema}[Lema 3.1, Dai \cite{Dai}]\label{Lema.3.}
Bajo el supuesto (A3), el conjunto
$B=\left\{x\in\mathbb{X}:|x|\leq k\right\}$ es un conjunto
peque\~no cerrado para cualquier $k>0$.
\end{Lema}

\begin{Teo}[Teorema 3.1, Dai \cite{Dai}]\label{Tma.3.1}
Si existe un $\delta>0$ tal que
\begin{equation}
lim_{|x|\rightarrow\infty}\frac{1}{|x|}\esp|X^{x}\left(|x|\delta\right)|=0,
\end{equation}
donde $X^{x}$ se utiliza para denotar que el proceso $X$ comienza
a partir de $x$, entonces la ecuaci\'on (\ref{Eq.3.1}) se cumple
para $B=\left\{x\in\mathbb{X}:|x|\leq k\right\}$ con alg\'un
$k>0$. En particular, $X$ es Harris recurrente positivo.
\end{Teo}

Entonces, tenemos que el proceso $X$ es un proceso de Markov que
cumple con los supuestos $A1)$-$A3)$, lo que falta de hacer es
construir el Modelo de Flujo bas\'andonos en lo hasta ahora
presentado.
%_______________________________________________________________________
\subsection{Modelo de Flujo}
%_______________________________________________________________________

Dada una condici\'on inicial $x\in\mathbb{X}$, sea

\begin{itemize}
\item $Q_{k}^{x}\left(t\right)$ la longitud de la cola al tiempo
$t$,

\item $T_{m,k}^{x}\left(t\right)$ el tiempo acumulado, al tiempo
$t$, que tarda el servidor $m$ en atender a los usuarios de la
cola $k$.

\item $T_{m,k}^{x,0}\left(t\right)$ el tiempo acumulado, al tiempo
$t$, que tarda el servidor $m$ en trasladarse a otra cola a partir de la $k$-\'esima.\\
\end{itemize}

Sup\'ongase que la funci\'on
$\left(\overline{Q}\left(\cdot\right),\overline{T}_{m}
\left(\cdot\right),\overline{T}_{m}^{0} \left(\cdot\right)\right)$
para $m=1,2,\ldots,M$ es un punto l\'imite de
\begin{equation}\label{Eq.Punto.Limite}
\left(\frac{1}{|x|}Q^{x}\left(|x|t\right),\frac{1}{|x|}T_{m}^{x}\left(|x|t\right),\frac{1}{|x|}T_{m}^{x,0}\left(|x|t\right)\right)
\end{equation}
para $m=1,2,\ldots,M$, cuando $x\rightarrow\infty$, ver
\cite{Down}. Entonces
$\left(\overline{Q}\left(t\right),\overline{T}_{m}
\left(t\right),\overline{T}_{m}^{0} \left(t\right)\right)$ es un
flujo l\'imite del sistema. Al conjunto de todos las posibles
flujos l\'imite se le llama {\emph{Modelo de Flujo}} y se le
denotar\'a por $\mathcal{Q}$, ver \cite{Down, Dai, DaiSean}.\\

El modelo de flujo satisface el siguiente conjunto de ecuaciones:

\begin{equation}\label{Eq.MF.1}
\overline{Q}_{k}\left(t\right)=\overline{Q}_{k}\left(0\right)+\lambda_{k}t-\sum_{m=1}^{M}\mu_{k}\overline{T}_{m,k}\left(t\right),\\
\end{equation}
para $k=1,2,\ldots,K$.\\
\begin{equation}\label{Eq.MF.2}
\overline{Q}_{k}\left(t\right)\geq0\textrm{ para
}k=1,2,\ldots,K.\\
\end{equation}

\begin{equation}\label{Eq.MF.3}
\overline{T}_{m,k}\left(0\right)=0,\textrm{ y }\overline{T}_{m,k}\left(\cdot\right)\textrm{ es no decreciente},\\
\end{equation}
para $k=1,2,\ldots,K$ y $m=1,2,\ldots,M$.\\
\begin{equation}\label{Eq.MF.4}
\sum_{k=1}^{K}\overline{T}_{m,k}^{0}\left(t\right)+\overline{T}_{m,k}\left(t\right)=t\textrm{
para }m=1,2,\ldots,M.\\
\end{equation}


\begin{Def}[Definici\'on 4.1, Dai \cite{Dai}]\label{Def.Modelo.Flujo}
Sea una disciplina de servicio espec\'ifica. Cualquier l\'imite
$\left(\overline{Q}\left(\cdot\right),\overline{T}\left(\cdot\right),\overline{T}^{0}\left(\cdot\right)\right)$
en (\ref{Eq.Punto.Limite}) es un {\em flujo l\'imite} de la
disciplina. Cualquier soluci\'on (\ref{Eq.MF.1})-(\ref{Eq.MF.4})
es llamado flujo soluci\'on de la disciplina.
\end{Def}

\begin{Def}
Se dice que el modelo de flujo l\'imite, modelo de flujo, de la
disciplina de la cola es estable si existe una constante
$\delta>0$ que depende de $\mu,\lambda$ y $P$ solamente, tal que
cualquier flujo l\'imite con
$|\overline{Q}\left(0\right)|+|\overline{U}|+|\overline{V}|=1$, se
tiene que $\overline{Q}\left(\cdot+\delta\right)\equiv0$.
\end{Def}

Si se hace $|x|\rightarrow\infty$ sin restringir ninguna de las
componentes, tambi\'en se obtienen un modelo de flujo, pero en
este caso el residual de los procesos de arribo y servicio
introducen un retraso:
\begin{Teo}[Teorema 4.2, Dai \cite{Dai}]\label{Tma.4.2.Dai}
Sea una disciplina fija para la cola, suponga que se cumplen las
condiciones (A1)-(A3). Si el modelo de flujo l\'imite de la
disciplina de la cola es estable, entonces la cadena de Markov $X$
que describe la din\'amica de la red bajo la disciplina es Harris
recurrente positiva.
\end{Teo}

Ahora se procede a escalar el espacio y el tiempo para reducir la
aparente fluctuaci\'on del modelo. Consid\'erese el proceso
\begin{equation}\label{Eq.3.7}
\overline{Q}^{x}\left(t\right)=\frac{1}{|x|}Q^{x}\left(|x|t\right).
\end{equation}
A este proceso se le conoce como el flujo escalado, y cualquier
l\'imite $\overline{Q}^{x}\left(t\right)$ es llamado flujo
l\'imite del proceso de longitud de la cola. Haciendo
$|q|\rightarrow\infty$ mientras se mantiene el resto de las
componentes fijas, cualquier punto l\'imite del proceso de
longitud de la cola normalizado $\overline{Q}^{x}$ es soluci\'on
del siguiente modelo de flujo.


\begin{Def}[Definici\'on 3.3, Dai y Meyn \cite{DaiSean}]
El modelo de flujo es estable si existe un tiempo fijo $t_{0}$ tal
que $\overline{Q}\left(t\right)=0$, con $t\geq t_{0}$, para
cualquier $\overline{Q}\left(\cdot\right)\in\mathcal{Q}$ que
cumple con $|\overline{Q}\left(0\right)|=1$.
\end{Def}

\begin{Lemma}[Lema 3.1, Dai y Meyn \cite{DaiSean}]
Si el modelo de flujo definido por (\ref{Eq.MF.1})-(\ref{Eq.MF.4})
es estable, entonces el modelo de flujo retrasado es tambi\'en
estable, es decir, existe $t_{0}>0$ tal que
$\overline{Q}\left(t\right)=0$ para cualquier $t\geq t_{0}$, para
cualquier soluci\'on del modelo de flujo retrasado cuya
condici\'on inicial $\overline{x}$ satisface que
$|\overline{x}|=|\overline{Q}\left(0\right)|+|\overline{A}\left(0\right)|+|\overline{B}\left(0\right)|\leq1$.
\end{Lemma}


Ahora ya estamos en condiciones de enunciar los resultados principales:


\begin{Teo}[Teorema 2.1, Down \cite{Down}]\label{Tma2.1.Down}
Suponga que el modelo de flujo es estable, y que se cumplen los supuestos (A1) y (A2), entonces
\begin{itemize}
\item[i)] Para alguna constante $\kappa_{p}$, y para cada
condici\'on inicial $x\in X$
\begin{equation}\label{Estability.Eq1}
\limsup_{t\rightarrow\infty}\frac{1}{t}\int_{0}^{t}\esp_{x}\left[|Q\left(s\right)|^{p}\right]ds\leq\kappa_{p},
\end{equation}
donde $p$ es el entero dado en (A2).
\end{itemize}
Si adem\'as se cumple la condici\'on (A3), entonces para cada
condici\'on inicial:
\begin{itemize}
\item[ii)] Los momentos transitorios convergen a su estado
estacionario:
 \begin{equation}\label{Estability.Eq2}
lim_{t\rightarrow\infty}\esp_{x}\left[Q_{k}\left(t\right)^{r}\right]=\esp_{\pi}\left[Q_{k}\left(0\right)^{r}\right]\leq\kappa_{r},
\end{equation}
para $r=1,2,\ldots,p$ y $k=1,2,\ldots,K$. Donde $\pi$ es la
probabilidad invariante para $X$.

\item[iii)]  El primer momento converge con raz\'on $t^{p-1}$:
\begin{equation}\label{Estability.Eq3}
lim_{t\rightarrow\infty}t^{p-1}|\esp_{x}\left[Q_{k}\left(t\right)\right]-\esp_{\pi}\left[Q_{k}\left(0\right)\right]|=0.
\end{equation}

\item[iv)] La {\em Ley Fuerte de los grandes n\'umeros} se cumple:
\begin{equation}\label{Estability.Eq4}
lim_{t\rightarrow\infty}\frac{1}{t}\int_{0}^{t}Q_{k}^{r}\left(s\right)ds=\esp_{\pi}\left[Q_{k}\left(0\right)^{r}\right],\textrm{
}\prob_{x}\textrm{-c.s.}
\end{equation}
para $r=1,2,\ldots,p$ y $k=1,2,\ldots,K$.
\end{itemize}
\end{Teo}

La contribuci\'on de Down a la teor\'ia de los {\emph {sistemas de
visitas c\'iclicas}}, es la relaci\'on que hay entre la
estabilidad del sistema con el comportamiento de las medidas de
desempe\~no, es decir, la condici\'on suficiente para poder
garantizar la convergencia del proceso de la longitud de la cola
as\'i como de por los menos los dos primeros momentos adem\'as de
una versi\'on de la Ley Fuerte de los Grandes N\'umeros para los
sistemas de visitas.


\begin{Teo}[Teorema 2.3, Down \cite{Down}]\label{Tma2.3.Down}
Considere el siguiente valor:
\begin{equation}\label{Eq.Rho.1serv}
\rho=\sum_{k=1}^{K}\rho_{k}+max_{1\leq j\leq K}\left(\frac{\lambda_{j}}{\sum_{s=1}^{S}p_{js}\overline{N}_{s}}\right)\delta^{*}
\end{equation}
\begin{itemize}
\item[i)] Si $\rho<1$ entonces la red es estable, es decir, se
cumple el Teorema \ref{Tma2.1.Down}.

\item[ii)] Si $\rho>1$ entonces la red es inestable, es decir, se
cumple el Teorema \ref{Tma2.2.Down}
\end{itemize}
\end{Teo}

%_________________________________________________________________________
%\section{DESARROLLO DEL TEMA Y/O METODOLOG\'IA}
%_________________________________________________________________________
\subsection{Supuestos}
%_________________________________________________________________________
Consideremos el caso en el que se tienen varias colas a las cuales
llegan uno o varios servidores para dar servicio a los usuarios
que se encuentran presentes en la cola, como ya se mencion\'o hay
varios tipos de pol\'iticas de servicio, incluso podr\'ia ocurrir
que la manera en que atiende al resto de las colas sea distinta a
como lo hizo en las anteriores.\\

Para ejemplificar los sistemas de visitas c\'iclicas se
considerar\'a el caso en que a las colas los usuarios son atendidos con
una s\'ola pol\'itica de servicio.\\


Si $\omega$ es el n\'umero de usuarios en la cola al comienzo del
periodo de servicio y $N\left(\omega\right)$ es el n\'umero de
usuarios que son atendidos con una pol\'itica en espec\'ifico
durante el periodo de servicio, entonces se asume que:
\begin{itemize}
\item[1)]\label{S1}$lim_{\omega\rightarrow\infty}\esp\left[N\left(\omega\right)\right]=\overline{N}>0$;
\item[2)]\label{S2}$\esp\left[N\left(\omega\right)\right]\leq\overline{N}$
para cualquier valor de $\omega$.
\end{itemize}
La manera en que atiende el servidor $m$-\'esimo, es la siguiente:
\begin{itemize}
\item Al t\'ermino de la visita a la cola $j$, el servidor cambia
a la cola $j^{'}$ con probabilidad $r_{j,j^{'}}^{m}$

\item La $n$-\'esima vez que el servidor cambia de la cola $j$ a
$j'$, va acompa\~nada con el tiempo de cambio de longitud
$\delta_{j,j^{'}}^{m}\left(n\right)$, con
$\delta_{j,j^{'}}^{m}\left(n\right)$, $n\geq1$, variables
aleatorias independientes e id\'enticamente distribuidas, tales
que $\esp\left[\delta_{j,j^{'}}^{m}\left(1\right)\right]\geq0$.

\item Sea $\left\{p_{j}^{m}\right\}$ la distribuci\'on invariante
estacionaria \'unica para la Cadena de Markov con matriz de
transici\'on $\left(r_{j,j^{'}}^{m}\right)$, se supone que \'esta
existe.

\item Finalmente, se define el tiempo promedio total de traslado
entre las colas como
\begin{equation}
\delta^{*}:=\sum_{j,j^{'}}p_{j}^{m}r_{j,j^{'}}^{m}\esp\left[\delta_{j,j^{'}}^{m}\left(i\right)\right].
\end{equation}
\end{itemize}

Consideremos el caso donde los tiempos entre arribo a cada una de
las colas, $\left\{\xi_{k}\left(n\right)\right\}_{n\geq1}$ son
variables aleatorias independientes a id\'enticamente
distribuidas, y los tiempos de servicio en cada una de las colas
se distribuyen de manera independiente e id\'enticamente
distribuidas $\left\{\eta_{k}\left(n\right)\right\}_{n\geq1}$;
adem\'as ambos procesos cumplen la condici\'on de ser
independientes entre s\'i. Para la $k$-\'esima cola se define la
tasa de arribo por
$\lambda_{k}=1/\esp\left[\xi_{k}\left(1\right)\right]$ y la tasa
de servicio como
$\mu_{k}=1/\esp\left[\eta_{k}\left(1\right)\right]$, finalmente se
define la carga de la cola como $\rho_{k}=\lambda_{k}/\mu_{k}$,
donde se pide que $\rho=\sum_{k=1}^{K}\rho_{k}<1$, para garantizar
la estabilidad del sistema, esto es cierto para las pol\'iticas de
servicio exhaustiva y cerrada, ver Geetor \cite{Getoor}.\\

Si denotamos por
\begin{itemize}
\item $Q_{k}\left(t\right)$ el n\'umero de usuarios presentes en
la cola $k$ al tiempo $t$; \item $A_{k}\left(t\right)$ los
residuales de los tiempos entre arribos a la cola $k$; para cada
servidor $m$; \item $B_{m}\left(t\right)$ denota a los residuales
de los tiempos de servicio al tiempo $t$; \item
$B_{m}^{0}\left(t\right)$ los residuales de los tiempos de
traslado de la cola $k$ a la pr\'oxima por atender al tiempo $t$,

\item sea
$C_{m}\left(t\right)$ el n\'umero de usuarios atendidos durante la
visita del servidor a la cola $k$ al tiempo $t$.
\end{itemize}


En este sentido, el proceso para el sistema de visitas se puede
definir como:

\begin{equation}\label{Esp.Edos.Down}
X\left(t\right)^{T}=\left(Q_{k}\left(t\right),A_{k}\left(t\right),B_{m}\left(t\right),B_{m}^{0}\left(t\right),C_{m}\left(t\right)\right),
\end{equation}
para $k=1,\ldots,K$ y $m=1,2,\ldots,M$, donde $T$ indica que es el
transpuesto del vector que se est\'a definiendo. El proceso $X$
evoluciona en el espacio de estados:
$\mathbb{X}=\ent_{+}^{K}\times\rea_{+}^{K}\times\left(\left\{1,2,\ldots,K\right\}\times\left\{1,2,\ldots,S\right\}\right)^{M}\times\rea_{+}^{K}\times\ent_{+}^{K}$.\\

El sistema aqu\'i descrito debe de cumplir con los siguientes supuestos b\'asicos de un sistema de visitas:
%__________________________________________________________________________
\subsubsection{Supuestos B\'asicos}
%__________________________________________________________________________
\begin{itemize}
\item[A1)] Los procesos
$\xi_{1},\ldots,\xi_{K},\eta_{1},\ldots,\eta_{K}$ son mutuamente
independientes y son sucesiones independientes e id\'enticamente
distribuidas.

\item[A2)] Para alg\'un entero $p\geq1$
\begin{eqnarray*}
\esp\left[\xi_{l}\left(1\right)^{p+1}\right]&<&\infty\textrm{ para }l=1,\ldots,K\textrm{ y }\\
\esp\left[\eta_{k}\left(1\right)^{p+1}\right]&<&\infty\textrm{
para }k=1,\ldots,K.
\end{eqnarray*}
donde $\mathcal{A}$ es la clase de posibles arribos.

\item[A3)] Para cada $k=1,2,\ldots,K$ existe una funci\'on
positiva $q_{k}\left(\cdot\right)$ definida en $\rea_{+}$, y un
entero $j_{k}$, tal que
\begin{eqnarray}
P\left(\xi_{k}\left(1\right)\geq x\right)&>&0\textrm{, para todo }x>0,\\
P\left\{a\leq\sum_{i=1}^{j_{k}}\xi_{k}\left(i\right)\leq
b\right\}&\geq&\int_{a}^{b}q_{k}\left(x\right)dx, \textrm{ }0\leq
a<b.
\end{eqnarray}
\end{itemize}

En lo que respecta al supuesto (A3), en Dai y Meyn \cite{DaiSean}
hacen ver que este se puede sustituir por

\begin{itemize}
\item[A3')] Para el Proceso de Markov $X$, cada subconjunto
compacto del espacio de estados de $X$ es un conjunto peque\~no,
ver definici\'on \ref{Def.Cto.Peq.}.
\end{itemize}

Es por esta raz\'on que con la finalidad de poder hacer uso de
$A3^{'})$ es necesario recurrir a los Procesos de Harris y en
particular a los Procesos Harris Recurrente, ver \cite{Dai,
DaiSean}.
%_______________________________________________________________________
\subsection{Procesos Harris Recurrente}
%_______________________________________________________________________

Por el supuesto (A1) conforme a Davis \cite{Davis}, se puede
definir el proceso de saltos correspondiente de manera tal que
satisfaga el supuesto (A3'), de hecho la demostraci\'on est\'a
basada en la l\'inea de argumentaci\'on de Davis, \cite{Davis},
p\'aginas 362-364.\\

Entonces se tiene un espacio de estados en el cual el proceso $X$
satisface la Propiedad Fuerte de Markov, ver Dai y Meyn
\cite{DaiSean}, dado por

\[\left(\Omega,\mathcal{F},\mathcal{F}_{t},X\left(t\right),\theta_{t},P_{x}\right),\]
adem\'as de ser un proceso de Borel Derecho (Sharpe \cite{Sharpe})
en el espacio de estados medible
$\left(\mathbb{X},\mathcal{B}_\mathbb{X}\right)$. El Proceso
$X=\left\{X\left(t\right),t\geq0\right\}$ tiene trayectorias
continuas por la derecha, est\'a definido en
$\left(\Omega,\mathcal{F}\right)$ y est\'a adaptado a
$\left\{\mathcal{F}_{t},t\geq0\right\}$; la colecci\'on
$\left\{P_{x},x\in \mathbb{X}\right\}$ son medidas de probabilidad
en $\left(\Omega,\mathcal{F}\right)$ tales que para todo $x\in
\mathbb{X}$
\[P_{x}\left\{X\left(0\right)=x\right\}=1,\] y
\[E_{x}\left\{f\left(X\circ\theta_{t}\right)|\mathcal{F}_{t}\right\}=E_{X}\left(\tau\right)f\left(X\right),\]
en $\left\{\tau<\infty\right\}$, $P_{x}$-c.s., con $\theta_{t}$
definido como el operador shift.


Donde $\tau$ es un $\mathcal{F}_{t}$-tiempo de paro
\[\left(X\circ\theta_{\tau}\right)\left(w\right)=\left\{X\left(\tau\left(w\right)+t,w\right),t\geq0\right\},\]
y $f$ es una funci\'on de valores reales acotada y medible, ver \cite{Dai, KaspiMandelbaum}.\\

Sea $P^{t}\left(x,D\right)$, $D\in\mathcal{B}_{\mathbb{X}}$,
$t\geq0$ la probabilidad de transici\'on de $X$ queda definida
como:
\[P^{t}\left(x,D\right)=P_{x}\left(X\left(t\right)\in
D\right).\]


\begin{Def}
Una medida no cero $\pi$ en
$\left(\mathbb{X},\mathcal{B}_{\mathbb{X}}\right)$ es invariante
para $X$ si $\pi$ es $\sigma$-finita y
\[\pi\left(D\right)=\int_{\mathbb{X}}P^{t}\left(x,D\right)\pi\left(dx\right),\]
para todo $D\in \mathcal{B}_{\mathbb{X}}$, con $t\geq0$.
\end{Def}

\begin{Def}
El proceso de Markov $X$ es llamado Harris Recurrente si existe
una medida de probabilidad $\nu$ en
$\left(\mathbb{X},\mathcal{B}_{\mathbb{X}}\right)$, tal que si
$\nu\left(D\right)>0$ y $D\in\mathcal{B}_{\mathbb{X}}$
\[P_{x}\left\{\tau_{D}<\infty\right\}\equiv1,\] cuando
$\tau_{D}=inf\left\{t\geq0:X_{t}\in D\right\}$.
\end{Def}

\begin{Note}
\begin{itemize}
\item[i)] Si $X$ es Harris recurrente, entonces existe una \'unica
medida invariante $\pi$ (Getoor \cite{Getoor}).

\item[ii)] Si la medida invariante es finita, entonces puede
normalizarse a una medida de probabilidad, en este caso al proceso
$X$ se le llama Harris recurrente positivo.


\item[iii)] Cuando $X$ es Harris recurrente positivo se dice que
la disciplina de servicio es estable. En este caso $\pi$ denota la
distribuci\'on estacionaria y hacemos
\[P_{\pi}\left(\cdot\right)=\int_{\mathbf{X}}P_{x}\left(\cdot\right)\pi\left(dx\right),\]
y se utiliza $E_{\pi}$ para denotar el operador esperanza
correspondiente, ver \cite{DaiSean}.
\end{itemize}
\end{Note}

\begin{Def}\label{Def.Cto.Peq.}
Un conjunto $D\in\mathcal{B_{\mathbb{X}}}$ es llamado peque\~no si
existe un $t>0$, una medida de probabilidad $\nu$ en
$\mathcal{B_{\mathbb{X}}}$, y un $\delta>0$ tal que
\[P^{t}\left(x,A\right)\geq\delta\nu\left(A\right),\] para $x\in
D,A\in\mathcal{B_{\mathbb{X}}}$.
\end{Def}

La siguiente serie de resultados vienen enunciados y demostrados
en Dai \cite{Dai}:
\begin{Lema}[Lema 3.1, Dai \cite{Dai}]
Sea $B$ conjunto peque\~no cerrado, supongamos que
$P_{x}\left(\tau_{B}<\infty\right)\equiv1$ y que para alg\'un
$\delta>0$ se cumple que
\begin{equation}\label{Eq.3.1}
\sup\esp_{x}\left[\tau_{B}\left(\delta\right)\right]<\infty,
\end{equation}
donde
$\tau_{B}\left(\delta\right)=inf\left\{t\geq\delta:X\left(t\right)\in
B\right\}$. Entonces, $X$ es un proceso Harris recurrente
positivo.
\end{Lema}

\begin{Lema}[Lema 3.1, Dai \cite{Dai}]\label{Lema.3.}
Bajo el supuesto (A3), el conjunto
$B=\left\{x\in\mathbb{X}:|x|\leq k\right\}$ es un conjunto
peque\~no cerrado para cualquier $k>0$.
\end{Lema}

\begin{Teo}[Teorema 3.1, Dai \cite{Dai}]\label{Tma.3.1}
Si existe un $\delta>0$ tal que
\begin{equation}
lim_{|x|\rightarrow\infty}\frac{1}{|x|}\esp|X^{x}\left(|x|\delta\right)|=0,
\end{equation}
donde $X^{x}$ se utiliza para denotar que el proceso $X$ comienza
a partir de $x$, entonces la ecuaci\'on (\ref{Eq.3.1}) se cumple
para $B=\left\{x\in\mathbb{X}:|x|\leq k\right\}$ con alg\'un
$k>0$. En particular, $X$ es Harris recurrente positivo.
\end{Teo}

Entonces, tenemos que el proceso $X$ es un proceso de Markov que
cumple con los supuestos $A1)$-$A3)$, lo que falta de hacer es
construir el Modelo de Flujo bas\'andonos en lo hasta ahora
presentado.
%_______________________________________________________________________
\subsection{Modelo de Flujo}
%_______________________________________________________________________

Dada una condici\'on inicial $x\in\mathbb{X}$, sea

\begin{itemize}
\item $Q_{k}^{x}\left(t\right)$ la longitud de la cola al tiempo
$t$,

\item $T_{m,k}^{x}\left(t\right)$ el tiempo acumulado, al tiempo
$t$, que tarda el servidor $m$ en atender a los usuarios de la
cola $k$.

\item $T_{m,k}^{x,0}\left(t\right)$ el tiempo acumulado, al tiempo
$t$, que tarda el servidor $m$ en trasladarse a otra cola a partir de la $k$-\'esima.\\
\end{itemize}

Sup\'ongase que la funci\'on
$\left(\overline{Q}\left(\cdot\right),\overline{T}_{m}
\left(\cdot\right),\overline{T}_{m}^{0} \left(\cdot\right)\right)$
para $m=1,2,\ldots,M$ es un punto l\'imite de
\begin{equation}\label{Eq.Punto.Limite}
\left(\frac{1}{|x|}Q^{x}\left(|x|t\right),\frac{1}{|x|}T_{m}^{x}\left(|x|t\right),\frac{1}{|x|}T_{m}^{x,0}\left(|x|t\right)\right)
\end{equation}
para $m=1,2,\ldots,M$, cuando $x\rightarrow\infty$, ver
\cite{Down}. Entonces
$\left(\overline{Q}\left(t\right),\overline{T}_{m}
\left(t\right),\overline{T}_{m}^{0} \left(t\right)\right)$ es un
flujo l\'imite del sistema. Al conjunto de todos las posibles
flujos l\'imite se le llama {\emph{Modelo de Flujo}} y se le
denotar\'a por $\mathcal{Q}$, ver \cite{Down, Dai, DaiSean}.\\

El modelo de flujo satisface el siguiente conjunto de ecuaciones:

\begin{equation}\label{Eq.MF.1}
\overline{Q}_{k}\left(t\right)=\overline{Q}_{k}\left(0\right)+\lambda_{k}t-\sum_{m=1}^{M}\mu_{k}\overline{T}_{m,k}\left(t\right),\\
\end{equation}
para $k=1,2,\ldots,K$.\\
\begin{equation}\label{Eq.MF.2}
\overline{Q}_{k}\left(t\right)\geq0\textrm{ para
}k=1,2,\ldots,K.\\
\end{equation}

\begin{equation}\label{Eq.MF.3}
\overline{T}_{m,k}\left(0\right)=0,\textrm{ y }\overline{T}_{m,k}\left(\cdot\right)\textrm{ es no decreciente},\\
\end{equation}
para $k=1,2,\ldots,K$ y $m=1,2,\ldots,M$.\\
\begin{equation}\label{Eq.MF.4}
\sum_{k=1}^{K}\overline{T}_{m,k}^{0}\left(t\right)+\overline{T}_{m,k}\left(t\right)=t\textrm{
para }m=1,2,\ldots,M.\\
\end{equation}


\begin{Def}[Definici\'on 4.1, Dai \cite{Dai}]\label{Def.Modelo.Flujo}
Sea una disciplina de servicio espec\'ifica. Cualquier l\'imite
$\left(\overline{Q}\left(\cdot\right),\overline{T}\left(\cdot\right),\overline{T}^{0}\left(\cdot\right)\right)$
en (\ref{Eq.Punto.Limite}) es un {\em flujo l\'imite} de la
disciplina. Cualquier soluci\'on (\ref{Eq.MF.1})-(\ref{Eq.MF.4})
es llamado flujo soluci\'on de la disciplina.
\end{Def}

\begin{Def}
Se dice que el modelo de flujo l\'imite, modelo de flujo, de la
disciplina de la cola es estable si existe una constante
$\delta>0$ que depende de $\mu,\lambda$ y $P$ solamente, tal que
cualquier flujo l\'imite con
$|\overline{Q}\left(0\right)|+|\overline{U}|+|\overline{V}|=1$, se
tiene que $\overline{Q}\left(\cdot+\delta\right)\equiv0$.
\end{Def}

Si se hace $|x|\rightarrow\infty$ sin restringir ninguna de las
componentes, tambi\'en se obtienen un modelo de flujo, pero en
este caso el residual de los procesos de arribo y servicio
introducen un retraso:
\begin{Teo}[Teorema 4.2, Dai \cite{Dai}]\label{Tma.4.2.Dai}
Sea una disciplina fija para la cola, suponga que se cumplen las
condiciones (A1)-(A3). Si el modelo de flujo l\'imite de la
disciplina de la cola es estable, entonces la cadena de Markov $X$
que describe la din\'amica de la red bajo la disciplina es Harris
recurrente positiva.
\end{Teo}

Ahora se procede a escalar el espacio y el tiempo para reducir la
aparente fluctuaci\'on del modelo. Consid\'erese el proceso
\begin{equation}\label{Eq.3.7}
\overline{Q}^{x}\left(t\right)=\frac{1}{|x|}Q^{x}\left(|x|t\right).
\end{equation}
A este proceso se le conoce como el flujo escalado, y cualquier
l\'imite $\overline{Q}^{x}\left(t\right)$ es llamado flujo
l\'imite del proceso de longitud de la cola. Haciendo
$|q|\rightarrow\infty$ mientras se mantiene el resto de las
componentes fijas, cualquier punto l\'imite del proceso de
longitud de la cola normalizado $\overline{Q}^{x}$ es soluci\'on
del siguiente modelo de flujo.


\begin{Def}[Definici\'on 3.3, Dai y Meyn \cite{DaiSean}]
El modelo de flujo es estable si existe un tiempo fijo $t_{0}$ tal
que $\overline{Q}\left(t\right)=0$, con $t\geq t_{0}$, para
cualquier $\overline{Q}\left(\cdot\right)\in\mathcal{Q}$ que
cumple con $|\overline{Q}\left(0\right)|=1$.
\end{Def}

\begin{Lemma}[Lema 3.1, Dai y Meyn \cite{DaiSean}]
Si el modelo de flujo definido por (\ref{Eq.MF.1})-(\ref{Eq.MF.4})
es estable, entonces el modelo de flujo retrasado es tambi\'en
estable, es decir, existe $t_{0}>0$ tal que
$\overline{Q}\left(t\right)=0$ para cualquier $t\geq t_{0}$, para
cualquier soluci\'on del modelo de flujo retrasado cuya
condici\'on inicial $\overline{x}$ satisface que
$|\overline{x}|=|\overline{Q}\left(0\right)|+|\overline{A}\left(0\right)|+|\overline{B}\left(0\right)|\leq1$.
\end{Lemma}


Ahora ya estamos en condiciones de enunciar los resultados principales:


\begin{Teo}[Teorema 2.1, Down \cite{Down}]\label{Tma2.1.Down}
Suponga que el modelo de flujo es estable, y que se cumplen los supuestos (A1) y (A2), entonces
\begin{itemize}
\item[i)] Para alguna constante $\kappa_{p}$, y para cada
condici\'on inicial $x\in X$
\begin{equation}\label{Estability.Eq1}
\limsup_{t\rightarrow\infty}\frac{1}{t}\int_{0}^{t}\esp_{x}\left[|Q\left(s\right)|^{p}\right]ds\leq\kappa_{p},
\end{equation}
donde $p$ es el entero dado en (A2).
\end{itemize}
Si adem\'as se cumple la condici\'on (A3), entonces para cada
condici\'on inicial:
\begin{itemize}
\item[ii)] Los momentos transitorios convergen a su estado
estacionario:
 \begin{equation}\label{Estability.Eq2}
lim_{t\rightarrow\infty}\esp_{x}\left[Q_{k}\left(t\right)^{r}\right]=\esp_{\pi}\left[Q_{k}\left(0\right)^{r}\right]\leq\kappa_{r},
\end{equation}
para $r=1,2,\ldots,p$ y $k=1,2,\ldots,K$. Donde $\pi$ es la
probabilidad invariante para $X$.

\item[iii)]  El primer momento converge con raz\'on $t^{p-1}$:
\begin{equation}\label{Estability.Eq3}
lim_{t\rightarrow\infty}t^{p-1}|\esp_{x}\left[Q_{k}\left(t\right)\right]-\esp_{\pi}\left[Q_{k}\left(0\right)\right]|=0.
\end{equation}

\item[iv)] La {\em Ley Fuerte de los grandes n\'umeros} se cumple:
\begin{equation}\label{Estability.Eq4}
lim_{t\rightarrow\infty}\frac{1}{t}\int_{0}^{t}Q_{k}^{r}\left(s\right)ds=\esp_{\pi}\left[Q_{k}\left(0\right)^{r}\right],\textrm{
}\prob_{x}\textrm{-c.s.}
\end{equation}
para $r=1,2,\ldots,p$ y $k=1,2,\ldots,K$.
\end{itemize}
\end{Teo}

La contribuci\'on de Down a la teor\'ia de los {\emph {sistemas de
visitas c\'iclicas}}, es la relaci\'on que hay entre la
estabilidad del sistema con el comportamiento de las medidas de
desempe\~no, es decir, la condici\'on suficiente para poder
garantizar la convergencia del proceso de la longitud de la cola
as\'i como de por los menos los dos primeros momentos adem\'as de
una versi\'on de la Ley Fuerte de los Grandes N\'umeros para los
sistemas de visitas.


\begin{Teo}[Teorema 2.3, Down \cite{Down}]\label{Tma2.3.Down}
Considere el siguiente valor:
\begin{equation}\label{Eq.Rho.1serv}
\rho=\sum_{k=1}^{K}\rho_{k}+max_{1\leq j\leq K}\left(\frac{\lambda_{j}}{\sum_{s=1}^{S}p_{js}\overline{N}_{s}}\right)\delta^{*}
\end{equation}
\begin{itemize}
\item[i)] Si $\rho<1$ entonces la red es estable, es decir, se
cumple el Teorema \ref{Tma2.1.Down}.

\item[ii)] Si $\rho>1$ entonces la red es inestable, es decir, se
cumple el Teorema \ref{Tma2.2.Down}
\end{itemize}
\end{Teo}



El sistema aqu\'i descrito debe de cumplir con los siguientes supuestos b\'asicos de un sistema de visitas:
%__________________________________________________________________________
\subsubsection{Supuestos B\'asicos}
%__________________________________________________________________________
\begin{itemize}
\item[A1)] Los procesos
$\xi_{1},\ldots,\xi_{K},\eta_{1},\ldots,\eta_{K}$ son mutuamente
independientes y son sucesiones independientes e id\'enticamente
distribuidas.

\item[A2)] Para alg\'un entero $p\geq1$
\begin{eqnarray*}
\esp\left[\xi_{l}\left(1\right)^{p+1}\right]&<&\infty\textrm{ para }l=1,\ldots,K\textrm{ y }\\
\esp\left[\eta_{k}\left(1\right)^{p+1}\right]&<&\infty\textrm{
para }k=1,\ldots,K.
\end{eqnarray*}
donde $\mathcal{A}$ es la clase de posibles arribos.

\item[A3)] Para cada $k=1,2,\ldots,K$ existe una funci\'on
positiva $q_{k}\left(\cdot\right)$ definida en $\rea_{+}$, y un
entero $j_{k}$, tal que
\begin{eqnarray}
P\left(\xi_{k}\left(1\right)\geq x\right)&>&0\textrm{, para todo }x>0,\\
P\left\{a\leq\sum_{i=1}^{j_{k}}\xi_{k}\left(i\right)\leq
b\right\}&\geq&\int_{a}^{b}q_{k}\left(x\right)dx, \textrm{ }0\leq
a<b.
\end{eqnarray}
\end{itemize}

En lo que respecta al supuesto (A3), en Dai y Meyn \cite{DaiSean}
hacen ver que este se puede sustituir por

\begin{itemize}
\item[A3')] Para el Proceso de Markov $X$, cada subconjunto
compacto del espacio de estados de $X$ es un conjunto peque\~no,
ver definici\'on \ref{Def.Cto.Peq.}.
\end{itemize}

Es por esta raz\'on que con la finalidad de poder hacer uso de
$A3^{'})$ es necesario recurrir a los Procesos de Harris y en
particular a los Procesos Harris Recurrente, ver \cite{Dai,
DaiSean}.
%_______________________________________________________________________
\subsection{Procesos Harris Recurrente}
%_______________________________________________________________________

Por el supuesto (A1) conforme a Davis \cite{Davis}, se puede
definir el proceso de saltos correspondiente de manera tal que
satisfaga el supuesto (A3'), de hecho la demostraci\'on est\'a
basada en la l\'inea de argumentaci\'on de Davis, \cite{Davis},
p\'aginas 362-364.\\

Entonces se tiene un espacio de estados en el cual el proceso $X$
satisface la Propiedad Fuerte de Markov, ver Dai y Meyn
\cite{DaiSean}, dado por

\[\left(\Omega,\mathcal{F},\mathcal{F}_{t},X\left(t\right),\theta_{t},P_{x}\right),\]
adem\'as de ser un proceso de Borel Derecho (Sharpe \cite{Sharpe})
en el espacio de estados medible
$\left(\mathbb{X},\mathcal{B}_\mathbb{X}\right)$. El Proceso
$X=\left\{X\left(t\right),t\geq0\right\}$ tiene trayectorias
continuas por la derecha, est\'a definido en
$\left(\Omega,\mathcal{F}\right)$ y est\'a adaptado a
$\left\{\mathcal{F}_{t},t\geq0\right\}$; la colecci\'on
$\left\{P_{x},x\in \mathbb{X}\right\}$ son medidas de probabilidad
en $\left(\Omega,\mathcal{F}\right)$ tales que para todo $x\in
\mathbb{X}$
\[P_{x}\left\{X\left(0\right)=x\right\}=1,\] y
\[E_{x}\left\{f\left(X\circ\theta_{t}\right)|\mathcal{F}_{t}\right\}=E_{X}\left(\tau\right)f\left(X\right),\]
en $\left\{\tau<\infty\right\}$, $P_{x}$-c.s., con $\theta_{t}$
definido como el operador shift.


Donde $\tau$ es un $\mathcal{F}_{t}$-tiempo de paro
\[\left(X\circ\theta_{\tau}\right)\left(w\right)=\left\{X\left(\tau\left(w\right)+t,w\right),t\geq0\right\},\]
y $f$ es una funci\'on de valores reales acotada y medible, ver \cite{Dai, KaspiMandelbaum}.\\

Sea $P^{t}\left(x,D\right)$, $D\in\mathcal{B}_{\mathbb{X}}$,
$t\geq0$ la probabilidad de transici\'on de $X$ queda definida
como:
\[P^{t}\left(x,D\right)=P_{x}\left(X\left(t\right)\in
D\right).\]


\begin{Def}
Una medida no cero $\pi$ en
$\left(\mathbb{X},\mathcal{B}_{\mathbb{X}}\right)$ es invariante
para $X$ si $\pi$ es $\sigma$-finita y
\[\pi\left(D\right)=\int_{\mathbb{X}}P^{t}\left(x,D\right)\pi\left(dx\right),\]
para todo $D\in \mathcal{B}_{\mathbb{X}}$, con $t\geq0$.
\end{Def}

\begin{Def}
El proceso de Markov $X$ es llamado Harris Recurrente si existe
una medida de probabilidad $\nu$ en
$\left(\mathbb{X},\mathcal{B}_{\mathbb{X}}\right)$, tal que si
$\nu\left(D\right)>0$ y $D\in\mathcal{B}_{\mathbb{X}}$
\[P_{x}\left\{\tau_{D}<\infty\right\}\equiv1,\] cuando
$\tau_{D}=inf\left\{t\geq0:X_{t}\in D\right\}$.
\end{Def}

\begin{Note}
\begin{itemize}
\item[i)] Si $X$ es Harris recurrente, entonces existe una \'unica
medida invariante $\pi$ (Getoor \cite{Getoor}).

\item[ii)] Si la medida invariante es finita, entonces puede
normalizarse a una medida de probabilidad, en este caso al proceso
$X$ se le llama Harris recurrente positivo.


\item[iii)] Cuando $X$ es Harris recurrente positivo se dice que
la disciplina de servicio es estable. En este caso $\pi$ denota la
distribuci\'on estacionaria y hacemos
\[P_{\pi}\left(\cdot\right)=\int_{\mathbf{X}}P_{x}\left(\cdot\right)\pi\left(dx\right),\]
y se utiliza $E_{\pi}$ para denotar el operador esperanza
correspondiente, ver \cite{DaiSean}.
\end{itemize}
\end{Note}

\begin{Def}\label{Def.Cto.Peq.}
Un conjunto $D\in\mathcal{B_{\mathbb{X}}}$ es llamado peque\~no si
existe un $t>0$, una medida de probabilidad $\nu$ en
$\mathcal{B_{\mathbb{X}}}$, y un $\delta>0$ tal que
\[P^{t}\left(x,A\right)\geq\delta\nu\left(A\right),\] para $x\in
D,A\in\mathcal{B_{\mathbb{X}}}$.
\end{Def}

La siguiente serie de resultados vienen enunciados y demostrados
en Dai \cite{Dai}:
\begin{Lema}[Lema 3.1, Dai \cite{Dai}]
Sea $B$ conjunto peque\~no cerrado, supongamos que
$P_{x}\left(\tau_{B}<\infty\right)\equiv1$ y que para alg\'un
$\delta>0$ se cumple que
\begin{equation}\label{Eq.3.1}
\sup\esp_{x}\left[\tau_{B}\left(\delta\right)\right]<\infty,
\end{equation}
donde
$\tau_{B}\left(\delta\right)=inf\left\{t\geq\delta:X\left(t\right)\in
B\right\}$. Entonces, $X$ es un proceso Harris recurrente
positivo.
\end{Lema}

\begin{Lema}[Lema 3.1, Dai \cite{Dai}]\label{Lema.3.}
Bajo el supuesto (A3), el conjunto
$B=\left\{x\in\mathbb{X}:|x|\leq k\right\}$ es un conjunto
peque\~no cerrado para cualquier $k>0$.
\end{Lema}

\begin{Teo}[Teorema 3.1, Dai \cite{Dai}]\label{Tma.3.1}
Si existe un $\delta>0$ tal que
\begin{equation}
lim_{|x|\rightarrow\infty}\frac{1}{|x|}\esp|X^{x}\left(|x|\delta\right)|=0,
\end{equation}
donde $X^{x}$ se utiliza para denotar que el proceso $X$ comienza
a partir de $x$, entonces la ecuaci\'on (\ref{Eq.3.1}) se cumple
para $B=\left\{x\in\mathbb{X}:|x|\leq k\right\}$ con alg\'un
$k>0$. En particular, $X$ es Harris recurrente positivo.
\end{Teo}

Entonces, tenemos que el proceso $X$ es un proceso de Markov que
cumple con los supuestos $A1)$-$A3)$, lo que falta de hacer es
construir el Modelo de Flujo bas\'andonos en lo hasta ahora
presentado.
%_______________________________________________________________________
\subsection{Modelo de Flujo}
%_______________________________________________________________________

Dada una condici\'on inicial $x\in\mathbb{X}$, sea

\begin{itemize}
\item $Q_{k}^{x}\left(t\right)$ la longitud de la cola al tiempo
$t$,

\item $T_{m,k}^{x}\left(t\right)$ el tiempo acumulado, al tiempo
$t$, que tarda el servidor $m$ en atender a los usuarios de la
cola $k$.

\item $T_{m,k}^{x,0}\left(t\right)$ el tiempo acumulado, al tiempo
$t$, que tarda el servidor $m$ en trasladarse a otra cola a partir de la $k$-\'esima.\\
\end{itemize}

Sup\'ongase que la funci\'on
$\left(\overline{Q}\left(\cdot\right),\overline{T}_{m}
\left(\cdot\right),\overline{T}_{m}^{0} \left(\cdot\right)\right)$
para $m=1,2,\ldots,M$ es un punto l\'imite de
\begin{equation}\label{Eq.Punto.Limite}
\left(\frac{1}{|x|}Q^{x}\left(|x|t\right),\frac{1}{|x|}T_{m}^{x}\left(|x|t\right),\frac{1}{|x|}T_{m}^{x,0}\left(|x|t\right)\right)
\end{equation}
para $m=1,2,\ldots,M$, cuando $x\rightarrow\infty$, ver
\cite{Down}. Entonces
$\left(\overline{Q}\left(t\right),\overline{T}_{m}
\left(t\right),\overline{T}_{m}^{0} \left(t\right)\right)$ es un
flujo l\'imite del sistema. Al conjunto de todos las posibles
flujos l\'imite se le llama {\emph{Modelo de Flujo}} y se le
denotar\'a por $\mathcal{Q}$, ver \cite{Down, Dai, DaiSean}.\\

El modelo de flujo satisface el siguiente conjunto de ecuaciones:

\begin{equation}\label{Eq.MF.1}
\overline{Q}_{k}\left(t\right)=\overline{Q}_{k}\left(0\right)+\lambda_{k}t-\sum_{m=1}^{M}\mu_{k}\overline{T}_{m,k}\left(t\right),\\
\end{equation}
para $k=1,2,\ldots,K$.\\
\begin{equation}\label{Eq.MF.2}
\overline{Q}_{k}\left(t\right)\geq0\textrm{ para
}k=1,2,\ldots,K.\\
\end{equation}

\begin{equation}\label{Eq.MF.3}
\overline{T}_{m,k}\left(0\right)=0,\textrm{ y }\overline{T}_{m,k}\left(\cdot\right)\textrm{ es no decreciente},\\
\end{equation}
para $k=1,2,\ldots,K$ y $m=1,2,\ldots,M$.\\
\begin{equation}\label{Eq.MF.4}
\sum_{k=1}^{K}\overline{T}_{m,k}^{0}\left(t\right)+\overline{T}_{m,k}\left(t\right)=t\textrm{
para }m=1,2,\ldots,M.\\
\end{equation}


\begin{Def}[Definici\'on 4.1, Dai \cite{Dai}]\label{Def.Modelo.Flujo}
Sea una disciplina de servicio espec\'ifica. Cualquier l\'imite
$\left(\overline{Q}\left(\cdot\right),\overline{T}\left(\cdot\right),\overline{T}^{0}\left(\cdot\right)\right)$
en (\ref{Eq.Punto.Limite}) es un {\em flujo l\'imite} de la
disciplina. Cualquier soluci\'on (\ref{Eq.MF.1})-(\ref{Eq.MF.4})
es llamado flujo soluci\'on de la disciplina.
\end{Def}

\begin{Def}
Se dice que el modelo de flujo l\'imite, modelo de flujo, de la
disciplina de la cola es estable si existe una constante
$\delta>0$ que depende de $\mu,\lambda$ y $P$ solamente, tal que
cualquier flujo l\'imite con
$|\overline{Q}\left(0\right)|+|\overline{U}|+|\overline{V}|=1$, se
tiene que $\overline{Q}\left(\cdot+\delta\right)\equiv0$.
\end{Def}

Si se hace $|x|\rightarrow\infty$ sin restringir ninguna de las
componentes, tambi\'en se obtienen un modelo de flujo, pero en
este caso el residual de los procesos de arribo y servicio
introducen un retraso:
\begin{Teo}[Teorema 4.2, Dai \cite{Dai}]\label{Tma.4.2.Dai}
Sea una disciplina fija para la cola, suponga que se cumplen las
condiciones (A1)-(A3). Si el modelo de flujo l\'imite de la
disciplina de la cola es estable, entonces la cadena de Markov $X$
que describe la din\'amica de la red bajo la disciplina es Harris
recurrente positiva.
\end{Teo}

Ahora se procede a escalar el espacio y el tiempo para reducir la
aparente fluctuaci\'on del modelo. Consid\'erese el proceso
\begin{equation}\label{Eq.3.7}
\overline{Q}^{x}\left(t\right)=\frac{1}{|x|}Q^{x}\left(|x|t\right).
\end{equation}
A este proceso se le conoce como el flujo escalado, y cualquier
l\'imite $\overline{Q}^{x}\left(t\right)$ es llamado flujo
l\'imite del proceso de longitud de la cola. Haciendo
$|q|\rightarrow\infty$ mientras se mantiene el resto de las
componentes fijas, cualquier punto l\'imite del proceso de
longitud de la cola normalizado $\overline{Q}^{x}$ es soluci\'on
del siguiente modelo de flujo.


\begin{Def}[Definici\'on 3.3, Dai y Meyn \cite{DaiSean}]
El modelo de flujo es estable si existe un tiempo fijo $t_{0}$ tal
que $\overline{Q}\left(t\right)=0$, con $t\geq t_{0}$, para
cualquier $\overline{Q}\left(\cdot\right)\in\mathcal{Q}$ que
cumple con $|\overline{Q}\left(0\right)|=1$.
\end{Def}

\begin{Lemma}[Lema 3.1, Dai y Meyn \cite{DaiSean}]
Si el modelo de flujo definido por (\ref{Eq.MF.1})-(\ref{Eq.MF.4})
es estable, entonces el modelo de flujo retrasado es tambi\'en
estable, es decir, existe $t_{0}>0$ tal que
$\overline{Q}\left(t\right)=0$ para cualquier $t\geq t_{0}$, para
cualquier soluci\'on del modelo de flujo retrasado cuya
condici\'on inicial $\overline{x}$ satisface que
$|\overline{x}|=|\overline{Q}\left(0\right)|+|\overline{A}\left(0\right)|+|\overline{B}\left(0\right)|\leq1$.
\end{Lemma}


Ahora ya estamos en condiciones de enunciar los resultados principales:


\begin{Teo}[Teorema 2.1, Down \cite{Down}]\label{Tma2.1.Down}
Suponga que el modelo de flujo es estable, y que se cumplen los supuestos (A1) y (A2), entonces
\begin{itemize}
\item[i)] Para alguna constante $\kappa_{p}$, y para cada
condici\'on inicial $x\in X$
\begin{equation}\label{Estability.Eq1}
\limsup_{t\rightarrow\infty}\frac{1}{t}\int_{0}^{t}\esp_{x}\left[|Q\left(s\right)|^{p}\right]ds\leq\kappa_{p},
\end{equation}
donde $p$ es el entero dado en (A2).
\end{itemize}
Si adem\'as se cumple la condici\'on (A3), entonces para cada
condici\'on inicial:
\begin{itemize}
\item[ii)] Los momentos transitorios convergen a su estado
estacionario:
 \begin{equation}\label{Estability.Eq2}
lim_{t\rightarrow\infty}\esp_{x}\left[Q_{k}\left(t\right)^{r}\right]=\esp_{\pi}\left[Q_{k}\left(0\right)^{r}\right]\leq\kappa_{r},
\end{equation}
para $r=1,2,\ldots,p$ y $k=1,2,\ldots,K$. Donde $\pi$ es la
probabilidad invariante para $X$.

\item[iii)]  El primer momento converge con raz\'on $t^{p-1}$:
\begin{equation}\label{Estability.Eq3}
lim_{t\rightarrow\infty}t^{p-1}|\esp_{x}\left[Q_{k}\left(t\right)\right]-\esp_{\pi}\left[Q_{k}\left(0\right)\right]|=0.
\end{equation}

\item[iv)] La {\em Ley Fuerte de los grandes n\'umeros} se cumple:
\begin{equation}\label{Estability.Eq4}
lim_{t\rightarrow\infty}\frac{1}{t}\int_{0}^{t}Q_{k}^{r}\left(s\right)ds=\esp_{\pi}\left[Q_{k}\left(0\right)^{r}\right],\textrm{
}\prob_{x}\textrm{-c.s.}
\end{equation}
para $r=1,2,\ldots,p$ y $k=1,2,\ldots,K$.
\end{itemize}
\end{Teo}

La contribuci\'on de Down a la teor\'ia de los {\emph {sistemas de
visitas c\'iclicas}}, es la relaci\'on que hay entre la
estabilidad del sistema con el comportamiento de las medidas de
desempe\~no, es decir, la condici\'on suficiente para poder
garantizar la convergencia del proceso de la longitud de la cola
as\'i como de por los menos los dos primeros momentos adem\'as de
una versi\'on de la Ley Fuerte de los Grandes N\'umeros para los
sistemas de visitas.


\begin{Teo}[Teorema 2.3, Down \cite{Down}]\label{Tma2.3.Down}
Considere el siguiente valor:
\begin{equation}\label{Eq.Rho.1serv}
\rho=\sum_{k=1}^{K}\rho_{k}+max_{1\leq j\leq K}\left(\frac{\lambda_{j}}{\sum_{s=1}^{S}p_{js}\overline{N}_{s}}\right)\delta^{*}
\end{equation}
\begin{itemize}
\item[i)] Si $\rho<1$ entonces la red es estable, es decir, se
cumple el Teorema \ref{Tma2.1.Down}.

\item[ii)] Si $\rho>1$ entonces la red es inestable, es decir, se
cumple el Teorema \ref{Tma2.2.Down}
\end{itemize}
\end{Teo}


%_________________________________________________________________________
\subsection{Supuestos}
%_________________________________________________________________________
Consideremos el caso en el que se tienen varias colas a las cuales
llegan uno o varios servidores para dar servicio a los usuarios
que se encuentran presentes en la cola, como ya se mencion\'o hay
varios tipos de pol\'iticas de servicio, incluso podr\'ia ocurrir
que la manera en que atiende al resto de las colas sea distinta a
como lo hizo en las anteriores.\\

Para ejemplificar los sistemas de visitas c\'iclicas se
considerar\'a el caso en que a las colas los usuarios son atendidos con
una s\'ola pol\'itica de servicio.\\



Si $\omega$ es el n\'umero de usuarios en la cola al comienzo del
periodo de servicio y $N\left(\omega\right)$ es el n\'umero de
usuarios que son atendidos con una pol\'itica en espec\'ifico
durante el periodo de servicio, entonces se asume que:
\begin{itemize}
\item[1)]\label{S1}$lim_{\omega\rightarrow\infty}\esp\left[N\left(\omega\right)\right]=\overline{N}>0$;
\item[2)]\label{S2}$\esp\left[N\left(\omega\right)\right]\leq\overline{N}$
para cualquier valor de $\omega$.
\end{itemize}
La manera en que atiende el servidor $m$-\'esimo, es la siguiente:
\begin{itemize}
\item Al t\'ermino de la visita a la cola $j$, el servidor cambia
a la cola $j^{'}$ con probabilidad $r_{j,j^{'}}^{m}$

\item La $n$-\'esima vez que el servidor cambia de la cola $j$ a
$j'$, va acompa\~nada con el tiempo de cambio de longitud
$\delta_{j,j^{'}}^{m}\left(n\right)$, con
$\delta_{j,j^{'}}^{m}\left(n\right)$, $n\geq1$, variables
aleatorias independientes e id\'enticamente distribuidas, tales
que $\esp\left[\delta_{j,j^{'}}^{m}\left(1\right)\right]\geq0$.

\item Sea $\left\{p_{j}^{m}\right\}$ la distribuci\'on invariante
estacionaria \'unica para la Cadena de Markov con matriz de
transici\'on $\left(r_{j,j^{'}}^{m}\right)$, se supone que \'esta
existe.

\item Finalmente, se define el tiempo promedio total de traslado
entre las colas como
\begin{equation}
\delta^{*}:=\sum_{j,j^{'}}p_{j}^{m}r_{j,j^{'}}^{m}\esp\left[\delta_{j,j^{'}}^{m}\left(i\right)\right].
\end{equation}
\end{itemize}

Consideremos el caso donde los tiempos entre arribo a cada una de
las colas, $\left\{\xi_{k}\left(n\right)\right\}_{n\geq1}$ son
variables aleatorias independientes a id\'enticamente
distribuidas, y los tiempos de servicio en cada una de las colas
se distribuyen de manera independiente e id\'enticamente
distribuidas $\left\{\eta_{k}\left(n\right)\right\}_{n\geq1}$;
adem\'as ambos procesos cumplen la condici\'on de ser
independientes entre s\'i. Para la $k$-\'esima cola se define la
tasa de arribo por
$\lambda_{k}=1/\esp\left[\xi_{k}\left(1\right)\right]$ y la tasa
de servicio como
$\mu_{k}=1/\esp\left[\eta_{k}\left(1\right)\right]$, finalmente se
define la carga de la cola como $\rho_{k}=\lambda_{k}/\mu_{k}$,
donde se pide que $\rho=\sum_{k=1}^{K}\rho_{k}<1$, para garantizar
la estabilidad del sistema, esto es cierto para las pol\'iticas de
servicio exhaustiva y cerrada, ver Geetor \cite{Getoor}.\\

Si denotamos por
\begin{itemize}
\item $Q_{k}\left(t\right)$ el n\'umero de usuarios presentes en
la cola $k$ al tiempo $t$; \item $A_{k}\left(t\right)$ los
residuales de los tiempos entre arribos a la cola $k$; para cada
servidor $m$; \item $B_{m}\left(t\right)$ denota a los residuales
de los tiempos de servicio al tiempo $t$; \item
$B_{m}^{0}\left(t\right)$ los residuales de los tiempos de
traslado de la cola $k$ a la pr\'oxima por atender al tiempo $t$,

\item sea
$C_{m}\left(t\right)$ el n\'umero de usuarios atendidos durante la
visita del servidor a la cola $k$ al tiempo $t$.
\end{itemize}


En este sentido, el proceso para el sistema de visitas se puede
definir como:

\begin{equation}\label{Esp.Edos.Down}
X\left(t\right)^{T}=\left(Q_{k}\left(t\right),A_{k}\left(t\right),B_{m}\left(t\right),B_{m}^{0}\left(t\right),C_{m}\left(t\right)\right),
\end{equation}
para $k=1,\ldots,K$ y $m=1,2,\ldots,M$, donde $T$ indica que es el
transpuesto del vector que se est\'a definiendo. El proceso $X$
evoluciona en el espacio de estados:
$\mathbb{X}=\ent_{+}^{K}\times\rea_{+}^{K}\times\left(\left\{1,2,\ldots,K\right\}\times\left\{1,2,\ldots,S\right\}\right)^{M}\times\rea_{+}^{K}\times\ent_{+}^{K}$.\\

El sistema aqu\'i descrito debe de cumplir con los siguientes supuestos b\'asicos de un sistema de visitas:
%__________________________________________________________________________
\subsubsection{Supuestos B\'asicos}
%__________________________________________________________________________
\begin{itemize}
\item[A1)] Los procesos
$\xi_{1},\ldots,\xi_{K},\eta_{1},\ldots,\eta_{K}$ son mutuamente
independientes y son sucesiones independientes e id\'enticamente
distribuidas.

\item[A2)] Para alg\'un entero $p\geq1$
\begin{eqnarray*}
\esp\left[\xi_{l}\left(1\right)^{p+1}\right]&<&\infty\textrm{ para }l=1,\ldots,K\textrm{ y }\\
\esp\left[\eta_{k}\left(1\right)^{p+1}\right]&<&\infty\textrm{
para }k=1,\ldots,K.
\end{eqnarray*}
donde $\mathcal{A}$ es la clase de posibles arribos.

\item[A3)] Para cada $k=1,2,\ldots,K$ existe una funci\'on
positiva $q_{k}\left(\cdot\right)$ definida en $\rea_{+}$, y un
entero $j_{k}$, tal que
\begin{eqnarray}
P\left(\xi_{k}\left(1\right)\geq x\right)&>&0\textrm{, para todo }x>0,\\
P\left\{a\leq\sum_{i=1}^{j_{k}}\xi_{k}\left(i\right)\leq
b\right\}&\geq&\int_{a}^{b}q_{k}\left(x\right)dx, \textrm{ }0\leq
a<b.
\end{eqnarray}
\end{itemize}

En lo que respecta al supuesto (A3), en Dai y Meyn \cite{DaiSean}
hacen ver que este se puede sustituir por

\begin{itemize}
\item[A3')] Para el Proceso de Markov $X$, cada subconjunto
compacto del espacio de estados de $X$ es un conjunto peque\~no,
ver definici\'on \ref{Def.Cto.Peq.}.
\end{itemize}

Es por esta raz\'on que con la finalidad de poder hacer uso de
$A3^{'})$ es necesario recurrir a los Procesos de Harris y en
particular a los Procesos Harris Recurrente, ver \cite{Dai,
DaiSean}.
%_______________________________________________________________________
\subsection{Procesos Harris Recurrente}
%_______________________________________________________________________

Por el supuesto (A1) conforme a Davis \cite{Davis}, se puede
definir el proceso de saltos correspondiente de manera tal que
satisfaga el supuesto (A3'), de hecho la demostraci\'on est\'a
basada en la l\'inea de argumentaci\'on de Davis, \cite{Davis},
p\'aginas 362-364.\\

Entonces se tiene un espacio de estados en el cual el proceso $X$
satisface la Propiedad Fuerte de Markov, ver Dai y Meyn
\cite{DaiSean}, dado por

\[\left(\Omega,\mathcal{F},\mathcal{F}_{t},X\left(t\right),\theta_{t},P_{x}\right),\]
adem\'as de ser un proceso de Borel Derecho (Sharpe \cite{Sharpe})
en el espacio de estados medible
$\left(\mathbb{X},\mathcal{B}_\mathbb{X}\right)$. El Proceso
$X=\left\{X\left(t\right),t\geq0\right\}$ tiene trayectorias
continuas por la derecha, est\'a definido en
$\left(\Omega,\mathcal{F}\right)$ y est\'a adaptado a
$\left\{\mathcal{F}_{t},t\geq0\right\}$; la colecci\'on
$\left\{P_{x},x\in \mathbb{X}\right\}$ son medidas de probabilidad
en $\left(\Omega,\mathcal{F}\right)$ tales que para todo $x\in
\mathbb{X}$
\[P_{x}\left\{X\left(0\right)=x\right\}=1,\] y
\[E_{x}\left\{f\left(X\circ\theta_{t}\right)|\mathcal{F}_{t}\right\}=E_{X}\left(\tau\right)f\left(X\right),\]
en $\left\{\tau<\infty\right\}$, $P_{x}$-c.s., con $\theta_{t}$
definido como el operador shift.


Donde $\tau$ es un $\mathcal{F}_{t}$-tiempo de paro
\[\left(X\circ\theta_{\tau}\right)\left(w\right)=\left\{X\left(\tau\left(w\right)+t,w\right),t\geq0\right\},\]
y $f$ es una funci\'on de valores reales acotada y medible, ver \cite{Dai, KaspiMandelbaum}.\\

Sea $P^{t}\left(x,D\right)$, $D\in\mathcal{B}_{\mathbb{X}}$,
$t\geq0$ la probabilidad de transici\'on de $X$ queda definida
como:
\[P^{t}\left(x,D\right)=P_{x}\left(X\left(t\right)\in
D\right).\]


\begin{Def}
Una medida no cero $\pi$ en
$\left(\mathbb{X},\mathcal{B}_{\mathbb{X}}\right)$ es invariante
para $X$ si $\pi$ es $\sigma$-finita y
\[\pi\left(D\right)=\int_{\mathbb{X}}P^{t}\left(x,D\right)\pi\left(dx\right),\]
para todo $D\in \mathcal{B}_{\mathbb{X}}$, con $t\geq0$.
\end{Def}

\begin{Def}
El proceso de Markov $X$ es llamado Harris Recurrente si existe
una medida de probabilidad $\nu$ en
$\left(\mathbb{X},\mathcal{B}_{\mathbb{X}}\right)$, tal que si
$\nu\left(D\right)>0$ y $D\in\mathcal{B}_{\mathbb{X}}$
\[P_{x}\left\{\tau_{D}<\infty\right\}\equiv1,\] cuando
$\tau_{D}=inf\left\{t\geq0:X_{t}\in D\right\}$.
\end{Def}

\begin{Note}
\begin{itemize}
\item[i)] Si $X$ es Harris recurrente, entonces existe una \'unica
medida invariante $\pi$ (Getoor \cite{Getoor}).

\item[ii)] Si la medida invariante es finita, entonces puede
normalizarse a una medida de probabilidad, en este caso al proceso
$X$ se le llama Harris recurrente positivo.


\item[iii)] Cuando $X$ es Harris recurrente positivo se dice que
la disciplina de servicio es estable. En este caso $\pi$ denota la
distribuci\'on estacionaria y hacemos
\[P_{\pi}\left(\cdot\right)=\int_{\mathbf{X}}P_{x}\left(\cdot\right)\pi\left(dx\right),\]
y se utiliza $E_{\pi}$ para denotar el operador esperanza
correspondiente, ver \cite{DaiSean}.
\end{itemize}
\end{Note}

\begin{Def}\label{Def.Cto.Peq.}
Un conjunto $D\in\mathcal{B_{\mathbb{X}}}$ es llamado peque\~no si
existe un $t>0$, una medida de probabilidad $\nu$ en
$\mathcal{B_{\mathbb{X}}}$, y un $\delta>0$ tal que
\[P^{t}\left(x,A\right)\geq\delta\nu\left(A\right),\] para $x\in
D,A\in\mathcal{B_{\mathbb{X}}}$.
\end{Def}

La siguiente serie de resultados vienen enunciados y demostrados
en Dai \cite{Dai}:
\begin{Lema}[Lema 3.1, Dai \cite{Dai}]
Sea $B$ conjunto peque\~no cerrado, supongamos que
$P_{x}\left(\tau_{B}<\infty\right)\equiv1$ y que para alg\'un
$\delta>0$ se cumple que
\begin{equation}\label{Eq.3.1}
\sup\esp_{x}\left[\tau_{B}\left(\delta\right)\right]<\infty,
\end{equation}
donde
$\tau_{B}\left(\delta\right)=inf\left\{t\geq\delta:X\left(t\right)\in
B\right\}$. Entonces, $X$ es un proceso Harris recurrente
positivo.
\end{Lema}

\begin{Lema}[Lema 3.1, Dai \cite{Dai}]\label{Lema.3.}
Bajo el supuesto (A3), el conjunto
$B=\left\{x\in\mathbb{X}:|x|\leq k\right\}$ es un conjunto
peque\~no cerrado para cualquier $k>0$.
\end{Lema}

\begin{Teo}[Teorema 3.1, Dai \cite{Dai}]\label{Tma.3.1}
Si existe un $\delta>0$ tal que
\begin{equation}
lim_{|x|\rightarrow\infty}\frac{1}{|x|}\esp|X^{x}\left(|x|\delta\right)|=0,
\end{equation}
donde $X^{x}$ se utiliza para denotar que el proceso $X$ comienza
a partir de $x$, entonces la ecuaci\'on (\ref{Eq.3.1}) se cumple
para $B=\left\{x\in\mathbb{X}:|x|\leq k\right\}$ con alg\'un
$k>0$. En particular, $X$ es Harris recurrente positivo.
\end{Teo}

Entonces, tenemos que el proceso $X$ es un proceso de Markov que
cumple con los supuestos $A1)$-$A3)$, lo que falta de hacer es
construir el Modelo de Flujo bas\'andonos en lo hasta ahora
presentado.
%_______________________________________________________________________
\subsection{Modelo de Flujo}
%_______________________________________________________________________

Dada una condici\'on inicial $x\in\mathbb{X}$, sea

\begin{itemize}
\item $Q_{k}^{x}\left(t\right)$ la longitud de la cola al tiempo
$t$,

\item $T_{m,k}^{x}\left(t\right)$ el tiempo acumulado, al tiempo
$t$, que tarda el servidor $m$ en atender a los usuarios de la
cola $k$.

\item $T_{m,k}^{x,0}\left(t\right)$ el tiempo acumulado, al tiempo
$t$, que tarda el servidor $m$ en trasladarse a otra cola a partir de la $k$-\'esima.\\
\end{itemize}

Sup\'ongase que la funci\'on
$\left(\overline{Q}\left(\cdot\right),\overline{T}_{m}
\left(\cdot\right),\overline{T}_{m}^{0} \left(\cdot\right)\right)$
para $m=1,2,\ldots,M$ es un punto l\'imite de
\begin{equation}\label{Eq.Punto.Limite}
\left(\frac{1}{|x|}Q^{x}\left(|x|t\right),\frac{1}{|x|}T_{m}^{x}\left(|x|t\right),\frac{1}{|x|}T_{m}^{x,0}\left(|x|t\right)\right)
\end{equation}
para $m=1,2,\ldots,M$, cuando $x\rightarrow\infty$, ver
\cite{Down}. Entonces
$\left(\overline{Q}\left(t\right),\overline{T}_{m}
\left(t\right),\overline{T}_{m}^{0} \left(t\right)\right)$ es un
flujo l\'imite del sistema. Al conjunto de todos las posibles
flujos l\'imite se le llama {\emph{Modelo de Flujo}} y se le
denotar\'a por $\mathcal{Q}$, ver \cite{Down, Dai, DaiSean}.\\

El modelo de flujo satisface el siguiente conjunto de ecuaciones:

\begin{equation}\label{Eq.MF.1}
\overline{Q}_{k}\left(t\right)=\overline{Q}_{k}\left(0\right)+\lambda_{k}t-\sum_{m=1}^{M}\mu_{k}\overline{T}_{m,k}\left(t\right),\\
\end{equation}
para $k=1,2,\ldots,K$.\\
\begin{equation}\label{Eq.MF.2}
\overline{Q}_{k}\left(t\right)\geq0\textrm{ para
}k=1,2,\ldots,K.\\
\end{equation}

\begin{equation}\label{Eq.MF.3}
\overline{T}_{m,k}\left(0\right)=0,\textrm{ y }\overline{T}_{m,k}\left(\cdot\right)\textrm{ es no decreciente},\\
\end{equation}
para $k=1,2,\ldots,K$ y $m=1,2,\ldots,M$.\\
\begin{equation}\label{Eq.MF.4}
\sum_{k=1}^{K}\overline{T}_{m,k}^{0}\left(t\right)+\overline{T}_{m,k}\left(t\right)=t\textrm{
para }m=1,2,\ldots,M.\\
\end{equation}


\begin{Def}[Definici\'on 4.1, Dai \cite{Dai}]\label{Def.Modelo.Flujo}
Sea una disciplina de servicio espec\'ifica. Cualquier l\'imite
$\left(\overline{Q}\left(\cdot\right),\overline{T}\left(\cdot\right),\overline{T}^{0}\left(\cdot\right)\right)$
en (\ref{Eq.Punto.Limite}) es un {\em flujo l\'imite} de la
disciplina. Cualquier soluci\'on (\ref{Eq.MF.1})-(\ref{Eq.MF.4})
es llamado flujo soluci\'on de la disciplina.
\end{Def}

\begin{Def}
Se dice que el modelo de flujo l\'imite, modelo de flujo, de la
disciplina de la cola es estable si existe una constante
$\delta>0$ que depende de $\mu,\lambda$ y $P$ solamente, tal que
cualquier flujo l\'imite con
$|\overline{Q}\left(0\right)|+|\overline{U}|+|\overline{V}|=1$, se
tiene que $\overline{Q}\left(\cdot+\delta\right)\equiv0$.
\end{Def}

Si se hace $|x|\rightarrow\infty$ sin restringir ninguna de las
componentes, tambi\'en se obtienen un modelo de flujo, pero en
este caso el residual de los procesos de arribo y servicio
introducen un retraso:
\begin{Teo}[Teorema 4.2, Dai \cite{Dai}]\label{Tma.4.2.Dai}
Sea una disciplina fija para la cola, suponga que se cumplen las
condiciones (A1)-(A3). Si el modelo de flujo l\'imite de la
disciplina de la cola es estable, entonces la cadena de Markov $X$
que describe la din\'amica de la red bajo la disciplina es Harris
recurrente positiva.
\end{Teo}

Ahora se procede a escalar el espacio y el tiempo para reducir la
aparente fluctuaci\'on del modelo. Consid\'erese el proceso
\begin{equation}\label{Eq.3.7}
\overline{Q}^{x}\left(t\right)=\frac{1}{|x|}Q^{x}\left(|x|t\right).
\end{equation}
A este proceso se le conoce como el flujo escalado, y cualquier
l\'imite $\overline{Q}^{x}\left(t\right)$ es llamado flujo
l\'imite del proceso de longitud de la cola. Haciendo
$|q|\rightarrow\infty$ mientras se mantiene el resto de las
componentes fijas, cualquier punto l\'imite del proceso de
longitud de la cola normalizado $\overline{Q}^{x}$ es soluci\'on
del siguiente modelo de flujo.


\begin{Def}[Definici\'on 3.3, Dai y Meyn \cite{DaiSean}]
El modelo de flujo es estable si existe un tiempo fijo $t_{0}$ tal
que $\overline{Q}\left(t\right)=0$, con $t\geq t_{0}$, para
cualquier $\overline{Q}\left(\cdot\right)\in\mathcal{Q}$ que
cumple con $|\overline{Q}\left(0\right)|=1$.
\end{Def}

\begin{Lemma}[Lema 3.1, Dai y Meyn \cite{DaiSean}]
Si el modelo de flujo definido por (\ref{Eq.MF.1})-(\ref{Eq.MF.4})
es estable, entonces el modelo de flujo retrasado es tambi\'en
estable, es decir, existe $t_{0}>0$ tal que
$\overline{Q}\left(t\right)=0$ para cualquier $t\geq t_{0}$, para
cualquier soluci\'on del modelo de flujo retrasado cuya
condici\'on inicial $\overline{x}$ satisface que
$|\overline{x}|=|\overline{Q}\left(0\right)|+|\overline{A}\left(0\right)|+|\overline{B}\left(0\right)|\leq1$.
\end{Lemma}


Ahora ya estamos en condiciones de enunciar los resultados principales:


\begin{Teo}[Teorema 2.1, Down \cite{Down}]\label{Tma2.1.Down}
Suponga que el modelo de flujo es estable, y que se cumplen los supuestos (A1) y (A2), entonces
\begin{itemize}
\item[i)] Para alguna constante $\kappa_{p}$, y para cada
condici\'on inicial $x\in X$
\begin{equation}\label{Estability.Eq1}
\limsup_{t\rightarrow\infty}\frac{1}{t}\int_{0}^{t}\esp_{x}\left[|Q\left(s\right)|^{p}\right]ds\leq\kappa_{p},
\end{equation}
donde $p$ es el entero dado en (A2).
\end{itemize}
Si adem\'as se cumple la condici\'on (A3), entonces para cada
condici\'on inicial:
\begin{itemize}
\item[ii)] Los momentos transitorios convergen a su estado
estacionario:
 \begin{equation}\label{Estability.Eq2}
lim_{t\rightarrow\infty}\esp_{x}\left[Q_{k}\left(t\right)^{r}\right]=\esp_{\pi}\left[Q_{k}\left(0\right)^{r}\right]\leq\kappa_{r},
\end{equation}
para $r=1,2,\ldots,p$ y $k=1,2,\ldots,K$. Donde $\pi$ es la
probabilidad invariante para $X$.

\item[iii)]  El primer momento converge con raz\'on $t^{p-1}$:
\begin{equation}\label{Estability.Eq3}
lim_{t\rightarrow\infty}t^{p-1}|\esp_{x}\left[Q_{k}\left(t\right)\right]-\esp_{\pi}\left[Q_{k}\left(0\right)\right]|=0.
\end{equation}

\item[iv)] La {\em Ley Fuerte de los grandes n\'umeros} se cumple:
\begin{equation}\label{Estability.Eq4}
lim_{t\rightarrow\infty}\frac{1}{t}\int_{0}^{t}Q_{k}^{r}\left(s\right)ds=\esp_{\pi}\left[Q_{k}\left(0\right)^{r}\right],\textrm{
}\prob_{x}\textrm{-c.s.}
\end{equation}
para $r=1,2,\ldots,p$ y $k=1,2,\ldots,K$.
\end{itemize}
\end{Teo}

La contribuci\'on de Down a la teor\'ia de los {\emph {sistemas de
visitas c\'iclicas}}, es la relaci\'on que hay entre la
estabilidad del sistema con el comportamiento de las medidas de
desempe\~no, es decir, la condici\'on suficiente para poder
garantizar la convergencia del proceso de la longitud de la cola
as\'i como de por los menos los dos primeros momentos adem\'as de
una versi\'on de la Ley Fuerte de los Grandes N\'umeros para los
sistemas de visitas.


\begin{Teo}[Teorema 2.3, Down \cite{Down}]\label{Tma2.3.Down}
Considere el siguiente valor:
\begin{equation}\label{Eq.Rho.1serv}
\rho=\sum_{k=1}^{K}\rho_{k}+max_{1\leq j\leq K}\left(\frac{\lambda_{j}}{\sum_{s=1}^{S}p_{js}\overline{N}_{s}}\right)\delta^{*}
\end{equation}
\begin{itemize}
\item[i)] Si $\rho<1$ entonces la red es estable, es decir, se
cumple el Teorema \ref{Tma2.1.Down}.

\item[ii)] Si $\rho>1$ entonces la red es inestable, es decir, se
cumple el Teorema \ref{Tma2.2.Down}
\end{itemize}
\end{Teo}




%-------------- CAPITULO YA INCLUIDO EN OTRO ----
%\chapter{Modelos de Flujo: Preliminario}
%
%___________________________________________________________________________________________
%\vspace{5.5cm}
\section{Preliminares: Modelos de Flujo}
%\vspace{-1.0cm}
%___________________________________________________________________________________________
%
\subsection{Procesos Regenerativos}
%_____________________________________________________

Si $x$ es el n{\'u}mero de usuarios en la cola al comienzo del
periodo de servicio y $N_{s}\left(x\right)=N\left(x\right)$ es el
n{\'u}mero de usuarios que son atendidos con la pol{\'\i}tica $s$,
{\'u}nica en nuestro caso, durante un periodo de servicio,
entonces se asume que:
\begin{itemize}
\item[(S1.)]
\begin{equation}\label{S1}
lim_{x\rightarrow\infty}\esp\left[N\left(x\right)\right]=\overline{N}>0.
\end{equation}
\item[(S2.)]
\begin{equation}\label{S2}
\esp\left[N\left(x\right)\right]\leq \overline{N}, \end{equation}
para cualquier valor de $x$. \item La $n$-{\'e}sima ocurrencia va
acompa{\~n}ada con el tiempo de cambio de longitud
$\delta_{j,j+1}\left(n\right)$, independientes e id{\'e}nticamente
distribuidas, con
$\esp\left[\delta_{j,j+1}\left(1\right)\right]\geq0$. \item Se
define
\begin{equation}
\delta^{*}:=\sum_{j,j+1}\esp\left[\delta_{j,j+1}\left(1\right)\right].
\end{equation}

\item Los tiempos de inter-arribo a la cola $k$,son de la forma
$\left\{\xi_{k}\left(n\right)\right\}_{n\geq1}$, con la propiedad
de que son independientes e id{\'e}nticamente distribuidos.

\item Los tiempos de servicio
$\left\{\eta_{k}\left(n\right)\right\}_{n\geq1}$ tienen la
propiedad de ser independientes e id{\'e}nticamente distribuidos.

\item Se define la tasa de arribo a la $k$-{\'e}sima cola como
$\lambda_{k}=1/\esp\left[\xi_{k}\left(1\right)\right]$ y
adem{\'a}s se define

\item la tasa de servicio para la $k$-{\'e}sima cola como
$\mu_{k}=1/\esp\left[\eta_{k}\left(1\right)\right]$

\item tambi{\'e}n se define $\rho_{k}=\lambda_{k}/\mu_{k}$, donde
es necesario que $\rho<1$ para cuestiones de estabilidad.

\item De las pol{\'\i}ticas posibles solamente consideraremos la
pol{\'\i}tica cerrada (Gated).
\end{itemize}

Las Colas C\'iclicas se pueden describir por medio de un proceso
de Markov $\left(X\left(t\right)\right)_{t\in\rea}$, donde el
estado del sistema al tiempo $t\geq0$ est\'a dado por
\begin{equation}
X\left(t\right)=\left(Q\left(t\right),A\left(t\right),H\left(t\right),B\left(t\right),B^{0}\left(t\right),C\left(t\right)\right)
\end{equation}
definido en el espacio producto:
\begin{equation}
\mathcal{X}=\mathbb{Z}^{K}\times\rea_{+}^{K}\times\left(\left\{1,2,\ldots,K\right\}\times\left\{1,2,\ldots,S\right\}\right)^{M}\times\rea_{+}^{K}\times\rea_{+}^{K}\times\mathbb{Z}^{K},
\end{equation}

\begin{itemize}
\item $Q\left(t\right)=\left(Q_{k}\left(t\right),1\leq k\leq
K\right)$, es el n\'umero de usuarios en la cola $k$, incluyendo
aquellos que est\'an siendo atendidos provenientes de la
$k$-\'esima cola.

\item $A\left(t\right)=\left(A_{k}\left(t\right),1\leq k\leq
K\right)$, son los residuales de los tiempos de arribo en la cola
$k$. \item $H\left(t\right)$ es el par ordenado que consiste en la
cola que esta siendo atendida y la pol\'itica de servicio que se
utilizar\'a.

\item $B\left(t\right)$ es el tiempo de servicio residual.

\item $B^{0}\left(t\right)$ es el tiempo residual del cambio de
cola.

\item $C\left(t\right)$ indica el n\'umero de usuarios atendidos
durante la visita del servidor a la cola dada en
$H\left(t\right)$.
\end{itemize}

$A_{k}\left(t\right),B_{m}\left(t\right)$ y
$B_{m}^{0}\left(t\right)$ se suponen continuas por la derecha y
que satisfacen la propiedad fuerte de Markov, (\cite{Dai})

\begin{itemize}
\item Los tiempos de interarribo a la cola $k$,son de la forma
$\left\{\xi_{k}\left(n\right)\right\}_{n\geq1}$, con la propiedad
de que son independientes e id{\'e}nticamente distribuidos.

\item Los tiempos de servicio
$\left\{\eta_{k}\left(n\right)\right\}_{n\geq1}$ tienen la
propiedad de ser independientes e id{\'e}nticamente distribuidos.

\item Se define la tasa de arribo a la $k$-{\'e}sima cola como
$\lambda_{k}=1/\esp\left[\xi_{k}\left(1\right)\right]$ y
adem{\'a}s se define

\item la tasa de servicio para la $k$-{\'e}sima cola como
$\mu_{k}=1/\esp\left[\eta_{k}\left(1\right)\right]$

\item tambi{\'e}n se define $\rho_{k}=\lambda_{k}/\mu_{k}$, donde
es necesario que $\rho<1$ para cuestiones de estabilidad.

\item De las pol{\'\i}ticas posibles solamente consideraremos la
pol{\'\i}tica cerrada (Gated).
\end{itemize}

%\section{Preliminares}



Sup\'ongase que el sistema consta de varias colas a los cuales
llegan uno o varios servidores a dar servicio a los usuarios
esperando en la cola.\\


Si $x$ es el n\'umero de usuarios en la cola al comienzo del
periodo de servicio y $N_{s}\left(x\right)=N\left(x\right)$ es el
n\'umero de usuarios que son atendidos con la pol\'itica $s$,
\'unica en nuestro caso, durante un periodo de servicio, entonces
se asume que:
\begin{itemize}
\item[1)]\label{S1}$lim_{x\rightarrow\infty}\esp\left[N\left(x\right)\right]=\overline{N}>0$
\item[2)]\label{S2}$\esp\left[N\left(x\right)\right]\leq\overline{N}$para
cualquier valor de $x$.
\end{itemize}
La manera en que atiende el servidor $m$-\'esimo, en este caso en
espec\'ifico solo lo ilustraremos con un s\'olo servidor, es la
siguiente:
\begin{itemize}
\item Al t\'ermino de la visita a la cola $j$, el servidor se
cambia a la cola $j^{'}$ con probabilidad
$r_{j,j^{'}}^{m}=r_{j,j^{'}}$

\item La $n$-\'esima ocurrencia va acompa\~nada con el tiempo de
cambio de longitud $\delta_{j,j^{'}}\left(n\right)$,
independientes e id\'enticamente distribuidas, con
$\esp\left[\delta_{j,j^{'}}\left(1\right)\right]\geq0$.

\item Sea $\left\{p_{j}\right\}$ la distribuci\'on invariante
estacionaria \'unica para la Cadena de Markov con matriz de
transici\'on $\left(r_{j,j^{'}}\right)$.

\item Finalmente, se define
\begin{equation}
\delta^{*}:=\sum_{j,j^{'}}p_{j}r_{j,j^{'}}\esp\left[\delta_{j,j^{'}}\left(i\right)\right].
\end{equation}
\end{itemize}

Veamos un caso muy espec\'ifico en el cual los tiempos de arribo a cada una de las colas se comportan de acuerdo a un proceso Poisson de la forma
$\left\{\xi_{k}\left(n\right)\right\}_{n\geq1}$, y los tiempos de servicio en cada una de las colas son variables aleatorias distribuidas exponencialmente e id\'enticamente distribuidas
$\left\{\eta_{k}\left(n\right)\right\}_{n\geq1}$, donde ambos procesos adem\'as cumplen la condici\'on de ser independientes entre si. Para la $k$-\'esima cola se define la tasa de arribo a la como
$\lambda_{k}=1/\esp\left[\xi_{k}\left(1\right)\right]$ y la tasa
de servicio como
$\mu_{k}=1/\esp\left[\eta_{k}\left(1\right)\right]$, finalmente se
define la carga de la cola como $\rho_{k}=\lambda_{k}/\mu_{k}$,
donde se pide que $\rho<1$, para garantizar la estabilidad del sistema.\\

Se denotar\'a por $Q_{k}\left(t\right)$ el n\'umero de usuarios en la cola $k$,
$A_{k}\left(t\right)$ los residuales de los tiempos entre arribos a la cola $k$;
para cada servidor $m$, se denota por $B_{m}\left(t\right)$ los residuales de los tiempos de servicio al tiempo $t$; $B_{m}^{0}\left(t\right)$ son los residuales de los tiempos de traslado de la cola $k$ a la pr\'oxima por atender, al tiempo $t$, finalmente sea $C_{m}\left(t\right)$ el n\'umero de usuarios atendidos durante la visita del servidor a la cola $k$ al tiempo $t$.\\


En este sentido el proceso para el sistema de visitas se puede definir como:

\begin{equation}\label{Esp.Edos.Down}
X\left(t\right)^{T}=\left(Q_{k}\left(t\right),A_{k}\left(t\right),B_{m}\left(t\right),B_{m}^{0}\left(t\right),C_{m}\left(t\right)\right)
\end{equation}
para $k=1,\ldots,K$ y $m=1,2,\ldots,M$. $X$ evoluciona en el
espacio de estados:
$X=\ent_{+}^{K}\times\rea_{+}^{K}\times\left(\left\{1,2,\ldots,K\right\}\times\left\{1,2,\ldots,S\right\}\right)^{M}\times\rea_{+}^{K}\times\ent_{+}^{K}$.\\

El sistema aqu\'i descrito debe de cumplir con los siguientes supuestos b\'asicos de un sistema de visitas:

Antes enunciemos los supuestos que regir\'an en la red.

\begin{itemize}
\item[A1)] $\xi_{1},\ldots,\xi_{K},\eta_{1},\ldots,\eta_{K}$ son
mutuamente independientes y son sucesiones independientes e
id\'enticamente distribuidas.

\item[A2)] Para alg\'un entero $p\geq1$
\begin{eqnarray*}
\esp\left[\xi_{l}\left(1\right)^{p+1}\right]<\infty\textrm{ para }l\in\mathcal{A}\textrm{ y }\\
\esp\left[\eta_{k}\left(1\right)^{p+1}\right]<\infty\textrm{ para
}k=1,\ldots,K.
\end{eqnarray*}
donde $\mathcal{A}$ es la clase de posibles arribos.

\item[A3)] Para $k=1,2,\ldots,K$ existe una funci\'on positiva
$q_{k}\left(x\right)$ definida en $\rea_{+}$, y un entero $j_{k}$,
tal que
\begin{eqnarray}
P\left(\xi_{k}\left(1\right)\geq x\right)>0\textrm{, para todo }x>0\\
P\left\{a\leq\sum_{i=1}^{j_{k}}\xi_{k}\left(i\right)\leq
b\right\}\geq\int_{a}^{b}q_{k}\left(x\right)dx, \textrm{ }0\leq
a<b.
\end{eqnarray}
\end{itemize}

En particular los procesos de tiempo entre arribos y de servicio
considerados con fines de ilustraci\'on de la metodolog\'ia
cumplen con el supuesto $A2)$ para $p=1$, es decir, ambos procesos
tienen primer y segundo momento finito.

En lo que respecta al supuesto (A3), en Dai y Meyn \cite{DaiSean}
hacen ver que este se puede sustituir por

\begin{itemize}
\item[A3')] Para el Proceso de Markov $X$, cada subconjunto
compacto de $X$ es un conjunto peque\~no, ver definici\'on
\ref{Def.Cto.Peq.}.
\end{itemize}

Es por esta raz\'on que con la finalidad de poder hacer uso de
$A3^{'})$ es necesario recurrir a los Procesos de Harris y en
particular a los Procesos Harris Recurrente:
%_______________________________________________________________________
\subsection{Procesos Harris Recurrente}
%_______________________________________________________________________

Por el supuesto (A1) conforme a Davis \cite{Davis}, se puede
definir el proceso de saltos correspondiente de manera tal que
satisfaga el supuesto (\ref{Sup3.1.Davis}), de hecho la
demostraci\'on est\'a basada en la l\'inea de argumentaci\'on de
Davis, (\cite{Davis}, p\'aginas 362-364).

Entonces se tiene un espacio de estados Markoviano. El espacio de
Markov descrito en Dai y Meyn \cite{DaiSean}

\[\left(\Omega,\mathcal{F},\mathcal{F}_{t},X\left(t\right),\theta_{t},P_{x}\right)\]
es un proceso de Borel Derecho (Sharpe \cite{Sharpe}) en el
espacio de estados medible $\left(X,\mathcal{B}_{X}\right)$. El
Proceso $X=\left\{X\left(t\right),t\geq0\right\}$ tiene
trayectorias continuas por la derecha, est\'a definida en
$\left(\Omega,\mathcal{F}\right)$ y est\'a adaptado a
$\left\{\mathcal{F}_{t},t\geq0\right\}$; la colecci\'on
$\left\{P_{x},x\in \mathbb{X}\right\}$ son medidas de probabilidad
en $\left(\Omega,\mathcal{F}\right)$ tales que para todo $x\in
\mathbb{X}$
\[P_{x}\left\{X\left(0\right)=x\right\}=1\] y
\[E_{x}\left\{f\left(X\circ\theta_{t}\right)|\mathcal{F}_{t}\right\}=E_{X}\left(\tau\right)f\left(X\right)\]
en $\left\{\tau<\infty\right\}$, $P_{x}$-c.s. Donde $\tau$ es un
$\mathcal{F}_{t}$-tiempo de paro
\[\left(X\circ\theta_{\tau}\right)\left(w\right)=\left\{X\left(\tau\left(w\right)+t,w\right),t\geq0\right\}\]
y $f$ es una funci\'on de valores reales acotada y medible con la
$\sigma$-algebra de Kolmogorov generada por los cilindros.\\

Sea $P^{t}\left(x,D\right)$, $D\in\mathcal{B}_{\mathbb{X}}$,
$t\geq0$ probabilidad de transici\'on de $X$ definida como
\[P^{t}\left(x,D\right)=P_{x}\left(X\left(t\right)\in
D\right)\]


\begin{Def}
Una medida no cero $\pi$ en
$\left(\mathbf{X},\mathcal{B}_{\mathbf{X}}\right)$ es {\bf
invariante} para $X$ si $\pi$ es $\sigma$-finita y
\[\pi\left(D\right)=\int_{\mathbf{X}}P^{t}\left(x,D\right)\pi\left(dx\right)\]
para todo $D\in \mathcal{B}_{\mathbf{X}}$, con $t\geq0$.
\end{Def}

\begin{Def}
El proceso de Markov $X$ es llamado Harris recurrente si existe
una medida de probabilidad $\nu$ en
$\left(\mathbf{X},\mathcal{B}_{\mathbf{X}}\right)$, tal que si
$\nu\left(D\right)>0$ y $D\in\mathcal{B}_{\mathbf{X}}$
\[P_{x}\left\{\tau_{D}<\infty\right\}\equiv1\] cuando
$\tau_{D}=inf\left\{t\geq0:X_{t}\in D\right\}$.
\end{Def}

\begin{Note}
\begin{itemize}
\item[i)] Si $X$ es Harris recurrente, entonces existe una \'unica
medida invariante $\pi$ (Getoor \cite{Getoor}).

\item[ii)] Si la medida invariante es finita, entonces puede
normalizarse a una medida de probabilidad, en este caso se le
llama Proceso {\em Harris recurrente positivo}.


\item[iii)] Cuando $X$ es Harris recurrente positivo se dice que
la disciplina de servicio es estable. En este caso $\pi$ denota la
distribuci\'on estacionaria y hacemos
\[P_{\pi}\left(\cdot\right)=\int_{\mathbf{X}}P_{x}\left(\cdot\right)\pi\left(dx\right)\]
y se utiliza $E_{\pi}$ para denotar el operador esperanza
correspondiente.
\end{itemize}
\end{Note}

\begin{Def}\label{Def.Cto.Peq.}
Un conjunto $D\in\mathcal{B_{\mathbf{X}}}$ es llamado peque\~no si
existe un $t>0$, una medida de probabilidad $\nu$ en
$\mathcal{B_{\mathbf{X}}}$, y un $\delta>0$ tal que
\[P^{t}\left(x,A\right)\geq\delta\nu\left(A\right)\] para $x\in
D,A\in\mathcal{B_{X}}$.
\end{Def}

La siguiente serie de resultados vienen enunciados y demostrados
en Dai \cite{Dai}:
\begin{Lema}[Lema 3.1, Dai\cite{Dai}]
Sea $B$ conjunto peque\~no cerrado, supongamos que
$P_{x}\left(\tau_{B}<\infty\right)\equiv1$ y que para alg\'un
$\delta>0$ se cumple que
\begin{equation}\label{Eq.3.1}
\sup\esp_{x}\left[\tau_{B}\left(\delta\right)\right]<\infty,
\end{equation}
donde
$\tau_{B}\left(\delta\right)=inf\left\{t\geq\delta:X\left(t\right)\in
B\right\}$. Entonces, $X$ es un proceso Harris Recurrente
Positivo.
\end{Lema}

\begin{Lema}[Lema 3.1, Dai \cite{Dai}]\label{Lema.3.}
Bajo el supuesto (A3), el conjunto $B=\left\{|x|\leq k\right\}$ es
un conjunto peque\~no cerrado para cualquier $k>0$.
\end{Lema}

\begin{Teo}[Teorema 3.1, Dai\cite{Dai}]\label{Tma.3.1}
Si existe un $\delta>0$ tal que
\begin{equation}
lim_{|x|\rightarrow\infty}\frac{1}{|x|}\esp|X^{x}\left(|x|\delta\right)|=0,
\end{equation}
entonces la ecuaci\'on (\ref{Eq.3.1}) se cumple para
$B=\left\{|x|\leq k\right\}$ con alg\'un $k>0$. En particular, $X$
es Harris Recurrente Positivo.
\end{Teo}

\begin{Note}
En Meyn and Tweedie \cite{MeynTweedie} muestran que si
$P_{x}\left\{\tau_{D}<\infty\right\}\equiv1$ incluso para solo un
conjunto peque\~no, entonces el proceso es Harris Recurrente.
\end{Note}

Entonces, tenemos que el proceso $X$ es un proceso de Markov que
cumple con los supuestos $A1)$-$A3)$, lo que falta de hacer es
construir el Modelo de Flujo bas\'andonos en lo hasta ahora
presentado.
%_______________________________________________________________________
\subsection{Modelo de Flujo}
%_______________________________________________________________________

Dada una condici\'on inicial $x\in\textrm{X}$, sea
$Q_{k}^{x}\left(t\right)$ la longitud de la cola al tiempo $t$,
$T_{m,k}^{x}\left(t\right)$ el tiempo acumulado, al tiempo $t$,
que tarda el servidor $m$ en atender a los usuarios de la cola
$k$. Finalmente sea $T_{m,k}^{x,0}\left(t\right)$ el tiempo
acumulado, al tiempo $t$, que tarda el servidor $m$ en trasladarse
a otra cola a partir de la $k$-\'esima.\\

Sup\'ongase que la funci\'on
$\left(\overline{Q}\left(\cdot\right),\overline{T}_{m}
\left(\cdot\right),\overline{T}_{m}^{0} \left(\cdot\right)\right)$
para $m=1,2,\ldots,M$ es un punto l\'imite de
\begin{equation}\label{Eq.Punto.Limite}
\left(\frac{1}{|x|}Q^{x}\left(|x|t\right),\frac{1}{|x|}T_{m}^{x}\left(|x|t\right),\frac{1}{|x|}T_{m}^{x,0}\left(|x|t\right)\right)
\end{equation}
para $m=1,2,\ldots,M$, cuando $x\rightarrow\infty$. Entonces
$\left(\overline{Q}\left(t\right),\overline{T}_{m}
\left(t\right),\overline{T}_{m}^{0} \left(t\right)\right)$ es un
flujo l\'imite del sistema. Al conjunto de todos las posibles
flujos l\'imite se le llama \textbf{Modelo de Flujo}.\\

El modelo de flujo satisface el siguiente conjunto de ecuaciones:

\begin{equation}\label{Eq.MF.1}
\overline{Q}_{k}\left(t\right)=\overline{Q}_{k}\left(0\right)+\lambda_{k}t-\sum_{m=1}^{M}\mu_{k}\overline{T}_{m,k}\left(t\right)\\
\end{equation}
para $k=1,2,\ldots,K$.\\
\begin{equation}\label{Eq.MF.2}
\overline{Q}_{k}\left(t\right)\geq0\textrm{ para
}k=1,2,\ldots,K,\\
\end{equation}

\begin{equation}\label{Eq.MF.3}
\overline{T}_{m,k}\left(0\right)=0,\textrm{ y }\overline{T}_{m,k}\left(\cdot\right)\textrm{ es no decreciente},\\
\end{equation}
para $k=1,2,\ldots,K$ y $m=1,2,\ldots,M$,\\
\begin{equation}\label{Eq.MF.4}
\sum_{k=1}^{K}\overline{T}_{m,k}^{0}\left(t\right)+\overline{T}_{m,k}\left(t\right)=t\textrm{
para }m=1,2,\ldots,M.\\
\end{equation}

De acuerdo a Dai \cite{Dai}, se tiene que el conjunto de posibles
l\'imites
$\left(\overline{Q}\left(\cdot\right),\overline{T}\left(\cdot\right),\overline{T}^{0}\left(\cdot\right)\right)$,
en el sentido de que deben de satisfacer las ecuaciones
(\ref{Eq.MF.1})-(\ref{Eq.MF.4}), se le llama {\em Modelo de
Flujo}.


\begin{Def}[Definici\'on 4.1, , Dai \cite{Dai}]\label{Def.Modelo.Flujo}
Sea una disciplina de servicio espec\'ifica. Cualquier l\'imite
$\left(\overline{Q}\left(\cdot\right),\overline{T}\left(\cdot\right)\right)$
en (\ref{Eq.Punto.Limite}) es un {\em flujo l\'imite} de la
disciplina. Cualquier soluci\'on (\ref{Eq.MF.1})-(\ref{Eq.MF.4})
es llamado flujo soluci\'on de la disciplina. Se dice que el
modelo de flujo l\'imite, modelo de flujo, de la disciplina de la
cola es estable si existe una constante $\delta>0$ que depende de
$\mu,\lambda$ y $P$ solamente, tal que cualquier flujo l\'imite
con
$|\overline{Q}\left(0\right)|+|\overline{U}|+|\overline{V}|=1$, se
tiene que $\overline{Q}\left(\cdot+\delta\right)\equiv0$.
\end{Def}

Al conjunto de ecuaciones dadas en \ref{Eq.MF.1}-\ref{Eq.MF.4} se
le llama {\em Modelo de flujo} y al conjunto de todas las
soluciones del modelo de flujo
$\left(\overline{Q}\left(\cdot\right),\overline{T}
\left(\cdot\right)\right)$ se le denotar\'a por $\mathcal{Q}$.

Si se hace $|x|\rightarrow\infty$ sin restringir ninguna de las
componentes, tambi\'en se obtienen un modelo de flujo, pero en
este caso el residual de los procesos de arribo y servicio
introducen un retraso:
\begin{Teo}[Teorema 4.2, Dai\cite{Dai}]\label{Tma.4.2.Dai}
Sea una disciplina fija para la cola, suponga que se cumplen las
condiciones (A1))-(A3)). Si el modelo de flujo l\'imite de la
disciplina de la cola es estable, entonces la cadena de Markov $X$
que describe la din\'amica de la red bajo la disciplina es Harris
recurrente positiva.
\end{Teo}

Ahora se procede a escalar el espacio y el tiempo para reducir la
aparente fluctuaci\'on del modelo. Consid\'erese el proceso
\begin{equation}\label{Eq.3.7}
\overline{Q}^{x}\left(t\right)=\frac{1}{|x|}Q^{x}\left(|x|t\right)
\end{equation}
A este proceso se le conoce como el fluido escalado, y cualquier
l\'imite $\overline{Q}^{x}\left(t\right)$ es llamado flujo
l\'imite del proceso de longitud de la cola. Haciendo
$|q|\rightarrow\infty$ mientras se mantiene el resto de las
componentes fijas, cualquier punto l\'imite del proceso de
longitud de la cola normalizado $\overline{Q}^{x}$ es soluci\'on
del siguiente modelo de flujo.


\begin{Def}[Definici\'on 3.3, Dai y Meyn \cite{DaiSean}]
El modelo de flujo es estable si existe un tiempo fijo $t_{0}$ tal
que $\overline{Q}\left(t\right)=0$, con $t\geq t_{0}$, para
cualquier $\overline{Q}\left(\cdot\right)\in\mathcal{Q}$ que
cumple con $|\overline{Q}\left(0\right)|=1$.
\end{Def}

El siguiente resultado se encuentra en Chen \cite{Chen}.
\begin{Lemma}[Lema 3.1, Dai y Meyn \cite{DaiSean}]
Si el modelo de flujo definido por \ref{Eq.MF.1}-\ref{Eq.MF.4} es
estable, entonces el modelo de flujo retrasado es tambi\'en
estable, es decir, existe $t_{0}>0$ tal que
$\overline{Q}\left(t\right)=0$ para cualquier $t\geq t_{0}$, para
cualquier soluci\'on del modelo de flujo retrasado cuya
condici\'on inicial $\overline{x}$ satisface que
$|\overline{x}|=|\overline{Q}\left(0\right)|+|\overline{A}\left(0\right)|+|\overline{B}\left(0\right)|\leq1$.
\end{Lemma}


Ahora ya estamos en condiciones de enunciar los resultados principales:


\begin{Teo}[Teorema 2.1, Down \cite{Down}]\label{Tma2.1.Down}
Suponga que el modelo de flujo es estable, y que se cumplen los supuestos (A1) y (A2), entonces
\begin{itemize}
\item[i)] Para alguna constante $\kappa_{p}$, y para cada
condici\'on inicial $x\in X$
\begin{equation}\label{Estability.Eq1}
limsup_{t\rightarrow\infty}\frac{1}{t}\int_{0}^{t}\esp_{x}\left[|Q\left(s\right)|^{p}\right]ds\leq\kappa_{p},
\end{equation}
donde $p$ es el entero dado en (A2).
\end{itemize}
Si adem\'as se cumple la condici\'on (A3), entonces para cada
condici\'on inicial:
\begin{itemize}
\item[ii)] Los momentos transitorios convergen a su estado
estacionario:
 \begin{equation}\label{Estability.Eq2}
lim_{t\rightarrow\infty}\esp_{x}\left[Q_{k}\left(t\right)^{r}\right]=\esp_{\pi}\left[Q_{k}\left(0\right)^{r}\right]\leq\kappa_{r},
\end{equation}
para $r=1,2,\ldots,p$ y $k=1,2,\ldots,K$. Donde $\pi$ es la
probabilidad invariante para $\mathbf{X}$.

\item[iii)]  El primer momento converge con raz\'on $t^{p-1}$:
\begin{equation}\label{Estability.Eq3}
lim_{t\rightarrow\infty}t^{p-1}|\esp_{x}\left[Q_{k}\left(t\right)\right]-\esp_{\pi}\left[Q_{k}\left(0\right)\right]=0.
\end{equation}

\item[iv)] La {\em Ley Fuerte de los grandes n\'umeros} se cumple:
\begin{equation}\label{Estability.Eq4}
lim_{t\rightarrow\infty}\frac{1}{t}\int_{0}^{t}Q_{k}^{r}\left(s\right)ds=\esp_{\pi}\left[Q_{k}\left(0\right)^{r}\right],\textrm{
}\prob_{x}\textrm{-c.s.}
\end{equation}
para $r=1,2,\ldots,p$ y $k=1,2,\ldots,K$.
\end{itemize}
\end{Teo}

La contribuci\'on de Down a la teor\'ia de los Sistemas de Visitas
C\'iclicas, es la relaci\'on que hay entre la estabilidad del
sistema con el comportamiento de las medidas de desempe\~no, es
decir, la condici\'on suficiente para poder garantizar la
convergencia del proceso de la longitud de la cola as\'i como de
por los menos los dos primeros momentos adem\'as de una versi\'on
de la Ley Fuerte de los Grandes N\'umeros para los sistemas de
visitas.


\begin{Teo}[Teorema 2.3, Down \cite{Down}]\label{Tma2.3.Down}
Considere el siguiente valor:
\begin{equation}\label{Eq.Rho.1serv}
\rho=\sum_{k=1}^{K}\rho_{k}+max_{1\leq j\leq K}\left(\frac{\lambda_{j}}{\sum_{s=1}^{S}p_{js}\overline{N}_{s}}\right)\delta^{*}
\end{equation}
\begin{itemize}
\item[i)] Si $\rho<1$ entonces la red es estable, es decir, se cumple el teorema \ref{Tma2.1.Down}.

\item[ii)] Si $\rho<1$ entonces la red es inestable, es decir, se cumple el teorema \ref{Tma2.2.Down}
\end{itemize}
\end{Teo}

\begin{Teo}
Sea $\left(X_{n},\mathcal{F}_{n},n=0,1,\ldots,\right\}$ Proceso de
Markov con espacio de estados $\left(S_{0},\chi_{0}\right)$
generado por una distribuici\'on inicial $P_{o}$ y probabilidad de
transici\'on $p_{mn}$, para $m,n=0,1,\ldots,$ $m<n$, que por
notaci\'on se escribir\'a como $p\left(m,n,x,B\right)\rightarrow
p_{mn}\left(x,B\right)$. Sea $S$ tiempo de paro relativo a la
$\sigma$-\'algebra $\mathcal{F}_{n}$. Sea $T$ funci\'on medible,
$T:\Omega\rightarrow\left\{0,1,\ldots,\right\}$. Sup\'ongase que
$T\geq S$, entonces $T$ es tiempo de paro. Si $B\in\chi_{0}$,
entonces
\begin{equation}\label{Prop.Fuerte.Markov}
P\left\{X\left(T\right)\in
B,T<\infty|\mathcal{F}\left(S\right)\right\} =
p\left(S,T,X\left(s\right),B\right)
\end{equation}
en $\left\{T<\infty\right\}$.
\end{Teo}


Sea $K$ conjunto numerable y sea $d:K\rightarrow\nat$ funci\'on.
Para $v\in K$, $M_{v}$ es un conjunto abierto de
$\rea^{d\left(v\right)}$. Entonces \[E=\cup_{v\in
K}M_{v}=\left\{\left(v,\zeta\right):v\in K,\zeta\in
M_{v}\right\}.\]

Sea $\mathcal{E}$ la clase de conjuntos medibles en $E$:
\[\mathcal{E}=\left\{\cup_{v\in K}A_{v}:A_{v}\in \mathcal{M}_{v}\right\}.\]

donde $\mathcal{M}$ son los conjuntos de Borel de $M_{v}$.
Entonces $\left(E,\mathcal{E}\right)$ es un espacio de Borel. El
estado del proceso se denotar\'a por
$\mathbf{x}_{t}=\left(v_{t},\zeta_{t}\right)$. La distribuci\'on
de $\left(\mathbf{x}_{t}\right)$ est\'a determinada por por los
siguientes objetos:

\begin{itemize}
\item[i)] Los campos vectoriales $\left(\mathcal{H}_{v},v\in
K\right)$. \item[ii)] Una funci\'on medible $\lambda:E\rightarrow
\rea_{+}$. \item[iii)] Una medida de transici\'on
$Q:\mathcal{E}\times\left(E\cup\Gamma^{*}\right)\rightarrow\left[0,1\right]$
donde
\begin{equation}
\Gamma^{*}=\cup_{v\in K}\partial^{*}M_{v}.
\end{equation}
y
\begin{equation}
\partial^{*}M_{v}=\left\{z\in\partial M_{v}:\mathbf{\mathbf{\phi}_{v}\left(t,\zeta\right)=\mathbf{z}}\textrm{ para alguna }\left(t,\zeta\right)\in\rea_{+}\times M_{v}\right\}.
\end{equation}
$\partial M_{v}$ denota  la frontera de $M_{v}$.
\end{itemize}

El campo vectorial $\left(\mathcal{H}_{v},v\in K\right)$ se supone
tal que para cada $\mathbf{z}\in M_{v}$ existe una \'unica curva
integral $\mathbf{\phi}_{v}\left(t,\zeta\right)$ que satisface la
ecuaci\'on

\begin{equation}
\frac{d}{dt}f\left(\zeta_{t}\right)=\mathcal{H}f\left(\zeta_{t}\right),
\end{equation}
con $\zeta_{0}=\mathbf{z}$, para cualquier funci\'on suave
$f:\rea^{d}\rightarrow\rea$ y $\mathcal{H}$ denota el operador
diferencial de primer orden, con $\mathcal{H}=\mathcal{H}_{v}$ y
$\zeta_{t}=\mathbf{\phi}\left(t,\mathbf{z}\right)$. Adem\'as se
supone que $\mathcal{H}_{v}$ es conservativo, es decir, las curvas
integrales est\'an definidas para todo $t>0$.

Para $\mathbf{x}=\left(v,\zeta\right)\in E$ se denota
\[t^{*}\mathbf{x}=inf\left\{t>0:\mathbf{\phi}_{v}\left(t,\zeta\right)\in\partial^{*}M_{v}\right\}\]

En lo que respecta a la funci\'on $\lambda$, se supondr\'a que
para cada $\left(v,\zeta\right)\in E$ existe un $\epsilon>0$ tal
que la funci\'on
$s\rightarrow\lambda\left(v,\phi_{v}\left(s,\zeta\right)\right)\in
E$ es integrable para $s\in\left[0,\epsilon\right)$. La medida de
transici\'on $Q\left(A;\mathbf{x}\right)$ es una funci\'on medible
de $\mathbf{x}$ para cada $A\in\mathcal{E}$, definida para
$\mathbf{x}\in E\cup\Gamma^{*}$ y es una medida de probabilidad en
$\left(E,\mathcal{E}\right)$ para cada $\mathbf{x}\in E$.

El movimiento del proceso $\left(\mathbf{x}_{t}\right)$ comenzando
en $\mathbf{x}=\left(n,\mathbf{z}\right)\in E$ se puede construir
de la siguiente manera, def\'inase la funci\'on $F$ por

\begin{equation}
F\left(t\right)=\left\{\begin{array}{ll}\\
exp\left(-\int_{0}^{t}\lambda\left(n,\phi_{n}\left(s,\mathbf{z}\right)\right)ds\right), & t<t^{*}\left(\mathbf{x}\right),\\
0, & t\geq t^{*}\left(\mathbf{x}\right)
\end{array}\right.
\end{equation}

Sea $T_{1}$ una variable aleatoria tal que
$\prob\left[T_{1}>t\right]=F\left(t\right)$, ahora sea la variable
aleatoria $\left(N,Z\right)$ con distribuici\'on
$Q\left(\cdot;\phi_{n}\left(T_{1},\mathbf{z}\right)\right)$. La
trayectoria de $\left(\mathbf{x}_{t}\right)$ para $t\leq T_{1}$
es\footnote{Revisar p\'agina 362, y 364 de Davis \cite{Davis}.}
\begin{eqnarray*}
\mathbf{x}_{t}=\left(v_{t},\zeta_{t}\right)=\left\{\begin{array}{ll}
\left(n,\phi_{n}\left(t,\mathbf{z}\right)\right), & t<T_{1},\\
\left(N,\mathbf{Z}\right), & t=t_{1}.
\end{array}\right.
\end{eqnarray*}

Comenzando en $\mathbf{x}_{T_{1}}$ se selecciona el siguiente
tiempo de intersalto $T_{2}-T_{1}$ lugar del post-salto
$\mathbf{x}_{T_{2}}$ de manera similar y as\'i sucesivamente. Este
procedimiento nos da una trayectoria determinista por partes
$\mathbf{x}_{t}$ con tiempos de salto $T_{1},T_{2},\ldots$. Bajo
las condiciones enunciadas para $\lambda,T_{1}>0$  y
$T_{1}-T_{2}>0$ para cada $i$, con probabilidad 1. Se supone que
se cumple la siguiente condici\'on.

\begin{Sup}[Supuesto 3.1, Davis \cite{Davis}]\label{Sup3.1.Davis}
Sea $N_{t}:=\sum_{t}\indora_{\left(t\geq t\right)}$ el n\'umero de
saltos en $\left[0,t\right]$. Entonces
\begin{equation}
\esp\left[N_{t}\right]<\infty\textrm{ para toda }t.
\end{equation}
\end{Sup}

es un proceso de Markov, m\'as a\'un, es un Proceso Fuerte de
Markov, es decir, la Propiedad Fuerte de Markov se cumple para
cualquier tiempo de paro.


Sea $E$ es un espacio m\'etrico separable y la m\'etrica $d$ es
compatible con la topolog\'ia.


\begin{Def}
Un espacio topol\'ogico $E$ es llamado de {\em Rad\'on} si es
homeomorfo a un subconjunto universalmente medible de un espacio
m\'etrico compacto.
\end{Def}

Equivalentemente, la definici\'on de un espacio de Rad\'on puede
encontrarse en los siguientes t\'erminos:


\begin{Def}
$E$ es un espacio de Rad\'on si cada medida finita en
$\left(E,\mathcal{B}\left(E\right)\right)$ es regular interior o
cerrada, {\em tight}.
\end{Def}

\begin{Def}
Una medida finita, $\lambda$ en la $\sigma$-\'algebra de Borel de
un espacio metrizable $E$ se dice cerrada si
\begin{equation}\label{Eq.A2.3}
\lambda\left(E\right)=sup\left\{\lambda\left(K\right):K\textrm{ es
compacto en }E\right\}.
\end{equation}
\end{Def}

El siguiente teorema nos permite tener una mejor caracterizaci\'on
de los espacios de Rad\'on:
\begin{Teo}\label{Tma.A2.2}
Sea $E$ espacio separable metrizable. Entonces $E$ es Radoniano si
y s\'olo s\'i cada medida finita en
$\left(E,\mathcal{B}\left(E\right)\right)$ es cerrada.
\end{Teo}

Sea $E$ espacio de estados, tal que $E$ es un espacio de Rad\'on,
$\mathcal{B}\left(E\right)$ $\sigma$-\'algebra de Borel en $E$,
que se denotar\'a por $\mathcal{E}$.

Sea $\left(X,\mathcal{G},\prob\right)$ espacio de probabilidad,
$I\subset\rea$ conjunto de \'indices. Sea $\mathcal{F}_{\leq t}$
la $\sigma$-\'algebra natural definida como
$\sigma\left\{f\left(X_{r}\right):r\in I, r\leq
t,f\in\mathcal{E}\right\}$. Se considerar\'a una
$\sigma$-\'algebra m\'as general, $ \left(\mathcal{G}_{t}\right)$
tal que $\left(X_{t}\right)$ sea $\mathcal{E}$-adaptado.

\begin{Def}
Una familia $\left(P_{s,t}\right)$ de kernels de Markov en
$\left(E,\mathcal{E}\right)$ indexada por pares $s,t\in I$, con
$s\leq t$ es una funci\'on de transici\'on en $\ER$, si  para todo
$r\leq s< t$ en $I$ y todo $x\in E$,
$B\in\mathcal{E}$\footnote{Ecuaci\'on de Chapman-Kolmogorov}
\begin{equation}\label{Eq.Kernels}
P_{r,t}\left(x,B\right)=\int_{E}P_{r,s}\left(x,dy\right)P_{s,t}\left(y,B\right).
\end{equation}
\end{Def}

Se dice que la funci\'on de transici\'on $\KM$ en $\ER$ es la
funci\'on de transici\'on para un proceso $\PE$  con valores en
$E$ y que satisface la propiedad de
Markov\footnote{\begin{equation}\label{Eq.1.4.S}
\prob\left\{H|\mathcal{G}_{t}\right\}=\prob\left\{H|X_{t}\right\}\textrm{
}H\in p\mathcal{F}_{\geq t}.
\end{equation}} (\ref{Eq.1.4.S}) relativa a $\left(\mathcal{G}_{t}\right)$ si

\begin{equation}\label{Eq.1.6.S}
\prob\left\{f\left(X_{t}\right)|\mathcal{G}_{s}\right\}=P_{s,t}f\left(X_{t}\right)\textrm{
}s\leq t\in I,\textrm{ }f\in b\mathcal{E}.
\end{equation}

\begin{Def}
Una familia $\left(P_{t}\right)_{t\geq0}$ de kernels de Markov en
$\ER$ es llamada {\em Semigrupo de Transici\'on de Markov} o {\em
Semigrupo de Transici\'on} si
\[P_{t+s}f\left(x\right)=P_{t}\left(P_{s}f\right)\left(x\right),\textrm{ }t,s\geq0,\textrm{ }x\in E\textrm{ }f\in b\mathcal{E}.\]
\end{Def}
\begin{Note}
Si la funci\'on de transici\'on $\KM$ es llamada homog\'enea si
$P_{s,t}=P_{t-s}$.
\end{Note}

Un proceso de Markov que satisface la ecuaci\'on (\ref{Eq.1.6.S})
con funci\'on de transici\'on homog\'enea $\left(P_{t}\right)$
tiene la propiedad caracter\'istica
\begin{equation}\label{Eq.1.8.S}
\prob\left\{f\left(X_{t+s}\right)|\mathcal{G}_{t}\right\}=P_{s}f\left(X_{t}\right)\textrm{
}t,s\geq0,\textrm{ }f\in b\mathcal{E}.
\end{equation}
La ecuaci\'on anterior es la {\em Propiedad Simple de Markov} de
$X$ relativa a $\left(P_{t}\right)$.

En este sentido el proceso $\PE$ cumple con la propiedad de Markov
(\ref{Eq.1.8.S}) relativa a
$\left(\Omega,\mathcal{G},\mathcal{G}_{t},\prob\right)$ con
semigrupo de transici\'on $\left(P_{t}\right)$.

\begin{Def}
Un proceso estoc\'astico $\PE$ definido en
$\left(\Omega,\mathcal{G},\prob\right)$ con valores en el espacio
topol\'ogico $E$ es continuo por la derecha si cada trayectoria
muestral $t\rightarrow X_{t}\left(w\right)$ es un mapeo continuo
por la derecha de $I$ en $E$.
\end{Def}

\begin{Def}[HD1]\label{Eq.2.1.S}
Un semigrupo de Markov $\left(P_{t}\right)$ en un espacio de
Rad\'on $E$ se dice que satisface la condici\'on {\em HD1} si,
dada una medida de probabilidad $\mu$ en $E$, existe una
$\sigma$-\'algebra $\mathcal{E^{*}}$ con
$\mathcal{E}\subset\mathcal{E}^{*}$ y
$P_{t}\left(b\mathcal{E}^{*}\right)\subset b\mathcal{E}^{*}$, y un
$\mathcal{E}^{*}$-proceso $E$-valuado continuo por la derecha
$\PE$ en alg\'un espacio de probabilidad filtrado
$\left(\Omega,\mathcal{G},\mathcal{G}_{t},\prob\right)$ tal que
$X=\left(\Omega,\mathcal{G},\mathcal{G}_{t},\prob\right)$ es de
Markov (Homog\'eneo) con semigrupo de transici\'on $(P_{t})$ y
distribuci\'on inicial $\mu$.
\end{Def}

Consid\'erese la colecci\'on de variables aleatorias $X_{t}$
definidas en alg\'un espacio de probabilidad, y una colecci\'on de
medidas $\mathbf{P}^{x}$ tales que
$\mathbf{P}^{x}\left\{X_{0}=x\right\}$, y bajo cualquier
$\mathbf{P}^{x}$, $X_{t}$ es de Markov con semigrupo
$\left(P_{t}\right)$. $\mathbf{P}^{x}$ puede considerarse como la
distribuci\'on condicional de $\mathbf{P}$ dado $X_{0}=x$.

\begin{Def}\label{Def.2.2.S}
Sea $E$ espacio de Rad\'on, $\SG$ semigrupo de Markov en $\ER$. La
colecci\'on
$\mathbf{X}=\left(\Omega,\mathcal{G},\mathcal{G}_{t},X_{t},\theta_{t},\CM\right)$
es un proceso $\mathcal{E}$-Markov continuo por la derecha simple,
con espacio de estados $E$ y semigrupo de transici\'on $\SG$ en
caso de que $\mathbf{X}$ satisfaga las siguientes condiciones:
\begin{itemize}
\item[i)] $\left(\Omega,\mathcal{G},\mathcal{G}_{t}\right)$ es un
espacio de medida filtrado, y $X_{t}$ es un proceso $E$-valuado
continuo por la derecha $\mathcal{E}^{*}$-adaptado a
$\left(\mathcal{G}_{t}\right)$;

\item[ii)] $\left(\theta_{t}\right)_{t\geq0}$ es una colecci\'on
de operadores {\em shift} para $X$, es decir, mapea $\Omega$ en
s\'i mismo satisfaciendo para $t,s\geq0$,

\begin{equation}\label{Eq.Shift}
\theta_{t}\circ\theta_{s}=\theta_{t+s}\textrm{ y
}X_{t}\circ\theta_{t}=X_{t+s};
\end{equation}

\item[iii)] Para cualquier $x\in E$,$\CM\left\{X_{0}=x\right\}=1$,
y el proceso $\PE$ tiene la propiedad de Markov (\ref{Eq.1.8.S})
con semigrupo de transici\'on $\SG$ relativo a
$\left(\Omega,\mathcal{G},\mathcal{G}_{t},\CM\right)$.
\end{itemize}
\end{Def}

\begin{Def}[HD2]\label{Eq.2.2.S}
Para cualquier $\alpha>0$ y cualquier $f\in S^{\alpha}$, el
proceso $t\rightarrow f\left(X_{t}\right)$ es continuo por la
derecha casi seguramente.
\end{Def}

\begin{Def}\label{Def.PD}
Un sistema
$\mathbf{X}=\left(\Omega,\mathcal{G},\mathcal{G}_{t},X_{t},\theta_{t},\CM\right)$
es un proceso derecho en el espacio de Rad\'on $E$ con semigrupo
de transici\'on $\SG$ provisto de:
\begin{itemize}
\item[i)] $\mathbf{X}$ es una realizaci\'on  continua por la
derecha, \ref{Def.2.2.S}, de $\SG$.

\item[ii)] $\mathbf{X}$ satisface la condicion HD2,
\ref{Eq.2.2.S}, relativa a $\mathcal{G}_{t}$.

\item[iii)] $\mathcal{G}_{t}$ es aumentado y continuo por la
derecha.
\end{itemize}
\end{Def}

\begin{Lema}[Lema 4.2, Dai\cite{Dai}]\label{Lema4.2}
Sea $\left\{x_{n}\right\}\subset \mathbf{X}$ con
$|x_{n}|\rightarrow\infty$, conforme $n\rightarrow\infty$. Suponga
que
\[lim_{n\rightarrow\infty}\frac{1}{|x_{n}|}U\left(0\right)=\overline{U}\]
y
\[lim_{n\rightarrow\infty}\frac{1}{|x_{n}|}V\left(0\right)=\overline{V}.\]

Entonces, conforme $n\rightarrow\infty$, casi seguramente

\begin{equation}\label{E1.4.2}
\frac{1}{|x_{n}|}\Phi^{k}\left(\left[|x_{n}|t\right]\right)\rightarrow
P_{k}^{'}t\textrm{, u.o.c.,}
\end{equation}

\begin{equation}\label{E1.4.3}
\frac{1}{|x_{n}|}E^{x_{n}}_{k}\left(|x_{n}|t\right)\rightarrow
\alpha_{k}\left(t-\overline{U}_{k}\right)^{+}\textrm{, u.o.c.,}
\end{equation}

\begin{equation}\label{E1.4.4}
\frac{1}{|x_{n}|}S^{x_{n}}_{k}\left(|x_{n}|t\right)\rightarrow
\mu_{k}\left(t-\overline{V}_{k}\right)^{+}\textrm{, u.o.c.,}
\end{equation}

donde $\left[t\right]$ es la parte entera de $t$ y
$\mu_{k}=1/m_{k}=1/\esp\left[\eta_{k}\left(1\right)\right]$.
\end{Lema}

\begin{Lema}[Lema 4.3, Dai\cite{Dai}]\label{Lema.4.3}
Sea $\left\{x_{n}\right\}\subset \mathbf{X}$ con
$|x_{n}|\rightarrow\infty$, conforme $n\rightarrow\infty$. Suponga
que
\[lim_{n\rightarrow\infty}\frac{1}{|x_{n}|}U\left(0\right)=\overline{U}_{k}\]
y
\[lim_{n\rightarrow\infty}\frac{1}{|x_{n}|}V\left(0\right)=\overline{V}_{k}.\]
\begin{itemize}
\item[a)] Conforme $n\rightarrow\infty$ casi seguramente,
\[lim_{n\rightarrow\infty}\frac{1}{|x_{n}|}U^{x_{n}}_{k}\left(|x_{n}|t\right)=\left(\overline{U}_{k}-t\right)^{+}\textrm{, u.o.c.}\]
y
\[lim_{n\rightarrow\infty}\frac{1}{|x_{n}|}V^{x_{n}}_{k}\left(|x_{n}|t\right)=\left(\overline{V}_{k}-t\right)^{+}.\]

\item[b)] Para cada $t\geq0$ fijo,
\[\left\{\frac{1}{|x_{n}|}U^{x_{n}}_{k}\left(|x_{n}|t\right),|x_{n}|\geq1\right\}\]
y
\[\left\{\frac{1}{|x_{n}|}V^{x_{n}}_{k}\left(|x_{n}|t\right),|x_{n}|\geq1\right\}\]
\end{itemize}
son uniformemente convergentes.
\end{Lema}

$S_{l}^{x}\left(t\right)$ es el n\'umero total de servicios
completados de la clase $l$, si la clase $l$ est\'a dando $t$
unidades de tiempo de servicio. Sea $T_{l}^{x}\left(x\right)$ el
monto acumulado del tiempo de servicio que el servidor
$s\left(l\right)$ gasta en los usuarios de la clase $l$ al tiempo
$t$. Entonces $S_{l}^{x}\left(T_{l}^{x}\left(t\right)\right)$ es
el n\'umero total de servicios completados para la clase $l$ al
tiempo $t$. Una fracci\'on de estos usuarios,
$\Phi_{l}^{x}\left(S_{l}^{x}\left(T_{l}^{x}\left(t\right)\right)\right)$,
se convierte en usuarios de la clase $k$.\\

Entonces, dado lo anterior, se tiene la siguiente representaci\'on
para el proceso de la longitud de la cola:\\

\begin{equation}
Q_{k}^{x}\left(t\right)=_{k}^{x}\left(0\right)+E_{k}^{x}\left(t\right)+\sum_{l=1}^{K}\Phi_{k}^{l}\left(S_{l}^{x}\left(T_{l}^{x}\left(t\right)\right)\right)-S_{k}^{x}\left(T_{k}^{x}\left(t\right)\right)
\end{equation}
para $k=1,\ldots,K$. Para $i=1,\ldots,d$, sea
\[I_{i}^{x}\left(t\right)=t-\sum_{j\in C_{i}}T_{k}^{x}\left(t\right).\]

Entonces $I_{i}^{x}\left(t\right)$ es el monto acumulado del
tiempo que el servidor $i$ ha estado desocupado al tiempo $t$. Se
est\'a asumiendo que las disciplinas satisfacen la ley de
conservaci\'on del trabajo, es decir, el servidor $i$ est\'a en
pausa solamente cuando no hay usuarios en la estaci\'on $i$.
Entonces, se tiene que

\begin{equation}
\int_{0}^{\infty}\left(\sum_{k\in
C_{i}}Q_{k}^{x}\left(t\right)\right)dI_{i}^{x}\left(t\right)=0,
\end{equation}
para $i=1,\ldots,d$.\\

Hacer
\[T^{x}\left(t\right)=\left(T_{1}^{x}\left(t\right),\ldots,T_{K}^{x}\left(t\right)\right)^{'},\]
\[I^{x}\left(t\right)=\left(I_{1}^{x}\left(t\right),\ldots,I_{K}^{x}\left(t\right)\right)^{'}\]
y
\[S^{x}\left(T^{x}\left(t\right)\right)=\left(S_{1}^{x}\left(T_{1}^{x}\left(t\right)\right),\ldots,S_{K}^{x}\left(T_{K}^{x}\left(t\right)\right)\right)^{'}.\]

Para una disciplina que cumple con la ley de conservaci\'on del
trabajo, en forma vectorial, se tiene el siguiente conjunto de
ecuaciones

\begin{equation}\label{Eq.MF.1.3}
Q^{x}\left(t\right)=Q^{x}\left(0\right)+E^{x}\left(t\right)+\sum_{l=1}^{K}\Phi^{l}\left(S_{l}^{x}\left(T_{l}^{x}\left(t\right)\right)\right)-S^{x}\left(T^{x}\left(t\right)\right),\\
\end{equation}

\begin{equation}\label{Eq.MF.2.3}
Q^{x}\left(t\right)\geq0,\\
\end{equation}

\begin{equation}\label{Eq.MF.3.3}
T^{x}\left(0\right)=0,\textrm{ y }\overline{T}^{x}\left(t\right)\textrm{ es no decreciente},\\
\end{equation}

\begin{equation}\label{Eq.MF.4.3}
I^{x}\left(t\right)=et-CT^{x}\left(t\right)\textrm{ es no
decreciente}\\
\end{equation}

\begin{equation}\label{Eq.MF.5.3}
\int_{0}^{\infty}\left(CQ^{x}\left(t\right)\right)dI_{i}^{x}\left(t\right)=0,\\
\end{equation}

\begin{equation}\label{Eq.MF.6.3}
\textrm{Condiciones adicionales en
}\left(\overline{Q}^{x}\left(\cdot\right),\overline{T}^{x}\left(\cdot\right)\right)\textrm{
espec\'ificas de la disciplina de la cola,}
\end{equation}

donde $e$ es un vector de unos de dimensi\'on $d$, $C$ es la
matriz definida por
\[C_{ik}=\left\{\begin{array}{cc}
1,& S\left(k\right)=i,\\
0,& \textrm{ en otro caso}.\\
\end{array}\right.
\]
Es necesario enunciar el siguiente Teorema que se utilizar\'a para
el Teorema \ref{Tma.4.2.Dai}:
\begin{Teo}[Teorema 4.1, Dai \cite{Dai}]
Considere una disciplina que cumpla la ley de conservaci\'on del
trabajo, para casi todas las trayectorias muestrales $\omega$ y
cualquier sucesi\'on de estados iniciales
$\left\{x_{n}\right\}\subset \mathbf{X}$, con
$|x_{n}|\rightarrow\infty$, existe una subsucesi\'on
$\left\{x_{n_{j}}\right\}$ con $|x_{n_{j}}|\rightarrow\infty$ tal
que
\begin{equation}\label{Eq.4.15}
\frac{1}{|x_{n_{j}}|}\left(Q^{x_{n_{j}}}\left(0\right),U^{x_{n_{j}}}\left(0\right),V^{x_{n_{j}}}\left(0\right)\right)\rightarrow\left(\overline{Q}\left(0\right),\overline{U},\overline{V}\right),
\end{equation}

\begin{equation}\label{Eq.4.16}
\frac{1}{|x_{n_{j}}|}\left(Q^{x_{n_{j}}}\left(|x_{n_{j}}|t\right),T^{x_{n_{j}}}\left(|x_{n_{j}}|t\right)\right)\rightarrow\left(\overline{Q}\left(t\right),\overline{T}\left(t\right)\right)\textrm{
u.o.c.}
\end{equation}

Adem\'as,
$\left(\overline{Q}\left(t\right),\overline{T}\left(t\right)\right)$
satisface las siguientes ecuaciones:
\begin{equation}\label{Eq.MF.1.3a}
\overline{Q}\left(t\right)=Q\left(0\right)+\left(\alpha
t-\overline{U}\right)^{+}-\left(I-P\right)^{'}M^{-1}\left(\overline{T}\left(t\right)-\overline{V}\right)^{+},
\end{equation}

\begin{equation}\label{Eq.MF.2.3a}
\overline{Q}\left(t\right)\geq0,\\
\end{equation}

\begin{equation}\label{Eq.MF.3.3a}
\overline{T}\left(t\right)\textrm{ es no decreciente y comienza en cero},\\
\end{equation}

\begin{equation}\label{Eq.MF.4.3a}
\overline{I}\left(t\right)=et-C\overline{T}\left(t\right)\textrm{
es no decreciente,}\\
\end{equation}

\begin{equation}\label{Eq.MF.5.3a}
\int_{0}^{\infty}\left(C\overline{Q}\left(t\right)\right)d\overline{I}\left(t\right)=0,\\
\end{equation}

\begin{equation}\label{Eq.MF.6.3a}
\textrm{Condiciones adicionales en
}\left(\overline{Q}\left(\cdot\right),\overline{T}\left(\cdot\right)\right)\textrm{
especficas de la disciplina de la cola,}
\end{equation}
\end{Teo}


Propiedades importantes para el modelo de flujo retrasado:

\begin{Prop}
 Sea $\left(\overline{Q},\overline{T},\overline{T}^{0}\right)$ un flujo l\'imite de \ref{Eq.4.4} y suponga que cuando $x\rightarrow\infty$ a lo largo de
una subsucesi\'on
\[\left(\frac{1}{|x|}Q_{k}^{x}\left(0\right),\frac{1}{|x|}A_{k}^{x}\left(0\right),\frac{1}{|x|}B_{k}^{x}\left(0\right),\frac{1}{|x|}B_{k}^{x,0}\left(0\right)\right)\rightarrow\left(\overline{Q}_{k}\left(0\right),0,0,0\right)\]
para $k=1,\ldots,K$. EL flujo l\'imite tiene las siguientes
propiedades, donde las propiedades de la derivada se cumplen donde
la derivada exista:
\begin{itemize}
 \item[i)] Los vectores de tiempo ocupado $\overline{T}\left(t\right)$ y $\overline{T}^{0}\left(t\right)$ son crecientes y continuas con
$\overline{T}\left(0\right)=\overline{T}^{0}\left(0\right)=0$.
\item[ii)] Para todo $t\geq0$
\[\sum_{k=1}^{K}\left[\overline{T}_{k}\left(t\right)+\overline{T}_{k}^{0}\left(t\right)\right]=t\]
\item[iii)] Para todo $1\leq k\leq K$
\[\overline{Q}_{k}\left(t\right)=\overline{Q}_{k}\left(0\right)+\alpha_{k}t-\mu_{k}\overline{T}_{k}\left(t\right)\]
\item[iv)]  Para todo $1\leq k\leq K$
\[\dot{{\overline{T}}}_{k}\left(t\right)=\beta_{k}\] para $\overline{Q}_{k}\left(t\right)=0$.
\item[v)] Para todo $k,j$
\[\mu_{k}^{0}\overline{T}_{k}^{0}\left(t\right)=\mu_{j}^{0}\overline{T}_{j}^{0}\left(t\right)\]
\item[vi)]  Para todo $1\leq k\leq K$
\[\mu_{k}\dot{{\overline{T}}}_{k}\left(t\right)=l_{k}\mu_{k}^{0}\dot{{\overline{T}}}_{k}^{0}\left(t\right)\] para $\overline{Q}_{k}\left(t\right)>0$.
\end{itemize}
\end{Prop}

\begin{Lema}[Lema 3.1 \cite{Chen}]\label{Lema3.1}
Si el modelo de flujo es estable, definido por las ecuaciones
(3.8)-(3.13), entonces el modelo de flujo retrasado tambi\'en es
estable.
\end{Lema}

\begin{Teo}[Teorema 5.1 \cite{Chen}]\label{Tma.5.1.Chen}
La red de colas es estable si existe una constante $t_{0}$ que
depende de $\left(\alpha,\mu,T,U\right)$ y $V$ que satisfagan las
ecuaciones (5.1)-(5.5), $Z\left(t\right)=0$, para toda $t\geq
t_{0}$.
\end{Teo}



\begin{Lema}[Lema 5.2 \cite{Gut}]\label{Lema.5.2.Gut}
Sea $\left\{\xi\left(k\right):k\in\ent\right\}$ sucesi\'on de
variables aleatorias i.i.d. con valores en
$\left(0,\infty\right)$, y sea $E\left(t\right)$ el proceso de
conteo
\[E\left(t\right)=max\left\{n\geq1:\xi\left(1\right)+\cdots+\xi\left(n-1\right)\leq t\right\}.\]
Si $E\left[\xi\left(1\right)\right]<\infty$, entonces para
cualquier entero $r\geq1$
\begin{equation}
lim_{t\rightarrow\infty}\esp\left[\left(\frac{E\left(t\right)}{t}\right)^{r}\right]=\left(\frac{1}{E\left[\xi_{1}\right]}\right)^{r}
\end{equation}
de aqu\'i, bajo estas condiciones
\begin{itemize}
\item[a)] Para cualquier $t>0$,
$sup_{t\geq\delta}\esp\left[\left(\frac{E\left(t\right)}{t}\right)^{r}\right]$

\item[b)] Las variables aleatorias
$\left\{\left(\frac{E\left(t\right)}{t}\right)^{r}:t\geq1\right\}$
son uniformemente integrables.
\end{itemize}
\end{Lema}

\begin{Teo}[Teorema 5.1: Ley Fuerte para Procesos de Conteo
\cite{Gut}]\label{Tma.5.1.Gut} Sea
$0<\mu<\esp\left(X_{1}\right]\leq\infty$. entonces

\begin{itemize}
\item[a)] $\frac{N\left(t\right)}{t}\rightarrow\frac{1}{\mu}$
a.s., cuando $t\rightarrow\infty$.


\item[b)]$\esp\left[\frac{N\left(t\right)}{t}\right]^{r}\rightarrow\frac{1}{\mu^{r}}$,
cuando $t\rightarrow\infty$ para todo $r>0$..
\end{itemize}
\end{Teo}


\begin{Prop}[Proposici\'on 5.1 \cite{DaiSean}]\label{Prop.5.1}
Suponga que los supuestos (A1) y (A2) se cumplen, adem\'as suponga
que el modelo de flujo es estable. Entonces existe $t_{0}>0$ tal
que
\begin{equation}\label{Eq.Prop.5.1}
lim_{|x|\rightarrow\infty}\frac{1}{|x|^{p+1}}\esp_{x}\left[|X\left(t_{0}|x|\right)|^{p+1}\right]=0.
\end{equation}

\end{Prop}


\begin{Prop}[Proposici\'on 5.3 \cite{DaiSean}]
Sea $X$ proceso de estados para la red de colas, y suponga que se
cumplen los supuestos (A1) y (A2), entonces para alguna constante
positiva $C_{p+1}<\infty$, $\delta>0$ y un conjunto compacto
$C\subset X$.

\begin{equation}\label{Eq.5.4}
\esp_{x}\left[\int_{0}^{\tau_{C}\left(\delta\right)}\left(1+|X\left(t\right)|^{p}\right)dt\right]\leq
C_{p+1}\left(1+|x|^{p+1}\right)
\end{equation}
\end{Prop}

\begin{Prop}[Proposici\'on 5.4 \cite{DaiSean}]
Sea $X$ un proceso de Markov Borel Derecho en $X$, sea
$f:X\leftarrow\rea_{+}$ y defina para alguna $\delta>0$, y un
conjunto cerrado $C\subset X$
\[V\left(x\right):=\esp_{x}\left[\int_{0}^{\tau_{C}\left(\delta\right)}f\left(X\left(t\right)\right)dt\right]\]
para $x\in X$. Si $V$ es finito en todas partes y uniformemente
acotada en $C$, entonces existe $k<\infty$ tal que
\begin{equation}\label{Eq.5.11}
\frac{1}{t}\esp_{x}\left[V\left(x\right)\right]+\frac{1}{t}\int_{0}^{t}\esp_{x}\left[f\left(X\left(s\right)\right)ds\right]\leq\frac{1}{t}V\left(x\right)+k,
\end{equation}
para $x\in X$ y $t>0$.
\end{Prop}


\begin{Teo}[Teorema 5.5 \cite{DaiSean}]
Suponga que se cumplen (A1) y (A2), adem\'as suponga que el modelo
de flujo es estable. Entonces existe una constante $k_{p}<\infty$
tal que
\begin{equation}\label{Eq.5.13}
\frac{1}{t}\int_{0}^{t}\esp_{x}\left[|Q\left(s\right)|^{p}\right]ds\leq
k_{p}\left\{\frac{1}{t}|x|^{p+1}+1\right\}
\end{equation}
para $t\geq0$, $x\in X$. En particular para cada condici\'on
inicial
\begin{equation}\label{Eq.5.14}
Limsup_{t\rightarrow\infty}\frac{1}{t}\int_{0}^{t}\esp_{x}\left[|Q\left(s\right)|^{p}\right]ds\leq
k_{p}
\end{equation}
\end{Teo}

\begin{Teo}[Teorema 6.2 \cite{DaiSean}]\label{Tma.6.2}
Suponga que se cumplen los supuestos (A1)-(A3) y que el modelo de
flujo es estable, entonces se tiene que
\[\parallel P^{t}\left(c,\cdot\right)-\pi\left(\cdot\right)\parallel_{f_{p}}\rightarrow0\]
para $t\rightarrow\infty$ y $x\in X$. En particular para cada
condici\'on inicial
\[lim_{t\rightarrow\infty}\esp_{x}\left[\left|Q_{t}\right|^{p}\right]=\esp_{\pi}\left[\left|Q_{0}\right|^{p}\right]<\infty\]
\end{Teo}


\begin{Teo}[Teorema 6.3 \cite{DaiSean}]\label{Tma.6.3}
Suponga que se cumplen los supuestos (A1)-(A3) y que el modelo de
flujo es estable, entonces con
$f\left(x\right)=f_{1}\left(x\right)$, se tiene que
\[lim_{t\rightarrow\infty}t^{(p-1)\left|P^{t}\left(c,\cdot\right)-\pi\left(\cdot\right)\right|_{f}=0},\]
para $x\in X$. En particular, para cada condici\'on inicial
\[lim_{t\rightarrow\infty}t^{(p-1)}\left|\esp_{x}\left[Q_{t}\right]-\esp_{\pi}\left[Q_{0}\right]\right|=0.\]
\end{Teo}



\begin{Prop}[Proposici\'on 5.1, Dai y Meyn \cite{DaiSean}]\label{Prop.5.1.DaiSean}
Suponga que los supuestos A1) y A2) son ciertos y que el modelo de
flujo es estable. Entonces existe $t_{0}>0$ tal que
\begin{equation}
lim_{|x|\rightarrow\infty}\frac{1}{|x|^{p+1}}\esp_{x}\left[|X\left(t_{0}|x|\right)|^{p+1}\right]=0
\end{equation}
\end{Prop}

\begin{Lemma}[Lema 5.2, Dai y Meyn, \cite{DaiSean}]\label{Lema.5.2.DaiSean}
 Sea $\left\{\zeta\left(k\right):k\in \mathbb{z}\right\}$ una sucesi\'on independiente e id\'enticamente distribuida que toma valores en $\left(0,\infty\right)$,
y sea
$E\left(t\right)=max\left(n\geq1:\zeta\left(1\right)+\cdots+\zeta\left(n-1\right)\leq
t\right)$. Si $\esp\left[\zeta\left(1\right)\right]<\infty$,
entonces para cualquier entero $r\geq1$
\begin{equation}
 lim_{t\rightarrow\infty}\esp\left[\left(\frac{E\left(t\right)}{t}\right)^{r}\right]=\left(\frac{1}{\esp\left[\zeta_{1}\right]}\right)^{r}.
\end{equation}
Luego, bajo estas condiciones:
\begin{itemize}
 \item[a)] para cualquier $\delta>0$, $\sup_{t\geq\delta}\esp\left[\left(\frac{E\left(t\right)}{t}\right)^{r}\right]<\infty$
\item[b)] las variables aleatorias
$\left\{\left(\frac{E\left(t\right)}{t}\right)^{r}:t\geq1\right\}$
son uniformemente integrables.
\end{itemize}
\end{Lemma}

\begin{Teo}[Teorema 5.5, Dai y Meyn \cite{DaiSean}]\label{Tma.5.5.DaiSean}
Suponga que los supuestos A1) y A2) se cumplen y que el modelo de
flujo es estable. Entonces existe una constante $\kappa_{p}$ tal
que
\begin{equation}
\frac{1}{t}\int_{0}^{t}\esp_{x}\left[|Q\left(s\right)|^{p}\right]ds\leq\kappa_{p}\left\{\frac{1}{t}|x|^{p+1}+1\right\}
\end{equation}
para $t>0$ y $x\in X$. En particular, para cada condici\'on
inicial
\begin{eqnarray*}
\limsup_{t\rightarrow\infty}\frac{1}{t}\int_{0}^{t}\esp_{x}\left[|Q\left(s\right)|^{p}\right]ds\leq\kappa_{p}.
\end{eqnarray*}
\end{Teo}

\begin{Teo}[Teorema 6.2, Dai y Meyn \cite{DaiSean}]\label{Tma.6.2.DaiSean}
Suponga que se cumplen los supuestos A1), A2) y A3) y que el
modelo de flujo es estable. Entonces se tiene que
\begin{equation}
\left\|P^{t}\left(x,\cdot\right)-\pi\left(\cdot\right)\right\|_{f_{p}}\textrm{,
}t\rightarrow\infty,x\in X.
\end{equation}
En particular para cada condici\'on inicial
\begin{eqnarray*}
\lim_{t\rightarrow\infty}\esp_{x}\left[|Q\left(t\right)|^{p}\right]=\esp_{\pi}\left[|Q\left(0\right)|^{p}\right]\leq\kappa_{r}
\end{eqnarray*}
\end{Teo}
\begin{Teo}[Teorema 6.3, Dai y Meyn \cite{DaiSean}]\label{Tma.6.3.DaiSean}
Suponga que se cumplen los supuestos A1), A2) y A3) y que el
modelo de flujo es estable. Entonces con
$f\left(x\right)=f_{1}\left(x\right)$ se tiene
\begin{equation}
\lim_{t\rightarrow\infty}t^{p-1}\left\|P^{t}\left(x,\cdot\right)-\pi\left(\cdot\right)\right\|_{f}=0.
\end{equation}
En particular para cada condici\'on inicial
\begin{eqnarray*}
\lim_{t\rightarrow\infty}t^{p-1}|\esp_{x}\left[Q\left(t\right)\right]-\esp_{\pi}\left[Q\left(0\right)\right]|=0.
\end{eqnarray*}
\end{Teo}

\begin{Teo}[Teorema 6.4, Dai y Meyn, \cite{DaiSean}]\label{Tma.6.4.DaiSean}
Suponga que se cumplen los supuestos A1), A2) y A3) y que el
modelo de flujo es estable. Sea $\nu$ cualquier distribuci\'on de
probabilidad en $\left(X,\mathcal{B}_{X}\right)$, y $\pi$ la
distribuci\'on estacionaria de $X$.
\begin{itemize}
\item[i)] Para cualquier $f:X\leftarrow\rea_{+}$
\begin{equation}
\lim_{t\rightarrow\infty}\frac{1}{t}\int_{o}^{t}f\left(X\left(s\right)\right)ds=\pi\left(f\right):=\int
f\left(x\right)\pi\left(dx\right)
\end{equation}
$\prob$-c.s.

\item[ii)] Para cualquier $f:X\leftarrow\rea_{+}$ con
$\pi\left(|f|\right)<\infty$, la ecuaci\'on anterior se cumple.
\end{itemize}
\end{Teo}

\begin{Teo}[Teorema 2.2, Down \cite{Down}]\label{Tma2.2.Down}
Suponga que el fluido modelo es inestable en el sentido de que
para alguna $\epsilon_{0},c_{0}\geq0$,
\begin{equation}\label{Eq.Inestability}
|Q\left(T\right)|\geq\epsilon_{0}T-c_{0}\textrm{,   }T\geq0,
\end{equation}
para cualquier condici\'on inicial $Q\left(0\right)$, con
$|Q\left(0\right)|=1$. Entonces para cualquier $0<q\leq1$, existe
$B<0$ tal que para cualquier $|x|\geq B$,
\begin{equation}
\prob_{x}\left\{\mathbb{X}\rightarrow\infty\right\}\geq q.
\end{equation}
\end{Teo}



\begin{Def}
Sea $X$ un conjunto y $\mathcal{F}$ una $\sigma$-\'algebra de
subconjuntos de $X$, la pareja $\left(X,\mathcal{F}\right)$ es
llamado espacio medible. Un subconjunto $A$ de $X$ es llamado
medible, o medible con respecto a $\mathcal{F}$, si
$A\in\mathcal{F}$.
\end{Def}

\begin{Def}
Sea $\left(X,\mathcal{F},\mu\right)$ espacio de medida. Se dice
que la medida $\mu$ es $\sigma$-finita si se puede escribir
$X=\bigcup_{n\geq1}X_{n}$ con $X_{n}\in\mathcal{F}$ y
$\mu\left(X_{n}\right)<\infty$.
\end{Def}

\begin{Def}\label{Cto.Borel}
Sea $X$ el conjunto de los n\'umeros reales $\rea$. El \'algebra
de Borel es la $\sigma$-\'algebra $B$ generada por los intervalos
abiertos $\left(a,b\right)\in\rea$. Cualquier conjunto en $B$ es
llamado {\em Conjunto de Borel}.
\end{Def}

\begin{Def}\label{Funcion.Medible}
Una funci\'on $f:X\rightarrow\rea$, es medible si para cualquier
n\'umero real $\alpha$ el conjunto
\[\left\{x\in X:f\left(x\right)>\alpha\right\}\]
pertenece a $\mathcal{F}$. Equivalentemente, se dice que $f$ es
medible si
\[f^{-1}\left(\left(\alpha,\infty\right)\right)=\left\{x\in X:f\left(x\right)>\alpha\right\}\in\mathcal{F}.\]
\end{Def}


\begin{Def}\label{Def.Cilindros}
Sean $\left(\Omega_{i},\mathcal{F}_{i}\right)$, $i=1,2,\ldots,$
espacios medibles y $\Omega=\prod_{i=1}^{\infty}\Omega_{i}$ el
conjunto de todas las sucesiones
$\left(\omega_{1},\omega_{2},\ldots,\right)$ tales que
$\omega_{i}\in\Omega_{i}$, $i=1,2,\ldots,$. Si
$B^{n}\subset\prod_{i=1}^{\infty}\Omega_{i}$, definimos
$B_{n}=\left\{\omega\in\Omega:\left(\omega_{1},\omega_{2},\ldots,\omega_{n}\right)\in
B^{n}\right\}$. Al conjunto $B_{n}$ se le llama {\em cilindro} con
base $B^{n}$, el cilindro es llamado medible si
$B^{n}\in\prod_{i=1}^{\infty}\mathcal{F}_{i}$.
\end{Def}


\begin{Def}\label{Def.Proc.Adaptado}[TSP, Ash \cite{RBA}]
Sea $X\left(t\right),t\geq0$ proceso estoc\'astico, el proceso es
adaptado a la familia de $\sigma$-\'algebras $\mathcal{F}_{t}$,
para $t\geq0$, si para $s<t$ implica que
$\mathcal{F}_{s}\subset\mathcal{F}_{t}$, y $X\left(t\right)$ es
$\mathcal{F}_{t}$-medible para cada $t$. Si no se especifica
$\mathcal{F}_{t}$ entonces se toma $\mathcal{F}_{t}$ como
$\mathcal{F}\left(X\left(s\right),s\leq t\right)$, la m\'as
peque\~na $\sigma$-\'algebra de subconjuntos de $\Omega$ que hace
que cada $X\left(s\right)$, con $s\leq t$ sea Borel medible.
\end{Def}


\begin{Def}\label{Def.Tiempo.Paro}[TSP, Ash \cite{RBA}]
Sea $\left\{\mathcal{F}\left(t\right),t\geq0\right\}$ familia
creciente de sub $\sigma$-\'algebras. es decir,
$\mathcal{F}\left(s\right)\subset\mathcal{F}\left(t\right)$ para
$s\leq t$. Un tiempo de paro para $\mathcal{F}\left(t\right)$ es
una funci\'on $T:\Omega\rightarrow\left[0,\infty\right]$ tal que
$\left\{T\leq t\right\}\in\mathcal{F}\left(t\right)$ para cada
$t\geq0$. Un tiempo de paro para el proceso estoc\'astico
$X\left(t\right),t\geq0$ es un tiempo de paro para las
$\sigma$-\'algebras
$\mathcal{F}\left(t\right)=\mathcal{F}\left(X\left(s\right)\right)$.
\end{Def}

\begin{Def}
Sea $X\left(t\right),t\geq0$ proceso estoc\'astico, con
$\left(S,\chi\right)$ espacio de estados. Se dice que el proceso
es adaptado a $\left\{\mathcal{F}\left(t\right)\right\}$, es
decir, si para cualquier $s,t\in I$, $I$ conjunto de \'indices,
$s<t$, se tiene que
$\mathcal{F}\left(s\right)\subset\mathcal{F}\left(t\right)$ y
$X\left(t\right)$ es $\mathcal{F}\left(t\right)$-medible,
\end{Def}

\begin{Def}
Sea $X\left(t\right),t\geq0$ proceso estoc\'astico, se dice que es
un Proceso de Markov relativo a $\mathcal{F}\left(t\right)$ o que
$\left\{X\left(t\right),\mathcal{F}\left(t\right)\right\}$ es de
Markov si y s\'olo si para cualquier conjunto $B\in\chi$,  y
$s,t\in I$, $s<t$ se cumple que
\begin{equation}\label{Prop.Markov}
P\left\{X\left(t\right)\in
B|\mathcal{F}\left(s\right)\right\}=P\left\{X\left(t\right)\in
B|X\left(s\right)\right\}.
\end{equation}
\end{Def}
\begin{Note}
Si se dice que $\left\{X\left(t\right)\right\}$ es un Proceso de
Markov sin mencionar $\mathcal{F}\left(t\right)$, se asumir\'a que
\begin{eqnarray*}
\mathcal{F}\left(t\right)=\mathcal{F}_{0}\left(t\right)=\mathcal{F}\left(X\left(r\right),r\leq
t\right),
\end{eqnarray*}
entonces la ecuaci\'on (\ref{Prop.Markov}) se puede escribir como
\begin{equation}
P\left\{X\left(t\right)\in B|X\left(r\right),r\leq s\right\} =
P\left\{X\left(t\right)\in B|X\left(s\right)\right\}
\end{equation}
\end{Note}
%_______________________________________________________________
\subsection{Procesos de Estados de Markov}
%_______________________________________________________________

\begin{Teo}
Sea $\left(X_{n},\mathcal{F}_{n},n=0,1,\ldots,\right\}$ Proceso de
Markov con espacio de estados $\left(S_{0},\chi_{0}\right)$
generado por una distribuici\'on inicial $P_{o}$ y probabilidad de
transici\'on $p_{mn}$, para $m,n=0,1,\ldots,$ $m<n$, que por
notaci\'on se escribir\'a como $p\left(m,n,x,B\right)\rightarrow
p_{mn}\left(x,B\right)$. Sea $S$ tiempo de paro relativo a la
$\sigma$-\'algebra $\mathcal{F}_{n}$. Sea $T$ funci\'on medible,
$T:\Omega\rightarrow\left\{0,1,\ldots,\right\}$. Sup\'ongase que
$T\geq S$, entonces $T$ es tiempo de paro. Si $B\in\chi_{0}$,
entonces
\begin{equation}\label{Prop.Fuerte.Markov}
P\left\{X\left(T\right)\in
B,T<\infty|\mathcal{F}\left(S\right)\right\} =
p\left(S,T,X\left(s\right),B\right)
\end{equation}
en $\left\{T<\infty\right\}$.
\end{Teo}


Sea $K$ conjunto numerable y sea $d:K\rightarrow\nat$ funci\'on.
Para $v\in K$, $M_{v}$ es un conjunto abierto de
$\rea^{d\left(v\right)}$. Entonces \[E=\bigcup_{v\in
K}M_{v}=\left\{\left(v,\zeta\right):v\in K,\zeta\in
M_{v}\right\}.\]

Sea $\mathcal{E}$ la clase de conjuntos medibles en $E$:
\[\mathcal{E}=\left\{\bigcup_{v\in K}A_{v}:A_{v}\in \mathcal{M}_{v}\right\}.\]

donde $\mathcal{M}$ son los conjuntos de Borel de $M_{v}$.
Entonces $\left(E,\mathcal{E}\right)$ es un espacio de Borel. El
estado del proceso se denotar\'a por
$\mathbf{x}_{t}=\left(v_{t},\zeta_{t}\right)$. La distribuci\'on
de $\left(\mathbf{x}_{t}\right)$ est\'a determinada por por los
siguientes objetos:

\begin{itemize}
\item[i)] Los campos vectoriales $\left(\mathcal{H}_{v},v\in
K\right)$. \item[ii)] Una funci\'on medible $\lambda:E\rightarrow
\rea_{+}$. \item[iii)] Una medida de transici\'on
$Q:\mathcal{E}\times\left(E\cup\Gamma^{*}\right)\rightarrow\left[0,1\right]$
donde
\begin{equation}
\Gamma^{*}=\bigcup_{v\in K}\partial^{*}M_{v}.
\end{equation}
y
\begin{equation}
\partial^{*}M_{v}=\left\{z\in\partial M_{v}:\mathbf{\mathbf{\phi}_{v}\left(t,\zeta\right)=\mathbf{z}}\textrm{ para alguna }\left(t,\zeta\right)\in\rea_{+}\times M_{v}\right\}.
\end{equation}
$\partial M_{v}$ denota  la frontera de $M_{v}$.
\end{itemize}

El campo vectorial $\left(\mathcal{H}_{v},v\in K\right)$ se supone
tal que para cada $\mathbf{z}\in M_{v}$ existe una \'unica curva
integral $\mathbf{\phi}_{v}\left(t,\zeta\right)$ que satisface la
ecuaci\'on

\begin{equation}
\frac{d}{dt}f\left(\zeta_{t}\right)=\mathcal{H}f\left(\zeta_{t}\right),
\end{equation}
con $\zeta_{0}=\mathbf{z}$, para cualquier funci\'on suave
$f:\rea^{d}\rightarrow\rea$ y $\mathcal{H}$ denota el operador
diferencial de primer orden, con $\mathcal{H}=\mathcal{H}_{v}$ y
$\zeta_{t}=\mathbf{\phi}\left(t,\mathbf{z}\right)$. Adem\'as se
supone que $\mathcal{H}_{v}$ es conservativo, es decir, las curvas
integrales est\'an definidas para todo $t>0$.

Para $\mathbf{x}=\left(v,\zeta\right)\in E$ se denota
\[t^{*}\mathbf{x}=inf\left\{t>0:\mathbf{\phi}_{v}\left(t,\zeta\right)\in\partial^{*}M_{v}\right\}\]

En lo que respecta a la funci\'on $\lambda$, se supondr\'a que
para cada $\left(v,\zeta\right)\in E$ existe un $\epsilon>0$ tal
que la funci\'on
$s\rightarrow\lambda\left(v,\phi_{v}\left(s,\zeta\right)\right)\in
E$ es integrable para $s\in\left[0,\epsilon\right)$. La medida de
transici\'on $Q\left(A;\mathbf{x}\right)$ es una funci\'on medible
de $\mathbf{x}$ para cada $A\in\mathcal{E}$, definida para
$\mathbf{x}\in E\cup\Gamma^{*}$ y es una medida de probabilidad en
$\left(E,\mathcal{E}\right)$ para cada $\mathbf{x}\in E$.

El movimiento del proceso $\left(\mathbf{x}_{t}\right)$ comenzando
en $\mathbf{x}=\left(n,\mathbf{z}\right)\in E$ se puede construir
de la siguiente manera, def\'inase la funci\'on $F$ por

\begin{equation}
F\left(t\right)=\left\{\begin{array}{ll}\\
exp\left(-\int_{0}^{t}\lambda\left(n,\phi_{n}\left(s,\mathbf{z}\right)\right)ds\right), & t<t^{*}\left(\mathbf{x}\right),\\
0, & t\geq t^{*}\left(\mathbf{x}\right)
\end{array}\right.
\end{equation}

Sea $T_{1}$ una variable aleatoria tal que
$\prob\left[T_{1}>t\right]=F\left(t\right)$, ahora sea la variable
aleatoria $\left(N,Z\right)$ con distribuici\'on
$Q\left(\cdot;\phi_{n}\left(T_{1},\mathbf{z}\right)\right)$. La
trayectoria de $\left(\mathbf{x}_{t}\right)$ para $t\leq T_{1}$ es
\begin{eqnarray*}
\mathbf{x}_{t}=\left(v_{t},\zeta_{t}\right)=\left\{\begin{array}{ll}
\left(n,\phi_{n}\left(t,\mathbf{z}\right)\right), & t<T_{1},\\
\left(N,\mathbf{Z}\right), & t=t_{1}.
\end{array}\right.
\end{eqnarray*}

Comenzando en $\mathbf{x}_{T_{1}}$ se selecciona el siguiente
tiempo de intersalto $T_{2}-T_{1}$ lugar del post-salto
$\mathbf{x}_{T_{2}}$ de manera similar y as\'i sucesivamente. Este
procedimiento nos da una trayectoria determinista por partes
$\mathbf{x}_{t}$ con tiempos de salto $T_{1},T_{2},\ldots$. Bajo
las condiciones enunciadas para $\lambda,T_{1}>0$  y
$T_{1}-T_{2}>0$ para cada $i$, con probabilidad 1. Se supone que
se cumple la siguiente condici\'on.

\begin{Sup}[Supuesto 3.1, Davis \cite{Davis}]\label{Sup3.1.Davis}
Sea $N_{t}:=\sum_{t}\indora_{\left(t\geq t\right)}$ el n\'umero de
saltos en $\left[0,t\right]$. Entonces
\begin{equation}
\esp\left[N_{t}\right]<\infty\textrm{ para toda }t.
\end{equation}
\end{Sup}

es un proceso de Markov, m\'as a\'un, es un Proceso Fuerte de
Markov, es decir, la Propiedad Fuerte de Markov\footnote{Revisar
p\'agina 362, y 364 de Davis \cite{Davis}.} se cumple para
cualquier tiempo de paro.
%_________________________________________________________________________
%\renewcommand{\refname}{PROCESOS ESTOC\'ASTICOS}
%\renewcommand{\appendixname}{PROCESOS ESTOC\'ASTICOS}
%\renewcommand{\appendixtocname}{PROCESOS ESTOC\'ASTICOS}
%\renewcommand{\appendixpagename}{PROCESOS ESTOC\'ASTICOS}
%\appendix
%\clearpage % o \cleardoublepage
%\addappheadtotoc
%\appendixpage
%_________________________________________________________________________
\subsection{Teor\'ia General de Procesos Estoc\'asticos}
%_________________________________________________________________________
En esta secci\'on se har\'an las siguientes consideraciones: $E$
es un espacio m\'etrico separable y la m\'etrica $d$ es compatible
con la topolog\'ia.

\begin{Def}
Una medida finita, $\lambda$ en la $\sigma$-\'algebra de Borel de
un espacio metrizable $E$ se dice cerrada si
\begin{equation}\label{Eq.A2.3}
\lambda\left(E\right)=sup\left\{\lambda\left(K\right):K\textrm{ es
compacto en }E\right\}.
\end{equation}
\end{Def}

\begin{Def}
$E$ es un espacio de Rad\'on si cada medida finita en
$\left(E,\mathcal{B}\left(E\right)\right)$ es regular interior o cerrada,
{\em tight}.
\end{Def}


El siguiente teorema nos permite tener una mejor caracterizaci\'on de los espacios de Rad\'on:
\begin{Teo}\label{Tma.A2.2}
Sea $E$ espacio separable metrizable. Entonces $E$ es de Rad\'on
si y s\'olo s\'i cada medida finita en
$\left(E,\mathcal{B}\left(E\right)\right)$ es cerrada.
\end{Teo}

%_________________________________________________________________________________________
\subsection{Propiedades de Markov}
%_________________________________________________________________________________________

Sea $E$ espacio de estados, tal que $E$ es un espacio de Rad\'on, $\mathcal{B}\left(E\right)$ $\sigma$-\'algebra de Borel en $E$, que se denotar\'a por $\mathcal{E}$.

Sea $\left(X,\mathcal{G},\prob\right)$ espacio de probabilidad,
$I\subset\rea$ conjunto de índices. Sea $\mathcal{F}_{\leq t}$ la
$\sigma$-\'algebra natural definida como
$\sigma\left\{f\left(X_{r}\right):r\in I, r\leq
t,f\in\mathcal{E}\right\}$. Se considerar\'a una
$\sigma$-\'algebra m\'as general\footnote{qu\'e se quiere decir
con el t\'ermino: m\'as general?}, $ \left(\mathcal{G}_{t}\right)$
tal que $\left(X_{t}\right)$ sea $\mathcal{E}$-adaptado.

\begin{Def}
Una familia $\left(P_{s,t}\right)$ de kernels de Markov en $\left(E,\mathcal{E}\right)$ indexada por pares $s,t\in I$, con $s\leq t$ es una funci\'on de transici\'on en $\ER$, si  para todo $r\leq s< t$ en $I$ y todo $x\in E$, $B\in\mathcal{E}$
\begin{equation}\label{Eq.Kernels}
P_{r,t}\left(x,B\right)=\int_{E}P_{r,s}\left(x,dy\right)P_{s,t}\left(y,B\right)\footnote{Ecuaci\'on de Chapman-Kolmogorov}.
\end{equation}
\end{Def}

Se dice que la funci\'on de transici\'on $\KM$ en $\ER$ es la funci\'on de transici\'on para un proceso $\PE$  con valores en $E$ y que satisface la propiedad de Markov\footnote{\begin{equation}\label{Eq.1.4.S}
\prob\left\{H|\mathcal{G}_{t}\right\}=\prob\left\{H|X_{t}\right\}\textrm{ }H\in p\mathcal{F}_{\geq t}.
\end{equation}} (\ref{Eq.1.4.S}) relativa a $\left(\mathcal{G}_{t}\right)$ si

\begin{equation}\label{Eq.1.6.S}
\prob\left\{f\left(X_{t}\right)|\mathcal{G}_{s}\right\}=P_{s,t}f\left(X_{t}\right)\textrm{ }s\leq t\in I,\textrm{ }f\in b\mathcal{E}.
\end{equation}

\begin{Def}
Una familia $\left(P_{t}\right)_{t\geq0}$ de kernels de Markov en $\ER$ es llamada {\em Semigrupo de Transici\'on de Markov} o {\em Semigrupo de Transici\'on} si
\[P_{t+s}f\left(x\right)=P_{t}\left(P_{s}f\right)\left(x\right),\textrm{ }t,s\geq0,\textrm{ }x\in E\textrm{ }f\in b\mathcal{E}\footnote{Definir los t\'ermino $b\mathcal{E}$ y $p\mathcal{E}$}.\]
\end{Def}
\begin{Note}
Si la funci\'on de transici\'on $\KM$ es llamada homog\'enea si $P_{s,t}=P_{t-s}$.
\end{Note}

Un proceso de Markov que satisface la ecuaci\'on (\ref{Eq.1.6.S}) con funci\'on de transici\'on homog\'enea $\left(P_{t}\right)$ tiene la propiedad caracter\'istica
\begin{equation}\label{Eq.1.8.S}
\prob\left\{f\left(X_{t+s}\right)|\mathcal{G}_{t}\right\}=P_{s}f\left(X_{t}\right)\textrm{ }t,s\geq0,\textrm{ }f\in b\mathcal{E}.
\end{equation}
La ecuaci\'on anterior es la {\em Propiedad Simple de Markov} de $X$ relativa a $\left(P_{t}\right)$.

En este sentido el proceso $\PE$ cumple con la propiedad de Markov (\ref{Eq.1.8.S}) relativa a $\left(\Omega,\mathcal{G},\mathcal{G}_{t},\prob\right)$ con semigrupo de transici\'on $\left(P_{t}\right)$.
%_________________________________________________________________________________________
\subsection{Primer Condici\'on de Regularidad}
%_________________________________________________________________________________________
%\newcommand{\EM}{\left(\Omega,\mathcal{G},\prob\right)}
%\newcommand{\E4}{\left(\Omega,\mathcal{G},\mathcal{G}_{t},\prob\right)}
\begin{Def}
Un proceso estoc\'astico $\PE$ definido en
$\left(\Omega,\mathcal{G},\prob\right)$ con valores en el espacio
topol\'ogico $E$ es continuo por la derecha si cada trayectoria
muestral $t\rightarrow X_{t}\left(w\right)$ es un mapeo continuo
por la derecha de $I$ en $E$.
\end{Def}

\begin{Def}[HD1]\label{Eq.2.1.S}
Un semigrupo de Markov $\left(P_{t}\right)$ en un espacio de
Rad\'on $E$ se dice que satisface la condici\'on {\em HD1} si,
dada una medida de probabilidad $\mu$ en $E$, existe una
$\sigma$-\'algebra $\mathcal{E^{*}}$ con
$\mathcal{E}\subset\mathcal{E}^{*}$ y
$P_{t}\left(b\mathcal{E}^{*}\right)\subset b\mathcal{E}^{*}$, y un
$\mathcal{E}^{*}$-proceso $E$-valuado continuo por la derecha
$\PE$ en alg\'un espacio de probabilidad filtrado
$\left(\Omega,\mathcal{G},\mathcal{G}_{t},\prob\right)$ tal que
$X=\left(\Omega,\mathcal{G},\mathcal{G}_{t},\prob\right)$ es de
Markov (Homog\'eneo) con semigrupo de transici\'on $(P_{t})$ y
distribuci\'on inicial $\mu$.
\end{Def}

Consid\'erese la colecci\'on de variables aleatorias $X_{t}$
definidas en alg\'un espacio de probabilidad, y una colecci\'on de
medidas $\mathbf{P}^{x}$ tales que
$\mathbf{P}^{x}\left\{X_{0}=x\right\}$, y bajo cualquier
$\mathbf{P}^{x}$, $X_{t}$ es de Markov con semigrupo
$\left(P_{t}\right)$. $\mathbf{P}^{x}$ puede considerarse como la
distribuci\'on condicional de $\mathbf{P}$ dado $X_{0}=x$.

\begin{Def}\label{Def.2.2.S}
Sea $E$ espacio de Rad\'on, $\SG$ semigrupo de Markov en $\ER$. La colecci\'on $\mathbf{X}=\left(\Omega,\mathcal{G},\mathcal{G}_{t},X_{t},\theta_{t},\CM\right)$ es un proceso $\mathcal{E}$-Markov continuo por la derecha simple, con espacio de estados $E$ y semigrupo de transici\'on $\SG$ en caso de que $\mathbf{X}$ satisfaga las siguientes condiciones:
\begin{itemize}
\item[i)] $\left(\Omega,\mathcal{G},\mathcal{G}_{t}\right)$ es un espacio de medida filtrado, y $X_{t}$ es un proceso $E$-valuado continuo por la derecha $\mathcal{E}^{*}$-adaptado a $\left(\mathcal{G}_{t}\right)$;

\item[ii)] $\left(\theta_{t}\right)_{t\geq0}$ es una colecci\'on de operadores {\em shift} para $X$, es decir, mapea $\Omega$ en s\'i mismo satisfaciendo para $t,s\geq0$,

\begin{equation}\label{Eq.Shift}
\theta_{t}\circ\theta_{s}=\theta_{t+s}\textrm{ y }X_{t}\circ\theta_{t}=X_{t+s};
\end{equation}

\item[iii)] Para cualquier $x\in E$,$\CM\left\{X_{0}=x\right\}=1$, y el proceso $\PE$ tiene la propiedad de Markov (\ref{Eq.1.8.S}) con semigrupo de transici\'on $\SG$ relativo a $\left(\Omega,\mathcal{G},\mathcal{G}_{t},\CM\right)$.
\end{itemize}
\end{Def}

\begin{Def}[HD2]\label{Eq.2.2.S}
Para cualquier $\alpha>0$ y cualquier $f\in S^{\alpha}$, el proceso $t\rightarrow f\left(X_{t}\right)$ es continuo por la derecha casi seguramente.
\end{Def}

\begin{Def}\label{Def.PD}
Un sistema $\mathbf{X}=\left(\Omega,\mathcal{G},\mathcal{G}_{t},X_{t},\theta_{t},\CM\right)$ es un proceso derecho en el espacio de Rad\'on $E$ con semigrupo de transici\'on $\SG$ provisto de:
\begin{itemize}
\item[i)] $\mathbf{X}$ es una realizaci\'on  continua por la derecha, \ref{Def.2.2.S}, de $\SG$.

\item[ii)] $\mathbf{X}$ satisface la condicion HD2, \ref{Eq.2.2.S}, relativa a $\mathcal{G}_{t}$.

\item[iii)] $\mathcal{G}_{t}$ es aumentado y continuo por la derecha.
\end{itemize}
\end{Def}


%_________________________________________________________________________
%\renewcommand{\refname}{MODELO DE FLUJO}
%\renewcommand{\appendixname}{MODELO DE FLUJO}
%\renewcommand{\appendixtocname}{MODELO DE FLUJO}
%\renewcommand{\appendixpagename}{MODELO DE FLUJO}
%\appendix
%\clearpage % o \cleardoublepage
%\addappheadtotoc
%\appendixpage

\subsection{Construcci\'on del Modelo de Flujo}


\begin{Lema}[Lema 4.2, Dai\cite{Dai}]\label{Lema4.2}
Sea $\left\{x_{n}\right\}\subset \mathbf{X}$ con
$|x_{n}|\rightarrow\infty$, conforme $n\rightarrow\infty$. Suponga
que
\[lim_{n\rightarrow\infty}\frac{1}{|x_{n}|}U\left(0\right)=\overline{U}\]
y
\[lim_{n\rightarrow\infty}\frac{1}{|x_{n}|}V\left(0\right)=\overline{V}.\]

Entonces, conforme $n\rightarrow\infty$, casi seguramente

\begin{equation}\label{E1.4.2}
\frac{1}{|x_{n}|}\Phi^{k}\left(\left[|x_{n}|t\right]\right)\rightarrow
P_{k}^{'}t\textrm{, u.o.c.,}
\end{equation}

\begin{equation}\label{E1.4.3}
\frac{1}{|x_{n}|}E^{x_{n}}_{k}\left(|x_{n}|t\right)\rightarrow
\alpha_{k}\left(t-\overline{U}_{k}\right)^{+}\textrm{, u.o.c.,}
\end{equation}

\begin{equation}\label{E1.4.4}
\frac{1}{|x_{n}|}S^{x_{n}}_{k}\left(|x_{n}|t\right)\rightarrow
\mu_{k}\left(t-\overline{V}_{k}\right)^{+}\textrm{, u.o.c.,}
\end{equation}

donde $\left[t\right]$ es la parte entera de $t$ y
$\mu_{k}=1/m_{k}=1/\esp\left[\eta_{k}\left(1\right)\right]$.
\end{Lema}

\begin{Lema}[Lema 4.3, Dai\cite{Dai}]\label{Lema.4.3}
Sea $\left\{x_{n}\right\}\subset \mathbf{X}$ con
$|x_{n}|\rightarrow\infty$, conforme $n\rightarrow\infty$. Suponga
que
\[lim_{n\rightarrow\infty}\frac{1}{|x_{n}|}U_{k}\left(0\right)=\overline{U}_{k}\]
y
\[lim_{n\rightarrow\infty}\frac{1}{|x_{n}|}V_{k}\left(0\right)=\overline{V}_{k}.\]
\begin{itemize}
\item[a)] Conforme $n\rightarrow\infty$ casi seguramente,
\[lim_{n\rightarrow\infty}\frac{1}{|x_{n}|}U^{x_{n}}_{k}\left(|x_{n}|t\right)=\left(\overline{U}_{k}-t\right)^{+}\textrm{, u.o.c.}\]
y
\[lim_{n\rightarrow\infty}\frac{1}{|x_{n}|}V^{x_{n}}_{k}\left(|x_{n}|t\right)=\left(\overline{V}_{k}-t\right)^{+}.\]

\item[b)] Para cada $t\geq0$ fijo,
\[\left\{\frac{1}{|x_{n}|}U^{x_{n}}_{k}\left(|x_{n}|t\right),|x_{n}|\geq1\right\}\]
y
\[\left\{\frac{1}{|x_{n}|}V^{x_{n}}_{k}\left(|x_{n}|t\right),|x_{n}|\geq1\right\}\]
\end{itemize}
son uniformemente convergentes.
\end{Lema}

Sea $S_{l}^{x}\left(t\right)$ el n\'umero total de servicios
completados de la clase $l$, si la clase $l$ est\'a dando $t$
unidades de tiempo de servicio. Sea $T_{l}^{x}\left(x\right)$ el
monto acumulado del tiempo de servicio que el servidor
$s\left(l\right)$ gasta en los usuarios de la clase $l$ al tiempo
$t$. Entonces $S_{l}^{x}\left(T_{l}^{x}\left(t\right)\right)$ es
el n\'umero total de servicios completados para la clase $l$ al
tiempo $t$. Una fracci\'on de estos usuarios,
$\Phi_{k}^{x}\left(S_{l}^{x}\left(T_{l}^{x}\left(t\right)\right)\right)$,
se convierte en usuarios de la clase $k$.\\

Entonces, dado lo anterior, se tiene la siguiente representaci\'on
para el proceso de la longitud de la cola:\\

\begin{equation}
Q_{k}^{x}\left(t\right)=Q_{k}^{x}\left(0\right)+E_{k}^{x}\left(t\right)+\sum_{l=1}^{K}\Phi_{k}^{l}\left(S_{l}^{x}\left(T_{l}^{x}\left(t\right)\right)\right)-S_{k}^{x}\left(T_{k}^{x}\left(t\right)\right)
\end{equation}
para $k=1,\ldots,K$. Para $i=1,\ldots,d$, sea
\[I_{i}^{x}\left(t\right)=t-\sum_{j\in C_{i}}T_{k}^{x}\left(t\right).\]

Entonces $I_{i}^{x}\left(t\right)$ es el monto acumulado del
tiempo que el servidor $i$ ha estado desocupado al tiempo $t$. Se
est\'a asumiendo que las disciplinas satisfacen la ley de
conservaci\'on del trabajo, es decir, el servidor $i$ est\'a en
pausa solamente cuando no hay usuarios en la estaci\'on $i$.
Entonces, se tiene que

\begin{equation}
\int_{0}^{\infty}\left(\sum_{k\in
C_{i}}Q_{k}^{x}\left(t\right)\right)dI_{i}^{x}\left(t\right)=0,
\end{equation}
para $i=1,\ldots,d$.\\

Hacer
\[T^{x}\left(t\right)=\left(T_{1}^{x}\left(t\right),\ldots,T_{K}^{x}\left(t\right)\right)^{'},\]
\[I^{x}\left(t\right)=\left(I_{1}^{x}\left(t\right),\ldots,I_{K}^{x}\left(t\right)\right)^{'}\]
y
\[S^{x}\left(T^{x}\left(t\right)\right)=\left(S_{1}^{x}\left(T_{1}^{x}\left(t\right)\right),\ldots,S_{K}^{x}\left(T_{K}^{x}\left(t\right)\right)\right)^{'}.\]

Para una disciplina que cumple con la ley de conservaci\'on del
trabajo, en forma vectorial, se tiene el siguiente conjunto de
ecuaciones

\begin{equation}\label{Eq.MF.1.3}
Q^{x}\left(t\right)=Q^{x}\left(0\right)+E^{x}\left(t\right)+\sum_{l=1}^{K}\Phi^{l}\left(S_{l}^{x}\left(T_{l}^{x}\left(t\right)\right)\right)-S^{x}\left(T^{x}\left(t\right)\right),\\
\end{equation}

\begin{equation}\label{Eq.MF.2.3}
Q^{x}\left(t\right)\geq0,\\
\end{equation}

\begin{equation}\label{Eq.MF.3.3}
T^{x}\left(0\right)=0,\textrm{ y }\overline{T}^{x}\left(t\right)\textrm{ es no decreciente},\\
\end{equation}

\begin{equation}\label{Eq.MF.4.3}
I^{x}\left(t\right)=et-CT^{x}\left(t\right)\textrm{ es no
decreciente}\\
\end{equation}

\begin{equation}\label{Eq.MF.5.3}
\int_{0}^{\infty}\left(CQ^{x}\left(t\right)\right)dI_{i}^{x}\left(t\right)=0,\\
\end{equation}

\begin{equation}\label{Eq.MF.6.3}
\textrm{Condiciones adicionales en
}\left(\overline{Q}^{x}\left(\cdot\right),\overline{T}^{x}\left(\cdot\right)\right)\textrm{
espec\'ificas de la disciplina de la cola,}
\end{equation}

donde $e$ es un vector de unos de dimensi\'on $d$, $C$ es la
matriz definida por
\[C_{ik}=\left\{\begin{array}{cc}
1,& S\left(k\right)=i,\\
0,& \textrm{ en otro caso}.\\
\end{array}\right.
\]
Es necesario enunciar el siguiente Teorema que se utilizar\'a para
el Teorema \ref{Tma.4.2.Dai}:
\begin{Teo}[Teorema 4.1, Dai \cite{Dai}]
Considere una disciplina que cumpla la ley de conservaci\'on del
trabajo, para casi todas las trayectorias muestrales $\omega$ y
cualquier sucesi\'on de estados iniciales
$\left\{x_{n}\right\}\subset \mathbf{X}$, con
$|x_{n}|\rightarrow\infty$, existe una subsucesi\'on
$\left\{x_{n_{j}}\right\}$ con $|x_{n_{j}}|\rightarrow\infty$ tal
que
\begin{equation}\label{Eq.4.15}
\frac{1}{|x_{n_{j}}|}\left(Q^{x_{n_{j}}}\left(0\right),U^{x_{n_{j}}}\left(0\right),V^{x_{n_{j}}}\left(0\right)\right)\rightarrow\left(\overline{Q}\left(0\right),\overline{U},\overline{V}\right),
\end{equation}

\begin{equation}\label{Eq.4.16}
\frac{1}{|x_{n_{j}}|}\left(Q^{x_{n_{j}}}\left(|x_{n_{j}}|t\right),T^{x_{n_{j}}}\left(|x_{n_{j}}|t\right)\right)\rightarrow\left(\overline{Q}\left(t\right),\overline{T}\left(t\right)\right)\textrm{
u.o.c.}
\end{equation}

Adem\'as,
$\left(\overline{Q}\left(t\right),\overline{T}\left(t\right)\right)$
satisface las siguientes ecuaciones:
\begin{equation}\label{Eq.MF.1.3a}
\overline{Q}\left(t\right)=Q\left(0\right)+\left(\alpha
t-\overline{U}\right)^{+}-\left(I-P\right)^{'}M^{-1}\left(\overline{T}\left(t\right)-\overline{V}\right)^{+},
\end{equation}

\begin{equation}\label{Eq.MF.2.3a}
\overline{Q}\left(t\right)\geq0,\\
\end{equation}

\begin{equation}\label{Eq.MF.3.3a}
\overline{T}\left(t\right)\textrm{ es no decreciente y comienza en cero},\\
\end{equation}

\begin{equation}\label{Eq.MF.4.3a}
\overline{I}\left(t\right)=et-C\overline{T}\left(t\right)\textrm{
es no decreciente,}\\
\end{equation}

\begin{equation}\label{Eq.MF.5.3a}
\int_{0}^{\infty}\left(C\overline{Q}\left(t\right)\right)d\overline{I}\left(t\right)=0,\\
\end{equation}

\begin{equation}\label{Eq.MF.6.3a}
\textrm{Condiciones adicionales en
}\left(\overline{Q}\left(\cdot\right),\overline{T}\left(\cdot\right)\right)\textrm{
especficas de la disciplina de la cola,}
\end{equation}
\end{Teo}


Propiedades importantes para el modelo de flujo retrasado:

\begin{Prop}
 Sea $\left(\overline{Q},\overline{T},\overline{T}^{0}\right)$ un flujo l\'imite de \ref{Eq.4.4} y suponga que cuando $x\rightarrow\infty$ a lo largo de
una subsucesi\'on
\[\left(\frac{1}{|x|}Q_{k}^{x}\left(0\right),\frac{1}{|x|}A_{k}^{x}\left(0\right),\frac{1}{|x|}B_{k}^{x}\left(0\right),\frac{1}{|x|}B_{k}^{x,0}\left(0\right)\right)\rightarrow\left(\overline{Q}_{k}\left(0\right),0,0,0\right)\]
para $k=1,\ldots,K$. EL flujo l\'imite tiene las siguientes
propiedades, donde las propiedades de la derivada se cumplen donde
la derivada exista:
\begin{itemize}
 \item[i)] Los vectores de tiempo ocupado $\overline{T}\left(t\right)$ y $\overline{T}^{0}\left(t\right)$ son crecientes y continuas con
$\overline{T}\left(0\right)=\overline{T}^{0}\left(0\right)=0$.
\item[ii)] Para todo $t\geq0$
\[\sum_{k=1}^{K}\left[\overline{T}_{k}\left(t\right)+\overline{T}_{k}^{0}\left(t\right)\right]=t\]
\item[iii)] Para todo $1\leq k\leq K$
\[\overline{Q}_{k}\left(t\right)=\overline{Q}_{k}\left(0\right)+\alpha_{k}t-\mu_{k}\overline{T}_{k}\left(t\right)\]
\item[iv)]  Para todo $1\leq k\leq K$
\[\dot{{\overline{T}}}_{k}\left(t\right)=\beta_{k}\] para $\overline{Q}_{k}\left(t\right)=0$.
\item[v)] Para todo $k,j$
\[\mu_{k}^{0}\overline{T}_{k}^{0}\left(t\right)=\mu_{j}^{0}\overline{T}_{j}^{0}\left(t\right)\]
\item[vi)]  Para todo $1\leq k\leq K$
\[\mu_{k}\dot{{\overline{T}}}_{k}\left(t\right)=l_{k}\mu_{k}^{0}\dot{{\overline{T}}}_{k}^{0}\left(t\right)\] para $\overline{Q}_{k}\left(t\right)>0$.
\end{itemize}
\end{Prop}

\begin{Teo}[Teorema 5.1: Ley Fuerte para Procesos de Conteo
\cite{Gut}]\label{Tma.5.1.Gut} Sea
$0<\mu<\esp\left(X_{1}\right]\leq\infty$. entonces

\begin{itemize}
\item[a)] $\frac{N\left(t\right)}{t}\rightarrow\frac{1}{\mu}$
a.s., cuando $t\rightarrow\infty$.


\item[b)]$\esp\left[\frac{N\left(t\right)}{t}\right]^{r}\rightarrow\frac{1}{\mu^{r}}$,
cuando $t\rightarrow\infty$ para todo $r>0$..
\end{itemize}
\end{Teo}


\begin{Prop}[Proposici\'on 5.3 \cite{DaiSean}]
Sea $X$ proceso de estados para la red de colas, y suponga que se
cumplen los supuestos (A1) y (A2), entonces para alguna constante
positiva $C_{p+1}<\infty$, $\delta>0$ y un conjunto compacto
$C\subset X$.

\begin{equation}\label{Eq.5.4}
\esp_{x}\left[\int_{0}^{\tau_{C}\left(\delta\right)}\left(1+|X\left(t\right)|^{p}\right)dt\right]\leq
C_{p+1}\left(1+|x|^{p+1}\right)
\end{equation}
\end{Prop}

\begin{Prop}[Proposici\'on 5.4 \cite{DaiSean}]
Sea $X$ un proceso de Markov Borel Derecho en $X$, sea
$f:X\leftarrow\rea_{+}$ y defina para alguna $\delta>0$, y un
conjunto cerrado $C\subset X$
\[V\left(x\right):=\esp_{x}\left[\int_{0}^{\tau_{C}\left(\delta\right)}f\left(X\left(t\right)\right)dt\right]\]
para $x\in X$. Si $V$ es finito en todas partes y uniformemente
acotada en $C$, entonces existe $k<\infty$ tal que
\begin{equation}\label{Eq.5.11}
\frac{1}{t}\esp_{x}\left[V\left(x\right)\right]+\frac{1}{t}\int_{0}^{t}\esp_{x}\left[f\left(X\left(s\right)\right)ds\right]\leq\frac{1}{t}V\left(x\right)+k,
\end{equation}
para $x\in X$ y $t>0$.
\end{Prop}


%_________________________________________________________________________
%\renewcommand{\refname}{Ap\'endice D}
%\renewcommand{\appendixname}{ESTABILIDAD}
%\renewcommand{\appendixtocname}{ESTABILIDAD}
%\renewcommand{\appendixpagename}{ESTABILIDAD}
%\appendix
%\clearpage % o \cleardoublepage
%\addappheadtotoc
%\appendixpage

\subsection{Estabilidad}

\begin{Def}[Definici\'on 3.2, Dai y Meyn \cite{DaiSean}]
El modelo de flujo retrasado de una disciplina de servicio en una
red con retraso
$\left(\overline{A}\left(0\right),\overline{B}\left(0\right)\right)\in\rea_{+}^{K+|A|}$
se define como el conjunto de ecuaciones dadas en
\ref{Eq.3.8}-\ref{Eq.3.13}, junto con la condici\'on:
\begin{equation}\label{CondAd.FluidModel}
\overline{Q}\left(t\right)=\overline{Q}\left(0\right)+\left(\alpha
t-\overline{A}\left(0\right)\right)^{+}-\left(I-P^{'}\right)M\left(\overline{T}\left(t\right)-\overline{B}\left(0\right)\right)^{+}
\end{equation}
\end{Def}

entonces si el modelo de flujo retrasado tambi\'en es estable:


\begin{Def}[Definici\'on 3.1, Dai y Meyn \cite{DaiSean}]
Un flujo l\'imite (retrasado) para una red bajo una disciplina de
servicio espec\'ifica se define como cualquier soluci\'on
 $\left(\overline{Q}\left(\cdot\right),\overline{T}\left(\cdot\right)\right)$ de las siguientes ecuaciones, donde
$\overline{Q}\left(t\right)=\left(\overline{Q}_{1}\left(t\right),\ldots,\overline{Q}_{K}\left(t\right)\right)^{'}$
y
$\overline{T}\left(t\right)=\left(\overline{T}_{1}\left(t\right),\ldots,\overline{T}_{K}\left(t\right)\right)^{'}$
\begin{equation}\label{Eq.3.8}
\overline{Q}_{k}\left(t\right)=\overline{Q}_{k}\left(0\right)+\alpha_{k}t-\mu_{k}\overline{T}_{k}\left(t\right)+\sum_{l=1}^{k}P_{lk}\mu_{l}\overline{T}_{l}\left(t\right)\\
\end{equation}
\begin{equation}\label{Eq.3.9}
\overline{Q}_{k}\left(t\right)\geq0\textrm{ para }k=1,2,\ldots,K,\\
\end{equation}
\begin{equation}\label{Eq.3.10}
\overline{T}_{k}\left(0\right)=0,\textrm{ y }\overline{T}_{k}\left(\cdot\right)\textrm{ es no decreciente},\\
\end{equation}
\begin{equation}\label{Eq.3.11}
\overline{I}_{i}\left(t\right)=t-\sum_{k\in C_{i}}\overline{T}_{k}\left(t\right)\textrm{ es no decreciente}\\
\end{equation}
\begin{equation}\label{Eq.3.12}
\overline{I}_{i}\left(\cdot\right)\textrm{ se incrementa al tiempo }t\textrm{ cuando }\sum_{k\in C_{i}}Q_{k}^{x}\left(t\right)dI_{i}^{x}\left(t\right)=0\\
\end{equation}
\begin{equation}\label{Eq.3.13}
\textrm{condiciones adicionales sobre
}\left(Q^{x}\left(\cdot\right),T^{x}\left(\cdot\right)\right)\textrm{
referentes a la disciplina de servicio}
\end{equation}
\end{Def}

\begin{Lema}[Lema 3.1 \cite{Chen}]\label{Lema3.1}
Si el modelo de flujo es estable, definido por las ecuaciones
(3.8)-(3.13), entonces el modelo de flujo retrasado tambin es
estable.
\end{Lema}

\begin{Teo}[Teorema 5.1 \cite{Chen}]\label{Tma.5.1.Chen}
La red de colas es estable si existe una constante $t_{0}$ que
depende de $\left(\alpha,\mu,T,U\right)$ y $V$ que satisfagan las
ecuaciones (5.1)-(5.5), $Z\left(t\right)=0$, para toda $t\geq
t_{0}$.
\end{Teo}

\begin{Prop}[Proposici\'on 5.1, Dai y Meyn \cite{DaiSean}]\label{Prop.5.1.DaiSean}
Suponga que los supuestos A1) y A2) son ciertos y que el modelo de flujo es estable. Entonces existe $t_{0}>0$ tal que
\begin{equation}
lim_{|x|\rightarrow\infty}\frac{1}{|x|^{p+1}}\esp_{x}\left[|X\left(t_{0}|x|\right)|^{p+1}\right]=0
\end{equation}
\end{Prop}

\begin{Lemma}[Lema 5.2, Dai y Meyn \cite{DaiSean}]\label{Lema.5.2.DaiSean}
 Sea $\left\{\zeta\left(k\right):k\in \mathbb{z}\right\}$ una sucesi\'on independiente e id\'enticamente distribuida que toma valores en $\left(0,\infty\right)$,
y sea
$E\left(t\right)=max\left(n\geq1:\zeta\left(1\right)+\cdots+\zeta\left(n-1\right)\leq
t\right)$. Si $\esp\left[\zeta\left(1\right)\right]<\infty$,
entonces para cualquier entero $r\geq1$
\begin{equation}
 lim_{t\rightarrow\infty}\esp\left[\left(\frac{E\left(t\right)}{t}\right)^{r}\right]=\left(\frac{1}{\esp\left[\zeta_{1}\right]}\right)^{r}.
\end{equation}
Luego, bajo estas condiciones:
\begin{itemize}
 \item[a)] para cualquier $\delta>0$, $\sup_{t\geq\delta}\esp\left[\left(\frac{E\left(t\right)}{t}\right)^{r}\right]<\infty$
\item[b)] las variables aleatorias
$\left\{\left(\frac{E\left(t\right)}{t}\right)^{r}:t\geq1\right\}$
son uniformemente integrables.
\end{itemize}
\end{Lemma}

\begin{Teo}[Teorema 5.5, Dai y Meyn \cite{DaiSean}]\label{Tma.5.5.DaiSean}
Suponga que los supuestos A1) y A2) se cumplen y que el modelo de
flujo es estable. Entonces existe una constante $\kappa_{p}$ tal
que
\begin{equation}
\frac{1}{t}\int_{0}^{t}\esp_{x}\left[|Q\left(s\right)|^{p}\right]ds\leq\kappa_{p}\left\{\frac{1}{t}|x|^{p+1}+1\right\}
\end{equation}
para $t>0$ y $x\in X$. En particular, para cada condici\'on
inicial
\begin{eqnarray*}
\limsup_{t\rightarrow\infty}\frac{1}{t}\int_{0}^{t}\esp_{x}\left[|Q\left(s\right)|^{p}\right]ds\leq\kappa_{p}.
\end{eqnarray*}
\end{Teo}

\begin{Teo}[Teorema 6.2, Dai y Meyn \cite{DaiSean}]\label{Tma.6.2.DaiSean}
Suponga que se cumplen los supuestos A1), A2) y A3) y que el
modelo de flujo es estable. Entonces se tiene que
\begin{equation}
\left\|P^{t}\left(x,\cdot\right)-\pi\left(\cdot\right)\right\|_{f_{p}}\textrm{,
}t\rightarrow\infty,x\in X.
\end{equation}
En particular para cada condici\'on inicial
\begin{eqnarray*}
\lim_{t\rightarrow\infty}\esp_{x}\left[|Q\left(t\right)|^{p}\right]=\esp_{\pi}\left[|Q\left(0\right)|^{p}\right]\leq\kappa_{r}
\end{eqnarray*}
\end{Teo}
\begin{Teo}[Teorema 6.3, Dai y Meyn \cite{DaiSean}]\label{Tma.6.3.DaiSean}
Suponga que se cumplen los supuestos A1), A2) y A3) y que el
modelo de flujo es estable. Entonces con
$f\left(x\right)=f_{1}\left(x\right)$ se tiene
\begin{equation}
\lim_{t\rightarrow\infty}t^{p-1}\left\|P^{t}\left(x,\cdot\right)-\pi\left(\cdot\right)\right\|_{f}=0.
\end{equation}
En particular para cada condici\'on inicial
\begin{eqnarray*}
\lim_{t\rightarrow\infty}t^{p-1}|\esp_{x}\left[Q\left(t\right)\right]-\esp_{\pi}\left[Q\left(0\right)\right]|=0.
\end{eqnarray*}
\end{Teo}

\begin{Teo}[Teorema 6.4, Dai y Meyn \cite{DaiSean}]\label{Tma.6.4.DaiSean}
Suponga que se cumplen los supuestos A1), A2) y A3) y que el
modelo de flujo es estable. Sea $\nu$ cualquier distribuci\'on de
probabilidad en $\left(X,\mathcal{B}_{X}\right)$, y $\pi$ la
distribuci\'on estacionaria de $X$.
\begin{itemize}
\item[i)] Para cualquier $f:X\leftarrow\rea_{+}$
\begin{equation}
\lim_{t\rightarrow\infty}\frac{1}{t}\int_{o}^{t}f\left(X\left(s\right)\right)ds=\pi\left(f\right):=\int
f\left(x\right)\pi\left(dx\right)
\end{equation}
$\prob$-c.s.

\item[ii)] Para cualquier $f:X\leftarrow\rea_{+}$ con
$\pi\left(|f|\right)<\infty$, la ecuaci\'on anterior se cumple.
\end{itemize}
\end{Teo}

\begin{Teo}[Teorema 2.2, Down \cite{Down}]\label{Tma2.2.Down}
Suponga que el fluido modelo es inestable en el sentido de que
para alguna $\epsilon_{0},c_{0}\geq0$,
\begin{equation}\label{Eq.Inestability}
|Q\left(T\right)|\geq\epsilon_{0}T-c_{0}\textrm{,   }T\geq0,
\end{equation}
para cualquier condici\'on inicial $Q\left(0\right)$, con
$|Q\left(0\right)|=1$. Entonces para cualquier $0<q\leq1$, existe
$B<0$ tal que para cualquier $|x|\geq B$,
\begin{equation}
\prob_{x}\left\{\mathbb{X}\rightarrow\infty\right\}\geq q.
\end{equation}
\end{Teo}


\begin{Def}
Sea $X$ un conjunto y $\mathcal{F}$ una $\sigma$-\'algebra de
subconjuntos de $X$, la pareja $\left(X,\mathcal{F}\right)$ es
llamado espacio medible. Un subconjunto $A$ de $X$ es llamado
medible, o medible con respecto a $\mathcal{F}$, si
$A\in\mathcal{F}$.
\end{Def}

\begin{Def}
Sea $\left(X,\mathcal{F},\mu\right)$ espacio de medida. Se dice
que la medida $\mu$ es $\sigma$-finita si se puede escribir
$X=\bigcup_{n\geq1}X_{n}$ con $X_{n}\in\mathcal{F}$ y
$\mu\left(X_{n}\right)<\infty$.
\end{Def}

\begin{Def}\label{Cto.Borel}
Sea $X$ el conjunto de los \'umeros reales $\rea$. El \'algebra de
Borel es la $\sigma$-\'algebra $B$ generada por los intervalos
abiertos $\left(a,b\right)\in\rea$. Cualquier conjunto en $B$ es
llamado {\em Conjunto de Borel}.
\end{Def}

\begin{Def}\label{Funcion.Medible}
Una funci\'on $f:X\rightarrow\rea$, es medible si para cualquier
n\'umero real $\alpha$ el conjunto
\[\left\{x\in X:f\left(x\right)>\alpha\right\}\]
pertenece a $X$. Equivalentemente, se dice que $f$ es medible si
\[f^{-1}\left(\left(\alpha,\infty\right)\right)=\left\{x\in X:f\left(x\right)>\alpha\right\}\in\mathcal{F}.\]
\end{Def}


\begin{Def}\label{Def.Cilindros}
Sean $\left(\Omega_{i},\mathcal{F}_{i}\right)$, $i=1,2,\ldots,$
espacios medibles y $\Omega=\prod_{i=1}^{\infty}\Omega_{i}$ el
conjunto de todas las sucesiones
$\left(\omega_{1},\omega_{2},\ldots,\right)$ tales que
$\omega_{i}\in\Omega_{i}$, $i=1,2,\ldots,$. Si
$B^{n}\subset\prod_{i=1}^{\infty}\Omega_{i}$, definimos
$B_{n}=\left\{\omega\in\Omega:\left(\omega_{1},\omega_{2},\ldots,\omega_{n}\right)\in
B^{n}\right\}$. Al conjunto $B_{n}$ se le llama {\em cilindro} con
base $B^{n}$, el cilindro es llamado medible si
$B^{n}\in\prod_{i=1}^{\infty}\mathcal{F}_{i}$.
\end{Def}


\begin{Def}\label{Def.Proc.Adaptado}[TSP, Ash \cite{RBA}]
Sea $X\left(t\right),t\geq0$ proceso estoc\'astico, el proceso es
adaptado a la familia de $\sigma$-\'algebras $\mathcal{F}_{t}$,
para $t\geq0$, si para $s<t$ implica que
$\mathcal{F}_{s}\subset\mathcal{F}_{t}$, y $X\left(t\right)$ es
$\mathcal{F}_{t}$-medible para cada $t$. Si no se especifica
$\mathcal{F}_{t}$ entonces se toma $\mathcal{F}_{t}$ como
$\mathcal{F}\left(X\left(s\right),s\leq t\right)$, la m\'as
peque\~na $\sigma$-\'algebra de subconjuntos de $\Omega$ que hace
que cada $X\left(s\right)$, con $s\leq t$ sea Borel medible.
\end{Def}


\begin{Def}\label{Def.Tiempo.Paro}[TSP, Ash \cite{RBA}]
Sea $\left\{\mathcal{F}\left(t\right),t\geq0\right\}$ familia
creciente de sub $\sigma$-\'algebras. es decir,
$\mathcal{F}\left(s\right)\subset\mathcal{F}\left(t\right)$ para
$s\leq t$. Un tiempo de paro para $\mathcal{F}\left(t\right)$ es
una funci\'on $T:\Omega\rightarrow\left[0,\infty\right]$ tal que
$\left\{T\leq t\right\}\in\mathcal{F}\left(t\right)$ para cada
$t\geq0$. Un tiempo de paro para el proceso estoc\'astico
$X\left(t\right),t\geq0$ es un tiempo de paro para las
$\sigma$-\'algebras
$\mathcal{F}\left(t\right)=\mathcal{F}\left(X\left(s\right)\right)$.
\end{Def}

\begin{Def}
Sea $X\left(t\right),t\geq0$ proceso estoc\'astico, con
$\left(S,\chi\right)$ espacio de estados. Se dice que el proceso
es adaptado a $\left\{\mathcal{F}\left(t\right)\right\}$, es
decir, si para cualquier $s,t\in I$, $I$ conjunto de \'indices,
$s<t$, se tiene que
$\mathcal{F}\left(s\right)\subset\mathcal{F}\left(t\right)$ y
$X\left(t\right)$ es $\mathcal{F}\left(t\right)$-medible,
\end{Def}

\begin{Def}
Sea $X\left(t\right),t\geq0$ proceso estoc\'astico, se dice que es
un Proceso de Markov relativo a $\mathcal{F}\left(t\right)$ o que
$\left\{X\left(t\right),\mathcal{F}\left(t\right)\right\}$ es de
Markov si y s\'olo si para cualquier conjunto $B\in\chi$,  y
$s,t\in I$, $s<t$ se cumple que
\begin{equation}\label{Prop.Markov}
P\left\{X\left(t\right)\in
B|\mathcal{F}\left(s\right)\right\}=P\left\{X\left(t\right)\in
B|X\left(s\right)\right\}.
\end{equation}
\end{Def}
\begin{Note}
Si se dice que $\left\{X\left(t\right)\right\}$ es un Proceso de
Markov sin mencionar $\mathcal{F}\left(t\right)$, se asumir\'a que
\begin{eqnarray*}
\mathcal{F}\left(t\right)=\mathcal{F}_{0}\left(t\right)=\mathcal{F}\left(X\left(r\right),r\leq
t\right),
\end{eqnarray*}
entonces la ecuaci\'on (\ref{Prop.Markov}) se puede escribir como
\begin{equation}
P\left\{X\left(t\right)\in B|X\left(r\right),r\leq s\right\} =
P\left\{X\left(t\right)\in B|X\left(s\right)\right\}
\end{equation}
\end{Note}

\begin{Teo}
Sea $\left(X_{n},\mathcal{F}_{n},n=0,1,\ldots,\right\}$ Proceso de
Markov con espacio de estados $\left(S_{0},\chi_{0}\right)$
generado por una distribuici\'on inicial $P_{o}$ y probabilidad de
transici\'on $p_{mn}$, para $m,n=0,1,\ldots,$ $m<n$, que por
notaci\'on se escribir\'a como $p\left(m,n,x,B\right)\rightarrow
p_{mn}\left(x,B\right)$. Sea $S$ tiempo de paro relativo a la
$\sigma$-\'algebra $\mathcal{F}_{n}$. Sea $T$ funci\'on medible,
$T:\Omega\rightarrow\left\{0,1,\ldots,\right\}$. Sup\'ongase que
$T\geq S$, entonces $T$ es tiempo de paro. Si $B\in\chi_{0}$,
entonces
\begin{equation}\label{Prop.Fuerte.Markov}
P\left\{X\left(T\right)\in
B,T<\infty|\mathcal{F}\left(S\right)\right\} =
p\left(S,T,X\left(s\right),B\right)
\end{equation}
en $\left\{T<\infty\right\}$.
\end{Teo}


Sea $K$ conjunto numerable y sea $d:K\rightarrow\nat$ funci\'on.
Para $v\in K$, $M_{v}$ es un conjunto abierto de
$\rea^{d\left(v\right)}$. Entonces \[E=\cup_{v\in
K}M_{v}=\left\{\left(v,\zeta\right):v\in K,\zeta\in
M_{v}\right\}.\]

Sea $\mathcal{E}$ la clase de conjuntos medibles en $E$:
\[\mathcal{E}=\left\{\cup_{v\in K}A_{v}:A_{v}\in \mathcal{M}_{v}\right\}.\]

donde $\mathcal{M}$ son los conjuntos de Borel de $M_{v}$.
Entonces $\left(E,\mathcal{E}\right)$ es un espacio de Borel. El
estado del proceso se denotar\'a por
$\mathbf{x}_{t}=\left(v_{t},\zeta_{t}\right)$. La distribuci\'on
de $\left(\mathbf{x}_{t}\right)$ est\'a determinada por por los
siguientes objetos:

\begin{itemize}
\item[i)] Los campos vectoriales $\left(\mathcal{H}_{v},v\in
K\right)$. \item[ii)] Una funci\'on medible $\lambda:E\rightarrow
\rea_{+}$. \item[iii)] Una medida de transici\'on
$Q:\mathcal{E}\times\left(E\cup\Gamma^{*}\right)\rightarrow\left[0,1\right]$
donde
\begin{equation}
\Gamma^{*}=\cup_{v\in K}\partial^{*}M_{v}.
\end{equation}
y
\begin{equation}
\partial^{*}M_{v}=\left\{z\in\partial M_{v}:\mathbf{\mathbf{\phi}_{v}\left(t,\zeta\right)=\mathbf{z}}\textrm{ para alguna }\left(t,\zeta\right)\in\rea_{+}\times M_{v}\right\}.
\end{equation}
$\partial M_{v}$ denota  la frontera de $M_{v}$.
\end{itemize}

El campo vectorial $\left(\mathcal{H}_{v},v\in K\right)$ se supone
tal que para cada $\mathbf{z}\in M_{v}$ existe una \'unica curva
integral $\mathbf{\phi}_{v}\left(t,\zeta\right)$ que satisface la
ecuaci\'on

\begin{equation}
\frac{d}{dt}f\left(\zeta_{t}\right)=\mathcal{H}f\left(\zeta_{t}\right),
\end{equation}
con $\zeta_{0}=\mathbf{z}$, para cualquier funci\'on suave
$f:\rea^{d}\rightarrow\rea$ y $\mathcal{H}$ denota el operador
diferencial de primer orden, con $\mathcal{H}=\mathcal{H}_{v}$ y
$\zeta_{t}=\mathbf{\phi}\left(t,\mathbf{z}\right)$. Adem\'as se
supone que $\mathcal{H}_{v}$ es conservativo, es decir, las curvas
integrales est\'an definidas para todo $t>0$.

Para $\mathbf{x}=\left(v,\zeta\right)\in E$ se denota
\[t^{*}\mathbf{x}=inf\left\{t>0:\mathbf{\phi}_{v}\left(t,\zeta\right)\in\partial^{*}M_{v}\right\}\]

En lo que respecta a la funci\'on $\lambda$, se supondr\'a que
para cada $\left(v,\zeta\right)\in E$ existe un $\epsilon>0$ tal
que la funci\'on
$s\rightarrow\lambda\left(v,\phi_{v}\left(s,\zeta\right)\right)\in
E$ es integrable para $s\in\left[0,\epsilon\right)$. La medida de
transici\'on $Q\left(A;\mathbf{x}\right)$ es una funci\'on medible
de $\mathbf{x}$ para cada $A\in\mathcal{E}$, definida para
$\mathbf{x}\in E\cup\Gamma^{*}$ y es una medida de probabilidad en
$\left(E,\mathcal{E}\right)$ para cada $\mathbf{x}\in E$.

El movimiento del proceso $\left(\mathbf{x}_{t}\right)$ comenzando
en $\mathbf{x}=\left(n,\mathbf{z}\right)\in E$ se puede construir
de la siguiente manera, def\'inase la funci\'on $F$ por

\begin{equation}
F\left(t\right)=\left\{\begin{array}{ll}\\
exp\left(-\int_{0}^{t}\lambda\left(n,\phi_{n}\left(s,\mathbf{z}\right)\right)ds\right), & t<t^{*}\left(\mathbf{x}\right),\\
0, & t\geq t^{*}\left(\mathbf{x}\right)
\end{array}\right.
\end{equation}

Sea $T_{1}$ una variable aleatoria tal que
$\prob\left[T_{1}>t\right]=F\left(t\right)$, ahora sea la variable
aleatoria $\left(N,Z\right)$ con distribuici\'on
$Q\left(\cdot;\phi_{n}\left(T_{1},\mathbf{z}\right)\right)$. La
trayectoria de $\left(\mathbf{x}_{t}\right)$ para $t\leq T_{1}$
es\footnote{Revisar p\'agina 362, y 364 de Davis \cite{Davis}.}
\begin{eqnarray*}
\mathbf{x}_{t}=\left(v_{t},\zeta_{t}\right)=\left\{\begin{array}{ll}
\left(n,\phi_{n}\left(t,\mathbf{z}\right)\right), & t<T_{1},\\
\left(N,\mathbf{Z}\right), & t=t_{1}.
\end{array}\right.
\end{eqnarray*}

Comenzando en $\mathbf{x}_{T_{1}}$ se selecciona el siguiente
tiempo de intersalto $T_{2}-T_{1}$ lugar del post-salto
$\mathbf{x}_{T_{2}}$ de manera similar y as\'i sucesivamente. Este
procedimiento nos da una trayectoria determinista por partes
$\mathbf{x}_{t}$ con tiempos de salto $T_{1},T_{2},\ldots$. Bajo
las condiciones enunciadas para $\lambda,T_{1}>0$  y
$T_{1}-T_{2}>0$ para cada $i$, con probabilidad 1. Se supone que
se cumple la siquiente condici\'on.

\begin{Sup}[Supuesto 3.1, Davis \cite{Davis}]\label{Sup3.1.Davis}
Sea $N_{t}:=\sum_{t}\indora_{\left(t\geq t\right)}$ el n\'umero de
saltos en $\left[0,t\right]$. Entonces
\begin{equation}
\esp\left[N_{t}\right]<\infty\textrm{ para toda }t.
\end{equation}
\end{Sup}

es un proceso de Markov, m\'as a\'un, es un Proceso Fuerte de
Markov, es decir, la Propiedad Fuerte de Markov se cumple para
cualquier tiempo de paro.
%_________________________________________________________________________

En esta secci\'on se har\'an las siguientes consideraciones: $E$
es un espacio m\'etrico separable y la m\'etrica $d$ es compatible
con la topolog\'ia.


\begin{Def}
Un espacio topol\'ogico $E$ es llamado {\em Luisin} si es
homeomorfo a un subconjunto de Borel de un espacio m\'etrico
compacto.
\end{Def}

\begin{Def}
Un espacio topol\'ogico $E$ es llamado de {\em Rad\'on} si es
homeomorfo a un subconjunto universalmente medible de un espacio
m\'etrico compacto.
\end{Def}

Equivalentemente, la definici\'on de un espacio de Rad\'on puede
encontrarse en los siguientes t\'erminos:


\begin{Def}
$E$ es un espacio de Rad\'on si cada medida finita en
$\left(E,\mathcal{B}\left(E\right)\right)$ es regular interior o cerrada,
{\em tight}.
\end{Def}

\begin{Def}
Una medida finita, $\lambda$ en la $\sigma$-\'algebra de Borel de
un espacio metrizable $E$ se dice cerrada si
\begin{equation}\label{Eq.A2.3}
\lambda\left(E\right)=sup\left\{\lambda\left(K\right):K\textrm{ es
compacto en }E\right\}.
\end{equation}
\end{Def}

El siguiente teorema nos permite tener una mejor caracterizaci\'on de los espacios de Rad\'on:
\begin{Teo}\label{Tma.A2.2}
Sea $E$ espacio separable metrizable. Entonces $E$ es Radoniano si y s\'olo s\'i cada medida finita en $\left(E,\mathcal{B}\left(E\right)\right)$ es cerrada.
\end{Teo}

%_________________________________________________________________________________________
\subsection{Propiedades de Markov}
%_________________________________________________________________________________________

Sea $E$ espacio de estados, tal que $E$ es un espacio de Rad\'on, $\mathcal{B}\left(E\right)$ $\sigma$-\'algebra de Borel en $E$, que se denotar\'a por $\mathcal{E}$.

Sea $\left(X,\mathcal{G},\prob\right)$ espacio de probabilidad, $I\subset\rea$ conjunto de índices. Sea $\mathcal{F}_{\leq t}$ la $\sigma$-\'algebra natural definida como $\sigma\left\{f\left(X_{r}\right):r\in I, rleq t,f\in\mathcal{E}\right\}$. Se considerar\'a una $\sigma$-\'algebra m\'as general, $ \left(\mathcal{G}_{t}\right)$ tal que $\left(X_{t}\right)$ sea $\mathcal{E}$-adaptado.

\begin{Def}
Una familia $\left(P_{s,t}\right)$ de kernels de Markov en $\left(E,\mathcal{E}\right)$ indexada por pares $s,t\in I$, con $s\leq t$ es una funci\'on de transici\'on en $\ER$, si  para todo $r\leq s< t$ en $I$ y todo $x\in E$, $B\in\mathcal{E}$
\begin{equation}\label{Eq.Kernels}
P_{r,t}\left(x,B\right)=\int_{E}P_{r,s}\left(x,dy\right)P_{s,t}\left(y,B\right)\footnote{Ecuaci\'on de Chapman-Kolmogorov}.
\end{equation}
\end{Def}

Se dice que la funci\'on de transici\'on $\KM$ en $\ER$ es la funci\'on de transici\'on para un proceso $\PE$  con valores en $E$ y que satisface la propiedad de Markov\footnote{\begin{equation}\label{Eq.1.4.S}
\prob\left\{H|\mathcal{G}_{t}\right\}=\prob\left\{H|X_{t}\right\}\textrm{ }H\in p\mathcal{F}_{\geq t}.
\end{equation}} (\ref{Eq.1.4.S}) relativa a $\left(\mathcal{G}_{t}\right)$ si 

\begin{equation}\label{Eq.1.6.S}
\prob\left\{f\left(X_{t}\right)|\mathcal{G}_{s}\right\}=P_{s,t}f\left(X_{t}\right)\textrm{ }s\leq t\in I,\textrm{ }f\in b\mathcal{E}.
\end{equation}

\begin{Def}
Una familia $\left(P_{t}\right)_{t\geq0}$ de kernels de Markov en $\ER$ es llamada {\em Semigrupo de Transici\'on de Markov} o {\em Semigrupo de Transici\'on} si
\[P_{t+s}f\left(x\right)=P_{t}\left(P_{s}f\right)\left(x\right),\textrm{ }t,s\geq0,\textrm{ }x\in E\textrm{ }f\in b\mathcal{E}.\]
\end{Def}
\begin{Note}
Si la funci\'on de transici\'on $\KM$ es llamada homog\'enea si $P_{s,t}=P_{t-s}$.
\end{Note}

Un proceso de Markov que satisface la ecuaci\'on (\ref{Eq.1.6.S}) con funci\'on de transici\'on homog\'enea $\left(P_{t}\right)$ tiene la propiedad caracter\'istica
\begin{equation}\label{Eq.1.8.S}
\prob\left\{f\left(X_{t+s}\right)|\mathcal{G}_{t}\right\}=P_{s}f\left(X_{t}\right)\textrm{ }t,s\geq0,\textrm{ }f\in b\mathcal{E}.
\end{equation}
La ecuaci\'on anterior es la {\em Propiedad Simple de Markov} de $X$ relativa a $\left(P_{t}\right)$.

En este sentido el proceso $\PE$ cumple con la propiedad de Markov (\ref{Eq.1.8.S}) relativa a $\left(\Omega,\mathcal{G},\mathcal{G}_{t},\prob\right)$ con semigrupo de transici\'on $\left(P_{t}\right)$.
%_________________________________________________________________________________________
\subsection{Primer Condici\'on de Regularidad}
%_________________________________________________________________________________________
%\newcommand{\EM}{\left(\Omega,\mathcal{G},\prob\right)}
%\newcommand{\E4}{\left(\Omega,\mathcal{G},\mathcal{G}_{t},\prob\right)}
\begin{Def}
Un proceso estoc\'astico $\PE$ definido en $\left(\Omega,\mathcal{G},\prob\right)$ con valores en el espacio topol\'ogico $E$ es continuo por la derecha si cada trayectoria muestral $t\rightarrow X_{t}\left(w\right)$ es un mapeo continuo por la derecha de $I$ en $E$.
\end{Def}

\begin{Def}[HD1]\label{Eq.2.1.S}
Un semigrupo de Markov $\left/P_{t}\right)$ en un espacio de Rad\'on $E$ se dice que satisface la condici\'on {\em HD1} si, dada una medida de probabilidad $\mu$ en $E$, existe una $\sigma$-\'algebra $\mathcal{E^{*}}$ con $\mathcal{E}\subset\mathcal{E}$ y $P_{t}\left(b\mathcal{E}^{*}\right)\subset b\mathcal{E}^{*}$, y un $\mathcal{E}^{*}$-proceso $E$-valuado continuo por la derecha $\PE$ en alg\'un espacio de probabilidad filtrado $\left(\Omega,\mathcal{G},\mathcal{G}_{t},\prob\right)$ tal que $X=\left(\Omega,\mathcal{G},\mathcal{G}_{t},\prob\right)$ es de Markov (Homog\'eneo) con semigrupo de transici\'on $(P_{t})$ y distribuci\'on inicial $\mu$.
\end{Def}

Considerese la colecci\'on de variables aleatorias $X_{t}$ definidas en alg\'un espacio de probabilidad, y una colecci\'on de medidas $\mathbf{P}^{x}$ tales que $\mathbf{P}^{x}\left\{X_{0}=x\right\}$, y bajo cualquier $\mathbf{P}^{x}$, $X_{t}$ es de Markov con semigrupo $\left(P_{t}\right)$. $\mathbf{P}^{x}$ puede considerarse como la distribuci\'on condicional de $\mathbf{P}$ dado $X_{0}=x$.

\begin{Def}\label{Def.2.2.S}
Sea $E$ espacio de Rad\'on, $\SG$ semigrupo de Markov en $\ER$. La colecci\'on $\mathbf{X}=\left(\Omega,\mathcal{G},\mathcal{G}_{t},X_{t},\theta_{t},\CM\right)$ es un proceso $\mathcal{E}$-Markov continuo por la derecha simple, con espacio de estados $E$ y semigrupo de transici\'on $\SG$ en caso de que $\mathbf{X}$ satisfaga las siguientes condiciones:
\begin{itemize}
\item[i)] $\left(\Omega,\mathcal{G},\mathcal{G}_{t}\right)$ es un espacio de medida filtrado, y $X_{t}$ es un proceso $E$-valuado continuo por la derecha $\mathcal{E}^{*}$-adaptado a $\left(\mathcal{G}_{t}\right)$;

\item[ii)] $\left(\theta_{t}\right)_{t\geq0}$ es una colecci\'on de operadores {\em shift} para $X$, es decir, mapea $\Omega$ en s\'i mismo satisfaciendo para $t,s\geq0$,

\begin{equation}\label{Eq.Shift}
\theta_{t}\circ\theta_{s}=\theta_{t+s}\textrm{ y }X_{t}\circ\theta_{t}=X_{t+s};
\end{equation}

\item[iii)] Para cualquier $x\in E$,$\CM\left\{X_{0}=x\right\}=1$, y el proceso $\PE$ tiene la propiedad de Markov (\ref{Eq.1.8.S}) con semigrupo de transici\'on $\SG$ relativo a $\left(\Omega,\mathcal{G},\mathcal{G}_{t},\CM\right)$.
\end{itemize}
\end{Def}

\begin{Def}[HD2]\label{Eq.2.2.S}
Para cualquier $\alpha>0$ y cualquier $f\in S^{\alpha}$, el proceso $t\rightarrow f\left(X_{t}\right)$ es continuo por la derecha casi seguramente.
\end{Def}

\begin{Def}\label{Def.PD}
Un sistema $\mathbf{X}=\left(\Omega,\mathcal{G},\mathcal{G}_{t},X_{t},\theta_{t},\CM\right)$ es un proceso derecho en el espacio de Rad\'on $E$ con semigrupo de transici\'on $\SG$ provisto de:
\begin{itemize}
\item[i)] $\mathbf{X}$ es una realizaci\'on  continua por la derecha, \ref{Def.2.2.S}, de $\SG$.

\item[ii)] $\mathbf{X}$ satisface la condicion HD2, \ref{Eq.2.2.S}, relativa a $\mathcal{G}_{t}$.

\item[iii)] $\mathcal{G}_{t}$ es aumentado y continuo por la derecha.
\end{itemize}
\end{Def}




\begin{Lema}[Lema 4.2, Dai\cite{Dai}]\label{Lema4.2}
Sea $\left\{x_{n}\right\}\subset \mathbf{X}$ con
$|x_{n}|\rightarrow\infty$, conforme $n\rightarrow\infty$. Suponga
que
\[lim_{n\rightarrow\infty}\frac{1}{|x_{n}|}U\left(0\right)=\overline{U}\]
y
\[lim_{n\rightarrow\infty}\frac{1}{|x_{n}|}V\left(0\right)=\overline{V}.\]

Entonces, conforme $n\rightarrow\infty$, casi seguramente

\begin{equation}\label{E1.4.2}
\frac{1}{|x_{n}|}\Phi^{k}\left(\left[|x_{n}|t\right]\right)\rightarrow
P_{k}^{'}t\textrm{, u.o.c.,}
\end{equation}

\begin{equation}\label{E1.4.3}
\frac{1}{|x_{n}|}E^{x_{n}}_{k}\left(|x_{n}|t\right)\rightarrow
\alpha_{k}\left(t-\overline{U}_{k}\right)^{+}\textrm{, u.o.c.,}
\end{equation}

\begin{equation}\label{E1.4.4}
\frac{1}{|x_{n}|}S^{x_{n}}_{k}\left(|x_{n}|t\right)\rightarrow
\mu_{k}\left(t-\overline{V}_{k}\right)^{+}\textrm{, u.o.c.,}
\end{equation}

donde $\left[t\right]$ es la parte entera de $t$ y
$\mu_{k}=1/m_{k}=1/\esp\left[\eta_{k}\left(1\right)\right]$.
\end{Lema}

\begin{Lema}[Lema 4.3, Dai\cite{Dai}]\label{Lema.4.3}
Sea $\left\{x_{n}\right\}\subset \mathbf{X}$ con
$|x_{n}|\rightarrow\infty$, conforme $n\rightarrow\infty$. Suponga
que
\[lim_{n\rightarrow\infty}\frac{1}{|x_{n}|}U\left(0\right)=\overline{U}_{k}\]
y
\[lim_{n\rightarrow\infty}\frac{1}{|x_{n}|}V\left(0\right)=\overline{V}_{k}.\]
\begin{itemize}
\item[a)] Conforme $n\rightarrow\infty$ casi seguramente,
\[lim_{n\rightarrow\infty}\frac{1}{|x_{n}|}U^{x_{n}}_{k}\left(|x_{n}|t\right)=\left(\overline{U}_{k}-t\right)^{+}\textrm{, u.o.c.}\]
y
\[lim_{n\rightarrow\infty}\frac{1}{|x_{n}|}V^{x_{n}}_{k}\left(|x_{n}|t\right)=\left(\overline{V}_{k}-t\right)^{+}.\]

\item[b)] Para cada $t\geq0$ fijo,
\[\left\{\frac{1}{|x_{n}|}U^{x_{n}}_{k}\left(|x_{n}|t\right),|x_{n}|\geq1\right\}\]
y
\[\left\{\frac{1}{|x_{n}|}V^{x_{n}}_{k}\left(|x_{n}|t\right),|x_{n}|\geq1\right\}\]
\end{itemize}
son uniformemente convergentes.
\end{Lema}

$S_{l}^{x}\left(t\right)$ es el n\'umero total de servicios
completados de la clase $l$, si la clase $l$ est\'a dando $t$
unidades de tiempo de servicio. Sea $T_{l}^{x}\left(x\right)$ el
monto acumulado del tiempo de servicio que el servidor
$s\left(l\right)$ gasta en los usuarios de la clase $l$ al tiempo
$t$. Entonces $S_{l}^{x}\left(T_{l}^{x}\left(t\right)\right)$ es
el n\'umero total de servicios completados para la clase $l$ al
tiempo $t$. Una fracci\'on de estos usuarios,
$\Phi_{l}^{x}\left(S_{l}^{x}\left(T_{l}^{x}\left(t\right)\right)\right)$,
se convierte en usuarios de la clase $k$.\\

Entonces, dado lo anterior, se tiene la siguiente representaci\'on
para el proceso de la longitud de la cola:\\

\begin{equation}
Q_{k}^{x}\left(t\right)=_{k}^{x}\left(0\right)+E_{k}^{x}\left(t\right)+\sum_{l=1}^{K}\Phi_{k}^{l}\left(S_{l}^{x}\left(T_{l}^{x}\left(t\right)\right)\right)-S_{k}^{x}\left(T_{k}^{x}\left(t\right)\right)
\end{equation}
para $k=1,\ldots,K$. Para $i=1,\ldots,d$, sea
\[I_{i}^{x}\left(t\right)=t-\sum_{j\in C_{i}}T_{k}^{x}\left(t\right).\]

Entonces $I_{i}^{x}\left(t\right)$ es el monto acumulado del
tiempo que el servidor $i$ ha estado desocupado al tiempo $t$. Se
est\'a asumiendo que las disciplinas satisfacen la ley de
conservaci\'on del trabajo, es decir, el servidor $i$ est\'a en
pausa solamente cuando no hay usuarios en la estaci\'on $i$.
Entonces, se tiene que

\begin{equation}
\int_{0}^{\infty}\left(\sum_{k\in
C_{i}}Q_{k}^{x}\left(t\right)\right)dI_{i}^{x}\left(t\right)=0,
\end{equation}
para $i=1,\ldots,d$.\\

Hacer
\[T^{x}\left(t\right)=\left(T_{1}^{x}\left(t\right),\ldots,T_{K}^{x}\left(t\right)\right)^{'},\]
\[I^{x}\left(t\right)=\left(I_{1}^{x}\left(t\right),\ldots,I_{K}^{x}\left(t\right)\right)^{'}\]
y
\[S^{x}\left(T^{x}\left(t\right)\right)=\left(S_{1}^{x}\left(T_{1}^{x}\left(t\right)\right),\ldots,S_{K}^{x}\left(T_{K}^{x}\left(t\right)\right)\right)^{'}.\]

Para una disciplina que cumple con la ley de conservaci\'on del
trabajo, en forma vectorial, se tiene el siguiente conjunto de
ecuaciones

\begin{equation}\label{Eq.MF.1.3}
Q^{x}\left(t\right)=Q^{x}\left(0\right)+E^{x}\left(t\right)+\sum_{l=1}^{K}\Phi^{l}\left(S_{l}^{x}\left(T_{l}^{x}\left(t\right)\right)\right)-S^{x}\left(T^{x}\left(t\right)\right),\\
\end{equation}

\begin{equation}\label{Eq.MF.2.3}
Q^{x}\left(t\right)\geq0,\\
\end{equation}

\begin{equation}\label{Eq.MF.3.3}
T^{x}\left(0\right)=0,\textrm{ y }\overline{T}^{x}\left(t\right)\textrm{ es no decreciente},\\
\end{equation}

\begin{equation}\label{Eq.MF.4.3}
I^{x}\left(t\right)=et-CT^{x}\left(t\right)\textrm{ es no
decreciente}\\
\end{equation}

\begin{equation}\label{Eq.MF.5.3}
\int_{0}^{\infty}\left(CQ^{x}\left(t\right)\right)dI_{i}^{x}\left(t\right)=0,\\
\end{equation}

\begin{equation}\label{Eq.MF.6.3}
\textrm{Condiciones adicionales en
}\left(\overline{Q}^{x}\left(\cdot\right),\overline{T}^{x}\left(\cdot\right)\right)\textrm{
espec\'ificas de la disciplina de la cola,}
\end{equation}

donde $e$ es un vector de unos de dimensi\'on $d$, $C$ es la
matriz definida por
\[C_{ik}=\left\{\begin{array}{cc}
1,& S\left(k\right)=i,\\
0,& \textrm{ en otro caso}.\\
\end{array}\right.
\]
Es necesario enunciar el siguiente Teorema que se utilizar\'a para
el Teorema \ref{Tma.4.2.Dai}:
\begin{Teo}[Teorema 4.1, Dai \cite{Dai}]
Considere una disciplina que cumpla la ley de conservaci\'on del
trabajo, para casi todas las trayectorias muestrales $\omega$ y
cualquier sucesi\'on de estados iniciales
$\left\{x_{n}\right\}\subset \mathbf{X}$, con
$|x_{n}|\rightarrow\infty$, existe una subsucesi\'on
$\left\{x_{n_{j}}\right\}$ con $|x_{n_{j}}|\rightarrow\infty$ tal
que
\begin{equation}\label{Eq.4.15}
\frac{1}{|x_{n_{j}}|}\left(Q^{x_{n_{j}}}\left(0\right),U^{x_{n_{j}}}\left(0\right),V^{x_{n_{j}}}\left(0\right)\right)\rightarrow\left(\overline{Q}\left(0\right),\overline{U},\overline{V}\right),
\end{equation}

\begin{equation}\label{Eq.4.16}
\frac{1}{|x_{n_{j}}|}\left(Q^{x_{n_{j}}}\left(|x_{n_{j}}|t\right),T^{x_{n_{j}}}\left(|x_{n_{j}}|t\right)\right)\rightarrow\left(\overline{Q}\left(t\right),\overline{T}\left(t\right)\right)\textrm{
u.o.c.}
\end{equation}

Adem\'as,
$\left(\overline{Q}\left(t\right),\overline{T}\left(t\right)\right)$
satisface las siguientes ecuaciones:
\begin{equation}\label{Eq.MF.1.3a}
\overline{Q}\left(t\right)=Q\left(0\right)+\left(\alpha
t-\overline{U}\right)^{+}-\left(I-P\right)^{'}M^{-1}\left(\overline{T}\left(t\right)-\overline{V}\right)^{+},
\end{equation}

\begin{equation}\label{Eq.MF.2.3a}
\overline{Q}\left(t\right)\geq0,\\
\end{equation}

\begin{equation}\label{Eq.MF.3.3a}
\overline{T}\left(t\right)\textrm{ es no decreciente y comienza en cero},\\
\end{equation}

\begin{equation}\label{Eq.MF.4.3a}
\overline{I}\left(t\right)=et-C\overline{T}\left(t\right)\textrm{
es no decreciente,}\\
\end{equation}

\begin{equation}\label{Eq.MF.5.3a}
\int_{0}^{\infty}\left(C\overline{Q}\left(t\right)\right)d\overline{I}\left(t\right)=0,\\
\end{equation}

\begin{equation}\label{Eq.MF.6.3a}
\textrm{Condiciones adicionales en
}\left(\overline{Q}\left(\cdot\right),\overline{T}\left(\cdot\right)\right)\textrm{
especficas de la disciplina de la cola,}
\end{equation}
\end{Teo}

\begin{Def}[Definici\'on 4.1, , Dai \cite{Dai}]
Sea una disciplina de servicio espec\'ifica. Cualquier l\'imite
$\left(\overline{Q}\left(\cdot\right),\overline{T}\left(\cdot\right)\right)$
en \ref{Eq.4.16} es un {\em flujo l\'imite} de la disciplina.
Cualquier soluci\'on (\ref{Eq.MF.1.3a})-(\ref{Eq.MF.6.3a}) es
llamado flujo soluci\'on de la disciplina. Se dice que el modelo de flujo l\'imite, modelo de flujo, de la disciplina de la cola es estable si existe una constante
$\delta>0$ que depende de $\mu,\alpha$ y $P$ solamente, tal que
cualquier flujo l\'imite con
$|\overline{Q}\left(0\right)|+|\overline{U}|+|\overline{V}|=1$, se
tiene que $\overline{Q}\left(\cdot+\delta\right)\equiv0$.
\end{Def}

\begin{Teo}[Teorema 4.2, Dai\cite{Dai}]\label{Tma.4.2.Dai}
Sea una disciplina fija para la cola, suponga que se cumplen las
condiciones (1.2)-(1.5). Si el modelo de flujo l\'imite de la
disciplina de la cola es estable, entonces la cadena de Markov $X$
que describe la din\'amica de la red bajo la disciplina es Harris
recurrente positiva.
\end{Teo}

Ahora se procede a escalar el espacio y el tiempo para reducir la
aparente fluctuaci\'on del modelo. Consid\'erese el proceso
\begin{equation}\label{Eq.3.7}
\overline{Q}^{x}\left(t\right)=\frac{1}{|x|}Q^{x}\left(|x|t\right)
\end{equation}
A este proceso se le conoce como el fluido escalado, y cualquier l\'imite $\overline{Q}^{x}\left(t\right)$ es llamado flujo l\'imite del proceso de longitud de la cola. Haciendo $|q|\rightarrow\infty$ mientras se mantiene el resto de las componentes fijas, cualquier punto l\'imite del proceso de longitud de la cola normalizado $\overline{Q}^{x}$ es soluci\'on del siguiente modelo de flujo.

Al conjunto de ecuaciones dadas en \ref{Eq.3.8}-\ref{Eq.3.13} se
le llama {\em Modelo de flujo} y al conjunto de todas las
soluciones del modelo de flujo
$\left(\overline{Q}\left(\cdot\right),\overline{T}
\left(\cdot\right)\right)$ se le denotar\'a por $\mathcal{Q}$.

Si se hace $|x|\rightarrow\infty$ sin restringir ninguna de las
componentes, tambi\'en se obtienen un modelo de flujo, pero en
este caso el residual de los procesos de arribo y servicio
introducen un retraso:

\begin{Def}[Definici\'on 3.3, Dai y Meyn \cite{DaiSean}]
El modelo de flujo es estable si existe un tiempo fijo $t_{0}$ tal
que $\overline{Q}\left(t\right)=0$, con $t\geq t_{0}$, para
cualquier $\overline{Q}\left(\cdot\right)\in\mathcal{Q}$ que
cumple con $|\overline{Q}\left(0\right)|=1$.
\end{Def}

El siguiente resultado se encuentra en Chen \cite{Chen}.
\begin{Lemma}[Lema 3.1, Dai y Meyn \cite{DaiSean}]
Si el modelo de flujo definido por \ref{Eq.3.8}-\ref{Eq.3.13} es
estable, entonces el modelo de flujo retrasado es tambi\'en
estable, es decir, existe $t_{0}>0$ tal que
$\overline{Q}\left(t\right)=0$ para cualquier $t\geq t_{0}$, para
cualquier soluci\'on del modelo de flujo retrasado cuya
condici\'on inicial $\overline{x}$ satisface que
$|\overline{x}|=|\overline{Q}\left(0\right)|+|\overline{A}\left(0\right)|+|\overline{B}\left(0\right)|\leq1$.
\end{Lemma}


Propiedades importantes para el modelo de flujo retrasado:

\begin{Prop}
 Sea $\left(\overline{Q},\overline{T},\overline{T}^{0}\right)$ un flujo l\'imite de \ref{Eq.4.4} y suponga que cuando $x\rightarrow\infty$ a lo largo de
una subsucesi\'on
\[\left(\frac{1}{|x|}Q_{k}^{x}\left(0\right),\frac{1}{|x|}A_{k}^{x}\left(0\right),\frac{1}{|x|}B_{k}^{x}\left(0\right),\frac{1}{|x|}B_{k}^{x,0}\left(0\right)\right)\rightarrow\left(\overline{Q}_{k}\left(0\right),0,0,0\right)\]
para $k=1,\ldots,K$. EL flujo l\'imite tiene las siguientes
propiedades, donde las propiedades de la derivada se cumplen donde
la derivada exista:
\begin{itemize}
 \item[i)] Los vectores de tiempo ocupado $\overline{T}\left(t\right)$ y $\overline{T}^{0}\left(t\right)$ son crecientes y continuas con
$\overline{T}\left(0\right)=\overline{T}^{0}\left(0\right)=0$.
\item[ii)] Para todo $t\geq0$
\[\sum_{k=1}^{K}\left[\overline{T}_{k}\left(t\right)+\overline{T}_{k}^{0}\left(t\right)\right]=t\]
\item[iii)] Para todo $1\leq k\leq K$
\[\overline{Q}_{k}\left(t\right)=\overline{Q}_{k}\left(0\right)+\alpha_{k}t-\mu_{k}\overline{T}_{k}\left(t\right)\]
\item[iv)]  Para todo $1\leq k\leq K$
\[\dot{{\overline{T}}}_{k}\left(t\right)=\beta_{k}\] para $\overline{Q}_{k}\left(t\right)=0$.
\item[v)] Para todo $k,j$
\[\mu_{k}^{0}\overline{T}_{k}^{0}\left(t\right)=\mu_{j}^{0}\overline{T}_{j}^{0}\left(t\right)\]
\item[vi)]  Para todo $1\leq k\leq K$
\[\mu_{k}\dot{{\overline{T}}}_{k}\left(t\right)=l_{k}\mu_{k}^{0}\dot{{\overline{T}}}_{k}^{0}\left(t\right)\] para $\overline{Q}_{k}\left(t\right)>0$.
\end{itemize}
\end{Prop}

\begin{Lema}[Lema 3.1 \cite{Chen}]\label{Lema3.1}
Si el modelo de flujo es estable, definido por las ecuaciones
(3.8)-(3.13), entonces el modelo de flujo retrasado tambin es
estable.
\end{Lema}

\begin{Teo}[Teorema 5.2 \cite{Chen}]\label{Tma.5.2}
Si el modelo de flujo lineal correspondiente a la red de cola es
estable, entonces la red de colas es estable.
\end{Teo}

\begin{Teo}[Teorema 5.1 \cite{Chen}]\label{Tma.5.1.Chen}
La red de colas es estable si existe una constante $t_{0}$ que
depende de $\left(\alpha,\mu,T,U\right)$ y $V$ que satisfagan las
ecuaciones (5.1)-(5.5), $Z\left(t\right)=0$, para toda $t\geq
t_{0}$.
\end{Teo}



\begin{Lema}[Lema 5.2 \cite{Gut}]\label{Lema.5.2.Gut}
Sea $\left\{\xi\left(k\right):k\in\ent\right\}$ sucesin de
variables aleatorias i.i.d. con valores en
$\left(0,\infty\right)$, y sea $E\left(t\right)$ el proceso de
conteo
\[E\left(t\right)=max\left\{n\geq1:\xi\left(1\right)+\cdots+\xi\left(n-1\right)\leq t\right\}.\]
Si $E\left[\xi\left(1\right)\right]<\infty$, entonces para
cualquier entero $r\geq1$
\begin{equation}
lim_{t\rightarrow\infty}\esp\left[\left(\frac{E\left(t\right)}{t}\right)^{r}\right]=\left(\frac{1}{E\left[\xi_{1}\right]}\right)^{r}
\end{equation}
de aqu, bajo estas condiciones
\begin{itemize}
\item[a)] Para cualquier $t>0$,
$sup_{t\geq\delta}\esp\left[\left(\frac{E\left(t\right)}{t}\right)^{r}\right]$

\item[b)] Las variables aleatorias
$\left\{\left(\frac{E\left(t\right)}{t}\right)^{r}:t\geq1\right\}$
son uniformemente integrables.
\end{itemize}
\end{Lema}

\begin{Teo}[Teorema 5.1: Ley Fuerte para Procesos de Conteo
\cite{Gut}]\label{Tma.5.1.Gut} Sea
$0<\mu<\esp\left(X_{1}\right]\leq\infty$. entonces

\begin{itemize}
\item[a)] $\frac{N\left(t\right)}{t}\rightarrow\frac{1}{\mu}$
a.s., cuando $t\rightarrow\infty$.


\item[b)]$\esp\left[\frac{N\left(t\right)}{t}\right]^{r}\rightarrow\frac{1}{\mu^{r}}$,
cuando $t\rightarrow\infty$ para todo $r>0$..
\end{itemize}
\end{Teo}


\begin{Prop}[Proposicin 5.1 \cite{DaiSean}]\label{Prop.5.1}
Suponga que los supuestos (A1) y (A2) se cumplen, adems suponga
que el modelo de flujo es estable. Entonces existe $t_{0}>0$ tal
que
\begin{equation}\label{Eq.Prop.5.1}
lim_{|x|\rightarrow\infty}\frac{1}{|x|^{p+1}}\esp_{x}\left[|X\left(t_{0}|x|\right)|^{p+1}\right]=0.
\end{equation}

\end{Prop}


\begin{Prop}[Proposici\'on 5.3 \cite{DaiSean}]
Sea $X$ proceso de estados para la red de colas, y suponga que se
cumplen los supuestos (A1) y (A2), entonces para alguna constante
positiva $C_{p+1}<\infty$, $\delta>0$ y un conjunto compacto
$C\subset X$.

\begin{equation}\label{Eq.5.4}
\esp_{x}\left[\int_{0}^{\tau_{C}\left(\delta\right)}\left(1+|X\left(t\right)|^{p}\right)dt\right]\leq
C_{p+1}\left(1+|x|^{p+1}\right)
\end{equation}
\end{Prop}

\begin{Prop}[Proposici\'on 5.4 \cite{DaiSean}]
Sea $X$ un proceso de Markov Borel Derecho en $X$, sea
$f:X\leftarrow\rea_{+}$ y defina para alguna $\delta>0$, y un
conjunto cerrado $C\subset X$
\[V\left(x\right):=\esp_{x}\left[\int_{0}^{\tau_{C}\left(\delta\right)}f\left(X\left(t\right)\right)dt\right]\]
para $x\in X$. Si $V$ es finito en todas partes y uniformemente
acotada en $C$, entonces existe $k<\infty$ tal que
\begin{equation}\label{Eq.5.11}
\frac{1}{t}\esp_{x}\left[V\left(x\right)\right]+\frac{1}{t}\int_{0}^{t}\esp_{x}\left[f\left(X\left(s\right)\right)ds\right]\leq\frac{1}{t}V\left(x\right)+k,
\end{equation}
para $x\in X$ y $t>0$.
\end{Prop}


\begin{Teo}[Teorema 5.5 \cite{DaiSean}]
Suponga que se cumplen (A1) y (A2), adems suponga que el modelo
de flujo es estable. Entonces existe una constante $k_{p}<\infty$
tal que
\begin{equation}\label{Eq.5.13}
\frac{1}{t}\int_{0}^{t}\esp_{x}\left[|Q\left(s\right)|^{p}\right]ds\leq
k_{p}\left\{\frac{1}{t}|x|^{p+1}+1\right\}
\end{equation}
para $t\geq0$, $x\in X$. En particular para cada condici\'on inicial
\begin{equation}\label{Eq.5.14}
Limsup_{t\rightarrow\infty}\frac{1}{t}\int_{0}^{t}\esp_{x}\left[|Q\left(s\right)|^{p}\right]ds\leq
k_{p}
\end{equation}
\end{Teo}

\begin{Teo}[Teorema 6.2\cite{DaiSean}]\label{Tma.6.2}
Suponga que se cumplen los supuestos (A1)-(A3) y que el modelo de
flujo es estable, entonces se tiene que
\[\parallel P^{t}\left(c,\cdot\right)-\pi\left(\cdot\right)\parallel_{f_{p}}\rightarrow0\]
para $t\rightarrow\infty$ y $x\in X$. En particular para cada
condicin inicial
\[lim_{t\rightarrow\infty}\esp_{x}\left[\left|Q_{t}\right|^{p}\right]=\esp_{\pi}\left[\left|Q_{0}\right|^{p}\right]<\infty\]
\end{Teo}


\begin{Teo}[Teorema 6.3\cite{DaiSean}]\label{Tma.6.3}
Suponga que se cumplen los supuestos (A1)-(A3) y que el modelo de
flujo es estable, entonces con
$f\left(x\right)=f_{1}\left(x\right)$, se tiene que
\[lim_{t\rightarrow\infty}t^{(p-1)\left|P^{t}\left(c,\cdot\right)-\pi\left(\cdot\right)\right|_{f}=0},\]
para $x\in X$. En particular, para cada condicin inicial
\[lim_{t\rightarrow\infty}t^{(p-1)\left|\esp_{x}\left[Q_{t}\right]-\esp_{\pi}\left[Q_{0}\right]\right|=0}.\]
\end{Teo}


\begin{Prop}[Proposici\'on 5.1, Dai y Meyn \cite{DaiSean}]\label{Prop.5.1.DaiSean}
Suponga que los supuestos A1) y A2) son ciertos y que el modelo de flujo es estable. Entonces existe $t_{0}>0$ tal que
\begin{equation}
lim_{|x|\rightarrow\infty}\frac{1}{|x|^{p+1}}\esp_{x}\left[|X\left(t_{0}|x|\right)|^{p+1}\right]=0
\end{equation}
\end{Prop}

\begin{Lemma}[Lema 5.2, Dai y Meyn \cite{DaiSean}]\label{Lema.5.2.DaiSean}
 Sea $\left\{\zeta\left(k\right):k\in \mathbb{z}\right\}$ una sucesi\'on independiente e id\'enticamente distribuida que toma valores en $\left(0,\infty\right)$,
y sea
$E\left(t\right)=max\left(n\geq1:\zeta\left(1\right)+\cdots+\zeta\left(n-1\right)\leq
t\right)$. Si $\esp\left[\zeta\left(1\right)\right]<\infty$,
entonces para cualquier entero $r\geq1$
\begin{equation}
 lim_{t\rightarrow\infty}\esp\left[\left(\frac{E\left(t\right)}{t}\right)^{r}\right]=\left(\frac{1}{\esp\left[\zeta_{1}\right]}\right)^{r}.
\end{equation}
Luego, bajo estas condiciones:
\begin{itemize}
 \item[a)] para cualquier $\delta>0$, $\sup_{t\geq\delta}\esp\left[\left(\frac{E\left(t\right)}{t}\right)^{r}\right]<\infty$
\item[b)] las variables aleatorias
$\left\{\left(\frac{E\left(t\right)}{t}\right)^{r}:t\geq1\right\}$
son uniformemente integrables.
\end{itemize}
\end{Lemma}

\begin{Teo}[Teorema 5.5, Dai y Meyn \cite{DaiSean}]\label{Tma.5.5.DaiSean}
Suponga que los supuestos A1) y A2) se cumplen y que el modelo de
flujo es estable. Entonces existe una constante $\kappa_{p}$ tal
que
\begin{equation}
\frac{1}{t}\int_{0}^{t}\esp_{x}\left[|Q\left(s\right)|^{p}\right]ds\leq\kappa_{p}\left\{\frac{1}{t}|x|^{p+1}+1\right\}
\end{equation}
para $t>0$ y $x\in X$. En particular, para cada condici\'on
inicial
\begin{eqnarray*}
\limsup_{t\rightarrow\infty}\frac{1}{t}\int_{0}^{t}\esp_{x}\left[|Q\left(s\right)|^{p}\right]ds\leq\kappa_{p}.
\end{eqnarray*}
\end{Teo}

\begin{Teo}[Teorema 6.2, Dai y Meyn \cite{DaiSean}]\label{Tma.6.2.DaiSean}
Suponga que se cumplen los supuestos A1), A2) y A3) y que el
modelo de flujo es estable. Entonces se tiene que
\begin{equation}
\left\|P^{t}\left(x,\cdot\right)-\pi\left(\cdot\right)\right\|_{f_{p}}\textrm{,
}t\rightarrow\infty,x\in X.
\end{equation}
En particular para cada condici\'on inicial
\begin{eqnarray*}
\lim_{t\rightarrow\infty}\esp_{x}\left[|Q\left(t\right)|^{p}\right]=\esp_{\pi}\left[|Q\left(0\right)|^{p}\right]\leq\kappa_{r}
\end{eqnarray*}
\end{Teo}
\begin{Teo}[Teorema 6.3, Dai y Meyn \cite{DaiSean}]\label{Tma.6.3.DaiSean}
Suponga que se cumplen los supuestos A1), A2) y A3) y que el
modelo de flujo es estable. Entonces con
$f\left(x\right)=f_{1}\left(x\right)$ se tiene
\begin{equation}
\lim_{t\rightarrow\infty}t^{p-1}\left\|P^{t}\left(x,\cdot\right)-\pi\left(\cdot\right)\right\|_{f}=0.
\end{equation}
En particular para cada condici\'on inicial
\begin{eqnarray*}
\lim_{t\rightarrow\infty}t^{p-1}|\esp_{x}\left[Q\left(t\right)\right]-\esp_{\pi}\left[Q\left(0\right)\right]|=0.
\end{eqnarray*}
\end{Teo}

\begin{Teo}[Teorema 6.4, Dai y Meyn \cite{DaiSean}]\label{Tma.6.4.DaiSean}
Suponga que se cumplen los supuestos A1), A2) y A3) y que el
modelo de flujo es estable. Sea $\nu$ cualquier distribuci\'on de
probabilidad en $\left(X,\mathcal{B}_{X}\right)$, y $\pi$ la
distribuci\'on estacionaria de $X$.
\begin{itemize}
\item[i)] Para cualquier $f:X\leftarrow\rea_{+}$
\begin{equation}
\lim_{t\rightarrow\infty}\frac{1}{t}\int_{o}^{t}f\left(X\left(s\right)\right)ds=\pi\left(f\right):=\int
f\left(x\right)\pi\left(dx\right)
\end{equation}
$\prob$-c.s.

\item[ii)] Para cualquier $f:X\leftarrow\rea_{+}$ con
$\pi\left(|f|\right)<\infty$, la ecuaci\'on anterior se cumple.
\end{itemize}
\end{Teo}

\begin{Teo}[Teorema 2.2, Down \cite{Down}]\label{Tma2.2.Down}
Suponga que el fluido modelo es inestable en el sentido de que
para alguna $\epsilon_{0},c_{0}\geq0$,
\begin{equation}\label{Eq.Inestability}
|Q\left(T\right)|\geq\epsilon_{0}T-c_{0}\textrm{,   }T\geq0,
\end{equation}
para cualquier condici\'on inicial $Q\left(0\right)$, con
$|Q\left(0\right)|=1$. Entonces para cualquier $0<q\leq1$, existe
$B<0$ tal que para cualquier $|x|\geq B$,
\begin{equation}
\prob_{x}\left\{\mathbb{X}\rightarrow\infty\right\}\geq q.
\end{equation}
\end{Teo}



Es necesario hacer los siguientes supuestos sobre el
comportamiento del sistema de visitas c\'iclicas:
\begin{itemize}
\item Los tiempos de interarribo a la $k$-\'esima cola, son de la
forma $\left\{\xi_{k}\left(n\right)\right\}_{n\geq1}$, con la
propiedad de que son independientes e id{\'e}nticamente
distribuidos,
\item Los tiempos de servicio
$\left\{\eta_{k}\left(n\right)\right\}_{n\geq1}$ tienen la
propiedad de ser independientes e id{\'e}nticamente distribuidos,
\item Se define la tasa de arribo a la $k$-{\'e}sima cola como
$\lambda_{k}=1/\esp\left[\xi_{k}\left(1\right)\right]$,
\item la tasa de servicio para la $k$-{\'e}sima cola se define
como $\mu_{k}=1/\esp\left[\eta_{k}\left(1\right)\right]$,
\item tambi{\'e}n se define $\rho_{k}:=\lambda_{k}/\mu_{k}$, la
intensidad de tr\'afico del sistema o carga de la red, donde es
necesario que $\rho<1$ para cuestiones de estabilidad.
\end{itemize}



%_________________________________________________________________________
\subsection{Procesos Fuerte de Markov}
%_________________________________________________________________________
En Dai \cite{Dai} se muestra que para una amplia serie de disciplinas
de servicio el proceso $X$ es un Proceso Fuerte de
Markov, y por tanto se puede asumir que


Para establecer que $X=\left\{X\left(t\right),t\geq0\right\}$ es
un Proceso Fuerte de Markov, se siguen las secciones 2.3 y 2.4 de Kaspi and Mandelbaum \cite{KaspiMandelbaum}. \\

%______________________________________________________________
\subsubsection{Construcci\'on de un Proceso Determinista por partes, Davis
\cite{Davis}}.
%______________________________________________________________

%_________________________________________________________________________
\subsection{Procesos Harris Recurrentes Positivos}
%_________________________________________________________________________
Sea el proceso de Markov $X=\left\{X\left(t\right),t\geq0\right\}$
que describe la din\'amica de la red de colas. En lo que respecta
al supuesto (A3), en Dai y Meyn \cite{DaiSean} y Meyn y Down
\cite{MeynDown} hacen ver que este se puede sustituir por

\begin{itemize}
\item[A3')] Para el Proceso de Markov $X$, cada subconjunto
compacto de $X$ es un conjunto peque\~no.
\end{itemize}

Este supuesto es importante pues es un requisito para deducir la ergodicidad de la red.

%_________________________________________________________________________
\subsection{Construcci\'on de un Modelo de Flujo L\'imite}
%_________________________________________________________________________

Consideremos un caso m\'as simple para poner en contexto lo
anterior: para un sistema de visitas c\'iclicas se tiene que el
estado al tiempo $t$ es
\begin{equation}
X\left(t\right)=\left(Q\left(t\right),U\left(t\right),V\left(t\right)\right),
\end{equation}

donde $Q\left(t\right)$ es el n\'umero de usuarios formados en
cada estaci\'on. $U\left(t\right)$ es el tiempo restante antes de
que la siguiente clase $k$ de usuarios lleguen desde fuera del
sistema, $V\left(t\right)$ es el tiempo restante de servicio para
la clase $k$ de usuarios que est\'an siendo atendidos. Tanto
$U\left(t\right)$ como $V\left(t\right)$ se puede asumir que son
continuas por la derecha.

Sea
$x=\left(Q\left(0\right),U\left(0\right),V\left(0\right)\right)=\left(q,a,b\right)$,
el estado inicial de la red bajo una disciplina espec\'ifica para
la cola. Para $l\in\mathcal{E}$, donde $\mathcal{E}$ es el conjunto de clases de arribos externos, y $k=1,\ldots,K$ se define\\
\begin{eqnarray*}
E_{l}^{x}\left(t\right)&=&max\left\{r:U_{l}\left(0\right)+\xi_{l}\left(1\right)+\cdots+\xi_{l}\left(r-1\right)\leq
t\right\}\textrm{   }t\geq0,\\
S_{k}^{x}\left(t\right)&=&max\left\{r:V_{k}\left(0\right)+\eta_{k}\left(1\right)+\cdots+\eta_{k}\left(r-1\right)\leq
t\right\}\textrm{   }t\geq0.
\end{eqnarray*}

Para cada $k$ y cada $n$ se define

\begin{eqnarray*}\label{Eq.phi}
\Phi^{k}\left(n\right):=\sum_{i=1}^{n}\phi^{k}\left(i\right).
\end{eqnarray*}

donde $\phi^{k}\left(n\right)$ se define como el vector de ruta
para el $n$-\'esimo usuario de la clase $k$ que termina en la
estaci\'on $s\left(k\right)$, la $s$-\'eima componente de
$\phi^{k}\left(n\right)$ es uno si estos usuarios se convierten en
usuarios de la clase $l$ y cero en otro caso, por lo tanto
$\phi^{k}\left(n\right)$ es un vector {\em Bernoulli} de
dimensi\'on $K$ con par\'ametro $P_{k}^{'}$, donde $P_{k}$ denota
el $k$-\'esimo rengl\'on de $P=\left(P_{kl}\right)$.

Se asume que cada para cada $k$ la sucesi\'on $\phi^{k}\left(n\right)=\left\{\phi^{k}\left(n\right),n\geq1\right\}$
es independiente e id\'enticamente distribuida y que las
$\phi^{1}\left(n\right),\ldots,\phi^{K}\left(n\right)$ son
mutuamente independientes, adem\'as de independientes de los
procesos de arribo y de servicio.\\

\begin{Lema}[Lema 4.2, Dai\cite{Dai}]\label{Lema4.2}
Sea $\left\{x_{n}\right\}\subset \mathbf{X}$ con
$|x_{n}|\rightarrow\infty$, conforme $n\rightarrow\infty$. Suponga
que
\[lim_{n\rightarrow\infty}\frac{1}{|x_{n}|}U\left(0\right)=\overline{U}\]
y
\[lim_{n\rightarrow\infty}\frac{1}{|x_{n}|}V\left(0\right)=\overline{V}.\]

Entonces, conforme $n\rightarrow\infty$, casi seguramente

\begin{equation}\label{E1.4.2}
\frac{1}{|x_{n}|}\Phi^{k}\left(\left[|x_{n}|t\right]\right)\rightarrow
P_{k}^{'}t\textrm{, u.o.c.,}
\end{equation}

\begin{equation}\label{E1.4.3}
\frac{1}{|x_{n}|}E^{x_{n}}_{k}\left(|x_{n}|t\right)\rightarrow
\alpha_{k}\left(t-\overline{U}_{k}\right)^{+}\textrm{, u.o.c.,}
\end{equation}

\begin{equation}\label{E1.4.4}
\frac{1}{|x_{n}|}S^{x_{n}}_{k}\left(|x_{n}|t\right)\rightarrow
\mu_{k}\left(t-\overline{V}_{k}\right)^{+}\textrm{, u.o.c.,}
\end{equation}

donde $\left[t\right]$ es la parte entera de $t$ y
$\mu_{k}=1/m_{k}=1/\esp\left[\eta_{k}\left(1\right)\right]$.
\end{Lema}

\begin{Lema}[Lema 4.3, Dai\cite{Dai}]\label{Lema.4.3}
Sea $\left\{x_{n}\right\}\subset \mathbf{X}$ con
$|x_{n}|\rightarrow\infty$, conforme $n\rightarrow\infty$. Suponga
que
\[lim_{n\rightarrow\infty}\frac{1}{|x_{n}|}U\left(0\right)=\overline{U}_{k}\]
y
\[lim_{n\rightarrow\infty}\frac{1}{|x_{n}|}V\left(0\right)=\overline{V}_{k}.\]
\begin{itemize}
\item[a)] Conforme $n\rightarrow\infty$ casi seguramente,
\[lim_{n\rightarrow\infty}\frac{1}{|x_{n}|}U^{x_{n}}_{k}\left(|x_{n}|t\right)=\left(\overline{U}_{k}-t\right)^{+}\textrm{, u.o.c.}\]
y
\[lim_{n\rightarrow\infty}\frac{1}{|x_{n}|}V^{x_{n}}_{k}\left(|x_{n}|t\right)=\left(\overline{V}_{k}-t\right)^{+}.\]

\item[b)] Para cada $t\geq0$ fijo,
\[\left\{\frac{1}{|x_{n}|}U^{x_{n}}_{k}\left(|x_{n}|t\right),|x_{n}|\geq1\right\}\]
y
\[\left\{\frac{1}{|x_{n}|}V^{x_{n}}_{k}\left(|x_{n}|t\right),|x_{n}|\geq1\right\}\]
\end{itemize}
son uniformemente convergentes.
\end{Lema}

$S_{l}^{x}\left(t\right)$ es el n\'umero total de servicios
completados de la clase $l$, si la clase $l$ est\'a dando $t$
unidades de tiempo de servicio. Sea $T_{l}^{x}\left(x\right)$ el
monto acumulado del tiempo de servicio que el servidor
$s\left(l\right)$ gasta en los usuarios de la clase $l$ al tiempo
$t$. Entonces $S_{l}^{x}\left(T_{l}^{x}\left(t\right)\right)$ es
el n\'umero total de servicios completados para la clase $l$ al
tiempo $t$. Una fracci\'on de estos usuarios,
$\Phi_{l}^{x}\left(S_{l}^{x}\left(T_{l}^{x}\left(t\right)\right)\right)$,
se convierte en usuarios de la clase $k$.\\

Entonces, dado lo anterior, se tiene la siguiente representaci\'on
para el proceso de la longitud de la cola:\\

\begin{equation}
Q_{k}^{x}\left(t\right)=_{k}^{x}\left(0\right)+E_{k}^{x}\left(t\right)+\sum_{l=1}^{K}\Phi_{k}^{l}\left(S_{l}^{x}\left(T_{l}^{x}\left(t\right)\right)\right)-S_{k}^{x}\left(T_{k}^{x}\left(t\right)\right)
\end{equation}
para $k=1,\ldots,K$. Para $i=1,\ldots,d$, sea
\[I_{i}^{x}\left(t\right)=t-\sum_{j\in C_{i}}T_{k}^{x}\left(t\right).\]

Entonces $I_{i}^{x}\left(t\right)$ es el monto acumulado del
tiempo que el servidor $i$ ha estado desocupado al tiempo $t$. Se
est\'a asumiendo que las disciplinas satisfacen la ley de
conservaci\'on del trabajo, es decir, el servidor $i$ est\'a en
pausa solamente cuando no hay usuarios en la estaci\'on $i$.
Entonces, se tiene que

\begin{equation}
\int_{0}^{\infty}\left(\sum_{k\in
C_{i}}Q_{k}^{x}\left(t\right)\right)dI_{i}^{x}\left(t\right)=0,
\end{equation}
para $i=1,\ldots,d$.\\

Hacer
\[T^{x}\left(t\right)=\left(T_{1}^{x}\left(t\right),\ldots,T_{K}^{x}\left(t\right)\right)^{'},\]
\[I^{x}\left(t\right)=\left(I_{1}^{x}\left(t\right),\ldots,I_{K}^{x}\left(t\right)\right)^{'}\]
y
\[S^{x}\left(T^{x}\left(t\right)\right)=\left(S_{1}^{x}\left(T_{1}^{x}\left(t\right)\right),\ldots,S_{K}^{x}\left(T_{K}^{x}\left(t\right)\right)\right)^{'}.\]

Para una disciplina que cumple con la ley de conservaci\'on del
trabajo, en forma vectorial, se tiene el siguiente conjunto de
ecuaciones

\begin{equation}\label{Eq.MF.1.3}
Q^{x}\left(t\right)=Q^{x}\left(0\right)+E^{x}\left(t\right)+\sum_{l=1}^{K}\Phi^{l}\left(S_{l}^{x}\left(T_{l}^{x}\left(t\right)\right)\right)-S^{x}\left(T^{x}\left(t\right)\right),\\
\end{equation}

\begin{equation}\label{Eq.MF.2.3}
Q^{x}\left(t\right)\geq0,\\
\end{equation}

\begin{equation}\label{Eq.MF.3.3}
T^{x}\left(0\right)=0,\textrm{ y }\overline{T}^{x}\left(t\right)\textrm{ es no decreciente},\\
\end{equation}

\begin{equation}\label{Eq.MF.4.3}
I^{x}\left(t\right)=et-CT^{x}\left(t\right)\textrm{ es no
decreciente}\\
\end{equation}

\begin{equation}\label{Eq.MF.5.3}
\int_{0}^{\infty}\left(CQ^{x}\left(t\right)\right)dI_{i}^{x}\left(t\right)=0,\\
\end{equation}

\begin{equation}\label{Eq.MF.6.3}
\textrm{Condiciones adicionales en
}\left(\overline{Q}^{x}\left(\cdot\right),\overline{T}^{x}\left(\cdot\right)\right)\textrm{
espec\'ificas de la disciplina de la cola,}
\end{equation}

donde $e$ es un vector de unos de dimensi\'on $d$, $C$ es la
matriz definida por
\[C_{ik}=\left\{\begin{array}{cc}
1,& S\left(k\right)=i,\\
0,& \textrm{ en otro caso}.\\
\end{array}\right.
\]
Es necesario enunciar el siguiente Teorema que se utilizar\'a para
el Teorema \ref{Tma.4.2.Dai}:
\begin{Teo}[Teorema 4.1, Dai \cite{Dai}]
Considere una disciplina que cumpla la ley de conservaci\'on del
trabajo, para casi todas las trayectorias muestrales $\omega$ y
cualquier sucesi\'on de estados iniciales
$\left\{x_{n}\right\}\subset \mathbf{X}$, con
$|x_{n}|\rightarrow\infty$, existe una subsucesi\'on
$\left\{x_{n_{j}}\right\}$ con $|x_{n_{j}}|\rightarrow\infty$ tal
que
\begin{equation}\label{Eq.4.15}
\frac{1}{|x_{n_{j}}|}\left(Q^{x_{n_{j}}}\left(0\right),U^{x_{n_{j}}}\left(0\right),V^{x_{n_{j}}}\left(0\right)\right)\rightarrow\left(\overline{Q}\left(0\right),\overline{U},\overline{V}\right),
\end{equation}

\begin{equation}\label{Eq.4.16}
\frac{1}{|x_{n_{j}}|}\left(Q^{x_{n_{j}}}\left(|x_{n_{j}}|t\right),T^{x_{n_{j}}}\left(|x_{n_{j}}|t\right)\right)\rightarrow\left(\overline{Q}\left(t\right),\overline{T}\left(t\right)\right)\textrm{
u.o.c.}
\end{equation}

Adem\'as,
$\left(\overline{Q}\left(t\right),\overline{T}\left(t\right)\right)$
satisface las siguientes ecuaciones:
\begin{equation}\label{Eq.MF.1.3a}
\overline{Q}\left(t\right)=Q\left(0\right)+\left(\alpha
t-\overline{U}\right)^{+}-\left(I-P\right)^{'}M^{-1}\left(\overline{T}\left(t\right)-\overline{V}\right)^{+},
\end{equation}

\begin{equation}\label{Eq.MF.2.3a}
\overline{Q}\left(t\right)\geq0,\\
\end{equation}

\begin{equation}\label{Eq.MF.3.3a}
\overline{T}\left(t\right)\textrm{ es no decreciente y comienza en cero},\\
\end{equation}

\begin{equation}\label{Eq.MF.4.3a}
\overline{I}\left(t\right)=et-C\overline{T}\left(t\right)\textrm{
es no decreciente,}\\
\end{equation}

\begin{equation}\label{Eq.MF.5.3a}
\int_{0}^{\infty}\left(C\overline{Q}\left(t\right)\right)d\overline{I}\left(t\right)=0,\\
\end{equation}

\begin{equation}\label{Eq.MF.6.3a}
\textrm{Condiciones adicionales en
}\left(\overline{Q}\left(\cdot\right),\overline{T}\left(\cdot\right)\right)\textrm{
especficas de la disciplina de la cola,}
\end{equation}
\end{Teo}

\begin{Def}[Definici\'on 4.1, , Dai \cite{Dai}]
Sea una disciplina de servicio espec\'ifica. Cualquier l\'imite
$\left(\overline{Q}\left(\cdot\right),\overline{T}\left(\cdot\right)\right)$
en \ref{Eq.4.16} es un {\em flujo l\'imite} de la disciplina.
Cualquier soluci\'on (\ref{Eq.MF.1.3a})-(\ref{Eq.MF.6.3a}) es
llamado flujo soluci\'on de la disciplina. Se dice que el modelo de flujo l\'imite, modelo de flujo, de la disciplina de la cola es estable si existe una constante
$\delta>0$ que depende de $\mu,\alpha$ y $P$ solamente, tal que
cualquier flujo l\'imite con
$|\overline{Q}\left(0\right)|+|\overline{U}|+|\overline{V}|=1$, se
tiene que $\overline{Q}\left(\cdot+\delta\right)\equiv0$.
\end{Def}

\begin{Teo}[Teorema 4.2, Dai\cite{Dai}]\label{Tma.4.2.Dai}
Sea una disciplina fija para la cola, suponga que se cumplen las
condiciones (1.2)-(1.5). Si el modelo de flujo l\'imite de la
disciplina de la cola es estable, entonces la cadena de Markov $X$
que describe la din\'amica de la red bajo la disciplina es Harris
recurrente positiva.
\end{Teo}

Ahora se procede a escalar el espacio y el tiempo para reducir la
aparente fluctuaci\'on del modelo. Consid\'erese el proceso
\begin{equation}\label{Eq.3.7}
\overline{Q}^{x}\left(t\right)=\frac{1}{|x|}Q^{x}\left(|x|t\right)
\end{equation}
A este proceso se le conoce como el fluido escalado, y cualquier l\'imite $\overline{Q}^{x}\left(t\right)$ es llamado flujo l\'imite del proceso de longitud de la cola. Haciendo $|q|\rightarrow\infty$ mientras se mantiene el resto de las componentes fijas, cualquier punto l\'imite del proceso de longitud de la cola normalizado $\overline{Q}^{x}$ es soluci\'on del siguiente modelo de flujo.

\begin{Def}[Definici\'on 3.1, Dai y Meyn \cite{DaiSean}]
Un flujo l\'imite (retrasado) para una red bajo una disciplina de
servicio espec\'ifica se define como cualquier soluci\'on
 $\left(\overline{Q}\left(\cdot\right),\overline{T}\left(\cdot\right)\right)$ de las siguientes ecuaciones, donde
$\overline{Q}\left(t\right)=\left(\overline{Q}_{1}\left(t\right),\ldots,\overline{Q}_{K}\left(t\right)\right)^{'}$
y
$\overline{T}\left(t\right)=\left(\overline{T}_{1}\left(t\right),\ldots,\overline{T}_{K}\left(t\right)\right)^{'}$
\begin{equation}\label{Eq.3.8}
\overline{Q}_{k}\left(t\right)=\overline{Q}_{k}\left(0\right)+\alpha_{k}t-\mu_{k}\overline{T}_{k}\left(t\right)+\sum_{l=1}^{k}P_{lk}\mu_{l}\overline{T}_{l}\left(t\right)\\
\end{equation}
\begin{equation}\label{Eq.3.9}
\overline{Q}_{k}\left(t\right)\geq0\textrm{ para }k=1,2,\ldots,K,\\
\end{equation}
\begin{equation}\label{Eq.3.10}
\overline{T}_{k}\left(0\right)=0,\textrm{ y }\overline{T}_{k}\left(\cdot\right)\textrm{ es no decreciente},\\
\end{equation}
\begin{equation}\label{Eq.3.11}
\overline{I}_{i}\left(t\right)=t-\sum_{k\in C_{i}}\overline{T}_{k}\left(t\right)\textrm{ es no decreciente}\\
\end{equation}
\begin{equation}\label{Eq.3.12}
\overline{I}_{i}\left(\cdot\right)\textrm{ se incrementa al tiempo }t\textrm{ cuando }\sum_{k\in C_{i}}Q_{k}^{x}\left(t\right)dI_{i}^{x}\left(t\right)=0\\
\end{equation}
\begin{equation}\label{Eq.3.13}
\textrm{condiciones adicionales sobre
}\left(Q^{x}\left(\cdot\right),T^{x}\left(\cdot\right)\right)\textrm{
referentes a la disciplina de servicio}
\end{equation}
\end{Def}

Al conjunto de ecuaciones dadas en \ref{Eq.3.8}-\ref{Eq.3.13} se
le llama {\em Modelo de flujo} y al conjunto de todas las
soluciones del modelo de flujo
$\left(\overline{Q}\left(\cdot\right),\overline{T}
\left(\cdot\right)\right)$ se le denotar\'a por $\mathcal{Q}$.

Si se hace $|x|\rightarrow\infty$ sin restringir ninguna de las
componentes, tambi\'en se obtienen un modelo de flujo, pero en
este caso el residual de los procesos de arribo y servicio
introducen un retraso:

\begin{Def}[Definici\'on 3.2, Dai y Meyn \cite{DaiSean}]
El modelo de flujo retrasado de una disciplina de servicio en una
red con retraso
$\left(\overline{A}\left(0\right),\overline{B}\left(0\right)\right)\in\rea_{+}^{K+|A|}$
se define como el conjunto de ecuaciones dadas en
\ref{Eq.3.8}-\ref{Eq.3.13}, junto con la condici\'on:
\begin{equation}\label{CondAd.FluidModel}
\overline{Q}\left(t\right)=\overline{Q}\left(0\right)+\left(\alpha
t-\overline{A}\left(0\right)\right)^{+}-\left(I-P^{'}\right)M\left(\overline{T}\left(t\right)-\overline{B}\left(0\right)\right)^{+}
\end{equation}
\end{Def}

\begin{Def}[Definici\'on 3.3, Dai y Meyn \cite{DaiSean}]
El modelo de flujo es estable si existe un tiempo fijo $t_{0}$ tal
que $\overline{Q}\left(t\right)=0$, con $t\geq t_{0}$, para
cualquier $\overline{Q}\left(\cdot\right)\in\mathcal{Q}$ que
cumple con $|\overline{Q}\left(0\right)|=1$.
\end{Def}

El siguiente resultado se encuentra en Chen \cite{Chen}.
\begin{Lemma}[Lema 3.1, Dai y Meyn \cite{DaiSean}]
Si el modelo de flujo definido por \ref{Eq.3.8}-\ref{Eq.3.13} es
estable, entonces el modelo de flujo retrasado es tambi\'en
estable, es decir, existe $t_{0}>0$ tal que
$\overline{Q}\left(t\right)=0$ para cualquier $t\geq t_{0}$, para
cualquier soluci\'on del modelo de flujo retrasado cuya
condici\'on inicial $\overline{x}$ satisface que
$|\overline{x}|=|\overline{Q}\left(0\right)|+|\overline{A}\left(0\right)|+|\overline{B}\left(0\right)|\leq1$.
\end{Lemma}

%_________________________________________________________________________
\subsection{Modelo de Visitas C\'iclicas con un Servidor: Estabilidad}
%_________________________________________________________________________

%_________________________________________________________________________
\subsection{Teorema 2.1}
%_________________________________________________________________________



El resultado principal de Down \cite{Down} que relaciona la estabilidad del modelo de flujo con la estabilidad del sistema original

\begin{Teo}[Teorema 2.1, Down \cite{Down}]\label{Tma.2.1.Down}
Suponga que el modelo de flujo es estable, y que se cumplen los supuestos (A1) y (A2), entonces
\begin{itemize}
\item[i)] Para alguna constante $\kappa_{p}$, y para cada
condici\'on inicial $x\in X$
\begin{equation}\label{Estability.Eq1}
lim_{t\rightarrow\infty}\sup\frac{1}{t}\int_{0}^{t}\esp_{x}\left[|Q\left(s\right)|^{p}\right]ds\leq\kappa_{p},
\end{equation}
donde $p$ es el entero dado en (A2). Si adem\'as se cumple
la condici\'on (A3), entonces para cada condici\'on inicial:

\item[ii)] Los momentos transitorios convergen a su estado estacionario:
 \begin{equation}\label{Estability.Eq2}
lim_{t\rightarrow\infty}\esp_{x}\left[Q_{k}\left(t\right)^{r}\right]=\esp_{\pi}\left[Q_{k}\left(0\right)^{r}\right]\leq\kappa_{r},
\end{equation}
para $r=1,2,\ldots,p$ y $k=1,2,\ldots,K$. Donde $\pi$ es la
probabilidad invariante para $\mathbf{X}$.

\item[iii)]  El primer momento converge con raz\'on $t^{p-1}$:
\begin{equation}\label{Estability.Eq3}
lim_{t\rightarrow\infty}t^{p-1}|\esp_{x}\left[Q_{k}\left(t\right)\right]-\esp_{\pi}\left[Q\left(0\right)\right]=0.
\end{equation}

\item[iv)] La {\em Ley Fuerte de los grandes n\'umeros} se cumple:
\begin{equation}\label{Estability.Eq4}
lim_{t\rightarrow\infty}\frac{1}{t}\int_{0}^{t}Q_{k}^{r}\left(s\right)ds=\esp_{\pi}\left[Q_{k}\left(0\right)^{r}\right],\textrm{
}\prob_{x}\textrm{-c.s.}
\end{equation}
para $r=1,2,\ldots,p$ y $k=1,2,\ldots,K$.
\end{itemize}
\end{Teo}


\begin{Prop}[Proposici\'on 5.1, Dai y Meyn \cite{DaiSean}]\label{Prop.5.1.DaiSean}
Suponga que los supuestos A1) y A2) son ciertos y que el modelo de flujo es estable. Entonces existe $t_{0}>0$ tal que
\begin{equation}
lim_{|x|\rightarrow\infty}\frac{1}{|x|^{p+1}}\esp_{x}\left[|X\left(t_{0}|x|\right)|^{p+1}\right]=0
\end{equation}
\end{Prop}

\begin{Lemma}[Lema 5.2, Dai y Meyn \cite{DaiSean}]\label{Lema.5.2.DaiSean}
 Sea $\left\{\zeta\left(k\right):k\in \mathbb{z}\right\}$ una sucesi\'on independiente e id\'enticamente distribuida que toma valores en $\left(0,\infty\right)$,
y sea
$E\left(t\right)=max\left(n\geq1:\zeta\left(1\right)+\cdots+\zeta\left(n-1\right)\leq
t\right)$. Si $\esp\left[\zeta\left(1\right)\right]<\infty$,
entonces para cualquier entero $r\geq1$
\begin{equation}
 lim_{t\rightarrow\infty}\esp\left[\left(\frac{E\left(t\right)}{t}\right)^{r}\right]=\left(\frac{1}{\esp\left[\zeta_{1}\right]}\right)^{r}.
\end{equation}
Luego, bajo estas condiciones:
\begin{itemize}
 \item[a)] para cualquier $\delta>0$, $\sup_{t\geq\delta}\esp\left[\left(\frac{E\left(t\right)}{t}\right)^{r}\right]<\infty$
\item[b)] las variables aleatorias
$\left\{\left(\frac{E\left(t\right)}{t}\right)^{r}:t\geq1\right\}$
son uniformemente integrables.
\end{itemize}
\end{Lemma}

\begin{Teo}[Teorema 5.5, Dai y Meyn \cite{DaiSean}]\label{Tma.5.5.DaiSean}
Suponga que los supuestos A1) y A2) se cumplen y que el modelo de
flujo es estable. Entonces existe una constante $\kappa_{p}$ tal
que
\begin{equation}
\frac{1}{t}\int_{0}^{t}\esp_{x}\left[|Q\left(s\right)|^{p}\right]ds\leq\kappa_{p}\left\{\frac{1}{t}|x|^{p+1}+1\right\}
\end{equation}
para $t>0$ y $x\in X$. En particular, para cada condici\'on
inicial
\begin{eqnarray*}
\limsup_{t\rightarrow\infty}\frac{1}{t}\int_{0}^{t}\esp_{x}\left[|Q\left(s\right)|^{p}\right]ds\leq\kappa_{p}.
\end{eqnarray*}
\end{Teo}

\begin{Teo}[Teorema 6.2, Dai y Meyn \cite{DaiSean}]\label{Tma.6.2.DaiSean}
Suponga que se cumplen los supuestos A1), A2) y A3) y que el
modelo de flujo es estable. Entonces se tiene que
\begin{equation}
\left\|P^{t}\left(x,\cdot\right)-\pi\left(\cdot\right)\right\|_{f_{p}}\textrm{,
}t\rightarrow\infty,x\in X.
\end{equation}
En particular para cada condici\'on inicial
\begin{eqnarray*}
\lim_{t\rightarrow\infty}\esp_{x}\left[|Q\left(t\right)|^{p}\right]=\esp_{\pi}\left[|Q\left(0\right)|^{p}\right]\leq\kappa_{r}
\end{eqnarray*}
\end{Teo}
\begin{Teo}[Teorema 6.3, Dai y Meyn \cite{DaiSean}]\label{Tma.6.3.DaiSean}
Suponga que se cumplen los supuestos A1), A2) y A3) y que el
modelo de flujo es estable. Entonces con
$f\left(x\right)=f_{1}\left(x\right)$ se tiene
\begin{equation}
\lim_{t\rightarrow\infty}t^{p-1}\left\|P^{t}\left(x,\cdot\right)-\pi\left(\cdot\right)\right\|_{f}=0.
\end{equation}
En particular para cada condici\'on inicial
\begin{eqnarray*}
\lim_{t\rightarrow\infty}t^{p-1}|\esp_{x}\left[Q\left(t\right)\right]-\esp_{\pi}\left[Q\left(0\right)\right]|=0.
\end{eqnarray*}
\end{Teo}

\begin{Teo}[Teorema 6.4, Dai y Meyn \cite{DaiSean}]\label{Tma.6.4.DaiSean}
Suponga que se cumplen los supuestos A1), A2) y A3) y que el
modelo de flujo es estable. Sea $\nu$ cualquier distribuci\'on de
probabilidad en $\left(X,\mathcal{B}_{X}\right)$, y $\pi$ la
distribuci\'on estacionaria de $X$.
\begin{itemize}
\item[i)] Para cualquier $f:X\leftarrow\rea_{+}$
\begin{equation}
\lim_{t\rightarrow\infty}\frac{1}{t}\int_{o}^{t}f\left(X\left(s\right)\right)ds=\pi\left(f\right):=\int
f\left(x\right)\pi\left(dx\right)
\end{equation}
$\prob$-c.s.

\item[ii)] Para cualquier $f:X\leftarrow\rea_{+}$ con
$\pi\left(|f|\right)<\infty$, la ecuaci\'on anterior se cumple.
\end{itemize}
\end{Teo}

%_________________________________________________________________________
\subsection{Teorema 2.2}
%_________________________________________________________________________

\begin{Teo}[Teorema 2.2, Down \cite{Down}]\label{Tma2.2.Down}
Suponga que el fluido modelo es inestable en el sentido de que
para alguna $\epsilon_{0},c_{0}\geq0$,
\begin{equation}\label{Eq.Inestability}
|Q\left(T\right)|\geq\epsilon_{0}T-c_{0}\textrm{,   }T\geq0,
\end{equation}
para cualquier condici\'on inicial $Q\left(0\right)$, con
$|Q\left(0\right)|=1$. Entonces para cualquier $0<q\leq1$, existe
$B<0$ tal que para cualquier $|x|\geq B$,
\begin{equation}
\prob_{x}\left\{\mathbb{X}\rightarrow\infty\right\}\geq q.
\end{equation}
\end{Teo}

%_________________________________________________________________________
\subsection{Teorema 2.3}
%_________________________________________________________________________
\begin{Teo}[Teorema 2.3, Down \cite{Down}]\label{Tma2.3.Down}
Considere el siguiente valor:
\begin{equation}\label{Eq.Rho.1serv}
\rho=\sum_{k=1}^{K}\rho_{k}+max_{1\leq j\leq K}\left(\frac{\lambda_{j}}{\sum_{s=1}^{S}p_{js}\overline{N}_{s}}\right)\delta^{*}
\end{equation}
\begin{itemize}
\item[i)] Si $\rho<1$ entonces la red es estable, es decir, se cumple el teorema \ref{Tma.2.1.Down}.

\item[ii)] Si $\rho<1$ entonces la red es inestable, es decir, se cumple el teorema \ref{Tma2.2.Down}
\end{itemize}
\end{Teo}
%_____________________________________________________________________
\subsection{Definiciones  B\'asicas}
%_____________________________________________________________________
\begin{Def}
Sea $X$ un conjunto y $\mathcal{F}$ una $\sigma$-\'algebra de
subconjuntos de $X$, la pareja $\left(X,\mathcal{F}\right)$ es
llamado espacio medible. Un subconjunto $A$ de $X$ es llamado
medible, o medible con respecto a $\mathcal{F}$, si
$A\in\mathcal{F}$.
\end{Def}

\begin{Def}
Sea $\left(X,\mathcal{F},\mu\right)$ espacio de medida. Se dice
que la medida $\mu$ es $\sigma$-finita si se puede escribir
$X=\bigcup_{n\geq1}X_{n}$ con $X_{n}\in\mathcal{F}$ y
$\mu\left(X_{n}\right)<\infty$.
\end{Def}

\begin{Def}\label{Cto.Borel}
Sea $X$ el conjunto de los \'umeros reales $\rea$. El \'algebra de
Borel es la $\sigma$-\'algebra $B$ generada por los intervalos
abiertos $\left(a,b\right)\in\rea$. Cualquier conjunto en $B$ es
llamado {\em Conjunto de Borel}.
\end{Def}

\begin{Def}\label{Funcion.Medible}
Una funci\'on $f:X\rightarrow\rea$, es medible si para cualquier
n\'umero real $\alpha$ el conjunto
\[\left\{x\in X:f\left(x\right)>\alpha\right\}\]
pertenece a $X$. Equivalentemente, se dice que $f$ es medible si
\[f^{-1}\left(\left(\alpha,\infty\right)\right)=\left\{x\in X:f\left(x\right)>\alpha\right\}\in\mathcal{F}.\]
\end{Def}


\begin{Def}\label{Def.Cilindros}
Sean $\left(\Omega_{i},\mathcal{F}_{i}\right)$, $i=1,2,\ldots,$
espacios medibles y $\Omega=\prod_{i=1}^{\infty}\Omega_{i}$ el
conjunto de todas las sucesiones
$\left(\omega_{1},\omega_{2},\ldots,\right)$ tales que
$\omega_{i}\in\Omega_{i}$, $i=1,2,\ldots,$. Si
$B^{n}\subset\prod_{i=1}^{\infty}\Omega_{i}$, definimos
$B_{n}=\left\{\omega\in\Omega:\left(\omega_{1},\omega_{2},\ldots,\omega_{n}\right)\in
B^{n}\right\}$. Al conjunto $B_{n}$ se le llama {\em cilindro} con
base $B^{n}$, el cilindro es llamado medible si
$B^{n}\in\prod_{i=1}^{\infty}\mathcal{F}_{i}$.
\end{Def}


\begin{Def}\label{Def.Proc.Adaptado}[TSP, Ash \cite{RBA}]
Sea $X\left(t\right),t\geq0$ proceso estoc\'astico, el proceso es
adaptado a la familia de $\sigma$-\'algebras $\mathcal{F}_{t}$,
para $t\geq0$, si para $s<t$ implica que
$\mathcal{F}_{s}\subset\mathcal{F}_{t}$, y $X\left(t\right)$ es
$\mathcal{F}_{t}$-medible para cada $t$. Si no se especifica
$\mathcal{F}_{t}$ entonces se toma $\mathcal{F}_{t}$ como
$\mathcal{F}\left(X\left(s\right),s\leq t\right)$, la m\'as
peque\~na $\sigma$-\'algebra de subconjuntos de $\Omega$ que hace
que cada $X\left(s\right)$, con $s\leq t$ sea Borel medible.
\end{Def}


\begin{Def}\label{Def.Tiempo.Paro}[TSP, Ash \cite{RBA}]
Sea $\left\{\mathcal{F}\left(t\right),t\geq0\right\}$ familia
creciente de sub $\sigma$-\'algebras. es decir,
$\mathcal{F}\left(s\right)\subset\mathcal{F}\left(t\right)$ para
$s\leq t$. Un tiempo de paro para $\mathcal{F}\left(t\right)$ es
una funci\'on $T:\Omega\rightarrow\left[0,\infty\right]$ tal que
$\left\{T\leq t\right\}\in\mathcal{F}\left(t\right)$ para cada
$t\geq0$. Un tiempo de paro para el proceso estoc\'astico
$X\left(t\right),t\geq0$ es un tiempo de paro para las
$\sigma$-\'algebras
$\mathcal{F}\left(t\right)=\mathcal{F}\left(X\left(s\right)\right)$.
\end{Def}

\begin{Def}
Sea $X\left(t\right),t\geq0$ proceso estoc\'astico, con
$\left(S,\chi\right)$ espacio de estados. Se dice que el proceso
es adaptado a $\left\{\mathcal{F}\left(t\right)\right\}$, es
decir, si para cualquier $s,t\in I$, $I$ conjunto de \'indices,
$s<t$, se tiene que
$\mathcal{F}\left(s\right)\subset\mathcal{F}\left(t\right)$ y
$X\left(t\right)$ es $\mathcal{F}\left(t\right)$-medible,
\end{Def}

\begin{Def}
Sea $X\left(t\right),t\geq0$ proceso estoc\'astico, se dice que es
un Proceso de Markov relativo a $\mathcal{F}\left(t\right)$ o que
$\left\{X\left(t\right),\mathcal{F}\left(t\right)\right\}$ es de
Markov si y s\'olo si para cualquier conjunto $B\in\chi$,  y
$s,t\in I$, $s<t$ se cumple que
\begin{equation}\label{Prop.Markov}
P\left\{X\left(t\right)\in
B|\mathcal{F}\left(s\right)\right\}=P\left\{X\left(t\right)\in
B|X\left(s\right)\right\}.
\end{equation}
\end{Def}
\begin{Note}
Si se dice que $\left\{X\left(t\right)\right\}$ es un Proceso de
Markov sin mencionar $\mathcal{F}\left(t\right)$, se asumir\'a que
\begin{eqnarray*}
\mathcal{F}\left(t\right)=\mathcal{F}_{0}\left(t\right)=\mathcal{F}\left(X\left(r\right),r\leq
t\right),
\end{eqnarray*}
entonces la ecuaci\'on (\ref{Prop.Markov}) se puede escribir como
\begin{equation}
P\left\{X\left(t\right)\in B|X\left(r\right),r\leq s\right\} =
P\left\{X\left(t\right)\in B|X\left(s\right)\right\}
\end{equation}
\end{Note}

\begin{Teo}
Sea $\left(X_{n},\mathcal{F}_{n},n=0,1,\ldots,\right\}$ Proceso de
Markov con espacio de estados $\left(S_{0},\chi_{0}\right)$
generado por una distribuici\'on inicial $P_{o}$ y probabilidad de
transici\'on $p_{mn}$, para $m,n=0,1,\ldots,$ $m<n$, que por
notaci\'on se escribir\'a como $p\left(m,n,x,B\right)\rightarrow
p_{mn}\left(x,B\right)$. Sea $S$ tiempo de paro relativo a la
$\sigma$-\'algebra $\mathcal{F}_{n}$. Sea $T$ funci\'on medible,
$T:\Omega\rightarrow\left\{0,1,\ldots,\right\}$. Sup\'ongase que
$T\geq S$, entonces $T$ es tiempo de paro. Si $B\in\chi_{0}$,
entonces
\begin{equation}\label{Prop.Fuerte.Markov}
P\left\{X\left(T\right)\in
B,T<\infty|\mathcal{F}\left(S\right)\right\} =
p\left(S,T,X\left(s\right),B\right)
\end{equation}
en $\left\{T<\infty\right\}$.
\end{Teo}

Propiedades importantes para el modelo de flujo retrasado:

\begin{Prop}
 Sea $\left(\overline{Q},\overline{T},\overline{T}^{0}\right)$ un flujo l\'imite de \ref{Equation.4.4} y suponga que cuando $x\rightarrow\infty$ a lo largo de
una subsucesi\'on
\[\left(\frac{1}{|x|}Q_{k}^{x}\left(0\right),\frac{1}{|x|}A_{k}^{x}\left(0\right),\frac{1}{|x|}B_{k}^{x}\left(0\right),\frac{1}{|x|}B_{k}^{x,0}\left(0\right)\right)\rightarrow\left(\overline{Q}_{k}\left(0\right),0,0,0\right)\]
para $k=1,\ldots,K$. EL flujo l\'imite tiene las siguientes
propiedades, donde las propiedades de la derivada se cumplen donde
la derivada exista:
\begin{itemize}
 \item[i)] Los vectores de tiempo ocupado $\overline{T}\left(t\right)$ y $\overline{T}^{0}\left(t\right)$ son crecientes y continuas con
$\overline{T}\left(0\right)=\overline{T}^{0}\left(0\right)=0$.
\item[ii)] Para todo $t\geq0$
\[\sum_{k=1}^{K}\left[\overline{T}_{k}\left(t\right)+\overline{T}_{k}^{0}\left(t\right)\right]=t\]
\item[iii)] Para todo $1\leq k\leq K$
\[\overline{Q}_{k}\left(t\right)=\overline{Q}_{k}\left(0\right)+\alpha_{k}t-\mu_{k}\overline{T}_{k}\left(t\right)\]
\item[iv)]  Para todo $1\leq k\leq K$
\[\dot{{\overline{T}}}_{k}\left(t\right)=\beta_{k}\] para $\overline{Q}_{k}\left(t\right)=0$.
\item[v)] Para todo $k,j$
\[\mu_{k}^{0}\overline{T}_{k}^{0}\left(t\right)=\mu_{j}^{0}\overline{T}_{j}^{0}\left(t\right)\]
\item[vi)]  Para todo $1\leq k\leq K$
\[\mu_{k}\dot{{\overline{T}}}_{k}\left(t\right)=l_{k}\mu_{k}^{0}\dot{{\overline{T}}}_{k}^{0}\left(t\right)\] para $\overline{Q}_{k}\left(t\right)>0$.
\end{itemize}
\end{Prop}

\begin{Lema}[Lema 3.1 \cite{Chen}]\label{Lema3.1}
Si el modelo de flujo es estable, definido por las ecuaciones
(3.8)-(3.13), entonces el modelo de flujo retrasado tambin es
estable.
\end{Lema}

\begin{Teo}[Teorema 5.2 \cite{Chen}]\label{Tma.5.2}
Si el modelo de flujo lineal correspondiente a la red de cola es
estable, entonces la red de colas es estable.
\end{Teo}

\begin{Teo}[Teorema 5.1 \cite{Chen}]\label{Tma.5.1.Chen}
La red de colas es estable si existe una constante $t_{0}$ que
depende de $\left(\alpha,\mu,T,U\right)$ y $V$ que satisfagan las
ecuaciones (5.1)-(5.5), $Z\left(t\right)=0$, para toda $t\geq
t_{0}$.
\end{Teo}



\begin{Lema}[Lema 5.2 \cite{Gut}]\label{Lema.5.2.Gut}
Sea $\left\{\xi\left(k\right):k\in\ent\right\}$ sucesin de
variables aleatorias i.i.d. con valores en
$\left(0,\infty\right)$, y sea $E\left(t\right)$ el proceso de
conteo
\[E\left(t\right)=max\left\{n\geq1:\xi\left(1\right)+\cdots+\xi\left(n-1\right)\leq t\right\}.\]
Si $E\left[\xi\left(1\right)\right]<\infty$, entonces para
cualquier entero $r\geq1$
\begin{equation}
lim_{t\rightarrow\infty}\esp\left[\left(\frac{E\left(t\right)}{t}\right)^{r}\right]=\left(\frac{1}{E\left[\xi_{1}\right]}\right)^{r}
\end{equation}
de aqu, bajo estas condiciones
\begin{itemize}
\item[a)] Para cualquier $t>0$,
$sup_{t\geq\delta}\esp\left[\left(\frac{E\left(t\right)}{t}\right)^{r}\right]$

\item[b)] Las variables aleatorias
$\left\{\left(\frac{E\left(t\right)}{t}\right)^{r}:t\geq1\right\}$
son uniformemente integrables.
\end{itemize}
\end{Lema}

\begin{Teo}[Teorema 5.1: Ley Fuerte para Procesos de Conteo
\cite{Gut}]\label{Tma.5.1.Gut} Sea
$0<\mu<\esp\left(X_{1}\right]\leq\infty$. entonces

\begin{itemize}
\item[a)] $\frac{N\left(t\right)}{t}\rightarrow\frac{1}{\mu}$
a.s., cuando $t\rightarrow\infty$.


\item[b)]$\esp\left[\frac{N\left(t\right)}{t}\right]^{r}\rightarrow\frac{1}{\mu^{r}}$,
cuando $t\rightarrow\infty$ para todo $r>0$..
\end{itemize}
\end{Teo}


\begin{Prop}[Proposicin 5.1 \cite{DaiSean}]\label{Prop.5.1}
Suponga que los supuestos (A1) y (A2) se cumplen, adems suponga
que el modelo de flujo es estable. Entonces existe $t_{0}>0$ tal
que
\begin{equation}\label{Eq.Prop.5.1}
lim_{|x|\rightarrow\infty}\frac{1}{|x|^{p+1}}\esp_{x}\left[|X\left(t_{0}|x|\right)|^{p+1}\right]=0.
\end{equation}

\end{Prop}


\begin{Prop}[Proposici\'on 5.3 \cite{DaiSean}]
Sea $X$ proceso de estados para la red de colas, y suponga que se
cumplen los supuestos (A1) y (A2), entonces para alguna constante
positiva $C_{p+1}<\infty$, $\delta>0$ y un conjunto compacto
$C\subset X$.

\begin{equation}\label{Eq.5.4}
\esp_{x}\left[\int_{0}^{\tau_{C}\left(\delta\right)}\left(1+|X\left(t\right)|^{p}\right)dt\right]\leq
C_{p+1}\left(1+|x|^{p+1}\right)
\end{equation}
\end{Prop}

\begin{Prop}[Proposici\'on 5.4 \cite{DaiSean}]
Sea $X$ un proceso de Markov Borel Derecho en $X$, sea
$f:X\leftarrow\rea_{+}$ y defina para alguna $\delta>0$, y un
conjunto cerrado $C\subset X$
\[V\left(x\right):=\esp_{x}\left[\int_{0}^{\tau_{C}\left(\delta\right)}f\left(X\left(t\right)\right)dt\right]\]
para $x\in X$. Si $V$ es finito en todas partes y uniformemente
acotada en $C$, entonces existe $k<\infty$ tal que
\begin{equation}\label{Eq.5.11}
\frac{1}{t}\esp_{x}\left[V\left(x\right)\right]+\frac{1}{t}\int_{0}^{t}\esp_{x}\left[f\left(X\left(s\right)\right)ds\right]\leq\frac{1}{t}V\left(x\right)+k,
\end{equation}
para $x\in X$ y $t>0$.
\end{Prop}


\begin{Teo}[Teorema 5.5 \cite{DaiSean}]
Suponga que se cumplen (A1) y (A2), adems suponga que el modelo
de flujo es estable. Entonces existe una constante $k_{p}<\infty$
tal que
\begin{equation}\label{Eq.5.13}
\frac{1}{t}\int_{0}^{t}\esp_{x}\left[|Q\left(s\right)|^{p}\right]ds\leq
k_{p}\left\{\frac{1}{t}|x|^{p+1}+1\right\}
\end{equation}
para $t\geq0$, $x\in X$. En particular para cada condicin inicial
\begin{equation}\label{Eq.5.14}
Limsup_{t\rightarrow\infty}\frac{1}{t}\int_{0}^{t}\esp_{x}\left[|Q\left(s\right)|^{p}\right]ds\leq
k_{p}
\end{equation}
\end{Teo}

\begin{Teo}[Teorema 6.2\cite{DaiSean}]\label{Tma.6.2}
Suponga que se cumplen los supuestos (A1)-(A3) y que el modelo de
flujo es estable, entonces se tiene que
\[\parallel P^{t}\left(c,\cdot\right)-\pi\left(\cdot\right)\parallel_{f_{p}}\rightarrow0\]
para $t\rightarrow\infty$ y $x\in X$. En particular para cada
condicin inicial
\[lim_{t\rightarrow\infty}\esp_{x}\left[\left|Q_{t}\right|^{p}\right]=\esp_{\pi}\left[\left|Q_{0}\right|^{p}\right]<\infty\]
\end{Teo}


\begin{Teo}[Teorema 6.3\cite{DaiSean}]\label{Tma.6.3}
Suponga que se cumplen los supuestos (A1)-(A3) y que el modelo de
flujo es estable, entonces con
$f\left(x\right)=f_{1}\left(x\right)$, se tiene que
\[lim_{t\rightarrow\infty}t^{(p-1)\left|P^{t}\left(c,\cdot\right)-\pi\left(\cdot\right)\right|_{f}=0},\]
para $x\in X$. En particular, para cada condicin inicial
\[lim_{t\rightarrow\infty}t^{(p-1)\left|\esp_{x}\left[Q_{t}\right]-\esp_{\pi}\left[Q_{0}\right]\right|=0}.\]
\end{Teo}



Si $x$ es el n{\'u}mero de usuarios en la cola al comienzo del
periodo de servicio y $N_{s}\left(x\right)=N\left(x\right)$ es el
n{\'u}mero de usuarios que son atendidos con la pol{\'\i}tica $s$,
{\'u}nica en nuestro caso, durante un periodo de servicio,
entonces se asume que:
\begin{itemize}
\item[(S1.)]
\begin{equation}\label{S1}
lim_{x\rightarrow\infty}\esp\left[N\left(x\right)\right]=\overline{N}>0.
\end{equation}
\item[(S2.)]
\begin{equation}\label{S2}
\esp\left[N\left(x\right)\right]\leq \overline{N}, \end{equation}
para cualquier valor de $x$. \item La $n$-{\'e}sima ocurrencia va
acompa{\~n}ada con el tiempo de cambio de longitud
$\delta_{j,j+1}\left(n\right)$, independientes e id{\'e}nticamente
distribuidas, con
$\esp\left[\delta_{j,j+1}\left(1\right)\right]\geq0$. \item Se
define
\begin{equation}
\delta^{*}:=\sum_{j,j+1}\esp\left[\delta_{j,j+1}\left(1\right)\right].
\end{equation}

\item Los tiempos de inter-arribo a la cola $k$,son de la forma
$\left\{\xi_{k}\left(n\right)\right\}_{n\geq1}$, con la propiedad
de que son independientes e id{\'e}nticamente distribuidos.

\item Los tiempos de servicio
$\left\{\eta_{k}\left(n\right)\right\}_{n\geq1}$ tienen la
propiedad de ser independientes e id{\'e}nticamente distribuidos.

\item Se define la tasa de arribo a la $k$-{\'e}sima cola como
$\lambda_{k}=1/\esp\left[\xi_{k}\left(1\right)\right]$ y
adem{\'a}s se define

\item la tasa de servicio para la $k$-{\'e}sima cola como
$\mu_{k}=1/\esp\left[\eta_{k}\left(1\right)\right]$

\item tambi{\'e}n se define $\rho_{k}=\lambda_{k}/\mu_{k}$, donde
es necesario que $\rho<1$ para cuestiones de estabilidad.

\item De las pol{\'\i}ticas posibles solamente consideraremos la
pol{\'\i}tica cerrada (Gated).
\end{itemize}

Las Colas C\'iclicas se pueden describir por medio de un proceso
de Markov $\left(X\left(t\right)\right)_{t\in\rea}$, donde el
estado del sistema al tiempo $t\geq0$ est\'a dado por
\begin{equation}
X\left(t\right)=\left(Q\left(t\right),A\left(t\right),H\left(t\right),B\left(t\right),B^{0}\left(t\right),C\left(t\right)\right)
\end{equation}
definido en el espacio producto:
\begin{equation}
\mathcal{X}=\mathbb{Z}^{K}\times\rea_{+}^{K}\times\left(\left\{1,2,\ldots,K\right\}\times\left\{1,2,\ldots,S\right\}\right)^{M}\times\rea_{+}^{K}\times\rea_{+}^{K}\times\mathbb{Z}^{K},
\end{equation}

\begin{itemize}
\item $Q\left(t\right)=\left(Q_{k}\left(t\right),1\leq k\leq
K\right)$, es el n\'umero de usuarios en la cola $k$, incluyendo
aquellos que est\'an siendo atendidos provenientes de la
$k$-\'esima cola.

\item $A\left(t\right)=\left(A_{k}\left(t\right),1\leq k\leq
K\right)$, son los residuales de los tiempos de arribo en la cola
$k$. \item $H\left(t\right)$ es el par ordenado que consiste en la
cola que esta siendo atendida y la pol\'itica de servicio que se
utilizar\'a.

\item $B\left(t\right)$ es el tiempo de servicio residual.

\item $B^{0}\left(t\right)$ es el tiempo residual del cambio de
cola.

\item $C\left(t\right)$ indica el n\'umero de usuarios atendidos
durante la visita del servidor a la cola dada en
$H\left(t\right)$.
\end{itemize}

$A_{k}\left(t\right),B_{m}\left(t\right)$ y
$B_{m}^{0}\left(t\right)$ se suponen continuas por la derecha y
que satisfacen la propiedad fuerte de Markov, (\cite{Dai})

\begin{itemize}
\item Los tiempos de interarribo a la cola $k$,son de la forma
$\left\{\xi_{k}\left(n\right)\right\}_{n\geq1}$, con la propiedad
de que son independientes e id{\'e}nticamente distribuidos.

\item Los tiempos de servicio
$\left\{\eta_{k}\left(n\right)\right\}_{n\geq1}$ tienen la
propiedad de ser independientes e id{\'e}nticamente distribuidos.

\item Se define la tasa de arribo a la $k$-{\'e}sima cola como
$\lambda_{k}=1/\esp\left[\xi_{k}\left(1\right)\right]$ y
adem{\'a}s se define

\item la tasa de servicio para la $k$-{\'e}sima cola como
$\mu_{k}=1/\esp\left[\eta_{k}\left(1\right)\right]$

\item tambi{\'e}n se define $\rho_{k}=\lambda_{k}/\mu_{k}$, donde
es necesario que $\rho<1$ para cuestiones de estabilidad.

\item De las pol{\'\i}ticas posibles solamente consideraremos la
pol{\'\i}tica cerrada (Gated).
\end{itemize}


%_____________________________________________________


\subsection{Preliminares}



Sup\'ongase que el sistema consta de varias colas a los cuales
llegan uno o varios servidores a dar servicio a los usuarios
esperando en la cola.\\


Si $x$ es el n\'umero de usuarios en la cola al comienzo del
periodo de servicio y $N_{s}\left(x\right)=N\left(x\right)$ es el
n\'umero de usuarios que son atendidos con la pol\'itica $s$,
\'unica en nuestro caso, durante un periodo de servicio, entonces
se asume que:
\begin{itemize}
\item[1)]\label{S1}$lim_{x\rightarrow\infty}\esp\left[N\left(x\right)\right]=\overline{N}>0$
\item[2)]\label{S2}$\esp\left[N\left(x\right)\right]\leq\overline{N}$para
cualquier valor de $x$.
\end{itemize}
La manera en que atiende el servidor $m$-\'esimo, en este caso en
espec\'ifico solo lo ilustraremos con un s\'olo servidor, es la
siguiente:
\begin{itemize}
\item Al t\'ermino de la visita a la cola $j$, el servidor se
cambia a la cola $j^{'}$ con probabilidad
$r_{j,j^{'}}^{m}=r_{j,j^{'}}$

\item La $n$-\'esima ocurrencia va acompa\~nada con el tiempo de
cambio de longitud $\delta_{j,j^{'}}\left(n\right)$,
independientes e id\'enticamente distribuidas, con
$\esp\left[\delta_{j,j^{'}}\left(1\right)\right]\geq0$.

\item Sea $\left\{p_{j}\right\}$ la distribuci\'on invariante
estacionaria \'unica para la Cadena de Markov con matriz de
transici\'on $\left(r_{j,j^{'}}\right)$.

\item Finalmente, se define
\begin{equation}
\delta^{*}:=\sum_{j,j^{'}}p_{j}r_{j,j^{'}}\esp\left[\delta_{j,j^{'}}\left(i\right)\right].
\end{equation}
\end{itemize}

Veamos un caso muy espec\'ifico en el cual los tiempos de arribo a cada una de las colas se comportan de acuerdo a un proceso Poisson de la forma
$\left\{\xi_{k}\left(n\right)\right\}_{n\geq1}$, y los tiempos de servicio en cada una de las colas son variables aleatorias distribuidas exponencialmente e id\'enticamente distribuidas
$\left\{\eta_{k}\left(n\right)\right\}_{n\geq1}$, donde ambos procesos adem\'as cumplen la condici\'on de ser independientes entre si. Para la $k$-\'esima cola se define la tasa de arribo a la como
$\lambda_{k}=1/\esp\left[\xi_{k}\left(1\right)\right]$ y la tasa
de servicio como
$\mu_{k}=1/\esp\left[\eta_{k}\left(1\right)\right]$, finalmente se
define la carga de la cola como $\rho_{k}=\lambda_{k}/\mu_{k}$,
donde se pide que $\rho<1$, para garantizar la estabilidad del sistema.\\

Se denotar\'a por $Q_{k}\left(t\right)$ el n\'umero de usuarios en la cola $k$,
$A_{k}\left(t\right)$ los residuales de los tiempos entre arribos a la cola $k$;
para cada servidor $m$, se denota por $B_{m}\left(t\right)$ los residuales de los tiempos de servicio al tiempo $t$; $B_{m}^{0}\left(t\right)$ son los residuales de los tiempos de traslado de la cola $k$ a la pr\'oxima por atender, al tiempo $t$, finalmente sea $C_{m}\left(t\right)$ el n\'umero de usuarios atendidos durante la visita del servidor a la cola $k$ al tiempo $t$.\\


En este sentido el proceso para el sistema de visitas se puede definir como:

\begin{equation}\label{Esp.Edos.Down}
X\left(t\right)^{T}=\left(Q_{k}\left(t\right),A_{k}\left(t\right),B_{m}\left(t\right),B_{m}^{0}\left(t\right),C_{m}\left(t\right)\right)
\end{equation}
para $k=1,\ldots,K$ y $m=1,2,\ldots,M$. $X$ evoluciona en el
espacio de estados:
$X=\ent_{+}^{K}\times\rea_{+}^{K}\times\left(\left\{1,2,\ldots,K\right\}\times\left\{1,2,\ldots,S\right\}\right)^{M}\times\rea_{+}^{K}\times\ent_{+}^{K}$.\\

El sistema aqu\'i descrito debe de cumplir con los siguientes supuestos b\'asicos de un sistema de visitas:

Antes enunciemos los supuestos que regir\'an en la red.

\begin{itemize}
\item[A1)] $\xi_{1},\ldots,\xi_{K},\eta_{1},\ldots,\eta_{K}$ son
mutuamente independientes y son sucesiones independientes e
id\'enticamente distribuidas.

\item[A2)] Para alg\'un entero $p\geq1$
\begin{eqnarray*}
\esp\left[\xi_{l}\left(1\right)^{p+1}\right]<\infty\textrm{ para }l\in\mathcal{A}\textrm{ y }\\
\esp\left[\eta_{k}\left(1\right)^{p+1}\right]<\infty\textrm{ para
}k=1,\ldots,K.
\end{eqnarray*}
donde $\mathcal{A}$ es la clase de posibles arribos.

\item[A3)] Para $k=1,2,\ldots,K$ existe una funci\'on positiva
$q_{k}\left(x\right)$ definida en $\rea_{+}$, y un entero $j_{k}$,
tal que
\begin{eqnarray}
P\left(\xi_{k}\left(1\right)\geq x\right)>0\textrm{, para todo }x>0\\
P\left\{a\leq\sum_{i=1}^{j_{k}}\xi_{k}\left(i\right)\leq
b\right\}\geq\int_{a}^{b}q_{k}\left(x\right)dx, \textrm{ }0\leq
a<b.
\end{eqnarray}
\end{itemize}

En particular los procesos de tiempo entre arribos y de servicio
considerados con fines de ilustraci\'on de la metodolog\'ia
cumplen con el supuesto $A2)$ para $p=1$, es decir, ambos procesos
tienen primer y segundo momento finito.

En lo que respecta al supuesto (A3), en Dai y Meyn \cite{DaiSean}
hacen ver que este se puede sustituir por

\begin{itemize}
\item[A3')] Para el Proceso de Markov $X$, cada subconjunto
compacto de $X$ es un conjunto peque\~no, ver definici\'on
\ref{Def.Cto.Peq.}.
\end{itemize}

Es por esta raz\'on que con la finalidad de poder hacer uso de
$A3^{'})$ es necesario recurrir a los Procesos de Harris y en
particular a los Procesos Harris Recurrente:
%_______________________________________________________________________
\subsection{Procesos Harris Recurrente}
%_______________________________________________________________________

Por el supuesto (A1) conforme a Davis \cite{Davis}, se puede
definir el proceso de saltos correspondiente de manera tal que
satisfaga el supuesto (\ref{Sup3.1.Davis}), de hecho la
demostraci\'on est\'a basada en la l\'inea de argumentaci\'on de
Davis, (\cite{Davis}, p\'aginas 362-364).

Entonces se tiene un espacio de estados Markoviano. El espacio de
Markov descrito en Dai y Meyn \cite{DaiSean}

\[\left(\Omega,\mathcal{F},\mathcal{F}_{t},X\left(t\right),\theta_{t},P_{x}\right)\]
es un proceso de Borel Derecho (Sharpe \cite{Sharpe}) en el
espacio de estados medible $\left(X,\mathcal{B}_{X}\right)$. El
Proceso $X=\left\{X\left(t\right),t\geq0\right\}$ tiene
trayectorias continuas por la derecha, est\'a definida en
$\left(\Omega,\mathcal{F}\right)$ y est\'a adaptado a
$\left\{\mathcal{F}_{t},t\geq0\right\}$; la colecci\'on
$\left\{P_{x},x\in \mathbb{X}\right\}$ son medidas de probabilidad
en $\left(\Omega,\mathcal{F}\right)$ tales que para todo $x\in
\mathbb{X}$
\[P_{x}\left\{X\left(0\right)=x\right\}=1\] y
\[E_{x}\left\{f\left(X\circ\theta_{t}\right)|\mathcal{F}_{t}\right\}=E_{X}\left(\tau\right)f\left(X\right)\]
en $\left\{\tau<\infty\right\}$, $P_{x}$-c.s. Donde $\tau$ es un
$\mathcal{F}_{t}$-tiempo de paro
\[\left(X\circ\theta_{\tau}\right)\left(w\right)=\left\{X\left(\tau\left(w\right)+t,w\right),t\geq0\right\}\]
y $f$ es una funci\'on de valores reales acotada y medible con la
$\sigma$-algebra de Kolmogorov generada por los cilindros.\\

Sea $P^{t}\left(x,D\right)$, $D\in\mathcal{B}_{\mathbb{X}}$,
$t\geq0$ probabilidad de transici\'on de $X$ definida como
\[P^{t}\left(x,D\right)=P_{x}\left(X\left(t\right)\in
D\right)\]


\begin{Def}
Una medida no cero $\pi$ en
$\left(\mathbf{X},\mathcal{B}_{\mathbf{X}}\right)$ es {\bf
invariante} para $X$ si $\pi$ es $\sigma$-finita y
\[\pi\left(D\right)=\int_{\mathbf{X}}P^{t}\left(x,D\right)\pi\left(dx\right)\]
para todo $D\in \mathcal{B}_{\mathbf{X}}$, con $t\geq0$.
\end{Def}

\begin{Def}
El proceso de Markov $X$ es llamado Harris recurrente si existe
una medida de probabilidad $\nu$ en
$\left(\mathbf{X},\mathcal{B}_{\mathbf{X}}\right)$, tal que si
$\nu\left(D\right)>0$ y $D\in\mathcal{B}_{\mathbf{X}}$
\[P_{x}\left\{\tau_{D}<\infty\right\}\equiv1\] cuando
$\tau_{D}=inf\left\{t\geq0:X_{t}\in D\right\}$.
\end{Def}

\begin{Note}
\begin{itemize}
\item[i)] Si $X$ es Harris recurrente, entonces existe una \'unica
medida invariante $\pi$ (Getoor \cite{Getoor}).

\item[ii)] Si la medida invariante es finita, entonces puede
normalizarse a una medida de probabilidad, en este caso se le
llama Proceso {\em Harris recurrente positivo}.


\item[iii)] Cuando $X$ es Harris recurrente positivo se dice que
la disciplina de servicio es estable. En este caso $\pi$ denota la
distribuci\'on estacionaria y hacemos
\[P_{\pi}\left(\cdot\right)=\int_{\mathbf{X}}P_{x}\left(\cdot\right)\pi\left(dx\right)\]
y se utiliza $E_{\pi}$ para denotar el operador esperanza
correspondiente.
\end{itemize}
\end{Note}

\begin{Def}\label{Def.Cto.Peq.}
Un conjunto $D\in\mathcal{B_{\mathbf{X}}}$ es llamado peque\~no si
existe un $t>0$, una medida de probabilidad $\nu$ en
$\mathcal{B_{\mathbf{X}}}$, y un $\delta>0$ tal que
\[P^{t}\left(x,A\right)\geq\delta\nu\left(A\right)\] para $x\in
D,A\in\mathcal{B_{X}}$.
\end{Def}

La siguiente serie de resultados vienen enunciados y demostrados
en Dai \cite{Dai}:
\begin{Lema}[Lema 3.1, Dai\cite{Dai}]
Sea $B$ conjunto peque\~no cerrado, supongamos que
$P_{x}\left(\tau_{B}<\infty\right)\equiv1$ y que para alg\'un
$\delta>0$ se cumple que
\begin{equation}\label{Eq.3.1}
\sup\esp_{x}\left[\tau_{B}\left(\delta\right)\right]<\infty,
\end{equation}
donde
$\tau_{B}\left(\delta\right)=inf\left\{t\geq\delta:X\left(t\right)\in
B\right\}$. Entonces, $X$ es un proceso Harris Recurrente
Positivo.
\end{Lema}

\begin{Lema}[Lema 3.1, Dai \cite{Dai}]\label{Lema.3.}
Bajo el supuesto (A3), el conjunto $B=\left\{|x|\leq k\right\}$ es
un conjunto peque\~no cerrado para cualquier $k>0$.
\end{Lema}

\begin{Teo}[Teorema 3.1, Dai\cite{Dai}]\label{Tma.3.1}
Si existe un $\delta>0$ tal que
\begin{equation}
lim_{|x|\rightarrow\infty}\frac{1}{|x|}\esp|X^{x}\left(|x|\delta\right)|=0,
\end{equation}
entonces la ecuaci\'on (\ref{Eq.3.1}) se cumple para
$B=\left\{|x|\leq k\right\}$ con alg\'un $k>0$. En particular, $X$
es Harris Recurrente Positivo.
\end{Teo}

\begin{Note}
En Meyn and Tweedie \cite{MeynTweedie} muestran que si
$P_{x}\left\{\tau_{D}<\infty\right\}\equiv1$ incluso para solo un
conjunto peque\~no, entonces el proceso es Harris Recurrente.
\end{Note}

Entonces, tenemos que el proceso $X$ es un proceso de Markov que
cumple con los supuestos $A1)$-$A3)$, lo que falta de hacer es
construir el Modelo de Flujo bas\'andonos en lo hasta ahora
presentado.
%_______________________________________________________________________
\subsection{Modelo de Flujo}
%_______________________________________________________________________

Dada una condici\'on inicial $x\in\textrm{X}$, sea
$Q_{k}^{x}\left(t\right)$ la longitud de la cola al tiempo $t$,
$T_{m,k}^{x}\left(t\right)$ el tiempo acumulado, al tiempo $t$,
que tarda el servidor $m$ en atender a los usuarios de la cola
$k$. Finalmente sea $T_{m,k}^{x,0}\left(t\right)$ el tiempo
acumulado, al tiempo $t$, que tarda el servidor $m$ en trasladarse
a otra cola a partir de la $k$-\'esima.\\

Sup\'ongase que la funci\'on
$\left(\overline{Q}\left(\cdot\right),\overline{T}_{m}
\left(\cdot\right),\overline{T}_{m}^{0} \left(\cdot\right)\right)$
para $m=1,2,\ldots,M$ es un punto l\'imite de
\begin{equation}\label{Eq.Punto.Limite}
\left(\frac{1}{|x|}Q^{x}\left(|x|t\right),\frac{1}{|x|}T_{m}^{x}\left(|x|t\right),\frac{1}{|x|}T_{m}^{x,0}\left(|x|t\right)\right)
\end{equation}
para $m=1,2,\ldots,M$, cuando $x\rightarrow\infty$. Entonces
$\left(\overline{Q}\left(t\right),\overline{T}_{m}
\left(t\right),\overline{T}_{m}^{0} \left(t\right)\right)$ es un
flujo l\'imite del sistema. Al conjunto de todos las posibles
flujos l\'imite se le llama \textbf{Modelo de Flujo}.\\

El modelo de flujo satisface el siguiente conjunto de ecuaciones:

\begin{equation}\label{Eq.MF.1}
\overline{Q}_{k}\left(t\right)=\overline{Q}_{k}\left(0\right)+\lambda_{k}t-\sum_{m=1}^{M}\mu_{k}\overline{T}_{m,k}\left(t\right)\\
\end{equation}
para $k=1,2,\ldots,K$.\\
\begin{equation}\label{Eq.MF.2}
\overline{Q}_{k}\left(t\right)\geq0\textrm{ para
}k=1,2,\ldots,K,\\
\end{equation}

\begin{equation}\label{Eq.MF.3}
\overline{T}_{m,k}\left(0\right)=0,\textrm{ y }\overline{T}_{m,k}\left(\cdot\right)\textrm{ es no decreciente},\\
\end{equation}
para $k=1,2,\ldots,K$ y $m=1,2,\ldots,M$,\\
\begin{equation}\label{Eq.MF.4}
\sum_{k=1}^{K}\overline{T}_{m,k}^{0}\left(t\right)+\overline{T}_{m,k}\left(t\right)=t\textrm{
para }m=1,2,\ldots,M.\\
\end{equation}

De acuerdo a Dai \cite{Dai}, se tiene que el conjunto de posibles
l\'imites
$\left(\overline{Q}\left(\cdot\right),\overline{T}\left(\cdot\right),\overline{T}^{0}\left(\cdot\right)\right)$,
en el sentido de que deben de satisfacer las ecuaciones
(\ref{Eq.MF.1})-(\ref{Eq.MF.4}), se le llama {\em Modelo de
Flujo}.


\begin{Def}[Definici\'on 4.1, , Dai \cite{Dai}]\label{Def.Modelo.Flujo}
Sea una disciplina de servicio espec\'ifica. Cualquier l\'imite
$\left(\overline{Q}\left(\cdot\right),\overline{T}\left(\cdot\right)\right)$
en (\ref{Eq.Punto.Limite}) es un {\em flujo l\'imite} de la
disciplina. Cualquier soluci\'on (\ref{Eq.MF.1})-(\ref{Eq.MF.4})
es llamado flujo soluci\'on de la disciplina. Se dice que el
modelo de flujo l\'imite, modelo de flujo, de la disciplina de la
cola es estable si existe una constante $\delta>0$ que depende de
$\mu,\lambda$ y $P$ solamente, tal que cualquier flujo l\'imite
con
$|\overline{Q}\left(0\right)|+|\overline{U}|+|\overline{V}|=1$, se
tiene que $\overline{Q}\left(\cdot+\delta\right)\equiv0$.
\end{Def}

Al conjunto de ecuaciones dadas en \ref{Eq.MF.1}-\ref{Eq.MF.4} se
le llama {\em Modelo de flujo} y al conjunto de todas las
soluciones del modelo de flujo
$\left(\overline{Q}\left(\cdot\right),\overline{T}
\left(\cdot\right)\right)$ se le denotar\'a por $\mathcal{Q}$.

Si se hace $|x|\rightarrow\infty$ sin restringir ninguna de las
componentes, tambi\'en se obtienen un modelo de flujo, pero en
este caso el residual de los procesos de arribo y servicio
introducen un retraso:
\begin{Teo}[Teorema 4.2, Dai\cite{Dai}]\label{Tma.4.2.Dai}
Sea una disciplina fija para la cola, suponga que se cumplen las
condiciones (A1))-(A3)). Si el modelo de flujo l\'imite de la
disciplina de la cola es estable, entonces la cadena de Markov $X$
que describe la din\'amica de la red bajo la disciplina es Harris
recurrente positiva.
\end{Teo}

Ahora se procede a escalar el espacio y el tiempo para reducir la
aparente fluctuaci\'on del modelo. Consid\'erese el proceso
\begin{equation}\label{Eq.3.7}
\overline{Q}^{x}\left(t\right)=\frac{1}{|x|}Q^{x}\left(|x|t\right)
\end{equation}
A este proceso se le conoce como el fluido escalado, y cualquier
l\'imite $\overline{Q}^{x}\left(t\right)$ es llamado flujo
l\'imite del proceso de longitud de la cola. Haciendo
$|q|\rightarrow\infty$ mientras se mantiene el resto de las
componentes fijas, cualquier punto l\'imite del proceso de
longitud de la cola normalizado $\overline{Q}^{x}$ es soluci\'on
del siguiente modelo de flujo.


\begin{Def}[Definici\'on 3.3, Dai y Meyn \cite{DaiSean}]
El modelo de flujo es estable si existe un tiempo fijo $t_{0}$ tal
que $\overline{Q}\left(t\right)=0$, con $t\geq t_{0}$, para
cualquier $\overline{Q}\left(\cdot\right)\in\mathcal{Q}$ que
cumple con $|\overline{Q}\left(0\right)|=1$.
\end{Def}

El siguiente resultado se encuentra en Chen \cite{Chen}.
\begin{Lemma}[Lema 3.1, Dai y Meyn \cite{DaiSean}]
Si el modelo de flujo definido por \ref{Eq.MF.1}-\ref{Eq.MF.4} es
estable, entonces el modelo de flujo retrasado es tambi\'en
estable, es decir, existe $t_{0}>0$ tal que
$\overline{Q}\left(t\right)=0$ para cualquier $t\geq t_{0}$, para
cualquier soluci\'on del modelo de flujo retrasado cuya
condici\'on inicial $\overline{x}$ satisface que
$|\overline{x}|=|\overline{Q}\left(0\right)|+|\overline{A}\left(0\right)|+|\overline{B}\left(0\right)|\leq1$.
\end{Lemma}


Ahora ya estamos en condiciones de enunciar los resultados principales:


\begin{Teo}[Teorema 2.1, Down \cite{Down}]\label{Tma2.1.Down}
Suponga que el modelo de flujo es estable, y que se cumplen los supuestos (A1) y (A2), entonces
\begin{itemize}
\item[i)] Para alguna constante $\kappa_{p}$, y para cada
condici\'on inicial $x\in X$
\begin{equation}\label{Estability.Eq1}
limsup_{t\rightarrow\infty}\frac{1}{t}\int_{0}^{t}\esp_{x}\left[|Q\left(s\right)|^{p}\right]ds\leq\kappa_{p},
\end{equation}
donde $p$ es el entero dado en (A2).
\end{itemize}
Si adem\'as se cumple la condici\'on (A3), entonces para cada
condici\'on inicial:
\begin{itemize}
\item[ii)] Los momentos transitorios convergen a su estado
estacionario:
 \begin{equation}\label{Estability.Eq2}
lim_{t\rightarrow\infty}\esp_{x}\left[Q_{k}\left(t\right)^{r}\right]=\esp_{\pi}\left[Q_{k}\left(0\right)^{r}\right]\leq\kappa_{r},
\end{equation}
para $r=1,2,\ldots,p$ y $k=1,2,\ldots,K$. Donde $\pi$ es la
probabilidad invariante para $\mathbf{X}$.

\item[iii)]  El primer momento converge con raz\'on $t^{p-1}$:
\begin{equation}\label{Estability.Eq3}
lim_{t\rightarrow\infty}t^{p-1}|\esp_{x}\left[Q_{k}\left(t\right)\right]-\esp_{\pi}\left[Q_{k}\left(0\right)\right]=0.
\end{equation}

\item[iv)] La {\em Ley Fuerte de los grandes n\'umeros} se cumple:
\begin{equation}\label{Estability.Eq4}
lim_{t\rightarrow\infty}\frac{1}{t}\int_{0}^{t}Q_{k}^{r}\left(s\right)ds=\esp_{\pi}\left[Q_{k}\left(0\right)^{r}\right],\textrm{
}\prob_{x}\textrm{-c.s.}
\end{equation}
para $r=1,2,\ldots,p$ y $k=1,2,\ldots,K$.
\end{itemize}
\end{Teo}

La contribuci\'on de Down a la teor\'ia de los Sistemas de Visitas
C\'iclicas, es la relaci\'on que hay entre la estabilidad del
sistema con el comportamiento de las medidas de desempe\~no, es
decir, la condici\'on suficiente para poder garantizar la
convergencia del proceso de la longitud de la cola as\'i como de
por los menos los dos primeros momentos adem\'as de una versi\'on
de la Ley Fuerte de los Grandes N\'umeros para los sistemas de
visitas.


\begin{Teo}[Teorema 2.3, Down \cite{Down}]\label{Tma2.3.Down}
Considere el siguiente valor:
\begin{equation}\label{Eq.Rho.1serv}
\rho=\sum_{k=1}^{K}\rho_{k}+max_{1\leq j\leq K}\left(\frac{\lambda_{j}}{\sum_{s=1}^{S}p_{js}\overline{N}_{s}}\right)\delta^{*}
\end{equation}
\begin{itemize}
\item[i)] Si $\rho<1$ entonces la red es estable, es decir, se cumple el teorema \ref{Tma2.1.Down}.

\item[ii)] Si $\rho<1$ entonces la red es inestable, es decir, se cumple el teorema \ref{Tma2.2.Down}
\end{itemize}
\end{Teo}

\begin{Teo}
Sea $\left(X_{n},\mathcal{F}_{n},n=0,1,\ldots,\right\}$ Proceso de
Markov con espacio de estados $\left(S_{0},\chi_{0}\right)$
generado por una distribuici\'on inicial $P_{o}$ y probabilidad de
transici\'on $p_{mn}$, para $m,n=0,1,\ldots,$ $m<n$, que por
notaci\'on se escribir\'a como $p\left(m,n,x,B\right)\rightarrow
p_{mn}\left(x,B\right)$. Sea $S$ tiempo de paro relativo a la
$\sigma$-\'algebra $\mathcal{F}_{n}$. Sea $T$ funci\'on medible,
$T:\Omega\rightarrow\left\{0,1,\ldots,\right\}$. Sup\'ongase que
$T\geq S$, entonces $T$ es tiempo de paro. Si $B\in\chi_{0}$,
entonces
\begin{equation}\label{Prop.Fuerte.Markov}
P\left\{X\left(T\right)\in
B,T<\infty|\mathcal{F}\left(S\right)\right\} =
p\left(S,T,X\left(s\right),B\right)
\end{equation}
en $\left\{T<\infty\right\}$.
\end{Teo}


Sea $K$ conjunto numerable y sea $d:K\rightarrow\nat$ funci\'on.
Para $v\in K$, $M_{v}$ es un conjunto abierto de
$\rea^{d\left(v\right)}$. Entonces \[E=\cup_{v\in
K}M_{v}=\left\{\left(v,\zeta\right):v\in K,\zeta\in
M_{v}\right\}.\]

Sea $\mathcal{E}$ la clase de conjuntos medibles en $E$:
\[\mathcal{E}=\left\{\cup_{v\in K}A_{v}:A_{v}\in \mathcal{M}_{v}\right\}.\]

donde $\mathcal{M}$ son los conjuntos de Borel de $M_{v}$.
Entonces $\left(E,\mathcal{E}\right)$ es un espacio de Borel. El
estado del proceso se denotar\'a por
$\mathbf{x}_{t}=\left(v_{t},\zeta_{t}\right)$. La distribuci\'on
de $\left(\mathbf{x}_{t}\right)$ est\'a determinada por por los
siguientes objetos:

\begin{itemize}
\item[i)] Los campos vectoriales $\left(\mathcal{H}_{v},v\in
K\right)$. \item[ii)] Una funci\'on medible $\lambda:E\rightarrow
\rea_{+}$. \item[iii)] Una medida de transici\'on
$Q:\mathcal{E}\times\left(E\cup\Gamma^{*}\right)\rightarrow\left[0,1\right]$
donde
\begin{equation}
\Gamma^{*}=\cup_{v\in K}\partial^{*}M_{v}.
\end{equation}
y
\begin{equation}
\partial^{*}M_{v}=\left\{z\in\partial M_{v}:\mathbf{\mathbf{\phi}_{v}\left(t,\zeta\right)=\mathbf{z}}\textrm{ para alguna }\left(t,\zeta\right)\in\rea_{+}\times M_{v}\right\}.
\end{equation}
$\partial M_{v}$ denota  la frontera de $M_{v}$.
\end{itemize}

El campo vectorial $\left(\mathcal{H}_{v},v\in K\right)$ se supone
tal que para cada $\mathbf{z}\in M_{v}$ existe una \'unica curva
integral $\mathbf{\phi}_{v}\left(t,\zeta\right)$ que satisface la
ecuaci\'on

\begin{equation}
\frac{d}{dt}f\left(\zeta_{t}\right)=\mathcal{H}f\left(\zeta_{t}\right),
\end{equation}
con $\zeta_{0}=\mathbf{z}$, para cualquier funci\'on suave
$f:\rea^{d}\rightarrow\rea$ y $\mathcal{H}$ denota el operador
diferencial de primer orden, con $\mathcal{H}=\mathcal{H}_{v}$ y
$\zeta_{t}=\mathbf{\phi}\left(t,\mathbf{z}\right)$. Adem\'as se
supone que $\mathcal{H}_{v}$ es conservativo, es decir, las curvas
integrales est\'an definidas para todo $t>0$.

Para $\mathbf{x}=\left(v,\zeta\right)\in E$ se denota
\[t^{*}\mathbf{x}=inf\left\{t>0:\mathbf{\phi}_{v}\left(t,\zeta\right)\in\partial^{*}M_{v}\right\}\]

En lo que respecta a la funci\'on $\lambda$, se supondr\'a que
para cada $\left(v,\zeta\right)\in E$ existe un $\epsilon>0$ tal
que la funci\'on
$s\rightarrow\lambda\left(v,\phi_{v}\left(s,\zeta\right)\right)\in
E$ es integrable para $s\in\left[0,\epsilon\right)$. La medida de
transici\'on $Q\left(A;\mathbf{x}\right)$ es una funci\'on medible
de $\mathbf{x}$ para cada $A\in\mathcal{E}$, definida para
$\mathbf{x}\in E\cup\Gamma^{*}$ y es una medida de probabilidad en
$\left(E,\mathcal{E}\right)$ para cada $\mathbf{x}\in E$.

El movimiento del proceso $\left(\mathbf{x}_{t}\right)$ comenzando
en $\mathbf{x}=\left(n,\mathbf{z}\right)\in E$ se puede construir
de la siguiente manera, def\'inase la funci\'on $F$ por

\begin{equation}
F\left(t\right)=\left\{\begin{array}{ll}\\
exp\left(-\int_{0}^{t}\lambda\left(n,\phi_{n}\left(s,\mathbf{z}\right)\right)ds\right), & t<t^{*}\left(\mathbf{x}\right),\\
0, & t\geq t^{*}\left(\mathbf{x}\right)
\end{array}\right.
\end{equation}

Sea $T_{1}$ una variable aleatoria tal que
$\prob\left[T_{1}>t\right]=F\left(t\right)$, ahora sea la variable
aleatoria $\left(N,Z\right)$ con distribuici\'on
$Q\left(\cdot;\phi_{n}\left(T_{1},\mathbf{z}\right)\right)$. La
trayectoria de $\left(\mathbf{x}_{t}\right)$ para $t\leq T_{1}$
es\footnote{Revisar p\'agina 362, y 364 de Davis \cite{Davis}.}
\begin{eqnarray*}
\mathbf{x}_{t}=\left(v_{t},\zeta_{t}\right)=\left\{\begin{array}{ll}
\left(n,\phi_{n}\left(t,\mathbf{z}\right)\right), & t<T_{1},\\
\left(N,\mathbf{Z}\right), & t=t_{1}.
\end{array}\right.
\end{eqnarray*}

Comenzando en $\mathbf{x}_{T_{1}}$ se selecciona el siguiente
tiempo de intersalto $T_{2}-T_{1}$ lugar del post-salto
$\mathbf{x}_{T_{2}}$ de manera similar y as\'i sucesivamente. Este
procedimiento nos da una trayectoria determinista por partes
$\mathbf{x}_{t}$ con tiempos de salto $T_{1},T_{2},\ldots$. Bajo
las condiciones enunciadas para $\lambda,T_{1}>0$  y
$T_{1}-T_{2}>0$ para cada $i$, con probabilidad 1. Se supone que
se cumple la siguiente condici\'on.

\begin{Sup}[Supuesto 3.1, Davis \cite{Davis}]\label{Sup3.1.Davis}
Sea $N_{t}:=\sum_{t}\indora_{\left(t\geq t\right)}$ el n\'umero de
saltos en $\left[0,t\right]$. Entonces
\begin{equation}
\esp\left[N_{t}\right]<\infty\textrm{ para toda }t.
\end{equation}
\end{Sup}

es un proceso de Markov, m\'as a\'un, es un Proceso Fuerte de
Markov, es decir, la Propiedad Fuerte de Markov se cumple para
cualquier tiempo de paro.


Sea $E$ es un espacio m\'etrico separable y la m\'etrica $d$ es
compatible con la topolog\'ia.


\begin{Def}
Un espacio topol\'ogico $E$ es llamado de {\em Rad\'on} si es
homeomorfo a un subconjunto universalmente medible de un espacio
m\'etrico compacto.
\end{Def}

Equivalentemente, la definici\'on de un espacio de Rad\'on puede
encontrarse en los siguientes t\'erminos:


\begin{Def}
$E$ es un espacio de Rad\'on si cada medida finita en
$\left(E,\mathcal{B}\left(E\right)\right)$ es regular interior o
cerrada, {\em tight}.
\end{Def}

\begin{Def}
Una medida finita, $\lambda$ en la $\sigma$-\'algebra de Borel de
un espacio metrizable $E$ se dice cerrada si
\begin{equation}\label{Eq.A2.3}
\lambda\left(E\right)=sup\left\{\lambda\left(K\right):K\textrm{ es
compacto en }E\right\}.
\end{equation}
\end{Def}

El siguiente teorema nos permite tener una mejor caracterizaci\'on
de los espacios de Rad\'on:
\begin{Teo}\label{Tma.A2.2}
Sea $E$ espacio separable metrizable. Entonces $E$ es Radoniano si
y s\'olo s\'i cada medida finita en
$\left(E,\mathcal{B}\left(E\right)\right)$ es cerrada.
\end{Teo}

Sea $E$ espacio de estados, tal que $E$ es un espacio de Rad\'on,
$\mathcal{B}\left(E\right)$ $\sigma$-\'algebra de Borel en $E$,
que se denotar\'a por $\mathcal{E}$.

Sea $\left(X,\mathcal{G},\prob\right)$ espacio de probabilidad,
$I\subset\rea$ conjunto de \'indices. Sea $\mathcal{F}_{\leq t}$
la $\sigma$-\'algebra natural definida como
$\sigma\left\{f\left(X_{r}\right):r\in I, r\leq
t,f\in\mathcal{E}\right\}$. Se considerar\'a una
$\sigma$-\'algebra m\'as general, $ \left(\mathcal{G}_{t}\right)$
tal que $\left(X_{t}\right)$ sea $\mathcal{E}$-adaptado.

\begin{Def}
Una familia $\left(P_{s,t}\right)$ de kernels de Markov en
$\left(E,\mathcal{E}\right)$ indexada por pares $s,t\in I$, con
$s\leq t$ es una funci\'on de transici\'on en $\ER$, si  para todo
$r\leq s< t$ en $I$ y todo $x\in E$,
$B\in\mathcal{E}$\footnote{Ecuaci\'on de Chapman-Kolmogorov}
\begin{equation}\label{Eq.Kernels}
P_{r,t}\left(x,B\right)=\int_{E}P_{r,s}\left(x,dy\right)P_{s,t}\left(y,B\right).
\end{equation}
\end{Def}

Se dice que la funci\'on de transici\'on $\KM$ en $\ER$ es la
funci\'on de transici\'on para un proceso $\PE$  con valores en
$E$ y que satisface la propiedad de
Markov\footnote{\begin{equation}\label{Eq.1.4.S}
\prob\left\{H|\mathcal{G}_{t}\right\}=\prob\left\{H|X_{t}\right\}\textrm{
}H\in p\mathcal{F}_{\geq t}.
\end{equation}} (\ref{Eq.1.4.S}) relativa a $\left(\mathcal{G}_{t}\right)$ si

\begin{equation}\label{Eq.1.6.S}
\prob\left\{f\left(X_{t}\right)|\mathcal{G}_{s}\right\}=P_{s,t}f\left(X_{t}\right)\textrm{
}s\leq t\in I,\textrm{ }f\in b\mathcal{E}.
\end{equation}

\begin{Def}
Una familia $\left(P_{t}\right)_{t\geq0}$ de kernels de Markov en
$\ER$ es llamada {\em Semigrupo de Transici\'on de Markov} o {\em
Semigrupo de Transici\'on} si
\[P_{t+s}f\left(x\right)=P_{t}\left(P_{s}f\right)\left(x\right),\textrm{ }t,s\geq0,\textrm{ }x\in E\textrm{ }f\in b\mathcal{E}.\]
\end{Def}
\begin{Note}
Si la funci\'on de transici\'on $\KM$ es llamada homog\'enea si
$P_{s,t}=P_{t-s}$.
\end{Note}

Un proceso de Markov que satisface la ecuaci\'on (\ref{Eq.1.6.S})
con funci\'on de transici\'on homog\'enea $\left(P_{t}\right)$
tiene la propiedad caracter\'istica
\begin{equation}\label{Eq.1.8.S}
\prob\left\{f\left(X_{t+s}\right)|\mathcal{G}_{t}\right\}=P_{s}f\left(X_{t}\right)\textrm{
}t,s\geq0,\textrm{ }f\in b\mathcal{E}.
\end{equation}
La ecuaci\'on anterior es la {\em Propiedad Simple de Markov} de
$X$ relativa a $\left(P_{t}\right)$.

En este sentido el proceso $\PE$ cumple con la propiedad de Markov
(\ref{Eq.1.8.S}) relativa a
$\left(\Omega,\mathcal{G},\mathcal{G}_{t},\prob\right)$ con
semigrupo de transici\'on $\left(P_{t}\right)$.

\begin{Def}
Un proceso estoc\'astico $\PE$ definido en
$\left(\Omega,\mathcal{G},\prob\right)$ con valores en el espacio
topol\'ogico $E$ es continuo por la derecha si cada trayectoria
muestral $t\rightarrow X_{t}\left(w\right)$ es un mapeo continuo
por la derecha de $I$ en $E$.
\end{Def}

\begin{Def}[HD1]\label{Eq.2.1.S}
Un semigrupo de Markov $\left(P_{t}\right)$ en un espacio de
Rad\'on $E$ se dice que satisface la condici\'on {\em HD1} si,
dada una medida de probabilidad $\mu$ en $E$, existe una
$\sigma$-\'algebra $\mathcal{E^{*}}$ con
$\mathcal{E}\subset\mathcal{E}^{*}$ y
$P_{t}\left(b\mathcal{E}^{*}\right)\subset b\mathcal{E}^{*}$, y un
$\mathcal{E}^{*}$-proceso $E$-valuado continuo por la derecha
$\PE$ en alg\'un espacio de probabilidad filtrado
$\left(\Omega,\mathcal{G},\mathcal{G}_{t},\prob\right)$ tal que
$X=\left(\Omega,\mathcal{G},\mathcal{G}_{t},\prob\right)$ es de
Markov (Homog\'eneo) con semigrupo de transici\'on $(P_{t})$ y
distribuci\'on inicial $\mu$.
\end{Def}

Consid\'erese la colecci\'on de variables aleatorias $X_{t}$
definidas en alg\'un espacio de probabilidad, y una colecci\'on de
medidas $\mathbf{P}^{x}$ tales que
$\mathbf{P}^{x}\left\{X_{0}=x\right\}$, y bajo cualquier
$\mathbf{P}^{x}$, $X_{t}$ es de Markov con semigrupo
$\left(P_{t}\right)$. $\mathbf{P}^{x}$ puede considerarse como la
distribuci\'on condicional de $\mathbf{P}$ dado $X_{0}=x$.

\begin{Def}\label{Def.2.2.S}
Sea $E$ espacio de Rad\'on, $\SG$ semigrupo de Markov en $\ER$. La
colecci\'on
$\mathbf{X}=\left(\Omega,\mathcal{G},\mathcal{G}_{t},X_{t},\theta_{t},\CM\right)$
es un proceso $\mathcal{E}$-Markov continuo por la derecha simple,
con espacio de estados $E$ y semigrupo de transici\'on $\SG$ en
caso de que $\mathbf{X}$ satisfaga las siguientes condiciones:
\begin{itemize}
\item[i)] $\left(\Omega,\mathcal{G},\mathcal{G}_{t}\right)$ es un
espacio de medida filtrado, y $X_{t}$ es un proceso $E$-valuado
continuo por la derecha $\mathcal{E}^{*}$-adaptado a
$\left(\mathcal{G}_{t}\right)$;

\item[ii)] $\left(\theta_{t}\right)_{t\geq0}$ es una colecci\'on
de operadores {\em shift} para $X$, es decir, mapea $\Omega$ en
s\'i mismo satisfaciendo para $t,s\geq0$,

\begin{equation}\label{Eq.Shift}
\theta_{t}\circ\theta_{s}=\theta_{t+s}\textrm{ y
}X_{t}\circ\theta_{t}=X_{t+s};
\end{equation}

\item[iii)] Para cualquier $x\in E$,$\CM\left\{X_{0}=x\right\}=1$,
y el proceso $\PE$ tiene la propiedad de Markov (\ref{Eq.1.8.S})
con semigrupo de transici\'on $\SG$ relativo a
$\left(\Omega,\mathcal{G},\mathcal{G}_{t},\CM\right)$.
\end{itemize}
\end{Def}

\begin{Def}[HD2]\label{Eq.2.2.S}
Para cualquier $\alpha>0$ y cualquier $f\in S^{\alpha}$, el
proceso $t\rightarrow f\left(X_{t}\right)$ es continuo por la
derecha casi seguramente.
\end{Def}

\begin{Def}\label{Def.PD}
Un sistema
$\mathbf{X}=\left(\Omega,\mathcal{G},\mathcal{G}_{t},X_{t},\theta_{t},\CM\right)$
es un proceso derecho en el espacio de Rad\'on $E$ con semigrupo
de transici\'on $\SG$ provisto de:
\begin{itemize}
\item[i)] $\mathbf{X}$ es una realizaci\'on  continua por la
derecha, \ref{Def.2.2.S}, de $\SG$.

\item[ii)] $\mathbf{X}$ satisface la condicion HD2,
\ref{Eq.2.2.S}, relativa a $\mathcal{G}_{t}$.

\item[iii)] $\mathcal{G}_{t}$ es aumentado y continuo por la
derecha.
\end{itemize}
\end{Def}

\begin{Lema}[Lema 4.2, Dai\cite{Dai}]\label{Lema4.2}
Sea $\left\{x_{n}\right\}\subset \mathbf{X}$ con
$|x_{n}|\rightarrow\infty$, conforme $n\rightarrow\infty$. Suponga
que
\[lim_{n\rightarrow\infty}\frac{1}{|x_{n}|}U\left(0\right)=\overline{U}\]
y
\[lim_{n\rightarrow\infty}\frac{1}{|x_{n}|}V\left(0\right)=\overline{V}.\]

Entonces, conforme $n\rightarrow\infty$, casi seguramente

\begin{equation}\label{E1.4.2}
\frac{1}{|x_{n}|}\Phi^{k}\left(\left[|x_{n}|t\right]\right)\rightarrow
P_{k}^{'}t\textrm{, u.o.c.,}
\end{equation}

\begin{equation}\label{E1.4.3}
\frac{1}{|x_{n}|}E^{x_{n}}_{k}\left(|x_{n}|t\right)\rightarrow
\alpha_{k}\left(t-\overline{U}_{k}\right)^{+}\textrm{, u.o.c.,}
\end{equation}

\begin{equation}\label{E1.4.4}
\frac{1}{|x_{n}|}S^{x_{n}}_{k}\left(|x_{n}|t\right)\rightarrow
\mu_{k}\left(t-\overline{V}_{k}\right)^{+}\textrm{, u.o.c.,}
\end{equation}

donde $\left[t\right]$ es la parte entera de $t$ y
$\mu_{k}=1/m_{k}=1/\esp\left[\eta_{k}\left(1\right)\right]$.
\end{Lema}

\begin{Lema}[Lema 4.3, Dai\cite{Dai}]\label{Lema.4.3}
Sea $\left\{x_{n}\right\}\subset \mathbf{X}$ con
$|x_{n}|\rightarrow\infty$, conforme $n\rightarrow\infty$. Suponga
que
\[lim_{n\rightarrow\infty}\frac{1}{|x_{n}|}U\left(0\right)=\overline{U}_{k}\]
y
\[lim_{n\rightarrow\infty}\frac{1}{|x_{n}|}V\left(0\right)=\overline{V}_{k}.\]
\begin{itemize}
\item[a)] Conforme $n\rightarrow\infty$ casi seguramente,
\[lim_{n\rightarrow\infty}\frac{1}{|x_{n}|}U^{x_{n}}_{k}\left(|x_{n}|t\right)=\left(\overline{U}_{k}-t\right)^{+}\textrm{, u.o.c.}\]
y
\[lim_{n\rightarrow\infty}\frac{1}{|x_{n}|}V^{x_{n}}_{k}\left(|x_{n}|t\right)=\left(\overline{V}_{k}-t\right)^{+}.\]

\item[b)] Para cada $t\geq0$ fijo,
\[\left\{\frac{1}{|x_{n}|}U^{x_{n}}_{k}\left(|x_{n}|t\right),|x_{n}|\geq1\right\}\]
y
\[\left\{\frac{1}{|x_{n}|}V^{x_{n}}_{k}\left(|x_{n}|t\right),|x_{n}|\geq1\right\}\]
\end{itemize}
son uniformemente convergentes.
\end{Lema}

$S_{l}^{x}\left(t\right)$ es el n\'umero total de servicios
completados de la clase $l$, si la clase $l$ est\'a dando $t$
unidades de tiempo de servicio. Sea $T_{l}^{x}\left(x\right)$ el
monto acumulado del tiempo de servicio que el servidor
$s\left(l\right)$ gasta en los usuarios de la clase $l$ al tiempo
$t$. Entonces $S_{l}^{x}\left(T_{l}^{x}\left(t\right)\right)$ es
el n\'umero total de servicios completados para la clase $l$ al
tiempo $t$. Una fracci\'on de estos usuarios,
$\Phi_{l}^{x}\left(S_{l}^{x}\left(T_{l}^{x}\left(t\right)\right)\right)$,
se convierte en usuarios de la clase $k$.\\

Entonces, dado lo anterior, se tiene la siguiente representaci\'on
para el proceso de la longitud de la cola:\\

\begin{equation}
Q_{k}^{x}\left(t\right)=_{k}^{x}\left(0\right)+E_{k}^{x}\left(t\right)+\sum_{l=1}^{K}\Phi_{k}^{l}\left(S_{l}^{x}\left(T_{l}^{x}\left(t\right)\right)\right)-S_{k}^{x}\left(T_{k}^{x}\left(t\right)\right)
\end{equation}
para $k=1,\ldots,K$. Para $i=1,\ldots,d$, sea
\[I_{i}^{x}\left(t\right)=t-\sum_{j\in C_{i}}T_{k}^{x}\left(t\right).\]

Entonces $I_{i}^{x}\left(t\right)$ es el monto acumulado del
tiempo que el servidor $i$ ha estado desocupado al tiempo $t$. Se
est\'a asumiendo que las disciplinas satisfacen la ley de
conservaci\'on del trabajo, es decir, el servidor $i$ est\'a en
pausa solamente cuando no hay usuarios en la estaci\'on $i$.
Entonces, se tiene que

\begin{equation}
\int_{0}^{\infty}\left(\sum_{k\in
C_{i}}Q_{k}^{x}\left(t\right)\right)dI_{i}^{x}\left(t\right)=0,
\end{equation}
para $i=1,\ldots,d$.\\

Hacer
\[T^{x}\left(t\right)=\left(T_{1}^{x}\left(t\right),\ldots,T_{K}^{x}\left(t\right)\right)^{'},\]
\[I^{x}\left(t\right)=\left(I_{1}^{x}\left(t\right),\ldots,I_{K}^{x}\left(t\right)\right)^{'}\]
y
\[S^{x}\left(T^{x}\left(t\right)\right)=\left(S_{1}^{x}\left(T_{1}^{x}\left(t\right)\right),\ldots,S_{K}^{x}\left(T_{K}^{x}\left(t\right)\right)\right)^{'}.\]

Para una disciplina que cumple con la ley de conservaci\'on del
trabajo, en forma vectorial, se tiene el siguiente conjunto de
ecuaciones

\begin{equation}\label{Eq.MF.1.3}
Q^{x}\left(t\right)=Q^{x}\left(0\right)+E^{x}\left(t\right)+\sum_{l=1}^{K}\Phi^{l}\left(S_{l}^{x}\left(T_{l}^{x}\left(t\right)\right)\right)-S^{x}\left(T^{x}\left(t\right)\right),\\
\end{equation}

\begin{equation}\label{Eq.MF.2.3}
Q^{x}\left(t\right)\geq0,\\
\end{equation}

\begin{equation}\label{Eq.MF.3.3}
T^{x}\left(0\right)=0,\textrm{ y }\overline{T}^{x}\left(t\right)\textrm{ es no decreciente},\\
\end{equation}

\begin{equation}\label{Eq.MF.4.3}
I^{x}\left(t\right)=et-CT^{x}\left(t\right)\textrm{ es no
decreciente}\\
\end{equation}

\begin{equation}\label{Eq.MF.5.3}
\int_{0}^{\infty}\left(CQ^{x}\left(t\right)\right)dI_{i}^{x}\left(t\right)=0,\\
\end{equation}

\begin{equation}\label{Eq.MF.6.3}
\textrm{Condiciones adicionales en
}\left(\overline{Q}^{x}\left(\cdot\right),\overline{T}^{x}\left(\cdot\right)\right)\textrm{
espec\'ificas de la disciplina de la cola,}
\end{equation}

donde $e$ es un vector de unos de dimensi\'on $d$, $C$ es la
matriz definida por
\[C_{ik}=\left\{\begin{array}{cc}
1,& S\left(k\right)=i,\\
0,& \textrm{ en otro caso}.\\
\end{array}\right.
\]
Es necesario enunciar el siguiente Teorema que se utilizar\'a para
el Teorema \ref{Tma.4.2.Dai}:
\begin{Teo}[Teorema 4.1, Dai \cite{Dai}]
Considere una disciplina que cumpla la ley de conservaci\'on del
trabajo, para casi todas las trayectorias muestrales $\omega$ y
cualquier sucesi\'on de estados iniciales
$\left\{x_{n}\right\}\subset \mathbf{X}$, con
$|x_{n}|\rightarrow\infty$, existe una subsucesi\'on
$\left\{x_{n_{j}}\right\}$ con $|x_{n_{j}}|\rightarrow\infty$ tal
que
\begin{equation}\label{Eq.4.15}
\frac{1}{|x_{n_{j}}|}\left(Q^{x_{n_{j}}}\left(0\right),U^{x_{n_{j}}}\left(0\right),V^{x_{n_{j}}}\left(0\right)\right)\rightarrow\left(\overline{Q}\left(0\right),\overline{U},\overline{V}\right),
\end{equation}

\begin{equation}\label{Eq.4.16}
\frac{1}{|x_{n_{j}}|}\left(Q^{x_{n_{j}}}\left(|x_{n_{j}}|t\right),T^{x_{n_{j}}}\left(|x_{n_{j}}|t\right)\right)\rightarrow\left(\overline{Q}\left(t\right),\overline{T}\left(t\right)\right)\textrm{
u.o.c.}
\end{equation}

Adem\'as,
$\left(\overline{Q}\left(t\right),\overline{T}\left(t\right)\right)$
satisface las siguientes ecuaciones:
\begin{equation}\label{Eq.MF.1.3a}
\overline{Q}\left(t\right)=Q\left(0\right)+\left(\alpha
t-\overline{U}\right)^{+}-\left(I-P\right)^{'}M^{-1}\left(\overline{T}\left(t\right)-\overline{V}\right)^{+},
\end{equation}

\begin{equation}\label{Eq.MF.2.3a}
\overline{Q}\left(t\right)\geq0,\\
\end{equation}

\begin{equation}\label{Eq.MF.3.3a}
\overline{T}\left(t\right)\textrm{ es no decreciente y comienza en cero},\\
\end{equation}

\begin{equation}\label{Eq.MF.4.3a}
\overline{I}\left(t\right)=et-C\overline{T}\left(t\right)\textrm{
es no decreciente,}\\
\end{equation}

\begin{equation}\label{Eq.MF.5.3a}
\int_{0}^{\infty}\left(C\overline{Q}\left(t\right)\right)d\overline{I}\left(t\right)=0,\\
\end{equation}

\begin{equation}\label{Eq.MF.6.3a}
\textrm{Condiciones adicionales en
}\left(\overline{Q}\left(\cdot\right),\overline{T}\left(\cdot\right)\right)\textrm{
especficas de la disciplina de la cola,}
\end{equation}
\end{Teo}


Propiedades importantes para el modelo de flujo retrasado:

\begin{Prop}
 Sea $\left(\overline{Q},\overline{T},\overline{T}^{0}\right)$ un flujo l\'imite de \ref{Eq.4.4} y suponga que cuando $x\rightarrow\infty$ a lo largo de
una subsucesi\'on
\[\left(\frac{1}{|x|}Q_{k}^{x}\left(0\right),\frac{1}{|x|}A_{k}^{x}\left(0\right),\frac{1}{|x|}B_{k}^{x}\left(0\right),\frac{1}{|x|}B_{k}^{x,0}\left(0\right)\right)\rightarrow\left(\overline{Q}_{k}\left(0\right),0,0,0\right)\]
para $k=1,\ldots,K$. EL flujo l\'imite tiene las siguientes
propiedades, donde las propiedades de la derivada se cumplen donde
la derivada exista:
\begin{itemize}
 \item[i)] Los vectores de tiempo ocupado $\overline{T}\left(t\right)$ y $\overline{T}^{0}\left(t\right)$ son crecientes y continuas con
$\overline{T}\left(0\right)=\overline{T}^{0}\left(0\right)=0$.
\item[ii)] Para todo $t\geq0$
\[\sum_{k=1}^{K}\left[\overline{T}_{k}\left(t\right)+\overline{T}_{k}^{0}\left(t\right)\right]=t\]
\item[iii)] Para todo $1\leq k\leq K$
\[\overline{Q}_{k}\left(t\right)=\overline{Q}_{k}\left(0\right)+\alpha_{k}t-\mu_{k}\overline{T}_{k}\left(t\right)\]
\item[iv)]  Para todo $1\leq k\leq K$
\[\dot{{\overline{T}}}_{k}\left(t\right)=\beta_{k}\] para $\overline{Q}_{k}\left(t\right)=0$.
\item[v)] Para todo $k,j$
\[\mu_{k}^{0}\overline{T}_{k}^{0}\left(t\right)=\mu_{j}^{0}\overline{T}_{j}^{0}\left(t\right)\]
\item[vi)]  Para todo $1\leq k\leq K$
\[\mu_{k}\dot{{\overline{T}}}_{k}\left(t\right)=l_{k}\mu_{k}^{0}\dot{{\overline{T}}}_{k}^{0}\left(t\right)\] para $\overline{Q}_{k}\left(t\right)>0$.
\end{itemize}
\end{Prop}

\begin{Lema}[Lema 3.1 \cite{Chen}]\label{Lema3.1}
Si el modelo de flujo es estable, definido por las ecuaciones
(3.8)-(3.13), entonces el modelo de flujo retrasado tambi\'en es
estable.
\end{Lema}

\begin{Teo}[Teorema 5.1 \cite{Chen}]\label{Tma.5.1.Chen}
La red de colas es estable si existe una constante $t_{0}$ que
depende de $\left(\alpha,\mu,T,U\right)$ y $V$ que satisfagan las
ecuaciones (5.1)-(5.5), $Z\left(t\right)=0$, para toda $t\geq
t_{0}$.
\end{Teo}



\begin{Lema}[Lema 5.2 \cite{Gut}]\label{Lema.5.2.Gut}
Sea $\left\{\xi\left(k\right):k\in\ent\right\}$ sucesi\'on de
variables aleatorias i.i.d. con valores en
$\left(0,\infty\right)$, y sea $E\left(t\right)$ el proceso de
conteo
\[E\left(t\right)=max\left\{n\geq1:\xi\left(1\right)+\cdots+\xi\left(n-1\right)\leq t\right\}.\]
Si $E\left[\xi\left(1\right)\right]<\infty$, entonces para
cualquier entero $r\geq1$
\begin{equation}
lim_{t\rightarrow\infty}\esp\left[\left(\frac{E\left(t\right)}{t}\right)^{r}\right]=\left(\frac{1}{E\left[\xi_{1}\right]}\right)^{r}
\end{equation}
de aqu\'i, bajo estas condiciones
\begin{itemize}
\item[a)] Para cualquier $t>0$,
$sup_{t\geq\delta}\esp\left[\left(\frac{E\left(t\right)}{t}\right)^{r}\right]$

\item[b)] Las variables aleatorias
$\left\{\left(\frac{E\left(t\right)}{t}\right)^{r}:t\geq1\right\}$
son uniformemente integrables.
\end{itemize}
\end{Lema}

\begin{Teo}[Teorema 5.1: Ley Fuerte para Procesos de Conteo
\cite{Gut}]\label{Tma.5.1.Gut} Sea
$0<\mu<\esp\left(X_{1}\right]\leq\infty$. entonces

\begin{itemize}
\item[a)] $\frac{N\left(t\right)}{t}\rightarrow\frac{1}{\mu}$
a.s., cuando $t\rightarrow\infty$.


\item[b)]$\esp\left[\frac{N\left(t\right)}{t}\right]^{r}\rightarrow\frac{1}{\mu^{r}}$,
cuando $t\rightarrow\infty$ para todo $r>0$..
\end{itemize}
\end{Teo}


\begin{Prop}[Proposici\'on 5.1 \cite{DaiSean}]\label{Prop.5.1}
Suponga que los supuestos (A1) y (A2) se cumplen, adem\'as suponga
que el modelo de flujo es estable. Entonces existe $t_{0}>0$ tal
que
\begin{equation}\label{Eq.Prop.5.1}
lim_{|x|\rightarrow\infty}\frac{1}{|x|^{p+1}}\esp_{x}\left[|X\left(t_{0}|x|\right)|^{p+1}\right]=0.
\end{equation}

\end{Prop}


\begin{Prop}[Proposici\'on 5.3 \cite{DaiSean}]
Sea $X$ proceso de estados para la red de colas, y suponga que se
cumplen los supuestos (A1) y (A2), entonces para alguna constante
positiva $C_{p+1}<\infty$, $\delta>0$ y un conjunto compacto
$C\subset X$.

\begin{equation}\label{Eq.5.4}
\esp_{x}\left[\int_{0}^{\tau_{C}\left(\delta\right)}\left(1+|X\left(t\right)|^{p}\right)dt\right]\leq
C_{p+1}\left(1+|x|^{p+1}\right)
\end{equation}
\end{Prop}

\begin{Prop}[Proposici\'on 5.4 \cite{DaiSean}]
Sea $X$ un proceso de Markov Borel Derecho en $X$, sea
$f:X\leftarrow\rea_{+}$ y defina para alguna $\delta>0$, y un
conjunto cerrado $C\subset X$
\[V\left(x\right):=\esp_{x}\left[\int_{0}^{\tau_{C}\left(\delta\right)}f\left(X\left(t\right)\right)dt\right]\]
para $x\in X$. Si $V$ es finito en todas partes y uniformemente
acotada en $C$, entonces existe $k<\infty$ tal que
\begin{equation}\label{Eq.5.11}
\frac{1}{t}\esp_{x}\left[V\left(x\right)\right]+\frac{1}{t}\int_{0}^{t}\esp_{x}\left[f\left(X\left(s\right)\right)ds\right]\leq\frac{1}{t}V\left(x\right)+k,
\end{equation}
para $x\in X$ y $t>0$.
\end{Prop}


\begin{Teo}[Teorema 5.5 \cite{DaiSean}]
Suponga que se cumplen (A1) y (A2), adem\'as suponga que el modelo
de flujo es estable. Entonces existe una constante $k_{p}<\infty$
tal que
\begin{equation}\label{Eq.5.13}
\frac{1}{t}\int_{0}^{t}\esp_{x}\left[|Q\left(s\right)|^{p}\right]ds\leq
k_{p}\left\{\frac{1}{t}|x|^{p+1}+1\right\}
\end{equation}
para $t\geq0$, $x\in X$. En particular para cada condici\'on
inicial
\begin{equation}\label{Eq.5.14}
Limsup_{t\rightarrow\infty}\frac{1}{t}\int_{0}^{t}\esp_{x}\left[|Q\left(s\right)|^{p}\right]ds\leq
k_{p}
\end{equation}
\end{Teo}

\begin{Teo}[Teorema 6.2 \cite{DaiSean}]\label{Tma.6.2}
Suponga que se cumplen los supuestos (A1)-(A3) y que el modelo de
flujo es estable, entonces se tiene que
\[\parallel P^{t}\left(c,\cdot\right)-\pi\left(\cdot\right)\parallel_{f_{p}}\rightarrow0\]
para $t\rightarrow\infty$ y $x\in X$. En particular para cada
condici\'on inicial
\[lim_{t\rightarrow\infty}\esp_{x}\left[\left|Q_{t}\right|^{p}\right]=\esp_{\pi}\left[\left|Q_{0}\right|^{p}\right]<\infty\]
\end{Teo}


\begin{Teo}[Teorema 6.3 \cite{DaiSean}]\label{Tma.6.3}
Suponga que se cumplen los supuestos (A1)-(A3) y que el modelo de
flujo es estable, entonces con
$f\left(x\right)=f_{1}\left(x\right)$, se tiene que
\[lim_{t\rightarrow\infty}t^{(p-1)\left|P^{t}\left(c,\cdot\right)-\pi\left(\cdot\right)\right|_{f}=0},\]
para $x\in X$. En particular, para cada condici\'on inicial
\[lim_{t\rightarrow\infty}t^{(p-1)}\left|\esp_{x}\left[Q_{t}\right]-\esp_{\pi}\left[Q_{0}\right]\right|=0.\]
\end{Teo}



\begin{Prop}[Proposici\'on 5.1, Dai y Meyn \cite{DaiSean}]\label{Prop.5.1.DaiSean}
Suponga que los supuestos A1) y A2) son ciertos y que el modelo de
flujo es estable. Entonces existe $t_{0}>0$ tal que
\begin{equation}
lim_{|x|\rightarrow\infty}\frac{1}{|x|^{p+1}}\esp_{x}\left[|X\left(t_{0}|x|\right)|^{p+1}\right]=0
\end{equation}
\end{Prop}

\begin{Lemma}[Lema 5.2, Dai y Meyn, \cite{DaiSean}]\label{Lema.5.2.DaiSean}
 Sea $\left\{\zeta\left(k\right):k\in \mathbb{z}\right\}$ una sucesi\'on independiente e id\'enticamente distribuida que toma valores en $\left(0,\infty\right)$,
y sea
$E\left(t\right)=max\left(n\geq1:\zeta\left(1\right)+\cdots+\zeta\left(n-1\right)\leq
t\right)$. Si $\esp\left[\zeta\left(1\right)\right]<\infty$,
entonces para cualquier entero $r\geq1$
\begin{equation}
 lim_{t\rightarrow\infty}\esp\left[\left(\frac{E\left(t\right)}{t}\right)^{r}\right]=\left(\frac{1}{\esp\left[\zeta_{1}\right]}\right)^{r}.
\end{equation}
Luego, bajo estas condiciones:
\begin{itemize}
 \item[a)] para cualquier $\delta>0$, $\sup_{t\geq\delta}\esp\left[\left(\frac{E\left(t\right)}{t}\right)^{r}\right]<\infty$
\item[b)] las variables aleatorias
$\left\{\left(\frac{E\left(t\right)}{t}\right)^{r}:t\geq1\right\}$
son uniformemente integrables.
\end{itemize}
\end{Lemma}

\begin{Teo}[Teorema 5.5, Dai y Meyn \cite{DaiSean}]\label{Tma.5.5.DaiSean}
Suponga que los supuestos A1) y A2) se cumplen y que el modelo de
flujo es estable. Entonces existe una constante $\kappa_{p}$ tal
que
\begin{equation}
\frac{1}{t}\int_{0}^{t}\esp_{x}\left[|Q\left(s\right)|^{p}\right]ds\leq\kappa_{p}\left\{\frac{1}{t}|x|^{p+1}+1\right\}
\end{equation}
para $t>0$ y $x\in X$. En particular, para cada condici\'on
inicial
\begin{eqnarray*}
\limsup_{t\rightarrow\infty}\frac{1}{t}\int_{0}^{t}\esp_{x}\left[|Q\left(s\right)|^{p}\right]ds\leq\kappa_{p}.
\end{eqnarray*}
\end{Teo}

\begin{Teo}[Teorema 6.2, Dai y Meyn \cite{DaiSean}]\label{Tma.6.2.DaiSean}
Suponga que se cumplen los supuestos A1), A2) y A3) y que el
modelo de flujo es estable. Entonces se tiene que
\begin{equation}
\left\|P^{t}\left(x,\cdot\right)-\pi\left(\cdot\right)\right\|_{f_{p}}\textrm{,
}t\rightarrow\infty,x\in X.
\end{equation}
En particular para cada condici\'on inicial
\begin{eqnarray*}
\lim_{t\rightarrow\infty}\esp_{x}\left[|Q\left(t\right)|^{p}\right]=\esp_{\pi}\left[|Q\left(0\right)|^{p}\right]\leq\kappa_{r}
\end{eqnarray*}
\end{Teo}
\begin{Teo}[Teorema 6.3, Dai y Meyn \cite{DaiSean}]\label{Tma.6.3.DaiSean}
Suponga que se cumplen los supuestos A1), A2) y A3) y que el
modelo de flujo es estable. Entonces con
$f\left(x\right)=f_{1}\left(x\right)$ se tiene
\begin{equation}
\lim_{t\rightarrow\infty}t^{p-1}\left\|P^{t}\left(x,\cdot\right)-\pi\left(\cdot\right)\right\|_{f}=0.
\end{equation}
En particular para cada condici\'on inicial
\begin{eqnarray*}
\lim_{t\rightarrow\infty}t^{p-1}|\esp_{x}\left[Q\left(t\right)\right]-\esp_{\pi}\left[Q\left(0\right)\right]|=0.
\end{eqnarray*}
\end{Teo}

\begin{Teo}[Teorema 6.4, Dai y Meyn, \cite{DaiSean}]\label{Tma.6.4.DaiSean}
Suponga que se cumplen los supuestos A1), A2) y A3) y que el
modelo de flujo es estable. Sea $\nu$ cualquier distribuci\'on de
probabilidad en $\left(X,\mathcal{B}_{X}\right)$, y $\pi$ la
distribuci\'on estacionaria de $X$.
\begin{itemize}
\item[i)] Para cualquier $f:X\leftarrow\rea_{+}$
\begin{equation}
\lim_{t\rightarrow\infty}\frac{1}{t}\int_{o}^{t}f\left(X\left(s\right)\right)ds=\pi\left(f\right):=\int
f\left(x\right)\pi\left(dx\right)
\end{equation}
$\prob$-c.s.

\item[ii)] Para cualquier $f:X\leftarrow\rea_{+}$ con
$\pi\left(|f|\right)<\infty$, la ecuaci\'on anterior se cumple.
\end{itemize}
\end{Teo}

\begin{Teo}[Teorema 2.2, Down \cite{Down}]\label{Tma2.2.Down}
Suponga que el fluido modelo es inestable en el sentido de que
para alguna $\epsilon_{0},c_{0}\geq0$,
\begin{equation}\label{Eq.Inestability}
|Q\left(T\right)|\geq\epsilon_{0}T-c_{0}\textrm{,   }T\geq0,
\end{equation}
para cualquier condici\'on inicial $Q\left(0\right)$, con
$|Q\left(0\right)|=1$. Entonces para cualquier $0<q\leq1$, existe
$B<0$ tal que para cualquier $|x|\geq B$,
\begin{equation}
\prob_{x}\left\{\mathbb{X}\rightarrow\infty\right\}\geq q.
\end{equation}
\end{Teo}



\begin{Def}
Sea $X$ un conjunto y $\mathcal{F}$ una $\sigma$-\'algebra de
subconjuntos de $X$, la pareja $\left(X,\mathcal{F}\right)$ es
llamado espacio medible. Un subconjunto $A$ de $X$ es llamado
medible, o medible con respecto a $\mathcal{F}$, si
$A\in\mathcal{F}$.
\end{Def}

\begin{Def}
Sea $\left(X,\mathcal{F},\mu\right)$ espacio de medida. Se dice
que la medida $\mu$ es $\sigma$-finita si se puede escribir
$X=\bigcup_{n\geq1}X_{n}$ con $X_{n}\in\mathcal{F}$ y
$\mu\left(X_{n}\right)<\infty$.
\end{Def}

\begin{Def}\label{Cto.Borel}
Sea $X$ el conjunto de los n\'umeros reales $\rea$. El \'algebra
de Borel es la $\sigma$-\'algebra $B$ generada por los intervalos
abiertos $\left(a,b\right)\in\rea$. Cualquier conjunto en $B$ es
llamado {\em Conjunto de Borel}.
\end{Def}

\begin{Def}\label{Funcion.Medible}
Una funci\'on $f:X\rightarrow\rea$, es medible si para cualquier
n\'umero real $\alpha$ el conjunto
\[\left\{x\in X:f\left(x\right)>\alpha\right\}\]
pertenece a $\mathcal{F}$. Equivalentemente, se dice que $f$ es
medible si
\[f^{-1}\left(\left(\alpha,\infty\right)\right)=\left\{x\in X:f\left(x\right)>\alpha\right\}\in\mathcal{F}.\]
\end{Def}


\begin{Def}\label{Def.Cilindros}
Sean $\left(\Omega_{i},\mathcal{F}_{i}\right)$, $i=1,2,\ldots,$
espacios medibles y $\Omega=\prod_{i=1}^{\infty}\Omega_{i}$ el
conjunto de todas las sucesiones
$\left(\omega_{1},\omega_{2},\ldots,\right)$ tales que
$\omega_{i}\in\Omega_{i}$, $i=1,2,\ldots,$. Si
$B^{n}\subset\prod_{i=1}^{\infty}\Omega_{i}$, definimos
$B_{n}=\left\{\omega\in\Omega:\left(\omega_{1},\omega_{2},\ldots,\omega_{n}\right)\in
B^{n}\right\}$. Al conjunto $B_{n}$ se le llama {\em cilindro} con
base $B^{n}$, el cilindro es llamado medible si
$B^{n}\in\prod_{i=1}^{\infty}\mathcal{F}_{i}$.
\end{Def}


\begin{Def}\label{Def.Proc.Adaptado}[TSP, Ash \cite{RBA}]
Sea $X\left(t\right),t\geq0$ proceso estoc\'astico, el proceso es
adaptado a la familia de $\sigma$-\'algebras $\mathcal{F}_{t}$,
para $t\geq0$, si para $s<t$ implica que
$\mathcal{F}_{s}\subset\mathcal{F}_{t}$, y $X\left(t\right)$ es
$\mathcal{F}_{t}$-medible para cada $t$. Si no se especifica
$\mathcal{F}_{t}$ entonces se toma $\mathcal{F}_{t}$ como
$\mathcal{F}\left(X\left(s\right),s\leq t\right)$, la m\'as
peque\~na $\sigma$-\'algebra de subconjuntos de $\Omega$ que hace
que cada $X\left(s\right)$, con $s\leq t$ sea Borel medible.
\end{Def}


\begin{Def}\label{Def.Tiempo.Paro}[TSP, Ash \cite{RBA}]
Sea $\left\{\mathcal{F}\left(t\right),t\geq0\right\}$ familia
creciente de sub $\sigma$-\'algebras. es decir,
$\mathcal{F}\left(s\right)\subset\mathcal{F}\left(t\right)$ para
$s\leq t$. Un tiempo de paro para $\mathcal{F}\left(t\right)$ es
una funci\'on $T:\Omega\rightarrow\left[0,\infty\right]$ tal que
$\left\{T\leq t\right\}\in\mathcal{F}\left(t\right)$ para cada
$t\geq0$. Un tiempo de paro para el proceso estoc\'astico
$X\left(t\right),t\geq0$ es un tiempo de paro para las
$\sigma$-\'algebras
$\mathcal{F}\left(t\right)=\mathcal{F}\left(X\left(s\right)\right)$.
\end{Def}

\begin{Def}
Sea $X\left(t\right),t\geq0$ proceso estoc\'astico, con
$\left(S,\chi\right)$ espacio de estados. Se dice que el proceso
es adaptado a $\left\{\mathcal{F}\left(t\right)\right\}$, es
decir, si para cualquier $s,t\in I$, $I$ conjunto de \'indices,
$s<t$, se tiene que
$\mathcal{F}\left(s\right)\subset\mathcal{F}\left(t\right)$ y
$X\left(t\right)$ es $\mathcal{F}\left(t\right)$-medible,
\end{Def}

\begin{Def}
Sea $X\left(t\right),t\geq0$ proceso estoc\'astico, se dice que es
un Proceso de Markov relativo a $\mathcal{F}\left(t\right)$ o que
$\left\{X\left(t\right),\mathcal{F}\left(t\right)\right\}$ es de
Markov si y s\'olo si para cualquier conjunto $B\in\chi$,  y
$s,t\in I$, $s<t$ se cumple que
\begin{equation}\label{Prop.Markov}
P\left\{X\left(t\right)\in
B|\mathcal{F}\left(s\right)\right\}=P\left\{X\left(t\right)\in
B|X\left(s\right)\right\}.
\end{equation}
\end{Def}
\begin{Note}
Si se dice que $\left\{X\left(t\right)\right\}$ es un Proceso de
Markov sin mencionar $\mathcal{F}\left(t\right)$, se asumir\'a que
\begin{eqnarray*}
\mathcal{F}\left(t\right)=\mathcal{F}_{0}\left(t\right)=\mathcal{F}\left(X\left(r\right),r\leq
t\right),
\end{eqnarray*}
entonces la ecuaci\'on (\ref{Prop.Markov}) se puede escribir como
\begin{equation}
P\left\{X\left(t\right)\in B|X\left(r\right),r\leq s\right\} =
P\left\{X\left(t\right)\in B|X\left(s\right)\right\}
\end{equation}
\end{Note}
%_______________________________________________________________
\subsection{Procesos de Estados de Markov}
%_______________________________________________________________

\begin{Teo}
Sea $\left(X_{n},\mathcal{F}_{n},n=0,1,\ldots,\right\}$ Proceso de
Markov con espacio de estados $\left(S_{0},\chi_{0}\right)$
generado por una distribuici\'on inicial $P_{o}$ y probabilidad de
transici\'on $p_{mn}$, para $m,n=0,1,\ldots,$ $m<n$, que por
notaci\'on se escribir\'a como $p\left(m,n,x,B\right)\rightarrow
p_{mn}\left(x,B\right)$. Sea $S$ tiempo de paro relativo a la
$\sigma$-\'algebra $\mathcal{F}_{n}$. Sea $T$ funci\'on medible,
$T:\Omega\rightarrow\left\{0,1,\ldots,\right\}$. Sup\'ongase que
$T\geq S$, entonces $T$ es tiempo de paro. Si $B\in\chi_{0}$,
entonces
\begin{equation}\label{Prop.Fuerte.Markov}
P\left\{X\left(T\right)\in
B,T<\infty|\mathcal{F}\left(S\right)\right\} =
p\left(S,T,X\left(s\right),B\right)
\end{equation}
en $\left\{T<\infty\right\}$.
\end{Teo}


Sea $K$ conjunto numerable y sea $d:K\rightarrow\nat$ funci\'on.
Para $v\in K$, $M_{v}$ es un conjunto abierto de
$\rea^{d\left(v\right)}$. Entonces \[E=\bigcup_{v\in
K}M_{v}=\left\{\left(v,\zeta\right):v\in K,\zeta\in
M_{v}\right\}.\]

Sea $\mathcal{E}$ la clase de conjuntos medibles en $E$:
\[\mathcal{E}=\left\{\bigcup_{v\in K}A_{v}:A_{v}\in \mathcal{M}_{v}\right\}.\]

donde $\mathcal{M}$ son los conjuntos de Borel de $M_{v}$.
Entonces $\left(E,\mathcal{E}\right)$ es un espacio de Borel. El
estado del proceso se denotar\'a por
$\mathbf{x}_{t}=\left(v_{t},\zeta_{t}\right)$. La distribuci\'on
de $\left(\mathbf{x}_{t}\right)$ est\'a determinada por por los
siguientes objetos:

\begin{itemize}
\item[i)] Los campos vectoriales $\left(\mathcal{H}_{v},v\in
K\right)$. \item[ii)] Una funci\'on medible $\lambda:E\rightarrow
\rea_{+}$. \item[iii)] Una medida de transici\'on
$Q:\mathcal{E}\times\left(E\cup\Gamma^{*}\right)\rightarrow\left[0,1\right]$
donde
\begin{equation}
\Gamma^{*}=\bigcup_{v\in K}\partial^{*}M_{v}.
\end{equation}
y
\begin{equation}
\partial^{*}M_{v}=\left\{z\in\partial M_{v}:\mathbf{\mathbf{\phi}_{v}\left(t,\zeta\right)=\mathbf{z}}\textrm{ para alguna }\left(t,\zeta\right)\in\rea_{+}\times M_{v}\right\}.
\end{equation}
$\partial M_{v}$ denota  la frontera de $M_{v}$.
\end{itemize}

El campo vectorial $\left(\mathcal{H}_{v},v\in K\right)$ se supone
tal que para cada $\mathbf{z}\in M_{v}$ existe una \'unica curva
integral $\mathbf{\phi}_{v}\left(t,\zeta\right)$ que satisface la
ecuaci\'on

\begin{equation}
\frac{d}{dt}f\left(\zeta_{t}\right)=\mathcal{H}f\left(\zeta_{t}\right),
\end{equation}
con $\zeta_{0}=\mathbf{z}$, para cualquier funci\'on suave
$f:\rea^{d}\rightarrow\rea$ y $\mathcal{H}$ denota el operador
diferencial de primer orden, con $\mathcal{H}=\mathcal{H}_{v}$ y
$\zeta_{t}=\mathbf{\phi}\left(t,\mathbf{z}\right)$. Adem\'as se
supone que $\mathcal{H}_{v}$ es conservativo, es decir, las curvas
integrales est\'an definidas para todo $t>0$.

Para $\mathbf{x}=\left(v,\zeta\right)\in E$ se denota
\[t^{*}\mathbf{x}=inf\left\{t>0:\mathbf{\phi}_{v}\left(t,\zeta\right)\in\partial^{*}M_{v}\right\}\]

En lo que respecta a la funci\'on $\lambda$, se supondr\'a que
para cada $\left(v,\zeta\right)\in E$ existe un $\epsilon>0$ tal
que la funci\'on
$s\rightarrow\lambda\left(v,\phi_{v}\left(s,\zeta\right)\right)\in
E$ es integrable para $s\in\left[0,\epsilon\right)$. La medida de
transici\'on $Q\left(A;\mathbf{x}\right)$ es una funci\'on medible
de $\mathbf{x}$ para cada $A\in\mathcal{E}$, definida para
$\mathbf{x}\in E\cup\Gamma^{*}$ y es una medida de probabilidad en
$\left(E,\mathcal{E}\right)$ para cada $\mathbf{x}\in E$.

El movimiento del proceso $\left(\mathbf{x}_{t}\right)$ comenzando
en $\mathbf{x}=\left(n,\mathbf{z}\right)\in E$ se puede construir
de la siguiente manera, def\'inase la funci\'on $F$ por

\begin{equation}
F\left(t\right)=\left\{\begin{array}{ll}\\
exp\left(-\int_{0}^{t}\lambda\left(n,\phi_{n}\left(s,\mathbf{z}\right)\right)ds\right), & t<t^{*}\left(\mathbf{x}\right),\\
0, & t\geq t^{*}\left(\mathbf{x}\right)
\end{array}\right.
\end{equation}

Sea $T_{1}$ una variable aleatoria tal que
$\prob\left[T_{1}>t\right]=F\left(t\right)$, ahora sea la variable
aleatoria $\left(N,Z\right)$ con distribuici\'on
$Q\left(\cdot;\phi_{n}\left(T_{1},\mathbf{z}\right)\right)$. La
trayectoria de $\left(\mathbf{x}_{t}\right)$ para $t\leq T_{1}$ es
\begin{eqnarray*}
\mathbf{x}_{t}=\left(v_{t},\zeta_{t}\right)=\left\{\begin{array}{ll}
\left(n,\phi_{n}\left(t,\mathbf{z}\right)\right), & t<T_{1},\\
\left(N,\mathbf{Z}\right), & t=t_{1}.
\end{array}\right.
\end{eqnarray*}

Comenzando en $\mathbf{x}_{T_{1}}$ se selecciona el siguiente
tiempo de intersalto $T_{2}-T_{1}$ lugar del post-salto
$\mathbf{x}_{T_{2}}$ de manera similar y as\'i sucesivamente. Este
procedimiento nos da una trayectoria determinista por partes
$\mathbf{x}_{t}$ con tiempos de salto $T_{1},T_{2},\ldots$. Bajo
las condiciones enunciadas para $\lambda,T_{1}>0$  y
$T_{1}-T_{2}>0$ para cada $i$, con probabilidad 1. Se supone que
se cumple la siguiente condici\'on.

\begin{Sup}[Supuesto 3.1, Davis \cite{Davis}]\label{Sup3.1.Davis}
Sea $N_{t}:=\sum_{t}\indora_{\left(t\geq t\right)}$ el n\'umero de
saltos en $\left[0,t\right]$. Entonces
\begin{equation}
\esp\left[N_{t}\right]<\infty\textrm{ para toda }t.
\end{equation}
\end{Sup}

es un proceso de Markov, m\'as a\'un, es un Proceso Fuerte de
Markov, es decir, la Propiedad Fuerte de Markov\footnote{Revisar
p\'agina 362, y 364 de Davis \cite{Davis}.} se cumple para
cualquier tiempo de paro.
%_________________________________________________________________________
%\renewcommand{\refname}{PROCESOS ESTOC\'ASTICOS}
%\renewcommand{\appendixname}{PROCESOS ESTOC\'ASTICOS}
%\renewcommand{\appendixtocname}{PROCESOS ESTOC\'ASTICOS}
%\renewcommand{\appendixpagename}{PROCESOS ESTOC\'ASTICOS}
%\appendix
%\clearpage % o \cleardoublepage
%\addappheadtotoc
%\appendixpage
%_________________________________________________________________________
\subsection{Teor\'ia General de Procesos Estoc\'asticos}
%_________________________________________________________________________
En esta secci\'on se har\'an las siguientes consideraciones: $E$
es un espacio m\'etrico separable y la m\'etrica $d$ es compatible
con la topolog\'ia.

\begin{Def}
Una medida finita, $\lambda$ en la $\sigma$-\'algebra de Borel de
un espacio metrizable $E$ se dice cerrada si
\begin{equation}\label{Eq.A2.3}
\lambda\left(E\right)=sup\left\{\lambda\left(K\right):K\textrm{ es
compacto en }E\right\}.
\end{equation}
\end{Def}

\begin{Def}
$E$ es un espacio de Rad\'on si cada medida finita en
$\left(E,\mathcal{B}\left(E\right)\right)$ es regular interior o cerrada,
{\em tight}.
\end{Def}


El siguiente teorema nos permite tener una mejor caracterizaci\'on de los espacios de Rad\'on:
\begin{Teo}\label{Tma.A2.2}
Sea $E$ espacio separable metrizable. Entonces $E$ es de Rad\'on
si y s\'olo s\'i cada medida finita en
$\left(E,\mathcal{B}\left(E\right)\right)$ es cerrada.
\end{Teo}

%_________________________________________________________________________________________
\subsection{Propiedades de Markov}
%_________________________________________________________________________________________

Sea $E$ espacio de estados, tal que $E$ es un espacio de Rad\'on, $\mathcal{B}\left(E\right)$ $\sigma$-\'algebra de Borel en $E$, que se denotar\'a por $\mathcal{E}$.

Sea $\left(X,\mathcal{G},\prob\right)$ espacio de probabilidad,
$I\subset\rea$ conjunto de índices. Sea $\mathcal{F}_{\leq t}$ la
$\sigma$-\'algebra natural definida como
$\sigma\left\{f\left(X_{r}\right):r\in I, r\leq
t,f\in\mathcal{E}\right\}$. Se considerar\'a una
$\sigma$-\'algebra m\'as general\footnote{qu\'e se quiere decir
con el t\'ermino: m\'as general?}, $ \left(\mathcal{G}_{t}\right)$
tal que $\left(X_{t}\right)$ sea $\mathcal{E}$-adaptado.

\begin{Def}
Una familia $\left(P_{s,t}\right)$ de kernels de Markov en $\left(E,\mathcal{E}\right)$ indexada por pares $s,t\in I$, con $s\leq t$ es una funci\'on de transici\'on en $\ER$, si  para todo $r\leq s< t$ en $I$ y todo $x\in E$, $B\in\mathcal{E}$
\begin{equation}\label{Eq.Kernels}
P_{r,t}\left(x,B\right)=\int_{E}P_{r,s}\left(x,dy\right)P_{s,t}\left(y,B\right)\footnote{Ecuaci\'on de Chapman-Kolmogorov}.
\end{equation}
\end{Def}

Se dice que la funci\'on de transici\'on $\KM$ en $\ER$ es la funci\'on de transici\'on para un proceso $\PE$  con valores en $E$ y que satisface la propiedad de Markov\footnote{\begin{equation}\label{Eq.1.4.S}
\prob\left\{H|\mathcal{G}_{t}\right\}=\prob\left\{H|X_{t}\right\}\textrm{ }H\in p\mathcal{F}_{\geq t}.
\end{equation}} (\ref{Eq.1.4.S}) relativa a $\left(\mathcal{G}_{t}\right)$ si

\begin{equation}\label{Eq.1.6.S}
\prob\left\{f\left(X_{t}\right)|\mathcal{G}_{s}\right\}=P_{s,t}f\left(X_{t}\right)\textrm{ }s\leq t\in I,\textrm{ }f\in b\mathcal{E}.
\end{equation}

\begin{Def}
Una familia $\left(P_{t}\right)_{t\geq0}$ de kernels de Markov en $\ER$ es llamada {\em Semigrupo de Transici\'on de Markov} o {\em Semigrupo de Transici\'on} si
\[P_{t+s}f\left(x\right)=P_{t}\left(P_{s}f\right)\left(x\right),\textrm{ }t,s\geq0,\textrm{ }x\in E\textrm{ }f\in b\mathcal{E}\footnote{Definir los t\'ermino $b\mathcal{E}$ y $p\mathcal{E}$}.\]
\end{Def}
\begin{Note}
Si la funci\'on de transici\'on $\KM$ es llamada homog\'enea si $P_{s,t}=P_{t-s}$.
\end{Note}

Un proceso de Markov que satisface la ecuaci\'on (\ref{Eq.1.6.S}) con funci\'on de transici\'on homog\'enea $\left(P_{t}\right)$ tiene la propiedad caracter\'istica
\begin{equation}\label{Eq.1.8.S}
\prob\left\{f\left(X_{t+s}\right)|\mathcal{G}_{t}\right\}=P_{s}f\left(X_{t}\right)\textrm{ }t,s\geq0,\textrm{ }f\in b\mathcal{E}.
\end{equation}
La ecuaci\'on anterior es la {\em Propiedad Simple de Markov} de $X$ relativa a $\left(P_{t}\right)$.

En este sentido el proceso $\PE$ cumple con la propiedad de Markov (\ref{Eq.1.8.S}) relativa a $\left(\Omega,\mathcal{G},\mathcal{G}_{t},\prob\right)$ con semigrupo de transici\'on $\left(P_{t}\right)$.
%_________________________________________________________________________________________
\subsection{Primer Condici\'on de Regularidad}
%_________________________________________________________________________________________
%\newcommand{\EM}{\left(\Omega,\mathcal{G},\prob\right)}
%\newcommand{\E4}{\left(\Omega,\mathcal{G},\mathcal{G}_{t},\prob\right)}
\begin{Def}
Un proceso estoc\'astico $\PE$ definido en
$\left(\Omega,\mathcal{G},\prob\right)$ con valores en el espacio
topol\'ogico $E$ es continuo por la derecha si cada trayectoria
muestral $t\rightarrow X_{t}\left(w\right)$ es un mapeo continuo
por la derecha de $I$ en $E$.
\end{Def}

\begin{Def}[HD1]\label{Eq.2.1.S}
Un semigrupo de Markov $\left(P_{t}\right)$ en un espacio de
Rad\'on $E$ se dice que satisface la condici\'on {\em HD1} si,
dada una medida de probabilidad $\mu$ en $E$, existe una
$\sigma$-\'algebra $\mathcal{E^{*}}$ con
$\mathcal{E}\subset\mathcal{E}^{*}$ y
$P_{t}\left(b\mathcal{E}^{*}\right)\subset b\mathcal{E}^{*}$, y un
$\mathcal{E}^{*}$-proceso $E$-valuado continuo por la derecha
$\PE$ en alg\'un espacio de probabilidad filtrado
$\left(\Omega,\mathcal{G},\mathcal{G}_{t},\prob\right)$ tal que
$X=\left(\Omega,\mathcal{G},\mathcal{G}_{t},\prob\right)$ es de
Markov (Homog\'eneo) con semigrupo de transici\'on $(P_{t})$ y
distribuci\'on inicial $\mu$.
\end{Def}

Consid\'erese la colecci\'on de variables aleatorias $X_{t}$
definidas en alg\'un espacio de probabilidad, y una colecci\'on de
medidas $\mathbf{P}^{x}$ tales que
$\mathbf{P}^{x}\left\{X_{0}=x\right\}$, y bajo cualquier
$\mathbf{P}^{x}$, $X_{t}$ es de Markov con semigrupo
$\left(P_{t}\right)$. $\mathbf{P}^{x}$ puede considerarse como la
distribuci\'on condicional de $\mathbf{P}$ dado $X_{0}=x$.

\begin{Def}\label{Def.2.2.S}
Sea $E$ espacio de Rad\'on, $\SG$ semigrupo de Markov en $\ER$. La colecci\'on $\mathbf{X}=\left(\Omega,\mathcal{G},\mathcal{G}_{t},X_{t},\theta_{t},\CM\right)$ es un proceso $\mathcal{E}$-Markov continuo por la derecha simple, con espacio de estados $E$ y semigrupo de transici\'on $\SG$ en caso de que $\mathbf{X}$ satisfaga las siguientes condiciones:
\begin{itemize}
\item[i)] $\left(\Omega,\mathcal{G},\mathcal{G}_{t}\right)$ es un espacio de medida filtrado, y $X_{t}$ es un proceso $E$-valuado continuo por la derecha $\mathcal{E}^{*}$-adaptado a $\left(\mathcal{G}_{t}\right)$;

\item[ii)] $\left(\theta_{t}\right)_{t\geq0}$ es una colecci\'on de operadores {\em shift} para $X$, es decir, mapea $\Omega$ en s\'i mismo satisfaciendo para $t,s\geq0$,

\begin{equation}\label{Eq.Shift}
\theta_{t}\circ\theta_{s}=\theta_{t+s}\textrm{ y }X_{t}\circ\theta_{t}=X_{t+s};
\end{equation}

\item[iii)] Para cualquier $x\in E$,$\CM\left\{X_{0}=x\right\}=1$, y el proceso $\PE$ tiene la propiedad de Markov (\ref{Eq.1.8.S}) con semigrupo de transici\'on $\SG$ relativo a $\left(\Omega,\mathcal{G},\mathcal{G}_{t},\CM\right)$.
\end{itemize}
\end{Def}

\begin{Def}[HD2]\label{Eq.2.2.S}
Para cualquier $\alpha>0$ y cualquier $f\in S^{\alpha}$, el proceso $t\rightarrow f\left(X_{t}\right)$ es continuo por la derecha casi seguramente.
\end{Def}

\begin{Def}\label{Def.PD}
Un sistema $\mathbf{X}=\left(\Omega,\mathcal{G},\mathcal{G}_{t},X_{t},\theta_{t},\CM\right)$ es un proceso derecho en el espacio de Rad\'on $E$ con semigrupo de transici\'on $\SG$ provisto de:
\begin{itemize}
\item[i)] $\mathbf{X}$ es una realizaci\'on  continua por la derecha, \ref{Def.2.2.S}, de $\SG$.

\item[ii)] $\mathbf{X}$ satisface la condicion HD2, \ref{Eq.2.2.S}, relativa a $\mathcal{G}_{t}$.

\item[iii)] $\mathcal{G}_{t}$ es aumentado y continuo por la derecha.
\end{itemize}
\end{Def}


%_________________________________________________________________________
%\renewcommand{\refname}{MODELO DE FLUJO}
%\renewcommand{\appendixname}{MODELO DE FLUJO}
%\renewcommand{\appendixtocname}{MODELO DE FLUJO}
%\renewcommand{\appendixpagename}{MODELO DE FLUJO}
%\appendix
%\clearpage % o \cleardoublepage
%\addappheadtotoc
%\appendixpage

\subsection{Construcci\'on del Modelo de Flujo}


\begin{Lema}[Lema 4.2, Dai\cite{Dai}]\label{Lema4.2}
Sea $\left\{x_{n}\right\}\subset \mathbf{X}$ con
$|x_{n}|\rightarrow\infty$, conforme $n\rightarrow\infty$. Suponga
que
\[lim_{n\rightarrow\infty}\frac{1}{|x_{n}|}U\left(0\right)=\overline{U}\]
y
\[lim_{n\rightarrow\infty}\frac{1}{|x_{n}|}V\left(0\right)=\overline{V}.\]

Entonces, conforme $n\rightarrow\infty$, casi seguramente

\begin{equation}\label{E1.4.2}
\frac{1}{|x_{n}|}\Phi^{k}\left(\left[|x_{n}|t\right]\right)\rightarrow
P_{k}^{'}t\textrm{, u.o.c.,}
\end{equation}

\begin{equation}\label{E1.4.3}
\frac{1}{|x_{n}|}E^{x_{n}}_{k}\left(|x_{n}|t\right)\rightarrow
\alpha_{k}\left(t-\overline{U}_{k}\right)^{+}\textrm{, u.o.c.,}
\end{equation}

\begin{equation}\label{E1.4.4}
\frac{1}{|x_{n}|}S^{x_{n}}_{k}\left(|x_{n}|t\right)\rightarrow
\mu_{k}\left(t-\overline{V}_{k}\right)^{+}\textrm{, u.o.c.,}
\end{equation}

donde $\left[t\right]$ es la parte entera de $t$ y
$\mu_{k}=1/m_{k}=1/\esp\left[\eta_{k}\left(1\right)\right]$.
\end{Lema}

\begin{Lema}[Lema 4.3, Dai\cite{Dai}]\label{Lema.4.3}
Sea $\left\{x_{n}\right\}\subset \mathbf{X}$ con
$|x_{n}|\rightarrow\infty$, conforme $n\rightarrow\infty$. Suponga
que
\[lim_{n\rightarrow\infty}\frac{1}{|x_{n}|}U_{k}\left(0\right)=\overline{U}_{k}\]
y
\[lim_{n\rightarrow\infty}\frac{1}{|x_{n}|}V_{k}\left(0\right)=\overline{V}_{k}.\]
\begin{itemize}
\item[a)] Conforme $n\rightarrow\infty$ casi seguramente,
\[lim_{n\rightarrow\infty}\frac{1}{|x_{n}|}U^{x_{n}}_{k}\left(|x_{n}|t\right)=\left(\overline{U}_{k}-t\right)^{+}\textrm{, u.o.c.}\]
y
\[lim_{n\rightarrow\infty}\frac{1}{|x_{n}|}V^{x_{n}}_{k}\left(|x_{n}|t\right)=\left(\overline{V}_{k}-t\right)^{+}.\]

\item[b)] Para cada $t\geq0$ fijo,
\[\left\{\frac{1}{|x_{n}|}U^{x_{n}}_{k}\left(|x_{n}|t\right),|x_{n}|\geq1\right\}\]
y
\[\left\{\frac{1}{|x_{n}|}V^{x_{n}}_{k}\left(|x_{n}|t\right),|x_{n}|\geq1\right\}\]
\end{itemize}
son uniformemente convergentes.
\end{Lema}

Sea $S_{l}^{x}\left(t\right)$ el n\'umero total de servicios
completados de la clase $l$, si la clase $l$ est\'a dando $t$
unidades de tiempo de servicio. Sea $T_{l}^{x}\left(x\right)$ el
monto acumulado del tiempo de servicio que el servidor
$s\left(l\right)$ gasta en los usuarios de la clase $l$ al tiempo
$t$. Entonces $S_{l}^{x}\left(T_{l}^{x}\left(t\right)\right)$ es
el n\'umero total de servicios completados para la clase $l$ al
tiempo $t$. Una fracci\'on de estos usuarios,
$\Phi_{k}^{x}\left(S_{l}^{x}\left(T_{l}^{x}\left(t\right)\right)\right)$,
se convierte en usuarios de la clase $k$.\\

Entonces, dado lo anterior, se tiene la siguiente representaci\'on
para el proceso de la longitud de la cola:\\

\begin{equation}
Q_{k}^{x}\left(t\right)=Q_{k}^{x}\left(0\right)+E_{k}^{x}\left(t\right)+\sum_{l=1}^{K}\Phi_{k}^{l}\left(S_{l}^{x}\left(T_{l}^{x}\left(t\right)\right)\right)-S_{k}^{x}\left(T_{k}^{x}\left(t\right)\right)
\end{equation}
para $k=1,\ldots,K$. Para $i=1,\ldots,d$, sea
\[I_{i}^{x}\left(t\right)=t-\sum_{j\in C_{i}}T_{k}^{x}\left(t\right).\]

Entonces $I_{i}^{x}\left(t\right)$ es el monto acumulado del
tiempo que el servidor $i$ ha estado desocupado al tiempo $t$. Se
est\'a asumiendo que las disciplinas satisfacen la ley de
conservaci\'on del trabajo, es decir, el servidor $i$ est\'a en
pausa solamente cuando no hay usuarios en la estaci\'on $i$.
Entonces, se tiene que

\begin{equation}
\int_{0}^{\infty}\left(\sum_{k\in
C_{i}}Q_{k}^{x}\left(t\right)\right)dI_{i}^{x}\left(t\right)=0,
\end{equation}
para $i=1,\ldots,d$.\\

Hacer
\[T^{x}\left(t\right)=\left(T_{1}^{x}\left(t\right),\ldots,T_{K}^{x}\left(t\right)\right)^{'},\]
\[I^{x}\left(t\right)=\left(I_{1}^{x}\left(t\right),\ldots,I_{K}^{x}\left(t\right)\right)^{'}\]
y
\[S^{x}\left(T^{x}\left(t\right)\right)=\left(S_{1}^{x}\left(T_{1}^{x}\left(t\right)\right),\ldots,S_{K}^{x}\left(T_{K}^{x}\left(t\right)\right)\right)^{'}.\]

Para una disciplina que cumple con la ley de conservaci\'on del
trabajo, en forma vectorial, se tiene el siguiente conjunto de
ecuaciones

\begin{equation}\label{Eq.MF.1.3}
Q^{x}\left(t\right)=Q^{x}\left(0\right)+E^{x}\left(t\right)+\sum_{l=1}^{K}\Phi^{l}\left(S_{l}^{x}\left(T_{l}^{x}\left(t\right)\right)\right)-S^{x}\left(T^{x}\left(t\right)\right),\\
\end{equation}

\begin{equation}\label{Eq.MF.2.3}
Q^{x}\left(t\right)\geq0,\\
\end{equation}

\begin{equation}\label{Eq.MF.3.3}
T^{x}\left(0\right)=0,\textrm{ y }\overline{T}^{x}\left(t\right)\textrm{ es no decreciente},\\
\end{equation}

\begin{equation}\label{Eq.MF.4.3}
I^{x}\left(t\right)=et-CT^{x}\left(t\right)\textrm{ es no
decreciente}\\
\end{equation}

\begin{equation}\label{Eq.MF.5.3}
\int_{0}^{\infty}\left(CQ^{x}\left(t\right)\right)dI_{i}^{x}\left(t\right)=0,\\
\end{equation}

\begin{equation}\label{Eq.MF.6.3}
\textrm{Condiciones adicionales en
}\left(\overline{Q}^{x}\left(\cdot\right),\overline{T}^{x}\left(\cdot\right)\right)\textrm{
espec\'ificas de la disciplina de la cola,}
\end{equation}

donde $e$ es un vector de unos de dimensi\'on $d$, $C$ es la
matriz definida por
\[C_{ik}=\left\{\begin{array}{cc}
1,& S\left(k\right)=i,\\
0,& \textrm{ en otro caso}.\\
\end{array}\right.
\]
Es necesario enunciar el siguiente Teorema que se utilizar\'a para
el Teorema \ref{Tma.4.2.Dai}:
\begin{Teo}[Teorema 4.1, Dai \cite{Dai}]
Considere una disciplina que cumpla la ley de conservaci\'on del
trabajo, para casi todas las trayectorias muestrales $\omega$ y
cualquier sucesi\'on de estados iniciales
$\left\{x_{n}\right\}\subset \mathbf{X}$, con
$|x_{n}|\rightarrow\infty$, existe una subsucesi\'on
$\left\{x_{n_{j}}\right\}$ con $|x_{n_{j}}|\rightarrow\infty$ tal
que
\begin{equation}\label{Eq.4.15}
\frac{1}{|x_{n_{j}}|}\left(Q^{x_{n_{j}}}\left(0\right),U^{x_{n_{j}}}\left(0\right),V^{x_{n_{j}}}\left(0\right)\right)\rightarrow\left(\overline{Q}\left(0\right),\overline{U},\overline{V}\right),
\end{equation}

\begin{equation}\label{Eq.4.16}
\frac{1}{|x_{n_{j}}|}\left(Q^{x_{n_{j}}}\left(|x_{n_{j}}|t\right),T^{x_{n_{j}}}\left(|x_{n_{j}}|t\right)\right)\rightarrow\left(\overline{Q}\left(t\right),\overline{T}\left(t\right)\right)\textrm{
u.o.c.}
\end{equation}

Adem\'as,
$\left(\overline{Q}\left(t\right),\overline{T}\left(t\right)\right)$
satisface las siguientes ecuaciones:
\begin{equation}\label{Eq.MF.1.3a}
\overline{Q}\left(t\right)=Q\left(0\right)+\left(\alpha
t-\overline{U}\right)^{+}-\left(I-P\right)^{'}M^{-1}\left(\overline{T}\left(t\right)-\overline{V}\right)^{+},
\end{equation}

\begin{equation}\label{Eq.MF.2.3a}
\overline{Q}\left(t\right)\geq0,\\
\end{equation}

\begin{equation}\label{Eq.MF.3.3a}
\overline{T}\left(t\right)\textrm{ es no decreciente y comienza en cero},\\
\end{equation}

\begin{equation}\label{Eq.MF.4.3a}
\overline{I}\left(t\right)=et-C\overline{T}\left(t\right)\textrm{
es no decreciente,}\\
\end{equation}

\begin{equation}\label{Eq.MF.5.3a}
\int_{0}^{\infty}\left(C\overline{Q}\left(t\right)\right)d\overline{I}\left(t\right)=0,\\
\end{equation}

\begin{equation}\label{Eq.MF.6.3a}
\textrm{Condiciones adicionales en
}\left(\overline{Q}\left(\cdot\right),\overline{T}\left(\cdot\right)\right)\textrm{
especficas de la disciplina de la cola,}
\end{equation}
\end{Teo}


Propiedades importantes para el modelo de flujo retrasado:

\begin{Prop}
 Sea $\left(\overline{Q},\overline{T},\overline{T}^{0}\right)$ un flujo l\'imite de \ref{Eq.4.4} y suponga que cuando $x\rightarrow\infty$ a lo largo de
una subsucesi\'on
\[\left(\frac{1}{|x|}Q_{k}^{x}\left(0\right),\frac{1}{|x|}A_{k}^{x}\left(0\right),\frac{1}{|x|}B_{k}^{x}\left(0\right),\frac{1}{|x|}B_{k}^{x,0}\left(0\right)\right)\rightarrow\left(\overline{Q}_{k}\left(0\right),0,0,0\right)\]
para $k=1,\ldots,K$. EL flujo l\'imite tiene las siguientes
propiedades, donde las propiedades de la derivada se cumplen donde
la derivada exista:
\begin{itemize}
 \item[i)] Los vectores de tiempo ocupado $\overline{T}\left(t\right)$ y $\overline{T}^{0}\left(t\right)$ son crecientes y continuas con
$\overline{T}\left(0\right)=\overline{T}^{0}\left(0\right)=0$.
\item[ii)] Para todo $t\geq0$
\[\sum_{k=1}^{K}\left[\overline{T}_{k}\left(t\right)+\overline{T}_{k}^{0}\left(t\right)\right]=t\]
\item[iii)] Para todo $1\leq k\leq K$
\[\overline{Q}_{k}\left(t\right)=\overline{Q}_{k}\left(0\right)+\alpha_{k}t-\mu_{k}\overline{T}_{k}\left(t\right)\]
\item[iv)]  Para todo $1\leq k\leq K$
\[\dot{{\overline{T}}}_{k}\left(t\right)=\beta_{k}\] para $\overline{Q}_{k}\left(t\right)=0$.
\item[v)] Para todo $k,j$
\[\mu_{k}^{0}\overline{T}_{k}^{0}\left(t\right)=\mu_{j}^{0}\overline{T}_{j}^{0}\left(t\right)\]
\item[vi)]  Para todo $1\leq k\leq K$
\[\mu_{k}\dot{{\overline{T}}}_{k}\left(t\right)=l_{k}\mu_{k}^{0}\dot{{\overline{T}}}_{k}^{0}\left(t\right)\] para $\overline{Q}_{k}\left(t\right)>0$.
\end{itemize}
\end{Prop}

\begin{Teo}[Teorema 5.1: Ley Fuerte para Procesos de Conteo
\cite{Gut}]\label{Tma.5.1.Gut} Sea
$0<\mu<\esp\left(X_{1}\right]\leq\infty$. entonces

\begin{itemize}
\item[a)] $\frac{N\left(t\right)}{t}\rightarrow\frac{1}{\mu}$
a.s., cuando $t\rightarrow\infty$.


\item[b)]$\esp\left[\frac{N\left(t\right)}{t}\right]^{r}\rightarrow\frac{1}{\mu^{r}}$,
cuando $t\rightarrow\infty$ para todo $r>0$..
\end{itemize}
\end{Teo}


\begin{Prop}[Proposici\'on 5.3 \cite{DaiSean}]
Sea $X$ proceso de estados para la red de colas, y suponga que se
cumplen los supuestos (A1) y (A2), entonces para alguna constante
positiva $C_{p+1}<\infty$, $\delta>0$ y un conjunto compacto
$C\subset X$.

\begin{equation}\label{Eq.5.4}
\esp_{x}\left[\int_{0}^{\tau_{C}\left(\delta\right)}\left(1+|X\left(t\right)|^{p}\right)dt\right]\leq
C_{p+1}\left(1+|x|^{p+1}\right)
\end{equation}
\end{Prop}

\begin{Prop}[Proposici\'on 5.4 \cite{DaiSean}]
Sea $X$ un proceso de Markov Borel Derecho en $X$, sea
$f:X\leftarrow\rea_{+}$ y defina para alguna $\delta>0$, y un
conjunto cerrado $C\subset X$
\[V\left(x\right):=\esp_{x}\left[\int_{0}^{\tau_{C}\left(\delta\right)}f\left(X\left(t\right)\right)dt\right]\]
para $x\in X$. Si $V$ es finito en todas partes y uniformemente
acotada en $C$, entonces existe $k<\infty$ tal que
\begin{equation}\label{Eq.5.11}
\frac{1}{t}\esp_{x}\left[V\left(x\right)\right]+\frac{1}{t}\int_{0}^{t}\esp_{x}\left[f\left(X\left(s\right)\right)ds\right]\leq\frac{1}{t}V\left(x\right)+k,
\end{equation}
para $x\in X$ y $t>0$.
\end{Prop}


%_________________________________________________________________________
%\renewcommand{\refname}{Ap\'endice D}
%\renewcommand{\appendixname}{ESTABILIDAD}
%\renewcommand{\appendixtocname}{ESTABILIDAD}
%\renewcommand{\appendixpagename}{ESTABILIDAD}
%\appendix
%\clearpage % o \cleardoublepage
%\addappheadtotoc
%\appendixpage

\subsection{Estabilidad}

\begin{Def}[Definici\'on 3.2, Dai y Meyn \cite{DaiSean}]
El modelo de flujo retrasado de una disciplina de servicio en una
red con retraso
$\left(\overline{A}\left(0\right),\overline{B}\left(0\right)\right)\in\rea_{+}^{K+|A|}$
se define como el conjunto de ecuaciones dadas en
\ref{Eq.3.8}-\ref{Eq.3.13}, junto con la condici\'on:
\begin{equation}\label{CondAd.FluidModel}
\overline{Q}\left(t\right)=\overline{Q}\left(0\right)+\left(\alpha
t-\overline{A}\left(0\right)\right)^{+}-\left(I-P^{'}\right)M\left(\overline{T}\left(t\right)-\overline{B}\left(0\right)\right)^{+}
\end{equation}
\end{Def}

entonces si el modelo de flujo retrasado tambi\'en es estable:


\begin{Def}[Definici\'on 3.1, Dai y Meyn \cite{DaiSean}]
Un flujo l\'imite (retrasado) para una red bajo una disciplina de
servicio espec\'ifica se define como cualquier soluci\'on
 $\left(\overline{Q}\left(\cdot\right),\overline{T}\left(\cdot\right)\right)$ de las siguientes ecuaciones, donde
$\overline{Q}\left(t\right)=\left(\overline{Q}_{1}\left(t\right),\ldots,\overline{Q}_{K}\left(t\right)\right)^{'}$
y
$\overline{T}\left(t\right)=\left(\overline{T}_{1}\left(t\right),\ldots,\overline{T}_{K}\left(t\right)\right)^{'}$
\begin{equation}\label{Eq.3.8}
\overline{Q}_{k}\left(t\right)=\overline{Q}_{k}\left(0\right)+\alpha_{k}t-\mu_{k}\overline{T}_{k}\left(t\right)+\sum_{l=1}^{k}P_{lk}\mu_{l}\overline{T}_{l}\left(t\right)\\
\end{equation}
\begin{equation}\label{Eq.3.9}
\overline{Q}_{k}\left(t\right)\geq0\textrm{ para }k=1,2,\ldots,K,\\
\end{equation}
\begin{equation}\label{Eq.3.10}
\overline{T}_{k}\left(0\right)=0,\textrm{ y }\overline{T}_{k}\left(\cdot\right)\textrm{ es no decreciente},\\
\end{equation}
\begin{equation}\label{Eq.3.11}
\overline{I}_{i}\left(t\right)=t-\sum_{k\in C_{i}}\overline{T}_{k}\left(t\right)\textrm{ es no decreciente}\\
\end{equation}
\begin{equation}\label{Eq.3.12}
\overline{I}_{i}\left(\cdot\right)\textrm{ se incrementa al tiempo }t\textrm{ cuando }\sum_{k\in C_{i}}Q_{k}^{x}\left(t\right)dI_{i}^{x}\left(t\right)=0\\
\end{equation}
\begin{equation}\label{Eq.3.13}
\textrm{condiciones adicionales sobre
}\left(Q^{x}\left(\cdot\right),T^{x}\left(\cdot\right)\right)\textrm{
referentes a la disciplina de servicio}
\end{equation}
\end{Def}

\begin{Lema}[Lema 3.1 \cite{Chen}]\label{Lema3.1}
Si el modelo de flujo es estable, definido por las ecuaciones
(3.8)-(3.13), entonces el modelo de flujo retrasado tambin es
estable.
\end{Lema}

\begin{Teo}[Teorema 5.1 \cite{Chen}]\label{Tma.5.1.Chen}
La red de colas es estable si existe una constante $t_{0}$ que
depende de $\left(\alpha,\mu,T,U\right)$ y $V$ que satisfagan las
ecuaciones (5.1)-(5.5), $Z\left(t\right)=0$, para toda $t\geq
t_{0}$.
\end{Teo}

\begin{Prop}[Proposici\'on 5.1, Dai y Meyn \cite{DaiSean}]\label{Prop.5.1.DaiSean}
Suponga que los supuestos A1) y A2) son ciertos y que el modelo de flujo es estable. Entonces existe $t_{0}>0$ tal que
\begin{equation}
lim_{|x|\rightarrow\infty}\frac{1}{|x|^{p+1}}\esp_{x}\left[|X\left(t_{0}|x|\right)|^{p+1}\right]=0
\end{equation}
\end{Prop}

\begin{Lemma}[Lema 5.2, Dai y Meyn \cite{DaiSean}]\label{Lema.5.2.DaiSean}
 Sea $\left\{\zeta\left(k\right):k\in \mathbb{z}\right\}$ una sucesi\'on independiente e id\'enticamente distribuida que toma valores en $\left(0,\infty\right)$,
y sea
$E\left(t\right)=max\left(n\geq1:\zeta\left(1\right)+\cdots+\zeta\left(n-1\right)\leq
t\right)$. Si $\esp\left[\zeta\left(1\right)\right]<\infty$,
entonces para cualquier entero $r\geq1$
\begin{equation}
 lim_{t\rightarrow\infty}\esp\left[\left(\frac{E\left(t\right)}{t}\right)^{r}\right]=\left(\frac{1}{\esp\left[\zeta_{1}\right]}\right)^{r}.
\end{equation}
Luego, bajo estas condiciones:
\begin{itemize}
 \item[a)] para cualquier $\delta>0$, $\sup_{t\geq\delta}\esp\left[\left(\frac{E\left(t\right)}{t}\right)^{r}\right]<\infty$
\item[b)] las variables aleatorias
$\left\{\left(\frac{E\left(t\right)}{t}\right)^{r}:t\geq1\right\}$
son uniformemente integrables.
\end{itemize}
\end{Lemma}

\begin{Teo}[Teorema 5.5, Dai y Meyn \cite{DaiSean}]\label{Tma.5.5.DaiSean}
Suponga que los supuestos A1) y A2) se cumplen y que el modelo de
flujo es estable. Entonces existe una constante $\kappa_{p}$ tal
que
\begin{equation}
\frac{1}{t}\int_{0}^{t}\esp_{x}\left[|Q\left(s\right)|^{p}\right]ds\leq\kappa_{p}\left\{\frac{1}{t}|x|^{p+1}+1\right\}
\end{equation}
para $t>0$ y $x\in X$. En particular, para cada condici\'on
inicial
\begin{eqnarray*}
\limsup_{t\rightarrow\infty}\frac{1}{t}\int_{0}^{t}\esp_{x}\left[|Q\left(s\right)|^{p}\right]ds\leq\kappa_{p}.
\end{eqnarray*}
\end{Teo}

\begin{Teo}[Teorema 6.2, Dai y Meyn \cite{DaiSean}]\label{Tma.6.2.DaiSean}
Suponga que se cumplen los supuestos A1), A2) y A3) y que el
modelo de flujo es estable. Entonces se tiene que
\begin{equation}
\left\|P^{t}\left(x,\cdot\right)-\pi\left(\cdot\right)\right\|_{f_{p}}\textrm{,
}t\rightarrow\infty,x\in X.
\end{equation}
En particular para cada condici\'on inicial
\begin{eqnarray*}
\lim_{t\rightarrow\infty}\esp_{x}\left[|Q\left(t\right)|^{p}\right]=\esp_{\pi}\left[|Q\left(0\right)|^{p}\right]\leq\kappa_{r}
\end{eqnarray*}
\end{Teo}
\begin{Teo}[Teorema 6.3, Dai y Meyn \cite{DaiSean}]\label{Tma.6.3.DaiSean}
Suponga que se cumplen los supuestos A1), A2) y A3) y que el
modelo de flujo es estable. Entonces con
$f\left(x\right)=f_{1}\left(x\right)$ se tiene
\begin{equation}
\lim_{t\rightarrow\infty}t^{p-1}\left\|P^{t}\left(x,\cdot\right)-\pi\left(\cdot\right)\right\|_{f}=0.
\end{equation}
En particular para cada condici\'on inicial
\begin{eqnarray*}
\lim_{t\rightarrow\infty}t^{p-1}|\esp_{x}\left[Q\left(t\right)\right]-\esp_{\pi}\left[Q\left(0\right)\right]|=0.
\end{eqnarray*}
\end{Teo}

\begin{Teo}[Teorema 6.4, Dai y Meyn \cite{DaiSean}]\label{Tma.6.4.DaiSean}
Suponga que se cumplen los supuestos A1), A2) y A3) y que el
modelo de flujo es estable. Sea $\nu$ cualquier distribuci\'on de
probabilidad en $\left(X,\mathcal{B}_{X}\right)$, y $\pi$ la
distribuci\'on estacionaria de $X$.
\begin{itemize}
\item[i)] Para cualquier $f:X\leftarrow\rea_{+}$
\begin{equation}
\lim_{t\rightarrow\infty}\frac{1}{t}\int_{o}^{t}f\left(X\left(s\right)\right)ds=\pi\left(f\right):=\int
f\left(x\right)\pi\left(dx\right)
\end{equation}
$\prob$-c.s.

\item[ii)] Para cualquier $f:X\leftarrow\rea_{+}$ con
$\pi\left(|f|\right)<\infty$, la ecuaci\'on anterior se cumple.
\end{itemize}
\end{Teo}

\begin{Teo}[Teorema 2.2, Down \cite{Down}]\label{Tma2.2.Down}
Suponga que el fluido modelo es inestable en el sentido de que
para alguna $\epsilon_{0},c_{0}\geq0$,
\begin{equation}\label{Eq.Inestability}
|Q\left(T\right)|\geq\epsilon_{0}T-c_{0}\textrm{,   }T\geq0,
\end{equation}
para cualquier condici\'on inicial $Q\left(0\right)$, con
$|Q\left(0\right)|=1$. Entonces para cualquier $0<q\leq1$, existe
$B<0$ tal que para cualquier $|x|\geq B$,
\begin{equation}
\prob_{x}\left\{\mathbb{X}\rightarrow\infty\right\}\geq q.
\end{equation}
\end{Teo}


\begin{Def}
Sea $X$ un conjunto y $\mathcal{F}$ una $\sigma$-\'algebra de
subconjuntos de $X$, la pareja $\left(X,\mathcal{F}\right)$ es
llamado espacio medible. Un subconjunto $A$ de $X$ es llamado
medible, o medible con respecto a $\mathcal{F}$, si
$A\in\mathcal{F}$.
\end{Def}

\begin{Def}
Sea $\left(X,\mathcal{F},\mu\right)$ espacio de medida. Se dice
que la medida $\mu$ es $\sigma$-finita si se puede escribir
$X=\bigcup_{n\geq1}X_{n}$ con $X_{n}\in\mathcal{F}$ y
$\mu\left(X_{n}\right)<\infty$.
\end{Def}

\begin{Def}\label{Cto.Borel}
Sea $X$ el conjunto de los \'umeros reales $\rea$. El \'algebra de
Borel es la $\sigma$-\'algebra $B$ generada por los intervalos
abiertos $\left(a,b\right)\in\rea$. Cualquier conjunto en $B$ es
llamado {\em Conjunto de Borel}.
\end{Def}

\begin{Def}\label{Funcion.Medible}
Una funci\'on $f:X\rightarrow\rea$, es medible si para cualquier
n\'umero real $\alpha$ el conjunto
\[\left\{x\in X:f\left(x\right)>\alpha\right\}\]
pertenece a $X$. Equivalentemente, se dice que $f$ es medible si
\[f^{-1}\left(\left(\alpha,\infty\right)\right)=\left\{x\in X:f\left(x\right)>\alpha\right\}\in\mathcal{F}.\]
\end{Def}


\begin{Def}\label{Def.Cilindros}
Sean $\left(\Omega_{i},\mathcal{F}_{i}\right)$, $i=1,2,\ldots,$
espacios medibles y $\Omega=\prod_{i=1}^{\infty}\Omega_{i}$ el
conjunto de todas las sucesiones
$\left(\omega_{1},\omega_{2},\ldots,\right)$ tales que
$\omega_{i}\in\Omega_{i}$, $i=1,2,\ldots,$. Si
$B^{n}\subset\prod_{i=1}^{\infty}\Omega_{i}$, definimos
$B_{n}=\left\{\omega\in\Omega:\left(\omega_{1},\omega_{2},\ldots,\omega_{n}\right)\in
B^{n}\right\}$. Al conjunto $B_{n}$ se le llama {\em cilindro} con
base $B^{n}$, el cilindro es llamado medible si
$B^{n}\in\prod_{i=1}^{\infty}\mathcal{F}_{i}$.
\end{Def}


\begin{Def}\label{Def.Proc.Adaptado}[TSP, Ash \cite{RBA}]
Sea $X\left(t\right),t\geq0$ proceso estoc\'astico, el proceso es
adaptado a la familia de $\sigma$-\'algebras $\mathcal{F}_{t}$,
para $t\geq0$, si para $s<t$ implica que
$\mathcal{F}_{s}\subset\mathcal{F}_{t}$, y $X\left(t\right)$ es
$\mathcal{F}_{t}$-medible para cada $t$. Si no se especifica
$\mathcal{F}_{t}$ entonces se toma $\mathcal{F}_{t}$ como
$\mathcal{F}\left(X\left(s\right),s\leq t\right)$, la m\'as
peque\~na $\sigma$-\'algebra de subconjuntos de $\Omega$ que hace
que cada $X\left(s\right)$, con $s\leq t$ sea Borel medible.
\end{Def}


\begin{Def}\label{Def.Tiempo.Paro}[TSP, Ash \cite{RBA}]
Sea $\left\{\mathcal{F}\left(t\right),t\geq0\right\}$ familia
creciente de sub $\sigma$-\'algebras. es decir,
$\mathcal{F}\left(s\right)\subset\mathcal{F}\left(t\right)$ para
$s\leq t$. Un tiempo de paro para $\mathcal{F}\left(t\right)$ es
una funci\'on $T:\Omega\rightarrow\left[0,\infty\right]$ tal que
$\left\{T\leq t\right\}\in\mathcal{F}\left(t\right)$ para cada
$t\geq0$. Un tiempo de paro para el proceso estoc\'astico
$X\left(t\right),t\geq0$ es un tiempo de paro para las
$\sigma$-\'algebras
$\mathcal{F}\left(t\right)=\mathcal{F}\left(X\left(s\right)\right)$.
\end{Def}

\begin{Def}
Sea $X\left(t\right),t\geq0$ proceso estoc\'astico, con
$\left(S,\chi\right)$ espacio de estados. Se dice que el proceso
es adaptado a $\left\{\mathcal{F}\left(t\right)\right\}$, es
decir, si para cualquier $s,t\in I$, $I$ conjunto de \'indices,
$s<t$, se tiene que
$\mathcal{F}\left(s\right)\subset\mathcal{F}\left(t\right)$ y
$X\left(t\right)$ es $\mathcal{F}\left(t\right)$-medible,
\end{Def}

\begin{Def}
Sea $X\left(t\right),t\geq0$ proceso estoc\'astico, se dice que es
un Proceso de Markov relativo a $\mathcal{F}\left(t\right)$ o que
$\left\{X\left(t\right),\mathcal{F}\left(t\right)\right\}$ es de
Markov si y s\'olo si para cualquier conjunto $B\in\chi$,  y
$s,t\in I$, $s<t$ se cumple que
\begin{equation}\label{Prop.Markov}
P\left\{X\left(t\right)\in
B|\mathcal{F}\left(s\right)\right\}=P\left\{X\left(t\right)\in
B|X\left(s\right)\right\}.
\end{equation}
\end{Def}
\begin{Note}
Si se dice que $\left\{X\left(t\right)\right\}$ es un Proceso de
Markov sin mencionar $\mathcal{F}\left(t\right)$, se asumir\'a que
\begin{eqnarray*}
\mathcal{F}\left(t\right)=\mathcal{F}_{0}\left(t\right)=\mathcal{F}\left(X\left(r\right),r\leq
t\right),
\end{eqnarray*}
entonces la ecuaci\'on (\ref{Prop.Markov}) se puede escribir como
\begin{equation}
P\left\{X\left(t\right)\in B|X\left(r\right),r\leq s\right\} =
P\left\{X\left(t\right)\in B|X\left(s\right)\right\}
\end{equation}
\end{Note}

\begin{Teo}
Sea $\left(X_{n},\mathcal{F}_{n},n=0,1,\ldots,\right\}$ Proceso de
Markov con espacio de estados $\left(S_{0},\chi_{0}\right)$
generado por una distribuici\'on inicial $P_{o}$ y probabilidad de
transici\'on $p_{mn}$, para $m,n=0,1,\ldots,$ $m<n$, que por
notaci\'on se escribir\'a como $p\left(m,n,x,B\right)\rightarrow
p_{mn}\left(x,B\right)$. Sea $S$ tiempo de paro relativo a la
$\sigma$-\'algebra $\mathcal{F}_{n}$. Sea $T$ funci\'on medible,
$T:\Omega\rightarrow\left\{0,1,\ldots,\right\}$. Sup\'ongase que
$T\geq S$, entonces $T$ es tiempo de paro. Si $B\in\chi_{0}$,
entonces
\begin{equation}\label{Prop.Fuerte.Markov}
P\left\{X\left(T\right)\in
B,T<\infty|\mathcal{F}\left(S\right)\right\} =
p\left(S,T,X\left(s\right),B\right)
\end{equation}
en $\left\{T<\infty\right\}$.
\end{Teo}


Sea $K$ conjunto numerable y sea $d:K\rightarrow\nat$ funci\'on.
Para $v\in K$, $M_{v}$ es un conjunto abierto de
$\rea^{d\left(v\right)}$. Entonces \[E=\cup_{v\in
K}M_{v}=\left\{\left(v,\zeta\right):v\in K,\zeta\in
M_{v}\right\}.\]

Sea $\mathcal{E}$ la clase de conjuntos medibles en $E$:
\[\mathcal{E}=\left\{\cup_{v\in K}A_{v}:A_{v}\in \mathcal{M}_{v}\right\}.\]

donde $\mathcal{M}$ son los conjuntos de Borel de $M_{v}$.
Entonces $\left(E,\mathcal{E}\right)$ es un espacio de Borel. El
estado del proceso se denotar\'a por
$\mathbf{x}_{t}=\left(v_{t},\zeta_{t}\right)$. La distribuci\'on
de $\left(\mathbf{x}_{t}\right)$ est\'a determinada por por los
siguientes objetos:

\begin{itemize}
\item[i)] Los campos vectoriales $\left(\mathcal{H}_{v},v\in
K\right)$. \item[ii)] Una funci\'on medible $\lambda:E\rightarrow
\rea_{+}$. \item[iii)] Una medida de transici\'on
$Q:\mathcal{E}\times\left(E\cup\Gamma^{*}\right)\rightarrow\left[0,1\right]$
donde
\begin{equation}
\Gamma^{*}=\cup_{v\in K}\partial^{*}M_{v}.
\end{equation}
y
\begin{equation}
\partial^{*}M_{v}=\left\{z\in\partial M_{v}:\mathbf{\mathbf{\phi}_{v}\left(t,\zeta\right)=\mathbf{z}}\textrm{ para alguna }\left(t,\zeta\right)\in\rea_{+}\times M_{v}\right\}.
\end{equation}
$\partial M_{v}$ denota  la frontera de $M_{v}$.
\end{itemize}

El campo vectorial $\left(\mathcal{H}_{v},v\in K\right)$ se supone
tal que para cada $\mathbf{z}\in M_{v}$ existe una \'unica curva
integral $\mathbf{\phi}_{v}\left(t,\zeta\right)$ que satisface la
ecuaci\'on

\begin{equation}
\frac{d}{dt}f\left(\zeta_{t}\right)=\mathcal{H}f\left(\zeta_{t}\right),
\end{equation}
con $\zeta_{0}=\mathbf{z}$, para cualquier funci\'on suave
$f:\rea^{d}\rightarrow\rea$ y $\mathcal{H}$ denota el operador
diferencial de primer orden, con $\mathcal{H}=\mathcal{H}_{v}$ y
$\zeta_{t}=\mathbf{\phi}\left(t,\mathbf{z}\right)$. Adem\'as se
supone que $\mathcal{H}_{v}$ es conservativo, es decir, las curvas
integrales est\'an definidas para todo $t>0$.

Para $\mathbf{x}=\left(v,\zeta\right)\in E$ se denota
\[t^{*}\mathbf{x}=inf\left\{t>0:\mathbf{\phi}_{v}\left(t,\zeta\right)\in\partial^{*}M_{v}\right\}\]

En lo que respecta a la funci\'on $\lambda$, se supondr\'a que
para cada $\left(v,\zeta\right)\in E$ existe un $\epsilon>0$ tal
que la funci\'on
$s\rightarrow\lambda\left(v,\phi_{v}\left(s,\zeta\right)\right)\in
E$ es integrable para $s\in\left[0,\epsilon\right)$. La medida de
transici\'on $Q\left(A;\mathbf{x}\right)$ es una funci\'on medible
de $\mathbf{x}$ para cada $A\in\mathcal{E}$, definida para
$\mathbf{x}\in E\cup\Gamma^{*}$ y es una medida de probabilidad en
$\left(E,\mathcal{E}\right)$ para cada $\mathbf{x}\in E$.

El movimiento del proceso $\left(\mathbf{x}_{t}\right)$ comenzando
en $\mathbf{x}=\left(n,\mathbf{z}\right)\in E$ se puede construir
de la siguiente manera, def\'inase la funci\'on $F$ por

\begin{equation}
F\left(t\right)=\left\{\begin{array}{ll}\\
exp\left(-\int_{0}^{t}\lambda\left(n,\phi_{n}\left(s,\mathbf{z}\right)\right)ds\right), & t<t^{*}\left(\mathbf{x}\right),\\
0, & t\geq t^{*}\left(\mathbf{x}\right)
\end{array}\right.
\end{equation}

Sea $T_{1}$ una variable aleatoria tal que
$\prob\left[T_{1}>t\right]=F\left(t\right)$, ahora sea la variable
aleatoria $\left(N,Z\right)$ con distribuici\'on
$Q\left(\cdot;\phi_{n}\left(T_{1},\mathbf{z}\right)\right)$. La
trayectoria de $\left(\mathbf{x}_{t}\right)$ para $t\leq T_{1}$
es\footnote{Revisar p\'agina 362, y 364 de Davis \cite{Davis}.}
\begin{eqnarray*}
\mathbf{x}_{t}=\left(v_{t},\zeta_{t}\right)=\left\{\begin{array}{ll}
\left(n,\phi_{n}\left(t,\mathbf{z}\right)\right), & t<T_{1},\\
\left(N,\mathbf{Z}\right), & t=t_{1}.
\end{array}\right.
\end{eqnarray*}

Comenzando en $\mathbf{x}_{T_{1}}$ se selecciona el siguiente
tiempo de intersalto $T_{2}-T_{1}$ lugar del post-salto
$\mathbf{x}_{T_{2}}$ de manera similar y as\'i sucesivamente. Este
procedimiento nos da una trayectoria determinista por partes
$\mathbf{x}_{t}$ con tiempos de salto $T_{1},T_{2},\ldots$. Bajo
las condiciones enunciadas para $\lambda,T_{1}>0$  y
$T_{1}-T_{2}>0$ para cada $i$, con probabilidad 1. Se supone que
se cumple la siquiente condici\'on.

\begin{Sup}[Supuesto 3.1, Davis \cite{Davis}]\label{Sup3.1.Davis}
Sea $N_{t}:=\sum_{t}\indora_{\left(t\geq t\right)}$ el n\'umero de
saltos en $\left[0,t\right]$. Entonces
\begin{equation}
\esp\left[N_{t}\right]<\infty\textrm{ para toda }t.
\end{equation}
\end{Sup}

es un proceso de Markov, m\'as a\'un, es un Proceso Fuerte de
Markov, es decir, la Propiedad Fuerte de Markov se cumple para
cualquier tiempo de paro.
%_________________________________________________________________________

En esta secci\'on se har\'an las siguientes consideraciones: $E$
es un espacio m\'etrico separable y la m\'etrica $d$ es compatible
con la topolog\'ia.


\begin{Def}
Un espacio topol\'ogico $E$ es llamado {\em Luisin} si es
homeomorfo a un subconjunto de Borel de un espacio m\'etrico
compacto.
\end{Def}

\begin{Def}
Un espacio topol\'ogico $E$ es llamado de {\em Rad\'on} si es
homeomorfo a un subconjunto universalmente medible de un espacio
m\'etrico compacto.
\end{Def}

Equivalentemente, la definici\'on de un espacio de Rad\'on puede
encontrarse en los siguientes t\'erminos:


\begin{Def}
$E$ es un espacio de Rad\'on si cada medida finita en
$\left(E,\mathcal{B}\left(E\right)\right)$ es regular interior o cerrada,
{\em tight}.
\end{Def}

\begin{Def}
Una medida finita, $\lambda$ en la $\sigma$-\'algebra de Borel de
un espacio metrizable $E$ se dice cerrada si
\begin{equation}\label{Eq.A2.3}
\lambda\left(E\right)=sup\left\{\lambda\left(K\right):K\textrm{ es
compacto en }E\right\}.
\end{equation}
\end{Def}

El siguiente teorema nos permite tener una mejor caracterizaci\'on de los espacios de Rad\'on:
\begin{Teo}\label{Tma.A2.2}
Sea $E$ espacio separable metrizable. Entonces $E$ es Radoniano si y s\'olo s\'i cada medida finita en $\left(E,\mathcal{B}\left(E\right)\right)$ es cerrada.
\end{Teo}

%_________________________________________________________________________________________
\subsection{Propiedades de Markov}
%_________________________________________________________________________________________

Sea $E$ espacio de estados, tal que $E$ es un espacio de Rad\'on, $\mathcal{B}\left(E\right)$ $\sigma$-\'algebra de Borel en $E$, que se denotar\'a por $\mathcal{E}$.

Sea $\left(X,\mathcal{G},\prob\right)$ espacio de probabilidad, $I\subset\rea$ conjunto de índices. Sea $\mathcal{F}_{\leq t}$ la $\sigma$-\'algebra natural definida como $\sigma\left\{f\left(X_{r}\right):r\in I, rleq t,f\in\mathcal{E}\right\}$. Se considerar\'a una $\sigma$-\'algebra m\'as general, $ \left(\mathcal{G}_{t}\right)$ tal que $\left(X_{t}\right)$ sea $\mathcal{E}$-adaptado.

\begin{Def}
Una familia $\left(P_{s,t}\right)$ de kernels de Markov en $\left(E,\mathcal{E}\right)$ indexada por pares $s,t\in I$, con $s\leq t$ es una funci\'on de transici\'on en $\ER$, si  para todo $r\leq s< t$ en $I$ y todo $x\in E$, $B\in\mathcal{E}$
\begin{equation}\label{Eq.Kernels}
P_{r,t}\left(x,B\right)=\int_{E}P_{r,s}\left(x,dy\right)P_{s,t}\left(y,B\right)\footnote{Ecuaci\'on de Chapman-Kolmogorov}.
\end{equation}
\end{Def}

Se dice que la funci\'on de transici\'on $\KM$ en $\ER$ es la funci\'on de transici\'on para un proceso $\PE$  con valores en $E$ y que satisface la propiedad de Markov\footnote{\begin{equation}\label{Eq.1.4.S}
\prob\left\{H|\mathcal{G}_{t}\right\}=\prob\left\{H|X_{t}\right\}\textrm{ }H\in p\mathcal{F}_{\geq t}.
\end{equation}} (\ref{Eq.1.4.S}) relativa a $\left(\mathcal{G}_{t}\right)$ si 

\begin{equation}\label{Eq.1.6.S}
\prob\left\{f\left(X_{t}\right)|\mathcal{G}_{s}\right\}=P_{s,t}f\left(X_{t}\right)\textrm{ }s\leq t\in I,\textrm{ }f\in b\mathcal{E}.
\end{equation}

\begin{Def}
Una familia $\left(P_{t}\right)_{t\geq0}$ de kernels de Markov en $\ER$ es llamada {\em Semigrupo de Transici\'on de Markov} o {\em Semigrupo de Transici\'on} si
\[P_{t+s}f\left(x\right)=P_{t}\left(P_{s}f\right)\left(x\right),\textrm{ }t,s\geq0,\textrm{ }x\in E\textrm{ }f\in b\mathcal{E}.\]
\end{Def}
\begin{Note}
Si la funci\'on de transici\'on $\KM$ es llamada homog\'enea si $P_{s,t}=P_{t-s}$.
\end{Note}

Un proceso de Markov que satisface la ecuaci\'on (\ref{Eq.1.6.S}) con funci\'on de transici\'on homog\'enea $\left(P_{t}\right)$ tiene la propiedad caracter\'istica
\begin{equation}\label{Eq.1.8.S}
\prob\left\{f\left(X_{t+s}\right)|\mathcal{G}_{t}\right\}=P_{s}f\left(X_{t}\right)\textrm{ }t,s\geq0,\textrm{ }f\in b\mathcal{E}.
\end{equation}
La ecuaci\'on anterior es la {\em Propiedad Simple de Markov} de $X$ relativa a $\left(P_{t}\right)$.

En este sentido el proceso $\PE$ cumple con la propiedad de Markov (\ref{Eq.1.8.S}) relativa a $\left(\Omega,\mathcal{G},\mathcal{G}_{t},\prob\right)$ con semigrupo de transici\'on $\left(P_{t}\right)$.
%_________________________________________________________________________________________
\subsection{Primer Condici\'on de Regularidad}
%_________________________________________________________________________________________
%\newcommand{\EM}{\left(\Omega,\mathcal{G},\prob\right)}
%\newcommand{\E4}{\left(\Omega,\mathcal{G},\mathcal{G}_{t},\prob\right)}
\begin{Def}
Un proceso estoc\'astico $\PE$ definido en $\left(\Omega,\mathcal{G},\prob\right)$ con valores en el espacio topol\'ogico $E$ es continuo por la derecha si cada trayectoria muestral $t\rightarrow X_{t}\left(w\right)$ es un mapeo continuo por la derecha de $I$ en $E$.
\end{Def}

\begin{Def}[HD1]\label{Eq.2.1.S}
Un semigrupo de Markov $\left/P_{t}\right)$ en un espacio de Rad\'on $E$ se dice que satisface la condici\'on {\em HD1} si, dada una medida de probabilidad $\mu$ en $E$, existe una $\sigma$-\'algebra $\mathcal{E^{*}}$ con $\mathcal{E}\subset\mathcal{E}$ y $P_{t}\left(b\mathcal{E}^{*}\right)\subset b\mathcal{E}^{*}$, y un $\mathcal{E}^{*}$-proceso $E$-valuado continuo por la derecha $\PE$ en alg\'un espacio de probabilidad filtrado $\left(\Omega,\mathcal{G},\mathcal{G}_{t},\prob\right)$ tal que $X=\left(\Omega,\mathcal{G},\mathcal{G}_{t},\prob\right)$ es de Markov (Homog\'eneo) con semigrupo de transici\'on $(P_{t})$ y distribuci\'on inicial $\mu$.
\end{Def}

Considerese la colecci\'on de variables aleatorias $X_{t}$ definidas en alg\'un espacio de probabilidad, y una colecci\'on de medidas $\mathbf{P}^{x}$ tales que $\mathbf{P}^{x}\left\{X_{0}=x\right\}$, y bajo cualquier $\mathbf{P}^{x}$, $X_{t}$ es de Markov con semigrupo $\left(P_{t}\right)$. $\mathbf{P}^{x}$ puede considerarse como la distribuci\'on condicional de $\mathbf{P}$ dado $X_{0}=x$.

\begin{Def}\label{Def.2.2.S}
Sea $E$ espacio de Rad\'on, $\SG$ semigrupo de Markov en $\ER$. La colecci\'on $\mathbf{X}=\left(\Omega,\mathcal{G},\mathcal{G}_{t},X_{t},\theta_{t},\CM\right)$ es un proceso $\mathcal{E}$-Markov continuo por la derecha simple, con espacio de estados $E$ y semigrupo de transici\'on $\SG$ en caso de que $\mathbf{X}$ satisfaga las siguientes condiciones:
\begin{itemize}
\item[i)] $\left(\Omega,\mathcal{G},\mathcal{G}_{t}\right)$ es un espacio de medida filtrado, y $X_{t}$ es un proceso $E$-valuado continuo por la derecha $\mathcal{E}^{*}$-adaptado a $\left(\mathcal{G}_{t}\right)$;

\item[ii)] $\left(\theta_{t}\right)_{t\geq0}$ es una colecci\'on de operadores {\em shift} para $X$, es decir, mapea $\Omega$ en s\'i mismo satisfaciendo para $t,s\geq0$,

\begin{equation}\label{Eq.Shift}
\theta_{t}\circ\theta_{s}=\theta_{t+s}\textrm{ y }X_{t}\circ\theta_{t}=X_{t+s};
\end{equation}

\item[iii)] Para cualquier $x\in E$,$\CM\left\{X_{0}=x\right\}=1$, y el proceso $\PE$ tiene la propiedad de Markov (\ref{Eq.1.8.S}) con semigrupo de transici\'on $\SG$ relativo a $\left(\Omega,\mathcal{G},\mathcal{G}_{t},\CM\right)$.
\end{itemize}
\end{Def}

\begin{Def}[HD2]\label{Eq.2.2.S}
Para cualquier $\alpha>0$ y cualquier $f\in S^{\alpha}$, el proceso $t\rightarrow f\left(X_{t}\right)$ es continuo por la derecha casi seguramente.
\end{Def}

\begin{Def}\label{Def.PD}
Un sistema $\mathbf{X}=\left(\Omega,\mathcal{G},\mathcal{G}_{t},X_{t},\theta_{t},\CM\right)$ es un proceso derecho en el espacio de Rad\'on $E$ con semigrupo de transici\'on $\SG$ provisto de:
\begin{itemize}
\item[i)] $\mathbf{X}$ es una realizaci\'on  continua por la derecha, \ref{Def.2.2.S}, de $\SG$.

\item[ii)] $\mathbf{X}$ satisface la condicion HD2, \ref{Eq.2.2.S}, relativa a $\mathcal{G}_{t}$.

\item[iii)] $\mathcal{G}_{t}$ es aumentado y continuo por la derecha.
\end{itemize}
\end{Def}




\begin{Lema}[Lema 4.2, Dai\cite{Dai}]\label{Lema4.2}
Sea $\left\{x_{n}\right\}\subset \mathbf{X}$ con
$|x_{n}|\rightarrow\infty$, conforme $n\rightarrow\infty$. Suponga
que
\[lim_{n\rightarrow\infty}\frac{1}{|x_{n}|}U\left(0\right)=\overline{U}\]
y
\[lim_{n\rightarrow\infty}\frac{1}{|x_{n}|}V\left(0\right)=\overline{V}.\]

Entonces, conforme $n\rightarrow\infty$, casi seguramente

\begin{equation}\label{E1.4.2}
\frac{1}{|x_{n}|}\Phi^{k}\left(\left[|x_{n}|t\right]\right)\rightarrow
P_{k}^{'}t\textrm{, u.o.c.,}
\end{equation}

\begin{equation}\label{E1.4.3}
\frac{1}{|x_{n}|}E^{x_{n}}_{k}\left(|x_{n}|t\right)\rightarrow
\alpha_{k}\left(t-\overline{U}_{k}\right)^{+}\textrm{, u.o.c.,}
\end{equation}

\begin{equation}\label{E1.4.4}
\frac{1}{|x_{n}|}S^{x_{n}}_{k}\left(|x_{n}|t\right)\rightarrow
\mu_{k}\left(t-\overline{V}_{k}\right)^{+}\textrm{, u.o.c.,}
\end{equation}

donde $\left[t\right]$ es la parte entera de $t$ y
$\mu_{k}=1/m_{k}=1/\esp\left[\eta_{k}\left(1\right)\right]$.
\end{Lema}

\begin{Lema}[Lema 4.3, Dai\cite{Dai}]\label{Lema.4.3}
Sea $\left\{x_{n}\right\}\subset \mathbf{X}$ con
$|x_{n}|\rightarrow\infty$, conforme $n\rightarrow\infty$. Suponga
que
\[lim_{n\rightarrow\infty}\frac{1}{|x_{n}|}U\left(0\right)=\overline{U}_{k}\]
y
\[lim_{n\rightarrow\infty}\frac{1}{|x_{n}|}V\left(0\right)=\overline{V}_{k}.\]
\begin{itemize}
\item[a)] Conforme $n\rightarrow\infty$ casi seguramente,
\[lim_{n\rightarrow\infty}\frac{1}{|x_{n}|}U^{x_{n}}_{k}\left(|x_{n}|t\right)=\left(\overline{U}_{k}-t\right)^{+}\textrm{, u.o.c.}\]
y
\[lim_{n\rightarrow\infty}\frac{1}{|x_{n}|}V^{x_{n}}_{k}\left(|x_{n}|t\right)=\left(\overline{V}_{k}-t\right)^{+}.\]

\item[b)] Para cada $t\geq0$ fijo,
\[\left\{\frac{1}{|x_{n}|}U^{x_{n}}_{k}\left(|x_{n}|t\right),|x_{n}|\geq1\right\}\]
y
\[\left\{\frac{1}{|x_{n}|}V^{x_{n}}_{k}\left(|x_{n}|t\right),|x_{n}|\geq1\right\}\]
\end{itemize}
son uniformemente convergentes.
\end{Lema}

$S_{l}^{x}\left(t\right)$ es el n\'umero total de servicios
completados de la clase $l$, si la clase $l$ est\'a dando $t$
unidades de tiempo de servicio. Sea $T_{l}^{x}\left(x\right)$ el
monto acumulado del tiempo de servicio que el servidor
$s\left(l\right)$ gasta en los usuarios de la clase $l$ al tiempo
$t$. Entonces $S_{l}^{x}\left(T_{l}^{x}\left(t\right)\right)$ es
el n\'umero total de servicios completados para la clase $l$ al
tiempo $t$. Una fracci\'on de estos usuarios,
$\Phi_{l}^{x}\left(S_{l}^{x}\left(T_{l}^{x}\left(t\right)\right)\right)$,
se convierte en usuarios de la clase $k$.\\

Entonces, dado lo anterior, se tiene la siguiente representaci\'on
para el proceso de la longitud de la cola:\\

\begin{equation}
Q_{k}^{x}\left(t\right)=_{k}^{x}\left(0\right)+E_{k}^{x}\left(t\right)+\sum_{l=1}^{K}\Phi_{k}^{l}\left(S_{l}^{x}\left(T_{l}^{x}\left(t\right)\right)\right)-S_{k}^{x}\left(T_{k}^{x}\left(t\right)\right)
\end{equation}
para $k=1,\ldots,K$. Para $i=1,\ldots,d$, sea
\[I_{i}^{x}\left(t\right)=t-\sum_{j\in C_{i}}T_{k}^{x}\left(t\right).\]

Entonces $I_{i}^{x}\left(t\right)$ es el monto acumulado del
tiempo que el servidor $i$ ha estado desocupado al tiempo $t$. Se
est\'a asumiendo que las disciplinas satisfacen la ley de
conservaci\'on del trabajo, es decir, el servidor $i$ est\'a en
pausa solamente cuando no hay usuarios en la estaci\'on $i$.
Entonces, se tiene que

\begin{equation}
\int_{0}^{\infty}\left(\sum_{k\in
C_{i}}Q_{k}^{x}\left(t\right)\right)dI_{i}^{x}\left(t\right)=0,
\end{equation}
para $i=1,\ldots,d$.\\

Hacer
\[T^{x}\left(t\right)=\left(T_{1}^{x}\left(t\right),\ldots,T_{K}^{x}\left(t\right)\right)^{'},\]
\[I^{x}\left(t\right)=\left(I_{1}^{x}\left(t\right),\ldots,I_{K}^{x}\left(t\right)\right)^{'}\]
y
\[S^{x}\left(T^{x}\left(t\right)\right)=\left(S_{1}^{x}\left(T_{1}^{x}\left(t\right)\right),\ldots,S_{K}^{x}\left(T_{K}^{x}\left(t\right)\right)\right)^{'}.\]

Para una disciplina que cumple con la ley de conservaci\'on del
trabajo, en forma vectorial, se tiene el siguiente conjunto de
ecuaciones

\begin{equation}\label{Eq.MF.1.3}
Q^{x}\left(t\right)=Q^{x}\left(0\right)+E^{x}\left(t\right)+\sum_{l=1}^{K}\Phi^{l}\left(S_{l}^{x}\left(T_{l}^{x}\left(t\right)\right)\right)-S^{x}\left(T^{x}\left(t\right)\right),\\
\end{equation}

\begin{equation}\label{Eq.MF.2.3}
Q^{x}\left(t\right)\geq0,\\
\end{equation}

\begin{equation}\label{Eq.MF.3.3}
T^{x}\left(0\right)=0,\textrm{ y }\overline{T}^{x}\left(t\right)\textrm{ es no decreciente},\\
\end{equation}

\begin{equation}\label{Eq.MF.4.3}
I^{x}\left(t\right)=et-CT^{x}\left(t\right)\textrm{ es no
decreciente}\\
\end{equation}

\begin{equation}\label{Eq.MF.5.3}
\int_{0}^{\infty}\left(CQ^{x}\left(t\right)\right)dI_{i}^{x}\left(t\right)=0,\\
\end{equation}

\begin{equation}\label{Eq.MF.6.3}
\textrm{Condiciones adicionales en
}\left(\overline{Q}^{x}\left(\cdot\right),\overline{T}^{x}\left(\cdot\right)\right)\textrm{
espec\'ificas de la disciplina de la cola,}
\end{equation}

donde $e$ es un vector de unos de dimensi\'on $d$, $C$ es la
matriz definida por
\[C_{ik}=\left\{\begin{array}{cc}
1,& S\left(k\right)=i,\\
0,& \textrm{ en otro caso}.\\
\end{array}\right.
\]
Es necesario enunciar el siguiente Teorema que se utilizar\'a para
el Teorema \ref{Tma.4.2.Dai}:
\begin{Teo}[Teorema 4.1, Dai \cite{Dai}]
Considere una disciplina que cumpla la ley de conservaci\'on del
trabajo, para casi todas las trayectorias muestrales $\omega$ y
cualquier sucesi\'on de estados iniciales
$\left\{x_{n}\right\}\subset \mathbf{X}$, con
$|x_{n}|\rightarrow\infty$, existe una subsucesi\'on
$\left\{x_{n_{j}}\right\}$ con $|x_{n_{j}}|\rightarrow\infty$ tal
que
\begin{equation}\label{Eq.4.15}
\frac{1}{|x_{n_{j}}|}\left(Q^{x_{n_{j}}}\left(0\right),U^{x_{n_{j}}}\left(0\right),V^{x_{n_{j}}}\left(0\right)\right)\rightarrow\left(\overline{Q}\left(0\right),\overline{U},\overline{V}\right),
\end{equation}

\begin{equation}\label{Eq.4.16}
\frac{1}{|x_{n_{j}}|}\left(Q^{x_{n_{j}}}\left(|x_{n_{j}}|t\right),T^{x_{n_{j}}}\left(|x_{n_{j}}|t\right)\right)\rightarrow\left(\overline{Q}\left(t\right),\overline{T}\left(t\right)\right)\textrm{
u.o.c.}
\end{equation}

Adem\'as,
$\left(\overline{Q}\left(t\right),\overline{T}\left(t\right)\right)$
satisface las siguientes ecuaciones:
\begin{equation}\label{Eq.MF.1.3a}
\overline{Q}\left(t\right)=Q\left(0\right)+\left(\alpha
t-\overline{U}\right)^{+}-\left(I-P\right)^{'}M^{-1}\left(\overline{T}\left(t\right)-\overline{V}\right)^{+},
\end{equation}

\begin{equation}\label{Eq.MF.2.3a}
\overline{Q}\left(t\right)\geq0,\\
\end{equation}

\begin{equation}\label{Eq.MF.3.3a}
\overline{T}\left(t\right)\textrm{ es no decreciente y comienza en cero},\\
\end{equation}

\begin{equation}\label{Eq.MF.4.3a}
\overline{I}\left(t\right)=et-C\overline{T}\left(t\right)\textrm{
es no decreciente,}\\
\end{equation}

\begin{equation}\label{Eq.MF.5.3a}
\int_{0}^{\infty}\left(C\overline{Q}\left(t\right)\right)d\overline{I}\left(t\right)=0,\\
\end{equation}

\begin{equation}\label{Eq.MF.6.3a}
\textrm{Condiciones adicionales en
}\left(\overline{Q}\left(\cdot\right),\overline{T}\left(\cdot\right)\right)\textrm{
especficas de la disciplina de la cola,}
\end{equation}
\end{Teo}

\begin{Def}[Definici\'on 4.1, , Dai \cite{Dai}]
Sea una disciplina de servicio espec\'ifica. Cualquier l\'imite
$\left(\overline{Q}\left(\cdot\right),\overline{T}\left(\cdot\right)\right)$
en \ref{Eq.4.16} es un {\em flujo l\'imite} de la disciplina.
Cualquier soluci\'on (\ref{Eq.MF.1.3a})-(\ref{Eq.MF.6.3a}) es
llamado flujo soluci\'on de la disciplina. Se dice que el modelo de flujo l\'imite, modelo de flujo, de la disciplina de la cola es estable si existe una constante
$\delta>0$ que depende de $\mu,\alpha$ y $P$ solamente, tal que
cualquier flujo l\'imite con
$|\overline{Q}\left(0\right)|+|\overline{U}|+|\overline{V}|=1$, se
tiene que $\overline{Q}\left(\cdot+\delta\right)\equiv0$.
\end{Def}

\begin{Teo}[Teorema 4.2, Dai\cite{Dai}]\label{Tma.4.2.Dai}
Sea una disciplina fija para la cola, suponga que se cumplen las
condiciones (1.2)-(1.5). Si el modelo de flujo l\'imite de la
disciplina de la cola es estable, entonces la cadena de Markov $X$
que describe la din\'amica de la red bajo la disciplina es Harris
recurrente positiva.
\end{Teo}

Ahora se procede a escalar el espacio y el tiempo para reducir la
aparente fluctuaci\'on del modelo. Consid\'erese el proceso
\begin{equation}\label{Eq.3.7}
\overline{Q}^{x}\left(t\right)=\frac{1}{|x|}Q^{x}\left(|x|t\right)
\end{equation}
A este proceso se le conoce como el fluido escalado, y cualquier l\'imite $\overline{Q}^{x}\left(t\right)$ es llamado flujo l\'imite del proceso de longitud de la cola. Haciendo $|q|\rightarrow\infty$ mientras se mantiene el resto de las componentes fijas, cualquier punto l\'imite del proceso de longitud de la cola normalizado $\overline{Q}^{x}$ es soluci\'on del siguiente modelo de flujo.

Al conjunto de ecuaciones dadas en \ref{Eq.3.8}-\ref{Eq.3.13} se
le llama {\em Modelo de flujo} y al conjunto de todas las
soluciones del modelo de flujo
$\left(\overline{Q}\left(\cdot\right),\overline{T}
\left(\cdot\right)\right)$ se le denotar\'a por $\mathcal{Q}$.

Si se hace $|x|\rightarrow\infty$ sin restringir ninguna de las
componentes, tambi\'en se obtienen un modelo de flujo, pero en
este caso el residual de los procesos de arribo y servicio
introducen un retraso:

\begin{Def}[Definici\'on 3.3, Dai y Meyn \cite{DaiSean}]
El modelo de flujo es estable si existe un tiempo fijo $t_{0}$ tal
que $\overline{Q}\left(t\right)=0$, con $t\geq t_{0}$, para
cualquier $\overline{Q}\left(\cdot\right)\in\mathcal{Q}$ que
cumple con $|\overline{Q}\left(0\right)|=1$.
\end{Def}

El siguiente resultado se encuentra en Chen \cite{Chen}.
\begin{Lemma}[Lema 3.1, Dai y Meyn \cite{DaiSean}]
Si el modelo de flujo definido por \ref{Eq.3.8}-\ref{Eq.3.13} es
estable, entonces el modelo de flujo retrasado es tambi\'en
estable, es decir, existe $t_{0}>0$ tal que
$\overline{Q}\left(t\right)=0$ para cualquier $t\geq t_{0}$, para
cualquier soluci\'on del modelo de flujo retrasado cuya
condici\'on inicial $\overline{x}$ satisface que
$|\overline{x}|=|\overline{Q}\left(0\right)|+|\overline{A}\left(0\right)|+|\overline{B}\left(0\right)|\leq1$.
\end{Lemma}


Propiedades importantes para el modelo de flujo retrasado:

\begin{Prop}
 Sea $\left(\overline{Q},\overline{T},\overline{T}^{0}\right)$ un flujo l\'imite de \ref{Eq.4.4} y suponga que cuando $x\rightarrow\infty$ a lo largo de
una subsucesi\'on
\[\left(\frac{1}{|x|}Q_{k}^{x}\left(0\right),\frac{1}{|x|}A_{k}^{x}\left(0\right),\frac{1}{|x|}B_{k}^{x}\left(0\right),\frac{1}{|x|}B_{k}^{x,0}\left(0\right)\right)\rightarrow\left(\overline{Q}_{k}\left(0\right),0,0,0\right)\]
para $k=1,\ldots,K$. EL flujo l\'imite tiene las siguientes
propiedades, donde las propiedades de la derivada se cumplen donde
la derivada exista:
\begin{itemize}
 \item[i)] Los vectores de tiempo ocupado $\overline{T}\left(t\right)$ y $\overline{T}^{0}\left(t\right)$ son crecientes y continuas con
$\overline{T}\left(0\right)=\overline{T}^{0}\left(0\right)=0$.
\item[ii)] Para todo $t\geq0$
\[\sum_{k=1}^{K}\left[\overline{T}_{k}\left(t\right)+\overline{T}_{k}^{0}\left(t\right)\right]=t\]
\item[iii)] Para todo $1\leq k\leq K$
\[\overline{Q}_{k}\left(t\right)=\overline{Q}_{k}\left(0\right)+\alpha_{k}t-\mu_{k}\overline{T}_{k}\left(t\right)\]
\item[iv)]  Para todo $1\leq k\leq K$
\[\dot{{\overline{T}}}_{k}\left(t\right)=\beta_{k}\] para $\overline{Q}_{k}\left(t\right)=0$.
\item[v)] Para todo $k,j$
\[\mu_{k}^{0}\overline{T}_{k}^{0}\left(t\right)=\mu_{j}^{0}\overline{T}_{j}^{0}\left(t\right)\]
\item[vi)]  Para todo $1\leq k\leq K$
\[\mu_{k}\dot{{\overline{T}}}_{k}\left(t\right)=l_{k}\mu_{k}^{0}\dot{{\overline{T}}}_{k}^{0}\left(t\right)\] para $\overline{Q}_{k}\left(t\right)>0$.
\end{itemize}
\end{Prop}

\begin{Lema}[Lema 3.1 \cite{Chen}]\label{Lema3.1}
Si el modelo de flujo es estable, definido por las ecuaciones
(3.8)-(3.13), entonces el modelo de flujo retrasado tambin es
estable.
\end{Lema}

\begin{Teo}[Teorema 5.2 \cite{Chen}]\label{Tma.5.2}
Si el modelo de flujo lineal correspondiente a la red de cola es
estable, entonces la red de colas es estable.
\end{Teo}

\begin{Teo}[Teorema 5.1 \cite{Chen}]\label{Tma.5.1.Chen}
La red de colas es estable si existe una constante $t_{0}$ que
depende de $\left(\alpha,\mu,T,U\right)$ y $V$ que satisfagan las
ecuaciones (5.1)-(5.5), $Z\left(t\right)=0$, para toda $t\geq
t_{0}$.
\end{Teo}



\begin{Lema}[Lema 5.2 \cite{Gut}]\label{Lema.5.2.Gut}
Sea $\left\{\xi\left(k\right):k\in\ent\right\}$ sucesin de
variables aleatorias i.i.d. con valores en
$\left(0,\infty\right)$, y sea $E\left(t\right)$ el proceso de
conteo
\[E\left(t\right)=max\left\{n\geq1:\xi\left(1\right)+\cdots+\xi\left(n-1\right)\leq t\right\}.\]
Si $E\left[\xi\left(1\right)\right]<\infty$, entonces para
cualquier entero $r\geq1$
\begin{equation}
lim_{t\rightarrow\infty}\esp\left[\left(\frac{E\left(t\right)}{t}\right)^{r}\right]=\left(\frac{1}{E\left[\xi_{1}\right]}\right)^{r}
\end{equation}
de aqu, bajo estas condiciones
\begin{itemize}
\item[a)] Para cualquier $t>0$,
$sup_{t\geq\delta}\esp\left[\left(\frac{E\left(t\right)}{t}\right)^{r}\right]$

\item[b)] Las variables aleatorias
$\left\{\left(\frac{E\left(t\right)}{t}\right)^{r}:t\geq1\right\}$
son uniformemente integrables.
\end{itemize}
\end{Lema}

\begin{Teo}[Teorema 5.1: Ley Fuerte para Procesos de Conteo
\cite{Gut}]\label{Tma.5.1.Gut} Sea
$0<\mu<\esp\left(X_{1}\right]\leq\infty$. entonces

\begin{itemize}
\item[a)] $\frac{N\left(t\right)}{t}\rightarrow\frac{1}{\mu}$
a.s., cuando $t\rightarrow\infty$.


\item[b)]$\esp\left[\frac{N\left(t\right)}{t}\right]^{r}\rightarrow\frac{1}{\mu^{r}}$,
cuando $t\rightarrow\infty$ para todo $r>0$..
\end{itemize}
\end{Teo}


\begin{Prop}[Proposicin 5.1 \cite{DaiSean}]\label{Prop.5.1}
Suponga que los supuestos (A1) y (A2) se cumplen, adems suponga
que el modelo de flujo es estable. Entonces existe $t_{0}>0$ tal
que
\begin{equation}\label{Eq.Prop.5.1}
lim_{|x|\rightarrow\infty}\frac{1}{|x|^{p+1}}\esp_{x}\left[|X\left(t_{0}|x|\right)|^{p+1}\right]=0.
\end{equation}

\end{Prop}


\begin{Prop}[Proposici\'on 5.3 \cite{DaiSean}]
Sea $X$ proceso de estados para la red de colas, y suponga que se
cumplen los supuestos (A1) y (A2), entonces para alguna constante
positiva $C_{p+1}<\infty$, $\delta>0$ y un conjunto compacto
$C\subset X$.

\begin{equation}\label{Eq.5.4}
\esp_{x}\left[\int_{0}^{\tau_{C}\left(\delta\right)}\left(1+|X\left(t\right)|^{p}\right)dt\right]\leq
C_{p+1}\left(1+|x|^{p+1}\right)
\end{equation}
\end{Prop}

\begin{Prop}[Proposici\'on 5.4 \cite{DaiSean}]
Sea $X$ un proceso de Markov Borel Derecho en $X$, sea
$f:X\leftarrow\rea_{+}$ y defina para alguna $\delta>0$, y un
conjunto cerrado $C\subset X$
\[V\left(x\right):=\esp_{x}\left[\int_{0}^{\tau_{C}\left(\delta\right)}f\left(X\left(t\right)\right)dt\right]\]
para $x\in X$. Si $V$ es finito en todas partes y uniformemente
acotada en $C$, entonces existe $k<\infty$ tal que
\begin{equation}\label{Eq.5.11}
\frac{1}{t}\esp_{x}\left[V\left(x\right)\right]+\frac{1}{t}\int_{0}^{t}\esp_{x}\left[f\left(X\left(s\right)\right)ds\right]\leq\frac{1}{t}V\left(x\right)+k,
\end{equation}
para $x\in X$ y $t>0$.
\end{Prop}


\begin{Teo}[Teorema 5.5 \cite{DaiSean}]
Suponga que se cumplen (A1) y (A2), adems suponga que el modelo
de flujo es estable. Entonces existe una constante $k_{p}<\infty$
tal que
\begin{equation}\label{Eq.5.13}
\frac{1}{t}\int_{0}^{t}\esp_{x}\left[|Q\left(s\right)|^{p}\right]ds\leq
k_{p}\left\{\frac{1}{t}|x|^{p+1}+1\right\}
\end{equation}
para $t\geq0$, $x\in X$. En particular para cada condici\'on inicial
\begin{equation}\label{Eq.5.14}
Limsup_{t\rightarrow\infty}\frac{1}{t}\int_{0}^{t}\esp_{x}\left[|Q\left(s\right)|^{p}\right]ds\leq
k_{p}
\end{equation}
\end{Teo}

\begin{Teo}[Teorema 6.2\cite{DaiSean}]\label{Tma.6.2}
Suponga que se cumplen los supuestos (A1)-(A3) y que el modelo de
flujo es estable, entonces se tiene que
\[\parallel P^{t}\left(c,\cdot\right)-\pi\left(\cdot\right)\parallel_{f_{p}}\rightarrow0\]
para $t\rightarrow\infty$ y $x\in X$. En particular para cada
condicin inicial
\[lim_{t\rightarrow\infty}\esp_{x}\left[\left|Q_{t}\right|^{p}\right]=\esp_{\pi}\left[\left|Q_{0}\right|^{p}\right]<\infty\]
\end{Teo}


\begin{Teo}[Teorema 6.3\cite{DaiSean}]\label{Tma.6.3}
Suponga que se cumplen los supuestos (A1)-(A3) y que el modelo de
flujo es estable, entonces con
$f\left(x\right)=f_{1}\left(x\right)$, se tiene que
\[lim_{t\rightarrow\infty}t^{(p-1)\left|P^{t}\left(c,\cdot\right)-\pi\left(\cdot\right)\right|_{f}=0},\]
para $x\in X$. En particular, para cada condicin inicial
\[lim_{t\rightarrow\infty}t^{(p-1)\left|\esp_{x}\left[Q_{t}\right]-\esp_{\pi}\left[Q_{0}\right]\right|=0}.\]
\end{Teo}


\begin{Prop}[Proposici\'on 5.1, Dai y Meyn \cite{DaiSean}]\label{Prop.5.1.DaiSean}
Suponga que los supuestos A1) y A2) son ciertos y que el modelo de flujo es estable. Entonces existe $t_{0}>0$ tal que
\begin{equation}
lim_{|x|\rightarrow\infty}\frac{1}{|x|^{p+1}}\esp_{x}\left[|X\left(t_{0}|x|\right)|^{p+1}\right]=0
\end{equation}
\end{Prop}

\begin{Lemma}[Lema 5.2, Dai y Meyn \cite{DaiSean}]\label{Lema.5.2.DaiSean}
 Sea $\left\{\zeta\left(k\right):k\in \mathbb{z}\right\}$ una sucesi\'on independiente e id\'enticamente distribuida que toma valores en $\left(0,\infty\right)$,
y sea
$E\left(t\right)=max\left(n\geq1:\zeta\left(1\right)+\cdots+\zeta\left(n-1\right)\leq
t\right)$. Si $\esp\left[\zeta\left(1\right)\right]<\infty$,
entonces para cualquier entero $r\geq1$
\begin{equation}
 lim_{t\rightarrow\infty}\esp\left[\left(\frac{E\left(t\right)}{t}\right)^{r}\right]=\left(\frac{1}{\esp\left[\zeta_{1}\right]}\right)^{r}.
\end{equation}
Luego, bajo estas condiciones:
\begin{itemize}
 \item[a)] para cualquier $\delta>0$, $\sup_{t\geq\delta}\esp\left[\left(\frac{E\left(t\right)}{t}\right)^{r}\right]<\infty$
\item[b)] las variables aleatorias
$\left\{\left(\frac{E\left(t\right)}{t}\right)^{r}:t\geq1\right\}$
son uniformemente integrables.
\end{itemize}
\end{Lemma}

\begin{Teo}[Teorema 5.5, Dai y Meyn \cite{DaiSean}]\label{Tma.5.5.DaiSean}
Suponga que los supuestos A1) y A2) se cumplen y que el modelo de
flujo es estable. Entonces existe una constante $\kappa_{p}$ tal
que
\begin{equation}
\frac{1}{t}\int_{0}^{t}\esp_{x}\left[|Q\left(s\right)|^{p}\right]ds\leq\kappa_{p}\left\{\frac{1}{t}|x|^{p+1}+1\right\}
\end{equation}
para $t>0$ y $x\in X$. En particular, para cada condici\'on
inicial
\begin{eqnarray*}
\limsup_{t\rightarrow\infty}\frac{1}{t}\int_{0}^{t}\esp_{x}\left[|Q\left(s\right)|^{p}\right]ds\leq\kappa_{p}.
\end{eqnarray*}
\end{Teo}

\begin{Teo}[Teorema 6.2, Dai y Meyn \cite{DaiSean}]\label{Tma.6.2.DaiSean}
Suponga que se cumplen los supuestos A1), A2) y A3) y que el
modelo de flujo es estable. Entonces se tiene que
\begin{equation}
\left\|P^{t}\left(x,\cdot\right)-\pi\left(\cdot\right)\right\|_{f_{p}}\textrm{,
}t\rightarrow\infty,x\in X.
\end{equation}
En particular para cada condici\'on inicial
\begin{eqnarray*}
\lim_{t\rightarrow\infty}\esp_{x}\left[|Q\left(t\right)|^{p}\right]=\esp_{\pi}\left[|Q\left(0\right)|^{p}\right]\leq\kappa_{r}
\end{eqnarray*}
\end{Teo}
\begin{Teo}[Teorema 6.3, Dai y Meyn \cite{DaiSean}]\label{Tma.6.3.DaiSean}
Suponga que se cumplen los supuestos A1), A2) y A3) y que el
modelo de flujo es estable. Entonces con
$f\left(x\right)=f_{1}\left(x\right)$ se tiene
\begin{equation}
\lim_{t\rightarrow\infty}t^{p-1}\left\|P^{t}\left(x,\cdot\right)-\pi\left(\cdot\right)\right\|_{f}=0.
\end{equation}
En particular para cada condici\'on inicial
\begin{eqnarray*}
\lim_{t\rightarrow\infty}t^{p-1}|\esp_{x}\left[Q\left(t\right)\right]-\esp_{\pi}\left[Q\left(0\right)\right]|=0.
\end{eqnarray*}
\end{Teo}

\begin{Teo}[Teorema 6.4, Dai y Meyn \cite{DaiSean}]\label{Tma.6.4.DaiSean}
Suponga que se cumplen los supuestos A1), A2) y A3) y que el
modelo de flujo es estable. Sea $\nu$ cualquier distribuci\'on de
probabilidad en $\left(X,\mathcal{B}_{X}\right)$, y $\pi$ la
distribuci\'on estacionaria de $X$.
\begin{itemize}
\item[i)] Para cualquier $f:X\leftarrow\rea_{+}$
\begin{equation}
\lim_{t\rightarrow\infty}\frac{1}{t}\int_{o}^{t}f\left(X\left(s\right)\right)ds=\pi\left(f\right):=\int
f\left(x\right)\pi\left(dx\right)
\end{equation}
$\prob$-c.s.

\item[ii)] Para cualquier $f:X\leftarrow\rea_{+}$ con
$\pi\left(|f|\right)<\infty$, la ecuaci\'on anterior se cumple.
\end{itemize}
\end{Teo}

\begin{Teo}[Teorema 2.2, Down \cite{Down}]\label{Tma2.2.Down}
Suponga que el fluido modelo es inestable en el sentido de que
para alguna $\epsilon_{0},c_{0}\geq0$,
\begin{equation}\label{Eq.Inestability}
|Q\left(T\right)|\geq\epsilon_{0}T-c_{0}\textrm{,   }T\geq0,
\end{equation}
para cualquier condici\'on inicial $Q\left(0\right)$, con
$|Q\left(0\right)|=1$. Entonces para cualquier $0<q\leq1$, existe
$B<0$ tal que para cualquier $|x|\geq B$,
\begin{equation}
\prob_{x}\left\{\mathbb{X}\rightarrow\infty\right\}\geq q.
\end{equation}
\end{Teo}



Es necesario hacer los siguientes supuestos sobre el
comportamiento del sistema de visitas c\'iclicas:
\begin{itemize}
\item Los tiempos de interarribo a la $k$-\'esima cola, son de la
forma $\left\{\xi_{k}\left(n\right)\right\}_{n\geq1}$, con la
propiedad de que son independientes e id{\'e}nticamente
distribuidos,
\item Los tiempos de servicio
$\left\{\eta_{k}\left(n\right)\right\}_{n\geq1}$ tienen la
propiedad de ser independientes e id{\'e}nticamente distribuidos,
\item Se define la tasa de arribo a la $k$-{\'e}sima cola como
$\lambda_{k}=1/\esp\left[\xi_{k}\left(1\right)\right]$,
\item la tasa de servicio para la $k$-{\'e}sima cola se define
como $\mu_{k}=1/\esp\left[\eta_{k}\left(1\right)\right]$,
\item tambi{\'e}n se define $\rho_{k}:=\lambda_{k}/\mu_{k}$, la
intensidad de tr\'afico del sistema o carga de la red, donde es
necesario que $\rho<1$ para cuestiones de estabilidad.
\end{itemize}



%_________________________________________________________________________
\subsection{Procesos de Estados Markoviano para el Sistema}
%_________________________________________________________________________

%_________________________________________________________________________
\subsection{Procesos Fuerte de Markov}
%_________________________________________________________________________
En Dai \cite{Dai} se muestra que para una amplia serie de disciplinas
de servicio el proceso $X$ es un Proceso Fuerte de
Markov, y por tanto se puede asumir que


Para establecer que $X=\left\{X\left(t\right),t\geq0\right\}$ es
un Proceso Fuerte de Markov, se siguen las secciones 2.3 y 2.4 de Kaspi and Mandelbaum \cite{KaspiMandelbaum}. \\

%______________________________________________________________
\subsubsection{Construcci\'on de un Proceso Determinista por partes, Davis
\cite{Davis}}.
%______________________________________________________________

%_________________________________________________________________________
\subsection{Procesos Harris Recurrentes Positivos}
%_________________________________________________________________________
Sea el proceso de Markov $X=\left\{X\left(t\right),t\geq0\right\}$
que describe la din\'amica de la red de colas. En lo que respecta
al supuesto (A3), en Dai y Meyn \cite{DaiSean} y Meyn y Down
\cite{MeynDown} hacen ver que este se puede sustituir por

\begin{itemize}
\item[A3')] Para el Proceso de Markov $X$, cada subconjunto
compacto de $X$ es un conjunto peque\~no.
\end{itemize}

Este supuesto es importante pues es un requisito para deducir la ergodicidad de la red.

%_________________________________________________________________________
\subsection{Construcci\'on de un Modelo de Flujo L\'imite}
%_________________________________________________________________________

Consideremos un caso m\'as simple para poner en contexto lo
anterior: para un sistema de visitas c\'iclicas se tiene que el
estado al tiempo $t$ es
\begin{equation}
X\left(t\right)=\left(Q\left(t\right),U\left(t\right),V\left(t\right)\right),
\end{equation}

donde $Q\left(t\right)$ es el n\'umero de usuarios formados en
cada estaci\'on. $U\left(t\right)$ es el tiempo restante antes de
que la siguiente clase $k$ de usuarios lleguen desde fuera del
sistema, $V\left(t\right)$ es el tiempo restante de servicio para
la clase $k$ de usuarios que est\'an siendo atendidos. Tanto
$U\left(t\right)$ como $V\left(t\right)$ se puede asumir que son
continuas por la derecha.

Sea
$x=\left(Q\left(0\right),U\left(0\right),V\left(0\right)\right)=\left(q,a,b\right)$,
el estado inicial de la red bajo una disciplina espec\'ifica para
la cola. Para $l\in\mathcal{E}$, donde $\mathcal{E}$ es el conjunto de clases de arribos externos, y $k=1,\ldots,K$ se define\\
\begin{eqnarray*}
E_{l}^{x}\left(t\right)&=&max\left\{r:U_{l}\left(0\right)+\xi_{l}\left(1\right)+\cdots+\xi_{l}\left(r-1\right)\leq
t\right\}\textrm{   }t\geq0,\\
S_{k}^{x}\left(t\right)&=&max\left\{r:V_{k}\left(0\right)+\eta_{k}\left(1\right)+\cdots+\eta_{k}\left(r-1\right)\leq
t\right\}\textrm{   }t\geq0.
\end{eqnarray*}

Para cada $k$ y cada $n$ se define

\begin{eqnarray*}\label{Eq.phi}
\Phi^{k}\left(n\right):=\sum_{i=1}^{n}\phi^{k}\left(i\right).
\end{eqnarray*}

donde $\phi^{k}\left(n\right)$ se define como el vector de ruta
para el $n$-\'esimo usuario de la clase $k$ que termina en la
estaci\'on $s\left(k\right)$, la $s$-\'eima componente de
$\phi^{k}\left(n\right)$ es uno si estos usuarios se convierten en
usuarios de la clase $l$ y cero en otro caso, por lo tanto
$\phi^{k}\left(n\right)$ es un vector {\em Bernoulli} de
dimensi\'on $K$ con par\'ametro $P_{k}^{'}$, donde $P_{k}$ denota
el $k$-\'esimo rengl\'on de $P=\left(P_{kl}\right)$.

Se asume que cada para cada $k$ la sucesi\'on $\phi^{k}\left(n\right)=\left\{\phi^{k}\left(n\right),n\geq1\right\}$
es independiente e id\'enticamente distribuida y que las
$\phi^{1}\left(n\right),\ldots,\phi^{K}\left(n\right)$ son
mutuamente independientes, adem\'as de independientes de los
procesos de arribo y de servicio.\\

\begin{Lema}[Lema 4.2, Dai\cite{Dai}]\label{Lema4.2}
Sea $\left\{x_{n}\right\}\subset \mathbf{X}$ con
$|x_{n}|\rightarrow\infty$, conforme $n\rightarrow\infty$. Suponga
que
\[lim_{n\rightarrow\infty}\frac{1}{|x_{n}|}U\left(0\right)=\overline{U}\]
y
\[lim_{n\rightarrow\infty}\frac{1}{|x_{n}|}V\left(0\right)=\overline{V}.\]

Entonces, conforme $n\rightarrow\infty$, casi seguramente

\begin{equation}\label{E1.4.2}
\frac{1}{|x_{n}|}\Phi^{k}\left(\left[|x_{n}|t\right]\right)\rightarrow
P_{k}^{'}t\textrm{, u.o.c.,}
\end{equation}

\begin{equation}\label{E1.4.3}
\frac{1}{|x_{n}|}E^{x_{n}}_{k}\left(|x_{n}|t\right)\rightarrow
\alpha_{k}\left(t-\overline{U}_{k}\right)^{+}\textrm{, u.o.c.,}
\end{equation}

\begin{equation}\label{E1.4.4}
\frac{1}{|x_{n}|}S^{x_{n}}_{k}\left(|x_{n}|t\right)\rightarrow
\mu_{k}\left(t-\overline{V}_{k}\right)^{+}\textrm{, u.o.c.,}
\end{equation}

donde $\left[t\right]$ es la parte entera de $t$ y
$\mu_{k}=1/m_{k}=1/\esp\left[\eta_{k}\left(1\right)\right]$.
\end{Lema}

\begin{Lema}[Lema 4.3, Dai\cite{Dai}]\label{Lema.4.3}
Sea $\left\{x_{n}\right\}\subset \mathbf{X}$ con
$|x_{n}|\rightarrow\infty$, conforme $n\rightarrow\infty$. Suponga
que
\[lim_{n\rightarrow\infty}\frac{1}{|x_{n}|}U\left(0\right)=\overline{U}_{k}\]
y
\[lim_{n\rightarrow\infty}\frac{1}{|x_{n}|}V\left(0\right)=\overline{V}_{k}.\]
\begin{itemize}
\item[a)] Conforme $n\rightarrow\infty$ casi seguramente,
\[lim_{n\rightarrow\infty}\frac{1}{|x_{n}|}U^{x_{n}}_{k}\left(|x_{n}|t\right)=\left(\overline{U}_{k}-t\right)^{+}\textrm{, u.o.c.}\]
y
\[lim_{n\rightarrow\infty}\frac{1}{|x_{n}|}V^{x_{n}}_{k}\left(|x_{n}|t\right)=\left(\overline{V}_{k}-t\right)^{+}.\]

\item[b)] Para cada $t\geq0$ fijo,
\[\left\{\frac{1}{|x_{n}|}U^{x_{n}}_{k}\left(|x_{n}|t\right),|x_{n}|\geq1\right\}\]
y
\[\left\{\frac{1}{|x_{n}|}V^{x_{n}}_{k}\left(|x_{n}|t\right),|x_{n}|\geq1\right\}\]
\end{itemize}
son uniformemente convergentes.
\end{Lema}

$S_{l}^{x}\left(t\right)$ es el n\'umero total de servicios
completados de la clase $l$, si la clase $l$ est\'a dando $t$
unidades de tiempo de servicio. Sea $T_{l}^{x}\left(x\right)$ el
monto acumulado del tiempo de servicio que el servidor
$s\left(l\right)$ gasta en los usuarios de la clase $l$ al tiempo
$t$. Entonces $S_{l}^{x}\left(T_{l}^{x}\left(t\right)\right)$ es
el n\'umero total de servicios completados para la clase $l$ al
tiempo $t$. Una fracci\'on de estos usuarios,
$\Phi_{l}^{x}\left(S_{l}^{x}\left(T_{l}^{x}\left(t\right)\right)\right)$,
se convierte en usuarios de la clase $k$.\\

Entonces, dado lo anterior, se tiene la siguiente representaci\'on
para el proceso de la longitud de la cola:\\

\begin{equation}
Q_{k}^{x}\left(t\right)=_{k}^{x}\left(0\right)+E_{k}^{x}\left(t\right)+\sum_{l=1}^{K}\Phi_{k}^{l}\left(S_{l}^{x}\left(T_{l}^{x}\left(t\right)\right)\right)-S_{k}^{x}\left(T_{k}^{x}\left(t\right)\right)
\end{equation}
para $k=1,\ldots,K$. Para $i=1,\ldots,d$, sea
\[I_{i}^{x}\left(t\right)=t-\sum_{j\in C_{i}}T_{k}^{x}\left(t\right).\]

Entonces $I_{i}^{x}\left(t\right)$ es el monto acumulado del
tiempo que el servidor $i$ ha estado desocupado al tiempo $t$. Se
est\'a asumiendo que las disciplinas satisfacen la ley de
conservaci\'on del trabajo, es decir, el servidor $i$ est\'a en
pausa solamente cuando no hay usuarios en la estaci\'on $i$.
Entonces, se tiene que

\begin{equation}
\int_{0}^{\infty}\left(\sum_{k\in
C_{i}}Q_{k}^{x}\left(t\right)\right)dI_{i}^{x}\left(t\right)=0,
\end{equation}
para $i=1,\ldots,d$.\\

Hacer
\[T^{x}\left(t\right)=\left(T_{1}^{x}\left(t\right),\ldots,T_{K}^{x}\left(t\right)\right)^{'},\]
\[I^{x}\left(t\right)=\left(I_{1}^{x}\left(t\right),\ldots,I_{K}^{x}\left(t\right)\right)^{'}\]
y
\[S^{x}\left(T^{x}\left(t\right)\right)=\left(S_{1}^{x}\left(T_{1}^{x}\left(t\right)\right),\ldots,S_{K}^{x}\left(T_{K}^{x}\left(t\right)\right)\right)^{'}.\]

Para una disciplina que cumple con la ley de conservaci\'on del
trabajo, en forma vectorial, se tiene el siguiente conjunto de
ecuaciones

\begin{equation}\label{Eq.MF.1.3}
Q^{x}\left(t\right)=Q^{x}\left(0\right)+E^{x}\left(t\right)+\sum_{l=1}^{K}\Phi^{l}\left(S_{l}^{x}\left(T_{l}^{x}\left(t\right)\right)\right)-S^{x}\left(T^{x}\left(t\right)\right),\\
\end{equation}

\begin{equation}\label{Eq.MF.2.3}
Q^{x}\left(t\right)\geq0,\\
\end{equation}

\begin{equation}\label{Eq.MF.3.3}
T^{x}\left(0\right)=0,\textrm{ y }\overline{T}^{x}\left(t\right)\textrm{ es no decreciente},\\
\end{equation}

\begin{equation}\label{Eq.MF.4.3}
I^{x}\left(t\right)=et-CT^{x}\left(t\right)\textrm{ es no
decreciente}\\
\end{equation}

\begin{equation}\label{Eq.MF.5.3}
\int_{0}^{\infty}\left(CQ^{x}\left(t\right)\right)dI_{i}^{x}\left(t\right)=0,\\
\end{equation}

\begin{equation}\label{Eq.MF.6.3}
\textrm{Condiciones adicionales en
}\left(\overline{Q}^{x}\left(\cdot\right),\overline{T}^{x}\left(\cdot\right)\right)\textrm{
espec\'ificas de la disciplina de la cola,}
\end{equation}

donde $e$ es un vector de unos de dimensi\'on $d$, $C$ es la
matriz definida por
\[C_{ik}=\left\{\begin{array}{cc}
1,& S\left(k\right)=i,\\
0,& \textrm{ en otro caso}.\\
\end{array}\right.
\]
Es necesario enunciar el siguiente Teorema que se utilizar\'a para
el Teorema \ref{Tma.4.2.Dai}:
\begin{Teo}[Teorema 4.1, Dai \cite{Dai}]
Considere una disciplina que cumpla la ley de conservaci\'on del
trabajo, para casi todas las trayectorias muestrales $\omega$ y
cualquier sucesi\'on de estados iniciales
$\left\{x_{n}\right\}\subset \mathbf{X}$, con
$|x_{n}|\rightarrow\infty$, existe una subsucesi\'on
$\left\{x_{n_{j}}\right\}$ con $|x_{n_{j}}|\rightarrow\infty$ tal
que
\begin{equation}\label{Eq.4.15}
\frac{1}{|x_{n_{j}}|}\left(Q^{x_{n_{j}}}\left(0\right),U^{x_{n_{j}}}\left(0\right),V^{x_{n_{j}}}\left(0\right)\right)\rightarrow\left(\overline{Q}\left(0\right),\overline{U},\overline{V}\right),
\end{equation}

\begin{equation}\label{Eq.4.16}
\frac{1}{|x_{n_{j}}|}\left(Q^{x_{n_{j}}}\left(|x_{n_{j}}|t\right),T^{x_{n_{j}}}\left(|x_{n_{j}}|t\right)\right)\rightarrow\left(\overline{Q}\left(t\right),\overline{T}\left(t\right)\right)\textrm{
u.o.c.}
\end{equation}

Adem\'as,
$\left(\overline{Q}\left(t\right),\overline{T}\left(t\right)\right)$
satisface las siguientes ecuaciones:
\begin{equation}\label{Eq.MF.1.3a}
\overline{Q}\left(t\right)=Q\left(0\right)+\left(\alpha
t-\overline{U}\right)^{+}-\left(I-P\right)^{'}M^{-1}\left(\overline{T}\left(t\right)-\overline{V}\right)^{+},
\end{equation}

\begin{equation}\label{Eq.MF.2.3a}
\overline{Q}\left(t\right)\geq0,\\
\end{equation}

\begin{equation}\label{Eq.MF.3.3a}
\overline{T}\left(t\right)\textrm{ es no decreciente y comienza en cero},\\
\end{equation}

\begin{equation}\label{Eq.MF.4.3a}
\overline{I}\left(t\right)=et-C\overline{T}\left(t\right)\textrm{
es no decreciente,}\\
\end{equation}

\begin{equation}\label{Eq.MF.5.3a}
\int_{0}^{\infty}\left(C\overline{Q}\left(t\right)\right)d\overline{I}\left(t\right)=0,\\
\end{equation}

\begin{equation}\label{Eq.MF.6.3a}
\textrm{Condiciones adicionales en
}\left(\overline{Q}\left(\cdot\right),\overline{T}\left(\cdot\right)\right)\textrm{
especficas de la disciplina de la cola,}
\end{equation}
\end{Teo}

\begin{Def}[Definici\'on 4.1, , Dai \cite{Dai}]
Sea una disciplina de servicio espec\'ifica. Cualquier l\'imite
$\left(\overline{Q}\left(\cdot\right),\overline{T}\left(\cdot\right)\right)$
en \ref{Eq.4.16} es un {\em flujo l\'imite} de la disciplina.
Cualquier soluci\'on (\ref{Eq.MF.1.3a})-(\ref{Eq.MF.6.3a}) es
llamado flujo soluci\'on de la disciplina. Se dice que el modelo de flujo l\'imite, modelo de flujo, de la disciplina de la cola es estable si existe una constante
$\delta>0$ que depende de $\mu,\alpha$ y $P$ solamente, tal que
cualquier flujo l\'imite con
$|\overline{Q}\left(0\right)|+|\overline{U}|+|\overline{V}|=1$, se
tiene que $\overline{Q}\left(\cdot+\delta\right)\equiv0$.
\end{Def}

\begin{Teo}[Teorema 4.2, Dai\cite{Dai}]\label{Tma.4.2.Dai}
Sea una disciplina fija para la cola, suponga que se cumplen las
condiciones (1.2)-(1.5). Si el modelo de flujo l\'imite de la
disciplina de la cola es estable, entonces la cadena de Markov $X$
que describe la din\'amica de la red bajo la disciplina es Harris
recurrente positiva.
\end{Teo}

Ahora se procede a escalar el espacio y el tiempo para reducir la
aparente fluctuaci\'on del modelo. Consid\'erese el proceso
\begin{equation}\label{Eq.3.7}
\overline{Q}^{x}\left(t\right)=\frac{1}{|x|}Q^{x}\left(|x|t\right)
\end{equation}
A este proceso se le conoce como el fluido escalado, y cualquier l\'imite $\overline{Q}^{x}\left(t\right)$ es llamado flujo l\'imite del proceso de longitud de la cola. Haciendo $|q|\rightarrow\infty$ mientras se mantiene el resto de las componentes fijas, cualquier punto l\'imite del proceso de longitud de la cola normalizado $\overline{Q}^{x}$ es soluci\'on del siguiente modelo de flujo.

\begin{Def}[Definici\'on 3.1, Dai y Meyn \cite{DaiSean}]
Un flujo l\'imite (retrasado) para una red bajo una disciplina de
servicio espec\'ifica se define como cualquier soluci\'on
 $\left(\overline{Q}\left(\cdot\right),\overline{T}\left(\cdot\right)\right)$ de las siguientes ecuaciones, donde
$\overline{Q}\left(t\right)=\left(\overline{Q}_{1}\left(t\right),\ldots,\overline{Q}_{K}\left(t\right)\right)^{'}$
y
$\overline{T}\left(t\right)=\left(\overline{T}_{1}\left(t\right),\ldots,\overline{T}_{K}\left(t\right)\right)^{'}$
\begin{equation}\label{Eq.3.8}
\overline{Q}_{k}\left(t\right)=\overline{Q}_{k}\left(0\right)+\alpha_{k}t-\mu_{k}\overline{T}_{k}\left(t\right)+\sum_{l=1}^{k}P_{lk}\mu_{l}\overline{T}_{l}\left(t\right)\\
\end{equation}
\begin{equation}\label{Eq.3.9}
\overline{Q}_{k}\left(t\right)\geq0\textrm{ para }k=1,2,\ldots,K,\\
\end{equation}
\begin{equation}\label{Eq.3.10}
\overline{T}_{k}\left(0\right)=0,\textrm{ y }\overline{T}_{k}\left(\cdot\right)\textrm{ es no decreciente},\\
\end{equation}
\begin{equation}\label{Eq.3.11}
\overline{I}_{i}\left(t\right)=t-\sum_{k\in C_{i}}\overline{T}_{k}\left(t\right)\textrm{ es no decreciente}\\
\end{equation}
\begin{equation}\label{Eq.3.12}
\overline{I}_{i}\left(\cdot\right)\textrm{ se incrementa al tiempo }t\textrm{ cuando }\sum_{k\in C_{i}}Q_{k}^{x}\left(t\right)dI_{i}^{x}\left(t\right)=0\\
\end{equation}
\begin{equation}\label{Eq.3.13}
\textrm{condiciones adicionales sobre
}\left(Q^{x}\left(\cdot\right),T^{x}\left(\cdot\right)\right)\textrm{
referentes a la disciplina de servicio}
\end{equation}
\end{Def}

Al conjunto de ecuaciones dadas en \ref{Eq.3.8}-\ref{Eq.3.13} se
le llama {\em Modelo de flujo} y al conjunto de todas las
soluciones del modelo de flujo
$\left(\overline{Q}\left(\cdot\right),\overline{T}
\left(\cdot\right)\right)$ se le denotar\'a por $\mathcal{Q}$.

Si se hace $|x|\rightarrow\infty$ sin restringir ninguna de las
componentes, tambi\'en se obtienen un modelo de flujo, pero en
este caso el residual de los procesos de arribo y servicio
introducen un retraso:

\begin{Def}[Definici\'on 3.2, Dai y Meyn \cite{DaiSean}]
El modelo de flujo retrasado de una disciplina de servicio en una
red con retraso
$\left(\overline{A}\left(0\right),\overline{B}\left(0\right)\right)\in\rea_{+}^{K+|A|}$
se define como el conjunto de ecuaciones dadas en
\ref{Eq.3.8}-\ref{Eq.3.13}, junto con la condici\'on:
\begin{equation}\label{CondAd.FluidModel}
\overline{Q}\left(t\right)=\overline{Q}\left(0\right)+\left(\alpha
t-\overline{A}\left(0\right)\right)^{+}-\left(I-P^{'}\right)M\left(\overline{T}\left(t\right)-\overline{B}\left(0\right)\right)^{+}
\end{equation}
\end{Def}

\begin{Def}[Definici\'on 3.3, Dai y Meyn \cite{DaiSean}]
El modelo de flujo es estable si existe un tiempo fijo $t_{0}$ tal
que $\overline{Q}\left(t\right)=0$, con $t\geq t_{0}$, para
cualquier $\overline{Q}\left(\cdot\right)\in\mathcal{Q}$ que
cumple con $|\overline{Q}\left(0\right)|=1$.
\end{Def}

El siguiente resultado se encuentra en Chen \cite{Chen}.
\begin{Lemma}[Lema 3.1, Dai y Meyn \cite{DaiSean}]
Si el modelo de flujo definido por \ref{Eq.3.8}-\ref{Eq.3.13} es
estable, entonces el modelo de flujo retrasado es tambi\'en
estable, es decir, existe $t_{0}>0$ tal que
$\overline{Q}\left(t\right)=0$ para cualquier $t\geq t_{0}$, para
cualquier soluci\'on del modelo de flujo retrasado cuya
condici\'on inicial $\overline{x}$ satisface que
$|\overline{x}|=|\overline{Q}\left(0\right)|+|\overline{A}\left(0\right)|+|\overline{B}\left(0\right)|\leq1$.
\end{Lemma}

%_________________________________________________________________________
\subsection{Modelo de Visitas C\'iclicas con un Servidor: Estabilidad}
%_________________________________________________________________________

%_________________________________________________________________________
\subsection{Teorema 2.1}
%_________________________________________________________________________



El resultado principal de Down \cite{Down} que relaciona la estabilidad del modelo de flujo con la estabilidad del sistema original

\begin{Teo}[Teorema 2.1, Down \cite{Down}]\label{Tma.2.1.Down}
Suponga que el modelo de flujo es estable, y que se cumplen los supuestos (A1) y (A2), entonces
\begin{itemize}
\item[i)] Para alguna constante $\kappa_{p}$, y para cada
condici\'on inicial $x\in X$
\begin{equation}\label{Estability.Eq1}
lim_{t\rightarrow\infty}\sup\frac{1}{t}\int_{0}^{t}\esp_{x}\left[|Q\left(s\right)|^{p}\right]ds\leq\kappa_{p},
\end{equation}
donde $p$ es el entero dado en (A2). Si adem\'as se cumple
la condici\'on (A3), entonces para cada condici\'on inicial:

\item[ii)] Los momentos transitorios convergen a su estado estacionario:
 \begin{equation}\label{Estability.Eq2}
lim_{t\rightarrow\infty}\esp_{x}\left[Q_{k}\left(t\right)^{r}\right]=\esp_{\pi}\left[Q_{k}\left(0\right)^{r}\right]\leq\kappa_{r},
\end{equation}
para $r=1,2,\ldots,p$ y $k=1,2,\ldots,K$. Donde $\pi$ es la
probabilidad invariante para $\mathbf{X}$.

\item[iii)]  El primer momento converge con raz\'on $t^{p-1}$:
\begin{equation}\label{Estability.Eq3}
lim_{t\rightarrow\infty}t^{p-1}|\esp_{x}\left[Q_{k}\left(t\right)\right]-\esp_{\pi}\left[Q\left(0\right)\right]=0.
\end{equation}

\item[iv)] La {\em Ley Fuerte de los grandes n\'umeros} se cumple:
\begin{equation}\label{Estability.Eq4}
lim_{t\rightarrow\infty}\frac{1}{t}\int_{0}^{t}Q_{k}^{r}\left(s\right)ds=\esp_{\pi}\left[Q_{k}\left(0\right)^{r}\right],\textrm{
}\prob_{x}\textrm{-c.s.}
\end{equation}
para $r=1,2,\ldots,p$ y $k=1,2,\ldots,K$.
\end{itemize}
\end{Teo}


\begin{Prop}[Proposici\'on 5.1, Dai y Meyn \cite{DaiSean}]\label{Prop.5.1.DaiSean}
Suponga que los supuestos A1) y A2) son ciertos y que el modelo de flujo es estable. Entonces existe $t_{0}>0$ tal que
\begin{equation}
lim_{|x|\rightarrow\infty}\frac{1}{|x|^{p+1}}\esp_{x}\left[|X\left(t_{0}|x|\right)|^{p+1}\right]=0
\end{equation}
\end{Prop}

\begin{Lemma}[Lema 5.2, Dai y Meyn \cite{DaiSean}]\label{Lema.5.2.DaiSean}
 Sea $\left\{\zeta\left(k\right):k\in \mathbb{z}\right\}$ una sucesi\'on independiente e id\'enticamente distribuida que toma valores en $\left(0,\infty\right)$,
y sea
$E\left(t\right)=max\left(n\geq1:\zeta\left(1\right)+\cdots+\zeta\left(n-1\right)\leq
t\right)$. Si $\esp\left[\zeta\left(1\right)\right]<\infty$,
entonces para cualquier entero $r\geq1$
\begin{equation}
 lim_{t\rightarrow\infty}\esp\left[\left(\frac{E\left(t\right)}{t}\right)^{r}\right]=\left(\frac{1}{\esp\left[\zeta_{1}\right]}\right)^{r}.
\end{equation}
Luego, bajo estas condiciones:
\begin{itemize}
 \item[a)] para cualquier $\delta>0$, $\sup_{t\geq\delta}\esp\left[\left(\frac{E\left(t\right)}{t}\right)^{r}\right]<\infty$
\item[b)] las variables aleatorias
$\left\{\left(\frac{E\left(t\right)}{t}\right)^{r}:t\geq1\right\}$
son uniformemente integrables.
\end{itemize}
\end{Lemma}

\begin{Teo}[Teorema 5.5, Dai y Meyn \cite{DaiSean}]\label{Tma.5.5.DaiSean}
Suponga que los supuestos A1) y A2) se cumplen y que el modelo de
flujo es estable. Entonces existe una constante $\kappa_{p}$ tal
que
\begin{equation}
\frac{1}{t}\int_{0}^{t}\esp_{x}\left[|Q\left(s\right)|^{p}\right]ds\leq\kappa_{p}\left\{\frac{1}{t}|x|^{p+1}+1\right\}
\end{equation}
para $t>0$ y $x\in X$. En particular, para cada condici\'on
inicial
\begin{eqnarray*}
\limsup_{t\rightarrow\infty}\frac{1}{t}\int_{0}^{t}\esp_{x}\left[|Q\left(s\right)|^{p}\right]ds\leq\kappa_{p}.
\end{eqnarray*}
\end{Teo}

\begin{Teo}[Teorema 6.2, Dai y Meyn \cite{DaiSean}]\label{Tma.6.2.DaiSean}
Suponga que se cumplen los supuestos A1), A2) y A3) y que el
modelo de flujo es estable. Entonces se tiene que
\begin{equation}
\left\|P^{t}\left(x,\cdot\right)-\pi\left(\cdot\right)\right\|_{f_{p}}\textrm{,
}t\rightarrow\infty,x\in X.
\end{equation}
En particular para cada condici\'on inicial
\begin{eqnarray*}
\lim_{t\rightarrow\infty}\esp_{x}\left[|Q\left(t\right)|^{p}\right]=\esp_{\pi}\left[|Q\left(0\right)|^{p}\right]\leq\kappa_{r}
\end{eqnarray*}
\end{Teo}
\begin{Teo}[Teorema 6.3, Dai y Meyn \cite{DaiSean}]\label{Tma.6.3.DaiSean}
Suponga que se cumplen los supuestos A1), A2) y A3) y que el
modelo de flujo es estable. Entonces con
$f\left(x\right)=f_{1}\left(x\right)$ se tiene
\begin{equation}
\lim_{t\rightarrow\infty}t^{p-1}\left\|P^{t}\left(x,\cdot\right)-\pi\left(\cdot\right)\right\|_{f}=0.
\end{equation}
En particular para cada condici\'on inicial
\begin{eqnarray*}
\lim_{t\rightarrow\infty}t^{p-1}|\esp_{x}\left[Q\left(t\right)\right]-\esp_{\pi}\left[Q\left(0\right)\right]|=0.
\end{eqnarray*}
\end{Teo}

\begin{Teo}[Teorema 6.4, Dai y Meyn \cite{DaiSean}]\label{Tma.6.4.DaiSean}
Suponga que se cumplen los supuestos A1), A2) y A3) y que el
modelo de flujo es estable. Sea $\nu$ cualquier distribuci\'on de
probabilidad en $\left(X,\mathcal{B}_{X}\right)$, y $\pi$ la
distribuci\'on estacionaria de $X$.
\begin{itemize}
\item[i)] Para cualquier $f:X\leftarrow\rea_{+}$
\begin{equation}
\lim_{t\rightarrow\infty}\frac{1}{t}\int_{o}^{t}f\left(X\left(s\right)\right)ds=\pi\left(f\right):=\int
f\left(x\right)\pi\left(dx\right)
\end{equation}
$\prob$-c.s.

\item[ii)] Para cualquier $f:X\leftarrow\rea_{+}$ con
$\pi\left(|f|\right)<\infty$, la ecuaci\'on anterior se cumple.
\end{itemize}
\end{Teo}

%_________________________________________________________________________
\subsection{Teorema 2.2}
%_________________________________________________________________________

\begin{Teo}[Teorema 2.2, Down \cite{Down}]\label{Tma2.2.Down}
Suponga que el fluido modelo es inestable en el sentido de que
para alguna $\epsilon_{0},c_{0}\geq0$,
\begin{equation}\label{Eq.Inestability}
|Q\left(T\right)|\geq\epsilon_{0}T-c_{0}\textrm{,   }T\geq0,
\end{equation}
para cualquier condici\'on inicial $Q\left(0\right)$, con
$|Q\left(0\right)|=1$. Entonces para cualquier $0<q\leq1$, existe
$B<0$ tal que para cualquier $|x|\geq B$,
\begin{equation}
\prob_{x}\left\{\mathbb{X}\rightarrow\infty\right\}\geq q.
\end{equation}
\end{Teo}

%_________________________________________________________________________
\subsection{Teorema 2.3}
%_________________________________________________________________________
\begin{Teo}[Teorema 2.3, Down \cite{Down}]\label{Tma2.3.Down}
Considere el siguiente valor:
\begin{equation}\label{Eq.Rho.1serv}
\rho=\sum_{k=1}^{K}\rho_{k}+max_{1\leq j\leq K}\left(\frac{\lambda_{j}}{\sum_{s=1}^{S}p_{js}\overline{N}_{s}}\right)\delta^{*}
\end{equation}
\begin{itemize}
\item[i)] Si $\rho<1$ entonces la red es estable, es decir, se cumple el teorema \ref{Tma.2.1.Down}.

\item[ii)] Si $\rho<1$ entonces la red es inestable, es decir, se cumple el teorema \ref{Tma2.2.Down}
\end{itemize}
\end{Teo}
\newpage
%_____________________________________________________________________
\subsection{Definiciones  B\'asicas}
%_____________________________________________________________________
\begin{Def}
Sea $X$ un conjunto y $\mathcal{F}$ una $\sigma$-\'algebra de
subconjuntos de $X$, la pareja $\left(X,\mathcal{F}\right)$ es
llamado espacio medible. Un subconjunto $A$ de $X$ es llamado
medible, o medible con respecto a $\mathcal{F}$, si
$A\in\mathcal{F}$.
\end{Def}

\begin{Def}
Sea $\left(X,\mathcal{F},\mu\right)$ espacio de medida. Se dice
que la medida $\mu$ es $\sigma$-finita si se puede escribir
$X=\bigcup_{n\geq1}X_{n}$ con $X_{n}\in\mathcal{F}$ y
$\mu\left(X_{n}\right)<\infty$.
\end{Def}

\begin{Def}\label{Cto.Borel}
Sea $X$ el conjunto de los \'umeros reales $\rea$. El \'algebra de
Borel es la $\sigma$-\'algebra $B$ generada por los intervalos
abiertos $\left(a,b\right)\in\rea$. Cualquier conjunto en $B$ es
llamado {\em Conjunto de Borel}.
\end{Def}

\begin{Def}\label{Funcion.Medible}
Una funci\'on $f:X\rightarrow\rea$, es medible si para cualquier
n\'umero real $\alpha$ el conjunto
\[\left\{x\in X:f\left(x\right)>\alpha\right\}\]
pertenece a $X$. Equivalentemente, se dice que $f$ es medible si
\[f^{-1}\left(\left(\alpha,\infty\right)\right)=\left\{x\in X:f\left(x\right)>\alpha\right\}\in\mathcal{F}.\]
\end{Def}


\begin{Def}\label{Def.Cilindros}
Sean $\left(\Omega_{i},\mathcal{F}_{i}\right)$, $i=1,2,\ldots,$
espacios medibles y $\Omega=\prod_{i=1}^{\infty}\Omega_{i}$ el
conjunto de todas las sucesiones
$\left(\omega_{1},\omega_{2},\ldots,\right)$ tales que
$\omega_{i}\in\Omega_{i}$, $i=1,2,\ldots,$. Si
$B^{n}\subset\prod_{i=1}^{\infty}\Omega_{i}$, definimos
$B_{n}=\left\{\omega\in\Omega:\left(\omega_{1},\omega_{2},\ldots,\omega_{n}\right)\in
B^{n}\right\}$. Al conjunto $B_{n}$ se le llama {\em cilindro} con
base $B^{n}$, el cilindro es llamado medible si
$B^{n}\in\prod_{i=1}^{\infty}\mathcal{F}_{i}$.
\end{Def}


\begin{Def}\label{Def.Proc.Adaptado}[TSP, Ash \cite{RBA}]
Sea $X\left(t\right),t\geq0$ proceso estoc\'astico, el proceso es
adaptado a la familia de $\sigma$-\'algebras $\mathcal{F}_{t}$,
para $t\geq0$, si para $s<t$ implica que
$\mathcal{F}_{s}\subset\mathcal{F}_{t}$, y $X\left(t\right)$ es
$\mathcal{F}_{t}$-medible para cada $t$. Si no se especifica
$\mathcal{F}_{t}$ entonces se toma $\mathcal{F}_{t}$ como
$\mathcal{F}\left(X\left(s\right),s\leq t\right)$, la m\'as
peque\~na $\sigma$-\'algebra de subconjuntos de $\Omega$ que hace
que cada $X\left(s\right)$, con $s\leq t$ sea Borel medible.
\end{Def}


\begin{Def}\label{Def.Tiempo.Paro}[TSP, Ash \cite{RBA}]
Sea $\left\{\mathcal{F}\left(t\right),t\geq0\right\}$ familia
creciente de sub $\sigma$-\'algebras. es decir,
$\mathcal{F}\left(s\right)\subset\mathcal{F}\left(t\right)$ para
$s\leq t$. Un tiempo de paro para $\mathcal{F}\left(t\right)$ es
una funci\'on $T:\Omega\rightarrow\left[0,\infty\right]$ tal que
$\left\{T\leq t\right\}\in\mathcal{F}\left(t\right)$ para cada
$t\geq0$. Un tiempo de paro para el proceso estoc\'astico
$X\left(t\right),t\geq0$ es un tiempo de paro para las
$\sigma$-\'algebras
$\mathcal{F}\left(t\right)=\mathcal{F}\left(X\left(s\right)\right)$.
\end{Def}

\begin{Def}
Sea $X\left(t\right),t\geq0$ proceso estoc\'astico, con
$\left(S,\chi\right)$ espacio de estados. Se dice que el proceso
es adaptado a $\left\{\mathcal{F}\left(t\right)\right\}$, es
decir, si para cualquier $s,t\in I$, $I$ conjunto de \'indices,
$s<t$, se tiene que
$\mathcal{F}\left(s\right)\subset\mathcal{F}\left(t\right)$ y
$X\left(t\right)$ es $\mathcal{F}\left(t\right)$-medible,
\end{Def}

\begin{Def}
Sea $X\left(t\right),t\geq0$ proceso estoc\'astico, se dice que es
un Proceso de Markov relativo a $\mathcal{F}\left(t\right)$ o que
$\left\{X\left(t\right),\mathcal{F}\left(t\right)\right\}$ es de
Markov si y s\'olo si para cualquier conjunto $B\in\chi$,  y
$s,t\in I$, $s<t$ se cumple que
\begin{equation}\label{Prop.Markov}
P\left\{X\left(t\right)\in
B|\mathcal{F}\left(s\right)\right\}=P\left\{X\left(t\right)\in
B|X\left(s\right)\right\}.
\end{equation}
\end{Def}
\begin{Note}
Si se dice que $\left\{X\left(t\right)\right\}$ es un Proceso de
Markov sin mencionar $\mathcal{F}\left(t\right)$, se asumir\'a que
\begin{eqnarray*}
\mathcal{F}\left(t\right)=\mathcal{F}_{0}\left(t\right)=\mathcal{F}\left(X\left(r\right),r\leq
t\right),
\end{eqnarray*}
entonces la ecuaci\'on (\ref{Prop.Markov}) se puede escribir como
\begin{equation}
P\left\{X\left(t\right)\in B|X\left(r\right),r\leq s\right\} =
P\left\{X\left(t\right)\in B|X\left(s\right)\right\}
\end{equation}
\end{Note}

\begin{Teo}
Sea $\left(X_{n},\mathcal{F}_{n},n=0,1,\ldots,\right\}$ Proceso de
Markov con espacio de estados $\left(S_{0},\chi_{0}\right)$
generado por una distribuici\'on inicial $P_{o}$ y probabilidad de
transici\'on $p_{mn}$, para $m,n=0,1,\ldots,$ $m<n$, que por
notaci\'on se escribir\'a como $p\left(m,n,x,B\right)\rightarrow
p_{mn}\left(x,B\right)$. Sea $S$ tiempo de paro relativo a la
$\sigma$-\'algebra $\mathcal{F}_{n}$. Sea $T$ funci\'on medible,
$T:\Omega\rightarrow\left\{0,1,\ldots,\right\}$. Sup\'ongase que
$T\geq S$, entonces $T$ es tiempo de paro. Si $B\in\chi_{0}$,
entonces
\begin{equation}\label{Prop.Fuerte.Markov}
P\left\{X\left(T\right)\in
B,T<\infty|\mathcal{F}\left(S\right)\right\} =
p\left(S,T,X\left(s\right),B\right)
\end{equation}
en $\left\{T<\infty\right\}$.
\end{Teo}

Propiedades importantes para el modelo de flujo retrasado:

\begin{Prop}
 Sea $\left(\overline{Q},\overline{T},\overline{T}^{0}\right)$ un flujo l\'imite de \ref{Equation.4.4} y suponga que cuando $x\rightarrow\infty$ a lo largo de
una subsucesi\'on
\[\left(\frac{1}{|x|}Q_{k}^{x}\left(0\right),\frac{1}{|x|}A_{k}^{x}\left(0\right),\frac{1}{|x|}B_{k}^{x}\left(0\right),\frac{1}{|x|}B_{k}^{x,0}\left(0\right)\right)\rightarrow\left(\overline{Q}_{k}\left(0\right),0,0,0\right)\]
para $k=1,\ldots,K$. EL flujo l\'imite tiene las siguientes
propiedades, donde las propiedades de la derivada se cumplen donde
la derivada exista:
\begin{itemize}
 \item[i)] Los vectores de tiempo ocupado $\overline{T}\left(t\right)$ y $\overline{T}^{0}\left(t\right)$ son crecientes y continuas con
$\overline{T}\left(0\right)=\overline{T}^{0}\left(0\right)=0$.
\item[ii)] Para todo $t\geq0$
\[\sum_{k=1}^{K}\left[\overline{T}_{k}\left(t\right)+\overline{T}_{k}^{0}\left(t\right)\right]=t\]
\item[iii)] Para todo $1\leq k\leq K$
\[\overline{Q}_{k}\left(t\right)=\overline{Q}_{k}\left(0\right)+\alpha_{k}t-\mu_{k}\overline{T}_{k}\left(t\right)\]
\item[iv)]  Para todo $1\leq k\leq K$
\[\dot{{\overline{T}}}_{k}\left(t\right)=\beta_{k}\] para $\overline{Q}_{k}\left(t\right)=0$.
\item[v)] Para todo $k,j$
\[\mu_{k}^{0}\overline{T}_{k}^{0}\left(t\right)=\mu_{j}^{0}\overline{T}_{j}^{0}\left(t\right)\]
\item[vi)]  Para todo $1\leq k\leq K$
\[\mu_{k}\dot{{\overline{T}}}_{k}\left(t\right)=l_{k}\mu_{k}^{0}\dot{{\overline{T}}}_{k}^{0}\left(t\right)\] para $\overline{Q}_{k}\left(t\right)>0$.
\end{itemize}
\end{Prop}

\begin{Lema}[Lema 3.1 \cite{Chen}]\label{Lema3.1}
Si el modelo de flujo es estable, definido por las ecuaciones
(3.8)-(3.13), entonces el modelo de flujo retrasado tambin es
estable.
\end{Lema}

\begin{Teo}[Teorema 5.2 \cite{Chen}]\label{Tma.5.2}
Si el modelo de flujo lineal correspondiente a la red de cola es
estable, entonces la red de colas es estable.
\end{Teo}

\begin{Teo}[Teorema 5.1 \cite{Chen}]\label{Tma.5.1.Chen}
La red de colas es estable si existe una constante $t_{0}$ que
depende de $\left(\alpha,\mu,T,U\right)$ y $V$ que satisfagan las
ecuaciones (5.1)-(5.5), $Z\left(t\right)=0$, para toda $t\geq
t_{0}$.
\end{Teo}



\begin{Lema}[Lema 5.2 \cite{Gut}]\label{Lema.5.2.Gut}
Sea $\left\{\xi\left(k\right):k\in\ent\right\}$ sucesin de
variables aleatorias i.i.d. con valores en
$\left(0,\infty\right)$, y sea $E\left(t\right)$ el proceso de
conteo
\[E\left(t\right)=max\left\{n\geq1:\xi\left(1\right)+\cdots+\xi\left(n-1\right)\leq t\right\}.\]
Si $E\left[\xi\left(1\right)\right]<\infty$, entonces para
cualquier entero $r\geq1$
\begin{equation}
lim_{t\rightarrow\infty}\esp\left[\left(\frac{E\left(t\right)}{t}\right)^{r}\right]=\left(\frac{1}{E\left[\xi_{1}\right]}\right)^{r}
\end{equation}
de aqu, bajo estas condiciones
\begin{itemize}
\item[a)] Para cualquier $t>0$,
$sup_{t\geq\delta}\esp\left[\left(\frac{E\left(t\right)}{t}\right)^{r}\right]$

\item[b)] Las variables aleatorias
$\left\{\left(\frac{E\left(t\right)}{t}\right)^{r}:t\geq1\right\}$
son uniformemente integrables.
\end{itemize}
\end{Lema}

\begin{Teo}[Teorema 5.1: Ley Fuerte para Procesos de Conteo
\cite{Gut}]\label{Tma.5.1.Gut} Sea
$0<\mu<\esp\left(X_{1}\right]\leq\infty$. entonces

\begin{itemize}
\item[a)] $\frac{N\left(t\right)}{t}\rightarrow\frac{1}{\mu}$
a.s., cuando $t\rightarrow\infty$.


\item[b)]$\esp\left[\frac{N\left(t\right)}{t}\right]^{r}\rightarrow\frac{1}{\mu^{r}}$,
cuando $t\rightarrow\infty$ para todo $r>0$..
\end{itemize}
\end{Teo}


\begin{Prop}[Proposicin 5.1 \cite{DaiSean}]\label{Prop.5.1}
Suponga que los supuestos (A1) y (A2) se cumplen, adems suponga
que el modelo de flujo es estable. Entonces existe $t_{0}>0$ tal
que
\begin{equation}\label{Eq.Prop.5.1}
lim_{|x|\rightarrow\infty}\frac{1}{|x|^{p+1}}\esp_{x}\left[|X\left(t_{0}|x|\right)|^{p+1}\right]=0.
\end{equation}

\end{Prop}


\begin{Prop}[Proposici\'on 5.3 \cite{DaiSean}]
Sea $X$ proceso de estados para la red de colas, y suponga que se
cumplen los supuestos (A1) y (A2), entonces para alguna constante
positiva $C_{p+1}<\infty$, $\delta>0$ y un conjunto compacto
$C\subset X$.

\begin{equation}\label{Eq.5.4}
\esp_{x}\left[\int_{0}^{\tau_{C}\left(\delta\right)}\left(1+|X\left(t\right)|^{p}\right)dt\right]\leq
C_{p+1}\left(1+|x|^{p+1}\right)
\end{equation}
\end{Prop}

\begin{Prop}[Proposici\'on 5.4 \cite{DaiSean}]
Sea $X$ un proceso de Markov Borel Derecho en $X$, sea
$f:X\leftarrow\rea_{+}$ y defina para alguna $\delta>0$, y un
conjunto cerrado $C\subset X$
\[V\left(x\right):=\esp_{x}\left[\int_{0}^{\tau_{C}\left(\delta\right)}f\left(X\left(t\right)\right)dt\right]\]
para $x\in X$. Si $V$ es finito en todas partes y uniformemente
acotada en $C$, entonces existe $k<\infty$ tal que
\begin{equation}\label{Eq.5.11}
\frac{1}{t}\esp_{x}\left[V\left(x\right)\right]+\frac{1}{t}\int_{0}^{t}\esp_{x}\left[f\left(X\left(s\right)\right)ds\right]\leq\frac{1}{t}V\left(x\right)+k,
\end{equation}
para $x\in X$ y $t>0$.
\end{Prop}


\begin{Teo}[Teorema 5.5 \cite{DaiSean}]
Suponga que se cumplen (A1) y (A2), adems suponga que el modelo
de flujo es estable. Entonces existe una constante $k_{p}<\infty$
tal que
\begin{equation}\label{Eq.5.13}
\frac{1}{t}\int_{0}^{t}\esp_{x}\left[|Q\left(s\right)|^{p}\right]ds\leq
k_{p}\left\{\frac{1}{t}|x|^{p+1}+1\right\}
\end{equation}
para $t\geq0$, $x\in X$. En particular para cada condicin inicial
\begin{equation}\label{Eq.5.14}
Limsup_{t\rightarrow\infty}\frac{1}{t}\int_{0}^{t}\esp_{x}\left[|Q\left(s\right)|^{p}\right]ds\leq
k_{p}
\end{equation}
\end{Teo}

\begin{Teo}[Teorema 6.2\cite{DaiSean}]\label{Tma.6.2}
Suponga que se cumplen los supuestos (A1)-(A3) y que el modelo de
flujo es estable, entonces se tiene que
\[\parallel P^{t}\left(c,\cdot\right)-\pi\left(\cdot\right)\parallel_{f_{p}}\rightarrow0\]
para $t\rightarrow\infty$ y $x\in X$. En particular para cada
condicin inicial
\[lim_{t\rightarrow\infty}\esp_{x}\left[\left|Q_{t}\right|^{p}\right]=\esp_{\pi}\left[\left|Q_{0}\right|^{p}\right]<\infty\]
\end{Teo}


\begin{Teo}[Teorema 6.3\cite{DaiSean}]\label{Tma.6.3}
Suponga que se cumplen los supuestos (A1)-(A3) y que el modelo de
flujo es estable, entonces con
$f\left(x\right)=f_{1}\left(x\right)$, se tiene que
\[lim_{t\rightarrow\infty}t^{(p-1)\left|P^{t}\left(c,\cdot\right)-\pi\left(\cdot\right)\right|_{f}=0},\]
para $x\in X$. En particular, para cada condicin inicial
\[lim_{t\rightarrow\infty}t^{(p-1)\left|\esp_{x}\left[Q_{t}\right]-\esp_{\pi}\left[Q_{0}\right]\right|=0}.\]
\end{Teo}



%_____________________________________________________________________________________
\subsection{Proceso de Estados Markoviano para el Sistema}
%_________________________________________________________________________


Sean $Q_{k}\left(t\right)$ el n\'umero de usuarios en la cola $k$,
$A_{k}\left(t\right)$ el tiempo residual de arribos a la cola $k$,
para cada servidor $m$, sea $H_{m}\left(t\right)$ par ordenado que
consiste en la cola que est\'a siendo atendida y la pol\'itica de
servicio que se est\'a utilizando. $B_{m}\left(t\right)$ los
tiempos de servicio residuales, $B_{m}^{0}\left(t\right)$ el
tiempo residual de traslado, $C_{m}\left(t\right)$ el n\'umero de
usuarios atendidos durante la visita del servidor a la cola dada
en $H_{m}\left(t\right)$.

El proceso para el sistema de visitas se puede definir como:

\begin{equation}\label{Esp.Edos.Down}
X\left(t\right)^{T}=\left(Q_{k}\left(t\right),A_{k}\left(t\right),B_{m}\left(t\right),B_{m}^{0}\left(t\right),C_{m}\left(t\right)\right)
\end{equation}
para $k=1,\ldots,K$ y $m=1,2,\ldots,M$. $X$ evoluciona en el
espacio de estados:
$X=\ent_{+}^{K}\times\rea_{+}^{K}\times\left(\left\{1,2,\ldots,K\right\}\times\left\{1,2,\ldots,S\right\}\right)^{M}\times\rea_{+}^{K}\times\rea_{+}^{K}\times\ent_{+}^{K}$.\\

Antes enunciemos los supuestos que regir\'an en la red.


\begin{itemize}
\item[A1)] $\xi_{1},\ldots,\xi_{K},\eta_{1},\ldots,\eta_{K}$ son
mutuamente independientes y son sucesiones independientes e
id\'enticamente distribuidas.

\item[A2)] Para alg\'un entero $p\geq1$
\begin{eqnarray*}
\esp\left[\xi_{l}\left(1\right)^{p+1}\right]<\infty\textrm{ para }l\in\mathcal{A}\textrm{ y }\\
\esp\left[\eta_{k}\left(1\right)^{p+1}\right]<\infty\textrm{ para
}k=1,\ldots,K.
\end{eqnarray*}
donde $\mathcal{A}$ es la clase de posibles arribos.

\item[A3)] Para $k=1,2,\ldots,K$ existe una funci\'on positiva
$q_{k}\left(x\right)$ definida en $\rea_{+}$, y un entero $j_{k}$,
tal que
\begin{eqnarray}
P\left(\xi_{k}\left(1\right)\geq x\right)>0\textrm{, para todo }x>0\\
P\left(\xi_{k}\left(1\right)+\ldots\xi_{k}\left(j_{k}\right)\in dx\right)\geq q_{k}\left(x\right)dx0\textrm{ y }\\
\int_{0}^{\infty}q_{k}\left(x\right)dx>0
\end{eqnarray}
\end{itemize}
%_________________________________________________________________________
\subsection{Procesos Fuerte de Markov}
%_________________________________________________________________________

En Dai \cite{Dai} se muestra que para una amplia serie de
disciplinas de servicio el proceso $X$ es un Proceso Fuerte de
Markov, y por tanto se puede asumir que
\[\left(\left(\Omega,\mathcal{F}\right),\mathcal{F}_{t},X\left(t\right),\theta_{t},P_{x}\right)\]
es un proceso de Borel Derecho, Sharpe \cite{Sharpe}, en el
espacio de estados medible
$\left(X,\mathcal{B}_{X}\right)$.


Se har\'an las siguientes consideraciones: $E$ es un espacio
m\'etrico separable.


\begin{Def}
Un espacio topol\'ogico $E$ es llamado {\em Luisin} si es
homeomorfo a un subconjunto de Borel de un espacio m\'etrico
compacto.
\end{Def}

\begin{Def}
Un espacio topol\'ogico $E$ es llamado de {\em Rad\'on} si es
homeomorfo a un subconjunto universalmente medible de un espacio
m\'etrico compacto.
\end{Def}

Equivalentemente, la definici\'on de un espacio de Rad\'on puede
encontrarse en los siguientes t\'erminos:

\begin{Def}
$E$ es un espacio de Rad\'on si cada medida finita en
$\left(E,\mathcal{B}\left(E\right)\right)$ es regular interior o
cerrada, {\em tight}.
\end{Def}

\begin{Def}
Una medida finita, $\lambda$ en la $\sigma$-\'algebra de Borel de
un espacio metrizable $E$ se dice cerrada si
\begin{equation}\label{Eq.A2.3}
\lambda\left(E\right)=sup\left\{\lambda\left(K\right):K\textrm{ es
compacto en }E\right\}.
\end{equation}
\end{Def}

El siguiente teorema nos permite tener una mejor caracterizaci\'on
de los espacios de Rad\'on:
\begin{Teo}\label{Tma.A2.2}
Sea $E$ espacio separable metrizable. Entonces $E$ es Radoniano si
y s\'olo s\'i cada medida finita en
$\left(E,\mathcal{B}\left(E\right)\right)$ es cerrada.
\end{Teo}

%_________________________________________________________________________________________
\subsection{Propiedades de Markov}
%_________________________________________________________________________________________

Sea $E$ espacio de estados, tal que $E$ es un espacio de Rad\'on,
$\mathcal{B}\left(E\right)$ $\sigma$-\'algebra de Borel en $E$,
que se denotar\'a por $\mathcal{E}$.

Sea $\left(X,\mathcal{G},\prob\right)$ espacio de probabilidad,
$I\subset\rea$ conjunto de índices. Sea $\mathcal{F}_{\leq
t}$ la $\sigma$-\'algebra natural definida como
$\sigma\left\{f\left(X_{r}\right):r\in I, r\leq
t,f\in\mathcal{E}\right\}$. Se considerar\'a una
$\sigma$-\'algebra m\'as general, $ \left(\mathcal{G}_{t}\right)$
tal que $\left(X_{t}\right)$ sea $\mathcal{E}$-adaptado.

\begin{Def}
Una familia $\left(P_{s,t}\right)$ de kernels de Markov en
$\left(E,\mathcal{E}\right)$ indexada por pares $s,t\in I$, con
$s\leq t$ es una funci\'on de transici\'on en $\ER$, si  para todo
$r\leq s< t$ en $I$ y todo $x\in E$, $B\in\mathcal{E}$
\begin{equation}\label{Eq.Kernels}
P_{r,t}\left(x,B\right)=\int_{E}P_{r,s}\left(x,dy\right)P_{s,t}\left(y,B\right)\footnote{Ecuaci\'on
de Chapman-Kolmogorov}.
\end{equation}
\end{Def}

Se dice que la funci\'on de transici\'on $\KM$ en $\ER$ es la
funci\'on de transici\'on para un proceso $\PE$  con valores en
$E$ y que satisface la propiedad de
Markov \footnote{\begin{equation}\label{Eq.1.4.S}
\prob\left\{H|\mathcal{G}_{t}\right\}=\prob\left\{H|X_{t}\right\}\textrm{
}H\in p\mathcal{F}_{\geq t}.
\end{equation}} (\ref{Eq.1.4.S}) relativa a $\left(\mathcal{G}_{t}\right)$ si

\begin{equation}\label{Eq.1.6.S}
\prob\left\{f\left(X_{t}\right)|\mathcal{G}_{s}\right\}=P_{s,t}f\left(X_{t}\right)\textrm{
}s\leq t\in I,\textrm{ }f\in b\mathcal{E}.
\end{equation}

\begin{Def}
Una familia $\left(P_{t}\right)_{t\geq0}$ de kernels de Markov en
$\ER$ es llamada {\em Semigrupo de Transici\'on de Markov} o {\em
Semigrupo de Transici\'on} si
\[P_{t+s}f\left(x\right)=P_{t}\left(P_{s}f\right)\left(x\right),\textrm{ }t,s\geq0,\textrm{ }x\in E\textrm{ }f\in b\mathcal{E}.\]
\end{Def}

\begin{Note}
Si la funci\'on de transici\'on $\KM$ es llamada homog\'enea si
$P_{s,t}=P_{t-s}$.
\end{Note}


Un proceso de Markov que satisface la ecuaci\'on (\ref{Eq.1.6.S})
con funci\'on de transici\'on homog\'enea $\left(P_{t}\right)$
tiene la propiedad caracter\'istica
\begin{equation}\label{Eq.1.8.S}
\prob\left\{f\left(X_{t+s}\right)|\mathcal{G}_{t}\right\}=P_{s}f\left(X_{t}\right)\textrm{
}t,s\geq0,\textrm{ }f\in b\mathcal{E}.
\end{equation}
La ecuaci\'on anterior es la {\em Propiedad Simple de Markov} de
$X$ relativa a $\left(P_{t}\right)$.

En este sentido el proceso $\PE$ cumple con la propiedad de Markov
(\ref{Eq.1.8.S}) relativa a
$\left(\Omega,\mathcal{G},\mathcal{G}_{t},\prob\right)$ con
semigrupo de transici\'on $\left(P_{t}\right)$.

%_________________________________________________________________________________________
\subsection{Primer Condici\'on de Regularidad}
%_________________________________________________________________________________________


\begin{Def}
Un proceso estoc\'astico $\PE$ definido en
$\left(\Omega,\mathcal{G},\prob\right)$ con valores en el espacio
topol\'ogico $E$ es continuo por la derecha si cada trayectoria
muestral $t\rightarrow X_{t}\left(w\right)$ es un mapeo continuo
por la derecha de $I$ en $E$.
\end{Def}

\begin{Def}[HD1]\label{Eq.2.1.S}
Un semigrupo de Markov $\left/P_{t}\right)$ en un espacio de
Rad\'on $E$ se dice que satisface la condici\'on {\em HD1} si,
dada una medida de probabilidad $\mu$ en $E$, existe una
$\sigma$-\'algebra $\mathcal{E^{*}}$ con
$\mathcal{E}\subset\mathcal{E}$ y
$P_{t}\left(b\mathcal{E}^{*}\right)\subset b\mathcal{E}^{*}$, y un
$\mathcal{E}^{*}$-proceso $E$-valuado continuo por la derecha
$\PE$ en alg\'un espacio de probabilidad filtrado
$\left(\Omega,\mathcal{G},\mathcal{G}_{t},\prob\right)$ tal que
$X=\left(\Omega,\mathcal{G},\mathcal{G}_{t},\prob\right)$ es de
Markov (Homog\'eneo) con semigrupo de transici\'on $(P_{t})$ y
distribuci\'on inicial $\mu$.
\end{Def}
Consid\'erese la colecci\'on de variables aleatorias $X_{t}$
definidas en alg\'un espacio de probabilidad, y una colecci\'on de
medidas $\mathbf{P}^{x}$ tales que
$\mathbf{P}^{x}\left\{X_{0}=x\right\}$, y bajo cualquier
$\mathbf{P}^{x}$, $X_{t}$ es de Markov con semigrupo
$\left(P_{t}\right)$. $\mathbf{P}^{x}$ puede considerarse como la
distribuci\'on condicional de $\mathbf{P}$ dado $X_{0}=x$.

\begin{Def}\label{Def.2.2.S}
Sea $E$ espacio de Rad\'on, $\SG$ semigrupo de Markov en $\ER$. La
colecci\'on
$\mathbf{X}=\left(\Omega,\mathcal{G},\mathcal{G}_{t},X_{t},\theta_{t},\CM\right)$
es un proceso $\mathcal{E}$-Markov continuo por la derecha simple,
con espacio de estados $E$ y semigrupo de transici\'on $\SG$ en
caso de que $\mathbf{X}$ satisfaga las siguientes
condiciones:
\begin{itemize}
\item[i)] $\left(\Omega,\mathcal{G},\mathcal{G}_{t}\right)$ es un
espacio de medida filtrado, y $X_{t}$ es un proceso $E$-valuado
continuo por la derecha $\mathcal{E}^{*}$-adaptado a
$\left(\mathcal{G}_{t}\right)$;

\item[ii)] $\left(\theta_{t}\right)_{t\geq0}$ es una colecci\'on
de operadores {\em shift} para $X$, es decir, mapea $\Omega$ en
s\'i mismo satisfaciendo para $t,s\geq0$,

\begin{equation}\label{Eq.Shift}
\theta_{t}\circ\theta_{s}=\theta_{t+s}\textrm{ y
}X_{t}\circ\theta_{t}=X_{t+s};
\end{equation}

\item[iii)] Para cualquier $x\in E$,$\CM\left\{X_{0}=x\right\}=1$,
y el proceso $\PE$ tiene la propiedad de Markov (\ref{Eq.1.8.S})
con semigrupo de transici\'on $\SG$ relativo a
$\left(\Omega,\mathcal{G},\mathcal{G}_{t},\CM\right)$.
\end{itemize}
\end{Def}


\begin{Def}[HD2]\label{Eq.2.2.S}
Para cualquier $\alpha>0$ y cualquier $f\in S^{\alpha}$, el
proceso $t\rightarrow f\left(X_{t}\right)$ es continuo por la
derecha casi seguramente.
\end{Def}

\begin{Def}\label{Def.PD}
Un sistema
$\mathbf{X}=\left(\Omega,\mathcal{G},\mathcal{G}_{t},X_{t},\theta_{t},\CM\right)$
es un proceso derecho en el espacio de Rad\'on $E$ con semigrupo
de transici\'on $\SG$ provisto de:
\begin{itemize}
\item[i)] $\mathbf{X}$ es una realizaci\'on  continua por la
derecha, \ref{Def.2.2.S}, de $\SG$.

\item[ii)] $\mathbf{X}$ satisface la condicion HD2,
\ref{Eq.2.2.S}, relativa a $\mathcal{G}_{t}$.

\item[iii)] $\mathcal{G}_{t}$ es aumentado y continuo por la
derecha.
\end{itemize}
\end{Def}


\begin{Def}
Sea $X$ un conjunto y $\mathcal{F}$ una $\sigma$-\'algebra de
subconjuntos de $X$, la pareja $\left(X,\mathcal{F}\right)$ es
llamado espacio medible. Un subconjunto $A$ de $X$ es llamado
medible, o medible con respecto a $\mathcal{F}$, si
$A\in\mathcal{F}$.
\end{Def}

\begin{Def}
Sea $\left(X,\mathcal{F},\mu\right)$ espacio de medida. Se dice
que la medida $\mu$ es $\sigma$-finita si se puede escribir
$X=\bigcup_{n\geq1}X_{n}$ con $X_{n}\in\mathcal{F}$ y
$\mu\left(X_{n}\right)<\infty$.
\end{Def}

\begin{Def}\label{Cto.Borel}
Sea $X$ el conjunto de los \'umeros reales $\rea$. El \'algebra de
Borel es la $\sigma$-\'algebra $B$ generada por los intervalos
abiertos $\left(a,b\right)\in\rea$. Cualquier conjunto en $B$ es
llamado {\em Conjunto de Borel}.
\end{Def}

\begin{Def}\label{Funcion.Medible}
Una funci\'on $f:X\rightarrow\rea$, es medible si para cualquier
n\'umero real $\alpha$ el conjunto
\[\left\{x\in X:f\left(x\right)>\alpha\right\}\]
pertenece a $X$. Equivalentemente, se dice que $f$ es medible si
\[f^{-1}\left(\left(\alpha,\infty\right)\right)=\left\{x\in X:f\left(x\right)>\alpha\right\}\in\mathcal{F}.\]
\end{Def}


\begin{Def}\label{Def.Cilindros}
Sean $\left(\Omega_{i},\mathcal{F}_{i}\right)$, $i=1,2,\ldots,$
espacios medibles y $\Omega=\prod_{i=1}^{\infty}\Omega_{i}$ el
conjunto de todas las sucesiones
$\left(\omega_{1},\omega_{2},\ldots,\right)$ tales que
$\omega_{i}\in\Omega_{i}$, $i=1,2,\ldots,$. Si
$B^{n}\subset\prod_{i=1}^{\infty}\Omega_{i}$, definimos
$B_{n}=\left\{\omega\in\Omega:\left(\omega_{1},\omega_{2},\ldots,\omega_{n}\right)\in
B^{n}\right\}$. Al conjunto $B_{n}$ se le llama {\em cilindro} con
base $B^{n}$, el cilindro es llamado medible si
$B^{n}\in\prod_{i=1}^{\infty}\mathcal{F}_{i}$.
\end{Def}


\begin{Def}\label{Def.Proc.Adaptado}[TSP, Ash \cite{RBA}]
Sea $X\left(t\right),t\geq0$ proceso estoc\'astico, el proceso es
adaptado a la familia de $\sigma$-\'algebras $\mathcal{F}_{t}$,
para $t\geq0$, si para $s<t$ implica que
$\mathcal{F}_{s}\subset\mathcal{F}_{t}$, y $X\left(t\right)$ es
$\mathcal{F}_{t}$-medible para cada $t$. Si no se especifica
$\mathcal{F}_{t}$ entonces se toma $\mathcal{F}_{t}$ como
$\mathcal{F}\left(X\left(s\right),s\leq t\right)$, la m\'as
peque\~na $\sigma$-\'algebra de subconjuntos de $\Omega$ que hace
que cada $X\left(s\right)$, con $s\leq t$ sea Borel medible.
\end{Def}


\begin{Def}\label{Def.Tiempo.Paro}[TSP, Ash \cite{RBA}]
Sea $\left\{\mathcal{F}\left(t\right),t\geq0\right\}$ familia
creciente de sub $\sigma$-\'algebras. es decir,
$\mathcal{F}\left(s\right)\subset\mathcal{F}\left(t\right)$ para
$s\leq t$. Un tiempo de paro para $\mathcal{F}\left(t\right)$ es
una funci\'on $T:\Omega\rightarrow\left[0,\infty\right]$ tal que
$\left\{T\leq t\right\}\in\mathcal{F}\left(t\right)$ para cada
$t\geq0$. Un tiempo de paro para el proceso estoc\'astico
$X\left(t\right),t\geq0$ es un tiempo de paro para las
$\sigma$-\'algebras
$\mathcal{F}\left(t\right)=\mathcal{F}\left(X\left(s\right)\right)$.
\end{Def}

\begin{Def}
Sea $X\left(t\right),t\geq0$ proceso estoc\'astico, con
$\left(S,\chi\right)$ espacio de estados. Se dice que el proceso
es adaptado a $\left\{\mathcal{F}\left(t\right)\right\}$, es
decir, si para cualquier $s,t\in I$, $I$ conjunto de \'indices,
$s<t$, se tiene que
$\mathcal{F}\left(s\right)\subset\mathcal{F}\left(t\right)$ y
$X\left(t\right)$ es $\mathcal{F}\left(t\right)$-medible,
\end{Def}

\begin{Def}
Sea $X\left(t\right),t\geq0$ proceso estoc\'astico, se dice que es
un Proceso de Markov relativo a $\mathcal{F}\left(t\right)$ o que
$\left\{X\left(t\right),\mathcal{F}\left(t\right)\right\}$ es de
Markov si y s\'olo si para cualquier conjunto $B\in\chi$,  y
$s,t\in I$, $s<t$ se cumple que
\begin{equation}\label{Prop.Markov}
P\left\{X\left(t\right)\in
B|\mathcal{F}\left(s\right)\right\}=P\left\{X\left(t\right)\in
B|X\left(s\right)\right\}.
\end{equation}
\end{Def}
\begin{Note}
Si se dice que $\left\{X\left(t\right)\right\}$ es un Proceso de
Markov sin mencionar $\mathcal{F}\left(t\right)$, se asumir\'a que
\begin{eqnarray*}
\mathcal{F}\left(t\right)=\mathcal{F}_{0}\left(t\right)=\mathcal{F}\left(X\left(r\right),r\leq
t\right),
\end{eqnarray*}
entonces la ecuaci\'on (\ref{Prop.Markov}) se puede escribir como
\begin{equation}
P\left\{X\left(t\right)\in B|X\left(r\right),r\leq s\right\} =
P\left\{X\left(t\right)\in B|X\left(s\right)\right\}
\end{equation}
\end{Note}

\begin{Teo}
Sea $\left(X_{n},\mathcal{F}_{n},n=0,1,\ldots,\right\}$ Proceso de
Markov con espacio de estados $\left(S_{0},\chi_{0}\right)$
generado por una distribuici\'on inicial $P_{o}$ y probabilidad de
transici\'on $p_{mn}$, para $m,n=0,1,\ldots,$ $m<n$, que por
notaci\'on se escribir\'a como $p\left(m,n,x,B\right)\rightarrow
p_{mn}\left(x,B\right)$. Sea $S$ tiempo de paro relativo a la
$\sigma$-\'algebra $\mathcal{F}_{n}$. Sea $T$ funci\'on medible,
$T:\Omega\rightarrow\left\{0,1,\ldots,\right\}$. Sup\'ongase que
$T\geq S$, entonces $T$ es tiempo de paro. Si $B\in\chi_{0}$,
entonces
\begin{equation}\label{Prop.Fuerte.Markov}
P\left\{X\left(T\right)\in
B,T<\infty|\mathcal{F}\left(S\right)\right\} =
p\left(S,T,X\left(s\right),B\right)
\end{equation}
en $\left\{T<\infty\right\}$.
\end{Teo}


Sea $K$ conjunto numerable y sea $d:K\rightarrow\nat$ funci\'on.
Para $v\in K$, $M_{v}$ es un conjunto abierto de
$\rea^{d\left(v\right)}$. Entonces \[E=\cup_{v\in
K}M_{v}=\left\{\left(v,\zeta\right):v\in K,\zeta\in
M_{v}\right\}.\]

Sea $\mathcal{E}$ la clase de conjuntos medibles en $E$:
\[\mathcal{E}=\left\{\cup_{v\in K}A_{v}:A_{v}\in \mathcal{M}_{v}\right\}.\]

donde $\mathcal{M}$ son los conjuntos de Borel de $M_{v}$.
Entonces $\left(E,\mathcal{E}\right)$ es un espacio de Borel. El
estado del proceso se denotar\'a por
$\mathbf{x}_{t}=\left(v_{t},\zeta_{t}\right)$. La distribuci\'on
de $\left(\mathbf{x}_{t}\right)$ est\'a determinada por por los
siguientes objetos:

\begin{itemize}
\item[i)] Los campos vectoriales $\left(\mathcal{H}_{v},v\in
K\right)$. \item[ii)] Una funci\'on medible $\lambda:E\rightarrow
\rea_{+}$. \item[iii)] Una medida de transici\'on
$Q:\mathcal{E}\times\left(E\cup\Gamma^{*}\right)\rightarrow\left[0,1\right]$
donde
\begin{equation}
\Gamma^{*}=\cup_{v\in K}\partial^{*}M_{v}.
\end{equation}
y
\begin{equation}
\partial^{*}M_{v}=\left\{z\in\partial M_{v}:\mathbf{\mathbf{\phi}_{v}\left(t,\zeta\right)=\mathbf{z}}\textrm{ para alguna }\left(t,\zeta\right)\in\rea_{+}\times M_{v}\right\}.
\end{equation}
$\partial M_{v}$ denota  la frontera de $M_{v}$.
\end{itemize}

El campo vectorial $\left(\mathcal{H}_{v},v\in K\right)$ se supone
tal que para cada $\mathbf{z}\in M_{v}$ existe una \'unica curva
integral $\mathbf{\phi}_{v}\left(t,\zeta\right)$ que satisface la
ecuaci\'on

\begin{equation}
\frac{d}{dt}f\left(\zeta_{t}\right)=\mathcal{H}f\left(\zeta_{t}\right),
\end{equation}
con $\zeta_{0}=\mathbf{z}$, para cualquier funci\'on suave
$f:\rea^{d}\rightarrow\rea$ y $\mathcal{H}$ denota el operador
diferencial de primer orden, con $\mathcal{H}=\mathcal{H}_{v}$ y
$\zeta_{t}=\mathbf{\phi}\left(t,\mathbf{z}\right)$. Adem\'as se
supone que $\mathcal{H}_{v}$ es conservativo, es decir, las curvas
integrales est\'an definidas para todo $t>0$.

Para $\mathbf{x}=\left(v,\zeta\right)\in E$ se denota
\[t^{*}\mathbf{x}=inf\left\{t>0:\mathbf{\phi}_{v}\left(t,\zeta\right)\in\partial^{*}M_{v}\right\}\]

En lo que respecta a la funci\'on $\lambda$, se supondr\'a que
para cada $\left(v,\zeta\right)\in E$ existe un $\epsilon>0$ tal
que la funci\'on
$s\rightarrow\lambda\left(v,\phi_{v}\left(s,\zeta\right)\right)\in
E$ es integrable para $s\in\left[0,\epsilon\right)$. La medida de
transici\'on $Q\left(A;\mathbf{x}\right)$ es una funci\'on medible
de $\mathbf{x}$ para cada $A\in\mathcal{E}$, definida para
$\mathbf{x}\in E\cup\Gamma^{*}$ y es una medida de probabilidad en
$\left(E,\mathcal{E}\right)$ para cada $\mathbf{x}\in E$.

El movimiento del proceso $\left(\mathbf{x}_{t}\right)$ comenzando
en $\mathbf{x}=\left(n,\mathbf{z}\right)\in E$ se puede construir
de la siguiente manera, def\'inase la funci\'on $F$ por

\begin{equation}
F\left(t\right)=\left\{\begin{array}{ll}\\
exp\left(-\int_{0}^{t}\lambda\left(n,\phi_{n}\left(s,\mathbf{z}\right)\right)ds\right), & t<t^{*}\left(\mathbf{x}\right),\\
0, & t\geq t^{*}\left(\mathbf{x}\right)
\end{array}\right.
\end{equation}

Sea $T_{1}$ una variable aleatoria tal que
$\prob\left[T_{1}>t\right]=F\left(t\right)$, ahora sea la variable
aleatoria $\left(N,Z\right)$ con distribuici\'on
$Q\left(\cdot;\phi_{n}\left(T_{1},\mathbf{z}\right)\right)$. La
trayectoria de $\left(\mathbf{x}_{t}\right)$ para $t\leq T_{1}$
es\footnote{Revisar p\'agina 362, y 364 de Davis \cite{Davis}.}
\begin{eqnarray*}
\mathbf{x}_{t}=\left(v_{t},\zeta_{t}\right)=\left\{\begin{array}{ll}
\left(n,\phi_{n}\left(t,\mathbf{z}\right)\right), & t<T_{1},\\
\left(N,\mathbf{Z}\right), & t=t_{1}.
\end{array}\right.
\end{eqnarray*}

Comenzando en $\mathbf{x}_{T_{1}}$ se selecciona el siguiente
tiempo de intersalto $T_{2}-T_{1}$ lugar del post-salto
$\mathbf{x}_{T_{2}}$ de manera similar y as\'i sucesivamente. Este
procedimiento nos da una trayectoria determinista por partes
$\mathbf{x}_{t}$ con tiempos de salto $T_{1},T_{2},\ldots$. Bajo
las condiciones enunciadas para $\lambda,T_{1}>0$  y
$T_{1}-T_{2}>0$ para cada $i$, con probabilidad 1. Se supone que
se cumple la siquiente condici\'on.

\begin{Sup}[Supuesto 3.1, Davis \cite{Davis}]\label{Sup3.1.Davis}
Sea $N_{t}:=\sum_{t}\indora_{\left(t\geq t\right)}$ el n\'umero de
saltos en $\left[0,t\right]$. Entonces
\begin{equation}
\esp\left[N_{t}\right]<\infty\textrm{ para toda }t.
\end{equation}
\end{Sup}

es un proceso de Markov, m\'as a\'un, es un Proceso Fuerte de
Markov, es decir, la Propiedad Fuerte de Markov se cumple para
cualquier tiempo de paro.
%_________________________________________________________________________

En esta secci\'on se har\'an las siguientes consideraciones: $E$
es un espacio m\'etrico separable y la m\'etrica $d$ es compatible
con la topolog\'ia.


\begin{Def}
Un espacio topol\'ogico $E$ es llamado {\em Luisin} si es
homeomorfo a un subconjunto de Borel de un espacio m\'etrico
compacto.
\end{Def}

\begin{Def}
Un espacio topol\'ogico $E$ es llamado de {\em Rad\'on} si es
homeomorfo a un subconjunto universalmente medible de un espacio
m\'etrico compacto.
\end{Def}

Equivalentemente, la definici\'on de un espacio de Rad\'on puede
encontrarse en los siguientes t\'erminos:


\begin{Def}
$E$ es un espacio de Rad\'on si cada medida finita en
$\left(E,\mathcal{B}\left(E\right)\right)$ es regular interior o cerrada,
{\em tight}.
\end{Def}

\begin{Def}
Una medida finita, $\lambda$ en la $\sigma$-\'algebra de Borel de
un espacio metrizable $E$ se dice cerrada si
\begin{equation}\label{Eq.A2.3}
\lambda\left(E\right)=sup\left\{\lambda\left(K\right):K\textrm{ es
compacto en }E\right\}.
\end{equation}
\end{Def}

El siguiente teorema nos permite tener una mejor caracterizaci\'on de los espacios de Rad\'on:
\begin{Teo}\label{Tma.A2.2}
Sea $E$ espacio separable metrizable. Entonces $E$ es Radoniano si y s\'olo s\'i cada medida finita en $\left(E,\mathcal{B}\left(E\right)\right)$ es cerrada.
\end{Teo}

%_________________________________________________________________________________________
\subsection{Propiedades de Markov}
%_________________________________________________________________________________________

Sea $E$ espacio de estados, tal que $E$ es un espacio de Rad\'on, $\mathcal{B}\left(E\right)$ $\sigma$-\'algebra de Borel en $E$, que se denotar\'a por $\mathcal{E}$.

Sea $\left(X,\mathcal{G},\prob\right)$ espacio de probabilidad, $I\subset\rea$ conjunto de índices. Sea $\mathcal{F}_{\leq t}$ la $\sigma$-\'algebra natural definida como $\sigma\left\{f\left(X_{r}\right):r\in I, rleq t,f\in\mathcal{E}\right\}$. Se considerar\'a una $\sigma$-\'algebra m\'as general, $ \left(\mathcal{G}_{t}\right)$ tal que $\left(X_{t}\right)$ sea $\mathcal{E}$-adaptado.

\begin{Def}
Una familia $\left(P_{s,t}\right)$ de kernels de Markov en $\left(E,\mathcal{E}\right)$ indexada por pares $s,t\in I$, con $s\leq t$ es una funci\'on de transici\'on en $\ER$, si  para todo $r\leq s< t$ en $I$ y todo $x\in E$, $B\in\mathcal{E}$
\begin{equation}\label{Eq.Kernels}
P_{r,t}\left(x,B\right)=\int_{E}P_{r,s}\left(x,dy\right)P_{s,t}\left(y,B\right)\footnote{Ecuaci\'on de Chapman-Kolmogorov}.
\end{equation}
\end{Def}

Se dice que la funci\'on de transici\'on $\KM$ en $\ER$ es la funci\'on de transici\'on para un proceso $\PE$  con valores en $E$ y que satisface la propiedad de Markov\footnote{\begin{equation}\label{Eq.1.4.S}
\prob\left\{H|\mathcal{G}_{t}\right\}=\prob\left\{H|X_{t}\right\}\textrm{ }H\in p\mathcal{F}_{\geq t}.
\end{equation}} (\ref{Eq.1.4.S}) relativa a $\left(\mathcal{G}_{t}\right)$ si 

\begin{equation}\label{Eq.1.6.S}
\prob\left\{f\left(X_{t}\right)|\mathcal{G}_{s}\right\}=P_{s,t}f\left(X_{t}\right)\textrm{ }s\leq t\in I,\textrm{ }f\in b\mathcal{E}.
\end{equation}

\begin{Def}
Una familia $\left(P_{t}\right)_{t\geq0}$ de kernels de Markov en $\ER$ es llamada {\em Semigrupo de Transici\'on de Markov} o {\em Semigrupo de Transici\'on} si
\[P_{t+s}f\left(x\right)=P_{t}\left(P_{s}f\right)\left(x\right),\textrm{ }t,s\geq0,\textrm{ }x\in E\textrm{ }f\in b\mathcal{E}.\]
\end{Def}
\begin{Note}
Si la funci\'on de transici\'on $\KM$ es llamada homog\'enea si $P_{s,t}=P_{t-s}$.
\end{Note}

Un proceso de Markov que satisface la ecuaci\'on (\ref{Eq.1.6.S}) con funci\'on de transici\'on homog\'enea $\left(P_{t}\right)$ tiene la propiedad caracter\'istica
\begin{equation}\label{Eq.1.8.S}
\prob\left\{f\left(X_{t+s}\right)|\mathcal{G}_{t}\right\}=P_{s}f\left(X_{t}\right)\textrm{ }t,s\geq0,\textrm{ }f\in b\mathcal{E}.
\end{equation}
La ecuaci\'on anterior es la {\em Propiedad Simple de Markov} de $X$ relativa a $\left(P_{t}\right)$.

En este sentido el proceso $\PE$ cumple con la propiedad de Markov (\ref{Eq.1.8.S}) relativa a $\left(\Omega,\mathcal{G},\mathcal{G}_{t},\prob\right)$ con semigrupo de transici\'on $\left(P_{t}\right)$.
%_________________________________________________________________________________________
\subsection{Primer Condici\'on de Regularidad}
%_________________________________________________________________________________________
%\newcommand{\EM}{\left(\Omega,\mathcal{G},\prob\right)}
%\newcommand{\E4}{\left(\Omega,\mathcal{G},\mathcal{G}_{t},\prob\right)}
\begin{Def}
Un proceso estoc\'astico $\PE$ definido en $\left(\Omega,\mathcal{G},\prob\right)$ con valores en el espacio topol\'ogico $E$ es continuo por la derecha si cada trayectoria muestral $t\rightarrow X_{t}\left(w\right)$ es un mapeo continuo por la derecha de $I$ en $E$.
\end{Def}

\begin{Def}[HD1]\label{Eq.2.1.S}
Un semigrupo de Markov $\left/P_{t}\right)$ en un espacio de Rad\'on $E$ se dice que satisface la condici\'on {\em HD1} si, dada una medida de probabilidad $\mu$ en $E$, existe una $\sigma$-\'algebra $\mathcal{E^{*}}$ con $\mathcal{E}\subset\mathcal{E}$ y $P_{t}\left(b\mathcal{E}^{*}\right)\subset b\mathcal{E}^{*}$, y un $\mathcal{E}^{*}$-proceso $E$-valuado continuo por la derecha $\PE$ en alg\'un espacio de probabilidad filtrado $\left(\Omega,\mathcal{G},\mathcal{G}_{t},\prob\right)$ tal que $X=\left(\Omega,\mathcal{G},\mathcal{G}_{t},\prob\right)$ es de Markov (Homog\'eneo) con semigrupo de transici\'on $(P_{t})$ y distribuci\'on inicial $\mu$.
\end{Def}

Considerese la colecci\'on de variables aleatorias $X_{t}$ definidas en alg\'un espacio de probabilidad, y una colecci\'on de medidas $\mathbf{P}^{x}$ tales que $\mathbf{P}^{x}\left\{X_{0}=x\right\}$, y bajo cualquier $\mathbf{P}^{x}$, $X_{t}$ es de Markov con semigrupo $\left(P_{t}\right)$. $\mathbf{P}^{x}$ puede considerarse como la distribuci\'on condicional de $\mathbf{P}$ dado $X_{0}=x$.

\begin{Def}\label{Def.2.2.S}
Sea $E$ espacio de Rad\'on, $\SG$ semigrupo de Markov en $\ER$. La colecci\'on $\mathbf{X}=\left(\Omega,\mathcal{G},\mathcal{G}_{t},X_{t},\theta_{t},\CM\right)$ es un proceso $\mathcal{E}$-Markov continuo por la derecha simple, con espacio de estados $E$ y semigrupo de transici\'on $\SG$ en caso de que $\mathbf{X}$ satisfaga las siguientes condiciones:
\begin{itemize}
\item[i)] $\left(\Omega,\mathcal{G},\mathcal{G}_{t}\right)$ es un espacio de medida filtrado, y $X_{t}$ es un proceso $E$-valuado continuo por la derecha $\mathcal{E}^{*}$-adaptado a $\left(\mathcal{G}_{t}\right)$;

\item[ii)] $\left(\theta_{t}\right)_{t\geq0}$ es una colecci\'on de operadores {\em shift} para $X$, es decir, mapea $\Omega$ en s\'i mismo satisfaciendo para $t,s\geq0$,

\begin{equation}\label{Eq.Shift}
\theta_{t}\circ\theta_{s}=\theta_{t+s}\textrm{ y }X_{t}\circ\theta_{t}=X_{t+s};
\end{equation}

\item[iii)] Para cualquier $x\in E$,$\CM\left\{X_{0}=x\right\}=1$, y el proceso $\PE$ tiene la propiedad de Markov (\ref{Eq.1.8.S}) con semigrupo de transici\'on $\SG$ relativo a $\left(\Omega,\mathcal{G},\mathcal{G}_{t},\CM\right)$.
\end{itemize}
\end{Def}

\begin{Def}[HD2]\label{Eq.2.2.S}
Para cualquier $\alpha>0$ y cualquier $f\in S^{\alpha}$, el proceso $t\rightarrow f\left(X_{t}\right)$ es continuo por la derecha casi seguramente.
\end{Def}

\begin{Def}\label{Def.PD}
Un sistema $\mathbf{X}=\left(\Omega,\mathcal{G},\mathcal{G}_{t},X_{t},\theta_{t},\CM\right)$ es un proceso derecho en el espacio de Rad\'on $E$ con semigrupo de transici\'on $\SG$ provisto de:
\begin{itemize}
\item[i)] $\mathbf{X}$ es una realizaci\'on  continua por la derecha, \ref{Def.2.2.S}, de $\SG$.

\item[ii)] $\mathbf{X}$ satisface la condicion HD2, \ref{Eq.2.2.S}, relativa a $\mathcal{G}_{t}$.

\item[iii)] $\mathcal{G}_{t}$ es aumentado y continuo por la derecha.
\end{itemize}
\end{Def}



\begin{Lema}[Lema 4.2, Dai\cite{Dai}]\label{Lema4.2}
Sea $\left\{x_{n}\right\}\subset \mathbf{X}$ con
$|x_{n}|\rightarrow\infty$, conforme $n\rightarrow\infty$. Suponga
que
\[lim_{n\rightarrow\infty}\frac{1}{|x_{n}|}U\left(0\right)=\overline{U}\]
y
\[lim_{n\rightarrow\infty}\frac{1}{|x_{n}|}V\left(0\right)=\overline{V}.\]

Entonces, conforme $n\rightarrow\infty$, casi seguramente

\begin{equation}\label{E1.4.2}
\frac{1}{|x_{n}|}\Phi^{k}\left(\left[|x_{n}|t\right]\right)\rightarrow
P_{k}^{'}t\textrm{, u.o.c.,}
\end{equation}

\begin{equation}\label{E1.4.3}
\frac{1}{|x_{n}|}E^{x_{n}}_{k}\left(|x_{n}|t\right)\rightarrow
\alpha_{k}\left(t-\overline{U}_{k}\right)^{+}\textrm{, u.o.c.,}
\end{equation}

\begin{equation}\label{E1.4.4}
\frac{1}{|x_{n}|}S^{x_{n}}_{k}\left(|x_{n}|t\right)\rightarrow
\mu_{k}\left(t-\overline{V}_{k}\right)^{+}\textrm{, u.o.c.,}
\end{equation}

donde $\left[t\right]$ es la parte entera de $t$ y
$\mu_{k}=1/m_{k}=1/\esp\left[\eta_{k}\left(1\right)\right]$.
\end{Lema}

\begin{Lema}[Lema 4.3, Dai\cite{Dai}]\label{Lema.4.3}
Sea $\left\{x_{n}\right\}\subset \mathbf{X}$ con
$|x_{n}|\rightarrow\infty$, conforme $n\rightarrow\infty$. Suponga
que
\[lim_{n\rightarrow\infty}\frac{1}{|x_{n}|}U\left(0\right)=\overline{U}_{k}\]
y
\[lim_{n\rightarrow\infty}\frac{1}{|x_{n}|}V\left(0\right)=\overline{V}_{k}.\]
\begin{itemize}
\item[a)] Conforme $n\rightarrow\infty$ casi seguramente,
\[lim_{n\rightarrow\infty}\frac{1}{|x_{n}|}U^{x_{n}}_{k}\left(|x_{n}|t\right)=\left(\overline{U}_{k}-t\right)^{+}\textrm{, u.o.c.}\]
y
\[lim_{n\rightarrow\infty}\frac{1}{|x_{n}|}V^{x_{n}}_{k}\left(|x_{n}|t\right)=\left(\overline{V}_{k}-t\right)^{+}.\]

\item[b)] Para cada $t\geq0$ fijo,
\[\left\{\frac{1}{|x_{n}|}U^{x_{n}}_{k}\left(|x_{n}|t\right),|x_{n}|\geq1\right\}\]
y
\[\left\{\frac{1}{|x_{n}|}V^{x_{n}}_{k}\left(|x_{n}|t\right),|x_{n}|\geq1\right\}\]
\end{itemize}
son uniformemente convergentes.
\end{Lema}

$S_{l}^{x}\left(t\right)$ es el n\'umero total de servicios
completados de la clase $l$, si la clase $l$ est\'a dando $t$
unidades de tiempo de servicio. Sea $T_{l}^{x}\left(x\right)$ el
monto acumulado del tiempo de servicio que el servidor
$s\left(l\right)$ gasta en los usuarios de la clase $l$ al tiempo
$t$. Entonces $S_{l}^{x}\left(T_{l}^{x}\left(t\right)\right)$ es
el n\'umero total de servicios completados para la clase $l$ al
tiempo $t$. Una fracci\'on de estos usuarios,
$\Phi_{l}^{x}\left(S_{l}^{x}\left(T_{l}^{x}\left(t\right)\right)\right)$,
se convierte en usuarios de la clase $k$.\\

Entonces, dado lo anterior, se tiene la siguiente representaci\'on
para el proceso de la longitud de la cola:\\

\begin{equation}
Q_{k}^{x}\left(t\right)=_{k}^{x}\left(0\right)+E_{k}^{x}\left(t\right)+\sum_{l=1}^{K}\Phi_{k}^{l}\left(S_{l}^{x}\left(T_{l}^{x}\left(t\right)\right)\right)-S_{k}^{x}\left(T_{k}^{x}\left(t\right)\right)
\end{equation}
para $k=1,\ldots,K$. Para $i=1,\ldots,d$, sea
\[I_{i}^{x}\left(t\right)=t-\sum_{j\in C_{i}}T_{k}^{x}\left(t\right).\]

Entonces $I_{i}^{x}\left(t\right)$ es el monto acumulado del
tiempo que el servidor $i$ ha estado desocupado al tiempo $t$. Se
est\'a asumiendo que las disciplinas satisfacen la ley de
conservaci\'on del trabajo, es decir, el servidor $i$ est\'a en
pausa solamente cuando no hay usuarios en la estaci\'on $i$.
Entonces, se tiene que

\begin{equation}
\int_{0}^{\infty}\left(\sum_{k\in
C_{i}}Q_{k}^{x}\left(t\right)\right)dI_{i}^{x}\left(t\right)=0,
\end{equation}
para $i=1,\ldots,d$.\\

Hacer
\[T^{x}\left(t\right)=\left(T_{1}^{x}\left(t\right),\ldots,T_{K}^{x}\left(t\right)\right)^{'},\]
\[I^{x}\left(t\right)=\left(I_{1}^{x}\left(t\right),\ldots,I_{K}^{x}\left(t\right)\right)^{'}\]
y
\[S^{x}\left(T^{x}\left(t\right)\right)=\left(S_{1}^{x}\left(T_{1}^{x}\left(t\right)\right),\ldots,S_{K}^{x}\left(T_{K}^{x}\left(t\right)\right)\right)^{'}.\]

Para una disciplina que cumple con la ley de conservaci\'on del
trabajo, en forma vectorial, se tiene el siguiente conjunto de
ecuaciones

\begin{equation}\label{Eq.MF.1.3}
Q^{x}\left(t\right)=Q^{x}\left(0\right)+E^{x}\left(t\right)+\sum_{l=1}^{K}\Phi^{l}\left(S_{l}^{x}\left(T_{l}^{x}\left(t\right)\right)\right)-S^{x}\left(T^{x}\left(t\right)\right),\\
\end{equation}

\begin{equation}\label{Eq.MF.2.3}
Q^{x}\left(t\right)\geq0,\\
\end{equation}

\begin{equation}\label{Eq.MF.3.3}
T^{x}\left(0\right)=0,\textrm{ y }\overline{T}^{x}\left(t\right)\textrm{ es no decreciente},\\
\end{equation}

\begin{equation}\label{Eq.MF.4.3}
I^{x}\left(t\right)=et-CT^{x}\left(t\right)\textrm{ es no
decreciente}\\
\end{equation}

\begin{equation}\label{Eq.MF.5.3}
\int_{0}^{\infty}\left(CQ^{x}\left(t\right)\right)dI_{i}^{x}\left(t\right)=0,\\
\end{equation}

\begin{equation}\label{Eq.MF.6.3}
\textrm{Condiciones adicionales en
}\left(\overline{Q}^{x}\left(\cdot\right),\overline{T}^{x}\left(\cdot\right)\right)\textrm{
espec\'ificas de la disciplina de la cola,}
\end{equation}

donde $e$ es un vector de unos de dimensi\'on $d$, $C$ es la
matriz definida por
\[C_{ik}=\left\{\begin{array}{cc}
1,& S\left(k\right)=i,\\
0,& \textrm{ en otro caso}.\\
\end{array}\right.
\]
Es necesario enunciar el siguiente Teorema que se utilizar\'a para
el Teorema \ref{Tma.4.2.Dai}:
\begin{Teo}[Teorema 4.1, Dai \cite{Dai}]
Considere una disciplina que cumpla la ley de conservaci\'on del
trabajo, para casi todas las trayectorias muestrales $\omega$ y
cualquier sucesi\'on de estados iniciales
$\left\{x_{n}\right\}\subset \mathbf{X}$, con
$|x_{n}|\rightarrow\infty$, existe una subsucesi\'on
$\left\{x_{n_{j}}\right\}$ con $|x_{n_{j}}|\rightarrow\infty$ tal
que
\begin{equation}\label{Eq.4.15}
\frac{1}{|x_{n_{j}}|}\left(Q^{x_{n_{j}}}\left(0\right),U^{x_{n_{j}}}\left(0\right),V^{x_{n_{j}}}\left(0\right)\right)\rightarrow\left(\overline{Q}\left(0\right),\overline{U},\overline{V}\right),
\end{equation}

\begin{equation}\label{Eq.4.16}
\frac{1}{|x_{n_{j}}|}\left(Q^{x_{n_{j}}}\left(|x_{n_{j}}|t\right),T^{x_{n_{j}}}\left(|x_{n_{j}}|t\right)\right)\rightarrow\left(\overline{Q}\left(t\right),\overline{T}\left(t\right)\right)\textrm{
u.o.c.}
\end{equation}

Adem\'as,
$\left(\overline{Q}\left(t\right),\overline{T}\left(t\right)\right)$
satisface las siguientes ecuaciones:
\begin{equation}\label{Eq.MF.1.3a}
\overline{Q}\left(t\right)=Q\left(0\right)+\left(\alpha
t-\overline{U}\right)^{+}-\left(I-P\right)^{'}M^{-1}\left(\overline{T}\left(t\right)-\overline{V}\right)^{+},
\end{equation}

\begin{equation}\label{Eq.MF.2.3a}
\overline{Q}\left(t\right)\geq0,\\
\end{equation}

\begin{equation}\label{Eq.MF.3.3a}
\overline{T}\left(t\right)\textrm{ es no decreciente y comienza en cero},\\
\end{equation}

\begin{equation}\label{Eq.MF.4.3a}
\overline{I}\left(t\right)=et-C\overline{T}\left(t\right)\textrm{
es no decreciente,}\\
\end{equation}

\begin{equation}\label{Eq.MF.5.3a}
\int_{0}^{\infty}\left(C\overline{Q}\left(t\right)\right)d\overline{I}\left(t\right)=0,\\
\end{equation}

\begin{equation}\label{Eq.MF.6.3a}
\textrm{Condiciones adicionales en
}\left(\overline{Q}\left(\cdot\right),\overline{T}\left(\cdot\right)\right)\textrm{
especficas de la disciplina de la cola,}
\end{equation}
\end{Teo}

\begin{Def}[Definici\'on 4.1, , Dai \cite{Dai}]
Sea una disciplina de servicio espec\'ifica. Cualquier l\'imite
$\left(\overline{Q}\left(\cdot\right),\overline{T}\left(\cdot\right)\right)$
en \ref{Eq.4.16} es un {\em flujo l\'imite} de la disciplina.
Cualquier soluci\'on (\ref{Eq.MF.1.3a})-(\ref{Eq.MF.6.3a}) es
llamado flujo soluci\'on de la disciplina. Se dice que el modelo de flujo l\'imite, modelo de flujo, de la disciplina de la cola es estable si existe una constante
$\delta>0$ que depende de $\mu,\alpha$ y $P$ solamente, tal que
cualquier flujo l\'imite con
$|\overline{Q}\left(0\right)|+|\overline{U}|+|\overline{V}|=1$, se
tiene que $\overline{Q}\left(\cdot+\delta\right)\equiv0$.
\end{Def}

\begin{Teo}[Teorema 4.2, Dai\cite{Dai}]\label{Tma.4.2.Dai}
Sea una disciplina fija para la cola, suponga que se cumplen las
condiciones (1.2)-(1.5). Si el modelo de flujo l\'imite de la
disciplina de la cola es estable, entonces la cadena de Markov $X$
que describe la din\'amica de la red bajo la disciplina es Harris
recurrente positiva.
\end{Teo}

Ahora se procede a escalar el espacio y el tiempo para reducir la
aparente fluctuaci\'on del modelo. Consid\'erese el proceso
\begin{equation}\label{Eq.3.7}
\overline{Q}^{x}\left(t\right)=\frac{1}{|x|}Q^{x}\left(|x|t\right)
\end{equation}
A este proceso se le conoce como el fluido escalado, y cualquier l\'imite $\overline{Q}^{x}\left(t\right)$ es llamado flujo l\'imite del proceso de longitud de la cola. Haciendo $|q|\rightarrow\infty$ mientras se mantiene el resto de las componentes fijas, cualquier punto l\'imite del proceso de longitud de la cola normalizado $\overline{Q}^{x}$ es soluci\'on del siguiente modelo de flujo.

Al conjunto de ecuaciones dadas en \ref{Eq.3.8}-\ref{Eq.3.13} se
le llama {\em Modelo de flujo} y al conjunto de todas las
soluciones del modelo de flujo
$\left(\overline{Q}\left(\cdot\right),\overline{T}
\left(\cdot\right)\right)$ se le denotar\'a por $\mathcal{Q}$.

Si se hace $|x|\rightarrow\infty$ sin restringir ninguna de las
componentes, tambi\'en se obtienen un modelo de flujo, pero en
este caso el residual de los procesos de arribo y servicio
introducen un retraso:

\begin{Def}[Definici\'on 3.3, Dai y Meyn \cite{DaiSean}]
El modelo de flujo es estable si existe un tiempo fijo $t_{0}$ tal
que $\overline{Q}\left(t\right)=0$, con $t\geq t_{0}$, para
cualquier $\overline{Q}\left(\cdot\right)\in\mathcal{Q}$ que
cumple con $|\overline{Q}\left(0\right)|=1$.
\end{Def}

El siguiente resultado se encuentra en Chen \cite{Chen}.
\begin{Lemma}[Lema 3.1, Dai y Meyn \cite{DaiSean}]
Si el modelo de flujo definido por \ref{Eq.3.8}-\ref{Eq.3.13} es
estable, entonces el modelo de flujo retrasado es tambi\'en
estable, es decir, existe $t_{0}>0$ tal que
$\overline{Q}\left(t\right)=0$ para cualquier $t\geq t_{0}$, para
cualquier soluci\'on del modelo de flujo retrasado cuya
condici\'on inicial $\overline{x}$ satisface que
$|\overline{x}|=|\overline{Q}\left(0\right)|+|\overline{A}\left(0\right)|+|\overline{B}\left(0\right)|\leq1$.
\end{Lemma}


Propiedades importantes para el modelo de flujo retrasado:

\begin{Prop}
 Sea $\left(\overline{Q},\overline{T},\overline{T}^{0}\right)$ un flujo l\'imite de \ref{Eq.4.4} y suponga que cuando $x\rightarrow\infty$ a lo largo de
una subsucesi\'on
\[\left(\frac{1}{|x|}Q_{k}^{x}\left(0\right),\frac{1}{|x|}A_{k}^{x}\left(0\right),\frac{1}{|x|}B_{k}^{x}\left(0\right),\frac{1}{|x|}B_{k}^{x,0}\left(0\right)\right)\rightarrow\left(\overline{Q}_{k}\left(0\right),0,0,0\right)\]
para $k=1,\ldots,K$. EL flujo l\'imite tiene las siguientes
propiedades, donde las propiedades de la derivada se cumplen donde
la derivada exista:
\begin{itemize}
 \item[i)] Los vectores de tiempo ocupado $\overline{T}\left(t\right)$ y $\overline{T}^{0}\left(t\right)$ son crecientes y continuas con
$\overline{T}\left(0\right)=\overline{T}^{0}\left(0\right)=0$.
\item[ii)] Para todo $t\geq0$
\[\sum_{k=1}^{K}\left[\overline{T}_{k}\left(t\right)+\overline{T}_{k}^{0}\left(t\right)\right]=t\]
\item[iii)] Para todo $1\leq k\leq K$
\[\overline{Q}_{k}\left(t\right)=\overline{Q}_{k}\left(0\right)+\alpha_{k}t-\mu_{k}\overline{T}_{k}\left(t\right)\]
\item[iv)]  Para todo $1\leq k\leq K$
\[\dot{{\overline{T}}}_{k}\left(t\right)=\beta_{k}\] para $\overline{Q}_{k}\left(t\right)=0$.
\item[v)] Para todo $k,j$
\[\mu_{k}^{0}\overline{T}_{k}^{0}\left(t\right)=\mu_{j}^{0}\overline{T}_{j}^{0}\left(t\right)\]
\item[vi)]  Para todo $1\leq k\leq K$
\[\mu_{k}\dot{{\overline{T}}}_{k}\left(t\right)=l_{k}\mu_{k}^{0}\dot{{\overline{T}}}_{k}^{0}\left(t\right)\] para $\overline{Q}_{k}\left(t\right)>0$.
\end{itemize}
\end{Prop}

\begin{Lema}[Lema 3.1 \cite{Chen}]\label{Lema3.1}
Si el modelo de flujo es estable, definido por las ecuaciones
(3.8)-(3.13), entonces el modelo de flujo retrasado tambin es
estable.
\end{Lema}

\begin{Teo}[Teorema 5.2 \cite{Chen}]\label{Tma.5.2}
Si el modelo de flujo lineal correspondiente a la red de cola es
estable, entonces la red de colas es estable.
\end{Teo}

\begin{Teo}[Teorema 5.1 \cite{Chen}]\label{Tma.5.1.Chen}
La red de colas es estable si existe una constante $t_{0}$ que
depende de $\left(\alpha,\mu,T,U\right)$ y $V$ que satisfagan las
ecuaciones (5.1)-(5.5), $Z\left(t\right)=0$, para toda $t\geq
t_{0}$.
\end{Teo}



\begin{Lema}[Lema 5.2 \cite{Gut}]\label{Lema.5.2.Gut}
Sea $\left\{\xi\left(k\right):k\in\ent\right\}$ sucesin de
variables aleatorias i.i.d. con valores en
$\left(0,\infty\right)$, y sea $E\left(t\right)$ el proceso de
conteo
\[E\left(t\right)=max\left\{n\geq1:\xi\left(1\right)+\cdots+\xi\left(n-1\right)\leq t\right\}.\]
Si $E\left[\xi\left(1\right)\right]<\infty$, entonces para
cualquier entero $r\geq1$
\begin{equation}
lim_{t\rightarrow\infty}\esp\left[\left(\frac{E\left(t\right)}{t}\right)^{r}\right]=\left(\frac{1}{E\left[\xi_{1}\right]}\right)^{r}
\end{equation}
de aqu, bajo estas condiciones
\begin{itemize}
\item[a)] Para cualquier $t>0$,
$sup_{t\geq\delta}\esp\left[\left(\frac{E\left(t\right)}{t}\right)^{r}\right]$

\item[b)] Las variables aleatorias
$\left\{\left(\frac{E\left(t\right)}{t}\right)^{r}:t\geq1\right\}$
son uniformemente integrables.
\end{itemize}
\end{Lema}

\begin{Teo}[Teorema 5.1: Ley Fuerte para Procesos de Conteo
\cite{Gut}]\label{Tma.5.1.Gut} Sea
$0<\mu<\esp\left(X_{1}\right]\leq\infty$. entonces

\begin{itemize}
\item[a)] $\frac{N\left(t\right)}{t}\rightarrow\frac{1}{\mu}$
a.s., cuando $t\rightarrow\infty$.


\item[b)]$\esp\left[\frac{N\left(t\right)}{t}\right]^{r}\rightarrow\frac{1}{\mu^{r}}$,
cuando $t\rightarrow\infty$ para todo $r>0$..
\end{itemize}
\end{Teo}


\begin{Prop}[Proposicin 5.1 \cite{DaiSean}]\label{Prop.5.1}
Suponga que los supuestos (A1) y (A2) se cumplen, adems suponga
que el modelo de flujo es estable. Entonces existe $t_{0}>0$ tal
que
\begin{equation}\label{Eq.Prop.5.1}
lim_{|x|\rightarrow\infty}\frac{1}{|x|^{p+1}}\esp_{x}\left[|X\left(t_{0}|x|\right)|^{p+1}\right]=0.
\end{equation}

\end{Prop}


\begin{Prop}[Proposici\'on 5.3 \cite{DaiSean}]
Sea $X$ proceso de estados para la red de colas, y suponga que se
cumplen los supuestos (A1) y (A2), entonces para alguna constante
positiva $C_{p+1}<\infty$, $\delta>0$ y un conjunto compacto
$C\subset X$.

\begin{equation}\label{Eq.5.4}
\esp_{x}\left[\int_{0}^{\tau_{C}\left(\delta\right)}\left(1+|X\left(t\right)|^{p}\right)dt\right]\leq
C_{p+1}\left(1+|x|^{p+1}\right)
\end{equation}
\end{Prop}

\begin{Prop}[Proposici\'on 5.4 \cite{DaiSean}]
Sea $X$ un proceso de Markov Borel Derecho en $X$, sea
$f:X\leftarrow\rea_{+}$ y defina para alguna $\delta>0$, y un
conjunto cerrado $C\subset X$
\[V\left(x\right):=\esp_{x}\left[\int_{0}^{\tau_{C}\left(\delta\right)}f\left(X\left(t\right)\right)dt\right]\]
para $x\in X$. Si $V$ es finito en todas partes y uniformemente
acotada en $C$, entonces existe $k<\infty$ tal que
\begin{equation}\label{Eq.5.11}
\frac{1}{t}\esp_{x}\left[V\left(x\right)\right]+\frac{1}{t}\int_{0}^{t}\esp_{x}\left[f\left(X\left(s\right)\right)ds\right]\leq\frac{1}{t}V\left(x\right)+k,
\end{equation}
para $x\in X$ y $t>0$.
\end{Prop}


\begin{Teo}[Teorema 5.5 \cite{DaiSean}]
Suponga que se cumplen (A1) y (A2), adems suponga que el modelo
de flujo es estable. Entonces existe una constante $k_{p}<\infty$
tal que
\begin{equation}\label{Eq.5.13}
\frac{1}{t}\int_{0}^{t}\esp_{x}\left[|Q\left(s\right)|^{p}\right]ds\leq
k_{p}\left\{\frac{1}{t}|x|^{p+1}+1\right\}
\end{equation}
para $t\geq0$, $x\in X$. En particular para cada condici\'on inicial
\begin{equation}\label{Eq.5.14}
Limsup_{t\rightarrow\infty}\frac{1}{t}\int_{0}^{t}\esp_{x}\left[|Q\left(s\right)|^{p}\right]ds\leq
k_{p}
\end{equation}
\end{Teo}

\begin{Teo}[Teorema 6.2\cite{DaiSean}]\label{Tma.6.2}
Suponga que se cumplen los supuestos (A1)-(A3) y que el modelo de
flujo es estable, entonces se tiene que
\[\parallel P^{t}\left(c,\cdot\right)-\pi\left(\cdot\right)\parallel_{f_{p}}\rightarrow0\]
para $t\rightarrow\infty$ y $x\in X$. En particular para cada
condicin inicial
\[lim_{t\rightarrow\infty}\esp_{x}\left[\left|Q_{t}\right|^{p}\right]=\esp_{\pi}\left[\left|Q_{0}\right|^{p}\right]<\infty\]
\end{Teo}


\begin{Teo}[Teorema 6.3\cite{DaiSean}]\label{Tma.6.3}
Suponga que se cumplen los supuestos (A1)-(A3) y que el modelo de
flujo es estable, entonces con
$f\left(x\right)=f_{1}\left(x\right)$, se tiene que
\[lim_{t\rightarrow\infty}t^{(p-1)\left|P^{t}\left(c,\cdot\right)-\pi\left(\cdot\right)\right|_{f}=0},\]
para $x\in X$. En particular, para cada condicin inicial
\[lim_{t\rightarrow\infty}t^{(p-1)\left|\esp_{x}\left[Q_{t}\right]-\esp_{\pi}\left[Q_{0}\right]\right|=0}.\]
\end{Teo}



\begin{Prop}[Proposici\'on 5.1, Dai y Meyn \cite{DaiSean}]\label{Prop.5.1.DaiSean}
Suponga que los supuestos A1) y A2) son ciertos y que el modelo de flujo es estable. Entonces existe $t_{0}>0$ tal que
\begin{equation}
lim_{|x|\rightarrow\infty}\frac{1}{|x|^{p+1}}\esp_{x}\left[|X\left(t_{0}|x|\right)|^{p+1}\right]=0
\end{equation}
\end{Prop}

\begin{Lemma}[Lema 5.2, Dai y Meyn \cite{DaiSean}]\label{Lema.5.2.DaiSean}
 Sea $\left\{\zeta\left(k\right):k\in \mathbb{z}\right\}$ una sucesi\'on independiente e id\'enticamente distribuida que toma valores en $\left(0,\infty\right)$,
y sea
$E\left(t\right)=max\left(n\geq1:\zeta\left(1\right)+\cdots+\zeta\left(n-1\right)\leq
t\right)$. Si $\esp\left[\zeta\left(1\right)\right]<\infty$,
entonces para cualquier entero $r\geq1$
\begin{equation}
 lim_{t\rightarrow\infty}\esp\left[\left(\frac{E\left(t\right)}{t}\right)^{r}\right]=\left(\frac{1}{\esp\left[\zeta_{1}\right]}\right)^{r}.
\end{equation}
Luego, bajo estas condiciones:
\begin{itemize}
 \item[a)] para cualquier $\delta>0$, $\sup_{t\geq\delta}\esp\left[\left(\frac{E\left(t\right)}{t}\right)^{r}\right]<\infty$
\item[b)] las variables aleatorias
$\left\{\left(\frac{E\left(t\right)}{t}\right)^{r}:t\geq1\right\}$
son uniformemente integrables.
\end{itemize}
\end{Lemma}

\begin{Teo}[Teorema 5.5, Dai y Meyn \cite{DaiSean}]\label{Tma.5.5.DaiSean}
Suponga que los supuestos A1) y A2) se cumplen y que el modelo de
flujo es estable. Entonces existe una constante $\kappa_{p}$ tal
que
\begin{equation}
\frac{1}{t}\int_{0}^{t}\esp_{x}\left[|Q\left(s\right)|^{p}\right]ds\leq\kappa_{p}\left\{\frac{1}{t}|x|^{p+1}+1\right\}
\end{equation}
para $t>0$ y $x\in X$. En particular, para cada condici\'on
inicial
\begin{eqnarray*}
\limsup_{t\rightarrow\infty}\frac{1}{t}\int_{0}^{t}\esp_{x}\left[|Q\left(s\right)|^{p}\right]ds\leq\kappa_{p}.
\end{eqnarray*}
\end{Teo}

\begin{Teo}[Teorema 6.2, Dai y Meyn \cite{DaiSean}]\label{Tma.6.2.DaiSean}
Suponga que se cumplen los supuestos A1), A2) y A3) y que el
modelo de flujo es estable. Entonces se tiene que
\begin{equation}
\left\|P^{t}\left(x,\cdot\right)-\pi\left(\cdot\right)\right\|_{f_{p}}\textrm{,
}t\rightarrow\infty,x\in X.
\end{equation}
En particular para cada condici\'on inicial
\begin{eqnarray*}
\lim_{t\rightarrow\infty}\esp_{x}\left[|Q\left(t\right)|^{p}\right]=\esp_{\pi}\left[|Q\left(0\right)|^{p}\right]\leq\kappa_{r}
\end{eqnarray*}
\end{Teo}
\begin{Teo}[Teorema 6.3, Dai y Meyn \cite{DaiSean}]\label{Tma.6.3.DaiSean}
Suponga que se cumplen los supuestos A1), A2) y A3) y que el
modelo de flujo es estable. Entonces con
$f\left(x\right)=f_{1}\left(x\right)$ se tiene
\begin{equation}
\lim_{t\rightarrow\infty}t^{p-1}\left\|P^{t}\left(x,\cdot\right)-\pi\left(\cdot\right)\right\|_{f}=0.
\end{equation}
En particular para cada condici\'on inicial
\begin{eqnarray*}
\lim_{t\rightarrow\infty}t^{p-1}|\esp_{x}\left[Q\left(t\right)\right]-\esp_{\pi}\left[Q\left(0\right)\right]|=0.
\end{eqnarray*}
\end{Teo}

\begin{Teo}[Teorema 6.4, Dai y Meyn \cite{DaiSean}]\label{Tma.6.4.DaiSean}
Suponga que se cumplen los supuestos A1), A2) y A3) y que el
modelo de flujo es estable. Sea $\nu$ cualquier distribuci\'on de
probabilidad en $\left(X,\mathcal{B}_{X}\right)$, y $\pi$ la
distribuci\'on estacionaria de $X$.
\begin{itemize}
\item[i)] Para cualquier $f:X\leftarrow\rea_{+}$
\begin{equation}
\lim_{t\rightarrow\infty}\frac{1}{t}\int_{o}^{t}f\left(X\left(s\right)\right)ds=\pi\left(f\right):=\int
f\left(x\right)\pi\left(dx\right)
\end{equation}
$\prob$-c.s.

\item[ii)] Para cualquier $f:X\leftarrow\rea_{+}$ con
$\pi\left(|f|\right)<\infty$, la ecuaci\'on anterior se cumple.
\end{itemize}
\end{Teo}

\begin{Teo}[Teorema 2.2, Down \cite{Down}]\label{Tma2.2.Down}
Suponga que el fluido modelo es inestable en el sentido de que
para alguna $\epsilon_{0},c_{0}\geq0$,
\begin{equation}\label{Eq.Inestability}
|Q\left(T\right)|\geq\epsilon_{0}T-c_{0}\textrm{,   }T\geq0,
\end{equation}
para cualquier condici\'on inicial $Q\left(0\right)$, con
$|Q\left(0\right)|=1$. Entonces para cualquier $0<q\leq1$, existe
$B<0$ tal que para cualquier $|x|\geq B$,
\begin{equation}
\prob_{x}\left\{\mathbb{X}\rightarrow\infty\right\}\geq q.
\end{equation}
\end{Teo}


%_________________________________________________________________________
\subsection{Supuestos}
%_________________________________________________________________________
Consideremos el caso en el que se tienen varias colas a las cuales
llegan uno o varios servidores para dar servicio a los usuarios
que se encuentran presentes en la cola, como ya se mencion\'o hay
varios tipos de pol\'iticas de servicio, incluso podr\'ia ocurrir
que la manera en que atiende al resto de las colas sea distinta a
como lo hizo en las anteriores.\\

Para ejemplificar los sistemas de visitas c\'iclicas se
considerar\'a el caso en que a las colas los usuarios son atendidos con
una s\'ola pol\'itica de servicio.\\


Si $\omega$ es el n\'umero de usuarios en la cola al comienzo del
periodo de servicio y $N\left(\omega\right)$ es el n\'umero de
usuarios que son atendidos con una pol\'itica en espec\'ifico
durante el periodo de servicio, entonces se asume que:
\begin{itemize}
\item[1)]\label{S1}$lim_{\omega\rightarrow\infty}\esp\left[N\left(\omega\right)\right]=\overline{N}>0$;
\item[2)]\label{S2}$\esp\left[N\left(\omega\right)\right]\leq\overline{N}$
para cualquier valor de $\omega$.
\end{itemize}
La manera en que atiende el servidor $m$-\'esimo, es la siguiente:
\begin{itemize}
\item Al t\'ermino de la visita a la cola $j$, el servidor cambia
a la cola $j^{'}$ con probabilidad $r_{j,j^{'}}^{m}$

\item La $n$-\'esima vez que el servidor cambia de la cola $j$ a
$j'$, va acompa\~nada con el tiempo de cambio de longitud
$\delta_{j,j^{'}}^{m}\left(n\right)$, con
$\delta_{j,j^{'}}^{m}\left(n\right)$, $n\geq1$, variables
aleatorias independientes e id\'enticamente distribuidas, tales
que $\esp\left[\delta_{j,j^{'}}^{m}\left(1\right)\right]\geq0$.

\item Sea $\left\{p_{j}^{m}\right\}$ la distribuci\'on invariante
estacionaria \'unica para la Cadena de Markov con matriz de
transici\'on $\left(r_{j,j^{'}}^{m}\right)$, se supone que \'esta
existe.

\item Finalmente, se define el tiempo promedio total de traslado
entre las colas como
\begin{equation}
\delta^{*}:=\sum_{j,j^{'}}p_{j}^{m}r_{j,j^{'}}^{m}\esp\left[\delta_{j,j^{'}}^{m}\left(i\right)\right].
\end{equation}
\end{itemize}

Consideremos el caso donde los tiempos entre arribo a cada una de
las colas, $\left\{\xi_{k}\left(n\right)\right\}_{n\geq1}$ son
variables aleatorias independientes a id\'enticamente
distribuidas, y los tiempos de servicio en cada una de las colas
se distribuyen de manera independiente e id\'enticamente
distribuidas $\left\{\eta_{k}\left(n\right)\right\}_{n\geq1}$;
adem\'as ambos procesos cumplen la condici\'on de ser
independientes entre s\'i. Para la $k$-\'esima cola se define la
tasa de arribo por
$\lambda_{k}=1/\esp\left[\xi_{k}\left(1\right)\right]$ y la tasa
de servicio como
$\mu_{k}=1/\esp\left[\eta_{k}\left(1\right)\right]$, finalmente se
define la carga de la cola como $\rho_{k}=\lambda_{k}/\mu_{k}$,
donde se pide que $\rho=\sum_{k=1}^{K}\rho_{k}<1$, para garantizar
la estabilidad del sistema, esto es cierto para las pol\'iticas de
servicio exhaustiva y cerrada, ver Geetor \cite{Getoor}.\\

Si denotamos por
\begin{itemize}
\item $Q_{k}\left(t\right)$ el n\'umero de usuarios presentes en
la cola $k$ al tiempo $t$; \item $A_{k}\left(t\right)$ los
residuales de los tiempos entre arribos a la cola $k$; para cada
servidor $m$; \item $B_{m}\left(t\right)$ denota a los residuales
de los tiempos de servicio al tiempo $t$; \item
$B_{m}^{0}\left(t\right)$ los residuales de los tiempos de
traslado de la cola $k$ a la pr\'oxima por atender al tiempo $t$,

\item sea
$C_{m}\left(t\right)$ el n\'umero de usuarios atendidos durante la
visita del servidor a la cola $k$ al tiempo $t$.
\end{itemize}


En este sentido, el proceso para el sistema de visitas se puede
definir como:

\begin{equation}\label{Esp.Edos.Down}
X\left(t\right)^{T}=\left(Q_{k}\left(t\right),A_{k}\left(t\right),B_{m}\left(t\right),B_{m}^{0}\left(t\right),C_{m}\left(t\right)\right),
\end{equation}
para $k=1,\ldots,K$ y $m=1,2,\ldots,M$, donde $T$ indica que es el
transpuesto del vector que se est\'a definiendo. El proceso $X$
evoluciona en el espacio de estados:
$\mathbb{X}=\ent_{+}^{K}\times\rea_{+}^{K}\times\left(\left\{1,2,\ldots,K\right\}\times\left\{1,2,\ldots,S\right\}\right)^{M}\times\rea_{+}^{K}\times\ent_{+}^{K}$.\\

El sistema aqu\'i descrito debe de cumplir con los siguientes supuestos b\'asicos de un sistema de visitas:
%__________________________________________________________________________
\subsubsection{Supuestos B\'asicos}
%__________________________________________________________________________
\begin{itemize}
\item[A1)] Los procesos
$\xi_{1},\ldots,\xi_{K},\eta_{1},\ldots,\eta_{K}$ son mutuamente
independientes y son sucesiones independientes e id\'enticamente
distribuidas.

\item[A2)] Para alg\'un entero $p\geq1$
\begin{eqnarray*}
\esp\left[\xi_{l}\left(1\right)^{p+1}\right]&<&\infty\textrm{ para }l=1,\ldots,K\textrm{ y }\\
\esp\left[\eta_{k}\left(1\right)^{p+1}\right]&<&\infty\textrm{
para }k=1,\ldots,K.
\end{eqnarray*}
donde $\mathcal{A}$ es la clase de posibles arribos.

\item[A3)] Para cada $k=1,2,\ldots,K$ existe una funci\'on
positiva $q_{k}\left(\cdot\right)$ definida en $\rea_{+}$, y un
entero $j_{k}$, tal que
\begin{eqnarray}
P\left(\xi_{k}\left(1\right)\geq x\right)&>&0\textrm{, para todo }x>0,\\
P\left\{a\leq\sum_{i=1}^{j_{k}}\xi_{k}\left(i\right)\leq
b\right\}&\geq&\int_{a}^{b}q_{k}\left(x\right)dx, \textrm{ }0\leq
a<b.
\end{eqnarray}
\end{itemize}

En lo que respecta al supuesto (A3), en Dai y Meyn \cite{DaiSean}
hacen ver que este se puede sustituir por

\begin{itemize}
\item[A3')] Para el Proceso de Markov $X$, cada subconjunto
compacto del espacio de estados de $X$ es un conjunto peque\~no,
ver definici\'on \ref{Def.Cto.Peq.}.
\end{itemize}

Es por esta raz\'on que con la finalidad de poder hacer uso de
$A3^{'})$ es necesario recurrir a los Procesos de Harris y en
particular a los Procesos Harris Recurrente, ver \cite{Dai,
DaiSean}.
%_______________________________________________________________________
\subsection{Procesos Harris Recurrente}
%_______________________________________________________________________

Por el supuesto (A1) conforme a Davis \cite{Davis}, se puede
definir el proceso de saltos correspondiente de manera tal que
satisfaga el supuesto (A3'), de hecho la demostraci\'on est\'a
basada en la l\'inea de argumentaci\'on de Davis, \cite{Davis},
p\'aginas 362-364.\\

Entonces se tiene un espacio de estados en el cual el proceso $X$
satisface la Propiedad Fuerte de Markov, ver Dai y Meyn
\cite{DaiSean}, dado por

\[\left(\Omega,\mathcal{F},\mathcal{F}_{t},X\left(t\right),\theta_{t},P_{x}\right),\]
adem\'as de ser un proceso de Borel Derecho (Sharpe \cite{Sharpe})
en el espacio de estados medible
$\left(\mathbb{X},\mathcal{B}_\mathbb{X}\right)$. El Proceso
$X=\left\{X\left(t\right),t\geq0\right\}$ tiene trayectorias
continuas por la derecha, est\'a definido en
$\left(\Omega,\mathcal{F}\right)$ y est\'a adaptado a
$\left\{\mathcal{F}_{t},t\geq0\right\}$; la colecci\'on
$\left\{P_{x},x\in \mathbb{X}\right\}$ son medidas de probabilidad
en $\left(\Omega,\mathcal{F}\right)$ tales que para todo $x\in
\mathbb{X}$
\[P_{x}\left\{X\left(0\right)=x\right\}=1,\] y
\[E_{x}\left\{f\left(X\circ\theta_{t}\right)|\mathcal{F}_{t}\right\}=E_{X}\left(\tau\right)f\left(X\right),\]
en $\left\{\tau<\infty\right\}$, $P_{x}$-c.s., con $\theta_{t}$
definido como el operador shift.


Donde $\tau$ es un $\mathcal{F}_{t}$-tiempo de paro
\[\left(X\circ\theta_{\tau}\right)\left(w\right)=\left\{X\left(\tau\left(w\right)+t,w\right),t\geq0\right\},\]
y $f$ es una funci\'on de valores reales acotada y medible, ver \cite{Dai, KaspiMandelbaum}.\\

Sea $P^{t}\left(x,D\right)$, $D\in\mathcal{B}_{\mathbb{X}}$,
$t\geq0$ la probabilidad de transici\'on de $X$ queda definida
como:
\[P^{t}\left(x,D\right)=P_{x}\left(X\left(t\right)\in
D\right).\]


\begin{Def}
Una medida no cero $\pi$ en
$\left(\mathbb{X},\mathcal{B}_{\mathbb{X}}\right)$ es invariante
para $X$ si $\pi$ es $\sigma$-finita y
\[\pi\left(D\right)=\int_{\mathbb{X}}P^{t}\left(x,D\right)\pi\left(dx\right),\]
para todo $D\in \mathcal{B}_{\mathbb{X}}$, con $t\geq0$.
\end{Def}

\begin{Def}
El proceso de Markov $X$ es llamado Harris Recurrente si existe
una medida de probabilidad $\nu$ en
$\left(\mathbb{X},\mathcal{B}_{\mathbb{X}}\right)$, tal que si
$\nu\left(D\right)>0$ y $D\in\mathcal{B}_{\mathbb{X}}$
\[P_{x}\left\{\tau_{D}<\infty\right\}\equiv1,\] cuando
$\tau_{D}=inf\left\{t\geq0:X_{t}\in D\right\}$.
\end{Def}

\begin{Note}
\begin{itemize}
\item[i)] Si $X$ es Harris recurrente, entonces existe una \'unica
medida invariante $\pi$ (Getoor \cite{Getoor}).

\item[ii)] Si la medida invariante es finita, entonces puede
normalizarse a una medida de probabilidad, en este caso al proceso
$X$ se le llama Harris recurrente positivo.


\item[iii)] Cuando $X$ es Harris recurrente positivo se dice que
la disciplina de servicio es estable. En este caso $\pi$ denota la
distribuci\'on estacionaria y hacemos
\[P_{\pi}\left(\cdot\right)=\int_{\mathbf{X}}P_{x}\left(\cdot\right)\pi\left(dx\right),\]
y se utiliza $E_{\pi}$ para denotar el operador esperanza
correspondiente, ver \cite{DaiSean}.
\end{itemize}
\end{Note}

\begin{Def}\label{Def.Cto.Peq.}
Un conjunto $D\in\mathcal{B_{\mathbb{X}}}$ es llamado peque\~no si
existe un $t>0$, una medida de probabilidad $\nu$ en
$\mathcal{B_{\mathbb{X}}}$, y un $\delta>0$ tal que
\[P^{t}\left(x,A\right)\geq\delta\nu\left(A\right),\] para $x\in
D,A\in\mathcal{B_{\mathbb{X}}}$.
\end{Def}

La siguiente serie de resultados vienen enunciados y demostrados
en Dai \cite{Dai}:
\begin{Lema}[Lema 3.1, Dai \cite{Dai}]
Sea $B$ conjunto peque\~no cerrado, supongamos que
$P_{x}\left(\tau_{B}<\infty\right)\equiv1$ y que para alg\'un
$\delta>0$ se cumple que
\begin{equation}\label{Eq.3.1}
\sup\esp_{x}\left[\tau_{B}\left(\delta\right)\right]<\infty,
\end{equation}
donde
$\tau_{B}\left(\delta\right)=inf\left\{t\geq\delta:X\left(t\right)\in
B\right\}$. Entonces, $X$ es un proceso Harris recurrente
positivo.
\end{Lema}

\begin{Lema}[Lema 3.1, Dai \cite{Dai}]\label{Lema.3.}
Bajo el supuesto (A3), el conjunto
$B=\left\{x\in\mathbb{X}:|x|\leq k\right\}$ es un conjunto
peque\~no cerrado para cualquier $k>0$.
\end{Lema}

\begin{Teo}[Teorema 3.1, Dai \cite{Dai}]\label{Tma.3.1}
Si existe un $\delta>0$ tal que
\begin{equation}
lim_{|x|\rightarrow\infty}\frac{1}{|x|}\esp|X^{x}\left(|x|\delta\right)|=0,
\end{equation}
donde $X^{x}$ se utiliza para denotar que el proceso $X$ comienza
a partir de $x$, entonces la ecuaci\'on (\ref{Eq.3.1}) se cumple
para $B=\left\{x\in\mathbb{X}:|x|\leq k\right\}$ con alg\'un
$k>0$. En particular, $X$ es Harris recurrente positivo.
\end{Teo}

Entonces, tenemos que el proceso $X$ es un proceso de Markov que
cumple con los supuestos $A1)$-$A3)$, lo que falta de hacer es
construir el Modelo de Flujo bas\'andonos en lo hasta ahora
presentado.
%_______________________________________________________________________
\subsection{Modelo de Flujo}
%_______________________________________________________________________

Dada una condici\'on inicial $x\in\mathbb{X}$, sea

\begin{itemize}
\item $Q_{k}^{x}\left(t\right)$ la longitud de la cola al tiempo
$t$,

\item $T_{m,k}^{x}\left(t\right)$ el tiempo acumulado, al tiempo
$t$, que tarda el servidor $m$ en atender a los usuarios de la
cola $k$.

\item $T_{m,k}^{x,0}\left(t\right)$ el tiempo acumulado, al tiempo
$t$, que tarda el servidor $m$ en trasladarse a otra cola a partir de la $k$-\'esima.\\
\end{itemize}

Sup\'ongase que la funci\'on
$\left(\overline{Q}\left(\cdot\right),\overline{T}_{m}
\left(\cdot\right),\overline{T}_{m}^{0} \left(\cdot\right)\right)$
para $m=1,2,\ldots,M$ es un punto l\'imite de
\begin{equation}\label{Eq.Punto.Limite}
\left(\frac{1}{|x|}Q^{x}\left(|x|t\right),\frac{1}{|x|}T_{m}^{x}\left(|x|t\right),\frac{1}{|x|}T_{m}^{x,0}\left(|x|t\right)\right)
\end{equation}
para $m=1,2,\ldots,M$, cuando $x\rightarrow\infty$, ver
\cite{Down}. Entonces
$\left(\overline{Q}\left(t\right),\overline{T}_{m}
\left(t\right),\overline{T}_{m}^{0} \left(t\right)\right)$ es un
flujo l\'imite del sistema. Al conjunto de todos las posibles
flujos l\'imite se le llama {\emph{Modelo de Flujo}} y se le
denotar\'a por $\mathcal{Q}$, ver \cite{Down, Dai, DaiSean}.\\

El modelo de flujo satisface el siguiente conjunto de ecuaciones:

\begin{equation}\label{Eq.MF.1}
\overline{Q}_{k}\left(t\right)=\overline{Q}_{k}\left(0\right)+\lambda_{k}t-\sum_{m=1}^{M}\mu_{k}\overline{T}_{m,k}\left(t\right),\\
\end{equation}
para $k=1,2,\ldots,K$.\\
\begin{equation}\label{Eq.MF.2}
\overline{Q}_{k}\left(t\right)\geq0\textrm{ para
}k=1,2,\ldots,K.\\
\end{equation}

\begin{equation}\label{Eq.MF.3}
\overline{T}_{m,k}\left(0\right)=0,\textrm{ y }\overline{T}_{m,k}\left(\cdot\right)\textrm{ es no decreciente},\\
\end{equation}
para $k=1,2,\ldots,K$ y $m=1,2,\ldots,M$.\\
\begin{equation}\label{Eq.MF.4}
\sum_{k=1}^{K}\overline{T}_{m,k}^{0}\left(t\right)+\overline{T}_{m,k}\left(t\right)=t\textrm{
para }m=1,2,\ldots,M.\\
\end{equation}


\begin{Def}[Definici\'on 4.1, Dai \cite{Dai}]\label{Def.Modelo.Flujo}
Sea una disciplina de servicio espec\'ifica. Cualquier l\'imite
$\left(\overline{Q}\left(\cdot\right),\overline{T}\left(\cdot\right),\overline{T}^{0}\left(\cdot\right)\right)$
en (\ref{Eq.Punto.Limite}) es un {\em flujo l\'imite} de la
disciplina. Cualquier soluci\'on (\ref{Eq.MF.1})-(\ref{Eq.MF.4})
es llamado flujo soluci\'on de la disciplina.
\end{Def}

\begin{Def}
Se dice que el modelo de flujo l\'imite, modelo de flujo, de la
disciplina de la cola es estable si existe una constante
$\delta>0$ que depende de $\mu,\lambda$ y $P$ solamente, tal que
cualquier flujo l\'imite con
$|\overline{Q}\left(0\right)|+|\overline{U}|+|\overline{V}|=1$, se
tiene que $\overline{Q}\left(\cdot+\delta\right)\equiv0$.
\end{Def}

Si se hace $|x|\rightarrow\infty$ sin restringir ninguna de las
componentes, tambi\'en se obtienen un modelo de flujo, pero en
este caso el residual de los procesos de arribo y servicio
introducen un retraso:
\begin{Teo}[Teorema 4.2, Dai \cite{Dai}]\label{Tma.4.2.Dai}
Sea una disciplina fija para la cola, suponga que se cumplen las
condiciones (A1)-(A3). Si el modelo de flujo l\'imite de la
disciplina de la cola es estable, entonces la cadena de Markov $X$
que describe la din\'amica de la red bajo la disciplina es Harris
recurrente positiva.
\end{Teo}

Ahora se procede a escalar el espacio y el tiempo para reducir la
aparente fluctuaci\'on del modelo. Consid\'erese el proceso
\begin{equation}\label{Eq.3.7}
\overline{Q}^{x}\left(t\right)=\frac{1}{|x|}Q^{x}\left(|x|t\right).
\end{equation}
A este proceso se le conoce como el flujo escalado, y cualquier
l\'imite $\overline{Q}^{x}\left(t\right)$ es llamado flujo
l\'imite del proceso de longitud de la cola. Haciendo
$|q|\rightarrow\infty$ mientras se mantiene el resto de las
componentes fijas, cualquier punto l\'imite del proceso de
longitud de la cola normalizado $\overline{Q}^{x}$ es soluci\'on
del siguiente modelo de flujo.


\begin{Def}[Definici\'on 3.3, Dai y Meyn \cite{DaiSean}]
El modelo de flujo es estable si existe un tiempo fijo $t_{0}$ tal
que $\overline{Q}\left(t\right)=0$, con $t\geq t_{0}$, para
cualquier $\overline{Q}\left(\cdot\right)\in\mathcal{Q}$ que
cumple con $|\overline{Q}\left(0\right)|=1$.
\end{Def}

\begin{Lemma}[Lema 3.1, Dai y Meyn \cite{DaiSean}]
Si el modelo de flujo definido por (\ref{Eq.MF.1})-(\ref{Eq.MF.4})
es estable, entonces el modelo de flujo retrasado es tambi\'en
estable, es decir, existe $t_{0}>0$ tal que
$\overline{Q}\left(t\right)=0$ para cualquier $t\geq t_{0}$, para
cualquier soluci\'on del modelo de flujo retrasado cuya
condici\'on inicial $\overline{x}$ satisface que
$|\overline{x}|=|\overline{Q}\left(0\right)|+|\overline{A}\left(0\right)|+|\overline{B}\left(0\right)|\leq1$.
\end{Lemma}


Ahora ya estamos en condiciones de enunciar los resultados principales:


\begin{Teo}[Teorema 2.1, Down \cite{Down}]\label{Tma2.1.Down}
Suponga que el modelo de flujo es estable, y que se cumplen los supuestos (A1) y (A2), entonces
\begin{itemize}
\item[i)] Para alguna constante $\kappa_{p}$, y para cada
condici\'on inicial $x\in X$
\begin{equation}\label{Estability.Eq1}
\limsup_{t\rightarrow\infty}\frac{1}{t}\int_{0}^{t}\esp_{x}\left[|Q\left(s\right)|^{p}\right]ds\leq\kappa_{p},
\end{equation}
donde $p$ es el entero dado en (A2).
\end{itemize}
Si adem\'as se cumple la condici\'on (A3), entonces para cada
condici\'on inicial:
\begin{itemize}
\item[ii)] Los momentos transitorios convergen a su estado
estacionario:
 \begin{equation}\label{Estability.Eq2}
lim_{t\rightarrow\infty}\esp_{x}\left[Q_{k}\left(t\right)^{r}\right]=\esp_{\pi}\left[Q_{k}\left(0\right)^{r}\right]\leq\kappa_{r},
\end{equation}
para $r=1,2,\ldots,p$ y $k=1,2,\ldots,K$. Donde $\pi$ es la
probabilidad invariante para $X$.

\item[iii)]  El primer momento converge con raz\'on $t^{p-1}$:
\begin{equation}\label{Estability.Eq3}
lim_{t\rightarrow\infty}t^{p-1}|\esp_{x}\left[Q_{k}\left(t\right)\right]-\esp_{\pi}\left[Q_{k}\left(0\right)\right]|=0.
\end{equation}

\item[iv)] La {\em Ley Fuerte de los grandes n\'umeros} se cumple:
\begin{equation}\label{Estability.Eq4}
lim_{t\rightarrow\infty}\frac{1}{t}\int_{0}^{t}Q_{k}^{r}\left(s\right)ds=\esp_{\pi}\left[Q_{k}\left(0\right)^{r}\right],\textrm{
}\prob_{x}\textrm{-c.s.}
\end{equation}
para $r=1,2,\ldots,p$ y $k=1,2,\ldots,K$.
\end{itemize}
\end{Teo}

La contribuci\'on de Down a la teor\'ia de los {\emph {sistemas de
visitas c\'iclicas}}, es la relaci\'on que hay entre la
estabilidad del sistema con el comportamiento de las medidas de
desempe\~no, es decir, la condici\'on suficiente para poder
garantizar la convergencia del proceso de la longitud de la cola
as\'i como de por los menos los dos primeros momentos adem\'as de
una versi\'on de la Ley Fuerte de los Grandes N\'umeros para los
sistemas de visitas.


\begin{Teo}[Teorema 2.3, Down \cite{Down}]\label{Tma2.3.Down}
Considere el siguiente valor:
\begin{equation}\label{Eq.Rho.1serv}
\rho=\sum_{k=1}^{K}\rho_{k}+max_{1\leq j\leq K}\left(\frac{\lambda_{j}}{\sum_{s=1}^{S}p_{js}\overline{N}_{s}}\right)\delta^{*}
\end{equation}
\begin{itemize}
\item[i)] Si $\rho<1$ entonces la red es estable, es decir, se
cumple el Teorema \ref{Tma2.1.Down}.

\item[ii)] Si $\rho>1$ entonces la red es inestable, es decir, se
cumple el Teorema \ref{Tma2.2.Down}
\end{itemize}
\end{Teo}




Dado el proceso $X=\left\{X\left(t\right),t\geq0\right\}$ definido
en (\ref{Esp.Edos.Down}) que describe la din\'amica del sistema de
visitas c\'iclicas, si $U\left(t\right)$ es el residual de los
tiempos de llegada al tiempo $t$ entre dos usuarios consecutivos y
$V\left(t\right)$ es el residual de los tiempos de servicio al
tiempo $t$ para el usuario que est\'as siendo atendido por el
servidor. Sea $\mathbb{X}$ el espacio de estados que puede tomar
el proceso $X$.


\begin{Lema}[Lema 4.3, Dai\cite{Dai}]\label{Lema.4.3}
Sea $\left\{x_{n}\right\}\subset \mathbf{X}$ con
$|x_{n}|\rightarrow\infty$, conforme $n\rightarrow\infty$. Suponga
que
\[lim_{n\rightarrow\infty}\frac{1}{|x_{n}|}U\left(0\right)=\overline{U}_{k},\]
y
\[lim_{n\rightarrow\infty}\frac{1}{|x_{n}|}V\left(0\right)=\overline{V}_{k}.\]
\begin{itemize}
\item[a)] Conforme $n\rightarrow\infty$ casi seguramente,
\[lim_{n\rightarrow\infty}\frac{1}{|x_{n}|}U^{x_{n}}_{k}\left(|x_{n}|t\right)=\left(\overline{U}_{k}-t\right)^{+}\textrm{, u.o.c.}\]
y
\[lim_{n\rightarrow\infty}\frac{1}{|x_{n}|}V^{x_{n}}_{k}\left(|x_{n}|t\right)=\left(\overline{V}_{k}-t\right)^{+}.\]

\item[b)] Para cada $t\geq0$ fijo,
\[\left\{\frac{1}{|x_{n}|}U^{x_{n}}_{k}\left(|x_{n}|t\right),|x_{n}|\geq1\right\}\]
y
\[\left\{\frac{1}{|x_{n}|}V^{x_{n}}_{k}\left(|x_{n}|t\right),|x_{n}|\geq1\right\}\]
\end{itemize}
son uniformemente convergentes.
\end{Lema}

Sea $e$ es un vector de unos, $C$ es la matriz definida por
\[C_{ik}=\left\{\begin{array}{cc}
1,& S\left(k\right)=i,\\
0,& \textrm{ en otro caso}.\\
\end{array}\right.
\]
Es necesario enunciar el siguiente Teorema que se utilizar\'a para
el Teorema (\ref{Tma.4.2.Dai}):
\begin{Teo}[Teorema 4.1, Dai \cite{Dai}]
Considere una disciplina que cumpla la ley de conservaci\'on, para
casi todas las trayectorias muestrales $\omega$ y cualquier
sucesi\'on de estados iniciales $\left\{x_{n}\right\}\subset
\mathbf{X}$, con $|x_{n}|\rightarrow\infty$, existe una
subsucesi\'on $\left\{x_{n_{j}}\right\}$ con
$|x_{n_{j}}|\rightarrow\infty$ tal que
\begin{equation}\label{Eq.4.15}
\frac{1}{|x_{n_{j}}|}\left(Q^{x_{n_{j}}}\left(0\right),U^{x_{n_{j}}}\left(0\right),V^{x_{n_{j}}}\left(0\right)\right)\rightarrow\left(\overline{Q}\left(0\right),\overline{U},\overline{V}\right),
\end{equation}

\begin{equation}\label{Eq.4.16}
\frac{1}{|x_{n_{j}}|}\left(Q^{x_{n_{j}}}\left(|x_{n_{j}}|t\right),T^{x_{n_{j}}}\left(|x_{n_{j}}|t\right)\right)\rightarrow\left(\overline{Q}\left(t\right),\overline{T}\left(t\right)\right)\textrm{
u.o.c.}
\end{equation}

Adem\'as,
$\left(\overline{Q}\left(t\right),\overline{T}\left(t\right)\right)$
satisface las siguientes ecuaciones:
\begin{equation}\label{Eq.MF.1.3a}
\overline{Q}\left(t\right)=Q\left(0\right)+\left(\alpha
t-\overline{U}\right)^{+}-\left(I-P\right)^{'}M^{-1}\left(\overline{T}\left(t\right)-\overline{V}\right)^{+},
\end{equation}

\begin{equation}\label{Eq.MF.2.3a}
\overline{Q}\left(t\right)\geq0,\\
\end{equation}

\begin{equation}\label{Eq.MF.3.3a}
\overline{T}\left(t\right)\textrm{ es no decreciente y comienza en cero},\\
\end{equation}

\begin{equation}\label{Eq.MF.4.3a}
\overline{I}\left(t\right)=et-C\overline{T}\left(t\right)\textrm{
es no decreciente,}\\
\end{equation}

\begin{equation}\label{Eq.MF.5.3a}
\int_{0}^{\infty}\left(C\overline{Q}\left(t\right)\right)d\overline{I}\left(t\right)=0,\\
\end{equation}

\begin{equation}\label{Eq.MF.6.3a}
\textrm{Condiciones en
}\left(\overline{Q}\left(\cdot\right),\overline{T}\left(\cdot\right)\right)\textrm{
espec\'ificas de la disciplina de la cola,}
\end{equation}
\end{Teo}


Propiedades importantes para el modelo de flujo retrasado:

\begin{Prop}[Proposici\'on 4.2, Dai \cite{Dai}]
 Sea $\left(\overline{Q},\overline{T},\overline{T}^{0}\right)$ un flujo l\'imite de \ref{Eq.Punto.Limite}
 y suponga que cuando $x\rightarrow\infty$ a lo largo de una subsucesi\'on
\[\left(\frac{1}{|x|}Q_{k}^{x}\left(0\right),\frac{1}{|x|}A_{k}^{x}\left(0\right),\frac{1}{|x|}B_{k}^{x}\left(0\right),\frac{1}{|x|}B_{k}^{x,0}\left(0\right)\right)\rightarrow\left(\overline{Q}_{k}\left(0\right),0,0,0\right)\]
para $k=1,\ldots,K$. El flujo l\'imite tiene las siguientes
propiedades, donde las propiedades de la derivada se cumplen donde
la derivada exista:
\begin{itemize}
 \item[i)] Los vectores de tiempo ocupado $\overline{T}\left(t\right)$ y $\overline{T}^{0}\left(t\right)$ son crecientes y continuas con
$\overline{T}\left(0\right)=\overline{T}^{0}\left(0\right)=0$.
\item[ii)] Para todo $t\geq0$
\[\sum_{k=1}^{K}\left[\overline{T}_{k}\left(t\right)+\overline{T}_{k}^{0}\left(t\right)\right]=t.\]
\item[iii)] Para todo $1\leq k\leq K$
\[\overline{Q}_{k}\left(t\right)=\overline{Q}_{k}\left(0\right)+\alpha_{k}t-\mu_{k}\overline{T}_{k}\left(t\right).\]
\item[iv)]  Para todo $1\leq k\leq K$
\[\dot{{\overline{T}}}_{k}\left(t\right)=\rho_{k}\] para $\overline{Q}_{k}\left(t\right)=0$.
\item[v)] Para todo $k,j$
\[\mu_{k}^{0}\overline{T}_{k}^{0}\left(t\right)=\mu_{j}^{0}\overline{T}_{j}^{0}\left(t\right).\]
\item[vi)]  Para todo $1\leq k\leq K$
\[\mu_{k}\dot{{\overline{T}}}_{k}\left(t\right)=l_{k}\mu_{k}^{0}\dot{{\overline{T}}}_{k}^{0}\left(t\right),\] para $\overline{Q}_{k}\left(t\right)>0$.
\end{itemize}
\end{Prop}

\begin{Lema}[Lema 3.1, Chen \cite{Chen}]\label{Lema3.1}
Si el modelo de flujo es estable, definido por las ecuaciones
(3.8)-(3.13), entonces el modelo de flujo retrasado tambi\'en es
estable.
\end{Lema}

\begin{Lema}[Lema 5.2, Gut \cite{Gut}]\label{Lema.5.2.Gut}
Sea $\left\{\xi\left(k\right):k\in\ent\right\}$ sucesi\'on de
variables aleatorias i.i.d. con valores en
$\left(0,\infty\right)$, y sea $E\left(t\right)$ el proceso de
conteo
\[E\left(t\right)=max\left\{n\geq1:\xi\left(1\right)+\cdots+\xi\left(n-1\right)\leq t\right\}.\]
Si $E\left[\xi\left(1\right)\right]<\infty$, entonces para
cualquier entero $r\geq1$
\begin{equation}
lim_{t\rightarrow\infty}\esp\left[\left(\frac{E\left(t\right)}{t}\right)^{r}\right]=\left(\frac{1}{E\left[\xi_{1}\right]}\right)^{r},
\end{equation}
de aqu\'i, bajo estas condiciones
\begin{itemize}
\item[a)] Para cualquier $t>0$,
$sup_{t\geq\delta}\esp\left[\left(\frac{E\left(t\right)}{t}\right)^{r}\right]<\infty$.

\item[b)] Las variables aleatorias
$\left\{\left(\frac{E\left(t\right)}{t}\right)^{r}:t\geq1\right\}$
son uniformemente integrables.
\end{itemize}
\end{Lema}

\begin{Teo}[Teorema 5.1: Ley Fuerte para Procesos de Conteo, Gut
\cite{Gut}]\label{Tma.5.1.Gut} Sea
$0<\mu<\esp\left(X_{1}\right]\leq\infty$. entonces

\begin{itemize}
\item[a)] $\frac{N\left(t\right)}{t}\rightarrow\frac{1}{\mu}$
a.s., cuando $t\rightarrow\infty$.


\item[b)]$\esp\left[\frac{N\left(t\right)}{t}\right]^{r}\rightarrow\frac{1}{\mu^{r}}$,
cuando $t\rightarrow\infty$ para todo $r>0$.
\end{itemize}
\end{Teo}


\begin{Prop}[Proposici\'on 5.1, Dai y Sean \cite{DaiSean}]\label{Prop.5.1}
Suponga que los supuestos (A1) y (A2) se cumplen, adem\'as suponga
que el modelo de flujo es estable. Entonces existe $t_{0}>0$ tal
que
\begin{equation}\label{Eq.Prop.5.1}
lim_{|x|\rightarrow\infty}\frac{1}{|x|^{p+1}}\esp_{x}\left[|X\left(t_{0}|x|\right)|^{p+1}\right]=0.
\end{equation}

\end{Prop}


\begin{Prop}[Proposici\'on 5.3, Dai y Sean \cite{DaiSean}]\label{Prop.5.3.DaiSean}
Sea $X$ proceso de estados para la red de colas, y suponga que se
cumplen los supuestos (A1) y (A2), entonces para alguna constante
positiva $C_{p+1}<\infty$, $\delta>0$ y un conjunto compacto
$C\subset X$.

\begin{equation}\label{Eq.5.4}
\esp_{x}\left[\int_{0}^{\tau_{C}\left(\delta\right)}\left(1+|X\left(t\right)|^{p}\right)dt\right]\leq
C_{p+1}\left(1+|x|^{p+1}\right).
\end{equation}
\end{Prop}

\begin{Prop}[Proposici\'on 5.4, Dai y Sean \cite{DaiSean}]\label{Prop.5.4.DaiSean}
Sea $X$ un proceso de Markov Borel Derecho en $X$, sea
$f:X\leftarrow\rea_{+}$ y defina para alguna $\delta>0$, y un
conjunto cerrado $C\subset X$
\[V\left(x\right):=\esp_{x}\left[\int_{0}^{\tau_{C}\left(\delta\right)}f\left(X\left(t\right)\right)dt\right],\]
para $x\in X$. Si $V$ es finito en todas partes y uniformemente
acotada en $C$, entonces existe $k<\infty$ tal que
\begin{equation}\label{Eq.5.11}
\frac{1}{t}\esp_{x}\left[V\left(x\right)\right]+\frac{1}{t}\int_{0}^{t}\esp_{x}\left[f\left(X\left(s\right)\right)ds\right]\leq\frac{1}{t}V\left(x\right)+k,
\end{equation}
para $x\in X$ y $t>0$.
\end{Prop}


\begin{Teo}[Teorema 5.5, Dai y Sean  \cite{DaiSean}]
Suponga que se cumplen (A1) y (A2), adem\'as suponga que el modelo
de flujo es estable. Entonces existe una constante $k_{p}<\infty$
tal que
\begin{equation}\label{Eq.5.13}
\frac{1}{t}\int_{0}^{t}\esp_{x}\left[|Q\left(s\right)|^{p}\right]ds\leq
k_{p}\left\{\frac{1}{t}|x|^{p+1}+1\right\},
\end{equation}
para $t\geq0$, $x\in X$. En particular para cada condici\'on
inicial
\begin{equation}\label{Eq.5.14}
\limsup_{t\rightarrow\infty}\frac{1}{t}\int_{0}^{t}\esp_{x}\left[|Q\left(s\right)|^{p}\right]ds\leq
k_{p}.
\end{equation}
\end{Teo}

\begin{Teo}[Teorema 6.2 Dai y Sean \cite{DaiSean}]\label{Tma.6.2}
Suponga que se cumplen los supuestos (A1)-(A3) y que el modelo de
flujo es estable, entonces se tiene que
\[\parallel P^{t}\left(x,\cdot\right)-\pi\left(\cdot\right)\parallel_{f_{p}}\rightarrow0,\]
para $t\rightarrow\infty$ y $x\in X$. En particular para cada
condici\'on inicial
\[lim_{t\rightarrow\infty}\esp_{x}\left[\left|Q_{t}\right|^{p}\right]=\esp_{\pi}\left[\left|Q_{0}\right|^{p}\right]<\infty,\]
\end{Teo}

donde

\begin{eqnarray*}
\parallel
P^{t}\left(c,\cdot\right)-\pi\left(\cdot\right)\parallel_{f}=sup_{|g\leq
f|}|\int\pi\left(dy\right)g\left(y\right)-\int
P^{t}\left(x,dy\right)g\left(y\right)|,
\end{eqnarray*}
para $x\in\mathbb{X}$.

\begin{Teo}[Teorema 6.3, Dai y Sean \cite{DaiSean}]\label{Tma.6.3}
Suponga que se cumplen los supuestos (A1)-(A3) y que el modelo de
flujo es estable, entonces con
$f\left(x\right)=f_{1}\left(x\right)$, se tiene que
\[lim_{t\rightarrow\infty}t^{(p-1)}\left|P^{t}\left(c,\cdot\right)-\pi\left(\cdot\right)\right|_{f}=0,\]
para $x\in X$. En particular, para cada condici\'on inicial
\[lim_{t\rightarrow\infty}t^{(p-1)}\left|\esp_{x}\left[Q_{t}\right]-\esp_{\pi}\left[Q_{0}\right]\right|=0.\]
\end{Teo}



\begin{Prop}[Proposici\'on 5.1, Dai y Meyn \cite{DaiSean}]\label{Prop.5.1.DaiSean}
Suponga que los supuestos A1) y A2) son ciertos y que el modelo de
flujo es estable. Entonces existe $t_{0}>0$ tal que
\begin{equation}
lim_{|x|\rightarrow\infty}\frac{1}{|x|^{p+1}}\esp_{x}\left[|X\left(t_{0}|x|\right)|^{p+1}\right]=0.
\end{equation}
\end{Prop}


\begin{Teo}[Teorema 5.5, Dai y Meyn \cite{DaiSean}]\label{Tma.5.5.DaiSean}
Suponga que los supuestos A1) y A2) se cumplen y que el modelo de
flujo es estable. Entonces existe una constante $\kappa_{p}$ tal
que
\begin{equation}
\frac{1}{t}\int_{0}^{t}\esp_{x}\left[|Q\left(s\right)|^{p}\right]ds\leq\kappa_{p}\left\{\frac{1}{t}|x|^{p+1}+1\right\},
\end{equation}
para $t>0$ y $x\in X$. En particular, para cada condici\'on
inicial
\begin{eqnarray*}
\limsup_{t\rightarrow\infty}\frac{1}{t}\int_{0}^{t}\esp_{x}\left[|Q\left(s\right)|^{p}\right]ds\leq\kappa_{p}.
\end{eqnarray*}
\end{Teo}


\begin{Teo}[Teorema 6.4, Dai y Meyn \cite{DaiSean}]\label{Tma.6.4.DaiSean}
Suponga que se cumplen los supuestos A1), A2) y A3) y que el
modelo de flujo es estable. Sea $\nu$ cualquier distribuci\'on de
probabilidad en
$\left(\mathbb{X},\mathcal{B}_{\mathbb{X}}\right)$, y $\pi$ la
distribuci\'on estacionaria de $X$.
\begin{itemize}
\item[i)] Para cualquier $f:X\leftarrow\rea_{+}$
\begin{equation}
\lim_{t\rightarrow\infty}\frac{1}{t}\int_{o}^{t}f\left(X\left(s\right)\right)ds=\pi\left(f\right):=\int
f\left(x\right)\pi\left(dx\right),
\end{equation}
$\prob$-c.s.

\item[ii)] Para cualquier $f:X\leftarrow\rea_{+}$ con
$\pi\left(|f|\right)<\infty$, la ecuaci\'on anterior se cumple.
\end{itemize}
\end{Teo}

\begin{Teo}[Teorema 2.2, Down \cite{Down}]\label{Tma2.2.Down}
Suponga que el fluido modelo es inestable en el sentido de que
para alguna $\epsilon_{0},c_{0}\geq0$,
\begin{equation}\label{Eq.Inestability}
|Q\left(T\right)|\geq\epsilon_{0}T-c_{0}\textrm{,   }T\geq0,
\end{equation}
para cualquier condici\'on inicial $Q\left(0\right)$, con
$|Q\left(0\right)|=1$. Entonces para cualquier $0<q\leq1$, existe
$B<0$ tal que para cualquier $|x|\geq B$,
\begin{equation}
\prob_{x}\left\{\mathbb{X}\rightarrow\infty\right\}\geq q.
\end{equation}
\end{Teo}

\begin{Dem}[Teorema \ref{Tma2.1.Down}] La demostraci\'on de este
teorema se da a continuaci\'on:\\
\begin{itemize}
\item[i)] Utilizando la proposici\'on \ref{Prop.5.3.DaiSean} se
tiene que la proposici\'on \ref{Prop.5.4.DaiSean} es cierta para
$f\left(x\right)=1+|x|^{p}$.

\item[i)] es consecuencia directa del Teorema \ref{Tma.6.2}.

\item[iii)] ver la demostraci\'on dada en Dai y Sean
\cite{DaiSean} p\'aginas 1901-1902.

\item[iv)] ver Dai y Sean \cite{DaiSean} p\'aginas 1902-1903 \'o
\cite{MeynTweedie2}.
\end{itemize}
\end{Dem}

%_____________________________________________________________________
\subsubsection{Modelo de Flujo y Estabilidad}
%_____________________________________________________________________

Para cada $k$ y cada $n$ se define

\numberwithin{equation}{section}
\begin{equation}
\Phi^{k}\left(n\right):=\sum_{i=1}^{n}\phi^{k}\left(i\right).
\end{equation}

suponiendo que el estado inicial de la red es
$x=\left(q,a,b\right)\in X$, entonces para cada $k$

\begin{eqnarray}
E_{k}^{x}\left(t\right):=\max\left\{n\geq0:A_{k}^{x}\left(0\right)+\psi_{k}\left(1\right)+\cdots+\psi_{k}\left(n-1\right)\leq t\right\}\\
S_{k}^{x}\left(t\right):=\max\left\{n\geq0:B_{k}^{x}\left(0\right)+\eta_{k}\left(1\right)+\cdots+\eta_{k}\left(n-1\right)\leq
t\right\}
\end{eqnarray}

Sea $T_{k}^{x}\left(t\right)$ el tiempo acumulado que el servidor
$s\left(k\right)$ ha utilizado en los usuarios de la clase $k$ en
el intervalo $\left[0,t\right]$. Entonces se tienen las siguientes
ecuaciones:

\begin{equation}
Q_{k}^{x}\left(t\right)=Q_{k}^{x}\left(0\right)+E_{k}^{x}\left(t\right)+\sum_{l=1}^{k}\Phi_{k}^{l}S_{l}^{x}\left(T_{l}^{x}\right)-S_{k}^{x}\left(T_{k}^{x}\right)\\
\end{equation}
\begin{equation}
Q^{x}\left(t\right)=\left(Q^{x}_{1}\left(t\right),\ldots,Q^{x}_{K}\left(t\right)\right)^{'}\geq0,\\
\end{equation}
\begin{equation}
T^{x}\left(t\right)=\left(T^{x}_{1}\left(t\right),\ldots,T^{x}_{K}\left(t\right)\right)^{'}\geq0,\textrm{ es no decreciente}\\
\end{equation}
\begin{equation}
I_{i}^{x}\left(t\right)=t-\sum_{k\in C_{i}}T_{k}^{x}\left(t\right)\textrm{ es no decreciente}\\
\end{equation}
\begin{equation}
\int_{0}^{\infty}\sum_{k\in C_{i}}Q_{k}^{x}\left(t\right)dI_{i}^{x}\left(t\right)=0\\
\end{equation}
\begin{equation}
\textrm{condiciones adicionales sobre
}\left(Q^{x}\left(\cdot\right),T^{x}\left(\cdot\right)\right)\textrm{
referentes a la disciplina de servicio}
\end{equation}

Para reducir la fluctuaci\'on del modelo se escala tanto el
espacio como el tiempo, entonces se tiene el proceso:

\begin{equation}
\overline{Q}^{x}\left(t\right)=\frac{1}{|x|}Q^{x}\left(|x|t\right)
\end{equation}
Cualquier l\'imite $\overline{Q}\left(t\right)$ es llamado un
flujo l\'imite del proceso longitud de la cola. Si se hace
$|q|\rightarrow\infty$ y se mantienen las componentes restantes
fijas, de la condici\'on inicial $x$, cualquier punto l\'imite del
proceso normalizado $\overline{Q}^{x}$ es una soluci\'on del
siguiente modelo de flujo, ver \cite{Dai}.

\begin{Def}
Un flujo l\'imite (retrasado) para una red bajo una disciplina de
servicio espec\'ifica se define como cualquier soluci\'on
 $\left(Q^{x}\left(\cdot\right),T^{x}\left(\cdot\right)\right)$ de las siguientes ecuaciones, donde
$\overline{Q}\left(t\right)=\left(\overline{Q}_{1}\left(t\right),\ldots,\overline{Q}_{K}\left(t\right)\right)^{'}$
y
$\overline{T}\left(t\right)=\left(\overline{T}_{1}\left(t\right),\ldots,\overline{T}_{K}\left(t\right)\right)^{'}$
\begin{equation}\label{Eq.3.8}
\overline{Q}_{k}\left(t\right)=\overline{Q}_{k}\left(0\right)+\alpha_{k}t-\mu_{k}\overline{T}_{k}\left(t\right)+\sum_{l=1}^{k}P_{lk}\mu_{l}\overline{T}_{l}\left(t\right)\\
\end{equation}
\begin{equation}\label{Eq.3.9}
\overline{Q}_{k}\left(t\right)\geq0\textrm{ para }k=1,2,\ldots,K,\\
\end{equation}
\begin{equation}\label{Eq.3.10}
\overline{T}_{k}\left(0\right)=0,\textrm{ y }\overline{T}_{k}\left(\cdot\right)\textrm{ es no decreciente},\\
\end{equation}
\begin{equation}\label{Eq.3.11}
\overline{I}_{i}\left(t\right)=t-\sum_{k\in C_{i}}\overline{T}_{k}\left(t\right)\textrm{ es no decreciente}\\
\end{equation}
\begin{equation}\label{Eq.3.12}
\overline{I}_{i}\left(\cdot\right)\textrm{ se incrementa al tiempo}t\textrm{ cuando }\sum_{k\in C_{i}}Q_{k}^{x}\left(t\right)dI_{i}^{x}\left(t\right)=0\\
\end{equation}
\begin{equation}\label{Eq.3.13}
\textrm{condiciones adicionales sobre
}\left(Q^{x}\left(\cdot\right),T^{x}\left(\cdot\right)\right)\textrm{
referentes a la disciplina de servicio}
\end{equation}
\end{Def}

Al conjunto de ecuaciones dadas en \ref{Eq.3.8}-\ref{Eq.3.13} se
le llama {\em Modelo de flujo} y al conjunto de todas las
soluciones del modelo de flujo
$\left(\overline{Q}\left(\cdot\right),\overline{T}
\left(\cdot\right)\right)$ se le denotar\'a por $\mathcal{Q}$.

Si se hace $|x|\rightarrow\infty$ sin restringir ninguna de las
componentes, tambi\'en se obtienen un modelo de flujo, pero en
este caso el residual de los procesos de arribo y servicio
introducen un retraso:

\begin{Def}
El modelo de flujo retrasado de una disciplina de servicio en una
red con retraso
$\left(\overline{A}\left(0\right),\overline{B}\left(0\right)\right)\in\rea_{+}^{K+|A|}$
se define como el conjunto de ecuaciones dadas en
\ref{Eq.3.8}-\ref{Eq.3.13}, junto con la condici\'on:
\begin{equation}\label{CondAd.FluidModel}
\overline{Q}\left(t\right)=\overline{Q}\left(0\right)+\left(\alpha
t-\overline{A}\left(0\right)\right)^{+}-\left(I-P^{'}\right)M\left(\overline{T}\left(t\right)-\overline{B}\left(0\right)\right)^{+}
\end{equation}
\end{Def}

\begin{Def}
El modelo de flujo es estable si existe un tiempo fijo $t_{0}$ tal
que $\overline{Q}\left(t\right)=0$, con $t\geq t_{0}$, para
cualquier $\overline{Q}\left(\cdot\right)\in\mathcal{Q}$ que
cumple con $|\overline{Q}\left(0\right)|=1$.
\end{Def}

El siguiente resultado se encuentra en \cite{Chen}.
\begin{Lemma}
Si el modelo de flujo definido por \ref{Eq.3.8}-\ref{Eq.3.13} es
estable, entonces el modelo de flujo retrasado es tambi\'en
estable, es decir, existe $t_{0}>0$ tal que
$\overline{Q}\left(t\right)=0$ para cualquier $t\geq t_{0}$, para
cualquier soluci\'on del modelo de flujo retrasado cuya
condici\'on inicial $\overline{x}$ satisface que
$|overline{x}|=|\overline{Q}\left(0\right)|+|\overline{A}\left(0\right)|+|\overline{B}\left(0\right)|\leq1$.
\end{Lemma}

%_____________________________________________________________________
\subsubsection{Resultados principales}
%_____________________________________________________________________
Supuestos necesarios sobre la red

\begin{Sup}
\begin{itemize}
\item[A1)] $\psi_{1},\ldots,\psi_{K},\eta_{1},\ldots,\eta_{K}$ son
mutuamente independientes y son sucesiones independientes e
id\'enticamente distribuidas.

\item[A2)] Para alg\'un entero $p\geq1$
\begin{eqnarray*}
\esp\left[\psi_{l}\left(1\right)^{p+1}\right]<\infty\textrm{ para }l\in\mathcal{A}\textrm{ y }\\
\esp\left[\eta_{k}\left(1\right)^{p+1}\right]<\infty\textrm{ para
}k=1,\ldots,K.
\end{eqnarray*}
\item[A3)] El conjunto $\left\{x\in X:|x|=0\right\}$ es un
singleton, y para cada $k\in\mathcal{A}$, existe una funci\'on
positiva $q_{k}\left(x\right)$ definida en $\rea_{+}$, y un entero
$j_{k}$, tal que
\begin{eqnarray}
P\left(\psi_{k}\left(1\right)\geq x\right)>0\textrm{, para todo }x>0\\
P\left(\psi_{k}\left(1\right)+\ldots\psi_{k}\left(j_{k}\right)\in dx\right)\geq q_{k}\left(x\right)dx0\textrm{ y }\\
\int_{0}^{\infty}q_{k}\left(x\right)dx>0
\end{eqnarray}
\end{itemize}
\end{Sup}

El argumento dado en \cite{MaynDown} en el lema
\ref{Lema.34.MeynDown} se puede aplicar para deducir que todos los
subconjuntos compactos de $X$ son peque\~nos.Entonces la
condici\'on $A3)$ se puede generalizar a
\begin{itemize}
\item[A3')] Para el proceso de Markov $X$, cada subconjunto
compacto de $X$ es peque\~no.
\end{itemize}

\begin{Teo}\label{Tma.4.1}
Suponga que el modelo de flujo para una disciplina de servicio es
estable, y suponga adem\'as que las condiciones A1) y A2) se
satisfacen. Entonces:
\begin{itemize}
\item[i)] Para alguna constante $\kappa_{p}$, y para cada
condici\'on inicial $x\in X$
\begin{equation}
\limsup_{t\rightarrow\infty}\frac{1}{t}\int_{0}^{t}\esp_{x}\left[|Q\left(t\right)|^{p}\right]ds\leq\kappa_{p}
\end{equation}
donde $p$ es el entero dado por A2). Suponga adem\'as que A3) o A3')
se cumple, entonces la disciplina de servicio es estable y adem\'as
para cada condici\'on inicial se tiene lo siguiente: \item[ii)] Los
momentos transitorios convergen a sus valores en estado
estacionario:
\begin{equation}
\lim_{t\rightarrow\infty}\esp_{x}\left[Q_{k}\left(t\right)^{r}\right]=\esp_{\pi}\left[Q_{k}\left(0\right)^{r}\right]\leq\kappa_{r}
\end{equation}
para $r=1,\ldots,p$ y $k=1,\ldots,K$. \item[iii)] EL primer
momento converge con raz\'on $t^{p-1}$:
\begin{equation}
\lim_{t\rightarrow\infty}t^{p-1}|\esp_{x}\left[Q\left(t\right)\right]-\esp_{\pi}\left[Q\left(0\right)\right]|=0.
\end{equation}
\item[iv)] Se cumple la Ley Fuerte de los Grandes N\'umeros:
\begin{equation}
\lim_{t\rightarrow\infty}\frac{1}{t}\int_{0}^{t}Q_{k}^{r}\left(s\right)ds=\esp_{\pi}\left[Q_{k}\left(0\right)^{r}\right]
\end{equation}
$\prob$-c.s., para $r=1,\ldots,p$ y $k=1,\ldots,K$.
\end{itemize}
\end{Teo}
\begin{Dem}
La demostraci\'on de este resultado se da aplicando los teoremas
\ref{Tma.5.5}, \ref{Tma.6.2}, \ref{Tma.6.3} y \ref{Tma.6.4}
\end{Dem}

%_____________________________________________________________________
\subsubsection{Definiciones Generales}
%_____________________________________________________________________
Definimos un proceso de estados para la red que depende de la
pol\'itica de servicio utilizada. Bajo cualquier {\em preemptive
buffer priority} disciplina de servicio, el estado
$\mathbb{X}\left(t\right)$ a cualquier tiempo $t$ puede definirse
como
\begin{equation}\label{Eq.Esp.Estados}
\mathbb{X}\left(t\right)=\left(Q_{k}\left(t\right),A_{l}\left(t\right),B_{k}\left(t\right):k=1,2,\ldots,K,l\in\mathcal{A}\right)
\end{equation}
donde $Q_{k}\left(t\right)$ es la longitud de la cola para los
usuarios de la clase $k$, incluyendo aquellos que est\'an siendo
atendidos, $B_{k}\left(t\right)$ son los tiempos de servicio
residuales para los usuarios de la clase $k$ que est\'an en
servicio. Los tiempos de arribo residuales, que son iguales al
tiempo que queda hasta que el pr\'oximo usuario de la clase $k$
llega, se denotan por $A_{k}\left(t\right)$. Tanto
$B_{k}\left(t\right)$ como $A_{k}\left(t\right)$ se suponen
continuos por la derecha.

Sea $\mathbb{X}$ el espacio de estados para el proceso de estados
que por definici\'on es igual  al conjunto de posibles valores
para el estado $\mathbb{X}\left(t\right)$, y sea
$x=\left(q,a,b\right)$ un estado gen\'erico en $\mathbb{X}$, la
componente $q$ determina la posici\'on del usuario en la red,
$|q|$ denota la longitud total de la cola en la red.

Para un estado $x=\left(q,a,b\right)\in\mathbb{X}$ definimos la
{\em norma} de $x$ como $\left\|x\right|=|q|+|a|+|b|$. En
\cite{Dai} se muestra que para una amplia serie de disciplinas de
servicio el proceso $\mathbb{X}$ es un Proceso Fuerte de Markov, y
por tanto se puede asumir que
\[\left(\left(\Omega,\mathcal{F}\right),\mathcal{F}_{t},\mathbb{X}\left(t\right),\theta_{t},P_{x}\right)\]
es un proceso de Borel Derecho en el espacio de estadio medible
$\left(\mathbb{X},\mathcal{B}_{\mathbb{X}}\right)$. El Proceso
$X=\left\{\mathbb{X}\left(t\right),t\geq0\right\}$ tiene
trayectorias continuas por la derecha, est definida en
$\left(\Omega,\mathcal{F}\right)$ y est adaptado a
$\left\{\mathcal{F}_{t},t\geq0\right\}$; $\left\{P_{x},x\in
X\right\}$ son medidas de probabilidad en
$\left(\Omega,\mathcal{F}\right)$ tales que para todo $x\in X$
\[P_{x}\left\{\mathbb{X}\left(0\right)=x\right\}=1\] y
\[E_{x}\left\{f\left(X\circ\theta_{t}\right)|\mathcal{F}_{t}\right\}=E_{X}\left(\tau\right)f\left(X\right)\]
en $\left\{\tau<\infty\right\}$, $P_{x}$-c.s. Donde $\tau$ es un
$\mathcal{F}_{t}$-tiempo de paro
\[\left(X\circ\theta_{\tau}\right)\left(w\right)=\left\{\mathbb{X}\left(\tau\left(w\right)+t,w\right),t\geq0\right\}\]
y $f$ es una funci\'on de valores reales acotada y medible con la
sigma algebra de Kolmogorov generada por los cilindros.

Sea $P^{t}\left(x,D\right)$, $D\in\mathcal{B}_{\mathbb{X}}$,
$t\geq0$ probabilidad de transici\'on de $X$ definida como
\[P^{t}\left(x,D\right)=P_{x}\left(\mathbb{X}\left(t\right)\in
D\right)\]

\begin{Def}
Una medida no cero $\pi$ en
$\left(\mathbb{X},\mathcal{B}_{\mathbb{X}}\right)$ es {\em
invariante} para $X$ si $\pi$ es $\sigma$-finita y
\[\pi\left(D\right)=\int_{X}P^{t}\left(x,D\right)\pi\left(dx\right)\]
para todo $D\in \mathcal{B}_{\mathbb{X}}$, con $t\geq0$.
\end{Def}

\begin{Def}
El proceso de Markov $X$ es llamado {\em Harris recurrente} si
existe una medida de probabilidad $\nu$ en
$\left(\mathbb{X},\mathcal{B}_{\mathbb{X}}\right)$, tal que si
$\nu\left(D\right)>0$ y $D\in\mathcal{B}_{\mathbb{X}}$
\[P_{x}\left\{\tau_{D}<\infty\right\}\equiv1\] cuando
$\tau_{D}=\inf\left\{t\geq0:\mathbb{X}_{t}\in D\right\}$.
\end{Def}

\begin{itemize}
\item Si $X$ es Harris recurrente, entonces una \'unica medida
invariante $\pi$ existe (\cite{Getoor}). \item Si la medida
invariante es finita, entonces puede normalizarse a una medida de
probabilidad, en este caso se le llama {\em Harris recurrente
positiva}. \item Cuando $X$ es Harris recurrente positivo se dice
que la disciplina de servicio es estable. En este caso $\pi$
denota la ditribuci\'on estacionaria y hacemos
\[P_{\pi}\left(\cdot\right)[=\int_{X}P_{x}\left(\cdot\right)\pi\left(dx\right)\]
y se utiliza $E_{\pi}$ para denotar el operador esperanza
correspondiente, as, el proceso
$X=\left\{\mathbb{X}\left(t\right),t\geq0\right\}$ es un proceso
estrictamente estacionario bajo $P_{\pi}$
\end{itemize}

\begin{Def}
Un conjunto $D\in\mathcal{B}_\mathbb{X}$ es llamado peque\~no si
existe un $t>0$, una medida de probabilidad $\nu$ en
$\mathcal{B}_\mathbb{X}$, y un $\delta>0$ tal que
\[P^{t}\left(x,A\right)\geq\delta\nu\left(A\right)\] para $x\in
D,A\in\mathcal{B}_\mathbb{X}$.\footnote{En \cite{MeynTweedie}
muestran que si $P_{x}\left\{\tau_{D}<\infty\right\}\equiv1$
solamente para uno conjunto peque\~no, entonces el proceso es
Harris recurrente}
\end{Def}

%_____________________________________________________________________
\subsubsection{Definiciones y Descripci\'on del Modelo}
%________________________________________________________________________
El modelo est\'a compuesto por $c$ colas de capacidad infinita,
etiquetadas de $1$ a $c$ las cuales son atendidas por $s$
servidores. Los servidores atienden de acuerdo a una cadena de
Markov independiente $\left(X^{i}_{n}\right)_{n}$ con $1\leq i\leq
s$ y $n\in\left\{1,2,\ldots,c\right\}$ con la misma matriz de
transici\'on $r_{k,l}$ y \'unica medida invariante
$\left(p_{k}\right)$. Cada servidor permanece atendiendo en la
cola un periodo llamado de visita y determinada por la pol\'itica de
servicio asignada a la cola.

Los usuarios llegan a la cola $k$ con una tasa $\lambda_{k}$ y son
atendidos a una raz\'on $\mu_{k}$. Las sucesiones de tiempos de
interarribo $\left(\tau_{k}\left(n\right)\right)_{n}$, la de
tiempos de servicio
$\left(\sigma_{k}^{i}\left(n\right)\right)_{n}$ y la de tiempos de
cambio $\left(\sigma_{k,l}^{0,i}\left(n\right)\right)_{n}$
requeridas en la cola $k$ para el servidor $i$ son sucesiones
independientes e id\'enticamente distribuidas con distribuci\'on
general independiente de $i$, con media
$\sigma_{k}=\frac{1}{\mu_{k}}$, respectivamente
$\sigma_{k,l}^{0}=\frac{1}{\mu_{k,l}^{0}}$, e independiente de las
cadenas de Markov $\left(X^{i}_{n}\right)_{n}$. Adem\'as se supone
que los tiempos de interarribo se asume son acotados, para cada
$\rho_{k}=\lambda_{k}\sigma_{k}<s$ para asegurar la estabilidad de
la cola $k$ cuando opera como una cola $M/GM/1$.
%________________________________________________________________________
\subsubsection{Pol\'iticas de Servicio}
%_____________________________________________________________________
Una pol\'itica de servicio determina el n\'umero de usuarios que ser\'an
atendidos sin interrupci\'on en periodo de servicio por los
servidores que atienden a la cola. Para un solo servidor esta se
define a trav\'es de una funci\'on $f$ donde $f\left(x,a\right)$ es el
n\'umero de usuarios que son atendidos sin interrupci\'on cuando el
servidor llega a la cola y encuentra $x$ usuarios esperando dado
el tiempo transcurrido de interarribo $a$. Sea $v\left(x,a\right)$
la duraci\'on del periodo de servicio para una sola condici\'on
inicial $\left(x,a\right)$.

Las pol\'iticas de servicio consideradas satisfacen las siguientes
propiedades:

\begin{itemize}
\item[i)] Hay conservaci\'on del trabajo, es decir
\[v\left(x,a\right)=\sum_{l=1}^{f\left(x,a\right)}\sigma\left(l\right)\]
con $f\left(0,a\right)=v\left(0,a\right)=0$, donde
$\left(\sigma\left(l\right)\right)_{l}$ es una sucesi\'on
independiente e id\'enticamente distribuida de los tiempos de
servicio solicitados. \item[ii)] La selecci\'on de usuarios para se
atendidos es independiente de sus correspondientes tiempos de
servicio y del pasado hasta el inicio del periodo de servicio. As\'i
las distribuci\'on $\left(f,v\right)$ no depende del orden en el
cu\'al son atendidos los usuarios. \item[iii)] La pol\'itica de
servicio es mon\'otona en el sentido de que para cada $a\geq0$ los
n\'umeros $f\left(x,a\right)$ son mon\'otonos en distribuci\'on en $x$ y
su l\'imite en distribuci\'on cuando $x\rightarrow\infty$ es una
variable aleatoria $F^{*0}$ que no depende de $a$. \item[iv)] El
n\'umero de usuarios atendidos por cada servidor es acotado por
$f^{min}\left(x\right)$ de la longitud de la cola $x$ que adem\'as
converge mon\'otonamente en distribuci\'on a $F^{*}$ cuando
$x\rightarrow\infty$
\end{itemize}
%________________________________________________________________________
\subsubsection{Proceso de Estados}
%_____________________________________________________________________
El sistema de colas se describe por medio del proceso de Markov
$\left(X\left(t\right)\right)_{t\in\rea}$ como se define a
continuaci\'on. El estado del sistema al tiempo $t\geq0$ est\'a dado
por
\[X\left(t\right)=\left(Q\left(t\right),P\left(t\right),A\left(t\right),R\left(t\right),C\left(t\right)\right)\]
donde
\begin{itemize}
\item
$Q\left(t\right)=\left(Q_{k}\left(t\right)\right)_{k=1}^{c}$,
n\'umero de usuarios en la cola $k$ al tiempo $t$. \item
$P\left(t\right)=\left(P^{i}\left(t\right)\right)_{i=1}^{s}$, es
la posici\'on del servidor $i$. \item
$A\left(t\right)=\left(A_{k}\left(t\right)\right)_{k=1}^{c}$, es
el residual del tiempo de arribo en la cola $k$ al tiempo $t$.
\item
$R\left(t\right)=\left(R_{k}^{i}\left(t\right),R_{k,l}^{0,i}\left(t\right)\right)_{k,l,i=1}^{c,c,s}$,
el primero es el residual del tiempo de servicio del usuario
atendido por servidor $i$ en la cola $k$ al tiempo $t$, la segunda
componente es el residual del tiempo de cambio del servidor $i$ de
la cola $k$ a la cola $l$ al tiempo $t$. \item
$C\left(t\right)=\left(C_{k}^{i}\left(t\right)\right)_{k,i=1}^{c,s}$,
es la componente correspondiente a la cola $k$ y al servidor $i$
que est\'a determinada por la pol\'itica de servicio en la cola $k$
y que hace al proceso $X\left(t\right)$ un proceso de Markov.
\end{itemize}
Todos los procesos definidos arriba se suponen continuos por la
derecha.

El proceso $X$ tiene la propiedad fuerte de Markov y su espacio de
estados es el espacio producto
\[\mathcal{X}=\nat^{c}\times E^{s}\times \rea_{+}^{c}\times\rea_{+}^{cs}\times\rea_{+}^{c^{2}s}\times \mathcal{C}\] donde $E=\left\{1,2,\ldots,c\right\}^{2}\cup\left\{1,2,\ldots,c\right\}$ y $\mathcal{C}$  depende de las pol\'iticas de servicio.

%_____________________________________________________________________________________
\subsubsection{Introducci{\'o}n}
%_____________________________________________________________________________________
%


Si $x$ es el n{\'u}mero de usuarios en la cola al comienzo del
periodo de servicio y $N_{s}\left(x\right)=N\left(x\right)$ es el
n{\'u}mero de usuarios que son atendidos con la pol{\'\i}tica $s$,
{\'u}nica en nuestro caso durante un periodo de servicio, entonces
se asume que:
\begin{enumerate}
\item
\begin{equation}\label{S1}
lim_{x\rightarrow\infty}\esp\left[N\left(x\right)\right]=\overline{N}>0
\end{equation}
\item
\begin{equation}\label{S2}
\esp\left[N\left(x\right)\right]\leq \overline{N} \end{equation}
para cualquier valor de $x$.
\end{enumerate}
La manera en que atiende el servidor $m$-{\'e}simo, en este caso
en espec{\'\i}fico solo lo ilustraremos con un s{\'o}lo servidor,
es la siguiente:
\begin{itemize}
\item Al t{\'e}rmino de la visita a la cola $j$, el servidor se
cambia a la cola $j^{'}$ con probabilidad
$r_{j,j^{'}}^{m}=r_{j,j^{'}}$

\item La $n$-{\'e}sima ocurencia va acompa{\~n}ada con el tiempo
de cambio de longitud $\delta_{j,j^{'}}\left(n\right)$,
independientes e id{\'e}nticamente distribuidas, con
$\esp\left[\delta_{j,j^{'}}\left(1\right)\right]\geq0$.

\item Sea $\left\{p_{j}\right\}$ la {\'u}nica distribuci{\'o}n
invariante estacionaria para la Cadena de Markov con matriz de
transici{\'o}n $\left(r_{j,j^{'}}\right)$.

\item Finalmente, se define
\begin{equation}
\delta^{*}:=\sum_{j,j^{'}}p_{j}r_{j,j^{'}}\esp\left[\delta_{j,j^{'}}\left(1\right)\right].
\end{equation}
\end{itemize}
%_____________________________________________________________________
\subsubsection{Colas C\'iclicas}
%_____________________________________________________________________
El {\em token passing ring} es una estaci\'on de un solo servidor
con $K$ clases de usuarios. Cada clase tiene su propio regulador
en la estaci\'on. Los usuarios llegan al regulador con raz\'on
$\alpha_{k}$ y son atendidos con taza $\mu_{k}$.

La red se puede modelar como un Proceso de Markov con espacio de
estados continuo, continuo en el tiempo:
\begin{equation}
 X\left(t\right)^{T}=\left(Q_{k}\left(t\right),A_{l}\left(t\right),B_{k}\left(t\right),B_{k}^{0}\left(t\right),C\left(t\right):k=1,\ldots,K,l\in\mathcal{A}\right)
\end{equation}
donde $Q_{k}\left(t\right), B_{k}\left(t\right)$ y
$A_{k}\left(t\right)$ se define como en \ref{Eq.Esp.Estados},
$B_{k}^{0}\left(t\right)$ es el tiempo residual de cambio de la
clase $k$ a la clase $k+1\left(mod K\right)$; $C\left(t\right)$
indica el n\'umero de servicios que han sido comenzados y/o
completados durante la sesi\'on activa del buffer.

Los par\'ametros cruciales son la carga nominal de la cola $k$:
$\beta_{k}=\alpha_{k}/\mu_{k}$ y la carga total es
$\rho_{0}=\sum\beta_{k}$, la media total del tiempo de cambio en
un ciclo del token est\'a definido por
\begin{equation}
 u^{0}=\sum_{k=1}^{K}\esp\left[\eta_{k}^{0}\left(1\right)\right]=\sum_{k=1}^{K}\frac{1}{\mu_{k}^{0}}
\end{equation}

El proceso de la longitud de la cola $Q_{k}^{x}\left(t\right)$ y
el proceso de acumulaci\'on del tiempo de servicio
$T_{k}^{x}\left(t\right)$ para el buffer $k$ y para el estado
inicial $x$ se definen como antes. Sea $T_{k}^{x,0}\left(t\right)$
el tiempo acumulado al tiempo $t$ que el token tarda en cambiar
del buffer $k$ al $k+1\mod K$. Suponga que la funci\'on
$\left(\overline{Q}\left(\cdot\right),\overline{T}\left(\cdot\right),\overline{T}^{0}\left(\cdot\right)\right)$
es un punto l\'imite de
\begin{equation}\label{Eq.4.4}
\left(\frac{1}{|x|}Q^{x}\left(|x|t\right),\frac{1}{|x|}T^{x}\left(|x|t\right),\frac{1}{|x|}T^{x,0}\left(|x|t\right)\right)
\end{equation}
cuando $|x|\rightarrow\infty$. Entonces
$\left(\overline{Q}\left(t\right),\overline{T}\left(t\right),\overline{T}^{0}\left(t\right)\right)$
es un flujo l\'imite retrasado del token ring.

Propiedades importantes para el modelo de flujo retrasado

\begin{Prop}
 Sea $\left(\overline{Q},\overline{T},\overline{T}^{0}\right)$ un flujo l\'imite de \ref{Eq.4.4} y suponga que cuando $x\rightarrow\infty$ a lo largo de
una subsucesi\'on
\[\left(\frac{1}{|x|}Q_{k}^{x}\left(0\right),\frac{1}{|x|}A_{k}^{x}\left(0\right),\frac{1}{|x|}B_{k}^{x}\left(0\right),\frac{1}{|x|}B_{k}^{x,0}\left(0\right)\right)\rightarrow\left(\overline{Q}_{k}\left(0\right),0,0,0\right)\]
para $k=1,\ldots,K$. EL flujo l\'imite tiene las siguientes
propiedades, donde las propiedades de la derivada se cumplen donde
la derivada exista:
\begin{itemize}
 \item[i)] Los vectores de tiempo ocupado $\overline{T}\left(t\right)$ y $\overline{T}^{0}\left(t\right)$ son crecientes y continuas con
$\overline{T}\left(0\right)=\overline{T}^{0}\left(0\right)=0$.
\item[ii)] Para todo $t\geq0$
\[\sum_{k=1}^{K}\left[\overline{T}_{k}\left(t\right)+\overline{T}_{k}^{0}\left(t\right)\right]=t\]
\item[iii)] Para todo $1\leq k\leq K$
\[\overline{Q}_{k}\left(t\right)=\overline{Q}_{k}\left(0\right)+\alpha_{k}t-\mu_{k}\overline{T}_{k}\left(t\right)\]
\item[iv)]  Para todo $1\leq k\leq K$
\[\dot{{\overline{T}}}_{k}\left(t\right)=\beta_{k}\] para $\overline{Q}_{k}\left(t\right)=0$.
\item[v)] Para todo $k,j$
\[\mu_{k}^{0}\overline{T}_{k}^{0}\left(t\right)=\mu_{j}^{0}\overline{T}_{j}^{0}\left(t\right)\]
\item[vi)]  Para todo $1\leq k\leq K$
\[\mu_{k}\dot{{\overline{T}}}_{k}\left(t\right)=l_{k}\mu_{k}^{0}\dot{{\overline{T}}}_{k}^{0}\left(t\right)\] para $\overline{Q}_{k}\left(t\right)>0$.
\end{itemize}
\end{Prop}

%_____________________________________________________________________
\subsubsection{Resultados Previos}
%_____________________________________________________________________

\begin{Lemma}\label{Lema.34.MeynDown}
El proceso estoc\'astico $\Phi$ es un proceso de markov fuerte,
temporalmente homog\'eneo, con trayectorias muestrales continuas
por la derecha, cuyo espacio de estados $Y$ es igual a
$X\times\rea$
\end{Lemma}
\begin{Prop}
 Suponga que los supuestos A1) y A2) son ciertos y que el modelo de flujo es estable. Entonces existe $t_{0}>0$ tal que
\begin{equation}
 lim_{|x|\rightarrow\infty}\frac{1}{|x|^{p+1}}\esp_{x}\left[|X\left(t_{0}|x|\right)|^{p+1}\right]=0
\end{equation}
\end{Prop}

\begin{Lemma}\label{Lema.5.2}
 Sea $\left\{\zeta\left(k\right):k\in \mathbb{z}\right\}$ una sucesi\'on independiente e id\'enticamente distribuida que toma valores en $\left(0,\infty\right)$,
y sea
$E\left(t\right)=max\left(n\geq1:\zeta\left(1\right)+\cdots+\zeta\left(n-1\right)\leq
t\right)$. Si $\esp\left[\zeta\left(1\right)\right]<\infty$,
entonces para cualquier entero $r\geq1$
\begin{equation}
 lim_{t\rightarrow\infty}\esp\left[\left(\frac{E\left(t\right)}{t}\right)^{r}\right]=\left(\frac{1}{\esp\left[\zeta_{1}\right]}\right)^{r}.
\end{equation}
Luego, bajo estas condiciones:
\begin{itemize}
 \item[a)] para cualquier $\delta>0$, $\sup_{t\geq\delta}\esp\left[\left(\frac{E\left(t\right)}{t}\right)^{r}\right]<\infty$
\item[b)] las variables aleatorias
$\left\{\left(\frac{E\left(t\right)}{t}\right)^{r}:t\geq1\right\}$
son uniformemente integrables.
\end{itemize}
\end{Lemma}

\begin{Teo}\label{Tma.5.5}
Suponga que los supuestos A1) y A2) se cumplen y que el modelo de
flujo es estable. Entonces existe una constante $\kappa_{p}$ tal
que
\begin{equation}
\frac{1}{t}\int_{0}^{t}\esp_{x}\left[|Q\left(s\right)|^{p}\right]ds\leq\kappa_{p}\left\{\frac{1}{t}|x|^{p+1}+1\right\}
\end{equation}
para $t>0$ y $x\in X$. En particular, para cada condici\'on inicial
\begin{eqnarray*}
\limsup_{t\rightarrow\infty}\frac{1}{t}\int_{0}^{t}\esp_{x}\left[|Q\left(s\right)|^{p}\right]ds\leq\kappa_{p}.
\end{eqnarray*}
\end{Teo}

\begin{Teo}\label{Tma.6.2}
Suponga que se cumplen los supuestos A1), A2) y A3) y que el
modelo de flujo es estable. Entonces se tiene que
\begin{equation}
|\left|P^{t}\left(x,\cdot\right)-\pi\left(\cdot\right)\right||_{f_{p}}\textrm{,
}t\rightarrow\infty,x\in X.
\end{equation}
En particular para cada condici\'on inicial
\begin{eqnarray*}
\lim_{t\rightarrow\infty}\esp_{x}\left[|Q\left(t\right)|^{p}\right]=\esp_{\pi}\left[|Q\left(0\right)|^{p}\right]\leq\kappa_{r}
\end{eqnarray*}
\end{Teo}
\begin{Teo}\label{Tma.6.3}
Suponga que se cumplen los supuestos A1), A2) y A3) y que el
modelo de flujo es estable. Entonces con
$f\left(x\right)=f_{1}\left(x\right)$ se tiene
\begin{equation}
\lim_{t\rightarrow\infty}t^{p-1}|\left|P^{t}\left(x,\cdot\right)-\pi\left(\cdot\right)\right||_{f}=0.
\end{equation}
En particular para cada condici\'on inicial
\begin{eqnarray*}
\lim_{t\rightarrow\infty}t^{p-1}|\esp_{x}\left[Q\left(t\right)\right]-\esp_{\pi}\left[Q\left(0\right)\right]|=0.
\end{eqnarray*}
\end{Teo}

\begin{Teo}\label{Tma.6.4}
Suponga que se cumplen los supuestos A1), A2) y A3) y que el
modelo de flujo es estable. Sea $\nu$ cualquier distribuci\'on de
probabilidad en $\left(X,\mathcal{B}_{X}\right)$, y $\pi$ la
distribuci\'on estacionaria de $X$.
\begin{itemize}
\item[i)] Para cualquier $f:X\leftarrow\rea_{+}$
\begin{equation}
\lim_{t\rightarrow\infty}\frac{1}{t}\int_{o}^{t}f\left(X\left(s\right)\right)ds=\pi\left(f\right):=\int
f\left(x\right)\pi\left(dx\right)
\end{equation}
$\prob$-c.s. \item[ii)] Para cualquier $f:X\leftarrow\rea_{+}$ con
$\pi\left(|f|\right)<\infty$, la ecuaci\'on anterior se cumple.
\end{itemize}
\end{Teo}

%_____________________________________________________________________________________
%
\subsubsection{Teorema de Estabilidad: Descripci{\'o}n}
%_____________________________________________________________________________________
%


Si $x$ es el n{\'u}mero de usuarios en la cola al comienzo del
periodo de servicio y $N_{s}\left(x\right)=N\left(x\right)$ es el
n{\'u}mero de usuarios que son atendidos con la pol{\'\i}tica $s$,
{\'u}nica en nuestro caso durante un periodo de servicio, entonces
se asume que:
\begin{enumerate}
\item
\begin{equation}\label{S1}
lim_{x\rightarrow\infty}\esp\left[N\left(x\right)\right]=\overline{N}>0
\end{equation}
\item
\begin{equation}\label{S2}
\esp\left[N\left(x\right)\right]\leq \overline{N} \end{equation}
para cualquier valor de $x$.
\end{enumerate}
La manera en que atiende el servidor $m$-{\'e}simo, en este caso
en espec{\'\i}fico solo lo ilustraremos con un s{\'o}lo servidor,
es la siguiente:
\begin{itemize}
\item Al t{\'e}rmino de la visita a la cola $j$, el servidor se
cambia a la cola $j^{'}$ con probabilidad
$r_{j,j^{'}}^{m}=r_{j,j^{'}}$

\item La $n$-{\'e}sima ocurencia va acompa{\~n}ada con el tiempo
de cambio de longitud $\delta_{j,j^{'}}\left(n\right)$,
independientes e id{\'e}nticamente distribuidas, con
$\esp\left[\delta_{j,j^{'}}\left(1\right)\right]\geq0$.

\item Sea $\left\{p_{j}\right\}$ la {\'u}nica distribuci{\'o}n
invariante estacionaria para la Cadena de Markov con matriz de
transici{\'o}n $\left(r_{j,j^{'}}\right)$.

\item Finalmente, se define
\begin{equation}
\delta^{*}:=\sum_{j,j^{'}}p_{j}r_{j,j^{'}}\esp\left[\delta_{j,j^{'}}\left(1\right)\right].
\end{equation}
\end{itemize}

%_________________________________________________________________________
\subsection{Supuestos}
%_________________________________________________________________________
Consideremos el caso en el que se tienen varias colas a las cuales
llegan uno o varios servidores para dar servicio a los usuarios
que se encuentran presentes en la cola, como ya se mencion\'o hay
varios tipos de pol\'iticas de servicio, incluso podr\'ia ocurrir
que la manera en que atiende al resto de las colas sea distinta a
como lo hizo en las anteriores.\\

Para ejemplificar los sistemas de visitas c\'iclicas se
considerar\'a el caso en que a las colas los usuarios son atendidos con
una s\'ola pol\'itica de servicio.\\



Si $\omega$ es el n\'umero de usuarios en la cola al comienzo del
periodo de servicio y $N\left(\omega\right)$ es el n\'umero de
usuarios que son atendidos con una pol\'itica en espec\'ifico
durante el periodo de servicio, entonces se asume que:
\begin{itemize}
\item[1)]\label{S1}$lim_{\omega\rightarrow\infty}\esp\left[N\left(\omega\right)\right]=\overline{N}>0$;
\item[2)]\label{S2}$\esp\left[N\left(\omega\right)\right]\leq\overline{N}$
para cualquier valor de $\omega$.
\end{itemize}
La manera en que atiende el servidor $m$-\'esimo, es la siguiente:
\begin{itemize}
\item Al t\'ermino de la visita a la cola $j$, el servidor cambia
a la cola $j^{'}$ con probabilidad $r_{j,j^{'}}^{m}$

\item La $n$-\'esima vez que el servidor cambia de la cola $j$ a
$j'$, va acompa\~nada con el tiempo de cambio de longitud
$\delta_{j,j^{'}}^{m}\left(n\right)$, con
$\delta_{j,j^{'}}^{m}\left(n\right)$, $n\geq1$, variables
aleatorias independientes e id\'enticamente distribuidas, tales
que $\esp\left[\delta_{j,j^{'}}^{m}\left(1\right)\right]\geq0$.

\item Sea $\left\{p_{j}^{m}\right\}$ la distribuci\'on invariante
estacionaria \'unica para la Cadena de Markov con matriz de
transici\'on $\left(r_{j,j^{'}}^{m}\right)$, se supone que \'esta
existe.

\item Finalmente, se define el tiempo promedio total de traslado
entre las colas como
\begin{equation}
\delta^{*}:=\sum_{j,j^{'}}p_{j}^{m}r_{j,j^{'}}^{m}\esp\left[\delta_{j,j^{'}}^{m}\left(i\right)\right].
\end{equation}
\end{itemize}

Consideremos el caso donde los tiempos entre arribo a cada una de
las colas, $\left\{\xi_{k}\left(n\right)\right\}_{n\geq1}$ son
variables aleatorias independientes a id\'enticamente
distribuidas, y los tiempos de servicio en cada una de las colas
se distribuyen de manera independiente e id\'enticamente
distribuidas $\left\{\eta_{k}\left(n\right)\right\}_{n\geq1}$;
adem\'as ambos procesos cumplen la condici\'on de ser
independientes entre s\'i. Para la $k$-\'esima cola se define la
tasa de arribo por
$\lambda_{k}=1/\esp\left[\xi_{k}\left(1\right)\right]$ y la tasa
de servicio como
$\mu_{k}=1/\esp\left[\eta_{k}\left(1\right)\right]$, finalmente se
define la carga de la cola como $\rho_{k}=\lambda_{k}/\mu_{k}$,
donde se pide que $\rho=\sum_{k=1}^{K}\rho_{k}<1$, para garantizar
la estabilidad del sistema, esto es cierto para las pol\'iticas de
servicio exhaustiva y cerrada, ver Geetor \cite{Getoor}.\\

Si denotamos por
\begin{itemize}
\item $Q_{k}\left(t\right)$ el n\'umero de usuarios presentes en
la cola $k$ al tiempo $t$; \item $A_{k}\left(t\right)$ los
residuales de los tiempos entre arribos a la cola $k$; para cada
servidor $m$; \item $B_{m}\left(t\right)$ denota a los residuales
de los tiempos de servicio al tiempo $t$; \item
$B_{m}^{0}\left(t\right)$ los residuales de los tiempos de
traslado de la cola $k$ a la pr\'oxima por atender al tiempo $t$,

\item sea
$C_{m}\left(t\right)$ el n\'umero de usuarios atendidos durante la
visita del servidor a la cola $k$ al tiempo $t$.
\end{itemize}


En este sentido, el proceso para el sistema de visitas se puede
definir como:

\begin{equation}\label{Esp.Edos.Down}
X\left(t\right)^{T}=\left(Q_{k}\left(t\right),A_{k}\left(t\right),B_{m}\left(t\right),B_{m}^{0}\left(t\right),C_{m}\left(t\right)\right),
\end{equation}
para $k=1,\ldots,K$ y $m=1,2,\ldots,M$, donde $T$ indica que es el
transpuesto del vector que se est\'a definiendo. El proceso $X$
evoluciona en el espacio de estados:
$\mathbb{X}=\ent_{+}^{K}\times\rea_{+}^{K}\times\left(\left\{1,2,\ldots,K\right\}\times\left\{1,2,\ldots,S\right\}\right)^{M}\times\rea_{+}^{K}\times\ent_{+}^{K}$.\\

El sistema aqu\'i descrito debe de cumplir con los siguientes supuestos b\'asicos de un sistema de visitas:
%__________________________________________________________________________
\subsubsection{Supuestos B\'asicos}
%__________________________________________________________________________
\begin{itemize}
\item[A1)] Los procesos
$\xi_{1},\ldots,\xi_{K},\eta_{1},\ldots,\eta_{K}$ son mutuamente
independientes y son sucesiones independientes e id\'enticamente
distribuidas.

\item[A2)] Para alg\'un entero $p\geq1$
\begin{eqnarray*}
\esp\left[\xi_{l}\left(1\right)^{p+1}\right]&<&\infty\textrm{ para }l=1,\ldots,K\textrm{ y }\\
\esp\left[\eta_{k}\left(1\right)^{p+1}\right]&<&\infty\textrm{
para }k=1,\ldots,K.
\end{eqnarray*}
donde $\mathcal{A}$ es la clase de posibles arribos.

\item[A3)] Para cada $k=1,2,\ldots,K$ existe una funci\'on
positiva $q_{k}\left(\cdot\right)$ definida en $\rea_{+}$, y un
entero $j_{k}$, tal que
\begin{eqnarray}
P\left(\xi_{k}\left(1\right)\geq x\right)&>&0\textrm{, para todo }x>0,\\
P\left\{a\leq\sum_{i=1}^{j_{k}}\xi_{k}\left(i\right)\leq
b\right\}&\geq&\int_{a}^{b}q_{k}\left(x\right)dx, \textrm{ }0\leq
a<b.
\end{eqnarray}
\end{itemize}

En lo que respecta al supuesto (A3), en Dai y Meyn \cite{DaiSean}
hacen ver que este se puede sustituir por

\begin{itemize}
\item[A3')] Para el Proceso de Markov $X$, cada subconjunto
compacto del espacio de estados de $X$ es un conjunto peque\~no,
ver definici\'on \ref{Def.Cto.Peq.}.
\end{itemize}

Es por esta raz\'on que con la finalidad de poder hacer uso de
$A3^{'})$ es necesario recurrir a los Procesos de Harris y en
particular a los Procesos Harris Recurrente, ver \cite{Dai,
DaiSean}.
%_______________________________________________________________________
\subsection{Procesos Harris Recurrente}
%_______________________________________________________________________

Por el supuesto (A1) conforme a Davis \cite{Davis}, se puede
definir el proceso de saltos correspondiente de manera tal que
satisfaga el supuesto (A3'), de hecho la demostraci\'on est\'a
basada en la l\'inea de argumentaci\'on de Davis, \cite{Davis},
p\'aginas 362-364.\\

Entonces se tiene un espacio de estados en el cual el proceso $X$
satisface la Propiedad Fuerte de Markov, ver Dai y Meyn
\cite{DaiSean}, dado por

\[\left(\Omega,\mathcal{F},\mathcal{F}_{t},X\left(t\right),\theta_{t},P_{x}\right),\]
adem\'as de ser un proceso de Borel Derecho (Sharpe \cite{Sharpe})
en el espacio de estados medible
$\left(\mathbb{X},\mathcal{B}_\mathbb{X}\right)$. El Proceso
$X=\left\{X\left(t\right),t\geq0\right\}$ tiene trayectorias
continuas por la derecha, est\'a definido en
$\left(\Omega,\mathcal{F}\right)$ y est\'a adaptado a
$\left\{\mathcal{F}_{t},t\geq0\right\}$; la colecci\'on
$\left\{P_{x},x\in \mathbb{X}\right\}$ son medidas de probabilidad
en $\left(\Omega,\mathcal{F}\right)$ tales que para todo $x\in
\mathbb{X}$
\[P_{x}\left\{X\left(0\right)=x\right\}=1,\] y
\[E_{x}\left\{f\left(X\circ\theta_{t}\right)|\mathcal{F}_{t}\right\}=E_{X}\left(\tau\right)f\left(X\right),\]
en $\left\{\tau<\infty\right\}$, $P_{x}$-c.s., con $\theta_{t}$
definido como el operador shift.


Donde $\tau$ es un $\mathcal{F}_{t}$-tiempo de paro
\[\left(X\circ\theta_{\tau}\right)\left(w\right)=\left\{X\left(\tau\left(w\right)+t,w\right),t\geq0\right\},\]
y $f$ es una funci\'on de valores reales acotada y medible, ver \cite{Dai, KaspiMandelbaum}.\\

Sea $P^{t}\left(x,D\right)$, $D\in\mathcal{B}_{\mathbb{X}}$,
$t\geq0$ la probabilidad de transici\'on de $X$ queda definida
como:
\[P^{t}\left(x,D\right)=P_{x}\left(X\left(t\right)\in
D\right).\]


\begin{Def}
Una medida no cero $\pi$ en
$\left(\mathbb{X},\mathcal{B}_{\mathbb{X}}\right)$ es invariante
para $X$ si $\pi$ es $\sigma$-finita y
\[\pi\left(D\right)=\int_{\mathbb{X}}P^{t}\left(x,D\right)\pi\left(dx\right),\]
para todo $D\in \mathcal{B}_{\mathbb{X}}$, con $t\geq0$.
\end{Def}

\begin{Def}
El proceso de Markov $X$ es llamado Harris Recurrente si existe
una medida de probabilidad $\nu$ en
$\left(\mathbb{X},\mathcal{B}_{\mathbb{X}}\right)$, tal que si
$\nu\left(D\right)>0$ y $D\in\mathcal{B}_{\mathbb{X}}$
\[P_{x}\left\{\tau_{D}<\infty\right\}\equiv1,\] cuando
$\tau_{D}=inf\left\{t\geq0:X_{t}\in D\right\}$.
\end{Def}

\begin{Note}
\begin{itemize}
\item[i)] Si $X$ es Harris recurrente, entonces existe una \'unica
medida invariante $\pi$ (Getoor \cite{Getoor}).

\item[ii)] Si la medida invariante es finita, entonces puede
normalizarse a una medida de probabilidad, en este caso al proceso
$X$ se le llama Harris recurrente positivo.


\item[iii)] Cuando $X$ es Harris recurrente positivo se dice que
la disciplina de servicio es estable. En este caso $\pi$ denota la
distribuci\'on estacionaria y hacemos
\[P_{\pi}\left(\cdot\right)=\int_{\mathbf{X}}P_{x}\left(\cdot\right)\pi\left(dx\right),\]
y se utiliza $E_{\pi}$ para denotar el operador esperanza
correspondiente, ver \cite{DaiSean}.
\end{itemize}
\end{Note}

\begin{Def}\label{Def.Cto.Peq.}
Un conjunto $D\in\mathcal{B_{\mathbb{X}}}$ es llamado peque\~no si
existe un $t>0$, una medida de probabilidad $\nu$ en
$\mathcal{B_{\mathbb{X}}}$, y un $\delta>0$ tal que
\[P^{t}\left(x,A\right)\geq\delta\nu\left(A\right),\] para $x\in
D,A\in\mathcal{B_{\mathbb{X}}}$.
\end{Def}

La siguiente serie de resultados vienen enunciados y demostrados
en Dai \cite{Dai}:
\begin{Lema}[Lema 3.1, Dai \cite{Dai}]
Sea $B$ conjunto peque\~no cerrado, supongamos que
$P_{x}\left(\tau_{B}<\infty\right)\equiv1$ y que para alg\'un
$\delta>0$ se cumple que
\begin{equation}\label{Eq.3.1}
\sup\esp_{x}\left[\tau_{B}\left(\delta\right)\right]<\infty,
\end{equation}
donde
$\tau_{B}\left(\delta\right)=inf\left\{t\geq\delta:X\left(t\right)\in
B\right\}$. Entonces, $X$ es un proceso Harris recurrente
positivo.
\end{Lema}

\begin{Lema}[Lema 3.1, Dai \cite{Dai}]\label{Lema.3.}
Bajo el supuesto (A3), el conjunto
$B=\left\{x\in\mathbb{X}:|x|\leq k\right\}$ es un conjunto
peque\~no cerrado para cualquier $k>0$.
\end{Lema}

\begin{Teo}[Teorema 3.1, Dai \cite{Dai}]\label{Tma.3.1}
Si existe un $\delta>0$ tal que
\begin{equation}
lim_{|x|\rightarrow\infty}\frac{1}{|x|}\esp|X^{x}\left(|x|\delta\right)|=0,
\end{equation}
donde $X^{x}$ se utiliza para denotar que el proceso $X$ comienza
a partir de $x$, entonces la ecuaci\'on (\ref{Eq.3.1}) se cumple
para $B=\left\{x\in\mathbb{X}:|x|\leq k\right\}$ con alg\'un
$k>0$. En particular, $X$ es Harris recurrente positivo.
\end{Teo}

Entonces, tenemos que el proceso $X$ es un proceso de Markov que
cumple con los supuestos $A1)$-$A3)$, lo que falta de hacer es
construir el Modelo de Flujo bas\'andonos en lo hasta ahora
presentado.
%_______________________________________________________________________
\subsection{Modelo de Flujo}
%_______________________________________________________________________

Dada una condici\'on inicial $x\in\mathbb{X}$, sea

\begin{itemize}
\item $Q_{k}^{x}\left(t\right)$ la longitud de la cola al tiempo
$t$,

\item $T_{m,k}^{x}\left(t\right)$ el tiempo acumulado, al tiempo
$t$, que tarda el servidor $m$ en atender a los usuarios de la
cola $k$.

\item $T_{m,k}^{x,0}\left(t\right)$ el tiempo acumulado, al tiempo
$t$, que tarda el servidor $m$ en trasladarse a otra cola a partir de la $k$-\'esima.\\
\end{itemize}

Sup\'ongase que la funci\'on
$\left(\overline{Q}\left(\cdot\right),\overline{T}_{m}
\left(\cdot\right),\overline{T}_{m}^{0} \left(\cdot\right)\right)$
para $m=1,2,\ldots,M$ es un punto l\'imite de
\begin{equation}\label{Eq.Punto.Limite}
\left(\frac{1}{|x|}Q^{x}\left(|x|t\right),\frac{1}{|x|}T_{m}^{x}\left(|x|t\right),\frac{1}{|x|}T_{m}^{x,0}\left(|x|t\right)\right)
\end{equation}
para $m=1,2,\ldots,M$, cuando $x\rightarrow\infty$, ver
\cite{Down}. Entonces
$\left(\overline{Q}\left(t\right),\overline{T}_{m}
\left(t\right),\overline{T}_{m}^{0} \left(t\right)\right)$ es un
flujo l\'imite del sistema. Al conjunto de todos las posibles
flujos l\'imite se le llama {\emph{Modelo de Flujo}} y se le
denotar\'a por $\mathcal{Q}$, ver \cite{Down, Dai, DaiSean}.\\

El modelo de flujo satisface el siguiente conjunto de ecuaciones:

\begin{equation}\label{Eq.MF.1}
\overline{Q}_{k}\left(t\right)=\overline{Q}_{k}\left(0\right)+\lambda_{k}t-\sum_{m=1}^{M}\mu_{k}\overline{T}_{m,k}\left(t\right),\\
\end{equation}
para $k=1,2,\ldots,K$.\\
\begin{equation}\label{Eq.MF.2}
\overline{Q}_{k}\left(t\right)\geq0\textrm{ para
}k=1,2,\ldots,K.\\
\end{equation}

\begin{equation}\label{Eq.MF.3}
\overline{T}_{m,k}\left(0\right)=0,\textrm{ y }\overline{T}_{m,k}\left(\cdot\right)\textrm{ es no decreciente},\\
\end{equation}
para $k=1,2,\ldots,K$ y $m=1,2,\ldots,M$.\\
\begin{equation}\label{Eq.MF.4}
\sum_{k=1}^{K}\overline{T}_{m,k}^{0}\left(t\right)+\overline{T}_{m,k}\left(t\right)=t\textrm{
para }m=1,2,\ldots,M.\\
\end{equation}


\begin{Def}[Definici\'on 4.1, Dai \cite{Dai}]\label{Def.Modelo.Flujo}
Sea una disciplina de servicio espec\'ifica. Cualquier l\'imite
$\left(\overline{Q}\left(\cdot\right),\overline{T}\left(\cdot\right),\overline{T}^{0}\left(\cdot\right)\right)$
en (\ref{Eq.Punto.Limite}) es un {\em flujo l\'imite} de la
disciplina. Cualquier soluci\'on (\ref{Eq.MF.1})-(\ref{Eq.MF.4})
es llamado flujo soluci\'on de la disciplina.
\end{Def}

\begin{Def}
Se dice que el modelo de flujo l\'imite, modelo de flujo, de la
disciplina de la cola es estable si existe una constante
$\delta>0$ que depende de $\mu,\lambda$ y $P$ solamente, tal que
cualquier flujo l\'imite con
$|\overline{Q}\left(0\right)|+|\overline{U}|+|\overline{V}|=1$, se
tiene que $\overline{Q}\left(\cdot+\delta\right)\equiv0$.
\end{Def}

Si se hace $|x|\rightarrow\infty$ sin restringir ninguna de las
componentes, tambi\'en se obtienen un modelo de flujo, pero en
este caso el residual de los procesos de arribo y servicio
introducen un retraso:
\begin{Teo}[Teorema 4.2, Dai \cite{Dai}]\label{Tma.4.2.Dai}
Sea una disciplina fija para la cola, suponga que se cumplen las
condiciones (A1)-(A3). Si el modelo de flujo l\'imite de la
disciplina de la cola es estable, entonces la cadena de Markov $X$
que describe la din\'amica de la red bajo la disciplina es Harris
recurrente positiva.
\end{Teo}

Ahora se procede a escalar el espacio y el tiempo para reducir la
aparente fluctuaci\'on del modelo. Consid\'erese el proceso
\begin{equation}\label{Eq.3.7}
\overline{Q}^{x}\left(t\right)=\frac{1}{|x|}Q^{x}\left(|x|t\right).
\end{equation}
A este proceso se le conoce como el flujo escalado, y cualquier
l\'imite $\overline{Q}^{x}\left(t\right)$ es llamado flujo
l\'imite del proceso de longitud de la cola. Haciendo
$|q|\rightarrow\infty$ mientras se mantiene el resto de las
componentes fijas, cualquier punto l\'imite del proceso de
longitud de la cola normalizado $\overline{Q}^{x}$ es soluci\'on
del siguiente modelo de flujo.


\begin{Def}[Definici\'on 3.3, Dai y Meyn \cite{DaiSean}]
El modelo de flujo es estable si existe un tiempo fijo $t_{0}$ tal
que $\overline{Q}\left(t\right)=0$, con $t\geq t_{0}$, para
cualquier $\overline{Q}\left(\cdot\right)\in\mathcal{Q}$ que
cumple con $|\overline{Q}\left(0\right)|=1$.
\end{Def}

\begin{Lemma}[Lema 3.1, Dai y Meyn \cite{DaiSean}]
Si el modelo de flujo definido por (\ref{Eq.MF.1})-(\ref{Eq.MF.4})
es estable, entonces el modelo de flujo retrasado es tambi\'en
estable, es decir, existe $t_{0}>0$ tal que
$\overline{Q}\left(t\right)=0$ para cualquier $t\geq t_{0}$, para
cualquier soluci\'on del modelo de flujo retrasado cuya
condici\'on inicial $\overline{x}$ satisface que
$|\overline{x}|=|\overline{Q}\left(0\right)|+|\overline{A}\left(0\right)|+|\overline{B}\left(0\right)|\leq1$.
\end{Lemma}


Ahora ya estamos en condiciones de enunciar los resultados principales:


\begin{Teo}[Teorema 2.1, Down \cite{Down}]\label{Tma2.1.Down}
Suponga que el modelo de flujo es estable, y que se cumplen los supuestos (A1) y (A2), entonces
\begin{itemize}
\item[i)] Para alguna constante $\kappa_{p}$, y para cada
condici\'on inicial $x\in X$
\begin{equation}\label{Estability.Eq1}
\limsup_{t\rightarrow\infty}\frac{1}{t}\int_{0}^{t}\esp_{x}\left[|Q\left(s\right)|^{p}\right]ds\leq\kappa_{p},
\end{equation}
donde $p$ es el entero dado en (A2).
\end{itemize}
Si adem\'as se cumple la condici\'on (A3), entonces para cada
condici\'on inicial:
\begin{itemize}
\item[ii)] Los momentos transitorios convergen a su estado
estacionario:
 \begin{equation}\label{Estability.Eq2}
lim_{t\rightarrow\infty}\esp_{x}\left[Q_{k}\left(t\right)^{r}\right]=\esp_{\pi}\left[Q_{k}\left(0\right)^{r}\right]\leq\kappa_{r},
\end{equation}
para $r=1,2,\ldots,p$ y $k=1,2,\ldots,K$. Donde $\pi$ es la
probabilidad invariante para $X$.

\item[iii)]  El primer momento converge con raz\'on $t^{p-1}$:
\begin{equation}\label{Estability.Eq3}
lim_{t\rightarrow\infty}t^{p-1}|\esp_{x}\left[Q_{k}\left(t\right)\right]-\esp_{\pi}\left[Q_{k}\left(0\right)\right]|=0.
\end{equation}

\item[iv)] La {\em Ley Fuerte de los grandes n\'umeros} se cumple:
\begin{equation}\label{Estability.Eq4}
lim_{t\rightarrow\infty}\frac{1}{t}\int_{0}^{t}Q_{k}^{r}\left(s\right)ds=\esp_{\pi}\left[Q_{k}\left(0\right)^{r}\right],\textrm{
}\prob_{x}\textrm{-c.s.}
\end{equation}
para $r=1,2,\ldots,p$ y $k=1,2,\ldots,K$.
\end{itemize}
\end{Teo}

La contribuci\'on de Down a la teor\'ia de los {\emph {sistemas de
visitas c\'iclicas}}, es la relaci\'on que hay entre la
estabilidad del sistema con el comportamiento de las medidas de
desempe\~no, es decir, la condici\'on suficiente para poder
garantizar la convergencia del proceso de la longitud de la cola
as\'i como de por los menos los dos primeros momentos adem\'as de
una versi\'on de la Ley Fuerte de los Grandes N\'umeros para los
sistemas de visitas.


\begin{Teo}[Teorema 2.3, Down \cite{Down}]\label{Tma2.3.Down}
Considere el siguiente valor:
\begin{equation}\label{Eq.Rho.1serv}
\rho=\sum_{k=1}^{K}\rho_{k}+max_{1\leq j\leq K}\left(\frac{\lambda_{j}}{\sum_{s=1}^{S}p_{js}\overline{N}_{s}}\right)\delta^{*}
\end{equation}
\begin{itemize}
\item[i)] Si $\rho<1$ entonces la red es estable, es decir, se
cumple el Teorema \ref{Tma2.1.Down}.

\item[ii)] Si $\rho>1$ entonces la red es inestable, es decir, se
cumple el Teorema \ref{Tma2.2.Down}
\end{itemize}
\end{Teo}



%_________________________________________________________________________
\subsection{Modelo de Flujo}
%_________________________________________________________________________
Sup\'ongase que el sistema consta de varias colas a los cuales
llegan uno o varios servidores a dar servicio a los usuarios
esperando en la cola.\\


Sea $x$ el n\'umero de usuarios en la cola esperando por servicio
y $N\left(x\right)$ es el n\'umero de usuarios que son atendidos
con una pol\'itica dada y fija mientras el servidor permanece
dando servicio, entonces se asume que:
\begin{itemize}
\item[(S1.)]
\begin{equation}\label{S1}
lim_{x\rightarrow\infty}\esp\left[N\left(x\right)\right]=\overline{N}>0.
\end{equation}
\item[(S2.)]
\begin{equation}\label{S2}
\esp\left[N\left(x\right)\right]\leq \overline{N},
\end{equation}

para cualquier valor de $x$.
\end{itemize}

El tiempo que tarda un servidor en volver a dar servicio despu\'es
de abandonar la cola inmediata anterior y llegar a la pr\'oxima se
llama tiempo de traslado o de cambio  de cola, al momento de la
$n$-\'esima visita del servidor a la cola $j$ se genera una
sucesi\'on de variables aleatorias $\delta_{j,j+1}\left(n\right)$,
independientes e id\'enticamente distribuidas, con la propiedad de
que $\esp\left[\delta_{j,j+1}\left(1\right)\right]\geq0$.\\


Se define
\begin{equation}
\delta^{*}:=\sum_{j,j+1}\esp\left[\delta_{j,j+1}\left(1\right)\right].
\end{equation}
%\begin{figure}[H]
%\centering
%\includegraphics[width=7cm]{switchovertime.jpg}
%\caption{Sistema de Visitas C\'iclicas}
%\end{figure}

Los tiempos entre arribos a la cola $k$, son de la forma
$\left\{\xi_{k}\left(n\right)\right\}_{n\geq1}$, con la propiedad
de que son independientes e id\'enticamente distribuidos. Los
tiempos de servicio
$\left\{\eta_{k}\left(n\right)\right\}_{n\geq1}$ tienen la
propiedad de ser independientes e id\'enticamente distribuidos.
Para la $k$-\'esima cola se define la tasa de arribo a la como
$\lambda_{k}=1/\esp\left[\xi_{k}\left(1\right)\right]$ y la tasa
de servicio como
$\mu_{k}=1/\esp\left[\eta_{k}\left(1\right)\right]$, finalmente se
define la carga de la cola como $\rho_{k}=\lambda_{k}/\mu_{k}$,
donde se pide que $\rho<1$, para garantizar la estabilidad del sistema.\\

%_____________________________________________________________________
%\subsubsection{Proceso de Estados}
%_____________________________________________________________________

Para el caso m\'as sencillo podemos definir un proceso de estados
para la red que depende de la pol\'itica de servicio utilizada, el
estado $\mathbb{X}\left(t\right)$ a cualquier tiempo $t$ puede
definirse como
\begin{equation}\label{Eq.Esp.Estados}
\mathbb{X}\left(t\right)=\left(Q_{k}\left(t\right),A_{l}\left(t\right),B_{k}\left(t\right):k=1,2,\ldots,K,l\in\mathcal{A}\right),
\end{equation}

donde $Q_{k}\left(t\right)$ es la longitud de la cola $k$ para los
usuarios esperando servicio, incluyendo aquellos que est\'an
siendo atendidos, $B_{k}\left(t\right)$ son los tiempos de
servicio residuales para los usuarios de la clase $k$ que est\'an
en servicio.\\

Los tiempos entre arribos residuales, que son el tiempo que queda
hasta que el pr\'oximo usuario llega a la cola para recibir
servicio, se denotan por $A_{k}\left(t\right)$. Tanto
$B_{k}\left(t\right)$ como $A_{k}\left(t\right)$ se suponen
continuos por la derecha \cite{Dai2}.\\

Sea $\mathcal{X}$ el espacio de estados para el proceso de estados
que por definici\'on es igual  al conjunto de posibles valores
para el estado $\mathbb{X}\left(t\right)$, y sea
$x=\left(q,a,b\right)$ un estado gen\'erico en $\mathbb{X}$, la
componente $q$ determina la posici\'on del usuario en la red,
$|q|$ denota la longitud total de la cola en la red.\\

Para un estado $x=\left(q,a,b\right)\in\mathbb{X}$ definimos la
{\em norma} de $x$ como $\left\|x\right\|=|q|+|a|+|b|$. En
\cite{Dai} se muestra que para una amplia serie de disciplinas de
servicio el proceso $\mathbb{X}$ es un Proceso Fuerte de Markov, y
por tanto se puede asumir que
\[\left(\left(\Omega,\mathcal{F}\right),\mathcal{F}_{t},\mathbb{X}\left(t\right),\theta_{t},P_{x}\right)\]
es un proceso de {\em Borel Derecho} en el espacio de estados
medible $\left(\mathcal{X},\mathcal{B}_{\mathcal{X}}\right)$.\\

Sea $P^{t}\left(x,D\right)$, $D\in\mathcal{B}_{\mathbb{X}}$,
$t\geq0$ probabilidad de transici\'on de $X$ definida como
\[P^{t}\left(x,D\right)=P_{x}\left(\mathbb{X}\left(t\right)\in
D\right).\]

\begin{Def}
Una medida no cero $\pi$ en
$\left(\mathbb{X},\mathcal{B}_{\mathbb{X}}\right)$ es {\em
invariante} para $X$ si $\pi$ es $\sigma$-finita y
\[\pi\left(D\right)=\int_{X}P^{t}\left(x,D\right)\pi\left(dx\right),\]
para todo $D\in \mathcal{B}_{\mathbb{X}}$, con $t\geq0$.
\end{Def}

\begin{Def}
El proceso de Markov $X$ es llamado {\em Harris recurrente} si
existe una medida de probabilidad $\nu$ en
$\left(\mathbb{X},\mathcal{B}_{\mathbb{X}}\right)$, tal que si
$\nu\left(D\right)>0$ y $D\in\mathcal{B}_{\mathbb{X}}$
\[P_{x}\left\{\tau_{D}<\infty\right\}\equiv1,\] cuando
$\tau_{D}=inf\left\{t\geq0:\mathbb{X}_{t}\in D\right\}$.
\end{Def}

\begin{Def}
Un conjunto $D\in\mathcal{B}_\mathbb{X}$ es llamado peque\~no si
existe un $t>0$, una medida de probabilidad $\nu$ en
$\mathcal{B}_\mathbb{X}$, y un $\delta>0$ tal que
\[P^{t}\left(x,A\right)\geq\delta\nu\left(A\right),\] para $x\in
D,A\in\mathcal{B}_\mathbb{X}$.
\end{Def}
\begin{Note}
\begin{itemize}

\item[i)] Si $X$ es Harris recurrente, entonces existe una \'unica medida
invariante $\pi$ (\cite{Getoor}).

\item[ii)] Si la medida invariante es finita, entonces puede
normalizarse a una medida de probabilidad, en este caso a la
medida se le llama \textbf{Harris recurrente positiva}.

\item[iii)] Cuando $X$ es Harris recurrente positivo se dice que
la disciplina de servicio es estable. En este caso $\pi$ denota la
ditribuci\'on estacionaria; se define
\[P_{\pi}\left(\cdot\right)=\int_{X}P_{x}\left(\cdot\right)\pi\left(dx\right).\]
Se utiliza $E_{\pi}$ para denotar el operador esperanza
correspondiente, as\'i, el proceso
$X=\left\{\mathbb{X}\left(t\right),t\geq0\right\}$ es un proceso
estrictamente estacionario bajo $P_{\pi}$.

\item[iv)] En \cite{MeynTweedie} se muestra que si
$P_{x}\left\{\tau_{D}<\infty\right\}=1$ incluso para solamente un
conjunto peque\~no, entonces el proceso de Harris es recurrente.
\end{itemize}
\end{Note}


%_________________________________________________________________________
%\newpage
%_________________________________________________________________________
%\subsection{Modelo de Flujo}
%_____________________________________________________________________
Las Colas C\'iclicas se pueden describir por medio de un proceso
de Markov $\left(X\left(t\right)\right)_{t\in\rea}$, donde el
estado del sistema al tiempo $t\geq0$ est\'a dado por
\begin{equation}
X\left(t\right)=\left(Q\left(t\right),A\left(t\right),H\left(t\right),B\left(t\right),B^{0}\left(t\right),C\left(t\right)\right)
\end{equation}
definido en el espacio producto:
\begin{equation}
\mathcal{X}=\mathbb{Z}^{K}\times\rea_{+}^{K}\times\left(\left\{1,2,\ldots,K\right\}\times\left\{1,2,\ldots,S\right\}\right)^{M}\times\rea_{+}^{K}\times\rea_{+}^{K}\times\mathbb{Z}^{K},
\end{equation}

\begin{itemize}
\item $Q\left(t\right)=\left(Q_{k}\left(t\right),1\leq k\leq
K\right)$, es el n\'umero de usuarios en la cola $k$, incluyendo
aquellos que est\'an siendo atendidos provenientes de la
$k$-\'esima cola.

\item $A\left(t\right)=\left(A_{k}\left(t\right),1\leq k\leq
K\right)$, son los residuales de los tiempos de arribo en la cola
$k$. \item $H\left(t\right)$ es el par ordenado que consiste en la
cola que esta siendo atendida y la pol\'itica de servicio que se
utilizar\'a.

\item $B\left(t\right)$ es el tiempo de servicio residual.

\item $B^{0}\left(t\right)$ es el tiempo residual del cambio de
cola.

\item $C\left(t\right)$ indica el n\'umero de usuarios atendidos
durante la visita del servidor a la cola dada en
$H\left(t\right)$.
\end{itemize}

$A_{k}\left(t\right),B_{m}\left(t\right)$ y
$B_{m}^{0}\left(t\right)$ se suponen continuas por la derecha y
que satisfacen la propiedad fuerte de Markov, (\cite{Dai}).

Dada una condici\'on inicial $x\in\mathcal{X}$, $Q_{k}^{x}\left(t\right)$ es la longitud de la cola $k$ al tiempo $t$
y $T_{m,k}^{x}\left(t\right)$  el tiempo acumulado al tiempo $t$ que el servidor tarda en atender a los usuarios de la cola $k$.
De igual manera se define $T_{m,k}^{x,0}\left(t\right)$ el tiempo acumulado al tiempo $t$ que el servidor tarda en
cambiar de cola para volver a atender a los usuarios.

Para reducir la fluctuaci\'on del modelo se escala tanto el espacio como el tiempo, entonces se
tiene el proceso:

\begin{eqnarray}
\overline{Q}^{x}\left(t\right)=\frac{1}{|x|}Q^{x}\left(|x|t\right),\\
\overline{T}_{m}^{x}\left(t\right)=\frac{1}{|x|}T_{m}^{x}\left(|x|t\right),\\
\overline{T}_{m}^{x,0}\left(t\right)=\frac{1}{|x|}T_{m}^{x,0}\left(|x|t\right).
\end{eqnarray}
Cualquier l\'imite $\overline{Q}\left(t\right)$ es llamado un
flujo l\'imite del proceso longitud de la cola, al conjunto de todos los posibles flujos l\'imite
se le llamar\'a \textbf{modelo de flujo}, (\cite{MaynDown}).

\begin{Def}
Un flujo l\'imite para un sistema de visitas bajo una disciplina de
servicio espec\'ifica se define como cualquier soluci\'on
 $\left(\overline{Q}\left(\cdot\right),\overline{T}_{m}\left(\cdot\right),\overline{T}_{m}^{0}\left(\cdot\right)\right)$
 de las siguientes ecuaciones, donde
$\overline{Q}\left(t\right)=\left(\overline{Q}_{1}\left(t\right),\ldots,\overline{Q}_{K}\left(t\right)\right)$
y
$\overline{T}\left(t\right)=\left(\overline{T}_{1}\left(t\right),\ldots,\overline{T}_{K}\left(t\right)\right)$
\begin{equation}\label{Eq.3.8}
\overline{Q}_{k}\left(t\right)=\overline{Q}_{k}\left(0\right)+\lambda_{k}t-\sum_{m=1}^{M}\mu_{k}\overline{T}_{m,k}\left(t\right)\\
\end{equation}
\begin{equation}\label{Eq.3.9}
\overline{Q}_{k}\left(t\right)\geq0\textrm{ para }k=1,2,\ldots,K,\\
\end{equation}
\begin{equation}\label{Eq.3.10}
\overline{T}_{m,k}\left(0\right)=0,\textrm{ y }\overline{T}_{m,k}\left(\cdot\right)\textrm{ es no decreciente},\\
\end{equation}
\begin{equation}\label{Eq.3.11}
\sum_{k=1}^{K}\overline{T}_{m,k}^{0}\left(t\right)+\overline{T}_{m,k}\left(t\right)=t\textrm{ para}m=1,2,\ldots,M\\
\end{equation}
\end{Def}

Al conjunto de ecuaciones dadas en (\ref{Eq.3.8})-(\ref{Eq.3.11}) se
le llama {\em Modelo de flujo} y al conjunto de todas las
soluciones del modelo de flujo
$\left(\overline{Q}\left(\cdot\right),\overline{T}
\left(\cdot\right)\right)$ se le denotar\'a por $\mathcal{Q}$.


\begin{Def}
El modelo de flujo es estable si existe un tiempo fijo $t_{0}$ tal
que $\overline{Q}\left(t\right)=0$, con $t\geq t_{0}$, para
cualquier $\overline{Q}\left(\cdot\right)\in\mathcal{Q}$ que
cumple con $|\overline{Q}\left(0\right)|=1$.
\end{Def}


%_____________________________________________________________________________________
%
\subsection{Estabilidad de los Sistemas de Visitas C\'iclicas}
%_________________________________________________________________________

Es necesario realizar los siguientes supuestos, ver (\cite{Dai2}) y (\cite{DaiSean}):

\begin{itemize}
\item[A1)] $\xi_{1},\ldots,\xi_{K},\eta_{1},\ldots,\eta_{K}$ son
mutuamente independientes y son sucesiones independientes e
id\'enticamente distribuidas.

\item[A2)] Para alg\'un entero $p\geq1$
\begin{eqnarray*}
\esp\left[\xi_{k}\left(1\right)^{p+1}\right]&<&\infty\textrm{ para }l\in\mathcal{A}\textrm{ y }\\
\esp\left[\eta_{k}\left(1\right)^{p+1}\right]&<&\infty\textrm{ para
}k=1,\ldots,K.
\end{eqnarray*}
\item[A3)] El conjunto $\left\{x\in X:|x|=0\right\}$ es un
singleton, y para cada $k\in\mathcal{A}$, existe una funci\'on
positiva $q_{k}\left(x\right)$ definida en $\rea_{+}$, y un entero
$j_{k}$, tal que
\begin{eqnarray}
P\left(\xi_{k}\left(1\right)\geq x\right)&>&0\textrm{, para todo }x>0\\
P\left(\xi_{k}\left(1\right)+\ldots\xi_{k}\left(j_{k}\right)\in dx\right)&\geq& q_{k}\left(x\right)dx0\textrm{ y }\\
\int_{0}^{\infty}q_{k}\left(x\right)dx>0
\end{eqnarray}
\end{itemize}


En \cite{MaynDown} ser da un argumento para deducir que todos los
subconjuntos compactos de $X$ son peque\~nos. Entonces la
condici\'on A3) se puede generalizar a
\begin{itemize}
\item[A3')] Para el proceso de Markov $X$, cada subconjunto
compacto de $X$ es peque\~no.
\end{itemize}

\begin{Teo}\label{Tma2.1}
Suponga que el modelo de flujo para una disciplina de servicio es
estable, y suponga adem\'as que las condiciones A1) y A2) se
satisfacen. Entonces:
\begin{itemize}
\item[i)] Para alguna constante $\kappa_{p}$, y para cada
condici\'on inicial $x\in X$
\begin{equation}
\limsup_{t\rightarrow\infty}\frac{1}{t}\int_{0}^{t}\esp_{x}\left[|Q\left(t\right)|^{p}\right]ds\leq\kappa_{p}
\end{equation}
donde $p$ es el entero dado por A2).
\end{itemize}

Suponga adem\'as que A3) o A3')
se cumple, entonces la disciplina de servicio es estable y adem\'as
para cada condici\'on inicial se tiene lo siguiente:
\begin{itemize}

\item[ii)] Los momentos transitorios convergen a sus valores en estado
estacionario:
\begin{equation}
\lim_{t\rightarrow\infty}\esp_{x}\left[Q_{k}\left(t\right)^{r}\right]=\esp_{\pi}\left[Q_{k}\left(0\right)^{r}\right]\leq\kappa_{r}
\end{equation}
para $r=1,\ldots,p$ y $k=1,\ldots,K$. \item[iii)] El primer
momento converge con raz\'on $t^{p-1}$:
\begin{equation}
\lim_{t\rightarrow\infty}t^{p-1}|\esp_{x}\left[Q\left(t\right)\right]-\esp_{\pi}\left[Q\left(0\right)\right]|=0.
\end{equation}
\item[iv)] Se cumple la Ley Fuerte de los Grandes N\'umeros:
\begin{equation}
\lim_{t\rightarrow\infty}\frac{1}{t}\int_{0}^{t}Q_{k}^{r}\left(s\right)ds=\esp_{\pi}\left[Q_{k}\left(0\right)^{r}\right]
\end{equation}
$\prob$-c.s., para $r=1,\ldots,p$ y $k=1,\ldots,K$.
\end{itemize}
\end{Teo}


\begin{Teo}\label{Tma2.2}
Suponga que el fluido modelo es inestable en el sentido de que
para alguna $\epsilon_{0},c_{0}\geq0$,
\begin{equation}\label{Eq.Inestability}
|Q\left(T\right)|\geq\epsilon_{0}T-c_{0}\textrm{, con }T\geq0,
\end{equation}
para cualquier condici\'on inicial $Q\left(0\right)$, con
$|Q\left(0\right)|=1$. Entonces para cualquier $0<q\leq1$, existe
$B<0$ tal que para cualquier $|x|\geq B$,
\begin{equation}
\prob_{x}\left\{\mathbb{X}\rightarrow\infty\right\}\geq q.
\end{equation}
\end{Teo}

%_____________________________________________________________________________________
%

%_____________________________________________________________________
\subsection{Resultados principales}
%_____________________________________________________________________
En el caso particular de un modelo con un solo servidor, $M=1$, se
tiene que si se define
\begin{Def}\label{Def.Ro}
\begin{equation}\label{RoM1}
\rho=\sum_{k=1}^{K}\rho_{k}+\max_{1\leq j\leq
K}\left(\frac{\lambda_{j}}{\overline{N}}\right)\delta^{*}.
\end{equation}
\end{Def}
entonces

\begin{Teo}\label{Teo.Down}
\begin{itemize}
\item[i)] Si $\rho<1$, entonces la red es estable, es decir el teorema
(\ref{Tma2.1}) se cumple. \item[ii)] De lo contrario, es decir, si
$\rho>1$ entonces la red es inestable, es decir, el teorema
(\ref{Tma2.2}).
\end{itemize}
\end{Teo}



%_________________________________________________________________________
\subsection{Supuestos}
%_________________________________________________________________________
Consideremos el caso en el que se tienen varias colas a las cuales
llegan uno o varios servidores para dar servicio a los usuarios
que se encuentran presentes en la cola, como ya se mencion\'o hay
varios tipos de pol\'iticas de servicio, incluso podr\'ia ocurrir
que la manera en que atiende al resto de las colas sea distinta a
como lo hizo en las anteriores.\\

Para ejemplificar los sistemas de visitas c\'iclicas se
considerar\'a el caso en que a las colas los usuarios son atendidos con
una s\'ola pol\'itica de servicio.\\


Si $\omega$ es el n\'umero de usuarios en la cola al comienzo del
periodo de servicio y $N\left(\omega\right)$ es el n\'umero de
usuarios que son atendidos con una pol\'itica en espec\'ifico
durante el periodo de servicio, entonces se asume que:
\begin{itemize}
\item[1)]\label{S1}$lim_{\omega\rightarrow\infty}\esp\left[N\left(\omega\right)\right]=\overline{N}>0$;
\item[2)]\label{S2}$\esp\left[N\left(\omega\right)\right]\leq\overline{N}$
para cualquier valor de $\omega$.
\end{itemize}
La manera en que atiende el servidor $m$-\'esimo, es la siguiente:
\begin{itemize}
\item Al t\'ermino de la visita a la cola $j$, el servidor cambia
a la cola $j^{'}$ con probabilidad $r_{j,j^{'}}^{m}$

\item La $n$-\'esima vez que el servidor cambia de la cola $j$ a
$j'$, va acompa\~nada con el tiempo de cambio de longitud
$\delta_{j,j^{'}}^{m}\left(n\right)$, con
$\delta_{j,j^{'}}^{m}\left(n\right)$, $n\geq1$, variables
aleatorias independientes e id\'enticamente distribuidas, tales
que $\esp\left[\delta_{j,j^{'}}^{m}\left(1\right)\right]\geq0$.

\item Sea $\left\{p_{j}^{m}\right\}$ la distribuci\'on invariante
estacionaria \'unica para la Cadena de Markov con matriz de
transici\'on $\left(r_{j,j^{'}}^{m}\right)$, se supone que \'esta
existe.

\item Finalmente, se define el tiempo promedio total de traslado
entre las colas como
\begin{equation}
\delta^{*}:=\sum_{j,j^{'}}p_{j}^{m}r_{j,j^{'}}^{m}\esp\left[\delta_{j,j^{'}}^{m}\left(i\right)\right].
\end{equation}
\end{itemize}

Consideremos el caso donde los tiempos entre arribo a cada una de
las colas, $\left\{\xi_{k}\left(n\right)\right\}_{n\geq1}$ son
variables aleatorias independientes a id\'enticamente
distribuidas, y los tiempos de servicio en cada una de las colas
se distribuyen de manera independiente e id\'enticamente
distribuidas $\left\{\eta_{k}\left(n\right)\right\}_{n\geq1}$;
adem\'as ambos procesos cumplen la condici\'on de ser
independientes entre s\'i. Para la $k$-\'esima cola se define la
tasa de arribo por
$\lambda_{k}=1/\esp\left[\xi_{k}\left(1\right)\right]$ y la tasa
de servicio como
$\mu_{k}=1/\esp\left[\eta_{k}\left(1\right)\right]$, finalmente se
define la carga de la cola como $\rho_{k}=\lambda_{k}/\mu_{k}$,
donde se pide que $\rho=\sum_{k=1}^{K}\rho_{k}<1$, para garantizar
la estabilidad del sistema, esto es cierto para las pol\'iticas de
servicio exhaustiva y cerrada, ver Geetor \cite{Getoor}.\\

Si denotamos por
\begin{itemize}
\item $Q_{k}\left(t\right)$ el n\'umero de usuarios presentes en
la cola $k$ al tiempo $t$; \item $A_{k}\left(t\right)$ los
residuales de los tiempos entre arribos a la cola $k$; para cada
servidor $m$; \item $B_{m}\left(t\right)$ denota a los residuales
de los tiempos de servicio al tiempo $t$; \item
$B_{m}^{0}\left(t\right)$ los residuales de los tiempos de
traslado de la cola $k$ a la pr\'oxima por atender al tiempo $t$,

\item sea
$C_{m}\left(t\right)$ el n\'umero de usuarios atendidos durante la
visita del servidor a la cola $k$ al tiempo $t$.
\end{itemize}


En este sentido, el proceso para el sistema de visitas se puede
definir como:

\begin{equation}\label{Esp.Edos.Down}
X\left(t\right)^{T}=\left(Q_{k}\left(t\right),A_{k}\left(t\right),B_{m}\left(t\right),B_{m}^{0}\left(t\right),C_{m}\left(t\right)\right),
\end{equation}
para $k=1,\ldots,K$ y $m=1,2,\ldots,M$, donde $T$ indica que es el
transpuesto del vector que se est\'a definiendo. El proceso $X$
evoluciona en el espacio de estados:
$\mathbb{X}=\ent_{+}^{K}\times\rea_{+}^{K}\times\left(\left\{1,2,\ldots,K\right\}\times\left\{1,2,\ldots,S\right\}\right)^{M}\times\rea_{+}^{K}\times\ent_{+}^{K}$.\\

El sistema aqu\'i descrito debe de cumplir con los siguientes supuestos b\'asicos de un sistema de visitas:
%__________________________________________________________________________
\subsubsection{Supuestos B\'asicos}
%__________________________________________________________________________
\begin{itemize}
\item[A1)] Los procesos
$\xi_{1},\ldots,\xi_{K},\eta_{1},\ldots,\eta_{K}$ son mutuamente
independientes y son sucesiones independientes e id\'enticamente
distribuidas.

\item[A2)] Para alg\'un entero $p\geq1$
\begin{eqnarray*}
\esp\left[\xi_{l}\left(1\right)^{p+1}\right]&<&\infty\textrm{ para }l=1,\ldots,K\textrm{ y }\\
\esp\left[\eta_{k}\left(1\right)^{p+1}\right]&<&\infty\textrm{
para }k=1,\ldots,K.
\end{eqnarray*}
donde $\mathcal{A}$ es la clase de posibles arribos.

\item[A3)] Para cada $k=1,2,\ldots,K$ existe una funci\'on
positiva $q_{k}\left(\cdot\right)$ definida en $\rea_{+}$, y un
entero $j_{k}$, tal que
\begin{eqnarray}
P\left(\xi_{k}\left(1\right)\geq x\right)&>&0\textrm{, para todo }x>0,\\
P\left\{a\leq\sum_{i=1}^{j_{k}}\xi_{k}\left(i\right)\leq
b\right\}&\geq&\int_{a}^{b}q_{k}\left(x\right)dx, \textrm{ }0\leq
a<b.
\end{eqnarray}
\end{itemize}

En lo que respecta al supuesto (A3), en Dai y Meyn \cite{DaiSean}
hacen ver que este se puede sustituir por

\begin{itemize}
\item[A3')] Para el Proceso de Markov $X$, cada subconjunto
compacto del espacio de estados de $X$ es un conjunto peque\~no,
ver definici\'on \ref{Def.Cto.Peq.}.
\end{itemize}

Es por esta raz\'on que con la finalidad de poder hacer uso de
$A3^{'})$ es necesario recurrir a los Procesos de Harris y en
particular a los Procesos Harris Recurrente, ver \cite{Dai,
DaiSean}.
%_______________________________________________________________________
\subsection{Procesos Harris Recurrente}
%_______________________________________________________________________

Por el supuesto (A1) conforme a Davis \cite{Davis}, se puede
definir el proceso de saltos correspondiente de manera tal que
satisfaga el supuesto (A3'), de hecho la demostraci\'on est\'a
basada en la l\'inea de argumentaci\'on de Davis, \cite{Davis},
p\'aginas 362-364.\\

Entonces se tiene un espacio de estados en el cual el proceso $X$
satisface la Propiedad Fuerte de Markov, ver Dai y Meyn
\cite{DaiSean}, dado por

\[\left(\Omega,\mathcal{F},\mathcal{F}_{t},X\left(t\right),\theta_{t},P_{x}\right),\]
adem\'as de ser un proceso de Borel Derecho (Sharpe \cite{Sharpe})
en el espacio de estados medible
$\left(\mathbb{X},\mathcal{B}_\mathbb{X}\right)$. El Proceso
$X=\left\{X\left(t\right),t\geq0\right\}$ tiene trayectorias
continuas por la derecha, est\'a definido en
$\left(\Omega,\mathcal{F}\right)$ y est\'a adaptado a
$\left\{\mathcal{F}_{t},t\geq0\right\}$; la colecci\'on
$\left\{P_{x},x\in \mathbb{X}\right\}$ son medidas de probabilidad
en $\left(\Omega,\mathcal{F}\right)$ tales que para todo $x\in
\mathbb{X}$
\[P_{x}\left\{X\left(0\right)=x\right\}=1,\] y
\[E_{x}\left\{f\left(X\circ\theta_{t}\right)|\mathcal{F}_{t}\right\}=E_{X}\left(\tau\right)f\left(X\right),\]
en $\left\{\tau<\infty\right\}$, $P_{x}$-c.s., con $\theta_{t}$
definido como el operador shift.


Donde $\tau$ es un $\mathcal{F}_{t}$-tiempo de paro
\[\left(X\circ\theta_{\tau}\right)\left(w\right)=\left\{X\left(\tau\left(w\right)+t,w\right),t\geq0\right\},\]
y $f$ es una funci\'on de valores reales acotada y medible, ver \cite{Dai, KaspiMandelbaum}.\\

Sea $P^{t}\left(x,D\right)$, $D\in\mathcal{B}_{\mathbb{X}}$,
$t\geq0$ la probabilidad de transici\'on de $X$ queda definida
como:
\[P^{t}\left(x,D\right)=P_{x}\left(X\left(t\right)\in
D\right).\]


\begin{Def}
Una medida no cero $\pi$ en
$\left(\mathbb{X},\mathcal{B}_{\mathbb{X}}\right)$ es invariante
para $X$ si $\pi$ es $\sigma$-finita y
\[\pi\left(D\right)=\int_{\mathbb{X}}P^{t}\left(x,D\right)\pi\left(dx\right),\]
para todo $D\in \mathcal{B}_{\mathbb{X}}$, con $t\geq0$.
\end{Def}

\begin{Def}
El proceso de Markov $X$ es llamado Harris Recurrente si existe
una medida de probabilidad $\nu$ en
$\left(\mathbb{X},\mathcal{B}_{\mathbb{X}}\right)$, tal que si
$\nu\left(D\right)>0$ y $D\in\mathcal{B}_{\mathbb{X}}$
\[P_{x}\left\{\tau_{D}<\infty\right\}\equiv1,\] cuando
$\tau_{D}=inf\left\{t\geq0:X_{t}\in D\right\}$.
\end{Def}

\begin{Note}
\begin{itemize}
\item[i)] Si $X$ es Harris recurrente, entonces existe una \'unica
medida invariante $\pi$ (Getoor \cite{Getoor}).

\item[ii)] Si la medida invariante es finita, entonces puede
normalizarse a una medida de probabilidad, en este caso al proceso
$X$ se le llama Harris recurrente positivo.


\item[iii)] Cuando $X$ es Harris recurrente positivo se dice que
la disciplina de servicio es estable. En este caso $\pi$ denota la
distribuci\'on estacionaria y hacemos
\[P_{\pi}\left(\cdot\right)=\int_{\mathbf{X}}P_{x}\left(\cdot\right)\pi\left(dx\right),\]
y se utiliza $E_{\pi}$ para denotar el operador esperanza
correspondiente, ver \cite{DaiSean}.
\end{itemize}
\end{Note}

\begin{Def}\label{Def.Cto.Peq.}
Un conjunto $D\in\mathcal{B_{\mathbb{X}}}$ es llamado peque\~no si
existe un $t>0$, una medida de probabilidad $\nu$ en
$\mathcal{B_{\mathbb{X}}}$, y un $\delta>0$ tal que
\[P^{t}\left(x,A\right)\geq\delta\nu\left(A\right),\] para $x\in
D,A\in\mathcal{B_{\mathbb{X}}}$.
\end{Def}

La siguiente serie de resultados vienen enunciados y demostrados
en Dai \cite{Dai}:
\begin{Lema}[Lema 3.1, Dai \cite{Dai}]
Sea $B$ conjunto peque\~no cerrado, supongamos que
$P_{x}\left(\tau_{B}<\infty\right)\equiv1$ y que para alg\'un
$\delta>0$ se cumple que
\begin{equation}\label{Eq.3.1}
\sup\esp_{x}\left[\tau_{B}\left(\delta\right)\right]<\infty,
\end{equation}
donde
$\tau_{B}\left(\delta\right)=inf\left\{t\geq\delta:X\left(t\right)\in
B\right\}$. Entonces, $X$ es un proceso Harris recurrente
positivo.
\end{Lema}

\begin{Lema}[Lema 3.1, Dai \cite{Dai}]\label{Lema.3.}
Bajo el supuesto (A3), el conjunto
$B=\left\{x\in\mathbb{X}:|x|\leq k\right\}$ es un conjunto
peque\~no cerrado para cualquier $k>0$.
\end{Lema}

\begin{Teo}[Teorema 3.1, Dai \cite{Dai}]\label{Tma.3.1}
Si existe un $\delta>0$ tal que
\begin{equation}
lim_{|x|\rightarrow\infty}\frac{1}{|x|}\esp|X^{x}\left(|x|\delta\right)|=0,
\end{equation}
donde $X^{x}$ se utiliza para denotar que el proceso $X$ comienza
a partir de $x$, entonces la ecuaci\'on (\ref{Eq.3.1}) se cumple
para $B=\left\{x\in\mathbb{X}:|x|\leq k\right\}$ con alg\'un
$k>0$. En particular, $X$ es Harris recurrente positivo.
\end{Teo}

Entonces, tenemos que el proceso $X$ es un proceso de Markov que
cumple con los supuestos $A1)$-$A3)$, lo que falta de hacer es
construir el Modelo de Flujo bas\'andonos en lo hasta ahora
presentado.
%_______________________________________________________________________
\subsection{Modelo de Flujo}
%_______________________________________________________________________

Dada una condici\'on inicial $x\in\mathbb{X}$, sea

\begin{itemize}
\item $Q_{k}^{x}\left(t\right)$ la longitud de la cola al tiempo
$t$,

\item $T_{m,k}^{x}\left(t\right)$ el tiempo acumulado, al tiempo
$t$, que tarda el servidor $m$ en atender a los usuarios de la
cola $k$.

\item $T_{m,k}^{x,0}\left(t\right)$ el tiempo acumulado, al tiempo
$t$, que tarda el servidor $m$ en trasladarse a otra cola a partir de la $k$-\'esima.\\
\end{itemize}

Sup\'ongase que la funci\'on
$\left(\overline{Q}\left(\cdot\right),\overline{T}_{m}
\left(\cdot\right),\overline{T}_{m}^{0} \left(\cdot\right)\right)$
para $m=1,2,\ldots,M$ es un punto l\'imite de
\begin{equation}\label{Eq.Punto.Limite}
\left(\frac{1}{|x|}Q^{x}\left(|x|t\right),\frac{1}{|x|}T_{m}^{x}\left(|x|t\right),\frac{1}{|x|}T_{m}^{x,0}\left(|x|t\right)\right)
\end{equation}
para $m=1,2,\ldots,M$, cuando $x\rightarrow\infty$, ver
\cite{Down}. Entonces
$\left(\overline{Q}\left(t\right),\overline{T}_{m}
\left(t\right),\overline{T}_{m}^{0} \left(t\right)\right)$ es un
flujo l\'imite del sistema. Al conjunto de todos las posibles
flujos l\'imite se le llama {\emph{Modelo de Flujo}} y se le
denotar\'a por $\mathcal{Q}$, ver \cite{Down, Dai, DaiSean}.\\

El modelo de flujo satisface el siguiente conjunto de ecuaciones:

\begin{equation}\label{Eq.MF.1}
\overline{Q}_{k}\left(t\right)=\overline{Q}_{k}\left(0\right)+\lambda_{k}t-\sum_{m=1}^{M}\mu_{k}\overline{T}_{m,k}\left(t\right),\\
\end{equation}
para $k=1,2,\ldots,K$.\\
\begin{equation}\label{Eq.MF.2}
\overline{Q}_{k}\left(t\right)\geq0\textrm{ para
}k=1,2,\ldots,K.\\
\end{equation}

\begin{equation}\label{Eq.MF.3}
\overline{T}_{m,k}\left(0\right)=0,\textrm{ y }\overline{T}_{m,k}\left(\cdot\right)\textrm{ es no decreciente},\\
\end{equation}
para $k=1,2,\ldots,K$ y $m=1,2,\ldots,M$.\\
\begin{equation}\label{Eq.MF.4}
\sum_{k=1}^{K}\overline{T}_{m,k}^{0}\left(t\right)+\overline{T}_{m,k}\left(t\right)=t\textrm{
para }m=1,2,\ldots,M.\\
\end{equation}


\begin{Def}[Definici\'on 4.1, Dai \cite{Dai}]\label{Def.Modelo.Flujo}
Sea una disciplina de servicio espec\'ifica. Cualquier l\'imite
$\left(\overline{Q}\left(\cdot\right),\overline{T}\left(\cdot\right),\overline{T}^{0}\left(\cdot\right)\right)$
en (\ref{Eq.Punto.Limite}) es un {\em flujo l\'imite} de la
disciplina. Cualquier soluci\'on (\ref{Eq.MF.1})-(\ref{Eq.MF.4})
es llamado flujo soluci\'on de la disciplina.
\end{Def}

\begin{Def}
Se dice que el modelo de flujo l\'imite, modelo de flujo, de la
disciplina de la cola es estable si existe una constante
$\delta>0$ que depende de $\mu,\lambda$ y $P$ solamente, tal que
cualquier flujo l\'imite con
$|\overline{Q}\left(0\right)|+|\overline{U}|+|\overline{V}|=1$, se
tiene que $\overline{Q}\left(\cdot+\delta\right)\equiv0$.
\end{Def}

Si se hace $|x|\rightarrow\infty$ sin restringir ninguna de las
componentes, tambi\'en se obtienen un modelo de flujo, pero en
este caso el residual de los procesos de arribo y servicio
introducen un retraso:
\begin{Teo}[Teorema 4.2, Dai \cite{Dai}]\label{Tma.4.2.Dai}
Sea una disciplina fija para la cola, suponga que se cumplen las
condiciones (A1)-(A3). Si el modelo de flujo l\'imite de la
disciplina de la cola es estable, entonces la cadena de Markov $X$
que describe la din\'amica de la red bajo la disciplina es Harris
recurrente positiva.
\end{Teo}

Ahora se procede a escalar el espacio y el tiempo para reducir la
aparente fluctuaci\'on del modelo. Consid\'erese el proceso
\begin{equation}\label{Eq.3.7}
\overline{Q}^{x}\left(t\right)=\frac{1}{|x|}Q^{x}\left(|x|t\right).
\end{equation}
A este proceso se le conoce como el flujo escalado, y cualquier
l\'imite $\overline{Q}^{x}\left(t\right)$ es llamado flujo
l\'imite del proceso de longitud de la cola. Haciendo
$|q|\rightarrow\infty$ mientras se mantiene el resto de las
componentes fijas, cualquier punto l\'imite del proceso de
longitud de la cola normalizado $\overline{Q}^{x}$ es soluci\'on
del siguiente modelo de flujo.


\begin{Def}[Definici\'on 3.3, Dai y Meyn \cite{DaiSean}]
El modelo de flujo es estable si existe un tiempo fijo $t_{0}$ tal
que $\overline{Q}\left(t\right)=0$, con $t\geq t_{0}$, para
cualquier $\overline{Q}\left(\cdot\right)\in\mathcal{Q}$ que
cumple con $|\overline{Q}\left(0\right)|=1$.
\end{Def}

\begin{Lemma}[Lema 3.1, Dai y Meyn \cite{DaiSean}]
Si el modelo de flujo definido por (\ref{Eq.MF.1})-(\ref{Eq.MF.4})
es estable, entonces el modelo de flujo retrasado es tambi\'en
estable, es decir, existe $t_{0}>0$ tal que
$\overline{Q}\left(t\right)=0$ para cualquier $t\geq t_{0}$, para
cualquier soluci\'on del modelo de flujo retrasado cuya
condici\'on inicial $\overline{x}$ satisface que
$|\overline{x}|=|\overline{Q}\left(0\right)|+|\overline{A}\left(0\right)|+|\overline{B}\left(0\right)|\leq1$.
\end{Lemma}


Ahora ya estamos en condiciones de enunciar los resultados principales:


\begin{Teo}[Teorema 2.1, Down \cite{Down}]\label{Tma2.1.Down}
Suponga que el modelo de flujo es estable, y que se cumplen los supuestos (A1) y (A2), entonces
\begin{itemize}
\item[i)] Para alguna constante $\kappa_{p}$, y para cada
condici\'on inicial $x\in X$
\begin{equation}\label{Estability.Eq1}
\limsup_{t\rightarrow\infty}\frac{1}{t}\int_{0}^{t}\esp_{x}\left[|Q\left(s\right)|^{p}\right]ds\leq\kappa_{p},
\end{equation}
donde $p$ es el entero dado en (A2).
\end{itemize}
Si adem\'as se cumple la condici\'on (A3), entonces para cada
condici\'on inicial:
\begin{itemize}
\item[ii)] Los momentos transitorios convergen a su estado
estacionario:
 \begin{equation}\label{Estability.Eq2}
lim_{t\rightarrow\infty}\esp_{x}\left[Q_{k}\left(t\right)^{r}\right]=\esp_{\pi}\left[Q_{k}\left(0\right)^{r}\right]\leq\kappa_{r},
\end{equation}
para $r=1,2,\ldots,p$ y $k=1,2,\ldots,K$. Donde $\pi$ es la
probabilidad invariante para $X$.

\item[iii)]  El primer momento converge con raz\'on $t^{p-1}$:
\begin{equation}\label{Estability.Eq3}
lim_{t\rightarrow\infty}t^{p-1}|\esp_{x}\left[Q_{k}\left(t\right)\right]-\esp_{\pi}\left[Q_{k}\left(0\right)\right]|=0.
\end{equation}

\item[iv)] La {\em Ley Fuerte de los grandes n\'umeros} se cumple:
\begin{equation}\label{Estability.Eq4}
lim_{t\rightarrow\infty}\frac{1}{t}\int_{0}^{t}Q_{k}^{r}\left(s\right)ds=\esp_{\pi}\left[Q_{k}\left(0\right)^{r}\right],\textrm{
}\prob_{x}\textrm{-c.s.}
\end{equation}
para $r=1,2,\ldots,p$ y $k=1,2,\ldots,K$.
\end{itemize}
\end{Teo}

La contribuci\'on de Down a la teor\'ia de los {\emph {sistemas de
visitas c\'iclicas}}, es la relaci\'on que hay entre la
estabilidad del sistema con el comportamiento de las medidas de
desempe\~no, es decir, la condici\'on suficiente para poder
garantizar la convergencia del proceso de la longitud de la cola
as\'i como de por los menos los dos primeros momentos adem\'as de
una versi\'on de la Ley Fuerte de los Grandes N\'umeros para los
sistemas de visitas.


\begin{Teo}[Teorema 2.3, Down \cite{Down}]\label{Tma2.3.Down}
Considere el siguiente valor:
\begin{equation}\label{Eq.Rho.1serv}
\rho=\sum_{k=1}^{K}\rho_{k}+max_{1\leq j\leq K}\left(\frac{\lambda_{j}}{\sum_{s=1}^{S}p_{js}\overline{N}_{s}}\right)\delta^{*}
\end{equation}
\begin{itemize}
\item[i)] Si $\rho<1$ entonces la red es estable, es decir, se
cumple el Teorema \ref{Tma2.1.Down}.

\item[ii)] Si $\rho>1$ entonces la red es inestable, es decir, se
cumple el Teorema \ref{Tma2.2.Down}
\end{itemize}
\end{Teo}

%_________________________________________________________________________
%\section{DESARROLLO DEL TEMA Y/O METODOLOG\'IA}
%_________________________________________________________________________
\subsection{Supuestos}
%_________________________________________________________________________
Consideremos el caso en el que se tienen varias colas a las cuales
llegan uno o varios servidores para dar servicio a los usuarios
que se encuentran presentes en la cola, como ya se mencion\'o hay
varios tipos de pol\'iticas de servicio, incluso podr\'ia ocurrir
que la manera en que atiende al resto de las colas sea distinta a
como lo hizo en las anteriores.\\

Para ejemplificar los sistemas de visitas c\'iclicas se
considerar\'a el caso en que a las colas los usuarios son atendidos con
una s\'ola pol\'itica de servicio.\\


Si $\omega$ es el n\'umero de usuarios en la cola al comienzo del
periodo de servicio y $N\left(\omega\right)$ es el n\'umero de
usuarios que son atendidos con una pol\'itica en espec\'ifico
durante el periodo de servicio, entonces se asume que:
\begin{itemize}
\item[1)]\label{S1}$lim_{\omega\rightarrow\infty}\esp\left[N\left(\omega\right)\right]=\overline{N}>0$;
\item[2)]\label{S2}$\esp\left[N\left(\omega\right)\right]\leq\overline{N}$
para cualquier valor de $\omega$.
\end{itemize}
La manera en que atiende el servidor $m$-\'esimo, es la siguiente:
\begin{itemize}
\item Al t\'ermino de la visita a la cola $j$, el servidor cambia
a la cola $j^{'}$ con probabilidad $r_{j,j^{'}}^{m}$

\item La $n$-\'esima vez que el servidor cambia de la cola $j$ a
$j'$, va acompa\~nada con el tiempo de cambio de longitud
$\delta_{j,j^{'}}^{m}\left(n\right)$, con
$\delta_{j,j^{'}}^{m}\left(n\right)$, $n\geq1$, variables
aleatorias independientes e id\'enticamente distribuidas, tales
que $\esp\left[\delta_{j,j^{'}}^{m}\left(1\right)\right]\geq0$.

\item Sea $\left\{p_{j}^{m}\right\}$ la distribuci\'on invariante
estacionaria \'unica para la Cadena de Markov con matriz de
transici\'on $\left(r_{j,j^{'}}^{m}\right)$, se supone que \'esta
existe.

\item Finalmente, se define el tiempo promedio total de traslado
entre las colas como
\begin{equation}
\delta^{*}:=\sum_{j,j^{'}}p_{j}^{m}r_{j,j^{'}}^{m}\esp\left[\delta_{j,j^{'}}^{m}\left(i\right)\right].
\end{equation}
\end{itemize}

Consideremos el caso donde los tiempos entre arribo a cada una de
las colas, $\left\{\xi_{k}\left(n\right)\right\}_{n\geq1}$ son
variables aleatorias independientes a id\'enticamente
distribuidas, y los tiempos de servicio en cada una de las colas
se distribuyen de manera independiente e id\'enticamente
distribuidas $\left\{\eta_{k}\left(n\right)\right\}_{n\geq1}$;
adem\'as ambos procesos cumplen la condici\'on de ser
independientes entre s\'i. Para la $k$-\'esima cola se define la
tasa de arribo por
$\lambda_{k}=1/\esp\left[\xi_{k}\left(1\right)\right]$ y la tasa
de servicio como
$\mu_{k}=1/\esp\left[\eta_{k}\left(1\right)\right]$, finalmente se
define la carga de la cola como $\rho_{k}=\lambda_{k}/\mu_{k}$,
donde se pide que $\rho=\sum_{k=1}^{K}\rho_{k}<1$, para garantizar
la estabilidad del sistema, esto es cierto para las pol\'iticas de
servicio exhaustiva y cerrada, ver Geetor \cite{Getoor}.\\

Si denotamos por
\begin{itemize}
\item $Q_{k}\left(t\right)$ el n\'umero de usuarios presentes en
la cola $k$ al tiempo $t$; \item $A_{k}\left(t\right)$ los
residuales de los tiempos entre arribos a la cola $k$; para cada
servidor $m$; \item $B_{m}\left(t\right)$ denota a los residuales
de los tiempos de servicio al tiempo $t$; \item
$B_{m}^{0}\left(t\right)$ los residuales de los tiempos de
traslado de la cola $k$ a la pr\'oxima por atender al tiempo $t$,

\item sea
$C_{m}\left(t\right)$ el n\'umero de usuarios atendidos durante la
visita del servidor a la cola $k$ al tiempo $t$.
\end{itemize}


En este sentido, el proceso para el sistema de visitas se puede
definir como:

\begin{equation}\label{Esp.Edos.Down}
X\left(t\right)^{T}=\left(Q_{k}\left(t\right),A_{k}\left(t\right),B_{m}\left(t\right),B_{m}^{0}\left(t\right),C_{m}\left(t\right)\right),
\end{equation}
para $k=1,\ldots,K$ y $m=1,2,\ldots,M$, donde $T$ indica que es el
transpuesto del vector que se est\'a definiendo. El proceso $X$
evoluciona en el espacio de estados:
$\mathbb{X}=\ent_{+}^{K}\times\rea_{+}^{K}\times\left(\left\{1,2,\ldots,K\right\}\times\left\{1,2,\ldots,S\right\}\right)^{M}\times\rea_{+}^{K}\times\ent_{+}^{K}$.\\

El sistema aqu\'i descrito debe de cumplir con los siguientes supuestos b\'asicos de un sistema de visitas:
%__________________________________________________________________________
\subsubsection{Supuestos B\'asicos}
%__________________________________________________________________________
\begin{itemize}
\item[A1)] Los procesos
$\xi_{1},\ldots,\xi_{K},\eta_{1},\ldots,\eta_{K}$ son mutuamente
independientes y son sucesiones independientes e id\'enticamente
distribuidas.

\item[A2)] Para alg\'un entero $p\geq1$
\begin{eqnarray*}
\esp\left[\xi_{l}\left(1\right)^{p+1}\right]&<&\infty\textrm{ para }l=1,\ldots,K\textrm{ y }\\
\esp\left[\eta_{k}\left(1\right)^{p+1}\right]&<&\infty\textrm{
para }k=1,\ldots,K.
\end{eqnarray*}
donde $\mathcal{A}$ es la clase de posibles arribos.

\item[A3)] Para cada $k=1,2,\ldots,K$ existe una funci\'on
positiva $q_{k}\left(\cdot\right)$ definida en $\rea_{+}$, y un
entero $j_{k}$, tal que
\begin{eqnarray}
P\left(\xi_{k}\left(1\right)\geq x\right)&>&0\textrm{, para todo }x>0,\\
P\left\{a\leq\sum_{i=1}^{j_{k}}\xi_{k}\left(i\right)\leq
b\right\}&\geq&\int_{a}^{b}q_{k}\left(x\right)dx, \textrm{ }0\leq
a<b.
\end{eqnarray}
\end{itemize}

En lo que respecta al supuesto (A3), en Dai y Meyn \cite{DaiSean}
hacen ver que este se puede sustituir por

\begin{itemize}
\item[A3')] Para el Proceso de Markov $X$, cada subconjunto
compacto del espacio de estados de $X$ es un conjunto peque\~no,
ver definici\'on \ref{Def.Cto.Peq.}.
\end{itemize}

Es por esta raz\'on que con la finalidad de poder hacer uso de
$A3^{'})$ es necesario recurrir a los Procesos de Harris y en
particular a los Procesos Harris Recurrente, ver \cite{Dai,
DaiSean}.
%_______________________________________________________________________
\subsection{Procesos Harris Recurrente}
%_______________________________________________________________________

Por el supuesto (A1) conforme a Davis \cite{Davis}, se puede
definir el proceso de saltos correspondiente de manera tal que
satisfaga el supuesto (A3'), de hecho la demostraci\'on est\'a
basada en la l\'inea de argumentaci\'on de Davis, \cite{Davis},
p\'aginas 362-364.\\

Entonces se tiene un espacio de estados en el cual el proceso $X$
satisface la Propiedad Fuerte de Markov, ver Dai y Meyn
\cite{DaiSean}, dado por

\[\left(\Omega,\mathcal{F},\mathcal{F}_{t},X\left(t\right),\theta_{t},P_{x}\right),\]
adem\'as de ser un proceso de Borel Derecho (Sharpe \cite{Sharpe})
en el espacio de estados medible
$\left(\mathbb{X},\mathcal{B}_\mathbb{X}\right)$. El Proceso
$X=\left\{X\left(t\right),t\geq0\right\}$ tiene trayectorias
continuas por la derecha, est\'a definido en
$\left(\Omega,\mathcal{F}\right)$ y est\'a adaptado a
$\left\{\mathcal{F}_{t},t\geq0\right\}$; la colecci\'on
$\left\{P_{x},x\in \mathbb{X}\right\}$ son medidas de probabilidad
en $\left(\Omega,\mathcal{F}\right)$ tales que para todo $x\in
\mathbb{X}$
\[P_{x}\left\{X\left(0\right)=x\right\}=1,\] y
\[E_{x}\left\{f\left(X\circ\theta_{t}\right)|\mathcal{F}_{t}\right\}=E_{X}\left(\tau\right)f\left(X\right),\]
en $\left\{\tau<\infty\right\}$, $P_{x}$-c.s., con $\theta_{t}$
definido como el operador shift.


Donde $\tau$ es un $\mathcal{F}_{t}$-tiempo de paro
\[\left(X\circ\theta_{\tau}\right)\left(w\right)=\left\{X\left(\tau\left(w\right)+t,w\right),t\geq0\right\},\]
y $f$ es una funci\'on de valores reales acotada y medible, ver \cite{Dai, KaspiMandelbaum}.\\

Sea $P^{t}\left(x,D\right)$, $D\in\mathcal{B}_{\mathbb{X}}$,
$t\geq0$ la probabilidad de transici\'on de $X$ queda definida
como:
\[P^{t}\left(x,D\right)=P_{x}\left(X\left(t\right)\in
D\right).\]


\begin{Def}
Una medida no cero $\pi$ en
$\left(\mathbb{X},\mathcal{B}_{\mathbb{X}}\right)$ es invariante
para $X$ si $\pi$ es $\sigma$-finita y
\[\pi\left(D\right)=\int_{\mathbb{X}}P^{t}\left(x,D\right)\pi\left(dx\right),\]
para todo $D\in \mathcal{B}_{\mathbb{X}}$, con $t\geq0$.
\end{Def}

\begin{Def}
El proceso de Markov $X$ es llamado Harris Recurrente si existe
una medida de probabilidad $\nu$ en
$\left(\mathbb{X},\mathcal{B}_{\mathbb{X}}\right)$, tal que si
$\nu\left(D\right)>0$ y $D\in\mathcal{B}_{\mathbb{X}}$
\[P_{x}\left\{\tau_{D}<\infty\right\}\equiv1,\] cuando
$\tau_{D}=inf\left\{t\geq0:X_{t}\in D\right\}$.
\end{Def}

\begin{Note}
\begin{itemize}
\item[i)] Si $X$ es Harris recurrente, entonces existe una \'unica
medida invariante $\pi$ (Getoor \cite{Getoor}).

\item[ii)] Si la medida invariante es finita, entonces puede
normalizarse a una medida de probabilidad, en este caso al proceso
$X$ se le llama Harris recurrente positivo.


\item[iii)] Cuando $X$ es Harris recurrente positivo se dice que
la disciplina de servicio es estable. En este caso $\pi$ denota la
distribuci\'on estacionaria y hacemos
\[P_{\pi}\left(\cdot\right)=\int_{\mathbf{X}}P_{x}\left(\cdot\right)\pi\left(dx\right),\]
y se utiliza $E_{\pi}$ para denotar el operador esperanza
correspondiente, ver \cite{DaiSean}.
\end{itemize}
\end{Note}

\begin{Def}\label{Def.Cto.Peq.}
Un conjunto $D\in\mathcal{B_{\mathbb{X}}}$ es llamado peque\~no si
existe un $t>0$, una medida de probabilidad $\nu$ en
$\mathcal{B_{\mathbb{X}}}$, y un $\delta>0$ tal que
\[P^{t}\left(x,A\right)\geq\delta\nu\left(A\right),\] para $x\in
D,A\in\mathcal{B_{\mathbb{X}}}$.
\end{Def}

La siguiente serie de resultados vienen enunciados y demostrados
en Dai \cite{Dai}:
\begin{Lema}[Lema 3.1, Dai \cite{Dai}]
Sea $B$ conjunto peque\~no cerrado, supongamos que
$P_{x}\left(\tau_{B}<\infty\right)\equiv1$ y que para alg\'un
$\delta>0$ se cumple que
\begin{equation}\label{Eq.3.1}
\sup\esp_{x}\left[\tau_{B}\left(\delta\right)\right]<\infty,
\end{equation}
donde
$\tau_{B}\left(\delta\right)=inf\left\{t\geq\delta:X\left(t\right)\in
B\right\}$. Entonces, $X$ es un proceso Harris recurrente
positivo.
\end{Lema}

\begin{Lema}[Lema 3.1, Dai \cite{Dai}]\label{Lema.3.}
Bajo el supuesto (A3), el conjunto
$B=\left\{x\in\mathbb{X}:|x|\leq k\right\}$ es un conjunto
peque\~no cerrado para cualquier $k>0$.
\end{Lema}

\begin{Teo}[Teorema 3.1, Dai \cite{Dai}]\label{Tma.3.1}
Si existe un $\delta>0$ tal que
\begin{equation}
lim_{|x|\rightarrow\infty}\frac{1}{|x|}\esp|X^{x}\left(|x|\delta\right)|=0,
\end{equation}
donde $X^{x}$ se utiliza para denotar que el proceso $X$ comienza
a partir de $x$, entonces la ecuaci\'on (\ref{Eq.3.1}) se cumple
para $B=\left\{x\in\mathbb{X}:|x|\leq k\right\}$ con alg\'un
$k>0$. En particular, $X$ es Harris recurrente positivo.
\end{Teo}

Entonces, tenemos que el proceso $X$ es un proceso de Markov que
cumple con los supuestos $A1)$-$A3)$, lo que falta de hacer es
construir el Modelo de Flujo bas\'andonos en lo hasta ahora
presentado.
%_______________________________________________________________________
\subsection{Modelo de Flujo}
%_______________________________________________________________________

Dada una condici\'on inicial $x\in\mathbb{X}$, sea

\begin{itemize}
\item $Q_{k}^{x}\left(t\right)$ la longitud de la cola al tiempo
$t$,

\item $T_{m,k}^{x}\left(t\right)$ el tiempo acumulado, al tiempo
$t$, que tarda el servidor $m$ en atender a los usuarios de la
cola $k$.

\item $T_{m,k}^{x,0}\left(t\right)$ el tiempo acumulado, al tiempo
$t$, que tarda el servidor $m$ en trasladarse a otra cola a partir de la $k$-\'esima.\\
\end{itemize}

Sup\'ongase que la funci\'on
$\left(\overline{Q}\left(\cdot\right),\overline{T}_{m}
\left(\cdot\right),\overline{T}_{m}^{0} \left(\cdot\right)\right)$
para $m=1,2,\ldots,M$ es un punto l\'imite de
\begin{equation}\label{Eq.Punto.Limite}
\left(\frac{1}{|x|}Q^{x}\left(|x|t\right),\frac{1}{|x|}T_{m}^{x}\left(|x|t\right),\frac{1}{|x|}T_{m}^{x,0}\left(|x|t\right)\right)
\end{equation}
para $m=1,2,\ldots,M$, cuando $x\rightarrow\infty$, ver
\cite{Down}. Entonces
$\left(\overline{Q}\left(t\right),\overline{T}_{m}
\left(t\right),\overline{T}_{m}^{0} \left(t\right)\right)$ es un
flujo l\'imite del sistema. Al conjunto de todos las posibles
flujos l\'imite se le llama {\emph{Modelo de Flujo}} y se le
denotar\'a por $\mathcal{Q}$, ver \cite{Down, Dai, DaiSean}.\\

El modelo de flujo satisface el siguiente conjunto de ecuaciones:

\begin{equation}\label{Eq.MF.1}
\overline{Q}_{k}\left(t\right)=\overline{Q}_{k}\left(0\right)+\lambda_{k}t-\sum_{m=1}^{M}\mu_{k}\overline{T}_{m,k}\left(t\right),\\
\end{equation}
para $k=1,2,\ldots,K$.\\
\begin{equation}\label{Eq.MF.2}
\overline{Q}_{k}\left(t\right)\geq0\textrm{ para
}k=1,2,\ldots,K.\\
\end{equation}

\begin{equation}\label{Eq.MF.3}
\overline{T}_{m,k}\left(0\right)=0,\textrm{ y }\overline{T}_{m,k}\left(\cdot\right)\textrm{ es no decreciente},\\
\end{equation}
para $k=1,2,\ldots,K$ y $m=1,2,\ldots,M$.\\
\begin{equation}\label{Eq.MF.4}
\sum_{k=1}^{K}\overline{T}_{m,k}^{0}\left(t\right)+\overline{T}_{m,k}\left(t\right)=t\textrm{
para }m=1,2,\ldots,M.\\
\end{equation}


\begin{Def}[Definici\'on 4.1, Dai \cite{Dai}]\label{Def.Modelo.Flujo}
Sea una disciplina de servicio espec\'ifica. Cualquier l\'imite
$\left(\overline{Q}\left(\cdot\right),\overline{T}\left(\cdot\right),\overline{T}^{0}\left(\cdot\right)\right)$
en (\ref{Eq.Punto.Limite}) es un {\em flujo l\'imite} de la
disciplina. Cualquier soluci\'on (\ref{Eq.MF.1})-(\ref{Eq.MF.4})
es llamado flujo soluci\'on de la disciplina.
\end{Def}

\begin{Def}
Se dice que el modelo de flujo l\'imite, modelo de flujo, de la
disciplina de la cola es estable si existe una constante
$\delta>0$ que depende de $\mu,\lambda$ y $P$ solamente, tal que
cualquier flujo l\'imite con
$|\overline{Q}\left(0\right)|+|\overline{U}|+|\overline{V}|=1$, se
tiene que $\overline{Q}\left(\cdot+\delta\right)\equiv0$.
\end{Def}

Si se hace $|x|\rightarrow\infty$ sin restringir ninguna de las
componentes, tambi\'en se obtienen un modelo de flujo, pero en
este caso el residual de los procesos de arribo y servicio
introducen un retraso:
\begin{Teo}[Teorema 4.2, Dai \cite{Dai}]\label{Tma.4.2.Dai}
Sea una disciplina fija para la cola, suponga que se cumplen las
condiciones (A1)-(A3). Si el modelo de flujo l\'imite de la
disciplina de la cola es estable, entonces la cadena de Markov $X$
que describe la din\'amica de la red bajo la disciplina es Harris
recurrente positiva.
\end{Teo}

Ahora se procede a escalar el espacio y el tiempo para reducir la
aparente fluctuaci\'on del modelo. Consid\'erese el proceso
\begin{equation}\label{Eq.3.7}
\overline{Q}^{x}\left(t\right)=\frac{1}{|x|}Q^{x}\left(|x|t\right).
\end{equation}
A este proceso se le conoce como el flujo escalado, y cualquier
l\'imite $\overline{Q}^{x}\left(t\right)$ es llamado flujo
l\'imite del proceso de longitud de la cola. Haciendo
$|q|\rightarrow\infty$ mientras se mantiene el resto de las
componentes fijas, cualquier punto l\'imite del proceso de
longitud de la cola normalizado $\overline{Q}^{x}$ es soluci\'on
del siguiente modelo de flujo.


\begin{Def}[Definici\'on 3.3, Dai y Meyn \cite{DaiSean}]
El modelo de flujo es estable si existe un tiempo fijo $t_{0}$ tal
que $\overline{Q}\left(t\right)=0$, con $t\geq t_{0}$, para
cualquier $\overline{Q}\left(\cdot\right)\in\mathcal{Q}$ que
cumple con $|\overline{Q}\left(0\right)|=1$.
\end{Def}

\begin{Lemma}[Lema 3.1, Dai y Meyn \cite{DaiSean}]
Si el modelo de flujo definido por (\ref{Eq.MF.1})-(\ref{Eq.MF.4})
es estable, entonces el modelo de flujo retrasado es tambi\'en
estable, es decir, existe $t_{0}>0$ tal que
$\overline{Q}\left(t\right)=0$ para cualquier $t\geq t_{0}$, para
cualquier soluci\'on del modelo de flujo retrasado cuya
condici\'on inicial $\overline{x}$ satisface que
$|\overline{x}|=|\overline{Q}\left(0\right)|+|\overline{A}\left(0\right)|+|\overline{B}\left(0\right)|\leq1$.
\end{Lemma}


Ahora ya estamos en condiciones de enunciar los resultados principales:


\begin{Teo}[Teorema 2.1, Down \cite{Down}]\label{Tma2.1.Down}
Suponga que el modelo de flujo es estable, y que se cumplen los supuestos (A1) y (A2), entonces
\begin{itemize}
\item[i)] Para alguna constante $\kappa_{p}$, y para cada
condici\'on inicial $x\in X$
\begin{equation}\label{Estability.Eq1}
\limsup_{t\rightarrow\infty}\frac{1}{t}\int_{0}^{t}\esp_{x}\left[|Q\left(s\right)|^{p}\right]ds\leq\kappa_{p},
\end{equation}
donde $p$ es el entero dado en (A2).
\end{itemize}
Si adem\'as se cumple la condici\'on (A3), entonces para cada
condici\'on inicial:
\begin{itemize}
\item[ii)] Los momentos transitorios convergen a su estado
estacionario:
 \begin{equation}\label{Estability.Eq2}
lim_{t\rightarrow\infty}\esp_{x}\left[Q_{k}\left(t\right)^{r}\right]=\esp_{\pi}\left[Q_{k}\left(0\right)^{r}\right]\leq\kappa_{r},
\end{equation}
para $r=1,2,\ldots,p$ y $k=1,2,\ldots,K$. Donde $\pi$ es la
probabilidad invariante para $X$.

\item[iii)]  El primer momento converge con raz\'on $t^{p-1}$:
\begin{equation}\label{Estability.Eq3}
lim_{t\rightarrow\infty}t^{p-1}|\esp_{x}\left[Q_{k}\left(t\right)\right]-\esp_{\pi}\left[Q_{k}\left(0\right)\right]|=0.
\end{equation}

\item[iv)] La {\em Ley Fuerte de los grandes n\'umeros} se cumple:
\begin{equation}\label{Estability.Eq4}
lim_{t\rightarrow\infty}\frac{1}{t}\int_{0}^{t}Q_{k}^{r}\left(s\right)ds=\esp_{\pi}\left[Q_{k}\left(0\right)^{r}\right],\textrm{
}\prob_{x}\textrm{-c.s.}
\end{equation}
para $r=1,2,\ldots,p$ y $k=1,2,\ldots,K$.
\end{itemize}
\end{Teo}

La contribuci\'on de Down a la teor\'ia de los {\emph {sistemas de
visitas c\'iclicas}}, es la relaci\'on que hay entre la
estabilidad del sistema con el comportamiento de las medidas de
desempe\~no, es decir, la condici\'on suficiente para poder
garantizar la convergencia del proceso de la longitud de la cola
as\'i como de por los menos los dos primeros momentos adem\'as de
una versi\'on de la Ley Fuerte de los Grandes N\'umeros para los
sistemas de visitas.


\begin{Teo}[Teorema 2.3, Down \cite{Down}]\label{Tma2.3.Down}
Considere el siguiente valor:
\begin{equation}\label{Eq.Rho.1serv}
\rho=\sum_{k=1}^{K}\rho_{k}+max_{1\leq j\leq K}\left(\frac{\lambda_{j}}{\sum_{s=1}^{S}p_{js}\overline{N}_{s}}\right)\delta^{*}
\end{equation}
\begin{itemize}
\item[i)] Si $\rho<1$ entonces la red es estable, es decir, se
cumple el Teorema \ref{Tma2.1.Down}.

\item[ii)] Si $\rho>1$ entonces la red es inestable, es decir, se
cumple el Teorema \ref{Tma2.2.Down}
\end{itemize}
\end{Teo}



El sistema aqu\'i descrito debe de cumplir con los siguientes supuestos b\'asicos de un sistema de visitas:
%__________________________________________________________________________
\subsubsection{Supuestos B\'asicos}
%__________________________________________________________________________
\begin{itemize}
\item[A1)] Los procesos
$\xi_{1},\ldots,\xi_{K},\eta_{1},\ldots,\eta_{K}$ son mutuamente
independientes y son sucesiones independientes e id\'enticamente
distribuidas.

\item[A2)] Para alg\'un entero $p\geq1$
\begin{eqnarray*}
\esp\left[\xi_{l}\left(1\right)^{p+1}\right]&<&\infty\textrm{ para }l=1,\ldots,K\textrm{ y }\\
\esp\left[\eta_{k}\left(1\right)^{p+1}\right]&<&\infty\textrm{
para }k=1,\ldots,K.
\end{eqnarray*}
donde $\mathcal{A}$ es la clase de posibles arribos.

\item[A3)] Para cada $k=1,2,\ldots,K$ existe una funci\'on
positiva $q_{k}\left(\cdot\right)$ definida en $\rea_{+}$, y un
entero $j_{k}$, tal que
\begin{eqnarray}
P\left(\xi_{k}\left(1\right)\geq x\right)&>&0\textrm{, para todo }x>0,\\
P\left\{a\leq\sum_{i=1}^{j_{k}}\xi_{k}\left(i\right)\leq
b\right\}&\geq&\int_{a}^{b}q_{k}\left(x\right)dx, \textrm{ }0\leq
a<b.
\end{eqnarray}
\end{itemize}

En lo que respecta al supuesto (A3), en Dai y Meyn \cite{DaiSean}
hacen ver que este se puede sustituir por

\begin{itemize}
\item[A3')] Para el Proceso de Markov $X$, cada subconjunto
compacto del espacio de estados de $X$ es un conjunto peque\~no,
ver definici\'on \ref{Def.Cto.Peq.}.
\end{itemize}

Es por esta raz\'on que con la finalidad de poder hacer uso de
$A3^{'})$ es necesario recurrir a los Procesos de Harris y en
particular a los Procesos Harris Recurrente, ver \cite{Dai,
DaiSean}.
%_______________________________________________________________________
\subsection{Procesos Harris Recurrente}
%_______________________________________________________________________

Por el supuesto (A1) conforme a Davis \cite{Davis}, se puede
definir el proceso de saltos correspondiente de manera tal que
satisfaga el supuesto (A3'), de hecho la demostraci\'on est\'a
basada en la l\'inea de argumentaci\'on de Davis, \cite{Davis},
p\'aginas 362-364.\\

Entonces se tiene un espacio de estados en el cual el proceso $X$
satisface la Propiedad Fuerte de Markov, ver Dai y Meyn
\cite{DaiSean}, dado por

\[\left(\Omega,\mathcal{F},\mathcal{F}_{t},X\left(t\right),\theta_{t},P_{x}\right),\]
adem\'as de ser un proceso de Borel Derecho (Sharpe \cite{Sharpe})
en el espacio de estados medible
$\left(\mathbb{X},\mathcal{B}_\mathbb{X}\right)$. El Proceso
$X=\left\{X\left(t\right),t\geq0\right\}$ tiene trayectorias
continuas por la derecha, est\'a definido en
$\left(\Omega,\mathcal{F}\right)$ y est\'a adaptado a
$\left\{\mathcal{F}_{t},t\geq0\right\}$; la colecci\'on
$\left\{P_{x},x\in \mathbb{X}\right\}$ son medidas de probabilidad
en $\left(\Omega,\mathcal{F}\right)$ tales que para todo $x\in
\mathbb{X}$
\[P_{x}\left\{X\left(0\right)=x\right\}=1,\] y
\[E_{x}\left\{f\left(X\circ\theta_{t}\right)|\mathcal{F}_{t}\right\}=E_{X}\left(\tau\right)f\left(X\right),\]
en $\left\{\tau<\infty\right\}$, $P_{x}$-c.s., con $\theta_{t}$
definido como el operador shift.


Donde $\tau$ es un $\mathcal{F}_{t}$-tiempo de paro
\[\left(X\circ\theta_{\tau}\right)\left(w\right)=\left\{X\left(\tau\left(w\right)+t,w\right),t\geq0\right\},\]
y $f$ es una funci\'on de valores reales acotada y medible, ver \cite{Dai, KaspiMandelbaum}.\\

Sea $P^{t}\left(x,D\right)$, $D\in\mathcal{B}_{\mathbb{X}}$,
$t\geq0$ la probabilidad de transici\'on de $X$ queda definida
como:
\[P^{t}\left(x,D\right)=P_{x}\left(X\left(t\right)\in
D\right).\]


\begin{Def}
Una medida no cero $\pi$ en
$\left(\mathbb{X},\mathcal{B}_{\mathbb{X}}\right)$ es invariante
para $X$ si $\pi$ es $\sigma$-finita y
\[\pi\left(D\right)=\int_{\mathbb{X}}P^{t}\left(x,D\right)\pi\left(dx\right),\]
para todo $D\in \mathcal{B}_{\mathbb{X}}$, con $t\geq0$.
\end{Def}

\begin{Def}
El proceso de Markov $X$ es llamado Harris Recurrente si existe
una medida de probabilidad $\nu$ en
$\left(\mathbb{X},\mathcal{B}_{\mathbb{X}}\right)$, tal que si
$\nu\left(D\right)>0$ y $D\in\mathcal{B}_{\mathbb{X}}$
\[P_{x}\left\{\tau_{D}<\infty\right\}\equiv1,\] cuando
$\tau_{D}=inf\left\{t\geq0:X_{t}\in D\right\}$.
\end{Def}

\begin{Note}
\begin{itemize}
\item[i)] Si $X$ es Harris recurrente, entonces existe una \'unica
medida invariante $\pi$ (Getoor \cite{Getoor}).

\item[ii)] Si la medida invariante es finita, entonces puede
normalizarse a una medida de probabilidad, en este caso al proceso
$X$ se le llama Harris recurrente positivo.


\item[iii)] Cuando $X$ es Harris recurrente positivo se dice que
la disciplina de servicio es estable. En este caso $\pi$ denota la
distribuci\'on estacionaria y hacemos
\[P_{\pi}\left(\cdot\right)=\int_{\mathbf{X}}P_{x}\left(\cdot\right)\pi\left(dx\right),\]
y se utiliza $E_{\pi}$ para denotar el operador esperanza
correspondiente, ver \cite{DaiSean}.
\end{itemize}
\end{Note}

\begin{Def}\label{Def.Cto.Peq.}
Un conjunto $D\in\mathcal{B_{\mathbb{X}}}$ es llamado peque\~no si
existe un $t>0$, una medida de probabilidad $\nu$ en
$\mathcal{B_{\mathbb{X}}}$, y un $\delta>0$ tal que
\[P^{t}\left(x,A\right)\geq\delta\nu\left(A\right),\] para $x\in
D,A\in\mathcal{B_{\mathbb{X}}}$.
\end{Def}

La siguiente serie de resultados vienen enunciados y demostrados
en Dai \cite{Dai}:
\begin{Lema}[Lema 3.1, Dai \cite{Dai}]
Sea $B$ conjunto peque\~no cerrado, supongamos que
$P_{x}\left(\tau_{B}<\infty\right)\equiv1$ y que para alg\'un
$\delta>0$ se cumple que
\begin{equation}\label{Eq.3.1}
\sup\esp_{x}\left[\tau_{B}\left(\delta\right)\right]<\infty,
\end{equation}
donde
$\tau_{B}\left(\delta\right)=inf\left\{t\geq\delta:X\left(t\right)\in
B\right\}$. Entonces, $X$ es un proceso Harris recurrente
positivo.
\end{Lema}

\begin{Lema}[Lema 3.1, Dai \cite{Dai}]\label{Lema.3.}
Bajo el supuesto (A3), el conjunto
$B=\left\{x\in\mathbb{X}:|x|\leq k\right\}$ es un conjunto
peque\~no cerrado para cualquier $k>0$.
\end{Lema}

\begin{Teo}[Teorema 3.1, Dai \cite{Dai}]\label{Tma.3.1}
Si existe un $\delta>0$ tal que
\begin{equation}
lim_{|x|\rightarrow\infty}\frac{1}{|x|}\esp|X^{x}\left(|x|\delta\right)|=0,
\end{equation}
donde $X^{x}$ se utiliza para denotar que el proceso $X$ comienza
a partir de $x$, entonces la ecuaci\'on (\ref{Eq.3.1}) se cumple
para $B=\left\{x\in\mathbb{X}:|x|\leq k\right\}$ con alg\'un
$k>0$. En particular, $X$ es Harris recurrente positivo.
\end{Teo}

Entonces, tenemos que el proceso $X$ es un proceso de Markov que
cumple con los supuestos $A1)$-$A3)$, lo que falta de hacer es
construir el Modelo de Flujo bas\'andonos en lo hasta ahora
presentado.
%_______________________________________________________________________
\subsection{Modelo de Flujo}
%_______________________________________________________________________

Dada una condici\'on inicial $x\in\mathbb{X}$, sea

\begin{itemize}
\item $Q_{k}^{x}\left(t\right)$ la longitud de la cola al tiempo
$t$,

\item $T_{m,k}^{x}\left(t\right)$ el tiempo acumulado, al tiempo
$t$, que tarda el servidor $m$ en atender a los usuarios de la
cola $k$.

\item $T_{m,k}^{x,0}\left(t\right)$ el tiempo acumulado, al tiempo
$t$, que tarda el servidor $m$ en trasladarse a otra cola a partir de la $k$-\'esima.\\
\end{itemize}

Sup\'ongase que la funci\'on
$\left(\overline{Q}\left(\cdot\right),\overline{T}_{m}
\left(\cdot\right),\overline{T}_{m}^{0} \left(\cdot\right)\right)$
para $m=1,2,\ldots,M$ es un punto l\'imite de
\begin{equation}\label{Eq.Punto.Limite}
\left(\frac{1}{|x|}Q^{x}\left(|x|t\right),\frac{1}{|x|}T_{m}^{x}\left(|x|t\right),\frac{1}{|x|}T_{m}^{x,0}\left(|x|t\right)\right)
\end{equation}
para $m=1,2,\ldots,M$, cuando $x\rightarrow\infty$, ver
\cite{Down}. Entonces
$\left(\overline{Q}\left(t\right),\overline{T}_{m}
\left(t\right),\overline{T}_{m}^{0} \left(t\right)\right)$ es un
flujo l\'imite del sistema. Al conjunto de todos las posibles
flujos l\'imite se le llama {\emph{Modelo de Flujo}} y se le
denotar\'a por $\mathcal{Q}$, ver \cite{Down, Dai, DaiSean}.\\

El modelo de flujo satisface el siguiente conjunto de ecuaciones:

\begin{equation}\label{Eq.MF.1}
\overline{Q}_{k}\left(t\right)=\overline{Q}_{k}\left(0\right)+\lambda_{k}t-\sum_{m=1}^{M}\mu_{k}\overline{T}_{m,k}\left(t\right),\\
\end{equation}
para $k=1,2,\ldots,K$.\\
\begin{equation}\label{Eq.MF.2}
\overline{Q}_{k}\left(t\right)\geq0\textrm{ para
}k=1,2,\ldots,K.\\
\end{equation}

\begin{equation}\label{Eq.MF.3}
\overline{T}_{m,k}\left(0\right)=0,\textrm{ y }\overline{T}_{m,k}\left(\cdot\right)\textrm{ es no decreciente},\\
\end{equation}
para $k=1,2,\ldots,K$ y $m=1,2,\ldots,M$.\\
\begin{equation}\label{Eq.MF.4}
\sum_{k=1}^{K}\overline{T}_{m,k}^{0}\left(t\right)+\overline{T}_{m,k}\left(t\right)=t\textrm{
para }m=1,2,\ldots,M.\\
\end{equation}


\begin{Def}[Definici\'on 4.1, Dai \cite{Dai}]\label{Def.Modelo.Flujo}
Sea una disciplina de servicio espec\'ifica. Cualquier l\'imite
$\left(\overline{Q}\left(\cdot\right),\overline{T}\left(\cdot\right),\overline{T}^{0}\left(\cdot\right)\right)$
en (\ref{Eq.Punto.Limite}) es un {\em flujo l\'imite} de la
disciplina. Cualquier soluci\'on (\ref{Eq.MF.1})-(\ref{Eq.MF.4})
es llamado flujo soluci\'on de la disciplina.
\end{Def}

\begin{Def}
Se dice que el modelo de flujo l\'imite, modelo de flujo, de la
disciplina de la cola es estable si existe una constante
$\delta>0$ que depende de $\mu,\lambda$ y $P$ solamente, tal que
cualquier flujo l\'imite con
$|\overline{Q}\left(0\right)|+|\overline{U}|+|\overline{V}|=1$, se
tiene que $\overline{Q}\left(\cdot+\delta\right)\equiv0$.
\end{Def}

Si se hace $|x|\rightarrow\infty$ sin restringir ninguna de las
componentes, tambi\'en se obtienen un modelo de flujo, pero en
este caso el residual de los procesos de arribo y servicio
introducen un retraso:
\begin{Teo}[Teorema 4.2, Dai \cite{Dai}]\label{Tma.4.2.Dai}
Sea una disciplina fija para la cola, suponga que se cumplen las
condiciones (A1)-(A3). Si el modelo de flujo l\'imite de la
disciplina de la cola es estable, entonces la cadena de Markov $X$
que describe la din\'amica de la red bajo la disciplina es Harris
recurrente positiva.
\end{Teo}

Ahora se procede a escalar el espacio y el tiempo para reducir la
aparente fluctuaci\'on del modelo. Consid\'erese el proceso
\begin{equation}\label{Eq.3.7}
\overline{Q}^{x}\left(t\right)=\frac{1}{|x|}Q^{x}\left(|x|t\right).
\end{equation}
A este proceso se le conoce como el flujo escalado, y cualquier
l\'imite $\overline{Q}^{x}\left(t\right)$ es llamado flujo
l\'imite del proceso de longitud de la cola. Haciendo
$|q|\rightarrow\infty$ mientras se mantiene el resto de las
componentes fijas, cualquier punto l\'imite del proceso de
longitud de la cola normalizado $\overline{Q}^{x}$ es soluci\'on
del siguiente modelo de flujo.


\begin{Def}[Definici\'on 3.3, Dai y Meyn \cite{DaiSean}]
El modelo de flujo es estable si existe un tiempo fijo $t_{0}$ tal
que $\overline{Q}\left(t\right)=0$, con $t\geq t_{0}$, para
cualquier $\overline{Q}\left(\cdot\right)\in\mathcal{Q}$ que
cumple con $|\overline{Q}\left(0\right)|=1$.
\end{Def}

\begin{Lemma}[Lema 3.1, Dai y Meyn \cite{DaiSean}]
Si el modelo de flujo definido por (\ref{Eq.MF.1})-(\ref{Eq.MF.4})
es estable, entonces el modelo de flujo retrasado es tambi\'en
estable, es decir, existe $t_{0}>0$ tal que
$\overline{Q}\left(t\right)=0$ para cualquier $t\geq t_{0}$, para
cualquier soluci\'on del modelo de flujo retrasado cuya
condici\'on inicial $\overline{x}$ satisface que
$|\overline{x}|=|\overline{Q}\left(0\right)|+|\overline{A}\left(0\right)|+|\overline{B}\left(0\right)|\leq1$.
\end{Lemma}


Ahora ya estamos en condiciones de enunciar los resultados principales:


\begin{Teo}[Teorema 2.1, Down \cite{Down}]\label{Tma2.1.Down}
Suponga que el modelo de flujo es estable, y que se cumplen los supuestos (A1) y (A2), entonces
\begin{itemize}
\item[i)] Para alguna constante $\kappa_{p}$, y para cada
condici\'on inicial $x\in X$
\begin{equation}\label{Estability.Eq1}
\limsup_{t\rightarrow\infty}\frac{1}{t}\int_{0}^{t}\esp_{x}\left[|Q\left(s\right)|^{p}\right]ds\leq\kappa_{p},
\end{equation}
donde $p$ es el entero dado en (A2).
\end{itemize}
Si adem\'as se cumple la condici\'on (A3), entonces para cada
condici\'on inicial:
\begin{itemize}
\item[ii)] Los momentos transitorios convergen a su estado
estacionario:
 \begin{equation}\label{Estability.Eq2}
lim_{t\rightarrow\infty}\esp_{x}\left[Q_{k}\left(t\right)^{r}\right]=\esp_{\pi}\left[Q_{k}\left(0\right)^{r}\right]\leq\kappa_{r},
\end{equation}
para $r=1,2,\ldots,p$ y $k=1,2,\ldots,K$. Donde $\pi$ es la
probabilidad invariante para $X$.

\item[iii)]  El primer momento converge con raz\'on $t^{p-1}$:
\begin{equation}\label{Estability.Eq3}
lim_{t\rightarrow\infty}t^{p-1}|\esp_{x}\left[Q_{k}\left(t\right)\right]-\esp_{\pi}\left[Q_{k}\left(0\right)\right]|=0.
\end{equation}

\item[iv)] La {\em Ley Fuerte de los grandes n\'umeros} se cumple:
\begin{equation}\label{Estability.Eq4}
lim_{t\rightarrow\infty}\frac{1}{t}\int_{0}^{t}Q_{k}^{r}\left(s\right)ds=\esp_{\pi}\left[Q_{k}\left(0\right)^{r}\right],\textrm{
}\prob_{x}\textrm{-c.s.}
\end{equation}
para $r=1,2,\ldots,p$ y $k=1,2,\ldots,K$.
\end{itemize}
\end{Teo}

La contribuci\'on de Down a la teor\'ia de los {\emph {sistemas de
visitas c\'iclicas}}, es la relaci\'on que hay entre la
estabilidad del sistema con el comportamiento de las medidas de
desempe\~no, es decir, la condici\'on suficiente para poder
garantizar la convergencia del proceso de la longitud de la cola
as\'i como de por los menos los dos primeros momentos adem\'as de
una versi\'on de la Ley Fuerte de los Grandes N\'umeros para los
sistemas de visitas.


\begin{Teo}[Teorema 2.3, Down \cite{Down}]\label{Tma2.3.Down}
Considere el siguiente valor:
\begin{equation}\label{Eq.Rho.1serv}
\rho=\sum_{k=1}^{K}\rho_{k}+max_{1\leq j\leq K}\left(\frac{\lambda_{j}}{\sum_{s=1}^{S}p_{js}\overline{N}_{s}}\right)\delta^{*}
\end{equation}
\begin{itemize}
\item[i)] Si $\rho<1$ entonces la red es estable, es decir, se
cumple el Teorema \ref{Tma2.1.Down}.

\item[ii)] Si $\rho>1$ entonces la red es inestable, es decir, se
cumple el Teorema \ref{Tma2.2.Down}
\end{itemize}
\end{Teo}


%_________________________________________________________________________
\subsection{Supuestos}
%_________________________________________________________________________
Consideremos el caso en el que se tienen varias colas a las cuales
llegan uno o varios servidores para dar servicio a los usuarios
que se encuentran presentes en la cola, como ya se mencion\'o hay
varios tipos de pol\'iticas de servicio, incluso podr\'ia ocurrir
que la manera en que atiende al resto de las colas sea distinta a
como lo hizo en las anteriores.\\

Para ejemplificar los sistemas de visitas c\'iclicas se
considerar\'a el caso en que a las colas los usuarios son atendidos con
una s\'ola pol\'itica de servicio.\\



Si $\omega$ es el n\'umero de usuarios en la cola al comienzo del
periodo de servicio y $N\left(\omega\right)$ es el n\'umero de
usuarios que son atendidos con una pol\'itica en espec\'ifico
durante el periodo de servicio, entonces se asume que:
\begin{itemize}
\item[1)]\label{S1}$lim_{\omega\rightarrow\infty}\esp\left[N\left(\omega\right)\right]=\overline{N}>0$;
\item[2)]\label{S2}$\esp\left[N\left(\omega\right)\right]\leq\overline{N}$
para cualquier valor de $\omega$.
\end{itemize}
La manera en que atiende el servidor $m$-\'esimo, es la siguiente:
\begin{itemize}
\item Al t\'ermino de la visita a la cola $j$, el servidor cambia
a la cola $j^{'}$ con probabilidad $r_{j,j^{'}}^{m}$

\item La $n$-\'esima vez que el servidor cambia de la cola $j$ a
$j'$, va acompa\~nada con el tiempo de cambio de longitud
$\delta_{j,j^{'}}^{m}\left(n\right)$, con
$\delta_{j,j^{'}}^{m}\left(n\right)$, $n\geq1$, variables
aleatorias independientes e id\'enticamente distribuidas, tales
que $\esp\left[\delta_{j,j^{'}}^{m}\left(1\right)\right]\geq0$.

\item Sea $\left\{p_{j}^{m}\right\}$ la distribuci\'on invariante
estacionaria \'unica para la Cadena de Markov con matriz de
transici\'on $\left(r_{j,j^{'}}^{m}\right)$, se supone que \'esta
existe.

\item Finalmente, se define el tiempo promedio total de traslado
entre las colas como
\begin{equation}
\delta^{*}:=\sum_{j,j^{'}}p_{j}^{m}r_{j,j^{'}}^{m}\esp\left[\delta_{j,j^{'}}^{m}\left(i\right)\right].
\end{equation}
\end{itemize}

Consideremos el caso donde los tiempos entre arribo a cada una de
las colas, $\left\{\xi_{k}\left(n\right)\right\}_{n\geq1}$ son
variables aleatorias independientes a id\'enticamente
distribuidas, y los tiempos de servicio en cada una de las colas
se distribuyen de manera independiente e id\'enticamente
distribuidas $\left\{\eta_{k}\left(n\right)\right\}_{n\geq1}$;
adem\'as ambos procesos cumplen la condici\'on de ser
independientes entre s\'i. Para la $k$-\'esima cola se define la
tasa de arribo por
$\lambda_{k}=1/\esp\left[\xi_{k}\left(1\right)\right]$ y la tasa
de servicio como
$\mu_{k}=1/\esp\left[\eta_{k}\left(1\right)\right]$, finalmente se
define la carga de la cola como $\rho_{k}=\lambda_{k}/\mu_{k}$,
donde se pide que $\rho=\sum_{k=1}^{K}\rho_{k}<1$, para garantizar
la estabilidad del sistema, esto es cierto para las pol\'iticas de
servicio exhaustiva y cerrada, ver Geetor \cite{Getoor}.\\

Si denotamos por
\begin{itemize}
\item $Q_{k}\left(t\right)$ el n\'umero de usuarios presentes en
la cola $k$ al tiempo $t$; \item $A_{k}\left(t\right)$ los
residuales de los tiempos entre arribos a la cola $k$; para cada
servidor $m$; \item $B_{m}\left(t\right)$ denota a los residuales
de los tiempos de servicio al tiempo $t$; \item
$B_{m}^{0}\left(t\right)$ los residuales de los tiempos de
traslado de la cola $k$ a la pr\'oxima por atender al tiempo $t$,

\item sea
$C_{m}\left(t\right)$ el n\'umero de usuarios atendidos durante la
visita del servidor a la cola $k$ al tiempo $t$.
\end{itemize}


En este sentido, el proceso para el sistema de visitas se puede
definir como:

\begin{equation}\label{Esp.Edos.Down}
X\left(t\right)^{T}=\left(Q_{k}\left(t\right),A_{k}\left(t\right),B_{m}\left(t\right),B_{m}^{0}\left(t\right),C_{m}\left(t\right)\right),
\end{equation}
para $k=1,\ldots,K$ y $m=1,2,\ldots,M$, donde $T$ indica que es el
transpuesto del vector que se est\'a definiendo. El proceso $X$
evoluciona en el espacio de estados:
$\mathbb{X}=\ent_{+}^{K}\times\rea_{+}^{K}\times\left(\left\{1,2,\ldots,K\right\}\times\left\{1,2,\ldots,S\right\}\right)^{M}\times\rea_{+}^{K}\times\ent_{+}^{K}$.\\

El sistema aqu\'i descrito debe de cumplir con los siguientes supuestos b\'asicos de un sistema de visitas:
%__________________________________________________________________________
\subsubsection{Supuestos B\'asicos}
%__________________________________________________________________________
\begin{itemize}
\item[A1)] Los procesos
$\xi_{1},\ldots,\xi_{K},\eta_{1},\ldots,\eta_{K}$ son mutuamente
independientes y son sucesiones independientes e id\'enticamente
distribuidas.

\item[A2)] Para alg\'un entero $p\geq1$
\begin{eqnarray*}
\esp\left[\xi_{l}\left(1\right)^{p+1}\right]&<&\infty\textrm{ para }l=1,\ldots,K\textrm{ y }\\
\esp\left[\eta_{k}\left(1\right)^{p+1}\right]&<&\infty\textrm{
para }k=1,\ldots,K.
\end{eqnarray*}
donde $\mathcal{A}$ es la clase de posibles arribos.

\item[A3)] Para cada $k=1,2,\ldots,K$ existe una funci\'on
positiva $q_{k}\left(\cdot\right)$ definida en $\rea_{+}$, y un
entero $j_{k}$, tal que
\begin{eqnarray}
P\left(\xi_{k}\left(1\right)\geq x\right)&>&0\textrm{, para todo }x>0,\\
P\left\{a\leq\sum_{i=1}^{j_{k}}\xi_{k}\left(i\right)\leq
b\right\}&\geq&\int_{a}^{b}q_{k}\left(x\right)dx, \textrm{ }0\leq
a<b.
\end{eqnarray}
\end{itemize}

En lo que respecta al supuesto (A3), en Dai y Meyn \cite{DaiSean}
hacen ver que este se puede sustituir por

\begin{itemize}
\item[A3')] Para el Proceso de Markov $X$, cada subconjunto
compacto del espacio de estados de $X$ es un conjunto peque\~no,
ver definici\'on \ref{Def.Cto.Peq.}.
\end{itemize}

Es por esta raz\'on que con la finalidad de poder hacer uso de
$A3^{'})$ es necesario recurrir a los Procesos de Harris y en
particular a los Procesos Harris Recurrente, ver \cite{Dai,
DaiSean}.
%_______________________________________________________________________
\subsection{Procesos Harris Recurrente}
%_______________________________________________________________________

Por el supuesto (A1) conforme a Davis \cite{Davis}, se puede
definir el proceso de saltos correspondiente de manera tal que
satisfaga el supuesto (A3'), de hecho la demostraci\'on est\'a
basada en la l\'inea de argumentaci\'on de Davis, \cite{Davis},
p\'aginas 362-364.\\

Entonces se tiene un espacio de estados en el cual el proceso $X$
satisface la Propiedad Fuerte de Markov, ver Dai y Meyn
\cite{DaiSean}, dado por

\[\left(\Omega,\mathcal{F},\mathcal{F}_{t},X\left(t\right),\theta_{t},P_{x}\right),\]
adem\'as de ser un proceso de Borel Derecho (Sharpe \cite{Sharpe})
en el espacio de estados medible
$\left(\mathbb{X},\mathcal{B}_\mathbb{X}\right)$. El Proceso
$X=\left\{X\left(t\right),t\geq0\right\}$ tiene trayectorias
continuas por la derecha, est\'a definido en
$\left(\Omega,\mathcal{F}\right)$ y est\'a adaptado a
$\left\{\mathcal{F}_{t},t\geq0\right\}$; la colecci\'on
$\left\{P_{x},x\in \mathbb{X}\right\}$ son medidas de probabilidad
en $\left(\Omega,\mathcal{F}\right)$ tales que para todo $x\in
\mathbb{X}$
\[P_{x}\left\{X\left(0\right)=x\right\}=1,\] y
\[E_{x}\left\{f\left(X\circ\theta_{t}\right)|\mathcal{F}_{t}\right\}=E_{X}\left(\tau\right)f\left(X\right),\]
en $\left\{\tau<\infty\right\}$, $P_{x}$-c.s., con $\theta_{t}$
definido como el operador shift.


Donde $\tau$ es un $\mathcal{F}_{t}$-tiempo de paro
\[\left(X\circ\theta_{\tau}\right)\left(w\right)=\left\{X\left(\tau\left(w\right)+t,w\right),t\geq0\right\},\]
y $f$ es una funci\'on de valores reales acotada y medible, ver \cite{Dai, KaspiMandelbaum}.\\

Sea $P^{t}\left(x,D\right)$, $D\in\mathcal{B}_{\mathbb{X}}$,
$t\geq0$ la probabilidad de transici\'on de $X$ queda definida
como:
\[P^{t}\left(x,D\right)=P_{x}\left(X\left(t\right)\in
D\right).\]


\begin{Def}
Una medida no cero $\pi$ en
$\left(\mathbb{X},\mathcal{B}_{\mathbb{X}}\right)$ es invariante
para $X$ si $\pi$ es $\sigma$-finita y
\[\pi\left(D\right)=\int_{\mathbb{X}}P^{t}\left(x,D\right)\pi\left(dx\right),\]
para todo $D\in \mathcal{B}_{\mathbb{X}}$, con $t\geq0$.
\end{Def}

\begin{Def}
El proceso de Markov $X$ es llamado Harris Recurrente si existe
una medida de probabilidad $\nu$ en
$\left(\mathbb{X},\mathcal{B}_{\mathbb{X}}\right)$, tal que si
$\nu\left(D\right)>0$ y $D\in\mathcal{B}_{\mathbb{X}}$
\[P_{x}\left\{\tau_{D}<\infty\right\}\equiv1,\] cuando
$\tau_{D}=inf\left\{t\geq0:X_{t}\in D\right\}$.
\end{Def}

\begin{Note}
\begin{itemize}
\item[i)] Si $X$ es Harris recurrente, entonces existe una \'unica
medida invariante $\pi$ (Getoor \cite{Getoor}).

\item[ii)] Si la medida invariante es finita, entonces puede
normalizarse a una medida de probabilidad, en este caso al proceso
$X$ se le llama Harris recurrente positivo.


\item[iii)] Cuando $X$ es Harris recurrente positivo se dice que
la disciplina de servicio es estable. En este caso $\pi$ denota la
distribuci\'on estacionaria y hacemos
\[P_{\pi}\left(\cdot\right)=\int_{\mathbf{X}}P_{x}\left(\cdot\right)\pi\left(dx\right),\]
y se utiliza $E_{\pi}$ para denotar el operador esperanza
correspondiente, ver \cite{DaiSean}.
\end{itemize}
\end{Note}

\begin{Def}\label{Def.Cto.Peq.}
Un conjunto $D\in\mathcal{B_{\mathbb{X}}}$ es llamado peque\~no si
existe un $t>0$, una medida de probabilidad $\nu$ en
$\mathcal{B_{\mathbb{X}}}$, y un $\delta>0$ tal que
\[P^{t}\left(x,A\right)\geq\delta\nu\left(A\right),\] para $x\in
D,A\in\mathcal{B_{\mathbb{X}}}$.
\end{Def}

La siguiente serie de resultados vienen enunciados y demostrados
en Dai \cite{Dai}:
\begin{Lema}[Lema 3.1, Dai \cite{Dai}]
Sea $B$ conjunto peque\~no cerrado, supongamos que
$P_{x}\left(\tau_{B}<\infty\right)\equiv1$ y que para alg\'un
$\delta>0$ se cumple que
\begin{equation}\label{Eq.3.1}
\sup\esp_{x}\left[\tau_{B}\left(\delta\right)\right]<\infty,
\end{equation}
donde
$\tau_{B}\left(\delta\right)=inf\left\{t\geq\delta:X\left(t\right)\in
B\right\}$. Entonces, $X$ es un proceso Harris recurrente
positivo.
\end{Lema}

\begin{Lema}[Lema 3.1, Dai \cite{Dai}]\label{Lema.3.}
Bajo el supuesto (A3), el conjunto
$B=\left\{x\in\mathbb{X}:|x|\leq k\right\}$ es un conjunto
peque\~no cerrado para cualquier $k>0$.
\end{Lema}

\begin{Teo}[Teorema 3.1, Dai \cite{Dai}]\label{Tma.3.1}
Si existe un $\delta>0$ tal que
\begin{equation}
lim_{|x|\rightarrow\infty}\frac{1}{|x|}\esp|X^{x}\left(|x|\delta\right)|=0,
\end{equation}
donde $X^{x}$ se utiliza para denotar que el proceso $X$ comienza
a partir de $x$, entonces la ecuaci\'on (\ref{Eq.3.1}) se cumple
para $B=\left\{x\in\mathbb{X}:|x|\leq k\right\}$ con alg\'un
$k>0$. En particular, $X$ es Harris recurrente positivo.
\end{Teo}

Entonces, tenemos que el proceso $X$ es un proceso de Markov que
cumple con los supuestos $A1)$-$A3)$, lo que falta de hacer es
construir el Modelo de Flujo bas\'andonos en lo hasta ahora
presentado.
%_______________________________________________________________________
\subsection{Modelo de Flujo}
%_______________________________________________________________________

Dada una condici\'on inicial $x\in\mathbb{X}$, sea

\begin{itemize}
\item $Q_{k}^{x}\left(t\right)$ la longitud de la cola al tiempo
$t$,

\item $T_{m,k}^{x}\left(t\right)$ el tiempo acumulado, al tiempo
$t$, que tarda el servidor $m$ en atender a los usuarios de la
cola $k$.

\item $T_{m,k}^{x,0}\left(t\right)$ el tiempo acumulado, al tiempo
$t$, que tarda el servidor $m$ en trasladarse a otra cola a partir de la $k$-\'esima.\\
\end{itemize}

Sup\'ongase que la funci\'on
$\left(\overline{Q}\left(\cdot\right),\overline{T}_{m}
\left(\cdot\right),\overline{T}_{m}^{0} \left(\cdot\right)\right)$
para $m=1,2,\ldots,M$ es un punto l\'imite de
\begin{equation}\label{Eq.Punto.Limite}
\left(\frac{1}{|x|}Q^{x}\left(|x|t\right),\frac{1}{|x|}T_{m}^{x}\left(|x|t\right),\frac{1}{|x|}T_{m}^{x,0}\left(|x|t\right)\right)
\end{equation}
para $m=1,2,\ldots,M$, cuando $x\rightarrow\infty$, ver
\cite{Down}. Entonces
$\left(\overline{Q}\left(t\right),\overline{T}_{m}
\left(t\right),\overline{T}_{m}^{0} \left(t\right)\right)$ es un
flujo l\'imite del sistema. Al conjunto de todos las posibles
flujos l\'imite se le llama {\emph{Modelo de Flujo}} y se le
denotar\'a por $\mathcal{Q}$, ver \cite{Down, Dai, DaiSean}.\\

El modelo de flujo satisface el siguiente conjunto de ecuaciones:

\begin{equation}\label{Eq.MF.1}
\overline{Q}_{k}\left(t\right)=\overline{Q}_{k}\left(0\right)+\lambda_{k}t-\sum_{m=1}^{M}\mu_{k}\overline{T}_{m,k}\left(t\right),\\
\end{equation}
para $k=1,2,\ldots,K$.\\
\begin{equation}\label{Eq.MF.2}
\overline{Q}_{k}\left(t\right)\geq0\textrm{ para
}k=1,2,\ldots,K.\\
\end{equation}

\begin{equation}\label{Eq.MF.3}
\overline{T}_{m,k}\left(0\right)=0,\textrm{ y }\overline{T}_{m,k}\left(\cdot\right)\textrm{ es no decreciente},\\
\end{equation}
para $k=1,2,\ldots,K$ y $m=1,2,\ldots,M$.\\
\begin{equation}\label{Eq.MF.4}
\sum_{k=1}^{K}\overline{T}_{m,k}^{0}\left(t\right)+\overline{T}_{m,k}\left(t\right)=t\textrm{
para }m=1,2,\ldots,M.\\
\end{equation}


\begin{Def}[Definici\'on 4.1, Dai \cite{Dai}]\label{Def.Modelo.Flujo}
Sea una disciplina de servicio espec\'ifica. Cualquier l\'imite
$\left(\overline{Q}\left(\cdot\right),\overline{T}\left(\cdot\right),\overline{T}^{0}\left(\cdot\right)\right)$
en (\ref{Eq.Punto.Limite}) es un {\em flujo l\'imite} de la
disciplina. Cualquier soluci\'on (\ref{Eq.MF.1})-(\ref{Eq.MF.4})
es llamado flujo soluci\'on de la disciplina.
\end{Def}

\begin{Def}
Se dice que el modelo de flujo l\'imite, modelo de flujo, de la
disciplina de la cola es estable si existe una constante
$\delta>0$ que depende de $\mu,\lambda$ y $P$ solamente, tal que
cualquier flujo l\'imite con
$|\overline{Q}\left(0\right)|+|\overline{U}|+|\overline{V}|=1$, se
tiene que $\overline{Q}\left(\cdot+\delta\right)\equiv0$.
\end{Def}

Si se hace $|x|\rightarrow\infty$ sin restringir ninguna de las
componentes, tambi\'en se obtienen un modelo de flujo, pero en
este caso el residual de los procesos de arribo y servicio
introducen un retraso:
\begin{Teo}[Teorema 4.2, Dai \cite{Dai}]\label{Tma.4.2.Dai}
Sea una disciplina fija para la cola, suponga que se cumplen las
condiciones (A1)-(A3). Si el modelo de flujo l\'imite de la
disciplina de la cola es estable, entonces la cadena de Markov $X$
que describe la din\'amica de la red bajo la disciplina es Harris
recurrente positiva.
\end{Teo}

Ahora se procede a escalar el espacio y el tiempo para reducir la
aparente fluctuaci\'on del modelo. Consid\'erese el proceso
\begin{equation}\label{Eq.3.7}
\overline{Q}^{x}\left(t\right)=\frac{1}{|x|}Q^{x}\left(|x|t\right).
\end{equation}
A este proceso se le conoce como el flujo escalado, y cualquier
l\'imite $\overline{Q}^{x}\left(t\right)$ es llamado flujo
l\'imite del proceso de longitud de la cola. Haciendo
$|q|\rightarrow\infty$ mientras se mantiene el resto de las
componentes fijas, cualquier punto l\'imite del proceso de
longitud de la cola normalizado $\overline{Q}^{x}$ es soluci\'on
del siguiente modelo de flujo.


\begin{Def}[Definici\'on 3.3, Dai y Meyn \cite{DaiSean}]
El modelo de flujo es estable si existe un tiempo fijo $t_{0}$ tal
que $\overline{Q}\left(t\right)=0$, con $t\geq t_{0}$, para
cualquier $\overline{Q}\left(\cdot\right)\in\mathcal{Q}$ que
cumple con $|\overline{Q}\left(0\right)|=1$.
\end{Def}

\begin{Lemma}[Lema 3.1, Dai y Meyn \cite{DaiSean}]
Si el modelo de flujo definido por (\ref{Eq.MF.1})-(\ref{Eq.MF.4})
es estable, entonces el modelo de flujo retrasado es tambi\'en
estable, es decir, existe $t_{0}>0$ tal que
$\overline{Q}\left(t\right)=0$ para cualquier $t\geq t_{0}$, para
cualquier soluci\'on del modelo de flujo retrasado cuya
condici\'on inicial $\overline{x}$ satisface que
$|\overline{x}|=|\overline{Q}\left(0\right)|+|\overline{A}\left(0\right)|+|\overline{B}\left(0\right)|\leq1$.
\end{Lemma}


Ahora ya estamos en condiciones de enunciar los resultados principales:


\begin{Teo}[Teorema 2.1, Down \cite{Down}]\label{Tma2.1.Down}
Suponga que el modelo de flujo es estable, y que se cumplen los supuestos (A1) y (A2), entonces
\begin{itemize}
\item[i)] Para alguna constante $\kappa_{p}$, y para cada
condici\'on inicial $x\in X$
\begin{equation}\label{Estability.Eq1}
\limsup_{t\rightarrow\infty}\frac{1}{t}\int_{0}^{t}\esp_{x}\left[|Q\left(s\right)|^{p}\right]ds\leq\kappa_{p},
\end{equation}
donde $p$ es el entero dado en (A2).
\end{itemize}
Si adem\'as se cumple la condici\'on (A3), entonces para cada
condici\'on inicial:
\begin{itemize}
\item[ii)] Los momentos transitorios convergen a su estado
estacionario:
 \begin{equation}\label{Estability.Eq2}
lim_{t\rightarrow\infty}\esp_{x}\left[Q_{k}\left(t\right)^{r}\right]=\esp_{\pi}\left[Q_{k}\left(0\right)^{r}\right]\leq\kappa_{r},
\end{equation}
para $r=1,2,\ldots,p$ y $k=1,2,\ldots,K$. Donde $\pi$ es la
probabilidad invariante para $X$.

\item[iii)]  El primer momento converge con raz\'on $t^{p-1}$:
\begin{equation}\label{Estability.Eq3}
lim_{t\rightarrow\infty}t^{p-1}|\esp_{x}\left[Q_{k}\left(t\right)\right]-\esp_{\pi}\left[Q_{k}\left(0\right)\right]|=0.
\end{equation}

\item[iv)] La {\em Ley Fuerte de los grandes n\'umeros} se cumple:
\begin{equation}\label{Estability.Eq4}
lim_{t\rightarrow\infty}\frac{1}{t}\int_{0}^{t}Q_{k}^{r}\left(s\right)ds=\esp_{\pi}\left[Q_{k}\left(0\right)^{r}\right],\textrm{
}\prob_{x}\textrm{-c.s.}
\end{equation}
para $r=1,2,\ldots,p$ y $k=1,2,\ldots,K$.
\end{itemize}
\end{Teo}

La contribuci\'on de Down a la teor\'ia de los {\emph {sistemas de
visitas c\'iclicas}}, es la relaci\'on que hay entre la
estabilidad del sistema con el comportamiento de las medidas de
desempe\~no, es decir, la condici\'on suficiente para poder
garantizar la convergencia del proceso de la longitud de la cola
as\'i como de por los menos los dos primeros momentos adem\'as de
una versi\'on de la Ley Fuerte de los Grandes N\'umeros para los
sistemas de visitas.


\begin{Teo}[Teorema 2.3, Down \cite{Down}]\label{Tma2.3.Down}
Considere el siguiente valor:
\begin{equation}\label{Eq.Rho.1serv}
\rho=\sum_{k=1}^{K}\rho_{k}+max_{1\leq j\leq K}\left(\frac{\lambda_{j}}{\sum_{s=1}^{S}p_{js}\overline{N}_{s}}\right)\delta^{*}
\end{equation}
\begin{itemize}
\item[i)] Si $\rho<1$ entonces la red es estable, es decir, se
cumple el Teorema \ref{Tma2.1.Down}.

\item[ii)] Si $\rho>1$ entonces la red es inestable, es decir, se
cumple el Teorema \ref{Tma2.2.Down}
\end{itemize}
\end{Teo}


%_______________________________________________________________________________________________________
\subsection{Ya revisado}
%_______________________________________________________________________________________________________


Def\'inanse los puntos de regenaraci\'on  en el proceso $\left[L_{1}\left(t\right),L_{2}\left(t\right),\ldots,L_{N}\left(t\right)\right]$. Los puntos cuando la cola $i$ es visitada y todos los $L_{j}\left(\tau_{i}\left(m\right)\right)=0$ para $i=1,2$  son puntos de regeneraci\'on. Se llama ciclo regenerativo al intervalo entre dos puntos regenerativos sucesivos.

Sea $M_{i}$  el n\'umero de ciclos de visita en un ciclo regenerativo, y sea $C_{i}^{(m)}$, para $m=1,2,\ldots,M_{i}$ la duraci\'on del $m$-\'esimo ciclo de visita en un ciclo regenerativo. Se define el ciclo del tiempo de visita promedio $\esp\left[C_{i}\right]$ como

\begin{eqnarray*}
\esp\left[C_{i}\right]&=&\frac{\esp\left[\sum_{m=1}^{M_{i}}C_{i}^{(m)}\right]}{\esp\left[M_{i}\right]}
\end{eqnarray*}




Sea la funci\'on generadora de momentos para $L_{i}$, el n\'umero de usuarios en la cola $Q_{i}\left(z\right)$ en cualquier momento, est\'a dada por el tiempo promedio de $z^{L_{i}\left(t\right)}$ sobre el ciclo regenerativo definido anteriormente:

\begin{eqnarray*}
Q_{i}\left(z\right)&=&\esp\left[z^{L_{i}\left(t\right)}\right]=\frac{\esp\left[\sum_{m=1}^{M_{i}}\sum_{t=\tau_{i}\left(m\right)}^{\tau_{i}\left(m+1\right)-1}z^{L_{i}\left(t\right)}\right]}{\esp\left[\sum_{m=1}^{M_{i}}\tau_{i}\left(m+1\right)-\tau_{i}\left(m\right)\right]}
\end{eqnarray*}

$M_{i}$ es un tiempo de paro en el proceso regenerativo con $\esp\left[M_{i}\right]<\infty$, se sigue del lema de Wald que:


\begin{eqnarray*}
\esp\left[\sum_{m=1}^{M_{i}}\sum_{t=\tau_{i}\left(m\right)}^{\tau_{i}\left(m+1\right)-1}z^{L_{i}\left(t\right)}\right]&=&\esp\left[M_{i}\right]\esp\left[\sum_{t=\tau_{i}\left(m\right)}^{\tau_{i}\left(m+1\right)-1}z^{L_{i}\left(t\right)}\right]\\
\esp\left[\sum_{m=1}^{M_{i}}\tau_{i}\left(m+1\right)-\tau_{i}\left(m\right)\right]&=&\esp\left[M_{i}\right]\esp\left[\tau_{i}\left(m+1\right)-\tau_{i}\left(m\right)\right]
\end{eqnarray*}

por tanto se tiene que


\begin{eqnarray*}
Q_{i}\left(z\right)&=&\frac{\esp\left[\sum_{t=\tau_{i}\left(m\right)}^{\tau_{i}\left(m+1\right)-1}z^{L_{i}\left(t\right)}\right]}{\esp\left[\tau_{i}\left(m+1\right)-\tau_{i}\left(m\right)\right]}
\end{eqnarray*}

observar que el denominador es simplemente la duraci\'on promedio del tiempo del ciclo.


Se puede demostrar (ver Hideaki Takagi 1986) que

\begin{eqnarray*}
\esp\left[\sum_{t=\tau_{i}\left(m\right)}^{\tau_{i}\left(m+1\right)-1}z^{L_{i}\left(t\right)}\right]=z\frac{F_{i}\left(z\right)-1}{z-P_{i}\left(z\right)}
\end{eqnarray*}

Durante el tiempo de intervisita para la cola $i$, $L_{i}\left(t\right)$ solamente se incrementa de manera que el incremento por intervalo de tiempo est\'a dado por la funci\'on generadora de probabilidades de $P_{i}\left(z\right)$, por tanto la suma sobre el tiempo de intervisita puede evaluarse como:

\begin{eqnarray*}
\esp\left[\sum_{t=\tau_{i}\left(m\right)}^{\tau_{i}\left(m+1\right)-1}z^{L_{i}\left(t\right)}\right]&=&\esp\left[\sum_{t=\tau_{i}\left(m\right)}^{\tau_{i}\left(m+1\right)-1}\left\{P_{i}\left(z\right)\right\}^{t-\overline{\tau}_{i}\left(m\right)}\right]=\frac{1-\esp\left[\left\{P_{i}\left(z\right)\right\}^{\tau_{i}\left(m+1\right)-\overline{\tau}_{i}\left(m\right)}\right]}{1-P_{i}\left(z\right)}\\
&=&\frac{1-I_{i}\left[P_{i}\left(z\right)\right]}{1-P_{i}\left(z\right)}
\end{eqnarray*}
por tanto

\begin{eqnarray*}
\esp\left[\sum_{t=\tau_{i}\left(m\right)}^{\tau_{i}\left(m+1\right)-1}z^{L_{i}\left(t\right)}\right]&=&\frac{1-F_{i}\left(z\right)}{1-P_{i}\left(z\right)}
\end{eqnarray*}

Haciendo uso de lo hasta ahora desarrollado se tiene que

\begin{eqnarray*}
Q_{i}\left(z\right)&=&\frac{1}{\esp\left[C_{i}\right]}\cdot\frac{1-F_{i}\left(z\right)}{P_{i}\left(z\right)-z}\cdot\frac{\left(1-z\right)P_{i}\left(z\right)}{1-P_{i}\left(z\right)}\\
&=&\frac{\mu_{i}\left(1-\mu_{i}\right)}{f_{i}\left(i\right)}\cdot\frac{1-F_{i}\left(z\right)}{P_{i}\left(z\right)-z}\cdot\frac{\left(1-z\right)P_{i}\left(z\right)}{1-P_{i}\left(z\right)}
\end{eqnarray*}

\begin{Def}
Sea $L_{i}^{*}$el n\'umero de usuarios en la cola $Q_{i}$ cuando es visitada por el servidor para dar servicio, entonces

\begin{eqnarray}
\esp\left[L_{i}^{*}\right]&=&f_{i}\left(i\right)\\
Var\left[L_{i}^{*}\right]&=&f_{i}\left(i,i\right)+\esp\left[L_{i}^{*}\right]-\esp\left[L_{i}^{*}\right]^{2}.
\end{eqnarray}

\end{Def}


\begin{Def}
El tiempo de intervisita $I_{i}$ es el periodo de tiempo que comienza cuando se ha completado el servicio en un ciclo y termina cuando es visitada nuevamente en el pr\'oximo ciclo. Su  duraci\'on del mismo est\'a dada por $\tau_{i}\left(m+1\right)-\overline{\tau}_{i}\left(m\right)$.
\end{Def}


Recordemos las siguientes expresiones:

\begin{eqnarray*}
S_{i}\left(z\right)&=&\esp\left[z^{\overline{\tau}_{i}\left(m\right)-\tau_{i}\left(m\right)}\right]=F_{i}\left(\theta\left(z\right)\right),\\
F\left(z\right)&=&\esp\left[z^{L_{0}}\right],\\
P\left(z\right)&=&\esp\left[z^{X_{n}}\right],\\
F_{i}\left(z\right)&=&\esp\left[z^{L_{i}\left(\tau_{i}\left(m\right)\right)}\right],
\theta_{i}\left(z\right)-zP_{i}
\end{eqnarray*}

entonces 

\begin{eqnarray*}
\esp\left[S_{i}\right]&=&\frac{\esp\left[L_{i}^{*}\right]}{1-\mu_{i}}=\frac{f_{i}\left(i\right)}{1-\mu_{i}},\\
Var\left[S_{i}\right]&=&\frac{Var\left[L_{i}^{*}\right]}{\left(1-\mu_{i}\right)^{2}}+\frac{\sigma^{2}\esp\left[L_{i}^{*}\right]}{\left(1-\mu_{i}\right)^{3}}
\end{eqnarray*}

donde recordemos que

\begin{eqnarray*}
Var\left[L_{i}^{*}\right]&=&f_{i}\left(i,i\right)+f_{i}\left(i\right)-f_{i}\left(i\right)^{2}.
\end{eqnarray*}

La duraci\'on del tiempo de intervisita es $\tau_{i}\left(m+1\right)-\overline{\tau}\left(m\right)$. Dado que el n\'umero de usuarios presentes en $Q_{i}$ al tiempo $t=\tau_{i}\left(m+1\right)$ es igual al n\'umero de arribos durante el intervalo de tiempo $\left[\overline{\tau}\left(m\right),\tau_{i}\left(m+1\right)\right]$ se tiene que


\begin{eqnarray*}
\esp\left[z_{i}^{L_{i}\left(\tau_{i}\left(m+1\right)\right)}\right]=\esp\left[\left\{P_{i}\left(z_{i}\right)\right\}^{\tau_{i}\left(m+1\right)-\overline{\tau}\left(m\right)}\right]
\end{eqnarray*}

entonces, si \begin{eqnarray*}I_{i}\left(z\right)&=&\esp\left[z^{\tau_{i}\left(m+1\right)-\overline{\tau}\left(m\right)}\right]\end{eqnarray*} se tienen que

\begin{eqnarray*}
F_{i}\left(z\right)=I_{i}\left[P_{i}\left(z\right)\right]
\end{eqnarray*}
para $i=1,2$, por tanto



\begin{eqnarray*}
\esp\left[L_{i}^{*}\right]&=&\mu_{i}\esp\left[I_{i}\right]\\
Var\left[L_{i}^{*}\right]&=&\mu_{i}^{2}Var\left[I_{i}\right]+\sigma^{2}\esp\left[I_{i}\right]
\end{eqnarray*}
para $i=1,2$, por tanto


\begin{eqnarray*}
\esp\left[I_{i}\right]&=&\frac{f_{i}\left(i\right)}{\mu_{i}},
\end{eqnarray*}
adem\'as

\begin{eqnarray*}
Var\left[I_{i}\right]&=&\frac{Var\left[L_{i}^{*}\right]}{\mu_{i}^{2}}-\frac{\sigma_{i}^{2}}{\mu_{i}^{2}}f_{i}\left(i\right).
\end{eqnarray*}


Si  $C_{i}\left(z\right)=\esp\left[z^{\overline{\tau}\left(m+1\right)-\overline{\tau}_{i}\left(m\right)}\right]$el tiempo de duraci\'on del ciclo, entonces, por lo hasta ahora establecido, se tiene que

\begin{eqnarray*}
C_{i}\left(z\right)=I_{i}\left[\theta_{i}\left(z\right)\right],
\end{eqnarray*}
entonces

\begin{eqnarray*}
\esp\left[C_{i}\right]&=&\esp\left[I_{i}\right]\esp\left[\theta_{i}\left(z\right)\right]=\frac{\esp\left[L_{i}^{*}\right]}{\mu_{i}}\frac{1}{1-\mu_{i}}=\frac{f_{i}\left(i\right)}{\mu_{i}\left(1-\mu_{i}\right)}\\
Var\left[C_{i}\right]&=&\frac{Var\left[L_{i}^{*}\right]}{\mu_{i}^{2}\left(1-\mu_{i}\right)^{2}}.
\end{eqnarray*}

Por tanto se tienen las siguientes igualdades


\begin{eqnarray*}
\esp\left[S_{i}\right]&=&\mu_{i}\esp\left[C_{i}\right],\\
\esp\left[I_{i}\right]&=&\left(1-\mu_{i}\right)\esp\left[C_{i}\right]\\
\end{eqnarray*}

derivando con respecto a $z$



\begin{eqnarray*}
\frac{d Q_{i}\left(z\right)}{d z}&=&\frac{\left(1-F_{i}\left(z\right)\right)P_{i}\left(z\right)}{\esp\left[C_{i}\right]\left(1-P_{i}\left(z\right)\right)\left(P_{i}\left(z\right)-z\right)}\\
&-&\frac{\left(1-z\right)P_{i}\left(z\right)F_{i}^{'}\left(z\right)}{\esp\left[C_{i}\right]\left(1-P_{i}\left(z\right)\right)\left(P_{i}\left(z\right)-z\right)}\\
&-&\frac{\left(1-z\right)\left(1-F_{i}\left(z\right)\right)P_{i}\left(z\right)\left(P_{i}^{'}\left(z\right)-1\right)}{\esp\left[C_{i}\right]\left(1-P_{i}\left(z\right)\right)\left(P_{i}\left(z\right)-z\right)^{2}}\\
&+&\frac{\left(1-z\right)\left(1-F_{i}\left(z\right)\right)P_{i}^{'}\left(z\right)}{\esp\left[C_{i}\right]\left(1-P_{i}\left(z\right)\right)\left(P_{i}\left(z\right)-z\right)}\\
&+&\frac{\left(1-z\right)\left(1-F_{i}\left(z\right)\right)P_{i}\left(z\right)P_{i}^{'}\left(z\right)}{\esp\left[C_{i}\right]\left(1-P_{i}\left(z\right)\right)^{2}\left(P_{i}\left(z\right)-z\right)}
\end{eqnarray*}

Calculando el l\'imite cuando $z\rightarrow1^{+}$:
\begin{eqnarray}
Q_{i}^{(1)}\left(z\right)=\lim_{z\rightarrow1^{+}}\frac{d Q_{i}\left(z\right)}{dz}&=&\lim_{z\rightarrow1}\frac{\left(1-F_{i}\left(z\right)\right)P_{i}\left(z\right)}{\esp\left[C_{i}\right]\left(1-P_{i}\left(z\right)\right)\left(P_{i}\left(z\right)-z\right)}\\
&-&\lim_{z\rightarrow1^{+}}\frac{\left(1-z\right)P_{i}\left(z\right)F_{i}^{'}\left(z\right)}{\esp\left[C_{i}\right]\left(1-P_{i}\left(z\right)\right)\left(P_{i}\left(z\right)-z\right)}\\
&-&\lim_{z\rightarrow1^{+}}\frac{\left(1-z\right)\left(1-F_{i}\left(z\right)\right)P_{i}\left(z\right)\left(P_{i}^{'}\left(z\right)-1\right)}{\esp\left[C_{i}\right]\left(1-P_{i}\left(z\right)\right)\left(P_{i}\left(z\right)-z\right)^{2}}\\
&+&\lim_{z\rightarrow1^{+}}\frac{\left(1-z\right)\left(1-F_{i}\left(z\right)\right)P_{i}^{'}\left(z\right)}{\esp\left[C_{i}\right]\left(1-P_{i}\left(z\right)\right)\left(P_{i}\left(z\right)-z\right)}\\
&+&\lim_{z\rightarrow1^{+}}\frac{\left(1-z\right)\left(1-F_{i}\left(z\right)\right)P_{i}\left(z\right)P_{i}^{'}\left(z\right)}{\esp\left[C_{i}\right]\left(1-P_{i}\left(z\right)\right)^{2}\left(P_{i}\left(z\right)-z\right)}
\end{eqnarray}

Entonces:
%______________________________________________________

\begin{eqnarray*}
\lim_{z\rightarrow1^{+}}\frac{\left(1-F_{i}\left(z\right)\right)P_{i}\left(z\right)}{\left(1-P_{i}\left(z\right)\right)\left(P_{i}\left(z\right)-z\right)}&=&\lim_{z\rightarrow1^{+}}\frac{\frac{d}{dz}\left[\left(1-F_{i}\left(z\right)\right)P_{i}\left(z\right)\right]}{\frac{d}{dz}\left[\left(1-P_{i}\left(z\right)\right)\left(-z+P_{i}\left(z\right)\right)\right]}\\
&=&\lim_{z\rightarrow1^{+}}\frac{-P_{i}\left(z\right)F_{i}^{'}\left(z\right)+\left(1-F_{i}\left(z\right)\right)P_{i}^{'}\left(z\right)}{\left(1-P_{i}\left(z\right)\right)\left(-1+P_{i}^{'}\left(z\right)\right)-\left(-z+P_{i}\left(z\right)\right)P_{i}^{'}\left(z\right)}
\end{eqnarray*}


%______________________________________________________


\begin{eqnarray*}
\lim_{z\rightarrow1^{+}}\frac{\left(1-z\right)P_{i}\left(z\right)F_{i}^{'}\left(z\right)}{\left(1-P_{i}\left(z\right)\right)\left(P_{i}\left(z\right)-z\right)}&=&\lim_{z\rightarrow1^{+}}\frac{\frac{d}{dz}\left[\left(1-z\right)P_{i}\left(z\right)F_{i}^{'}\left(z\right)\right]}{\frac{d}{dz}\left[\left(1-P_{i}\left(z\right)\right)\left(P_{i}\left(z\right)-z\right)\right]}\\
&=&\lim_{z\rightarrow1^{+}}\frac{-P_{i}\left(z\right) F_{i}^{'}\left(z\right)+(1-z) F_{i}^{'}\left(z\right) P_{i}^{'}\left(z\right)+(1-z) P_{i}\left(z\right)F_{i}^{''}\left(z\right)}{\left(1-P_{i}\left(z\right)\right)\left(-1+P_{i}^{'}\left(z\right)\right)-\left(-z+P_{i}\left(z\right)\right)P_{i}^{'}\left(z\right)}
\end{eqnarray*}


%______________________________________________________

\begin{eqnarray*}
&&\lim_{z\rightarrow1^{+}}\frac{\left(1-z\right)\left(1-F_{i}\left(z\right)\right)P_{i}\left(z\right)\left(P_{i}^{'}\left(z\right)-1\right)}{\left(1-P_{i}\left(z\right)\right)\left(P_{i}\left(z\right)-z\right)^{2}}=\lim_{z\rightarrow1^{+}}\frac{\frac{d}{dz}\left[\left(1-z\right)\left(1-F_{i}\left(z\right)\right)P_{i}\left(z\right)\left(P_{i}^{'}\left(z\right)-1\right)\right]}{\frac{d}{dz}\left[\left(1-P_{i}\left(z\right)\right)\left(P_{i}\left(z\right)-z\right)^{2}\right]}\\
&=&\lim_{z\rightarrow1^{+}}\frac{-\left(1-F_{i}\left(z\right)\right) P_{i}\left(z\right)\left(-1+P_{i}^{'}\left(z\right)\right)-(1-z) P_{i}\left(z\right)F_{i}^{'}\left(z\right)\left(-1+P_{i}^{'}\left(z\right)\right)}{2\left(1-P_{i}\left(z\right)\right)\left(-z+P_{i}\left(z\right)\right) \left(-1+P_{i}^{'}\left(z\right)\right)-\left(-z+P_{i}\left(z\right)\right)^2 P_{i}^{'}\left(z\right)}\\
&+&\lim_{z\rightarrow1^{+}}\frac{+(1-z) \left(1-F_{i}\left(z\right)\right) \left(-1+P_{i}^{'}\left(z\right)\right) P_{i}^{'}\left(z\right)}{{2\left(1-P_{i}\left(z\right)\right)\left(-z+P_{i}\left(z\right)\right) \left(-1+P_{i}^{'}\left(z\right)\right)-\left(-z+P_{i}\left(z\right)\right)^2 P_{i}^{'}\left(z\right)}}\\
&+&\lim_{z\rightarrow1^{+}}\frac{+(1-z) \left(1-F_{i}\left(z\right)\right) P_{i}\left(z\right)P_{i}^{''}\left(z\right)}{{2\left(1-P_{i}\left(z\right)\right)\left(-z+P_{i}\left(z\right)\right) \left(-1+P_{i}^{'}\left(z\right)\right)-\left(-z+P_{i}\left(z\right)\right)^2 P_{i}^{'}\left(z\right)}}
\end{eqnarray*}











%______________________________________________________
\begin{eqnarray*}
&&\lim_{z\rightarrow1^{+}}\frac{\left(1-z\right)\left(1-F_{i}\left(z\right)\right)P_{i}^{'}\left(z\right)}{\left(1-P_{i}\left(z\right)\right)\left(P_{i}\left(z\right)-z\right)}=\lim_{z\rightarrow1^{+}}\frac{\frac{d}{dz}\left[\left(1-z\right)\left(1-F_{i}\left(z\right)\right)P_{i}^{'}\left(z\right)\right]}{\frac{d}{dz}\left[\left(1-P_{i}\left(z\right)\right)\left(P_{i}\left(z\right)-z\right)\right]}\\
&=&\lim_{z\rightarrow1^{+}}\frac{-\left(1-F_{i}\left(z\right)\right) P_{i}^{'}\left(z\right)-(1-z) F_{i}^{'}\left(z\right) P_{i}^{'}\left(z\right)+(1-z) \left(1-F_{i}\left(z\right)\right) P_{i}^{''}\left(z\right)}{\left(1-P_{i}\left(z\right)\right) \left(-1+P_{i}^{'}\left(z\right)\right)-\left(-z+P_{i}\left(z\right)\right) P_{i}^{'}\left(z\right)}\frac{}{}
\end{eqnarray*}

%______________________________________________________
\begin{eqnarray*}
&&\lim_{z\rightarrow1^{+}}\frac{\left(1-z\right)\left(1-F_{i}\left(z\right)\right)P_{i}\left(z\right)P_{i}^{'}\left(z\right)}{\left(1-P_{i}\left(z\right)\right)^{2}\left(P_{i}\left(z\right)-z\right)}=\lim_{z\rightarrow1^{+}}\frac{\frac{d}{dz}\left[\left(1-z\right)\left(1-F_{i}\left(z\right)\right)P_{i}\left(z\right)P_{i}^{'}\left(z\right)\right]}{\frac{d}{dz}\left[\left(1-P_{i}\left(z\right)\right)^{2}\left(P_{i}\left(z\right)-z\right)\right]}\\
&=&\lim_{z\rightarrow1^{+}}\frac{-\left(1-F_{i}\left(z\right)\right) P_{i}\left(z\right) P_{i}^{'}\left(z\right)-(1-z) P_{i}\left(z\right) F_{i}^{'}\left(z\right)P_i'[z]}{\left(1-P_{i}\left(z\right)\right)^2 \left(-1+P_{i}^{'}\left(z\right)\right)-2 \left(1-P_{i}\left(z\right)\right) \left(-z+P_{i}\left(z\right)\right) P_{i}^{'}\left(z\right)}\\
&+&\lim_{z\rightarrow1^{+}}\frac{(1-z) \left(1-F_{i}\left(z\right)\right) P_{i}^{'}\left(z\right)^2+(1-z) \left(1-F_{i}\left(z\right)\right) P_{i}\left(z\right) P_{i}^{''}\left(z\right)}{\left(1-P_{i}\left(z\right)\right)^2 \left(-1+P_{i}^{'}\left(z\right)\right)-2 \left(1-P_{i}\left(z\right)\right) \left(-z+P_{i}\left(z\right)\right) P_{i}^{'}\left(z\right)}\\
\end{eqnarray*}



En nuestra notaci\'on $V\left(t\right)\equiv C_{i}$ y $X_{i}=C_{i}^{(m)}$ para nuestra segunda definici\'on, mientras que para la primera la notaci\'on es: $X\left(t\right)\equiv C_{i}$ y $R_{i}\equiv C_{i}^{(m)}$.


%___________________________________________________________________________________________
%\section{Tiempos de Ciclo e Intervisita}
%___________________________________________________________________________________________


\begin{Def}
Sea $L_{i}^{*}$el n\'umero de usuarios en la cola $Q_{i}$ cuando es visitada por el servidor para dar servicio, entonces

\begin{eqnarray}
\esp\left[L_{i}^{*}\right]&=&f_{i}\left(i\right)\\
Var\left[L_{i}^{*}\right]&=&f_{i}\left(i,i\right)+\esp\left[L_{i}^{*}\right]-\esp\left[L_{i}^{*}\right]^{2}.
\end{eqnarray}

\end{Def}

\begin{Def}
El tiempo de Ciclo $C_{i}$ es e periodo de tiempo que comienza cuando la cola $i$ es visitada por primera vez en un ciclo, y termina cuando es visitado nuevamente en el pr\'oximo ciclo. La duraci\'on del mismo est\'a dada por $\tau_{i}\left(m+1\right)-\tau_{i}\left(m\right)$, o equivalentemente $\overline{\tau}_{i}\left(m+1\right)-\overline{\tau}_{i}\left(m\right)$ bajo condiciones de estabilidad.
\end{Def}

\begin{Def}
El tiempo de intervisita $I_{i}$ es el periodo de tiempo que comienza cuando se ha completado el servicio en un ciclo y termina cuando es visitada nuevamente en el pr\'oximo ciclo. Su  duraci\'on del mismo est\'a dada por $\tau_{i}\left(m+1\right)-\overline{\tau}_{i}\left(m\right)$.
\end{Def}


Recordemos las siguientes expresiones:

\begin{eqnarray*}
S_{i}\left(z\right)&=&\esp\left[z^{\overline{\tau}_{i}\left(m\right)-\tau_{i}\left(m\right)}\right]=F_{i}\left(\theta\left(z\right)\right),\\
F\left(z\right)&=&\esp\left[z^{L_{0}}\right],\\
P\left(z\right)&=&\esp\left[z^{X_{n}}\right],\\
F_{i}\left(z\right)&=&\esp\left[z^{L_{i}\left(\tau_{i}\left(m\right)\right)}\right],
\theta_{i}\left(z\right)-zP_{i}
\end{eqnarray*}

entonces 

\begin{eqnarray*}
\esp\left[S_{i}\right]&=&\frac{\esp\left[L_{i}^{*}\right]}{1-\mu_{i}}=\frac{f_{i}\left(i\right)}{1-\mu_{i}},\\
Var\left[S_{i}\right]&=&\frac{Var\left[L_{i}^{*}\right]}{\left(1-\mu_{i}\right)^{2}}+\frac{\sigma^{2}\esp\left[L_{i}^{*}\right]}{\left(1-\mu_{i}\right)^{3}}
\end{eqnarray*}

donde recordemos que

\begin{eqnarray*}
Var\left[L_{i}^{*}\right]&=&f_{i}\left(i,i\right)+f_{i}\left(i\right)-f_{i}\left(i\right)^{2}.
\end{eqnarray*}

La duraci\'on del tiempo de intervisita es $\tau_{i}\left(m+1\right)-\overline{\tau}\left(m\right)$. Dado que el n\'umero de usuarios presentes en $Q_{i}$ al tiempo $t=\tau_{i}\left(m+1\right)$ es igual al n\'umero de arribos durante el intervalo de tiempo $\left[\overline{\tau}\left(m\right),\tau_{i}\left(m+1\right)\right]$ se tiene que


\begin{eqnarray*}
\esp\left[z_{i}^{L_{i}\left(\tau_{i}\left(m+1\right)\right)}\right]=\esp\left[\left\{P_{i}\left(z_{i}\right)\right\}^{\tau_{i}\left(m+1\right)-\overline{\tau}\left(m\right)}\right]
\end{eqnarray*}

entonces, si \begin{eqnarray*}I_{i}\left(z\right)&=&\esp\left[z^{\tau_{i}\left(m+1\right)-\overline{\tau}\left(m\right)}\right]\end{eqnarray*} se tienen que

\begin{eqnarray*}
F_{i}\left(z\right)=I_{i}\left[P_{i}\left(z\right)\right]
\end{eqnarray*}
para $i=1,2$, por tanto



\begin{eqnarray*}
\esp\left[L_{i}^{*}\right]&=&\mu_{i}\esp\left[I_{i}\right]\\
Var\left[L_{i}^{*}\right]&=&\mu_{i}^{2}Var\left[I_{i}\right]+\sigma^{2}\esp\left[I_{i}\right]
\end{eqnarray*}
para $i=1,2$, por tanto


\begin{eqnarray*}
\esp\left[I_{i}\right]&=&\frac{f_{i}\left(i\right)}{\mu_{i}},
\end{eqnarray*}
adem\'as

\begin{eqnarray*}
Var\left[I_{i}\right]&=&\frac{Var\left[L_{i}^{*}\right]}{\mu_{i}^{2}}-\frac{\sigma_{i}^{2}}{\mu_{i}^{2}}f_{i}\left(i\right).
\end{eqnarray*}


Si  $C_{i}\left(z\right)=\esp\left[z^{\overline{\tau}\left(m+1\right)-\overline{\tau}_{i}\left(m\right)}\right]$el tiempo de duraci\'on del ciclo, entonces, por lo hasta ahora establecido, se tiene que

\begin{eqnarray*}
C_{i}\left(z\right)=I_{i}\left[\theta_{i}\left(z\right)\right],
\end{eqnarray*}
entonces

\begin{eqnarray*}
\esp\left[C_{i}\right]&=&\esp\left[I_{i}\right]\esp\left[\theta_{i}\left(z\right)\right]=\frac{\esp\left[L_{i}^{*}\right]}{\mu_{i}}\frac{1}{1-\mu_{i}}=\frac{f_{i}\left(i\right)}{\mu_{i}\left(1-\mu_{i}\right)}\\
Var\left[C_{i}\right]&=&\frac{Var\left[L_{i}^{*}\right]}{\mu_{i}^{2}\left(1-\mu_{i}\right)^{2}}.
\end{eqnarray*}

Por tanto se tienen las siguientes igualdades


\begin{eqnarray*}
\esp\left[S_{i}\right]&=&\mu_{i}\esp\left[C_{i}\right],\\
\esp\left[I_{i}\right]&=&\left(1-\mu_{i}\right)\esp\left[C_{i}\right]\\
\end{eqnarray*}

Def\'inanse los puntos de regenaraci\'on  en el proceso $\left[L_{1}\left(t\right),L_{2}\left(t\right),\ldots,L_{N}\left(t\right)\right]$. Los puntos cuando la cola $i$ es visitada y todos los $L_{j}\left(\tau_{i}\left(m\right)\right)=0$ para $i=1,2$  son puntos de regeneraci\'on. Se llama ciclo regenerativo al intervalo entre dos puntos regenerativos sucesivos.

Sea $M_{i}$  el n\'umero de ciclos de visita en un ciclo regenerativo, y sea $C_{i}^{(m)}$, para $m=1,2,\ldots,M_{i}$ la duraci\'on del $m$-\'esimo ciclo de visita en un ciclo regenerativo. Se define el ciclo del tiempo de visita promedio $\esp\left[C_{i}\right]$ como

\begin{eqnarray*}
\esp\left[C_{i}\right]&=&\frac{\esp\left[\sum_{m=1}^{M_{i}}C_{i}^{(m)}\right]}{\esp\left[M_{i}\right]}
\end{eqnarray*}


En Stid72 y Heym82 se muestra que una condici\'on suficiente para que el proceso regenerativo 
estacionario sea un procesoo estacionario es que el valor esperado del tiempo del ciclo regenerativo sea finito:

\begin{eqnarray*}
\esp\left[\sum_{m=1}^{M_{i}}C_{i}^{(m)}\right]<\infty.
\end{eqnarray*}

como cada $C_{i}^{(m)}$ contiene intervalos de r\'eplica positivos, se tiene que $\esp\left[M_{i}\right]<\infty$, adem\'as, como $M_{i}>0$, se tiene que la condici\'on anterior es equivalente a tener que 

\begin{eqnarray*}
\esp\left[C_{i}\right]<\infty,
\end{eqnarray*}
por lo tanto una condici\'on suficiente para la existencia del proceso regenerativo est\'a dada por

\begin{eqnarray*}
\sum_{k=1}^{N}\mu_{k}<1.
\end{eqnarray*}



\begin{Note}\label{Cita1.Stidham}
En Stidham\cite{Stidham} y Heyman  se muestra que una condici\'on suficiente para que el proceso regenerativo 
estacionario sea un procesoo estacionario es que el valor esperado del tiempo del ciclo regenerativo sea finito:

\begin{eqnarray*}
\esp\left[\sum_{m=1}^{M_{i}}C_{i}^{(m)}\right]<\infty.
\end{eqnarray*}

como cada $C_{i}^{(m)}$ contiene intervalos de r\'eplica positivos, se tiene que $\esp\left[M_{i}\right]<\infty$, adem\'as, como $M_{i}>0$, se tiene que la condici\'on anterior es equivalente a tener que 

\begin{eqnarray*}
\esp\left[C_{i}\right]<\infty,
\end{eqnarray*}
por lo tanto una condici\'on suficiente para la existencia del proceso regenerativo est\'a dada por

\begin{eqnarray*}
\sum_{k=1}^{N}\mu_{k}<1.
\end{eqnarray*}

{\centering{\Huge{\textbf{Nota incompleta!!}}}}
\end{Note}

%_______________________________________________________________________________________
\subsection{Procesos de Renovaci\'on y Regenerativos}
%_______________________________________________________________________________________



Se puede demostrar (ver Hideaki Takagi 1986) que

\begin{eqnarray*}
\esp\left[\sum_{t=\tau_{i}\left(m\right)}^{\tau_{i}\left(m+1\right)-1}z^{L_{i}\left(t\right)}\right]=z\frac{F_{i}\left(z\right)-1}{z-P_{i}\left(z\right)}
\end{eqnarray*}

Durante el tiempo de intervisita para la cola $i$, $L_{i}\left(t\right)$ solamente se incrementa de manera que el incremento por intervalo de tiempo est\'a dado por la funci\'on generadora de probabilidades de $P_{i}\left(z\right)$, por tanto la suma sobre el tiempo de intervisita puede evaluarse como:

\begin{eqnarray*}
\esp\left[\sum_{t=\tau_{i}\left(m\right)}^{\tau_{i}\left(m+1\right)-1}z^{L_{i}\left(t\right)}\right]&=&\esp\left[\sum_{t=\tau_{i}\left(m\right)}^{\tau_{i}\left(m+1\right)-1}\left\{P_{i}\left(z\right)\right\}^{t-\overline{\tau}_{i}\left(m\right)}\right]=\frac{1-\esp\left[\left\{P_{i}\left(z\right)\right\}^{\tau_{i}\left(m+1\right)-\overline{\tau}_{i}\left(m\right)}\right]}{1-P_{i}\left(z\right)}\\
&=&\frac{1-I_{i}\left[P_{i}\left(z\right)\right]}{1-P_{i}\left(z\right)}
\end{eqnarray*}
por tanto

\begin{eqnarray*}
\esp\left[\sum_{t=\tau_{i}\left(m\right)}^{\tau_{i}\left(m+1\right)-1}z^{L_{i}\left(t\right)}\right]&=&\frac{1-F_{i}\left(z\right)}{1-P_{i}\left(z\right)}
\end{eqnarray*}

Haciendo uso de lo hasta ahora desarrollado se tiene que



%___________________________________________________________________________________________
%\subsection{Longitudes de la Cola en cualquier tiempo}
%___________________________________________________________________________________________
Sea 
\begin{eqnarray*}
Q_{i}\left(z\right)&=&\frac{1}{\esp\left[C_{i}\right]}\cdot\frac{1-F_{i}\left(z\right)}{P_{i}\left(z\right)-z}\cdot\frac{\left(1-z\right)P_{i}\left(z\right)}{1-P_{i}\left(z\right)}
\end{eqnarray*}

Consideremos una cola de la red de sistemas de visitas c\'iclicas fija, $Q_{l}$.


Conforme a la definici\'on dada al principio del cap\'itulo, definici\'on (\ref{Def.Tn}), sean $T_{1},T_{2},\ldots$ los puntos donde las longitudes de las colas de la red de sistemas de visitas c\'iclicas son cero simult\'aneamente, cuando la cola $Q_{l}$ es visitada por el servidor para dar servicio, es decir, $L_{1}\left(T_{i}\right)=0,L_{2}\left(T_{i}\right)=0,\hat{L}_{1}\left(T_{i}\right)=0$ y $\hat{L}_{2}\left(T_{i}\right)=0$, a estos puntos se les denominar\'a puntos regenerativos. Entonces, 

\begin{Def}
Al intervalo de tiempo entre dos puntos regenerativos se le llamar\'a ciclo regenerativo.
\end{Def}

\begin{Def}
Para $T_{i}$ se define, $M_{i}$, el n\'umero de ciclos de visita a la cola $Q_{l}$, durante el ciclo regenerativo, es decir, $M_{i}$ es un proceso de renovaci\'on.
\end{Def}

\begin{Def}
Para cada uno de los $M_{i}$'s, se definen a su vez la duraci\'on de cada uno de estos ciclos de visita en el ciclo regenerativo, $C_{i}^{(m)}$, para $m=1,2,\ldots,M_{i}$, que a su vez, tambi\'en es n proceso de renovaci\'on.
\end{Def}

En nuestra notaci\'on $V\left(t\right)\equiv C_{i}$ y $X_{i}=C_{i}^{(m)}$ para nuestra segunda definici\'on, mientras que para la primera la notaci\'on es: $X\left(t\right)\equiv C_{i}$ y $R_{i}\equiv C_{i}^{(m)}$.


%___________________________________________________________________________________________
%\subsection{Tiempos de Ciclo e Intervisita}
%___________________________________________________________________________________________


\begin{Def}
Sea $L_{i}^{*}$el n\'umero de usuarios en la cola $Q_{i}$ cuando es visitada por el servidor para dar servicio, entonces

\begin{eqnarray}
\esp\left[L_{i}^{*}\right]&=&f_{i}\left(i\right)\\
Var\left[L_{i}^{*}\right]&=&f_{i}\left(i,i\right)+\esp\left[L_{i}^{*}\right]-\esp\left[L_{i}^{*}\right]^{2}.
\end{eqnarray}

\end{Def}

\begin{Def}
El tiempo de Ciclo $C_{i}$ es e periodo de tiempo que comienza cuando la cola $i$ es visitada por primera vez en un ciclo, y termina cuando es visitado nuevamente en el pr\'oximo ciclo. La duraci\'on del mismo est\'a dada por $\tau_{i}\left(m+1\right)-\tau_{i}\left(m\right)$, o equivalentemente $\overline{\tau}_{i}\left(m+1\right)-\overline{\tau}_{i}\left(m\right)$ bajo condiciones de estabilidad.
\end{Def}



Recordemos las siguientes expresiones:

\begin{eqnarray*}
S_{i}\left(z\right)&=&\esp\left[z^{\overline{\tau}_{i}\left(m\right)-\tau_{i}\left(m\right)}\right]=F_{i}\left(\theta\left(z\right)\right),\\
F\left(z\right)&=&\esp\left[z^{L_{0}}\right],\\
P\left(z\right)&=&\esp\left[z^{X_{n}}\right],\\
F_{i}\left(z\right)&=&\esp\left[z^{L_{i}\left(\tau_{i}\left(m\right)\right)}\right],
\theta_{i}\left(z\right)-zP_{i}
\end{eqnarray*}

entonces 

\begin{eqnarray*}
\esp\left[S_{i}\right]&=&\frac{\esp\left[L_{i}^{*}\right]}{1-\mu_{i}}=\frac{f_{i}\left(i\right)}{1-\mu_{i}},\\
Var\left[S_{i}\right]&=&\frac{Var\left[L_{i}^{*}\right]}{\left(1-\mu_{i}\right)^{2}}+\frac{\sigma^{2}\esp\left[L_{i}^{*}\right]}{\left(1-\mu_{i}\right)^{3}}
\end{eqnarray*}

donde recordemos que

\begin{eqnarray*}
Var\left[L_{i}^{*}\right]&=&f_{i}\left(i,i\right)+f_{i}\left(i\right)-f_{i}\left(i\right)^{2}.
\end{eqnarray*}

 por tanto


\begin{eqnarray*}
\esp\left[I_{i}\right]&=&\frac{f_{i}\left(i\right)}{\mu_{i}},
\end{eqnarray*}
adem\'as

\begin{eqnarray*}
Var\left[I_{i}\right]&=&\frac{Var\left[L_{i}^{*}\right]}{\mu_{i}^{2}}-\frac{\sigma_{i}^{2}}{\mu_{i}^{2}}f_{i}\left(i\right).
\end{eqnarray*}


Si  $C_{i}\left(z\right)=\esp\left[z^{\overline{\tau}\left(m+1\right)-\overline{\tau}_{i}\left(m\right)}\right]$el tiempo de duraci\'on del ciclo, entonces, por lo hasta ahora establecido, se tiene que

\begin{eqnarray*}
C_{i}\left(z\right)=I_{i}\left[\theta_{i}\left(z\right)\right],
\end{eqnarray*}
entonces

\begin{eqnarray*}
\esp\left[C_{i}\right]&=&\esp\left[I_{i}\right]\esp\left[\theta_{i}\left(z\right)\right]=\frac{\esp\left[L_{i}^{*}\right]}{\mu_{i}}\frac{1}{1-\mu_{i}}=\frac{f_{i}\left(i\right)}{\mu_{i}\left(1-\mu_{i}\right)}\\
Var\left[C_{i}\right]&=&\frac{Var\left[L_{i}^{*}\right]}{\mu_{i}^{2}\left(1-\mu_{i}\right)^{2}}.
\end{eqnarray*}

Por tanto se tienen las siguientes igualdades


\begin{eqnarray*}
\esp\left[S_{i}\right]&=&\mu_{i}\esp\left[C_{i}\right],\\
\esp\left[I_{i}\right]&=&\left(1-\mu_{i}\right)\esp\left[C_{i}\right]\\
\end{eqnarray*}

Def\'inanse los puntos de regenaraci\'on  en el proceso $\left[L_{1}\left(t\right),L_{2}\left(t\right),\ldots,L_{N}\left(t\right)\right]$. Los puntos cuando la cola $i$ es visitada y todos los $L_{j}\left(\tau_{i}\left(m\right)\right)=0$ para $i=1,2$  son puntos de regeneraci\'on. Se llama ciclo regenerativo al intervalo entre dos puntos regenerativos sucesivos.

Sea $M_{i}$  el n\'umero de ciclos de visita en un ciclo regenerativo, y sea $C_{i}^{(m)}$, para $m=1,2,\ldots,M_{i}$ la duraci\'on del $m$-\'esimo ciclo de visita en un ciclo regenerativo. Se define el ciclo del tiempo de visita promedio $\esp\left[C_{i}\right]$ como

\begin{eqnarray*}
\esp\left[C_{i}\right]&=&\frac{\esp\left[\sum_{m=1}^{M_{i}}C_{i}^{(m)}\right]}{\esp\left[M_{i}\right]}
\end{eqnarray*}


En Stid72 y Heym82 se muestra que una condici\'on suficiente para que el proceso regenerativo 
estacionario sea un procesoo estacionario es que el valor esperado del tiempo del ciclo regenerativo sea finito:

\begin{eqnarray*}
\esp\left[\sum_{m=1}^{M_{i}}C_{i}^{(m)}\right]<\infty.
\end{eqnarray*}

como cada $C_{i}^{(m)}$ contiene intervalos de r\'eplica positivos, se tiene que $\esp\left[M_{i}\right]<\infty$, adem\'as, como $M_{i}>0$, se tiene que la condici\'on anterior es equivalente a tener que 

\begin{eqnarray*}
\esp\left[C_{i}\right]<\infty,
\end{eqnarray*}
por lo tanto una condici\'on suficiente para la existencia del proceso regenerativo est\'a dada por

\begin{eqnarray*}
\sum_{k=1}^{N}\mu_{k}<1.
\end{eqnarray*}

Sea la funci\'on generadora de momentos para $L_{i}$, el n\'umero de usuarios en la cola $Q_{i}\left(z\right)$ en cualquier momento, est\'a dada por el tiempo promedio de $z^{L_{i}\left(t\right)}$ sobre el ciclo regenerativo definido anteriormente:

\begin{eqnarray*}
Q_{i}\left(z\right)&=&\esp\left[z^{L_{i}\left(t\right)}\right]=\frac{\esp\left[\sum_{m=1}^{M_{i}}\sum_{t=\tau_{i}\left(m\right)}^{\tau_{i}\left(m+1\right)-1}z^{L_{i}\left(t\right)}\right]}{\esp\left[\sum_{m=1}^{M_{i}}\tau_{i}\left(m+1\right)-\tau_{i}\left(m\right)\right]}
\end{eqnarray*}

$M_{i}$ es un tiempo de paro en el proceso regenerativo con $\esp\left[M_{i}\right]<\infty$, se sigue del lema de Wald que:


\begin{eqnarray*}
\esp\left[\sum_{m=1}^{M_{i}}\sum_{t=\tau_{i}\left(m\right)}^{\tau_{i}\left(m+1\right)-1}z^{L_{i}\left(t\right)}\right]&=&\esp\left[M_{i}\right]\esp\left[\sum_{t=\tau_{i}\left(m\right)}^{\tau_{i}\left(m+1\right)-1}z^{L_{i}\left(t\right)}\right]\\
\esp\left[\sum_{m=1}^{M_{i}}\tau_{i}\left(m+1\right)-\tau_{i}\left(m\right)\right]&=&\esp\left[M_{i}\right]\esp\left[\tau_{i}\left(m+1\right)-\tau_{i}\left(m\right)\right]
\end{eqnarray*}

por tanto se tiene que


\begin{eqnarray*}
Q_{i}\left(z\right)&=&\frac{\esp\left[\sum_{t=\tau_{i}\left(m\right)}^{\tau_{i}\left(m+1\right)-1}z^{L_{i}\left(t\right)}\right]}{\esp\left[\tau_{i}\left(m+1\right)-\tau_{i}\left(m\right)\right]}
\end{eqnarray*}

observar que el denominador es simplemente la duraci\'on promedio del tiempo del ciclo.


Se puede demostrar (ver Hideaki Takagi 1986) que

\begin{eqnarray*}
\esp\left[\sum_{t=\tau_{i}\left(m\right)}^{\tau_{i}\left(m+1\right)-1}z^{L_{i}\left(t\right)}\right]=z\frac{F_{i}\left(z\right)-1}{z-P_{i}\left(z\right)}
\end{eqnarray*}

Durante el tiempo de intervisita para la cola $i$, $L_{i}\left(t\right)$ solamente se incrementa de manera que el incremento por intervalo de tiempo est\'a dado por la funci\'on generadora de probabilidades de $P_{i}\left(z\right)$, por tanto la suma sobre el tiempo de intervisita puede evaluarse como:

\begin{eqnarray*}
\esp\left[\sum_{t=\tau_{i}\left(m\right)}^{\tau_{i}\left(m+1\right)-1}z^{L_{i}\left(t\right)}\right]&=&\esp\left[\sum_{t=\tau_{i}\left(m\right)}^{\tau_{i}\left(m+1\right)-1}\left\{P_{i}\left(z\right)\right\}^{t-\overline{\tau}_{i}\left(m\right)}\right]=\frac{1-\esp\left[\left\{P_{i}\left(z\right)\right\}^{\tau_{i}\left(m+1\right)-\overline{\tau}_{i}\left(m\right)}\right]}{1-P_{i}\left(z\right)}\\
&=&\frac{1-I_{i}\left[P_{i}\left(z\right)\right]}{1-P_{i}\left(z\right)}
\end{eqnarray*}
por tanto

\begin{eqnarray*}
\esp\left[\sum_{t=\tau_{i}\left(m\right)}^{\tau_{i}\left(m+1\right)-1}z^{L_{i}\left(t\right)}\right]&=&\frac{1-F_{i}\left(z\right)}{1-P_{i}\left(z\right)}
\end{eqnarray*}

Haciendo uso de lo hasta ahora desarrollado se tiene que

\begin{eqnarray*}
Q_{i}\left(z\right)&=&\frac{1}{\esp\left[C_{i}\right]}\cdot\frac{1-F_{i}\left(z\right)}{P_{i}\left(z\right)-z}\cdot\frac{\left(1-z\right)P_{i}\left(z\right)}{1-P_{i}\left(z\right)}\\
&=&\frac{\mu_{i}\left(1-\mu_{i}\right)}{f_{i}\left(i\right)}\cdot\frac{1-F_{i}\left(z\right)}{P_{i}\left(z\right)-z}\cdot\frac{\left(1-z\right)P_{i}\left(z\right)}{1-P_{i}\left(z\right)}
\end{eqnarray*}

derivando con respecto a $z$



\begin{eqnarray*}
\frac{d Q_{i}\left(z\right)}{d z}&=&\frac{\left(1-F_{i}\left(z\right)\right)P_{i}\left(z\right)}{\esp\left[C_{i}\right]\left(1-P_{i}\left(z\right)\right)\left(P_{i}\left(z\right)-z\right)}\\
&-&\frac{\left(1-z\right)P_{i}\left(z\right)F_{i}^{'}\left(z\right)}{\esp\left[C_{i}\right]\left(1-P_{i}\left(z\right)\right)\left(P_{i}\left(z\right)-z\right)}\\
&-&\frac{\left(1-z\right)\left(1-F_{i}\left(z\right)\right)P_{i}\left(z\right)\left(P_{i}^{'}\left(z\right)-1\right)}{\esp\left[C_{i}\right]\left(1-P_{i}\left(z\right)\right)\left(P_{i}\left(z\right)-z\right)^{2}}\\
&+&\frac{\left(1-z\right)\left(1-F_{i}\left(z\right)\right)P_{i}^{'}\left(z\right)}{\esp\left[C_{i}\right]\left(1-P_{i}\left(z\right)\right)\left(P_{i}\left(z\right)-z\right)}\\
&+&\frac{\left(1-z\right)\left(1-F_{i}\left(z\right)\right)P_{i}\left(z\right)P_{i}^{'}\left(z\right)}{\esp\left[C_{i}\right]\left(1-P_{i}\left(z\right)\right)^{2}\left(P_{i}\left(z\right)-z\right)}
\end{eqnarray*}

Calculando el l\'imite cuando $z\rightarrow1^{+}$:
\begin{eqnarray}
Q_{i}^{(1)}\left(z\right)=\lim_{z\rightarrow1^{+}}\frac{d Q_{i}\left(z\right)}{dz}&=&\lim_{z\rightarrow1}\frac{\left(1-F_{i}\left(z\right)\right)P_{i}\left(z\right)}{\esp\left[C_{i}\right]\left(1-P_{i}\left(z\right)\right)\left(P_{i}\left(z\right)-z\right)}\\
&-&\lim_{z\rightarrow1^{+}}\frac{\left(1-z\right)P_{i}\left(z\right)F_{i}^{'}\left(z\right)}{\esp\left[C_{i}\right]\left(1-P_{i}\left(z\right)\right)\left(P_{i}\left(z\right)-z\right)}\\
&-&\lim_{z\rightarrow1^{+}}\frac{\left(1-z\right)\left(1-F_{i}\left(z\right)\right)P_{i}\left(z\right)\left(P_{i}^{'}\left(z\right)-1\right)}{\esp\left[C_{i}\right]\left(1-P_{i}\left(z\right)\right)\left(P_{i}\left(z\right)-z\right)^{2}}\\
&+&\lim_{z\rightarrow1^{+}}\frac{\left(1-z\right)\left(1-F_{i}\left(z\right)\right)P_{i}^{'}\left(z\right)}{\esp\left[C_{i}\right]\left(1-P_{i}\left(z\right)\right)\left(P_{i}\left(z\right)-z\right)}\\
&+&\lim_{z\rightarrow1^{+}}\frac{\left(1-z\right)\left(1-F_{i}\left(z\right)\right)P_{i}\left(z\right)P_{i}^{'}\left(z\right)}{\esp\left[C_{i}\right]\left(1-P_{i}\left(z\right)\right)^{2}\left(P_{i}\left(z\right)-z\right)}
\end{eqnarray}

Entonces:
%______________________________________________________

\begin{eqnarray*}
\lim_{z\rightarrow1^{+}}\frac{\left(1-F_{i}\left(z\right)\right)P_{i}\left(z\right)}{\left(1-P_{i}\left(z\right)\right)\left(P_{i}\left(z\right)-z\right)}&=&\lim_{z\rightarrow1^{+}}\frac{\frac{d}{dz}\left[\left(1-F_{i}\left(z\right)\right)P_{i}\left(z\right)\right]}{\frac{d}{dz}\left[\left(1-P_{i}\left(z\right)\right)\left(-z+P_{i}\left(z\right)\right)\right]}\\
&=&\lim_{z\rightarrow1^{+}}\frac{-P_{i}\left(z\right)F_{i}^{'}\left(z\right)+\left(1-F_{i}\left(z\right)\right)P_{i}^{'}\left(z\right)}{\left(1-P_{i}\left(z\right)\right)\left(-1+P_{i}^{'}\left(z\right)\right)-\left(-z+P_{i}\left(z\right)\right)P_{i}^{'}\left(z\right)}
\end{eqnarray*}


%______________________________________________________


\begin{eqnarray*}
\lim_{z\rightarrow1^{+}}\frac{\left(1-z\right)P_{i}\left(z\right)F_{i}^{'}\left(z\right)}{\left(1-P_{i}\left(z\right)\right)\left(P_{i}\left(z\right)-z\right)}&=&\lim_{z\rightarrow1^{+}}\frac{\frac{d}{dz}\left[\left(1-z\right)P_{i}\left(z\right)F_{i}^{'}\left(z\right)\right]}{\frac{d}{dz}\left[\left(1-P_{i}\left(z\right)\right)\left(P_{i}\left(z\right)-z\right)\right]}\\
&=&\lim_{z\rightarrow1^{+}}\frac{-P_{i}\left(z\right) F_{i}^{'}\left(z\right)+(1-z) F_{i}^{'}\left(z\right) P_{i}^{'}\left(z\right)+(1-z) P_{i}\left(z\right)F_{i}^{''}\left(z\right)}{\left(1-P_{i}\left(z\right)\right)\left(-1+P_{i}^{'}\left(z\right)\right)-\left(-z+P_{i}\left(z\right)\right)P_{i}^{'}\left(z\right)}
\end{eqnarray*}


%______________________________________________________

\begin{eqnarray*}
&&\lim_{z\rightarrow1^{+}}\frac{\left(1-z\right)\left(1-F_{i}\left(z\right)\right)P_{i}\left(z\right)\left(P_{i}^{'}\left(z\right)-1\right)}{\left(1-P_{i}\left(z\right)\right)\left(P_{i}\left(z\right)-z\right)^{2}}=\lim_{z\rightarrow1^{+}}\frac{\frac{d}{dz}\left[\left(1-z\right)\left(1-F_{i}\left(z\right)\right)P_{i}\left(z\right)\left(P_{i}^{'}\left(z\right)-1\right)\right]}{\frac{d}{dz}\left[\left(1-P_{i}\left(z\right)\right)\left(P_{i}\left(z\right)-z\right)^{2}\right]}\\
&=&\lim_{z\rightarrow1^{+}}\frac{-\left(1-F_{i}\left(z\right)\right) P_{i}\left(z\right)\left(-1+P_{i}^{'}\left(z\right)\right)-(1-z) P_{i}\left(z\right)F_{i}^{'}\left(z\right)\left(-1+P_{i}^{'}\left(z\right)\right)}{2\left(1-P_{i}\left(z\right)\right)\left(-z+P_{i}\left(z\right)\right) \left(-1+P_{i}^{'}\left(z\right)\right)-\left(-z+P_{i}\left(z\right)\right)^2 P_{i}^{'}\left(z\right)}\\
&+&\lim_{z\rightarrow1^{+}}\frac{+(1-z) \left(1-F_{i}\left(z\right)\right) \left(-1+P_{i}^{'}\left(z\right)\right) P_{i}^{'}\left(z\right)}{{2\left(1-P_{i}\left(z\right)\right)\left(-z+P_{i}\left(z\right)\right) \left(-1+P_{i}^{'}\left(z\right)\right)-\left(-z+P_{i}\left(z\right)\right)^2 P_{i}^{'}\left(z\right)}}\\
&+&\lim_{z\rightarrow1^{+}}\frac{+(1-z) \left(1-F_{i}\left(z\right)\right) P_{i}\left(z\right)P_{i}^{''}\left(z\right)}{{2\left(1-P_{i}\left(z\right)\right)\left(-z+P_{i}\left(z\right)\right) \left(-1+P_{i}^{'}\left(z\right)\right)-\left(-z+P_{i}\left(z\right)\right)^2 P_{i}^{'}\left(z\right)}}
\end{eqnarray*}











%______________________________________________________
\begin{eqnarray*}
&&\lim_{z\rightarrow1^{+}}\frac{\left(1-z\right)\left(1-F_{i}\left(z\right)\right)P_{i}^{'}\left(z\right)}{\left(1-P_{i}\left(z\right)\right)\left(P_{i}\left(z\right)-z\right)}=\lim_{z\rightarrow1^{+}}\frac{\frac{d}{dz}\left[\left(1-z\right)\left(1-F_{i}\left(z\right)\right)P_{i}^{'}\left(z\right)\right]}{\frac{d}{dz}\left[\left(1-P_{i}\left(z\right)\right)\left(P_{i}\left(z\right)-z\right)\right]}\\
&=&\lim_{z\rightarrow1^{+}}\frac{-\left(1-F_{i}\left(z\right)\right) P_{i}^{'}\left(z\right)-(1-z) F_{i}^{'}\left(z\right) P_{i}^{'}\left(z\right)+(1-z) \left(1-F_{i}\left(z\right)\right) P_{i}^{''}\left(z\right)}{\left(1-P_{i}\left(z\right)\right) \left(-1+P_{i}^{'}\left(z\right)\right)-\left(-z+P_{i}\left(z\right)\right) P_{i}^{'}\left(z\right)}\frac{}{}
\end{eqnarray*}

%______________________________________________________
\begin{eqnarray*}
&&\lim_{z\rightarrow1^{+}}\frac{\left(1-z\right)\left(1-F_{i}\left(z\right)\right)P_{i}\left(z\right)P_{i}^{'}\left(z\right)}{\left(1-P_{i}\left(z\right)\right)^{2}\left(P_{i}\left(z\right)-z\right)}=\lim_{z\rightarrow1^{+}}\frac{\frac{d}{dz}\left[\left(1-z\right)\left(1-F_{i}\left(z\right)\right)P_{i}\left(z\right)P_{i}^{'}\left(z\right)\right]}{\frac{d}{dz}\left[\left(1-P_{i}\left(z\right)\right)^{2}\left(P_{i}\left(z\right)-z\right)\right]}\\
&=&\lim_{z\rightarrow1^{+}}\frac{-\left(1-F_{i}\left(z\right)\right) P_{i}\left(z\right) P_{i}^{'}\left(z\right)-(1-z) P_{i}\left(z\right) F_{i}^{'}\left(z\right)P_i'[z]}{\left(1-P_{i}\left(z\right)\right)^2 \left(-1+P_{i}^{'}\left(z\right)\right)-2 \left(1-P_{i}\left(z\right)\right) \left(-z+P_{i}\left(z\right)\right) P_{i}^{'}\left(z\right)}\\
&+&\lim_{z\rightarrow1^{+}}\frac{(1-z) \left(1-F_{i}\left(z\right)\right) P_{i}^{'}\left(z\right)^2+(1-z) \left(1-F_{i}\left(z\right)\right) P_{i}\left(z\right) P_{i}^{''}\left(z\right)}{\left(1-P_{i}\left(z\right)\right)^2 \left(-1+P_{i}^{'}\left(z\right)\right)-2 \left(1-P_{i}\left(z\right)\right) \left(-z+P_{i}\left(z\right)\right) P_{i}^{'}\left(z\right)}\\
\end{eqnarray*}
%___________________________________________________________________________________________
%\subsection{Tiempos de Ciclo e Intervisita}
%___________________________________________________________________________________________


\begin{Def}
Sea $L_{i}^{*}$el n\'umero de usuarios en la cola $Q_{i}$ cuando es visitada por el servidor para dar servicio, entonces

\begin{eqnarray}
\esp\left[L_{i}^{*}\right]&=&f_{i}\left(i\right)\\
Var\left[L_{i}^{*}\right]&=&f_{i}\left(i,i\right)+\esp\left[L_{i}^{*}\right]-\esp\left[L_{i}^{*}\right]^{2}.
\end{eqnarray}

\end{Def}

\begin{Def}
El tiempo de Ciclo $C_{i}$ es e periodo de tiempo que comienza cuando la cola $i$ es visitada por primera vez en un ciclo, y termina cuando es visitado nuevamente en el pr\'oximo ciclo. La duraci\'on del mismo est\'a dada por $\tau_{i}\left(m+1\right)-\tau_{i}\left(m\right)$, o equivalentemente $\overline{\tau}_{i}\left(m+1\right)-\overline{\tau}_{i}\left(m\right)$ bajo condiciones de estabilidad.
\end{Def}

\begin{Def}
El tiempo de intervisita $I_{i}$ es el periodo de tiempo que comienza cuando se ha completado el servicio en un ciclo y termina cuando es visitada nuevamente en el pr\'oximo ciclo. Su  duraci\'on del mismo est\'a dada por $\tau_{i}\left(m+1\right)-\overline{\tau}_{i}\left(m\right)$.
\end{Def}


Recordemos las siguientes expresiones:

\begin{eqnarray*}
S_{i}\left(z\right)&=&\esp\left[z^{\overline{\tau}_{i}\left(m\right)-\tau_{i}\left(m\right)}\right]=F_{i}\left(\theta\left(z\right)\right),\\
F\left(z\right)&=&\esp\left[z^{L_{0}}\right],\\
P\left(z\right)&=&\esp\left[z^{X_{n}}\right],\\
F_{i}\left(z\right)&=&\esp\left[z^{L_{i}\left(\tau_{i}\left(m\right)\right)}\right],
\theta_{i}\left(z\right)-zP_{i}
\end{eqnarray*}

entonces 

\begin{eqnarray*}
\esp\left[S_{i}\right]&=&\frac{\esp\left[L_{i}^{*}\right]}{1-\mu_{i}}=\frac{f_{i}\left(i\right)}{1-\mu_{i}},\\
Var\left[S_{i}\right]&=&\frac{Var\left[L_{i}^{*}\right]}{\left(1-\mu_{i}\right)^{2}}+\frac{\sigma^{2}\esp\left[L_{i}^{*}\right]}{\left(1-\mu_{i}\right)^{3}}
\end{eqnarray*}

donde recordemos que

\begin{eqnarray*}
Var\left[L_{i}^{*}\right]&=&f_{i}\left(i,i\right)+f_{i}\left(i\right)-f_{i}\left(i\right)^{2}.
\end{eqnarray*}

La duraci\'on del tiempo de intervisita es $\tau_{i}\left(m+1\right)-\overline{\tau}\left(m\right)$. Dado que el n\'umero de usuarios presentes en $Q_{i}$ al tiempo $t=\tau_{i}\left(m+1\right)$ es igual al n\'umero de arribos durante el intervalo de tiempo $\left[\overline{\tau}\left(m\right),\tau_{i}\left(m+1\right)\right]$ se tiene que


\begin{eqnarray*}
\esp\left[z_{i}^{L_{i}\left(\tau_{i}\left(m+1\right)\right)}\right]=\esp\left[\left\{P_{i}\left(z_{i}\right)\right\}^{\tau_{i}\left(m+1\right)-\overline{\tau}\left(m\right)}\right]
\end{eqnarray*}

entonces, si \begin{eqnarray*}I_{i}\left(z\right)&=&\esp\left[z^{\tau_{i}\left(m+1\right)-\overline{\tau}\left(m\right)}\right]\end{eqnarray*} se tienen que

\begin{eqnarray*}
F_{i}\left(z\right)=I_{i}\left[P_{i}\left(z\right)\right]
\end{eqnarray*}
para $i=1,2$, por tanto



\begin{eqnarray*}
\esp\left[L_{i}^{*}\right]&=&\mu_{i}\esp\left[I_{i}\right]\\
Var\left[L_{i}^{*}\right]&=&\mu_{i}^{2}Var\left[I_{i}\right]+\sigma^{2}\esp\left[I_{i}\right]
\end{eqnarray*}
para $i=1,2$, por tanto


\begin{eqnarray*}
\esp\left[I_{i}\right]&=&\frac{f_{i}\left(i\right)}{\mu_{i}},
\end{eqnarray*}
adem\'as

\begin{eqnarray*}
Var\left[I_{i}\right]&=&\frac{Var\left[L_{i}^{*}\right]}{\mu_{i}^{2}}-\frac{\sigma_{i}^{2}}{\mu_{i}^{2}}f_{i}\left(i\right).
\end{eqnarray*}


Si  $C_{i}\left(z\right)=\esp\left[z^{\overline{\tau}\left(m+1\right)-\overline{\tau}_{i}\left(m\right)}\right]$el tiempo de duraci\'on del ciclo, entonces, por lo hasta ahora establecido, se tiene que

\begin{eqnarray*}
C_{i}\left(z\right)=I_{i}\left[\theta_{i}\left(z\right)\right],
\end{eqnarray*}
entonces

\begin{eqnarray*}
\esp\left[C_{i}\right]&=&\esp\left[I_{i}\right]\esp\left[\theta_{i}\left(z\right)\right]=\frac{\esp\left[L_{i}^{*}\right]}{\mu_{i}}\frac{1}{1-\mu_{i}}=\frac{f_{i}\left(i\right)}{\mu_{i}\left(1-\mu_{i}\right)}\\
Var\left[C_{i}\right]&=&\frac{Var\left[L_{i}^{*}\right]}{\mu_{i}^{2}\left(1-\mu_{i}\right)^{2}}.
\end{eqnarray*}

Por tanto se tienen las siguientes igualdades


\begin{eqnarray*}
\esp\left[S_{i}\right]&=&\mu_{i}\esp\left[C_{i}\right],\\
\esp\left[I_{i}\right]&=&\left(1-\mu_{i}\right)\esp\left[C_{i}\right]\\
\end{eqnarray*}

Def\'inanse los puntos de regenaraci\'on  en el proceso $\left[L_{1}\left(t\right),L_{2}\left(t\right),\ldots,L_{N}\left(t\right)\right]$. Los puntos cuando la cola $i$ es visitada y todos los $L_{j}\left(\tau_{i}\left(m\right)\right)=0$ para $i=1,2$  son puntos de regeneraci\'on. Se llama ciclo regenerativo al intervalo entre dos puntos regenerativos sucesivos.

Sea $M_{i}$  el n\'umero de ciclos de visita en un ciclo regenerativo, y sea $C_{i}^{(m)}$, para $m=1,2,\ldots,M_{i}$ la duraci\'on del $m$-\'esimo ciclo de visita en un ciclo regenerativo. Se define el ciclo del tiempo de visita promedio $\esp\left[C_{i}\right]$ como

\begin{eqnarray*}
\esp\left[C_{i}\right]&=&\frac{\esp\left[\sum_{m=1}^{M_{i}}C_{i}^{(m)}\right]}{\esp\left[M_{i}\right]}
\end{eqnarray*}


En Stid72 y Heym82 se muestra que una condici\'on suficiente para que el proceso regenerativo 
estacionario sea un procesoo estacionario es que el valor esperado del tiempo del ciclo regenerativo sea finito:

\begin{eqnarray*}
\esp\left[\sum_{m=1}^{M_{i}}C_{i}^{(m)}\right]<\infty.
\end{eqnarray*}

como cada $C_{i}^{(m)}$ contiene intervalos de r\'eplica positivos, se tiene que $\esp\left[M_{i}\right]<\infty$, adem\'as, como $M_{i}>0$, se tiene que la condici\'on anterior es equivalente a tener que 

\begin{eqnarray*}
\esp\left[C_{i}\right]<\infty,
\end{eqnarray*}
por lo tanto una condici\'on suficiente para la existencia del proceso regenerativo est\'a dada por

\begin{eqnarray*}
\sum_{k=1}^{N}\mu_{k}<1.
\end{eqnarray*}

Sea la funci\'on generadora de momentos para $L_{i}$, el n\'umero de usuarios en la cola $Q_{i}\left(z\right)$ en cualquier momento, est\'a dada por el tiempo promedio de $z^{L_{i}\left(t\right)}$ sobre el ciclo regenerativo definido anteriormente:

\begin{eqnarray*}
Q_{i}\left(z\right)&=&\esp\left[z^{L_{i}\left(t\right)}\right]=\frac{\esp\left[\sum_{m=1}^{M_{i}}\sum_{t=\tau_{i}\left(m\right)}^{\tau_{i}\left(m+1\right)-1}z^{L_{i}\left(t\right)}\right]}{\esp\left[\sum_{m=1}^{M_{i}}\tau_{i}\left(m+1\right)-\tau_{i}\left(m\right)\right]}
\end{eqnarray*}

$M_{i}$ es un tiempo de paro en el proceso regenerativo con $\esp\left[M_{i}\right]<\infty$, se sigue del lema de Wald que:


\begin{eqnarray*}
\esp\left[\sum_{m=1}^{M_{i}}\sum_{t=\tau_{i}\left(m\right)}^{\tau_{i}\left(m+1\right)-1}z^{L_{i}\left(t\right)}\right]&=&\esp\left[M_{i}\right]\esp\left[\sum_{t=\tau_{i}\left(m\right)}^{\tau_{i}\left(m+1\right)-1}z^{L_{i}\left(t\right)}\right]\\
\esp\left[\sum_{m=1}^{M_{i}}\tau_{i}\left(m+1\right)-\tau_{i}\left(m\right)\right]&=&\esp\left[M_{i}\right]\esp\left[\tau_{i}\left(m+1\right)-\tau_{i}\left(m\right)\right]
\end{eqnarray*}

por tanto se tiene que


\begin{eqnarray*}
Q_{i}\left(z\right)&=&\frac{\esp\left[\sum_{t=\tau_{i}\left(m\right)}^{\tau_{i}\left(m+1\right)-1}z^{L_{i}\left(t\right)}\right]}{\esp\left[\tau_{i}\left(m+1\right)-\tau_{i}\left(m\right)\right]}
\end{eqnarray*}

observar que el denominador es simplemente la duraci\'on promedio del tiempo del ciclo.


Se puede demostrar (ver Hideaki Takagi 1986) que

\begin{eqnarray*}
\esp\left[\sum_{t=\tau_{i}\left(m\right)}^{\tau_{i}\left(m+1\right)-1}z^{L_{i}\left(t\right)}\right]=z\frac{F_{i}\left(z\right)-1}{z-P_{i}\left(z\right)}
\end{eqnarray*}

Durante el tiempo de intervisita para la cola $i$, $L_{i}\left(t\right)$ solamente se incrementa de manera que el incremento por intervalo de tiempo est\'a dado por la funci\'on generadora de probabilidades de $P_{i}\left(z\right)$, por tanto la suma sobre el tiempo de intervisita puede evaluarse como:

\begin{eqnarray*}
\esp\left[\sum_{t=\tau_{i}\left(m\right)}^{\tau_{i}\left(m+1\right)-1}z^{L_{i}\left(t\right)}\right]&=&\esp\left[\sum_{t=\tau_{i}\left(m\right)}^{\tau_{i}\left(m+1\right)-1}\left\{P_{i}\left(z\right)\right\}^{t-\overline{\tau}_{i}\left(m\right)}\right]=\frac{1-\esp\left[\left\{P_{i}\left(z\right)\right\}^{\tau_{i}\left(m+1\right)-\overline{\tau}_{i}\left(m\right)}\right]}{1-P_{i}\left(z\right)}\\
&=&\frac{1-I_{i}\left[P_{i}\left(z\right)\right]}{1-P_{i}\left(z\right)}
\end{eqnarray*}
por tanto

\begin{eqnarray*}
\esp\left[\sum_{t=\tau_{i}\left(m\right)}^{\tau_{i}\left(m+1\right)-1}z^{L_{i}\left(t\right)}\right]&=&\frac{1-F_{i}\left(z\right)}{1-P_{i}\left(z\right)}
\end{eqnarray*}

Haciendo uso de lo hasta ahora desarrollado se tiene que

\begin{eqnarray*}
Q_{i}\left(z\right)&=&\frac{1}{\esp\left[C_{i}\right]}\cdot\frac{1-F_{i}\left(z\right)}{P_{i}\left(z\right)-z}\cdot\frac{\left(1-z\right)P_{i}\left(z\right)}{1-P_{i}\left(z\right)}\\
&=&\frac{\mu_{i}\left(1-\mu_{i}\right)}{f_{i}\left(i\right)}\cdot\frac{1-F_{i}\left(z\right)}{P_{i}\left(z\right)-z}\cdot\frac{\left(1-z\right)P_{i}\left(z\right)}{1-P_{i}\left(z\right)}
\end{eqnarray*}


%___________________________________________________________________________________________
%\subsection{Longitudes de la Cola en cualquier tiempo}
%___________________________________________________________________________________________

Sea
$V_{i}\left(z\right)=\frac{1}{\esp\left[C_{i}\right]}\frac{I_{i}\left(z\right)-1}{z-P_{i}\left(z\right)}$

%{\esp\lef[I_{i}\right]}\frac{1-\mu_{i}}{z-P_{i}\left(z\right)}

\begin{eqnarray*}
\frac{\partial V_{i}\left(z\right)}{\partial z}&=&\frac{1}{\esp\left[C_{i}\right]}\left[\frac{I_{i}{'}\left(z\right)\left(z-P_{i}\left(z\right)\right)}{z-P_{i}\left(z\right)}-\frac{\left(I_{i}\left(z\right)-1\right)\left(1-P_{i}{'}\left(z\right)\right)}{\left(z-P_{i}\left(z\right)\right)^{2}}\right]
\end{eqnarray*}


La FGP para el tiempo de espera para cualquier usuario en la cola est\'a dada por:
\[U_{i}\left(z\right)=\frac{1}{\esp\left[C_{i}\right]}\cdot\frac{1-P_{i}\left(z\right)}{z-P_{i}\left(z\right)}\cdot\frac{I_{i}\left(z\right)-1}{1-z}\]

entonces


\begin{eqnarray*}
\frac{d}{dz}V_{i}\left(z\right)&=&\frac{1}{\esp\left[C_{i}\right]}\left\{\frac{d}{dz}\left(\frac{1-P_{i}\left(z\right)}{z-P_{i}\left(z\right)}\right)\frac{I_{i}\left(z\right)-1}{1-z}+\frac{1-P_{i}\left(z\right)}{z-P_{i}\left(z\right)}\frac{d}{dz}\left(\frac{I_{i}\left(z\right)-1}{1-z}\right)\right\}\\
&=&\frac{1}{\esp\left[C_{i}\right]}\left\{\frac{-P_{i}\left(z\right)\left(z-P_{i}\left(z\right)\right)-\left(1-P_{i}\left(z\right)\right)\left(1-P_{i}^{'}\left(z\right)\right)}{\left(z-P_{i}\left(z\right)\right)^{2}}\cdot\frac{I_{i}\left(z\right)-1}{1-z}\right\}\\
&+&\frac{1}{\esp\left[C_{i}\right]}\left\{\frac{1-P_{i}\left(z\right)}{z-P_{i}\left(z\right)}\cdot\frac{I_{i}^{'}\left(z\right)\left(1-z\right)+\left(I_{i}\left(z\right)-1\right)}{\left(1-z\right)^{2}}\right\}
\end{eqnarray*}
%\frac{I_{i}\left(z\right)-1}{1-z}
%+\frac{1-P_{i}\left(z\right)}{z-P_{i}\frac{d}{dz}\left(\frac{I_{i}\left(z\right)-1}{1-z}\right)


\begin{eqnarray*}
\frac{\partial U_{i}\left(z\right)}{\partial z}&=&\frac{(-1+I_{i}[z]) (1-P_{i}[z])}{(1-z)^2 \esp[I_{i}] (z-P_{i}[z])}+\frac{(1-P_{i}[z]) I_{i}^{'}[z]}{(1-z) \esp[I_{i}] (z-P_{i}[z])}-\frac{(-1+I_{i}[z]) (1-P_{i}[z])\left(1-P{'}[z]\right)}{(1-z) \esp[I_{i}] (z-P_{i}[z])^2}\\
&-&\frac{(-1+I_{i}[z]) P_{i}{'}[z]}{(1-z) \esp[I_{i}](z-P_{i}[z])}
\end{eqnarray*}
%___________________________________________________________________________________________
%\subsection{Longitudes de la Cola en cualquier tiempo}
%___________________________________________________________________________________________
Sea 
\begin{eqnarray*}
Q_{i}\left(z\right)&=&\frac{1}{\esp\left[C_{i}\right]}\cdot\frac{1-F_{i}\left(z\right)}{P_{i}\left(z\right)-z}\cdot\frac{\left(1-z\right)P_{i}\left(z\right)}{1-P_{i}\left(z\right)}
\end{eqnarray*}

derivando con respecto a $z$



\begin{eqnarray*}
\frac{d Q_{i}\left(z\right)}{d z}&=&\frac{\left(1-F_{i}\left(z\right)\right)P_{i}\left(z\right)}{\esp\left[C_{i}\right]\left(1-P_{i}\left(z\right)\right)\left(P_{i}\left(z\right)-z\right)}\\
&-&\frac{\left(1-z\right)P_{i}\left(z\right)F_{i}^{'}\left(z\right)}{\esp\left[C_{i}\right]\left(1-P_{i}\left(z\right)\right)\left(P_{i}\left(z\right)-z\right)}\\
&-&\frac{\left(1-z\right)\left(1-F_{i}\left(z\right)\right)P_{i}\left(z\right)\left(P_{i}^{'}\left(z\right)-1\right)}{\esp\left[C_{i}\right]\left(1-P_{i}\left(z\right)\right)\left(P_{i}\left(z\right)-z\right)^{2}}\\
&+&\frac{\left(1-z\right)\left(1-F_{i}\left(z\right)\right)P_{i}^{'}\left(z\right)}{\esp\left[C_{i}\right]\left(1-P_{i}\left(z\right)\right)\left(P_{i}\left(z\right)-z\right)}\\
&+&\frac{\left(1-z\right)\left(1-F_{i}\left(z\right)\right)P_{i}\left(z\right)P_{i}^{'}\left(z\right)}{\esp\left[C_{i}\right]\left(1-P_{i}\left(z\right)\right)^{2}\left(P_{i}\left(z\right)-z\right)}
\end{eqnarray*}

Calculando el l\'imite cuando $z\rightarrow1^{+}$:
\begin{eqnarray}
Q_{i}^{(1)}\left(z\right)=\lim_{z\rightarrow1^{+}}\frac{d Q_{i}\left(z\right)}{dz}&=&\lim_{z\rightarrow1}\frac{\left(1-F_{i}\left(z\right)\right)P_{i}\left(z\right)}{\esp\left[C_{i}\right]\left(1-P_{i}\left(z\right)\right)\left(P_{i}\left(z\right)-z\right)}\\
&-&\lim_{z\rightarrow1^{+}}\frac{\left(1-z\right)P_{i}\left(z\right)F_{i}^{'}\left(z\right)}{\esp\left[C_{i}\right]\left(1-P_{i}\left(z\right)\right)\left(P_{i}\left(z\right)-z\right)}\\
&-&\lim_{z\rightarrow1^{+}}\frac{\left(1-z\right)\left(1-F_{i}\left(z\right)\right)P_{i}\left(z\right)\left(P_{i}^{'}\left(z\right)-1\right)}{\esp\left[C_{i}\right]\left(1-P_{i}\left(z\right)\right)\left(P_{i}\left(z\right)-z\right)^{2}}\\
&+&\lim_{z\rightarrow1^{+}}\frac{\left(1-z\right)\left(1-F_{i}\left(z\right)\right)P_{i}^{'}\left(z\right)}{\esp\left[C_{i}\right]\left(1-P_{i}\left(z\right)\right)\left(P_{i}\left(z\right)-z\right)}\\
&+&\lim_{z\rightarrow1^{+}}\frac{\left(1-z\right)\left(1-F_{i}\left(z\right)\right)P_{i}\left(z\right)P_{i}^{'}\left(z\right)}{\esp\left[C_{i}\right]\left(1-P_{i}\left(z\right)\right)^{2}\left(P_{i}\left(z\right)-z\right)}
\end{eqnarray}

Entonces:
%______________________________________________________

\begin{eqnarray*}
\lim_{z\rightarrow1^{+}}\frac{\left(1-F_{i}\left(z\right)\right)P_{i}\left(z\right)}{\left(1-P_{i}\left(z\right)\right)\left(P_{i}\left(z\right)-z\right)}&=&\lim_{z\rightarrow1^{+}}\frac{\frac{d}{dz}\left[\left(1-F_{i}\left(z\right)\right)P_{i}\left(z\right)\right]}{\frac{d}{dz}\left[\left(1-P_{i}\left(z\right)\right)\left(-z+P_{i}\left(z\right)\right)\right]}\\
&=&\lim_{z\rightarrow1^{+}}\frac{-P_{i}\left(z\right)F_{i}^{'}\left(z\right)+\left(1-F_{i}\left(z\right)\right)P_{i}^{'}\left(z\right)}{\left(1-P_{i}\left(z\right)\right)\left(-1+P_{i}^{'}\left(z\right)\right)-\left(-z+P_{i}\left(z\right)\right)P_{i}^{'}\left(z\right)}
\end{eqnarray*}


%______________________________________________________


\begin{eqnarray*}
\lim_{z\rightarrow1^{+}}\frac{\left(1-z\right)P_{i}\left(z\right)F_{i}^{'}\left(z\right)}{\left(1-P_{i}\left(z\right)\right)\left(P_{i}\left(z\right)-z\right)}&=&\lim_{z\rightarrow1^{+}}\frac{\frac{d}{dz}\left[\left(1-z\right)P_{i}\left(z\right)F_{i}^{'}\left(z\right)\right]}{\frac{d}{dz}\left[\left(1-P_{i}\left(z\right)\right)\left(P_{i}\left(z\right)-z\right)\right]}\\
&=&\lim_{z\rightarrow1^{+}}\frac{-P_{i}\left(z\right) F_{i}^{'}\left(z\right)+(1-z) F_{i}^{'}\left(z\right) P_{i}^{'}\left(z\right)+(1-z) P_{i}\left(z\right)F_{i}^{''}\left(z\right)}{\left(1-P_{i}\left(z\right)\right)\left(-1+P_{i}^{'}\left(z\right)\right)-\left(-z+P_{i}\left(z\right)\right)P_{i}^{'}\left(z\right)}
\end{eqnarray*}


%______________________________________________________

\begin{eqnarray*}
&&\lim_{z\rightarrow1^{+}}\frac{\left(1-z\right)\left(1-F_{i}\left(z\right)\right)P_{i}\left(z\right)\left(P_{i}^{'}\left(z\right)-1\right)}{\left(1-P_{i}\left(z\right)\right)\left(P_{i}\left(z\right)-z\right)^{2}}=\lim_{z\rightarrow1^{+}}\frac{\frac{d}{dz}\left[\left(1-z\right)\left(1-F_{i}\left(z\right)\right)P_{i}\left(z\right)\left(P_{i}^{'}\left(z\right)-1\right)\right]}{\frac{d}{dz}\left[\left(1-P_{i}\left(z\right)\right)\left(P_{i}\left(z\right)-z\right)^{2}\right]}\\
&=&\lim_{z\rightarrow1^{+}}\frac{-\left(1-F_{i}\left(z\right)\right) P_{i}\left(z\right)\left(-1+P_{i}^{'}\left(z\right)\right)-(1-z) P_{i}\left(z\right)F_{i}^{'}\left(z\right)\left(-1+P_{i}^{'}\left(z\right)\right)}{2\left(1-P_{i}\left(z\right)\right)\left(-z+P_{i}\left(z\right)\right) \left(-1+P_{i}^{'}\left(z\right)\right)-\left(-z+P_{i}\left(z\right)\right)^2 P_{i}^{'}\left(z\right)}\\
&+&\lim_{z\rightarrow1^{+}}\frac{+(1-z) \left(1-F_{i}\left(z\right)\right) \left(-1+P_{i}^{'}\left(z\right)\right) P_{i}^{'}\left(z\right)}{{2\left(1-P_{i}\left(z\right)\right)\left(-z+P_{i}\left(z\right)\right) \left(-1+P_{i}^{'}\left(z\right)\right)-\left(-z+P_{i}\left(z\right)\right)^2 P_{i}^{'}\left(z\right)}}\\
&+&\lim_{z\rightarrow1^{+}}\frac{+(1-z) \left(1-F_{i}\left(z\right)\right) P_{i}\left(z\right)P_{i}^{''}\left(z\right)}{{2\left(1-P_{i}\left(z\right)\right)\left(-z+P_{i}\left(z\right)\right) \left(-1+P_{i}^{'}\left(z\right)\right)-\left(-z+P_{i}\left(z\right)\right)^2 P_{i}^{'}\left(z\right)}}
\end{eqnarray*}











%______________________________________________________
\begin{eqnarray*}
&&\lim_{z\rightarrow1^{+}}\frac{\left(1-z\right)\left(1-F_{i}\left(z\right)\right)P_{i}^{'}\left(z\right)}{\left(1-P_{i}\left(z\right)\right)\left(P_{i}\left(z\right)-z\right)}=\lim_{z\rightarrow1^{+}}\frac{\frac{d}{dz}\left[\left(1-z\right)\left(1-F_{i}\left(z\right)\right)P_{i}^{'}\left(z\right)\right]}{\frac{d}{dz}\left[\left(1-P_{i}\left(z\right)\right)\left(P_{i}\left(z\right)-z\right)\right]}\\
&=&\lim_{z\rightarrow1^{+}}\frac{-\left(1-F_{i}\left(z\right)\right) P_{i}^{'}\left(z\right)-(1-z) F_{i}^{'}\left(z\right) P_{i}^{'}\left(z\right)+(1-z) \left(1-F_{i}\left(z\right)\right) P_{i}^{''}\left(z\right)}{\left(1-P_{i}\left(z\right)\right) \left(-1+P_{i}^{'}\left(z\right)\right)-\left(-z+P_{i}\left(z\right)\right) P_{i}^{'}\left(z\right)}\frac{}{}
\end{eqnarray*}

%______________________________________________________
\begin{eqnarray*}
&&\lim_{z\rightarrow1^{+}}\frac{\left(1-z\right)\left(1-F_{i}\left(z\right)\right)P_{i}\left(z\right)P_{i}^{'}\left(z\right)}{\left(1-P_{i}\left(z\right)\right)^{2}\left(P_{i}\left(z\right)-z\right)}=\lim_{z\rightarrow1^{+}}\frac{\frac{d}{dz}\left[\left(1-z\right)\left(1-F_{i}\left(z\right)\right)P_{i}\left(z\right)P_{i}^{'}\left(z\right)\right]}{\frac{d}{dz}\left[\left(1-P_{i}\left(z\right)\right)^{2}\left(P_{i}\left(z\right)-z\right)\right]}\\
&=&\lim_{z\rightarrow1^{+}}\frac{-\left(1-F_{i}\left(z\right)\right) P_{i}\left(z\right) P_{i}^{'}\left(z\right)-(1-z) P_{i}\left(z\right) F_{i}^{'}\left(z\right)P_i'[z]}{\left(1-P_{i}\left(z\right)\right)^2 \left(-1+P_{i}^{'}\left(z\right)\right)-2 \left(1-P_{i}\left(z\right)\right) \left(-z+P_{i}\left(z\right)\right) P_{i}^{'}\left(z\right)}\\
&+&\lim_{z\rightarrow1^{+}}\frac{(1-z) \left(1-F_{i}\left(z\right)\right) P_{i}^{'}\left(z\right)^2+(1-z) \left(1-F_{i}\left(z\right)\right) P_{i}\left(z\right) P_{i}^{''}\left(z\right)}{\left(1-P_{i}\left(z\right)\right)^2 \left(-1+P_{i}^{'}\left(z\right)\right)-2 \left(1-P_{i}\left(z\right)\right) \left(-z+P_{i}\left(z\right)\right) P_{i}^{'}\left(z\right)}\\
\end{eqnarray*}




%_______________________________________________________________________________________________________
\subsection{Tiempo de Ciclo Promedio}
%_______________________________________________________________________________________________________

Consideremos una cola de la red de sistemas de visitas c\'iclicas fija, $Q_{l}$.


Conforme a la definici\'on dada al principio del cap\'itulo, definici\'on (\ref{Def.Tn}), sean $T_{1},T_{2},\ldots$ los puntos donde las longitudes de las colas de la red de sistemas de visitas c\'iclicas son cero simult\'aneamente, cuando la cola $Q_{l}$ es visitada por el servidor para dar servicio, es decir, $L_{1}\left(T_{i}\right)=0,L_{2}\left(T_{i}\right)=0,\hat{L}_{1}\left(T_{i}\right)=0$ y $\hat{L}_{2}\left(T_{i}\right)=0$, a estos puntos se les denominar\'a puntos regenerativos. Entonces, 

\begin{Def}
Al intervalo de tiempo entre dos puntos regenerativos se le llamar\'a ciclo regenerativo.
\end{Def}

\begin{Def}
Para $T_{i}$ se define, $M_{i}$, el n\'umero de ciclos de visita a la cola $Q_{l}$, durante el ciclo regenerativo, es decir, $M_{i}$ es un proceso de renovaci\'on.
\end{Def}

\begin{Def}
Para cada uno de los $M_{i}$'s, se definen a su vez la duraci\'on de cada uno de estos ciclos de visita en el ciclo regenerativo, $C_{i}^{(m)}$, para $m=1,2,\ldots,M_{i}$, que a su vez, tambi\'en es n proceso de renovaci\'on.
\end{Def}

En nuestra notaci\'on $V\left(t\right)\equiv C_{i}$ y $X_{i}=C_{i}^{(m)}$ para nuestra segunda definici\'on, mientras que para la primera la notaci\'on es: $X\left(t\right)\equiv C_{i}$ y $R_{i}\equiv C_{i}^{(m)}$.


%___________________________________________________________________________________________
\subsection{Tiempos de Ciclo e Intervisita}
%___________________________________________________________________________________________


\begin{Def}
Sea $L_{i}^{*}$el n\'umero de usuarios en la cola $Q_{i}$ cuando es visitada por el servidor para dar servicio, entonces

\begin{eqnarray}
\esp\left[L_{i}^{*}\right]&=&f_{i}\left(i\right)\\
Var\left[L_{i}^{*}\right]&=&f_{i}\left(i,i\right)+\esp\left[L_{i}^{*}\right]-\esp\left[L_{i}^{*}\right]^{2}.
\end{eqnarray}

\end{Def}

\begin{Def}
El tiempo de Ciclo $C_{i}$ es e periodo de tiempo que comienza cuando la cola $i$ es visitada por primera vez en un ciclo, y termina cuando es visitado nuevamente en el pr\'oximo ciclo. La duraci\'on del mismo est\'a dada por $\tau_{i}\left(m+1\right)-\tau_{i}\left(m\right)$, o equivalentemente $\overline{\tau}_{i}\left(m+1\right)-\overline{\tau}_{i}\left(m\right)$ bajo condiciones de estabilidad.
\end{Def}

\begin{Def}
El tiempo de intervisita $I_{i}$ es el periodo de tiempo que comienza cuando se ha completado el servicio en un ciclo y termina cuando es visitada nuevamente en el pr\'oximo ciclo. Su  duraci\'on del mismo est\'a dada por $\tau_{i}\left(m+1\right)-\overline{\tau}_{i}\left(m\right)$.
\end{Def}


Recordemos las siguientes expresiones:

\begin{eqnarray*}
S_{i}\left(z\right)&=&\esp\left[z^{\overline{\tau}_{i}\left(m\right)-\tau_{i}\left(m\right)}\right]=F_{i}\left(\theta\left(z\right)\right),\\
F\left(z\right)&=&\esp\left[z^{L_{0}}\right],\\
P\left(z\right)&=&\esp\left[z^{X_{n}}\right],\\
F_{i}\left(z\right)&=&\esp\left[z^{L_{i}\left(\tau_{i}\left(m\right)\right)}\right],
\theta_{i}\left(z\right)-zP_{i}
\end{eqnarray*}

entonces 

\begin{eqnarray*}
\esp\left[S_{i}\right]&=&\frac{\esp\left[L_{i}^{*}\right]}{1-\mu_{i}}=\frac{f_{i}\left(i\right)}{1-\mu_{i}},\\
Var\left[S_{i}\right]&=&\frac{Var\left[L_{i}^{*}\right]}{\left(1-\mu_{i}\right)^{2}}+\frac{\sigma^{2}\esp\left[L_{i}^{*}\right]}{\left(1-\mu_{i}\right)^{3}}
\end{eqnarray*}

donde recordemos que

\begin{eqnarray*}
Var\left[L_{i}^{*}\right]&=&f_{i}\left(i,i\right)+f_{i}\left(i\right)-f_{i}\left(i\right)^{2}.
\end{eqnarray*}

La duraci\'on del tiempo de intervisita es $\tau_{i}\left(m+1\right)-\overline{\tau}\left(m\right)$. Dado que el n\'umero de usuarios presentes en $Q_{i}$ al tiempo $t=\tau_{i}\left(m+1\right)$ es igual al n\'umero de arribos durante el intervalo de tiempo $\left[\overline{\tau}\left(m\right),\tau_{i}\left(m+1\right)\right]$ se tiene que


\begin{eqnarray*}
\esp\left[z_{i}^{L_{i}\left(\tau_{i}\left(m+1\right)\right)}\right]=\esp\left[\left\{P_{i}\left(z_{i}\right)\right\}^{\tau_{i}\left(m+1\right)-\overline{\tau}\left(m\right)}\right]
\end{eqnarray*}

entonces, si \begin{eqnarray*}I_{i}\left(z\right)&=&\esp\left[z^{\tau_{i}\left(m+1\right)-\overline{\tau}\left(m\right)}\right]\end{eqnarray*} se tienen que

\begin{eqnarray*}
F_{i}\left(z\right)=I_{i}\left[P_{i}\left(z\right)\right]
\end{eqnarray*}
para $i=1,2$, por tanto



\begin{eqnarray*}
\esp\left[L_{i}^{*}\right]&=&\mu_{i}\esp\left[I_{i}\right]\\
Var\left[L_{i}^{*}\right]&=&\mu_{i}^{2}Var\left[I_{i}\right]+\sigma^{2}\esp\left[I_{i}\right]
\end{eqnarray*}
para $i=1,2$, por tanto


\begin{eqnarray*}
\esp\left[I_{i}\right]&=&\frac{f_{i}\left(i\right)}{\mu_{i}},
\end{eqnarray*}
adem\'as

\begin{eqnarray*}
Var\left[I_{i}\right]&=&\frac{Var\left[L_{i}^{*}\right]}{\mu_{i}^{2}}-\frac{\sigma_{i}^{2}}{\mu_{i}^{2}}f_{i}\left(i\right).
\end{eqnarray*}


Si  $C_{i}\left(z\right)=\esp\left[z^{\overline{\tau}\left(m+1\right)-\overline{\tau}_{i}\left(m\right)}\right]$el tiempo de duraci\'on del ciclo, entonces, por lo hasta ahora establecido, se tiene que

\begin{eqnarray*}
C_{i}\left(z\right)=I_{i}\left[\theta_{i}\left(z\right)\right],
\end{eqnarray*}
entonces

\begin{eqnarray*}
\esp\left[C_{i}\right]&=&\esp\left[I_{i}\right]\esp\left[\theta_{i}\left(z\right)\right]=\frac{\esp\left[L_{i}^{*}\right]}{\mu_{i}}\frac{1}{1-\mu_{i}}=\frac{f_{i}\left(i\right)}{\mu_{i}\left(1-\mu_{i}\right)}\\
Var\left[C_{i}\right]&=&\frac{Var\left[L_{i}^{*}\right]}{\mu_{i}^{2}\left(1-\mu_{i}\right)^{2}}.
\end{eqnarray*}

Por tanto se tienen las siguientes igualdades


\begin{eqnarray*}
\esp\left[S_{i}\right]&=&\mu_{i}\esp\left[C_{i}\right],\\
\esp\left[I_{i}\right]&=&\left(1-\mu_{i}\right)\esp\left[C_{i}\right]\\
\end{eqnarray*}

Def\'inanse los puntos de regenaraci\'on  en el proceso $\left[L_{1}\left(t\right),L_{2}\left(t\right),\ldots,L_{N}\left(t\right)\right]$. Los puntos cuando la cola $i$ es visitada y todos los $L_{j}\left(\tau_{i}\left(m\right)\right)=0$ para $i=1,2$  son puntos de regeneraci\'on. Se llama ciclo regenerativo al intervalo entre dos puntos regenerativos sucesivos.

Sea $M_{i}$  el n\'umero de ciclos de visita en un ciclo regenerativo, y sea $C_{i}^{(m)}$, para $m=1,2,\ldots,M_{i}$ la duraci\'on del $m$-\'esimo ciclo de visita en un ciclo regenerativo. Se define el ciclo del tiempo de visita promedio $\esp\left[C_{i}\right]$ como

\begin{eqnarray*}
\esp\left[C_{i}\right]&=&\frac{\esp\left[\sum_{m=1}^{M_{i}}C_{i}^{(m)}\right]}{\esp\left[M_{i}\right]}
\end{eqnarray*}


En Stid72 y Heym82 se muestra que una condici\'on suficiente para que el proceso regenerativo 
estacionario sea un procesoo estacionario es que el valor esperado del tiempo del ciclo regenerativo sea finito:

\begin{eqnarray*}
\esp\left[\sum_{m=1}^{M_{i}}C_{i}^{(m)}\right]<\infty.
\end{eqnarray*}

como cada $C_{i}^{(m)}$ contiene intervalos de r\'eplica positivos, se tiene que $\esp\left[M_{i}\right]<\infty$, adem\'as, como $M_{i}>0$, se tiene que la condici\'on anterior es equivalente a tener que 

\begin{eqnarray*}
\esp\left[C_{i}\right]<\infty,
\end{eqnarray*}
por lo tanto una condici\'on suficiente para la existencia del proceso regenerativo est\'a dada por

\begin{eqnarray*}
\sum_{k=1}^{N}\mu_{k}<1.
\end{eqnarray*}

Sea la funci\'on generadora de momentos para $L_{i}$, el n\'umero de usuarios en la cola $Q_{i}\left(z\right)$ en cualquier momento, est\'a dada por el tiempo promedio de $z^{L_{i}\left(t\right)}$ sobre el ciclo regenerativo definido anteriormente:

\begin{eqnarray*}
Q_{i}\left(z\right)&=&\esp\left[z^{L_{i}\left(t\right)}\right]=\frac{\esp\left[\sum_{m=1}^{M_{i}}\sum_{t=\tau_{i}\left(m\right)}^{\tau_{i}\left(m+1\right)-1}z^{L_{i}\left(t\right)}\right]}{\esp\left[\sum_{m=1}^{M_{i}}\tau_{i}\left(m+1\right)-\tau_{i}\left(m\right)\right]}
\end{eqnarray*}

$M_{i}$ es un tiempo de paro en el proceso regenerativo con $\esp\left[M_{i}\right]<\infty$, se sigue del lema de Wald que:


\begin{eqnarray*}
\esp\left[\sum_{m=1}^{M_{i}}\sum_{t=\tau_{i}\left(m\right)}^{\tau_{i}\left(m+1\right)-1}z^{L_{i}\left(t\right)}\right]&=&\esp\left[M_{i}\right]\esp\left[\sum_{t=\tau_{i}\left(m\right)}^{\tau_{i}\left(m+1\right)-1}z^{L_{i}\left(t\right)}\right]\\
\esp\left[\sum_{m=1}^{M_{i}}\tau_{i}\left(m+1\right)-\tau_{i}\left(m\right)\right]&=&\esp\left[M_{i}\right]\esp\left[\tau_{i}\left(m+1\right)-\tau_{i}\left(m\right)\right]
\end{eqnarray*}

por tanto se tiene que


\begin{eqnarray*}
Q_{i}\left(z\right)&=&\frac{\esp\left[\sum_{t=\tau_{i}\left(m\right)}^{\tau_{i}\left(m+1\right)-1}z^{L_{i}\left(t\right)}\right]}{\esp\left[\tau_{i}\left(m+1\right)-\tau_{i}\left(m\right)\right]}
\end{eqnarray*}

observar que el denominador es simplemente la duraci\'on promedio del tiempo del ciclo.


Se puede demostrar (ver Hideaki Takagi 1986) que

\begin{eqnarray*}
\esp\left[\sum_{t=\tau_{i}\left(m\right)}^{\tau_{i}\left(m+1\right)-1}z^{L_{i}\left(t\right)}\right]=z\frac{F_{i}\left(z\right)-1}{z-P_{i}\left(z\right)}
\end{eqnarray*}

Durante el tiempo de intervisita para la cola $i$, $L_{i}\left(t\right)$ solamente se incrementa de manera que el incremento por intervalo de tiempo est\'a dado por la funci\'on generadora de probabilidades de $P_{i}\left(z\right)$, por tanto la suma sobre el tiempo de intervisita puede evaluarse como:

\begin{eqnarray*}
\esp\left[\sum_{t=\tau_{i}\left(m\right)}^{\tau_{i}\left(m+1\right)-1}z^{L_{i}\left(t\right)}\right]&=&\esp\left[\sum_{t=\tau_{i}\left(m\right)}^{\tau_{i}\left(m+1\right)-1}\left\{P_{i}\left(z\right)\right\}^{t-\overline{\tau}_{i}\left(m\right)}\right]=\frac{1-\esp\left[\left\{P_{i}\left(z\right)\right\}^{\tau_{i}\left(m+1\right)-\overline{\tau}_{i}\left(m\right)}\right]}{1-P_{i}\left(z\right)}\\
&=&\frac{1-I_{i}\left[P_{i}\left(z\right)\right]}{1-P_{i}\left(z\right)}
\end{eqnarray*}
por tanto

\begin{eqnarray*}
\esp\left[\sum_{t=\tau_{i}\left(m\right)}^{\tau_{i}\left(m+1\right)-1}z^{L_{i}\left(t\right)}\right]&=&\frac{1-F_{i}\left(z\right)}{1-P_{i}\left(z\right)}
\end{eqnarray*}

Haciendo uso de lo hasta ahora desarrollado se tiene que

\begin{eqnarray*}
Q_{i}\left(z\right)&=&\frac{1}{\esp\left[C_{i}\right]}\cdot\frac{1-F_{i}\left(z\right)}{P_{i}\left(z\right)-z}\cdot\frac{\left(1-z\right)P_{i}\left(z\right)}{1-P_{i}\left(z\right)}\\
&=&\frac{\mu_{i}\left(1-\mu_{i}\right)}{f_{i}\left(i\right)}\cdot\frac{1-F_{i}\left(z\right)}{P_{i}\left(z\right)-z}\cdot\frac{\left(1-z\right)P_{i}\left(z\right)}{1-P_{i}\left(z\right)}
\end{eqnarray*}


%___________________________________________________________________________________________
\subsection{Longitudes de la Cola en cualquier tiempo}
%___________________________________________________________________________________________
Sea 
\begin{eqnarray*}
Q_{i}\left(z\right)&=&\frac{1}{\esp\left[C_{i}\right]}\cdot\frac{1-F_{i}\left(z\right)}{P_{i}\left(z\right)-z}\cdot\frac{\left(1-z\right)P_{i}\left(z\right)}{1-P_{i}\left(z\right)}
\end{eqnarray*}

derivando con respecto a $z$



\begin{eqnarray*}
\frac{d Q_{i}\left(z\right)}{d z}&=&\frac{\left(1-F_{i}\left(z\right)\right)P_{i}\left(z\right)}{\esp\left[C_{i}\right]\left(1-P_{i}\left(z\right)\right)\left(P_{i}\left(z\right)-z\right)}\\
&-&\frac{\left(1-z\right)P_{i}\left(z\right)F_{i}^{'}\left(z\right)}{\esp\left[C_{i}\right]\left(1-P_{i}\left(z\right)\right)\left(P_{i}\left(z\right)-z\right)}\\
&-&\frac{\left(1-z\right)\left(1-F_{i}\left(z\right)\right)P_{i}\left(z\right)\left(P_{i}^{'}\left(z\right)-1\right)}{\esp\left[C_{i}\right]\left(1-P_{i}\left(z\right)\right)\left(P_{i}\left(z\right)-z\right)^{2}}\\
&+&\frac{\left(1-z\right)\left(1-F_{i}\left(z\right)\right)P_{i}^{'}\left(z\right)}{\esp\left[C_{i}\right]\left(1-P_{i}\left(z\right)\right)\left(P_{i}\left(z\right)-z\right)}\\
&+&\frac{\left(1-z\right)\left(1-F_{i}\left(z\right)\right)P_{i}\left(z\right)P_{i}^{'}\left(z\right)}{\esp\left[C_{i}\right]\left(1-P_{i}\left(z\right)\right)^{2}\left(P_{i}\left(z\right)-z\right)}
\end{eqnarray*}

Calculando el l\'imite cuando $z\rightarrow1^{+}$:
\begin{eqnarray}
Q_{i}^{(1)}\left(z\right)=\lim_{z\rightarrow1^{+}}\frac{d Q_{i}\left(z\right)}{dz}&=&\lim_{z\rightarrow1}\frac{\left(1-F_{i}\left(z\right)\right)P_{i}\left(z\right)}{\esp\left[C_{i}\right]\left(1-P_{i}\left(z\right)\right)\left(P_{i}\left(z\right)-z\right)}\\
&-&\lim_{z\rightarrow1^{+}}\frac{\left(1-z\right)P_{i}\left(z\right)F_{i}^{'}\left(z\right)}{\esp\left[C_{i}\right]\left(1-P_{i}\left(z\right)\right)\left(P_{i}\left(z\right)-z\right)}\\
&-&\lim_{z\rightarrow1^{+}}\frac{\left(1-z\right)\left(1-F_{i}\left(z\right)\right)P_{i}\left(z\right)\left(P_{i}^{'}\left(z\right)-1\right)}{\esp\left[C_{i}\right]\left(1-P_{i}\left(z\right)\right)\left(P_{i}\left(z\right)-z\right)^{2}}\\
&+&\lim_{z\rightarrow1^{+}}\frac{\left(1-z\right)\left(1-F_{i}\left(z\right)\right)P_{i}^{'}\left(z\right)}{\esp\left[C_{i}\right]\left(1-P_{i}\left(z\right)\right)\left(P_{i}\left(z\right)-z\right)}\\
&+&\lim_{z\rightarrow1^{+}}\frac{\left(1-z\right)\left(1-F_{i}\left(z\right)\right)P_{i}\left(z\right)P_{i}^{'}\left(z\right)}{\esp\left[C_{i}\right]\left(1-P_{i}\left(z\right)\right)^{2}\left(P_{i}\left(z\right)-z\right)}
\end{eqnarray}

Entonces:
%______________________________________________________

\begin{eqnarray*}
\lim_{z\rightarrow1^{+}}\frac{\left(1-F_{i}\left(z\right)\right)P_{i}\left(z\right)}{\left(1-P_{i}\left(z\right)\right)\left(P_{i}\left(z\right)-z\right)}&=&\lim_{z\rightarrow1^{+}}\frac{\frac{d}{dz}\left[\left(1-F_{i}\left(z\right)\right)P_{i}\left(z\right)\right]}{\frac{d}{dz}\left[\left(1-P_{i}\left(z\right)\right)\left(-z+P_{i}\left(z\right)\right)\right]}\\
&=&\lim_{z\rightarrow1^{+}}\frac{-P_{i}\left(z\right)F_{i}^{'}\left(z\right)+\left(1-F_{i}\left(z\right)\right)P_{i}^{'}\left(z\right)}{\left(1-P_{i}\left(z\right)\right)\left(-1+P_{i}^{'}\left(z\right)\right)-\left(-z+P_{i}\left(z\right)\right)P_{i}^{'}\left(z\right)}
\end{eqnarray*}


%______________________________________________________


\begin{eqnarray*}
\lim_{z\rightarrow1^{+}}\frac{\left(1-z\right)P_{i}\left(z\right)F_{i}^{'}\left(z\right)}{\left(1-P_{i}\left(z\right)\right)\left(P_{i}\left(z\right)-z\right)}&=&\lim_{z\rightarrow1^{+}}\frac{\frac{d}{dz}\left[\left(1-z\right)P_{i}\left(z\right)F_{i}^{'}\left(z\right)\right]}{\frac{d}{dz}\left[\left(1-P_{i}\left(z\right)\right)\left(P_{i}\left(z\right)-z\right)\right]}\\
&=&\lim_{z\rightarrow1^{+}}\frac{-P_{i}\left(z\right) F_{i}^{'}\left(z\right)+(1-z) F_{i}^{'}\left(z\right) P_{i}^{'}\left(z\right)+(1-z) P_{i}\left(z\right)F_{i}^{''}\left(z\right)}{\left(1-P_{i}\left(z\right)\right)\left(-1+P_{i}^{'}\left(z\right)\right)-\left(-z+P_{i}\left(z\right)\right)P_{i}^{'}\left(z\right)}
\end{eqnarray*}


%______________________________________________________

\begin{eqnarray*}
&&\lim_{z\rightarrow1^{+}}\frac{\left(1-z\right)\left(1-F_{i}\left(z\right)\right)P_{i}\left(z\right)\left(P_{i}^{'}\left(z\right)-1\right)}{\left(1-P_{i}\left(z\right)\right)\left(P_{i}\left(z\right)-z\right)^{2}}=\lim_{z\rightarrow1^{+}}\frac{\frac{d}{dz}\left[\left(1-z\right)\left(1-F_{i}\left(z\right)\right)P_{i}\left(z\right)\left(P_{i}^{'}\left(z\right)-1\right)\right]}{\frac{d}{dz}\left[\left(1-P_{i}\left(z\right)\right)\left(P_{i}\left(z\right)-z\right)^{2}\right]}\\
&=&\lim_{z\rightarrow1^{+}}\frac{-\left(1-F_{i}\left(z\right)\right) P_{i}\left(z\right)\left(-1+P_{i}^{'}\left(z\right)\right)-(1-z) P_{i}\left(z\right)F_{i}^{'}\left(z\right)\left(-1+P_{i}^{'}\left(z\right)\right)}{2\left(1-P_{i}\left(z\right)\right)\left(-z+P_{i}\left(z\right)\right) \left(-1+P_{i}^{'}\left(z\right)\right)-\left(-z+P_{i}\left(z\right)\right)^2 P_{i}^{'}\left(z\right)}\\
&+&\lim_{z\rightarrow1^{+}}\frac{+(1-z) \left(1-F_{i}\left(z\right)\right) \left(-1+P_{i}^{'}\left(z\right)\right) P_{i}^{'}\left(z\right)}{{2\left(1-P_{i}\left(z\right)\right)\left(-z+P_{i}\left(z\right)\right) \left(-1+P_{i}^{'}\left(z\right)\right)-\left(-z+P_{i}\left(z\right)\right)^2 P_{i}^{'}\left(z\right)}}\\
&+&\lim_{z\rightarrow1^{+}}\frac{+(1-z) \left(1-F_{i}\left(z\right)\right) P_{i}\left(z\right)P_{i}^{''}\left(z\right)}{{2\left(1-P_{i}\left(z\right)\right)\left(-z+P_{i}\left(z\right)\right) \left(-1+P_{i}^{'}\left(z\right)\right)-\left(-z+P_{i}\left(z\right)\right)^2 P_{i}^{'}\left(z\right)}}
\end{eqnarray*}











%______________________________________________________
\begin{eqnarray*}
&&\lim_{z\rightarrow1^{+}}\frac{\left(1-z\right)\left(1-F_{i}\left(z\right)\right)P_{i}^{'}\left(z\right)}{\left(1-P_{i}\left(z\right)\right)\left(P_{i}\left(z\right)-z\right)}=\lim_{z\rightarrow1^{+}}\frac{\frac{d}{dz}\left[\left(1-z\right)\left(1-F_{i}\left(z\right)\right)P_{i}^{'}\left(z\right)\right]}{\frac{d}{dz}\left[\left(1-P_{i}\left(z\right)\right)\left(P_{i}\left(z\right)-z\right)\right]}\\
&=&\lim_{z\rightarrow1^{+}}\frac{-\left(1-F_{i}\left(z\right)\right) P_{i}^{'}\left(z\right)-(1-z) F_{i}^{'}\left(z\right) P_{i}^{'}\left(z\right)+(1-z) \left(1-F_{i}\left(z\right)\right) P_{i}^{''}\left(z\right)}{\left(1-P_{i}\left(z\right)\right) \left(-1+P_{i}^{'}\left(z\right)\right)-\left(-z+P_{i}\left(z\right)\right) P_{i}^{'}\left(z\right)}\frac{}{}
\end{eqnarray*}

%______________________________________________________
\begin{eqnarray*}
&&\lim_{z\rightarrow1^{+}}\frac{\left(1-z\right)\left(1-F_{i}\left(z\right)\right)P_{i}\left(z\right)P_{i}^{'}\left(z\right)}{\left(1-P_{i}\left(z\right)\right)^{2}\left(P_{i}\left(z\right)-z\right)}=\lim_{z\rightarrow1^{+}}\frac{\frac{d}{dz}\left[\left(1-z\right)\left(1-F_{i}\left(z\right)\right)P_{i}\left(z\right)P_{i}^{'}\left(z\right)\right]}{\frac{d}{dz}\left[\left(1-P_{i}\left(z\right)\right)^{2}\left(P_{i}\left(z\right)-z\right)\right]}\\
&=&\lim_{z\rightarrow1^{+}}\frac{-\left(1-F_{i}\left(z\right)\right) P_{i}\left(z\right) P_{i}^{'}\left(z\right)-(1-z) P_{i}\left(z\right) F_{i}^{'}\left(z\right)P_i'[z]}{\left(1-P_{i}\left(z\right)\right)^2 \left(-1+P_{i}^{'}\left(z\right)\right)-2 \left(1-P_{i}\left(z\right)\right) \left(-z+P_{i}\left(z\right)\right) P_{i}^{'}\left(z\right)}\\
&+&\lim_{z\rightarrow1^{+}}\frac{(1-z) \left(1-F_{i}\left(z\right)\right) P_{i}^{'}\left(z\right)^2+(1-z) \left(1-F_{i}\left(z\right)\right) P_{i}\left(z\right) P_{i}^{''}\left(z\right)}{\left(1-P_{i}\left(z\right)\right)^2 \left(-1+P_{i}^{'}\left(z\right)\right)-2 \left(1-P_{i}\left(z\right)\right) \left(-z+P_{i}\left(z\right)\right) P_{i}^{'}\left(z\right)}\\
\end{eqnarray*}

\subsection{Por resolver}



\begin{eqnarray*}
&&\frac{\partial Q_{i}\left(z\right)}{\partial z}=\frac{1}{\esp\left[C_{i}\right]}\frac{\partial}{\partial z}\left\{\frac{1-F_{i}\left(z\right)}{P_{i}\left(z\right)-z}\cdot\frac{\left(1-z\right)P_{i}\left(z\right)}{1-P_{i}\left(z\right)}\right\}\\
&=&\frac{1}{\esp\left[C_{i}\right]}\left\{\frac{\partial}{\partial z}\left(\frac{1-F_{i}\left(z\right)}{P_{i}\left(z\right)-z}\right)\cdot\frac{\left(1-z\right)P_{i}\left(z\right)}{1-P_{i}\left(z\right)}+\frac{1-F_{i}\left(z\right)}{P_{i}\left(z\right)-z}\cdot\frac{\partial}{\partial z}\left(\frac{\left(1-z\right)P_{i}\left(z\right)}{1-P_{i}\left(z\right)}\right)\right\}\\
&=&\frac{1}{\esp\left[C_{i}\right]}\cdot\frac{\left(1-z\right)P_{i}\left(z\right)}{1-P_{i}\left(z\right)}\cdot\frac{\partial}{\partial z}\left(\frac{1-F_{i}\left(z\right)}{P_{i}\left(z\right)-z}\right)+\frac{1}{\esp\left[C_{i}\right]}\cdot\frac{1-F_{i}\left(z\right)}{P_{i}\left(z\right)-z}\cdot\frac{\partial}{\partial z}\left(\frac{\left(1-z\right)P_{i}\left(z\right)}{1-P_{i}\left(z\right)}\right)\\
&=&\frac{1}{\esp\left[C_{i}\right]}\cdot\frac{\left(1-z\right)P_{i}\left(z\right)}{1-P_{i}\left(z\right)}\cdot\frac{-F_{i}^{'}\left(z\right)\left(P_{i}\left(z\right)-z\right)-\left(1-F_{i}\left(z\right)\right)\left(P_{i}^{'}\left(z\right)-1\right)}{\left(P_{i}\left(z\right)-z\right)^{2}}\\
&+&\frac{1}{\esp\left[C_{i}\right]}\cdot\frac{1-F_{i}\left(z\right)}{P_{i}\left(z\right)-z}\cdot\frac{\left(1-z\right)P_{i}^{'}\left(z\right)-P_{i}\left(z\right)}{\left(1-P_{i}\left(z\right)\right)^{2}}
\end{eqnarray*}



\begin{eqnarray*}
Q_{i}^{(1)}\left(z\right)&=& \frac{\left(1-F_{i}\left(z\right)\right)P_{i}\left(z\right)}{\esp\left[C_{i}\right]\left(1-P_{i}\left(z\right)\right)\left(P_{i}\left(z\right)-z\right)}
-\frac{\left(1-z\right)P_{i}\left(z\right)F_{i}^{'}\left(z\right)}{\esp\left[C_{i}\right]\left(1-P_{i}\left(z\right)\right)\left(P_{i}\left(z\right)-z\right)}\\
&-&\frac{\left(1-z\right)\left(1-F_{i}\left(z\right)\right)P_{i}\left(z\right)\left(P_{i}^{'}\left(z\right)-1\right)}{\esp\left[C_{i}\right]\left(1-P_{i}\left(z\right)\right)\left(P_{i}\left(z\right)-z\right)^{2}}+\frac{\left(1-z\right)\left(1-F_{i}\left(z\right)\right)P_{i}^{'}\left(z\right)}{\esp\left[C_{i}\right]\left(1-P_{i}\left(z\right)\right)\left(P_{i}\left(z\right)-z\right)}\\
&+&\frac{\left(1-z\right)\left(1-F_{i}\left(z\right)\right)P_{i}\left(z\right)P_{i}^{'}\left(z\right)}{\esp\left[C_{i}\right]\left(1-P_{i}\left(z\right)\right)^{2}\left(P_{i}\left(z\right)-z\right)}
\end{eqnarray*}
%___________________________________________________________________________________________
%\subsection{Operaciones Matemathica: Tiempos de Espera}
%___________________________________________________________________________________________
Sea
$V_{i}\left(z\right)=\frac{1}{\esp\left[C_{i}\right]}\frac{I_{i}\left(z\right)-1}{z-P_{i}\left(z\right)}$

%{\esp\lef[I_{i}\right]}\frac{1-\mu_{i}}{z-P_{i}\left(z\right)}

\begin{eqnarray*}
\frac{\partial V_{i}\left(z\right)}{\partial z}&=&\frac{1}{\esp\left[C_{i}\right]}\left[\frac{I_{i}{'}\left(z\right)\left(z-P_{i}\left(z\right)\right)}{z-P_{i}\left(z\right)}-\frac{\left(I_{i}\left(z\right)-1\right)\left(1-P_{i}{'}\left(z\right)\right)}{\left(z-P_{i}\left(z\right)\right)^{2}}\right]
\end{eqnarray*}


La FGP para el tiempo de espera para cualquier usuario en la cola est\'a dada por:
\[U_{i}\left(z\right)=\frac{1}{\esp\left[C_{i}\right]}\cdot\frac{1-P_{i}\left(z\right)}{z-P_{i}\left(z\right)}\cdot\frac{I_{i}\left(z\right)-1}{1-z}\]

entonces


\begin{eqnarray*}
\frac{d}{dz}V_{i}\left(z\right)&=&\frac{1}{\esp\left[C_{i}\right]}\left\{\frac{d}{dz}\left(\frac{1-P_{i}\left(z\right)}{z-P_{i}\left(z\right)}\right)\frac{I_{i}\left(z\right)-1}{1-z}+\frac{1-P_{i}\left(z\right)}{z-P_{i}\left(z\right)}\frac{d}{dz}\left(\frac{I_{i}\left(z\right)-1}{1-z}\right)\right\}\\
&=&\frac{1}{\esp\left[C_{i}\right]}\left\{\frac{-P_{i}\left(z\right)\left(z-P_{i}\left(z\right)\right)-\left(1-P_{i}\left(z\right)\right)\left(1-P_{i}^{'}\left(z\right)\right)}{\left(z-P_{i}\left(z\right)\right)^{2}}\cdot\frac{I_{i}\left(z\right)-1}{1-z}\right\}\\
&+&\frac{1}{\esp\left[C_{i}\right]}\left\{\frac{1-P_{i}\left(z\right)}{z-P_{i}\left(z\right)}\cdot\frac{I_{i}^{'}\left(z\right)\left(1-z\right)+\left(I_{i}\left(z\right)-1\right)}{\left(1-z\right)^{2}}\right\}
\end{eqnarray*}
%\frac{I_{i}\left(z\right)-1}{1-z}
%+\frac{1-P_{i}\left(z\right)}{z-P_{i}\frac{d}{dz}\left(\frac{I_{i}\left(z\right)-1}{1-z}\right)


\begin{eqnarray*}
\frac{\partial U_{i}\left(z\right)}{\partial z}&=&\frac{(-1+I_{i}[z]) (1-P_{i}[z])}{(1-z)^2 \esp[I_{i}] (z-P_{i}[z])}+\frac{(1-P_{i}[z]) I_{i}^{'}[z]}{(1-z) \esp[I_{i}] (z-P_{i}[z])}-\frac{(-1+I_{i}[z]) (1-P_{i}[z])\left(1-P{'}[z]\right)}{(1-z) \esp[I_{i}] (z-P_{i}[z])^2}\\
&-&\frac{(-1+I_{i}[z]) P_{i}{'}[z]}{(1-z) \esp[I_{i}](z-P_{i}[z])}
\end{eqnarray*}

%___________________________________________________________________________________________
\subsection{Tiempos de Ciclo e Intervisita}
%___________________________________________________________________________________________


\begin{Def}
Sea $L_{i}^{*}$el n\'umero de usuarios en la cola $Q_{i}$ cuando es visitada por el servidor para dar servicio, entonces

\begin{eqnarray}
\esp\left[L_{i}^{*}\right]&=&f_{i}\left(i\right)\\
Var\left[L_{i}^{*}\right]&=&f_{i}\left(i,i\right)+\esp\left[L_{i}^{*}\right]-\esp\left[L_{i}^{*}\right]^{2}.
\end{eqnarray}

\end{Def}

\begin{Def}
El tiempo de Ciclo $C_{i}$ es e periodo de tiempo que comienza cuando la cola $i$ es visitada por primera vez en un ciclo, y termina cuando es visitado nuevamente en el pr\'oximo ciclo. La duraci\'on del mismo est\'a dada por $\tau_{i}\left(m+1\right)-\tau_{i}\left(m\right)$, o equivalentemente $\overline{\tau}_{i}\left(m+1\right)-\overline{\tau}_{i}\left(m\right)$ bajo condiciones de estabilidad.
\end{Def}

\begin{Def}
El tiempo de intervisita $I_{i}$ es el periodo de tiempo que comienza cuando se ha completado el servicio en un ciclo y termina cuando es visitada nuevamente en el pr\'oximo ciclo. Su  duraci\'on del mismo est\'a dada por $\tau_{i}\left(m+1\right)-\overline{\tau}_{i}\left(m\right)$.
\end{Def}


Recordemos las siguientes expresiones:

\begin{eqnarray*}
S_{i}\left(z\right)&=&\esp\left[z^{\overline{\tau}_{i}\left(m\right)-\tau_{i}\left(m\right)}\right]=F_{i}\left(\theta\left(z\right)\right),\\
F\left(z\right)&=&\esp\left[z^{L_{0}}\right],\\
P\left(z\right)&=&\esp\left[z^{X_{n}}\right],\\
F_{i}\left(z\right)&=&\esp\left[z^{L_{i}\left(\tau_{i}\left(m\right)\right)}\right],
\theta_{i}\left(z\right)-zP_{i}
\end{eqnarray*}

entonces 

\begin{eqnarray*}
\esp\left[S_{i}\right]&=&\frac{\esp\left[L_{i}^{*}\right]}{1-\mu_{i}}=\frac{f_{i}\left(i\right)}{1-\mu_{i}},\\
Var\left[S_{i}\right]&=&\frac{Var\left[L_{i}^{*}\right]}{\left(1-\mu_{i}\right)^{2}}+\frac{\sigma^{2}\esp\left[L_{i}^{*}\right]}{\left(1-\mu_{i}\right)^{3}}
\end{eqnarray*}

donde recordemos que

\begin{eqnarray*}
Var\left[L_{i}^{*}\right]&=&f_{i}\left(i,i\right)+f_{i}\left(i\right)-f_{i}\left(i\right)^{2}.
\end{eqnarray*}

La duraci\'on del tiempo de intervisita es $\tau_{i}\left(m+1\right)-\overline{\tau}\left(m\right)$. Dado que el n\'umero de usuarios presentes en $Q_{i}$ al tiempo $t=\tau_{i}\left(m+1\right)$ es igual al n\'umero de arribos durante el intervalo de tiempo $\left[\overline{\tau}\left(m\right),\tau_{i}\left(m+1\right)\right]$ se tiene que


\begin{eqnarray*}
\esp\left[z_{i}^{L_{i}\left(\tau_{i}\left(m+1\right)\right)}\right]=\esp\left[\left\{P_{i}\left(z_{i}\right)\right\}^{\tau_{i}\left(m+1\right)-\overline{\tau}\left(m\right)}\right]
\end{eqnarray*}

entonces, si \begin{eqnarray*}I_{i}\left(z\right)&=&\esp\left[z^{\tau_{i}\left(m+1\right)-\overline{\tau}\left(m\right)}\right]\end{eqnarray*} se tienen que

\begin{eqnarray*}
F_{i}\left(z\right)=I_{i}\left[P_{i}\left(z\right)\right]
\end{eqnarray*}
para $i=1,2$, por tanto



\begin{eqnarray*}
\esp\left[L_{i}^{*}\right]&=&\mu_{i}\esp\left[I_{i}\right]\\
Var\left[L_{i}^{*}\right]&=&\mu_{i}^{2}Var\left[I_{i}\right]+\sigma^{2}\esp\left[I_{i}\right]
\end{eqnarray*}
para $i=1,2$, por tanto


\begin{eqnarray*}
\esp\left[I_{i}\right]&=&\frac{f_{i}\left(i\right)}{\mu_{i}},
\end{eqnarray*}
adem\'as

\begin{eqnarray*}
Var\left[I_{i}\right]&=&\frac{Var\left[L_{i}^{*}\right]}{\mu_{i}^{2}}-\frac{\sigma_{i}^{2}}{\mu_{i}^{2}}f_{i}\left(i\right).
\end{eqnarray*}


Si  $C_{i}\left(z\right)=\esp\left[z^{\overline{\tau}\left(m+1\right)-\overline{\tau}_{i}\left(m\right)}\right]$el tiempo de duraci\'on del ciclo, entonces, por lo hasta ahora establecido, se tiene que

\begin{eqnarray*}
C_{i}\left(z\right)=I_{i}\left[\theta_{i}\left(z\right)\right],
\end{eqnarray*}
entonces

\begin{eqnarray*}
\esp\left[C_{i}\right]&=&\esp\left[I_{i}\right]\esp\left[\theta_{i}\left(z\right)\right]=\frac{\esp\left[L_{i}^{*}\right]}{\mu_{i}}\frac{1}{1-\mu_{i}}=\frac{f_{i}\left(i\right)}{\mu_{i}\left(1-\mu_{i}\right)}\\
Var\left[C_{i}\right]&=&\frac{Var\left[L_{i}^{*}\right]}{\mu_{i}^{2}\left(1-\mu_{i}\right)^{2}}.
\end{eqnarray*}

Por tanto se tienen las siguientes igualdades


\begin{eqnarray*}
\esp\left[S_{i}\right]&=&\mu_{i}\esp\left[C_{i}\right],\\
\esp\left[I_{i}\right]&=&\left(1-\mu_{i}\right)\esp\left[C_{i}\right]\\
\end{eqnarray*}

Def\'inanse los puntos de regenaraci\'on  en el proceso $\left[L_{1}\left(t\right),L_{2}\left(t\right),\ldots,L_{N}\left(t\right)\right]$. Los puntos cuando la cola $i$ es visitada y todos los $L_{j}\left(\tau_{i}\left(m\right)\right)=0$ para $i=1,2$  son puntos de regeneraci\'on. Se llama ciclo regenerativo al intervalo entre dos puntos regenerativos sucesivos.

Sea $M_{i}$  el n\'umero de ciclos de visita en un ciclo regenerativo, y sea $C_{i}^{(m)}$, para $m=1,2,\ldots,M_{i}$ la duraci\'on del $m$-\'esimo ciclo de visita en un ciclo regenerativo. Se define el ciclo del tiempo de visita promedio $\esp\left[C_{i}\right]$ como

\begin{eqnarray*}
\esp\left[C_{i}\right]&=&\frac{\esp\left[\sum_{m=1}^{M_{i}}C_{i}^{(m)}\right]}{\esp\left[M_{i}\right]}
\end{eqnarray*}


En Stid72 y Heym82 se muestra que una condici\'on suficiente para que el proceso regenerativo 
estacionario sea un procesoo estacionario es que el valor esperado del tiempo del ciclo regenerativo sea finito:

\begin{eqnarray*}
\esp\left[\sum_{m=1}^{M_{i}}C_{i}^{(m)}\right]<\infty.
\end{eqnarray*}

como cada $C_{i}^{(m)}$ contiene intervalos de r\'eplica positivos, se tiene que $\esp\left[M_{i}\right]<\infty$, adem\'as, como $M_{i}>0$, se tiene que la condici\'on anterior es equivalente a tener que 

\begin{eqnarray*}
\esp\left[C_{i}\right]<\infty,
\end{eqnarray*}
por lo tanto una condici\'on suficiente para la existencia del proceso regenerativo est\'a dada por

\begin{eqnarray*}
\sum_{k=1}^{N}\mu_{k}<1.
\end{eqnarray*}

Sea la funci\'on generadora de momentos para $L_{i}$, el n\'umero de usuarios en la cola $Q_{i}\left(z\right)$ en cualquier momento, est\'a dada por el tiempo promedio de $z^{L_{i}\left(t\right)}$ sobre el ciclo regenerativo definido anteriormente:

\begin{eqnarray*}
Q_{i}\left(z\right)&=&\esp\left[z^{L_{i}\left(t\right)}\right]=\frac{\esp\left[\sum_{m=1}^{M_{i}}\sum_{t=\tau_{i}\left(m\right)}^{\tau_{i}\left(m+1\right)-1}z^{L_{i}\left(t\right)}\right]}{\esp\left[\sum_{m=1}^{M_{i}}\tau_{i}\left(m+1\right)-\tau_{i}\left(m\right)\right]}
\end{eqnarray*}

$M_{i}$ es un tiempo de paro en el proceso regenerativo con $\esp\left[M_{i}\right]<\infty$, se sigue del lema de Wald que:


\begin{eqnarray*}
\esp\left[\sum_{m=1}^{M_{i}}\sum_{t=\tau_{i}\left(m\right)}^{\tau_{i}\left(m+1\right)-1}z^{L_{i}\left(t\right)}\right]&=&\esp\left[M_{i}\right]\esp\left[\sum_{t=\tau_{i}\left(m\right)}^{\tau_{i}\left(m+1\right)-1}z^{L_{i}\left(t\right)}\right]\\
\esp\left[\sum_{m=1}^{M_{i}}\tau_{i}\left(m+1\right)-\tau_{i}\left(m\right)\right]&=&\esp\left[M_{i}\right]\esp\left[\tau_{i}\left(m+1\right)-\tau_{i}\left(m\right)\right]
\end{eqnarray*}

por tanto se tiene que


\begin{eqnarray*}
Q_{i}\left(z\right)&=&\frac{\esp\left[\sum_{t=\tau_{i}\left(m\right)}^{\tau_{i}\left(m+1\right)-1}z^{L_{i}\left(t\right)}\right]}{\esp\left[\tau_{i}\left(m+1\right)-\tau_{i}\left(m\right)\right]}
\end{eqnarray*}

observar que el denominador es simplemente la duraci\'on promedio del tiempo del ciclo.


Se puede demostrar (ver Hideaki Takagi 1986) que

\begin{eqnarray*}
\esp\left[\sum_{t=\tau_{i}\left(m\right)}^{\tau_{i}\left(m+1\right)-1}z^{L_{i}\left(t\right)}\right]=z\frac{F_{i}\left(z\right)-1}{z-P_{i}\left(z\right)}
\end{eqnarray*}

Durante el tiempo de intervisita para la cola $i$, $L_{i}\left(t\right)$ solamente se incrementa de manera que el incremento por intervalo de tiempo est\'a dado por la funci\'on generadora de probabilidades de $P_{i}\left(z\right)$, por tanto la suma sobre el tiempo de intervisita puede evaluarse como:

\begin{eqnarray*}
\esp\left[\sum_{t=\tau_{i}\left(m\right)}^{\tau_{i}\left(m+1\right)-1}z^{L_{i}\left(t\right)}\right]&=&\esp\left[\sum_{t=\tau_{i}\left(m\right)}^{\tau_{i}\left(m+1\right)-1}\left\{P_{i}\left(z\right)\right\}^{t-\overline{\tau}_{i}\left(m\right)}\right]=\frac{1-\esp\left[\left\{P_{i}\left(z\right)\right\}^{\tau_{i}\left(m+1\right)-\overline{\tau}_{i}\left(m\right)}\right]}{1-P_{i}\left(z\right)}\\
&=&\frac{1-I_{i}\left[P_{i}\left(z\right)\right]}{1-P_{i}\left(z\right)}
\end{eqnarray*}
por tanto

\begin{eqnarray*}
\esp\left[\sum_{t=\tau_{i}\left(m\right)}^{\tau_{i}\left(m+1\right)-1}z^{L_{i}\left(t\right)}\right]&=&\frac{1-F_{i}\left(z\right)}{1-P_{i}\left(z\right)}
\end{eqnarray*}

Haciendo uso de lo hasta ahora desarrollado se tiene que

\begin{eqnarray*}
Q_{i}\left(z\right)&=&\frac{1}{\esp\left[C_{i}\right]}\cdot\frac{1-F_{i}\left(z\right)}{P_{i}\left(z\right)-z}\cdot\frac{\left(1-z\right)P_{i}\left(z\right)}{1-P_{i}\left(z\right)}\\
&=&\frac{\mu_{i}\left(1-\mu_{i}\right)}{f_{i}\left(i\right)}\cdot\frac{1-F_{i}\left(z\right)}{P_{i}\left(z\right)-z}\cdot\frac{\left(1-z\right)P_{i}\left(z\right)}{1-P_{i}\left(z\right)}
\end{eqnarray*}

derivando con respecto a $z$



\begin{eqnarray*}
\frac{d Q_{i}\left(z\right)}{d z}&=&\frac{\left(1-F_{i}\left(z\right)\right)P_{i}\left(z\right)}{\esp\left[C_{i}\right]\left(1-P_{i}\left(z\right)\right)\left(P_{i}\left(z\right)-z\right)}\\
&-&\frac{\left(1-z\right)P_{i}\left(z\right)F_{i}^{'}\left(z\right)}{\esp\left[C_{i}\right]\left(1-P_{i}\left(z\right)\right)\left(P_{i}\left(z\right)-z\right)}\\
&-&\frac{\left(1-z\right)\left(1-F_{i}\left(z\right)\right)P_{i}\left(z\right)\left(P_{i}^{'}\left(z\right)-1\right)}{\esp\left[C_{i}\right]\left(1-P_{i}\left(z\right)\right)\left(P_{i}\left(z\right)-z\right)^{2}}\\
&+&\frac{\left(1-z\right)\left(1-F_{i}\left(z\right)\right)P_{i}^{'}\left(z\right)}{\esp\left[C_{i}\right]\left(1-P_{i}\left(z\right)\right)\left(P_{i}\left(z\right)-z\right)}\\
&+&\frac{\left(1-z\right)\left(1-F_{i}\left(z\right)\right)P_{i}\left(z\right)P_{i}^{'}\left(z\right)}{\esp\left[C_{i}\right]\left(1-P_{i}\left(z\right)\right)^{2}\left(P_{i}\left(z\right)-z\right)}
\end{eqnarray*}

Calculando el l\'imite cuando $z\rightarrow1^{+}$:
\begin{eqnarray}
Q_{i}^{(1)}\left(z\right)=\lim_{z\rightarrow1^{+}}\frac{d Q_{i}\left(z\right)}{dz}&=&\lim_{z\rightarrow1}\frac{\left(1-F_{i}\left(z\right)\right)P_{i}\left(z\right)}{\esp\left[C_{i}\right]\left(1-P_{i}\left(z\right)\right)\left(P_{i}\left(z\right)-z\right)}\\
&-&\lim_{z\rightarrow1^{+}}\frac{\left(1-z\right)P_{i}\left(z\right)F_{i}^{'}\left(z\right)}{\esp\left[C_{i}\right]\left(1-P_{i}\left(z\right)\right)\left(P_{i}\left(z\right)-z\right)}\\
&-&\lim_{z\rightarrow1^{+}}\frac{\left(1-z\right)\left(1-F_{i}\left(z\right)\right)P_{i}\left(z\right)\left(P_{i}^{'}\left(z\right)-1\right)}{\esp\left[C_{i}\right]\left(1-P_{i}\left(z\right)\right)\left(P_{i}\left(z\right)-z\right)^{2}}\\
&+&\lim_{z\rightarrow1^{+}}\frac{\left(1-z\right)\left(1-F_{i}\left(z\right)\right)P_{i}^{'}\left(z\right)}{\esp\left[C_{i}\right]\left(1-P_{i}\left(z\right)\right)\left(P_{i}\left(z\right)-z\right)}\\
&+&\lim_{z\rightarrow1^{+}}\frac{\left(1-z\right)\left(1-F_{i}\left(z\right)\right)P_{i}\left(z\right)P_{i}^{'}\left(z\right)}{\esp\left[C_{i}\right]\left(1-P_{i}\left(z\right)\right)^{2}\left(P_{i}\left(z\right)-z\right)}
\end{eqnarray}

Entonces:
%______________________________________________________

\begin{eqnarray*}
\lim_{z\rightarrow1^{+}}\frac{\left(1-F_{i}\left(z\right)\right)P_{i}\left(z\right)}{\left(1-P_{i}\left(z\right)\right)\left(P_{i}\left(z\right)-z\right)}&=&\lim_{z\rightarrow1^{+}}\frac{\frac{d}{dz}\left[\left(1-F_{i}\left(z\right)\right)P_{i}\left(z\right)\right]}{\frac{d}{dz}\left[\left(1-P_{i}\left(z\right)\right)\left(-z+P_{i}\left(z\right)\right)\right]}\\
&=&\lim_{z\rightarrow1^{+}}\frac{-P_{i}\left(z\right)F_{i}^{'}\left(z\right)+\left(1-F_{i}\left(z\right)\right)P_{i}^{'}\left(z\right)}{\left(1-P_{i}\left(z\right)\right)\left(-1+P_{i}^{'}\left(z\right)\right)-\left(-z+P_{i}\left(z\right)\right)P_{i}^{'}\left(z\right)}
\end{eqnarray*}


%______________________________________________________


\begin{eqnarray*}
\lim_{z\rightarrow1^{+}}\frac{\left(1-z\right)P_{i}\left(z\right)F_{i}^{'}\left(z\right)}{\left(1-P_{i}\left(z\right)\right)\left(P_{i}\left(z\right)-z\right)}&=&\lim_{z\rightarrow1^{+}}\frac{\frac{d}{dz}\left[\left(1-z\right)P_{i}\left(z\right)F_{i}^{'}\left(z\right)\right]}{\frac{d}{dz}\left[\left(1-P_{i}\left(z\right)\right)\left(P_{i}\left(z\right)-z\right)\right]}\\
&=&\lim_{z\rightarrow1^{+}}\frac{-P_{i}\left(z\right) F_{i}^{'}\left(z\right)+(1-z) F_{i}^{'}\left(z\right) P_{i}^{'}\left(z\right)+(1-z) P_{i}\left(z\right)F_{i}^{''}\left(z\right)}{\left(1-P_{i}\left(z\right)\right)\left(-1+P_{i}^{'}\left(z\right)\right)-\left(-z+P_{i}\left(z\right)\right)P_{i}^{'}\left(z\right)}
\end{eqnarray*}


%______________________________________________________

\begin{eqnarray*}
&&\lim_{z\rightarrow1^{+}}\frac{\left(1-z\right)\left(1-F_{i}\left(z\right)\right)P_{i}\left(z\right)\left(P_{i}^{'}\left(z\right)-1\right)}{\left(1-P_{i}\left(z\right)\right)\left(P_{i}\left(z\right)-z\right)^{2}}=\lim_{z\rightarrow1^{+}}\frac{\frac{d}{dz}\left[\left(1-z\right)\left(1-F_{i}\left(z\right)\right)P_{i}\left(z\right)\left(P_{i}^{'}\left(z\right)-1\right)\right]}{\frac{d}{dz}\left[\left(1-P_{i}\left(z\right)\right)\left(P_{i}\left(z\right)-z\right)^{2}\right]}\\
&=&\lim_{z\rightarrow1^{+}}\frac{-\left(1-F_{i}\left(z\right)\right) P_{i}\left(z\right)\left(-1+P_{i}^{'}\left(z\right)\right)-(1-z) P_{i}\left(z\right)F_{i}^{'}\left(z\right)\left(-1+P_{i}^{'}\left(z\right)\right)}{2\left(1-P_{i}\left(z\right)\right)\left(-z+P_{i}\left(z\right)\right) \left(-1+P_{i}^{'}\left(z\right)\right)-\left(-z+P_{i}\left(z\right)\right)^2 P_{i}^{'}\left(z\right)}\\
&+&\lim_{z\rightarrow1^{+}}\frac{+(1-z) \left(1-F_{i}\left(z\right)\right) \left(-1+P_{i}^{'}\left(z\right)\right) P_{i}^{'}\left(z\right)}{{2\left(1-P_{i}\left(z\right)\right)\left(-z+P_{i}\left(z\right)\right) \left(-1+P_{i}^{'}\left(z\right)\right)-\left(-z+P_{i}\left(z\right)\right)^2 P_{i}^{'}\left(z\right)}}\\
&+&\lim_{z\rightarrow1^{+}}\frac{+(1-z) \left(1-F_{i}\left(z\right)\right) P_{i}\left(z\right)P_{i}^{''}\left(z\right)}{{2\left(1-P_{i}\left(z\right)\right)\left(-z+P_{i}\left(z\right)\right) \left(-1+P_{i}^{'}\left(z\right)\right)-\left(-z+P_{i}\left(z\right)\right)^2 P_{i}^{'}\left(z\right)}}
\end{eqnarray*}











%______________________________________________________
\begin{eqnarray*}
&&\lim_{z\rightarrow1^{+}}\frac{\left(1-z\right)\left(1-F_{i}\left(z\right)\right)P_{i}^{'}\left(z\right)}{\left(1-P_{i}\left(z\right)\right)\left(P_{i}\left(z\right)-z\right)}=\lim_{z\rightarrow1^{+}}\frac{\frac{d}{dz}\left[\left(1-z\right)\left(1-F_{i}\left(z\right)\right)P_{i}^{'}\left(z\right)\right]}{\frac{d}{dz}\left[\left(1-P_{i}\left(z\right)\right)\left(P_{i}\left(z\right)-z\right)\right]}\\
&=&\lim_{z\rightarrow1^{+}}\frac{-\left(1-F_{i}\left(z\right)\right) P_{i}^{'}\left(z\right)-(1-z) F_{i}^{'}\left(z\right) P_{i}^{'}\left(z\right)+(1-z) \left(1-F_{i}\left(z\right)\right) P_{i}^{''}\left(z\right)}{\left(1-P_{i}\left(z\right)\right) \left(-1+P_{i}^{'}\left(z\right)\right)-\left(-z+P_{i}\left(z\right)\right) P_{i}^{'}\left(z\right)}\frac{}{}
\end{eqnarray*}

%______________________________________________________
\begin{eqnarray*}
&&\lim_{z\rightarrow1^{+}}\frac{\left(1-z\right)\left(1-F_{i}\left(z\right)\right)P_{i}\left(z\right)P_{i}^{'}\left(z\right)}{\left(1-P_{i}\left(z\right)\right)^{2}\left(P_{i}\left(z\right)-z\right)}=\lim_{z\rightarrow1^{+}}\frac{\frac{d}{dz}\left[\left(1-z\right)\left(1-F_{i}\left(z\right)\right)P_{i}\left(z\right)P_{i}^{'}\left(z\right)\right]}{\frac{d}{dz}\left[\left(1-P_{i}\left(z\right)\right)^{2}\left(P_{i}\left(z\right)-z\right)\right]}\\
&=&\lim_{z\rightarrow1^{+}}\frac{-\left(1-F_{i}\left(z\right)\right) P_{i}\left(z\right) P_{i}^{'}\left(z\right)-(1-z) P_{i}\left(z\right) F_{i}^{'}\left(z\right)P_i'[z]}{\left(1-P_{i}\left(z\right)\right)^2 \left(-1+P_{i}^{'}\left(z\right)\right)-2 \left(1-P_{i}\left(z\right)\right) \left(-z+P_{i}\left(z\right)\right) P_{i}^{'}\left(z\right)}\\
&+&\lim_{z\rightarrow1^{+}}\frac{(1-z) \left(1-F_{i}\left(z\right)\right) P_{i}^{'}\left(z\right)^2+(1-z) \left(1-F_{i}\left(z\right)\right) P_{i}\left(z\right) P_{i}^{''}\left(z\right)}{\left(1-P_{i}\left(z\right)\right)^2 \left(-1+P_{i}^{'}\left(z\right)\right)-2 \left(1-P_{i}\left(z\right)\right) \left(-z+P_{i}\left(z\right)\right) P_{i}^{'}\left(z\right)}\\
\end{eqnarray*}

%___________________________________________________________________________________________
\subsection{Longitudes de la Cola en cualquier tiempo}
%___________________________________________________________________________________________

Sea
$V_{i}\left(z\right)=\frac{1}{\esp\left[C_{i}\right]}\frac{I_{i}\left(z\right)-1}{z-P_{i}\left(z\right)}$

%{\esp\lef[I_{i}\right]}\frac{1-\mu_{i}}{z-P_{i}\left(z\right)}

\begin{eqnarray*}
\frac{\partial V_{i}\left(z\right)}{\partial z}&=&\frac{1}{\esp\left[C_{i}\right]}\left[\frac{I_{i}{'}\left(z\right)\left(z-P_{i}\left(z\right)\right)}{z-P_{i}\left(z\right)}-\frac{\left(I_{i}\left(z\right)-1\right)\left(1-P_{i}{'}\left(z\right)\right)}{\left(z-P_{i}\left(z\right)\right)^{2}}\right]
\end{eqnarray*}


La FGP para el tiempo de espera para cualquier usuario en la cola est\'a dada por:
\[U_{i}\left(z\right)=\frac{1}{\esp\left[C_{i}\right]}\cdot\frac{1-P_{i}\left(z\right)}{z-P_{i}\left(z\right)}\cdot\frac{I_{i}\left(z\right)-1}{1-z}\]

entonces


\begin{eqnarray*}
\frac{d}{dz}V_{i}\left(z\right)&=&\frac{1}{\esp\left[C_{i}\right]}\left\{\frac{d}{dz}\left(\frac{1-P_{i}\left(z\right)}{z-P_{i}\left(z\right)}\right)\frac{I_{i}\left(z\right)-1}{1-z}+\frac{1-P_{i}\left(z\right)}{z-P_{i}\left(z\right)}\frac{d}{dz}\left(\frac{I_{i}\left(z\right)-1}{1-z}\right)\right\}\\
&=&\frac{1}{\esp\left[C_{i}\right]}\left\{\frac{-P_{i}\left(z\right)\left(z-P_{i}\left(z\right)\right)-\left(1-P_{i}\left(z\right)\right)\left(1-P_{i}^{'}\left(z\right)\right)}{\left(z-P_{i}\left(z\right)\right)^{2}}\cdot\frac{I_{i}\left(z\right)-1}{1-z}\right\}\\
&+&\frac{1}{\esp\left[C_{i}\right]}\left\{\frac{1-P_{i}\left(z\right)}{z-P_{i}\left(z\right)}\cdot\frac{I_{i}^{'}\left(z\right)\left(1-z\right)+\left(I_{i}\left(z\right)-1\right)}{\left(1-z\right)^{2}}\right\}
\end{eqnarray*}
%\frac{I_{i}\left(z\right)-1}{1-z}
%+\frac{1-P_{i}\left(z\right)}{z-P_{i}\frac{d}{dz}\left(\frac{I_{i}\left(z\right)-1}{1-z}\right)


\begin{eqnarray*}
\frac{\partial U_{i}\left(z\right)}{\partial z}&=&\frac{(-1+I_{i}[z]) (1-P_{i}[z])}{(1-z)^2 \esp[I_{i}] (z-P_{i}[z])}+\frac{(1-P_{i}[z]) I_{i}^{'}[z]}{(1-z) \esp[I_{i}] (z-P_{i}[z])}-\frac{(-1+I_{i}[z]) (1-P_{i}[z])\left(1-P{'}[z]\right)}{(1-z) \esp[I_{i}] (z-P_{i}[z])^2}\\
&-&\frac{(-1+I_{i}[z]) P_{i}{'}[z]}{(1-z) \esp[I_{i}](z-P_{i}[z])}
\end{eqnarray*}


\subsection{Material por agregar}


\begin{Teo}
Dada una Red de Sistemas de Visitas C\'iclicas (RSVC), conformada por dos Sistemas de Visitas C\'iclicas (SVC), donde cada uno de ellos consta de dos colas tipo $M/M/1$. Los dos sistemas est\'an comunicados entre s\'i por medio de la transferencia de usuarios entre las colas $Q_{1}\leftrightarrow Q_{3}$ y $Q_{2}\leftrightarrow Q_{4}$. Se definen los eventos para los procesos de arribos al tiempo $t$, $A_{j}\left(t\right)=\left\{0 \textrm{ arribos en }Q_{j}\textrm{ al tiempo }t\right\}$ para alg\'un tiempo $t\geq0$ y $Q_{j}$ la cola $j$-\'esima en la RSVC, para $j=1,2,3,4$.  Existe un intervalo $I\neq\emptyset$ tal que para $T^{*}\in I$, tal que $\prob\left\{A_{1}\left(T^{*}\right),A_{2}\left(Tt^{*}\right),
A_{3}\left(T^{*}\right),A_{4}\left(T^{*}\right)|T^{*}\in I\right\}>0$.
\end{Teo}



\begin{proof}
Sin p\'erdida de generalidad podemos considerar como base del an\'alisis a la cola $Q_{1}$ del primer sistema que conforma la RSVC.\medskip 

Sea $n\geq1$, ciclo en el primer sistema en el que se sabe que $L_{j}\left(\overline{\tau}_{1}\left(n\right)\right)=0$, pues la pol\'itica de servicio con que atienden los servidores es la exhaustiva. Como es sabido, para trasladarse a la siguiente cola, el servidor incurre en un tiempo de traslado $r_{1}\left(n\right)>0$, entonces tenemos que $\tau_{2}\left(n\right)=\overline{\tau}_{1}\left(n\right)+r_{1}\left(n\right)$.\medskip 


Definamos el intervalo $I_{1}\equiv\left[\overline{\tau}_{1}\left(n\right),\tau_{2}\left(n\right)\right]$ de longitud $\xi_{1}=r_{1}\left(n\right)$.

Dado que los tiempos entre arribo son exponenciales con tasa $\tilde{\mu}_{1}=\mu_{1}+\hat{\mu}_{1}$ ($\mu_{1}$ son los arribos a $Q_{1}$ por primera vez al sistema, mientras que $\hat{\mu}_{1}$ son los arribos de traslado procedentes de $Q_{3}$) se tiene que la probabilidad del evento $A_{1}\left(t\right)$ est\'a dada por 

\begin{equation}
\prob\left\{A_{1}\left(t\right)|t\in I_{1}\left(n\right)\right\}=e^{-\tilde{\mu}_{1}\xi_{1}\left(n\right)}.
\end{equation} 


Por otra parte, para la cola $Q_{2}$ el tiempo $\overline{\tau}_{2}\left(n-1\right)$ es tal que $L_{2}\left(\overline{\tau}_{2}\left(n-1\right)\right)=0$, es decir, es el tiempo en que la cola queda totalmente vac\'ia en el ciclo anterior a $n$. \medskip 


Entonces tenemos un sgundo intervalo $I_{2}\equiv\left[\overline{\tau}_{2}\left(n-1\right),\tau_{2}\left(n\right)\right]$. Por lo tanto la probabilidad del evento $A_{2}\left(t\right)$ tiene probabilidad dada por

\begin{eqnarray}
\prob\left\{A_{2}\left(t\right)|t\in I_{2}\left(n\right)\right\}=e^{-\tilde{\mu}_{2}\xi_{2}\left(n\right)},\\
\xi_{2}\left(n\right)=\tau_{2}\left(n\right)-\overline{\tau}_{2}\left(n-1\right)
\end{eqnarray}
%\end{equation} 

%donde $$.

Ahora, dado que $I_{1}\left(n\right)\subset I_{2}\left(n\right)$, se tiene que

\begin{eqnarray*}
\xi_{1}\left(n\right)\leq\xi_{2}\left(n\right)&\Leftrightarrow& -\xi_{1}\left(n\right)\geq-\xi_{2}\left(n\right)
\\
-\tilde{\mu}_{2}\xi_{1}\left(n\right)\geq-\tilde{\mu}_{2}\xi_{2}\left(n\right)&\Leftrightarrow&
e^{-\tilde{\mu}_{2}\xi_{1}\left(n\right)}\geq e^{-\tilde{\mu}_{2}\xi_{2}\left(n\right)}\\
\prob\left\{A_{2}\left(t\right)|t\in I_{1}\left(n\right)\right\}&\geq&
\prob\left\{A_{2}\left(t\right)|t\in I_{2}\left(n\right)\right\}.
\end{eqnarray*}


Entonces se tiene que
\small{
\begin{eqnarray*}
\prob\left\{A_{1}\left(t\right),A_{2}\left(t\right)|t\in I_{1}\left(n\right)\right\}&=&
\prob\left\{A_{1}\left(t\right)|t\in I_{1}\left(n\right)\right\}
\prob\left\{A_{2}\left(t\right)|t\in I_{1}\left(n\right)\right\}\\
&\geq&
\prob\left\{A_{1}\left(t\right)|t\in I_{1}\left(n\right)\right\}
\prob\left\{A_{2}\left(t\right)|t\in I_{2}\left(n\right)\right\}\\
&=&e^{-\tilde{\mu}_{1}\xi_{1}\left(n\right)}e^{-\tilde{\mu}_{2}\xi_{2}\left(n\right)}
=e^{-\left[\tilde{\mu}_{1}\xi_{1}\left(n\right)+\tilde{\mu}_{2}\xi_{2}\left(n\right)\right]}.
\end{eqnarray*}}


Es decir, 

\begin{equation}
\prob\left\{A_{1}\left(t\right),A_{2}\left(t\right)|t\in I_{1}\left(n\right)\right\}
=e^{-\left[\tilde{\mu}_{1}\xi_{1}\left(n\right)+\tilde{\mu}_{2}\xi_{2}
\left(n\right)\right]}>0.
\end{equation}
En lo que respecta a la relaci\'on entre los dos SVC que conforman la RSVC para alg\'un $m\geq1$ se tiene que $\tau_{3}\left(m\right)<\tau_{2}\left(n\right)<\tau_{4}\left(m\right)$ por lo tanto se cumple cualquiera de los siguientes cuatro casos
\begin{itemize}
\item[a)] $\tau_{3}\left(m\right)<\tau_{2}\left(n\right)<\overline{\tau}_{3}\left(m\right)$

\item[b)] $\overline{\tau}_{3}\left(m\right)<\tau_{2}\left(n\right)
<\tau_{4}\left(m\right)$

\item[c)] $\tau_{4}\left(m\right)<\tau_{2}\left(n\right)<
\overline{\tau}_{4}\left(m\right)$

\item[d)] $\overline{\tau}_{4}\left(m\right)<\tau_{2}\left(n\right)
<\tau_{3}\left(m+1\right)$
\end{itemize}


Sea el intervalo $I_{3}\left(m\right)\equiv\left[\tau_{3}\left(m\right),\overline{\tau}_{3}\left(m\right)\right]$ tal que $\tau_{2}\left(n\right)\in I_{3}\left(m\right)$, con longitud de intervalo $\xi_{3}\equiv\overline{\tau}_{3}\left(m\right)-\tau_{3}\left(m\right)$, entonces se tiene que para $Q_{3}$
\begin{equation}
\prob\left\{A_{3}\left(t\right)|t\in I_{3}\left(m\right)\right\}=e^{-\tilde{\mu}_{3}\xi_{3}\left(m\right)}.
\end{equation} 

mientras que para $Q_{4}$ consideremos el intervalo $I_{4}\left(m\right)\equiv\left[\tau_{4}\left(m-1\right),\overline{\tau}_{3}\left(m\right)\right]$, entonces por construcci\'on  $I_{3}\left(m\right)\subset I_{4}\left(m\right)$, por lo tanto


\begin{eqnarray*}
\xi_{3}\left(m\right)\leq\xi_{4}\left(m\right)&\Leftrightarrow& -\xi_{3}\left(m\right)\geq-\xi_{4}\left(m\right)
\\
-\tilde{\mu}_{4}\xi_{3}\left(m\right)\geq-\tilde{\mu}_{4}\xi_{4}\left(m\right)&\Leftrightarrow&
e^{-\tilde{\mu}_{4}\xi_{3}\left(m\right)}\geq e^{-\tilde{\mu}_{4}\xi_{4}\left(n\right)}\\
\prob\left\{A_{4}\left(t\right)|t\in I_{3}\left(m\right)\right\}&\geq&
\prob\left\{A_{4}\left(t\right)|t\in I_{4}\left(m\right)\right\}.
\end{eqnarray*}



Entonces se tiene que
\small{
\begin{eqnarray*}
\prob\left\{A_{3}\left(t\right),A_{4}\left(t\right)|t\in I_{3}\left(m\right)\right\}&=&
\prob\left\{A_{3}\left(t\right)|t\in I_{3}\left(m\right)\right\}
\prob\left\{A_{4}\left(t\right)|t\in I_{3}\left(m\right)\right\}\\
&\geq&
\prob\left\{A_{3}\left(t\right)|t\in I_{3}\left(m\right)\right\}
\prob\left\{A_{4}\left(t\right)|t\in I_{4}\left(m\right)\right\}\\
&=&e^{-\tilde{\mu}_{3}\xi_{3}\left(m\right)}e^{-\tilde{\mu}_{4}\xi_{4}
\left(m\right)}
=e^{-\left(\tilde{\mu}_{3}\xi_{3}\left(m\right)+\tilde{\mu}_{4}\xi_{4}\left(m\right)\right)}.
\end{eqnarray*}}

Es decir, 

\begin{equation}
\prob\left\{A_{3}\left(t\right),A_{4}\left(t\right)|t\in I_{3}\left(m\right)\right\}\geq
e^{-\left(\tilde{\mu}_{3}\xi_{3}\left(m\right)+\tilde{\mu}_{4}\xi_{4}\left(m\right)\right)}>0.
\end{equation}


Sea el intervalo $I_{3}\left(m\right)\equiv\left[\overline{\tau}_{3}\left(m\right),\tau_{4}\left(m\right)\right]$ con longitud $\xi_{3}\equiv\tau_{4}\left(m\right)-\overline{\tau}_{3}\left(m\right)$, entonces se tiene que para $Q_{3}$
\begin{equation}
\prob\left\{A_{3}\left(t\right)|t\in I_{3}\left(m\right)\right\}=e^{-\tilde{\mu}_{3}\xi_{3}\left(m\right)}.
\end{equation} 

mientras que para $Q_{4}$ consideremos el intervalo $I_{4}\left(m\right)\equiv\left[\overline{\tau}_{4}\left(m-1\right),\tau_{4}\left(m\right)\right]$, entonces por construcci\'on  $I_{3}\left(m\right)\subset I_{4}\left(m\right)$, y al igual que en el caso anterior se tiene que 

\begin{eqnarray*}
\xi_{3}\left(m\right)\leq\xi_{4}\left(m\right)&\Leftrightarrow& -\xi_{3}\left(m\right)\geq-\xi_{4}\left(m\right)
\\
-\tilde{\mu}_{4}\xi_{3}\left(m\right)\geq-\tilde{\mu}_{4}\xi_{4}\left(m\right)&\Leftrightarrow&
e^{-\tilde{\mu}_{4}\xi_{3}\left(m\right)}\geq e^{-\tilde{\mu}_{4}\xi_{4}\left(n\right)}\\
\prob\left\{A_{4}\left(t\right)|t\in I_{3}\left(m\right)\right\}&\geq&
\prob\left\{A_{4}\left(t\right)|t\in I_{4}\left(m\right)\right\}.
\end{eqnarray*}


Entonces se tiene que
\small{
\begin{eqnarray*}
\prob\left\{A_{3}\left(t\right),A_{4}\left(t\right)|t\in I_{3}\left(m\right)\right\}&=&
\prob\left\{A_{3}\left(t\right)|t\in I_{3}\left(m\right)\right\}
\prob\left\{A_{4}\left(t\right)|t\in I_{3}\left(m\right)\right\}\\
&\geq&
\prob\left\{A_{3}\left(t\right)|t\in I_{3}\left(m\right)\right\}
\prob\left\{A_{4}\left(t\right)|t\in I_{4}\left(m\right)\right\}\\
&=&e^{-\tilde{\mu}_{3}\xi_{3}\left(m\right)}e^{-\tilde{\mu}_{4}\xi_{4}\left(m\right)}
=e^{-\left(\tilde{\mu}_{3}\xi_{3}\left(m\right)+\tilde{\mu}_{4}\xi_{4}\left(m\right)\right)}.
\end{eqnarray*}}

Es decir, 

\begin{equation}
\prob\left\{A_{3}\left(t\right),A_{4}\left(t\right)|t\in I_{4}\left(m\right)\right\}\geq
e^{-\left(\tilde{\mu}_{3}+\tilde{\mu}_{4}\right)\xi_{3}\left(m\right)}>0.
\end{equation}


Para el intervalo $I_{3}\left(m\right)=\left[\tau_{4}\left(m\right),\overline{\tau}_{4}\left(m\right)\right]$, se tiene que este caso es an\'alogo al caso (a).


Para el intevalo $I_{3}\left(m\right)\equiv\left[\overline{\tau}_{4}\left(m\right),\tau_{4}\left(m+1\right)\right]$, se tiene que es an\'alogo al caso (b).


Por construcci\'on se tiene que $I\left(n,m\right)\equiv I_{1}\left(n\right)\cap I_{3}\left(m\right)\neq\emptyset$,entonces en particular se tienen las contenciones $I\left(n,m\right)\subseteq I_{1}\left(n\right)$ y $I\left(n,m\right)\subseteq I_{3}\left(m\right)$, por lo tanto si definimos $\xi_{n,m}\equiv\ell\left(I\left(n,m\right)\right)$ tenemos que

\begin{eqnarray*}
\xi_{n,m}\leq\xi_{1}\left(n\right)\textrm{ y }\xi_{n,m}\leq\xi_{3}\left(m\right)\textrm{ entonces }\\
-\xi_{n,m}\geq-\xi_{1}\left(n\right)\textrm{ y }-\xi_{n,m}\leq-\xi_{3}\left(m\right)\\
\end{eqnarray*}
por lo tanto tenemos las desigualdades 


\begin{eqnarray*}
\begin{array}{ll}
-\tilde{\mu}_{1}\xi_{n,m}\geq-\tilde{\mu}_{1}\xi_{1}\left(n\right),&
-\tilde{\mu}_{2}\xi_{n,m}\geq-\tilde{\mu}_{2}\xi_{1}\left(n\right)
\geq-\tilde{\mu}_{2}\xi_{2}\left(n\right),\\
-\tilde{\mu}_{3}\xi_{n,m}\geq-\tilde{\mu}_{3}\xi_{3}\left(m\right),&
-\tilde{\mu}_{4}\xi_{n,m}\geq-\tilde{\mu}_{4}\xi_{3}\left(m\right)
\geq-\tilde{\mu}_{4}\xi_{4}\left(m\right).
\end{array}
\end{eqnarray*}

Sea $T^{*}\in I\left(n,m\right)$, entonces dado que en particular $T^{*}\in I_{1}\left(n\right)$, se cumple con probabilidad positiva que no hay arribos a las colas $Q_{1}$ y $Q_{2}$, en consecuencia, tampoco hay usuarios de transferencia para $Q_{3}$ y $Q_{4}$, es decir, $\tilde{\mu}_{1}=\mu_{1}$, $\tilde{\mu}_{2}=\mu_{2}$, $\tilde{\mu}_{3}=\mu_{3}$, $\tilde{\mu}_{4}=\mu_{4}$, es decir, los eventos $Q_{1}$ y $Q_{3}$ son condicionalmente independientes en el intervalo $I\left(n,m\right)$; lo mismo ocurre para las colas $Q_{2}$ y $Q_{4}$, por lo tanto tenemos que
%\small{
\begin{eqnarray}
\begin{array}{l}
\prob\left\{A_{1}\left(T^{*}\right),A_{2}\left(T^{*}\right),
A_{3}\left(T^{*}\right),A_{4}\left(T^{*}\right)|T^{*}\in I\left(n,m\right)\right\}\\
=\prod_{j=1}^{4}\prob\left\{A_{j}\left(T^{*}\right)|T^{*}\in I\left(n,m\right)\right\}\\
\geq\prob\left\{A_{1}\left(T^{*}\right)|T^{*}\in I_{1}\left(n\right)\right\}
\prob\left\{A_{2}\left(T^{*}\right)|T^{*}\in I_{2}\left(n\right)\right\}\\
\prob\left\{A_{3}\left(T^{*}\right)|T^{*}\in I_{3}\left(m\right)\right\}
\prob\left\{A_{4}\left(T^{*}\right)|T^{*}\in I_{4}\left(m\right)\right\}\\
=e^{-\mu_{1}\xi_{1}\left(n\right)}
e^{-\mu_{2}\xi_{2}\left(n\right)}
e^{-\mu_{3}\xi_{3}\left(m\right)}
e^{-\mu_{4}\xi_{4}\left(m\right)}\\
=e^{-\left[\tilde{\mu}_{1}\xi_{1}\left(n\right)
+\tilde{\mu}_{2}\xi_{2}\left(n\right)
+\tilde{\mu}_{3}\xi_{3}\left(m\right)
+\tilde{\mu}_{4}\xi_{4}
\left(m\right)\right]}>0.
\end{array}
\end{eqnarray}


Ahora solo resta demostrar que para $n\ge1$, existe $m\geq1$ tal que se cumplen cualquiera de los cuatro casos arriba mencionados: 

\begin{itemize}
\item[a)] $\tau_{3}\left(m\right)<\tau_{2}\left(n\right)<\overline{\tau}_{3}\left(m\right)$

\item[b)] $\overline{\tau}_{3}\left(m\right)<\tau_{2}\left(n\right)
<\tau_{4}\left(m\right)$

\item[c)] $\tau_{4}\left(m\right)<\tau_{2}\left(n\right)<
\overline{\tau}_{4}\left(m\right)$

\item[d)] $\overline{\tau}_{4}\left(m\right)<\tau_{2}\left(n\right)
<\tau_{3}\left(m+1\right)$
\end{itemize}

Consideremos nuevamente el primer caso. Supongamos que no existe $m\geq1$, tal que $I_{1}\left(n\right)\cap I_{3}\left(m\right)\neq\emptyset$, es decir, para toda $m\geq1$, $I_{1}\left(n\right)\cap I_{3}\left(m\right)=\emptyset$, entonces se tiene que ocurren cualquiera de los dos casos

\begin{itemize}
\item[a)] $\tau_{2}\left(n\right)\leq\tau_{3}\left(m\right)$: Recordemos que $\tau_{2}\left(m\right)=\overline{\tau}_{1}+r_{1}\left(m\right)$ donde cada una de las variables aleatorias son tales que $\esp\left[\overline{\tau}_{1}\left(n\right)-\tau_{1}\left(n\right)\right]<\infty$, $\esp\left[R_{1}\right]<\infty$ y $\esp\left[\tau_{3}\left(m\right)\right]<\infty$, lo cual contradice el hecho de que no exista un ciclo $m\geq1$ que satisfaga la condici\'on deseada.

\item[b)] $\tau_{2}\left(n\right)\geq\overline{\tau}_{3}\left(m\right)$: por un argumento similar al anterior se tiene que no es posible que no exista un ciclo $m\geq1$ tal que satisaface la condici\'on deseada.

\end{itemize}

Para el resto de los casos la demostraci\'on es an\'aloga. Por lo tanto, se tiene que efectivamente existe $m\geq1$ tal que $\tau_{3}\left(m\right)<\tau_{2}\left(n\right)<\tau_{4}\left(m\right)$.
\end{proof}
\newpage

En Sigman, Thorison y Wolff \cite{Sigman2} prueban que para la existencia de un una sucesi\'on infinita no decreciente de tiempos de regeneraci\'on $\tau_{1}\leq\tau_{2}\leq\cdots$ en los cuales el proceso se regenera, basta un tiempo de regeneraci\'on $R_{1}$, donde $R_{j}=\tau_{j}-\tau_{j-1}$. Para tal efecto se requiere la existencia de un espacio de probabilidad $\left(\Omega,\mathcal{F},\prob\right)$, y proceso estoc\'astico $\textit{X}=\left\{X\left(t\right):t\geq0\right\}$ con espacio de estados $\left(S,\mathcal{R}\right)$, con $\mathcal{R}$ $\sigma$-\'algebra.

\begin{Prop}
Si existe una variable aleatoria no negativa $R_{1}$ tal que $\theta_{R1}X=_{D}X$, entonces $\left(\Omega,\mathcal{F},\prob\right)$ puede extenderse para soportar una sucesi\'on estacionaria de variables aleatorias $R=\left\{R_{k}:k\geq1\right\}$, tal que para $k\geq1$,
\begin{eqnarray*}
\theta_{k}\left(X,R\right)=_{D}\left(X,R\right).
\end{eqnarray*}

Adem\'as, para $k\geq1$, $\theta_{k}R$ es condicionalmente independiente de $\left(X,R_{1},\ldots,R_{k}\right)$, dado $\theta_{\tau k}X$.

\end{Prop}


\begin{itemize}
\item Doob en 1953 demostr\'o que el estado estacionario de un proceso de partida en un sistema de espera $M/G/\infty$, es Poisson con la misma tasa que el proceso de arribos.

\item Burke en 1968, fue el primero en demostrar que el estado estacionario de un proceso de salida de una cola $M/M/s$ es un proceso Poisson.

\item Disney en 1973 obtuvo el siguiente resultado:

\begin{Teo}
Para el sistema de espera $M/G/1/L$ con disciplina FIFO, el proceso $\textbf{I}$ es un proceso de renovaci\'on si y s\'olo si el proceso denominado longitud de la cola es estacionario y se cumple cualquiera de los siguientes casos:

\begin{itemize}
\item[a)] Los tiempos de servicio son identicamente cero;
\item[b)] $L=0$, para cualquier proceso de servicio $S$;
\item[c)] $L=1$ y $G=D$;
\item[d)] $L=\infty$ y $G=M$.
\end{itemize}
En estos casos, respectivamente, las distribuciones de interpartida $P\left\{T_{n+1}-T_{n}\leq t\right\}$ son


\begin{itemize}
\item[a)] $1-e^{-\lambda t}$, $t\geq0$;
\item[b)] $1-e^{-\lambda t}*F\left(t\right)$, $t\geq0$;
\item[c)] $1-e^{-\lambda t}*\indora_{d}\left(t\right)$, $t\geq0$;
\item[d)] $1-e^{-\lambda t}*F\left(t\right)$, $t\geq0$.
\end{itemize}
\end{Teo}


\item Finch (1959) mostr\'o que para los sistemas $M/G/1/L$, con $1\leq L\leq \infty$ con distribuciones de servicio dos veces diferenciable, solamente el sistema $M/M/1/\infty$ tiene proceso de salida de renovaci\'on estacionario.

\item King (1971) demostro que un sistema de colas estacionario $M/G/1/1$ tiene sus tiempos de interpartida sucesivas $D_{n}$ y $D_{n+1}$ son independientes, si y s\'olo si, $G=D$, en cuyo caso le proceso de salida es de renovaci\'on.

\item Disney (1973) demostr\'o que el \'unico sistema estacionario $M/G/1/L$, que tiene proceso de salida de renovaci\'on  son los sistemas $M/M/1$ y $M/D/1/1$.



\item El siguiente resultado es de Disney y Koning (1985)
\begin{Teo}
En un sistema de espera $M/G/s$, el estado estacionario del proceso de salida es un proceso Poisson para cualquier distribuci\'on de los tiempos de servicio si el sistema tiene cualquiera de las siguientes cuatro propiedades.

\begin{itemize}
\item[a)] $s=\infty$
\item[b)] La disciplina de servicio es de procesador compartido.
\item[c)] La disciplina de servicio es LCFS y preemptive resume, esto se cumple para $L<\infty$
\item[d)] $G=M$.
\end{itemize}

\end{Teo}

\item El siguiente resultado es de Alamatsaz (1983)

\begin{Teo}
En cualquier sistema de colas $GI/G/1/L$ con $1\leq L<\infty$ y distribuci\'on de interarribos $A$ y distribuci\'on de los tiempos de servicio $B$, tal que $A\left(0\right)=0$, $A\left(t\right)\left(1-B\left(t\right)\right)>0$ para alguna $t>0$ y $B\left(t\right)$ para toda $t>0$, es imposible que el proceso de salida estacionario sea de renovaci\'on.
\end{Teo}

\end{itemize}



%________________________________________________________________________
%\subsection{Procesos Regenerativos Sigman, Thorisson y Wolff \cite{Sigman1}}
%________________________________________________________________________


\begin{Def}[Definici\'on Cl\'asica]
Un proceso estoc\'astico $X=\left\{X\left(t\right):t\geq0\right\}$ es llamado regenerativo is existe una variable aleatoria $R_{1}>0$ tal que
\begin{itemize}
\item[i)] $\left\{X\left(t+R_{1}\right):t\geq0\right\}$ es independiente de $\left\{\left\{X\left(t\right):t<R_{1}\right\},\right\}$
\item[ii)] $\left\{X\left(t+R_{1}\right):t\geq0\right\}$ es estoc\'asticamente equivalente a $\left\{X\left(t\right):t>0\right\}$
\end{itemize}

Llamamos a $R_{1}$ tiempo de regeneraci\'on, y decimos que $X$ se regenera en este punto.
\end{Def}

$\left\{X\left(t+R_{1}\right)\right\}$ es regenerativo con tiempo de regeneraci\'on $R_{2}$, independiente de $R_{1}$ pero con la misma distribuci\'on que $R_{1}$. Procediendo de esta manera se obtiene una secuencia de variables aleatorias independientes e id\'enticamente distribuidas $\left\{R_{n}\right\}$ llamados longitudes de ciclo. Si definimos a $Z_{k}\equiv R_{1}+R_{2}+\cdots+R_{k}$, se tiene un proceso de renovaci\'on llamado proceso de renovaci\'on encajado para $X$.


\begin{Note}
La existencia de un primer tiempo de regeneraci\'on, $R_{1}$, implica la existencia de una sucesi\'on completa de estos tiempos $R_{1},R_{2}\ldots,$ que satisfacen la propiedad deseada \cite{Sigman2}.
\end{Note}


\begin{Note} Para la cola $GI/GI/1$ los usuarios arriban con tiempos $t_{n}$ y son atendidos con tiempos de servicio $S_{n}$, los tiempos de arribo forman un proceso de renovaci\'on  con tiempos entre arribos independientes e identicamente distribuidos (\texttt{i.i.d.})$T_{n}=t_{n}-t_{n-1}$, adem\'as los tiempos de servicio son \texttt{i.i.d.} e independientes de los procesos de arribo. Por \textit{estable} se entiende que $\esp S_{n}<\esp T_{n}<\infty$.
\end{Note}
 


\begin{Def}
Para $x$ fijo y para cada $t\geq0$, sea $I_{x}\left(t\right)=1$ si $X\left(t\right)\leq x$,  $I_{x}\left(t\right)=0$ en caso contrario, y def\'inanse los tiempos promedio
\begin{eqnarray*}
\overline{X}&=&lim_{t\rightarrow\infty}\frac{1}{t}\int_{0}^{\infty}X\left(u\right)du\\
\prob\left(X_{\infty}\leq x\right)&=&lim_{t\rightarrow\infty}\frac{1}{t}\int_{0}^{\infty}I_{x}\left(u\right)du,
\end{eqnarray*}
cuando estos l\'imites existan.
\end{Def}

Como consecuencia del teorema de Renovaci\'on-Recompensa, se tiene que el primer l\'imite  existe y es igual a la constante
\begin{eqnarray*}
\overline{X}&=&\frac{\esp\left[\int_{0}^{R_{1}}X\left(t\right)dt\right]}{\esp\left[R_{1}\right]},
\end{eqnarray*}
suponiendo que ambas esperanzas son finitas.
 
\begin{Note}
Funciones de procesos regenerativos son regenerativas, es decir, si $X\left(t\right)$ es regenerativo y se define el proceso $Y\left(t\right)$ por $Y\left(t\right)=f\left(X\left(t\right)\right)$ para alguna funci\'on Borel medible $f\left(\cdot\right)$. Adem\'as $Y$ es regenerativo con los mismos tiempos de renovaci\'on que $X$. 

En general, los tiempos de renovaci\'on, $Z_{k}$ de un proceso regenerativo no requieren ser tiempos de paro con respecto a la evoluci\'on de $X\left(t\right)$.
\end{Note} 

\begin{Note}
Una funci\'on de un proceso de Markov, usualmente no ser\'a un proceso de Markov, sin embargo ser\'a regenerativo si el proceso de Markov lo es.
\end{Note}

 
\begin{Note}
Un proceso regenerativo con media de la longitud de ciclo finita es llamado positivo recurrente.
\end{Note}


\begin{Note}
\begin{itemize}
\item[a)] Si el proceso regenerativo $X$ es positivo recurrente y tiene trayectorias muestrales no negativas, entonces la ecuaci\'on anterior es v\'alida.
\item[b)] Si $X$ es positivo recurrente regenerativo, podemos construir una \'unica versi\'on estacionaria de este proceso, $X_{e}=\left\{X_{e}\left(t\right)\right\}$, donde $X_{e}$ es un proceso estoc\'astico regenerativo y estrictamente estacionario, con distribuci\'on marginal distribuida como $X_{\infty}$
\end{itemize}
\end{Note}


%__________________________________________________________________________________________
%\subsection{Procesos Regenerativos Estacionarios - Stidham \cite{Stidham}}
%__________________________________________________________________________________________


Un proceso estoc\'astico a tiempo continuo $\left\{V\left(t\right),t\geq0\right\}$ es un proceso regenerativo si existe una sucesi\'on de variables aleatorias independientes e id\'enticamente distribuidas $\left\{X_{1},X_{2},\ldots\right\}$, sucesi\'on de renovaci\'on, tal que para cualquier conjunto de Borel $A$, 

\begin{eqnarray*}
\prob\left\{V\left(t\right)\in A|X_{1}+X_{2}+\cdots+X_{R\left(t\right)}=s,\left\{V\left(\tau\right),\tau<s\right\}\right\}=\prob\left\{V\left(t-s\right)\in A|X_{1}>t-s\right\},
\end{eqnarray*}
para todo $0\leq s\leq t$, donde $R\left(t\right)=\max\left\{X_{1}+X_{2}+\cdots+X_{j}\leq t\right\}=$n\'umero de renovaciones ({\emph{puntos de regeneraci\'on}}) que ocurren en $\left[0,t\right]$. El intervalo $\left[0,X_{1}\right)$ es llamado {\emph{primer ciclo de regeneraci\'on}} de $\left\{V\left(t \right),t\geq0\right\}$, $\left[X_{1},X_{1}+X_{2}\right)$ el {\emph{segundo ciclo de regeneraci\'on}}, y as\'i sucesivamente.

Sea $X=X_{1}$ y sea $F$ la funci\'on de distrbuci\'on de $X$


\begin{Def}
Se define el proceso estacionario, $\left\{V^{*}\left(t\right),t\geq0\right\}$, para $\left\{V\left(t\right),t\geq0\right\}$ por

\begin{eqnarray*}
\prob\left\{V\left(t\right)\in A\right\}=\frac{1}{\esp\left[X\right]}\int_{0}^{\infty}\prob\left\{V\left(t+x\right)\in A|X>x\right\}\left(1-F\left(x\right)\right)dx,
\end{eqnarray*} 
para todo $t\geq0$ y todo conjunto de Borel $A$.
\end{Def}

\begin{Def}
Una distribuci\'on se dice que es {\emph{aritm\'etica}} si todos sus puntos de incremento son m\'ultiplos de la forma $0,\lambda, 2\lambda,\ldots$ para alguna $\lambda>0$ entera.
\end{Def}


\begin{Def}
Una modificaci\'on medible de un proceso $\left\{V\left(t\right),t\geq0\right\}$, es una versi\'on de este, $\left\{V\left(t,w\right)\right\}$ conjuntamente medible para $t\geq0$ y para $w\in S$, $S$ espacio de estados para $\left\{V\left(t\right),t\geq0\right\}$.
\end{Def}

\begin{Teo}
Sea $\left\{V\left(t\right),t\geq\right\}$ un proceso regenerativo no negativo con modificaci\'on medible. Sea $\esp\left[X\right]<\infty$. Entonces el proceso estacionario dado por la ecuaci\'on anterior est\'a bien definido y tiene funci\'on de distribuci\'on independiente de $t$, adem\'as
\begin{itemize}
\item[i)] \begin{eqnarray*}
\esp\left[V^{*}\left(0\right)\right]&=&\frac{\esp\left[\int_{0}^{X}V\left(s\right)ds\right]}{\esp\left[X\right]}\end{eqnarray*}
\item[ii)] Si $\esp\left[V^{*}\left(0\right)\right]<\infty$, equivalentemente, si $\esp\left[\int_{0}^{X}V\left(s\right)ds\right]<\infty$,entonces
\begin{eqnarray*}
\frac{\int_{0}^{t}V\left(s\right)ds}{t}\rightarrow\frac{\esp\left[\int_{0}^{X}V\left(s\right)ds\right]}{\esp\left[X\right]}
\end{eqnarray*}
con probabilidad 1 y en media, cuando $t\rightarrow\infty$.
\end{itemize}
\end{Teo}

\begin{Coro}
Sea $\left\{V\left(t\right),t\geq0\right\}$ un proceso regenerativo no negativo, con modificaci\'on medible. Si $\esp <\infty$, $F$ es no-aritm\'etica, y para todo $x\geq0$, $P\left\{V\left(t\right)\leq x,C>x\right\}$ es de variaci\'on acotada como funci\'on de $t$ en cada intervalo finito $\left[0,\tau\right]$, entonces $V\left(t\right)$ converge en distribuci\'on  cuando $t\rightarrow\infty$ y $$\esp V=\frac{\esp \int_{0}^{X}V\left(s\right)ds}{\esp X}$$
Donde $V$ tiene la distribuci\'on l\'imite de $V\left(t\right)$ cuando $t\rightarrow\infty$.

\end{Coro}

Para el caso discreto se tienen resultados similares.



%______________________________________________________________________
%\subsection{Procesos de Renovaci\'on}
%______________________________________________________________________

\begin{Def}%\label{Def.Tn}
Sean $0\leq T_{1}\leq T_{2}\leq \ldots$ son tiempos aleatorios infinitos en los cuales ocurren ciertos eventos. El n\'umero de tiempos $T_{n}$ en el intervalo $\left[0,t\right)$ es

\begin{eqnarray}
N\left(t\right)=\sum_{n=1}^{\infty}\indora\left(T_{n}\leq t\right),
\end{eqnarray}
para $t\geq0$.
\end{Def}

Si se consideran los puntos $T_{n}$ como elementos de $\rea_{+}$, y $N\left(t\right)$ es el n\'umero de puntos en $\rea$. El proceso denotado por $\left\{N\left(t\right):t\geq0\right\}$, denotado por $N\left(t\right)$, es un proceso puntual en $\rea_{+}$. Los $T_{n}$ son los tiempos de ocurrencia, el proceso puntual $N\left(t\right)$ es simple si su n\'umero de ocurrencias son distintas: $0<T_{1}<T_{2}<\ldots$ casi seguramente.

\begin{Def}
Un proceso puntual $N\left(t\right)$ es un proceso de renovaci\'on si los tiempos de interocurrencia $\xi_{n}=T_{n}-T_{n-1}$, para $n\geq1$, son independientes e identicamente distribuidos con distribuci\'on $F$, donde $F\left(0\right)=0$ y $T_{0}=0$. Los $T_{n}$ son llamados tiempos de renovaci\'on, referente a la independencia o renovaci\'on de la informaci\'on estoc\'astica en estos tiempos. Los $\xi_{n}$ son los tiempos de inter-renovaci\'on, y $N\left(t\right)$ es el n\'umero de renovaciones en el intervalo $\left[0,t\right)$
\end{Def}


\begin{Note}
Para definir un proceso de renovaci\'on para cualquier contexto, solamente hay que especificar una distribuci\'on $F$, con $F\left(0\right)=0$, para los tiempos de inter-renovaci\'on. La funci\'on $F$ en turno degune las otra variables aleatorias. De manera formal, existe un espacio de probabilidad y una sucesi\'on de variables aleatorias $\xi_{1},\xi_{2},\ldots$ definidas en este con distribuci\'on $F$. Entonces las otras cantidades son $T_{n}=\sum_{k=1}^{n}\xi_{k}$ y $N\left(t\right)=\sum_{n=1}^{\infty}\indora\left(T_{n}\leq t\right)$, donde $T_{n}\rightarrow\infty$ casi seguramente por la Ley Fuerte de los Grandes Números.
\end{Note}

%___________________________________________________________________________________________
%
%\subsection{Teorema Principal de Renovaci\'on}
%___________________________________________________________________________________________
%

\begin{Note} Una funci\'on $h:\rea_{+}\rightarrow\rea$ es Directamente Riemann Integrable en los siguientes casos:
\begin{itemize}
\item[a)] $h\left(t\right)\geq0$ es decreciente y Riemann Integrable.
\item[b)] $h$ es continua excepto posiblemente en un conjunto de Lebesgue de medida 0, y $|h\left(t\right)|\leq b\left(t\right)$, donde $b$ es DRI.
\end{itemize}
\end{Note}

\begin{Teo}[Teorema Principal de Renovaci\'on]
Si $F$ es no aritm\'etica y $h\left(t\right)$ es Directamente Riemann Integrable (DRI), entonces

\begin{eqnarray*}
lim_{t\rightarrow\infty}U\star h=\frac{1}{\mu}\int_{\rea_{+}}h\left(s\right)ds.
\end{eqnarray*}
\end{Teo}

\begin{Prop}
Cualquier funci\'on $H\left(t\right)$ acotada en intervalos finitos y que es 0 para $t<0$ puede expresarse como
\begin{eqnarray*}
H\left(t\right)=U\star h\left(t\right)\textrm{,  donde }h\left(t\right)=H\left(t\right)-F\star H\left(t\right)
\end{eqnarray*}
\end{Prop}

\begin{Def}
Un proceso estoc\'astico $X\left(t\right)$ es crudamente regenerativo en un tiempo aleatorio positivo $T$ si
\begin{eqnarray*}
\esp\left[X\left(T+t\right)|T\right]=\esp\left[X\left(t\right)\right]\textrm{, para }t\geq0,\end{eqnarray*}
y con las esperanzas anteriores finitas.
\end{Def}

\begin{Prop}
Sup\'ongase que $X\left(t\right)$ es un proceso crudamente regenerativo en $T$, que tiene distribuci\'on $F$. Si $\esp\left[X\left(t\right)\right]$ es acotado en intervalos finitos, entonces
\begin{eqnarray*}
\esp\left[X\left(t\right)\right]=U\star h\left(t\right)\textrm{,  donde }h\left(t\right)=\esp\left[X\left(t\right)\indora\left(T>t\right)\right].
\end{eqnarray*}
\end{Prop}

\begin{Teo}[Regeneraci\'on Cruda]
Sup\'ongase que $X\left(t\right)$ es un proceso con valores positivo crudamente regenerativo en $T$, y def\'inase $M=\sup\left\{|X\left(t\right)|:t\leq T\right\}$. Si $T$ es no aritm\'etico y $M$ y $MT$ tienen media finita, entonces
\begin{eqnarray*}
lim_{t\rightarrow\infty}\esp\left[X\left(t\right)\right]=\frac{1}{\mu}\int_{\rea_{+}}h\left(s\right)ds,
\end{eqnarray*}
donde $h\left(t\right)=\esp\left[X\left(t\right)\indora\left(T>t\right)\right]$.
\end{Teo}

%___________________________________________________________________________________________
%
%\subsection{Propiedades de los Procesos de Renovaci\'on}
%___________________________________________________________________________________________
%

Los tiempos $T_{n}$ est\'an relacionados con los conteos de $N\left(t\right)$ por

\begin{eqnarray*}
\left\{N\left(t\right)\geq n\right\}&=&\left\{T_{n}\leq t\right\}\\
T_{N\left(t\right)}\leq &t&<T_{N\left(t\right)+1},
\end{eqnarray*}

adem\'as $N\left(T_{n}\right)=n$, y 

\begin{eqnarray*}
N\left(t\right)=\max\left\{n:T_{n}\leq t\right\}=\min\left\{n:T_{n+1}>t\right\}
\end{eqnarray*}

Por propiedades de la convoluci\'on se sabe que

\begin{eqnarray*}
P\left\{T_{n}\leq t\right\}=F^{n\star}\left(t\right)
\end{eqnarray*}
que es la $n$-\'esima convoluci\'on de $F$. Entonces 

\begin{eqnarray*}
\left\{N\left(t\right)\geq n\right\}&=&\left\{T_{n}\leq t\right\}\\
P\left\{N\left(t\right)\leq n\right\}&=&1-F^{\left(n+1\right)\star}\left(t\right)
\end{eqnarray*}

Adem\'as usando el hecho de que $\esp\left[N\left(t\right)\right]=\sum_{n=1}^{\infty}P\left\{N\left(t\right)\geq n\right\}$
se tiene que

\begin{eqnarray*}
\esp\left[N\left(t\right)\right]=\sum_{n=1}^{\infty}F^{n\star}\left(t\right)
\end{eqnarray*}

\begin{Prop}
Para cada $t\geq0$, la funci\'on generadora de momentos $\esp\left[e^{\alpha N\left(t\right)}\right]$ existe para alguna $\alpha$ en una vecindad del 0, y de aqu\'i que $\esp\left[N\left(t\right)^{m}\right]<\infty$, para $m\geq1$.
\end{Prop}


\begin{Note}
Si el primer tiempo de renovaci\'on $\xi_{1}$ no tiene la misma distribuci\'on que el resto de las $\xi_{n}$, para $n\geq2$, a $N\left(t\right)$ se le llama Proceso de Renovaci\'on retardado, donde si $\xi$ tiene distribuci\'on $G$, entonces el tiempo $T_{n}$ de la $n$-\'esima renovaci\'on tiene distribuci\'on $G\star F^{\left(n-1\right)\star}\left(t\right)$
\end{Note}


\begin{Teo}
Para una constante $\mu\leq\infty$ ( o variable aleatoria), las siguientes expresiones son equivalentes:

\begin{eqnarray}
lim_{n\rightarrow\infty}n^{-1}T_{n}&=&\mu,\textrm{ c.s.}\\
lim_{t\rightarrow\infty}t^{-1}N\left(t\right)&=&1/\mu,\textrm{ c.s.}
\end{eqnarray}
\end{Teo}


Es decir, $T_{n}$ satisface la Ley Fuerte de los Grandes N\'umeros s\'i y s\'olo s\'i $N\left/t\right)$ la cumple.


\begin{Coro}[Ley Fuerte de los Grandes N\'umeros para Procesos de Renovaci\'on]
Si $N\left(t\right)$ es un proceso de renovaci\'on cuyos tiempos de inter-renovaci\'on tienen media $\mu\leq\infty$, entonces
\begin{eqnarray}
t^{-1}N\left(t\right)\rightarrow 1/\mu,\textrm{ c.s. cuando }t\rightarrow\infty.
\end{eqnarray}

\end{Coro}


Considerar el proceso estoc\'astico de valores reales $\left\{Z\left(t\right):t\geq0\right\}$ en el mismo espacio de probabilidad que $N\left(t\right)$

\begin{Def}
Para el proceso $\left\{Z\left(t\right):t\geq0\right\}$ se define la fluctuaci\'on m\'axima de $Z\left(t\right)$ en el intervalo $\left(T_{n-1},T_{n}\right]$:
\begin{eqnarray*}
M_{n}=\sup_{T_{n-1}<t\leq T_{n}}|Z\left(t\right)-Z\left(T_{n-1}\right)|
\end{eqnarray*}
\end{Def}

\begin{Teo}
Sup\'ongase que $n^{-1}T_{n}\rightarrow\mu$ c.s. cuando $n\rightarrow\infty$, donde $\mu\leq\infty$ es una constante o variable aleatoria. Sea $a$ una constante o variable aleatoria que puede ser infinita cuando $\mu$ es finita, y considere las expresiones l\'imite:
\begin{eqnarray}
lim_{n\rightarrow\infty}n^{-1}Z\left(T_{n}\right)&=&a,\textrm{ c.s.}\\
lim_{t\rightarrow\infty}t^{-1}Z\left(t\right)&=&a/\mu,\textrm{ c.s.}
\end{eqnarray}
La segunda expresi\'on implica la primera. Conversamente, la primera implica la segunda si el proceso $Z\left(t\right)$ es creciente, o si $lim_{n\rightarrow\infty}n^{-1}M_{n}=0$ c.s.
\end{Teo}

\begin{Coro}
Si $N\left(t\right)$ es un proceso de renovaci\'on, y $\left(Z\left(T_{n}\right)-Z\left(T_{n-1}\right),M_{n}\right)$, para $n\geq1$, son variables aleatorias independientes e id\'enticamente distribuidas con media finita, entonces,
\begin{eqnarray}
lim_{t\rightarrow\infty}t^{-1}Z\left(t\right)\rightarrow\frac{\esp\left[Z\left(T_{1}\right)-Z\left(T_{0}\right)\right]}{\esp\left[T_{1}\right]},\textrm{ c.s. cuando  }t\rightarrow\infty.
\end{eqnarray}
\end{Coro}



%___________________________________________________________________________________________
%
%\subsection{Propiedades de los Procesos de Renovaci\'on}
%___________________________________________________________________________________________
%

Los tiempos $T_{n}$ est\'an relacionados con los conteos de $N\left(t\right)$ por

\begin{eqnarray*}
\left\{N\left(t\right)\geq n\right\}&=&\left\{T_{n}\leq t\right\}\\
T_{N\left(t\right)}\leq &t&<T_{N\left(t\right)+1},
\end{eqnarray*}

adem\'as $N\left(T_{n}\right)=n$, y 

\begin{eqnarray*}
N\left(t\right)=\max\left\{n:T_{n}\leq t\right\}=\min\left\{n:T_{n+1}>t\right\}
\end{eqnarray*}

Por propiedades de la convoluci\'on se sabe que

\begin{eqnarray*}
P\left\{T_{n}\leq t\right\}=F^{n\star}\left(t\right)
\end{eqnarray*}
que es la $n$-\'esima convoluci\'on de $F$. Entonces 

\begin{eqnarray*}
\left\{N\left(t\right)\geq n\right\}&=&\left\{T_{n}\leq t\right\}\\
P\left\{N\left(t\right)\leq n\right\}&=&1-F^{\left(n+1\right)\star}\left(t\right)
\end{eqnarray*}

Adem\'as usando el hecho de que $\esp\left[N\left(t\right)\right]=\sum_{n=1}^{\infty}P\left\{N\left(t\right)\geq n\right\}$
se tiene que

\begin{eqnarray*}
\esp\left[N\left(t\right)\right]=\sum_{n=1}^{\infty}F^{n\star}\left(t\right)
\end{eqnarray*}

\begin{Prop}
Para cada $t\geq0$, la funci\'on generadora de momentos $\esp\left[e^{\alpha N\left(t\right)}\right]$ existe para alguna $\alpha$ en una vecindad del 0, y de aqu\'i que $\esp\left[N\left(t\right)^{m}\right]<\infty$, para $m\geq1$.
\end{Prop}


\begin{Note}
Si el primer tiempo de renovaci\'on $\xi_{1}$ no tiene la misma distribuci\'on que el resto de las $\xi_{n}$, para $n\geq2$, a $N\left(t\right)$ se le llama Proceso de Renovaci\'on retardado, donde si $\xi$ tiene distribuci\'on $G$, entonces el tiempo $T_{n}$ de la $n$-\'esima renovaci\'on tiene distribuci\'on $G\star F^{\left(n-1\right)\star}\left(t\right)$
\end{Note}


\begin{Teo}
Para una constante $\mu\leq\infty$ ( o variable aleatoria), las siguientes expresiones son equivalentes:

\begin{eqnarray}
lim_{n\rightarrow\infty}n^{-1}T_{n}&=&\mu,\textrm{ c.s.}\\
lim_{t\rightarrow\infty}t^{-1}N\left(t\right)&=&1/\mu,\textrm{ c.s.}
\end{eqnarray}
\end{Teo}


Es decir, $T_{n}$ satisface la Ley Fuerte de los Grandes N\'umeros s\'i y s\'olo s\'i $N\left/t\right)$ la cumple.


\begin{Coro}[Ley Fuerte de los Grandes N\'umeros para Procesos de Renovaci\'on]
Si $N\left(t\right)$ es un proceso de renovaci\'on cuyos tiempos de inter-renovaci\'on tienen media $\mu\leq\infty$, entonces
\begin{eqnarray}
t^{-1}N\left(t\right)\rightarrow 1/\mu,\textrm{ c.s. cuando }t\rightarrow\infty.
\end{eqnarray}

\end{Coro}


Considerar el proceso estoc\'astico de valores reales $\left\{Z\left(t\right):t\geq0\right\}$ en el mismo espacio de probabilidad que $N\left(t\right)$

\begin{Def}
Para el proceso $\left\{Z\left(t\right):t\geq0\right\}$ se define la fluctuaci\'on m\'axima de $Z\left(t\right)$ en el intervalo $\left(T_{n-1},T_{n}\right]$:
\begin{eqnarray*}
M_{n}=\sup_{T_{n-1}<t\leq T_{n}}|Z\left(t\right)-Z\left(T_{n-1}\right)|
\end{eqnarray*}
\end{Def}

\begin{Teo}
Sup\'ongase que $n^{-1}T_{n}\rightarrow\mu$ c.s. cuando $n\rightarrow\infty$, donde $\mu\leq\infty$ es una constante o variable aleatoria. Sea $a$ una constante o variable aleatoria que puede ser infinita cuando $\mu$ es finita, y considere las expresiones l\'imite:
\begin{eqnarray}
lim_{n\rightarrow\infty}n^{-1}Z\left(T_{n}\right)&=&a,\textrm{ c.s.}\\
lim_{t\rightarrow\infty}t^{-1}Z\left(t\right)&=&a/\mu,\textrm{ c.s.}
\end{eqnarray}
La segunda expresi\'on implica la primera. Conversamente, la primera implica la segunda si el proceso $Z\left(t\right)$ es creciente, o si $lim_{n\rightarrow\infty}n^{-1}M_{n}=0$ c.s.
\end{Teo}

\begin{Coro}
Si $N\left(t\right)$ es un proceso de renovaci\'on, y $\left(Z\left(T_{n}\right)-Z\left(T_{n-1}\right),M_{n}\right)$, para $n\geq1$, son variables aleatorias independientes e id\'enticamente distribuidas con media finita, entonces,
\begin{eqnarray}
lim_{t\rightarrow\infty}t^{-1}Z\left(t\right)\rightarrow\frac{\esp\left[Z\left(T_{1}\right)-Z\left(T_{0}\right)\right]}{\esp\left[T_{1}\right]},\textrm{ c.s. cuando  }t\rightarrow\infty.
\end{eqnarray}
\end{Coro}


%___________________________________________________________________________________________
%
%\subsection{Propiedades de los Procesos de Renovaci\'on}
%___________________________________________________________________________________________
%

Los tiempos $T_{n}$ est\'an relacionados con los conteos de $N\left(t\right)$ por

\begin{eqnarray*}
\left\{N\left(t\right)\geq n\right\}&=&\left\{T_{n}\leq t\right\}\\
T_{N\left(t\right)}\leq &t&<T_{N\left(t\right)+1},
\end{eqnarray*}

adem\'as $N\left(T_{n}\right)=n$, y 

\begin{eqnarray*}
N\left(t\right)=\max\left\{n:T_{n}\leq t\right\}=\min\left\{n:T_{n+1}>t\right\}
\end{eqnarray*}

Por propiedades de la convoluci\'on se sabe que

\begin{eqnarray*}
P\left\{T_{n}\leq t\right\}=F^{n\star}\left(t\right)
\end{eqnarray*}
que es la $n$-\'esima convoluci\'on de $F$. Entonces 

\begin{eqnarray*}
\left\{N\left(t\right)\geq n\right\}&=&\left\{T_{n}\leq t\right\}\\
P\left\{N\left(t\right)\leq n\right\}&=&1-F^{\left(n+1\right)\star}\left(t\right)
\end{eqnarray*}

Adem\'as usando el hecho de que $\esp\left[N\left(t\right)\right]=\sum_{n=1}^{\infty}P\left\{N\left(t\right)\geq n\right\}$
se tiene que

\begin{eqnarray*}
\esp\left[N\left(t\right)\right]=\sum_{n=1}^{\infty}F^{n\star}\left(t\right)
\end{eqnarray*}

\begin{Prop}
Para cada $t\geq0$, la funci\'on generadora de momentos $\esp\left[e^{\alpha N\left(t\right)}\right]$ existe para alguna $\alpha$ en una vecindad del 0, y de aqu\'i que $\esp\left[N\left(t\right)^{m}\right]<\infty$, para $m\geq1$.
\end{Prop}


\begin{Note}
Si el primer tiempo de renovaci\'on $\xi_{1}$ no tiene la misma distribuci\'on que el resto de las $\xi_{n}$, para $n\geq2$, a $N\left(t\right)$ se le llama Proceso de Renovaci\'on retardado, donde si $\xi$ tiene distribuci\'on $G$, entonces el tiempo $T_{n}$ de la $n$-\'esima renovaci\'on tiene distribuci\'on $G\star F^{\left(n-1\right)\star}\left(t\right)$
\end{Note}


\begin{Teo}
Para una constante $\mu\leq\infty$ ( o variable aleatoria), las siguientes expresiones son equivalentes:

\begin{eqnarray}
lim_{n\rightarrow\infty}n^{-1}T_{n}&=&\mu,\textrm{ c.s.}\\
lim_{t\rightarrow\infty}t^{-1}N\left(t\right)&=&1/\mu,\textrm{ c.s.}
\end{eqnarray}
\end{Teo}


Es decir, $T_{n}$ satisface la Ley Fuerte de los Grandes N\'umeros s\'i y s\'olo s\'i $N\left/t\right)$ la cumple.


\begin{Coro}[Ley Fuerte de los Grandes N\'umeros para Procesos de Renovaci\'on]
Si $N\left(t\right)$ es un proceso de renovaci\'on cuyos tiempos de inter-renovaci\'on tienen media $\mu\leq\infty$, entonces
\begin{eqnarray}
t^{-1}N\left(t\right)\rightarrow 1/\mu,\textrm{ c.s. cuando }t\rightarrow\infty.
\end{eqnarray}

\end{Coro}


Considerar el proceso estoc\'astico de valores reales $\left\{Z\left(t\right):t\geq0\right\}$ en el mismo espacio de probabilidad que $N\left(t\right)$

\begin{Def}
Para el proceso $\left\{Z\left(t\right):t\geq0\right\}$ se define la fluctuaci\'on m\'axima de $Z\left(t\right)$ en el intervalo $\left(T_{n-1},T_{n}\right]$:
\begin{eqnarray*}
M_{n}=\sup_{T_{n-1}<t\leq T_{n}}|Z\left(t\right)-Z\left(T_{n-1}\right)|
\end{eqnarray*}
\end{Def}

\begin{Teo}
Sup\'ongase que $n^{-1}T_{n}\rightarrow\mu$ c.s. cuando $n\rightarrow\infty$, donde $\mu\leq\infty$ es una constante o variable aleatoria. Sea $a$ una constante o variable aleatoria que puede ser infinita cuando $\mu$ es finita, y considere las expresiones l\'imite:
\begin{eqnarray}
lim_{n\rightarrow\infty}n^{-1}Z\left(T_{n}\right)&=&a,\textrm{ c.s.}\\
lim_{t\rightarrow\infty}t^{-1}Z\left(t\right)&=&a/\mu,\textrm{ c.s.}
\end{eqnarray}
La segunda expresi\'on implica la primera. Conversamente, la primera implica la segunda si el proceso $Z\left(t\right)$ es creciente, o si $lim_{n\rightarrow\infty}n^{-1}M_{n}=0$ c.s.
\end{Teo}

\begin{Coro}
Si $N\left(t\right)$ es un proceso de renovaci\'on, y $\left(Z\left(T_{n}\right)-Z\left(T_{n-1}\right),M_{n}\right)$, para $n\geq1$, son variables aleatorias independientes e id\'enticamente distribuidas con media finita, entonces,
\begin{eqnarray}
lim_{t\rightarrow\infty}t^{-1}Z\left(t\right)\rightarrow\frac{\esp\left[Z\left(T_{1}\right)-Z\left(T_{0}\right)\right]}{\esp\left[T_{1}\right]},\textrm{ c.s. cuando  }t\rightarrow\infty.
\end{eqnarray}
\end{Coro}

%___________________________________________________________________________________________
%
%\subsection{Propiedades de los Procesos de Renovaci\'on}
%___________________________________________________________________________________________
%

Los tiempos $T_{n}$ est\'an relacionados con los conteos de $N\left(t\right)$ por

\begin{eqnarray*}
\left\{N\left(t\right)\geq n\right\}&=&\left\{T_{n}\leq t\right\}\\
T_{N\left(t\right)}\leq &t&<T_{N\left(t\right)+1},
\end{eqnarray*}

adem\'as $N\left(T_{n}\right)=n$, y 

\begin{eqnarray*}
N\left(t\right)=\max\left\{n:T_{n}\leq t\right\}=\min\left\{n:T_{n+1}>t\right\}
\end{eqnarray*}

Por propiedades de la convoluci\'on se sabe que

\begin{eqnarray*}
P\left\{T_{n}\leq t\right\}=F^{n\star}\left(t\right)
\end{eqnarray*}
que es la $n$-\'esima convoluci\'on de $F$. Entonces 

\begin{eqnarray*}
\left\{N\left(t\right)\geq n\right\}&=&\left\{T_{n}\leq t\right\}\\
P\left\{N\left(t\right)\leq n\right\}&=&1-F^{\left(n+1\right)\star}\left(t\right)
\end{eqnarray*}

Adem\'as usando el hecho de que $\esp\left[N\left(t\right)\right]=\sum_{n=1}^{\infty}P\left\{N\left(t\right)\geq n\right\}$
se tiene que

\begin{eqnarray*}
\esp\left[N\left(t\right)\right]=\sum_{n=1}^{\infty}F^{n\star}\left(t\right)
\end{eqnarray*}

\begin{Prop}
Para cada $t\geq0$, la funci\'on generadora de momentos $\esp\left[e^{\alpha N\left(t\right)}\right]$ existe para alguna $\alpha$ en una vecindad del 0, y de aqu\'i que $\esp\left[N\left(t\right)^{m}\right]<\infty$, para $m\geq1$.
\end{Prop}


\begin{Note}
Si el primer tiempo de renovaci\'on $\xi_{1}$ no tiene la misma distribuci\'on que el resto de las $\xi_{n}$, para $n\geq2$, a $N\left(t\right)$ se le llama Proceso de Renovaci\'on retardado, donde si $\xi$ tiene distribuci\'on $G$, entonces el tiempo $T_{n}$ de la $n$-\'esima renovaci\'on tiene distribuci\'on $G\star F^{\left(n-1\right)\star}\left(t\right)$
\end{Note}


\begin{Teo}
Para una constante $\mu\leq\infty$ ( o variable aleatoria), las siguientes expresiones son equivalentes:

\begin{eqnarray}
lim_{n\rightarrow\infty}n^{-1}T_{n}&=&\mu,\textrm{ c.s.}\\
lim_{t\rightarrow\infty}t^{-1}N\left(t\right)&=&1/\mu,\textrm{ c.s.}
\end{eqnarray}
\end{Teo}


Es decir, $T_{n}$ satisface la Ley Fuerte de los Grandes N\'umeros s\'i y s\'olo s\'i $N\left/t\right)$ la cumple.


\begin{Coro}[Ley Fuerte de los Grandes N\'umeros para Procesos de Renovaci\'on]
Si $N\left(t\right)$ es un proceso de renovaci\'on cuyos tiempos de inter-renovaci\'on tienen media $\mu\leq\infty$, entonces
\begin{eqnarray}
t^{-1}N\left(t\right)\rightarrow 1/\mu,\textrm{ c.s. cuando }t\rightarrow\infty.
\end{eqnarray}

\end{Coro}


Considerar el proceso estoc\'astico de valores reales $\left\{Z\left(t\right):t\geq0\right\}$ en el mismo espacio de probabilidad que $N\left(t\right)$

\begin{Def}
Para el proceso $\left\{Z\left(t\right):t\geq0\right\}$ se define la fluctuaci\'on m\'axima de $Z\left(t\right)$ en el intervalo $\left(T_{n-1},T_{n}\right]$:
\begin{eqnarray*}
M_{n}=\sup_{T_{n-1}<t\leq T_{n}}|Z\left(t\right)-Z\left(T_{n-1}\right)|
\end{eqnarray*}
\end{Def}

\begin{Teo}
Sup\'ongase que $n^{-1}T_{n}\rightarrow\mu$ c.s. cuando $n\rightarrow\infty$, donde $\mu\leq\infty$ es una constante o variable aleatoria. Sea $a$ una constante o variable aleatoria que puede ser infinita cuando $\mu$ es finita, y considere las expresiones l\'imite:
\begin{eqnarray}
lim_{n\rightarrow\infty}n^{-1}Z\left(T_{n}\right)&=&a,\textrm{ c.s.}\\
lim_{t\rightarrow\infty}t^{-1}Z\left(t\right)&=&a/\mu,\textrm{ c.s.}
\end{eqnarray}
La segunda expresi\'on implica la primera. Conversamente, la primera implica la segunda si el proceso $Z\left(t\right)$ es creciente, o si $lim_{n\rightarrow\infty}n^{-1}M_{n}=0$ c.s.
\end{Teo}

\begin{Coro}
Si $N\left(t\right)$ es un proceso de renovaci\'on, y $\left(Z\left(T_{n}\right)-Z\left(T_{n-1}\right),M_{n}\right)$, para $n\geq1$, son variables aleatorias independientes e id\'enticamente distribuidas con media finita, entonces,
\begin{eqnarray}
lim_{t\rightarrow\infty}t^{-1}Z\left(t\right)\rightarrow\frac{\esp\left[Z\left(T_{1}\right)-Z\left(T_{0}\right)\right]}{\esp\left[T_{1}\right]},\textrm{ c.s. cuando  }t\rightarrow\infty.
\end{eqnarray}
\end{Coro}
%___________________________________________________________________________________________
%
\subsection{Propiedades de los Procesos de Renovaci\'on}
%___________________________________________________________________________________________
%

Los tiempos $T_{n}$ est\'an relacionados con los conteos de $N\left(t\right)$ por

\begin{eqnarray*}
\left\{N\left(t\right)\geq n\right\}&=&\left\{T_{n}\leq t\right\}\\
T_{N\left(t\right)}\leq &t&<T_{N\left(t\right)+1},
\end{eqnarray*}

adem\'as $N\left(T_{n}\right)=n$, y 

\begin{eqnarray*}
N\left(t\right)=\max\left\{n:T_{n}\leq t\right\}=\min\left\{n:T_{n+1}>t\right\}
\end{eqnarray*}

Por propiedades de la convoluci\'on se sabe que

\begin{eqnarray*}
P\left\{T_{n}\leq t\right\}=F^{n\star}\left(t\right)
\end{eqnarray*}
que es la $n$-\'esima convoluci\'on de $F$. Entonces 

\begin{eqnarray*}
\left\{N\left(t\right)\geq n\right\}&=&\left\{T_{n}\leq t\right\}\\
P\left\{N\left(t\right)\leq n\right\}&=&1-F^{\left(n+1\right)\star}\left(t\right)
\end{eqnarray*}

Adem\'as usando el hecho de que $\esp\left[N\left(t\right)\right]=\sum_{n=1}^{\infty}P\left\{N\left(t\right)\geq n\right\}$
se tiene que

\begin{eqnarray*}
\esp\left[N\left(t\right)\right]=\sum_{n=1}^{\infty}F^{n\star}\left(t\right)
\end{eqnarray*}

\begin{Prop}
Para cada $t\geq0$, la funci\'on generadora de momentos $\esp\left[e^{\alpha N\left(t\right)}\right]$ existe para alguna $\alpha$ en una vecindad del 0, y de aqu\'i que $\esp\left[N\left(t\right)^{m}\right]<\infty$, para $m\geq1$.
\end{Prop}


\begin{Note}
Si el primer tiempo de renovaci\'on $\xi_{1}$ no tiene la misma distribuci\'on que el resto de las $\xi_{n}$, para $n\geq2$, a $N\left(t\right)$ se le llama Proceso de Renovaci\'on retardado, donde si $\xi$ tiene distribuci\'on $G$, entonces el tiempo $T_{n}$ de la $n$-\'esima renovaci\'on tiene distribuci\'on $G\star F^{\left(n-1\right)\star}\left(t\right)$
\end{Note}


\begin{Teo}
Para una constante $\mu\leq\infty$ ( o variable aleatoria), las siguientes expresiones son equivalentes:

\begin{eqnarray}
lim_{n\rightarrow\infty}n^{-1}T_{n}&=&\mu,\textrm{ c.s.}\\
lim_{t\rightarrow\infty}t^{-1}N\left(t\right)&=&1/\mu,\textrm{ c.s.}
\end{eqnarray}
\end{Teo}


Es decir, $T_{n}$ satisface la Ley Fuerte de los Grandes N\'umeros s\'i y s\'olo s\'i $N\left/t\right)$ la cumple.


\begin{Coro}[Ley Fuerte de los Grandes N\'umeros para Procesos de Renovaci\'on]
Si $N\left(t\right)$ es un proceso de renovaci\'on cuyos tiempos de inter-renovaci\'on tienen media $\mu\leq\infty$, entonces
\begin{eqnarray}
t^{-1}N\left(t\right)\rightarrow 1/\mu,\textrm{ c.s. cuando }t\rightarrow\infty.
\end{eqnarray}

\end{Coro}


Considerar el proceso estoc\'astico de valores reales $\left\{Z\left(t\right):t\geq0\right\}$ en el mismo espacio de probabilidad que $N\left(t\right)$

\begin{Def}
Para el proceso $\left\{Z\left(t\right):t\geq0\right\}$ se define la fluctuaci\'on m\'axima de $Z\left(t\right)$ en el intervalo $\left(T_{n-1},T_{n}\right]$:
\begin{eqnarray*}
M_{n}=\sup_{T_{n-1}<t\leq T_{n}}|Z\left(t\right)-Z\left(T_{n-1}\right)|
\end{eqnarray*}
\end{Def}

\begin{Teo}
Sup\'ongase que $n^{-1}T_{n}\rightarrow\mu$ c.s. cuando $n\rightarrow\infty$, donde $\mu\leq\infty$ es una constante o variable aleatoria. Sea $a$ una constante o variable aleatoria que puede ser infinita cuando $\mu$ es finita, y considere las expresiones l\'imite:
\begin{eqnarray}
lim_{n\rightarrow\infty}n^{-1}Z\left(T_{n}\right)&=&a,\textrm{ c.s.}\\
lim_{t\rightarrow\infty}t^{-1}Z\left(t\right)&=&a/\mu,\textrm{ c.s.}
\end{eqnarray}
La segunda expresi\'on implica la primera. Conversamente, la primera implica la segunda si el proceso $Z\left(t\right)$ es creciente, o si $lim_{n\rightarrow\infty}n^{-1}M_{n}=0$ c.s.
\end{Teo}

\begin{Coro}
Si $N\left(t\right)$ es un proceso de renovaci\'on, y $\left(Z\left(T_{n}\right)-Z\left(T_{n-1}\right),M_{n}\right)$, para $n\geq1$, son variables aleatorias independientes e id\'enticamente distribuidas con media finita, entonces,
\begin{eqnarray}
lim_{t\rightarrow\infty}t^{-1}Z\left(t\right)\rightarrow\frac{\esp\left[Z\left(T_{1}\right)-Z\left(T_{0}\right)\right]}{\esp\left[T_{1}\right]},\textrm{ c.s. cuando  }t\rightarrow\infty.
\end{eqnarray}
\end{Coro}


%___________________________________________________________________________________________
%
%\subsection{Funci\'on de Renovaci\'on}
%___________________________________________________________________________________________
%


\begin{Def}
Sea $h\left(t\right)$ funci\'on de valores reales en $\rea$ acotada en intervalos finitos e igual a cero para $t<0$ La ecuaci\'on de renovaci\'on para $h\left(t\right)$ y la distribuci\'on $F$ es

\begin{eqnarray}%\label{Ec.Renovacion}
H\left(t\right)=h\left(t\right)+\int_{\left[0,t\right]}H\left(t-s\right)dF\left(s\right)\textrm{,    }t\geq0,
\end{eqnarray}
donde $H\left(t\right)$ es una funci\'on de valores reales. Esto es $H=h+F\star H$. Decimos que $H\left(t\right)$ es soluci\'on de esta ecuaci\'on si satisface la ecuaci\'on, y es acotada en intervalos finitos e iguales a cero para $t<0$.
\end{Def}

\begin{Prop}
La funci\'on $U\star h\left(t\right)$ es la \'unica soluci\'on de la ecuaci\'on de renovaci\'on (\ref{Ec.Renovacion}).
\end{Prop}

\begin{Teo}[Teorema Renovaci\'on Elemental]
\begin{eqnarray*}
t^{-1}U\left(t\right)\rightarrow 1/\mu\textrm{,    cuando }t\rightarrow\infty.
\end{eqnarray*}
\end{Teo}

%___________________________________________________________________________________________
%
%\subsection{Funci\'on de Renovaci\'on}
%___________________________________________________________________________________________
%


Sup\'ongase que $N\left(t\right)$ es un proceso de renovaci\'on con distribuci\'on $F$ con media finita $\mu$.

\begin{Def}
La funci\'on de renovaci\'on asociada con la distribuci\'on $F$, del proceso $N\left(t\right)$, es
\begin{eqnarray*}
U\left(t\right)=\sum_{n=1}^{\infty}F^{n\star}\left(t\right),\textrm{   }t\geq0,
\end{eqnarray*}
donde $F^{0\star}\left(t\right)=\indora\left(t\geq0\right)$.
\end{Def}


\begin{Prop}
Sup\'ongase que la distribuci\'on de inter-renovaci\'on $F$ tiene densidad $f$. Entonces $U\left(t\right)$ tambi\'en tiene densidad, para $t>0$, y es $U^{'}\left(t\right)=\sum_{n=0}^{\infty}f^{n\star}\left(t\right)$. Adem\'as
\begin{eqnarray*}
\prob\left\{N\left(t\right)>N\left(t-\right)\right\}=0\textrm{,   }t\geq0.
\end{eqnarray*}
\end{Prop}

\begin{Def}
La Transformada de Laplace-Stieljes de $F$ est\'a dada por

\begin{eqnarray*}
\hat{F}\left(\alpha\right)=\int_{\rea_{+}}e^{-\alpha t}dF\left(t\right)\textrm{,  }\alpha\geq0.
\end{eqnarray*}
\end{Def}

Entonces

\begin{eqnarray*}
\hat{U}\left(\alpha\right)=\sum_{n=0}^{\infty}\hat{F^{n\star}}\left(\alpha\right)=\sum_{n=0}^{\infty}\hat{F}\left(\alpha\right)^{n}=\frac{1}{1-\hat{F}\left(\alpha\right)}.
\end{eqnarray*}


\begin{Prop}
La Transformada de Laplace $\hat{U}\left(\alpha\right)$ y $\hat{F}\left(\alpha\right)$ determina una a la otra de manera \'unica por la relaci\'on $\hat{U}\left(\alpha\right)=\frac{1}{1-\hat{F}\left(\alpha\right)}$.
\end{Prop}


\begin{Note}
Un proceso de renovaci\'on $N\left(t\right)$ cuyos tiempos de inter-renovaci\'on tienen media finita, es un proceso Poisson con tasa $\lambda$ si y s\'olo s\'i $\esp\left[U\left(t\right)\right]=\lambda t$, para $t\geq0$.
\end{Note}


\begin{Teo}
Sea $N\left(t\right)$ un proceso puntual simple con puntos de localizaci\'on $T_{n}$ tal que $\eta\left(t\right)=\esp\left[N\left(\right)\right]$ es finita para cada $t$. Entonces para cualquier funci\'on $f:\rea_{+}\rightarrow\rea$,
\begin{eqnarray*}
\esp\left[\sum_{n=1}^{N\left(\right)}f\left(T_{n}\right)\right]=\int_{\left(0,t\right]}f\left(s\right)d\eta\left(s\right)\textrm{,  }t\geq0,
\end{eqnarray*}
suponiendo que la integral exista. Adem\'as si $X_{1},X_{2},\ldots$ son variables aleatorias definidas en el mismo espacio de probabilidad que el proceso $N\left(t\right)$ tal que $\esp\left[X_{n}|T_{n}=s\right]=f\left(s\right)$, independiente de $n$. Entonces
\begin{eqnarray*}
\esp\left[\sum_{n=1}^{N\left(t\right)}X_{n}\right]=\int_{\left(0,t\right]}f\left(s\right)d\eta\left(s\right)\textrm{,  }t\geq0,
\end{eqnarray*} 
suponiendo que la integral exista. 
\end{Teo}

\begin{Coro}[Identidad de Wald para Renovaciones]
Para el proceso de renovaci\'on $N\left(t\right)$,
\begin{eqnarray*}
\esp\left[T_{N\left(t\right)+1}\right]=\mu\esp\left[N\left(t\right)+1\right]\textrm{,  }t\geq0,
\end{eqnarray*}  
\end{Coro}

%______________________________________________________________________
%\subsection{Procesos de Renovaci\'on}
%______________________________________________________________________

\begin{Def}%\label{Def.Tn}
Sean $0\leq T_{1}\leq T_{2}\leq \ldots$ son tiempos aleatorios infinitos en los cuales ocurren ciertos eventos. El n\'umero de tiempos $T_{n}$ en el intervalo $\left[0,t\right)$ es

\begin{eqnarray}
N\left(t\right)=\sum_{n=1}^{\infty}\indora\left(T_{n}\leq t\right),
\end{eqnarray}
para $t\geq0$.
\end{Def}

Si se consideran los puntos $T_{n}$ como elementos de $\rea_{+}$, y $N\left(t\right)$ es el n\'umero de puntos en $\rea$. El proceso denotado por $\left\{N\left(t\right):t\geq0\right\}$, denotado por $N\left(t\right)$, es un proceso puntual en $\rea_{+}$. Los $T_{n}$ son los tiempos de ocurrencia, el proceso puntual $N\left(t\right)$ es simple si su n\'umero de ocurrencias son distintas: $0<T_{1}<T_{2}<\ldots$ casi seguramente.

\begin{Def}
Un proceso puntual $N\left(t\right)$ es un proceso de renovaci\'on si los tiempos de interocurrencia $\xi_{n}=T_{n}-T_{n-1}$, para $n\geq1$, son independientes e identicamente distribuidos con distribuci\'on $F$, donde $F\left(0\right)=0$ y $T_{0}=0$. Los $T_{n}$ son llamados tiempos de renovaci\'on, referente a la independencia o renovaci\'on de la informaci\'on estoc\'astica en estos tiempos. Los $\xi_{n}$ son los tiempos de inter-renovaci\'on, y $N\left(t\right)$ es el n\'umero de renovaciones en el intervalo $\left[0,t\right)$
\end{Def}


\begin{Note}
Para definir un proceso de renovaci\'on para cualquier contexto, solamente hay que especificar una distribuci\'on $F$, con $F\left(0\right)=0$, para los tiempos de inter-renovaci\'on. La funci\'on $F$ en turno degune las otra variables aleatorias. De manera formal, existe un espacio de probabilidad y una sucesi\'on de variables aleatorias $\xi_{1},\xi_{2},\ldots$ definidas en este con distribuci\'on $F$. Entonces las otras cantidades son $T_{n}=\sum_{k=1}^{n}\xi_{k}$ y $N\left(t\right)=\sum_{n=1}^{\infty}\indora\left(T_{n}\leq t\right)$, donde $T_{n}\rightarrow\infty$ casi seguramente por la Ley Fuerte de los Grandes Números.
\end{Note}

%___________________________________________________________________________________________
%
%\subsection{Renewal and Regenerative Processes: Serfozo\cite{Serfozo}}
%___________________________________________________________________________________________
%
\begin{Def}%\label{Def.Tn}
Sean $0\leq T_{1}\leq T_{2}\leq \ldots$ son tiempos aleatorios infinitos en los cuales ocurren ciertos eventos. El n\'umero de tiempos $T_{n}$ en el intervalo $\left[0,t\right)$ es

\begin{eqnarray}
N\left(t\right)=\sum_{n=1}^{\infty}\indora\left(T_{n}\leq t\right),
\end{eqnarray}
para $t\geq0$.
\end{Def}

Si se consideran los puntos $T_{n}$ como elementos de $\rea_{+}$, y $N\left(t\right)$ es el n\'umero de puntos en $\rea$. El proceso denotado por $\left\{N\left(t\right):t\geq0\right\}$, denotado por $N\left(t\right)$, es un proceso puntual en $\rea_{+}$. Los $T_{n}$ son los tiempos de ocurrencia, el proceso puntual $N\left(t\right)$ es simple si su n\'umero de ocurrencias son distintas: $0<T_{1}<T_{2}<\ldots$ casi seguramente.

\begin{Def}
Un proceso puntual $N\left(t\right)$ es un proceso de renovaci\'on si los tiempos de interocurrencia $\xi_{n}=T_{n}-T_{n-1}$, para $n\geq1$, son independientes e identicamente distribuidos con distribuci\'on $F$, donde $F\left(0\right)=0$ y $T_{0}=0$. Los $T_{n}$ son llamados tiempos de renovaci\'on, referente a la independencia o renovaci\'on de la informaci\'on estoc\'astica en estos tiempos. Los $\xi_{n}$ son los tiempos de inter-renovaci\'on, y $N\left(t\right)$ es el n\'umero de renovaciones en el intervalo $\left[0,t\right)$
\end{Def}


\begin{Note}
Para definir un proceso de renovaci\'on para cualquier contexto, solamente hay que especificar una distribuci\'on $F$, con $F\left(0\right)=0$, para los tiempos de inter-renovaci\'on. La funci\'on $F$ en turno degune las otra variables aleatorias. De manera formal, existe un espacio de probabilidad y una sucesi\'on de variables aleatorias $\xi_{1},\xi_{2},\ldots$ definidas en este con distribuci\'on $F$. Entonces las otras cantidades son $T_{n}=\sum_{k=1}^{n}\xi_{k}$ y $N\left(t\right)=\sum_{n=1}^{\infty}\indora\left(T_{n}\leq t\right)$, donde $T_{n}\rightarrow\infty$ casi seguramente por la Ley Fuerte de los Grandes N\'umeros.
\end{Note}







Los tiempos $T_{n}$ est\'an relacionados con los conteos de $N\left(t\right)$ por

\begin{eqnarray*}
\left\{N\left(t\right)\geq n\right\}&=&\left\{T_{n}\leq t\right\}\\
T_{N\left(t\right)}\leq &t&<T_{N\left(t\right)+1},
\end{eqnarray*}

adem\'as $N\left(T_{n}\right)=n$, y 

\begin{eqnarray*}
N\left(t\right)=\max\left\{n:T_{n}\leq t\right\}=\min\left\{n:T_{n+1}>t\right\}
\end{eqnarray*}

Por propiedades de la convoluci\'on se sabe que

\begin{eqnarray*}
P\left\{T_{n}\leq t\right\}=F^{n\star}\left(t\right)
\end{eqnarray*}
que es la $n$-\'esima convoluci\'on de $F$. Entonces 

\begin{eqnarray*}
\left\{N\left(t\right)\geq n\right\}&=&\left\{T_{n}\leq t\right\}\\
P\left\{N\left(t\right)\leq n\right\}&=&1-F^{\left(n+1\right)\star}\left(t\right)
\end{eqnarray*}

Adem\'as usando el hecho de que $\esp\left[N\left(t\right)\right]=\sum_{n=1}^{\infty}P\left\{N\left(t\right)\geq n\right\}$
se tiene que

\begin{eqnarray*}
\esp\left[N\left(t\right)\right]=\sum_{n=1}^{\infty}F^{n\star}\left(t\right)
\end{eqnarray*}

\begin{Prop}
Para cada $t\geq0$, la funci\'on generadora de momentos $\esp\left[e^{\alpha N\left(t\right)}\right]$ existe para alguna $\alpha$ en una vecindad del 0, y de aqu\'i que $\esp\left[N\left(t\right)^{m}\right]<\infty$, para $m\geq1$.
\end{Prop}

\begin{Ejem}[\textbf{Proceso Poisson}]

Suponga que se tienen tiempos de inter-renovaci\'on \textit{i.i.d.} del proceso de renovaci\'on $N\left(t\right)$ tienen distribuci\'on exponencial $F\left(t\right)=q-e^{-\lambda t}$ con tasa $\lambda$. Entonces $N\left(t\right)$ es un proceso Poisson con tasa $\lambda$.

\end{Ejem}


\begin{Note}
Si el primer tiempo de renovaci\'on $\xi_{1}$ no tiene la misma distribuci\'on que el resto de las $\xi_{n}$, para $n\geq2$, a $N\left(t\right)$ se le llama Proceso de Renovaci\'on retardado, donde si $\xi$ tiene distribuci\'on $G$, entonces el tiempo $T_{n}$ de la $n$-\'esima renovaci\'on tiene distribuci\'on $G\star F^{\left(n-1\right)\star}\left(t\right)$
\end{Note}


\begin{Teo}
Para una constante $\mu\leq\infty$ ( o variable aleatoria), las siguientes expresiones son equivalentes:

\begin{eqnarray}
lim_{n\rightarrow\infty}n^{-1}T_{n}&=&\mu,\textrm{ c.s.}\\
lim_{t\rightarrow\infty}t^{-1}N\left(t\right)&=&1/\mu,\textrm{ c.s.}
\end{eqnarray}
\end{Teo}


Es decir, $T_{n}$ satisface la Ley Fuerte de los Grandes N\'umeros s\'i y s\'olo s\'i $N\left/t\right)$ la cumple.


\begin{Coro}[Ley Fuerte de los Grandes N\'umeros para Procesos de Renovaci\'on]
Si $N\left(t\right)$ es un proceso de renovaci\'on cuyos tiempos de inter-renovaci\'on tienen media $\mu\leq\infty$, entonces
\begin{eqnarray}
t^{-1}N\left(t\right)\rightarrow 1/\mu,\textrm{ c.s. cuando }t\rightarrow\infty.
\end{eqnarray}

\end{Coro}


Considerar el proceso estoc\'astico de valores reales $\left\{Z\left(t\right):t\geq0\right\}$ en el mismo espacio de probabilidad que $N\left(t\right)$

\begin{Def}
Para el proceso $\left\{Z\left(t\right):t\geq0\right\}$ se define la fluctuaci\'on m\'axima de $Z\left(t\right)$ en el intervalo $\left(T_{n-1},T_{n}\right]$:
\begin{eqnarray*}
M_{n}=\sup_{T_{n-1}<t\leq T_{n}}|Z\left(t\right)-Z\left(T_{n-1}\right)|
\end{eqnarray*}
\end{Def}

\begin{Teo}
Sup\'ongase que $n^{-1}T_{n}\rightarrow\mu$ c.s. cuando $n\rightarrow\infty$, donde $\mu\leq\infty$ es una constante o variable aleatoria. Sea $a$ una constante o variable aleatoria que puede ser infinita cuando $\mu$ es finita, y considere las expresiones l\'imite:
\begin{eqnarray}
lim_{n\rightarrow\infty}n^{-1}Z\left(T_{n}\right)&=&a,\textrm{ c.s.}\\
lim_{t\rightarrow\infty}t^{-1}Z\left(t\right)&=&a/\mu,\textrm{ c.s.}
\end{eqnarray}
La segunda expresi\'on implica la primera. Conversamente, la primera implica la segunda si el proceso $Z\left(t\right)$ es creciente, o si $lim_{n\rightarrow\infty}n^{-1}M_{n}=0$ c.s.
\end{Teo}

\begin{Coro}
Si $N\left(t\right)$ es un proceso de renovaci\'on, y $\left(Z\left(T_{n}\right)-Z\left(T_{n-1}\right),M_{n}\right)$, para $n\geq1$, son variables aleatorias independientes e id\'enticamente distribuidas con media finita, entonces,
\begin{eqnarray}
lim_{t\rightarrow\infty}t^{-1}Z\left(t\right)\rightarrow\frac{\esp\left[Z\left(T_{1}\right)-Z\left(T_{0}\right)\right]}{\esp\left[T_{1}\right]},\textrm{ c.s. cuando  }t\rightarrow\infty.
\end{eqnarray}
\end{Coro}


Sup\'ongase que $N\left(t\right)$ es un proceso de renovaci\'on con distribuci\'on $F$ con media finita $\mu$.

\begin{Def}
La funci\'on de renovaci\'on asociada con la distribuci\'on $F$, del proceso $N\left(t\right)$, es
\begin{eqnarray*}
U\left(t\right)=\sum_{n=1}^{\infty}F^{n\star}\left(t\right),\textrm{   }t\geq0,
\end{eqnarray*}
donde $F^{0\star}\left(t\right)=\indora\left(t\geq0\right)$.
\end{Def}


\begin{Prop}
Sup\'ongase que la distribuci\'on de inter-renovaci\'on $F$ tiene densidad $f$. Entonces $U\left(t\right)$ tambi\'en tiene densidad, para $t>0$, y es $U^{'}\left(t\right)=\sum_{n=0}^{\infty}f^{n\star}\left(t\right)$. Adem\'as
\begin{eqnarray*}
\prob\left\{N\left(t\right)>N\left(t-\right)\right\}=0\textrm{,   }t\geq0.
\end{eqnarray*}
\end{Prop}

\begin{Def}
La Transformada de Laplace-Stieljes de $F$ est\'a dada por

\begin{eqnarray*}
\hat{F}\left(\alpha\right)=\int_{\rea_{+}}e^{-\alpha t}dF\left(t\right)\textrm{,  }\alpha\geq0.
\end{eqnarray*}
\end{Def}

Entonces

\begin{eqnarray*}
\hat{U}\left(\alpha\right)=\sum_{n=0}^{\infty}\hat{F^{n\star}}\left(\alpha\right)=\sum_{n=0}^{\infty}\hat{F}\left(\alpha\right)^{n}=\frac{1}{1-\hat{F}\left(\alpha\right)}.
\end{eqnarray*}


\begin{Prop}
La Transformada de Laplace $\hat{U}\left(\alpha\right)$ y $\hat{F}\left(\alpha\right)$ determina una a la otra de manera \'unica por la relaci\'on $\hat{U}\left(\alpha\right)=\frac{1}{1-\hat{F}\left(\alpha\right)}$.
\end{Prop}


\begin{Note}
Un proceso de renovaci\'on $N\left(t\right)$ cuyos tiempos de inter-renovaci\'on tienen media finita, es un proceso Poisson con tasa $\lambda$ si y s\'olo s\'i $\esp\left[U\left(t\right)\right]=\lambda t$, para $t\geq0$.
\end{Note}


\begin{Teo}
Sea $N\left(t\right)$ un proceso puntual simple con puntos de localizaci\'on $T_{n}$ tal que $\eta\left(t\right)=\esp\left[N\left(\right)\right]$ es finita para cada $t$. Entonces para cualquier funci\'on $f:\rea_{+}\rightarrow\rea$,
\begin{eqnarray*}
\esp\left[\sum_{n=1}^{N\left(\right)}f\left(T_{n}\right)\right]=\int_{\left(0,t\right]}f\left(s\right)d\eta\left(s\right)\textrm{,  }t\geq0,
\end{eqnarray*}
suponiendo que la integral exista. Adem\'as si $X_{1},X_{2},\ldots$ son variables aleatorias definidas en el mismo espacio de probabilidad que el proceso $N\left(t\right)$ tal que $\esp\left[X_{n}|T_{n}=s\right]=f\left(s\right)$, independiente de $n$. Entonces
\begin{eqnarray*}
\esp\left[\sum_{n=1}^{N\left(t\right)}X_{n}\right]=\int_{\left(0,t\right]}f\left(s\right)d\eta\left(s\right)\textrm{,  }t\geq0,
\end{eqnarray*} 
suponiendo que la integral exista. 
\end{Teo}

\begin{Coro}[Identidad de Wald para Renovaciones]
Para el proceso de renovaci\'on $N\left(t\right)$,
\begin{eqnarray*}
\esp\left[T_{N\left(t\right)+1}\right]=\mu\esp\left[N\left(t\right)+1\right]\textrm{,  }t\geq0,
\end{eqnarray*}  
\end{Coro}


\begin{Def}
Sea $h\left(t\right)$ funci\'on de valores reales en $\rea$ acotada en intervalos finitos e igual a cero para $t<0$ La ecuaci\'on de renovaci\'on para $h\left(t\right)$ y la distribuci\'on $F$ es

\begin{eqnarray}%\label{Ec.Renovacion}
H\left(t\right)=h\left(t\right)+\int_{\left[0,t\right]}H\left(t-s\right)dF\left(s\right)\textrm{,    }t\geq0,
\end{eqnarray}
donde $H\left(t\right)$ es una funci\'on de valores reales. Esto es $H=h+F\star H$. Decimos que $H\left(t\right)$ es soluci\'on de esta ecuaci\'on si satisface la ecuaci\'on, y es acotada en intervalos finitos e iguales a cero para $t<0$.
\end{Def}

\begin{Prop}
La funci\'on $U\star h\left(t\right)$ es la \'unica soluci\'on de la ecuaci\'on de renovaci\'on (\ref{Ec.Renovacion}).
\end{Prop}

\begin{Teo}[Teorema Renovaci\'on Elemental]
\begin{eqnarray*}
t^{-1}U\left(t\right)\rightarrow 1/\mu\textrm{,    cuando }t\rightarrow\infty.
\end{eqnarray*}
\end{Teo}



Sup\'ongase que $N\left(t\right)$ es un proceso de renovaci\'on con distribuci\'on $F$ con media finita $\mu$.

\begin{Def}
La funci\'on de renovaci\'on asociada con la distribuci\'on $F$, del proceso $N\left(t\right)$, es
\begin{eqnarray*}
U\left(t\right)=\sum_{n=1}^{\infty}F^{n\star}\left(t\right),\textrm{   }t\geq0,
\end{eqnarray*}
donde $F^{0\star}\left(t\right)=\indora\left(t\geq0\right)$.
\end{Def}


\begin{Prop}
Sup\'ongase que la distribuci\'on de inter-renovaci\'on $F$ tiene densidad $f$. Entonces $U\left(t\right)$ tambi\'en tiene densidad, para $t>0$, y es $U^{'}\left(t\right)=\sum_{n=0}^{\infty}f^{n\star}\left(t\right)$. Adem\'as
\begin{eqnarray*}
\prob\left\{N\left(t\right)>N\left(t-\right)\right\}=0\textrm{,   }t\geq0.
\end{eqnarray*}
\end{Prop}

\begin{Def}
La Transformada de Laplace-Stieljes de $F$ est\'a dada por

\begin{eqnarray*}
\hat{F}\left(\alpha\right)=\int_{\rea_{+}}e^{-\alpha t}dF\left(t\right)\textrm{,  }\alpha\geq0.
\end{eqnarray*}
\end{Def}

Entonces

\begin{eqnarray*}
\hat{U}\left(\alpha\right)=\sum_{n=0}^{\infty}\hat{F^{n\star}}\left(\alpha\right)=\sum_{n=0}^{\infty}\hat{F}\left(\alpha\right)^{n}=\frac{1}{1-\hat{F}\left(\alpha\right)}.
\end{eqnarray*}


\begin{Prop}
La Transformada de Laplace $\hat{U}\left(\alpha\right)$ y $\hat{F}\left(\alpha\right)$ determina una a la otra de manera \'unica por la relaci\'on $\hat{U}\left(\alpha\right)=\frac{1}{1-\hat{F}\left(\alpha\right)}$.
\end{Prop}


\begin{Note}
Un proceso de renovaci\'on $N\left(t\right)$ cuyos tiempos de inter-renovaci\'on tienen media finita, es un proceso Poisson con tasa $\lambda$ si y s\'olo s\'i $\esp\left[U\left(t\right)\right]=\lambda t$, para $t\geq0$.
\end{Note}


\begin{Teo}
Sea $N\left(t\right)$ un proceso puntual simple con puntos de localizaci\'on $T_{n}$ tal que $\eta\left(t\right)=\esp\left[N\left(\right)\right]$ es finita para cada $t$. Entonces para cualquier funci\'on $f:\rea_{+}\rightarrow\rea$,
\begin{eqnarray*}
\esp\left[\sum_{n=1}^{N\left(\right)}f\left(T_{n}\right)\right]=\int_{\left(0,t\right]}f\left(s\right)d\eta\left(s\right)\textrm{,  }t\geq0,
\end{eqnarray*}
suponiendo que la integral exista. Adem\'as si $X_{1},X_{2},\ldots$ son variables aleatorias definidas en el mismo espacio de probabilidad que el proceso $N\left(t\right)$ tal que $\esp\left[X_{n}|T_{n}=s\right]=f\left(s\right)$, independiente de $n$. Entonces
\begin{eqnarray*}
\esp\left[\sum_{n=1}^{N\left(t\right)}X_{n}\right]=\int_{\left(0,t\right]}f\left(s\right)d\eta\left(s\right)\textrm{,  }t\geq0,
\end{eqnarray*} 
suponiendo que la integral exista. 
\end{Teo}

\begin{Coro}[Identidad de Wald para Renovaciones]
Para el proceso de renovaci\'on $N\left(t\right)$,
\begin{eqnarray*}
\esp\left[T_{N\left(t\right)+1}\right]=\mu\esp\left[N\left(t\right)+1\right]\textrm{,  }t\geq0,
\end{eqnarray*}  
\end{Coro}


\begin{Def}
Sea $h\left(t\right)$ funci\'on de valores reales en $\rea$ acotada en intervalos finitos e igual a cero para $t<0$ La ecuaci\'on de renovaci\'on para $h\left(t\right)$ y la distribuci\'on $F$ es

\begin{eqnarray}%\label{Ec.Renovacion}
H\left(t\right)=h\left(t\right)+\int_{\left[0,t\right]}H\left(t-s\right)dF\left(s\right)\textrm{,    }t\geq0,
\end{eqnarray}
donde $H\left(t\right)$ es una funci\'on de valores reales. Esto es $H=h+F\star H$. Decimos que $H\left(t\right)$ es soluci\'on de esta ecuaci\'on si satisface la ecuaci\'on, y es acotada en intervalos finitos e iguales a cero para $t<0$.
\end{Def}

\begin{Prop}
La funci\'on $U\star h\left(t\right)$ es la \'unica soluci\'on de la ecuaci\'on de renovaci\'on (\ref{Ec.Renovacion}).
\end{Prop}

\begin{Teo}[Teorema Renovaci\'on Elemental]
\begin{eqnarray*}
t^{-1}U\left(t\right)\rightarrow 1/\mu\textrm{,    cuando }t\rightarrow\infty.
\end{eqnarray*}
\end{Teo}


\begin{Note} Una funci\'on $h:\rea_{+}\rightarrow\rea$ es Directamente Riemann Integrable en los siguientes casos:
\begin{itemize}
\item[a)] $h\left(t\right)\geq0$ es decreciente y Riemann Integrable.
\item[b)] $h$ es continua excepto posiblemente en un conjunto de Lebesgue de medida 0, y $|h\left(t\right)|\leq b\left(t\right)$, donde $b$ es DRI.
\end{itemize}
\end{Note}

\begin{Teo}[Teorema Principal de Renovaci\'on]
Si $F$ es no aritm\'etica y $h\left(t\right)$ es Directamente Riemann Integrable (DRI), entonces

\begin{eqnarray*}
lim_{t\rightarrow\infty}U\star h=\frac{1}{\mu}\int_{\rea_{+}}h\left(s\right)ds.
\end{eqnarray*}
\end{Teo}

\begin{Prop}
Cualquier funci\'on $H\left(t\right)$ acotada en intervalos finitos y que es 0 para $t<0$ puede expresarse como
\begin{eqnarray*}
H\left(t\right)=U\star h\left(t\right)\textrm{,  donde }h\left(t\right)=H\left(t\right)-F\star H\left(t\right)
\end{eqnarray*}
\end{Prop}

\begin{Def}
Un proceso estoc\'astico $X\left(t\right)$ es crudamente regenerativo en un tiempo aleatorio positivo $T$ si
\begin{eqnarray*}
\esp\left[X\left(T+t\right)|T\right]=\esp\left[X\left(t\right)\right]\textrm{, para }t\geq0,\end{eqnarray*}
y con las esperanzas anteriores finitas.
\end{Def}

\begin{Prop}
Sup\'ongase que $X\left(t\right)$ es un proceso crudamente regenerativo en $T$, que tiene distribuci\'on $F$. Si $\esp\left[X\left(t\right)\right]$ es acotado en intervalos finitos, entonces
\begin{eqnarray*}
\esp\left[X\left(t\right)\right]=U\star h\left(t\right)\textrm{,  donde }h\left(t\right)=\esp\left[X\left(t\right)\indora\left(T>t\right)\right].
\end{eqnarray*}
\end{Prop}

\begin{Teo}[Regeneraci\'on Cruda]
Sup\'ongase que $X\left(t\right)$ es un proceso con valores positivo crudamente regenerativo en $T$, y def\'inase $M=\sup\left\{|X\left(t\right)|:t\leq T\right\}$. Si $T$ es no aritm\'etico y $M$ y $MT$ tienen media finita, entonces
\begin{eqnarray*}
lim_{t\rightarrow\infty}\esp\left[X\left(t\right)\right]=\frac{1}{\mu}\int_{\rea_{+}}h\left(s\right)ds,
\end{eqnarray*}
donde $h\left(t\right)=\esp\left[X\left(t\right)\indora\left(T>t\right)\right]$.
\end{Teo}


\begin{Note} Una funci\'on $h:\rea_{+}\rightarrow\rea$ es Directamente Riemann Integrable en los siguientes casos:
\begin{itemize}
\item[a)] $h\left(t\right)\geq0$ es decreciente y Riemann Integrable.
\item[b)] $h$ es continua excepto posiblemente en un conjunto de Lebesgue de medida 0, y $|h\left(t\right)|\leq b\left(t\right)$, donde $b$ es DRI.
\end{itemize}
\end{Note}

\begin{Teo}[Teorema Principal de Renovaci\'on]
Si $F$ es no aritm\'etica y $h\left(t\right)$ es Directamente Riemann Integrable (DRI), entonces

\begin{eqnarray*}
lim_{t\rightarrow\infty}U\star h=\frac{1}{\mu}\int_{\rea_{+}}h\left(s\right)ds.
\end{eqnarray*}
\end{Teo}

\begin{Prop}
Cualquier funci\'on $H\left(t\right)$ acotada en intervalos finitos y que es 0 para $t<0$ puede expresarse como
\begin{eqnarray*}
H\left(t\right)=U\star h\left(t\right)\textrm{,  donde }h\left(t\right)=H\left(t\right)-F\star H\left(t\right)
\end{eqnarray*}
\end{Prop}

\begin{Def}
Un proceso estoc\'astico $X\left(t\right)$ es crudamente regenerativo en un tiempo aleatorio positivo $T$ si
\begin{eqnarray*}
\esp\left[X\left(T+t\right)|T\right]=\esp\left[X\left(t\right)\right]\textrm{, para }t\geq0,\end{eqnarray*}
y con las esperanzas anteriores finitas.
\end{Def}

\begin{Prop}
Sup\'ongase que $X\left(t\right)$ es un proceso crudamente regenerativo en $T$, que tiene distribuci\'on $F$. Si $\esp\left[X\left(t\right)\right]$ es acotado en intervalos finitos, entonces
\begin{eqnarray*}
\esp\left[X\left(t\right)\right]=U\star h\left(t\right)\textrm{,  donde }h\left(t\right)=\esp\left[X\left(t\right)\indora\left(T>t\right)\right].
\end{eqnarray*}
\end{Prop}

\begin{Teo}[Regeneraci\'on Cruda]
Sup\'ongase que $X\left(t\right)$ es un proceso con valores positivo crudamente regenerativo en $T$, y def\'inase $M=\sup\left\{|X\left(t\right)|:t\leq T\right\}$. Si $T$ es no aritm\'etico y $M$ y $MT$ tienen media finita, entonces
\begin{eqnarray*}
lim_{t\rightarrow\infty}\esp\left[X\left(t\right)\right]=\frac{1}{\mu}\int_{\rea_{+}}h\left(s\right)ds,
\end{eqnarray*}
donde $h\left(t\right)=\esp\left[X\left(t\right)\indora\left(T>t\right)\right]$.
\end{Teo}

\begin{Def}
Para el proceso $\left\{\left(N\left(t\right),X\left(t\right)\right):t\geq0\right\}$, sus trayectoria muestrales en el intervalo de tiempo $\left[T_{n-1},T_{n}\right)$ est\'an descritas por
\begin{eqnarray*}
\zeta_{n}=\left(\xi_{n},\left\{X\left(T_{n-1}+t\right):0\leq t<\xi_{n}\right\}\right)
\end{eqnarray*}
Este $\zeta_{n}$ es el $n$-\'esimo segmento del proceso. El proceso es regenerativo sobre los tiempos $T_{n}$ si sus segmentos $\zeta_{n}$ son independientes e id\'enticamennte distribuidos.
\end{Def}


\begin{Note}
Si $\tilde{X}\left(t\right)$ con espacio de estados $\tilde{S}$ es regenerativo sobre $T_{n}$, entonces $X\left(t\right)=f\left(\tilde{X}\left(t\right)\right)$ tambi\'en es regenerativo sobre $T_{n}$, para cualquier funci\'on $f:\tilde{S}\rightarrow S$.
\end{Note}

\begin{Note}
Los procesos regenerativos son crudamente regenerativos, pero no al rev\'es.
\end{Note}


\begin{Note}
Un proceso estoc\'astico a tiempo continuo o discreto es regenerativo si existe un proceso de renovaci\'on  tal que los segmentos del proceso entre tiempos de renovaci\'on sucesivos son i.i.d., es decir, para $\left\{X\left(t\right):t\geq0\right\}$ proceso estoc\'astico a tiempo continuo con espacio de estados $S$, espacio m\'etrico.
\end{Note}

Para $\left\{X\left(t\right):t\geq0\right\}$ Proceso Estoc\'astico a tiempo continuo con estado de espacios $S$, que es un espacio m\'etrico, con trayectorias continuas por la derecha y con l\'imites por la izquierda c.s. Sea $N\left(t\right)$ un proceso de renovaci\'on en $\rea_{+}$ definido en el mismo espacio de probabilidad que $X\left(t\right)$, con tiempos de renovaci\'on $T$ y tiempos de inter-renovaci\'on $\xi_{n}=T_{n}-T_{n-1}$, con misma distribuci\'on $F$ de media finita $\mu$.



\begin{Def}
Para el proceso $\left\{\left(N\left(t\right),X\left(t\right)\right):t\geq0\right\}$, sus trayectoria muestrales en el intervalo de tiempo $\left[T_{n-1},T_{n}\right)$ est\'an descritas por
\begin{eqnarray*}
\zeta_{n}=\left(\xi_{n},\left\{X\left(T_{n-1}+t\right):0\leq t<\xi_{n}\right\}\right)
\end{eqnarray*}
Este $\zeta_{n}$ es el $n$-\'esimo segmento del proceso. El proceso es regenerativo sobre los tiempos $T_{n}$ si sus segmentos $\zeta_{n}$ son independientes e id\'enticamennte distribuidos.
\end{Def}

\begin{Note}
Un proceso regenerativo con media de la longitud de ciclo finita es llamado positivo recurrente.
\end{Note}

\begin{Teo}[Procesos Regenerativos]
Suponga que el proceso
\end{Teo}


\begin{Def}[Renewal Process Trinity]
Para un proceso de renovaci\'on $N\left(t\right)$, los siguientes procesos proveen de informaci\'on sobre los tiempos de renovaci\'on.
\begin{itemize}
\item $A\left(t\right)=t-T_{N\left(t\right)}$, el tiempo de recurrencia hacia atr\'as al tiempo $t$, que es el tiempo desde la \'ultima renovaci\'on para $t$.

\item $B\left(t\right)=T_{N\left(t\right)+1}-t$, el tiempo de recurrencia hacia adelante al tiempo $t$, residual del tiempo de renovaci\'on, que es el tiempo para la pr\'oxima renovaci\'on despu\'es de $t$.

\item $L\left(t\right)=\xi_{N\left(t\right)+1}=A\left(t\right)+B\left(t\right)$, la longitud del intervalo de renovaci\'on que contiene a $t$.
\end{itemize}
\end{Def}

\begin{Note}
El proceso tridimensional $\left(A\left(t\right),B\left(t\right),L\left(t\right)\right)$ es regenerativo sobre $T_{n}$, y por ende cada proceso lo es. Cada proceso $A\left(t\right)$ y $B\left(t\right)$ son procesos de MArkov a tiempo continuo con trayectorias continuas por partes en el espacio de estados $\rea_{+}$. Una expresi\'on conveniente para su distribuci\'on conjunta es, para $0\leq x<t,y\geq0$
\begin{equation}\label{NoRenovacion}
P\left\{A\left(t\right)>x,B\left(t\right)>y\right\}=
P\left\{N\left(t+y\right)-N\left((t-x)\right)=0\right\}
\end{equation}
\end{Note}

\begin{Ejem}[Tiempos de recurrencia Poisson]
Si $N\left(t\right)$ es un proceso Poisson con tasa $\lambda$, entonces de la expresi\'on (\ref{NoRenovacion}) se tiene que

\begin{eqnarray*}
\begin{array}{lc}
P\left\{A\left(t\right)>x,B\left(t\right)>y\right\}=e^{-\lambda\left(x+y\right)},&0\leq x<t,y\geq0,
\end{array}
\end{eqnarray*}
que es la probabilidad Poisson de no renovaciones en un intervalo de longitud $x+y$.

\end{Ejem}

\begin{Note}
Una cadena de Markov erg\'odica tiene la propiedad de ser estacionaria si la distribuci\'on de su estado al tiempo $0$ es su distribuci\'on estacionaria.
\end{Note}


\begin{Def}
Un proceso estoc\'astico a tiempo continuo $\left\{X\left(t\right):t\geq0\right\}$ en un espacio general es estacionario si sus distribuciones finito dimensionales son invariantes bajo cualquier  traslado: para cada $0\leq s_{1}<s_{2}<\cdots<s_{k}$ y $t\geq0$,
\begin{eqnarray*}
\left(X\left(s_{1}+t\right),\ldots,X\left(s_{k}+t\right)\right)=_{d}\left(X\left(s_{1}\right),\ldots,X\left(s_{k}\right)\right).
\end{eqnarray*}
\end{Def}

\begin{Note}
Un proceso de Markov es estacionario si $X\left(t\right)=_{d}X\left(0\right)$, $t\geq0$.
\end{Note}

Considerese el proceso $N\left(t\right)=\sum_{n}\indora\left(\tau_{n}\leq t\right)$ en $\rea_{+}$, con puntos $0<\tau_{1}<\tau_{2}<\cdots$.

\begin{Prop}
Si $N$ es un proceso puntual estacionario y $\esp\left[N\left(1\right)\right]<\infty$, entonces $\esp\left[N\left(t\right)\right]=t\esp\left[N\left(1\right)\right]$, $t\geq0$

\end{Prop}

\begin{Teo}
Los siguientes enunciados son equivalentes
\begin{itemize}
\item[i)] El proceso retardado de renovaci\'on $N$ es estacionario.

\item[ii)] EL proceso de tiempos de recurrencia hacia adelante $B\left(t\right)$ es estacionario.


\item[iii)] $\esp\left[N\left(t\right)\right]=t/\mu$,


\item[iv)] $G\left(t\right)=F_{e}\left(t\right)=\frac{1}{\mu}\int_{0}^{t}\left[1-F\left(s\right)\right]ds$
\end{itemize}
Cuando estos enunciados son ciertos, $P\left\{B\left(t\right)\leq x\right\}=F_{e}\left(x\right)$, para $t,x\geq0$.

\end{Teo}

\begin{Note}
Una consecuencia del teorema anterior es que el Proceso Poisson es el \'unico proceso sin retardo que es estacionario.
\end{Note}

\begin{Coro}
El proceso de renovaci\'on $N\left(t\right)$ sin retardo, y cuyos tiempos de inter renonaci\'on tienen media finita, es estacionario si y s\'olo si es un proceso Poisson.

\end{Coro}


%________________________________________________________________________
%\subsection{Procesos Regenerativos}
%________________________________________________________________________



\begin{Note}
Si $\tilde{X}\left(t\right)$ con espacio de estados $\tilde{S}$ es regenerativo sobre $T_{n}$, entonces $X\left(t\right)=f\left(\tilde{X}\left(t\right)\right)$ tambi\'en es regenerativo sobre $T_{n}$, para cualquier funci\'on $f:\tilde{S}\rightarrow S$.
\end{Note}

\begin{Note}
Los procesos regenerativos son crudamente regenerativos, pero no al rev\'es.
\end{Note}
%\subsection*{Procesos Regenerativos: Sigman\cite{Sigman1}}
\begin{Def}[Definici\'on Cl\'asica]
Un proceso estoc\'astico $X=\left\{X\left(t\right):t\geq0\right\}$ es llamado regenerativo is existe una variable aleatoria $R_{1}>0$ tal que
\begin{itemize}
\item[i)] $\left\{X\left(t+R_{1}\right):t\geq0\right\}$ es independiente de $\left\{\left\{X\left(t\right):t<R_{1}\right\},\right\}$
\item[ii)] $\left\{X\left(t+R_{1}\right):t\geq0\right\}$ es estoc\'asticamente equivalente a $\left\{X\left(t\right):t>0\right\}$
\end{itemize}

Llamamos a $R_{1}$ tiempo de regeneraci\'on, y decimos que $X$ se regenera en este punto.
\end{Def}

$\left\{X\left(t+R_{1}\right)\right\}$ es regenerativo con tiempo de regeneraci\'on $R_{2}$, independiente de $R_{1}$ pero con la misma distribuci\'on que $R_{1}$. Procediendo de esta manera se obtiene una secuencia de variables aleatorias independientes e id\'enticamente distribuidas $\left\{R_{n}\right\}$ llamados longitudes de ciclo. Si definimos a $Z_{k}\equiv R_{1}+R_{2}+\cdots+R_{k}$, se tiene un proceso de renovaci\'on llamado proceso de renovaci\'on encajado para $X$.




\begin{Def}
Para $x$ fijo y para cada $t\geq0$, sea $I_{x}\left(t\right)=1$ si $X\left(t\right)\leq x$,  $I_{x}\left(t\right)=0$ en caso contrario, y def\'inanse los tiempos promedio
\begin{eqnarray*}
\overline{X}&=&lim_{t\rightarrow\infty}\frac{1}{t}\int_{0}^{\infty}X\left(u\right)du\\
\prob\left(X_{\infty}\leq x\right)&=&lim_{t\rightarrow\infty}\frac{1}{t}\int_{0}^{\infty}I_{x}\left(u\right)du,
\end{eqnarray*}
cuando estos l\'imites existan.
\end{Def}

Como consecuencia del teorema de Renovaci\'on-Recompensa, se tiene que el primer l\'imite  existe y es igual a la constante
\begin{eqnarray*}
\overline{X}&=&\frac{\esp\left[\int_{0}^{R_{1}}X\left(t\right)dt\right]}{\esp\left[R_{1}\right]},
\end{eqnarray*}
suponiendo que ambas esperanzas son finitas.

\begin{Note}
\begin{itemize}
\item[a)] Si el proceso regenerativo $X$ es positivo recurrente y tiene trayectorias muestrales no negativas, entonces la ecuaci\'on anterior es v\'alida.
\item[b)] Si $X$ es positivo recurrente regenerativo, podemos construir una \'unica versi\'on estacionaria de este proceso, $X_{e}=\left\{X_{e}\left(t\right)\right\}$, donde $X_{e}$ es un proceso estoc\'astico regenerativo y estrictamente estacionario, con distribuci\'on marginal distribuida como $X_{\infty}$
\end{itemize}
\end{Note}

%________________________________________________________________________
%\subsection{Procesos Regenerativos}
%________________________________________________________________________

Para $\left\{X\left(t\right):t\geq0\right\}$ Proceso Estoc\'astico a tiempo continuo con estado de espacios $S$, que es un espacio m\'etrico, con trayectorias continuas por la derecha y con l\'imites por la izquierda c.s. Sea $N\left(t\right)$ un proceso de renovaci\'on en $\rea_{+}$ definido en el mismo espacio de probabilidad que $X\left(t\right)$, con tiempos de renovaci\'on $T$ y tiempos de inter-renovaci\'on $\xi_{n}=T_{n}-T_{n-1}$, con misma distribuci\'on $F$ de media finita $\mu$.



\begin{Def}
Para el proceso $\left\{\left(N\left(t\right),X\left(t\right)\right):t\geq0\right\}$, sus trayectoria muestrales en el intervalo de tiempo $\left[T_{n-1},T_{n}\right)$ est\'an descritas por
\begin{eqnarray*}
\zeta_{n}=\left(\xi_{n},\left\{X\left(T_{n-1}+t\right):0\leq t<\xi_{n}\right\}\right)
\end{eqnarray*}
Este $\zeta_{n}$ es el $n$-\'esimo segmento del proceso. El proceso es regenerativo sobre los tiempos $T_{n}$ si sus segmentos $\zeta_{n}$ son independientes e id\'enticamennte distribuidos.
\end{Def}


\begin{Note}
Si $\tilde{X}\left(t\right)$ con espacio de estados $\tilde{S}$ es regenerativo sobre $T_{n}$, entonces $X\left(t\right)=f\left(\tilde{X}\left(t\right)\right)$ tambi\'en es regenerativo sobre $T_{n}$, para cualquier funci\'on $f:\tilde{S}\rightarrow S$.
\end{Note}

\begin{Note}
Los procesos regenerativos son crudamente regenerativos, pero no al rev\'es.
\end{Note}

\begin{Def}[Definici\'on Cl\'asica]
Un proceso estoc\'astico $X=\left\{X\left(t\right):t\geq0\right\}$ es llamado regenerativo is existe una variable aleatoria $R_{1}>0$ tal que
\begin{itemize}
\item[i)] $\left\{X\left(t+R_{1}\right):t\geq0\right\}$ es independiente de $\left\{\left\{X\left(t\right):t<R_{1}\right\},\right\}$
\item[ii)] $\left\{X\left(t+R_{1}\right):t\geq0\right\}$ es estoc\'asticamente equivalente a $\left\{X\left(t\right):t>0\right\}$
\end{itemize}

Llamamos a $R_{1}$ tiempo de regeneraci\'on, y decimos que $X$ se regenera en este punto.
\end{Def}

$\left\{X\left(t+R_{1}\right)\right\}$ es regenerativo con tiempo de regeneraci\'on $R_{2}$, independiente de $R_{1}$ pero con la misma distribuci\'on que $R_{1}$. Procediendo de esta manera se obtiene una secuencia de variables aleatorias independientes e id\'enticamente distribuidas $\left\{R_{n}\right\}$ llamados longitudes de ciclo. Si definimos a $Z_{k}\equiv R_{1}+R_{2}+\cdots+R_{k}$, se tiene un proceso de renovaci\'on llamado proceso de renovaci\'on encajado para $X$.

\begin{Note}
Un proceso regenerativo con media de la longitud de ciclo finita es llamado positivo recurrente.
\end{Note}


\begin{Def}
Para $x$ fijo y para cada $t\geq0$, sea $I_{x}\left(t\right)=1$ si $X\left(t\right)\leq x$,  $I_{x}\left(t\right)=0$ en caso contrario, y def\'inanse los tiempos promedio
\begin{eqnarray*}
\overline{X}&=&lim_{t\rightarrow\infty}\frac{1}{t}\int_{0}^{\infty}X\left(u\right)du\\
\prob\left(X_{\infty}\leq x\right)&=&lim_{t\rightarrow\infty}\frac{1}{t}\int_{0}^{\infty}I_{x}\left(u\right)du,
\end{eqnarray*}
cuando estos l\'imites existan.
\end{Def}

Como consecuencia del teorema de Renovaci\'on-Recompensa, se tiene que el primer l\'imite  existe y es igual a la constante
\begin{eqnarray*}
\overline{X}&=&\frac{\esp\left[\int_{0}^{R_{1}}X\left(t\right)dt\right]}{\esp\left[R_{1}\right]},
\end{eqnarray*}
suponiendo que ambas esperanzas son finitas.

\begin{Note}
\begin{itemize}
\item[a)] Si el proceso regenerativo $X$ es positivo recurrente y tiene trayectorias muestrales no negativas, entonces la ecuaci\'on anterior es v\'alida.
\item[b)] Si $X$ es positivo recurrente regenerativo, podemos construir una \'unica versi\'on estacionaria de este proceso, $X_{e}=\left\{X_{e}\left(t\right)\right\}$, donde $X_{e}$ es un proceso estoc\'astico regenerativo y estrictamente estacionario, con distribuci\'on marginal distribuida como $X_{\infty}$
\end{itemize}
\end{Note}

%__________________________________________________________________________________________
%\subsection{Procesos Regenerativos Estacionarios - Stidham \cite{Stidham}}
%__________________________________________________________________________________________


Un proceso estoc\'astico a tiempo continuo $\left\{V\left(t\right),t\geq0\right\}$ es un proceso regenerativo si existe una sucesi\'on de variables aleatorias independientes e id\'enticamente distribuidas $\left\{X_{1},X_{2},\ldots\right\}$, sucesi\'on de renovaci\'on, tal que para cualquier conjunto de Borel $A$, 

\begin{eqnarray*}
\prob\left\{V\left(t\right)\in A|X_{1}+X_{2}+\cdots+X_{R\left(t\right)}=s,\left\{V\left(\tau\right),\tau<s\right\}\right\}=\prob\left\{V\left(t-s\right)\in A|X_{1}>t-s\right\},
\end{eqnarray*}
para todo $0\leq s\leq t$, donde $R\left(t\right)=\max\left\{X_{1}+X_{2}+\cdots+X_{j}\leq t\right\}=$n\'umero de renovaciones ({\emph{puntos de regeneraci\'on}}) que ocurren en $\left[0,t\right]$. El intervalo $\left[0,X_{1}\right)$ es llamado {\emph{primer ciclo de regeneraci\'on}} de $\left\{V\left(t \right),t\geq0\right\}$, $\left[X_{1},X_{1}+X_{2}\right)$ el {\emph{segundo ciclo de regeneraci\'on}}, y as\'i sucesivamente.

Sea $X=X_{1}$ y sea $F$ la funci\'on de distrbuci\'on de $X$


\begin{Def}
Se define el proceso estacionario, $\left\{V^{*}\left(t\right),t\geq0\right\}$, para $\left\{V\left(t\right),t\geq0\right\}$ por

\begin{eqnarray*}
\prob\left\{V\left(t\right)\in A\right\}=\frac{1}{\esp\left[X\right]}\int_{0}^{\infty}\prob\left\{V\left(t+x\right)\in A|X>x\right\}\left(1-F\left(x\right)\right)dx,
\end{eqnarray*} 
para todo $t\geq0$ y todo conjunto de Borel $A$.
\end{Def}

\begin{Def}
Una distribuci\'on se dice que es {\emph{aritm\'etica}} si todos sus puntos de incremento son m\'ultiplos de la forma $0,\lambda, 2\lambda,\ldots$ para alguna $\lambda>0$ entera.
\end{Def}


\begin{Def}
Una modificaci\'on medible de un proceso $\left\{V\left(t\right),t\geq0\right\}$, es una versi\'on de este, $\left\{V\left(t,w\right)\right\}$ conjuntamente medible para $t\geq0$ y para $w\in S$, $S$ espacio de estados para $\left\{V\left(t\right),t\geq0\right\}$.
\end{Def}

\begin{Teo}
Sea $\left\{V\left(t\right),t\geq\right\}$ un proceso regenerativo no negativo con modificaci\'on medible. Sea $\esp\left[X\right]<\infty$. Entonces el proceso estacionario dado por la ecuaci\'on anterior est\'a bien definido y tiene funci\'on de distribuci\'on independiente de $t$, adem\'as
\begin{itemize}
\item[i)] \begin{eqnarray*}
\esp\left[V^{*}\left(0\right)\right]&=&\frac{\esp\left[\int_{0}^{X}V\left(s\right)ds\right]}{\esp\left[X\right]}\end{eqnarray*}
\item[ii)] Si $\esp\left[V^{*}\left(0\right)\right]<\infty$, equivalentemente, si $\esp\left[\int_{0}^{X}V\left(s\right)ds\right]<\infty$,entonces
\begin{eqnarray*}
\frac{\int_{0}^{t}V\left(s\right)ds}{t}\rightarrow\frac{\esp\left[\int_{0}^{X}V\left(s\right)ds\right]}{\esp\left[X\right]}
\end{eqnarray*}
con probabilidad 1 y en media, cuando $t\rightarrow\infty$.
\end{itemize}
\end{Teo}
%
%___________________________________________________________________________________________
%\vspace{5.5cm}
%\chapter{Cadenas de Markov estacionarias}
%\vspace{-1.0cm}


%__________________________________________________________________________________________
%\subsection{Procesos Regenerativos Estacionarios - Stidham \cite{Stidham}}
%__________________________________________________________________________________________


Un proceso estoc\'astico a tiempo continuo $\left\{V\left(t\right),t\geq0\right\}$ es un proceso regenerativo si existe una sucesi\'on de variables aleatorias independientes e id\'enticamente distribuidas $\left\{X_{1},X_{2},\ldots\right\}$, sucesi\'on de renovaci\'on, tal que para cualquier conjunto de Borel $A$, 

\begin{eqnarray*}
\prob\left\{V\left(t\right)\in A|X_{1}+X_{2}+\cdots+X_{R\left(t\right)}=s,\left\{V\left(\tau\right),\tau<s\right\}\right\}=\prob\left\{V\left(t-s\right)\in A|X_{1}>t-s\right\},
\end{eqnarray*}
para todo $0\leq s\leq t$, donde $R\left(t\right)=\max\left\{X_{1}+X_{2}+\cdots+X_{j}\leq t\right\}=$n\'umero de renovaciones ({\emph{puntos de regeneraci\'on}}) que ocurren en $\left[0,t\right]$. El intervalo $\left[0,X_{1}\right)$ es llamado {\emph{primer ciclo de regeneraci\'on}} de $\left\{V\left(t \right),t\geq0\right\}$, $\left[X_{1},X_{1}+X_{2}\right)$ el {\emph{segundo ciclo de regeneraci\'on}}, y as\'i sucesivamente.

Sea $X=X_{1}$ y sea $F$ la funci\'on de distrbuci\'on de $X$


\begin{Def}
Se define el proceso estacionario, $\left\{V^{*}\left(t\right),t\geq0\right\}$, para $\left\{V\left(t\right),t\geq0\right\}$ por

\begin{eqnarray*}
\prob\left\{V\left(t\right)\in A\right\}=\frac{1}{\esp\left[X\right]}\int_{0}^{\infty}\prob\left\{V\left(t+x\right)\in A|X>x\right\}\left(1-F\left(x\right)\right)dx,
\end{eqnarray*} 
para todo $t\geq0$ y todo conjunto de Borel $A$.
\end{Def}

\begin{Def}
Una distribuci\'on se dice que es {\emph{aritm\'etica}} si todos sus puntos de incremento son m\'ultiplos de la forma $0,\lambda, 2\lambda,\ldots$ para alguna $\lambda>0$ entera.
\end{Def}


\begin{Def}
Una modificaci\'on medible de un proceso $\left\{V\left(t\right),t\geq0\right\}$, es una versi\'on de este, $\left\{V\left(t,w\right)\right\}$ conjuntamente medible para $t\geq0$ y para $w\in S$, $S$ espacio de estados para $\left\{V\left(t\right),t\geq0\right\}$.
\end{Def}

\begin{Teo}
Sea $\left\{V\left(t\right),t\geq\right\}$ un proceso regenerativo no negativo con modificaci\'on medible. Sea $\esp\left[X\right]<\infty$. Entonces el proceso estacionario dado por la ecuaci\'on anterior est\'a bien definido y tiene funci\'on de distribuci\'on independiente de $t$, adem\'as
\begin{itemize}
\item[i)] \begin{eqnarray*}
\esp\left[V^{*}\left(0\right)\right]&=&\frac{\esp\left[\int_{0}^{X}V\left(s\right)ds\right]}{\esp\left[X\right]}\end{eqnarray*}
\item[ii)] Si $\esp\left[V^{*}\left(0\right)\right]<\infty$, equivalentemente, si $\esp\left[\int_{0}^{X}V\left(s\right)ds\right]<\infty$,entonces
\begin{eqnarray*}
\frac{\int_{0}^{t}V\left(s\right)ds}{t}\rightarrow\frac{\esp\left[\int_{0}^{X}V\left(s\right)ds\right]}{\esp\left[X\right]}
\end{eqnarray*}
con probabilidad 1 y en media, cuando $t\rightarrow\infty$.
\end{itemize}
\end{Teo}

Para $\left\{X\left(t\right):t\geq0\right\}$ Proceso Estoc\'astico a tiempo continuo con estado de espacios $S$, que es un espacio m\'etrico, con trayectorias continuas por la derecha y con l\'imites por la izquierda c.s. Sea $N\left(t\right)$ un proceso de renovaci\'on en $\rea_{+}$ definido en el mismo espacio de probabilidad que $X\left(t\right)$, con tiempos de renovaci\'on $T$ y tiempos de inter-renovaci\'on $\xi_{n}=T_{n}-T_{n-1}$, con misma distribuci\'on $F$ de media finita $\mu$.


%______________________________________________________________________
%\subsection{Ejemplos, Notas importantes}


Sean $T_{1},T_{2},\ldots$ los puntos donde las longitudes de las colas de la red de sistemas de visitas c\'iclicas son cero simult\'aneamente, cuando la cola $Q_{j}$ es visitada por el servidor para dar servicio, es decir, $L_{1}\left(T_{i}\right)=0,L_{2}\left(T_{i}\right)=0,\hat{L}_{1}\left(T_{i}\right)=0$ y $\hat{L}_{2}\left(T_{i}\right)=0$, a estos puntos se les denominar\'a puntos regenerativos. Sea la funci\'on generadora de momentos para $L_{i}$, el n\'umero de usuarios en la cola $Q_{i}\left(z\right)$ en cualquier momento, est\'a dada por el tiempo promedio de $z^{L_{i}\left(t\right)}$ sobre el ciclo regenerativo definido anteriormente:

\begin{eqnarray*}
Q_{i}\left(z\right)&=&\esp\left[z^{L_{i}\left(t\right)}\right]=\frac{\esp\left[\sum_{m=1}^{M_{i}}\sum_{t=\tau_{i}\left(m\right)}^{\tau_{i}\left(m+1\right)-1}z^{L_{i}\left(t\right)}\right]}{\esp\left[\sum_{m=1}^{M_{i}}\tau_{i}\left(m+1\right)-\tau_{i}\left(m\right)\right]}
\end{eqnarray*}

$M_{i}$ es un tiempo de paro en el proceso regenerativo con $\esp\left[M_{i}\right]<\infty$\footnote{En Stidham\cite{Stidham} y Heyman  se muestra que una condici\'on suficiente para que el proceso regenerativo 
estacionario sea un procesoo estacionario es que el valor esperado del tiempo del ciclo regenerativo sea finito, es decir: $\esp\left[\sum_{m=1}^{M_{i}}C_{i}^{(m)}\right]<\infty$, como cada $C_{i}^{(m)}$ contiene intervalos de r\'eplica positivos, se tiene que $\esp\left[M_{i}\right]<\infty$, adem\'as, como $M_{i}>0$, se tiene que la condici\'on anterior es equivalente a tener que $\esp\left[C_{i}\right]<\infty$,
por lo tanto una condici\'on suficiente para la existencia del proceso regenerativo est\'a dada por $\sum_{k=1}^{N}\mu_{k}<1.$}, se sigue del lema de Wald que:


\begin{eqnarray*}
\esp\left[\sum_{m=1}^{M_{i}}\sum_{t=\tau_{i}\left(m\right)}^{\tau_{i}\left(m+1\right)-1}z^{L_{i}\left(t\right)}\right]&=&\esp\left[M_{i}\right]\esp\left[\sum_{t=\tau_{i}\left(m\right)}^{\tau_{i}\left(m+1\right)-1}z^{L_{i}\left(t\right)}\right]\\
\esp\left[\sum_{m=1}^{M_{i}}\tau_{i}\left(m+1\right)-\tau_{i}\left(m\right)\right]&=&\esp\left[M_{i}\right]\esp\left[\tau_{i}\left(m+1\right)-\tau_{i}\left(m\right)\right]
\end{eqnarray*}

por tanto se tiene que


\begin{eqnarray*}
Q_{i}\left(z\right)&=&\frac{\esp\left[\sum_{t=\tau_{i}\left(m\right)}^{\tau_{i}\left(m+1\right)-1}z^{L_{i}\left(t\right)}\right]}{\esp\left[\tau_{i}\left(m+1\right)-\tau_{i}\left(m\right)\right]}
\end{eqnarray*}

observar que el denominador es simplemente la duraci\'on promedio del tiempo del ciclo.


Haciendo las siguientes sustituciones en la ecuaci\'on (\ref{Corolario2}): $n\rightarrow t-\tau_{i}\left(m\right)$, $T \rightarrow \overline{\tau}_{i}\left(m\right)-\tau_{i}\left(m\right)$, $L_{n}\rightarrow L_{i}\left(t\right)$ y $F\left(z\right)=\esp\left[z^{L_{0}}\right]\rightarrow F_{i}\left(z\right)=\esp\left[z^{L_{i}\tau_{i}\left(m\right)}\right]$, se puede ver que

\begin{eqnarray}\label{Eq.Arribos.Primera}
\esp\left[\sum_{n=0}^{T-1}z^{L_{n}}\right]=
\esp\left[\sum_{t=\tau_{i}\left(m\right)}^{\overline{\tau}_{i}\left(m\right)-1}z^{L_{i}\left(t\right)}\right]
=z\frac{F_{i}\left(z\right)-1}{z-P_{i}\left(z\right)}
\end{eqnarray}

Por otra parte durante el tiempo de intervisita para la cola $i$, $L_{i}\left(t\right)$ solamente se incrementa de manera que el incremento por intervalo de tiempo est\'a dado por la funci\'on generadora de probabilidades de $P_{i}\left(z\right)$, por tanto la suma sobre el tiempo de intervisita puede evaluarse como:

\begin{eqnarray*}
\esp\left[\sum_{t=\tau_{i}\left(m\right)}^{\tau_{i}\left(m+1\right)-1}z^{L_{i}\left(t\right)}\right]&=&\esp\left[\sum_{t=\tau_{i}\left(m\right)}^{\tau_{i}\left(m+1\right)-1}\left\{P_{i}\left(z\right)\right\}^{t-\overline{\tau}_{i}\left(m\right)}\right]=\frac{1-\esp\left[\left\{P_{i}\left(z\right)\right\}^{\tau_{i}\left(m+1\right)-\overline{\tau}_{i}\left(m\right)}\right]}{1-P_{i}\left(z\right)}\\
&=&\frac{1-I_{i}\left[P_{i}\left(z\right)\right]}{1-P_{i}\left(z\right)}
\end{eqnarray*}
por tanto

\begin{eqnarray*}
\esp\left[\sum_{t=\tau_{i}\left(m\right)}^{\tau_{i}\left(m+1\right)-1}z^{L_{i}\left(t\right)}\right]&=&
\frac{1-F_{i}\left(z\right)}{1-P_{i}\left(z\right)}
\end{eqnarray*}

Por lo tanto

\begin{eqnarray*}
Q_{i}\left(z\right)&=&\frac{\esp\left[\sum_{t=\tau_{i}\left(m\right)}^{\tau_{i}
\left(m+1\right)-1}z^{L_{i}\left(t\right)}\right]}{\esp\left[\tau_{i}\left(m+1\right)-\tau_{i}\left(m\right)\right]}\\
&=&\frac{1}{\esp\left[\tau_{i}\left(m+1\right)-\tau_{i}\left(m\right)\right]}
\left\{
\esp\left[\sum_{t=\tau_{i}\left(m\right)}^{\overline{\tau}_{i}\left(m\right)-1}
z^{L_{i}\left(t\right)}\right]
+\esp\left[\sum_{t=\overline{\tau}_{i}\left(m\right)}^{\tau_{i}\left(m+1\right)-1}
z^{L_{i}\left(t\right)}\right]\right\}\\
&=&\frac{1}{\esp\left[\tau_{i}\left(m+1\right)-\tau_{i}\left(m\right)\right]}
\left\{
z\frac{F_{i}\left(z\right)-1}{z-P_{i}\left(z\right)}+\frac{1-F_{i}\left(z\right)}
{1-P_{i}\left(z\right)}
\right\}
\end{eqnarray*}

es decir

\begin{equation}
Q_{i}\left(z\right)=\frac{1}{\esp\left[C_{i}\right]}\cdot\frac{1-F_{i}\left(z\right)}{P_{i}\left(z\right)-z}\cdot\frac{\left(1-z\right)P_{i}\left(z\right)}{1-P_{i}\left(z\right)}
\end{equation}

\begin{Teo}
Dada una Red de Sistemas de Visitas C\'iclicas (RSVC), conformada por dos Sistemas de Visitas C\'iclicas (SVC), donde cada uno de ellos consta de dos colas tipo $M/M/1$. Los dos sistemas est\'an comunicados entre s\'i por medio de la transferencia de usuarios entre las colas $Q_{1}\leftrightarrow Q_{3}$ y $Q_{2}\leftrightarrow Q_{4}$. Se definen los eventos para los procesos de arribos al tiempo $t$, $A_{j}\left(t\right)=\left\{0 \textrm{ arribos en }Q_{j}\textrm{ al tiempo }t\right\}$ para alg\'un tiempo $t\geq0$ y $Q_{j}$ la cola $j$-\'esima en la RSVC, para $j=1,2,3,4$.  Existe un intervalo $I\neq\emptyset$ tal que para $T^{*}\in I$, tal que $\prob\left\{A_{1}\left(T^{*}\right),A_{2}\left(Tt^{*}\right),
A_{3}\left(T^{*}\right),A_{4}\left(T^{*}\right)|T^{*}\in I\right\}>0$.
\end{Teo}

\begin{proof}
Sin p\'erdida de generalidad podemos considerar como base del an\'alisis a la cola $Q_{1}$ del primer sistema que conforma la RSVC.

Sea $n>0$, ciclo en el primer sistema en el que se sabe que $L_{j}\left(\overline{\tau}_{1}\left(n\right)\right)=0$, pues la pol\'itica de servicio con que atienden los servidores es la exhaustiva. Como es sabido, para trasladarse a la siguiente cola, el servidor incurre en un tiempo de traslado $r_{1}\left(n\right)>0$, entonces tenemos que $\tau_{2}\left(n\right)=\overline{\tau}_{1}\left(n\right)+r_{1}\left(n\right)$.


Definamos el intervalo $I_{1}\equiv\left[\overline{\tau}_{1}\left(n\right),\tau_{2}\left(n\right)\right]$ de longitud $\xi_{1}=r_{1}\left(n\right)$. Dado que los tiempos entre arribo son exponenciales con tasa $\tilde{\mu}_{1}=\mu_{1}+\hat{\mu}_{1}$ ($\mu_{1}$ son los arribos a $Q_{1}$ por primera vez al sistema, mientras que $\hat{\mu}_{1}$ son los arribos de traslado procedentes de $Q_{3}$) se tiene que la probabilidad del evento $A_{1}\left(t\right)$ est\'a dada por 

\begin{equation}
\prob\left\{A_{1}\left(t\right)|t\in I_{1}\left(n\right)\right\}=e^{-\tilde{\mu}_{1}\xi_{1}\left(n\right)}.
\end{equation} 

Por otra parte, para la cola $Q_{2}$, el tiempo $\overline{\tau}_{2}\left(n-1\right)$ es tal que $L_{2}\left(\overline{\tau}_{2}\left(n-1\right)\right)=0$, es decir, es el tiempo en que la cola queda totalmente vac\'ia en el ciclo anterior a $n$. Entonces tenemos un sgundo intervalo $I_{2}\equiv\left[\overline{\tau}_{2}\left(n-1\right),\tau_{2}\left(n\right)\right]$. Por lo tanto la probabilidad del evento $A_{2}\left(t\right)$ tiene probabilidad dada por

\begin{equation}
\prob\left\{A_{2}\left(t\right)|t\in I_{2}\left(n\right)\right\}=e^{-\tilde{\mu}_{2}\xi_{2}\left(n\right)},
\end{equation} 

donde $\xi_{2}\left(n\right)=\tau_{2}\left(n\right)-\overline{\tau}_{2}\left(n-1\right)$.



Entonces, se tiene que

\begin{eqnarray*}
\prob\left\{A_{1}\left(t\right),A_{2}\left(t\right)|t\in I_{1}\left(n\right)\right\}&=&
\prob\left\{A_{1}\left(t\right)|t\in I_{1}\left(n\right)\right\}
\prob\left\{A_{2}\left(t\right)|t\in I_{1}\left(n\right)\right\}\\
&\geq&
\prob\left\{A_{1}\left(t\right)|t\in I_{1}\left(n\right)\right\}
\prob\left\{A_{2}\left(t\right)|t\in I_{2}\left(n\right)\right\}\\
&=&e^{-\tilde{\mu}_{1}\xi_{1}\left(n\right)}e^{-\tilde{\mu}_{2}\xi_{2}\left(n\right)}
=e^{-\left[\tilde{\mu}_{1}\xi_{1}\left(n\right)+\tilde{\mu}_{2}\xi_{2}\left(n\right)\right]}.
\end{eqnarray*}


es decir, 

\begin{equation}
\prob\left\{A_{1}\left(t\right),A_{2}\left(t\right)|t\in I_{1}\left(n\right)\right\}
=e^{-\left[\tilde{\mu}_{1}\xi_{1}\left(n\right)+\tilde{\mu}_{2}\xi_{2}
\left(n\right)\right]}>0.
\end{equation}

En lo que respecta a la relaci\'on entre los dos SVC que conforman la RSVC, se afirma que existe $m>0$ tal que $\overline{\tau}_{3}\left(m\right)<\tau_{2}\left(n\right)<\tau_{4}\left(m\right)$.

Para $Q_{3}$ sea $I_{3}=\left[\overline{\tau}_{3}\left(m\right),\tau_{4}\left(m\right)\right]$ con longitud  $\xi_{3}\left(m\right)=r_{3}\left(m\right)$, entonces 

\begin{equation}
\prob\left\{A_{3}\left(t\right)|t\in I_{3}\left(n\right)\right\}=e^{-\tilde{\mu}_{3}\xi_{3}\left(n\right)}.
\end{equation} 

An\'alogamente que como se hizo para $Q_{2}$, tenemos que para $Q_{4}$ se tiene el intervalo $I_{4}=\left[\overline{\tau}_{4}\left(m-1\right),\tau_{4}\left(m\right)\right]$ con longitud $\xi_{4}\left(m\right)=\tau_{4}\left(m\right)-\overline{\tau}_{4}\left(m-1\right)$, entonces


\begin{equation}
\prob\left\{A_{4}\left(t\right)|t\in I_{4}\left(m\right)\right\}=e^{-\tilde{\mu}_{4}\xi_{4}\left(n\right)}.
\end{equation} 

Al igual que para el primer sistema, dado que $I_{3}\left(m\right)\subset I_{4}\left(m\right)$, se tiene que

\begin{eqnarray*}
\xi_{3}\left(m\right)\leq\xi_{4}\left(m\right)&\Leftrightarrow& -\xi_{3}\left(m\right)\geq-\xi_{4}\left(m\right)
\\
-\tilde{\mu}_{4}\xi_{3}\left(m\right)\geq-\tilde{\mu}_{4}\xi_{4}\left(m\right)&\Leftrightarrow&
e^{-\tilde{\mu}_{4}\xi_{3}\left(m\right)}\geq e^{-\tilde{\mu}_{4}\xi_{4}\left(m\right)}\\
\prob\left\{A_{4}\left(t\right)|t\in I_{3}\left(m\right)\right\}&\geq&
\prob\left\{A_{4}\left(t\right)|t\in I_{4}\left(m\right)\right\}
\end{eqnarray*}

Entonces, dado que los eventos $A_{3}$ y $A_{4}$ son independientes, se tiene que

\begin{eqnarray*}
\prob\left\{A_{3}\left(t\right),A_{4}\left(t\right)|t\in I_{3}\left(m\right)\right\}&=&
\prob\left\{A_{3}\left(t\right)|t\in I_{3}\left(m\right)\right\}
\prob\left\{A_{4}\left(t\right)|t\in I_{3}\left(m\right)\right\}\\
&\geq&
\prob\left\{A_{3}\left(t\right)|t\in I_{3}\left(n\right)\right\}
\prob\left\{A_{4}\left(t\right)|t\in I_{4}\left(n\right)\right\}\\
&=&e^{-\tilde{\mu}_{3}\xi_{3}\left(m\right)}e^{-\tilde{\mu}_{4}\xi_{4}
\left(m\right)}
=e^{-\left[\tilde{\mu}_{3}\xi_{3}\left(m\right)+\tilde{\mu}_{4}\xi_{4}
\left(m\right)\right]}.
\end{eqnarray*}


es decir, 

\begin{equation}
\prob\left\{A_{3}\left(t\right),A_{4}\left(t\right)|t\in I_{3}\left(m\right)\right\}
=e^{-\left[\tilde{\mu}_{3}\xi_{3}\left(m\right)+\tilde{\mu}_{4}\xi_{4}
\left(m\right)\right]}>0.
\end{equation}

Por construcci\'on se tiene que $I\left(n,m\right)\equiv I_{1}\left(n\right)\cap I_{3}\left(m\right)\neq\emptyset$,entonces en particular se tienen las contenciones $I\left(n,m\right)\subseteq I_{1}\left(n\right)$ y $I\left(n,m\right)\subseteq I_{3}\left(m\right)$, por lo tanto si definimos $\xi_{n,m}\equiv\ell\left(I\left(n,m\right)\right)$ tenemos que

\begin{eqnarray*}
\xi_{n,m}\leq\xi_{1}\left(n\right)\textrm{ y }\xi_{n,m}\leq\xi_{3}\left(m\right)\textrm{ entonces }
-\xi_{n,m}\geq-\xi_{1}\left(n\right)\textrm{ y }-\xi_{n,m}\leq-\xi_{3}\left(m\right)\\
\end{eqnarray*}
por lo tanto tenemos las desigualdades 



\begin{eqnarray*}
\begin{array}{ll}
-\tilde{\mu}_{1}\xi_{n,m}\geq-\tilde{\mu}_{1}\xi_{1}\left(n\right),&
-\tilde{\mu}_{2}\xi_{n,m}\geq-\tilde{\mu}_{2}\xi_{1}\left(n\right)
\geq-\tilde{\mu}_{2}\xi_{2}\left(n\right),\\
-\tilde{\mu}_{3}\xi_{n,m}\geq-\tilde{\mu}_{3}\xi_{3}\left(m\right),&
-\tilde{\mu}_{4}\xi_{n,m}\geq-\tilde{\mu}_{4}\xi_{3}\left(m\right)
\geq-\tilde{\mu}_{4}\xi_{4}\left(m\right).
\end{array}
\end{eqnarray*}

Sea $T^{*}\in I_{n,m}$, entonces dado que en particular $T^{*}\in I_{1}\left(n\right)$ se cumple con probabilidad positiva que no hay arribos a las colas $Q_{1}$ y $Q_{2}$, en consecuencia, tampoco hay usuarios de transferencia para $Q_{3}$ y $Q_{4}$, es decir, $\tilde{\mu}_{1}=\mu_{1}$, $\tilde{\mu}_{2}=\mu_{2}$, $\tilde{\mu}_{3}=\mu_{3}$, $\tilde{\mu}_{4}=\mu_{4}$, es decir, los eventos $Q_{1}$ y $Q_{3}$ son condicionalmente independientes en el intervalo $I_{n,m}$; lo mismo ocurre para las colas $Q_{2}$ y $Q_{4}$, por lo tanto tenemos que


\begin{eqnarray}
\begin{array}{l}
\prob\left\{A_{1}\left(T^{*}\right),A_{2}\left(T^{*}\right),
A_{3}\left(T^{*}\right),A_{4}\left(T^{*}\right)|T^{*}\in I_{n,m}\right\}
=\prod_{j=1}^{4}\prob\left\{A_{j}\left(T^{*}\right)|T^{*}\in I_{n,m}\right\}\\
\geq\prob\left\{A_{1}\left(T^{*}\right)|T^{*}\in I_{1}\left(n\right)\right\}
\prob\left\{A_{2}\left(T^{*}\right)|T^{*}\in I_{2}\left(n\right)\right\}
\prob\left\{A_{3}\left(T^{*}\right)|T^{*}\in I_{3}\left(m\right)\right\}
\prob\left\{A_{4}\left(T^{*}\right)|T^{*}\in I_{4}\left(m\right)\right\}\\
=e^{-\mu_{1}\xi_{1}\left(n\right)}
e^{-\mu_{2}\xi_{2}\left(n\right)}
e^{-\mu_{3}\xi_{3}\left(m\right)}
e^{-\mu_{4}\xi_{4}\left(m\right)}
=e^{-\left[\tilde{\mu}_{1}\xi_{1}\left(n\right)
+\tilde{\mu}_{2}\xi_{2}\left(n\right)
+\tilde{\mu}_{3}\xi_{3}\left(m\right)
+\tilde{\mu}_{4}\xi_{4}
\left(m\right)\right]}>0.
\end{array}
\end{eqnarray}
\end{proof}


Estos resultados aparecen en Daley (1968) \cite{Daley68} para $\left\{T_{n}\right\}$ intervalos de inter-arribo, $\left\{D_{n}\right\}$ intervalos de inter-salida y $\left\{S_{n}\right\}$ tiempos de servicio.

\begin{itemize}
\item Si el proceso $\left\{T_{n}\right\}$ es Poisson, el proceso $\left\{D_{n}\right\}$ es no correlacionado si y s\'olo si es un proceso Poisso, lo cual ocurre si y s\'olo si $\left\{S_{n}\right\}$ son exponenciales negativas.

\item Si $\left\{S_{n}\right\}$ son exponenciales negativas, $\left\{D_{n}\right\}$ es un proceso de renovaci\'on  si y s\'olo si es un proceso Poisson, lo cual ocurre si y s\'olo si $\left\{T_{n}\right\}$ es un proceso Poisson.

\item $\esp\left(D_{n}\right)=\esp\left(T_{n}\right)$.

\item Para un sistema de visitas $GI/M/1$ se tiene el siguiente teorema:

\begin{Teo}
En un sistema estacionario $GI/M/1$ los intervalos de interpartida tienen
\begin{eqnarray*}
\esp\left(e^{-\theta D_{n}}\right)&=&\mu\left(\mu+\theta\right)^{-1}\left[\delta\theta
-\mu\left(1-\delta\right)\alpha\left(\theta\right)\right]
\left[\theta-\mu\left(1-\delta\right)^{-1}\right]\\
\alpha\left(\theta\right)&=&\esp\left[e^{-\theta T_{0}}\right]\\
var\left(D_{n}\right)&=&var\left(T_{0}\right)-\left(\tau^{-1}-\delta^{-1}\right)
2\delta\left(\esp\left(S_{0}\right)\right)^{2}\left(1-\delta\right)^{-1}.
\end{eqnarray*}
\end{Teo}



\begin{Teo}
El proceso de salida de un sistema de colas estacionario $GI/M/1$ es un proceso de renovaci\'on si y s\'olo si el proceso de entrada es un proceso Poisson, en cuyo caso el proceso de salida es un proceso Poisson.
\end{Teo}


\begin{Teo}
Los intervalos de interpartida $\left\{D_{n}\right\}$ de un sistema $M/G/1$ estacionario son no correlacionados si y s\'olo si la distribuci\'on de los tiempos de servicio es exponencial negativa, es decir, el sistema es de tipo  $M/M/1$.

\end{Teo}



\end{itemize}


%\section{Resultados para Procesos de Salida}

En Sigman, Thorison y Wolff \cite{Sigman2} prueban que para la existencia de un una sucesi\'on infinita no decreciente de tiempos de regeneraci\'on $\tau_{1}\leq\tau_{2}\leq\cdots$ en los cuales el proceso se regenera, basta un tiempo de regeneraci\'on $R_{1}$, donde $R_{j}=\tau_{j}-\tau_{j-1}$. Para tal efecto se requiere la existencia de un espacio de probabilidad $\left(\Omega,\mathcal{F},\prob\right)$, y proceso estoc\'astico $\textit{X}=\left\{X\left(t\right):t\geq0\right\}$ con espacio de estados $\left(S,\mathcal{R}\right)$, con $\mathcal{R}$ $\sigma$-\'algebra.

\begin{Prop}
Si existe una variable aleatoria no negativa $R_{1}$ tal que $\theta_{R\footnotesize{1}}X=_{D}X$, entonces $\left(\Omega,\mathcal{F},\prob\right)$ puede extenderse para soportar una sucesi\'on estacionaria de variables aleatorias $R=\left\{R_{k}:k\geq1\right\}$, tal que para $k\geq1$,
\begin{eqnarray*}
\theta_{k}\left(X,R\right)=_{D}\left(X,R\right).
\end{eqnarray*}

Adem\'as, para $k\geq1$, $\theta_{k}R$ es condicionalmente independiente de $\left(X,R_{1},\ldots,R_{k}\right)$, dado $\theta_{\tau k}X$.

\end{Prop}


\begin{itemize}
\item Doob en 1953 demostr\'o que el estado estacionario de un proceso de partida en un sistema de espera $M/G/\infty$, es Poisson con la misma tasa que el proceso de arribos.

\item Burke en 1968, fue el primero en demostrar que el estado estacionario de un proceso de salida de una cola $M/M/s$ es un proceso Poisson.

\item Disney en 1973 obtuvo el siguiente resultado:

\begin{Teo}
Para el sistema de espera $M/G/1/L$ con disciplina FIFO, el proceso $\textbf{I}$ es un proceso de renovaci\'on si y s\'olo si el proceso denominado longitud de la cola es estacionario y se cumple cualquiera de los siguientes casos:

\begin{itemize}
\item[a)] Los tiempos de servicio son identicamente cero;
\item[b)] $L=0$, para cualquier proceso de servicio $S$;
\item[c)] $L=1$ y $G=D$;
\item[d)] $L=\infty$ y $G=M$.
\end{itemize}
En estos casos, respectivamente, las distribuciones de interpartida $P\left\{T_{n+1}-T_{n}\leq t\right\}$ son


\begin{itemize}
\item[a)] $1-e^{-\lambda t}$, $t\geq0$;
\item[b)] $1-e^{-\lambda t}*F\left(t\right)$, $t\geq0$;
\item[c)] $1-e^{-\lambda t}*\indora_{d}\left(t\right)$, $t\geq0$;
\item[d)] $1-e^{-\lambda t}*F\left(t\right)$, $t\geq0$.
\end{itemize}
\end{Teo}


\item Finch (1959) mostr\'o que para los sistemas $M/G/1/L$, con $1\leq L\leq \infty$ con distribuciones de servicio dos veces diferenciable, solamente el sistema $M/M/1/\infty$ tiene proceso de salida de renovaci\'on estacionario.

\item King (1971) demostro que un sistema de colas estacionario $M/G/1/1$ tiene sus tiempos de interpartida sucesivas $D_{n}$ y $D_{n+1}$ son independientes, si y s\'olo si, $G=D$, en cuyo caso le proceso de salida es de renovaci\'on.

\item Disney (1973) demostr\'o que el \'unico sistema estacionario $M/G/1/L$, que tiene proceso de salida de renovaci\'on  son los sistemas $M/M/1$ y $M/D/1/1$.



\item El siguiente resultado es de Disney y Koning (1985)
\begin{Teo}
En un sistema de espera $M/G/s$, el estado estacionario del proceso de salida es un proceso Poisson para cualquier distribuci\'on de los tiempos de servicio si el sistema tiene cualquiera de las siguientes cuatro propiedades.

\begin{itemize}
\item[a)] $s=\infty$
\item[b)] La disciplina de servicio es de procesador compartido.
\item[c)] La disciplina de servicio es LCFS y preemptive resume, esto se cumple para $L<\infty$
\item[d)] $G=M$.
\end{itemize}

\end{Teo}

\item El siguiente resultado es de Alamatsaz (1983)

\begin{Teo}
En cualquier sistema de colas $GI/G/1/L$ con $1\leq L<\infty$ y distribuci\'on de interarribos $A$ y distribuci\'on de los tiempos de servicio $B$, tal que $A\left(0\right)=0$, $A\left(t\right)\left(1-B\left(t\right)\right)>0$ para alguna $t>0$ y $B\left(t\right)$ para toda $t>0$, es imposible que el proceso de salida estacionario sea de renovaci\'on.
\end{Teo}

\end{itemize}

Estos resultados aparecen en Daley (1968) \cite{Daley68} para $\left\{T_{n}\right\}$ intervalos de inter-arribo, $\left\{D_{n}\right\}$ intervalos de inter-salida y $\left\{S_{n}\right\}$ tiempos de servicio.

\begin{itemize}
\item Si el proceso $\left\{T_{n}\right\}$ es Poisson, el proceso $\left\{D_{n}\right\}$ es no correlacionado si y s\'olo si es un proceso Poisso, lo cual ocurre si y s\'olo si $\left\{S_{n}\right\}$ son exponenciales negativas.

\item Si $\left\{S_{n}\right\}$ son exponenciales negativas, $\left\{D_{n}\right\}$ es un proceso de renovaci\'on  si y s\'olo si es un proceso Poisson, lo cual ocurre si y s\'olo si $\left\{T_{n}\right\}$ es un proceso Poisson.

\item $\esp\left(D_{n}\right)=\esp\left(T_{n}\right)$.

\item Para un sistema de visitas $GI/M/1$ se tiene el siguiente teorema:

\begin{Teo}
En un sistema estacionario $GI/M/1$ los intervalos de interpartida tienen
\begin{eqnarray*}
\esp\left(e^{-\theta D_{n}}\right)&=&\mu\left(\mu+\theta\right)^{-1}\left[\delta\theta
-\mu\left(1-\delta\right)\alpha\left(\theta\right)\right]
\left[\theta-\mu\left(1-\delta\right)^{-1}\right]\\
\alpha\left(\theta\right)&=&\esp\left[e^{-\theta T_{0}}\right]\\
var\left(D_{n}\right)&=&var\left(T_{0}\right)-\left(\tau^{-1}-\delta^{-1}\right)
2\delta\left(\esp\left(S_{0}\right)\right)^{2}\left(1-\delta\right)^{-1}.
\end{eqnarray*}
\end{Teo}



\begin{Teo}
El proceso de salida de un sistema de colas estacionario $GI/M/1$ es un proceso de renovaci\'on si y s\'olo si el proceso de entrada es un proceso Poisson, en cuyo caso el proceso de salida es un proceso Poisson.
\end{Teo}


\begin{Teo}
Los intervalos de interpartida $\left\{D_{n}\right\}$ de un sistema $M/G/1$ estacionario son no correlacionados si y s\'olo si la distribuci\'on de los tiempos de servicio es exponencial negativa, es decir, el sistema es de tipo  $M/M/1$.

\end{Teo}



\end{itemize}
%\newpage


%___________________________________________________________________________________________
%\vspace{5.5cm}
\section{Preliminares: Modelos de Flujo}
%\vspace{-1.0cm}
%___________________________________________________________________________________________
%
\subsection{Procesos Regenerativos}
%_____________________________________________________

Si $x$ es el n{\'u}mero de usuarios en la cola al comienzo del
periodo de servicio y $N_{s}\left(x\right)=N\left(x\right)$ es el
n{\'u}mero de usuarios que son atendidos con la pol{\'\i}tica $s$,
{\'u}nica en nuestro caso, durante un periodo de servicio,
entonces se asume que:
\begin{itemize}
\item[(S1.)]
\begin{equation}\label{S1}
lim_{x\rightarrow\infty}\esp\left[N\left(x\right)\right]=\overline{N}>0.
\end{equation}
\item[(S2.)]
\begin{equation}\label{S2}
\esp\left[N\left(x\right)\right]\leq \overline{N}, \end{equation}
para cualquier valor de $x$. \item La $n$-{\'e}sima ocurrencia va
acompa{\~n}ada con el tiempo de cambio de longitud
$\delta_{j,j+1}\left(n\right)$, independientes e id{\'e}nticamente
distribuidas, con
$\esp\left[\delta_{j,j+1}\left(1\right)\right]\geq0$. \item Se
define
\begin{equation}
\delta^{*}:=\sum_{j,j+1}\esp\left[\delta_{j,j+1}\left(1\right)\right].
\end{equation}

\item Los tiempos de inter-arribo a la cola $k$,son de la forma
$\left\{\xi_{k}\left(n\right)\right\}_{n\geq1}$, con la propiedad
de que son independientes e id{\'e}nticamente distribuidos.

\item Los tiempos de servicio
$\left\{\eta_{k}\left(n\right)\right\}_{n\geq1}$ tienen la
propiedad de ser independientes e id{\'e}nticamente distribuidos.

\item Se define la tasa de arribo a la $k$-{\'e}sima cola como
$\lambda_{k}=1/\esp\left[\xi_{k}\left(1\right)\right]$ y
adem{\'a}s se define

\item la tasa de servicio para la $k$-{\'e}sima cola como
$\mu_{k}=1/\esp\left[\eta_{k}\left(1\right)\right]$

\item tambi{\'e}n se define $\rho_{k}=\lambda_{k}/\mu_{k}$, donde
es necesario que $\rho<1$ para cuestiones de estabilidad.

\item De las pol{\'\i}ticas posibles solamente consideraremos la
pol{\'\i}tica cerrada (Gated).
\end{itemize}

Las Colas C\'iclicas se pueden describir por medio de un proceso
de Markov $\left(X\left(t\right)\right)_{t\in\rea}$, donde el
estado del sistema al tiempo $t\geq0$ est\'a dado por
\begin{equation}
X\left(t\right)=\left(Q\left(t\right),A\left(t\right),H\left(t\right),B\left(t\right),B^{0}\left(t\right),C\left(t\right)\right)
\end{equation}
definido en el espacio producto:
\begin{equation}
\mathcal{X}=\mathbb{Z}^{K}\times\rea_{+}^{K}\times\left(\left\{1,2,\ldots,K\right\}\times\left\{1,2,\ldots,S\right\}\right)^{M}\times\rea_{+}^{K}\times\rea_{+}^{K}\times\mathbb{Z}^{K},
\end{equation}

\begin{itemize}
\item $Q\left(t\right)=\left(Q_{k}\left(t\right),1\leq k\leq
K\right)$, es el n\'umero de usuarios en la cola $k$, incluyendo
aquellos que est\'an siendo atendidos provenientes de la
$k$-\'esima cola.

\item $A\left(t\right)=\left(A_{k}\left(t\right),1\leq k\leq
K\right)$, son los residuales de los tiempos de arribo en la cola
$k$. \item $H\left(t\right)$ es el par ordenado que consiste en la
cola que esta siendo atendida y la pol\'itica de servicio que se
utilizar\'a.

\item $B\left(t\right)$ es el tiempo de servicio residual.

\item $B^{0}\left(t\right)$ es el tiempo residual del cambio de
cola.

\item $C\left(t\right)$ indica el n\'umero de usuarios atendidos
durante la visita del servidor a la cola dada en
$H\left(t\right)$.
\end{itemize}

$A_{k}\left(t\right),B_{m}\left(t\right)$ y
$B_{m}^{0}\left(t\right)$ se suponen continuas por la derecha y
que satisfacen la propiedad fuerte de Markov, (\cite{Dai})

\begin{itemize}
\item Los tiempos de interarribo a la cola $k$,son de la forma
$\left\{\xi_{k}\left(n\right)\right\}_{n\geq1}$, con la propiedad
de que son independientes e id{\'e}nticamente distribuidos.

\item Los tiempos de servicio
$\left\{\eta_{k}\left(n\right)\right\}_{n\geq1}$ tienen la
propiedad de ser independientes e id{\'e}nticamente distribuidos.

\item Se define la tasa de arribo a la $k$-{\'e}sima cola como
$\lambda_{k}=1/\esp\left[\xi_{k}\left(1\right)\right]$ y
adem{\'a}s se define

\item la tasa de servicio para la $k$-{\'e}sima cola como
$\mu_{k}=1/\esp\left[\eta_{k}\left(1\right)\right]$

\item tambi{\'e}n se define $\rho_{k}=\lambda_{k}/\mu_{k}$, donde
es necesario que $\rho<1$ para cuestiones de estabilidad.

\item De las pol{\'\i}ticas posibles solamente consideraremos la
pol{\'\i}tica cerrada (Gated).
\end{itemize}

%\section{Preliminares}



Sup\'ongase que el sistema consta de varias colas a los cuales
llegan uno o varios servidores a dar servicio a los usuarios
esperando en la cola.\\


Si $x$ es el n\'umero de usuarios en la cola al comienzo del
periodo de servicio y $N_{s}\left(x\right)=N\left(x\right)$ es el
n\'umero de usuarios que son atendidos con la pol\'itica $s$,
\'unica en nuestro caso, durante un periodo de servicio, entonces
se asume que:
\begin{itemize}
\item[1)]\label{S1}$lim_{x\rightarrow\infty}\esp\left[N\left(x\right)\right]=\overline{N}>0$
\item[2)]\label{S2}$\esp\left[N\left(x\right)\right]\leq\overline{N}$para
cualquier valor de $x$.
\end{itemize}
La manera en que atiende el servidor $m$-\'esimo, en este caso en
espec\'ifico solo lo ilustraremos con un s\'olo servidor, es la
siguiente:
\begin{itemize}
\item Al t\'ermino de la visita a la cola $j$, el servidor se
cambia a la cola $j^{'}$ con probabilidad
$r_{j,j^{'}}^{m}=r_{j,j^{'}}$

\item La $n$-\'esima ocurrencia va acompa\~nada con el tiempo de
cambio de longitud $\delta_{j,j^{'}}\left(n\right)$,
independientes e id\'enticamente distribuidas, con
$\esp\left[\delta_{j,j^{'}}\left(1\right)\right]\geq0$.

\item Sea $\left\{p_{j}\right\}$ la distribuci\'on invariante
estacionaria \'unica para la Cadena de Markov con matriz de
transici\'on $\left(r_{j,j^{'}}\right)$.

\item Finalmente, se define
\begin{equation}
\delta^{*}:=\sum_{j,j^{'}}p_{j}r_{j,j^{'}}\esp\left[\delta_{j,j^{'}}\left(i\right)\right].
\end{equation}
\end{itemize}

Veamos un caso muy espec\'ifico en el cual los tiempos de arribo a cada una de las colas se comportan de acuerdo a un proceso Poisson de la forma
$\left\{\xi_{k}\left(n\right)\right\}_{n\geq1}$, y los tiempos de servicio en cada una de las colas son variables aleatorias distribuidas exponencialmente e id\'enticamente distribuidas
$\left\{\eta_{k}\left(n\right)\right\}_{n\geq1}$, donde ambos procesos adem\'as cumplen la condici\'on de ser independientes entre si. Para la $k$-\'esima cola se define la tasa de arribo a la como
$\lambda_{k}=1/\esp\left[\xi_{k}\left(1\right)\right]$ y la tasa
de servicio como
$\mu_{k}=1/\esp\left[\eta_{k}\left(1\right)\right]$, finalmente se
define la carga de la cola como $\rho_{k}=\lambda_{k}/\mu_{k}$,
donde se pide que $\rho<1$, para garantizar la estabilidad del sistema.\\

Se denotar\'a por $Q_{k}\left(t\right)$ el n\'umero de usuarios en la cola $k$,
$A_{k}\left(t\right)$ los residuales de los tiempos entre arribos a la cola $k$;
para cada servidor $m$, se denota por $B_{m}\left(t\right)$ los residuales de los tiempos de servicio al tiempo $t$; $B_{m}^{0}\left(t\right)$ son los residuales de los tiempos de traslado de la cola $k$ a la pr\'oxima por atender, al tiempo $t$, finalmente sea $C_{m}\left(t\right)$ el n\'umero de usuarios atendidos durante la visita del servidor a la cola $k$ al tiempo $t$.\\


En este sentido el proceso para el sistema de visitas se puede definir como:

\begin{equation}\label{Esp.Edos.Down}
X\left(t\right)^{T}=\left(Q_{k}\left(t\right),A_{k}\left(t\right),B_{m}\left(t\right),B_{m}^{0}\left(t\right),C_{m}\left(t\right)\right)
\end{equation}
para $k=1,\ldots,K$ y $m=1,2,\ldots,M$. $X$ evoluciona en el
espacio de estados:
$X=\ent_{+}^{K}\times\rea_{+}^{K}\times\left(\left\{1,2,\ldots,K\right\}\times\left\{1,2,\ldots,S\right\}\right)^{M}\times\rea_{+}^{K}\times\ent_{+}^{K}$.\\

El sistema aqu\'i descrito debe de cumplir con los siguientes supuestos b\'asicos de un sistema de visitas:

Antes enunciemos los supuestos que regir\'an en la red.

\begin{itemize}
\item[A1)] $\xi_{1},\ldots,\xi_{K},\eta_{1},\ldots,\eta_{K}$ son
mutuamente independientes y son sucesiones independientes e
id\'enticamente distribuidas.

\item[A2)] Para alg\'un entero $p\geq1$
\begin{eqnarray*}
\esp\left[\xi_{l}\left(1\right)^{p+1}\right]<\infty\textrm{ para }l\in\mathcal{A}\textrm{ y }\\
\esp\left[\eta_{k}\left(1\right)^{p+1}\right]<\infty\textrm{ para
}k=1,\ldots,K.
\end{eqnarray*}
donde $\mathcal{A}$ es la clase de posibles arribos.

\item[A3)] Para $k=1,2,\ldots,K$ existe una funci\'on positiva
$q_{k}\left(x\right)$ definida en $\rea_{+}$, y un entero $j_{k}$,
tal que
\begin{eqnarray}
P\left(\xi_{k}\left(1\right)\geq x\right)>0\textrm{, para todo }x>0\\
P\left\{a\leq\sum_{i=1}^{j_{k}}\xi_{k}\left(i\right)\leq
b\right\}\geq\int_{a}^{b}q_{k}\left(x\right)dx, \textrm{ }0\leq
a<b.
\end{eqnarray}
\end{itemize}

En particular los procesos de tiempo entre arribos y de servicio
considerados con fines de ilustraci\'on de la metodolog\'ia
cumplen con el supuesto $A2)$ para $p=1$, es decir, ambos procesos
tienen primer y segundo momento finito.

En lo que respecta al supuesto (A3), en Dai y Meyn \cite{DaiSean}
hacen ver que este se puede sustituir por

\begin{itemize}
\item[A3')] Para el Proceso de Markov $X$, cada subconjunto
compacto de $X$ es un conjunto peque\~no, ver definici\'on
\ref{Def.Cto.Peq.}.
\end{itemize}

Es por esta raz\'on que con la finalidad de poder hacer uso de
$A3^{'})$ es necesario recurrir a los Procesos de Harris y en
particular a los Procesos Harris Recurrente:
%_______________________________________________________________________
\subsection{Procesos Harris Recurrente}
%_______________________________________________________________________

Por el supuesto (A1) conforme a Davis \cite{Davis}, se puede
definir el proceso de saltos correspondiente de manera tal que
satisfaga el supuesto (\ref{Sup3.1.Davis}), de hecho la
demostraci\'on est\'a basada en la l\'inea de argumentaci\'on de
Davis, (\cite{Davis}, p\'aginas 362-364).

Entonces se tiene un espacio de estados Markoviano. El espacio de
Markov descrito en Dai y Meyn \cite{DaiSean}

\[\left(\Omega,\mathcal{F},\mathcal{F}_{t},X\left(t\right),\theta_{t},P_{x}\right)\]
es un proceso de Borel Derecho (Sharpe \cite{Sharpe}) en el
espacio de estados medible $\left(X,\mathcal{B}_{X}\right)$. El
Proceso $X=\left\{X\left(t\right),t\geq0\right\}$ tiene
trayectorias continuas por la derecha, est\'a definida en
$\left(\Omega,\mathcal{F}\right)$ y est\'a adaptado a
$\left\{\mathcal{F}_{t},t\geq0\right\}$; la colecci\'on
$\left\{P_{x},x\in \mathbb{X}\right\}$ son medidas de probabilidad
en $\left(\Omega,\mathcal{F}\right)$ tales que para todo $x\in
\mathbb{X}$
\[P_{x}\left\{X\left(0\right)=x\right\}=1\] y
\[E_{x}\left\{f\left(X\circ\theta_{t}\right)|\mathcal{F}_{t}\right\}=E_{X}\left(\tau\right)f\left(X\right)\]
en $\left\{\tau<\infty\right\}$, $P_{x}$-c.s. Donde $\tau$ es un
$\mathcal{F}_{t}$-tiempo de paro
\[\left(X\circ\theta_{\tau}\right)\left(w\right)=\left\{X\left(\tau\left(w\right)+t,w\right),t\geq0\right\}\]
y $f$ es una funci\'on de valores reales acotada y medible con la
$\sigma$-algebra de Kolmogorov generada por los cilindros.\\

Sea $P^{t}\left(x,D\right)$, $D\in\mathcal{B}_{\mathbb{X}}$,
$t\geq0$ probabilidad de transici\'on de $X$ definida como
\[P^{t}\left(x,D\right)=P_{x}\left(X\left(t\right)\in
D\right)\]


\begin{Def}
Una medida no cero $\pi$ en
$\left(\mathbf{X},\mathcal{B}_{\mathbf{X}}\right)$ es {\bf
invariante} para $X$ si $\pi$ es $\sigma$-finita y
\[\pi\left(D\right)=\int_{\mathbf{X}}P^{t}\left(x,D\right)\pi\left(dx\right)\]
para todo $D\in \mathcal{B}_{\mathbf{X}}$, con $t\geq0$.
\end{Def}

\begin{Def}
El proceso de Markov $X$ es llamado Harris recurrente si existe
una medida de probabilidad $\nu$ en
$\left(\mathbf{X},\mathcal{B}_{\mathbf{X}}\right)$, tal que si
$\nu\left(D\right)>0$ y $D\in\mathcal{B}_{\mathbf{X}}$
\[P_{x}\left\{\tau_{D}<\infty\right\}\equiv1\] cuando
$\tau_{D}=inf\left\{t\geq0:X_{t}\in D\right\}$.
\end{Def}

\begin{Note}
\begin{itemize}
\item[i)] Si $X$ es Harris recurrente, entonces existe una \'unica
medida invariante $\pi$ (Getoor \cite{Getoor}).

\item[ii)] Si la medida invariante es finita, entonces puede
normalizarse a una medida de probabilidad, en este caso se le
llama Proceso {\em Harris recurrente positivo}.


\item[iii)] Cuando $X$ es Harris recurrente positivo se dice que
la disciplina de servicio es estable. En este caso $\pi$ denota la
distribuci\'on estacionaria y hacemos
\[P_{\pi}\left(\cdot\right)=\int_{\mathbf{X}}P_{x}\left(\cdot\right)\pi\left(dx\right)\]
y se utiliza $E_{\pi}$ para denotar el operador esperanza
correspondiente.
\end{itemize}
\end{Note}

\begin{Def}\label{Def.Cto.Peq.}
Un conjunto $D\in\mathcal{B_{\mathbf{X}}}$ es llamado peque\~no si
existe un $t>0$, una medida de probabilidad $\nu$ en
$\mathcal{B_{\mathbf{X}}}$, y un $\delta>0$ tal que
\[P^{t}\left(x,A\right)\geq\delta\nu\left(A\right)\] para $x\in
D,A\in\mathcal{B_{X}}$.
\end{Def}

La siguiente serie de resultados vienen enunciados y demostrados
en Dai \cite{Dai}:
\begin{Lema}[Lema 3.1, Dai\cite{Dai}]
Sea $B$ conjunto peque\~no cerrado, supongamos que
$P_{x}\left(\tau_{B}<\infty\right)\equiv1$ y que para alg\'un
$\delta>0$ se cumple que
\begin{equation}\label{Eq.3.1}
\sup\esp_{x}\left[\tau_{B}\left(\delta\right)\right]<\infty,
\end{equation}
donde
$\tau_{B}\left(\delta\right)=inf\left\{t\geq\delta:X\left(t\right)\in
B\right\}$. Entonces, $X$ es un proceso Harris Recurrente
Positivo.
\end{Lema}

\begin{Lema}[Lema 3.1, Dai \cite{Dai}]\label{Lema.3.}
Bajo el supuesto (A3), el conjunto $B=\left\{|x|\leq k\right\}$ es
un conjunto peque\~no cerrado para cualquier $k>0$.
\end{Lema}

\begin{Teo}[Teorema 3.1, Dai\cite{Dai}]\label{Tma.3.1}
Si existe un $\delta>0$ tal que
\begin{equation}
lim_{|x|\rightarrow\infty}\frac{1}{|x|}\esp|X^{x}\left(|x|\delta\right)|=0,
\end{equation}
entonces la ecuaci\'on (\ref{Eq.3.1}) se cumple para
$B=\left\{|x|\leq k\right\}$ con alg\'un $k>0$. En particular, $X$
es Harris Recurrente Positivo.
\end{Teo}

\begin{Note}
En Meyn and Tweedie \cite{MeynTweedie} muestran que si
$P_{x}\left\{\tau_{D}<\infty\right\}\equiv1$ incluso para solo un
conjunto peque\~no, entonces el proceso es Harris Recurrente.
\end{Note}

Entonces, tenemos que el proceso $X$ es un proceso de Markov que
cumple con los supuestos $A1)$-$A3)$, lo que falta de hacer es
construir el Modelo de Flujo bas\'andonos en lo hasta ahora
presentado.
%_______________________________________________________________________
\subsection{Modelo de Flujo}
%_______________________________________________________________________

Dada una condici\'on inicial $x\in\textrm{X}$, sea
$Q_{k}^{x}\left(t\right)$ la longitud de la cola al tiempo $t$,
$T_{m,k}^{x}\left(t\right)$ el tiempo acumulado, al tiempo $t$,
que tarda el servidor $m$ en atender a los usuarios de la cola
$k$. Finalmente sea $T_{m,k}^{x,0}\left(t\right)$ el tiempo
acumulado, al tiempo $t$, que tarda el servidor $m$ en trasladarse
a otra cola a partir de la $k$-\'esima.\\

Sup\'ongase que la funci\'on
$\left(\overline{Q}\left(\cdot\right),\overline{T}_{m}
\left(\cdot\right),\overline{T}_{m}^{0} \left(\cdot\right)\right)$
para $m=1,2,\ldots,M$ es un punto l\'imite de
\begin{equation}\label{Eq.Punto.Limite}
\left(\frac{1}{|x|}Q^{x}\left(|x|t\right),\frac{1}{|x|}T_{m}^{x}\left(|x|t\right),\frac{1}{|x|}T_{m}^{x,0}\left(|x|t\right)\right)
\end{equation}
para $m=1,2,\ldots,M$, cuando $x\rightarrow\infty$. Entonces
$\left(\overline{Q}\left(t\right),\overline{T}_{m}
\left(t\right),\overline{T}_{m}^{0} \left(t\right)\right)$ es un
flujo l\'imite del sistema. Al conjunto de todos las posibles
flujos l\'imite se le llama \textbf{Modelo de Flujo}.\\

El modelo de flujo satisface el siguiente conjunto de ecuaciones:

\begin{equation}\label{Eq.MF.1}
\overline{Q}_{k}\left(t\right)=\overline{Q}_{k}\left(0\right)+\lambda_{k}t-\sum_{m=1}^{M}\mu_{k}\overline{T}_{m,k}\left(t\right)\\
\end{equation}
para $k=1,2,\ldots,K$.\\
\begin{equation}\label{Eq.MF.2}
\overline{Q}_{k}\left(t\right)\geq0\textrm{ para
}k=1,2,\ldots,K,\\
\end{equation}

\begin{equation}\label{Eq.MF.3}
\overline{T}_{m,k}\left(0\right)=0,\textrm{ y }\overline{T}_{m,k}\left(\cdot\right)\textrm{ es no decreciente},\\
\end{equation}
para $k=1,2,\ldots,K$ y $m=1,2,\ldots,M$,\\
\begin{equation}\label{Eq.MF.4}
\sum_{k=1}^{K}\overline{T}_{m,k}^{0}\left(t\right)+\overline{T}_{m,k}\left(t\right)=t\textrm{
para }m=1,2,\ldots,M.\\
\end{equation}

De acuerdo a Dai \cite{Dai}, se tiene que el conjunto de posibles
l\'imites
$\left(\overline{Q}\left(\cdot\right),\overline{T}\left(\cdot\right),\overline{T}^{0}\left(\cdot\right)\right)$,
en el sentido de que deben de satisfacer las ecuaciones
(\ref{Eq.MF.1})-(\ref{Eq.MF.4}), se le llama {\em Modelo de
Flujo}.


\begin{Def}[Definici\'on 4.1, , Dai \cite{Dai}]\label{Def.Modelo.Flujo}
Sea una disciplina de servicio espec\'ifica. Cualquier l\'imite
$\left(\overline{Q}\left(\cdot\right),\overline{T}\left(\cdot\right)\right)$
en (\ref{Eq.Punto.Limite}) es un {\em flujo l\'imite} de la
disciplina. Cualquier soluci\'on (\ref{Eq.MF.1})-(\ref{Eq.MF.4})
es llamado flujo soluci\'on de la disciplina. Se dice que el
modelo de flujo l\'imite, modelo de flujo, de la disciplina de la
cola es estable si existe una constante $\delta>0$ que depende de
$\mu,\lambda$ y $P$ solamente, tal que cualquier flujo l\'imite
con
$|\overline{Q}\left(0\right)|+|\overline{U}|+|\overline{V}|=1$, se
tiene que $\overline{Q}\left(\cdot+\delta\right)\equiv0$.
\end{Def}

Al conjunto de ecuaciones dadas en \ref{Eq.MF.1}-\ref{Eq.MF.4} se
le llama {\em Modelo de flujo} y al conjunto de todas las
soluciones del modelo de flujo
$\left(\overline{Q}\left(\cdot\right),\overline{T}
\left(\cdot\right)\right)$ se le denotar\'a por $\mathcal{Q}$.

Si se hace $|x|\rightarrow\infty$ sin restringir ninguna de las
componentes, tambi\'en se obtienen un modelo de flujo, pero en
este caso el residual de los procesos de arribo y servicio
introducen un retraso:
\begin{Teo}[Teorema 4.2, Dai\cite{Dai}]\label{Tma.4.2.Dai}
Sea una disciplina fija para la cola, suponga que se cumplen las
condiciones (A1))-(A3)). Si el modelo de flujo l\'imite de la
disciplina de la cola es estable, entonces la cadena de Markov $X$
que describe la din\'amica de la red bajo la disciplina es Harris
recurrente positiva.
\end{Teo}

Ahora se procede a escalar el espacio y el tiempo para reducir la
aparente fluctuaci\'on del modelo. Consid\'erese el proceso
\begin{equation}\label{Eq.3.7}
\overline{Q}^{x}\left(t\right)=\frac{1}{|x|}Q^{x}\left(|x|t\right)
\end{equation}
A este proceso se le conoce como el fluido escalado, y cualquier
l\'imite $\overline{Q}^{x}\left(t\right)$ es llamado flujo
l\'imite del proceso de longitud de la cola. Haciendo
$|q|\rightarrow\infty$ mientras se mantiene el resto de las
componentes fijas, cualquier punto l\'imite del proceso de
longitud de la cola normalizado $\overline{Q}^{x}$ es soluci\'on
del siguiente modelo de flujo.


\begin{Def}[Definici\'on 3.3, Dai y Meyn \cite{DaiSean}]
El modelo de flujo es estable si existe un tiempo fijo $t_{0}$ tal
que $\overline{Q}\left(t\right)=0$, con $t\geq t_{0}$, para
cualquier $\overline{Q}\left(\cdot\right)\in\mathcal{Q}$ que
cumple con $|\overline{Q}\left(0\right)|=1$.
\end{Def}

El siguiente resultado se encuentra en Chen \cite{Chen}.
\begin{Lemma}[Lema 3.1, Dai y Meyn \cite{DaiSean}]
Si el modelo de flujo definido por \ref{Eq.MF.1}-\ref{Eq.MF.4} es
estable, entonces el modelo de flujo retrasado es tambi\'en
estable, es decir, existe $t_{0}>0$ tal que
$\overline{Q}\left(t\right)=0$ para cualquier $t\geq t_{0}$, para
cualquier soluci\'on del modelo de flujo retrasado cuya
condici\'on inicial $\overline{x}$ satisface que
$|\overline{x}|=|\overline{Q}\left(0\right)|+|\overline{A}\left(0\right)|+|\overline{B}\left(0\right)|\leq1$.
\end{Lemma}


Ahora ya estamos en condiciones de enunciar los resultados principales:


\begin{Teo}[Teorema 2.1, Down \cite{Down}]\label{Tma2.1.Down}
Suponga que el modelo de flujo es estable, y que se cumplen los supuestos (A1) y (A2), entonces
\begin{itemize}
\item[i)] Para alguna constante $\kappa_{p}$, y para cada
condici\'on inicial $x\in X$
\begin{equation}\label{Estability.Eq1}
limsup_{t\rightarrow\infty}\frac{1}{t}\int_{0}^{t}\esp_{x}\left[|Q\left(s\right)|^{p}\right]ds\leq\kappa_{p},
\end{equation}
donde $p$ es el entero dado en (A2).
\end{itemize}
Si adem\'as se cumple la condici\'on (A3), entonces para cada
condici\'on inicial:
\begin{itemize}
\item[ii)] Los momentos transitorios convergen a su estado
estacionario:
 \begin{equation}\label{Estability.Eq2}
lim_{t\rightarrow\infty}\esp_{x}\left[Q_{k}\left(t\right)^{r}\right]=\esp_{\pi}\left[Q_{k}\left(0\right)^{r}\right]\leq\kappa_{r},
\end{equation}
para $r=1,2,\ldots,p$ y $k=1,2,\ldots,K$. Donde $\pi$ es la
probabilidad invariante para $\mathbf{X}$.

\item[iii)]  El primer momento converge con raz\'on $t^{p-1}$:
\begin{equation}\label{Estability.Eq3}
lim_{t\rightarrow\infty}t^{p-1}|\esp_{x}\left[Q_{k}\left(t\right)\right]-\esp_{\pi}\left[Q_{k}\left(0\right)\right]=0.
\end{equation}

\item[iv)] La {\em Ley Fuerte de los grandes n\'umeros} se cumple:
\begin{equation}\label{Estability.Eq4}
lim_{t\rightarrow\infty}\frac{1}{t}\int_{0}^{t}Q_{k}^{r}\left(s\right)ds=\esp_{\pi}\left[Q_{k}\left(0\right)^{r}\right],\textrm{
}\prob_{x}\textrm{-c.s.}
\end{equation}
para $r=1,2,\ldots,p$ y $k=1,2,\ldots,K$.
\end{itemize}
\end{Teo}

La contribuci\'on de Down a la teor\'ia de los Sistemas de Visitas
C\'iclicas, es la relaci\'on que hay entre la estabilidad del
sistema con el comportamiento de las medidas de desempe\~no, es
decir, la condici\'on suficiente para poder garantizar la
convergencia del proceso de la longitud de la cola as\'i como de
por los menos los dos primeros momentos adem\'as de una versi\'on
de la Ley Fuerte de los Grandes N\'umeros para los sistemas de
visitas.


\begin{Teo}[Teorema 2.3, Down \cite{Down}]\label{Tma2.3.Down}
Considere el siguiente valor:
\begin{equation}\label{Eq.Rho.1serv}
\rho=\sum_{k=1}^{K}\rho_{k}+max_{1\leq j\leq K}\left(\frac{\lambda_{j}}{\sum_{s=1}^{S}p_{js}\overline{N}_{s}}\right)\delta^{*}
\end{equation}
\begin{itemize}
\item[i)] Si $\rho<1$ entonces la red es estable, es decir, se cumple el teorema \ref{Tma2.1.Down}.

\item[ii)] Si $\rho<1$ entonces la red es inestable, es decir, se cumple el teorema \ref{Tma2.2.Down}
\end{itemize}
\end{Teo}

\begin{Teo}
Sea $\left(X_{n},\mathcal{F}_{n},n=0,1,\ldots,\right\}$ Proceso de
Markov con espacio de estados $\left(S_{0},\chi_{0}\right)$
generado por una distribuici\'on inicial $P_{o}$ y probabilidad de
transici\'on $p_{mn}$, para $m,n=0,1,\ldots,$ $m<n$, que por
notaci\'on se escribir\'a como $p\left(m,n,x,B\right)\rightarrow
p_{mn}\left(x,B\right)$. Sea $S$ tiempo de paro relativo a la
$\sigma$-\'algebra $\mathcal{F}_{n}$. Sea $T$ funci\'on medible,
$T:\Omega\rightarrow\left\{0,1,\ldots,\right\}$. Sup\'ongase que
$T\geq S$, entonces $T$ es tiempo de paro. Si $B\in\chi_{0}$,
entonces
\begin{equation}\label{Prop.Fuerte.Markov}
P\left\{X\left(T\right)\in
B,T<\infty|\mathcal{F}\left(S\right)\right\} =
p\left(S,T,X\left(s\right),B\right)
\end{equation}
en $\left\{T<\infty\right\}$.
\end{Teo}


Sea $K$ conjunto numerable y sea $d:K\rightarrow\nat$ funci\'on.
Para $v\in K$, $M_{v}$ es un conjunto abierto de
$\rea^{d\left(v\right)}$. Entonces \[E=\cup_{v\in
K}M_{v}=\left\{\left(v,\zeta\right):v\in K,\zeta\in
M_{v}\right\}.\]

Sea $\mathcal{E}$ la clase de conjuntos medibles en $E$:
\[\mathcal{E}=\left\{\cup_{v\in K}A_{v}:A_{v}\in \mathcal{M}_{v}\right\}.\]

donde $\mathcal{M}$ son los conjuntos de Borel de $M_{v}$.
Entonces $\left(E,\mathcal{E}\right)$ es un espacio de Borel. El
estado del proceso se denotar\'a por
$\mathbf{x}_{t}=\left(v_{t},\zeta_{t}\right)$. La distribuci\'on
de $\left(\mathbf{x}_{t}\right)$ est\'a determinada por por los
siguientes objetos:

\begin{itemize}
\item[i)] Los campos vectoriales $\left(\mathcal{H}_{v},v\in
K\right)$. \item[ii)] Una funci\'on medible $\lambda:E\rightarrow
\rea_{+}$. \item[iii)] Una medida de transici\'on
$Q:\mathcal{E}\times\left(E\cup\Gamma^{*}\right)\rightarrow\left[0,1\right]$
donde
\begin{equation}
\Gamma^{*}=\cup_{v\in K}\partial^{*}M_{v}.
\end{equation}
y
\begin{equation}
\partial^{*}M_{v}=\left\{z\in\partial M_{v}:\mathbf{\mathbf{\phi}_{v}\left(t,\zeta\right)=\mathbf{z}}\textrm{ para alguna }\left(t,\zeta\right)\in\rea_{+}\times M_{v}\right\}.
\end{equation}
$\partial M_{v}$ denota  la frontera de $M_{v}$.
\end{itemize}

El campo vectorial $\left(\mathcal{H}_{v},v\in K\right)$ se supone
tal que para cada $\mathbf{z}\in M_{v}$ existe una \'unica curva
integral $\mathbf{\phi}_{v}\left(t,\zeta\right)$ que satisface la
ecuaci\'on

\begin{equation}
\frac{d}{dt}f\left(\zeta_{t}\right)=\mathcal{H}f\left(\zeta_{t}\right),
\end{equation}
con $\zeta_{0}=\mathbf{z}$, para cualquier funci\'on suave
$f:\rea^{d}\rightarrow\rea$ y $\mathcal{H}$ denota el operador
diferencial de primer orden, con $\mathcal{H}=\mathcal{H}_{v}$ y
$\zeta_{t}=\mathbf{\phi}\left(t,\mathbf{z}\right)$. Adem\'as se
supone que $\mathcal{H}_{v}$ es conservativo, es decir, las curvas
integrales est\'an definidas para todo $t>0$.

Para $\mathbf{x}=\left(v,\zeta\right)\in E$ se denota
\[t^{*}\mathbf{x}=inf\left\{t>0:\mathbf{\phi}_{v}\left(t,\zeta\right)\in\partial^{*}M_{v}\right\}\]

En lo que respecta a la funci\'on $\lambda$, se supondr\'a que
para cada $\left(v,\zeta\right)\in E$ existe un $\epsilon>0$ tal
que la funci\'on
$s\rightarrow\lambda\left(v,\phi_{v}\left(s,\zeta\right)\right)\in
E$ es integrable para $s\in\left[0,\epsilon\right)$. La medida de
transici\'on $Q\left(A;\mathbf{x}\right)$ es una funci\'on medible
de $\mathbf{x}$ para cada $A\in\mathcal{E}$, definida para
$\mathbf{x}\in E\cup\Gamma^{*}$ y es una medida de probabilidad en
$\left(E,\mathcal{E}\right)$ para cada $\mathbf{x}\in E$.

El movimiento del proceso $\left(\mathbf{x}_{t}\right)$ comenzando
en $\mathbf{x}=\left(n,\mathbf{z}\right)\in E$ se puede construir
de la siguiente manera, def\'inase la funci\'on $F$ por

\begin{equation}
F\left(t\right)=\left\{\begin{array}{ll}\\
exp\left(-\int_{0}^{t}\lambda\left(n,\phi_{n}\left(s,\mathbf{z}\right)\right)ds\right), & t<t^{*}\left(\mathbf{x}\right),\\
0, & t\geq t^{*}\left(\mathbf{x}\right)
\end{array}\right.
\end{equation}

Sea $T_{1}$ una variable aleatoria tal que
$\prob\left[T_{1}>t\right]=F\left(t\right)$, ahora sea la variable
aleatoria $\left(N,Z\right)$ con distribuici\'on
$Q\left(\cdot;\phi_{n}\left(T_{1},\mathbf{z}\right)\right)$. La
trayectoria de $\left(\mathbf{x}_{t}\right)$ para $t\leq T_{1}$
es\footnote{Revisar p\'agina 362, y 364 de Davis \cite{Davis}.}
\begin{eqnarray*}
\mathbf{x}_{t}=\left(v_{t},\zeta_{t}\right)=\left\{\begin{array}{ll}
\left(n,\phi_{n}\left(t,\mathbf{z}\right)\right), & t<T_{1},\\
\left(N,\mathbf{Z}\right), & t=t_{1}.
\end{array}\right.
\end{eqnarray*}

Comenzando en $\mathbf{x}_{T_{1}}$ se selecciona el siguiente
tiempo de intersalto $T_{2}-T_{1}$ lugar del post-salto
$\mathbf{x}_{T_{2}}$ de manera similar y as\'i sucesivamente. Este
procedimiento nos da una trayectoria determinista por partes
$\mathbf{x}_{t}$ con tiempos de salto $T_{1},T_{2},\ldots$. Bajo
las condiciones enunciadas para $\lambda,T_{1}>0$  y
$T_{1}-T_{2}>0$ para cada $i$, con probabilidad 1. Se supone que
se cumple la siguiente condici\'on.

\begin{Sup}[Supuesto 3.1, Davis \cite{Davis}]\label{Sup3.1.Davis}
Sea $N_{t}:=\sum_{t}\indora_{\left(t\geq t\right)}$ el n\'umero de
saltos en $\left[0,t\right]$. Entonces
\begin{equation}
\esp\left[N_{t}\right]<\infty\textrm{ para toda }t.
\end{equation}
\end{Sup}

es un proceso de Markov, m\'as a\'un, es un Proceso Fuerte de
Markov, es decir, la Propiedad Fuerte de Markov se cumple para
cualquier tiempo de paro.


Sea $E$ es un espacio m\'etrico separable y la m\'etrica $d$ es
compatible con la topolog\'ia.


\begin{Def}
Un espacio topol\'ogico $E$ es llamado de {\em Rad\'on} si es
homeomorfo a un subconjunto universalmente medible de un espacio
m\'etrico compacto.
\end{Def}

Equivalentemente, la definici\'on de un espacio de Rad\'on puede
encontrarse en los siguientes t\'erminos:


\begin{Def}
$E$ es un espacio de Rad\'on si cada medida finita en
$\left(E,\mathcal{B}\left(E\right)\right)$ es regular interior o
cerrada, {\em tight}.
\end{Def}

\begin{Def}
Una medida finita, $\lambda$ en la $\sigma$-\'algebra de Borel de
un espacio metrizable $E$ se dice cerrada si
\begin{equation}\label{Eq.A2.3}
\lambda\left(E\right)=sup\left\{\lambda\left(K\right):K\textrm{ es
compacto en }E\right\}.
\end{equation}
\end{Def}

El siguiente teorema nos permite tener una mejor caracterizaci\'on
de los espacios de Rad\'on:
\begin{Teo}\label{Tma.A2.2}
Sea $E$ espacio separable metrizable. Entonces $E$ es Radoniano si
y s\'olo s\'i cada medida finita en
$\left(E,\mathcal{B}\left(E\right)\right)$ es cerrada.
\end{Teo}

Sea $E$ espacio de estados, tal que $E$ es un espacio de Rad\'on,
$\mathcal{B}\left(E\right)$ $\sigma$-\'algebra de Borel en $E$,
que se denotar\'a por $\mathcal{E}$.

Sea $\left(X,\mathcal{G},\prob\right)$ espacio de probabilidad,
$I\subset\rea$ conjunto de \'indices. Sea $\mathcal{F}_{\leq t}$
la $\sigma$-\'algebra natural definida como
$\sigma\left\{f\left(X_{r}\right):r\in I, r\leq
t,f\in\mathcal{E}\right\}$. Se considerar\'a una
$\sigma$-\'algebra m\'as general, $ \left(\mathcal{G}_{t}\right)$
tal que $\left(X_{t}\right)$ sea $\mathcal{E}$-adaptado.

\begin{Def}
Una familia $\left(P_{s,t}\right)$ de kernels de Markov en
$\left(E,\mathcal{E}\right)$ indexada por pares $s,t\in I$, con
$s\leq t$ es una funci\'on de transici\'on en $\ER$, si  para todo
$r\leq s< t$ en $I$ y todo $x\in E$,
$B\in\mathcal{E}$\footnote{Ecuaci\'on de Chapman-Kolmogorov}
\begin{equation}\label{Eq.Kernels}
P_{r,t}\left(x,B\right)=\int_{E}P_{r,s}\left(x,dy\right)P_{s,t}\left(y,B\right).
\end{equation}
\end{Def}

Se dice que la funci\'on de transici\'on $\KM$ en $\ER$ es la
funci\'on de transici\'on para un proceso $\PE$  con valores en
$E$ y que satisface la propiedad de
Markov\footnote{\begin{equation}\label{Eq.1.4.S}
\prob\left\{H|\mathcal{G}_{t}\right\}=\prob\left\{H|X_{t}\right\}\textrm{
}H\in p\mathcal{F}_{\geq t}.
\end{equation}} (\ref{Eq.1.4.S}) relativa a $\left(\mathcal{G}_{t}\right)$ si

\begin{equation}\label{Eq.1.6.S}
\prob\left\{f\left(X_{t}\right)|\mathcal{G}_{s}\right\}=P_{s,t}f\left(X_{t}\right)\textrm{
}s\leq t\in I,\textrm{ }f\in b\mathcal{E}.
\end{equation}

\begin{Def}
Una familia $\left(P_{t}\right)_{t\geq0}$ de kernels de Markov en
$\ER$ es llamada {\em Semigrupo de Transici\'on de Markov} o {\em
Semigrupo de Transici\'on} si
\[P_{t+s}f\left(x\right)=P_{t}\left(P_{s}f\right)\left(x\right),\textrm{ }t,s\geq0,\textrm{ }x\in E\textrm{ }f\in b\mathcal{E}.\]
\end{Def}
\begin{Note}
Si la funci\'on de transici\'on $\KM$ es llamada homog\'enea si
$P_{s,t}=P_{t-s}$.
\end{Note}

Un proceso de Markov que satisface la ecuaci\'on (\ref{Eq.1.6.S})
con funci\'on de transici\'on homog\'enea $\left(P_{t}\right)$
tiene la propiedad caracter\'istica
\begin{equation}\label{Eq.1.8.S}
\prob\left\{f\left(X_{t+s}\right)|\mathcal{G}_{t}\right\}=P_{s}f\left(X_{t}\right)\textrm{
}t,s\geq0,\textrm{ }f\in b\mathcal{E}.
\end{equation}
La ecuaci\'on anterior es la {\em Propiedad Simple de Markov} de
$X$ relativa a $\left(P_{t}\right)$.

En este sentido el proceso $\PE$ cumple con la propiedad de Markov
(\ref{Eq.1.8.S}) relativa a
$\left(\Omega,\mathcal{G},\mathcal{G}_{t},\prob\right)$ con
semigrupo de transici\'on $\left(P_{t}\right)$.

\begin{Def}
Un proceso estoc\'astico $\PE$ definido en
$\left(\Omega,\mathcal{G},\prob\right)$ con valores en el espacio
topol\'ogico $E$ es continuo por la derecha si cada trayectoria
muestral $t\rightarrow X_{t}\left(w\right)$ es un mapeo continuo
por la derecha de $I$ en $E$.
\end{Def}

\begin{Def}[HD1]\label{Eq.2.1.S}
Un semigrupo de Markov $\left(P_{t}\right)$ en un espacio de
Rad\'on $E$ se dice que satisface la condici\'on {\em HD1} si,
dada una medida de probabilidad $\mu$ en $E$, existe una
$\sigma$-\'algebra $\mathcal{E^{*}}$ con
$\mathcal{E}\subset\mathcal{E}^{*}$ y
$P_{t}\left(b\mathcal{E}^{*}\right)\subset b\mathcal{E}^{*}$, y un
$\mathcal{E}^{*}$-proceso $E$-valuado continuo por la derecha
$\PE$ en alg\'un espacio de probabilidad filtrado
$\left(\Omega,\mathcal{G},\mathcal{G}_{t},\prob\right)$ tal que
$X=\left(\Omega,\mathcal{G},\mathcal{G}_{t},\prob\right)$ es de
Markov (Homog\'eneo) con semigrupo de transici\'on $(P_{t})$ y
distribuci\'on inicial $\mu$.
\end{Def}

Consid\'erese la colecci\'on de variables aleatorias $X_{t}$
definidas en alg\'un espacio de probabilidad, y una colecci\'on de
medidas $\mathbf{P}^{x}$ tales que
$\mathbf{P}^{x}\left\{X_{0}=x\right\}$, y bajo cualquier
$\mathbf{P}^{x}$, $X_{t}$ es de Markov con semigrupo
$\left(P_{t}\right)$. $\mathbf{P}^{x}$ puede considerarse como la
distribuci\'on condicional de $\mathbf{P}$ dado $X_{0}=x$.

\begin{Def}\label{Def.2.2.S}
Sea $E$ espacio de Rad\'on, $\SG$ semigrupo de Markov en $\ER$. La
colecci\'on
$\mathbf{X}=\left(\Omega,\mathcal{G},\mathcal{G}_{t},X_{t},\theta_{t},\CM\right)$
es un proceso $\mathcal{E}$-Markov continuo por la derecha simple,
con espacio de estados $E$ y semigrupo de transici\'on $\SG$ en
caso de que $\mathbf{X}$ satisfaga las siguientes condiciones:
\begin{itemize}
\item[i)] $\left(\Omega,\mathcal{G},\mathcal{G}_{t}\right)$ es un
espacio de medida filtrado, y $X_{t}$ es un proceso $E$-valuado
continuo por la derecha $\mathcal{E}^{*}$-adaptado a
$\left(\mathcal{G}_{t}\right)$;

\item[ii)] $\left(\theta_{t}\right)_{t\geq0}$ es una colecci\'on
de operadores {\em shift} para $X$, es decir, mapea $\Omega$ en
s\'i mismo satisfaciendo para $t,s\geq0$,

\begin{equation}\label{Eq.Shift}
\theta_{t}\circ\theta_{s}=\theta_{t+s}\textrm{ y
}X_{t}\circ\theta_{t}=X_{t+s};
\end{equation}

\item[iii)] Para cualquier $x\in E$,$\CM\left\{X_{0}=x\right\}=1$,
y el proceso $\PE$ tiene la propiedad de Markov (\ref{Eq.1.8.S})
con semigrupo de transici\'on $\SG$ relativo a
$\left(\Omega,\mathcal{G},\mathcal{G}_{t},\CM\right)$.
\end{itemize}
\end{Def}

\begin{Def}[HD2]\label{Eq.2.2.S}
Para cualquier $\alpha>0$ y cualquier $f\in S^{\alpha}$, el
proceso $t\rightarrow f\left(X_{t}\right)$ es continuo por la
derecha casi seguramente.
\end{Def}

\begin{Def}\label{Def.PD}
Un sistema
$\mathbf{X}=\left(\Omega,\mathcal{G},\mathcal{G}_{t},X_{t},\theta_{t},\CM\right)$
es un proceso derecho en el espacio de Rad\'on $E$ con semigrupo
de transici\'on $\SG$ provisto de:
\begin{itemize}
\item[i)] $\mathbf{X}$ es una realizaci\'on  continua por la
derecha, \ref{Def.2.2.S}, de $\SG$.

\item[ii)] $\mathbf{X}$ satisface la condicion HD2,
\ref{Eq.2.2.S}, relativa a $\mathcal{G}_{t}$.

\item[iii)] $\mathcal{G}_{t}$ es aumentado y continuo por la
derecha.
\end{itemize}
\end{Def}

\begin{Lema}[Lema 4.2, Dai\cite{Dai}]\label{Lema4.2}
Sea $\left\{x_{n}\right\}\subset \mathbf{X}$ con
$|x_{n}|\rightarrow\infty$, conforme $n\rightarrow\infty$. Suponga
que
\[lim_{n\rightarrow\infty}\frac{1}{|x_{n}|}U\left(0\right)=\overline{U}\]
y
\[lim_{n\rightarrow\infty}\frac{1}{|x_{n}|}V\left(0\right)=\overline{V}.\]

Entonces, conforme $n\rightarrow\infty$, casi seguramente

\begin{equation}\label{E1.4.2}
\frac{1}{|x_{n}|}\Phi^{k}\left(\left[|x_{n}|t\right]\right)\rightarrow
P_{k}^{'}t\textrm{, u.o.c.,}
\end{equation}

\begin{equation}\label{E1.4.3}
\frac{1}{|x_{n}|}E^{x_{n}}_{k}\left(|x_{n}|t\right)\rightarrow
\alpha_{k}\left(t-\overline{U}_{k}\right)^{+}\textrm{, u.o.c.,}
\end{equation}

\begin{equation}\label{E1.4.4}
\frac{1}{|x_{n}|}S^{x_{n}}_{k}\left(|x_{n}|t\right)\rightarrow
\mu_{k}\left(t-\overline{V}_{k}\right)^{+}\textrm{, u.o.c.,}
\end{equation}

donde $\left[t\right]$ es la parte entera de $t$ y
$\mu_{k}=1/m_{k}=1/\esp\left[\eta_{k}\left(1\right)\right]$.
\end{Lema}

\begin{Lema}[Lema 4.3, Dai\cite{Dai}]\label{Lema.4.3}
Sea $\left\{x_{n}\right\}\subset \mathbf{X}$ con
$|x_{n}|\rightarrow\infty$, conforme $n\rightarrow\infty$. Suponga
que
\[lim_{n\rightarrow\infty}\frac{1}{|x_{n}|}U\left(0\right)=\overline{U}_{k}\]
y
\[lim_{n\rightarrow\infty}\frac{1}{|x_{n}|}V\left(0\right)=\overline{V}_{k}.\]
\begin{itemize}
\item[a)] Conforme $n\rightarrow\infty$ casi seguramente,
\[lim_{n\rightarrow\infty}\frac{1}{|x_{n}|}U^{x_{n}}_{k}\left(|x_{n}|t\right)=\left(\overline{U}_{k}-t\right)^{+}\textrm{, u.o.c.}\]
y
\[lim_{n\rightarrow\infty}\frac{1}{|x_{n}|}V^{x_{n}}_{k}\left(|x_{n}|t\right)=\left(\overline{V}_{k}-t\right)^{+}.\]

\item[b)] Para cada $t\geq0$ fijo,
\[\left\{\frac{1}{|x_{n}|}U^{x_{n}}_{k}\left(|x_{n}|t\right),|x_{n}|\geq1\right\}\]
y
\[\left\{\frac{1}{|x_{n}|}V^{x_{n}}_{k}\left(|x_{n}|t\right),|x_{n}|\geq1\right\}\]
\end{itemize}
son uniformemente convergentes.
\end{Lema}

$S_{l}^{x}\left(t\right)$ es el n\'umero total de servicios
completados de la clase $l$, si la clase $l$ est\'a dando $t$
unidades de tiempo de servicio. Sea $T_{l}^{x}\left(x\right)$ el
monto acumulado del tiempo de servicio que el servidor
$s\left(l\right)$ gasta en los usuarios de la clase $l$ al tiempo
$t$. Entonces $S_{l}^{x}\left(T_{l}^{x}\left(t\right)\right)$ es
el n\'umero total de servicios completados para la clase $l$ al
tiempo $t$. Una fracci\'on de estos usuarios,
$\Phi_{l}^{x}\left(S_{l}^{x}\left(T_{l}^{x}\left(t\right)\right)\right)$,
se convierte en usuarios de la clase $k$.\\

Entonces, dado lo anterior, se tiene la siguiente representaci\'on
para el proceso de la longitud de la cola:\\

\begin{equation}
Q_{k}^{x}\left(t\right)=_{k}^{x}\left(0\right)+E_{k}^{x}\left(t\right)+\sum_{l=1}^{K}\Phi_{k}^{l}\left(S_{l}^{x}\left(T_{l}^{x}\left(t\right)\right)\right)-S_{k}^{x}\left(T_{k}^{x}\left(t\right)\right)
\end{equation}
para $k=1,\ldots,K$. Para $i=1,\ldots,d$, sea
\[I_{i}^{x}\left(t\right)=t-\sum_{j\in C_{i}}T_{k}^{x}\left(t\right).\]

Entonces $I_{i}^{x}\left(t\right)$ es el monto acumulado del
tiempo que el servidor $i$ ha estado desocupado al tiempo $t$. Se
est\'a asumiendo que las disciplinas satisfacen la ley de
conservaci\'on del trabajo, es decir, el servidor $i$ est\'a en
pausa solamente cuando no hay usuarios en la estaci\'on $i$.
Entonces, se tiene que

\begin{equation}
\int_{0}^{\infty}\left(\sum_{k\in
C_{i}}Q_{k}^{x}\left(t\right)\right)dI_{i}^{x}\left(t\right)=0,
\end{equation}
para $i=1,\ldots,d$.\\

Hacer
\[T^{x}\left(t\right)=\left(T_{1}^{x}\left(t\right),\ldots,T_{K}^{x}\left(t\right)\right)^{'},\]
\[I^{x}\left(t\right)=\left(I_{1}^{x}\left(t\right),\ldots,I_{K}^{x}\left(t\right)\right)^{'}\]
y
\[S^{x}\left(T^{x}\left(t\right)\right)=\left(S_{1}^{x}\left(T_{1}^{x}\left(t\right)\right),\ldots,S_{K}^{x}\left(T_{K}^{x}\left(t\right)\right)\right)^{'}.\]

Para una disciplina que cumple con la ley de conservaci\'on del
trabajo, en forma vectorial, se tiene el siguiente conjunto de
ecuaciones

\begin{equation}\label{Eq.MF.1.3}
Q^{x}\left(t\right)=Q^{x}\left(0\right)+E^{x}\left(t\right)+\sum_{l=1}^{K}\Phi^{l}\left(S_{l}^{x}\left(T_{l}^{x}\left(t\right)\right)\right)-S^{x}\left(T^{x}\left(t\right)\right),\\
\end{equation}

\begin{equation}\label{Eq.MF.2.3}
Q^{x}\left(t\right)\geq0,\\
\end{equation}

\begin{equation}\label{Eq.MF.3.3}
T^{x}\left(0\right)=0,\textrm{ y }\overline{T}^{x}\left(t\right)\textrm{ es no decreciente},\\
\end{equation}

\begin{equation}\label{Eq.MF.4.3}
I^{x}\left(t\right)=et-CT^{x}\left(t\right)\textrm{ es no
decreciente}\\
\end{equation}

\begin{equation}\label{Eq.MF.5.3}
\int_{0}^{\infty}\left(CQ^{x}\left(t\right)\right)dI_{i}^{x}\left(t\right)=0,\\
\end{equation}

\begin{equation}\label{Eq.MF.6.3}
\textrm{Condiciones adicionales en
}\left(\overline{Q}^{x}\left(\cdot\right),\overline{T}^{x}\left(\cdot\right)\right)\textrm{
espec\'ificas de la disciplina de la cola,}
\end{equation}

donde $e$ es un vector de unos de dimensi\'on $d$, $C$ es la
matriz definida por
\[C_{ik}=\left\{\begin{array}{cc}
1,& S\left(k\right)=i,\\
0,& \textrm{ en otro caso}.\\
\end{array}\right.
\]
Es necesario enunciar el siguiente Teorema que se utilizar\'a para
el Teorema \ref{Tma.4.2.Dai}:
\begin{Teo}[Teorema 4.1, Dai \cite{Dai}]
Considere una disciplina que cumpla la ley de conservaci\'on del
trabajo, para casi todas las trayectorias muestrales $\omega$ y
cualquier sucesi\'on de estados iniciales
$\left\{x_{n}\right\}\subset \mathbf{X}$, con
$|x_{n}|\rightarrow\infty$, existe una subsucesi\'on
$\left\{x_{n_{j}}\right\}$ con $|x_{n_{j}}|\rightarrow\infty$ tal
que
\begin{equation}\label{Eq.4.15}
\frac{1}{|x_{n_{j}}|}\left(Q^{x_{n_{j}}}\left(0\right),U^{x_{n_{j}}}\left(0\right),V^{x_{n_{j}}}\left(0\right)\right)\rightarrow\left(\overline{Q}\left(0\right),\overline{U},\overline{V}\right),
\end{equation}

\begin{equation}\label{Eq.4.16}
\frac{1}{|x_{n_{j}}|}\left(Q^{x_{n_{j}}}\left(|x_{n_{j}}|t\right),T^{x_{n_{j}}}\left(|x_{n_{j}}|t\right)\right)\rightarrow\left(\overline{Q}\left(t\right),\overline{T}\left(t\right)\right)\textrm{
u.o.c.}
\end{equation}

Adem\'as,
$\left(\overline{Q}\left(t\right),\overline{T}\left(t\right)\right)$
satisface las siguientes ecuaciones:
\begin{equation}\label{Eq.MF.1.3a}
\overline{Q}\left(t\right)=Q\left(0\right)+\left(\alpha
t-\overline{U}\right)^{+}-\left(I-P\right)^{'}M^{-1}\left(\overline{T}\left(t\right)-\overline{V}\right)^{+},
\end{equation}

\begin{equation}\label{Eq.MF.2.3a}
\overline{Q}\left(t\right)\geq0,\\
\end{equation}

\begin{equation}\label{Eq.MF.3.3a}
\overline{T}\left(t\right)\textrm{ es no decreciente y comienza en cero},\\
\end{equation}

\begin{equation}\label{Eq.MF.4.3a}
\overline{I}\left(t\right)=et-C\overline{T}\left(t\right)\textrm{
es no decreciente,}\\
\end{equation}

\begin{equation}\label{Eq.MF.5.3a}
\int_{0}^{\infty}\left(C\overline{Q}\left(t\right)\right)d\overline{I}\left(t\right)=0,\\
\end{equation}

\begin{equation}\label{Eq.MF.6.3a}
\textrm{Condiciones adicionales en
}\left(\overline{Q}\left(\cdot\right),\overline{T}\left(\cdot\right)\right)\textrm{
especficas de la disciplina de la cola,}
\end{equation}
\end{Teo}


Propiedades importantes para el modelo de flujo retrasado:

\begin{Prop}
 Sea $\left(\overline{Q},\overline{T},\overline{T}^{0}\right)$ un flujo l\'imite de \ref{Eq.4.4} y suponga que cuando $x\rightarrow\infty$ a lo largo de
una subsucesi\'on
\[\left(\frac{1}{|x|}Q_{k}^{x}\left(0\right),\frac{1}{|x|}A_{k}^{x}\left(0\right),\frac{1}{|x|}B_{k}^{x}\left(0\right),\frac{1}{|x|}B_{k}^{x,0}\left(0\right)\right)\rightarrow\left(\overline{Q}_{k}\left(0\right),0,0,0\right)\]
para $k=1,\ldots,K$. EL flujo l\'imite tiene las siguientes
propiedades, donde las propiedades de la derivada se cumplen donde
la derivada exista:
\begin{itemize}
 \item[i)] Los vectores de tiempo ocupado $\overline{T}\left(t\right)$ y $\overline{T}^{0}\left(t\right)$ son crecientes y continuas con
$\overline{T}\left(0\right)=\overline{T}^{0}\left(0\right)=0$.
\item[ii)] Para todo $t\geq0$
\[\sum_{k=1}^{K}\left[\overline{T}_{k}\left(t\right)+\overline{T}_{k}^{0}\left(t\right)\right]=t\]
\item[iii)] Para todo $1\leq k\leq K$
\[\overline{Q}_{k}\left(t\right)=\overline{Q}_{k}\left(0\right)+\alpha_{k}t-\mu_{k}\overline{T}_{k}\left(t\right)\]
\item[iv)]  Para todo $1\leq k\leq K$
\[\dot{{\overline{T}}}_{k}\left(t\right)=\beta_{k}\] para $\overline{Q}_{k}\left(t\right)=0$.
\item[v)] Para todo $k,j$
\[\mu_{k}^{0}\overline{T}_{k}^{0}\left(t\right)=\mu_{j}^{0}\overline{T}_{j}^{0}\left(t\right)\]
\item[vi)]  Para todo $1\leq k\leq K$
\[\mu_{k}\dot{{\overline{T}}}_{k}\left(t\right)=l_{k}\mu_{k}^{0}\dot{{\overline{T}}}_{k}^{0}\left(t\right)\] para $\overline{Q}_{k}\left(t\right)>0$.
\end{itemize}
\end{Prop}

\begin{Lema}[Lema 3.1 \cite{Chen}]\label{Lema3.1}
Si el modelo de flujo es estable, definido por las ecuaciones
(3.8)-(3.13), entonces el modelo de flujo retrasado tambi\'en es
estable.
\end{Lema}

\begin{Teo}[Teorema 5.1 \cite{Chen}]\label{Tma.5.1.Chen}
La red de colas es estable si existe una constante $t_{0}$ que
depende de $\left(\alpha,\mu,T,U\right)$ y $V$ que satisfagan las
ecuaciones (5.1)-(5.5), $Z\left(t\right)=0$, para toda $t\geq
t_{0}$.
\end{Teo}



\begin{Lema}[Lema 5.2 \cite{Gut}]\label{Lema.5.2.Gut}
Sea $\left\{\xi\left(k\right):k\in\ent\right\}$ sucesi\'on de
variables aleatorias i.i.d. con valores en
$\left(0,\infty\right)$, y sea $E\left(t\right)$ el proceso de
conteo
\[E\left(t\right)=max\left\{n\geq1:\xi\left(1\right)+\cdots+\xi\left(n-1\right)\leq t\right\}.\]
Si $E\left[\xi\left(1\right)\right]<\infty$, entonces para
cualquier entero $r\geq1$
\begin{equation}
lim_{t\rightarrow\infty}\esp\left[\left(\frac{E\left(t\right)}{t}\right)^{r}\right]=\left(\frac{1}{E\left[\xi_{1}\right]}\right)^{r}
\end{equation}
de aqu\'i, bajo estas condiciones
\begin{itemize}
\item[a)] Para cualquier $t>0$,
$sup_{t\geq\delta}\esp\left[\left(\frac{E\left(t\right)}{t}\right)^{r}\right]$

\item[b)] Las variables aleatorias
$\left\{\left(\frac{E\left(t\right)}{t}\right)^{r}:t\geq1\right\}$
son uniformemente integrables.
\end{itemize}
\end{Lema}

\begin{Teo}[Teorema 5.1: Ley Fuerte para Procesos de Conteo
\cite{Gut}]\label{Tma.5.1.Gut} Sea
$0<\mu<\esp\left(X_{1}\right]\leq\infty$. entonces

\begin{itemize}
\item[a)] $\frac{N\left(t\right)}{t}\rightarrow\frac{1}{\mu}$
a.s., cuando $t\rightarrow\infty$.


\item[b)]$\esp\left[\frac{N\left(t\right)}{t}\right]^{r}\rightarrow\frac{1}{\mu^{r}}$,
cuando $t\rightarrow\infty$ para todo $r>0$..
\end{itemize}
\end{Teo}


\begin{Prop}[Proposici\'on 5.1 \cite{DaiSean}]\label{Prop.5.1}
Suponga que los supuestos (A1) y (A2) se cumplen, adem\'as suponga
que el modelo de flujo es estable. Entonces existe $t_{0}>0$ tal
que
\begin{equation}\label{Eq.Prop.5.1}
lim_{|x|\rightarrow\infty}\frac{1}{|x|^{p+1}}\esp_{x}\left[|X\left(t_{0}|x|\right)|^{p+1}\right]=0.
\end{equation}

\end{Prop}


\begin{Prop}[Proposici\'on 5.3 \cite{DaiSean}]
Sea $X$ proceso de estados para la red de colas, y suponga que se
cumplen los supuestos (A1) y (A2), entonces para alguna constante
positiva $C_{p+1}<\infty$, $\delta>0$ y un conjunto compacto
$C\subset X$.

\begin{equation}\label{Eq.5.4}
\esp_{x}\left[\int_{0}^{\tau_{C}\left(\delta\right)}\left(1+|X\left(t\right)|^{p}\right)dt\right]\leq
C_{p+1}\left(1+|x|^{p+1}\right)
\end{equation}
\end{Prop}

\begin{Prop}[Proposici\'on 5.4 \cite{DaiSean}]
Sea $X$ un proceso de Markov Borel Derecho en $X$, sea
$f:X\leftarrow\rea_{+}$ y defina para alguna $\delta>0$, y un
conjunto cerrado $C\subset X$
\[V\left(x\right):=\esp_{x}\left[\int_{0}^{\tau_{C}\left(\delta\right)}f\left(X\left(t\right)\right)dt\right]\]
para $x\in X$. Si $V$ es finito en todas partes y uniformemente
acotada en $C$, entonces existe $k<\infty$ tal que
\begin{equation}\label{Eq.5.11}
\frac{1}{t}\esp_{x}\left[V\left(x\right)\right]+\frac{1}{t}\int_{0}^{t}\esp_{x}\left[f\left(X\left(s\right)\right)ds\right]\leq\frac{1}{t}V\left(x\right)+k,
\end{equation}
para $x\in X$ y $t>0$.
\end{Prop}


\begin{Teo}[Teorema 5.5 \cite{DaiSean}]
Suponga que se cumplen (A1) y (A2), adem\'as suponga que el modelo
de flujo es estable. Entonces existe una constante $k_{p}<\infty$
tal que
\begin{equation}\label{Eq.5.13}
\frac{1}{t}\int_{0}^{t}\esp_{x}\left[|Q\left(s\right)|^{p}\right]ds\leq
k_{p}\left\{\frac{1}{t}|x|^{p+1}+1\right\}
\end{equation}
para $t\geq0$, $x\in X$. En particular para cada condici\'on
inicial
\begin{equation}\label{Eq.5.14}
Limsup_{t\rightarrow\infty}\frac{1}{t}\int_{0}^{t}\esp_{x}\left[|Q\left(s\right)|^{p}\right]ds\leq
k_{p}
\end{equation}
\end{Teo}

\begin{Teo}[Teorema 6.2 \cite{DaiSean}]\label{Tma.6.2}
Suponga que se cumplen los supuestos (A1)-(A3) y que el modelo de
flujo es estable, entonces se tiene que
\[\parallel P^{t}\left(c,\cdot\right)-\pi\left(\cdot\right)\parallel_{f_{p}}\rightarrow0\]
para $t\rightarrow\infty$ y $x\in X$. En particular para cada
condici\'on inicial
\[lim_{t\rightarrow\infty}\esp_{x}\left[\left|Q_{t}\right|^{p}\right]=\esp_{\pi}\left[\left|Q_{0}\right|^{p}\right]<\infty\]
\end{Teo}


\begin{Teo}[Teorema 6.3 \cite{DaiSean}]\label{Tma.6.3}
Suponga que se cumplen los supuestos (A1)-(A3) y que el modelo de
flujo es estable, entonces con
$f\left(x\right)=f_{1}\left(x\right)$, se tiene que
\[lim_{t\rightarrow\infty}t^{(p-1)\left|P^{t}\left(c,\cdot\right)-\pi\left(\cdot\right)\right|_{f}=0},\]
para $x\in X$. En particular, para cada condici\'on inicial
\[lim_{t\rightarrow\infty}t^{(p-1)}\left|\esp_{x}\left[Q_{t}\right]-\esp_{\pi}\left[Q_{0}\right]\right|=0.\]
\end{Teo}



\begin{Prop}[Proposici\'on 5.1, Dai y Meyn \cite{DaiSean}]\label{Prop.5.1.DaiSean}
Suponga que los supuestos A1) y A2) son ciertos y que el modelo de
flujo es estable. Entonces existe $t_{0}>0$ tal que
\begin{equation}
lim_{|x|\rightarrow\infty}\frac{1}{|x|^{p+1}}\esp_{x}\left[|X\left(t_{0}|x|\right)|^{p+1}\right]=0
\end{equation}
\end{Prop}

\begin{Lemma}[Lema 5.2, Dai y Meyn, \cite{DaiSean}]\label{Lema.5.2.DaiSean}
 Sea $\left\{\zeta\left(k\right):k\in \mathbb{z}\right\}$ una sucesi\'on independiente e id\'enticamente distribuida que toma valores en $\left(0,\infty\right)$,
y sea
$E\left(t\right)=max\left(n\geq1:\zeta\left(1\right)+\cdots+\zeta\left(n-1\right)\leq
t\right)$. Si $\esp\left[\zeta\left(1\right)\right]<\infty$,
entonces para cualquier entero $r\geq1$
\begin{equation}
 lim_{t\rightarrow\infty}\esp\left[\left(\frac{E\left(t\right)}{t}\right)^{r}\right]=\left(\frac{1}{\esp\left[\zeta_{1}\right]}\right)^{r}.
\end{equation}
Luego, bajo estas condiciones:
\begin{itemize}
 \item[a)] para cualquier $\delta>0$, $\sup_{t\geq\delta}\esp\left[\left(\frac{E\left(t\right)}{t}\right)^{r}\right]<\infty$
\item[b)] las variables aleatorias
$\left\{\left(\frac{E\left(t\right)}{t}\right)^{r}:t\geq1\right\}$
son uniformemente integrables.
\end{itemize}
\end{Lemma}

\begin{Teo}[Teorema 5.5, Dai y Meyn \cite{DaiSean}]\label{Tma.5.5.DaiSean}
Suponga que los supuestos A1) y A2) se cumplen y que el modelo de
flujo es estable. Entonces existe una constante $\kappa_{p}$ tal
que
\begin{equation}
\frac{1}{t}\int_{0}^{t}\esp_{x}\left[|Q\left(s\right)|^{p}\right]ds\leq\kappa_{p}\left\{\frac{1}{t}|x|^{p+1}+1\right\}
\end{equation}
para $t>0$ y $x\in X$. En particular, para cada condici\'on
inicial
\begin{eqnarray*}
\limsup_{t\rightarrow\infty}\frac{1}{t}\int_{0}^{t}\esp_{x}\left[|Q\left(s\right)|^{p}\right]ds\leq\kappa_{p}.
\end{eqnarray*}
\end{Teo}

\begin{Teo}[Teorema 6.2, Dai y Meyn \cite{DaiSean}]\label{Tma.6.2.DaiSean}
Suponga que se cumplen los supuestos A1), A2) y A3) y que el
modelo de flujo es estable. Entonces se tiene que
\begin{equation}
\left\|P^{t}\left(x,\cdot\right)-\pi\left(\cdot\right)\right\|_{f_{p}}\textrm{,
}t\rightarrow\infty,x\in X.
\end{equation}
En particular para cada condici\'on inicial
\begin{eqnarray*}
\lim_{t\rightarrow\infty}\esp_{x}\left[|Q\left(t\right)|^{p}\right]=\esp_{\pi}\left[|Q\left(0\right)|^{p}\right]\leq\kappa_{r}
\end{eqnarray*}
\end{Teo}
\begin{Teo}[Teorema 6.3, Dai y Meyn \cite{DaiSean}]\label{Tma.6.3.DaiSean}
Suponga que se cumplen los supuestos A1), A2) y A3) y que el
modelo de flujo es estable. Entonces con
$f\left(x\right)=f_{1}\left(x\right)$ se tiene
\begin{equation}
\lim_{t\rightarrow\infty}t^{p-1}\left\|P^{t}\left(x,\cdot\right)-\pi\left(\cdot\right)\right\|_{f}=0.
\end{equation}
En particular para cada condici\'on inicial
\begin{eqnarray*}
\lim_{t\rightarrow\infty}t^{p-1}|\esp_{x}\left[Q\left(t\right)\right]-\esp_{\pi}\left[Q\left(0\right)\right]|=0.
\end{eqnarray*}
\end{Teo}

\begin{Teo}[Teorema 6.4, Dai y Meyn, \cite{DaiSean}]\label{Tma.6.4.DaiSean}
Suponga que se cumplen los supuestos A1), A2) y A3) y que el
modelo de flujo es estable. Sea $\nu$ cualquier distribuci\'on de
probabilidad en $\left(X,\mathcal{B}_{X}\right)$, y $\pi$ la
distribuci\'on estacionaria de $X$.
\begin{itemize}
\item[i)] Para cualquier $f:X\leftarrow\rea_{+}$
\begin{equation}
\lim_{t\rightarrow\infty}\frac{1}{t}\int_{o}^{t}f\left(X\left(s\right)\right)ds=\pi\left(f\right):=\int
f\left(x\right)\pi\left(dx\right)
\end{equation}
$\prob$-c.s.

\item[ii)] Para cualquier $f:X\leftarrow\rea_{+}$ con
$\pi\left(|f|\right)<\infty$, la ecuaci\'on anterior se cumple.
\end{itemize}
\end{Teo}

\begin{Teo}[Teorema 2.2, Down \cite{Down}]\label{Tma2.2.Down}
Suponga que el fluido modelo es inestable en el sentido de que
para alguna $\epsilon_{0},c_{0}\geq0$,
\begin{equation}\label{Eq.Inestability}
|Q\left(T\right)|\geq\epsilon_{0}T-c_{0}\textrm{,   }T\geq0,
\end{equation}
para cualquier condici\'on inicial $Q\left(0\right)$, con
$|Q\left(0\right)|=1$. Entonces para cualquier $0<q\leq1$, existe
$B<0$ tal que para cualquier $|x|\geq B$,
\begin{equation}
\prob_{x}\left\{\mathbb{X}\rightarrow\infty\right\}\geq q.
\end{equation}
\end{Teo}



\begin{Def}
Sea $X$ un conjunto y $\mathcal{F}$ una $\sigma$-\'algebra de
subconjuntos de $X$, la pareja $\left(X,\mathcal{F}\right)$ es
llamado espacio medible. Un subconjunto $A$ de $X$ es llamado
medible, o medible con respecto a $\mathcal{F}$, si
$A\in\mathcal{F}$.
\end{Def}

\begin{Def}
Sea $\left(X,\mathcal{F},\mu\right)$ espacio de medida. Se dice
que la medida $\mu$ es $\sigma$-finita si se puede escribir
$X=\bigcup_{n\geq1}X_{n}$ con $X_{n}\in\mathcal{F}$ y
$\mu\left(X_{n}\right)<\infty$.
\end{Def}

\begin{Def}\label{Cto.Borel}
Sea $X$ el conjunto de los n\'umeros reales $\rea$. El \'algebra
de Borel es la $\sigma$-\'algebra $B$ generada por los intervalos
abiertos $\left(a,b\right)\in\rea$. Cualquier conjunto en $B$ es
llamado {\em Conjunto de Borel}.
\end{Def}

\begin{Def}\label{Funcion.Medible}
Una funci\'on $f:X\rightarrow\rea$, es medible si para cualquier
n\'umero real $\alpha$ el conjunto
\[\left\{x\in X:f\left(x\right)>\alpha\right\}\]
pertenece a $\mathcal{F}$. Equivalentemente, se dice que $f$ es
medible si
\[f^{-1}\left(\left(\alpha,\infty\right)\right)=\left\{x\in X:f\left(x\right)>\alpha\right\}\in\mathcal{F}.\]
\end{Def}


\begin{Def}\label{Def.Cilindros}
Sean $\left(\Omega_{i},\mathcal{F}_{i}\right)$, $i=1,2,\ldots,$
espacios medibles y $\Omega=\prod_{i=1}^{\infty}\Omega_{i}$ el
conjunto de todas las sucesiones
$\left(\omega_{1},\omega_{2},\ldots,\right)$ tales que
$\omega_{i}\in\Omega_{i}$, $i=1,2,\ldots,$. Si
$B^{n}\subset\prod_{i=1}^{\infty}\Omega_{i}$, definimos
$B_{n}=\left\{\omega\in\Omega:\left(\omega_{1},\omega_{2},\ldots,\omega_{n}\right)\in
B^{n}\right\}$. Al conjunto $B_{n}$ se le llama {\em cilindro} con
base $B^{n}$, el cilindro es llamado medible si
$B^{n}\in\prod_{i=1}^{\infty}\mathcal{F}_{i}$.
\end{Def}


\begin{Def}\label{Def.Proc.Adaptado}[TSP, Ash \cite{RBA}]
Sea $X\left(t\right),t\geq0$ proceso estoc\'astico, el proceso es
adaptado a la familia de $\sigma$-\'algebras $\mathcal{F}_{t}$,
para $t\geq0$, si para $s<t$ implica que
$\mathcal{F}_{s}\subset\mathcal{F}_{t}$, y $X\left(t\right)$ es
$\mathcal{F}_{t}$-medible para cada $t$. Si no se especifica
$\mathcal{F}_{t}$ entonces se toma $\mathcal{F}_{t}$ como
$\mathcal{F}\left(X\left(s\right),s\leq t\right)$, la m\'as
peque\~na $\sigma$-\'algebra de subconjuntos de $\Omega$ que hace
que cada $X\left(s\right)$, con $s\leq t$ sea Borel medible.
\end{Def}


\begin{Def}\label{Def.Tiempo.Paro}[TSP, Ash \cite{RBA}]
Sea $\left\{\mathcal{F}\left(t\right),t\geq0\right\}$ familia
creciente de sub $\sigma$-\'algebras. es decir,
$\mathcal{F}\left(s\right)\subset\mathcal{F}\left(t\right)$ para
$s\leq t$. Un tiempo de paro para $\mathcal{F}\left(t\right)$ es
una funci\'on $T:\Omega\rightarrow\left[0,\infty\right]$ tal que
$\left\{T\leq t\right\}\in\mathcal{F}\left(t\right)$ para cada
$t\geq0$. Un tiempo de paro para el proceso estoc\'astico
$X\left(t\right),t\geq0$ es un tiempo de paro para las
$\sigma$-\'algebras
$\mathcal{F}\left(t\right)=\mathcal{F}\left(X\left(s\right)\right)$.
\end{Def}

\begin{Def}
Sea $X\left(t\right),t\geq0$ proceso estoc\'astico, con
$\left(S,\chi\right)$ espacio de estados. Se dice que el proceso
es adaptado a $\left\{\mathcal{F}\left(t\right)\right\}$, es
decir, si para cualquier $s,t\in I$, $I$ conjunto de \'indices,
$s<t$, se tiene que
$\mathcal{F}\left(s\right)\subset\mathcal{F}\left(t\right)$ y
$X\left(t\right)$ es $\mathcal{F}\left(t\right)$-medible,
\end{Def}

\begin{Def}
Sea $X\left(t\right),t\geq0$ proceso estoc\'astico, se dice que es
un Proceso de Markov relativo a $\mathcal{F}\left(t\right)$ o que
$\left\{X\left(t\right),\mathcal{F}\left(t\right)\right\}$ es de
Markov si y s\'olo si para cualquier conjunto $B\in\chi$,  y
$s,t\in I$, $s<t$ se cumple que
\begin{equation}\label{Prop.Markov}
P\left\{X\left(t\right)\in
B|\mathcal{F}\left(s\right)\right\}=P\left\{X\left(t\right)\in
B|X\left(s\right)\right\}.
\end{equation}
\end{Def}
\begin{Note}
Si se dice que $\left\{X\left(t\right)\right\}$ es un Proceso de
Markov sin mencionar $\mathcal{F}\left(t\right)$, se asumir\'a que
\begin{eqnarray*}
\mathcal{F}\left(t\right)=\mathcal{F}_{0}\left(t\right)=\mathcal{F}\left(X\left(r\right),r\leq
t\right),
\end{eqnarray*}
entonces la ecuaci\'on (\ref{Prop.Markov}) se puede escribir como
\begin{equation}
P\left\{X\left(t\right)\in B|X\left(r\right),r\leq s\right\} =
P\left\{X\left(t\right)\in B|X\left(s\right)\right\}
\end{equation}
\end{Note}
%_______________________________________________________________
\subsection{Procesos de Estados de Markov}
%_______________________________________________________________

\begin{Teo}
Sea $\left(X_{n},\mathcal{F}_{n},n=0,1,\ldots,\right\}$ Proceso de
Markov con espacio de estados $\left(S_{0},\chi_{0}\right)$
generado por una distribuici\'on inicial $P_{o}$ y probabilidad de
transici\'on $p_{mn}$, para $m,n=0,1,\ldots,$ $m<n$, que por
notaci\'on se escribir\'a como $p\left(m,n,x,B\right)\rightarrow
p_{mn}\left(x,B\right)$. Sea $S$ tiempo de paro relativo a la
$\sigma$-\'algebra $\mathcal{F}_{n}$. Sea $T$ funci\'on medible,
$T:\Omega\rightarrow\left\{0,1,\ldots,\right\}$. Sup\'ongase que
$T\geq S$, entonces $T$ es tiempo de paro. Si $B\in\chi_{0}$,
entonces
\begin{equation}\label{Prop.Fuerte.Markov}
P\left\{X\left(T\right)\in
B,T<\infty|\mathcal{F}\left(S\right)\right\} =
p\left(S,T,X\left(s\right),B\right)
\end{equation}
en $\left\{T<\infty\right\}$.
\end{Teo}


Sea $K$ conjunto numerable y sea $d:K\rightarrow\nat$ funci\'on.
Para $v\in K$, $M_{v}$ es un conjunto abierto de
$\rea^{d\left(v\right)}$. Entonces \[E=\bigcup_{v\in
K}M_{v}=\left\{\left(v,\zeta\right):v\in K,\zeta\in
M_{v}\right\}.\]

Sea $\mathcal{E}$ la clase de conjuntos medibles en $E$:
\[\mathcal{E}=\left\{\bigcup_{v\in K}A_{v}:A_{v}\in \mathcal{M}_{v}\right\}.\]

donde $\mathcal{M}$ son los conjuntos de Borel de $M_{v}$.
Entonces $\left(E,\mathcal{E}\right)$ es un espacio de Borel. El
estado del proceso se denotar\'a por
$\mathbf{x}_{t}=\left(v_{t},\zeta_{t}\right)$. La distribuci\'on
de $\left(\mathbf{x}_{t}\right)$ est\'a determinada por por los
siguientes objetos:

\begin{itemize}
\item[i)] Los campos vectoriales $\left(\mathcal{H}_{v},v\in
K\right)$. \item[ii)] Una funci\'on medible $\lambda:E\rightarrow
\rea_{+}$. \item[iii)] Una medida de transici\'on
$Q:\mathcal{E}\times\left(E\cup\Gamma^{*}\right)\rightarrow\left[0,1\right]$
donde
\begin{equation}
\Gamma^{*}=\bigcup_{v\in K}\partial^{*}M_{v}.
\end{equation}
y
\begin{equation}
\partial^{*}M_{v}=\left\{z\in\partial M_{v}:\mathbf{\mathbf{\phi}_{v}\left(t,\zeta\right)=\mathbf{z}}\textrm{ para alguna }\left(t,\zeta\right)\in\rea_{+}\times M_{v}\right\}.
\end{equation}
$\partial M_{v}$ denota  la frontera de $M_{v}$.
\end{itemize}

El campo vectorial $\left(\mathcal{H}_{v},v\in K\right)$ se supone
tal que para cada $\mathbf{z}\in M_{v}$ existe una \'unica curva
integral $\mathbf{\phi}_{v}\left(t,\zeta\right)$ que satisface la
ecuaci\'on

\begin{equation}
\frac{d}{dt}f\left(\zeta_{t}\right)=\mathcal{H}f\left(\zeta_{t}\right),
\end{equation}
con $\zeta_{0}=\mathbf{z}$, para cualquier funci\'on suave
$f:\rea^{d}\rightarrow\rea$ y $\mathcal{H}$ denota el operador
diferencial de primer orden, con $\mathcal{H}=\mathcal{H}_{v}$ y
$\zeta_{t}=\mathbf{\phi}\left(t,\mathbf{z}\right)$. Adem\'as se
supone que $\mathcal{H}_{v}$ es conservativo, es decir, las curvas
integrales est\'an definidas para todo $t>0$.

Para $\mathbf{x}=\left(v,\zeta\right)\in E$ se denota
\[t^{*}\mathbf{x}=inf\left\{t>0:\mathbf{\phi}_{v}\left(t,\zeta\right)\in\partial^{*}M_{v}\right\}\]

En lo que respecta a la funci\'on $\lambda$, se supondr\'a que
para cada $\left(v,\zeta\right)\in E$ existe un $\epsilon>0$ tal
que la funci\'on
$s\rightarrow\lambda\left(v,\phi_{v}\left(s,\zeta\right)\right)\in
E$ es integrable para $s\in\left[0,\epsilon\right)$. La medida de
transici\'on $Q\left(A;\mathbf{x}\right)$ es una funci\'on medible
de $\mathbf{x}$ para cada $A\in\mathcal{E}$, definida para
$\mathbf{x}\in E\cup\Gamma^{*}$ y es una medida de probabilidad en
$\left(E,\mathcal{E}\right)$ para cada $\mathbf{x}\in E$.

El movimiento del proceso $\left(\mathbf{x}_{t}\right)$ comenzando
en $\mathbf{x}=\left(n,\mathbf{z}\right)\in E$ se puede construir
de la siguiente manera, def\'inase la funci\'on $F$ por

\begin{equation}
F\left(t\right)=\left\{\begin{array}{ll}\\
exp\left(-\int_{0}^{t}\lambda\left(n,\phi_{n}\left(s,\mathbf{z}\right)\right)ds\right), & t<t^{*}\left(\mathbf{x}\right),\\
0, & t\geq t^{*}\left(\mathbf{x}\right)
\end{array}\right.
\end{equation}

Sea $T_{1}$ una variable aleatoria tal que
$\prob\left[T_{1}>t\right]=F\left(t\right)$, ahora sea la variable
aleatoria $\left(N,Z\right)$ con distribuici\'on
$Q\left(\cdot;\phi_{n}\left(T_{1},\mathbf{z}\right)\right)$. La
trayectoria de $\left(\mathbf{x}_{t}\right)$ para $t\leq T_{1}$ es
\begin{eqnarray*}
\mathbf{x}_{t}=\left(v_{t},\zeta_{t}\right)=\left\{\begin{array}{ll}
\left(n,\phi_{n}\left(t,\mathbf{z}\right)\right), & t<T_{1},\\
\left(N,\mathbf{Z}\right), & t=t_{1}.
\end{array}\right.
\end{eqnarray*}

Comenzando en $\mathbf{x}_{T_{1}}$ se selecciona el siguiente
tiempo de intersalto $T_{2}-T_{1}$ lugar del post-salto
$\mathbf{x}_{T_{2}}$ de manera similar y as\'i sucesivamente. Este
procedimiento nos da una trayectoria determinista por partes
$\mathbf{x}_{t}$ con tiempos de salto $T_{1},T_{2},\ldots$. Bajo
las condiciones enunciadas para $\lambda,T_{1}>0$  y
$T_{1}-T_{2}>0$ para cada $i$, con probabilidad 1. Se supone que
se cumple la siguiente condici\'on.

\begin{Sup}[Supuesto 3.1, Davis \cite{Davis}]\label{Sup3.1.Davis}
Sea $N_{t}:=\sum_{t}\indora_{\left(t\geq t\right)}$ el n\'umero de
saltos en $\left[0,t\right]$. Entonces
\begin{equation}
\esp\left[N_{t}\right]<\infty\textrm{ para toda }t.
\end{equation}
\end{Sup}

es un proceso de Markov, m\'as a\'un, es un Proceso Fuerte de
Markov, es decir, la Propiedad Fuerte de Markov\footnote{Revisar
p\'agina 362, y 364 de Davis \cite{Davis}.} se cumple para
cualquier tiempo de paro.
%_________________________________________________________________________
%\renewcommand{\refname}{PROCESOS ESTOC\'ASTICOS}
%\renewcommand{\appendixname}{PROCESOS ESTOC\'ASTICOS}
%\renewcommand{\appendixtocname}{PROCESOS ESTOC\'ASTICOS}
%\renewcommand{\appendixpagename}{PROCESOS ESTOC\'ASTICOS}
%\appendix
%\clearpage % o \cleardoublepage
%\addappheadtotoc
%\appendixpage
%_________________________________________________________________________
\subsection{Teor\'ia General de Procesos Estoc\'asticos}
%_________________________________________________________________________
En esta secci\'on se har\'an las siguientes consideraciones: $E$
es un espacio m\'etrico separable y la m\'etrica $d$ es compatible
con la topolog\'ia.

\begin{Def}
Una medida finita, $\lambda$ en la $\sigma$-\'algebra de Borel de
un espacio metrizable $E$ se dice cerrada si
\begin{equation}\label{Eq.A2.3}
\lambda\left(E\right)=sup\left\{\lambda\left(K\right):K\textrm{ es
compacto en }E\right\}.
\end{equation}
\end{Def}

\begin{Def}
$E$ es un espacio de Rad\'on si cada medida finita en
$\left(E,\mathcal{B}\left(E\right)\right)$ es regular interior o cerrada,
{\em tight}.
\end{Def}


El siguiente teorema nos permite tener una mejor caracterizaci\'on de los espacios de Rad\'on:
\begin{Teo}\label{Tma.A2.2}
Sea $E$ espacio separable metrizable. Entonces $E$ es de Rad\'on
si y s\'olo s\'i cada medida finita en
$\left(E,\mathcal{B}\left(E\right)\right)$ es cerrada.
\end{Teo}

%_________________________________________________________________________________________
\subsection{Propiedades de Markov}
%_________________________________________________________________________________________

Sea $E$ espacio de estados, tal que $E$ es un espacio de Rad\'on, $\mathcal{B}\left(E\right)$ $\sigma$-\'algebra de Borel en $E$, que se denotar\'a por $\mathcal{E}$.

Sea $\left(X,\mathcal{G},\prob\right)$ espacio de probabilidad,
$I\subset\rea$ conjunto de índices. Sea $\mathcal{F}_{\leq t}$ la
$\sigma$-\'algebra natural definida como
$\sigma\left\{f\left(X_{r}\right):r\in I, r\leq
t,f\in\mathcal{E}\right\}$. Se considerar\'a una
$\sigma$-\'algebra m\'as general\footnote{qu\'e se quiere decir
con el t\'ermino: m\'as general?}, $ \left(\mathcal{G}_{t}\right)$
tal que $\left(X_{t}\right)$ sea $\mathcal{E}$-adaptado.

\begin{Def}
Una familia $\left(P_{s,t}\right)$ de kernels de Markov en $\left(E,\mathcal{E}\right)$ indexada por pares $s,t\in I$, con $s\leq t$ es una funci\'on de transici\'on en $\ER$, si  para todo $r\leq s< t$ en $I$ y todo $x\in E$, $B\in\mathcal{E}$
\begin{equation}\label{Eq.Kernels}
P_{r,t}\left(x,B\right)=\int_{E}P_{r,s}\left(x,dy\right)P_{s,t}\left(y,B\right)\footnote{Ecuaci\'on de Chapman-Kolmogorov}.
\end{equation}
\end{Def}

Se dice que la funci\'on de transici\'on $\KM$ en $\ER$ es la funci\'on de transici\'on para un proceso $\PE$  con valores en $E$ y que satisface la propiedad de Markov\footnote{\begin{equation}\label{Eq.1.4.S}
\prob\left\{H|\mathcal{G}_{t}\right\}=\prob\left\{H|X_{t}\right\}\textrm{ }H\in p\mathcal{F}_{\geq t}.
\end{equation}} (\ref{Eq.1.4.S}) relativa a $\left(\mathcal{G}_{t}\right)$ si

\begin{equation}\label{Eq.1.6.S}
\prob\left\{f\left(X_{t}\right)|\mathcal{G}_{s}\right\}=P_{s,t}f\left(X_{t}\right)\textrm{ }s\leq t\in I,\textrm{ }f\in b\mathcal{E}.
\end{equation}

\begin{Def}
Una familia $\left(P_{t}\right)_{t\geq0}$ de kernels de Markov en $\ER$ es llamada {\em Semigrupo de Transici\'on de Markov} o {\em Semigrupo de Transici\'on} si
\[P_{t+s}f\left(x\right)=P_{t}\left(P_{s}f\right)\left(x\right),\textrm{ }t,s\geq0,\textrm{ }x\in E\textrm{ }f\in b\mathcal{E}\footnote{Definir los t\'ermino $b\mathcal{E}$ y $p\mathcal{E}$}.\]
\end{Def}
\begin{Note}
Si la funci\'on de transici\'on $\KM$ es llamada homog\'enea si $P_{s,t}=P_{t-s}$.
\end{Note}

Un proceso de Markov que satisface la ecuaci\'on (\ref{Eq.1.6.S}) con funci\'on de transici\'on homog\'enea $\left(P_{t}\right)$ tiene la propiedad caracter\'istica
\begin{equation}\label{Eq.1.8.S}
\prob\left\{f\left(X_{t+s}\right)|\mathcal{G}_{t}\right\}=P_{s}f\left(X_{t}\right)\textrm{ }t,s\geq0,\textrm{ }f\in b\mathcal{E}.
\end{equation}
La ecuaci\'on anterior es la {\em Propiedad Simple de Markov} de $X$ relativa a $\left(P_{t}\right)$.

En este sentido el proceso $\PE$ cumple con la propiedad de Markov (\ref{Eq.1.8.S}) relativa a $\left(\Omega,\mathcal{G},\mathcal{G}_{t},\prob\right)$ con semigrupo de transici\'on $\left(P_{t}\right)$.
%_________________________________________________________________________________________
\subsection{Primer Condici\'on de Regularidad}
%_________________________________________________________________________________________
%\newcommand{\EM}{\left(\Omega,\mathcal{G},\prob\right)}
%\newcommand{\E4}{\left(\Omega,\mathcal{G},\mathcal{G}_{t},\prob\right)}
\begin{Def}
Un proceso estoc\'astico $\PE$ definido en
$\left(\Omega,\mathcal{G},\prob\right)$ con valores en el espacio
topol\'ogico $E$ es continuo por la derecha si cada trayectoria
muestral $t\rightarrow X_{t}\left(w\right)$ es un mapeo continuo
por la derecha de $I$ en $E$.
\end{Def}

\begin{Def}[HD1]\label{Eq.2.1.S}
Un semigrupo de Markov $\left(P_{t}\right)$ en un espacio de
Rad\'on $E$ se dice que satisface la condici\'on {\em HD1} si,
dada una medida de probabilidad $\mu$ en $E$, existe una
$\sigma$-\'algebra $\mathcal{E^{*}}$ con
$\mathcal{E}\subset\mathcal{E}^{*}$ y
$P_{t}\left(b\mathcal{E}^{*}\right)\subset b\mathcal{E}^{*}$, y un
$\mathcal{E}^{*}$-proceso $E$-valuado continuo por la derecha
$\PE$ en alg\'un espacio de probabilidad filtrado
$\left(\Omega,\mathcal{G},\mathcal{G}_{t},\prob\right)$ tal que
$X=\left(\Omega,\mathcal{G},\mathcal{G}_{t},\prob\right)$ es de
Markov (Homog\'eneo) con semigrupo de transici\'on $(P_{t})$ y
distribuci\'on inicial $\mu$.
\end{Def}

Consid\'erese la colecci\'on de variables aleatorias $X_{t}$
definidas en alg\'un espacio de probabilidad, y una colecci\'on de
medidas $\mathbf{P}^{x}$ tales que
$\mathbf{P}^{x}\left\{X_{0}=x\right\}$, y bajo cualquier
$\mathbf{P}^{x}$, $X_{t}$ es de Markov con semigrupo
$\left(P_{t}\right)$. $\mathbf{P}^{x}$ puede considerarse como la
distribuci\'on condicional de $\mathbf{P}$ dado $X_{0}=x$.

\begin{Def}\label{Def.2.2.S}
Sea $E$ espacio de Rad\'on, $\SG$ semigrupo de Markov en $\ER$. La colecci\'on $\mathbf{X}=\left(\Omega,\mathcal{G},\mathcal{G}_{t},X_{t},\theta_{t},\CM\right)$ es un proceso $\mathcal{E}$-Markov continuo por la derecha simple, con espacio de estados $E$ y semigrupo de transici\'on $\SG$ en caso de que $\mathbf{X}$ satisfaga las siguientes condiciones:
\begin{itemize}
\item[i)] $\left(\Omega,\mathcal{G},\mathcal{G}_{t}\right)$ es un espacio de medida filtrado, y $X_{t}$ es un proceso $E$-valuado continuo por la derecha $\mathcal{E}^{*}$-adaptado a $\left(\mathcal{G}_{t}\right)$;

\item[ii)] $\left(\theta_{t}\right)_{t\geq0}$ es una colecci\'on de operadores {\em shift} para $X$, es decir, mapea $\Omega$ en s\'i mismo satisfaciendo para $t,s\geq0$,

\begin{equation}\label{Eq.Shift}
\theta_{t}\circ\theta_{s}=\theta_{t+s}\textrm{ y }X_{t}\circ\theta_{t}=X_{t+s};
\end{equation}

\item[iii)] Para cualquier $x\in E$,$\CM\left\{X_{0}=x\right\}=1$, y el proceso $\PE$ tiene la propiedad de Markov (\ref{Eq.1.8.S}) con semigrupo de transici\'on $\SG$ relativo a $\left(\Omega,\mathcal{G},\mathcal{G}_{t},\CM\right)$.
\end{itemize}
\end{Def}

\begin{Def}[HD2]\label{Eq.2.2.S}
Para cualquier $\alpha>0$ y cualquier $f\in S^{\alpha}$, el proceso $t\rightarrow f\left(X_{t}\right)$ es continuo por la derecha casi seguramente.
\end{Def}

\begin{Def}\label{Def.PD}
Un sistema $\mathbf{X}=\left(\Omega,\mathcal{G},\mathcal{G}_{t},X_{t},\theta_{t},\CM\right)$ es un proceso derecho en el espacio de Rad\'on $E$ con semigrupo de transici\'on $\SG$ provisto de:
\begin{itemize}
\item[i)] $\mathbf{X}$ es una realizaci\'on  continua por la derecha, \ref{Def.2.2.S}, de $\SG$.

\item[ii)] $\mathbf{X}$ satisface la condicion HD2, \ref{Eq.2.2.S}, relativa a $\mathcal{G}_{t}$.

\item[iii)] $\mathcal{G}_{t}$ es aumentado y continuo por la derecha.
\end{itemize}
\end{Def}


%_________________________________________________________________________
%\renewcommand{\refname}{MODELO DE FLUJO}
%\renewcommand{\appendixname}{MODELO DE FLUJO}
%\renewcommand{\appendixtocname}{MODELO DE FLUJO}
%\renewcommand{\appendixpagename}{MODELO DE FLUJO}
%\appendix
%\clearpage % o \cleardoublepage
%\addappheadtotoc
%\appendixpage

\subsection{Construcci\'on del Modelo de Flujo}


\begin{Lema}[Lema 4.2, Dai\cite{Dai}]\label{Lema4.2}
Sea $\left\{x_{n}\right\}\subset \mathbf{X}$ con
$|x_{n}|\rightarrow\infty$, conforme $n\rightarrow\infty$. Suponga
que
\[lim_{n\rightarrow\infty}\frac{1}{|x_{n}|}U\left(0\right)=\overline{U}\]
y
\[lim_{n\rightarrow\infty}\frac{1}{|x_{n}|}V\left(0\right)=\overline{V}.\]

Entonces, conforme $n\rightarrow\infty$, casi seguramente

\begin{equation}\label{E1.4.2}
\frac{1}{|x_{n}|}\Phi^{k}\left(\left[|x_{n}|t\right]\right)\rightarrow
P_{k}^{'}t\textrm{, u.o.c.,}
\end{equation}

\begin{equation}\label{E1.4.3}
\frac{1}{|x_{n}|}E^{x_{n}}_{k}\left(|x_{n}|t\right)\rightarrow
\alpha_{k}\left(t-\overline{U}_{k}\right)^{+}\textrm{, u.o.c.,}
\end{equation}

\begin{equation}\label{E1.4.4}
\frac{1}{|x_{n}|}S^{x_{n}}_{k}\left(|x_{n}|t\right)\rightarrow
\mu_{k}\left(t-\overline{V}_{k}\right)^{+}\textrm{, u.o.c.,}
\end{equation}

donde $\left[t\right]$ es la parte entera de $t$ y
$\mu_{k}=1/m_{k}=1/\esp\left[\eta_{k}\left(1\right)\right]$.
\end{Lema}

\begin{Lema}[Lema 4.3, Dai\cite{Dai}]\label{Lema.4.3}
Sea $\left\{x_{n}\right\}\subset \mathbf{X}$ con
$|x_{n}|\rightarrow\infty$, conforme $n\rightarrow\infty$. Suponga
que
\[lim_{n\rightarrow\infty}\frac{1}{|x_{n}|}U_{k}\left(0\right)=\overline{U}_{k}\]
y
\[lim_{n\rightarrow\infty}\frac{1}{|x_{n}|}V_{k}\left(0\right)=\overline{V}_{k}.\]
\begin{itemize}
\item[a)] Conforme $n\rightarrow\infty$ casi seguramente,
\[lim_{n\rightarrow\infty}\frac{1}{|x_{n}|}U^{x_{n}}_{k}\left(|x_{n}|t\right)=\left(\overline{U}_{k}-t\right)^{+}\textrm{, u.o.c.}\]
y
\[lim_{n\rightarrow\infty}\frac{1}{|x_{n}|}V^{x_{n}}_{k}\left(|x_{n}|t\right)=\left(\overline{V}_{k}-t\right)^{+}.\]

\item[b)] Para cada $t\geq0$ fijo,
\[\left\{\frac{1}{|x_{n}|}U^{x_{n}}_{k}\left(|x_{n}|t\right),|x_{n}|\geq1\right\}\]
y
\[\left\{\frac{1}{|x_{n}|}V^{x_{n}}_{k}\left(|x_{n}|t\right),|x_{n}|\geq1\right\}\]
\end{itemize}
son uniformemente convergentes.
\end{Lema}

Sea $S_{l}^{x}\left(t\right)$ el n\'umero total de servicios
completados de la clase $l$, si la clase $l$ est\'a dando $t$
unidades de tiempo de servicio. Sea $T_{l}^{x}\left(x\right)$ el
monto acumulado del tiempo de servicio que el servidor
$s\left(l\right)$ gasta en los usuarios de la clase $l$ al tiempo
$t$. Entonces $S_{l}^{x}\left(T_{l}^{x}\left(t\right)\right)$ es
el n\'umero total de servicios completados para la clase $l$ al
tiempo $t$. Una fracci\'on de estos usuarios,
$\Phi_{k}^{x}\left(S_{l}^{x}\left(T_{l}^{x}\left(t\right)\right)\right)$,
se convierte en usuarios de la clase $k$.\\

Entonces, dado lo anterior, se tiene la siguiente representaci\'on
para el proceso de la longitud de la cola:\\

\begin{equation}
Q_{k}^{x}\left(t\right)=Q_{k}^{x}\left(0\right)+E_{k}^{x}\left(t\right)+\sum_{l=1}^{K}\Phi_{k}^{l}\left(S_{l}^{x}\left(T_{l}^{x}\left(t\right)\right)\right)-S_{k}^{x}\left(T_{k}^{x}\left(t\right)\right)
\end{equation}
para $k=1,\ldots,K$. Para $i=1,\ldots,d$, sea
\[I_{i}^{x}\left(t\right)=t-\sum_{j\in C_{i}}T_{k}^{x}\left(t\right).\]

Entonces $I_{i}^{x}\left(t\right)$ es el monto acumulado del
tiempo que el servidor $i$ ha estado desocupado al tiempo $t$. Se
est\'a asumiendo que las disciplinas satisfacen la ley de
conservaci\'on del trabajo, es decir, el servidor $i$ est\'a en
pausa solamente cuando no hay usuarios en la estaci\'on $i$.
Entonces, se tiene que

\begin{equation}
\int_{0}^{\infty}\left(\sum_{k\in
C_{i}}Q_{k}^{x}\left(t\right)\right)dI_{i}^{x}\left(t\right)=0,
\end{equation}
para $i=1,\ldots,d$.\\

Hacer
\[T^{x}\left(t\right)=\left(T_{1}^{x}\left(t\right),\ldots,T_{K}^{x}\left(t\right)\right)^{'},\]
\[I^{x}\left(t\right)=\left(I_{1}^{x}\left(t\right),\ldots,I_{K}^{x}\left(t\right)\right)^{'}\]
y
\[S^{x}\left(T^{x}\left(t\right)\right)=\left(S_{1}^{x}\left(T_{1}^{x}\left(t\right)\right),\ldots,S_{K}^{x}\left(T_{K}^{x}\left(t\right)\right)\right)^{'}.\]

Para una disciplina que cumple con la ley de conservaci\'on del
trabajo, en forma vectorial, se tiene el siguiente conjunto de
ecuaciones

\begin{equation}\label{Eq.MF.1.3}
Q^{x}\left(t\right)=Q^{x}\left(0\right)+E^{x}\left(t\right)+\sum_{l=1}^{K}\Phi^{l}\left(S_{l}^{x}\left(T_{l}^{x}\left(t\right)\right)\right)-S^{x}\left(T^{x}\left(t\right)\right),\\
\end{equation}

\begin{equation}\label{Eq.MF.2.3}
Q^{x}\left(t\right)\geq0,\\
\end{equation}

\begin{equation}\label{Eq.MF.3.3}
T^{x}\left(0\right)=0,\textrm{ y }\overline{T}^{x}\left(t\right)\textrm{ es no decreciente},\\
\end{equation}

\begin{equation}\label{Eq.MF.4.3}
I^{x}\left(t\right)=et-CT^{x}\left(t\right)\textrm{ es no
decreciente}\\
\end{equation}

\begin{equation}\label{Eq.MF.5.3}
\int_{0}^{\infty}\left(CQ^{x}\left(t\right)\right)dI_{i}^{x}\left(t\right)=0,\\
\end{equation}

\begin{equation}\label{Eq.MF.6.3}
\textrm{Condiciones adicionales en
}\left(\overline{Q}^{x}\left(\cdot\right),\overline{T}^{x}\left(\cdot\right)\right)\textrm{
espec\'ificas de la disciplina de la cola,}
\end{equation}

donde $e$ es un vector de unos de dimensi\'on $d$, $C$ es la
matriz definida por
\[C_{ik}=\left\{\begin{array}{cc}
1,& S\left(k\right)=i,\\
0,& \textrm{ en otro caso}.\\
\end{array}\right.
\]
Es necesario enunciar el siguiente Teorema que se utilizar\'a para
el Teorema \ref{Tma.4.2.Dai}:
\begin{Teo}[Teorema 4.1, Dai \cite{Dai}]
Considere una disciplina que cumpla la ley de conservaci\'on del
trabajo, para casi todas las trayectorias muestrales $\omega$ y
cualquier sucesi\'on de estados iniciales
$\left\{x_{n}\right\}\subset \mathbf{X}$, con
$|x_{n}|\rightarrow\infty$, existe una subsucesi\'on
$\left\{x_{n_{j}}\right\}$ con $|x_{n_{j}}|\rightarrow\infty$ tal
que
\begin{equation}\label{Eq.4.15}
\frac{1}{|x_{n_{j}}|}\left(Q^{x_{n_{j}}}\left(0\right),U^{x_{n_{j}}}\left(0\right),V^{x_{n_{j}}}\left(0\right)\right)\rightarrow\left(\overline{Q}\left(0\right),\overline{U},\overline{V}\right),
\end{equation}

\begin{equation}\label{Eq.4.16}
\frac{1}{|x_{n_{j}}|}\left(Q^{x_{n_{j}}}\left(|x_{n_{j}}|t\right),T^{x_{n_{j}}}\left(|x_{n_{j}}|t\right)\right)\rightarrow\left(\overline{Q}\left(t\right),\overline{T}\left(t\right)\right)\textrm{
u.o.c.}
\end{equation}

Adem\'as,
$\left(\overline{Q}\left(t\right),\overline{T}\left(t\right)\right)$
satisface las siguientes ecuaciones:
\begin{equation}\label{Eq.MF.1.3a}
\overline{Q}\left(t\right)=Q\left(0\right)+\left(\alpha
t-\overline{U}\right)^{+}-\left(I-P\right)^{'}M^{-1}\left(\overline{T}\left(t\right)-\overline{V}\right)^{+},
\end{equation}

\begin{equation}\label{Eq.MF.2.3a}
\overline{Q}\left(t\right)\geq0,\\
\end{equation}

\begin{equation}\label{Eq.MF.3.3a}
\overline{T}\left(t\right)\textrm{ es no decreciente y comienza en cero},\\
\end{equation}

\begin{equation}\label{Eq.MF.4.3a}
\overline{I}\left(t\right)=et-C\overline{T}\left(t\right)\textrm{
es no decreciente,}\\
\end{equation}

\begin{equation}\label{Eq.MF.5.3a}
\int_{0}^{\infty}\left(C\overline{Q}\left(t\right)\right)d\overline{I}\left(t\right)=0,\\
\end{equation}

\begin{equation}\label{Eq.MF.6.3a}
\textrm{Condiciones adicionales en
}\left(\overline{Q}\left(\cdot\right),\overline{T}\left(\cdot\right)\right)\textrm{
especficas de la disciplina de la cola,}
\end{equation}
\end{Teo}


Propiedades importantes para el modelo de flujo retrasado:

\begin{Prop}
 Sea $\left(\overline{Q},\overline{T},\overline{T}^{0}\right)$ un flujo l\'imite de \ref{Eq.4.4} y suponga que cuando $x\rightarrow\infty$ a lo largo de
una subsucesi\'on
\[\left(\frac{1}{|x|}Q_{k}^{x}\left(0\right),\frac{1}{|x|}A_{k}^{x}\left(0\right),\frac{1}{|x|}B_{k}^{x}\left(0\right),\frac{1}{|x|}B_{k}^{x,0}\left(0\right)\right)\rightarrow\left(\overline{Q}_{k}\left(0\right),0,0,0\right)\]
para $k=1,\ldots,K$. EL flujo l\'imite tiene las siguientes
propiedades, donde las propiedades de la derivada se cumplen donde
la derivada exista:
\begin{itemize}
 \item[i)] Los vectores de tiempo ocupado $\overline{T}\left(t\right)$ y $\overline{T}^{0}\left(t\right)$ son crecientes y continuas con
$\overline{T}\left(0\right)=\overline{T}^{0}\left(0\right)=0$.
\item[ii)] Para todo $t\geq0$
\[\sum_{k=1}^{K}\left[\overline{T}_{k}\left(t\right)+\overline{T}_{k}^{0}\left(t\right)\right]=t\]
\item[iii)] Para todo $1\leq k\leq K$
\[\overline{Q}_{k}\left(t\right)=\overline{Q}_{k}\left(0\right)+\alpha_{k}t-\mu_{k}\overline{T}_{k}\left(t\right)\]
\item[iv)]  Para todo $1\leq k\leq K$
\[\dot{{\overline{T}}}_{k}\left(t\right)=\beta_{k}\] para $\overline{Q}_{k}\left(t\right)=0$.
\item[v)] Para todo $k,j$
\[\mu_{k}^{0}\overline{T}_{k}^{0}\left(t\right)=\mu_{j}^{0}\overline{T}_{j}^{0}\left(t\right)\]
\item[vi)]  Para todo $1\leq k\leq K$
\[\mu_{k}\dot{{\overline{T}}}_{k}\left(t\right)=l_{k}\mu_{k}^{0}\dot{{\overline{T}}}_{k}^{0}\left(t\right)\] para $\overline{Q}_{k}\left(t\right)>0$.
\end{itemize}
\end{Prop}

\begin{Teo}[Teorema 5.1: Ley Fuerte para Procesos de Conteo
\cite{Gut}]\label{Tma.5.1.Gut} Sea
$0<\mu<\esp\left(X_{1}\right]\leq\infty$. entonces

\begin{itemize}
\item[a)] $\frac{N\left(t\right)}{t}\rightarrow\frac{1}{\mu}$
a.s., cuando $t\rightarrow\infty$.


\item[b)]$\esp\left[\frac{N\left(t\right)}{t}\right]^{r}\rightarrow\frac{1}{\mu^{r}}$,
cuando $t\rightarrow\infty$ para todo $r>0$..
\end{itemize}
\end{Teo}


\begin{Prop}[Proposici\'on 5.3 \cite{DaiSean}]
Sea $X$ proceso de estados para la red de colas, y suponga que se
cumplen los supuestos (A1) y (A2), entonces para alguna constante
positiva $C_{p+1}<\infty$, $\delta>0$ y un conjunto compacto
$C\subset X$.

\begin{equation}\label{Eq.5.4}
\esp_{x}\left[\int_{0}^{\tau_{C}\left(\delta\right)}\left(1+|X\left(t\right)|^{p}\right)dt\right]\leq
C_{p+1}\left(1+|x|^{p+1}\right)
\end{equation}
\end{Prop}

\begin{Prop}[Proposici\'on 5.4 \cite{DaiSean}]
Sea $X$ un proceso de Markov Borel Derecho en $X$, sea
$f:X\leftarrow\rea_{+}$ y defina para alguna $\delta>0$, y un
conjunto cerrado $C\subset X$
\[V\left(x\right):=\esp_{x}\left[\int_{0}^{\tau_{C}\left(\delta\right)}f\left(X\left(t\right)\right)dt\right]\]
para $x\in X$. Si $V$ es finito en todas partes y uniformemente
acotada en $C$, entonces existe $k<\infty$ tal que
\begin{equation}\label{Eq.5.11}
\frac{1}{t}\esp_{x}\left[V\left(x\right)\right]+\frac{1}{t}\int_{0}^{t}\esp_{x}\left[f\left(X\left(s\right)\right)ds\right]\leq\frac{1}{t}V\left(x\right)+k,
\end{equation}
para $x\in X$ y $t>0$.
\end{Prop}


%_________________________________________________________________________
%\renewcommand{\refname}{Ap\'endice D}
%\renewcommand{\appendixname}{ESTABILIDAD}
%\renewcommand{\appendixtocname}{ESTABILIDAD}
%\renewcommand{\appendixpagename}{ESTABILIDAD}
%\appendix
%\clearpage % o \cleardoublepage
%\addappheadtotoc
%\appendixpage

\subsection{Estabilidad}

\begin{Def}[Definici\'on 3.2, Dai y Meyn \cite{DaiSean}]
El modelo de flujo retrasado de una disciplina de servicio en una
red con retraso
$\left(\overline{A}\left(0\right),\overline{B}\left(0\right)\right)\in\rea_{+}^{K+|A|}$
se define como el conjunto de ecuaciones dadas en
\ref{Eq.3.8}-\ref{Eq.3.13}, junto con la condici\'on:
\begin{equation}\label{CondAd.FluidModel}
\overline{Q}\left(t\right)=\overline{Q}\left(0\right)+\left(\alpha
t-\overline{A}\left(0\right)\right)^{+}-\left(I-P^{'}\right)M\left(\overline{T}\left(t\right)-\overline{B}\left(0\right)\right)^{+}
\end{equation}
\end{Def}

entonces si el modelo de flujo retrasado tambi\'en es estable:


\begin{Def}[Definici\'on 3.1, Dai y Meyn \cite{DaiSean}]
Un flujo l\'imite (retrasado) para una red bajo una disciplina de
servicio espec\'ifica se define como cualquier soluci\'on
 $\left(\overline{Q}\left(\cdot\right),\overline{T}\left(\cdot\right)\right)$ de las siguientes ecuaciones, donde
$\overline{Q}\left(t\right)=\left(\overline{Q}_{1}\left(t\right),\ldots,\overline{Q}_{K}\left(t\right)\right)^{'}$
y
$\overline{T}\left(t\right)=\left(\overline{T}_{1}\left(t\right),\ldots,\overline{T}_{K}\left(t\right)\right)^{'}$
\begin{equation}\label{Eq.3.8}
\overline{Q}_{k}\left(t\right)=\overline{Q}_{k}\left(0\right)+\alpha_{k}t-\mu_{k}\overline{T}_{k}\left(t\right)+\sum_{l=1}^{k}P_{lk}\mu_{l}\overline{T}_{l}\left(t\right)\\
\end{equation}
\begin{equation}\label{Eq.3.9}
\overline{Q}_{k}\left(t\right)\geq0\textrm{ para }k=1,2,\ldots,K,\\
\end{equation}
\begin{equation}\label{Eq.3.10}
\overline{T}_{k}\left(0\right)=0,\textrm{ y }\overline{T}_{k}\left(\cdot\right)\textrm{ es no decreciente},\\
\end{equation}
\begin{equation}\label{Eq.3.11}
\overline{I}_{i}\left(t\right)=t-\sum_{k\in C_{i}}\overline{T}_{k}\left(t\right)\textrm{ es no decreciente}\\
\end{equation}
\begin{equation}\label{Eq.3.12}
\overline{I}_{i}\left(\cdot\right)\textrm{ se incrementa al tiempo }t\textrm{ cuando }\sum_{k\in C_{i}}Q_{k}^{x}\left(t\right)dI_{i}^{x}\left(t\right)=0\\
\end{equation}
\begin{equation}\label{Eq.3.13}
\textrm{condiciones adicionales sobre
}\left(Q^{x}\left(\cdot\right),T^{x}\left(\cdot\right)\right)\textrm{
referentes a la disciplina de servicio}
\end{equation}
\end{Def}

\begin{Lema}[Lema 3.1 \cite{Chen}]\label{Lema3.1}
Si el modelo de flujo es estable, definido por las ecuaciones
(3.8)-(3.13), entonces el modelo de flujo retrasado tambin es
estable.
\end{Lema}

\begin{Teo}[Teorema 5.1 \cite{Chen}]\label{Tma.5.1.Chen}
La red de colas es estable si existe una constante $t_{0}$ que
depende de $\left(\alpha,\mu,T,U\right)$ y $V$ que satisfagan las
ecuaciones (5.1)-(5.5), $Z\left(t\right)=0$, para toda $t\geq
t_{0}$.
\end{Teo}

\begin{Prop}[Proposici\'on 5.1, Dai y Meyn \cite{DaiSean}]\label{Prop.5.1.DaiSean}
Suponga que los supuestos A1) y A2) son ciertos y que el modelo de flujo es estable. Entonces existe $t_{0}>0$ tal que
\begin{equation}
lim_{|x|\rightarrow\infty}\frac{1}{|x|^{p+1}}\esp_{x}\left[|X\left(t_{0}|x|\right)|^{p+1}\right]=0
\end{equation}
\end{Prop}

\begin{Lemma}[Lema 5.2, Dai y Meyn \cite{DaiSean}]\label{Lema.5.2.DaiSean}
 Sea $\left\{\zeta\left(k\right):k\in \mathbb{z}\right\}$ una sucesi\'on independiente e id\'enticamente distribuida que toma valores en $\left(0,\infty\right)$,
y sea
$E\left(t\right)=max\left(n\geq1:\zeta\left(1\right)+\cdots+\zeta\left(n-1\right)\leq
t\right)$. Si $\esp\left[\zeta\left(1\right)\right]<\infty$,
entonces para cualquier entero $r\geq1$
\begin{equation}
 lim_{t\rightarrow\infty}\esp\left[\left(\frac{E\left(t\right)}{t}\right)^{r}\right]=\left(\frac{1}{\esp\left[\zeta_{1}\right]}\right)^{r}.
\end{equation}
Luego, bajo estas condiciones:
\begin{itemize}
 \item[a)] para cualquier $\delta>0$, $\sup_{t\geq\delta}\esp\left[\left(\frac{E\left(t\right)}{t}\right)^{r}\right]<\infty$
\item[b)] las variables aleatorias
$\left\{\left(\frac{E\left(t\right)}{t}\right)^{r}:t\geq1\right\}$
son uniformemente integrables.
\end{itemize}
\end{Lemma}

\begin{Teo}[Teorema 5.5, Dai y Meyn \cite{DaiSean}]\label{Tma.5.5.DaiSean}
Suponga que los supuestos A1) y A2) se cumplen y que el modelo de
flujo es estable. Entonces existe una constante $\kappa_{p}$ tal
que
\begin{equation}
\frac{1}{t}\int_{0}^{t}\esp_{x}\left[|Q\left(s\right)|^{p}\right]ds\leq\kappa_{p}\left\{\frac{1}{t}|x|^{p+1}+1\right\}
\end{equation}
para $t>0$ y $x\in X$. En particular, para cada condici\'on
inicial
\begin{eqnarray*}
\limsup_{t\rightarrow\infty}\frac{1}{t}\int_{0}^{t}\esp_{x}\left[|Q\left(s\right)|^{p}\right]ds\leq\kappa_{p}.
\end{eqnarray*}
\end{Teo}

\begin{Teo}[Teorema 6.2, Dai y Meyn \cite{DaiSean}]\label{Tma.6.2.DaiSean}
Suponga que se cumplen los supuestos A1), A2) y A3) y que el
modelo de flujo es estable. Entonces se tiene que
\begin{equation}
\left\|P^{t}\left(x,\cdot\right)-\pi\left(\cdot\right)\right\|_{f_{p}}\textrm{,
}t\rightarrow\infty,x\in X.
\end{equation}
En particular para cada condici\'on inicial
\begin{eqnarray*}
\lim_{t\rightarrow\infty}\esp_{x}\left[|Q\left(t\right)|^{p}\right]=\esp_{\pi}\left[|Q\left(0\right)|^{p}\right]\leq\kappa_{r}
\end{eqnarray*}
\end{Teo}
\begin{Teo}[Teorema 6.3, Dai y Meyn \cite{DaiSean}]\label{Tma.6.3.DaiSean}
Suponga que se cumplen los supuestos A1), A2) y A3) y que el
modelo de flujo es estable. Entonces con
$f\left(x\right)=f_{1}\left(x\right)$ se tiene
\begin{equation}
\lim_{t\rightarrow\infty}t^{p-1}\left\|P^{t}\left(x,\cdot\right)-\pi\left(\cdot\right)\right\|_{f}=0.
\end{equation}
En particular para cada condici\'on inicial
\begin{eqnarray*}
\lim_{t\rightarrow\infty}t^{p-1}|\esp_{x}\left[Q\left(t\right)\right]-\esp_{\pi}\left[Q\left(0\right)\right]|=0.
\end{eqnarray*}
\end{Teo}

\begin{Teo}[Teorema 6.4, Dai y Meyn \cite{DaiSean}]\label{Tma.6.4.DaiSean}
Suponga que se cumplen los supuestos A1), A2) y A3) y que el
modelo de flujo es estable. Sea $\nu$ cualquier distribuci\'on de
probabilidad en $\left(X,\mathcal{B}_{X}\right)$, y $\pi$ la
distribuci\'on estacionaria de $X$.
\begin{itemize}
\item[i)] Para cualquier $f:X\leftarrow\rea_{+}$
\begin{equation}
\lim_{t\rightarrow\infty}\frac{1}{t}\int_{o}^{t}f\left(X\left(s\right)\right)ds=\pi\left(f\right):=\int
f\left(x\right)\pi\left(dx\right)
\end{equation}
$\prob$-c.s.

\item[ii)] Para cualquier $f:X\leftarrow\rea_{+}$ con
$\pi\left(|f|\right)<\infty$, la ecuaci\'on anterior se cumple.
\end{itemize}
\end{Teo}

\begin{Teo}[Teorema 2.2, Down \cite{Down}]\label{Tma2.2.Down}
Suponga que el fluido modelo es inestable en el sentido de que
para alguna $\epsilon_{0},c_{0}\geq0$,
\begin{equation}\label{Eq.Inestability}
|Q\left(T\right)|\geq\epsilon_{0}T-c_{0}\textrm{,   }T\geq0,
\end{equation}
para cualquier condici\'on inicial $Q\left(0\right)$, con
$|Q\left(0\right)|=1$. Entonces para cualquier $0<q\leq1$, existe
$B<0$ tal que para cualquier $|x|\geq B$,
\begin{equation}
\prob_{x}\left\{\mathbb{X}\rightarrow\infty\right\}\geq q.
\end{equation}
\end{Teo}


\begin{Def}
Sea $X$ un conjunto y $\mathcal{F}$ una $\sigma$-\'algebra de
subconjuntos de $X$, la pareja $\left(X,\mathcal{F}\right)$ es
llamado espacio medible. Un subconjunto $A$ de $X$ es llamado
medible, o medible con respecto a $\mathcal{F}$, si
$A\in\mathcal{F}$.
\end{Def}

\begin{Def}
Sea $\left(X,\mathcal{F},\mu\right)$ espacio de medida. Se dice
que la medida $\mu$ es $\sigma$-finita si se puede escribir
$X=\bigcup_{n\geq1}X_{n}$ con $X_{n}\in\mathcal{F}$ y
$\mu\left(X_{n}\right)<\infty$.
\end{Def}

\begin{Def}\label{Cto.Borel}
Sea $X$ el conjunto de los \'umeros reales $\rea$. El \'algebra de
Borel es la $\sigma$-\'algebra $B$ generada por los intervalos
abiertos $\left(a,b\right)\in\rea$. Cualquier conjunto en $B$ es
llamado {\em Conjunto de Borel}.
\end{Def}

\begin{Def}\label{Funcion.Medible}
Una funci\'on $f:X\rightarrow\rea$, es medible si para cualquier
n\'umero real $\alpha$ el conjunto
\[\left\{x\in X:f\left(x\right)>\alpha\right\}\]
pertenece a $X$. Equivalentemente, se dice que $f$ es medible si
\[f^{-1}\left(\left(\alpha,\infty\right)\right)=\left\{x\in X:f\left(x\right)>\alpha\right\}\in\mathcal{F}.\]
\end{Def}


\begin{Def}\label{Def.Cilindros}
Sean $\left(\Omega_{i},\mathcal{F}_{i}\right)$, $i=1,2,\ldots,$
espacios medibles y $\Omega=\prod_{i=1}^{\infty}\Omega_{i}$ el
conjunto de todas las sucesiones
$\left(\omega_{1},\omega_{2},\ldots,\right)$ tales que
$\omega_{i}\in\Omega_{i}$, $i=1,2,\ldots,$. Si
$B^{n}\subset\prod_{i=1}^{\infty}\Omega_{i}$, definimos
$B_{n}=\left\{\omega\in\Omega:\left(\omega_{1},\omega_{2},\ldots,\omega_{n}\right)\in
B^{n}\right\}$. Al conjunto $B_{n}$ se le llama {\em cilindro} con
base $B^{n}$, el cilindro es llamado medible si
$B^{n}\in\prod_{i=1}^{\infty}\mathcal{F}_{i}$.
\end{Def}


\begin{Def}\label{Def.Proc.Adaptado}[TSP, Ash \cite{RBA}]
Sea $X\left(t\right),t\geq0$ proceso estoc\'astico, el proceso es
adaptado a la familia de $\sigma$-\'algebras $\mathcal{F}_{t}$,
para $t\geq0$, si para $s<t$ implica que
$\mathcal{F}_{s}\subset\mathcal{F}_{t}$, y $X\left(t\right)$ es
$\mathcal{F}_{t}$-medible para cada $t$. Si no se especifica
$\mathcal{F}_{t}$ entonces se toma $\mathcal{F}_{t}$ como
$\mathcal{F}\left(X\left(s\right),s\leq t\right)$, la m\'as
peque\~na $\sigma$-\'algebra de subconjuntos de $\Omega$ que hace
que cada $X\left(s\right)$, con $s\leq t$ sea Borel medible.
\end{Def}


\begin{Def}\label{Def.Tiempo.Paro}[TSP, Ash \cite{RBA}]
Sea $\left\{\mathcal{F}\left(t\right),t\geq0\right\}$ familia
creciente de sub $\sigma$-\'algebras. es decir,
$\mathcal{F}\left(s\right)\subset\mathcal{F}\left(t\right)$ para
$s\leq t$. Un tiempo de paro para $\mathcal{F}\left(t\right)$ es
una funci\'on $T:\Omega\rightarrow\left[0,\infty\right]$ tal que
$\left\{T\leq t\right\}\in\mathcal{F}\left(t\right)$ para cada
$t\geq0$. Un tiempo de paro para el proceso estoc\'astico
$X\left(t\right),t\geq0$ es un tiempo de paro para las
$\sigma$-\'algebras
$\mathcal{F}\left(t\right)=\mathcal{F}\left(X\left(s\right)\right)$.
\end{Def}

\begin{Def}
Sea $X\left(t\right),t\geq0$ proceso estoc\'astico, con
$\left(S,\chi\right)$ espacio de estados. Se dice que el proceso
es adaptado a $\left\{\mathcal{F}\left(t\right)\right\}$, es
decir, si para cualquier $s,t\in I$, $I$ conjunto de \'indices,
$s<t$, se tiene que
$\mathcal{F}\left(s\right)\subset\mathcal{F}\left(t\right)$ y
$X\left(t\right)$ es $\mathcal{F}\left(t\right)$-medible,
\end{Def}

\begin{Def}
Sea $X\left(t\right),t\geq0$ proceso estoc\'astico, se dice que es
un Proceso de Markov relativo a $\mathcal{F}\left(t\right)$ o que
$\left\{X\left(t\right),\mathcal{F}\left(t\right)\right\}$ es de
Markov si y s\'olo si para cualquier conjunto $B\in\chi$,  y
$s,t\in I$, $s<t$ se cumple que
\begin{equation}\label{Prop.Markov}
P\left\{X\left(t\right)\in
B|\mathcal{F}\left(s\right)\right\}=P\left\{X\left(t\right)\in
B|X\left(s\right)\right\}.
\end{equation}
\end{Def}
\begin{Note}
Si se dice que $\left\{X\left(t\right)\right\}$ es un Proceso de
Markov sin mencionar $\mathcal{F}\left(t\right)$, se asumir\'a que
\begin{eqnarray*}
\mathcal{F}\left(t\right)=\mathcal{F}_{0}\left(t\right)=\mathcal{F}\left(X\left(r\right),r\leq
t\right),
\end{eqnarray*}
entonces la ecuaci\'on (\ref{Prop.Markov}) se puede escribir como
\begin{equation}
P\left\{X\left(t\right)\in B|X\left(r\right),r\leq s\right\} =
P\left\{X\left(t\right)\in B|X\left(s\right)\right\}
\end{equation}
\end{Note}

\begin{Teo}
Sea $\left(X_{n},\mathcal{F}_{n},n=0,1,\ldots,\right\}$ Proceso de
Markov con espacio de estados $\left(S_{0},\chi_{0}\right)$
generado por una distribuici\'on inicial $P_{o}$ y probabilidad de
transici\'on $p_{mn}$, para $m,n=0,1,\ldots,$ $m<n$, que por
notaci\'on se escribir\'a como $p\left(m,n,x,B\right)\rightarrow
p_{mn}\left(x,B\right)$. Sea $S$ tiempo de paro relativo a la
$\sigma$-\'algebra $\mathcal{F}_{n}$. Sea $T$ funci\'on medible,
$T:\Omega\rightarrow\left\{0,1,\ldots,\right\}$. Sup\'ongase que
$T\geq S$, entonces $T$ es tiempo de paro. Si $B\in\chi_{0}$,
entonces
\begin{equation}\label{Prop.Fuerte.Markov}
P\left\{X\left(T\right)\in
B,T<\infty|\mathcal{F}\left(S\right)\right\} =
p\left(S,T,X\left(s\right),B\right)
\end{equation}
en $\left\{T<\infty\right\}$.
\end{Teo}


Sea $K$ conjunto numerable y sea $d:K\rightarrow\nat$ funci\'on.
Para $v\in K$, $M_{v}$ es un conjunto abierto de
$\rea^{d\left(v\right)}$. Entonces \[E=\cup_{v\in
K}M_{v}=\left\{\left(v,\zeta\right):v\in K,\zeta\in
M_{v}\right\}.\]

Sea $\mathcal{E}$ la clase de conjuntos medibles en $E$:
\[\mathcal{E}=\left\{\cup_{v\in K}A_{v}:A_{v}\in \mathcal{M}_{v}\right\}.\]

donde $\mathcal{M}$ son los conjuntos de Borel de $M_{v}$.
Entonces $\left(E,\mathcal{E}\right)$ es un espacio de Borel. El
estado del proceso se denotar\'a por
$\mathbf{x}_{t}=\left(v_{t},\zeta_{t}\right)$. La distribuci\'on
de $\left(\mathbf{x}_{t}\right)$ est\'a determinada por por los
siguientes objetos:

\begin{itemize}
\item[i)] Los campos vectoriales $\left(\mathcal{H}_{v},v\in
K\right)$. \item[ii)] Una funci\'on medible $\lambda:E\rightarrow
\rea_{+}$. \item[iii)] Una medida de transici\'on
$Q:\mathcal{E}\times\left(E\cup\Gamma^{*}\right)\rightarrow\left[0,1\right]$
donde
\begin{equation}
\Gamma^{*}=\cup_{v\in K}\partial^{*}M_{v}.
\end{equation}
y
\begin{equation}
\partial^{*}M_{v}=\left\{z\in\partial M_{v}:\mathbf{\mathbf{\phi}_{v}\left(t,\zeta\right)=\mathbf{z}}\textrm{ para alguna }\left(t,\zeta\right)\in\rea_{+}\times M_{v}\right\}.
\end{equation}
$\partial M_{v}$ denota  la frontera de $M_{v}$.
\end{itemize}

El campo vectorial $\left(\mathcal{H}_{v},v\in K\right)$ se supone
tal que para cada $\mathbf{z}\in M_{v}$ existe una \'unica curva
integral $\mathbf{\phi}_{v}\left(t,\zeta\right)$ que satisface la
ecuaci\'on

\begin{equation}
\frac{d}{dt}f\left(\zeta_{t}\right)=\mathcal{H}f\left(\zeta_{t}\right),
\end{equation}
con $\zeta_{0}=\mathbf{z}$, para cualquier funci\'on suave
$f:\rea^{d}\rightarrow\rea$ y $\mathcal{H}$ denota el operador
diferencial de primer orden, con $\mathcal{H}=\mathcal{H}_{v}$ y
$\zeta_{t}=\mathbf{\phi}\left(t,\mathbf{z}\right)$. Adem\'as se
supone que $\mathcal{H}_{v}$ es conservativo, es decir, las curvas
integrales est\'an definidas para todo $t>0$.

Para $\mathbf{x}=\left(v,\zeta\right)\in E$ se denota
\[t^{*}\mathbf{x}=inf\left\{t>0:\mathbf{\phi}_{v}\left(t,\zeta\right)\in\partial^{*}M_{v}\right\}\]

En lo que respecta a la funci\'on $\lambda$, se supondr\'a que
para cada $\left(v,\zeta\right)\in E$ existe un $\epsilon>0$ tal
que la funci\'on
$s\rightarrow\lambda\left(v,\phi_{v}\left(s,\zeta\right)\right)\in
E$ es integrable para $s\in\left[0,\epsilon\right)$. La medida de
transici\'on $Q\left(A;\mathbf{x}\right)$ es una funci\'on medible
de $\mathbf{x}$ para cada $A\in\mathcal{E}$, definida para
$\mathbf{x}\in E\cup\Gamma^{*}$ y es una medida de probabilidad en
$\left(E,\mathcal{E}\right)$ para cada $\mathbf{x}\in E$.

El movimiento del proceso $\left(\mathbf{x}_{t}\right)$ comenzando
en $\mathbf{x}=\left(n,\mathbf{z}\right)\in E$ se puede construir
de la siguiente manera, def\'inase la funci\'on $F$ por

\begin{equation}
F\left(t\right)=\left\{\begin{array}{ll}\\
exp\left(-\int_{0}^{t}\lambda\left(n,\phi_{n}\left(s,\mathbf{z}\right)\right)ds\right), & t<t^{*}\left(\mathbf{x}\right),\\
0, & t\geq t^{*}\left(\mathbf{x}\right)
\end{array}\right.
\end{equation}

Sea $T_{1}$ una variable aleatoria tal que
$\prob\left[T_{1}>t\right]=F\left(t\right)$, ahora sea la variable
aleatoria $\left(N,Z\right)$ con distribuici\'on
$Q\left(\cdot;\phi_{n}\left(T_{1},\mathbf{z}\right)\right)$. La
trayectoria de $\left(\mathbf{x}_{t}\right)$ para $t\leq T_{1}$
es\footnote{Revisar p\'agina 362, y 364 de Davis \cite{Davis}.}
\begin{eqnarray*}
\mathbf{x}_{t}=\left(v_{t},\zeta_{t}\right)=\left\{\begin{array}{ll}
\left(n,\phi_{n}\left(t,\mathbf{z}\right)\right), & t<T_{1},\\
\left(N,\mathbf{Z}\right), & t=t_{1}.
\end{array}\right.
\end{eqnarray*}

Comenzando en $\mathbf{x}_{T_{1}}$ se selecciona el siguiente
tiempo de intersalto $T_{2}-T_{1}$ lugar del post-salto
$\mathbf{x}_{T_{2}}$ de manera similar y as\'i sucesivamente. Este
procedimiento nos da una trayectoria determinista por partes
$\mathbf{x}_{t}$ con tiempos de salto $T_{1},T_{2},\ldots$. Bajo
las condiciones enunciadas para $\lambda,T_{1}>0$  y
$T_{1}-T_{2}>0$ para cada $i$, con probabilidad 1. Se supone que
se cumple la siquiente condici\'on.

\begin{Sup}[Supuesto 3.1, Davis \cite{Davis}]\label{Sup3.1.Davis}
Sea $N_{t}:=\sum_{t}\indora_{\left(t\geq t\right)}$ el n\'umero de
saltos en $\left[0,t\right]$. Entonces
\begin{equation}
\esp\left[N_{t}\right]<\infty\textrm{ para toda }t.
\end{equation}
\end{Sup}

es un proceso de Markov, m\'as a\'un, es un Proceso Fuerte de
Markov, es decir, la Propiedad Fuerte de Markov se cumple para
cualquier tiempo de paro.
%_________________________________________________________________________

En esta secci\'on se har\'an las siguientes consideraciones: $E$
es un espacio m\'etrico separable y la m\'etrica $d$ es compatible
con la topolog\'ia.


\begin{Def}
Un espacio topol\'ogico $E$ es llamado {\em Luisin} si es
homeomorfo a un subconjunto de Borel de un espacio m\'etrico
compacto.
\end{Def}

\begin{Def}
Un espacio topol\'ogico $E$ es llamado de {\em Rad\'on} si es
homeomorfo a un subconjunto universalmente medible de un espacio
m\'etrico compacto.
\end{Def}

Equivalentemente, la definici\'on de un espacio de Rad\'on puede
encontrarse en los siguientes t\'erminos:


\begin{Def}
$E$ es un espacio de Rad\'on si cada medida finita en
$\left(E,\mathcal{B}\left(E\right)\right)$ es regular interior o cerrada,
{\em tight}.
\end{Def}

\begin{Def}
Una medida finita, $\lambda$ en la $\sigma$-\'algebra de Borel de
un espacio metrizable $E$ se dice cerrada si
\begin{equation}\label{Eq.A2.3}
\lambda\left(E\right)=sup\left\{\lambda\left(K\right):K\textrm{ es
compacto en }E\right\}.
\end{equation}
\end{Def}

El siguiente teorema nos permite tener una mejor caracterizaci\'on de los espacios de Rad\'on:
\begin{Teo}\label{Tma.A2.2}
Sea $E$ espacio separable metrizable. Entonces $E$ es Radoniano si y s\'olo s\'i cada medida finita en $\left(E,\mathcal{B}\left(E\right)\right)$ es cerrada.
\end{Teo}

%_________________________________________________________________________________________
\subsection{Propiedades de Markov}
%_________________________________________________________________________________________

Sea $E$ espacio de estados, tal que $E$ es un espacio de Rad\'on, $\mathcal{B}\left(E\right)$ $\sigma$-\'algebra de Borel en $E$, que se denotar\'a por $\mathcal{E}$.

Sea $\left(X,\mathcal{G},\prob\right)$ espacio de probabilidad, $I\subset\rea$ conjunto de índices. Sea $\mathcal{F}_{\leq t}$ la $\sigma$-\'algebra natural definida como $\sigma\left\{f\left(X_{r}\right):r\in I, rleq t,f\in\mathcal{E}\right\}$. Se considerar\'a una $\sigma$-\'algebra m\'as general, $ \left(\mathcal{G}_{t}\right)$ tal que $\left(X_{t}\right)$ sea $\mathcal{E}$-adaptado.

\begin{Def}
Una familia $\left(P_{s,t}\right)$ de kernels de Markov en $\left(E,\mathcal{E}\right)$ indexada por pares $s,t\in I$, con $s\leq t$ es una funci\'on de transici\'on en $\ER$, si  para todo $r\leq s< t$ en $I$ y todo $x\in E$, $B\in\mathcal{E}$
\begin{equation}\label{Eq.Kernels}
P_{r,t}\left(x,B\right)=\int_{E}P_{r,s}\left(x,dy\right)P_{s,t}\left(y,B\right)\footnote{Ecuaci\'on de Chapman-Kolmogorov}.
\end{equation}
\end{Def}

Se dice que la funci\'on de transici\'on $\KM$ en $\ER$ es la funci\'on de transici\'on para un proceso $\PE$  con valores en $E$ y que satisface la propiedad de Markov\footnote{\begin{equation}\label{Eq.1.4.S}
\prob\left\{H|\mathcal{G}_{t}\right\}=\prob\left\{H|X_{t}\right\}\textrm{ }H\in p\mathcal{F}_{\geq t}.
\end{equation}} (\ref{Eq.1.4.S}) relativa a $\left(\mathcal{G}_{t}\right)$ si 

\begin{equation}\label{Eq.1.6.S}
\prob\left\{f\left(X_{t}\right)|\mathcal{G}_{s}\right\}=P_{s,t}f\left(X_{t}\right)\textrm{ }s\leq t\in I,\textrm{ }f\in b\mathcal{E}.
\end{equation}

\begin{Def}
Una familia $\left(P_{t}\right)_{t\geq0}$ de kernels de Markov en $\ER$ es llamada {\em Semigrupo de Transici\'on de Markov} o {\em Semigrupo de Transici\'on} si
\[P_{t+s}f\left(x\right)=P_{t}\left(P_{s}f\right)\left(x\right),\textrm{ }t,s\geq0,\textrm{ }x\in E\textrm{ }f\in b\mathcal{E}.\]
\end{Def}
\begin{Note}
Si la funci\'on de transici\'on $\KM$ es llamada homog\'enea si $P_{s,t}=P_{t-s}$.
\end{Note}

Un proceso de Markov que satisface la ecuaci\'on (\ref{Eq.1.6.S}) con funci\'on de transici\'on homog\'enea $\left(P_{t}\right)$ tiene la propiedad caracter\'istica
\begin{equation}\label{Eq.1.8.S}
\prob\left\{f\left(X_{t+s}\right)|\mathcal{G}_{t}\right\}=P_{s}f\left(X_{t}\right)\textrm{ }t,s\geq0,\textrm{ }f\in b\mathcal{E}.
\end{equation}
La ecuaci\'on anterior es la {\em Propiedad Simple de Markov} de $X$ relativa a $\left(P_{t}\right)$.

En este sentido el proceso $\PE$ cumple con la propiedad de Markov (\ref{Eq.1.8.S}) relativa a $\left(\Omega,\mathcal{G},\mathcal{G}_{t},\prob\right)$ con semigrupo de transici\'on $\left(P_{t}\right)$.
%_________________________________________________________________________________________
\subsection{Primer Condici\'on de Regularidad}
%_________________________________________________________________________________________
%\newcommand{\EM}{\left(\Omega,\mathcal{G},\prob\right)}
%\newcommand{\E4}{\left(\Omega,\mathcal{G},\mathcal{G}_{t},\prob\right)}
\begin{Def}
Un proceso estoc\'astico $\PE$ definido en $\left(\Omega,\mathcal{G},\prob\right)$ con valores en el espacio topol\'ogico $E$ es continuo por la derecha si cada trayectoria muestral $t\rightarrow X_{t}\left(w\right)$ es un mapeo continuo por la derecha de $I$ en $E$.
\end{Def}

\begin{Def}[HD1]\label{Eq.2.1.S}
Un semigrupo de Markov $\left/P_{t}\right)$ en un espacio de Rad\'on $E$ se dice que satisface la condici\'on {\em HD1} si, dada una medida de probabilidad $\mu$ en $E$, existe una $\sigma$-\'algebra $\mathcal{E^{*}}$ con $\mathcal{E}\subset\mathcal{E}$ y $P_{t}\left(b\mathcal{E}^{*}\right)\subset b\mathcal{E}^{*}$, y un $\mathcal{E}^{*}$-proceso $E$-valuado continuo por la derecha $\PE$ en alg\'un espacio de probabilidad filtrado $\left(\Omega,\mathcal{G},\mathcal{G}_{t},\prob\right)$ tal que $X=\left(\Omega,\mathcal{G},\mathcal{G}_{t},\prob\right)$ es de Markov (Homog\'eneo) con semigrupo de transici\'on $(P_{t})$ y distribuci\'on inicial $\mu$.
\end{Def}

Considerese la colecci\'on de variables aleatorias $X_{t}$ definidas en alg\'un espacio de probabilidad, y una colecci\'on de medidas $\mathbf{P}^{x}$ tales que $\mathbf{P}^{x}\left\{X_{0}=x\right\}$, y bajo cualquier $\mathbf{P}^{x}$, $X_{t}$ es de Markov con semigrupo $\left(P_{t}\right)$. $\mathbf{P}^{x}$ puede considerarse como la distribuci\'on condicional de $\mathbf{P}$ dado $X_{0}=x$.

\begin{Def}\label{Def.2.2.S}
Sea $E$ espacio de Rad\'on, $\SG$ semigrupo de Markov en $\ER$. La colecci\'on $\mathbf{X}=\left(\Omega,\mathcal{G},\mathcal{G}_{t},X_{t},\theta_{t},\CM\right)$ es un proceso $\mathcal{E}$-Markov continuo por la derecha simple, con espacio de estados $E$ y semigrupo de transici\'on $\SG$ en caso de que $\mathbf{X}$ satisfaga las siguientes condiciones:
\begin{itemize}
\item[i)] $\left(\Omega,\mathcal{G},\mathcal{G}_{t}\right)$ es un espacio de medida filtrado, y $X_{t}$ es un proceso $E$-valuado continuo por la derecha $\mathcal{E}^{*}$-adaptado a $\left(\mathcal{G}_{t}\right)$;

\item[ii)] $\left(\theta_{t}\right)_{t\geq0}$ es una colecci\'on de operadores {\em shift} para $X$, es decir, mapea $\Omega$ en s\'i mismo satisfaciendo para $t,s\geq0$,

\begin{equation}\label{Eq.Shift}
\theta_{t}\circ\theta_{s}=\theta_{t+s}\textrm{ y }X_{t}\circ\theta_{t}=X_{t+s};
\end{equation}

\item[iii)] Para cualquier $x\in E$,$\CM\left\{X_{0}=x\right\}=1$, y el proceso $\PE$ tiene la propiedad de Markov (\ref{Eq.1.8.S}) con semigrupo de transici\'on $\SG$ relativo a $\left(\Omega,\mathcal{G},\mathcal{G}_{t},\CM\right)$.
\end{itemize}
\end{Def}

\begin{Def}[HD2]\label{Eq.2.2.S}
Para cualquier $\alpha>0$ y cualquier $f\in S^{\alpha}$, el proceso $t\rightarrow f\left(X_{t}\right)$ es continuo por la derecha casi seguramente.
\end{Def}

\begin{Def}\label{Def.PD}
Un sistema $\mathbf{X}=\left(\Omega,\mathcal{G},\mathcal{G}_{t},X_{t},\theta_{t},\CM\right)$ es un proceso derecho en el espacio de Rad\'on $E$ con semigrupo de transici\'on $\SG$ provisto de:
\begin{itemize}
\item[i)] $\mathbf{X}$ es una realizaci\'on  continua por la derecha, \ref{Def.2.2.S}, de $\SG$.

\item[ii)] $\mathbf{X}$ satisface la condicion HD2, \ref{Eq.2.2.S}, relativa a $\mathcal{G}_{t}$.

\item[iii)] $\mathcal{G}_{t}$ es aumentado y continuo por la derecha.
\end{itemize}
\end{Def}




\begin{Lema}[Lema 4.2, Dai\cite{Dai}]\label{Lema4.2}
Sea $\left\{x_{n}\right\}\subset \mathbf{X}$ con
$|x_{n}|\rightarrow\infty$, conforme $n\rightarrow\infty$. Suponga
que
\[lim_{n\rightarrow\infty}\frac{1}{|x_{n}|}U\left(0\right)=\overline{U}\]
y
\[lim_{n\rightarrow\infty}\frac{1}{|x_{n}|}V\left(0\right)=\overline{V}.\]

Entonces, conforme $n\rightarrow\infty$, casi seguramente

\begin{equation}\label{E1.4.2}
\frac{1}{|x_{n}|}\Phi^{k}\left(\left[|x_{n}|t\right]\right)\rightarrow
P_{k}^{'}t\textrm{, u.o.c.,}
\end{equation}

\begin{equation}\label{E1.4.3}
\frac{1}{|x_{n}|}E^{x_{n}}_{k}\left(|x_{n}|t\right)\rightarrow
\alpha_{k}\left(t-\overline{U}_{k}\right)^{+}\textrm{, u.o.c.,}
\end{equation}

\begin{equation}\label{E1.4.4}
\frac{1}{|x_{n}|}S^{x_{n}}_{k}\left(|x_{n}|t\right)\rightarrow
\mu_{k}\left(t-\overline{V}_{k}\right)^{+}\textrm{, u.o.c.,}
\end{equation}

donde $\left[t\right]$ es la parte entera de $t$ y
$\mu_{k}=1/m_{k}=1/\esp\left[\eta_{k}\left(1\right)\right]$.
\end{Lema}

\begin{Lema}[Lema 4.3, Dai\cite{Dai}]\label{Lema.4.3}
Sea $\left\{x_{n}\right\}\subset \mathbf{X}$ con
$|x_{n}|\rightarrow\infty$, conforme $n\rightarrow\infty$. Suponga
que
\[lim_{n\rightarrow\infty}\frac{1}{|x_{n}|}U\left(0\right)=\overline{U}_{k}\]
y
\[lim_{n\rightarrow\infty}\frac{1}{|x_{n}|}V\left(0\right)=\overline{V}_{k}.\]
\begin{itemize}
\item[a)] Conforme $n\rightarrow\infty$ casi seguramente,
\[lim_{n\rightarrow\infty}\frac{1}{|x_{n}|}U^{x_{n}}_{k}\left(|x_{n}|t\right)=\left(\overline{U}_{k}-t\right)^{+}\textrm{, u.o.c.}\]
y
\[lim_{n\rightarrow\infty}\frac{1}{|x_{n}|}V^{x_{n}}_{k}\left(|x_{n}|t\right)=\left(\overline{V}_{k}-t\right)^{+}.\]

\item[b)] Para cada $t\geq0$ fijo,
\[\left\{\frac{1}{|x_{n}|}U^{x_{n}}_{k}\left(|x_{n}|t\right),|x_{n}|\geq1\right\}\]
y
\[\left\{\frac{1}{|x_{n}|}V^{x_{n}}_{k}\left(|x_{n}|t\right),|x_{n}|\geq1\right\}\]
\end{itemize}
son uniformemente convergentes.
\end{Lema}

$S_{l}^{x}\left(t\right)$ es el n\'umero total de servicios
completados de la clase $l$, si la clase $l$ est\'a dando $t$
unidades de tiempo de servicio. Sea $T_{l}^{x}\left(x\right)$ el
monto acumulado del tiempo de servicio que el servidor
$s\left(l\right)$ gasta en los usuarios de la clase $l$ al tiempo
$t$. Entonces $S_{l}^{x}\left(T_{l}^{x}\left(t\right)\right)$ es
el n\'umero total de servicios completados para la clase $l$ al
tiempo $t$. Una fracci\'on de estos usuarios,
$\Phi_{l}^{x}\left(S_{l}^{x}\left(T_{l}^{x}\left(t\right)\right)\right)$,
se convierte en usuarios de la clase $k$.\\

Entonces, dado lo anterior, se tiene la siguiente representaci\'on
para el proceso de la longitud de la cola:\\

\begin{equation}
Q_{k}^{x}\left(t\right)=_{k}^{x}\left(0\right)+E_{k}^{x}\left(t\right)+\sum_{l=1}^{K}\Phi_{k}^{l}\left(S_{l}^{x}\left(T_{l}^{x}\left(t\right)\right)\right)-S_{k}^{x}\left(T_{k}^{x}\left(t\right)\right)
\end{equation}
para $k=1,\ldots,K$. Para $i=1,\ldots,d$, sea
\[I_{i}^{x}\left(t\right)=t-\sum_{j\in C_{i}}T_{k}^{x}\left(t\right).\]

Entonces $I_{i}^{x}\left(t\right)$ es el monto acumulado del
tiempo que el servidor $i$ ha estado desocupado al tiempo $t$. Se
est\'a asumiendo que las disciplinas satisfacen la ley de
conservaci\'on del trabajo, es decir, el servidor $i$ est\'a en
pausa solamente cuando no hay usuarios en la estaci\'on $i$.
Entonces, se tiene que

\begin{equation}
\int_{0}^{\infty}\left(\sum_{k\in
C_{i}}Q_{k}^{x}\left(t\right)\right)dI_{i}^{x}\left(t\right)=0,
\end{equation}
para $i=1,\ldots,d$.\\

Hacer
\[T^{x}\left(t\right)=\left(T_{1}^{x}\left(t\right),\ldots,T_{K}^{x}\left(t\right)\right)^{'},\]
\[I^{x}\left(t\right)=\left(I_{1}^{x}\left(t\right),\ldots,I_{K}^{x}\left(t\right)\right)^{'}\]
y
\[S^{x}\left(T^{x}\left(t\right)\right)=\left(S_{1}^{x}\left(T_{1}^{x}\left(t\right)\right),\ldots,S_{K}^{x}\left(T_{K}^{x}\left(t\right)\right)\right)^{'}.\]

Para una disciplina que cumple con la ley de conservaci\'on del
trabajo, en forma vectorial, se tiene el siguiente conjunto de
ecuaciones

\begin{equation}\label{Eq.MF.1.3}
Q^{x}\left(t\right)=Q^{x}\left(0\right)+E^{x}\left(t\right)+\sum_{l=1}^{K}\Phi^{l}\left(S_{l}^{x}\left(T_{l}^{x}\left(t\right)\right)\right)-S^{x}\left(T^{x}\left(t\right)\right),\\
\end{equation}

\begin{equation}\label{Eq.MF.2.3}
Q^{x}\left(t\right)\geq0,\\
\end{equation}

\begin{equation}\label{Eq.MF.3.3}
T^{x}\left(0\right)=0,\textrm{ y }\overline{T}^{x}\left(t\right)\textrm{ es no decreciente},\\
\end{equation}

\begin{equation}\label{Eq.MF.4.3}
I^{x}\left(t\right)=et-CT^{x}\left(t\right)\textrm{ es no
decreciente}\\
\end{equation}

\begin{equation}\label{Eq.MF.5.3}
\int_{0}^{\infty}\left(CQ^{x}\left(t\right)\right)dI_{i}^{x}\left(t\right)=0,\\
\end{equation}

\begin{equation}\label{Eq.MF.6.3}
\textrm{Condiciones adicionales en
}\left(\overline{Q}^{x}\left(\cdot\right),\overline{T}^{x}\left(\cdot\right)\right)\textrm{
espec\'ificas de la disciplina de la cola,}
\end{equation}

donde $e$ es un vector de unos de dimensi\'on $d$, $C$ es la
matriz definida por
\[C_{ik}=\left\{\begin{array}{cc}
1,& S\left(k\right)=i,\\
0,& \textrm{ en otro caso}.\\
\end{array}\right.
\]
Es necesario enunciar el siguiente Teorema que se utilizar\'a para
el Teorema \ref{Tma.4.2.Dai}:
\begin{Teo}[Teorema 4.1, Dai \cite{Dai}]
Considere una disciplina que cumpla la ley de conservaci\'on del
trabajo, para casi todas las trayectorias muestrales $\omega$ y
cualquier sucesi\'on de estados iniciales
$\left\{x_{n}\right\}\subset \mathbf{X}$, con
$|x_{n}|\rightarrow\infty$, existe una subsucesi\'on
$\left\{x_{n_{j}}\right\}$ con $|x_{n_{j}}|\rightarrow\infty$ tal
que
\begin{equation}\label{Eq.4.15}
\frac{1}{|x_{n_{j}}|}\left(Q^{x_{n_{j}}}\left(0\right),U^{x_{n_{j}}}\left(0\right),V^{x_{n_{j}}}\left(0\right)\right)\rightarrow\left(\overline{Q}\left(0\right),\overline{U},\overline{V}\right),
\end{equation}

\begin{equation}\label{Eq.4.16}
\frac{1}{|x_{n_{j}}|}\left(Q^{x_{n_{j}}}\left(|x_{n_{j}}|t\right),T^{x_{n_{j}}}\left(|x_{n_{j}}|t\right)\right)\rightarrow\left(\overline{Q}\left(t\right),\overline{T}\left(t\right)\right)\textrm{
u.o.c.}
\end{equation}

Adem\'as,
$\left(\overline{Q}\left(t\right),\overline{T}\left(t\right)\right)$
satisface las siguientes ecuaciones:
\begin{equation}\label{Eq.MF.1.3a}
\overline{Q}\left(t\right)=Q\left(0\right)+\left(\alpha
t-\overline{U}\right)^{+}-\left(I-P\right)^{'}M^{-1}\left(\overline{T}\left(t\right)-\overline{V}\right)^{+},
\end{equation}

\begin{equation}\label{Eq.MF.2.3a}
\overline{Q}\left(t\right)\geq0,\\
\end{equation}

\begin{equation}\label{Eq.MF.3.3a}
\overline{T}\left(t\right)\textrm{ es no decreciente y comienza en cero},\\
\end{equation}

\begin{equation}\label{Eq.MF.4.3a}
\overline{I}\left(t\right)=et-C\overline{T}\left(t\right)\textrm{
es no decreciente,}\\
\end{equation}

\begin{equation}\label{Eq.MF.5.3a}
\int_{0}^{\infty}\left(C\overline{Q}\left(t\right)\right)d\overline{I}\left(t\right)=0,\\
\end{equation}

\begin{equation}\label{Eq.MF.6.3a}
\textrm{Condiciones adicionales en
}\left(\overline{Q}\left(\cdot\right),\overline{T}\left(\cdot\right)\right)\textrm{
especficas de la disciplina de la cola,}
\end{equation}
\end{Teo}

\begin{Def}[Definici\'on 4.1, , Dai \cite{Dai}]
Sea una disciplina de servicio espec\'ifica. Cualquier l\'imite
$\left(\overline{Q}\left(\cdot\right),\overline{T}\left(\cdot\right)\right)$
en \ref{Eq.4.16} es un {\em flujo l\'imite} de la disciplina.
Cualquier soluci\'on (\ref{Eq.MF.1.3a})-(\ref{Eq.MF.6.3a}) es
llamado flujo soluci\'on de la disciplina. Se dice que el modelo de flujo l\'imite, modelo de flujo, de la disciplina de la cola es estable si existe una constante
$\delta>0$ que depende de $\mu,\alpha$ y $P$ solamente, tal que
cualquier flujo l\'imite con
$|\overline{Q}\left(0\right)|+|\overline{U}|+|\overline{V}|=1$, se
tiene que $\overline{Q}\left(\cdot+\delta\right)\equiv0$.
\end{Def}

\begin{Teo}[Teorema 4.2, Dai\cite{Dai}]\label{Tma.4.2.Dai}
Sea una disciplina fija para la cola, suponga que se cumplen las
condiciones (1.2)-(1.5). Si el modelo de flujo l\'imite de la
disciplina de la cola es estable, entonces la cadena de Markov $X$
que describe la din\'amica de la red bajo la disciplina es Harris
recurrente positiva.
\end{Teo}

Ahora se procede a escalar el espacio y el tiempo para reducir la
aparente fluctuaci\'on del modelo. Consid\'erese el proceso
\begin{equation}\label{Eq.3.7}
\overline{Q}^{x}\left(t\right)=\frac{1}{|x|}Q^{x}\left(|x|t\right)
\end{equation}
A este proceso se le conoce como el fluido escalado, y cualquier l\'imite $\overline{Q}^{x}\left(t\right)$ es llamado flujo l\'imite del proceso de longitud de la cola. Haciendo $|q|\rightarrow\infty$ mientras se mantiene el resto de las componentes fijas, cualquier punto l\'imite del proceso de longitud de la cola normalizado $\overline{Q}^{x}$ es soluci\'on del siguiente modelo de flujo.

Al conjunto de ecuaciones dadas en \ref{Eq.3.8}-\ref{Eq.3.13} se
le llama {\em Modelo de flujo} y al conjunto de todas las
soluciones del modelo de flujo
$\left(\overline{Q}\left(\cdot\right),\overline{T}
\left(\cdot\right)\right)$ se le denotar\'a por $\mathcal{Q}$.

Si se hace $|x|\rightarrow\infty$ sin restringir ninguna de las
componentes, tambi\'en se obtienen un modelo de flujo, pero en
este caso el residual de los procesos de arribo y servicio
introducen un retraso:

\begin{Def}[Definici\'on 3.3, Dai y Meyn \cite{DaiSean}]
El modelo de flujo es estable si existe un tiempo fijo $t_{0}$ tal
que $\overline{Q}\left(t\right)=0$, con $t\geq t_{0}$, para
cualquier $\overline{Q}\left(\cdot\right)\in\mathcal{Q}$ que
cumple con $|\overline{Q}\left(0\right)|=1$.
\end{Def}

El siguiente resultado se encuentra en Chen \cite{Chen}.
\begin{Lemma}[Lema 3.1, Dai y Meyn \cite{DaiSean}]
Si el modelo de flujo definido por \ref{Eq.3.8}-\ref{Eq.3.13} es
estable, entonces el modelo de flujo retrasado es tambi\'en
estable, es decir, existe $t_{0}>0$ tal que
$\overline{Q}\left(t\right)=0$ para cualquier $t\geq t_{0}$, para
cualquier soluci\'on del modelo de flujo retrasado cuya
condici\'on inicial $\overline{x}$ satisface que
$|\overline{x}|=|\overline{Q}\left(0\right)|+|\overline{A}\left(0\right)|+|\overline{B}\left(0\right)|\leq1$.
\end{Lemma}


Propiedades importantes para el modelo de flujo retrasado:

\begin{Prop}
 Sea $\left(\overline{Q},\overline{T},\overline{T}^{0}\right)$ un flujo l\'imite de \ref{Eq.4.4} y suponga que cuando $x\rightarrow\infty$ a lo largo de
una subsucesi\'on
\[\left(\frac{1}{|x|}Q_{k}^{x}\left(0\right),\frac{1}{|x|}A_{k}^{x}\left(0\right),\frac{1}{|x|}B_{k}^{x}\left(0\right),\frac{1}{|x|}B_{k}^{x,0}\left(0\right)\right)\rightarrow\left(\overline{Q}_{k}\left(0\right),0,0,0\right)\]
para $k=1,\ldots,K$. EL flujo l\'imite tiene las siguientes
propiedades, donde las propiedades de la derivada se cumplen donde
la derivada exista:
\begin{itemize}
 \item[i)] Los vectores de tiempo ocupado $\overline{T}\left(t\right)$ y $\overline{T}^{0}\left(t\right)$ son crecientes y continuas con
$\overline{T}\left(0\right)=\overline{T}^{0}\left(0\right)=0$.
\item[ii)] Para todo $t\geq0$
\[\sum_{k=1}^{K}\left[\overline{T}_{k}\left(t\right)+\overline{T}_{k}^{0}\left(t\right)\right]=t\]
\item[iii)] Para todo $1\leq k\leq K$
\[\overline{Q}_{k}\left(t\right)=\overline{Q}_{k}\left(0\right)+\alpha_{k}t-\mu_{k}\overline{T}_{k}\left(t\right)\]
\item[iv)]  Para todo $1\leq k\leq K$
\[\dot{{\overline{T}}}_{k}\left(t\right)=\beta_{k}\] para $\overline{Q}_{k}\left(t\right)=0$.
\item[v)] Para todo $k,j$
\[\mu_{k}^{0}\overline{T}_{k}^{0}\left(t\right)=\mu_{j}^{0}\overline{T}_{j}^{0}\left(t\right)\]
\item[vi)]  Para todo $1\leq k\leq K$
\[\mu_{k}\dot{{\overline{T}}}_{k}\left(t\right)=l_{k}\mu_{k}^{0}\dot{{\overline{T}}}_{k}^{0}\left(t\right)\] para $\overline{Q}_{k}\left(t\right)>0$.
\end{itemize}
\end{Prop}

\begin{Lema}[Lema 3.1 \cite{Chen}]\label{Lema3.1}
Si el modelo de flujo es estable, definido por las ecuaciones
(3.8)-(3.13), entonces el modelo de flujo retrasado tambin es
estable.
\end{Lema}

\begin{Teo}[Teorema 5.2 \cite{Chen}]\label{Tma.5.2}
Si el modelo de flujo lineal correspondiente a la red de cola es
estable, entonces la red de colas es estable.
\end{Teo}

\begin{Teo}[Teorema 5.1 \cite{Chen}]\label{Tma.5.1.Chen}
La red de colas es estable si existe una constante $t_{0}$ que
depende de $\left(\alpha,\mu,T,U\right)$ y $V$ que satisfagan las
ecuaciones (5.1)-(5.5), $Z\left(t\right)=0$, para toda $t\geq
t_{0}$.
\end{Teo}



\begin{Lema}[Lema 5.2 \cite{Gut}]\label{Lema.5.2.Gut}
Sea $\left\{\xi\left(k\right):k\in\ent\right\}$ sucesin de
variables aleatorias i.i.d. con valores en
$\left(0,\infty\right)$, y sea $E\left(t\right)$ el proceso de
conteo
\[E\left(t\right)=max\left\{n\geq1:\xi\left(1\right)+\cdots+\xi\left(n-1\right)\leq t\right\}.\]
Si $E\left[\xi\left(1\right)\right]<\infty$, entonces para
cualquier entero $r\geq1$
\begin{equation}
lim_{t\rightarrow\infty}\esp\left[\left(\frac{E\left(t\right)}{t}\right)^{r}\right]=\left(\frac{1}{E\left[\xi_{1}\right]}\right)^{r}
\end{equation}
de aqu, bajo estas condiciones
\begin{itemize}
\item[a)] Para cualquier $t>0$,
$sup_{t\geq\delta}\esp\left[\left(\frac{E\left(t\right)}{t}\right)^{r}\right]$

\item[b)] Las variables aleatorias
$\left\{\left(\frac{E\left(t\right)}{t}\right)^{r}:t\geq1\right\}$
son uniformemente integrables.
\end{itemize}
\end{Lema}

\begin{Teo}[Teorema 5.1: Ley Fuerte para Procesos de Conteo
\cite{Gut}]\label{Tma.5.1.Gut} Sea
$0<\mu<\esp\left(X_{1}\right]\leq\infty$. entonces

\begin{itemize}
\item[a)] $\frac{N\left(t\right)}{t}\rightarrow\frac{1}{\mu}$
a.s., cuando $t\rightarrow\infty$.


\item[b)]$\esp\left[\frac{N\left(t\right)}{t}\right]^{r}\rightarrow\frac{1}{\mu^{r}}$,
cuando $t\rightarrow\infty$ para todo $r>0$..
\end{itemize}
\end{Teo}


\begin{Prop}[Proposicin 5.1 \cite{DaiSean}]\label{Prop.5.1}
Suponga que los supuestos (A1) y (A2) se cumplen, adems suponga
que el modelo de flujo es estable. Entonces existe $t_{0}>0$ tal
que
\begin{equation}\label{Eq.Prop.5.1}
lim_{|x|\rightarrow\infty}\frac{1}{|x|^{p+1}}\esp_{x}\left[|X\left(t_{0}|x|\right)|^{p+1}\right]=0.
\end{equation}

\end{Prop}


\begin{Prop}[Proposici\'on 5.3 \cite{DaiSean}]
Sea $X$ proceso de estados para la red de colas, y suponga que se
cumplen los supuestos (A1) y (A2), entonces para alguna constante
positiva $C_{p+1}<\infty$, $\delta>0$ y un conjunto compacto
$C\subset X$.

\begin{equation}\label{Eq.5.4}
\esp_{x}\left[\int_{0}^{\tau_{C}\left(\delta\right)}\left(1+|X\left(t\right)|^{p}\right)dt\right]\leq
C_{p+1}\left(1+|x|^{p+1}\right)
\end{equation}
\end{Prop}

\begin{Prop}[Proposici\'on 5.4 \cite{DaiSean}]
Sea $X$ un proceso de Markov Borel Derecho en $X$, sea
$f:X\leftarrow\rea_{+}$ y defina para alguna $\delta>0$, y un
conjunto cerrado $C\subset X$
\[V\left(x\right):=\esp_{x}\left[\int_{0}^{\tau_{C}\left(\delta\right)}f\left(X\left(t\right)\right)dt\right]\]
para $x\in X$. Si $V$ es finito en todas partes y uniformemente
acotada en $C$, entonces existe $k<\infty$ tal que
\begin{equation}\label{Eq.5.11}
\frac{1}{t}\esp_{x}\left[V\left(x\right)\right]+\frac{1}{t}\int_{0}^{t}\esp_{x}\left[f\left(X\left(s\right)\right)ds\right]\leq\frac{1}{t}V\left(x\right)+k,
\end{equation}
para $x\in X$ y $t>0$.
\end{Prop}


\begin{Teo}[Teorema 5.5 \cite{DaiSean}]
Suponga que se cumplen (A1) y (A2), adems suponga que el modelo
de flujo es estable. Entonces existe una constante $k_{p}<\infty$
tal que
\begin{equation}\label{Eq.5.13}
\frac{1}{t}\int_{0}^{t}\esp_{x}\left[|Q\left(s\right)|^{p}\right]ds\leq
k_{p}\left\{\frac{1}{t}|x|^{p+1}+1\right\}
\end{equation}
para $t\geq0$, $x\in X$. En particular para cada condici\'on inicial
\begin{equation}\label{Eq.5.14}
Limsup_{t\rightarrow\infty}\frac{1}{t}\int_{0}^{t}\esp_{x}\left[|Q\left(s\right)|^{p}\right]ds\leq
k_{p}
\end{equation}
\end{Teo}

\begin{Teo}[Teorema 6.2\cite{DaiSean}]\label{Tma.6.2}
Suponga que se cumplen los supuestos (A1)-(A3) y que el modelo de
flujo es estable, entonces se tiene que
\[\parallel P^{t}\left(c,\cdot\right)-\pi\left(\cdot\right)\parallel_{f_{p}}\rightarrow0\]
para $t\rightarrow\infty$ y $x\in X$. En particular para cada
condicin inicial
\[lim_{t\rightarrow\infty}\esp_{x}\left[\left|Q_{t}\right|^{p}\right]=\esp_{\pi}\left[\left|Q_{0}\right|^{p}\right]<\infty\]
\end{Teo}


\begin{Teo}[Teorema 6.3\cite{DaiSean}]\label{Tma.6.3}
Suponga que se cumplen los supuestos (A1)-(A3) y que el modelo de
flujo es estable, entonces con
$f\left(x\right)=f_{1}\left(x\right)$, se tiene que
\[lim_{t\rightarrow\infty}t^{(p-1)\left|P^{t}\left(c,\cdot\right)-\pi\left(\cdot\right)\right|_{f}=0},\]
para $x\in X$. En particular, para cada condicin inicial
\[lim_{t\rightarrow\infty}t^{(p-1)\left|\esp_{x}\left[Q_{t}\right]-\esp_{\pi}\left[Q_{0}\right]\right|=0}.\]
\end{Teo}


\begin{Prop}[Proposici\'on 5.1, Dai y Meyn \cite{DaiSean}]\label{Prop.5.1.DaiSean}
Suponga que los supuestos A1) y A2) son ciertos y que el modelo de flujo es estable. Entonces existe $t_{0}>0$ tal que
\begin{equation}
lim_{|x|\rightarrow\infty}\frac{1}{|x|^{p+1}}\esp_{x}\left[|X\left(t_{0}|x|\right)|^{p+1}\right]=0
\end{equation}
\end{Prop}

\begin{Lemma}[Lema 5.2, Dai y Meyn \cite{DaiSean}]\label{Lema.5.2.DaiSean}
 Sea $\left\{\zeta\left(k\right):k\in \mathbb{z}\right\}$ una sucesi\'on independiente e id\'enticamente distribuida que toma valores en $\left(0,\infty\right)$,
y sea
$E\left(t\right)=max\left(n\geq1:\zeta\left(1\right)+\cdots+\zeta\left(n-1\right)\leq
t\right)$. Si $\esp\left[\zeta\left(1\right)\right]<\infty$,
entonces para cualquier entero $r\geq1$
\begin{equation}
 lim_{t\rightarrow\infty}\esp\left[\left(\frac{E\left(t\right)}{t}\right)^{r}\right]=\left(\frac{1}{\esp\left[\zeta_{1}\right]}\right)^{r}.
\end{equation}
Luego, bajo estas condiciones:
\begin{itemize}
 \item[a)] para cualquier $\delta>0$, $\sup_{t\geq\delta}\esp\left[\left(\frac{E\left(t\right)}{t}\right)^{r}\right]<\infty$
\item[b)] las variables aleatorias
$\left\{\left(\frac{E\left(t\right)}{t}\right)^{r}:t\geq1\right\}$
son uniformemente integrables.
\end{itemize}
\end{Lemma}

\begin{Teo}[Teorema 5.5, Dai y Meyn \cite{DaiSean}]\label{Tma.5.5.DaiSean}
Suponga que los supuestos A1) y A2) se cumplen y que el modelo de
flujo es estable. Entonces existe una constante $\kappa_{p}$ tal
que
\begin{equation}
\frac{1}{t}\int_{0}^{t}\esp_{x}\left[|Q\left(s\right)|^{p}\right]ds\leq\kappa_{p}\left\{\frac{1}{t}|x|^{p+1}+1\right\}
\end{equation}
para $t>0$ y $x\in X$. En particular, para cada condici\'on
inicial
\begin{eqnarray*}
\limsup_{t\rightarrow\infty}\frac{1}{t}\int_{0}^{t}\esp_{x}\left[|Q\left(s\right)|^{p}\right]ds\leq\kappa_{p}.
\end{eqnarray*}
\end{Teo}

\begin{Teo}[Teorema 6.2, Dai y Meyn \cite{DaiSean}]\label{Tma.6.2.DaiSean}
Suponga que se cumplen los supuestos A1), A2) y A3) y que el
modelo de flujo es estable. Entonces se tiene que
\begin{equation}
\left\|P^{t}\left(x,\cdot\right)-\pi\left(\cdot\right)\right\|_{f_{p}}\textrm{,
}t\rightarrow\infty,x\in X.
\end{equation}
En particular para cada condici\'on inicial
\begin{eqnarray*}
\lim_{t\rightarrow\infty}\esp_{x}\left[|Q\left(t\right)|^{p}\right]=\esp_{\pi}\left[|Q\left(0\right)|^{p}\right]\leq\kappa_{r}
\end{eqnarray*}
\end{Teo}
\begin{Teo}[Teorema 6.3, Dai y Meyn \cite{DaiSean}]\label{Tma.6.3.DaiSean}
Suponga que se cumplen los supuestos A1), A2) y A3) y que el
modelo de flujo es estable. Entonces con
$f\left(x\right)=f_{1}\left(x\right)$ se tiene
\begin{equation}
\lim_{t\rightarrow\infty}t^{p-1}\left\|P^{t}\left(x,\cdot\right)-\pi\left(\cdot\right)\right\|_{f}=0.
\end{equation}
En particular para cada condici\'on inicial
\begin{eqnarray*}
\lim_{t\rightarrow\infty}t^{p-1}|\esp_{x}\left[Q\left(t\right)\right]-\esp_{\pi}\left[Q\left(0\right)\right]|=0.
\end{eqnarray*}
\end{Teo}

\begin{Teo}[Teorema 6.4, Dai y Meyn \cite{DaiSean}]\label{Tma.6.4.DaiSean}
Suponga que se cumplen los supuestos A1), A2) y A3) y que el
modelo de flujo es estable. Sea $\nu$ cualquier distribuci\'on de
probabilidad en $\left(X,\mathcal{B}_{X}\right)$, y $\pi$ la
distribuci\'on estacionaria de $X$.
\begin{itemize}
\item[i)] Para cualquier $f:X\leftarrow\rea_{+}$
\begin{equation}
\lim_{t\rightarrow\infty}\frac{1}{t}\int_{o}^{t}f\left(X\left(s\right)\right)ds=\pi\left(f\right):=\int
f\left(x\right)\pi\left(dx\right)
\end{equation}
$\prob$-c.s.

\item[ii)] Para cualquier $f:X\leftarrow\rea_{+}$ con
$\pi\left(|f|\right)<\infty$, la ecuaci\'on anterior se cumple.
\end{itemize}
\end{Teo}

\begin{Teo}[Teorema 2.2, Down \cite{Down}]\label{Tma2.2.Down}
Suponga que el fluido modelo es inestable en el sentido de que
para alguna $\epsilon_{0},c_{0}\geq0$,
\begin{equation}\label{Eq.Inestability}
|Q\left(T\right)|\geq\epsilon_{0}T-c_{0}\textrm{,   }T\geq0,
\end{equation}
para cualquier condici\'on inicial $Q\left(0\right)$, con
$|Q\left(0\right)|=1$. Entonces para cualquier $0<q\leq1$, existe
$B<0$ tal que para cualquier $|x|\geq B$,
\begin{equation}
\prob_{x}\left\{\mathbb{X}\rightarrow\infty\right\}\geq q.
\end{equation}
\end{Teo}



Es necesario hacer los siguientes supuestos sobre el
comportamiento del sistema de visitas c\'iclicas:
\begin{itemize}
\item Los tiempos de interarribo a la $k$-\'esima cola, son de la
forma $\left\{\xi_{k}\left(n\right)\right\}_{n\geq1}$, con la
propiedad de que son independientes e id{\'e}nticamente
distribuidos,
\item Los tiempos de servicio
$\left\{\eta_{k}\left(n\right)\right\}_{n\geq1}$ tienen la
propiedad de ser independientes e id{\'e}nticamente distribuidos,
\item Se define la tasa de arribo a la $k$-{\'e}sima cola como
$\lambda_{k}=1/\esp\left[\xi_{k}\left(1\right)\right]$,
\item la tasa de servicio para la $k$-{\'e}sima cola se define
como $\mu_{k}=1/\esp\left[\eta_{k}\left(1\right)\right]$,
\item tambi{\'e}n se define $\rho_{k}:=\lambda_{k}/\mu_{k}$, la
intensidad de tr\'afico del sistema o carga de la red, donde es
necesario que $\rho<1$ para cuestiones de estabilidad.
\end{itemize}



%_________________________________________________________________________
\subsection{Procesos Fuerte de Markov}
%_________________________________________________________________________
En Dai \cite{Dai} se muestra que para una amplia serie de disciplinas
de servicio el proceso $X$ es un Proceso Fuerte de
Markov, y por tanto se puede asumir que


Para establecer que $X=\left\{X\left(t\right),t\geq0\right\}$ es
un Proceso Fuerte de Markov, se siguen las secciones 2.3 y 2.4 de Kaspi and Mandelbaum \cite{KaspiMandelbaum}. \\

%______________________________________________________________
\subsubsection{Construcci\'on de un Proceso Determinista por partes, Davis
\cite{Davis}}.
%______________________________________________________________

%_________________________________________________________________________
\subsection{Procesos Harris Recurrentes Positivos}
%_________________________________________________________________________
Sea el proceso de Markov $X=\left\{X\left(t\right),t\geq0\right\}$
que describe la din\'amica de la red de colas. En lo que respecta
al supuesto (A3), en Dai y Meyn \cite{DaiSean} y Meyn y Down
\cite{MeynDown} hacen ver que este se puede sustituir por

\begin{itemize}
\item[A3')] Para el Proceso de Markov $X$, cada subconjunto
compacto de $X$ es un conjunto peque\~no.
\end{itemize}

Este supuesto es importante pues es un requisito para deducir la ergodicidad de la red.

%_________________________________________________________________________
\subsection{Construcci\'on de un Modelo de Flujo L\'imite}
%_________________________________________________________________________

Consideremos un caso m\'as simple para poner en contexto lo
anterior: para un sistema de visitas c\'iclicas se tiene que el
estado al tiempo $t$ es
\begin{equation}
X\left(t\right)=\left(Q\left(t\right),U\left(t\right),V\left(t\right)\right),
\end{equation}

donde $Q\left(t\right)$ es el n\'umero de usuarios formados en
cada estaci\'on. $U\left(t\right)$ es el tiempo restante antes de
que la siguiente clase $k$ de usuarios lleguen desde fuera del
sistema, $V\left(t\right)$ es el tiempo restante de servicio para
la clase $k$ de usuarios que est\'an siendo atendidos. Tanto
$U\left(t\right)$ como $V\left(t\right)$ se puede asumir que son
continuas por la derecha.

Sea
$x=\left(Q\left(0\right),U\left(0\right),V\left(0\right)\right)=\left(q,a,b\right)$,
el estado inicial de la red bajo una disciplina espec\'ifica para
la cola. Para $l\in\mathcal{E}$, donde $\mathcal{E}$ es el conjunto de clases de arribos externos, y $k=1,\ldots,K$ se define\\
\begin{eqnarray*}
E_{l}^{x}\left(t\right)&=&max\left\{r:U_{l}\left(0\right)+\xi_{l}\left(1\right)+\cdots+\xi_{l}\left(r-1\right)\leq
t\right\}\textrm{   }t\geq0,\\
S_{k}^{x}\left(t\right)&=&max\left\{r:V_{k}\left(0\right)+\eta_{k}\left(1\right)+\cdots+\eta_{k}\left(r-1\right)\leq
t\right\}\textrm{   }t\geq0.
\end{eqnarray*}

Para cada $k$ y cada $n$ se define

\begin{eqnarray*}\label{Eq.phi}
\Phi^{k}\left(n\right):=\sum_{i=1}^{n}\phi^{k}\left(i\right).
\end{eqnarray*}

donde $\phi^{k}\left(n\right)$ se define como el vector de ruta
para el $n$-\'esimo usuario de la clase $k$ que termina en la
estaci\'on $s\left(k\right)$, la $s$-\'eima componente de
$\phi^{k}\left(n\right)$ es uno si estos usuarios se convierten en
usuarios de la clase $l$ y cero en otro caso, por lo tanto
$\phi^{k}\left(n\right)$ es un vector {\em Bernoulli} de
dimensi\'on $K$ con par\'ametro $P_{k}^{'}$, donde $P_{k}$ denota
el $k$-\'esimo rengl\'on de $P=\left(P_{kl}\right)$.

Se asume que cada para cada $k$ la sucesi\'on $\phi^{k}\left(n\right)=\left\{\phi^{k}\left(n\right),n\geq1\right\}$
es independiente e id\'enticamente distribuida y que las
$\phi^{1}\left(n\right),\ldots,\phi^{K}\left(n\right)$ son
mutuamente independientes, adem\'as de independientes de los
procesos de arribo y de servicio.\\

\begin{Lema}[Lema 4.2, Dai\cite{Dai}]\label{Lema4.2}
Sea $\left\{x_{n}\right\}\subset \mathbf{X}$ con
$|x_{n}|\rightarrow\infty$, conforme $n\rightarrow\infty$. Suponga
que
\[lim_{n\rightarrow\infty}\frac{1}{|x_{n}|}U\left(0\right)=\overline{U}\]
y
\[lim_{n\rightarrow\infty}\frac{1}{|x_{n}|}V\left(0\right)=\overline{V}.\]

Entonces, conforme $n\rightarrow\infty$, casi seguramente

\begin{equation}\label{E1.4.2}
\frac{1}{|x_{n}|}\Phi^{k}\left(\left[|x_{n}|t\right]\right)\rightarrow
P_{k}^{'}t\textrm{, u.o.c.,}
\end{equation}

\begin{equation}\label{E1.4.3}
\frac{1}{|x_{n}|}E^{x_{n}}_{k}\left(|x_{n}|t\right)\rightarrow
\alpha_{k}\left(t-\overline{U}_{k}\right)^{+}\textrm{, u.o.c.,}
\end{equation}

\begin{equation}\label{E1.4.4}
\frac{1}{|x_{n}|}S^{x_{n}}_{k}\left(|x_{n}|t\right)\rightarrow
\mu_{k}\left(t-\overline{V}_{k}\right)^{+}\textrm{, u.o.c.,}
\end{equation}

donde $\left[t\right]$ es la parte entera de $t$ y
$\mu_{k}=1/m_{k}=1/\esp\left[\eta_{k}\left(1\right)\right]$.
\end{Lema}

\begin{Lema}[Lema 4.3, Dai\cite{Dai}]\label{Lema.4.3}
Sea $\left\{x_{n}\right\}\subset \mathbf{X}$ con
$|x_{n}|\rightarrow\infty$, conforme $n\rightarrow\infty$. Suponga
que
\[lim_{n\rightarrow\infty}\frac{1}{|x_{n}|}U\left(0\right)=\overline{U}_{k}\]
y
\[lim_{n\rightarrow\infty}\frac{1}{|x_{n}|}V\left(0\right)=\overline{V}_{k}.\]
\begin{itemize}
\item[a)] Conforme $n\rightarrow\infty$ casi seguramente,
\[lim_{n\rightarrow\infty}\frac{1}{|x_{n}|}U^{x_{n}}_{k}\left(|x_{n}|t\right)=\left(\overline{U}_{k}-t\right)^{+}\textrm{, u.o.c.}\]
y
\[lim_{n\rightarrow\infty}\frac{1}{|x_{n}|}V^{x_{n}}_{k}\left(|x_{n}|t\right)=\left(\overline{V}_{k}-t\right)^{+}.\]

\item[b)] Para cada $t\geq0$ fijo,
\[\left\{\frac{1}{|x_{n}|}U^{x_{n}}_{k}\left(|x_{n}|t\right),|x_{n}|\geq1\right\}\]
y
\[\left\{\frac{1}{|x_{n}|}V^{x_{n}}_{k}\left(|x_{n}|t\right),|x_{n}|\geq1\right\}\]
\end{itemize}
son uniformemente convergentes.
\end{Lema}

$S_{l}^{x}\left(t\right)$ es el n\'umero total de servicios
completados de la clase $l$, si la clase $l$ est\'a dando $t$
unidades de tiempo de servicio. Sea $T_{l}^{x}\left(x\right)$ el
monto acumulado del tiempo de servicio que el servidor
$s\left(l\right)$ gasta en los usuarios de la clase $l$ al tiempo
$t$. Entonces $S_{l}^{x}\left(T_{l}^{x}\left(t\right)\right)$ es
el n\'umero total de servicios completados para la clase $l$ al
tiempo $t$. Una fracci\'on de estos usuarios,
$\Phi_{l}^{x}\left(S_{l}^{x}\left(T_{l}^{x}\left(t\right)\right)\right)$,
se convierte en usuarios de la clase $k$.\\

Entonces, dado lo anterior, se tiene la siguiente representaci\'on
para el proceso de la longitud de la cola:\\

\begin{equation}
Q_{k}^{x}\left(t\right)=_{k}^{x}\left(0\right)+E_{k}^{x}\left(t\right)+\sum_{l=1}^{K}\Phi_{k}^{l}\left(S_{l}^{x}\left(T_{l}^{x}\left(t\right)\right)\right)-S_{k}^{x}\left(T_{k}^{x}\left(t\right)\right)
\end{equation}
para $k=1,\ldots,K$. Para $i=1,\ldots,d$, sea
\[I_{i}^{x}\left(t\right)=t-\sum_{j\in C_{i}}T_{k}^{x}\left(t\right).\]

Entonces $I_{i}^{x}\left(t\right)$ es el monto acumulado del
tiempo que el servidor $i$ ha estado desocupado al tiempo $t$. Se
est\'a asumiendo que las disciplinas satisfacen la ley de
conservaci\'on del trabajo, es decir, el servidor $i$ est\'a en
pausa solamente cuando no hay usuarios en la estaci\'on $i$.
Entonces, se tiene que

\begin{equation}
\int_{0}^{\infty}\left(\sum_{k\in
C_{i}}Q_{k}^{x}\left(t\right)\right)dI_{i}^{x}\left(t\right)=0,
\end{equation}
para $i=1,\ldots,d$.\\

Hacer
\[T^{x}\left(t\right)=\left(T_{1}^{x}\left(t\right),\ldots,T_{K}^{x}\left(t\right)\right)^{'},\]
\[I^{x}\left(t\right)=\left(I_{1}^{x}\left(t\right),\ldots,I_{K}^{x}\left(t\right)\right)^{'}\]
y
\[S^{x}\left(T^{x}\left(t\right)\right)=\left(S_{1}^{x}\left(T_{1}^{x}\left(t\right)\right),\ldots,S_{K}^{x}\left(T_{K}^{x}\left(t\right)\right)\right)^{'}.\]

Para una disciplina que cumple con la ley de conservaci\'on del
trabajo, en forma vectorial, se tiene el siguiente conjunto de
ecuaciones

\begin{equation}\label{Eq.MF.1.3}
Q^{x}\left(t\right)=Q^{x}\left(0\right)+E^{x}\left(t\right)+\sum_{l=1}^{K}\Phi^{l}\left(S_{l}^{x}\left(T_{l}^{x}\left(t\right)\right)\right)-S^{x}\left(T^{x}\left(t\right)\right),\\
\end{equation}

\begin{equation}\label{Eq.MF.2.3}
Q^{x}\left(t\right)\geq0,\\
\end{equation}

\begin{equation}\label{Eq.MF.3.3}
T^{x}\left(0\right)=0,\textrm{ y }\overline{T}^{x}\left(t\right)\textrm{ es no decreciente},\\
\end{equation}

\begin{equation}\label{Eq.MF.4.3}
I^{x}\left(t\right)=et-CT^{x}\left(t\right)\textrm{ es no
decreciente}\\
\end{equation}

\begin{equation}\label{Eq.MF.5.3}
\int_{0}^{\infty}\left(CQ^{x}\left(t\right)\right)dI_{i}^{x}\left(t\right)=0,\\
\end{equation}

\begin{equation}\label{Eq.MF.6.3}
\textrm{Condiciones adicionales en
}\left(\overline{Q}^{x}\left(\cdot\right),\overline{T}^{x}\left(\cdot\right)\right)\textrm{
espec\'ificas de la disciplina de la cola,}
\end{equation}

donde $e$ es un vector de unos de dimensi\'on $d$, $C$ es la
matriz definida por
\[C_{ik}=\left\{\begin{array}{cc}
1,& S\left(k\right)=i,\\
0,& \textrm{ en otro caso}.\\
\end{array}\right.
\]
Es necesario enunciar el siguiente Teorema que se utilizar\'a para
el Teorema \ref{Tma.4.2.Dai}:
\begin{Teo}[Teorema 4.1, Dai \cite{Dai}]
Considere una disciplina que cumpla la ley de conservaci\'on del
trabajo, para casi todas las trayectorias muestrales $\omega$ y
cualquier sucesi\'on de estados iniciales
$\left\{x_{n}\right\}\subset \mathbf{X}$, con
$|x_{n}|\rightarrow\infty$, existe una subsucesi\'on
$\left\{x_{n_{j}}\right\}$ con $|x_{n_{j}}|\rightarrow\infty$ tal
que
\begin{equation}\label{Eq.4.15}
\frac{1}{|x_{n_{j}}|}\left(Q^{x_{n_{j}}}\left(0\right),U^{x_{n_{j}}}\left(0\right),V^{x_{n_{j}}}\left(0\right)\right)\rightarrow\left(\overline{Q}\left(0\right),\overline{U},\overline{V}\right),
\end{equation}

\begin{equation}\label{Eq.4.16}
\frac{1}{|x_{n_{j}}|}\left(Q^{x_{n_{j}}}\left(|x_{n_{j}}|t\right),T^{x_{n_{j}}}\left(|x_{n_{j}}|t\right)\right)\rightarrow\left(\overline{Q}\left(t\right),\overline{T}\left(t\right)\right)\textrm{
u.o.c.}
\end{equation}

Adem\'as,
$\left(\overline{Q}\left(t\right),\overline{T}\left(t\right)\right)$
satisface las siguientes ecuaciones:
\begin{equation}\label{Eq.MF.1.3a}
\overline{Q}\left(t\right)=Q\left(0\right)+\left(\alpha
t-\overline{U}\right)^{+}-\left(I-P\right)^{'}M^{-1}\left(\overline{T}\left(t\right)-\overline{V}\right)^{+},
\end{equation}

\begin{equation}\label{Eq.MF.2.3a}
\overline{Q}\left(t\right)\geq0,\\
\end{equation}

\begin{equation}\label{Eq.MF.3.3a}
\overline{T}\left(t\right)\textrm{ es no decreciente y comienza en cero},\\
\end{equation}

\begin{equation}\label{Eq.MF.4.3a}
\overline{I}\left(t\right)=et-C\overline{T}\left(t\right)\textrm{
es no decreciente,}\\
\end{equation}

\begin{equation}\label{Eq.MF.5.3a}
\int_{0}^{\infty}\left(C\overline{Q}\left(t\right)\right)d\overline{I}\left(t\right)=0,\\
\end{equation}

\begin{equation}\label{Eq.MF.6.3a}
\textrm{Condiciones adicionales en
}\left(\overline{Q}\left(\cdot\right),\overline{T}\left(\cdot\right)\right)\textrm{
especficas de la disciplina de la cola,}
\end{equation}
\end{Teo}

\begin{Def}[Definici\'on 4.1, , Dai \cite{Dai}]
Sea una disciplina de servicio espec\'ifica. Cualquier l\'imite
$\left(\overline{Q}\left(\cdot\right),\overline{T}\left(\cdot\right)\right)$
en \ref{Eq.4.16} es un {\em flujo l\'imite} de la disciplina.
Cualquier soluci\'on (\ref{Eq.MF.1.3a})-(\ref{Eq.MF.6.3a}) es
llamado flujo soluci\'on de la disciplina. Se dice que el modelo de flujo l\'imite, modelo de flujo, de la disciplina de la cola es estable si existe una constante
$\delta>0$ que depende de $\mu,\alpha$ y $P$ solamente, tal que
cualquier flujo l\'imite con
$|\overline{Q}\left(0\right)|+|\overline{U}|+|\overline{V}|=1$, se
tiene que $\overline{Q}\left(\cdot+\delta\right)\equiv0$.
\end{Def}

\begin{Teo}[Teorema 4.2, Dai\cite{Dai}]\label{Tma.4.2.Dai}
Sea una disciplina fija para la cola, suponga que se cumplen las
condiciones (1.2)-(1.5). Si el modelo de flujo l\'imite de la
disciplina de la cola es estable, entonces la cadena de Markov $X$
que describe la din\'amica de la red bajo la disciplina es Harris
recurrente positiva.
\end{Teo}

Ahora se procede a escalar el espacio y el tiempo para reducir la
aparente fluctuaci\'on del modelo. Consid\'erese el proceso
\begin{equation}\label{Eq.3.7}
\overline{Q}^{x}\left(t\right)=\frac{1}{|x|}Q^{x}\left(|x|t\right)
\end{equation}
A este proceso se le conoce como el fluido escalado, y cualquier l\'imite $\overline{Q}^{x}\left(t\right)$ es llamado flujo l\'imite del proceso de longitud de la cola. Haciendo $|q|\rightarrow\infty$ mientras se mantiene el resto de las componentes fijas, cualquier punto l\'imite del proceso de longitud de la cola normalizado $\overline{Q}^{x}$ es soluci\'on del siguiente modelo de flujo.

\begin{Def}[Definici\'on 3.1, Dai y Meyn \cite{DaiSean}]
Un flujo l\'imite (retrasado) para una red bajo una disciplina de
servicio espec\'ifica se define como cualquier soluci\'on
 $\left(\overline{Q}\left(\cdot\right),\overline{T}\left(\cdot\right)\right)$ de las siguientes ecuaciones, donde
$\overline{Q}\left(t\right)=\left(\overline{Q}_{1}\left(t\right),\ldots,\overline{Q}_{K}\left(t\right)\right)^{'}$
y
$\overline{T}\left(t\right)=\left(\overline{T}_{1}\left(t\right),\ldots,\overline{T}_{K}\left(t\right)\right)^{'}$
\begin{equation}\label{Eq.3.8}
\overline{Q}_{k}\left(t\right)=\overline{Q}_{k}\left(0\right)+\alpha_{k}t-\mu_{k}\overline{T}_{k}\left(t\right)+\sum_{l=1}^{k}P_{lk}\mu_{l}\overline{T}_{l}\left(t\right)\\
\end{equation}
\begin{equation}\label{Eq.3.9}
\overline{Q}_{k}\left(t\right)\geq0\textrm{ para }k=1,2,\ldots,K,\\
\end{equation}
\begin{equation}\label{Eq.3.10}
\overline{T}_{k}\left(0\right)=0,\textrm{ y }\overline{T}_{k}\left(\cdot\right)\textrm{ es no decreciente},\\
\end{equation}
\begin{equation}\label{Eq.3.11}
\overline{I}_{i}\left(t\right)=t-\sum_{k\in C_{i}}\overline{T}_{k}\left(t\right)\textrm{ es no decreciente}\\
\end{equation}
\begin{equation}\label{Eq.3.12}
\overline{I}_{i}\left(\cdot\right)\textrm{ se incrementa al tiempo }t\textrm{ cuando }\sum_{k\in C_{i}}Q_{k}^{x}\left(t\right)dI_{i}^{x}\left(t\right)=0\\
\end{equation}
\begin{equation}\label{Eq.3.13}
\textrm{condiciones adicionales sobre
}\left(Q^{x}\left(\cdot\right),T^{x}\left(\cdot\right)\right)\textrm{
referentes a la disciplina de servicio}
\end{equation}
\end{Def}

Al conjunto de ecuaciones dadas en \ref{Eq.3.8}-\ref{Eq.3.13} se
le llama {\em Modelo de flujo} y al conjunto de todas las
soluciones del modelo de flujo
$\left(\overline{Q}\left(\cdot\right),\overline{T}
\left(\cdot\right)\right)$ se le denotar\'a por $\mathcal{Q}$.

Si se hace $|x|\rightarrow\infty$ sin restringir ninguna de las
componentes, tambi\'en se obtienen un modelo de flujo, pero en
este caso el residual de los procesos de arribo y servicio
introducen un retraso:

\begin{Def}[Definici\'on 3.2, Dai y Meyn \cite{DaiSean}]
El modelo de flujo retrasado de una disciplina de servicio en una
red con retraso
$\left(\overline{A}\left(0\right),\overline{B}\left(0\right)\right)\in\rea_{+}^{K+|A|}$
se define como el conjunto de ecuaciones dadas en
\ref{Eq.3.8}-\ref{Eq.3.13}, junto con la condici\'on:
\begin{equation}\label{CondAd.FluidModel}
\overline{Q}\left(t\right)=\overline{Q}\left(0\right)+\left(\alpha
t-\overline{A}\left(0\right)\right)^{+}-\left(I-P^{'}\right)M\left(\overline{T}\left(t\right)-\overline{B}\left(0\right)\right)^{+}
\end{equation}
\end{Def}

\begin{Def}[Definici\'on 3.3, Dai y Meyn \cite{DaiSean}]
El modelo de flujo es estable si existe un tiempo fijo $t_{0}$ tal
que $\overline{Q}\left(t\right)=0$, con $t\geq t_{0}$, para
cualquier $\overline{Q}\left(\cdot\right)\in\mathcal{Q}$ que
cumple con $|\overline{Q}\left(0\right)|=1$.
\end{Def}

El siguiente resultado se encuentra en Chen \cite{Chen}.
\begin{Lemma}[Lema 3.1, Dai y Meyn \cite{DaiSean}]
Si el modelo de flujo definido por \ref{Eq.3.8}-\ref{Eq.3.13} es
estable, entonces el modelo de flujo retrasado es tambi\'en
estable, es decir, existe $t_{0}>0$ tal que
$\overline{Q}\left(t\right)=0$ para cualquier $t\geq t_{0}$, para
cualquier soluci\'on del modelo de flujo retrasado cuya
condici\'on inicial $\overline{x}$ satisface que
$|\overline{x}|=|\overline{Q}\left(0\right)|+|\overline{A}\left(0\right)|+|\overline{B}\left(0\right)|\leq1$.
\end{Lemma}

%_________________________________________________________________________
\subsection{Modelo de Visitas C\'iclicas con un Servidor: Estabilidad}
%_________________________________________________________________________

%_________________________________________________________________________
\subsection{Teorema 2.1}
%_________________________________________________________________________



El resultado principal de Down \cite{Down} que relaciona la estabilidad del modelo de flujo con la estabilidad del sistema original

\begin{Teo}[Teorema 2.1, Down \cite{Down}]\label{Tma.2.1.Down}
Suponga que el modelo de flujo es estable, y que se cumplen los supuestos (A1) y (A2), entonces
\begin{itemize}
\item[i)] Para alguna constante $\kappa_{p}$, y para cada
condici\'on inicial $x\in X$
\begin{equation}\label{Estability.Eq1}
lim_{t\rightarrow\infty}\sup\frac{1}{t}\int_{0}^{t}\esp_{x}\left[|Q\left(s\right)|^{p}\right]ds\leq\kappa_{p},
\end{equation}
donde $p$ es el entero dado en (A2). Si adem\'as se cumple
la condici\'on (A3), entonces para cada condici\'on inicial:

\item[ii)] Los momentos transitorios convergen a su estado estacionario:
 \begin{equation}\label{Estability.Eq2}
lim_{t\rightarrow\infty}\esp_{x}\left[Q_{k}\left(t\right)^{r}\right]=\esp_{\pi}\left[Q_{k}\left(0\right)^{r}\right]\leq\kappa_{r},
\end{equation}
para $r=1,2,\ldots,p$ y $k=1,2,\ldots,K$. Donde $\pi$ es la
probabilidad invariante para $\mathbf{X}$.

\item[iii)]  El primer momento converge con raz\'on $t^{p-1}$:
\begin{equation}\label{Estability.Eq3}
lim_{t\rightarrow\infty}t^{p-1}|\esp_{x}\left[Q_{k}\left(t\right)\right]-\esp_{\pi}\left[Q\left(0\right)\right]=0.
\end{equation}

\item[iv)] La {\em Ley Fuerte de los grandes n\'umeros} se cumple:
\begin{equation}\label{Estability.Eq4}
lim_{t\rightarrow\infty}\frac{1}{t}\int_{0}^{t}Q_{k}^{r}\left(s\right)ds=\esp_{\pi}\left[Q_{k}\left(0\right)^{r}\right],\textrm{
}\prob_{x}\textrm{-c.s.}
\end{equation}
para $r=1,2,\ldots,p$ y $k=1,2,\ldots,K$.
\end{itemize}
\end{Teo}


\begin{Prop}[Proposici\'on 5.1, Dai y Meyn \cite{DaiSean}]\label{Prop.5.1.DaiSean}
Suponga que los supuestos A1) y A2) son ciertos y que el modelo de flujo es estable. Entonces existe $t_{0}>0$ tal que
\begin{equation}
lim_{|x|\rightarrow\infty}\frac{1}{|x|^{p+1}}\esp_{x}\left[|X\left(t_{0}|x|\right)|^{p+1}\right]=0
\end{equation}
\end{Prop}

\begin{Lemma}[Lema 5.2, Dai y Meyn \cite{DaiSean}]\label{Lema.5.2.DaiSean}
 Sea $\left\{\zeta\left(k\right):k\in \mathbb{z}\right\}$ una sucesi\'on independiente e id\'enticamente distribuida que toma valores en $\left(0,\infty\right)$,
y sea
$E\left(t\right)=max\left(n\geq1:\zeta\left(1\right)+\cdots+\zeta\left(n-1\right)\leq
t\right)$. Si $\esp\left[\zeta\left(1\right)\right]<\infty$,
entonces para cualquier entero $r\geq1$
\begin{equation}
 lim_{t\rightarrow\infty}\esp\left[\left(\frac{E\left(t\right)}{t}\right)^{r}\right]=\left(\frac{1}{\esp\left[\zeta_{1}\right]}\right)^{r}.
\end{equation}
Luego, bajo estas condiciones:
\begin{itemize}
 \item[a)] para cualquier $\delta>0$, $\sup_{t\geq\delta}\esp\left[\left(\frac{E\left(t\right)}{t}\right)^{r}\right]<\infty$
\item[b)] las variables aleatorias
$\left\{\left(\frac{E\left(t\right)}{t}\right)^{r}:t\geq1\right\}$
son uniformemente integrables.
\end{itemize}
\end{Lemma}

\begin{Teo}[Teorema 5.5, Dai y Meyn \cite{DaiSean}]\label{Tma.5.5.DaiSean}
Suponga que los supuestos A1) y A2) se cumplen y que el modelo de
flujo es estable. Entonces existe una constante $\kappa_{p}$ tal
que
\begin{equation}
\frac{1}{t}\int_{0}^{t}\esp_{x}\left[|Q\left(s\right)|^{p}\right]ds\leq\kappa_{p}\left\{\frac{1}{t}|x|^{p+1}+1\right\}
\end{equation}
para $t>0$ y $x\in X$. En particular, para cada condici\'on
inicial
\begin{eqnarray*}
\limsup_{t\rightarrow\infty}\frac{1}{t}\int_{0}^{t}\esp_{x}\left[|Q\left(s\right)|^{p}\right]ds\leq\kappa_{p}.
\end{eqnarray*}
\end{Teo}

\begin{Teo}[Teorema 6.2, Dai y Meyn \cite{DaiSean}]\label{Tma.6.2.DaiSean}
Suponga que se cumplen los supuestos A1), A2) y A3) y que el
modelo de flujo es estable. Entonces se tiene que
\begin{equation}
\left\|P^{t}\left(x,\cdot\right)-\pi\left(\cdot\right)\right\|_{f_{p}}\textrm{,
}t\rightarrow\infty,x\in X.
\end{equation}
En particular para cada condici\'on inicial
\begin{eqnarray*}
\lim_{t\rightarrow\infty}\esp_{x}\left[|Q\left(t\right)|^{p}\right]=\esp_{\pi}\left[|Q\left(0\right)|^{p}\right]\leq\kappa_{r}
\end{eqnarray*}
\end{Teo}
\begin{Teo}[Teorema 6.3, Dai y Meyn \cite{DaiSean}]\label{Tma.6.3.DaiSean}
Suponga que se cumplen los supuestos A1), A2) y A3) y que el
modelo de flujo es estable. Entonces con
$f\left(x\right)=f_{1}\left(x\right)$ se tiene
\begin{equation}
\lim_{t\rightarrow\infty}t^{p-1}\left\|P^{t}\left(x,\cdot\right)-\pi\left(\cdot\right)\right\|_{f}=0.
\end{equation}
En particular para cada condici\'on inicial
\begin{eqnarray*}
\lim_{t\rightarrow\infty}t^{p-1}|\esp_{x}\left[Q\left(t\right)\right]-\esp_{\pi}\left[Q\left(0\right)\right]|=0.
\end{eqnarray*}
\end{Teo}

\begin{Teo}[Teorema 6.4, Dai y Meyn \cite{DaiSean}]\label{Tma.6.4.DaiSean}
Suponga que se cumplen los supuestos A1), A2) y A3) y que el
modelo de flujo es estable. Sea $\nu$ cualquier distribuci\'on de
probabilidad en $\left(X,\mathcal{B}_{X}\right)$, y $\pi$ la
distribuci\'on estacionaria de $X$.
\begin{itemize}
\item[i)] Para cualquier $f:X\leftarrow\rea_{+}$
\begin{equation}
\lim_{t\rightarrow\infty}\frac{1}{t}\int_{o}^{t}f\left(X\left(s\right)\right)ds=\pi\left(f\right):=\int
f\left(x\right)\pi\left(dx\right)
\end{equation}
$\prob$-c.s.

\item[ii)] Para cualquier $f:X\leftarrow\rea_{+}$ con
$\pi\left(|f|\right)<\infty$, la ecuaci\'on anterior se cumple.
\end{itemize}
\end{Teo}

%_________________________________________________________________________
\subsection{Teorema 2.2}
%_________________________________________________________________________

\begin{Teo}[Teorema 2.2, Down \cite{Down}]\label{Tma2.2.Down}
Suponga que el fluido modelo es inestable en el sentido de que
para alguna $\epsilon_{0},c_{0}\geq0$,
\begin{equation}\label{Eq.Inestability}
|Q\left(T\right)|\geq\epsilon_{0}T-c_{0}\textrm{,   }T\geq0,
\end{equation}
para cualquier condici\'on inicial $Q\left(0\right)$, con
$|Q\left(0\right)|=1$. Entonces para cualquier $0<q\leq1$, existe
$B<0$ tal que para cualquier $|x|\geq B$,
\begin{equation}
\prob_{x}\left\{\mathbb{X}\rightarrow\infty\right\}\geq q.
\end{equation}
\end{Teo}

%_________________________________________________________________________
\subsection{Teorema 2.3}
%_________________________________________________________________________
\begin{Teo}[Teorema 2.3, Down \cite{Down}]\label{Tma2.3.Down}
Considere el siguiente valor:
\begin{equation}\label{Eq.Rho.1serv}
\rho=\sum_{k=1}^{K}\rho_{k}+max_{1\leq j\leq K}\left(\frac{\lambda_{j}}{\sum_{s=1}^{S}p_{js}\overline{N}_{s}}\right)\delta^{*}
\end{equation}
\begin{itemize}
\item[i)] Si $\rho<1$ entonces la red es estable, es decir, se cumple el teorema \ref{Tma.2.1.Down}.

\item[ii)] Si $\rho<1$ entonces la red es inestable, es decir, se cumple el teorema \ref{Tma2.2.Down}
\end{itemize}
\end{Teo}
%_____________________________________________________________________
\subsection{Definiciones  B\'asicas}
%_____________________________________________________________________
\begin{Def}
Sea $X$ un conjunto y $\mathcal{F}$ una $\sigma$-\'algebra de
subconjuntos de $X$, la pareja $\left(X,\mathcal{F}\right)$ es
llamado espacio medible. Un subconjunto $A$ de $X$ es llamado
medible, o medible con respecto a $\mathcal{F}$, si
$A\in\mathcal{F}$.
\end{Def}

\begin{Def}
Sea $\left(X,\mathcal{F},\mu\right)$ espacio de medida. Se dice
que la medida $\mu$ es $\sigma$-finita si se puede escribir
$X=\bigcup_{n\geq1}X_{n}$ con $X_{n}\in\mathcal{F}$ y
$\mu\left(X_{n}\right)<\infty$.
\end{Def}

\begin{Def}\label{Cto.Borel}
Sea $X$ el conjunto de los \'umeros reales $\rea$. El \'algebra de
Borel es la $\sigma$-\'algebra $B$ generada por los intervalos
abiertos $\left(a,b\right)\in\rea$. Cualquier conjunto en $B$ es
llamado {\em Conjunto de Borel}.
\end{Def}

\begin{Def}\label{Funcion.Medible}
Una funci\'on $f:X\rightarrow\rea$, es medible si para cualquier
n\'umero real $\alpha$ el conjunto
\[\left\{x\in X:f\left(x\right)>\alpha\right\}\]
pertenece a $X$. Equivalentemente, se dice que $f$ es medible si
\[f^{-1}\left(\left(\alpha,\infty\right)\right)=\left\{x\in X:f\left(x\right)>\alpha\right\}\in\mathcal{F}.\]
\end{Def}


\begin{Def}\label{Def.Cilindros}
Sean $\left(\Omega_{i},\mathcal{F}_{i}\right)$, $i=1,2,\ldots,$
espacios medibles y $\Omega=\prod_{i=1}^{\infty}\Omega_{i}$ el
conjunto de todas las sucesiones
$\left(\omega_{1},\omega_{2},\ldots,\right)$ tales que
$\omega_{i}\in\Omega_{i}$, $i=1,2,\ldots,$. Si
$B^{n}\subset\prod_{i=1}^{\infty}\Omega_{i}$, definimos
$B_{n}=\left\{\omega\in\Omega:\left(\omega_{1},\omega_{2},\ldots,\omega_{n}\right)\in
B^{n}\right\}$. Al conjunto $B_{n}$ se le llama {\em cilindro} con
base $B^{n}$, el cilindro es llamado medible si
$B^{n}\in\prod_{i=1}^{\infty}\mathcal{F}_{i}$.
\end{Def}


\begin{Def}\label{Def.Proc.Adaptado}[TSP, Ash \cite{RBA}]
Sea $X\left(t\right),t\geq0$ proceso estoc\'astico, el proceso es
adaptado a la familia de $\sigma$-\'algebras $\mathcal{F}_{t}$,
para $t\geq0$, si para $s<t$ implica que
$\mathcal{F}_{s}\subset\mathcal{F}_{t}$, y $X\left(t\right)$ es
$\mathcal{F}_{t}$-medible para cada $t$. Si no se especifica
$\mathcal{F}_{t}$ entonces se toma $\mathcal{F}_{t}$ como
$\mathcal{F}\left(X\left(s\right),s\leq t\right)$, la m\'as
peque\~na $\sigma$-\'algebra de subconjuntos de $\Omega$ que hace
que cada $X\left(s\right)$, con $s\leq t$ sea Borel medible.
\end{Def}


\begin{Def}\label{Def.Tiempo.Paro}[TSP, Ash \cite{RBA}]
Sea $\left\{\mathcal{F}\left(t\right),t\geq0\right\}$ familia
creciente de sub $\sigma$-\'algebras. es decir,
$\mathcal{F}\left(s\right)\subset\mathcal{F}\left(t\right)$ para
$s\leq t$. Un tiempo de paro para $\mathcal{F}\left(t\right)$ es
una funci\'on $T:\Omega\rightarrow\left[0,\infty\right]$ tal que
$\left\{T\leq t\right\}\in\mathcal{F}\left(t\right)$ para cada
$t\geq0$. Un tiempo de paro para el proceso estoc\'astico
$X\left(t\right),t\geq0$ es un tiempo de paro para las
$\sigma$-\'algebras
$\mathcal{F}\left(t\right)=\mathcal{F}\left(X\left(s\right)\right)$.
\end{Def}

\begin{Def}
Sea $X\left(t\right),t\geq0$ proceso estoc\'astico, con
$\left(S,\chi\right)$ espacio de estados. Se dice que el proceso
es adaptado a $\left\{\mathcal{F}\left(t\right)\right\}$, es
decir, si para cualquier $s,t\in I$, $I$ conjunto de \'indices,
$s<t$, se tiene que
$\mathcal{F}\left(s\right)\subset\mathcal{F}\left(t\right)$ y
$X\left(t\right)$ es $\mathcal{F}\left(t\right)$-medible,
\end{Def}

\begin{Def}
Sea $X\left(t\right),t\geq0$ proceso estoc\'astico, se dice que es
un Proceso de Markov relativo a $\mathcal{F}\left(t\right)$ o que
$\left\{X\left(t\right),\mathcal{F}\left(t\right)\right\}$ es de
Markov si y s\'olo si para cualquier conjunto $B\in\chi$,  y
$s,t\in I$, $s<t$ se cumple que
\begin{equation}\label{Prop.Markov}
P\left\{X\left(t\right)\in
B|\mathcal{F}\left(s\right)\right\}=P\left\{X\left(t\right)\in
B|X\left(s\right)\right\}.
\end{equation}
\end{Def}
\begin{Note}
Si se dice que $\left\{X\left(t\right)\right\}$ es un Proceso de
Markov sin mencionar $\mathcal{F}\left(t\right)$, se asumir\'a que
\begin{eqnarray*}
\mathcal{F}\left(t\right)=\mathcal{F}_{0}\left(t\right)=\mathcal{F}\left(X\left(r\right),r\leq
t\right),
\end{eqnarray*}
entonces la ecuaci\'on (\ref{Prop.Markov}) se puede escribir como
\begin{equation}
P\left\{X\left(t\right)\in B|X\left(r\right),r\leq s\right\} =
P\left\{X\left(t\right)\in B|X\left(s\right)\right\}
\end{equation}
\end{Note}

\begin{Teo}
Sea $\left(X_{n},\mathcal{F}_{n},n=0,1,\ldots,\right\}$ Proceso de
Markov con espacio de estados $\left(S_{0},\chi_{0}\right)$
generado por una distribuici\'on inicial $P_{o}$ y probabilidad de
transici\'on $p_{mn}$, para $m,n=0,1,\ldots,$ $m<n$, que por
notaci\'on se escribir\'a como $p\left(m,n,x,B\right)\rightarrow
p_{mn}\left(x,B\right)$. Sea $S$ tiempo de paro relativo a la
$\sigma$-\'algebra $\mathcal{F}_{n}$. Sea $T$ funci\'on medible,
$T:\Omega\rightarrow\left\{0,1,\ldots,\right\}$. Sup\'ongase que
$T\geq S$, entonces $T$ es tiempo de paro. Si $B\in\chi_{0}$,
entonces
\begin{equation}\label{Prop.Fuerte.Markov}
P\left\{X\left(T\right)\in
B,T<\infty|\mathcal{F}\left(S\right)\right\} =
p\left(S,T,X\left(s\right),B\right)
\end{equation}
en $\left\{T<\infty\right\}$.
\end{Teo}

Propiedades importantes para el modelo de flujo retrasado:

\begin{Prop}
 Sea $\left(\overline{Q},\overline{T},\overline{T}^{0}\right)$ un flujo l\'imite de \ref{Equation.4.4} y suponga que cuando $x\rightarrow\infty$ a lo largo de
una subsucesi\'on
\[\left(\frac{1}{|x|}Q_{k}^{x}\left(0\right),\frac{1}{|x|}A_{k}^{x}\left(0\right),\frac{1}{|x|}B_{k}^{x}\left(0\right),\frac{1}{|x|}B_{k}^{x,0}\left(0\right)\right)\rightarrow\left(\overline{Q}_{k}\left(0\right),0,0,0\right)\]
para $k=1,\ldots,K$. EL flujo l\'imite tiene las siguientes
propiedades, donde las propiedades de la derivada se cumplen donde
la derivada exista:
\begin{itemize}
 \item[i)] Los vectores de tiempo ocupado $\overline{T}\left(t\right)$ y $\overline{T}^{0}\left(t\right)$ son crecientes y continuas con
$\overline{T}\left(0\right)=\overline{T}^{0}\left(0\right)=0$.
\item[ii)] Para todo $t\geq0$
\[\sum_{k=1}^{K}\left[\overline{T}_{k}\left(t\right)+\overline{T}_{k}^{0}\left(t\right)\right]=t\]
\item[iii)] Para todo $1\leq k\leq K$
\[\overline{Q}_{k}\left(t\right)=\overline{Q}_{k}\left(0\right)+\alpha_{k}t-\mu_{k}\overline{T}_{k}\left(t\right)\]
\item[iv)]  Para todo $1\leq k\leq K$
\[\dot{{\overline{T}}}_{k}\left(t\right)=\beta_{k}\] para $\overline{Q}_{k}\left(t\right)=0$.
\item[v)] Para todo $k,j$
\[\mu_{k}^{0}\overline{T}_{k}^{0}\left(t\right)=\mu_{j}^{0}\overline{T}_{j}^{0}\left(t\right)\]
\item[vi)]  Para todo $1\leq k\leq K$
\[\mu_{k}\dot{{\overline{T}}}_{k}\left(t\right)=l_{k}\mu_{k}^{0}\dot{{\overline{T}}}_{k}^{0}\left(t\right)\] para $\overline{Q}_{k}\left(t\right)>0$.
\end{itemize}
\end{Prop}

\begin{Lema}[Lema 3.1 \cite{Chen}]\label{Lema3.1}
Si el modelo de flujo es estable, definido por las ecuaciones
(3.8)-(3.13), entonces el modelo de flujo retrasado tambin es
estable.
\end{Lema}

\begin{Teo}[Teorema 5.2 \cite{Chen}]\label{Tma.5.2}
Si el modelo de flujo lineal correspondiente a la red de cola es
estable, entonces la red de colas es estable.
\end{Teo}

\begin{Teo}[Teorema 5.1 \cite{Chen}]\label{Tma.5.1.Chen}
La red de colas es estable si existe una constante $t_{0}$ que
depende de $\left(\alpha,\mu,T,U\right)$ y $V$ que satisfagan las
ecuaciones (5.1)-(5.5), $Z\left(t\right)=0$, para toda $t\geq
t_{0}$.
\end{Teo}



\begin{Lema}[Lema 5.2 \cite{Gut}]\label{Lema.5.2.Gut}
Sea $\left\{\xi\left(k\right):k\in\ent\right\}$ sucesin de
variables aleatorias i.i.d. con valores en
$\left(0,\infty\right)$, y sea $E\left(t\right)$ el proceso de
conteo
\[E\left(t\right)=max\left\{n\geq1:\xi\left(1\right)+\cdots+\xi\left(n-1\right)\leq t\right\}.\]
Si $E\left[\xi\left(1\right)\right]<\infty$, entonces para
cualquier entero $r\geq1$
\begin{equation}
lim_{t\rightarrow\infty}\esp\left[\left(\frac{E\left(t\right)}{t}\right)^{r}\right]=\left(\frac{1}{E\left[\xi_{1}\right]}\right)^{r}
\end{equation}
de aqu, bajo estas condiciones
\begin{itemize}
\item[a)] Para cualquier $t>0$,
$sup_{t\geq\delta}\esp\left[\left(\frac{E\left(t\right)}{t}\right)^{r}\right]$

\item[b)] Las variables aleatorias
$\left\{\left(\frac{E\left(t\right)}{t}\right)^{r}:t\geq1\right\}$
son uniformemente integrables.
\end{itemize}
\end{Lema}

\begin{Teo}[Teorema 5.1: Ley Fuerte para Procesos de Conteo
\cite{Gut}]\label{Tma.5.1.Gut} Sea
$0<\mu<\esp\left(X_{1}\right]\leq\infty$. entonces

\begin{itemize}
\item[a)] $\frac{N\left(t\right)}{t}\rightarrow\frac{1}{\mu}$
a.s., cuando $t\rightarrow\infty$.


\item[b)]$\esp\left[\frac{N\left(t\right)}{t}\right]^{r}\rightarrow\frac{1}{\mu^{r}}$,
cuando $t\rightarrow\infty$ para todo $r>0$..
\end{itemize}
\end{Teo}


\begin{Prop}[Proposicin 5.1 \cite{DaiSean}]\label{Prop.5.1}
Suponga que los supuestos (A1) y (A2) se cumplen, adems suponga
que el modelo de flujo es estable. Entonces existe $t_{0}>0$ tal
que
\begin{equation}\label{Eq.Prop.5.1}
lim_{|x|\rightarrow\infty}\frac{1}{|x|^{p+1}}\esp_{x}\left[|X\left(t_{0}|x|\right)|^{p+1}\right]=0.
\end{equation}

\end{Prop}


\begin{Prop}[Proposici\'on 5.3 \cite{DaiSean}]
Sea $X$ proceso de estados para la red de colas, y suponga que se
cumplen los supuestos (A1) y (A2), entonces para alguna constante
positiva $C_{p+1}<\infty$, $\delta>0$ y un conjunto compacto
$C\subset X$.

\begin{equation}\label{Eq.5.4}
\esp_{x}\left[\int_{0}^{\tau_{C}\left(\delta\right)}\left(1+|X\left(t\right)|^{p}\right)dt\right]\leq
C_{p+1}\left(1+|x|^{p+1}\right)
\end{equation}
\end{Prop}

\begin{Prop}[Proposici\'on 5.4 \cite{DaiSean}]
Sea $X$ un proceso de Markov Borel Derecho en $X$, sea
$f:X\leftarrow\rea_{+}$ y defina para alguna $\delta>0$, y un
conjunto cerrado $C\subset X$
\[V\left(x\right):=\esp_{x}\left[\int_{0}^{\tau_{C}\left(\delta\right)}f\left(X\left(t\right)\right)dt\right]\]
para $x\in X$. Si $V$ es finito en todas partes y uniformemente
acotada en $C$, entonces existe $k<\infty$ tal que
\begin{equation}\label{Eq.5.11}
\frac{1}{t}\esp_{x}\left[V\left(x\right)\right]+\frac{1}{t}\int_{0}^{t}\esp_{x}\left[f\left(X\left(s\right)\right)ds\right]\leq\frac{1}{t}V\left(x\right)+k,
\end{equation}
para $x\in X$ y $t>0$.
\end{Prop}


\begin{Teo}[Teorema 5.5 \cite{DaiSean}]
Suponga que se cumplen (A1) y (A2), adems suponga que el modelo
de flujo es estable. Entonces existe una constante $k_{p}<\infty$
tal que
\begin{equation}\label{Eq.5.13}
\frac{1}{t}\int_{0}^{t}\esp_{x}\left[|Q\left(s\right)|^{p}\right]ds\leq
k_{p}\left\{\frac{1}{t}|x|^{p+1}+1\right\}
\end{equation}
para $t\geq0$, $x\in X$. En particular para cada condicin inicial
\begin{equation}\label{Eq.5.14}
Limsup_{t\rightarrow\infty}\frac{1}{t}\int_{0}^{t}\esp_{x}\left[|Q\left(s\right)|^{p}\right]ds\leq
k_{p}
\end{equation}
\end{Teo}

\begin{Teo}[Teorema 6.2\cite{DaiSean}]\label{Tma.6.2}
Suponga que se cumplen los supuestos (A1)-(A3) y que el modelo de
flujo es estable, entonces se tiene que
\[\parallel P^{t}\left(c,\cdot\right)-\pi\left(\cdot\right)\parallel_{f_{p}}\rightarrow0\]
para $t\rightarrow\infty$ y $x\in X$. En particular para cada
condicin inicial
\[lim_{t\rightarrow\infty}\esp_{x}\left[\left|Q_{t}\right|^{p}\right]=\esp_{\pi}\left[\left|Q_{0}\right|^{p}\right]<\infty\]
\end{Teo}


\begin{Teo}[Teorema 6.3\cite{DaiSean}]\label{Tma.6.3}
Suponga que se cumplen los supuestos (A1)-(A3) y que el modelo de
flujo es estable, entonces con
$f\left(x\right)=f_{1}\left(x\right)$, se tiene que
\[lim_{t\rightarrow\infty}t^{(p-1)\left|P^{t}\left(c,\cdot\right)-\pi\left(\cdot\right)\right|_{f}=0},\]
para $x\in X$. En particular, para cada condicin inicial
\[lim_{t\rightarrow\infty}t^{(p-1)\left|\esp_{x}\left[Q_{t}\right]-\esp_{\pi}\left[Q_{0}\right]\right|=0}.\]
\end{Teo}



Si $x$ es el n{\'u}mero de usuarios en la cola al comienzo del
periodo de servicio y $N_{s}\left(x\right)=N\left(x\right)$ es el
n{\'u}mero de usuarios que son atendidos con la pol{\'\i}tica $s$,
{\'u}nica en nuestro caso, durante un periodo de servicio,
entonces se asume que:
\begin{itemize}
\item[(S1.)]
\begin{equation}\label{S1}
lim_{x\rightarrow\infty}\esp\left[N\left(x\right)\right]=\overline{N}>0.
\end{equation}
\item[(S2.)]
\begin{equation}\label{S2}
\esp\left[N\left(x\right)\right]\leq \overline{N}, \end{equation}
para cualquier valor de $x$. \item La $n$-{\'e}sima ocurrencia va
acompa{\~n}ada con el tiempo de cambio de longitud
$\delta_{j,j+1}\left(n\right)$, independientes e id{\'e}nticamente
distribuidas, con
$\esp\left[\delta_{j,j+1}\left(1\right)\right]\geq0$. \item Se
define
\begin{equation}
\delta^{*}:=\sum_{j,j+1}\esp\left[\delta_{j,j+1}\left(1\right)\right].
\end{equation}

\item Los tiempos de inter-arribo a la cola $k$,son de la forma
$\left\{\xi_{k}\left(n\right)\right\}_{n\geq1}$, con la propiedad
de que son independientes e id{\'e}nticamente distribuidos.

\item Los tiempos de servicio
$\left\{\eta_{k}\left(n\right)\right\}_{n\geq1}$ tienen la
propiedad de ser independientes e id{\'e}nticamente distribuidos.

\item Se define la tasa de arribo a la $k$-{\'e}sima cola como
$\lambda_{k}=1/\esp\left[\xi_{k}\left(1\right)\right]$ y
adem{\'a}s se define

\item la tasa de servicio para la $k$-{\'e}sima cola como
$\mu_{k}=1/\esp\left[\eta_{k}\left(1\right)\right]$

\item tambi{\'e}n se define $\rho_{k}=\lambda_{k}/\mu_{k}$, donde
es necesario que $\rho<1$ para cuestiones de estabilidad.

\item De las pol{\'\i}ticas posibles solamente consideraremos la
pol{\'\i}tica cerrada (Gated).
\end{itemize}

Las Colas C\'iclicas se pueden describir por medio de un proceso
de Markov $\left(X\left(t\right)\right)_{t\in\rea}$, donde el
estado del sistema al tiempo $t\geq0$ est\'a dado por
\begin{equation}
X\left(t\right)=\left(Q\left(t\right),A\left(t\right),H\left(t\right),B\left(t\right),B^{0}\left(t\right),C\left(t\right)\right)
\end{equation}
definido en el espacio producto:
\begin{equation}
\mathcal{X}=\mathbb{Z}^{K}\times\rea_{+}^{K}\times\left(\left\{1,2,\ldots,K\right\}\times\left\{1,2,\ldots,S\right\}\right)^{M}\times\rea_{+}^{K}\times\rea_{+}^{K}\times\mathbb{Z}^{K},
\end{equation}

\begin{itemize}
\item $Q\left(t\right)=\left(Q_{k}\left(t\right),1\leq k\leq
K\right)$, es el n\'umero de usuarios en la cola $k$, incluyendo
aquellos que est\'an siendo atendidos provenientes de la
$k$-\'esima cola.

\item $A\left(t\right)=\left(A_{k}\left(t\right),1\leq k\leq
K\right)$, son los residuales de los tiempos de arribo en la cola
$k$. \item $H\left(t\right)$ es el par ordenado que consiste en la
cola que esta siendo atendida y la pol\'itica de servicio que se
utilizar\'a.

\item $B\left(t\right)$ es el tiempo de servicio residual.

\item $B^{0}\left(t\right)$ es el tiempo residual del cambio de
cola.

\item $C\left(t\right)$ indica el n\'umero de usuarios atendidos
durante la visita del servidor a la cola dada en
$H\left(t\right)$.
\end{itemize}

$A_{k}\left(t\right),B_{m}\left(t\right)$ y
$B_{m}^{0}\left(t\right)$ se suponen continuas por la derecha y
que satisfacen la propiedad fuerte de Markov, (\cite{Dai})

\begin{itemize}
\item Los tiempos de interarribo a la cola $k$,son de la forma
$\left\{\xi_{k}\left(n\right)\right\}_{n\geq1}$, con la propiedad
de que son independientes e id{\'e}nticamente distribuidos.

\item Los tiempos de servicio
$\left\{\eta_{k}\left(n\right)\right\}_{n\geq1}$ tienen la
propiedad de ser independientes e id{\'e}nticamente distribuidos.

\item Se define la tasa de arribo a la $k$-{\'e}sima cola como
$\lambda_{k}=1/\esp\left[\xi_{k}\left(1\right)\right]$ y
adem{\'a}s se define

\item la tasa de servicio para la $k$-{\'e}sima cola como
$\mu_{k}=1/\esp\left[\eta_{k}\left(1\right)\right]$

\item tambi{\'e}n se define $\rho_{k}=\lambda_{k}/\mu_{k}$, donde
es necesario que $\rho<1$ para cuestiones de estabilidad.

\item De las pol{\'\i}ticas posibles solamente consideraremos la
pol{\'\i}tica cerrada (Gated).
\end{itemize}


%_____________________________________________________


\subsection{Preliminares}



Sup\'ongase que el sistema consta de varias colas a los cuales
llegan uno o varios servidores a dar servicio a los usuarios
esperando en la cola.\\


Si $x$ es el n\'umero de usuarios en la cola al comienzo del
periodo de servicio y $N_{s}\left(x\right)=N\left(x\right)$ es el
n\'umero de usuarios que son atendidos con la pol\'itica $s$,
\'unica en nuestro caso, durante un periodo de servicio, entonces
se asume que:
\begin{itemize}
\item[1)]\label{S1}$lim_{x\rightarrow\infty}\esp\left[N\left(x\right)\right]=\overline{N}>0$
\item[2)]\label{S2}$\esp\left[N\left(x\right)\right]\leq\overline{N}$para
cualquier valor de $x$.
\end{itemize}
La manera en que atiende el servidor $m$-\'esimo, en este caso en
espec\'ifico solo lo ilustraremos con un s\'olo servidor, es la
siguiente:
\begin{itemize}
\item Al t\'ermino de la visita a la cola $j$, el servidor se
cambia a la cola $j^{'}$ con probabilidad
$r_{j,j^{'}}^{m}=r_{j,j^{'}}$

\item La $n$-\'esima ocurrencia va acompa\~nada con el tiempo de
cambio de longitud $\delta_{j,j^{'}}\left(n\right)$,
independientes e id\'enticamente distribuidas, con
$\esp\left[\delta_{j,j^{'}}\left(1\right)\right]\geq0$.

\item Sea $\left\{p_{j}\right\}$ la distribuci\'on invariante
estacionaria \'unica para la Cadena de Markov con matriz de
transici\'on $\left(r_{j,j^{'}}\right)$.

\item Finalmente, se define
\begin{equation}
\delta^{*}:=\sum_{j,j^{'}}p_{j}r_{j,j^{'}}\esp\left[\delta_{j,j^{'}}\left(i\right)\right].
\end{equation}
\end{itemize}

Veamos un caso muy espec\'ifico en el cual los tiempos de arribo a cada una de las colas se comportan de acuerdo a un proceso Poisson de la forma
$\left\{\xi_{k}\left(n\right)\right\}_{n\geq1}$, y los tiempos de servicio en cada una de las colas son variables aleatorias distribuidas exponencialmente e id\'enticamente distribuidas
$\left\{\eta_{k}\left(n\right)\right\}_{n\geq1}$, donde ambos procesos adem\'as cumplen la condici\'on de ser independientes entre si. Para la $k$-\'esima cola se define la tasa de arribo a la como
$\lambda_{k}=1/\esp\left[\xi_{k}\left(1\right)\right]$ y la tasa
de servicio como
$\mu_{k}=1/\esp\left[\eta_{k}\left(1\right)\right]$, finalmente se
define la carga de la cola como $\rho_{k}=\lambda_{k}/\mu_{k}$,
donde se pide que $\rho<1$, para garantizar la estabilidad del sistema.\\

Se denotar\'a por $Q_{k}\left(t\right)$ el n\'umero de usuarios en la cola $k$,
$A_{k}\left(t\right)$ los residuales de los tiempos entre arribos a la cola $k$;
para cada servidor $m$, se denota por $B_{m}\left(t\right)$ los residuales de los tiempos de servicio al tiempo $t$; $B_{m}^{0}\left(t\right)$ son los residuales de los tiempos de traslado de la cola $k$ a la pr\'oxima por atender, al tiempo $t$, finalmente sea $C_{m}\left(t\right)$ el n\'umero de usuarios atendidos durante la visita del servidor a la cola $k$ al tiempo $t$.\\


En este sentido el proceso para el sistema de visitas se puede definir como:

\begin{equation}\label{Esp.Edos.Down}
X\left(t\right)^{T}=\left(Q_{k}\left(t\right),A_{k}\left(t\right),B_{m}\left(t\right),B_{m}^{0}\left(t\right),C_{m}\left(t\right)\right)
\end{equation}
para $k=1,\ldots,K$ y $m=1,2,\ldots,M$. $X$ evoluciona en el
espacio de estados:
$X=\ent_{+}^{K}\times\rea_{+}^{K}\times\left(\left\{1,2,\ldots,K\right\}\times\left\{1,2,\ldots,S\right\}\right)^{M}\times\rea_{+}^{K}\times\ent_{+}^{K}$.\\

El sistema aqu\'i descrito debe de cumplir con los siguientes supuestos b\'asicos de un sistema de visitas:

Antes enunciemos los supuestos que regir\'an en la red.

\begin{itemize}
\item[A1)] $\xi_{1},\ldots,\xi_{K},\eta_{1},\ldots,\eta_{K}$ son
mutuamente independientes y son sucesiones independientes e
id\'enticamente distribuidas.

\item[A2)] Para alg\'un entero $p\geq1$
\begin{eqnarray*}
\esp\left[\xi_{l}\left(1\right)^{p+1}\right]<\infty\textrm{ para }l\in\mathcal{A}\textrm{ y }\\
\esp\left[\eta_{k}\left(1\right)^{p+1}\right]<\infty\textrm{ para
}k=1,\ldots,K.
\end{eqnarray*}
donde $\mathcal{A}$ es la clase de posibles arribos.

\item[A3)] Para $k=1,2,\ldots,K$ existe una funci\'on positiva
$q_{k}\left(x\right)$ definida en $\rea_{+}$, y un entero $j_{k}$,
tal que
\begin{eqnarray}
P\left(\xi_{k}\left(1\right)\geq x\right)>0\textrm{, para todo }x>0\\
P\left\{a\leq\sum_{i=1}^{j_{k}}\xi_{k}\left(i\right)\leq
b\right\}\geq\int_{a}^{b}q_{k}\left(x\right)dx, \textrm{ }0\leq
a<b.
\end{eqnarray}
\end{itemize}

En particular los procesos de tiempo entre arribos y de servicio
considerados con fines de ilustraci\'on de la metodolog\'ia
cumplen con el supuesto $A2)$ para $p=1$, es decir, ambos procesos
tienen primer y segundo momento finito.

En lo que respecta al supuesto (A3), en Dai y Meyn \cite{DaiSean}
hacen ver que este se puede sustituir por

\begin{itemize}
\item[A3')] Para el Proceso de Markov $X$, cada subconjunto
compacto de $X$ es un conjunto peque\~no, ver definici\'on
\ref{Def.Cto.Peq.}.
\end{itemize}

Es por esta raz\'on que con la finalidad de poder hacer uso de
$A3^{'})$ es necesario recurrir a los Procesos de Harris y en
particular a los Procesos Harris Recurrente:
%_______________________________________________________________________
\subsection{Procesos Harris Recurrente}
%_______________________________________________________________________

Por el supuesto (A1) conforme a Davis \cite{Davis}, se puede
definir el proceso de saltos correspondiente de manera tal que
satisfaga el supuesto (\ref{Sup3.1.Davis}), de hecho la
demostraci\'on est\'a basada en la l\'inea de argumentaci\'on de
Davis, (\cite{Davis}, p\'aginas 362-364).

Entonces se tiene un espacio de estados Markoviano. El espacio de
Markov descrito en Dai y Meyn \cite{DaiSean}

\[\left(\Omega,\mathcal{F},\mathcal{F}_{t},X\left(t\right),\theta_{t},P_{x}\right)\]
es un proceso de Borel Derecho (Sharpe \cite{Sharpe}) en el
espacio de estados medible $\left(X,\mathcal{B}_{X}\right)$. El
Proceso $X=\left\{X\left(t\right),t\geq0\right\}$ tiene
trayectorias continuas por la derecha, est\'a definida en
$\left(\Omega,\mathcal{F}\right)$ y est\'a adaptado a
$\left\{\mathcal{F}_{t},t\geq0\right\}$; la colecci\'on
$\left\{P_{x},x\in \mathbb{X}\right\}$ son medidas de probabilidad
en $\left(\Omega,\mathcal{F}\right)$ tales que para todo $x\in
\mathbb{X}$
\[P_{x}\left\{X\left(0\right)=x\right\}=1\] y
\[E_{x}\left\{f\left(X\circ\theta_{t}\right)|\mathcal{F}_{t}\right\}=E_{X}\left(\tau\right)f\left(X\right)\]
en $\left\{\tau<\infty\right\}$, $P_{x}$-c.s. Donde $\tau$ es un
$\mathcal{F}_{t}$-tiempo de paro
\[\left(X\circ\theta_{\tau}\right)\left(w\right)=\left\{X\left(\tau\left(w\right)+t,w\right),t\geq0\right\}\]
y $f$ es una funci\'on de valores reales acotada y medible con la
$\sigma$-algebra de Kolmogorov generada por los cilindros.\\

Sea $P^{t}\left(x,D\right)$, $D\in\mathcal{B}_{\mathbb{X}}$,
$t\geq0$ probabilidad de transici\'on de $X$ definida como
\[P^{t}\left(x,D\right)=P_{x}\left(X\left(t\right)\in
D\right)\]


\begin{Def}
Una medida no cero $\pi$ en
$\left(\mathbf{X},\mathcal{B}_{\mathbf{X}}\right)$ es {\bf
invariante} para $X$ si $\pi$ es $\sigma$-finita y
\[\pi\left(D\right)=\int_{\mathbf{X}}P^{t}\left(x,D\right)\pi\left(dx\right)\]
para todo $D\in \mathcal{B}_{\mathbf{X}}$, con $t\geq0$.
\end{Def}

\begin{Def}
El proceso de Markov $X$ es llamado Harris recurrente si existe
una medida de probabilidad $\nu$ en
$\left(\mathbf{X},\mathcal{B}_{\mathbf{X}}\right)$, tal que si
$\nu\left(D\right)>0$ y $D\in\mathcal{B}_{\mathbf{X}}$
\[P_{x}\left\{\tau_{D}<\infty\right\}\equiv1\] cuando
$\tau_{D}=inf\left\{t\geq0:X_{t}\in D\right\}$.
\end{Def}

\begin{Note}
\begin{itemize}
\item[i)] Si $X$ es Harris recurrente, entonces existe una \'unica
medida invariante $\pi$ (Getoor \cite{Getoor}).

\item[ii)] Si la medida invariante es finita, entonces puede
normalizarse a una medida de probabilidad, en este caso se le
llama Proceso {\em Harris recurrente positivo}.


\item[iii)] Cuando $X$ es Harris recurrente positivo se dice que
la disciplina de servicio es estable. En este caso $\pi$ denota la
distribuci\'on estacionaria y hacemos
\[P_{\pi}\left(\cdot\right)=\int_{\mathbf{X}}P_{x}\left(\cdot\right)\pi\left(dx\right)\]
y se utiliza $E_{\pi}$ para denotar el operador esperanza
correspondiente.
\end{itemize}
\end{Note}

\begin{Def}\label{Def.Cto.Peq.}
Un conjunto $D\in\mathcal{B_{\mathbf{X}}}$ es llamado peque\~no si
existe un $t>0$, una medida de probabilidad $\nu$ en
$\mathcal{B_{\mathbf{X}}}$, y un $\delta>0$ tal que
\[P^{t}\left(x,A\right)\geq\delta\nu\left(A\right)\] para $x\in
D,A\in\mathcal{B_{X}}$.
\end{Def}

La siguiente serie de resultados vienen enunciados y demostrados
en Dai \cite{Dai}:
\begin{Lema}[Lema 3.1, Dai\cite{Dai}]
Sea $B$ conjunto peque\~no cerrado, supongamos que
$P_{x}\left(\tau_{B}<\infty\right)\equiv1$ y que para alg\'un
$\delta>0$ se cumple que
\begin{equation}\label{Eq.3.1}
\sup\esp_{x}\left[\tau_{B}\left(\delta\right)\right]<\infty,
\end{equation}
donde
$\tau_{B}\left(\delta\right)=inf\left\{t\geq\delta:X\left(t\right)\in
B\right\}$. Entonces, $X$ es un proceso Harris Recurrente
Positivo.
\end{Lema}

\begin{Lema}[Lema 3.1, Dai \cite{Dai}]\label{Lema.3.}
Bajo el supuesto (A3), el conjunto $B=\left\{|x|\leq k\right\}$ es
un conjunto peque\~no cerrado para cualquier $k>0$.
\end{Lema}

\begin{Teo}[Teorema 3.1, Dai\cite{Dai}]\label{Tma.3.1}
Si existe un $\delta>0$ tal que
\begin{equation}
lim_{|x|\rightarrow\infty}\frac{1}{|x|}\esp|X^{x}\left(|x|\delta\right)|=0,
\end{equation}
entonces la ecuaci\'on (\ref{Eq.3.1}) se cumple para
$B=\left\{|x|\leq k\right\}$ con alg\'un $k>0$. En particular, $X$
es Harris Recurrente Positivo.
\end{Teo}

\begin{Note}
En Meyn and Tweedie \cite{MeynTweedie} muestran que si
$P_{x}\left\{\tau_{D}<\infty\right\}\equiv1$ incluso para solo un
conjunto peque\~no, entonces el proceso es Harris Recurrente.
\end{Note}

Entonces, tenemos que el proceso $X$ es un proceso de Markov que
cumple con los supuestos $A1)$-$A3)$, lo que falta de hacer es
construir el Modelo de Flujo bas\'andonos en lo hasta ahora
presentado.
%_______________________________________________________________________
\subsection{Modelo de Flujo}
%_______________________________________________________________________

Dada una condici\'on inicial $x\in\textrm{X}$, sea
$Q_{k}^{x}\left(t\right)$ la longitud de la cola al tiempo $t$,
$T_{m,k}^{x}\left(t\right)$ el tiempo acumulado, al tiempo $t$,
que tarda el servidor $m$ en atender a los usuarios de la cola
$k$. Finalmente sea $T_{m,k}^{x,0}\left(t\right)$ el tiempo
acumulado, al tiempo $t$, que tarda el servidor $m$ en trasladarse
a otra cola a partir de la $k$-\'esima.\\

Sup\'ongase que la funci\'on
$\left(\overline{Q}\left(\cdot\right),\overline{T}_{m}
\left(\cdot\right),\overline{T}_{m}^{0} \left(\cdot\right)\right)$
para $m=1,2,\ldots,M$ es un punto l\'imite de
\begin{equation}\label{Eq.Punto.Limite}
\left(\frac{1}{|x|}Q^{x}\left(|x|t\right),\frac{1}{|x|}T_{m}^{x}\left(|x|t\right),\frac{1}{|x|}T_{m}^{x,0}\left(|x|t\right)\right)
\end{equation}
para $m=1,2,\ldots,M$, cuando $x\rightarrow\infty$. Entonces
$\left(\overline{Q}\left(t\right),\overline{T}_{m}
\left(t\right),\overline{T}_{m}^{0} \left(t\right)\right)$ es un
flujo l\'imite del sistema. Al conjunto de todos las posibles
flujos l\'imite se le llama \textbf{Modelo de Flujo}.\\

El modelo de flujo satisface el siguiente conjunto de ecuaciones:

\begin{equation}\label{Eq.MF.1}
\overline{Q}_{k}\left(t\right)=\overline{Q}_{k}\left(0\right)+\lambda_{k}t-\sum_{m=1}^{M}\mu_{k}\overline{T}_{m,k}\left(t\right)\\
\end{equation}
para $k=1,2,\ldots,K$.\\
\begin{equation}\label{Eq.MF.2}
\overline{Q}_{k}\left(t\right)\geq0\textrm{ para
}k=1,2,\ldots,K,\\
\end{equation}

\begin{equation}\label{Eq.MF.3}
\overline{T}_{m,k}\left(0\right)=0,\textrm{ y }\overline{T}_{m,k}\left(\cdot\right)\textrm{ es no decreciente},\\
\end{equation}
para $k=1,2,\ldots,K$ y $m=1,2,\ldots,M$,\\
\begin{equation}\label{Eq.MF.4}
\sum_{k=1}^{K}\overline{T}_{m,k}^{0}\left(t\right)+\overline{T}_{m,k}\left(t\right)=t\textrm{
para }m=1,2,\ldots,M.\\
\end{equation}

De acuerdo a Dai \cite{Dai}, se tiene que el conjunto de posibles
l\'imites
$\left(\overline{Q}\left(\cdot\right),\overline{T}\left(\cdot\right),\overline{T}^{0}\left(\cdot\right)\right)$,
en el sentido de que deben de satisfacer las ecuaciones
(\ref{Eq.MF.1})-(\ref{Eq.MF.4}), se le llama {\em Modelo de
Flujo}.


\begin{Def}[Definici\'on 4.1, , Dai \cite{Dai}]\label{Def.Modelo.Flujo}
Sea una disciplina de servicio espec\'ifica. Cualquier l\'imite
$\left(\overline{Q}\left(\cdot\right),\overline{T}\left(\cdot\right)\right)$
en (\ref{Eq.Punto.Limite}) es un {\em flujo l\'imite} de la
disciplina. Cualquier soluci\'on (\ref{Eq.MF.1})-(\ref{Eq.MF.4})
es llamado flujo soluci\'on de la disciplina. Se dice que el
modelo de flujo l\'imite, modelo de flujo, de la disciplina de la
cola es estable si existe una constante $\delta>0$ que depende de
$\mu,\lambda$ y $P$ solamente, tal que cualquier flujo l\'imite
con
$|\overline{Q}\left(0\right)|+|\overline{U}|+|\overline{V}|=1$, se
tiene que $\overline{Q}\left(\cdot+\delta\right)\equiv0$.
\end{Def}

Al conjunto de ecuaciones dadas en \ref{Eq.MF.1}-\ref{Eq.MF.4} se
le llama {\em Modelo de flujo} y al conjunto de todas las
soluciones del modelo de flujo
$\left(\overline{Q}\left(\cdot\right),\overline{T}
\left(\cdot\right)\right)$ se le denotar\'a por $\mathcal{Q}$.

Si se hace $|x|\rightarrow\infty$ sin restringir ninguna de las
componentes, tambi\'en se obtienen un modelo de flujo, pero en
este caso el residual de los procesos de arribo y servicio
introducen un retraso:
\begin{Teo}[Teorema 4.2, Dai\cite{Dai}]\label{Tma.4.2.Dai}
Sea una disciplina fija para la cola, suponga que se cumplen las
condiciones (A1))-(A3)). Si el modelo de flujo l\'imite de la
disciplina de la cola es estable, entonces la cadena de Markov $X$
que describe la din\'amica de la red bajo la disciplina es Harris
recurrente positiva.
\end{Teo}

Ahora se procede a escalar el espacio y el tiempo para reducir la
aparente fluctuaci\'on del modelo. Consid\'erese el proceso
\begin{equation}\label{Eq.3.7}
\overline{Q}^{x}\left(t\right)=\frac{1}{|x|}Q^{x}\left(|x|t\right)
\end{equation}
A este proceso se le conoce como el fluido escalado, y cualquier
l\'imite $\overline{Q}^{x}\left(t\right)$ es llamado flujo
l\'imite del proceso de longitud de la cola. Haciendo
$|q|\rightarrow\infty$ mientras se mantiene el resto de las
componentes fijas, cualquier punto l\'imite del proceso de
longitud de la cola normalizado $\overline{Q}^{x}$ es soluci\'on
del siguiente modelo de flujo.


\begin{Def}[Definici\'on 3.3, Dai y Meyn \cite{DaiSean}]
El modelo de flujo es estable si existe un tiempo fijo $t_{0}$ tal
que $\overline{Q}\left(t\right)=0$, con $t\geq t_{0}$, para
cualquier $\overline{Q}\left(\cdot\right)\in\mathcal{Q}$ que
cumple con $|\overline{Q}\left(0\right)|=1$.
\end{Def}

El siguiente resultado se encuentra en Chen \cite{Chen}.
\begin{Lemma}[Lema 3.1, Dai y Meyn \cite{DaiSean}]
Si el modelo de flujo definido por \ref{Eq.MF.1}-\ref{Eq.MF.4} es
estable, entonces el modelo de flujo retrasado es tambi\'en
estable, es decir, existe $t_{0}>0$ tal que
$\overline{Q}\left(t\right)=0$ para cualquier $t\geq t_{0}$, para
cualquier soluci\'on del modelo de flujo retrasado cuya
condici\'on inicial $\overline{x}$ satisface que
$|\overline{x}|=|\overline{Q}\left(0\right)|+|\overline{A}\left(0\right)|+|\overline{B}\left(0\right)|\leq1$.
\end{Lemma}


Ahora ya estamos en condiciones de enunciar los resultados principales:


\begin{Teo}[Teorema 2.1, Down \cite{Down}]\label{Tma2.1.Down}
Suponga que el modelo de flujo es estable, y que se cumplen los supuestos (A1) y (A2), entonces
\begin{itemize}
\item[i)] Para alguna constante $\kappa_{p}$, y para cada
condici\'on inicial $x\in X$
\begin{equation}\label{Estability.Eq1}
limsup_{t\rightarrow\infty}\frac{1}{t}\int_{0}^{t}\esp_{x}\left[|Q\left(s\right)|^{p}\right]ds\leq\kappa_{p},
\end{equation}
donde $p$ es el entero dado en (A2).
\end{itemize}
Si adem\'as se cumple la condici\'on (A3), entonces para cada
condici\'on inicial:
\begin{itemize}
\item[ii)] Los momentos transitorios convergen a su estado
estacionario:
 \begin{equation}\label{Estability.Eq2}
lim_{t\rightarrow\infty}\esp_{x}\left[Q_{k}\left(t\right)^{r}\right]=\esp_{\pi}\left[Q_{k}\left(0\right)^{r}\right]\leq\kappa_{r},
\end{equation}
para $r=1,2,\ldots,p$ y $k=1,2,\ldots,K$. Donde $\pi$ es la
probabilidad invariante para $\mathbf{X}$.

\item[iii)]  El primer momento converge con raz\'on $t^{p-1}$:
\begin{equation}\label{Estability.Eq3}
lim_{t\rightarrow\infty}t^{p-1}|\esp_{x}\left[Q_{k}\left(t\right)\right]-\esp_{\pi}\left[Q_{k}\left(0\right)\right]=0.
\end{equation}

\item[iv)] La {\em Ley Fuerte de los grandes n\'umeros} se cumple:
\begin{equation}\label{Estability.Eq4}
lim_{t\rightarrow\infty}\frac{1}{t}\int_{0}^{t}Q_{k}^{r}\left(s\right)ds=\esp_{\pi}\left[Q_{k}\left(0\right)^{r}\right],\textrm{
}\prob_{x}\textrm{-c.s.}
\end{equation}
para $r=1,2,\ldots,p$ y $k=1,2,\ldots,K$.
\end{itemize}
\end{Teo}

La contribuci\'on de Down a la teor\'ia de los Sistemas de Visitas
C\'iclicas, es la relaci\'on que hay entre la estabilidad del
sistema con el comportamiento de las medidas de desempe\~no, es
decir, la condici\'on suficiente para poder garantizar la
convergencia del proceso de la longitud de la cola as\'i como de
por los menos los dos primeros momentos adem\'as de una versi\'on
de la Ley Fuerte de los Grandes N\'umeros para los sistemas de
visitas.


\begin{Teo}[Teorema 2.3, Down \cite{Down}]\label{Tma2.3.Down}
Considere el siguiente valor:
\begin{equation}\label{Eq.Rho.1serv}
\rho=\sum_{k=1}^{K}\rho_{k}+max_{1\leq j\leq K}\left(\frac{\lambda_{j}}{\sum_{s=1}^{S}p_{js}\overline{N}_{s}}\right)\delta^{*}
\end{equation}
\begin{itemize}
\item[i)] Si $\rho<1$ entonces la red es estable, es decir, se cumple el teorema \ref{Tma2.1.Down}.

\item[ii)] Si $\rho<1$ entonces la red es inestable, es decir, se cumple el teorema \ref{Tma2.2.Down}
\end{itemize}
\end{Teo}

\begin{Teo}
Sea $\left(X_{n},\mathcal{F}_{n},n=0,1,\ldots,\right\}$ Proceso de
Markov con espacio de estados $\left(S_{0},\chi_{0}\right)$
generado por una distribuici\'on inicial $P_{o}$ y probabilidad de
transici\'on $p_{mn}$, para $m,n=0,1,\ldots,$ $m<n$, que por
notaci\'on se escribir\'a como $p\left(m,n,x,B\right)\rightarrow
p_{mn}\left(x,B\right)$. Sea $S$ tiempo de paro relativo a la
$\sigma$-\'algebra $\mathcal{F}_{n}$. Sea $T$ funci\'on medible,
$T:\Omega\rightarrow\left\{0,1,\ldots,\right\}$. Sup\'ongase que
$T\geq S$, entonces $T$ es tiempo de paro. Si $B\in\chi_{0}$,
entonces
\begin{equation}\label{Prop.Fuerte.Markov}
P\left\{X\left(T\right)\in
B,T<\infty|\mathcal{F}\left(S\right)\right\} =
p\left(S,T,X\left(s\right),B\right)
\end{equation}
en $\left\{T<\infty\right\}$.
\end{Teo}


Sea $K$ conjunto numerable y sea $d:K\rightarrow\nat$ funci\'on.
Para $v\in K$, $M_{v}$ es un conjunto abierto de
$\rea^{d\left(v\right)}$. Entonces \[E=\cup_{v\in
K}M_{v}=\left\{\left(v,\zeta\right):v\in K,\zeta\in
M_{v}\right\}.\]

Sea $\mathcal{E}$ la clase de conjuntos medibles en $E$:
\[\mathcal{E}=\left\{\cup_{v\in K}A_{v}:A_{v}\in \mathcal{M}_{v}\right\}.\]

donde $\mathcal{M}$ son los conjuntos de Borel de $M_{v}$.
Entonces $\left(E,\mathcal{E}\right)$ es un espacio de Borel. El
estado del proceso se denotar\'a por
$\mathbf{x}_{t}=\left(v_{t},\zeta_{t}\right)$. La distribuci\'on
de $\left(\mathbf{x}_{t}\right)$ est\'a determinada por por los
siguientes objetos:

\begin{itemize}
\item[i)] Los campos vectoriales $\left(\mathcal{H}_{v},v\in
K\right)$. \item[ii)] Una funci\'on medible $\lambda:E\rightarrow
\rea_{+}$. \item[iii)] Una medida de transici\'on
$Q:\mathcal{E}\times\left(E\cup\Gamma^{*}\right)\rightarrow\left[0,1\right]$
donde
\begin{equation}
\Gamma^{*}=\cup_{v\in K}\partial^{*}M_{v}.
\end{equation}
y
\begin{equation}
\partial^{*}M_{v}=\left\{z\in\partial M_{v}:\mathbf{\mathbf{\phi}_{v}\left(t,\zeta\right)=\mathbf{z}}\textrm{ para alguna }\left(t,\zeta\right)\in\rea_{+}\times M_{v}\right\}.
\end{equation}
$\partial M_{v}$ denota  la frontera de $M_{v}$.
\end{itemize}

El campo vectorial $\left(\mathcal{H}_{v},v\in K\right)$ se supone
tal que para cada $\mathbf{z}\in M_{v}$ existe una \'unica curva
integral $\mathbf{\phi}_{v}\left(t,\zeta\right)$ que satisface la
ecuaci\'on

\begin{equation}
\frac{d}{dt}f\left(\zeta_{t}\right)=\mathcal{H}f\left(\zeta_{t}\right),
\end{equation}
con $\zeta_{0}=\mathbf{z}$, para cualquier funci\'on suave
$f:\rea^{d}\rightarrow\rea$ y $\mathcal{H}$ denota el operador
diferencial de primer orden, con $\mathcal{H}=\mathcal{H}_{v}$ y
$\zeta_{t}=\mathbf{\phi}\left(t,\mathbf{z}\right)$. Adem\'as se
supone que $\mathcal{H}_{v}$ es conservativo, es decir, las curvas
integrales est\'an definidas para todo $t>0$.

Para $\mathbf{x}=\left(v,\zeta\right)\in E$ se denota
\[t^{*}\mathbf{x}=inf\left\{t>0:\mathbf{\phi}_{v}\left(t,\zeta\right)\in\partial^{*}M_{v}\right\}\]

En lo que respecta a la funci\'on $\lambda$, se supondr\'a que
para cada $\left(v,\zeta\right)\in E$ existe un $\epsilon>0$ tal
que la funci\'on
$s\rightarrow\lambda\left(v,\phi_{v}\left(s,\zeta\right)\right)\in
E$ es integrable para $s\in\left[0,\epsilon\right)$. La medida de
transici\'on $Q\left(A;\mathbf{x}\right)$ es una funci\'on medible
de $\mathbf{x}$ para cada $A\in\mathcal{E}$, definida para
$\mathbf{x}\in E\cup\Gamma^{*}$ y es una medida de probabilidad en
$\left(E,\mathcal{E}\right)$ para cada $\mathbf{x}\in E$.

El movimiento del proceso $\left(\mathbf{x}_{t}\right)$ comenzando
en $\mathbf{x}=\left(n,\mathbf{z}\right)\in E$ se puede construir
de la siguiente manera, def\'inase la funci\'on $F$ por

\begin{equation}
F\left(t\right)=\left\{\begin{array}{ll}\\
exp\left(-\int_{0}^{t}\lambda\left(n,\phi_{n}\left(s,\mathbf{z}\right)\right)ds\right), & t<t^{*}\left(\mathbf{x}\right),\\
0, & t\geq t^{*}\left(\mathbf{x}\right)
\end{array}\right.
\end{equation}

Sea $T_{1}$ una variable aleatoria tal que
$\prob\left[T_{1}>t\right]=F\left(t\right)$, ahora sea la variable
aleatoria $\left(N,Z\right)$ con distribuici\'on
$Q\left(\cdot;\phi_{n}\left(T_{1},\mathbf{z}\right)\right)$. La
trayectoria de $\left(\mathbf{x}_{t}\right)$ para $t\leq T_{1}$
es\footnote{Revisar p\'agina 362, y 364 de Davis \cite{Davis}.}
\begin{eqnarray*}
\mathbf{x}_{t}=\left(v_{t},\zeta_{t}\right)=\left\{\begin{array}{ll}
\left(n,\phi_{n}\left(t,\mathbf{z}\right)\right), & t<T_{1},\\
\left(N,\mathbf{Z}\right), & t=t_{1}.
\end{array}\right.
\end{eqnarray*}

Comenzando en $\mathbf{x}_{T_{1}}$ se selecciona el siguiente
tiempo de intersalto $T_{2}-T_{1}$ lugar del post-salto
$\mathbf{x}_{T_{2}}$ de manera similar y as\'i sucesivamente. Este
procedimiento nos da una trayectoria determinista por partes
$\mathbf{x}_{t}$ con tiempos de salto $T_{1},T_{2},\ldots$. Bajo
las condiciones enunciadas para $\lambda,T_{1}>0$  y
$T_{1}-T_{2}>0$ para cada $i$, con probabilidad 1. Se supone que
se cumple la siguiente condici\'on.

\begin{Sup}[Supuesto 3.1, Davis \cite{Davis}]\label{Sup3.1.Davis}
Sea $N_{t}:=\sum_{t}\indora_{\left(t\geq t\right)}$ el n\'umero de
saltos en $\left[0,t\right]$. Entonces
\begin{equation}
\esp\left[N_{t}\right]<\infty\textrm{ para toda }t.
\end{equation}
\end{Sup}

es un proceso de Markov, m\'as a\'un, es un Proceso Fuerte de
Markov, es decir, la Propiedad Fuerte de Markov se cumple para
cualquier tiempo de paro.


Sea $E$ es un espacio m\'etrico separable y la m\'etrica $d$ es
compatible con la topolog\'ia.


\begin{Def}
Un espacio topol\'ogico $E$ es llamado de {\em Rad\'on} si es
homeomorfo a un subconjunto universalmente medible de un espacio
m\'etrico compacto.
\end{Def}

Equivalentemente, la definici\'on de un espacio de Rad\'on puede
encontrarse en los siguientes t\'erminos:


\begin{Def}
$E$ es un espacio de Rad\'on si cada medida finita en
$\left(E,\mathcal{B}\left(E\right)\right)$ es regular interior o
cerrada, {\em tight}.
\end{Def}

\begin{Def}
Una medida finita, $\lambda$ en la $\sigma$-\'algebra de Borel de
un espacio metrizable $E$ se dice cerrada si
\begin{equation}\label{Eq.A2.3}
\lambda\left(E\right)=sup\left\{\lambda\left(K\right):K\textrm{ es
compacto en }E\right\}.
\end{equation}
\end{Def}

El siguiente teorema nos permite tener una mejor caracterizaci\'on
de los espacios de Rad\'on:
\begin{Teo}\label{Tma.A2.2}
Sea $E$ espacio separable metrizable. Entonces $E$ es Radoniano si
y s\'olo s\'i cada medida finita en
$\left(E,\mathcal{B}\left(E\right)\right)$ es cerrada.
\end{Teo}

Sea $E$ espacio de estados, tal que $E$ es un espacio de Rad\'on,
$\mathcal{B}\left(E\right)$ $\sigma$-\'algebra de Borel en $E$,
que se denotar\'a por $\mathcal{E}$.

Sea $\left(X,\mathcal{G},\prob\right)$ espacio de probabilidad,
$I\subset\rea$ conjunto de \'indices. Sea $\mathcal{F}_{\leq t}$
la $\sigma$-\'algebra natural definida como
$\sigma\left\{f\left(X_{r}\right):r\in I, r\leq
t,f\in\mathcal{E}\right\}$. Se considerar\'a una
$\sigma$-\'algebra m\'as general, $ \left(\mathcal{G}_{t}\right)$
tal que $\left(X_{t}\right)$ sea $\mathcal{E}$-adaptado.

\begin{Def}
Una familia $\left(P_{s,t}\right)$ de kernels de Markov en
$\left(E,\mathcal{E}\right)$ indexada por pares $s,t\in I$, con
$s\leq t$ es una funci\'on de transici\'on en $\ER$, si  para todo
$r\leq s< t$ en $I$ y todo $x\in E$,
$B\in\mathcal{E}$\footnote{Ecuaci\'on de Chapman-Kolmogorov}
\begin{equation}\label{Eq.Kernels}
P_{r,t}\left(x,B\right)=\int_{E}P_{r,s}\left(x,dy\right)P_{s,t}\left(y,B\right).
\end{equation}
\end{Def}

Se dice que la funci\'on de transici\'on $\KM$ en $\ER$ es la
funci\'on de transici\'on para un proceso $\PE$  con valores en
$E$ y que satisface la propiedad de
Markov\footnote{\begin{equation}\label{Eq.1.4.S}
\prob\left\{H|\mathcal{G}_{t}\right\}=\prob\left\{H|X_{t}\right\}\textrm{
}H\in p\mathcal{F}_{\geq t}.
\end{equation}} (\ref{Eq.1.4.S}) relativa a $\left(\mathcal{G}_{t}\right)$ si

\begin{equation}\label{Eq.1.6.S}
\prob\left\{f\left(X_{t}\right)|\mathcal{G}_{s}\right\}=P_{s,t}f\left(X_{t}\right)\textrm{
}s\leq t\in I,\textrm{ }f\in b\mathcal{E}.
\end{equation}

\begin{Def}
Una familia $\left(P_{t}\right)_{t\geq0}$ de kernels de Markov en
$\ER$ es llamada {\em Semigrupo de Transici\'on de Markov} o {\em
Semigrupo de Transici\'on} si
\[P_{t+s}f\left(x\right)=P_{t}\left(P_{s}f\right)\left(x\right),\textrm{ }t,s\geq0,\textrm{ }x\in E\textrm{ }f\in b\mathcal{E}.\]
\end{Def}
\begin{Note}
Si la funci\'on de transici\'on $\KM$ es llamada homog\'enea si
$P_{s,t}=P_{t-s}$.
\end{Note}

Un proceso de Markov que satisface la ecuaci\'on (\ref{Eq.1.6.S})
con funci\'on de transici\'on homog\'enea $\left(P_{t}\right)$
tiene la propiedad caracter\'istica
\begin{equation}\label{Eq.1.8.S}
\prob\left\{f\left(X_{t+s}\right)|\mathcal{G}_{t}\right\}=P_{s}f\left(X_{t}\right)\textrm{
}t,s\geq0,\textrm{ }f\in b\mathcal{E}.
\end{equation}
La ecuaci\'on anterior es la {\em Propiedad Simple de Markov} de
$X$ relativa a $\left(P_{t}\right)$.

En este sentido el proceso $\PE$ cumple con la propiedad de Markov
(\ref{Eq.1.8.S}) relativa a
$\left(\Omega,\mathcal{G},\mathcal{G}_{t},\prob\right)$ con
semigrupo de transici\'on $\left(P_{t}\right)$.

\begin{Def}
Un proceso estoc\'astico $\PE$ definido en
$\left(\Omega,\mathcal{G},\prob\right)$ con valores en el espacio
topol\'ogico $E$ es continuo por la derecha si cada trayectoria
muestral $t\rightarrow X_{t}\left(w\right)$ es un mapeo continuo
por la derecha de $I$ en $E$.
\end{Def}

\begin{Def}[HD1]\label{Eq.2.1.S}
Un semigrupo de Markov $\left(P_{t}\right)$ en un espacio de
Rad\'on $E$ se dice que satisface la condici\'on {\em HD1} si,
dada una medida de probabilidad $\mu$ en $E$, existe una
$\sigma$-\'algebra $\mathcal{E^{*}}$ con
$\mathcal{E}\subset\mathcal{E}^{*}$ y
$P_{t}\left(b\mathcal{E}^{*}\right)\subset b\mathcal{E}^{*}$, y un
$\mathcal{E}^{*}$-proceso $E$-valuado continuo por la derecha
$\PE$ en alg\'un espacio de probabilidad filtrado
$\left(\Omega,\mathcal{G},\mathcal{G}_{t},\prob\right)$ tal que
$X=\left(\Omega,\mathcal{G},\mathcal{G}_{t},\prob\right)$ es de
Markov (Homog\'eneo) con semigrupo de transici\'on $(P_{t})$ y
distribuci\'on inicial $\mu$.
\end{Def}

Consid\'erese la colecci\'on de variables aleatorias $X_{t}$
definidas en alg\'un espacio de probabilidad, y una colecci\'on de
medidas $\mathbf{P}^{x}$ tales que
$\mathbf{P}^{x}\left\{X_{0}=x\right\}$, y bajo cualquier
$\mathbf{P}^{x}$, $X_{t}$ es de Markov con semigrupo
$\left(P_{t}\right)$. $\mathbf{P}^{x}$ puede considerarse como la
distribuci\'on condicional de $\mathbf{P}$ dado $X_{0}=x$.

\begin{Def}\label{Def.2.2.S}
Sea $E$ espacio de Rad\'on, $\SG$ semigrupo de Markov en $\ER$. La
colecci\'on
$\mathbf{X}=\left(\Omega,\mathcal{G},\mathcal{G}_{t},X_{t},\theta_{t},\CM\right)$
es un proceso $\mathcal{E}$-Markov continuo por la derecha simple,
con espacio de estados $E$ y semigrupo de transici\'on $\SG$ en
caso de que $\mathbf{X}$ satisfaga las siguientes condiciones:
\begin{itemize}
\item[i)] $\left(\Omega,\mathcal{G},\mathcal{G}_{t}\right)$ es un
espacio de medida filtrado, y $X_{t}$ es un proceso $E$-valuado
continuo por la derecha $\mathcal{E}^{*}$-adaptado a
$\left(\mathcal{G}_{t}\right)$;

\item[ii)] $\left(\theta_{t}\right)_{t\geq0}$ es una colecci\'on
de operadores {\em shift} para $X$, es decir, mapea $\Omega$ en
s\'i mismo satisfaciendo para $t,s\geq0$,

\begin{equation}\label{Eq.Shift}
\theta_{t}\circ\theta_{s}=\theta_{t+s}\textrm{ y
}X_{t}\circ\theta_{t}=X_{t+s};
\end{equation}

\item[iii)] Para cualquier $x\in E$,$\CM\left\{X_{0}=x\right\}=1$,
y el proceso $\PE$ tiene la propiedad de Markov (\ref{Eq.1.8.S})
con semigrupo de transici\'on $\SG$ relativo a
$\left(\Omega,\mathcal{G},\mathcal{G}_{t},\CM\right)$.
\end{itemize}
\end{Def}

\begin{Def}[HD2]\label{Eq.2.2.S}
Para cualquier $\alpha>0$ y cualquier $f\in S^{\alpha}$, el
proceso $t\rightarrow f\left(X_{t}\right)$ es continuo por la
derecha casi seguramente.
\end{Def}

\begin{Def}\label{Def.PD}
Un sistema
$\mathbf{X}=\left(\Omega,\mathcal{G},\mathcal{G}_{t},X_{t},\theta_{t},\CM\right)$
es un proceso derecho en el espacio de Rad\'on $E$ con semigrupo
de transici\'on $\SG$ provisto de:
\begin{itemize}
\item[i)] $\mathbf{X}$ es una realizaci\'on  continua por la
derecha, \ref{Def.2.2.S}, de $\SG$.

\item[ii)] $\mathbf{X}$ satisface la condicion HD2,
\ref{Eq.2.2.S}, relativa a $\mathcal{G}_{t}$.

\item[iii)] $\mathcal{G}_{t}$ es aumentado y continuo por la
derecha.
\end{itemize}
\end{Def}

\begin{Lema}[Lema 4.2, Dai\cite{Dai}]\label{Lema4.2}
Sea $\left\{x_{n}\right\}\subset \mathbf{X}$ con
$|x_{n}|\rightarrow\infty$, conforme $n\rightarrow\infty$. Suponga
que
\[lim_{n\rightarrow\infty}\frac{1}{|x_{n}|}U\left(0\right)=\overline{U}\]
y
\[lim_{n\rightarrow\infty}\frac{1}{|x_{n}|}V\left(0\right)=\overline{V}.\]

Entonces, conforme $n\rightarrow\infty$, casi seguramente

\begin{equation}\label{E1.4.2}
\frac{1}{|x_{n}|}\Phi^{k}\left(\left[|x_{n}|t\right]\right)\rightarrow
P_{k}^{'}t\textrm{, u.o.c.,}
\end{equation}

\begin{equation}\label{E1.4.3}
\frac{1}{|x_{n}|}E^{x_{n}}_{k}\left(|x_{n}|t\right)\rightarrow
\alpha_{k}\left(t-\overline{U}_{k}\right)^{+}\textrm{, u.o.c.,}
\end{equation}

\begin{equation}\label{E1.4.4}
\frac{1}{|x_{n}|}S^{x_{n}}_{k}\left(|x_{n}|t\right)\rightarrow
\mu_{k}\left(t-\overline{V}_{k}\right)^{+}\textrm{, u.o.c.,}
\end{equation}

donde $\left[t\right]$ es la parte entera de $t$ y
$\mu_{k}=1/m_{k}=1/\esp\left[\eta_{k}\left(1\right)\right]$.
\end{Lema}

\begin{Lema}[Lema 4.3, Dai\cite{Dai}]\label{Lema.4.3}
Sea $\left\{x_{n}\right\}\subset \mathbf{X}$ con
$|x_{n}|\rightarrow\infty$, conforme $n\rightarrow\infty$. Suponga
que
\[lim_{n\rightarrow\infty}\frac{1}{|x_{n}|}U\left(0\right)=\overline{U}_{k}\]
y
\[lim_{n\rightarrow\infty}\frac{1}{|x_{n}|}V\left(0\right)=\overline{V}_{k}.\]
\begin{itemize}
\item[a)] Conforme $n\rightarrow\infty$ casi seguramente,
\[lim_{n\rightarrow\infty}\frac{1}{|x_{n}|}U^{x_{n}}_{k}\left(|x_{n}|t\right)=\left(\overline{U}_{k}-t\right)^{+}\textrm{, u.o.c.}\]
y
\[lim_{n\rightarrow\infty}\frac{1}{|x_{n}|}V^{x_{n}}_{k}\left(|x_{n}|t\right)=\left(\overline{V}_{k}-t\right)^{+}.\]

\item[b)] Para cada $t\geq0$ fijo,
\[\left\{\frac{1}{|x_{n}|}U^{x_{n}}_{k}\left(|x_{n}|t\right),|x_{n}|\geq1\right\}\]
y
\[\left\{\frac{1}{|x_{n}|}V^{x_{n}}_{k}\left(|x_{n}|t\right),|x_{n}|\geq1\right\}\]
\end{itemize}
son uniformemente convergentes.
\end{Lema}

$S_{l}^{x}\left(t\right)$ es el n\'umero total de servicios
completados de la clase $l$, si la clase $l$ est\'a dando $t$
unidades de tiempo de servicio. Sea $T_{l}^{x}\left(x\right)$ el
monto acumulado del tiempo de servicio que el servidor
$s\left(l\right)$ gasta en los usuarios de la clase $l$ al tiempo
$t$. Entonces $S_{l}^{x}\left(T_{l}^{x}\left(t\right)\right)$ es
el n\'umero total de servicios completados para la clase $l$ al
tiempo $t$. Una fracci\'on de estos usuarios,
$\Phi_{l}^{x}\left(S_{l}^{x}\left(T_{l}^{x}\left(t\right)\right)\right)$,
se convierte en usuarios de la clase $k$.\\

Entonces, dado lo anterior, se tiene la siguiente representaci\'on
para el proceso de la longitud de la cola:\\

\begin{equation}
Q_{k}^{x}\left(t\right)=_{k}^{x}\left(0\right)+E_{k}^{x}\left(t\right)+\sum_{l=1}^{K}\Phi_{k}^{l}\left(S_{l}^{x}\left(T_{l}^{x}\left(t\right)\right)\right)-S_{k}^{x}\left(T_{k}^{x}\left(t\right)\right)
\end{equation}
para $k=1,\ldots,K$. Para $i=1,\ldots,d$, sea
\[I_{i}^{x}\left(t\right)=t-\sum_{j\in C_{i}}T_{k}^{x}\left(t\right).\]

Entonces $I_{i}^{x}\left(t\right)$ es el monto acumulado del
tiempo que el servidor $i$ ha estado desocupado al tiempo $t$. Se
est\'a asumiendo que las disciplinas satisfacen la ley de
conservaci\'on del trabajo, es decir, el servidor $i$ est\'a en
pausa solamente cuando no hay usuarios en la estaci\'on $i$.
Entonces, se tiene que

\begin{equation}
\int_{0}^{\infty}\left(\sum_{k\in
C_{i}}Q_{k}^{x}\left(t\right)\right)dI_{i}^{x}\left(t\right)=0,
\end{equation}
para $i=1,\ldots,d$.\\

Hacer
\[T^{x}\left(t\right)=\left(T_{1}^{x}\left(t\right),\ldots,T_{K}^{x}\left(t\right)\right)^{'},\]
\[I^{x}\left(t\right)=\left(I_{1}^{x}\left(t\right),\ldots,I_{K}^{x}\left(t\right)\right)^{'}\]
y
\[S^{x}\left(T^{x}\left(t\right)\right)=\left(S_{1}^{x}\left(T_{1}^{x}\left(t\right)\right),\ldots,S_{K}^{x}\left(T_{K}^{x}\left(t\right)\right)\right)^{'}.\]

Para una disciplina que cumple con la ley de conservaci\'on del
trabajo, en forma vectorial, se tiene el siguiente conjunto de
ecuaciones

\begin{equation}\label{Eq.MF.1.3}
Q^{x}\left(t\right)=Q^{x}\left(0\right)+E^{x}\left(t\right)+\sum_{l=1}^{K}\Phi^{l}\left(S_{l}^{x}\left(T_{l}^{x}\left(t\right)\right)\right)-S^{x}\left(T^{x}\left(t\right)\right),\\
\end{equation}

\begin{equation}\label{Eq.MF.2.3}
Q^{x}\left(t\right)\geq0,\\
\end{equation}

\begin{equation}\label{Eq.MF.3.3}
T^{x}\left(0\right)=0,\textrm{ y }\overline{T}^{x}\left(t\right)\textrm{ es no decreciente},\\
\end{equation}

\begin{equation}\label{Eq.MF.4.3}
I^{x}\left(t\right)=et-CT^{x}\left(t\right)\textrm{ es no
decreciente}\\
\end{equation}

\begin{equation}\label{Eq.MF.5.3}
\int_{0}^{\infty}\left(CQ^{x}\left(t\right)\right)dI_{i}^{x}\left(t\right)=0,\\
\end{equation}

\begin{equation}\label{Eq.MF.6.3}
\textrm{Condiciones adicionales en
}\left(\overline{Q}^{x}\left(\cdot\right),\overline{T}^{x}\left(\cdot\right)\right)\textrm{
espec\'ificas de la disciplina de la cola,}
\end{equation}

donde $e$ es un vector de unos de dimensi\'on $d$, $C$ es la
matriz definida por
\[C_{ik}=\left\{\begin{array}{cc}
1,& S\left(k\right)=i,\\
0,& \textrm{ en otro caso}.\\
\end{array}\right.
\]
Es necesario enunciar el siguiente Teorema que se utilizar\'a para
el Teorema \ref{Tma.4.2.Dai}:
\begin{Teo}[Teorema 4.1, Dai \cite{Dai}]
Considere una disciplina que cumpla la ley de conservaci\'on del
trabajo, para casi todas las trayectorias muestrales $\omega$ y
cualquier sucesi\'on de estados iniciales
$\left\{x_{n}\right\}\subset \mathbf{X}$, con
$|x_{n}|\rightarrow\infty$, existe una subsucesi\'on
$\left\{x_{n_{j}}\right\}$ con $|x_{n_{j}}|\rightarrow\infty$ tal
que
\begin{equation}\label{Eq.4.15}
\frac{1}{|x_{n_{j}}|}\left(Q^{x_{n_{j}}}\left(0\right),U^{x_{n_{j}}}\left(0\right),V^{x_{n_{j}}}\left(0\right)\right)\rightarrow\left(\overline{Q}\left(0\right),\overline{U},\overline{V}\right),
\end{equation}

\begin{equation}\label{Eq.4.16}
\frac{1}{|x_{n_{j}}|}\left(Q^{x_{n_{j}}}\left(|x_{n_{j}}|t\right),T^{x_{n_{j}}}\left(|x_{n_{j}}|t\right)\right)\rightarrow\left(\overline{Q}\left(t\right),\overline{T}\left(t\right)\right)\textrm{
u.o.c.}
\end{equation}

Adem\'as,
$\left(\overline{Q}\left(t\right),\overline{T}\left(t\right)\right)$
satisface las siguientes ecuaciones:
\begin{equation}\label{Eq.MF.1.3a}
\overline{Q}\left(t\right)=Q\left(0\right)+\left(\alpha
t-\overline{U}\right)^{+}-\left(I-P\right)^{'}M^{-1}\left(\overline{T}\left(t\right)-\overline{V}\right)^{+},
\end{equation}

\begin{equation}\label{Eq.MF.2.3a}
\overline{Q}\left(t\right)\geq0,\\
\end{equation}

\begin{equation}\label{Eq.MF.3.3a}
\overline{T}\left(t\right)\textrm{ es no decreciente y comienza en cero},\\
\end{equation}

\begin{equation}\label{Eq.MF.4.3a}
\overline{I}\left(t\right)=et-C\overline{T}\left(t\right)\textrm{
es no decreciente,}\\
\end{equation}

\begin{equation}\label{Eq.MF.5.3a}
\int_{0}^{\infty}\left(C\overline{Q}\left(t\right)\right)d\overline{I}\left(t\right)=0,\\
\end{equation}

\begin{equation}\label{Eq.MF.6.3a}
\textrm{Condiciones adicionales en
}\left(\overline{Q}\left(\cdot\right),\overline{T}\left(\cdot\right)\right)\textrm{
especficas de la disciplina de la cola,}
\end{equation}
\end{Teo}


Propiedades importantes para el modelo de flujo retrasado:

\begin{Prop}
 Sea $\left(\overline{Q},\overline{T},\overline{T}^{0}\right)$ un flujo l\'imite de \ref{Eq.4.4} y suponga que cuando $x\rightarrow\infty$ a lo largo de
una subsucesi\'on
\[\left(\frac{1}{|x|}Q_{k}^{x}\left(0\right),\frac{1}{|x|}A_{k}^{x}\left(0\right),\frac{1}{|x|}B_{k}^{x}\left(0\right),\frac{1}{|x|}B_{k}^{x,0}\left(0\right)\right)\rightarrow\left(\overline{Q}_{k}\left(0\right),0,0,0\right)\]
para $k=1,\ldots,K$. EL flujo l\'imite tiene las siguientes
propiedades, donde las propiedades de la derivada se cumplen donde
la derivada exista:
\begin{itemize}
 \item[i)] Los vectores de tiempo ocupado $\overline{T}\left(t\right)$ y $\overline{T}^{0}\left(t\right)$ son crecientes y continuas con
$\overline{T}\left(0\right)=\overline{T}^{0}\left(0\right)=0$.
\item[ii)] Para todo $t\geq0$
\[\sum_{k=1}^{K}\left[\overline{T}_{k}\left(t\right)+\overline{T}_{k}^{0}\left(t\right)\right]=t\]
\item[iii)] Para todo $1\leq k\leq K$
\[\overline{Q}_{k}\left(t\right)=\overline{Q}_{k}\left(0\right)+\alpha_{k}t-\mu_{k}\overline{T}_{k}\left(t\right)\]
\item[iv)]  Para todo $1\leq k\leq K$
\[\dot{{\overline{T}}}_{k}\left(t\right)=\beta_{k}\] para $\overline{Q}_{k}\left(t\right)=0$.
\item[v)] Para todo $k,j$
\[\mu_{k}^{0}\overline{T}_{k}^{0}\left(t\right)=\mu_{j}^{0}\overline{T}_{j}^{0}\left(t\right)\]
\item[vi)]  Para todo $1\leq k\leq K$
\[\mu_{k}\dot{{\overline{T}}}_{k}\left(t\right)=l_{k}\mu_{k}^{0}\dot{{\overline{T}}}_{k}^{0}\left(t\right)\] para $\overline{Q}_{k}\left(t\right)>0$.
\end{itemize}
\end{Prop}

\begin{Lema}[Lema 3.1 \cite{Chen}]\label{Lema3.1}
Si el modelo de flujo es estable, definido por las ecuaciones
(3.8)-(3.13), entonces el modelo de flujo retrasado tambi\'en es
estable.
\end{Lema}

\begin{Teo}[Teorema 5.1 \cite{Chen}]\label{Tma.5.1.Chen}
La red de colas es estable si existe una constante $t_{0}$ que
depende de $\left(\alpha,\mu,T,U\right)$ y $V$ que satisfagan las
ecuaciones (5.1)-(5.5), $Z\left(t\right)=0$, para toda $t\geq
t_{0}$.
\end{Teo}



\begin{Lema}[Lema 5.2 \cite{Gut}]\label{Lema.5.2.Gut}
Sea $\left\{\xi\left(k\right):k\in\ent\right\}$ sucesi\'on de
variables aleatorias i.i.d. con valores en
$\left(0,\infty\right)$, y sea $E\left(t\right)$ el proceso de
conteo
\[E\left(t\right)=max\left\{n\geq1:\xi\left(1\right)+\cdots+\xi\left(n-1\right)\leq t\right\}.\]
Si $E\left[\xi\left(1\right)\right]<\infty$, entonces para
cualquier entero $r\geq1$
\begin{equation}
lim_{t\rightarrow\infty}\esp\left[\left(\frac{E\left(t\right)}{t}\right)^{r}\right]=\left(\frac{1}{E\left[\xi_{1}\right]}\right)^{r}
\end{equation}
de aqu\'i, bajo estas condiciones
\begin{itemize}
\item[a)] Para cualquier $t>0$,
$sup_{t\geq\delta}\esp\left[\left(\frac{E\left(t\right)}{t}\right)^{r}\right]$

\item[b)] Las variables aleatorias
$\left\{\left(\frac{E\left(t\right)}{t}\right)^{r}:t\geq1\right\}$
son uniformemente integrables.
\end{itemize}
\end{Lema}

\begin{Teo}[Teorema 5.1: Ley Fuerte para Procesos de Conteo
\cite{Gut}]\label{Tma.5.1.Gut} Sea
$0<\mu<\esp\left(X_{1}\right]\leq\infty$. entonces

\begin{itemize}
\item[a)] $\frac{N\left(t\right)}{t}\rightarrow\frac{1}{\mu}$
a.s., cuando $t\rightarrow\infty$.


\item[b)]$\esp\left[\frac{N\left(t\right)}{t}\right]^{r}\rightarrow\frac{1}{\mu^{r}}$,
cuando $t\rightarrow\infty$ para todo $r>0$..
\end{itemize}
\end{Teo}


\begin{Prop}[Proposici\'on 5.1 \cite{DaiSean}]\label{Prop.5.1}
Suponga que los supuestos (A1) y (A2) se cumplen, adem\'as suponga
que el modelo de flujo es estable. Entonces existe $t_{0}>0$ tal
que
\begin{equation}\label{Eq.Prop.5.1}
lim_{|x|\rightarrow\infty}\frac{1}{|x|^{p+1}}\esp_{x}\left[|X\left(t_{0}|x|\right)|^{p+1}\right]=0.
\end{equation}

\end{Prop}


\begin{Prop}[Proposici\'on 5.3 \cite{DaiSean}]
Sea $X$ proceso de estados para la red de colas, y suponga que se
cumplen los supuestos (A1) y (A2), entonces para alguna constante
positiva $C_{p+1}<\infty$, $\delta>0$ y un conjunto compacto
$C\subset X$.

\begin{equation}\label{Eq.5.4}
\esp_{x}\left[\int_{0}^{\tau_{C}\left(\delta\right)}\left(1+|X\left(t\right)|^{p}\right)dt\right]\leq
C_{p+1}\left(1+|x|^{p+1}\right)
\end{equation}
\end{Prop}

\begin{Prop}[Proposici\'on 5.4 \cite{DaiSean}]
Sea $X$ un proceso de Markov Borel Derecho en $X$, sea
$f:X\leftarrow\rea_{+}$ y defina para alguna $\delta>0$, y un
conjunto cerrado $C\subset X$
\[V\left(x\right):=\esp_{x}\left[\int_{0}^{\tau_{C}\left(\delta\right)}f\left(X\left(t\right)\right)dt\right]\]
para $x\in X$. Si $V$ es finito en todas partes y uniformemente
acotada en $C$, entonces existe $k<\infty$ tal que
\begin{equation}\label{Eq.5.11}
\frac{1}{t}\esp_{x}\left[V\left(x\right)\right]+\frac{1}{t}\int_{0}^{t}\esp_{x}\left[f\left(X\left(s\right)\right)ds\right]\leq\frac{1}{t}V\left(x\right)+k,
\end{equation}
para $x\in X$ y $t>0$.
\end{Prop}


\begin{Teo}[Teorema 5.5 \cite{DaiSean}]
Suponga que se cumplen (A1) y (A2), adem\'as suponga que el modelo
de flujo es estable. Entonces existe una constante $k_{p}<\infty$
tal que
\begin{equation}\label{Eq.5.13}
\frac{1}{t}\int_{0}^{t}\esp_{x}\left[|Q\left(s\right)|^{p}\right]ds\leq
k_{p}\left\{\frac{1}{t}|x|^{p+1}+1\right\}
\end{equation}
para $t\geq0$, $x\in X$. En particular para cada condici\'on
inicial
\begin{equation}\label{Eq.5.14}
Limsup_{t\rightarrow\infty}\frac{1}{t}\int_{0}^{t}\esp_{x}\left[|Q\left(s\right)|^{p}\right]ds\leq
k_{p}
\end{equation}
\end{Teo}

\begin{Teo}[Teorema 6.2 \cite{DaiSean}]\label{Tma.6.2}
Suponga que se cumplen los supuestos (A1)-(A3) y que el modelo de
flujo es estable, entonces se tiene que
\[\parallel P^{t}\left(c,\cdot\right)-\pi\left(\cdot\right)\parallel_{f_{p}}\rightarrow0\]
para $t\rightarrow\infty$ y $x\in X$. En particular para cada
condici\'on inicial
\[lim_{t\rightarrow\infty}\esp_{x}\left[\left|Q_{t}\right|^{p}\right]=\esp_{\pi}\left[\left|Q_{0}\right|^{p}\right]<\infty\]
\end{Teo}


\begin{Teo}[Teorema 6.3 \cite{DaiSean}]\label{Tma.6.3}
Suponga que se cumplen los supuestos (A1)-(A3) y que el modelo de
flujo es estable, entonces con
$f\left(x\right)=f_{1}\left(x\right)$, se tiene que
\[lim_{t\rightarrow\infty}t^{(p-1)\left|P^{t}\left(c,\cdot\right)-\pi\left(\cdot\right)\right|_{f}=0},\]
para $x\in X$. En particular, para cada condici\'on inicial
\[lim_{t\rightarrow\infty}t^{(p-1)}\left|\esp_{x}\left[Q_{t}\right]-\esp_{\pi}\left[Q_{0}\right]\right|=0.\]
\end{Teo}



\begin{Prop}[Proposici\'on 5.1, Dai y Meyn \cite{DaiSean}]\label{Prop.5.1.DaiSean}
Suponga que los supuestos A1) y A2) son ciertos y que el modelo de
flujo es estable. Entonces existe $t_{0}>0$ tal que
\begin{equation}
lim_{|x|\rightarrow\infty}\frac{1}{|x|^{p+1}}\esp_{x}\left[|X\left(t_{0}|x|\right)|^{p+1}\right]=0
\end{equation}
\end{Prop}

\begin{Lemma}[Lema 5.2, Dai y Meyn, \cite{DaiSean}]\label{Lema.5.2.DaiSean}
 Sea $\left\{\zeta\left(k\right):k\in \mathbb{z}\right\}$ una sucesi\'on independiente e id\'enticamente distribuida que toma valores en $\left(0,\infty\right)$,
y sea
$E\left(t\right)=max\left(n\geq1:\zeta\left(1\right)+\cdots+\zeta\left(n-1\right)\leq
t\right)$. Si $\esp\left[\zeta\left(1\right)\right]<\infty$,
entonces para cualquier entero $r\geq1$
\begin{equation}
 lim_{t\rightarrow\infty}\esp\left[\left(\frac{E\left(t\right)}{t}\right)^{r}\right]=\left(\frac{1}{\esp\left[\zeta_{1}\right]}\right)^{r}.
\end{equation}
Luego, bajo estas condiciones:
\begin{itemize}
 \item[a)] para cualquier $\delta>0$, $\sup_{t\geq\delta}\esp\left[\left(\frac{E\left(t\right)}{t}\right)^{r}\right]<\infty$
\item[b)] las variables aleatorias
$\left\{\left(\frac{E\left(t\right)}{t}\right)^{r}:t\geq1\right\}$
son uniformemente integrables.
\end{itemize}
\end{Lemma}

\begin{Teo}[Teorema 5.5, Dai y Meyn \cite{DaiSean}]\label{Tma.5.5.DaiSean}
Suponga que los supuestos A1) y A2) se cumplen y que el modelo de
flujo es estable. Entonces existe una constante $\kappa_{p}$ tal
que
\begin{equation}
\frac{1}{t}\int_{0}^{t}\esp_{x}\left[|Q\left(s\right)|^{p}\right]ds\leq\kappa_{p}\left\{\frac{1}{t}|x|^{p+1}+1\right\}
\end{equation}
para $t>0$ y $x\in X$. En particular, para cada condici\'on
inicial
\begin{eqnarray*}
\limsup_{t\rightarrow\infty}\frac{1}{t}\int_{0}^{t}\esp_{x}\left[|Q\left(s\right)|^{p}\right]ds\leq\kappa_{p}.
\end{eqnarray*}
\end{Teo}

\begin{Teo}[Teorema 6.2, Dai y Meyn \cite{DaiSean}]\label{Tma.6.2.DaiSean}
Suponga que se cumplen los supuestos A1), A2) y A3) y que el
modelo de flujo es estable. Entonces se tiene que
\begin{equation}
\left\|P^{t}\left(x,\cdot\right)-\pi\left(\cdot\right)\right\|_{f_{p}}\textrm{,
}t\rightarrow\infty,x\in X.
\end{equation}
En particular para cada condici\'on inicial
\begin{eqnarray*}
\lim_{t\rightarrow\infty}\esp_{x}\left[|Q\left(t\right)|^{p}\right]=\esp_{\pi}\left[|Q\left(0\right)|^{p}\right]\leq\kappa_{r}
\end{eqnarray*}
\end{Teo}
\begin{Teo}[Teorema 6.3, Dai y Meyn \cite{DaiSean}]\label{Tma.6.3.DaiSean}
Suponga que se cumplen los supuestos A1), A2) y A3) y que el
modelo de flujo es estable. Entonces con
$f\left(x\right)=f_{1}\left(x\right)$ se tiene
\begin{equation}
\lim_{t\rightarrow\infty}t^{p-1}\left\|P^{t}\left(x,\cdot\right)-\pi\left(\cdot\right)\right\|_{f}=0.
\end{equation}
En particular para cada condici\'on inicial
\begin{eqnarray*}
\lim_{t\rightarrow\infty}t^{p-1}|\esp_{x}\left[Q\left(t\right)\right]-\esp_{\pi}\left[Q\left(0\right)\right]|=0.
\end{eqnarray*}
\end{Teo}

\begin{Teo}[Teorema 6.4, Dai y Meyn, \cite{DaiSean}]\label{Tma.6.4.DaiSean}
Suponga que se cumplen los supuestos A1), A2) y A3) y que el
modelo de flujo es estable. Sea $\nu$ cualquier distribuci\'on de
probabilidad en $\left(X,\mathcal{B}_{X}\right)$, y $\pi$ la
distribuci\'on estacionaria de $X$.
\begin{itemize}
\item[i)] Para cualquier $f:X\leftarrow\rea_{+}$
\begin{equation}
\lim_{t\rightarrow\infty}\frac{1}{t}\int_{o}^{t}f\left(X\left(s\right)\right)ds=\pi\left(f\right):=\int
f\left(x\right)\pi\left(dx\right)
\end{equation}
$\prob$-c.s.

\item[ii)] Para cualquier $f:X\leftarrow\rea_{+}$ con
$\pi\left(|f|\right)<\infty$, la ecuaci\'on anterior se cumple.
\end{itemize}
\end{Teo}

\begin{Teo}[Teorema 2.2, Down \cite{Down}]\label{Tma2.2.Down}
Suponga que el fluido modelo es inestable en el sentido de que
para alguna $\epsilon_{0},c_{0}\geq0$,
\begin{equation}\label{Eq.Inestability}
|Q\left(T\right)|\geq\epsilon_{0}T-c_{0}\textrm{,   }T\geq0,
\end{equation}
para cualquier condici\'on inicial $Q\left(0\right)$, con
$|Q\left(0\right)|=1$. Entonces para cualquier $0<q\leq1$, existe
$B<0$ tal que para cualquier $|x|\geq B$,
\begin{equation}
\prob_{x}\left\{\mathbb{X}\rightarrow\infty\right\}\geq q.
\end{equation}
\end{Teo}



\begin{Def}
Sea $X$ un conjunto y $\mathcal{F}$ una $\sigma$-\'algebra de
subconjuntos de $X$, la pareja $\left(X,\mathcal{F}\right)$ es
llamado espacio medible. Un subconjunto $A$ de $X$ es llamado
medible, o medible con respecto a $\mathcal{F}$, si
$A\in\mathcal{F}$.
\end{Def}

\begin{Def}
Sea $\left(X,\mathcal{F},\mu\right)$ espacio de medida. Se dice
que la medida $\mu$ es $\sigma$-finita si se puede escribir
$X=\bigcup_{n\geq1}X_{n}$ con $X_{n}\in\mathcal{F}$ y
$\mu\left(X_{n}\right)<\infty$.
\end{Def}

\begin{Def}\label{Cto.Borel}
Sea $X$ el conjunto de los n\'umeros reales $\rea$. El \'algebra
de Borel es la $\sigma$-\'algebra $B$ generada por los intervalos
abiertos $\left(a,b\right)\in\rea$. Cualquier conjunto en $B$ es
llamado {\em Conjunto de Borel}.
\end{Def}

\begin{Def}\label{Funcion.Medible}
Una funci\'on $f:X\rightarrow\rea$, es medible si para cualquier
n\'umero real $\alpha$ el conjunto
\[\left\{x\in X:f\left(x\right)>\alpha\right\}\]
pertenece a $\mathcal{F}$. Equivalentemente, se dice que $f$ es
medible si
\[f^{-1}\left(\left(\alpha,\infty\right)\right)=\left\{x\in X:f\left(x\right)>\alpha\right\}\in\mathcal{F}.\]
\end{Def}


\begin{Def}\label{Def.Cilindros}
Sean $\left(\Omega_{i},\mathcal{F}_{i}\right)$, $i=1,2,\ldots,$
espacios medibles y $\Omega=\prod_{i=1}^{\infty}\Omega_{i}$ el
conjunto de todas las sucesiones
$\left(\omega_{1},\omega_{2},\ldots,\right)$ tales que
$\omega_{i}\in\Omega_{i}$, $i=1,2,\ldots,$. Si
$B^{n}\subset\prod_{i=1}^{\infty}\Omega_{i}$, definimos
$B_{n}=\left\{\omega\in\Omega:\left(\omega_{1},\omega_{2},\ldots,\omega_{n}\right)\in
B^{n}\right\}$. Al conjunto $B_{n}$ se le llama {\em cilindro} con
base $B^{n}$, el cilindro es llamado medible si
$B^{n}\in\prod_{i=1}^{\infty}\mathcal{F}_{i}$.
\end{Def}


\begin{Def}\label{Def.Proc.Adaptado}[TSP, Ash \cite{RBA}]
Sea $X\left(t\right),t\geq0$ proceso estoc\'astico, el proceso es
adaptado a la familia de $\sigma$-\'algebras $\mathcal{F}_{t}$,
para $t\geq0$, si para $s<t$ implica que
$\mathcal{F}_{s}\subset\mathcal{F}_{t}$, y $X\left(t\right)$ es
$\mathcal{F}_{t}$-medible para cada $t$. Si no se especifica
$\mathcal{F}_{t}$ entonces se toma $\mathcal{F}_{t}$ como
$\mathcal{F}\left(X\left(s\right),s\leq t\right)$, la m\'as
peque\~na $\sigma$-\'algebra de subconjuntos de $\Omega$ que hace
que cada $X\left(s\right)$, con $s\leq t$ sea Borel medible.
\end{Def}


\begin{Def}\label{Def.Tiempo.Paro}[TSP, Ash \cite{RBA}]
Sea $\left\{\mathcal{F}\left(t\right),t\geq0\right\}$ familia
creciente de sub $\sigma$-\'algebras. es decir,
$\mathcal{F}\left(s\right)\subset\mathcal{F}\left(t\right)$ para
$s\leq t$. Un tiempo de paro para $\mathcal{F}\left(t\right)$ es
una funci\'on $T:\Omega\rightarrow\left[0,\infty\right]$ tal que
$\left\{T\leq t\right\}\in\mathcal{F}\left(t\right)$ para cada
$t\geq0$. Un tiempo de paro para el proceso estoc\'astico
$X\left(t\right),t\geq0$ es un tiempo de paro para las
$\sigma$-\'algebras
$\mathcal{F}\left(t\right)=\mathcal{F}\left(X\left(s\right)\right)$.
\end{Def}

\begin{Def}
Sea $X\left(t\right),t\geq0$ proceso estoc\'astico, con
$\left(S,\chi\right)$ espacio de estados. Se dice que el proceso
es adaptado a $\left\{\mathcal{F}\left(t\right)\right\}$, es
decir, si para cualquier $s,t\in I$, $I$ conjunto de \'indices,
$s<t$, se tiene que
$\mathcal{F}\left(s\right)\subset\mathcal{F}\left(t\right)$ y
$X\left(t\right)$ es $\mathcal{F}\left(t\right)$-medible,
\end{Def}

\begin{Def}
Sea $X\left(t\right),t\geq0$ proceso estoc\'astico, se dice que es
un Proceso de Markov relativo a $\mathcal{F}\left(t\right)$ o que
$\left\{X\left(t\right),\mathcal{F}\left(t\right)\right\}$ es de
Markov si y s\'olo si para cualquier conjunto $B\in\chi$,  y
$s,t\in I$, $s<t$ se cumple que
\begin{equation}\label{Prop.Markov}
P\left\{X\left(t\right)\in
B|\mathcal{F}\left(s\right)\right\}=P\left\{X\left(t\right)\in
B|X\left(s\right)\right\}.
\end{equation}
\end{Def}
\begin{Note}
Si se dice que $\left\{X\left(t\right)\right\}$ es un Proceso de
Markov sin mencionar $\mathcal{F}\left(t\right)$, se asumir\'a que
\begin{eqnarray*}
\mathcal{F}\left(t\right)=\mathcal{F}_{0}\left(t\right)=\mathcal{F}\left(X\left(r\right),r\leq
t\right),
\end{eqnarray*}
entonces la ecuaci\'on (\ref{Prop.Markov}) se puede escribir como
\begin{equation}
P\left\{X\left(t\right)\in B|X\left(r\right),r\leq s\right\} =
P\left\{X\left(t\right)\in B|X\left(s\right)\right\}
\end{equation}
\end{Note}
%_______________________________________________________________
\subsection{Procesos de Estados de Markov}
%_______________________________________________________________

\begin{Teo}
Sea $\left(X_{n},\mathcal{F}_{n},n=0,1,\ldots,\right\}$ Proceso de
Markov con espacio de estados $\left(S_{0},\chi_{0}\right)$
generado por una distribuici\'on inicial $P_{o}$ y probabilidad de
transici\'on $p_{mn}$, para $m,n=0,1,\ldots,$ $m<n$, que por
notaci\'on se escribir\'a como $p\left(m,n,x,B\right)\rightarrow
p_{mn}\left(x,B\right)$. Sea $S$ tiempo de paro relativo a la
$\sigma$-\'algebra $\mathcal{F}_{n}$. Sea $T$ funci\'on medible,
$T:\Omega\rightarrow\left\{0,1,\ldots,\right\}$. Sup\'ongase que
$T\geq S$, entonces $T$ es tiempo de paro. Si $B\in\chi_{0}$,
entonces
\begin{equation}\label{Prop.Fuerte.Markov}
P\left\{X\left(T\right)\in
B,T<\infty|\mathcal{F}\left(S\right)\right\} =
p\left(S,T,X\left(s\right),B\right)
\end{equation}
en $\left\{T<\infty\right\}$.
\end{Teo}


Sea $K$ conjunto numerable y sea $d:K\rightarrow\nat$ funci\'on.
Para $v\in K$, $M_{v}$ es un conjunto abierto de
$\rea^{d\left(v\right)}$. Entonces \[E=\bigcup_{v\in
K}M_{v}=\left\{\left(v,\zeta\right):v\in K,\zeta\in
M_{v}\right\}.\]

Sea $\mathcal{E}$ la clase de conjuntos medibles en $E$:
\[\mathcal{E}=\left\{\bigcup_{v\in K}A_{v}:A_{v}\in \mathcal{M}_{v}\right\}.\]

donde $\mathcal{M}$ son los conjuntos de Borel de $M_{v}$.
Entonces $\left(E,\mathcal{E}\right)$ es un espacio de Borel. El
estado del proceso se denotar\'a por
$\mathbf{x}_{t}=\left(v_{t},\zeta_{t}\right)$. La distribuci\'on
de $\left(\mathbf{x}_{t}\right)$ est\'a determinada por por los
siguientes objetos:

\begin{itemize}
\item[i)] Los campos vectoriales $\left(\mathcal{H}_{v},v\in
K\right)$. \item[ii)] Una funci\'on medible $\lambda:E\rightarrow
\rea_{+}$. \item[iii)] Una medida de transici\'on
$Q:\mathcal{E}\times\left(E\cup\Gamma^{*}\right)\rightarrow\left[0,1\right]$
donde
\begin{equation}
\Gamma^{*}=\bigcup_{v\in K}\partial^{*}M_{v}.
\end{equation}
y
\begin{equation}
\partial^{*}M_{v}=\left\{z\in\partial M_{v}:\mathbf{\mathbf{\phi}_{v}\left(t,\zeta\right)=\mathbf{z}}\textrm{ para alguna }\left(t,\zeta\right)\in\rea_{+}\times M_{v}\right\}.
\end{equation}
$\partial M_{v}$ denota  la frontera de $M_{v}$.
\end{itemize}

El campo vectorial $\left(\mathcal{H}_{v},v\in K\right)$ se supone
tal que para cada $\mathbf{z}\in M_{v}$ existe una \'unica curva
integral $\mathbf{\phi}_{v}\left(t,\zeta\right)$ que satisface la
ecuaci\'on

\begin{equation}
\frac{d}{dt}f\left(\zeta_{t}\right)=\mathcal{H}f\left(\zeta_{t}\right),
\end{equation}
con $\zeta_{0}=\mathbf{z}$, para cualquier funci\'on suave
$f:\rea^{d}\rightarrow\rea$ y $\mathcal{H}$ denota el operador
diferencial de primer orden, con $\mathcal{H}=\mathcal{H}_{v}$ y
$\zeta_{t}=\mathbf{\phi}\left(t,\mathbf{z}\right)$. Adem\'as se
supone que $\mathcal{H}_{v}$ es conservativo, es decir, las curvas
integrales est\'an definidas para todo $t>0$.

Para $\mathbf{x}=\left(v,\zeta\right)\in E$ se denota
\[t^{*}\mathbf{x}=inf\left\{t>0:\mathbf{\phi}_{v}\left(t,\zeta\right)\in\partial^{*}M_{v}\right\}\]

En lo que respecta a la funci\'on $\lambda$, se supondr\'a que
para cada $\left(v,\zeta\right)\in E$ existe un $\epsilon>0$ tal
que la funci\'on
$s\rightarrow\lambda\left(v,\phi_{v}\left(s,\zeta\right)\right)\in
E$ es integrable para $s\in\left[0,\epsilon\right)$. La medida de
transici\'on $Q\left(A;\mathbf{x}\right)$ es una funci\'on medible
de $\mathbf{x}$ para cada $A\in\mathcal{E}$, definida para
$\mathbf{x}\in E\cup\Gamma^{*}$ y es una medida de probabilidad en
$\left(E,\mathcal{E}\right)$ para cada $\mathbf{x}\in E$.

El movimiento del proceso $\left(\mathbf{x}_{t}\right)$ comenzando
en $\mathbf{x}=\left(n,\mathbf{z}\right)\in E$ se puede construir
de la siguiente manera, def\'inase la funci\'on $F$ por

\begin{equation}
F\left(t\right)=\left\{\begin{array}{ll}\\
exp\left(-\int_{0}^{t}\lambda\left(n,\phi_{n}\left(s,\mathbf{z}\right)\right)ds\right), & t<t^{*}\left(\mathbf{x}\right),\\
0, & t\geq t^{*}\left(\mathbf{x}\right)
\end{array}\right.
\end{equation}

Sea $T_{1}$ una variable aleatoria tal que
$\prob\left[T_{1}>t\right]=F\left(t\right)$, ahora sea la variable
aleatoria $\left(N,Z\right)$ con distribuici\'on
$Q\left(\cdot;\phi_{n}\left(T_{1},\mathbf{z}\right)\right)$. La
trayectoria de $\left(\mathbf{x}_{t}\right)$ para $t\leq T_{1}$ es
\begin{eqnarray*}
\mathbf{x}_{t}=\left(v_{t},\zeta_{t}\right)=\left\{\begin{array}{ll}
\left(n,\phi_{n}\left(t,\mathbf{z}\right)\right), & t<T_{1},\\
\left(N,\mathbf{Z}\right), & t=t_{1}.
\end{array}\right.
\end{eqnarray*}

Comenzando en $\mathbf{x}_{T_{1}}$ se selecciona el siguiente
tiempo de intersalto $T_{2}-T_{1}$ lugar del post-salto
$\mathbf{x}_{T_{2}}$ de manera similar y as\'i sucesivamente. Este
procedimiento nos da una trayectoria determinista por partes
$\mathbf{x}_{t}$ con tiempos de salto $T_{1},T_{2},\ldots$. Bajo
las condiciones enunciadas para $\lambda,T_{1}>0$  y
$T_{1}-T_{2}>0$ para cada $i$, con probabilidad 1. Se supone que
se cumple la siguiente condici\'on.

\begin{Sup}[Supuesto 3.1, Davis \cite{Davis}]\label{Sup3.1.Davis}
Sea $N_{t}:=\sum_{t}\indora_{\left(t\geq t\right)}$ el n\'umero de
saltos en $\left[0,t\right]$. Entonces
\begin{equation}
\esp\left[N_{t}\right]<\infty\textrm{ para toda }t.
\end{equation}
\end{Sup}

es un proceso de Markov, m\'as a\'un, es un Proceso Fuerte de
Markov, es decir, la Propiedad Fuerte de Markov\footnote{Revisar
p\'agina 362, y 364 de Davis \cite{Davis}.} se cumple para
cualquier tiempo de paro.
%_________________________________________________________________________
%\renewcommand{\refname}{PROCESOS ESTOC\'ASTICOS}
%\renewcommand{\appendixname}{PROCESOS ESTOC\'ASTICOS}
%\renewcommand{\appendixtocname}{PROCESOS ESTOC\'ASTICOS}
%\renewcommand{\appendixpagename}{PROCESOS ESTOC\'ASTICOS}
%\appendix
%\clearpage % o \cleardoublepage
%\addappheadtotoc
%\appendixpage
%_________________________________________________________________________
\subsection{Teor\'ia General de Procesos Estoc\'asticos}
%_________________________________________________________________________
En esta secci\'on se har\'an las siguientes consideraciones: $E$
es un espacio m\'etrico separable y la m\'etrica $d$ es compatible
con la topolog\'ia.

\begin{Def}
Una medida finita, $\lambda$ en la $\sigma$-\'algebra de Borel de
un espacio metrizable $E$ se dice cerrada si
\begin{equation}\label{Eq.A2.3}
\lambda\left(E\right)=sup\left\{\lambda\left(K\right):K\textrm{ es
compacto en }E\right\}.
\end{equation}
\end{Def}

\begin{Def}
$E$ es un espacio de Rad\'on si cada medida finita en
$\left(E,\mathcal{B}\left(E\right)\right)$ es regular interior o cerrada,
{\em tight}.
\end{Def}


El siguiente teorema nos permite tener una mejor caracterizaci\'on de los espacios de Rad\'on:
\begin{Teo}\label{Tma.A2.2}
Sea $E$ espacio separable metrizable. Entonces $E$ es de Rad\'on
si y s\'olo s\'i cada medida finita en
$\left(E,\mathcal{B}\left(E\right)\right)$ es cerrada.
\end{Teo}

%_________________________________________________________________________________________
\subsection{Propiedades de Markov}
%_________________________________________________________________________________________

Sea $E$ espacio de estados, tal que $E$ es un espacio de Rad\'on, $\mathcal{B}\left(E\right)$ $\sigma$-\'algebra de Borel en $E$, que se denotar\'a por $\mathcal{E}$.

Sea $\left(X,\mathcal{G},\prob\right)$ espacio de probabilidad,
$I\subset\rea$ conjunto de índices. Sea $\mathcal{F}_{\leq t}$ la
$\sigma$-\'algebra natural definida como
$\sigma\left\{f\left(X_{r}\right):r\in I, r\leq
t,f\in\mathcal{E}\right\}$. Se considerar\'a una
$\sigma$-\'algebra m\'as general\footnote{qu\'e se quiere decir
con el t\'ermino: m\'as general?}, $ \left(\mathcal{G}_{t}\right)$
tal que $\left(X_{t}\right)$ sea $\mathcal{E}$-adaptado.

\begin{Def}
Una familia $\left(P_{s,t}\right)$ de kernels de Markov en $\left(E,\mathcal{E}\right)$ indexada por pares $s,t\in I$, con $s\leq t$ es una funci\'on de transici\'on en $\ER$, si  para todo $r\leq s< t$ en $I$ y todo $x\in E$, $B\in\mathcal{E}$
\begin{equation}\label{Eq.Kernels}
P_{r,t}\left(x,B\right)=\int_{E}P_{r,s}\left(x,dy\right)P_{s,t}\left(y,B\right)\footnote{Ecuaci\'on de Chapman-Kolmogorov}.
\end{equation}
\end{Def}

Se dice que la funci\'on de transici\'on $\KM$ en $\ER$ es la funci\'on de transici\'on para un proceso $\PE$  con valores en $E$ y que satisface la propiedad de Markov\footnote{\begin{equation}\label{Eq.1.4.S}
\prob\left\{H|\mathcal{G}_{t}\right\}=\prob\left\{H|X_{t}\right\}\textrm{ }H\in p\mathcal{F}_{\geq t}.
\end{equation}} (\ref{Eq.1.4.S}) relativa a $\left(\mathcal{G}_{t}\right)$ si

\begin{equation}\label{Eq.1.6.S}
\prob\left\{f\left(X_{t}\right)|\mathcal{G}_{s}\right\}=P_{s,t}f\left(X_{t}\right)\textrm{ }s\leq t\in I,\textrm{ }f\in b\mathcal{E}.
\end{equation}

\begin{Def}
Una familia $\left(P_{t}\right)_{t\geq0}$ de kernels de Markov en $\ER$ es llamada {\em Semigrupo de Transici\'on de Markov} o {\em Semigrupo de Transici\'on} si
\[P_{t+s}f\left(x\right)=P_{t}\left(P_{s}f\right)\left(x\right),\textrm{ }t,s\geq0,\textrm{ }x\in E\textrm{ }f\in b\mathcal{E}\footnote{Definir los t\'ermino $b\mathcal{E}$ y $p\mathcal{E}$}.\]
\end{Def}
\begin{Note}
Si la funci\'on de transici\'on $\KM$ es llamada homog\'enea si $P_{s,t}=P_{t-s}$.
\end{Note}

Un proceso de Markov que satisface la ecuaci\'on (\ref{Eq.1.6.S}) con funci\'on de transici\'on homog\'enea $\left(P_{t}\right)$ tiene la propiedad caracter\'istica
\begin{equation}\label{Eq.1.8.S}
\prob\left\{f\left(X_{t+s}\right)|\mathcal{G}_{t}\right\}=P_{s}f\left(X_{t}\right)\textrm{ }t,s\geq0,\textrm{ }f\in b\mathcal{E}.
\end{equation}
La ecuaci\'on anterior es la {\em Propiedad Simple de Markov} de $X$ relativa a $\left(P_{t}\right)$.

En este sentido el proceso $\PE$ cumple con la propiedad de Markov (\ref{Eq.1.8.S}) relativa a $\left(\Omega,\mathcal{G},\mathcal{G}_{t},\prob\right)$ con semigrupo de transici\'on $\left(P_{t}\right)$.
%_________________________________________________________________________________________
\subsection{Primer Condici\'on de Regularidad}
%_________________________________________________________________________________________
%\newcommand{\EM}{\left(\Omega,\mathcal{G},\prob\right)}
%\newcommand{\E4}{\left(\Omega,\mathcal{G},\mathcal{G}_{t},\prob\right)}
\begin{Def}
Un proceso estoc\'astico $\PE$ definido en
$\left(\Omega,\mathcal{G},\prob\right)$ con valores en el espacio
topol\'ogico $E$ es continuo por la derecha si cada trayectoria
muestral $t\rightarrow X_{t}\left(w\right)$ es un mapeo continuo
por la derecha de $I$ en $E$.
\end{Def}

\begin{Def}[HD1]\label{Eq.2.1.S}
Un semigrupo de Markov $\left(P_{t}\right)$ en un espacio de
Rad\'on $E$ se dice que satisface la condici\'on {\em HD1} si,
dada una medida de probabilidad $\mu$ en $E$, existe una
$\sigma$-\'algebra $\mathcal{E^{*}}$ con
$\mathcal{E}\subset\mathcal{E}^{*}$ y
$P_{t}\left(b\mathcal{E}^{*}\right)\subset b\mathcal{E}^{*}$, y un
$\mathcal{E}^{*}$-proceso $E$-valuado continuo por la derecha
$\PE$ en alg\'un espacio de probabilidad filtrado
$\left(\Omega,\mathcal{G},\mathcal{G}_{t},\prob\right)$ tal que
$X=\left(\Omega,\mathcal{G},\mathcal{G}_{t},\prob\right)$ es de
Markov (Homog\'eneo) con semigrupo de transici\'on $(P_{t})$ y
distribuci\'on inicial $\mu$.
\end{Def}

Consid\'erese la colecci\'on de variables aleatorias $X_{t}$
definidas en alg\'un espacio de probabilidad, y una colecci\'on de
medidas $\mathbf{P}^{x}$ tales que
$\mathbf{P}^{x}\left\{X_{0}=x\right\}$, y bajo cualquier
$\mathbf{P}^{x}$, $X_{t}$ es de Markov con semigrupo
$\left(P_{t}\right)$. $\mathbf{P}^{x}$ puede considerarse como la
distribuci\'on condicional de $\mathbf{P}$ dado $X_{0}=x$.

\begin{Def}\label{Def.2.2.S}
Sea $E$ espacio de Rad\'on, $\SG$ semigrupo de Markov en $\ER$. La colecci\'on $\mathbf{X}=\left(\Omega,\mathcal{G},\mathcal{G}_{t},X_{t},\theta_{t},\CM\right)$ es un proceso $\mathcal{E}$-Markov continuo por la derecha simple, con espacio de estados $E$ y semigrupo de transici\'on $\SG$ en caso de que $\mathbf{X}$ satisfaga las siguientes condiciones:
\begin{itemize}
\item[i)] $\left(\Omega,\mathcal{G},\mathcal{G}_{t}\right)$ es un espacio de medida filtrado, y $X_{t}$ es un proceso $E$-valuado continuo por la derecha $\mathcal{E}^{*}$-adaptado a $\left(\mathcal{G}_{t}\right)$;

\item[ii)] $\left(\theta_{t}\right)_{t\geq0}$ es una colecci\'on de operadores {\em shift} para $X$, es decir, mapea $\Omega$ en s\'i mismo satisfaciendo para $t,s\geq0$,

\begin{equation}\label{Eq.Shift}
\theta_{t}\circ\theta_{s}=\theta_{t+s}\textrm{ y }X_{t}\circ\theta_{t}=X_{t+s};
\end{equation}

\item[iii)] Para cualquier $x\in E$,$\CM\left\{X_{0}=x\right\}=1$, y el proceso $\PE$ tiene la propiedad de Markov (\ref{Eq.1.8.S}) con semigrupo de transici\'on $\SG$ relativo a $\left(\Omega,\mathcal{G},\mathcal{G}_{t},\CM\right)$.
\end{itemize}
\end{Def}

\begin{Def}[HD2]\label{Eq.2.2.S}
Para cualquier $\alpha>0$ y cualquier $f\in S^{\alpha}$, el proceso $t\rightarrow f\left(X_{t}\right)$ es continuo por la derecha casi seguramente.
\end{Def}

\begin{Def}\label{Def.PD}
Un sistema $\mathbf{X}=\left(\Omega,\mathcal{G},\mathcal{G}_{t},X_{t},\theta_{t},\CM\right)$ es un proceso derecho en el espacio de Rad\'on $E$ con semigrupo de transici\'on $\SG$ provisto de:
\begin{itemize}
\item[i)] $\mathbf{X}$ es una realizaci\'on  continua por la derecha, \ref{Def.2.2.S}, de $\SG$.

\item[ii)] $\mathbf{X}$ satisface la condicion HD2, \ref{Eq.2.2.S}, relativa a $\mathcal{G}_{t}$.

\item[iii)] $\mathcal{G}_{t}$ es aumentado y continuo por la derecha.
\end{itemize}
\end{Def}


%_________________________________________________________________________
%\renewcommand{\refname}{MODELO DE FLUJO}
%\renewcommand{\appendixname}{MODELO DE FLUJO}
%\renewcommand{\appendixtocname}{MODELO DE FLUJO}
%\renewcommand{\appendixpagename}{MODELO DE FLUJO}
%\appendix
%\clearpage % o \cleardoublepage
%\addappheadtotoc
%\appendixpage

\subsection{Construcci\'on del Modelo de Flujo}


\begin{Lema}[Lema 4.2, Dai\cite{Dai}]\label{Lema4.2}
Sea $\left\{x_{n}\right\}\subset \mathbf{X}$ con
$|x_{n}|\rightarrow\infty$, conforme $n\rightarrow\infty$. Suponga
que
\[lim_{n\rightarrow\infty}\frac{1}{|x_{n}|}U\left(0\right)=\overline{U}\]
y
\[lim_{n\rightarrow\infty}\frac{1}{|x_{n}|}V\left(0\right)=\overline{V}.\]

Entonces, conforme $n\rightarrow\infty$, casi seguramente

\begin{equation}\label{E1.4.2}
\frac{1}{|x_{n}|}\Phi^{k}\left(\left[|x_{n}|t\right]\right)\rightarrow
P_{k}^{'}t\textrm{, u.o.c.,}
\end{equation}

\begin{equation}\label{E1.4.3}
\frac{1}{|x_{n}|}E^{x_{n}}_{k}\left(|x_{n}|t\right)\rightarrow
\alpha_{k}\left(t-\overline{U}_{k}\right)^{+}\textrm{, u.o.c.,}
\end{equation}

\begin{equation}\label{E1.4.4}
\frac{1}{|x_{n}|}S^{x_{n}}_{k}\left(|x_{n}|t\right)\rightarrow
\mu_{k}\left(t-\overline{V}_{k}\right)^{+}\textrm{, u.o.c.,}
\end{equation}

donde $\left[t\right]$ es la parte entera de $t$ y
$\mu_{k}=1/m_{k}=1/\esp\left[\eta_{k}\left(1\right)\right]$.
\end{Lema}

\begin{Lema}[Lema 4.3, Dai\cite{Dai}]\label{Lema.4.3}
Sea $\left\{x_{n}\right\}\subset \mathbf{X}$ con
$|x_{n}|\rightarrow\infty$, conforme $n\rightarrow\infty$. Suponga
que
\[lim_{n\rightarrow\infty}\frac{1}{|x_{n}|}U_{k}\left(0\right)=\overline{U}_{k}\]
y
\[lim_{n\rightarrow\infty}\frac{1}{|x_{n}|}V_{k}\left(0\right)=\overline{V}_{k}.\]
\begin{itemize}
\item[a)] Conforme $n\rightarrow\infty$ casi seguramente,
\[lim_{n\rightarrow\infty}\frac{1}{|x_{n}|}U^{x_{n}}_{k}\left(|x_{n}|t\right)=\left(\overline{U}_{k}-t\right)^{+}\textrm{, u.o.c.}\]
y
\[lim_{n\rightarrow\infty}\frac{1}{|x_{n}|}V^{x_{n}}_{k}\left(|x_{n}|t\right)=\left(\overline{V}_{k}-t\right)^{+}.\]

\item[b)] Para cada $t\geq0$ fijo,
\[\left\{\frac{1}{|x_{n}|}U^{x_{n}}_{k}\left(|x_{n}|t\right),|x_{n}|\geq1\right\}\]
y
\[\left\{\frac{1}{|x_{n}|}V^{x_{n}}_{k}\left(|x_{n}|t\right),|x_{n}|\geq1\right\}\]
\end{itemize}
son uniformemente convergentes.
\end{Lema}

Sea $S_{l}^{x}\left(t\right)$ el n\'umero total de servicios
completados de la clase $l$, si la clase $l$ est\'a dando $t$
unidades de tiempo de servicio. Sea $T_{l}^{x}\left(x\right)$ el
monto acumulado del tiempo de servicio que el servidor
$s\left(l\right)$ gasta en los usuarios de la clase $l$ al tiempo
$t$. Entonces $S_{l}^{x}\left(T_{l}^{x}\left(t\right)\right)$ es
el n\'umero total de servicios completados para la clase $l$ al
tiempo $t$. Una fracci\'on de estos usuarios,
$\Phi_{k}^{x}\left(S_{l}^{x}\left(T_{l}^{x}\left(t\right)\right)\right)$,
se convierte en usuarios de la clase $k$.\\

Entonces, dado lo anterior, se tiene la siguiente representaci\'on
para el proceso de la longitud de la cola:\\

\begin{equation}
Q_{k}^{x}\left(t\right)=Q_{k}^{x}\left(0\right)+E_{k}^{x}\left(t\right)+\sum_{l=1}^{K}\Phi_{k}^{l}\left(S_{l}^{x}\left(T_{l}^{x}\left(t\right)\right)\right)-S_{k}^{x}\left(T_{k}^{x}\left(t\right)\right)
\end{equation}
para $k=1,\ldots,K$. Para $i=1,\ldots,d$, sea
\[I_{i}^{x}\left(t\right)=t-\sum_{j\in C_{i}}T_{k}^{x}\left(t\right).\]

Entonces $I_{i}^{x}\left(t\right)$ es el monto acumulado del
tiempo que el servidor $i$ ha estado desocupado al tiempo $t$. Se
est\'a asumiendo que las disciplinas satisfacen la ley de
conservaci\'on del trabajo, es decir, el servidor $i$ est\'a en
pausa solamente cuando no hay usuarios en la estaci\'on $i$.
Entonces, se tiene que

\begin{equation}
\int_{0}^{\infty}\left(\sum_{k\in
C_{i}}Q_{k}^{x}\left(t\right)\right)dI_{i}^{x}\left(t\right)=0,
\end{equation}
para $i=1,\ldots,d$.\\

Hacer
\[T^{x}\left(t\right)=\left(T_{1}^{x}\left(t\right),\ldots,T_{K}^{x}\left(t\right)\right)^{'},\]
\[I^{x}\left(t\right)=\left(I_{1}^{x}\left(t\right),\ldots,I_{K}^{x}\left(t\right)\right)^{'}\]
y
\[S^{x}\left(T^{x}\left(t\right)\right)=\left(S_{1}^{x}\left(T_{1}^{x}\left(t\right)\right),\ldots,S_{K}^{x}\left(T_{K}^{x}\left(t\right)\right)\right)^{'}.\]

Para una disciplina que cumple con la ley de conservaci\'on del
trabajo, en forma vectorial, se tiene el siguiente conjunto de
ecuaciones

\begin{equation}\label{Eq.MF.1.3}
Q^{x}\left(t\right)=Q^{x}\left(0\right)+E^{x}\left(t\right)+\sum_{l=1}^{K}\Phi^{l}\left(S_{l}^{x}\left(T_{l}^{x}\left(t\right)\right)\right)-S^{x}\left(T^{x}\left(t\right)\right),\\
\end{equation}

\begin{equation}\label{Eq.MF.2.3}
Q^{x}\left(t\right)\geq0,\\
\end{equation}

\begin{equation}\label{Eq.MF.3.3}
T^{x}\left(0\right)=0,\textrm{ y }\overline{T}^{x}\left(t\right)\textrm{ es no decreciente},\\
\end{equation}

\begin{equation}\label{Eq.MF.4.3}
I^{x}\left(t\right)=et-CT^{x}\left(t\right)\textrm{ es no
decreciente}\\
\end{equation}

\begin{equation}\label{Eq.MF.5.3}
\int_{0}^{\infty}\left(CQ^{x}\left(t\right)\right)dI_{i}^{x}\left(t\right)=0,\\
\end{equation}

\begin{equation}\label{Eq.MF.6.3}
\textrm{Condiciones adicionales en
}\left(\overline{Q}^{x}\left(\cdot\right),\overline{T}^{x}\left(\cdot\right)\right)\textrm{
espec\'ificas de la disciplina de la cola,}
\end{equation}

donde $e$ es un vector de unos de dimensi\'on $d$, $C$ es la
matriz definida por
\[C_{ik}=\left\{\begin{array}{cc}
1,& S\left(k\right)=i,\\
0,& \textrm{ en otro caso}.\\
\end{array}\right.
\]
Es necesario enunciar el siguiente Teorema que se utilizar\'a para
el Teorema \ref{Tma.4.2.Dai}:
\begin{Teo}[Teorema 4.1, Dai \cite{Dai}]
Considere una disciplina que cumpla la ley de conservaci\'on del
trabajo, para casi todas las trayectorias muestrales $\omega$ y
cualquier sucesi\'on de estados iniciales
$\left\{x_{n}\right\}\subset \mathbf{X}$, con
$|x_{n}|\rightarrow\infty$, existe una subsucesi\'on
$\left\{x_{n_{j}}\right\}$ con $|x_{n_{j}}|\rightarrow\infty$ tal
que
\begin{equation}\label{Eq.4.15}
\frac{1}{|x_{n_{j}}|}\left(Q^{x_{n_{j}}}\left(0\right),U^{x_{n_{j}}}\left(0\right),V^{x_{n_{j}}}\left(0\right)\right)\rightarrow\left(\overline{Q}\left(0\right),\overline{U},\overline{V}\right),
\end{equation}

\begin{equation}\label{Eq.4.16}
\frac{1}{|x_{n_{j}}|}\left(Q^{x_{n_{j}}}\left(|x_{n_{j}}|t\right),T^{x_{n_{j}}}\left(|x_{n_{j}}|t\right)\right)\rightarrow\left(\overline{Q}\left(t\right),\overline{T}\left(t\right)\right)\textrm{
u.o.c.}
\end{equation}

Adem\'as,
$\left(\overline{Q}\left(t\right),\overline{T}\left(t\right)\right)$
satisface las siguientes ecuaciones:
\begin{equation}\label{Eq.MF.1.3a}
\overline{Q}\left(t\right)=Q\left(0\right)+\left(\alpha
t-\overline{U}\right)^{+}-\left(I-P\right)^{'}M^{-1}\left(\overline{T}\left(t\right)-\overline{V}\right)^{+},
\end{equation}

\begin{equation}\label{Eq.MF.2.3a}
\overline{Q}\left(t\right)\geq0,\\
\end{equation}

\begin{equation}\label{Eq.MF.3.3a}
\overline{T}\left(t\right)\textrm{ es no decreciente y comienza en cero},\\
\end{equation}

\begin{equation}\label{Eq.MF.4.3a}
\overline{I}\left(t\right)=et-C\overline{T}\left(t\right)\textrm{
es no decreciente,}\\
\end{equation}

\begin{equation}\label{Eq.MF.5.3a}
\int_{0}^{\infty}\left(C\overline{Q}\left(t\right)\right)d\overline{I}\left(t\right)=0,\\
\end{equation}

\begin{equation}\label{Eq.MF.6.3a}
\textrm{Condiciones adicionales en
}\left(\overline{Q}\left(\cdot\right),\overline{T}\left(\cdot\right)\right)\textrm{
especficas de la disciplina de la cola,}
\end{equation}
\end{Teo}


Propiedades importantes para el modelo de flujo retrasado:

\begin{Prop}
 Sea $\left(\overline{Q},\overline{T},\overline{T}^{0}\right)$ un flujo l\'imite de \ref{Eq.4.4} y suponga que cuando $x\rightarrow\infty$ a lo largo de
una subsucesi\'on
\[\left(\frac{1}{|x|}Q_{k}^{x}\left(0\right),\frac{1}{|x|}A_{k}^{x}\left(0\right),\frac{1}{|x|}B_{k}^{x}\left(0\right),\frac{1}{|x|}B_{k}^{x,0}\left(0\right)\right)\rightarrow\left(\overline{Q}_{k}\left(0\right),0,0,0\right)\]
para $k=1,\ldots,K$. EL flujo l\'imite tiene las siguientes
propiedades, donde las propiedades de la derivada se cumplen donde
la derivada exista:
\begin{itemize}
 \item[i)] Los vectores de tiempo ocupado $\overline{T}\left(t\right)$ y $\overline{T}^{0}\left(t\right)$ son crecientes y continuas con
$\overline{T}\left(0\right)=\overline{T}^{0}\left(0\right)=0$.
\item[ii)] Para todo $t\geq0$
\[\sum_{k=1}^{K}\left[\overline{T}_{k}\left(t\right)+\overline{T}_{k}^{0}\left(t\right)\right]=t\]
\item[iii)] Para todo $1\leq k\leq K$
\[\overline{Q}_{k}\left(t\right)=\overline{Q}_{k}\left(0\right)+\alpha_{k}t-\mu_{k}\overline{T}_{k}\left(t\right)\]
\item[iv)]  Para todo $1\leq k\leq K$
\[\dot{{\overline{T}}}_{k}\left(t\right)=\beta_{k}\] para $\overline{Q}_{k}\left(t\right)=0$.
\item[v)] Para todo $k,j$
\[\mu_{k}^{0}\overline{T}_{k}^{0}\left(t\right)=\mu_{j}^{0}\overline{T}_{j}^{0}\left(t\right)\]
\item[vi)]  Para todo $1\leq k\leq K$
\[\mu_{k}\dot{{\overline{T}}}_{k}\left(t\right)=l_{k}\mu_{k}^{0}\dot{{\overline{T}}}_{k}^{0}\left(t\right)\] para $\overline{Q}_{k}\left(t\right)>0$.
\end{itemize}
\end{Prop}

\begin{Teo}[Teorema 5.1: Ley Fuerte para Procesos de Conteo
\cite{Gut}]\label{Tma.5.1.Gut} Sea
$0<\mu<\esp\left(X_{1}\right]\leq\infty$. entonces

\begin{itemize}
\item[a)] $\frac{N\left(t\right)}{t}\rightarrow\frac{1}{\mu}$
a.s., cuando $t\rightarrow\infty$.


\item[b)]$\esp\left[\frac{N\left(t\right)}{t}\right]^{r}\rightarrow\frac{1}{\mu^{r}}$,
cuando $t\rightarrow\infty$ para todo $r>0$..
\end{itemize}
\end{Teo}


\begin{Prop}[Proposici\'on 5.3 \cite{DaiSean}]
Sea $X$ proceso de estados para la red de colas, y suponga que se
cumplen los supuestos (A1) y (A2), entonces para alguna constante
positiva $C_{p+1}<\infty$, $\delta>0$ y un conjunto compacto
$C\subset X$.

\begin{equation}\label{Eq.5.4}
\esp_{x}\left[\int_{0}^{\tau_{C}\left(\delta\right)}\left(1+|X\left(t\right)|^{p}\right)dt\right]\leq
C_{p+1}\left(1+|x|^{p+1}\right)
\end{equation}
\end{Prop}

\begin{Prop}[Proposici\'on 5.4 \cite{DaiSean}]
Sea $X$ un proceso de Markov Borel Derecho en $X$, sea
$f:X\leftarrow\rea_{+}$ y defina para alguna $\delta>0$, y un
conjunto cerrado $C\subset X$
\[V\left(x\right):=\esp_{x}\left[\int_{0}^{\tau_{C}\left(\delta\right)}f\left(X\left(t\right)\right)dt\right]\]
para $x\in X$. Si $V$ es finito en todas partes y uniformemente
acotada en $C$, entonces existe $k<\infty$ tal que
\begin{equation}\label{Eq.5.11}
\frac{1}{t}\esp_{x}\left[V\left(x\right)\right]+\frac{1}{t}\int_{0}^{t}\esp_{x}\left[f\left(X\left(s\right)\right)ds\right]\leq\frac{1}{t}V\left(x\right)+k,
\end{equation}
para $x\in X$ y $t>0$.
\end{Prop}


%_________________________________________________________________________
%\renewcommand{\refname}{Ap\'endice D}
%\renewcommand{\appendixname}{ESTABILIDAD}
%\renewcommand{\appendixtocname}{ESTABILIDAD}
%\renewcommand{\appendixpagename}{ESTABILIDAD}
%\appendix
%\clearpage % o \cleardoublepage
%\addappheadtotoc
%\appendixpage

\subsection{Estabilidad}

\begin{Def}[Definici\'on 3.2, Dai y Meyn \cite{DaiSean}]
El modelo de flujo retrasado de una disciplina de servicio en una
red con retraso
$\left(\overline{A}\left(0\right),\overline{B}\left(0\right)\right)\in\rea_{+}^{K+|A|}$
se define como el conjunto de ecuaciones dadas en
\ref{Eq.3.8}-\ref{Eq.3.13}, junto con la condici\'on:
\begin{equation}\label{CondAd.FluidModel}
\overline{Q}\left(t\right)=\overline{Q}\left(0\right)+\left(\alpha
t-\overline{A}\left(0\right)\right)^{+}-\left(I-P^{'}\right)M\left(\overline{T}\left(t\right)-\overline{B}\left(0\right)\right)^{+}
\end{equation}
\end{Def}

entonces si el modelo de flujo retrasado tambi\'en es estable:


\begin{Def}[Definici\'on 3.1, Dai y Meyn \cite{DaiSean}]
Un flujo l\'imite (retrasado) para una red bajo una disciplina de
servicio espec\'ifica se define como cualquier soluci\'on
 $\left(\overline{Q}\left(\cdot\right),\overline{T}\left(\cdot\right)\right)$ de las siguientes ecuaciones, donde
$\overline{Q}\left(t\right)=\left(\overline{Q}_{1}\left(t\right),\ldots,\overline{Q}_{K}\left(t\right)\right)^{'}$
y
$\overline{T}\left(t\right)=\left(\overline{T}_{1}\left(t\right),\ldots,\overline{T}_{K}\left(t\right)\right)^{'}$
\begin{equation}\label{Eq.3.8}
\overline{Q}_{k}\left(t\right)=\overline{Q}_{k}\left(0\right)+\alpha_{k}t-\mu_{k}\overline{T}_{k}\left(t\right)+\sum_{l=1}^{k}P_{lk}\mu_{l}\overline{T}_{l}\left(t\right)\\
\end{equation}
\begin{equation}\label{Eq.3.9}
\overline{Q}_{k}\left(t\right)\geq0\textrm{ para }k=1,2,\ldots,K,\\
\end{equation}
\begin{equation}\label{Eq.3.10}
\overline{T}_{k}\left(0\right)=0,\textrm{ y }\overline{T}_{k}\left(\cdot\right)\textrm{ es no decreciente},\\
\end{equation}
\begin{equation}\label{Eq.3.11}
\overline{I}_{i}\left(t\right)=t-\sum_{k\in C_{i}}\overline{T}_{k}\left(t\right)\textrm{ es no decreciente}\\
\end{equation}
\begin{equation}\label{Eq.3.12}
\overline{I}_{i}\left(\cdot\right)\textrm{ se incrementa al tiempo }t\textrm{ cuando }\sum_{k\in C_{i}}Q_{k}^{x}\left(t\right)dI_{i}^{x}\left(t\right)=0\\
\end{equation}
\begin{equation}\label{Eq.3.13}
\textrm{condiciones adicionales sobre
}\left(Q^{x}\left(\cdot\right),T^{x}\left(\cdot\right)\right)\textrm{
referentes a la disciplina de servicio}
\end{equation}
\end{Def}

\begin{Lema}[Lema 3.1 \cite{Chen}]\label{Lema3.1}
Si el modelo de flujo es estable, definido por las ecuaciones
(3.8)-(3.13), entonces el modelo de flujo retrasado tambin es
estable.
\end{Lema}

\begin{Teo}[Teorema 5.1 \cite{Chen}]\label{Tma.5.1.Chen}
La red de colas es estable si existe una constante $t_{0}$ que
depende de $\left(\alpha,\mu,T,U\right)$ y $V$ que satisfagan las
ecuaciones (5.1)-(5.5), $Z\left(t\right)=0$, para toda $t\geq
t_{0}$.
\end{Teo}

\begin{Prop}[Proposici\'on 5.1, Dai y Meyn \cite{DaiSean}]\label{Prop.5.1.DaiSean}
Suponga que los supuestos A1) y A2) son ciertos y que el modelo de flujo es estable. Entonces existe $t_{0}>0$ tal que
\begin{equation}
lim_{|x|\rightarrow\infty}\frac{1}{|x|^{p+1}}\esp_{x}\left[|X\left(t_{0}|x|\right)|^{p+1}\right]=0
\end{equation}
\end{Prop}

\begin{Lemma}[Lema 5.2, Dai y Meyn \cite{DaiSean}]\label{Lema.5.2.DaiSean}
 Sea $\left\{\zeta\left(k\right):k\in \mathbb{z}\right\}$ una sucesi\'on independiente e id\'enticamente distribuida que toma valores en $\left(0,\infty\right)$,
y sea
$E\left(t\right)=max\left(n\geq1:\zeta\left(1\right)+\cdots+\zeta\left(n-1\right)\leq
t\right)$. Si $\esp\left[\zeta\left(1\right)\right]<\infty$,
entonces para cualquier entero $r\geq1$
\begin{equation}
 lim_{t\rightarrow\infty}\esp\left[\left(\frac{E\left(t\right)}{t}\right)^{r}\right]=\left(\frac{1}{\esp\left[\zeta_{1}\right]}\right)^{r}.
\end{equation}
Luego, bajo estas condiciones:
\begin{itemize}
 \item[a)] para cualquier $\delta>0$, $\sup_{t\geq\delta}\esp\left[\left(\frac{E\left(t\right)}{t}\right)^{r}\right]<\infty$
\item[b)] las variables aleatorias
$\left\{\left(\frac{E\left(t\right)}{t}\right)^{r}:t\geq1\right\}$
son uniformemente integrables.
\end{itemize}
\end{Lemma}

\begin{Teo}[Teorema 5.5, Dai y Meyn \cite{DaiSean}]\label{Tma.5.5.DaiSean}
Suponga que los supuestos A1) y A2) se cumplen y que el modelo de
flujo es estable. Entonces existe una constante $\kappa_{p}$ tal
que
\begin{equation}
\frac{1}{t}\int_{0}^{t}\esp_{x}\left[|Q\left(s\right)|^{p}\right]ds\leq\kappa_{p}\left\{\frac{1}{t}|x|^{p+1}+1\right\}
\end{equation}
para $t>0$ y $x\in X$. En particular, para cada condici\'on
inicial
\begin{eqnarray*}
\limsup_{t\rightarrow\infty}\frac{1}{t}\int_{0}^{t}\esp_{x}\left[|Q\left(s\right)|^{p}\right]ds\leq\kappa_{p}.
\end{eqnarray*}
\end{Teo}

\begin{Teo}[Teorema 6.2, Dai y Meyn \cite{DaiSean}]\label{Tma.6.2.DaiSean}
Suponga que se cumplen los supuestos A1), A2) y A3) y que el
modelo de flujo es estable. Entonces se tiene que
\begin{equation}
\left\|P^{t}\left(x,\cdot\right)-\pi\left(\cdot\right)\right\|_{f_{p}}\textrm{,
}t\rightarrow\infty,x\in X.
\end{equation}
En particular para cada condici\'on inicial
\begin{eqnarray*}
\lim_{t\rightarrow\infty}\esp_{x}\left[|Q\left(t\right)|^{p}\right]=\esp_{\pi}\left[|Q\left(0\right)|^{p}\right]\leq\kappa_{r}
\end{eqnarray*}
\end{Teo}
\begin{Teo}[Teorema 6.3, Dai y Meyn \cite{DaiSean}]\label{Tma.6.3.DaiSean}
Suponga que se cumplen los supuestos A1), A2) y A3) y que el
modelo de flujo es estable. Entonces con
$f\left(x\right)=f_{1}\left(x\right)$ se tiene
\begin{equation}
\lim_{t\rightarrow\infty}t^{p-1}\left\|P^{t}\left(x,\cdot\right)-\pi\left(\cdot\right)\right\|_{f}=0.
\end{equation}
En particular para cada condici\'on inicial
\begin{eqnarray*}
\lim_{t\rightarrow\infty}t^{p-1}|\esp_{x}\left[Q\left(t\right)\right]-\esp_{\pi}\left[Q\left(0\right)\right]|=0.
\end{eqnarray*}
\end{Teo}

\begin{Teo}[Teorema 6.4, Dai y Meyn \cite{DaiSean}]\label{Tma.6.4.DaiSean}
Suponga que se cumplen los supuestos A1), A2) y A3) y que el
modelo de flujo es estable. Sea $\nu$ cualquier distribuci\'on de
probabilidad en $\left(X,\mathcal{B}_{X}\right)$, y $\pi$ la
distribuci\'on estacionaria de $X$.
\begin{itemize}
\item[i)] Para cualquier $f:X\leftarrow\rea_{+}$
\begin{equation}
\lim_{t\rightarrow\infty}\frac{1}{t}\int_{o}^{t}f\left(X\left(s\right)\right)ds=\pi\left(f\right):=\int
f\left(x\right)\pi\left(dx\right)
\end{equation}
$\prob$-c.s.

\item[ii)] Para cualquier $f:X\leftarrow\rea_{+}$ con
$\pi\left(|f|\right)<\infty$, la ecuaci\'on anterior se cumple.
\end{itemize}
\end{Teo}

\begin{Teo}[Teorema 2.2, Down \cite{Down}]\label{Tma2.2.Down}
Suponga que el fluido modelo es inestable en el sentido de que
para alguna $\epsilon_{0},c_{0}\geq0$,
\begin{equation}\label{Eq.Inestability}
|Q\left(T\right)|\geq\epsilon_{0}T-c_{0}\textrm{,   }T\geq0,
\end{equation}
para cualquier condici\'on inicial $Q\left(0\right)$, con
$|Q\left(0\right)|=1$. Entonces para cualquier $0<q\leq1$, existe
$B<0$ tal que para cualquier $|x|\geq B$,
\begin{equation}
\prob_{x}\left\{\mathbb{X}\rightarrow\infty\right\}\geq q.
\end{equation}
\end{Teo}


\begin{Def}
Sea $X$ un conjunto y $\mathcal{F}$ una $\sigma$-\'algebra de
subconjuntos de $X$, la pareja $\left(X,\mathcal{F}\right)$ es
llamado espacio medible. Un subconjunto $A$ de $X$ es llamado
medible, o medible con respecto a $\mathcal{F}$, si
$A\in\mathcal{F}$.
\end{Def}

\begin{Def}
Sea $\left(X,\mathcal{F},\mu\right)$ espacio de medida. Se dice
que la medida $\mu$ es $\sigma$-finita si se puede escribir
$X=\bigcup_{n\geq1}X_{n}$ con $X_{n}\in\mathcal{F}$ y
$\mu\left(X_{n}\right)<\infty$.
\end{Def}

\begin{Def}\label{Cto.Borel}
Sea $X$ el conjunto de los \'umeros reales $\rea$. El \'algebra de
Borel es la $\sigma$-\'algebra $B$ generada por los intervalos
abiertos $\left(a,b\right)\in\rea$. Cualquier conjunto en $B$ es
llamado {\em Conjunto de Borel}.
\end{Def}

\begin{Def}\label{Funcion.Medible}
Una funci\'on $f:X\rightarrow\rea$, es medible si para cualquier
n\'umero real $\alpha$ el conjunto
\[\left\{x\in X:f\left(x\right)>\alpha\right\}\]
pertenece a $X$. Equivalentemente, se dice que $f$ es medible si
\[f^{-1}\left(\left(\alpha,\infty\right)\right)=\left\{x\in X:f\left(x\right)>\alpha\right\}\in\mathcal{F}.\]
\end{Def}


\begin{Def}\label{Def.Cilindros}
Sean $\left(\Omega_{i},\mathcal{F}_{i}\right)$, $i=1,2,\ldots,$
espacios medibles y $\Omega=\prod_{i=1}^{\infty}\Omega_{i}$ el
conjunto de todas las sucesiones
$\left(\omega_{1},\omega_{2},\ldots,\right)$ tales que
$\omega_{i}\in\Omega_{i}$, $i=1,2,\ldots,$. Si
$B^{n}\subset\prod_{i=1}^{\infty}\Omega_{i}$, definimos
$B_{n}=\left\{\omega\in\Omega:\left(\omega_{1},\omega_{2},\ldots,\omega_{n}\right)\in
B^{n}\right\}$. Al conjunto $B_{n}$ se le llama {\em cilindro} con
base $B^{n}$, el cilindro es llamado medible si
$B^{n}\in\prod_{i=1}^{\infty}\mathcal{F}_{i}$.
\end{Def}


\begin{Def}\label{Def.Proc.Adaptado}[TSP, Ash \cite{RBA}]
Sea $X\left(t\right),t\geq0$ proceso estoc\'astico, el proceso es
adaptado a la familia de $\sigma$-\'algebras $\mathcal{F}_{t}$,
para $t\geq0$, si para $s<t$ implica que
$\mathcal{F}_{s}\subset\mathcal{F}_{t}$, y $X\left(t\right)$ es
$\mathcal{F}_{t}$-medible para cada $t$. Si no se especifica
$\mathcal{F}_{t}$ entonces se toma $\mathcal{F}_{t}$ como
$\mathcal{F}\left(X\left(s\right),s\leq t\right)$, la m\'as
peque\~na $\sigma$-\'algebra de subconjuntos de $\Omega$ que hace
que cada $X\left(s\right)$, con $s\leq t$ sea Borel medible.
\end{Def}


\begin{Def}\label{Def.Tiempo.Paro}[TSP, Ash \cite{RBA}]
Sea $\left\{\mathcal{F}\left(t\right),t\geq0\right\}$ familia
creciente de sub $\sigma$-\'algebras. es decir,
$\mathcal{F}\left(s\right)\subset\mathcal{F}\left(t\right)$ para
$s\leq t$. Un tiempo de paro para $\mathcal{F}\left(t\right)$ es
una funci\'on $T:\Omega\rightarrow\left[0,\infty\right]$ tal que
$\left\{T\leq t\right\}\in\mathcal{F}\left(t\right)$ para cada
$t\geq0$. Un tiempo de paro para el proceso estoc\'astico
$X\left(t\right),t\geq0$ es un tiempo de paro para las
$\sigma$-\'algebras
$\mathcal{F}\left(t\right)=\mathcal{F}\left(X\left(s\right)\right)$.
\end{Def}

\begin{Def}
Sea $X\left(t\right),t\geq0$ proceso estoc\'astico, con
$\left(S,\chi\right)$ espacio de estados. Se dice que el proceso
es adaptado a $\left\{\mathcal{F}\left(t\right)\right\}$, es
decir, si para cualquier $s,t\in I$, $I$ conjunto de \'indices,
$s<t$, se tiene que
$\mathcal{F}\left(s\right)\subset\mathcal{F}\left(t\right)$ y
$X\left(t\right)$ es $\mathcal{F}\left(t\right)$-medible,
\end{Def}

\begin{Def}
Sea $X\left(t\right),t\geq0$ proceso estoc\'astico, se dice que es
un Proceso de Markov relativo a $\mathcal{F}\left(t\right)$ o que
$\left\{X\left(t\right),\mathcal{F}\left(t\right)\right\}$ es de
Markov si y s\'olo si para cualquier conjunto $B\in\chi$,  y
$s,t\in I$, $s<t$ se cumple que
\begin{equation}\label{Prop.Markov}
P\left\{X\left(t\right)\in
B|\mathcal{F}\left(s\right)\right\}=P\left\{X\left(t\right)\in
B|X\left(s\right)\right\}.
\end{equation}
\end{Def}
\begin{Note}
Si se dice que $\left\{X\left(t\right)\right\}$ es un Proceso de
Markov sin mencionar $\mathcal{F}\left(t\right)$, se asumir\'a que
\begin{eqnarray*}
\mathcal{F}\left(t\right)=\mathcal{F}_{0}\left(t\right)=\mathcal{F}\left(X\left(r\right),r\leq
t\right),
\end{eqnarray*}
entonces la ecuaci\'on (\ref{Prop.Markov}) se puede escribir como
\begin{equation}
P\left\{X\left(t\right)\in B|X\left(r\right),r\leq s\right\} =
P\left\{X\left(t\right)\in B|X\left(s\right)\right\}
\end{equation}
\end{Note}

\begin{Teo}
Sea $\left(X_{n},\mathcal{F}_{n},n=0,1,\ldots,\right\}$ Proceso de
Markov con espacio de estados $\left(S_{0},\chi_{0}\right)$
generado por una distribuici\'on inicial $P_{o}$ y probabilidad de
transici\'on $p_{mn}$, para $m,n=0,1,\ldots,$ $m<n$, que por
notaci\'on se escribir\'a como $p\left(m,n,x,B\right)\rightarrow
p_{mn}\left(x,B\right)$. Sea $S$ tiempo de paro relativo a la
$\sigma$-\'algebra $\mathcal{F}_{n}$. Sea $T$ funci\'on medible,
$T:\Omega\rightarrow\left\{0,1,\ldots,\right\}$. Sup\'ongase que
$T\geq S$, entonces $T$ es tiempo de paro. Si $B\in\chi_{0}$,
entonces
\begin{equation}\label{Prop.Fuerte.Markov}
P\left\{X\left(T\right)\in
B,T<\infty|\mathcal{F}\left(S\right)\right\} =
p\left(S,T,X\left(s\right),B\right)
\end{equation}
en $\left\{T<\infty\right\}$.
\end{Teo}


Sea $K$ conjunto numerable y sea $d:K\rightarrow\nat$ funci\'on.
Para $v\in K$, $M_{v}$ es un conjunto abierto de
$\rea^{d\left(v\right)}$. Entonces \[E=\cup_{v\in
K}M_{v}=\left\{\left(v,\zeta\right):v\in K,\zeta\in
M_{v}\right\}.\]

Sea $\mathcal{E}$ la clase de conjuntos medibles en $E$:
\[\mathcal{E}=\left\{\cup_{v\in K}A_{v}:A_{v}\in \mathcal{M}_{v}\right\}.\]

donde $\mathcal{M}$ son los conjuntos de Borel de $M_{v}$.
Entonces $\left(E,\mathcal{E}\right)$ es un espacio de Borel. El
estado del proceso se denotar\'a por
$\mathbf{x}_{t}=\left(v_{t},\zeta_{t}\right)$. La distribuci\'on
de $\left(\mathbf{x}_{t}\right)$ est\'a determinada por por los
siguientes objetos:

\begin{itemize}
\item[i)] Los campos vectoriales $\left(\mathcal{H}_{v},v\in
K\right)$. \item[ii)] Una funci\'on medible $\lambda:E\rightarrow
\rea_{+}$. \item[iii)] Una medida de transici\'on
$Q:\mathcal{E}\times\left(E\cup\Gamma^{*}\right)\rightarrow\left[0,1\right]$
donde
\begin{equation}
\Gamma^{*}=\cup_{v\in K}\partial^{*}M_{v}.
\end{equation}
y
\begin{equation}
\partial^{*}M_{v}=\left\{z\in\partial M_{v}:\mathbf{\mathbf{\phi}_{v}\left(t,\zeta\right)=\mathbf{z}}\textrm{ para alguna }\left(t,\zeta\right)\in\rea_{+}\times M_{v}\right\}.
\end{equation}
$\partial M_{v}$ denota  la frontera de $M_{v}$.
\end{itemize}

El campo vectorial $\left(\mathcal{H}_{v},v\in K\right)$ se supone
tal que para cada $\mathbf{z}\in M_{v}$ existe una \'unica curva
integral $\mathbf{\phi}_{v}\left(t,\zeta\right)$ que satisface la
ecuaci\'on

\begin{equation}
\frac{d}{dt}f\left(\zeta_{t}\right)=\mathcal{H}f\left(\zeta_{t}\right),
\end{equation}
con $\zeta_{0}=\mathbf{z}$, para cualquier funci\'on suave
$f:\rea^{d}\rightarrow\rea$ y $\mathcal{H}$ denota el operador
diferencial de primer orden, con $\mathcal{H}=\mathcal{H}_{v}$ y
$\zeta_{t}=\mathbf{\phi}\left(t,\mathbf{z}\right)$. Adem\'as se
supone que $\mathcal{H}_{v}$ es conservativo, es decir, las curvas
integrales est\'an definidas para todo $t>0$.

Para $\mathbf{x}=\left(v,\zeta\right)\in E$ se denota
\[t^{*}\mathbf{x}=inf\left\{t>0:\mathbf{\phi}_{v}\left(t,\zeta\right)\in\partial^{*}M_{v}\right\}\]

En lo que respecta a la funci\'on $\lambda$, se supondr\'a que
para cada $\left(v,\zeta\right)\in E$ existe un $\epsilon>0$ tal
que la funci\'on
$s\rightarrow\lambda\left(v,\phi_{v}\left(s,\zeta\right)\right)\in
E$ es integrable para $s\in\left[0,\epsilon\right)$. La medida de
transici\'on $Q\left(A;\mathbf{x}\right)$ es una funci\'on medible
de $\mathbf{x}$ para cada $A\in\mathcal{E}$, definida para
$\mathbf{x}\in E\cup\Gamma^{*}$ y es una medida de probabilidad en
$\left(E,\mathcal{E}\right)$ para cada $\mathbf{x}\in E$.

El movimiento del proceso $\left(\mathbf{x}_{t}\right)$ comenzando
en $\mathbf{x}=\left(n,\mathbf{z}\right)\in E$ se puede construir
de la siguiente manera, def\'inase la funci\'on $F$ por

\begin{equation}
F\left(t\right)=\left\{\begin{array}{ll}\\
exp\left(-\int_{0}^{t}\lambda\left(n,\phi_{n}\left(s,\mathbf{z}\right)\right)ds\right), & t<t^{*}\left(\mathbf{x}\right),\\
0, & t\geq t^{*}\left(\mathbf{x}\right)
\end{array}\right.
\end{equation}

Sea $T_{1}$ una variable aleatoria tal que
$\prob\left[T_{1}>t\right]=F\left(t\right)$, ahora sea la variable
aleatoria $\left(N,Z\right)$ con distribuici\'on
$Q\left(\cdot;\phi_{n}\left(T_{1},\mathbf{z}\right)\right)$. La
trayectoria de $\left(\mathbf{x}_{t}\right)$ para $t\leq T_{1}$
es\footnote{Revisar p\'agina 362, y 364 de Davis \cite{Davis}.}
\begin{eqnarray*}
\mathbf{x}_{t}=\left(v_{t},\zeta_{t}\right)=\left\{\begin{array}{ll}
\left(n,\phi_{n}\left(t,\mathbf{z}\right)\right), & t<T_{1},\\
\left(N,\mathbf{Z}\right), & t=t_{1}.
\end{array}\right.
\end{eqnarray*}

Comenzando en $\mathbf{x}_{T_{1}}$ se selecciona el siguiente
tiempo de intersalto $T_{2}-T_{1}$ lugar del post-salto
$\mathbf{x}_{T_{2}}$ de manera similar y as\'i sucesivamente. Este
procedimiento nos da una trayectoria determinista por partes
$\mathbf{x}_{t}$ con tiempos de salto $T_{1},T_{2},\ldots$. Bajo
las condiciones enunciadas para $\lambda,T_{1}>0$  y
$T_{1}-T_{2}>0$ para cada $i$, con probabilidad 1. Se supone que
se cumple la siquiente condici\'on.

\begin{Sup}[Supuesto 3.1, Davis \cite{Davis}]\label{Sup3.1.Davis}
Sea $N_{t}:=\sum_{t}\indora_{\left(t\geq t\right)}$ el n\'umero de
saltos en $\left[0,t\right]$. Entonces
\begin{equation}
\esp\left[N_{t}\right]<\infty\textrm{ para toda }t.
\end{equation}
\end{Sup}

es un proceso de Markov, m\'as a\'un, es un Proceso Fuerte de
Markov, es decir, la Propiedad Fuerte de Markov se cumple para
cualquier tiempo de paro.
%_________________________________________________________________________

En esta secci\'on se har\'an las siguientes consideraciones: $E$
es un espacio m\'etrico separable y la m\'etrica $d$ es compatible
con la topolog\'ia.


\begin{Def}
Un espacio topol\'ogico $E$ es llamado {\em Luisin} si es
homeomorfo a un subconjunto de Borel de un espacio m\'etrico
compacto.
\end{Def}

\begin{Def}
Un espacio topol\'ogico $E$ es llamado de {\em Rad\'on} si es
homeomorfo a un subconjunto universalmente medible de un espacio
m\'etrico compacto.
\end{Def}

Equivalentemente, la definici\'on de un espacio de Rad\'on puede
encontrarse en los siguientes t\'erminos:


\begin{Def}
$E$ es un espacio de Rad\'on si cada medida finita en
$\left(E,\mathcal{B}\left(E\right)\right)$ es regular interior o cerrada,
{\em tight}.
\end{Def}

\begin{Def}
Una medida finita, $\lambda$ en la $\sigma$-\'algebra de Borel de
un espacio metrizable $E$ se dice cerrada si
\begin{equation}\label{Eq.A2.3}
\lambda\left(E\right)=sup\left\{\lambda\left(K\right):K\textrm{ es
compacto en }E\right\}.
\end{equation}
\end{Def}

El siguiente teorema nos permite tener una mejor caracterizaci\'on de los espacios de Rad\'on:
\begin{Teo}\label{Tma.A2.2}
Sea $E$ espacio separable metrizable. Entonces $E$ es Radoniano si y s\'olo s\'i cada medida finita en $\left(E,\mathcal{B}\left(E\right)\right)$ es cerrada.
\end{Teo}

%_________________________________________________________________________________________
\subsection{Propiedades de Markov}
%_________________________________________________________________________________________

Sea $E$ espacio de estados, tal que $E$ es un espacio de Rad\'on, $\mathcal{B}\left(E\right)$ $\sigma$-\'algebra de Borel en $E$, que se denotar\'a por $\mathcal{E}$.

Sea $\left(X,\mathcal{G},\prob\right)$ espacio de probabilidad, $I\subset\rea$ conjunto de índices. Sea $\mathcal{F}_{\leq t}$ la $\sigma$-\'algebra natural definida como $\sigma\left\{f\left(X_{r}\right):r\in I, rleq t,f\in\mathcal{E}\right\}$. Se considerar\'a una $\sigma$-\'algebra m\'as general, $ \left(\mathcal{G}_{t}\right)$ tal que $\left(X_{t}\right)$ sea $\mathcal{E}$-adaptado.

\begin{Def}
Una familia $\left(P_{s,t}\right)$ de kernels de Markov en $\left(E,\mathcal{E}\right)$ indexada por pares $s,t\in I$, con $s\leq t$ es una funci\'on de transici\'on en $\ER$, si  para todo $r\leq s< t$ en $I$ y todo $x\in E$, $B\in\mathcal{E}$
\begin{equation}\label{Eq.Kernels}
P_{r,t}\left(x,B\right)=\int_{E}P_{r,s}\left(x,dy\right)P_{s,t}\left(y,B\right)\footnote{Ecuaci\'on de Chapman-Kolmogorov}.
\end{equation}
\end{Def}

Se dice que la funci\'on de transici\'on $\KM$ en $\ER$ es la funci\'on de transici\'on para un proceso $\PE$  con valores en $E$ y que satisface la propiedad de Markov\footnote{\begin{equation}\label{Eq.1.4.S}
\prob\left\{H|\mathcal{G}_{t}\right\}=\prob\left\{H|X_{t}\right\}\textrm{ }H\in p\mathcal{F}_{\geq t}.
\end{equation}} (\ref{Eq.1.4.S}) relativa a $\left(\mathcal{G}_{t}\right)$ si 

\begin{equation}\label{Eq.1.6.S}
\prob\left\{f\left(X_{t}\right)|\mathcal{G}_{s}\right\}=P_{s,t}f\left(X_{t}\right)\textrm{ }s\leq t\in I,\textrm{ }f\in b\mathcal{E}.
\end{equation}

\begin{Def}
Una familia $\left(P_{t}\right)_{t\geq0}$ de kernels de Markov en $\ER$ es llamada {\em Semigrupo de Transici\'on de Markov} o {\em Semigrupo de Transici\'on} si
\[P_{t+s}f\left(x\right)=P_{t}\left(P_{s}f\right)\left(x\right),\textrm{ }t,s\geq0,\textrm{ }x\in E\textrm{ }f\in b\mathcal{E}.\]
\end{Def}
\begin{Note}
Si la funci\'on de transici\'on $\KM$ es llamada homog\'enea si $P_{s,t}=P_{t-s}$.
\end{Note}

Un proceso de Markov que satisface la ecuaci\'on (\ref{Eq.1.6.S}) con funci\'on de transici\'on homog\'enea $\left(P_{t}\right)$ tiene la propiedad caracter\'istica
\begin{equation}\label{Eq.1.8.S}
\prob\left\{f\left(X_{t+s}\right)|\mathcal{G}_{t}\right\}=P_{s}f\left(X_{t}\right)\textrm{ }t,s\geq0,\textrm{ }f\in b\mathcal{E}.
\end{equation}
La ecuaci\'on anterior es la {\em Propiedad Simple de Markov} de $X$ relativa a $\left(P_{t}\right)$.

En este sentido el proceso $\PE$ cumple con la propiedad de Markov (\ref{Eq.1.8.S}) relativa a $\left(\Omega,\mathcal{G},\mathcal{G}_{t},\prob\right)$ con semigrupo de transici\'on $\left(P_{t}\right)$.
%_________________________________________________________________________________________
\subsection{Primer Condici\'on de Regularidad}
%_________________________________________________________________________________________
%\newcommand{\EM}{\left(\Omega,\mathcal{G},\prob\right)}
%\newcommand{\E4}{\left(\Omega,\mathcal{G},\mathcal{G}_{t},\prob\right)}
\begin{Def}
Un proceso estoc\'astico $\PE$ definido en $\left(\Omega,\mathcal{G},\prob\right)$ con valores en el espacio topol\'ogico $E$ es continuo por la derecha si cada trayectoria muestral $t\rightarrow X_{t}\left(w\right)$ es un mapeo continuo por la derecha de $I$ en $E$.
\end{Def}

\begin{Def}[HD1]\label{Eq.2.1.S}
Un semigrupo de Markov $\left/P_{t}\right)$ en un espacio de Rad\'on $E$ se dice que satisface la condici\'on {\em HD1} si, dada una medida de probabilidad $\mu$ en $E$, existe una $\sigma$-\'algebra $\mathcal{E^{*}}$ con $\mathcal{E}\subset\mathcal{E}$ y $P_{t}\left(b\mathcal{E}^{*}\right)\subset b\mathcal{E}^{*}$, y un $\mathcal{E}^{*}$-proceso $E$-valuado continuo por la derecha $\PE$ en alg\'un espacio de probabilidad filtrado $\left(\Omega,\mathcal{G},\mathcal{G}_{t},\prob\right)$ tal que $X=\left(\Omega,\mathcal{G},\mathcal{G}_{t},\prob\right)$ es de Markov (Homog\'eneo) con semigrupo de transici\'on $(P_{t})$ y distribuci\'on inicial $\mu$.
\end{Def}

Considerese la colecci\'on de variables aleatorias $X_{t}$ definidas en alg\'un espacio de probabilidad, y una colecci\'on de medidas $\mathbf{P}^{x}$ tales que $\mathbf{P}^{x}\left\{X_{0}=x\right\}$, y bajo cualquier $\mathbf{P}^{x}$, $X_{t}$ es de Markov con semigrupo $\left(P_{t}\right)$. $\mathbf{P}^{x}$ puede considerarse como la distribuci\'on condicional de $\mathbf{P}$ dado $X_{0}=x$.

\begin{Def}\label{Def.2.2.S}
Sea $E$ espacio de Rad\'on, $\SG$ semigrupo de Markov en $\ER$. La colecci\'on $\mathbf{X}=\left(\Omega,\mathcal{G},\mathcal{G}_{t},X_{t},\theta_{t},\CM\right)$ es un proceso $\mathcal{E}$-Markov continuo por la derecha simple, con espacio de estados $E$ y semigrupo de transici\'on $\SG$ en caso de que $\mathbf{X}$ satisfaga las siguientes condiciones:
\begin{itemize}
\item[i)] $\left(\Omega,\mathcal{G},\mathcal{G}_{t}\right)$ es un espacio de medida filtrado, y $X_{t}$ es un proceso $E$-valuado continuo por la derecha $\mathcal{E}^{*}$-adaptado a $\left(\mathcal{G}_{t}\right)$;

\item[ii)] $\left(\theta_{t}\right)_{t\geq0}$ es una colecci\'on de operadores {\em shift} para $X$, es decir, mapea $\Omega$ en s\'i mismo satisfaciendo para $t,s\geq0$,

\begin{equation}\label{Eq.Shift}
\theta_{t}\circ\theta_{s}=\theta_{t+s}\textrm{ y }X_{t}\circ\theta_{t}=X_{t+s};
\end{equation}

\item[iii)] Para cualquier $x\in E$,$\CM\left\{X_{0}=x\right\}=1$, y el proceso $\PE$ tiene la propiedad de Markov (\ref{Eq.1.8.S}) con semigrupo de transici\'on $\SG$ relativo a $\left(\Omega,\mathcal{G},\mathcal{G}_{t},\CM\right)$.
\end{itemize}
\end{Def}

\begin{Def}[HD2]\label{Eq.2.2.S}
Para cualquier $\alpha>0$ y cualquier $f\in S^{\alpha}$, el proceso $t\rightarrow f\left(X_{t}\right)$ es continuo por la derecha casi seguramente.
\end{Def}

\begin{Def}\label{Def.PD}
Un sistema $\mathbf{X}=\left(\Omega,\mathcal{G},\mathcal{G}_{t},X_{t},\theta_{t},\CM\right)$ es un proceso derecho en el espacio de Rad\'on $E$ con semigrupo de transici\'on $\SG$ provisto de:
\begin{itemize}
\item[i)] $\mathbf{X}$ es una realizaci\'on  continua por la derecha, \ref{Def.2.2.S}, de $\SG$.

\item[ii)] $\mathbf{X}$ satisface la condicion HD2, \ref{Eq.2.2.S}, relativa a $\mathcal{G}_{t}$.

\item[iii)] $\mathcal{G}_{t}$ es aumentado y continuo por la derecha.
\end{itemize}
\end{Def}




\begin{Lema}[Lema 4.2, Dai\cite{Dai}]\label{Lema4.2}
Sea $\left\{x_{n}\right\}\subset \mathbf{X}$ con
$|x_{n}|\rightarrow\infty$, conforme $n\rightarrow\infty$. Suponga
que
\[lim_{n\rightarrow\infty}\frac{1}{|x_{n}|}U\left(0\right)=\overline{U}\]
y
\[lim_{n\rightarrow\infty}\frac{1}{|x_{n}|}V\left(0\right)=\overline{V}.\]

Entonces, conforme $n\rightarrow\infty$, casi seguramente

\begin{equation}\label{E1.4.2}
\frac{1}{|x_{n}|}\Phi^{k}\left(\left[|x_{n}|t\right]\right)\rightarrow
P_{k}^{'}t\textrm{, u.o.c.,}
\end{equation}

\begin{equation}\label{E1.4.3}
\frac{1}{|x_{n}|}E^{x_{n}}_{k}\left(|x_{n}|t\right)\rightarrow
\alpha_{k}\left(t-\overline{U}_{k}\right)^{+}\textrm{, u.o.c.,}
\end{equation}

\begin{equation}\label{E1.4.4}
\frac{1}{|x_{n}|}S^{x_{n}}_{k}\left(|x_{n}|t\right)\rightarrow
\mu_{k}\left(t-\overline{V}_{k}\right)^{+}\textrm{, u.o.c.,}
\end{equation}

donde $\left[t\right]$ es la parte entera de $t$ y
$\mu_{k}=1/m_{k}=1/\esp\left[\eta_{k}\left(1\right)\right]$.
\end{Lema}

\begin{Lema}[Lema 4.3, Dai\cite{Dai}]\label{Lema.4.3}
Sea $\left\{x_{n}\right\}\subset \mathbf{X}$ con
$|x_{n}|\rightarrow\infty$, conforme $n\rightarrow\infty$. Suponga
que
\[lim_{n\rightarrow\infty}\frac{1}{|x_{n}|}U\left(0\right)=\overline{U}_{k}\]
y
\[lim_{n\rightarrow\infty}\frac{1}{|x_{n}|}V\left(0\right)=\overline{V}_{k}.\]
\begin{itemize}
\item[a)] Conforme $n\rightarrow\infty$ casi seguramente,
\[lim_{n\rightarrow\infty}\frac{1}{|x_{n}|}U^{x_{n}}_{k}\left(|x_{n}|t\right)=\left(\overline{U}_{k}-t\right)^{+}\textrm{, u.o.c.}\]
y
\[lim_{n\rightarrow\infty}\frac{1}{|x_{n}|}V^{x_{n}}_{k}\left(|x_{n}|t\right)=\left(\overline{V}_{k}-t\right)^{+}.\]

\item[b)] Para cada $t\geq0$ fijo,
\[\left\{\frac{1}{|x_{n}|}U^{x_{n}}_{k}\left(|x_{n}|t\right),|x_{n}|\geq1\right\}\]
y
\[\left\{\frac{1}{|x_{n}|}V^{x_{n}}_{k}\left(|x_{n}|t\right),|x_{n}|\geq1\right\}\]
\end{itemize}
son uniformemente convergentes.
\end{Lema}

$S_{l}^{x}\left(t\right)$ es el n\'umero total de servicios
completados de la clase $l$, si la clase $l$ est\'a dando $t$
unidades de tiempo de servicio. Sea $T_{l}^{x}\left(x\right)$ el
monto acumulado del tiempo de servicio que el servidor
$s\left(l\right)$ gasta en los usuarios de la clase $l$ al tiempo
$t$. Entonces $S_{l}^{x}\left(T_{l}^{x}\left(t\right)\right)$ es
el n\'umero total de servicios completados para la clase $l$ al
tiempo $t$. Una fracci\'on de estos usuarios,
$\Phi_{l}^{x}\left(S_{l}^{x}\left(T_{l}^{x}\left(t\right)\right)\right)$,
se convierte en usuarios de la clase $k$.\\

Entonces, dado lo anterior, se tiene la siguiente representaci\'on
para el proceso de la longitud de la cola:\\

\begin{equation}
Q_{k}^{x}\left(t\right)=_{k}^{x}\left(0\right)+E_{k}^{x}\left(t\right)+\sum_{l=1}^{K}\Phi_{k}^{l}\left(S_{l}^{x}\left(T_{l}^{x}\left(t\right)\right)\right)-S_{k}^{x}\left(T_{k}^{x}\left(t\right)\right)
\end{equation}
para $k=1,\ldots,K$. Para $i=1,\ldots,d$, sea
\[I_{i}^{x}\left(t\right)=t-\sum_{j\in C_{i}}T_{k}^{x}\left(t\right).\]

Entonces $I_{i}^{x}\left(t\right)$ es el monto acumulado del
tiempo que el servidor $i$ ha estado desocupado al tiempo $t$. Se
est\'a asumiendo que las disciplinas satisfacen la ley de
conservaci\'on del trabajo, es decir, el servidor $i$ est\'a en
pausa solamente cuando no hay usuarios en la estaci\'on $i$.
Entonces, se tiene que

\begin{equation}
\int_{0}^{\infty}\left(\sum_{k\in
C_{i}}Q_{k}^{x}\left(t\right)\right)dI_{i}^{x}\left(t\right)=0,
\end{equation}
para $i=1,\ldots,d$.\\

Hacer
\[T^{x}\left(t\right)=\left(T_{1}^{x}\left(t\right),\ldots,T_{K}^{x}\left(t\right)\right)^{'},\]
\[I^{x}\left(t\right)=\left(I_{1}^{x}\left(t\right),\ldots,I_{K}^{x}\left(t\right)\right)^{'}\]
y
\[S^{x}\left(T^{x}\left(t\right)\right)=\left(S_{1}^{x}\left(T_{1}^{x}\left(t\right)\right),\ldots,S_{K}^{x}\left(T_{K}^{x}\left(t\right)\right)\right)^{'}.\]

Para una disciplina que cumple con la ley de conservaci\'on del
trabajo, en forma vectorial, se tiene el siguiente conjunto de
ecuaciones

\begin{equation}\label{Eq.MF.1.3}
Q^{x}\left(t\right)=Q^{x}\left(0\right)+E^{x}\left(t\right)+\sum_{l=1}^{K}\Phi^{l}\left(S_{l}^{x}\left(T_{l}^{x}\left(t\right)\right)\right)-S^{x}\left(T^{x}\left(t\right)\right),\\
\end{equation}

\begin{equation}\label{Eq.MF.2.3}
Q^{x}\left(t\right)\geq0,\\
\end{equation}

\begin{equation}\label{Eq.MF.3.3}
T^{x}\left(0\right)=0,\textrm{ y }\overline{T}^{x}\left(t\right)\textrm{ es no decreciente},\\
\end{equation}

\begin{equation}\label{Eq.MF.4.3}
I^{x}\left(t\right)=et-CT^{x}\left(t\right)\textrm{ es no
decreciente}\\
\end{equation}

\begin{equation}\label{Eq.MF.5.3}
\int_{0}^{\infty}\left(CQ^{x}\left(t\right)\right)dI_{i}^{x}\left(t\right)=0,\\
\end{equation}

\begin{equation}\label{Eq.MF.6.3}
\textrm{Condiciones adicionales en
}\left(\overline{Q}^{x}\left(\cdot\right),\overline{T}^{x}\left(\cdot\right)\right)\textrm{
espec\'ificas de la disciplina de la cola,}
\end{equation}

donde $e$ es un vector de unos de dimensi\'on $d$, $C$ es la
matriz definida por
\[C_{ik}=\left\{\begin{array}{cc}
1,& S\left(k\right)=i,\\
0,& \textrm{ en otro caso}.\\
\end{array}\right.
\]
Es necesario enunciar el siguiente Teorema que se utilizar\'a para
el Teorema \ref{Tma.4.2.Dai}:
\begin{Teo}[Teorema 4.1, Dai \cite{Dai}]
Considere una disciplina que cumpla la ley de conservaci\'on del
trabajo, para casi todas las trayectorias muestrales $\omega$ y
cualquier sucesi\'on de estados iniciales
$\left\{x_{n}\right\}\subset \mathbf{X}$, con
$|x_{n}|\rightarrow\infty$, existe una subsucesi\'on
$\left\{x_{n_{j}}\right\}$ con $|x_{n_{j}}|\rightarrow\infty$ tal
que
\begin{equation}\label{Eq.4.15}
\frac{1}{|x_{n_{j}}|}\left(Q^{x_{n_{j}}}\left(0\right),U^{x_{n_{j}}}\left(0\right),V^{x_{n_{j}}}\left(0\right)\right)\rightarrow\left(\overline{Q}\left(0\right),\overline{U},\overline{V}\right),
\end{equation}

\begin{equation}\label{Eq.4.16}
\frac{1}{|x_{n_{j}}|}\left(Q^{x_{n_{j}}}\left(|x_{n_{j}}|t\right),T^{x_{n_{j}}}\left(|x_{n_{j}}|t\right)\right)\rightarrow\left(\overline{Q}\left(t\right),\overline{T}\left(t\right)\right)\textrm{
u.o.c.}
\end{equation}

Adem\'as,
$\left(\overline{Q}\left(t\right),\overline{T}\left(t\right)\right)$
satisface las siguientes ecuaciones:
\begin{equation}\label{Eq.MF.1.3a}
\overline{Q}\left(t\right)=Q\left(0\right)+\left(\alpha
t-\overline{U}\right)^{+}-\left(I-P\right)^{'}M^{-1}\left(\overline{T}\left(t\right)-\overline{V}\right)^{+},
\end{equation}

\begin{equation}\label{Eq.MF.2.3a}
\overline{Q}\left(t\right)\geq0,\\
\end{equation}

\begin{equation}\label{Eq.MF.3.3a}
\overline{T}\left(t\right)\textrm{ es no decreciente y comienza en cero},\\
\end{equation}

\begin{equation}\label{Eq.MF.4.3a}
\overline{I}\left(t\right)=et-C\overline{T}\left(t\right)\textrm{
es no decreciente,}\\
\end{equation}

\begin{equation}\label{Eq.MF.5.3a}
\int_{0}^{\infty}\left(C\overline{Q}\left(t\right)\right)d\overline{I}\left(t\right)=0,\\
\end{equation}

\begin{equation}\label{Eq.MF.6.3a}
\textrm{Condiciones adicionales en
}\left(\overline{Q}\left(\cdot\right),\overline{T}\left(\cdot\right)\right)\textrm{
especficas de la disciplina de la cola,}
\end{equation}
\end{Teo}

\begin{Def}[Definici\'on 4.1, , Dai \cite{Dai}]
Sea una disciplina de servicio espec\'ifica. Cualquier l\'imite
$\left(\overline{Q}\left(\cdot\right),\overline{T}\left(\cdot\right)\right)$
en \ref{Eq.4.16} es un {\em flujo l\'imite} de la disciplina.
Cualquier soluci\'on (\ref{Eq.MF.1.3a})-(\ref{Eq.MF.6.3a}) es
llamado flujo soluci\'on de la disciplina. Se dice que el modelo de flujo l\'imite, modelo de flujo, de la disciplina de la cola es estable si existe una constante
$\delta>0$ que depende de $\mu,\alpha$ y $P$ solamente, tal que
cualquier flujo l\'imite con
$|\overline{Q}\left(0\right)|+|\overline{U}|+|\overline{V}|=1$, se
tiene que $\overline{Q}\left(\cdot+\delta\right)\equiv0$.
\end{Def}

\begin{Teo}[Teorema 4.2, Dai\cite{Dai}]\label{Tma.4.2.Dai}
Sea una disciplina fija para la cola, suponga que se cumplen las
condiciones (1.2)-(1.5). Si el modelo de flujo l\'imite de la
disciplina de la cola es estable, entonces la cadena de Markov $X$
que describe la din\'amica de la red bajo la disciplina es Harris
recurrente positiva.
\end{Teo}

Ahora se procede a escalar el espacio y el tiempo para reducir la
aparente fluctuaci\'on del modelo. Consid\'erese el proceso
\begin{equation}\label{Eq.3.7}
\overline{Q}^{x}\left(t\right)=\frac{1}{|x|}Q^{x}\left(|x|t\right)
\end{equation}
A este proceso se le conoce como el fluido escalado, y cualquier l\'imite $\overline{Q}^{x}\left(t\right)$ es llamado flujo l\'imite del proceso de longitud de la cola. Haciendo $|q|\rightarrow\infty$ mientras se mantiene el resto de las componentes fijas, cualquier punto l\'imite del proceso de longitud de la cola normalizado $\overline{Q}^{x}$ es soluci\'on del siguiente modelo de flujo.

Al conjunto de ecuaciones dadas en \ref{Eq.3.8}-\ref{Eq.3.13} se
le llama {\em Modelo de flujo} y al conjunto de todas las
soluciones del modelo de flujo
$\left(\overline{Q}\left(\cdot\right),\overline{T}
\left(\cdot\right)\right)$ se le denotar\'a por $\mathcal{Q}$.

Si se hace $|x|\rightarrow\infty$ sin restringir ninguna de las
componentes, tambi\'en se obtienen un modelo de flujo, pero en
este caso el residual de los procesos de arribo y servicio
introducen un retraso:

\begin{Def}[Definici\'on 3.3, Dai y Meyn \cite{DaiSean}]
El modelo de flujo es estable si existe un tiempo fijo $t_{0}$ tal
que $\overline{Q}\left(t\right)=0$, con $t\geq t_{0}$, para
cualquier $\overline{Q}\left(\cdot\right)\in\mathcal{Q}$ que
cumple con $|\overline{Q}\left(0\right)|=1$.
\end{Def}

El siguiente resultado se encuentra en Chen \cite{Chen}.
\begin{Lemma}[Lema 3.1, Dai y Meyn \cite{DaiSean}]
Si el modelo de flujo definido por \ref{Eq.3.8}-\ref{Eq.3.13} es
estable, entonces el modelo de flujo retrasado es tambi\'en
estable, es decir, existe $t_{0}>0$ tal que
$\overline{Q}\left(t\right)=0$ para cualquier $t\geq t_{0}$, para
cualquier soluci\'on del modelo de flujo retrasado cuya
condici\'on inicial $\overline{x}$ satisface que
$|\overline{x}|=|\overline{Q}\left(0\right)|+|\overline{A}\left(0\right)|+|\overline{B}\left(0\right)|\leq1$.
\end{Lemma}


Propiedades importantes para el modelo de flujo retrasado:

\begin{Prop}
 Sea $\left(\overline{Q},\overline{T},\overline{T}^{0}\right)$ un flujo l\'imite de \ref{Eq.4.4} y suponga que cuando $x\rightarrow\infty$ a lo largo de
una subsucesi\'on
\[\left(\frac{1}{|x|}Q_{k}^{x}\left(0\right),\frac{1}{|x|}A_{k}^{x}\left(0\right),\frac{1}{|x|}B_{k}^{x}\left(0\right),\frac{1}{|x|}B_{k}^{x,0}\left(0\right)\right)\rightarrow\left(\overline{Q}_{k}\left(0\right),0,0,0\right)\]
para $k=1,\ldots,K$. EL flujo l\'imite tiene las siguientes
propiedades, donde las propiedades de la derivada se cumplen donde
la derivada exista:
\begin{itemize}
 \item[i)] Los vectores de tiempo ocupado $\overline{T}\left(t\right)$ y $\overline{T}^{0}\left(t\right)$ son crecientes y continuas con
$\overline{T}\left(0\right)=\overline{T}^{0}\left(0\right)=0$.
\item[ii)] Para todo $t\geq0$
\[\sum_{k=1}^{K}\left[\overline{T}_{k}\left(t\right)+\overline{T}_{k}^{0}\left(t\right)\right]=t\]
\item[iii)] Para todo $1\leq k\leq K$
\[\overline{Q}_{k}\left(t\right)=\overline{Q}_{k}\left(0\right)+\alpha_{k}t-\mu_{k}\overline{T}_{k}\left(t\right)\]
\item[iv)]  Para todo $1\leq k\leq K$
\[\dot{{\overline{T}}}_{k}\left(t\right)=\beta_{k}\] para $\overline{Q}_{k}\left(t\right)=0$.
\item[v)] Para todo $k,j$
\[\mu_{k}^{0}\overline{T}_{k}^{0}\left(t\right)=\mu_{j}^{0}\overline{T}_{j}^{0}\left(t\right)\]
\item[vi)]  Para todo $1\leq k\leq K$
\[\mu_{k}\dot{{\overline{T}}}_{k}\left(t\right)=l_{k}\mu_{k}^{0}\dot{{\overline{T}}}_{k}^{0}\left(t\right)\] para $\overline{Q}_{k}\left(t\right)>0$.
\end{itemize}
\end{Prop}

\begin{Lema}[Lema 3.1 \cite{Chen}]\label{Lema3.1}
Si el modelo de flujo es estable, definido por las ecuaciones
(3.8)-(3.13), entonces el modelo de flujo retrasado tambin es
estable.
\end{Lema}

\begin{Teo}[Teorema 5.2 \cite{Chen}]\label{Tma.5.2}
Si el modelo de flujo lineal correspondiente a la red de cola es
estable, entonces la red de colas es estable.
\end{Teo}

\begin{Teo}[Teorema 5.1 \cite{Chen}]\label{Tma.5.1.Chen}
La red de colas es estable si existe una constante $t_{0}$ que
depende de $\left(\alpha,\mu,T,U\right)$ y $V$ que satisfagan las
ecuaciones (5.1)-(5.5), $Z\left(t\right)=0$, para toda $t\geq
t_{0}$.
\end{Teo}



\begin{Lema}[Lema 5.2 \cite{Gut}]\label{Lema.5.2.Gut}
Sea $\left\{\xi\left(k\right):k\in\ent\right\}$ sucesin de
variables aleatorias i.i.d. con valores en
$\left(0,\infty\right)$, y sea $E\left(t\right)$ el proceso de
conteo
\[E\left(t\right)=max\left\{n\geq1:\xi\left(1\right)+\cdots+\xi\left(n-1\right)\leq t\right\}.\]
Si $E\left[\xi\left(1\right)\right]<\infty$, entonces para
cualquier entero $r\geq1$
\begin{equation}
lim_{t\rightarrow\infty}\esp\left[\left(\frac{E\left(t\right)}{t}\right)^{r}\right]=\left(\frac{1}{E\left[\xi_{1}\right]}\right)^{r}
\end{equation}
de aqu, bajo estas condiciones
\begin{itemize}
\item[a)] Para cualquier $t>0$,
$sup_{t\geq\delta}\esp\left[\left(\frac{E\left(t\right)}{t}\right)^{r}\right]$

\item[b)] Las variables aleatorias
$\left\{\left(\frac{E\left(t\right)}{t}\right)^{r}:t\geq1\right\}$
son uniformemente integrables.
\end{itemize}
\end{Lema}

\begin{Teo}[Teorema 5.1: Ley Fuerte para Procesos de Conteo
\cite{Gut}]\label{Tma.5.1.Gut} Sea
$0<\mu<\esp\left(X_{1}\right]\leq\infty$. entonces

\begin{itemize}
\item[a)] $\frac{N\left(t\right)}{t}\rightarrow\frac{1}{\mu}$
a.s., cuando $t\rightarrow\infty$.


\item[b)]$\esp\left[\frac{N\left(t\right)}{t}\right]^{r}\rightarrow\frac{1}{\mu^{r}}$,
cuando $t\rightarrow\infty$ para todo $r>0$..
\end{itemize}
\end{Teo}


\begin{Prop}[Proposicin 5.1 \cite{DaiSean}]\label{Prop.5.1}
Suponga que los supuestos (A1) y (A2) se cumplen, adems suponga
que el modelo de flujo es estable. Entonces existe $t_{0}>0$ tal
que
\begin{equation}\label{Eq.Prop.5.1}
lim_{|x|\rightarrow\infty}\frac{1}{|x|^{p+1}}\esp_{x}\left[|X\left(t_{0}|x|\right)|^{p+1}\right]=0.
\end{equation}

\end{Prop}


\begin{Prop}[Proposici\'on 5.3 \cite{DaiSean}]
Sea $X$ proceso de estados para la red de colas, y suponga que se
cumplen los supuestos (A1) y (A2), entonces para alguna constante
positiva $C_{p+1}<\infty$, $\delta>0$ y un conjunto compacto
$C\subset X$.

\begin{equation}\label{Eq.5.4}
\esp_{x}\left[\int_{0}^{\tau_{C}\left(\delta\right)}\left(1+|X\left(t\right)|^{p}\right)dt\right]\leq
C_{p+1}\left(1+|x|^{p+1}\right)
\end{equation}
\end{Prop}

\begin{Prop}[Proposici\'on 5.4 \cite{DaiSean}]
Sea $X$ un proceso de Markov Borel Derecho en $X$, sea
$f:X\leftarrow\rea_{+}$ y defina para alguna $\delta>0$, y un
conjunto cerrado $C\subset X$
\[V\left(x\right):=\esp_{x}\left[\int_{0}^{\tau_{C}\left(\delta\right)}f\left(X\left(t\right)\right)dt\right]\]
para $x\in X$. Si $V$ es finito en todas partes y uniformemente
acotada en $C$, entonces existe $k<\infty$ tal que
\begin{equation}\label{Eq.5.11}
\frac{1}{t}\esp_{x}\left[V\left(x\right)\right]+\frac{1}{t}\int_{0}^{t}\esp_{x}\left[f\left(X\left(s\right)\right)ds\right]\leq\frac{1}{t}V\left(x\right)+k,
\end{equation}
para $x\in X$ y $t>0$.
\end{Prop}


\begin{Teo}[Teorema 5.5 \cite{DaiSean}]
Suponga que se cumplen (A1) y (A2), adems suponga que el modelo
de flujo es estable. Entonces existe una constante $k_{p}<\infty$
tal que
\begin{equation}\label{Eq.5.13}
\frac{1}{t}\int_{0}^{t}\esp_{x}\left[|Q\left(s\right)|^{p}\right]ds\leq
k_{p}\left\{\frac{1}{t}|x|^{p+1}+1\right\}
\end{equation}
para $t\geq0$, $x\in X$. En particular para cada condici\'on inicial
\begin{equation}\label{Eq.5.14}
Limsup_{t\rightarrow\infty}\frac{1}{t}\int_{0}^{t}\esp_{x}\left[|Q\left(s\right)|^{p}\right]ds\leq
k_{p}
\end{equation}
\end{Teo}

\begin{Teo}[Teorema 6.2\cite{DaiSean}]\label{Tma.6.2}
Suponga que se cumplen los supuestos (A1)-(A3) y que el modelo de
flujo es estable, entonces se tiene que
\[\parallel P^{t}\left(c,\cdot\right)-\pi\left(\cdot\right)\parallel_{f_{p}}\rightarrow0\]
para $t\rightarrow\infty$ y $x\in X$. En particular para cada
condicin inicial
\[lim_{t\rightarrow\infty}\esp_{x}\left[\left|Q_{t}\right|^{p}\right]=\esp_{\pi}\left[\left|Q_{0}\right|^{p}\right]<\infty\]
\end{Teo}


\begin{Teo}[Teorema 6.3\cite{DaiSean}]\label{Tma.6.3}
Suponga que se cumplen los supuestos (A1)-(A3) y que el modelo de
flujo es estable, entonces con
$f\left(x\right)=f_{1}\left(x\right)$, se tiene que
\[lim_{t\rightarrow\infty}t^{(p-1)\left|P^{t}\left(c,\cdot\right)-\pi\left(\cdot\right)\right|_{f}=0},\]
para $x\in X$. En particular, para cada condicin inicial
\[lim_{t\rightarrow\infty}t^{(p-1)\left|\esp_{x}\left[Q_{t}\right]-\esp_{\pi}\left[Q_{0}\right]\right|=0}.\]
\end{Teo}


\begin{Prop}[Proposici\'on 5.1, Dai y Meyn \cite{DaiSean}]\label{Prop.5.1.DaiSean}
Suponga que los supuestos A1) y A2) son ciertos y que el modelo de flujo es estable. Entonces existe $t_{0}>0$ tal que
\begin{equation}
lim_{|x|\rightarrow\infty}\frac{1}{|x|^{p+1}}\esp_{x}\left[|X\left(t_{0}|x|\right)|^{p+1}\right]=0
\end{equation}
\end{Prop}

\begin{Lemma}[Lema 5.2, Dai y Meyn \cite{DaiSean}]\label{Lema.5.2.DaiSean}
 Sea $\left\{\zeta\left(k\right):k\in \mathbb{z}\right\}$ una sucesi\'on independiente e id\'enticamente distribuida que toma valores en $\left(0,\infty\right)$,
y sea
$E\left(t\right)=max\left(n\geq1:\zeta\left(1\right)+\cdots+\zeta\left(n-1\right)\leq
t\right)$. Si $\esp\left[\zeta\left(1\right)\right]<\infty$,
entonces para cualquier entero $r\geq1$
\begin{equation}
 lim_{t\rightarrow\infty}\esp\left[\left(\frac{E\left(t\right)}{t}\right)^{r}\right]=\left(\frac{1}{\esp\left[\zeta_{1}\right]}\right)^{r}.
\end{equation}
Luego, bajo estas condiciones:
\begin{itemize}
 \item[a)] para cualquier $\delta>0$, $\sup_{t\geq\delta}\esp\left[\left(\frac{E\left(t\right)}{t}\right)^{r}\right]<\infty$
\item[b)] las variables aleatorias
$\left\{\left(\frac{E\left(t\right)}{t}\right)^{r}:t\geq1\right\}$
son uniformemente integrables.
\end{itemize}
\end{Lemma}

\begin{Teo}[Teorema 5.5, Dai y Meyn \cite{DaiSean}]\label{Tma.5.5.DaiSean}
Suponga que los supuestos A1) y A2) se cumplen y que el modelo de
flujo es estable. Entonces existe una constante $\kappa_{p}$ tal
que
\begin{equation}
\frac{1}{t}\int_{0}^{t}\esp_{x}\left[|Q\left(s\right)|^{p}\right]ds\leq\kappa_{p}\left\{\frac{1}{t}|x|^{p+1}+1\right\}
\end{equation}
para $t>0$ y $x\in X$. En particular, para cada condici\'on
inicial
\begin{eqnarray*}
\limsup_{t\rightarrow\infty}\frac{1}{t}\int_{0}^{t}\esp_{x}\left[|Q\left(s\right)|^{p}\right]ds\leq\kappa_{p}.
\end{eqnarray*}
\end{Teo}

\begin{Teo}[Teorema 6.2, Dai y Meyn \cite{DaiSean}]\label{Tma.6.2.DaiSean}
Suponga que se cumplen los supuestos A1), A2) y A3) y que el
modelo de flujo es estable. Entonces se tiene que
\begin{equation}
\left\|P^{t}\left(x,\cdot\right)-\pi\left(\cdot\right)\right\|_{f_{p}}\textrm{,
}t\rightarrow\infty,x\in X.
\end{equation}
En particular para cada condici\'on inicial
\begin{eqnarray*}
\lim_{t\rightarrow\infty}\esp_{x}\left[|Q\left(t\right)|^{p}\right]=\esp_{\pi}\left[|Q\left(0\right)|^{p}\right]\leq\kappa_{r}
\end{eqnarray*}
\end{Teo}
\begin{Teo}[Teorema 6.3, Dai y Meyn \cite{DaiSean}]\label{Tma.6.3.DaiSean}
Suponga que se cumplen los supuestos A1), A2) y A3) y que el
modelo de flujo es estable. Entonces con
$f\left(x\right)=f_{1}\left(x\right)$ se tiene
\begin{equation}
\lim_{t\rightarrow\infty}t^{p-1}\left\|P^{t}\left(x,\cdot\right)-\pi\left(\cdot\right)\right\|_{f}=0.
\end{equation}
En particular para cada condici\'on inicial
\begin{eqnarray*}
\lim_{t\rightarrow\infty}t^{p-1}|\esp_{x}\left[Q\left(t\right)\right]-\esp_{\pi}\left[Q\left(0\right)\right]|=0.
\end{eqnarray*}
\end{Teo}

\begin{Teo}[Teorema 6.4, Dai y Meyn \cite{DaiSean}]\label{Tma.6.4.DaiSean}
Suponga que se cumplen los supuestos A1), A2) y A3) y que el
modelo de flujo es estable. Sea $\nu$ cualquier distribuci\'on de
probabilidad en $\left(X,\mathcal{B}_{X}\right)$, y $\pi$ la
distribuci\'on estacionaria de $X$.
\begin{itemize}
\item[i)] Para cualquier $f:X\leftarrow\rea_{+}$
\begin{equation}
\lim_{t\rightarrow\infty}\frac{1}{t}\int_{o}^{t}f\left(X\left(s\right)\right)ds=\pi\left(f\right):=\int
f\left(x\right)\pi\left(dx\right)
\end{equation}
$\prob$-c.s.

\item[ii)] Para cualquier $f:X\leftarrow\rea_{+}$ con
$\pi\left(|f|\right)<\infty$, la ecuaci\'on anterior se cumple.
\end{itemize}
\end{Teo}

\begin{Teo}[Teorema 2.2, Down \cite{Down}]\label{Tma2.2.Down}
Suponga que el fluido modelo es inestable en el sentido de que
para alguna $\epsilon_{0},c_{0}\geq0$,
\begin{equation}\label{Eq.Inestability}
|Q\left(T\right)|\geq\epsilon_{0}T-c_{0}\textrm{,   }T\geq0,
\end{equation}
para cualquier condici\'on inicial $Q\left(0\right)$, con
$|Q\left(0\right)|=1$. Entonces para cualquier $0<q\leq1$, existe
$B<0$ tal que para cualquier $|x|\geq B$,
\begin{equation}
\prob_{x}\left\{\mathbb{X}\rightarrow\infty\right\}\geq q.
\end{equation}
\end{Teo}



Es necesario hacer los siguientes supuestos sobre el
comportamiento del sistema de visitas c\'iclicas:
\begin{itemize}
\item Los tiempos de interarribo a la $k$-\'esima cola, son de la
forma $\left\{\xi_{k}\left(n\right)\right\}_{n\geq1}$, con la
propiedad de que son independientes e id{\'e}nticamente
distribuidos,
\item Los tiempos de servicio
$\left\{\eta_{k}\left(n\right)\right\}_{n\geq1}$ tienen la
propiedad de ser independientes e id{\'e}nticamente distribuidos,
\item Se define la tasa de arribo a la $k$-{\'e}sima cola como
$\lambda_{k}=1/\esp\left[\xi_{k}\left(1\right)\right]$,
\item la tasa de servicio para la $k$-{\'e}sima cola se define
como $\mu_{k}=1/\esp\left[\eta_{k}\left(1\right)\right]$,
\item tambi{\'e}n se define $\rho_{k}:=\lambda_{k}/\mu_{k}$, la
intensidad de tr\'afico del sistema o carga de la red, donde es
necesario que $\rho<1$ para cuestiones de estabilidad.
\end{itemize}



%_________________________________________________________________________
\subsection{Procesos de Estados Markoviano para el Sistema}
%_________________________________________________________________________

%_________________________________________________________________________
\subsection{Procesos Fuerte de Markov}
%_________________________________________________________________________
En Dai \cite{Dai} se muestra que para una amplia serie de disciplinas
de servicio el proceso $X$ es un Proceso Fuerte de
Markov, y por tanto se puede asumir que


Para establecer que $X=\left\{X\left(t\right),t\geq0\right\}$ es
un Proceso Fuerte de Markov, se siguen las secciones 2.3 y 2.4 de Kaspi and Mandelbaum \cite{KaspiMandelbaum}. \\

%______________________________________________________________
\subsubsection{Construcci\'on de un Proceso Determinista por partes, Davis
\cite{Davis}}.
%______________________________________________________________

%_________________________________________________________________________
\subsection{Procesos Harris Recurrentes Positivos}
%_________________________________________________________________________
Sea el proceso de Markov $X=\left\{X\left(t\right),t\geq0\right\}$
que describe la din\'amica de la red de colas. En lo que respecta
al supuesto (A3), en Dai y Meyn \cite{DaiSean} y Meyn y Down
\cite{MeynDown} hacen ver que este se puede sustituir por

\begin{itemize}
\item[A3')] Para el Proceso de Markov $X$, cada subconjunto
compacto de $X$ es un conjunto peque\~no.
\end{itemize}

Este supuesto es importante pues es un requisito para deducir la ergodicidad de la red.

%_________________________________________________________________________
\subsection{Construcci\'on de un Modelo de Flujo L\'imite}
%_________________________________________________________________________

Consideremos un caso m\'as simple para poner en contexto lo
anterior: para un sistema de visitas c\'iclicas se tiene que el
estado al tiempo $t$ es
\begin{equation}
X\left(t\right)=\left(Q\left(t\right),U\left(t\right),V\left(t\right)\right),
\end{equation}

donde $Q\left(t\right)$ es el n\'umero de usuarios formados en
cada estaci\'on. $U\left(t\right)$ es el tiempo restante antes de
que la siguiente clase $k$ de usuarios lleguen desde fuera del
sistema, $V\left(t\right)$ es el tiempo restante de servicio para
la clase $k$ de usuarios que est\'an siendo atendidos. Tanto
$U\left(t\right)$ como $V\left(t\right)$ se puede asumir que son
continuas por la derecha.

Sea
$x=\left(Q\left(0\right),U\left(0\right),V\left(0\right)\right)=\left(q,a,b\right)$,
el estado inicial de la red bajo una disciplina espec\'ifica para
la cola. Para $l\in\mathcal{E}$, donde $\mathcal{E}$ es el conjunto de clases de arribos externos, y $k=1,\ldots,K$ se define\\
\begin{eqnarray*}
E_{l}^{x}\left(t\right)&=&max\left\{r:U_{l}\left(0\right)+\xi_{l}\left(1\right)+\cdots+\xi_{l}\left(r-1\right)\leq
t\right\}\textrm{   }t\geq0,\\
S_{k}^{x}\left(t\right)&=&max\left\{r:V_{k}\left(0\right)+\eta_{k}\left(1\right)+\cdots+\eta_{k}\left(r-1\right)\leq
t\right\}\textrm{   }t\geq0.
\end{eqnarray*}

Para cada $k$ y cada $n$ se define

\begin{eqnarray*}\label{Eq.phi}
\Phi^{k}\left(n\right):=\sum_{i=1}^{n}\phi^{k}\left(i\right).
\end{eqnarray*}

donde $\phi^{k}\left(n\right)$ se define como el vector de ruta
para el $n$-\'esimo usuario de la clase $k$ que termina en la
estaci\'on $s\left(k\right)$, la $s$-\'eima componente de
$\phi^{k}\left(n\right)$ es uno si estos usuarios se convierten en
usuarios de la clase $l$ y cero en otro caso, por lo tanto
$\phi^{k}\left(n\right)$ es un vector {\em Bernoulli} de
dimensi\'on $K$ con par\'ametro $P_{k}^{'}$, donde $P_{k}$ denota
el $k$-\'esimo rengl\'on de $P=\left(P_{kl}\right)$.

Se asume que cada para cada $k$ la sucesi\'on $\phi^{k}\left(n\right)=\left\{\phi^{k}\left(n\right),n\geq1\right\}$
es independiente e id\'enticamente distribuida y que las
$\phi^{1}\left(n\right),\ldots,\phi^{K}\left(n\right)$ son
mutuamente independientes, adem\'as de independientes de los
procesos de arribo y de servicio.\\

\begin{Lema}[Lema 4.2, Dai\cite{Dai}]\label{Lema4.2}
Sea $\left\{x_{n}\right\}\subset \mathbf{X}$ con
$|x_{n}|\rightarrow\infty$, conforme $n\rightarrow\infty$. Suponga
que
\[lim_{n\rightarrow\infty}\frac{1}{|x_{n}|}U\left(0\right)=\overline{U}\]
y
\[lim_{n\rightarrow\infty}\frac{1}{|x_{n}|}V\left(0\right)=\overline{V}.\]

Entonces, conforme $n\rightarrow\infty$, casi seguramente

\begin{equation}\label{E1.4.2}
\frac{1}{|x_{n}|}\Phi^{k}\left(\left[|x_{n}|t\right]\right)\rightarrow
P_{k}^{'}t\textrm{, u.o.c.,}
\end{equation}

\begin{equation}\label{E1.4.3}
\frac{1}{|x_{n}|}E^{x_{n}}_{k}\left(|x_{n}|t\right)\rightarrow
\alpha_{k}\left(t-\overline{U}_{k}\right)^{+}\textrm{, u.o.c.,}
\end{equation}

\begin{equation}\label{E1.4.4}
\frac{1}{|x_{n}|}S^{x_{n}}_{k}\left(|x_{n}|t\right)\rightarrow
\mu_{k}\left(t-\overline{V}_{k}\right)^{+}\textrm{, u.o.c.,}
\end{equation}

donde $\left[t\right]$ es la parte entera de $t$ y
$\mu_{k}=1/m_{k}=1/\esp\left[\eta_{k}\left(1\right)\right]$.
\end{Lema}

\begin{Lema}[Lema 4.3, Dai\cite{Dai}]\label{Lema.4.3}
Sea $\left\{x_{n}\right\}\subset \mathbf{X}$ con
$|x_{n}|\rightarrow\infty$, conforme $n\rightarrow\infty$. Suponga
que
\[lim_{n\rightarrow\infty}\frac{1}{|x_{n}|}U\left(0\right)=\overline{U}_{k}\]
y
\[lim_{n\rightarrow\infty}\frac{1}{|x_{n}|}V\left(0\right)=\overline{V}_{k}.\]
\begin{itemize}
\item[a)] Conforme $n\rightarrow\infty$ casi seguramente,
\[lim_{n\rightarrow\infty}\frac{1}{|x_{n}|}U^{x_{n}}_{k}\left(|x_{n}|t\right)=\left(\overline{U}_{k}-t\right)^{+}\textrm{, u.o.c.}\]
y
\[lim_{n\rightarrow\infty}\frac{1}{|x_{n}|}V^{x_{n}}_{k}\left(|x_{n}|t\right)=\left(\overline{V}_{k}-t\right)^{+}.\]

\item[b)] Para cada $t\geq0$ fijo,
\[\left\{\frac{1}{|x_{n}|}U^{x_{n}}_{k}\left(|x_{n}|t\right),|x_{n}|\geq1\right\}\]
y
\[\left\{\frac{1}{|x_{n}|}V^{x_{n}}_{k}\left(|x_{n}|t\right),|x_{n}|\geq1\right\}\]
\end{itemize}
son uniformemente convergentes.
\end{Lema}

$S_{l}^{x}\left(t\right)$ es el n\'umero total de servicios
completados de la clase $l$, si la clase $l$ est\'a dando $t$
unidades de tiempo de servicio. Sea $T_{l}^{x}\left(x\right)$ el
monto acumulado del tiempo de servicio que el servidor
$s\left(l\right)$ gasta en los usuarios de la clase $l$ al tiempo
$t$. Entonces $S_{l}^{x}\left(T_{l}^{x}\left(t\right)\right)$ es
el n\'umero total de servicios completados para la clase $l$ al
tiempo $t$. Una fracci\'on de estos usuarios,
$\Phi_{l}^{x}\left(S_{l}^{x}\left(T_{l}^{x}\left(t\right)\right)\right)$,
se convierte en usuarios de la clase $k$.\\

Entonces, dado lo anterior, se tiene la siguiente representaci\'on
para el proceso de la longitud de la cola:\\

\begin{equation}
Q_{k}^{x}\left(t\right)=_{k}^{x}\left(0\right)+E_{k}^{x}\left(t\right)+\sum_{l=1}^{K}\Phi_{k}^{l}\left(S_{l}^{x}\left(T_{l}^{x}\left(t\right)\right)\right)-S_{k}^{x}\left(T_{k}^{x}\left(t\right)\right)
\end{equation}
para $k=1,\ldots,K$. Para $i=1,\ldots,d$, sea
\[I_{i}^{x}\left(t\right)=t-\sum_{j\in C_{i}}T_{k}^{x}\left(t\right).\]

Entonces $I_{i}^{x}\left(t\right)$ es el monto acumulado del
tiempo que el servidor $i$ ha estado desocupado al tiempo $t$. Se
est\'a asumiendo que las disciplinas satisfacen la ley de
conservaci\'on del trabajo, es decir, el servidor $i$ est\'a en
pausa solamente cuando no hay usuarios en la estaci\'on $i$.
Entonces, se tiene que

\begin{equation}
\int_{0}^{\infty}\left(\sum_{k\in
C_{i}}Q_{k}^{x}\left(t\right)\right)dI_{i}^{x}\left(t\right)=0,
\end{equation}
para $i=1,\ldots,d$.\\

Hacer
\[T^{x}\left(t\right)=\left(T_{1}^{x}\left(t\right),\ldots,T_{K}^{x}\left(t\right)\right)^{'},\]
\[I^{x}\left(t\right)=\left(I_{1}^{x}\left(t\right),\ldots,I_{K}^{x}\left(t\right)\right)^{'}\]
y
\[S^{x}\left(T^{x}\left(t\right)\right)=\left(S_{1}^{x}\left(T_{1}^{x}\left(t\right)\right),\ldots,S_{K}^{x}\left(T_{K}^{x}\left(t\right)\right)\right)^{'}.\]

Para una disciplina que cumple con la ley de conservaci\'on del
trabajo, en forma vectorial, se tiene el siguiente conjunto de
ecuaciones

\begin{equation}\label{Eq.MF.1.3}
Q^{x}\left(t\right)=Q^{x}\left(0\right)+E^{x}\left(t\right)+\sum_{l=1}^{K}\Phi^{l}\left(S_{l}^{x}\left(T_{l}^{x}\left(t\right)\right)\right)-S^{x}\left(T^{x}\left(t\right)\right),\\
\end{equation}

\begin{equation}\label{Eq.MF.2.3}
Q^{x}\left(t\right)\geq0,\\
\end{equation}

\begin{equation}\label{Eq.MF.3.3}
T^{x}\left(0\right)=0,\textrm{ y }\overline{T}^{x}\left(t\right)\textrm{ es no decreciente},\\
\end{equation}

\begin{equation}\label{Eq.MF.4.3}
I^{x}\left(t\right)=et-CT^{x}\left(t\right)\textrm{ es no
decreciente}\\
\end{equation}

\begin{equation}\label{Eq.MF.5.3}
\int_{0}^{\infty}\left(CQ^{x}\left(t\right)\right)dI_{i}^{x}\left(t\right)=0,\\
\end{equation}

\begin{equation}\label{Eq.MF.6.3}
\textrm{Condiciones adicionales en
}\left(\overline{Q}^{x}\left(\cdot\right),\overline{T}^{x}\left(\cdot\right)\right)\textrm{
espec\'ificas de la disciplina de la cola,}
\end{equation}

donde $e$ es un vector de unos de dimensi\'on $d$, $C$ es la
matriz definida por
\[C_{ik}=\left\{\begin{array}{cc}
1,& S\left(k\right)=i,\\
0,& \textrm{ en otro caso}.\\
\end{array}\right.
\]
Es necesario enunciar el siguiente Teorema que se utilizar\'a para
el Teorema \ref{Tma.4.2.Dai}:
\begin{Teo}[Teorema 4.1, Dai \cite{Dai}]
Considere una disciplina que cumpla la ley de conservaci\'on del
trabajo, para casi todas las trayectorias muestrales $\omega$ y
cualquier sucesi\'on de estados iniciales
$\left\{x_{n}\right\}\subset \mathbf{X}$, con
$|x_{n}|\rightarrow\infty$, existe una subsucesi\'on
$\left\{x_{n_{j}}\right\}$ con $|x_{n_{j}}|\rightarrow\infty$ tal
que
\begin{equation}\label{Eq.4.15}
\frac{1}{|x_{n_{j}}|}\left(Q^{x_{n_{j}}}\left(0\right),U^{x_{n_{j}}}\left(0\right),V^{x_{n_{j}}}\left(0\right)\right)\rightarrow\left(\overline{Q}\left(0\right),\overline{U},\overline{V}\right),
\end{equation}

\begin{equation}\label{Eq.4.16}
\frac{1}{|x_{n_{j}}|}\left(Q^{x_{n_{j}}}\left(|x_{n_{j}}|t\right),T^{x_{n_{j}}}\left(|x_{n_{j}}|t\right)\right)\rightarrow\left(\overline{Q}\left(t\right),\overline{T}\left(t\right)\right)\textrm{
u.o.c.}
\end{equation}

Adem\'as,
$\left(\overline{Q}\left(t\right),\overline{T}\left(t\right)\right)$
satisface las siguientes ecuaciones:
\begin{equation}\label{Eq.MF.1.3a}
\overline{Q}\left(t\right)=Q\left(0\right)+\left(\alpha
t-\overline{U}\right)^{+}-\left(I-P\right)^{'}M^{-1}\left(\overline{T}\left(t\right)-\overline{V}\right)^{+},
\end{equation}

\begin{equation}\label{Eq.MF.2.3a}
\overline{Q}\left(t\right)\geq0,\\
\end{equation}

\begin{equation}\label{Eq.MF.3.3a}
\overline{T}\left(t\right)\textrm{ es no decreciente y comienza en cero},\\
\end{equation}

\begin{equation}\label{Eq.MF.4.3a}
\overline{I}\left(t\right)=et-C\overline{T}\left(t\right)\textrm{
es no decreciente,}\\
\end{equation}

\begin{equation}\label{Eq.MF.5.3a}
\int_{0}^{\infty}\left(C\overline{Q}\left(t\right)\right)d\overline{I}\left(t\right)=0,\\
\end{equation}

\begin{equation}\label{Eq.MF.6.3a}
\textrm{Condiciones adicionales en
}\left(\overline{Q}\left(\cdot\right),\overline{T}\left(\cdot\right)\right)\textrm{
especficas de la disciplina de la cola,}
\end{equation}
\end{Teo}

\begin{Def}[Definici\'on 4.1, , Dai \cite{Dai}]
Sea una disciplina de servicio espec\'ifica. Cualquier l\'imite
$\left(\overline{Q}\left(\cdot\right),\overline{T}\left(\cdot\right)\right)$
en \ref{Eq.4.16} es un {\em flujo l\'imite} de la disciplina.
Cualquier soluci\'on (\ref{Eq.MF.1.3a})-(\ref{Eq.MF.6.3a}) es
llamado flujo soluci\'on de la disciplina. Se dice que el modelo de flujo l\'imite, modelo de flujo, de la disciplina de la cola es estable si existe una constante
$\delta>0$ que depende de $\mu,\alpha$ y $P$ solamente, tal que
cualquier flujo l\'imite con
$|\overline{Q}\left(0\right)|+|\overline{U}|+|\overline{V}|=1$, se
tiene que $\overline{Q}\left(\cdot+\delta\right)\equiv0$.
\end{Def}

\begin{Teo}[Teorema 4.2, Dai\cite{Dai}]\label{Tma.4.2.Dai}
Sea una disciplina fija para la cola, suponga que se cumplen las
condiciones (1.2)-(1.5). Si el modelo de flujo l\'imite de la
disciplina de la cola es estable, entonces la cadena de Markov $X$
que describe la din\'amica de la red bajo la disciplina es Harris
recurrente positiva.
\end{Teo}

Ahora se procede a escalar el espacio y el tiempo para reducir la
aparente fluctuaci\'on del modelo. Consid\'erese el proceso
\begin{equation}\label{Eq.3.7}
\overline{Q}^{x}\left(t\right)=\frac{1}{|x|}Q^{x}\left(|x|t\right)
\end{equation}
A este proceso se le conoce como el fluido escalado, y cualquier l\'imite $\overline{Q}^{x}\left(t\right)$ es llamado flujo l\'imite del proceso de longitud de la cola. Haciendo $|q|\rightarrow\infty$ mientras se mantiene el resto de las componentes fijas, cualquier punto l\'imite del proceso de longitud de la cola normalizado $\overline{Q}^{x}$ es soluci\'on del siguiente modelo de flujo.

\begin{Def}[Definici\'on 3.1, Dai y Meyn \cite{DaiSean}]
Un flujo l\'imite (retrasado) para una red bajo una disciplina de
servicio espec\'ifica se define como cualquier soluci\'on
 $\left(\overline{Q}\left(\cdot\right),\overline{T}\left(\cdot\right)\right)$ de las siguientes ecuaciones, donde
$\overline{Q}\left(t\right)=\left(\overline{Q}_{1}\left(t\right),\ldots,\overline{Q}_{K}\left(t\right)\right)^{'}$
y
$\overline{T}\left(t\right)=\left(\overline{T}_{1}\left(t\right),\ldots,\overline{T}_{K}\left(t\right)\right)^{'}$
\begin{equation}\label{Eq.3.8}
\overline{Q}_{k}\left(t\right)=\overline{Q}_{k}\left(0\right)+\alpha_{k}t-\mu_{k}\overline{T}_{k}\left(t\right)+\sum_{l=1}^{k}P_{lk}\mu_{l}\overline{T}_{l}\left(t\right)\\
\end{equation}
\begin{equation}\label{Eq.3.9}
\overline{Q}_{k}\left(t\right)\geq0\textrm{ para }k=1,2,\ldots,K,\\
\end{equation}
\begin{equation}\label{Eq.3.10}
\overline{T}_{k}\left(0\right)=0,\textrm{ y }\overline{T}_{k}\left(\cdot\right)\textrm{ es no decreciente},\\
\end{equation}
\begin{equation}\label{Eq.3.11}
\overline{I}_{i}\left(t\right)=t-\sum_{k\in C_{i}}\overline{T}_{k}\left(t\right)\textrm{ es no decreciente}\\
\end{equation}
\begin{equation}\label{Eq.3.12}
\overline{I}_{i}\left(\cdot\right)\textrm{ se incrementa al tiempo }t\textrm{ cuando }\sum_{k\in C_{i}}Q_{k}^{x}\left(t\right)dI_{i}^{x}\left(t\right)=0\\
\end{equation}
\begin{equation}\label{Eq.3.13}
\textrm{condiciones adicionales sobre
}\left(Q^{x}\left(\cdot\right),T^{x}\left(\cdot\right)\right)\textrm{
referentes a la disciplina de servicio}
\end{equation}
\end{Def}

Al conjunto de ecuaciones dadas en \ref{Eq.3.8}-\ref{Eq.3.13} se
le llama {\em Modelo de flujo} y al conjunto de todas las
soluciones del modelo de flujo
$\left(\overline{Q}\left(\cdot\right),\overline{T}
\left(\cdot\right)\right)$ se le denotar\'a por $\mathcal{Q}$.

Si se hace $|x|\rightarrow\infty$ sin restringir ninguna de las
componentes, tambi\'en se obtienen un modelo de flujo, pero en
este caso el residual de los procesos de arribo y servicio
introducen un retraso:

\begin{Def}[Definici\'on 3.2, Dai y Meyn \cite{DaiSean}]
El modelo de flujo retrasado de una disciplina de servicio en una
red con retraso
$\left(\overline{A}\left(0\right),\overline{B}\left(0\right)\right)\in\rea_{+}^{K+|A|}$
se define como el conjunto de ecuaciones dadas en
\ref{Eq.3.8}-\ref{Eq.3.13}, junto con la condici\'on:
\begin{equation}\label{CondAd.FluidModel}
\overline{Q}\left(t\right)=\overline{Q}\left(0\right)+\left(\alpha
t-\overline{A}\left(0\right)\right)^{+}-\left(I-P^{'}\right)M\left(\overline{T}\left(t\right)-\overline{B}\left(0\right)\right)^{+}
\end{equation}
\end{Def}

\begin{Def}[Definici\'on 3.3, Dai y Meyn \cite{DaiSean}]
El modelo de flujo es estable si existe un tiempo fijo $t_{0}$ tal
que $\overline{Q}\left(t\right)=0$, con $t\geq t_{0}$, para
cualquier $\overline{Q}\left(\cdot\right)\in\mathcal{Q}$ que
cumple con $|\overline{Q}\left(0\right)|=1$.
\end{Def}

El siguiente resultado se encuentra en Chen \cite{Chen}.
\begin{Lemma}[Lema 3.1, Dai y Meyn \cite{DaiSean}]
Si el modelo de flujo definido por \ref{Eq.3.8}-\ref{Eq.3.13} es
estable, entonces el modelo de flujo retrasado es tambi\'en
estable, es decir, existe $t_{0}>0$ tal que
$\overline{Q}\left(t\right)=0$ para cualquier $t\geq t_{0}$, para
cualquier soluci\'on del modelo de flujo retrasado cuya
condici\'on inicial $\overline{x}$ satisface que
$|\overline{x}|=|\overline{Q}\left(0\right)|+|\overline{A}\left(0\right)|+|\overline{B}\left(0\right)|\leq1$.
\end{Lemma}

%_________________________________________________________________________
\subsection{Modelo de Visitas C\'iclicas con un Servidor: Estabilidad}
%_________________________________________________________________________

%_________________________________________________________________________
\subsection{Teorema 2.1}
%_________________________________________________________________________



El resultado principal de Down \cite{Down} que relaciona la estabilidad del modelo de flujo con la estabilidad del sistema original

\begin{Teo}[Teorema 2.1, Down \cite{Down}]\label{Tma.2.1.Down}
Suponga que el modelo de flujo es estable, y que se cumplen los supuestos (A1) y (A2), entonces
\begin{itemize}
\item[i)] Para alguna constante $\kappa_{p}$, y para cada
condici\'on inicial $x\in X$
\begin{equation}\label{Estability.Eq1}
lim_{t\rightarrow\infty}\sup\frac{1}{t}\int_{0}^{t}\esp_{x}\left[|Q\left(s\right)|^{p}\right]ds\leq\kappa_{p},
\end{equation}
donde $p$ es el entero dado en (A2). Si adem\'as se cumple
la condici\'on (A3), entonces para cada condici\'on inicial:

\item[ii)] Los momentos transitorios convergen a su estado estacionario:
 \begin{equation}\label{Estability.Eq2}
lim_{t\rightarrow\infty}\esp_{x}\left[Q_{k}\left(t\right)^{r}\right]=\esp_{\pi}\left[Q_{k}\left(0\right)^{r}\right]\leq\kappa_{r},
\end{equation}
para $r=1,2,\ldots,p$ y $k=1,2,\ldots,K$. Donde $\pi$ es la
probabilidad invariante para $\mathbf{X}$.

\item[iii)]  El primer momento converge con raz\'on $t^{p-1}$:
\begin{equation}\label{Estability.Eq3}
lim_{t\rightarrow\infty}t^{p-1}|\esp_{x}\left[Q_{k}\left(t\right)\right]-\esp_{\pi}\left[Q\left(0\right)\right]=0.
\end{equation}

\item[iv)] La {\em Ley Fuerte de los grandes n\'umeros} se cumple:
\begin{equation}\label{Estability.Eq4}
lim_{t\rightarrow\infty}\frac{1}{t}\int_{0}^{t}Q_{k}^{r}\left(s\right)ds=\esp_{\pi}\left[Q_{k}\left(0\right)^{r}\right],\textrm{
}\prob_{x}\textrm{-c.s.}
\end{equation}
para $r=1,2,\ldots,p$ y $k=1,2,\ldots,K$.
\end{itemize}
\end{Teo}


\begin{Prop}[Proposici\'on 5.1, Dai y Meyn \cite{DaiSean}]\label{Prop.5.1.DaiSean}
Suponga que los supuestos A1) y A2) son ciertos y que el modelo de flujo es estable. Entonces existe $t_{0}>0$ tal que
\begin{equation}
lim_{|x|\rightarrow\infty}\frac{1}{|x|^{p+1}}\esp_{x}\left[|X\left(t_{0}|x|\right)|^{p+1}\right]=0
\end{equation}
\end{Prop}

\begin{Lemma}[Lema 5.2, Dai y Meyn \cite{DaiSean}]\label{Lema.5.2.DaiSean}
 Sea $\left\{\zeta\left(k\right):k\in \mathbb{z}\right\}$ una sucesi\'on independiente e id\'enticamente distribuida que toma valores en $\left(0,\infty\right)$,
y sea
$E\left(t\right)=max\left(n\geq1:\zeta\left(1\right)+\cdots+\zeta\left(n-1\right)\leq
t\right)$. Si $\esp\left[\zeta\left(1\right)\right]<\infty$,
entonces para cualquier entero $r\geq1$
\begin{equation}
 lim_{t\rightarrow\infty}\esp\left[\left(\frac{E\left(t\right)}{t}\right)^{r}\right]=\left(\frac{1}{\esp\left[\zeta_{1}\right]}\right)^{r}.
\end{equation}
Luego, bajo estas condiciones:
\begin{itemize}
 \item[a)] para cualquier $\delta>0$, $\sup_{t\geq\delta}\esp\left[\left(\frac{E\left(t\right)}{t}\right)^{r}\right]<\infty$
\item[b)] las variables aleatorias
$\left\{\left(\frac{E\left(t\right)}{t}\right)^{r}:t\geq1\right\}$
son uniformemente integrables.
\end{itemize}
\end{Lemma}

\begin{Teo}[Teorema 5.5, Dai y Meyn \cite{DaiSean}]\label{Tma.5.5.DaiSean}
Suponga que los supuestos A1) y A2) se cumplen y que el modelo de
flujo es estable. Entonces existe una constante $\kappa_{p}$ tal
que
\begin{equation}
\frac{1}{t}\int_{0}^{t}\esp_{x}\left[|Q\left(s\right)|^{p}\right]ds\leq\kappa_{p}\left\{\frac{1}{t}|x|^{p+1}+1\right\}
\end{equation}
para $t>0$ y $x\in X$. En particular, para cada condici\'on
inicial
\begin{eqnarray*}
\limsup_{t\rightarrow\infty}\frac{1}{t}\int_{0}^{t}\esp_{x}\left[|Q\left(s\right)|^{p}\right]ds\leq\kappa_{p}.
\end{eqnarray*}
\end{Teo}

\begin{Teo}[Teorema 6.2, Dai y Meyn \cite{DaiSean}]\label{Tma.6.2.DaiSean}
Suponga que se cumplen los supuestos A1), A2) y A3) y que el
modelo de flujo es estable. Entonces se tiene que
\begin{equation}
\left\|P^{t}\left(x,\cdot\right)-\pi\left(\cdot\right)\right\|_{f_{p}}\textrm{,
}t\rightarrow\infty,x\in X.
\end{equation}
En particular para cada condici\'on inicial
\begin{eqnarray*}
\lim_{t\rightarrow\infty}\esp_{x}\left[|Q\left(t\right)|^{p}\right]=\esp_{\pi}\left[|Q\left(0\right)|^{p}\right]\leq\kappa_{r}
\end{eqnarray*}
\end{Teo}
\begin{Teo}[Teorema 6.3, Dai y Meyn \cite{DaiSean}]\label{Tma.6.3.DaiSean}
Suponga que se cumplen los supuestos A1), A2) y A3) y que el
modelo de flujo es estable. Entonces con
$f\left(x\right)=f_{1}\left(x\right)$ se tiene
\begin{equation}
\lim_{t\rightarrow\infty}t^{p-1}\left\|P^{t}\left(x,\cdot\right)-\pi\left(\cdot\right)\right\|_{f}=0.
\end{equation}
En particular para cada condici\'on inicial
\begin{eqnarray*}
\lim_{t\rightarrow\infty}t^{p-1}|\esp_{x}\left[Q\left(t\right)\right]-\esp_{\pi}\left[Q\left(0\right)\right]|=0.
\end{eqnarray*}
\end{Teo}

\begin{Teo}[Teorema 6.4, Dai y Meyn \cite{DaiSean}]\label{Tma.6.4.DaiSean}
Suponga que se cumplen los supuestos A1), A2) y A3) y que el
modelo de flujo es estable. Sea $\nu$ cualquier distribuci\'on de
probabilidad en $\left(X,\mathcal{B}_{X}\right)$, y $\pi$ la
distribuci\'on estacionaria de $X$.
\begin{itemize}
\item[i)] Para cualquier $f:X\leftarrow\rea_{+}$
\begin{equation}
\lim_{t\rightarrow\infty}\frac{1}{t}\int_{o}^{t}f\left(X\left(s\right)\right)ds=\pi\left(f\right):=\int
f\left(x\right)\pi\left(dx\right)
\end{equation}
$\prob$-c.s.

\item[ii)] Para cualquier $f:X\leftarrow\rea_{+}$ con
$\pi\left(|f|\right)<\infty$, la ecuaci\'on anterior se cumple.
\end{itemize}
\end{Teo}

%_________________________________________________________________________
\subsection{Teorema 2.2}
%_________________________________________________________________________

\begin{Teo}[Teorema 2.2, Down \cite{Down}]\label{Tma2.2.Down}
Suponga que el fluido modelo es inestable en el sentido de que
para alguna $\epsilon_{0},c_{0}\geq0$,
\begin{equation}\label{Eq.Inestability}
|Q\left(T\right)|\geq\epsilon_{0}T-c_{0}\textrm{,   }T\geq0,
\end{equation}
para cualquier condici\'on inicial $Q\left(0\right)$, con
$|Q\left(0\right)|=1$. Entonces para cualquier $0<q\leq1$, existe
$B<0$ tal que para cualquier $|x|\geq B$,
\begin{equation}
\prob_{x}\left\{\mathbb{X}\rightarrow\infty\right\}\geq q.
\end{equation}
\end{Teo}

%_________________________________________________________________________
\subsection{Teorema 2.3}
%_________________________________________________________________________
\begin{Teo}[Teorema 2.3, Down \cite{Down}]\label{Tma2.3.Down}
Considere el siguiente valor:
\begin{equation}\label{Eq.Rho.1serv}
\rho=\sum_{k=1}^{K}\rho_{k}+max_{1\leq j\leq K}\left(\frac{\lambda_{j}}{\sum_{s=1}^{S}p_{js}\overline{N}_{s}}\right)\delta^{*}
\end{equation}
\begin{itemize}
\item[i)] Si $\rho<1$ entonces la red es estable, es decir, se cumple el teorema \ref{Tma.2.1.Down}.

\item[ii)] Si $\rho<1$ entonces la red es inestable, es decir, se cumple el teorema \ref{Tma2.2.Down}
\end{itemize}
\end{Teo}
\newpage
%_____________________________________________________________________
\subsection{Definiciones  B\'asicas}
%_____________________________________________________________________
\begin{Def}
Sea $X$ un conjunto y $\mathcal{F}$ una $\sigma$-\'algebra de
subconjuntos de $X$, la pareja $\left(X,\mathcal{F}\right)$ es
llamado espacio medible. Un subconjunto $A$ de $X$ es llamado
medible, o medible con respecto a $\mathcal{F}$, si
$A\in\mathcal{F}$.
\end{Def}

\begin{Def}
Sea $\left(X,\mathcal{F},\mu\right)$ espacio de medida. Se dice
que la medida $\mu$ es $\sigma$-finita si se puede escribir
$X=\bigcup_{n\geq1}X_{n}$ con $X_{n}\in\mathcal{F}$ y
$\mu\left(X_{n}\right)<\infty$.
\end{Def}

\begin{Def}\label{Cto.Borel}
Sea $X$ el conjunto de los \'umeros reales $\rea$. El \'algebra de
Borel es la $\sigma$-\'algebra $B$ generada por los intervalos
abiertos $\left(a,b\right)\in\rea$. Cualquier conjunto en $B$ es
llamado {\em Conjunto de Borel}.
\end{Def}

\begin{Def}\label{Funcion.Medible}
Una funci\'on $f:X\rightarrow\rea$, es medible si para cualquier
n\'umero real $\alpha$ el conjunto
\[\left\{x\in X:f\left(x\right)>\alpha\right\}\]
pertenece a $X$. Equivalentemente, se dice que $f$ es medible si
\[f^{-1}\left(\left(\alpha,\infty\right)\right)=\left\{x\in X:f\left(x\right)>\alpha\right\}\in\mathcal{F}.\]
\end{Def}


\begin{Def}\label{Def.Cilindros}
Sean $\left(\Omega_{i},\mathcal{F}_{i}\right)$, $i=1,2,\ldots,$
espacios medibles y $\Omega=\prod_{i=1}^{\infty}\Omega_{i}$ el
conjunto de todas las sucesiones
$\left(\omega_{1},\omega_{2},\ldots,\right)$ tales que
$\omega_{i}\in\Omega_{i}$, $i=1,2,\ldots,$. Si
$B^{n}\subset\prod_{i=1}^{\infty}\Omega_{i}$, definimos
$B_{n}=\left\{\omega\in\Omega:\left(\omega_{1},\omega_{2},\ldots,\omega_{n}\right)\in
B^{n}\right\}$. Al conjunto $B_{n}$ se le llama {\em cilindro} con
base $B^{n}$, el cilindro es llamado medible si
$B^{n}\in\prod_{i=1}^{\infty}\mathcal{F}_{i}$.
\end{Def}


\begin{Def}\label{Def.Proc.Adaptado}[TSP, Ash \cite{RBA}]
Sea $X\left(t\right),t\geq0$ proceso estoc\'astico, el proceso es
adaptado a la familia de $\sigma$-\'algebras $\mathcal{F}_{t}$,
para $t\geq0$, si para $s<t$ implica que
$\mathcal{F}_{s}\subset\mathcal{F}_{t}$, y $X\left(t\right)$ es
$\mathcal{F}_{t}$-medible para cada $t$. Si no se especifica
$\mathcal{F}_{t}$ entonces se toma $\mathcal{F}_{t}$ como
$\mathcal{F}\left(X\left(s\right),s\leq t\right)$, la m\'as
peque\~na $\sigma$-\'algebra de subconjuntos de $\Omega$ que hace
que cada $X\left(s\right)$, con $s\leq t$ sea Borel medible.
\end{Def}


\begin{Def}\label{Def.Tiempo.Paro}[TSP, Ash \cite{RBA}]
Sea $\left\{\mathcal{F}\left(t\right),t\geq0\right\}$ familia
creciente de sub $\sigma$-\'algebras. es decir,
$\mathcal{F}\left(s\right)\subset\mathcal{F}\left(t\right)$ para
$s\leq t$. Un tiempo de paro para $\mathcal{F}\left(t\right)$ es
una funci\'on $T:\Omega\rightarrow\left[0,\infty\right]$ tal que
$\left\{T\leq t\right\}\in\mathcal{F}\left(t\right)$ para cada
$t\geq0$. Un tiempo de paro para el proceso estoc\'astico
$X\left(t\right),t\geq0$ es un tiempo de paro para las
$\sigma$-\'algebras
$\mathcal{F}\left(t\right)=\mathcal{F}\left(X\left(s\right)\right)$.
\end{Def}

\begin{Def}
Sea $X\left(t\right),t\geq0$ proceso estoc\'astico, con
$\left(S,\chi\right)$ espacio de estados. Se dice que el proceso
es adaptado a $\left\{\mathcal{F}\left(t\right)\right\}$, es
decir, si para cualquier $s,t\in I$, $I$ conjunto de \'indices,
$s<t$, se tiene que
$\mathcal{F}\left(s\right)\subset\mathcal{F}\left(t\right)$ y
$X\left(t\right)$ es $\mathcal{F}\left(t\right)$-medible,
\end{Def}

\begin{Def}
Sea $X\left(t\right),t\geq0$ proceso estoc\'astico, se dice que es
un Proceso de Markov relativo a $\mathcal{F}\left(t\right)$ o que
$\left\{X\left(t\right),\mathcal{F}\left(t\right)\right\}$ es de
Markov si y s\'olo si para cualquier conjunto $B\in\chi$,  y
$s,t\in I$, $s<t$ se cumple que
\begin{equation}\label{Prop.Markov}
P\left\{X\left(t\right)\in
B|\mathcal{F}\left(s\right)\right\}=P\left\{X\left(t\right)\in
B|X\left(s\right)\right\}.
\end{equation}
\end{Def}
\begin{Note}
Si se dice que $\left\{X\left(t\right)\right\}$ es un Proceso de
Markov sin mencionar $\mathcal{F}\left(t\right)$, se asumir\'a que
\begin{eqnarray*}
\mathcal{F}\left(t\right)=\mathcal{F}_{0}\left(t\right)=\mathcal{F}\left(X\left(r\right),r\leq
t\right),
\end{eqnarray*}
entonces la ecuaci\'on (\ref{Prop.Markov}) se puede escribir como
\begin{equation}
P\left\{X\left(t\right)\in B|X\left(r\right),r\leq s\right\} =
P\left\{X\left(t\right)\in B|X\left(s\right)\right\}
\end{equation}
\end{Note}

\begin{Teo}
Sea $\left(X_{n},\mathcal{F}_{n},n=0,1,\ldots,\right\}$ Proceso de
Markov con espacio de estados $\left(S_{0},\chi_{0}\right)$
generado por una distribuici\'on inicial $P_{o}$ y probabilidad de
transici\'on $p_{mn}$, para $m,n=0,1,\ldots,$ $m<n$, que por
notaci\'on se escribir\'a como $p\left(m,n,x,B\right)\rightarrow
p_{mn}\left(x,B\right)$. Sea $S$ tiempo de paro relativo a la
$\sigma$-\'algebra $\mathcal{F}_{n}$. Sea $T$ funci\'on medible,
$T:\Omega\rightarrow\left\{0,1,\ldots,\right\}$. Sup\'ongase que
$T\geq S$, entonces $T$ es tiempo de paro. Si $B\in\chi_{0}$,
entonces
\begin{equation}\label{Prop.Fuerte.Markov}
P\left\{X\left(T\right)\in
B,T<\infty|\mathcal{F}\left(S\right)\right\} =
p\left(S,T,X\left(s\right),B\right)
\end{equation}
en $\left\{T<\infty\right\}$.
\end{Teo}

Propiedades importantes para el modelo de flujo retrasado:

\begin{Prop}
 Sea $\left(\overline{Q},\overline{T},\overline{T}^{0}\right)$ un flujo l\'imite de \ref{Equation.4.4} y suponga que cuando $x\rightarrow\infty$ a lo largo de
una subsucesi\'on
\[\left(\frac{1}{|x|}Q_{k}^{x}\left(0\right),\frac{1}{|x|}A_{k}^{x}\left(0\right),\frac{1}{|x|}B_{k}^{x}\left(0\right),\frac{1}{|x|}B_{k}^{x,0}\left(0\right)\right)\rightarrow\left(\overline{Q}_{k}\left(0\right),0,0,0\right)\]
para $k=1,\ldots,K$. EL flujo l\'imite tiene las siguientes
propiedades, donde las propiedades de la derivada se cumplen donde
la derivada exista:
\begin{itemize}
 \item[i)] Los vectores de tiempo ocupado $\overline{T}\left(t\right)$ y $\overline{T}^{0}\left(t\right)$ son crecientes y continuas con
$\overline{T}\left(0\right)=\overline{T}^{0}\left(0\right)=0$.
\item[ii)] Para todo $t\geq0$
\[\sum_{k=1}^{K}\left[\overline{T}_{k}\left(t\right)+\overline{T}_{k}^{0}\left(t\right)\right]=t\]
\item[iii)] Para todo $1\leq k\leq K$
\[\overline{Q}_{k}\left(t\right)=\overline{Q}_{k}\left(0\right)+\alpha_{k}t-\mu_{k}\overline{T}_{k}\left(t\right)\]
\item[iv)]  Para todo $1\leq k\leq K$
\[\dot{{\overline{T}}}_{k}\left(t\right)=\beta_{k}\] para $\overline{Q}_{k}\left(t\right)=0$.
\item[v)] Para todo $k,j$
\[\mu_{k}^{0}\overline{T}_{k}^{0}\left(t\right)=\mu_{j}^{0}\overline{T}_{j}^{0}\left(t\right)\]
\item[vi)]  Para todo $1\leq k\leq K$
\[\mu_{k}\dot{{\overline{T}}}_{k}\left(t\right)=l_{k}\mu_{k}^{0}\dot{{\overline{T}}}_{k}^{0}\left(t\right)\] para $\overline{Q}_{k}\left(t\right)>0$.
\end{itemize}
\end{Prop}

\begin{Lema}[Lema 3.1 \cite{Chen}]\label{Lema3.1}
Si el modelo de flujo es estable, definido por las ecuaciones
(3.8)-(3.13), entonces el modelo de flujo retrasado tambin es
estable.
\end{Lema}

\begin{Teo}[Teorema 5.2 \cite{Chen}]\label{Tma.5.2}
Si el modelo de flujo lineal correspondiente a la red de cola es
estable, entonces la red de colas es estable.
\end{Teo}

\begin{Teo}[Teorema 5.1 \cite{Chen}]\label{Tma.5.1.Chen}
La red de colas es estable si existe una constante $t_{0}$ que
depende de $\left(\alpha,\mu,T,U\right)$ y $V$ que satisfagan las
ecuaciones (5.1)-(5.5), $Z\left(t\right)=0$, para toda $t\geq
t_{0}$.
\end{Teo}



\begin{Lema}[Lema 5.2 \cite{Gut}]\label{Lema.5.2.Gut}
Sea $\left\{\xi\left(k\right):k\in\ent\right\}$ sucesin de
variables aleatorias i.i.d. con valores en
$\left(0,\infty\right)$, y sea $E\left(t\right)$ el proceso de
conteo
\[E\left(t\right)=max\left\{n\geq1:\xi\left(1\right)+\cdots+\xi\left(n-1\right)\leq t\right\}.\]
Si $E\left[\xi\left(1\right)\right]<\infty$, entonces para
cualquier entero $r\geq1$
\begin{equation}
lim_{t\rightarrow\infty}\esp\left[\left(\frac{E\left(t\right)}{t}\right)^{r}\right]=\left(\frac{1}{E\left[\xi_{1}\right]}\right)^{r}
\end{equation}
de aqu, bajo estas condiciones
\begin{itemize}
\item[a)] Para cualquier $t>0$,
$sup_{t\geq\delta}\esp\left[\left(\frac{E\left(t\right)}{t}\right)^{r}\right]$

\item[b)] Las variables aleatorias
$\left\{\left(\frac{E\left(t\right)}{t}\right)^{r}:t\geq1\right\}$
son uniformemente integrables.
\end{itemize}
\end{Lema}

\begin{Teo}[Teorema 5.1: Ley Fuerte para Procesos de Conteo
\cite{Gut}]\label{Tma.5.1.Gut} Sea
$0<\mu<\esp\left(X_{1}\right]\leq\infty$. entonces

\begin{itemize}
\item[a)] $\frac{N\left(t\right)}{t}\rightarrow\frac{1}{\mu}$
a.s., cuando $t\rightarrow\infty$.


\item[b)]$\esp\left[\frac{N\left(t\right)}{t}\right]^{r}\rightarrow\frac{1}{\mu^{r}}$,
cuando $t\rightarrow\infty$ para todo $r>0$..
\end{itemize}
\end{Teo}


\begin{Prop}[Proposicin 5.1 \cite{DaiSean}]\label{Prop.5.1}
Suponga que los supuestos (A1) y (A2) se cumplen, adems suponga
que el modelo de flujo es estable. Entonces existe $t_{0}>0$ tal
que
\begin{equation}\label{Eq.Prop.5.1}
lim_{|x|\rightarrow\infty}\frac{1}{|x|^{p+1}}\esp_{x}\left[|X\left(t_{0}|x|\right)|^{p+1}\right]=0.
\end{equation}

\end{Prop}


\begin{Prop}[Proposici\'on 5.3 \cite{DaiSean}]
Sea $X$ proceso de estados para la red de colas, y suponga que se
cumplen los supuestos (A1) y (A2), entonces para alguna constante
positiva $C_{p+1}<\infty$, $\delta>0$ y un conjunto compacto
$C\subset X$.

\begin{equation}\label{Eq.5.4}
\esp_{x}\left[\int_{0}^{\tau_{C}\left(\delta\right)}\left(1+|X\left(t\right)|^{p}\right)dt\right]\leq
C_{p+1}\left(1+|x|^{p+1}\right)
\end{equation}
\end{Prop}

\begin{Prop}[Proposici\'on 5.4 \cite{DaiSean}]
Sea $X$ un proceso de Markov Borel Derecho en $X$, sea
$f:X\leftarrow\rea_{+}$ y defina para alguna $\delta>0$, y un
conjunto cerrado $C\subset X$
\[V\left(x\right):=\esp_{x}\left[\int_{0}^{\tau_{C}\left(\delta\right)}f\left(X\left(t\right)\right)dt\right]\]
para $x\in X$. Si $V$ es finito en todas partes y uniformemente
acotada en $C$, entonces existe $k<\infty$ tal que
\begin{equation}\label{Eq.5.11}
\frac{1}{t}\esp_{x}\left[V\left(x\right)\right]+\frac{1}{t}\int_{0}^{t}\esp_{x}\left[f\left(X\left(s\right)\right)ds\right]\leq\frac{1}{t}V\left(x\right)+k,
\end{equation}
para $x\in X$ y $t>0$.
\end{Prop}


\begin{Teo}[Teorema 5.5 \cite{DaiSean}]
Suponga que se cumplen (A1) y (A2), adems suponga que el modelo
de flujo es estable. Entonces existe una constante $k_{p}<\infty$
tal que
\begin{equation}\label{Eq.5.13}
\frac{1}{t}\int_{0}^{t}\esp_{x}\left[|Q\left(s\right)|^{p}\right]ds\leq
k_{p}\left\{\frac{1}{t}|x|^{p+1}+1\right\}
\end{equation}
para $t\geq0$, $x\in X$. En particular para cada condicin inicial
\begin{equation}\label{Eq.5.14}
Limsup_{t\rightarrow\infty}\frac{1}{t}\int_{0}^{t}\esp_{x}\left[|Q\left(s\right)|^{p}\right]ds\leq
k_{p}
\end{equation}
\end{Teo}

\begin{Teo}[Teorema 6.2\cite{DaiSean}]\label{Tma.6.2}
Suponga que se cumplen los supuestos (A1)-(A3) y que el modelo de
flujo es estable, entonces se tiene que
\[\parallel P^{t}\left(c,\cdot\right)-\pi\left(\cdot\right)\parallel_{f_{p}}\rightarrow0\]
para $t\rightarrow\infty$ y $x\in X$. En particular para cada
condicin inicial
\[lim_{t\rightarrow\infty}\esp_{x}\left[\left|Q_{t}\right|^{p}\right]=\esp_{\pi}\left[\left|Q_{0}\right|^{p}\right]<\infty\]
\end{Teo}


\begin{Teo}[Teorema 6.3\cite{DaiSean}]\label{Tma.6.3}
Suponga que se cumplen los supuestos (A1)-(A3) y que el modelo de
flujo es estable, entonces con
$f\left(x\right)=f_{1}\left(x\right)$, se tiene que
\[lim_{t\rightarrow\infty}t^{(p-1)\left|P^{t}\left(c,\cdot\right)-\pi\left(\cdot\right)\right|_{f}=0},\]
para $x\in X$. En particular, para cada condicin inicial
\[lim_{t\rightarrow\infty}t^{(p-1)\left|\esp_{x}\left[Q_{t}\right]-\esp_{\pi}\left[Q_{0}\right]\right|=0}.\]
\end{Teo}



%_____________________________________________________________________________________
\subsection{Proceso de Estados Markoviano para el Sistema}
%_________________________________________________________________________


Sean $Q_{k}\left(t\right)$ el n\'umero de usuarios en la cola $k$,
$A_{k}\left(t\right)$ el tiempo residual de arribos a la cola $k$,
para cada servidor $m$, sea $H_{m}\left(t\right)$ par ordenado que
consiste en la cola que est\'a siendo atendida y la pol\'itica de
servicio que se est\'a utilizando. $B_{m}\left(t\right)$ los
tiempos de servicio residuales, $B_{m}^{0}\left(t\right)$ el
tiempo residual de traslado, $C_{m}\left(t\right)$ el n\'umero de
usuarios atendidos durante la visita del servidor a la cola dada
en $H_{m}\left(t\right)$.

El proceso para el sistema de visitas se puede definir como:

\begin{equation}\label{Esp.Edos.Down}
X\left(t\right)^{T}=\left(Q_{k}\left(t\right),A_{k}\left(t\right),B_{m}\left(t\right),B_{m}^{0}\left(t\right),C_{m}\left(t\right)\right)
\end{equation}
para $k=1,\ldots,K$ y $m=1,2,\ldots,M$. $X$ evoluciona en el
espacio de estados:
$X=\ent_{+}^{K}\times\rea_{+}^{K}\times\left(\left\{1,2,\ldots,K\right\}\times\left\{1,2,\ldots,S\right\}\right)^{M}\times\rea_{+}^{K}\times\rea_{+}^{K}\times\ent_{+}^{K}$.\\

Antes enunciemos los supuestos que regir\'an en la red.


\begin{itemize}
\item[A1)] $\xi_{1},\ldots,\xi_{K},\eta_{1},\ldots,\eta_{K}$ son
mutuamente independientes y son sucesiones independientes e
id\'enticamente distribuidas.

\item[A2)] Para alg\'un entero $p\geq1$
\begin{eqnarray*}
\esp\left[\xi_{l}\left(1\right)^{p+1}\right]<\infty\textrm{ para }l\in\mathcal{A}\textrm{ y }\\
\esp\left[\eta_{k}\left(1\right)^{p+1}\right]<\infty\textrm{ para
}k=1,\ldots,K.
\end{eqnarray*}
donde $\mathcal{A}$ es la clase de posibles arribos.

\item[A3)] Para $k=1,2,\ldots,K$ existe una funci\'on positiva
$q_{k}\left(x\right)$ definida en $\rea_{+}$, y un entero $j_{k}$,
tal que
\begin{eqnarray}
P\left(\xi_{k}\left(1\right)\geq x\right)>0\textrm{, para todo }x>0\\
P\left(\xi_{k}\left(1\right)+\ldots\xi_{k}\left(j_{k}\right)\in dx\right)\geq q_{k}\left(x\right)dx0\textrm{ y }\\
\int_{0}^{\infty}q_{k}\left(x\right)dx>0
\end{eqnarray}
\end{itemize}
%_________________________________________________________________________
\subsection{Procesos Fuerte de Markov}
%_________________________________________________________________________

En Dai \cite{Dai} se muestra que para una amplia serie de
disciplinas de servicio el proceso $X$ es un Proceso Fuerte de
Markov, y por tanto se puede asumir que
\[\left(\left(\Omega,\mathcal{F}\right),\mathcal{F}_{t},X\left(t\right),\theta_{t},P_{x}\right)\]
es un proceso de Borel Derecho, Sharpe \cite{Sharpe}, en el
espacio de estados medible
$\left(X,\mathcal{B}_{X}\right)$.


Se har\'an las siguientes consideraciones: $E$ es un espacio
m\'etrico separable.


\begin{Def}
Un espacio topol\'ogico $E$ es llamado {\em Luisin} si es
homeomorfo a un subconjunto de Borel de un espacio m\'etrico
compacto.
\end{Def}

\begin{Def}
Un espacio topol\'ogico $E$ es llamado de {\em Rad\'on} si es
homeomorfo a un subconjunto universalmente medible de un espacio
m\'etrico compacto.
\end{Def}

Equivalentemente, la definici\'on de un espacio de Rad\'on puede
encontrarse en los siguientes t\'erminos:

\begin{Def}
$E$ es un espacio de Rad\'on si cada medida finita en
$\left(E,\mathcal{B}\left(E\right)\right)$ es regular interior o
cerrada, {\em tight}.
\end{Def}

\begin{Def}
Una medida finita, $\lambda$ en la $\sigma$-\'algebra de Borel de
un espacio metrizable $E$ se dice cerrada si
\begin{equation}\label{Eq.A2.3}
\lambda\left(E\right)=sup\left\{\lambda\left(K\right):K\textrm{ es
compacto en }E\right\}.
\end{equation}
\end{Def}

El siguiente teorema nos permite tener una mejor caracterizaci\'on
de los espacios de Rad\'on:
\begin{Teo}\label{Tma.A2.2}
Sea $E$ espacio separable metrizable. Entonces $E$ es Radoniano si
y s\'olo s\'i cada medida finita en
$\left(E,\mathcal{B}\left(E\right)\right)$ es cerrada.
\end{Teo}

%_________________________________________________________________________________________
\subsection{Propiedades de Markov}
%_________________________________________________________________________________________

Sea $E$ espacio de estados, tal que $E$ es un espacio de Rad\'on,
$\mathcal{B}\left(E\right)$ $\sigma$-\'algebra de Borel en $E$,
que se denotar\'a por $\mathcal{E}$.

Sea $\left(X,\mathcal{G},\prob\right)$ espacio de probabilidad,
$I\subset\rea$ conjunto de índices. Sea $\mathcal{F}_{\leq
t}$ la $\sigma$-\'algebra natural definida como
$\sigma\left\{f\left(X_{r}\right):r\in I, r\leq
t,f\in\mathcal{E}\right\}$. Se considerar\'a una
$\sigma$-\'algebra m\'as general, $ \left(\mathcal{G}_{t}\right)$
tal que $\left(X_{t}\right)$ sea $\mathcal{E}$-adaptado.

\begin{Def}
Una familia $\left(P_{s,t}\right)$ de kernels de Markov en
$\left(E,\mathcal{E}\right)$ indexada por pares $s,t\in I$, con
$s\leq t$ es una funci\'on de transici\'on en $\ER$, si  para todo
$r\leq s< t$ en $I$ y todo $x\in E$, $B\in\mathcal{E}$
\begin{equation}\label{Eq.Kernels}
P_{r,t}\left(x,B\right)=\int_{E}P_{r,s}\left(x,dy\right)P_{s,t}\left(y,B\right)\footnote{Ecuaci\'on
de Chapman-Kolmogorov}.
\end{equation}
\end{Def}

Se dice que la funci\'on de transici\'on $\KM$ en $\ER$ es la
funci\'on de transici\'on para un proceso $\PE$  con valores en
$E$ y que satisface la propiedad de
Markov \footnote{\begin{equation}\label{Eq.1.4.S}
\prob\left\{H|\mathcal{G}_{t}\right\}=\prob\left\{H|X_{t}\right\}\textrm{
}H\in p\mathcal{F}_{\geq t}.
\end{equation}} (\ref{Eq.1.4.S}) relativa a $\left(\mathcal{G}_{t}\right)$ si

\begin{equation}\label{Eq.1.6.S}
\prob\left\{f\left(X_{t}\right)|\mathcal{G}_{s}\right\}=P_{s,t}f\left(X_{t}\right)\textrm{
}s\leq t\in I,\textrm{ }f\in b\mathcal{E}.
\end{equation}

\begin{Def}
Una familia $\left(P_{t}\right)_{t\geq0}$ de kernels de Markov en
$\ER$ es llamada {\em Semigrupo de Transici\'on de Markov} o {\em
Semigrupo de Transici\'on} si
\[P_{t+s}f\left(x\right)=P_{t}\left(P_{s}f\right)\left(x\right),\textrm{ }t,s\geq0,\textrm{ }x\in E\textrm{ }f\in b\mathcal{E}.\]
\end{Def}

\begin{Note}
Si la funci\'on de transici\'on $\KM$ es llamada homog\'enea si
$P_{s,t}=P_{t-s}$.
\end{Note}


Un proceso de Markov que satisface la ecuaci\'on (\ref{Eq.1.6.S})
con funci\'on de transici\'on homog\'enea $\left(P_{t}\right)$
tiene la propiedad caracter\'istica
\begin{equation}\label{Eq.1.8.S}
\prob\left\{f\left(X_{t+s}\right)|\mathcal{G}_{t}\right\}=P_{s}f\left(X_{t}\right)\textrm{
}t,s\geq0,\textrm{ }f\in b\mathcal{E}.
\end{equation}
La ecuaci\'on anterior es la {\em Propiedad Simple de Markov} de
$X$ relativa a $\left(P_{t}\right)$.

En este sentido el proceso $\PE$ cumple con la propiedad de Markov
(\ref{Eq.1.8.S}) relativa a
$\left(\Omega,\mathcal{G},\mathcal{G}_{t},\prob\right)$ con
semigrupo de transici\'on $\left(P_{t}\right)$.

%_________________________________________________________________________________________
\subsection{Primer Condici\'on de Regularidad}
%_________________________________________________________________________________________


\begin{Def}
Un proceso estoc\'astico $\PE$ definido en
$\left(\Omega,\mathcal{G},\prob\right)$ con valores en el espacio
topol\'ogico $E$ es continuo por la derecha si cada trayectoria
muestral $t\rightarrow X_{t}\left(w\right)$ es un mapeo continuo
por la derecha de $I$ en $E$.
\end{Def}

\begin{Def}[HD1]\label{Eq.2.1.S}
Un semigrupo de Markov $\left/P_{t}\right)$ en un espacio de
Rad\'on $E$ se dice que satisface la condici\'on {\em HD1} si,
dada una medida de probabilidad $\mu$ en $E$, existe una
$\sigma$-\'algebra $\mathcal{E^{*}}$ con
$\mathcal{E}\subset\mathcal{E}$ y
$P_{t}\left(b\mathcal{E}^{*}\right)\subset b\mathcal{E}^{*}$, y un
$\mathcal{E}^{*}$-proceso $E$-valuado continuo por la derecha
$\PE$ en alg\'un espacio de probabilidad filtrado
$\left(\Omega,\mathcal{G},\mathcal{G}_{t},\prob\right)$ tal que
$X=\left(\Omega,\mathcal{G},\mathcal{G}_{t},\prob\right)$ es de
Markov (Homog\'eneo) con semigrupo de transici\'on $(P_{t})$ y
distribuci\'on inicial $\mu$.
\end{Def}
Consid\'erese la colecci\'on de variables aleatorias $X_{t}$
definidas en alg\'un espacio de probabilidad, y una colecci\'on de
medidas $\mathbf{P}^{x}$ tales que
$\mathbf{P}^{x}\left\{X_{0}=x\right\}$, y bajo cualquier
$\mathbf{P}^{x}$, $X_{t}$ es de Markov con semigrupo
$\left(P_{t}\right)$. $\mathbf{P}^{x}$ puede considerarse como la
distribuci\'on condicional de $\mathbf{P}$ dado $X_{0}=x$.

\begin{Def}\label{Def.2.2.S}
Sea $E$ espacio de Rad\'on, $\SG$ semigrupo de Markov en $\ER$. La
colecci\'on
$\mathbf{X}=\left(\Omega,\mathcal{G},\mathcal{G}_{t},X_{t},\theta_{t},\CM\right)$
es un proceso $\mathcal{E}$-Markov continuo por la derecha simple,
con espacio de estados $E$ y semigrupo de transici\'on $\SG$ en
caso de que $\mathbf{X}$ satisfaga las siguientes
condiciones:
\begin{itemize}
\item[i)] $\left(\Omega,\mathcal{G},\mathcal{G}_{t}\right)$ es un
espacio de medida filtrado, y $X_{t}$ es un proceso $E$-valuado
continuo por la derecha $\mathcal{E}^{*}$-adaptado a
$\left(\mathcal{G}_{t}\right)$;

\item[ii)] $\left(\theta_{t}\right)_{t\geq0}$ es una colecci\'on
de operadores {\em shift} para $X$, es decir, mapea $\Omega$ en
s\'i mismo satisfaciendo para $t,s\geq0$,

\begin{equation}\label{Eq.Shift}
\theta_{t}\circ\theta_{s}=\theta_{t+s}\textrm{ y
}X_{t}\circ\theta_{t}=X_{t+s};
\end{equation}

\item[iii)] Para cualquier $x\in E$,$\CM\left\{X_{0}=x\right\}=1$,
y el proceso $\PE$ tiene la propiedad de Markov (\ref{Eq.1.8.S})
con semigrupo de transici\'on $\SG$ relativo a
$\left(\Omega,\mathcal{G},\mathcal{G}_{t},\CM\right)$.
\end{itemize}
\end{Def}


\begin{Def}[HD2]\label{Eq.2.2.S}
Para cualquier $\alpha>0$ y cualquier $f\in S^{\alpha}$, el
proceso $t\rightarrow f\left(X_{t}\right)$ es continuo por la
derecha casi seguramente.
\end{Def}

\begin{Def}\label{Def.PD}
Un sistema
$\mathbf{X}=\left(\Omega,\mathcal{G},\mathcal{G}_{t},X_{t},\theta_{t},\CM\right)$
es un proceso derecho en el espacio de Rad\'on $E$ con semigrupo
de transici\'on $\SG$ provisto de:
\begin{itemize}
\item[i)] $\mathbf{X}$ es una realizaci\'on  continua por la
derecha, \ref{Def.2.2.S}, de $\SG$.

\item[ii)] $\mathbf{X}$ satisface la condicion HD2,
\ref{Eq.2.2.S}, relativa a $\mathcal{G}_{t}$.

\item[iii)] $\mathcal{G}_{t}$ es aumentado y continuo por la
derecha.
\end{itemize}
\end{Def}


\begin{Def}
Sea $X$ un conjunto y $\mathcal{F}$ una $\sigma$-\'algebra de
subconjuntos de $X$, la pareja $\left(X,\mathcal{F}\right)$ es
llamado espacio medible. Un subconjunto $A$ de $X$ es llamado
medible, o medible con respecto a $\mathcal{F}$, si
$A\in\mathcal{F}$.
\end{Def}

\begin{Def}
Sea $\left(X,\mathcal{F},\mu\right)$ espacio de medida. Se dice
que la medida $\mu$ es $\sigma$-finita si se puede escribir
$X=\bigcup_{n\geq1}X_{n}$ con $X_{n}\in\mathcal{F}$ y
$\mu\left(X_{n}\right)<\infty$.
\end{Def}

\begin{Def}\label{Cto.Borel}
Sea $X$ el conjunto de los \'umeros reales $\rea$. El \'algebra de
Borel es la $\sigma$-\'algebra $B$ generada por los intervalos
abiertos $\left(a,b\right)\in\rea$. Cualquier conjunto en $B$ es
llamado {\em Conjunto de Borel}.
\end{Def}

\begin{Def}\label{Funcion.Medible}
Una funci\'on $f:X\rightarrow\rea$, es medible si para cualquier
n\'umero real $\alpha$ el conjunto
\[\left\{x\in X:f\left(x\right)>\alpha\right\}\]
pertenece a $X$. Equivalentemente, se dice que $f$ es medible si
\[f^{-1}\left(\left(\alpha,\infty\right)\right)=\left\{x\in X:f\left(x\right)>\alpha\right\}\in\mathcal{F}.\]
\end{Def}


\begin{Def}\label{Def.Cilindros}
Sean $\left(\Omega_{i},\mathcal{F}_{i}\right)$, $i=1,2,\ldots,$
espacios medibles y $\Omega=\prod_{i=1}^{\infty}\Omega_{i}$ el
conjunto de todas las sucesiones
$\left(\omega_{1},\omega_{2},\ldots,\right)$ tales que
$\omega_{i}\in\Omega_{i}$, $i=1,2,\ldots,$. Si
$B^{n}\subset\prod_{i=1}^{\infty}\Omega_{i}$, definimos
$B_{n}=\left\{\omega\in\Omega:\left(\omega_{1},\omega_{2},\ldots,\omega_{n}\right)\in
B^{n}\right\}$. Al conjunto $B_{n}$ se le llama {\em cilindro} con
base $B^{n}$, el cilindro es llamado medible si
$B^{n}\in\prod_{i=1}^{\infty}\mathcal{F}_{i}$.
\end{Def}


\begin{Def}\label{Def.Proc.Adaptado}[TSP, Ash \cite{RBA}]
Sea $X\left(t\right),t\geq0$ proceso estoc\'astico, el proceso es
adaptado a la familia de $\sigma$-\'algebras $\mathcal{F}_{t}$,
para $t\geq0$, si para $s<t$ implica que
$\mathcal{F}_{s}\subset\mathcal{F}_{t}$, y $X\left(t\right)$ es
$\mathcal{F}_{t}$-medible para cada $t$. Si no se especifica
$\mathcal{F}_{t}$ entonces se toma $\mathcal{F}_{t}$ como
$\mathcal{F}\left(X\left(s\right),s\leq t\right)$, la m\'as
peque\~na $\sigma$-\'algebra de subconjuntos de $\Omega$ que hace
que cada $X\left(s\right)$, con $s\leq t$ sea Borel medible.
\end{Def}


\begin{Def}\label{Def.Tiempo.Paro}[TSP, Ash \cite{RBA}]
Sea $\left\{\mathcal{F}\left(t\right),t\geq0\right\}$ familia
creciente de sub $\sigma$-\'algebras. es decir,
$\mathcal{F}\left(s\right)\subset\mathcal{F}\left(t\right)$ para
$s\leq t$. Un tiempo de paro para $\mathcal{F}\left(t\right)$ es
una funci\'on $T:\Omega\rightarrow\left[0,\infty\right]$ tal que
$\left\{T\leq t\right\}\in\mathcal{F}\left(t\right)$ para cada
$t\geq0$. Un tiempo de paro para el proceso estoc\'astico
$X\left(t\right),t\geq0$ es un tiempo de paro para las
$\sigma$-\'algebras
$\mathcal{F}\left(t\right)=\mathcal{F}\left(X\left(s\right)\right)$.
\end{Def}

\begin{Def}
Sea $X\left(t\right),t\geq0$ proceso estoc\'astico, con
$\left(S,\chi\right)$ espacio de estados. Se dice que el proceso
es adaptado a $\left\{\mathcal{F}\left(t\right)\right\}$, es
decir, si para cualquier $s,t\in I$, $I$ conjunto de \'indices,
$s<t$, se tiene que
$\mathcal{F}\left(s\right)\subset\mathcal{F}\left(t\right)$ y
$X\left(t\right)$ es $\mathcal{F}\left(t\right)$-medible,
\end{Def}

\begin{Def}
Sea $X\left(t\right),t\geq0$ proceso estoc\'astico, se dice que es
un Proceso de Markov relativo a $\mathcal{F}\left(t\right)$ o que
$\left\{X\left(t\right),\mathcal{F}\left(t\right)\right\}$ es de
Markov si y s\'olo si para cualquier conjunto $B\in\chi$,  y
$s,t\in I$, $s<t$ se cumple que
\begin{equation}\label{Prop.Markov}
P\left\{X\left(t\right)\in
B|\mathcal{F}\left(s\right)\right\}=P\left\{X\left(t\right)\in
B|X\left(s\right)\right\}.
\end{equation}
\end{Def}
\begin{Note}
Si se dice que $\left\{X\left(t\right)\right\}$ es un Proceso de
Markov sin mencionar $\mathcal{F}\left(t\right)$, se asumir\'a que
\begin{eqnarray*}
\mathcal{F}\left(t\right)=\mathcal{F}_{0}\left(t\right)=\mathcal{F}\left(X\left(r\right),r\leq
t\right),
\end{eqnarray*}
entonces la ecuaci\'on (\ref{Prop.Markov}) se puede escribir como
\begin{equation}
P\left\{X\left(t\right)\in B|X\left(r\right),r\leq s\right\} =
P\left\{X\left(t\right)\in B|X\left(s\right)\right\}
\end{equation}
\end{Note}

\begin{Teo}
Sea $\left(X_{n},\mathcal{F}_{n},n=0,1,\ldots,\right\}$ Proceso de
Markov con espacio de estados $\left(S_{0},\chi_{0}\right)$
generado por una distribuici\'on inicial $P_{o}$ y probabilidad de
transici\'on $p_{mn}$, para $m,n=0,1,\ldots,$ $m<n$, que por
notaci\'on se escribir\'a como $p\left(m,n,x,B\right)\rightarrow
p_{mn}\left(x,B\right)$. Sea $S$ tiempo de paro relativo a la
$\sigma$-\'algebra $\mathcal{F}_{n}$. Sea $T$ funci\'on medible,
$T:\Omega\rightarrow\left\{0,1,\ldots,\right\}$. Sup\'ongase que
$T\geq S$, entonces $T$ es tiempo de paro. Si $B\in\chi_{0}$,
entonces
\begin{equation}\label{Prop.Fuerte.Markov}
P\left\{X\left(T\right)\in
B,T<\infty|\mathcal{F}\left(S\right)\right\} =
p\left(S,T,X\left(s\right),B\right)
\end{equation}
en $\left\{T<\infty\right\}$.
\end{Teo}


Sea $K$ conjunto numerable y sea $d:K\rightarrow\nat$ funci\'on.
Para $v\in K$, $M_{v}$ es un conjunto abierto de
$\rea^{d\left(v\right)}$. Entonces \[E=\cup_{v\in
K}M_{v}=\left\{\left(v,\zeta\right):v\in K,\zeta\in
M_{v}\right\}.\]

Sea $\mathcal{E}$ la clase de conjuntos medibles en $E$:
\[\mathcal{E}=\left\{\cup_{v\in K}A_{v}:A_{v}\in \mathcal{M}_{v}\right\}.\]

donde $\mathcal{M}$ son los conjuntos de Borel de $M_{v}$.
Entonces $\left(E,\mathcal{E}\right)$ es un espacio de Borel. El
estado del proceso se denotar\'a por
$\mathbf{x}_{t}=\left(v_{t},\zeta_{t}\right)$. La distribuci\'on
de $\left(\mathbf{x}_{t}\right)$ est\'a determinada por por los
siguientes objetos:

\begin{itemize}
\item[i)] Los campos vectoriales $\left(\mathcal{H}_{v},v\in
K\right)$. \item[ii)] Una funci\'on medible $\lambda:E\rightarrow
\rea_{+}$. \item[iii)] Una medida de transici\'on
$Q:\mathcal{E}\times\left(E\cup\Gamma^{*}\right)\rightarrow\left[0,1\right]$
donde
\begin{equation}
\Gamma^{*}=\cup_{v\in K}\partial^{*}M_{v}.
\end{equation}
y
\begin{equation}
\partial^{*}M_{v}=\left\{z\in\partial M_{v}:\mathbf{\mathbf{\phi}_{v}\left(t,\zeta\right)=\mathbf{z}}\textrm{ para alguna }\left(t,\zeta\right)\in\rea_{+}\times M_{v}\right\}.
\end{equation}
$\partial M_{v}$ denota  la frontera de $M_{v}$.
\end{itemize}

El campo vectorial $\left(\mathcal{H}_{v},v\in K\right)$ se supone
tal que para cada $\mathbf{z}\in M_{v}$ existe una \'unica curva
integral $\mathbf{\phi}_{v}\left(t,\zeta\right)$ que satisface la
ecuaci\'on

\begin{equation}
\frac{d}{dt}f\left(\zeta_{t}\right)=\mathcal{H}f\left(\zeta_{t}\right),
\end{equation}
con $\zeta_{0}=\mathbf{z}$, para cualquier funci\'on suave
$f:\rea^{d}\rightarrow\rea$ y $\mathcal{H}$ denota el operador
diferencial de primer orden, con $\mathcal{H}=\mathcal{H}_{v}$ y
$\zeta_{t}=\mathbf{\phi}\left(t,\mathbf{z}\right)$. Adem\'as se
supone que $\mathcal{H}_{v}$ es conservativo, es decir, las curvas
integrales est\'an definidas para todo $t>0$.

Para $\mathbf{x}=\left(v,\zeta\right)\in E$ se denota
\[t^{*}\mathbf{x}=inf\left\{t>0:\mathbf{\phi}_{v}\left(t,\zeta\right)\in\partial^{*}M_{v}\right\}\]

En lo que respecta a la funci\'on $\lambda$, se supondr\'a que
para cada $\left(v,\zeta\right)\in E$ existe un $\epsilon>0$ tal
que la funci\'on
$s\rightarrow\lambda\left(v,\phi_{v}\left(s,\zeta\right)\right)\in
E$ es integrable para $s\in\left[0,\epsilon\right)$. La medida de
transici\'on $Q\left(A;\mathbf{x}\right)$ es una funci\'on medible
de $\mathbf{x}$ para cada $A\in\mathcal{E}$, definida para
$\mathbf{x}\in E\cup\Gamma^{*}$ y es una medida de probabilidad en
$\left(E,\mathcal{E}\right)$ para cada $\mathbf{x}\in E$.

El movimiento del proceso $\left(\mathbf{x}_{t}\right)$ comenzando
en $\mathbf{x}=\left(n,\mathbf{z}\right)\in E$ se puede construir
de la siguiente manera, def\'inase la funci\'on $F$ por

\begin{equation}
F\left(t\right)=\left\{\begin{array}{ll}\\
exp\left(-\int_{0}^{t}\lambda\left(n,\phi_{n}\left(s,\mathbf{z}\right)\right)ds\right), & t<t^{*}\left(\mathbf{x}\right),\\
0, & t\geq t^{*}\left(\mathbf{x}\right)
\end{array}\right.
\end{equation}

Sea $T_{1}$ una variable aleatoria tal que
$\prob\left[T_{1}>t\right]=F\left(t\right)$, ahora sea la variable
aleatoria $\left(N,Z\right)$ con distribuici\'on
$Q\left(\cdot;\phi_{n}\left(T_{1},\mathbf{z}\right)\right)$. La
trayectoria de $\left(\mathbf{x}_{t}\right)$ para $t\leq T_{1}$
es\footnote{Revisar p\'agina 362, y 364 de Davis \cite{Davis}.}
\begin{eqnarray*}
\mathbf{x}_{t}=\left(v_{t},\zeta_{t}\right)=\left\{\begin{array}{ll}
\left(n,\phi_{n}\left(t,\mathbf{z}\right)\right), & t<T_{1},\\
\left(N,\mathbf{Z}\right), & t=t_{1}.
\end{array}\right.
\end{eqnarray*}

Comenzando en $\mathbf{x}_{T_{1}}$ se selecciona el siguiente
tiempo de intersalto $T_{2}-T_{1}$ lugar del post-salto
$\mathbf{x}_{T_{2}}$ de manera similar y as\'i sucesivamente. Este
procedimiento nos da una trayectoria determinista por partes
$\mathbf{x}_{t}$ con tiempos de salto $T_{1},T_{2},\ldots$. Bajo
las condiciones enunciadas para $\lambda,T_{1}>0$  y
$T_{1}-T_{2}>0$ para cada $i$, con probabilidad 1. Se supone que
se cumple la siquiente condici\'on.

\begin{Sup}[Supuesto 3.1, Davis \cite{Davis}]\label{Sup3.1.Davis}
Sea $N_{t}:=\sum_{t}\indora_{\left(t\geq t\right)}$ el n\'umero de
saltos en $\left[0,t\right]$. Entonces
\begin{equation}
\esp\left[N_{t}\right]<\infty\textrm{ para toda }t.
\end{equation}
\end{Sup}

es un proceso de Markov, m\'as a\'un, es un Proceso Fuerte de
Markov, es decir, la Propiedad Fuerte de Markov se cumple para
cualquier tiempo de paro.
%_________________________________________________________________________

En esta secci\'on se har\'an las siguientes consideraciones: $E$
es un espacio m\'etrico separable y la m\'etrica $d$ es compatible
con la topolog\'ia.


\begin{Def}
Un espacio topol\'ogico $E$ es llamado {\em Luisin} si es
homeomorfo a un subconjunto de Borel de un espacio m\'etrico
compacto.
\end{Def}

\begin{Def}
Un espacio topol\'ogico $E$ es llamado de {\em Rad\'on} si es
homeomorfo a un subconjunto universalmente medible de un espacio
m\'etrico compacto.
\end{Def}

Equivalentemente, la definici\'on de un espacio de Rad\'on puede
encontrarse en los siguientes t\'erminos:


\begin{Def}
$E$ es un espacio de Rad\'on si cada medida finita en
$\left(E,\mathcal{B}\left(E\right)\right)$ es regular interior o cerrada,
{\em tight}.
\end{Def}

\begin{Def}
Una medida finita, $\lambda$ en la $\sigma$-\'algebra de Borel de
un espacio metrizable $E$ se dice cerrada si
\begin{equation}\label{Eq.A2.3}
\lambda\left(E\right)=sup\left\{\lambda\left(K\right):K\textrm{ es
compacto en }E\right\}.
\end{equation}
\end{Def}

El siguiente teorema nos permite tener una mejor caracterizaci\'on de los espacios de Rad\'on:
\begin{Teo}\label{Tma.A2.2}
Sea $E$ espacio separable metrizable. Entonces $E$ es Radoniano si y s\'olo s\'i cada medida finita en $\left(E,\mathcal{B}\left(E\right)\right)$ es cerrada.
\end{Teo}

%_________________________________________________________________________________________
\subsection{Propiedades de Markov}
%_________________________________________________________________________________________

Sea $E$ espacio de estados, tal que $E$ es un espacio de Rad\'on, $\mathcal{B}\left(E\right)$ $\sigma$-\'algebra de Borel en $E$, que se denotar\'a por $\mathcal{E}$.

Sea $\left(X,\mathcal{G},\prob\right)$ espacio de probabilidad, $I\subset\rea$ conjunto de índices. Sea $\mathcal{F}_{\leq t}$ la $\sigma$-\'algebra natural definida como $\sigma\left\{f\left(X_{r}\right):r\in I, rleq t,f\in\mathcal{E}\right\}$. Se considerar\'a una $\sigma$-\'algebra m\'as general, $ \left(\mathcal{G}_{t}\right)$ tal que $\left(X_{t}\right)$ sea $\mathcal{E}$-adaptado.

\begin{Def}
Una familia $\left(P_{s,t}\right)$ de kernels de Markov en $\left(E,\mathcal{E}\right)$ indexada por pares $s,t\in I$, con $s\leq t$ es una funci\'on de transici\'on en $\ER$, si  para todo $r\leq s< t$ en $I$ y todo $x\in E$, $B\in\mathcal{E}$
\begin{equation}\label{Eq.Kernels}
P_{r,t}\left(x,B\right)=\int_{E}P_{r,s}\left(x,dy\right)P_{s,t}\left(y,B\right)\footnote{Ecuaci\'on de Chapman-Kolmogorov}.
\end{equation}
\end{Def}

Se dice que la funci\'on de transici\'on $\KM$ en $\ER$ es la funci\'on de transici\'on para un proceso $\PE$  con valores en $E$ y que satisface la propiedad de Markov\footnote{\begin{equation}\label{Eq.1.4.S}
\prob\left\{H|\mathcal{G}_{t}\right\}=\prob\left\{H|X_{t}\right\}\textrm{ }H\in p\mathcal{F}_{\geq t}.
\end{equation}} (\ref{Eq.1.4.S}) relativa a $\left(\mathcal{G}_{t}\right)$ si 

\begin{equation}\label{Eq.1.6.S}
\prob\left\{f\left(X_{t}\right)|\mathcal{G}_{s}\right\}=P_{s,t}f\left(X_{t}\right)\textrm{ }s\leq t\in I,\textrm{ }f\in b\mathcal{E}.
\end{equation}

\begin{Def}
Una familia $\left(P_{t}\right)_{t\geq0}$ de kernels de Markov en $\ER$ es llamada {\em Semigrupo de Transici\'on de Markov} o {\em Semigrupo de Transici\'on} si
\[P_{t+s}f\left(x\right)=P_{t}\left(P_{s}f\right)\left(x\right),\textrm{ }t,s\geq0,\textrm{ }x\in E\textrm{ }f\in b\mathcal{E}.\]
\end{Def}
\begin{Note}
Si la funci\'on de transici\'on $\KM$ es llamada homog\'enea si $P_{s,t}=P_{t-s}$.
\end{Note}

Un proceso de Markov que satisface la ecuaci\'on (\ref{Eq.1.6.S}) con funci\'on de transici\'on homog\'enea $\left(P_{t}\right)$ tiene la propiedad caracter\'istica
\begin{equation}\label{Eq.1.8.S}
\prob\left\{f\left(X_{t+s}\right)|\mathcal{G}_{t}\right\}=P_{s}f\left(X_{t}\right)\textrm{ }t,s\geq0,\textrm{ }f\in b\mathcal{E}.
\end{equation}
La ecuaci\'on anterior es la {\em Propiedad Simple de Markov} de $X$ relativa a $\left(P_{t}\right)$.

En este sentido el proceso $\PE$ cumple con la propiedad de Markov (\ref{Eq.1.8.S}) relativa a $\left(\Omega,\mathcal{G},\mathcal{G}_{t},\prob\right)$ con semigrupo de transici\'on $\left(P_{t}\right)$.
%_________________________________________________________________________________________
\subsection{Primer Condici\'on de Regularidad}
%_________________________________________________________________________________________
%\newcommand{\EM}{\left(\Omega,\mathcal{G},\prob\right)}
%\newcommand{\E4}{\left(\Omega,\mathcal{G},\mathcal{G}_{t},\prob\right)}
\begin{Def}
Un proceso estoc\'astico $\PE$ definido en $\left(\Omega,\mathcal{G},\prob\right)$ con valores en el espacio topol\'ogico $E$ es continuo por la derecha si cada trayectoria muestral $t\rightarrow X_{t}\left(w\right)$ es un mapeo continuo por la derecha de $I$ en $E$.
\end{Def}

\begin{Def}[HD1]\label{Eq.2.1.S}
Un semigrupo de Markov $\left/P_{t}\right)$ en un espacio de Rad\'on $E$ se dice que satisface la condici\'on {\em HD1} si, dada una medida de probabilidad $\mu$ en $E$, existe una $\sigma$-\'algebra $\mathcal{E^{*}}$ con $\mathcal{E}\subset\mathcal{E}$ y $P_{t}\left(b\mathcal{E}^{*}\right)\subset b\mathcal{E}^{*}$, y un $\mathcal{E}^{*}$-proceso $E$-valuado continuo por la derecha $\PE$ en alg\'un espacio de probabilidad filtrado $\left(\Omega,\mathcal{G},\mathcal{G}_{t},\prob\right)$ tal que $X=\left(\Omega,\mathcal{G},\mathcal{G}_{t},\prob\right)$ es de Markov (Homog\'eneo) con semigrupo de transici\'on $(P_{t})$ y distribuci\'on inicial $\mu$.
\end{Def}

Considerese la colecci\'on de variables aleatorias $X_{t}$ definidas en alg\'un espacio de probabilidad, y una colecci\'on de medidas $\mathbf{P}^{x}$ tales que $\mathbf{P}^{x}\left\{X_{0}=x\right\}$, y bajo cualquier $\mathbf{P}^{x}$, $X_{t}$ es de Markov con semigrupo $\left(P_{t}\right)$. $\mathbf{P}^{x}$ puede considerarse como la distribuci\'on condicional de $\mathbf{P}$ dado $X_{0}=x$.

\begin{Def}\label{Def.2.2.S}
Sea $E$ espacio de Rad\'on, $\SG$ semigrupo de Markov en $\ER$. La colecci\'on $\mathbf{X}=\left(\Omega,\mathcal{G},\mathcal{G}_{t},X_{t},\theta_{t},\CM\right)$ es un proceso $\mathcal{E}$-Markov continuo por la derecha simple, con espacio de estados $E$ y semigrupo de transici\'on $\SG$ en caso de que $\mathbf{X}$ satisfaga las siguientes condiciones:
\begin{itemize}
\item[i)] $\left(\Omega,\mathcal{G},\mathcal{G}_{t}\right)$ es un espacio de medida filtrado, y $X_{t}$ es un proceso $E$-valuado continuo por la derecha $\mathcal{E}^{*}$-adaptado a $\left(\mathcal{G}_{t}\right)$;

\item[ii)] $\left(\theta_{t}\right)_{t\geq0}$ es una colecci\'on de operadores {\em shift} para $X$, es decir, mapea $\Omega$ en s\'i mismo satisfaciendo para $t,s\geq0$,

\begin{equation}\label{Eq.Shift}
\theta_{t}\circ\theta_{s}=\theta_{t+s}\textrm{ y }X_{t}\circ\theta_{t}=X_{t+s};
\end{equation}

\item[iii)] Para cualquier $x\in E$,$\CM\left\{X_{0}=x\right\}=1$, y el proceso $\PE$ tiene la propiedad de Markov (\ref{Eq.1.8.S}) con semigrupo de transici\'on $\SG$ relativo a $\left(\Omega,\mathcal{G},\mathcal{G}_{t},\CM\right)$.
\end{itemize}
\end{Def}

\begin{Def}[HD2]\label{Eq.2.2.S}
Para cualquier $\alpha>0$ y cualquier $f\in S^{\alpha}$, el proceso $t\rightarrow f\left(X_{t}\right)$ es continuo por la derecha casi seguramente.
\end{Def}

\begin{Def}\label{Def.PD}
Un sistema $\mathbf{X}=\left(\Omega,\mathcal{G},\mathcal{G}_{t},X_{t},\theta_{t},\CM\right)$ es un proceso derecho en el espacio de Rad\'on $E$ con semigrupo de transici\'on $\SG$ provisto de:
\begin{itemize}
\item[i)] $\mathbf{X}$ es una realizaci\'on  continua por la derecha, \ref{Def.2.2.S}, de $\SG$.

\item[ii)] $\mathbf{X}$ satisface la condicion HD2, \ref{Eq.2.2.S}, relativa a $\mathcal{G}_{t}$.

\item[iii)] $\mathcal{G}_{t}$ es aumentado y continuo por la derecha.
\end{itemize}
\end{Def}



\begin{Lema}[Lema 4.2, Dai\cite{Dai}]\label{Lema4.2}
Sea $\left\{x_{n}\right\}\subset \mathbf{X}$ con
$|x_{n}|\rightarrow\infty$, conforme $n\rightarrow\infty$. Suponga
que
\[lim_{n\rightarrow\infty}\frac{1}{|x_{n}|}U\left(0\right)=\overline{U}\]
y
\[lim_{n\rightarrow\infty}\frac{1}{|x_{n}|}V\left(0\right)=\overline{V}.\]

Entonces, conforme $n\rightarrow\infty$, casi seguramente

\begin{equation}\label{E1.4.2}
\frac{1}{|x_{n}|}\Phi^{k}\left(\left[|x_{n}|t\right]\right)\rightarrow
P_{k}^{'}t\textrm{, u.o.c.,}
\end{equation}

\begin{equation}\label{E1.4.3}
\frac{1}{|x_{n}|}E^{x_{n}}_{k}\left(|x_{n}|t\right)\rightarrow
\alpha_{k}\left(t-\overline{U}_{k}\right)^{+}\textrm{, u.o.c.,}
\end{equation}

\begin{equation}\label{E1.4.4}
\frac{1}{|x_{n}|}S^{x_{n}}_{k}\left(|x_{n}|t\right)\rightarrow
\mu_{k}\left(t-\overline{V}_{k}\right)^{+}\textrm{, u.o.c.,}
\end{equation}

donde $\left[t\right]$ es la parte entera de $t$ y
$\mu_{k}=1/m_{k}=1/\esp\left[\eta_{k}\left(1\right)\right]$.
\end{Lema}

\begin{Lema}[Lema 4.3, Dai\cite{Dai}]\label{Lema.4.3}
Sea $\left\{x_{n}\right\}\subset \mathbf{X}$ con
$|x_{n}|\rightarrow\infty$, conforme $n\rightarrow\infty$. Suponga
que
\[lim_{n\rightarrow\infty}\frac{1}{|x_{n}|}U\left(0\right)=\overline{U}_{k}\]
y
\[lim_{n\rightarrow\infty}\frac{1}{|x_{n}|}V\left(0\right)=\overline{V}_{k}.\]
\begin{itemize}
\item[a)] Conforme $n\rightarrow\infty$ casi seguramente,
\[lim_{n\rightarrow\infty}\frac{1}{|x_{n}|}U^{x_{n}}_{k}\left(|x_{n}|t\right)=\left(\overline{U}_{k}-t\right)^{+}\textrm{, u.o.c.}\]
y
\[lim_{n\rightarrow\infty}\frac{1}{|x_{n}|}V^{x_{n}}_{k}\left(|x_{n}|t\right)=\left(\overline{V}_{k}-t\right)^{+}.\]

\item[b)] Para cada $t\geq0$ fijo,
\[\left\{\frac{1}{|x_{n}|}U^{x_{n}}_{k}\left(|x_{n}|t\right),|x_{n}|\geq1\right\}\]
y
\[\left\{\frac{1}{|x_{n}|}V^{x_{n}}_{k}\left(|x_{n}|t\right),|x_{n}|\geq1\right\}\]
\end{itemize}
son uniformemente convergentes.
\end{Lema}

$S_{l}^{x}\left(t\right)$ es el n\'umero total de servicios
completados de la clase $l$, si la clase $l$ est\'a dando $t$
unidades de tiempo de servicio. Sea $T_{l}^{x}\left(x\right)$ el
monto acumulado del tiempo de servicio que el servidor
$s\left(l\right)$ gasta en los usuarios de la clase $l$ al tiempo
$t$. Entonces $S_{l}^{x}\left(T_{l}^{x}\left(t\right)\right)$ es
el n\'umero total de servicios completados para la clase $l$ al
tiempo $t$. Una fracci\'on de estos usuarios,
$\Phi_{l}^{x}\left(S_{l}^{x}\left(T_{l}^{x}\left(t\right)\right)\right)$,
se convierte en usuarios de la clase $k$.\\

Entonces, dado lo anterior, se tiene la siguiente representaci\'on
para el proceso de la longitud de la cola:\\

\begin{equation}
Q_{k}^{x}\left(t\right)=_{k}^{x}\left(0\right)+E_{k}^{x}\left(t\right)+\sum_{l=1}^{K}\Phi_{k}^{l}\left(S_{l}^{x}\left(T_{l}^{x}\left(t\right)\right)\right)-S_{k}^{x}\left(T_{k}^{x}\left(t\right)\right)
\end{equation}
para $k=1,\ldots,K$. Para $i=1,\ldots,d$, sea
\[I_{i}^{x}\left(t\right)=t-\sum_{j\in C_{i}}T_{k}^{x}\left(t\right).\]

Entonces $I_{i}^{x}\left(t\right)$ es el monto acumulado del
tiempo que el servidor $i$ ha estado desocupado al tiempo $t$. Se
est\'a asumiendo que las disciplinas satisfacen la ley de
conservaci\'on del trabajo, es decir, el servidor $i$ est\'a en
pausa solamente cuando no hay usuarios en la estaci\'on $i$.
Entonces, se tiene que

\begin{equation}
\int_{0}^{\infty}\left(\sum_{k\in
C_{i}}Q_{k}^{x}\left(t\right)\right)dI_{i}^{x}\left(t\right)=0,
\end{equation}
para $i=1,\ldots,d$.\\

Hacer
\[T^{x}\left(t\right)=\left(T_{1}^{x}\left(t\right),\ldots,T_{K}^{x}\left(t\right)\right)^{'},\]
\[I^{x}\left(t\right)=\left(I_{1}^{x}\left(t\right),\ldots,I_{K}^{x}\left(t\right)\right)^{'}\]
y
\[S^{x}\left(T^{x}\left(t\right)\right)=\left(S_{1}^{x}\left(T_{1}^{x}\left(t\right)\right),\ldots,S_{K}^{x}\left(T_{K}^{x}\left(t\right)\right)\right)^{'}.\]

Para una disciplina que cumple con la ley de conservaci\'on del
trabajo, en forma vectorial, se tiene el siguiente conjunto de
ecuaciones

\begin{equation}\label{Eq.MF.1.3}
Q^{x}\left(t\right)=Q^{x}\left(0\right)+E^{x}\left(t\right)+\sum_{l=1}^{K}\Phi^{l}\left(S_{l}^{x}\left(T_{l}^{x}\left(t\right)\right)\right)-S^{x}\left(T^{x}\left(t\right)\right),\\
\end{equation}

\begin{equation}\label{Eq.MF.2.3}
Q^{x}\left(t\right)\geq0,\\
\end{equation}

\begin{equation}\label{Eq.MF.3.3}
T^{x}\left(0\right)=0,\textrm{ y }\overline{T}^{x}\left(t\right)\textrm{ es no decreciente},\\
\end{equation}

\begin{equation}\label{Eq.MF.4.3}
I^{x}\left(t\right)=et-CT^{x}\left(t\right)\textrm{ es no
decreciente}\\
\end{equation}

\begin{equation}\label{Eq.MF.5.3}
\int_{0}^{\infty}\left(CQ^{x}\left(t\right)\right)dI_{i}^{x}\left(t\right)=0,\\
\end{equation}

\begin{equation}\label{Eq.MF.6.3}
\textrm{Condiciones adicionales en
}\left(\overline{Q}^{x}\left(\cdot\right),\overline{T}^{x}\left(\cdot\right)\right)\textrm{
espec\'ificas de la disciplina de la cola,}
\end{equation}

donde $e$ es un vector de unos de dimensi\'on $d$, $C$ es la
matriz definida por
\[C_{ik}=\left\{\begin{array}{cc}
1,& S\left(k\right)=i,\\
0,& \textrm{ en otro caso}.\\
\end{array}\right.
\]
Es necesario enunciar el siguiente Teorema que se utilizar\'a para
el Teorema \ref{Tma.4.2.Dai}:
\begin{Teo}[Teorema 4.1, Dai \cite{Dai}]
Considere una disciplina que cumpla la ley de conservaci\'on del
trabajo, para casi todas las trayectorias muestrales $\omega$ y
cualquier sucesi\'on de estados iniciales
$\left\{x_{n}\right\}\subset \mathbf{X}$, con
$|x_{n}|\rightarrow\infty$, existe una subsucesi\'on
$\left\{x_{n_{j}}\right\}$ con $|x_{n_{j}}|\rightarrow\infty$ tal
que
\begin{equation}\label{Eq.4.15}
\frac{1}{|x_{n_{j}}|}\left(Q^{x_{n_{j}}}\left(0\right),U^{x_{n_{j}}}\left(0\right),V^{x_{n_{j}}}\left(0\right)\right)\rightarrow\left(\overline{Q}\left(0\right),\overline{U},\overline{V}\right),
\end{equation}

\begin{equation}\label{Eq.4.16}
\frac{1}{|x_{n_{j}}|}\left(Q^{x_{n_{j}}}\left(|x_{n_{j}}|t\right),T^{x_{n_{j}}}\left(|x_{n_{j}}|t\right)\right)\rightarrow\left(\overline{Q}\left(t\right),\overline{T}\left(t\right)\right)\textrm{
u.o.c.}
\end{equation}

Adem\'as,
$\left(\overline{Q}\left(t\right),\overline{T}\left(t\right)\right)$
satisface las siguientes ecuaciones:
\begin{equation}\label{Eq.MF.1.3a}
\overline{Q}\left(t\right)=Q\left(0\right)+\left(\alpha
t-\overline{U}\right)^{+}-\left(I-P\right)^{'}M^{-1}\left(\overline{T}\left(t\right)-\overline{V}\right)^{+},
\end{equation}

\begin{equation}\label{Eq.MF.2.3a}
\overline{Q}\left(t\right)\geq0,\\
\end{equation}

\begin{equation}\label{Eq.MF.3.3a}
\overline{T}\left(t\right)\textrm{ es no decreciente y comienza en cero},\\
\end{equation}

\begin{equation}\label{Eq.MF.4.3a}
\overline{I}\left(t\right)=et-C\overline{T}\left(t\right)\textrm{
es no decreciente,}\\
\end{equation}

\begin{equation}\label{Eq.MF.5.3a}
\int_{0}^{\infty}\left(C\overline{Q}\left(t\right)\right)d\overline{I}\left(t\right)=0,\\
\end{equation}

\begin{equation}\label{Eq.MF.6.3a}
\textrm{Condiciones adicionales en
}\left(\overline{Q}\left(\cdot\right),\overline{T}\left(\cdot\right)\right)\textrm{
especficas de la disciplina de la cola,}
\end{equation}
\end{Teo}

\begin{Def}[Definici\'on 4.1, , Dai \cite{Dai}]
Sea una disciplina de servicio espec\'ifica. Cualquier l\'imite
$\left(\overline{Q}\left(\cdot\right),\overline{T}\left(\cdot\right)\right)$
en \ref{Eq.4.16} es un {\em flujo l\'imite} de la disciplina.
Cualquier soluci\'on (\ref{Eq.MF.1.3a})-(\ref{Eq.MF.6.3a}) es
llamado flujo soluci\'on de la disciplina. Se dice que el modelo de flujo l\'imite, modelo de flujo, de la disciplina de la cola es estable si existe una constante
$\delta>0$ que depende de $\mu,\alpha$ y $P$ solamente, tal que
cualquier flujo l\'imite con
$|\overline{Q}\left(0\right)|+|\overline{U}|+|\overline{V}|=1$, se
tiene que $\overline{Q}\left(\cdot+\delta\right)\equiv0$.
\end{Def}

\begin{Teo}[Teorema 4.2, Dai\cite{Dai}]\label{Tma.4.2.Dai}
Sea una disciplina fija para la cola, suponga que se cumplen las
condiciones (1.2)-(1.5). Si el modelo de flujo l\'imite de la
disciplina de la cola es estable, entonces la cadena de Markov $X$
que describe la din\'amica de la red bajo la disciplina es Harris
recurrente positiva.
\end{Teo}

Ahora se procede a escalar el espacio y el tiempo para reducir la
aparente fluctuaci\'on del modelo. Consid\'erese el proceso
\begin{equation}\label{Eq.3.7}
\overline{Q}^{x}\left(t\right)=\frac{1}{|x|}Q^{x}\left(|x|t\right)
\end{equation}
A este proceso se le conoce como el fluido escalado, y cualquier l\'imite $\overline{Q}^{x}\left(t\right)$ es llamado flujo l\'imite del proceso de longitud de la cola. Haciendo $|q|\rightarrow\infty$ mientras se mantiene el resto de las componentes fijas, cualquier punto l\'imite del proceso de longitud de la cola normalizado $\overline{Q}^{x}$ es soluci\'on del siguiente modelo de flujo.

Al conjunto de ecuaciones dadas en \ref{Eq.3.8}-\ref{Eq.3.13} se
le llama {\em Modelo de flujo} y al conjunto de todas las
soluciones del modelo de flujo
$\left(\overline{Q}\left(\cdot\right),\overline{T}
\left(\cdot\right)\right)$ se le denotar\'a por $\mathcal{Q}$.

Si se hace $|x|\rightarrow\infty$ sin restringir ninguna de las
componentes, tambi\'en se obtienen un modelo de flujo, pero en
este caso el residual de los procesos de arribo y servicio
introducen un retraso:

\begin{Def}[Definici\'on 3.3, Dai y Meyn \cite{DaiSean}]
El modelo de flujo es estable si existe un tiempo fijo $t_{0}$ tal
que $\overline{Q}\left(t\right)=0$, con $t\geq t_{0}$, para
cualquier $\overline{Q}\left(\cdot\right)\in\mathcal{Q}$ que
cumple con $|\overline{Q}\left(0\right)|=1$.
\end{Def}

El siguiente resultado se encuentra en Chen \cite{Chen}.
\begin{Lemma}[Lema 3.1, Dai y Meyn \cite{DaiSean}]
Si el modelo de flujo definido por \ref{Eq.3.8}-\ref{Eq.3.13} es
estable, entonces el modelo de flujo retrasado es tambi\'en
estable, es decir, existe $t_{0}>0$ tal que
$\overline{Q}\left(t\right)=0$ para cualquier $t\geq t_{0}$, para
cualquier soluci\'on del modelo de flujo retrasado cuya
condici\'on inicial $\overline{x}$ satisface que
$|\overline{x}|=|\overline{Q}\left(0\right)|+|\overline{A}\left(0\right)|+|\overline{B}\left(0\right)|\leq1$.
\end{Lemma}


Propiedades importantes para el modelo de flujo retrasado:

\begin{Prop}
 Sea $\left(\overline{Q},\overline{T},\overline{T}^{0}\right)$ un flujo l\'imite de \ref{Eq.4.4} y suponga que cuando $x\rightarrow\infty$ a lo largo de
una subsucesi\'on
\[\left(\frac{1}{|x|}Q_{k}^{x}\left(0\right),\frac{1}{|x|}A_{k}^{x}\left(0\right),\frac{1}{|x|}B_{k}^{x}\left(0\right),\frac{1}{|x|}B_{k}^{x,0}\left(0\right)\right)\rightarrow\left(\overline{Q}_{k}\left(0\right),0,0,0\right)\]
para $k=1,\ldots,K$. EL flujo l\'imite tiene las siguientes
propiedades, donde las propiedades de la derivada se cumplen donde
la derivada exista:
\begin{itemize}
 \item[i)] Los vectores de tiempo ocupado $\overline{T}\left(t\right)$ y $\overline{T}^{0}\left(t\right)$ son crecientes y continuas con
$\overline{T}\left(0\right)=\overline{T}^{0}\left(0\right)=0$.
\item[ii)] Para todo $t\geq0$
\[\sum_{k=1}^{K}\left[\overline{T}_{k}\left(t\right)+\overline{T}_{k}^{0}\left(t\right)\right]=t\]
\item[iii)] Para todo $1\leq k\leq K$
\[\overline{Q}_{k}\left(t\right)=\overline{Q}_{k}\left(0\right)+\alpha_{k}t-\mu_{k}\overline{T}_{k}\left(t\right)\]
\item[iv)]  Para todo $1\leq k\leq K$
\[\dot{{\overline{T}}}_{k}\left(t\right)=\beta_{k}\] para $\overline{Q}_{k}\left(t\right)=0$.
\item[v)] Para todo $k,j$
\[\mu_{k}^{0}\overline{T}_{k}^{0}\left(t\right)=\mu_{j}^{0}\overline{T}_{j}^{0}\left(t\right)\]
\item[vi)]  Para todo $1\leq k\leq K$
\[\mu_{k}\dot{{\overline{T}}}_{k}\left(t\right)=l_{k}\mu_{k}^{0}\dot{{\overline{T}}}_{k}^{0}\left(t\right)\] para $\overline{Q}_{k}\left(t\right)>0$.
\end{itemize}
\end{Prop}

\begin{Lema}[Lema 3.1 \cite{Chen}]\label{Lema3.1}
Si el modelo de flujo es estable, definido por las ecuaciones
(3.8)-(3.13), entonces el modelo de flujo retrasado tambin es
estable.
\end{Lema}

\begin{Teo}[Teorema 5.2 \cite{Chen}]\label{Tma.5.2}
Si el modelo de flujo lineal correspondiente a la red de cola es
estable, entonces la red de colas es estable.
\end{Teo}

\begin{Teo}[Teorema 5.1 \cite{Chen}]\label{Tma.5.1.Chen}
La red de colas es estable si existe una constante $t_{0}$ que
depende de $\left(\alpha,\mu,T,U\right)$ y $V$ que satisfagan las
ecuaciones (5.1)-(5.5), $Z\left(t\right)=0$, para toda $t\geq
t_{0}$.
\end{Teo}



\begin{Lema}[Lema 5.2 \cite{Gut}]\label{Lema.5.2.Gut}
Sea $\left\{\xi\left(k\right):k\in\ent\right\}$ sucesin de
variables aleatorias i.i.d. con valores en
$\left(0,\infty\right)$, y sea $E\left(t\right)$ el proceso de
conteo
\[E\left(t\right)=max\left\{n\geq1:\xi\left(1\right)+\cdots+\xi\left(n-1\right)\leq t\right\}.\]
Si $E\left[\xi\left(1\right)\right]<\infty$, entonces para
cualquier entero $r\geq1$
\begin{equation}
lim_{t\rightarrow\infty}\esp\left[\left(\frac{E\left(t\right)}{t}\right)^{r}\right]=\left(\frac{1}{E\left[\xi_{1}\right]}\right)^{r}
\end{equation}
de aqu, bajo estas condiciones
\begin{itemize}
\item[a)] Para cualquier $t>0$,
$sup_{t\geq\delta}\esp\left[\left(\frac{E\left(t\right)}{t}\right)^{r}\right]$

\item[b)] Las variables aleatorias
$\left\{\left(\frac{E\left(t\right)}{t}\right)^{r}:t\geq1\right\}$
son uniformemente integrables.
\end{itemize}
\end{Lema}

\begin{Teo}[Teorema 5.1: Ley Fuerte para Procesos de Conteo
\cite{Gut}]\label{Tma.5.1.Gut} Sea
$0<\mu<\esp\left(X_{1}\right]\leq\infty$. entonces

\begin{itemize}
\item[a)] $\frac{N\left(t\right)}{t}\rightarrow\frac{1}{\mu}$
a.s., cuando $t\rightarrow\infty$.


\item[b)]$\esp\left[\frac{N\left(t\right)}{t}\right]^{r}\rightarrow\frac{1}{\mu^{r}}$,
cuando $t\rightarrow\infty$ para todo $r>0$..
\end{itemize}
\end{Teo}


\begin{Prop}[Proposicin 5.1 \cite{DaiSean}]\label{Prop.5.1}
Suponga que los supuestos (A1) y (A2) se cumplen, adems suponga
que el modelo de flujo es estable. Entonces existe $t_{0}>0$ tal
que
\begin{equation}\label{Eq.Prop.5.1}
lim_{|x|\rightarrow\infty}\frac{1}{|x|^{p+1}}\esp_{x}\left[|X\left(t_{0}|x|\right)|^{p+1}\right]=0.
\end{equation}

\end{Prop}


\begin{Prop}[Proposici\'on 5.3 \cite{DaiSean}]
Sea $X$ proceso de estados para la red de colas, y suponga que se
cumplen los supuestos (A1) y (A2), entonces para alguna constante
positiva $C_{p+1}<\infty$, $\delta>0$ y un conjunto compacto
$C\subset X$.

\begin{equation}\label{Eq.5.4}
\esp_{x}\left[\int_{0}^{\tau_{C}\left(\delta\right)}\left(1+|X\left(t\right)|^{p}\right)dt\right]\leq
C_{p+1}\left(1+|x|^{p+1}\right)
\end{equation}
\end{Prop}

\begin{Prop}[Proposici\'on 5.4 \cite{DaiSean}]
Sea $X$ un proceso de Markov Borel Derecho en $X$, sea
$f:X\leftarrow\rea_{+}$ y defina para alguna $\delta>0$, y un
conjunto cerrado $C\subset X$
\[V\left(x\right):=\esp_{x}\left[\int_{0}^{\tau_{C}\left(\delta\right)}f\left(X\left(t\right)\right)dt\right]\]
para $x\in X$. Si $V$ es finito en todas partes y uniformemente
acotada en $C$, entonces existe $k<\infty$ tal que
\begin{equation}\label{Eq.5.11}
\frac{1}{t}\esp_{x}\left[V\left(x\right)\right]+\frac{1}{t}\int_{0}^{t}\esp_{x}\left[f\left(X\left(s\right)\right)ds\right]\leq\frac{1}{t}V\left(x\right)+k,
\end{equation}
para $x\in X$ y $t>0$.
\end{Prop}


\begin{Teo}[Teorema 5.5 \cite{DaiSean}]
Suponga que se cumplen (A1) y (A2), adems suponga que el modelo
de flujo es estable. Entonces existe una constante $k_{p}<\infty$
tal que
\begin{equation}\label{Eq.5.13}
\frac{1}{t}\int_{0}^{t}\esp_{x}\left[|Q\left(s\right)|^{p}\right]ds\leq
k_{p}\left\{\frac{1}{t}|x|^{p+1}+1\right\}
\end{equation}
para $t\geq0$, $x\in X$. En particular para cada condici\'on inicial
\begin{equation}\label{Eq.5.14}
Limsup_{t\rightarrow\infty}\frac{1}{t}\int_{0}^{t}\esp_{x}\left[|Q\left(s\right)|^{p}\right]ds\leq
k_{p}
\end{equation}
\end{Teo}

\begin{Teo}[Teorema 6.2\cite{DaiSean}]\label{Tma.6.2}
Suponga que se cumplen los supuestos (A1)-(A3) y que el modelo de
flujo es estable, entonces se tiene que
\[\parallel P^{t}\left(c,\cdot\right)-\pi\left(\cdot\right)\parallel_{f_{p}}\rightarrow0\]
para $t\rightarrow\infty$ y $x\in X$. En particular para cada
condicin inicial
\[lim_{t\rightarrow\infty}\esp_{x}\left[\left|Q_{t}\right|^{p}\right]=\esp_{\pi}\left[\left|Q_{0}\right|^{p}\right]<\infty\]
\end{Teo}


\begin{Teo}[Teorema 6.3\cite{DaiSean}]\label{Tma.6.3}
Suponga que se cumplen los supuestos (A1)-(A3) y que el modelo de
flujo es estable, entonces con
$f\left(x\right)=f_{1}\left(x\right)$, se tiene que
\[lim_{t\rightarrow\infty}t^{(p-1)\left|P^{t}\left(c,\cdot\right)-\pi\left(\cdot\right)\right|_{f}=0},\]
para $x\in X$. En particular, para cada condicin inicial
\[lim_{t\rightarrow\infty}t^{(p-1)\left|\esp_{x}\left[Q_{t}\right]-\esp_{\pi}\left[Q_{0}\right]\right|=0}.\]
\end{Teo}



\begin{Prop}[Proposici\'on 5.1, Dai y Meyn \cite{DaiSean}]\label{Prop.5.1.DaiSean}
Suponga que los supuestos A1) y A2) son ciertos y que el modelo de flujo es estable. Entonces existe $t_{0}>0$ tal que
\begin{equation}
lim_{|x|\rightarrow\infty}\frac{1}{|x|^{p+1}}\esp_{x}\left[|X\left(t_{0}|x|\right)|^{p+1}\right]=0
\end{equation}
\end{Prop}

\begin{Lemma}[Lema 5.2, Dai y Meyn \cite{DaiSean}]\label{Lema.5.2.DaiSean}
 Sea $\left\{\zeta\left(k\right):k\in \mathbb{z}\right\}$ una sucesi\'on independiente e id\'enticamente distribuida que toma valores en $\left(0,\infty\right)$,
y sea
$E\left(t\right)=max\left(n\geq1:\zeta\left(1\right)+\cdots+\zeta\left(n-1\right)\leq
t\right)$. Si $\esp\left[\zeta\left(1\right)\right]<\infty$,
entonces para cualquier entero $r\geq1$
\begin{equation}
 lim_{t\rightarrow\infty}\esp\left[\left(\frac{E\left(t\right)}{t}\right)^{r}\right]=\left(\frac{1}{\esp\left[\zeta_{1}\right]}\right)^{r}.
\end{equation}
Luego, bajo estas condiciones:
\begin{itemize}
 \item[a)] para cualquier $\delta>0$, $\sup_{t\geq\delta}\esp\left[\left(\frac{E\left(t\right)}{t}\right)^{r}\right]<\infty$
\item[b)] las variables aleatorias
$\left\{\left(\frac{E\left(t\right)}{t}\right)^{r}:t\geq1\right\}$
son uniformemente integrables.
\end{itemize}
\end{Lemma}

\begin{Teo}[Teorema 5.5, Dai y Meyn \cite{DaiSean}]\label{Tma.5.5.DaiSean}
Suponga que los supuestos A1) y A2) se cumplen y que el modelo de
flujo es estable. Entonces existe una constante $\kappa_{p}$ tal
que
\begin{equation}
\frac{1}{t}\int_{0}^{t}\esp_{x}\left[|Q\left(s\right)|^{p}\right]ds\leq\kappa_{p}\left\{\frac{1}{t}|x|^{p+1}+1\right\}
\end{equation}
para $t>0$ y $x\in X$. En particular, para cada condici\'on
inicial
\begin{eqnarray*}
\limsup_{t\rightarrow\infty}\frac{1}{t}\int_{0}^{t}\esp_{x}\left[|Q\left(s\right)|^{p}\right]ds\leq\kappa_{p}.
\end{eqnarray*}
\end{Teo}

\begin{Teo}[Teorema 6.2, Dai y Meyn \cite{DaiSean}]\label{Tma.6.2.DaiSean}
Suponga que se cumplen los supuestos A1), A2) y A3) y que el
modelo de flujo es estable. Entonces se tiene que
\begin{equation}
\left\|P^{t}\left(x,\cdot\right)-\pi\left(\cdot\right)\right\|_{f_{p}}\textrm{,
}t\rightarrow\infty,x\in X.
\end{equation}
En particular para cada condici\'on inicial
\begin{eqnarray*}
\lim_{t\rightarrow\infty}\esp_{x}\left[|Q\left(t\right)|^{p}\right]=\esp_{\pi}\left[|Q\left(0\right)|^{p}\right]\leq\kappa_{r}
\end{eqnarray*}
\end{Teo}
\begin{Teo}[Teorema 6.3, Dai y Meyn \cite{DaiSean}]\label{Tma.6.3.DaiSean}
Suponga que se cumplen los supuestos A1), A2) y A3) y que el
modelo de flujo es estable. Entonces con
$f\left(x\right)=f_{1}\left(x\right)$ se tiene
\begin{equation}
\lim_{t\rightarrow\infty}t^{p-1}\left\|P^{t}\left(x,\cdot\right)-\pi\left(\cdot\right)\right\|_{f}=0.
\end{equation}
En particular para cada condici\'on inicial
\begin{eqnarray*}
\lim_{t\rightarrow\infty}t^{p-1}|\esp_{x}\left[Q\left(t\right)\right]-\esp_{\pi}\left[Q\left(0\right)\right]|=0.
\end{eqnarray*}
\end{Teo}

\begin{Teo}[Teorema 6.4, Dai y Meyn \cite{DaiSean}]\label{Tma.6.4.DaiSean}
Suponga que se cumplen los supuestos A1), A2) y A3) y que el
modelo de flujo es estable. Sea $\nu$ cualquier distribuci\'on de
probabilidad en $\left(X,\mathcal{B}_{X}\right)$, y $\pi$ la
distribuci\'on estacionaria de $X$.
\begin{itemize}
\item[i)] Para cualquier $f:X\leftarrow\rea_{+}$
\begin{equation}
\lim_{t\rightarrow\infty}\frac{1}{t}\int_{o}^{t}f\left(X\left(s\right)\right)ds=\pi\left(f\right):=\int
f\left(x\right)\pi\left(dx\right)
\end{equation}
$\prob$-c.s.

\item[ii)] Para cualquier $f:X\leftarrow\rea_{+}$ con
$\pi\left(|f|\right)<\infty$, la ecuaci\'on anterior se cumple.
\end{itemize}
\end{Teo}

\begin{Teo}[Teorema 2.2, Down \cite{Down}]\label{Tma2.2.Down}
Suponga que el fluido modelo es inestable en el sentido de que
para alguna $\epsilon_{0},c_{0}\geq0$,
\begin{equation}\label{Eq.Inestability}
|Q\left(T\right)|\geq\epsilon_{0}T-c_{0}\textrm{,   }T\geq0,
\end{equation}
para cualquier condici\'on inicial $Q\left(0\right)$, con
$|Q\left(0\right)|=1$. Entonces para cualquier $0<q\leq1$, existe
$B<0$ tal que para cualquier $|x|\geq B$,
\begin{equation}
\prob_{x}\left\{\mathbb{X}\rightarrow\infty\right\}\geq q.
\end{equation}
\end{Teo}


%_________________________________________________________________________
\subsection{Supuestos}
%_________________________________________________________________________
Consideremos el caso en el que se tienen varias colas a las cuales
llegan uno o varios servidores para dar servicio a los usuarios
que se encuentran presentes en la cola, como ya se mencion\'o hay
varios tipos de pol\'iticas de servicio, incluso podr\'ia ocurrir
que la manera en que atiende al resto de las colas sea distinta a
como lo hizo en las anteriores.\\

Para ejemplificar los sistemas de visitas c\'iclicas se
considerar\'a el caso en que a las colas los usuarios son atendidos con
una s\'ola pol\'itica de servicio.\\


Si $\omega$ es el n\'umero de usuarios en la cola al comienzo del
periodo de servicio y $N\left(\omega\right)$ es el n\'umero de
usuarios que son atendidos con una pol\'itica en espec\'ifico
durante el periodo de servicio, entonces se asume que:
\begin{itemize}
\item[1)]\label{S1}$lim_{\omega\rightarrow\infty}\esp\left[N\left(\omega\right)\right]=\overline{N}>0$;
\item[2)]\label{S2}$\esp\left[N\left(\omega\right)\right]\leq\overline{N}$
para cualquier valor de $\omega$.
\end{itemize}
La manera en que atiende el servidor $m$-\'esimo, es la siguiente:
\begin{itemize}
\item Al t\'ermino de la visita a la cola $j$, el servidor cambia
a la cola $j^{'}$ con probabilidad $r_{j,j^{'}}^{m}$

\item La $n$-\'esima vez que el servidor cambia de la cola $j$ a
$j'$, va acompa\~nada con el tiempo de cambio de longitud
$\delta_{j,j^{'}}^{m}\left(n\right)$, con
$\delta_{j,j^{'}}^{m}\left(n\right)$, $n\geq1$, variables
aleatorias independientes e id\'enticamente distribuidas, tales
que $\esp\left[\delta_{j,j^{'}}^{m}\left(1\right)\right]\geq0$.

\item Sea $\left\{p_{j}^{m}\right\}$ la distribuci\'on invariante
estacionaria \'unica para la Cadena de Markov con matriz de
transici\'on $\left(r_{j,j^{'}}^{m}\right)$, se supone que \'esta
existe.

\item Finalmente, se define el tiempo promedio total de traslado
entre las colas como
\begin{equation}
\delta^{*}:=\sum_{j,j^{'}}p_{j}^{m}r_{j,j^{'}}^{m}\esp\left[\delta_{j,j^{'}}^{m}\left(i\right)\right].
\end{equation}
\end{itemize}

Consideremos el caso donde los tiempos entre arribo a cada una de
las colas, $\left\{\xi_{k}\left(n\right)\right\}_{n\geq1}$ son
variables aleatorias independientes a id\'enticamente
distribuidas, y los tiempos de servicio en cada una de las colas
se distribuyen de manera independiente e id\'enticamente
distribuidas $\left\{\eta_{k}\left(n\right)\right\}_{n\geq1}$;
adem\'as ambos procesos cumplen la condici\'on de ser
independientes entre s\'i. Para la $k$-\'esima cola se define la
tasa de arribo por
$\lambda_{k}=1/\esp\left[\xi_{k}\left(1\right)\right]$ y la tasa
de servicio como
$\mu_{k}=1/\esp\left[\eta_{k}\left(1\right)\right]$, finalmente se
define la carga de la cola como $\rho_{k}=\lambda_{k}/\mu_{k}$,
donde se pide que $\rho=\sum_{k=1}^{K}\rho_{k}<1$, para garantizar
la estabilidad del sistema, esto es cierto para las pol\'iticas de
servicio exhaustiva y cerrada, ver Geetor \cite{Getoor}.\\

Si denotamos por
\begin{itemize}
\item $Q_{k}\left(t\right)$ el n\'umero de usuarios presentes en
la cola $k$ al tiempo $t$; \item $A_{k}\left(t\right)$ los
residuales de los tiempos entre arribos a la cola $k$; para cada
servidor $m$; \item $B_{m}\left(t\right)$ denota a los residuales
de los tiempos de servicio al tiempo $t$; \item
$B_{m}^{0}\left(t\right)$ los residuales de los tiempos de
traslado de la cola $k$ a la pr\'oxima por atender al tiempo $t$,

\item sea
$C_{m}\left(t\right)$ el n\'umero de usuarios atendidos durante la
visita del servidor a la cola $k$ al tiempo $t$.
\end{itemize}


En este sentido, el proceso para el sistema de visitas se puede
definir como:

\begin{equation}\label{Esp.Edos.Down}
X\left(t\right)^{T}=\left(Q_{k}\left(t\right),A_{k}\left(t\right),B_{m}\left(t\right),B_{m}^{0}\left(t\right),C_{m}\left(t\right)\right),
\end{equation}
para $k=1,\ldots,K$ y $m=1,2,\ldots,M$, donde $T$ indica que es el
transpuesto del vector que se est\'a definiendo. El proceso $X$
evoluciona en el espacio de estados:
$\mathbb{X}=\ent_{+}^{K}\times\rea_{+}^{K}\times\left(\left\{1,2,\ldots,K\right\}\times\left\{1,2,\ldots,S\right\}\right)^{M}\times\rea_{+}^{K}\times\ent_{+}^{K}$.\\

El sistema aqu\'i descrito debe de cumplir con los siguientes supuestos b\'asicos de un sistema de visitas:
%__________________________________________________________________________
\subsubsection{Supuestos B\'asicos}
%__________________________________________________________________________
\begin{itemize}
\item[A1)] Los procesos
$\xi_{1},\ldots,\xi_{K},\eta_{1},\ldots,\eta_{K}$ son mutuamente
independientes y son sucesiones independientes e id\'enticamente
distribuidas.

\item[A2)] Para alg\'un entero $p\geq1$
\begin{eqnarray*}
\esp\left[\xi_{l}\left(1\right)^{p+1}\right]&<&\infty\textrm{ para }l=1,\ldots,K\textrm{ y }\\
\esp\left[\eta_{k}\left(1\right)^{p+1}\right]&<&\infty\textrm{
para }k=1,\ldots,K.
\end{eqnarray*}
donde $\mathcal{A}$ es la clase de posibles arribos.

\item[A3)] Para cada $k=1,2,\ldots,K$ existe una funci\'on
positiva $q_{k}\left(\cdot\right)$ definida en $\rea_{+}$, y un
entero $j_{k}$, tal que
\begin{eqnarray}
P\left(\xi_{k}\left(1\right)\geq x\right)&>&0\textrm{, para todo }x>0,\\
P\left\{a\leq\sum_{i=1}^{j_{k}}\xi_{k}\left(i\right)\leq
b\right\}&\geq&\int_{a}^{b}q_{k}\left(x\right)dx, \textrm{ }0\leq
a<b.
\end{eqnarray}
\end{itemize}

En lo que respecta al supuesto (A3), en Dai y Meyn \cite{DaiSean}
hacen ver que este se puede sustituir por

\begin{itemize}
\item[A3')] Para el Proceso de Markov $X$, cada subconjunto
compacto del espacio de estados de $X$ es un conjunto peque\~no,
ver definici\'on \ref{Def.Cto.Peq.}.
\end{itemize}

Es por esta raz\'on que con la finalidad de poder hacer uso de
$A3^{'})$ es necesario recurrir a los Procesos de Harris y en
particular a los Procesos Harris Recurrente, ver \cite{Dai,
DaiSean}.
%_______________________________________________________________________
\subsection{Procesos Harris Recurrente}
%_______________________________________________________________________

Por el supuesto (A1) conforme a Davis \cite{Davis}, se puede
definir el proceso de saltos correspondiente de manera tal que
satisfaga el supuesto (A3'), de hecho la demostraci\'on est\'a
basada en la l\'inea de argumentaci\'on de Davis, \cite{Davis},
p\'aginas 362-364.\\

Entonces se tiene un espacio de estados en el cual el proceso $X$
satisface la Propiedad Fuerte de Markov, ver Dai y Meyn
\cite{DaiSean}, dado por

\[\left(\Omega,\mathcal{F},\mathcal{F}_{t},X\left(t\right),\theta_{t},P_{x}\right),\]
adem\'as de ser un proceso de Borel Derecho (Sharpe \cite{Sharpe})
en el espacio de estados medible
$\left(\mathbb{X},\mathcal{B}_\mathbb{X}\right)$. El Proceso
$X=\left\{X\left(t\right),t\geq0\right\}$ tiene trayectorias
continuas por la derecha, est\'a definido en
$\left(\Omega,\mathcal{F}\right)$ y est\'a adaptado a
$\left\{\mathcal{F}_{t},t\geq0\right\}$; la colecci\'on
$\left\{P_{x},x\in \mathbb{X}\right\}$ son medidas de probabilidad
en $\left(\Omega,\mathcal{F}\right)$ tales que para todo $x\in
\mathbb{X}$
\[P_{x}\left\{X\left(0\right)=x\right\}=1,\] y
\[E_{x}\left\{f\left(X\circ\theta_{t}\right)|\mathcal{F}_{t}\right\}=E_{X}\left(\tau\right)f\left(X\right),\]
en $\left\{\tau<\infty\right\}$, $P_{x}$-c.s., con $\theta_{t}$
definido como el operador shift.


Donde $\tau$ es un $\mathcal{F}_{t}$-tiempo de paro
\[\left(X\circ\theta_{\tau}\right)\left(w\right)=\left\{X\left(\tau\left(w\right)+t,w\right),t\geq0\right\},\]
y $f$ es una funci\'on de valores reales acotada y medible, ver \cite{Dai, KaspiMandelbaum}.\\

Sea $P^{t}\left(x,D\right)$, $D\in\mathcal{B}_{\mathbb{X}}$,
$t\geq0$ la probabilidad de transici\'on de $X$ queda definida
como:
\[P^{t}\left(x,D\right)=P_{x}\left(X\left(t\right)\in
D\right).\]


\begin{Def}
Una medida no cero $\pi$ en
$\left(\mathbb{X},\mathcal{B}_{\mathbb{X}}\right)$ es invariante
para $X$ si $\pi$ es $\sigma$-finita y
\[\pi\left(D\right)=\int_{\mathbb{X}}P^{t}\left(x,D\right)\pi\left(dx\right),\]
para todo $D\in \mathcal{B}_{\mathbb{X}}$, con $t\geq0$.
\end{Def}

\begin{Def}
El proceso de Markov $X$ es llamado Harris Recurrente si existe
una medida de probabilidad $\nu$ en
$\left(\mathbb{X},\mathcal{B}_{\mathbb{X}}\right)$, tal que si
$\nu\left(D\right)>0$ y $D\in\mathcal{B}_{\mathbb{X}}$
\[P_{x}\left\{\tau_{D}<\infty\right\}\equiv1,\] cuando
$\tau_{D}=inf\left\{t\geq0:X_{t}\in D\right\}$.
\end{Def}

\begin{Note}
\begin{itemize}
\item[i)] Si $X$ es Harris recurrente, entonces existe una \'unica
medida invariante $\pi$ (Getoor \cite{Getoor}).

\item[ii)] Si la medida invariante es finita, entonces puede
normalizarse a una medida de probabilidad, en este caso al proceso
$X$ se le llama Harris recurrente positivo.


\item[iii)] Cuando $X$ es Harris recurrente positivo se dice que
la disciplina de servicio es estable. En este caso $\pi$ denota la
distribuci\'on estacionaria y hacemos
\[P_{\pi}\left(\cdot\right)=\int_{\mathbf{X}}P_{x}\left(\cdot\right)\pi\left(dx\right),\]
y se utiliza $E_{\pi}$ para denotar el operador esperanza
correspondiente, ver \cite{DaiSean}.
\end{itemize}
\end{Note}

\begin{Def}\label{Def.Cto.Peq.}
Un conjunto $D\in\mathcal{B_{\mathbb{X}}}$ es llamado peque\~no si
existe un $t>0$, una medida de probabilidad $\nu$ en
$\mathcal{B_{\mathbb{X}}}$, y un $\delta>0$ tal que
\[P^{t}\left(x,A\right)\geq\delta\nu\left(A\right),\] para $x\in
D,A\in\mathcal{B_{\mathbb{X}}}$.
\end{Def}

La siguiente serie de resultados vienen enunciados y demostrados
en Dai \cite{Dai}:
\begin{Lema}[Lema 3.1, Dai \cite{Dai}]
Sea $B$ conjunto peque\~no cerrado, supongamos que
$P_{x}\left(\tau_{B}<\infty\right)\equiv1$ y que para alg\'un
$\delta>0$ se cumple que
\begin{equation}\label{Eq.3.1}
\sup\esp_{x}\left[\tau_{B}\left(\delta\right)\right]<\infty,
\end{equation}
donde
$\tau_{B}\left(\delta\right)=inf\left\{t\geq\delta:X\left(t\right)\in
B\right\}$. Entonces, $X$ es un proceso Harris recurrente
positivo.
\end{Lema}

\begin{Lema}[Lema 3.1, Dai \cite{Dai}]\label{Lema.3.}
Bajo el supuesto (A3), el conjunto
$B=\left\{x\in\mathbb{X}:|x|\leq k\right\}$ es un conjunto
peque\~no cerrado para cualquier $k>0$.
\end{Lema}

\begin{Teo}[Teorema 3.1, Dai \cite{Dai}]\label{Tma.3.1}
Si existe un $\delta>0$ tal que
\begin{equation}
lim_{|x|\rightarrow\infty}\frac{1}{|x|}\esp|X^{x}\left(|x|\delta\right)|=0,
\end{equation}
donde $X^{x}$ se utiliza para denotar que el proceso $X$ comienza
a partir de $x$, entonces la ecuaci\'on (\ref{Eq.3.1}) se cumple
para $B=\left\{x\in\mathbb{X}:|x|\leq k\right\}$ con alg\'un
$k>0$. En particular, $X$ es Harris recurrente positivo.
\end{Teo}

Entonces, tenemos que el proceso $X$ es un proceso de Markov que
cumple con los supuestos $A1)$-$A3)$, lo que falta de hacer es
construir el Modelo de Flujo bas\'andonos en lo hasta ahora
presentado.
%_______________________________________________________________________
\subsection{Modelo de Flujo}
%_______________________________________________________________________

Dada una condici\'on inicial $x\in\mathbb{X}$, sea

\begin{itemize}
\item $Q_{k}^{x}\left(t\right)$ la longitud de la cola al tiempo
$t$,

\item $T_{m,k}^{x}\left(t\right)$ el tiempo acumulado, al tiempo
$t$, que tarda el servidor $m$ en atender a los usuarios de la
cola $k$.

\item $T_{m,k}^{x,0}\left(t\right)$ el tiempo acumulado, al tiempo
$t$, que tarda el servidor $m$ en trasladarse a otra cola a partir de la $k$-\'esima.\\
\end{itemize}

Sup\'ongase que la funci\'on
$\left(\overline{Q}\left(\cdot\right),\overline{T}_{m}
\left(\cdot\right),\overline{T}_{m}^{0} \left(\cdot\right)\right)$
para $m=1,2,\ldots,M$ es un punto l\'imite de
\begin{equation}\label{Eq.Punto.Limite}
\left(\frac{1}{|x|}Q^{x}\left(|x|t\right),\frac{1}{|x|}T_{m}^{x}\left(|x|t\right),\frac{1}{|x|}T_{m}^{x,0}\left(|x|t\right)\right)
\end{equation}
para $m=1,2,\ldots,M$, cuando $x\rightarrow\infty$, ver
\cite{Down}. Entonces
$\left(\overline{Q}\left(t\right),\overline{T}_{m}
\left(t\right),\overline{T}_{m}^{0} \left(t\right)\right)$ es un
flujo l\'imite del sistema. Al conjunto de todos las posibles
flujos l\'imite se le llama {\emph{Modelo de Flujo}} y se le
denotar\'a por $\mathcal{Q}$, ver \cite{Down, Dai, DaiSean}.\\

El modelo de flujo satisface el siguiente conjunto de ecuaciones:

\begin{equation}\label{Eq.MF.1}
\overline{Q}_{k}\left(t\right)=\overline{Q}_{k}\left(0\right)+\lambda_{k}t-\sum_{m=1}^{M}\mu_{k}\overline{T}_{m,k}\left(t\right),\\
\end{equation}
para $k=1,2,\ldots,K$.\\
\begin{equation}\label{Eq.MF.2}
\overline{Q}_{k}\left(t\right)\geq0\textrm{ para
}k=1,2,\ldots,K.\\
\end{equation}

\begin{equation}\label{Eq.MF.3}
\overline{T}_{m,k}\left(0\right)=0,\textrm{ y }\overline{T}_{m,k}\left(\cdot\right)\textrm{ es no decreciente},\\
\end{equation}
para $k=1,2,\ldots,K$ y $m=1,2,\ldots,M$.\\
\begin{equation}\label{Eq.MF.4}
\sum_{k=1}^{K}\overline{T}_{m,k}^{0}\left(t\right)+\overline{T}_{m,k}\left(t\right)=t\textrm{
para }m=1,2,\ldots,M.\\
\end{equation}


\begin{Def}[Definici\'on 4.1, Dai \cite{Dai}]\label{Def.Modelo.Flujo}
Sea una disciplina de servicio espec\'ifica. Cualquier l\'imite
$\left(\overline{Q}\left(\cdot\right),\overline{T}\left(\cdot\right),\overline{T}^{0}\left(\cdot\right)\right)$
en (\ref{Eq.Punto.Limite}) es un {\em flujo l\'imite} de la
disciplina. Cualquier soluci\'on (\ref{Eq.MF.1})-(\ref{Eq.MF.4})
es llamado flujo soluci\'on de la disciplina.
\end{Def}

\begin{Def}
Se dice que el modelo de flujo l\'imite, modelo de flujo, de la
disciplina de la cola es estable si existe una constante
$\delta>0$ que depende de $\mu,\lambda$ y $P$ solamente, tal que
cualquier flujo l\'imite con
$|\overline{Q}\left(0\right)|+|\overline{U}|+|\overline{V}|=1$, se
tiene que $\overline{Q}\left(\cdot+\delta\right)\equiv0$.
\end{Def}

Si se hace $|x|\rightarrow\infty$ sin restringir ninguna de las
componentes, tambi\'en se obtienen un modelo de flujo, pero en
este caso el residual de los procesos de arribo y servicio
introducen un retraso:
\begin{Teo}[Teorema 4.2, Dai \cite{Dai}]\label{Tma.4.2.Dai}
Sea una disciplina fija para la cola, suponga que se cumplen las
condiciones (A1)-(A3). Si el modelo de flujo l\'imite de la
disciplina de la cola es estable, entonces la cadena de Markov $X$
que describe la din\'amica de la red bajo la disciplina es Harris
recurrente positiva.
\end{Teo}

Ahora se procede a escalar el espacio y el tiempo para reducir la
aparente fluctuaci\'on del modelo. Consid\'erese el proceso
\begin{equation}\label{Eq.3.7}
\overline{Q}^{x}\left(t\right)=\frac{1}{|x|}Q^{x}\left(|x|t\right).
\end{equation}
A este proceso se le conoce como el flujo escalado, y cualquier
l\'imite $\overline{Q}^{x}\left(t\right)$ es llamado flujo
l\'imite del proceso de longitud de la cola. Haciendo
$|q|\rightarrow\infty$ mientras se mantiene el resto de las
componentes fijas, cualquier punto l\'imite del proceso de
longitud de la cola normalizado $\overline{Q}^{x}$ es soluci\'on
del siguiente modelo de flujo.


\begin{Def}[Definici\'on 3.3, Dai y Meyn \cite{DaiSean}]
El modelo de flujo es estable si existe un tiempo fijo $t_{0}$ tal
que $\overline{Q}\left(t\right)=0$, con $t\geq t_{0}$, para
cualquier $\overline{Q}\left(\cdot\right)\in\mathcal{Q}$ que
cumple con $|\overline{Q}\left(0\right)|=1$.
\end{Def}

\begin{Lemma}[Lema 3.1, Dai y Meyn \cite{DaiSean}]
Si el modelo de flujo definido por (\ref{Eq.MF.1})-(\ref{Eq.MF.4})
es estable, entonces el modelo de flujo retrasado es tambi\'en
estable, es decir, existe $t_{0}>0$ tal que
$\overline{Q}\left(t\right)=0$ para cualquier $t\geq t_{0}$, para
cualquier soluci\'on del modelo de flujo retrasado cuya
condici\'on inicial $\overline{x}$ satisface que
$|\overline{x}|=|\overline{Q}\left(0\right)|+|\overline{A}\left(0\right)|+|\overline{B}\left(0\right)|\leq1$.
\end{Lemma}


Ahora ya estamos en condiciones de enunciar los resultados principales:


\begin{Teo}[Teorema 2.1, Down \cite{Down}]\label{Tma2.1.Down}
Suponga que el modelo de flujo es estable, y que se cumplen los supuestos (A1) y (A2), entonces
\begin{itemize}
\item[i)] Para alguna constante $\kappa_{p}$, y para cada
condici\'on inicial $x\in X$
\begin{equation}\label{Estability.Eq1}
\limsup_{t\rightarrow\infty}\frac{1}{t}\int_{0}^{t}\esp_{x}\left[|Q\left(s\right)|^{p}\right]ds\leq\kappa_{p},
\end{equation}
donde $p$ es el entero dado en (A2).
\end{itemize}
Si adem\'as se cumple la condici\'on (A3), entonces para cada
condici\'on inicial:
\begin{itemize}
\item[ii)] Los momentos transitorios convergen a su estado
estacionario:
 \begin{equation}\label{Estability.Eq2}
lim_{t\rightarrow\infty}\esp_{x}\left[Q_{k}\left(t\right)^{r}\right]=\esp_{\pi}\left[Q_{k}\left(0\right)^{r}\right]\leq\kappa_{r},
\end{equation}
para $r=1,2,\ldots,p$ y $k=1,2,\ldots,K$. Donde $\pi$ es la
probabilidad invariante para $X$.

\item[iii)]  El primer momento converge con raz\'on $t^{p-1}$:
\begin{equation}\label{Estability.Eq3}
lim_{t\rightarrow\infty}t^{p-1}|\esp_{x}\left[Q_{k}\left(t\right)\right]-\esp_{\pi}\left[Q_{k}\left(0\right)\right]|=0.
\end{equation}

\item[iv)] La {\em Ley Fuerte de los grandes n\'umeros} se cumple:
\begin{equation}\label{Estability.Eq4}
lim_{t\rightarrow\infty}\frac{1}{t}\int_{0}^{t}Q_{k}^{r}\left(s\right)ds=\esp_{\pi}\left[Q_{k}\left(0\right)^{r}\right],\textrm{
}\prob_{x}\textrm{-c.s.}
\end{equation}
para $r=1,2,\ldots,p$ y $k=1,2,\ldots,K$.
\end{itemize}
\end{Teo}

La contribuci\'on de Down a la teor\'ia de los {\emph {sistemas de
visitas c\'iclicas}}, es la relaci\'on que hay entre la
estabilidad del sistema con el comportamiento de las medidas de
desempe\~no, es decir, la condici\'on suficiente para poder
garantizar la convergencia del proceso de la longitud de la cola
as\'i como de por los menos los dos primeros momentos adem\'as de
una versi\'on de la Ley Fuerte de los Grandes N\'umeros para los
sistemas de visitas.


\begin{Teo}[Teorema 2.3, Down \cite{Down}]\label{Tma2.3.Down}
Considere el siguiente valor:
\begin{equation}\label{Eq.Rho.1serv}
\rho=\sum_{k=1}^{K}\rho_{k}+max_{1\leq j\leq K}\left(\frac{\lambda_{j}}{\sum_{s=1}^{S}p_{js}\overline{N}_{s}}\right)\delta^{*}
\end{equation}
\begin{itemize}
\item[i)] Si $\rho<1$ entonces la red es estable, es decir, se
cumple el Teorema \ref{Tma2.1.Down}.

\item[ii)] Si $\rho>1$ entonces la red es inestable, es decir, se
cumple el Teorema \ref{Tma2.2.Down}
\end{itemize}
\end{Teo}




Dado el proceso $X=\left\{X\left(t\right),t\geq0\right\}$ definido
en (\ref{Esp.Edos.Down}) que describe la din\'amica del sistema de
visitas c\'iclicas, si $U\left(t\right)$ es el residual de los
tiempos de llegada al tiempo $t$ entre dos usuarios consecutivos y
$V\left(t\right)$ es el residual de los tiempos de servicio al
tiempo $t$ para el usuario que est\'as siendo atendido por el
servidor. Sea $\mathbb{X}$ el espacio de estados que puede tomar
el proceso $X$.


\begin{Lema}[Lema 4.3, Dai\cite{Dai}]\label{Lema.4.3}
Sea $\left\{x_{n}\right\}\subset \mathbf{X}$ con
$|x_{n}|\rightarrow\infty$, conforme $n\rightarrow\infty$. Suponga
que
\[lim_{n\rightarrow\infty}\frac{1}{|x_{n}|}U\left(0\right)=\overline{U}_{k},\]
y
\[lim_{n\rightarrow\infty}\frac{1}{|x_{n}|}V\left(0\right)=\overline{V}_{k}.\]
\begin{itemize}
\item[a)] Conforme $n\rightarrow\infty$ casi seguramente,
\[lim_{n\rightarrow\infty}\frac{1}{|x_{n}|}U^{x_{n}}_{k}\left(|x_{n}|t\right)=\left(\overline{U}_{k}-t\right)^{+}\textrm{, u.o.c.}\]
y
\[lim_{n\rightarrow\infty}\frac{1}{|x_{n}|}V^{x_{n}}_{k}\left(|x_{n}|t\right)=\left(\overline{V}_{k}-t\right)^{+}.\]

\item[b)] Para cada $t\geq0$ fijo,
\[\left\{\frac{1}{|x_{n}|}U^{x_{n}}_{k}\left(|x_{n}|t\right),|x_{n}|\geq1\right\}\]
y
\[\left\{\frac{1}{|x_{n}|}V^{x_{n}}_{k}\left(|x_{n}|t\right),|x_{n}|\geq1\right\}\]
\end{itemize}
son uniformemente convergentes.
\end{Lema}

Sea $e$ es un vector de unos, $C$ es la matriz definida por
\[C_{ik}=\left\{\begin{array}{cc}
1,& S\left(k\right)=i,\\
0,& \textrm{ en otro caso}.\\
\end{array}\right.
\]
Es necesario enunciar el siguiente Teorema que se utilizar\'a para
el Teorema (\ref{Tma.4.2.Dai}):
\begin{Teo}[Teorema 4.1, Dai \cite{Dai}]
Considere una disciplina que cumpla la ley de conservaci\'on, para
casi todas las trayectorias muestrales $\omega$ y cualquier
sucesi\'on de estados iniciales $\left\{x_{n}\right\}\subset
\mathbf{X}$, con $|x_{n}|\rightarrow\infty$, existe una
subsucesi\'on $\left\{x_{n_{j}}\right\}$ con
$|x_{n_{j}}|\rightarrow\infty$ tal que
\begin{equation}\label{Eq.4.15}
\frac{1}{|x_{n_{j}}|}\left(Q^{x_{n_{j}}}\left(0\right),U^{x_{n_{j}}}\left(0\right),V^{x_{n_{j}}}\left(0\right)\right)\rightarrow\left(\overline{Q}\left(0\right),\overline{U},\overline{V}\right),
\end{equation}

\begin{equation}\label{Eq.4.16}
\frac{1}{|x_{n_{j}}|}\left(Q^{x_{n_{j}}}\left(|x_{n_{j}}|t\right),T^{x_{n_{j}}}\left(|x_{n_{j}}|t\right)\right)\rightarrow\left(\overline{Q}\left(t\right),\overline{T}\left(t\right)\right)\textrm{
u.o.c.}
\end{equation}

Adem\'as,
$\left(\overline{Q}\left(t\right),\overline{T}\left(t\right)\right)$
satisface las siguientes ecuaciones:
\begin{equation}\label{Eq.MF.1.3a}
\overline{Q}\left(t\right)=Q\left(0\right)+\left(\alpha
t-\overline{U}\right)^{+}-\left(I-P\right)^{'}M^{-1}\left(\overline{T}\left(t\right)-\overline{V}\right)^{+},
\end{equation}

\begin{equation}\label{Eq.MF.2.3a}
\overline{Q}\left(t\right)\geq0,\\
\end{equation}

\begin{equation}\label{Eq.MF.3.3a}
\overline{T}\left(t\right)\textrm{ es no decreciente y comienza en cero},\\
\end{equation}

\begin{equation}\label{Eq.MF.4.3a}
\overline{I}\left(t\right)=et-C\overline{T}\left(t\right)\textrm{
es no decreciente,}\\
\end{equation}

\begin{equation}\label{Eq.MF.5.3a}
\int_{0}^{\infty}\left(C\overline{Q}\left(t\right)\right)d\overline{I}\left(t\right)=0,\\
\end{equation}

\begin{equation}\label{Eq.MF.6.3a}
\textrm{Condiciones en
}\left(\overline{Q}\left(\cdot\right),\overline{T}\left(\cdot\right)\right)\textrm{
espec\'ificas de la disciplina de la cola,}
\end{equation}
\end{Teo}


Propiedades importantes para el modelo de flujo retrasado:

\begin{Prop}[Proposici\'on 4.2, Dai \cite{Dai}]
 Sea $\left(\overline{Q},\overline{T},\overline{T}^{0}\right)$ un flujo l\'imite de \ref{Eq.Punto.Limite}
 y suponga que cuando $x\rightarrow\infty$ a lo largo de una subsucesi\'on
\[\left(\frac{1}{|x|}Q_{k}^{x}\left(0\right),\frac{1}{|x|}A_{k}^{x}\left(0\right),\frac{1}{|x|}B_{k}^{x}\left(0\right),\frac{1}{|x|}B_{k}^{x,0}\left(0\right)\right)\rightarrow\left(\overline{Q}_{k}\left(0\right),0,0,0\right)\]
para $k=1,\ldots,K$. El flujo l\'imite tiene las siguientes
propiedades, donde las propiedades de la derivada se cumplen donde
la derivada exista:
\begin{itemize}
 \item[i)] Los vectores de tiempo ocupado $\overline{T}\left(t\right)$ y $\overline{T}^{0}\left(t\right)$ son crecientes y continuas con
$\overline{T}\left(0\right)=\overline{T}^{0}\left(0\right)=0$.
\item[ii)] Para todo $t\geq0$
\[\sum_{k=1}^{K}\left[\overline{T}_{k}\left(t\right)+\overline{T}_{k}^{0}\left(t\right)\right]=t.\]
\item[iii)] Para todo $1\leq k\leq K$
\[\overline{Q}_{k}\left(t\right)=\overline{Q}_{k}\left(0\right)+\alpha_{k}t-\mu_{k}\overline{T}_{k}\left(t\right).\]
\item[iv)]  Para todo $1\leq k\leq K$
\[\dot{{\overline{T}}}_{k}\left(t\right)=\rho_{k}\] para $\overline{Q}_{k}\left(t\right)=0$.
\item[v)] Para todo $k,j$
\[\mu_{k}^{0}\overline{T}_{k}^{0}\left(t\right)=\mu_{j}^{0}\overline{T}_{j}^{0}\left(t\right).\]
\item[vi)]  Para todo $1\leq k\leq K$
\[\mu_{k}\dot{{\overline{T}}}_{k}\left(t\right)=l_{k}\mu_{k}^{0}\dot{{\overline{T}}}_{k}^{0}\left(t\right),\] para $\overline{Q}_{k}\left(t\right)>0$.
\end{itemize}
\end{Prop}

\begin{Lema}[Lema 3.1, Chen \cite{Chen}]\label{Lema3.1}
Si el modelo de flujo es estable, definido por las ecuaciones
(3.8)-(3.13), entonces el modelo de flujo retrasado tambi\'en es
estable.
\end{Lema}

\begin{Lema}[Lema 5.2, Gut \cite{Gut}]\label{Lema.5.2.Gut}
Sea $\left\{\xi\left(k\right):k\in\ent\right\}$ sucesi\'on de
variables aleatorias i.i.d. con valores en
$\left(0,\infty\right)$, y sea $E\left(t\right)$ el proceso de
conteo
\[E\left(t\right)=max\left\{n\geq1:\xi\left(1\right)+\cdots+\xi\left(n-1\right)\leq t\right\}.\]
Si $E\left[\xi\left(1\right)\right]<\infty$, entonces para
cualquier entero $r\geq1$
\begin{equation}
lim_{t\rightarrow\infty}\esp\left[\left(\frac{E\left(t\right)}{t}\right)^{r}\right]=\left(\frac{1}{E\left[\xi_{1}\right]}\right)^{r},
\end{equation}
de aqu\'i, bajo estas condiciones
\begin{itemize}
\item[a)] Para cualquier $t>0$,
$sup_{t\geq\delta}\esp\left[\left(\frac{E\left(t\right)}{t}\right)^{r}\right]<\infty$.

\item[b)] Las variables aleatorias
$\left\{\left(\frac{E\left(t\right)}{t}\right)^{r}:t\geq1\right\}$
son uniformemente integrables.
\end{itemize}
\end{Lema}

\begin{Teo}[Teorema 5.1: Ley Fuerte para Procesos de Conteo, Gut
\cite{Gut}]\label{Tma.5.1.Gut} Sea
$0<\mu<\esp\left(X_{1}\right]\leq\infty$. entonces

\begin{itemize}
\item[a)] $\frac{N\left(t\right)}{t}\rightarrow\frac{1}{\mu}$
a.s., cuando $t\rightarrow\infty$.


\item[b)]$\esp\left[\frac{N\left(t\right)}{t}\right]^{r}\rightarrow\frac{1}{\mu^{r}}$,
cuando $t\rightarrow\infty$ para todo $r>0$.
\end{itemize}
\end{Teo}


\begin{Prop}[Proposici\'on 5.1, Dai y Sean \cite{DaiSean}]\label{Prop.5.1}
Suponga que los supuestos (A1) y (A2) se cumplen, adem\'as suponga
que el modelo de flujo es estable. Entonces existe $t_{0}>0$ tal
que
\begin{equation}\label{Eq.Prop.5.1}
lim_{|x|\rightarrow\infty}\frac{1}{|x|^{p+1}}\esp_{x}\left[|X\left(t_{0}|x|\right)|^{p+1}\right]=0.
\end{equation}

\end{Prop}


\begin{Prop}[Proposici\'on 5.3, Dai y Sean \cite{DaiSean}]\label{Prop.5.3.DaiSean}
Sea $X$ proceso de estados para la red de colas, y suponga que se
cumplen los supuestos (A1) y (A2), entonces para alguna constante
positiva $C_{p+1}<\infty$, $\delta>0$ y un conjunto compacto
$C\subset X$.

\begin{equation}\label{Eq.5.4}
\esp_{x}\left[\int_{0}^{\tau_{C}\left(\delta\right)}\left(1+|X\left(t\right)|^{p}\right)dt\right]\leq
C_{p+1}\left(1+|x|^{p+1}\right).
\end{equation}
\end{Prop}

\begin{Prop}[Proposici\'on 5.4, Dai y Sean \cite{DaiSean}]\label{Prop.5.4.DaiSean}
Sea $X$ un proceso de Markov Borel Derecho en $X$, sea
$f:X\leftarrow\rea_{+}$ y defina para alguna $\delta>0$, y un
conjunto cerrado $C\subset X$
\[V\left(x\right):=\esp_{x}\left[\int_{0}^{\tau_{C}\left(\delta\right)}f\left(X\left(t\right)\right)dt\right],\]
para $x\in X$. Si $V$ es finito en todas partes y uniformemente
acotada en $C$, entonces existe $k<\infty$ tal que
\begin{equation}\label{Eq.5.11}
\frac{1}{t}\esp_{x}\left[V\left(x\right)\right]+\frac{1}{t}\int_{0}^{t}\esp_{x}\left[f\left(X\left(s\right)\right)ds\right]\leq\frac{1}{t}V\left(x\right)+k,
\end{equation}
para $x\in X$ y $t>0$.
\end{Prop}


\begin{Teo}[Teorema 5.5, Dai y Sean  \cite{DaiSean}]
Suponga que se cumplen (A1) y (A2), adem\'as suponga que el modelo
de flujo es estable. Entonces existe una constante $k_{p}<\infty$
tal que
\begin{equation}\label{Eq.5.13}
\frac{1}{t}\int_{0}^{t}\esp_{x}\left[|Q\left(s\right)|^{p}\right]ds\leq
k_{p}\left\{\frac{1}{t}|x|^{p+1}+1\right\},
\end{equation}
para $t\geq0$, $x\in X$. En particular para cada condici\'on
inicial
\begin{equation}\label{Eq.5.14}
\limsup_{t\rightarrow\infty}\frac{1}{t}\int_{0}^{t}\esp_{x}\left[|Q\left(s\right)|^{p}\right]ds\leq
k_{p}.
\end{equation}
\end{Teo}

\begin{Teo}[Teorema 6.2 Dai y Sean \cite{DaiSean}]\label{Tma.6.2}
Suponga que se cumplen los supuestos (A1)-(A3) y que el modelo de
flujo es estable, entonces se tiene que
\[\parallel P^{t}\left(x,\cdot\right)-\pi\left(\cdot\right)\parallel_{f_{p}}\rightarrow0,\]
para $t\rightarrow\infty$ y $x\in X$. En particular para cada
condici\'on inicial
\[lim_{t\rightarrow\infty}\esp_{x}\left[\left|Q_{t}\right|^{p}\right]=\esp_{\pi}\left[\left|Q_{0}\right|^{p}\right]<\infty,\]
\end{Teo}

donde

\begin{eqnarray*}
\parallel
P^{t}\left(c,\cdot\right)-\pi\left(\cdot\right)\parallel_{f}=sup_{|g\leq
f|}|\int\pi\left(dy\right)g\left(y\right)-\int
P^{t}\left(x,dy\right)g\left(y\right)|,
\end{eqnarray*}
para $x\in\mathbb{X}$.

\begin{Teo}[Teorema 6.3, Dai y Sean \cite{DaiSean}]\label{Tma.6.3}
Suponga que se cumplen los supuestos (A1)-(A3) y que el modelo de
flujo es estable, entonces con
$f\left(x\right)=f_{1}\left(x\right)$, se tiene que
\[lim_{t\rightarrow\infty}t^{(p-1)}\left|P^{t}\left(c,\cdot\right)-\pi\left(\cdot\right)\right|_{f}=0,\]
para $x\in X$. En particular, para cada condici\'on inicial
\[lim_{t\rightarrow\infty}t^{(p-1)}\left|\esp_{x}\left[Q_{t}\right]-\esp_{\pi}\left[Q_{0}\right]\right|=0.\]
\end{Teo}



\begin{Prop}[Proposici\'on 5.1, Dai y Meyn \cite{DaiSean}]\label{Prop.5.1.DaiSean}
Suponga que los supuestos A1) y A2) son ciertos y que el modelo de
flujo es estable. Entonces existe $t_{0}>0$ tal que
\begin{equation}
lim_{|x|\rightarrow\infty}\frac{1}{|x|^{p+1}}\esp_{x}\left[|X\left(t_{0}|x|\right)|^{p+1}\right]=0.
\end{equation}
\end{Prop}


\begin{Teo}[Teorema 5.5, Dai y Meyn \cite{DaiSean}]\label{Tma.5.5.DaiSean}
Suponga que los supuestos A1) y A2) se cumplen y que el modelo de
flujo es estable. Entonces existe una constante $\kappa_{p}$ tal
que
\begin{equation}
\frac{1}{t}\int_{0}^{t}\esp_{x}\left[|Q\left(s\right)|^{p}\right]ds\leq\kappa_{p}\left\{\frac{1}{t}|x|^{p+1}+1\right\},
\end{equation}
para $t>0$ y $x\in X$. En particular, para cada condici\'on
inicial
\begin{eqnarray*}
\limsup_{t\rightarrow\infty}\frac{1}{t}\int_{0}^{t}\esp_{x}\left[|Q\left(s\right)|^{p}\right]ds\leq\kappa_{p}.
\end{eqnarray*}
\end{Teo}


\begin{Teo}[Teorema 6.4, Dai y Meyn \cite{DaiSean}]\label{Tma.6.4.DaiSean}
Suponga que se cumplen los supuestos A1), A2) y A3) y que el
modelo de flujo es estable. Sea $\nu$ cualquier distribuci\'on de
probabilidad en
$\left(\mathbb{X},\mathcal{B}_{\mathbb{X}}\right)$, y $\pi$ la
distribuci\'on estacionaria de $X$.
\begin{itemize}
\item[i)] Para cualquier $f:X\leftarrow\rea_{+}$
\begin{equation}
\lim_{t\rightarrow\infty}\frac{1}{t}\int_{o}^{t}f\left(X\left(s\right)\right)ds=\pi\left(f\right):=\int
f\left(x\right)\pi\left(dx\right),
\end{equation}
$\prob$-c.s.

\item[ii)] Para cualquier $f:X\leftarrow\rea_{+}$ con
$\pi\left(|f|\right)<\infty$, la ecuaci\'on anterior se cumple.
\end{itemize}
\end{Teo}

\begin{Teo}[Teorema 2.2, Down \cite{Down}]\label{Tma2.2.Down}
Suponga que el fluido modelo es inestable en el sentido de que
para alguna $\epsilon_{0},c_{0}\geq0$,
\begin{equation}\label{Eq.Inestability}
|Q\left(T\right)|\geq\epsilon_{0}T-c_{0}\textrm{,   }T\geq0,
\end{equation}
para cualquier condici\'on inicial $Q\left(0\right)$, con
$|Q\left(0\right)|=1$. Entonces para cualquier $0<q\leq1$, existe
$B<0$ tal que para cualquier $|x|\geq B$,
\begin{equation}
\prob_{x}\left\{\mathbb{X}\rightarrow\infty\right\}\geq q.
\end{equation}
\end{Teo}

\begin{Dem}[Teorema \ref{Tma2.1.Down}] La demostraci\'on de este
teorema se da a continuaci\'on:\\
\begin{itemize}
\item[i)] Utilizando la proposici\'on \ref{Prop.5.3.DaiSean} se
tiene que la proposici\'on \ref{Prop.5.4.DaiSean} es cierta para
$f\left(x\right)=1+|x|^{p}$.

\item[i)] es consecuencia directa del Teorema \ref{Tma.6.2}.

\item[iii)] ver la demostraci\'on dada en Dai y Sean
\cite{DaiSean} p\'aginas 1901-1902.

\item[iv)] ver Dai y Sean \cite{DaiSean} p\'aginas 1902-1903 \'o
\cite{MeynTweedie2}.
\end{itemize}
\end{Dem}

%_____________________________________________________________________
\subsubsection{Modelo de Flujo y Estabilidad}
%_____________________________________________________________________

Para cada $k$ y cada $n$ se define

\numberwithin{equation}{section}
\begin{equation}
\Phi^{k}\left(n\right):=\sum_{i=1}^{n}\phi^{k}\left(i\right).
\end{equation}

suponiendo que el estado inicial de la red es
$x=\left(q,a,b\right)\in X$, entonces para cada $k$

\begin{eqnarray}
E_{k}^{x}\left(t\right):=\max\left\{n\geq0:A_{k}^{x}\left(0\right)+\psi_{k}\left(1\right)+\cdots+\psi_{k}\left(n-1\right)\leq t\right\}\\
S_{k}^{x}\left(t\right):=\max\left\{n\geq0:B_{k}^{x}\left(0\right)+\eta_{k}\left(1\right)+\cdots+\eta_{k}\left(n-1\right)\leq
t\right\}
\end{eqnarray}

Sea $T_{k}^{x}\left(t\right)$ el tiempo acumulado que el servidor
$s\left(k\right)$ ha utilizado en los usuarios de la clase $k$ en
el intervalo $\left[0,t\right]$. Entonces se tienen las siguientes
ecuaciones:

\begin{equation}
Q_{k}^{x}\left(t\right)=Q_{k}^{x}\left(0\right)+E_{k}^{x}\left(t\right)+\sum_{l=1}^{k}\Phi_{k}^{l}S_{l}^{x}\left(T_{l}^{x}\right)-S_{k}^{x}\left(T_{k}^{x}\right)\\
\end{equation}
\begin{equation}
Q^{x}\left(t\right)=\left(Q^{x}_{1}\left(t\right),\ldots,Q^{x}_{K}\left(t\right)\right)^{'}\geq0,\\
\end{equation}
\begin{equation}
T^{x}\left(t\right)=\left(T^{x}_{1}\left(t\right),\ldots,T^{x}_{K}\left(t\right)\right)^{'}\geq0,\textrm{ es no decreciente}\\
\end{equation}
\begin{equation}
I_{i}^{x}\left(t\right)=t-\sum_{k\in C_{i}}T_{k}^{x}\left(t\right)\textrm{ es no decreciente}\\
\end{equation}
\begin{equation}
\int_{0}^{\infty}\sum_{k\in C_{i}}Q_{k}^{x}\left(t\right)dI_{i}^{x}\left(t\right)=0\\
\end{equation}
\begin{equation}
\textrm{condiciones adicionales sobre
}\left(Q^{x}\left(\cdot\right),T^{x}\left(\cdot\right)\right)\textrm{
referentes a la disciplina de servicio}
\end{equation}

Para reducir la fluctuaci\'on del modelo se escala tanto el
espacio como el tiempo, entonces se tiene el proceso:

\begin{equation}
\overline{Q}^{x}\left(t\right)=\frac{1}{|x|}Q^{x}\left(|x|t\right)
\end{equation}
Cualquier l\'imite $\overline{Q}\left(t\right)$ es llamado un
flujo l\'imite del proceso longitud de la cola. Si se hace
$|q|\rightarrow\infty$ y se mantienen las componentes restantes
fijas, de la condici\'on inicial $x$, cualquier punto l\'imite del
proceso normalizado $\overline{Q}^{x}$ es una soluci\'on del
siguiente modelo de flujo, ver \cite{Dai}.

\begin{Def}
Un flujo l\'imite (retrasado) para una red bajo una disciplina de
servicio espec\'ifica se define como cualquier soluci\'on
 $\left(Q^{x}\left(\cdot\right),T^{x}\left(\cdot\right)\right)$ de las siguientes ecuaciones, donde
$\overline{Q}\left(t\right)=\left(\overline{Q}_{1}\left(t\right),\ldots,\overline{Q}_{K}\left(t\right)\right)^{'}$
y
$\overline{T}\left(t\right)=\left(\overline{T}_{1}\left(t\right),\ldots,\overline{T}_{K}\left(t\right)\right)^{'}$
\begin{equation}\label{Eq.3.8}
\overline{Q}_{k}\left(t\right)=\overline{Q}_{k}\left(0\right)+\alpha_{k}t-\mu_{k}\overline{T}_{k}\left(t\right)+\sum_{l=1}^{k}P_{lk}\mu_{l}\overline{T}_{l}\left(t\right)\\
\end{equation}
\begin{equation}\label{Eq.3.9}
\overline{Q}_{k}\left(t\right)\geq0\textrm{ para }k=1,2,\ldots,K,\\
\end{equation}
\begin{equation}\label{Eq.3.10}
\overline{T}_{k}\left(0\right)=0,\textrm{ y }\overline{T}_{k}\left(\cdot\right)\textrm{ es no decreciente},\\
\end{equation}
\begin{equation}\label{Eq.3.11}
\overline{I}_{i}\left(t\right)=t-\sum_{k\in C_{i}}\overline{T}_{k}\left(t\right)\textrm{ es no decreciente}\\
\end{equation}
\begin{equation}\label{Eq.3.12}
\overline{I}_{i}\left(\cdot\right)\textrm{ se incrementa al tiempo}t\textrm{ cuando }\sum_{k\in C_{i}}Q_{k}^{x}\left(t\right)dI_{i}^{x}\left(t\right)=0\\
\end{equation}
\begin{equation}\label{Eq.3.13}
\textrm{condiciones adicionales sobre
}\left(Q^{x}\left(\cdot\right),T^{x}\left(\cdot\right)\right)\textrm{
referentes a la disciplina de servicio}
\end{equation}
\end{Def}

Al conjunto de ecuaciones dadas en \ref{Eq.3.8}-\ref{Eq.3.13} se
le llama {\em Modelo de flujo} y al conjunto de todas las
soluciones del modelo de flujo
$\left(\overline{Q}\left(\cdot\right),\overline{T}
\left(\cdot\right)\right)$ se le denotar\'a por $\mathcal{Q}$.

Si se hace $|x|\rightarrow\infty$ sin restringir ninguna de las
componentes, tambi\'en se obtienen un modelo de flujo, pero en
este caso el residual de los procesos de arribo y servicio
introducen un retraso:

\begin{Def}
El modelo de flujo retrasado de una disciplina de servicio en una
red con retraso
$\left(\overline{A}\left(0\right),\overline{B}\left(0\right)\right)\in\rea_{+}^{K+|A|}$
se define como el conjunto de ecuaciones dadas en
\ref{Eq.3.8}-\ref{Eq.3.13}, junto con la condici\'on:
\begin{equation}\label{CondAd.FluidModel}
\overline{Q}\left(t\right)=\overline{Q}\left(0\right)+\left(\alpha
t-\overline{A}\left(0\right)\right)^{+}-\left(I-P^{'}\right)M\left(\overline{T}\left(t\right)-\overline{B}\left(0\right)\right)^{+}
\end{equation}
\end{Def}

\begin{Def}
El modelo de flujo es estable si existe un tiempo fijo $t_{0}$ tal
que $\overline{Q}\left(t\right)=0$, con $t\geq t_{0}$, para
cualquier $\overline{Q}\left(\cdot\right)\in\mathcal{Q}$ que
cumple con $|\overline{Q}\left(0\right)|=1$.
\end{Def}

El siguiente resultado se encuentra en \cite{Chen}.
\begin{Lemma}
Si el modelo de flujo definido por \ref{Eq.3.8}-\ref{Eq.3.13} es
estable, entonces el modelo de flujo retrasado es tambi\'en
estable, es decir, existe $t_{0}>0$ tal que
$\overline{Q}\left(t\right)=0$ para cualquier $t\geq t_{0}$, para
cualquier soluci\'on del modelo de flujo retrasado cuya
condici\'on inicial $\overline{x}$ satisface que
$|overline{x}|=|\overline{Q}\left(0\right)|+|\overline{A}\left(0\right)|+|\overline{B}\left(0\right)|\leq1$.
\end{Lemma}

%_____________________________________________________________________
\subsubsection{Resultados principales}
%_____________________________________________________________________
Supuestos necesarios sobre la red

\begin{Sup}
\begin{itemize}
\item[A1)] $\psi_{1},\ldots,\psi_{K},\eta_{1},\ldots,\eta_{K}$ son
mutuamente independientes y son sucesiones independientes e
id\'enticamente distribuidas.

\item[A2)] Para alg\'un entero $p\geq1$
\begin{eqnarray*}
\esp\left[\psi_{l}\left(1\right)^{p+1}\right]<\infty\textrm{ para }l\in\mathcal{A}\textrm{ y }\\
\esp\left[\eta_{k}\left(1\right)^{p+1}\right]<\infty\textrm{ para
}k=1,\ldots,K.
\end{eqnarray*}
\item[A3)] El conjunto $\left\{x\in X:|x|=0\right\}$ es un
singleton, y para cada $k\in\mathcal{A}$, existe una funci\'on
positiva $q_{k}\left(x\right)$ definida en $\rea_{+}$, y un entero
$j_{k}$, tal que
\begin{eqnarray}
P\left(\psi_{k}\left(1\right)\geq x\right)>0\textrm{, para todo }x>0\\
P\left(\psi_{k}\left(1\right)+\ldots\psi_{k}\left(j_{k}\right)\in dx\right)\geq q_{k}\left(x\right)dx0\textrm{ y }\\
\int_{0}^{\infty}q_{k}\left(x\right)dx>0
\end{eqnarray}
\end{itemize}
\end{Sup}

El argumento dado en \cite{MaynDown} en el lema
\ref{Lema.34.MeynDown} se puede aplicar para deducir que todos los
subconjuntos compactos de $X$ son peque\~nos.Entonces la
condici\'on $A3)$ se puede generalizar a
\begin{itemize}
\item[A3')] Para el proceso de Markov $X$, cada subconjunto
compacto de $X$ es peque\~no.
\end{itemize}

\begin{Teo}\label{Tma.4.1}
Suponga que el modelo de flujo para una disciplina de servicio es
estable, y suponga adem\'as que las condiciones A1) y A2) se
satisfacen. Entonces:
\begin{itemize}
\item[i)] Para alguna constante $\kappa_{p}$, y para cada
condici\'on inicial $x\in X$
\begin{equation}
\limsup_{t\rightarrow\infty}\frac{1}{t}\int_{0}^{t}\esp_{x}\left[|Q\left(t\right)|^{p}\right]ds\leq\kappa_{p}
\end{equation}
donde $p$ es el entero dado por A2). Suponga adem\'as que A3) o A3')
se cumple, entonces la disciplina de servicio es estable y adem\'as
para cada condici\'on inicial se tiene lo siguiente: \item[ii)] Los
momentos transitorios convergen a sus valores en estado
estacionario:
\begin{equation}
\lim_{t\rightarrow\infty}\esp_{x}\left[Q_{k}\left(t\right)^{r}\right]=\esp_{\pi}\left[Q_{k}\left(0\right)^{r}\right]\leq\kappa_{r}
\end{equation}
para $r=1,\ldots,p$ y $k=1,\ldots,K$. \item[iii)] EL primer
momento converge con raz\'on $t^{p-1}$:
\begin{equation}
\lim_{t\rightarrow\infty}t^{p-1}|\esp_{x}\left[Q\left(t\right)\right]-\esp_{\pi}\left[Q\left(0\right)\right]|=0.
\end{equation}
\item[iv)] Se cumple la Ley Fuerte de los Grandes N\'umeros:
\begin{equation}
\lim_{t\rightarrow\infty}\frac{1}{t}\int_{0}^{t}Q_{k}^{r}\left(s\right)ds=\esp_{\pi}\left[Q_{k}\left(0\right)^{r}\right]
\end{equation}
$\prob$-c.s., para $r=1,\ldots,p$ y $k=1,\ldots,K$.
\end{itemize}
\end{Teo}
\begin{Dem}
La demostraci\'on de este resultado se da aplicando los teoremas
\ref{Tma.5.5}, \ref{Tma.6.2}, \ref{Tma.6.3} y \ref{Tma.6.4}
\end{Dem}

%_____________________________________________________________________
\subsubsection{Definiciones Generales}
%_____________________________________________________________________
Definimos un proceso de estados para la red que depende de la
pol\'itica de servicio utilizada. Bajo cualquier {\em preemptive
buffer priority} disciplina de servicio, el estado
$\mathbb{X}\left(t\right)$ a cualquier tiempo $t$ puede definirse
como
\begin{equation}\label{Eq.Esp.Estados}
\mathbb{X}\left(t\right)=\left(Q_{k}\left(t\right),A_{l}\left(t\right),B_{k}\left(t\right):k=1,2,\ldots,K,l\in\mathcal{A}\right)
\end{equation}
donde $Q_{k}\left(t\right)$ es la longitud de la cola para los
usuarios de la clase $k$, incluyendo aquellos que est\'an siendo
atendidos, $B_{k}\left(t\right)$ son los tiempos de servicio
residuales para los usuarios de la clase $k$ que est\'an en
servicio. Los tiempos de arribo residuales, que son iguales al
tiempo que queda hasta que el pr\'oximo usuario de la clase $k$
llega, se denotan por $A_{k}\left(t\right)$. Tanto
$B_{k}\left(t\right)$ como $A_{k}\left(t\right)$ se suponen
continuos por la derecha.

Sea $\mathbb{X}$ el espacio de estados para el proceso de estados
que por definici\'on es igual  al conjunto de posibles valores
para el estado $\mathbb{X}\left(t\right)$, y sea
$x=\left(q,a,b\right)$ un estado gen\'erico en $\mathbb{X}$, la
componente $q$ determina la posici\'on del usuario en la red,
$|q|$ denota la longitud total de la cola en la red.

Para un estado $x=\left(q,a,b\right)\in\mathbb{X}$ definimos la
{\em norma} de $x$ como $\left\|x\right|=|q|+|a|+|b|$. En
\cite{Dai} se muestra que para una amplia serie de disciplinas de
servicio el proceso $\mathbb{X}$ es un Proceso Fuerte de Markov, y
por tanto se puede asumir que
\[\left(\left(\Omega,\mathcal{F}\right),\mathcal{F}_{t},\mathbb{X}\left(t\right),\theta_{t},P_{x}\right)\]
es un proceso de Borel Derecho en el espacio de estadio medible
$\left(\mathbb{X},\mathcal{B}_{\mathbb{X}}\right)$. El Proceso
$X=\left\{\mathbb{X}\left(t\right),t\geq0\right\}$ tiene
trayectorias continuas por la derecha, est definida en
$\left(\Omega,\mathcal{F}\right)$ y est adaptado a
$\left\{\mathcal{F}_{t},t\geq0\right\}$; $\left\{P_{x},x\in
X\right\}$ son medidas de probabilidad en
$\left(\Omega,\mathcal{F}\right)$ tales que para todo $x\in X$
\[P_{x}\left\{\mathbb{X}\left(0\right)=x\right\}=1\] y
\[E_{x}\left\{f\left(X\circ\theta_{t}\right)|\mathcal{F}_{t}\right\}=E_{X}\left(\tau\right)f\left(X\right)\]
en $\left\{\tau<\infty\right\}$, $P_{x}$-c.s. Donde $\tau$ es un
$\mathcal{F}_{t}$-tiempo de paro
\[\left(X\circ\theta_{\tau}\right)\left(w\right)=\left\{\mathbb{X}\left(\tau\left(w\right)+t,w\right),t\geq0\right\}\]
y $f$ es una funci\'on de valores reales acotada y medible con la
sigma algebra de Kolmogorov generada por los cilindros.

Sea $P^{t}\left(x,D\right)$, $D\in\mathcal{B}_{\mathbb{X}}$,
$t\geq0$ probabilidad de transici\'on de $X$ definida como
\[P^{t}\left(x,D\right)=P_{x}\left(\mathbb{X}\left(t\right)\in
D\right)\]

\begin{Def}
Una medida no cero $\pi$ en
$\left(\mathbb{X},\mathcal{B}_{\mathbb{X}}\right)$ es {\em
invariante} para $X$ si $\pi$ es $\sigma$-finita y
\[\pi\left(D\right)=\int_{X}P^{t}\left(x,D\right)\pi\left(dx\right)\]
para todo $D\in \mathcal{B}_{\mathbb{X}}$, con $t\geq0$.
\end{Def}

\begin{Def}
El proceso de Markov $X$ es llamado {\em Harris recurrente} si
existe una medida de probabilidad $\nu$ en
$\left(\mathbb{X},\mathcal{B}_{\mathbb{X}}\right)$, tal que si
$\nu\left(D\right)>0$ y $D\in\mathcal{B}_{\mathbb{X}}$
\[P_{x}\left\{\tau_{D}<\infty\right\}\equiv1\] cuando
$\tau_{D}=\inf\left\{t\geq0:\mathbb{X}_{t}\in D\right\}$.
\end{Def}

\begin{itemize}
\item Si $X$ es Harris recurrente, entonces una \'unica medida
invariante $\pi$ existe (\cite{Getoor}). \item Si la medida
invariante es finita, entonces puede normalizarse a una medida de
probabilidad, en este caso se le llama {\em Harris recurrente
positiva}. \item Cuando $X$ es Harris recurrente positivo se dice
que la disciplina de servicio es estable. En este caso $\pi$
denota la ditribuci\'on estacionaria y hacemos
\[P_{\pi}\left(\cdot\right)[=\int_{X}P_{x}\left(\cdot\right)\pi\left(dx\right)\]
y se utiliza $E_{\pi}$ para denotar el operador esperanza
correspondiente, as, el proceso
$X=\left\{\mathbb{X}\left(t\right),t\geq0\right\}$ es un proceso
estrictamente estacionario bajo $P_{\pi}$
\end{itemize}

\begin{Def}
Un conjunto $D\in\mathcal{B}_\mathbb{X}$ es llamado peque\~no si
existe un $t>0$, una medida de probabilidad $\nu$ en
$\mathcal{B}_\mathbb{X}$, y un $\delta>0$ tal que
\[P^{t}\left(x,A\right)\geq\delta\nu\left(A\right)\] para $x\in
D,A\in\mathcal{B}_\mathbb{X}$.\footnote{En \cite{MeynTweedie}
muestran que si $P_{x}\left\{\tau_{D}<\infty\right\}\equiv1$
solamente para uno conjunto peque\~no, entonces el proceso es
Harris recurrente}
\end{Def}

%_____________________________________________________________________
\subsubsection{Definiciones y Descripci\'on del Modelo}
%________________________________________________________________________
El modelo est\'a compuesto por $c$ colas de capacidad infinita,
etiquetadas de $1$ a $c$ las cuales son atendidas por $s$
servidores. Los servidores atienden de acuerdo a una cadena de
Markov independiente $\left(X^{i}_{n}\right)_{n}$ con $1\leq i\leq
s$ y $n\in\left\{1,2,\ldots,c\right\}$ con la misma matriz de
transici\'on $r_{k,l}$ y \'unica medida invariante
$\left(p_{k}\right)$. Cada servidor permanece atendiendo en la
cola un periodo llamado de visita y determinada por la pol\'itica de
servicio asignada a la cola.

Los usuarios llegan a la cola $k$ con una tasa $\lambda_{k}$ y son
atendidos a una raz\'on $\mu_{k}$. Las sucesiones de tiempos de
interarribo $\left(\tau_{k}\left(n\right)\right)_{n}$, la de
tiempos de servicio
$\left(\sigma_{k}^{i}\left(n\right)\right)_{n}$ y la de tiempos de
cambio $\left(\sigma_{k,l}^{0,i}\left(n\right)\right)_{n}$
requeridas en la cola $k$ para el servidor $i$ son sucesiones
independientes e id\'enticamente distribuidas con distribuci\'on
general independiente de $i$, con media
$\sigma_{k}=\frac{1}{\mu_{k}}$, respectivamente
$\sigma_{k,l}^{0}=\frac{1}{\mu_{k,l}^{0}}$, e independiente de las
cadenas de Markov $\left(X^{i}_{n}\right)_{n}$. Adem\'as se supone
que los tiempos de interarribo se asume son acotados, para cada
$\rho_{k}=\lambda_{k}\sigma_{k}<s$ para asegurar la estabilidad de
la cola $k$ cuando opera como una cola $M/GM/1$.
%________________________________________________________________________
\subsubsection{Pol\'iticas de Servicio}
%_____________________________________________________________________
Una pol\'itica de servicio determina el n\'umero de usuarios que ser\'an
atendidos sin interrupci\'on en periodo de servicio por los
servidores que atienden a la cola. Para un solo servidor esta se
define a trav\'es de una funci\'on $f$ donde $f\left(x,a\right)$ es el
n\'umero de usuarios que son atendidos sin interrupci\'on cuando el
servidor llega a la cola y encuentra $x$ usuarios esperando dado
el tiempo transcurrido de interarribo $a$. Sea $v\left(x,a\right)$
la duraci\'on del periodo de servicio para una sola condici\'on
inicial $\left(x,a\right)$.

Las pol\'iticas de servicio consideradas satisfacen las siguientes
propiedades:

\begin{itemize}
\item[i)] Hay conservaci\'on del trabajo, es decir
\[v\left(x,a\right)=\sum_{l=1}^{f\left(x,a\right)}\sigma\left(l\right)\]
con $f\left(0,a\right)=v\left(0,a\right)=0$, donde
$\left(\sigma\left(l\right)\right)_{l}$ es una sucesi\'on
independiente e id\'enticamente distribuida de los tiempos de
servicio solicitados. \item[ii)] La selecci\'on de usuarios para se
atendidos es independiente de sus correspondientes tiempos de
servicio y del pasado hasta el inicio del periodo de servicio. As\'i
las distribuci\'on $\left(f,v\right)$ no depende del orden en el
cu\'al son atendidos los usuarios. \item[iii)] La pol\'itica de
servicio es mon\'otona en el sentido de que para cada $a\geq0$ los
n\'umeros $f\left(x,a\right)$ son mon\'otonos en distribuci\'on en $x$ y
su l\'imite en distribuci\'on cuando $x\rightarrow\infty$ es una
variable aleatoria $F^{*0}$ que no depende de $a$. \item[iv)] El
n\'umero de usuarios atendidos por cada servidor es acotado por
$f^{min}\left(x\right)$ de la longitud de la cola $x$ que adem\'as
converge mon\'otonamente en distribuci\'on a $F^{*}$ cuando
$x\rightarrow\infty$
\end{itemize}
%________________________________________________________________________
\subsubsection{Proceso de Estados}
%_____________________________________________________________________
El sistema de colas se describe por medio del proceso de Markov
$\left(X\left(t\right)\right)_{t\in\rea}$ como se define a
continuaci\'on. El estado del sistema al tiempo $t\geq0$ est\'a dado
por
\[X\left(t\right)=\left(Q\left(t\right),P\left(t\right),A\left(t\right),R\left(t\right),C\left(t\right)\right)\]
donde
\begin{itemize}
\item
$Q\left(t\right)=\left(Q_{k}\left(t\right)\right)_{k=1}^{c}$,
n\'umero de usuarios en la cola $k$ al tiempo $t$. \item
$P\left(t\right)=\left(P^{i}\left(t\right)\right)_{i=1}^{s}$, es
la posici\'on del servidor $i$. \item
$A\left(t\right)=\left(A_{k}\left(t\right)\right)_{k=1}^{c}$, es
el residual del tiempo de arribo en la cola $k$ al tiempo $t$.
\item
$R\left(t\right)=\left(R_{k}^{i}\left(t\right),R_{k,l}^{0,i}\left(t\right)\right)_{k,l,i=1}^{c,c,s}$,
el primero es el residual del tiempo de servicio del usuario
atendido por servidor $i$ en la cola $k$ al tiempo $t$, la segunda
componente es el residual del tiempo de cambio del servidor $i$ de
la cola $k$ a la cola $l$ al tiempo $t$. \item
$C\left(t\right)=\left(C_{k}^{i}\left(t\right)\right)_{k,i=1}^{c,s}$,
es la componente correspondiente a la cola $k$ y al servidor $i$
que est\'a determinada por la pol\'itica de servicio en la cola $k$
y que hace al proceso $X\left(t\right)$ un proceso de Markov.
\end{itemize}
Todos los procesos definidos arriba se suponen continuos por la
derecha.

El proceso $X$ tiene la propiedad fuerte de Markov y su espacio de
estados es el espacio producto
\[\mathcal{X}=\nat^{c}\times E^{s}\times \rea_{+}^{c}\times\rea_{+}^{cs}\times\rea_{+}^{c^{2}s}\times \mathcal{C}\] donde $E=\left\{1,2,\ldots,c\right\}^{2}\cup\left\{1,2,\ldots,c\right\}$ y $\mathcal{C}$  depende de las pol\'iticas de servicio.

%_____________________________________________________________________________________
\subsubsection{Introducci{\'o}n}
%_____________________________________________________________________________________
%


Si $x$ es el n{\'u}mero de usuarios en la cola al comienzo del
periodo de servicio y $N_{s}\left(x\right)=N\left(x\right)$ es el
n{\'u}mero de usuarios que son atendidos con la pol{\'\i}tica $s$,
{\'u}nica en nuestro caso durante un periodo de servicio, entonces
se asume que:
\begin{enumerate}
\item
\begin{equation}\label{S1}
lim_{x\rightarrow\infty}\esp\left[N\left(x\right)\right]=\overline{N}>0
\end{equation}
\item
\begin{equation}\label{S2}
\esp\left[N\left(x\right)\right]\leq \overline{N} \end{equation}
para cualquier valor de $x$.
\end{enumerate}
La manera en que atiende el servidor $m$-{\'e}simo, en este caso
en espec{\'\i}fico solo lo ilustraremos con un s{\'o}lo servidor,
es la siguiente:
\begin{itemize}
\item Al t{\'e}rmino de la visita a la cola $j$, el servidor se
cambia a la cola $j^{'}$ con probabilidad
$r_{j,j^{'}}^{m}=r_{j,j^{'}}$

\item La $n$-{\'e}sima ocurencia va acompa{\~n}ada con el tiempo
de cambio de longitud $\delta_{j,j^{'}}\left(n\right)$,
independientes e id{\'e}nticamente distribuidas, con
$\esp\left[\delta_{j,j^{'}}\left(1\right)\right]\geq0$.

\item Sea $\left\{p_{j}\right\}$ la {\'u}nica distribuci{\'o}n
invariante estacionaria para la Cadena de Markov con matriz de
transici{\'o}n $\left(r_{j,j^{'}}\right)$.

\item Finalmente, se define
\begin{equation}
\delta^{*}:=\sum_{j,j^{'}}p_{j}r_{j,j^{'}}\esp\left[\delta_{j,j^{'}}\left(1\right)\right].
\end{equation}
\end{itemize}
%_____________________________________________________________________
\subsubsection{Colas C\'iclicas}
%_____________________________________________________________________
El {\em token passing ring} es una estaci\'on de un solo servidor
con $K$ clases de usuarios. Cada clase tiene su propio regulador
en la estaci\'on. Los usuarios llegan al regulador con raz\'on
$\alpha_{k}$ y son atendidos con taza $\mu_{k}$.

La red se puede modelar como un Proceso de Markov con espacio de
estados continuo, continuo en el tiempo:
\begin{equation}
 X\left(t\right)^{T}=\left(Q_{k}\left(t\right),A_{l}\left(t\right),B_{k}\left(t\right),B_{k}^{0}\left(t\right),C\left(t\right):k=1,\ldots,K,l\in\mathcal{A}\right)
\end{equation}
donde $Q_{k}\left(t\right), B_{k}\left(t\right)$ y
$A_{k}\left(t\right)$ se define como en \ref{Eq.Esp.Estados},
$B_{k}^{0}\left(t\right)$ es el tiempo residual de cambio de la
clase $k$ a la clase $k+1\left(mod K\right)$; $C\left(t\right)$
indica el n\'umero de servicios que han sido comenzados y/o
completados durante la sesi\'on activa del buffer.

Los par\'ametros cruciales son la carga nominal de la cola $k$:
$\beta_{k}=\alpha_{k}/\mu_{k}$ y la carga total es
$\rho_{0}=\sum\beta_{k}$, la media total del tiempo de cambio en
un ciclo del token est\'a definido por
\begin{equation}
 u^{0}=\sum_{k=1}^{K}\esp\left[\eta_{k}^{0}\left(1\right)\right]=\sum_{k=1}^{K}\frac{1}{\mu_{k}^{0}}
\end{equation}

El proceso de la longitud de la cola $Q_{k}^{x}\left(t\right)$ y
el proceso de acumulaci\'on del tiempo de servicio
$T_{k}^{x}\left(t\right)$ para el buffer $k$ y para el estado
inicial $x$ se definen como antes. Sea $T_{k}^{x,0}\left(t\right)$
el tiempo acumulado al tiempo $t$ que el token tarda en cambiar
del buffer $k$ al $k+1\mod K$. Suponga que la funci\'on
$\left(\overline{Q}\left(\cdot\right),\overline{T}\left(\cdot\right),\overline{T}^{0}\left(\cdot\right)\right)$
es un punto l\'imite de
\begin{equation}\label{Eq.4.4}
\left(\frac{1}{|x|}Q^{x}\left(|x|t\right),\frac{1}{|x|}T^{x}\left(|x|t\right),\frac{1}{|x|}T^{x,0}\left(|x|t\right)\right)
\end{equation}
cuando $|x|\rightarrow\infty$. Entonces
$\left(\overline{Q}\left(t\right),\overline{T}\left(t\right),\overline{T}^{0}\left(t\right)\right)$
es un flujo l\'imite retrasado del token ring.

Propiedades importantes para el modelo de flujo retrasado

\begin{Prop}
 Sea $\left(\overline{Q},\overline{T},\overline{T}^{0}\right)$ un flujo l\'imite de \ref{Eq.4.4} y suponga que cuando $x\rightarrow\infty$ a lo largo de
una subsucesi\'on
\[\left(\frac{1}{|x|}Q_{k}^{x}\left(0\right),\frac{1}{|x|}A_{k}^{x}\left(0\right),\frac{1}{|x|}B_{k}^{x}\left(0\right),\frac{1}{|x|}B_{k}^{x,0}\left(0\right)\right)\rightarrow\left(\overline{Q}_{k}\left(0\right),0,0,0\right)\]
para $k=1,\ldots,K$. EL flujo l\'imite tiene las siguientes
propiedades, donde las propiedades de la derivada se cumplen donde
la derivada exista:
\begin{itemize}
 \item[i)] Los vectores de tiempo ocupado $\overline{T}\left(t\right)$ y $\overline{T}^{0}\left(t\right)$ son crecientes y continuas con
$\overline{T}\left(0\right)=\overline{T}^{0}\left(0\right)=0$.
\item[ii)] Para todo $t\geq0$
\[\sum_{k=1}^{K}\left[\overline{T}_{k}\left(t\right)+\overline{T}_{k}^{0}\left(t\right)\right]=t\]
\item[iii)] Para todo $1\leq k\leq K$
\[\overline{Q}_{k}\left(t\right)=\overline{Q}_{k}\left(0\right)+\alpha_{k}t-\mu_{k}\overline{T}_{k}\left(t\right)\]
\item[iv)]  Para todo $1\leq k\leq K$
\[\dot{{\overline{T}}}_{k}\left(t\right)=\beta_{k}\] para $\overline{Q}_{k}\left(t\right)=0$.
\item[v)] Para todo $k,j$
\[\mu_{k}^{0}\overline{T}_{k}^{0}\left(t\right)=\mu_{j}^{0}\overline{T}_{j}^{0}\left(t\right)\]
\item[vi)]  Para todo $1\leq k\leq K$
\[\mu_{k}\dot{{\overline{T}}}_{k}\left(t\right)=l_{k}\mu_{k}^{0}\dot{{\overline{T}}}_{k}^{0}\left(t\right)\] para $\overline{Q}_{k}\left(t\right)>0$.
\end{itemize}
\end{Prop}

%_____________________________________________________________________
\subsubsection{Resultados Previos}
%_____________________________________________________________________

\begin{Lemma}\label{Lema.34.MeynDown}
El proceso estoc\'astico $\Phi$ es un proceso de markov fuerte,
temporalmente homog\'eneo, con trayectorias muestrales continuas
por la derecha, cuyo espacio de estados $Y$ es igual a
$X\times\rea$
\end{Lemma}
\begin{Prop}
 Suponga que los supuestos A1) y A2) son ciertos y que el modelo de flujo es estable. Entonces existe $t_{0}>0$ tal que
\begin{equation}
 lim_{|x|\rightarrow\infty}\frac{1}{|x|^{p+1}}\esp_{x}\left[|X\left(t_{0}|x|\right)|^{p+1}\right]=0
\end{equation}
\end{Prop}

\begin{Lemma}\label{Lema.5.2}
 Sea $\left\{\zeta\left(k\right):k\in \mathbb{z}\right\}$ una sucesi\'on independiente e id\'enticamente distribuida que toma valores en $\left(0,\infty\right)$,
y sea
$E\left(t\right)=max\left(n\geq1:\zeta\left(1\right)+\cdots+\zeta\left(n-1\right)\leq
t\right)$. Si $\esp\left[\zeta\left(1\right)\right]<\infty$,
entonces para cualquier entero $r\geq1$
\begin{equation}
 lim_{t\rightarrow\infty}\esp\left[\left(\frac{E\left(t\right)}{t}\right)^{r}\right]=\left(\frac{1}{\esp\left[\zeta_{1}\right]}\right)^{r}.
\end{equation}
Luego, bajo estas condiciones:
\begin{itemize}
 \item[a)] para cualquier $\delta>0$, $\sup_{t\geq\delta}\esp\left[\left(\frac{E\left(t\right)}{t}\right)^{r}\right]<\infty$
\item[b)] las variables aleatorias
$\left\{\left(\frac{E\left(t\right)}{t}\right)^{r}:t\geq1\right\}$
son uniformemente integrables.
\end{itemize}
\end{Lemma}

\begin{Teo}\label{Tma.5.5}
Suponga que los supuestos A1) y A2) se cumplen y que el modelo de
flujo es estable. Entonces existe una constante $\kappa_{p}$ tal
que
\begin{equation}
\frac{1}{t}\int_{0}^{t}\esp_{x}\left[|Q\left(s\right)|^{p}\right]ds\leq\kappa_{p}\left\{\frac{1}{t}|x|^{p+1}+1\right\}
\end{equation}
para $t>0$ y $x\in X$. En particular, para cada condici\'on inicial
\begin{eqnarray*}
\limsup_{t\rightarrow\infty}\frac{1}{t}\int_{0}^{t}\esp_{x}\left[|Q\left(s\right)|^{p}\right]ds\leq\kappa_{p}.
\end{eqnarray*}
\end{Teo}

\begin{Teo}\label{Tma.6.2}
Suponga que se cumplen los supuestos A1), A2) y A3) y que el
modelo de flujo es estable. Entonces se tiene que
\begin{equation}
|\left|P^{t}\left(x,\cdot\right)-\pi\left(\cdot\right)\right||_{f_{p}}\textrm{,
}t\rightarrow\infty,x\in X.
\end{equation}
En particular para cada condici\'on inicial
\begin{eqnarray*}
\lim_{t\rightarrow\infty}\esp_{x}\left[|Q\left(t\right)|^{p}\right]=\esp_{\pi}\left[|Q\left(0\right)|^{p}\right]\leq\kappa_{r}
\end{eqnarray*}
\end{Teo}
\begin{Teo}\label{Tma.6.3}
Suponga que se cumplen los supuestos A1), A2) y A3) y que el
modelo de flujo es estable. Entonces con
$f\left(x\right)=f_{1}\left(x\right)$ se tiene
\begin{equation}
\lim_{t\rightarrow\infty}t^{p-1}|\left|P^{t}\left(x,\cdot\right)-\pi\left(\cdot\right)\right||_{f}=0.
\end{equation}
En particular para cada condici\'on inicial
\begin{eqnarray*}
\lim_{t\rightarrow\infty}t^{p-1}|\esp_{x}\left[Q\left(t\right)\right]-\esp_{\pi}\left[Q\left(0\right)\right]|=0.
\end{eqnarray*}
\end{Teo}

\begin{Teo}\label{Tma.6.4}
Suponga que se cumplen los supuestos A1), A2) y A3) y que el
modelo de flujo es estable. Sea $\nu$ cualquier distribuci\'on de
probabilidad en $\left(X,\mathcal{B}_{X}\right)$, y $\pi$ la
distribuci\'on estacionaria de $X$.
\begin{itemize}
\item[i)] Para cualquier $f:X\leftarrow\rea_{+}$
\begin{equation}
\lim_{t\rightarrow\infty}\frac{1}{t}\int_{o}^{t}f\left(X\left(s\right)\right)ds=\pi\left(f\right):=\int
f\left(x\right)\pi\left(dx\right)
\end{equation}
$\prob$-c.s. \item[ii)] Para cualquier $f:X\leftarrow\rea_{+}$ con
$\pi\left(|f|\right)<\infty$, la ecuaci\'on anterior se cumple.
\end{itemize}
\end{Teo}

%_____________________________________________________________________________________
%
\subsubsection{Teorema de Estabilidad: Descripci{\'o}n}
%_____________________________________________________________________________________
%


Si $x$ es el n{\'u}mero de usuarios en la cola al comienzo del
periodo de servicio y $N_{s}\left(x\right)=N\left(x\right)$ es el
n{\'u}mero de usuarios que son atendidos con la pol{\'\i}tica $s$,
{\'u}nica en nuestro caso durante un periodo de servicio, entonces
se asume que:
\begin{enumerate}
\item
\begin{equation}\label{S1}
lim_{x\rightarrow\infty}\esp\left[N\left(x\right)\right]=\overline{N}>0
\end{equation}
\item
\begin{equation}\label{S2}
\esp\left[N\left(x\right)\right]\leq \overline{N} \end{equation}
para cualquier valor de $x$.
\end{enumerate}
La manera en que atiende el servidor $m$-{\'e}simo, en este caso
en espec{\'\i}fico solo lo ilustraremos con un s{\'o}lo servidor,
es la siguiente:
\begin{itemize}
\item Al t{\'e}rmino de la visita a la cola $j$, el servidor se
cambia a la cola $j^{'}$ con probabilidad
$r_{j,j^{'}}^{m}=r_{j,j^{'}}$

\item La $n$-{\'e}sima ocurencia va acompa{\~n}ada con el tiempo
de cambio de longitud $\delta_{j,j^{'}}\left(n\right)$,
independientes e id{\'e}nticamente distribuidas, con
$\esp\left[\delta_{j,j^{'}}\left(1\right)\right]\geq0$.

\item Sea $\left\{p_{j}\right\}$ la {\'u}nica distribuci{\'o}n
invariante estacionaria para la Cadena de Markov con matriz de
transici{\'o}n $\left(r_{j,j^{'}}\right)$.

\item Finalmente, se define
\begin{equation}
\delta^{*}:=\sum_{j,j^{'}}p_{j}r_{j,j^{'}}\esp\left[\delta_{j,j^{'}}\left(1\right)\right].
\end{equation}
\end{itemize}

%_________________________________________________________________________
\subsection{Supuestos}
%_________________________________________________________________________
Consideremos el caso en el que se tienen varias colas a las cuales
llegan uno o varios servidores para dar servicio a los usuarios
que se encuentran presentes en la cola, como ya se mencion\'o hay
varios tipos de pol\'iticas de servicio, incluso podr\'ia ocurrir
que la manera en que atiende al resto de las colas sea distinta a
como lo hizo en las anteriores.\\

Para ejemplificar los sistemas de visitas c\'iclicas se
considerar\'a el caso en que a las colas los usuarios son atendidos con
una s\'ola pol\'itica de servicio.\\



Si $\omega$ es el n\'umero de usuarios en la cola al comienzo del
periodo de servicio y $N\left(\omega\right)$ es el n\'umero de
usuarios que son atendidos con una pol\'itica en espec\'ifico
durante el periodo de servicio, entonces se asume que:
\begin{itemize}
\item[1)]\label{S1}$lim_{\omega\rightarrow\infty}\esp\left[N\left(\omega\right)\right]=\overline{N}>0$;
\item[2)]\label{S2}$\esp\left[N\left(\omega\right)\right]\leq\overline{N}$
para cualquier valor de $\omega$.
\end{itemize}
La manera en que atiende el servidor $m$-\'esimo, es la siguiente:
\begin{itemize}
\item Al t\'ermino de la visita a la cola $j$, el servidor cambia
a la cola $j^{'}$ con probabilidad $r_{j,j^{'}}^{m}$

\item La $n$-\'esima vez que el servidor cambia de la cola $j$ a
$j'$, va acompa\~nada con el tiempo de cambio de longitud
$\delta_{j,j^{'}}^{m}\left(n\right)$, con
$\delta_{j,j^{'}}^{m}\left(n\right)$, $n\geq1$, variables
aleatorias independientes e id\'enticamente distribuidas, tales
que $\esp\left[\delta_{j,j^{'}}^{m}\left(1\right)\right]\geq0$.

\item Sea $\left\{p_{j}^{m}\right\}$ la distribuci\'on invariante
estacionaria \'unica para la Cadena de Markov con matriz de
transici\'on $\left(r_{j,j^{'}}^{m}\right)$, se supone que \'esta
existe.

\item Finalmente, se define el tiempo promedio total de traslado
entre las colas como
\begin{equation}
\delta^{*}:=\sum_{j,j^{'}}p_{j}^{m}r_{j,j^{'}}^{m}\esp\left[\delta_{j,j^{'}}^{m}\left(i\right)\right].
\end{equation}
\end{itemize}

Consideremos el caso donde los tiempos entre arribo a cada una de
las colas, $\left\{\xi_{k}\left(n\right)\right\}_{n\geq1}$ son
variables aleatorias independientes a id\'enticamente
distribuidas, y los tiempos de servicio en cada una de las colas
se distribuyen de manera independiente e id\'enticamente
distribuidas $\left\{\eta_{k}\left(n\right)\right\}_{n\geq1}$;
adem\'as ambos procesos cumplen la condici\'on de ser
independientes entre s\'i. Para la $k$-\'esima cola se define la
tasa de arribo por
$\lambda_{k}=1/\esp\left[\xi_{k}\left(1\right)\right]$ y la tasa
de servicio como
$\mu_{k}=1/\esp\left[\eta_{k}\left(1\right)\right]$, finalmente se
define la carga de la cola como $\rho_{k}=\lambda_{k}/\mu_{k}$,
donde se pide que $\rho=\sum_{k=1}^{K}\rho_{k}<1$, para garantizar
la estabilidad del sistema, esto es cierto para las pol\'iticas de
servicio exhaustiva y cerrada, ver Geetor \cite{Getoor}.\\

Si denotamos por
\begin{itemize}
\item $Q_{k}\left(t\right)$ el n\'umero de usuarios presentes en
la cola $k$ al tiempo $t$; \item $A_{k}\left(t\right)$ los
residuales de los tiempos entre arribos a la cola $k$; para cada
servidor $m$; \item $B_{m}\left(t\right)$ denota a los residuales
de los tiempos de servicio al tiempo $t$; \item
$B_{m}^{0}\left(t\right)$ los residuales de los tiempos de
traslado de la cola $k$ a la pr\'oxima por atender al tiempo $t$,

\item sea
$C_{m}\left(t\right)$ el n\'umero de usuarios atendidos durante la
visita del servidor a la cola $k$ al tiempo $t$.
\end{itemize}


En este sentido, el proceso para el sistema de visitas se puede
definir como:

\begin{equation}\label{Esp.Edos.Down}
X\left(t\right)^{T}=\left(Q_{k}\left(t\right),A_{k}\left(t\right),B_{m}\left(t\right),B_{m}^{0}\left(t\right),C_{m}\left(t\right)\right),
\end{equation}
para $k=1,\ldots,K$ y $m=1,2,\ldots,M$, donde $T$ indica que es el
transpuesto del vector que se est\'a definiendo. El proceso $X$
evoluciona en el espacio de estados:
$\mathbb{X}=\ent_{+}^{K}\times\rea_{+}^{K}\times\left(\left\{1,2,\ldots,K\right\}\times\left\{1,2,\ldots,S\right\}\right)^{M}\times\rea_{+}^{K}\times\ent_{+}^{K}$.\\

El sistema aqu\'i descrito debe de cumplir con los siguientes supuestos b\'asicos de un sistema de visitas:
%__________________________________________________________________________
\subsubsection{Supuestos B\'asicos}
%__________________________________________________________________________
\begin{itemize}
\item[A1)] Los procesos
$\xi_{1},\ldots,\xi_{K},\eta_{1},\ldots,\eta_{K}$ son mutuamente
independientes y son sucesiones independientes e id\'enticamente
distribuidas.

\item[A2)] Para alg\'un entero $p\geq1$
\begin{eqnarray*}
\esp\left[\xi_{l}\left(1\right)^{p+1}\right]&<&\infty\textrm{ para }l=1,\ldots,K\textrm{ y }\\
\esp\left[\eta_{k}\left(1\right)^{p+1}\right]&<&\infty\textrm{
para }k=1,\ldots,K.
\end{eqnarray*}
donde $\mathcal{A}$ es la clase de posibles arribos.

\item[A3)] Para cada $k=1,2,\ldots,K$ existe una funci\'on
positiva $q_{k}\left(\cdot\right)$ definida en $\rea_{+}$, y un
entero $j_{k}$, tal que
\begin{eqnarray}
P\left(\xi_{k}\left(1\right)\geq x\right)&>&0\textrm{, para todo }x>0,\\
P\left\{a\leq\sum_{i=1}^{j_{k}}\xi_{k}\left(i\right)\leq
b\right\}&\geq&\int_{a}^{b}q_{k}\left(x\right)dx, \textrm{ }0\leq
a<b.
\end{eqnarray}
\end{itemize}

En lo que respecta al supuesto (A3), en Dai y Meyn \cite{DaiSean}
hacen ver que este se puede sustituir por

\begin{itemize}
\item[A3')] Para el Proceso de Markov $X$, cada subconjunto
compacto del espacio de estados de $X$ es un conjunto peque\~no,
ver definici\'on \ref{Def.Cto.Peq.}.
\end{itemize}

Es por esta raz\'on que con la finalidad de poder hacer uso de
$A3^{'})$ es necesario recurrir a los Procesos de Harris y en
particular a los Procesos Harris Recurrente, ver \cite{Dai,
DaiSean}.
%_______________________________________________________________________
\subsection{Procesos Harris Recurrente}
%_______________________________________________________________________

Por el supuesto (A1) conforme a Davis \cite{Davis}, se puede
definir el proceso de saltos correspondiente de manera tal que
satisfaga el supuesto (A3'), de hecho la demostraci\'on est\'a
basada en la l\'inea de argumentaci\'on de Davis, \cite{Davis},
p\'aginas 362-364.\\

Entonces se tiene un espacio de estados en el cual el proceso $X$
satisface la Propiedad Fuerte de Markov, ver Dai y Meyn
\cite{DaiSean}, dado por

\[\left(\Omega,\mathcal{F},\mathcal{F}_{t},X\left(t\right),\theta_{t},P_{x}\right),\]
adem\'as de ser un proceso de Borel Derecho (Sharpe \cite{Sharpe})
en el espacio de estados medible
$\left(\mathbb{X},\mathcal{B}_\mathbb{X}\right)$. El Proceso
$X=\left\{X\left(t\right),t\geq0\right\}$ tiene trayectorias
continuas por la derecha, est\'a definido en
$\left(\Omega,\mathcal{F}\right)$ y est\'a adaptado a
$\left\{\mathcal{F}_{t},t\geq0\right\}$; la colecci\'on
$\left\{P_{x},x\in \mathbb{X}\right\}$ son medidas de probabilidad
en $\left(\Omega,\mathcal{F}\right)$ tales que para todo $x\in
\mathbb{X}$
\[P_{x}\left\{X\left(0\right)=x\right\}=1,\] y
\[E_{x}\left\{f\left(X\circ\theta_{t}\right)|\mathcal{F}_{t}\right\}=E_{X}\left(\tau\right)f\left(X\right),\]
en $\left\{\tau<\infty\right\}$, $P_{x}$-c.s., con $\theta_{t}$
definido como el operador shift.


Donde $\tau$ es un $\mathcal{F}_{t}$-tiempo de paro
\[\left(X\circ\theta_{\tau}\right)\left(w\right)=\left\{X\left(\tau\left(w\right)+t,w\right),t\geq0\right\},\]
y $f$ es una funci\'on de valores reales acotada y medible, ver \cite{Dai, KaspiMandelbaum}.\\

Sea $P^{t}\left(x,D\right)$, $D\in\mathcal{B}_{\mathbb{X}}$,
$t\geq0$ la probabilidad de transici\'on de $X$ queda definida
como:
\[P^{t}\left(x,D\right)=P_{x}\left(X\left(t\right)\in
D\right).\]


\begin{Def}
Una medida no cero $\pi$ en
$\left(\mathbb{X},\mathcal{B}_{\mathbb{X}}\right)$ es invariante
para $X$ si $\pi$ es $\sigma$-finita y
\[\pi\left(D\right)=\int_{\mathbb{X}}P^{t}\left(x,D\right)\pi\left(dx\right),\]
para todo $D\in \mathcal{B}_{\mathbb{X}}$, con $t\geq0$.
\end{Def}

\begin{Def}
El proceso de Markov $X$ es llamado Harris Recurrente si existe
una medida de probabilidad $\nu$ en
$\left(\mathbb{X},\mathcal{B}_{\mathbb{X}}\right)$, tal que si
$\nu\left(D\right)>0$ y $D\in\mathcal{B}_{\mathbb{X}}$
\[P_{x}\left\{\tau_{D}<\infty\right\}\equiv1,\] cuando
$\tau_{D}=inf\left\{t\geq0:X_{t}\in D\right\}$.
\end{Def}

\begin{Note}
\begin{itemize}
\item[i)] Si $X$ es Harris recurrente, entonces existe una \'unica
medida invariante $\pi$ (Getoor \cite{Getoor}).

\item[ii)] Si la medida invariante es finita, entonces puede
normalizarse a una medida de probabilidad, en este caso al proceso
$X$ se le llama Harris recurrente positivo.


\item[iii)] Cuando $X$ es Harris recurrente positivo se dice que
la disciplina de servicio es estable. En este caso $\pi$ denota la
distribuci\'on estacionaria y hacemos
\[P_{\pi}\left(\cdot\right)=\int_{\mathbf{X}}P_{x}\left(\cdot\right)\pi\left(dx\right),\]
y se utiliza $E_{\pi}$ para denotar el operador esperanza
correspondiente, ver \cite{DaiSean}.
\end{itemize}
\end{Note}

\begin{Def}\label{Def.Cto.Peq.}
Un conjunto $D\in\mathcal{B_{\mathbb{X}}}$ es llamado peque\~no si
existe un $t>0$, una medida de probabilidad $\nu$ en
$\mathcal{B_{\mathbb{X}}}$, y un $\delta>0$ tal que
\[P^{t}\left(x,A\right)\geq\delta\nu\left(A\right),\] para $x\in
D,A\in\mathcal{B_{\mathbb{X}}}$.
\end{Def}

La siguiente serie de resultados vienen enunciados y demostrados
en Dai \cite{Dai}:
\begin{Lema}[Lema 3.1, Dai \cite{Dai}]
Sea $B$ conjunto peque\~no cerrado, supongamos que
$P_{x}\left(\tau_{B}<\infty\right)\equiv1$ y que para alg\'un
$\delta>0$ se cumple que
\begin{equation}\label{Eq.3.1}
\sup\esp_{x}\left[\tau_{B}\left(\delta\right)\right]<\infty,
\end{equation}
donde
$\tau_{B}\left(\delta\right)=inf\left\{t\geq\delta:X\left(t\right)\in
B\right\}$. Entonces, $X$ es un proceso Harris recurrente
positivo.
\end{Lema}

\begin{Lema}[Lema 3.1, Dai \cite{Dai}]\label{Lema.3.}
Bajo el supuesto (A3), el conjunto
$B=\left\{x\in\mathbb{X}:|x|\leq k\right\}$ es un conjunto
peque\~no cerrado para cualquier $k>0$.
\end{Lema}

\begin{Teo}[Teorema 3.1, Dai \cite{Dai}]\label{Tma.3.1}
Si existe un $\delta>0$ tal que
\begin{equation}
lim_{|x|\rightarrow\infty}\frac{1}{|x|}\esp|X^{x}\left(|x|\delta\right)|=0,
\end{equation}
donde $X^{x}$ se utiliza para denotar que el proceso $X$ comienza
a partir de $x$, entonces la ecuaci\'on (\ref{Eq.3.1}) se cumple
para $B=\left\{x\in\mathbb{X}:|x|\leq k\right\}$ con alg\'un
$k>0$. En particular, $X$ es Harris recurrente positivo.
\end{Teo}

Entonces, tenemos que el proceso $X$ es un proceso de Markov que
cumple con los supuestos $A1)$-$A3)$, lo que falta de hacer es
construir el Modelo de Flujo bas\'andonos en lo hasta ahora
presentado.
%_______________________________________________________________________
\subsection{Modelo de Flujo}
%_______________________________________________________________________

Dada una condici\'on inicial $x\in\mathbb{X}$, sea

\begin{itemize}
\item $Q_{k}^{x}\left(t\right)$ la longitud de la cola al tiempo
$t$,

\item $T_{m,k}^{x}\left(t\right)$ el tiempo acumulado, al tiempo
$t$, que tarda el servidor $m$ en atender a los usuarios de la
cola $k$.

\item $T_{m,k}^{x,0}\left(t\right)$ el tiempo acumulado, al tiempo
$t$, que tarda el servidor $m$ en trasladarse a otra cola a partir de la $k$-\'esima.\\
\end{itemize}

Sup\'ongase que la funci\'on
$\left(\overline{Q}\left(\cdot\right),\overline{T}_{m}
\left(\cdot\right),\overline{T}_{m}^{0} \left(\cdot\right)\right)$
para $m=1,2,\ldots,M$ es un punto l\'imite de
\begin{equation}\label{Eq.Punto.Limite}
\left(\frac{1}{|x|}Q^{x}\left(|x|t\right),\frac{1}{|x|}T_{m}^{x}\left(|x|t\right),\frac{1}{|x|}T_{m}^{x,0}\left(|x|t\right)\right)
\end{equation}
para $m=1,2,\ldots,M$, cuando $x\rightarrow\infty$, ver
\cite{Down}. Entonces
$\left(\overline{Q}\left(t\right),\overline{T}_{m}
\left(t\right),\overline{T}_{m}^{0} \left(t\right)\right)$ es un
flujo l\'imite del sistema. Al conjunto de todos las posibles
flujos l\'imite se le llama {\emph{Modelo de Flujo}} y se le
denotar\'a por $\mathcal{Q}$, ver \cite{Down, Dai, DaiSean}.\\

El modelo de flujo satisface el siguiente conjunto de ecuaciones:

\begin{equation}\label{Eq.MF.1}
\overline{Q}_{k}\left(t\right)=\overline{Q}_{k}\left(0\right)+\lambda_{k}t-\sum_{m=1}^{M}\mu_{k}\overline{T}_{m,k}\left(t\right),\\
\end{equation}
para $k=1,2,\ldots,K$.\\
\begin{equation}\label{Eq.MF.2}
\overline{Q}_{k}\left(t\right)\geq0\textrm{ para
}k=1,2,\ldots,K.\\
\end{equation}

\begin{equation}\label{Eq.MF.3}
\overline{T}_{m,k}\left(0\right)=0,\textrm{ y }\overline{T}_{m,k}\left(\cdot\right)\textrm{ es no decreciente},\\
\end{equation}
para $k=1,2,\ldots,K$ y $m=1,2,\ldots,M$.\\
\begin{equation}\label{Eq.MF.4}
\sum_{k=1}^{K}\overline{T}_{m,k}^{0}\left(t\right)+\overline{T}_{m,k}\left(t\right)=t\textrm{
para }m=1,2,\ldots,M.\\
\end{equation}


\begin{Def}[Definici\'on 4.1, Dai \cite{Dai}]\label{Def.Modelo.Flujo}
Sea una disciplina de servicio espec\'ifica. Cualquier l\'imite
$\left(\overline{Q}\left(\cdot\right),\overline{T}\left(\cdot\right),\overline{T}^{0}\left(\cdot\right)\right)$
en (\ref{Eq.Punto.Limite}) es un {\em flujo l\'imite} de la
disciplina. Cualquier soluci\'on (\ref{Eq.MF.1})-(\ref{Eq.MF.4})
es llamado flujo soluci\'on de la disciplina.
\end{Def}

\begin{Def}
Se dice que el modelo de flujo l\'imite, modelo de flujo, de la
disciplina de la cola es estable si existe una constante
$\delta>0$ que depende de $\mu,\lambda$ y $P$ solamente, tal que
cualquier flujo l\'imite con
$|\overline{Q}\left(0\right)|+|\overline{U}|+|\overline{V}|=1$, se
tiene que $\overline{Q}\left(\cdot+\delta\right)\equiv0$.
\end{Def}

Si se hace $|x|\rightarrow\infty$ sin restringir ninguna de las
componentes, tambi\'en se obtienen un modelo de flujo, pero en
este caso el residual de los procesos de arribo y servicio
introducen un retraso:
\begin{Teo}[Teorema 4.2, Dai \cite{Dai}]\label{Tma.4.2.Dai}
Sea una disciplina fija para la cola, suponga que se cumplen las
condiciones (A1)-(A3). Si el modelo de flujo l\'imite de la
disciplina de la cola es estable, entonces la cadena de Markov $X$
que describe la din\'amica de la red bajo la disciplina es Harris
recurrente positiva.
\end{Teo}

Ahora se procede a escalar el espacio y el tiempo para reducir la
aparente fluctuaci\'on del modelo. Consid\'erese el proceso
\begin{equation}\label{Eq.3.7}
\overline{Q}^{x}\left(t\right)=\frac{1}{|x|}Q^{x}\left(|x|t\right).
\end{equation}
A este proceso se le conoce como el flujo escalado, y cualquier
l\'imite $\overline{Q}^{x}\left(t\right)$ es llamado flujo
l\'imite del proceso de longitud de la cola. Haciendo
$|q|\rightarrow\infty$ mientras se mantiene el resto de las
componentes fijas, cualquier punto l\'imite del proceso de
longitud de la cola normalizado $\overline{Q}^{x}$ es soluci\'on
del siguiente modelo de flujo.


\begin{Def}[Definici\'on 3.3, Dai y Meyn \cite{DaiSean}]
El modelo de flujo es estable si existe un tiempo fijo $t_{0}$ tal
que $\overline{Q}\left(t\right)=0$, con $t\geq t_{0}$, para
cualquier $\overline{Q}\left(\cdot\right)\in\mathcal{Q}$ que
cumple con $|\overline{Q}\left(0\right)|=1$.
\end{Def}

\begin{Lemma}[Lema 3.1, Dai y Meyn \cite{DaiSean}]
Si el modelo de flujo definido por (\ref{Eq.MF.1})-(\ref{Eq.MF.4})
es estable, entonces el modelo de flujo retrasado es tambi\'en
estable, es decir, existe $t_{0}>0$ tal que
$\overline{Q}\left(t\right)=0$ para cualquier $t\geq t_{0}$, para
cualquier soluci\'on del modelo de flujo retrasado cuya
condici\'on inicial $\overline{x}$ satisface que
$|\overline{x}|=|\overline{Q}\left(0\right)|+|\overline{A}\left(0\right)|+|\overline{B}\left(0\right)|\leq1$.
\end{Lemma}


Ahora ya estamos en condiciones de enunciar los resultados principales:


\begin{Teo}[Teorema 2.1, Down \cite{Down}]\label{Tma2.1.Down}
Suponga que el modelo de flujo es estable, y que se cumplen los supuestos (A1) y (A2), entonces
\begin{itemize}
\item[i)] Para alguna constante $\kappa_{p}$, y para cada
condici\'on inicial $x\in X$
\begin{equation}\label{Estability.Eq1}
\limsup_{t\rightarrow\infty}\frac{1}{t}\int_{0}^{t}\esp_{x}\left[|Q\left(s\right)|^{p}\right]ds\leq\kappa_{p},
\end{equation}
donde $p$ es el entero dado en (A2).
\end{itemize}
Si adem\'as se cumple la condici\'on (A3), entonces para cada
condici\'on inicial:
\begin{itemize}
\item[ii)] Los momentos transitorios convergen a su estado
estacionario:
 \begin{equation}\label{Estability.Eq2}
lim_{t\rightarrow\infty}\esp_{x}\left[Q_{k}\left(t\right)^{r}\right]=\esp_{\pi}\left[Q_{k}\left(0\right)^{r}\right]\leq\kappa_{r},
\end{equation}
para $r=1,2,\ldots,p$ y $k=1,2,\ldots,K$. Donde $\pi$ es la
probabilidad invariante para $X$.

\item[iii)]  El primer momento converge con raz\'on $t^{p-1}$:
\begin{equation}\label{Estability.Eq3}
lim_{t\rightarrow\infty}t^{p-1}|\esp_{x}\left[Q_{k}\left(t\right)\right]-\esp_{\pi}\left[Q_{k}\left(0\right)\right]|=0.
\end{equation}

\item[iv)] La {\em Ley Fuerte de los grandes n\'umeros} se cumple:
\begin{equation}\label{Estability.Eq4}
lim_{t\rightarrow\infty}\frac{1}{t}\int_{0}^{t}Q_{k}^{r}\left(s\right)ds=\esp_{\pi}\left[Q_{k}\left(0\right)^{r}\right],\textrm{
}\prob_{x}\textrm{-c.s.}
\end{equation}
para $r=1,2,\ldots,p$ y $k=1,2,\ldots,K$.
\end{itemize}
\end{Teo}

La contribuci\'on de Down a la teor\'ia de los {\emph {sistemas de
visitas c\'iclicas}}, es la relaci\'on que hay entre la
estabilidad del sistema con el comportamiento de las medidas de
desempe\~no, es decir, la condici\'on suficiente para poder
garantizar la convergencia del proceso de la longitud de la cola
as\'i como de por los menos los dos primeros momentos adem\'as de
una versi\'on de la Ley Fuerte de los Grandes N\'umeros para los
sistemas de visitas.


\begin{Teo}[Teorema 2.3, Down \cite{Down}]\label{Tma2.3.Down}
Considere el siguiente valor:
\begin{equation}\label{Eq.Rho.1serv}
\rho=\sum_{k=1}^{K}\rho_{k}+max_{1\leq j\leq K}\left(\frac{\lambda_{j}}{\sum_{s=1}^{S}p_{js}\overline{N}_{s}}\right)\delta^{*}
\end{equation}
\begin{itemize}
\item[i)] Si $\rho<1$ entonces la red es estable, es decir, se
cumple el Teorema \ref{Tma2.1.Down}.

\item[ii)] Si $\rho>1$ entonces la red es inestable, es decir, se
cumple el Teorema \ref{Tma2.2.Down}
\end{itemize}
\end{Teo}



%_________________________________________________________________________
\subsection{Modelo de Flujo}
%_________________________________________________________________________
Sup\'ongase que el sistema consta de varias colas a los cuales
llegan uno o varios servidores a dar servicio a los usuarios
esperando en la cola.\\


Sea $x$ el n\'umero de usuarios en la cola esperando por servicio
y $N\left(x\right)$ es el n\'umero de usuarios que son atendidos
con una pol\'itica dada y fija mientras el servidor permanece
dando servicio, entonces se asume que:
\begin{itemize}
\item[(S1.)]
\begin{equation}\label{S1}
lim_{x\rightarrow\infty}\esp\left[N\left(x\right)\right]=\overline{N}>0.
\end{equation}
\item[(S2.)]
\begin{equation}\label{S2}
\esp\left[N\left(x\right)\right]\leq \overline{N},
\end{equation}

para cualquier valor de $x$.
\end{itemize}

El tiempo que tarda un servidor en volver a dar servicio despu\'es
de abandonar la cola inmediata anterior y llegar a la pr\'oxima se
llama tiempo de traslado o de cambio  de cola, al momento de la
$n$-\'esima visita del servidor a la cola $j$ se genera una
sucesi\'on de variables aleatorias $\delta_{j,j+1}\left(n\right)$,
independientes e id\'enticamente distribuidas, con la propiedad de
que $\esp\left[\delta_{j,j+1}\left(1\right)\right]\geq0$.\\


Se define
\begin{equation}
\delta^{*}:=\sum_{j,j+1}\esp\left[\delta_{j,j+1}\left(1\right)\right].
\end{equation}
%\begin{figure}[H]
%\centering
%\includegraphics[width=7cm]{switchovertime.jpg}
%\caption{Sistema de Visitas C\'iclicas}
%\end{figure}

Los tiempos entre arribos a la cola $k$, son de la forma
$\left\{\xi_{k}\left(n\right)\right\}_{n\geq1}$, con la propiedad
de que son independientes e id\'enticamente distribuidos. Los
tiempos de servicio
$\left\{\eta_{k}\left(n\right)\right\}_{n\geq1}$ tienen la
propiedad de ser independientes e id\'enticamente distribuidos.
Para la $k$-\'esima cola se define la tasa de arribo a la como
$\lambda_{k}=1/\esp\left[\xi_{k}\left(1\right)\right]$ y la tasa
de servicio como
$\mu_{k}=1/\esp\left[\eta_{k}\left(1\right)\right]$, finalmente se
define la carga de la cola como $\rho_{k}=\lambda_{k}/\mu_{k}$,
donde se pide que $\rho<1$, para garantizar la estabilidad del sistema.\\

%_____________________________________________________________________
%\subsubsection{Proceso de Estados}
%_____________________________________________________________________

Para el caso m\'as sencillo podemos definir un proceso de estados
para la red que depende de la pol\'itica de servicio utilizada, el
estado $\mathbb{X}\left(t\right)$ a cualquier tiempo $t$ puede
definirse como
\begin{equation}\label{Eq.Esp.Estados}
\mathbb{X}\left(t\right)=\left(Q_{k}\left(t\right),A_{l}\left(t\right),B_{k}\left(t\right):k=1,2,\ldots,K,l\in\mathcal{A}\right),
\end{equation}

donde $Q_{k}\left(t\right)$ es la longitud de la cola $k$ para los
usuarios esperando servicio, incluyendo aquellos que est\'an
siendo atendidos, $B_{k}\left(t\right)$ son los tiempos de
servicio residuales para los usuarios de la clase $k$ que est\'an
en servicio.\\

Los tiempos entre arribos residuales, que son el tiempo que queda
hasta que el pr\'oximo usuario llega a la cola para recibir
servicio, se denotan por $A_{k}\left(t\right)$. Tanto
$B_{k}\left(t\right)$ como $A_{k}\left(t\right)$ se suponen
continuos por la derecha \cite{Dai2}.\\

Sea $\mathcal{X}$ el espacio de estados para el proceso de estados
que por definici\'on es igual  al conjunto de posibles valores
para el estado $\mathbb{X}\left(t\right)$, y sea
$x=\left(q,a,b\right)$ un estado gen\'erico en $\mathbb{X}$, la
componente $q$ determina la posici\'on del usuario en la red,
$|q|$ denota la longitud total de la cola en la red.\\

Para un estado $x=\left(q,a,b\right)\in\mathbb{X}$ definimos la
{\em norma} de $x$ como $\left\|x\right\|=|q|+|a|+|b|$. En
\cite{Dai} se muestra que para una amplia serie de disciplinas de
servicio el proceso $\mathbb{X}$ es un Proceso Fuerte de Markov, y
por tanto se puede asumir que
\[\left(\left(\Omega,\mathcal{F}\right),\mathcal{F}_{t},\mathbb{X}\left(t\right),\theta_{t},P_{x}\right)\]
es un proceso de {\em Borel Derecho} en el espacio de estados
medible $\left(\mathcal{X},\mathcal{B}_{\mathcal{X}}\right)$.\\

Sea $P^{t}\left(x,D\right)$, $D\in\mathcal{B}_{\mathbb{X}}$,
$t\geq0$ probabilidad de transici\'on de $X$ definida como
\[P^{t}\left(x,D\right)=P_{x}\left(\mathbb{X}\left(t\right)\in
D\right).\]

\begin{Def}
Una medida no cero $\pi$ en
$\left(\mathbb{X},\mathcal{B}_{\mathbb{X}}\right)$ es {\em
invariante} para $X$ si $\pi$ es $\sigma$-finita y
\[\pi\left(D\right)=\int_{X}P^{t}\left(x,D\right)\pi\left(dx\right),\]
para todo $D\in \mathcal{B}_{\mathbb{X}}$, con $t\geq0$.
\end{Def}

\begin{Def}
El proceso de Markov $X$ es llamado {\em Harris recurrente} si
existe una medida de probabilidad $\nu$ en
$\left(\mathbb{X},\mathcal{B}_{\mathbb{X}}\right)$, tal que si
$\nu\left(D\right)>0$ y $D\in\mathcal{B}_{\mathbb{X}}$
\[P_{x}\left\{\tau_{D}<\infty\right\}\equiv1,\] cuando
$\tau_{D}=inf\left\{t\geq0:\mathbb{X}_{t}\in D\right\}$.
\end{Def}

\begin{Def}
Un conjunto $D\in\mathcal{B}_\mathbb{X}$ es llamado peque\~no si
existe un $t>0$, una medida de probabilidad $\nu$ en
$\mathcal{B}_\mathbb{X}$, y un $\delta>0$ tal que
\[P^{t}\left(x,A\right)\geq\delta\nu\left(A\right),\] para $x\in
D,A\in\mathcal{B}_\mathbb{X}$.
\end{Def}
\begin{Note}
\begin{itemize}

\item[i)] Si $X$ es Harris recurrente, entonces existe una \'unica medida
invariante $\pi$ (\cite{Getoor}).

\item[ii)] Si la medida invariante es finita, entonces puede
normalizarse a una medida de probabilidad, en este caso a la
medida se le llama \textbf{Harris recurrente positiva}.

\item[iii)] Cuando $X$ es Harris recurrente positivo se dice que
la disciplina de servicio es estable. En este caso $\pi$ denota la
ditribuci\'on estacionaria; se define
\[P_{\pi}\left(\cdot\right)=\int_{X}P_{x}\left(\cdot\right)\pi\left(dx\right).\]
Se utiliza $E_{\pi}$ para denotar el operador esperanza
correspondiente, as\'i, el proceso
$X=\left\{\mathbb{X}\left(t\right),t\geq0\right\}$ es un proceso
estrictamente estacionario bajo $P_{\pi}$.

\item[iv)] En \cite{MeynTweedie} se muestra que si
$P_{x}\left\{\tau_{D}<\infty\right\}=1$ incluso para solamente un
conjunto peque\~no, entonces el proceso de Harris es recurrente.
\end{itemize}
\end{Note}


%_________________________________________________________________________
%\newpage
%_________________________________________________________________________
%\subsection{Modelo de Flujo}
%_____________________________________________________________________
Las Colas C\'iclicas se pueden describir por medio de un proceso
de Markov $\left(X\left(t\right)\right)_{t\in\rea}$, donde el
estado del sistema al tiempo $t\geq0$ est\'a dado por
\begin{equation}
X\left(t\right)=\left(Q\left(t\right),A\left(t\right),H\left(t\right),B\left(t\right),B^{0}\left(t\right),C\left(t\right)\right)
\end{equation}
definido en el espacio producto:
\begin{equation}
\mathcal{X}=\mathbb{Z}^{K}\times\rea_{+}^{K}\times\left(\left\{1,2,\ldots,K\right\}\times\left\{1,2,\ldots,S\right\}\right)^{M}\times\rea_{+}^{K}\times\rea_{+}^{K}\times\mathbb{Z}^{K},
\end{equation}

\begin{itemize}
\item $Q\left(t\right)=\left(Q_{k}\left(t\right),1\leq k\leq
K\right)$, es el n\'umero de usuarios en la cola $k$, incluyendo
aquellos que est\'an siendo atendidos provenientes de la
$k$-\'esima cola.

\item $A\left(t\right)=\left(A_{k}\left(t\right),1\leq k\leq
K\right)$, son los residuales de los tiempos de arribo en la cola
$k$. \item $H\left(t\right)$ es el par ordenado que consiste en la
cola que esta siendo atendida y la pol\'itica de servicio que se
utilizar\'a.

\item $B\left(t\right)$ es el tiempo de servicio residual.

\item $B^{0}\left(t\right)$ es el tiempo residual del cambio de
cola.

\item $C\left(t\right)$ indica el n\'umero de usuarios atendidos
durante la visita del servidor a la cola dada en
$H\left(t\right)$.
\end{itemize}

$A_{k}\left(t\right),B_{m}\left(t\right)$ y
$B_{m}^{0}\left(t\right)$ se suponen continuas por la derecha y
que satisfacen la propiedad fuerte de Markov, (\cite{Dai}).

Dada una condici\'on inicial $x\in\mathcal{X}$, $Q_{k}^{x}\left(t\right)$ es la longitud de la cola $k$ al tiempo $t$
y $T_{m,k}^{x}\left(t\right)$  el tiempo acumulado al tiempo $t$ que el servidor tarda en atender a los usuarios de la cola $k$.
De igual manera se define $T_{m,k}^{x,0}\left(t\right)$ el tiempo acumulado al tiempo $t$ que el servidor tarda en
cambiar de cola para volver a atender a los usuarios.

Para reducir la fluctuaci\'on del modelo se escala tanto el espacio como el tiempo, entonces se
tiene el proceso:

\begin{eqnarray}
\overline{Q}^{x}\left(t\right)=\frac{1}{|x|}Q^{x}\left(|x|t\right),\\
\overline{T}_{m}^{x}\left(t\right)=\frac{1}{|x|}T_{m}^{x}\left(|x|t\right),\\
\overline{T}_{m}^{x,0}\left(t\right)=\frac{1}{|x|}T_{m}^{x,0}\left(|x|t\right).
\end{eqnarray}
Cualquier l\'imite $\overline{Q}\left(t\right)$ es llamado un
flujo l\'imite del proceso longitud de la cola, al conjunto de todos los posibles flujos l\'imite
se le llamar\'a \textbf{modelo de flujo}, (\cite{MaynDown}).

\begin{Def}
Un flujo l\'imite para un sistema de visitas bajo una disciplina de
servicio espec\'ifica se define como cualquier soluci\'on
 $\left(\overline{Q}\left(\cdot\right),\overline{T}_{m}\left(\cdot\right),\overline{T}_{m}^{0}\left(\cdot\right)\right)$
 de las siguientes ecuaciones, donde
$\overline{Q}\left(t\right)=\left(\overline{Q}_{1}\left(t\right),\ldots,\overline{Q}_{K}\left(t\right)\right)$
y
$\overline{T}\left(t\right)=\left(\overline{T}_{1}\left(t\right),\ldots,\overline{T}_{K}\left(t\right)\right)$
\begin{equation}\label{Eq.3.8}
\overline{Q}_{k}\left(t\right)=\overline{Q}_{k}\left(0\right)+\lambda_{k}t-\sum_{m=1}^{M}\mu_{k}\overline{T}_{m,k}\left(t\right)\\
\end{equation}
\begin{equation}\label{Eq.3.9}
\overline{Q}_{k}\left(t\right)\geq0\textrm{ para }k=1,2,\ldots,K,\\
\end{equation}
\begin{equation}\label{Eq.3.10}
\overline{T}_{m,k}\left(0\right)=0,\textrm{ y }\overline{T}_{m,k}\left(\cdot\right)\textrm{ es no decreciente},\\
\end{equation}
\begin{equation}\label{Eq.3.11}
\sum_{k=1}^{K}\overline{T}_{m,k}^{0}\left(t\right)+\overline{T}_{m,k}\left(t\right)=t\textrm{ para}m=1,2,\ldots,M\\
\end{equation}
\end{Def}

Al conjunto de ecuaciones dadas en (\ref{Eq.3.8})-(\ref{Eq.3.11}) se
le llama {\em Modelo de flujo} y al conjunto de todas las
soluciones del modelo de flujo
$\left(\overline{Q}\left(\cdot\right),\overline{T}
\left(\cdot\right)\right)$ se le denotar\'a por $\mathcal{Q}$.


\begin{Def}
El modelo de flujo es estable si existe un tiempo fijo $t_{0}$ tal
que $\overline{Q}\left(t\right)=0$, con $t\geq t_{0}$, para
cualquier $\overline{Q}\left(\cdot\right)\in\mathcal{Q}$ que
cumple con $|\overline{Q}\left(0\right)|=1$.
\end{Def}


%_____________________________________________________________________________________
%
\subsection{Estabilidad de los Sistemas de Visitas C\'iclicas}
%_________________________________________________________________________

Es necesario realizar los siguientes supuestos, ver (\cite{Dai2}) y (\cite{DaiSean}):

\begin{itemize}
\item[A1)] $\xi_{1},\ldots,\xi_{K},\eta_{1},\ldots,\eta_{K}$ son
mutuamente independientes y son sucesiones independientes e
id\'enticamente distribuidas.

\item[A2)] Para alg\'un entero $p\geq1$
\begin{eqnarray*}
\esp\left[\xi_{k}\left(1\right)^{p+1}\right]&<&\infty\textrm{ para }l\in\mathcal{A}\textrm{ y }\\
\esp\left[\eta_{k}\left(1\right)^{p+1}\right]&<&\infty\textrm{ para
}k=1,\ldots,K.
\end{eqnarray*}
\item[A3)] El conjunto $\left\{x\in X:|x|=0\right\}$ es un
singleton, y para cada $k\in\mathcal{A}$, existe una funci\'on
positiva $q_{k}\left(x\right)$ definida en $\rea_{+}$, y un entero
$j_{k}$, tal que
\begin{eqnarray}
P\left(\xi_{k}\left(1\right)\geq x\right)&>&0\textrm{, para todo }x>0\\
P\left(\xi_{k}\left(1\right)+\ldots\xi_{k}\left(j_{k}\right)\in dx\right)&\geq& q_{k}\left(x\right)dx0\textrm{ y }\\
\int_{0}^{\infty}q_{k}\left(x\right)dx>0
\end{eqnarray}
\end{itemize}


En \cite{MaynDown} ser da un argumento para deducir que todos los
subconjuntos compactos de $X$ son peque\~nos. Entonces la
condici\'on A3) se puede generalizar a
\begin{itemize}
\item[A3')] Para el proceso de Markov $X$, cada subconjunto
compacto de $X$ es peque\~no.
\end{itemize}

\begin{Teo}\label{Tma2.1}
Suponga que el modelo de flujo para una disciplina de servicio es
estable, y suponga adem\'as que las condiciones A1) y A2) se
satisfacen. Entonces:
\begin{itemize}
\item[i)] Para alguna constante $\kappa_{p}$, y para cada
condici\'on inicial $x\in X$
\begin{equation}
\limsup_{t\rightarrow\infty}\frac{1}{t}\int_{0}^{t}\esp_{x}\left[|Q\left(t\right)|^{p}\right]ds\leq\kappa_{p}
\end{equation}
donde $p$ es el entero dado por A2).
\end{itemize}

Suponga adem\'as que A3) o A3')
se cumple, entonces la disciplina de servicio es estable y adem\'as
para cada condici\'on inicial se tiene lo siguiente:
\begin{itemize}

\item[ii)] Los momentos transitorios convergen a sus valores en estado
estacionario:
\begin{equation}
\lim_{t\rightarrow\infty}\esp_{x}\left[Q_{k}\left(t\right)^{r}\right]=\esp_{\pi}\left[Q_{k}\left(0\right)^{r}\right]\leq\kappa_{r}
\end{equation}
para $r=1,\ldots,p$ y $k=1,\ldots,K$. \item[iii)] El primer
momento converge con raz\'on $t^{p-1}$:
\begin{equation}
\lim_{t\rightarrow\infty}t^{p-1}|\esp_{x}\left[Q\left(t\right)\right]-\esp_{\pi}\left[Q\left(0\right)\right]|=0.
\end{equation}
\item[iv)] Se cumple la Ley Fuerte de los Grandes N\'umeros:
\begin{equation}
\lim_{t\rightarrow\infty}\frac{1}{t}\int_{0}^{t}Q_{k}^{r}\left(s\right)ds=\esp_{\pi}\left[Q_{k}\left(0\right)^{r}\right]
\end{equation}
$\prob$-c.s., para $r=1,\ldots,p$ y $k=1,\ldots,K$.
\end{itemize}
\end{Teo}


\begin{Teo}\label{Tma2.2}
Suponga que el fluido modelo es inestable en el sentido de que
para alguna $\epsilon_{0},c_{0}\geq0$,
\begin{equation}\label{Eq.Inestability}
|Q\left(T\right)|\geq\epsilon_{0}T-c_{0}\textrm{, con }T\geq0,
\end{equation}
para cualquier condici\'on inicial $Q\left(0\right)$, con
$|Q\left(0\right)|=1$. Entonces para cualquier $0<q\leq1$, existe
$B<0$ tal que para cualquier $|x|\geq B$,
\begin{equation}
\prob_{x}\left\{\mathbb{X}\rightarrow\infty\right\}\geq q.
\end{equation}
\end{Teo}

%_____________________________________________________________________________________
%

%_____________________________________________________________________
\subsection{Resultados principales}
%_____________________________________________________________________
En el caso particular de un modelo con un solo servidor, $M=1$, se
tiene que si se define
\begin{Def}\label{Def.Ro}
\begin{equation}\label{RoM1}
\rho=\sum_{k=1}^{K}\rho_{k}+\max_{1\leq j\leq
K}\left(\frac{\lambda_{j}}{\overline{N}}\right)\delta^{*}.
\end{equation}
\end{Def}
entonces

\begin{Teo}\label{Teo.Down}
\begin{itemize}
\item[i)] Si $\rho<1$, entonces la red es estable, es decir el teorema
(\ref{Tma2.1}) se cumple. \item[ii)] De lo contrario, es decir, si
$\rho>1$ entonces la red es inestable, es decir, el teorema
(\ref{Tma2.2}).
\end{itemize}
\end{Teo}



%_________________________________________________________________________
\subsection{Supuestos}
%_________________________________________________________________________
Consideremos el caso en el que se tienen varias colas a las cuales
llegan uno o varios servidores para dar servicio a los usuarios
que se encuentran presentes en la cola, como ya se mencion\'o hay
varios tipos de pol\'iticas de servicio, incluso podr\'ia ocurrir
que la manera en que atiende al resto de las colas sea distinta a
como lo hizo en las anteriores.\\

Para ejemplificar los sistemas de visitas c\'iclicas se
considerar\'a el caso en que a las colas los usuarios son atendidos con
una s\'ola pol\'itica de servicio.\\


Si $\omega$ es el n\'umero de usuarios en la cola al comienzo del
periodo de servicio y $N\left(\omega\right)$ es el n\'umero de
usuarios que son atendidos con una pol\'itica en espec\'ifico
durante el periodo de servicio, entonces se asume que:
\begin{itemize}
\item[1)]\label{S1}$lim_{\omega\rightarrow\infty}\esp\left[N\left(\omega\right)\right]=\overline{N}>0$;
\item[2)]\label{S2}$\esp\left[N\left(\omega\right)\right]\leq\overline{N}$
para cualquier valor de $\omega$.
\end{itemize}
La manera en que atiende el servidor $m$-\'esimo, es la siguiente:
\begin{itemize}
\item Al t\'ermino de la visita a la cola $j$, el servidor cambia
a la cola $j^{'}$ con probabilidad $r_{j,j^{'}}^{m}$

\item La $n$-\'esima vez que el servidor cambia de la cola $j$ a
$j'$, va acompa\~nada con el tiempo de cambio de longitud
$\delta_{j,j^{'}}^{m}\left(n\right)$, con
$\delta_{j,j^{'}}^{m}\left(n\right)$, $n\geq1$, variables
aleatorias independientes e id\'enticamente distribuidas, tales
que $\esp\left[\delta_{j,j^{'}}^{m}\left(1\right)\right]\geq0$.

\item Sea $\left\{p_{j}^{m}\right\}$ la distribuci\'on invariante
estacionaria \'unica para la Cadena de Markov con matriz de
transici\'on $\left(r_{j,j^{'}}^{m}\right)$, se supone que \'esta
existe.

\item Finalmente, se define el tiempo promedio total de traslado
entre las colas como
\begin{equation}
\delta^{*}:=\sum_{j,j^{'}}p_{j}^{m}r_{j,j^{'}}^{m}\esp\left[\delta_{j,j^{'}}^{m}\left(i\right)\right].
\end{equation}
\end{itemize}

Consideremos el caso donde los tiempos entre arribo a cada una de
las colas, $\left\{\xi_{k}\left(n\right)\right\}_{n\geq1}$ son
variables aleatorias independientes a id\'enticamente
distribuidas, y los tiempos de servicio en cada una de las colas
se distribuyen de manera independiente e id\'enticamente
distribuidas $\left\{\eta_{k}\left(n\right)\right\}_{n\geq1}$;
adem\'as ambos procesos cumplen la condici\'on de ser
independientes entre s\'i. Para la $k$-\'esima cola se define la
tasa de arribo por
$\lambda_{k}=1/\esp\left[\xi_{k}\left(1\right)\right]$ y la tasa
de servicio como
$\mu_{k}=1/\esp\left[\eta_{k}\left(1\right)\right]$, finalmente se
define la carga de la cola como $\rho_{k}=\lambda_{k}/\mu_{k}$,
donde se pide que $\rho=\sum_{k=1}^{K}\rho_{k}<1$, para garantizar
la estabilidad del sistema, esto es cierto para las pol\'iticas de
servicio exhaustiva y cerrada, ver Geetor \cite{Getoor}.\\

Si denotamos por
\begin{itemize}
\item $Q_{k}\left(t\right)$ el n\'umero de usuarios presentes en
la cola $k$ al tiempo $t$; \item $A_{k}\left(t\right)$ los
residuales de los tiempos entre arribos a la cola $k$; para cada
servidor $m$; \item $B_{m}\left(t\right)$ denota a los residuales
de los tiempos de servicio al tiempo $t$; \item
$B_{m}^{0}\left(t\right)$ los residuales de los tiempos de
traslado de la cola $k$ a la pr\'oxima por atender al tiempo $t$,

\item sea
$C_{m}\left(t\right)$ el n\'umero de usuarios atendidos durante la
visita del servidor a la cola $k$ al tiempo $t$.
\end{itemize}


En este sentido, el proceso para el sistema de visitas se puede
definir como:

\begin{equation}\label{Esp.Edos.Down}
X\left(t\right)^{T}=\left(Q_{k}\left(t\right),A_{k}\left(t\right),B_{m}\left(t\right),B_{m}^{0}\left(t\right),C_{m}\left(t\right)\right),
\end{equation}
para $k=1,\ldots,K$ y $m=1,2,\ldots,M$, donde $T$ indica que es el
transpuesto del vector que se est\'a definiendo. El proceso $X$
evoluciona en el espacio de estados:
$\mathbb{X}=\ent_{+}^{K}\times\rea_{+}^{K}\times\left(\left\{1,2,\ldots,K\right\}\times\left\{1,2,\ldots,S\right\}\right)^{M}\times\rea_{+}^{K}\times\ent_{+}^{K}$.\\

El sistema aqu\'i descrito debe de cumplir con los siguientes supuestos b\'asicos de un sistema de visitas:
%__________________________________________________________________________
\subsubsection{Supuestos B\'asicos}
%__________________________________________________________________________
\begin{itemize}
\item[A1)] Los procesos
$\xi_{1},\ldots,\xi_{K},\eta_{1},\ldots,\eta_{K}$ son mutuamente
independientes y son sucesiones independientes e id\'enticamente
distribuidas.

\item[A2)] Para alg\'un entero $p\geq1$
\begin{eqnarray*}
\esp\left[\xi_{l}\left(1\right)^{p+1}\right]&<&\infty\textrm{ para }l=1,\ldots,K\textrm{ y }\\
\esp\left[\eta_{k}\left(1\right)^{p+1}\right]&<&\infty\textrm{
para }k=1,\ldots,K.
\end{eqnarray*}
donde $\mathcal{A}$ es la clase de posibles arribos.

\item[A3)] Para cada $k=1,2,\ldots,K$ existe una funci\'on
positiva $q_{k}\left(\cdot\right)$ definida en $\rea_{+}$, y un
entero $j_{k}$, tal que
\begin{eqnarray}
P\left(\xi_{k}\left(1\right)\geq x\right)&>&0\textrm{, para todo }x>0,\\
P\left\{a\leq\sum_{i=1}^{j_{k}}\xi_{k}\left(i\right)\leq
b\right\}&\geq&\int_{a}^{b}q_{k}\left(x\right)dx, \textrm{ }0\leq
a<b.
\end{eqnarray}
\end{itemize}

En lo que respecta al supuesto (A3), en Dai y Meyn \cite{DaiSean}
hacen ver que este se puede sustituir por

\begin{itemize}
\item[A3')] Para el Proceso de Markov $X$, cada subconjunto
compacto del espacio de estados de $X$ es un conjunto peque\~no,
ver definici\'on \ref{Def.Cto.Peq.}.
\end{itemize}

Es por esta raz\'on que con la finalidad de poder hacer uso de
$A3^{'})$ es necesario recurrir a los Procesos de Harris y en
particular a los Procesos Harris Recurrente, ver \cite{Dai,
DaiSean}.
%_______________________________________________________________________
\subsection{Procesos Harris Recurrente}
%_______________________________________________________________________

Por el supuesto (A1) conforme a Davis \cite{Davis}, se puede
definir el proceso de saltos correspondiente de manera tal que
satisfaga el supuesto (A3'), de hecho la demostraci\'on est\'a
basada en la l\'inea de argumentaci\'on de Davis, \cite{Davis},
p\'aginas 362-364.\\

Entonces se tiene un espacio de estados en el cual el proceso $X$
satisface la Propiedad Fuerte de Markov, ver Dai y Meyn
\cite{DaiSean}, dado por

\[\left(\Omega,\mathcal{F},\mathcal{F}_{t},X\left(t\right),\theta_{t},P_{x}\right),\]
adem\'as de ser un proceso de Borel Derecho (Sharpe \cite{Sharpe})
en el espacio de estados medible
$\left(\mathbb{X},\mathcal{B}_\mathbb{X}\right)$. El Proceso
$X=\left\{X\left(t\right),t\geq0\right\}$ tiene trayectorias
continuas por la derecha, est\'a definido en
$\left(\Omega,\mathcal{F}\right)$ y est\'a adaptado a
$\left\{\mathcal{F}_{t},t\geq0\right\}$; la colecci\'on
$\left\{P_{x},x\in \mathbb{X}\right\}$ son medidas de probabilidad
en $\left(\Omega,\mathcal{F}\right)$ tales que para todo $x\in
\mathbb{X}$
\[P_{x}\left\{X\left(0\right)=x\right\}=1,\] y
\[E_{x}\left\{f\left(X\circ\theta_{t}\right)|\mathcal{F}_{t}\right\}=E_{X}\left(\tau\right)f\left(X\right),\]
en $\left\{\tau<\infty\right\}$, $P_{x}$-c.s., con $\theta_{t}$
definido como el operador shift.


Donde $\tau$ es un $\mathcal{F}_{t}$-tiempo de paro
\[\left(X\circ\theta_{\tau}\right)\left(w\right)=\left\{X\left(\tau\left(w\right)+t,w\right),t\geq0\right\},\]
y $f$ es una funci\'on de valores reales acotada y medible, ver \cite{Dai, KaspiMandelbaum}.\\

Sea $P^{t}\left(x,D\right)$, $D\in\mathcal{B}_{\mathbb{X}}$,
$t\geq0$ la probabilidad de transici\'on de $X$ queda definida
como:
\[P^{t}\left(x,D\right)=P_{x}\left(X\left(t\right)\in
D\right).\]


\begin{Def}
Una medida no cero $\pi$ en
$\left(\mathbb{X},\mathcal{B}_{\mathbb{X}}\right)$ es invariante
para $X$ si $\pi$ es $\sigma$-finita y
\[\pi\left(D\right)=\int_{\mathbb{X}}P^{t}\left(x,D\right)\pi\left(dx\right),\]
para todo $D\in \mathcal{B}_{\mathbb{X}}$, con $t\geq0$.
\end{Def}

\begin{Def}
El proceso de Markov $X$ es llamado Harris Recurrente si existe
una medida de probabilidad $\nu$ en
$\left(\mathbb{X},\mathcal{B}_{\mathbb{X}}\right)$, tal que si
$\nu\left(D\right)>0$ y $D\in\mathcal{B}_{\mathbb{X}}$
\[P_{x}\left\{\tau_{D}<\infty\right\}\equiv1,\] cuando
$\tau_{D}=inf\left\{t\geq0:X_{t}\in D\right\}$.
\end{Def}

\begin{Note}
\begin{itemize}
\item[i)] Si $X$ es Harris recurrente, entonces existe una \'unica
medida invariante $\pi$ (Getoor \cite{Getoor}).

\item[ii)] Si la medida invariante es finita, entonces puede
normalizarse a una medida de probabilidad, en este caso al proceso
$X$ se le llama Harris recurrente positivo.


\item[iii)] Cuando $X$ es Harris recurrente positivo se dice que
la disciplina de servicio es estable. En este caso $\pi$ denota la
distribuci\'on estacionaria y hacemos
\[P_{\pi}\left(\cdot\right)=\int_{\mathbf{X}}P_{x}\left(\cdot\right)\pi\left(dx\right),\]
y se utiliza $E_{\pi}$ para denotar el operador esperanza
correspondiente, ver \cite{DaiSean}.
\end{itemize}
\end{Note}

\begin{Def}\label{Def.Cto.Peq.}
Un conjunto $D\in\mathcal{B_{\mathbb{X}}}$ es llamado peque\~no si
existe un $t>0$, una medida de probabilidad $\nu$ en
$\mathcal{B_{\mathbb{X}}}$, y un $\delta>0$ tal que
\[P^{t}\left(x,A\right)\geq\delta\nu\left(A\right),\] para $x\in
D,A\in\mathcal{B_{\mathbb{X}}}$.
\end{Def}

La siguiente serie de resultados vienen enunciados y demostrados
en Dai \cite{Dai}:
\begin{Lema}[Lema 3.1, Dai \cite{Dai}]
Sea $B$ conjunto peque\~no cerrado, supongamos que
$P_{x}\left(\tau_{B}<\infty\right)\equiv1$ y que para alg\'un
$\delta>0$ se cumple que
\begin{equation}\label{Eq.3.1}
\sup\esp_{x}\left[\tau_{B}\left(\delta\right)\right]<\infty,
\end{equation}
donde
$\tau_{B}\left(\delta\right)=inf\left\{t\geq\delta:X\left(t\right)\in
B\right\}$. Entonces, $X$ es un proceso Harris recurrente
positivo.
\end{Lema}

\begin{Lema}[Lema 3.1, Dai \cite{Dai}]\label{Lema.3.}
Bajo el supuesto (A3), el conjunto
$B=\left\{x\in\mathbb{X}:|x|\leq k\right\}$ es un conjunto
peque\~no cerrado para cualquier $k>0$.
\end{Lema}

\begin{Teo}[Teorema 3.1, Dai \cite{Dai}]\label{Tma.3.1}
Si existe un $\delta>0$ tal que
\begin{equation}
lim_{|x|\rightarrow\infty}\frac{1}{|x|}\esp|X^{x}\left(|x|\delta\right)|=0,
\end{equation}
donde $X^{x}$ se utiliza para denotar que el proceso $X$ comienza
a partir de $x$, entonces la ecuaci\'on (\ref{Eq.3.1}) se cumple
para $B=\left\{x\in\mathbb{X}:|x|\leq k\right\}$ con alg\'un
$k>0$. En particular, $X$ es Harris recurrente positivo.
\end{Teo}

Entonces, tenemos que el proceso $X$ es un proceso de Markov que
cumple con los supuestos $A1)$-$A3)$, lo que falta de hacer es
construir el Modelo de Flujo bas\'andonos en lo hasta ahora
presentado.
%_______________________________________________________________________
\subsection{Modelo de Flujo}
%_______________________________________________________________________

Dada una condici\'on inicial $x\in\mathbb{X}$, sea

\begin{itemize}
\item $Q_{k}^{x}\left(t\right)$ la longitud de la cola al tiempo
$t$,

\item $T_{m,k}^{x}\left(t\right)$ el tiempo acumulado, al tiempo
$t$, que tarda el servidor $m$ en atender a los usuarios de la
cola $k$.

\item $T_{m,k}^{x,0}\left(t\right)$ el tiempo acumulado, al tiempo
$t$, que tarda el servidor $m$ en trasladarse a otra cola a partir de la $k$-\'esima.\\
\end{itemize}

Sup\'ongase que la funci\'on
$\left(\overline{Q}\left(\cdot\right),\overline{T}_{m}
\left(\cdot\right),\overline{T}_{m}^{0} \left(\cdot\right)\right)$
para $m=1,2,\ldots,M$ es un punto l\'imite de
\begin{equation}\label{Eq.Punto.Limite}
\left(\frac{1}{|x|}Q^{x}\left(|x|t\right),\frac{1}{|x|}T_{m}^{x}\left(|x|t\right),\frac{1}{|x|}T_{m}^{x,0}\left(|x|t\right)\right)
\end{equation}
para $m=1,2,\ldots,M$, cuando $x\rightarrow\infty$, ver
\cite{Down}. Entonces
$\left(\overline{Q}\left(t\right),\overline{T}_{m}
\left(t\right),\overline{T}_{m}^{0} \left(t\right)\right)$ es un
flujo l\'imite del sistema. Al conjunto de todos las posibles
flujos l\'imite se le llama {\emph{Modelo de Flujo}} y se le
denotar\'a por $\mathcal{Q}$, ver \cite{Down, Dai, DaiSean}.\\

El modelo de flujo satisface el siguiente conjunto de ecuaciones:

\begin{equation}\label{Eq.MF.1}
\overline{Q}_{k}\left(t\right)=\overline{Q}_{k}\left(0\right)+\lambda_{k}t-\sum_{m=1}^{M}\mu_{k}\overline{T}_{m,k}\left(t\right),\\
\end{equation}
para $k=1,2,\ldots,K$.\\
\begin{equation}\label{Eq.MF.2}
\overline{Q}_{k}\left(t\right)\geq0\textrm{ para
}k=1,2,\ldots,K.\\
\end{equation}

\begin{equation}\label{Eq.MF.3}
\overline{T}_{m,k}\left(0\right)=0,\textrm{ y }\overline{T}_{m,k}\left(\cdot\right)\textrm{ es no decreciente},\\
\end{equation}
para $k=1,2,\ldots,K$ y $m=1,2,\ldots,M$.\\
\begin{equation}\label{Eq.MF.4}
\sum_{k=1}^{K}\overline{T}_{m,k}^{0}\left(t\right)+\overline{T}_{m,k}\left(t\right)=t\textrm{
para }m=1,2,\ldots,M.\\
\end{equation}


\begin{Def}[Definici\'on 4.1, Dai \cite{Dai}]\label{Def.Modelo.Flujo}
Sea una disciplina de servicio espec\'ifica. Cualquier l\'imite
$\left(\overline{Q}\left(\cdot\right),\overline{T}\left(\cdot\right),\overline{T}^{0}\left(\cdot\right)\right)$
en (\ref{Eq.Punto.Limite}) es un {\em flujo l\'imite} de la
disciplina. Cualquier soluci\'on (\ref{Eq.MF.1})-(\ref{Eq.MF.4})
es llamado flujo soluci\'on de la disciplina.
\end{Def}

\begin{Def}
Se dice que el modelo de flujo l\'imite, modelo de flujo, de la
disciplina de la cola es estable si existe una constante
$\delta>0$ que depende de $\mu,\lambda$ y $P$ solamente, tal que
cualquier flujo l\'imite con
$|\overline{Q}\left(0\right)|+|\overline{U}|+|\overline{V}|=1$, se
tiene que $\overline{Q}\left(\cdot+\delta\right)\equiv0$.
\end{Def}

Si se hace $|x|\rightarrow\infty$ sin restringir ninguna de las
componentes, tambi\'en se obtienen un modelo de flujo, pero en
este caso el residual de los procesos de arribo y servicio
introducen un retraso:
\begin{Teo}[Teorema 4.2, Dai \cite{Dai}]\label{Tma.4.2.Dai}
Sea una disciplina fija para la cola, suponga que se cumplen las
condiciones (A1)-(A3). Si el modelo de flujo l\'imite de la
disciplina de la cola es estable, entonces la cadena de Markov $X$
que describe la din\'amica de la red bajo la disciplina es Harris
recurrente positiva.
\end{Teo}

Ahora se procede a escalar el espacio y el tiempo para reducir la
aparente fluctuaci\'on del modelo. Consid\'erese el proceso
\begin{equation}\label{Eq.3.7}
\overline{Q}^{x}\left(t\right)=\frac{1}{|x|}Q^{x}\left(|x|t\right).
\end{equation}
A este proceso se le conoce como el flujo escalado, y cualquier
l\'imite $\overline{Q}^{x}\left(t\right)$ es llamado flujo
l\'imite del proceso de longitud de la cola. Haciendo
$|q|\rightarrow\infty$ mientras se mantiene el resto de las
componentes fijas, cualquier punto l\'imite del proceso de
longitud de la cola normalizado $\overline{Q}^{x}$ es soluci\'on
del siguiente modelo de flujo.


\begin{Def}[Definici\'on 3.3, Dai y Meyn \cite{DaiSean}]
El modelo de flujo es estable si existe un tiempo fijo $t_{0}$ tal
que $\overline{Q}\left(t\right)=0$, con $t\geq t_{0}$, para
cualquier $\overline{Q}\left(\cdot\right)\in\mathcal{Q}$ que
cumple con $|\overline{Q}\left(0\right)|=1$.
\end{Def}

\begin{Lemma}[Lema 3.1, Dai y Meyn \cite{DaiSean}]
Si el modelo de flujo definido por (\ref{Eq.MF.1})-(\ref{Eq.MF.4})
es estable, entonces el modelo de flujo retrasado es tambi\'en
estable, es decir, existe $t_{0}>0$ tal que
$\overline{Q}\left(t\right)=0$ para cualquier $t\geq t_{0}$, para
cualquier soluci\'on del modelo de flujo retrasado cuya
condici\'on inicial $\overline{x}$ satisface que
$|\overline{x}|=|\overline{Q}\left(0\right)|+|\overline{A}\left(0\right)|+|\overline{B}\left(0\right)|\leq1$.
\end{Lemma}


Ahora ya estamos en condiciones de enunciar los resultados principales:


\begin{Teo}[Teorema 2.1, Down \cite{Down}]\label{Tma2.1.Down}
Suponga que el modelo de flujo es estable, y que se cumplen los supuestos (A1) y (A2), entonces
\begin{itemize}
\item[i)] Para alguna constante $\kappa_{p}$, y para cada
condici\'on inicial $x\in X$
\begin{equation}\label{Estability.Eq1}
\limsup_{t\rightarrow\infty}\frac{1}{t}\int_{0}^{t}\esp_{x}\left[|Q\left(s\right)|^{p}\right]ds\leq\kappa_{p},
\end{equation}
donde $p$ es el entero dado en (A2).
\end{itemize}
Si adem\'as se cumple la condici\'on (A3), entonces para cada
condici\'on inicial:
\begin{itemize}
\item[ii)] Los momentos transitorios convergen a su estado
estacionario:
 \begin{equation}\label{Estability.Eq2}
lim_{t\rightarrow\infty}\esp_{x}\left[Q_{k}\left(t\right)^{r}\right]=\esp_{\pi}\left[Q_{k}\left(0\right)^{r}\right]\leq\kappa_{r},
\end{equation}
para $r=1,2,\ldots,p$ y $k=1,2,\ldots,K$. Donde $\pi$ es la
probabilidad invariante para $X$.

\item[iii)]  El primer momento converge con raz\'on $t^{p-1}$:
\begin{equation}\label{Estability.Eq3}
lim_{t\rightarrow\infty}t^{p-1}|\esp_{x}\left[Q_{k}\left(t\right)\right]-\esp_{\pi}\left[Q_{k}\left(0\right)\right]|=0.
\end{equation}

\item[iv)] La {\em Ley Fuerte de los grandes n\'umeros} se cumple:
\begin{equation}\label{Estability.Eq4}
lim_{t\rightarrow\infty}\frac{1}{t}\int_{0}^{t}Q_{k}^{r}\left(s\right)ds=\esp_{\pi}\left[Q_{k}\left(0\right)^{r}\right],\textrm{
}\prob_{x}\textrm{-c.s.}
\end{equation}
para $r=1,2,\ldots,p$ y $k=1,2,\ldots,K$.
\end{itemize}
\end{Teo}

La contribuci\'on de Down a la teor\'ia de los {\emph {sistemas de
visitas c\'iclicas}}, es la relaci\'on que hay entre la
estabilidad del sistema con el comportamiento de las medidas de
desempe\~no, es decir, la condici\'on suficiente para poder
garantizar la convergencia del proceso de la longitud de la cola
as\'i como de por los menos los dos primeros momentos adem\'as de
una versi\'on de la Ley Fuerte de los Grandes N\'umeros para los
sistemas de visitas.


\begin{Teo}[Teorema 2.3, Down \cite{Down}]\label{Tma2.3.Down}
Considere el siguiente valor:
\begin{equation}\label{Eq.Rho.1serv}
\rho=\sum_{k=1}^{K}\rho_{k}+max_{1\leq j\leq K}\left(\frac{\lambda_{j}}{\sum_{s=1}^{S}p_{js}\overline{N}_{s}}\right)\delta^{*}
\end{equation}
\begin{itemize}
\item[i)] Si $\rho<1$ entonces la red es estable, es decir, se
cumple el Teorema \ref{Tma2.1.Down}.

\item[ii)] Si $\rho>1$ entonces la red es inestable, es decir, se
cumple el Teorema \ref{Tma2.2.Down}
\end{itemize}
\end{Teo}

%_________________________________________________________________________
%\section{DESARROLLO DEL TEMA Y/O METODOLOG\'IA}
%_________________________________________________________________________
\subsection{Supuestos}
%_________________________________________________________________________
Consideremos el caso en el que se tienen varias colas a las cuales
llegan uno o varios servidores para dar servicio a los usuarios
que se encuentran presentes en la cola, como ya se mencion\'o hay
varios tipos de pol\'iticas de servicio, incluso podr\'ia ocurrir
que la manera en que atiende al resto de las colas sea distinta a
como lo hizo en las anteriores.\\

Para ejemplificar los sistemas de visitas c\'iclicas se
considerar\'a el caso en que a las colas los usuarios son atendidos con
una s\'ola pol\'itica de servicio.\\


Si $\omega$ es el n\'umero de usuarios en la cola al comienzo del
periodo de servicio y $N\left(\omega\right)$ es el n\'umero de
usuarios que son atendidos con una pol\'itica en espec\'ifico
durante el periodo de servicio, entonces se asume que:
\begin{itemize}
\item[1)]\label{S1}$lim_{\omega\rightarrow\infty}\esp\left[N\left(\omega\right)\right]=\overline{N}>0$;
\item[2)]\label{S2}$\esp\left[N\left(\omega\right)\right]\leq\overline{N}$
para cualquier valor de $\omega$.
\end{itemize}
La manera en que atiende el servidor $m$-\'esimo, es la siguiente:
\begin{itemize}
\item Al t\'ermino de la visita a la cola $j$, el servidor cambia
a la cola $j^{'}$ con probabilidad $r_{j,j^{'}}^{m}$

\item La $n$-\'esima vez que el servidor cambia de la cola $j$ a
$j'$, va acompa\~nada con el tiempo de cambio de longitud
$\delta_{j,j^{'}}^{m}\left(n\right)$, con
$\delta_{j,j^{'}}^{m}\left(n\right)$, $n\geq1$, variables
aleatorias independientes e id\'enticamente distribuidas, tales
que $\esp\left[\delta_{j,j^{'}}^{m}\left(1\right)\right]\geq0$.

\item Sea $\left\{p_{j}^{m}\right\}$ la distribuci\'on invariante
estacionaria \'unica para la Cadena de Markov con matriz de
transici\'on $\left(r_{j,j^{'}}^{m}\right)$, se supone que \'esta
existe.

\item Finalmente, se define el tiempo promedio total de traslado
entre las colas como
\begin{equation}
\delta^{*}:=\sum_{j,j^{'}}p_{j}^{m}r_{j,j^{'}}^{m}\esp\left[\delta_{j,j^{'}}^{m}\left(i\right)\right].
\end{equation}
\end{itemize}

Consideremos el caso donde los tiempos entre arribo a cada una de
las colas, $\left\{\xi_{k}\left(n\right)\right\}_{n\geq1}$ son
variables aleatorias independientes a id\'enticamente
distribuidas, y los tiempos de servicio en cada una de las colas
se distribuyen de manera independiente e id\'enticamente
distribuidas $\left\{\eta_{k}\left(n\right)\right\}_{n\geq1}$;
adem\'as ambos procesos cumplen la condici\'on de ser
independientes entre s\'i. Para la $k$-\'esima cola se define la
tasa de arribo por
$\lambda_{k}=1/\esp\left[\xi_{k}\left(1\right)\right]$ y la tasa
de servicio como
$\mu_{k}=1/\esp\left[\eta_{k}\left(1\right)\right]$, finalmente se
define la carga de la cola como $\rho_{k}=\lambda_{k}/\mu_{k}$,
donde se pide que $\rho=\sum_{k=1}^{K}\rho_{k}<1$, para garantizar
la estabilidad del sistema, esto es cierto para las pol\'iticas de
servicio exhaustiva y cerrada, ver Geetor \cite{Getoor}.\\

Si denotamos por
\begin{itemize}
\item $Q_{k}\left(t\right)$ el n\'umero de usuarios presentes en
la cola $k$ al tiempo $t$; \item $A_{k}\left(t\right)$ los
residuales de los tiempos entre arribos a la cola $k$; para cada
servidor $m$; \item $B_{m}\left(t\right)$ denota a los residuales
de los tiempos de servicio al tiempo $t$; \item
$B_{m}^{0}\left(t\right)$ los residuales de los tiempos de
traslado de la cola $k$ a la pr\'oxima por atender al tiempo $t$,

\item sea
$C_{m}\left(t\right)$ el n\'umero de usuarios atendidos durante la
visita del servidor a la cola $k$ al tiempo $t$.
\end{itemize}


En este sentido, el proceso para el sistema de visitas se puede
definir como:

\begin{equation}\label{Esp.Edos.Down}
X\left(t\right)^{T}=\left(Q_{k}\left(t\right),A_{k}\left(t\right),B_{m}\left(t\right),B_{m}^{0}\left(t\right),C_{m}\left(t\right)\right),
\end{equation}
para $k=1,\ldots,K$ y $m=1,2,\ldots,M$, donde $T$ indica que es el
transpuesto del vector que se est\'a definiendo. El proceso $X$
evoluciona en el espacio de estados:
$\mathbb{X}=\ent_{+}^{K}\times\rea_{+}^{K}\times\left(\left\{1,2,\ldots,K\right\}\times\left\{1,2,\ldots,S\right\}\right)^{M}\times\rea_{+}^{K}\times\ent_{+}^{K}$.\\

El sistema aqu\'i descrito debe de cumplir con los siguientes supuestos b\'asicos de un sistema de visitas:
%__________________________________________________________________________
\subsubsection{Supuestos B\'asicos}
%__________________________________________________________________________
\begin{itemize}
\item[A1)] Los procesos
$\xi_{1},\ldots,\xi_{K},\eta_{1},\ldots,\eta_{K}$ son mutuamente
independientes y son sucesiones independientes e id\'enticamente
distribuidas.

\item[A2)] Para alg\'un entero $p\geq1$
\begin{eqnarray*}
\esp\left[\xi_{l}\left(1\right)^{p+1}\right]&<&\infty\textrm{ para }l=1,\ldots,K\textrm{ y }\\
\esp\left[\eta_{k}\left(1\right)^{p+1}\right]&<&\infty\textrm{
para }k=1,\ldots,K.
\end{eqnarray*}
donde $\mathcal{A}$ es la clase de posibles arribos.

\item[A3)] Para cada $k=1,2,\ldots,K$ existe una funci\'on
positiva $q_{k}\left(\cdot\right)$ definida en $\rea_{+}$, y un
entero $j_{k}$, tal que
\begin{eqnarray}
P\left(\xi_{k}\left(1\right)\geq x\right)&>&0\textrm{, para todo }x>0,\\
P\left\{a\leq\sum_{i=1}^{j_{k}}\xi_{k}\left(i\right)\leq
b\right\}&\geq&\int_{a}^{b}q_{k}\left(x\right)dx, \textrm{ }0\leq
a<b.
\end{eqnarray}
\end{itemize}

En lo que respecta al supuesto (A3), en Dai y Meyn \cite{DaiSean}
hacen ver que este se puede sustituir por

\begin{itemize}
\item[A3')] Para el Proceso de Markov $X$, cada subconjunto
compacto del espacio de estados de $X$ es un conjunto peque\~no,
ver definici\'on \ref{Def.Cto.Peq.}.
\end{itemize}

Es por esta raz\'on que con la finalidad de poder hacer uso de
$A3^{'})$ es necesario recurrir a los Procesos de Harris y en
particular a los Procesos Harris Recurrente, ver \cite{Dai,
DaiSean}.
%_______________________________________________________________________
\subsection{Procesos Harris Recurrente}
%_______________________________________________________________________

Por el supuesto (A1) conforme a Davis \cite{Davis}, se puede
definir el proceso de saltos correspondiente de manera tal que
satisfaga el supuesto (A3'), de hecho la demostraci\'on est\'a
basada en la l\'inea de argumentaci\'on de Davis, \cite{Davis},
p\'aginas 362-364.\\

Entonces se tiene un espacio de estados en el cual el proceso $X$
satisface la Propiedad Fuerte de Markov, ver Dai y Meyn
\cite{DaiSean}, dado por

\[\left(\Omega,\mathcal{F},\mathcal{F}_{t},X\left(t\right),\theta_{t},P_{x}\right),\]
adem\'as de ser un proceso de Borel Derecho (Sharpe \cite{Sharpe})
en el espacio de estados medible
$\left(\mathbb{X},\mathcal{B}_\mathbb{X}\right)$. El Proceso
$X=\left\{X\left(t\right),t\geq0\right\}$ tiene trayectorias
continuas por la derecha, est\'a definido en
$\left(\Omega,\mathcal{F}\right)$ y est\'a adaptado a
$\left\{\mathcal{F}_{t},t\geq0\right\}$; la colecci\'on
$\left\{P_{x},x\in \mathbb{X}\right\}$ son medidas de probabilidad
en $\left(\Omega,\mathcal{F}\right)$ tales que para todo $x\in
\mathbb{X}$
\[P_{x}\left\{X\left(0\right)=x\right\}=1,\] y
\[E_{x}\left\{f\left(X\circ\theta_{t}\right)|\mathcal{F}_{t}\right\}=E_{X}\left(\tau\right)f\left(X\right),\]
en $\left\{\tau<\infty\right\}$, $P_{x}$-c.s., con $\theta_{t}$
definido como el operador shift.


Donde $\tau$ es un $\mathcal{F}_{t}$-tiempo de paro
\[\left(X\circ\theta_{\tau}\right)\left(w\right)=\left\{X\left(\tau\left(w\right)+t,w\right),t\geq0\right\},\]
y $f$ es una funci\'on de valores reales acotada y medible, ver \cite{Dai, KaspiMandelbaum}.\\

Sea $P^{t}\left(x,D\right)$, $D\in\mathcal{B}_{\mathbb{X}}$,
$t\geq0$ la probabilidad de transici\'on de $X$ queda definida
como:
\[P^{t}\left(x,D\right)=P_{x}\left(X\left(t\right)\in
D\right).\]


\begin{Def}
Una medida no cero $\pi$ en
$\left(\mathbb{X},\mathcal{B}_{\mathbb{X}}\right)$ es invariante
para $X$ si $\pi$ es $\sigma$-finita y
\[\pi\left(D\right)=\int_{\mathbb{X}}P^{t}\left(x,D\right)\pi\left(dx\right),\]
para todo $D\in \mathcal{B}_{\mathbb{X}}$, con $t\geq0$.
\end{Def}

\begin{Def}
El proceso de Markov $X$ es llamado Harris Recurrente si existe
una medida de probabilidad $\nu$ en
$\left(\mathbb{X},\mathcal{B}_{\mathbb{X}}\right)$, tal que si
$\nu\left(D\right)>0$ y $D\in\mathcal{B}_{\mathbb{X}}$
\[P_{x}\left\{\tau_{D}<\infty\right\}\equiv1,\] cuando
$\tau_{D}=inf\left\{t\geq0:X_{t}\in D\right\}$.
\end{Def}

\begin{Note}
\begin{itemize}
\item[i)] Si $X$ es Harris recurrente, entonces existe una \'unica
medida invariante $\pi$ (Getoor \cite{Getoor}).

\item[ii)] Si la medida invariante es finita, entonces puede
normalizarse a una medida de probabilidad, en este caso al proceso
$X$ se le llama Harris recurrente positivo.


\item[iii)] Cuando $X$ es Harris recurrente positivo se dice que
la disciplina de servicio es estable. En este caso $\pi$ denota la
distribuci\'on estacionaria y hacemos
\[P_{\pi}\left(\cdot\right)=\int_{\mathbf{X}}P_{x}\left(\cdot\right)\pi\left(dx\right),\]
y se utiliza $E_{\pi}$ para denotar el operador esperanza
correspondiente, ver \cite{DaiSean}.
\end{itemize}
\end{Note}

\begin{Def}\label{Def.Cto.Peq.}
Un conjunto $D\in\mathcal{B_{\mathbb{X}}}$ es llamado peque\~no si
existe un $t>0$, una medida de probabilidad $\nu$ en
$\mathcal{B_{\mathbb{X}}}$, y un $\delta>0$ tal que
\[P^{t}\left(x,A\right)\geq\delta\nu\left(A\right),\] para $x\in
D,A\in\mathcal{B_{\mathbb{X}}}$.
\end{Def}

La siguiente serie de resultados vienen enunciados y demostrados
en Dai \cite{Dai}:
\begin{Lema}[Lema 3.1, Dai \cite{Dai}]
Sea $B$ conjunto peque\~no cerrado, supongamos que
$P_{x}\left(\tau_{B}<\infty\right)\equiv1$ y que para alg\'un
$\delta>0$ se cumple que
\begin{equation}\label{Eq.3.1}
\sup\esp_{x}\left[\tau_{B}\left(\delta\right)\right]<\infty,
\end{equation}
donde
$\tau_{B}\left(\delta\right)=inf\left\{t\geq\delta:X\left(t\right)\in
B\right\}$. Entonces, $X$ es un proceso Harris recurrente
positivo.
\end{Lema}

\begin{Lema}[Lema 3.1, Dai \cite{Dai}]\label{Lema.3.}
Bajo el supuesto (A3), el conjunto
$B=\left\{x\in\mathbb{X}:|x|\leq k\right\}$ es un conjunto
peque\~no cerrado para cualquier $k>0$.
\end{Lema}

\begin{Teo}[Teorema 3.1, Dai \cite{Dai}]\label{Tma.3.1}
Si existe un $\delta>0$ tal que
\begin{equation}
lim_{|x|\rightarrow\infty}\frac{1}{|x|}\esp|X^{x}\left(|x|\delta\right)|=0,
\end{equation}
donde $X^{x}$ se utiliza para denotar que el proceso $X$ comienza
a partir de $x$, entonces la ecuaci\'on (\ref{Eq.3.1}) se cumple
para $B=\left\{x\in\mathbb{X}:|x|\leq k\right\}$ con alg\'un
$k>0$. En particular, $X$ es Harris recurrente positivo.
\end{Teo}

Entonces, tenemos que el proceso $X$ es un proceso de Markov que
cumple con los supuestos $A1)$-$A3)$, lo que falta de hacer es
construir el Modelo de Flujo bas\'andonos en lo hasta ahora
presentado.
%_______________________________________________________________________
\subsection{Modelo de Flujo}
%_______________________________________________________________________

Dada una condici\'on inicial $x\in\mathbb{X}$, sea

\begin{itemize}
\item $Q_{k}^{x}\left(t\right)$ la longitud de la cola al tiempo
$t$,

\item $T_{m,k}^{x}\left(t\right)$ el tiempo acumulado, al tiempo
$t$, que tarda el servidor $m$ en atender a los usuarios de la
cola $k$.

\item $T_{m,k}^{x,0}\left(t\right)$ el tiempo acumulado, al tiempo
$t$, que tarda el servidor $m$ en trasladarse a otra cola a partir de la $k$-\'esima.\\
\end{itemize}

Sup\'ongase que la funci\'on
$\left(\overline{Q}\left(\cdot\right),\overline{T}_{m}
\left(\cdot\right),\overline{T}_{m}^{0} \left(\cdot\right)\right)$
para $m=1,2,\ldots,M$ es un punto l\'imite de
\begin{equation}\label{Eq.Punto.Limite}
\left(\frac{1}{|x|}Q^{x}\left(|x|t\right),\frac{1}{|x|}T_{m}^{x}\left(|x|t\right),\frac{1}{|x|}T_{m}^{x,0}\left(|x|t\right)\right)
\end{equation}
para $m=1,2,\ldots,M$, cuando $x\rightarrow\infty$, ver
\cite{Down}. Entonces
$\left(\overline{Q}\left(t\right),\overline{T}_{m}
\left(t\right),\overline{T}_{m}^{0} \left(t\right)\right)$ es un
flujo l\'imite del sistema. Al conjunto de todos las posibles
flujos l\'imite se le llama {\emph{Modelo de Flujo}} y se le
denotar\'a por $\mathcal{Q}$, ver \cite{Down, Dai, DaiSean}.\\

El modelo de flujo satisface el siguiente conjunto de ecuaciones:

\begin{equation}\label{Eq.MF.1}
\overline{Q}_{k}\left(t\right)=\overline{Q}_{k}\left(0\right)+\lambda_{k}t-\sum_{m=1}^{M}\mu_{k}\overline{T}_{m,k}\left(t\right),\\
\end{equation}
para $k=1,2,\ldots,K$.\\
\begin{equation}\label{Eq.MF.2}
\overline{Q}_{k}\left(t\right)\geq0\textrm{ para
}k=1,2,\ldots,K.\\
\end{equation}

\begin{equation}\label{Eq.MF.3}
\overline{T}_{m,k}\left(0\right)=0,\textrm{ y }\overline{T}_{m,k}\left(\cdot\right)\textrm{ es no decreciente},\\
\end{equation}
para $k=1,2,\ldots,K$ y $m=1,2,\ldots,M$.\\
\begin{equation}\label{Eq.MF.4}
\sum_{k=1}^{K}\overline{T}_{m,k}^{0}\left(t\right)+\overline{T}_{m,k}\left(t\right)=t\textrm{
para }m=1,2,\ldots,M.\\
\end{equation}


\begin{Def}[Definici\'on 4.1, Dai \cite{Dai}]\label{Def.Modelo.Flujo}
Sea una disciplina de servicio espec\'ifica. Cualquier l\'imite
$\left(\overline{Q}\left(\cdot\right),\overline{T}\left(\cdot\right),\overline{T}^{0}\left(\cdot\right)\right)$
en (\ref{Eq.Punto.Limite}) es un {\em flujo l\'imite} de la
disciplina. Cualquier soluci\'on (\ref{Eq.MF.1})-(\ref{Eq.MF.4})
es llamado flujo soluci\'on de la disciplina.
\end{Def}

\begin{Def}
Se dice que el modelo de flujo l\'imite, modelo de flujo, de la
disciplina de la cola es estable si existe una constante
$\delta>0$ que depende de $\mu,\lambda$ y $P$ solamente, tal que
cualquier flujo l\'imite con
$|\overline{Q}\left(0\right)|+|\overline{U}|+|\overline{V}|=1$, se
tiene que $\overline{Q}\left(\cdot+\delta\right)\equiv0$.
\end{Def}

Si se hace $|x|\rightarrow\infty$ sin restringir ninguna de las
componentes, tambi\'en se obtienen un modelo de flujo, pero en
este caso el residual de los procesos de arribo y servicio
introducen un retraso:
\begin{Teo}[Teorema 4.2, Dai \cite{Dai}]\label{Tma.4.2.Dai}
Sea una disciplina fija para la cola, suponga que se cumplen las
condiciones (A1)-(A3). Si el modelo de flujo l\'imite de la
disciplina de la cola es estable, entonces la cadena de Markov $X$
que describe la din\'amica de la red bajo la disciplina es Harris
recurrente positiva.
\end{Teo}

Ahora se procede a escalar el espacio y el tiempo para reducir la
aparente fluctuaci\'on del modelo. Consid\'erese el proceso
\begin{equation}\label{Eq.3.7}
\overline{Q}^{x}\left(t\right)=\frac{1}{|x|}Q^{x}\left(|x|t\right).
\end{equation}
A este proceso se le conoce como el flujo escalado, y cualquier
l\'imite $\overline{Q}^{x}\left(t\right)$ es llamado flujo
l\'imite del proceso de longitud de la cola. Haciendo
$|q|\rightarrow\infty$ mientras se mantiene el resto de las
componentes fijas, cualquier punto l\'imite del proceso de
longitud de la cola normalizado $\overline{Q}^{x}$ es soluci\'on
del siguiente modelo de flujo.


\begin{Def}[Definici\'on 3.3, Dai y Meyn \cite{DaiSean}]
El modelo de flujo es estable si existe un tiempo fijo $t_{0}$ tal
que $\overline{Q}\left(t\right)=0$, con $t\geq t_{0}$, para
cualquier $\overline{Q}\left(\cdot\right)\in\mathcal{Q}$ que
cumple con $|\overline{Q}\left(0\right)|=1$.
\end{Def}

\begin{Lemma}[Lema 3.1, Dai y Meyn \cite{DaiSean}]
Si el modelo de flujo definido por (\ref{Eq.MF.1})-(\ref{Eq.MF.4})
es estable, entonces el modelo de flujo retrasado es tambi\'en
estable, es decir, existe $t_{0}>0$ tal que
$\overline{Q}\left(t\right)=0$ para cualquier $t\geq t_{0}$, para
cualquier soluci\'on del modelo de flujo retrasado cuya
condici\'on inicial $\overline{x}$ satisface que
$|\overline{x}|=|\overline{Q}\left(0\right)|+|\overline{A}\left(0\right)|+|\overline{B}\left(0\right)|\leq1$.
\end{Lemma}


Ahora ya estamos en condiciones de enunciar los resultados principales:


\begin{Teo}[Teorema 2.1, Down \cite{Down}]\label{Tma2.1.Down}
Suponga que el modelo de flujo es estable, y que se cumplen los supuestos (A1) y (A2), entonces
\begin{itemize}
\item[i)] Para alguna constante $\kappa_{p}$, y para cada
condici\'on inicial $x\in X$
\begin{equation}\label{Estability.Eq1}
\limsup_{t\rightarrow\infty}\frac{1}{t}\int_{0}^{t}\esp_{x}\left[|Q\left(s\right)|^{p}\right]ds\leq\kappa_{p},
\end{equation}
donde $p$ es el entero dado en (A2).
\end{itemize}
Si adem\'as se cumple la condici\'on (A3), entonces para cada
condici\'on inicial:
\begin{itemize}
\item[ii)] Los momentos transitorios convergen a su estado
estacionario:
 \begin{equation}\label{Estability.Eq2}
lim_{t\rightarrow\infty}\esp_{x}\left[Q_{k}\left(t\right)^{r}\right]=\esp_{\pi}\left[Q_{k}\left(0\right)^{r}\right]\leq\kappa_{r},
\end{equation}
para $r=1,2,\ldots,p$ y $k=1,2,\ldots,K$. Donde $\pi$ es la
probabilidad invariante para $X$.

\item[iii)]  El primer momento converge con raz\'on $t^{p-1}$:
\begin{equation}\label{Estability.Eq3}
lim_{t\rightarrow\infty}t^{p-1}|\esp_{x}\left[Q_{k}\left(t\right)\right]-\esp_{\pi}\left[Q_{k}\left(0\right)\right]|=0.
\end{equation}

\item[iv)] La {\em Ley Fuerte de los grandes n\'umeros} se cumple:
\begin{equation}\label{Estability.Eq4}
lim_{t\rightarrow\infty}\frac{1}{t}\int_{0}^{t}Q_{k}^{r}\left(s\right)ds=\esp_{\pi}\left[Q_{k}\left(0\right)^{r}\right],\textrm{
}\prob_{x}\textrm{-c.s.}
\end{equation}
para $r=1,2,\ldots,p$ y $k=1,2,\ldots,K$.
\end{itemize}
\end{Teo}

La contribuci\'on de Down a la teor\'ia de los {\emph {sistemas de
visitas c\'iclicas}}, es la relaci\'on que hay entre la
estabilidad del sistema con el comportamiento de las medidas de
desempe\~no, es decir, la condici\'on suficiente para poder
garantizar la convergencia del proceso de la longitud de la cola
as\'i como de por los menos los dos primeros momentos adem\'as de
una versi\'on de la Ley Fuerte de los Grandes N\'umeros para los
sistemas de visitas.


\begin{Teo}[Teorema 2.3, Down \cite{Down}]\label{Tma2.3.Down}
Considere el siguiente valor:
\begin{equation}\label{Eq.Rho.1serv}
\rho=\sum_{k=1}^{K}\rho_{k}+max_{1\leq j\leq K}\left(\frac{\lambda_{j}}{\sum_{s=1}^{S}p_{js}\overline{N}_{s}}\right)\delta^{*}
\end{equation}
\begin{itemize}
\item[i)] Si $\rho<1$ entonces la red es estable, es decir, se
cumple el Teorema \ref{Tma2.1.Down}.

\item[ii)] Si $\rho>1$ entonces la red es inestable, es decir, se
cumple el Teorema \ref{Tma2.2.Down}
\end{itemize}
\end{Teo}



El sistema aqu\'i descrito debe de cumplir con los siguientes supuestos b\'asicos de un sistema de visitas:
%__________________________________________________________________________
\subsubsection{Supuestos B\'asicos}
%__________________________________________________________________________
\begin{itemize}
\item[A1)] Los procesos
$\xi_{1},\ldots,\xi_{K},\eta_{1},\ldots,\eta_{K}$ son mutuamente
independientes y son sucesiones independientes e id\'enticamente
distribuidas.

\item[A2)] Para alg\'un entero $p\geq1$
\begin{eqnarray*}
\esp\left[\xi_{l}\left(1\right)^{p+1}\right]&<&\infty\textrm{ para }l=1,\ldots,K\textrm{ y }\\
\esp\left[\eta_{k}\left(1\right)^{p+1}\right]&<&\infty\textrm{
para }k=1,\ldots,K.
\end{eqnarray*}
donde $\mathcal{A}$ es la clase de posibles arribos.

\item[A3)] Para cada $k=1,2,\ldots,K$ existe una funci\'on
positiva $q_{k}\left(\cdot\right)$ definida en $\rea_{+}$, y un
entero $j_{k}$, tal que
\begin{eqnarray}
P\left(\xi_{k}\left(1\right)\geq x\right)&>&0\textrm{, para todo }x>0,\\
P\left\{a\leq\sum_{i=1}^{j_{k}}\xi_{k}\left(i\right)\leq
b\right\}&\geq&\int_{a}^{b}q_{k}\left(x\right)dx, \textrm{ }0\leq
a<b.
\end{eqnarray}
\end{itemize}

En lo que respecta al supuesto (A3), en Dai y Meyn \cite{DaiSean}
hacen ver que este se puede sustituir por

\begin{itemize}
\item[A3')] Para el Proceso de Markov $X$, cada subconjunto
compacto del espacio de estados de $X$ es un conjunto peque\~no,
ver definici\'on \ref{Def.Cto.Peq.}.
\end{itemize}

Es por esta raz\'on que con la finalidad de poder hacer uso de
$A3^{'})$ es necesario recurrir a los Procesos de Harris y en
particular a los Procesos Harris Recurrente, ver \cite{Dai,
DaiSean}.
%_______________________________________________________________________
\subsection{Procesos Harris Recurrente}
%_______________________________________________________________________

Por el supuesto (A1) conforme a Davis \cite{Davis}, se puede
definir el proceso de saltos correspondiente de manera tal que
satisfaga el supuesto (A3'), de hecho la demostraci\'on est\'a
basada en la l\'inea de argumentaci\'on de Davis, \cite{Davis},
p\'aginas 362-364.\\

Entonces se tiene un espacio de estados en el cual el proceso $X$
satisface la Propiedad Fuerte de Markov, ver Dai y Meyn
\cite{DaiSean}, dado por

\[\left(\Omega,\mathcal{F},\mathcal{F}_{t},X\left(t\right),\theta_{t},P_{x}\right),\]
adem\'as de ser un proceso de Borel Derecho (Sharpe \cite{Sharpe})
en el espacio de estados medible
$\left(\mathbb{X},\mathcal{B}_\mathbb{X}\right)$. El Proceso
$X=\left\{X\left(t\right),t\geq0\right\}$ tiene trayectorias
continuas por la derecha, est\'a definido en
$\left(\Omega,\mathcal{F}\right)$ y est\'a adaptado a
$\left\{\mathcal{F}_{t},t\geq0\right\}$; la colecci\'on
$\left\{P_{x},x\in \mathbb{X}\right\}$ son medidas de probabilidad
en $\left(\Omega,\mathcal{F}\right)$ tales que para todo $x\in
\mathbb{X}$
\[P_{x}\left\{X\left(0\right)=x\right\}=1,\] y
\[E_{x}\left\{f\left(X\circ\theta_{t}\right)|\mathcal{F}_{t}\right\}=E_{X}\left(\tau\right)f\left(X\right),\]
en $\left\{\tau<\infty\right\}$, $P_{x}$-c.s., con $\theta_{t}$
definido como el operador shift.


Donde $\tau$ es un $\mathcal{F}_{t}$-tiempo de paro
\[\left(X\circ\theta_{\tau}\right)\left(w\right)=\left\{X\left(\tau\left(w\right)+t,w\right),t\geq0\right\},\]
y $f$ es una funci\'on de valores reales acotada y medible, ver \cite{Dai, KaspiMandelbaum}.\\

Sea $P^{t}\left(x,D\right)$, $D\in\mathcal{B}_{\mathbb{X}}$,
$t\geq0$ la probabilidad de transici\'on de $X$ queda definida
como:
\[P^{t}\left(x,D\right)=P_{x}\left(X\left(t\right)\in
D\right).\]


\begin{Def}
Una medida no cero $\pi$ en
$\left(\mathbb{X},\mathcal{B}_{\mathbb{X}}\right)$ es invariante
para $X$ si $\pi$ es $\sigma$-finita y
\[\pi\left(D\right)=\int_{\mathbb{X}}P^{t}\left(x,D\right)\pi\left(dx\right),\]
para todo $D\in \mathcal{B}_{\mathbb{X}}$, con $t\geq0$.
\end{Def}

\begin{Def}
El proceso de Markov $X$ es llamado Harris Recurrente si existe
una medida de probabilidad $\nu$ en
$\left(\mathbb{X},\mathcal{B}_{\mathbb{X}}\right)$, tal que si
$\nu\left(D\right)>0$ y $D\in\mathcal{B}_{\mathbb{X}}$
\[P_{x}\left\{\tau_{D}<\infty\right\}\equiv1,\] cuando
$\tau_{D}=inf\left\{t\geq0:X_{t}\in D\right\}$.
\end{Def}

\begin{Note}
\begin{itemize}
\item[i)] Si $X$ es Harris recurrente, entonces existe una \'unica
medida invariante $\pi$ (Getoor \cite{Getoor}).

\item[ii)] Si la medida invariante es finita, entonces puede
normalizarse a una medida de probabilidad, en este caso al proceso
$X$ se le llama Harris recurrente positivo.


\item[iii)] Cuando $X$ es Harris recurrente positivo se dice que
la disciplina de servicio es estable. En este caso $\pi$ denota la
distribuci\'on estacionaria y hacemos
\[P_{\pi}\left(\cdot\right)=\int_{\mathbf{X}}P_{x}\left(\cdot\right)\pi\left(dx\right),\]
y se utiliza $E_{\pi}$ para denotar el operador esperanza
correspondiente, ver \cite{DaiSean}.
\end{itemize}
\end{Note}

\begin{Def}\label{Def.Cto.Peq.}
Un conjunto $D\in\mathcal{B_{\mathbb{X}}}$ es llamado peque\~no si
existe un $t>0$, una medida de probabilidad $\nu$ en
$\mathcal{B_{\mathbb{X}}}$, y un $\delta>0$ tal que
\[P^{t}\left(x,A\right)\geq\delta\nu\left(A\right),\] para $x\in
D,A\in\mathcal{B_{\mathbb{X}}}$.
\end{Def}

La siguiente serie de resultados vienen enunciados y demostrados
en Dai \cite{Dai}:
\begin{Lema}[Lema 3.1, Dai \cite{Dai}]
Sea $B$ conjunto peque\~no cerrado, supongamos que
$P_{x}\left(\tau_{B}<\infty\right)\equiv1$ y que para alg\'un
$\delta>0$ se cumple que
\begin{equation}\label{Eq.3.1}
\sup\esp_{x}\left[\tau_{B}\left(\delta\right)\right]<\infty,
\end{equation}
donde
$\tau_{B}\left(\delta\right)=inf\left\{t\geq\delta:X\left(t\right)\in
B\right\}$. Entonces, $X$ es un proceso Harris recurrente
positivo.
\end{Lema}

\begin{Lema}[Lema 3.1, Dai \cite{Dai}]\label{Lema.3.}
Bajo el supuesto (A3), el conjunto
$B=\left\{x\in\mathbb{X}:|x|\leq k\right\}$ es un conjunto
peque\~no cerrado para cualquier $k>0$.
\end{Lema}

\begin{Teo}[Teorema 3.1, Dai \cite{Dai}]\label{Tma.3.1}
Si existe un $\delta>0$ tal que
\begin{equation}
lim_{|x|\rightarrow\infty}\frac{1}{|x|}\esp|X^{x}\left(|x|\delta\right)|=0,
\end{equation}
donde $X^{x}$ se utiliza para denotar que el proceso $X$ comienza
a partir de $x$, entonces la ecuaci\'on (\ref{Eq.3.1}) se cumple
para $B=\left\{x\in\mathbb{X}:|x|\leq k\right\}$ con alg\'un
$k>0$. En particular, $X$ es Harris recurrente positivo.
\end{Teo}

Entonces, tenemos que el proceso $X$ es un proceso de Markov que
cumple con los supuestos $A1)$-$A3)$, lo que falta de hacer es
construir el Modelo de Flujo bas\'andonos en lo hasta ahora
presentado.
%_______________________________________________________________________
\subsection{Modelo de Flujo}
%_______________________________________________________________________

Dada una condici\'on inicial $x\in\mathbb{X}$, sea

\begin{itemize}
\item $Q_{k}^{x}\left(t\right)$ la longitud de la cola al tiempo
$t$,

\item $T_{m,k}^{x}\left(t\right)$ el tiempo acumulado, al tiempo
$t$, que tarda el servidor $m$ en atender a los usuarios de la
cola $k$.

\item $T_{m,k}^{x,0}\left(t\right)$ el tiempo acumulado, al tiempo
$t$, que tarda el servidor $m$ en trasladarse a otra cola a partir de la $k$-\'esima.\\
\end{itemize}

Sup\'ongase que la funci\'on
$\left(\overline{Q}\left(\cdot\right),\overline{T}_{m}
\left(\cdot\right),\overline{T}_{m}^{0} \left(\cdot\right)\right)$
para $m=1,2,\ldots,M$ es un punto l\'imite de
\begin{equation}\label{Eq.Punto.Limite}
\left(\frac{1}{|x|}Q^{x}\left(|x|t\right),\frac{1}{|x|}T_{m}^{x}\left(|x|t\right),\frac{1}{|x|}T_{m}^{x,0}\left(|x|t\right)\right)
\end{equation}
para $m=1,2,\ldots,M$, cuando $x\rightarrow\infty$, ver
\cite{Down}. Entonces
$\left(\overline{Q}\left(t\right),\overline{T}_{m}
\left(t\right),\overline{T}_{m}^{0} \left(t\right)\right)$ es un
flujo l\'imite del sistema. Al conjunto de todos las posibles
flujos l\'imite se le llama {\emph{Modelo de Flujo}} y se le
denotar\'a por $\mathcal{Q}$, ver \cite{Down, Dai, DaiSean}.\\

El modelo de flujo satisface el siguiente conjunto de ecuaciones:

\begin{equation}\label{Eq.MF.1}
\overline{Q}_{k}\left(t\right)=\overline{Q}_{k}\left(0\right)+\lambda_{k}t-\sum_{m=1}^{M}\mu_{k}\overline{T}_{m,k}\left(t\right),\\
\end{equation}
para $k=1,2,\ldots,K$.\\
\begin{equation}\label{Eq.MF.2}
\overline{Q}_{k}\left(t\right)\geq0\textrm{ para
}k=1,2,\ldots,K.\\
\end{equation}

\begin{equation}\label{Eq.MF.3}
\overline{T}_{m,k}\left(0\right)=0,\textrm{ y }\overline{T}_{m,k}\left(\cdot\right)\textrm{ es no decreciente},\\
\end{equation}
para $k=1,2,\ldots,K$ y $m=1,2,\ldots,M$.\\
\begin{equation}\label{Eq.MF.4}
\sum_{k=1}^{K}\overline{T}_{m,k}^{0}\left(t\right)+\overline{T}_{m,k}\left(t\right)=t\textrm{
para }m=1,2,\ldots,M.\\
\end{equation}


\begin{Def}[Definici\'on 4.1, Dai \cite{Dai}]\label{Def.Modelo.Flujo}
Sea una disciplina de servicio espec\'ifica. Cualquier l\'imite
$\left(\overline{Q}\left(\cdot\right),\overline{T}\left(\cdot\right),\overline{T}^{0}\left(\cdot\right)\right)$
en (\ref{Eq.Punto.Limite}) es un {\em flujo l\'imite} de la
disciplina. Cualquier soluci\'on (\ref{Eq.MF.1})-(\ref{Eq.MF.4})
es llamado flujo soluci\'on de la disciplina.
\end{Def}

\begin{Def}
Se dice que el modelo de flujo l\'imite, modelo de flujo, de la
disciplina de la cola es estable si existe una constante
$\delta>0$ que depende de $\mu,\lambda$ y $P$ solamente, tal que
cualquier flujo l\'imite con
$|\overline{Q}\left(0\right)|+|\overline{U}|+|\overline{V}|=1$, se
tiene que $\overline{Q}\left(\cdot+\delta\right)\equiv0$.
\end{Def}

Si se hace $|x|\rightarrow\infty$ sin restringir ninguna de las
componentes, tambi\'en se obtienen un modelo de flujo, pero en
este caso el residual de los procesos de arribo y servicio
introducen un retraso:
\begin{Teo}[Teorema 4.2, Dai \cite{Dai}]\label{Tma.4.2.Dai}
Sea una disciplina fija para la cola, suponga que se cumplen las
condiciones (A1)-(A3). Si el modelo de flujo l\'imite de la
disciplina de la cola es estable, entonces la cadena de Markov $X$
que describe la din\'amica de la red bajo la disciplina es Harris
recurrente positiva.
\end{Teo}

Ahora se procede a escalar el espacio y el tiempo para reducir la
aparente fluctuaci\'on del modelo. Consid\'erese el proceso
\begin{equation}\label{Eq.3.7}
\overline{Q}^{x}\left(t\right)=\frac{1}{|x|}Q^{x}\left(|x|t\right).
\end{equation}
A este proceso se le conoce como el flujo escalado, y cualquier
l\'imite $\overline{Q}^{x}\left(t\right)$ es llamado flujo
l\'imite del proceso de longitud de la cola. Haciendo
$|q|\rightarrow\infty$ mientras se mantiene el resto de las
componentes fijas, cualquier punto l\'imite del proceso de
longitud de la cola normalizado $\overline{Q}^{x}$ es soluci\'on
del siguiente modelo de flujo.


\begin{Def}[Definici\'on 3.3, Dai y Meyn \cite{DaiSean}]
El modelo de flujo es estable si existe un tiempo fijo $t_{0}$ tal
que $\overline{Q}\left(t\right)=0$, con $t\geq t_{0}$, para
cualquier $\overline{Q}\left(\cdot\right)\in\mathcal{Q}$ que
cumple con $|\overline{Q}\left(0\right)|=1$.
\end{Def}

\begin{Lemma}[Lema 3.1, Dai y Meyn \cite{DaiSean}]
Si el modelo de flujo definido por (\ref{Eq.MF.1})-(\ref{Eq.MF.4})
es estable, entonces el modelo de flujo retrasado es tambi\'en
estable, es decir, existe $t_{0}>0$ tal que
$\overline{Q}\left(t\right)=0$ para cualquier $t\geq t_{0}$, para
cualquier soluci\'on del modelo de flujo retrasado cuya
condici\'on inicial $\overline{x}$ satisface que
$|\overline{x}|=|\overline{Q}\left(0\right)|+|\overline{A}\left(0\right)|+|\overline{B}\left(0\right)|\leq1$.
\end{Lemma}


Ahora ya estamos en condiciones de enunciar los resultados principales:


\begin{Teo}[Teorema 2.1, Down \cite{Down}]\label{Tma2.1.Down}
Suponga que el modelo de flujo es estable, y que se cumplen los supuestos (A1) y (A2), entonces
\begin{itemize}
\item[i)] Para alguna constante $\kappa_{p}$, y para cada
condici\'on inicial $x\in X$
\begin{equation}\label{Estability.Eq1}
\limsup_{t\rightarrow\infty}\frac{1}{t}\int_{0}^{t}\esp_{x}\left[|Q\left(s\right)|^{p}\right]ds\leq\kappa_{p},
\end{equation}
donde $p$ es el entero dado en (A2).
\end{itemize}
Si adem\'as se cumple la condici\'on (A3), entonces para cada
condici\'on inicial:
\begin{itemize}
\item[ii)] Los momentos transitorios convergen a su estado
estacionario:
 \begin{equation}\label{Estability.Eq2}
lim_{t\rightarrow\infty}\esp_{x}\left[Q_{k}\left(t\right)^{r}\right]=\esp_{\pi}\left[Q_{k}\left(0\right)^{r}\right]\leq\kappa_{r},
\end{equation}
para $r=1,2,\ldots,p$ y $k=1,2,\ldots,K$. Donde $\pi$ es la
probabilidad invariante para $X$.

\item[iii)]  El primer momento converge con raz\'on $t^{p-1}$:
\begin{equation}\label{Estability.Eq3}
lim_{t\rightarrow\infty}t^{p-1}|\esp_{x}\left[Q_{k}\left(t\right)\right]-\esp_{\pi}\left[Q_{k}\left(0\right)\right]|=0.
\end{equation}

\item[iv)] La {\em Ley Fuerte de los grandes n\'umeros} se cumple:
\begin{equation}\label{Estability.Eq4}
lim_{t\rightarrow\infty}\frac{1}{t}\int_{0}^{t}Q_{k}^{r}\left(s\right)ds=\esp_{\pi}\left[Q_{k}\left(0\right)^{r}\right],\textrm{
}\prob_{x}\textrm{-c.s.}
\end{equation}
para $r=1,2,\ldots,p$ y $k=1,2,\ldots,K$.
\end{itemize}
\end{Teo}

La contribuci\'on de Down a la teor\'ia de los {\emph {sistemas de
visitas c\'iclicas}}, es la relaci\'on que hay entre la
estabilidad del sistema con el comportamiento de las medidas de
desempe\~no, es decir, la condici\'on suficiente para poder
garantizar la convergencia del proceso de la longitud de la cola
as\'i como de por los menos los dos primeros momentos adem\'as de
una versi\'on de la Ley Fuerte de los Grandes N\'umeros para los
sistemas de visitas.


\begin{Teo}[Teorema 2.3, Down \cite{Down}]\label{Tma2.3.Down}
Considere el siguiente valor:
\begin{equation}\label{Eq.Rho.1serv}
\rho=\sum_{k=1}^{K}\rho_{k}+max_{1\leq j\leq K}\left(\frac{\lambda_{j}}{\sum_{s=1}^{S}p_{js}\overline{N}_{s}}\right)\delta^{*}
\end{equation}
\begin{itemize}
\item[i)] Si $\rho<1$ entonces la red es estable, es decir, se
cumple el Teorema \ref{Tma2.1.Down}.

\item[ii)] Si $\rho>1$ entonces la red es inestable, es decir, se
cumple el Teorema \ref{Tma2.2.Down}
\end{itemize}
\end{Teo}


%_________________________________________________________________________
\subsection{Supuestos}
%_________________________________________________________________________
Consideremos el caso en el que se tienen varias colas a las cuales
llegan uno o varios servidores para dar servicio a los usuarios
que se encuentran presentes en la cola, como ya se mencion\'o hay
varios tipos de pol\'iticas de servicio, incluso podr\'ia ocurrir
que la manera en que atiende al resto de las colas sea distinta a
como lo hizo en las anteriores.\\

Para ejemplificar los sistemas de visitas c\'iclicas se
considerar\'a el caso en que a las colas los usuarios son atendidos con
una s\'ola pol\'itica de servicio.\\



Si $\omega$ es el n\'umero de usuarios en la cola al comienzo del
periodo de servicio y $N\left(\omega\right)$ es el n\'umero de
usuarios que son atendidos con una pol\'itica en espec\'ifico
durante el periodo de servicio, entonces se asume que:
\begin{itemize}
\item[1)]\label{S1}$lim_{\omega\rightarrow\infty}\esp\left[N\left(\omega\right)\right]=\overline{N}>0$;
\item[2)]\label{S2}$\esp\left[N\left(\omega\right)\right]\leq\overline{N}$
para cualquier valor de $\omega$.
\end{itemize}
La manera en que atiende el servidor $m$-\'esimo, es la siguiente:
\begin{itemize}
\item Al t\'ermino de la visita a la cola $j$, el servidor cambia
a la cola $j^{'}$ con probabilidad $r_{j,j^{'}}^{m}$

\item La $n$-\'esima vez que el servidor cambia de la cola $j$ a
$j'$, va acompa\~nada con el tiempo de cambio de longitud
$\delta_{j,j^{'}}^{m}\left(n\right)$, con
$\delta_{j,j^{'}}^{m}\left(n\right)$, $n\geq1$, variables
aleatorias independientes e id\'enticamente distribuidas, tales
que $\esp\left[\delta_{j,j^{'}}^{m}\left(1\right)\right]\geq0$.

\item Sea $\left\{p_{j}^{m}\right\}$ la distribuci\'on invariante
estacionaria \'unica para la Cadena de Markov con matriz de
transici\'on $\left(r_{j,j^{'}}^{m}\right)$, se supone que \'esta
existe.

\item Finalmente, se define el tiempo promedio total de traslado
entre las colas como
\begin{equation}
\delta^{*}:=\sum_{j,j^{'}}p_{j}^{m}r_{j,j^{'}}^{m}\esp\left[\delta_{j,j^{'}}^{m}\left(i\right)\right].
\end{equation}
\end{itemize}

Consideremos el caso donde los tiempos entre arribo a cada una de
las colas, $\left\{\xi_{k}\left(n\right)\right\}_{n\geq1}$ son
variables aleatorias independientes a id\'enticamente
distribuidas, y los tiempos de servicio en cada una de las colas
se distribuyen de manera independiente e id\'enticamente
distribuidas $\left\{\eta_{k}\left(n\right)\right\}_{n\geq1}$;
adem\'as ambos procesos cumplen la condici\'on de ser
independientes entre s\'i. Para la $k$-\'esima cola se define la
tasa de arribo por
$\lambda_{k}=1/\esp\left[\xi_{k}\left(1\right)\right]$ y la tasa
de servicio como
$\mu_{k}=1/\esp\left[\eta_{k}\left(1\right)\right]$, finalmente se
define la carga de la cola como $\rho_{k}=\lambda_{k}/\mu_{k}$,
donde se pide que $\rho=\sum_{k=1}^{K}\rho_{k}<1$, para garantizar
la estabilidad del sistema, esto es cierto para las pol\'iticas de
servicio exhaustiva y cerrada, ver Geetor \cite{Getoor}.\\

Si denotamos por
\begin{itemize}
\item $Q_{k}\left(t\right)$ el n\'umero de usuarios presentes en
la cola $k$ al tiempo $t$; \item $A_{k}\left(t\right)$ los
residuales de los tiempos entre arribos a la cola $k$; para cada
servidor $m$; \item $B_{m}\left(t\right)$ denota a los residuales
de los tiempos de servicio al tiempo $t$; \item
$B_{m}^{0}\left(t\right)$ los residuales de los tiempos de
traslado de la cola $k$ a la pr\'oxima por atender al tiempo $t$,

\item sea
$C_{m}\left(t\right)$ el n\'umero de usuarios atendidos durante la
visita del servidor a la cola $k$ al tiempo $t$.
\end{itemize}


En este sentido, el proceso para el sistema de visitas se puede
definir como:

\begin{equation}\label{Esp.Edos.Down}
X\left(t\right)^{T}=\left(Q_{k}\left(t\right),A_{k}\left(t\right),B_{m}\left(t\right),B_{m}^{0}\left(t\right),C_{m}\left(t\right)\right),
\end{equation}
para $k=1,\ldots,K$ y $m=1,2,\ldots,M$, donde $T$ indica que es el
transpuesto del vector que se est\'a definiendo. El proceso $X$
evoluciona en el espacio de estados:
$\mathbb{X}=\ent_{+}^{K}\times\rea_{+}^{K}\times\left(\left\{1,2,\ldots,K\right\}\times\left\{1,2,\ldots,S\right\}\right)^{M}\times\rea_{+}^{K}\times\ent_{+}^{K}$.\\

El sistema aqu\'i descrito debe de cumplir con los siguientes supuestos b\'asicos de un sistema de visitas:
%__________________________________________________________________________
\subsubsection{Supuestos B\'asicos}
%__________________________________________________________________________
\begin{itemize}
\item[A1)] Los procesos
$\xi_{1},\ldots,\xi_{K},\eta_{1},\ldots,\eta_{K}$ son mutuamente
independientes y son sucesiones independientes e id\'enticamente
distribuidas.

\item[A2)] Para alg\'un entero $p\geq1$
\begin{eqnarray*}
\esp\left[\xi_{l}\left(1\right)^{p+1}\right]&<&\infty\textrm{ para }l=1,\ldots,K\textrm{ y }\\
\esp\left[\eta_{k}\left(1\right)^{p+1}\right]&<&\infty\textrm{
para }k=1,\ldots,K.
\end{eqnarray*}
donde $\mathcal{A}$ es la clase de posibles arribos.

\item[A3)] Para cada $k=1,2,\ldots,K$ existe una funci\'on
positiva $q_{k}\left(\cdot\right)$ definida en $\rea_{+}$, y un
entero $j_{k}$, tal que
\begin{eqnarray}
P\left(\xi_{k}\left(1\right)\geq x\right)&>&0\textrm{, para todo }x>0,\\
P\left\{a\leq\sum_{i=1}^{j_{k}}\xi_{k}\left(i\right)\leq
b\right\}&\geq&\int_{a}^{b}q_{k}\left(x\right)dx, \textrm{ }0\leq
a<b.
\end{eqnarray}
\end{itemize}

En lo que respecta al supuesto (A3), en Dai y Meyn \cite{DaiSean}
hacen ver que este se puede sustituir por

\begin{itemize}
\item[A3')] Para el Proceso de Markov $X$, cada subconjunto
compacto del espacio de estados de $X$ es un conjunto peque\~no,
ver definici\'on \ref{Def.Cto.Peq.}.
\end{itemize}

Es por esta raz\'on que con la finalidad de poder hacer uso de
$A3^{'})$ es necesario recurrir a los Procesos de Harris y en
particular a los Procesos Harris Recurrente, ver \cite{Dai,
DaiSean}.
%_______________________________________________________________________
\subsection{Procesos Harris Recurrente}
%_______________________________________________________________________

Por el supuesto (A1) conforme a Davis \cite{Davis}, se puede
definir el proceso de saltos correspondiente de manera tal que
satisfaga el supuesto (A3'), de hecho la demostraci\'on est\'a
basada en la l\'inea de argumentaci\'on de Davis, \cite{Davis},
p\'aginas 362-364.\\

Entonces se tiene un espacio de estados en el cual el proceso $X$
satisface la Propiedad Fuerte de Markov, ver Dai y Meyn
\cite{DaiSean}, dado por

\[\left(\Omega,\mathcal{F},\mathcal{F}_{t},X\left(t\right),\theta_{t},P_{x}\right),\]
adem\'as de ser un proceso de Borel Derecho (Sharpe \cite{Sharpe})
en el espacio de estados medible
$\left(\mathbb{X},\mathcal{B}_\mathbb{X}\right)$. El Proceso
$X=\left\{X\left(t\right),t\geq0\right\}$ tiene trayectorias
continuas por la derecha, est\'a definido en
$\left(\Omega,\mathcal{F}\right)$ y est\'a adaptado a
$\left\{\mathcal{F}_{t},t\geq0\right\}$; la colecci\'on
$\left\{P_{x},x\in \mathbb{X}\right\}$ son medidas de probabilidad
en $\left(\Omega,\mathcal{F}\right)$ tales que para todo $x\in
\mathbb{X}$
\[P_{x}\left\{X\left(0\right)=x\right\}=1,\] y
\[E_{x}\left\{f\left(X\circ\theta_{t}\right)|\mathcal{F}_{t}\right\}=E_{X}\left(\tau\right)f\left(X\right),\]
en $\left\{\tau<\infty\right\}$, $P_{x}$-c.s., con $\theta_{t}$
definido como el operador shift.


Donde $\tau$ es un $\mathcal{F}_{t}$-tiempo de paro
\[\left(X\circ\theta_{\tau}\right)\left(w\right)=\left\{X\left(\tau\left(w\right)+t,w\right),t\geq0\right\},\]
y $f$ es una funci\'on de valores reales acotada y medible, ver \cite{Dai, KaspiMandelbaum}.\\

Sea $P^{t}\left(x,D\right)$, $D\in\mathcal{B}_{\mathbb{X}}$,
$t\geq0$ la probabilidad de transici\'on de $X$ queda definida
como:
\[P^{t}\left(x,D\right)=P_{x}\left(X\left(t\right)\in
D\right).\]


\begin{Def}
Una medida no cero $\pi$ en
$\left(\mathbb{X},\mathcal{B}_{\mathbb{X}}\right)$ es invariante
para $X$ si $\pi$ es $\sigma$-finita y
\[\pi\left(D\right)=\int_{\mathbb{X}}P^{t}\left(x,D\right)\pi\left(dx\right),\]
para todo $D\in \mathcal{B}_{\mathbb{X}}$, con $t\geq0$.
\end{Def}

\begin{Def}
El proceso de Markov $X$ es llamado Harris Recurrente si existe
una medida de probabilidad $\nu$ en
$\left(\mathbb{X},\mathcal{B}_{\mathbb{X}}\right)$, tal que si
$\nu\left(D\right)>0$ y $D\in\mathcal{B}_{\mathbb{X}}$
\[P_{x}\left\{\tau_{D}<\infty\right\}\equiv1,\] cuando
$\tau_{D}=inf\left\{t\geq0:X_{t}\in D\right\}$.
\end{Def}

\begin{Note}
\begin{itemize}
\item[i)] Si $X$ es Harris recurrente, entonces existe una \'unica
medida invariante $\pi$ (Getoor \cite{Getoor}).

\item[ii)] Si la medida invariante es finita, entonces puede
normalizarse a una medida de probabilidad, en este caso al proceso
$X$ se le llama Harris recurrente positivo.


\item[iii)] Cuando $X$ es Harris recurrente positivo se dice que
la disciplina de servicio es estable. En este caso $\pi$ denota la
distribuci\'on estacionaria y hacemos
\[P_{\pi}\left(\cdot\right)=\int_{\mathbf{X}}P_{x}\left(\cdot\right)\pi\left(dx\right),\]
y se utiliza $E_{\pi}$ para denotar el operador esperanza
correspondiente, ver \cite{DaiSean}.
\end{itemize}
\end{Note}

\begin{Def}\label{Def.Cto.Peq.}
Un conjunto $D\in\mathcal{B_{\mathbb{X}}}$ es llamado peque\~no si
existe un $t>0$, una medida de probabilidad $\nu$ en
$\mathcal{B_{\mathbb{X}}}$, y un $\delta>0$ tal que
\[P^{t}\left(x,A\right)\geq\delta\nu\left(A\right),\] para $x\in
D,A\in\mathcal{B_{\mathbb{X}}}$.
\end{Def}

La siguiente serie de resultados vienen enunciados y demostrados
en Dai \cite{Dai}:
\begin{Lema}[Lema 3.1, Dai \cite{Dai}]
Sea $B$ conjunto peque\~no cerrado, supongamos que
$P_{x}\left(\tau_{B}<\infty\right)\equiv1$ y que para alg\'un
$\delta>0$ se cumple que
\begin{equation}\label{Eq.3.1}
\sup\esp_{x}\left[\tau_{B}\left(\delta\right)\right]<\infty,
\end{equation}
donde
$\tau_{B}\left(\delta\right)=inf\left\{t\geq\delta:X\left(t\right)\in
B\right\}$. Entonces, $X$ es un proceso Harris recurrente
positivo.
\end{Lema}

\begin{Lema}[Lema 3.1, Dai \cite{Dai}]\label{Lema.3.}
Bajo el supuesto (A3), el conjunto
$B=\left\{x\in\mathbb{X}:|x|\leq k\right\}$ es un conjunto
peque\~no cerrado para cualquier $k>0$.
\end{Lema}

\begin{Teo}[Teorema 3.1, Dai \cite{Dai}]\label{Tma.3.1}
Si existe un $\delta>0$ tal que
\begin{equation}
lim_{|x|\rightarrow\infty}\frac{1}{|x|}\esp|X^{x}\left(|x|\delta\right)|=0,
\end{equation}
donde $X^{x}$ se utiliza para denotar que el proceso $X$ comienza
a partir de $x$, entonces la ecuaci\'on (\ref{Eq.3.1}) se cumple
para $B=\left\{x\in\mathbb{X}:|x|\leq k\right\}$ con alg\'un
$k>0$. En particular, $X$ es Harris recurrente positivo.
\end{Teo}

Entonces, tenemos que el proceso $X$ es un proceso de Markov que
cumple con los supuestos $A1)$-$A3)$, lo que falta de hacer es
construir el Modelo de Flujo bas\'andonos en lo hasta ahora
presentado.
%_______________________________________________________________________
\subsection{Modelo de Flujo}
%_______________________________________________________________________

Dada una condici\'on inicial $x\in\mathbb{X}$, sea

\begin{itemize}
\item $Q_{k}^{x}\left(t\right)$ la longitud de la cola al tiempo
$t$,

\item $T_{m,k}^{x}\left(t\right)$ el tiempo acumulado, al tiempo
$t$, que tarda el servidor $m$ en atender a los usuarios de la
cola $k$.

\item $T_{m,k}^{x,0}\left(t\right)$ el tiempo acumulado, al tiempo
$t$, que tarda el servidor $m$ en trasladarse a otra cola a partir de la $k$-\'esima.\\
\end{itemize}

Sup\'ongase que la funci\'on
$\left(\overline{Q}\left(\cdot\right),\overline{T}_{m}
\left(\cdot\right),\overline{T}_{m}^{0} \left(\cdot\right)\right)$
para $m=1,2,\ldots,M$ es un punto l\'imite de
\begin{equation}\label{Eq.Punto.Limite}
\left(\frac{1}{|x|}Q^{x}\left(|x|t\right),\frac{1}{|x|}T_{m}^{x}\left(|x|t\right),\frac{1}{|x|}T_{m}^{x,0}\left(|x|t\right)\right)
\end{equation}
para $m=1,2,\ldots,M$, cuando $x\rightarrow\infty$, ver
\cite{Down}. Entonces
$\left(\overline{Q}\left(t\right),\overline{T}_{m}
\left(t\right),\overline{T}_{m}^{0} \left(t\right)\right)$ es un
flujo l\'imite del sistema. Al conjunto de todos las posibles
flujos l\'imite se le llama {\emph{Modelo de Flujo}} y se le
denotar\'a por $\mathcal{Q}$, ver \cite{Down, Dai, DaiSean}.\\

El modelo de flujo satisface el siguiente conjunto de ecuaciones:

\begin{equation}\label{Eq.MF.1}
\overline{Q}_{k}\left(t\right)=\overline{Q}_{k}\left(0\right)+\lambda_{k}t-\sum_{m=1}^{M}\mu_{k}\overline{T}_{m,k}\left(t\right),\\
\end{equation}
para $k=1,2,\ldots,K$.\\
\begin{equation}\label{Eq.MF.2}
\overline{Q}_{k}\left(t\right)\geq0\textrm{ para
}k=1,2,\ldots,K.\\
\end{equation}

\begin{equation}\label{Eq.MF.3}
\overline{T}_{m,k}\left(0\right)=0,\textrm{ y }\overline{T}_{m,k}\left(\cdot\right)\textrm{ es no decreciente},\\
\end{equation}
para $k=1,2,\ldots,K$ y $m=1,2,\ldots,M$.\\
\begin{equation}\label{Eq.MF.4}
\sum_{k=1}^{K}\overline{T}_{m,k}^{0}\left(t\right)+\overline{T}_{m,k}\left(t\right)=t\textrm{
para }m=1,2,\ldots,M.\\
\end{equation}


\begin{Def}[Definici\'on 4.1, Dai \cite{Dai}]\label{Def.Modelo.Flujo}
Sea una disciplina de servicio espec\'ifica. Cualquier l\'imite
$\left(\overline{Q}\left(\cdot\right),\overline{T}\left(\cdot\right),\overline{T}^{0}\left(\cdot\right)\right)$
en (\ref{Eq.Punto.Limite}) es un {\em flujo l\'imite} de la
disciplina. Cualquier soluci\'on (\ref{Eq.MF.1})-(\ref{Eq.MF.4})
es llamado flujo soluci\'on de la disciplina.
\end{Def}

\begin{Def}
Se dice que el modelo de flujo l\'imite, modelo de flujo, de la
disciplina de la cola es estable si existe una constante
$\delta>0$ que depende de $\mu,\lambda$ y $P$ solamente, tal que
cualquier flujo l\'imite con
$|\overline{Q}\left(0\right)|+|\overline{U}|+|\overline{V}|=1$, se
tiene que $\overline{Q}\left(\cdot+\delta\right)\equiv0$.
\end{Def}

Si se hace $|x|\rightarrow\infty$ sin restringir ninguna de las
componentes, tambi\'en se obtienen un modelo de flujo, pero en
este caso el residual de los procesos de arribo y servicio
introducen un retraso:
\begin{Teo}[Teorema 4.2, Dai \cite{Dai}]\label{Tma.4.2.Dai}
Sea una disciplina fija para la cola, suponga que se cumplen las
condiciones (A1)-(A3). Si el modelo de flujo l\'imite de la
disciplina de la cola es estable, entonces la cadena de Markov $X$
que describe la din\'amica de la red bajo la disciplina es Harris
recurrente positiva.
\end{Teo}

Ahora se procede a escalar el espacio y el tiempo para reducir la
aparente fluctuaci\'on del modelo. Consid\'erese el proceso
\begin{equation}\label{Eq.3.7}
\overline{Q}^{x}\left(t\right)=\frac{1}{|x|}Q^{x}\left(|x|t\right).
\end{equation}
A este proceso se le conoce como el flujo escalado, y cualquier
l\'imite $\overline{Q}^{x}\left(t\right)$ es llamado flujo
l\'imite del proceso de longitud de la cola. Haciendo
$|q|\rightarrow\infty$ mientras se mantiene el resto de las
componentes fijas, cualquier punto l\'imite del proceso de
longitud de la cola normalizado $\overline{Q}^{x}$ es soluci\'on
del siguiente modelo de flujo.


\begin{Def}[Definici\'on 3.3, Dai y Meyn \cite{DaiSean}]
El modelo de flujo es estable si existe un tiempo fijo $t_{0}$ tal
que $\overline{Q}\left(t\right)=0$, con $t\geq t_{0}$, para
cualquier $\overline{Q}\left(\cdot\right)\in\mathcal{Q}$ que
cumple con $|\overline{Q}\left(0\right)|=1$.
\end{Def}

\begin{Lemma}[Lema 3.1, Dai y Meyn \cite{DaiSean}]
Si el modelo de flujo definido por (\ref{Eq.MF.1})-(\ref{Eq.MF.4})
es estable, entonces el modelo de flujo retrasado es tambi\'en
estable, es decir, existe $t_{0}>0$ tal que
$\overline{Q}\left(t\right)=0$ para cualquier $t\geq t_{0}$, para
cualquier soluci\'on del modelo de flujo retrasado cuya
condici\'on inicial $\overline{x}$ satisface que
$|\overline{x}|=|\overline{Q}\left(0\right)|+|\overline{A}\left(0\right)|+|\overline{B}\left(0\right)|\leq1$.
\end{Lemma}


Ahora ya estamos en condiciones de enunciar los resultados principales:


\begin{Teo}[Teorema 2.1, Down \cite{Down}]\label{Tma2.1.Down}
Suponga que el modelo de flujo es estable, y que se cumplen los supuestos (A1) y (A2), entonces
\begin{itemize}
\item[i)] Para alguna constante $\kappa_{p}$, y para cada
condici\'on inicial $x\in X$
\begin{equation}\label{Estability.Eq1}
\limsup_{t\rightarrow\infty}\frac{1}{t}\int_{0}^{t}\esp_{x}\left[|Q\left(s\right)|^{p}\right]ds\leq\kappa_{p},
\end{equation}
donde $p$ es el entero dado en (A2).
\end{itemize}
Si adem\'as se cumple la condici\'on (A3), entonces para cada
condici\'on inicial:
\begin{itemize}
\item[ii)] Los momentos transitorios convergen a su estado
estacionario:
 \begin{equation}\label{Estability.Eq2}
lim_{t\rightarrow\infty}\esp_{x}\left[Q_{k}\left(t\right)^{r}\right]=\esp_{\pi}\left[Q_{k}\left(0\right)^{r}\right]\leq\kappa_{r},
\end{equation}
para $r=1,2,\ldots,p$ y $k=1,2,\ldots,K$. Donde $\pi$ es la
probabilidad invariante para $X$.

\item[iii)]  El primer momento converge con raz\'on $t^{p-1}$:
\begin{equation}\label{Estability.Eq3}
lim_{t\rightarrow\infty}t^{p-1}|\esp_{x}\left[Q_{k}\left(t\right)\right]-\esp_{\pi}\left[Q_{k}\left(0\right)\right]|=0.
\end{equation}

\item[iv)] La {\em Ley Fuerte de los grandes n\'umeros} se cumple:
\begin{equation}\label{Estability.Eq4}
lim_{t\rightarrow\infty}\frac{1}{t}\int_{0}^{t}Q_{k}^{r}\left(s\right)ds=\esp_{\pi}\left[Q_{k}\left(0\right)^{r}\right],\textrm{
}\prob_{x}\textrm{-c.s.}
\end{equation}
para $r=1,2,\ldots,p$ y $k=1,2,\ldots,K$.
\end{itemize}
\end{Teo}

La contribuci\'on de Down a la teor\'ia de los {\emph {sistemas de
visitas c\'iclicas}}, es la relaci\'on que hay entre la
estabilidad del sistema con el comportamiento de las medidas de
desempe\~no, es decir, la condici\'on suficiente para poder
garantizar la convergencia del proceso de la longitud de la cola
as\'i como de por los menos los dos primeros momentos adem\'as de
una versi\'on de la Ley Fuerte de los Grandes N\'umeros para los
sistemas de visitas.


\begin{Teo}[Teorema 2.3, Down \cite{Down}]\label{Tma2.3.Down}
Considere el siguiente valor:
\begin{equation}\label{Eq.Rho.1serv}
\rho=\sum_{k=1}^{K}\rho_{k}+max_{1\leq j\leq K}\left(\frac{\lambda_{j}}{\sum_{s=1}^{S}p_{js}\overline{N}_{s}}\right)\delta^{*}
\end{equation}
\begin{itemize}
\item[i)] Si $\rho<1$ entonces la red es estable, es decir, se
cumple el Teorema \ref{Tma2.1.Down}.

\item[ii)] Si $\rho>1$ entonces la red es inestable, es decir, se
cumple el Teorema \ref{Tma2.2.Down}
\end{itemize}
\end{Teo}



\section{Fuera de tema}

\begin{Def}%\label{Def.Tn}
Sean $0\leq T_{1}\leq T_{2}\leq \ldots$ son tiempos aleatorios infinitos en los cuales ocurren ciertos eventos. El n\'umero de tiempos $T_{n}$ en el intervalo $\left[0,t\right)$ es

\begin{eqnarray}
N\left(t\right)=\sum_{n=1}^{\infty}\indora\left(T_{n}\leq t\right),
\end{eqnarray}
para $t\geq0$.
\end{Def}

Si se consideran los puntos $T_{n}$ como elementos de $\rea_{+}$, y $N\left(t\right)$ es el n\'umero de puntos en $\rea$. El proceso denotado por $\left\{N\left(t\right):t\geq0\right\}$, denotado por $N\left(t\right)$, es un proceso puntual en $\rea_{+}$. Los $T_{n}$ son los tiempos de ocurrencia, el proceso puntual $N\left(t\right)$ es simple si su n\'umero de ocurrencias son distintas: $0<T_{1}<T_{2}<\ldots$ casi seguramente.

\begin{Def}
Un proceso puntual $N\left(t\right)$ es un proceso de renovaci\'on si los tiempos de interocurrencia $\xi_{n}=T_{n}-T_{n-1}$, para $n\geq1$, son independientes e identicamente distribuidos con distribuci\'on $F$, donde $F\left(0\right)=0$ y $T_{0}=0$. Los $T_{n}$ son llamados tiempos de renovaci\'on, referente a la independencia o renovaci\'on de la informaci\'on estoc\'astica en estos tiempos. Los $\xi_{n}$ son los tiempos de inter-renovaci\'on, y $N\left(t\right)$ es el n\'umero de renovaciones en el intervalo $\left[0,t\right)$
\end{Def}


\begin{Note}
Para definir un proceso de renovaci\'on para cualquier contexto, solamente hay que especificar una distribuci\'on $F$, con $F\left(0\right)=0$, para los tiempos de inter-renovaci\'on. La funci\'on $F$ en turno degune las otra variables aleatorias. De manera formal, existe un espacio de probabilidad y una sucesi\'on de variables aleatorias $\xi_{1},\xi_{2},\ldots$ definidas en este con distribuci\'on $F$. Entonces las otras cantidades son $T_{n}=\sum_{k=1}^{n}\xi_{k}$ y $N\left(t\right)=\sum_{n=1}^{\infty}\indora\left(T_{n}\leq t\right)$, donde $T_{n}\rightarrow\infty$ casi seguramente por la Ley Fuerte de los Grandes Números.
\end{Note}

%___________________________________________________________________________________________
%
\subsection{Teorema Principal de Renovaci\'on}
%___________________________________________________________________________________________
%

\begin{Note} Una funci\'on $h:\rea_{+}\rightarrow\rea$ es Directamente Riemann Integrable en los siguientes casos:
\begin{itemize}
\item[a)] $h\left(t\right)\geq0$ es decreciente y Riemann Integrable.
\item[b)] $h$ es continua excepto posiblemente en un conjunto de Lebesgue de medida 0, y $|h\left(t\right)|\leq b\left(t\right)$, donde $b$ es DRI.
\end{itemize}
\end{Note}

\begin{Teo}[Teorema Principal de Renovaci\'on]
Si $F$ es no aritm\'etica y $h\left(t\right)$ es Directamente Riemann Integrable (DRI), entonces

\begin{eqnarray*}
lim_{t\rightarrow\infty}U\star h=\frac{1}{\mu}\int_{\rea_{+}}h\left(s\right)ds.
\end{eqnarray*}
\end{Teo}

\begin{Prop}
Cualquier funci\'on $H\left(t\right)$ acotada en intervalos finitos y que es 0 para $t<0$ puede expresarse como
\begin{eqnarray*}
H\left(t\right)=U\star h\left(t\right)\textrm{,  donde }h\left(t\right)=H\left(t\right)-F\star H\left(t\right)
\end{eqnarray*}
\end{Prop}

\begin{Def}
Un proceso estoc\'astico $X\left(t\right)$ es crudamente regenerativo en un tiempo aleatorio positivo $T$ si
\begin{eqnarray*}
\esp\left[X\left(T+t\right)|T\right]=\esp\left[X\left(t\right)\right]\textrm{, para }t\geq0,\end{eqnarray*}
y con las esperanzas anteriores finitas.
\end{Def}

\begin{Prop}
Sup\'ongase que $X\left(t\right)$ es un proceso crudamente regenerativo en $T$, que tiene distribuci\'on $F$. Si $\esp\left[X\left(t\right)\right]$ es acotado en intervalos finitos, entonces
\begin{eqnarray*}
\esp\left[X\left(t\right)\right]=U\star h\left(t\right)\textrm{,  donde }h\left(t\right)=\esp\left[X\left(t\right)\indora\left(T>t\right)\right].
\end{eqnarray*}
\end{Prop}

\begin{Teo}[Regeneraci\'on Cruda]
Sup\'ongase que $X\left(t\right)$ es un proceso con valores positivo crudamente regenerativo en $T$, y def\'inase $M=\sup\left\{|X\left(t\right)|:t\leq T\right\}$. Si $T$ es no aritm\'etico y $M$ y $MT$ tienen media finita, entonces
\begin{eqnarray*}
lim_{t\rightarrow\infty}\esp\left[X\left(t\right)\right]=\frac{1}{\mu}\int_{\rea_{+}}h\left(s\right)ds,
\end{eqnarray*}
donde $h\left(t\right)=\esp\left[X\left(t\right)\indora\left(T>t\right)\right]$.
\end{Teo}

%___________________________________________________________________________________________
%
\subsection{Propiedades de los Procesos de Renovaci\'on}
%___________________________________________________________________________________________
%

Los tiempos $T_{n}$ est\'an relacionados con los conteos de $N\left(t\right)$ por

\begin{eqnarray*}
\left\{N\left(t\right)\geq n\right\}&=&\left\{T_{n}\leq t\right\}\\
T_{N\left(t\right)}\leq &t&<T_{N\left(t\right)+1},
\end{eqnarray*}

adem\'as $N\left(T_{n}\right)=n$, y 

\begin{eqnarray*}
N\left(t\right)=\max\left\{n:T_{n}\leq t\right\}=\min\left\{n:T_{n+1}>t\right\}
\end{eqnarray*}

Por propiedades de la convoluci\'on se sabe que

\begin{eqnarray*}
P\left\{T_{n}\leq t\right\}=F^{n\star}\left(t\right)
\end{eqnarray*}
que es la $n$-\'esima convoluci\'on de $F$. Entonces 

\begin{eqnarray*}
\left\{N\left(t\right)\geq n\right\}&=&\left\{T_{n}\leq t\right\}\\
P\left\{N\left(t\right)\leq n\right\}&=&1-F^{\left(n+1\right)\star}\left(t\right)
\end{eqnarray*}

Adem\'as usando el hecho de que $\esp\left[N\left(t\right)\right]=\sum_{n=1}^{\infty}P\left\{N\left(t\right)\geq n\right\}$
se tiene que

\begin{eqnarray*}
\esp\left[N\left(t\right)\right]=\sum_{n=1}^{\infty}F^{n\star}\left(t\right)
\end{eqnarray*}

\begin{Prop}
Para cada $t\geq0$, la funci\'on generadora de momentos $\esp\left[e^{\alpha N\left(t\right)}\right]$ existe para alguna $\alpha$ en una vecindad del 0, y de aqu\'i que $\esp\left[N\left(t\right)^{m}\right]<\infty$, para $m\geq1$.
\end{Prop}


\begin{Note}
Si el primer tiempo de renovaci\'on $\xi_{1}$ no tiene la misma distribuci\'on que el resto de las $\xi_{n}$, para $n\geq2$, a $N\left(t\right)$ se le llama Proceso de Renovaci\'on retardado, donde si $\xi$ tiene distribuci\'on $G$, entonces el tiempo $T_{n}$ de la $n$-\'esima renovaci\'on tiene distribuci\'on $G\star F^{\left(n-1\right)\star}\left(t\right)$
\end{Note}


\begin{Teo}
Para una constante $\mu\leq\infty$ ( o variable aleatoria), las siguientes expresiones son equivalentes:

\begin{eqnarray}
lim_{n\rightarrow\infty}n^{-1}T_{n}&=&\mu,\textrm{ c.s.}\\
lim_{t\rightarrow\infty}t^{-1}N\left(t\right)&=&1/\mu,\textrm{ c.s.}
\end{eqnarray}
\end{Teo}


Es decir, $T_{n}$ satisface la Ley Fuerte de los Grandes N\'umeros s\'i y s\'olo s\'i $N\left/t\right)$ la cumple.


\begin{Coro}[Ley Fuerte de los Grandes N\'umeros para Procesos de Renovaci\'on]
Si $N\left(t\right)$ es un proceso de renovaci\'on cuyos tiempos de inter-renovaci\'on tienen media $\mu\leq\infty$, entonces
\begin{eqnarray}
t^{-1}N\left(t\right)\rightarrow 1/\mu,\textrm{ c.s. cuando }t\rightarrow\infty.
\end{eqnarray}

\end{Coro}


Considerar el proceso estoc\'astico de valores reales $\left\{Z\left(t\right):t\geq0\right\}$ en el mismo espacio de probabilidad que $N\left(t\right)$

\begin{Def}
Para el proceso $\left\{Z\left(t\right):t\geq0\right\}$ se define la fluctuaci\'on m\'axima de $Z\left(t\right)$ en el intervalo $\left(T_{n-1},T_{n}\right]$:
\begin{eqnarray*}
M_{n}=\sup_{T_{n-1}<t\leq T_{n}}|Z\left(t\right)-Z\left(T_{n-1}\right)|
\end{eqnarray*}
\end{Def}

\begin{Teo}
Sup\'ongase que $n^{-1}T_{n}\rightarrow\mu$ c.s. cuando $n\rightarrow\infty$, donde $\mu\leq\infty$ es una constante o variable aleatoria. Sea $a$ una constante o variable aleatoria que puede ser infinita cuando $\mu$ es finita, y considere las expresiones l\'imite:
\begin{eqnarray}
lim_{n\rightarrow\infty}n^{-1}Z\left(T_{n}\right)&=&a,\textrm{ c.s.}\\
lim_{t\rightarrow\infty}t^{-1}Z\left(t\right)&=&a/\mu,\textrm{ c.s.}
\end{eqnarray}
La segunda expresi\'on implica la primera. Conversamente, la primera implica la segunda si el proceso $Z\left(t\right)$ es creciente, o si $lim_{n\rightarrow\infty}n^{-1}M_{n}=0$ c.s.
\end{Teo}

\begin{Coro}
Si $N\left(t\right)$ es un proceso de renovaci\'on, y $\left(Z\left(T_{n}\right)-Z\left(T_{n-1}\right),M_{n}\right)$, para $n\geq1$, son variables aleatorias independientes e id\'enticamente distribuidas con media finita, entonces,
\begin{eqnarray}
lim_{t\rightarrow\infty}t^{-1}Z\left(t\right)\rightarrow\frac{\esp\left[Z\left(T_{1}\right)-Z\left(T_{0}\right)\right]}{\esp\left[T_{1}\right]},\textrm{ c.s. cuando  }t\rightarrow\infty.
\end{eqnarray}
\end{Coro}



%______________________________________________________________________
\subsection{Procesos de Renovaci\'on}
%______________________________________________________________________

\begin{Def}%\label{Def.Tn}
Sean $0\leq T_{1}\leq T_{2}\leq \ldots$ son tiempos aleatorios infinitos en los cuales ocurren ciertos eventos. El n\'umero de tiempos $T_{n}$ en el intervalo $\left[0,t\right)$ es

\begin{eqnarray}
N\left(t\right)=\sum_{n=1}^{\infty}\indora\left(T_{n}\leq t\right),
\end{eqnarray}
para $t\geq0$.
\end{Def}

Si se consideran los puntos $T_{n}$ como elementos de $\rea_{+}$, y $N\left(t\right)$ es el n\'umero de puntos en $\rea$. El proceso denotado por $\left\{N\left(t\right):t\geq0\right\}$, denotado por $N\left(t\right)$, es un proceso puntual en $\rea_{+}$. Los $T_{n}$ son los tiempos de ocurrencia, el proceso puntual $N\left(t\right)$ es simple si su n\'umero de ocurrencias son distintas: $0<T_{1}<T_{2}<\ldots$ casi seguramente.

\begin{Def}
Un proceso puntual $N\left(t\right)$ es un proceso de renovaci\'on si los tiempos de interocurrencia $\xi_{n}=T_{n}-T_{n-1}$, para $n\geq1$, son independientes e identicamente distribuidos con distribuci\'on $F$, donde $F\left(0\right)=0$ y $T_{0}=0$. Los $T_{n}$ son llamados tiempos de renovaci\'on, referente a la independencia o renovaci\'on de la informaci\'on estoc\'astica en estos tiempos. Los $\xi_{n}$ son los tiempos de inter-renovaci\'on, y $N\left(t\right)$ es el n\'umero de renovaciones en el intervalo $\left[0,t\right)$
\end{Def}


\begin{Note}
Para definir un proceso de renovaci\'on para cualquier contexto, solamente hay que especificar una distribuci\'on $F$, con $F\left(0\right)=0$, para los tiempos de inter-renovaci\'on. La funci\'on $F$ en turno degune las otra variables aleatorias. De manera formal, existe un espacio de probabilidad y una sucesi\'on de variables aleatorias $\xi_{1},\xi_{2},\ldots$ definidas en este con distribuci\'on $F$. Entonces las otras cantidades son $T_{n}=\sum_{k=1}^{n}\xi_{k}$ y $N\left(t\right)=\sum_{n=1}^{\infty}\indora\left(T_{n}\leq t\right)$, donde $T_{n}\rightarrow\infty$ casi seguramente por la Ley Fuerte de los Grandes Números.
\end{Note}

%___________________________________________________________________________________________
%
\subsection{Renewal and Regenerative Processes: Serfozo\cite{Serfozo}}
%___________________________________________________________________________________________
%
\begin{Def}%\label{Def.Tn}
Sean $0\leq T_{1}\leq T_{2}\leq \ldots$ son tiempos aleatorios infinitos en los cuales ocurren ciertos eventos. El n\'umero de tiempos $T_{n}$ en el intervalo $\left[0,t\right)$ es

\begin{eqnarray}
N\left(t\right)=\sum_{n=1}^{\infty}\indora\left(T_{n}\leq t\right),
\end{eqnarray}
para $t\geq0$.
\end{Def}

Si se consideran los puntos $T_{n}$ como elementos de $\rea_{+}$, y $N\left(t\right)$ es el n\'umero de puntos en $\rea$. El proceso denotado por $\left\{N\left(t\right):t\geq0\right\}$, denotado por $N\left(t\right)$, es un proceso puntual en $\rea_{+}$. Los $T_{n}$ son los tiempos de ocurrencia, el proceso puntual $N\left(t\right)$ es simple si su n\'umero de ocurrencias son distintas: $0<T_{1}<T_{2}<\ldots$ casi seguramente.

\begin{Def}
Un proceso puntual $N\left(t\right)$ es un proceso de renovaci\'on si los tiempos de interocurrencia $\xi_{n}=T_{n}-T_{n-1}$, para $n\geq1$, son independientes e identicamente distribuidos con distribuci\'on $F$, donde $F\left(0\right)=0$ y $T_{0}=0$. Los $T_{n}$ son llamados tiempos de renovaci\'on, referente a la independencia o renovaci\'on de la informaci\'on estoc\'astica en estos tiempos. Los $\xi_{n}$ son los tiempos de inter-renovaci\'on, y $N\left(t\right)$ es el n\'umero de renovaciones en el intervalo $\left[0,t\right)$
\end{Def}


\begin{Note}
Para definir un proceso de renovaci\'on para cualquier contexto, solamente hay que especificar una distribuci\'on $F$, con $F\left(0\right)=0$, para los tiempos de inter-renovaci\'on. La funci\'on $F$ en turno degune las otra variables aleatorias. De manera formal, existe un espacio de probabilidad y una sucesi\'on de variables aleatorias $\xi_{1},\xi_{2},\ldots$ definidas en este con distribuci\'on $F$. Entonces las otras cantidades son $T_{n}=\sum_{k=1}^{n}\xi_{k}$ y $N\left(t\right)=\sum_{n=1}^{\infty}\indora\left(T_{n}\leq t\right)$, donde $T_{n}\rightarrow\infty$ casi seguramente por la Ley Fuerte de los Grandes N\'umeros.
\end{Note}







Los tiempos $T_{n}$ est\'an relacionados con los conteos de $N\left(t\right)$ por

\begin{eqnarray*}
\left\{N\left(t\right)\geq n\right\}&=&\left\{T_{n}\leq t\right\}\\
T_{N\left(t\right)}\leq &t&<T_{N\left(t\right)+1},
\end{eqnarray*}

adem\'as $N\left(T_{n}\right)=n$, y 

\begin{eqnarray*}
N\left(t\right)=\max\left\{n:T_{n}\leq t\right\}=\min\left\{n:T_{n+1}>t\right\}
\end{eqnarray*}

Por propiedades de la convoluci\'on se sabe que

\begin{eqnarray*}
P\left\{T_{n}\leq t\right\}=F^{n\star}\left(t\right)
\end{eqnarray*}
que es la $n$-\'esima convoluci\'on de $F$. Entonces 

\begin{eqnarray*}
\left\{N\left(t\right)\geq n\right\}&=&\left\{T_{n}\leq t\right\}\\
P\left\{N\left(t\right)\leq n\right\}&=&1-F^{\left(n+1\right)\star}\left(t\right)
\end{eqnarray*}

Adem\'as usando el hecho de que $\esp\left[N\left(t\right)\right]=\sum_{n=1}^{\infty}P\left\{N\left(t\right)\geq n\right\}$
se tiene que

\begin{eqnarray*}
\esp\left[N\left(t\right)\right]=\sum_{n=1}^{\infty}F^{n\star}\left(t\right)
\end{eqnarray*}

\begin{Prop}
Para cada $t\geq0$, la funci\'on generadora de momentos $\esp\left[e^{\alpha N\left(t\right)}\right]$ existe para alguna $\alpha$ en una vecindad del 0, y de aqu\'i que $\esp\left[N\left(t\right)^{m}\right]<\infty$, para $m\geq1$.
\end{Prop}

\begin{Ejem}[\textbf{Proceso Poisson}]

Suponga que se tienen tiempos de inter-renovaci\'on \textit{i.i.d.} del proceso de renovaci\'on $N\left(t\right)$ tienen distribuci\'on exponencial $F\left(t\right)=q-e^{-\lambda t}$ con tasa $\lambda$. Entonces $N\left(t\right)$ es un proceso Poisson con tasa $\lambda$.

\end{Ejem}


\begin{Note}
Si el primer tiempo de renovaci\'on $\xi_{1}$ no tiene la misma distribuci\'on que el resto de las $\xi_{n}$, para $n\geq2$, a $N\left(t\right)$ se le llama Proceso de Renovaci\'on retardado, donde si $\xi$ tiene distribuci\'on $G$, entonces el tiempo $T_{n}$ de la $n$-\'esima renovaci\'on tiene distribuci\'on $G\star F^{\left(n-1\right)\star}\left(t\right)$
\end{Note}


\begin{Teo}
Para una constante $\mu\leq\infty$ ( o variable aleatoria), las siguientes expresiones son equivalentes:

\begin{eqnarray}
lim_{n\rightarrow\infty}n^{-1}T_{n}&=&\mu,\textrm{ c.s.}\\
lim_{t\rightarrow\infty}t^{-1}N\left(t\right)&=&1/\mu,\textrm{ c.s.}
\end{eqnarray}
\end{Teo}


Es decir, $T_{n}$ satisface la Ley Fuerte de los Grandes N\'umeros s\'i y s\'olo s\'i $N\left/t\right)$ la cumple.


\begin{Coro}[Ley Fuerte de los Grandes N\'umeros para Procesos de Renovaci\'on]
Si $N\left(t\right)$ es un proceso de renovaci\'on cuyos tiempos de inter-renovaci\'on tienen media $\mu\leq\infty$, entonces
\begin{eqnarray}
t^{-1}N\left(t\right)\rightarrow 1/\mu,\textrm{ c.s. cuando }t\rightarrow\infty.
\end{eqnarray}

\end{Coro}


Considerar el proceso estoc\'astico de valores reales $\left\{Z\left(t\right):t\geq0\right\}$ en el mismo espacio de probabilidad que $N\left(t\right)$

\begin{Def}
Para el proceso $\left\{Z\left(t\right):t\geq0\right\}$ se define la fluctuaci\'on m\'axima de $Z\left(t\right)$ en el intervalo $\left(T_{n-1},T_{n}\right]$:
\begin{eqnarray*}
M_{n}=\sup_{T_{n-1}<t\leq T_{n}}|Z\left(t\right)-Z\left(T_{n-1}\right)|
\end{eqnarray*}
\end{Def}

\begin{Teo}
Sup\'ongase que $n^{-1}T_{n}\rightarrow\mu$ c.s. cuando $n\rightarrow\infty$, donde $\mu\leq\infty$ es una constante o variable aleatoria. Sea $a$ una constante o variable aleatoria que puede ser infinita cuando $\mu$ es finita, y considere las expresiones l\'imite:
\begin{eqnarray}
lim_{n\rightarrow\infty}n^{-1}Z\left(T_{n}\right)&=&a,\textrm{ c.s.}\\
lim_{t\rightarrow\infty}t^{-1}Z\left(t\right)&=&a/\mu,\textrm{ c.s.}
\end{eqnarray}
La segunda expresi\'on implica la primera. Conversamente, la primera implica la segunda si el proceso $Z\left(t\right)$ es creciente, o si $lim_{n\rightarrow\infty}n^{-1}M_{n}=0$ c.s.
\end{Teo}

\begin{Coro}
Si $N\left(t\right)$ es un proceso de renovaci\'on, y $\left(Z\left(T_{n}\right)-Z\left(T_{n-1}\right),M_{n}\right)$, para $n\geq1$, son variables aleatorias independientes e id\'enticamente distribuidas con media finita, entonces,
\begin{eqnarray}
lim_{t\rightarrow\infty}t^{-1}Z\left(t\right)\rightarrow\frac{\esp\left[Z\left(T_{1}\right)-Z\left(T_{0}\right)\right]}{\esp\left[T_{1}\right]},\textrm{ c.s. cuando  }t\rightarrow\infty.
\end{eqnarray}
\end{Coro}


Sup\'ongase que $N\left(t\right)$ es un proceso de renovaci\'on con distribuci\'on $F$ con media finita $\mu$.

\begin{Def}
La funci\'on de renovaci\'on asociada con la distribuci\'on $F$, del proceso $N\left(t\right)$, es
\begin{eqnarray*}
U\left(t\right)=\sum_{n=1}^{\infty}F^{n\star}\left(t\right),\textrm{   }t\geq0,
\end{eqnarray*}
donde $F^{0\star}\left(t\right)=\indora\left(t\geq0\right)$.
\end{Def}


\begin{Prop}
Sup\'ongase que la distribuci\'on de inter-renovaci\'on $F$ tiene densidad $f$. Entonces $U\left(t\right)$ tambi\'en tiene densidad, para $t>0$, y es $U^{'}\left(t\right)=\sum_{n=0}^{\infty}f^{n\star}\left(t\right)$. Adem\'as
\begin{eqnarray*}
\prob\left\{N\left(t\right)>N\left(t-\right)\right\}=0\textrm{,   }t\geq0.
\end{eqnarray*}
\end{Prop}

\begin{Def}
La Transformada de Laplace-Stieljes de $F$ est\'a dada por

\begin{eqnarray*}
\hat{F}\left(\alpha\right)=\int_{\rea_{+}}e^{-\alpha t}dF\left(t\right)\textrm{,  }\alpha\geq0.
\end{eqnarray*}
\end{Def}

Entonces

\begin{eqnarray*}
\hat{U}\left(\alpha\right)=\sum_{n=0}^{\infty}\hat{F^{n\star}}\left(\alpha\right)=\sum_{n=0}^{\infty}\hat{F}\left(\alpha\right)^{n}=\frac{1}{1-\hat{F}\left(\alpha\right)}.
\end{eqnarray*}


\begin{Prop}
La Transformada de Laplace $\hat{U}\left(\alpha\right)$ y $\hat{F}\left(\alpha\right)$ determina una a la otra de manera \'unica por la relaci\'on $\hat{U}\left(\alpha\right)=\frac{1}{1-\hat{F}\left(\alpha\right)}$.
\end{Prop}


\begin{Note}
Un proceso de renovaci\'on $N\left(t\right)$ cuyos tiempos de inter-renovaci\'on tienen media finita, es un proceso Poisson con tasa $\lambda$ si y s\'olo s\'i $\esp\left[U\left(t\right)\right]=\lambda t$, para $t\geq0$.
\end{Note}


\begin{Teo}
Sea $N\left(t\right)$ un proceso puntual simple con puntos de localizaci\'on $T_{n}$ tal que $\eta\left(t\right)=\esp\left[N\left(\right)\right]$ es finita para cada $t$. Entonces para cualquier funci\'on $f:\rea_{+}\rightarrow\rea$,
\begin{eqnarray*}
\esp\left[\sum_{n=1}^{N\left(\right)}f\left(T_{n}\right)\right]=\int_{\left(0,t\right]}f\left(s\right)d\eta\left(s\right)\textrm{,  }t\geq0,
\end{eqnarray*}
suponiendo que la integral exista. Adem\'as si $X_{1},X_{2},\ldots$ son variables aleatorias definidas en el mismo espacio de probabilidad que el proceso $N\left(t\right)$ tal que $\esp\left[X_{n}|T_{n}=s\right]=f\left(s\right)$, independiente de $n$. Entonces
\begin{eqnarray*}
\esp\left[\sum_{n=1}^{N\left(t\right)}X_{n}\right]=\int_{\left(0,t\right]}f\left(s\right)d\eta\left(s\right)\textrm{,  }t\geq0,
\end{eqnarray*} 
suponiendo que la integral exista. 
\end{Teo}

\begin{Coro}[Identidad de Wald para Renovaciones]
Para el proceso de renovaci\'on $N\left(t\right)$,
\begin{eqnarray*}
\esp\left[T_{N\left(t\right)+1}\right]=\mu\esp\left[N\left(t\right)+1\right]\textrm{,  }t\geq0,
\end{eqnarray*}  
\end{Coro}


\begin{Def}
Sea $h\left(t\right)$ funci\'on de valores reales en $\rea$ acotada en intervalos finitos e igual a cero para $t<0$ La ecuaci\'on de renovaci\'on para $h\left(t\right)$ y la distribuci\'on $F$ es

\begin{eqnarray}%\label{Ec.Renovacion}
H\left(t\right)=h\left(t\right)+\int_{\left[0,t\right]}H\left(t-s\right)dF\left(s\right)\textrm{,    }t\geq0,
\end{eqnarray}
donde $H\left(t\right)$ es una funci\'on de valores reales. Esto es $H=h+F\star H$. Decimos que $H\left(t\right)$ es soluci\'on de esta ecuaci\'on si satisface la ecuaci\'on, y es acotada en intervalos finitos e iguales a cero para $t<0$.
\end{Def}

\begin{Prop}
La funci\'on $U\star h\left(t\right)$ es la \'unica soluci\'on de la ecuaci\'on de renovaci\'on (\ref{Ec.Renovacion}).
\end{Prop}

\begin{Teo}[Teorema Renovaci\'on Elemental]
\begin{eqnarray*}
t^{-1}U\left(t\right)\rightarrow 1/\mu\textrm{,    cuando }t\rightarrow\infty.
\end{eqnarray*}
\end{Teo}



Sup\'ongase que $N\left(t\right)$ es un proceso de renovaci\'on con distribuci\'on $F$ con media finita $\mu$.

\begin{Def}
La funci\'on de renovaci\'on asociada con la distribuci\'on $F$, del proceso $N\left(t\right)$, es
\begin{eqnarray*}
U\left(t\right)=\sum_{n=1}^{\infty}F^{n\star}\left(t\right),\textrm{   }t\geq0,
\end{eqnarray*}
donde $F^{0\star}\left(t\right)=\indora\left(t\geq0\right)$.
\end{Def}


\begin{Prop}
Sup\'ongase que la distribuci\'on de inter-renovaci\'on $F$ tiene densidad $f$. Entonces $U\left(t\right)$ tambi\'en tiene densidad, para $t>0$, y es $U^{'}\left(t\right)=\sum_{n=0}^{\infty}f^{n\star}\left(t\right)$. Adem\'as
\begin{eqnarray*}
\prob\left\{N\left(t\right)>N\left(t-\right)\right\}=0\textrm{,   }t\geq0.
\end{eqnarray*}
\end{Prop}

\begin{Def}
La Transformada de Laplace-Stieljes de $F$ est\'a dada por

\begin{eqnarray*}
\hat{F}\left(\alpha\right)=\int_{\rea_{+}}e^{-\alpha t}dF\left(t\right)\textrm{,  }\alpha\geq0.
\end{eqnarray*}
\end{Def}

Entonces

\begin{eqnarray*}
\hat{U}\left(\alpha\right)=\sum_{n=0}^{\infty}\hat{F^{n\star}}\left(\alpha\right)=\sum_{n=0}^{\infty}\hat{F}\left(\alpha\right)^{n}=\frac{1}{1-\hat{F}\left(\alpha\right)}.
\end{eqnarray*}


\begin{Prop}
La Transformada de Laplace $\hat{U}\left(\alpha\right)$ y $\hat{F}\left(\alpha\right)$ determina una a la otra de manera \'unica por la relaci\'on $\hat{U}\left(\alpha\right)=\frac{1}{1-\hat{F}\left(\alpha\right)}$.
\end{Prop}


\begin{Note}
Un proceso de renovaci\'on $N\left(t\right)$ cuyos tiempos de inter-renovaci\'on tienen media finita, es un proceso Poisson con tasa $\lambda$ si y s\'olo s\'i $\esp\left[U\left(t\right)\right]=\lambda t$, para $t\geq0$.
\end{Note}


\begin{Teo}
Sea $N\left(t\right)$ un proceso puntual simple con puntos de localizaci\'on $T_{n}$ tal que $\eta\left(t\right)=\esp\left[N\left(\right)\right]$ es finita para cada $t$. Entonces para cualquier funci\'on $f:\rea_{+}\rightarrow\rea$,
\begin{eqnarray*}
\esp\left[\sum_{n=1}^{N\left(\right)}f\left(T_{n}\right)\right]=\int_{\left(0,t\right]}f\left(s\right)d\eta\left(s\right)\textrm{,  }t\geq0,
\end{eqnarray*}
suponiendo que la integral exista. Adem\'as si $X_{1},X_{2},\ldots$ son variables aleatorias definidas en el mismo espacio de probabilidad que el proceso $N\left(t\right)$ tal que $\esp\left[X_{n}|T_{n}=s\right]=f\left(s\right)$, independiente de $n$. Entonces
\begin{eqnarray*}
\esp\left[\sum_{n=1}^{N\left(t\right)}X_{n}\right]=\int_{\left(0,t\right]}f\left(s\right)d\eta\left(s\right)\textrm{,  }t\geq0,
\end{eqnarray*} 
suponiendo que la integral exista. 
\end{Teo}

\begin{Coro}[Identidad de Wald para Renovaciones]
Para el proceso de renovaci\'on $N\left(t\right)$,
\begin{eqnarray*}
\esp\left[T_{N\left(t\right)+1}\right]=\mu\esp\left[N\left(t\right)+1\right]\textrm{,  }t\geq0,
\end{eqnarray*}  
\end{Coro}


\begin{Def}
Sea $h\left(t\right)$ funci\'on de valores reales en $\rea$ acotada en intervalos finitos e igual a cero para $t<0$ La ecuaci\'on de renovaci\'on para $h\left(t\right)$ y la distribuci\'on $F$ es

\begin{eqnarray}%\label{Ec.Renovacion}
H\left(t\right)=h\left(t\right)+\int_{\left[0,t\right]}H\left(t-s\right)dF\left(s\right)\textrm{,    }t\geq0,
\end{eqnarray}
donde $H\left(t\right)$ es una funci\'on de valores reales. Esto es $H=h+F\star H$. Decimos que $H\left(t\right)$ es soluci\'on de esta ecuaci\'on si satisface la ecuaci\'on, y es acotada en intervalos finitos e iguales a cero para $t<0$.
\end{Def}

\begin{Prop}
La funci\'on $U\star h\left(t\right)$ es la \'unica soluci\'on de la ecuaci\'on de renovaci\'on (\ref{Ec.Renovacion}).
\end{Prop}

\begin{Teo}[Teorema Renovaci\'on Elemental]
\begin{eqnarray*}
t^{-1}U\left(t\right)\rightarrow 1/\mu\textrm{,    cuando }t\rightarrow\infty.
\end{eqnarray*}
\end{Teo}


\begin{Note} Una funci\'on $h:\rea_{+}\rightarrow\rea$ es Directamente Riemann Integrable en los siguientes casos:
\begin{itemize}
\item[a)] $h\left(t\right)\geq0$ es decreciente y Riemann Integrable.
\item[b)] $h$ es continua excepto posiblemente en un conjunto de Lebesgue de medida 0, y $|h\left(t\right)|\leq b\left(t\right)$, donde $b$ es DRI.
\end{itemize}
\end{Note}

\begin{Teo}[Teorema Principal de Renovaci\'on]
Si $F$ es no aritm\'etica y $h\left(t\right)$ es Directamente Riemann Integrable (DRI), entonces

\begin{eqnarray*}
lim_{t\rightarrow\infty}U\star h=\frac{1}{\mu}\int_{\rea_{+}}h\left(s\right)ds.
\end{eqnarray*}
\end{Teo}

\begin{Prop}
Cualquier funci\'on $H\left(t\right)$ acotada en intervalos finitos y que es 0 para $t<0$ puede expresarse como
\begin{eqnarray*}
H\left(t\right)=U\star h\left(t\right)\textrm{,  donde }h\left(t\right)=H\left(t\right)-F\star H\left(t\right)
\end{eqnarray*}
\end{Prop}

\begin{Def}
Un proceso estoc\'astico $X\left(t\right)$ es crudamente regenerativo en un tiempo aleatorio positivo $T$ si
\begin{eqnarray*}
\esp\left[X\left(T+t\right)|T\right]=\esp\left[X\left(t\right)\right]\textrm{, para }t\geq0,\end{eqnarray*}
y con las esperanzas anteriores finitas.
\end{Def}

\begin{Prop}
Sup\'ongase que $X\left(t\right)$ es un proceso crudamente regenerativo en $T$, que tiene distribuci\'on $F$. Si $\esp\left[X\left(t\right)\right]$ es acotado en intervalos finitos, entonces
\begin{eqnarray*}
\esp\left[X\left(t\right)\right]=U\star h\left(t\right)\textrm{,  donde }h\left(t\right)=\esp\left[X\left(t\right)\indora\left(T>t\right)\right].
\end{eqnarray*}
\end{Prop}

\begin{Teo}[Regeneraci\'on Cruda]
Sup\'ongase que $X\left(t\right)$ es un proceso con valores positivo crudamente regenerativo en $T$, y def\'inase $M=\sup\left\{|X\left(t\right)|:t\leq T\right\}$. Si $T$ es no aritm\'etico y $M$ y $MT$ tienen media finita, entonces
\begin{eqnarray*}
lim_{t\rightarrow\infty}\esp\left[X\left(t\right)\right]=\frac{1}{\mu}\int_{\rea_{+}}h\left(s\right)ds,
\end{eqnarray*}
donde $h\left(t\right)=\esp\left[X\left(t\right)\indora\left(T>t\right)\right]$.
\end{Teo}


\begin{Note} Una funci\'on $h:\rea_{+}\rightarrow\rea$ es Directamente Riemann Integrable en los siguientes casos:
\begin{itemize}
\item[a)] $h\left(t\right)\geq0$ es decreciente y Riemann Integrable.
\item[b)] $h$ es continua excepto posiblemente en un conjunto de Lebesgue de medida 0, y $|h\left(t\right)|\leq b\left(t\right)$, donde $b$ es DRI.
\end{itemize}
\end{Note}

\begin{Teo}[Teorema Principal de Renovaci\'on]
Si $F$ es no aritm\'etica y $h\left(t\right)$ es Directamente Riemann Integrable (DRI), entonces

\begin{eqnarray*}
lim_{t\rightarrow\infty}U\star h=\frac{1}{\mu}\int_{\rea_{+}}h\left(s\right)ds.
\end{eqnarray*}
\end{Teo}

\begin{Prop}
Cualquier funci\'on $H\left(t\right)$ acotada en intervalos finitos y que es 0 para $t<0$ puede expresarse como
\begin{eqnarray*}
H\left(t\right)=U\star h\left(t\right)\textrm{,  donde }h\left(t\right)=H\left(t\right)-F\star H\left(t\right)
\end{eqnarray*}
\end{Prop}

\begin{Def}
Un proceso estoc\'astico $X\left(t\right)$ es crudamente regenerativo en un tiempo aleatorio positivo $T$ si
\begin{eqnarray*}
\esp\left[X\left(T+t\right)|T\right]=\esp\left[X\left(t\right)\right]\textrm{, para }t\geq0,\end{eqnarray*}
y con las esperanzas anteriores finitas.
\end{Def}

\begin{Prop}
Sup\'ongase que $X\left(t\right)$ es un proceso crudamente regenerativo en $T$, que tiene distribuci\'on $F$. Si $\esp\left[X\left(t\right)\right]$ es acotado en intervalos finitos, entonces
\begin{eqnarray*}
\esp\left[X\left(t\right)\right]=U\star h\left(t\right)\textrm{,  donde }h\left(t\right)=\esp\left[X\left(t\right)\indora\left(T>t\right)\right].
\end{eqnarray*}
\end{Prop}

\begin{Teo}[Regeneraci\'on Cruda]
Sup\'ongase que $X\left(t\right)$ es un proceso con valores positivo crudamente regenerativo en $T$, y def\'inase $M=\sup\left\{|X\left(t\right)|:t\leq T\right\}$. Si $T$ es no aritm\'etico y $M$ y $MT$ tienen media finita, entonces
\begin{eqnarray*}
lim_{t\rightarrow\infty}\esp\left[X\left(t\right)\right]=\frac{1}{\mu}\int_{\rea_{+}}h\left(s\right)ds,
\end{eqnarray*}
donde $h\left(t\right)=\esp\left[X\left(t\right)\indora\left(T>t\right)\right]$.
\end{Teo}

\begin{Def}
Para el proceso $\left\{\left(N\left(t\right),X\left(t\right)\right):t\geq0\right\}$, sus trayectoria muestrales en el intervalo de tiempo $\left[T_{n-1},T_{n}\right)$ est\'an descritas por
\begin{eqnarray*}
\zeta_{n}=\left(\xi_{n},\left\{X\left(T_{n-1}+t\right):0\leq t<\xi_{n}\right\}\right)
\end{eqnarray*}
Este $\zeta_{n}$ es el $n$-\'esimo segmento del proceso. El proceso es regenerativo sobre los tiempos $T_{n}$ si sus segmentos $\zeta_{n}$ son independientes e id\'enticamennte distribuidos.
\end{Def}


\begin{Note}
Si $\tilde{X}\left(t\right)$ con espacio de estados $\tilde{S}$ es regenerativo sobre $T_{n}$, entonces $X\left(t\right)=f\left(\tilde{X}\left(t\right)\right)$ tambi\'en es regenerativo sobre $T_{n}$, para cualquier funci\'on $f:\tilde{S}\rightarrow S$.
\end{Note}

\begin{Note}
Los procesos regenerativos son crudamente regenerativos, pero no al rev\'es.
\end{Note}


\begin{Note}
Un proceso estoc\'astico a tiempo continuo o discreto es regenerativo si existe un proceso de renovaci\'on  tal que los segmentos del proceso entre tiempos de renovaci\'on sucesivos son i.i.d., es decir, para $\left\{X\left(t\right):t\geq0\right\}$ proceso estoc\'astico a tiempo continuo con espacio de estados $S$, espacio m\'etrico.
\end{Note}

Para $\left\{X\left(t\right):t\geq0\right\}$ Proceso Estoc\'astico a tiempo continuo con estado de espacios $S$, que es un espacio m\'etrico, con trayectorias continuas por la derecha y con l\'imites por la izquierda c.s. Sea $N\left(t\right)$ un proceso de renovaci\'on en $\rea_{+}$ definido en el mismo espacio de probabilidad que $X\left(t\right)$, con tiempos de renovaci\'on $T$ y tiempos de inter-renovaci\'on $\xi_{n}=T_{n}-T_{n-1}$, con misma distribuci\'on $F$ de media finita $\mu$.



\begin{Def}
Para el proceso $\left\{\left(N\left(t\right),X\left(t\right)\right):t\geq0\right\}$, sus trayectoria muestrales en el intervalo de tiempo $\left[T_{n-1},T_{n}\right)$ est\'an descritas por
\begin{eqnarray*}
\zeta_{n}=\left(\xi_{n},\left\{X\left(T_{n-1}+t\right):0\leq t<\xi_{n}\right\}\right)
\end{eqnarray*}
Este $\zeta_{n}$ es el $n$-\'esimo segmento del proceso. El proceso es regenerativo sobre los tiempos $T_{n}$ si sus segmentos $\zeta_{n}$ son independientes e id\'enticamennte distribuidos.
\end{Def}

\begin{Note}
Un proceso regenerativo con media de la longitud de ciclo finita es llamado positivo recurrente.
\end{Note}

\begin{Teo}[Procesos Regenerativos]
Suponga que el proceso
\end{Teo}


\begin{Def}[Renewal Process Trinity]
Para un proceso de renovaci\'on $N\left(t\right)$, los siguientes procesos proveen de informaci\'on sobre los tiempos de renovaci\'on.
\begin{itemize}
\item $A\left(t\right)=t-T_{N\left(t\right)}$, el tiempo de recurrencia hacia atr\'as al tiempo $t$, que es el tiempo desde la \'ultima renovaci\'on para $t$.

\item $B\left(t\right)=T_{N\left(t\right)+1}-t$, el tiempo de recurrencia hacia adelante al tiempo $t$, residual del tiempo de renovaci\'on, que es el tiempo para la pr\'oxima renovaci\'on despu\'es de $t$.

\item $L\left(t\right)=\xi_{N\left(t\right)+1}=A\left(t\right)+B\left(t\right)$, la longitud del intervalo de renovaci\'on que contiene a $t$.
\end{itemize}
\end{Def}

\begin{Note}
El proceso tridimensional $\left(A\left(t\right),B\left(t\right),L\left(t\right)\right)$ es regenerativo sobre $T_{n}$, y por ende cada proceso lo es. Cada proceso $A\left(t\right)$ y $B\left(t\right)$ son procesos de MArkov a tiempo continuo con trayectorias continuas por partes en el espacio de estados $\rea_{+}$. Una expresi\'on conveniente para su distribuci\'on conjunta es, para $0\leq x<t,y\geq0$
\begin{equation}\label{NoRenovacion}
P\left\{A\left(t\right)>x,B\left(t\right)>y\right\}=
P\left\{N\left(t+y\right)-N\left((t-x)\right)=0\right\}
\end{equation}
\end{Note}

\begin{Ejem}[Tiempos de recurrencia Poisson]
Si $N\left(t\right)$ es un proceso Poisson con tasa $\lambda$, entonces de la expresi\'on (\ref{NoRenovacion}) se tiene que

\begin{eqnarray*}
\begin{array}{lc}
P\left\{A\left(t\right)>x,B\left(t\right)>y\right\}=e^{-\lambda\left(x+y\right)},&0\leq x<t,y\geq0,
\end{array}
\end{eqnarray*}
que es la probabilidad Poisson de no renovaciones en un intervalo de longitud $x+y$.

\end{Ejem}

%\begin{Note}
Una cadena de Markov erg\'odica tiene la propiedad de ser estacionaria si la distribuci\'on de su estado al tiempo $0$ es su distribuci\'on estacionaria.
%\end{Note}


\begin{Def}
Un proceso estoc\'astico a tiempo continuo $\left\{X\left(t\right):t\geq0\right\}$ en un espacio general es estacionario si sus distribuciones finito dimensionales son invariantes bajo cualquier  traslado: para cada $0\leq s_{1}<s_{2}<\cdots<s_{k}$ y $t\geq0$,
\begin{eqnarray*}
\left(X\left(s_{1}+t\right),\ldots,X\left(s_{k}+t\right)\right)=_{d}\left(X\left(s_{1}\right),\ldots,X\left(s_{k}\right)\right).
\end{eqnarray*}
\end{Def}

\begin{Note}
Un proceso de Markov es estacionario si $X\left(t\right)=_{d}X\left(0\right)$, $t\geq0$.
\end{Note}

Considerese el proceso $N\left(t\right)=\sum_{n}\indora\left(\tau_{n}\leq t\right)$ en $\rea_{+}$, con puntos $0<\tau_{1}<\tau_{2}<\cdots$.

\begin{Prop}
Si $N$ es un proceso puntual estacionario y $\esp\left[N\left(1\right)\right]<\infty$, entonces $\esp\left[N\left(t\right)\right]=t\esp\left[N\left(1\right)\right]$, $t\geq0$

\end{Prop}

\begin{Teo}
Los siguientes enunciados son equivalentes
\begin{itemize}
\item[i)] El proceso retardado de renovaci\'on $N$ es estacionario.

\item[ii)] EL proceso de tiempos de recurrencia hacia adelante $B\left(t\right)$ es estacionario.


\item[iii)] $\esp\left[N\left(t\right)\right]=t/\mu$,


\item[iv)] $G\left(t\right)=F_{e}\left(t\right)=\frac{1}{\mu}\int_{0}^{t}\left[1-F\left(s\right)\right]ds$
\end{itemize}
Cuando estos enunciados son ciertos, $P\left\{B\left(t\right)\leq x\right\}=F_{e}\left(x\right)$, para $t,x\geq0$.

\end{Teo}

\begin{Note}
Una consecuencia del teorema anterior es que el Proceso Poisson es el \'unico proceso sin retardo que es estacionario.
\end{Note}

\begin{Coro}
El proceso de renovaci\'on $N\left(t\right)$ sin retardo, y cuyos tiempos de inter renonaci\'on tienen media finita, es estacionario si y s\'olo si es un proceso Poisson.

\end{Coro}





%___________________________________________________________________________________________
%
\subsection{Renewal and Regenerative Processes: Serfozo\cite{Serfozo}}
%___________________________________________________________________________________________
%
\begin{Def}%\label{Def.Tn}
Sean $0\leq T_{1}\leq T_{2}\leq \ldots$ son tiempos aleatorios infinitos en los cuales ocurren ciertos eventos. El n\'umero de tiempos $T_{n}$ en el intervalo $\left[0,t\right)$ es

\begin{eqnarray}
N\left(t\right)=\sum_{n=1}^{\infty}\indora\left(T_{n}\leq t\right),
\end{eqnarray}
para $t\geq0$.
\end{Def}

Si se consideran los puntos $T_{n}$ como elementos de $\rea_{+}$, y $N\left(t\right)$ es el n\'umero de puntos en $\rea$. El proceso denotado por $\left\{N\left(t\right):t\geq0\right\}$, denotado por $N\left(t\right)$, es un proceso puntual en $\rea_{+}$. Los $T_{n}$ son los tiempos de ocurrencia, el proceso puntual $N\left(t\right)$ es simple si su n\'umero de ocurrencias son distintas: $0<T_{1}<T_{2}<\ldots$ casi seguramente.

\begin{Def}
Un proceso puntual $N\left(t\right)$ es un proceso de renovaci\'on si los tiempos de interocurrencia $\xi_{n}=T_{n}-T_{n-1}$, para $n\geq1$, son independientes e identicamente distribuidos con distribuci\'on $F$, donde $F\left(0\right)=0$ y $T_{0}=0$. Los $T_{n}$ son llamados tiempos de renovaci\'on, referente a la independencia o renovaci\'on de la informaci\'on estoc\'astica en estos tiempos. Los $\xi_{n}$ son los tiempos de inter-renovaci\'on, y $N\left(t\right)$ es el n\'umero de renovaciones en el intervalo $\left[0,t\right)$
\end{Def}


\begin{Note}
Para definir un proceso de renovaci\'on para cualquier contexto, solamente hay que especificar una distribuci\'on $F$, con $F\left(0\right)=0$, para los tiempos de inter-renovaci\'on. La funci\'on $F$ en turno degune las otra variables aleatorias. De manera formal, existe un espacio de probabilidad y una sucesi\'on de variables aleatorias $\xi_{1},\xi_{2},\ldots$ definidas en este con distribuci\'on $F$. Entonces las otras cantidades son $T_{n}=\sum_{k=1}^{n}\xi_{k}$ y $N\left(t\right)=\sum_{n=1}^{\infty}\indora\left(T_{n}\leq t\right)$, donde $T_{n}\rightarrow\infty$ casi seguramente por la Ley Fuerte de los Grandes N\'umeros.
\end{Note}







Los tiempos $T_{n}$ est\'an relacionados con los conteos de $N\left(t\right)$ por

\begin{eqnarray*}
\left\{N\left(t\right)\geq n\right\}&=&\left\{T_{n}\leq t\right\}\\
T_{N\left(t\right)}\leq &t&<T_{N\left(t\right)+1},
\end{eqnarray*}

adem\'as $N\left(T_{n}\right)=n$, y 

\begin{eqnarray*}
N\left(t\right)=\max\left\{n:T_{n}\leq t\right\}=\min\left\{n:T_{n+1}>t\right\}
\end{eqnarray*}

Por propiedades de la convoluci\'on se sabe que

\begin{eqnarray*}
P\left\{T_{n}\leq t\right\}=F^{n\star}\left(t\right)
\end{eqnarray*}
que es la $n$-\'esima convoluci\'on de $F$. Entonces 

\begin{eqnarray*}
\left\{N\left(t\right)\geq n\right\}&=&\left\{T_{n}\leq t\right\}\\
P\left\{N\left(t\right)\leq n\right\}&=&1-F^{\left(n+1\right)\star}\left(t\right)
\end{eqnarray*}

Adem\'as usando el hecho de que $\esp\left[N\left(t\right)\right]=\sum_{n=1}^{\infty}P\left\{N\left(t\right)\geq n\right\}$
se tiene que

\begin{eqnarray*}
\esp\left[N\left(t\right)\right]=\sum_{n=1}^{\infty}F^{n\star}\left(t\right)
\end{eqnarray*}

\begin{Prop}
Para cada $t\geq0$, la funci\'on generadora de momentos $\esp\left[e^{\alpha N\left(t\right)}\right]$ existe para alguna $\alpha$ en una vecindad del 0, y de aqu\'i que $\esp\left[N\left(t\right)^{m}\right]<\infty$, para $m\geq1$.
\end{Prop}

\begin{Ejem}[\textbf{Proceso Poisson}]

Suponga que se tienen tiempos de inter-renovaci\'on \textit{i.i.d.} del proceso de renovaci\'on $N\left(t\right)$ tienen distribuci\'on exponencial $F\left(t\right)=q-e^{-\lambda t}$ con tasa $\lambda$. Entonces $N\left(t\right)$ es un proceso Poisson con tasa $\lambda$.

\end{Ejem}


\begin{Note}
Si el primer tiempo de renovaci\'on $\xi_{1}$ no tiene la misma distribuci\'on que el resto de las $\xi_{n}$, para $n\geq2$, a $N\left(t\right)$ se le llama Proceso de Renovaci\'on retardado, donde si $\xi$ tiene distribuci\'on $G$, entonces el tiempo $T_{n}$ de la $n$-\'esima renovaci\'on tiene distribuci\'on $G\star F^{\left(n-1\right)\star}\left(t\right)$
\end{Note}


\begin{Teo}
Para una constante $\mu\leq\infty$ ( o variable aleatoria), las siguientes expresiones son equivalentes:

\begin{eqnarray}
lim_{n\rightarrow\infty}n^{-1}T_{n}&=&\mu,\textrm{ c.s.}\\
lim_{t\rightarrow\infty}t^{-1}N\left(t\right)&=&1/\mu,\textrm{ c.s.}
\end{eqnarray}
\end{Teo}


Es decir, $T_{n}$ satisface la Ley Fuerte de los Grandes N\'umeros s\'i y s\'olo s\'i $N\left/t\right)$ la cumple.


\begin{Coro}[Ley Fuerte de los Grandes N\'umeros para Procesos de Renovaci\'on]
Si $N\left(t\right)$ es un proceso de renovaci\'on cuyos tiempos de inter-renovaci\'on tienen media $\mu\leq\infty$, entonces
\begin{eqnarray}
t^{-1}N\left(t\right)\rightarrow 1/\mu,\textrm{ c.s. cuando }t\rightarrow\infty.
\end{eqnarray}

\end{Coro}


Considerar el proceso estoc\'astico de valores reales $\left\{Z\left(t\right):t\geq0\right\}$ en el mismo espacio de probabilidad que $N\left(t\right)$

\begin{Def}
Para el proceso $\left\{Z\left(t\right):t\geq0\right\}$ se define la fluctuaci\'on m\'axima de $Z\left(t\right)$ en el intervalo $\left(T_{n-1},T_{n}\right]$:
\begin{eqnarray*}
M_{n}=\sup_{T_{n-1}<t\leq T_{n}}|Z\left(t\right)-Z\left(T_{n-1}\right)|
\end{eqnarray*}
\end{Def}

\begin{Teo}
Sup\'ongase que $n^{-1}T_{n}\rightarrow\mu$ c.s. cuando $n\rightarrow\infty$, donde $\mu\leq\infty$ es una constante o variable aleatoria. Sea $a$ una constante o variable aleatoria que puede ser infinita cuando $\mu$ es finita, y considere las expresiones l\'imite:
\begin{eqnarray}
lim_{n\rightarrow\infty}n^{-1}Z\left(T_{n}\right)&=&a,\textrm{ c.s.}\\
lim_{t\rightarrow\infty}t^{-1}Z\left(t\right)&=&a/\mu,\textrm{ c.s.}
\end{eqnarray}
La segunda expresi\'on implica la primera. Conversamente, la primera implica la segunda si el proceso $Z\left(t\right)$ es creciente, o si $lim_{n\rightarrow\infty}n^{-1}M_{n}=0$ c.s.
\end{Teo}

\begin{Coro}
Si $N\left(t\right)$ es un proceso de renovaci\'on, y $\left(Z\left(T_{n}\right)-Z\left(T_{n-1}\right),M_{n}\right)$, para $n\geq1$, son variables aleatorias independientes e id\'enticamente distribuidas con media finita, entonces,
\begin{eqnarray}
lim_{t\rightarrow\infty}t^{-1}Z\left(t\right)\rightarrow\frac{\esp\left[Z\left(T_{1}\right)-Z\left(T_{0}\right)\right]}{\esp\left[T_{1}\right]},\textrm{ c.s. cuando  }t\rightarrow\infty.
\end{eqnarray}
\end{Coro}


Sup\'ongase que $N\left(t\right)$ es un proceso de renovaci\'on con distribuci\'on $F$ con media finita $\mu$.

\begin{Def}
La funci\'on de renovaci\'on asociada con la distribuci\'on $F$, del proceso $N\left(t\right)$, es
\begin{eqnarray*}
U\left(t\right)=\sum_{n=1}^{\infty}F^{n\star}\left(t\right),\textrm{   }t\geq0,
\end{eqnarray*}
donde $F^{0\star}\left(t\right)=\indora\left(t\geq0\right)$.
\end{Def}


\begin{Prop}
Sup\'ongase que la distribuci\'on de inter-renovaci\'on $F$ tiene densidad $f$. Entonces $U\left(t\right)$ tambi\'en tiene densidad, para $t>0$, y es $U^{'}\left(t\right)=\sum_{n=0}^{\infty}f^{n\star}\left(t\right)$. Adem\'as
\begin{eqnarray*}
\prob\left\{N\left(t\right)>N\left(t-\right)\right\}=0\textrm{,   }t\geq0.
\end{eqnarray*}
\end{Prop}

\begin{Def}
La Transformada de Laplace-Stieljes de $F$ est\'a dada por

\begin{eqnarray*}
\hat{F}\left(\alpha\right)=\int_{\rea_{+}}e^{-\alpha t}dF\left(t\right)\textrm{,  }\alpha\geq0.
\end{eqnarray*}
\end{Def}

Entonces

\begin{eqnarray*}
\hat{U}\left(\alpha\right)=\sum_{n=0}^{\infty}\hat{F^{n\star}}\left(\alpha\right)=\sum_{n=0}^{\infty}\hat{F}\left(\alpha\right)^{n}=\frac{1}{1-\hat{F}\left(\alpha\right)}.
\end{eqnarray*}


\begin{Prop}
La Transformada de Laplace $\hat{U}\left(\alpha\right)$ y $\hat{F}\left(\alpha\right)$ determina una a la otra de manera \'unica por la relaci\'on $\hat{U}\left(\alpha\right)=\frac{1}{1-\hat{F}\left(\alpha\right)}$.
\end{Prop}


\begin{Note}
Un proceso de renovaci\'on $N\left(t\right)$ cuyos tiempos de inter-renovaci\'on tienen media finita, es un proceso Poisson con tasa $\lambda$ si y s\'olo s\'i $\esp\left[U\left(t\right)\right]=\lambda t$, para $t\geq0$.
\end{Note}


\begin{Teo}
Sea $N\left(t\right)$ un proceso puntual simple con puntos de localizaci\'on $T_{n}$ tal que $\eta\left(t\right)=\esp\left[N\left(\right)\right]$ es finita para cada $t$. Entonces para cualquier funci\'on $f:\rea_{+}\rightarrow\rea$,
\begin{eqnarray*}
\esp\left[\sum_{n=1}^{N\left(\right)}f\left(T_{n}\right)\right]=\int_{\left(0,t\right]}f\left(s\right)d\eta\left(s\right)\textrm{,  }t\geq0,
\end{eqnarray*}
suponiendo que la integral exista. Adem\'as si $X_{1},X_{2},\ldots$ son variables aleatorias definidas en el mismo espacio de probabilidad que el proceso $N\left(t\right)$ tal que $\esp\left[X_{n}|T_{n}=s\right]=f\left(s\right)$, independiente de $n$. Entonces
\begin{eqnarray*}
\esp\left[\sum_{n=1}^{N\left(t\right)}X_{n}\right]=\int_{\left(0,t\right]}f\left(s\right)d\eta\left(s\right)\textrm{,  }t\geq0,
\end{eqnarray*} 
suponiendo que la integral exista. 
\end{Teo}

\begin{Coro}[Identidad de Wald para Renovaciones]
Para el proceso de renovaci\'on $N\left(t\right)$,
\begin{eqnarray*}
\esp\left[T_{N\left(t\right)+1}\right]=\mu\esp\left[N\left(t\right)+1\right]\textrm{,  }t\geq0,
\end{eqnarray*}  
\end{Coro}


\begin{Def}
Sea $h\left(t\right)$ funci\'on de valores reales en $\rea$ acotada en intervalos finitos e igual a cero para $t<0$ La ecuaci\'on de renovaci\'on para $h\left(t\right)$ y la distribuci\'on $F$ es

\begin{eqnarray}%\label{Ec.Renovacion}
H\left(t\right)=h\left(t\right)+\int_{\left[0,t\right]}H\left(t-s\right)dF\left(s\right)\textrm{,    }t\geq0,
\end{eqnarray}
donde $H\left(t\right)$ es una funci\'on de valores reales. Esto es $H=h+F\star H$. Decimos que $H\left(t\right)$ es soluci\'on de esta ecuaci\'on si satisface la ecuaci\'on, y es acotada en intervalos finitos e iguales a cero para $t<0$.
\end{Def}

\begin{Prop}
La funci\'on $U\star h\left(t\right)$ es la \'unica soluci\'on de la ecuaci\'on de renovaci\'on (\ref{Ec.Renovacion}).
\end{Prop}

\begin{Teo}[Teorema Renovaci\'on Elemental]
\begin{eqnarray*}
t^{-1}U\left(t\right)\rightarrow 1/\mu\textrm{,    cuando }t\rightarrow\infty.
\end{eqnarray*}
\end{Teo}

Sup\'ongase que $N\left(t\right)$ es un proceso de renovaci\'on con distribuci\'on $F$ con media finita $\mu$.

\begin{Def}
La funci\'on de renovaci\'on asociada con la distribuci\'on $F$, del proceso $N\left(t\right)$, es
\begin{eqnarray*}
U\left(t\right)=\sum_{n=1}^{\infty}F^{n\star}\left(t\right),\textrm{   }t\geq0,
\end{eqnarray*}
donde $F^{0\star}\left(t\right)=\indora\left(t\geq0\right)$.
\end{Def}


\begin{Prop}
Sup\'ongase que la distribuci\'on de inter-renovaci\'on $F$ tiene densidad $f$. Entonces $U\left(t\right)$ tambi\'en tiene densidad, para $t>0$, y es $U^{'}\left(t\right)=\sum_{n=0}^{\infty}f^{n\star}\left(t\right)$. Adem\'as
\begin{eqnarray*}
\prob\left\{N\left(t\right)>N\left(t-\right)\right\}=0\textrm{,   }t\geq0.
\end{eqnarray*}
\end{Prop}

\begin{Def}
La Transformada de Laplace-Stieljes de $F$ est\'a dada por

\begin{eqnarray*}
\hat{F}\left(\alpha\right)=\int_{\rea_{+}}e^{-\alpha t}dF\left(t\right)\textrm{,  }\alpha\geq0.
\end{eqnarray*}
\end{Def}

Entonces

\begin{eqnarray*}
\hat{U}\left(\alpha\right)=\sum_{n=0}^{\infty}\hat{F^{n\star}}\left(\alpha\right)=\sum_{n=0}^{\infty}\hat{F}\left(\alpha\right)^{n}=\frac{1}{1-\hat{F}\left(\alpha\right)}.
\end{eqnarray*}


\begin{Prop}
La Transformada de Laplace $\hat{U}\left(\alpha\right)$ y $\hat{F}\left(\alpha\right)$ determina una a la otra de manera \'unica por la relaci\'on $\hat{U}\left(\alpha\right)=\frac{1}{1-\hat{F}\left(\alpha\right)}$.
\end{Prop}


\begin{Note}
Un proceso de renovaci\'on $N\left(t\right)$ cuyos tiempos de inter-renovaci\'on tienen media finita, es un proceso Poisson con tasa $\lambda$ si y s\'olo s\'i $\esp\left[U\left(t\right)\right]=\lambda t$, para $t\geq0$.
\end{Note}


\begin{Teo}
Sea $N\left(t\right)$ un proceso puntual simple con puntos de localizaci\'on $T_{n}$ tal que $\eta\left(t\right)=\esp\left[N\left(\right)\right]$ es finita para cada $t$. Entonces para cualquier funci\'on $f:\rea_{+}\rightarrow\rea$,
\begin{eqnarray*}
\esp\left[\sum_{n=1}^{N\left(\right)}f\left(T_{n}\right)\right]=\int_{\left(0,t\right]}f\left(s\right)d\eta\left(s\right)\textrm{,  }t\geq0,
\end{eqnarray*}
suponiendo que la integral exista. Adem\'as si $X_{1},X_{2},\ldots$ son variables aleatorias definidas en el mismo espacio de probabilidad que el proceso $N\left(t\right)$ tal que $\esp\left[X_{n}|T_{n}=s\right]=f\left(s\right)$, independiente de $n$. Entonces
\begin{eqnarray*}
\esp\left[\sum_{n=1}^{N\left(t\right)}X_{n}\right]=\int_{\left(0,t\right]}f\left(s\right)d\eta\left(s\right)\textrm{,  }t\geq0,
\end{eqnarray*} 
suponiendo que la integral exista. 
\end{Teo}

\begin{Coro}[Identidad de Wald para Renovaciones]
Para el proceso de renovaci\'on $N\left(t\right)$,
\begin{eqnarray*}
\esp\left[T_{N\left(t\right)+1}\right]=\mu\esp\left[N\left(t\right)+1\right]\textrm{,  }t\geq0,
\end{eqnarray*}  
\end{Coro}


\begin{Def}
Sea $h\left(t\right)$ funci\'on de valores reales en $\rea$ acotada en intervalos finitos e igual a cero para $t<0$ La ecuaci\'on de renovaci\'on para $h\left(t\right)$ y la distribuci\'on $F$ es

\begin{eqnarray}%\label{Ec.Renovacion}
H\left(t\right)=h\left(t\right)+\int_{\left[0,t\right]}H\left(t-s\right)dF\left(s\right)\textrm{,    }t\geq0,
\end{eqnarray}
donde $H\left(t\right)$ es una funci\'on de valores reales. Esto es $H=h+F\star H$. Decimos que $H\left(t\right)$ es soluci\'on de esta ecuaci\'on si satisface la ecuaci\'on, y es acotada en intervalos finitos e iguales a cero para $t<0$.
\end{Def}

\begin{Prop}
La funci\'on $U\star h\left(t\right)$ es la \'unica soluci\'on de la ecuaci\'on de renovaci\'on (\ref{Ec.Renovacion}).
\end{Prop}

\begin{Teo}[Teorema Renovaci\'on Elemental]
\begin{eqnarray*}
t^{-1}U\left(t\right)\rightarrow 1/\mu\textrm{,    cuando }t\rightarrow\infty.
\end{eqnarray*}
\end{Teo}


\begin{Note} Una funci\'on $h:\rea_{+}\rightarrow\rea$ es Directamente Riemann Integrable en los siguientes casos:
\begin{itemize}
\item[a)] $h\left(t\right)\geq0$ es decreciente y Riemann Integrable.
\item[b)] $h$ es continua excepto posiblemente en un conjunto de Lebesgue de medida 0, y $|h\left(t\right)|\leq b\left(t\right)$, donde $b$ es DRI.
\end{itemize}
\end{Note}

\begin{Teo}[Teorema Principal de Renovaci\'on]
Si $F$ es no aritm\'etica y $h\left(t\right)$ es Directamente Riemann Integrable (DRI), entonces

\begin{eqnarray*}
lim_{t\rightarrow\infty}U\star h=\frac{1}{\mu}\int_{\rea_{+}}h\left(s\right)ds.
\end{eqnarray*}
\end{Teo}

\begin{Prop}
Cualquier funci\'on $H\left(t\right)$ acotada en intervalos finitos y que es 0 para $t<0$ puede expresarse como
\begin{eqnarray*}
H\left(t\right)=U\star h\left(t\right)\textrm{,  donde }h\left(t\right)=H\left(t\right)-F\star H\left(t\right)
\end{eqnarray*}
\end{Prop}

\begin{Def}
Un proceso estoc\'astico $X\left(t\right)$ es crudamente regenerativo en un tiempo aleatorio positivo $T$ si
\begin{eqnarray*}
\esp\left[X\left(T+t\right)|T\right]=\esp\left[X\left(t\right)\right]\textrm{, para }t\geq0,\end{eqnarray*}
y con las esperanzas anteriores finitas.
\end{Def}

\begin{Prop}
Sup\'ongase que $X\left(t\right)$ es un proceso crudamente regenerativo en $T$, que tiene distribuci\'on $F$. Si $\esp\left[X\left(t\right)\right]$ es acotado en intervalos finitos, entonces
\begin{eqnarray*}
\esp\left[X\left(t\right)\right]=U\star h\left(t\right)\textrm{,  donde }h\left(t\right)=\esp\left[X\left(t\right)\indora\left(T>t\right)\right].
\end{eqnarray*}
\end{Prop}

\begin{Teo}[Regeneraci\'on Cruda]
Sup\'ongase que $X\left(t\right)$ es un proceso con valores positivo crudamente regenerativo en $T$, y def\'inase $M=\sup\left\{|X\left(t\right)|:t\leq T\right\}$. Si $T$ es no aritm\'etico y $M$ y $MT$ tienen media finita, entonces
\begin{eqnarray*}
lim_{t\rightarrow\infty}\esp\left[X\left(t\right)\right]=\frac{1}{\mu}\int_{\rea_{+}}h\left(s\right)ds,
\end{eqnarray*}
donde $h\left(t\right)=\esp\left[X\left(t\right)\indora\left(T>t\right)\right]$.
\end{Teo}


\begin{Note} Una funci\'on $h:\rea_{+}\rightarrow\rea$ es Directamente Riemann Integrable en los siguientes casos:
\begin{itemize}
\item[a)] $h\left(t\right)\geq0$ es decreciente y Riemann Integrable.
\item[b)] $h$ es continua excepto posiblemente en un conjunto de Lebesgue de medida 0, y $|h\left(t\right)|\leq b\left(t\right)$, donde $b$ es DRI.
\end{itemize}
\end{Note}

\begin{Teo}[Teorema Principal de Renovaci\'on]
Si $F$ es no aritm\'etica y $h\left(t\right)$ es Directamente Riemann Integrable (DRI), entonces

\begin{eqnarray*}
lim_{t\rightarrow\infty}U\star h=\frac{1}{\mu}\int_{\rea_{+}}h\left(s\right)ds.
\end{eqnarray*}
\end{Teo}

\begin{Prop}
Cualquier funci\'on $H\left(t\right)$ acotada en intervalos finitos y que es 0 para $t<0$ puede expresarse como
\begin{eqnarray*}
H\left(t\right)=U\star h\left(t\right)\textrm{,  donde }h\left(t\right)=H\left(t\right)-F\star H\left(t\right)
\end{eqnarray*}
\end{Prop}

\begin{Def}
Un proceso estoc\'astico $X\left(t\right)$ es crudamente regenerativo en un tiempo aleatorio positivo $T$ si
\begin{eqnarray*}
\esp\left[X\left(T+t\right)|T\right]=\esp\left[X\left(t\right)\right]\textrm{, para }t\geq0,\end{eqnarray*}
y con las esperanzas anteriores finitas.
\end{Def}

\begin{Prop}
Sup\'ongase que $X\left(t\right)$ es un proceso crudamente regenerativo en $T$, que tiene distribuci\'on $F$. Si $\esp\left[X\left(t\right)\right]$ es acotado en intervalos finitos, entonces
\begin{eqnarray*}
\esp\left[X\left(t\right)\right]=U\star h\left(t\right)\textrm{,  donde }h\left(t\right)=\esp\left[X\left(t\right)\indora\left(T>t\right)\right].
\end{eqnarray*}
\end{Prop}

\begin{Teo}[Regeneraci\'on Cruda]
Sup\'ongase que $X\left(t\right)$ es un proceso con valores positivo crudamente regenerativo en $T$, y def\'inase $M=\sup\left\{|X\left(t\right)|:t\leq T\right\}$. Si $T$ es no aritm\'etico y $M$ y $MT$ tienen media finita, entonces
\begin{eqnarray*}
lim_{t\rightarrow\infty}\esp\left[X\left(t\right)\right]=\frac{1}{\mu}\int_{\rea_{+}}h\left(s\right)ds,
\end{eqnarray*}
donde $h\left(t\right)=\esp\left[X\left(t\right)\indora\left(T>t\right)\right]$.
\end{Teo}

\begin{Def}
Para el proceso $\left\{\left(N\left(t\right),X\left(t\right)\right):t\geq0\right\}$, sus trayectoria muestrales en el intervalo de tiempo $\left[T_{n-1},T_{n}\right)$ est\'an descritas por
\begin{eqnarray*}
\zeta_{n}=\left(\xi_{n},\left\{X\left(T_{n-1}+t\right):0\leq t<\xi_{n}\right\}\right)
\end{eqnarray*}
Este $\zeta_{n}$ es el $n$-\'esimo segmento del proceso. El proceso es regenerativo sobre los tiempos $T_{n}$ si sus segmentos $\zeta_{n}$ son independientes e id\'enticamennte distribuidos.
\end{Def}


\begin{Note}
Si $\tilde{X}\left(t\right)$ con espacio de estados $\tilde{S}$ es regenerativo sobre $T_{n}$, entonces $X\left(t\right)=f\left(\tilde{X}\left(t\right)\right)$ tambi\'en es regenerativo sobre $T_{n}$, para cualquier funci\'on $f:\tilde{S}\rightarrow S$.
\end{Note}

\begin{Note}
Los procesos regenerativos son crudamente regenerativos, pero no al rev\'es.
\end{Note}


\begin{Note}
Un proceso estoc\'astico a tiempo continuo o discreto es regenerativo si existe un proceso de renovaci\'on  tal que los segmentos del proceso entre tiempos de renovaci\'on sucesivos son i.i.d., es decir, para $\left\{X\left(t\right):t\geq0\right\}$ proceso estoc\'astico a tiempo continuo con espacio de estados $S$, espacio m\'etrico.
\end{Note}

Para $\left\{X\left(t\right):t\geq0\right\}$ Proceso Estoc\'astico a tiempo continuo con estado de espacios $S$, que es un espacio m\'etrico, con trayectorias continuas por la derecha y con l\'imites por la izquierda c.s. Sea $N\left(t\right)$ un proceso de renovaci\'on en $\rea_{+}$ definido en el mismo espacio de probabilidad que $X\left(t\right)$, con tiempos de renovaci\'on $T$ y tiempos de inter-renovaci\'on $\xi_{n}=T_{n}-T_{n-1}$, con misma distribuci\'on $F$ de media finita $\mu$.



\begin{Def}
Para el proceso $\left\{\left(N\left(t\right),X\left(t\right)\right):t\geq0\right\}$, sus trayectoria muestrales en el intervalo de tiempo $\left[T_{n-1},T_{n}\right)$ est\'an descritas por
\begin{eqnarray*}
\zeta_{n}=\left(\xi_{n},\left\{X\left(T_{n-1}+t\right):0\leq t<\xi_{n}\right\}\right)
\end{eqnarray*}
Este $\zeta_{n}$ es el $n$-\'esimo segmento del proceso. El proceso es regenerativo sobre los tiempos $T_{n}$ si sus segmentos $\zeta_{n}$ son independientes e id\'enticamennte distribuidos.
\end{Def}

\begin{Note}
Un proceso regenerativo con media de la longitud de ciclo finita es llamado positivo recurrente.
\end{Note}

\begin{Teo}[Procesos Regenerativos]
Suponga que el proceso
\end{Teo}


\begin{Def}[Renewal Process Trinity]
Para un proceso de renovaci\'on $N\left(t\right)$, los siguientes procesos proveen de informaci\'on sobre los tiempos de renovaci\'on.
\begin{itemize}
\item $A\left(t\right)=t-T_{N\left(t\right)}$, el tiempo de recurrencia hacia atr\'as al tiempo $t$, que es el tiempo desde la \'ultima renovaci\'on para $t$.

\item $B\left(t\right)=T_{N\left(t\right)+1}-t$, el tiempo de recurrencia hacia adelante al tiempo $t$, residual del tiempo de renovaci\'on, que es el tiempo para la pr\'oxima renovaci\'on despu\'es de $t$.

\item $L\left(t\right)=\xi_{N\left(t\right)+1}=A\left(t\right)+B\left(t\right)$, la longitud del intervalo de renovaci\'on que contiene a $t$.
\end{itemize}
\end{Def}

\begin{Note}
El proceso tridimensional $\left(A\left(t\right),B\left(t\right),L\left(t\right)\right)$ es regenerativo sobre $T_{n}$, y por ende cada proceso lo es. Cada proceso $A\left(t\right)$ y $B\left(t\right)$ son procesos de MArkov a tiempo continuo con trayectorias continuas por partes en el espacio de estados $\rea_{+}$. Una expresi\'on conveniente para su distribuci\'on conjunta es, para $0\leq x<t,y\geq0$
\begin{equation}\label{NoRenovacion}
P\left\{A\left(t\right)>x,B\left(t\right)>y\right\}=
P\left\{N\left(t+y\right)-N\left((t-x)\right)=0\right\}
\end{equation}
\end{Note}

\begin{Ejem}[Tiempos de recurrencia Poisson]
Si $N\left(t\right)$ es un proceso Poisson con tasa $\lambda$, entonces de la expresi\'on (\ref{NoRenovacion}) se tiene que

\begin{eqnarray*}
\begin{array}{lc}
P\left\{A\left(t\right)>x,B\left(t\right)>y\right\}=e^{-\lambda\left(x+y\right)},&0\leq x<t,y\geq0,
\end{array}
\end{eqnarray*}
que es la probabilidad Poisson de no renovaciones en un intervalo de longitud $x+y$.

\end{Ejem}

%\begin{Note}
Una cadena de Markov erg\'odica tiene la propiedad de ser estacionaria si la distribuci\'on de su estado al tiempo $0$ es su distribuci\'on estacionaria.
%\end{Note}


\begin{Def}
Un proceso estoc\'astico a tiempo continuo $\left\{X\left(t\right):t\geq0\right\}$ en un espacio general es estacionario si sus distribuciones finito dimensionales son invariantes bajo cualquier  traslado: para cada $0\leq s_{1}<s_{2}<\cdots<s_{k}$ y $t\geq0$,
\begin{eqnarray*}
\left(X\left(s_{1}+t\right),\ldots,X\left(s_{k}+t\right)\right)=_{d}\left(X\left(s_{1}\right),\ldots,X\left(s_{k}\right)\right).
\end{eqnarray*}
\end{Def}

\begin{Note}
Un proceso de Markov es estacionario si $X\left(t\right)=_{d}X\left(0\right)$, $t\geq0$.
\end{Note}

Considerese el proceso $N\left(t\right)=\sum_{n}\indora\left(\tau_{n}\leq t\right)$ en $\rea_{+}$, con puntos $0<\tau_{1}<\tau_{2}<\cdots$.

\begin{Prop}
Si $N$ es un proceso puntual estacionario y $\esp\left[N\left(1\right)\right]<\infty$, entonces $\esp\left[N\left(t\right)\right]=t\esp\left[N\left(1\right)\right]$, $t\geq0$

\end{Prop}

\begin{Teo}
Los siguientes enunciados son equivalentes
\begin{itemize}
\item[i)] El proceso retardado de renovaci\'on $N$ es estacionario.

\item[ii)] EL proceso de tiempos de recurrencia hacia adelante $B\left(t\right)$ es estacionario.


\item[iii)] $\esp\left[N\left(t\right)\right]=t/\mu$,


\item[iv)] $G\left(t\right)=F_{e}\left(t\right)=\frac{1}{\mu}\int_{0}^{t}\left[1-F\left(s\right)\right]ds$
\end{itemize}
Cuando estos enunciados son ciertos, $P\left\{B\left(t\right)\leq x\right\}=F_{e}\left(x\right)$, para $t,x\geq0$.

\end{Teo}

\begin{Note}
Una consecuencia del teorema anterior es que el Proceso Poisson es el \'unico proceso sin retardo que es estacionario.
\end{Note}

\begin{Coro}
El proceso de renovaci\'on $N\left(t\right)$ sin retardo, y cuyos tiempos de inter renonaci\'on tienen media finita, es estacionario si y s\'olo si es un proceso Poisson.

\end{Coro}


%________________________________________________________________________
\subsection{Procesos Regenerativos}
%________________________________________________________________________

Para $\left\{X\left(t\right):t\geq0\right\}$ Proceso Estoc\'astico a tiempo continuo con estado de espacios $S$, que es un espacio m\'etrico, con trayectorias continuas por la derecha y con l\'imites por la izquierda c.s. Sea $N\left(t\right)$ un proceso de renovaci\'on en $\rea_{+}$ definido en el mismo espacio de probabilidad que $X\left(t\right)$, con tiempos de renovaci\'on $T$ y tiempos de inter-renovaci\'on $\xi_{n}=T_{n}-T_{n-1}$, con misma distribuci\'on $F$ de media finita $\mu$.



\begin{Def}
Para el proceso $\left\{\left(N\left(t\right),X\left(t\right)\right):t\geq0\right\}$, sus trayectoria muestrales en el intervalo de tiempo $\left[T_{n-1},T_{n}\right)$ est\'an descritas por
\begin{eqnarray*}
\zeta_{n}=\left(\xi_{n},\left\{X\left(T_{n-1}+t\right):0\leq t<\xi_{n}\right\}\right)
\end{eqnarray*}
Este $\zeta_{n}$ es el $n$-\'esimo segmento del proceso. El proceso es regenerativo sobre los tiempos $T_{n}$ si sus segmentos $\zeta_{n}$ son independientes e id\'enticamennte distribuidos.
\end{Def}


\begin{Note}
Si $\tilde{X}\left(t\right)$ con espacio de estados $\tilde{S}$ es regenerativo sobre $T_{n}$, entonces $X\left(t\right)=f\left(\tilde{X}\left(t\right)\right)$ tambi\'en es regenerativo sobre $T_{n}$, para cualquier funci\'on $f:\tilde{S}\rightarrow S$.
\end{Note}

\begin{Note}
Los procesos regenerativos son crudamente regenerativos, pero no al rev\'es.
\end{Note}

\begin{Def}[Definici\'on Cl\'asica]
Un proceso estoc\'astico $X=\left\{X\left(t\right):t\geq0\right\}$ es llamado regenerativo is existe una variable aleatoria $R_{1}>0$ tal que
\begin{itemize}
\item[i)] $\left\{X\left(t+R_{1}\right):t\geq0\right\}$ es independiente de $\left\{\left\{X\left(t\right):t<R_{1}\right\},\right\}$
\item[ii)] $\left\{X\left(t+R_{1}\right):t\geq0\right\}$ es estoc\'asticamente equivalente a $\left\{X\left(t\right):t>0\right\}$
\end{itemize}

Llamamos a $R_{1}$ tiempo de regeneraci\'on, y decimos que $X$ se regenera en este punto.
\end{Def}

$\left\{X\left(t+R_{1}\right)\right\}$ es regenerativo con tiempo de regeneraci\'on $R_{2}$, independiente de $R_{1}$ pero con la misma distribuci\'on que $R_{1}$. Procediendo de esta manera se obtiene una secuencia de variables aleatorias independientes e id\'enticamente distribuidas $\left\{R_{n}\right\}$ llamados longitudes de ciclo. Si definimos a $Z_{k}\equiv R_{1}+R_{2}+\cdots+R_{k}$, se tiene un proceso de renovaci\'on llamado proceso de renovaci\'on encajado para $X$.

\begin{Note}
Un proceso regenerativo con media de la longitud de ciclo finita es llamado positivo recurrente.
\end{Note}


\begin{Def}
Para $x$ fijo y para cada $t\geq0$, sea $I_{x}\left(t\right)=1$ si $X\left(t\right)\leq x$,  $I_{x}\left(t\right)=0$ en caso contrario, y def\'inanse los tiempos promedio
\begin{eqnarray*}
\overline{X}&=&lim_{t\rightarrow\infty}\frac{1}{t}\int_{0}^{\infty}X\left(u\right)du\\
\prob\left(X_{\infty}\leq x\right)&=&lim_{t\rightarrow\infty}\frac{1}{t}\int_{0}^{\infty}I_{x}\left(u\right)du,
\end{eqnarray*}
cuando estos l\'imites existan.
\end{Def}

Como consecuencia del teorema de Renovaci\'on-Recompensa, se tiene que el primer l\'imite  existe y es igual a la constante
\begin{eqnarray*}
\overline{X}&=&\frac{\esp\left[\int_{0}^{R_{1}}X\left(t\right)dt\right]}{\esp\left[R_{1}\right]},
\end{eqnarray*}
suponiendo que ambas esperanzas son finitas.

\begin{Note}
\begin{itemize}
\item[a)] Si el proceso regenerativo $X$ es positivo recurrente y tiene trayectorias muestrales no negativas, entonces la ecuaci\'on anterior es v\'alida.
\item[b)] Si $X$ es positivo recurrente regenerativo, podemos construir una \'unica versi\'on estacionaria de este proceso, $X_{e}=\left\{X_{e}\left(t\right)\right\}$, donde $X_{e}$ es un proceso estoc\'astico regenerativo y estrictamente estacionario, con distribuci\'on marginal distribuida como $X_{\infty}$
\end{itemize}
\end{Note}

%__________________________________________________________________________________________
\subsection{Procesos Regenerativos Estacionarios - Stidham \cite{Stidham}}
%__________________________________________________________________________________________


Un proceso estoc\'astico a tiempo continuo $\left\{V\left(t\right),t\geq0\right\}$ es un proceso regenerativo si existe una sucesi\'on de variables aleatorias independientes e id\'enticamente distribuidas $\left\{X_{1},X_{2},\ldots\right\}$, sucesi\'on de renovaci\'on, tal que para cualquier conjunto de Borel $A$, 

\begin{eqnarray*}
\prob\left\{V\left(t\right)\in A|X_{1}+X_{2}+\cdots+X_{R\left(t\right)}=s,\left\{V\left(\tau\right),\tau<s\right\}\right\}=\prob\left\{V\left(t-s\right)\in A|X_{1}>t-s\right\},
\end{eqnarray*}
para todo $0\leq s\leq t$, donde $R\left(t\right)=\max\left\{X_{1}+X_{2}+\cdots+X_{j}\leq t\right\}=$n\'umero de renovaciones ({\emph{puntos de regeneraci\'on}}) que ocurren en $\left[0,t\right]$. El intervalo $\left[0,X_{1}\right)$ es llamado {\emph{primer ciclo de regeneraci\'on}} de $\left\{V\left(t \right),t\geq0\right\}$, $\left[X_{1},X_{1}+X_{2}\right)$ el {\emph{segundo ciclo de regeneraci\'on}}, y as\'i sucesivamente.

Sea $X=X_{1}$ y sea $F$ la funci\'on de distrbuci\'on de $X$


\begin{Def}
Se define el proceso estacionario, $\left\{V^{*}\left(t\right),t\geq0\right\}$, para $\left\{V\left(t\right),t\geq0\right\}$ por

\begin{eqnarray*}
\prob\left\{V\left(t\right)\in A\right\}=\frac{1}{\esp\left[X\right]}\int_{0}^{\infty}\prob\left\{V\left(t+x\right)\in A|X>x\right\}\left(1-F\left(x\right)\right)dx,
\end{eqnarray*} 
para todo $t\geq0$ y todo conjunto de Borel $A$.
\end{Def}

\begin{Def}
Una distribuci\'on se dice que es {\emph{aritm\'etica}} si todos sus puntos de incremento son m\'ultiplos de la forma $0,\lambda, 2\lambda,\ldots$ para alguna $\lambda>0$ entera.
\end{Def}


\begin{Def}
Una modificaci\'on medible de un proceso $\left\{V\left(t\right),t\geq0\right\}$, es una versi\'on de este, $\left\{V\left(t,w\right)\right\}$ conjuntamente medible para $t\geq0$ y para $w\in S$, $S$ espacio de estados para $\left\{V\left(t\right),t\geq0\right\}$.
\end{Def}

\begin{Teo}
Sea $\left\{V\left(t\right),t\geq\right\}$ un proceso regenerativo no negativo con modificaci\'on medible. Sea $\esp\left[X\right]<\infty$. Entonces el proceso estacionario dado por la ecuaci\'on anterior est\'a bien definido y tiene funci\'on de distribuci\'on independiente de $t$, adem\'as
\begin{itemize}
\item[i)] \begin{eqnarray*}
\esp\left[V^{*}\left(0\right)\right]&=&\frac{\esp\left[\int_{0}^{X}V\left(s\right)ds\right]}{\esp\left[X\right]}\end{eqnarray*}
\item[ii)] Si $\esp\left[V^{*}\left(0\right)\right]<\infty$, equivalentemente, si $\esp\left[\int_{0}^{X}V\left(s\right)ds\right]<\infty$,entonces
\begin{eqnarray*}
\frac{\int_{0}^{t}V\left(s\right)ds}{t}\rightarrow\frac{\esp\left[\int_{0}^{X}V\left(s\right)ds\right]}{\esp\left[X\right]}
\end{eqnarray*}
con probabilidad 1 y en media, cuando $t\rightarrow\infty$.
\end{itemize}
\end{Teo}

Para $\left\{X\left(t\right):t\geq0\right\}$ Proceso Estoc\'astico a tiempo continuo con estado de espacios $S$, que es un espacio m\'etrico, con trayectorias continuas por la derecha y con l\'imites por la izquierda c.s. Sea $N\left(t\right)$ un proceso de renovaci\'on en $\rea_{+}$ definido en el mismo espacio de probabilidad que $X\left(t\right)$, con tiempos de renovaci\'on $T$ y tiempos de inter-renovaci\'on $\xi_{n}=T_{n}-T_{n-1}$, con misma distribuci\'on $F$ de media finita $\mu$.

%______________________________________________________________________
\section{De nuevo}
%______________________________________________________________________

\subsection{Renewal and Regenerative Processes: Serfozo\cite{Serfozo}}
\begin{Def}\label{Def.Tn}
Sean $0\leq T_{1}\leq T_{2}\leq \ldots$ son tiempos aleatorios infinitos en los cuales ocurren ciertos eventos. El n\'umero de tiempos $T_{n}$ en el intervalo $\left[0,t\right)$ es

\begin{eqnarray}
N\left(t\right)=\sum_{n=1}^{\infty}\indora\left(T_{n}\leq t\right),
\end{eqnarray}
para $t\geq0$.
\end{Def}

Si se consideran los puntos $T_{n}$ como elementos de $\rea_{+}$, y $N\left(t\right)$ es el n\'umero de puntos en $\rea$. El proceso denotado por $\left\{N\left(t\right):t\geq0\right\}$, denotado por $N\left(t\right)$, es un proceso puntual en $\rea_{+}$. Los $T_{n}$ son los tiempos de ocurrencia, el proceso puntual $N\left(t\right)$ es simple si su n\'umero de ocurrencias son distintas: $0<T_{1}<T_{2}<\ldots$ casi seguramente.

\begin{Def}
Un proceso puntual $N\left(t\right)$ es un proceso de renovaci\'on si los tiempos de interocurrencia $\xi_{n}=T_{n}-T_{n-1}$, para $n\geq1$, son independientes e identicamente distribuidos con distribuci\'on $F$, donde $F\left(0\right)=0$ y $T_{0}=0$. Los $T_{n}$ son llamados tiempos de renovaci\'on, referente a la independencia o renovaci\'on de la informaci\'on estoc\'astica en estos tiempos. Los $\xi_{n}$ son los tiempos de inter-renovaci\'on, y $N\left(t\right)$ es el n\'umero de renovaciones en el intervalo $\left[0,t\right)$
\end{Def}


\begin{Note}
Para definir un proceso de renovaci\'on para cualquier contexto, solamente hay que especificar una distribuci\'on $F$, con $F\left(0\right)=0$, para los tiempos de inter-renovaci\'on. La funci\'on $F$ en turno degune las otra variables aleatorias. De manera formal, existe un espacio de probabilidad y una sucesi\'on de variables aleatorias $\xi_{1},\xi_{2},\ldots$ definidas en este con distribuci\'on $F$. Entonces las otras cantidades son $T_{n}=\sum_{k=1}^{n}\xi_{k}$ y $N\left(t\right)=\sum_{n=1}^{\infty}\indora\left(T_{n}\leq t\right)$, donde $T_{n}\rightarrow\infty$ casi seguramente por la Ley Fuerte de los Grandes N\'umeros.
\end{Note}







Los tiempos $T_{n}$ est\'an relacionados con los conteos de $N\left(t\right)$ por

\begin{eqnarray*}
\left\{N\left(t\right)\geq n\right\}&=&\left\{T_{n}\leq t\right\}\\
T_{N\left(t\right)}\leq &t&<T_{N\left(t\right)+1},
\end{eqnarray*}

adem\'as $N\left(T_{n}\right)=n$, y 

\begin{eqnarray*}
N\left(t\right)=\max\left\{n:T_{n}\leq t\right\}=\min\left\{n:T_{n+1}>t\right\}
\end{eqnarray*}

Por propiedades de la convoluci\'on se sabe que

\begin{eqnarray*}
P\left\{T_{n}\leq t\right\}=F^{n\star}\left(t\right)
\end{eqnarray*}
que es la $n$-\'esima convoluci\'on de $F$. Entonces 

\begin{eqnarray*}
\left\{N\left(t\right)\geq n\right\}&=&\left\{T_{n}\leq t\right\}\\
P\left\{N\left(t\right)\leq n\right\}&=&1-F^{\left(n+1\right)\star}\left(t\right)
\end{eqnarray*}

Adem\'as usando el hecho de que $\esp\left[N\left(t\right)\right]=\sum_{n=1}^{\infty}P\left\{N\left(t\right)\geq n\right\}$
se tiene que

\begin{eqnarray*}
\esp\left[N\left(t\right)\right]=\sum_{n=1}^{\infty}F^{n\star}\left(t\right)
\end{eqnarray*}

\begin{Prop}
Para cada $t\geq0$, la funci\'on generadora de momentos $\esp\left[e^{\alpha N\left(t\right)}\right]$ existe para alguna $\alpha$ en una vecindad del 0, y de aqu\'i que $\esp\left[N\left(t\right)^{m}\right]<\infty$, para $m\geq1$.
\end{Prop}


\begin{Note}
Si el primer tiempo de renovaci\'on $\xi_{1}$ no tiene la misma distribuci\'on que el resto de las $\xi_{n}$, para $n\geq2$, a $N\left(t\right)$ se le llama Proceso de Renovaci\'on retardado, donde si $\xi$ tiene distribuci\'on $G$, entonces el tiempo $T_{n}$ de la $n$-\'esima renovaci\'on tiene distribuci\'on $G\star F^{\left(n-1\right)\star}\left(t\right)$
\end{Note}


\begin{Teo}
Para una constante $\mu\leq\infty$ ( o variable aleatoria), las siguientes expresiones son equivalentes:

\begin{eqnarray}
lim_{n\rightarrow\infty}n^{-1}T_{n}&=&\mu,\textrm{ c.s.}\\
lim_{t\rightarrow\infty}t^{-1}N\left(t\right)&=&1/\mu,\textrm{ c.s.}
\end{eqnarray}
\end{Teo}


Es decir, $T_{n}$ satisface la Ley Fuerte de los Grandes N\'umeros s\'i y s\'olo s\'i $N\left/t\right)$ la cumple.


\begin{Coro}[Ley Fuerte de los Grandes N\'umeros para Procesos de Renovaci\'on]
Si $N\left(t\right)$ es un proceso de renovaci\'on cuyos tiempos de inter-renovaci\'on tienen media $\mu\leq\infty$, entonces
\begin{eqnarray}
t^{-1}N\left(t\right)\rightarrow 1/\mu,\textrm{ c.s. cuando }t\rightarrow\infty.
\end{eqnarray}

\end{Coro}


Considerar el proceso estoc\'astico de valores reales $\left\{Z\left(t\right):t\geq0\right\}$ en el mismo espacio de probabilidad que $N\left(t\right)$

\begin{Def}
Para el proceso $\left\{Z\left(t\right):t\geq0\right\}$ se define la fluctuaci\'on m\'axima de $Z\left(t\right)$ en el intervalo $\left(T_{n-1},T_{n}\right]$:
\begin{eqnarray*}
M_{n}=\sup_{T_{n-1}<t\leq T_{n}}|Z\left(t\right)-Z\left(T_{n-1}\right)|
\end{eqnarray*}
\end{Def}

\begin{Teo}
Sup\'ongase que $n^{-1}T_{n}\rightarrow\mu$ c.s. cuando $n\rightarrow\infty$, donde $\mu\leq\infty$ es una constante o variable aleatoria. Sea $a$ una constante o variable aleatoria que puede ser infinita cuando $\mu$ es finita, y considere las expresiones l\'imite:
\begin{eqnarray}
lim_{n\rightarrow\infty}n^{-1}Z\left(T_{n}\right)&=&a,\textrm{ c.s.}\\
lim_{t\rightarrow\infty}t^{-1}Z\left(t\right)&=&a/\mu,\textrm{ c.s.}
\end{eqnarray}
La segunda expresi\'on implica la primera. Conversamente, la primera implica la segunda si el proceso $Z\left(t\right)$ es creciente, o si $lim_{n\rightarrow\infty}n^{-1}M_{n}=0$ c.s.
\end{Teo}

\begin{Coro}
Si $N\left(t\right)$ es un proceso de renovaci\'on, y $\left(Z\left(T_{n}\right)-Z\left(T_{n-1}\right),M_{n}\right)$, para $n\geq1$, son variables aleatorias independientes e id\'enticamente distribuidas con media finita, entonces,
\begin{eqnarray}
lim_{t\rightarrow\infty}t^{-1}Z\left(t\right)\rightarrow\frac{\esp\left[Z\left(T_{1}\right)-Z\left(T_{0}\right)\right]}{\esp\left[T_{1}\right]},\textrm{ c.s. cuando  }t\rightarrow\infty.
\end{eqnarray}
\end{Coro}


Sup\'ongase que $N\left(t\right)$ es un proceso de renovaci\'on con distribuci\'on $F$ con media finita $\mu$.

\begin{Def}
La funci\'on de renovaci\'on asociada con la distribuci\'on $F$, del proceso $N\left(t\right)$, es
\begin{eqnarray*}
U\left(t\right)=\sum_{n=1}^{\infty}F^{n\star}\left(t\right),\textrm{   }t\geq0,
\end{eqnarray*}
donde $F^{0\star}\left(t\right)=\indora\left(t\geq0\right)$.
\end{Def}


\begin{Prop}
Sup\'ongase que la distribuci\'on de inter-renovaci\'on $F$ tiene densidad $f$. Entonces $U\left(t\right)$ tambi\'en tiene densidad, para $t>0$, y es $U^{'}\left(t\right)=\sum_{n=0}^{\infty}f^{n\star}\left(t\right)$. Adem\'as
\begin{eqnarray*}
\prob\left\{N\left(t\right)>N\left(t-\right)\right\}=0\textrm{,   }t\geq0.
\end{eqnarray*}
\end{Prop}

\begin{Def}
La Transformada de Laplace-Stieljes de $F$ est\'a dada por

\begin{eqnarray*}
\hat{F}\left(\alpha\right)=\int_{\rea_{+}}e^{-\alpha t}dF\left(t\right)\textrm{,  }\alpha\geq0.
\end{eqnarray*}
\end{Def}

Entonces

\begin{eqnarray*}
\hat{U}\left(\alpha\right)=\sum_{n=0}^{\infty}\hat{F^{n\star}}\left(\alpha\right)=\sum_{n=0}^{\infty}\hat{F}\left(\alpha\right)^{n}=\frac{1}{1-\hat{F}\left(\alpha\right)}.
\end{eqnarray*}


\begin{Prop}
La Transformada de Laplace $\hat{U}\left(\alpha\right)$ y $\hat{F}\left(\alpha\right)$ determina una a la otra de manera \'unica por la relaci\'on $\hat{U}\left(\alpha\right)=\frac{1}{1-\hat{F}\left(\alpha\right)}$.
\end{Prop}


\begin{Note}
Un proceso de renovaci\'on $N\left(t\right)$ cuyos tiempos de inter-renovaci\'on tienen media finita, es un proceso Poisson con tasa $\lambda$ si y s\'olo s\'i $\esp\left[U\left(t\right)\right]=\lambda t$, para $t\geq0$.
\end{Note}


\begin{Teo}
Sea $N\left(t\right)$ un proceso puntual simple con puntos de localizaci\'on $T_{n}$ tal que $\eta\left(t\right)=\esp\left[N\left(\right)\right]$ es finita para cada $t$. Entonces para cualquier funci\'on $f:\rea_{+}\rightarrow\rea$,
\begin{eqnarray*}
\esp\left[\sum_{n=1}^{N\left(\right)}f\left(T_{n}\right)\right]=\int_{\left(0,t\right]}f\left(s\right)d\eta\left(s\right)\textrm{,  }t\geq0,
\end{eqnarray*}
suponiendo que la integral exista. Adem\'as si $X_{1},X_{2},\ldots$ son variables aleatorias definidas en el mismo espacio de probabilidad que el proceso $N\left(t\right)$ tal que $\esp\left[X_{n}|T_{n}=s\right]=f\left(s\right)$, independiente de $n$. Entonces
\begin{eqnarray*}
\esp\left[\sum_{n=1}^{N\left(t\right)}X_{n}\right]=\int_{\left(0,t\right]}f\left(s\right)d\eta\left(s\right)\textrm{,  }t\geq0,
\end{eqnarray*} 
suponiendo que la integral exista. 
\end{Teo}

\begin{Coro}[Identidad de Wald para Renovaciones]
Para el proceso de renovaci\'on $N\left(t\right)$,
\begin{eqnarray*}
\esp\left[T_{N\left(t\right)+1}\right]=\mu\esp\left[N\left(t\right)+1\right]\textrm{,  }t\geq0,
\end{eqnarray*}  
\end{Coro}


\begin{Def}
Sea $h\left(t\right)$ funci\'on de valores reales en $\rea$ acotada en intervalos finitos e igual a cero para $t<0$ La ecuaci\'on de renovaci\'on para $h\left(t\right)$ y la distribuci\'on $F$ es

\begin{eqnarray}\label{Ec.Renovacion}
H\left(t\right)=h\left(t\right)+\int_{\left[0,t\right]}H\left(t-s\right)dF\left(s\right)\textrm{,    }t\geq0,
\end{eqnarray}
donde $H\left(t\right)$ es una funci\'on de valores reales. Esto es $H=h+F\star H$. Decimos que $H\left(t\right)$ es soluci\'on de esta ecuaci\'on si satisface la ecuaci\'on, y es acotada en intervalos finitos e iguales a cero para $t<0$.
\end{Def}

\begin{Prop}
La funci\'on $U\star h\left(t\right)$ es la \'unica soluci\'on de la ecuaci\'on de renovaci\'on (\ref{Ec.Renovacion}).
\end{Prop}

\begin{Teo}[Teorema Renovaci\'on Elemental]
\begin{eqnarray*}
t^{-1}U\left(t\right)\rightarrow 1/\mu\textrm{,    cuando }t\rightarrow\infty.
\end{eqnarray*}
\end{Teo}



Sup\'ongase que $N\left(t\right)$ es un proceso de renovaci\'on con distribuci\'on $F$ con media finita $\mu$.

\begin{Def}
La funci\'on de renovaci\'on asociada con la distribuci\'on $F$, del proceso $N\left(t\right)$, es
\begin{eqnarray*}
U\left(t\right)=\sum_{n=1}^{\infty}F^{n\star}\left(t\right),\textrm{   }t\geq0,
\end{eqnarray*}
donde $F^{0\star}\left(t\right)=\indora\left(t\geq0\right)$.
\end{Def}


\begin{Prop}
Sup\'ongase que la distribuci\'on de inter-renovaci\'on $F$ tiene densidad $f$. Entonces $U\left(t\right)$ tambi\'en tiene densidad, para $t>0$, y es $U^{'}\left(t\right)=\sum_{n=0}^{\infty}f^{n\star}\left(t\right)$. Adem\'as
\begin{eqnarray*}
\prob\left\{N\left(t\right)>N\left(t-\right)\right\}=0\textrm{,   }t\geq0.
\end{eqnarray*}
\end{Prop}

\begin{Def}
La Transformada de Laplace-Stieljes de $F$ est\'a dada por

\begin{eqnarray*}
\hat{F}\left(\alpha\right)=\int_{\rea_{+}}e^{-\alpha t}dF\left(t\right)\textrm{,  }\alpha\geq0.
\end{eqnarray*}
\end{Def}

Entonces

\begin{eqnarray*}
\hat{U}\left(\alpha\right)=\sum_{n=0}^{\infty}\hat{F^{n\star}}\left(\alpha\right)=\sum_{n=0}^{\infty}\hat{F}\left(\alpha\right)^{n}=\frac{1}{1-\hat{F}\left(\alpha\right)}.
\end{eqnarray*}


\begin{Prop}
La Transformada de Laplace $\hat{U}\left(\alpha\right)$ y $\hat{F}\left(\alpha\right)$ determina una a la otra de manera \'unica por la relaci\'on $\hat{U}\left(\alpha\right)=\frac{1}{1-\hat{F}\left(\alpha\right)}$.
\end{Prop}


\begin{Note}
Un proceso de renovaci\'on $N\left(t\right)$ cuyos tiempos de inter-renovaci\'on tienen media finita, es un proceso Poisson con tasa $\lambda$ si y s\'olo s\'i $\esp\left[U\left(t\right)\right]=\lambda t$, para $t\geq0$.
\end{Note}


\begin{Teo}
Sea $N\left(t\right)$ un proceso puntual simple con puntos de localizaci\'on $T_{n}$ tal que $\eta\left(t\right)=\esp\left[N\left(\right)\right]$ es finita para cada $t$. Entonces para cualquier funci\'on $f:\rea_{+}\rightarrow\rea$,
\begin{eqnarray*}
\esp\left[\sum_{n=1}^{N\left(\right)}f\left(T_{n}\right)\right]=\int_{\left(0,t\right]}f\left(s\right)d\eta\left(s\right)\textrm{,  }t\geq0,
\end{eqnarray*}
suponiendo que la integral exista. Adem\'as si $X_{1},X_{2},\ldots$ son variables aleatorias definidas en el mismo espacio de probabilidad que el proceso $N\left(t\right)$ tal que $\esp\left[X_{n}|T_{n}=s\right]=f\left(s\right)$, independiente de $n$. Entonces
\begin{eqnarray*}
\esp\left[\sum_{n=1}^{N\left(t\right)}X_{n}\right]=\int_{\left(0,t\right]}f\left(s\right)d\eta\left(s\right)\textrm{,  }t\geq0,
\end{eqnarray*} 
suponiendo que la integral exista. 
\end{Teo}

\begin{Coro}[Identidad de Wald para Renovaciones]
Para el proceso de renovaci\'on $N\left(t\right)$,
\begin{eqnarray*}
\esp\left[T_{N\left(t\right)+1}\right]=\mu\esp\left[N\left(t\right)+1\right]\textrm{,  }t\geq0,
\end{eqnarray*}  
\end{Coro}


\begin{Def}
Sea $h\left(t\right)$ funci\'on de valores reales en $\rea$ acotada en intervalos finitos e igual a cero para $t<0$ La ecuaci\'on de renovaci\'on para $h\left(t\right)$ y la distribuci\'on $F$ es

\begin{eqnarray}\label{Ec.Renovacion}
H\left(t\right)=h\left(t\right)+\int_{\left[0,t\right]}H\left(t-s\right)dF\left(s\right)\textrm{,    }t\geq0,
\end{eqnarray}
donde $H\left(t\right)$ es una funci\'on de valores reales. Esto es $H=h+F\star H$. Decimos que $H\left(t\right)$ es soluci\'on de esta ecuaci\'on si satisface la ecuaci\'on, y es acotada en intervalos finitos e iguales a cero para $t<0$.
\end{Def}

\begin{Prop}
La funci\'on $U\star h\left(t\right)$ es la \'unica soluci\'on de la ecuaci\'on de renovaci\'on (\ref{Ec.Renovacion}).
\end{Prop}

\begin{Teo}[Teorema Renovaci\'on Elemental]
\begin{eqnarray*}
t^{-1}U\left(t\right)\rightarrow 1/\mu\textrm{,    cuando }t\rightarrow\infty.
\end{eqnarray*}
\end{Teo}


\begin{Note} Una funci\'on $h:\rea_{+}\rightarrow\rea$ es Directamente Riemann Integrable en los siguientes casos:
\begin{itemize}
\item[a)] $h\left(t\right)\geq0$ es decreciente y Riemann Integrable.
\item[b)] $h$ es continua excepto posiblemente en un conjunto de Lebesgue de medida 0, y $|h\left(t\right)|\leq b\left(t\right)$, donde $b$ es DRI.
\end{itemize}
\end{Note}

\begin{Teo}[Teorema Principal de Renovaci\'on]
Si $F$ es no aritm\'etica y $h\left(t\right)$ es Directamente Riemann Integrable (DRI), entonces

\begin{eqnarray*}
lim_{t\rightarrow\infty}U\star h=\frac{1}{\mu}\int_{\rea_{+}}h\left(s\right)ds.
\end{eqnarray*}
\end{Teo}

\begin{Prop}
Cualquier funci\'on $H\left(t\right)$ acotada en intervalos finitos y que es 0 para $t<0$ puede expresarse como
\begin{eqnarray*}
H\left(t\right)=U\star h\left(t\right)\textrm{,  donde }h\left(t\right)=H\left(t\right)-F\star H\left(t\right)
\end{eqnarray*}
\end{Prop}

\begin{Def}
Un proceso estoc\'astico $X\left(t\right)$ es crudamente regenerativo en un tiempo aleatorio positivo $T$ si
\begin{eqnarray*}
\esp\left[X\left(T+t\right)|T\right]=\esp\left[X\left(t\right)\right]\textrm{, para }t\geq0,\end{eqnarray*}
y con las esperanzas anteriores finitas.
\end{Def}

\begin{Prop}
Sup\'ongase que $X\left(t\right)$ es un proceso crudamente regenerativo en $T$, que tiene distribuci\'on $F$. Si $\esp\left[X\left(t\right)\right]$ es acotado en intervalos finitos, entonces
\begin{eqnarray*}
\esp\left[X\left(t\right)\right]=U\star h\left(t\right)\textrm{,  donde }h\left(t\right)=\esp\left[X\left(t\right)\indora\left(T>t\right)\right].
\end{eqnarray*}
\end{Prop}

\begin{Teo}[Regeneraci\'on Cruda]
Sup\'ongase que $X\left(t\right)$ es un proceso con valores positivo crudamente regenerativo en $T$, y def\'inase $M=\sup\left\{|X\left(t\right)|:t\leq T\right\}$. Si $T$ es no aritm\'etico y $M$ y $MT$ tienen media finita, entonces
\begin{eqnarray*}
lim_{t\rightarrow\infty}\esp\left[X\left(t\right)\right]=\frac{1}{\mu}\int_{\rea_{+}}h\left(s\right)ds,
\end{eqnarray*}
donde $h\left(t\right)=\esp\left[X\left(t\right)\indora\left(T>t\right)\right]$.
\end{Teo}


\begin{Note} Una funci\'on $h:\rea_{+}\rightarrow\rea$ es Directamente Riemann Integrable en los siguientes casos:
\begin{itemize}
\item[a)] $h\left(t\right)\geq0$ es decreciente y Riemann Integrable.
\item[b)] $h$ es continua excepto posiblemente en un conjunto de Lebesgue de medida 0, y $|h\left(t\right)|\leq b\left(t\right)$, donde $b$ es DRI.
\end{itemize}
\end{Note}

\begin{Teo}[Teorema Principal de Renovaci\'on]
Si $F$ es no aritm\'etica y $h\left(t\right)$ es Directamente Riemann Integrable (DRI), entonces

\begin{eqnarray*}
lim_{t\rightarrow\infty}U\star h=\frac{1}{\mu}\int_{\rea_{+}}h\left(s\right)ds.
\end{eqnarray*}
\end{Teo}

\begin{Prop}
Cualquier funci\'on $H\left(t\right)$ acotada en intervalos finitos y que es 0 para $t<0$ puede expresarse como
\begin{eqnarray*}
H\left(t\right)=U\star h\left(t\right)\textrm{,  donde }h\left(t\right)=H\left(t\right)-F\star H\left(t\right)
\end{eqnarray*}
\end{Prop}

\begin{Def}
Un proceso estoc\'astico $X\left(t\right)$ es crudamente regenerativo en un tiempo aleatorio positivo $T$ si
\begin{eqnarray*}
\esp\left[X\left(T+t\right)|T\right]=\esp\left[X\left(t\right)\right]\textrm{, para }t\geq0,\end{eqnarray*}
y con las esperanzas anteriores finitas.
\end{Def}

\begin{Prop}
Sup\'ongase que $X\left(t\right)$ es un proceso crudamente regenerativo en $T$, que tiene distribuci\'on $F$. Si $\esp\left[X\left(t\right)\right]$ es acotado en intervalos finitos, entonces
\begin{eqnarray*}
\esp\left[X\left(t\right)\right]=U\star h\left(t\right)\textrm{,  donde }h\left(t\right)=\esp\left[X\left(t\right)\indora\left(T>t\right)\right].
\end{eqnarray*}
\end{Prop}

\begin{Teo}[Regeneraci\'on Cruda]
Sup\'ongase que $X\left(t\right)$ es un proceso con valores positivo crudamente regenerativo en $T$, y def\'inase $M=\sup\left\{|X\left(t\right)|:t\leq T\right\}$. Si $T$ es no aritm\'etico y $M$ y $MT$ tienen media finita, entonces
\begin{eqnarray*}
lim_{t\rightarrow\infty}\esp\left[X\left(t\right)\right]=\frac{1}{\mu}\int_{\rea_{+}}h\left(s\right)ds,
\end{eqnarray*}
donde $h\left(t\right)=\esp\left[X\left(t\right)\indora\left(T>t\right)\right]$.
\end{Teo}

%__________________________________________________________________________________________
\subsection{Procesos Regenerativos Estacionarios - Stidham \cite{Stidham}}
%__________________________________________________________________________________________


Un proceso estoc\'astico a tiempo continuo $\left\{V\left(t\right),t\geq0\right\}$ es un proceso regenerativo si existe una sucesi\'on de variables aleatorias independientes e id\'enticamente distribuidas $\left\{X_{1},X_{2},\ldots\right\}$, sucesi\'on de renovaci\'on, tal que para cualquier conjunto de Borel $A$, 

\begin{eqnarray*}
\prob\left\{V\left(t\right)\in A|X_{1}+X_{2}+\cdots+X_{R\left(t\right)}=s,\left\{V\left(\tau\right),\tau<s\right\}\right\}=\prob\left\{V\left(t-s\right)\in A|X_{1}>t-s\right\},
\end{eqnarray*}
para todo $0\leq s\leq t$, donde $R\left(t\right)=\max\left\{X_{1}+X_{2}+\cdots+X_{j}\leq t\right\}=$n\'umero de renovaciones ({\emph{puntos de regeneraci\'on}}) que ocurren en $\left[0,t\right]$. El intervalo $\left[0,X_{1}\right)$ es llamado {\emph{primer ciclo de regeneraci\'on}} de $\left\{V\left(t \right),t\geq0\right\}$, $\left[X_{1},X_{1}+X_{2}\right)$ el {\emph{segundo ciclo de regeneraci\'on}}, y as\'i sucesivamente.

Sea $X=X_{1}$ y sea $F$ la funci\'on de distrbuci\'on de $X$


\begin{Def}
Se define el proceso estacionario, $\left\{V^{*}\left(t\right),t\geq0\right\}$, para $\left\{V\left(t\right),t\geq0\right\}$ por

\begin{eqnarray*}
\prob\left\{V\left(t\right)\in A\right\}=\frac{1}{\esp\left[X\right]}\int_{0}^{\infty}\prob\left\{V\left(t+x\right)\in A|X>x\right\}\left(1-F\left(x\right)\right)dx,
\end{eqnarray*} 
para todo $t\geq0$ y todo conjunto de Borel $A$.
\end{Def}

\begin{Def}
Una distribuci\'on se dice que es {\emph{aritm\'etica}} si todos sus puntos de incremento son m\'ultiplos de la forma $0,\lambda, 2\lambda,\ldots$ para alguna $\lambda>0$ entera.
\end{Def}


\begin{Def}
Una modificaci\'on medible de un proceso $\left\{V\left(t\right),t\geq0\right\}$, es una versi\'on de este, $\left\{V\left(t,w\right)\right\}$ conjuntamente medible para $t\geq0$ y para $w\in S$, $S$ espacio de estados para $\left\{V\left(t\right),t\geq0\right\}$.
\end{Def}

\begin{Teo}
Sea $\left\{V\left(t\right),t\geq\right\}$ un proceso regenerativo no negativo con modificaci\'on medible. Sea $\esp\left[X\right]<\infty$. Entonces el proceso estacionario dado por la ecuaci\'on anterior est\'a bien definido y tiene funci\'on de distribuci\'on independiente de $t$, adem\'as
\begin{itemize}
\item[i)] \begin{eqnarray*}
\esp\left[V^{*}\left(0\right)\right]&=&\frac{\esp\left[\int_{0}^{X}V\left(s\right)ds\right]}{\esp\left[X\right]}\end{eqnarray*}
\item[ii)] Si $\esp\left[V^{*}\left(0\right)\right]<\infty$, equivalentemente, si $\esp\left[\int_{0}^{X}V\left(s\right)ds\right]<\infty$,entonces
\begin{eqnarray*}
\frac{\int_{0}^{t}V\left(s\right)ds}{t}\rightarrow\frac{\esp\left[\int_{0}^{X}V\left(s\right)ds\right]}{\esp\left[X\right]}
\end{eqnarray*}
con probabilidad 1 y en media, cuando $t\rightarrow\infty$.
\end{itemize}
\end{Teo}


%__________________________________________________________________________________________
\subsection{Procesos Regenerativos Estacionarios - Stidham \cite{Stidham}}
%__________________________________________________________________________________________


Un proceso estoc\'astico a tiempo continuo $\left\{V\left(t\right),t\geq0\right\}$ es un proceso regenerativo si existe una sucesi\'on de variables aleatorias independientes e id\'enticamente distribuidas $\left\{X_{1},X_{2},\ldots\right\}$, sucesi\'on de renovaci\'on, tal que para cualquier conjunto de Borel $A$, 

\begin{eqnarray*}
\prob\left\{V\left(t\right)\in A|X_{1}+X_{2}+\cdots+X_{R\left(t\right)}=s,\left\{V\left(\tau\right),\tau<s\right\}\right\}=\prob\left\{V\left(t-s\right)\in A|X_{1}>t-s\right\},
\end{eqnarray*}
para todo $0\leq s\leq t$, donde $R\left(t\right)=\max\left\{X_{1}+X_{2}+\cdots+X_{j}\leq t\right\}=$n\'umero de renovaciones ({\emph{puntos de regeneraci\'on}}) que ocurren en $\left[0,t\right]$. El intervalo $\left[0,X_{1}\right)$ es llamado {\emph{primer ciclo de regeneraci\'on}} de $\left\{V\left(t \right),t\geq0\right\}$, $\left[X_{1},X_{1}+X_{2}\right)$ el {\emph{segundo ciclo de regeneraci\'on}}, y as\'i sucesivamente.

Sea $X=X_{1}$ y sea $F$ la funci\'on de distrbuci\'on de $X$


\begin{Def}
Se define el proceso estacionario, $\left\{V^{*}\left(t\right),t\geq0\right\}$, para $\left\{V\left(t\right),t\geq0\right\}$ por

\begin{eqnarray*}
\prob\left\{V\left(t\right)\in A\right\}=\frac{1}{\esp\left[X\right]}\int_{0}^{\infty}\prob\left\{V\left(t+x\right)\in A|X>x\right\}\left(1-F\left(x\right)\right)dx,
\end{eqnarray*} 
para todo $t\geq0$ y todo conjunto de Borel $A$.
\end{Def}

\begin{Def}
Una distribuci\'on se dice que es {\emph{aritm\'etica}} si todos sus puntos de incremento son m\'ultiplos de la forma $0,\lambda, 2\lambda,\ldots$ para alguna $\lambda>0$ entera.
\end{Def}


\begin{Def}
Una modificaci\'on medible de un proceso $\left\{V\left(t\right),t\geq0\right\}$, es una versi\'on de este, $\left\{V\left(t,w\right)\right\}$ conjuntamente medible para $t\geq0$ y para $w\in S$, $S$ espacio de estados para $\left\{V\left(t\right),t\geq0\right\}$.
\end{Def}

\begin{Teo}
Sea $\left\{V\left(t\right),t\geq\right\}$ un proceso regenerativo no negativo con modificaci\'on medible. Sea $\esp\left[X\right]<\infty$. Entonces el proceso estacionario dado por la ecuaci\'on anterior est\'a bien definido y tiene funci\'on de distribuci\'on independiente de $t$, adem\'as
\begin{itemize}
\item[i)] \begin{eqnarray*}
\esp\left[V^{*}\left(0\right)\right]&=&\frac{\esp\left[\int_{0}^{X}V\left(s\right)ds\right]}{\esp\left[X\right]}\end{eqnarray*}
\item[ii)] Si $\esp\left[V^{*}\left(0\right)\right]<\infty$, equivalentemente, si $\esp\left[\int_{0}^{X}V\left(s\right)ds\right]<\infty$,entonces
\begin{eqnarray*}
\frac{\int_{0}^{t}V\left(s\right)ds}{t}\rightarrow\frac{\esp\left[\int_{0}^{X}V\left(s\right)ds\right]}{\esp\left[X\right]}
\end{eqnarray*}
con probabilidad 1 y en media, cuando $t\rightarrow\infty$.
\end{itemize}
\end{Teo}

%__________________________________________________________________________________________
\subsection{Procesos Regenerativos Estacionarios - Stidham \cite{Stidham}}
%__________________________________________________________________________________________


Un proceso estoc\'astico a tiempo continuo $\left\{V\left(t\right),t\geq0\right\}$ es un proceso regenerativo si existe una sucesi\'on de variables aleatorias independientes e id\'enticamente distribuidas $\left\{X_{1},X_{2},\ldots\right\}$, sucesi\'on de renovaci\'on, tal que para cualquier conjunto de Borel $A$, 

\begin{eqnarray*}
\prob\left\{V\left(t\right)\in A|X_{1}+X_{2}+\cdots+X_{R\left(t\right)}=s,\left\{V\left(\tau\right),\tau<s\right\}\right\}=\prob\left\{V\left(t-s\right)\in A|X_{1}>t-s\right\},
\end{eqnarray*}
para todo $0\leq s\leq t$, donde $R\left(t\right)=\max\left\{X_{1}+X_{2}+\cdots+X_{j}\leq t\right\}=$n\'umero de renovaciones ({\emph{puntos de regeneraci\'on}}) que ocurren en $\left[0,t\right]$. El intervalo $\left[0,X_{1}\right)$ es llamado {\emph{primer ciclo de regeneraci\'on}} de $\left\{V\left(t \right),t\geq0\right\}$, $\left[X_{1},X_{1}+X_{2}\right)$ el {\emph{segundo ciclo de regeneraci\'on}}, y as\'i sucesivamente.

Sea $X=X_{1}$ y sea $F$ la funci\'on de distrbuci\'on de $X$


\begin{Def}
Se define el proceso estacionario, $\left\{V^{*}\left(t\right),t\geq0\right\}$, para $\left\{V\left(t\right),t\geq0\right\}$ por

\begin{eqnarray*}
\prob\left\{V\left(t\right)\in A\right\}=\frac{1}{\esp\left[X\right]}\int_{0}^{\infty}\prob\left\{V\left(t+x\right)\in A|X>x\right\}\left(1-F\left(x\right)\right)dx,
\end{eqnarray*} 
para todo $t\geq0$ y todo conjunto de Borel $A$.
\end{Def}

\begin{Def}
Una distribuci\'on se dice que es {\emph{aritm\'etica}} si todos sus puntos de incremento son m\'ultiplos de la forma $0,\lambda, 2\lambda,\ldots$ para alguna $\lambda>0$ entera.
\end{Def}


\begin{Def}
Una modificaci\'on medible de un proceso $\left\{V\left(t\right),t\geq0\right\}$, es una versi\'on de este, $\left\{V\left(t,w\right)\right\}$ conjuntamente medible para $t\geq0$ y para $w\in S$, $S$ espacio de estados para $\left\{V\left(t\right),t\geq0\right\}$.
\end{Def}

\begin{Teo}
Sea $\left\{V\left(t\right),t\geq\right\}$ un proceso regenerativo no negativo con modificaci\'on medible. Sea $\esp\left[X\right]<\infty$. Entonces el proceso estacionario dado por la ecuaci\'on anterior est\'a bien definido y tiene funci\'on de distribuci\'on independiente de $t$, adem\'as
\begin{itemize}
\item[i)] \begin{eqnarray*}
\esp\left[V^{*}\left(0\right)\right]&=&\frac{\esp\left[\int_{0}^{X}V\left(s\right)ds\right]}{\esp\left[X\right]}\end{eqnarray*}
\item[ii)] Si $\esp\left[V^{*}\left(0\right)\right]<\infty$, equivalentemente, si $\esp\left[\int_{0}^{X}V\left(s\right)ds\right]<\infty$,entonces
\begin{eqnarray*}
\frac{\int_{0}^{t}V\left(s\right)ds}{t}\rightarrow\frac{\esp\left[\int_{0}^{X}V\left(s\right)ds\right]}{\esp\left[X\right]}
\end{eqnarray*}
con probabilidad 1 y en media, cuando $t\rightarrow\infty$.
\end{itemize}
\end{Teo}

%______________________________________________________________________
\subsection{Procesos de Renovaci\'on}
%______________________________________________________________________

\begin{Def}\label{Def.Tn}
Sean $0\leq T_{1}\leq T_{2}\leq \ldots$ son tiempos aleatorios infinitos en los cuales ocurren ciertos eventos. El n\'umero de tiempos $T_{n}$ en el intervalo $\left[0,t\right)$ es

\begin{eqnarray}
N\left(t\right)=\sum_{n=1}^{\infty}\indora\left(T_{n}\leq t\right),
\end{eqnarray}
para $t\geq0$.
\end{Def}

Si se consideran los puntos $T_{n}$ como elementos de $\rea_{+}$, y $N\left(t\right)$ es el n\'umero de puntos en $\rea$. El proceso denotado por $\left\{N\left(t\right):t\geq0\right\}$, denotado por $N\left(t\right)$, es un proceso puntual en $\rea_{+}$. Los $T_{n}$ son los tiempos de ocurrencia, el proceso puntual $N\left(t\right)$ es simple si su n\'umero de ocurrencias son distintas: $0<T_{1}<T_{2}<\ldots$ casi seguramente.

\begin{Def}
Un proceso puntual $N\left(t\right)$ es un proceso de renovaci\'on si los tiempos de interocurrencia $\xi_{n}=T_{n}-T_{n-1}$, para $n\geq1$, son independientes e identicamente distribuidos con distribuci\'on $F$, donde $F\left(0\right)=0$ y $T_{0}=0$. Los $T_{n}$ son llamados tiempos de renovaci\'on, referente a la independencia o renovaci\'on de la informaci\'on estoc\'astica en estos tiempos. Los $\xi_{n}$ son los tiempos de inter-renovaci\'on, y $N\left(t\right)$ es el n\'umero de renovaciones en el intervalo $\left[0,t\right)$
\end{Def}


\begin{Note}
Para definir un proceso de renovaci\'on para cualquier contexto, solamente hay que especificar una distribuci\'on $F$, con $F\left(0\right)=0$, para los tiempos de inter-renovaci\'on. La funci\'on $F$ en turno degune las otra variables aleatorias. De manera formal, existe un espacio de probabilidad y una sucesi\'on de variables aleatorias $\xi_{1},\xi_{2},\ldots$ definidas en este con distribuci\'on $F$. Entonces las otras cantidades son $T_{n}=\sum_{k=1}^{n}\xi_{k}$ y $N\left(t\right)=\sum_{n=1}^{\infty}\indora\left(T_{n}\leq t\right)$, donde $T_{n}\rightarrow\infty$ casi seguramente por la Ley Fuerte de los Grandes Números.
\end{Note}

%___________________________________________________________________________________________
%
\subsection{Teorema Principal de Renovaci\'on}
%___________________________________________________________________________________________
%

\begin{Note} Una funci\'on $h:\rea_{+}\rightarrow\rea$ es Directamente Riemann Integrable en los siguientes casos:
\begin{itemize}
\item[a)] $h\left(t\right)\geq0$ es decreciente y Riemann Integrable.
\item[b)] $h$ es continua excepto posiblemente en un conjunto de Lebesgue de medida 0, y $|h\left(t\right)|\leq b\left(t\right)$, donde $b$ es DRI.
\end{itemize}
\end{Note}

\begin{Teo}[Teorema Principal de Renovaci\'on]
Si $F$ es no aritm\'etica y $h\left(t\right)$ es Directamente Riemann Integrable (DRI), entonces

\begin{eqnarray*}
lim_{t\rightarrow\infty}U\star h=\frac{1}{\mu}\int_{\rea_{+}}h\left(s\right)ds.
\end{eqnarray*}
\end{Teo}

\begin{Prop}
Cualquier funci\'on $H\left(t\right)$ acotada en intervalos finitos y que es 0 para $t<0$ puede expresarse como
\begin{eqnarray*}
H\left(t\right)=U\star h\left(t\right)\textrm{,  donde }h\left(t\right)=H\left(t\right)-F\star H\left(t\right)
\end{eqnarray*}
\end{Prop}

\begin{Def}
Un proceso estoc\'astico $X\left(t\right)$ es crudamente regenerativo en un tiempo aleatorio positivo $T$ si
\begin{eqnarray*}
\esp\left[X\left(T+t\right)|T\right]=\esp\left[X\left(t\right)\right]\textrm{, para }t\geq0,\end{eqnarray*}
y con las esperanzas anteriores finitas.
\end{Def}

\begin{Prop}
Sup\'ongase que $X\left(t\right)$ es un proceso crudamente regenerativo en $T$, que tiene distribuci\'on $F$. Si $\esp\left[X\left(t\right)\right]$ es acotado en intervalos finitos, entonces
\begin{eqnarray*}
\esp\left[X\left(t\right)\right]=U\star h\left(t\right)\textrm{,  donde }h\left(t\right)=\esp\left[X\left(t\right)\indora\left(T>t\right)\right].
\end{eqnarray*}
\end{Prop}

\begin{Teo}[Regeneraci\'on Cruda]
Sup\'ongase que $X\left(t\right)$ es un proceso con valores positivo crudamente regenerativo en $T$, y def\'inase $M=\sup\left\{|X\left(t\right)|:t\leq T\right\}$. Si $T$ es no aritm\'etico y $M$ y $MT$ tienen media finita, entonces
\begin{eqnarray*}
lim_{t\rightarrow\infty}\esp\left[X\left(t\right)\right]=\frac{1}{\mu}\int_{\rea_{+}}h\left(s\right)ds,
\end{eqnarray*}
donde $h\left(t\right)=\esp\left[X\left(t\right)\indora\left(T>t\right)\right]$.
\end{Teo}



%___________________________________________________________________________________________
%
\subsection{Funci\'on de Renovaci\'on}
%___________________________________________________________________________________________
%


\begin{Def}
Sea $h\left(t\right)$ funci\'on de valores reales en $\rea$ acotada en intervalos finitos e igual a cero para $t<0$ La ecuaci\'on de renovaci\'on para $h\left(t\right)$ y la distribuci\'on $F$ es

\begin{eqnarray}\label{Ec.Renovacion}
H\left(t\right)=h\left(t\right)+\int_{\left[0,t\right]}H\left(t-s\right)dF\left(s\right)\textrm{,    }t\geq0,
\end{eqnarray}
donde $H\left(t\right)$ es una funci\'on de valores reales. Esto es $H=h+F\star H$. Decimos que $H\left(t\right)$ es soluci\'on de esta ecuaci\'on si satisface la ecuaci\'on, y es acotada en intervalos finitos e iguales a cero para $t<0$.
\end{Def}

\begin{Prop}
La funci\'on $U\star h\left(t\right)$ es la \'unica soluci\'on de la ecuaci\'on de renovaci\'on (\ref{Ec.Renovacion}).
\end{Prop}

\begin{Teo}[Teorema Renovaci\'on Elemental]
\begin{eqnarray*}
t^{-1}U\left(t\right)\rightarrow 1/\mu\textrm{,    cuando }t\rightarrow\infty.
\end{eqnarray*}
\end{Teo}

%___________________________________________________________________________________________
%
\subsection{Propiedades de los Procesos de Renovaci\'on}
%___________________________________________________________________________________________
%

Los tiempos $T_{n}$ est\'an relacionados con los conteos de $N\left(t\right)$ por

\begin{eqnarray*}
\left\{N\left(t\right)\geq n\right\}&=&\left\{T_{n}\leq t\right\}\\
T_{N\left(t\right)}\leq &t&<T_{N\left(t\right)+1},
\end{eqnarray*}

adem\'as $N\left(T_{n}\right)=n$, y 

\begin{eqnarray*}
N\left(t\right)=\max\left\{n:T_{n}\leq t\right\}=\min\left\{n:T_{n+1}>t\right\}
\end{eqnarray*}

Por propiedades de la convoluci\'on se sabe que

\begin{eqnarray*}
P\left\{T_{n}\leq t\right\}=F^{n\star}\left(t\right)
\end{eqnarray*}
que es la $n$-\'esima convoluci\'on de $F$. Entonces 

\begin{eqnarray*}
\left\{N\left(t\right)\geq n\right\}&=&\left\{T_{n}\leq t\right\}\\
P\left\{N\left(t\right)\leq n\right\}&=&1-F^{\left(n+1\right)\star}\left(t\right)
\end{eqnarray*}

Adem\'as usando el hecho de que $\esp\left[N\left(t\right)\right]=\sum_{n=1}^{\infty}P\left\{N\left(t\right)\geq n\right\}$
se tiene que

\begin{eqnarray*}
\esp\left[N\left(t\right)\right]=\sum_{n=1}^{\infty}F^{n\star}\left(t\right)
\end{eqnarray*}

\begin{Prop}
Para cada $t\geq0$, la funci\'on generadora de momentos $\esp\left[e^{\alpha N\left(t\right)}\right]$ existe para alguna $\alpha$ en una vecindad del 0, y de aqu\'i que $\esp\left[N\left(t\right)^{m}\right]<\infty$, para $m\geq1$.
\end{Prop}


\begin{Note}
Si el primer tiempo de renovaci\'on $\xi_{1}$ no tiene la misma distribuci\'on que el resto de las $\xi_{n}$, para $n\geq2$, a $N\left(t\right)$ se le llama Proceso de Renovaci\'on retardado, donde si $\xi$ tiene distribuci\'on $G$, entonces el tiempo $T_{n}$ de la $n$-\'esima renovaci\'on tiene distribuci\'on $G\star F^{\left(n-1\right)\star}\left(t\right)$
\end{Note}


\begin{Teo}
Para una constante $\mu\leq\infty$ ( o variable aleatoria), las siguientes expresiones son equivalentes:

\begin{eqnarray}
lim_{n\rightarrow\infty}n^{-1}T_{n}&=&\mu,\textrm{ c.s.}\\
lim_{t\rightarrow\infty}t^{-1}N\left(t\right)&=&1/\mu,\textrm{ c.s.}
\end{eqnarray}
\end{Teo}


Es decir, $T_{n}$ satisface la Ley Fuerte de los Grandes N\'umeros s\'i y s\'olo s\'i $N\left/t\right)$ la cumple.


\begin{Coro}[Ley Fuerte de los Grandes N\'umeros para Procesos de Renovaci\'on]
Si $N\left(t\right)$ es un proceso de renovaci\'on cuyos tiempos de inter-renovaci\'on tienen media $\mu\leq\infty$, entonces
\begin{eqnarray}
t^{-1}N\left(t\right)\rightarrow 1/\mu,\textrm{ c.s. cuando }t\rightarrow\infty.
\end{eqnarray}

\end{Coro}


Considerar el proceso estoc\'astico de valores reales $\left\{Z\left(t\right):t\geq0\right\}$ en el mismo espacio de probabilidad que $N\left(t\right)$

\begin{Def}
Para el proceso $\left\{Z\left(t\right):t\geq0\right\}$ se define la fluctuaci\'on m\'axima de $Z\left(t\right)$ en el intervalo $\left(T_{n-1},T_{n}\right]$:
\begin{eqnarray*}
M_{n}=\sup_{T_{n-1}<t\leq T_{n}}|Z\left(t\right)-Z\left(T_{n-1}\right)|
\end{eqnarray*}
\end{Def}

\begin{Teo}
Sup\'ongase que $n^{-1}T_{n}\rightarrow\mu$ c.s. cuando $n\rightarrow\infty$, donde $\mu\leq\infty$ es una constante o variable aleatoria. Sea $a$ una constante o variable aleatoria que puede ser infinita cuando $\mu$ es finita, y considere las expresiones l\'imite:
\begin{eqnarray}
lim_{n\rightarrow\infty}n^{-1}Z\left(T_{n}\right)&=&a,\textrm{ c.s.}\\
lim_{t\rightarrow\infty}t^{-1}Z\left(t\right)&=&a/\mu,\textrm{ c.s.}
\end{eqnarray}
La segunda expresi\'on implica la primera. Conversamente, la primera implica la segunda si el proceso $Z\left(t\right)$ es creciente, o si $lim_{n\rightarrow\infty}n^{-1}M_{n}=0$ c.s.
\end{Teo}

\begin{Coro}
Si $N\left(t\right)$ es un proceso de renovaci\'on, y $\left(Z\left(T_{n}\right)-Z\left(T_{n-1}\right),M_{n}\right)$, para $n\geq1$, son variables aleatorias independientes e id\'enticamente distribuidas con media finita, entonces,
\begin{eqnarray}
lim_{t\rightarrow\infty}t^{-1}Z\left(t\right)\rightarrow\frac{\esp\left[Z\left(T_{1}\right)-Z\left(T_{0}\right)\right]}{\esp\left[T_{1}\right]},\textrm{ c.s. cuando  }t\rightarrow\infty.
\end{eqnarray}
\end{Coro}

%___________________________________________________________________________________________
%
\subsection{Funci\'on de Renovaci\'on}
%___________________________________________________________________________________________
%


Sup\'ongase que $N\left(t\right)$ es un proceso de renovaci\'on con distribuci\'on $F$ con media finita $\mu$.

\begin{Def}
La funci\'on de renovaci\'on asociada con la distribuci\'on $F$, del proceso $N\left(t\right)$, es
\begin{eqnarray*}
U\left(t\right)=\sum_{n=1}^{\infty}F^{n\star}\left(t\right),\textrm{   }t\geq0,
\end{eqnarray*}
donde $F^{0\star}\left(t\right)=\indora\left(t\geq0\right)$.
\end{Def}


\begin{Prop}
Sup\'ongase que la distribuci\'on de inter-renovaci\'on $F$ tiene densidad $f$. Entonces $U\left(t\right)$ tambi\'en tiene densidad, para $t>0$, y es $U^{'}\left(t\right)=\sum_{n=0}^{\infty}f^{n\star}\left(t\right)$. Adem\'as
\begin{eqnarray*}
\prob\left\{N\left(t\right)>N\left(t-\right)\right\}=0\textrm{,   }t\geq0.
\end{eqnarray*}
\end{Prop}

\begin{Def}
La Transformada de Laplace-Stieljes de $F$ est\'a dada por

\begin{eqnarray*}
\hat{F}\left(\alpha\right)=\int_{\rea_{+}}e^{-\alpha t}dF\left(t\right)\textrm{,  }\alpha\geq0.
\end{eqnarray*}
\end{Def}

Entonces

\begin{eqnarray*}
\hat{U}\left(\alpha\right)=\sum_{n=0}^{\infty}\hat{F^{n\star}}\left(\alpha\right)=\sum_{n=0}^{\infty}\hat{F}\left(\alpha\right)^{n}=\frac{1}{1-\hat{F}\left(\alpha\right)}.
\end{eqnarray*}


\begin{Prop}
La Transformada de Laplace $\hat{U}\left(\alpha\right)$ y $\hat{F}\left(\alpha\right)$ determina una a la otra de manera \'unica por la relaci\'on $\hat{U}\left(\alpha\right)=\frac{1}{1-\hat{F}\left(\alpha\right)}$.
\end{Prop}


\begin{Note}
Un proceso de renovaci\'on $N\left(t\right)$ cuyos tiempos de inter-renovaci\'on tienen media finita, es un proceso Poisson con tasa $\lambda$ si y s\'olo s\'i $\esp\left[U\left(t\right)\right]=\lambda t$, para $t\geq0$.
\end{Note}


\begin{Teo}
Sea $N\left(t\right)$ un proceso puntual simple con puntos de localizaci\'on $T_{n}$ tal que $\eta\left(t\right)=\esp\left[N\left(\right)\right]$ es finita para cada $t$. Entonces para cualquier funci\'on $f:\rea_{+}\rightarrow\rea$,
\begin{eqnarray*}
\esp\left[\sum_{n=1}^{N\left(\right)}f\left(T_{n}\right)\right]=\int_{\left(0,t\right]}f\left(s\right)d\eta\left(s\right)\textrm{,  }t\geq0,
\end{eqnarray*}
suponiendo que la integral exista. Adem\'as si $X_{1},X_{2},\ldots$ son variables aleatorias definidas en el mismo espacio de probabilidad que el proceso $N\left(t\right)$ tal que $\esp\left[X_{n}|T_{n}=s\right]=f\left(s\right)$, independiente de $n$. Entonces
\begin{eqnarray*}
\esp\left[\sum_{n=1}^{N\left(t\right)}X_{n}\right]=\int_{\left(0,t\right]}f\left(s\right)d\eta\left(s\right)\textrm{,  }t\geq0,
\end{eqnarray*} 
suponiendo que la integral exista. 
\end{Teo}

\begin{Coro}[Identidad de Wald para Renovaciones]
Para el proceso de renovaci\'on $N\left(t\right)$,
\begin{eqnarray*}
\esp\left[T_{N\left(t\right)+1}\right]=\mu\esp\left[N\left(t\right)+1\right]\textrm{,  }t\geq0,
\end{eqnarray*}  
\end{Coro}

%______________________________________________________________________
\subsection{Procesos de Renovaci\'on}
%______________________________________________________________________

\begin{Def}\label{Def.Tn}
Sean $0\leq T_{1}\leq T_{2}\leq \ldots$ son tiempos aleatorios infinitos en los cuales ocurren ciertos eventos. El n\'umero de tiempos $T_{n}$ en el intervalo $\left[0,t\right)$ es

\begin{eqnarray}
N\left(t\right)=\sum_{n=1}^{\infty}\indora\left(T_{n}\leq t\right),
\end{eqnarray}
para $t\geq0$.
\end{Def}

Si se consideran los puntos $T_{n}$ como elementos de $\rea_{+}$, y $N\left(t\right)$ es el n\'umero de puntos en $\rea$. El proceso denotado por $\left\{N\left(t\right):t\geq0\right\}$, denotado por $N\left(t\right)$, es un proceso puntual en $\rea_{+}$. Los $T_{n}$ son los tiempos de ocurrencia, el proceso puntual $N\left(t\right)$ es simple si su n\'umero de ocurrencias son distintas: $0<T_{1}<T_{2}<\ldots$ casi seguramente.

\begin{Def}
Un proceso puntual $N\left(t\right)$ es un proceso de renovaci\'on si los tiempos de interocurrencia $\xi_{n}=T_{n}-T_{n-1}$, para $n\geq1$, son independientes e identicamente distribuidos con distribuci\'on $F$, donde $F\left(0\right)=0$ y $T_{0}=0$. Los $T_{n}$ son llamados tiempos de renovaci\'on, referente a la independencia o renovaci\'on de la informaci\'on estoc\'astica en estos tiempos. Los $\xi_{n}$ son los tiempos de inter-renovaci\'on, y $N\left(t\right)$ es el n\'umero de renovaciones en el intervalo $\left[0,t\right)$
\end{Def}


\begin{Note}
Para definir un proceso de renovaci\'on para cualquier contexto, solamente hay que especificar una distribuci\'on $F$, con $F\left(0\right)=0$, para los tiempos de inter-renovaci\'on. La funci\'on $F$ en turno degune las otra variables aleatorias. De manera formal, existe un espacio de probabilidad y una sucesi\'on de variables aleatorias $\xi_{1},\xi_{2},\ldots$ definidas en este con distribuci\'on $F$. Entonces las otras cantidades son $T_{n}=\sum_{k=1}^{n}\xi_{k}$ y $N\left(t\right)=\sum_{n=1}^{\infty}\indora\left(T_{n}\leq t\right)$, donde $T_{n}\rightarrow\infty$ casi seguramente por la Ley Fuerte de los Grandes Números.
\end{Note}

%________________________________________________________________________
\subsection{Procesos Regenerativos}
%________________________________________________________________________

%________________________________________________________________________
\subsection*{Procesos Regenerativos Sigman, Thorisson y Wolff \cite{Sigman1}}
%________________________________________________________________________


\begin{Def}[Definici\'on Cl\'asica]
Un proceso estoc\'astico $X=\left\{X\left(t\right):t\geq0\right\}$ es llamado regenerativo is existe una variable aleatoria $R_{1}>0$ tal que
\begin{itemize}
\item[i)] $\left\{X\left(t+R_{1}\right):t\geq0\right\}$ es independiente de $\left\{\left\{X\left(t\right):t<R_{1}\right\},\right\}$
\item[ii)] $\left\{X\left(t+R_{1}\right):t\geq0\right\}$ es estoc\'asticamente equivalente a $\left\{X\left(t\right):t>0\right\}$
\end{itemize}

Llamamos a $R_{1}$ tiempo de regeneraci\'on, y decimos que $X$ se regenera en este punto.
\end{Def}

$\left\{X\left(t+R_{1}\right)\right\}$ es regenerativo con tiempo de regeneraci\'on $R_{2}$, independiente de $R_{1}$ pero con la misma distribuci\'on que $R_{1}$. Procediendo de esta manera se obtiene una secuencia de variables aleatorias independientes e id\'enticamente distribuidas $\left\{R_{n}\right\}$ llamados longitudes de ciclo. Si definimos a $Z_{k}\equiv R_{1}+R_{2}+\cdots+R_{k}$, se tiene un proceso de renovaci\'on llamado proceso de renovaci\'on encajado para $X$.


\begin{Note}
La existencia de un primer tiempo de regeneraci\'on, $R_{1}$, implica la existencia de una sucesi\'on completa de estos tiempos $R_{1},R_{2}\ldots,$ que satisfacen la propiedad deseada \cite{Sigman2}.
\end{Note}


\begin{Note} Para la cola $GI/GI/1$ los usuarios arriban con tiempos $t_{n}$ y son atendidos con tiempos de servicio $S_{n}$, los tiempos de arribo forman un proceso de renovaci\'on  con tiempos entre arribos independientes e identicamente distribuidos (\texttt{i.i.d.})$T_{n}=t_{n}-t_{n-1}$, adem\'as los tiempos de servicio son \texttt{i.i.d.} e independientes de los procesos de arribo. Por \textit{estable} se entiende que $\esp S_{n}<\esp T_{n}<\infty$.
\end{Note}
 


\begin{Def}
Para $x$ fijo y para cada $t\geq0$, sea $I_{x}\left(t\right)=1$ si $X\left(t\right)\leq x$,  $I_{x}\left(t\right)=0$ en caso contrario, y def\'inanse los tiempos promedio
\begin{eqnarray*}
\overline{X}&=&lim_{t\rightarrow\infty}\frac{1}{t}\int_{0}^{\infty}X\left(u\right)du\\
\prob\left(X_{\infty}\leq x\right)&=&lim_{t\rightarrow\infty}\frac{1}{t}\int_{0}^{\infty}I_{x}\left(u\right)du,
\end{eqnarray*}
cuando estos l\'imites existan.
\end{Def}

Como consecuencia del teorema de Renovaci\'on-Recompensa, se tiene que el primer l\'imite  existe y es igual a la constante
\begin{eqnarray*}
\overline{X}&=&\frac{\esp\left[\int_{0}^{R_{1}}X\left(t\right)dt\right]}{\esp\left[R_{1}\right]},
\end{eqnarray*}
suponiendo que ambas esperanzas son finitas.
 
\begin{Note}
Funciones de procesos regenerativos son regenerativas, es decir, si $X\left(t\right)$ es regenerativo y se define el proceso $Y\left(t\right)$ por $Y\left(t\right)=f\left(X\left(t\right)\right)$ para alguna funci\'on Borel medible $f\left(\cdot\right)$. Adem\'as $Y$ es regenerativo con los mismos tiempos de renovaci\'on que $X$. 

En general, los tiempos de renovaci\'on, $Z_{k}$ de un proceso regenerativo no requieren ser tiempos de paro con respecto a la evoluci\'on de $X\left(t\right)$.
\end{Note} 

\begin{Note}
Una funci\'on de un proceso de Markov, usualmente no ser\'a un proceso de Markov, sin embargo ser\'a regenerativo si el proceso de Markov lo es.
\end{Note}

 
\begin{Note}
Un proceso regenerativo con media de la longitud de ciclo finita es llamado positivo recurrente.
\end{Note}


\begin{Note}
\begin{itemize}
\item[a)] Si el proceso regenerativo $X$ es positivo recurrente y tiene trayectorias muestrales no negativas, entonces la ecuaci\'on anterior es v\'alida.
\item[b)] Si $X$ es positivo recurrente regenerativo, podemos construir una \'unica versi\'on estacionaria de este proceso, $X_{e}=\left\{X_{e}\left(t\right)\right\}$, donde $X_{e}$ es un proceso estoc\'astico regenerativo y estrictamente estacionario, con distribuci\'on marginal distribuida como $X_{\infty}$
\end{itemize}
\end{Note}


%__________________________________________________________________________________________
\subsection*{Procesos Regenerativos Estacionarios - Stidham \cite{Stidham}}
%__________________________________________________________________________________________


Un proceso estoc\'astico a tiempo continuo $\left\{V\left(t\right),t\geq0\right\}$ es un proceso regenerativo si existe una sucesi\'on de variables aleatorias independientes e id\'enticamente distribuidas $\left\{X_{1},X_{2},\ldots\right\}$, sucesi\'on de renovaci\'on, tal que para cualquier conjunto de Borel $A$, 

\begin{eqnarray*}
\prob\left\{V\left(t\right)\in A|X_{1}+X_{2}+\cdots+X_{R\left(t\right)}=s,\left\{V\left(\tau\right),\tau<s\right\}\right\}=\prob\left\{V\left(t-s\right)\in A|X_{1}>t-s\right\},
\end{eqnarray*}
para todo $0\leq s\leq t$, donde $R\left(t\right)=\max\left\{X_{1}+X_{2}+\cdots+X_{j}\leq t\right\}=$n\'umero de renovaciones ({\emph{puntos de regeneraci\'on}}) que ocurren en $\left[0,t\right]$. El intervalo $\left[0,X_{1}\right)$ es llamado {\emph{primer ciclo de regeneraci\'on}} de $\left\{V\left(t \right),t\geq0\right\}$, $\left[X_{1},X_{1}+X_{2}\right)$ el {\emph{segundo ciclo de regeneraci\'on}}, y as\'i sucesivamente.

Sea $X=X_{1}$ y sea $F$ la funci\'on de distrbuci\'on de $X$


\begin{Def}
Se define el proceso estacionario, $\left\{V^{*}\left(t\right),t\geq0\right\}$, para $\left\{V\left(t\right),t\geq0\right\}$ por

\begin{eqnarray*}
\prob\left\{V\left(t\right)\in A\right\}=\frac{1}{\esp\left[X\right]}\int_{0}^{\infty}\prob\left\{V\left(t+x\right)\in A|X>x\right\}\left(1-F\left(x\right)\right)dx,
\end{eqnarray*} 
para todo $t\geq0$ y todo conjunto de Borel $A$.
\end{Def}

\begin{Def}
Una distribuci\'on se dice que es {\emph{aritm\'etica}} si todos sus puntos de incremento son m\'ultiplos de la forma $0,\lambda, 2\lambda,\ldots$ para alguna $\lambda>0$ entera.
\end{Def}


\begin{Def}
Una modificaci\'on medible de un proceso $\left\{V\left(t\right),t\geq0\right\}$, es una versi\'on de este, $\left\{V\left(t,w\right)\right\}$ conjuntamente medible para $t\geq0$ y para $w\in S$, $S$ espacio de estados para $\left\{V\left(t\right),t\geq0\right\}$.
\end{Def}

\begin{Teo}
Sea $\left\{V\left(t\right),t\geq\right\}$ un proceso regenerativo no negativo con modificaci\'on medible. Sea $\esp\left[X\right]<\infty$. Entonces el proceso estacionario dado por la ecuaci\'on anterior est\'a bien definido y tiene funci\'on de distribuci\'on independiente de $t$, adem\'as
\begin{itemize}
\item[i)] \begin{eqnarray*}
\esp\left[V^{*}\left(0\right)\right]&=&\frac{\esp\left[\int_{0}^{X}V\left(s\right)ds\right]}{\esp\left[X\right]}\end{eqnarray*}
\item[ii)] Si $\esp\left[V^{*}\left(0\right)\right]<\infty$, equivalentemente, si $\esp\left[\int_{0}^{X}V\left(s\right)ds\right]<\infty$,entonces
\begin{eqnarray*}
\frac{\int_{0}^{t}V\left(s\right)ds}{t}\rightarrow\frac{\esp\left[\int_{0}^{X}V\left(s\right)ds\right]}{\esp\left[X\right]}
\end{eqnarray*}
con probabilidad 1 y en media, cuando $t\rightarrow\infty$.
\end{itemize}
\end{Teo}

\begin{Coro}
Sea $\left\{V\left(t\right),t\geq0\right\}$ un proceso regenerativo no negativo, con modificaci\'on medible. Si $\esp <\infty$, $F$ es no-aritm\'etica, y para todo $x\geq0$, $P\left\{V\left(t\right)\leq x,C>x\right\}$ es de variaci\'on acotada como funci\'on de $t$ en cada intervalo finito $\left[0,\tau\right]$, entonces $V\left(t\right)$ converge en distribuci\'on  cuando $t\rightarrow\infty$ y $$\esp V=\frac{\esp \int_{0}^{X}V\left(s\right)ds}{\esp X}$$
Donde $V$ tiene la distribuci\'on l\'imite de $V\left(t\right)$ cuando $t\rightarrow\infty$.

\end{Coro}

Para el caso discreto se tienen resultados similares.



%______________________________________________________________________
\subsection{Procesos de Renovaci\'on}
%______________________________________________________________________

\begin{Def}\label{Def.Tn}
Sean $0\leq T_{1}\leq T_{2}\leq \ldots$ son tiempos aleatorios infinitos en los cuales ocurren ciertos eventos. El n\'umero de tiempos $T_{n}$ en el intervalo $\left[0,t\right)$ es

\begin{eqnarray}
N\left(t\right)=\sum_{n=1}^{\infty}\indora\left(T_{n}\leq t\right),
\end{eqnarray}
para $t\geq0$.
\end{Def}

Si se consideran los puntos $T_{n}$ como elementos de $\rea_{+}$, y $N\left(t\right)$ es el n\'umero de puntos en $\rea$. El proceso denotado por $\left\{N\left(t\right):t\geq0\right\}$, denotado por $N\left(t\right)$, es un proceso puntual en $\rea_{+}$. Los $T_{n}$ son los tiempos de ocurrencia, el proceso puntual $N\left(t\right)$ es simple si su n\'umero de ocurrencias son distintas: $0<T_{1}<T_{2}<\ldots$ casi seguramente.

\begin{Def}
Un proceso puntual $N\left(t\right)$ es un proceso de renovaci\'on si los tiempos de interocurrencia $\xi_{n}=T_{n}-T_{n-1}$, para $n\geq1$, son independientes e identicamente distribuidos con distribuci\'on $F$, donde $F\left(0\right)=0$ y $T_{0}=0$. Los $T_{n}$ son llamados tiempos de renovaci\'on, referente a la independencia o renovaci\'on de la informaci\'on estoc\'astica en estos tiempos. Los $\xi_{n}$ son los tiempos de inter-renovaci\'on, y $N\left(t\right)$ es el n\'umero de renovaciones en el intervalo $\left[0,t\right)$
\end{Def}


\begin{Note}
Para definir un proceso de renovaci\'on para cualquier contexto, solamente hay que especificar una distribuci\'on $F$, con $F\left(0\right)=0$, para los tiempos de inter-renovaci\'on. La funci\'on $F$ en turno degune las otra variables aleatorias. De manera formal, existe un espacio de probabilidad y una sucesi\'on de variables aleatorias $\xi_{1},\xi_{2},\ldots$ definidas en este con distribuci\'on $F$. Entonces las otras cantidades son $T_{n}=\sum_{k=1}^{n}\xi_{k}$ y $N\left(t\right)=\sum_{n=1}^{\infty}\indora\left(T_{n}\leq t\right)$, donde $T_{n}\rightarrow\infty$ casi seguramente por la Ley Fuerte de los Grandes Números.
\end{Note}

%___________________________________________________________________________________________
%
\subsection{Teorema Principal de Renovaci\'on}
%___________________________________________________________________________________________
%

\begin{Note} Una funci\'on $h:\rea_{+}\rightarrow\rea$ es Directamente Riemann Integrable en los siguientes casos:
\begin{itemize}
\item[a)] $h\left(t\right)\geq0$ es decreciente y Riemann Integrable.
\item[b)] $h$ es continua excepto posiblemente en un conjunto de Lebesgue de medida 0, y $|h\left(t\right)|\leq b\left(t\right)$, donde $b$ es DRI.
\end{itemize}
\end{Note}

\begin{Teo}[Teorema Principal de Renovaci\'on]
Si $F$ es no aritm\'etica y $h\left(t\right)$ es Directamente Riemann Integrable (DRI), entonces

\begin{eqnarray*}
lim_{t\rightarrow\infty}U\star h=\frac{1}{\mu}\int_{\rea_{+}}h\left(s\right)ds.
\end{eqnarray*}
\end{Teo}

\begin{Prop}
Cualquier funci\'on $H\left(t\right)$ acotada en intervalos finitos y que es 0 para $t<0$ puede expresarse como
\begin{eqnarray*}
H\left(t\right)=U\star h\left(t\right)\textrm{,  donde }h\left(t\right)=H\left(t\right)-F\star H\left(t\right)
\end{eqnarray*}
\end{Prop}

\begin{Def}
Un proceso estoc\'astico $X\left(t\right)$ es crudamente regenerativo en un tiempo aleatorio positivo $T$ si
\begin{eqnarray*}
\esp\left[X\left(T+t\right)|T\right]=\esp\left[X\left(t\right)\right]\textrm{, para }t\geq0,\end{eqnarray*}
y con las esperanzas anteriores finitas.
\end{Def}

\begin{Prop}
Sup\'ongase que $X\left(t\right)$ es un proceso crudamente regenerativo en $T$, que tiene distribuci\'on $F$. Si $\esp\left[X\left(t\right)\right]$ es acotado en intervalos finitos, entonces
\begin{eqnarray*}
\esp\left[X\left(t\right)\right]=U\star h\left(t\right)\textrm{,  donde }h\left(t\right)=\esp\left[X\left(t\right)\indora\left(T>t\right)\right].
\end{eqnarray*}
\end{Prop}

\begin{Teo}[Regeneraci\'on Cruda]
Sup\'ongase que $X\left(t\right)$ es un proceso con valores positivo crudamente regenerativo en $T$, y def\'inase $M=\sup\left\{|X\left(t\right)|:t\leq T\right\}$. Si $T$ es no aritm\'etico y $M$ y $MT$ tienen media finita, entonces
\begin{eqnarray*}
lim_{t\rightarrow\infty}\esp\left[X\left(t\right)\right]=\frac{1}{\mu}\int_{\rea_{+}}h\left(s\right)ds,
\end{eqnarray*}
donde $h\left(t\right)=\esp\left[X\left(t\right)\indora\left(T>t\right)\right]$.
\end{Teo}


%___________________________________________________________________________________________
%
\subsection{Funci\'on de Renovaci\'on}
%___________________________________________________________________________________________
%


\begin{Def}
Sea $h\left(t\right)$ funci\'on de valores reales en $\rea$ acotada en intervalos finitos e igual a cero para $t<0$ La ecuaci\'on de renovaci\'on para $h\left(t\right)$ y la distribuci\'on $F$ es

\begin{eqnarray}\label{Ec.Renovacion}
H\left(t\right)=h\left(t\right)+\int_{\left[0,t\right]}H\left(t-s\right)dF\left(s\right)\textrm{,    }t\geq0,
\end{eqnarray}
donde $H\left(t\right)$ es una funci\'on de valores reales. Esto es $H=h+F\star H$. Decimos que $H\left(t\right)$ es soluci\'on de esta ecuaci\'on si satisface la ecuaci\'on, y es acotada en intervalos finitos e iguales a cero para $t<0$.
\end{Def}

\begin{Prop}
La funci\'on $U\star h\left(t\right)$ es la \'unica soluci\'on de la ecuaci\'on de renovaci\'on (\ref{Ec.Renovacion}).
\end{Prop}

\begin{Teo}[Teorema Renovaci\'on Elemental]
\begin{eqnarray*}
t^{-1}U\left(t\right)\rightarrow 1/\mu\textrm{,    cuando }t\rightarrow\infty.
\end{eqnarray*}
\end{Teo}

%___________________________________________________________________________________________
%
\subsection{Funci\'on de Renovaci\'on}
%___________________________________________________________________________________________
%


Sup\'ongase que $N\left(t\right)$ es un proceso de renovaci\'on con distribuci\'on $F$ con media finita $\mu$.

\begin{Def}
La funci\'on de renovaci\'on asociada con la distribuci\'on $F$, del proceso $N\left(t\right)$, es
\begin{eqnarray*}
U\left(t\right)=\sum_{n=1}^{\infty}F^{n\star}\left(t\right),\textrm{   }t\geq0,
\end{eqnarray*}
donde $F^{0\star}\left(t\right)=\indora\left(t\geq0\right)$.
\end{Def}


\begin{Prop}
Sup\'ongase que la distribuci\'on de inter-renovaci\'on $F$ tiene densidad $f$. Entonces $U\left(t\right)$ tambi\'en tiene densidad, para $t>0$, y es $U^{'}\left(t\right)=\sum_{n=0}^{\infty}f^{n\star}\left(t\right)$. Adem\'as
\begin{eqnarray*}
\prob\left\{N\left(t\right)>N\left(t-\right)\right\}=0\textrm{,   }t\geq0.
\end{eqnarray*}
\end{Prop}

\begin{Def}
La Transformada de Laplace-Stieljes de $F$ est\'a dada por

\begin{eqnarray*}
\hat{F}\left(\alpha\right)=\int_{\rea_{+}}e^{-\alpha t}dF\left(t\right)\textrm{,  }\alpha\geq0.
\end{eqnarray*}
\end{Def}

Entonces

\begin{eqnarray*}
\hat{U}\left(\alpha\right)=\sum_{n=0}^{\infty}\hat{F^{n\star}}\left(\alpha\right)=\sum_{n=0}^{\infty}\hat{F}\left(\alpha\right)^{n}=\frac{1}{1-\hat{F}\left(\alpha\right)}.
\end{eqnarray*}


\begin{Prop}
La Transformada de Laplace $\hat{U}\left(\alpha\right)$ y $\hat{F}\left(\alpha\right)$ determina una a la otra de manera \'unica por la relaci\'on $\hat{U}\left(\alpha\right)=\frac{1}{1-\hat{F}\left(\alpha\right)}$.
\end{Prop}


\begin{Note}
Un proceso de renovaci\'on $N\left(t\right)$ cuyos tiempos de inter-renovaci\'on tienen media finita, es un proceso Poisson con tasa $\lambda$ si y s\'olo s\'i $\esp\left[U\left(t\right)\right]=\lambda t$, para $t\geq0$.
\end{Note}


\begin{Teo}
Sea $N\left(t\right)$ un proceso puntual simple con puntos de localizaci\'on $T_{n}$ tal que $\eta\left(t\right)=\esp\left[N\left(\right)\right]$ es finita para cada $t$. Entonces para cualquier funci\'on $f:\rea_{+}\rightarrow\rea$,
\begin{eqnarray*}
\esp\left[\sum_{n=1}^{N\left(\right)}f\left(T_{n}\right)\right]=\int_{\left(0,t\right]}f\left(s\right)d\eta\left(s\right)\textrm{,  }t\geq0,
\end{eqnarray*}
suponiendo que la integral exista. Adem\'as si $X_{1},X_{2},\ldots$ son variables aleatorias definidas en el mismo espacio de probabilidad que el proceso $N\left(t\right)$ tal que $\esp\left[X_{n}|T_{n}=s\right]=f\left(s\right)$, independiente de $n$. Entonces
\begin{eqnarray*}
\esp\left[\sum_{n=1}^{N\left(t\right)}X_{n}\right]=\int_{\left(0,t\right]}f\left(s\right)d\eta\left(s\right)\textrm{,  }t\geq0,
\end{eqnarray*} 
suponiendo que la integral exista. 
\end{Teo}

\begin{Coro}[Identidad de Wald para Renovaciones]
Para el proceso de renovaci\'on $N\left(t\right)$,
\begin{eqnarray*}
\esp\left[T_{N\left(t\right)+1}\right]=\mu\esp\left[N\left(t\right)+1\right]\textrm{,  }t\geq0,
\end{eqnarray*}  
\end{Coro}

%______________________________________________________________________
\subsection{Procesos de Renovaci\'on}
%______________________________________________________________________

\begin{Def}\label{Def.Tn}
Sean $0\leq T_{1}\leq T_{2}\leq \ldots$ son tiempos aleatorios infinitos en los cuales ocurren ciertos eventos. El n\'umero de tiempos $T_{n}$ en el intervalo $\left[0,t\right)$ es

\begin{eqnarray}
N\left(t\right)=\sum_{n=1}^{\infty}\indora\left(T_{n}\leq t\right),
\end{eqnarray}
para $t\geq0$.
\end{Def}

Si se consideran los puntos $T_{n}$ como elementos de $\rea_{+}$, y $N\left(t\right)$ es el n\'umero de puntos en $\rea$. El proceso denotado por $\left\{N\left(t\right):t\geq0\right\}$, denotado por $N\left(t\right)$, es un proceso puntual en $\rea_{+}$. Los $T_{n}$ son los tiempos de ocurrencia, el proceso puntual $N\left(t\right)$ es simple si su n\'umero de ocurrencias son distintas: $0<T_{1}<T_{2}<\ldots$ casi seguramente.

\begin{Def}
Un proceso puntual $N\left(t\right)$ es un proceso de renovaci\'on si los tiempos de interocurrencia $\xi_{n}=T_{n}-T_{n-1}$, para $n\geq1$, son independientes e identicamente distribuidos con distribuci\'on $F$, donde $F\left(0\right)=0$ y $T_{0}=0$. Los $T_{n}$ son llamados tiempos de renovaci\'on, referente a la independencia o renovaci\'on de la informaci\'on estoc\'astica en estos tiempos. Los $\xi_{n}$ son los tiempos de inter-renovaci\'on, y $N\left(t\right)$ es el n\'umero de renovaciones en el intervalo $\left[0,t\right)$
\end{Def}


\begin{Note}
Para definir un proceso de renovaci\'on para cualquier contexto, solamente hay que especificar una distribuci\'on $F$, con $F\left(0\right)=0$, para los tiempos de inter-renovaci\'on. La funci\'on $F$ en turno degune las otra variables aleatorias. De manera formal, existe un espacio de probabilidad y una sucesi\'on de variables aleatorias $\xi_{1},\xi_{2},\ldots$ definidas en este con distribuci\'on $F$. Entonces las otras cantidades son $T_{n}=\sum_{k=1}^{n}\xi_{k}$ y $N\left(t\right)=\sum_{n=1}^{\infty}\indora\left(T_{n}\leq t\right)$, donde $T_{n}\rightarrow\infty$ casi seguramente por la Ley Fuerte de los Grandes Números.
\end{Note}

\begin{Def}\label{Def.Tn}
Sean $0\leq T_{1}\leq T_{2}\leq \ldots$ son tiempos aleatorios infinitos en los cuales ocurren ciertos eventos. El n\'umero de tiempos $T_{n}$ en el intervalo $\left[0,t\right)$ es

\begin{eqnarray}
N\left(t\right)=\sum_{n=1}^{\infty}\indora\left(T_{n}\leq t\right),
\end{eqnarray}
para $t\geq0$.
\end{Def}

Si se consideran los puntos $T_{n}$ como elementos de $\rea_{+}$, y $N\left(t\right)$ es el n\'umero de puntos en $\rea$. El proceso denotado por $\left\{N\left(t\right):t\geq0\right\}$, denotado por $N\left(t\right)$, es un proceso puntual en $\rea_{+}$. Los $T_{n}$ son los tiempos de ocurrencia, el proceso puntual $N\left(t\right)$ es simple si su n\'umero de ocurrencias son distintas: $0<T_{1}<T_{2}<\ldots$ casi seguramente.

\begin{Def}
Un proceso puntual $N\left(t\right)$ es un proceso de renovaci\'on si los tiempos de interocurrencia $\xi_{n}=T_{n}-T_{n-1}$, para $n\geq1$, son independientes e identicamente distribuidos con distribuci\'on $F$, donde $F\left(0\right)=0$ y $T_{0}=0$. Los $T_{n}$ son llamados tiempos de renovaci\'on, referente a la independencia o renovaci\'on de la informaci\'on estoc\'astica en estos tiempos. Los $\xi_{n}$ son los tiempos de inter-renovaci\'on, y $N\left(t\right)$ es el n\'umero de renovaciones en el intervalo $\left[0,t\right)$
\end{Def}


\begin{Note}
Para definir un proceso de renovaci\'on para cualquier contexto, solamente hay que especificar una distribuci\'on $F$, con $F\left(0\right)=0$, para los tiempos de inter-renovaci\'on. La funci\'on $F$ en turno degune las otra variables aleatorias. De manera formal, existe un espacio de probabilidad y una sucesi\'on de variables aleatorias $\xi_{1},\xi_{2},\ldots$ definidas en este con distribuci\'on $F$. Entonces las otras cantidades son $T_{n}=\sum_{k=1}^{n}\xi_{k}$ y $N\left(t\right)=\sum_{n=1}^{\infty}\indora\left(T_{n}\leq t\right)$, donde $T_{n}\rightarrow\infty$ casi seguramente por la Ley Fuerte de los Grandes N\'umeros.
\end{Note}







Los tiempos $T_{n}$ est\'an relacionados con los conteos de $N\left(t\right)$ por

\begin{eqnarray*}
\left\{N\left(t\right)\geq n\right\}&=&\left\{T_{n}\leq t\right\}\\
T_{N\left(t\right)}\leq &t&<T_{N\left(t\right)+1},
\end{eqnarray*}

adem\'as $N\left(T_{n}\right)=n$, y 

\begin{eqnarray*}
N\left(t\right)=\max\left\{n:T_{n}\leq t\right\}=\min\left\{n:T_{n+1}>t\right\}
\end{eqnarray*}

Por propiedades de la convoluci\'on se sabe que

\begin{eqnarray*}
P\left\{T_{n}\leq t\right\}=F^{n\star}\left(t\right)
\end{eqnarray*}
que es la $n$-\'esima convoluci\'on de $F$. Entonces 

\begin{eqnarray*}
\left\{N\left(t\right)\geq n\right\}&=&\left\{T_{n}\leq t\right\}\\
P\left\{N\left(t\right)\leq n\right\}&=&1-F^{\left(n+1\right)\star}\left(t\right)
\end{eqnarray*}

Adem\'as usando el hecho de que $\esp\left[N\left(t\right)\right]=\sum_{n=1}^{\infty}P\left\{N\left(t\right)\geq n\right\}$
se tiene que

\begin{eqnarray*}
\esp\left[N\left(t\right)\right]=\sum_{n=1}^{\infty}F^{n\star}\left(t\right)
\end{eqnarray*}

\begin{Prop}
Para cada $t\geq0$, la funci\'on generadora de momentos $\esp\left[e^{\alpha N\left(t\right)}\right]$ existe para alguna $\alpha$ en una vecindad del 0, y de aqu\'i que $\esp\left[N\left(t\right)^{m}\right]<\infty$, para $m\geq1$.
\end{Prop}

\begin{Ejem}[\textbf{Proceso Poisson}]

Suponga que se tienen tiempos de inter-renovaci\'on \textit{i.i.d.} del proceso de renovaci\'on $N\left(t\right)$ tienen distribuci\'on exponencial $F\left(t\right)=q-e^{-\lambda t}$ con tasa $\lambda$. Entonces $N\left(t\right)$ es un proceso Poisson con tasa $\lambda$.

\end{Ejem}


\begin{Note}
Si el primer tiempo de renovaci\'on $\xi_{1}$ no tiene la misma distribuci\'on que el resto de las $\xi_{n}$, para $n\geq2$, a $N\left(t\right)$ se le llama Proceso de Renovaci\'on retardado, donde si $\xi$ tiene distribuci\'on $G$, entonces el tiempo $T_{n}$ de la $n$-\'esima renovaci\'on tiene distribuci\'on $G\star F^{\left(n-1\right)\star}\left(t\right)$
\end{Note}


\begin{Teo}
Para una constante $\mu\leq\infty$ ( o variable aleatoria), las siguientes expresiones son equivalentes:

\begin{eqnarray}
lim_{n\rightarrow\infty}n^{-1}T_{n}&=&\mu,\textrm{ c.s.}\\
lim_{t\rightarrow\infty}t^{-1}N\left(t\right)&=&1/\mu,\textrm{ c.s.}
\end{eqnarray}
\end{Teo}


Es decir, $T_{n}$ satisface la Ley Fuerte de los Grandes N\'umeros s\'i y s\'olo s\'i $N\left/t\right)$ la cumple.


\begin{Coro}[Ley Fuerte de los Grandes N\'umeros para Procesos de Renovaci\'on]
Si $N\left(t\right)$ es un proceso de renovaci\'on cuyos tiempos de inter-renovaci\'on tienen media $\mu\leq\infty$, entonces
\begin{eqnarray}
t^{-1}N\left(t\right)\rightarrow 1/\mu,\textrm{ c.s. cuando }t\rightarrow\infty.
\end{eqnarray}

\end{Coro}


Considerar el proceso estoc\'astico de valores reales $\left\{Z\left(t\right):t\geq0\right\}$ en el mismo espacio de probabilidad que $N\left(t\right)$

\begin{Def}
Para el proceso $\left\{Z\left(t\right):t\geq0\right\}$ se define la fluctuaci\'on m\'axima de $Z\left(t\right)$ en el intervalo $\left(T_{n-1},T_{n}\right]$:
\begin{eqnarray*}
M_{n}=\sup_{T_{n-1}<t\leq T_{n}}|Z\left(t\right)-Z\left(T_{n-1}\right)|
\end{eqnarray*}
\end{Def}

\begin{Teo}
Sup\'ongase que $n^{-1}T_{n}\rightarrow\mu$ c.s. cuando $n\rightarrow\infty$, donde $\mu\leq\infty$ es una constante o variable aleatoria. Sea $a$ una constante o variable aleatoria que puede ser infinita cuando $\mu$ es finita, y considere las expresiones l\'imite:
\begin{eqnarray}
lim_{n\rightarrow\infty}n^{-1}Z\left(T_{n}\right)&=&a,\textrm{ c.s.}\\
lim_{t\rightarrow\infty}t^{-1}Z\left(t\right)&=&a/\mu,\textrm{ c.s.}
\end{eqnarray}
La segunda expresi\'on implica la primera. Conversamente, la primera implica la segunda si el proceso $Z\left(t\right)$ es creciente, o si $lim_{n\rightarrow\infty}n^{-1}M_{n}=0$ c.s.
\end{Teo}

\begin{Coro}
Si $N\left(t\right)$ es un proceso de renovaci\'on, y $\left(Z\left(T_{n}\right)-Z\left(T_{n-1}\right),M_{n}\right)$, para $n\geq1$, son variables aleatorias independientes e id\'enticamente distribuidas con media finita, entonces,
\begin{eqnarray}
lim_{t\rightarrow\infty}t^{-1}Z\left(t\right)\rightarrow\frac{\esp\left[Z\left(T_{1}\right)-Z\left(T_{0}\right)\right]}{\esp\left[T_{1}\right]},\textrm{ c.s. cuando  }t\rightarrow\infty.
\end{eqnarray}
\end{Coro}


Sup\'ongase que $N\left(t\right)$ es un proceso de renovaci\'on con distribuci\'on $F$ con media finita $\mu$.

\begin{Def}
La funci\'on de renovaci\'on asociada con la distribuci\'on $F$, del proceso $N\left(t\right)$, es
\begin{eqnarray*}
U\left(t\right)=\sum_{n=1}^{\infty}F^{n\star}\left(t\right),\textrm{   }t\geq0,
\end{eqnarray*}
donde $F^{0\star}\left(t\right)=\indora\left(t\geq0\right)$.
\end{Def}


\begin{Prop}
Sup\'ongase que la distribuci\'on de inter-renovaci\'on $F$ tiene densidad $f$. Entonces $U\left(t\right)$ tambi\'en tiene densidad, para $t>0$, y es $U^{'}\left(t\right)=\sum_{n=0}^{\infty}f^{n\star}\left(t\right)$. Adem\'as
\begin{eqnarray*}
\prob\left\{N\left(t\right)>N\left(t-\right)\right\}=0\textrm{,   }t\geq0.
\end{eqnarray*}
\end{Prop}

\begin{Def}
La Transformada de Laplace-Stieljes de $F$ est\'a dada por

\begin{eqnarray*}
\hat{F}\left(\alpha\right)=\int_{\rea_{+}}e^{-\alpha t}dF\left(t\right)\textrm{,  }\alpha\geq0.
\end{eqnarray*}
\end{Def}

Entonces

\begin{eqnarray*}
\hat{U}\left(\alpha\right)=\sum_{n=0}^{\infty}\hat{F^{n\star}}\left(\alpha\right)=\sum_{n=0}^{\infty}\hat{F}\left(\alpha\right)^{n}=\frac{1}{1-\hat{F}\left(\alpha\right)}.
\end{eqnarray*}


\begin{Prop}
La Transformada de Laplace $\hat{U}\left(\alpha\right)$ y $\hat{F}\left(\alpha\right)$ determina una a la otra de manera \'unica por la relaci\'on $\hat{U}\left(\alpha\right)=\frac{1}{1-\hat{F}\left(\alpha\right)}$.
\end{Prop}


\begin{Note}
Un proceso de renovaci\'on $N\left(t\right)$ cuyos tiempos de inter-renovaci\'on tienen media finita, es un proceso Poisson con tasa $\lambda$ si y s\'olo s\'i $\esp\left[U\left(t\right)\right]=\lambda t$, para $t\geq0$.
\end{Note}


\begin{Teo}
Sea $N\left(t\right)$ un proceso puntual simple con puntos de localizaci\'on $T_{n}$ tal que $\eta\left(t\right)=\esp\left[N\left(\right)\right]$ es finita para cada $t$. Entonces para cualquier funci\'on $f:\rea_{+}\rightarrow\rea$,
\begin{eqnarray*}
\esp\left[\sum_{n=1}^{N\left(\right)}f\left(T_{n}\right)\right]=\int_{\left(0,t\right]}f\left(s\right)d\eta\left(s\right)\textrm{,  }t\geq0,
\end{eqnarray*}
suponiendo que la integral exista. Adem\'as si $X_{1},X_{2},\ldots$ son variables aleatorias definidas en el mismo espacio de probabilidad que el proceso $N\left(t\right)$ tal que $\esp\left[X_{n}|T_{n}=s\right]=f\left(s\right)$, independiente de $n$. Entonces
\begin{eqnarray*}
\esp\left[\sum_{n=1}^{N\left(t\right)}X_{n}\right]=\int_{\left(0,t\right]}f\left(s\right)d\eta\left(s\right)\textrm{,  }t\geq0,
\end{eqnarray*} 
suponiendo que la integral exista. 
\end{Teo}

\begin{Coro}[Identidad de Wald para Renovaciones]
Para el proceso de renovaci\'on $N\left(t\right)$,
\begin{eqnarray*}
\esp\left[T_{N\left(t\right)+1}\right]=\mu\esp\left[N\left(t\right)+1\right]\textrm{,  }t\geq0,
\end{eqnarray*}  
\end{Coro}


\begin{Def}
Sea $h\left(t\right)$ funci\'on de valores reales en $\rea$ acotada en intervalos finitos e igual a cero para $t<0$ La ecuaci\'on de renovaci\'on para $h\left(t\right)$ y la distribuci\'on $F$ es

\begin{eqnarray}\label{Ec.Renovacion}
H\left(t\right)=h\left(t\right)+\int_{\left[0,t\right]}H\left(t-s\right)dF\left(s\right)\textrm{,    }t\geq0,
\end{eqnarray}
donde $H\left(t\right)$ es una funci\'on de valores reales. Esto es $H=h+F\star H$. Decimos que $H\left(t\right)$ es soluci\'on de esta ecuaci\'on si satisface la ecuaci\'on, y es acotada en intervalos finitos e iguales a cero para $t<0$.
\end{Def}

\begin{Prop}
La funci\'on $U\star h\left(t\right)$ es la \'unica soluci\'on de la ecuaci\'on de renovaci\'on (\ref{Ec.Renovacion}).
\end{Prop}

\begin{Teo}[Teorema Renovaci\'on Elemental]
\begin{eqnarray*}
t^{-1}U\left(t\right)\rightarrow 1/\mu\textrm{,    cuando }t\rightarrow\infty.
\end{eqnarray*}
\end{Teo}



Sup\'ongase que $N\left(t\right)$ es un proceso de renovaci\'on con distribuci\'on $F$ con media finita $\mu$.

\begin{Def}
La funci\'on de renovaci\'on asociada con la distribuci\'on $F$, del proceso $N\left(t\right)$, es
\begin{eqnarray*}
U\left(t\right)=\sum_{n=1}^{\infty}F^{n\star}\left(t\right),\textrm{   }t\geq0,
\end{eqnarray*}
donde $F^{0\star}\left(t\right)=\indora\left(t\geq0\right)$.
\end{Def}


\begin{Prop}
Sup\'ongase que la distribuci\'on de inter-renovaci\'on $F$ tiene densidad $f$. Entonces $U\left(t\right)$ tambi\'en tiene densidad, para $t>0$, y es $U^{'}\left(t\right)=\sum_{n=0}^{\infty}f^{n\star}\left(t\right)$. Adem\'as
\begin{eqnarray*}
\prob\left\{N\left(t\right)>N\left(t-\right)\right\}=0\textrm{,   }t\geq0.
\end{eqnarray*}
\end{Prop}

\begin{Def}
La Transformada de Laplace-Stieljes de $F$ est\'a dada por

\begin{eqnarray*}
\hat{F}\left(\alpha\right)=\int_{\rea_{+}}e^{-\alpha t}dF\left(t\right)\textrm{,  }\alpha\geq0.
\end{eqnarray*}
\end{Def}

Entonces

\begin{eqnarray*}
\hat{U}\left(\alpha\right)=\sum_{n=0}^{\infty}\hat{F^{n\star}}\left(\alpha\right)=\sum_{n=0}^{\infty}\hat{F}\left(\alpha\right)^{n}=\frac{1}{1-\hat{F}\left(\alpha\right)}.
\end{eqnarray*}


\begin{Prop}
La Transformada de Laplace $\hat{U}\left(\alpha\right)$ y $\hat{F}\left(\alpha\right)$ determina una a la otra de manera \'unica por la relaci\'on $\hat{U}\left(\alpha\right)=\frac{1}{1-\hat{F}\left(\alpha\right)}$.
\end{Prop}


\begin{Note}
Un proceso de renovaci\'on $N\left(t\right)$ cuyos tiempos de inter-renovaci\'on tienen media finita, es un proceso Poisson con tasa $\lambda$ si y s\'olo s\'i $\esp\left[U\left(t\right)\right]=\lambda t$, para $t\geq0$.
\end{Note}


\begin{Teo}
Sea $N\left(t\right)$ un proceso puntual simple con puntos de localizaci\'on $T_{n}$ tal que $\eta\left(t\right)=\esp\left[N\left(\right)\right]$ es finita para cada $t$. Entonces para cualquier funci\'on $f:\rea_{+}\rightarrow\rea$,
\begin{eqnarray*}
\esp\left[\sum_{n=1}^{N\left(\right)}f\left(T_{n}\right)\right]=\int_{\left(0,t\right]}f\left(s\right)d\eta\left(s\right)\textrm{,  }t\geq0,
\end{eqnarray*}
suponiendo que la integral exista. Adem\'as si $X_{1},X_{2},\ldots$ son variables aleatorias definidas en el mismo espacio de probabilidad que el proceso $N\left(t\right)$ tal que $\esp\left[X_{n}|T_{n}=s\right]=f\left(s\right)$, independiente de $n$. Entonces
\begin{eqnarray*}
\esp\left[\sum_{n=1}^{N\left(t\right)}X_{n}\right]=\int_{\left(0,t\right]}f\left(s\right)d\eta\left(s\right)\textrm{,  }t\geq0,
\end{eqnarray*} 
suponiendo que la integral exista. 
\end{Teo}

\begin{Coro}[Identidad de Wald para Renovaciones]
Para el proceso de renovaci\'on $N\left(t\right)$,
\begin{eqnarray*}
\esp\left[T_{N\left(t\right)+1}\right]=\mu\esp\left[N\left(t\right)+1\right]\textrm{,  }t\geq0,
\end{eqnarray*}  
\end{Coro}


\begin{Def}
Sea $h\left(t\right)$ funci\'on de valores reales en $\rea$ acotada en intervalos finitos e igual a cero para $t<0$ La ecuaci\'on de renovaci\'on para $h\left(t\right)$ y la distribuci\'on $F$ es

\begin{eqnarray}\label{Ec.Renovacion}
H\left(t\right)=h\left(t\right)+\int_{\left[0,t\right]}H\left(t-s\right)dF\left(s\right)\textrm{,    }t\geq0,
\end{eqnarray}
donde $H\left(t\right)$ es una funci\'on de valores reales. Esto es $H=h+F\star H$. Decimos que $H\left(t\right)$ es soluci\'on de esta ecuaci\'on si satisface la ecuaci\'on, y es acotada en intervalos finitos e iguales a cero para $t<0$.
\end{Def}

\begin{Prop}
La funci\'on $U\star h\left(t\right)$ es la \'unica soluci\'on de la ecuaci\'on de renovaci\'on (\ref{Ec.Renovacion}).
\end{Prop}

\begin{Teo}[Teorema Renovaci\'on Elemental]
\begin{eqnarray*}
t^{-1}U\left(t\right)\rightarrow 1/\mu\textrm{,    cuando }t\rightarrow\infty.
\end{eqnarray*}
\end{Teo}


\begin{Note} Una funci\'on $h:\rea_{+}\rightarrow\rea$ es Directamente Riemann Integrable en los siguientes casos:
\begin{itemize}
\item[a)] $h\left(t\right)\geq0$ es decreciente y Riemann Integrable.
\item[b)] $h$ es continua excepto posiblemente en un conjunto de Lebesgue de medida 0, y $|h\left(t\right)|\leq b\left(t\right)$, donde $b$ es DRI.
\end{itemize}
\end{Note}

\begin{Teo}[Teorema Principal de Renovaci\'on]
Si $F$ es no aritm\'etica y $h\left(t\right)$ es Directamente Riemann Integrable (DRI), entonces

\begin{eqnarray*}
lim_{t\rightarrow\infty}U\star h=\frac{1}{\mu}\int_{\rea_{+}}h\left(s\right)ds.
\end{eqnarray*}
\end{Teo}

\begin{Prop}
Cualquier funci\'on $H\left(t\right)$ acotada en intervalos finitos y que es 0 para $t<0$ puede expresarse como
\begin{eqnarray*}
H\left(t\right)=U\star h\left(t\right)\textrm{,  donde }h\left(t\right)=H\left(t\right)-F\star H\left(t\right)
\end{eqnarray*}
\end{Prop}

\begin{Def}
Un proceso estoc\'astico $X\left(t\right)$ es crudamente regenerativo en un tiempo aleatorio positivo $T$ si
\begin{eqnarray*}
\esp\left[X\left(T+t\right)|T\right]=\esp\left[X\left(t\right)\right]\textrm{, para }t\geq0,\end{eqnarray*}
y con las esperanzas anteriores finitas.
\end{Def}

\begin{Prop}
Sup\'ongase que $X\left(t\right)$ es un proceso crudamente regenerativo en $T$, que tiene distribuci\'on $F$. Si $\esp\left[X\left(t\right)\right]$ es acotado en intervalos finitos, entonces
\begin{eqnarray*}
\esp\left[X\left(t\right)\right]=U\star h\left(t\right)\textrm{,  donde }h\left(t\right)=\esp\left[X\left(t\right)\indora\left(T>t\right)\right].
\end{eqnarray*}
\end{Prop}

\begin{Teo}[Regeneraci\'on Cruda]
Sup\'ongase que $X\left(t\right)$ es un proceso con valores positivo crudamente regenerativo en $T$, y def\'inase $M=\sup\left\{|X\left(t\right)|:t\leq T\right\}$. Si $T$ es no aritm\'etico y $M$ y $MT$ tienen media finita, entonces
\begin{eqnarray*}
lim_{t\rightarrow\infty}\esp\left[X\left(t\right)\right]=\frac{1}{\mu}\int_{\rea_{+}}h\left(s\right)ds,
\end{eqnarray*}
donde $h\left(t\right)=\esp\left[X\left(t\right)\indora\left(T>t\right)\right]$.
\end{Teo}


\begin{Note} Una funci\'on $h:\rea_{+}\rightarrow\rea$ es Directamente Riemann Integrable en los siguientes casos:
\begin{itemize}
\item[a)] $h\left(t\right)\geq0$ es decreciente y Riemann Integrable.
\item[b)] $h$ es continua excepto posiblemente en un conjunto de Lebesgue de medida 0, y $|h\left(t\right)|\leq b\left(t\right)$, donde $b$ es DRI.
\end{itemize}
\end{Note}

\begin{Teo}[Teorema Principal de Renovaci\'on]
Si $F$ es no aritm\'etica y $h\left(t\right)$ es Directamente Riemann Integrable (DRI), entonces

\begin{eqnarray*}
lim_{t\rightarrow\infty}U\star h=\frac{1}{\mu}\int_{\rea_{+}}h\left(s\right)ds.
\end{eqnarray*}
\end{Teo}

\begin{Prop}
Cualquier funci\'on $H\left(t\right)$ acotada en intervalos finitos y que es 0 para $t<0$ puede expresarse como
\begin{eqnarray*}
H\left(t\right)=U\star h\left(t\right)\textrm{,  donde }h\left(t\right)=H\left(t\right)-F\star H\left(t\right)
\end{eqnarray*}
\end{Prop}

\begin{Def}
Un proceso estoc\'astico $X\left(t\right)$ es crudamente regenerativo en un tiempo aleatorio positivo $T$ si
\begin{eqnarray*}
\esp\left[X\left(T+t\right)|T\right]=\esp\left[X\left(t\right)\right]\textrm{, para }t\geq0,\end{eqnarray*}
y con las esperanzas anteriores finitas.
\end{Def}

\begin{Prop}
Sup\'ongase que $X\left(t\right)$ es un proceso crudamente regenerativo en $T$, que tiene distribuci\'on $F$. Si $\esp\left[X\left(t\right)\right]$ es acotado en intervalos finitos, entonces
\begin{eqnarray*}
\esp\left[X\left(t\right)\right]=U\star h\left(t\right)\textrm{,  donde }h\left(t\right)=\esp\left[X\left(t\right)\indora\left(T>t\right)\right].
\end{eqnarray*}
\end{Prop}

\begin{Teo}[Regeneraci\'on Cruda]
Sup\'ongase que $X\left(t\right)$ es un proceso con valores positivo crudamente regenerativo en $T$, y def\'inase $M=\sup\left\{|X\left(t\right)|:t\leq T\right\}$. Si $T$ es no aritm\'etico y $M$ y $MT$ tienen media finita, entonces
\begin{eqnarray*}
lim_{t\rightarrow\infty}\esp\left[X\left(t\right)\right]=\frac{1}{\mu}\int_{\rea_{+}}h\left(s\right)ds,
\end{eqnarray*}
donde $h\left(t\right)=\esp\left[X\left(t\right)\indora\left(T>t\right)\right]$.
\end{Teo}

\begin{Def}
Para el proceso $\left\{\left(N\left(t\right),X\left(t\right)\right):t\geq0\right\}$, sus trayectoria muestrales en el intervalo de tiempo $\left[T_{n-1},T_{n}\right)$ est\'an descritas por
\begin{eqnarray*}
\zeta_{n}=\left(\xi_{n},\left\{X\left(T_{n-1}+t\right):0\leq t<\xi_{n}\right\}\right)
\end{eqnarray*}
Este $\zeta_{n}$ es el $n$-\'esimo segmento del proceso. El proceso es regenerativo sobre los tiempos $T_{n}$ si sus segmentos $\zeta_{n}$ son independientes e id\'enticamennte distribuidos.
\end{Def}


\begin{Note}
Si $\tilde{X}\left(t\right)$ con espacio de estados $\tilde{S}$ es regenerativo sobre $T_{n}$, entonces $X\left(t\right)=f\left(\tilde{X}\left(t\right)\right)$ tambi\'en es regenerativo sobre $T_{n}$, para cualquier funci\'on $f:\tilde{S}\rightarrow S$.
\end{Note}

\begin{Note}
Los procesos regenerativos son crudamente regenerativos, pero no al rev\'es.
\end{Note}


\begin{Note}
Un proceso estoc\'astico a tiempo continuo o discreto es regenerativo si existe un proceso de renovaci\'on  tal que los segmentos del proceso entre tiempos de renovaci\'on sucesivos son i.i.d., es decir, para $\left\{X\left(t\right):t\geq0\right\}$ proceso estoc\'astico a tiempo continuo con espacio de estados $S$, espacio m\'etrico.
\end{Note}

Para $\left\{X\left(t\right):t\geq0\right\}$ Proceso Estoc\'astico a tiempo continuo con estado de espacios $S$, que es un espacio m\'etrico, con trayectorias continuas por la derecha y con l\'imites por la izquierda c.s. Sea $N\left(t\right)$ un proceso de renovaci\'on en $\rea_{+}$ definido en el mismo espacio de probabilidad que $X\left(t\right)$, con tiempos de renovaci\'on $T$ y tiempos de inter-renovaci\'on $\xi_{n}=T_{n}-T_{n-1}$, con misma distribuci\'on $F$ de media finita $\mu$.



\begin{Def}
Para el proceso $\left\{\left(N\left(t\right),X\left(t\right)\right):t\geq0\right\}$, sus trayectoria muestrales en el intervalo de tiempo $\left[T_{n-1},T_{n}\right)$ est\'an descritas por
\begin{eqnarray*}
\zeta_{n}=\left(\xi_{n},\left\{X\left(T_{n-1}+t\right):0\leq t<\xi_{n}\right\}\right)
\end{eqnarray*}
Este $\zeta_{n}$ es el $n$-\'esimo segmento del proceso. El proceso es regenerativo sobre los tiempos $T_{n}$ si sus segmentos $\zeta_{n}$ son independientes e id\'enticamennte distribuidos.
\end{Def}

\begin{Note}
Un proceso regenerativo con media de la longitud de ciclo finita es llamado positivo recurrente.
\end{Note}

\begin{Teo}[Procesos Regenerativos]
Suponga que el proceso
\end{Teo}


\begin{Def}[Renewal Process Trinity]
Para un proceso de renovaci\'on $N\left(t\right)$, los siguientes procesos proveen de informaci\'on sobre los tiempos de renovaci\'on.
\begin{itemize}
\item $A\left(t\right)=t-T_{N\left(t\right)}$, el tiempo de recurrencia hacia atr\'as al tiempo $t$, que es el tiempo desde la \'ultima renovaci\'on para $t$.

\item $B\left(t\right)=T_{N\left(t\right)+1}-t$, el tiempo de recurrencia hacia adelante al tiempo $t$, residual del tiempo de renovaci\'on, que es el tiempo para la pr\'oxima renovaci\'on despu\'es de $t$.

\item $L\left(t\right)=\xi_{N\left(t\right)+1}=A\left(t\right)+B\left(t\right)$, la longitud del intervalo de renovaci\'on que contiene a $t$.
\end{itemize}
\end{Def}

\begin{Note}
El proceso tridimensional $\left(A\left(t\right),B\left(t\right),L\left(t\right)\right)$ es regenerativo sobre $T_{n}$, y por ende cada proceso lo es. Cada proceso $A\left(t\right)$ y $B\left(t\right)$ son procesos de MArkov a tiempo continuo con trayectorias continuas por partes en el espacio de estados $\rea_{+}$. Una expresi\'on conveniente para su distribuci\'on conjunta es, para $0\leq x<t,y\geq0$
\begin{equation}\label{NoRenovacion}
P\left\{A\left(t\right)>x,B\left(t\right)>y\right\}=
P\left\{N\left(t+y\right)-N\left((t-x)\right)=0\right\}
\end{equation}
\end{Note}

\begin{Ejem}[Tiempos de recurrencia Poisson]
Si $N\left(t\right)$ es un proceso Poisson con tasa $\lambda$, entonces de la expresi\'on (\ref{NoRenovacion}) se tiene que

\begin{eqnarray*}
\begin{array}{lc}
P\left\{A\left(t\right)>x,B\left(t\right)>y\right\}=e^{-\lambda\left(x+y\right)},&0\leq x<t,y\geq0,
\end{array}
\end{eqnarray*}
que es la probabilidad Poisson de no renovaciones en un intervalo de longitud $x+y$.

\end{Ejem}

\begin{Note}
Una cadena de Markov erg\'odica tiene la propiedad de ser estacionaria si la distribuci\'on de su estado al tiempo $0$ es su distribuci\'on estacionaria.
\end{Note}


\begin{Def}
Un proceso estoc\'astico a tiempo continuo $\left\{X\left(t\right):t\geq0\right\}$ en un espacio general es estacionario si sus distribuciones finito dimensionales son invariantes bajo cualquier  traslado: para cada $0\leq s_{1}<s_{2}<\cdots<s_{k}$ y $t\geq0$,
\begin{eqnarray*}
\left(X\left(s_{1}+t\right),\ldots,X\left(s_{k}+t\right)\right)=_{d}\left(X\left(s_{1}\right),\ldots,X\left(s_{k}\right)\right).
\end{eqnarray*}
\end{Def}

\begin{Note}
Un proceso de Markov es estacionario si $X\left(t\right)=_{d}X\left(0\right)$, $t\geq0$.
\end{Note}

Considerese el proceso $N\left(t\right)=\sum_{n}\indora\left(\tau_{n}\leq t\right)$ en $\rea_{+}$, con puntos $0<\tau_{1}<\tau_{2}<\cdots$.

\begin{Prop}
Si $N$ es un proceso puntual estacionario y $\esp\left[N\left(1\right)\right]<\infty$, entonces $\esp\left[N\left(t\right)\right]=t\esp\left[N\left(1\right)\right]$, $t\geq0$

\end{Prop}

\begin{Teo}
Los siguientes enunciados son equivalentes
\begin{itemize}
\item[i)] El proceso retardado de renovaci\'on $N$ es estacionario.

\item[ii)] EL proceso de tiempos de recurrencia hacia adelante $B\left(t\right)$ es estacionario.


\item[iii)] $\esp\left[N\left(t\right)\right]=t/\mu$,


\item[iv)] $G\left(t\right)=F_{e}\left(t\right)=\frac{1}{\mu}\int_{0}^{t}\left[1-F\left(s\right)\right]ds$
\end{itemize}
Cuando estos enunciados son ciertos, $P\left\{B\left(t\right)\leq x\right\}=F_{e}\left(x\right)$, para $t,x\geq0$.

\end{Teo}

\begin{Note}
Una consecuencia del teorema anterior es que el Proceso Poisson es el \'unico proceso sin retardo que es estacionario.
\end{Note}

\begin{Coro}
El proceso de renovaci\'on $N\left(t\right)$ sin retardo, y cuyos tiempos de inter renonaci\'on tienen media finita, es estacionario si y s\'olo si es un proceso Poisson.

\end{Coro}

%________________________________________________________________________
\subsection*{Procesos Regenerativos}
%________________________________________________________________________



\begin{Note}
Si $\tilde{X}\left(t\right)$ con espacio de estados $\tilde{S}$ es regenerativo sobre $T_{n}$, entonces $X\left(t\right)=f\left(\tilde{X}\left(t\right)\right)$ tambi\'en es regenerativo sobre $T_{n}$, para cualquier funci\'on $f:\tilde{S}\rightarrow S$.
\end{Note}

\begin{Note}
Los procesos regenerativos son crudamente regenerativos, pero no al rev\'es.
\end{Note}
%\subsection*{Procesos Regenerativos: Sigman\cite{Sigman1}}
\begin{Def}[Definici\'on Cl\'asica]
Un proceso estoc\'astico $X=\left\{X\left(t\right):t\geq0\right\}$ es llamado regenerativo is existe una variable aleatoria $R_{1}>0$ tal que
\begin{itemize}
\item[i)] $\left\{X\left(t+R_{1}\right):t\geq0\right\}$ es independiente de $\left\{\left\{X\left(t\right):t<R_{1}\right\},\right\}$
\item[ii)] $\left\{X\left(t+R_{1}\right):t\geq0\right\}$ es estoc\'asticamente equivalente a $\left\{X\left(t\right):t>0\right\}$
\end{itemize}

Llamamos a $R_{1}$ tiempo de regeneraci\'on, y decimos que $X$ se regenera en este punto.
\end{Def}

$\left\{X\left(t+R_{1}\right)\right\}$ es regenerativo con tiempo de regeneraci\'on $R_{2}$, independiente de $R_{1}$ pero con la misma distribuci\'on que $R_{1}$. Procediendo de esta manera se obtiene una secuencia de variables aleatorias independientes e id\'enticamente distribuidas $\left\{R_{n}\right\}$ llamados longitudes de ciclo. Si definimos a $Z_{k}\equiv R_{1}+R_{2}+\cdots+R_{k}$, se tiene un proceso de renovaci\'on llamado proceso de renovaci\'on encajado para $X$.




\begin{Def}
Para $x$ fijo y para cada $t\geq0$, sea $I_{x}\left(t\right)=1$ si $X\left(t\right)\leq x$,  $I_{x}\left(t\right)=0$ en caso contrario, y def\'inanse los tiempos promedio
\begin{eqnarray*}
\overline{X}&=&lim_{t\rightarrow\infty}\frac{1}{t}\int_{0}^{\infty}X\left(u\right)du\\
\prob\left(X_{\infty}\leq x\right)&=&lim_{t\rightarrow\infty}\frac{1}{t}\int_{0}^{\infty}I_{x}\left(u\right)du,
\end{eqnarray*}
cuando estos l\'imites existan.
\end{Def}

Como consecuencia del teorema de Renovaci\'on-Recompensa, se tiene que el primer l\'imite  existe y es igual a la constante
\begin{eqnarray*}
\overline{X}&=&\frac{\esp\left[\int_{0}^{R_{1}}X\left(t\right)dt\right]}{\esp\left[R_{1}\right]},
\end{eqnarray*}
suponiendo que ambas esperanzas son finitas.

\begin{Note}
\begin{itemize}
\item[a)] Si el proceso regenerativo $X$ es positivo recurrente y tiene trayectorias muestrales no negativas, entonces la ecuaci\'on anterior es v\'alida.
\item[b)] Si $X$ es positivo recurrente regenerativo, podemos construir una \'unica versi\'on estacionaria de este proceso, $X_{e}=\left\{X_{e}\left(t\right)\right\}$, donde $X_{e}$ es un proceso estoc\'astico regenerativo y estrictamente estacionario, con distribuci\'on marginal distribuida como $X_{\infty}$
\end{itemize}
\end{Note}

Para $\left\{X\left(t\right):t\geq0\right\}$ Proceso Estoc\'astico a tiempo continuo con estado de espacios $S$, que es un espacio m\'etrico, con trayectorias continuas por la derecha y con l\'imites por la izquierda c.s. Sea $N\left(t\right)$ un proceso de renovaci\'on en $\rea_{+}$ definido en el mismo espacio de probabilidad que $X\left(t\right)$, con tiempos de renovaci\'on $T$ y tiempos de inter-renovaci\'on $\xi_{n}=T_{n}-T_{n-1}$, con misma distribuci\'on $F$ de media finita $\mu$.


\begin{Def}
Para el proceso $\left\{\left(N\left(t\right),X\left(t\right)\right):t\geq0\right\}$, sus trayectoria muestrales en el intervalo de tiempo $\left[T_{n-1},T_{n}\right)$ est\'an descritas por
\begin{eqnarray*}
\zeta_{n}=\left(\xi_{n},\left\{X\left(T_{n-1}+t\right):0\leq t<\xi_{n}\right\}\right)
\end{eqnarray*}
Este $\zeta_{n}$ es el $n$-\'esimo segmento del proceso. El proceso es regenerativo sobre los tiempos $T_{n}$ si sus segmentos $\zeta_{n}$ son independientes e id\'enticamennte distribuidos.
\end{Def}


\begin{Note}
Si $\tilde{X}\left(t\right)$ con espacio de estados $\tilde{S}$ es regenerativo sobre $T_{n}$, entonces $X\left(t\right)=f\left(\tilde{X}\left(t\right)\right)$ tambi\'en es regenerativo sobre $T_{n}$, para cualquier funci\'on $f:\tilde{S}\rightarrow S$.
\end{Note}

\begin{Note}
Los procesos regenerativos son crudamente regenerativos, pero no al rev\'es.
\end{Note}

\begin{Def}[Definici\'on Cl\'asica]
Un proceso estoc\'astico $X=\left\{X\left(t\right):t\geq0\right\}$ es llamado regenerativo is existe una variable aleatoria $R_{1}>0$ tal que
\begin{itemize}
\item[i)] $\left\{X\left(t+R_{1}\right):t\geq0\right\}$ es independiente de $\left\{\left\{X\left(t\right):t<R_{1}\right\},\right\}$
\item[ii)] $\left\{X\left(t+R_{1}\right):t\geq0\right\}$ es estoc\'asticamente equivalente a $\left\{X\left(t\right):t>0\right\}$
\end{itemize}

Llamamos a $R_{1}$ tiempo de regeneraci\'on, y decimos que $X$ se regenera en este punto.
\end{Def}

$\left\{X\left(t+R_{1}\right)\right\}$ es regenerativo con tiempo de regeneraci\'on $R_{2}$, independiente de $R_{1}$ pero con la misma distribuci\'on que $R_{1}$. Procediendo de esta manera se obtiene una secuencia de variables aleatorias independientes e id\'enticamente distribuidas $\left\{R_{n}\right\}$ llamados longitudes de ciclo. Si definimos a $Z_{k}\equiv R_{1}+R_{2}+\cdots+R_{k}$, se tiene un proceso de renovaci\'on llamado proceso de renovaci\'on encajado para $X$.

\begin{Note}
Un proceso regenerativo con media de la longitud de ciclo finita es llamado positivo recurrente.
\end{Note}


\begin{Def}
Para $x$ fijo y para cada $t\geq0$, sea $I_{x}\left(t\right)=1$ si $X\left(t\right)\leq x$,  $I_{x}\left(t\right)=0$ en caso contrario, y def\'inanse los tiempos promedio
\begin{eqnarray*}
\overline{X}&=&lim_{t\rightarrow\infty}\frac{1}{t}\int_{0}^{\infty}X\left(u\right)du\\
\prob\left(X_{\infty}\leq x\right)&=&lim_{t\rightarrow\infty}\frac{1}{t}\int_{0}^{\infty}I_{x}\left(u\right)du,
\end{eqnarray*}
cuando estos l\'imites existan.
\end{Def}

Como consecuencia del teorema de Renovaci\'on-Recompensa, se tiene que el primer l\'imite  existe y es igual a la constante
\begin{eqnarray*}
\overline{X}&=&\frac{\esp\left[\int_{0}^{R_{1}}X\left(t\right)dt\right]}{\esp\left[R_{1}\right]},
\end{eqnarray*}
suponiendo que ambas esperanzas son finitas.

\begin{Note}
\begin{itemize}
\item[a)] Si el proceso regenerativo $X$ es positivo recurrente y tiene trayectorias muestrales no negativas, entonces la ecuaci\'on anterior es v\'alida.
\item[b)] Si $X$ es positivo recurrente regenerativo, podemos construir una \'unica versi\'on estacionaria de este proceso, $X_{e}=\left\{X_{e}\left(t\right)\right\}$, donde $X_{e}$ es un proceso estoc\'astico regenerativo y estrictamente estacionario, con distribuci\'on marginal distribuida como $X_{\infty}$
\end{itemize}
\end{Note}

%__________________________________________________________________________________________
\subsection{Procesos Regenerativos Estacionarios - Stidham \cite{Stidham}}
%__________________________________________________________________________________________


Un proceso estoc\'astico a tiempo continuo $\left\{V\left(t\right),t\geq0\right\}$ es un proceso regenerativo si existe una sucesi\'on de variables aleatorias independientes e id\'enticamente distribuidas $\left\{X_{1},X_{2},\ldots\right\}$, sucesi\'on de renovaci\'on, tal que para cualquier conjunto de Borel $A$, 

\begin{eqnarray*}
\prob\left\{V\left(t\right)\in A|X_{1}+X_{2}+\cdots+X_{R\left(t\right)}=s,\left\{V\left(\tau\right),\tau<s\right\}\right\}=\prob\left\{V\left(t-s\right)\in A|X_{1}>t-s\right\},
\end{eqnarray*}
para todo $0\leq s\leq t$, donde $R\left(t\right)=\max\left\{X_{1}+X_{2}+\cdots+X_{j}\leq t\right\}=$n\'umero de renovaciones ({\emph{puntos de regeneraci\'on}}) que ocurren en $\left[0,t\right]$. El intervalo $\left[0,X_{1}\right)$ es llamado {\emph{primer ciclo de regeneraci\'on}} de $\left\{V\left(t \right),t\geq0\right\}$, $\left[X_{1},X_{1}+X_{2}\right)$ el {\emph{segundo ciclo de regeneraci\'on}}, y as\'i sucesivamente.

Sea $X=X_{1}$ y sea $F$ la funci\'on de distrbuci\'on de $X$


\begin{Def}
Se define el proceso estacionario, $\left\{V^{*}\left(t\right),t\geq0\right\}$, para $\left\{V\left(t\right),t\geq0\right\}$ por

\begin{eqnarray*}
\prob\left\{V\left(t\right)\in A\right\}=\frac{1}{\esp\left[X\right]}\int_{0}^{\infty}\prob\left\{V\left(t+x\right)\in A|X>x\right\}\left(1-F\left(x\right)\right)dx,
\end{eqnarray*} 
para todo $t\geq0$ y todo conjunto de Borel $A$.
\end{Def}

\begin{Def}
Una distribuci\'on se dice que es {\emph{aritm\'etica}} si todos sus puntos de incremento son m\'ultiplos de la forma $0,\lambda, 2\lambda,\ldots$ para alguna $\lambda>0$ entera.
\end{Def}


\begin{Def}
Una modificaci\'on medible de un proceso $\left\{V\left(t\right),t\geq0\right\}$, es una versi\'on de este, $\left\{V\left(t,w\right)\right\}$ conjuntamente medible para $t\geq0$ y para $w\in S$, $S$ espacio de estados para $\left\{V\left(t\right),t\geq0\right\}$.
\end{Def}

\begin{Teo}
Sea $\left\{V\left(t\right),t\geq\right\}$ un proceso regenerativo no negativo con modificaci\'on medible. Sea $\esp\left[X\right]<\infty$. Entonces el proceso estacionario dado por la ecuaci\'on anterior est\'a bien definido y tiene funci\'on de distribuci\'on independiente de $t$, adem\'as
\begin{itemize}
\item[i)] \begin{eqnarray*}
\esp\left[V^{*}\left(0\right)\right]&=&\frac{\esp\left[\int_{0}^{X}V\left(s\right)ds\right]}{\esp\left[X\right]}\end{eqnarray*}
\item[ii)] Si $\esp\left[V^{*}\left(0\right)\right]<\infty$, equivalentemente, si $\esp\left[\int_{0}^{X}V\left(s\right)ds\right]<\infty$,entonces
\begin{eqnarray*}
\frac{\int_{0}^{t}V\left(s\right)ds}{t}\rightarrow\frac{\esp\left[\int_{0}^{X}V\left(s\right)ds\right]}{\esp\left[X\right]}
\end{eqnarray*}
con probabilidad 1 y en media, cuando $t\rightarrow\infty$.
\end{itemize}
\end{Teo}
%_________________________________________________________________________
%
\subsection{Output Process and Regenerative Processes}
%_________________________________________________________________________
%
En Sigman, Thorison y Wolff \cite{Sigman2} prueban que para la existencia de un una sucesi\'on infinita no decreciente de tiempos de regeneraci\'on $\tau_{1}\leq\tau_{2}\leq\cdots$ en los cuales el proceso se regenera, basta un tiempo de regeneraci\'on $R_{1}$, donde $R_{j}=\tau_{j}-\tau_{j-1}$. Para tal efecto se requiere la existencia de un espacio de probabilidad $\left(\Omega,\mathcal{F},\prob\right)$, y proceso estoc\'astico $\textit{X}=\left\{X\left(t\right):t\geq0\right\}$ con espacio de estados $\left(S,\mathcal{R}\right)$, con $\mathcal{R}$ $\sigma$-\'algebra.

\begin{Prop}
Si existe una variable aleatoria no negativa $R_{1}$ tal que $\theta_{R1}X=_{D}X$, entonces $\left(\Omega,\mathcal{F},\prob\right)$ puede extenderse para soportar una sucesi\'on estacionaria de variables aleatorias $R=\left\{R_{k}:k\geq1\right\}$, tal que para $k\geq1$,
\begin{eqnarray*}
\theta_{k}\left(X,R\right)=_{D}\left(X,R\right).
\end{eqnarray*}

Adem\'as, para $k\geq1$, $\theta_{k}R$ es condicionalmente independiente de $\left(X,R_{1},\ldots,R_{k}\right)$, dado $\theta_{\tau k}X$.

\end{Prop}


\begin{itemize}
\item Doob en 1953 demostr\'o que el estado estacionario de un proceso de partida en un sistema de espera $M/G/\infty$, es Poisson con la misma tasa que el proceso de arribos.

\item Burke en 1968, fue el primero en demostrar que el estado estacionario de un proceso de salida de una cola $M/M/s$ es un proceso Poisson.

\item Disney en 1973 obtuvo el siguiente resultado:

\begin{Teo}
Para el sistema de espera $M/G/1/L$ con disciplina FIFO, el proceso $\textbf{I}$ es un proceso de renovaci\'on si y s\'olo si el proceso denominado longitud de la cola es estacionario y se cumple cualquiera de los siguientes casos:

\begin{itemize}
\item[a)] Los tiempos de servicio son identicamente cero;
\item[b)] $L=0$, para cualquier proceso de servicio $S$;
\item[c)] $L=1$ y $G=D$;
\item[d)] $L=\infty$ y $G=M$.
\end{itemize}
En estos casos, respectivamente, las distribuciones de interpartida $P\left\{T_{n+1}-T_{n}\leq t\right\}$ son


\begin{itemize}
\item[a)] $1-e^{-\lambda t}$, $t\geq0$;
\item[b)] $1-e^{-\lambda t}*F\left(t\right)$, $t\geq0$;
\item[c)] $1-e^{-\lambda t}*\indora_{d}\left(t\right)$, $t\geq0$;
\item[d)] $1-e^{-\lambda t}*F\left(t\right)$, $t\geq0$.
\end{itemize}
\end{Teo}


\item Finch (1959) mostr\'o que para los sistemas $M/G/1/L$, con $1\leq L\leq \infty$ con distribuciones de servicio dos veces diferenciable, solamente el sistema $M/M/1/\infty$ tiene proceso de salida de renovaci\'on estacionario.

\item King (1971) demostro que un sistema de colas estacionario $M/G/1/1$ tiene sus tiempos de interpartida sucesivas $D_{n}$ y $D_{n+1}$ son independientes, si y s\'olo si, $G=D$, en cuyo caso le proceso de salida es de renovaci\'on.

\item Disney (1973) demostr\'o que el \'unico sistema estacionario $M/G/1/L$, que tiene proceso de salida de renovaci\'on  son los sistemas $M/M/1$ y $M/D/1/1$.



\item El siguiente resultado es de Disney y Koning (1985)
\begin{Teo}
En un sistema de espera $M/G/s$, el estado estacionario del proceso de salida es un proceso Poisson para cualquier distribuci\'on de los tiempos de servicio si el sistema tiene cualquiera de las siguientes cuatro propiedades.

\begin{itemize}
\item[a)] $s=\infty$
\item[b)] La disciplina de servicio es de procesador compartido.
\item[c)] La disciplina de servicio es LCFS y preemptive resume, esto se cumple para $L<\infty$
\item[d)] $G=M$.
\end{itemize}

\end{Teo}

\item El siguiente resultado es de Alamatsaz (1983)

\begin{Teo}
En cualquier sistema de colas $GI/G/1/L$ con $1\leq L<\infty$ y distribuci\'on de interarribos $A$ y distribuci\'on de los tiempos de servicio $B$, tal que $A\left(0\right)=0$, $A\left(t\right)\left(1-B\left(t\right)\right)>0$ para alguna $t>0$ y $B\left(t\right)$ para toda $t>0$, es imposible que el proceso de salida estacionario sea de renovaci\'on.
\end{Teo}

\end{itemize}



%________________________________________________________________________
\subsection{Procesos Regenerativos Sigman, Thorisson y Wolff \cite{Sigman1}}
%________________________________________________________________________


\begin{Def}[Definici\'on Cl\'asica]
Un proceso estoc\'astico $X=\left\{X\left(t\right):t\geq0\right\}$ es llamado regenerativo is existe una variable aleatoria $R_{1}>0$ tal que
\begin{itemize}
\item[i)] $\left\{X\left(t+R_{1}\right):t\geq0\right\}$ es independiente de $\left\{\left\{X\left(t\right):t<R_{1}\right\},\right\}$
\item[ii)] $\left\{X\left(t+R_{1}\right):t\geq0\right\}$ es estoc\'asticamente equivalente a $\left\{X\left(t\right):t>0\right\}$
\end{itemize}

Llamamos a $R_{1}$ tiempo de regeneraci\'on, y decimos que $X$ se regenera en este punto.
\end{Def}

$\left\{X\left(t+R_{1}\right)\right\}$ es regenerativo con tiempo de regeneraci\'on $R_{2}$, independiente de $R_{1}$ pero con la misma distribuci\'on que $R_{1}$. Procediendo de esta manera se obtiene una secuencia de variables aleatorias independientes e id\'enticamente distribuidas $\left\{R_{n}\right\}$ llamados longitudes de ciclo. Si definimos a $Z_{k}\equiv R_{1}+R_{2}+\cdots+R_{k}$, se tiene un proceso de renovaci\'on llamado proceso de renovaci\'on encajado para $X$.


\begin{Note}
La existencia de un primer tiempo de regeneraci\'on, $R_{1}$, implica la existencia de una sucesi\'on completa de estos tiempos $R_{1},R_{2}\ldots,$ que satisfacen la propiedad deseada \cite{Sigman2}.
\end{Note}


\begin{Note} Para la cola $GI/GI/1$ los usuarios arriban con tiempos $t_{n}$ y son atendidos con tiempos de servicio $S_{n}$, los tiempos de arribo forman un proceso de renovaci\'on  con tiempos entre arribos independientes e identicamente distribuidos (\texttt{i.i.d.})$T_{n}=t_{n}-t_{n-1}$, adem\'as los tiempos de servicio son \texttt{i.i.d.} e independientes de los procesos de arribo. Por \textit{estable} se entiende que $\esp S_{n}<\esp T_{n}<\infty$.
\end{Note}
 


\begin{Def}
Para $x$ fijo y para cada $t\geq0$, sea $I_{x}\left(t\right)=1$ si $X\left(t\right)\leq x$,  $I_{x}\left(t\right)=0$ en caso contrario, y def\'inanse los tiempos promedio
\begin{eqnarray*}
\overline{X}&=&lim_{t\rightarrow\infty}\frac{1}{t}\int_{0}^{\infty}X\left(u\right)du\\
\prob\left(X_{\infty}\leq x\right)&=&lim_{t\rightarrow\infty}\frac{1}{t}\int_{0}^{\infty}I_{x}\left(u\right)du,
\end{eqnarray*}
cuando estos l\'imites existan.
\end{Def}

Como consecuencia del teorema de Renovaci\'on-Recompensa, se tiene que el primer l\'imite  existe y es igual a la constante
\begin{eqnarray*}
\overline{X}&=&\frac{\esp\left[\int_{0}^{R_{1}}X\left(t\right)dt\right]}{\esp\left[R_{1}\right]},
\end{eqnarray*}
suponiendo que ambas esperanzas son finitas.
 
\begin{Note}
Funciones de procesos regenerativos son regenerativas, es decir, si $X\left(t\right)$ es regenerativo y se define el proceso $Y\left(t\right)$ por $Y\left(t\right)=f\left(X\left(t\right)\right)$ para alguna funci\'on Borel medible $f\left(\cdot\right)$. Adem\'as $Y$ es regenerativo con los mismos tiempos de renovaci\'on que $X$. 

En general, los tiempos de renovaci\'on, $Z_{k}$ de un proceso regenerativo no requieren ser tiempos de paro con respecto a la evoluci\'on de $X\left(t\right)$.
\end{Note} 

\begin{Note}
Una funci\'on de un proceso de Markov, usualmente no ser\'a un proceso de Markov, sin embargo ser\'a regenerativo si el proceso de Markov lo es.
\end{Note}

 
\begin{Note}
Un proceso regenerativo con media de la longitud de ciclo finita es llamado positivo recurrente.
\end{Note}


\begin{Note}
\begin{itemize}
\item[a)] Si el proceso regenerativo $X$ es positivo recurrente y tiene trayectorias muestrales no negativas, entonces la ecuaci\'on anterior es v\'alida.
\item[b)] Si $X$ es positivo recurrente regenerativo, podemos construir una \'unica versi\'on estacionaria de este proceso, $X_{e}=\left\{X_{e}\left(t\right)\right\}$, donde $X_{e}$ es un proceso estoc\'astico regenerativo y estrictamente estacionario, con distribuci\'on marginal distribuida como $X_{\infty}$
\end{itemize}
\end{Note}


%__________________________________________________________________________________________
%\subsection{Procesos Regenerativos Estacionarios - Stidham \cite{Stidham}}
%__________________________________________________________________________________________


Un proceso estoc\'astico a tiempo continuo $\left\{V\left(t\right),t\geq0\right\}$ es un proceso regenerativo si existe una sucesi\'on de variables aleatorias independientes e id\'enticamente distribuidas $\left\{X_{1},X_{2},\ldots\right\}$, sucesi\'on de renovaci\'on, tal que para cualquier conjunto de Borel $A$, 

\begin{eqnarray*}
\prob\left\{V\left(t\right)\in A|X_{1}+X_{2}+\cdots+X_{R\left(t\right)}=s,\left\{V\left(\tau\right),\tau<s\right\}\right\}=\prob\left\{V\left(t-s\right)\in A|X_{1}>t-s\right\},
\end{eqnarray*}
para todo $0\leq s\leq t$, donde $R\left(t\right)=\max\left\{X_{1}+X_{2}+\cdots+X_{j}\leq t\right\}=$n\'umero de renovaciones ({\emph{puntos de regeneraci\'on}}) que ocurren en $\left[0,t\right]$. El intervalo $\left[0,X_{1}\right)$ es llamado {\emph{primer ciclo de regeneraci\'on}} de $\left\{V\left(t \right),t\geq0\right\}$, $\left[X_{1},X_{1}+X_{2}\right)$ el {\emph{segundo ciclo de regeneraci\'on}}, y as\'i sucesivamente.

Sea $X=X_{1}$ y sea $F$ la funci\'on de distrbuci\'on de $X$


\begin{Def}
Se define el proceso estacionario, $\left\{V^{*}\left(t\right),t\geq0\right\}$, para $\left\{V\left(t\right),t\geq0\right\}$ por

\begin{eqnarray*}
\prob\left\{V\left(t\right)\in A\right\}=\frac{1}{\esp\left[X\right]}\int_{0}^{\infty}\prob\left\{V\left(t+x\right)\in A|X>x\right\}\left(1-F\left(x\right)\right)dx,
\end{eqnarray*} 
para todo $t\geq0$ y todo conjunto de Borel $A$.
\end{Def}

\begin{Def}
Una distribuci\'on se dice que es {\emph{aritm\'etica}} si todos sus puntos de incremento son m\'ultiplos de la forma $0,\lambda, 2\lambda,\ldots$ para alguna $\lambda>0$ entera.
\end{Def}


\begin{Def}
Una modificaci\'on medible de un proceso $\left\{V\left(t\right),t\geq0\right\}$, es una versi\'on de este, $\left\{V\left(t,w\right)\right\}$ conjuntamente medible para $t\geq0$ y para $w\in S$, $S$ espacio de estados para $\left\{V\left(t\right),t\geq0\right\}$.
\end{Def}

\begin{Teo}
Sea $\left\{V\left(t\right),t\geq\right\}$ un proceso regenerativo no negativo con modificaci\'on medible. Sea $\esp\left[X\right]<\infty$. Entonces el proceso estacionario dado por la ecuaci\'on anterior est\'a bien definido y tiene funci\'on de distribuci\'on independiente de $t$, adem\'as
\begin{itemize}
\item[i)] \begin{eqnarray*}
\esp\left[V^{*}\left(0\right)\right]&=&\frac{\esp\left[\int_{0}^{X}V\left(s\right)ds\right]}{\esp\left[X\right]}\end{eqnarray*}
\item[ii)] Si $\esp\left[V^{*}\left(0\right)\right]<\infty$, equivalentemente, si $\esp\left[\int_{0}^{X}V\left(s\right)ds\right]<\infty$,entonces
\begin{eqnarray*}
\frac{\int_{0}^{t}V\left(s\right)ds}{t}\rightarrow\frac{\esp\left[\int_{0}^{X}V\left(s\right)ds\right]}{\esp\left[X\right]}
\end{eqnarray*}
con probabilidad 1 y en media, cuando $t\rightarrow\infty$.
\end{itemize}
\end{Teo}

\begin{Coro}
Sea $\left\{V\left(t\right),t\geq0\right\}$ un proceso regenerativo no negativo, con modificaci\'on medible. Si $\esp <\infty$, $F$ es no-aritm\'etica, y para todo $x\geq0$, $P\left\{V\left(t\right)\leq x,C>x\right\}$ es de variaci\'on acotada como funci\'on de $t$ en cada intervalo finito $\left[0,\tau\right]$, entonces $V\left(t\right)$ converge en distribuci\'on  cuando $t\rightarrow\infty$ y $$\esp V=\frac{\esp \int_{0}^{X}V\left(s\right)ds}{\esp X}$$
Donde $V$ tiene la distribuci\'on l\'imite de $V\left(t\right)$ cuando $t\rightarrow\infty$.

\end{Coro}

Para el caso discreto se tienen resultados similares.



%______________________________________________________________________
%\subsection{Procesos de Renovaci\'on}
%______________________________________________________________________

\begin{Def}%\label{Def.Tn}
Sean $0\leq T_{1}\leq T_{2}\leq \ldots$ son tiempos aleatorios infinitos en los cuales ocurren ciertos eventos. El n\'umero de tiempos $T_{n}$ en el intervalo $\left[0,t\right)$ es

\begin{eqnarray}
N\left(t\right)=\sum_{n=1}^{\infty}\indora\left(T_{n}\leq t\right),
\end{eqnarray}
para $t\geq0$.
\end{Def}

Si se consideran los puntos $T_{n}$ como elementos de $\rea_{+}$, y $N\left(t\right)$ es el n\'umero de puntos en $\rea$. El proceso denotado por $\left\{N\left(t\right):t\geq0\right\}$, denotado por $N\left(t\right)$, es un proceso puntual en $\rea_{+}$. Los $T_{n}$ son los tiempos de ocurrencia, el proceso puntual $N\left(t\right)$ es simple si su n\'umero de ocurrencias son distintas: $0<T_{1}<T_{2}<\ldots$ casi seguramente.

\begin{Def}
Un proceso puntual $N\left(t\right)$ es un proceso de renovaci\'on si los tiempos de interocurrencia $\xi_{n}=T_{n}-T_{n-1}$, para $n\geq1$, son independientes e identicamente distribuidos con distribuci\'on $F$, donde $F\left(0\right)=0$ y $T_{0}=0$. Los $T_{n}$ son llamados tiempos de renovaci\'on, referente a la independencia o renovaci\'on de la informaci\'on estoc\'astica en estos tiempos. Los $\xi_{n}$ son los tiempos de inter-renovaci\'on, y $N\left(t\right)$ es el n\'umero de renovaciones en el intervalo $\left[0,t\right)$
\end{Def}


\begin{Note}
Para definir un proceso de renovaci\'on para cualquier contexto, solamente hay que especificar una distribuci\'on $F$, con $F\left(0\right)=0$, para los tiempos de inter-renovaci\'on. La funci\'on $F$ en turno degune las otra variables aleatorias. De manera formal, existe un espacio de probabilidad y una sucesi\'on de variables aleatorias $\xi_{1},\xi_{2},\ldots$ definidas en este con distribuci\'on $F$. Entonces las otras cantidades son $T_{n}=\sum_{k=1}^{n}\xi_{k}$ y $N\left(t\right)=\sum_{n=1}^{\infty}\indora\left(T_{n}\leq t\right)$, donde $T_{n}\rightarrow\infty$ casi seguramente por la Ley Fuerte de los Grandes Números.
\end{Note}

%___________________________________________________________________________________________
%
%\subsection{Teorema Principal de Renovaci\'on}
%___________________________________________________________________________________________
%

\begin{Note} Una funci\'on $h:\rea_{+}\rightarrow\rea$ es Directamente Riemann Integrable en los siguientes casos:
\begin{itemize}
\item[a)] $h\left(t\right)\geq0$ es decreciente y Riemann Integrable.
\item[b)] $h$ es continua excepto posiblemente en un conjunto de Lebesgue de medida 0, y $|h\left(t\right)|\leq b\left(t\right)$, donde $b$ es DRI.
\end{itemize}
\end{Note}

\begin{Teo}[Teorema Principal de Renovaci\'on]
Si $F$ es no aritm\'etica y $h\left(t\right)$ es Directamente Riemann Integrable (DRI), entonces

\begin{eqnarray*}
lim_{t\rightarrow\infty}U\star h=\frac{1}{\mu}\int_{\rea_{+}}h\left(s\right)ds.
\end{eqnarray*}
\end{Teo}

\begin{Prop}
Cualquier funci\'on $H\left(t\right)$ acotada en intervalos finitos y que es 0 para $t<0$ puede expresarse como
\begin{eqnarray*}
H\left(t\right)=U\star h\left(t\right)\textrm{,  donde }h\left(t\right)=H\left(t\right)-F\star H\left(t\right)
\end{eqnarray*}
\end{Prop}

\begin{Def}
Un proceso estoc\'astico $X\left(t\right)$ es crudamente regenerativo en un tiempo aleatorio positivo $T$ si
\begin{eqnarray*}
\esp\left[X\left(T+t\right)|T\right]=\esp\left[X\left(t\right)\right]\textrm{, para }t\geq0,\end{eqnarray*}
y con las esperanzas anteriores finitas.
\end{Def}

\begin{Prop}
Sup\'ongase que $X\left(t\right)$ es un proceso crudamente regenerativo en $T$, que tiene distribuci\'on $F$. Si $\esp\left[X\left(t\right)\right]$ es acotado en intervalos finitos, entonces
\begin{eqnarray*}
\esp\left[X\left(t\right)\right]=U\star h\left(t\right)\textrm{,  donde }h\left(t\right)=\esp\left[X\left(t\right)\indora\left(T>t\right)\right].
\end{eqnarray*}
\end{Prop}

\begin{Teo}[Regeneraci\'on Cruda]
Sup\'ongase que $X\left(t\right)$ es un proceso con valores positivo crudamente regenerativo en $T$, y def\'inase $M=\sup\left\{|X\left(t\right)|:t\leq T\right\}$. Si $T$ es no aritm\'etico y $M$ y $MT$ tienen media finita, entonces
\begin{eqnarray*}
lim_{t\rightarrow\infty}\esp\left[X\left(t\right)\right]=\frac{1}{\mu}\int_{\rea_{+}}h\left(s\right)ds,
\end{eqnarray*}
donde $h\left(t\right)=\esp\left[X\left(t\right)\indora\left(T>t\right)\right]$.
\end{Teo}

%___________________________________________________________________________________________
%
%\subsection{Propiedades de los Procesos de Renovaci\'on}
%___________________________________________________________________________________________
%

Los tiempos $T_{n}$ est\'an relacionados con los conteos de $N\left(t\right)$ por

\begin{eqnarray*}
\left\{N\left(t\right)\geq n\right\}&=&\left\{T_{n}\leq t\right\}\\
T_{N\left(t\right)}\leq &t&<T_{N\left(t\right)+1},
\end{eqnarray*}

adem\'as $N\left(T_{n}\right)=n$, y 

\begin{eqnarray*}
N\left(t\right)=\max\left\{n:T_{n}\leq t\right\}=\min\left\{n:T_{n+1}>t\right\}
\end{eqnarray*}

Por propiedades de la convoluci\'on se sabe que

\begin{eqnarray*}
P\left\{T_{n}\leq t\right\}=F^{n\star}\left(t\right)
\end{eqnarray*}
que es la $n$-\'esima convoluci\'on de $F$. Entonces 

\begin{eqnarray*}
\left\{N\left(t\right)\geq n\right\}&=&\left\{T_{n}\leq t\right\}\\
P\left\{N\left(t\right)\leq n\right\}&=&1-F^{\left(n+1\right)\star}\left(t\right)
\end{eqnarray*}

Adem\'as usando el hecho de que $\esp\left[N\left(t\right)\right]=\sum_{n=1}^{\infty}P\left\{N\left(t\right)\geq n\right\}$
se tiene que

\begin{eqnarray*}
\esp\left[N\left(t\right)\right]=\sum_{n=1}^{\infty}F^{n\star}\left(t\right)
\end{eqnarray*}

\begin{Prop}
Para cada $t\geq0$, la funci\'on generadora de momentos $\esp\left[e^{\alpha N\left(t\right)}\right]$ existe para alguna $\alpha$ en una vecindad del 0, y de aqu\'i que $\esp\left[N\left(t\right)^{m}\right]<\infty$, para $m\geq1$.
\end{Prop}


\begin{Note}
Si el primer tiempo de renovaci\'on $\xi_{1}$ no tiene la misma distribuci\'on que el resto de las $\xi_{n}$, para $n\geq2$, a $N\left(t\right)$ se le llama Proceso de Renovaci\'on retardado, donde si $\xi$ tiene distribuci\'on $G$, entonces el tiempo $T_{n}$ de la $n$-\'esima renovaci\'on tiene distribuci\'on $G\star F^{\left(n-1\right)\star}\left(t\right)$
\end{Note}


\begin{Teo}
Para una constante $\mu\leq\infty$ ( o variable aleatoria), las siguientes expresiones son equivalentes:

\begin{eqnarray}
lim_{n\rightarrow\infty}n^{-1}T_{n}&=&\mu,\textrm{ c.s.}\\
lim_{t\rightarrow\infty}t^{-1}N\left(t\right)&=&1/\mu,\textrm{ c.s.}
\end{eqnarray}
\end{Teo}


Es decir, $T_{n}$ satisface la Ley Fuerte de los Grandes N\'umeros s\'i y s\'olo s\'i $N\left/t\right)$ la cumple.


\begin{Coro}[Ley Fuerte de los Grandes N\'umeros para Procesos de Renovaci\'on]
Si $N\left(t\right)$ es un proceso de renovaci\'on cuyos tiempos de inter-renovaci\'on tienen media $\mu\leq\infty$, entonces
\begin{eqnarray}
t^{-1}N\left(t\right)\rightarrow 1/\mu,\textrm{ c.s. cuando }t\rightarrow\infty.
\end{eqnarray}

\end{Coro}


Considerar el proceso estoc\'astico de valores reales $\left\{Z\left(t\right):t\geq0\right\}$ en el mismo espacio de probabilidad que $N\left(t\right)$

\begin{Def}
Para el proceso $\left\{Z\left(t\right):t\geq0\right\}$ se define la fluctuaci\'on m\'axima de $Z\left(t\right)$ en el intervalo $\left(T_{n-1},T_{n}\right]$:
\begin{eqnarray*}
M_{n}=\sup_{T_{n-1}<t\leq T_{n}}|Z\left(t\right)-Z\left(T_{n-1}\right)|
\end{eqnarray*}
\end{Def}

\begin{Teo}
Sup\'ongase que $n^{-1}T_{n}\rightarrow\mu$ c.s. cuando $n\rightarrow\infty$, donde $\mu\leq\infty$ es una constante o variable aleatoria. Sea $a$ una constante o variable aleatoria que puede ser infinita cuando $\mu$ es finita, y considere las expresiones l\'imite:
\begin{eqnarray}
lim_{n\rightarrow\infty}n^{-1}Z\left(T_{n}\right)&=&a,\textrm{ c.s.}\\
lim_{t\rightarrow\infty}t^{-1}Z\left(t\right)&=&a/\mu,\textrm{ c.s.}
\end{eqnarray}
La segunda expresi\'on implica la primera. Conversamente, la primera implica la segunda si el proceso $Z\left(t\right)$ es creciente, o si $lim_{n\rightarrow\infty}n^{-1}M_{n}=0$ c.s.
\end{Teo}

\begin{Coro}
Si $N\left(t\right)$ es un proceso de renovaci\'on, y $\left(Z\left(T_{n}\right)-Z\left(T_{n-1}\right),M_{n}\right)$, para $n\geq1$, son variables aleatorias independientes e id\'enticamente distribuidas con media finita, entonces,
\begin{eqnarray}
lim_{t\rightarrow\infty}t^{-1}Z\left(t\right)\rightarrow\frac{\esp\left[Z\left(T_{1}\right)-Z\left(T_{0}\right)\right]}{\esp\left[T_{1}\right]},\textrm{ c.s. cuando  }t\rightarrow\infty.
\end{eqnarray}
\end{Coro}



%___________________________________________________________________________________________
%
%\subsection{Propiedades de los Procesos de Renovaci\'on}
%___________________________________________________________________________________________
%

Los tiempos $T_{n}$ est\'an relacionados con los conteos de $N\left(t\right)$ por

\begin{eqnarray*}
\left\{N\left(t\right)\geq n\right\}&=&\left\{T_{n}\leq t\right\}\\
T_{N\left(t\right)}\leq &t&<T_{N\left(t\right)+1},
\end{eqnarray*}

adem\'as $N\left(T_{n}\right)=n$, y 

\begin{eqnarray*}
N\left(t\right)=\max\left\{n:T_{n}\leq t\right\}=\min\left\{n:T_{n+1}>t\right\}
\end{eqnarray*}

Por propiedades de la convoluci\'on se sabe que

\begin{eqnarray*}
P\left\{T_{n}\leq t\right\}=F^{n\star}\left(t\right)
\end{eqnarray*}
que es la $n$-\'esima convoluci\'on de $F$. Entonces 

\begin{eqnarray*}
\left\{N\left(t\right)\geq n\right\}&=&\left\{T_{n}\leq t\right\}\\
P\left\{N\left(t\right)\leq n\right\}&=&1-F^{\left(n+1\right)\star}\left(t\right)
\end{eqnarray*}

Adem\'as usando el hecho de que $\esp\left[N\left(t\right)\right]=\sum_{n=1}^{\infty}P\left\{N\left(t\right)\geq n\right\}$
se tiene que

\begin{eqnarray*}
\esp\left[N\left(t\right)\right]=\sum_{n=1}^{\infty}F^{n\star}\left(t\right)
\end{eqnarray*}

\begin{Prop}
Para cada $t\geq0$, la funci\'on generadora de momentos $\esp\left[e^{\alpha N\left(t\right)}\right]$ existe para alguna $\alpha$ en una vecindad del 0, y de aqu\'i que $\esp\left[N\left(t\right)^{m}\right]<\infty$, para $m\geq1$.
\end{Prop}


\begin{Note}
Si el primer tiempo de renovaci\'on $\xi_{1}$ no tiene la misma distribuci\'on que el resto de las $\xi_{n}$, para $n\geq2$, a $N\left(t\right)$ se le llama Proceso de Renovaci\'on retardado, donde si $\xi$ tiene distribuci\'on $G$, entonces el tiempo $T_{n}$ de la $n$-\'esima renovaci\'on tiene distribuci\'on $G\star F^{\left(n-1\right)\star}\left(t\right)$
\end{Note}


\begin{Teo}
Para una constante $\mu\leq\infty$ ( o variable aleatoria), las siguientes expresiones son equivalentes:

\begin{eqnarray}
lim_{n\rightarrow\infty}n^{-1}T_{n}&=&\mu,\textrm{ c.s.}\\
lim_{t\rightarrow\infty}t^{-1}N\left(t\right)&=&1/\mu,\textrm{ c.s.}
\end{eqnarray}
\end{Teo}


Es decir, $T_{n}$ satisface la Ley Fuerte de los Grandes N\'umeros s\'i y s\'olo s\'i $N\left/t\right)$ la cumple.


\begin{Coro}[Ley Fuerte de los Grandes N\'umeros para Procesos de Renovaci\'on]
Si $N\left(t\right)$ es un proceso de renovaci\'on cuyos tiempos de inter-renovaci\'on tienen media $\mu\leq\infty$, entonces
\begin{eqnarray}
t^{-1}N\left(t\right)\rightarrow 1/\mu,\textrm{ c.s. cuando }t\rightarrow\infty.
\end{eqnarray}

\end{Coro}


Considerar el proceso estoc\'astico de valores reales $\left\{Z\left(t\right):t\geq0\right\}$ en el mismo espacio de probabilidad que $N\left(t\right)$

\begin{Def}
Para el proceso $\left\{Z\left(t\right):t\geq0\right\}$ se define la fluctuaci\'on m\'axima de $Z\left(t\right)$ en el intervalo $\left(T_{n-1},T_{n}\right]$:
\begin{eqnarray*}
M_{n}=\sup_{T_{n-1}<t\leq T_{n}}|Z\left(t\right)-Z\left(T_{n-1}\right)|
\end{eqnarray*}
\end{Def}

\begin{Teo}
Sup\'ongase que $n^{-1}T_{n}\rightarrow\mu$ c.s. cuando $n\rightarrow\infty$, donde $\mu\leq\infty$ es una constante o variable aleatoria. Sea $a$ una constante o variable aleatoria que puede ser infinita cuando $\mu$ es finita, y considere las expresiones l\'imite:
\begin{eqnarray}
lim_{n\rightarrow\infty}n^{-1}Z\left(T_{n}\right)&=&a,\textrm{ c.s.}\\
lim_{t\rightarrow\infty}t^{-1}Z\left(t\right)&=&a/\mu,\textrm{ c.s.}
\end{eqnarray}
La segunda expresi\'on implica la primera. Conversamente, la primera implica la segunda si el proceso $Z\left(t\right)$ es creciente, o si $lim_{n\rightarrow\infty}n^{-1}M_{n}=0$ c.s.
\end{Teo}

\begin{Coro}
Si $N\left(t\right)$ es un proceso de renovaci\'on, y $\left(Z\left(T_{n}\right)-Z\left(T_{n-1}\right),M_{n}\right)$, para $n\geq1$, son variables aleatorias independientes e id\'enticamente distribuidas con media finita, entonces,
\begin{eqnarray}
lim_{t\rightarrow\infty}t^{-1}Z\left(t\right)\rightarrow\frac{\esp\left[Z\left(T_{1}\right)-Z\left(T_{0}\right)\right]}{\esp\left[T_{1}\right]},\textrm{ c.s. cuando  }t\rightarrow\infty.
\end{eqnarray}
\end{Coro}


%___________________________________________________________________________________________
%
%\subsection{Propiedades de los Procesos de Renovaci\'on}
%___________________________________________________________________________________________
%

Los tiempos $T_{n}$ est\'an relacionados con los conteos de $N\left(t\right)$ por

\begin{eqnarray*}
\left\{N\left(t\right)\geq n\right\}&=&\left\{T_{n}\leq t\right\}\\
T_{N\left(t\right)}\leq &t&<T_{N\left(t\right)+1},
\end{eqnarray*}

adem\'as $N\left(T_{n}\right)=n$, y 

\begin{eqnarray*}
N\left(t\right)=\max\left\{n:T_{n}\leq t\right\}=\min\left\{n:T_{n+1}>t\right\}
\end{eqnarray*}

Por propiedades de la convoluci\'on se sabe que

\begin{eqnarray*}
P\left\{T_{n}\leq t\right\}=F^{n\star}\left(t\right)
\end{eqnarray*}
que es la $n$-\'esima convoluci\'on de $F$. Entonces 

\begin{eqnarray*}
\left\{N\left(t\right)\geq n\right\}&=&\left\{T_{n}\leq t\right\}\\
P\left\{N\left(t\right)\leq n\right\}&=&1-F^{\left(n+1\right)\star}\left(t\right)
\end{eqnarray*}

Adem\'as usando el hecho de que $\esp\left[N\left(t\right)\right]=\sum_{n=1}^{\infty}P\left\{N\left(t\right)\geq n\right\}$
se tiene que

\begin{eqnarray*}
\esp\left[N\left(t\right)\right]=\sum_{n=1}^{\infty}F^{n\star}\left(t\right)
\end{eqnarray*}

\begin{Prop}
Para cada $t\geq0$, la funci\'on generadora de momentos $\esp\left[e^{\alpha N\left(t\right)}\right]$ existe para alguna $\alpha$ en una vecindad del 0, y de aqu\'i que $\esp\left[N\left(t\right)^{m}\right]<\infty$, para $m\geq1$.
\end{Prop}


\begin{Note}
Si el primer tiempo de renovaci\'on $\xi_{1}$ no tiene la misma distribuci\'on que el resto de las $\xi_{n}$, para $n\geq2$, a $N\left(t\right)$ se le llama Proceso de Renovaci\'on retardado, donde si $\xi$ tiene distribuci\'on $G$, entonces el tiempo $T_{n}$ de la $n$-\'esima renovaci\'on tiene distribuci\'on $G\star F^{\left(n-1\right)\star}\left(t\right)$
\end{Note}


\begin{Teo}
Para una constante $\mu\leq\infty$ ( o variable aleatoria), las siguientes expresiones son equivalentes:

\begin{eqnarray}
lim_{n\rightarrow\infty}n^{-1}T_{n}&=&\mu,\textrm{ c.s.}\\
lim_{t\rightarrow\infty}t^{-1}N\left(t\right)&=&1/\mu,\textrm{ c.s.}
\end{eqnarray}
\end{Teo}


Es decir, $T_{n}$ satisface la Ley Fuerte de los Grandes N\'umeros s\'i y s\'olo s\'i $N\left/t\right)$ la cumple.


\begin{Coro}[Ley Fuerte de los Grandes N\'umeros para Procesos de Renovaci\'on]
Si $N\left(t\right)$ es un proceso de renovaci\'on cuyos tiempos de inter-renovaci\'on tienen media $\mu\leq\infty$, entonces
\begin{eqnarray}
t^{-1}N\left(t\right)\rightarrow 1/\mu,\textrm{ c.s. cuando }t\rightarrow\infty.
\end{eqnarray}

\end{Coro}


Considerar el proceso estoc\'astico de valores reales $\left\{Z\left(t\right):t\geq0\right\}$ en el mismo espacio de probabilidad que $N\left(t\right)$

\begin{Def}
Para el proceso $\left\{Z\left(t\right):t\geq0\right\}$ se define la fluctuaci\'on m\'axima de $Z\left(t\right)$ en el intervalo $\left(T_{n-1},T_{n}\right]$:
\begin{eqnarray*}
M_{n}=\sup_{T_{n-1}<t\leq T_{n}}|Z\left(t\right)-Z\left(T_{n-1}\right)|
\end{eqnarray*}
\end{Def}

\begin{Teo}
Sup\'ongase que $n^{-1}T_{n}\rightarrow\mu$ c.s. cuando $n\rightarrow\infty$, donde $\mu\leq\infty$ es una constante o variable aleatoria. Sea $a$ una constante o variable aleatoria que puede ser infinita cuando $\mu$ es finita, y considere las expresiones l\'imite:
\begin{eqnarray}
lim_{n\rightarrow\infty}n^{-1}Z\left(T_{n}\right)&=&a,\textrm{ c.s.}\\
lim_{t\rightarrow\infty}t^{-1}Z\left(t\right)&=&a/\mu,\textrm{ c.s.}
\end{eqnarray}
La segunda expresi\'on implica la primera. Conversamente, la primera implica la segunda si el proceso $Z\left(t\right)$ es creciente, o si $lim_{n\rightarrow\infty}n^{-1}M_{n}=0$ c.s.
\end{Teo}

\begin{Coro}
Si $N\left(t\right)$ es un proceso de renovaci\'on, y $\left(Z\left(T_{n}\right)-Z\left(T_{n-1}\right),M_{n}\right)$, para $n\geq1$, son variables aleatorias independientes e id\'enticamente distribuidas con media finita, entonces,
\begin{eqnarray}
lim_{t\rightarrow\infty}t^{-1}Z\left(t\right)\rightarrow\frac{\esp\left[Z\left(T_{1}\right)-Z\left(T_{0}\right)\right]}{\esp\left[T_{1}\right]},\textrm{ c.s. cuando  }t\rightarrow\infty.
\end{eqnarray}
\end{Coro}

%___________________________________________________________________________________________
%
%\subsection{Propiedades de los Procesos de Renovaci\'on}
%___________________________________________________________________________________________
%

Los tiempos $T_{n}$ est\'an relacionados con los conteos de $N\left(t\right)$ por

\begin{eqnarray*}
\left\{N\left(t\right)\geq n\right\}&=&\left\{T_{n}\leq t\right\}\\
T_{N\left(t\right)}\leq &t&<T_{N\left(t\right)+1},
\end{eqnarray*}

adem\'as $N\left(T_{n}\right)=n$, y 

\begin{eqnarray*}
N\left(t\right)=\max\left\{n:T_{n}\leq t\right\}=\min\left\{n:T_{n+1}>t\right\}
\end{eqnarray*}

Por propiedades de la convoluci\'on se sabe que

\begin{eqnarray*}
P\left\{T_{n}\leq t\right\}=F^{n\star}\left(t\right)
\end{eqnarray*}
que es la $n$-\'esima convoluci\'on de $F$. Entonces 

\begin{eqnarray*}
\left\{N\left(t\right)\geq n\right\}&=&\left\{T_{n}\leq t\right\}\\
P\left\{N\left(t\right)\leq n\right\}&=&1-F^{\left(n+1\right)\star}\left(t\right)
\end{eqnarray*}

Adem\'as usando el hecho de que $\esp\left[N\left(t\right)\right]=\sum_{n=1}^{\infty}P\left\{N\left(t\right)\geq n\right\}$
se tiene que

\begin{eqnarray*}
\esp\left[N\left(t\right)\right]=\sum_{n=1}^{\infty}F^{n\star}\left(t\right)
\end{eqnarray*}

\begin{Prop}
Para cada $t\geq0$, la funci\'on generadora de momentos $\esp\left[e^{\alpha N\left(t\right)}\right]$ existe para alguna $\alpha$ en una vecindad del 0, y de aqu\'i que $\esp\left[N\left(t\right)^{m}\right]<\infty$, para $m\geq1$.
\end{Prop}


\begin{Note}
Si el primer tiempo de renovaci\'on $\xi_{1}$ no tiene la misma distribuci\'on que el resto de las $\xi_{n}$, para $n\geq2$, a $N\left(t\right)$ se le llama Proceso de Renovaci\'on retardado, donde si $\xi$ tiene distribuci\'on $G$, entonces el tiempo $T_{n}$ de la $n$-\'esima renovaci\'on tiene distribuci\'on $G\star F^{\left(n-1\right)\star}\left(t\right)$
\end{Note}


\begin{Teo}
Para una constante $\mu\leq\infty$ ( o variable aleatoria), las siguientes expresiones son equivalentes:

\begin{eqnarray}
lim_{n\rightarrow\infty}n^{-1}T_{n}&=&\mu,\textrm{ c.s.}\\
lim_{t\rightarrow\infty}t^{-1}N\left(t\right)&=&1/\mu,\textrm{ c.s.}
\end{eqnarray}
\end{Teo}


Es decir, $T_{n}$ satisface la Ley Fuerte de los Grandes N\'umeros s\'i y s\'olo s\'i $N\left/t\right)$ la cumple.


\begin{Coro}[Ley Fuerte de los Grandes N\'umeros para Procesos de Renovaci\'on]
Si $N\left(t\right)$ es un proceso de renovaci\'on cuyos tiempos de inter-renovaci\'on tienen media $\mu\leq\infty$, entonces
\begin{eqnarray}
t^{-1}N\left(t\right)\rightarrow 1/\mu,\textrm{ c.s. cuando }t\rightarrow\infty.
\end{eqnarray}

\end{Coro}


Considerar el proceso estoc\'astico de valores reales $\left\{Z\left(t\right):t\geq0\right\}$ en el mismo espacio de probabilidad que $N\left(t\right)$

\begin{Def}
Para el proceso $\left\{Z\left(t\right):t\geq0\right\}$ se define la fluctuaci\'on m\'axima de $Z\left(t\right)$ en el intervalo $\left(T_{n-1},T_{n}\right]$:
\begin{eqnarray*}
M_{n}=\sup_{T_{n-1}<t\leq T_{n}}|Z\left(t\right)-Z\left(T_{n-1}\right)|
\end{eqnarray*}
\end{Def}

\begin{Teo}
Sup\'ongase que $n^{-1}T_{n}\rightarrow\mu$ c.s. cuando $n\rightarrow\infty$, donde $\mu\leq\infty$ es una constante o variable aleatoria. Sea $a$ una constante o variable aleatoria que puede ser infinita cuando $\mu$ es finita, y considere las expresiones l\'imite:
\begin{eqnarray}
lim_{n\rightarrow\infty}n^{-1}Z\left(T_{n}\right)&=&a,\textrm{ c.s.}\\
lim_{t\rightarrow\infty}t^{-1}Z\left(t\right)&=&a/\mu,\textrm{ c.s.}
\end{eqnarray}
La segunda expresi\'on implica la primera. Conversamente, la primera implica la segunda si el proceso $Z\left(t\right)$ es creciente, o si $lim_{n\rightarrow\infty}n^{-1}M_{n}=0$ c.s.
\end{Teo}

\begin{Coro}
Si $N\left(t\right)$ es un proceso de renovaci\'on, y $\left(Z\left(T_{n}\right)-Z\left(T_{n-1}\right),M_{n}\right)$, para $n\geq1$, son variables aleatorias independientes e id\'enticamente distribuidas con media finita, entonces,
\begin{eqnarray}
lim_{t\rightarrow\infty}t^{-1}Z\left(t\right)\rightarrow\frac{\esp\left[Z\left(T_{1}\right)-Z\left(T_{0}\right)\right]}{\esp\left[T_{1}\right]},\textrm{ c.s. cuando  }t\rightarrow\infty.
\end{eqnarray}
\end{Coro}
%___________________________________________________________________________________________
%
%\subsection{Propiedades de los Procesos de Renovaci\'on}
%___________________________________________________________________________________________
%

Los tiempos $T_{n}$ est\'an relacionados con los conteos de $N\left(t\right)$ por

\begin{eqnarray*}
\left\{N\left(t\right)\geq n\right\}&=&\left\{T_{n}\leq t\right\}\\
T_{N\left(t\right)}\leq &t&<T_{N\left(t\right)+1},
\end{eqnarray*}

adem\'as $N\left(T_{n}\right)=n$, y 

\begin{eqnarray*}
N\left(t\right)=\max\left\{n:T_{n}\leq t\right\}=\min\left\{n:T_{n+1}>t\right\}
\end{eqnarray*}

Por propiedades de la convoluci\'on se sabe que

\begin{eqnarray*}
P\left\{T_{n}\leq t\right\}=F^{n\star}\left(t\right)
\end{eqnarray*}
que es la $n$-\'esima convoluci\'on de $F$. Entonces 

\begin{eqnarray*}
\left\{N\left(t\right)\geq n\right\}&=&\left\{T_{n}\leq t\right\}\\
P\left\{N\left(t\right)\leq n\right\}&=&1-F^{\left(n+1\right)\star}\left(t\right)
\end{eqnarray*}

Adem\'as usando el hecho de que $\esp\left[N\left(t\right)\right]=\sum_{n=1}^{\infty}P\left\{N\left(t\right)\geq n\right\}$
se tiene que

\begin{eqnarray*}
\esp\left[N\left(t\right)\right]=\sum_{n=1}^{\infty}F^{n\star}\left(t\right)
\end{eqnarray*}

\begin{Prop}
Para cada $t\geq0$, la funci\'on generadora de momentos $\esp\left[e^{\alpha N\left(t\right)}\right]$ existe para alguna $\alpha$ en una vecindad del 0, y de aqu\'i que $\esp\left[N\left(t\right)^{m}\right]<\infty$, para $m\geq1$.
\end{Prop}


\begin{Note}
Si el primer tiempo de renovaci\'on $\xi_{1}$ no tiene la misma distribuci\'on que el resto de las $\xi_{n}$, para $n\geq2$, a $N\left(t\right)$ se le llama Proceso de Renovaci\'on retardado, donde si $\xi$ tiene distribuci\'on $G$, entonces el tiempo $T_{n}$ de la $n$-\'esima renovaci\'on tiene distribuci\'on $G\star F^{\left(n-1\right)\star}\left(t\right)$
\end{Note}


\begin{Teo}
Para una constante $\mu\leq\infty$ ( o variable aleatoria), las siguientes expresiones son equivalentes:

\begin{eqnarray}
lim_{n\rightarrow\infty}n^{-1}T_{n}&=&\mu,\textrm{ c.s.}\\
lim_{t\rightarrow\infty}t^{-1}N\left(t\right)&=&1/\mu,\textrm{ c.s.}
\end{eqnarray}
\end{Teo}


Es decir, $T_{n}$ satisface la Ley Fuerte de los Grandes N\'umeros s\'i y s\'olo s\'i $N\left/t\right)$ la cumple.


\begin{Coro}[Ley Fuerte de los Grandes N\'umeros para Procesos de Renovaci\'on]
Si $N\left(t\right)$ es un proceso de renovaci\'on cuyos tiempos de inter-renovaci\'on tienen media $\mu\leq\infty$, entonces
\begin{eqnarray}
t^{-1}N\left(t\right)\rightarrow 1/\mu,\textrm{ c.s. cuando }t\rightarrow\infty.
\end{eqnarray}

\end{Coro}


Considerar el proceso estoc\'astico de valores reales $\left\{Z\left(t\right):t\geq0\right\}$ en el mismo espacio de probabilidad que $N\left(t\right)$

\begin{Def}
Para el proceso $\left\{Z\left(t\right):t\geq0\right\}$ se define la fluctuaci\'on m\'axima de $Z\left(t\right)$ en el intervalo $\left(T_{n-1},T_{n}\right]$:
\begin{eqnarray*}
M_{n}=\sup_{T_{n-1}<t\leq T_{n}}|Z\left(t\right)-Z\left(T_{n-1}\right)|
\end{eqnarray*}
\end{Def}

\begin{Teo}
Sup\'ongase que $n^{-1}T_{n}\rightarrow\mu$ c.s. cuando $n\rightarrow\infty$, donde $\mu\leq\infty$ es una constante o variable aleatoria. Sea $a$ una constante o variable aleatoria que puede ser infinita cuando $\mu$ es finita, y considere las expresiones l\'imite:
\begin{eqnarray}
lim_{n\rightarrow\infty}n^{-1}Z\left(T_{n}\right)&=&a,\textrm{ c.s.}\\
lim_{t\rightarrow\infty}t^{-1}Z\left(t\right)&=&a/\mu,\textrm{ c.s.}
\end{eqnarray}
La segunda expresi\'on implica la primera. Conversamente, la primera implica la segunda si el proceso $Z\left(t\right)$ es creciente, o si $lim_{n\rightarrow\infty}n^{-1}M_{n}=0$ c.s.
\end{Teo}

\begin{Coro}
Si $N\left(t\right)$ es un proceso de renovaci\'on, y $\left(Z\left(T_{n}\right)-Z\left(T_{n-1}\right),M_{n}\right)$, para $n\geq1$, son variables aleatorias independientes e id\'enticamente distribuidas con media finita, entonces,
\begin{eqnarray}
lim_{t\rightarrow\infty}t^{-1}Z\left(t\right)\rightarrow\frac{\esp\left[Z\left(T_{1}\right)-Z\left(T_{0}\right)\right]}{\esp\left[T_{1}\right]},\textrm{ c.s. cuando  }t\rightarrow\infty.
\end{eqnarray}
\end{Coro}


%___________________________________________________________________________________________
%
%\subsection{Funci\'on de Renovaci\'on}
%___________________________________________________________________________________________
%


\begin{Def}
Sea $h\left(t\right)$ funci\'on de valores reales en $\rea$ acotada en intervalos finitos e igual a cero para $t<0$ La ecuaci\'on de renovaci\'on para $h\left(t\right)$ y la distribuci\'on $F$ es

\begin{eqnarray}%\label{Ec.Renovacion}
H\left(t\right)=h\left(t\right)+\int_{\left[0,t\right]}H\left(t-s\right)dF\left(s\right)\textrm{,    }t\geq0,
\end{eqnarray}
donde $H\left(t\right)$ es una funci\'on de valores reales. Esto es $H=h+F\star H$. Decimos que $H\left(t\right)$ es soluci\'on de esta ecuaci\'on si satisface la ecuaci\'on, y es acotada en intervalos finitos e iguales a cero para $t<0$.
\end{Def}

\begin{Prop}
La funci\'on $U\star h\left(t\right)$ es la \'unica soluci\'on de la ecuaci\'on de renovaci\'on (\ref{Ec.Renovacion}).
\end{Prop}

\begin{Teo}[Teorema Renovaci\'on Elemental]
\begin{eqnarray*}
t^{-1}U\left(t\right)\rightarrow 1/\mu\textrm{,    cuando }t\rightarrow\infty.
\end{eqnarray*}
\end{Teo}

%___________________________________________________________________________________________
%
%\subsection{Funci\'on de Renovaci\'on}
%___________________________________________________________________________________________
%


Sup\'ongase que $N\left(t\right)$ es un proceso de renovaci\'on con distribuci\'on $F$ con media finita $\mu$.

\begin{Def}
La funci\'on de renovaci\'on asociada con la distribuci\'on $F$, del proceso $N\left(t\right)$, es
\begin{eqnarray*}
U\left(t\right)=\sum_{n=1}^{\infty}F^{n\star}\left(t\right),\textrm{   }t\geq0,
\end{eqnarray*}
donde $F^{0\star}\left(t\right)=\indora\left(t\geq0\right)$.
\end{Def}


\begin{Prop}
Sup\'ongase que la distribuci\'on de inter-renovaci\'on $F$ tiene densidad $f$. Entonces $U\left(t\right)$ tambi\'en tiene densidad, para $t>0$, y es $U^{'}\left(t\right)=\sum_{n=0}^{\infty}f^{n\star}\left(t\right)$. Adem\'as
\begin{eqnarray*}
\prob\left\{N\left(t\right)>N\left(t-\right)\right\}=0\textrm{,   }t\geq0.
\end{eqnarray*}
\end{Prop}

\begin{Def}
La Transformada de Laplace-Stieljes de $F$ est\'a dada por

\begin{eqnarray*}
\hat{F}\left(\alpha\right)=\int_{\rea_{+}}e^{-\alpha t}dF\left(t\right)\textrm{,  }\alpha\geq0.
\end{eqnarray*}
\end{Def}

Entonces

\begin{eqnarray*}
\hat{U}\left(\alpha\right)=\sum_{n=0}^{\infty}\hat{F^{n\star}}\left(\alpha\right)=\sum_{n=0}^{\infty}\hat{F}\left(\alpha\right)^{n}=\frac{1}{1-\hat{F}\left(\alpha\right)}.
\end{eqnarray*}


\begin{Prop}
La Transformada de Laplace $\hat{U}\left(\alpha\right)$ y $\hat{F}\left(\alpha\right)$ determina una a la otra de manera \'unica por la relaci\'on $\hat{U}\left(\alpha\right)=\frac{1}{1-\hat{F}\left(\alpha\right)}$.
\end{Prop}


\begin{Note}
Un proceso de renovaci\'on $N\left(t\right)$ cuyos tiempos de inter-renovaci\'on tienen media finita, es un proceso Poisson con tasa $\lambda$ si y s\'olo s\'i $\esp\left[U\left(t\right)\right]=\lambda t$, para $t\geq0$.
\end{Note}


\begin{Teo}
Sea $N\left(t\right)$ un proceso puntual simple con puntos de localizaci\'on $T_{n}$ tal que $\eta\left(t\right)=\esp\left[N\left(\right)\right]$ es finita para cada $t$. Entonces para cualquier funci\'on $f:\rea_{+}\rightarrow\rea$,
\begin{eqnarray*}
\esp\left[\sum_{n=1}^{N\left(\right)}f\left(T_{n}\right)\right]=\int_{\left(0,t\right]}f\left(s\right)d\eta\left(s\right)\textrm{,  }t\geq0,
\end{eqnarray*}
suponiendo que la integral exista. Adem\'as si $X_{1},X_{2},\ldots$ son variables aleatorias definidas en el mismo espacio de probabilidad que el proceso $N\left(t\right)$ tal que $\esp\left[X_{n}|T_{n}=s\right]=f\left(s\right)$, independiente de $n$. Entonces
\begin{eqnarray*}
\esp\left[\sum_{n=1}^{N\left(t\right)}X_{n}\right]=\int_{\left(0,t\right]}f\left(s\right)d\eta\left(s\right)\textrm{,  }t\geq0,
\end{eqnarray*} 
suponiendo que la integral exista. 
\end{Teo}

\begin{Coro}[Identidad de Wald para Renovaciones]
Para el proceso de renovaci\'on $N\left(t\right)$,
\begin{eqnarray*}
\esp\left[T_{N\left(t\right)+1}\right]=\mu\esp\left[N\left(t\right)+1\right]\textrm{,  }t\geq0,
\end{eqnarray*}  
\end{Coro}

%______________________________________________________________________
%\subsection{Procesos de Renovaci\'on}
%______________________________________________________________________

\begin{Def}%\label{Def.Tn}
Sean $0\leq T_{1}\leq T_{2}\leq \ldots$ son tiempos aleatorios infinitos en los cuales ocurren ciertos eventos. El n\'umero de tiempos $T_{n}$ en el intervalo $\left[0,t\right)$ es

\begin{eqnarray}
N\left(t\right)=\sum_{n=1}^{\infty}\indora\left(T_{n}\leq t\right),
\end{eqnarray}
para $t\geq0$.
\end{Def}

Si se consideran los puntos $T_{n}$ como elementos de $\rea_{+}$, y $N\left(t\right)$ es el n\'umero de puntos en $\rea$. El proceso denotado por $\left\{N\left(t\right):t\geq0\right\}$, denotado por $N\left(t\right)$, es un proceso puntual en $\rea_{+}$. Los $T_{n}$ son los tiempos de ocurrencia, el proceso puntual $N\left(t\right)$ es simple si su n\'umero de ocurrencias son distintas: $0<T_{1}<T_{2}<\ldots$ casi seguramente.

\begin{Def}
Un proceso puntual $N\left(t\right)$ es un proceso de renovaci\'on si los tiempos de interocurrencia $\xi_{n}=T_{n}-T_{n-1}$, para $n\geq1$, son independientes e identicamente distribuidos con distribuci\'on $F$, donde $F\left(0\right)=0$ y $T_{0}=0$. Los $T_{n}$ son llamados tiempos de renovaci\'on, referente a la independencia o renovaci\'on de la informaci\'on estoc\'astica en estos tiempos. Los $\xi_{n}$ son los tiempos de inter-renovaci\'on, y $N\left(t\right)$ es el n\'umero de renovaciones en el intervalo $\left[0,t\right)$
\end{Def}


\begin{Note}
Para definir un proceso de renovaci\'on para cualquier contexto, solamente hay que especificar una distribuci\'on $F$, con $F\left(0\right)=0$, para los tiempos de inter-renovaci\'on. La funci\'on $F$ en turno degune las otra variables aleatorias. De manera formal, existe un espacio de probabilidad y una sucesi\'on de variables aleatorias $\xi_{1},\xi_{2},\ldots$ definidas en este con distribuci\'on $F$. Entonces las otras cantidades son $T_{n}=\sum_{k=1}^{n}\xi_{k}$ y $N\left(t\right)=\sum_{n=1}^{\infty}\indora\left(T_{n}\leq t\right)$, donde $T_{n}\rightarrow\infty$ casi seguramente por la Ley Fuerte de los Grandes Números.
\end{Note}

%___________________________________________________________________________________________
%
%\subsection{Renewal and Regenerative Processes: Serfozo\cite{Serfozo}}
%___________________________________________________________________________________________
%
\begin{Def}%\label{Def.Tn}
Sean $0\leq T_{1}\leq T_{2}\leq \ldots$ son tiempos aleatorios infinitos en los cuales ocurren ciertos eventos. El n\'umero de tiempos $T_{n}$ en el intervalo $\left[0,t\right)$ es

\begin{eqnarray}
N\left(t\right)=\sum_{n=1}^{\infty}\indora\left(T_{n}\leq t\right),
\end{eqnarray}
para $t\geq0$.
\end{Def}

Si se consideran los puntos $T_{n}$ como elementos de $\rea_{+}$, y $N\left(t\right)$ es el n\'umero de puntos en $\rea$. El proceso denotado por $\left\{N\left(t\right):t\geq0\right\}$, denotado por $N\left(t\right)$, es un proceso puntual en $\rea_{+}$. Los $T_{n}$ son los tiempos de ocurrencia, el proceso puntual $N\left(t\right)$ es simple si su n\'umero de ocurrencias son distintas: $0<T_{1}<T_{2}<\ldots$ casi seguramente.

\begin{Def}
Un proceso puntual $N\left(t\right)$ es un proceso de renovaci\'on si los tiempos de interocurrencia $\xi_{n}=T_{n}-T_{n-1}$, para $n\geq1$, son independientes e identicamente distribuidos con distribuci\'on $F$, donde $F\left(0\right)=0$ y $T_{0}=0$. Los $T_{n}$ son llamados tiempos de renovaci\'on, referente a la independencia o renovaci\'on de la informaci\'on estoc\'astica en estos tiempos. Los $\xi_{n}$ son los tiempos de inter-renovaci\'on, y $N\left(t\right)$ es el n\'umero de renovaciones en el intervalo $\left[0,t\right)$
\end{Def}


\begin{Note}
Para definir un proceso de renovaci\'on para cualquier contexto, solamente hay que especificar una distribuci\'on $F$, con $F\left(0\right)=0$, para los tiempos de inter-renovaci\'on. La funci\'on $F$ en turno degune las otra variables aleatorias. De manera formal, existe un espacio de probabilidad y una sucesi\'on de variables aleatorias $\xi_{1},\xi_{2},\ldots$ definidas en este con distribuci\'on $F$. Entonces las otras cantidades son $T_{n}=\sum_{k=1}^{n}\xi_{k}$ y $N\left(t\right)=\sum_{n=1}^{\infty}\indora\left(T_{n}\leq t\right)$, donde $T_{n}\rightarrow\infty$ casi seguramente por la Ley Fuerte de los Grandes N\'umeros.
\end{Note}







Los tiempos $T_{n}$ est\'an relacionados con los conteos de $N\left(t\right)$ por

\begin{eqnarray*}
\left\{N\left(t\right)\geq n\right\}&=&\left\{T_{n}\leq t\right\}\\
T_{N\left(t\right)}\leq &t&<T_{N\left(t\right)+1},
\end{eqnarray*}

adem\'as $N\left(T_{n}\right)=n$, y 

\begin{eqnarray*}
N\left(t\right)=\max\left\{n:T_{n}\leq t\right\}=\min\left\{n:T_{n+1}>t\right\}
\end{eqnarray*}

Por propiedades de la convoluci\'on se sabe que

\begin{eqnarray*}
P\left\{T_{n}\leq t\right\}=F^{n\star}\left(t\right)
\end{eqnarray*}
que es la $n$-\'esima convoluci\'on de $F$. Entonces 

\begin{eqnarray*}
\left\{N\left(t\right)\geq n\right\}&=&\left\{T_{n}\leq t\right\}\\
P\left\{N\left(t\right)\leq n\right\}&=&1-F^{\left(n+1\right)\star}\left(t\right)
\end{eqnarray*}

Adem\'as usando el hecho de que $\esp\left[N\left(t\right)\right]=\sum_{n=1}^{\infty}P\left\{N\left(t\right)\geq n\right\}$
se tiene que

\begin{eqnarray*}
\esp\left[N\left(t\right)\right]=\sum_{n=1}^{\infty}F^{n\star}\left(t\right)
\end{eqnarray*}

\begin{Prop}
Para cada $t\geq0$, la funci\'on generadora de momentos $\esp\left[e^{\alpha N\left(t\right)}\right]$ existe para alguna $\alpha$ en una vecindad del 0, y de aqu\'i que $\esp\left[N\left(t\right)^{m}\right]<\infty$, para $m\geq1$.
\end{Prop}

\begin{Ejem}[\textbf{Proceso Poisson}]

Suponga que se tienen tiempos de inter-renovaci\'on \textit{i.i.d.} del proceso de renovaci\'on $N\left(t\right)$ tienen distribuci\'on exponencial $F\left(t\right)=q-e^{-\lambda t}$ con tasa $\lambda$. Entonces $N\left(t\right)$ es un proceso Poisson con tasa $\lambda$.

\end{Ejem}


\begin{Note}
Si el primer tiempo de renovaci\'on $\xi_{1}$ no tiene la misma distribuci\'on que el resto de las $\xi_{n}$, para $n\geq2$, a $N\left(t\right)$ se le llama Proceso de Renovaci\'on retardado, donde si $\xi$ tiene distribuci\'on $G$, entonces el tiempo $T_{n}$ de la $n$-\'esima renovaci\'on tiene distribuci\'on $G\star F^{\left(n-1\right)\star}\left(t\right)$
\end{Note}


\begin{Teo}
Para una constante $\mu\leq\infty$ ( o variable aleatoria), las siguientes expresiones son equivalentes:

\begin{eqnarray}
lim_{n\rightarrow\infty}n^{-1}T_{n}&=&\mu,\textrm{ c.s.}\\
lim_{t\rightarrow\infty}t^{-1}N\left(t\right)&=&1/\mu,\textrm{ c.s.}
\end{eqnarray}
\end{Teo}


Es decir, $T_{n}$ satisface la Ley Fuerte de los Grandes N\'umeros s\'i y s\'olo s\'i $N\left/t\right)$ la cumple.


\begin{Coro}[Ley Fuerte de los Grandes N\'umeros para Procesos de Renovaci\'on]
Si $N\left(t\right)$ es un proceso de renovaci\'on cuyos tiempos de inter-renovaci\'on tienen media $\mu\leq\infty$, entonces
\begin{eqnarray}
t^{-1}N\left(t\right)\rightarrow 1/\mu,\textrm{ c.s. cuando }t\rightarrow\infty.
\end{eqnarray}

\end{Coro}


Considerar el proceso estoc\'astico de valores reales $\left\{Z\left(t\right):t\geq0\right\}$ en el mismo espacio de probabilidad que $N\left(t\right)$

\begin{Def}
Para el proceso $\left\{Z\left(t\right):t\geq0\right\}$ se define la fluctuaci\'on m\'axima de $Z\left(t\right)$ en el intervalo $\left(T_{n-1},T_{n}\right]$:
\begin{eqnarray*}
M_{n}=\sup_{T_{n-1}<t\leq T_{n}}|Z\left(t\right)-Z\left(T_{n-1}\right)|
\end{eqnarray*}
\end{Def}

\begin{Teo}
Sup\'ongase que $n^{-1}T_{n}\rightarrow\mu$ c.s. cuando $n\rightarrow\infty$, donde $\mu\leq\infty$ es una constante o variable aleatoria. Sea $a$ una constante o variable aleatoria que puede ser infinita cuando $\mu$ es finita, y considere las expresiones l\'imite:
\begin{eqnarray}
lim_{n\rightarrow\infty}n^{-1}Z\left(T_{n}\right)&=&a,\textrm{ c.s.}\\
lim_{t\rightarrow\infty}t^{-1}Z\left(t\right)&=&a/\mu,\textrm{ c.s.}
\end{eqnarray}
La segunda expresi\'on implica la primera. Conversamente, la primera implica la segunda si el proceso $Z\left(t\right)$ es creciente, o si $lim_{n\rightarrow\infty}n^{-1}M_{n}=0$ c.s.
\end{Teo}

\begin{Coro}
Si $N\left(t\right)$ es un proceso de renovaci\'on, y $\left(Z\left(T_{n}\right)-Z\left(T_{n-1}\right),M_{n}\right)$, para $n\geq1$, son variables aleatorias independientes e id\'enticamente distribuidas con media finita, entonces,
\begin{eqnarray}
lim_{t\rightarrow\infty}t^{-1}Z\left(t\right)\rightarrow\frac{\esp\left[Z\left(T_{1}\right)-Z\left(T_{0}\right)\right]}{\esp\left[T_{1}\right]},\textrm{ c.s. cuando  }t\rightarrow\infty.
\end{eqnarray}
\end{Coro}


Sup\'ongase que $N\left(t\right)$ es un proceso de renovaci\'on con distribuci\'on $F$ con media finita $\mu$.

\begin{Def}
La funci\'on de renovaci\'on asociada con la distribuci\'on $F$, del proceso $N\left(t\right)$, es
\begin{eqnarray*}
U\left(t\right)=\sum_{n=1}^{\infty}F^{n\star}\left(t\right),\textrm{   }t\geq0,
\end{eqnarray*}
donde $F^{0\star}\left(t\right)=\indora\left(t\geq0\right)$.
\end{Def}


\begin{Prop}
Sup\'ongase que la distribuci\'on de inter-renovaci\'on $F$ tiene densidad $f$. Entonces $U\left(t\right)$ tambi\'en tiene densidad, para $t>0$, y es $U^{'}\left(t\right)=\sum_{n=0}^{\infty}f^{n\star}\left(t\right)$. Adem\'as
\begin{eqnarray*}
\prob\left\{N\left(t\right)>N\left(t-\right)\right\}=0\textrm{,   }t\geq0.
\end{eqnarray*}
\end{Prop}

\begin{Def}
La Transformada de Laplace-Stieljes de $F$ est\'a dada por

\begin{eqnarray*}
\hat{F}\left(\alpha\right)=\int_{\rea_{+}}e^{-\alpha t}dF\left(t\right)\textrm{,  }\alpha\geq0.
\end{eqnarray*}
\end{Def}

Entonces

\begin{eqnarray*}
\hat{U}\left(\alpha\right)=\sum_{n=0}^{\infty}\hat{F^{n\star}}\left(\alpha\right)=\sum_{n=0}^{\infty}\hat{F}\left(\alpha\right)^{n}=\frac{1}{1-\hat{F}\left(\alpha\right)}.
\end{eqnarray*}


\begin{Prop}
La Transformada de Laplace $\hat{U}\left(\alpha\right)$ y $\hat{F}\left(\alpha\right)$ determina una a la otra de manera \'unica por la relaci\'on $\hat{U}\left(\alpha\right)=\frac{1}{1-\hat{F}\left(\alpha\right)}$.
\end{Prop}


\begin{Note}
Un proceso de renovaci\'on $N\left(t\right)$ cuyos tiempos de inter-renovaci\'on tienen media finita, es un proceso Poisson con tasa $\lambda$ si y s\'olo s\'i $\esp\left[U\left(t\right)\right]=\lambda t$, para $t\geq0$.
\end{Note}


\begin{Teo}
Sea $N\left(t\right)$ un proceso puntual simple con puntos de localizaci\'on $T_{n}$ tal que $\eta\left(t\right)=\esp\left[N\left(\right)\right]$ es finita para cada $t$. Entonces para cualquier funci\'on $f:\rea_{+}\rightarrow\rea$,
\begin{eqnarray*}
\esp\left[\sum_{n=1}^{N\left(\right)}f\left(T_{n}\right)\right]=\int_{\left(0,t\right]}f\left(s\right)d\eta\left(s\right)\textrm{,  }t\geq0,
\end{eqnarray*}
suponiendo que la integral exista. Adem\'as si $X_{1},X_{2},\ldots$ son variables aleatorias definidas en el mismo espacio de probabilidad que el proceso $N\left(t\right)$ tal que $\esp\left[X_{n}|T_{n}=s\right]=f\left(s\right)$, independiente de $n$. Entonces
\begin{eqnarray*}
\esp\left[\sum_{n=1}^{N\left(t\right)}X_{n}\right]=\int_{\left(0,t\right]}f\left(s\right)d\eta\left(s\right)\textrm{,  }t\geq0,
\end{eqnarray*} 
suponiendo que la integral exista. 
\end{Teo}

\begin{Coro}[Identidad de Wald para Renovaciones]
Para el proceso de renovaci\'on $N\left(t\right)$,
\begin{eqnarray*}
\esp\left[T_{N\left(t\right)+1}\right]=\mu\esp\left[N\left(t\right)+1\right]\textrm{,  }t\geq0,
\end{eqnarray*}  
\end{Coro}


\begin{Def}
Sea $h\left(t\right)$ funci\'on de valores reales en $\rea$ acotada en intervalos finitos e igual a cero para $t<0$ La ecuaci\'on de renovaci\'on para $h\left(t\right)$ y la distribuci\'on $F$ es

\begin{eqnarray}%\label{Ec.Renovacion}
H\left(t\right)=h\left(t\right)+\int_{\left[0,t\right]}H\left(t-s\right)dF\left(s\right)\textrm{,    }t\geq0,
\end{eqnarray}
donde $H\left(t\right)$ es una funci\'on de valores reales. Esto es $H=h+F\star H$. Decimos que $H\left(t\right)$ es soluci\'on de esta ecuaci\'on si satisface la ecuaci\'on, y es acotada en intervalos finitos e iguales a cero para $t<0$.
\end{Def}

\begin{Prop}
La funci\'on $U\star h\left(t\right)$ es la \'unica soluci\'on de la ecuaci\'on de renovaci\'on (\ref{Ec.Renovacion}).
\end{Prop}

\begin{Teo}[Teorema Renovaci\'on Elemental]
\begin{eqnarray*}
t^{-1}U\left(t\right)\rightarrow 1/\mu\textrm{,    cuando }t\rightarrow\infty.
\end{eqnarray*}
\end{Teo}



Sup\'ongase que $N\left(t\right)$ es un proceso de renovaci\'on con distribuci\'on $F$ con media finita $\mu$.

\begin{Def}
La funci\'on de renovaci\'on asociada con la distribuci\'on $F$, del proceso $N\left(t\right)$, es
\begin{eqnarray*}
U\left(t\right)=\sum_{n=1}^{\infty}F^{n\star}\left(t\right),\textrm{   }t\geq0,
\end{eqnarray*}
donde $F^{0\star}\left(t\right)=\indora\left(t\geq0\right)$.
\end{Def}


\begin{Prop}
Sup\'ongase que la distribuci\'on de inter-renovaci\'on $F$ tiene densidad $f$. Entonces $U\left(t\right)$ tambi\'en tiene densidad, para $t>0$, y es $U^{'}\left(t\right)=\sum_{n=0}^{\infty}f^{n\star}\left(t\right)$. Adem\'as
\begin{eqnarray*}
\prob\left\{N\left(t\right)>N\left(t-\right)\right\}=0\textrm{,   }t\geq0.
\end{eqnarray*}
\end{Prop}

\begin{Def}
La Transformada de Laplace-Stieljes de $F$ est\'a dada por

\begin{eqnarray*}
\hat{F}\left(\alpha\right)=\int_{\rea_{+}}e^{-\alpha t}dF\left(t\right)\textrm{,  }\alpha\geq0.
\end{eqnarray*}
\end{Def}

Entonces

\begin{eqnarray*}
\hat{U}\left(\alpha\right)=\sum_{n=0}^{\infty}\hat{F^{n\star}}\left(\alpha\right)=\sum_{n=0}^{\infty}\hat{F}\left(\alpha\right)^{n}=\frac{1}{1-\hat{F}\left(\alpha\right)}.
\end{eqnarray*}


\begin{Prop}
La Transformada de Laplace $\hat{U}\left(\alpha\right)$ y $\hat{F}\left(\alpha\right)$ determina una a la otra de manera \'unica por la relaci\'on $\hat{U}\left(\alpha\right)=\frac{1}{1-\hat{F}\left(\alpha\right)}$.
\end{Prop}


\begin{Note}
Un proceso de renovaci\'on $N\left(t\right)$ cuyos tiempos de inter-renovaci\'on tienen media finita, es un proceso Poisson con tasa $\lambda$ si y s\'olo s\'i $\esp\left[U\left(t\right)\right]=\lambda t$, para $t\geq0$.
\end{Note}


\begin{Teo}
Sea $N\left(t\right)$ un proceso puntual simple con puntos de localizaci\'on $T_{n}$ tal que $\eta\left(t\right)=\esp\left[N\left(\right)\right]$ es finita para cada $t$. Entonces para cualquier funci\'on $f:\rea_{+}\rightarrow\rea$,
\begin{eqnarray*}
\esp\left[\sum_{n=1}^{N\left(\right)}f\left(T_{n}\right)\right]=\int_{\left(0,t\right]}f\left(s\right)d\eta\left(s\right)\textrm{,  }t\geq0,
\end{eqnarray*}
suponiendo que la integral exista. Adem\'as si $X_{1},X_{2},\ldots$ son variables aleatorias definidas en el mismo espacio de probabilidad que el proceso $N\left(t\right)$ tal que $\esp\left[X_{n}|T_{n}=s\right]=f\left(s\right)$, independiente de $n$. Entonces
\begin{eqnarray*}
\esp\left[\sum_{n=1}^{N\left(t\right)}X_{n}\right]=\int_{\left(0,t\right]}f\left(s\right)d\eta\left(s\right)\textrm{,  }t\geq0,
\end{eqnarray*} 
suponiendo que la integral exista. 
\end{Teo}

\begin{Coro}[Identidad de Wald para Renovaciones]
Para el proceso de renovaci\'on $N\left(t\right)$,
\begin{eqnarray*}
\esp\left[T_{N\left(t\right)+1}\right]=\mu\esp\left[N\left(t\right)+1\right]\textrm{,  }t\geq0,
\end{eqnarray*}  
\end{Coro}


\begin{Def}
Sea $h\left(t\right)$ funci\'on de valores reales en $\rea$ acotada en intervalos finitos e igual a cero para $t<0$ La ecuaci\'on de renovaci\'on para $h\left(t\right)$ y la distribuci\'on $F$ es

\begin{eqnarray}%\label{Ec.Renovacion}
H\left(t\right)=h\left(t\right)+\int_{\left[0,t\right]}H\left(t-s\right)dF\left(s\right)\textrm{,    }t\geq0,
\end{eqnarray}
donde $H\left(t\right)$ es una funci\'on de valores reales. Esto es $H=h+F\star H$. Decimos que $H\left(t\right)$ es soluci\'on de esta ecuaci\'on si satisface la ecuaci\'on, y es acotada en intervalos finitos e iguales a cero para $t<0$.
\end{Def}

\begin{Prop}
La funci\'on $U\star h\left(t\right)$ es la \'unica soluci\'on de la ecuaci\'on de renovaci\'on (\ref{Ec.Renovacion}).
\end{Prop}

\begin{Teo}[Teorema Renovaci\'on Elemental]
\begin{eqnarray*}
t^{-1}U\left(t\right)\rightarrow 1/\mu\textrm{,    cuando }t\rightarrow\infty.
\end{eqnarray*}
\end{Teo}


\begin{Note} Una funci\'on $h:\rea_{+}\rightarrow\rea$ es Directamente Riemann Integrable en los siguientes casos:
\begin{itemize}
\item[a)] $h\left(t\right)\geq0$ es decreciente y Riemann Integrable.
\item[b)] $h$ es continua excepto posiblemente en un conjunto de Lebesgue de medida 0, y $|h\left(t\right)|\leq b\left(t\right)$, donde $b$ es DRI.
\end{itemize}
\end{Note}

\begin{Teo}[Teorema Principal de Renovaci\'on]
Si $F$ es no aritm\'etica y $h\left(t\right)$ es Directamente Riemann Integrable (DRI), entonces

\begin{eqnarray*}
lim_{t\rightarrow\infty}U\star h=\frac{1}{\mu}\int_{\rea_{+}}h\left(s\right)ds.
\end{eqnarray*}
\end{Teo}

\begin{Prop}
Cualquier funci\'on $H\left(t\right)$ acotada en intervalos finitos y que es 0 para $t<0$ puede expresarse como
\begin{eqnarray*}
H\left(t\right)=U\star h\left(t\right)\textrm{,  donde }h\left(t\right)=H\left(t\right)-F\star H\left(t\right)
\end{eqnarray*}
\end{Prop}

\begin{Def}
Un proceso estoc\'astico $X\left(t\right)$ es crudamente regenerativo en un tiempo aleatorio positivo $T$ si
\begin{eqnarray*}
\esp\left[X\left(T+t\right)|T\right]=\esp\left[X\left(t\right)\right]\textrm{, para }t\geq0,\end{eqnarray*}
y con las esperanzas anteriores finitas.
\end{Def}

\begin{Prop}
Sup\'ongase que $X\left(t\right)$ es un proceso crudamente regenerativo en $T$, que tiene distribuci\'on $F$. Si $\esp\left[X\left(t\right)\right]$ es acotado en intervalos finitos, entonces
\begin{eqnarray*}
\esp\left[X\left(t\right)\right]=U\star h\left(t\right)\textrm{,  donde }h\left(t\right)=\esp\left[X\left(t\right)\indora\left(T>t\right)\right].
\end{eqnarray*}
\end{Prop}

\begin{Teo}[Regeneraci\'on Cruda]
Sup\'ongase que $X\left(t\right)$ es un proceso con valores positivo crudamente regenerativo en $T$, y def\'inase $M=\sup\left\{|X\left(t\right)|:t\leq T\right\}$. Si $T$ es no aritm\'etico y $M$ y $MT$ tienen media finita, entonces
\begin{eqnarray*}
lim_{t\rightarrow\infty}\esp\left[X\left(t\right)\right]=\frac{1}{\mu}\int_{\rea_{+}}h\left(s\right)ds,
\end{eqnarray*}
donde $h\left(t\right)=\esp\left[X\left(t\right)\indora\left(T>t\right)\right]$.
\end{Teo}


\begin{Note} Una funci\'on $h:\rea_{+}\rightarrow\rea$ es Directamente Riemann Integrable en los siguientes casos:
\begin{itemize}
\item[a)] $h\left(t\right)\geq0$ es decreciente y Riemann Integrable.
\item[b)] $h$ es continua excepto posiblemente en un conjunto de Lebesgue de medida 0, y $|h\left(t\right)|\leq b\left(t\right)$, donde $b$ es DRI.
\end{itemize}
\end{Note}

\begin{Teo}[Teorema Principal de Renovaci\'on]
Si $F$ es no aritm\'etica y $h\left(t\right)$ es Directamente Riemann Integrable (DRI), entonces

\begin{eqnarray*}
lim_{t\rightarrow\infty}U\star h=\frac{1}{\mu}\int_{\rea_{+}}h\left(s\right)ds.
\end{eqnarray*}
\end{Teo}

\begin{Prop}
Cualquier funci\'on $H\left(t\right)$ acotada en intervalos finitos y que es 0 para $t<0$ puede expresarse como
\begin{eqnarray*}
H\left(t\right)=U\star h\left(t\right)\textrm{,  donde }h\left(t\right)=H\left(t\right)-F\star H\left(t\right)
\end{eqnarray*}
\end{Prop}

\begin{Def}
Un proceso estoc\'astico $X\left(t\right)$ es crudamente regenerativo en un tiempo aleatorio positivo $T$ si
\begin{eqnarray*}
\esp\left[X\left(T+t\right)|T\right]=\esp\left[X\left(t\right)\right]\textrm{, para }t\geq0,\end{eqnarray*}
y con las esperanzas anteriores finitas.
\end{Def}

\begin{Prop}
Sup\'ongase que $X\left(t\right)$ es un proceso crudamente regenerativo en $T$, que tiene distribuci\'on $F$. Si $\esp\left[X\left(t\right)\right]$ es acotado en intervalos finitos, entonces
\begin{eqnarray*}
\esp\left[X\left(t\right)\right]=U\star h\left(t\right)\textrm{,  donde }h\left(t\right)=\esp\left[X\left(t\right)\indora\left(T>t\right)\right].
\end{eqnarray*}
\end{Prop}

\begin{Teo}[Regeneraci\'on Cruda]
Sup\'ongase que $X\left(t\right)$ es un proceso con valores positivo crudamente regenerativo en $T$, y def\'inase $M=\sup\left\{|X\left(t\right)|:t\leq T\right\}$. Si $T$ es no aritm\'etico y $M$ y $MT$ tienen media finita, entonces
\begin{eqnarray*}
lim_{t\rightarrow\infty}\esp\left[X\left(t\right)\right]=\frac{1}{\mu}\int_{\rea_{+}}h\left(s\right)ds,
\end{eqnarray*}
donde $h\left(t\right)=\esp\left[X\left(t\right)\indora\left(T>t\right)\right]$.
\end{Teo}

\begin{Def}
Para el proceso $\left\{\left(N\left(t\right),X\left(t\right)\right):t\geq0\right\}$, sus trayectoria muestrales en el intervalo de tiempo $\left[T_{n-1},T_{n}\right)$ est\'an descritas por
\begin{eqnarray*}
\zeta_{n}=\left(\xi_{n},\left\{X\left(T_{n-1}+t\right):0\leq t<\xi_{n}\right\}\right)
\end{eqnarray*}
Este $\zeta_{n}$ es el $n$-\'esimo segmento del proceso. El proceso es regenerativo sobre los tiempos $T_{n}$ si sus segmentos $\zeta_{n}$ son independientes e id\'enticamennte distribuidos.
\end{Def}


\begin{Note}
Si $\tilde{X}\left(t\right)$ con espacio de estados $\tilde{S}$ es regenerativo sobre $T_{n}$, entonces $X\left(t\right)=f\left(\tilde{X}\left(t\right)\right)$ tambi\'en es regenerativo sobre $T_{n}$, para cualquier funci\'on $f:\tilde{S}\rightarrow S$.
\end{Note}

\begin{Note}
Los procesos regenerativos son crudamente regenerativos, pero no al rev\'es.
\end{Note}


\begin{Note}
Un proceso estoc\'astico a tiempo continuo o discreto es regenerativo si existe un proceso de renovaci\'on  tal que los segmentos del proceso entre tiempos de renovaci\'on sucesivos son i.i.d., es decir, para $\left\{X\left(t\right):t\geq0\right\}$ proceso estoc\'astico a tiempo continuo con espacio de estados $S$, espacio m\'etrico.
\end{Note}

Para $\left\{X\left(t\right):t\geq0\right\}$ Proceso Estoc\'astico a tiempo continuo con estado de espacios $S$, que es un espacio m\'etrico, con trayectorias continuas por la derecha y con l\'imites por la izquierda c.s. Sea $N\left(t\right)$ un proceso de renovaci\'on en $\rea_{+}$ definido en el mismo espacio de probabilidad que $X\left(t\right)$, con tiempos de renovaci\'on $T$ y tiempos de inter-renovaci\'on $\xi_{n}=T_{n}-T_{n-1}$, con misma distribuci\'on $F$ de media finita $\mu$.



\begin{Def}
Para el proceso $\left\{\left(N\left(t\right),X\left(t\right)\right):t\geq0\right\}$, sus trayectoria muestrales en el intervalo de tiempo $\left[T_{n-1},T_{n}\right)$ est\'an descritas por
\begin{eqnarray*}
\zeta_{n}=\left(\xi_{n},\left\{X\left(T_{n-1}+t\right):0\leq t<\xi_{n}\right\}\right)
\end{eqnarray*}
Este $\zeta_{n}$ es el $n$-\'esimo segmento del proceso. El proceso es regenerativo sobre los tiempos $T_{n}$ si sus segmentos $\zeta_{n}$ son independientes e id\'enticamennte distribuidos.
\end{Def}

\begin{Note}
Un proceso regenerativo con media de la longitud de ciclo finita es llamado positivo recurrente.
\end{Note}

\begin{Teo}[Procesos Regenerativos]
Suponga que el proceso
\end{Teo}


\begin{Def}[Renewal Process Trinity]
Para un proceso de renovaci\'on $N\left(t\right)$, los siguientes procesos proveen de informaci\'on sobre los tiempos de renovaci\'on.
\begin{itemize}
\item $A\left(t\right)=t-T_{N\left(t\right)}$, el tiempo de recurrencia hacia atr\'as al tiempo $t$, que es el tiempo desde la \'ultima renovaci\'on para $t$.

\item $B\left(t\right)=T_{N\left(t\right)+1}-t$, el tiempo de recurrencia hacia adelante al tiempo $t$, residual del tiempo de renovaci\'on, que es el tiempo para la pr\'oxima renovaci\'on despu\'es de $t$.

\item $L\left(t\right)=\xi_{N\left(t\right)+1}=A\left(t\right)+B\left(t\right)$, la longitud del intervalo de renovaci\'on que contiene a $t$.
\end{itemize}
\end{Def}

\begin{Note}
El proceso tridimensional $\left(A\left(t\right),B\left(t\right),L\left(t\right)\right)$ es regenerativo sobre $T_{n}$, y por ende cada proceso lo es. Cada proceso $A\left(t\right)$ y $B\left(t\right)$ son procesos de MArkov a tiempo continuo con trayectorias continuas por partes en el espacio de estados $\rea_{+}$. Una expresi\'on conveniente para su distribuci\'on conjunta es, para $0\leq x<t,y\geq0$
\begin{equation}\label{NoRenovacion}
P\left\{A\left(t\right)>x,B\left(t\right)>y\right\}=
P\left\{N\left(t+y\right)-N\left((t-x)\right)=0\right\}
\end{equation}
\end{Note}

\begin{Ejem}[Tiempos de recurrencia Poisson]
Si $N\left(t\right)$ es un proceso Poisson con tasa $\lambda$, entonces de la expresi\'on (\ref{NoRenovacion}) se tiene que

\begin{eqnarray*}
\begin{array}{lc}
P\left\{A\left(t\right)>x,B\left(t\right)>y\right\}=e^{-\lambda\left(x+y\right)},&0\leq x<t,y\geq0,
\end{array}
\end{eqnarray*}
que es la probabilidad Poisson de no renovaciones en un intervalo de longitud $x+y$.

\end{Ejem}

\begin{Note}
Una cadena de Markov erg\'odica tiene la propiedad de ser estacionaria si la distribuci\'on de su estado al tiempo $0$ es su distribuci\'on estacionaria.
\end{Note}


\begin{Def}
Un proceso estoc\'astico a tiempo continuo $\left\{X\left(t\right):t\geq0\right\}$ en un espacio general es estacionario si sus distribuciones finito dimensionales son invariantes bajo cualquier  traslado: para cada $0\leq s_{1}<s_{2}<\cdots<s_{k}$ y $t\geq0$,
\begin{eqnarray*}
\left(X\left(s_{1}+t\right),\ldots,X\left(s_{k}+t\right)\right)=_{d}\left(X\left(s_{1}\right),\ldots,X\left(s_{k}\right)\right).
\end{eqnarray*}
\end{Def}

\begin{Note}
Un proceso de Markov es estacionario si $X\left(t\right)=_{d}X\left(0\right)$, $t\geq0$.
\end{Note}

Considerese el proceso $N\left(t\right)=\sum_{n}\indora\left(\tau_{n}\leq t\right)$ en $\rea_{+}$, con puntos $0<\tau_{1}<\tau_{2}<\cdots$.

\begin{Prop}
Si $N$ es un proceso puntual estacionario y $\esp\left[N\left(1\right)\right]<\infty$, entonces $\esp\left[N\left(t\right)\right]=t\esp\left[N\left(1\right)\right]$, $t\geq0$

\end{Prop}

\begin{Teo}
Los siguientes enunciados son equivalentes
\begin{itemize}
\item[i)] El proceso retardado de renovaci\'on $N$ es estacionario.

\item[ii)] EL proceso de tiempos de recurrencia hacia adelante $B\left(t\right)$ es estacionario.


\item[iii)] $\esp\left[N\left(t\right)\right]=t/\mu$,


\item[iv)] $G\left(t\right)=F_{e}\left(t\right)=\frac{1}{\mu}\int_{0}^{t}\left[1-F\left(s\right)\right]ds$
\end{itemize}
Cuando estos enunciados son ciertos, $P\left\{B\left(t\right)\leq x\right\}=F_{e}\left(x\right)$, para $t,x\geq0$.

\end{Teo}

\begin{Note}
Una consecuencia del teorema anterior es que el Proceso Poisson es el \'unico proceso sin retardo que es estacionario.
\end{Note}

\begin{Coro}
El proceso de renovaci\'on $N\left(t\right)$ sin retardo, y cuyos tiempos de inter renonaci\'on tienen media finita, es estacionario si y s\'olo si es un proceso Poisson.

\end{Coro}


%________________________________________________________________________
%\subsection{Procesos Regenerativos}
%________________________________________________________________________



\begin{Note}
Si $\tilde{X}\left(t\right)$ con espacio de estados $\tilde{S}$ es regenerativo sobre $T_{n}$, entonces $X\left(t\right)=f\left(\tilde{X}\left(t\right)\right)$ tambi\'en es regenerativo sobre $T_{n}$, para cualquier funci\'on $f:\tilde{S}\rightarrow S$.
\end{Note}

\begin{Note}
Los procesos regenerativos son crudamente regenerativos, pero no al rev\'es.
\end{Note}
%\subsection*{Procesos Regenerativos: Sigman\cite{Sigman1}}
\begin{Def}[Definici\'on Cl\'asica]
Un proceso estoc\'astico $X=\left\{X\left(t\right):t\geq0\right\}$ es llamado regenerativo is existe una variable aleatoria $R_{1}>0$ tal que
\begin{itemize}
\item[i)] $\left\{X\left(t+R_{1}\right):t\geq0\right\}$ es independiente de $\left\{\left\{X\left(t\right):t<R_{1}\right\},\right\}$
\item[ii)] $\left\{X\left(t+R_{1}\right):t\geq0\right\}$ es estoc\'asticamente equivalente a $\left\{X\left(t\right):t>0\right\}$
\end{itemize}

Llamamos a $R_{1}$ tiempo de regeneraci\'on, y decimos que $X$ se regenera en este punto.
\end{Def}

$\left\{X\left(t+R_{1}\right)\right\}$ es regenerativo con tiempo de regeneraci\'on $R_{2}$, independiente de $R_{1}$ pero con la misma distribuci\'on que $R_{1}$. Procediendo de esta manera se obtiene una secuencia de variables aleatorias independientes e id\'enticamente distribuidas $\left\{R_{n}\right\}$ llamados longitudes de ciclo. Si definimos a $Z_{k}\equiv R_{1}+R_{2}+\cdots+R_{k}$, se tiene un proceso de renovaci\'on llamado proceso de renovaci\'on encajado para $X$.




\begin{Def}
Para $x$ fijo y para cada $t\geq0$, sea $I_{x}\left(t\right)=1$ si $X\left(t\right)\leq x$,  $I_{x}\left(t\right)=0$ en caso contrario, y def\'inanse los tiempos promedio
\begin{eqnarray*}
\overline{X}&=&lim_{t\rightarrow\infty}\frac{1}{t}\int_{0}^{\infty}X\left(u\right)du\\
\prob\left(X_{\infty}\leq x\right)&=&lim_{t\rightarrow\infty}\frac{1}{t}\int_{0}^{\infty}I_{x}\left(u\right)du,
\end{eqnarray*}
cuando estos l\'imites existan.
\end{Def}

Como consecuencia del teorema de Renovaci\'on-Recompensa, se tiene que el primer l\'imite  existe y es igual a la constante
\begin{eqnarray*}
\overline{X}&=&\frac{\esp\left[\int_{0}^{R_{1}}X\left(t\right)dt\right]}{\esp\left[R_{1}\right]},
\end{eqnarray*}
suponiendo que ambas esperanzas son finitas.

\begin{Note}
\begin{itemize}
\item[a)] Si el proceso regenerativo $X$ es positivo recurrente y tiene trayectorias muestrales no negativas, entonces la ecuaci\'on anterior es v\'alida.
\item[b)] Si $X$ es positivo recurrente regenerativo, podemos construir una \'unica versi\'on estacionaria de este proceso, $X_{e}=\left\{X_{e}\left(t\right)\right\}$, donde $X_{e}$ es un proceso estoc\'astico regenerativo y estrictamente estacionario, con distribuci\'on marginal distribuida como $X_{\infty}$
\end{itemize}
\end{Note}

%________________________________________________________________________
%\subsection{Procesos Regenerativos}
%________________________________________________________________________

Para $\left\{X\left(t\right):t\geq0\right\}$ Proceso Estoc\'astico a tiempo continuo con estado de espacios $S$, que es un espacio m\'etrico, con trayectorias continuas por la derecha y con l\'imites por la izquierda c.s. Sea $N\left(t\right)$ un proceso de renovaci\'on en $\rea_{+}$ definido en el mismo espacio de probabilidad que $X\left(t\right)$, con tiempos de renovaci\'on $T$ y tiempos de inter-renovaci\'on $\xi_{n}=T_{n}-T_{n-1}$, con misma distribuci\'on $F$ de media finita $\mu$.



\begin{Def}
Para el proceso $\left\{\left(N\left(t\right),X\left(t\right)\right):t\geq0\right\}$, sus trayectoria muestrales en el intervalo de tiempo $\left[T_{n-1},T_{n}\right)$ est\'an descritas por
\begin{eqnarray*}
\zeta_{n}=\left(\xi_{n},\left\{X\left(T_{n-1}+t\right):0\leq t<\xi_{n}\right\}\right)
\end{eqnarray*}
Este $\zeta_{n}$ es el $n$-\'esimo segmento del proceso. El proceso es regenerativo sobre los tiempos $T_{n}$ si sus segmentos $\zeta_{n}$ son independientes e id\'enticamennte distribuidos.
\end{Def}


\begin{Note}
Si $\tilde{X}\left(t\right)$ con espacio de estados $\tilde{S}$ es regenerativo sobre $T_{n}$, entonces $X\left(t\right)=f\left(\tilde{X}\left(t\right)\right)$ tambi\'en es regenerativo sobre $T_{n}$, para cualquier funci\'on $f:\tilde{S}\rightarrow S$.
\end{Note}

\begin{Note}
Los procesos regenerativos son crudamente regenerativos, pero no al rev\'es.
\end{Note}

\begin{Def}[Definici\'on Cl\'asica]
Un proceso estoc\'astico $X=\left\{X\left(t\right):t\geq0\right\}$ es llamado regenerativo is existe una variable aleatoria $R_{1}>0$ tal que
\begin{itemize}
\item[i)] $\left\{X\left(t+R_{1}\right):t\geq0\right\}$ es independiente de $\left\{\left\{X\left(t\right):t<R_{1}\right\},\right\}$
\item[ii)] $\left\{X\left(t+R_{1}\right):t\geq0\right\}$ es estoc\'asticamente equivalente a $\left\{X\left(t\right):t>0\right\}$
\end{itemize}

Llamamos a $R_{1}$ tiempo de regeneraci\'on, y decimos que $X$ se regenera en este punto.
\end{Def}

$\left\{X\left(t+R_{1}\right)\right\}$ es regenerativo con tiempo de regeneraci\'on $R_{2}$, independiente de $R_{1}$ pero con la misma distribuci\'on que $R_{1}$. Procediendo de esta manera se obtiene una secuencia de variables aleatorias independientes e id\'enticamente distribuidas $\left\{R_{n}\right\}$ llamados longitudes de ciclo. Si definimos a $Z_{k}\equiv R_{1}+R_{2}+\cdots+R_{k}$, se tiene un proceso de renovaci\'on llamado proceso de renovaci\'on encajado para $X$.

\begin{Note}
Un proceso regenerativo con media de la longitud de ciclo finita es llamado positivo recurrente.
\end{Note}


\begin{Def}
Para $x$ fijo y para cada $t\geq0$, sea $I_{x}\left(t\right)=1$ si $X\left(t\right)\leq x$,  $I_{x}\left(t\right)=0$ en caso contrario, y def\'inanse los tiempos promedio
\begin{eqnarray*}
\overline{X}&=&lim_{t\rightarrow\infty}\frac{1}{t}\int_{0}^{\infty}X\left(u\right)du\\
\prob\left(X_{\infty}\leq x\right)&=&lim_{t\rightarrow\infty}\frac{1}{t}\int_{0}^{\infty}I_{x}\left(u\right)du,
\end{eqnarray*}
cuando estos l\'imites existan.
\end{Def}

Como consecuencia del teorema de Renovaci\'on-Recompensa, se tiene que el primer l\'imite  existe y es igual a la constante
\begin{eqnarray*}
\overline{X}&=&\frac{\esp\left[\int_{0}^{R_{1}}X\left(t\right)dt\right]}{\esp\left[R_{1}\right]},
\end{eqnarray*}
suponiendo que ambas esperanzas son finitas.

\begin{Note}
\begin{itemize}
\item[a)] Si el proceso regenerativo $X$ es positivo recurrente y tiene trayectorias muestrales no negativas, entonces la ecuaci\'on anterior es v\'alida.
\item[b)] Si $X$ es positivo recurrente regenerativo, podemos construir una \'unica versi\'on estacionaria de este proceso, $X_{e}=\left\{X_{e}\left(t\right)\right\}$, donde $X_{e}$ es un proceso estoc\'astico regenerativo y estrictamente estacionario, con distribuci\'on marginal distribuida como $X_{\infty}$
\end{itemize}
\end{Note}

%__________________________________________________________________________________________
%\subsection{Procesos Regenerativos Estacionarios - Stidham \cite{Stidham}}
%__________________________________________________________________________________________


Un proceso estoc\'astico a tiempo continuo $\left\{V\left(t\right),t\geq0\right\}$ es un proceso regenerativo si existe una sucesi\'on de variables aleatorias independientes e id\'enticamente distribuidas $\left\{X_{1},X_{2},\ldots\right\}$, sucesi\'on de renovaci\'on, tal que para cualquier conjunto de Borel $A$, 

\begin{eqnarray*}
\prob\left\{V\left(t\right)\in A|X_{1}+X_{2}+\cdots+X_{R\left(t\right)}=s,\left\{V\left(\tau\right),\tau<s\right\}\right\}=\prob\left\{V\left(t-s\right)\in A|X_{1}>t-s\right\},
\end{eqnarray*}
para todo $0\leq s\leq t$, donde $R\left(t\right)=\max\left\{X_{1}+X_{2}+\cdots+X_{j}\leq t\right\}=$n\'umero de renovaciones ({\emph{puntos de regeneraci\'on}}) que ocurren en $\left[0,t\right]$. El intervalo $\left[0,X_{1}\right)$ es llamado {\emph{primer ciclo de regeneraci\'on}} de $\left\{V\left(t \right),t\geq0\right\}$, $\left[X_{1},X_{1}+X_{2}\right)$ el {\emph{segundo ciclo de regeneraci\'on}}, y as\'i sucesivamente.

Sea $X=X_{1}$ y sea $F$ la funci\'on de distrbuci\'on de $X$


\begin{Def}
Se define el proceso estacionario, $\left\{V^{*}\left(t\right),t\geq0\right\}$, para $\left\{V\left(t\right),t\geq0\right\}$ por

\begin{eqnarray*}
\prob\left\{V\left(t\right)\in A\right\}=\frac{1}{\esp\left[X\right]}\int_{0}^{\infty}\prob\left\{V\left(t+x\right)\in A|X>x\right\}\left(1-F\left(x\right)\right)dx,
\end{eqnarray*} 
para todo $t\geq0$ y todo conjunto de Borel $A$.
\end{Def}

\begin{Def}
Una distribuci\'on se dice que es {\emph{aritm\'etica}} si todos sus puntos de incremento son m\'ultiplos de la forma $0,\lambda, 2\lambda,\ldots$ para alguna $\lambda>0$ entera.
\end{Def}


\begin{Def}
Una modificaci\'on medible de un proceso $\left\{V\left(t\right),t\geq0\right\}$, es una versi\'on de este, $\left\{V\left(t,w\right)\right\}$ conjuntamente medible para $t\geq0$ y para $w\in S$, $S$ espacio de estados para $\left\{V\left(t\right),t\geq0\right\}$.
\end{Def}

\begin{Teo}
Sea $\left\{V\left(t\right),t\geq\right\}$ un proceso regenerativo no negativo con modificaci\'on medible. Sea $\esp\left[X\right]<\infty$. Entonces el proceso estacionario dado por la ecuaci\'on anterior est\'a bien definido y tiene funci\'on de distribuci\'on independiente de $t$, adem\'as
\begin{itemize}
\item[i)] \begin{eqnarray*}
\esp\left[V^{*}\left(0\right)\right]&=&\frac{\esp\left[\int_{0}^{X}V\left(s\right)ds\right]}{\esp\left[X\right]}\end{eqnarray*}
\item[ii)] Si $\esp\left[V^{*}\left(0\right)\right]<\infty$, equivalentemente, si $\esp\left[\int_{0}^{X}V\left(s\right)ds\right]<\infty$,entonces
\begin{eqnarray*}
\frac{\int_{0}^{t}V\left(s\right)ds}{t}\rightarrow\frac{\esp\left[\int_{0}^{X}V\left(s\right)ds\right]}{\esp\left[X\right]}
\end{eqnarray*}
con probabilidad 1 y en media, cuando $t\rightarrow\infty$.
\end{itemize}
\end{Teo}
%
%___________________________________________________________________________________________
%\vspace{5.5cm}
%\chapter{Cadenas de Markov estacionarias}
%\vspace{-1.0cm}


%__________________________________________________________________________________________
%\subsection{Procesos Regenerativos Estacionarios - Stidham \cite{Stidham}}
%__________________________________________________________________________________________


Un proceso estoc\'astico a tiempo continuo $\left\{V\left(t\right),t\geq0\right\}$ es un proceso regenerativo si existe una sucesi\'on de variables aleatorias independientes e id\'enticamente distribuidas $\left\{X_{1},X_{2},\ldots\right\}$, sucesi\'on de renovaci\'on, tal que para cualquier conjunto de Borel $A$, 

\begin{eqnarray*}
\prob\left\{V\left(t\right)\in A|X_{1}+X_{2}+\cdots+X_{R\left(t\right)}=s,\left\{V\left(\tau\right),\tau<s\right\}\right\}=\prob\left\{V\left(t-s\right)\in A|X_{1}>t-s\right\},
\end{eqnarray*}
para todo $0\leq s\leq t$, donde $R\left(t\right)=\max\left\{X_{1}+X_{2}+\cdots+X_{j}\leq t\right\}=$n\'umero de renovaciones ({\emph{puntos de regeneraci\'on}}) que ocurren en $\left[0,t\right]$. El intervalo $\left[0,X_{1}\right)$ es llamado {\emph{primer ciclo de regeneraci\'on}} de $\left\{V\left(t \right),t\geq0\right\}$, $\left[X_{1},X_{1}+X_{2}\right)$ el {\emph{segundo ciclo de regeneraci\'on}}, y as\'i sucesivamente.

Sea $X=X_{1}$ y sea $F$ la funci\'on de distrbuci\'on de $X$


\begin{Def}
Se define el proceso estacionario, $\left\{V^{*}\left(t\right),t\geq0\right\}$, para $\left\{V\left(t\right),t\geq0\right\}$ por

\begin{eqnarray*}
\prob\left\{V\left(t\right)\in A\right\}=\frac{1}{\esp\left[X\right]}\int_{0}^{\infty}\prob\left\{V\left(t+x\right)\in A|X>x\right\}\left(1-F\left(x\right)\right)dx,
\end{eqnarray*} 
para todo $t\geq0$ y todo conjunto de Borel $A$.
\end{Def}

\begin{Def}
Una distribuci\'on se dice que es {\emph{aritm\'etica}} si todos sus puntos de incremento son m\'ultiplos de la forma $0,\lambda, 2\lambda,\ldots$ para alguna $\lambda>0$ entera.
\end{Def}


\begin{Def}
Una modificaci\'on medible de un proceso $\left\{V\left(t\right),t\geq0\right\}$, es una versi\'on de este, $\left\{V\left(t,w\right)\right\}$ conjuntamente medible para $t\geq0$ y para $w\in S$, $S$ espacio de estados para $\left\{V\left(t\right),t\geq0\right\}$.
\end{Def}

\begin{Teo}
Sea $\left\{V\left(t\right),t\geq\right\}$ un proceso regenerativo no negativo con modificaci\'on medible. Sea $\esp\left[X\right]<\infty$. Entonces el proceso estacionario dado por la ecuaci\'on anterior est\'a bien definido y tiene funci\'on de distribuci\'on independiente de $t$, adem\'as
\begin{itemize}
\item[i)] \begin{eqnarray*}
\esp\left[V^{*}\left(0\right)\right]&=&\frac{\esp\left[\int_{0}^{X}V\left(s\right)ds\right]}{\esp\left[X\right]}\end{eqnarray*}
\item[ii)] Si $\esp\left[V^{*}\left(0\right)\right]<\infty$, equivalentemente, si $\esp\left[\int_{0}^{X}V\left(s\right)ds\right]<\infty$,entonces
\begin{eqnarray*}
\frac{\int_{0}^{t}V\left(s\right)ds}{t}\rightarrow\frac{\esp\left[\int_{0}^{X}V\left(s\right)ds\right]}{\esp\left[X\right]}
\end{eqnarray*}
con probabilidad 1 y en media, cuando $t\rightarrow\infty$.
\end{itemize}
\end{Teo}

Para $\left\{X\left(t\right):t\geq0\right\}$ Proceso Estoc\'astico a tiempo continuo con estado de espacios $S$, que es un espacio m\'etrico, con trayectorias continuas por la derecha y con l\'imites por la izquierda c.s. Sea $N\left(t\right)$ un proceso de renovaci\'on en $\rea_{+}$ definido en el mismo espacio de probabilidad que $X\left(t\right)$, con tiempos de renovaci\'on $T$ y tiempos de inter-renovaci\'on $\xi_{n}=T_{n}-T_{n-1}$, con misma distribuci\'on $F$ de media finita $\mu$.


%______________________________________________________________________
%\subsection{Ejemplos, Notas importantes}


Sean $T_{1},T_{2},\ldots$ los puntos donde las longitudes de las colas de la red de sistemas de visitas c\'iclicas son cero simult\'aneamente, cuando la cola $Q_{j}$ es visitada por el servidor para dar servicio, es decir, $L_{1}\left(T_{i}\right)=0,L_{2}\left(T_{i}\right)=0,\hat{L}_{1}\left(T_{i}\right)=0$ y $\hat{L}_{2}\left(T_{i}\right)=0$, a estos puntos se les denominar\'a puntos regenerativos. Sea la funci\'on generadora de momentos para $L_{i}$, el n\'umero de usuarios en la cola $Q_{i}\left(z\right)$ en cualquier momento, est\'a dada por el tiempo promedio de $z^{L_{i}\left(t\right)}$ sobre el ciclo regenerativo definido anteriormente:

\begin{eqnarray*}
Q_{i}\left(z\right)&=&\esp\left[z^{L_{i}\left(t\right)}\right]=\frac{\esp\left[\sum_{m=1}^{M_{i}}\sum_{t=\tau_{i}\left(m\right)}^{\tau_{i}\left(m+1\right)-1}z^{L_{i}\left(t\right)}\right]}{\esp\left[\sum_{m=1}^{M_{i}}\tau_{i}\left(m+1\right)-\tau_{i}\left(m\right)\right]}
\end{eqnarray*}

$M_{i}$ es un tiempo de paro en el proceso regenerativo con $\esp\left[M_{i}\right]<\infty$\footnote{En Stidham\cite{Stidham} y Heyman  se muestra que una condici\'on suficiente para que el proceso regenerativo 
estacionario sea un procesoo estacionario es que el valor esperado del tiempo del ciclo regenerativo sea finito, es decir: $\esp\left[\sum_{m=1}^{M_{i}}C_{i}^{(m)}\right]<\infty$, como cada $C_{i}^{(m)}$ contiene intervalos de r\'eplica positivos, se tiene que $\esp\left[M_{i}\right]<\infty$, adem\'as, como $M_{i}>0$, se tiene que la condici\'on anterior es equivalente a tener que $\esp\left[C_{i}\right]<\infty$,
por lo tanto una condici\'on suficiente para la existencia del proceso regenerativo est\'a dada por $\sum_{k=1}^{N}\mu_{k}<1.$}, se sigue del lema de Wald que:


\begin{eqnarray*}
\esp\left[\sum_{m=1}^{M_{i}}\sum_{t=\tau_{i}\left(m\right)}^{\tau_{i}\left(m+1\right)-1}z^{L_{i}\left(t\right)}\right]&=&\esp\left[M_{i}\right]\esp\left[\sum_{t=\tau_{i}\left(m\right)}^{\tau_{i}\left(m+1\right)-1}z^{L_{i}\left(t\right)}\right]\\
\esp\left[\sum_{m=1}^{M_{i}}\tau_{i}\left(m+1\right)-\tau_{i}\left(m\right)\right]&=&\esp\left[M_{i}\right]\esp\left[\tau_{i}\left(m+1\right)-\tau_{i}\left(m\right)\right]
\end{eqnarray*}

por tanto se tiene que


\begin{eqnarray*}
Q_{i}\left(z\right)&=&\frac{\esp\left[\sum_{t=\tau_{i}\left(m\right)}^{\tau_{i}\left(m+1\right)-1}z^{L_{i}\left(t\right)}\right]}{\esp\left[\tau_{i}\left(m+1\right)-\tau_{i}\left(m\right)\right]}
\end{eqnarray*}

observar que el denominador es simplemente la duraci\'on promedio del tiempo del ciclo.


Haciendo las siguientes sustituciones en la ecuaci\'on (\ref{Corolario2}): $n\rightarrow t-\tau_{i}\left(m\right)$, $T \rightarrow \overline{\tau}_{i}\left(m\right)-\tau_{i}\left(m\right)$, $L_{n}\rightarrow L_{i}\left(t\right)$ y $F\left(z\right)=\esp\left[z^{L_{0}}\right]\rightarrow F_{i}\left(z\right)=\esp\left[z^{L_{i}\tau_{i}\left(m\right)}\right]$, se puede ver que

\begin{eqnarray}\label{Eq.Arribos.Primera}
\esp\left[\sum_{n=0}^{T-1}z^{L_{n}}\right]=
\esp\left[\sum_{t=\tau_{i}\left(m\right)}^{\overline{\tau}_{i}\left(m\right)-1}z^{L_{i}\left(t\right)}\right]
=z\frac{F_{i}\left(z\right)-1}{z-P_{i}\left(z\right)}
\end{eqnarray}

Por otra parte durante el tiempo de intervisita para la cola $i$, $L_{i}\left(t\right)$ solamente se incrementa de manera que el incremento por intervalo de tiempo est\'a dado por la funci\'on generadora de probabilidades de $P_{i}\left(z\right)$, por tanto la suma sobre el tiempo de intervisita puede evaluarse como:

\begin{eqnarray*}
\esp\left[\sum_{t=\tau_{i}\left(m\right)}^{\tau_{i}\left(m+1\right)-1}z^{L_{i}\left(t\right)}\right]&=&\esp\left[\sum_{t=\tau_{i}\left(m\right)}^{\tau_{i}\left(m+1\right)-1}\left\{P_{i}\left(z\right)\right\}^{t-\overline{\tau}_{i}\left(m\right)}\right]=\frac{1-\esp\left[\left\{P_{i}\left(z\right)\right\}^{\tau_{i}\left(m+1\right)-\overline{\tau}_{i}\left(m\right)}\right]}{1-P_{i}\left(z\right)}\\
&=&\frac{1-I_{i}\left[P_{i}\left(z\right)\right]}{1-P_{i}\left(z\right)}
\end{eqnarray*}
por tanto

\begin{eqnarray*}
\esp\left[\sum_{t=\tau_{i}\left(m\right)}^{\tau_{i}\left(m+1\right)-1}z^{L_{i}\left(t\right)}\right]&=&
\frac{1-F_{i}\left(z\right)}{1-P_{i}\left(z\right)}
\end{eqnarray*}

Por lo tanto

\begin{eqnarray*}
Q_{i}\left(z\right)&=&\frac{\esp\left[\sum_{t=\tau_{i}\left(m\right)}^{\tau_{i}
\left(m+1\right)-1}z^{L_{i}\left(t\right)}\right]}{\esp\left[\tau_{i}\left(m+1\right)-\tau_{i}\left(m\right)\right]}\\
&=&\frac{1}{\esp\left[\tau_{i}\left(m+1\right)-\tau_{i}\left(m\right)\right]}
\left\{
\esp\left[\sum_{t=\tau_{i}\left(m\right)}^{\overline{\tau}_{i}\left(m\right)-1}
z^{L_{i}\left(t\right)}\right]
+\esp\left[\sum_{t=\overline{\tau}_{i}\left(m\right)}^{\tau_{i}\left(m+1\right)-1}
z^{L_{i}\left(t\right)}\right]\right\}\\
&=&\frac{1}{\esp\left[\tau_{i}\left(m+1\right)-\tau_{i}\left(m\right)\right]}
\left\{
z\frac{F_{i}\left(z\right)-1}{z-P_{i}\left(z\right)}+\frac{1-F_{i}\left(z\right)}
{1-P_{i}\left(z\right)}
\right\}
\end{eqnarray*}

es decir

\begin{equation}
Q_{i}\left(z\right)=\frac{1}{\esp\left[C_{i}\right]}\cdot\frac{1-F_{i}\left(z\right)}{P_{i}\left(z\right)-z}\cdot\frac{\left(1-z\right)P_{i}\left(z\right)}{1-P_{i}\left(z\right)}
\end{equation}

\begin{Teo}
Dada una Red de Sistemas de Visitas C\'iclicas (RSVC), conformada por dos Sistemas de Visitas C\'iclicas (SVC), donde cada uno de ellos consta de dos colas tipo $M/M/1$. Los dos sistemas est\'an comunicados entre s\'i por medio de la transferencia de usuarios entre las colas $Q_{1}\leftrightarrow Q_{3}$ y $Q_{2}\leftrightarrow Q_{4}$. Se definen los eventos para los procesos de arribos al tiempo $t$, $A_{j}\left(t\right)=\left\{0 \textrm{ arribos en }Q_{j}\textrm{ al tiempo }t\right\}$ para alg\'un tiempo $t\geq0$ y $Q_{j}$ la cola $j$-\'esima en la RSVC, para $j=1,2,3,4$.  Existe un intervalo $I\neq\emptyset$ tal que para $T^{*}\in I$, tal que $\prob\left\{A_{1}\left(T^{*}\right),A_{2}\left(Tt^{*}\right),
A_{3}\left(T^{*}\right),A_{4}\left(T^{*}\right)|T^{*}\in I\right\}>0$.
\end{Teo}

\begin{proof}
Sin p\'erdida de generalidad podemos considerar como base del an\'alisis a la cola $Q_{1}$ del primer sistema que conforma la RSVC.

Sea $n>0$, ciclo en el primer sistema en el que se sabe que $L_{j}\left(\overline{\tau}_{1}\left(n\right)\right)=0$, pues la pol\'itica de servicio con que atienden los servidores es la exhaustiva. Como es sabido, para trasladarse a la siguiente cola, el servidor incurre en un tiempo de traslado $r_{1}\left(n\right)>0$, entonces tenemos que $\tau_{2}\left(n\right)=\overline{\tau}_{1}\left(n\right)+r_{1}\left(n\right)$.


Definamos el intervalo $I_{1}\equiv\left[\overline{\tau}_{1}\left(n\right),\tau_{2}\left(n\right)\right]$ de longitud $\xi_{1}=r_{1}\left(n\right)$. Dado que los tiempos entre arribo son exponenciales con tasa $\tilde{\mu}_{1}=\mu_{1}+\hat{\mu}_{1}$ ($\mu_{1}$ son los arribos a $Q_{1}$ por primera vez al sistema, mientras que $\hat{\mu}_{1}$ son los arribos de traslado procedentes de $Q_{3}$) se tiene que la probabilidad del evento $A_{1}\left(t\right)$ est\'a dada por 

\begin{equation}
\prob\left\{A_{1}\left(t\right)|t\in I_{1}\left(n\right)\right\}=e^{-\tilde{\mu}_{1}\xi_{1}\left(n\right)}.
\end{equation} 

Por otra parte, para la cola $Q_{2}$, el tiempo $\overline{\tau}_{2}\left(n-1\right)$ es tal que $L_{2}\left(\overline{\tau}_{2}\left(n-1\right)\right)=0$, es decir, es el tiempo en que la cola queda totalmente vac\'ia en el ciclo anterior a $n$. Entonces tenemos un sgundo intervalo $I_{2}\equiv\left[\overline{\tau}_{2}\left(n-1\right),\tau_{2}\left(n\right)\right]$. Por lo tanto la probabilidad del evento $A_{2}\left(t\right)$ tiene probabilidad dada por

\begin{equation}
\prob\left\{A_{2}\left(t\right)|t\in I_{2}\left(n\right)\right\}=e^{-\tilde{\mu}_{2}\xi_{2}\left(n\right)},
\end{equation} 

donde $\xi_{2}\left(n\right)=\tau_{2}\left(n\right)-\overline{\tau}_{2}\left(n-1\right)$.



Entonces, se tiene que

\begin{eqnarray*}
\prob\left\{A_{1}\left(t\right),A_{2}\left(t\right)|t\in I_{1}\left(n\right)\right\}&=&
\prob\left\{A_{1}\left(t\right)|t\in I_{1}\left(n\right)\right\}
\prob\left\{A_{2}\left(t\right)|t\in I_{1}\left(n\right)\right\}\\
&\geq&
\prob\left\{A_{1}\left(t\right)|t\in I_{1}\left(n\right)\right\}
\prob\left\{A_{2}\left(t\right)|t\in I_{2}\left(n\right)\right\}\\
&=&e^{-\tilde{\mu}_{1}\xi_{1}\left(n\right)}e^{-\tilde{\mu}_{2}\xi_{2}\left(n\right)}
=e^{-\left[\tilde{\mu}_{1}\xi_{1}\left(n\right)+\tilde{\mu}_{2}\xi_{2}\left(n\right)\right]}.
\end{eqnarray*}


es decir, 

\begin{equation}
\prob\left\{A_{1}\left(t\right),A_{2}\left(t\right)|t\in I_{1}\left(n\right)\right\}
=e^{-\left[\tilde{\mu}_{1}\xi_{1}\left(n\right)+\tilde{\mu}_{2}\xi_{2}
\left(n\right)\right]}>0.
\end{equation}

En lo que respecta a la relaci\'on entre los dos SVC que conforman la RSVC, se afirma que existe $m>0$ tal que $\overline{\tau}_{3}\left(m\right)<\tau_{2}\left(n\right)<\tau_{4}\left(m\right)$.

Para $Q_{3}$ sea $I_{3}=\left[\overline{\tau}_{3}\left(m\right),\tau_{4}\left(m\right)\right]$ con longitud  $\xi_{3}\left(m\right)=r_{3}\left(m\right)$, entonces 

\begin{equation}
\prob\left\{A_{3}\left(t\right)|t\in I_{3}\left(n\right)\right\}=e^{-\tilde{\mu}_{3}\xi_{3}\left(n\right)}.
\end{equation} 

An\'alogamente que como se hizo para $Q_{2}$, tenemos que para $Q_{4}$ se tiene el intervalo $I_{4}=\left[\overline{\tau}_{4}\left(m-1\right),\tau_{4}\left(m\right)\right]$ con longitud $\xi_{4}\left(m\right)=\tau_{4}\left(m\right)-\overline{\tau}_{4}\left(m-1\right)$, entonces


\begin{equation}
\prob\left\{A_{4}\left(t\right)|t\in I_{4}\left(m\right)\right\}=e^{-\tilde{\mu}_{4}\xi_{4}\left(n\right)}.
\end{equation} 

Al igual que para el primer sistema, dado que $I_{3}\left(m\right)\subset I_{4}\left(m\right)$, se tiene que

\begin{eqnarray*}
\xi_{3}\left(m\right)\leq\xi_{4}\left(m\right)&\Leftrightarrow& -\xi_{3}\left(m\right)\geq-\xi_{4}\left(m\right)
\\
-\tilde{\mu}_{4}\xi_{3}\left(m\right)\geq-\tilde{\mu}_{4}\xi_{4}\left(m\right)&\Leftrightarrow&
e^{-\tilde{\mu}_{4}\xi_{3}\left(m\right)}\geq e^{-\tilde{\mu}_{4}\xi_{4}\left(m\right)}\\
\prob\left\{A_{4}\left(t\right)|t\in I_{3}\left(m\right)\right\}&\geq&
\prob\left\{A_{4}\left(t\right)|t\in I_{4}\left(m\right)\right\}
\end{eqnarray*}

Entonces, dado que los eventos $A_{3}$ y $A_{4}$ son independientes, se tiene que

\begin{eqnarray*}
\prob\left\{A_{3}\left(t\right),A_{4}\left(t\right)|t\in I_{3}\left(m\right)\right\}&=&
\prob\left\{A_{3}\left(t\right)|t\in I_{3}\left(m\right)\right\}
\prob\left\{A_{4}\left(t\right)|t\in I_{3}\left(m\right)\right\}\\
&\geq&
\prob\left\{A_{3}\left(t\right)|t\in I_{3}\left(n\right)\right\}
\prob\left\{A_{4}\left(t\right)|t\in I_{4}\left(n\right)\right\}\\
&=&e^{-\tilde{\mu}_{3}\xi_{3}\left(m\right)}e^{-\tilde{\mu}_{4}\xi_{4}
\left(m\right)}
=e^{-\left[\tilde{\mu}_{3}\xi_{3}\left(m\right)+\tilde{\mu}_{4}\xi_{4}
\left(m\right)\right]}.
\end{eqnarray*}


es decir, 

\begin{equation}
\prob\left\{A_{3}\left(t\right),A_{4}\left(t\right)|t\in I_{3}\left(m\right)\right\}
=e^{-\left[\tilde{\mu}_{3}\xi_{3}\left(m\right)+\tilde{\mu}_{4}\xi_{4}
\left(m\right)\right]}>0.
\end{equation}

Por construcci\'on se tiene que $I\left(n,m\right)\equiv I_{1}\left(n\right)\cap I_{3}\left(m\right)\neq\emptyset$,entonces en particular se tienen las contenciones $I\left(n,m\right)\subseteq I_{1}\left(n\right)$ y $I\left(n,m\right)\subseteq I_{3}\left(m\right)$, por lo tanto si definimos $\xi_{n,m}\equiv\ell\left(I\left(n,m\right)\right)$ tenemos que

\begin{eqnarray*}
\xi_{n,m}\leq\xi_{1}\left(n\right)\textrm{ y }\xi_{n,m}\leq\xi_{3}\left(m\right)\textrm{ entonces }
-\xi_{n,m}\geq-\xi_{1}\left(n\right)\textrm{ y }-\xi_{n,m}\leq-\xi_{3}\left(m\right)\\
\end{eqnarray*}
por lo tanto tenemos las desigualdades 



\begin{eqnarray*}
\begin{array}{ll}
-\tilde{\mu}_{1}\xi_{n,m}\geq-\tilde{\mu}_{1}\xi_{1}\left(n\right),&
-\tilde{\mu}_{2}\xi_{n,m}\geq-\tilde{\mu}_{2}\xi_{1}\left(n\right)
\geq-\tilde{\mu}_{2}\xi_{2}\left(n\right),\\
-\tilde{\mu}_{3}\xi_{n,m}\geq-\tilde{\mu}_{3}\xi_{3}\left(m\right),&
-\tilde{\mu}_{4}\xi_{n,m}\geq-\tilde{\mu}_{4}\xi_{3}\left(m\right)
\geq-\tilde{\mu}_{4}\xi_{4}\left(m\right).
\end{array}
\end{eqnarray*}

Sea $T^{*}\in I_{n,m}$, entonces dado que en particular $T^{*}\in I_{1}\left(n\right)$ se cumple con probabilidad positiva que no hay arribos a las colas $Q_{1}$ y $Q_{2}$, en consecuencia, tampoco hay usuarios de transferencia para $Q_{3}$ y $Q_{4}$, es decir, $\tilde{\mu}_{1}=\mu_{1}$, $\tilde{\mu}_{2}=\mu_{2}$, $\tilde{\mu}_{3}=\mu_{3}$, $\tilde{\mu}_{4}=\mu_{4}$, es decir, los eventos $Q_{1}$ y $Q_{3}$ son condicionalmente independientes en el intervalo $I_{n,m}$; lo mismo ocurre para las colas $Q_{2}$ y $Q_{4}$, por lo tanto tenemos que


\begin{eqnarray}
\begin{array}{l}
\prob\left\{A_{1}\left(T^{*}\right),A_{2}\left(T^{*}\right),
A_{3}\left(T^{*}\right),A_{4}\left(T^{*}\right)|T^{*}\in I_{n,m}\right\}
=\prod_{j=1}^{4}\prob\left\{A_{j}\left(T^{*}\right)|T^{*}\in I_{n,m}\right\}\\
\geq\prob\left\{A_{1}\left(T^{*}\right)|T^{*}\in I_{1}\left(n\right)\right\}
\prob\left\{A_{2}\left(T^{*}\right)|T^{*}\in I_{2}\left(n\right)\right\}
\prob\left\{A_{3}\left(T^{*}\right)|T^{*}\in I_{3}\left(m\right)\right\}
\prob\left\{A_{4}\left(T^{*}\right)|T^{*}\in I_{4}\left(m\right)\right\}\\
=e^{-\mu_{1}\xi_{1}\left(n\right)}
e^{-\mu_{2}\xi_{2}\left(n\right)}
e^{-\mu_{3}\xi_{3}\left(m\right)}
e^{-\mu_{4}\xi_{4}\left(m\right)}
=e^{-\left[\tilde{\mu}_{1}\xi_{1}\left(n\right)
+\tilde{\mu}_{2}\xi_{2}\left(n\right)
+\tilde{\mu}_{3}\xi_{3}\left(m\right)
+\tilde{\mu}_{4}\xi_{4}
\left(m\right)\right]}>0.
\end{array}
\end{eqnarray}
\end{proof}


Estos resultados aparecen en Daley (1968) \cite{Daley68} para $\left\{T_{n}\right\}$ intervalos de inter-arribo, $\left\{D_{n}\right\}$ intervalos de inter-salida y $\left\{S_{n}\right\}$ tiempos de servicio.

\begin{itemize}
\item Si el proceso $\left\{T_{n}\right\}$ es Poisson, el proceso $\left\{D_{n}\right\}$ es no correlacionado si y s\'olo si es un proceso Poisso, lo cual ocurre si y s\'olo si $\left\{S_{n}\right\}$ son exponenciales negativas.

\item Si $\left\{S_{n}\right\}$ son exponenciales negativas, $\left\{D_{n}\right\}$ es un proceso de renovaci\'on  si y s\'olo si es un proceso Poisson, lo cual ocurre si y s\'olo si $\left\{T_{n}\right\}$ es un proceso Poisson.

\item $\esp\left(D_{n}\right)=\esp\left(T_{n}\right)$.

\item Para un sistema de visitas $GI/M/1$ se tiene el siguiente teorema:

\begin{Teo}
En un sistema estacionario $GI/M/1$ los intervalos de interpartida tienen
\begin{eqnarray*}
\esp\left(e^{-\theta D_{n}}\right)&=&\mu\left(\mu+\theta\right)^{-1}\left[\delta\theta
-\mu\left(1-\delta\right)\alpha\left(\theta\right)\right]
\left[\theta-\mu\left(1-\delta\right)^{-1}\right]\\
\alpha\left(\theta\right)&=&\esp\left[e^{-\theta T_{0}}\right]\\
var\left(D_{n}\right)&=&var\left(T_{0}\right)-\left(\tau^{-1}-\delta^{-1}\right)
2\delta\left(\esp\left(S_{0}\right)\right)^{2}\left(1-\delta\right)^{-1}.
\end{eqnarray*}
\end{Teo}



\begin{Teo}
El proceso de salida de un sistema de colas estacionario $GI/M/1$ es un proceso de renovaci\'on si y s\'olo si el proceso de entrada es un proceso Poisson, en cuyo caso el proceso de salida es un proceso Poisson.
\end{Teo}


\begin{Teo}
Los intervalos de interpartida $\left\{D_{n}\right\}$ de un sistema $M/G/1$ estacionario son no correlacionados si y s\'olo si la distribuci\'on de los tiempos de servicio es exponencial negativa, es decir, el sistema es de tipo  $M/M/1$.

\end{Teo}



\end{itemize}


%\section{Resultados para Procesos de Salida}

En Sigman, Thorison y Wolff \cite{Sigman2} prueban que para la existencia de un una sucesi\'on infinita no decreciente de tiempos de regeneraci\'on $\tau_{1}\leq\tau_{2}\leq\cdots$ en los cuales el proceso se regenera, basta un tiempo de regeneraci\'on $R_{1}$, donde $R_{j}=\tau_{j}-\tau_{j-1}$. Para tal efecto se requiere la existencia de un espacio de probabilidad $\left(\Omega,\mathcal{F},\prob\right)$, y proceso estoc\'astico $\textit{X}=\left\{X\left(t\right):t\geq0\right\}$ con espacio de estados $\left(S,\mathcal{R}\right)$, con $\mathcal{R}$ $\sigma$-\'algebra.

\begin{Prop}
Si existe una variable aleatoria no negativa $R_{1}$ tal que $\theta_{R\footnotesize{1}}X=_{D}X$, entonces $\left(\Omega,\mathcal{F},\prob\right)$ puede extenderse para soportar una sucesi\'on estacionaria de variables aleatorias $R=\left\{R_{k}:k\geq1\right\}$, tal que para $k\geq1$,
\begin{eqnarray*}
\theta_{k}\left(X,R\right)=_{D}\left(X,R\right).
\end{eqnarray*}

Adem\'as, para $k\geq1$, $\theta_{k}R$ es condicionalmente independiente de $\left(X,R_{1},\ldots,R_{k}\right)$, dado $\theta_{\tau k}X$.

\end{Prop}


\begin{itemize}
\item Doob en 1953 demostr\'o que el estado estacionario de un proceso de partida en un sistema de espera $M/G/\infty$, es Poisson con la misma tasa que el proceso de arribos.

\item Burke en 1968, fue el primero en demostrar que el estado estacionario de un proceso de salida de una cola $M/M/s$ es un proceso Poisson.

\item Disney en 1973 obtuvo el siguiente resultado:

\begin{Teo}
Para el sistema de espera $M/G/1/L$ con disciplina FIFO, el proceso $\textbf{I}$ es un proceso de renovaci\'on si y s\'olo si el proceso denominado longitud de la cola es estacionario y se cumple cualquiera de los siguientes casos:

\begin{itemize}
\item[a)] Los tiempos de servicio son identicamente cero;
\item[b)] $L=0$, para cualquier proceso de servicio $S$;
\item[c)] $L=1$ y $G=D$;
\item[d)] $L=\infty$ y $G=M$.
\end{itemize}
En estos casos, respectivamente, las distribuciones de interpartida $P\left\{T_{n+1}-T_{n}\leq t\right\}$ son


\begin{itemize}
\item[a)] $1-e^{-\lambda t}$, $t\geq0$;
\item[b)] $1-e^{-\lambda t}*F\left(t\right)$, $t\geq0$;
\item[c)] $1-e^{-\lambda t}*\indora_{d}\left(t\right)$, $t\geq0$;
\item[d)] $1-e^{-\lambda t}*F\left(t\right)$, $t\geq0$.
\end{itemize}
\end{Teo}


\item Finch (1959) mostr\'o que para los sistemas $M/G/1/L$, con $1\leq L\leq \infty$ con distribuciones de servicio dos veces diferenciable, solamente el sistema $M/M/1/\infty$ tiene proceso de salida de renovaci\'on estacionario.

\item King (1971) demostro que un sistema de colas estacionario $M/G/1/1$ tiene sus tiempos de interpartida sucesivas $D_{n}$ y $D_{n+1}$ son independientes, si y s\'olo si, $G=D$, en cuyo caso le proceso de salida es de renovaci\'on.

\item Disney (1973) demostr\'o que el \'unico sistema estacionario $M/G/1/L$, que tiene proceso de salida de renovaci\'on  son los sistemas $M/M/1$ y $M/D/1/1$.



\item El siguiente resultado es de Disney y Koning (1985)
\begin{Teo}
En un sistema de espera $M/G/s$, el estado estacionario del proceso de salida es un proceso Poisson para cualquier distribuci\'on de los tiempos de servicio si el sistema tiene cualquiera de las siguientes cuatro propiedades.

\begin{itemize}
\item[a)] $s=\infty$
\item[b)] La disciplina de servicio es de procesador compartido.
\item[c)] La disciplina de servicio es LCFS y preemptive resume, esto se cumple para $L<\infty$
\item[d)] $G=M$.
\end{itemize}

\end{Teo}

\item El siguiente resultado es de Alamatsaz (1983)

\begin{Teo}
En cualquier sistema de colas $GI/G/1/L$ con $1\leq L<\infty$ y distribuci\'on de interarribos $A$ y distribuci\'on de los tiempos de servicio $B$, tal que $A\left(0\right)=0$, $A\left(t\right)\left(1-B\left(t\right)\right)>0$ para alguna $t>0$ y $B\left(t\right)$ para toda $t>0$, es imposible que el proceso de salida estacionario sea de renovaci\'on.
\end{Teo}

\end{itemize}

Estos resultados aparecen en Daley (1968) \cite{Daley68} para $\left\{T_{n}\right\}$ intervalos de inter-arribo, $\left\{D_{n}\right\}$ intervalos de inter-salida y $\left\{S_{n}\right\}$ tiempos de servicio.

\begin{itemize}
\item Si el proceso $\left\{T_{n}\right\}$ es Poisson, el proceso $\left\{D_{n}\right\}$ es no correlacionado si y s\'olo si es un proceso Poisso, lo cual ocurre si y s\'olo si $\left\{S_{n}\right\}$ son exponenciales negativas.

\item Si $\left\{S_{n}\right\}$ son exponenciales negativas, $\left\{D_{n}\right\}$ es un proceso de renovaci\'on  si y s\'olo si es un proceso Poisson, lo cual ocurre si y s\'olo si $\left\{T_{n}\right\}$ es un proceso Poisson.

\item $\esp\left(D_{n}\right)=\esp\left(T_{n}\right)$.

\item Para un sistema de visitas $GI/M/1$ se tiene el siguiente teorema:

\begin{Teo}
En un sistema estacionario $GI/M/1$ los intervalos de interpartida tienen
\begin{eqnarray*}
\esp\left(e^{-\theta D_{n}}\right)&=&\mu\left(\mu+\theta\right)^{-1}\left[\delta\theta
-\mu\left(1-\delta\right)\alpha\left(\theta\right)\right]
\left[\theta-\mu\left(1-\delta\right)^{-1}\right]\\
\alpha\left(\theta\right)&=&\esp\left[e^{-\theta T_{0}}\right]\\
var\left(D_{n}\right)&=&var\left(T_{0}\right)-\left(\tau^{-1}-\delta^{-1}\right)
2\delta\left(\esp\left(S_{0}\right)\right)^{2}\left(1-\delta\right)^{-1}.
\end{eqnarray*}
\end{Teo}



\begin{Teo}
El proceso de salida de un sistema de colas estacionario $GI/M/1$ es un proceso de renovaci\'on si y s\'olo si el proceso de entrada es un proceso Poisson, en cuyo caso el proceso de salida es un proceso Poisson.
\end{Teo}


\begin{Teo}
Los intervalos de interpartida $\left\{D_{n}\right\}$ de un sistema $M/G/1$ estacionario son no correlacionados si y s\'olo si la distribuci\'on de los tiempos de servicio es exponencial negativa, es decir, el sistema es de tipo  $M/M/1$.

\end{Teo}



\end{itemize}
%\newpage
%________________________________________________________________________
%\section{Redes de Sistemas de Visitas C\'iclicas}
%________________________________________________________________________

Sean $Q_{1},Q_{2},Q_{3}$ y $Q_{4}$ en una Red de Sistemas de Visitas C\'iclicas (RSVC). Supongamos que cada una de las colas es del tipo $M/M/1$ con tasa de arribo $\mu_{i}$ y que la transferencia de usuarios entre los dos sistemas ocurre entre $Q_{1}\leftrightarrow Q_{3}$ y $Q_{2}\leftrightarrow Q_{4}$ con respectiva tasa de arribo igual a la tasa de salida $\hat{\mu}_{i}=\mu_{i}$, esto se sabe por lo desarrollado en la secci\'on anterior.  

Consideremos, sin p\'erdida de generalidad como base del an\'alisis, la cola $Q_{1}$ adem\'as supongamos al servidor lo comenzamos a observar una vez que termina de atender a la misma para desplazarse y llegar a $Q_{2}$, es decir al tiempo $\tau_{2}$.

Sea $n\in\nat$, $n>0$, ciclo del servidor en que regresa a $Q_{1}$ para dar servicio y atender conforme a la pol\'itica exhaustiva, entonces se tiene que $\overline{\tau}_{1}\left(n\right)$ es el tiempo del servidor en el sistema 1 en que termina de dar servicio a todos los usuarios presentes en la cola, por lo tanto se cumple que $L_{1}\left(\overline{\tau}_{1}\left(n\right)\right)=0$, entonces el servidor para llegar a $Q_{2}$ incurre en un tiempo de traslado $r_{1}$ y por tanto se cumple que $\tau_{2}\left(n\right)=\overline{\tau}_{1}\left(n\right)+r_{1}$. Dado que los tiempos entre arribos son exponenciales se cumple que 

\begin{eqnarray*}
\prob\left\{0 \textrm{ arribos en }Q_{1}\textrm{ en el intervalo }\left[\overline{\tau}_{1}\left(n\right),\overline{\tau}_{1}\left(n\right)+r_{1}\right]\right\}=e^{-\tilde{\mu}_{1}r_{1}},\\
\prob\left\{0 \textrm{ arribos en }Q_{2}\textrm{ en el intervalo }\left[\overline{\tau}_{1}\left(n\right),\overline{\tau}_{1}\left(n\right)+r_{1}\right]\right\}=e^{-\tilde{\mu}_{2}r_{1}}.
\end{eqnarray*}

El evento que nos interesa consiste en que no haya arribos desde que el servidor abandon\'o $Q_{2}$ y regresa nuevamente para dar servicio, es decir en el intervalo de tiempo $\left[\overline{\tau}_{2}\left(n-1\right),\tau_{2}\left(n\right)\right]$. Entonces, si hacemos


\begin{eqnarray*}
\varphi_{1}\left(n\right)&\equiv&\overline{\tau}_{1}\left(n\right)+r_{1}=\overline{\tau}_{2}\left(n-1\right)+r_{1}+r_{2}+\overline{\tau}_{1}\left(n\right)-\tau_{1}\left(n\right)\\
&=&\overline{\tau}_{2}\left(n-1\right)+\overline{\tau}_{1}\left(n\right)-\tau_{1}\left(n\right)+r,,
\end{eqnarray*}

y longitud del intervalo

\begin{eqnarray*}
\xi&\equiv&\overline{\tau}_{1}\left(n\right)+r_{1}-\overline{\tau}_{2}\left(n-1\right)
=\overline{\tau}_{2}\left(n-1\right)+\overline{\tau}_{1}\left(n\right)-\tau_{1}\left(n\right)+r-\overline{\tau}_{2}\left(n-1\right)\\
&=&\overline{\tau}_{1}\left(n\right)-\tau_{1}\left(n\right)+r.
\end{eqnarray*}


Entonces, determinemos la probabilidad del evento no arribos a $Q_{2}$ en $\left[\overline{\tau}_{2}\left(n-1\right),\varphi_{1}\left(n\right)\right]$:

\begin{eqnarray}
\prob\left\{0 \textrm{ arribos en }Q_{2}\textrm{ en el intervalo }\left[\overline{\tau}_{2}\left(n-1\right),\varphi_{1}\left(n\right)\right]\right\}
=e^{-\tilde{\mu}_{2}\xi}.
\end{eqnarray}

De manera an\'aloga, tenemos que la probabilidad de no arribos a $Q_{1}$ en $\left[\overline{\tau}_{2}\left(n-1\right),\varphi_{1}\left(n\right)\right]$ esta dada por

\begin{eqnarray}
\prob\left\{0 \textrm{ arribos en }Q_{1}\textrm{ en el intervalo }\left[\overline{\tau}_{2}\left(n-1\right),\varphi_{1}\left(n\right)\right]\right\}
=e^{-\tilde{\mu}_{1}\xi},
\end{eqnarray}

\begin{eqnarray}
\prob\left\{0 \textrm{ arribos en }Q_{2}\textrm{ en el intervalo }\left[\overline{\tau}_{2}\left(n-1\right),\varphi_{1}\left(n\right)\right]\right\}
=e^{-\tilde{\mu}_{2}\xi}.
\end{eqnarray}

Por tanto 

\begin{eqnarray}
\begin{array}{l}
\prob\left\{0 \textrm{ arribos en }Q_{1}\textrm{ y }Q_{2}\textrm{ en el intervalo }\left[\overline{\tau}_{2}\left(n-1\right),\varphi_{1}\left(n\right)\right]\right\}\\
=\prob\left\{0 \textrm{ arribos en }Q_{1}\textrm{ en el intervalo }\left[\overline{\tau}_{2}\left(n-1\right),\varphi_{1}\left(n\right)\right]\right\}\\
\times
\prob\left\{0 \textrm{ arribos en }Q_{2}\textrm{ en el intervalo }\left[\overline{\tau}_{2}\left(n-1\right),\varphi_{1}\left(n\right)\right]\right\}=e^{-\tilde{\mu}_{1}\xi}e^{-\tilde{\mu}_{2}\xi}
=e^{-\tilde{\mu}\xi}.
\end{array}
\end{eqnarray}

Para el segundo sistema, consideremos nuevamente $\overline{\tau}_{1}\left(n\right)+r_{1}$, sin p\'erdida de generalidad podemos suponer que existe $m>0$ tal que $\overline{\tau}_{3}\left(m\right)<\overline{\tau}_{1}+r_{1}<\tau_{4}\left(m\right)$, entonces

\begin{eqnarray}
\prob\left\{0 \textrm{ arribos en }Q_{3}\textrm{ en el intervalo }\left[\overline{\tau}_{3}\left(m\right),\overline{\tau}_{1}\left(n\right)+r_{1}\right]\right\}
=e^{-\tilde{\mu}_{3}\xi_{3}},
\end{eqnarray}
donde 
\begin{eqnarray}
\xi_{3}=\overline{\tau}_{1}\left(n\right)+r_{1}-\overline{\tau}_{3}\left(m\right)=
\overline{\tau}_{1}\left(n\right)-\overline{\tau}_{3}\left(m\right)+r_{1},
\end{eqnarray}

mientras que para $Q_{4}$ al igual que con $Q_{2}$ escribiremos $\tau_{4}\left(m\right)$ en t\'erminos de $\overline{\tau}_{4}\left(m-1\right)$:

$\varphi_{2}\equiv\tau_{4}\left(m\right)=\overline{\tau}_{4}\left(m-1\right)+r_{4}+\overline{\tau}_{3}\left(m\right)
-\tau_{3}\left(m\right)+r_{3}=\overline{\tau}_{4}\left(m-1\right)+\overline{\tau}_{3}\left(m\right)
-\tau_{3}\left(m\right)+\hat{r}$, adem\'as,

$\xi_{2}\equiv\varphi_{2}\left(m\right)-\overline{\tau}_{1}-r_{1}=\overline{\tau}_{4}\left(m-1\right)+\overline{\tau}_{3}\left(m\right)
-\tau_{3}\left(m\right)-\overline{\tau}_{1}\left(n\right)+\hat{r}-r_{1}$. 

Entonces


\begin{eqnarray}
\prob\left\{0 \textrm{ arribos en }Q_{4}\textrm{ en el intervalo }\left[\overline{\tau}_{1}\left(n\right)+r_{1},\varphi_{2}\left(m\right)\right]\right\}
=e^{-\tilde{\mu}_{4}\xi_{2}},
\end{eqnarray}

mientras que para $Q_{3}$ se tiene que 

\begin{eqnarray}
\prob\left\{0 \textrm{ arribos en }Q_{3}\textrm{ en el intervalo }\left[\overline{\tau}_{1}\left(n\right)+r_{1},\varphi_{2}\left(m\right)\right]\right\}
=e^{-\tilde{\mu}_{3}\xi_{2}}
\end{eqnarray}

Por tanto

\begin{eqnarray}
\prob\left\{0 \textrm{ arribos en }Q_{3}\wedge Q_{4}\textrm{ en el intervalo }\left[\overline{\tau}_{1}\left(n\right)+r_{1},\varphi_{2}\left(m\right)\right]\right\}
=e^{-\hat{\mu}\xi_{2}}
\end{eqnarray}
donde $\hat{\mu}=\tilde{\mu}_{3}+\tilde{\mu}_{4}$.

Ahora, definamos los intervalos $\mathcal{I}_{1}=\left[\overline{\tau}_{1}\left(n\right)+r_{1},\varphi_{1}\left(n\right)\right]$  y $\mathcal{I}_{2}=\left[\overline{\tau}_{1}\left(n\right)+r_{1},\varphi_{2}\left(m\right)\right]$, entonces, sea $\mathcal{I}=\mathcal{I}_{1}\cap\mathcal{I}_{2}$ el intervalo donde cada una de las colas se encuentran vac\'ias, es decir, si tomamos $T^{*}\in\mathcal{I}$, entonces  $L_{1}\left(T^{*}\right)=L_{2}\left(T^{*}\right)=L_{3}\left(T^{*}\right)=L_{4}\left(T^{*}\right)=0$.

Ahora, dado que por construcci\'on $\mathcal{I}\neq\emptyset$ y que para $T^{*}\in\mathcal{I}$ en ninguna de las colas han llegado usuarios, se tiene que no hay transferencia entre las colas, por lo tanto, el sistema 1 y el sistema 2 son condicionalmente independientes en $\mathcal{I}$, es decir

\begin{eqnarray}
\prob\left\{L_{1}\left(T^{*}\right),L_{2}\left(T^{*}\right),L_{3}\left(T^{*}\right),L_{4}\left(T^{*}\right)|T^{*}\in\mathcal{I}\right\}=\prod_{j=1}^{4}\prob\left\{L_{j}\left(T^{*}\right)\right\},
\end{eqnarray}

para $T^{*}\in\mathcal{I}$. 

%\newpage























%________________________________________________________________________
%\section{Procesos Regenerativos}
%________________________________________________________________________

%________________________________________________________________________
%\subsection*{Procesos Regenerativos Sigman, Thorisson y Wolff \cite{Sigman1}}
%________________________________________________________________________


\begin{Def}[Definici\'on Cl\'asica]
Un proceso estoc\'astico $X=\left\{X\left(t\right):t\geq0\right\}$ es llamado regenerativo is existe una variable aleatoria $R_{1}>0$ tal que
\begin{itemize}
\item[i)] $\left\{X\left(t+R_{1}\right):t\geq0\right\}$ es independiente de $\left\{\left\{X\left(t\right):t<R_{1}\right\},\right\}$
\item[ii)] $\left\{X\left(t+R_{1}\right):t\geq0\right\}$ es estoc\'asticamente equivalente a $\left\{X\left(t\right):t>0\right\}$
\end{itemize}

Llamamos a $R_{1}$ tiempo de regeneraci\'on, y decimos que $X$ se regenera en este punto.
\end{Def}

$\left\{X\left(t+R_{1}\right)\right\}$ es regenerativo con tiempo de regeneraci\'on $R_{2}$, independiente de $R_{1}$ pero con la misma distribuci\'on que $R_{1}$. Procediendo de esta manera se obtiene una secuencia de variables aleatorias independientes e id\'enticamente distribuidas $\left\{R_{n}\right\}$ llamados longitudes de ciclo. Si definimos a $Z_{k}\equiv R_{1}+R_{2}+\cdots+R_{k}$, se tiene un proceso de renovaci\'on llamado proceso de renovaci\'on encajado para $X$.


\begin{Note}
La existencia de un primer tiempo de regeneraci\'on, $R_{1}$, implica la existencia de una sucesi\'on completa de estos tiempos $R_{1},R_{2}\ldots,$ que satisfacen la propiedad deseada \cite{Sigman2}.
\end{Note}


\begin{Note} Para la cola $GI/GI/1$ los usuarios arriban con tiempos $t_{n}$ y son atendidos con tiempos de servicio $S_{n}$, los tiempos de arribo forman un proceso de renovaci\'on  con tiempos entre arribos independientes e identicamente distribuidos (\texttt{i.i.d.})$T_{n}=t_{n}-t_{n-1}$, adem\'as los tiempos de servicio son \texttt{i.i.d.} e independientes de los procesos de arribo. Por \textit{estable} se entiende que $\esp S_{n}<\esp T_{n}<\infty$.
\end{Note}
 


\begin{Def}
Para $x$ fijo y para cada $t\geq0$, sea $I_{x}\left(t\right)=1$ si $X\left(t\right)\leq x$,  $I_{x}\left(t\right)=0$ en caso contrario, y def\'inanse los tiempos promedio
\begin{eqnarray*}
\overline{X}&=&lim_{t\rightarrow\infty}\frac{1}{t}\int_{0}^{\infty}X\left(u\right)du\\
\prob\left(X_{\infty}\leq x\right)&=&lim_{t\rightarrow\infty}\frac{1}{t}\int_{0}^{\infty}I_{x}\left(u\right)du,
\end{eqnarray*}
cuando estos l\'imites existan.
\end{Def}

Como consecuencia del teorema de Renovaci\'on-Recompensa, se tiene que el primer l\'imite  existe y es igual a la constante
\begin{eqnarray*}
\overline{X}&=&\frac{\esp\left[\int_{0}^{R_{1}}X\left(t\right)dt\right]}{\esp\left[R_{1}\right]},
\end{eqnarray*}
suponiendo que ambas esperanzas son finitas.
 
\begin{Note}
Funciones de procesos regenerativos son regenerativas, es decir, si $X\left(t\right)$ es regenerativo y se define el proceso $Y\left(t\right)$ por $Y\left(t\right)=f\left(X\left(t\right)\right)$ para alguna funci\'on Borel medible $f\left(\cdot\right)$. Adem\'as $Y$ es regenerativo con los mismos tiempos de renovaci\'on que $X$. 

En general, los tiempos de renovaci\'on, $Z_{k}$ de un proceso regenerativo no requieren ser tiempos de paro con respecto a la evoluci\'on de $X\left(t\right)$.
\end{Note} 

\begin{Note}
Una funci\'on de un proceso de Markov, usualmente no ser\'a un proceso de Markov, sin embargo ser\'a regenerativo si el proceso de Markov lo es.
\end{Note}

 
\begin{Note}
Un proceso regenerativo con media de la longitud de ciclo finita es llamado positivo recurrente.
\end{Note}


\begin{Note}
\begin{itemize}
\item[a)] Si el proceso regenerativo $X$ es positivo recurrente y tiene trayectorias muestrales no negativas, entonces la ecuaci\'on anterior es v\'alida.
\item[b)] Si $X$ es positivo recurrente regenerativo, podemos construir una \'unica versi\'on estacionaria de este proceso, $X_{e}=\left\{X_{e}\left(t\right)\right\}$, donde $X_{e}$ es un proceso estoc\'astico regenerativo y estrictamente estacionario, con distribuci\'on marginal distribuida como $X_{\infty}$
\end{itemize}
\end{Note}


%__________________________________________________________________________________________
%\subsection*{Procesos Regenerativos Estacionarios - Stidham \cite{Stidham}}
%__________________________________________________________________________________________


Un proceso estoc\'astico a tiempo continuo $\left\{V\left(t\right),t\geq0\right\}$ es un proceso regenerativo si existe una sucesi\'on de variables aleatorias independientes e id\'enticamente distribuidas $\left\{X_{1},X_{2},\ldots\right\}$, sucesi\'on de renovaci\'on, tal que para cualquier conjunto de Borel $A$, 

\begin{eqnarray*}
\prob\left\{V\left(t\right)\in A|X_{1}+X_{2}+\cdots+X_{R\left(t\right)}=s,\left\{V\left(\tau\right),\tau<s\right\}\right\}=\prob\left\{V\left(t-s\right)\in A|X_{1}>t-s\right\},
\end{eqnarray*}
para todo $0\leq s\leq t$, donde $R\left(t\right)=\max\left\{X_{1}+X_{2}+\cdots+X_{j}\leq t\right\}=$n\'umero de renovaciones ({\emph{puntos de regeneraci\'on}}) que ocurren en $\left[0,t\right]$. El intervalo $\left[0,X_{1}\right)$ es llamado {\emph{primer ciclo de regeneraci\'on}} de $\left\{V\left(t \right),t\geq0\right\}$, $\left[X_{1},X_{1}+X_{2}\right)$ el {\emph{segundo ciclo de regeneraci\'on}}, y as\'i sucesivamente.

Sea $X=X_{1}$ y sea $F$ la funci\'on de distrbuci\'on de $X$


\begin{Def}
Se define el proceso estacionario, $\left\{V^{*}\left(t\right),t\geq0\right\}$, para $\left\{V\left(t\right),t\geq0\right\}$ por

\begin{eqnarray*}
\prob\left\{V\left(t\right)\in A\right\}=\frac{1}{\esp\left[X\right]}\int_{0}^{\infty}\prob\left\{V\left(t+x\right)\in A|X>x\right\}\left(1-F\left(x\right)\right)dx,
\end{eqnarray*} 
para todo $t\geq0$ y todo conjunto de Borel $A$.
\end{Def}

\begin{Def}
Una distribuci\'on se dice que es {\emph{aritm\'etica}} si todos sus puntos de incremento son m\'ultiplos de la forma $0,\lambda, 2\lambda,\ldots$ para alguna $\lambda>0$ entera.
\end{Def}


\begin{Def}
Una modificaci\'on medible de un proceso $\left\{V\left(t\right),t\geq0\right\}$, es una versi\'on de este, $\left\{V\left(t,w\right)\right\}$ conjuntamente medible para $t\geq0$ y para $w\in S$, $S$ espacio de estados para $\left\{V\left(t\right),t\geq0\right\}$.
\end{Def}

\begin{Teo}
Sea $\left\{V\left(t\right),t\geq\right\}$ un proceso regenerativo no negativo con modificaci\'on medible. Sea $\esp\left[X\right]<\infty$. Entonces el proceso estacionario dado por la ecuaci\'on anterior est\'a bien definido y tiene funci\'on de distribuci\'on independiente de $t$, adem\'as
\begin{itemize}
\item[i)] \begin{eqnarray*}
\esp\left[V^{*}\left(0\right)\right]&=&\frac{\esp\left[\int_{0}^{X}V\left(s\right)ds\right]}{\esp\left[X\right]}\end{eqnarray*}
\item[ii)] Si $\esp\left[V^{*}\left(0\right)\right]<\infty$, equivalentemente, si $\esp\left[\int_{0}^{X}V\left(s\right)ds\right]<\infty$,entonces
\begin{eqnarray*}
\frac{\int_{0}^{t}V\left(s\right)ds}{t}\rightarrow\frac{\esp\left[\int_{0}^{X}V\left(s\right)ds\right]}{\esp\left[X\right]}
\end{eqnarray*}
con probabilidad 1 y en media, cuando $t\rightarrow\infty$.
\end{itemize}
\end{Teo}

\begin{Coro}
Sea $\left\{V\left(t\right),t\geq0\right\}$ un proceso regenerativo no negativo, con modificaci\'on medible. Si $\esp <\infty$, $F$ es no-aritm\'etica, y para todo $x\geq0$, $P\left\{V\left(t\right)\leq x,C>x\right\}$ es de variaci\'on acotada como funci\'on de $t$ en cada intervalo finito $\left[0,\tau\right]$, entonces $V\left(t\right)$ converge en distribuci\'on  cuando $t\rightarrow\infty$ y $$\esp V=\frac{\esp \int_{0}^{X}V\left(s\right)ds}{\esp X}$$
Donde $V$ tiene la distribuci\'on l\'imite de $V\left(t\right)$ cuando $t\rightarrow\infty$.

\end{Coro}

Para el caso discreto se tienen resultados similares.



%______________________________________________________________________
%\section{Procesos de Renovaci\'on}
%______________________________________________________________________

\begin{Def}\label{Def.Tn}
Sean $0\leq T_{1}\leq T_{2}\leq \ldots$ son tiempos aleatorios infinitos en los cuales ocurren ciertos eventos. El n\'umero de tiempos $T_{n}$ en el intervalo $\left[0,t\right)$ es

\begin{eqnarray}
N\left(t\right)=\sum_{n=1}^{\infty}\indora\left(T_{n}\leq t\right),
\end{eqnarray}
para $t\geq0$.
\end{Def}

Si se consideran los puntos $T_{n}$ como elementos de $\rea_{+}$, y $N\left(t\right)$ es el n\'umero de puntos en $\rea$. El proceso denotado por $\left\{N\left(t\right):t\geq0\right\}$, denotado por $N\left(t\right)$, es un proceso puntual en $\rea_{+}$. Los $T_{n}$ son los tiempos de ocurrencia, el proceso puntual $N\left(t\right)$ es simple si su n\'umero de ocurrencias son distintas: $0<T_{1}<T_{2}<\ldots$ casi seguramente.

\begin{Def}
Un proceso puntual $N\left(t\right)$ es un proceso de renovaci\'on si los tiempos de interocurrencia $\xi_{n}=T_{n}-T_{n-1}$, para $n\geq1$, son independientes e identicamente distribuidos con distribuci\'on $F$, donde $F\left(0\right)=0$ y $T_{0}=0$. Los $T_{n}$ son llamados tiempos de renovaci\'on, referente a la independencia o renovaci\'on de la informaci\'on estoc\'astica en estos tiempos. Los $\xi_{n}$ son los tiempos de inter-renovaci\'on, y $N\left(t\right)$ es el n\'umero de renovaciones en el intervalo $\left[0,t\right)$
\end{Def}


\begin{Note}
Para definir un proceso de renovaci\'on para cualquier contexto, solamente hay que especificar una distribuci\'on $F$, con $F\left(0\right)=0$, para los tiempos de inter-renovaci\'on. La funci\'on $F$ en turno degune las otra variables aleatorias. De manera formal, existe un espacio de probabilidad y una sucesi\'on de variables aleatorias $\xi_{1},\xi_{2},\ldots$ definidas en este con distribuci\'on $F$. Entonces las otras cantidades son $T_{n}=\sum_{k=1}^{n}\xi_{k}$ y $N\left(t\right)=\sum_{n=1}^{\infty}\indora\left(T_{n}\leq t\right)$, donde $T_{n}\rightarrow\infty$ casi seguramente por la Ley Fuerte de los Grandes Números.
\end{Note}

%___________________________________________________________________________________________
%
%\subsection*{Teorema Principal de Renovaci\'on}
%___________________________________________________________________________________________
%

\begin{Note} Una funci\'on $h:\rea_{+}\rightarrow\rea$ es Directamente Riemann Integrable en los siguientes casos:
\begin{itemize}
\item[a)] $h\left(t\right)\geq0$ es decreciente y Riemann Integrable.
\item[b)] $h$ es continua excepto posiblemente en un conjunto de Lebesgue de medida 0, y $|h\left(t\right)|\leq b\left(t\right)$, donde $b$ es DRI.
\end{itemize}
\end{Note}

\begin{Teo}[Teorema Principal de Renovaci\'on]
Si $F$ es no aritm\'etica y $h\left(t\right)$ es Directamente Riemann Integrable (DRI), entonces

\begin{eqnarray*}
lim_{t\rightarrow\infty}U\star h=\frac{1}{\mu}\int_{\rea_{+}}h\left(s\right)ds.
\end{eqnarray*}
\end{Teo}

\begin{Prop}
Cualquier funci\'on $H\left(t\right)$ acotada en intervalos finitos y que es 0 para $t<0$ puede expresarse como
\begin{eqnarray*}
H\left(t\right)=U\star h\left(t\right)\textrm{,  donde }h\left(t\right)=H\left(t\right)-F\star H\left(t\right)
\end{eqnarray*}
\end{Prop}

\begin{Def}
Un proceso estoc\'astico $X\left(t\right)$ es crudamente regenerativo en un tiempo aleatorio positivo $T$ si
\begin{eqnarray*}
\esp\left[X\left(T+t\right)|T\right]=\esp\left[X\left(t\right)\right]\textrm{, para }t\geq0,\end{eqnarray*}
y con las esperanzas anteriores finitas.
\end{Def}

\begin{Prop}
Sup\'ongase que $X\left(t\right)$ es un proceso crudamente regenerativo en $T$, que tiene distribuci\'on $F$. Si $\esp\left[X\left(t\right)\right]$ es acotado en intervalos finitos, entonces
\begin{eqnarray*}
\esp\left[X\left(t\right)\right]=U\star h\left(t\right)\textrm{,  donde }h\left(t\right)=\esp\left[X\left(t\right)\indora\left(T>t\right)\right].
\end{eqnarray*}
\end{Prop}

\begin{Teo}[Regeneraci\'on Cruda]
Sup\'ongase que $X\left(t\right)$ es un proceso con valores positivo crudamente regenerativo en $T$, y def\'inase $M=\sup\left\{|X\left(t\right)|:t\leq T\right\}$. Si $T$ es no aritm\'etico y $M$ y $MT$ tienen media finita, entonces
\begin{eqnarray*}
lim_{t\rightarrow\infty}\esp\left[X\left(t\right)\right]=\frac{1}{\mu}\int_{\rea_{+}}h\left(s\right)ds,
\end{eqnarray*}
donde $h\left(t\right)=\esp\left[X\left(t\right)\indora\left(T>t\right)\right]$.
\end{Teo}

%___________________________________________________________________________________________
%
%\subsection*{Propiedades de los Procesos de Renovaci\'on}
%___________________________________________________________________________________________
%

Los tiempos $T_{n}$ est\'an relacionados con los conteos de $N\left(t\right)$ por

\begin{eqnarray*}
\left\{N\left(t\right)\geq n\right\}&=&\left\{T_{n}\leq t\right\}\\
T_{N\left(t\right)}\leq &t&<T_{N\left(t\right)+1},
\end{eqnarray*}

adem\'as $N\left(T_{n}\right)=n$, y 

\begin{eqnarray*}
N\left(t\right)=\max\left\{n:T_{n}\leq t\right\}=\min\left\{n:T_{n+1}>t\right\}
\end{eqnarray*}

Por propiedades de la convoluci\'on se sabe que

\begin{eqnarray*}
P\left\{T_{n}\leq t\right\}=F^{n\star}\left(t\right)
\end{eqnarray*}
que es la $n$-\'esima convoluci\'on de $F$. Entonces 

\begin{eqnarray*}
\left\{N\left(t\right)\geq n\right\}&=&\left\{T_{n}\leq t\right\}\\
P\left\{N\left(t\right)\leq n\right\}&=&1-F^{\left(n+1\right)\star}\left(t\right)
\end{eqnarray*}

Adem\'as usando el hecho de que $\esp\left[N\left(t\right)\right]=\sum_{n=1}^{\infty}P\left\{N\left(t\right)\geq n\right\}$
se tiene que

\begin{eqnarray*}
\esp\left[N\left(t\right)\right]=\sum_{n=1}^{\infty}F^{n\star}\left(t\right)
\end{eqnarray*}

\begin{Prop}
Para cada $t\geq0$, la funci\'on generadora de momentos $\esp\left[e^{\alpha N\left(t\right)}\right]$ existe para alguna $\alpha$ en una vecindad del 0, y de aqu\'i que $\esp\left[N\left(t\right)^{m}\right]<\infty$, para $m\geq1$.
\end{Prop}


\begin{Note}
Si el primer tiempo de renovaci\'on $\xi_{1}$ no tiene la misma distribuci\'on que el resto de las $\xi_{n}$, para $n\geq2$, a $N\left(t\right)$ se le llama Proceso de Renovaci\'on retardado, donde si $\xi$ tiene distribuci\'on $G$, entonces el tiempo $T_{n}$ de la $n$-\'esima renovaci\'on tiene distribuci\'on $G\star F^{\left(n-1\right)\star}\left(t\right)$
\end{Note}


\begin{Teo}
Para una constante $\mu\leq\infty$ ( o variable aleatoria), las siguientes expresiones son equivalentes:

\begin{eqnarray}
lim_{n\rightarrow\infty}n^{-1}T_{n}&=&\mu,\textrm{ c.s.}\\
lim_{t\rightarrow\infty}t^{-1}N\left(t\right)&=&1/\mu,\textrm{ c.s.}
\end{eqnarray}
\end{Teo}


Es decir, $T_{n}$ satisface la Ley Fuerte de los Grandes N\'umeros s\'i y s\'olo s\'i $N\left/t\right)$ la cumple.


\begin{Coro}[Ley Fuerte de los Grandes N\'umeros para Procesos de Renovaci\'on]
Si $N\left(t\right)$ es un proceso de renovaci\'on cuyos tiempos de inter-renovaci\'on tienen media $\mu\leq\infty$, entonces
\begin{eqnarray}
t^{-1}N\left(t\right)\rightarrow 1/\mu,\textrm{ c.s. cuando }t\rightarrow\infty.
\end{eqnarray}

\end{Coro}


Considerar el proceso estoc\'astico de valores reales $\left\{Z\left(t\right):t\geq0\right\}$ en el mismo espacio de probabilidad que $N\left(t\right)$

\begin{Def}
Para el proceso $\left\{Z\left(t\right):t\geq0\right\}$ se define la fluctuaci\'on m\'axima de $Z\left(t\right)$ en el intervalo $\left(T_{n-1},T_{n}\right]$:
\begin{eqnarray*}
M_{n}=\sup_{T_{n-1}<t\leq T_{n}}|Z\left(t\right)-Z\left(T_{n-1}\right)|
\end{eqnarray*}
\end{Def}

\begin{Teo}
Sup\'ongase que $n^{-1}T_{n}\rightarrow\mu$ c.s. cuando $n\rightarrow\infty$, donde $\mu\leq\infty$ es una constante o variable aleatoria. Sea $a$ una constante o variable aleatoria que puede ser infinita cuando $\mu$ es finita, y considere las expresiones l\'imite:
\begin{eqnarray}
lim_{n\rightarrow\infty}n^{-1}Z\left(T_{n}\right)&=&a,\textrm{ c.s.}\\
lim_{t\rightarrow\infty}t^{-1}Z\left(t\right)&=&a/\mu,\textrm{ c.s.}
\end{eqnarray}
La segunda expresi\'on implica la primera. Conversamente, la primera implica la segunda si el proceso $Z\left(t\right)$ es creciente, o si $lim_{n\rightarrow\infty}n^{-1}M_{n}=0$ c.s.
\end{Teo}

\begin{Coro}
Si $N\left(t\right)$ es un proceso de renovaci\'on, y $\left(Z\left(T_{n}\right)-Z\left(T_{n-1}\right),M_{n}\right)$, para $n\geq1$, son variables aleatorias independientes e id\'enticamente distribuidas con media finita, entonces,
\begin{eqnarray}
lim_{t\rightarrow\infty}t^{-1}Z\left(t\right)\rightarrow\frac{\esp\left[Z\left(T_{1}\right)-Z\left(T_{0}\right)\right]}{\esp\left[T_{1}\right]},\textrm{ c.s. cuando  }t\rightarrow\infty.
\end{eqnarray}
\end{Coro}



%___________________________________________________________________________________________
%
%\subsection{Propiedades de los Procesos de Renovaci\'on}
%___________________________________________________________________________________________
%

Los tiempos $T_{n}$ est\'an relacionados con los conteos de $N\left(t\right)$ por

\begin{eqnarray*}
\left\{N\left(t\right)\geq n\right\}&=&\left\{T_{n}\leq t\right\}\\
T_{N\left(t\right)}\leq &t&<T_{N\left(t\right)+1},
\end{eqnarray*}

adem\'as $N\left(T_{n}\right)=n$, y 

\begin{eqnarray*}
N\left(t\right)=\max\left\{n:T_{n}\leq t\right\}=\min\left\{n:T_{n+1}>t\right\}
\end{eqnarray*}

Por propiedades de la convoluci\'on se sabe que

\begin{eqnarray*}
P\left\{T_{n}\leq t\right\}=F^{n\star}\left(t\right)
\end{eqnarray*}
que es la $n$-\'esima convoluci\'on de $F$. Entonces 

\begin{eqnarray*}
\left\{N\left(t\right)\geq n\right\}&=&\left\{T_{n}\leq t\right\}\\
P\left\{N\left(t\right)\leq n\right\}&=&1-F^{\left(n+1\right)\star}\left(t\right)
\end{eqnarray*}

Adem\'as usando el hecho de que $\esp\left[N\left(t\right)\right]=\sum_{n=1}^{\infty}P\left\{N\left(t\right)\geq n\right\}$
se tiene que

\begin{eqnarray*}
\esp\left[N\left(t\right)\right]=\sum_{n=1}^{\infty}F^{n\star}\left(t\right)
\end{eqnarray*}

\begin{Prop}
Para cada $t\geq0$, la funci\'on generadora de momentos $\esp\left[e^{\alpha N\left(t\right)}\right]$ existe para alguna $\alpha$ en una vecindad del 0, y de aqu\'i que $\esp\left[N\left(t\right)^{m}\right]<\infty$, para $m\geq1$.
\end{Prop}


\begin{Note}
Si el primer tiempo de renovaci\'on $\xi_{1}$ no tiene la misma distribuci\'on que el resto de las $\xi_{n}$, para $n\geq2$, a $N\left(t\right)$ se le llama Proceso de Renovaci\'on retardado, donde si $\xi$ tiene distribuci\'on $G$, entonces el tiempo $T_{n}$ de la $n$-\'esima renovaci\'on tiene distribuci\'on $G\star F^{\left(n-1\right)\star}\left(t\right)$
\end{Note}


\begin{Teo}
Para una constante $\mu\leq\infty$ ( o variable aleatoria), las siguientes expresiones son equivalentes:

\begin{eqnarray}
lim_{n\rightarrow\infty}n^{-1}T_{n}&=&\mu,\textrm{ c.s.}\\
lim_{t\rightarrow\infty}t^{-1}N\left(t\right)&=&1/\mu,\textrm{ c.s.}
\end{eqnarray}
\end{Teo}


Es decir, $T_{n}$ satisface la Ley Fuerte de los Grandes N\'umeros s\'i y s\'olo s\'i $N\left/t\right)$ la cumple.


\begin{Coro}[Ley Fuerte de los Grandes N\'umeros para Procesos de Renovaci\'on]
Si $N\left(t\right)$ es un proceso de renovaci\'on cuyos tiempos de inter-renovaci\'on tienen media $\mu\leq\infty$, entonces
\begin{eqnarray}
t^{-1}N\left(t\right)\rightarrow 1/\mu,\textrm{ c.s. cuando }t\rightarrow\infty.
\end{eqnarray}

\end{Coro}


Considerar el proceso estoc\'astico de valores reales $\left\{Z\left(t\right):t\geq0\right\}$ en el mismo espacio de probabilidad que $N\left(t\right)$

\begin{Def}
Para el proceso $\left\{Z\left(t\right):t\geq0\right\}$ se define la fluctuaci\'on m\'axima de $Z\left(t\right)$ en el intervalo $\left(T_{n-1},T_{n}\right]$:
\begin{eqnarray*}
M_{n}=\sup_{T_{n-1}<t\leq T_{n}}|Z\left(t\right)-Z\left(T_{n-1}\right)|
\end{eqnarray*}
\end{Def}

\begin{Teo}
Sup\'ongase que $n^{-1}T_{n}\rightarrow\mu$ c.s. cuando $n\rightarrow\infty$, donde $\mu\leq\infty$ es una constante o variable aleatoria. Sea $a$ una constante o variable aleatoria que puede ser infinita cuando $\mu$ es finita, y considere las expresiones l\'imite:
\begin{eqnarray}
lim_{n\rightarrow\infty}n^{-1}Z\left(T_{n}\right)&=&a,\textrm{ c.s.}\\
lim_{t\rightarrow\infty}t^{-1}Z\left(t\right)&=&a/\mu,\textrm{ c.s.}
\end{eqnarray}
La segunda expresi\'on implica la primera. Conversamente, la primera implica la segunda si el proceso $Z\left(t\right)$ es creciente, o si $lim_{n\rightarrow\infty}n^{-1}M_{n}=0$ c.s.
\end{Teo}

\begin{Coro}
Si $N\left(t\right)$ es un proceso de renovaci\'on, y $\left(Z\left(T_{n}\right)-Z\left(T_{n-1}\right),M_{n}\right)$, para $n\geq1$, son variables aleatorias independientes e id\'enticamente distribuidas con media finita, entonces,
\begin{eqnarray}
lim_{t\rightarrow\infty}t^{-1}Z\left(t\right)\rightarrow\frac{\esp\left[Z\left(T_{1}\right)-Z\left(T_{0}\right)\right]}{\esp\left[T_{1}\right]},\textrm{ c.s. cuando  }t\rightarrow\infty.
\end{eqnarray}
\end{Coro}


%___________________________________________________________________________________________
%
%\subsection{Propiedades de los Procesos de Renovaci\'on}
%___________________________________________________________________________________________
%

Los tiempos $T_{n}$ est\'an relacionados con los conteos de $N\left(t\right)$ por

\begin{eqnarray*}
\left\{N\left(t\right)\geq n\right\}&=&\left\{T_{n}\leq t\right\}\\
T_{N\left(t\right)}\leq &t&<T_{N\left(t\right)+1},
\end{eqnarray*}

adem\'as $N\left(T_{n}\right)=n$, y 

\begin{eqnarray*}
N\left(t\right)=\max\left\{n:T_{n}\leq t\right\}=\min\left\{n:T_{n+1}>t\right\}
\end{eqnarray*}

Por propiedades de la convoluci\'on se sabe que

\begin{eqnarray*}
P\left\{T_{n}\leq t\right\}=F^{n\star}\left(t\right)
\end{eqnarray*}
que es la $n$-\'esima convoluci\'on de $F$. Entonces 

\begin{eqnarray*}
\left\{N\left(t\right)\geq n\right\}&=&\left\{T_{n}\leq t\right\}\\
P\left\{N\left(t\right)\leq n\right\}&=&1-F^{\left(n+1\right)\star}\left(t\right)
\end{eqnarray*}

Adem\'as usando el hecho de que $\esp\left[N\left(t\right)\right]=\sum_{n=1}^{\infty}P\left\{N\left(t\right)\geq n\right\}$
se tiene que

\begin{eqnarray*}
\esp\left[N\left(t\right)\right]=\sum_{n=1}^{\infty}F^{n\star}\left(t\right)
\end{eqnarray*}

\begin{Prop}
Para cada $t\geq0$, la funci\'on generadora de momentos $\esp\left[e^{\alpha N\left(t\right)}\right]$ existe para alguna $\alpha$ en una vecindad del 0, y de aqu\'i que $\esp\left[N\left(t\right)^{m}\right]<\infty$, para $m\geq1$.
\end{Prop}


\begin{Note}
Si el primer tiempo de renovaci\'on $\xi_{1}$ no tiene la misma distribuci\'on que el resto de las $\xi_{n}$, para $n\geq2$, a $N\left(t\right)$ se le llama Proceso de Renovaci\'on retardado, donde si $\xi$ tiene distribuci\'on $G$, entonces el tiempo $T_{n}$ de la $n$-\'esima renovaci\'on tiene distribuci\'on $G\star F^{\left(n-1\right)\star}\left(t\right)$
\end{Note}


\begin{Teo}
Para una constante $\mu\leq\infty$ ( o variable aleatoria), las siguientes expresiones son equivalentes:

\begin{eqnarray}
lim_{n\rightarrow\infty}n^{-1}T_{n}&=&\mu,\textrm{ c.s.}\\
lim_{t\rightarrow\infty}t^{-1}N\left(t\right)&=&1/\mu,\textrm{ c.s.}
\end{eqnarray}
\end{Teo}


Es decir, $T_{n}$ satisface la Ley Fuerte de los Grandes N\'umeros s\'i y s\'olo s\'i $N\left/t\right)$ la cumple.


\begin{Coro}[Ley Fuerte de los Grandes N\'umeros para Procesos de Renovaci\'on]
Si $N\left(t\right)$ es un proceso de renovaci\'on cuyos tiempos de inter-renovaci\'on tienen media $\mu\leq\infty$, entonces
\begin{eqnarray}
t^{-1}N\left(t\right)\rightarrow 1/\mu,\textrm{ c.s. cuando }t\rightarrow\infty.
\end{eqnarray}

\end{Coro}


Considerar el proceso estoc\'astico de valores reales $\left\{Z\left(t\right):t\geq0\right\}$ en el mismo espacio de probabilidad que $N\left(t\right)$

\begin{Def}
Para el proceso $\left\{Z\left(t\right):t\geq0\right\}$ se define la fluctuaci\'on m\'axima de $Z\left(t\right)$ en el intervalo $\left(T_{n-1},T_{n}\right]$:
\begin{eqnarray*}
M_{n}=\sup_{T_{n-1}<t\leq T_{n}}|Z\left(t\right)-Z\left(T_{n-1}\right)|
\end{eqnarray*}
\end{Def}

\begin{Teo}
Sup\'ongase que $n^{-1}T_{n}\rightarrow\mu$ c.s. cuando $n\rightarrow\infty$, donde $\mu\leq\infty$ es una constante o variable aleatoria. Sea $a$ una constante o variable aleatoria que puede ser infinita cuando $\mu$ es finita, y considere las expresiones l\'imite:
\begin{eqnarray}
lim_{n\rightarrow\infty}n^{-1}Z\left(T_{n}\right)&=&a,\textrm{ c.s.}\\
lim_{t\rightarrow\infty}t^{-1}Z\left(t\right)&=&a/\mu,\textrm{ c.s.}
\end{eqnarray}
La segunda expresi\'on implica la primera. Conversamente, la primera implica la segunda si el proceso $Z\left(t\right)$ es creciente, o si $lim_{n\rightarrow\infty}n^{-1}M_{n}=0$ c.s.
\end{Teo}

\begin{Coro}
Si $N\left(t\right)$ es un proceso de renovaci\'on, y $\left(Z\left(T_{n}\right)-Z\left(T_{n-1}\right),M_{n}\right)$, para $n\geq1$, son variables aleatorias independientes e id\'enticamente distribuidas con media finita, entonces,
\begin{eqnarray}
lim_{t\rightarrow\infty}t^{-1}Z\left(t\right)\rightarrow\frac{\esp\left[Z\left(T_{1}\right)-Z\left(T_{0}\right)\right]}{\esp\left[T_{1}\right]},\textrm{ c.s. cuando  }t\rightarrow\infty.
\end{eqnarray}
\end{Coro}

%___________________________________________________________________________________________
%
%\subsection{Propiedades de los Procesos de Renovaci\'on}
%___________________________________________________________________________________________
%

Los tiempos $T_{n}$ est\'an relacionados con los conteos de $N\left(t\right)$ por

\begin{eqnarray*}
\left\{N\left(t\right)\geq n\right\}&=&\left\{T_{n}\leq t\right\}\\
T_{N\left(t\right)}\leq &t&<T_{N\left(t\right)+1},
\end{eqnarray*}

adem\'as $N\left(T_{n}\right)=n$, y 

\begin{eqnarray*}
N\left(t\right)=\max\left\{n:T_{n}\leq t\right\}=\min\left\{n:T_{n+1}>t\right\}
\end{eqnarray*}

Por propiedades de la convoluci\'on se sabe que

\begin{eqnarray*}
P\left\{T_{n}\leq t\right\}=F^{n\star}\left(t\right)
\end{eqnarray*}
que es la $n$-\'esima convoluci\'on de $F$. Entonces 

\begin{eqnarray*}
\left\{N\left(t\right)\geq n\right\}&=&\left\{T_{n}\leq t\right\}\\
P\left\{N\left(t\right)\leq n\right\}&=&1-F^{\left(n+1\right)\star}\left(t\right)
\end{eqnarray*}

Adem\'as usando el hecho de que $\esp\left[N\left(t\right)\right]=\sum_{n=1}^{\infty}P\left\{N\left(t\right)\geq n\right\}$
se tiene que

\begin{eqnarray*}
\esp\left[N\left(t\right)\right]=\sum_{n=1}^{\infty}F^{n\star}\left(t\right)
\end{eqnarray*}

\begin{Prop}
Para cada $t\geq0$, la funci\'on generadora de momentos $\esp\left[e^{\alpha N\left(t\right)}\right]$ existe para alguna $\alpha$ en una vecindad del 0, y de aqu\'i que $\esp\left[N\left(t\right)^{m}\right]<\infty$, para $m\geq1$.
\end{Prop}


\begin{Note}
Si el primer tiempo de renovaci\'on $\xi_{1}$ no tiene la misma distribuci\'on que el resto de las $\xi_{n}$, para $n\geq2$, a $N\left(t\right)$ se le llama Proceso de Renovaci\'on retardado, donde si $\xi$ tiene distribuci\'on $G$, entonces el tiempo $T_{n}$ de la $n$-\'esima renovaci\'on tiene distribuci\'on $G\star F^{\left(n-1\right)\star}\left(t\right)$
\end{Note}


\begin{Teo}
Para una constante $\mu\leq\infty$ ( o variable aleatoria), las siguientes expresiones son equivalentes:

\begin{eqnarray}
lim_{n\rightarrow\infty}n^{-1}T_{n}&=&\mu,\textrm{ c.s.}\\
lim_{t\rightarrow\infty}t^{-1}N\left(t\right)&=&1/\mu,\textrm{ c.s.}
\end{eqnarray}
\end{Teo}


Es decir, $T_{n}$ satisface la Ley Fuerte de los Grandes N\'umeros s\'i y s\'olo s\'i $N\left/t\right)$ la cumple.


\begin{Coro}[Ley Fuerte de los Grandes N\'umeros para Procesos de Renovaci\'on]
Si $N\left(t\right)$ es un proceso de renovaci\'on cuyos tiempos de inter-renovaci\'on tienen media $\mu\leq\infty$, entonces
\begin{eqnarray}
t^{-1}N\left(t\right)\rightarrow 1/\mu,\textrm{ c.s. cuando }t\rightarrow\infty.
\end{eqnarray}

\end{Coro}


Considerar el proceso estoc\'astico de valores reales $\left\{Z\left(t\right):t\geq0\right\}$ en el mismo espacio de probabilidad que $N\left(t\right)$

\begin{Def}
Para el proceso $\left\{Z\left(t\right):t\geq0\right\}$ se define la fluctuaci\'on m\'axima de $Z\left(t\right)$ en el intervalo $\left(T_{n-1},T_{n}\right]$:
\begin{eqnarray*}
M_{n}=\sup_{T_{n-1}<t\leq T_{n}}|Z\left(t\right)-Z\left(T_{n-1}\right)|
\end{eqnarray*}
\end{Def}

\begin{Teo}
Sup\'ongase que $n^{-1}T_{n}\rightarrow\mu$ c.s. cuando $n\rightarrow\infty$, donde $\mu\leq\infty$ es una constante o variable aleatoria. Sea $a$ una constante o variable aleatoria que puede ser infinita cuando $\mu$ es finita, y considere las expresiones l\'imite:
\begin{eqnarray}
lim_{n\rightarrow\infty}n^{-1}Z\left(T_{n}\right)&=&a,\textrm{ c.s.}\\
lim_{t\rightarrow\infty}t^{-1}Z\left(t\right)&=&a/\mu,\textrm{ c.s.}
\end{eqnarray}
La segunda expresi\'on implica la primera. Conversamente, la primera implica la segunda si el proceso $Z\left(t\right)$ es creciente, o si $lim_{n\rightarrow\infty}n^{-1}M_{n}=0$ c.s.
\end{Teo}

\begin{Coro}
Si $N\left(t\right)$ es un proceso de renovaci\'on, y $\left(Z\left(T_{n}\right)-Z\left(T_{n-1}\right),M_{n}\right)$, para $n\geq1$, son variables aleatorias independientes e id\'enticamente distribuidas con media finita, entonces,
\begin{eqnarray}
lim_{t\rightarrow\infty}t^{-1}Z\left(t\right)\rightarrow\frac{\esp\left[Z\left(T_{1}\right)-Z\left(T_{0}\right)\right]}{\esp\left[T_{1}\right]},\textrm{ c.s. cuando  }t\rightarrow\infty.
\end{eqnarray}
\end{Coro}
%___________________________________________________________________________________________
%
%\subsection{Propiedades de los Procesos de Renovaci\'on}
%___________________________________________________________________________________________
%

Los tiempos $T_{n}$ est\'an relacionados con los conteos de $N\left(t\right)$ por

\begin{eqnarray*}
\left\{N\left(t\right)\geq n\right\}&=&\left\{T_{n}\leq t\right\}\\
T_{N\left(t\right)}\leq &t&<T_{N\left(t\right)+1},
\end{eqnarray*}

adem\'as $N\left(T_{n}\right)=n$, y 

\begin{eqnarray*}
N\left(t\right)=\max\left\{n:T_{n}\leq t\right\}=\min\left\{n:T_{n+1}>t\right\}
\end{eqnarray*}

Por propiedades de la convoluci\'on se sabe que

\begin{eqnarray*}
P\left\{T_{n}\leq t\right\}=F^{n\star}\left(t\right)
\end{eqnarray*}
que es la $n$-\'esima convoluci\'on de $F$. Entonces 

\begin{eqnarray*}
\left\{N\left(t\right)\geq n\right\}&=&\left\{T_{n}\leq t\right\}\\
P\left\{N\left(t\right)\leq n\right\}&=&1-F^{\left(n+1\right)\star}\left(t\right)
\end{eqnarray*}

Adem\'as usando el hecho de que $\esp\left[N\left(t\right)\right]=\sum_{n=1}^{\infty}P\left\{N\left(t\right)\geq n\right\}$
se tiene que

\begin{eqnarray*}
\esp\left[N\left(t\right)\right]=\sum_{n=1}^{\infty}F^{n\star}\left(t\right)
\end{eqnarray*}

\begin{Prop}
Para cada $t\geq0$, la funci\'on generadora de momentos $\esp\left[e^{\alpha N\left(t\right)}\right]$ existe para alguna $\alpha$ en una vecindad del 0, y de aqu\'i que $\esp\left[N\left(t\right)^{m}\right]<\infty$, para $m\geq1$.
\end{Prop}


\begin{Note}
Si el primer tiempo de renovaci\'on $\xi_{1}$ no tiene la misma distribuci\'on que el resto de las $\xi_{n}$, para $n\geq2$, a $N\left(t\right)$ se le llama Proceso de Renovaci\'on retardado, donde si $\xi$ tiene distribuci\'on $G$, entonces el tiempo $T_{n}$ de la $n$-\'esima renovaci\'on tiene distribuci\'on $G\star F^{\left(n-1\right)\star}\left(t\right)$
\end{Note}


\begin{Teo}
Para una constante $\mu\leq\infty$ ( o variable aleatoria), las siguientes expresiones son equivalentes:

\begin{eqnarray}
lim_{n\rightarrow\infty}n^{-1}T_{n}&=&\mu,\textrm{ c.s.}\\
lim_{t\rightarrow\infty}t^{-1}N\left(t\right)&=&1/\mu,\textrm{ c.s.}
\end{eqnarray}
\end{Teo}


Es decir, $T_{n}$ satisface la Ley Fuerte de los Grandes N\'umeros s\'i y s\'olo s\'i $N\left/t\right)$ la cumple.


\begin{Coro}[Ley Fuerte de los Grandes N\'umeros para Procesos de Renovaci\'on]
Si $N\left(t\right)$ es un proceso de renovaci\'on cuyos tiempos de inter-renovaci\'on tienen media $\mu\leq\infty$, entonces
\begin{eqnarray}
t^{-1}N\left(t\right)\rightarrow 1/\mu,\textrm{ c.s. cuando }t\rightarrow\infty.
\end{eqnarray}

\end{Coro}


Considerar el proceso estoc\'astico de valores reales $\left\{Z\left(t\right):t\geq0\right\}$ en el mismo espacio de probabilidad que $N\left(t\right)$

\begin{Def}
Para el proceso $\left\{Z\left(t\right):t\geq0\right\}$ se define la fluctuaci\'on m\'axima de $Z\left(t\right)$ en el intervalo $\left(T_{n-1},T_{n}\right]$:
\begin{eqnarray*}
M_{n}=\sup_{T_{n-1}<t\leq T_{n}}|Z\left(t\right)-Z\left(T_{n-1}\right)|
\end{eqnarray*}
\end{Def}

\begin{Teo}
Sup\'ongase que $n^{-1}T_{n}\rightarrow\mu$ c.s. cuando $n\rightarrow\infty$, donde $\mu\leq\infty$ es una constante o variable aleatoria. Sea $a$ una constante o variable aleatoria que puede ser infinita cuando $\mu$ es finita, y considere las expresiones l\'imite:
\begin{eqnarray}
lim_{n\rightarrow\infty}n^{-1}Z\left(T_{n}\right)&=&a,\textrm{ c.s.}\\
lim_{t\rightarrow\infty}t^{-1}Z\left(t\right)&=&a/\mu,\textrm{ c.s.}
\end{eqnarray}
La segunda expresi\'on implica la primera. Conversamente, la primera implica la segunda si el proceso $Z\left(t\right)$ es creciente, o si $lim_{n\rightarrow\infty}n^{-1}M_{n}=0$ c.s.
\end{Teo}

\begin{Coro}
Si $N\left(t\right)$ es un proceso de renovaci\'on, y $\left(Z\left(T_{n}\right)-Z\left(T_{n-1}\right),M_{n}\right)$, para $n\geq1$, son variables aleatorias independientes e id\'enticamente distribuidas con media finita, entonces,
\begin{eqnarray}
lim_{t\rightarrow\infty}t^{-1}Z\left(t\right)\rightarrow\frac{\esp\left[Z\left(T_{1}\right)-Z\left(T_{0}\right)\right]}{\esp\left[T_{1}\right]},\textrm{ c.s. cuando  }t\rightarrow\infty.
\end{eqnarray}
\end{Coro}


%___________________________________________________________________________________________
%
%\subsection*{Funci\'on de Renovaci\'on}
%___________________________________________________________________________________________
%


\begin{Def}
Sea $h\left(t\right)$ funci\'on de valores reales en $\rea$ acotada en intervalos finitos e igual a cero para $t<0$ La ecuaci\'on de renovaci\'on para $h\left(t\right)$ y la distribuci\'on $F$ es

\begin{eqnarray}\label{Ec.Renovacion}
H\left(t\right)=h\left(t\right)+\int_{\left[0,t\right]}H\left(t-s\right)dF\left(s\right)\textrm{,    }t\geq0,
\end{eqnarray}
donde $H\left(t\right)$ es una funci\'on de valores reales. Esto es $H=h+F\star H$. Decimos que $H\left(t\right)$ es soluci\'on de esta ecuaci\'on si satisface la ecuaci\'on, y es acotada en intervalos finitos e iguales a cero para $t<0$.
\end{Def}

\begin{Prop}
La funci\'on $U\star h\left(t\right)$ es la \'unica soluci\'on de la ecuaci\'on de renovaci\'on (\ref{Ec.Renovacion}).
\end{Prop}

\begin{Teo}[Teorema Renovaci\'on Elemental]
\begin{eqnarray*}
t^{-1}U\left(t\right)\rightarrow 1/\mu\textrm{,    cuando }t\rightarrow\infty.
\end{eqnarray*}
\end{Teo}

%___________________________________________________________________________________________
%
%\subsection{Funci\'on de Renovaci\'on}
%___________________________________________________________________________________________
%


Sup\'ongase que $N\left(t\right)$ es un proceso de renovaci\'on con distribuci\'on $F$ con media finita $\mu$.

\begin{Def}
La funci\'on de renovaci\'on asociada con la distribuci\'on $F$, del proceso $N\left(t\right)$, es
\begin{eqnarray*}
U\left(t\right)=\sum_{n=1}^{\infty}F^{n\star}\left(t\right),\textrm{   }t\geq0,
\end{eqnarray*}
donde $F^{0\star}\left(t\right)=\indora\left(t\geq0\right)$.
\end{Def}


\begin{Prop}
Sup\'ongase que la distribuci\'on de inter-renovaci\'on $F$ tiene densidad $f$. Entonces $U\left(t\right)$ tambi\'en tiene densidad, para $t>0$, y es $U^{'}\left(t\right)=\sum_{n=0}^{\infty}f^{n\star}\left(t\right)$. Adem\'as
\begin{eqnarray*}
\prob\left\{N\left(t\right)>N\left(t-\right)\right\}=0\textrm{,   }t\geq0.
\end{eqnarray*}
\end{Prop}

\begin{Def}
La Transformada de Laplace-Stieljes de $F$ est\'a dada por

\begin{eqnarray*}
\hat{F}\left(\alpha\right)=\int_{\rea_{+}}e^{-\alpha t}dF\left(t\right)\textrm{,  }\alpha\geq0.
\end{eqnarray*}
\end{Def}

Entonces

\begin{eqnarray*}
\hat{U}\left(\alpha\right)=\sum_{n=0}^{\infty}\hat{F^{n\star}}\left(\alpha\right)=\sum_{n=0}^{\infty}\hat{F}\left(\alpha\right)^{n}=\frac{1}{1-\hat{F}\left(\alpha\right)}.
\end{eqnarray*}


\begin{Prop}
La Transformada de Laplace $\hat{U}\left(\alpha\right)$ y $\hat{F}\left(\alpha\right)$ determina una a la otra de manera \'unica por la relaci\'on $\hat{U}\left(\alpha\right)=\frac{1}{1-\hat{F}\left(\alpha\right)}$.
\end{Prop}


\begin{Note}
Un proceso de renovaci\'on $N\left(t\right)$ cuyos tiempos de inter-renovaci\'on tienen media finita, es un proceso Poisson con tasa $\lambda$ si y s\'olo s\'i $\esp\left[U\left(t\right)\right]=\lambda t$, para $t\geq0$.
\end{Note}


\begin{Teo}
Sea $N\left(t\right)$ un proceso puntual simple con puntos de localizaci\'on $T_{n}$ tal que $\eta\left(t\right)=\esp\left[N\left(\right)\right]$ es finita para cada $t$. Entonces para cualquier funci\'on $f:\rea_{+}\rightarrow\rea$,
\begin{eqnarray*}
\esp\left[\sum_{n=1}^{N\left(\right)}f\left(T_{n}\right)\right]=\int_{\left(0,t\right]}f\left(s\right)d\eta\left(s\right)\textrm{,  }t\geq0,
\end{eqnarray*}
suponiendo que la integral exista. Adem\'as si $X_{1},X_{2},\ldots$ son variables aleatorias definidas en el mismo espacio de probabilidad que el proceso $N\left(t\right)$ tal que $\esp\left[X_{n}|T_{n}=s\right]=f\left(s\right)$, independiente de $n$. Entonces
\begin{eqnarray*}
\esp\left[\sum_{n=1}^{N\left(t\right)}X_{n}\right]=\int_{\left(0,t\right]}f\left(s\right)d\eta\left(s\right)\textrm{,  }t\geq0,
\end{eqnarray*} 
suponiendo que la integral exista. 
\end{Teo}

\begin{Coro}[Identidad de Wald para Renovaciones]
Para el proceso de renovaci\'on $N\left(t\right)$,
\begin{eqnarray*}
\esp\left[T_{N\left(t\right)+1}\right]=\mu\esp\left[N\left(t\right)+1\right]\textrm{,  }t\geq0,
\end{eqnarray*}  
\end{Coro}

%______________________________________________________________________
%\subsection{Procesos de Renovaci\'on}
%______________________________________________________________________

\begin{Def}\label{Def.Tn}
Sean $0\leq T_{1}\leq T_{2}\leq \ldots$ son tiempos aleatorios infinitos en los cuales ocurren ciertos eventos. El n\'umero de tiempos $T_{n}$ en el intervalo $\left[0,t\right)$ es

\begin{eqnarray}
N\left(t\right)=\sum_{n=1}^{\infty}\indora\left(T_{n}\leq t\right),
\end{eqnarray}
para $t\geq0$.
\end{Def}

Si se consideran los puntos $T_{n}$ como elementos de $\rea_{+}$, y $N\left(t\right)$ es el n\'umero de puntos en $\rea$. El proceso denotado por $\left\{N\left(t\right):t\geq0\right\}$, denotado por $N\left(t\right)$, es un proceso puntual en $\rea_{+}$. Los $T_{n}$ son los tiempos de ocurrencia, el proceso puntual $N\left(t\right)$ es simple si su n\'umero de ocurrencias son distintas: $0<T_{1}<T_{2}<\ldots$ casi seguramente.

\begin{Def}
Un proceso puntual $N\left(t\right)$ es un proceso de renovaci\'on si los tiempos de interocurrencia $\xi_{n}=T_{n}-T_{n-1}$, para $n\geq1$, son independientes e identicamente distribuidos con distribuci\'on $F$, donde $F\left(0\right)=0$ y $T_{0}=0$. Los $T_{n}$ son llamados tiempos de renovaci\'on, referente a la independencia o renovaci\'on de la informaci\'on estoc\'astica en estos tiempos. Los $\xi_{n}$ son los tiempos de inter-renovaci\'on, y $N\left(t\right)$ es el n\'umero de renovaciones en el intervalo $\left[0,t\right)$
\end{Def}


\begin{Note}
Para definir un proceso de renovaci\'on para cualquier contexto, solamente hay que especificar una distribuci\'on $F$, con $F\left(0\right)=0$, para los tiempos de inter-renovaci\'on. La funci\'on $F$ en turno degune las otra variables aleatorias. De manera formal, existe un espacio de probabilidad y una sucesi\'on de variables aleatorias $\xi_{1},\xi_{2},\ldots$ definidas en este con distribuci\'on $F$. Entonces las otras cantidades son $T_{n}=\sum_{k=1}^{n}\xi_{k}$ y $N\left(t\right)=\sum_{n=1}^{\infty}\indora\left(T_{n}\leq t\right)$, donde $T_{n}\rightarrow\infty$ casi seguramente por la Ley Fuerte de los Grandes Números.
\end{Note}

%___________________________________________________________________________________________
%
%\section{Renewal and Regenerative Processes: Serfozo\cite{Serfozo}}
%___________________________________________________________________________________________
%
\begin{Def}\label{Def.Tn}
Sean $0\leq T_{1}\leq T_{2}\leq \ldots$ son tiempos aleatorios infinitos en los cuales ocurren ciertos eventos. El n\'umero de tiempos $T_{n}$ en el intervalo $\left[0,t\right)$ es

\begin{eqnarray}
N\left(t\right)=\sum_{n=1}^{\infty}\indora\left(T_{n}\leq t\right),
\end{eqnarray}
para $t\geq0$.
\end{Def}

Si se consideran los puntos $T_{n}$ como elementos de $\rea_{+}$, y $N\left(t\right)$ es el n\'umero de puntos en $\rea$. El proceso denotado por $\left\{N\left(t\right):t\geq0\right\}$, denotado por $N\left(t\right)$, es un proceso puntual en $\rea_{+}$. Los $T_{n}$ son los tiempos de ocurrencia, el proceso puntual $N\left(t\right)$ es simple si su n\'umero de ocurrencias son distintas: $0<T_{1}<T_{2}<\ldots$ casi seguramente.

\begin{Def}
Un proceso puntual $N\left(t\right)$ es un proceso de renovaci\'on si los tiempos de interocurrencia $\xi_{n}=T_{n}-T_{n-1}$, para $n\geq1$, son independientes e identicamente distribuidos con distribuci\'on $F$, donde $F\left(0\right)=0$ y $T_{0}=0$. Los $T_{n}$ son llamados tiempos de renovaci\'on, referente a la independencia o renovaci\'on de la informaci\'on estoc\'astica en estos tiempos. Los $\xi_{n}$ son los tiempos de inter-renovaci\'on, y $N\left(t\right)$ es el n\'umero de renovaciones en el intervalo $\left[0,t\right)$
\end{Def}


\begin{Note}
Para definir un proceso de renovaci\'on para cualquier contexto, solamente hay que especificar una distribuci\'on $F$, con $F\left(0\right)=0$, para los tiempos de inter-renovaci\'on. La funci\'on $F$ en turno degune las otra variables aleatorias. De manera formal, existe un espacio de probabilidad y una sucesi\'on de variables aleatorias $\xi_{1},\xi_{2},\ldots$ definidas en este con distribuci\'on $F$. Entonces las otras cantidades son $T_{n}=\sum_{k=1}^{n}\xi_{k}$ y $N\left(t\right)=\sum_{n=1}^{\infty}\indora\left(T_{n}\leq t\right)$, donde $T_{n}\rightarrow\infty$ casi seguramente por la Ley Fuerte de los Grandes N\'umeros.
\end{Note}







Los tiempos $T_{n}$ est\'an relacionados con los conteos de $N\left(t\right)$ por

\begin{eqnarray*}
\left\{N\left(t\right)\geq n\right\}&=&\left\{T_{n}\leq t\right\}\\
T_{N\left(t\right)}\leq &t&<T_{N\left(t\right)+1},
\end{eqnarray*}

adem\'as $N\left(T_{n}\right)=n$, y 

\begin{eqnarray*}
N\left(t\right)=\max\left\{n:T_{n}\leq t\right\}=\min\left\{n:T_{n+1}>t\right\}
\end{eqnarray*}

Por propiedades de la convoluci\'on se sabe que

\begin{eqnarray*}
P\left\{T_{n}\leq t\right\}=F^{n\star}\left(t\right)
\end{eqnarray*}
que es la $n$-\'esima convoluci\'on de $F$. Entonces 

\begin{eqnarray*}
\left\{N\left(t\right)\geq n\right\}&=&\left\{T_{n}\leq t\right\}\\
P\left\{N\left(t\right)\leq n\right\}&=&1-F^{\left(n+1\right)\star}\left(t\right)
\end{eqnarray*}

Adem\'as usando el hecho de que $\esp\left[N\left(t\right)\right]=\sum_{n=1}^{\infty}P\left\{N\left(t\right)\geq n\right\}$
se tiene que

\begin{eqnarray*}
\esp\left[N\left(t\right)\right]=\sum_{n=1}^{\infty}F^{n\star}\left(t\right)
\end{eqnarray*}

\begin{Prop}
Para cada $t\geq0$, la funci\'on generadora de momentos $\esp\left[e^{\alpha N\left(t\right)}\right]$ existe para alguna $\alpha$ en una vecindad del 0, y de aqu\'i que $\esp\left[N\left(t\right)^{m}\right]<\infty$, para $m\geq1$.
\end{Prop}

\begin{Ejem}[\textbf{Proceso Poisson}]

Suponga que se tienen tiempos de inter-renovaci\'on \textit{i.i.d.} del proceso de renovaci\'on $N\left(t\right)$ tienen distribuci\'on exponencial $F\left(t\right)=q-e^{-\lambda t}$ con tasa $\lambda$. Entonces $N\left(t\right)$ es un proceso Poisson con tasa $\lambda$.

\end{Ejem}


\begin{Note}
Si el primer tiempo de renovaci\'on $\xi_{1}$ no tiene la misma distribuci\'on que el resto de las $\xi_{n}$, para $n\geq2$, a $N\left(t\right)$ se le llama Proceso de Renovaci\'on retardado, donde si $\xi$ tiene distribuci\'on $G$, entonces el tiempo $T_{n}$ de la $n$-\'esima renovaci\'on tiene distribuci\'on $G\star F^{\left(n-1\right)\star}\left(t\right)$
\end{Note}


\begin{Teo}
Para una constante $\mu\leq\infty$ ( o variable aleatoria), las siguientes expresiones son equivalentes:

\begin{eqnarray}
lim_{n\rightarrow\infty}n^{-1}T_{n}&=&\mu,\textrm{ c.s.}\\
lim_{t\rightarrow\infty}t^{-1}N\left(t\right)&=&1/\mu,\textrm{ c.s.}
\end{eqnarray}
\end{Teo}


Es decir, $T_{n}$ satisface la Ley Fuerte de los Grandes N\'umeros s\'i y s\'olo s\'i $N\left/t\right)$ la cumple.


\begin{Coro}[Ley Fuerte de los Grandes N\'umeros para Procesos de Renovaci\'on]
Si $N\left(t\right)$ es un proceso de renovaci\'on cuyos tiempos de inter-renovaci\'on tienen media $\mu\leq\infty$, entonces
\begin{eqnarray}
t^{-1}N\left(t\right)\rightarrow 1/\mu,\textrm{ c.s. cuando }t\rightarrow\infty.
\end{eqnarray}

\end{Coro}


Considerar el proceso estoc\'astico de valores reales $\left\{Z\left(t\right):t\geq0\right\}$ en el mismo espacio de probabilidad que $N\left(t\right)$

\begin{Def}
Para el proceso $\left\{Z\left(t\right):t\geq0\right\}$ se define la fluctuaci\'on m\'axima de $Z\left(t\right)$ en el intervalo $\left(T_{n-1},T_{n}\right]$:
\begin{eqnarray*}
M_{n}=\sup_{T_{n-1}<t\leq T_{n}}|Z\left(t\right)-Z\left(T_{n-1}\right)|
\end{eqnarray*}
\end{Def}

\begin{Teo}
Sup\'ongase que $n^{-1}T_{n}\rightarrow\mu$ c.s. cuando $n\rightarrow\infty$, donde $\mu\leq\infty$ es una constante o variable aleatoria. Sea $a$ una constante o variable aleatoria que puede ser infinita cuando $\mu$ es finita, y considere las expresiones l\'imite:
\begin{eqnarray}
lim_{n\rightarrow\infty}n^{-1}Z\left(T_{n}\right)&=&a,\textrm{ c.s.}\\
lim_{t\rightarrow\infty}t^{-1}Z\left(t\right)&=&a/\mu,\textrm{ c.s.}
\end{eqnarray}
La segunda expresi\'on implica la primera. Conversamente, la primera implica la segunda si el proceso $Z\left(t\right)$ es creciente, o si $lim_{n\rightarrow\infty}n^{-1}M_{n}=0$ c.s.
\end{Teo}

\begin{Coro}
Si $N\left(t\right)$ es un proceso de renovaci\'on, y $\left(Z\left(T_{n}\right)-Z\left(T_{n-1}\right),M_{n}\right)$, para $n\geq1$, son variables aleatorias independientes e id\'enticamente distribuidas con media finita, entonces,
\begin{eqnarray}
lim_{t\rightarrow\infty}t^{-1}Z\left(t\right)\rightarrow\frac{\esp\left[Z\left(T_{1}\right)-Z\left(T_{0}\right)\right]}{\esp\left[T_{1}\right]},\textrm{ c.s. cuando  }t\rightarrow\infty.
\end{eqnarray}
\end{Coro}


Sup\'ongase que $N\left(t\right)$ es un proceso de renovaci\'on con distribuci\'on $F$ con media finita $\mu$.

\begin{Def}
La funci\'on de renovaci\'on asociada con la distribuci\'on $F$, del proceso $N\left(t\right)$, es
\begin{eqnarray*}
U\left(t\right)=\sum_{n=1}^{\infty}F^{n\star}\left(t\right),\textrm{   }t\geq0,
\end{eqnarray*}
donde $F^{0\star}\left(t\right)=\indora\left(t\geq0\right)$.
\end{Def}


\begin{Prop}
Sup\'ongase que la distribuci\'on de inter-renovaci\'on $F$ tiene densidad $f$. Entonces $U\left(t\right)$ tambi\'en tiene densidad, para $t>0$, y es $U^{'}\left(t\right)=\sum_{n=0}^{\infty}f^{n\star}\left(t\right)$. Adem\'as
\begin{eqnarray*}
\prob\left\{N\left(t\right)>N\left(t-\right)\right\}=0\textrm{,   }t\geq0.
\end{eqnarray*}
\end{Prop}

\begin{Def}
La Transformada de Laplace-Stieljes de $F$ est\'a dada por

\begin{eqnarray*}
\hat{F}\left(\alpha\right)=\int_{\rea_{+}}e^{-\alpha t}dF\left(t\right)\textrm{,  }\alpha\geq0.
\end{eqnarray*}
\end{Def}

Entonces

\begin{eqnarray*}
\hat{U}\left(\alpha\right)=\sum_{n=0}^{\infty}\hat{F^{n\star}}\left(\alpha\right)=\sum_{n=0}^{\infty}\hat{F}\left(\alpha\right)^{n}=\frac{1}{1-\hat{F}\left(\alpha\right)}.
\end{eqnarray*}


\begin{Prop}
La Transformada de Laplace $\hat{U}\left(\alpha\right)$ y $\hat{F}\left(\alpha\right)$ determina una a la otra de manera \'unica por la relaci\'on $\hat{U}\left(\alpha\right)=\frac{1}{1-\hat{F}\left(\alpha\right)}$.
\end{Prop}


\begin{Note}
Un proceso de renovaci\'on $N\left(t\right)$ cuyos tiempos de inter-renovaci\'on tienen media finita, es un proceso Poisson con tasa $\lambda$ si y s\'olo s\'i $\esp\left[U\left(t\right)\right]=\lambda t$, para $t\geq0$.
\end{Note}


\begin{Teo}
Sea $N\left(t\right)$ un proceso puntual simple con puntos de localizaci\'on $T_{n}$ tal que $\eta\left(t\right)=\esp\left[N\left(\right)\right]$ es finita para cada $t$. Entonces para cualquier funci\'on $f:\rea_{+}\rightarrow\rea$,
\begin{eqnarray*}
\esp\left[\sum_{n=1}^{N\left(\right)}f\left(T_{n}\right)\right]=\int_{\left(0,t\right]}f\left(s\right)d\eta\left(s\right)\textrm{,  }t\geq0,
\end{eqnarray*}
suponiendo que la integral exista. Adem\'as si $X_{1},X_{2},\ldots$ son variables aleatorias definidas en el mismo espacio de probabilidad que el proceso $N\left(t\right)$ tal que $\esp\left[X_{n}|T_{n}=s\right]=f\left(s\right)$, independiente de $n$. Entonces
\begin{eqnarray*}
\esp\left[\sum_{n=1}^{N\left(t\right)}X_{n}\right]=\int_{\left(0,t\right]}f\left(s\right)d\eta\left(s\right)\textrm{,  }t\geq0,
\end{eqnarray*} 
suponiendo que la integral exista. 
\end{Teo}

\begin{Coro}[Identidad de Wald para Renovaciones]
Para el proceso de renovaci\'on $N\left(t\right)$,
\begin{eqnarray*}
\esp\left[T_{N\left(t\right)+1}\right]=\mu\esp\left[N\left(t\right)+1\right]\textrm{,  }t\geq0,
\end{eqnarray*}  
\end{Coro}


\begin{Def}
Sea $h\left(t\right)$ funci\'on de valores reales en $\rea$ acotada en intervalos finitos e igual a cero para $t<0$ La ecuaci\'on de renovaci\'on para $h\left(t\right)$ y la distribuci\'on $F$ es

\begin{eqnarray}\label{Ec.Renovacion}
H\left(t\right)=h\left(t\right)+\int_{\left[0,t\right]}H\left(t-s\right)dF\left(s\right)\textrm{,    }t\geq0,
\end{eqnarray}
donde $H\left(t\right)$ es una funci\'on de valores reales. Esto es $H=h+F\star H$. Decimos que $H\left(t\right)$ es soluci\'on de esta ecuaci\'on si satisface la ecuaci\'on, y es acotada en intervalos finitos e iguales a cero para $t<0$.
\end{Def}

\begin{Prop}
La funci\'on $U\star h\left(t\right)$ es la \'unica soluci\'on de la ecuaci\'on de renovaci\'on (\ref{Ec.Renovacion}).
\end{Prop}

\begin{Teo}[Teorema Renovaci\'on Elemental]
\begin{eqnarray*}
t^{-1}U\left(t\right)\rightarrow 1/\mu\textrm{,    cuando }t\rightarrow\infty.
\end{eqnarray*}
\end{Teo}



Sup\'ongase que $N\left(t\right)$ es un proceso de renovaci\'on con distribuci\'on $F$ con media finita $\mu$.

\begin{Def}
La funci\'on de renovaci\'on asociada con la distribuci\'on $F$, del proceso $N\left(t\right)$, es
\begin{eqnarray*}
U\left(t\right)=\sum_{n=1}^{\infty}F^{n\star}\left(t\right),\textrm{   }t\geq0,
\end{eqnarray*}
donde $F^{0\star}\left(t\right)=\indora\left(t\geq0\right)$.
\end{Def}


\begin{Prop}
Sup\'ongase que la distribuci\'on de inter-renovaci\'on $F$ tiene densidad $f$. Entonces $U\left(t\right)$ tambi\'en tiene densidad, para $t>0$, y es $U^{'}\left(t\right)=\sum_{n=0}^{\infty}f^{n\star}\left(t\right)$. Adem\'as
\begin{eqnarray*}
\prob\left\{N\left(t\right)>N\left(t-\right)\right\}=0\textrm{,   }t\geq0.
\end{eqnarray*}
\end{Prop}

\begin{Def}
La Transformada de Laplace-Stieljes de $F$ est\'a dada por

\begin{eqnarray*}
\hat{F}\left(\alpha\right)=\int_{\rea_{+}}e^{-\alpha t}dF\left(t\right)\textrm{,  }\alpha\geq0.
\end{eqnarray*}
\end{Def}

Entonces

\begin{eqnarray*}
\hat{U}\left(\alpha\right)=\sum_{n=0}^{\infty}\hat{F^{n\star}}\left(\alpha\right)=\sum_{n=0}^{\infty}\hat{F}\left(\alpha\right)^{n}=\frac{1}{1-\hat{F}\left(\alpha\right)}.
\end{eqnarray*}


\begin{Prop}
La Transformada de Laplace $\hat{U}\left(\alpha\right)$ y $\hat{F}\left(\alpha\right)$ determina una a la otra de manera \'unica por la relaci\'on $\hat{U}\left(\alpha\right)=\frac{1}{1-\hat{F}\left(\alpha\right)}$.
\end{Prop}


\begin{Note}
Un proceso de renovaci\'on $N\left(t\right)$ cuyos tiempos de inter-renovaci\'on tienen media finita, es un proceso Poisson con tasa $\lambda$ si y s\'olo s\'i $\esp\left[U\left(t\right)\right]=\lambda t$, para $t\geq0$.
\end{Note}


\begin{Teo}
Sea $N\left(t\right)$ un proceso puntual simple con puntos de localizaci\'on $T_{n}$ tal que $\eta\left(t\right)=\esp\left[N\left(\right)\right]$ es finita para cada $t$. Entonces para cualquier funci\'on $f:\rea_{+}\rightarrow\rea$,
\begin{eqnarray*}
\esp\left[\sum_{n=1}^{N\left(\right)}f\left(T_{n}\right)\right]=\int_{\left(0,t\right]}f\left(s\right)d\eta\left(s\right)\textrm{,  }t\geq0,
\end{eqnarray*}
suponiendo que la integral exista. Adem\'as si $X_{1},X_{2},\ldots$ son variables aleatorias definidas en el mismo espacio de probabilidad que el proceso $N\left(t\right)$ tal que $\esp\left[X_{n}|T_{n}=s\right]=f\left(s\right)$, independiente de $n$. Entonces
\begin{eqnarray*}
\esp\left[\sum_{n=1}^{N\left(t\right)}X_{n}\right]=\int_{\left(0,t\right]}f\left(s\right)d\eta\left(s\right)\textrm{,  }t\geq0,
\end{eqnarray*} 
suponiendo que la integral exista. 
\end{Teo}

\begin{Coro}[Identidad de Wald para Renovaciones]
Para el proceso de renovaci\'on $N\left(t\right)$,
\begin{eqnarray*}
\esp\left[T_{N\left(t\right)+1}\right]=\mu\esp\left[N\left(t\right)+1\right]\textrm{,  }t\geq0,
\end{eqnarray*}  
\end{Coro}


\begin{Def}
Sea $h\left(t\right)$ funci\'on de valores reales en $\rea$ acotada en intervalos finitos e igual a cero para $t<0$ La ecuaci\'on de renovaci\'on para $h\left(t\right)$ y la distribuci\'on $F$ es

\begin{eqnarray}\label{Ec.Renovacion}
H\left(t\right)=h\left(t\right)+\int_{\left[0,t\right]}H\left(t-s\right)dF\left(s\right)\textrm{,    }t\geq0,
\end{eqnarray}
donde $H\left(t\right)$ es una funci\'on de valores reales. Esto es $H=h+F\star H$. Decimos que $H\left(t\right)$ es soluci\'on de esta ecuaci\'on si satisface la ecuaci\'on, y es acotada en intervalos finitos e iguales a cero para $t<0$.
\end{Def}

\begin{Prop}
La funci\'on $U\star h\left(t\right)$ es la \'unica soluci\'on de la ecuaci\'on de renovaci\'on (\ref{Ec.Renovacion}).
\end{Prop}

\begin{Teo}[Teorema Renovaci\'on Elemental]
\begin{eqnarray*}
t^{-1}U\left(t\right)\rightarrow 1/\mu\textrm{,    cuando }t\rightarrow\infty.
\end{eqnarray*}
\end{Teo}


\begin{Note} Una funci\'on $h:\rea_{+}\rightarrow\rea$ es Directamente Riemann Integrable en los siguientes casos:
\begin{itemize}
\item[a)] $h\left(t\right)\geq0$ es decreciente y Riemann Integrable.
\item[b)] $h$ es continua excepto posiblemente en un conjunto de Lebesgue de medida 0, y $|h\left(t\right)|\leq b\left(t\right)$, donde $b$ es DRI.
\end{itemize}
\end{Note}

\begin{Teo}[Teorema Principal de Renovaci\'on]
Si $F$ es no aritm\'etica y $h\left(t\right)$ es Directamente Riemann Integrable (DRI), entonces

\begin{eqnarray*}
lim_{t\rightarrow\infty}U\star h=\frac{1}{\mu}\int_{\rea_{+}}h\left(s\right)ds.
\end{eqnarray*}
\end{Teo}

\begin{Prop}
Cualquier funci\'on $H\left(t\right)$ acotada en intervalos finitos y que es 0 para $t<0$ puede expresarse como
\begin{eqnarray*}
H\left(t\right)=U\star h\left(t\right)\textrm{,  donde }h\left(t\right)=H\left(t\right)-F\star H\left(t\right)
\end{eqnarray*}
\end{Prop}

\begin{Def}
Un proceso estoc\'astico $X\left(t\right)$ es crudamente regenerativo en un tiempo aleatorio positivo $T$ si
\begin{eqnarray*}
\esp\left[X\left(T+t\right)|T\right]=\esp\left[X\left(t\right)\right]\textrm{, para }t\geq0,\end{eqnarray*}
y con las esperanzas anteriores finitas.
\end{Def}

\begin{Prop}
Sup\'ongase que $X\left(t\right)$ es un proceso crudamente regenerativo en $T$, que tiene distribuci\'on $F$. Si $\esp\left[X\left(t\right)\right]$ es acotado en intervalos finitos, entonces
\begin{eqnarray*}
\esp\left[X\left(t\right)\right]=U\star h\left(t\right)\textrm{,  donde }h\left(t\right)=\esp\left[X\left(t\right)\indora\left(T>t\right)\right].
\end{eqnarray*}
\end{Prop}

\begin{Teo}[Regeneraci\'on Cruda]
Sup\'ongase que $X\left(t\right)$ es un proceso con valores positivo crudamente regenerativo en $T$, y def\'inase $M=\sup\left\{|X\left(t\right)|:t\leq T\right\}$. Si $T$ es no aritm\'etico y $M$ y $MT$ tienen media finita, entonces
\begin{eqnarray*}
lim_{t\rightarrow\infty}\esp\left[X\left(t\right)\right]=\frac{1}{\mu}\int_{\rea_{+}}h\left(s\right)ds,
\end{eqnarray*}
donde $h\left(t\right)=\esp\left[X\left(t\right)\indora\left(T>t\right)\right]$.
\end{Teo}


\begin{Note} Una funci\'on $h:\rea_{+}\rightarrow\rea$ es Directamente Riemann Integrable en los siguientes casos:
\begin{itemize}
\item[a)] $h\left(t\right)\geq0$ es decreciente y Riemann Integrable.
\item[b)] $h$ es continua excepto posiblemente en un conjunto de Lebesgue de medida 0, y $|h\left(t\right)|\leq b\left(t\right)$, donde $b$ es DRI.
\end{itemize}
\end{Note}

\begin{Teo}[Teorema Principal de Renovaci\'on]
Si $F$ es no aritm\'etica y $h\left(t\right)$ es Directamente Riemann Integrable (DRI), entonces

\begin{eqnarray*}
lim_{t\rightarrow\infty}U\star h=\frac{1}{\mu}\int_{\rea_{+}}h\left(s\right)ds.
\end{eqnarray*}
\end{Teo}

\begin{Prop}
Cualquier funci\'on $H\left(t\right)$ acotada en intervalos finitos y que es 0 para $t<0$ puede expresarse como
\begin{eqnarray*}
H\left(t\right)=U\star h\left(t\right)\textrm{,  donde }h\left(t\right)=H\left(t\right)-F\star H\left(t\right)
\end{eqnarray*}
\end{Prop}

\begin{Def}
Un proceso estoc\'astico $X\left(t\right)$ es crudamente regenerativo en un tiempo aleatorio positivo $T$ si
\begin{eqnarray*}
\esp\left[X\left(T+t\right)|T\right]=\esp\left[X\left(t\right)\right]\textrm{, para }t\geq0,\end{eqnarray*}
y con las esperanzas anteriores finitas.
\end{Def}

\begin{Prop}
Sup\'ongase que $X\left(t\right)$ es un proceso crudamente regenerativo en $T$, que tiene distribuci\'on $F$. Si $\esp\left[X\left(t\right)\right]$ es acotado en intervalos finitos, entonces
\begin{eqnarray*}
\esp\left[X\left(t\right)\right]=U\star h\left(t\right)\textrm{,  donde }h\left(t\right)=\esp\left[X\left(t\right)\indora\left(T>t\right)\right].
\end{eqnarray*}
\end{Prop}

\begin{Teo}[Regeneraci\'on Cruda]
Sup\'ongase que $X\left(t\right)$ es un proceso con valores positivo crudamente regenerativo en $T$, y def\'inase $M=\sup\left\{|X\left(t\right)|:t\leq T\right\}$. Si $T$ es no aritm\'etico y $M$ y $MT$ tienen media finita, entonces
\begin{eqnarray*}
lim_{t\rightarrow\infty}\esp\left[X\left(t\right)\right]=\frac{1}{\mu}\int_{\rea_{+}}h\left(s\right)ds,
\end{eqnarray*}
donde $h\left(t\right)=\esp\left[X\left(t\right)\indora\left(T>t\right)\right]$.
\end{Teo}

\begin{Def}
Para el proceso $\left\{\left(N\left(t\right),X\left(t\right)\right):t\geq0\right\}$, sus trayectoria muestrales en el intervalo de tiempo $\left[T_{n-1},T_{n}\right)$ est\'an descritas por
\begin{eqnarray*}
\zeta_{n}=\left(\xi_{n},\left\{X\left(T_{n-1}+t\right):0\leq t<\xi_{n}\right\}\right)
\end{eqnarray*}
Este $\zeta_{n}$ es el $n$-\'esimo segmento del proceso. El proceso es regenerativo sobre los tiempos $T_{n}$ si sus segmentos $\zeta_{n}$ son independientes e id\'enticamennte distribuidos.
\end{Def}


\begin{Note}
Si $\tilde{X}\left(t\right)$ con espacio de estados $\tilde{S}$ es regenerativo sobre $T_{n}$, entonces $X\left(t\right)=f\left(\tilde{X}\left(t\right)\right)$ tambi\'en es regenerativo sobre $T_{n}$, para cualquier funci\'on $f:\tilde{S}\rightarrow S$.
\end{Note}

\begin{Note}
Los procesos regenerativos son crudamente regenerativos, pero no al rev\'es.
\end{Note}


\begin{Note}
Un proceso estoc\'astico a tiempo continuo o discreto es regenerativo si existe un proceso de renovaci\'on  tal que los segmentos del proceso entre tiempos de renovaci\'on sucesivos son i.i.d., es decir, para $\left\{X\left(t\right):t\geq0\right\}$ proceso estoc\'astico a tiempo continuo con espacio de estados $S$, espacio m\'etrico.
\end{Note}

Para $\left\{X\left(t\right):t\geq0\right\}$ Proceso Estoc\'astico a tiempo continuo con estado de espacios $S$, que es un espacio m\'etrico, con trayectorias continuas por la derecha y con l\'imites por la izquierda c.s. Sea $N\left(t\right)$ un proceso de renovaci\'on en $\rea_{+}$ definido en el mismo espacio de probabilidad que $X\left(t\right)$, con tiempos de renovaci\'on $T$ y tiempos de inter-renovaci\'on $\xi_{n}=T_{n}-T_{n-1}$, con misma distribuci\'on $F$ de media finita $\mu$.



\begin{Def}
Para el proceso $\left\{\left(N\left(t\right),X\left(t\right)\right):t\geq0\right\}$, sus trayectoria muestrales en el intervalo de tiempo $\left[T_{n-1},T_{n}\right)$ est\'an descritas por
\begin{eqnarray*}
\zeta_{n}=\left(\xi_{n},\left\{X\left(T_{n-1}+t\right):0\leq t<\xi_{n}\right\}\right)
\end{eqnarray*}
Este $\zeta_{n}$ es el $n$-\'esimo segmento del proceso. El proceso es regenerativo sobre los tiempos $T_{n}$ si sus segmentos $\zeta_{n}$ son independientes e id\'enticamennte distribuidos.
\end{Def}

\begin{Note}
Un proceso regenerativo con media de la longitud de ciclo finita es llamado positivo recurrente.
\end{Note}

\begin{Teo}[Procesos Regenerativos]
Suponga que el proceso
\end{Teo}


\begin{Def}[Renewal Process Trinity]
Para un proceso de renovaci\'on $N\left(t\right)$, los siguientes procesos proveen de informaci\'on sobre los tiempos de renovaci\'on.
\begin{itemize}
\item $A\left(t\right)=t-T_{N\left(t\right)}$, el tiempo de recurrencia hacia atr\'as al tiempo $t$, que es el tiempo desde la \'ultima renovaci\'on para $t$.

\item $B\left(t\right)=T_{N\left(t\right)+1}-t$, el tiempo de recurrencia hacia adelante al tiempo $t$, residual del tiempo de renovaci\'on, que es el tiempo para la pr\'oxima renovaci\'on despu\'es de $t$.

\item $L\left(t\right)=\xi_{N\left(t\right)+1}=A\left(t\right)+B\left(t\right)$, la longitud del intervalo de renovaci\'on que contiene a $t$.
\end{itemize}
\end{Def}

\begin{Note}
El proceso tridimensional $\left(A\left(t\right),B\left(t\right),L\left(t\right)\right)$ es regenerativo sobre $T_{n}$, y por ende cada proceso lo es. Cada proceso $A\left(t\right)$ y $B\left(t\right)$ son procesos de MArkov a tiempo continuo con trayectorias continuas por partes en el espacio de estados $\rea_{+}$. Una expresi\'on conveniente para su distribuci\'on conjunta es, para $0\leq x<t,y\geq0$
\begin{equation}\label{NoRenovacion}
P\left\{A\left(t\right)>x,B\left(t\right)>y\right\}=
P\left\{N\left(t+y\right)-N\left((t-x)\right)=0\right\}
\end{equation}
\end{Note}

\begin{Ejem}[Tiempos de recurrencia Poisson]
Si $N\left(t\right)$ es un proceso Poisson con tasa $\lambda$, entonces de la expresi\'on (\ref{NoRenovacion}) se tiene que

\begin{eqnarray*}
\begin{array}{lc}
P\left\{A\left(t\right)>x,B\left(t\right)>y\right\}=e^{-\lambda\left(x+y\right)},&0\leq x<t,y\geq0,
\end{array}
\end{eqnarray*}
que es la probabilidad Poisson de no renovaciones en un intervalo de longitud $x+y$.

\end{Ejem}

\begin{Note}
Una cadena de Markov erg\'odica tiene la propiedad de ser estacionaria si la distribuci\'on de su estado al tiempo $0$ es su distribuci\'on estacionaria.
\end{Note}


\begin{Def}
Un proceso estoc\'astico a tiempo continuo $\left\{X\left(t\right):t\geq0\right\}$ en un espacio general es estacionario si sus distribuciones finito dimensionales son invariantes bajo cualquier  traslado: para cada $0\leq s_{1}<s_{2}<\cdots<s_{k}$ y $t\geq0$,
\begin{eqnarray*}
\left(X\left(s_{1}+t\right),\ldots,X\left(s_{k}+t\right)\right)=_{d}\left(X\left(s_{1}\right),\ldots,X\left(s_{k}\right)\right).
\end{eqnarray*}
\end{Def}

\begin{Note}
Un proceso de Markov es estacionario si $X\left(t\right)=_{d}X\left(0\right)$, $t\geq0$.
\end{Note}

Considerese el proceso $N\left(t\right)=\sum_{n}\indora\left(\tau_{n}\leq t\right)$ en $\rea_{+}$, con puntos $0<\tau_{1}<\tau_{2}<\cdots$.

\begin{Prop}
Si $N$ es un proceso puntual estacionario y $\esp\left[N\left(1\right)\right]<\infty$, entonces $\esp\left[N\left(t\right)\right]=t\esp\left[N\left(1\right)\right]$, $t\geq0$

\end{Prop}

\begin{Teo}
Los siguientes enunciados son equivalentes
\begin{itemize}
\item[i)] El proceso retardado de renovaci\'on $N$ es estacionario.

\item[ii)] EL proceso de tiempos de recurrencia hacia adelante $B\left(t\right)$ es estacionario.


\item[iii)] $\esp\left[N\left(t\right)\right]=t/\mu$,


\item[iv)] $G\left(t\right)=F_{e}\left(t\right)=\frac{1}{\mu}\int_{0}^{t}\left[1-F\left(s\right)\right]ds$
\end{itemize}
Cuando estos enunciados son ciertos, $P\left\{B\left(t\right)\leq x\right\}=F_{e}\left(x\right)$, para $t,x\geq0$.

\end{Teo}

\begin{Note}
Una consecuencia del teorema anterior es que el Proceso Poisson es el \'unico proceso sin retardo que es estacionario.
\end{Note}

\begin{Coro}
El proceso de renovaci\'on $N\left(t\right)$ sin retardo, y cuyos tiempos de inter renonaci\'on tienen media finita, es estacionario si y s\'olo si es un proceso Poisson.

\end{Coro}

%______________________________________________________________________

%\section{Ejemplos, Notas importantes}
%______________________________________________________________________
%\section*{Ap\'endice A}
%__________________________________________________________________

%________________________________________________________________________
%\subsection*{Procesos Regenerativos}
%________________________________________________________________________



\begin{Note}
Si $\tilde{X}\left(t\right)$ con espacio de estados $\tilde{S}$ es regenerativo sobre $T_{n}$, entonces $X\left(t\right)=f\left(\tilde{X}\left(t\right)\right)$ tambi\'en es regenerativo sobre $T_{n}$, para cualquier funci\'on $f:\tilde{S}\rightarrow S$.
\end{Note}

\begin{Note}
Los procesos regenerativos son crudamente regenerativos, pero no al rev\'es.
\end{Note}
%\subsection*{Procesos Regenerativos: Sigman\cite{Sigman1}}
\begin{Def}[Definici\'on Cl\'asica]
Un proceso estoc\'astico $X=\left\{X\left(t\right):t\geq0\right\}$ es llamado regenerativo is existe una variable aleatoria $R_{1}>0$ tal que
\begin{itemize}
\item[i)] $\left\{X\left(t+R_{1}\right):t\geq0\right\}$ es independiente de $\left\{\left\{X\left(t\right):t<R_{1}\right\},\right\}$
\item[ii)] $\left\{X\left(t+R_{1}\right):t\geq0\right\}$ es estoc\'asticamente equivalente a $\left\{X\left(t\right):t>0\right\}$
\end{itemize}

Llamamos a $R_{1}$ tiempo de regeneraci\'on, y decimos que $X$ se regenera en este punto.
\end{Def}

$\left\{X\left(t+R_{1}\right)\right\}$ es regenerativo con tiempo de regeneraci\'on $R_{2}$, independiente de $R_{1}$ pero con la misma distribuci\'on que $R_{1}$. Procediendo de esta manera se obtiene una secuencia de variables aleatorias independientes e id\'enticamente distribuidas $\left\{R_{n}\right\}$ llamados longitudes de ciclo. Si definimos a $Z_{k}\equiv R_{1}+R_{2}+\cdots+R_{k}$, se tiene un proceso de renovaci\'on llamado proceso de renovaci\'on encajado para $X$.




\begin{Def}
Para $x$ fijo y para cada $t\geq0$, sea $I_{x}\left(t\right)=1$ si $X\left(t\right)\leq x$,  $I_{x}\left(t\right)=0$ en caso contrario, y def\'inanse los tiempos promedio
\begin{eqnarray*}
\overline{X}&=&lim_{t\rightarrow\infty}\frac{1}{t}\int_{0}^{\infty}X\left(u\right)du\\
\prob\left(X_{\infty}\leq x\right)&=&lim_{t\rightarrow\infty}\frac{1}{t}\int_{0}^{\infty}I_{x}\left(u\right)du,
\end{eqnarray*}
cuando estos l\'imites existan.
\end{Def}

Como consecuencia del teorema de Renovaci\'on-Recompensa, se tiene que el primer l\'imite  existe y es igual a la constante
\begin{eqnarray*}
\overline{X}&=&\frac{\esp\left[\int_{0}^{R_{1}}X\left(t\right)dt\right]}{\esp\left[R_{1}\right]},
\end{eqnarray*}
suponiendo que ambas esperanzas son finitas.

\begin{Note}
\begin{itemize}
\item[a)] Si el proceso regenerativo $X$ es positivo recurrente y tiene trayectorias muestrales no negativas, entonces la ecuaci\'on anterior es v\'alida.
\item[b)] Si $X$ es positivo recurrente regenerativo, podemos construir una \'unica versi\'on estacionaria de este proceso, $X_{e}=\left\{X_{e}\left(t\right)\right\}$, donde $X_{e}$ es un proceso estoc\'astico regenerativo y estrictamente estacionario, con distribuci\'on marginal distribuida como $X_{\infty}$
\end{itemize}
\end{Note}

Para $\left\{X\left(t\right):t\geq0\right\}$ Proceso Estoc\'astico a tiempo continuo con estado de espacios $S$, que es un espacio m\'etrico, con trayectorias continuas por la derecha y con l\'imites por la izquierda c.s. Sea $N\left(t\right)$ un proceso de renovaci\'on en $\rea_{+}$ definido en el mismo espacio de probabilidad que $X\left(t\right)$, con tiempos de renovaci\'on $T$ y tiempos de inter-renovaci\'on $\xi_{n}=T_{n}-T_{n-1}$, con misma distribuci\'on $F$ de media finita $\mu$.


\begin{Def}
Para el proceso $\left\{\left(N\left(t\right),X\left(t\right)\right):t\geq0\right\}$, sus trayectoria muestrales en el intervalo de tiempo $\left[T_{n-1},T_{n}\right)$ est\'an descritas por
\begin{eqnarray*}
\zeta_{n}=\left(\xi_{n},\left\{X\left(T_{n-1}+t\right):0\leq t<\xi_{n}\right\}\right)
\end{eqnarray*}
Este $\zeta_{n}$ es el $n$-\'esimo segmento del proceso. El proceso es regenerativo sobre los tiempos $T_{n}$ si sus segmentos $\zeta_{n}$ son independientes e id\'enticamennte distribuidos.
\end{Def}


\begin{Note}
Si $\tilde{X}\left(t\right)$ con espacio de estados $\tilde{S}$ es regenerativo sobre $T_{n}$, entonces $X\left(t\right)=f\left(\tilde{X}\left(t\right)\right)$ tambi\'en es regenerativo sobre $T_{n}$, para cualquier funci\'on $f:\tilde{S}\rightarrow S$.
\end{Note}

\begin{Note}
Los procesos regenerativos son crudamente regenerativos, pero no al rev\'es.
\end{Note}

\begin{Def}[Definici\'on Cl\'asica]
Un proceso estoc\'astico $X=\left\{X\left(t\right):t\geq0\right\}$ es llamado regenerativo is existe una variable aleatoria $R_{1}>0$ tal que
\begin{itemize}
\item[i)] $\left\{X\left(t+R_{1}\right):t\geq0\right\}$ es independiente de $\left\{\left\{X\left(t\right):t<R_{1}\right\},\right\}$
\item[ii)] $\left\{X\left(t+R_{1}\right):t\geq0\right\}$ es estoc\'asticamente equivalente a $\left\{X\left(t\right):t>0\right\}$
\end{itemize}

Llamamos a $R_{1}$ tiempo de regeneraci\'on, y decimos que $X$ se regenera en este punto.
\end{Def}

$\left\{X\left(t+R_{1}\right)\right\}$ es regenerativo con tiempo de regeneraci\'on $R_{2}$, independiente de $R_{1}$ pero con la misma distribuci\'on que $R_{1}$. Procediendo de esta manera se obtiene una secuencia de variables aleatorias independientes e id\'enticamente distribuidas $\left\{R_{n}\right\}$ llamados longitudes de ciclo. Si definimos a $Z_{k}\equiv R_{1}+R_{2}+\cdots+R_{k}$, se tiene un proceso de renovaci\'on llamado proceso de renovaci\'on encajado para $X$.

\begin{Note}
Un proceso regenerativo con media de la longitud de ciclo finita es llamado positivo recurrente.
\end{Note}


\begin{Def}
Para $x$ fijo y para cada $t\geq0$, sea $I_{x}\left(t\right)=1$ si $X\left(t\right)\leq x$,  $I_{x}\left(t\right)=0$ en caso contrario, y def\'inanse los tiempos promedio
\begin{eqnarray*}
\overline{X}&=&lim_{t\rightarrow\infty}\frac{1}{t}\int_{0}^{\infty}X\left(u\right)du\\
\prob\left(X_{\infty}\leq x\right)&=&lim_{t\rightarrow\infty}\frac{1}{t}\int_{0}^{\infty}I_{x}\left(u\right)du,
\end{eqnarray*}
cuando estos l\'imites existan.
\end{Def}

Como consecuencia del teorema de Renovaci\'on-Recompensa, se tiene que el primer l\'imite  existe y es igual a la constante
\begin{eqnarray*}
\overline{X}&=&\frac{\esp\left[\int_{0}^{R_{1}}X\left(t\right)dt\right]}{\esp\left[R_{1}\right]},
\end{eqnarray*}
suponiendo que ambas esperanzas son finitas.

\begin{Note}
\begin{itemize}
\item[a)] Si el proceso regenerativo $X$ es positivo recurrente y tiene trayectorias muestrales no negativas, entonces la ecuaci\'on anterior es v\'alida.
\item[b)] Si $X$ es positivo recurrente regenerativo, podemos construir una \'unica versi\'on estacionaria de este proceso, $X_{e}=\left\{X_{e}\left(t\right)\right\}$, donde $X_{e}$ es un proceso estoc\'astico regenerativo y estrictamente estacionario, con distribuci\'on marginal distribuida como $X_{\infty}$
\end{itemize}
\end{Note}

%__________________________________________________________________________________________
%\subsection{Procesos Regenerativos Estacionarios - Stidham \cite{Stidham}}
%__________________________________________________________________________________________


Un proceso estoc\'astico a tiempo continuo $\left\{V\left(t\right),t\geq0\right\}$ es un proceso regenerativo si existe una sucesi\'on de variables aleatorias independientes e id\'enticamente distribuidas $\left\{X_{1},X_{2},\ldots\right\}$, sucesi\'on de renovaci\'on, tal que para cualquier conjunto de Borel $A$, 

\begin{eqnarray*}
\prob\left\{V\left(t\right)\in A|X_{1}+X_{2}+\cdots+X_{R\left(t\right)}=s,\left\{V\left(\tau\right),\tau<s\right\}\right\}=\prob\left\{V\left(t-s\right)\in A|X_{1}>t-s\right\},
\end{eqnarray*}
para todo $0\leq s\leq t$, donde $R\left(t\right)=\max\left\{X_{1}+X_{2}+\cdots+X_{j}\leq t\right\}=$n\'umero de renovaciones ({\emph{puntos de regeneraci\'on}}) que ocurren en $\left[0,t\right]$. El intervalo $\left[0,X_{1}\right)$ es llamado {\emph{primer ciclo de regeneraci\'on}} de $\left\{V\left(t \right),t\geq0\right\}$, $\left[X_{1},X_{1}+X_{2}\right)$ el {\emph{segundo ciclo de regeneraci\'on}}, y as\'i sucesivamente.

Sea $X=X_{1}$ y sea $F$ la funci\'on de distrbuci\'on de $X$


\begin{Def}
Se define el proceso estacionario, $\left\{V^{*}\left(t\right),t\geq0\right\}$, para $\left\{V\left(t\right),t\geq0\right\}$ por

\begin{eqnarray*}
\prob\left\{V\left(t\right)\in A\right\}=\frac{1}{\esp\left[X\right]}\int_{0}^{\infty}\prob\left\{V\left(t+x\right)\in A|X>x\right\}\left(1-F\left(x\right)\right)dx,
\end{eqnarray*} 
para todo $t\geq0$ y todo conjunto de Borel $A$.
\end{Def}

\begin{Def}
Una distribuci\'on se dice que es {\emph{aritm\'etica}} si todos sus puntos de incremento son m\'ultiplos de la forma $0,\lambda, 2\lambda,\ldots$ para alguna $\lambda>0$ entera.
\end{Def}


\begin{Def}
Una modificaci\'on medible de un proceso $\left\{V\left(t\right),t\geq0\right\}$, es una versi\'on de este, $\left\{V\left(t,w\right)\right\}$ conjuntamente medible para $t\geq0$ y para $w\in S$, $S$ espacio de estados para $\left\{V\left(t\right),t\geq0\right\}$.
\end{Def}

\begin{Teo}
Sea $\left\{V\left(t\right),t\geq\right\}$ un proceso regenerativo no negativo con modificaci\'on medible. Sea $\esp\left[X\right]<\infty$. Entonces el proceso estacionario dado por la ecuaci\'on anterior est\'a bien definido y tiene funci\'on de distribuci\'on independiente de $t$, adem\'as
\begin{itemize}
\item[i)] \begin{eqnarray*}
\esp\left[V^{*}\left(0\right)\right]&=&\frac{\esp\left[\int_{0}^{X}V\left(s\right)ds\right]}{\esp\left[X\right]}\end{eqnarray*}
\item[ii)] Si $\esp\left[V^{*}\left(0\right)\right]<\infty$, equivalentemente, si $\esp\left[\int_{0}^{X}V\left(s\right)ds\right]<\infty$,entonces
\begin{eqnarray*}
\frac{\int_{0}^{t}V\left(s\right)ds}{t}\rightarrow\frac{\esp\left[\int_{0}^{X}V\left(s\right)ds\right]}{\esp\left[X\right]}
\end{eqnarray*}
con probabilidad 1 y en media, cuando $t\rightarrow\infty$.
\end{itemize}
\end{Teo}
%________________________________________________________________________
\subsection{Procesos Regenerativos Sigman, Thorisson y Wolff \cite{Sigman2}}
%________________________________________________________________________


\begin{Def}[Definici\'on Cl\'asica]
Un proceso estoc\'astico $X=\left\{X\left(t\right):t\geq0\right\}$ es llamado regenerativo is existe una variable aleatoria $R_{1}>0$ tal que
\begin{itemize}
\item[i)] $\left\{X\left(t+R_{1}\right):t\geq0\right\}$ es independiente de $\left\{\left\{X\left(t\right):t<R_{1}\right\},\right\}$
\item[ii)] $\left\{X\left(t+R_{1}\right):t\geq0\right\}$ es estoc\'asticamente equivalente a $\left\{X\left(t\right):t>0\right\}$
\end{itemize}

Llamamos a $R_{1}$ tiempo de regeneraci\'on, y decimos que $X$ se regenera en este punto.
\end{Def}

$\left\{X\left(t+R_{1}\right)\right\}$ es regenerativo con tiempo de regeneraci\'on $R_{2}$, independiente de $R_{1}$ pero con la misma distribuci\'on que $R_{1}$. Procediendo de esta manera se obtiene una secuencia de variables aleatorias independientes e id\'enticamente distribuidas $\left\{R_{n}\right\}$ llamados longitudes de ciclo. Si definimos a $Z_{k}\equiv R_{1}+R_{2}+\cdots+R_{k}$, se tiene un proceso de renovaci\'on llamado proceso de renovaci\'on encajado para $X$.


\begin{Note}
La existencia de un primer tiempo de regeneraci\'on, $R_{1}$, implica la existencia de una sucesi\'on completa de estos tiempos $R_{1},R_{2}\ldots,$ que satisfacen la propiedad deseada \cite{Sigman2}.
\end{Note}


\begin{Note} Para la cola $GI/GI/1$ los usuarios arriban con tiempos $t_{n}$ y son atendidos con tiempos de servicio $S_{n}$, los tiempos de arribo forman un proceso de renovaci\'on  con tiempos entre arribos independientes e identicamente distribuidos (\texttt{i.i.d.})$T_{n}=t_{n}-t_{n-1}$, adem\'as los tiempos de servicio son \texttt{i.i.d.} e independientes de los procesos de arribo. Por \textit{estable} se entiende que $\esp S_{n}<\esp T_{n}<\infty$.
\end{Note}
 


\begin{Def}
Para $x$ fijo y para cada $t\geq0$, sea $I_{x}\left(t\right)=1$ si $X\left(t\right)\leq x$,  $I_{x}\left(t\right)=0$ en caso contrario, y def\'inanse los tiempos promedio
\begin{eqnarray*}
\overline{X}&=&lim_{t\rightarrow\infty}\frac{1}{t}\int_{0}^{\infty}X\left(u\right)du\\
\prob\left(X_{\infty}\leq x\right)&=&lim_{t\rightarrow\infty}\frac{1}{t}\int_{0}^{\infty}I_{x}\left(u\right)du,
\end{eqnarray*}
cuando estos l\'imites existan.
\end{Def}

Como consecuencia del teorema de Renovaci\'on-Recompensa, se tiene que el primer l\'imite  existe y es igual a la constante
\begin{eqnarray*}
\overline{X}&=&\frac{\esp\left[\int_{0}^{R_{1}}X\left(t\right)dt\right]}{\esp\left[R_{1}\right]},
\end{eqnarray*}
suponiendo que ambas esperanzas son finitas.
 
\begin{Note}
Funciones de procesos regenerativos son regenerativas, es decir, si $X\left(t\right)$ es regenerativo y se define el proceso $Y\left(t\right)$ por $Y\left(t\right)=f\left(X\left(t\right)\right)$ para alguna funci\'on Borel medible $f\left(\cdot\right)$. Adem\'as $Y$ es regenerativo con los mismos tiempos de renovaci\'on que $X$. 

En general, los tiempos de renovaci\'on, $Z_{k}$ de un proceso regenerativo no requieren ser tiempos de paro con respecto a la evoluci\'on de $X\left(t\right)$.
\end{Note} 

\begin{Note}
Una funci\'on de un proceso de Markov, usualmente no ser\'a un proceso de Markov, sin embargo ser\'a regenerativo si el proceso de Markov lo es.
\end{Note}

 
\begin{Note}
Un proceso regenerativo con media de la longitud de ciclo finita es llamado positivo recurrente.
\end{Note}


\begin{Note}
\begin{itemize}
\item[a)] Si el proceso regenerativo $X$ es positivo recurrente y tiene trayectorias muestrales no negativas, entonces la ecuaci\'on anterior es v\'alida.
\item[b)] Si $X$ es positivo recurrente regenerativo, podemos construir una \'unica versi\'on estacionaria de este proceso, $X_{e}=\left\{X_{e}\left(t\right)\right\}$, donde $X_{e}$ es un proceso estoc\'astico regenerativo y estrictamente estacionario, con distribuci\'on marginal distribuida como $X_{\infty}$
\end{itemize}
\end{Note}


%__________________________________________________________________________________________
\subsection{Procesos Regenerativos Estacionarios - Stidham \cite{Stidham}}
%__________________________________________________________________________________________


Un proceso estoc\'astico a tiempo continuo $\left\{V\left(t\right),t\geq0\right\}$ es un proceso regenerativo si existe una sucesi\'on de variables aleatorias independientes e id\'enticamente distribuidas $\left\{X_{1},X_{2},\ldots\right\}$, sucesi\'on de renovaci\'on, tal que para cualquier conjunto de Borel $A$, 

\begin{eqnarray*}
\prob\left\{V\left(t\right)\in A|X_{1}+X_{2}+\cdots+X_{R\left(t\right)}=s,\left\{V\left(\tau\right),\tau<s\right\}\right\}=\prob\left\{V\left(t-s\right)\in A|X_{1}>t-s\right\},
\end{eqnarray*}
para todo $0\leq s\leq t$, donde $R\left(t\right)=\max\left\{X_{1}+X_{2}+\cdots+X_{j}\leq t\right\}=$n\'umero de renovaciones ({\emph{puntos de regeneraci\'on}}) que ocurren en $\left[0,t\right]$. El intervalo $\left[0,X_{1}\right)$ es llamado {\emph{primer ciclo de regeneraci\'on}} de $\left\{V\left(t \right),t\geq0\right\}$, $\left[X_{1},X_{1}+X_{2}\right)$ el {\emph{segundo ciclo de regeneraci\'on}}, y as\'i sucesivamente.

Sea $X=X_{1}$ y sea $F$ la funci\'on de distrbuci\'on de $X$


\begin{Def}
Se define el proceso estacionario, $\left\{V^{*}\left(t\right),t\geq0\right\}$, para $\left\{V\left(t\right),t\geq0\right\}$ por

\begin{eqnarray*}
\prob\left\{V\left(t\right)\in A\right\}=\frac{1}{\esp\left[X\right]}\int_{0}^{\infty}\prob\left\{V\left(t+x\right)\in A|X>x\right\}\left(1-F\left(x\right)\right)dx,
\end{eqnarray*} 
para todo $t\geq0$ y todo conjunto de Borel $A$.
\end{Def}

\begin{Def}
Una distribuci\'on se dice que es {\emph{aritm\'etica}} si todos sus puntos de incremento son m\'ultiplos de la forma $0,\lambda, 2\lambda,\ldots$ para alguna $\lambda>0$ entera.
\end{Def}


\begin{Def}
Una modificaci\'on medible de un proceso $\left\{V\left(t\right),t\geq0\right\}$, es una versi\'on de este, $\left\{V\left(t,w\right)\right\}$ conjuntamente medible para $t\geq0$ y para $w\in S$, $S$ espacio de estados para $\left\{V\left(t\right),t\geq0\right\}$.
\end{Def}

\begin{Teo}
Sea $\left\{V\left(t\right),t\geq\right\}$ un proceso regenerativo no negativo con modificaci\'on medible. Sea $\esp\left[X\right]<\infty$. Entonces el proceso estacionario dado por la ecuaci\'on anterior est\'a bien definido y tiene funci\'on de distribuci\'on independiente de $t$, adem\'as
\begin{itemize}
\item[i)] \begin{eqnarray*}
\esp\left[V^{*}\left(0\right)\right]&=&\frac{\esp\left[\int_{0}^{X}V\left(s\right)ds\right]}{\esp\left[X\right]}\end{eqnarray*}
\item[ii)] Si $\esp\left[V^{*}\left(0\right)\right]<\infty$, equivalentemente, si $\esp\left[\int_{0}^{X}V\left(s\right)ds\right]<\infty$,entonces
\begin{eqnarray*}
\frac{\int_{0}^{t}V\left(s\right)ds}{t}\rightarrow\frac{\esp\left[\int_{0}^{X}V\left(s\right)ds\right]}{\esp\left[X\right]}
\end{eqnarray*}
con probabilidad 1 y en media, cuando $t\rightarrow\infty$.
\end{itemize}
\end{Teo}

\begin{Coro}
Sea $\left\{V\left(t\right),t\geq0\right\}$ un proceso regenerativo no negativo, con modificaci\'on medible. Si $\esp <\infty$, $F$ es no-aritm\'etica, y para todo $x\geq0$, $P\left\{V\left(t\right)\leq x,C>x\right\}$ es de variaci\'on acotada como funci\'on de $t$ en cada intervalo finito $\left[0,\tau\right]$, entonces $V\left(t\right)$ converge en distribuci\'on  cuando $t\rightarrow\infty$ y $$\esp V=\frac{\esp \int_{0}^{X}V\left(s\right)ds}{\esp X}$$
Donde $V$ tiene la distribuci\'on l\'imite de $V\left(t\right)$ cuando $t\rightarrow\infty$.

\end{Coro}

Para el caso discreto se tienen resultados similares.



%______________________________________________________________________
%\subsubsection{Procesos de Renovaci\'on}
%______________________________________________________________________

\begin{Def}%\label{Def.Tn}
Sean $0\leq T_{1}\leq T_{2}\leq \ldots$ son tiempos aleatorios infinitos en los cuales ocurren ciertos eventos. El n\'umero de tiempos $T_{n}$ en el intervalo $\left[0,t\right)$ es

\begin{eqnarray}
N\left(t\right)=\sum_{n=1}^{\infty}\indora\left(T_{n}\leq t\right),
\end{eqnarray}
para $t\geq0$.
\end{Def}

Si se consideran los puntos $T_{n}$ como elementos de $\rea_{+}$, y $N\left(t\right)$ es el n\'umero de puntos en $\rea$. El proceso denotado por $\left\{N\left(t\right):t\geq0\right\}$, denotado por $N\left(t\right)$, es un proceso puntual en $\rea_{+}$. Los $T_{n}$ son los tiempos de ocurrencia, el proceso puntual $N\left(t\right)$ es simple si su n\'umero de ocurrencias son distintas: $0<T_{1}<T_{2}<\ldots$ casi seguramente.

\begin{Def}
Un proceso puntual $N\left(t\right)$ es un proceso de renovaci\'on si los tiempos de interocurrencia $\xi_{n}=T_{n}-T_{n-1}$, para $n\geq1$, son independientes e identicamente distribuidos con distribuci\'on $F$, donde $F\left(0\right)=0$ y $T_{0}=0$. Los $T_{n}$ son llamados tiempos de renovaci\'on, referente a la independencia o renovaci\'on de la informaci\'on estoc\'astica en estos tiempos. Los $\xi_{n}$ son los tiempos de inter-renovaci\'on, y $N\left(t\right)$ es el n\'umero de renovaciones en el intervalo $\left[0,t\right)$
\end{Def}


\begin{Note}
Para definir un proceso de renovaci\'on para cualquier contexto, solamente hay que especificar una distribuci\'on $F$, con $F\left(0\right)=0$, para los tiempos de inter-renovaci\'on. La funci\'on $F$ en turno degune las otra variables aleatorias. De manera formal, existe un espacio de probabilidad y una sucesi\'on de variables aleatorias $\xi_{1},\xi_{2},\ldots$ definidas en este con distribuci\'on $F$. Entonces las otras cantidades son $T_{n}=\sum_{k=1}^{n}\xi_{k}$ y $N\left(t\right)=\sum_{n=1}^{\infty}\indora\left(T_{n}\leq t\right)$, donde $T_{n}\rightarrow\infty$ casi seguramente por la Ley Fuerte de los Grandes Números.
\end{Note}

%___________________________________________________________________________________________
%
%\subsubsection{Teorema Principal de Renovaci\'on}
%___________________________________________________________________________________________
%

\begin{Note} Una funci\'on $h:\rea_{+}\rightarrow\rea$ es Directamente Riemann Integrable en los siguientes casos:
\begin{itemize}
\item[a)] $h\left(t\right)\geq0$ es decreciente y Riemann Integrable.
\item[b)] $h$ es continua excepto posiblemente en un conjunto de Lebesgue de medida 0, y $|h\left(t\right)|\leq b\left(t\right)$, donde $b$ es DRI.
\end{itemize}
\end{Note}

\begin{Teo}[Teorema Principal de Renovaci\'on]
Si $F$ es no aritm\'etica y $h\left(t\right)$ es Directamente Riemann Integrable (DRI), entonces

\begin{eqnarray*}
lim_{t\rightarrow\infty}U\star h=\frac{1}{\mu}\int_{\rea_{+}}h\left(s\right)ds.
\end{eqnarray*}
\end{Teo}

\begin{Prop}
Cualquier funci\'on $H\left(t\right)$ acotada en intervalos finitos y que es 0 para $t<0$ puede expresarse como
\begin{eqnarray*}
H\left(t\right)=U\star h\left(t\right)\textrm{,  donde }h\left(t\right)=H\left(t\right)-F\star H\left(t\right)
\end{eqnarray*}
\end{Prop}

\begin{Def}
Un proceso estoc\'astico $X\left(t\right)$ es crudamente regenerativo en un tiempo aleatorio positivo $T$ si
\begin{eqnarray*}
\esp\left[X\left(T+t\right)|T\right]=\esp\left[X\left(t\right)\right]\textrm{, para }t\geq0,\end{eqnarray*}
y con las esperanzas anteriores finitas.
\end{Def}

\begin{Prop}
Sup\'ongase que $X\left(t\right)$ es un proceso crudamente regenerativo en $T$, que tiene distribuci\'on $F$. Si $\esp\left[X\left(t\right)\right]$ es acotado en intervalos finitos, entonces
\begin{eqnarray*}
\esp\left[X\left(t\right)\right]=U\star h\left(t\right)\textrm{,  donde }h\left(t\right)=\esp\left[X\left(t\right)\indora\left(T>t\right)\right].
\end{eqnarray*}
\end{Prop}

\begin{Teo}[Regeneraci\'on Cruda]
Sup\'ongase que $X\left(t\right)$ es un proceso con valores positivo crudamente regenerativo en $T$, y def\'inase $M=\sup\left\{|X\left(t\right)|:t\leq T\right\}$. Si $T$ es no aritm\'etico y $M$ y $MT$ tienen media finita, entonces
\begin{eqnarray*}
lim_{t\rightarrow\infty}\esp\left[X\left(t\right)\right]=\frac{1}{\mu}\int_{\rea_{+}}h\left(s\right)ds,
\end{eqnarray*}
donde $h\left(t\right)=\esp\left[X\left(t\right)\indora\left(T>t\right)\right]$.
\end{Teo}

%___________________________________________________________________________________________
%
\subsection{Propiedades de los Procesos de Renovaci\'on}
%___________________________________________________________________________________________
%

Los tiempos $T_{n}$ est\'an relacionados con los conteos de $N\left(t\right)$ por

\begin{eqnarray*}
\left\{N\left(t\right)\geq n\right\}&=&\left\{T_{n}\leq t\right\}\\
T_{N\left(t\right)}\leq &t&<T_{N\left(t\right)+1},
\end{eqnarray*}

adem\'as $N\left(T_{n}\right)=n$, y 

\begin{eqnarray*}
N\left(t\right)=\max\left\{n:T_{n}\leq t\right\}=\min\left\{n:T_{n+1}>t\right\}
\end{eqnarray*}

Por propiedades de la convoluci\'on se sabe que

\begin{eqnarray*}
P\left\{T_{n}\leq t\right\}=F^{n\star}\left(t\right)
\end{eqnarray*}
que es la $n$-\'esima convoluci\'on de $F$. Entonces 

\begin{eqnarray*}
\left\{N\left(t\right)\geq n\right\}&=&\left\{T_{n}\leq t\right\}\\
P\left\{N\left(t\right)\leq n\right\}&=&1-F^{\left(n+1\right)\star}\left(t\right)
\end{eqnarray*}

Adem\'as usando el hecho de que $\esp\left[N\left(t\right)\right]=\sum_{n=1}^{\infty}P\left\{N\left(t\right)\geq n\right\}$
se tiene que

\begin{eqnarray*}
\esp\left[N\left(t\right)\right]=\sum_{n=1}^{\infty}F^{n\star}\left(t\right)
\end{eqnarray*}

\begin{Prop}
Para cada $t\geq0$, la funci\'on generadora de momentos $\esp\left[e^{\alpha N\left(t\right)}\right]$ existe para alguna $\alpha$ en una vecindad del 0, y de aqu\'i que $\esp\left[N\left(t\right)^{m}\right]<\infty$, para $m\geq1$.
\end{Prop}


\begin{Note}
Si el primer tiempo de renovaci\'on $\xi_{1}$ no tiene la misma distribuci\'on que el resto de las $\xi_{n}$, para $n\geq2$, a $N\left(t\right)$ se le llama Proceso de Renovaci\'on retardado, donde si $\xi$ tiene distribuci\'on $G$, entonces el tiempo $T_{n}$ de la $n$-\'esima renovaci\'on tiene distribuci\'on $G\star F^{\left(n-1\right)\star}\left(t\right)$
\end{Note}


\begin{Teo}
Para una constante $\mu\leq\infty$ ( o variable aleatoria), las siguientes expresiones son equivalentes:

\begin{eqnarray}
lim_{n\rightarrow\infty}n^{-1}T_{n}&=&\mu,\textrm{ c.s.}\\
lim_{t\rightarrow\infty}t^{-1}N\left(t\right)&=&1/\mu,\textrm{ c.s.}
\end{eqnarray}
\end{Teo}


Es decir, $T_{n}$ satisface la Ley Fuerte de los Grandes N\'umeros s\'i y s\'olo s\'i $N\left/t\right)$ la cumple.


\begin{Coro}[Ley Fuerte de los Grandes N\'umeros para Procesos de Renovaci\'on]
Si $N\left(t\right)$ es un proceso de renovaci\'on cuyos tiempos de inter-renovaci\'on tienen media $\mu\leq\infty$, entonces
\begin{eqnarray}
t^{-1}N\left(t\right)\rightarrow 1/\mu,\textrm{ c.s. cuando }t\rightarrow\infty.
\end{eqnarray}

\end{Coro}


Considerar el proceso estoc\'astico de valores reales $\left\{Z\left(t\right):t\geq0\right\}$ en el mismo espacio de probabilidad que $N\left(t\right)$

\begin{Def}
Para el proceso $\left\{Z\left(t\right):t\geq0\right\}$ se define la fluctuaci\'on m\'axima de $Z\left(t\right)$ en el intervalo $\left(T_{n-1},T_{n}\right]$:
\begin{eqnarray*}
M_{n}=\sup_{T_{n-1}<t\leq T_{n}}|Z\left(t\right)-Z\left(T_{n-1}\right)|
\end{eqnarray*}
\end{Def}

\begin{Teo}
Sup\'ongase que $n^{-1}T_{n}\rightarrow\mu$ c.s. cuando $n\rightarrow\infty$, donde $\mu\leq\infty$ es una constante o variable aleatoria. Sea $a$ una constante o variable aleatoria que puede ser infinita cuando $\mu$ es finita, y considere las expresiones l\'imite:
\begin{eqnarray}
lim_{n\rightarrow\infty}n^{-1}Z\left(T_{n}\right)&=&a,\textrm{ c.s.}\\
lim_{t\rightarrow\infty}t^{-1}Z\left(t\right)&=&a/\mu,\textrm{ c.s.}
\end{eqnarray}
La segunda expresi\'on implica la primera. Conversamente, la primera implica la segunda si el proceso $Z\left(t\right)$ es creciente, o si $lim_{n\rightarrow\infty}n^{-1}M_{n}=0$ c.s.
\end{Teo}

\begin{Coro}
Si $N\left(t\right)$ es un proceso de renovaci\'on, y $\left(Z\left(T_{n}\right)-Z\left(T_{n-1}\right),M_{n}\right)$, para $n\geq1$, son variables aleatorias independientes e id\'enticamente distribuidas con media finita, entonces,
\begin{eqnarray}
lim_{t\rightarrow\infty}t^{-1}Z\left(t\right)\rightarrow\frac{\esp\left[Z\left(T_{1}\right)-Z\left(T_{0}\right)\right]}{\esp\left[T_{1}\right]},\textrm{ c.s. cuando  }t\rightarrow\infty.
\end{eqnarray}
\end{Coro}



%___________________________________________________________________________________________
%
%\subsection{Propiedades de los Procesos de Renovaci\'on}
%___________________________________________________________________________________________
%

Los tiempos $T_{n}$ est\'an relacionados con los conteos de $N\left(t\right)$ por

\begin{eqnarray*}
\left\{N\left(t\right)\geq n\right\}&=&\left\{T_{n}\leq t\right\}\\
T_{N\left(t\right)}\leq &t&<T_{N\left(t\right)+1},
\end{eqnarray*}

adem\'as $N\left(T_{n}\right)=n$, y 

\begin{eqnarray*}
N\left(t\right)=\max\left\{n:T_{n}\leq t\right\}=\min\left\{n:T_{n+1}>t\right\}
\end{eqnarray*}

Por propiedades de la convoluci\'on se sabe que

\begin{eqnarray*}
P\left\{T_{n}\leq t\right\}=F^{n\star}\left(t\right)
\end{eqnarray*}
que es la $n$-\'esima convoluci\'on de $F$. Entonces 

\begin{eqnarray*}
\left\{N\left(t\right)\geq n\right\}&=&\left\{T_{n}\leq t\right\}\\
P\left\{N\left(t\right)\leq n\right\}&=&1-F^{\left(n+1\right)\star}\left(t\right)
\end{eqnarray*}

Adem\'as usando el hecho de que $\esp\left[N\left(t\right)\right]=\sum_{n=1}^{\infty}P\left\{N\left(t\right)\geq n\right\}$
se tiene que

\begin{eqnarray*}
\esp\left[N\left(t\right)\right]=\sum_{n=1}^{\infty}F^{n\star}\left(t\right)
\end{eqnarray*}

\begin{Prop}
Para cada $t\geq0$, la funci\'on generadora de momentos $\esp\left[e^{\alpha N\left(t\right)}\right]$ existe para alguna $\alpha$ en una vecindad del 0, y de aqu\'i que $\esp\left[N\left(t\right)^{m}\right]<\infty$, para $m\geq1$.
\end{Prop}


\begin{Note}
Si el primer tiempo de renovaci\'on $\xi_{1}$ no tiene la misma distribuci\'on que el resto de las $\xi_{n}$, para $n\geq2$, a $N\left(t\right)$ se le llama Proceso de Renovaci\'on retardado, donde si $\xi$ tiene distribuci\'on $G$, entonces el tiempo $T_{n}$ de la $n$-\'esima renovaci\'on tiene distribuci\'on $G\star F^{\left(n-1\right)\star}\left(t\right)$
\end{Note}


\begin{Teo}
Para una constante $\mu\leq\infty$ ( o variable aleatoria), las siguientes expresiones son equivalentes:

\begin{eqnarray}
lim_{n\rightarrow\infty}n^{-1}T_{n}&=&\mu,\textrm{ c.s.}\\
lim_{t\rightarrow\infty}t^{-1}N\left(t\right)&=&1/\mu,\textrm{ c.s.}
\end{eqnarray}
\end{Teo}


Es decir, $T_{n}$ satisface la Ley Fuerte de los Grandes N\'umeros s\'i y s\'olo s\'i $N\left/t\right)$ la cumple.


\begin{Coro}[Ley Fuerte de los Grandes N\'umeros para Procesos de Renovaci\'on]
Si $N\left(t\right)$ es un proceso de renovaci\'on cuyos tiempos de inter-renovaci\'on tienen media $\mu\leq\infty$, entonces
\begin{eqnarray}
t^{-1}N\left(t\right)\rightarrow 1/\mu,\textrm{ c.s. cuando }t\rightarrow\infty.
\end{eqnarray}

\end{Coro}


Considerar el proceso estoc\'astico de valores reales $\left\{Z\left(t\right):t\geq0\right\}$ en el mismo espacio de probabilidad que $N\left(t\right)$

\begin{Def}
Para el proceso $\left\{Z\left(t\right):t\geq0\right\}$ se define la fluctuaci\'on m\'axima de $Z\left(t\right)$ en el intervalo $\left(T_{n-1},T_{n}\right]$:
\begin{eqnarray*}
M_{n}=\sup_{T_{n-1}<t\leq T_{n}}|Z\left(t\right)-Z\left(T_{n-1}\right)|
\end{eqnarray*}
\end{Def}

\begin{Teo}
Sup\'ongase que $n^{-1}T_{n}\rightarrow\mu$ c.s. cuando $n\rightarrow\infty$, donde $\mu\leq\infty$ es una constante o variable aleatoria. Sea $a$ una constante o variable aleatoria que puede ser infinita cuando $\mu$ es finita, y considere las expresiones l\'imite:
\begin{eqnarray}
lim_{n\rightarrow\infty}n^{-1}Z\left(T_{n}\right)&=&a,\textrm{ c.s.}\\
lim_{t\rightarrow\infty}t^{-1}Z\left(t\right)&=&a/\mu,\textrm{ c.s.}
\end{eqnarray}
La segunda expresi\'on implica la primera. Conversamente, la primera implica la segunda si el proceso $Z\left(t\right)$ es creciente, o si $lim_{n\rightarrow\infty}n^{-1}M_{n}=0$ c.s.
\end{Teo}

\begin{Coro}
Si $N\left(t\right)$ es un proceso de renovaci\'on, y $\left(Z\left(T_{n}\right)-Z\left(T_{n-1}\right),M_{n}\right)$, para $n\geq1$, son variables aleatorias independientes e id\'enticamente distribuidas con media finita, entonces,
\begin{eqnarray}
lim_{t\rightarrow\infty}t^{-1}Z\left(t\right)\rightarrow\frac{\esp\left[Z\left(T_{1}\right)-Z\left(T_{0}\right)\right]}{\esp\left[T_{1}\right]},\textrm{ c.s. cuando  }t\rightarrow\infty.
\end{eqnarray}
\end{Coro}


%___________________________________________________________________________________________
%
%\subsection{Propiedades de los Procesos de Renovaci\'on}
%___________________________________________________________________________________________
%

Los tiempos $T_{n}$ est\'an relacionados con los conteos de $N\left(t\right)$ por

\begin{eqnarray*}
\left\{N\left(t\right)\geq n\right\}&=&\left\{T_{n}\leq t\right\}\\
T_{N\left(t\right)}\leq &t&<T_{N\left(t\right)+1},
\end{eqnarray*}

adem\'as $N\left(T_{n}\right)=n$, y 

\begin{eqnarray*}
N\left(t\right)=\max\left\{n:T_{n}\leq t\right\}=\min\left\{n:T_{n+1}>t\right\}
\end{eqnarray*}

Por propiedades de la convoluci\'on se sabe que

\begin{eqnarray*}
P\left\{T_{n}\leq t\right\}=F^{n\star}\left(t\right)
\end{eqnarray*}
que es la $n$-\'esima convoluci\'on de $F$. Entonces 

\begin{eqnarray*}
\left\{N\left(t\right)\geq n\right\}&=&\left\{T_{n}\leq t\right\}\\
P\left\{N\left(t\right)\leq n\right\}&=&1-F^{\left(n+1\right)\star}\left(t\right)
\end{eqnarray*}

Adem\'as usando el hecho de que $\esp\left[N\left(t\right)\right]=\sum_{n=1}^{\infty}P\left\{N\left(t\right)\geq n\right\}$
se tiene que

\begin{eqnarray*}
\esp\left[N\left(t\right)\right]=\sum_{n=1}^{\infty}F^{n\star}\left(t\right)
\end{eqnarray*}

\begin{Prop}
Para cada $t\geq0$, la funci\'on generadora de momentos $\esp\left[e^{\alpha N\left(t\right)}\right]$ existe para alguna $\alpha$ en una vecindad del 0, y de aqu\'i que $\esp\left[N\left(t\right)^{m}\right]<\infty$, para $m\geq1$.
\end{Prop}


\begin{Note}
Si el primer tiempo de renovaci\'on $\xi_{1}$ no tiene la misma distribuci\'on que el resto de las $\xi_{n}$, para $n\geq2$, a $N\left(t\right)$ se le llama Proceso de Renovaci\'on retardado, donde si $\xi$ tiene distribuci\'on $G$, entonces el tiempo $T_{n}$ de la $n$-\'esima renovaci\'on tiene distribuci\'on $G\star F^{\left(n-1\right)\star}\left(t\right)$
\end{Note}


\begin{Teo}
Para una constante $\mu\leq\infty$ ( o variable aleatoria), las siguientes expresiones son equivalentes:

\begin{eqnarray}
lim_{n\rightarrow\infty}n^{-1}T_{n}&=&\mu,\textrm{ c.s.}\\
lim_{t\rightarrow\infty}t^{-1}N\left(t\right)&=&1/\mu,\textrm{ c.s.}
\end{eqnarray}
\end{Teo}


Es decir, $T_{n}$ satisface la Ley Fuerte de los Grandes N\'umeros s\'i y s\'olo s\'i $N\left/t\right)$ la cumple.


\begin{Coro}[Ley Fuerte de los Grandes N\'umeros para Procesos de Renovaci\'on]
Si $N\left(t\right)$ es un proceso de renovaci\'on cuyos tiempos de inter-renovaci\'on tienen media $\mu\leq\infty$, entonces
\begin{eqnarray}
t^{-1}N\left(t\right)\rightarrow 1/\mu,\textrm{ c.s. cuando }t\rightarrow\infty.
\end{eqnarray}

\end{Coro}


Considerar el proceso estoc\'astico de valores reales $\left\{Z\left(t\right):t\geq0\right\}$ en el mismo espacio de probabilidad que $N\left(t\right)$

\begin{Def}
Para el proceso $\left\{Z\left(t\right):t\geq0\right\}$ se define la fluctuaci\'on m\'axima de $Z\left(t\right)$ en el intervalo $\left(T_{n-1},T_{n}\right]$:
\begin{eqnarray*}
M_{n}=\sup_{T_{n-1}<t\leq T_{n}}|Z\left(t\right)-Z\left(T_{n-1}\right)|
\end{eqnarray*}
\end{Def}

\begin{Teo}
Sup\'ongase que $n^{-1}T_{n}\rightarrow\mu$ c.s. cuando $n\rightarrow\infty$, donde $\mu\leq\infty$ es una constante o variable aleatoria. Sea $a$ una constante o variable aleatoria que puede ser infinita cuando $\mu$ es finita, y considere las expresiones l\'imite:
\begin{eqnarray}
lim_{n\rightarrow\infty}n^{-1}Z\left(T_{n}\right)&=&a,\textrm{ c.s.}\\
lim_{t\rightarrow\infty}t^{-1}Z\left(t\right)&=&a/\mu,\textrm{ c.s.}
\end{eqnarray}
La segunda expresi\'on implica la primera. Conversamente, la primera implica la segunda si el proceso $Z\left(t\right)$ es creciente, o si $lim_{n\rightarrow\infty}n^{-1}M_{n}=0$ c.s.
\end{Teo}

\begin{Coro}
Si $N\left(t\right)$ es un proceso de renovaci\'on, y $\left(Z\left(T_{n}\right)-Z\left(T_{n-1}\right),M_{n}\right)$, para $n\geq1$, son variables aleatorias independientes e id\'enticamente distribuidas con media finita, entonces,
\begin{eqnarray}
lim_{t\rightarrow\infty}t^{-1}Z\left(t\right)\rightarrow\frac{\esp\left[Z\left(T_{1}\right)-Z\left(T_{0}\right)\right]}{\esp\left[T_{1}\right]},\textrm{ c.s. cuando  }t\rightarrow\infty.
\end{eqnarray}
\end{Coro}

%___________________________________________________________________________________________
%
%\subsection{Propiedades de los Procesos de Renovaci\'on}
%___________________________________________________________________________________________
%

Los tiempos $T_{n}$ est\'an relacionados con los conteos de $N\left(t\right)$ por

\begin{eqnarray*}
\left\{N\left(t\right)\geq n\right\}&=&\left\{T_{n}\leq t\right\}\\
T_{N\left(t\right)}\leq &t&<T_{N\left(t\right)+1},
\end{eqnarray*}

adem\'as $N\left(T_{n}\right)=n$, y 

\begin{eqnarray*}
N\left(t\right)=\max\left\{n:T_{n}\leq t\right\}=\min\left\{n:T_{n+1}>t\right\}
\end{eqnarray*}

Por propiedades de la convoluci\'on se sabe que

\begin{eqnarray*}
P\left\{T_{n}\leq t\right\}=F^{n\star}\left(t\right)
\end{eqnarray*}
que es la $n$-\'esima convoluci\'on de $F$. Entonces 

\begin{eqnarray*}
\left\{N\left(t\right)\geq n\right\}&=&\left\{T_{n}\leq t\right\}\\
P\left\{N\left(t\right)\leq n\right\}&=&1-F^{\left(n+1\right)\star}\left(t\right)
\end{eqnarray*}

Adem\'as usando el hecho de que $\esp\left[N\left(t\right)\right]=\sum_{n=1}^{\infty}P\left\{N\left(t\right)\geq n\right\}$
se tiene que

\begin{eqnarray*}
\esp\left[N\left(t\right)\right]=\sum_{n=1}^{\infty}F^{n\star}\left(t\right)
\end{eqnarray*}

\begin{Prop}
Para cada $t\geq0$, la funci\'on generadora de momentos $\esp\left[e^{\alpha N\left(t\right)}\right]$ existe para alguna $\alpha$ en una vecindad del 0, y de aqu\'i que $\esp\left[N\left(t\right)^{m}\right]<\infty$, para $m\geq1$.
\end{Prop}


\begin{Note}
Si el primer tiempo de renovaci\'on $\xi_{1}$ no tiene la misma distribuci\'on que el resto de las $\xi_{n}$, para $n\geq2$, a $N\left(t\right)$ se le llama Proceso de Renovaci\'on retardado, donde si $\xi$ tiene distribuci\'on $G$, entonces el tiempo $T_{n}$ de la $n$-\'esima renovaci\'on tiene distribuci\'on $G\star F^{\left(n-1\right)\star}\left(t\right)$
\end{Note}


\begin{Teo}
Para una constante $\mu\leq\infty$ ( o variable aleatoria), las siguientes expresiones son equivalentes:

\begin{eqnarray}
lim_{n\rightarrow\infty}n^{-1}T_{n}&=&\mu,\textrm{ c.s.}\\
lim_{t\rightarrow\infty}t^{-1}N\left(t\right)&=&1/\mu,\textrm{ c.s.}
\end{eqnarray}
\end{Teo}


Es decir, $T_{n}$ satisface la Ley Fuerte de los Grandes N\'umeros s\'i y s\'olo s\'i $N\left/t\right)$ la cumple.


\begin{Coro}[Ley Fuerte de los Grandes N\'umeros para Procesos de Renovaci\'on]
Si $N\left(t\right)$ es un proceso de renovaci\'on cuyos tiempos de inter-renovaci\'on tienen media $\mu\leq\infty$, entonces
\begin{eqnarray}
t^{-1}N\left(t\right)\rightarrow 1/\mu,\textrm{ c.s. cuando }t\rightarrow\infty.
\end{eqnarray}

\end{Coro}


Considerar el proceso estoc\'astico de valores reales $\left\{Z\left(t\right):t\geq0\right\}$ en el mismo espacio de probabilidad que $N\left(t\right)$

\begin{Def}
Para el proceso $\left\{Z\left(t\right):t\geq0\right\}$ se define la fluctuaci\'on m\'axima de $Z\left(t\right)$ en el intervalo $\left(T_{n-1},T_{n}\right]$:
\begin{eqnarray*}
M_{n}=\sup_{T_{n-1}<t\leq T_{n}}|Z\left(t\right)-Z\left(T_{n-1}\right)|
\end{eqnarray*}
\end{Def}

\begin{Teo}
Sup\'ongase que $n^{-1}T_{n}\rightarrow\mu$ c.s. cuando $n\rightarrow\infty$, donde $\mu\leq\infty$ es una constante o variable aleatoria. Sea $a$ una constante o variable aleatoria que puede ser infinita cuando $\mu$ es finita, y considere las expresiones l\'imite:
\begin{eqnarray}
lim_{n\rightarrow\infty}n^{-1}Z\left(T_{n}\right)&=&a,\textrm{ c.s.}\\
lim_{t\rightarrow\infty}t^{-1}Z\left(t\right)&=&a/\mu,\textrm{ c.s.}
\end{eqnarray}
La segunda expresi\'on implica la primera. Conversamente, la primera implica la segunda si el proceso $Z\left(t\right)$ es creciente, o si $lim_{n\rightarrow\infty}n^{-1}M_{n}=0$ c.s.
\end{Teo}

\begin{Coro}
Si $N\left(t\right)$ es un proceso de renovaci\'on, y $\left(Z\left(T_{n}\right)-Z\left(T_{n-1}\right),M_{n}\right)$, para $n\geq1$, son variables aleatorias independientes e id\'enticamente distribuidas con media finita, entonces,
\begin{eqnarray}
lim_{t\rightarrow\infty}t^{-1}Z\left(t\right)\rightarrow\frac{\esp\left[Z\left(T_{1}\right)-Z\left(T_{0}\right)\right]}{\esp\left[T_{1}\right]},\textrm{ c.s. cuando  }t\rightarrow\infty.
\end{eqnarray}
\end{Coro}
%___________________________________________________________________________________________
%
%\subsection{Propiedades de los Procesos de Renovaci\'on}
%___________________________________________________________________________________________
%

Los tiempos $T_{n}$ est\'an relacionados con los conteos de $N\left(t\right)$ por

\begin{eqnarray*}
\left\{N\left(t\right)\geq n\right\}&=&\left\{T_{n}\leq t\right\}\\
T_{N\left(t\right)}\leq &t&<T_{N\left(t\right)+1},
\end{eqnarray*}

adem\'as $N\left(T_{n}\right)=n$, y 

\begin{eqnarray*}
N\left(t\right)=\max\left\{n:T_{n}\leq t\right\}=\min\left\{n:T_{n+1}>t\right\}
\end{eqnarray*}

Por propiedades de la convoluci\'on se sabe que

\begin{eqnarray*}
P\left\{T_{n}\leq t\right\}=F^{n\star}\left(t\right)
\end{eqnarray*}
que es la $n$-\'esima convoluci\'on de $F$. Entonces 

\begin{eqnarray*}
\left\{N\left(t\right)\geq n\right\}&=&\left\{T_{n}\leq t\right\}\\
P\left\{N\left(t\right)\leq n\right\}&=&1-F^{\left(n+1\right)\star}\left(t\right)
\end{eqnarray*}

Adem\'as usando el hecho de que $\esp\left[N\left(t\right)\right]=\sum_{n=1}^{\infty}P\left\{N\left(t\right)\geq n\right\}$
se tiene que

\begin{eqnarray*}
\esp\left[N\left(t\right)\right]=\sum_{n=1}^{\infty}F^{n\star}\left(t\right)
\end{eqnarray*}

\begin{Prop}
Para cada $t\geq0$, la funci\'on generadora de momentos $\esp\left[e^{\alpha N\left(t\right)}\right]$ existe para alguna $\alpha$ en una vecindad del 0, y de aqu\'i que $\esp\left[N\left(t\right)^{m}\right]<\infty$, para $m\geq1$.
\end{Prop}


\begin{Note}
Si el primer tiempo de renovaci\'on $\xi_{1}$ no tiene la misma distribuci\'on que el resto de las $\xi_{n}$, para $n\geq2$, a $N\left(t\right)$ se le llama Proceso de Renovaci\'on retardado, donde si $\xi$ tiene distribuci\'on $G$, entonces el tiempo $T_{n}$ de la $n$-\'esima renovaci\'on tiene distribuci\'on $G\star F^{\left(n-1\right)\star}\left(t\right)$
\end{Note}


\begin{Teo}
Para una constante $\mu\leq\infty$ ( o variable aleatoria), las siguientes expresiones son equivalentes:

\begin{eqnarray}
lim_{n\rightarrow\infty}n^{-1}T_{n}&=&\mu,\textrm{ c.s.}\\
lim_{t\rightarrow\infty}t^{-1}N\left(t\right)&=&1/\mu,\textrm{ c.s.}
\end{eqnarray}
\end{Teo}


Es decir, $T_{n}$ satisface la Ley Fuerte de los Grandes N\'umeros s\'i y s\'olo s\'i $N\left/t\right)$ la cumple.


\begin{Coro}[Ley Fuerte de los Grandes N\'umeros para Procesos de Renovaci\'on]
Si $N\left(t\right)$ es un proceso de renovaci\'on cuyos tiempos de inter-renovaci\'on tienen media $\mu\leq\infty$, entonces
\begin{eqnarray}
t^{-1}N\left(t\right)\rightarrow 1/\mu,\textrm{ c.s. cuando }t\rightarrow\infty.
\end{eqnarray}

\end{Coro}


Considerar el proceso estoc\'astico de valores reales $\left\{Z\left(t\right):t\geq0\right\}$ en el mismo espacio de probabilidad que $N\left(t\right)$

\begin{Def}
Para el proceso $\left\{Z\left(t\right):t\geq0\right\}$ se define la fluctuaci\'on m\'axima de $Z\left(t\right)$ en el intervalo $\left(T_{n-1},T_{n}\right]$:
\begin{eqnarray*}
M_{n}=\sup_{T_{n-1}<t\leq T_{n}}|Z\left(t\right)-Z\left(T_{n-1}\right)|
\end{eqnarray*}
\end{Def}

\begin{Teo}
Sup\'ongase que $n^{-1}T_{n}\rightarrow\mu$ c.s. cuando $n\rightarrow\infty$, donde $\mu\leq\infty$ es una constante o variable aleatoria. Sea $a$ una constante o variable aleatoria que puede ser infinita cuando $\mu$ es finita, y considere las expresiones l\'imite:
\begin{eqnarray}
lim_{n\rightarrow\infty}n^{-1}Z\left(T_{n}\right)&=&a,\textrm{ c.s.}\\
lim_{t\rightarrow\infty}t^{-1}Z\left(t\right)&=&a/\mu,\textrm{ c.s.}
\end{eqnarray}
La segunda expresi\'on implica la primera. Conversamente, la primera implica la segunda si el proceso $Z\left(t\right)$ es creciente, o si $lim_{n\rightarrow\infty}n^{-1}M_{n}=0$ c.s.
\end{Teo}

\begin{Coro}
Si $N\left(t\right)$ es un proceso de renovaci\'on, y $\left(Z\left(T_{n}\right)-Z\left(T_{n-1}\right),M_{n}\right)$, para $n\geq1$, son variables aleatorias independientes e id\'enticamente distribuidas con media finita, entonces,
\begin{eqnarray}
lim_{t\rightarrow\infty}t^{-1}Z\left(t\right)\rightarrow\frac{\esp\left[Z\left(T_{1}\right)-Z\left(T_{0}\right)\right]}{\esp\left[T_{1}\right]},\textrm{ c.s. cuando  }t\rightarrow\infty.
\end{eqnarray}
\end{Coro}


%___________________________________________________________________________________________
%
\subsection{Funci\'on de Renovaci\'on}
%___________________________________________________________________________________________
%


\begin{Def}
Sea $h\left(t\right)$ funci\'on de valores reales en $\rea$ acotada en intervalos finitos e igual a cero para $t<0$ La ecuaci\'on de renovaci\'on para $h\left(t\right)$ y la distribuci\'on $F$ es

\begin{eqnarray}%\label{Ec.Renovacion}
H\left(t\right)=h\left(t\right)+\int_{\left[0,t\right]}H\left(t-s\right)dF\left(s\right)\textrm{,    }t\geq0,
\end{eqnarray}
donde $H\left(t\right)$ es una funci\'on de valores reales. Esto es $H=h+F\star H$. Decimos que $H\left(t\right)$ es soluci\'on de esta ecuaci\'on si satisface la ecuaci\'on, y es acotada en intervalos finitos e iguales a cero para $t<0$.
\end{Def}

\begin{Prop}
La funci\'on $U\star h\left(t\right)$ es la \'unica soluci\'on de la ecuaci\'on de renovaci\'on (\ref{Ec.Renovacion}).
\end{Prop}

\begin{Teo}[Teorema Renovaci\'on Elemental]
\begin{eqnarray*}
t^{-1}U\left(t\right)\rightarrow 1/\mu\textrm{,    cuando }t\rightarrow\infty.
\end{eqnarray*}
\end{Teo}

%___________________________________________________________________________________________
%
%\subsection{Funci\'on de Renovaci\'on}
%___________________________________________________________________________________________
%


Sup\'ongase que $N\left(t\right)$ es un proceso de renovaci\'on con distribuci\'on $F$ con media finita $\mu$.

\begin{Def}
La funci\'on de renovaci\'on asociada con la distribuci\'on $F$, del proceso $N\left(t\right)$, es
\begin{eqnarray*}
U\left(t\right)=\sum_{n=1}^{\infty}F^{n\star}\left(t\right),\textrm{   }t\geq0,
\end{eqnarray*}
donde $F^{0\star}\left(t\right)=\indora\left(t\geq0\right)$.
\end{Def}


\begin{Prop}
Sup\'ongase que la distribuci\'on de inter-renovaci\'on $F$ tiene densidad $f$. Entonces $U\left(t\right)$ tambi\'en tiene densidad, para $t>0$, y es $U^{'}\left(t\right)=\sum_{n=0}^{\infty}f^{n\star}\left(t\right)$. Adem\'as
\begin{eqnarray*}
\prob\left\{N\left(t\right)>N\left(t-\right)\right\}=0\textrm{,   }t\geq0.
\end{eqnarray*}
\end{Prop}

\begin{Def}
La Transformada de Laplace-Stieljes de $F$ est\'a dada por

\begin{eqnarray*}
\hat{F}\left(\alpha\right)=\int_{\rea_{+}}e^{-\alpha t}dF\left(t\right)\textrm{,  }\alpha\geq0.
\end{eqnarray*}
\end{Def}

Entonces

\begin{eqnarray*}
\hat{U}\left(\alpha\right)=\sum_{n=0}^{\infty}\hat{F^{n\star}}\left(\alpha\right)=\sum_{n=0}^{\infty}\hat{F}\left(\alpha\right)^{n}=\frac{1}{1-\hat{F}\left(\alpha\right)}.
\end{eqnarray*}


\begin{Prop}
La Transformada de Laplace $\hat{U}\left(\alpha\right)$ y $\hat{F}\left(\alpha\right)$ determina una a la otra de manera \'unica por la relaci\'on $\hat{U}\left(\alpha\right)=\frac{1}{1-\hat{F}\left(\alpha\right)}$.
\end{Prop}


\begin{Note}
Un proceso de renovaci\'on $N\left(t\right)$ cuyos tiempos de inter-renovaci\'on tienen media finita, es un proceso Poisson con tasa $\lambda$ si y s\'olo s\'i $\esp\left[U\left(t\right)\right]=\lambda t$, para $t\geq0$.
\end{Note}


\begin{Teo}
Sea $N\left(t\right)$ un proceso puntual simple con puntos de localizaci\'on $T_{n}$ tal que $\eta\left(t\right)=\esp\left[N\left(\right)\right]$ es finita para cada $t$. Entonces para cualquier funci\'on $f:\rea_{+}\rightarrow\rea$,
\begin{eqnarray*}
\esp\left[\sum_{n=1}^{N\left(\right)}f\left(T_{n}\right)\right]=\int_{\left(0,t\right]}f\left(s\right)d\eta\left(s\right)\textrm{,  }t\geq0,
\end{eqnarray*}
suponiendo que la integral exista. Adem\'as si $X_{1},X_{2},\ldots$ son variables aleatorias definidas en el mismo espacio de probabilidad que el proceso $N\left(t\right)$ tal que $\esp\left[X_{n}|T_{n}=s\right]=f\left(s\right)$, independiente de $n$. Entonces
\begin{eqnarray*}
\esp\left[\sum_{n=1}^{N\left(t\right)}X_{n}\right]=\int_{\left(0,t\right]}f\left(s\right)d\eta\left(s\right)\textrm{,  }t\geq0,
\end{eqnarray*} 
suponiendo que la integral exista. 
\end{Teo}

\begin{Coro}[Identidad de Wald para Renovaciones]
Para el proceso de renovaci\'on $N\left(t\right)$,
\begin{eqnarray*}
\esp\left[T_{N\left(t\right)+1}\right]=\mu\esp\left[N\left(t\right)+1\right]\textrm{,  }t\geq0,
\end{eqnarray*}  
\end{Coro}

%______________________________________________________________________
%\subsection{Procesos de Renovaci\'on}
%______________________________________________________________________

\begin{Def}%\label{Def.Tn}
Sean $0\leq T_{1}\leq T_{2}\leq \ldots$ son tiempos aleatorios infinitos en los cuales ocurren ciertos eventos. El n\'umero de tiempos $T_{n}$ en el intervalo $\left[0,t\right)$ es

\begin{eqnarray}
N\left(t\right)=\sum_{n=1}^{\infty}\indora\left(T_{n}\leq t\right),
\end{eqnarray}
para $t\geq0$.
\end{Def}

Si se consideran los puntos $T_{n}$ como elementos de $\rea_{+}$, y $N\left(t\right)$ es el n\'umero de puntos en $\rea$. El proceso denotado por $\left\{N\left(t\right):t\geq0\right\}$, denotado por $N\left(t\right)$, es un proceso puntual en $\rea_{+}$. Los $T_{n}$ son los tiempos de ocurrencia, el proceso puntual $N\left(t\right)$ es simple si su n\'umero de ocurrencias son distintas: $0<T_{1}<T_{2}<\ldots$ casi seguramente.

\begin{Def}
Un proceso puntual $N\left(t\right)$ es un proceso de renovaci\'on si los tiempos de interocurrencia $\xi_{n}=T_{n}-T_{n-1}$, para $n\geq1$, son independientes e identicamente distribuidos con distribuci\'on $F$, donde $F\left(0\right)=0$ y $T_{0}=0$. Los $T_{n}$ son llamados tiempos de renovaci\'on, referente a la independencia o renovaci\'on de la informaci\'on estoc\'astica en estos tiempos. Los $\xi_{n}$ son los tiempos de inter-renovaci\'on, y $N\left(t\right)$ es el n\'umero de renovaciones en el intervalo $\left[0,t\right)$
\end{Def}


\begin{Note}
Para definir un proceso de renovaci\'on para cualquier contexto, solamente hay que especificar una distribuci\'on $F$, con $F\left(0\right)=0$, para los tiempos de inter-renovaci\'on. La funci\'on $F$ en turno degune las otra variables aleatorias. De manera formal, existe un espacio de probabilidad y una sucesi\'on de variables aleatorias $\xi_{1},\xi_{2},\ldots$ definidas en este con distribuci\'on $F$. Entonces las otras cantidades son $T_{n}=\sum_{k=1}^{n}\xi_{k}$ y $N\left(t\right)=\sum_{n=1}^{\infty}\indora\left(T_{n}\leq t\right)$, donde $T_{n}\rightarrow\infty$ casi seguramente por la Ley Fuerte de los Grandes Números.
\end{Note}

%___________________________________________________________________________________________
%
\subsection{Renewal and Regenerative Processes: Serfozo\cite{Serfozo}}
%___________________________________________________________________________________________
%
\begin{Def}%\label{Def.Tn}
Sean $0\leq T_{1}\leq T_{2}\leq \ldots$ son tiempos aleatorios infinitos en los cuales ocurren ciertos eventos. El n\'umero de tiempos $T_{n}$ en el intervalo $\left[0,t\right)$ es

\begin{eqnarray}
N\left(t\right)=\sum_{n=1}^{\infty}\indora\left(T_{n}\leq t\right),
\end{eqnarray}
para $t\geq0$.
\end{Def}

Si se consideran los puntos $T_{n}$ como elementos de $\rea_{+}$, y $N\left(t\right)$ es el n\'umero de puntos en $\rea$. El proceso denotado por $\left\{N\left(t\right):t\geq0\right\}$, denotado por $N\left(t\right)$, es un proceso puntual en $\rea_{+}$. Los $T_{n}$ son los tiempos de ocurrencia, el proceso puntual $N\left(t\right)$ es simple si su n\'umero de ocurrencias son distintas: $0<T_{1}<T_{2}<\ldots$ casi seguramente.

\begin{Def}
Un proceso puntual $N\left(t\right)$ es un proceso de renovaci\'on si los tiempos de interocurrencia $\xi_{n}=T_{n}-T_{n-1}$, para $n\geq1$, son independientes e identicamente distribuidos con distribuci\'on $F$, donde $F\left(0\right)=0$ y $T_{0}=0$. Los $T_{n}$ son llamados tiempos de renovaci\'on, referente a la independencia o renovaci\'on de la informaci\'on estoc\'astica en estos tiempos. Los $\xi_{n}$ son los tiempos de inter-renovaci\'on, y $N\left(t\right)$ es el n\'umero de renovaciones en el intervalo $\left[0,t\right)$
\end{Def}


\begin{Note}
Para definir un proceso de renovaci\'on para cualquier contexto, solamente hay que especificar una distribuci\'on $F$, con $F\left(0\right)=0$, para los tiempos de inter-renovaci\'on. La funci\'on $F$ en turno degune las otra variables aleatorias. De manera formal, existe un espacio de probabilidad y una sucesi\'on de variables aleatorias $\xi_{1},\xi_{2},\ldots$ definidas en este con distribuci\'on $F$. Entonces las otras cantidades son $T_{n}=\sum_{k=1}^{n}\xi_{k}$ y $N\left(t\right)=\sum_{n=1}^{\infty}\indora\left(T_{n}\leq t\right)$, donde $T_{n}\rightarrow\infty$ casi seguramente por la Ley Fuerte de los Grandes N\'umeros.
\end{Note}

Los tiempos $T_{n}$ est\'an relacionados con los conteos de $N\left(t\right)$ por

\begin{eqnarray*}
\left\{N\left(t\right)\geq n\right\}&=&\left\{T_{n}\leq t\right\}\\
T_{N\left(t\right)}\leq &t&<T_{N\left(t\right)+1},
\end{eqnarray*}

adem\'as $N\left(T_{n}\right)=n$, y 

\begin{eqnarray*}
N\left(t\right)=\max\left\{n:T_{n}\leq t\right\}=\min\left\{n:T_{n+1}>t\right\}
\end{eqnarray*}

Por propiedades de la convoluci\'on se sabe que

\begin{eqnarray*}
P\left\{T_{n}\leq t\right\}=F^{n\star}\left(t\right)
\end{eqnarray*}
que es la $n$-\'esima convoluci\'on de $F$. Entonces 

\begin{eqnarray*}
\left\{N\left(t\right)\geq n\right\}&=&\left\{T_{n}\leq t\right\}\\
P\left\{N\left(t\right)\leq n\right\}&=&1-F^{\left(n+1\right)\star}\left(t\right)
\end{eqnarray*}

Adem\'as usando el hecho de que $\esp\left[N\left(t\right)\right]=\sum_{n=1}^{\infty}P\left\{N\left(t\right)\geq n\right\}$
se tiene que

\begin{eqnarray*}
\esp\left[N\left(t\right)\right]=\sum_{n=1}^{\infty}F^{n\star}\left(t\right)
\end{eqnarray*}

\begin{Prop}
Para cada $t\geq0$, la funci\'on generadora de momentos $\esp\left[e^{\alpha N\left(t\right)}\right]$ existe para alguna $\alpha$ en una vecindad del 0, y de aqu\'i que $\esp\left[N\left(t\right)^{m}\right]<\infty$, para $m\geq1$.
\end{Prop}

\begin{Ejem}[\textbf{Proceso Poisson}]

Suponga que se tienen tiempos de inter-renovaci\'on \textit{i.i.d.} del proceso de renovaci\'on $N\left(t\right)$ tienen distribuci\'on exponencial $F\left(t\right)=q-e^{-\lambda t}$ con tasa $\lambda$. Entonces $N\left(t\right)$ es un proceso Poisson con tasa $\lambda$.

\end{Ejem}


\begin{Note}
Si el primer tiempo de renovaci\'on $\xi_{1}$ no tiene la misma distribuci\'on que el resto de las $\xi_{n}$, para $n\geq2$, a $N\left(t\right)$ se le llama Proceso de Renovaci\'on retardado, donde si $\xi$ tiene distribuci\'on $G$, entonces el tiempo $T_{n}$ de la $n$-\'esima renovaci\'on tiene distribuci\'on $G\star F^{\left(n-1\right)\star}\left(t\right)$
\end{Note}


\begin{Teo}
Para una constante $\mu\leq\infty$ ( o variable aleatoria), las siguientes expresiones son equivalentes:

\begin{eqnarray}
lim_{n\rightarrow\infty}n^{-1}T_{n}&=&\mu,\textrm{ c.s.}\\
lim_{t\rightarrow\infty}t^{-1}N\left(t\right)&=&1/\mu,\textrm{ c.s.}
\end{eqnarray}
\end{Teo}


Es decir, $T_{n}$ satisface la Ley Fuerte de los Grandes N\'umeros s\'i y s\'olo s\'i $N\left/t\right)$ la cumple.


\begin{Coro}[Ley Fuerte de los Grandes N\'umeros para Procesos de Renovaci\'on]
Si $N\left(t\right)$ es un proceso de renovaci\'on cuyos tiempos de inter-renovaci\'on tienen media $\mu\leq\infty$, entonces
\begin{eqnarray}
t^{-1}N\left(t\right)\rightarrow 1/\mu,\textrm{ c.s. cuando }t\rightarrow\infty.
\end{eqnarray}

\end{Coro}


Considerar el proceso estoc\'astico de valores reales $\left\{Z\left(t\right):t\geq0\right\}$ en el mismo espacio de probabilidad que $N\left(t\right)$

\begin{Def}
Para el proceso $\left\{Z\left(t\right):t\geq0\right\}$ se define la fluctuaci\'on m\'axima de $Z\left(t\right)$ en el intervalo $\left(T_{n-1},T_{n}\right]$:
\begin{eqnarray*}
M_{n}=\sup_{T_{n-1}<t\leq T_{n}}|Z\left(t\right)-Z\left(T_{n-1}\right)|
\end{eqnarray*}
\end{Def}

\begin{Teo}
Sup\'ongase que $n^{-1}T_{n}\rightarrow\mu$ c.s. cuando $n\rightarrow\infty$, donde $\mu\leq\infty$ es una constante o variable aleatoria. Sea $a$ una constante o variable aleatoria que puede ser infinita cuando $\mu$ es finita, y considere las expresiones l\'imite:
\begin{eqnarray}
lim_{n\rightarrow\infty}n^{-1}Z\left(T_{n}\right)&=&a,\textrm{ c.s.}\\
lim_{t\rightarrow\infty}t^{-1}Z\left(t\right)&=&a/\mu,\textrm{ c.s.}
\end{eqnarray}
La segunda expresi\'on implica la primera. Conversamente, la primera implica la segunda si el proceso $Z\left(t\right)$ es creciente, o si $lim_{n\rightarrow\infty}n^{-1}M_{n}=0$ c.s.
\end{Teo}

\begin{Coro}
Si $N\left(t\right)$ es un proceso de renovaci\'on, y $\left(Z\left(T_{n}\right)-Z\left(T_{n-1}\right),M_{n}\right)$, para $n\geq1$, son variables aleatorias independientes e id\'enticamente distribuidas con media finita, entonces,
\begin{eqnarray}
lim_{t\rightarrow\infty}t^{-1}Z\left(t\right)\rightarrow\frac{\esp\left[Z\left(T_{1}\right)-Z\left(T_{0}\right)\right]}{\esp\left[T_{1}\right]},\textrm{ c.s. cuando  }t\rightarrow\infty.
\end{eqnarray}
\end{Coro}


Sup\'ongase que $N\left(t\right)$ es un proceso de renovaci\'on con distribuci\'on $F$ con media finita $\mu$.

\begin{Def}
La funci\'on de renovaci\'on asociada con la distribuci\'on $F$, del proceso $N\left(t\right)$, es
\begin{eqnarray*}
U\left(t\right)=\sum_{n=1}^{\infty}F^{n\star}\left(t\right),\textrm{   }t\geq0,
\end{eqnarray*}
donde $F^{0\star}\left(t\right)=\indora\left(t\geq0\right)$.
\end{Def}


\begin{Prop}
Sup\'ongase que la distribuci\'on de inter-renovaci\'on $F$ tiene densidad $f$. Entonces $U\left(t\right)$ tambi\'en tiene densidad, para $t>0$, y es $U^{'}\left(t\right)=\sum_{n=0}^{\infty}f^{n\star}\left(t\right)$. Adem\'as
\begin{eqnarray*}
\prob\left\{N\left(t\right)>N\left(t-\right)\right\}=0\textrm{,   }t\geq0.
\end{eqnarray*}
\end{Prop}

\begin{Def}
La Transformada de Laplace-Stieljes de $F$ est\'a dada por

\begin{eqnarray*}
\hat{F}\left(\alpha\right)=\int_{\rea_{+}}e^{-\alpha t}dF\left(t\right)\textrm{,  }\alpha\geq0.
\end{eqnarray*}
\end{Def}

Entonces

\begin{eqnarray*}
\hat{U}\left(\alpha\right)=\sum_{n=0}^{\infty}\hat{F^{n\star}}\left(\alpha\right)=\sum_{n=0}^{\infty}\hat{F}\left(\alpha\right)^{n}=\frac{1}{1-\hat{F}\left(\alpha\right)}.
\end{eqnarray*}


\begin{Prop}
La Transformada de Laplace $\hat{U}\left(\alpha\right)$ y $\hat{F}\left(\alpha\right)$ determina una a la otra de manera \'unica por la relaci\'on $\hat{U}\left(\alpha\right)=\frac{1}{1-\hat{F}\left(\alpha\right)}$.
\end{Prop}


\begin{Note}
Un proceso de renovaci\'on $N\left(t\right)$ cuyos tiempos de inter-renovaci\'on tienen media finita, es un proceso Poisson con tasa $\lambda$ si y s\'olo s\'i $\esp\left[U\left(t\right)\right]=\lambda t$, para $t\geq0$.
\end{Note}


\begin{Teo}
Sea $N\left(t\right)$ un proceso puntual simple con puntos de localizaci\'on $T_{n}$ tal que $\eta\left(t\right)=\esp\left[N\left(\right)\right]$ es finita para cada $t$. Entonces para cualquier funci\'on $f:\rea_{+}\rightarrow\rea$,
\begin{eqnarray*}
\esp\left[\sum_{n=1}^{N\left(\right)}f\left(T_{n}\right)\right]=\int_{\left(0,t\right]}f\left(s\right)d\eta\left(s\right)\textrm{,  }t\geq0,
\end{eqnarray*}
suponiendo que la integral exista. Adem\'as si $X_{1},X_{2},\ldots$ son variables aleatorias definidas en el mismo espacio de probabilidad que el proceso $N\left(t\right)$ tal que $\esp\left[X_{n}|T_{n}=s\right]=f\left(s\right)$, independiente de $n$. Entonces
\begin{eqnarray*}
\esp\left[\sum_{n=1}^{N\left(t\right)}X_{n}\right]=\int_{\left(0,t\right]}f\left(s\right)d\eta\left(s\right)\textrm{,  }t\geq0,
\end{eqnarray*} 
suponiendo que la integral exista. 
\end{Teo}

\begin{Coro}[Identidad de Wald para Renovaciones]
Para el proceso de renovaci\'on $N\left(t\right)$,
\begin{eqnarray*}
\esp\left[T_{N\left(t\right)+1}\right]=\mu\esp\left[N\left(t\right)+1\right]\textrm{,  }t\geq0,
\end{eqnarray*}  
\end{Coro}


\begin{Def}
Sea $h\left(t\right)$ funci\'on de valores reales en $\rea$ acotada en intervalos finitos e igual a cero para $t<0$ La ecuaci\'on de renovaci\'on para $h\left(t\right)$ y la distribuci\'on $F$ es

\begin{eqnarray}%\label{Ec.Renovacion}
H\left(t\right)=h\left(t\right)+\int_{\left[0,t\right]}H\left(t-s\right)dF\left(s\right)\textrm{,    }t\geq0,
\end{eqnarray}
donde $H\left(t\right)$ es una funci\'on de valores reales. Esto es $H=h+F\star H$. Decimos que $H\left(t\right)$ es soluci\'on de esta ecuaci\'on si satisface la ecuaci\'on, y es acotada en intervalos finitos e iguales a cero para $t<0$.
\end{Def}

\begin{Prop}
La funci\'on $U\star h\left(t\right)$ es la \'unica soluci\'on de la ecuaci\'on de renovaci\'on (\ref{Ec.Renovacion}).
\end{Prop}

\begin{Teo}[Teorema Renovaci\'on Elemental]
\begin{eqnarray*}
t^{-1}U\left(t\right)\rightarrow 1/\mu\textrm{,    cuando }t\rightarrow\infty.
\end{eqnarray*}
\end{Teo}



Sup\'ongase que $N\left(t\right)$ es un proceso de renovaci\'on con distribuci\'on $F$ con media finita $\mu$.

\begin{Def}
La funci\'on de renovaci\'on asociada con la distribuci\'on $F$, del proceso $N\left(t\right)$, es
\begin{eqnarray*}
U\left(t\right)=\sum_{n=1}^{\infty}F^{n\star}\left(t\right),\textrm{   }t\geq0,
\end{eqnarray*}
donde $F^{0\star}\left(t\right)=\indora\left(t\geq0\right)$.
\end{Def}


\begin{Prop}
Sup\'ongase que la distribuci\'on de inter-renovaci\'on $F$ tiene densidad $f$. Entonces $U\left(t\right)$ tambi\'en tiene densidad, para $t>0$, y es $U^{'}\left(t\right)=\sum_{n=0}^{\infty}f^{n\star}\left(t\right)$. Adem\'as
\begin{eqnarray*}
\prob\left\{N\left(t\right)>N\left(t-\right)\right\}=0\textrm{,   }t\geq0.
\end{eqnarray*}
\end{Prop}

\begin{Def}
La Transformada de Laplace-Stieljes de $F$ est\'a dada por

\begin{eqnarray*}
\hat{F}\left(\alpha\right)=\int_{\rea_{+}}e^{-\alpha t}dF\left(t\right)\textrm{,  }\alpha\geq0.
\end{eqnarray*}
\end{Def}

Entonces

\begin{eqnarray*}
\hat{U}\left(\alpha\right)=\sum_{n=0}^{\infty}\hat{F^{n\star}}\left(\alpha\right)=\sum_{n=0}^{\infty}\hat{F}\left(\alpha\right)^{n}=\frac{1}{1-\hat{F}\left(\alpha\right)}.
\end{eqnarray*}


\begin{Prop}
La Transformada de Laplace $\hat{U}\left(\alpha\right)$ y $\hat{F}\left(\alpha\right)$ determina una a la otra de manera \'unica por la relaci\'on $\hat{U}\left(\alpha\right)=\frac{1}{1-\hat{F}\left(\alpha\right)}$.
\end{Prop}


\begin{Note}
Un proceso de renovaci\'on $N\left(t\right)$ cuyos tiempos de inter-renovaci\'on tienen media finita, es un proceso Poisson con tasa $\lambda$ si y s\'olo s\'i $\esp\left[U\left(t\right)\right]=\lambda t$, para $t\geq0$.
\end{Note}


\begin{Teo}
Sea $N\left(t\right)$ un proceso puntual simple con puntos de localizaci\'on $T_{n}$ tal que $\eta\left(t\right)=\esp\left[N\left(\right)\right]$ es finita para cada $t$. Entonces para cualquier funci\'on $f:\rea_{+}\rightarrow\rea$,
\begin{eqnarray*}
\esp\left[\sum_{n=1}^{N\left(\right)}f\left(T_{n}\right)\right]=\int_{\left(0,t\right]}f\left(s\right)d\eta\left(s\right)\textrm{,  }t\geq0,
\end{eqnarray*}
suponiendo que la integral exista. Adem\'as si $X_{1},X_{2},\ldots$ son variables aleatorias definidas en el mismo espacio de probabilidad que el proceso $N\left(t\right)$ tal que $\esp\left[X_{n}|T_{n}=s\right]=f\left(s\right)$, independiente de $n$. Entonces
\begin{eqnarray*}
\esp\left[\sum_{n=1}^{N\left(t\right)}X_{n}\right]=\int_{\left(0,t\right]}f\left(s\right)d\eta\left(s\right)\textrm{,  }t\geq0,
\end{eqnarray*} 
suponiendo que la integral exista. 
\end{Teo}

\begin{Coro}[Identidad de Wald para Renovaciones]
Para el proceso de renovaci\'on $N\left(t\right)$,
\begin{eqnarray*}
\esp\left[T_{N\left(t\right)+1}\right]=\mu\esp\left[N\left(t\right)+1\right]\textrm{,  }t\geq0,
\end{eqnarray*}  
\end{Coro}


\begin{Def}
Sea $h\left(t\right)$ funci\'on de valores reales en $\rea$ acotada en intervalos finitos e igual a cero para $t<0$ La ecuaci\'on de renovaci\'on para $h\left(t\right)$ y la distribuci\'on $F$ es

\begin{eqnarray}%\label{Ec.Renovacion}
H\left(t\right)=h\left(t\right)+\int_{\left[0,t\right]}H\left(t-s\right)dF\left(s\right)\textrm{,    }t\geq0,
\end{eqnarray}
donde $H\left(t\right)$ es una funci\'on de valores reales. Esto es $H=h+F\star H$. Decimos que $H\left(t\right)$ es soluci\'on de esta ecuaci\'on si satisface la ecuaci\'on, y es acotada en intervalos finitos e iguales a cero para $t<0$.
\end{Def}

\begin{Prop}
La funci\'on $U\star h\left(t\right)$ es la \'unica soluci\'on de la ecuaci\'on de renovaci\'on (\ref{Ec.Renovacion}).
\end{Prop}

\begin{Teo}[Teorema Renovaci\'on Elemental]
\begin{eqnarray*}
t^{-1}U\left(t\right)\rightarrow 1/\mu\textrm{,    cuando }t\rightarrow\infty.
\end{eqnarray*}
\end{Teo}


\begin{Note} Una funci\'on $h:\rea_{+}\rightarrow\rea$ es Directamente Riemann Integrable en los siguientes casos:
\begin{itemize}
\item[a)] $h\left(t\right)\geq0$ es decreciente y Riemann Integrable.
\item[b)] $h$ es continua excepto posiblemente en un conjunto de Lebesgue de medida 0, y $|h\left(t\right)|\leq b\left(t\right)$, donde $b$ es DRI.
\end{itemize}
\end{Note}

\begin{Teo}[Teorema Principal de Renovaci\'on]
Si $F$ es no aritm\'etica y $h\left(t\right)$ es Directamente Riemann Integrable (DRI), entonces

\begin{eqnarray*}
lim_{t\rightarrow\infty}U\star h=\frac{1}{\mu}\int_{\rea_{+}}h\left(s\right)ds.
\end{eqnarray*}
\end{Teo}

\begin{Prop}
Cualquier funci\'on $H\left(t\right)$ acotada en intervalos finitos y que es 0 para $t<0$ puede expresarse como
\begin{eqnarray*}
H\left(t\right)=U\star h\left(t\right)\textrm{,  donde }h\left(t\right)=H\left(t\right)-F\star H\left(t\right)
\end{eqnarray*}
\end{Prop}

\begin{Def}
Un proceso estoc\'astico $X\left(t\right)$ es crudamente regenerativo en un tiempo aleatorio positivo $T$ si
\begin{eqnarray*}
\esp\left[X\left(T+t\right)|T\right]=\esp\left[X\left(t\right)\right]\textrm{, para }t\geq0,\end{eqnarray*}
y con las esperanzas anteriores finitas.
\end{Def}

\begin{Prop}
Sup\'ongase que $X\left(t\right)$ es un proceso crudamente regenerativo en $T$, que tiene distribuci\'on $F$. Si $\esp\left[X\left(t\right)\right]$ es acotado en intervalos finitos, entonces
\begin{eqnarray*}
\esp\left[X\left(t\right)\right]=U\star h\left(t\right)\textrm{,  donde }h\left(t\right)=\esp\left[X\left(t\right)\indora\left(T>t\right)\right].
\end{eqnarray*}
\end{Prop}

\begin{Teo}[Regeneraci\'on Cruda]
Sup\'ongase que $X\left(t\right)$ es un proceso con valores positivo crudamente regenerativo en $T$, y def\'inase $M=\sup\left\{|X\left(t\right)|:t\leq T\right\}$. Si $T$ es no aritm\'etico y $M$ y $MT$ tienen media finita, entonces
\begin{eqnarray*}
lim_{t\rightarrow\infty}\esp\left[X\left(t\right)\right]=\frac{1}{\mu}\int_{\rea_{+}}h\left(s\right)ds,
\end{eqnarray*}
donde $h\left(t\right)=\esp\left[X\left(t\right)\indora\left(T>t\right)\right]$.
\end{Teo}


\begin{Note} Una funci\'on $h:\rea_{+}\rightarrow\rea$ es Directamente Riemann Integrable en los siguientes casos:
\begin{itemize}
\item[a)] $h\left(t\right)\geq0$ es decreciente y Riemann Integrable.
\item[b)] $h$ es continua excepto posiblemente en un conjunto de Lebesgue de medida 0, y $|h\left(t\right)|\leq b\left(t\right)$, donde $b$ es DRI.
\end{itemize}
\end{Note}

\begin{Teo}[Teorema Principal de Renovaci\'on]
Si $F$ es no aritm\'etica y $h\left(t\right)$ es Directamente Riemann Integrable (DRI), entonces

\begin{eqnarray*}
lim_{t\rightarrow\infty}U\star h=\frac{1}{\mu}\int_{\rea_{+}}h\left(s\right)ds.
\end{eqnarray*}
\end{Teo}

\begin{Prop}
Cualquier funci\'on $H\left(t\right)$ acotada en intervalos finitos y que es 0 para $t<0$ puede expresarse como
\begin{eqnarray*}
H\left(t\right)=U\star h\left(t\right)\textrm{,  donde }h\left(t\right)=H\left(t\right)-F\star H\left(t\right)
\end{eqnarray*}
\end{Prop}

\begin{Def}
Un proceso estoc\'astico $X\left(t\right)$ es crudamente regenerativo en un tiempo aleatorio positivo $T$ si
\begin{eqnarray*}
\esp\left[X\left(T+t\right)|T\right]=\esp\left[X\left(t\right)\right]\textrm{, para }t\geq0,\end{eqnarray*}
y con las esperanzas anteriores finitas.
\end{Def}

\begin{Prop}
Sup\'ongase que $X\left(t\right)$ es un proceso crudamente regenerativo en $T$, que tiene distribuci\'on $F$. Si $\esp\left[X\left(t\right)\right]$ es acotado en intervalos finitos, entonces
\begin{eqnarray*}
\esp\left[X\left(t\right)\right]=U\star h\left(t\right)\textrm{,  donde }h\left(t\right)=\esp\left[X\left(t\right)\indora\left(T>t\right)\right].
\end{eqnarray*}
\end{Prop}

\begin{Teo}[Regeneraci\'on Cruda]
Sup\'ongase que $X\left(t\right)$ es un proceso con valores positivo crudamente regenerativo en $T$, y def\'inase $M=\sup\left\{|X\left(t\right)|:t\leq T\right\}$. Si $T$ es no aritm\'etico y $M$ y $MT$ tienen media finita, entonces
\begin{eqnarray*}
lim_{t\rightarrow\infty}\esp\left[X\left(t\right)\right]=\frac{1}{\mu}\int_{\rea_{+}}h\left(s\right)ds,
\end{eqnarray*}
donde $h\left(t\right)=\esp\left[X\left(t\right)\indora\left(T>t\right)\right]$.
\end{Teo}

\begin{Def}
Para el proceso $\left\{\left(N\left(t\right),X\left(t\right)\right):t\geq0\right\}$, sus trayectoria muestrales en el intervalo de tiempo $\left[T_{n-1},T_{n}\right)$ est\'an descritas por
\begin{eqnarray*}
\zeta_{n}=\left(\xi_{n},\left\{X\left(T_{n-1}+t\right):0\leq t<\xi_{n}\right\}\right)
\end{eqnarray*}
Este $\zeta_{n}$ es el $n$-\'esimo segmento del proceso. El proceso es regenerativo sobre los tiempos $T_{n}$ si sus segmentos $\zeta_{n}$ son independientes e id\'enticamennte distribuidos.
\end{Def}


\begin{Note}
Si $\tilde{X}\left(t\right)$ con espacio de estados $\tilde{S}$ es regenerativo sobre $T_{n}$, entonces $X\left(t\right)=f\left(\tilde{X}\left(t\right)\right)$ tambi\'en es regenerativo sobre $T_{n}$, para cualquier funci\'on $f:\tilde{S}\rightarrow S$.
\end{Note}

\begin{Note}
Los procesos regenerativos son crudamente regenerativos, pero no al rev\'es.
\end{Note}


\begin{Note}
Un proceso estoc\'astico a tiempo continuo o discreto es regenerativo si existe un proceso de renovaci\'on  tal que los segmentos del proceso entre tiempos de renovaci\'on sucesivos son i.i.d., es decir, para $\left\{X\left(t\right):t\geq0\right\}$ proceso estoc\'astico a tiempo continuo con espacio de estados $S$, espacio m\'etrico.
\end{Note}

Para $\left\{X\left(t\right):t\geq0\right\}$ Proceso Estoc\'astico a tiempo continuo con estado de espacios $S$, que es un espacio m\'etrico, con trayectorias continuas por la derecha y con l\'imites por la izquierda c.s. Sea $N\left(t\right)$ un proceso de renovaci\'on en $\rea_{+}$ definido en el mismo espacio de probabilidad que $X\left(t\right)$, con tiempos de renovaci\'on $T$ y tiempos de inter-renovaci\'on $\xi_{n}=T_{n}-T_{n-1}$, con misma distribuci\'on $F$ de media finita $\mu$.



\begin{Def}
Para el proceso $\left\{\left(N\left(t\right),X\left(t\right)\right):t\geq0\right\}$, sus trayectoria muestrales en el intervalo de tiempo $\left[T_{n-1},T_{n}\right)$ est\'an descritas por
\begin{eqnarray*}
\zeta_{n}=\left(\xi_{n},\left\{X\left(T_{n-1}+t\right):0\leq t<\xi_{n}\right\}\right)
\end{eqnarray*}
Este $\zeta_{n}$ es el $n$-\'esimo segmento del proceso. El proceso es regenerativo sobre los tiempos $T_{n}$ si sus segmentos $\zeta_{n}$ son independientes e id\'enticamennte distribuidos.
\end{Def}

\begin{Note}
Un proceso regenerativo con media de la longitud de ciclo finita es llamado positivo recurrente.
\end{Note}

\begin{Teo}[Procesos Regenerativos]
Suponga que el proceso
\end{Teo}


\begin{Def}[Renewal Process Trinity]
Para un proceso de renovaci\'on $N\left(t\right)$, los siguientes procesos proveen de informaci\'on sobre los tiempos de renovaci\'on.
\begin{itemize}
\item $A\left(t\right)=t-T_{N\left(t\right)}$, el tiempo de recurrencia hacia atr\'as al tiempo $t$, que es el tiempo desde la \'ultima renovaci\'on para $t$.

\item $B\left(t\right)=T_{N\left(t\right)+1}-t$, el tiempo de recurrencia hacia adelante al tiempo $t$, residual del tiempo de renovaci\'on, que es el tiempo para la pr\'oxima renovaci\'on despu\'es de $t$.

\item $L\left(t\right)=\xi_{N\left(t\right)+1}=A\left(t\right)+B\left(t\right)$, la longitud del intervalo de renovaci\'on que contiene a $t$.
\end{itemize}
\end{Def}

\begin{Note}
El proceso tridimensional $\left(A\left(t\right),B\left(t\right),L\left(t\right)\right)$ es regenerativo sobre $T_{n}$, y por ende cada proceso lo es. Cada proceso $A\left(t\right)$ y $B\left(t\right)$ son procesos de MArkov a tiempo continuo con trayectorias continuas por partes en el espacio de estados $\rea_{+}$. Una expresi\'on conveniente para su distribuci\'on conjunta es, para $0\leq x<t,y\geq0$
\begin{equation}\label{NoRenovacion}
P\left\{A\left(t\right)>x,B\left(t\right)>y\right\}=
P\left\{N\left(t+y\right)-N\left((t-x)\right)=0\right\}
\end{equation}
\end{Note}

\begin{Ejem}[Tiempos de recurrencia Poisson]
Si $N\left(t\right)$ es un proceso Poisson con tasa $\lambda$, entonces de la expresi\'on (\ref{NoRenovacion}) se tiene que

\begin{eqnarray*}
\begin{array}{lc}
P\left\{A\left(t\right)>x,B\left(t\right)>y\right\}=e^{-\lambda\left(x+y\right)},&0\leq x<t,y\geq0,
\end{array}
\end{eqnarray*}
que es la probabilidad Poisson de no renovaciones en un intervalo de longitud $x+y$.

\end{Ejem}

%\begin{Note}
Una cadena de Markov erg\'odica tiene la propiedad de ser estacionaria si la distribuci\'on de su estado al tiempo $0$ es su distribuci\'on estacionaria.
%\end{Note}


\begin{Def}
Un proceso estoc\'astico a tiempo continuo $\left\{X\left(t\right):t\geq0\right\}$ en un espacio general es estacionario si sus distribuciones finito dimensionales son invariantes bajo cualquier  traslado: para cada $0\leq s_{1}<s_{2}<\cdots<s_{k}$ y $t\geq0$,
\begin{eqnarray*}
\left(X\left(s_{1}+t\right),\ldots,X\left(s_{k}+t\right)\right)=_{d}\left(X\left(s_{1}\right),\ldots,X\left(s_{k}\right)\right).
\end{eqnarray*}
\end{Def}

\begin{Note}
Un proceso de Markov es estacionario si $X\left(t\right)=_{d}X\left(0\right)$, $t\geq0$.
\end{Note}

Considerese el proceso $N\left(t\right)=\sum_{n}\indora\left(\tau_{n}\leq t\right)$ en $\rea_{+}$, con puntos $0<\tau_{1}<\tau_{2}<\cdots$.

\begin{Prop}
Si $N$ es un proceso puntual estacionario y $\esp\left[N\left(1\right)\right]<\infty$, entonces $\esp\left[N\left(t\right)\right]=t\esp\left[N\left(1\right)\right]$, $t\geq0$

\end{Prop}

\begin{Teo}
Los siguientes enunciados son equivalentes
\begin{itemize}
\item[i)] El proceso retardado de renovaci\'on $N$ es estacionario.

\item[ii)] EL proceso de tiempos de recurrencia hacia adelante $B\left(t\right)$ es estacionario.


\item[iii)] $\esp\left[N\left(t\right)\right]=t/\mu$,


\item[iv)] $G\left(t\right)=F_{e}\left(t\right)=\frac{1}{\mu}\int_{0}^{t}\left[1-F\left(s\right)\right]ds$
\end{itemize}
Cuando estos enunciados son ciertos, $P\left\{B\left(t\right)\leq x\right\}=F_{e}\left(x\right)$, para $t,x\geq0$.

\end{Teo}

\begin{Note}
Una consecuencia del teorema anterior es que el Proceso Poisson es el \'unico proceso sin retardo que es estacionario.
\end{Note}

\begin{Coro}
El proceso de renovaci\'on $N\left(t\right)$ sin retardo, y cuyos tiempos de inter renonaci\'on tienen media finita, es estacionario si y s\'olo si es un proceso Poisson.

\end{Coro}


%________________________________________________________________________
\subsection{Procesos Regenerativos}
%________________________________________________________________________



\begin{Note}
Si $\tilde{X}\left(t\right)$ con espacio de estados $\tilde{S}$ es regenerativo sobre $T_{n}$, entonces $X\left(t\right)=f\left(\tilde{X}\left(t\right)\right)$ tambi\'en es regenerativo sobre $T_{n}$, para cualquier funci\'on $f:\tilde{S}\rightarrow S$.
\end{Note}

\begin{Note}
Los procesos regenerativos son crudamente regenerativos, pero no al rev\'es.
\end{Note}
%\subsection*{Procesos Regenerativos: Sigman\cite{Sigman1}}
\begin{Def}[Definici\'on Cl\'asica]
Un proceso estoc\'astico $X=\left\{X\left(t\right):t\geq0\right\}$ es llamado regenerativo is existe una variable aleatoria $R_{1}>0$ tal que
\begin{itemize}
\item[i)] $\left\{X\left(t+R_{1}\right):t\geq0\right\}$ es independiente de $\left\{\left\{X\left(t\right):t<R_{1}\right\},\right\}$
\item[ii)] $\left\{X\left(t+R_{1}\right):t\geq0\right\}$ es estoc\'asticamente equivalente a $\left\{X\left(t\right):t>0\right\}$
\end{itemize}

Llamamos a $R_{1}$ tiempo de regeneraci\'on, y decimos que $X$ se regenera en este punto.
\end{Def}

$\left\{X\left(t+R_{1}\right)\right\}$ es regenerativo con tiempo de regeneraci\'on $R_{2}$, independiente de $R_{1}$ pero con la misma distribuci\'on que $R_{1}$. Procediendo de esta manera se obtiene una secuencia de variables aleatorias independientes e id\'enticamente distribuidas $\left\{R_{n}\right\}$ llamados longitudes de ciclo. Si definimos a $Z_{k}\equiv R_{1}+R_{2}+\cdots+R_{k}$, se tiene un proceso de renovaci\'on llamado proceso de renovaci\'on encajado para $X$.




\begin{Def}
Para $x$ fijo y para cada $t\geq0$, sea $I_{x}\left(t\right)=1$ si $X\left(t\right)\leq x$,  $I_{x}\left(t\right)=0$ en caso contrario, y def\'inanse los tiempos promedio
\begin{eqnarray*}
\overline{X}&=&lim_{t\rightarrow\infty}\frac{1}{t}\int_{0}^{\infty}X\left(u\right)du\\
\prob\left(X_{\infty}\leq x\right)&=&lim_{t\rightarrow\infty}\frac{1}{t}\int_{0}^{\infty}I_{x}\left(u\right)du,
\end{eqnarray*}
cuando estos l\'imites existan.
\end{Def}

Como consecuencia del teorema de Renovaci\'on-Recompensa, se tiene que el primer l\'imite  existe y es igual a la constante
\begin{eqnarray*}
\overline{X}&=&\frac{\esp\left[\int_{0}^{R_{1}}X\left(t\right)dt\right]}{\esp\left[R_{1}\right]},
\end{eqnarray*}
suponiendo que ambas esperanzas son finitas.

\begin{Note}
\begin{itemize}
\item[a)] Si el proceso regenerativo $X$ es positivo recurrente y tiene trayectorias muestrales no negativas, entonces la ecuaci\'on anterior es v\'alida.
\item[b)] Si $X$ es positivo recurrente regenerativo, podemos construir una \'unica versi\'on estacionaria de este proceso, $X_{e}=\left\{X_{e}\left(t\right)\right\}$, donde $X_{e}$ es un proceso estoc\'astico regenerativo y estrictamente estacionario, con distribuci\'on marginal distribuida como $X_{\infty}$
\end{itemize}
\end{Note}

%________________________________________________________________________
%\subsection{Procesos Regenerativos}
%________________________________________________________________________

Para $\left\{X\left(t\right):t\geq0\right\}$ Proceso Estoc\'astico a tiempo continuo con estado de espacios $S$, que es un espacio m\'etrico, con trayectorias continuas por la derecha y con l\'imites por la izquierda c.s. Sea $N\left(t\right)$ un proceso de renovaci\'on en $\rea_{+}$ definido en el mismo espacio de probabilidad que $X\left(t\right)$, con tiempos de renovaci\'on $T$ y tiempos de inter-renovaci\'on $\xi_{n}=T_{n}-T_{n-1}$, con misma distribuci\'on $F$ de media finita $\mu$.



\begin{Def}
Para el proceso $\left\{\left(N\left(t\right),X\left(t\right)\right):t\geq0\right\}$, sus trayectoria muestrales en el intervalo de tiempo $\left[T_{n-1},T_{n}\right)$ est\'an descritas por
\begin{eqnarray*}
\zeta_{n}=\left(\xi_{n},\left\{X\left(T_{n-1}+t\right):0\leq t<\xi_{n}\right\}\right)
\end{eqnarray*}
Este $\zeta_{n}$ es el $n$-\'esimo segmento del proceso. El proceso es regenerativo sobre los tiempos $T_{n}$ si sus segmentos $\zeta_{n}$ son independientes e id\'enticamennte distribuidos.
\end{Def}


\begin{Note}
Si $\tilde{X}\left(t\right)$ con espacio de estados $\tilde{S}$ es regenerativo sobre $T_{n}$, entonces $X\left(t\right)=f\left(\tilde{X}\left(t\right)\right)$ tambi\'en es regenerativo sobre $T_{n}$, para cualquier funci\'on $f:\tilde{S}\rightarrow S$.
\end{Note}

\begin{Note}
Los procesos regenerativos son crudamente regenerativos, pero no al rev\'es.
\end{Note}

\begin{Def}[Definici\'on Cl\'asica]
Un proceso estoc\'astico $X=\left\{X\left(t\right):t\geq0\right\}$ es llamado regenerativo is existe una variable aleatoria $R_{1}>0$ tal que
\begin{itemize}
\item[i)] $\left\{X\left(t+R_{1}\right):t\geq0\right\}$ es independiente de $\left\{\left\{X\left(t\right):t<R_{1}\right\},\right\}$
\item[ii)] $\left\{X\left(t+R_{1}\right):t\geq0\right\}$ es estoc\'asticamente equivalente a $\left\{X\left(t\right):t>0\right\}$
\end{itemize}

Llamamos a $R_{1}$ tiempo de regeneraci\'on, y decimos que $X$ se regenera en este punto.
\end{Def}

$\left\{X\left(t+R_{1}\right)\right\}$ es regenerativo con tiempo de regeneraci\'on $R_{2}$, independiente de $R_{1}$ pero con la misma distribuci\'on que $R_{1}$. Procediendo de esta manera se obtiene una secuencia de variables aleatorias independientes e id\'enticamente distribuidas $\left\{R_{n}\right\}$ llamados longitudes de ciclo. Si definimos a $Z_{k}\equiv R_{1}+R_{2}+\cdots+R_{k}$, se tiene un proceso de renovaci\'on llamado proceso de renovaci\'on encajado para $X$.

\begin{Note}
Un proceso regenerativo con media de la longitud de ciclo finita es llamado positivo recurrente.
\end{Note}


\begin{Def}
Para $x$ fijo y para cada $t\geq0$, sea $I_{x}\left(t\right)=1$ si $X\left(t\right)\leq x$,  $I_{x}\left(t\right)=0$ en caso contrario, y def\'inanse los tiempos promedio
\begin{eqnarray*}
\overline{X}&=&lim_{t\rightarrow\infty}\frac{1}{t}\int_{0}^{\infty}X\left(u\right)du\\
\prob\left(X_{\infty}\leq x\right)&=&lim_{t\rightarrow\infty}\frac{1}{t}\int_{0}^{\infty}I_{x}\left(u\right)du,
\end{eqnarray*}
cuando estos l\'imites existan.
\end{Def}

Como consecuencia del teorema de Renovaci\'on-Recompensa, se tiene que el primer l\'imite  existe y es igual a la constante
\begin{eqnarray*}
\overline{X}&=&\frac{\esp\left[\int_{0}^{R_{1}}X\left(t\right)dt\right]}{\esp\left[R_{1}\right]},
\end{eqnarray*}
suponiendo que ambas esperanzas son finitas.

\begin{Note}
\begin{itemize}
\item[a)] Si el proceso regenerativo $X$ es positivo recurrente y tiene trayectorias muestrales no negativas, entonces la ecuaci\'on anterior es v\'alida.
\item[b)] Si $X$ es positivo recurrente regenerativo, podemos construir una \'unica versi\'on estacionaria de este proceso, $X_{e}=\left\{X_{e}\left(t\right)\right\}$, donde $X_{e}$ es un proceso estoc\'astico regenerativo y estrictamente estacionario, con distribuci\'on marginal distribuida como $X_{\infty}$
\end{itemize}
\end{Note}

%__________________________________________________________________________________________
%\subsection{Procesos Regenerativos Estacionarios - Stidham \cite{Stidham}}
%__________________________________________________________________________________________


Un proceso estoc\'astico a tiempo continuo $\left\{V\left(t\right),t\geq0\right\}$ es un proceso regenerativo si existe una sucesi\'on de variables aleatorias independientes e id\'enticamente distribuidas $\left\{X_{1},X_{2},\ldots\right\}$, sucesi\'on de renovaci\'on, tal que para cualquier conjunto de Borel $A$, 

\begin{eqnarray*}
\prob\left\{V\left(t\right)\in A|X_{1}+X_{2}+\cdots+X_{R\left(t\right)}=s,\left\{V\left(\tau\right),\tau<s\right\}\right\}=\prob\left\{V\left(t-s\right)\in A|X_{1}>t-s\right\},
\end{eqnarray*}
para todo $0\leq s\leq t$, donde $R\left(t\right)=\max\left\{X_{1}+X_{2}+\cdots+X_{j}\leq t\right\}=$n\'umero de renovaciones ({\emph{puntos de regeneraci\'on}}) que ocurren en $\left[0,t\right]$. El intervalo $\left[0,X_{1}\right)$ es llamado {\emph{primer ciclo de regeneraci\'on}} de $\left\{V\left(t \right),t\geq0\right\}$, $\left[X_{1},X_{1}+X_{2}\right)$ el {\emph{segundo ciclo de regeneraci\'on}}, y as\'i sucesivamente.

Sea $X=X_{1}$ y sea $F$ la funci\'on de distrbuci\'on de $X$


\begin{Def}
Se define el proceso estacionario, $\left\{V^{*}\left(t\right),t\geq0\right\}$, para $\left\{V\left(t\right),t\geq0\right\}$ por

\begin{eqnarray*}
\prob\left\{V\left(t\right)\in A\right\}=\frac{1}{\esp\left[X\right]}\int_{0}^{\infty}\prob\left\{V\left(t+x\right)\in A|X>x\right\}\left(1-F\left(x\right)\right)dx,
\end{eqnarray*} 
para todo $t\geq0$ y todo conjunto de Borel $A$.
\end{Def}

\begin{Def}
Una distribuci\'on se dice que es {\emph{aritm\'etica}} si todos sus puntos de incremento son m\'ultiplos de la forma $0,\lambda, 2\lambda,\ldots$ para alguna $\lambda>0$ entera.
\end{Def}


\begin{Def}
Una modificaci\'on medible de un proceso $\left\{V\left(t\right),t\geq0\right\}$, es una versi\'on de este, $\left\{V\left(t,w\right)\right\}$ conjuntamente medible para $t\geq0$ y para $w\in S$, $S$ espacio de estados para $\left\{V\left(t\right),t\geq0\right\}$.
\end{Def}

\begin{Teo}
Sea $\left\{V\left(t\right),t\geq\right\}$ un proceso regenerativo no negativo con modificaci\'on medible. Sea $\esp\left[X\right]<\infty$. Entonces el proceso estacionario dado por la ecuaci\'on anterior est\'a bien definido y tiene funci\'on de distribuci\'on independiente de $t$, adem\'as
\begin{itemize}
\item[i)] \begin{eqnarray*}
\esp\left[V^{*}\left(0\right)\right]&=&\frac{\esp\left[\int_{0}^{X}V\left(s\right)ds\right]}{\esp\left[X\right]}\end{eqnarray*}
\item[ii)] Si $\esp\left[V^{*}\left(0\right)\right]<\infty$, equivalentemente, si $\esp\left[\int_{0}^{X}V\left(s\right)ds\right]<\infty$,entonces
\begin{eqnarray*}
\frac{\int_{0}^{t}V\left(s\right)ds}{t}\rightarrow\frac{\esp\left[\int_{0}^{X}V\left(s\right)ds\right]}{\esp\left[X\right]}
\end{eqnarray*}
con probabilidad 1 y en media, cuando $t\rightarrow\infty$.
\end{itemize}
\end{Teo}

%__________________________________________________________________________________________
%\subsection{Procesos Regenerativos Estacionarios - Stidham \cite{Stidham}}
%__________________________________________________________________________________________


Un proceso estoc\'astico a tiempo continuo $\left\{V\left(t\right),t\geq0\right\}$ es un proceso regenerativo si existe una sucesi\'on de variables aleatorias independientes e id\'enticamente distribuidas $\left\{X_{1},X_{2},\ldots\right\}$, sucesi\'on de renovaci\'on, tal que para cualquier conjunto de Borel $A$, 

\begin{eqnarray*}
\prob\left\{V\left(t\right)\in A|X_{1}+X_{2}+\cdots+X_{R\left(t\right)}=s,\left\{V\left(\tau\right),\tau<s\right\}\right\}=\prob\left\{V\left(t-s\right)\in A|X_{1}>t-s\right\},
\end{eqnarray*}
para todo $0\leq s\leq t$, donde $R\left(t\right)=\max\left\{X_{1}+X_{2}+\cdots+X_{j}\leq t\right\}=$n\'umero de renovaciones ({\emph{puntos de regeneraci\'on}}) que ocurren en $\left[0,t\right]$. El intervalo $\left[0,X_{1}\right)$ es llamado {\emph{primer ciclo de regeneraci\'on}} de $\left\{V\left(t \right),t\geq0\right\}$, $\left[X_{1},X_{1}+X_{2}\right)$ el {\emph{segundo ciclo de regeneraci\'on}}, y as\'i sucesivamente.

Sea $X=X_{1}$ y sea $F$ la funci\'on de distrbuci\'on de $X$


\begin{Def}
Se define el proceso estacionario, $\left\{V^{*}\left(t\right),t\geq0\right\}$, para $\left\{V\left(t\right),t\geq0\right\}$ por

\begin{eqnarray*}
\prob\left\{V\left(t\right)\in A\right\}=\frac{1}{\esp\left[X\right]}\int_{0}^{\infty}\prob\left\{V\left(t+x\right)\in A|X>x\right\}\left(1-F\left(x\right)\right)dx,
\end{eqnarray*} 
para todo $t\geq0$ y todo conjunto de Borel $A$.
\end{Def}

\begin{Def}
Una distribuci\'on se dice que es {\emph{aritm\'etica}} si todos sus puntos de incremento son m\'ultiplos de la forma $0,\lambda, 2\lambda,\ldots$ para alguna $\lambda>0$ entera.
\end{Def}


\begin{Def}
Una modificaci\'on medible de un proceso $\left\{V\left(t\right),t\geq0\right\}$, es una versi\'on de este, $\left\{V\left(t,w\right)\right\}$ conjuntamente medible para $t\geq0$ y para $w\in S$, $S$ espacio de estados para $\left\{V\left(t\right),t\geq0\right\}$.
\end{Def}

\begin{Teo}
Sea $\left\{V\left(t\right),t\geq\right\}$ un proceso regenerativo no negativo con modificaci\'on medible. Sea $\esp\left[X\right]<\infty$. Entonces el proceso estacionario dado por la ecuaci\'on anterior est\'a bien definido y tiene funci\'on de distribuci\'on independiente de $t$, adem\'as
\begin{itemize}
\item[i)] \begin{eqnarray*}
\esp\left[V^{*}\left(0\right)\right]&=&\frac{\esp\left[\int_{0}^{X}V\left(s\right)ds\right]}{\esp\left[X\right]}\end{eqnarray*}
\item[ii)] Si $\esp\left[V^{*}\left(0\right)\right]<\infty$, equivalentemente, si $\esp\left[\int_{0}^{X}V\left(s\right)ds\right]<\infty$,entonces
\begin{eqnarray*}
\frac{\int_{0}^{t}V\left(s\right)ds}{t}\rightarrow\frac{\esp\left[\int_{0}^{X}V\left(s\right)ds\right]}{\esp\left[X\right]}
\end{eqnarray*}
con probabilidad 1 y en media, cuando $t\rightarrow\infty$.
\end{itemize}
\end{Teo}

Para $\left\{X\left(t\right):t\geq0\right\}$ Proceso Estoc\'astico a tiempo continuo con estado de espacios $S$, que es un espacio m\'etrico, con trayectorias continuas por la derecha y con l\'imites por la izquierda c.s. Sea $N\left(t\right)$ un proceso de renovaci\'on en $\rea_{+}$ definido en el mismo espacio de probabilidad que $X\left(t\right)$, con tiempos de renovaci\'on $T$ y tiempos de inter-renovaci\'on $\xi_{n}=T_{n}-T_{n-1}$, con misma distribuci\'on $F$ de media finita $\mu$.
%_____________________________________________________
\subsection{Puntos de Renovaci\'on}
%_____________________________________________________

Para cada cola $Q_{i}$ se tienen los procesos de arribo a la cola, para estas, los tiempos de arribo est\'an dados por $$\left\{T_{1}^{i},T_{2}^{i},\ldots,T_{k}^{i},\ldots\right\},$$ entonces, consideremos solamente los primeros tiempos de arribo a cada una de las colas, es decir, $$\left\{T_{1}^{1},T_{1}^{2},T_{1}^{3},T_{1}^{4}\right\},$$ se sabe que cada uno de estos tiempos se distribuye de manera exponencial con par\'ametro $1/mu_{i}$. Adem\'as se sabe que para $$T^{*}=\min\left\{T_{1}^{1},T_{1}^{2},T_{1}^{3},T_{1}^{4}\right\},$$ $T^{*}$ se distribuye de manera exponencial con par\'ametro $$\mu^{*}=\sum_{i=1}^{4}\mu_{i}.$$ Ahora, dado que 
\begin{center}
\begin{tabular}{lcl}
$\tilde{r}=r_{1}+r_{2}$ & y &$\hat{r}=r_{3}+r_{4}.$
\end{tabular}
\end{center}


Supongamos que $$\tilde{r},\hat{r}<\mu^{*},$$ entonces si tomamos $$r^{*}=\min\left\{\tilde{r},\hat{r}\right\},$$ se tiene que para  $$t^{*}\in\left(0,r^{*}\right)$$ se cumple que 
\begin{center}
\begin{tabular}{lcl}
$\tau_{1}\left(1\right)=0$ & y por tanto & $\overline{\tau}_{1}=0,$
\end{tabular}
\end{center}
entonces para la segunda cola en este primer ciclo se cumple que $$\tau_{2}=\overline{\tau}_{1}+r_{1}=r_{1}<\mu^{*},$$ y por tanto se tiene que  $$\overline{\tau}_{2}=\tau_{2}.$$ Por lo tanto, nuevamente para la primer cola en el segundo ciclo $$\tau_{1}\left(2\right)=\tau_{2}\left(1\right)+r_{2}=\tilde{r}<\mu^{*}.$$ An\'alogamente para el segundo sistema se tiene que ambas colas est\'an vac\'ias, es decir, existe un valor $t^{*}$ tal que en el intervalo $\left(0,t^{*}\right)$ no ha llegado ning\'un usuario, es decir, $$L_{i}\left(t^{*}\right)=0$$ para $i=1,2,3,4$.

\subsection{Resultados para Procesos de Salida}




%________________________________________________________________________
\subsection{Procesos Regenerativos}
%________________________________________________________________________

Para $\left\{X\left(t\right):t\geq0\right\}$ Proceso Estoc\'astico a tiempo continuo con estado de espacios $S$, que es un espacio m\'etrico, con trayectorias continuas por la derecha y con l\'imites por la izquierda c.s. Sea $N\left(t\right)$ un proceso de renovaci\'on en $\rea_{+}$ definido en el mismo espacio de probabilidad que $X\left(t\right)$, con tiempos de renovaci\'on $T$ y tiempos de inter-renovaci\'on $\xi_{n}=T_{n}-T_{n-1}$, con misma distribuci\'on $F$ de media finita $\mu$.



\begin{Def}
Para el proceso $\left\{\left(N\left(t\right),X\left(t\right)\right):t\geq0\right\}$, sus trayectoria muestrales en el intervalo de tiempo $\left[T_{n-1},T_{n}\right)$ est\'an descritas por
\begin{eqnarray*}
\zeta_{n}=\left(\xi_{n},\left\{X\left(T_{n-1}+t\right):0\leq t<\xi_{n}\right\}\right)
\end{eqnarray*}
Este $\zeta_{n}$ es el $n$-\'esimo segmento del proceso. El proceso es regenerativo sobre los tiempos $T_{n}$ si sus segmentos $\zeta_{n}$ son independientes e id\'enticamennte distribuidos.
\end{Def}


\begin{Obs}
Si $\tilde{X}\left(t\right)$ con espacio de estados $\tilde{S}$ es regenerativo sobre $T_{n}$, entonces $X\left(t\right)=f\left(\tilde{X}\left(t\right)\right)$ tambi\'en es regenerativo sobre $T_{n}$, para cualquier funci\'on $f:\tilde{S}\rightarrow S$.
\end{Obs}

\begin{Obs}
Los procesos regenerativos son crudamente regenerativos, pero no al rev\'es.
\end{Obs}

\begin{Def}[Definici\'on Cl\'asica]
Un proceso estoc\'astico $X=\left\{X\left(t\right):t\geq0\right\}$ es llamado regenerativo is existe una variable aleatoria $R_{1}>0$ tal que
\begin{itemize}
\item[i)] $\left\{X\left(t+R_{1}\right):t\geq0\right\}$ es independiente de $\left\{\left\{X\left(t\right):t<R_{1}\right\},\right\}$
\item[ii)] $\left\{X\left(t+R_{1}\right):t\geq0\right\}$ es estoc\'asticamente equivalente a $\left\{X\left(t\right):t>0\right\}$
\end{itemize}

Llamamos a $R_{1}$ tiempo de regeneraci\'on, y decimos que $X$ se regenera en este punto.
\end{Def}

$\left\{X\left(t+R_{1}\right)\right\}$ es regenerativo con tiempo de regeneraci\'on $R_{2}$, independiente de $R_{1}$ pero con la misma distribuci\'on que $R_{1}$. Procediendo de esta manera se obtiene una secuencia de variables aleatorias independientes e id\'enticamente distribuidas $\left\{R_{n}\right\}$ llamados longitudes de ciclo. Si definimos a $Z_{k}\equiv R_{1}+R_{2}+\cdots+R_{k}$, se tiene un proceso de renovaci\'on llamado proceso de renovaci\'on encajado para $X$.

\begin{Note}
Un proceso regenerativo con media de la longitud de ciclo finita es llamado positivo recurrente.
\end{Note}


\begin{Def}
Para $x$ fijo y para cada $t\geq0$, sea $I_{x}\left(t\right)=1$ si $X\left(t\right)\leq x$,  $I_{x}\left(t\right)=0$ en caso contrario, y def\'inanse los tiempos promedio
\begin{eqnarray*}
\overline{X}&=&lim_{t\rightarrow\infty}\frac{1}{t}\int_{0}^{\infty}X\left(u\right)du\\
\prob\left(X_{\infty}\leq x\right)&=&lim_{t\rightarrow\infty}\frac{1}{t}\int_{0}^{\infty}I_{x}\left(u\right)du,
\end{eqnarray*}
cuando estos l\'imites existan.
\end{Def}

Como consecuencia del teorema de Renovaci\'on-Recompensa, se tiene que el primer l\'imite  existe y es igual a la constante
\begin{eqnarray*}
\overline{X}&=&\frac{\esp\left[\int_{0}^{R_{1}}X\left(t\right)dt\right]}{\esp\left[R_{1}\right]},
\end{eqnarray*}
suponiendo que ambas esperanzas son finitas.

\begin{Note}
\begin{itemize}
\item[a)] Si el proceso regenerativo $X$ es positivo recurrente y tiene trayectorias muestrales no negativas, entonces la ecuaci\'on anterior es v\'alida.
\item[b)] Si $X$ es positivo recurrente regenerativo, podemos construir una \'unica versi\'on estacionaria de este proceso, $X_{e}=\left\{X_{e}\left(t\right)\right\}$, donde $X_{e}$ es un proceso estoc\'astico regenerativo y estrictamente estacionario, con distribuci\'on marginal distribuida como $X_{\infty}$
\end{itemize}
\end{Note}

\subsection{Renewal and Regenerative Processes: Serfozo\cite{Serfozo}}
\begin{Def}\label{Def.Tn}
Sean $0\leq T_{1}\leq T_{2}\leq \ldots$ son tiempos aleatorios infinitos en los cuales ocurren ciertos eventos. El n\'umero de tiempos $T_{n}$ en el intervalo $\left[0,t\right)$ es

\begin{eqnarray}
N\left(t\right)=\sum_{n=1}^{\infty}\indora\left(T_{n}\leq t\right),
\end{eqnarray}
para $t\geq0$.
\end{Def}

Si se consideran los puntos $T_{n}$ como elementos de $\rea_{+}$, y $N\left(t\right)$ es el n\'umero de puntos en $\rea$. El proceso denotado por $\left\{N\left(t\right):t\geq0\right\}$, denotado por $N\left(t\right)$, es un proceso puntual en $\rea_{+}$. Los $T_{n}$ son los tiempos de ocurrencia, el proceso puntual $N\left(t\right)$ es simple si su n\'umero de ocurrencias son distintas: $0<T_{1}<T_{2}<\ldots$ casi seguramente.

\begin{Def}
Un proceso puntual $N\left(t\right)$ es un proceso de renovaci\'on si los tiempos de interocurrencia $\xi_{n}=T_{n}-T_{n-1}$, para $n\geq1$, son independientes e identicamente distribuidos con distribuci\'on $F$, donde $F\left(0\right)=0$ y $T_{0}=0$. Los $T_{n}$ son llamados tiempos de renovaci\'on, referente a la independencia o renovaci\'on de la informaci\'on estoc\'astica en estos tiempos. Los $\xi_{n}$ son los tiempos de inter-renovaci\'on, y $N\left(t\right)$ es el n\'umero de renovaciones en el intervalo $\left[0,t\right)$
\end{Def}


\begin{Note}
Para definir un proceso de renovaci\'on para cualquier contexto, solamente hay que especificar una distribuci\'on $F$, con $F\left(0\right)=0$, para los tiempos de inter-renovaci\'on. La funci\'on $F$ en turno degune las otra variables aleatorias. De manera formal, existe un espacio de probabilidad y una sucesi\'on de variables aleatorias $\xi_{1},\xi_{2},\ldots$ definidas en este con distribuci\'on $F$. Entonces las otras cantidades son $T_{n}=\sum_{k=1}^{n}\xi_{k}$ y $N\left(t\right)=\sum_{n=1}^{\infty}\indora\left(T_{n}\leq t\right)$, donde $T_{n}\rightarrow\infty$ casi seguramente por la Ley Fuerte de los Grandes N\'umeros.
\end{Note}


Los tiempos $T_{n}$ est\'an relacionados con los conteos de $N\left(t\right)$ por

\begin{eqnarray*}
\left\{N\left(t\right)\geq n\right\}&=&\left\{T_{n}\leq t\right\}\\
T_{N\left(t\right)}\leq &t&<T_{N\left(t\right)+1},
\end{eqnarray*}

adem\'as $N\left(T_{n}\right)=n$, y 

\begin{eqnarray*}
N\left(t\right)=\max\left\{n:T_{n}\leq t\right\}=\min\left\{n:T_{n+1}>t\right\}
\end{eqnarray*}

Por propiedades de la convoluci\'on se sabe que

\begin{eqnarray*}
P\left\{T_{n}\leq t\right\}=F^{n\star}\left(t\right)
\end{eqnarray*}
que es la $n$-\'esima convoluci\'on de $F$. Entonces 

\begin{eqnarray*}
\left\{N\left(t\right)\geq n\right\}&=&\left\{T_{n}\leq t\right\}\\
P\left\{N\left(t\right)\leq n\right\}&=&1-F^{\left(n+1\right)\star}\left(t\right)
\end{eqnarray*}

Adem\'as usando el hecho de que $\esp\left[N\left(t\right)\right]=\sum_{n=1}^{\infty}P\left\{N\left(t\right)\geq n\right\}$
se tiene que

\begin{eqnarray*}
\esp\left[N\left(t\right)\right]=\sum_{n=1}^{\infty}F^{n\star}\left(t\right)
\end{eqnarray*}

\begin{Prop}
Para cada $t\geq0$, la funci\'on generadora de momentos $\esp\left[e^{\alpha N\left(t\right)}\right]$ existe para alguna $\alpha$ en una vecindad del 0, y de aqu\'i que $\esp\left[N\left(t\right)^{m}\right]<\infty$, para $m\geq1$.
\end{Prop}


\begin{Note}
Si el primer tiempo de renovaci\'on $\xi_{1}$ no tiene la misma distribuci\'on que el resto de las $\xi_{n}$, para $n\geq2$, a $N\left(t\right)$ se le llama Proceso de Renovaci\'on retardado, donde si $\xi$ tiene distribuci\'on $G$, entonces el tiempo $T_{n}$ de la $n$-\'esima renovaci\'on tiene distribuci\'on $G\star F^{\left(n-1\right)\star}\left(t\right)$
\end{Note}


\begin{Teo}
Para una constante $\mu\leq\infty$ ( o variable aleatoria), las siguientes expresiones son equivalentes:

\begin{eqnarray}
lim_{n\rightarrow\infty}n^{-1}T_{n}&=&\mu,\textrm{ c.s.}\\
lim_{t\rightarrow\infty}t^{-1}N\left(t\right)&=&1/\mu,\textrm{ c.s.}
\end{eqnarray}
\end{Teo}


Es decir, $T_{n}$ satisface la Ley Fuerte de los Grandes N\'umeros s\'i y s\'olo s\'i $N\left/t\right)$ la cumple.


\begin{Coro}[Ley Fuerte de los Grandes N\'umeros para Procesos de Renovaci\'on]
Si $N\left(t\right)$ es un proceso de renovaci\'on cuyos tiempos de inter-renovaci\'on tienen media $\mu\leq\infty$, entonces
\begin{eqnarray}
t^{-1}N\left(t\right)\rightarrow 1/\mu,\textrm{ c.s. cuando }t\rightarrow\infty.
\end{eqnarray}

\end{Coro}


Considerar el proceso estoc\'astico de valores reales $\left\{Z\left(t\right):t\geq0\right\}$ en el mismo espacio de probabilidad que $N\left(t\right)$

\begin{Def}
Para el proceso $\left\{Z\left(t\right):t\geq0\right\}$ se define la fluctuaci\'on m\'axima de $Z\left(t\right)$ en el intervalo $\left(T_{n-1},T_{n}\right]$:
\begin{eqnarray*}
M_{n}=\sup_{T_{n-1}<t\leq T_{n}}|Z\left(t\right)-Z\left(T_{n-1}\right)|
\end{eqnarray*}
\end{Def}

\begin{Teo}
Sup\'ongase que $n^{-1}T_{n}\rightarrow\mu$ c.s. cuando $n\rightarrow\infty$, donde $\mu\leq\infty$ es una constante o variable aleatoria. Sea $a$ una constante o variable aleatoria que puede ser infinita cuando $\mu$ es finita, y considere las expresiones l\'imite:
\begin{eqnarray}
lim_{n\rightarrow\infty}n^{-1}Z\left(T_{n}\right)&=&a,\textrm{ c.s.}\\
lim_{t\rightarrow\infty}t^{-1}Z\left(t\right)&=&a/\mu,\textrm{ c.s.}
\end{eqnarray}
La segunda expresi\'on implica la primera. Conversamente, la primera implica la segunda si el proceso $Z\left(t\right)$ es creciente, o si $lim_{n\rightarrow\infty}n^{-1}M_{n}=0$ c.s.
\end{Teo}

\begin{Coro}
Si $N\left(t\right)$ es un proceso de renovaci\'on, y $\left(Z\left(T_{n}\right)-Z\left(T_{n-1}\right),M_{n}\right)$, para $n\geq1$, son variables aleatorias independientes e id\'enticamente distribuidas con media finita, entonces,
\begin{eqnarray}
lim_{t\rightarrow\infty}t^{-1}Z\left(t\right)\rightarrow\frac{\esp\left[Z\left(T_{1}\right)-Z\left(T_{0}\right)\right]}{\esp\left[T_{1}\right]},\textrm{ c.s. cuando  }t\rightarrow\infty.
\end{eqnarray}
\end{Coro}


Sup\'ongase que $N\left(t\right)$ es un proceso de renovaci\'on con distribuci\'on $F$ con media finita $\mu$.

\begin{Def}
La funci\'on de renovaci\'on asociada con la distribuci\'on $F$, del proceso $N\left(t\right)$, es
\begin{eqnarray*}
U\left(t\right)=\sum_{n=1}^{\infty}F^{n\star}\left(t\right),\textrm{   }t\geq0,
\end{eqnarray*}
donde $F^{0\star}\left(t\right)=\indora\left(t\geq0\right)$.
\end{Def}


\begin{Prop}
Sup\'ongase que la distribuci\'on de inter-renovaci\'on $F$ tiene densidad $f$. Entonces $U\left(t\right)$ tambi\'en tiene densidad, para $t>0$, y es $U^{'}\left(t\right)=\sum_{n=0}^{\infty}f^{n\star}\left(t\right)$. Adem\'as
\begin{eqnarray*}
\prob\left\{N\left(t\right)>N\left(t-\right)\right\}=0\textrm{,   }t\geq0.
\end{eqnarray*}
\end{Prop}

\begin{Def}
La Transformada de Laplace-Stieljes de $F$ est\'a dada por

\begin{eqnarray*}
\hat{F}\left(\alpha\right)=\int_{\rea_{+}}e^{-\alpha t}dF\left(t\right)\textrm{,  }\alpha\geq0.
\end{eqnarray*}
\end{Def}

Entonces

\begin{eqnarray*}
\hat{U}\left(\alpha\right)=\sum_{n=0}^{\infty}\hat{F^{n\star}}\left(\alpha\right)=\sum_{n=0}^{\infty}\hat{F}\left(\alpha\right)^{n}=\frac{1}{1-\hat{F}\left(\alpha\right)}.
\end{eqnarray*}


\begin{Prop}
La Transformada de Laplace $\hat{U}\left(\alpha\right)$ y $\hat{F}\left(\alpha\right)$ determina una a la otra de manera \'unica por la relaci\'on $\hat{U}\left(\alpha\right)=\frac{1}{1-\hat{F}\left(\alpha\right)}$.
\end{Prop}


\begin{Note}
Un proceso de renovaci\'on $N\left(t\right)$ cuyos tiempos de inter-renovaci\'on tienen media finita, es un proceso Poisson con tasa $\lambda$ si y s\'olo s\'i $\esp\left[U\left(t\right)\right]=\lambda t$, para $t\geq0$.
\end{Note}


\begin{Teo}
Sea $N\left(t\right)$ un proceso puntual simple con puntos de localizaci\'on $T_{n}$ tal que $\eta\left(t\right)=\esp\left[N\left(\right)\right]$ es finita para cada $t$. Entonces para cualquier funci\'on $f:\rea_{+}\rightarrow\rea$,
\begin{eqnarray*}
\esp\left[\sum_{n=1}^{N\left(\right)}f\left(T_{n}\right)\right]=\int_{\left(0,t\right]}f\left(s\right)d\eta\left(s\right)\textrm{,  }t\geq0,
\end{eqnarray*}
suponiendo que la integral exista. Adem\'as si $X_{1},X_{2},\ldots$ son variables aleatorias definidas en el mismo espacio de probabilidad que el proceso $N\left(t\right)$ tal que $\esp\left[X_{n}|T_{n}=s\right]=f\left(s\right)$, independiente de $n$. Entonces
\begin{eqnarray*}
\esp\left[\sum_{n=1}^{N\left(t\right)}X_{n}\right]=\int_{\left(0,t\right]}f\left(s\right)d\eta\left(s\right)\textrm{,  }t\geq0,
\end{eqnarray*} 
suponiendo que la integral exista. 
\end{Teo}

\begin{Coro}[Identidad de Wald para Renovaciones]
Para el proceso de renovaci\'on $N\left(t\right)$,
\begin{eqnarray*}
\esp\left[T_{N\left(t\right)+1}\right]=\mu\esp\left[N\left(t\right)+1\right]\textrm{,  }t\geq0,
\end{eqnarray*}  
\end{Coro}


\begin{Def}
Sea $h\left(t\right)$ funci\'on de valores reales en $\rea$ acotada en intervalos finitos e igual a cero para $t<0$ La ecuaci\'on de renovaci\'on para $h\left(t\right)$ y la distribuci\'on $F$ es

\begin{eqnarray}\label{Ec.Renovacion}
H\left(t\right)=h\left(t\right)+\int_{\left[0,t\right]}H\left(t-s\right)dF\left(s\right)\textrm{,    }t\geq0,
\end{eqnarray}
donde $H\left(t\right)$ es una funci\'on de valores reales. Esto es $H=h+F\star H$. Decimos que $H\left(t\right)$ es soluci\'on de esta ecuaci\'on si satisface la ecuaci\'on, y es acotada en intervalos finitos e iguales a cero para $t<0$.
\end{Def}

\begin{Prop}
La funci\'on $U\star h\left(t\right)$ es la \'unica soluci\'on de la ecuaci\'on de renovaci\'on (\ref{Ec.Renovacion}).
\end{Prop}

\begin{Teo}[Teorema Renovaci\'on Elemental]
\begin{eqnarray*}
t^{-1}U\left(t\right)\rightarrow 1/\mu\textrm{,    cuando }t\rightarrow\infty.
\end{eqnarray*}
\end{Teo}



Sup\'ongase que $N\left(t\right)$ es un proceso de renovaci\'on con distribuci\'on $F$ con media finita $\mu$.

\begin{Def}
La funci\'on de renovaci\'on asociada con la distribuci\'on $F$, del proceso $N\left(t\right)$, es
\begin{eqnarray*}
U\left(t\right)=\sum_{n=1}^{\infty}F^{n\star}\left(t\right),\textrm{   }t\geq0,
\end{eqnarray*}
donde $F^{0\star}\left(t\right)=\indora\left(t\geq0\right)$.
\end{Def}


\begin{Prop}
Sup\'ongase que la distribuci\'on de inter-renovaci\'on $F$ tiene densidad $f$. Entonces $U\left(t\right)$ tambi\'en tiene densidad, para $t>0$, y es $U^{'}\left(t\right)=\sum_{n=0}^{\infty}f^{n\star}\left(t\right)$. Adem\'as
\begin{eqnarray*}
\prob\left\{N\left(t\right)>N\left(t-\right)\right\}=0\textrm{,   }t\geq0.
\end{eqnarray*}
\end{Prop}

\begin{Def}
La Transformada de Laplace-Stieljes de $F$ est\'a dada por

\begin{eqnarray*}
\hat{F}\left(\alpha\right)=\int_{\rea_{+}}e^{-\alpha t}dF\left(t\right)\textrm{,  }\alpha\geq0.
\end{eqnarray*}
\end{Def}

Entonces

\begin{eqnarray*}
\hat{U}\left(\alpha\right)=\sum_{n=0}^{\infty}\hat{F^{n\star}}\left(\alpha\right)=\sum_{n=0}^{\infty}\hat{F}\left(\alpha\right)^{n}=\frac{1}{1-\hat{F}\left(\alpha\right)}.
\end{eqnarray*}


\begin{Prop}
La Transformada de Laplace $\hat{U}\left(\alpha\right)$ y $\hat{F}\left(\alpha\right)$ determina una a la otra de manera \'unica por la relaci\'on $\hat{U}\left(\alpha\right)=\frac{1}{1-\hat{F}\left(\alpha\right)}$.
\end{Prop}


\begin{Note}
Un proceso de renovaci\'on $N\left(t\right)$ cuyos tiempos de inter-renovaci\'on tienen media finita, es un proceso Poisson con tasa $\lambda$ si y s\'olo s\'i $\esp\left[U\left(t\right)\right]=\lambda t$, para $t\geq0$.
\end{Note}


\begin{Teo}
Sea $N\left(t\right)$ un proceso puntual simple con puntos de localizaci\'on $T_{n}$ tal que $\eta\left(t\right)=\esp\left[N\left(\right)\right]$ es finita para cada $t$. Entonces para cualquier funci\'on $f:\rea_{+}\rightarrow\rea$,
\begin{eqnarray*}
\esp\left[\sum_{n=1}^{N\left(\right)}f\left(T_{n}\right)\right]=\int_{\left(0,t\right]}f\left(s\right)d\eta\left(s\right)\textrm{,  }t\geq0,
\end{eqnarray*}
suponiendo que la integral exista. Adem\'as si $X_{1},X_{2},\ldots$ son variables aleatorias definidas en el mismo espacio de probabilidad que el proceso $N\left(t\right)$ tal que $\esp\left[X_{n}|T_{n}=s\right]=f\left(s\right)$, independiente de $n$. Entonces
\begin{eqnarray*}
\esp\left[\sum_{n=1}^{N\left(t\right)}X_{n}\right]=\int_{\left(0,t\right]}f\left(s\right)d\eta\left(s\right)\textrm{,  }t\geq0,
\end{eqnarray*} 
suponiendo que la integral exista. 
\end{Teo}

\begin{Coro}[Identidad de Wald para Renovaciones]
Para el proceso de renovaci\'on $N\left(t\right)$,
\begin{eqnarray*}
\esp\left[T_{N\left(t\right)+1}\right]=\mu\esp\left[N\left(t\right)+1\right]\textrm{,  }t\geq0,
\end{eqnarray*}  
\end{Coro}


\begin{Def}
Sea $h\left(t\right)$ funci\'on de valores reales en $\rea$ acotada en intervalos finitos e igual a cero para $t<0$ La ecuaci\'on de renovaci\'on para $h\left(t\right)$ y la distribuci\'on $F$ es

\begin{eqnarray}\label{Ec.Renovacion}
H\left(t\right)=h\left(t\right)+\int_{\left[0,t\right]}H\left(t-s\right)dF\left(s\right)\textrm{,    }t\geq0,
\end{eqnarray}
donde $H\left(t\right)$ es una funci\'on de valores reales. Esto es $H=h+F\star H$. Decimos que $H\left(t\right)$ es soluci\'on de esta ecuaci\'on si satisface la ecuaci\'on, y es acotada en intervalos finitos e iguales a cero para $t<0$.
\end{Def}

\begin{Prop}
La funci\'on $U\star h\left(t\right)$ es la \'unica soluci\'on de la ecuaci\'on de renovaci\'on (\ref{Ec.Renovacion}).
\end{Prop}

\begin{Teo}[Teorema Renovaci\'on Elemental]
\begin{eqnarray*}
t^{-1}U\left(t\right)\rightarrow 1/\mu\textrm{,    cuando }t\rightarrow\infty.
\end{eqnarray*}
\end{Teo}


\begin{Note} Una funci\'on $h:\rea_{+}\rightarrow\rea$ es Directamente Riemann Integrable en los siguientes casos:
\begin{itemize}
\item[a)] $h\left(t\right)\geq0$ es decreciente y Riemann Integrable.
\item[b)] $h$ es continua excepto posiblemente en un conjunto de Lebesgue de medida 0, y $|h\left(t\right)|\leq b\left(t\right)$, donde $b$ es DRI.
\end{itemize}
\end{Note}

\begin{Teo}[Teorema Principal de Renovaci\'on]
Si $F$ es no aritm\'etica y $h\left(t\right)$ es Directamente Riemann Integrable (DRI), entonces

\begin{eqnarray*}
lim_{t\rightarrow\infty}U\star h=\frac{1}{\mu}\int_{\rea_{+}}h\left(s\right)ds.
\end{eqnarray*}
\end{Teo}

\begin{Prop}
Cualquier funci\'on $H\left(t\right)$ acotada en intervalos finitos y que es 0 para $t<0$ puede expresarse como
\begin{eqnarray*}
H\left(t\right)=U\star h\left(t\right)\textrm{,  donde }h\left(t\right)=H\left(t\right)-F\star H\left(t\right)
\end{eqnarray*}
\end{Prop}

\begin{Def}
Un proceso estoc\'astico $X\left(t\right)$ es crudamente regenerativo en un tiempo aleatorio positivo $T$ si
\begin{eqnarray*}
\esp\left[X\left(T+t\right)|T\right]=\esp\left[X\left(t\right)\right]\textrm{, para }t\geq0,\end{eqnarray*}
y con las esperanzas anteriores finitas.
\end{Def}

\begin{Prop}
Sup\'ongase que $X\left(t\right)$ es un proceso crudamente regenerativo en $T$, que tiene distribuci\'on $F$. Si $\esp\left[X\left(t\right)\right]$ es acotado en intervalos finitos, entonces
\begin{eqnarray*}
\esp\left[X\left(t\right)\right]=U\star h\left(t\right)\textrm{,  donde }h\left(t\right)=\esp\left[X\left(t\right)\indora\left(T>t\right)\right].
\end{eqnarray*}
\end{Prop}

\begin{Teo}[Regeneraci\'on Cruda]
Sup\'ongase que $X\left(t\right)$ es un proceso con valores positivo crudamente regenerativo en $T$, y def\'inase $M=\sup\left\{|X\left(t\right)|:t\leq T\right\}$. Si $T$ es no aritm\'etico y $M$ y $MT$ tienen media finita, entonces
\begin{eqnarray*}
lim_{t\rightarrow\infty}\esp\left[X\left(t\right)\right]=\frac{1}{\mu}\int_{\rea_{+}}h\left(s\right)ds,
\end{eqnarray*}
donde $h\left(t\right)=\esp\left[X\left(t\right)\indora\left(T>t\right)\right]$.
\end{Teo}


\begin{Note} Una funci\'on $h:\rea_{+}\rightarrow\rea$ es Directamente Riemann Integrable en los siguientes casos:
\begin{itemize}
\item[a)] $h\left(t\right)\geq0$ es decreciente y Riemann Integrable.
\item[b)] $h$ es continua excepto posiblemente en un conjunto de Lebesgue de medida 0, y $|h\left(t\right)|\leq b\left(t\right)$, donde $b$ es DRI.
\end{itemize}
\end{Note}

\begin{Teo}[Teorema Principal de Renovaci\'on]
Si $F$ es no aritm\'etica y $h\left(t\right)$ es Directamente Riemann Integrable (DRI), entonces

\begin{eqnarray*}
lim_{t\rightarrow\infty}U\star h=\frac{1}{\mu}\int_{\rea_{+}}h\left(s\right)ds.
\end{eqnarray*}
\end{Teo}

\begin{Prop}
Cualquier funci\'on $H\left(t\right)$ acotada en intervalos finitos y que es 0 para $t<0$ puede expresarse como
\begin{eqnarray*}
H\left(t\right)=U\star h\left(t\right)\textrm{,  donde }h\left(t\right)=H\left(t\right)-F\star H\left(t\right)
\end{eqnarray*}
\end{Prop}

\begin{Def}
Un proceso estoc\'astico $X\left(t\right)$ es crudamente regenerativo en un tiempo aleatorio positivo $T$ si
\begin{eqnarray*}
\esp\left[X\left(T+t\right)|T\right]=\esp\left[X\left(t\right)\right]\textrm{, para }t\geq0,\end{eqnarray*}
y con las esperanzas anteriores finitas.
\end{Def}

\begin{Prop}
Sup\'ongase que $X\left(t\right)$ es un proceso crudamente regenerativo en $T$, que tiene distribuci\'on $F$. Si $\esp\left[X\left(t\right)\right]$ es acotado en intervalos finitos, entonces
\begin{eqnarray*}
\esp\left[X\left(t\right)\right]=U\star h\left(t\right)\textrm{,  donde }h\left(t\right)=\esp\left[X\left(t\right)\indora\left(T>t\right)\right].
\end{eqnarray*}
\end{Prop}

\begin{Teo}[Regeneraci\'on Cruda]
Sup\'ongase que $X\left(t\right)$ es un proceso con valores positivo crudamente regenerativo en $T$, y def\'inase $M=\sup\left\{|X\left(t\right)|:t\leq T\right\}$. Si $T$ es no aritm\'etico y $M$ y $MT$ tienen media finita, entonces
\begin{eqnarray*}
lim_{t\rightarrow\infty}\esp\left[X\left(t\right)\right]=\frac{1}{\mu}\int_{\rea_{+}}h\left(s\right)ds,
\end{eqnarray*}
donde $h\left(t\right)=\esp\left[X\left(t\right)\indora\left(T>t\right)\right]$.
\end{Teo}
%________________________________________________________________________
\subsection{Procesos Regenerativos}
%________________________________________________________________________

Para $\left\{X\left(t\right):t\geq0\right\}$ Proceso Estoc\'astico a tiempo continuo con estado de espacios $S$, que es un espacio m\'etrico, con trayectorias continuas por la derecha y con l\'imites por la izquierda c.s. Sea $N\left(t\right)$ un proceso de renovaci\'on en $\rea_{+}$ definido en el mismo espacio de probabilidad que $X\left(t\right)$, con tiempos de renovaci\'on $T$ y tiempos de inter-renovaci\'on $\xi_{n}=T_{n}-T_{n-1}$, con misma distribuci\'on $F$ de media finita $\mu$.



\begin{Def}
Para el proceso $\left\{\left(N\left(t\right),X\left(t\right)\right):t\geq0\right\}$, sus trayectoria muestrales en el intervalo de tiempo $\left[T_{n-1},T_{n}\right)$ est\'an descritas por
\begin{eqnarray*}
\zeta_{n}=\left(\xi_{n},\left\{X\left(T_{n-1}+t\right):0\leq t<\xi_{n}\right\}\right)
\end{eqnarray*}
Este $\zeta_{n}$ es el $n$-\'esimo segmento del proceso. El proceso es regenerativo sobre los tiempos $T_{n}$ si sus segmentos $\zeta_{n}$ son independientes e id\'enticamennte distribuidos.
\end{Def}


\begin{Obs}
Si $\tilde{X}\left(t\right)$ con espacio de estados $\tilde{S}$ es regenerativo sobre $T_{n}$, entonces $X\left(t\right)=f\left(\tilde{X}\left(t\right)\right)$ tambi\'en es regenerativo sobre $T_{n}$, para cualquier funci\'on $f:\tilde{S}\rightarrow S$.
\end{Obs}

\begin{Obs}
Los procesos regenerativos son crudamente regenerativos, pero no al rev\'es.
\end{Obs}

\begin{Def}[Definici\'on Cl\'asica]
Un proceso estoc\'astico $X=\left\{X\left(t\right):t\geq0\right\}$ es llamado regenerativo is existe una variable aleatoria $R_{1}>0$ tal que
\begin{itemize}
\item[i)] $\left\{X\left(t+R_{1}\right):t\geq0\right\}$ es independiente de $\left\{\left\{X\left(t\right):t<R_{1}\right\},\right\}$
\item[ii)] $\left\{X\left(t+R_{1}\right):t\geq0\right\}$ es estoc\'asticamente equivalente a $\left\{X\left(t\right):t>0\right\}$
\end{itemize}

Llamamos a $R_{1}$ tiempo de regeneraci\'on, y decimos que $X$ se regenera en este punto.
\end{Def}

$\left\{X\left(t+R_{1}\right)\right\}$ es regenerativo con tiempo de regeneraci\'on $R_{2}$, independiente de $R_{1}$ pero con la misma distribuci\'on que $R_{1}$. Procediendo de esta manera se obtiene una secuencia de variables aleatorias independientes e id\'enticamente distribuidas $\left\{R_{n}\right\}$ llamados longitudes de ciclo. Si definimos a $Z_{k}\equiv R_{1}+R_{2}+\cdots+R_{k}$, se tiene un proceso de renovaci\'on llamado proceso de renovaci\'on encajado para $X$.

\begin{Note}
Un proceso regenerativo con media de la longitud de ciclo finita es llamado positivo recurrente.
\end{Note}


\begin{Def}
Para $x$ fijo y para cada $t\geq0$, sea $I_{x}\left(t\right)=1$ si $X\left(t\right)\leq x$,  $I_{x}\left(t\right)=0$ en caso contrario, y def\'inanse los tiempos promedio
\begin{eqnarray*}
\overline{X}&=&lim_{t\rightarrow\infty}\frac{1}{t}\int_{0}^{\infty}X\left(u\right)du\\
\prob\left(X_{\infty}\leq x\right)&=&lim_{t\rightarrow\infty}\frac{1}{t}\int_{0}^{\infty}I_{x}\left(u\right)du,
\end{eqnarray*}
cuando estos l\'imites existan.
\end{Def}

Como consecuencia del teorema de Renovaci\'on-Recompensa, se tiene que el primer l\'imite  existe y es igual a la constante
\begin{eqnarray*}
\overline{X}&=&\frac{\esp\left[\int_{0}^{R_{1}}X\left(t\right)dt\right]}{\esp\left[R_{1}\right]},
\end{eqnarray*}
suponiendo que ambas esperanzas son finitas.

\begin{Note}
\begin{itemize}
\item[a)] Si el proceso regenerativo $X$ es positivo recurrente y tiene trayectorias muestrales no negativas, entonces la ecuaci\'on anterior es v\'alida.
\item[b)] Si $X$ es positivo recurrente regenerativo, podemos construir una \'unica versi\'on estacionaria de este proceso, $X_{e}=\left\{X_{e}\left(t\right)\right\}$, donde $X_{e}$ es un proceso estoc\'astico regenerativo y estrictamente estacionario, con distribuci\'on marginal distribuida como $X_{\infty}$
\end{itemize}
\end{Note}

%________________________________________________________________________
\subsection{Procesos Regenerativos}
%________________________________________________________________________

Para $\left\{X\left(t\right):t\geq0\right\}$ Proceso Estoc\'astico a tiempo continuo con estado de espacios $S$, que es un espacio m\'etrico, con trayectorias continuas por la derecha y con l\'imites por la izquierda c.s. Sea $N\left(t\right)$ un proceso de renovaci\'on en $\rea_{+}$ definido en el mismo espacio de probabilidad que $X\left(t\right)$, con tiempos de renovaci\'on $T$ y tiempos de inter-renovaci\'on $\xi_{n}=T_{n}-T_{n-1}$, con misma distribuci\'on $F$ de media finita $\mu$.



\begin{Def}
Para el proceso $\left\{\left(N\left(t\right),X\left(t\right)\right):t\geq0\right\}$, sus trayectoria muestrales en el intervalo de tiempo $\left[T_{n-1},T_{n}\right)$ est\'an descritas por
\begin{eqnarray*}
\zeta_{n}=\left(\xi_{n},\left\{X\left(T_{n-1}+t\right):0\leq t<\xi_{n}\right\}\right)
\end{eqnarray*}
Este $\zeta_{n}$ es el $n$-\'esimo segmento del proceso. El proceso es regenerativo sobre los tiempos $T_{n}$ si sus segmentos $\zeta_{n}$ son independientes e id\'enticamennte distribuidos.
\end{Def}


\begin{Obs}
Si $\tilde{X}\left(t\right)$ con espacio de estados $\tilde{S}$ es regenerativo sobre $T_{n}$, entonces $X\left(t\right)=f\left(\tilde{X}\left(t\right)\right)$ tambi\'en es regenerativo sobre $T_{n}$, para cualquier funci\'on $f:\tilde{S}\rightarrow S$.
\end{Obs}

\begin{Obs}
Los procesos regenerativos son crudamente regenerativos, pero no al rev\'es.
\end{Obs}

\begin{Def}[Definici\'on Cl\'asica]
Un proceso estoc\'astico $X=\left\{X\left(t\right):t\geq0\right\}$ es llamado regenerativo is existe una variable aleatoria $R_{1}>0$ tal que
\begin{itemize}
\item[i)] $\left\{X\left(t+R_{1}\right):t\geq0\right\}$ es independiente de $\left\{\left\{X\left(t\right):t<R_{1}\right\},\right\}$
\item[ii)] $\left\{X\left(t+R_{1}\right):t\geq0\right\}$ es estoc\'asticamente equivalente a $\left\{X\left(t\right):t>0\right\}$
\end{itemize}

Llamamos a $R_{1}$ tiempo de regeneraci\'on, y decimos que $X$ se regenera en este punto.
\end{Def}

$\left\{X\left(t+R_{1}\right)\right\}$ es regenerativo con tiempo de regeneraci\'on $R_{2}$, independiente de $R_{1}$ pero con la misma distribuci\'on que $R_{1}$. Procediendo de esta manera se obtiene una secuencia de variables aleatorias independientes e id\'enticamente distribuidas $\left\{R_{n}\right\}$ llamados longitudes de ciclo. Si definimos a $Z_{k}\equiv R_{1}+R_{2}+\cdots+R_{k}$, se tiene un proceso de renovaci\'on llamado proceso de renovaci\'on encajado para $X$.

\begin{Note}
Un proceso regenerativo con media de la longitud de ciclo finita es llamado positivo recurrente.
\end{Note}


\begin{Def}
Para $x$ fijo y para cada $t\geq0$, sea $I_{x}\left(t\right)=1$ si $X\left(t\right)\leq x$,  $I_{x}\left(t\right)=0$ en caso contrario, y def\'inanse los tiempos promedio
\begin{eqnarray*}
\overline{X}&=&lim_{t\rightarrow\infty}\frac{1}{t}\int_{0}^{\infty}X\left(u\right)du\\
\prob\left(X_{\infty}\leq x\right)&=&lim_{t\rightarrow\infty}\frac{1}{t}\int_{0}^{\infty}I_{x}\left(u\right)du,
\end{eqnarray*}
cuando estos l\'imites existan.
\end{Def}

Como consecuencia del teorema de Renovaci\'on-Recompensa, se tiene que el primer l\'imite  existe y es igual a la constante
\begin{eqnarray*}
\overline{X}&=&\frac{\esp\left[\int_{0}^{R_{1}}X\left(t\right)dt\right]}{\esp\left[R_{1}\right]},
\end{eqnarray*}
suponiendo que ambas esperanzas son finitas.

\begin{Note}
\begin{itemize}
\item[a)] Si el proceso regenerativo $X$ es positivo recurrente y tiene trayectorias muestrales no negativas, entonces la ecuaci\'on anterior es v\'alida.
\item[b)] Si $X$ es positivo recurrente regenerativo, podemos construir una \'unica versi\'on estacionaria de este proceso, $X_{e}=\left\{X_{e}\left(t\right)\right\}$, donde $X_{e}$ es un proceso estoc\'astico regenerativo y estrictamente estacionario, con distribuci\'on marginal distribuida como $X_{\infty}$
\end{itemize}
\end{Note}
%__________________________________________________________________________________________
\subsection{Procesos Regenerativos Estacionarios - Stidham \cite{Stidham}}
%__________________________________________________________________________________________


Un proceso estoc\'astico a tiempo continuo $\left\{V\left(t\right),t\geq0\right\}$ es un proceso regenerativo si existe una sucesi\'on de variables aleatorias independientes e id\'enticamente distribuidas $\left\{X_{1},X_{2},\ldots\right\}$, sucesi\'on de renovaci\'on, tal que para cualquier conjunto de Borel $A$, 

\begin{eqnarray*}
\prob\left\{V\left(t\right)\in A|X_{1}+X_{2}+\cdots+X_{R\left(t\right)}=s,\left\{V\left(\tau\right),\tau<s\right\}\right\}=\prob\left\{V\left(t-s\right)\in A|X_{1}>t-s\right\},
\end{eqnarray*}
para todo $0\leq s\leq t$, donde $R\left(t\right)=\max\left\{X_{1}+X_{2}+\cdots+X_{j}\leq t\right\}=$n\'umero de renovaciones ({\emph{puntos de regeneraci\'on}}) que ocurren en $\left[0,t\right]$. El intervalo $\left[0,X_{1}\right)$ es llamado {\emph{primer ciclo de regeneraci\'on}} de $\left\{V\left(t \right),t\geq0\right\}$, $\left[X_{1},X_{1}+X_{2}\right)$ el {\emph{segundo ciclo de regeneraci\'on}}, y as\'i sucesivamente.

Sea $X=X_{1}$ y sea $F$ la funci\'on de distrbuci\'on de $X$


\begin{Def}
Se define el proceso estacionario, $\left\{V^{*}\left(t\right),t\geq0\right\}$, para $\left\{V\left(t\right),t\geq0\right\}$ por

\begin{eqnarray*}
\prob\left\{V\left(t\right)\in A\right\}=\frac{1}{\esp\left[X\right]}\int_{0}^{\infty}\prob\left\{V\left(t+x\right)\in A|X>x\right\}\left(1-F\left(x\right)\right)dx,
\end{eqnarray*} 
para todo $t\geq0$ y todo conjunto de Borel $A$.
\end{Def}

\begin{Def}
Una distribuci\'on se dice que es {\emph{aritm\'etica}} si todos sus puntos de incremento son m\'ultiplos de la forma $0,\lambda, 2\lambda,\ldots$ para alguna $\lambda>0$ entera.
\end{Def}


\begin{Def}
Una modificaci\'on medible de un proceso $\left\{V\left(t\right),t\geq0\right\}$, es una versi\'on de este, $\left\{V\left(t,w\right)\right\}$ conjuntamente medible para $t\geq0$ y para $w\in S$, $S$ espacio de estados para $\left\{V\left(t\right),t\geq0\right\}$.
\end{Def}

\begin{Teo}
Sea $\left\{V\left(t\right),t\geq\right\}$ un proceso regenerativo no negativo con modificaci\'on medible. Sea $\esp\left[X\right]<\infty$. Entonces el proceso estacionario dado por la ecuaci\'on anterior est\'a bien definido y tiene funci\'on de distribuci\'on independiente de $t$, adem\'as
\begin{itemize}
\item[i)] \begin{eqnarray*}
\esp\left[V^{*}\left(0\right)\right]&=&\frac{\esp\left[\int_{0}^{X}V\left(s\right)ds\right]}{\esp\left[X\right]}\end{eqnarray*}
\item[ii)] Si $\esp\left[V^{*}\left(0\right)\right]<\infty$, equivalentemente, si $\esp\left[\int_{0}^{X}V\left(s\right)ds\right]<\infty$,entonces
\begin{eqnarray*}
\frac{\int_{0}^{t}V\left(s\right)ds}{t}\rightarrow\frac{\esp\left[\int_{0}^{X}V\left(s\right)ds\right]}{\esp\left[X\right]}
\end{eqnarray*}
con probabilidad 1 y en media, cuando $t\rightarrow\infty$.
\end{itemize}
\end{Teo}


%__________________________________________________________________________________________
\subsection{Procesos Regenerativos Estacionarios - Stidham \cite{Stidham}}
%__________________________________________________________________________________________


Un proceso estoc\'astico a tiempo continuo $\left\{V\left(t\right),t\geq0\right\}$ es un proceso regenerativo si existe una sucesi\'on de variables aleatorias independientes e id\'enticamente distribuidas $\left\{X_{1},X_{2},\ldots\right\}$, sucesi\'on de renovaci\'on, tal que para cualquier conjunto de Borel $A$, 

\begin{eqnarray*}
\prob\left\{V\left(t\right)\in A|X_{1}+X_{2}+\cdots+X_{R\left(t\right)}=s,\left\{V\left(\tau\right),\tau<s\right\}\right\}=\prob\left\{V\left(t-s\right)\in A|X_{1}>t-s\right\},
\end{eqnarray*}
para todo $0\leq s\leq t$, donde $R\left(t\right)=\max\left\{X_{1}+X_{2}+\cdots+X_{j}\leq t\right\}=$n\'umero de renovaciones ({\emph{puntos de regeneraci\'on}}) que ocurren en $\left[0,t\right]$. El intervalo $\left[0,X_{1}\right)$ es llamado {\emph{primer ciclo de regeneraci\'on}} de $\left\{V\left(t \right),t\geq0\right\}$, $\left[X_{1},X_{1}+X_{2}\right)$ el {\emph{segundo ciclo de regeneraci\'on}}, y as\'i sucesivamente.

Sea $X=X_{1}$ y sea $F$ la funci\'on de distrbuci\'on de $X$


\begin{Def}
Se define el proceso estacionario, $\left\{V^{*}\left(t\right),t\geq0\right\}$, para $\left\{V\left(t\right),t\geq0\right\}$ por

\begin{eqnarray*}
\prob\left\{V\left(t\right)\in A\right\}=\frac{1}{\esp\left[X\right]}\int_{0}^{\infty}\prob\left\{V\left(t+x\right)\in A|X>x\right\}\left(1-F\left(x\right)\right)dx,
\end{eqnarray*} 
para todo $t\geq0$ y todo conjunto de Borel $A$.
\end{Def}

\begin{Def}
Una distribuci\'on se dice que es {\emph{aritm\'etica}} si todos sus puntos de incremento son m\'ultiplos de la forma $0,\lambda, 2\lambda,\ldots$ para alguna $\lambda>0$ entera.
\end{Def}


\begin{Def}
Una modificaci\'on medible de un proceso $\left\{V\left(t\right),t\geq0\right\}$, es una versi\'on de este, $\left\{V\left(t,w\right)\right\}$ conjuntamente medible para $t\geq0$ y para $w\in S$, $S$ espacio de estados para $\left\{V\left(t\right),t\geq0\right\}$.
\end{Def}

\begin{Teo}
Sea $\left\{V\left(t\right),t\geq\right\}$ un proceso regenerativo no negativo con modificaci\'on medible. Sea $\esp\left[X\right]<\infty$. Entonces el proceso estacionario dado por la ecuaci\'on anterior est\'a bien definido y tiene funci\'on de distribuci\'on independiente de $t$, adem\'as
\begin{itemize}
\item[i)] \begin{eqnarray*}
\esp\left[V^{*}\left(0\right)\right]&=&\frac{\esp\left[\int_{0}^{X}V\left(s\right)ds\right]}{\esp\left[X\right]}\end{eqnarray*}
\item[ii)] Si $\esp\left[V^{*}\left(0\right)\right]<\infty$, equivalentemente, si $\esp\left[\int_{0}^{X}V\left(s\right)ds\right]<\infty$,entonces
\begin{eqnarray*}
\frac{\int_{0}^{t}V\left(s\right)ds}{t}\rightarrow\frac{\esp\left[\int_{0}^{X}V\left(s\right)ds\right]}{\esp\left[X\right]}
\end{eqnarray*}
con probabilidad 1 y en media, cuando $t\rightarrow\infty$.
\end{itemize}
\end{Teo}
%___________________________________________________________________________________________
%
\subsection{Propiedades de los Procesos de Renovaci\'on}
%___________________________________________________________________________________________
%

Los tiempos $T_{n}$ est\'an relacionados con los conteos de $N\left(t\right)$ por

\begin{eqnarray*}
\left\{N\left(t\right)\geq n\right\}&=&\left\{T_{n}\leq t\right\}\\
T_{N\left(t\right)}\leq &t&<T_{N\left(t\right)+1},
\end{eqnarray*}

adem\'as $N\left(T_{n}\right)=n$, y 

\begin{eqnarray*}
N\left(t\right)=\max\left\{n:T_{n}\leq t\right\}=\min\left\{n:T_{n+1}>t\right\}
\end{eqnarray*}

Por propiedades de la convoluci\'on se sabe que

\begin{eqnarray*}
P\left\{T_{n}\leq t\right\}=F^{n\star}\left(t\right)
\end{eqnarray*}
que es la $n$-\'esima convoluci\'on de $F$. Entonces 

\begin{eqnarray*}
\left\{N\left(t\right)\geq n\right\}&=&\left\{T_{n}\leq t\right\}\\
P\left\{N\left(t\right)\leq n\right\}&=&1-F^{\left(n+1\right)\star}\left(t\right)
\end{eqnarray*}

Adem\'as usando el hecho de que $\esp\left[N\left(t\right)\right]=\sum_{n=1}^{\infty}P\left\{N\left(t\right)\geq n\right\}$
se tiene que

\begin{eqnarray*}
\esp\left[N\left(t\right)\right]=\sum_{n=1}^{\infty}F^{n\star}\left(t\right)
\end{eqnarray*}

\begin{Prop}
Para cada $t\geq0$, la funci\'on generadora de momentos $\esp\left[e^{\alpha N\left(t\right)}\right]$ existe para alguna $\alpha$ en una vecindad del 0, y de aqu\'i que $\esp\left[N\left(t\right)^{m}\right]<\infty$, para $m\geq1$.
\end{Prop}


\begin{Note}
Si el primer tiempo de renovaci\'on $\xi_{1}$ no tiene la misma distribuci\'on que el resto de las $\xi_{n}$, para $n\geq2$, a $N\left(t\right)$ se le llama Proceso de Renovaci\'on retardado, donde si $\xi$ tiene distribuci\'on $G$, entonces el tiempo $T_{n}$ de la $n$-\'esima renovaci\'on tiene distribuci\'on $G\star F^{\left(n-1\right)\star}\left(t\right)$
\end{Note}


\begin{Teo}
Para una constante $\mu\leq\infty$ ( o variable aleatoria), las siguientes expresiones son equivalentes:

\begin{eqnarray}
lim_{n\rightarrow\infty}n^{-1}T_{n}&=&\mu,\textrm{ c.s.}\\
lim_{t\rightarrow\infty}t^{-1}N\left(t\right)&=&1/\mu,\textrm{ c.s.}
\end{eqnarray}
\end{Teo}


Es decir, $T_{n}$ satisface la Ley Fuerte de los Grandes N\'umeros s\'i y s\'olo s\'i $N\left/t\right)$ la cumple.


\begin{Coro}[Ley Fuerte de los Grandes N\'umeros para Procesos de Renovaci\'on]
Si $N\left(t\right)$ es un proceso de renovaci\'on cuyos tiempos de inter-renovaci\'on tienen media $\mu\leq\infty$, entonces
\begin{eqnarray}
t^{-1}N\left(t\right)\rightarrow 1/\mu,\textrm{ c.s. cuando }t\rightarrow\infty.
\end{eqnarray}

\end{Coro}


Considerar el proceso estoc\'astico de valores reales $\left\{Z\left(t\right):t\geq0\right\}$ en el mismo espacio de probabilidad que $N\left(t\right)$

\begin{Def}
Para el proceso $\left\{Z\left(t\right):t\geq0\right\}$ se define la fluctuaci\'on m\'axima de $Z\left(t\right)$ en el intervalo $\left(T_{n-1},T_{n}\right]$:
\begin{eqnarray*}
M_{n}=\sup_{T_{n-1}<t\leq T_{n}}|Z\left(t\right)-Z\left(T_{n-1}\right)|
\end{eqnarray*}
\end{Def}

\begin{Teo}
Sup\'ongase que $n^{-1}T_{n}\rightarrow\mu$ c.s. cuando $n\rightarrow\infty$, donde $\mu\leq\infty$ es una constante o variable aleatoria. Sea $a$ una constante o variable aleatoria que puede ser infinita cuando $\mu$ es finita, y considere las expresiones l\'imite:
\begin{eqnarray}
lim_{n\rightarrow\infty}n^{-1}Z\left(T_{n}\right)&=&a,\textrm{ c.s.}\\
lim_{t\rightarrow\infty}t^{-1}Z\left(t\right)&=&a/\mu,\textrm{ c.s.}
\end{eqnarray}
La segunda expresi\'on implica la primera. Conversamente, la primera implica la segunda si el proceso $Z\left(t\right)$ es creciente, o si $lim_{n\rightarrow\infty}n^{-1}M_{n}=0$ c.s.
\end{Teo}

\begin{Coro}
Si $N\left(t\right)$ es un proceso de renovaci\'on, y $\left(Z\left(T_{n}\right)-Z\left(T_{n-1}\right),M_{n}\right)$, para $n\geq1$, son variables aleatorias independientes e id\'enticamente distribuidas con media finita, entonces,
\begin{eqnarray}
lim_{t\rightarrow\infty}t^{-1}Z\left(t\right)\rightarrow\frac{\esp\left[Z\left(T_{1}\right)-Z\left(T_{0}\right)\right]}{\esp\left[T_{1}\right]},\textrm{ c.s. cuando  }t\rightarrow\infty.
\end{eqnarray}
\end{Coro}


%___________________________________________________________________________________________
%
\subsection{Propiedades de los Procesos de Renovaci\'on}
%___________________________________________________________________________________________
%

Los tiempos $T_{n}$ est\'an relacionados con los conteos de $N\left(t\right)$ por

\begin{eqnarray*}
\left\{N\left(t\right)\geq n\right\}&=&\left\{T_{n}\leq t\right\}\\
T_{N\left(t\right)}\leq &t&<T_{N\left(t\right)+1},
\end{eqnarray*}

adem\'as $N\left(T_{n}\right)=n$, y 

\begin{eqnarray*}
N\left(t\right)=\max\left\{n:T_{n}\leq t\right\}=\min\left\{n:T_{n+1}>t\right\}
\end{eqnarray*}

Por propiedades de la convoluci\'on se sabe que

\begin{eqnarray*}
P\left\{T_{n}\leq t\right\}=F^{n\star}\left(t\right)
\end{eqnarray*}
que es la $n$-\'esima convoluci\'on de $F$. Entonces 

\begin{eqnarray*}
\left\{N\left(t\right)\geq n\right\}&=&\left\{T_{n}\leq t\right\}\\
P\left\{N\left(t\right)\leq n\right\}&=&1-F^{\left(n+1\right)\star}\left(t\right)
\end{eqnarray*}

Adem\'as usando el hecho de que $\esp\left[N\left(t\right)\right]=\sum_{n=1}^{\infty}P\left\{N\left(t\right)\geq n\right\}$
se tiene que

\begin{eqnarray*}
\esp\left[N\left(t\right)\right]=\sum_{n=1}^{\infty}F^{n\star}\left(t\right)
\end{eqnarray*}

\begin{Prop}
Para cada $t\geq0$, la funci\'on generadora de momentos $\esp\left[e^{\alpha N\left(t\right)}\right]$ existe para alguna $\alpha$ en una vecindad del 0, y de aqu\'i que $\esp\left[N\left(t\right)^{m}\right]<\infty$, para $m\geq1$.
\end{Prop}


\begin{Note}
Si el primer tiempo de renovaci\'on $\xi_{1}$ no tiene la misma distribuci\'on que el resto de las $\xi_{n}$, para $n\geq2$, a $N\left(t\right)$ se le llama Proceso de Renovaci\'on retardado, donde si $\xi$ tiene distribuci\'on $G$, entonces el tiempo $T_{n}$ de la $n$-\'esima renovaci\'on tiene distribuci\'on $G\star F^{\left(n-1\right)\star}\left(t\right)$
\end{Note}


\begin{Teo}
Para una constante $\mu\leq\infty$ ( o variable aleatoria), las siguientes expresiones son equivalentes:

\begin{eqnarray}
lim_{n\rightarrow\infty}n^{-1}T_{n}&=&\mu,\textrm{ c.s.}\\
lim_{t\rightarrow\infty}t^{-1}N\left(t\right)&=&1/\mu,\textrm{ c.s.}
\end{eqnarray}
\end{Teo}


Es decir, $T_{n}$ satisface la Ley Fuerte de los Grandes N\'umeros s\'i y s\'olo s\'i $N\left/t\right)$ la cumple.


\begin{Coro}[Ley Fuerte de los Grandes N\'umeros para Procesos de Renovaci\'on]
Si $N\left(t\right)$ es un proceso de renovaci\'on cuyos tiempos de inter-renovaci\'on tienen media $\mu\leq\infty$, entonces
\begin{eqnarray}
t^{-1}N\left(t\right)\rightarrow 1/\mu,\textrm{ c.s. cuando }t\rightarrow\infty.
\end{eqnarray}

\end{Coro}


Considerar el proceso estoc\'astico de valores reales $\left\{Z\left(t\right):t\geq0\right\}$ en el mismo espacio de probabilidad que $N\left(t\right)$

\begin{Def}
Para el proceso $\left\{Z\left(t\right):t\geq0\right\}$ se define la fluctuaci\'on m\'axima de $Z\left(t\right)$ en el intervalo $\left(T_{n-1},T_{n}\right]$:
\begin{eqnarray*}
M_{n}=\sup_{T_{n-1}<t\leq T_{n}}|Z\left(t\right)-Z\left(T_{n-1}\right)|
\end{eqnarray*}
\end{Def}

\begin{Teo}
Sup\'ongase que $n^{-1}T_{n}\rightarrow\mu$ c.s. cuando $n\rightarrow\infty$, donde $\mu\leq\infty$ es una constante o variable aleatoria. Sea $a$ una constante o variable aleatoria que puede ser infinita cuando $\mu$ es finita, y considere las expresiones l\'imite:
\begin{eqnarray}
lim_{n\rightarrow\infty}n^{-1}Z\left(T_{n}\right)&=&a,\textrm{ c.s.}\\
lim_{t\rightarrow\infty}t^{-1}Z\left(t\right)&=&a/\mu,\textrm{ c.s.}
\end{eqnarray}
La segunda expresi\'on implica la primera. Conversamente, la primera implica la segunda si el proceso $Z\left(t\right)$ es creciente, o si $lim_{n\rightarrow\infty}n^{-1}M_{n}=0$ c.s.
\end{Teo}

\begin{Coro}
Si $N\left(t\right)$ es un proceso de renovaci\'on, y $\left(Z\left(T_{n}\right)-Z\left(T_{n-1}\right),M_{n}\right)$, para $n\geq1$, son variables aleatorias independientes e id\'enticamente distribuidas con media finita, entonces,
\begin{eqnarray}
lim_{t\rightarrow\infty}t^{-1}Z\left(t\right)\rightarrow\frac{\esp\left[Z\left(T_{1}\right)-Z\left(T_{0}\right)\right]}{\esp\left[T_{1}\right]},\textrm{ c.s. cuando  }t\rightarrow\infty.
\end{eqnarray}
\end{Coro}

%___________________________________________________________________________________________
%
\subsection{Propiedades de los Procesos de Renovaci\'on}
%___________________________________________________________________________________________
%

Los tiempos $T_{n}$ est\'an relacionados con los conteos de $N\left(t\right)$ por

\begin{eqnarray*}
\left\{N\left(t\right)\geq n\right\}&=&\left\{T_{n}\leq t\right\}\\
T_{N\left(t\right)}\leq &t&<T_{N\left(t\right)+1},
\end{eqnarray*}

adem\'as $N\left(T_{n}\right)=n$, y 

\begin{eqnarray*}
N\left(t\right)=\max\left\{n:T_{n}\leq t\right\}=\min\left\{n:T_{n+1}>t\right\}
\end{eqnarray*}

Por propiedades de la convoluci\'on se sabe que

\begin{eqnarray*}
P\left\{T_{n}\leq t\right\}=F^{n\star}\left(t\right)
\end{eqnarray*}
que es la $n$-\'esima convoluci\'on de $F$. Entonces 

\begin{eqnarray*}
\left\{N\left(t\right)\geq n\right\}&=&\left\{T_{n}\leq t\right\}\\
P\left\{N\left(t\right)\leq n\right\}&=&1-F^{\left(n+1\right)\star}\left(t\right)
\end{eqnarray*}

Adem\'as usando el hecho de que $\esp\left[N\left(t\right)\right]=\sum_{n=1}^{\infty}P\left\{N\left(t\right)\geq n\right\}$
se tiene que

\begin{eqnarray*}
\esp\left[N\left(t\right)\right]=\sum_{n=1}^{\infty}F^{n\star}\left(t\right)
\end{eqnarray*}

\begin{Prop}
Para cada $t\geq0$, la funci\'on generadora de momentos $\esp\left[e^{\alpha N\left(t\right)}\right]$ existe para alguna $\alpha$ en una vecindad del 0, y de aqu\'i que $\esp\left[N\left(t\right)^{m}\right]<\infty$, para $m\geq1$.
\end{Prop}


\begin{Note}
Si el primer tiempo de renovaci\'on $\xi_{1}$ no tiene la misma distribuci\'on que el resto de las $\xi_{n}$, para $n\geq2$, a $N\left(t\right)$ se le llama Proceso de Renovaci\'on retardado, donde si $\xi$ tiene distribuci\'on $G$, entonces el tiempo $T_{n}$ de la $n$-\'esima renovaci\'on tiene distribuci\'on $G\star F^{\left(n-1\right)\star}\left(t\right)$
\end{Note}


\begin{Teo}
Para una constante $\mu\leq\infty$ ( o variable aleatoria), las siguientes expresiones son equivalentes:

\begin{eqnarray}
lim_{n\rightarrow\infty}n^{-1}T_{n}&=&\mu,\textrm{ c.s.}\\
lim_{t\rightarrow\infty}t^{-1}N\left(t\right)&=&1/\mu,\textrm{ c.s.}
\end{eqnarray}
\end{Teo}


Es decir, $T_{n}$ satisface la Ley Fuerte de los Grandes N\'umeros s\'i y s\'olo s\'i $N\left/t\right)$ la cumple.


\begin{Coro}[Ley Fuerte de los Grandes N\'umeros para Procesos de Renovaci\'on]
Si $N\left(t\right)$ es un proceso de renovaci\'on cuyos tiempos de inter-renovaci\'on tienen media $\mu\leq\infty$, entonces
\begin{eqnarray}
t^{-1}N\left(t\right)\rightarrow 1/\mu,\textrm{ c.s. cuando }t\rightarrow\infty.
\end{eqnarray}

\end{Coro}


Considerar el proceso estoc\'astico de valores reales $\left\{Z\left(t\right):t\geq0\right\}$ en el mismo espacio de probabilidad que $N\left(t\right)$

\begin{Def}
Para el proceso $\left\{Z\left(t\right):t\geq0\right\}$ se define la fluctuaci\'on m\'axima de $Z\left(t\right)$ en el intervalo $\left(T_{n-1},T_{n}\right]$:
\begin{eqnarray*}
M_{n}=\sup_{T_{n-1}<t\leq T_{n}}|Z\left(t\right)-Z\left(T_{n-1}\right)|
\end{eqnarray*}
\end{Def}

\begin{Teo}
Sup\'ongase que $n^{-1}T_{n}\rightarrow\mu$ c.s. cuando $n\rightarrow\infty$, donde $\mu\leq\infty$ es una constante o variable aleatoria. Sea $a$ una constante o variable aleatoria que puede ser infinita cuando $\mu$ es finita, y considere las expresiones l\'imite:
\begin{eqnarray}
lim_{n\rightarrow\infty}n^{-1}Z\left(T_{n}\right)&=&a,\textrm{ c.s.}\\
lim_{t\rightarrow\infty}t^{-1}Z\left(t\right)&=&a/\mu,\textrm{ c.s.}
\end{eqnarray}
La segunda expresi\'on implica la primera. Conversamente, la primera implica la segunda si el proceso $Z\left(t\right)$ es creciente, o si $lim_{n\rightarrow\infty}n^{-1}M_{n}=0$ c.s.
\end{Teo}

\begin{Coro}
Si $N\left(t\right)$ es un proceso de renovaci\'on, y $\left(Z\left(T_{n}\right)-Z\left(T_{n-1}\right),M_{n}\right)$, para $n\geq1$, son variables aleatorias independientes e id\'enticamente distribuidas con media finita, entonces,
\begin{eqnarray}
lim_{t\rightarrow\infty}t^{-1}Z\left(t\right)\rightarrow\frac{\esp\left[Z\left(T_{1}\right)-Z\left(T_{0}\right)\right]}{\esp\left[T_{1}\right]},\textrm{ c.s. cuando  }t\rightarrow\infty.
\end{eqnarray}
\end{Coro}

%___________________________________________________________________________________________
%
\subsection{Propiedades de los Procesos de Renovaci\'on}
%___________________________________________________________________________________________
%

Los tiempos $T_{n}$ est\'an relacionados con los conteos de $N\left(t\right)$ por

\begin{eqnarray*}
\left\{N\left(t\right)\geq n\right\}&=&\left\{T_{n}\leq t\right\}\\
T_{N\left(t\right)}\leq &t&<T_{N\left(t\right)+1},
\end{eqnarray*}

adem\'as $N\left(T_{n}\right)=n$, y 

\begin{eqnarray*}
N\left(t\right)=\max\left\{n:T_{n}\leq t\right\}=\min\left\{n:T_{n+1}>t\right\}
\end{eqnarray*}

Por propiedades de la convoluci\'on se sabe que

\begin{eqnarray*}
P\left\{T_{n}\leq t\right\}=F^{n\star}\left(t\right)
\end{eqnarray*}
que es la $n$-\'esima convoluci\'on de $F$. Entonces 

\begin{eqnarray*}
\left\{N\left(t\right)\geq n\right\}&=&\left\{T_{n}\leq t\right\}\\
P\left\{N\left(t\right)\leq n\right\}&=&1-F^{\left(n+1\right)\star}\left(t\right)
\end{eqnarray*}

Adem\'as usando el hecho de que $\esp\left[N\left(t\right)\right]=\sum_{n=1}^{\infty}P\left\{N\left(t\right)\geq n\right\}$
se tiene que

\begin{eqnarray*}
\esp\left[N\left(t\right)\right]=\sum_{n=1}^{\infty}F^{n\star}\left(t\right)
\end{eqnarray*}

\begin{Prop}
Para cada $t\geq0$, la funci\'on generadora de momentos $\esp\left[e^{\alpha N\left(t\right)}\right]$ existe para alguna $\alpha$ en una vecindad del 0, y de aqu\'i que $\esp\left[N\left(t\right)^{m}\right]<\infty$, para $m\geq1$.
\end{Prop}


\begin{Note}
Si el primer tiempo de renovaci\'on $\xi_{1}$ no tiene la misma distribuci\'on que el resto de las $\xi_{n}$, para $n\geq2$, a $N\left(t\right)$ se le llama Proceso de Renovaci\'on retardado, donde si $\xi$ tiene distribuci\'on $G$, entonces el tiempo $T_{n}$ de la $n$-\'esima renovaci\'on tiene distribuci\'on $G\star F^{\left(n-1\right)\star}\left(t\right)$
\end{Note}


\begin{Teo}
Para una constante $\mu\leq\infty$ ( o variable aleatoria), las siguientes expresiones son equivalentes:

\begin{eqnarray}
lim_{n\rightarrow\infty}n^{-1}T_{n}&=&\mu,\textrm{ c.s.}\\
lim_{t\rightarrow\infty}t^{-1}N\left(t\right)&=&1/\mu,\textrm{ c.s.}
\end{eqnarray}
\end{Teo}


Es decir, $T_{n}$ satisface la Ley Fuerte de los Grandes N\'umeros s\'i y s\'olo s\'i $N\left/t\right)$ la cumple.


\begin{Coro}[Ley Fuerte de los Grandes N\'umeros para Procesos de Renovaci\'on]
Si $N\left(t\right)$ es un proceso de renovaci\'on cuyos tiempos de inter-renovaci\'on tienen media $\mu\leq\infty$, entonces
\begin{eqnarray}
t^{-1}N\left(t\right)\rightarrow 1/\mu,\textrm{ c.s. cuando }t\rightarrow\infty.
\end{eqnarray}

\end{Coro}


Considerar el proceso estoc\'astico de valores reales $\left\{Z\left(t\right):t\geq0\right\}$ en el mismo espacio de probabilidad que $N\left(t\right)$

\begin{Def}
Para el proceso $\left\{Z\left(t\right):t\geq0\right\}$ se define la fluctuaci\'on m\'axima de $Z\left(t\right)$ en el intervalo $\left(T_{n-1},T_{n}\right]$:
\begin{eqnarray*}
M_{n}=\sup_{T_{n-1}<t\leq T_{n}}|Z\left(t\right)-Z\left(T_{n-1}\right)|
\end{eqnarray*}
\end{Def}

\begin{Teo}
Sup\'ongase que $n^{-1}T_{n}\rightarrow\mu$ c.s. cuando $n\rightarrow\infty$, donde $\mu\leq\infty$ es una constante o variable aleatoria. Sea $a$ una constante o variable aleatoria que puede ser infinita cuando $\mu$ es finita, y considere las expresiones l\'imite:
\begin{eqnarray}
lim_{n\rightarrow\infty}n^{-1}Z\left(T_{n}\right)&=&a,\textrm{ c.s.}\\
lim_{t\rightarrow\infty}t^{-1}Z\left(t\right)&=&a/\mu,\textrm{ c.s.}
\end{eqnarray}
La segunda expresi\'on implica la primera. Conversamente, la primera implica la segunda si el proceso $Z\left(t\right)$ es creciente, o si $lim_{n\rightarrow\infty}n^{-1}M_{n}=0$ c.s.
\end{Teo}

\begin{Coro}
Si $N\left(t\right)$ es un proceso de renovaci\'on, y $\left(Z\left(T_{n}\right)-Z\left(T_{n-1}\right),M_{n}\right)$, para $n\geq1$, son variables aleatorias independientes e id\'enticamente distribuidas con media finita, entonces,
\begin{eqnarray}
lim_{t\rightarrow\infty}t^{-1}Z\left(t\right)\rightarrow\frac{\esp\left[Z\left(T_{1}\right)-Z\left(T_{0}\right)\right]}{\esp\left[T_{1}\right]},\textrm{ c.s. cuando  }t\rightarrow\infty.
\end{eqnarray}
\end{Coro}


%__________________________________________________________________________________________
\subsection{Procesos Regenerativos Estacionarios - Stidham \cite{Stidham}}
%__________________________________________________________________________________________


Un proceso estoc\'astico a tiempo continuo $\left\{V\left(t\right),t\geq0\right\}$ es un proceso regenerativo si existe una sucesi\'on de variables aleatorias independientes e id\'enticamente distribuidas $\left\{X_{1},X_{2},\ldots\right\}$, sucesi\'on de renovaci\'on, tal que para cualquier conjunto de Borel $A$, 

\begin{eqnarray*}
\prob\left\{V\left(t\right)\in A|X_{1}+X_{2}+\cdots+X_{R\left(t\right)}=s,\left\{V\left(\tau\right),\tau<s\right\}\right\}=\prob\left\{V\left(t-s\right)\in A|X_{1}>t-s\right\},
\end{eqnarray*}
para todo $0\leq s\leq t$, donde $R\left(t\right)=\max\left\{X_{1}+X_{2}+\cdots+X_{j}\leq t\right\}=$n\'umero de renovaciones ({\emph{puntos de regeneraci\'on}}) que ocurren en $\left[0,t\right]$. El intervalo $\left[0,X_{1}\right)$ es llamado {\emph{primer ciclo de regeneraci\'on}} de $\left\{V\left(t \right),t\geq0\right\}$, $\left[X_{1},X_{1}+X_{2}\right)$ el {\emph{segundo ciclo de regeneraci\'on}}, y as\'i sucesivamente.

Sea $X=X_{1}$ y sea $F$ la funci\'on de distrbuci\'on de $X$


\begin{Def}
Se define el proceso estacionario, $\left\{V^{*}\left(t\right),t\geq0\right\}$, para $\left\{V\left(t\right),t\geq0\right\}$ por

\begin{eqnarray*}
\prob\left\{V\left(t\right)\in A\right\}=\frac{1}{\esp\left[X\right]}\int_{0}^{\infty}\prob\left\{V\left(t+x\right)\in A|X>x\right\}\left(1-F\left(x\right)\right)dx,
\end{eqnarray*} 
para todo $t\geq0$ y todo conjunto de Borel $A$.
\end{Def}

\begin{Def}
Una distribuci\'on se dice que es {\emph{aritm\'etica}} si todos sus puntos de incremento son m\'ultiplos de la forma $0,\lambda, 2\lambda,\ldots$ para alguna $\lambda>0$ entera.
\end{Def}


\begin{Def}
Una modificaci\'on medible de un proceso $\left\{V\left(t\right),t\geq0\right\}$, es una versi\'on de este, $\left\{V\left(t,w\right)\right\}$ conjuntamente medible para $t\geq0$ y para $w\in S$, $S$ espacio de estados para $\left\{V\left(t\right),t\geq0\right\}$.
\end{Def}

\begin{Teo}
Sea $\left\{V\left(t\right),t\geq\right\}$ un proceso regenerativo no negativo con modificaci\'on medible. Sea $\esp\left[X\right]<\infty$. Entonces el proceso estacionario dado por la ecuaci\'on anterior est\'a bien definido y tiene funci\'on de distribuci\'on independiente de $t$, adem\'as
\begin{itemize}
\item[i)] \begin{eqnarray*}
\esp\left[V^{*}\left(0\right)\right]&=&\frac{\esp\left[\int_{0}^{X}V\left(s\right)ds\right]}{\esp\left[X\right]}\end{eqnarray*}
\item[ii)] Si $\esp\left[V^{*}\left(0\right)\right]<\infty$, equivalentemente, si $\esp\left[\int_{0}^{X}V\left(s\right)ds\right]<\infty$,entonces
\begin{eqnarray*}
\frac{\int_{0}^{t}V\left(s\right)ds}{t}\rightarrow\frac{\esp\left[\int_{0}^{X}V\left(s\right)ds\right]}{\esp\left[X\right]}
\end{eqnarray*}
con probabilidad 1 y en media, cuando $t\rightarrow\infty$.
\end{itemize}
\end{Teo}

%______________________________________________________________________
\subsection{Procesos de Renovaci\'on}
%______________________________________________________________________

\begin{Def}\label{Def.Tn}
Sean $0\leq T_{1}\leq T_{2}\leq \ldots$ son tiempos aleatorios infinitos en los cuales ocurren ciertos eventos. El n\'umero de tiempos $T_{n}$ en el intervalo $\left[0,t\right)$ es

\begin{eqnarray}
N\left(t\right)=\sum_{n=1}^{\infty}\indora\left(T_{n}\leq t\right),
\end{eqnarray}
para $t\geq0$.
\end{Def}

Si se consideran los puntos $T_{n}$ como elementos de $\rea_{+}$, y $N\left(t\right)$ es el n\'umero de puntos en $\rea$. El proceso denotado por $\left\{N\left(t\right):t\geq0\right\}$, denotado por $N\left(t\right)$, es un proceso puntual en $\rea_{+}$. Los $T_{n}$ son los tiempos de ocurrencia, el proceso puntual $N\left(t\right)$ es simple si su n\'umero de ocurrencias son distintas: $0<T_{1}<T_{2}<\ldots$ casi seguramente.

\begin{Def}
Un proceso puntual $N\left(t\right)$ es un proceso de renovaci\'on si los tiempos de interocurrencia $\xi_{n}=T_{n}-T_{n-1}$, para $n\geq1$, son independientes e identicamente distribuidos con distribuci\'on $F$, donde $F\left(0\right)=0$ y $T_{0}=0$. Los $T_{n}$ son llamados tiempos de renovaci\'on, referente a la independencia o renovaci\'on de la informaci\'on estoc\'astica en estos tiempos. Los $\xi_{n}$ son los tiempos de inter-renovaci\'on, y $N\left(t\right)$ es el n\'umero de renovaciones en el intervalo $\left[0,t\right)$
\end{Def}


\begin{Note}
Para definir un proceso de renovaci\'on para cualquier contexto, solamente hay que especificar una distribuci\'on $F$, con $F\left(0\right)=0$, para los tiempos de inter-renovaci\'on. La funci\'on $F$ en turno degune las otra variables aleatorias. De manera formal, existe un espacio de probabilidad y una sucesi\'on de variables aleatorias $\xi_{1},\xi_{2},\ldots$ definidas en este con distribuci\'on $F$. Entonces las otras cantidades son $T_{n}=\sum_{k=1}^{n}\xi_{k}$ y $N\left(t\right)=\sum_{n=1}^{\infty}\indora\left(T_{n}\leq t\right)$, donde $T_{n}\rightarrow\infty$ casi seguramente por la Ley Fuerte de los Grandes Números.
\end{Note}

%___________________________________________________________________________________________
%
\subsection{Teorema Principal de Renovaci\'on}
%___________________________________________________________________________________________
%

\begin{Note} Una funci\'on $h:\rea_{+}\rightarrow\rea$ es Directamente Riemann Integrable en los siguientes casos:
\begin{itemize}
\item[a)] $h\left(t\right)\geq0$ es decreciente y Riemann Integrable.
\item[b)] $h$ es continua excepto posiblemente en un conjunto de Lebesgue de medida 0, y $|h\left(t\right)|\leq b\left(t\right)$, donde $b$ es DRI.
\end{itemize}
\end{Note}

\begin{Teo}[Teorema Principal de Renovaci\'on]
Si $F$ es no aritm\'etica y $h\left(t\right)$ es Directamente Riemann Integrable (DRI), entonces

\begin{eqnarray*}
lim_{t\rightarrow\infty}U\star h=\frac{1}{\mu}\int_{\rea_{+}}h\left(s\right)ds.
\end{eqnarray*}
\end{Teo}

\begin{Prop}
Cualquier funci\'on $H\left(t\right)$ acotada en intervalos finitos y que es 0 para $t<0$ puede expresarse como
\begin{eqnarray*}
H\left(t\right)=U\star h\left(t\right)\textrm{,  donde }h\left(t\right)=H\left(t\right)-F\star H\left(t\right)
\end{eqnarray*}
\end{Prop}

\begin{Def}
Un proceso estoc\'astico $X\left(t\right)$ es crudamente regenerativo en un tiempo aleatorio positivo $T$ si
\begin{eqnarray*}
\esp\left[X\left(T+t\right)|T\right]=\esp\left[X\left(t\right)\right]\textrm{, para }t\geq0,\end{eqnarray*}
y con las esperanzas anteriores finitas.
\end{Def}

\begin{Prop}
Sup\'ongase que $X\left(t\right)$ es un proceso crudamente regenerativo en $T$, que tiene distribuci\'on $F$. Si $\esp\left[X\left(t\right)\right]$ es acotado en intervalos finitos, entonces
\begin{eqnarray*}
\esp\left[X\left(t\right)\right]=U\star h\left(t\right)\textrm{,  donde }h\left(t\right)=\esp\left[X\left(t\right)\indora\left(T>t\right)\right].
\end{eqnarray*}
\end{Prop}

\begin{Teo}[Regeneraci\'on Cruda]
Sup\'ongase que $X\left(t\right)$ es un proceso con valores positivo crudamente regenerativo en $T$, y def\'inase $M=\sup\left\{|X\left(t\right)|:t\leq T\right\}$. Si $T$ es no aritm\'etico y $M$ y $MT$ tienen media finita, entonces
\begin{eqnarray*}
lim_{t\rightarrow\infty}\esp\left[X\left(t\right)\right]=\frac{1}{\mu}\int_{\rea_{+}}h\left(s\right)ds,
\end{eqnarray*}
donde $h\left(t\right)=\esp\left[X\left(t\right)\indora\left(T>t\right)\right]$.
\end{Teo}



%___________________________________________________________________________________________
%
\subsection{Funci\'on de Renovaci\'on}
%___________________________________________________________________________________________
%


\begin{Def}
Sea $h\left(t\right)$ funci\'on de valores reales en $\rea$ acotada en intervalos finitos e igual a cero para $t<0$ La ecuaci\'on de renovaci\'on para $h\left(t\right)$ y la distribuci\'on $F$ es

\begin{eqnarray}\label{Ec.Renovacion}
H\left(t\right)=h\left(t\right)+\int_{\left[0,t\right]}H\left(t-s\right)dF\left(s\right)\textrm{,    }t\geq0,
\end{eqnarray}
donde $H\left(t\right)$ es una funci\'on de valores reales. Esto es $H=h+F\star H$. Decimos que $H\left(t\right)$ es soluci\'on de esta ecuaci\'on si satisface la ecuaci\'on, y es acotada en intervalos finitos e iguales a cero para $t<0$.
\end{Def}

\begin{Prop}
La funci\'on $U\star h\left(t\right)$ es la \'unica soluci\'on de la ecuaci\'on de renovaci\'on (\ref{Ec.Renovacion}).
\end{Prop}

\begin{Teo}[Teorema Renovaci\'on Elemental]
\begin{eqnarray*}
t^{-1}U\left(t\right)\rightarrow 1/\mu\textrm{,    cuando }t\rightarrow\infty.
\end{eqnarray*}
\end{Teo}

%___________________________________________________________________________________________
%
\subsection{Propiedades de los Procesos de Renovaci\'on}
%___________________________________________________________________________________________
%

Los tiempos $T_{n}$ est\'an relacionados con los conteos de $N\left(t\right)$ por

\begin{eqnarray*}
\left\{N\left(t\right)\geq n\right\}&=&\left\{T_{n}\leq t\right\}\\
T_{N\left(t\right)}\leq &t&<T_{N\left(t\right)+1},
\end{eqnarray*}

adem\'as $N\left(T_{n}\right)=n$, y 

\begin{eqnarray*}
N\left(t\right)=\max\left\{n:T_{n}\leq t\right\}=\min\left\{n:T_{n+1}>t\right\}
\end{eqnarray*}

Por propiedades de la convoluci\'on se sabe que

\begin{eqnarray*}
P\left\{T_{n}\leq t\right\}=F^{n\star}\left(t\right)
\end{eqnarray*}
que es la $n$-\'esima convoluci\'on de $F$. Entonces 

\begin{eqnarray*}
\left\{N\left(t\right)\geq n\right\}&=&\left\{T_{n}\leq t\right\}\\
P\left\{N\left(t\right)\leq n\right\}&=&1-F^{\left(n+1\right)\star}\left(t\right)
\end{eqnarray*}

Adem\'as usando el hecho de que $\esp\left[N\left(t\right)\right]=\sum_{n=1}^{\infty}P\left\{N\left(t\right)\geq n\right\}$
se tiene que

\begin{eqnarray*}
\esp\left[N\left(t\right)\right]=\sum_{n=1}^{\infty}F^{n\star}\left(t\right)
\end{eqnarray*}

\begin{Prop}
Para cada $t\geq0$, la funci\'on generadora de momentos $\esp\left[e^{\alpha N\left(t\right)}\right]$ existe para alguna $\alpha$ en una vecindad del 0, y de aqu\'i que $\esp\left[N\left(t\right)^{m}\right]<\infty$, para $m\geq1$.
\end{Prop}


\begin{Note}
Si el primer tiempo de renovaci\'on $\xi_{1}$ no tiene la misma distribuci\'on que el resto de las $\xi_{n}$, para $n\geq2$, a $N\left(t\right)$ se le llama Proceso de Renovaci\'on retardado, donde si $\xi$ tiene distribuci\'on $G$, entonces el tiempo $T_{n}$ de la $n$-\'esima renovaci\'on tiene distribuci\'on $G\star F^{\left(n-1\right)\star}\left(t\right)$
\end{Note}


\begin{Teo}
Para una constante $\mu\leq\infty$ ( o variable aleatoria), las siguientes expresiones son equivalentes:

\begin{eqnarray}
lim_{n\rightarrow\infty}n^{-1}T_{n}&=&\mu,\textrm{ c.s.}\\
lim_{t\rightarrow\infty}t^{-1}N\left(t\right)&=&1/\mu,\textrm{ c.s.}
\end{eqnarray}
\end{Teo}


Es decir, $T_{n}$ satisface la Ley Fuerte de los Grandes N\'umeros s\'i y s\'olo s\'i $N\left/t\right)$ la cumple.


\begin{Coro}[Ley Fuerte de los Grandes N\'umeros para Procesos de Renovaci\'on]
Si $N\left(t\right)$ es un proceso de renovaci\'on cuyos tiempos de inter-renovaci\'on tienen media $\mu\leq\infty$, entonces
\begin{eqnarray}
t^{-1}N\left(t\right)\rightarrow 1/\mu,\textrm{ c.s. cuando }t\rightarrow\infty.
\end{eqnarray}

\end{Coro}


Considerar el proceso estoc\'astico de valores reales $\left\{Z\left(t\right):t\geq0\right\}$ en el mismo espacio de probabilidad que $N\left(t\right)$

\begin{Def}
Para el proceso $\left\{Z\left(t\right):t\geq0\right\}$ se define la fluctuaci\'on m\'axima de $Z\left(t\right)$ en el intervalo $\left(T_{n-1},T_{n}\right]$:
\begin{eqnarray*}
M_{n}=\sup_{T_{n-1}<t\leq T_{n}}|Z\left(t\right)-Z\left(T_{n-1}\right)|
\end{eqnarray*}
\end{Def}

\begin{Teo}
Sup\'ongase que $n^{-1}T_{n}\rightarrow\mu$ c.s. cuando $n\rightarrow\infty$, donde $\mu\leq\infty$ es una constante o variable aleatoria. Sea $a$ una constante o variable aleatoria que puede ser infinita cuando $\mu$ es finita, y considere las expresiones l\'imite:
\begin{eqnarray}
lim_{n\rightarrow\infty}n^{-1}Z\left(T_{n}\right)&=&a,\textrm{ c.s.}\\
lim_{t\rightarrow\infty}t^{-1}Z\left(t\right)&=&a/\mu,\textrm{ c.s.}
\end{eqnarray}
La segunda expresi\'on implica la primera. Conversamente, la primera implica la segunda si el proceso $Z\left(t\right)$ es creciente, o si $lim_{n\rightarrow\infty}n^{-1}M_{n}=0$ c.s.
\end{Teo}

\begin{Coro}
Si $N\left(t\right)$ es un proceso de renovaci\'on, y $\left(Z\left(T_{n}\right)-Z\left(T_{n-1}\right),M_{n}\right)$, para $n\geq1$, son variables aleatorias independientes e id\'enticamente distribuidas con media finita, entonces,
\begin{eqnarray}
lim_{t\rightarrow\infty}t^{-1}Z\left(t\right)\rightarrow\frac{\esp\left[Z\left(T_{1}\right)-Z\left(T_{0}\right)\right]}{\esp\left[T_{1}\right]},\textrm{ c.s. cuando  }t\rightarrow\infty.
\end{eqnarray}
\end{Coro}

%___________________________________________________________________________________________
%
\subsection{Funci\'on de Renovaci\'on}
%___________________________________________________________________________________________
%


Sup\'ongase que $N\left(t\right)$ es un proceso de renovaci\'on con distribuci\'on $F$ con media finita $\mu$.

\begin{Def}
La funci\'on de renovaci\'on asociada con la distribuci\'on $F$, del proceso $N\left(t\right)$, es
\begin{eqnarray*}
U\left(t\right)=\sum_{n=1}^{\infty}F^{n\star}\left(t\right),\textrm{   }t\geq0,
\end{eqnarray*}
donde $F^{0\star}\left(t\right)=\indora\left(t\geq0\right)$.
\end{Def}


\begin{Prop}
Sup\'ongase que la distribuci\'on de inter-renovaci\'on $F$ tiene densidad $f$. Entonces $U\left(t\right)$ tambi\'en tiene densidad, para $t>0$, y es $U^{'}\left(t\right)=\sum_{n=0}^{\infty}f^{n\star}\left(t\right)$. Adem\'as
\begin{eqnarray*}
\prob\left\{N\left(t\right)>N\left(t-\right)\right\}=0\textrm{,   }t\geq0.
\end{eqnarray*}
\end{Prop}

\begin{Def}
La Transformada de Laplace-Stieljes de $F$ est\'a dada por

\begin{eqnarray*}
\hat{F}\left(\alpha\right)=\int_{\rea_{+}}e^{-\alpha t}dF\left(t\right)\textrm{,  }\alpha\geq0.
\end{eqnarray*}
\end{Def}

Entonces

\begin{eqnarray*}
\hat{U}\left(\alpha\right)=\sum_{n=0}^{\infty}\hat{F^{n\star}}\left(\alpha\right)=\sum_{n=0}^{\infty}\hat{F}\left(\alpha\right)^{n}=\frac{1}{1-\hat{F}\left(\alpha\right)}.
\end{eqnarray*}


\begin{Prop}
La Transformada de Laplace $\hat{U}\left(\alpha\right)$ y $\hat{F}\left(\alpha\right)$ determina una a la otra de manera \'unica por la relaci\'on $\hat{U}\left(\alpha\right)=\frac{1}{1-\hat{F}\left(\alpha\right)}$.
\end{Prop}


\begin{Note}
Un proceso de renovaci\'on $N\left(t\right)$ cuyos tiempos de inter-renovaci\'on tienen media finita, es un proceso Poisson con tasa $\lambda$ si y s\'olo s\'i $\esp\left[U\left(t\right)\right]=\lambda t$, para $t\geq0$.
\end{Note}


\begin{Teo}
Sea $N\left(t\right)$ un proceso puntual simple con puntos de localizaci\'on $T_{n}$ tal que $\eta\left(t\right)=\esp\left[N\left(\right)\right]$ es finita para cada $t$. Entonces para cualquier funci\'on $f:\rea_{+}\rightarrow\rea$,
\begin{eqnarray*}
\esp\left[\sum_{n=1}^{N\left(\right)}f\left(T_{n}\right)\right]=\int_{\left(0,t\right]}f\left(s\right)d\eta\left(s\right)\textrm{,  }t\geq0,
\end{eqnarray*}
suponiendo que la integral exista. Adem\'as si $X_{1},X_{2},\ldots$ son variables aleatorias definidas en el mismo espacio de probabilidad que el proceso $N\left(t\right)$ tal que $\esp\left[X_{n}|T_{n}=s\right]=f\left(s\right)$, independiente de $n$. Entonces
\begin{eqnarray*}
\esp\left[\sum_{n=1}^{N\left(t\right)}X_{n}\right]=\int_{\left(0,t\right]}f\left(s\right)d\eta\left(s\right)\textrm{,  }t\geq0,
\end{eqnarray*} 
suponiendo que la integral exista. 
\end{Teo}

\begin{Coro}[Identidad de Wald para Renovaciones]
Para el proceso de renovaci\'on $N\left(t\right)$,
\begin{eqnarray*}
\esp\left[T_{N\left(t\right)+1}\right]=\mu\esp\left[N\left(t\right)+1\right]\textrm{,  }t\geq0,
\end{eqnarray*}  
\end{Coro}

%_____________________________________________________
\subsection{Puntos de Renovaci\'on}
%_____________________________________________________

Para cada cola $Q_{i}$ se tienen los procesos de arribo a la cola, para estas, los tiempos de arribo est\'an dados por $$\left\{T_{1}^{i},T_{2}^{i},\ldots,T_{k}^{i},\ldots\right\},$$ entonces, consideremos solamente los primeros tiempos de arribo a cada una de las colas, es decir, $$\left\{T_{1}^{1},T_{1}^{2},T_{1}^{3},T_{1}^{4}\right\},$$ se sabe que cada uno de estos tiempos se distribuye de manera exponencial con par\'ametro $1/mu_{i}$. Adem\'as se sabe que para $$T^{*}=\min\left\{T_{1}^{1},T_{1}^{2},T_{1}^{3},T_{1}^{4}\right\},$$ $T^{*}$ se distribuye de manera exponencial con par\'ametro $$\mu^{*}=\sum_{i=1}^{4}\mu_{i}.$$ Ahora, dado que 
\begin{center}
\begin{tabular}{lcl}
$\tilde{r}=r_{1}+r_{2}$ & y &$\hat{r}=r_{3}+r_{4}.$
\end{tabular}
\end{center}


Supongamos que $$\tilde{r},\hat{r}<\mu^{*},$$ entonces si tomamos $$r^{*}=\min\left\{\tilde{r},\hat{r}\right\},$$ se tiene que para  $$t^{*}\in\left(0,r^{*}\right)$$ se cumple que 
\begin{center}
\begin{tabular}{lcl}
$\tau_{1}\left(1\right)=0$ & y por tanto & $\overline{\tau}_{1}=0,$
\end{tabular}
\end{center}
entonces para la segunda cola en este primer ciclo se cumple que $$\tau_{2}=\overline{\tau}_{1}+r_{1}=r_{1}<\mu^{*},$$ y por tanto se tiene que  $$\overline{\tau}_{2}=\tau_{2}.$$ Por lo tanto, nuevamente para la primer cola en el segundo ciclo $$\tau_{1}\left(2\right)=\tau_{2}\left(1\right)+r_{2}=\tilde{r}<\mu^{*}.$$ An\'alogamente para el segundo sistema se tiene que ambas colas est\'an vac\'ias, es decir, existe un valor $t^{*}$ tal que en el intervalo $\left(0,t^{*}\right)$ no ha llegado ning\'un usuario, es decir, $$L_{i}\left(t^{*}\right)=0$$ para $i=1,2,3,4$.

\subsection{Resultados para Procesos de Salida}

En \cite{Sigman2} prueban que para la existencia de un una sucesi\'on infinita no decreciente de tiempos de regeneraci\'on $\tau_{1}\leq\tau_{2}\leq\cdots$ en los cuales el proceso se regenera, basta un tiempo de regeneraci\'on $R_{1}$, donde $R_{j}=\tau_{j}-\tau_{j-1}$. Para tal efecto se requiere la existencia de un espacio de probabilidad $\left(\Omega,\mathcal{F},\prob\right)$, y proceso estoc\'astico $\textit{X}=\left\{X\left(t\right):t\geq0\right\}$ con espacio de estados $\left(S,\mathcal{R}\right)$, con $\mathcal{R}$ $\sigma$-\'algebra.

\begin{Prop}
Si existe una variable aleatoria no negativa $R_{1}$ tal que $\theta_{R\footnotesize{1}}X=_{D}X$, entonces $\left(\Omega,\mathcal{F},\prob\right)$ puede extenderse para soportar una sucesi\'on estacionaria de variables aleatorias $R=\left\{R_{k}:k\geq1\right\}$, tal que para $k\geq1$,
\begin{eqnarray*}
\theta_{k}\left(X,R\right)=_{D}\left(X,R\right).
\end{eqnarray*}

Adem\'as, para $k\geq1$, $\theta_{k}R$ es condicionalmente independiente de $\left(X,R_{1},\ldots,R_{k}\right)$, dado $\theta_{\tau k}X$.

\end{Prop}


\begin{itemize}
\item Doob en 1953 demostr\'o que el estado estacionario de un proceso de partida en un sistema de espera $M/G/\infty$, es Poisson con la misma tasa que el proceso de arribos.

\item Burke en 1968, fue el primero en demostrar que el estado estacionario de un proceso de salida de una cola $M/M/s$ es un proceso Poisson.

\item Disney en 1973 obtuvo el siguiente resultado:

\begin{Teo}
Para el sistema de espera $M/G/1/L$ con disciplina FIFO, el proceso $\textbf{I}$ es un proceso de renovaci\'on si y s\'olo si el proceso denominado longitud de la cola es estacionario y se cumple cualquiera de los siguientes casos:

\begin{itemize}
\item[a)] Los tiempos de servicio son identicamente cero;
\item[b)] $L=0$, para cualquier proceso de servicio $S$;
\item[c)] $L=1$ y $G=D$;
\item[d)] $L=\infty$ y $G=M$.
\end{itemize}
En estos casos, respectivamente, las distribuciones de interpartida $P\left\{T_{n+1}-T_{n}\leq t\right\}$ son


\begin{itemize}
\item[a)] $1-e^{-\lambda t}$, $t\geq0$;
\item[b)] $1-e^{-\lambda t}*F\left(t\right)$, $t\geq0$;
\item[c)] $1-e^{-\lambda t}*\indora_{d}\left(t\right)$, $t\geq0$;
\item[d)] $1-e^{-\lambda t}*F\left(t\right)$, $t\geq0$.
\end{itemize}
\end{Teo}


\item Finch (1959) mostr\'o que para los sistemas $M/G/1/L$, con $1\leq L\leq \infty$ con distribuciones de servicio dos veces diferenciable, solamente el sistema $M/M/1/\infty$ tiene proceso de salida de renovaci\'on estacionario.

\item King (1971) demostro que un sistema de colas estacionario $M/G/1/1$ tiene sus tiempos de interpartida sucesivas $D_{n}$ y $D_{n+1}$ son independientes, si y s\'olo si, $G=D$, en cuyo caso le proceso de salida es de renovaci\'on.

\item Disney (1973) demostr\'o que el \'unico sistema estacionario $M/G/1/L$, que tiene proceso de salida de renovaci\'on  son los sistemas $M/M/1$ y $M/D/1/1$.



\item El siguiente resultado es de Disney y Koning (1985)
\begin{Teo}
En un sistema de espera $M/G/s$, el estado estacionario del proceso de salida es un proceso Poisson para cualquier distribuci\'on de los tiempos de servicio si el sistema tiene cualquiera de las siguientes cuatro propiedades.

\begin{itemize}
\item[a)] $s=\infty$
\item[b)] La disciplina de servicio es de procesador compartido.
\item[c)] La disciplina de servicio es LCFS y preemptive resume, esto se cumple para $L<\infty$
\item[d)] $G=M$.
\end{itemize}

\end{Teo}

\item El siguiente resultado es de Alamatsaz (1983)

\begin{Teo}
En cualquier sistema de colas $GI/G/1/L$ con $1\leq L<\infty$ y distribuci\'on de interarribos $A$ y distribuci\'on de los tiempos de servicio $B$, tal que $A\left(0\right)=0$, $A\left(t\right)\left(1-B\left(t\right)\right)>0$ para alguna $t>0$ y $B\left(t\right)$ para toda $t>0$, es imposible que el proceso de salida estacionario sea de renovaci\'on.
\end{Teo}

\end{itemize}

Estos resultados aparecen en Daley (1968) \cite{Daley68} para $\left\{T_{n}\right\}$ intervalos de inter-arribo, $\left\{D_{n}\right\}$ intervalos de inter-salida y $\left\{S_{n}\right\}$ tiempos de servicio.

\begin{itemize}
\item Si el proceso $\left\{T_{n}\right\}$ es Poisson, el proceso $\left\{D_{n}\right\}$ es no correlacionado si y s\'olo si es un proceso Poisso, lo cual ocurre si y s\'olo si $\left\{S_{n}\right\}$ son exponenciales negativas.

\item Si $\left\{S_{n}\right\}$ son exponenciales negativas, $\left\{D_{n}\right\}$ es un proceso de renovaci\'on  si y s\'olo si es un proceso Poisson, lo cual ocurre si y s\'olo si $\left\{T_{n}\right\}$ es un proceso Poisson.

\item $\esp\left(D_{n}\right)=\esp\left(T_{n}\right)$.

\item Para un sistema de visitas $GI/M/1$ se tiene el siguiente teorema:

\begin{Teo}
En un sistema estacionario $GI/M/1$ los intervalos de interpartida tienen
\begin{eqnarray*}
\esp\left(e^{-\theta D_{n}}\right)&=&\mu\left(\mu+\theta\right)^{-1}\left[\delta\theta
-\mu\left(1-\delta\right)\alpha\left(\theta\right)\right]
\left[\theta-\mu\left(1-\delta\right)^{-1}\right]\\
\alpha\left(\theta\right)&=&\esp\left[e^{-\theta T_{0}}\right]\\
var\left(D_{n}\right)&=&var\left(T_{0}\right)-\left(\tau^{-1}-\delta^{-1}\right)
2\delta\left(\esp\left(S_{0}\right)\right)^{2}\left(1-\delta\right)^{-1}.
\end{eqnarray*}
\end{Teo}



\begin{Teo}
El proceso de salida de un sistema de colas estacionario $GI/M/1$ es un proceso de renovaci\'on si y s\'olo si el proceso de entrada es un proceso Poisson, en cuyo caso el proceso de salida es un proceso Poisson.
\end{Teo}


\begin{Teo}
Los intervalos de interpartida $\left\{D_{n}\right\}$ de un sistema $M/G/1$ estacionario son no correlacionados si y s\'olo si la distribuci\'on de los tiempos de servicio es exponencial negativa, es decir, el sistema es de tipo  $M/M/1$.

\end{Teo}



\end{itemize}


%________________________________________________________________________
\subsection{Procesos Regenerativos}
%________________________________________________________________________

Para $\left\{X\left(t\right):t\geq0\right\}$ Proceso Estoc\'astico a tiempo continuo con estado de espacios $S$, que es un espacio m\'etrico, con trayectorias continuas por la derecha y con l\'imites por la izquierda c.s. Sea $N\left(t\right)$ un proceso de renovaci\'on en $\rea_{+}$ definido en el mismo espacio de probabilidad que $X\left(t\right)$, con tiempos de renovaci\'on $T$ y tiempos de inter-renovaci\'on $\xi_{n}=T_{n}-T_{n-1}$, con misma distribuci\'on $F$ de media finita $\mu$.



\begin{Def}
Para el proceso $\left\{\left(N\left(t\right),X\left(t\right)\right):t\geq0\right\}$, sus trayectoria muestrales en el intervalo de tiempo $\left[T_{n-1},T_{n}\right)$ est\'an descritas por
\begin{eqnarray*}
\zeta_{n}=\left(\xi_{n},\left\{X\left(T_{n-1}+t\right):0\leq t<\xi_{n}\right\}\right)
\end{eqnarray*}
Este $\zeta_{n}$ es el $n$-\'esimo segmento del proceso. El proceso es regenerativo sobre los tiempos $T_{n}$ si sus segmentos $\zeta_{n}$ son independientes e id\'enticamennte distribuidos.
\end{Def}


\begin{Obs}
Si $\tilde{X}\left(t\right)$ con espacio de estados $\tilde{S}$ es regenerativo sobre $T_{n}$, entonces $X\left(t\right)=f\left(\tilde{X}\left(t\right)\right)$ tambi\'en es regenerativo sobre $T_{n}$, para cualquier funci\'on $f:\tilde{S}\rightarrow S$.
\end{Obs}

\begin{Obs}
Los procesos regenerativos son crudamente regenerativos, pero no al rev\'es.
\end{Obs}

\begin{Def}[Definici\'on Cl\'asica]
Un proceso estoc\'astico $X=\left\{X\left(t\right):t\geq0\right\}$ es llamado regenerativo is existe una variable aleatoria $R_{1}>0$ tal que
\begin{itemize}
\item[i)] $\left\{X\left(t+R_{1}\right):t\geq0\right\}$ es independiente de $\left\{\left\{X\left(t\right):t<R_{1}\right\},\right\}$
\item[ii)] $\left\{X\left(t+R_{1}\right):t\geq0\right\}$ es estoc\'asticamente equivalente a $\left\{X\left(t\right):t>0\right\}$
\end{itemize}

Llamamos a $R_{1}$ tiempo de regeneraci\'on, y decimos que $X$ se regenera en este punto.
\end{Def}

$\left\{X\left(t+R_{1}\right)\right\}$ es regenerativo con tiempo de regeneraci\'on $R_{2}$, independiente de $R_{1}$ pero con la misma distribuci\'on que $R_{1}$. Procediendo de esta manera se obtiene una secuencia de variables aleatorias independientes e id\'enticamente distribuidas $\left\{R_{n}\right\}$ llamados longitudes de ciclo. Si definimos a $Z_{k}\equiv R_{1}+R_{2}+\cdots+R_{k}$, se tiene un proceso de renovaci\'on llamado proceso de renovaci\'on encajado para $X$.

\begin{Note}
Un proceso regenerativo con media de la longitud de ciclo finita es llamado positivo recurrente.
\end{Note}


\begin{Def}
Para $x$ fijo y para cada $t\geq0$, sea $I_{x}\left(t\right)=1$ si $X\left(t\right)\leq x$,  $I_{x}\left(t\right)=0$ en caso contrario, y def\'inanse los tiempos promedio
\begin{eqnarray*}
\overline{X}&=&lim_{t\rightarrow\infty}\frac{1}{t}\int_{0}^{\infty}X\left(u\right)du\\
\prob\left(X_{\infty}\leq x\right)&=&lim_{t\rightarrow\infty}\frac{1}{t}\int_{0}^{\infty}I_{x}\left(u\right)du,
\end{eqnarray*}
cuando estos l\'imites existan.
\end{Def}

Como consecuencia del teorema de Renovaci\'on-Recompensa, se tiene que el primer l\'imite  existe y es igual a la constante
\begin{eqnarray*}
\overline{X}&=&\frac{\esp\left[\int_{0}^{R_{1}}X\left(t\right)dt\right]}{\esp\left[R_{1}\right]},
\end{eqnarray*}
suponiendo que ambas esperanzas son finitas.

\begin{Note}
\begin{itemize}
\item[a)] Si el proceso regenerativo $X$ es positivo recurrente y tiene trayectorias muestrales no negativas, entonces la ecuaci\'on anterior es v\'alida.
\item[b)] Si $X$ es positivo recurrente regenerativo, podemos construir una \'unica versi\'on estacionaria de este proceso, $X_{e}=\left\{X_{e}\left(t\right)\right\}$, donde $X_{e}$ es un proceso estoc\'astico regenerativo y estrictamente estacionario, con distribuci\'on marginal distribuida como $X_{\infty}$
\end{itemize}
\end{Note}

\subsection{Renewal and Regenerative Processes: Serfozo\cite{Serfozo}}
\begin{Def}\label{Def.Tn}
Sean $0\leq T_{1}\leq T_{2}\leq \ldots$ son tiempos aleatorios infinitos en los cuales ocurren ciertos eventos. El n\'umero de tiempos $T_{n}$ en el intervalo $\left[0,t\right)$ es

\begin{eqnarray}
N\left(t\right)=\sum_{n=1}^{\infty}\indora\left(T_{n}\leq t\right),
\end{eqnarray}
para $t\geq0$.
\end{Def}

Si se consideran los puntos $T_{n}$ como elementos de $\rea_{+}$, y $N\left(t\right)$ es el n\'umero de puntos en $\rea$. El proceso denotado por $\left\{N\left(t\right):t\geq0\right\}$, denotado por $N\left(t\right)$, es un proceso puntual en $\rea_{+}$. Los $T_{n}$ son los tiempos de ocurrencia, el proceso puntual $N\left(t\right)$ es simple si su n\'umero de ocurrencias son distintas: $0<T_{1}<T_{2}<\ldots$ casi seguramente.

\begin{Def}
Un proceso puntual $N\left(t\right)$ es un proceso de renovaci\'on si los tiempos de interocurrencia $\xi_{n}=T_{n}-T_{n-1}$, para $n\geq1$, son independientes e identicamente distribuidos con distribuci\'on $F$, donde $F\left(0\right)=0$ y $T_{0}=0$. Los $T_{n}$ son llamados tiempos de renovaci\'on, referente a la independencia o renovaci\'on de la informaci\'on estoc\'astica en estos tiempos. Los $\xi_{n}$ son los tiempos de inter-renovaci\'on, y $N\left(t\right)$ es el n\'umero de renovaciones en el intervalo $\left[0,t\right)$
\end{Def}


\begin{Note}
Para definir un proceso de renovaci\'on para cualquier contexto, solamente hay que especificar una distribuci\'on $F$, con $F\left(0\right)=0$, para los tiempos de inter-renovaci\'on. La funci\'on $F$ en turno degune las otra variables aleatorias. De manera formal, existe un espacio de probabilidad y una sucesi\'on de variables aleatorias $\xi_{1},\xi_{2},\ldots$ definidas en este con distribuci\'on $F$. Entonces las otras cantidades son $T_{n}=\sum_{k=1}^{n}\xi_{k}$ y $N\left(t\right)=\sum_{n=1}^{\infty}\indora\left(T_{n}\leq t\right)$, donde $T_{n}\rightarrow\infty$ casi seguramente por la Ley Fuerte de los Grandes N\'umeros.
\end{Note}







Los tiempos $T_{n}$ est\'an relacionados con los conteos de $N\left(t\right)$ por

\begin{eqnarray*}
\left\{N\left(t\right)\geq n\right\}&=&\left\{T_{n}\leq t\right\}\\
T_{N\left(t\right)}\leq &t&<T_{N\left(t\right)+1},
\end{eqnarray*}

adem\'as $N\left(T_{n}\right)=n$, y 

\begin{eqnarray*}
N\left(t\right)=\max\left\{n:T_{n}\leq t\right\}=\min\left\{n:T_{n+1}>t\right\}
\end{eqnarray*}

Por propiedades de la convoluci\'on se sabe que

\begin{eqnarray*}
P\left\{T_{n}\leq t\right\}=F^{n\star}\left(t\right)
\end{eqnarray*}
que es la $n$-\'esima convoluci\'on de $F$. Entonces 

\begin{eqnarray*}
\left\{N\left(t\right)\geq n\right\}&=&\left\{T_{n}\leq t\right\}\\
P\left\{N\left(t\right)\leq n\right\}&=&1-F^{\left(n+1\right)\star}\left(t\right)
\end{eqnarray*}

Adem\'as usando el hecho de que $\esp\left[N\left(t\right)\right]=\sum_{n=1}^{\infty}P\left\{N\left(t\right)\geq n\right\}$
se tiene que

\begin{eqnarray*}
\esp\left[N\left(t\right)\right]=\sum_{n=1}^{\infty}F^{n\star}\left(t\right)
\end{eqnarray*}

\begin{Prop}
Para cada $t\geq0$, la funci\'on generadora de momentos $\esp\left[e^{\alpha N\left(t\right)}\right]$ existe para alguna $\alpha$ en una vecindad del 0, y de aqu\'i que $\esp\left[N\left(t\right)^{m}\right]<\infty$, para $m\geq1$.
\end{Prop}


\begin{Note}
Si el primer tiempo de renovaci\'on $\xi_{1}$ no tiene la misma distribuci\'on que el resto de las $\xi_{n}$, para $n\geq2$, a $N\left(t\right)$ se le llama Proceso de Renovaci\'on retardado, donde si $\xi$ tiene distribuci\'on $G$, entonces el tiempo $T_{n}$ de la $n$-\'esima renovaci\'on tiene distribuci\'on $G\star F^{\left(n-1\right)\star}\left(t\right)$
\end{Note}


\begin{Teo}
Para una constante $\mu\leq\infty$ ( o variable aleatoria), las siguientes expresiones son equivalentes:

\begin{eqnarray}
lim_{n\rightarrow\infty}n^{-1}T_{n}&=&\mu,\textrm{ c.s.}\\
lim_{t\rightarrow\infty}t^{-1}N\left(t\right)&=&1/\mu,\textrm{ c.s.}
\end{eqnarray}
\end{Teo}


Es decir, $T_{n}$ satisface la Ley Fuerte de los Grandes N\'umeros s\'i y s\'olo s\'i $N\left/t\right)$ la cumple.


\begin{Coro}[Ley Fuerte de los Grandes N\'umeros para Procesos de Renovaci\'on]
Si $N\left(t\right)$ es un proceso de renovaci\'on cuyos tiempos de inter-renovaci\'on tienen media $\mu\leq\infty$, entonces
\begin{eqnarray}
t^{-1}N\left(t\right)\rightarrow 1/\mu,\textrm{ c.s. cuando }t\rightarrow\infty.
\end{eqnarray}

\end{Coro}


Considerar el proceso estoc\'astico de valores reales $\left\{Z\left(t\right):t\geq0\right\}$ en el mismo espacio de probabilidad que $N\left(t\right)$

\begin{Def}
Para el proceso $\left\{Z\left(t\right):t\geq0\right\}$ se define la fluctuaci\'on m\'axima de $Z\left(t\right)$ en el intervalo $\left(T_{n-1},T_{n}\right]$:
\begin{eqnarray*}
M_{n}=\sup_{T_{n-1}<t\leq T_{n}}|Z\left(t\right)-Z\left(T_{n-1}\right)|
\end{eqnarray*}
\end{Def}

\begin{Teo}
Sup\'ongase que $n^{-1}T_{n}\rightarrow\mu$ c.s. cuando $n\rightarrow\infty$, donde $\mu\leq\infty$ es una constante o variable aleatoria. Sea $a$ una constante o variable aleatoria que puede ser infinita cuando $\mu$ es finita, y considere las expresiones l\'imite:
\begin{eqnarray}
lim_{n\rightarrow\infty}n^{-1}Z\left(T_{n}\right)&=&a,\textrm{ c.s.}\\
lim_{t\rightarrow\infty}t^{-1}Z\left(t\right)&=&a/\mu,\textrm{ c.s.}
\end{eqnarray}
La segunda expresi\'on implica la primera. Conversamente, la primera implica la segunda si el proceso $Z\left(t\right)$ es creciente, o si $lim_{n\rightarrow\infty}n^{-1}M_{n}=0$ c.s.
\end{Teo}

\begin{Coro}
Si $N\left(t\right)$ es un proceso de renovaci\'on, y $\left(Z\left(T_{n}\right)-Z\left(T_{n-1}\right),M_{n}\right)$, para $n\geq1$, son variables aleatorias independientes e id\'enticamente distribuidas con media finita, entonces,
\begin{eqnarray}
lim_{t\rightarrow\infty}t^{-1}Z\left(t\right)\rightarrow\frac{\esp\left[Z\left(T_{1}\right)-Z\left(T_{0}\right)\right]}{\esp\left[T_{1}\right]},\textrm{ c.s. cuando  }t\rightarrow\infty.
\end{eqnarray}
\end{Coro}


Sup\'ongase que $N\left(t\right)$ es un proceso de renovaci\'on con distribuci\'on $F$ con media finita $\mu$.

\begin{Def}
La funci\'on de renovaci\'on asociada con la distribuci\'on $F$, del proceso $N\left(t\right)$, es
\begin{eqnarray*}
U\left(t\right)=\sum_{n=1}^{\infty}F^{n\star}\left(t\right),\textrm{   }t\geq0,
\end{eqnarray*}
donde $F^{0\star}\left(t\right)=\indora\left(t\geq0\right)$.
\end{Def}


\begin{Prop}
Sup\'ongase que la distribuci\'on de inter-renovaci\'on $F$ tiene densidad $f$. Entonces $U\left(t\right)$ tambi\'en tiene densidad, para $t>0$, y es $U^{'}\left(t\right)=\sum_{n=0}^{\infty}f^{n\star}\left(t\right)$. Adem\'as
\begin{eqnarray*}
\prob\left\{N\left(t\right)>N\left(t-\right)\right\}=0\textrm{,   }t\geq0.
\end{eqnarray*}
\end{Prop}

\begin{Def}
La Transformada de Laplace-Stieljes de $F$ est\'a dada por

\begin{eqnarray*}
\hat{F}\left(\alpha\right)=\int_{\rea_{+}}e^{-\alpha t}dF\left(t\right)\textrm{,  }\alpha\geq0.
\end{eqnarray*}
\end{Def}

Entonces

\begin{eqnarray*}
\hat{U}\left(\alpha\right)=\sum_{n=0}^{\infty}\hat{F^{n\star}}\left(\alpha\right)=\sum_{n=0}^{\infty}\hat{F}\left(\alpha\right)^{n}=\frac{1}{1-\hat{F}\left(\alpha\right)}.
\end{eqnarray*}


\begin{Prop}
La Transformada de Laplace $\hat{U}\left(\alpha\right)$ y $\hat{F}\left(\alpha\right)$ determina una a la otra de manera \'unica por la relaci\'on $\hat{U}\left(\alpha\right)=\frac{1}{1-\hat{F}\left(\alpha\right)}$.
\end{Prop}


\begin{Note}
Un proceso de renovaci\'on $N\left(t\right)$ cuyos tiempos de inter-renovaci\'on tienen media finita, es un proceso Poisson con tasa $\lambda$ si y s\'olo s\'i $\esp\left[U\left(t\right)\right]=\lambda t$, para $t\geq0$.
\end{Note}


\begin{Teo}
Sea $N\left(t\right)$ un proceso puntual simple con puntos de localizaci\'on $T_{n}$ tal que $\eta\left(t\right)=\esp\left[N\left(\right)\right]$ es finita para cada $t$. Entonces para cualquier funci\'on $f:\rea_{+}\rightarrow\rea$,
\begin{eqnarray*}
\esp\left[\sum_{n=1}^{N\left(\right)}f\left(T_{n}\right)\right]=\int_{\left(0,t\right]}f\left(s\right)d\eta\left(s\right)\textrm{,  }t\geq0,
\end{eqnarray*}
suponiendo que la integral exista. Adem\'as si $X_{1},X_{2},\ldots$ son variables aleatorias definidas en el mismo espacio de probabilidad que el proceso $N\left(t\right)$ tal que $\esp\left[X_{n}|T_{n}=s\right]=f\left(s\right)$, independiente de $n$. Entonces
\begin{eqnarray*}
\esp\left[\sum_{n=1}^{N\left(t\right)}X_{n}\right]=\int_{\left(0,t\right]}f\left(s\right)d\eta\left(s\right)\textrm{,  }t\geq0,
\end{eqnarray*} 
suponiendo que la integral exista. 
\end{Teo}

\begin{Coro}[Identidad de Wald para Renovaciones]
Para el proceso de renovaci\'on $N\left(t\right)$,
\begin{eqnarray*}
\esp\left[T_{N\left(t\right)+1}\right]=\mu\esp\left[N\left(t\right)+1\right]\textrm{,  }t\geq0,
\end{eqnarray*}  
\end{Coro}


\begin{Def}
Sea $h\left(t\right)$ funci\'on de valores reales en $\rea$ acotada en intervalos finitos e igual a cero para $t<0$ La ecuaci\'on de renovaci\'on para $h\left(t\right)$ y la distribuci\'on $F$ es

\begin{eqnarray}\label{Ec.Renovacion}
H\left(t\right)=h\left(t\right)+\int_{\left[0,t\right]}H\left(t-s\right)dF\left(s\right)\textrm{,    }t\geq0,
\end{eqnarray}
donde $H\left(t\right)$ es una funci\'on de valores reales. Esto es $H=h+F\star H$. Decimos que $H\left(t\right)$ es soluci\'on de esta ecuaci\'on si satisface la ecuaci\'on, y es acotada en intervalos finitos e iguales a cero para $t<0$.
\end{Def}

\begin{Prop}
La funci\'on $U\star h\left(t\right)$ es la \'unica soluci\'on de la ecuaci\'on de renovaci\'on (\ref{Ec.Renovacion}).
\end{Prop}

\begin{Teo}[Teorema Renovaci\'on Elemental]
\begin{eqnarray*}
t^{-1}U\left(t\right)\rightarrow 1/\mu\textrm{,    cuando }t\rightarrow\infty.
\end{eqnarray*}
\end{Teo}



Sup\'ongase que $N\left(t\right)$ es un proceso de renovaci\'on con distribuci\'on $F$ con media finita $\mu$.

\begin{Def}
La funci\'on de renovaci\'on asociada con la distribuci\'on $F$, del proceso $N\left(t\right)$, es
\begin{eqnarray*}
U\left(t\right)=\sum_{n=1}^{\infty}F^{n\star}\left(t\right),\textrm{   }t\geq0,
\end{eqnarray*}
donde $F^{0\star}\left(t\right)=\indora\left(t\geq0\right)$.
\end{Def}


\begin{Prop}
Sup\'ongase que la distribuci\'on de inter-renovaci\'on $F$ tiene densidad $f$. Entonces $U\left(t\right)$ tambi\'en tiene densidad, para $t>0$, y es $U^{'}\left(t\right)=\sum_{n=0}^{\infty}f^{n\star}\left(t\right)$. Adem\'as
\begin{eqnarray*}
\prob\left\{N\left(t\right)>N\left(t-\right)\right\}=0\textrm{,   }t\geq0.
\end{eqnarray*}
\end{Prop}

\begin{Def}
La Transformada de Laplace-Stieljes de $F$ est\'a dada por

\begin{eqnarray*}
\hat{F}\left(\alpha\right)=\int_{\rea_{+}}e^{-\alpha t}dF\left(t\right)\textrm{,  }\alpha\geq0.
\end{eqnarray*}
\end{Def}

Entonces

\begin{eqnarray*}
\hat{U}\left(\alpha\right)=\sum_{n=0}^{\infty}\hat{F^{n\star}}\left(\alpha\right)=\sum_{n=0}^{\infty}\hat{F}\left(\alpha\right)^{n}=\frac{1}{1-\hat{F}\left(\alpha\right)}.
\end{eqnarray*}


\begin{Prop}
La Transformada de Laplace $\hat{U}\left(\alpha\right)$ y $\hat{F}\left(\alpha\right)$ determina una a la otra de manera \'unica por la relaci\'on $\hat{U}\left(\alpha\right)=\frac{1}{1-\hat{F}\left(\alpha\right)}$.
\end{Prop}


\begin{Note}
Un proceso de renovaci\'on $N\left(t\right)$ cuyos tiempos de inter-renovaci\'on tienen media finita, es un proceso Poisson con tasa $\lambda$ si y s\'olo s\'i $\esp\left[U\left(t\right)\right]=\lambda t$, para $t\geq0$.
\end{Note}


\begin{Teo}
Sea $N\left(t\right)$ un proceso puntual simple con puntos de localizaci\'on $T_{n}$ tal que $\eta\left(t\right)=\esp\left[N\left(\right)\right]$ es finita para cada $t$. Entonces para cualquier funci\'on $f:\rea_{+}\rightarrow\rea$,
\begin{eqnarray*}
\esp\left[\sum_{n=1}^{N\left(\right)}f\left(T_{n}\right)\right]=\int_{\left(0,t\right]}f\left(s\right)d\eta\left(s\right)\textrm{,  }t\geq0,
\end{eqnarray*}
suponiendo que la integral exista. Adem\'as si $X_{1},X_{2},\ldots$ son variables aleatorias definidas en el mismo espacio de probabilidad que el proceso $N\left(t\right)$ tal que $\esp\left[X_{n}|T_{n}=s\right]=f\left(s\right)$, independiente de $n$. Entonces
\begin{eqnarray*}
\esp\left[\sum_{n=1}^{N\left(t\right)}X_{n}\right]=\int_{\left(0,t\right]}f\left(s\right)d\eta\left(s\right)\textrm{,  }t\geq0,
\end{eqnarray*} 
suponiendo que la integral exista. 
\end{Teo}

\begin{Coro}[Identidad de Wald para Renovaciones]
Para el proceso de renovaci\'on $N\left(t\right)$,
\begin{eqnarray*}
\esp\left[T_{N\left(t\right)+1}\right]=\mu\esp\left[N\left(t\right)+1\right]\textrm{,  }t\geq0,
\end{eqnarray*}  
\end{Coro}


\begin{Def}
Sea $h\left(t\right)$ funci\'on de valores reales en $\rea$ acotada en intervalos finitos e igual a cero para $t<0$ La ecuaci\'on de renovaci\'on para $h\left(t\right)$ y la distribuci\'on $F$ es

\begin{eqnarray}\label{Ec.Renovacion}
H\left(t\right)=h\left(t\right)+\int_{\left[0,t\right]}H\left(t-s\right)dF\left(s\right)\textrm{,    }t\geq0,
\end{eqnarray}
donde $H\left(t\right)$ es una funci\'on de valores reales. Esto es $H=h+F\star H$. Decimos que $H\left(t\right)$ es soluci\'on de esta ecuaci\'on si satisface la ecuaci\'on, y es acotada en intervalos finitos e iguales a cero para $t<0$.
\end{Def}

\begin{Prop}
La funci\'on $U\star h\left(t\right)$ es la \'unica soluci\'on de la ecuaci\'on de renovaci\'on (\ref{Ec.Renovacion}).
\end{Prop}

\begin{Teo}[Teorema Renovaci\'on Elemental]
\begin{eqnarray*}
t^{-1}U\left(t\right)\rightarrow 1/\mu\textrm{,    cuando }t\rightarrow\infty.
\end{eqnarray*}
\end{Teo}


\begin{Note} Una funci\'on $h:\rea_{+}\rightarrow\rea$ es Directamente Riemann Integrable en los siguientes casos:
\begin{itemize}
\item[a)] $h\left(t\right)\geq0$ es decreciente y Riemann Integrable.
\item[b)] $h$ es continua excepto posiblemente en un conjunto de Lebesgue de medida 0, y $|h\left(t\right)|\leq b\left(t\right)$, donde $b$ es DRI.
\end{itemize}
\end{Note}

\begin{Teo}[Teorema Principal de Renovaci\'on]
Si $F$ es no aritm\'etica y $h\left(t\right)$ es Directamente Riemann Integrable (DRI), entonces

\begin{eqnarray*}
lim_{t\rightarrow\infty}U\star h=\frac{1}{\mu}\int_{\rea_{+}}h\left(s\right)ds.
\end{eqnarray*}
\end{Teo}

\begin{Prop}
Cualquier funci\'on $H\left(t\right)$ acotada en intervalos finitos y que es 0 para $t<0$ puede expresarse como
\begin{eqnarray*}
H\left(t\right)=U\star h\left(t\right)\textrm{,  donde }h\left(t\right)=H\left(t\right)-F\star H\left(t\right)
\end{eqnarray*}
\end{Prop}

\begin{Def}
Un proceso estoc\'astico $X\left(t\right)$ es crudamente regenerativo en un tiempo aleatorio positivo $T$ si
\begin{eqnarray*}
\esp\left[X\left(T+t\right)|T\right]=\esp\left[X\left(t\right)\right]\textrm{, para }t\geq0,\end{eqnarray*}
y con las esperanzas anteriores finitas.
\end{Def}

\begin{Prop}
Sup\'ongase que $X\left(t\right)$ es un proceso crudamente regenerativo en $T$, que tiene distribuci\'on $F$. Si $\esp\left[X\left(t\right)\right]$ es acotado en intervalos finitos, entonces
\begin{eqnarray*}
\esp\left[X\left(t\right)\right]=U\star h\left(t\right)\textrm{,  donde }h\left(t\right)=\esp\left[X\left(t\right)\indora\left(T>t\right)\right].
\end{eqnarray*}
\end{Prop}

\begin{Teo}[Regeneraci\'on Cruda]
Sup\'ongase que $X\left(t\right)$ es un proceso con valores positivo crudamente regenerativo en $T$, y def\'inase $M=\sup\left\{|X\left(t\right)|:t\leq T\right\}$. Si $T$ es no aritm\'etico y $M$ y $MT$ tienen media finita, entonces
\begin{eqnarray*}
lim_{t\rightarrow\infty}\esp\left[X\left(t\right)\right]=\frac{1}{\mu}\int_{\rea_{+}}h\left(s\right)ds,
\end{eqnarray*}
donde $h\left(t\right)=\esp\left[X\left(t\right)\indora\left(T>t\right)\right]$.
\end{Teo}


\begin{Note} Una funci\'on $h:\rea_{+}\rightarrow\rea$ es Directamente Riemann Integrable en los siguientes casos:
\begin{itemize}
\item[a)] $h\left(t\right)\geq0$ es decreciente y Riemann Integrable.
\item[b)] $h$ es continua excepto posiblemente en un conjunto de Lebesgue de medida 0, y $|h\left(t\right)|\leq b\left(t\right)$, donde $b$ es DRI.
\end{itemize}
\end{Note}

\begin{Teo}[Teorema Principal de Renovaci\'on]
Si $F$ es no aritm\'etica y $h\left(t\right)$ es Directamente Riemann Integrable (DRI), entonces

\begin{eqnarray*}
lim_{t\rightarrow\infty}U\star h=\frac{1}{\mu}\int_{\rea_{+}}h\left(s\right)ds.
\end{eqnarray*}
\end{Teo}

\begin{Prop}
Cualquier funci\'on $H\left(t\right)$ acotada en intervalos finitos y que es 0 para $t<0$ puede expresarse como
\begin{eqnarray*}
H\left(t\right)=U\star h\left(t\right)\textrm{,  donde }h\left(t\right)=H\left(t\right)-F\star H\left(t\right)
\end{eqnarray*}
\end{Prop}

\begin{Def}
Un proceso estoc\'astico $X\left(t\right)$ es crudamente regenerativo en un tiempo aleatorio positivo $T$ si
\begin{eqnarray*}
\esp\left[X\left(T+t\right)|T\right]=\esp\left[X\left(t\right)\right]\textrm{, para }t\geq0,\end{eqnarray*}
y con las esperanzas anteriores finitas.
\end{Def}

\begin{Prop}
Sup\'ongase que $X\left(t\right)$ es un proceso crudamente regenerativo en $T$, que tiene distribuci\'on $F$. Si $\esp\left[X\left(t\right)\right]$ es acotado en intervalos finitos, entonces
\begin{eqnarray*}
\esp\left[X\left(t\right)\right]=U\star h\left(t\right)\textrm{,  donde }h\left(t\right)=\esp\left[X\left(t\right)\indora\left(T>t\right)\right].
\end{eqnarray*}
\end{Prop}

\begin{Teo}[Regeneraci\'on Cruda]
Sup\'ongase que $X\left(t\right)$ es un proceso con valores positivo crudamente regenerativo en $T$, y def\'inase $M=\sup\left\{|X\left(t\right)|:t\leq T\right\}$. Si $T$ es no aritm\'etico y $M$ y $MT$ tienen media finita, entonces
\begin{eqnarray*}
lim_{t\rightarrow\infty}\esp\left[X\left(t\right)\right]=\frac{1}{\mu}\int_{\rea_{+}}h\left(s\right)ds,
\end{eqnarray*}
donde $h\left(t\right)=\esp\left[X\left(t\right)\indora\left(T>t\right)\right]$.
\end{Teo}

%________________________________________________________________________
\subsection{Procesos Regenerativos}
%________________________________________________________________________

Para $\left\{X\left(t\right):t\geq0\right\}$ Proceso Estoc\'astico a tiempo continuo con estado de espacios $S$, que es un espacio m\'etrico, con trayectorias continuas por la derecha y con l\'imites por la izquierda c.s. Sea $N\left(t\right)$ un proceso de renovaci\'on en $\rea_{+}$ definido en el mismo espacio de probabilidad que $X\left(t\right)$, con tiempos de renovaci\'on $T$ y tiempos de inter-renovaci\'on $\xi_{n}=T_{n}-T_{n-1}$, con misma distribuci\'on $F$ de media finita $\mu$.



\begin{Def}
Para el proceso $\left\{\left(N\left(t\right),X\left(t\right)\right):t\geq0\right\}$, sus trayectoria muestrales en el intervalo de tiempo $\left[T_{n-1},T_{n}\right)$ est\'an descritas por
\begin{eqnarray*}
\zeta_{n}=\left(\xi_{n},\left\{X\left(T_{n-1}+t\right):0\leq t<\xi_{n}\right\}\right)
\end{eqnarray*}
Este $\zeta_{n}$ es el $n$-\'esimo segmento del proceso. El proceso es regenerativo sobre los tiempos $T_{n}$ si sus segmentos $\zeta_{n}$ son independientes e id\'enticamennte distribuidos.
\end{Def}


\begin{Obs}
Si $\tilde{X}\left(t\right)$ con espacio de estados $\tilde{S}$ es regenerativo sobre $T_{n}$, entonces $X\left(t\right)=f\left(\tilde{X}\left(t\right)\right)$ tambi\'en es regenerativo sobre $T_{n}$, para cualquier funci\'on $f:\tilde{S}\rightarrow S$.
\end{Obs}

\begin{Obs}
Los procesos regenerativos son crudamente regenerativos, pero no al rev\'es.
\end{Obs}

\begin{Def}[Definici\'on Cl\'asica]
Un proceso estoc\'astico $X=\left\{X\left(t\right):t\geq0\right\}$ es llamado regenerativo is existe una variable aleatoria $R_{1}>0$ tal que
\begin{itemize}
\item[i)] $\left\{X\left(t+R_{1}\right):t\geq0\right\}$ es independiente de $\left\{\left\{X\left(t\right):t<R_{1}\right\},\right\}$
\item[ii)] $\left\{X\left(t+R_{1}\right):t\geq0\right\}$ es estoc\'asticamente equivalente a $\left\{X\left(t\right):t>0\right\}$
\end{itemize}

Llamamos a $R_{1}$ tiempo de regeneraci\'on, y decimos que $X$ se regenera en este punto.
\end{Def}

$\left\{X\left(t+R_{1}\right)\right\}$ es regenerativo con tiempo de regeneraci\'on $R_{2}$, independiente de $R_{1}$ pero con la misma distribuci\'on que $R_{1}$. Procediendo de esta manera se obtiene una secuencia de variables aleatorias independientes e id\'enticamente distribuidas $\left\{R_{n}\right\}$ llamados longitudes de ciclo. Si definimos a $Z_{k}\equiv R_{1}+R_{2}+\cdots+R_{k}$, se tiene un proceso de renovaci\'on llamado proceso de renovaci\'on encajado para $X$.

\begin{Note}
Un proceso regenerativo con media de la longitud de ciclo finita es llamado positivo recurrente.
\end{Note}


\begin{Def}
Para $x$ fijo y para cada $t\geq0$, sea $I_{x}\left(t\right)=1$ si $X\left(t\right)\leq x$,  $I_{x}\left(t\right)=0$ en caso contrario, y def\'inanse los tiempos promedio
\begin{eqnarray*}
\overline{X}&=&lim_{t\rightarrow\infty}\frac{1}{t}\int_{0}^{\infty}X\left(u\right)du\\
\prob\left(X_{\infty}\leq x\right)&=&lim_{t\rightarrow\infty}\frac{1}{t}\int_{0}^{\infty}I_{x}\left(u\right)du,
\end{eqnarray*}
cuando estos l\'imites existan.
\end{Def}

Como consecuencia del teorema de Renovaci\'on-Recompensa, se tiene que el primer l\'imite  existe y es igual a la constante
\begin{eqnarray*}
\overline{X}&=&\frac{\esp\left[\int_{0}^{R_{1}}X\left(t\right)dt\right]}{\esp\left[R_{1}\right]},
\end{eqnarray*}
suponiendo que ambas esperanzas son finitas.

\begin{Note}
\begin{itemize}
\item[a)] Si el proceso regenerativo $X$ es positivo recurrente y tiene trayectorias muestrales no negativas, entonces la ecuaci\'on anterior es v\'alida.
\item[b)] Si $X$ es positivo recurrente regenerativo, podemos construir una \'unica versi\'on estacionaria de este proceso, $X_{e}=\left\{X_{e}\left(t\right)\right\}$, donde $X_{e}$ es un proceso estoc\'astico regenerativo y estrictamente estacionario, con distribuci\'on marginal distribuida como $X_{\infty}$
\end{itemize}
\end{Note}

%________________________________________________________________________
\subsection{Procesos Regenerativos}
%________________________________________________________________________

Para $\left\{X\left(t\right):t\geq0\right\}$ Proceso Estoc\'astico a tiempo continuo con estado de espacios $S$, que es un espacio m\'etrico, con trayectorias continuas por la derecha y con l\'imites por la izquierda c.s. Sea $N\left(t\right)$ un proceso de renovaci\'on en $\rea_{+}$ definido en el mismo espacio de probabilidad que $X\left(t\right)$, con tiempos de renovaci\'on $T$ y tiempos de inter-renovaci\'on $\xi_{n}=T_{n}-T_{n-1}$, con misma distribuci\'on $F$ de media finita $\mu$.



\begin{Def}
Para el proceso $\left\{\left(N\left(t\right),X\left(t\right)\right):t\geq0\right\}$, sus trayectoria muestrales en el intervalo de tiempo $\left[T_{n-1},T_{n}\right)$ est\'an descritas por
\begin{eqnarray*}
\zeta_{n}=\left(\xi_{n},\left\{X\left(T_{n-1}+t\right):0\leq t<\xi_{n}\right\}\right)
\end{eqnarray*}
Este $\zeta_{n}$ es el $n$-\'esimo segmento del proceso. El proceso es regenerativo sobre los tiempos $T_{n}$ si sus segmentos $\zeta_{n}$ son independientes e id\'enticamennte distribuidos.
\end{Def}


\begin{Obs}
Si $\tilde{X}\left(t\right)$ con espacio de estados $\tilde{S}$ es regenerativo sobre $T_{n}$, entonces $X\left(t\right)=f\left(\tilde{X}\left(t\right)\right)$ tambi\'en es regenerativo sobre $T_{n}$, para cualquier funci\'on $f:\tilde{S}\rightarrow S$.
\end{Obs}

\begin{Obs}
Los procesos regenerativos son crudamente regenerativos, pero no al rev\'es.
\end{Obs}

\begin{Def}[Definici\'on Cl\'asica]
Un proceso estoc\'astico $X=\left\{X\left(t\right):t\geq0\right\}$ es llamado regenerativo is existe una variable aleatoria $R_{1}>0$ tal que
\begin{itemize}
\item[i)] $\left\{X\left(t+R_{1}\right):t\geq0\right\}$ es independiente de $\left\{\left\{X\left(t\right):t<R_{1}\right\},\right\}$
\item[ii)] $\left\{X\left(t+R_{1}\right):t\geq0\right\}$ es estoc\'asticamente equivalente a $\left\{X\left(t\right):t>0\right\}$
\end{itemize}

Llamamos a $R_{1}$ tiempo de regeneraci\'on, y decimos que $X$ se regenera en este punto.
\end{Def}

$\left\{X\left(t+R_{1}\right)\right\}$ es regenerativo con tiempo de regeneraci\'on $R_{2}$, independiente de $R_{1}$ pero con la misma distribuci\'on que $R_{1}$. Procediendo de esta manera se obtiene una secuencia de variables aleatorias independientes e id\'enticamente distribuidas $\left\{R_{n}\right\}$ llamados longitudes de ciclo. Si definimos a $Z_{k}\equiv R_{1}+R_{2}+\cdots+R_{k}$, se tiene un proceso de renovaci\'on llamado proceso de renovaci\'on encajado para $X$.

\begin{Note}
Un proceso regenerativo con media de la longitud de ciclo finita es llamado positivo recurrente.
\end{Note}


\begin{Def}
Para $x$ fijo y para cada $t\geq0$, sea $I_{x}\left(t\right)=1$ si $X\left(t\right)\leq x$,  $I_{x}\left(t\right)=0$ en caso contrario, y def\'inanse los tiempos promedio
\begin{eqnarray*}
\overline{X}&=&lim_{t\rightarrow\infty}\frac{1}{t}\int_{0}^{\infty}X\left(u\right)du\\
\prob\left(X_{\infty}\leq x\right)&=&lim_{t\rightarrow\infty}\frac{1}{t}\int_{0}^{\infty}I_{x}\left(u\right)du,
\end{eqnarray*}
cuando estos l\'imites existan.
\end{Def}

Como consecuencia del teorema de Renovaci\'on-Recompensa, se tiene que el primer l\'imite  existe y es igual a la constante
\begin{eqnarray*}
\overline{X}&=&\frac{\esp\left[\int_{0}^{R_{1}}X\left(t\right)dt\right]}{\esp\left[R_{1}\right]},
\end{eqnarray*}
suponiendo que ambas esperanzas son finitas.

\begin{Note}
\begin{itemize}
\item[a)] Si el proceso regenerativo $X$ es positivo recurrente y tiene trayectorias muestrales no negativas, entonces la ecuaci\'on anterior es v\'alida.
\item[b)] Si $X$ es positivo recurrente regenerativo, podemos construir una \'unica versi\'on estacionaria de este proceso, $X_{e}=\left\{X_{e}\left(t\right)\right\}$, donde $X_{e}$ es un proceso estoc\'astico regenerativo y estrictamente estacionario, con distribuci\'on marginal distribuida como $X_{\infty}$
\end{itemize}
\end{Note}
%__________________________________________________________________________________________
\subsection{Procesos Regenerativos Estacionarios - Stidham \cite{Stidham}}
%__________________________________________________________________________________________


Un proceso estoc\'astico a tiempo continuo $\left\{V\left(t\right),t\geq0\right\}$ es un proceso regenerativo si existe una sucesi\'on de variables aleatorias independientes e id\'enticamente distribuidas $\left\{X_{1},X_{2},\ldots\right\}$, sucesi\'on de renovaci\'on, tal que para cualquier conjunto de Borel $A$, 

\begin{eqnarray*}
\prob\left\{V\left(t\right)\in A|X_{1}+X_{2}+\cdots+X_{R\left(t\right)}=s,\left\{V\left(\tau\right),\tau<s\right\}\right\}=\prob\left\{V\left(t-s\right)\in A|X_{1}>t-s\right\},
\end{eqnarray*}
para todo $0\leq s\leq t$, donde $R\left(t\right)=\max\left\{X_{1}+X_{2}+\cdots+X_{j}\leq t\right\}=$n\'umero de renovaciones ({\emph{puntos de regeneraci\'on}}) que ocurren en $\left[0,t\right]$. El intervalo $\left[0,X_{1}\right)$ es llamado {\emph{primer ciclo de regeneraci\'on}} de $\left\{V\left(t \right),t\geq0\right\}$, $\left[X_{1},X_{1}+X_{2}\right)$ el {\emph{segundo ciclo de regeneraci\'on}}, y as\'i sucesivamente.

Sea $X=X_{1}$ y sea $F$ la funci\'on de distrbuci\'on de $X$


\begin{Def}
Se define el proceso estacionario, $\left\{V^{*}\left(t\right),t\geq0\right\}$, para $\left\{V\left(t\right),t\geq0\right\}$ por

\begin{eqnarray*}
\prob\left\{V\left(t\right)\in A\right\}=\frac{1}{\esp\left[X\right]}\int_{0}^{\infty}\prob\left\{V\left(t+x\right)\in A|X>x\right\}\left(1-F\left(x\right)\right)dx,
\end{eqnarray*} 
para todo $t\geq0$ y todo conjunto de Borel $A$.
\end{Def}

\begin{Def}
Una distribuci\'on se dice que es {\emph{aritm\'etica}} si todos sus puntos de incremento son m\'ultiplos de la forma $0,\lambda, 2\lambda,\ldots$ para alguna $\lambda>0$ entera.
\end{Def}


\begin{Def}
Una modificaci\'on medible de un proceso $\left\{V\left(t\right),t\geq0\right\}$, es una versi\'on de este, $\left\{V\left(t,w\right)\right\}$ conjuntamente medible para $t\geq0$ y para $w\in S$, $S$ espacio de estados para $\left\{V\left(t\right),t\geq0\right\}$.
\end{Def}

\begin{Teo}
Sea $\left\{V\left(t\right),t\geq\right\}$ un proceso regenerativo no negativo con modificaci\'on medible. Sea $\esp\left[X\right]<\infty$. Entonces el proceso estacionario dado por la ecuaci\'on anterior est\'a bien definido y tiene funci\'on de distribuci\'on independiente de $t$, adem\'as
\begin{itemize}
\item[i)] \begin{eqnarray*}
\esp\left[V^{*}\left(0\right)\right]&=&\frac{\esp\left[\int_{0}^{X}V\left(s\right)ds\right]}{\esp\left[X\right]}\end{eqnarray*}
\item[ii)] Si $\esp\left[V^{*}\left(0\right)\right]<\infty$, equivalentemente, si $\esp\left[\int_{0}^{X}V\left(s\right)ds\right]<\infty$,entonces
\begin{eqnarray*}
\frac{\int_{0}^{t}V\left(s\right)ds}{t}\rightarrow\frac{\esp\left[\int_{0}^{X}V\left(s\right)ds\right]}{\esp\left[X\right]}
\end{eqnarray*}
con probabilidad 1 y en media, cuando $t\rightarrow\infty$.
\end{itemize}
\end{Teo}


%__________________________________________________________________________________________
\subsection{Procesos Regenerativos Estacionarios - Stidham \cite{Stidham}}
%__________________________________________________________________________________________


Un proceso estoc\'astico a tiempo continuo $\left\{V\left(t\right),t\geq0\right\}$ es un proceso regenerativo si existe una sucesi\'on de variables aleatorias independientes e id\'enticamente distribuidas $\left\{X_{1},X_{2},\ldots\right\}$, sucesi\'on de renovaci\'on, tal que para cualquier conjunto de Borel $A$, 

\begin{eqnarray*}
\prob\left\{V\left(t\right)\in A|X_{1}+X_{2}+\cdots+X_{R\left(t\right)}=s,\left\{V\left(\tau\right),\tau<s\right\}\right\}=\prob\left\{V\left(t-s\right)\in A|X_{1}>t-s\right\},
\end{eqnarray*}
para todo $0\leq s\leq t$, donde $R\left(t\right)=\max\left\{X_{1}+X_{2}+\cdots+X_{j}\leq t\right\}=$n\'umero de renovaciones ({\emph{puntos de regeneraci\'on}}) que ocurren en $\left[0,t\right]$. El intervalo $\left[0,X_{1}\right)$ es llamado {\emph{primer ciclo de regeneraci\'on}} de $\left\{V\left(t \right),t\geq0\right\}$, $\left[X_{1},X_{1}+X_{2}\right)$ el {\emph{segundo ciclo de regeneraci\'on}}, y as\'i sucesivamente.

Sea $X=X_{1}$ y sea $F$ la funci\'on de distrbuci\'on de $X$


\begin{Def}
Se define el proceso estacionario, $\left\{V^{*}\left(t\right),t\geq0\right\}$, para $\left\{V\left(t\right),t\geq0\right\}$ por

\begin{eqnarray*}
\prob\left\{V\left(t\right)\in A\right\}=\frac{1}{\esp\left[X\right]}\int_{0}^{\infty}\prob\left\{V\left(t+x\right)\in A|X>x\right\}\left(1-F\left(x\right)\right)dx,
\end{eqnarray*} 
para todo $t\geq0$ y todo conjunto de Borel $A$.
\end{Def}

\begin{Def}
Una distribuci\'on se dice que es {\emph{aritm\'etica}} si todos sus puntos de incremento son m\'ultiplos de la forma $0,\lambda, 2\lambda,\ldots$ para alguna $\lambda>0$ entera.
\end{Def}


\begin{Def}
Una modificaci\'on medible de un proceso $\left\{V\left(t\right),t\geq0\right\}$, es una versi\'on de este, $\left\{V\left(t,w\right)\right\}$ conjuntamente medible para $t\geq0$ y para $w\in S$, $S$ espacio de estados para $\left\{V\left(t\right),t\geq0\right\}$.
\end{Def}

\begin{Teo}
Sea $\left\{V\left(t\right),t\geq\right\}$ un proceso regenerativo no negativo con modificaci\'on medible. Sea $\esp\left[X\right]<\infty$. Entonces el proceso estacionario dado por la ecuaci\'on anterior est\'a bien definido y tiene funci\'on de distribuci\'on independiente de $t$, adem\'as
\begin{itemize}
\item[i)] \begin{eqnarray*}
\esp\left[V^{*}\left(0\right)\right]&=&\frac{\esp\left[\int_{0}^{X}V\left(s\right)ds\right]}{\esp\left[X\right]}\end{eqnarray*}
\item[ii)] Si $\esp\left[V^{*}\left(0\right)\right]<\infty$, equivalentemente, si $\esp\left[\int_{0}^{X}V\left(s\right)ds\right]<\infty$,entonces
\begin{eqnarray*}
\frac{\int_{0}^{t}V\left(s\right)ds}{t}\rightarrow\frac{\esp\left[\int_{0}^{X}V\left(s\right)ds\right]}{\esp\left[X\right]}
\end{eqnarray*}
con probabilidad 1 y en media, cuando $t\rightarrow\infty$.
\end{itemize}
\end{Teo}
%
%___________________________________________________________________________________________
%\vspace{5.5cm}
%\chapter{Cadenas de Markov estacionarias}
%\vspace{-1.0cm}
%___________________________________________________________________________________________
%
\subsection{Propiedades de los Procesos de Renovaci\'on}
%___________________________________________________________________________________________
%

Los tiempos $T_{n}$ est\'an relacionados con los conteos de $N\left(t\right)$ por

\begin{eqnarray*}
\left\{N\left(t\right)\geq n\right\}&=&\left\{T_{n}\leq t\right\}\\
T_{N\left(t\right)}\leq &t&<T_{N\left(t\right)+1},
\end{eqnarray*}

adem\'as $N\left(T_{n}\right)=n$, y 

\begin{eqnarray*}
N\left(t\right)=\max\left\{n:T_{n}\leq t\right\}=\min\left\{n:T_{n+1}>t\right\}
\end{eqnarray*}

Por propiedades de la convoluci\'on se sabe que

\begin{eqnarray*}
P\left\{T_{n}\leq t\right\}=F^{n\star}\left(t\right)
\end{eqnarray*}
que es la $n$-\'esima convoluci\'on de $F$. Entonces 

\begin{eqnarray*}
\left\{N\left(t\right)\geq n\right\}&=&\left\{T_{n}\leq t\right\}\\
P\left\{N\left(t\right)\leq n\right\}&=&1-F^{\left(n+1\right)\star}\left(t\right)
\end{eqnarray*}

Adem\'as usando el hecho de que $\esp\left[N\left(t\right)\right]=\sum_{n=1}^{\infty}P\left\{N\left(t\right)\geq n\right\}$
se tiene que

\begin{eqnarray*}
\esp\left[N\left(t\right)\right]=\sum_{n=1}^{\infty}F^{n\star}\left(t\right)
\end{eqnarray*}

\begin{Prop}
Para cada $t\geq0$, la funci\'on generadora de momentos $\esp\left[e^{\alpha N\left(t\right)}\right]$ existe para alguna $\alpha$ en una vecindad del 0, y de aqu\'i que $\esp\left[N\left(t\right)^{m}\right]<\infty$, para $m\geq1$.
\end{Prop}


\begin{Note}
Si el primer tiempo de renovaci\'on $\xi_{1}$ no tiene la misma distribuci\'on que el resto de las $\xi_{n}$, para $n\geq2$, a $N\left(t\right)$ se le llama Proceso de Renovaci\'on retardado, donde si $\xi$ tiene distribuci\'on $G$, entonces el tiempo $T_{n}$ de la $n$-\'esima renovaci\'on tiene distribuci\'on $G\star F^{\left(n-1\right)\star}\left(t\right)$
\end{Note}


\begin{Teo}
Para una constante $\mu\leq\infty$ ( o variable aleatoria), las siguientes expresiones son equivalentes:

\begin{eqnarray}
lim_{n\rightarrow\infty}n^{-1}T_{n}&=&\mu,\textrm{ c.s.}\\
lim_{t\rightarrow\infty}t^{-1}N\left(t\right)&=&1/\mu,\textrm{ c.s.}
\end{eqnarray}
\end{Teo}


Es decir, $T_{n}$ satisface la Ley Fuerte de los Grandes N\'umeros s\'i y s\'olo s\'i $N\left/t\right)$ la cumple.


\begin{Coro}[Ley Fuerte de los Grandes N\'umeros para Procesos de Renovaci\'on]
Si $N\left(t\right)$ es un proceso de renovaci\'on cuyos tiempos de inter-renovaci\'on tienen media $\mu\leq\infty$, entonces
\begin{eqnarray}
t^{-1}N\left(t\right)\rightarrow 1/\mu,\textrm{ c.s. cuando }t\rightarrow\infty.
\end{eqnarray}

\end{Coro}


Considerar el proceso estoc\'astico de valores reales $\left\{Z\left(t\right):t\geq0\right\}$ en el mismo espacio de probabilidad que $N\left(t\right)$

\begin{Def}
Para el proceso $\left\{Z\left(t\right):t\geq0\right\}$ se define la fluctuaci\'on m\'axima de $Z\left(t\right)$ en el intervalo $\left(T_{n-1},T_{n}\right]$:
\begin{eqnarray*}
M_{n}=\sup_{T_{n-1}<t\leq T_{n}}|Z\left(t\right)-Z\left(T_{n-1}\right)|
\end{eqnarray*}
\end{Def}

\begin{Teo}
Sup\'ongase que $n^{-1}T_{n}\rightarrow\mu$ c.s. cuando $n\rightarrow\infty$, donde $\mu\leq\infty$ es una constante o variable aleatoria. Sea $a$ una constante o variable aleatoria que puede ser infinita cuando $\mu$ es finita, y considere las expresiones l\'imite:
\begin{eqnarray}
lim_{n\rightarrow\infty}n^{-1}Z\left(T_{n}\right)&=&a,\textrm{ c.s.}\\
lim_{t\rightarrow\infty}t^{-1}Z\left(t\right)&=&a/\mu,\textrm{ c.s.}
\end{eqnarray}
La segunda expresi\'on implica la primera. Conversamente, la primera implica la segunda si el proceso $Z\left(t\right)$ es creciente, o si $lim_{n\rightarrow\infty}n^{-1}M_{n}=0$ c.s.
\end{Teo}

\begin{Coro}
Si $N\left(t\right)$ es un proceso de renovaci\'on, y $\left(Z\left(T_{n}\right)-Z\left(T_{n-1}\right),M_{n}\right)$, para $n\geq1$, son variables aleatorias independientes e id\'enticamente distribuidas con media finita, entonces,
\begin{eqnarray}
lim_{t\rightarrow\infty}t^{-1}Z\left(t\right)\rightarrow\frac{\esp\left[Z\left(T_{1}\right)-Z\left(T_{0}\right)\right]}{\esp\left[T_{1}\right]},\textrm{ c.s. cuando  }t\rightarrow\infty.
\end{eqnarray}
\end{Coro}


%___________________________________________________________________________________________
%
%\subsection{Propiedades de los Procesos de Renovaci\'on}
%___________________________________________________________________________________________
%

Los tiempos $T_{n}$ est\'an relacionados con los conteos de $N\left(t\right)$ por

\begin{eqnarray*}
\left\{N\left(t\right)\geq n\right\}&=&\left\{T_{n}\leq t\right\}\\
T_{N\left(t\right)}\leq &t&<T_{N\left(t\right)+1},
\end{eqnarray*}

adem\'as $N\left(T_{n}\right)=n$, y 

\begin{eqnarray*}
N\left(t\right)=\max\left\{n:T_{n}\leq t\right\}=\min\left\{n:T_{n+1}>t\right\}
\end{eqnarray*}

Por propiedades de la convoluci\'on se sabe que

\begin{eqnarray*}
P\left\{T_{n}\leq t\right\}=F^{n\star}\left(t\right)
\end{eqnarray*}
que es la $n$-\'esima convoluci\'on de $F$. Entonces 

\begin{eqnarray*}
\left\{N\left(t\right)\geq n\right\}&=&\left\{T_{n}\leq t\right\}\\
P\left\{N\left(t\right)\leq n\right\}&=&1-F^{\left(n+1\right)\star}\left(t\right)
\end{eqnarray*}

Adem\'as usando el hecho de que $\esp\left[N\left(t\right)\right]=\sum_{n=1}^{\infty}P\left\{N\left(t\right)\geq n\right\}$
se tiene que

\begin{eqnarray*}
\esp\left[N\left(t\right)\right]=\sum_{n=1}^{\infty}F^{n\star}\left(t\right)
\end{eqnarray*}

\begin{Prop}
Para cada $t\geq0$, la funci\'on generadora de momentos $\esp\left[e^{\alpha N\left(t\right)}\right]$ existe para alguna $\alpha$ en una vecindad del 0, y de aqu\'i que $\esp\left[N\left(t\right)^{m}\right]<\infty$, para $m\geq1$.
\end{Prop}


\begin{Note}
Si el primer tiempo de renovaci\'on $\xi_{1}$ no tiene la misma distribuci\'on que el resto de las $\xi_{n}$, para $n\geq2$, a $N\left(t\right)$ se le llama Proceso de Renovaci\'on retardado, donde si $\xi$ tiene distribuci\'on $G$, entonces el tiempo $T_{n}$ de la $n$-\'esima renovaci\'on tiene distribuci\'on $G\star F^{\left(n-1\right)\star}\left(t\right)$
\end{Note}


\begin{Teo}
Para una constante $\mu\leq\infty$ ( o variable aleatoria), las siguientes expresiones son equivalentes:

\begin{eqnarray}
lim_{n\rightarrow\infty}n^{-1}T_{n}&=&\mu,\textrm{ c.s.}\\
lim_{t\rightarrow\infty}t^{-1}N\left(t\right)&=&1/\mu,\textrm{ c.s.}
\end{eqnarray}
\end{Teo}


Es decir, $T_{n}$ satisface la Ley Fuerte de los Grandes N\'umeros s\'i y s\'olo s\'i $N\left/t\right)$ la cumple.


\begin{Coro}[Ley Fuerte de los Grandes N\'umeros para Procesos de Renovaci\'on]
Si $N\left(t\right)$ es un proceso de renovaci\'on cuyos tiempos de inter-renovaci\'on tienen media $\mu\leq\infty$, entonces
\begin{eqnarray}
t^{-1}N\left(t\right)\rightarrow 1/\mu,\textrm{ c.s. cuando }t\rightarrow\infty.
\end{eqnarray}

\end{Coro}


Considerar el proceso estoc\'astico de valores reales $\left\{Z\left(t\right):t\geq0\right\}$ en el mismo espacio de probabilidad que $N\left(t\right)$

\begin{Def}
Para el proceso $\left\{Z\left(t\right):t\geq0\right\}$ se define la fluctuaci\'on m\'axima de $Z\left(t\right)$ en el intervalo $\left(T_{n-1},T_{n}\right]$:
\begin{eqnarray*}
M_{n}=\sup_{T_{n-1}<t\leq T_{n}}|Z\left(t\right)-Z\left(T_{n-1}\right)|
\end{eqnarray*}
\end{Def}

\begin{Teo}
Sup\'ongase que $n^{-1}T_{n}\rightarrow\mu$ c.s. cuando $n\rightarrow\infty$, donde $\mu\leq\infty$ es una constante o variable aleatoria. Sea $a$ una constante o variable aleatoria que puede ser infinita cuando $\mu$ es finita, y considere las expresiones l\'imite:
\begin{eqnarray}
lim_{n\rightarrow\infty}n^{-1}Z\left(T_{n}\right)&=&a,\textrm{ c.s.}\\
lim_{t\rightarrow\infty}t^{-1}Z\left(t\right)&=&a/\mu,\textrm{ c.s.}
\end{eqnarray}
La segunda expresi\'on implica la primera. Conversamente, la primera implica la segunda si el proceso $Z\left(t\right)$ es creciente, o si $lim_{n\rightarrow\infty}n^{-1}M_{n}=0$ c.s.
\end{Teo}

\begin{Coro}
Si $N\left(t\right)$ es un proceso de renovaci\'on, y $\left(Z\left(T_{n}\right)-Z\left(T_{n-1}\right),M_{n}\right)$, para $n\geq1$, son variables aleatorias independientes e id\'enticamente distribuidas con media finita, entonces,
\begin{eqnarray}
lim_{t\rightarrow\infty}t^{-1}Z\left(t\right)\rightarrow\frac{\esp\left[Z\left(T_{1}\right)-Z\left(T_{0}\right)\right]}{\esp\left[T_{1}\right]},\textrm{ c.s. cuando  }t\rightarrow\infty.
\end{eqnarray}
\end{Coro}

%___________________________________________________________________________________________
%
%\subsection{Propiedades de los Procesos de Renovaci\'on}
%___________________________________________________________________________________________
%

Los tiempos $T_{n}$ est\'an relacionados con los conteos de $N\left(t\right)$ por

\begin{eqnarray*}
\left\{N\left(t\right)\geq n\right\}&=&\left\{T_{n}\leq t\right\}\\
T_{N\left(t\right)}\leq &t&<T_{N\left(t\right)+1},
\end{eqnarray*}

adem\'as $N\left(T_{n}\right)=n$, y 

\begin{eqnarray*}
N\left(t\right)=\max\left\{n:T_{n}\leq t\right\}=\min\left\{n:T_{n+1}>t\right\}
\end{eqnarray*}

Por propiedades de la convoluci\'on se sabe que

\begin{eqnarray*}
P\left\{T_{n}\leq t\right\}=F^{n\star}\left(t\right)
\end{eqnarray*}
que es la $n$-\'esima convoluci\'on de $F$. Entonces 

\begin{eqnarray*}
\left\{N\left(t\right)\geq n\right\}&=&\left\{T_{n}\leq t\right\}\\
P\left\{N\left(t\right)\leq n\right\}&=&1-F^{\left(n+1\right)\star}\left(t\right)
\end{eqnarray*}

Adem\'as usando el hecho de que $\esp\left[N\left(t\right)\right]=\sum_{n=1}^{\infty}P\left\{N\left(t\right)\geq n\right\}$
se tiene que

\begin{eqnarray*}
\esp\left[N\left(t\right)\right]=\sum_{n=1}^{\infty}F^{n\star}\left(t\right)
\end{eqnarray*}

\begin{Prop}
Para cada $t\geq0$, la funci\'on generadora de momentos $\esp\left[e^{\alpha N\left(t\right)}\right]$ existe para alguna $\alpha$ en una vecindad del 0, y de aqu\'i que $\esp\left[N\left(t\right)^{m}\right]<\infty$, para $m\geq1$.
\end{Prop}


\begin{Note}
Si el primer tiempo de renovaci\'on $\xi_{1}$ no tiene la misma distribuci\'on que el resto de las $\xi_{n}$, para $n\geq2$, a $N\left(t\right)$ se le llama Proceso de Renovaci\'on retardado, donde si $\xi$ tiene distribuci\'on $G$, entonces el tiempo $T_{n}$ de la $n$-\'esima renovaci\'on tiene distribuci\'on $G\star F^{\left(n-1\right)\star}\left(t\right)$
\end{Note}


\begin{Teo}
Para una constante $\mu\leq\infty$ ( o variable aleatoria), las siguientes expresiones son equivalentes:

\begin{eqnarray}
lim_{n\rightarrow\infty}n^{-1}T_{n}&=&\mu,\textrm{ c.s.}\\
lim_{t\rightarrow\infty}t^{-1}N\left(t\right)&=&1/\mu,\textrm{ c.s.}
\end{eqnarray}
\end{Teo}


Es decir, $T_{n}$ satisface la Ley Fuerte de los Grandes N\'umeros s\'i y s\'olo s\'i $N\left/t\right)$ la cumple.


\begin{Coro}[Ley Fuerte de los Grandes N\'umeros para Procesos de Renovaci\'on]
Si $N\left(t\right)$ es un proceso de renovaci\'on cuyos tiempos de inter-renovaci\'on tienen media $\mu\leq\infty$, entonces
\begin{eqnarray}
t^{-1}N\left(t\right)\rightarrow 1/\mu,\textrm{ c.s. cuando }t\rightarrow\infty.
\end{eqnarray}

\end{Coro}


Considerar el proceso estoc\'astico de valores reales $\left\{Z\left(t\right):t\geq0\right\}$ en el mismo espacio de probabilidad que $N\left(t\right)$

\begin{Def}
Para el proceso $\left\{Z\left(t\right):t\geq0\right\}$ se define la fluctuaci\'on m\'axima de $Z\left(t\right)$ en el intervalo $\left(T_{n-1},T_{n}\right]$:
\begin{eqnarray*}
M_{n}=\sup_{T_{n-1}<t\leq T_{n}}|Z\left(t\right)-Z\left(T_{n-1}\right)|
\end{eqnarray*}
\end{Def}

\begin{Teo}
Sup\'ongase que $n^{-1}T_{n}\rightarrow\mu$ c.s. cuando $n\rightarrow\infty$, donde $\mu\leq\infty$ es una constante o variable aleatoria. Sea $a$ una constante o variable aleatoria que puede ser infinita cuando $\mu$ es finita, y considere las expresiones l\'imite:
\begin{eqnarray}
lim_{n\rightarrow\infty}n^{-1}Z\left(T_{n}\right)&=&a,\textrm{ c.s.}\\
lim_{t\rightarrow\infty}t^{-1}Z\left(t\right)&=&a/\mu,\textrm{ c.s.}
\end{eqnarray}
La segunda expresi\'on implica la primera. Conversamente, la primera implica la segunda si el proceso $Z\left(t\right)$ es creciente, o si $lim_{n\rightarrow\infty}n^{-1}M_{n}=0$ c.s.
\end{Teo}

\begin{Coro}
Si $N\left(t\right)$ es un proceso de renovaci\'on, y $\left(Z\left(T_{n}\right)-Z\left(T_{n-1}\right),M_{n}\right)$, para $n\geq1$, son variables aleatorias independientes e id\'enticamente distribuidas con media finita, entonces,
\begin{eqnarray}
lim_{t\rightarrow\infty}t^{-1}Z\left(t\right)\rightarrow\frac{\esp\left[Z\left(T_{1}\right)-Z\left(T_{0}\right)\right]}{\esp\left[T_{1}\right]},\textrm{ c.s. cuando  }t\rightarrow\infty.
\end{eqnarray}
\end{Coro}



%___________________________________________________________________________________________
%
\subsection{Propiedades de los Procesos de Renovaci\'on}
%___________________________________________________________________________________________
%

Los tiempos $T_{n}$ est\'an relacionados con los conteos de $N\left(t\right)$ por

\begin{eqnarray*}
\left\{N\left(t\right)\geq n\right\}&=&\left\{T_{n}\leq t\right\}\\
T_{N\left(t\right)}\leq &t&<T_{N\left(t\right)+1},
\end{eqnarray*}

adem\'as $N\left(T_{n}\right)=n$, y 

\begin{eqnarray*}
N\left(t\right)=\max\left\{n:T_{n}\leq t\right\}=\min\left\{n:T_{n+1}>t\right\}
\end{eqnarray*}

Por propiedades de la convoluci\'on se sabe que

\begin{eqnarray*}
P\left\{T_{n}\leq t\right\}=F^{n\star}\left(t\right)
\end{eqnarray*}
que es la $n$-\'esima convoluci\'on de $F$. Entonces 

\begin{eqnarray*}
\left\{N\left(t\right)\geq n\right\}&=&\left\{T_{n}\leq t\right\}\\
P\left\{N\left(t\right)\leq n\right\}&=&1-F^{\left(n+1\right)\star}\left(t\right)
\end{eqnarray*}

Adem\'as usando el hecho de que $\esp\left[N\left(t\right)\right]=\sum_{n=1}^{\infty}P\left\{N\left(t\right)\geq n\right\}$
se tiene que

\begin{eqnarray*}
\esp\left[N\left(t\right)\right]=\sum_{n=1}^{\infty}F^{n\star}\left(t\right)
\end{eqnarray*}

\begin{Prop}
Para cada $t\geq0$, la funci\'on generadora de momentos $\esp\left[e^{\alpha N\left(t\right)}\right]$ existe para alguna $\alpha$ en una vecindad del 0, y de aqu\'i que $\esp\left[N\left(t\right)^{m}\right]<\infty$, para $m\geq1$.
\end{Prop}


\begin{Note}
Si el primer tiempo de renovaci\'on $\xi_{1}$ no tiene la misma distribuci\'on que el resto de las $\xi_{n}$, para $n\geq2$, a $N\left(t\right)$ se le llama Proceso de Renovaci\'on retardado, donde si $\xi$ tiene distribuci\'on $G$, entonces el tiempo $T_{n}$ de la $n$-\'esima renovaci\'on tiene distribuci\'on $G\star F^{\left(n-1\right)\star}\left(t\right)$
\end{Note}


\begin{Teo}
Para una constante $\mu\leq\infty$ ( o variable aleatoria), las siguientes expresiones son equivalentes:

\begin{eqnarray}
lim_{n\rightarrow\infty}n^{-1}T_{n}&=&\mu,\textrm{ c.s.}\\
lim_{t\rightarrow\infty}t^{-1}N\left(t\right)&=&1/\mu,\textrm{ c.s.}
\end{eqnarray}
\end{Teo}


Es decir, $T_{n}$ satisface la Ley Fuerte de los Grandes N\'umeros s\'i y s\'olo s\'i $N\left/t\right)$ la cumple.


\begin{Coro}[Ley Fuerte de los Grandes N\'umeros para Procesos de Renovaci\'on]
Si $N\left(t\right)$ es un proceso de renovaci\'on cuyos tiempos de inter-renovaci\'on tienen media $\mu\leq\infty$, entonces
\begin{eqnarray}
t^{-1}N\left(t\right)\rightarrow 1/\mu,\textrm{ c.s. cuando }t\rightarrow\infty.
\end{eqnarray}

\end{Coro}


Considerar el proceso estoc\'astico de valores reales $\left\{Z\left(t\right):t\geq0\right\}$ en el mismo espacio de probabilidad que $N\left(t\right)$

\begin{Def}
Para el proceso $\left\{Z\left(t\right):t\geq0\right\}$ se define la fluctuaci\'on m\'axima de $Z\left(t\right)$ en el intervalo $\left(T_{n-1},T_{n}\right]$:
\begin{eqnarray*}
M_{n}=\sup_{T_{n-1}<t\leq T_{n}}|Z\left(t\right)-Z\left(T_{n-1}\right)|
\end{eqnarray*}
\end{Def}

\begin{Teo}
Sup\'ongase que $n^{-1}T_{n}\rightarrow\mu$ c.s. cuando $n\rightarrow\infty$, donde $\mu\leq\infty$ es una constante o variable aleatoria. Sea $a$ una constante o variable aleatoria que puede ser infinita cuando $\mu$ es finita, y considere las expresiones l\'imite:
\begin{eqnarray}
lim_{n\rightarrow\infty}n^{-1}Z\left(T_{n}\right)&=&a,\textrm{ c.s.}\\
lim_{t\rightarrow\infty}t^{-1}Z\left(t\right)&=&a/\mu,\textrm{ c.s.}
\end{eqnarray}
La segunda expresi\'on implica la primera. Conversamente, la primera implica la segunda si el proceso $Z\left(t\right)$ es creciente, o si $lim_{n\rightarrow\infty}n^{-1}M_{n}=0$ c.s.
\end{Teo}

\begin{Coro}
Si $N\left(t\right)$ es un proceso de renovaci\'on, y $\left(Z\left(T_{n}\right)-Z\left(T_{n-1}\right),M_{n}\right)$, para $n\geq1$, son variables aleatorias independientes e id\'enticamente distribuidas con media finita, entonces,
\begin{eqnarray}
lim_{t\rightarrow\infty}t^{-1}Z\left(t\right)\rightarrow\frac{\esp\left[Z\left(T_{1}\right)-Z\left(T_{0}\right)\right]}{\esp\left[T_{1}\right]},\textrm{ c.s. cuando  }t\rightarrow\infty.
\end{eqnarray}
\end{Coro}




%__________________________________________________________________________________________
\subsection{Procesos Regenerativos Estacionarios - Stidham \cite{Stidham}}
%__________________________________________________________________________________________


Un proceso estoc\'astico a tiempo continuo $\left\{V\left(t\right),t\geq0\right\}$ es un proceso regenerativo si existe una sucesi\'on de variables aleatorias independientes e id\'enticamente distribuidas $\left\{X_{1},X_{2},\ldots\right\}$, sucesi\'on de renovaci\'on, tal que para cualquier conjunto de Borel $A$, 

\begin{eqnarray*}
\prob\left\{V\left(t\right)\in A|X_{1}+X_{2}+\cdots+X_{R\left(t\right)}=s,\left\{V\left(\tau\right),\tau<s\right\}\right\}=\prob\left\{V\left(t-s\right)\in A|X_{1}>t-s\right\},
\end{eqnarray*}
para todo $0\leq s\leq t$, donde $R\left(t\right)=\max\left\{X_{1}+X_{2}+\cdots+X_{j}\leq t\right\}=$n\'umero de renovaciones ({\emph{puntos de regeneraci\'on}}) que ocurren en $\left[0,t\right]$. El intervalo $\left[0,X_{1}\right)$ es llamado {\emph{primer ciclo de regeneraci\'on}} de $\left\{V\left(t \right),t\geq0\right\}$, $\left[X_{1},X_{1}+X_{2}\right)$ el {\emph{segundo ciclo de regeneraci\'on}}, y as\'i sucesivamente.

Sea $X=X_{1}$ y sea $F$ la funci\'on de distrbuci\'on de $X$


\begin{Def}
Se define el proceso estacionario, $\left\{V^{*}\left(t\right),t\geq0\right\}$, para $\left\{V\left(t\right),t\geq0\right\}$ por

\begin{eqnarray*}
\prob\left\{V\left(t\right)\in A\right\}=\frac{1}{\esp\left[X\right]}\int_{0}^{\infty}\prob\left\{V\left(t+x\right)\in A|X>x\right\}\left(1-F\left(x\right)\right)dx,
\end{eqnarray*} 
para todo $t\geq0$ y todo conjunto de Borel $A$.
\end{Def}

\begin{Def}
Una distribuci\'on se dice que es {\emph{aritm\'etica}} si todos sus puntos de incremento son m\'ultiplos de la forma $0,\lambda, 2\lambda,\ldots$ para alguna $\lambda>0$ entera.
\end{Def}


\begin{Def}
Una modificaci\'on medible de un proceso $\left\{V\left(t\right),t\geq0\right\}$, es una versi\'on de este, $\left\{V\left(t,w\right)\right\}$ conjuntamente medible para $t\geq0$ y para $w\in S$, $S$ espacio de estados para $\left\{V\left(t\right),t\geq0\right\}$.
\end{Def}

\begin{Teo}
Sea $\left\{V\left(t\right),t\geq\right\}$ un proceso regenerativo no negativo con modificaci\'on medible. Sea $\esp\left[X\right]<\infty$. Entonces el proceso estacionario dado por la ecuaci\'on anterior est\'a bien definido y tiene funci\'on de distribuci\'on independiente de $t$, adem\'as
\begin{itemize}
\item[i)] \begin{eqnarray*}
\esp\left[V^{*}\left(0\right)\right]&=&\frac{\esp\left[\int_{0}^{X}V\left(s\right)ds\right]}{\esp\left[X\right]}\end{eqnarray*}
\item[ii)] Si $\esp\left[V^{*}\left(0\right)\right]<\infty$, equivalentemente, si $\esp\left[\int_{0}^{X}V\left(s\right)ds\right]<\infty$,entonces
\begin{eqnarray*}
\frac{\int_{0}^{t}V\left(s\right)ds}{t}\rightarrow\frac{\esp\left[\int_{0}^{X}V\left(s\right)ds\right]}{\esp\left[X\right]}
\end{eqnarray*}
con probabilidad 1 y en media, cuando $t\rightarrow\infty$.
\end{itemize}
\end{Teo}

%______________________________________________________________________
\subsection{Procesos de Renovaci\'on}
%______________________________________________________________________

\begin{Def}\label{Def.Tn}
Sean $0\leq T_{1}\leq T_{2}\leq \ldots$ son tiempos aleatorios infinitos en los cuales ocurren ciertos eventos. El n\'umero de tiempos $T_{n}$ en el intervalo $\left[0,t\right)$ es

\begin{eqnarray}
N\left(t\right)=\sum_{n=1}^{\infty}\indora\left(T_{n}\leq t\right),
\end{eqnarray}
para $t\geq0$.
\end{Def}

Si se consideran los puntos $T_{n}$ como elementos de $\rea_{+}$, y $N\left(t\right)$ es el n\'umero de puntos en $\rea$. El proceso denotado por $\left\{N\left(t\right):t\geq0\right\}$, denotado por $N\left(t\right)$, es un proceso puntual en $\rea_{+}$. Los $T_{n}$ son los tiempos de ocurrencia, el proceso puntual $N\left(t\right)$ es simple si su n\'umero de ocurrencias son distintas: $0<T_{1}<T_{2}<\ldots$ casi seguramente.

\begin{Def}
Un proceso puntual $N\left(t\right)$ es un proceso de renovaci\'on si los tiempos de interocurrencia $\xi_{n}=T_{n}-T_{n-1}$, para $n\geq1$, son independientes e identicamente distribuidos con distribuci\'on $F$, donde $F\left(0\right)=0$ y $T_{0}=0$. Los $T_{n}$ son llamados tiempos de renovaci\'on, referente a la independencia o renovaci\'on de la informaci\'on estoc\'astica en estos tiempos. Los $\xi_{n}$ son los tiempos de inter-renovaci\'on, y $N\left(t\right)$ es el n\'umero de renovaciones en el intervalo $\left[0,t\right)$
\end{Def}


\begin{Note}
Para definir un proceso de renovaci\'on para cualquier contexto, solamente hay que especificar una distribuci\'on $F$, con $F\left(0\right)=0$, para los tiempos de inter-renovaci\'on. La funci\'on $F$ en turno degune las otra variables aleatorias. De manera formal, existe un espacio de probabilidad y una sucesi\'on de variables aleatorias $\xi_{1},\xi_{2},\ldots$ definidas en este con distribuci\'on $F$. Entonces las otras cantidades son $T_{n}=\sum_{k=1}^{n}\xi_{k}$ y $N\left(t\right)=\sum_{n=1}^{\infty}\indora\left(T_{n}\leq t\right)$, donde $T_{n}\rightarrow\infty$ casi seguramente por la Ley Fuerte de los Grandes Números.
\end{Note}

%___________________________________________________________________________________________
%
\subsection{Teorema Principal de Renovaci\'on}
%___________________________________________________________________________________________
%

\begin{Note} Una funci\'on $h:\rea_{+}\rightarrow\rea$ es Directamente Riemann Integrable en los siguientes casos:
\begin{itemize}
\item[a)] $h\left(t\right)\geq0$ es decreciente y Riemann Integrable.
\item[b)] $h$ es continua excepto posiblemente en un conjunto de Lebesgue de medida 0, y $|h\left(t\right)|\leq b\left(t\right)$, donde $b$ es DRI.
\end{itemize}
\end{Note}

\begin{Teo}[Teorema Principal de Renovaci\'on]
Si $F$ es no aritm\'etica y $h\left(t\right)$ es Directamente Riemann Integrable (DRI), entonces

\begin{eqnarray*}
lim_{t\rightarrow\infty}U\star h=\frac{1}{\mu}\int_{\rea_{+}}h\left(s\right)ds.
\end{eqnarray*}
\end{Teo}

\begin{Prop}
Cualquier funci\'on $H\left(t\right)$ acotada en intervalos finitos y que es 0 para $t<0$ puede expresarse como
\begin{eqnarray*}
H\left(t\right)=U\star h\left(t\right)\textrm{,  donde }h\left(t\right)=H\left(t\right)-F\star H\left(t\right)
\end{eqnarray*}
\end{Prop}

\begin{Def}
Un proceso estoc\'astico $X\left(t\right)$ es crudamente regenerativo en un tiempo aleatorio positivo $T$ si
\begin{eqnarray*}
\esp\left[X\left(T+t\right)|T\right]=\esp\left[X\left(t\right)\right]\textrm{, para }t\geq0,\end{eqnarray*}
y con las esperanzas anteriores finitas.
\end{Def}

\begin{Prop}
Sup\'ongase que $X\left(t\right)$ es un proceso crudamente regenerativo en $T$, que tiene distribuci\'on $F$. Si $\esp\left[X\left(t\right)\right]$ es acotado en intervalos finitos, entonces
\begin{eqnarray*}
\esp\left[X\left(t\right)\right]=U\star h\left(t\right)\textrm{,  donde }h\left(t\right)=\esp\left[X\left(t\right)\indora\left(T>t\right)\right].
\end{eqnarray*}
\end{Prop}

\begin{Teo}[Regeneraci\'on Cruda]
Sup\'ongase que $X\left(t\right)$ es un proceso con valores positivo crudamente regenerativo en $T$, y def\'inase $M=\sup\left\{|X\left(t\right)|:t\leq T\right\}$. Si $T$ es no aritm\'etico y $M$ y $MT$ tienen media finita, entonces
\begin{eqnarray*}
lim_{t\rightarrow\infty}\esp\left[X\left(t\right)\right]=\frac{1}{\mu}\int_{\rea_{+}}h\left(s\right)ds,
\end{eqnarray*}
donde $h\left(t\right)=\esp\left[X\left(t\right)\indora\left(T>t\right)\right]$.
\end{Teo}



%___________________________________________________________________________________________
%
\subsection{Funci\'on de Renovaci\'on}
%___________________________________________________________________________________________
%


\begin{Def}
Sea $h\left(t\right)$ funci\'on de valores reales en $\rea$ acotada en intervalos finitos e igual a cero para $t<0$ La ecuaci\'on de renovaci\'on para $h\left(t\right)$ y la distribuci\'on $F$ es

\begin{eqnarray}\label{Ec.Renovacion}
H\left(t\right)=h\left(t\right)+\int_{\left[0,t\right]}H\left(t-s\right)dF\left(s\right)\textrm{,    }t\geq0,
\end{eqnarray}
donde $H\left(t\right)$ es una funci\'on de valores reales. Esto es $H=h+F\star H$. Decimos que $H\left(t\right)$ es soluci\'on de esta ecuaci\'on si satisface la ecuaci\'on, y es acotada en intervalos finitos e iguales a cero para $t<0$.
\end{Def}

\begin{Prop}
La funci\'on $U\star h\left(t\right)$ es la \'unica soluci\'on de la ecuaci\'on de renovaci\'on (\ref{Ec.Renovacion}).
\end{Prop}

\begin{Teo}[Teorema Renovaci\'on Elemental]
\begin{eqnarray*}
t^{-1}U\left(t\right)\rightarrow 1/\mu\textrm{,    cuando }t\rightarrow\infty.
\end{eqnarray*}
\end{Teo}

%___________________________________________________________________________________________
%
\subsection{Propiedades de los Procesos de Renovaci\'on}
%___________________________________________________________________________________________
%

Los tiempos $T_{n}$ est\'an relacionados con los conteos de $N\left(t\right)$ por

\begin{eqnarray*}
\left\{N\left(t\right)\geq n\right\}&=&\left\{T_{n}\leq t\right\}\\
T_{N\left(t\right)}\leq &t&<T_{N\left(t\right)+1},
\end{eqnarray*}

adem\'as $N\left(T_{n}\right)=n$, y 

\begin{eqnarray*}
N\left(t\right)=\max\left\{n:T_{n}\leq t\right\}=\min\left\{n:T_{n+1}>t\right\}
\end{eqnarray*}

Por propiedades de la convoluci\'on se sabe que

\begin{eqnarray*}
P\left\{T_{n}\leq t\right\}=F^{n\star}\left(t\right)
\end{eqnarray*}
que es la $n$-\'esima convoluci\'on de $F$. Entonces 

\begin{eqnarray*}
\left\{N\left(t\right)\geq n\right\}&=&\left\{T_{n}\leq t\right\}\\
P\left\{N\left(t\right)\leq n\right\}&=&1-F^{\left(n+1\right)\star}\left(t\right)
\end{eqnarray*}

Adem\'as usando el hecho de que $\esp\left[N\left(t\right)\right]=\sum_{n=1}^{\infty}P\left\{N\left(t\right)\geq n\right\}$
se tiene que

\begin{eqnarray*}
\esp\left[N\left(t\right)\right]=\sum_{n=1}^{\infty}F^{n\star}\left(t\right)
\end{eqnarray*}

\begin{Prop}
Para cada $t\geq0$, la funci\'on generadora de momentos $\esp\left[e^{\alpha N\left(t\right)}\right]$ existe para alguna $\alpha$ en una vecindad del 0, y de aqu\'i que $\esp\left[N\left(t\right)^{m}\right]<\infty$, para $m\geq1$.
\end{Prop}


\begin{Note}
Si el primer tiempo de renovaci\'on $\xi_{1}$ no tiene la misma distribuci\'on que el resto de las $\xi_{n}$, para $n\geq2$, a $N\left(t\right)$ se le llama Proceso de Renovaci\'on retardado, donde si $\xi$ tiene distribuci\'on $G$, entonces el tiempo $T_{n}$ de la $n$-\'esima renovaci\'on tiene distribuci\'on $G\star F^{\left(n-1\right)\star}\left(t\right)$
\end{Note}


\begin{Teo}
Para una constante $\mu\leq\infty$ ( o variable aleatoria), las siguientes expresiones son equivalentes:

\begin{eqnarray}
lim_{n\rightarrow\infty}n^{-1}T_{n}&=&\mu,\textrm{ c.s.}\\
lim_{t\rightarrow\infty}t^{-1}N\left(t\right)&=&1/\mu,\textrm{ c.s.}
\end{eqnarray}
\end{Teo}


Es decir, $T_{n}$ satisface la Ley Fuerte de los Grandes N\'umeros s\'i y s\'olo s\'i $N\left/t\right)$ la cumple.


\begin{Coro}[Ley Fuerte de los Grandes N\'umeros para Procesos de Renovaci\'on]
Si $N\left(t\right)$ es un proceso de renovaci\'on cuyos tiempos de inter-renovaci\'on tienen media $\mu\leq\infty$, entonces
\begin{eqnarray}
t^{-1}N\left(t\right)\rightarrow 1/\mu,\textrm{ c.s. cuando }t\rightarrow\infty.
\end{eqnarray}

\end{Coro}


Considerar el proceso estoc\'astico de valores reales $\left\{Z\left(t\right):t\geq0\right\}$ en el mismo espacio de probabilidad que $N\left(t\right)$

\begin{Def}
Para el proceso $\left\{Z\left(t\right):t\geq0\right\}$ se define la fluctuaci\'on m\'axima de $Z\left(t\right)$ en el intervalo $\left(T_{n-1},T_{n}\right]$:
\begin{eqnarray*}
M_{n}=\sup_{T_{n-1}<t\leq T_{n}}|Z\left(t\right)-Z\left(T_{n-1}\right)|
\end{eqnarray*}
\end{Def}

\begin{Teo}
Sup\'ongase que $n^{-1}T_{n}\rightarrow\mu$ c.s. cuando $n\rightarrow\infty$, donde $\mu\leq\infty$ es una constante o variable aleatoria. Sea $a$ una constante o variable aleatoria que puede ser infinita cuando $\mu$ es finita, y considere las expresiones l\'imite:
\begin{eqnarray}
lim_{n\rightarrow\infty}n^{-1}Z\left(T_{n}\right)&=&a,\textrm{ c.s.}\\
lim_{t\rightarrow\infty}t^{-1}Z\left(t\right)&=&a/\mu,\textrm{ c.s.}
\end{eqnarray}
La segunda expresi\'on implica la primera. Conversamente, la primera implica la segunda si el proceso $Z\left(t\right)$ es creciente, o si $lim_{n\rightarrow\infty}n^{-1}M_{n}=0$ c.s.
\end{Teo}

\begin{Coro}
Si $N\left(t\right)$ es un proceso de renovaci\'on, y $\left(Z\left(T_{n}\right)-Z\left(T_{n-1}\right),M_{n}\right)$, para $n\geq1$, son variables aleatorias independientes e id\'enticamente distribuidas con media finita, entonces,
\begin{eqnarray}
lim_{t\rightarrow\infty}t^{-1}Z\left(t\right)\rightarrow\frac{\esp\left[Z\left(T_{1}\right)-Z\left(T_{0}\right)\right]}{\esp\left[T_{1}\right]},\textrm{ c.s. cuando  }t\rightarrow\infty.
\end{eqnarray}
\end{Coro}

%___________________________________________________________________________________________
%
\subsection{Funci\'on de Renovaci\'on}
%___________________________________________________________________________________________
%


Sup\'ongase que $N\left(t\right)$ es un proceso de renovaci\'on con distribuci\'on $F$ con media finita $\mu$.

\begin{Def}
La funci\'on de renovaci\'on asociada con la distribuci\'on $F$, del proceso $N\left(t\right)$, es
\begin{eqnarray*}
U\left(t\right)=\sum_{n=1}^{\infty}F^{n\star}\left(t\right),\textrm{   }t\geq0,
\end{eqnarray*}
donde $F^{0\star}\left(t\right)=\indora\left(t\geq0\right)$.
\end{Def}


\begin{Prop}
Sup\'ongase que la distribuci\'on de inter-renovaci\'on $F$ tiene densidad $f$. Entonces $U\left(t\right)$ tambi\'en tiene densidad, para $t>0$, y es $U^{'}\left(t\right)=\sum_{n=0}^{\infty}f^{n\star}\left(t\right)$. Adem\'as
\begin{eqnarray*}
\prob\left\{N\left(t\right)>N\left(t-\right)\right\}=0\textrm{,   }t\geq0.
\end{eqnarray*}
\end{Prop}

\begin{Def}
La Transformada de Laplace-Stieljes de $F$ est\'a dada por

\begin{eqnarray*}
\hat{F}\left(\alpha\right)=\int_{\rea_{+}}e^{-\alpha t}dF\left(t\right)\textrm{,  }\alpha\geq0.
\end{eqnarray*}
\end{Def}

Entonces

\begin{eqnarray*}
\hat{U}\left(\alpha\right)=\sum_{n=0}^{\infty}\hat{F^{n\star}}\left(\alpha\right)=\sum_{n=0}^{\infty}\hat{F}\left(\alpha\right)^{n}=\frac{1}{1-\hat{F}\left(\alpha\right)}.
\end{eqnarray*}


\begin{Prop}
La Transformada de Laplace $\hat{U}\left(\alpha\right)$ y $\hat{F}\left(\alpha\right)$ determina una a la otra de manera \'unica por la relaci\'on $\hat{U}\left(\alpha\right)=\frac{1}{1-\hat{F}\left(\alpha\right)}$.
\end{Prop}


\begin{Note}
Un proceso de renovaci\'on $N\left(t\right)$ cuyos tiempos de inter-renovaci\'on tienen media finita, es un proceso Poisson con tasa $\lambda$ si y s\'olo s\'i $\esp\left[U\left(t\right)\right]=\lambda t$, para $t\geq0$.
\end{Note}


\begin{Teo}
Sea $N\left(t\right)$ un proceso puntual simple con puntos de localizaci\'on $T_{n}$ tal que $\eta\left(t\right)=\esp\left[N\left(\right)\right]$ es finita para cada $t$. Entonces para cualquier funci\'on $f:\rea_{+}\rightarrow\rea$,
\begin{eqnarray*}
\esp\left[\sum_{n=1}^{N\left(\right)}f\left(T_{n}\right)\right]=\int_{\left(0,t\right]}f\left(s\right)d\eta\left(s\right)\textrm{,  }t\geq0,
\end{eqnarray*}
suponiendo que la integral exista. Adem\'as si $X_{1},X_{2},\ldots$ son variables aleatorias definidas en el mismo espacio de probabilidad que el proceso $N\left(t\right)$ tal que $\esp\left[X_{n}|T_{n}=s\right]=f\left(s\right)$, independiente de $n$. Entonces
\begin{eqnarray*}
\esp\left[\sum_{n=1}^{N\left(t\right)}X_{n}\right]=\int_{\left(0,t\right]}f\left(s\right)d\eta\left(s\right)\textrm{,  }t\geq0,
\end{eqnarray*} 
suponiendo que la integral exista. 
\end{Teo}

\begin{Coro}[Identidad de Wald para Renovaciones]
Para el proceso de renovaci\'on $N\left(t\right)$,
\begin{eqnarray*}
\esp\left[T_{N\left(t\right)+1}\right]=\mu\esp\left[N\left(t\right)+1\right]\textrm{,  }t\geq0,
\end{eqnarray*}  
\end{Coro}

%______________________________________________________________________
\subsection{Procesos de Renovaci\'on}
%______________________________________________________________________

\begin{Def}\label{Def.Tn}
Sean $0\leq T_{1}\leq T_{2}\leq \ldots$ son tiempos aleatorios infinitos en los cuales ocurren ciertos eventos. El n\'umero de tiempos $T_{n}$ en el intervalo $\left[0,t\right)$ es

\begin{eqnarray}
N\left(t\right)=\sum_{n=1}^{\infty}\indora\left(T_{n}\leq t\right),
\end{eqnarray}
para $t\geq0$.
\end{Def}

Si se consideran los puntos $T_{n}$ como elementos de $\rea_{+}$, y $N\left(t\right)$ es el n\'umero de puntos en $\rea$. El proceso denotado por $\left\{N\left(t\right):t\geq0\right\}$, denotado por $N\left(t\right)$, es un proceso puntual en $\rea_{+}$. Los $T_{n}$ son los tiempos de ocurrencia, el proceso puntual $N\left(t\right)$ es simple si su n\'umero de ocurrencias son distintas: $0<T_{1}<T_{2}<\ldots$ casi seguramente.

\begin{Def}
Un proceso puntual $N\left(t\right)$ es un proceso de renovaci\'on si los tiempos de interocurrencia $\xi_{n}=T_{n}-T_{n-1}$, para $n\geq1$, son independientes e identicamente distribuidos con distribuci\'on $F$, donde $F\left(0\right)=0$ y $T_{0}=0$. Los $T_{n}$ son llamados tiempos de renovaci\'on, referente a la independencia o renovaci\'on de la informaci\'on estoc\'astica en estos tiempos. Los $\xi_{n}$ son los tiempos de inter-renovaci\'on, y $N\left(t\right)$ es el n\'umero de renovaciones en el intervalo $\left[0,t\right)$
\end{Def}


\begin{Note}
Para definir un proceso de renovaci\'on para cualquier contexto, solamente hay que especificar una distribuci\'on $F$, con $F\left(0\right)=0$, para los tiempos de inter-renovaci\'on. La funci\'on $F$ en turno degune las otra variables aleatorias. De manera formal, existe un espacio de probabilidad y una sucesi\'on de variables aleatorias $\xi_{1},\xi_{2},\ldots$ definidas en este con distribuci\'on $F$. Entonces las otras cantidades son $T_{n}=\sum_{k=1}^{n}\xi_{k}$ y $N\left(t\right)=\sum_{n=1}^{\infty}\indora\left(T_{n}\leq t\right)$, donde $T_{n}\rightarrow\infty$ casi seguramente por la Ley Fuerte de los Grandes Números.
\end{Note}
%_____________________________________________________
\subsection{Puntos de Renovaci\'on}
%_____________________________________________________

Para cada cola $Q_{i}$ se tienen los procesos de arribo a la cola, para estas, los tiempos de arribo est\'an dados por $$\left\{T_{1}^{i},T_{2}^{i},\ldots,T_{k}^{i},\ldots\right\},$$ entonces, consideremos solamente los primeros tiempos de arribo a cada una de las colas, es decir, $$\left\{T_{1}^{1},T_{1}^{2},T_{1}^{3},T_{1}^{4}\right\},$$ se sabe que cada uno de estos tiempos se distribuye de manera exponencial con par\'ametro $1/mu_{i}$. Adem\'as se sabe que para $$T^{*}=\min\left\{T_{1}^{1},T_{1}^{2},T_{1}^{3},T_{1}^{4}\right\},$$ $T^{*}$ se distribuye de manera exponencial con par\'ametro $$\mu^{*}=\sum_{i=1}^{4}\mu_{i}.$$ Ahora, dado que 
\begin{center}
\begin{tabular}{lcl}
$\tilde{r}=r_{1}+r_{2}$ & y &$\hat{r}=r_{3}+r_{4}.$
\end{tabular}
\end{center}


Supongamos que $$\tilde{r},\hat{r}<\mu^{*},$$ entonces si tomamos $$r^{*}=\min\left\{\tilde{r},\hat{r}\right\},$$ se tiene que para  $$t^{*}\in\left(0,r^{*}\right)$$ se cumple que 
\begin{center}
\begin{tabular}{lcl}
$\tau_{1}\left(1\right)=0$ & y por tanto & $\overline{\tau}_{1}=0,$
\end{tabular}
\end{center}
entonces para la segunda cola en este primer ciclo se cumple que $$\tau_{2}=\overline{\tau}_{1}+r_{1}=r_{1}<\mu^{*},$$ y por tanto se tiene que  $$\overline{\tau}_{2}=\tau_{2}.$$ Por lo tanto, nuevamente para la primer cola en el segundo ciclo $$\tau_{1}\left(2\right)=\tau_{2}\left(1\right)+r_{2}=\tilde{r}<\mu^{*}.$$ An\'alogamente para el segundo sistema se tiene que ambas colas est\'an vac\'ias, es decir, existe un valor $t^{*}$ tal que en el intervalo $\left(0,t^{*}\right)$ no ha llegado ning\'un usuario, es decir, $$L_{i}\left(t^{*}\right)=0$$ para $i=1,2,3,4$.

\subsection{Resultados para Procesos de Salida}

En \cite{Sigman2} prueban que para la existencia de un una sucesi\'on infinita no decreciente de tiempos de regeneraci\'on $\tau_{1}\leq\tau_{2}\leq\cdots$ en los cuales el proceso se regenera, basta un tiempo de regeneraci\'on $R_{1}$, donde $R_{j}=\tau_{j}-\tau_{j-1}$. Para tal efecto se requiere la existencia de un espacio de probabilidad $\left(\Omega,\mathcal{F},\prob\right)$, y proceso estoc\'astico $\textit{X}=\left\{X\left(t\right):t\geq0\right\}$ con espacio de estados $\left(S,\mathcal{R}\right)$, con $\mathcal{R}$ $\sigma$-\'algebra.

\begin{Prop}
Si existe una variable aleatoria no negativa $R_{1}$ tal que $\theta_{R\footnotesize{1}}X=_{D}X$, entonces $\left(\Omega,\mathcal{F},\prob\right)$ puede extenderse para soportar una sucesi\'on estacionaria de variables aleatorias $R=\left\{R_{k}:k\geq1\right\}$, tal que para $k\geq1$,
\begin{eqnarray*}
\theta_{k}\left(X,R\right)=_{D}\left(X,R\right).
\end{eqnarray*}

Adem\'as, para $k\geq1$, $\theta_{k}R$ es condicionalmente independiente de $\left(X,R_{1},\ldots,R_{k}\right)$, dado $\theta_{\tau k}X$.

\end{Prop}


\begin{itemize}
\item Doob en 1953 demostr\'o que el estado estacionario de un proceso de partida en un sistema de espera $M/G/\infty$, es Poisson con la misma tasa que el proceso de arribos.

\item Burke en 1968, fue el primero en demostrar que el estado estacionario de un proceso de salida de una cola $M/M/s$ es un proceso Poisson.

\item Disney en 1973 obtuvo el siguiente resultado:

\begin{Teo}
Para el sistema de espera $M/G/1/L$ con disciplina FIFO, el proceso $\textbf{I}$ es un proceso de renovaci\'on si y s\'olo si el proceso denominado longitud de la cola es estacionario y se cumple cualquiera de los siguientes casos:

\begin{itemize}
\item[a)] Los tiempos de servicio son identicamente cero;
\item[b)] $L=0$, para cualquier proceso de servicio $S$;
\item[c)] $L=1$ y $G=D$;
\item[d)] $L=\infty$ y $G=M$.
\end{itemize}
En estos casos, respectivamente, las distribuciones de interpartida $P\left\{T_{n+1}-T_{n}\leq t\right\}$ son


\begin{itemize}
\item[a)] $1-e^{-\lambda t}$, $t\geq0$;
\item[b)] $1-e^{-\lambda t}*F\left(t\right)$, $t\geq0$;
\item[c)] $1-e^{-\lambda t}*\indora_{d}\left(t\right)$, $t\geq0$;
\item[d)] $1-e^{-\lambda t}*F\left(t\right)$, $t\geq0$.
\end{itemize}
\end{Teo}


\item Finch (1959) mostr\'o que para los sistemas $M/G/1/L$, con $1\leq L\leq \infty$ con distribuciones de servicio dos veces diferenciable, solamente el sistema $M/M/1/\infty$ tiene proceso de salida de renovaci\'on estacionario.

\item King (1971) demostro que un sistema de colas estacionario $M/G/1/1$ tiene sus tiempos de interpartida sucesivas $D_{n}$ y $D_{n+1}$ son independientes, si y s\'olo si, $G=D$, en cuyo caso le proceso de salida es de renovaci\'on.

\item Disney (1973) demostr\'o que el \'unico sistema estacionario $M/G/1/L$, que tiene proceso de salida de renovaci\'on  son los sistemas $M/M/1$ y $M/D/1/1$.



\item El siguiente resultado es de Disney y Koning (1985)
\begin{Teo}
En un sistema de espera $M/G/s$, el estado estacionario del proceso de salida es un proceso Poisson para cualquier distribuci\'on de los tiempos de servicio si el sistema tiene cualquiera de las siguientes cuatro propiedades.

\begin{itemize}
\item[a)] $s=\infty$
\item[b)] La disciplina de servicio es de procesador compartido.
\item[c)] La disciplina de servicio es LCFS y preemptive resume, esto se cumple para $L<\infty$
\item[d)] $G=M$.
\end{itemize}

\end{Teo}

\item El siguiente resultado es de Alamatsaz (1983)

\begin{Teo}
En cualquier sistema de colas $GI/G/1/L$ con $1\leq L<\infty$ y distribuci\'on de interarribos $A$ y distribuci\'on de los tiempos de servicio $B$, tal que $A\left(0\right)=0$, $A\left(t\right)\left(1-B\left(t\right)\right)>0$ para alguna $t>0$ y $B\left(t\right)$ para toda $t>0$, es imposible que el proceso de salida estacionario sea de renovaci\'on.
\end{Teo}

\end{itemize}

Estos resultados aparecen en Daley (1968) \cite{Daley68} para $\left\{T_{n}\right\}$ intervalos de inter-arribo, $\left\{D_{n}\right\}$ intervalos de inter-salida y $\left\{S_{n}\right\}$ tiempos de servicio.

\begin{itemize}
\item Si el proceso $\left\{T_{n}\right\}$ es Poisson, el proceso $\left\{D_{n}\right\}$ es no correlacionado si y s\'olo si es un proceso Poisso, lo cual ocurre si y s\'olo si $\left\{S_{n}\right\}$ son exponenciales negativas.

\item Si $\left\{S_{n}\right\}$ son exponenciales negativas, $\left\{D_{n}\right\}$ es un proceso de renovaci\'on  si y s\'olo si es un proceso Poisson, lo cual ocurre si y s\'olo si $\left\{T_{n}\right\}$ es un proceso Poisson.

\item $\esp\left(D_{n}\right)=\esp\left(T_{n}\right)$.

\item Para un sistema de visitas $GI/M/1$ se tiene el siguiente teorema:

\begin{Teo}
En un sistema estacionario $GI/M/1$ los intervalos de interpartida tienen
\begin{eqnarray*}
\esp\left(e^{-\theta D_{n}}\right)&=&\mu\left(\mu+\theta\right)^{-1}\left[\delta\theta
-\mu\left(1-\delta\right)\alpha\left(\theta\right)\right]
\left[\theta-\mu\left(1-\delta\right)^{-1}\right]\\
\alpha\left(\theta\right)&=&\esp\left[e^{-\theta T_{0}}\right]\\
var\left(D_{n}\right)&=&var\left(T_{0}\right)-\left(\tau^{-1}-\delta^{-1}\right)
2\delta\left(\esp\left(S_{0}\right)\right)^{2}\left(1-\delta\right)^{-1}.
\end{eqnarray*}
\end{Teo}



\begin{Teo}
El proceso de salida de un sistema de colas estacionario $GI/M/1$ es un proceso de renovaci\'on si y s\'olo si el proceso de entrada es un proceso Poisson, en cuyo caso el proceso de salida es un proceso Poisson.
\end{Teo}


\begin{Teo}
Los intervalos de interpartida $\left\{D_{n}\right\}$ de un sistema $M/G/1$ estacionario son no correlacionados si y s\'olo si la distribuci\'on de los tiempos de servicio es exponencial negativa, es decir, el sistema es de tipo  $M/M/1$.

\end{Teo}



\end{itemize}
%________________________________________________________________________
\subsection{Procesos Regenerativos}
%________________________________________________________________________

%________________________________________________________________________
\subsection*{Procesos Regenerativos Sigman, Thorisson y Wolff \cite{Sigman1}}
%________________________________________________________________________


\begin{Def}[Definici\'on Cl\'asica]
Un proceso estoc\'astico $X=\left\{X\left(t\right):t\geq0\right\}$ es llamado regenerativo is existe una variable aleatoria $R_{1}>0$ tal que
\begin{itemize}
\item[i)] $\left\{X\left(t+R_{1}\right):t\geq0\right\}$ es independiente de $\left\{\left\{X\left(t\right):t<R_{1}\right\},\right\}$
\item[ii)] $\left\{X\left(t+R_{1}\right):t\geq0\right\}$ es estoc\'asticamente equivalente a $\left\{X\left(t\right):t>0\right\}$
\end{itemize}

Llamamos a $R_{1}$ tiempo de regeneraci\'on, y decimos que $X$ se regenera en este punto.
\end{Def}

$\left\{X\left(t+R_{1}\right)\right\}$ es regenerativo con tiempo de regeneraci\'on $R_{2}$, independiente de $R_{1}$ pero con la misma distribuci\'on que $R_{1}$. Procediendo de esta manera se obtiene una secuencia de variables aleatorias independientes e id\'enticamente distribuidas $\left\{R_{n}\right\}$ llamados longitudes de ciclo. Si definimos a $Z_{k}\equiv R_{1}+R_{2}+\cdots+R_{k}$, se tiene un proceso de renovaci\'on llamado proceso de renovaci\'on encajado para $X$.


\begin{Note}
La existencia de un primer tiempo de regeneraci\'on, $R_{1}$, implica la existencia de una sucesi\'on completa de estos tiempos $R_{1},R_{2}\ldots,$ que satisfacen la propiedad deseada \cite{Sigman2}.
\end{Note}


\begin{Note} Para la cola $GI/GI/1$ los usuarios arriban con tiempos $t_{n}$ y son atendidos con tiempos de servicio $S_{n}$, los tiempos de arribo forman un proceso de renovaci\'on  con tiempos entre arribos independientes e identicamente distribuidos (\texttt{i.i.d.})$T_{n}=t_{n}-t_{n-1}$, adem\'as los tiempos de servicio son \texttt{i.i.d.} e independientes de los procesos de arribo. Por \textit{estable} se entiende que $\esp S_{n}<\esp T_{n}<\infty$.
\end{Note}
 


\begin{Def}
Para $x$ fijo y para cada $t\geq0$, sea $I_{x}\left(t\right)=1$ si $X\left(t\right)\leq x$,  $I_{x}\left(t\right)=0$ en caso contrario, y def\'inanse los tiempos promedio
\begin{eqnarray*}
\overline{X}&=&lim_{t\rightarrow\infty}\frac{1}{t}\int_{0}^{\infty}X\left(u\right)du\\
\prob\left(X_{\infty}\leq x\right)&=&lim_{t\rightarrow\infty}\frac{1}{t}\int_{0}^{\infty}I_{x}\left(u\right)du,
\end{eqnarray*}
cuando estos l\'imites existan.
\end{Def}

Como consecuencia del teorema de Renovaci\'on-Recompensa, se tiene que el primer l\'imite  existe y es igual a la constante
\begin{eqnarray*}
\overline{X}&=&\frac{\esp\left[\int_{0}^{R_{1}}X\left(t\right)dt\right]}{\esp\left[R_{1}\right]},
\end{eqnarray*}
suponiendo que ambas esperanzas son finitas.
 
\begin{Note}
Funciones de procesos regenerativos son regenerativas, es decir, si $X\left(t\right)$ es regenerativo y se define el proceso $Y\left(t\right)$ por $Y\left(t\right)=f\left(X\left(t\right)\right)$ para alguna funci\'on Borel medible $f\left(\cdot\right)$. Adem\'as $Y$ es regenerativo con los mismos tiempos de renovaci\'on que $X$. 

En general, los tiempos de renovaci\'on, $Z_{k}$ de un proceso regenerativo no requieren ser tiempos de paro con respecto a la evoluci\'on de $X\left(t\right)$.
\end{Note} 

\begin{Note}
Una funci\'on de un proceso de Markov, usualmente no ser\'a un proceso de Markov, sin embargo ser\'a regenerativo si el proceso de Markov lo es.
\end{Note}

 
\begin{Note}
Un proceso regenerativo con media de la longitud de ciclo finita es llamado positivo recurrente.
\end{Note}


\begin{Note}
\begin{itemize}
\item[a)] Si el proceso regenerativo $X$ es positivo recurrente y tiene trayectorias muestrales no negativas, entonces la ecuaci\'on anterior es v\'alida.
\item[b)] Si $X$ es positivo recurrente regenerativo, podemos construir una \'unica versi\'on estacionaria de este proceso, $X_{e}=\left\{X_{e}\left(t\right)\right\}$, donde $X_{e}$ es un proceso estoc\'astico regenerativo y estrictamente estacionario, con distribuci\'on marginal distribuida como $X_{\infty}$
\end{itemize}
\end{Note}


%__________________________________________________________________________________________
%\subsection*{Procesos Regenerativos Estacionarios - Stidham \cite{Stidham}}
%__________________________________________________________________________________________


Un proceso estoc\'astico a tiempo continuo $\left\{V\left(t\right),t\geq0\right\}$ es un proceso regenerativo si existe una sucesi\'on de variables aleatorias independientes e id\'enticamente distribuidas $\left\{X_{1},X_{2},\ldots\right\}$, sucesi\'on de renovaci\'on, tal que para cualquier conjunto de Borel $A$, 

\begin{eqnarray*}
\prob\left\{V\left(t\right)\in A|X_{1}+X_{2}+\cdots+X_{R\left(t\right)}=s,\left\{V\left(\tau\right),\tau<s\right\}\right\}=\prob\left\{V\left(t-s\right)\in A|X_{1}>t-s\right\},
\end{eqnarray*}
para todo $0\leq s\leq t$, donde $R\left(t\right)=\max\left\{X_{1}+X_{2}+\cdots+X_{j}\leq t\right\}=$n\'umero de renovaciones ({\emph{puntos de regeneraci\'on}}) que ocurren en $\left[0,t\right]$. El intervalo $\left[0,X_{1}\right)$ es llamado {\emph{primer ciclo de regeneraci\'on}} de $\left\{V\left(t \right),t\geq0\right\}$, $\left[X_{1},X_{1}+X_{2}\right)$ el {\emph{segundo ciclo de regeneraci\'on}}, y as\'i sucesivamente.

Sea $X=X_{1}$ y sea $F$ la funci\'on de distrbuci\'on de $X$


\begin{Def}
Se define el proceso estacionario, $\left\{V^{*}\left(t\right),t\geq0\right\}$, para $\left\{V\left(t\right),t\geq0\right\}$ por

\begin{eqnarray*}
\prob\left\{V\left(t\right)\in A\right\}=\frac{1}{\esp\left[X\right]}\int_{0}^{\infty}\prob\left\{V\left(t+x\right)\in A|X>x\right\}\left(1-F\left(x\right)\right)dx,
\end{eqnarray*} 
para todo $t\geq0$ y todo conjunto de Borel $A$.
\end{Def}

\begin{Def}
Una distribuci\'on se dice que es {\emph{aritm\'etica}} si todos sus puntos de incremento son m\'ultiplos de la forma $0,\lambda, 2\lambda,\ldots$ para alguna $\lambda>0$ entera.
\end{Def}


\begin{Def}
Una modificaci\'on medible de un proceso $\left\{V\left(t\right),t\geq0\right\}$, es una versi\'on de este, $\left\{V\left(t,w\right)\right\}$ conjuntamente medible para $t\geq0$ y para $w\in S$, $S$ espacio de estados para $\left\{V\left(t\right),t\geq0\right\}$.
\end{Def}

\begin{Teo}
Sea $\left\{V\left(t\right),t\geq\right\}$ un proceso regenerativo no negativo con modificaci\'on medible. Sea $\esp\left[X\right]<\infty$. Entonces el proceso estacionario dado por la ecuaci\'on anterior est\'a bien definido y tiene funci\'on de distribuci\'on independiente de $t$, adem\'as
\begin{itemize}
\item[i)] \begin{eqnarray*}
\esp\left[V^{*}\left(0\right)\right]&=&\frac{\esp\left[\int_{0}^{X}V\left(s\right)ds\right]}{\esp\left[X\right]}\end{eqnarray*}
\item[ii)] Si $\esp\left[V^{*}\left(0\right)\right]<\infty$, equivalentemente, si $\esp\left[\int_{0}^{X}V\left(s\right)ds\right]<\infty$,entonces
\begin{eqnarray*}
\frac{\int_{0}^{t}V\left(s\right)ds}{t}\rightarrow\frac{\esp\left[\int_{0}^{X}V\left(s\right)ds\right]}{\esp\left[X\right]}
\end{eqnarray*}
con probabilidad 1 y en media, cuando $t\rightarrow\infty$.
\end{itemize}
\end{Teo}

\begin{Coro}
Sea $\left\{V\left(t\right),t\geq0\right\}$ un proceso regenerativo no negativo, con modificaci\'on medible. Si $\esp <\infty$, $F$ es no-aritm\'etica, y para todo $x\geq0$, $P\left\{V\left(t\right)\leq x,C>x\right\}$ es de variaci\'on acotada como funci\'on de $t$ en cada intervalo finito $\left[0,\tau\right]$, entonces $V\left(t\right)$ converge en distribuci\'on  cuando $t\rightarrow\infty$ y $$\esp V=\frac{\esp \int_{0}^{X}V\left(s\right)ds}{\esp X}$$
Donde $V$ tiene la distribuci\'on l\'imite de $V\left(t\right)$ cuando $t\rightarrow\infty$.

\end{Coro}

Para el caso discreto se tienen resultados similares.

%______________________________________________________________________
%\section{Procesos de Renovaci\'on}
%______________________________________________________________________

\begin{Def}\label{Def.Tn}
Sean $0\leq T_{1}\leq T_{2}\leq \ldots$ son tiempos aleatorios infinitos en los cuales ocurren ciertos eventos. El n\'umero de tiempos $T_{n}$ en el intervalo $\left[0,t\right)$ es

\begin{eqnarray}
N\left(t\right)=\sum_{n=1}^{\infty}\indora\left(T_{n}\leq t\right),
\end{eqnarray}
para $t\geq0$.
\end{Def}

Si se consideran los puntos $T_{n}$ como elementos de $\rea_{+}$, y $N\left(t\right)$ es el n\'umero de puntos en $\rea$. El proceso denotado por $\left\{N\left(t\right):t\geq0\right\}$, denotado por $N\left(t\right)$, es un proceso puntual en $\rea_{+}$. Los $T_{n}$ son los tiempos de ocurrencia, el proceso puntual $N\left(t\right)$ es simple si su n\'umero de ocurrencias son distintas: $0<T_{1}<T_{2}<\ldots$ casi seguramente.

\begin{Def}
Un proceso puntual $N\left(t\right)$ es un proceso de renovaci\'on si los tiempos de interocurrencia $\xi_{n}=T_{n}-T_{n-1}$, para $n\geq1$, son independientes e identicamente distribuidos con distribuci\'on $F$, donde $F\left(0\right)=0$ y $T_{0}=0$. Los $T_{n}$ son llamados tiempos de renovaci\'on, referente a la independencia o renovaci\'on de la informaci\'on estoc\'astica en estos tiempos. Los $\xi_{n}$ son los tiempos de inter-renovaci\'on, y $N\left(t\right)$ es el n\'umero de renovaciones en el intervalo $\left[0,t\right)$
\end{Def}


\begin{Note}
Para definir un proceso de renovaci\'on para cualquier contexto, solamente hay que especificar una distribuci\'on $F$, con $F\left(0\right)=0$, para los tiempos de inter-renovaci\'on. La funci\'on $F$ en turno degune las otra variables aleatorias. De manera formal, existe un espacio de probabilidad y una sucesi\'on de variables aleatorias $\xi_{1},\xi_{2},\ldots$ definidas en este con distribuci\'on $F$. Entonces las otras cantidades son $T_{n}=\sum_{k=1}^{n}\xi_{k}$ y $N\left(t\right)=\sum_{n=1}^{\infty}\indora\left(T_{n}\leq t\right)$, donde $T_{n}\rightarrow\infty$ casi seguramente por la Ley Fuerte de los Grandes Números.
\end{Note}

%___________________________________________________________________________________________
%
%\subsection*{Teorema Principal de Renovaci\'on}
%___________________________________________________________________________________________
%

\begin{Note} Una funci\'on $h:\rea_{+}\rightarrow\rea$ es Directamente Riemann Integrable en los siguientes casos:
\begin{itemize}
\item[a)] $h\left(t\right)\geq0$ es decreciente y Riemann Integrable.
\item[b)] $h$ es continua excepto posiblemente en un conjunto de Lebesgue de medida 0, y $|h\left(t\right)|\leq b\left(t\right)$, donde $b$ es DRI.
\end{itemize}
\end{Note}

\begin{Teo}[Teorema Principal de Renovaci\'on]
Si $F$ es no aritm\'etica y $h\left(t\right)$ es Directamente Riemann Integrable (DRI), entonces

\begin{eqnarray*}
lim_{t\rightarrow\infty}U\star h=\frac{1}{\mu}\int_{\rea_{+}}h\left(s\right)ds.
\end{eqnarray*}
\end{Teo}

\begin{Prop}
Cualquier funci\'on $H\left(t\right)$ acotada en intervalos finitos y que es 0 para $t<0$ puede expresarse como
\begin{eqnarray*}
H\left(t\right)=U\star h\left(t\right)\textrm{,  donde }h\left(t\right)=H\left(t\right)-F\star H\left(t\right)
\end{eqnarray*}
\end{Prop}

\begin{Def}
Un proceso estoc\'astico $X\left(t\right)$ es crudamente regenerativo en un tiempo aleatorio positivo $T$ si
\begin{eqnarray*}
\esp\left[X\left(T+t\right)|T\right]=\esp\left[X\left(t\right)\right]\textrm{, para }t\geq0,\end{eqnarray*}
y con las esperanzas anteriores finitas.
\end{Def}

\begin{Prop}
Sup\'ongase que $X\left(t\right)$ es un proceso crudamente regenerativo en $T$, que tiene distribuci\'on $F$. Si $\esp\left[X\left(t\right)\right]$ es acotado en intervalos finitos, entonces
\begin{eqnarray*}
\esp\left[X\left(t\right)\right]=U\star h\left(t\right)\textrm{,  donde }h\left(t\right)=\esp\left[X\left(t\right)\indora\left(T>t\right)\right].
\end{eqnarray*}
\end{Prop}

\begin{Teo}[Regeneraci\'on Cruda]
Sup\'ongase que $X\left(t\right)$ es un proceso con valores positivo crudamente regenerativo en $T$, y def\'inase $M=\sup\left\{|X\left(t\right)|:t\leq T\right\}$. Si $T$ es no aritm\'etico y $M$ y $MT$ tienen media finita, entonces
\begin{eqnarray*}
lim_{t\rightarrow\infty}\esp\left[X\left(t\right)\right]=\frac{1}{\mu}\int_{\rea_{+}}h\left(s\right)ds,
\end{eqnarray*}
donde $h\left(t\right)=\esp\left[X\left(t\right)\indora\left(T>t\right)\right]$.
\end{Teo}

%___________________________________________________________________________________________
%
%\subsection*{Propiedades de los Procesos de Renovaci\'on}
%___________________________________________________________________________________________
%

Los tiempos $T_{n}$ est\'an relacionados con los conteos de $N\left(t\right)$ por

\begin{eqnarray*}
\left\{N\left(t\right)\geq n\right\}&=&\left\{T_{n}\leq t\right\}\\
T_{N\left(t\right)}\leq &t&<T_{N\left(t\right)+1},
\end{eqnarray*}

adem\'as $N\left(T_{n}\right)=n$, y 

\begin{eqnarray*}
N\left(t\right)=\max\left\{n:T_{n}\leq t\right\}=\min\left\{n:T_{n+1}>t\right\}
\end{eqnarray*}

Por propiedades de la convoluci\'on se sabe que

\begin{eqnarray*}
P\left\{T_{n}\leq t\right\}=F^{n\star}\left(t\right)
\end{eqnarray*}
que es la $n$-\'esima convoluci\'on de $F$. Entonces 

\begin{eqnarray*}
\left\{N\left(t\right)\geq n\right\}&=&\left\{T_{n}\leq t\right\}\\
P\left\{N\left(t\right)\leq n\right\}&=&1-F^{\left(n+1\right)\star}\left(t\right)
\end{eqnarray*}

Adem\'as usando el hecho de que $\esp\left[N\left(t\right)\right]=\sum_{n=1}^{\infty}P\left\{N\left(t\right)\geq n\right\}$
se tiene que

\begin{eqnarray*}
\esp\left[N\left(t\right)\right]=\sum_{n=1}^{\infty}F^{n\star}\left(t\right)
\end{eqnarray*}

\begin{Prop}
Para cada $t\geq0$, la funci\'on generadora de momentos $\esp\left[e^{\alpha N\left(t\right)}\right]$ existe para alguna $\alpha$ en una vecindad del 0, y de aqu\'i que $\esp\left[N\left(t\right)^{m}\right]<\infty$, para $m\geq1$.
\end{Prop}


\begin{Note}
Si el primer tiempo de renovaci\'on $\xi_{1}$ no tiene la misma distribuci\'on que el resto de las $\xi_{n}$, para $n\geq2$, a $N\left(t\right)$ se le llama Proceso de Renovaci\'on retardado, donde si $\xi$ tiene distribuci\'on $G$, entonces el tiempo $T_{n}$ de la $n$-\'esima renovaci\'on tiene distribuci\'on $G\star F^{\left(n-1\right)\star}\left(t\right)$
\end{Note}


\begin{Teo}
Para una constante $\mu\leq\infty$ ( o variable aleatoria), las siguientes expresiones son equivalentes:

\begin{eqnarray}
lim_{n\rightarrow\infty}n^{-1}T_{n}&=&\mu,\textrm{ c.s.}\\
lim_{t\rightarrow\infty}t^{-1}N\left(t\right)&=&1/\mu,\textrm{ c.s.}
\end{eqnarray}
\end{Teo}


Es decir, $T_{n}$ satisface la Ley Fuerte de los Grandes N\'umeros s\'i y s\'olo s\'i $N\left/t\right)$ la cumple.


\begin{Coro}[Ley Fuerte de los Grandes N\'umeros para Procesos de Renovaci\'on]
Si $N\left(t\right)$ es un proceso de renovaci\'on cuyos tiempos de inter-renovaci\'on tienen media $\mu\leq\infty$, entonces
\begin{eqnarray}
t^{-1}N\left(t\right)\rightarrow 1/\mu,\textrm{ c.s. cuando }t\rightarrow\infty.
\end{eqnarray}

\end{Coro}


Considerar el proceso estoc\'astico de valores reales $\left\{Z\left(t\right):t\geq0\right\}$ en el mismo espacio de probabilidad que $N\left(t\right)$

\begin{Def}
Para el proceso $\left\{Z\left(t\right):t\geq0\right\}$ se define la fluctuaci\'on m\'axima de $Z\left(t\right)$ en el intervalo $\left(T_{n-1},T_{n}\right]$:
\begin{eqnarray*}
M_{n}=\sup_{T_{n-1}<t\leq T_{n}}|Z\left(t\right)-Z\left(T_{n-1}\right)|
\end{eqnarray*}
\end{Def}

\begin{Teo}
Sup\'ongase que $n^{-1}T_{n}\rightarrow\mu$ c.s. cuando $n\rightarrow\infty$, donde $\mu\leq\infty$ es una constante o variable aleatoria. Sea $a$ una constante o variable aleatoria que puede ser infinita cuando $\mu$ es finita, y considere las expresiones l\'imite:
\begin{eqnarray}
lim_{n\rightarrow\infty}n^{-1}Z\left(T_{n}\right)&=&a,\textrm{ c.s.}\\
lim_{t\rightarrow\infty}t^{-1}Z\left(t\right)&=&a/\mu,\textrm{ c.s.}
\end{eqnarray}
La segunda expresi\'on implica la primera. Conversamente, la primera implica la segunda si el proceso $Z\left(t\right)$ es creciente, o si $lim_{n\rightarrow\infty}n^{-1}M_{n}=0$ c.s.
\end{Teo}

\begin{Coro}
Si $N\left(t\right)$ es un proceso de renovaci\'on, y $\left(Z\left(T_{n}\right)-Z\left(T_{n-1}\right),M_{n}\right)$, para $n\geq1$, son variables aleatorias independientes e id\'enticamente distribuidas con media finita, entonces,
\begin{eqnarray}
lim_{t\rightarrow\infty}t^{-1}Z\left(t\right)\rightarrow\frac{\esp\left[Z\left(T_{1}\right)-Z\left(T_{0}\right)\right]}{\esp\left[T_{1}\right]},\textrm{ c.s. cuando  }t\rightarrow\infty.
\end{eqnarray}
\end{Coro}



%___________________________________________________________________________________________
%
%\subsection{Propiedades de los Procesos de Renovaci\'on}
%___________________________________________________________________________________________
%

Los tiempos $T_{n}$ est\'an relacionados con los conteos de $N\left(t\right)$ por

\begin{eqnarray*}
\left\{N\left(t\right)\geq n\right\}&=&\left\{T_{n}\leq t\right\}\\
T_{N\left(t\right)}\leq &t&<T_{N\left(t\right)+1},
\end{eqnarray*}

adem\'as $N\left(T_{n}\right)=n$, y 

\begin{eqnarray*}
N\left(t\right)=\max\left\{n:T_{n}\leq t\right\}=\min\left\{n:T_{n+1}>t\right\}
\end{eqnarray*}

Por propiedades de la convoluci\'on se sabe que

\begin{eqnarray*}
P\left\{T_{n}\leq t\right\}=F^{n\star}\left(t\right)
\end{eqnarray*}
que es la $n$-\'esima convoluci\'on de $F$. Entonces 

\begin{eqnarray*}
\left\{N\left(t\right)\geq n\right\}&=&\left\{T_{n}\leq t\right\}\\
P\left\{N\left(t\right)\leq n\right\}&=&1-F^{\left(n+1\right)\star}\left(t\right)
\end{eqnarray*}

Adem\'as usando el hecho de que $\esp\left[N\left(t\right)\right]=\sum_{n=1}^{\infty}P\left\{N\left(t\right)\geq n\right\}$
se tiene que

\begin{eqnarray*}
\esp\left[N\left(t\right)\right]=\sum_{n=1}^{\infty}F^{n\star}\left(t\right)
\end{eqnarray*}

\begin{Prop}
Para cada $t\geq0$, la funci\'on generadora de momentos $\esp\left[e^{\alpha N\left(t\right)}\right]$ existe para alguna $\alpha$ en una vecindad del 0, y de aqu\'i que $\esp\left[N\left(t\right)^{m}\right]<\infty$, para $m\geq1$.
\end{Prop}


\begin{Note}
Si el primer tiempo de renovaci\'on $\xi_{1}$ no tiene la misma distribuci\'on que el resto de las $\xi_{n}$, para $n\geq2$, a $N\left(t\right)$ se le llama Proceso de Renovaci\'on retardado, donde si $\xi$ tiene distribuci\'on $G$, entonces el tiempo $T_{n}$ de la $n$-\'esima renovaci\'on tiene distribuci\'on $G\star F^{\left(n-1\right)\star}\left(t\right)$
\end{Note}


\begin{Teo}
Para una constante $\mu\leq\infty$ ( o variable aleatoria), las siguientes expresiones son equivalentes:

\begin{eqnarray}
lim_{n\rightarrow\infty}n^{-1}T_{n}&=&\mu,\textrm{ c.s.}\\
lim_{t\rightarrow\infty}t^{-1}N\left(t\right)&=&1/\mu,\textrm{ c.s.}
\end{eqnarray}
\end{Teo}


Es decir, $T_{n}$ satisface la Ley Fuerte de los Grandes N\'umeros s\'i y s\'olo s\'i $N\left/t\right)$ la cumple.


\begin{Coro}[Ley Fuerte de los Grandes N\'umeros para Procesos de Renovaci\'on]
Si $N\left(t\right)$ es un proceso de renovaci\'on cuyos tiempos de inter-renovaci\'on tienen media $\mu\leq\infty$, entonces
\begin{eqnarray}
t^{-1}N\left(t\right)\rightarrow 1/\mu,\textrm{ c.s. cuando }t\rightarrow\infty.
\end{eqnarray}

\end{Coro}


Considerar el proceso estoc\'astico de valores reales $\left\{Z\left(t\right):t\geq0\right\}$ en el mismo espacio de probabilidad que $N\left(t\right)$

\begin{Def}
Para el proceso $\left\{Z\left(t\right):t\geq0\right\}$ se define la fluctuaci\'on m\'axima de $Z\left(t\right)$ en el intervalo $\left(T_{n-1},T_{n}\right]$:
\begin{eqnarray*}
M_{n}=\sup_{T_{n-1}<t\leq T_{n}}|Z\left(t\right)-Z\left(T_{n-1}\right)|
\end{eqnarray*}
\end{Def}

\begin{Teo}
Sup\'ongase que $n^{-1}T_{n}\rightarrow\mu$ c.s. cuando $n\rightarrow\infty$, donde $\mu\leq\infty$ es una constante o variable aleatoria. Sea $a$ una constante o variable aleatoria que puede ser infinita cuando $\mu$ es finita, y considere las expresiones l\'imite:
\begin{eqnarray}
lim_{n\rightarrow\infty}n^{-1}Z\left(T_{n}\right)&=&a,\textrm{ c.s.}\\
lim_{t\rightarrow\infty}t^{-1}Z\left(t\right)&=&a/\mu,\textrm{ c.s.}
\end{eqnarray}
La segunda expresi\'on implica la primera. Conversamente, la primera implica la segunda si el proceso $Z\left(t\right)$ es creciente, o si $lim_{n\rightarrow\infty}n^{-1}M_{n}=0$ c.s.
\end{Teo}

\begin{Coro}
Si $N\left(t\right)$ es un proceso de renovaci\'on, y $\left(Z\left(T_{n}\right)-Z\left(T_{n-1}\right),M_{n}\right)$, para $n\geq1$, son variables aleatorias independientes e id\'enticamente distribuidas con media finita, entonces,
\begin{eqnarray}
lim_{t\rightarrow\infty}t^{-1}Z\left(t\right)\rightarrow\frac{\esp\left[Z\left(T_{1}\right)-Z\left(T_{0}\right)\right]}{\esp\left[T_{1}\right]},\textrm{ c.s. cuando  }t\rightarrow\infty.
\end{eqnarray}
\end{Coro}


%___________________________________________________________________________________________
%
%\subsection{Propiedades de los Procesos de Renovaci\'on}
%___________________________________________________________________________________________
%

Los tiempos $T_{n}$ est\'an relacionados con los conteos de $N\left(t\right)$ por

\begin{eqnarray*}
\left\{N\left(t\right)\geq n\right\}&=&\left\{T_{n}\leq t\right\}\\
T_{N\left(t\right)}\leq &t&<T_{N\left(t\right)+1},
\end{eqnarray*}

adem\'as $N\left(T_{n}\right)=n$, y 

\begin{eqnarray*}
N\left(t\right)=\max\left\{n:T_{n}\leq t\right\}=\min\left\{n:T_{n+1}>t\right\}
\end{eqnarray*}

Por propiedades de la convoluci\'on se sabe que

\begin{eqnarray*}
P\left\{T_{n}\leq t\right\}=F^{n\star}\left(t\right)
\end{eqnarray*}
que es la $n$-\'esima convoluci\'on de $F$. Entonces 

\begin{eqnarray*}
\left\{N\left(t\right)\geq n\right\}&=&\left\{T_{n}\leq t\right\}\\
P\left\{N\left(t\right)\leq n\right\}&=&1-F^{\left(n+1\right)\star}\left(t\right)
\end{eqnarray*}

Adem\'as usando el hecho de que $\esp\left[N\left(t\right)\right]=\sum_{n=1}^{\infty}P\left\{N\left(t\right)\geq n\right\}$
se tiene que

\begin{eqnarray*}
\esp\left[N\left(t\right)\right]=\sum_{n=1}^{\infty}F^{n\star}\left(t\right)
\end{eqnarray*}

\begin{Prop}
Para cada $t\geq0$, la funci\'on generadora de momentos $\esp\left[e^{\alpha N\left(t\right)}\right]$ existe para alguna $\alpha$ en una vecindad del 0, y de aqu\'i que $\esp\left[N\left(t\right)^{m}\right]<\infty$, para $m\geq1$.
\end{Prop}


\begin{Note}
Si el primer tiempo de renovaci\'on $\xi_{1}$ no tiene la misma distribuci\'on que el resto de las $\xi_{n}$, para $n\geq2$, a $N\left(t\right)$ se le llama Proceso de Renovaci\'on retardado, donde si $\xi$ tiene distribuci\'on $G$, entonces el tiempo $T_{n}$ de la $n$-\'esima renovaci\'on tiene distribuci\'on $G\star F^{\left(n-1\right)\star}\left(t\right)$
\end{Note}


\begin{Teo}
Para una constante $\mu\leq\infty$ ( o variable aleatoria), las siguientes expresiones son equivalentes:

\begin{eqnarray}
lim_{n\rightarrow\infty}n^{-1}T_{n}&=&\mu,\textrm{ c.s.}\\
lim_{t\rightarrow\infty}t^{-1}N\left(t\right)&=&1/\mu,\textrm{ c.s.}
\end{eqnarray}
\end{Teo}


Es decir, $T_{n}$ satisface la Ley Fuerte de los Grandes N\'umeros s\'i y s\'olo s\'i $N\left/t\right)$ la cumple.


\begin{Coro}[Ley Fuerte de los Grandes N\'umeros para Procesos de Renovaci\'on]
Si $N\left(t\right)$ es un proceso de renovaci\'on cuyos tiempos de inter-renovaci\'on tienen media $\mu\leq\infty$, entonces
\begin{eqnarray}
t^{-1}N\left(t\right)\rightarrow 1/\mu,\textrm{ c.s. cuando }t\rightarrow\infty.
\end{eqnarray}

\end{Coro}


Considerar el proceso estoc\'astico de valores reales $\left\{Z\left(t\right):t\geq0\right\}$ en el mismo espacio de probabilidad que $N\left(t\right)$

\begin{Def}
Para el proceso $\left\{Z\left(t\right):t\geq0\right\}$ se define la fluctuaci\'on m\'axima de $Z\left(t\right)$ en el intervalo $\left(T_{n-1},T_{n}\right]$:
\begin{eqnarray*}
M_{n}=\sup_{T_{n-1}<t\leq T_{n}}|Z\left(t\right)-Z\left(T_{n-1}\right)|
\end{eqnarray*}
\end{Def}

\begin{Teo}
Sup\'ongase que $n^{-1}T_{n}\rightarrow\mu$ c.s. cuando $n\rightarrow\infty$, donde $\mu\leq\infty$ es una constante o variable aleatoria. Sea $a$ una constante o variable aleatoria que puede ser infinita cuando $\mu$ es finita, y considere las expresiones l\'imite:
\begin{eqnarray}
lim_{n\rightarrow\infty}n^{-1}Z\left(T_{n}\right)&=&a,\textrm{ c.s.}\\
lim_{t\rightarrow\infty}t^{-1}Z\left(t\right)&=&a/\mu,\textrm{ c.s.}
\end{eqnarray}
La segunda expresi\'on implica la primera. Conversamente, la primera implica la segunda si el proceso $Z\left(t\right)$ es creciente, o si $lim_{n\rightarrow\infty}n^{-1}M_{n}=0$ c.s.
\end{Teo}

\begin{Coro}
Si $N\left(t\right)$ es un proceso de renovaci\'on, y $\left(Z\left(T_{n}\right)-Z\left(T_{n-1}\right),M_{n}\right)$, para $n\geq1$, son variables aleatorias independientes e id\'enticamente distribuidas con media finita, entonces,
\begin{eqnarray}
lim_{t\rightarrow\infty}t^{-1}Z\left(t\right)\rightarrow\frac{\esp\left[Z\left(T_{1}\right)-Z\left(T_{0}\right)\right]}{\esp\left[T_{1}\right]},\textrm{ c.s. cuando  }t\rightarrow\infty.
\end{eqnarray}
\end{Coro}

%___________________________________________________________________________________________
%
%\subsection{Propiedades de los Procesos de Renovaci\'on}
%___________________________________________________________________________________________
%

Los tiempos $T_{n}$ est\'an relacionados con los conteos de $N\left(t\right)$ por

\begin{eqnarray*}
\left\{N\left(t\right)\geq n\right\}&=&\left\{T_{n}\leq t\right\}\\
T_{N\left(t\right)}\leq &t&<T_{N\left(t\right)+1},
\end{eqnarray*}

adem\'as $N\left(T_{n}\right)=n$, y 

\begin{eqnarray*}
N\left(t\right)=\max\left\{n:T_{n}\leq t\right\}=\min\left\{n:T_{n+1}>t\right\}
\end{eqnarray*}

Por propiedades de la convoluci\'on se sabe que

\begin{eqnarray*}
P\left\{T_{n}\leq t\right\}=F^{n\star}\left(t\right)
\end{eqnarray*}
que es la $n$-\'esima convoluci\'on de $F$. Entonces 

\begin{eqnarray*}
\left\{N\left(t\right)\geq n\right\}&=&\left\{T_{n}\leq t\right\}\\
P\left\{N\left(t\right)\leq n\right\}&=&1-F^{\left(n+1\right)\star}\left(t\right)
\end{eqnarray*}

Adem\'as usando el hecho de que $\esp\left[N\left(t\right)\right]=\sum_{n=1}^{\infty}P\left\{N\left(t\right)\geq n\right\}$
se tiene que

\begin{eqnarray*}
\esp\left[N\left(t\right)\right]=\sum_{n=1}^{\infty}F^{n\star}\left(t\right)
\end{eqnarray*}

\begin{Prop}
Para cada $t\geq0$, la funci\'on generadora de momentos $\esp\left[e^{\alpha N\left(t\right)}\right]$ existe para alguna $\alpha$ en una vecindad del 0, y de aqu\'i que $\esp\left[N\left(t\right)^{m}\right]<\infty$, para $m\geq1$.
\end{Prop}


\begin{Note}
Si el primer tiempo de renovaci\'on $\xi_{1}$ no tiene la misma distribuci\'on que el resto de las $\xi_{n}$, para $n\geq2$, a $N\left(t\right)$ se le llama Proceso de Renovaci\'on retardado, donde si $\xi$ tiene distribuci\'on $G$, entonces el tiempo $T_{n}$ de la $n$-\'esima renovaci\'on tiene distribuci\'on $G\star F^{\left(n-1\right)\star}\left(t\right)$
\end{Note}


\begin{Teo}
Para una constante $\mu\leq\infty$ ( o variable aleatoria), las siguientes expresiones son equivalentes:

\begin{eqnarray}
lim_{n\rightarrow\infty}n^{-1}T_{n}&=&\mu,\textrm{ c.s.}\\
lim_{t\rightarrow\infty}t^{-1}N\left(t\right)&=&1/\mu,\textrm{ c.s.}
\end{eqnarray}
\end{Teo}


Es decir, $T_{n}$ satisface la Ley Fuerte de los Grandes N\'umeros s\'i y s\'olo s\'i $N\left/t\right)$ la cumple.


\begin{Coro}[Ley Fuerte de los Grandes N\'umeros para Procesos de Renovaci\'on]
Si $N\left(t\right)$ es un proceso de renovaci\'on cuyos tiempos de inter-renovaci\'on tienen media $\mu\leq\infty$, entonces
\begin{eqnarray}
t^{-1}N\left(t\right)\rightarrow 1/\mu,\textrm{ c.s. cuando }t\rightarrow\infty.
\end{eqnarray}

\end{Coro}


Considerar el proceso estoc\'astico de valores reales $\left\{Z\left(t\right):t\geq0\right\}$ en el mismo espacio de probabilidad que $N\left(t\right)$

\begin{Def}
Para el proceso $\left\{Z\left(t\right):t\geq0\right\}$ se define la fluctuaci\'on m\'axima de $Z\left(t\right)$ en el intervalo $\left(T_{n-1},T_{n}\right]$:
\begin{eqnarray*}
M_{n}=\sup_{T_{n-1}<t\leq T_{n}}|Z\left(t\right)-Z\left(T_{n-1}\right)|
\end{eqnarray*}
\end{Def}

\begin{Teo}
Sup\'ongase que $n^{-1}T_{n}\rightarrow\mu$ c.s. cuando $n\rightarrow\infty$, donde $\mu\leq\infty$ es una constante o variable aleatoria. Sea $a$ una constante o variable aleatoria que puede ser infinita cuando $\mu$ es finita, y considere las expresiones l\'imite:
\begin{eqnarray}
lim_{n\rightarrow\infty}n^{-1}Z\left(T_{n}\right)&=&a,\textrm{ c.s.}\\
lim_{t\rightarrow\infty}t^{-1}Z\left(t\right)&=&a/\mu,\textrm{ c.s.}
\end{eqnarray}
La segunda expresi\'on implica la primera. Conversamente, la primera implica la segunda si el proceso $Z\left(t\right)$ es creciente, o si $lim_{n\rightarrow\infty}n^{-1}M_{n}=0$ c.s.
\end{Teo}

\begin{Coro}
Si $N\left(t\right)$ es un proceso de renovaci\'on, y $\left(Z\left(T_{n}\right)-Z\left(T_{n-1}\right),M_{n}\right)$, para $n\geq1$, son variables aleatorias independientes e id\'enticamente distribuidas con media finita, entonces,
\begin{eqnarray}
lim_{t\rightarrow\infty}t^{-1}Z\left(t\right)\rightarrow\frac{\esp\left[Z\left(T_{1}\right)-Z\left(T_{0}\right)\right]}{\esp\left[T_{1}\right]},\textrm{ c.s. cuando  }t\rightarrow\infty.
\end{eqnarray}
\end{Coro}
%___________________________________________________________________________________________
%
%\subsection{Propiedades de los Procesos de Renovaci\'on}
%___________________________________________________________________________________________
%

Los tiempos $T_{n}$ est\'an relacionados con los conteos de $N\left(t\right)$ por

\begin{eqnarray*}
\left\{N\left(t\right)\geq n\right\}&=&\left\{T_{n}\leq t\right\}\\
T_{N\left(t\right)}\leq &t&<T_{N\left(t\right)+1},
\end{eqnarray*}

adem\'as $N\left(T_{n}\right)=n$, y 

\begin{eqnarray*}
N\left(t\right)=\max\left\{n:T_{n}\leq t\right\}=\min\left\{n:T_{n+1}>t\right\}
\end{eqnarray*}

Por propiedades de la convoluci\'on se sabe que

\begin{eqnarray*}
P\left\{T_{n}\leq t\right\}=F^{n\star}\left(t\right)
\end{eqnarray*}
que es la $n$-\'esima convoluci\'on de $F$. Entonces 

\begin{eqnarray*}
\left\{N\left(t\right)\geq n\right\}&=&\left\{T_{n}\leq t\right\}\\
P\left\{N\left(t\right)\leq n\right\}&=&1-F^{\left(n+1\right)\star}\left(t\right)
\end{eqnarray*}

Adem\'as usando el hecho de que $\esp\left[N\left(t\right)\right]=\sum_{n=1}^{\infty}P\left\{N\left(t\right)\geq n\right\}$
se tiene que

\begin{eqnarray*}
\esp\left[N\left(t\right)\right]=\sum_{n=1}^{\infty}F^{n\star}\left(t\right)
\end{eqnarray*}

\begin{Prop}
Para cada $t\geq0$, la funci\'on generadora de momentos $\esp\left[e^{\alpha N\left(t\right)}\right]$ existe para alguna $\alpha$ en una vecindad del 0, y de aqu\'i que $\esp\left[N\left(t\right)^{m}\right]<\infty$, para $m\geq1$.
\end{Prop}


\begin{Note}
Si el primer tiempo de renovaci\'on $\xi_{1}$ no tiene la misma distribuci\'on que el resto de las $\xi_{n}$, para $n\geq2$, a $N\left(t\right)$ se le llama Proceso de Renovaci\'on retardado, donde si $\xi$ tiene distribuci\'on $G$, entonces el tiempo $T_{n}$ de la $n$-\'esima renovaci\'on tiene distribuci\'on $G\star F^{\left(n-1\right)\star}\left(t\right)$
\end{Note}


\begin{Teo}
Para una constante $\mu\leq\infty$ ( o variable aleatoria), las siguientes expresiones son equivalentes:

\begin{eqnarray}
lim_{n\rightarrow\infty}n^{-1}T_{n}&=&\mu,\textrm{ c.s.}\\
lim_{t\rightarrow\infty}t^{-1}N\left(t\right)&=&1/\mu,\textrm{ c.s.}
\end{eqnarray}
\end{Teo}


Es decir, $T_{n}$ satisface la Ley Fuerte de los Grandes N\'umeros s\'i y s\'olo s\'i $N\left/t\right)$ la cumple.


\begin{Coro}[Ley Fuerte de los Grandes N\'umeros para Procesos de Renovaci\'on]
Si $N\left(t\right)$ es un proceso de renovaci\'on cuyos tiempos de inter-renovaci\'on tienen media $\mu\leq\infty$, entonces
\begin{eqnarray}
t^{-1}N\left(t\right)\rightarrow 1/\mu,\textrm{ c.s. cuando }t\rightarrow\infty.
\end{eqnarray}

\end{Coro}


Considerar el proceso estoc\'astico de valores reales $\left\{Z\left(t\right):t\geq0\right\}$ en el mismo espacio de probabilidad que $N\left(t\right)$

\begin{Def}
Para el proceso $\left\{Z\left(t\right):t\geq0\right\}$ se define la fluctuaci\'on m\'axima de $Z\left(t\right)$ en el intervalo $\left(T_{n-1},T_{n}\right]$:
\begin{eqnarray*}
M_{n}=\sup_{T_{n-1}<t\leq T_{n}}|Z\left(t\right)-Z\left(T_{n-1}\right)|
\end{eqnarray*}
\end{Def}

\begin{Teo}
Sup\'ongase que $n^{-1}T_{n}\rightarrow\mu$ c.s. cuando $n\rightarrow\infty$, donde $\mu\leq\infty$ es una constante o variable aleatoria. Sea $a$ una constante o variable aleatoria que puede ser infinita cuando $\mu$ es finita, y considere las expresiones l\'imite:
\begin{eqnarray}
lim_{n\rightarrow\infty}n^{-1}Z\left(T_{n}\right)&=&a,\textrm{ c.s.}\\
lim_{t\rightarrow\infty}t^{-1}Z\left(t\right)&=&a/\mu,\textrm{ c.s.}
\end{eqnarray}
La segunda expresi\'on implica la primera. Conversamente, la primera implica la segunda si el proceso $Z\left(t\right)$ es creciente, o si $lim_{n\rightarrow\infty}n^{-1}M_{n}=0$ c.s.
\end{Teo}

\begin{Coro}
Si $N\left(t\right)$ es un proceso de renovaci\'on, y $\left(Z\left(T_{n}\right)-Z\left(T_{n-1}\right),M_{n}\right)$, para $n\geq1$, son variables aleatorias independientes e id\'enticamente distribuidas con media finita, entonces,
\begin{eqnarray}
lim_{t\rightarrow\infty}t^{-1}Z\left(t\right)\rightarrow\frac{\esp\left[Z\left(T_{1}\right)-Z\left(T_{0}\right)\right]}{\esp\left[T_{1}\right]},\textrm{ c.s. cuando  }t\rightarrow\infty.
\end{eqnarray}
\end{Coro}


%___________________________________________________________________________________________
%
%\subsection*{Funci\'on de Renovaci\'on}
%___________________________________________________________________________________________
%


\begin{Def}
Sea $h\left(t\right)$ funci\'on de valores reales en $\rea$ acotada en intervalos finitos e igual a cero para $t<0$ La ecuaci\'on de renovaci\'on para $h\left(t\right)$ y la distribuci\'on $F$ es

\begin{eqnarray}\label{Ec.Renovacion}
H\left(t\right)=h\left(t\right)+\int_{\left[0,t\right]}H\left(t-s\right)dF\left(s\right)\textrm{,    }t\geq0,
\end{eqnarray}
donde $H\left(t\right)$ es una funci\'on de valores reales. Esto es $H=h+F\star H$. Decimos que $H\left(t\right)$ es soluci\'on de esta ecuaci\'on si satisface la ecuaci\'on, y es acotada en intervalos finitos e iguales a cero para $t<0$.
\end{Def}

\begin{Prop}
La funci\'on $U\star h\left(t\right)$ es la \'unica soluci\'on de la ecuaci\'on de renovaci\'on (\ref{Ec.Renovacion}).
\end{Prop}

\begin{Teo}[Teorema Renovaci\'on Elemental]
\begin{eqnarray*}
t^{-1}U\left(t\right)\rightarrow 1/\mu\textrm{,    cuando }t\rightarrow\infty.
\end{eqnarray*}
\end{Teo}

%___________________________________________________________________________________________
%
%\subsection{Funci\'on de Renovaci\'on}
%___________________________________________________________________________________________
%


Sup\'ongase que $N\left(t\right)$ es un proceso de renovaci\'on con distribuci\'on $F$ con media finita $\mu$.

\begin{Def}
La funci\'on de renovaci\'on asociada con la distribuci\'on $F$, del proceso $N\left(t\right)$, es
\begin{eqnarray*}
U\left(t\right)=\sum_{n=1}^{\infty}F^{n\star}\left(t\right),\textrm{   }t\geq0,
\end{eqnarray*}
donde $F^{0\star}\left(t\right)=\indora\left(t\geq0\right)$.
\end{Def}


\begin{Prop}
Sup\'ongase que la distribuci\'on de inter-renovaci\'on $F$ tiene densidad $f$. Entonces $U\left(t\right)$ tambi\'en tiene densidad, para $t>0$, y es $U^{'}\left(t\right)=\sum_{n=0}^{\infty}f^{n\star}\left(t\right)$. Adem\'as
\begin{eqnarray*}
\prob\left\{N\left(t\right)>N\left(t-\right)\right\}=0\textrm{,   }t\geq0.
\end{eqnarray*}
\end{Prop}

\begin{Def}
La Transformada de Laplace-Stieljes de $F$ est\'a dada por

\begin{eqnarray*}
\hat{F}\left(\alpha\right)=\int_{\rea_{+}}e^{-\alpha t}dF\left(t\right)\textrm{,  }\alpha\geq0.
\end{eqnarray*}
\end{Def}

Entonces

\begin{eqnarray*}
\hat{U}\left(\alpha\right)=\sum_{n=0}^{\infty}\hat{F^{n\star}}\left(\alpha\right)=\sum_{n=0}^{\infty}\hat{F}\left(\alpha\right)^{n}=\frac{1}{1-\hat{F}\left(\alpha\right)}.
\end{eqnarray*}


\begin{Prop}
La Transformada de Laplace $\hat{U}\left(\alpha\right)$ y $\hat{F}\left(\alpha\right)$ determina una a la otra de manera \'unica por la relaci\'on $\hat{U}\left(\alpha\right)=\frac{1}{1-\hat{F}\left(\alpha\right)}$.
\end{Prop}


\begin{Note}
Un proceso de renovaci\'on $N\left(t\right)$ cuyos tiempos de inter-renovaci\'on tienen media finita, es un proceso Poisson con tasa $\lambda$ si y s\'olo s\'i $\esp\left[U\left(t\right)\right]=\lambda t$, para $t\geq0$.
\end{Note}


\begin{Teo}
Sea $N\left(t\right)$ un proceso puntual simple con puntos de localizaci\'on $T_{n}$ tal que $\eta\left(t\right)=\esp\left[N\left(\right)\right]$ es finita para cada $t$. Entonces para cualquier funci\'on $f:\rea_{+}\rightarrow\rea$,
\begin{eqnarray*}
\esp\left[\sum_{n=1}^{N\left(\right)}f\left(T_{n}\right)\right]=\int_{\left(0,t\right]}f\left(s\right)d\eta\left(s\right)\textrm{,  }t\geq0,
\end{eqnarray*}
suponiendo que la integral exista. Adem\'as si $X_{1},X_{2},\ldots$ son variables aleatorias definidas en el mismo espacio de probabilidad que el proceso $N\left(t\right)$ tal que $\esp\left[X_{n}|T_{n}=s\right]=f\left(s\right)$, independiente de $n$. Entonces
\begin{eqnarray*}
\esp\left[\sum_{n=1}^{N\left(t\right)}X_{n}\right]=\int_{\left(0,t\right]}f\left(s\right)d\eta\left(s\right)\textrm{,  }t\geq0,
\end{eqnarray*} 
suponiendo que la integral exista. 
\end{Teo}

\begin{Coro}[Identidad de Wald para Renovaciones]
Para el proceso de renovaci\'on $N\left(t\right)$,
\begin{eqnarray*}
\esp\left[T_{N\left(t\right)+1}\right]=\mu\esp\left[N\left(t\right)+1\right]\textrm{,  }t\geq0,
\end{eqnarray*}  
\end{Coro}

%______________________________________________________________________
%\subsection{Procesos de Renovaci\'on}
%______________________________________________________________________

\begin{Def}\label{Def.Tn}
Sean $0\leq T_{1}\leq T_{2}\leq \ldots$ son tiempos aleatorios infinitos en los cuales ocurren ciertos eventos. El n\'umero de tiempos $T_{n}$ en el intervalo $\left[0,t\right)$ es

\begin{eqnarray}
N\left(t\right)=\sum_{n=1}^{\infty}\indora\left(T_{n}\leq t\right),
\end{eqnarray}
para $t\geq0$.
\end{Def}

Si se consideran los puntos $T_{n}$ como elementos de $\rea_{+}$, y $N\left(t\right)$ es el n\'umero de puntos en $\rea$. El proceso denotado por $\left\{N\left(t\right):t\geq0\right\}$, denotado por $N\left(t\right)$, es un proceso puntual en $\rea_{+}$. Los $T_{n}$ son los tiempos de ocurrencia, el proceso puntual $N\left(t\right)$ es simple si su n\'umero de ocurrencias son distintas: $0<T_{1}<T_{2}<\ldots$ casi seguramente.

\begin{Def}
Un proceso puntual $N\left(t\right)$ es un proceso de renovaci\'on si los tiempos de interocurrencia $\xi_{n}=T_{n}-T_{n-1}$, para $n\geq1$, son independientes e identicamente distribuidos con distribuci\'on $F$, donde $F\left(0\right)=0$ y $T_{0}=0$. Los $T_{n}$ son llamados tiempos de renovaci\'on, referente a la independencia o renovaci\'on de la informaci\'on estoc\'astica en estos tiempos. Los $\xi_{n}$ son los tiempos de inter-renovaci\'on, y $N\left(t\right)$ es el n\'umero de renovaciones en el intervalo $\left[0,t\right)$
\end{Def}


\begin{Note}
Para definir un proceso de renovaci\'on para cualquier contexto, solamente hay que especificar una distribuci\'on $F$, con $F\left(0\right)=0$, para los tiempos de inter-renovaci\'on. La funci\'on $F$ en turno degune las otra variables aleatorias. De manera formal, existe un espacio de probabilidad y una sucesi\'on de variables aleatorias $\xi_{1},\xi_{2},\ldots$ definidas en este con distribuci\'on $F$. Entonces las otras cantidades son $T_{n}=\sum_{k=1}^{n}\xi_{k}$ y $N\left(t\right)=\sum_{n=1}^{\infty}\indora\left(T_{n}\leq t\right)$, donde $T_{n}\rightarrow\infty$ casi seguramente por la Ley Fuerte de los Grandes Números.
\end{Note}

\begin{Def}\label{Def.Tn}
Sean $0\leq T_{1}\leq T_{2}\leq \ldots$ son tiempos aleatorios infinitos en los cuales ocurren ciertos eventos. El n\'umero de tiempos $T_{n}$ en el intervalo $\left[0,t\right)$ es

\begin{eqnarray}
N\left(t\right)=\sum_{n=1}^{\infty}\indora\left(T_{n}\leq t\right),
\end{eqnarray}
para $t\geq0$.
\end{Def}

Si se consideran los puntos $T_{n}$ como elementos de $\rea_{+}$, y $N\left(t\right)$ es el n\'umero de puntos en $\rea$. El proceso denotado por $\left\{N\left(t\right):t\geq0\right\}$, denotado por $N\left(t\right)$, es un proceso puntual en $\rea_{+}$. Los $T_{n}$ son los tiempos de ocurrencia, el proceso puntual $N\left(t\right)$ es simple si su n\'umero de ocurrencias son distintas: $0<T_{1}<T_{2}<\ldots$ casi seguramente.

\begin{Def}
Un proceso puntual $N\left(t\right)$ es un proceso de renovaci\'on si los tiempos de interocurrencia $\xi_{n}=T_{n}-T_{n-1}$, para $n\geq1$, son independientes e identicamente distribuidos con distribuci\'on $F$, donde $F\left(0\right)=0$ y $T_{0}=0$. Los $T_{n}$ son llamados tiempos de renovaci\'on, referente a la independencia o renovaci\'on de la informaci\'on estoc\'astica en estos tiempos. Los $\xi_{n}$ son los tiempos de inter-renovaci\'on, y $N\left(t\right)$ es el n\'umero de renovaciones en el intervalo $\left[0,t\right)$
\end{Def}


\begin{Note}
Para definir un proceso de renovaci\'on para cualquier contexto, solamente hay que especificar una distribuci\'on $F$, con $F\left(0\right)=0$, para los tiempos de inter-renovaci\'on. La funci\'on $F$ en turno degune las otra variables aleatorias. De manera formal, existe un espacio de probabilidad y una sucesi\'on de variables aleatorias $\xi_{1},\xi_{2},\ldots$ definidas en este con distribuci\'on $F$. Entonces las otras cantidades son $T_{n}=\sum_{k=1}^{n}\xi_{k}$ y $N\left(t\right)=\sum_{n=1}^{\infty}\indora\left(T_{n}\leq t\right)$, donde $T_{n}\rightarrow\infty$ casi seguramente por la Ley Fuerte de los Grandes N\'umeros.
\end{Note}







Los tiempos $T_{n}$ est\'an relacionados con los conteos de $N\left(t\right)$ por

\begin{eqnarray*}
\left\{N\left(t\right)\geq n\right\}&=&\left\{T_{n}\leq t\right\}\\
T_{N\left(t\right)}\leq &t&<T_{N\left(t\right)+1},
\end{eqnarray*}

adem\'as $N\left(T_{n}\right)=n$, y 

\begin{eqnarray*}
N\left(t\right)=\max\left\{n:T_{n}\leq t\right\}=\min\left\{n:T_{n+1}>t\right\}
\end{eqnarray*}

Por propiedades de la convoluci\'on se sabe que

\begin{eqnarray*}
P\left\{T_{n}\leq t\right\}=F^{n\star}\left(t\right)
\end{eqnarray*}
que es la $n$-\'esima convoluci\'on de $F$. Entonces 

\begin{eqnarray*}
\left\{N\left(t\right)\geq n\right\}&=&\left\{T_{n}\leq t\right\}\\
P\left\{N\left(t\right)\leq n\right\}&=&1-F^{\left(n+1\right)\star}\left(t\right)
\end{eqnarray*}

Adem\'as usando el hecho de que $\esp\left[N\left(t\right)\right]=\sum_{n=1}^{\infty}P\left\{N\left(t\right)\geq n\right\}$
se tiene que

\begin{eqnarray*}
\esp\left[N\left(t\right)\right]=\sum_{n=1}^{\infty}F^{n\star}\left(t\right)
\end{eqnarray*}

\begin{Prop}
Para cada $t\geq0$, la funci\'on generadora de momentos $\esp\left[e^{\alpha N\left(t\right)}\right]$ existe para alguna $\alpha$ en una vecindad del 0, y de aqu\'i que $\esp\left[N\left(t\right)^{m}\right]<\infty$, para $m\geq1$.
\end{Prop}

\begin{Ejem}[\textbf{Proceso Poisson}]

Suponga que se tienen tiempos de inter-renovaci\'on \textit{i.i.d.} del proceso de renovaci\'on $N\left(t\right)$ tienen distribuci\'on exponencial $F\left(t\right)=q-e^{-\lambda t}$ con tasa $\lambda$. Entonces $N\left(t\right)$ es un proceso Poisson con tasa $\lambda$.

\end{Ejem}


\begin{Note}
Si el primer tiempo de renovaci\'on $\xi_{1}$ no tiene la misma distribuci\'on que el resto de las $\xi_{n}$, para $n\geq2$, a $N\left(t\right)$ se le llama Proceso de Renovaci\'on retardado, donde si $\xi$ tiene distribuci\'on $G$, entonces el tiempo $T_{n}$ de la $n$-\'esima renovaci\'on tiene distribuci\'on $G\star F^{\left(n-1\right)\star}\left(t\right)$
\end{Note}


\begin{Teo}
Para una constante $\mu\leq\infty$ ( o variable aleatoria), las siguientes expresiones son equivalentes:

\begin{eqnarray}
lim_{n\rightarrow\infty}n^{-1}T_{n}&=&\mu,\textrm{ c.s.}\\
lim_{t\rightarrow\infty}t^{-1}N\left(t\right)&=&1/\mu,\textrm{ c.s.}
\end{eqnarray}
\end{Teo}


Es decir, $T_{n}$ satisface la Ley Fuerte de los Grandes N\'umeros s\'i y s\'olo s\'i $N\left/t\right)$ la cumple.


\begin{Coro}[Ley Fuerte de los Grandes N\'umeros para Procesos de Renovaci\'on]
Si $N\left(t\right)$ es un proceso de renovaci\'on cuyos tiempos de inter-renovaci\'on tienen media $\mu\leq\infty$, entonces
\begin{eqnarray}
t^{-1}N\left(t\right)\rightarrow 1/\mu,\textrm{ c.s. cuando }t\rightarrow\infty.
\end{eqnarray}

\end{Coro}


Considerar el proceso estoc\'astico de valores reales $\left\{Z\left(t\right):t\geq0\right\}$ en el mismo espacio de probabilidad que $N\left(t\right)$

\begin{Def}
Para el proceso $\left\{Z\left(t\right):t\geq0\right\}$ se define la fluctuaci\'on m\'axima de $Z\left(t\right)$ en el intervalo $\left(T_{n-1},T_{n}\right]$:
\begin{eqnarray*}
M_{n}=\sup_{T_{n-1}<t\leq T_{n}}|Z\left(t\right)-Z\left(T_{n-1}\right)|
\end{eqnarray*}
\end{Def}

\begin{Teo}
Sup\'ongase que $n^{-1}T_{n}\rightarrow\mu$ c.s. cuando $n\rightarrow\infty$, donde $\mu\leq\infty$ es una constante o variable aleatoria. Sea $a$ una constante o variable aleatoria que puede ser infinita cuando $\mu$ es finita, y considere las expresiones l\'imite:
\begin{eqnarray}
lim_{n\rightarrow\infty}n^{-1}Z\left(T_{n}\right)&=&a,\textrm{ c.s.}\\
lim_{t\rightarrow\infty}t^{-1}Z\left(t\right)&=&a/\mu,\textrm{ c.s.}
\end{eqnarray}
La segunda expresi\'on implica la primera. Conversamente, la primera implica la segunda si el proceso $Z\left(t\right)$ es creciente, o si $lim_{n\rightarrow\infty}n^{-1}M_{n}=0$ c.s.
\end{Teo}

\begin{Coro}
Si $N\left(t\right)$ es un proceso de renovaci\'on, y $\left(Z\left(T_{n}\right)-Z\left(T_{n-1}\right),M_{n}\right)$, para $n\geq1$, son variables aleatorias independientes e id\'enticamente distribuidas con media finita, entonces,
\begin{eqnarray}
lim_{t\rightarrow\infty}t^{-1}Z\left(t\right)\rightarrow\frac{\esp\left[Z\left(T_{1}\right)-Z\left(T_{0}\right)\right]}{\esp\left[T_{1}\right]},\textrm{ c.s. cuando  }t\rightarrow\infty.
\end{eqnarray}
\end{Coro}


Sup\'ongase que $N\left(t\right)$ es un proceso de renovaci\'on con distribuci\'on $F$ con media finita $\mu$.

\begin{Def}
La funci\'on de renovaci\'on asociada con la distribuci\'on $F$, del proceso $N\left(t\right)$, es
\begin{eqnarray*}
U\left(t\right)=\sum_{n=1}^{\infty}F^{n\star}\left(t\right),\textrm{   }t\geq0,
\end{eqnarray*}
donde $F^{0\star}\left(t\right)=\indora\left(t\geq0\right)$.
\end{Def}


\begin{Prop}
Sup\'ongase que la distribuci\'on de inter-renovaci\'on $F$ tiene densidad $f$. Entonces $U\left(t\right)$ tambi\'en tiene densidad, para $t>0$, y es $U^{'}\left(t\right)=\sum_{n=0}^{\infty}f^{n\star}\left(t\right)$. Adem\'as
\begin{eqnarray*}
\prob\left\{N\left(t\right)>N\left(t-\right)\right\}=0\textrm{,   }t\geq0.
\end{eqnarray*}
\end{Prop}

\begin{Def}
La Transformada de Laplace-Stieljes de $F$ est\'a dada por

\begin{eqnarray*}
\hat{F}\left(\alpha\right)=\int_{\rea_{+}}e^{-\alpha t}dF\left(t\right)\textrm{,  }\alpha\geq0.
\end{eqnarray*}
\end{Def}

Entonces

\begin{eqnarray*}
\hat{U}\left(\alpha\right)=\sum_{n=0}^{\infty}\hat{F^{n\star}}\left(\alpha\right)=\sum_{n=0}^{\infty}\hat{F}\left(\alpha\right)^{n}=\frac{1}{1-\hat{F}\left(\alpha\right)}.
\end{eqnarray*}


\begin{Prop}
La Transformada de Laplace $\hat{U}\left(\alpha\right)$ y $\hat{F}\left(\alpha\right)$ determina una a la otra de manera \'unica por la relaci\'on $\hat{U}\left(\alpha\right)=\frac{1}{1-\hat{F}\left(\alpha\right)}$.
\end{Prop}


\begin{Note}
Un proceso de renovaci\'on $N\left(t\right)$ cuyos tiempos de inter-renovaci\'on tienen media finita, es un proceso Poisson con tasa $\lambda$ si y s\'olo s\'i $\esp\left[U\left(t\right)\right]=\lambda t$, para $t\geq0$.
\end{Note}


\begin{Teo}
Sea $N\left(t\right)$ un proceso puntual simple con puntos de localizaci\'on $T_{n}$ tal que $\eta\left(t\right)=\esp\left[N\left(\right)\right]$ es finita para cada $t$. Entonces para cualquier funci\'on $f:\rea_{+}\rightarrow\rea$,
\begin{eqnarray*}
\esp\left[\sum_{n=1}^{N\left(\right)}f\left(T_{n}\right)\right]=\int_{\left(0,t\right]}f\left(s\right)d\eta\left(s\right)\textrm{,  }t\geq0,
\end{eqnarray*}
suponiendo que la integral exista. Adem\'as si $X_{1},X_{2},\ldots$ son variables aleatorias definidas en el mismo espacio de probabilidad que el proceso $N\left(t\right)$ tal que $\esp\left[X_{n}|T_{n}=s\right]=f\left(s\right)$, independiente de $n$. Entonces
\begin{eqnarray*}
\esp\left[\sum_{n=1}^{N\left(t\right)}X_{n}\right]=\int_{\left(0,t\right]}f\left(s\right)d\eta\left(s\right)\textrm{,  }t\geq0,
\end{eqnarray*} 
suponiendo que la integral exista. 
\end{Teo}

\begin{Coro}[Identidad de Wald para Renovaciones]
Para el proceso de renovaci\'on $N\left(t\right)$,
\begin{eqnarray*}
\esp\left[T_{N\left(t\right)+1}\right]=\mu\esp\left[N\left(t\right)+1\right]\textrm{,  }t\geq0,
\end{eqnarray*}  
\end{Coro}


\begin{Def}
Sea $h\left(t\right)$ funci\'on de valores reales en $\rea$ acotada en intervalos finitos e igual a cero para $t<0$ La ecuaci\'on de renovaci\'on para $h\left(t\right)$ y la distribuci\'on $F$ es

\begin{eqnarray}\label{Ec.Renovacion}
H\left(t\right)=h\left(t\right)+\int_{\left[0,t\right]}H\left(t-s\right)dF\left(s\right)\textrm{,    }t\geq0,
\end{eqnarray}
donde $H\left(t\right)$ es una funci\'on de valores reales. Esto es $H=h+F\star H$. Decimos que $H\left(t\right)$ es soluci\'on de esta ecuaci\'on si satisface la ecuaci\'on, y es acotada en intervalos finitos e iguales a cero para $t<0$.
\end{Def}

\begin{Prop}
La funci\'on $U\star h\left(t\right)$ es la \'unica soluci\'on de la ecuaci\'on de renovaci\'on (\ref{Ec.Renovacion}).
\end{Prop}

\begin{Teo}[Teorema Renovaci\'on Elemental]
\begin{eqnarray*}
t^{-1}U\left(t\right)\rightarrow 1/\mu\textrm{,    cuando }t\rightarrow\infty.
\end{eqnarray*}
\end{Teo}



Sup\'ongase que $N\left(t\right)$ es un proceso de renovaci\'on con distribuci\'on $F$ con media finita $\mu$.

\begin{Def}
La funci\'on de renovaci\'on asociada con la distribuci\'on $F$, del proceso $N\left(t\right)$, es
\begin{eqnarray*}
U\left(t\right)=\sum_{n=1}^{\infty}F^{n\star}\left(t\right),\textrm{   }t\geq0,
\end{eqnarray*}
donde $F^{0\star}\left(t\right)=\indora\left(t\geq0\right)$.
\end{Def}


\begin{Prop}
Sup\'ongase que la distribuci\'on de inter-renovaci\'on $F$ tiene densidad $f$. Entonces $U\left(t\right)$ tambi\'en tiene densidad, para $t>0$, y es $U^{'}\left(t\right)=\sum_{n=0}^{\infty}f^{n\star}\left(t\right)$. Adem\'as
\begin{eqnarray*}
\prob\left\{N\left(t\right)>N\left(t-\right)\right\}=0\textrm{,   }t\geq0.
\end{eqnarray*}
\end{Prop}

\begin{Def}
La Transformada de Laplace-Stieljes de $F$ est\'a dada por

\begin{eqnarray*}
\hat{F}\left(\alpha\right)=\int_{\rea_{+}}e^{-\alpha t}dF\left(t\right)\textrm{,  }\alpha\geq0.
\end{eqnarray*}
\end{Def}

Entonces

\begin{eqnarray*}
\hat{U}\left(\alpha\right)=\sum_{n=0}^{\infty}\hat{F^{n\star}}\left(\alpha\right)=\sum_{n=0}^{\infty}\hat{F}\left(\alpha\right)^{n}=\frac{1}{1-\hat{F}\left(\alpha\right)}.
\end{eqnarray*}


\begin{Prop}
La Transformada de Laplace $\hat{U}\left(\alpha\right)$ y $\hat{F}\left(\alpha\right)$ determina una a la otra de manera \'unica por la relaci\'on $\hat{U}\left(\alpha\right)=\frac{1}{1-\hat{F}\left(\alpha\right)}$.
\end{Prop}


\begin{Note}
Un proceso de renovaci\'on $N\left(t\right)$ cuyos tiempos de inter-renovaci\'on tienen media finita, es un proceso Poisson con tasa $\lambda$ si y s\'olo s\'i $\esp\left[U\left(t\right)\right]=\lambda t$, para $t\geq0$.
\end{Note}


\begin{Teo}
Sea $N\left(t\right)$ un proceso puntual simple con puntos de localizaci\'on $T_{n}$ tal que $\eta\left(t\right)=\esp\left[N\left(\right)\right]$ es finita para cada $t$. Entonces para cualquier funci\'on $f:\rea_{+}\rightarrow\rea$,
\begin{eqnarray*}
\esp\left[\sum_{n=1}^{N\left(\right)}f\left(T_{n}\right)\right]=\int_{\left(0,t\right]}f\left(s\right)d\eta\left(s\right)\textrm{,  }t\geq0,
\end{eqnarray*}
suponiendo que la integral exista. Adem\'as si $X_{1},X_{2},\ldots$ son variables aleatorias definidas en el mismo espacio de probabilidad que el proceso $N\left(t\right)$ tal que $\esp\left[X_{n}|T_{n}=s\right]=f\left(s\right)$, independiente de $n$. Entonces
\begin{eqnarray*}
\esp\left[\sum_{n=1}^{N\left(t\right)}X_{n}\right]=\int_{\left(0,t\right]}f\left(s\right)d\eta\left(s\right)\textrm{,  }t\geq0,
\end{eqnarray*} 
suponiendo que la integral exista. 
\end{Teo}

\begin{Coro}[Identidad de Wald para Renovaciones]
Para el proceso de renovaci\'on $N\left(t\right)$,
\begin{eqnarray*}
\esp\left[T_{N\left(t\right)+1}\right]=\mu\esp\left[N\left(t\right)+1\right]\textrm{,  }t\geq0,
\end{eqnarray*}  
\end{Coro}


\begin{Def}
Sea $h\left(t\right)$ funci\'on de valores reales en $\rea$ acotada en intervalos finitos e igual a cero para $t<0$ La ecuaci\'on de renovaci\'on para $h\left(t\right)$ y la distribuci\'on $F$ es

\begin{eqnarray}\label{Ec.Renovacion}
H\left(t\right)=h\left(t\right)+\int_{\left[0,t\right]}H\left(t-s\right)dF\left(s\right)\textrm{,    }t\geq0,
\end{eqnarray}
donde $H\left(t\right)$ es una funci\'on de valores reales. Esto es $H=h+F\star H$. Decimos que $H\left(t\right)$ es soluci\'on de esta ecuaci\'on si satisface la ecuaci\'on, y es acotada en intervalos finitos e iguales a cero para $t<0$.
\end{Def}

\begin{Prop}
La funci\'on $U\star h\left(t\right)$ es la \'unica soluci\'on de la ecuaci\'on de renovaci\'on (\ref{Ec.Renovacion}).
\end{Prop}

\begin{Teo}[Teorema Renovaci\'on Elemental]
\begin{eqnarray*}
t^{-1}U\left(t\right)\rightarrow 1/\mu\textrm{,    cuando }t\rightarrow\infty.
\end{eqnarray*}
\end{Teo}


\begin{Note} Una funci\'on $h:\rea_{+}\rightarrow\rea$ es Directamente Riemann Integrable en los siguientes casos:
\begin{itemize}
\item[a)] $h\left(t\right)\geq0$ es decreciente y Riemann Integrable.
\item[b)] $h$ es continua excepto posiblemente en un conjunto de Lebesgue de medida 0, y $|h\left(t\right)|\leq b\left(t\right)$, donde $b$ es DRI.
\end{itemize}
\end{Note}

\begin{Teo}[Teorema Principal de Renovaci\'on]
Si $F$ es no aritm\'etica y $h\left(t\right)$ es Directamente Riemann Integrable (DRI), entonces

\begin{eqnarray*}
lim_{t\rightarrow\infty}U\star h=\frac{1}{\mu}\int_{\rea_{+}}h\left(s\right)ds.
\end{eqnarray*}
\end{Teo}

\begin{Prop}
Cualquier funci\'on $H\left(t\right)$ acotada en intervalos finitos y que es 0 para $t<0$ puede expresarse como
\begin{eqnarray*}
H\left(t\right)=U\star h\left(t\right)\textrm{,  donde }h\left(t\right)=H\left(t\right)-F\star H\left(t\right)
\end{eqnarray*}
\end{Prop}

\begin{Def}
Un proceso estoc\'astico $X\left(t\right)$ es crudamente regenerativo en un tiempo aleatorio positivo $T$ si
\begin{eqnarray*}
\esp\left[X\left(T+t\right)|T\right]=\esp\left[X\left(t\right)\right]\textrm{, para }t\geq0,\end{eqnarray*}
y con las esperanzas anteriores finitas.
\end{Def}

\begin{Prop}
Sup\'ongase que $X\left(t\right)$ es un proceso crudamente regenerativo en $T$, que tiene distribuci\'on $F$. Si $\esp\left[X\left(t\right)\right]$ es acotado en intervalos finitos, entonces
\begin{eqnarray*}
\esp\left[X\left(t\right)\right]=U\star h\left(t\right)\textrm{,  donde }h\left(t\right)=\esp\left[X\left(t\right)\indora\left(T>t\right)\right].
\end{eqnarray*}
\end{Prop}

\begin{Teo}[Regeneraci\'on Cruda]
Sup\'ongase que $X\left(t\right)$ es un proceso con valores positivo crudamente regenerativo en $T$, y def\'inase $M=\sup\left\{|X\left(t\right)|:t\leq T\right\}$. Si $T$ es no aritm\'etico y $M$ y $MT$ tienen media finita, entonces
\begin{eqnarray*}
lim_{t\rightarrow\infty}\esp\left[X\left(t\right)\right]=\frac{1}{\mu}\int_{\rea_{+}}h\left(s\right)ds,
\end{eqnarray*}
donde $h\left(t\right)=\esp\left[X\left(t\right)\indora\left(T>t\right)\right]$.
\end{Teo}


\begin{Note} Una funci\'on $h:\rea_{+}\rightarrow\rea$ es Directamente Riemann Integrable en los siguientes casos:
\begin{itemize}
\item[a)] $h\left(t\right)\geq0$ es decreciente y Riemann Integrable.
\item[b)] $h$ es continua excepto posiblemente en un conjunto de Lebesgue de medida 0, y $|h\left(t\right)|\leq b\left(t\right)$, donde $b$ es DRI.
\end{itemize}
\end{Note}

\begin{Teo}[Teorema Principal de Renovaci\'on]
Si $F$ es no aritm\'etica y $h\left(t\right)$ es Directamente Riemann Integrable (DRI), entonces

\begin{eqnarray*}
lim_{t\rightarrow\infty}U\star h=\frac{1}{\mu}\int_{\rea_{+}}h\left(s\right)ds.
\end{eqnarray*}
\end{Teo}

\begin{Prop}
Cualquier funci\'on $H\left(t\right)$ acotada en intervalos finitos y que es 0 para $t<0$ puede expresarse como
\begin{eqnarray*}
H\left(t\right)=U\star h\left(t\right)\textrm{,  donde }h\left(t\right)=H\left(t\right)-F\star H\left(t\right)
\end{eqnarray*}
\end{Prop}

\begin{Def}
Un proceso estoc\'astico $X\left(t\right)$ es crudamente regenerativo en un tiempo aleatorio positivo $T$ si
\begin{eqnarray*}
\esp\left[X\left(T+t\right)|T\right]=\esp\left[X\left(t\right)\right]\textrm{, para }t\geq0,\end{eqnarray*}
y con las esperanzas anteriores finitas.
\end{Def}

\begin{Prop}
Sup\'ongase que $X\left(t\right)$ es un proceso crudamente regenerativo en $T$, que tiene distribuci\'on $F$. Si $\esp\left[X\left(t\right)\right]$ es acotado en intervalos finitos, entonces
\begin{eqnarray*}
\esp\left[X\left(t\right)\right]=U\star h\left(t\right)\textrm{,  donde }h\left(t\right)=\esp\left[X\left(t\right)\indora\left(T>t\right)\right].
\end{eqnarray*}
\end{Prop}

\begin{Teo}[Regeneraci\'on Cruda]
Sup\'ongase que $X\left(t\right)$ es un proceso con valores positivo crudamente regenerativo en $T$, y def\'inase $M=\sup\left\{|X\left(t\right)|:t\leq T\right\}$. Si $T$ es no aritm\'etico y $M$ y $MT$ tienen media finita, entonces
\begin{eqnarray*}
lim_{t\rightarrow\infty}\esp\left[X\left(t\right)\right]=\frac{1}{\mu}\int_{\rea_{+}}h\left(s\right)ds,
\end{eqnarray*}
donde $h\left(t\right)=\esp\left[X\left(t\right)\indora\left(T>t\right)\right]$.
\end{Teo}

\begin{Def}
Para el proceso $\left\{\left(N\left(t\right),X\left(t\right)\right):t\geq0\right\}$, sus trayectoria muestrales en el intervalo de tiempo $\left[T_{n-1},T_{n}\right)$ est\'an descritas por
\begin{eqnarray*}
\zeta_{n}=\left(\xi_{n},\left\{X\left(T_{n-1}+t\right):0\leq t<\xi_{n}\right\}\right)
\end{eqnarray*}
Este $\zeta_{n}$ es el $n$-\'esimo segmento del proceso. El proceso es regenerativo sobre los tiempos $T_{n}$ si sus segmentos $\zeta_{n}$ son independientes e id\'enticamennte distribuidos.
\end{Def}


\begin{Note}
Si $\tilde{X}\left(t\right)$ con espacio de estados $\tilde{S}$ es regenerativo sobre $T_{n}$, entonces $X\left(t\right)=f\left(\tilde{X}\left(t\right)\right)$ tambi\'en es regenerativo sobre $T_{n}$, para cualquier funci\'on $f:\tilde{S}\rightarrow S$.
\end{Note}

\begin{Note}
Los procesos regenerativos son crudamente regenerativos, pero no al rev\'es.
\end{Note}


\begin{Note}
Un proceso estoc\'astico a tiempo continuo o discreto es regenerativo si existe un proceso de renovaci\'on  tal que los segmentos del proceso entre tiempos de renovaci\'on sucesivos son i.i.d., es decir, para $\left\{X\left(t\right):t\geq0\right\}$ proceso estoc\'astico a tiempo continuo con espacio de estados $S$, espacio m\'etrico.
\end{Note}

Para $\left\{X\left(t\right):t\geq0\right\}$ Proceso Estoc\'astico a tiempo continuo con estado de espacios $S$, que es un espacio m\'etrico, con trayectorias continuas por la derecha y con l\'imites por la izquierda c.s. Sea $N\left(t\right)$ un proceso de renovaci\'on en $\rea_{+}$ definido en el mismo espacio de probabilidad que $X\left(t\right)$, con tiempos de renovaci\'on $T$ y tiempos de inter-renovaci\'on $\xi_{n}=T_{n}-T_{n-1}$, con misma distribuci\'on $F$ de media finita $\mu$.



\begin{Def}
Para el proceso $\left\{\left(N\left(t\right),X\left(t\right)\right):t\geq0\right\}$, sus trayectoria muestrales en el intervalo de tiempo $\left[T_{n-1},T_{n}\right)$ est\'an descritas por
\begin{eqnarray*}
\zeta_{n}=\left(\xi_{n},\left\{X\left(T_{n-1}+t\right):0\leq t<\xi_{n}\right\}\right)
\end{eqnarray*}
Este $\zeta_{n}$ es el $n$-\'esimo segmento del proceso. El proceso es regenerativo sobre los tiempos $T_{n}$ si sus segmentos $\zeta_{n}$ son independientes e id\'enticamennte distribuidos.
\end{Def}

\begin{Note}
Un proceso regenerativo con media de la longitud de ciclo finita es llamado positivo recurrente.
\end{Note}

\begin{Teo}[Procesos Regenerativos]
Suponga que el proceso
\end{Teo}


\begin{Def}[Renewal Process Trinity]
Para un proceso de renovaci\'on $N\left(t\right)$, los siguientes procesos proveen de informaci\'on sobre los tiempos de renovaci\'on.
\begin{itemize}
\item $A\left(t\right)=t-T_{N\left(t\right)}$, el tiempo de recurrencia hacia atr\'as al tiempo $t$, que es el tiempo desde la \'ultima renovaci\'on para $t$.

\item $B\left(t\right)=T_{N\left(t\right)+1}-t$, el tiempo de recurrencia hacia adelante al tiempo $t$, residual del tiempo de renovaci\'on, que es el tiempo para la pr\'oxima renovaci\'on despu\'es de $t$.

\item $L\left(t\right)=\xi_{N\left(t\right)+1}=A\left(t\right)+B\left(t\right)$, la longitud del intervalo de renovaci\'on que contiene a $t$.
\end{itemize}
\end{Def}

\begin{Note}
El proceso tridimensional $\left(A\left(t\right),B\left(t\right),L\left(t\right)\right)$ es regenerativo sobre $T_{n}$, y por ende cada proceso lo es. Cada proceso $A\left(t\right)$ y $B\left(t\right)$ son procesos de MArkov a tiempo continuo con trayectorias continuas por partes en el espacio de estados $\rea_{+}$. Una expresi\'on conveniente para su distribuci\'on conjunta es, para $0\leq x<t,y\geq0$
\begin{equation}\label{NoRenovacion}
P\left\{A\left(t\right)>x,B\left(t\right)>y\right\}=
P\left\{N\left(t+y\right)-N\left((t-x)\right)=0\right\}
\end{equation}
\end{Note}

\begin{Ejem}[Tiempos de recurrencia Poisson]
Si $N\left(t\right)$ es un proceso Poisson con tasa $\lambda$, entonces de la expresi\'on (\ref{NoRenovacion}) se tiene que

\begin{eqnarray*}
\begin{array}{lc}
P\left\{A\left(t\right)>x,B\left(t\right)>y\right\}=e^{-\lambda\left(x+y\right)},&0\leq x<t,y\geq0,
\end{array}
\end{eqnarray*}
que es la probabilidad Poisson de no renovaciones en un intervalo de longitud $x+y$.

\end{Ejem}

\begin{Note}
Una cadena de Markov erg\'odica tiene la propiedad de ser estacionaria si la distribuci\'on de su estado al tiempo $0$ es su distribuci\'on estacionaria.
\end{Note}


\begin{Def}
Un proceso estoc\'astico a tiempo continuo $\left\{X\left(t\right):t\geq0\right\}$ en un espacio general es estacionario si sus distribuciones finito dimensionales son invariantes bajo cualquier  traslado: para cada $0\leq s_{1}<s_{2}<\cdots<s_{k}$ y $t\geq0$,
\begin{eqnarray*}
\left(X\left(s_{1}+t\right),\ldots,X\left(s_{k}+t\right)\right)=_{d}\left(X\left(s_{1}\right),\ldots,X\left(s_{k}\right)\right).
\end{eqnarray*}
\end{Def}

\begin{Note}
Un proceso de Markov es estacionario si $X\left(t\right)=_{d}X\left(0\right)$, $t\geq0$.
\end{Note}

Considerese el proceso $N\left(t\right)=\sum_{n}\indora\left(\tau_{n}\leq t\right)$ en $\rea_{+}$, con puntos $0<\tau_{1}<\tau_{2}<\cdots$.

\begin{Prop}
Si $N$ es un proceso puntual estacionario y $\esp\left[N\left(1\right)\right]<\infty$, entonces $\esp\left[N\left(t\right)\right]=t\esp\left[N\left(1\right)\right]$, $t\geq0$

\end{Prop}

\begin{Teo}
Los siguientes enunciados son equivalentes
\begin{itemize}
\item[i)] El proceso retardado de renovaci\'on $N$ es estacionario.

\item[ii)] EL proceso de tiempos de recurrencia hacia adelante $B\left(t\right)$ es estacionario.


\item[iii)] $\esp\left[N\left(t\right)\right]=t/\mu$,


\item[iv)] $G\left(t\right)=F_{e}\left(t\right)=\frac{1}{\mu}\int_{0}^{t}\left[1-F\left(s\right)\right]ds$
\end{itemize}
Cuando estos enunciados son ciertos, $P\left\{B\left(t\right)\leq x\right\}=F_{e}\left(x\right)$, para $t,x\geq0$.

\end{Teo}

\begin{Note}
Una consecuencia del teorema anterior es que el Proceso Poisson es el \'unico proceso sin retardo que es estacionario.
\end{Note}

\begin{Coro}
El proceso de renovaci\'on $N\left(t\right)$ sin retardo, y cuyos tiempos de inter renonaci\'on tienen media finita, es estacionario si y s\'olo si es un proceso Poisson.

\end{Coro}

%______________________________________________________________________

%\section{Ejemplos, Notas importantes}
%______________________________________________________________________
%\section*{Ap\'endice A}
%__________________________________________________________________

%________________________________________________________________________
%\subsection*{Procesos Regenerativos}
%________________________________________________________________________



\begin{Note}
Si $\tilde{X}\left(t\right)$ con espacio de estados $\tilde{S}$ es regenerativo sobre $T_{n}$, entonces $X\left(t\right)=f\left(\tilde{X}\left(t\right)\right)$ tambi\'en es regenerativo sobre $T_{n}$, para cualquier funci\'on $f:\tilde{S}\rightarrow S$.
\end{Note}

\begin{Note}
Los procesos regenerativos son crudamente regenerativos, pero no al rev\'es.
\end{Note}
%\subsection*{Procesos Regenerativos: Sigman\cite{Sigman1}}
\begin{Def}[Definici\'on Cl\'asica]
Un proceso estoc\'astico $X=\left\{X\left(t\right):t\geq0\right\}$ es llamado regenerativo is existe una variable aleatoria $R_{1}>0$ tal que
\begin{itemize}
\item[i)] $\left\{X\left(t+R_{1}\right):t\geq0\right\}$ es independiente de $\left\{\left\{X\left(t\right):t<R_{1}\right\},\right\}$
\item[ii)] $\left\{X\left(t+R_{1}\right):t\geq0\right\}$ es estoc\'asticamente equivalente a $\left\{X\left(t\right):t>0\right\}$
\end{itemize}

Llamamos a $R_{1}$ tiempo de regeneraci\'on, y decimos que $X$ se regenera en este punto.
\end{Def}

$\left\{X\left(t+R_{1}\right)\right\}$ es regenerativo con tiempo de regeneraci\'on $R_{2}$, independiente de $R_{1}$ pero con la misma distribuci\'on que $R_{1}$. Procediendo de esta manera se obtiene una secuencia de variables aleatorias independientes e id\'enticamente distribuidas $\left\{R_{n}\right\}$ llamados longitudes de ciclo. Si definimos a $Z_{k}\equiv R_{1}+R_{2}+\cdots+R_{k}$, se tiene un proceso de renovaci\'on llamado proceso de renovaci\'on encajado para $X$.




\begin{Def}
Para $x$ fijo y para cada $t\geq0$, sea $I_{x}\left(t\right)=1$ si $X\left(t\right)\leq x$,  $I_{x}\left(t\right)=0$ en caso contrario, y def\'inanse los tiempos promedio
\begin{eqnarray*}
\overline{X}&=&lim_{t\rightarrow\infty}\frac{1}{t}\int_{0}^{\infty}X\left(u\right)du\\
\prob\left(X_{\infty}\leq x\right)&=&lim_{t\rightarrow\infty}\frac{1}{t}\int_{0}^{\infty}I_{x}\left(u\right)du,
\end{eqnarray*}
cuando estos l\'imites existan.
\end{Def}

Como consecuencia del teorema de Renovaci\'on-Recompensa, se tiene que el primer l\'imite  existe y es igual a la constante
\begin{eqnarray*}
\overline{X}&=&\frac{\esp\left[\int_{0}^{R_{1}}X\left(t\right)dt\right]}{\esp\left[R_{1}\right]},
\end{eqnarray*}
suponiendo que ambas esperanzas son finitas.

\begin{Note}
\begin{itemize}
\item[a)] Si el proceso regenerativo $X$ es positivo recurrente y tiene trayectorias muestrales no negativas, entonces la ecuaci\'on anterior es v\'alida.
\item[b)] Si $X$ es positivo recurrente regenerativo, podemos construir una \'unica versi\'on estacionaria de este proceso, $X_{e}=\left\{X_{e}\left(t\right)\right\}$, donde $X_{e}$ es un proceso estoc\'astico regenerativo y estrictamente estacionario, con distribuci\'on marginal distribuida como $X_{\infty}$
\end{itemize}
\end{Note}

Para $\left\{X\left(t\right):t\geq0\right\}$ Proceso Estoc\'astico a tiempo continuo con estado de espacios $S$, que es un espacio m\'etrico, con trayectorias continuas por la derecha y con l\'imites por la izquierda c.s. Sea $N\left(t\right)$ un proceso de renovaci\'on en $\rea_{+}$ definido en el mismo espacio de probabilidad que $X\left(t\right)$, con tiempos de renovaci\'on $T$ y tiempos de inter-renovaci\'on $\xi_{n}=T_{n}-T_{n-1}$, con misma distribuci\'on $F$ de media finita $\mu$.


\begin{Def}
Para el proceso $\left\{\left(N\left(t\right),X\left(t\right)\right):t\geq0\right\}$, sus trayectoria muestrales en el intervalo de tiempo $\left[T_{n-1},T_{n}\right)$ est\'an descritas por
\begin{eqnarray*}
\zeta_{n}=\left(\xi_{n},\left\{X\left(T_{n-1}+t\right):0\leq t<\xi_{n}\right\}\right)
\end{eqnarray*}
Este $\zeta_{n}$ es el $n$-\'esimo segmento del proceso. El proceso es regenerativo sobre los tiempos $T_{n}$ si sus segmentos $\zeta_{n}$ son independientes e id\'enticamennte distribuidos.
\end{Def}


\begin{Note}
Si $\tilde{X}\left(t\right)$ con espacio de estados $\tilde{S}$ es regenerativo sobre $T_{n}$, entonces $X\left(t\right)=f\left(\tilde{X}\left(t\right)\right)$ tambi\'en es regenerativo sobre $T_{n}$, para cualquier funci\'on $f:\tilde{S}\rightarrow S$.
\end{Note}

\begin{Note}
Los procesos regenerativos son crudamente regenerativos, pero no al rev\'es.
\end{Note}

\begin{Def}[Definici\'on Cl\'asica]
Un proceso estoc\'astico $X=\left\{X\left(t\right):t\geq0\right\}$ es llamado regenerativo is existe una variable aleatoria $R_{1}>0$ tal que
\begin{itemize}
\item[i)] $\left\{X\left(t+R_{1}\right):t\geq0\right\}$ es independiente de $\left\{\left\{X\left(t\right):t<R_{1}\right\},\right\}$
\item[ii)] $\left\{X\left(t+R_{1}\right):t\geq0\right\}$ es estoc\'asticamente equivalente a $\left\{X\left(t\right):t>0\right\}$
\end{itemize}

Llamamos a $R_{1}$ tiempo de regeneraci\'on, y decimos que $X$ se regenera en este punto.
\end{Def}

$\left\{X\left(t+R_{1}\right)\right\}$ es regenerativo con tiempo de regeneraci\'on $R_{2}$, independiente de $R_{1}$ pero con la misma distribuci\'on que $R_{1}$. Procediendo de esta manera se obtiene una secuencia de variables aleatorias independientes e id\'enticamente distribuidas $\left\{R_{n}\right\}$ llamados longitudes de ciclo. Si definimos a $Z_{k}\equiv R_{1}+R_{2}+\cdots+R_{k}$, se tiene un proceso de renovaci\'on llamado proceso de renovaci\'on encajado para $X$.

\begin{Note}
Un proceso regenerativo con media de la longitud de ciclo finita es llamado positivo recurrente.
\end{Note}


\begin{Def}
Para $x$ fijo y para cada $t\geq0$, sea $I_{x}\left(t\right)=1$ si $X\left(t\right)\leq x$,  $I_{x}\left(t\right)=0$ en caso contrario, y def\'inanse los tiempos promedio
\begin{eqnarray*}
\overline{X}&=&lim_{t\rightarrow\infty}\frac{1}{t}\int_{0}^{\infty}X\left(u\right)du\\
\prob\left(X_{\infty}\leq x\right)&=&lim_{t\rightarrow\infty}\frac{1}{t}\int_{0}^{\infty}I_{x}\left(u\right)du,
\end{eqnarray*}
cuando estos l\'imites existan.
\end{Def}

Como consecuencia del teorema de Renovaci\'on-Recompensa, se tiene que el primer l\'imite  existe y es igual a la constante
\begin{eqnarray*}
\overline{X}&=&\frac{\esp\left[\int_{0}^{R_{1}}X\left(t\right)dt\right]}{\esp\left[R_{1}\right]},
\end{eqnarray*}
suponiendo que ambas esperanzas son finitas.

\begin{Note}
\begin{itemize}
\item[a)] Si el proceso regenerativo $X$ es positivo recurrente y tiene trayectorias muestrales no negativas, entonces la ecuaci\'on anterior es v\'alida.
\item[b)] Si $X$ es positivo recurrente regenerativo, podemos construir una \'unica versi\'on estacionaria de este proceso, $X_{e}=\left\{X_{e}\left(t\right)\right\}$, donde $X_{e}$ es un proceso estoc\'astico regenerativo y estrictamente estacionario, con distribuci\'on marginal distribuida como $X_{\infty}$
\end{itemize}
\end{Note}

%__________________________________________________________________________________________
%\subsection{Procesos Regenerativos Estacionarios - Stidham \cite{Stidham}}
%__________________________________________________________________________________________


Un proceso estoc\'astico a tiempo continuo $\left\{V\left(t\right),t\geq0\right\}$ es un proceso regenerativo si existe una sucesi\'on de variables aleatorias independientes e id\'enticamente distribuidas $\left\{X_{1},X_{2},\ldots\right\}$, sucesi\'on de renovaci\'on, tal que para cualquier conjunto de Borel $A$, 

\begin{eqnarray*}
\prob\left\{V\left(t\right)\in A|X_{1}+X_{2}+\cdots+X_{R\left(t\right)}=s,\left\{V\left(\tau\right),\tau<s\right\}\right\}=\prob\left\{V\left(t-s\right)\in A|X_{1}>t-s\right\},
\end{eqnarray*}
para todo $0\leq s\leq t$, donde $R\left(t\right)=\max\left\{X_{1}+X_{2}+\cdots+X_{j}\leq t\right\}=$n\'umero de renovaciones ({\emph{puntos de regeneraci\'on}}) que ocurren en $\left[0,t\right]$. El intervalo $\left[0,X_{1}\right)$ es llamado {\emph{primer ciclo de regeneraci\'on}} de $\left\{V\left(t \right),t\geq0\right\}$, $\left[X_{1},X_{1}+X_{2}\right)$ el {\emph{segundo ciclo de regeneraci\'on}}, y as\'i sucesivamente.

Sea $X=X_{1}$ y sea $F$ la funci\'on de distrbuci\'on de $X$


\begin{Def}
Se define el proceso estacionario, $\left\{V^{*}\left(t\right),t\geq0\right\}$, para $\left\{V\left(t\right),t\geq0\right\}$ por

\begin{eqnarray*}
\prob\left\{V\left(t\right)\in A\right\}=\frac{1}{\esp\left[X\right]}\int_{0}^{\infty}\prob\left\{V\left(t+x\right)\in A|X>x\right\}\left(1-F\left(x\right)\right)dx,
\end{eqnarray*} 
para todo $t\geq0$ y todo conjunto de Borel $A$.
\end{Def}

\begin{Def}
Una distribuci\'on se dice que es {\emph{aritm\'etica}} si todos sus puntos de incremento son m\'ultiplos de la forma $0,\lambda, 2\lambda,\ldots$ para alguna $\lambda>0$ entera.
\end{Def}


\begin{Def}
Una modificaci\'on medible de un proceso $\left\{V\left(t\right),t\geq0\right\}$, es una versi\'on de este, $\left\{V\left(t,w\right)\right\}$ conjuntamente medible para $t\geq0$ y para $w\in S$, $S$ espacio de estados para $\left\{V\left(t\right),t\geq0\right\}$.
\end{Def}

\begin{Teo}
Sea $\left\{V\left(t\right),t\geq\right\}$ un proceso regenerativo no negativo con modificaci\'on medible. Sea $\esp\left[X\right]<\infty$. Entonces el proceso estacionario dado por la ecuaci\'on anterior est\'a bien definido y tiene funci\'on de distribuci\'on independiente de $t$, adem\'as
\begin{itemize}
\item[i)] \begin{eqnarray*}
\esp\left[V^{*}\left(0\right)\right]&=&\frac{\esp\left[\int_{0}^{X}V\left(s\right)ds\right]}{\esp\left[X\right]}\end{eqnarray*}
\item[ii)] Si $\esp\left[V^{*}\left(0\right)\right]<\infty$, equivalentemente, si $\esp\left[\int_{0}^{X}V\left(s\right)ds\right]<\infty$,entonces
\begin{eqnarray*}
\frac{\int_{0}^{t}V\left(s\right)ds}{t}\rightarrow\frac{\esp\left[\int_{0}^{X}V\left(s\right)ds\right]}{\esp\left[X\right]}
\end{eqnarray*}
con probabilidad 1 y en media, cuando $t\rightarrow\infty$.
\end{itemize}
\end{Teo}
%
%___________________________________________________________________________________________
%\vspace{5.5cm}
%\chapter{Cadenas de Markov estacionarias}
%\vspace{-1.0cm}


%__________________________________________________________________________________________
%\subsection{Procesos Regenerativos Estacionarios - Stidham \cite{Stidham}}
%__________________________________________________________________________________________


Un proceso estoc\'astico a tiempo continuo $\left\{V\left(t\right),t\geq0\right\}$ es un proceso regenerativo si existe una sucesi\'on de variables aleatorias independientes e id\'enticamente distribuidas $\left\{X_{1},X_{2},\ldots\right\}$, sucesi\'on de renovaci\'on, tal que para cualquier conjunto de Borel $A$, 

\begin{eqnarray*}
\prob\left\{V\left(t\right)\in A|X_{1}+X_{2}+\cdots+X_{R\left(t\right)}=s,\left\{V\left(\tau\right),\tau<s\right\}\right\}=\prob\left\{V\left(t-s\right)\in A|X_{1}>t-s\right\},
\end{eqnarray*}
para todo $0\leq s\leq t$, donde $R\left(t\right)=\max\left\{X_{1}+X_{2}+\cdots+X_{j}\leq t\right\}=$n\'umero de renovaciones ({\emph{puntos de regeneraci\'on}}) que ocurren en $\left[0,t\right]$. El intervalo $\left[0,X_{1}\right)$ es llamado {\emph{primer ciclo de regeneraci\'on}} de $\left\{V\left(t \right),t\geq0\right\}$, $\left[X_{1},X_{1}+X_{2}\right)$ el {\emph{segundo ciclo de regeneraci\'on}}, y as\'i sucesivamente.

Sea $X=X_{1}$ y sea $F$ la funci\'on de distrbuci\'on de $X$


\begin{Def}
Se define el proceso estacionario, $\left\{V^{*}\left(t\right),t\geq0\right\}$, para $\left\{V\left(t\right),t\geq0\right\}$ por

\begin{eqnarray*}
\prob\left\{V\left(t\right)\in A\right\}=\frac{1}{\esp\left[X\right]}\int_{0}^{\infty}\prob\left\{V\left(t+x\right)\in A|X>x\right\}\left(1-F\left(x\right)\right)dx,
\end{eqnarray*} 
para todo $t\geq0$ y todo conjunto de Borel $A$.
\end{Def}

\begin{Def}
Una distribuci\'on se dice que es {\emph{aritm\'etica}} si todos sus puntos de incremento son m\'ultiplos de la forma $0,\lambda, 2\lambda,\ldots$ para alguna $\lambda>0$ entera.
\end{Def}


\begin{Def}
Una modificaci\'on medible de un proceso $\left\{V\left(t\right),t\geq0\right\}$, es una versi\'on de este, $\left\{V\left(t,w\right)\right\}$ conjuntamente medible para $t\geq0$ y para $w\in S$, $S$ espacio de estados para $\left\{V\left(t\right),t\geq0\right\}$.
\end{Def}

\begin{Teo}
Sea $\left\{V\left(t\right),t\geq\right\}$ un proceso regenerativo no negativo con modificaci\'on medible. Sea $\esp\left[X\right]<\infty$. Entonces el proceso estacionario dado por la ecuaci\'on anterior est\'a bien definido y tiene funci\'on de distribuci\'on independiente de $t$, adem\'as
\begin{itemize}
\item[i)] \begin{eqnarray*}
\esp\left[V^{*}\left(0\right)\right]&=&\frac{\esp\left[\int_{0}^{X}V\left(s\right)ds\right]}{\esp\left[X\right]}\end{eqnarray*}
\item[ii)] Si $\esp\left[V^{*}\left(0\right)\right]<\infty$, equivalentemente, si $\esp\left[\int_{0}^{X}V\left(s\right)ds\right]<\infty$,entonces
\begin{eqnarray*}
\frac{\int_{0}^{t}V\left(s\right)ds}{t}\rightarrow\frac{\esp\left[\int_{0}^{X}V\left(s\right)ds\right]}{\esp\left[X\right]}
\end{eqnarray*}
con probabilidad 1 y en media, cuando $t\rightarrow\infty$.
\end{itemize}
\end{Teo}

Sea la funci\'on generadora de momentos para $L_{i}$, el n\'umero de usuarios en la cola $Q_{i}\left(z\right)$ en cualquier momento, est\'a dada por el tiempo promedio de $z^{L_{i}\left(t\right)}$ sobre el ciclo regenerativo definido anteriormente. Entonces 



Es decir, es posible determinar las longitudes de las colas a cualquier tiempo $t$. Entonces, determinando el primer momento es posible ver que


\begin{Def}
El tiempo de Ciclo $C_{i}$ es el periodo de tiempo que comienza cuando la cola $i$ es visitada por primera vez en un ciclo, y termina cuando es visitado nuevamente en el pr\'oximo ciclo. La duraci\'on del mismo est\'a dada por $\tau_{i}\left(m+1\right)-\tau_{i}\left(m\right)$, o equivalentemente $\overline{\tau}_{i}\left(m+1\right)-\overline{\tau}_{i}\left(m\right)$ bajo condiciones de estabilidad.
\end{Def}


\begin{Def}
El tiempo de intervisita $I_{i}$ es el periodo de tiempo que comienza cuando se ha completado el servicio en un ciclo y termina cuando es visitada nuevamente en el pr\'oximo ciclo. Su  duraci\'on del mismo est\'a dada por $\tau_{i}\left(m+1\right)-\overline{\tau}_{i}\left(m\right)$.
\end{Def}

La duraci\'on del tiempo de intervisita es $\tau_{i}\left(m+1\right)-\overline{\tau}\left(m\right)$. Dado que el n\'umero de usuarios presentes en $Q_{i}$ al tiempo $t=\tau_{i}\left(m+1\right)$ es igual al n\'umero de arribos durante el intervalo de tiempo $\left[\overline{\tau}\left(m\right),\tau_{i}\left(m+1\right)\right]$ se tiene que


\begin{eqnarray*}
\esp\left[z_{i}^{L_{i}\left(\tau_{i}\left(m+1\right)\right)}\right]=\esp\left[\left\{P_{i}\left(z_{i}\right)\right\}^{\tau_{i}\left(m+1\right)-\overline{\tau}\left(m\right)}\right]
\end{eqnarray*}

entonces, si $I_{i}\left(z\right)=\esp\left[z^{\tau_{i}\left(m+1\right)-\overline{\tau}\left(m\right)}\right]$
se tiene que $F_{i}\left(z\right)=I_{i}\left[P_{i}\left(z\right)\right]$
para $i=1,2$.

Conforme a la definici\'on dada al principio del cap\'itulo, definici\'on (\ref{Def.Tn}), sean $T_{1},T_{2},\ldots$ los puntos donde las longitudes de las colas de la red de sistemas de visitas c\'iclicas son cero simult\'aneamente, cuando la cola $Q_{j}$ es visitada por el servidor para dar servicio, es decir, $L_{1}\left(T_{i}\right)=0,L_{2}\left(T_{i}\right)=0,\hat{L}_{1}\left(T_{i}\right)=0$ y $\hat{L}_{2}\left(T_{i}\right)=0$, a estos puntos se les denominar\'a puntos regenerativos. Entonces, 

\begin{Def}
Al intervalo de tiempo entre dos puntos regenerativos se le llamar\'a ciclo regenerativo.
\end{Def}

\begin{Def}
Para $T_{i}$ se define, $M_{i}$, el n\'umero de ciclos de visita a la cola $Q_{l}$, durante el ciclo regenerativo, es decir, $M_{i}$ es un proceso de renovaci\'on.
\end{Def}

\begin{Def}
Para cada uno de los $M_{i}$'s, se definen a su vez la duraci\'on de cada uno de estos ciclos de visita en el ciclo regenerativo, $C_{i}^{(m)}$, para $m=1,2,\ldots,M_{i}$, que a su vez, tambi\'en es n proceso de renovaci\'on.
\end{Def}

\footnote{In Stidham and  Heyman \cite{Stidham} shows that is sufficient for the regenerative process to be stationary that the mean regenerative cycle time is finite: $\esp\left[\sum_{m=1}^{M_{i}}C_{i}^{(m)}\right]<\infty$, 


 como cada $C_{i}^{(m)}$ contiene intervalos de r\'eplica positivos, se tiene que $\esp\left[M_{i}\right]<\infty$, adem\'as, como $M_{i}>0$, se tiene que la condici\'on anterior es equivalente a tener que $\esp\left[C_{i}\right]<\infty$,
por lo tanto una condici\'on suficiente para la existencia del proceso regenerativo est\'a dada por $\sum_{k=1}^{N}\mu_{k}<1.$}

Para $\left\{X\left(t\right):t\geq0\right\}$ Proceso Estoc\'astico a tiempo continuo con estado de espacios $S$, que es un espacio m\'etrico, con trayectorias continuas por la derecha y con l\'imites por la izquierda c.s. Sea $N\left(t\right)$ un proceso de renovaci\'on en $\rea_{+}$ definido en el mismo espacio de probabilidad que $X\left(t\right)$, con tiempos de renovaci\'on $T$ y tiempos de inter-renovaci\'on $\xi_{n}=T_{n}-T_{n-1}$, con misma distribuci\'on $F$ de media finita $\mu$.

\begin{Def}
Un elemento aleatorio en un espacio medible $\left(E,\mathcal{E}\right)$ en un espacio de probabilidad $\left(\Omega,\mathcal{F},\prob\right)$ a $\left(E,\mathcal{E}\right)$, es decir,
para $A\in \mathcal{E}$,  se tiene que $\left\{Y\in A\right\}\in\mathcal{F}$, donde $\left\{Y\in A\right\}:=\left\{w\in\Omega:Y\left(w\right)\in A\right\}=:Y^{-1}A$.
\end{Def}

\begin{Note}
Tambi\'en se dice que $Y$ est\'a soportado por el espacio de probabilidad $\left(\Omega,\mathcal{F},\prob\right)$ y que $Y$ es un mapeo medible de $\Omega$ en $E$, es decir, es $\mathcal{F}/\mathcal{E}$ medible.
\end{Note}

\begin{Def}
Para cada $i\in \mathbb{I}$ sea $P_{i}$ una medida de probabilidad en un espacio medible $\left(E_{i},\mathcal{E}_{i}\right)$. Se define el espacio producto
$\otimes_{i\in\mathbb{I}}\left(E_{i},\mathcal{E}_{i}\right):=\left(\prod_{i\in\mathbb{I}}E_{i},\otimes_{i\in\mathbb{I}}\mathcal{E}_{i}\right)$, donde $\prod_{i\in\mathbb{I}}E_{i}$ es el producto cartesiano de los $E_{i}$'s, y $\otimes_{i\in\mathbb{I}}\mathcal{E}_{i}$ es la $\sigma$-\'algebra producto, es decir, es la $\sigma$-\'algebra m\'as peque\~na en $\prod_{i\in\mathbb{I}}E_{i}$ que hace al $i$-\'esimo mapeo proyecci\'on en $E_{i}$ medible para toda $i\in\mathbb{I}$ es la $\sigma$-\'algebra inducida por los mapeos proyecci\'on. $$\otimes_{i\in\mathbb{I}}\mathcal{E}_{i}:=\sigma\left\{\left\{y:y_{i}\in A\right\}:i\in\mathbb{I}\textrm{ y }A\in\mathcal{E}_{i}\right\}.$$
\end{Def}

\begin{Def}
Un espacio de probabilidad $\left(\tilde{\Omega},\tilde{\mathcal{F}},\tilde{\prob}\right)$ es una extensi\'on de otro espacio de probabilidad $\left(\Omega,\mathcal{F},\prob\right)$ si $\left(\tilde{\Omega},\tilde{\mathcal{F}},\tilde{\prob}\right)$ soporta un elemento aleatorio $\xi\in\left(\Omega,\mathcal{F}\right)$ que tienen a $\prob$ como distribuci\'on.
\end{Def}

\begin{Teo}
Sea $\mathbb{I}$ un conjunto de \'indices arbitrario. Para cada $i\in\mathbb{I}$ sea $P_{i}$ una medida de probabilidad en un espacio medible $\left(E_{i},\mathcal{E}_{i}\right)$. Entonces existe una \'unica medida de probabilidad $\otimes_{i\in\mathbb{I}}P_{i}$ en $\otimes_{i\in\mathbb{I}}\left(E_{i},\mathcal{E}_{i}\right)$ tal que 

\begin{eqnarray*}
\otimes_{i\in\mathbb{I}}P_{i}\left(y\in\prod_{i\in\mathbb{I}}E_{i}:y_{i}\in A_{i_{1}},\ldots,y_{n}\in A_{i_{n}}\right)=P_{i_{1}}\left(A_{i_{n}}\right)\cdots P_{i_{n}}\left(A_{i_{n}}\right)
\end{eqnarray*}
para todos los enteros $n>0$, toda $i_{1},\ldots,i_{n}\in\mathbb{I}$ y todo $A_{i_{1}}\in\mathcal{E}_{i_{1}},\ldots,A_{i_{n}}\in\mathcal{E}_{i_{n}}$
\end{Teo}

La medida $\otimes_{i\in\mathbb{I}}P_{i}$ es llamada la medida producto y $\otimes_{i\in\mathbb{I}}\left(E_{i},\mathcal{E}_{i},P_{i}\right):=\left(\prod_{i\in\mathbb{I}},E_{i},\otimes_{i\in\mathbb{I}}\mathcal{E}_{i},\otimes_{i\in\mathbb{I}}P_{i}\right)$, es llamado espacio de probabilidad producto.


\begin{Def}
Un espacio medible $\left(E,\mathcal{E}\right)$ es \textit{Polaco} si existe una m\'etrica en $E$ tal que $E$ es completo, es decir cada sucesi\'on de Cauchy converge a un l\'imite en $E$, y \textit{separable}, $E$ tienen un subconjunto denso numerable, y tal que $\mathcal{E}$ es generado por conjuntos abiertos.
\end{Def}


\begin{Def}
Dos espacios medibles $\left(E,\mathcal{E}\right)$ y $\left(G,\mathcal{G}\right)$ son Borel equivalentes \textit{isomorfos} si existe una biyecci\'on $f:E\rightarrow G$ tal que $f$ es $\mathcal{E}/\mathcal{G}$ medible y su inversa $f^{-1}$ es $\mathcal{G}/\mathcal{E}$ medible. La biyecci\'on es una equivalencia de Borel.
\end{Def}

\begin{Def}
Un espacio medible  $\left(E,\mathcal{E}\right)$ es un \textit{espacio est\'andar} si es Borel equivalente a $\left(G,\mathcal{G}\right)$, donde $G$ es un subconjunto de Borel de $\left[0,1\right]$ y $\mathcal{G}$ son los subconjuntos de Borel de $G$.
\end{Def}

\begin{Note}
Cualquier espacio Polaco es un espacio est\'andar.
\end{Note}


\begin{Def}
Un proceso estoc\'astico con conjunto de \'indices $\mathbb{I}$ y espacio de estados $\left(E,\mathcal{E}\right)$ es una familia $Z=\left(\mathbb{Z}_{s}\right)_{s\in\mathbb{I}}$ donde $\mathbb{Z}_{s}$ son elementos aleatorios definidos en un espacio de probabilidad com\'un $\left(\Omega,\mathcal{F},\prob\right)$ y todos toman valores en $\left(E,\mathcal{E}\right)$.
\end{Def}

\begin{Def}
Un proceso estoc\'astico \textit{one-sided contiuous time} (\textbf{PEOSCT}) es un proceso estoc\'astico con conjunto de \'indices $\mathbb{I}=\left[0,\infty\right)$.
\end{Def}


Sea $\left(E^{\mathbb{I}},\mathcal{E}^{\mathbb{I}}\right)$ denota el espacio producto $\left(E^{\mathbb{I}},\mathcal{E}^{\mathbb{I}}\right):=\otimes_{s\in\mathbb{I}}\left(E,\mathcal{E}\right)$. Vamos a considerar $\mathbb{Z}$ como un mapeo aleatorio, es decir, como un elemento aleatorio en $\left(E^{\mathbb{I}},\mathcal{E}^{\mathbb{I}}\right)$ definido por $Z\left(w\right)=\left(Z_{s}\left(w\right)\right)_{s\in\mathbb{I}}$ y $w\in\Omega$.

\begin{Note}
La distribuci\'on de un proceso estoc\'astico $Z$ es la distribuci\'on de $Z$ como un elemento aleatorio en $\left(E^{\mathbb{I}},\mathcal{E}^{\mathbb{I}}\right)$. La distribuci\'on de $Z$ esta determinada de manera \'unica por las distribuciones finito dimensionales.
\end{Note}

\begin{Note}
En particular cuando $Z$ toma valores reales, es decir, $\left(E,\mathcal{E}\right)=\left(\mathbb{R},\mathcal{B}\right)$ las distribuciones finito dimensionales est\'an determinadas por las funciones de distribuci\'on finito dimensionales

\begin{eqnarray}
\prob\left(Z_{t_{1}}\leq x_{1},\ldots,Z_{t_{n}}\leq x_{n}\right),x_{1},\ldots,x_{n}\in\mathbb{R},t_{1},\ldots,t_{n}\in\mathbb{I},n\geq1.
\end{eqnarray}
\end{Note}

\begin{Note}
Para espacios polacos $\left(E,\mathcal{E}\right)$ el Teorema de Consistencia de Kolmogorov asegura que dada una colecci\'on de distribuciones finito dimensionales consistentes, siempre existe un proceso estoc\'astico que posee tales distribuciones finito dimensionales.
\end{Note}


\begin{Def}
Las trayectorias de $Z$ son las realizaciones $Z\left(w\right)$ para $w\in\Omega$ del mapeo aleatorio $Z$.
\end{Def}

\begin{Note}
Algunas restricciones se imponen sobre las trayectorias, por ejemplo que sean continuas por la derecha, o continuas por la derecha con l\'imites por la izquierda, o de manera m\'as general, se pedir\'a que caigan en alg\'un subconjunto $H$ de $E^{\mathbb{I}}$. En este caso es natural considerar a $Z$ como un elemento aleatorio que no est\'a en $\left(E^{\mathbb{I}},\mathcal{E}^{\mathbb{I}}\right)$ sino en $\left(H,\mathcal{H}\right)$, donde $\mathcal{H}$ es la $\sigma$-\'algebra generada por los mapeos proyecci\'on que toman a $z\in H$ a $z_{t}\in E$ para $t\in\mathbb{I}$. A $\mathcal{H}$ se le conoce como la traza de $H$ en $E^{\mathbb{I}}$, es decir,
\begin{eqnarray}
\mathcal{H}:=E^{\mathbb{I}}\cap H:=\left\{A\cap H:A\in E^{\mathbb{I}}\right\}.
\end{eqnarray}
\end{Note}


\begin{Note}
$Z$ tiene trayectorias con valores en $H$ y cada $Z_{t}$ es un mapeo medible de $\left(\Omega,\mathcal{F}\right)$ a $\left(H,\mathcal{H}\right)$. Cuando se considera un espacio de trayectorias en particular $H$, al espacio $\left(H,\mathcal{H}\right)$ se le llama el espacio de trayectorias de $Z$.
\end{Note}

\begin{Note}
La distribuci\'on del proceso estoc\'astico $Z$ con espacio de trayectorias $\left(H,\mathcal{H}\right)$ es la distribuci\'on de $Z$ como  un elemento aleatorio en $\left(H,\mathcal{H}\right)$. La distribuci\'on, nuevemente, est\'a determinada de manera \'unica por las distribuciones finito dimensionales.
\end{Note}


\begin{Def}
Sea $Z$ un PEOSCT  con espacio de estados $\left(E,\mathcal{E}\right)$ y sea $T$ un tiempo aleatorio en $\left[0,\infty\right)$. Por $Z_{T}$ se entiende el mapeo con valores en $E$ definido en $\Omega$ en la manera obvia:
\begin{eqnarray*}
Z_{T}\left(w\right):=Z_{T\left(w\right)}\left(w\right). w\in\Omega.
\end{eqnarray*}
\end{Def}

\begin{Def}
Un PEOSCT $Z$ es conjuntamente medible (\textbf{CM}) si el mapeo que toma $\left(w,t\right)\in\Omega\times\left[0,\infty\right)$ a $Z_{t}\left(w\right)\in E$ es $\mathcal{F}\otimes\mathcal{B}\left[0,\infty\right)/\mathcal{E}$ medible.
\end{Def}

\begin{Note}
Un PEOSCT-CM implica que el proceso es medible, dado que $Z_{T}$ es una composici\'on  de dos mapeos continuos: el primero que toma $w$ en $\left(w,T\left(w\right)\right)$ es $\mathcal{F}/\mathcal{F}\otimes\mathcal{B}\left[0,\infty\right)$ medible, mientras que el segundo toma $\left(w,T\left(w\right)\right)$ en $Z_{T\left(w\right)}\left(w\right)$ es $\mathcal{F}\otimes\mathcal{B}\left[0,\infty\right)/\mathcal{E}$ medible.
\end{Note}


\begin{Def}
Un PEOSCT con espacio de estados $\left(H,\mathcal{H}\right)$ es can\'onicamente conjuntamente medible (\textbf{CCM}) si el mapeo $\left(z,t\right)\in H\times\left[0,\infty\right)$ en $Z_{t}\in E$ es $\mathcal{H}\otimes\mathcal{B}\left[0,\infty\right)/\mathcal{E}$ medible.
\end{Def}

\begin{Note}
Un PEOSCTCCM implica que el proceso es CM, dado que un PECCM $Z$ es un mapeo de $\Omega\times\left[0,\infty\right)$ a $E$, es la composici\'on de dos mapeos medibles: el primero, toma $\left(w,t\right)$ en $\left(Z\left(w\right),t\right)$ es $\mathcal{F}\otimes\mathcal{B}\left[0,\infty\right)/\mathcal{H}\otimes\mathcal{B}\left[0,\infty\right)$ medible, y el segundo que toma $\left(Z\left(w\right),t\right)$  en $Z_{t}\left(w\right)$ es $\mathcal{H}\otimes\mathcal{B}\left[0,\infty\right)/\mathcal{E}$ medible. Por tanto CCM es una condici\'on m\'as fuerte que CM.
\end{Note}

\begin{Def}
Un conjunto de trayectorias $H$ de un PEOSCT $Z$, es internamente shift-invariante (\textbf{ISI}) si 
\begin{eqnarray*}
\left\{\left(z_{t+s}\right)_{s\in\left[0,\infty\right)}:z\in H\right\}=H\textrm{, }t\in\left[0,\infty\right).
\end{eqnarray*}
\end{Def}


\begin{Def}
Dado un PEOSCTISI, se define el mapeo-shift $\theta_{t}$, $t\in\left[0,\infty\right)$, de $H$ a $H$ por 
\begin{eqnarray*}
\theta_{t}z=\left(z_{t+s}\right)_{s\in\left[0,\infty\right)}\textrm{, }z\in H.
\end{eqnarray*}
\end{Def}

\begin{Def}
Se dice que un proceso $Z$ es shift-medible (\textbf{SM}) si $Z$ tiene un conjunto de trayectorias $H$ que es ISI y adem\'as el mapeo que toma $\left(z,t\right)\in H\times\left[0,\infty\right)$ en $\theta_{t}z\in H$ es $\mathcal{H}\otimes\mathcal{B}\left[0,\infty\right)/\mathcal{H}$ medible.
\end{Def}

\begin{Note}
Un proceso estoc\'astico con conjunto de trayectorias $H$ ISI es shift-medible si y s\'olo si es CCM
\end{Note}

\begin{Note}
\begin{itemize}
\item Dado el espacio polaco $\left(E,\mathcal{E}\right)$ se tiene el  conjunto de trayectorias $D_{E}\left[0,\infty\right)$ que es ISI, entonces cumpe con ser CCM.

\item Si $G$ es abierto, podemos cubrirlo por bolas abiertas cuay cerradura este contenida en $G$, y como $G$ es segundo numerable como subespacio de $E$, lo podemos cubrir por una cantidad numerable de bolas abiertas.

\end{itemize}
\end{Note}


\begin{Note}
Los procesos estoc\'asticos $Z$ a tiempo discreto con espacio de estados polaco, tambi\'en tiene un espacio de trayectorias polaco y por tanto tiene distribuciones condicionales regulares.
\end{Note}

\begin{Teo}
El producto numerable de espacios polacos es polaco.
\end{Teo}


\begin{Def}
Sea $\left(\Omega,\mathcal{F},\prob\right)$ espacio de probabilidad que soporta al proceso $Z=\left(Z_{s}\right)_{s\in\left[0,\infty\right)}$ y $S=\left(S_{k}\right)_{0}^{\infty}$ donde $Z$ es un PEOSCTM con espacio de estados $\left(E,\mathcal{E}\right)$  y espacio de trayectorias $\left(H,\mathcal{H}\right)$  y adem\'as $S$ es una sucesi\'on de tiempos aleatorios one-sided que satisfacen la condici\'on $0\leq S_{0}<S_{1}<\cdots\rightarrow\infty$. Considerando $S$ como un mapeo medible de $\left(\Omega,\mathcal{F}\right)$ al espacio sucesi\'on $\left(L,\mathcal{L}\right)$, donde 
\begin{eqnarray*}
L=\left\{\left(s_{k}\right)_{0}^{\infty}\in\left[0,\infty\right)^{\left\{0,1,\ldots\right\}}:s_{0}<s_{1}<\cdots\rightarrow\infty\right\},
\end{eqnarray*}
donde $\mathcal{L}$ son los subconjuntos de Borel de $L$, es decir, $\mathcal{L}=L\cap\mathcal{B}^{\left\{0,1,\ldots\right\}}$.

As\'i el par $\left(Z,S\right)$ es un mapeo medible de  $\left(\Omega,\mathcal{F}\right)$ en $\left(H\times L,\mathcal{H}\otimes\mathcal{L}\right)$. El par $\mathcal{H}\otimes\mathcal{L}^{+}$ denotar\'a la clase de todas las funciones medibles de $\left(H\times L,\mathcal{H}\otimes\mathcal{L}\right)$ en $\left(\left[0,\infty\right),\mathcal{B}\left[0,\infty\right)\right)$.
\end{Def}


\begin{Def}
Sea $\theta_{t}$ el mapeo-shift conjunto de $H\times L$ en $H\times L$ dado por
\begin{eqnarray*}
\theta_{t}\left(z,\left(s_{k}\right)_{0}^{\infty}\right)=\theta_{t}\left(z,\left(s_{n_{t-}+k}-t\right)_{0}^{\infty}\right)
\end{eqnarray*}
donde 
$n_{t-}=inf\left\{n\geq1:s_{n}\geq t\right\}$.
\end{Def}

\begin{Note}
Con la finalidad de poder realizar los shift's sin complicaciones de medibilidad, se supondr\'a que $Z$ es shit-medible, es decir, el conjunto de trayectorias $H$ es invariante bajo shifts del tiempo y el mapeo que toma $\left(z,t\right)\in H\times\left[0,\infty\right)$ en $z_{t}\in E$ es $\mathcal{H}\otimes\mathcal{B}\left[0,\infty\right)/\mathcal{E}$ medible.
\end{Note}

\begin{Def}
Dado un proceso \textbf{PEOSSM} (Proceso Estoc\'astico One Side Shift Medible) $Z$, se dice regenerativo cl\'asico con tiempos de regeneraci\'on $S$ si 

\begin{eqnarray*}
\theta_{S_{n}}\left(Z,S\right)=\left(Z^{0},S^{0}\right),n\geq0
\end{eqnarray*}
y adem\'as $\theta_{S_{n}}\left(Z,S\right)$ es independiente de $\left(\left(Z_{s}\right)s\in\left[0,S_{n}\right),S_{0},\ldots,S_{n}\right)$
Si lo anterior se cumple, al par $\left(Z,S\right)$ se le llama regenerativo cl\'asico.
\end{Def}

\begin{Note}
Si el par $\left(Z,S\right)$ es regenerativo cl\'asico, entonces las longitudes de los ciclos $X_{1},X_{2},\ldots,$ son i.i.d. e independientes de la longitud del retraso $S_{0}$, es decir, $S$ es un proceso de renovaci\'on. Las longitudes de los ciclos tambi\'en son llamados tiempos de inter-regeneraci\'on y tiempos de ocurrencia.

\end{Note}

\begin{Teo}
Sup\'ongase que el par $\left(Z,S\right)$ es regenerativo cl\'asico con $\esp\left[X_{1}\right]<\infty$. Entonces $\left(Z^{*},S^{*}\right)$ en el teorema 2.1 es una versi\'on estacionaria de $\left(Z,S\right)$. Adem\'as, si $X_{1}$ es lattice con span $d$, entonces $\left(Z^{**},S^{**}\right)$ en el teorema 2.2 es una versi\'on periodicamente estacionaria de $\left(Z,S\right)$ con periodo $d$.

\end{Teo}

\begin{Def}
Una variable aleatoria $X_{1}$ es \textit{spread out} si existe una $n\geq1$ y una  funci\'on $f\in\mathcal{B}^{+}$ tal que $\int_{\rea}f\left(x\right)dx>0$ con $X_{2},X_{3},\ldots,X_{n}$ copias i.i.d  de $X_{1}$, $$\prob\left(X_{1}+\cdots+X_{n}\in B\right)\geq\int_{B}f\left(x\right)dx$$ para $B\in\mathcal{B}$.

\end{Def}



\begin{Def}
Dado un proceso estoc\'astico $Z$ se le llama \textit{wide-sense regenerative} (\textbf{WSR}) con tiempos de regeneraci\'on $S$ si $\theta_{S_{n}}\left(Z,S\right)=\left(Z^{0},S^{0}\right)$ para $n\geq0$ en distribuci\'on y $\theta_{S_{n}}\left(Z,S\right)$ es independiente de $\left(S_{0},S_{1},\ldots,S_{n}\right)$ para $n\geq0$.
Se dice que el par $\left(Z,S\right)$ es WSR si lo anterior se cumple.
\end{Def}


\begin{Note}
\begin{itemize}
\item El proceso de trayectorias $\left(\theta_{s}Z\right)_{s\in\left[0,\infty\right)}$ es WSR con tiempos de regeneraci\'on $S$ pero no es regenerativo cl\'asico.

\item Si $Z$ es cualquier proceso estacionario y $S$ es un proceso de renovaci\'on que es independiente de $Z$, entonces $\left(Z,S\right)$ es WSR pero en general no es regenerativo cl\'asico

\end{itemize}

\end{Note}


\begin{Note}
Para cualquier proceso estoc\'astico $Z$, el proceso de trayectorias $\left(\theta_{s}Z\right)_{s\in\left[0,\infty\right)}$ es siempre un proceso de Markov.
\end{Note}



\begin{Teo}
Supongase que el par $\left(Z,S\right)$ es WSR con $\esp\left[X_{1}\right]<\infty$. Entonces $\left(Z^{*},S^{*}\right)$ en el teorema 2.1 es una versi\'on estacionaria de 
$\left(Z,S\right)$.
\end{Teo}


\begin{Teo}
Supongase que $\left(Z,S\right)$ es cycle-stationary con $\esp\left[X_{1}\right]<\infty$. Sea $U$ distribuida uniformemente en $\left[0,1\right)$ e independiente de $\left(Z^{0},S^{0}\right)$ y sea $\prob^{*}$ la medida de probabilidad en $\left(\Omega,\prob\right)$ definida por $$d\prob^{*}=\frac{X_{1}}{\esp\left[X_{1}\right]}d\prob$$. Sea $\left(Z^{*},S^{*}\right)$ con distribuci\'on $\prob^{*}\left(\theta_{UX_{1}}\left(Z^{0},S^{0}\right)\in\cdot\right)$. Entonces $\left(Z^{}*,S^{*}\right)$ es estacionario,
\begin{eqnarray*}
\esp\left[f\left(Z^{*},S^{*}\right)\right]=\esp\left[\int_{0}^{X_{1}}f\left(\theta_{s}\left(Z^{0},S^{0}\right)\right)ds\right]/\esp\left[X_{1}\right]
\end{eqnarray*}
$f\in\mathcal{H}\otimes\mathcal{L}^{+}$, and $S_{0}^{*}$ es continuo con funci\'on distribuci\'on $G_{\infty}$ definida por $$G_{\infty}\left(x\right):=\frac{\esp\left[X_{1}\right]\wedge x}{\esp\left[X_{1}\right]}$$ para $x\geq0$ y densidad $\prob\left[X_{1}>x\right]/\esp\left[X_{1}\right]$, con $x\geq0$.

\end{Teo}


\begin{Teo}
Sea $Z$ un Proceso Estoc\'astico un lado shift-medible \textit{one-sided shift-measurable stochastic process}, (PEOSSM),
y $S_{0}$ y $S_{1}$ tiempos aleatorios tales que $0\leq S_{0}<S_{1}$ y
\begin{equation}
\theta_{S_{1}}Z=\theta_{S_{0}}Z\textrm{ en distribuci\'on}.
\end{equation}

Entonces el espacio de probabilidad subyacente $\left(\Omega,\mathcal{F},\prob\right)$ puede extenderse para soportar una sucesi\'on de tiempos aleatorios $S$ tales que

\begin{eqnarray}
\theta_{S_{n}}\left(Z,S\right)=\left(Z^{0},S^{0}\right),n\geq0,\textrm{ en distribuci\'on},\\
\left(Z,S_{0},S_{1}\right)\textrm{ depende de }\left(X_{2},X_{3},\ldots\right)\textrm{ solamente a traves de }\theta_{S_{1}}Z.
\end{eqnarray}
\end{Teo}





%_________________________________________________________________________
%
%\subsection{Una vez que se tiene estabilidad}
%_________________________________________________________________________
%

Also the intervisit time $I_{i}$ is defined as the period beginning at the time of its service completion in a cycle and ending at the time when it is polled in the next cycle; its duration is given by $\tau_{i}\left(m+1\right)-\overline{\tau}_{i}\left(m\right)$.

So we the following are still true 

\begin{eqnarray}
\begin{array}{ll}
\esp\left[L_{i}\right]=\mu_{i}\esp\left[I_{i}\right], &
\esp\left[C_{i}\right]=\frac{f_{i}\left(i\right)}{\mu_{i}\left(1-\mu_{i}\right)},\\
\esp\left[S_{i}\right]=\mu_{i}\esp\left[C_{i}\right],&
\esp\left[I_{i}\right]=\left(1-\mu_{i}\right)\esp\left[C_{i}\right],\\
Var\left[L_{i}\right]= \mu_{i}^{2}Var\left[I_{i}\right]+\sigma^{2}\esp\left[I_{i}\right],& 
Var\left[C_{i}\right]=\frac{Var\left[L_{i}^{*}\right]}{\mu_{i}^{2}\left(1-\mu_{i}\right)^{2}},\\
Var\left[S_{i}\right]= \frac{Var\left[L_{i}^{*}\right]}{\left(1-\mu_{i}\right)^{2}}+\frac{\sigma^{2}\esp\left[L_{i}^{*}\right]}{\left(1-\mu_{i}\right)^{3}},&
Var\left[I_{i}\right]= \frac{Var\left[L_{i}^{*}\right]}{\mu_{i}^{2}}-\frac{\sigma_{i}^{2}}{\mu_{i}^{2}}f_{i}\left(i\right).
\end{array}
\end{eqnarray}
\begin{Def}
El tiempo de Ciclo $C_{i}$ es el periodo de tiempo que comienza cuando la cola $i$ es visitada por primera vez en un ciclo, y termina cuando es visitado nuevamente en el pr\'oximo ciclo. La duraci\'on del mismo est\'a dada por $\tau_{i}\left(m+1\right)-\tau_{i}\left(m\right)$, o equivalentemente $\overline{\tau}_{i}\left(m+1\right)-\overline{\tau}_{i}\left(m\right)$ bajo condiciones de estabilidad.
\end{Def}


\begin{Def}
El tiempo de intervisita $I_{i}$ es el periodo de tiempo que comienza cuando se ha completado el servicio en un ciclo y termina cuando es visitada nuevamente en el pr\'oximo ciclo. Su  duraci\'on del mismo est\'a dada por $\tau_{i}\left(m+1\right)-\overline{\tau}_{i}\left(m\right)$.
\end{Def}

La duraci\'on del tiempo de intervisita es $\tau_{i}\left(m+1\right)-\overline{\tau}\left(m\right)$. Dado que el n\'umero de usuarios presentes en $Q_{i}$ al tiempo $t=\tau_{i}\left(m+1\right)$ es igual al n\'umero de arribos durante el intervalo de tiempo $\left[\overline{\tau}\left(m\right),\tau_{i}\left(m+1\right)\right]$ se tiene que


\begin{eqnarray*}
\esp\left[z_{i}^{L_{i}\left(\tau_{i}\left(m+1\right)\right)}\right]=\esp\left[\left\{P_{i}\left(z_{i}\right)\right\}^{\tau_{i}\left(m+1\right)-\overline{\tau}\left(m\right)}\right]
\end{eqnarray*}

entonces, si $I_{i}\left(z\right)=\esp\left[z^{\tau_{i}\left(m+1\right)-\overline{\tau}\left(m\right)}\right]$
se tiene que $F_{i}\left(z\right)=I_{i}\left[P_{i}\left(z\right)\right]$
para $i=1,2$.

Conforme a la definici\'on dada al principio del cap\'itulo, definici\'on (\ref{Def.Tn}), sean $T_{1},T_{2},\ldots$ los puntos donde las longitudes de las colas de la red de sistemas de visitas c\'iclicas son cero simult\'aneamente, cuando la cola $Q_{j}$ es visitada por el servidor para dar servicio, es decir, $L_{1}\left(T_{i}\right)=0,L_{2}\left(T_{i}\right)=0,\hat{L}_{1}\left(T_{i}\right)=0$ y $\hat{L}_{2}\left(T_{i}\right)=0$, a estos puntos se les denominar\'a puntos regenerativos. Entonces, 

\begin{Def}
Al intervalo de tiempo entre dos puntos regenerativos se le llamar\'a ciclo regenerativo.
\end{Def}

\begin{Def}
Para $T_{i}$ se define, $M_{i}$, el n\'umero de ciclos de visita a la cola $Q_{l}$, durante el ciclo regenerativo, es decir, $M_{i}$ es un proceso de renovaci\'on.
\end{Def}

\begin{Def}
Para cada uno de los $M_{i}$'s, se definen a su vez la duraci\'on de cada uno de estos ciclos de visita en el ciclo regenerativo, $C_{i}^{(m)}$, para $m=1,2,\ldots,M_{i}$, que a su vez, tambi\'en es n proceso de renovaci\'on.
\end{Def}


Sea la funci\'on generadora de momentos para $L_{i}$, el n\'umero de usuarios en la cola $Q_{i}\left(z\right)$ en cualquier momento, est\'a dada por el tiempo promedio de $z^{L_{i}\left(t\right)}$ sobre el ciclo regenerativo definido anteriormente:

\begin{eqnarray*}
Q_{i}\left(z\right)&=&\esp\left[z^{L_{i}\left(t\right)}\right]=\frac{\esp\left[\sum_{m=1}^{M_{i}}\sum_{t=\tau_{i}\left(m\right)}^{\tau_{i}\left(m+1\right)-1}z^{L_{i}\left(t\right)}\right]}{\esp\left[\sum_{m=1}^{M_{i}}\tau_{i}\left(m+1\right)-\tau_{i}\left(m\right)\right]}
\end{eqnarray*}

$M_{i}$ es un tiempo de paro en el proceso regenerativo con $\esp\left[M_{i}\right]<\infty$\footnote{En Stidham\cite{Stidham} y Heyman  se muestra que una condici\'on suficiente para que el proceso regenerativo 
estacionario sea un procesoo estacionario es que el valor esperado del tiempo del ciclo regenerativo sea finito, es decir: $\esp\left[\sum_{m=1}^{M_{i}}C_{i}^{(m)}\right]<\infty$, como cada $C_{i}^{(m)}$ contiene intervalos de r\'eplica positivos, se tiene que $\esp\left[M_{i}\right]<\infty$, adem\'as, como $M_{i}>0$, se tiene que la condici\'on anterior es equivalente a tener que $\esp\left[C_{i}\right]<\infty$,
por lo tanto una condici\'on suficiente para la existencia del proceso regenerativo est\'a dada por $\sum_{k=1}^{N}\mu_{k}<1.$}, se sigue del lema de Wald que:


\begin{eqnarray*}
\esp\left[\sum_{m=1}^{M_{i}}\sum_{t=\tau_{i}\left(m\right)}^{\tau_{i}\left(m+1\right)-1}z^{L_{i}\left(t\right)}\right]&=&\esp\left[M_{i}\right]\esp\left[\sum_{t=\tau_{i}\left(m\right)}^{\tau_{i}\left(m+1\right)-1}z^{L_{i}\left(t\right)}\right]\\
\esp\left[\sum_{m=1}^{M_{i}}\tau_{i}\left(m+1\right)-\tau_{i}\left(m\right)\right]&=&\esp\left[M_{i}\right]\esp\left[\tau_{i}\left(m+1\right)-\tau_{i}\left(m\right)\right]
\end{eqnarray*}

por tanto se tiene que


\begin{eqnarray*}
Q_{i}\left(z\right)&=&\frac{\esp\left[\sum_{t=\tau_{i}\left(m\right)}^{\tau_{i}\left(m+1\right)-1}z^{L_{i}\left(t\right)}\right]}{\esp\left[\tau_{i}\left(m+1\right)-\tau_{i}\left(m\right)\right]}
\end{eqnarray*}

observar que el denominador es simplemente la duraci\'on promedio del tiempo del ciclo.


Haciendo las siguientes sustituciones en la ecuaci\'on (\ref{Corolario2}): $n\rightarrow t-\tau_{i}\left(m\right)$, $T \rightarrow \overline{\tau}_{i}\left(m\right)-\tau_{i}\left(m\right)$, $L_{n}\rightarrow L_{i}\left(t\right)$ y $F\left(z\right)=\esp\left[z^{L_{0}}\right]\rightarrow F_{i}\left(z\right)=\esp\left[z^{L_{i}\tau_{i}\left(m\right)}\right]$, se puede ver que

\begin{eqnarray}\label{Eq.Arribos.Primera}
\esp\left[\sum_{n=0}^{T-1}z^{L_{n}}\right]=
\esp\left[\sum_{t=\tau_{i}\left(m\right)}^{\overline{\tau}_{i}\left(m\right)-1}z^{L_{i}\left(t\right)}\right]
=z\frac{F_{i}\left(z\right)-1}{z-P_{i}\left(z\right)}
\end{eqnarray}

Por otra parte durante el tiempo de intervisita para la cola $i$, $L_{i}\left(t\right)$ solamente se incrementa de manera que el incremento por intervalo de tiempo est\'a dado por la funci\'on generadora de probabilidades de $P_{i}\left(z\right)$, por tanto la suma sobre el tiempo de intervisita puede evaluarse como:

\begin{eqnarray*}
\esp\left[\sum_{t=\tau_{i}\left(m\right)}^{\tau_{i}\left(m+1\right)-1}z^{L_{i}\left(t\right)}\right]&=&\esp\left[\sum_{t=\tau_{i}\left(m\right)}^{\tau_{i}\left(m+1\right)-1}\left\{P_{i}\left(z\right)\right\}^{t-\overline{\tau}_{i}\left(m\right)}\right]=\frac{1-\esp\left[\left\{P_{i}\left(z\right)\right\}^{\tau_{i}\left(m+1\right)-\overline{\tau}_{i}\left(m\right)}\right]}{1-P_{i}\left(z\right)}\\
&=&\frac{1-I_{i}\left[P_{i}\left(z\right)\right]}{1-P_{i}\left(z\right)}
\end{eqnarray*}
por tanto

\begin{eqnarray*}
\esp\left[\sum_{t=\tau_{i}\left(m\right)}^{\tau_{i}\left(m+1\right)-1}z^{L_{i}\left(t\right)}\right]&=&
\frac{1-F_{i}\left(z\right)}{1-P_{i}\left(z\right)}
\end{eqnarray*}

Por lo tanto

\begin{eqnarray*}
Q_{i}\left(z\right)&=&\frac{\esp\left[\sum_{t=\tau_{i}\left(m\right)}^{\tau_{i}\left(m+1\right)-1}z^{L_{i}\left(t\right)}\right]}{\esp\left[\tau_{i}\left(m+1\right)-\tau_{i}\left(m\right)\right]}
=\frac{1}{\esp\left[\tau_{i}\left(m+1\right)-\tau_{i}\left(m\right)\right]}
\esp\left[\sum_{t=\tau_{i}\left(m\right)}^{\tau_{i}\left(m+1\right)-1}z^{L_{i}\left(t\right)}\right]\\
&=&\frac{1}{\esp\left[\tau_{i}\left(m+1\right)-\tau_{i}\left(m\right)\right]}
\esp\left[\sum_{t=\tau_{i}\left(m\right)}^{\overline{\tau}_{i}\left(m\right)-1}z^{L_{i}\left(t\right)}
+\sum_{t=\overline{\tau}_{i}\left(m\right)}^{\tau_{i}\left(m+1\right)-1}z^{L_{i}\left(t\right)}\right]\\
&=&\frac{1}{\esp\left[\tau_{i}\left(m+1\right)-\tau_{i}\left(m\right)\right]}\left\{
\esp\left[\sum_{t=\tau_{i}\left(m\right)}^{\overline{\tau}_{i}\left(m\right)-1}z^{L_{i}\left(t\right)}\right]
+\esp\left[\sum_{t=\overline{\tau}_{i}\left(m\right)}^{\tau_{i}\left(m+1\right)-1}z^{L_{i}\left(t\right)}\right]\right\}\\
&=&\frac{1}{\esp\left[\tau_{i}\left(m+1\right)-\tau_{i}\left(m\right)\right]}\left\{
z\frac{F_{i}\left(z\right)-1}{z-P_{i}\left(z\right)}+\frac{1-F_{i}\left(z\right)}{1-P_{i}\left(z\right)}
\right\}\\
&=&\frac{1}{\esp\left[C_{i}\right]}\cdot\frac{1-F_{i}\left(z\right)}{P_{i}\left(z\right)-z}\cdot\frac{\left(1-z\right)P_{i}\left(z\right)}{1-P_{i}\left(z\right)}
\end{eqnarray*}

es decir

\begin{equation}
Q_{i}\left(z\right)=\frac{1}{\esp\left[C_{i}\right]}\cdot\frac{1-F_{i}\left(z\right)}{P_{i}\left(z\right)-z}\cdot\frac{\left(1-z\right)P_{i}\left(z\right)}{1-P_{i}\left(z\right)}
\end{equation}


Si hacemos:

\begin{eqnarray}
S\left(z\right)&=&1-F\left(z\right)\\
T\left(z\right)&=&z-P\left(z\right)\\
U\left(z\right)&=&1-P\left(z\right)
\end{eqnarray}
entonces 

\begin{eqnarray}
\esp\left[C_{i}\right]Q\left(z\right)=\frac{\left(z-1\right)S\left(z\right)P\left(z\right)}{T\left(z\right)U\left(z\right)}
\end{eqnarray}

A saber, si $a_{k}=P\left\{L\left(t\right)=k\right\}$
\begin{eqnarray*}
S\left(z\right)=1-F\left(z\right)=1-\sum_{k=0}^{+\infty}a_{k}z^{k}
\end{eqnarray*}
entonces

%\begin{eqnarray}
%\begin{array}{ll}
%S^{'}\left(z\right)=-\sum_{k=1}^{+\infty}ka_{k}z^{k-1},& %S^{(1)}\left(1\right)=-\sum_{k=1}^{+\infty}ka_{k}=-\esp\left[L\left(t\right)\right],\\
%S^{''}\left(z\right)=-\sum_{k=2}^{+\infty}k(k-1)a_{k}z^{k-2},& S^{(2)}\left(1\right)=-\sum_{k=2}^{+\infty}k(k-1)a_{k}=\esp\left[L\left(L-1\right)\right],\\
%S^{'''}\left(z\right)=-\sum_{k=3}^{+\infty}k(k-1)(k-2)a_{k}z^{k-3},&
%S^{(3)}\left(1\right)=-\sum_{k=3}^{+\infty}k(k-1)(k-2)a_{k}\\
%&=-\esp\left[L\left(L-1\right)\left(L-2\right)\right]\\
%&=-\esp\left[L^{3}\right]+3-\esp\left[L^{2}\right]-2-\esp\left[L\right];
%\end{array}
%\end{eqnarray}

$S^{'}\left(z\right)=-\sum_{k=1}^{+\infty}ka_{k}z^{k-1}$, por tanto $S^{(1)}\left(1\right)=-\sum_{k=1}^{+\infty}ka_{k}=-\esp\left[L\left(t\right)\right]$,
luego $S^{''}\left(z\right)=-\sum_{k=2}^{+\infty}k(k-1)a_{k}z^{k-2}$ y $S^{(2)}\left(1\right)=-\sum_{k=2}^{+\infty}k(k-1)a_{k}=\esp\left[L\left(L-1\right)\right]$;
de la misma manera $S^{'''}\left(z\right)=-\sum_{k=3}^{+\infty}k(k-1)(k-2)a_{k}z^{k-3}$ y $S^{(3)}\left(1\right)=-\sum_{k=3}^{+\infty}k(k-1)(k-2)a_{k}=-\esp\left[L\left(L-1\right)\left(L-2\right)\right]
=-\esp\left[L^{3}\right]+3-\esp\left[L^{2}\right]-2-\esp\left[L\right]$. 

Es decir

\begin{eqnarray*}
S^{(1)}\left(1\right)&=&-\esp\left[L\left(t\right)\right],\\ S^{(2)}\left(1\right)&=&-\esp\left[L\left(L-1\right)\right]
=-\esp\left[L^{2}\right]+\esp\left[L\right],\\
S^{(3)}\left(1\right)&=&-\esp\left[L\left(L-1\right)\left(L-2\right)\right]
=-\esp\left[L^{3}\right]+3\esp\left[L^{2}\right]-2\esp\left[L\right].
\end{eqnarray*}


Expandiendo alrededor de $z=1$

\begin{eqnarray*}
S\left(z\right)&=&S\left(1\right)+\frac{S^{'}\left(1\right)}{1!}\left(z-1\right)+\frac{S^{''}\left(1\right)}{2!}\left(z-1\right)^{2}+\frac{S^{'''}\left(1\right)}{3!}\left(z-1\right)^{3}+\ldots+\\
&=&\left(z-1\right)\left\{S^{'}\left(1\right)+\frac{S^{''}\left(1\right)}{2!}\left(z-1\right)+\frac{S^{'''}\left(1\right)}{3!}\left(z-1\right)^{2}+\ldots+\right\}\\
&=&\left(z-1\right)R_{1}\left(z\right)
\end{eqnarray*}
con $R_{1}\left(z\right)\neq0$, pues

\begin{eqnarray}\label{Eq.R1}
R_{1}\left(z\right)=-\esp\left[L\right]
\end{eqnarray}
entonces

\begin{eqnarray}
R_{1}\left(z\right)&=&S^{'}\left(1\right)+\frac{S^{''}\left(1\right)}{2!}\left(z-1\right)+\frac{S^{'''}\left(1\right)}{3!}\left(z-1\right)^{2}+\frac{S^{iv}\left(1\right)}{4!}\left(z-1\right)^{3}+\ldots+
\end{eqnarray}
Calculando las derivadas y evaluando en $z=1$

\begin{eqnarray}
R_{1}\left(1\right)&=&S^{(1)}\left(1\right)=-\esp\left[L\right]\\
R_{1}^{(1)}\left(1\right)&=&\frac{1}{2}S^{(2)}\left(1\right)=-\frac{1}{2}\esp\left[L^{2}\right]+\frac{1}{2}\esp\left[L\right]\\
R_{1}^{(2)}\left(1\right)&=&\frac{2}{3!}S^{(3)}\left(1\right)
=-\frac{1}{3}\esp\left[L^{3}\right]+\esp\left[L^{2}\right]-\frac{2}{3}\esp\left[L\right]
\end{eqnarray}

De manera an\'aloga se puede ver que para $T\left(z\right)=z-P\left(z\right)$ se puede encontrar una expanci\'on alrededor de $z=1$

Expandiendo alrededor de $z=1$

\begin{eqnarray*}
T\left(z\right)&=&T\left(1\right)+\frac{T^{'}\left(1\right)}{1!}\left(z-1\right)+\frac{T^{''}\left(1\right)}{2!}\left(z-1\right)^{2}+\frac{T^{'''}\left(1\right)}{3!}\left(z-1\right)^{3}+\ldots+\\
&=&\left(z-1\right)\left\{T^{'}\left(1\right)+\frac{T^{''}\left(1\right)}{2!}\left(z-1\right)+\frac{T^{'''}\left(1\right)}{3!}\left(z-1\right)^{2}+\ldots+\right\}\\
&=&\left(z-1\right)R_{2}\left(z\right)
\end{eqnarray*}

donde 
\begin{eqnarray*}
T^{(1)}\left(1\right)&=&-\esp\left[X\left(t\right)\right]=-\mu,\\ T^{(2)}\left(1\right)&=&-\esp\left[X\left(X-1\right)\right]
=-\esp\left[X^{2}\right]+\esp\left[X\right]=-\esp\left[X^{2}\right]+\mu,\\
T^{(3)}\left(1\right)&=&-\esp\left[X\left(X-1\right)\left(X-2\right)\right]
=-\esp\left[X^{3}\right]+3\esp\left[X^{2}\right]-2\esp\left[X\right]\\
&=&-\esp\left[X^{3}\right]+3\esp\left[X^{2}\right]-2\mu.
\end{eqnarray*}

Por lo tanto $R_{2}\left(1\right)\neq0$, pues

\begin{eqnarray}\label{Eq.R2}
R_{2}\left(1\right)=1-\esp\left[X\right]=1-\mu
\end{eqnarray}
entonces

\begin{eqnarray}
R_{2}\left(z\right)&=&T^{'}\left(1\right)+\frac{T^{''}\left(1\right)}{2!}\left(z-1\right)+\frac{T^{'''}\left(1\right)}{3!}\left(z-1\right)^{2}+\frac{T^{(iv)}\left(1\right)}{4!}\left(z-1\right)^{3}+\ldots+
\end{eqnarray}
Calculando las derivadas y evaluando en $z=1$

\begin{eqnarray}
R_{2}\left(1\right)&=&T^{(1)}\left(1\right)=1-\mu\\
R_{2}^{(1)}\left(1\right)&=&\frac{1}{2}T^{(2)}\left(1\right)=-\frac{1}{2}\esp\left[X^{2}\right]+\frac{1}{2}\mu\\
R_{2}^{(2)}\left(1\right)&=&\frac{2}{3!}T^{(3)}\left(1\right)
=-\frac{1}{3}\esp\left[X^{3}\right]+\esp\left[X^{2}\right]-\frac{2}{3}\mu
\end{eqnarray}

Finalmente para de manera an\'aloga se puede ver que para $U\left(z\right)=1-P\left(z\right)$ se puede encontrar una expanci\'on alrededor de $z=1$

\begin{eqnarray*}
U\left(z\right)&=&U\left(1\right)+\frac{U^{'}\left(1\right)}{1!}\left(z-1\right)+\frac{U^{''}\left(1\right)}{2!}\left(z-1\right)^{2}+\frac{U^{'''}\left(1\right)}{3!}\left(z-1\right)^{3}+\ldots+\\
&=&\left(z-1\right)\left\{U^{'}\left(1\right)+\frac{U^{''}\left(1\right)}{2!}\left(z-1\right)+\frac{U^{'''}\left(1\right)}{3!}\left(z-1\right)^{2}+\ldots+\right\}\\
&=&\left(z-1\right)R_{3}\left(z\right)
\end{eqnarray*}

donde 
\begin{eqnarray*}
U^{(1)}\left(1\right)&=&-\esp\left[X\left(t\right)\right]=-\mu,\\ U^{(2)}\left(1\right)&=&-\esp\left[X\left(X-1\right)\right]
=-\esp\left[X^{2}\right]+\esp\left[X\right]=-\esp\left[X^{2}\right]+\mu,\\
U^{(3)}\left(1\right)&=&-\esp\left[X\left(X-1\right)\left(X-2\right)\right]
=-\esp\left[X^{3}\right]+3\esp\left[X^{2}\right]-2\esp\left[X\right]\\
&=&-\esp\left[X^{3}\right]+3\esp\left[X^{2}\right]-2\mu.
\end{eqnarray*}

Por lo tanto $R_{3}\left(1\right)\neq0$, pues

\begin{eqnarray}\label{Eq.R2}
R_{3}\left(1\right)=-\esp\left[X\right]=-\mu
\end{eqnarray}
entonces

\begin{eqnarray}
R_{3}\left(z\right)&=&U^{'}\left(1\right)+\frac{U^{''}\left(1\right)}{2!}\left(z-1\right)+\frac{U^{'''}\left(1\right)}{3!}\left(z-1\right)^{2}+\frac{U^{(iv)}\left(1\right)}{4!}\left(z-1\right)^{3}+\ldots+
\end{eqnarray}

Calculando las derivadas y evaluando en $z=1$

\begin{eqnarray}
R_{3}\left(1\right)&=&U^{(1)}\left(1\right)=-\mu\\
R_{3}^{(1)}\left(1\right)&=&\frac{1}{2}U^{(2)}\left(1\right)=-\frac{1}{2}\esp\left[X^{2}\right]+\frac{1}{2}\mu\\
R_{3}^{(2)}\left(1\right)&=&\frac{2}{3!}U^{(3)}\left(1\right)
=-\frac{1}{3}\esp\left[X^{3}\right]+\esp\left[X^{2}\right]-\frac{2}{3}\mu
\end{eqnarray}

Por lo tanto

\begin{eqnarray}
\esp\left[C_{i}\right]Q\left(z\right)&=&\frac{\left(z-1\right)\left(z-1\right)R_{1}\left(z\right)P\left(z\right)}{\left(z-1\right)R_{2}\left(z\right)\left(z-1\right)R_{3}\left(z\right)}
=\frac{R_{1}\left(z\right)P\left(z\right)}{R_{2}\left(z\right)R_{3}\left(z\right)}\equiv\frac{R_{1}P}{R_{2}R_{3}}
\end{eqnarray}

Entonces

\begin{eqnarray}\label{Eq.Primer.Derivada.Q}
\left[\frac{R_{1}\left(z\right)P\left(z\right)}{R_{2}\left(z\right)R_{3}\left(z\right)}\right]^{'}&=&\frac{PR_{2}R_{3}R_{1}^{'}
+R_{1}R_{2}R_{3}P^{'}-R_{3}R_{1}PR_{2}-R_{2}R_{1}PR_{3}^{'}}{\left(R_{2}R_{3}\right)^{2}}
\end{eqnarray}
Evaluando en $z=1$
\begin{eqnarray*}
&=&\frac{R_{2}(1)R_{3}(1)R_{1}^{(1)}(1)+R_{1}(1)R_{2}(1)R_{3}(1)P^{'}(1)-R_{3}(1)R_{1}(1)R_{2}(1)^{(1)}-R_{2}(1)R_{1}(1)R_{3}^{'}(1)}{\left(R_{2}(1)R_{3}(1)\right)^{2}}\\
&=&\frac{1}{\left(1-\mu\right)^{2}\mu^{2}}\left\{\left(-\frac{1}{2}\esp L^{2}+\frac{1}{2}\esp L\right)\left(1-\mu\right)\left(-\mu\right)+\left(-\esp L\right)\left(1-\mu\right)\left(-\mu\right)\mu\right.\\
&&\left.-\left(-\frac{1}{2}\esp X^{2}+\frac{1}{2}\mu\right)\left(-\mu\right)\left(-\esp L\right)-\left(1-\mu\right)\left(-\esp L\right)\left(-\frac{1}{2}\esp X^{2}+\frac{1}{2}\mu\right)\right\}\\
&=&\frac{1}{\left(1-\mu\right)^{2}\mu^{2}}\left\{\left(-\frac{1}{2}\esp L^{2}+\frac{1}{2}\esp L\right)\left(\mu^{2}-\mu\right)
+\left(\mu^{2}-\mu^{3}\right)\esp L\right.\\
&&\left.-\mu\esp L\left(-\frac{1}{2}\esp X^{2}+\frac{1}{2}\mu\right)
+\left(\esp L-\mu\esp L\right)\left(-\frac{1}{2}\esp X^{2}+\frac{1}{2}\mu\right)\right\}\\
&=&\frac{1}{\left(1-\mu\right)^{2}\mu^{2}}\left\{-\frac{1}{2}\mu^{2}\esp L^{2}
+\frac{1}{2}\mu\esp L^{2}
+\frac{1}{2}\mu^{2}\esp L
-\mu^{3}\esp L
+\mu\esp L\esp X^{2}
-\frac{1}{2}\esp L\esp X^{2}\right\}\\
&=&\frac{1}{\left(1-\mu\right)^{2}\mu^{2}}\left\{
\frac{1}{2}\mu\esp L^{2}\left(1-\mu\right)
+\esp L\left(\frac{1}{2}-\mu\right)\left(\mu^{2}-\esp X^{2}\right)\right\}\\
&=&\frac{1}{2\mu\left(1-\mu\right)}\esp L^{2}-\frac{\frac{1}{2}-\mu}{\left(1-\mu\right)^{2}\mu^{2}}\sigma^{2}\esp L
\end{eqnarray*}

por lo tanto (para Takagi)

\begin{eqnarray*}
Q^{(1)}=\frac{1}{\esp C}\left\{\frac{1}{2\mu\left(1-\mu\right)}\esp L^{2}-\frac{\frac{1}{2}-\mu}{\left(1-\mu\right)^{2}\mu^{2}}\sigma^{2}\esp L\right\}
\end{eqnarray*}
donde 

\begin{eqnarray*}
\esp C = \frac{\esp L}{\mu\left(1-\mu\right)}
\end{eqnarray*}
entonces

\begin{eqnarray*}
Q^{(1)}&=&\frac{1}{2}\frac{\esp L^{2}}{\esp L}-\frac{\frac{1}{2}-\mu}{\left(1-\mu\right)\mu}\sigma^{2}
=\frac{\esp L^{2}}{2\esp L}-\frac{\sigma^{2}}{2}\left\{\frac{2\mu-1}{\left(1-\mu\right)\mu}\right\}\\
&=&\frac{\esp L^{2}}{2\esp L}+\frac{\sigma^{2}}{2}\left\{\frac{1}{1-\mu}+\frac{1}{\mu}\right\}
\end{eqnarray*}

Mientras que para nosotros

\begin{eqnarray*}
Q^{(1)}=\frac{1}{\mu\left(1-\mu\right)}\frac{\esp L^{2}}{2\esp C}
-\sigma^{2}\frac{\esp L}{2\esp C}\cdot\frac{1-2\mu}{\left(1-\mu\right)^{2}\mu^{2}}
\end{eqnarray*}

Retomando la ecuaci\'on (\ref{Eq.Primer.Derivada.Q})

\begin{eqnarray*}
\left[\frac{R_{1}\left(z\right)P\left(z\right)}{R_{2}\left(z\right)R_{3}\left(z\right)}\right]^{'}&=&\frac{PR_{2}R_{3}R_{1}^{'}
+R_{1}R_{2}R_{3}P^{'}-R_{3}R_{1}PR_{2}-R_{2}R_{1}PR_{3}^{'}}{\left(R_{2}R_{3}\right)^{2}}
=\frac{F\left(z\right)}{G\left(z\right)}
\end{eqnarray*}

donde 

\begin{eqnarray*}
F\left(z\right)&=&PR_{2}R_{3}R_{1}^{'}
+R_{1}R_{2}R_{3}P^{'}-R_{3}R_{1}PR_{2}^{'}-R_{2}R_{1}PR_{3}^{'}\\
G\left(z\right)&=&R_{2}^{2}R_{3}^{2}\\
G^{2}\left(z\right)&=&R_{2}^{4}R_{3}^{4}=\left(1-\mu\right)^{4}\mu^{4}
\end{eqnarray*}
y por tanto

\begin{eqnarray*}
G^{'}\left(z\right)&=&2R_{2}R_{3}\left[R_{2}^{'}R_{3}+R_{2}R_{3}^{'}\right]\\
G^{'}\left(1\right)&=&-2\left(1-\mu\right)\mu\left[\left(-\frac{1}{2}\esp\left[X^{2}\right]+\frac{1}{2}\mu\right)\left(-\mu\right)+\left(1-\mu\right)\left(-\frac{1}{2}\esp\left[X^{2}\right]+\frac{1}{2}\mu\right)\right]
\end{eqnarray*}


\begin{eqnarray*}
F^{'}\left(z\right)&=&\left[\left(R_{2}R_{3}\right)R_{1}^{''}
-\left(R_{1}R_{3}\right)R_{2}^{''}
-\left(R_{1}R_{2}\right)R_{3}^{''}
-2\left(R_{2}^{'}R_{3}^{'}\right)R_{1}\right]P
+2\left(R_{2}R_{3}\right)R_{1}^{'}P^{'}
+\left(R_{1}R_{2}R_{3}\right)P^{''}
\end{eqnarray*}

Por lo tanto, encontremos $F^{'}\left(z\right)G\left(z\right)+F\left(z\right)G^{'}\left(z\right)$:

\begin{eqnarray*}
&&F^{'}\left(z\right)G\left(z\right)+F\left(z\right)G^{'}\left(z\right)=
\left\{\left[\left(R_{2}R_{3}\right)R_{1}^{''}
-\left(R_{1}R_{3}\right)R_{2}^{''}
-\left(R_{1}R_{2}\right)R_{3}^{''}
-2\left(R_{2}^{'}R_{3}^{'}\right)R_{1}\right]P\right.\\
&&\left.+2\left(R_{2}R_{3}\right)R_{1}^{'}P^{'}
+\left(R_{1}R_{2}R_{3}\right)P^{''}\right\}R_{2}^{2}R_{3}^{2}
-\left\{\left[PR_{2}R_{3}R_{1}^{'}+R_{1}R_{2}R_{3}P^{'}
-R_{3}R_{1}PR_{2}^{'}\right.\right.\\
&&\left.\left.
-R_{2}R_{1}PR_{3}^{'}\right]\left[2R_{2}R_{3}\left(R_{2}^{'}R_{3}+R_{2}R_{3}^{'}\right)\right]\right\}
\end{eqnarray*}
Evaluando en $z=1$

\begin{eqnarray*}
&=&\left(1+R_{3}\right)^{3}R_{3}^{3}R_{1}^{''}-\left(1+R_{3}\right)^{2}R_{1}R_{3}^{3}R_{3}^{''}
-\left(1+R_{3}\right)^{3}R_{3}^{2}R_{1}R_{3}^{''}-2\left(1+R_{3}\right)^{2}R_{3}^{2}
\left(R_{3}^{'}\right)^{2}\\
&+&2\left(1+R_{3}\right)^{3}R_{3}^{3}R_{1}^{'}P^{'}
+\left(1+R_{3}\right)^{3}R_{3}^{3}R_{1}P^{''}
-2\left(1+R_{3}\right)^{2}R_{3}^{2}\left(1+2R_{3}\right)R_{3}^{'}R_{1}^{'}\\
&-&2\left(1+R_{3}\right)^{2}R_{3}^{2}R_{1}R_{3}^{'}\left(1+2R_{3}\right)P^{'}
+2\left(1+R_{3}\right)\left(1+2R_{3}\right)R_{3}^{3}R_{1}\left(R_{3}^{'}\right)^{2}\\
&+&2\left(1+R_{3}\right)^{2}\left(1+2R_{3}\right)R_{1}R_{3}R_{3}^{'}\\
&=&-\left(1-\mu\right)^{3}\mu^{3}R_{1}^{''}-\left(1-\mu\right)^{2}\mu^{2}R_{1}\left(1-2\mu\right)R_{3}^{''}
-\left(1-\mu\right)^{3}\mu^{3}R_{1}P^{''}\\
&+&2\left(1-\mu\right)\mu^{2}\left[\left(1-2\mu\right)R_{1}-\left(1-\mu\right)\right]\left(R_{3}^{'}\right)^{2}
-2\left(1-\mu\right)^{2}\mu R_{1}\left(1-2\mu\right)R_{3}^{'}\\
&-&2\left(1-\mu\right)^{3}\mu^{4}R_{1}^{'}-2\mu\left(1-\mu\right)\left(1-2\mu\right)R_{3}^{'}R_{1}^{'}
-2\mu^{3}\left(1-\mu\right)^{2}\left(1-2\mu\right)R_{1}R_{1}^{'}
\end{eqnarray*}

por tanto

\begin{eqnarray*}
\left[\frac{F\left(z\right)}{G\left(z\right)}\right]^{'}&=&\frac{1}{\mu^{3}\left(1-\mu\right)^{3}}\left\{
-\left(1-\mu\right)^{2}\mu^{2}R_{1}^{''}-\mu\left(1-\mu\right)\left(1-2\mu\right)R_{1}R_{3}^{''}
-\mu^{2}\left(1-\mu\right)^{2}R_{1}P^{''}\right.\\
&&\left.+2\mu\left[\left(1-2\mu\right)R_{1}-\left(1-\mu\right)\right]\left(R_{3}^{'}\right)^{2}
-2\left(1-\mu\right)\left(1-2\mu\right)R_{1}R_{3}^{'}-2\mu^{3}\left(1-\mu\right)^{2}R_{1}^{'}\right.\\
&&\left.-2\left(1-2\mu\right)R_{3}^{'}R_{1}^{'}-2\mu^{2}\left(1-\mu\right)\left(1-2\mu\right)R_{1}R_{1}^{'}\right\}
\end{eqnarray*}

recordemos que


\begin{eqnarray*}
R_{1}&=&-\esp L\\
R_{3}&=& -\mu\\
R_{1}^{'}&=&-\frac{1}{2}\esp L^{2}+\frac{1}{2}\esp L\\
R_{3}^{'}&=&-\frac{1}{2}\esp X^{2}+\frac{1}{2}\mu\\
R_{1}^{''}&=&-\frac{1}{3}\esp L^{3}+\esp L^{2}-\frac{2}{3}\esp L\\
R_{3}^{''}&=&-\frac{1}{3}\esp X^{3}+\esp X^{2}-\frac{2}{3}\mu\\
R_{1}R_{3}^{'}&=&\frac{1}{2}\esp X^{2}\esp L-\frac{1}{2}\esp X\esp L\\
R_{1}R_{1}^{'}&=&\frac{1}{2}\esp L^{2}\esp L+\frac{1}{2}\esp^{2}L\\
R_{3}^{'}R_{1}^{'}&=&\frac{1}{4}\esp X^{2}\esp L^{2}-\frac{1}{4}\esp X^{2}\esp L-\frac{1}{4}\esp L^{2}\esp X+\frac{1}{4}\esp X\esp L\\
R_{1}R_{3}^{''}&=&\frac{1}{6}\esp X^{3}\esp L^{2}-\frac{1}{6}\esp X^{3}\esp L-\frac{1}{2}\esp L^{2}\esp X^{2}+\frac{1}{2}\esp X^{2}\esp L+\frac{1}{3}\esp X\esp L^{2}-\frac{1}{3}\esp X\esp L\\
R_{1}P^{''}&=&-\esp X^{2}\esp L\\
\left(R_{3}^{'}\right)^{2}&=&\frac{1}{4}\esp^{2}X^{2}-\frac{1}{2}\esp X^{2}\esp X+\frac{1}{4}\esp^{2} X
\end{eqnarray*}




\begin{Def}
Let $L_{i}^{*}$ be the number of users at queue $Q_{i}$ when it is polled, then
\begin{eqnarray}
\begin{array}{cc}
\esp\left[L_{i}^{*}\right]=f_{i}\left(i\right), &
Var\left[L_{i}^{*}\right]=f_{i}\left(i,i\right)+\esp\left[L_{i}^{*}\right]-\esp\left[L_{i}^{*}\right]^{2}.
\end{array}
\end{eqnarray}
\end{Def}

\begin{Def}
The cycle time $C_{i}$ for the queue $Q_{i}$ is the period beginning at the time when it is polled in a cycle and ending at the time when it is polled in the next cycle; it's duration is given by $\tau_{i}\left(m+1\right)-\tau_{i}\left(m\right)$, equivalently $\overline{\tau}_{i}\left(m+1\right)-\overline{\tau}_{i}\left(m\right)$ under steady state assumption.
\end{Def}

\begin{Def}
The intervisit time $I_{i}$ is defined as the period beginning at the time of its service completion in a cycle and ending at the time when it is polled in the next cycle; its duration is given by $\tau_{i}\left(m+1\right)-\overline{\tau}_{i}\left(m\right)$.
\end{Def}

The intervisit time duration $\tau_{i}\left(m+1\right)-\overline{\tau}\left(m\right)$ given the number of users found at queue $Q_{i}$ at time $t=\tau_{i}\left(m+1\right)$ is equal to the number of arrivals during the preceding intervisit time $\left[\overline{\tau}\left(m\right),\tau_{i}\left(m+1\right)\right]$. 

So we have



\begin{eqnarray*}
\esp\left[z_{i}^{L_{i}\left(\tau_{i}\left(m+1\right)\right)}\right]=\esp\left[\left\{P_{i}\left(z_{i}\right)\right\}^{\tau_{i}\left(m+1\right)-\overline{\tau}\left(m\right)}\right]
\end{eqnarray*}

if $I_{i}\left(z\right)=\esp\left[z^{\tau_{i}\left(m+1\right)-\overline{\tau}\left(m\right)}\right]$
we have $F_{i}\left(z\right)=I_{i}\left[P_{i}\left(z\right)\right]$
for $i=1,2$. Futhermore can be proved that

\begin{eqnarray}
\begin{array}{ll}
\esp\left[L_{i}\right]=\mu_{i}\esp\left[I_{i}\right], &
\esp\left[C_{i}\right]=\frac{f_{i}\left(i\right)}{\mu_{i}\left(1-\mu_{i}\right)},\\
\esp\left[S_{i}\right]=\mu_{i}\esp\left[C_{i}\right],&
\esp\left[I_{i}\right]=\left(1-\mu_{i}\right)\esp\left[C_{i}\right],\\
Var\left[L_{i}\right]= \mu_{i}^{2}Var\left[I_{i}\right]+\sigma^{2}\esp\left[I_{i}\right],& 
Var\left[C_{i}\right]=\frac{Var\left[L_{i}^{*}\right]}{\mu_{i}^{2}\left(1-\mu_{i}\right)^{2}},\\
Var\left[S_{i}\right]= \frac{Var\left[L_{i}^{*}\right]}{\left(1-\mu_{i}\right)^{2}}+\frac{\sigma^{2}\esp\left[L_{i}^{*}\right]}{\left(1-\mu_{i}\right)^{3}},&
Var\left[I_{i}\right]= \frac{Var\left[L_{i}^{*}\right]}{\mu_{i}^{2}}-\frac{\sigma_{i}^{2}}{\mu_{i}^{2}}f_{i}\left(i\right).
\end{array}
\end{eqnarray}

Let consider the points when the process $\left[L_{1}\left(1\right),L_{2}\left(1\right),L_{3}\left(1\right),L_{4}\left(1\right)
\right]$ becomes zero at the same time, this points, $T_{1},T_{2},\ldots$ will be denoted as regeneration points, then we have that

\begin{Def}
the interval between two such succesive regeneration points will be called regenerative cycle.
\end{Def}

\begin{Def}
Para $T_{i}$ se define, $M_{i}$, el n\'umero de ciclos de visita a la cola $Q_{l}$, durante el ciclo regenerativo, es decir, $M_{i}$ es un proceso de renovaci\'on.
\end{Def}

\begin{Def}
Para cada uno de los $M_{i}$'s, se definen a su vez la duraci\'on de cada uno de estos ciclos de visita en el ciclo regenerativo, $C_{i}^{(m)}$, para $m=1,2,\ldots,M_{i}$, que a su vez, tambi\'en es n proceso de renovaci\'on.
\end{Def}



Sea la funci\'on generadora de momentos para $L_{i}$, el n\'umero de usuarios en la cola $Q_{i}\left(z\right)$ en cualquier momento, est\'a dada por el tiempo promedio de $z^{L_{i}\left(t\right)}$ sobre el ciclo regenerativo definido anteriormente. Entonces 

\begin{equation}\label{Eq.Longitud.Tiempo.t}
Q_{i}\left(z\right)=\frac{1}{\esp\left[C_{i}\right]}\cdot\frac{1-F_{i}\left(z\right)}{P_{i}\left(z\right)-z}\cdot\frac{\left(1-z\right)P_{i}\left(z\right)}{1-P_{i}\left(z\right)}.
\end{equation}

Es decir, es posible determinar las longitudes de las colas a cualquier tiempo $t$. Entonces, determinando el primer momento es posible ver que


$M_{i}$ is an stopping time for the regenerative process with $\esp\left[M_{i}\right]<\infty$, from Wald's lemma follows that:


\begin{eqnarray*}
\esp\left[\sum_{m=1}^{M_{i}}\sum_{t=\tau_{i}\left(m\right)}^{\tau_{i}\left(m+1\right)-1}z^{L_{i}\left(t\right)}\right]&=&\esp\left[M_{i}\right]\esp\left[\sum_{t=\tau_{i}\left(m\right)}^{\tau_{i}\left(m+1\right)-1}z^{L_{i}\left(t\right)}\right]\\
\esp\left[\sum_{m=1}^{M_{i}}\tau_{i}\left(m+1\right)-\tau_{i}\left(m\right)\right]&=&\esp\left[M_{i}\right]\esp\left[\tau_{i}\left(m+1\right)-\tau_{i}\left(m\right)\right]
\end{eqnarray*}
therefore 

\begin{eqnarray*}
Q_{i}\left(z\right)&=&\frac{\esp\left[\sum_{t=\tau_{i}\left(m\right)}^{\tau_{i}\left(m+1\right)-1}z^{L_{i}\left(t\right)}\right]}{\esp\left[\tau_{i}\left(m+1\right)-\tau_{i}\left(m\right)\right]}
\end{eqnarray*}

Doing the following substitutions en (\ref{Corolario2}): $n\rightarrow t-\tau_{i}\left(m\right)$, $T \rightarrow \overline{\tau}_{i}\left(m\right)-\tau_{i}\left(m\right)$, $L_{n}\rightarrow L_{i}\left(t\right)$ and $F\left(z\right)=\esp\left[z^{L_{0}}\right]\rightarrow F_{i}\left(z\right)=\esp\left[z^{L_{i}\tau_{i}\left(m\right)}\right]$, 
we obtain

\begin{eqnarray}\label{Eq.Arribos.Primera}
\esp\left[\sum_{n=0}^{T-1}z^{L_{n}}\right]=
\esp\left[\sum_{t=\tau_{i}\left(m\right)}^{\overline{\tau}_{i}\left(m\right)-1}z^{L_{i}\left(t\right)}\right]
=z\frac{F_{i}\left(z\right)-1}{z-P_{i}\left(z\right)}
\end{eqnarray}



Por otra parte durante el tiempo de intervisita para la cola $i$, $L_{i}\left(t\right)$ solamente se incrementa de manera que el incremento por intervalo de tiempo est\'a dado por la funci\'on generadora de probabilidades de $P_{i}\left(z\right)$, por tanto la suma sobre el tiempo de intervisita puede evaluarse como:

\begin{eqnarray*}
\esp\left[\sum_{t=\tau_{i}\left(m\right)}^{\tau_{i}\left(m+1\right)-1}z^{L_{i}\left(t\right)}\right]&=&\esp\left[\sum_{t=\tau_{i}\left(m\right)}^{\tau_{i}\left(m+1\right)-1}\left\{P_{i}\left(z\right)\right\}^{t-\overline{\tau}_{i}\left(m\right)}\right]=\frac{1-\esp\left[\left\{P_{i}\left(z\right)\right\}^{\tau_{i}\left(m+1\right)-\overline{\tau}_{i}\left(m\right)}\right]}{1-P_{i}\left(z\right)}\\
&=&\frac{1-I_{i}\left[P_{i}\left(z\right)\right]}{1-P_{i}\left(z\right)}
\end{eqnarray*}
por tanto

\begin{eqnarray*}
\esp\left[\sum_{t=\tau_{i}\left(m\right)}^{\tau_{i}\left(m+1\right)-1}z^{L_{i}\left(t\right)}\right]&=&
\frac{1-F_{i}\left(z\right)}{1-P_{i}\left(z\right)}
\end{eqnarray*}

Por lo tanto

\begin{eqnarray*}
Q_{i}\left(z\right)&=&\frac{\esp\left[\sum_{t=\tau_{i}\left(m\right)}^{\tau_{i}\left(m+1\right)-1}z^{L_{i}\left(t\right)}\right]}{\esp\left[\tau_{i}\left(m+1\right)-\tau_{i}\left(m\right)\right]}
=\frac{1}{\esp\left[\tau_{i}\left(m+1\right)-\tau_{i}\left(m\right)\right]}
\esp\left[\sum_{t=\tau_{i}\left(m\right)}^{\tau_{i}\left(m+1\right)-1}z^{L_{i}\left(t\right)}\right]\\
&=&\frac{1}{\esp\left[\tau_{i}\left(m+1\right)-\tau_{i}\left(m\right)\right]}
\esp\left[\sum_{t=\tau_{i}\left(m\right)}^{\overline{\tau}_{i}\left(m\right)-1}z^{L_{i}\left(t\right)}
+\sum_{t=\overline{\tau}_{i}\left(m\right)}^{\tau_{i}\left(m+1\right)-1}z^{L_{i}\left(t\right)}\right]\\
&=&\frac{1}{\esp\left[\tau_{i}\left(m+1\right)-\tau_{i}\left(m\right)\right]}\left\{
\esp\left[\sum_{t=\tau_{i}\left(m\right)}^{\overline{\tau}_{i}\left(m\right)-1}z^{L_{i}\left(t\right)}\right]
+\esp\left[\sum_{t=\overline{\tau}_{i}\left(m\right)}^{\tau_{i}\left(m+1\right)-1}z^{L_{i}\left(t\right)}\right]\right\}\\
&=&\frac{1}{\esp\left[\tau_{i}\left(m+1\right)-\tau_{i}\left(m\right)\right]}\left\{
z\frac{F_{i}\left(z\right)-1}{z-P_{i}\left(z\right)}+\frac{1-F_{i}\left(z\right)}{1-P_{i}\left(z\right)}
\right\}\\
&=&\frac{1}{\esp\left[C_{i}\right]}\cdot\frac{1-F_{i}\left(z\right)}{P_{i}\left(z\right)-z}\cdot\frac{\left(1-z\right)P_{i}\left(z\right)}{1-P_{i}\left(z\right)}
\end{eqnarray*}

es decir

\begin{eqnarray}
\begin{array}{ll}
S^{'}\left(z\right)=-\sum_{k=1}^{+\infty}ka_{k}z^{k-1},& S^{(1)}\left(1\right)=-\sum_{k=1}^{+\infty}ka_{k}=-\esp\left[L\left(t\right)\right],\\
S^{''}\left(z\right)=-\sum_{k=2}^{+\infty}k(k-1)a_{k}z^{k-2},& S^{(2)}\left(1\right)=-\sum_{k=2}^{+\infty}k(k-1)a_{k}=\esp\left[L\left(L-1\right)\right],\\
S^{'''}\left(z\right)=-\sum_{k=3}^{+\infty}k(k-1)(k-2)a_{k}z^{k-3},&
S^{(3)}\left(1\right)=-\sum_{k=3}^{+\infty}k(k-1)(k-2)a_{k}\\
&=-\esp\left[L\left(L-1\right)\left(L-2\right)\right]\\
&=-\esp\left[L^{3}\right]+3-\esp\left[L^{2}\right]-2-\esp\left[L\right];
\end{array}
\end{eqnarray}


%________________________________________________________________________
\subsection{Procesos Regenerativos Sigman, Thorisson y Wolff \cite{Sigman1}}
%________________________________________________________________________


\begin{Def}[Definici\'on Cl\'asica]
Un proceso estoc\'astico $X=\left\{X\left(t\right):t\geq0\right\}$ es llamado regenerativo is existe una variable aleatoria $R_{1}>0$ tal que
\begin{itemize}
\item[i)] $\left\{X\left(t+R_{1}\right):t\geq0\right\}$ es independiente de $\left\{\left\{X\left(t\right):t<R_{1}\right\},\right\}$
\item[ii)] $\left\{X\left(t+R_{1}\right):t\geq0\right\}$ es estoc\'asticamente equivalente a $\left\{X\left(t\right):t>0\right\}$
\end{itemize}

Llamamos a $R_{1}$ tiempo de regeneraci\'on, y decimos que $X$ se regenera en este punto.
\end{Def}

$\left\{X\left(t+R_{1}\right)\right\}$ es regenerativo con tiempo de regeneraci\'on $R_{2}$, independiente de $R_{1}$ pero con la misma distribuci\'on que $R_{1}$. Procediendo de esta manera se obtiene una secuencia de variables aleatorias independientes e id\'enticamente distribuidas $\left\{R_{n}\right\}$ llamados longitudes de ciclo. Si definimos a $Z_{k}\equiv R_{1}+R_{2}+\cdots+R_{k}$, se tiene un proceso de renovaci\'on llamado proceso de renovaci\'on encajado para $X$.


\begin{Note}
La existencia de un primer tiempo de regeneraci\'on, $R_{1}$, implica la existencia de una sucesi\'on completa de estos tiempos $R_{1},R_{2}\ldots,$ que satisfacen la propiedad deseada \cite{Sigman2}.
\end{Note}


\begin{Note} Para la cola $GI/GI/1$ los usuarios arriban con tiempos $t_{n}$ y son atendidos con tiempos de servicio $S_{n}$, los tiempos de arribo forman un proceso de renovaci\'on  con tiempos entre arribos independientes e identicamente distribuidos (\texttt{i.i.d.})$T_{n}=t_{n}-t_{n-1}$, adem\'as los tiempos de servicio son \texttt{i.i.d.} e independientes de los procesos de arribo. Por \textit{estable} se entiende que $\esp S_{n}<\esp T_{n}<\infty$.
\end{Note}
 


\begin{Def}
Para $x$ fijo y para cada $t\geq0$, sea $I_{x}\left(t\right)=1$ si $X\left(t\right)\leq x$,  $I_{x}\left(t\right)=0$ en caso contrario, y def\'inanse los tiempos promedio
\begin{eqnarray*}
\overline{X}&=&lim_{t\rightarrow\infty}\frac{1}{t}\int_{0}^{\infty}X\left(u\right)du\\
\prob\left(X_{\infty}\leq x\right)&=&lim_{t\rightarrow\infty}\frac{1}{t}\int_{0}^{\infty}I_{x}\left(u\right)du,
\end{eqnarray*}
cuando estos l\'imites existan.
\end{Def}

Como consecuencia del teorema de Renovaci\'on-Recompensa, se tiene que el primer l\'imite  existe y es igual a la constante
\begin{eqnarray*}
\overline{X}&=&\frac{\esp\left[\int_{0}^{R_{1}}X\left(t\right)dt\right]}{\esp\left[R_{1}\right]},
\end{eqnarray*}
suponiendo que ambas esperanzas son finitas.
 
\begin{Note}
Funciones de procesos regenerativos son regenerativas, es decir, si $X\left(t\right)$ es regenerativo y se define el proceso $Y\left(t\right)$ por $Y\left(t\right)=f\left(X\left(t\right)\right)$ para alguna funci\'on Borel medible $f\left(\cdot\right)$. Adem\'as $Y$ es regenerativo con los mismos tiempos de renovaci\'on que $X$. 

En general, los tiempos de renovaci\'on, $Z_{k}$ de un proceso regenerativo no requieren ser tiempos de paro con respecto a la evoluci\'on de $X\left(t\right)$.
\end{Note} 

\begin{Note}
Una funci\'on de un proceso de Markov, usualmente no ser\'a un proceso de Markov, sin embargo ser\'a regenerativo si el proceso de Markov lo es.
\end{Note}

 
\begin{Note}
Un proceso regenerativo con media de la longitud de ciclo finita es llamado positivo recurrente.
\end{Note}


\begin{Note}
\begin{itemize}
\item[a)] Si el proceso regenerativo $X$ es positivo recurrente y tiene trayectorias muestrales no negativas, entonces la ecuaci\'on anterior es v\'alida.
\item[b)] Si $X$ es positivo recurrente regenerativo, podemos construir una \'unica versi\'on estacionaria de este proceso, $X_{e}=\left\{X_{e}\left(t\right)\right\}$, donde $X_{e}$ es un proceso estoc\'astico regenerativo y estrictamente estacionario, con distribuci\'on marginal distribuida como $X_{\infty}$
\end{itemize}
\end{Note}


%__________________________________________________________________________________________
\subsection{Procesos Regenerativos Estacionarios - Stidham \cite{Stidham}}
%__________________________________________________________________________________________


Un proceso estoc\'astico a tiempo continuo $\left\{V\left(t\right),t\geq0\right\}$ es un proceso regenerativo si existe una sucesi\'on de variables aleatorias independientes e id\'enticamente distribuidas $\left\{X_{1},X_{2},\ldots\right\}$, sucesi\'on de renovaci\'on, tal que para cualquier conjunto de Borel $A$, 

\begin{eqnarray*}
\prob\left\{V\left(t\right)\in A|X_{1}+X_{2}+\cdots+X_{R\left(t\right)}=s,\left\{V\left(\tau\right),\tau<s\right\}\right\}=\prob\left\{V\left(t-s\right)\in A|X_{1}>t-s\right\},
\end{eqnarray*}
para todo $0\leq s\leq t$, donde $R\left(t\right)=\max\left\{X_{1}+X_{2}+\cdots+X_{j}\leq t\right\}=$n\'umero de renovaciones ({\emph{puntos de regeneraci\'on}}) que ocurren en $\left[0,t\right]$. El intervalo $\left[0,X_{1}\right)$ es llamado {\emph{primer ciclo de regeneraci\'on}} de $\left\{V\left(t \right),t\geq0\right\}$, $\left[X_{1},X_{1}+X_{2}\right)$ el {\emph{segundo ciclo de regeneraci\'on}}, y as\'i sucesivamente.

Sea $X=X_{1}$ y sea $F$ la funci\'on de distrbuci\'on de $X$


\begin{Def}
Se define el proceso estacionario, $\left\{V^{*}\left(t\right),t\geq0\right\}$, para $\left\{V\left(t\right),t\geq0\right\}$ por

\begin{eqnarray*}
\prob\left\{V\left(t\right)\in A\right\}=\frac{1}{\esp\left[X\right]}\int_{0}^{\infty}\prob\left\{V\left(t+x\right)\in A|X>x\right\}\left(1-F\left(x\right)\right)dx,
\end{eqnarray*} 
para todo $t\geq0$ y todo conjunto de Borel $A$.
\end{Def}

\begin{Def}
Una distribuci\'on se dice que es {\emph{aritm\'etica}} si todos sus puntos de incremento son m\'ultiplos de la forma $0,\lambda, 2\lambda,\ldots$ para alguna $\lambda>0$ entera.
\end{Def}


\begin{Def}
Una modificaci\'on medible de un proceso $\left\{V\left(t\right),t\geq0\right\}$, es una versi\'on de este, $\left\{V\left(t,w\right)\right\}$ conjuntamente medible para $t\geq0$ y para $w\in S$, $S$ espacio de estados para $\left\{V\left(t\right),t\geq0\right\}$.
\end{Def}

\begin{Teo}
Sea $\left\{V\left(t\right),t\geq\right\}$ un proceso regenerativo no negativo con modificaci\'on medible. Sea $\esp\left[X\right]<\infty$. Entonces el proceso estacionario dado por la ecuaci\'on anterior est\'a bien definido y tiene funci\'on de distribuci\'on independiente de $t$, adem\'as
\begin{itemize}
\item[i)] \begin{eqnarray*}
\esp\left[V^{*}\left(0\right)\right]&=&\frac{\esp\left[\int_{0}^{X}V\left(s\right)ds\right]}{\esp\left[X\right]}\end{eqnarray*}
\item[ii)] Si $\esp\left[V^{*}\left(0\right)\right]<\infty$, equivalentemente, si $\esp\left[\int_{0}^{X}V\left(s\right)ds\right]<\infty$,entonces
\begin{eqnarray*}
\frac{\int_{0}^{t}V\left(s\right)ds}{t}\rightarrow\frac{\esp\left[\int_{0}^{X}V\left(s\right)ds\right]}{\esp\left[X\right]}
\end{eqnarray*}
con probabilidad 1 y en media, cuando $t\rightarrow\infty$.
\end{itemize}
\end{Teo}

\begin{Coro}
Sea $\left\{V\left(t\right),t\geq0\right\}$ un proceso regenerativo no negativo, con modificaci\'on medible. Si $\esp <\infty$, $F$ es no-aritm\'etica, y para todo $x\geq0$, $P\left\{V\left(t\right)\leq x,C>x\right\}$ es de variaci\'on acotada como funci\'on de $t$ en cada intervalo finito $\left[0,\tau\right]$, entonces $V\left(t\right)$ converge en distribuci\'on  cuando $t\rightarrow\infty$ y $$\esp V=\frac{\esp \int_{0}^{X}V\left(s\right)ds}{\esp X}$$
Donde $V$ tiene la distribuci\'on l\'imite de $V\left(t\right)$ cuando $t\rightarrow\infty$.

\end{Coro}

Para el caso discreto se tienen resultados similares.



%______________________________________________________________________
\subsection{Procesos de Renovaci\'on}
%______________________________________________________________________

\begin{Def}%\label{Def.Tn}
Sean $0\leq T_{1}\leq T_{2}\leq \ldots$ son tiempos aleatorios infinitos en los cuales ocurren ciertos eventos. El n\'umero de tiempos $T_{n}$ en el intervalo $\left[0,t\right)$ es

\begin{eqnarray}
N\left(t\right)=\sum_{n=1}^{\infty}\indora\left(T_{n}\leq t\right),
\end{eqnarray}
para $t\geq0$.
\end{Def}

Si se consideran los puntos $T_{n}$ como elementos de $\rea_{+}$, y $N\left(t\right)$ es el n\'umero de puntos en $\rea$. El proceso denotado por $\left\{N\left(t\right):t\geq0\right\}$, denotado por $N\left(t\right)$, es un proceso puntual en $\rea_{+}$. Los $T_{n}$ son los tiempos de ocurrencia, el proceso puntual $N\left(t\right)$ es simple si su n\'umero de ocurrencias son distintas: $0<T_{1}<T_{2}<\ldots$ casi seguramente.

\begin{Def}
Un proceso puntual $N\left(t\right)$ es un proceso de renovaci\'on si los tiempos de interocurrencia $\xi_{n}=T_{n}-T_{n-1}$, para $n\geq1$, son independientes e identicamente distribuidos con distribuci\'on $F$, donde $F\left(0\right)=0$ y $T_{0}=0$. Los $T_{n}$ son llamados tiempos de renovaci\'on, referente a la independencia o renovaci\'on de la informaci\'on estoc\'astica en estos tiempos. Los $\xi_{n}$ son los tiempos de inter-renovaci\'on, y $N\left(t\right)$ es el n\'umero de renovaciones en el intervalo $\left[0,t\right)$
\end{Def}


\begin{Note}
Para definir un proceso de renovaci\'on para cualquier contexto, solamente hay que especificar una distribuci\'on $F$, con $F\left(0\right)=0$, para los tiempos de inter-renovaci\'on. La funci\'on $F$ en turno degune las otra variables aleatorias. De manera formal, existe un espacio de probabilidad y una sucesi\'on de variables aleatorias $\xi_{1},\xi_{2},\ldots$ definidas en este con distribuci\'on $F$. Entonces las otras cantidades son $T_{n}=\sum_{k=1}^{n}\xi_{k}$ y $N\left(t\right)=\sum_{n=1}^{\infty}\indora\left(T_{n}\leq t\right)$, donde $T_{n}\rightarrow\infty$ casi seguramente por la Ley Fuerte de los Grandes Números.
\end{Note}

%___________________________________________________________________________________________
%
%\subsection{Teorema Principal de Renovaci\'on}
%___________________________________________________________________________________________
%

\begin{Note} Una funci\'on $h:\rea_{+}\rightarrow\rea$ es Directamente Riemann Integrable en los siguientes casos:
\begin{itemize}
\item[a)] $h\left(t\right)\geq0$ es decreciente y Riemann Integrable.
\item[b)] $h$ es continua excepto posiblemente en un conjunto de Lebesgue de medida 0, y $|h\left(t\right)|\leq b\left(t\right)$, donde $b$ es DRI.
\end{itemize}
\end{Note}

\begin{Teo}[Teorema Principal de Renovaci\'on]
Si $F$ es no aritm\'etica y $h\left(t\right)$ es Directamente Riemann Integrable (DRI), entonces

\begin{eqnarray*}
lim_{t\rightarrow\infty}U\star h=\frac{1}{\mu}\int_{\rea_{+}}h\left(s\right)ds.
\end{eqnarray*}
\end{Teo}

\begin{Prop}
Cualquier funci\'on $H\left(t\right)$ acotada en intervalos finitos y que es 0 para $t<0$ puede expresarse como
\begin{eqnarray*}
H\left(t\right)=U\star h\left(t\right)\textrm{,  donde }h\left(t\right)=H\left(t\right)-F\star H\left(t\right)
\end{eqnarray*}
\end{Prop}

\begin{Def}
Un proceso estoc\'astico $X\left(t\right)$ es crudamente regenerativo en un tiempo aleatorio positivo $T$ si
\begin{eqnarray*}
\esp\left[X\left(T+t\right)|T\right]=\esp\left[X\left(t\right)\right]\textrm{, para }t\geq0,\end{eqnarray*}
y con las esperanzas anteriores finitas.
\end{Def}

\begin{Prop}
Sup\'ongase que $X\left(t\right)$ es un proceso crudamente regenerativo en $T$, que tiene distribuci\'on $F$. Si $\esp\left[X\left(t\right)\right]$ es acotado en intervalos finitos, entonces
\begin{eqnarray*}
\esp\left[X\left(t\right)\right]=U\star h\left(t\right)\textrm{,  donde }h\left(t\right)=\esp\left[X\left(t\right)\indora\left(T>t\right)\right].
\end{eqnarray*}
\end{Prop}

\begin{Teo}[Regeneraci\'on Cruda]
Sup\'ongase que $X\left(t\right)$ es un proceso con valores positivo crudamente regenerativo en $T$, y def\'inase $M=\sup\left\{|X\left(t\right)|:t\leq T\right\}$. Si $T$ es no aritm\'etico y $M$ y $MT$ tienen media finita, entonces
\begin{eqnarray*}
lim_{t\rightarrow\infty}\esp\left[X\left(t\right)\right]=\frac{1}{\mu}\int_{\rea_{+}}h\left(s\right)ds,
\end{eqnarray*}
donde $h\left(t\right)=\esp\left[X\left(t\right)\indora\left(T>t\right)\right]$.
\end{Teo}

%___________________________________________________________________________________________
%
\subsection{Propiedades de los Procesos de Renovaci\'on}
%___________________________________________________________________________________________
%

Los tiempos $T_{n}$ est\'an relacionados con los conteos de $N\left(t\right)$ por

\begin{eqnarray*}
\left\{N\left(t\right)\geq n\right\}&=&\left\{T_{n}\leq t\right\}\\
T_{N\left(t\right)}\leq &t&<T_{N\left(t\right)+1},
\end{eqnarray*}

adem\'as $N\left(T_{n}\right)=n$, y 

\begin{eqnarray*}
N\left(t\right)=\max\left\{n:T_{n}\leq t\right\}=\min\left\{n:T_{n+1}>t\right\}
\end{eqnarray*}

Por propiedades de la convoluci\'on se sabe que

\begin{eqnarray*}
P\left\{T_{n}\leq t\right\}=F^{n\star}\left(t\right)
\end{eqnarray*}
que es la $n$-\'esima convoluci\'on de $F$. Entonces 

\begin{eqnarray*}
\left\{N\left(t\right)\geq n\right\}&=&\left\{T_{n}\leq t\right\}\\
P\left\{N\left(t\right)\leq n\right\}&=&1-F^{\left(n+1\right)\star}\left(t\right)
\end{eqnarray*}

Adem\'as usando el hecho de que $\esp\left[N\left(t\right)\right]=\sum_{n=1}^{\infty}P\left\{N\left(t\right)\geq n\right\}$
se tiene que

\begin{eqnarray*}
\esp\left[N\left(t\right)\right]=\sum_{n=1}^{\infty}F^{n\star}\left(t\right)
\end{eqnarray*}

\begin{Prop}
Para cada $t\geq0$, la funci\'on generadora de momentos $\esp\left[e^{\alpha N\left(t\right)}\right]$ existe para alguna $\alpha$ en una vecindad del 0, y de aqu\'i que $\esp\left[N\left(t\right)^{m}\right]<\infty$, para $m\geq1$.
\end{Prop}


\begin{Note}
Si el primer tiempo de renovaci\'on $\xi_{1}$ no tiene la misma distribuci\'on que el resto de las $\xi_{n}$, para $n\geq2$, a $N\left(t\right)$ se le llama Proceso de Renovaci\'on retardado, donde si $\xi$ tiene distribuci\'on $G$, entonces el tiempo $T_{n}$ de la $n$-\'esima renovaci\'on tiene distribuci\'on $G\star F^{\left(n-1\right)\star}\left(t\right)$
\end{Note}


\begin{Teo}
Para una constante $\mu\leq\infty$ ( o variable aleatoria), las siguientes expresiones son equivalentes:

\begin{eqnarray}
lim_{n\rightarrow\infty}n^{-1}T_{n}&=&\mu,\textrm{ c.s.}\\
lim_{t\rightarrow\infty}t^{-1}N\left(t\right)&=&1/\mu,\textrm{ c.s.}
\end{eqnarray}
\end{Teo}


Es decir, $T_{n}$ satisface la Ley Fuerte de los Grandes N\'umeros s\'i y s\'olo s\'i $N\left/t\right)$ la cumple.


\begin{Coro}[Ley Fuerte de los Grandes N\'umeros para Procesos de Renovaci\'on]
Si $N\left(t\right)$ es un proceso de renovaci\'on cuyos tiempos de inter-renovaci\'on tienen media $\mu\leq\infty$, entonces
\begin{eqnarray}
t^{-1}N\left(t\right)\rightarrow 1/\mu,\textrm{ c.s. cuando }t\rightarrow\infty.
\end{eqnarray}

\end{Coro}


Considerar el proceso estoc\'astico de valores reales $\left\{Z\left(t\right):t\geq0\right\}$ en el mismo espacio de probabilidad que $N\left(t\right)$

\begin{Def}
Para el proceso $\left\{Z\left(t\right):t\geq0\right\}$ se define la fluctuaci\'on m\'axima de $Z\left(t\right)$ en el intervalo $\left(T_{n-1},T_{n}\right]$:
\begin{eqnarray*}
M_{n}=\sup_{T_{n-1}<t\leq T_{n}}|Z\left(t\right)-Z\left(T_{n-1}\right)|
\end{eqnarray*}
\end{Def}

\begin{Teo}
Sup\'ongase que $n^{-1}T_{n}\rightarrow\mu$ c.s. cuando $n\rightarrow\infty$, donde $\mu\leq\infty$ es una constante o variable aleatoria. Sea $a$ una constante o variable aleatoria que puede ser infinita cuando $\mu$ es finita, y considere las expresiones l\'imite:
\begin{eqnarray}
lim_{n\rightarrow\infty}n^{-1}Z\left(T_{n}\right)&=&a,\textrm{ c.s.}\\
lim_{t\rightarrow\infty}t^{-1}Z\left(t\right)&=&a/\mu,\textrm{ c.s.}
\end{eqnarray}
La segunda expresi\'on implica la primera. Conversamente, la primera implica la segunda si el proceso $Z\left(t\right)$ es creciente, o si $lim_{n\rightarrow\infty}n^{-1}M_{n}=0$ c.s.
\end{Teo}

\begin{Coro}
Si $N\left(t\right)$ es un proceso de renovaci\'on, y $\left(Z\left(T_{n}\right)-Z\left(T_{n-1}\right),M_{n}\right)$, para $n\geq1$, son variables aleatorias independientes e id\'enticamente distribuidas con media finita, entonces,
\begin{eqnarray}
lim_{t\rightarrow\infty}t^{-1}Z\left(t\right)\rightarrow\frac{\esp\left[Z\left(T_{1}\right)-Z\left(T_{0}\right)\right]}{\esp\left[T_{1}\right]},\textrm{ c.s. cuando  }t\rightarrow\infty.
\end{eqnarray}
\end{Coro}



%___________________________________________________________________________________________
%
\subsection{Propiedades de los Procesos de Renovaci\'on}
%___________________________________________________________________________________________
%

Los tiempos $T_{n}$ est\'an relacionados con los conteos de $N\left(t\right)$ por

\begin{eqnarray*}
\left\{N\left(t\right)\geq n\right\}&=&\left\{T_{n}\leq t\right\}\\
T_{N\left(t\right)}\leq &t&<T_{N\left(t\right)+1},
\end{eqnarray*}

adem\'as $N\left(T_{n}\right)=n$, y 

\begin{eqnarray*}
N\left(t\right)=\max\left\{n:T_{n}\leq t\right\}=\min\left\{n:T_{n+1}>t\right\}
\end{eqnarray*}

Por propiedades de la convoluci\'on se sabe que

\begin{eqnarray*}
P\left\{T_{n}\leq t\right\}=F^{n\star}\left(t\right)
\end{eqnarray*}
que es la $n$-\'esima convoluci\'on de $F$. Entonces 

\begin{eqnarray*}
\left\{N\left(t\right)\geq n\right\}&=&\left\{T_{n}\leq t\right\}\\
P\left\{N\left(t\right)\leq n\right\}&=&1-F^{\left(n+1\right)\star}\left(t\right)
\end{eqnarray*}

Adem\'as usando el hecho de que $\esp\left[N\left(t\right)\right]=\sum_{n=1}^{\infty}P\left\{N\left(t\right)\geq n\right\}$
se tiene que

\begin{eqnarray*}
\esp\left[N\left(t\right)\right]=\sum_{n=1}^{\infty}F^{n\star}\left(t\right)
\end{eqnarray*}

\begin{Prop}
Para cada $t\geq0$, la funci\'on generadora de momentos $\esp\left[e^{\alpha N\left(t\right)}\right]$ existe para alguna $\alpha$ en una vecindad del 0, y de aqu\'i que $\esp\left[N\left(t\right)^{m}\right]<\infty$, para $m\geq1$.
\end{Prop}


\begin{Note}
Si el primer tiempo de renovaci\'on $\xi_{1}$ no tiene la misma distribuci\'on que el resto de las $\xi_{n}$, para $n\geq2$, a $N\left(t\right)$ se le llama Proceso de Renovaci\'on retardado, donde si $\xi$ tiene distribuci\'on $G$, entonces el tiempo $T_{n}$ de la $n$-\'esima renovaci\'on tiene distribuci\'on $G\star F^{\left(n-1\right)\star}\left(t\right)$
\end{Note}


\begin{Teo}
Para una constante $\mu\leq\infty$ ( o variable aleatoria), las siguientes expresiones son equivalentes:

\begin{eqnarray}
lim_{n\rightarrow\infty}n^{-1}T_{n}&=&\mu,\textrm{ c.s.}\\
lim_{t\rightarrow\infty}t^{-1}N\left(t\right)&=&1/\mu,\textrm{ c.s.}
\end{eqnarray}
\end{Teo}


Es decir, $T_{n}$ satisface la Ley Fuerte de los Grandes N\'umeros s\'i y s\'olo s\'i $N\left/t\right)$ la cumple.


\begin{Coro}[Ley Fuerte de los Grandes N\'umeros para Procesos de Renovaci\'on]
Si $N\left(t\right)$ es un proceso de renovaci\'on cuyos tiempos de inter-renovaci\'on tienen media $\mu\leq\infty$, entonces
\begin{eqnarray}
t^{-1}N\left(t\right)\rightarrow 1/\mu,\textrm{ c.s. cuando }t\rightarrow\infty.
\end{eqnarray}

\end{Coro}


Considerar el proceso estoc\'astico de valores reales $\left\{Z\left(t\right):t\geq0\right\}$ en el mismo espacio de probabilidad que $N\left(t\right)$

\begin{Def}
Para el proceso $\left\{Z\left(t\right):t\geq0\right\}$ se define la fluctuaci\'on m\'axima de $Z\left(t\right)$ en el intervalo $\left(T_{n-1},T_{n}\right]$:
\begin{eqnarray*}
M_{n}=\sup_{T_{n-1}<t\leq T_{n}}|Z\left(t\right)-Z\left(T_{n-1}\right)|
\end{eqnarray*}
\end{Def}

\begin{Teo}
Sup\'ongase que $n^{-1}T_{n}\rightarrow\mu$ c.s. cuando $n\rightarrow\infty$, donde $\mu\leq\infty$ es una constante o variable aleatoria. Sea $a$ una constante o variable aleatoria que puede ser infinita cuando $\mu$ es finita, y considere las expresiones l\'imite:
\begin{eqnarray}
lim_{n\rightarrow\infty}n^{-1}Z\left(T_{n}\right)&=&a,\textrm{ c.s.}\\
lim_{t\rightarrow\infty}t^{-1}Z\left(t\right)&=&a/\mu,\textrm{ c.s.}
\end{eqnarray}
La segunda expresi\'on implica la primera. Conversamente, la primera implica la segunda si el proceso $Z\left(t\right)$ es creciente, o si $lim_{n\rightarrow\infty}n^{-1}M_{n}=0$ c.s.
\end{Teo}

\begin{Coro}
Si $N\left(t\right)$ es un proceso de renovaci\'on, y $\left(Z\left(T_{n}\right)-Z\left(T_{n-1}\right),M_{n}\right)$, para $n\geq1$, son variables aleatorias independientes e id\'enticamente distribuidas con media finita, entonces,
\begin{eqnarray}
lim_{t\rightarrow\infty}t^{-1}Z\left(t\right)\rightarrow\frac{\esp\left[Z\left(T_{1}\right)-Z\left(T_{0}\right)\right]}{\esp\left[T_{1}\right]},\textrm{ c.s. cuando  }t\rightarrow\infty.
\end{eqnarray}
\end{Coro}


%___________________________________________________________________________________________
%
\subsection{Propiedades de los Procesos de Renovaci\'on}
%___________________________________________________________________________________________
%

Los tiempos $T_{n}$ est\'an relacionados con los conteos de $N\left(t\right)$ por

\begin{eqnarray*}
\left\{N\left(t\right)\geq n\right\}&=&\left\{T_{n}\leq t\right\}\\
T_{N\left(t\right)}\leq &t&<T_{N\left(t\right)+1},
\end{eqnarray*}

adem\'as $N\left(T_{n}\right)=n$, y 

\begin{eqnarray*}
N\left(t\right)=\max\left\{n:T_{n}\leq t\right\}=\min\left\{n:T_{n+1}>t\right\}
\end{eqnarray*}

Por propiedades de la convoluci\'on se sabe que

\begin{eqnarray*}
P\left\{T_{n}\leq t\right\}=F^{n\star}\left(t\right)
\end{eqnarray*}
que es la $n$-\'esima convoluci\'on de $F$. Entonces 

\begin{eqnarray*}
\left\{N\left(t\right)\geq n\right\}&=&\left\{T_{n}\leq t\right\}\\
P\left\{N\left(t\right)\leq n\right\}&=&1-F^{\left(n+1\right)\star}\left(t\right)
\end{eqnarray*}

Adem\'as usando el hecho de que $\esp\left[N\left(t\right)\right]=\sum_{n=1}^{\infty}P\left\{N\left(t\right)\geq n\right\}$
se tiene que

\begin{eqnarray*}
\esp\left[N\left(t\right)\right]=\sum_{n=1}^{\infty}F^{n\star}\left(t\right)
\end{eqnarray*}

\begin{Prop}
Para cada $t\geq0$, la funci\'on generadora de momentos $\esp\left[e^{\alpha N\left(t\right)}\right]$ existe para alguna $\alpha$ en una vecindad del 0, y de aqu\'i que $\esp\left[N\left(t\right)^{m}\right]<\infty$, para $m\geq1$.
\end{Prop}


\begin{Note}
Si el primer tiempo de renovaci\'on $\xi_{1}$ no tiene la misma distribuci\'on que el resto de las $\xi_{n}$, para $n\geq2$, a $N\left(t\right)$ se le llama Proceso de Renovaci\'on retardado, donde si $\xi$ tiene distribuci\'on $G$, entonces el tiempo $T_{n}$ de la $n$-\'esima renovaci\'on tiene distribuci\'on $G\star F^{\left(n-1\right)\star}\left(t\right)$
\end{Note}


\begin{Teo}
Para una constante $\mu\leq\infty$ ( o variable aleatoria), las siguientes expresiones son equivalentes:

\begin{eqnarray}
lim_{n\rightarrow\infty}n^{-1}T_{n}&=&\mu,\textrm{ c.s.}\\
lim_{t\rightarrow\infty}t^{-1}N\left(t\right)&=&1/\mu,\textrm{ c.s.}
\end{eqnarray}
\end{Teo}


Es decir, $T_{n}$ satisface la Ley Fuerte de los Grandes N\'umeros s\'i y s\'olo s\'i $N\left/t\right)$ la cumple.


\begin{Coro}[Ley Fuerte de los Grandes N\'umeros para Procesos de Renovaci\'on]
Si $N\left(t\right)$ es un proceso de renovaci\'on cuyos tiempos de inter-renovaci\'on tienen media $\mu\leq\infty$, entonces
\begin{eqnarray}
t^{-1}N\left(t\right)\rightarrow 1/\mu,\textrm{ c.s. cuando }t\rightarrow\infty.
\end{eqnarray}

\end{Coro}


Considerar el proceso estoc\'astico de valores reales $\left\{Z\left(t\right):t\geq0\right\}$ en el mismo espacio de probabilidad que $N\left(t\right)$

\begin{Def}
Para el proceso $\left\{Z\left(t\right):t\geq0\right\}$ se define la fluctuaci\'on m\'axima de $Z\left(t\right)$ en el intervalo $\left(T_{n-1},T_{n}\right]$:
\begin{eqnarray*}
M_{n}=\sup_{T_{n-1}<t\leq T_{n}}|Z\left(t\right)-Z\left(T_{n-1}\right)|
\end{eqnarray*}
\end{Def}

\begin{Teo}
Sup\'ongase que $n^{-1}T_{n}\rightarrow\mu$ c.s. cuando $n\rightarrow\infty$, donde $\mu\leq\infty$ es una constante o variable aleatoria. Sea $a$ una constante o variable aleatoria que puede ser infinita cuando $\mu$ es finita, y considere las expresiones l\'imite:
\begin{eqnarray}
lim_{n\rightarrow\infty}n^{-1}Z\left(T_{n}\right)&=&a,\textrm{ c.s.}\\
lim_{t\rightarrow\infty}t^{-1}Z\left(t\right)&=&a/\mu,\textrm{ c.s.}
\end{eqnarray}
La segunda expresi\'on implica la primera. Conversamente, la primera implica la segunda si el proceso $Z\left(t\right)$ es creciente, o si $lim_{n\rightarrow\infty}n^{-1}M_{n}=0$ c.s.
\end{Teo}

\begin{Coro}
Si $N\left(t\right)$ es un proceso de renovaci\'on, y $\left(Z\left(T_{n}\right)-Z\left(T_{n-1}\right),M_{n}\right)$, para $n\geq1$, son variables aleatorias independientes e id\'enticamente distribuidas con media finita, entonces,
\begin{eqnarray}
lim_{t\rightarrow\infty}t^{-1}Z\left(t\right)\rightarrow\frac{\esp\left[Z\left(T_{1}\right)-Z\left(T_{0}\right)\right]}{\esp\left[T_{1}\right]},\textrm{ c.s. cuando  }t\rightarrow\infty.
\end{eqnarray}
\end{Coro}

%___________________________________________________________________________________________
%
\subsection{Propiedades de los Procesos de Renovaci\'on}
%___________________________________________________________________________________________
%

Los tiempos $T_{n}$ est\'an relacionados con los conteos de $N\left(t\right)$ por

\begin{eqnarray*}
\left\{N\left(t\right)\geq n\right\}&=&\left\{T_{n}\leq t\right\}\\
T_{N\left(t\right)}\leq &t&<T_{N\left(t\right)+1},
\end{eqnarray*}

adem\'as $N\left(T_{n}\right)=n$, y 

\begin{eqnarray*}
N\left(t\right)=\max\left\{n:T_{n}\leq t\right\}=\min\left\{n:T_{n+1}>t\right\}
\end{eqnarray*}

Por propiedades de la convoluci\'on se sabe que

\begin{eqnarray*}
P\left\{T_{n}\leq t\right\}=F^{n\star}\left(t\right)
\end{eqnarray*}
que es la $n$-\'esima convoluci\'on de $F$. Entonces 

\begin{eqnarray*}
\left\{N\left(t\right)\geq n\right\}&=&\left\{T_{n}\leq t\right\}\\
P\left\{N\left(t\right)\leq n\right\}&=&1-F^{\left(n+1\right)\star}\left(t\right)
\end{eqnarray*}

Adem\'as usando el hecho de que $\esp\left[N\left(t\right)\right]=\sum_{n=1}^{\infty}P\left\{N\left(t\right)\geq n\right\}$
se tiene que

\begin{eqnarray*}
\esp\left[N\left(t\right)\right]=\sum_{n=1}^{\infty}F^{n\star}\left(t\right)
\end{eqnarray*}

\begin{Prop}
Para cada $t\geq0$, la funci\'on generadora de momentos $\esp\left[e^{\alpha N\left(t\right)}\right]$ existe para alguna $\alpha$ en una vecindad del 0, y de aqu\'i que $\esp\left[N\left(t\right)^{m}\right]<\infty$, para $m\geq1$.
\end{Prop}


\begin{Note}
Si el primer tiempo de renovaci\'on $\xi_{1}$ no tiene la misma distribuci\'on que el resto de las $\xi_{n}$, para $n\geq2$, a $N\left(t\right)$ se le llama Proceso de Renovaci\'on retardado, donde si $\xi$ tiene distribuci\'on $G$, entonces el tiempo $T_{n}$ de la $n$-\'esima renovaci\'on tiene distribuci\'on $G\star F^{\left(n-1\right)\star}\left(t\right)$
\end{Note}


\begin{Teo}
Para una constante $\mu\leq\infty$ ( o variable aleatoria), las siguientes expresiones son equivalentes:

\begin{eqnarray}
lim_{n\rightarrow\infty}n^{-1}T_{n}&=&\mu,\textrm{ c.s.}\\
lim_{t\rightarrow\infty}t^{-1}N\left(t\right)&=&1/\mu,\textrm{ c.s.}
\end{eqnarray}
\end{Teo}


Es decir, $T_{n}$ satisface la Ley Fuerte de los Grandes N\'umeros s\'i y s\'olo s\'i $N\left/t\right)$ la cumple.


\begin{Coro}[Ley Fuerte de los Grandes N\'umeros para Procesos de Renovaci\'on]
Si $N\left(t\right)$ es un proceso de renovaci\'on cuyos tiempos de inter-renovaci\'on tienen media $\mu\leq\infty$, entonces
\begin{eqnarray}
t^{-1}N\left(t\right)\rightarrow 1/\mu,\textrm{ c.s. cuando }t\rightarrow\infty.
\end{eqnarray}

\end{Coro}


Considerar el proceso estoc\'astico de valores reales $\left\{Z\left(t\right):t\geq0\right\}$ en el mismo espacio de probabilidad que $N\left(t\right)$

\begin{Def}
Para el proceso $\left\{Z\left(t\right):t\geq0\right\}$ se define la fluctuaci\'on m\'axima de $Z\left(t\right)$ en el intervalo $\left(T_{n-1},T_{n}\right]$:
\begin{eqnarray*}
M_{n}=\sup_{T_{n-1}<t\leq T_{n}}|Z\left(t\right)-Z\left(T_{n-1}\right)|
\end{eqnarray*}
\end{Def}

\begin{Teo}
Sup\'ongase que $n^{-1}T_{n}\rightarrow\mu$ c.s. cuando $n\rightarrow\infty$, donde $\mu\leq\infty$ es una constante o variable aleatoria. Sea $a$ una constante o variable aleatoria que puede ser infinita cuando $\mu$ es finita, y considere las expresiones l\'imite:
\begin{eqnarray}
lim_{n\rightarrow\infty}n^{-1}Z\left(T_{n}\right)&=&a,\textrm{ c.s.}\\
lim_{t\rightarrow\infty}t^{-1}Z\left(t\right)&=&a/\mu,\textrm{ c.s.}
\end{eqnarray}
La segunda expresi\'on implica la primera. Conversamente, la primera implica la segunda si el proceso $Z\left(t\right)$ es creciente, o si $lim_{n\rightarrow\infty}n^{-1}M_{n}=0$ c.s.
\end{Teo}

\begin{Coro}
Si $N\left(t\right)$ es un proceso de renovaci\'on, y $\left(Z\left(T_{n}\right)-Z\left(T_{n-1}\right),M_{n}\right)$, para $n\geq1$, son variables aleatorias independientes e id\'enticamente distribuidas con media finita, entonces,
\begin{eqnarray}
lim_{t\rightarrow\infty}t^{-1}Z\left(t\right)\rightarrow\frac{\esp\left[Z\left(T_{1}\right)-Z\left(T_{0}\right)\right]}{\esp\left[T_{1}\right]},\textrm{ c.s. cuando  }t\rightarrow\infty.
\end{eqnarray}
\end{Coro}
%___________________________________________________________________________________________
%
\subsection{Propiedades de los Procesos de Renovaci\'on}
%___________________________________________________________________________________________
%

Los tiempos $T_{n}$ est\'an relacionados con los conteos de $N\left(t\right)$ por

\begin{eqnarray*}
\left\{N\left(t\right)\geq n\right\}&=&\left\{T_{n}\leq t\right\}\\
T_{N\left(t\right)}\leq &t&<T_{N\left(t\right)+1},
\end{eqnarray*}

adem\'as $N\left(T_{n}\right)=n$, y 

\begin{eqnarray*}
N\left(t\right)=\max\left\{n:T_{n}\leq t\right\}=\min\left\{n:T_{n+1}>t\right\}
\end{eqnarray*}

Por propiedades de la convoluci\'on se sabe que

\begin{eqnarray*}
P\left\{T_{n}\leq t\right\}=F^{n\star}\left(t\right)
\end{eqnarray*}
que es la $n$-\'esima convoluci\'on de $F$. Entonces 

\begin{eqnarray*}
\left\{N\left(t\right)\geq n\right\}&=&\left\{T_{n}\leq t\right\}\\
P\left\{N\left(t\right)\leq n\right\}&=&1-F^{\left(n+1\right)\star}\left(t\right)
\end{eqnarray*}

Adem\'as usando el hecho de que $\esp\left[N\left(t\right)\right]=\sum_{n=1}^{\infty}P\left\{N\left(t\right)\geq n\right\}$
se tiene que

\begin{eqnarray*}
\esp\left[N\left(t\right)\right]=\sum_{n=1}^{\infty}F^{n\star}\left(t\right)
\end{eqnarray*}

\begin{Prop}
Para cada $t\geq0$, la funci\'on generadora de momentos $\esp\left[e^{\alpha N\left(t\right)}\right]$ existe para alguna $\alpha$ en una vecindad del 0, y de aqu\'i que $\esp\left[N\left(t\right)^{m}\right]<\infty$, para $m\geq1$.
\end{Prop}


\begin{Note}
Si el primer tiempo de renovaci\'on $\xi_{1}$ no tiene la misma distribuci\'on que el resto de las $\xi_{n}$, para $n\geq2$, a $N\left(t\right)$ se le llama Proceso de Renovaci\'on retardado, donde si $\xi$ tiene distribuci\'on $G$, entonces el tiempo $T_{n}$ de la $n$-\'esima renovaci\'on tiene distribuci\'on $G\star F^{\left(n-1\right)\star}\left(t\right)$
\end{Note}


\begin{Teo}
Para una constante $\mu\leq\infty$ ( o variable aleatoria), las siguientes expresiones son equivalentes:

\begin{eqnarray}
lim_{n\rightarrow\infty}n^{-1}T_{n}&=&\mu,\textrm{ c.s.}\\
lim_{t\rightarrow\infty}t^{-1}N\left(t\right)&=&1/\mu,\textrm{ c.s.}
\end{eqnarray}
\end{Teo}


Es decir, $T_{n}$ satisface la Ley Fuerte de los Grandes N\'umeros s\'i y s\'olo s\'i $N\left/t\right)$ la cumple.


\begin{Coro}[Ley Fuerte de los Grandes N\'umeros para Procesos de Renovaci\'on]
Si $N\left(t\right)$ es un proceso de renovaci\'on cuyos tiempos de inter-renovaci\'on tienen media $\mu\leq\infty$, entonces
\begin{eqnarray}
t^{-1}N\left(t\right)\rightarrow 1/\mu,\textrm{ c.s. cuando }t\rightarrow\infty.
\end{eqnarray}

\end{Coro}


Considerar el proceso estoc\'astico de valores reales $\left\{Z\left(t\right):t\geq0\right\}$ en el mismo espacio de probabilidad que $N\left(t\right)$

\begin{Def}
Para el proceso $\left\{Z\left(t\right):t\geq0\right\}$ se define la fluctuaci\'on m\'axima de $Z\left(t\right)$ en el intervalo $\left(T_{n-1},T_{n}\right]$:
\begin{eqnarray*}
M_{n}=\sup_{T_{n-1}<t\leq T_{n}}|Z\left(t\right)-Z\left(T_{n-1}\right)|
\end{eqnarray*}
\end{Def}

\begin{Teo}
Sup\'ongase que $n^{-1}T_{n}\rightarrow\mu$ c.s. cuando $n\rightarrow\infty$, donde $\mu\leq\infty$ es una constante o variable aleatoria. Sea $a$ una constante o variable aleatoria que puede ser infinita cuando $\mu$ es finita, y considere las expresiones l\'imite:
\begin{eqnarray}
lim_{n\rightarrow\infty}n^{-1}Z\left(T_{n}\right)&=&a,\textrm{ c.s.}\\
lim_{t\rightarrow\infty}t^{-1}Z\left(t\right)&=&a/\mu,\textrm{ c.s.}
\end{eqnarray}
La segunda expresi\'on implica la primera. Conversamente, la primera implica la segunda si el proceso $Z\left(t\right)$ es creciente, o si $lim_{n\rightarrow\infty}n^{-1}M_{n}=0$ c.s.
\end{Teo}

\begin{Coro}
Si $N\left(t\right)$ es un proceso de renovaci\'on, y $\left(Z\left(T_{n}\right)-Z\left(T_{n-1}\right),M_{n}\right)$, para $n\geq1$, son variables aleatorias independientes e id\'enticamente distribuidas con media finita, entonces,
\begin{eqnarray}
lim_{t\rightarrow\infty}t^{-1}Z\left(t\right)\rightarrow\frac{\esp\left[Z\left(T_{1}\right)-Z\left(T_{0}\right)\right]}{\esp\left[T_{1}\right]},\textrm{ c.s. cuando  }t\rightarrow\infty.
\end{eqnarray}
\end{Coro}


%___________________________________________________________________________________________
%
\subsection{Funci\'on de Renovaci\'on}
%___________________________________________________________________________________________
%


\begin{Def}
Sea $h\left(t\right)$ funci\'on de valores reales en $\rea$ acotada en intervalos finitos e igual a cero para $t<0$ La ecuaci\'on de renovaci\'on para $h\left(t\right)$ y la distribuci\'on $F$ es

\begin{eqnarray}%\label{Ec.Renovacion}
H\left(t\right)=h\left(t\right)+\int_{\left[0,t\right]}H\left(t-s\right)dF\left(s\right)\textrm{,    }t\geq0,
\end{eqnarray}
donde $H\left(t\right)$ es una funci\'on de valores reales. Esto es $H=h+F\star H$. Decimos que $H\left(t\right)$ es soluci\'on de esta ecuaci\'on si satisface la ecuaci\'on, y es acotada en intervalos finitos e iguales a cero para $t<0$.
\end{Def}

\begin{Prop}
La funci\'on $U\star h\left(t\right)$ es la \'unica soluci\'on de la ecuaci\'on de renovaci\'on (\ref{Ec.Renovacion}).
\end{Prop}

\begin{Teo}[Teorema Renovaci\'on Elemental]
\begin{eqnarray*}
t^{-1}U\left(t\right)\rightarrow 1/\mu\textrm{,    cuando }t\rightarrow\infty.
\end{eqnarray*}
\end{Teo}

%___________________________________________________________________________________________
%
\subsection{Funci\'on de Renovaci\'on}
%___________________________________________________________________________________________
%


Sup\'ongase que $N\left(t\right)$ es un proceso de renovaci\'on con distribuci\'on $F$ con media finita $\mu$.

\begin{Def}
La funci\'on de renovaci\'on asociada con la distribuci\'on $F$, del proceso $N\left(t\right)$, es
\begin{eqnarray*}
U\left(t\right)=\sum_{n=1}^{\infty}F^{n\star}\left(t\right),\textrm{   }t\geq0,
\end{eqnarray*}
donde $F^{0\star}\left(t\right)=\indora\left(t\geq0\right)$.
\end{Def}


\begin{Prop}
Sup\'ongase que la distribuci\'on de inter-renovaci\'on $F$ tiene densidad $f$. Entonces $U\left(t\right)$ tambi\'en tiene densidad, para $t>0$, y es $U^{'}\left(t\right)=\sum_{n=0}^{\infty}f^{n\star}\left(t\right)$. Adem\'as
\begin{eqnarray*}
\prob\left\{N\left(t\right)>N\left(t-\right)\right\}=0\textrm{,   }t\geq0.
\end{eqnarray*}
\end{Prop}

\begin{Def}
La Transformada de Laplace-Stieljes de $F$ est\'a dada por

\begin{eqnarray*}
\hat{F}\left(\alpha\right)=\int_{\rea_{+}}e^{-\alpha t}dF\left(t\right)\textrm{,  }\alpha\geq0.
\end{eqnarray*}
\end{Def}

Entonces

\begin{eqnarray*}
\hat{U}\left(\alpha\right)=\sum_{n=0}^{\infty}\hat{F^{n\star}}\left(\alpha\right)=\sum_{n=0}^{\infty}\hat{F}\left(\alpha\right)^{n}=\frac{1}{1-\hat{F}\left(\alpha\right)}.
\end{eqnarray*}


\begin{Prop}
La Transformada de Laplace $\hat{U}\left(\alpha\right)$ y $\hat{F}\left(\alpha\right)$ determina una a la otra de manera \'unica por la relaci\'on $\hat{U}\left(\alpha\right)=\frac{1}{1-\hat{F}\left(\alpha\right)}$.
\end{Prop}


\begin{Note}
Un proceso de renovaci\'on $N\left(t\right)$ cuyos tiempos de inter-renovaci\'on tienen media finita, es un proceso Poisson con tasa $\lambda$ si y s\'olo s\'i $\esp\left[U\left(t\right)\right]=\lambda t$, para $t\geq0$.
\end{Note}


\begin{Teo}
Sea $N\left(t\right)$ un proceso puntual simple con puntos de localizaci\'on $T_{n}$ tal que $\eta\left(t\right)=\esp\left[N\left(\right)\right]$ es finita para cada $t$. Entonces para cualquier funci\'on $f:\rea_{+}\rightarrow\rea$,
\begin{eqnarray*}
\esp\left[\sum_{n=1}^{N\left(\right)}f\left(T_{n}\right)\right]=\int_{\left(0,t\right]}f\left(s\right)d\eta\left(s\right)\textrm{,  }t\geq0,
\end{eqnarray*}
suponiendo que la integral exista. Adem\'as si $X_{1},X_{2},\ldots$ son variables aleatorias definidas en el mismo espacio de probabilidad que el proceso $N\left(t\right)$ tal que $\esp\left[X_{n}|T_{n}=s\right]=f\left(s\right)$, independiente de $n$. Entonces
\begin{eqnarray*}
\esp\left[\sum_{n=1}^{N\left(t\right)}X_{n}\right]=\int_{\left(0,t\right]}f\left(s\right)d\eta\left(s\right)\textrm{,  }t\geq0,
\end{eqnarray*} 
suponiendo que la integral exista. 
\end{Teo}

\begin{Coro}[Identidad de Wald para Renovaciones]
Para el proceso de renovaci\'on $N\left(t\right)$,
\begin{eqnarray*}
\esp\left[T_{N\left(t\right)+1}\right]=\mu\esp\left[N\left(t\right)+1\right]\textrm{,  }t\geq0,
\end{eqnarray*}  
\end{Coro}

%______________________________________________________________________
\subsection{Procesos de Renovaci\'on}
%______________________________________________________________________

\begin{Def}%\label{Def.Tn}
Sean $0\leq T_{1}\leq T_{2}\leq \ldots$ son tiempos aleatorios infinitos en los cuales ocurren ciertos eventos. El n\'umero de tiempos $T_{n}$ en el intervalo $\left[0,t\right)$ es

\begin{eqnarray}
N\left(t\right)=\sum_{n=1}^{\infty}\indora\left(T_{n}\leq t\right),
\end{eqnarray}
para $t\geq0$.
\end{Def}

Si se consideran los puntos $T_{n}$ como elementos de $\rea_{+}$, y $N\left(t\right)$ es el n\'umero de puntos en $\rea$. El proceso denotado por $\left\{N\left(t\right):t\geq0\right\}$, denotado por $N\left(t\right)$, es un proceso puntual en $\rea_{+}$. Los $T_{n}$ son los tiempos de ocurrencia, el proceso puntual $N\left(t\right)$ es simple si su n\'umero de ocurrencias son distintas: $0<T_{1}<T_{2}<\ldots$ casi seguramente.

\begin{Def}
Un proceso puntual $N\left(t\right)$ es un proceso de renovaci\'on si los tiempos de interocurrencia $\xi_{n}=T_{n}-T_{n-1}$, para $n\geq1$, son independientes e identicamente distribuidos con distribuci\'on $F$, donde $F\left(0\right)=0$ y $T_{0}=0$. Los $T_{n}$ son llamados tiempos de renovaci\'on, referente a la independencia o renovaci\'on de la informaci\'on estoc\'astica en estos tiempos. Los $\xi_{n}$ son los tiempos de inter-renovaci\'on, y $N\left(t\right)$ es el n\'umero de renovaciones en el intervalo $\left[0,t\right)$
\end{Def}


\begin{Note}
Para definir un proceso de renovaci\'on para cualquier contexto, solamente hay que especificar una distribuci\'on $F$, con $F\left(0\right)=0$, para los tiempos de inter-renovaci\'on. La funci\'on $F$ en turno degune las otra variables aleatorias. De manera formal, existe un espacio de probabilidad y una sucesi\'on de variables aleatorias $\xi_{1},\xi_{2},\ldots$ definidas en este con distribuci\'on $F$. Entonces las otras cantidades son $T_{n}=\sum_{k=1}^{n}\xi_{k}$ y $N\left(t\right)=\sum_{n=1}^{\infty}\indora\left(T_{n}\leq t\right)$, donde $T_{n}\rightarrow\infty$ casi seguramente por la Ley Fuerte de los Grandes Números.
\end{Note}

%___________________________________________________________________________________________
%
%\subsection{Renewal and Regenerative Processes: Serfozo\cite{Serfozo}}
%___________________________________________________________________________________________
%
\begin{Def}%\label{Def.Tn}
Sean $0\leq T_{1}\leq T_{2}\leq \ldots$ son tiempos aleatorios infinitos en los cuales ocurren ciertos eventos. El n\'umero de tiempos $T_{n}$ en el intervalo $\left[0,t\right)$ es

\begin{eqnarray}
N\left(t\right)=\sum_{n=1}^{\infty}\indora\left(T_{n}\leq t\right),
\end{eqnarray}
para $t\geq0$.
\end{Def}

Si se consideran los puntos $T_{n}$ como elementos de $\rea_{+}$, y $N\left(t\right)$ es el n\'umero de puntos en $\rea$. El proceso denotado por $\left\{N\left(t\right):t\geq0\right\}$, denotado por $N\left(t\right)$, es un proceso puntual en $\rea_{+}$. Los $T_{n}$ son los tiempos de ocurrencia, el proceso puntual $N\left(t\right)$ es simple si su n\'umero de ocurrencias son distintas: $0<T_{1}<T_{2}<\ldots$ casi seguramente.

\begin{Def}
Un proceso puntual $N\left(t\right)$ es un proceso de renovaci\'on si los tiempos de interocurrencia $\xi_{n}=T_{n}-T_{n-1}$, para $n\geq1$, son independientes e identicamente distribuidos con distribuci\'on $F$, donde $F\left(0\right)=0$ y $T_{0}=0$. Los $T_{n}$ son llamados tiempos de renovaci\'on, referente a la independencia o renovaci\'on de la informaci\'on estoc\'astica en estos tiempos. Los $\xi_{n}$ son los tiempos de inter-renovaci\'on, y $N\left(t\right)$ es el n\'umero de renovaciones en el intervalo $\left[0,t\right)$
\end{Def}


\begin{Note}
Para definir un proceso de renovaci\'on para cualquier contexto, solamente hay que especificar una distribuci\'on $F$, con $F\left(0\right)=0$, para los tiempos de inter-renovaci\'on. La funci\'on $F$ en turno degune las otra variables aleatorias. De manera formal, existe un espacio de probabilidad y una sucesi\'on de variables aleatorias $\xi_{1},\xi_{2},\ldots$ definidas en este con distribuci\'on $F$. Entonces las otras cantidades son $T_{n}=\sum_{k=1}^{n}\xi_{k}$ y $N\left(t\right)=\sum_{n=1}^{\infty}\indora\left(T_{n}\leq t\right)$, donde $T_{n}\rightarrow\infty$ casi seguramente por la Ley Fuerte de los Grandes N\'umeros.
\end{Note}







Los tiempos $T_{n}$ est\'an relacionados con los conteos de $N\left(t\right)$ por

\begin{eqnarray*}
\left\{N\left(t\right)\geq n\right\}&=&\left\{T_{n}\leq t\right\}\\
T_{N\left(t\right)}\leq &t&<T_{N\left(t\right)+1},
\end{eqnarray*}

adem\'as $N\left(T_{n}\right)=n$, y 

\begin{eqnarray*}
N\left(t\right)=\max\left\{n:T_{n}\leq t\right\}=\min\left\{n:T_{n+1}>t\right\}
\end{eqnarray*}

Por propiedades de la convoluci\'on se sabe que

\begin{eqnarray*}
P\left\{T_{n}\leq t\right\}=F^{n\star}\left(t\right)
\end{eqnarray*}
que es la $n$-\'esima convoluci\'on de $F$. Entonces 

\begin{eqnarray*}
\left\{N\left(t\right)\geq n\right\}&=&\left\{T_{n}\leq t\right\}\\
P\left\{N\left(t\right)\leq n\right\}&=&1-F^{\left(n+1\right)\star}\left(t\right)
\end{eqnarray*}

Adem\'as usando el hecho de que $\esp\left[N\left(t\right)\right]=\sum_{n=1}^{\infty}P\left\{N\left(t\right)\geq n\right\}$
se tiene que

\begin{eqnarray*}
\esp\left[N\left(t\right)\right]=\sum_{n=1}^{\infty}F^{n\star}\left(t\right)
\end{eqnarray*}

\begin{Prop}
Para cada $t\geq0$, la funci\'on generadora de momentos $\esp\left[e^{\alpha N\left(t\right)}\right]$ existe para alguna $\alpha$ en una vecindad del 0, y de aqu\'i que $\esp\left[N\left(t\right)^{m}\right]<\infty$, para $m\geq1$.
\end{Prop}

\begin{Ejem}[\textbf{Proceso Poisson}]

Suponga que se tienen tiempos de inter-renovaci\'on \textit{i.i.d.} del proceso de renovaci\'on $N\left(t\right)$ tienen distribuci\'on exponencial $F\left(t\right)=q-e^{-\lambda t}$ con tasa $\lambda$. Entonces $N\left(t\right)$ es un proceso Poisson con tasa $\lambda$.

\end{Ejem}


\begin{Note}
Si el primer tiempo de renovaci\'on $\xi_{1}$ no tiene la misma distribuci\'on que el resto de las $\xi_{n}$, para $n\geq2$, a $N\left(t\right)$ se le llama Proceso de Renovaci\'on retardado, donde si $\xi$ tiene distribuci\'on $G$, entonces el tiempo $T_{n}$ de la $n$-\'esima renovaci\'on tiene distribuci\'on $G\star F^{\left(n-1\right)\star}\left(t\right)$
\end{Note}


\begin{Teo}
Para una constante $\mu\leq\infty$ ( o variable aleatoria), las siguientes expresiones son equivalentes:

\begin{eqnarray}
lim_{n\rightarrow\infty}n^{-1}T_{n}&=&\mu,\textrm{ c.s.}\\
lim_{t\rightarrow\infty}t^{-1}N\left(t\right)&=&1/\mu,\textrm{ c.s.}
\end{eqnarray}
\end{Teo}


Es decir, $T_{n}$ satisface la Ley Fuerte de los Grandes N\'umeros s\'i y s\'olo s\'i $N\left/t\right)$ la cumple.


\begin{Coro}[Ley Fuerte de los Grandes N\'umeros para Procesos de Renovaci\'on]
Si $N\left(t\right)$ es un proceso de renovaci\'on cuyos tiempos de inter-renovaci\'on tienen media $\mu\leq\infty$, entonces
\begin{eqnarray}
t^{-1}N\left(t\right)\rightarrow 1/\mu,\textrm{ c.s. cuando }t\rightarrow\infty.
\end{eqnarray}

\end{Coro}


Considerar el proceso estoc\'astico de valores reales $\left\{Z\left(t\right):t\geq0\right\}$ en el mismo espacio de probabilidad que $N\left(t\right)$

\begin{Def}
Para el proceso $\left\{Z\left(t\right):t\geq0\right\}$ se define la fluctuaci\'on m\'axima de $Z\left(t\right)$ en el intervalo $\left(T_{n-1},T_{n}\right]$:
\begin{eqnarray*}
M_{n}=\sup_{T_{n-1}<t\leq T_{n}}|Z\left(t\right)-Z\left(T_{n-1}\right)|
\end{eqnarray*}
\end{Def}

\begin{Teo}
Sup\'ongase que $n^{-1}T_{n}\rightarrow\mu$ c.s. cuando $n\rightarrow\infty$, donde $\mu\leq\infty$ es una constante o variable aleatoria. Sea $a$ una constante o variable aleatoria que puede ser infinita cuando $\mu$ es finita, y considere las expresiones l\'imite:
\begin{eqnarray}
lim_{n\rightarrow\infty}n^{-1}Z\left(T_{n}\right)&=&a,\textrm{ c.s.}\\
lim_{t\rightarrow\infty}t^{-1}Z\left(t\right)&=&a/\mu,\textrm{ c.s.}
\end{eqnarray}
La segunda expresi\'on implica la primera. Conversamente, la primera implica la segunda si el proceso $Z\left(t\right)$ es creciente, o si $lim_{n\rightarrow\infty}n^{-1}M_{n}=0$ c.s.
\end{Teo}

\begin{Coro}
Si $N\left(t\right)$ es un proceso de renovaci\'on, y $\left(Z\left(T_{n}\right)-Z\left(T_{n-1}\right),M_{n}\right)$, para $n\geq1$, son variables aleatorias independientes e id\'enticamente distribuidas con media finita, entonces,
\begin{eqnarray}
lim_{t\rightarrow\infty}t^{-1}Z\left(t\right)\rightarrow\frac{\esp\left[Z\left(T_{1}\right)-Z\left(T_{0}\right)\right]}{\esp\left[T_{1}\right]},\textrm{ c.s. cuando  }t\rightarrow\infty.
\end{eqnarray}
\end{Coro}


Sup\'ongase que $N\left(t\right)$ es un proceso de renovaci\'on con distribuci\'on $F$ con media finita $\mu$.

\begin{Def}
La funci\'on de renovaci\'on asociada con la distribuci\'on $F$, del proceso $N\left(t\right)$, es
\begin{eqnarray*}
U\left(t\right)=\sum_{n=1}^{\infty}F^{n\star}\left(t\right),\textrm{   }t\geq0,
\end{eqnarray*}
donde $F^{0\star}\left(t\right)=\indora\left(t\geq0\right)$.
\end{Def}


\begin{Prop}
Sup\'ongase que la distribuci\'on de inter-renovaci\'on $F$ tiene densidad $f$. Entonces $U\left(t\right)$ tambi\'en tiene densidad, para $t>0$, y es $U^{'}\left(t\right)=\sum_{n=0}^{\infty}f^{n\star}\left(t\right)$. Adem\'as
\begin{eqnarray*}
\prob\left\{N\left(t\right)>N\left(t-\right)\right\}=0\textrm{,   }t\geq0.
\end{eqnarray*}
\end{Prop}

\begin{Def}
La Transformada de Laplace-Stieljes de $F$ est\'a dada por

\begin{eqnarray*}
\hat{F}\left(\alpha\right)=\int_{\rea_{+}}e^{-\alpha t}dF\left(t\right)\textrm{,  }\alpha\geq0.
\end{eqnarray*}
\end{Def}

Entonces

\begin{eqnarray*}
\hat{U}\left(\alpha\right)=\sum_{n=0}^{\infty}\hat{F^{n\star}}\left(\alpha\right)=\sum_{n=0}^{\infty}\hat{F}\left(\alpha\right)^{n}=\frac{1}{1-\hat{F}\left(\alpha\right)}.
\end{eqnarray*}


\begin{Prop}
La Transformada de Laplace $\hat{U}\left(\alpha\right)$ y $\hat{F}\left(\alpha\right)$ determina una a la otra de manera \'unica por la relaci\'on $\hat{U}\left(\alpha\right)=\frac{1}{1-\hat{F}\left(\alpha\right)}$.
\end{Prop}


\begin{Note}
Un proceso de renovaci\'on $N\left(t\right)$ cuyos tiempos de inter-renovaci\'on tienen media finita, es un proceso Poisson con tasa $\lambda$ si y s\'olo s\'i $\esp\left[U\left(t\right)\right]=\lambda t$, para $t\geq0$.
\end{Note}


\begin{Teo}
Sea $N\left(t\right)$ un proceso puntual simple con puntos de localizaci\'on $T_{n}$ tal que $\eta\left(t\right)=\esp\left[N\left(\right)\right]$ es finita para cada $t$. Entonces para cualquier funci\'on $f:\rea_{+}\rightarrow\rea$,
\begin{eqnarray*}
\esp\left[\sum_{n=1}^{N\left(\right)}f\left(T_{n}\right)\right]=\int_{\left(0,t\right]}f\left(s\right)d\eta\left(s\right)\textrm{,  }t\geq0,
\end{eqnarray*}
suponiendo que la integral exista. Adem\'as si $X_{1},X_{2},\ldots$ son variables aleatorias definidas en el mismo espacio de probabilidad que el proceso $N\left(t\right)$ tal que $\esp\left[X_{n}|T_{n}=s\right]=f\left(s\right)$, independiente de $n$. Entonces
\begin{eqnarray*}
\esp\left[\sum_{n=1}^{N\left(t\right)}X_{n}\right]=\int_{\left(0,t\right]}f\left(s\right)d\eta\left(s\right)\textrm{,  }t\geq0,
\end{eqnarray*} 
suponiendo que la integral exista. 
\end{Teo}

\begin{Coro}[Identidad de Wald para Renovaciones]
Para el proceso de renovaci\'on $N\left(t\right)$,
\begin{eqnarray*}
\esp\left[T_{N\left(t\right)+1}\right]=\mu\esp\left[N\left(t\right)+1\right]\textrm{,  }t\geq0,
\end{eqnarray*}  
\end{Coro}


\begin{Def}
Sea $h\left(t\right)$ funci\'on de valores reales en $\rea$ acotada en intervalos finitos e igual a cero para $t<0$ La ecuaci\'on de renovaci\'on para $h\left(t\right)$ y la distribuci\'on $F$ es

\begin{eqnarray}%\label{Ec.Renovacion}
H\left(t\right)=h\left(t\right)+\int_{\left[0,t\right]}H\left(t-s\right)dF\left(s\right)\textrm{,    }t\geq0,
\end{eqnarray}
donde $H\left(t\right)$ es una funci\'on de valores reales. Esto es $H=h+F\star H$. Decimos que $H\left(t\right)$ es soluci\'on de esta ecuaci\'on si satisface la ecuaci\'on, y es acotada en intervalos finitos e iguales a cero para $t<0$.
\end{Def}

\begin{Prop}
La funci\'on $U\star h\left(t\right)$ es la \'unica soluci\'on de la ecuaci\'on de renovaci\'on (\ref{Ec.Renovacion}).
\end{Prop}

\begin{Teo}[Teorema Renovaci\'on Elemental]
\begin{eqnarray*}
t^{-1}U\left(t\right)\rightarrow 1/\mu\textrm{,    cuando }t\rightarrow\infty.
\end{eqnarray*}
\end{Teo}



Sup\'ongase que $N\left(t\right)$ es un proceso de renovaci\'on con distribuci\'on $F$ con media finita $\mu$.

\begin{Def}
La funci\'on de renovaci\'on asociada con la distribuci\'on $F$, del proceso $N\left(t\right)$, es
\begin{eqnarray*}
U\left(t\right)=\sum_{n=1}^{\infty}F^{n\star}\left(t\right),\textrm{   }t\geq0,
\end{eqnarray*}
donde $F^{0\star}\left(t\right)=\indora\left(t\geq0\right)$.
\end{Def}


\begin{Prop}
Sup\'ongase que la distribuci\'on de inter-renovaci\'on $F$ tiene densidad $f$. Entonces $U\left(t\right)$ tambi\'en tiene densidad, para $t>0$, y es $U^{'}\left(t\right)=\sum_{n=0}^{\infty}f^{n\star}\left(t\right)$. Adem\'as
\begin{eqnarray*}
\prob\left\{N\left(t\right)>N\left(t-\right)\right\}=0\textrm{,   }t\geq0.
\end{eqnarray*}
\end{Prop}

\begin{Def}
La Transformada de Laplace-Stieljes de $F$ est\'a dada por

\begin{eqnarray*}
\hat{F}\left(\alpha\right)=\int_{\rea_{+}}e^{-\alpha t}dF\left(t\right)\textrm{,  }\alpha\geq0.
\end{eqnarray*}
\end{Def}

Entonces

\begin{eqnarray*}
\hat{U}\left(\alpha\right)=\sum_{n=0}^{\infty}\hat{F^{n\star}}\left(\alpha\right)=\sum_{n=0}^{\infty}\hat{F}\left(\alpha\right)^{n}=\frac{1}{1-\hat{F}\left(\alpha\right)}.
\end{eqnarray*}


\begin{Prop}
La Transformada de Laplace $\hat{U}\left(\alpha\right)$ y $\hat{F}\left(\alpha\right)$ determina una a la otra de manera \'unica por la relaci\'on $\hat{U}\left(\alpha\right)=\frac{1}{1-\hat{F}\left(\alpha\right)}$.
\end{Prop}


\begin{Note}
Un proceso de renovaci\'on $N\left(t\right)$ cuyos tiempos de inter-renovaci\'on tienen media finita, es un proceso Poisson con tasa $\lambda$ si y s\'olo s\'i $\esp\left[U\left(t\right)\right]=\lambda t$, para $t\geq0$.
\end{Note}


\begin{Teo}
Sea $N\left(t\right)$ un proceso puntual simple con puntos de localizaci\'on $T_{n}$ tal que $\eta\left(t\right)=\esp\left[N\left(\right)\right]$ es finita para cada $t$. Entonces para cualquier funci\'on $f:\rea_{+}\rightarrow\rea$,
\begin{eqnarray*}
\esp\left[\sum_{n=1}^{N\left(\right)}f\left(T_{n}\right)\right]=\int_{\left(0,t\right]}f\left(s\right)d\eta\left(s\right)\textrm{,  }t\geq0,
\end{eqnarray*}
suponiendo que la integral exista. Adem\'as si $X_{1},X_{2},\ldots$ son variables aleatorias definidas en el mismo espacio de probabilidad que el proceso $N\left(t\right)$ tal que $\esp\left[X_{n}|T_{n}=s\right]=f\left(s\right)$, independiente de $n$. Entonces
\begin{eqnarray*}
\esp\left[\sum_{n=1}^{N\left(t\right)}X_{n}\right]=\int_{\left(0,t\right]}f\left(s\right)d\eta\left(s\right)\textrm{,  }t\geq0,
\end{eqnarray*} 
suponiendo que la integral exista. 
\end{Teo}

\begin{Coro}[Identidad de Wald para Renovaciones]
Para el proceso de renovaci\'on $N\left(t\right)$,
\begin{eqnarray*}
\esp\left[T_{N\left(t\right)+1}\right]=\mu\esp\left[N\left(t\right)+1\right]\textrm{,  }t\geq0,
\end{eqnarray*}  
\end{Coro}


\begin{Def}
Sea $h\left(t\right)$ funci\'on de valores reales en $\rea$ acotada en intervalos finitos e igual a cero para $t<0$ La ecuaci\'on de renovaci\'on para $h\left(t\right)$ y la distribuci\'on $F$ es

\begin{eqnarray}%\label{Ec.Renovacion}
H\left(t\right)=h\left(t\right)+\int_{\left[0,t\right]}H\left(t-s\right)dF\left(s\right)\textrm{,    }t\geq0,
\end{eqnarray}
donde $H\left(t\right)$ es una funci\'on de valores reales. Esto es $H=h+F\star H$. Decimos que $H\left(t\right)$ es soluci\'on de esta ecuaci\'on si satisface la ecuaci\'on, y es acotada en intervalos finitos e iguales a cero para $t<0$.
\end{Def}

\begin{Prop}
La funci\'on $U\star h\left(t\right)$ es la \'unica soluci\'on de la ecuaci\'on de renovaci\'on (\ref{Ec.Renovacion}).
\end{Prop}

\begin{Teo}[Teorema Renovaci\'on Elemental]
\begin{eqnarray*}
t^{-1}U\left(t\right)\rightarrow 1/\mu\textrm{,    cuando }t\rightarrow\infty.
\end{eqnarray*}
\end{Teo}


\begin{Note} Una funci\'on $h:\rea_{+}\rightarrow\rea$ es Directamente Riemann Integrable en los siguientes casos:
\begin{itemize}
\item[a)] $h\left(t\right)\geq0$ es decreciente y Riemann Integrable.
\item[b)] $h$ es continua excepto posiblemente en un conjunto de Lebesgue de medida 0, y $|h\left(t\right)|\leq b\left(t\right)$, donde $b$ es DRI.
\end{itemize}
\end{Note}

\begin{Teo}[Teorema Principal de Renovaci\'on]
Si $F$ es no aritm\'etica y $h\left(t\right)$ es Directamente Riemann Integrable (DRI), entonces

\begin{eqnarray*}
lim_{t\rightarrow\infty}U\star h=\frac{1}{\mu}\int_{\rea_{+}}h\left(s\right)ds.
\end{eqnarray*}
\end{Teo}

\begin{Prop}
Cualquier funci\'on $H\left(t\right)$ acotada en intervalos finitos y que es 0 para $t<0$ puede expresarse como
\begin{eqnarray*}
H\left(t\right)=U\star h\left(t\right)\textrm{,  donde }h\left(t\right)=H\left(t\right)-F\star H\left(t\right)
\end{eqnarray*}
\end{Prop}

\begin{Def}
Un proceso estoc\'astico $X\left(t\right)$ es crudamente regenerativo en un tiempo aleatorio positivo $T$ si
\begin{eqnarray*}
\esp\left[X\left(T+t\right)|T\right]=\esp\left[X\left(t\right)\right]\textrm{, para }t\geq0,\end{eqnarray*}
y con las esperanzas anteriores finitas.
\end{Def}

\begin{Prop}
Sup\'ongase que $X\left(t\right)$ es un proceso crudamente regenerativo en $T$, que tiene distribuci\'on $F$. Si $\esp\left[X\left(t\right)\right]$ es acotado en intervalos finitos, entonces
\begin{eqnarray*}
\esp\left[X\left(t\right)\right]=U\star h\left(t\right)\textrm{,  donde }h\left(t\right)=\esp\left[X\left(t\right)\indora\left(T>t\right)\right].
\end{eqnarray*}
\end{Prop}

\begin{Teo}[Regeneraci\'on Cruda]
Sup\'ongase que $X\left(t\right)$ es un proceso con valores positivo crudamente regenerativo en $T$, y def\'inase $M=\sup\left\{|X\left(t\right)|:t\leq T\right\}$. Si $T$ es no aritm\'etico y $M$ y $MT$ tienen media finita, entonces
\begin{eqnarray*}
lim_{t\rightarrow\infty}\esp\left[X\left(t\right)\right]=\frac{1}{\mu}\int_{\rea_{+}}h\left(s\right)ds,
\end{eqnarray*}
donde $h\left(t\right)=\esp\left[X\left(t\right)\indora\left(T>t\right)\right]$.
\end{Teo}


\begin{Note} Una funci\'on $h:\rea_{+}\rightarrow\rea$ es Directamente Riemann Integrable en los siguientes casos:
\begin{itemize}
\item[a)] $h\left(t\right)\geq0$ es decreciente y Riemann Integrable.
\item[b)] $h$ es continua excepto posiblemente en un conjunto de Lebesgue de medida 0, y $|h\left(t\right)|\leq b\left(t\right)$, donde $b$ es DRI.
\end{itemize}
\end{Note}

\begin{Teo}[Teorema Principal de Renovaci\'on]
Si $F$ es no aritm\'etica y $h\left(t\right)$ es Directamente Riemann Integrable (DRI), entonces

\begin{eqnarray*}
lim_{t\rightarrow\infty}U\star h=\frac{1}{\mu}\int_{\rea_{+}}h\left(s\right)ds.
\end{eqnarray*}
\end{Teo}

\begin{Prop}
Cualquier funci\'on $H\left(t\right)$ acotada en intervalos finitos y que es 0 para $t<0$ puede expresarse como
\begin{eqnarray*}
H\left(t\right)=U\star h\left(t\right)\textrm{,  donde }h\left(t\right)=H\left(t\right)-F\star H\left(t\right)
\end{eqnarray*}
\end{Prop}

\begin{Def}
Un proceso estoc\'astico $X\left(t\right)$ es crudamente regenerativo en un tiempo aleatorio positivo $T$ si
\begin{eqnarray*}
\esp\left[X\left(T+t\right)|T\right]=\esp\left[X\left(t\right)\right]\textrm{, para }t\geq0,\end{eqnarray*}
y con las esperanzas anteriores finitas.
\end{Def}

\begin{Prop}
Sup\'ongase que $X\left(t\right)$ es un proceso crudamente regenerativo en $T$, que tiene distribuci\'on $F$. Si $\esp\left[X\left(t\right)\right]$ es acotado en intervalos finitos, entonces
\begin{eqnarray*}
\esp\left[X\left(t\right)\right]=U\star h\left(t\right)\textrm{,  donde }h\left(t\right)=\esp\left[X\left(t\right)\indora\left(T>t\right)\right].
\end{eqnarray*}
\end{Prop}

\begin{Teo}[Regeneraci\'on Cruda]
Sup\'ongase que $X\left(t\right)$ es un proceso con valores positivo crudamente regenerativo en $T$, y def\'inase $M=\sup\left\{|X\left(t\right)|:t\leq T\right\}$. Si $T$ es no aritm\'etico y $M$ y $MT$ tienen media finita, entonces
\begin{eqnarray*}
lim_{t\rightarrow\infty}\esp\left[X\left(t\right)\right]=\frac{1}{\mu}\int_{\rea_{+}}h\left(s\right)ds,
\end{eqnarray*}
donde $h\left(t\right)=\esp\left[X\left(t\right)\indora\left(T>t\right)\right]$.
\end{Teo}

\begin{Def}
Para el proceso $\left\{\left(N\left(t\right),X\left(t\right)\right):t\geq0\right\}$, sus trayectoria muestrales en el intervalo de tiempo $\left[T_{n-1},T_{n}\right)$ est\'an descritas por
\begin{eqnarray*}
\zeta_{n}=\left(\xi_{n},\left\{X\left(T_{n-1}+t\right):0\leq t<\xi_{n}\right\}\right)
\end{eqnarray*}
Este $\zeta_{n}$ es el $n$-\'esimo segmento del proceso. El proceso es regenerativo sobre los tiempos $T_{n}$ si sus segmentos $\zeta_{n}$ son independientes e id\'enticamennte distribuidos.
\end{Def}


\begin{Note}
Si $\tilde{X}\left(t\right)$ con espacio de estados $\tilde{S}$ es regenerativo sobre $T_{n}$, entonces $X\left(t\right)=f\left(\tilde{X}\left(t\right)\right)$ tambi\'en es regenerativo sobre $T_{n}$, para cualquier funci\'on $f:\tilde{S}\rightarrow S$.
\end{Note}

\begin{Note}
Los procesos regenerativos son crudamente regenerativos, pero no al rev\'es.
\end{Note}


\begin{Note}
Un proceso estoc\'astico a tiempo continuo o discreto es regenerativo si existe un proceso de renovaci\'on  tal que los segmentos del proceso entre tiempos de renovaci\'on sucesivos son i.i.d., es decir, para $\left\{X\left(t\right):t\geq0\right\}$ proceso estoc\'astico a tiempo continuo con espacio de estados $S$, espacio m\'etrico.
\end{Note}

Para $\left\{X\left(t\right):t\geq0\right\}$ Proceso Estoc\'astico a tiempo continuo con estado de espacios $S$, que es un espacio m\'etrico, con trayectorias continuas por la derecha y con l\'imites por la izquierda c.s. Sea $N\left(t\right)$ un proceso de renovaci\'on en $\rea_{+}$ definido en el mismo espacio de probabilidad que $X\left(t\right)$, con tiempos de renovaci\'on $T$ y tiempos de inter-renovaci\'on $\xi_{n}=T_{n}-T_{n-1}$, con misma distribuci\'on $F$ de media finita $\mu$.



\begin{Def}
Para el proceso $\left\{\left(N\left(t\right),X\left(t\right)\right):t\geq0\right\}$, sus trayectoria muestrales en el intervalo de tiempo $\left[T_{n-1},T_{n}\right)$ est\'an descritas por
\begin{eqnarray*}
\zeta_{n}=\left(\xi_{n},\left\{X\left(T_{n-1}+t\right):0\leq t<\xi_{n}\right\}\right)
\end{eqnarray*}
Este $\zeta_{n}$ es el $n$-\'esimo segmento del proceso. El proceso es regenerativo sobre los tiempos $T_{n}$ si sus segmentos $\zeta_{n}$ son independientes e id\'enticamennte distribuidos.
\end{Def}

\begin{Note}
Un proceso regenerativo con media de la longitud de ciclo finita es llamado positivo recurrente.
\end{Note}

\begin{Teo}[Procesos Regenerativos]
Suponga que el proceso
\end{Teo}


\begin{Def}[Renewal Process Trinity]
Para un proceso de renovaci\'on $N\left(t\right)$, los siguientes procesos proveen de informaci\'on sobre los tiempos de renovaci\'on.
\begin{itemize}
\item $A\left(t\right)=t-T_{N\left(t\right)}$, el tiempo de recurrencia hacia atr\'as al tiempo $t$, que es el tiempo desde la \'ultima renovaci\'on para $t$.

\item $B\left(t\right)=T_{N\left(t\right)+1}-t$, el tiempo de recurrencia hacia adelante al tiempo $t$, residual del tiempo de renovaci\'on, que es el tiempo para la pr\'oxima renovaci\'on despu\'es de $t$.

\item $L\left(t\right)=\xi_{N\left(t\right)+1}=A\left(t\right)+B\left(t\right)$, la longitud del intervalo de renovaci\'on que contiene a $t$.
\end{itemize}
\end{Def}

\begin{Note}
El proceso tridimensional $\left(A\left(t\right),B\left(t\right),L\left(t\right)\right)$ es regenerativo sobre $T_{n}$, y por ende cada proceso lo es. Cada proceso $A\left(t\right)$ y $B\left(t\right)$ son procesos de MArkov a tiempo continuo con trayectorias continuas por partes en el espacio de estados $\rea_{+}$. Una expresi\'on conveniente para su distribuci\'on conjunta es, para $0\leq x<t,y\geq0$
\begin{equation}\label{NoRenovacion}
P\left\{A\left(t\right)>x,B\left(t\right)>y\right\}=
P\left\{N\left(t+y\right)-N\left((t-x)\right)=0\right\}
\end{equation}
\end{Note}

\begin{Ejem}[Tiempos de recurrencia Poisson]
Si $N\left(t\right)$ es un proceso Poisson con tasa $\lambda$, entonces de la expresi\'on (\ref{NoRenovacion}) se tiene que

\begin{eqnarray*}
\begin{array}{lc}
P\left\{A\left(t\right)>x,B\left(t\right)>y\right\}=e^{-\lambda\left(x+y\right)},&0\leq x<t,y\geq0,
\end{array}
\end{eqnarray*}
que es la probabilidad Poisson de no renovaciones en un intervalo de longitud $x+y$.

\end{Ejem}

\begin{Note}
Una cadena de Markov erg\'odica tiene la propiedad de ser estacionaria si la distribuci\'on de su estado al tiempo $0$ es su distribuci\'on estacionaria.
\end{Note}


\begin{Def}
Un proceso estoc\'astico a tiempo continuo $\left\{X\left(t\right):t\geq0\right\}$ en un espacio general es estacionario si sus distribuciones finito dimensionales son invariantes bajo cualquier  traslado: para cada $0\leq s_{1}<s_{2}<\cdots<s_{k}$ y $t\geq0$,
\begin{eqnarray*}
\left(X\left(s_{1}+t\right),\ldots,X\left(s_{k}+t\right)\right)=_{d}\left(X\left(s_{1}\right),\ldots,X\left(s_{k}\right)\right).
\end{eqnarray*}
\end{Def}

\begin{Note}
Un proceso de Markov es estacionario si $X\left(t\right)=_{d}X\left(0\right)$, $t\geq0$.
\end{Note}

Considerese el proceso $N\left(t\right)=\sum_{n}\indora\left(\tau_{n}\leq t\right)$ en $\rea_{+}$, con puntos $0<\tau_{1}<\tau_{2}<\cdots$.

\begin{Prop}
Si $N$ es un proceso puntual estacionario y $\esp\left[N\left(1\right)\right]<\infty$, entonces $\esp\left[N\left(t\right)\right]=t\esp\left[N\left(1\right)\right]$, $t\geq0$

\end{Prop}

\begin{Teo}
Los siguientes enunciados son equivalentes
\begin{itemize}
\item[i)] El proceso retardado de renovaci\'on $N$ es estacionario.

\item[ii)] EL proceso de tiempos de recurrencia hacia adelante $B\left(t\right)$ es estacionario.


\item[iii)] $\esp\left[N\left(t\right)\right]=t/\mu$,


\item[iv)] $G\left(t\right)=F_{e}\left(t\right)=\frac{1}{\mu}\int_{0}^{t}\left[1-F\left(s\right)\right]ds$
\end{itemize}
Cuando estos enunciados son ciertos, $P\left\{B\left(t\right)\leq x\right\}=F_{e}\left(x\right)$, para $t,x\geq0$.

\end{Teo}

\begin{Note}
Una consecuencia del teorema anterior es que el Proceso Poisson es el \'unico proceso sin retardo que es estacionario.
\end{Note}

\begin{Coro}
El proceso de renovaci\'on $N\left(t\right)$ sin retardo, y cuyos tiempos de inter renonaci\'on tienen media finita, es estacionario si y s\'olo si es un proceso Poisson.

\end{Coro}


%________________________________________________________________________
\subsection{Procesos Regenerativos}
%________________________________________________________________________



\begin{Note}
Si $\tilde{X}\left(t\right)$ con espacio de estados $\tilde{S}$ es regenerativo sobre $T_{n}$, entonces $X\left(t\right)=f\left(\tilde{X}\left(t\right)\right)$ tambi\'en es regenerativo sobre $T_{n}$, para cualquier funci\'on $f:\tilde{S}\rightarrow S$.
\end{Note}

\begin{Note}
Los procesos regenerativos son crudamente regenerativos, pero no al rev\'es.
\end{Note}
%\subsection*{Procesos Regenerativos: Sigman\cite{Sigman1}}
\begin{Def}[Definici\'on Cl\'asica]
Un proceso estoc\'astico $X=\left\{X\left(t\right):t\geq0\right\}$ es llamado regenerativo is existe una variable aleatoria $R_{1}>0$ tal que
\begin{itemize}
\item[i)] $\left\{X\left(t+R_{1}\right):t\geq0\right\}$ es independiente de $\left\{\left\{X\left(t\right):t<R_{1}\right\},\right\}$
\item[ii)] $\left\{X\left(t+R_{1}\right):t\geq0\right\}$ es estoc\'asticamente equivalente a $\left\{X\left(t\right):t>0\right\}$
\end{itemize}

Llamamos a $R_{1}$ tiempo de regeneraci\'on, y decimos que $X$ se regenera en este punto.
\end{Def}

$\left\{X\left(t+R_{1}\right)\right\}$ es regenerativo con tiempo de regeneraci\'on $R_{2}$, independiente de $R_{1}$ pero con la misma distribuci\'on que $R_{1}$. Procediendo de esta manera se obtiene una secuencia de variables aleatorias independientes e id\'enticamente distribuidas $\left\{R_{n}\right\}$ llamados longitudes de ciclo. Si definimos a $Z_{k}\equiv R_{1}+R_{2}+\cdots+R_{k}$, se tiene un proceso de renovaci\'on llamado proceso de renovaci\'on encajado para $X$.




\begin{Def}
Para $x$ fijo y para cada $t\geq0$, sea $I_{x}\left(t\right)=1$ si $X\left(t\right)\leq x$,  $I_{x}\left(t\right)=0$ en caso contrario, y def\'inanse los tiempos promedio
\begin{eqnarray*}
\overline{X}&=&lim_{t\rightarrow\infty}\frac{1}{t}\int_{0}^{\infty}X\left(u\right)du\\
\prob\left(X_{\infty}\leq x\right)&=&lim_{t\rightarrow\infty}\frac{1}{t}\int_{0}^{\infty}I_{x}\left(u\right)du,
\end{eqnarray*}
cuando estos l\'imites existan.
\end{Def}

Como consecuencia del teorema de Renovaci\'on-Recompensa, se tiene que el primer l\'imite  existe y es igual a la constante
\begin{eqnarray*}
\overline{X}&=&\frac{\esp\left[\int_{0}^{R_{1}}X\left(t\right)dt\right]}{\esp\left[R_{1}\right]},
\end{eqnarray*}
suponiendo que ambas esperanzas son finitas.

\begin{Note}
\begin{itemize}
\item[a)] Si el proceso regenerativo $X$ es positivo recurrente y tiene trayectorias muestrales no negativas, entonces la ecuaci\'on anterior es v\'alida.
\item[b)] Si $X$ es positivo recurrente regenerativo, podemos construir una \'unica versi\'on estacionaria de este proceso, $X_{e}=\left\{X_{e}\left(t\right)\right\}$, donde $X_{e}$ es un proceso estoc\'astico regenerativo y estrictamente estacionario, con distribuci\'on marginal distribuida como $X_{\infty}$
\end{itemize}
\end{Note}

%________________________________________________________________________
\subsection{Procesos Regenerativos}
%________________________________________________________________________

Para $\left\{X\left(t\right):t\geq0\right\}$ Proceso Estoc\'astico a tiempo continuo con estado de espacios $S$, que es un espacio m\'etrico, con trayectorias continuas por la derecha y con l\'imites por la izquierda c.s. Sea $N\left(t\right)$ un proceso de renovaci\'on en $\rea_{+}$ definido en el mismo espacio de probabilidad que $X\left(t\right)$, con tiempos de renovaci\'on $T$ y tiempos de inter-renovaci\'on $\xi_{n}=T_{n}-T_{n-1}$, con misma distribuci\'on $F$ de media finita $\mu$.



\begin{Def}
Para el proceso $\left\{\left(N\left(t\right),X\left(t\right)\right):t\geq0\right\}$, sus trayectoria muestrales en el intervalo de tiempo $\left[T_{n-1},T_{n}\right)$ est\'an descritas por
\begin{eqnarray*}
\zeta_{n}=\left(\xi_{n},\left\{X\left(T_{n-1}+t\right):0\leq t<\xi_{n}\right\}\right)
\end{eqnarray*}
Este $\zeta_{n}$ es el $n$-\'esimo segmento del proceso. El proceso es regenerativo sobre los tiempos $T_{n}$ si sus segmentos $\zeta_{n}$ son independientes e id\'enticamennte distribuidos.
\end{Def}


\begin{Note}
Si $\tilde{X}\left(t\right)$ con espacio de estados $\tilde{S}$ es regenerativo sobre $T_{n}$, entonces $X\left(t\right)=f\left(\tilde{X}\left(t\right)\right)$ tambi\'en es regenerativo sobre $T_{n}$, para cualquier funci\'on $f:\tilde{S}\rightarrow S$.
\end{Note}

\begin{Note}
Los procesos regenerativos son crudamente regenerativos, pero no al rev\'es.
\end{Note}

\begin{Def}[Definici\'on Cl\'asica]
Un proceso estoc\'astico $X=\left\{X\left(t\right):t\geq0\right\}$ es llamado regenerativo is existe una variable aleatoria $R_{1}>0$ tal que
\begin{itemize}
\item[i)] $\left\{X\left(t+R_{1}\right):t\geq0\right\}$ es independiente de $\left\{\left\{X\left(t\right):t<R_{1}\right\},\right\}$
\item[ii)] $\left\{X\left(t+R_{1}\right):t\geq0\right\}$ es estoc\'asticamente equivalente a $\left\{X\left(t\right):t>0\right\}$
\end{itemize}

Llamamos a $R_{1}$ tiempo de regeneraci\'on, y decimos que $X$ se regenera en este punto.
\end{Def}

$\left\{X\left(t+R_{1}\right)\right\}$ es regenerativo con tiempo de regeneraci\'on $R_{2}$, independiente de $R_{1}$ pero con la misma distribuci\'on que $R_{1}$. Procediendo de esta manera se obtiene una secuencia de variables aleatorias independientes e id\'enticamente distribuidas $\left\{R_{n}\right\}$ llamados longitudes de ciclo. Si definimos a $Z_{k}\equiv R_{1}+R_{2}+\cdots+R_{k}$, se tiene un proceso de renovaci\'on llamado proceso de renovaci\'on encajado para $X$.

\begin{Note}
Un proceso regenerativo con media de la longitud de ciclo finita es llamado positivo recurrente.
\end{Note}


\begin{Def}
Para $x$ fijo y para cada $t\geq0$, sea $I_{x}\left(t\right)=1$ si $X\left(t\right)\leq x$,  $I_{x}\left(t\right)=0$ en caso contrario, y def\'inanse los tiempos promedio
\begin{eqnarray*}
\overline{X}&=&lim_{t\rightarrow\infty}\frac{1}{t}\int_{0}^{\infty}X\left(u\right)du\\
\prob\left(X_{\infty}\leq x\right)&=&lim_{t\rightarrow\infty}\frac{1}{t}\int_{0}^{\infty}I_{x}\left(u\right)du,
\end{eqnarray*}
cuando estos l\'imites existan.
\end{Def}

Como consecuencia del teorema de Renovaci\'on-Recompensa, se tiene que el primer l\'imite  existe y es igual a la constante
\begin{eqnarray*}
\overline{X}&=&\frac{\esp\left[\int_{0}^{R_{1}}X\left(t\right)dt\right]}{\esp\left[R_{1}\right]},
\end{eqnarray*}
suponiendo que ambas esperanzas son finitas.

\begin{Note}
\begin{itemize}
\item[a)] Si el proceso regenerativo $X$ es positivo recurrente y tiene trayectorias muestrales no negativas, entonces la ecuaci\'on anterior es v\'alida.
\item[b)] Si $X$ es positivo recurrente regenerativo, podemos construir una \'unica versi\'on estacionaria de este proceso, $X_{e}=\left\{X_{e}\left(t\right)\right\}$, donde $X_{e}$ es un proceso estoc\'astico regenerativo y estrictamente estacionario, con distribuci\'on marginal distribuida como $X_{\infty}$
\end{itemize}
\end{Note}

%__________________________________________________________________________________________
\subsection{Procesos Regenerativos Estacionarios - Stidham \cite{Stidham}}
%__________________________________________________________________________________________


Un proceso estoc\'astico a tiempo continuo $\left\{V\left(t\right),t\geq0\right\}$ es un proceso regenerativo si existe una sucesi\'on de variables aleatorias independientes e id\'enticamente distribuidas $\left\{X_{1},X_{2},\ldots\right\}$, sucesi\'on de renovaci\'on, tal que para cualquier conjunto de Borel $A$, 

\begin{eqnarray*}
\prob\left\{V\left(t\right)\in A|X_{1}+X_{2}+\cdots+X_{R\left(t\right)}=s,\left\{V\left(\tau\right),\tau<s\right\}\right\}=\prob\left\{V\left(t-s\right)\in A|X_{1}>t-s\right\},
\end{eqnarray*}
para todo $0\leq s\leq t$, donde $R\left(t\right)=\max\left\{X_{1}+X_{2}+\cdots+X_{j}\leq t\right\}=$n\'umero de renovaciones ({\emph{puntos de regeneraci\'on}}) que ocurren en $\left[0,t\right]$. El intervalo $\left[0,X_{1}\right)$ es llamado {\emph{primer ciclo de regeneraci\'on}} de $\left\{V\left(t \right),t\geq0\right\}$, $\left[X_{1},X_{1}+X_{2}\right)$ el {\emph{segundo ciclo de regeneraci\'on}}, y as\'i sucesivamente.

Sea $X=X_{1}$ y sea $F$ la funci\'on de distrbuci\'on de $X$


\begin{Def}
Se define el proceso estacionario, $\left\{V^{*}\left(t\right),t\geq0\right\}$, para $\left\{V\left(t\right),t\geq0\right\}$ por

\begin{eqnarray*}
\prob\left\{V\left(t\right)\in A\right\}=\frac{1}{\esp\left[X\right]}\int_{0}^{\infty}\prob\left\{V\left(t+x\right)\in A|X>x\right\}\left(1-F\left(x\right)\right)dx,
\end{eqnarray*} 
para todo $t\geq0$ y todo conjunto de Borel $A$.
\end{Def}

\begin{Def}
Una distribuci\'on se dice que es {\emph{aritm\'etica}} si todos sus puntos de incremento son m\'ultiplos de la forma $0,\lambda, 2\lambda,\ldots$ para alguna $\lambda>0$ entera.
\end{Def}


\begin{Def}
Una modificaci\'on medible de un proceso $\left\{V\left(t\right),t\geq0\right\}$, es una versi\'on de este, $\left\{V\left(t,w\right)\right\}$ conjuntamente medible para $t\geq0$ y para $w\in S$, $S$ espacio de estados para $\left\{V\left(t\right),t\geq0\right\}$.
\end{Def}

\begin{Teo}
Sea $\left\{V\left(t\right),t\geq\right\}$ un proceso regenerativo no negativo con modificaci\'on medible. Sea $\esp\left[X\right]<\infty$. Entonces el proceso estacionario dado por la ecuaci\'on anterior est\'a bien definido y tiene funci\'on de distribuci\'on independiente de $t$, adem\'as
\begin{itemize}
\item[i)] \begin{eqnarray*}
\esp\left[V^{*}\left(0\right)\right]&=&\frac{\esp\left[\int_{0}^{X}V\left(s\right)ds\right]}{\esp\left[X\right]}\end{eqnarray*}
\item[ii)] Si $\esp\left[V^{*}\left(0\right)\right]<\infty$, equivalentemente, si $\esp\left[\int_{0}^{X}V\left(s\right)ds\right]<\infty$,entonces
\begin{eqnarray*}
\frac{\int_{0}^{t}V\left(s\right)ds}{t}\rightarrow\frac{\esp\left[\int_{0}^{X}V\left(s\right)ds\right]}{\esp\left[X\right]}
\end{eqnarray*}
con probabilidad 1 y en media, cuando $t\rightarrow\infty$.
\end{itemize}
\end{Teo}


%___________________________________________________________
%
\subsection{Existencia de Tiempos de Regeneraci\'on}
%___________________________________________________________
%

%________________________________________________________________________
\subsection{Procesos Regenerativos: Thorisson}
%________________________________________________________________________

Para $\left\{X\left(t\right):t\geq0\right\}$ Proceso Estoc\'astico a tiempo continuo con estado de espacios $S$, que es un espacio m\'etrico, con trayectorias continuas por la derecha y con l\'imites por la izquierda c.s. Sea $N\left(t\right)$ un proceso de renovaci\'on en $\rea_{+}$ definido en el mismo espacio de probabilidad que $X\left(t\right)$, con tiempos de renovaci\'on $T$ y tiempos de inter-renovaci\'on $\xi_{n}=T_{n}-T_{n-1}$, con misma distribuci\'on $F$ de media finita $\mu$.

\begin{Def}
Un elemento aleatorio en un espacio medible $\left(E,\mathcal{E}\right)$ en un espacio de probabilidad $\left(\Omega,\mathcal{F},\prob\right)$ a $\left(E,\mathcal{E}\right)$, es decir,
para $A\in \mathcal{E}$,  se tiene que $\left\{Y\in A\right\}\in\mathcal{F}$, donde $\left\{Y\in A\right\}:=\left\{w\in\Omega:Y\left(w\right)\in A\right\}=:Y^{-1}A$.
\end{Def}

\begin{Note}
Tambi\'en se dice que $Y$ est\'a soportado por el espacio de probabilidad $\left(\Omega,\mathcal{F},\prob\right)$ y que $Y$ es un mapeo medible de $\Omega$ en $E$, es decir, es $\mathcal{F}/\mathcal{E}$ medible.
\end{Note}

\begin{Def}
Para cada $i\in \mathbb{I}$ sea $P_{i}$ una medida de probabilidad en un espacio medible $\left(E_{i},\mathcal{E}_{i}\right)$. Se define el espacio producto
$\otimes_{i\in\mathbb{I}}\left(E_{i},\mathcal{E}_{i}\right):=\left(\prod_{i\in\mathbb{I}}E_{i},\otimes_{i\in\mathbb{I}}\mathcal{E}_{i}\right)$, donde $\prod_{i\in\mathbb{I}}E_{i}$ es el producto cartesiano de los $E_{i}$'s, y $\otimes_{i\in\mathbb{I}}\mathcal{E}_{i}$ es la $\sigma$-\'algebra producto, es decir, es la $\sigma$-\'algebra m\'as peque\~na en $\prod_{i\in\mathbb{I}}E_{i}$ que hace al $i$-\'esimo mapeo proyecci\'on en $E_{i}$ medible para toda $i\in\mathbb{I}$ es la $\sigma$-\'algebra inducida por los mapeos proyecci\'on. $$\otimes_{i\in\mathbb{I}}\mathcal{E}_{i}:=\sigma\left\{\left\{y:y_{i}\in A\right\}:i\in\mathbb{I}\textrm{ y }A\in\mathcal{E}_{i}\right\}.$$
\end{Def}

\begin{Def}
Un espacio de probabilidad $\left(\tilde{\Omega},\tilde{\mathcal{F}},\tilde{\prob}\right)$ es una extensi\'on de otro espacio de probabilidad $\left(\Omega,\mathcal{F},\prob\right)$ si $\left(\tilde{\Omega},\tilde{\mathcal{F}},\tilde{\prob}\right)$ soporta un elemento aleatorio $\xi\in\left(\Omega,\mathcal{F}\right)$ que tienen a $\prob$ como distribuci\'on.
\end{Def}

\begin{Teo}
Sea $\mathbb{I}$ un conjunto de \'indices arbitrario. Para cada $i\in\mathbb{I}$ sea $P_{i}$ una medida de probabilidad en un espacio medible $\left(E_{i},\mathcal{E}_{i}\right)$. Entonces existe una \'unica medida de probabilidad $\otimes_{i\in\mathbb{I}}P_{i}$ en $\otimes_{i\in\mathbb{I}}\left(E_{i},\mathcal{E}_{i}\right)$ tal que 

\begin{eqnarray*}
\otimes_{i\in\mathbb{I}}P_{i}\left(y\in\prod_{i\in\mathbb{I}}E_{i}:y_{i}\in A_{i_{1}},\ldots,y_{n}\in A_{i_{n}}\right)=P_{i_{1}}\left(A_{i_{n}}\right)\cdots P_{i_{n}}\left(A_{i_{n}}\right)
\end{eqnarray*}
para todos los enteros $n>0$, toda $i_{1},\ldots,i_{n}\in\mathbb{I}$ y todo $A_{i_{1}}\in\mathcal{E}_{i_{1}},\ldots,A_{i_{n}}\in\mathcal{E}_{i_{n}}$
\end{Teo}

La medida $\otimes_{i\in\mathbb{I}}P_{i}$ es llamada la medida producto y $\otimes_{i\in\mathbb{I}}\left(E_{i},\mathcal{E}_{i},P_{i}\right):=\left(\prod_{i\in\mathbb{I}},E_{i},\otimes_{i\in\mathbb{I}}\mathcal{E}_{i},\otimes_{i\in\mathbb{I}}P_{i}\right)$, es llamado espacio de probabilidad producto.


\begin{Def}
Un espacio medible $\left(E,\mathcal{E}\right)$ es \textit{Polaco} si existe una m\'etrica en $E$ tal que $E$ es completo, es decir cada sucesi\'on de Cauchy converge a un l\'imite en $E$, y \textit{separable}, $E$ tienen un subconjunto denso numerable, y tal que $\mathcal{E}$ es generado por conjuntos abiertos.
\end{Def}


\begin{Def}
Dos espacios medibles $\left(E,\mathcal{E}\right)$ y $\left(G,\mathcal{G}\right)$ son Borel equivalentes \textit{isomorfos} si existe una biyecci\'on $f:E\rightarrow G$ tal que $f$ es $\mathcal{E}/\mathcal{G}$ medible y su inversa $f^{-1}$ es $\mathcal{G}/\mathcal{E}$ medible. La biyecci\'on es una equivalencia de Borel.
\end{Def}

\begin{Def}
Un espacio medible  $\left(E,\mathcal{E}\right)$ es un \textit{espacio est\'andar} si es Borel equivalente a $\left(G,\mathcal{G}\right)$, donde $G$ es un subconjunto de Borel de $\left[0,1\right]$ y $\mathcal{G}$ son los subconjuntos de Borel de $G$.
\end{Def}

\begin{Note}
Cualquier espacio Polaco es un espacio est\'andar.
\end{Note}


\begin{Def}
Un proceso estoc\'astico con conjunto de \'indices $\mathbb{I}$ y espacio de estados $\left(E,\mathcal{E}\right)$ es una familia $Z=\left(\mathbb{Z}_{s}\right)_{s\in\mathbb{I}}$ donde $\mathbb{Z}_{s}$ son elementos aleatorios definidos en un espacio de probabilidad com\'un $\left(\Omega,\mathcal{F},\prob\right)$ y todos toman valores en $\left(E,\mathcal{E}\right)$.
\end{Def}

\begin{Def}
Un proceso estoc\'astico \textit{one-sided contiuous time} (\textbf{PEOSCT}) es un proceso estoc\'astico con conjunto de \'indices $\mathbb{I}=\left[0,\infty\right)$.
\end{Def}


Sea $\left(E^{\mathbb{I}},\mathcal{E}^{\mathbb{I}}\right)$ denota el espacio producto $\left(E^{\mathbb{I}},\mathcal{E}^{\mathbb{I}}\right):=\otimes_{s\in\mathbb{I}}\left(E,\mathcal{E}\right)$. Vamos a considerar $\mathbb{Z}$ como un mapeo aleatorio, es decir, como un elemento aleatorio en $\left(E^{\mathbb{I}},\mathcal{E}^{\mathbb{I}}\right)$ definido por $Z\left(w\right)=\left(Z_{s}\left(w\right)\right)_{s\in\mathbb{I}}$ y $w\in\Omega$.

\begin{Note}
La distribuci\'on de un proceso estoc\'astico $Z$ es la distribuci\'on de $Z$ como un elemento aleatorio en $\left(E^{\mathbb{I}},\mathcal{E}^{\mathbb{I}}\right)$. La distribuci\'on de $Z$ esta determinada de manera \'unica por las distribuciones finito dimensionales.
\end{Note}

\begin{Note}
En particular cuando $Z$ toma valores reales, es decir, $\left(E,\mathcal{E}\right)=\left(\mathbb{R},\mathcal{B}\right)$ las distribuciones finito dimensionales est\'an determinadas por las funciones de distribuci\'on finito dimensionales

\begin{eqnarray}
\prob\left(Z_{t_{1}}\leq x_{1},\ldots,Z_{t_{n}}\leq x_{n}\right),x_{1},\ldots,x_{n}\in\mathbb{R},t_{1},\ldots,t_{n}\in\mathbb{I},n\geq1.
\end{eqnarray}
\end{Note}

\begin{Note}
Para espacios polacos $\left(E,\mathcal{E}\right)$ el Teorema de Consistencia de Kolmogorov asegura que dada una colecci\'on de distribuciones finito dimensionales consistentes, siempre existe un proceso estoc\'astico que posee tales distribuciones finito dimensionales.
\end{Note}


\begin{Def}
Las trayectorias de $Z$ son las realizaciones $Z\left(w\right)$ para $w\in\Omega$ del mapeo aleatorio $Z$.
\end{Def}

\begin{Note}
Algunas restricciones se imponen sobre las trayectorias, por ejemplo que sean continuas por la derecha, o continuas por la derecha con l\'imites por la izquierda, o de manera m\'as general, se pedir\'a que caigan en alg\'un subconjunto $H$ de $E^{\mathbb{I}}$. En este caso es natural considerar a $Z$ como un elemento aleatorio que no est\'a en $\left(E^{\mathbb{I}},\mathcal{E}^{\mathbb{I}}\right)$ sino en $\left(H,\mathcal{H}\right)$, donde $\mathcal{H}$ es la $\sigma$-\'algebra generada por los mapeos proyecci\'on que toman a $z\in H$ a $z_{t}\in E$ para $t\in\mathbb{I}$. A $\mathcal{H}$ se le conoce como la traza de $H$ en $E^{\mathbb{I}}$, es decir,
\begin{eqnarray}
\mathcal{H}:=E^{\mathbb{I}}\cap H:=\left\{A\cap H:A\in E^{\mathbb{I}}\right\}.
\end{eqnarray}
\end{Note}


\begin{Note}
$Z$ tiene trayectorias con valores en $H$ y cada $Z_{t}$ es un mapeo medible de $\left(\Omega,\mathcal{F}\right)$ a $\left(H,\mathcal{H}\right)$. Cuando se considera un espacio de trayectorias en particular $H$, al espacio $\left(H,\mathcal{H}\right)$ se le llama el espacio de trayectorias de $Z$.
\end{Note}

\begin{Note}
La distribuci\'on del proceso estoc\'astico $Z$ con espacio de trayectorias $\left(H,\mathcal{H}\right)$ es la distribuci\'on de $Z$ como  un elemento aleatorio en $\left(H,\mathcal{H}\right)$. La distribuci\'on, nuevemente, est\'a determinada de manera \'unica por las distribuciones finito dimensionales.
\end{Note}


\begin{Def}
Sea $Z$ un PEOSCT  con espacio de estados $\left(E,\mathcal{E}\right)$ y sea $T$ un tiempo aleatorio en $\left[0,\infty\right)$. Por $Z_{T}$ se entiende el mapeo con valores en $E$ definido en $\Omega$ en la manera obvia:
\begin{eqnarray*}
Z_{T}\left(w\right):=Z_{T\left(w\right)}\left(w\right). w\in\Omega.
\end{eqnarray*}
\end{Def}

\begin{Def}
Un PEOSCT $Z$ es conjuntamente medible (\textbf{CM}) si el mapeo que toma $\left(w,t\right)\in\Omega\times\left[0,\infty\right)$ a $Z_{t}\left(w\right)\in E$ es $\mathcal{F}\otimes\mathcal{B}\left[0,\infty\right)/\mathcal{E}$ medible.
\end{Def}

\begin{Note}
Un PEOSCT-CM implica que el proceso es medible, dado que $Z_{T}$ es una composici\'on  de dos mapeos continuos: el primero que toma $w$ en $\left(w,T\left(w\right)\right)$ es $\mathcal{F}/\mathcal{F}\otimes\mathcal{B}\left[0,\infty\right)$ medible, mientras que el segundo toma $\left(w,T\left(w\right)\right)$ en $Z_{T\left(w\right)}\left(w\right)$ es $\mathcal{F}\otimes\mathcal{B}\left[0,\infty\right)/\mathcal{E}$ medible.
\end{Note}


\begin{Def}
Un PEOSCT con espacio de estados $\left(H,\mathcal{H}\right)$ es can\'onicamente conjuntamente medible (\textbf{CCM}) si el mapeo $\left(z,t\right)\in H\times\left[0,\infty\right)$ en $Z_{t}\in E$ es $\mathcal{H}\otimes\mathcal{B}\left[0,\infty\right)/\mathcal{E}$ medible.
\end{Def}

\begin{Note}
Un PEOSCTCCM implica que el proceso es CM, dado que un PECCM $Z$ es un mapeo de $\Omega\times\left[0,\infty\right)$ a $E$, es la composici\'on de dos mapeos medibles: el primero, toma $\left(w,t\right)$ en $\left(Z\left(w\right),t\right)$ es $\mathcal{F}\otimes\mathcal{B}\left[0,\infty\right)/\mathcal{H}\otimes\mathcal{B}\left[0,\infty\right)$ medible, y el segundo que toma $\left(Z\left(w\right),t\right)$  en $Z_{t}\left(w\right)$ es $\mathcal{H}\otimes\mathcal{B}\left[0,\infty\right)/\mathcal{E}$ medible. Por tanto CCM es una condici\'on m\'as fuerte que CM.
\end{Note}

\begin{Def}
Un conjunto de trayectorias $H$ de un PEOSCT $Z$, es internamente shift-invariante (\textbf{ISI}) si 
\begin{eqnarray*}
\left\{\left(z_{t+s}\right)_{s\in\left[0,\infty\right)}:z\in H\right\}=H\textrm{, }t\in\left[0,\infty\right).
\end{eqnarray*}
\end{Def}


\begin{Def}
Dado un PEOSCTISI, se define el mapeo-shift $\theta_{t}$, $t\in\left[0,\infty\right)$, de $H$ a $H$ por 
\begin{eqnarray*}
\theta_{t}z=\left(z_{t+s}\right)_{s\in\left[0,\infty\right)}\textrm{, }z\in H.
\end{eqnarray*}
\end{Def}

\begin{Def}
Se dice que un proceso $Z$ es shift-medible (\textbf{SM}) si $Z$ tiene un conjunto de trayectorias $H$ que es ISI y adem\'as el mapeo que toma $\left(z,t\right)\in H\times\left[0,\infty\right)$ en $\theta_{t}z\in H$ es $\mathcal{H}\otimes\mathcal{B}\left[0,\infty\right)/\mathcal{H}$ medible.
\end{Def}

\begin{Note}
Un proceso estoc\'astico con conjunto de trayectorias $H$ ISI es shift-medible si y s\'olo si es CCM
\end{Note}

\begin{Note}
\begin{itemize}
\item Dado el espacio polaco $\left(E,\mathcal{E}\right)$ se tiene el  conjunto de trayectorias $D_{E}\left[0,\infty\right)$ que es ISI, entonces cumpe con ser CCM.

\item Si $G$ es abierto, podemos cubrirlo por bolas abiertas cuay cerradura este contenida en $G$, y como $G$ es segundo numerable como subespacio de $E$, lo podemos cubrir por una cantidad numerable de bolas abiertas.

\end{itemize}
\end{Note}


\begin{Note}
Los procesos estoc\'asticos $Z$ a tiempo discreto con espacio de estados polaco, tambi\'en tiene un espacio de trayectorias polaco y por tanto tiene distribuciones condicionales regulares.
\end{Note}

\begin{Teo}
El producto numerable de espacios polacos es polaco.
\end{Teo}


\begin{Def}
Sea $\left(\Omega,\mathcal{F},\prob\right)$ espacio de probabilidad que soporta al proceso $Z=\left(Z_{s}\right)_{s\in\left[0,\infty\right)}$ y $S=\left(S_{k}\right)_{0}^{\infty}$ donde $Z$ es un PEOSCTM con espacio de estados $\left(E,\mathcal{E}\right)$  y espacio de trayectorias $\left(H,\mathcal{H}\right)$  y adem\'as $S$ es una sucesi\'on de tiempos aleatorios one-sided que satisfacen la condici\'on $0\leq S_{0}<S_{1}<\cdots\rightarrow\infty$. Considerando $S$ como un mapeo medible de $\left(\Omega,\mathcal{F}\right)$ al espacio sucesi\'on $\left(L,\mathcal{L}\right)$, donde 
\begin{eqnarray*}
L=\left\{\left(s_{k}\right)_{0}^{\infty}\in\left[0,\infty\right)^{\left\{0,1,\ldots\right\}}:s_{0}<s_{1}<\cdots\rightarrow\infty\right\},
\end{eqnarray*}
donde $\mathcal{L}$ son los subconjuntos de Borel de $L$, es decir, $\mathcal{L}=L\cap\mathcal{B}^{\left\{0,1,\ldots\right\}}$.

As\'i el par $\left(Z,S\right)$ es un mapeo medible de  $\left(\Omega,\mathcal{F}\right)$ en $\left(H\times L,\mathcal{H}\otimes\mathcal{L}\right)$. El par $\mathcal{H}\otimes\mathcal{L}^{+}$ denotar\'a la clase de todas las funciones medibles de $\left(H\times L,\mathcal{H}\otimes\mathcal{L}\right)$ en $\left(\left[0,\infty\right),\mathcal{B}\left[0,\infty\right)\right)$.
\end{Def}


\begin{Def}
Sea $\theta_{t}$ el mapeo-shift conjunto de $H\times L$ en $H\times L$ dado por
\begin{eqnarray*}
\theta_{t}\left(z,\left(s_{k}\right)_{0}^{\infty}\right)=\theta_{t}\left(z,\left(s_{n_{t-}+k}-t\right)_{0}^{\infty}\right)
\end{eqnarray*}
donde 
$n_{t-}=inf\left\{n\geq1:s_{n}\geq t\right\}$.
\end{Def}

\begin{Note}
Con la finalidad de poder realizar los shift's sin complicaciones de medibilidad, se supondr\'a que $Z$ es shit-medible, es decir, el conjunto de trayectorias $H$ es invariante bajo shifts del tiempo y el mapeo que toma $\left(z,t\right)\in H\times\left[0,\infty\right)$ en $z_{t}\in E$ es $\mathcal{H}\otimes\mathcal{B}\left[0,\infty\right)/\mathcal{E}$ medible.
\end{Note}

\begin{Def}
Dado un proceso \textbf{PEOSSM} (Proceso Estoc\'astico One Side Shift Medible) $Z$, se dice regenerativo cl\'asico con tiempos de regeneraci\'on $S$ si 

\begin{eqnarray*}
\theta_{S_{n}}\left(Z,S\right)=\left(Z^{0},S^{0}\right),n\geq0
\end{eqnarray*}
y adem\'as $\theta_{S_{n}}\left(Z,S\right)$ es independiente de $\left(\left(Z_{s}\right)s\in\left[0,S_{n}\right),S_{0},\ldots,S_{n}\right)$
Si lo anterior se cumple, al par $\left(Z,S\right)$ se le llama regenerativo cl\'asico.
\end{Def}

\begin{Note}
Si el par $\left(Z,S\right)$ es regenerativo cl\'asico, entonces las longitudes de los ciclos $X_{1},X_{2},\ldots,$ son i.i.d. e independientes de la longitud del retraso $S_{0}$, es decir, $S$ es un proceso de renovaci\'on. Las longitudes de los ciclos tambi\'en son llamados tiempos de inter-regeneraci\'on y tiempos de ocurrencia.

\end{Note}

\begin{Teo}
Sup\'ongase que el par $\left(Z,S\right)$ es regenerativo cl\'asico con $\esp\left[X_{1}\right]<\infty$. Entonces $\left(Z^{*},S^{*}\right)$ en el teorema 2.1 es una versi\'on estacionaria de $\left(Z,S\right)$. Adem\'as, si $X_{1}$ es lattice con span $d$, entonces $\left(Z^{**},S^{**}\right)$ en el teorema 2.2 es una versi\'on periodicamente estacionaria de $\left(Z,S\right)$ con periodo $d$.

\end{Teo}

\begin{Def}
Una variable aleatoria $X_{1}$ es \textit{spread out} si existe una $n\geq1$ y una  funci\'on $f\in\mathcal{B}^{+}$ tal que $\int_{\rea}f\left(x\right)dx>0$ con $X_{2},X_{3},\ldots,X_{n}$ copias i.i.d  de $X_{1}$, $$\prob\left(X_{1}+\cdots+X_{n}\in B\right)\geq\int_{B}f\left(x\right)dx$$ para $B\in\mathcal{B}$.

\end{Def}



\begin{Def}
Dado un proceso estoc\'astico $Z$ se le llama \textit{wide-sense regenerative} (\textbf{WSR}) con tiempos de regeneraci\'on $S$ si $\theta_{S_{n}}\left(Z,S\right)=\left(Z^{0},S^{0}\right)$ para $n\geq0$ en distribuci\'on y $\theta_{S_{n}}\left(Z,S\right)$ es independiente de $\left(S_{0},S_{1},\ldots,S_{n}\right)$ para $n\geq0$.
Se dice que el par $\left(Z,S\right)$ es WSR si lo anterior se cumple.
\end{Def}


\begin{Note}
\begin{itemize}
\item El proceso de trayectorias $\left(\theta_{s}Z\right)_{s\in\left[0,\infty\right)}$ es WSR con tiempos de regeneraci\'on $S$ pero no es regenerativo cl\'asico.

\item Si $Z$ es cualquier proceso estacionario y $S$ es un proceso de renovaci\'on que es independiente de $Z$, entonces $\left(Z,S\right)$ es WSR pero en general no es regenerativo cl\'asico

\end{itemize}

\end{Note}


\begin{Note}
Para cualquier proceso estoc\'astico $Z$, el proceso de trayectorias $\left(\theta_{s}Z\right)_{s\in\left[0,\infty\right)}$ es siempre un proceso de Markov.
\end{Note}



\begin{Teo}
Supongase que el par $\left(Z,S\right)$ es WSR con $\esp\left[X_{1}\right]<\infty$. Entonces $\left(Z^{*},S^{*}\right)$ en el teorema 2.1 es una versi\'on estacionaria de 
$\left(Z,S\right)$.
\end{Teo}


\begin{Teo}
Supongase que $\left(Z,S\right)$ es cycle-stationary con $\esp\left[X_{1}\right]<\infty$. Sea $U$ distribuida uniformemente en $\left[0,1\right)$ e independiente de $\left(Z^{0},S^{0}\right)$ y sea $\prob^{*}$ la medida de probabilidad en $\left(\Omega,\prob\right)$ definida por $$d\prob^{*}=\frac{X_{1}}{\esp\left[X_{1}\right]}d\prob$$. Sea $\left(Z^{*},S^{*}\right)$ con distribuci\'on $\prob^{*}\left(\theta_{UX_{1}}\left(Z^{0},S^{0}\right)\in\cdot\right)$. Entonces $\left(Z^{}*,S^{*}\right)$ es estacionario,
\begin{eqnarray*}
\esp\left[f\left(Z^{*},S^{*}\right)\right]=\esp\left[\int_{0}^{X_{1}}f\left(\theta_{s}\left(Z^{0},S^{0}\right)\right)ds\right]/\esp\left[X_{1}\right]
\end{eqnarray*}
$f\in\mathcal{H}\otimes\mathcal{L}^{+}$, and $S_{0}^{*}$ es continuo con funci\'on distribuci\'on $G_{\infty}$ definida por $$G_{\infty}\left(x\right):=\frac{\esp\left[X_{1}\right]\wedge x}{\esp\left[X_{1}\right]}$$ para $x\geq0$ y densidad $\prob\left[X_{1}>x\right]/\esp\left[X_{1}\right]$, con $x\geq0$.

\end{Teo}


\begin{Teo}
Sea $Z$ un Proceso Estoc\'astico un lado shift-medible \textit{one-sided shift-measurable stochastic process}, (PEOSSM),
y $S_{0}$ y $S_{1}$ tiempos aleatorios tales que $0\leq S_{0}<S_{1}$ y
\begin{equation}
\theta_{S_{1}}Z=\theta_{S_{0}}Z\textrm{ en distribuci\'on}.
\end{equation}

Entonces el espacio de probabilidad subyacente $\left(\Omega,\mathcal{F},\prob\right)$ puede extenderse para soportar una sucesi\'on de tiempos aleatorios $S$ tales que

\begin{eqnarray}
\theta_{S_{n}}\left(Z,S\right)=\left(Z^{0},S^{0}\right),n\geq0,\textrm{ en distribuci\'on},\\
\left(Z,S_{0},S_{1}\right)\textrm{ depende de }\left(X_{2},X_{3},\ldots\right)\textrm{ solamente a traves de }\theta_{S_{1}}Z.
\end{eqnarray}
\end{Teo}


\begin{Def}
Un elemento aleatorio en un espacio medible $\left(E,\mathcal{E}\right)$ en un espacio de probabilidad $\left(\Omega,\mathcal{F},\prob\right)$ a $\left(E,\mathcal{E}\right)$, es decir,
para $A\in \mathcal{E}$,  se tiene que $\left\{Y\in A\right\}\in\mathcal{F}$, donde $\left\{Y\in A\right\}:=\left\{w\in\Omega:Y\left(w\right)\in A\right\}=:Y^{-1}A$.
\end{Def}

\begin{Note}
Tambi\'en se dice que $Y$ est\'a soportado por el espacio de probabilidad $\left(\Omega,\mathcal{F},\prob\right)$ y que $Y$ es un mapeo medible de $\Omega$ en $E$, es decir, es $\mathcal{F}/\mathcal{E}$ medible.
\end{Note}

\begin{Def}
Para cada $i\in \mathbb{I}$ sea $P_{i}$ una medida de probabilidad en un espacio medible $\left(E_{i},\mathcal{E}_{i}\right)$. Se define el espacio producto
$\otimes_{i\in\mathbb{I}}\left(E_{i},\mathcal{E}_{i}\right):=\left(\prod_{i\in\mathbb{I}}E_{i},\otimes_{i\in\mathbb{I}}\mathcal{E}_{i}\right)$, donde $\prod_{i\in\mathbb{I}}E_{i}$ es el producto cartesiano de los $E_{i}$'s, y $\otimes_{i\in\mathbb{I}}\mathcal{E}_{i}$ es la $\sigma$-\'algebra producto, es decir, es la $\sigma$-\'algebra m\'as peque\~na en $\prod_{i\in\mathbb{I}}E_{i}$ que hace al $i$-\'esimo mapeo proyecci\'on en $E_{i}$ medible para toda $i\in\mathbb{I}$ es la $\sigma$-\'algebra inducida por los mapeos proyecci\'on. $$\otimes_{i\in\mathbb{I}}\mathcal{E}_{i}:=\sigma\left\{\left\{y:y_{i}\in A\right\}:i\in\mathbb{I}\textrm{ y }A\in\mathcal{E}_{i}\right\}.$$
\end{Def}

\begin{Def}
Un espacio de probabilidad $\left(\tilde{\Omega},\tilde{\mathcal{F}},\tilde{\prob}\right)$ es una extensi\'on de otro espacio de probabilidad $\left(\Omega,\mathcal{F},\prob\right)$ si $\left(\tilde{\Omega},\tilde{\mathcal{F}},\tilde{\prob}\right)$ soporta un elemento aleatorio $\xi\in\left(\Omega,\mathcal{F}\right)$ que tienen a $\prob$ como distribuci\'on.
\end{Def}

\begin{Teo}
Sea $\mathbb{I}$ un conjunto de \'indices arbitrario. Para cada $i\in\mathbb{I}$ sea $P_{i}$ una medida de probabilidad en un espacio medible $\left(E_{i},\mathcal{E}_{i}\right)$. Entonces existe una \'unica medida de probabilidad $\otimes_{i\in\mathbb{I}}P_{i}$ en $\otimes_{i\in\mathbb{I}}\left(E_{i},\mathcal{E}_{i}\right)$ tal que 

\begin{eqnarray*}
\otimes_{i\in\mathbb{I}}P_{i}\left(y\in\prod_{i\in\mathbb{I}}E_{i}:y_{i}\in A_{i_{1}},\ldots,y_{n}\in A_{i_{n}}\right)=P_{i_{1}}\left(A_{i_{n}}\right)\cdots P_{i_{n}}\left(A_{i_{n}}\right)
\end{eqnarray*}
para todos los enteros $n>0$, toda $i_{1},\ldots,i_{n}\in\mathbb{I}$ y todo $A_{i_{1}}\in\mathcal{E}_{i_{1}},\ldots,A_{i_{n}}\in\mathcal{E}_{i_{n}}$
\end{Teo}

La medida $\otimes_{i\in\mathbb{I}}P_{i}$ es llamada la medida producto y $\otimes_{i\in\mathbb{I}}\left(E_{i},\mathcal{E}_{i},P_{i}\right):=\left(\prod_{i\in\mathbb{I}},E_{i},\otimes_{i\in\mathbb{I}}\mathcal{E}_{i},\otimes_{i\in\mathbb{I}}P_{i}\right)$, es llamado espacio de probabilidad producto.


\begin{Def}
Un espacio medible $\left(E,\mathcal{E}\right)$ es \textit{Polaco} si existe una m\'etrica en $E$ tal que $E$ es completo, es decir cada sucesi\'on de Cauchy converge a un l\'imite en $E$, y \textit{separable}, $E$ tienen un subconjunto denso numerable, y tal que $\mathcal{E}$ es generado por conjuntos abiertos.
\end{Def}


\begin{Def}
Dos espacios medibles $\left(E,\mathcal{E}\right)$ y $\left(G,\mathcal{G}\right)$ son Borel equivalentes \textit{isomorfos} si existe una biyecci\'on $f:E\rightarrow G$ tal que $f$ es $\mathcal{E}/\mathcal{G}$ medible y su inversa $f^{-1}$ es $\mathcal{G}/\mathcal{E}$ medible. La biyecci\'on es una equivalencia de Borel.
\end{Def}

\begin{Def}
Un espacio medible  $\left(E,\mathcal{E}\right)$ es un \textit{espacio est\'andar} si es Borel equivalente a $\left(G,\mathcal{G}\right)$, donde $G$ es un subconjunto de Borel de $\left[0,1\right]$ y $\mathcal{G}$ son los subconjuntos de Borel de $G$.
\end{Def}

\begin{Note}
Cualquier espacio Polaco es un espacio est\'andar.
\end{Note}


\begin{Def}
Un proceso estoc\'astico con conjunto de \'indices $\mathbb{I}$ y espacio de estados $\left(E,\mathcal{E}\right)$ es una familia $Z=\left(\mathbb{Z}_{s}\right)_{s\in\mathbb{I}}$ donde $\mathbb{Z}_{s}$ son elementos aleatorios definidos en un espacio de probabilidad com\'un $\left(\Omega,\mathcal{F},\prob\right)$ y todos toman valores en $\left(E,\mathcal{E}\right)$.
\end{Def}

\begin{Def}
Un proceso estoc\'astico \textit{one-sided contiuous time} (\textbf{PEOSCT}) es un proceso estoc\'astico con conjunto de \'indices $\mathbb{I}=\left[0,\infty\right)$.
\end{Def}


Sea $\left(E^{\mathbb{I}},\mathcal{E}^{\mathbb{I}}\right)$ denota el espacio producto $\left(E^{\mathbb{I}},\mathcal{E}^{\mathbb{I}}\right):=\otimes_{s\in\mathbb{I}}\left(E,\mathcal{E}\right)$. Vamos a considerar $\mathbb{Z}$ como un mapeo aleatorio, es decir, como un elemento aleatorio en $\left(E^{\mathbb{I}},\mathcal{E}^{\mathbb{I}}\right)$ definido por $Z\left(w\right)=\left(Z_{s}\left(w\right)\right)_{s\in\mathbb{I}}$ y $w\in\Omega$.

\begin{Note}
La distribuci\'on de un proceso estoc\'astico $Z$ es la distribuci\'on de $Z$ como un elemento aleatorio en $\left(E^{\mathbb{I}},\mathcal{E}^{\mathbb{I}}\right)$. La distribuci\'on de $Z$ esta determinada de manera \'unica por las distribuciones finito dimensionales.
\end{Note}

\begin{Note}
En particular cuando $Z$ toma valores reales, es decir, $\left(E,\mathcal{E}\right)=\left(\mathbb{R},\mathcal{B}\right)$ las distribuciones finito dimensionales est\'an determinadas por las funciones de distribuci\'on finito dimensionales

\begin{eqnarray}
\prob\left(Z_{t_{1}}\leq x_{1},\ldots,Z_{t_{n}}\leq x_{n}\right),x_{1},\ldots,x_{n}\in\mathbb{R},t_{1},\ldots,t_{n}\in\mathbb{I},n\geq1.
\end{eqnarray}
\end{Note}

\begin{Note}
Para espacios polacos $\left(E,\mathcal{E}\right)$ el Teorema de Consistencia de Kolmogorov asegura que dada una colecci\'on de distribuciones finito dimensionales consistentes, siempre existe un proceso estoc\'astico que posee tales distribuciones finito dimensionales.
\end{Note}


\begin{Def}
Las trayectorias de $Z$ son las realizaciones $Z\left(w\right)$ para $w\in\Omega$ del mapeo aleatorio $Z$.
\end{Def}

\begin{Note}
Algunas restricciones se imponen sobre las trayectorias, por ejemplo que sean continuas por la derecha, o continuas por la derecha con l\'imites por la izquierda, o de manera m\'as general, se pedir\'a que caigan en alg\'un subconjunto $H$ de $E^{\mathbb{I}}$. En este caso es natural considerar a $Z$ como un elemento aleatorio que no est\'a en $\left(E^{\mathbb{I}},\mathcal{E}^{\mathbb{I}}\right)$ sino en $\left(H,\mathcal{H}\right)$, donde $\mathcal{H}$ es la $\sigma$-\'algebra generada por los mapeos proyecci\'on que toman a $z\in H$ a $z_{t}\in E$ para $t\in\mathbb{I}$. A $\mathcal{H}$ se le conoce como la traza de $H$ en $E^{\mathbb{I}}$, es decir,
\begin{eqnarray}
\mathcal{H}:=E^{\mathbb{I}}\cap H:=\left\{A\cap H:A\in E^{\mathbb{I}}\right\}.
\end{eqnarray}
\end{Note}


\begin{Note}
$Z$ tiene trayectorias con valores en $H$ y cada $Z_{t}$ es un mapeo medible de $\left(\Omega,\mathcal{F}\right)$ a $\left(H,\mathcal{H}\right)$. Cuando se considera un espacio de trayectorias en particular $H$, al espacio $\left(H,\mathcal{H}\right)$ se le llama el espacio de trayectorias de $Z$.
\end{Note}

\begin{Note}
La distribuci\'on del proceso estoc\'astico $Z$ con espacio de trayectorias $\left(H,\mathcal{H}\right)$ es la distribuci\'on de $Z$ como  un elemento aleatorio en $\left(H,\mathcal{H}\right)$. La distribuci\'on, nuevemente, est\'a determinada de manera \'unica por las distribuciones finito dimensionales.
\end{Note}


\begin{Def}
Sea $Z$ un PEOSCT  con espacio de estados $\left(E,\mathcal{E}\right)$ y sea $T$ un tiempo aleatorio en $\left[0,\infty\right)$. Por $Z_{T}$ se entiende el mapeo con valores en $E$ definido en $\Omega$ en la manera obvia:
\begin{eqnarray*}
Z_{T}\left(w\right):=Z_{T\left(w\right)}\left(w\right). w\in\Omega.
\end{eqnarray*}
\end{Def}

\begin{Def}
Un PEOSCT $Z$ es conjuntamente medible (\textbf{CM}) si el mapeo que toma $\left(w,t\right)\in\Omega\times\left[0,\infty\right)$ a $Z_{t}\left(w\right)\in E$ es $\mathcal{F}\otimes\mathcal{B}\left[0,\infty\right)/\mathcal{E}$ medible.
\end{Def}

\begin{Note}
Un PEOSCT-CM implica que el proceso es medible, dado que $Z_{T}$ es una composici\'on  de dos mapeos continuos: el primero que toma $w$ en $\left(w,T\left(w\right)\right)$ es $\mathcal{F}/\mathcal{F}\otimes\mathcal{B}\left[0,\infty\right)$ medible, mientras que el segundo toma $\left(w,T\left(w\right)\right)$ en $Z_{T\left(w\right)}\left(w\right)$ es $\mathcal{F}\otimes\mathcal{B}\left[0,\infty\right)/\mathcal{E}$ medible.
\end{Note}


\begin{Def}
Un PEOSCT con espacio de estados $\left(H,\mathcal{H}\right)$ es can\'onicamente conjuntamente medible (\textbf{CCM}) si el mapeo $\left(z,t\right)\in H\times\left[0,\infty\right)$ en $Z_{t}\in E$ es $\mathcal{H}\otimes\mathcal{B}\left[0,\infty\right)/\mathcal{E}$ medible.
\end{Def}

\begin{Note}
Un PEOSCTCCM implica que el proceso es CM, dado que un PECCM $Z$ es un mapeo de $\Omega\times\left[0,\infty\right)$ a $E$, es la composici\'on de dos mapeos medibles: el primero, toma $\left(w,t\right)$ en $\left(Z\left(w\right),t\right)$ es $\mathcal{F}\otimes\mathcal{B}\left[0,\infty\right)/\mathcal{H}\otimes\mathcal{B}\left[0,\infty\right)$ medible, y el segundo que toma $\left(Z\left(w\right),t\right)$  en $Z_{t}\left(w\right)$ es $\mathcal{H}\otimes\mathcal{B}\left[0,\infty\right)/\mathcal{E}$ medible. Por tanto CCM es una condici\'on m\'as fuerte que CM.
\end{Note}

\begin{Def}
Un conjunto de trayectorias $H$ de un PEOSCT $Z$, es internamente shift-invariante (\textbf{ISI}) si 
\begin{eqnarray*}
\left\{\left(z_{t+s}\right)_{s\in\left[0,\infty\right)}:z\in H\right\}=H\textrm{, }t\in\left[0,\infty\right).
\end{eqnarray*}
\end{Def}


\begin{Def}
Dado un PEOSCTISI, se define el mapeo-shift $\theta_{t}$, $t\in\left[0,\infty\right)$, de $H$ a $H$ por 
\begin{eqnarray*}
\theta_{t}z=\left(z_{t+s}\right)_{s\in\left[0,\infty\right)}\textrm{, }z\in H.
\end{eqnarray*}
\end{Def}

\begin{Def}
Se dice que un proceso $Z$ es shift-medible (\textbf{SM}) si $Z$ tiene un conjunto de trayectorias $H$ que es ISI y adem\'as el mapeo que toma $\left(z,t\right)\in H\times\left[0,\infty\right)$ en $\theta_{t}z\in H$ es $\mathcal{H}\otimes\mathcal{B}\left[0,\infty\right)/\mathcal{H}$ medible.
\end{Def}

\begin{Note}
Un proceso estoc\'astico con conjunto de trayectorias $H$ ISI es shift-medible si y s\'olo si es CCM
\end{Note}

\begin{Note}
\begin{itemize}
\item Dado el espacio polaco $\left(E,\mathcal{E}\right)$ se tiene el  conjunto de trayectorias $D_{E}\left[0,\infty\right)$ que es ISI, entonces cumpe con ser CCM.

\item Si $G$ es abierto, podemos cubrirlo por bolas abiertas cuay cerradura este contenida en $G$, y como $G$ es segundo numerable como subespacio de $E$, lo podemos cubrir por una cantidad numerable de bolas abiertas.

\end{itemize}
\end{Note}


\begin{Note}
Los procesos estoc\'asticos $Z$ a tiempo discreto con espacio de estados polaco, tambi\'en tiene un espacio de trayectorias polaco y por tanto tiene distribuciones condicionales regulares.
\end{Note}

\begin{Teo}
El producto numerable de espacios polacos es polaco.
\end{Teo}


\begin{Def}
Sea $\left(\Omega,\mathcal{F},\prob\right)$ espacio de probabilidad que soporta al proceso $Z=\left(Z_{s}\right)_{s\in\left[0,\infty\right)}$ y $S=\left(S_{k}\right)_{0}^{\infty}$ donde $Z$ es un PEOSCTM con espacio de estados $\left(E,\mathcal{E}\right)$  y espacio de trayectorias $\left(H,\mathcal{H}\right)$  y adem\'as $S$ es una sucesi\'on de tiempos aleatorios one-sided que satisfacen la condici\'on $0\leq S_{0}<S_{1}<\cdots\rightarrow\infty$. Considerando $S$ como un mapeo medible de $\left(\Omega,\mathcal{F}\right)$ al espacio sucesi\'on $\left(L,\mathcal{L}\right)$, donde 
\begin{eqnarray*}
L=\left\{\left(s_{k}\right)_{0}^{\infty}\in\left[0,\infty\right)^{\left\{0,1,\ldots\right\}}:s_{0}<s_{1}<\cdots\rightarrow\infty\right\},
\end{eqnarray*}
donde $\mathcal{L}$ son los subconjuntos de Borel de $L$, es decir, $\mathcal{L}=L\cap\mathcal{B}^{\left\{0,1,\ldots\right\}}$.

As\'i el par $\left(Z,S\right)$ es un mapeo medible de  $\left(\Omega,\mathcal{F}\right)$ en $\left(H\times L,\mathcal{H}\otimes\mathcal{L}\right)$. El par $\mathcal{H}\otimes\mathcal{L}^{+}$ denotar\'a la clase de todas las funciones medibles de $\left(H\times L,\mathcal{H}\otimes\mathcal{L}\right)$ en $\left(\left[0,\infty\right),\mathcal{B}\left[0,\infty\right)\right)$.
\end{Def}


\begin{Def}
Sea $\theta_{t}$ el mapeo-shift conjunto de $H\times L$ en $H\times L$ dado por
\begin{eqnarray*}
\theta_{t}\left(z,\left(s_{k}\right)_{0}^{\infty}\right)=\theta_{t}\left(z,\left(s_{n_{t-}+k}-t\right)_{0}^{\infty}\right)
\end{eqnarray*}
donde 
$n_{t-}=inf\left\{n\geq1:s_{n}\geq t\right\}$.
\end{Def}

\begin{Note}
Con la finalidad de poder realizar los shift's sin complicaciones de medibilidad, se supondr\'a que $Z$ es shit-medible, es decir, el conjunto de trayectorias $H$ es invariante bajo shifts del tiempo y el mapeo que toma $\left(z,t\right)\in H\times\left[0,\infty\right)$ en $z_{t}\in E$ es $\mathcal{H}\otimes\mathcal{B}\left[0,\infty\right)/\mathcal{E}$ medible.
\end{Note}

\begin{Def}
Dado un proceso \textbf{PEOSSM} (Proceso Estoc\'astico One Side Shift Medible) $Z$, se dice regenerativo cl\'asico con tiempos de regeneraci\'on $S$ si 

\begin{eqnarray*}
\theta_{S_{n}}\left(Z,S\right)=\left(Z^{0},S^{0}\right),n\geq0
\end{eqnarray*}
y adem\'as $\theta_{S_{n}}\left(Z,S\right)$ es independiente de $\left(\left(Z_{s}\right)s\in\left[0,S_{n}\right),S_{0},\ldots,S_{n}\right)$
Si lo anterior se cumple, al par $\left(Z,S\right)$ se le llama regenerativo cl\'asico.
\end{Def}

\begin{Note}
Si el par $\left(Z,S\right)$ es regenerativo cl\'asico, entonces las longitudes de los ciclos $X_{1},X_{2},\ldots,$ son i.i.d. e independientes de la longitud del retraso $S_{0}$, es decir, $S$ es un proceso de renovaci\'on. Las longitudes de los ciclos tambi\'en son llamados tiempos de inter-regeneraci\'on y tiempos de ocurrencia.

\end{Note}

\begin{Teo}
Sup\'ongase que el par $\left(Z,S\right)$ es regenerativo cl\'asico con $\esp\left[X_{1}\right]<\infty$. Entonces $\left(Z^{*},S^{*}\right)$ en el teorema 2.1 es una versi\'on estacionaria de $\left(Z,S\right)$. Adem\'as, si $X_{1}$ es lattice con span $d$, entonces $\left(Z^{**},S^{**}\right)$ en el teorema 2.2 es una versi\'on periodicamente estacionaria de $\left(Z,S\right)$ con periodo $d$.

\end{Teo}

\begin{Def}
Una variable aleatoria $X_{1}$ es \textit{spread out} si existe una $n\geq1$ y una  funci\'on $f\in\mathcal{B}^{+}$ tal que $\int_{\rea}f\left(x\right)dx>0$ con $X_{2},X_{3},\ldots,X_{n}$ copias i.i.d  de $X_{1}$, $$\prob\left(X_{1}+\cdots+X_{n}\in B\right)\geq\int_{B}f\left(x\right)dx$$ para $B\in\mathcal{B}$.

\end{Def}



\begin{Def}
Dado un proceso estoc\'astico $Z$ se le llama \textit{wide-sense regenerative} (\textbf{WSR}) con tiempos de regeneraci\'on $S$ si $\theta_{S_{n}}\left(Z,S\right)=\left(Z^{0},S^{0}\right)$ para $n\geq0$ en distribuci\'on y $\theta_{S_{n}}\left(Z,S\right)$ es independiente de $\left(S_{0},S_{1},\ldots,S_{n}\right)$ para $n\geq0$.
Se dice que el par $\left(Z,S\right)$ es WSR si lo anterior se cumple.
\end{Def}


\begin{Note}
\begin{itemize}
\item El proceso de trayectorias $\left(\theta_{s}Z\right)_{s\in\left[0,\infty\right)}$ es WSR con tiempos de regeneraci\'on $S$ pero no es regenerativo cl\'asico.

\item Si $Z$ es cualquier proceso estacionario y $S$ es un proceso de renovaci\'on que es independiente de $Z$, entonces $\left(Z,S\right)$ es WSR pero en general no es regenerativo cl\'asico

\end{itemize}

\end{Note}


\begin{Note}
Para cualquier proceso estoc\'astico $Z$, el proceso de trayectorias $\left(\theta_{s}Z\right)_{s\in\left[0,\infty\right)}$ es siempre un proceso de Markov.
\end{Note}



\begin{Teo}
Supongase que el par $\left(Z,S\right)$ es WSR con $\esp\left[X_{1}\right]<\infty$. Entonces $\left(Z^{*},S^{*}\right)$ en el teorema 2.1 es una versi\'on estacionaria de 
$\left(Z,S\right)$.
\end{Teo}


\begin{Teo}
Supongase que $\left(Z,S\right)$ es cycle-stationary con $\esp\left[X_{1}\right]<\infty$. Sea $U$ distribuida uniformemente en $\left[0,1\right)$ e independiente de $\left(Z^{0},S^{0}\right)$ y sea $\prob^{*}$ la medida de probabilidad en $\left(\Omega,\prob\right)$ definida por $$d\prob^{*}=\frac{X_{1}}{\esp\left[X_{1}\right]}d\prob$$. Sea $\left(Z^{*},S^{*}\right)$ con distribuci\'on $\prob^{*}\left(\theta_{UX_{1}}\left(Z^{0},S^{0}\right)\in\cdot\right)$. Entonces $\left(Z^{}*,S^{*}\right)$ es estacionario,
\begin{eqnarray*}
\esp\left[f\left(Z^{*},S^{*}\right)\right]=\esp\left[\int_{0}^{X_{1}}f\left(\theta_{s}\left(Z^{0},S^{0}\right)\right)ds\right]/\esp\left[X_{1}\right]
\end{eqnarray*}
$f\in\mathcal{H}\otimes\mathcal{L}^{+}$, and $S_{0}^{*}$ es continuo con funci\'on distribuci\'on $G_{\infty}$ definida por $$G_{\infty}\left(x\right):=\frac{\esp\left[X_{1}\right]\wedge x}{\esp\left[X_{1}\right]}$$ para $x\geq0$ y densidad $\prob\left[X_{1}>x\right]/\esp\left[X_{1}\right]$, con $x\geq0$.

\end{Teo}


\begin{Teo}
Sea $Z$ un Proceso Estoc\'astico un lado shift-medible \textit{one-sided shift-measurable stochastic process}, (PEOSSM),
y $S_{0}$ y $S_{1}$ tiempos aleatorios tales que $0\leq S_{0}<S_{1}$ y
\begin{equation}
\theta_{S_{1}}Z=\theta_{S_{0}}Z\textrm{ en distribuci\'on}.
\end{equation}

Entonces el espacio de probabilidad subyacente $\left(\Omega,\mathcal{F},\prob\right)$ puede extenderse para soportar una sucesi\'on de tiempos aleatorios $S$ tales que

\begin{eqnarray}
\theta_{S_{n}}\left(Z,S\right)=\left(Z^{0},S^{0}\right),n\geq0,\textrm{ en distribuci\'on},\\
\left(Z,S_{0},S_{1}\right)\textrm{ depende de }\left(X_{2},X_{3},\ldots\right)\textrm{ solamente a traves de }\theta_{S_{1}}Z.
\end{eqnarray}
\end{Teo}

\begin{Def}
Un elemento aleatorio en un espacio medible $\left(E,\mathcal{E}\right)$ en un espacio de probabilidad $\left(\Omega,\mathcal{F},\prob\right)$ a $\left(E,\mathcal{E}\right)$, es decir,
para $A\in \mathcal{E}$,  se tiene que $\left\{Y\in A\right\}\in\mathcal{F}$, donde $\left\{Y\in A\right\}:=\left\{w\in\Omega:Y\left(w\right)\in A\right\}=:Y^{-1}A$.
\end{Def}

\begin{Note}
Tambi\'en se dice que $Y$ est\'a soportado por el espacio de probabilidad $\left(\Omega,\mathcal{F},\prob\right)$ y que $Y$ es un mapeo medible de $\Omega$ en $E$, es decir, es $\mathcal{F}/\mathcal{E}$ medible.
\end{Note}

\begin{Def}
Para cada $i\in \mathbb{I}$ sea $P_{i}$ una medida de probabilidad en un espacio medible $\left(E_{i},\mathcal{E}_{i}\right)$. Se define el espacio producto
$\otimes_{i\in\mathbb{I}}\left(E_{i},\mathcal{E}_{i}\right):=\left(\prod_{i\in\mathbb{I}}E_{i},\otimes_{i\in\mathbb{I}}\mathcal{E}_{i}\right)$, donde $\prod_{i\in\mathbb{I}}E_{i}$ es el producto cartesiano de los $E_{i}$'s, y $\otimes_{i\in\mathbb{I}}\mathcal{E}_{i}$ es la $\sigma$-\'algebra producto, es decir, es la $\sigma$-\'algebra m\'as peque\~na en $\prod_{i\in\mathbb{I}}E_{i}$ que hace al $i$-\'esimo mapeo proyecci\'on en $E_{i}$ medible para toda $i\in\mathbb{I}$ es la $\sigma$-\'algebra inducida por los mapeos proyecci\'on. $$\otimes_{i\in\mathbb{I}}\mathcal{E}_{i}:=\sigma\left\{\left\{y:y_{i}\in A\right\}:i\in\mathbb{I}\textrm{ y }A\in\mathcal{E}_{i}\right\}.$$
\end{Def}

\begin{Def}
Un espacio de probabilidad $\left(\tilde{\Omega},\tilde{\mathcal{F}},\tilde{\prob}\right)$ es una extensi\'on de otro espacio de probabilidad $\left(\Omega,\mathcal{F},\prob\right)$ si $\left(\tilde{\Omega},\tilde{\mathcal{F}},\tilde{\prob}\right)$ soporta un elemento aleatorio $\xi\in\left(\Omega,\mathcal{F}\right)$ que tienen a $\prob$ como distribuci\'on.
\end{Def}

\begin{Teo}
Sea $\mathbb{I}$ un conjunto de \'indices arbitrario. Para cada $i\in\mathbb{I}$ sea $P_{i}$ una medida de probabilidad en un espacio medible $\left(E_{i},\mathcal{E}_{i}\right)$. Entonces existe una \'unica medida de probabilidad $\otimes_{i\in\mathbb{I}}P_{i}$ en $\otimes_{i\in\mathbb{I}}\left(E_{i},\mathcal{E}_{i}\right)$ tal que 

\begin{eqnarray*}
\otimes_{i\in\mathbb{I}}P_{i}\left(y\in\prod_{i\in\mathbb{I}}E_{i}:y_{i}\in A_{i_{1}},\ldots,y_{n}\in A_{i_{n}}\right)=P_{i_{1}}\left(A_{i_{n}}\right)\cdots P_{i_{n}}\left(A_{i_{n}}\right)
\end{eqnarray*}
para todos los enteros $n>0$, toda $i_{1},\ldots,i_{n}\in\mathbb{I}$ y todo $A_{i_{1}}\in\mathcal{E}_{i_{1}},\ldots,A_{i_{n}}\in\mathcal{E}_{i_{n}}$
\end{Teo}

La medida $\otimes_{i\in\mathbb{I}}P_{i}$ es llamada la medida producto y $\otimes_{i\in\mathbb{I}}\left(E_{i},\mathcal{E}_{i},P_{i}\right):=\left(\prod_{i\in\mathbb{I}},E_{i},\otimes_{i\in\mathbb{I}}\mathcal{E}_{i},\otimes_{i\in\mathbb{I}}P_{i}\right)$, es llamado espacio de probabilidad producto.


\begin{Def}
Un espacio medible $\left(E,\mathcal{E}\right)$ es \textit{Polaco} si existe una m\'etrica en $E$ tal que $E$ es completo, es decir cada sucesi\'on de Cauchy converge a un l\'imite en $E$, y \textit{separable}, $E$ tienen un subconjunto denso numerable, y tal que $\mathcal{E}$ es generado por conjuntos abiertos.
\end{Def}


\begin{Def}
Dos espacios medibles $\left(E,\mathcal{E}\right)$ y $\left(G,\mathcal{G}\right)$ son Borel equivalentes \textit{isomorfos} si existe una biyecci\'on $f:E\rightarrow G$ tal que $f$ es $\mathcal{E}/\mathcal{G}$ medible y su inversa $f^{-1}$ es $\mathcal{G}/\mathcal{E}$ medible. La biyecci\'on es una equivalencia de Borel.
\end{Def}

\begin{Def}
Un espacio medible  $\left(E,\mathcal{E}\right)$ es un \textit{espacio est\'andar} si es Borel equivalente a $\left(G,\mathcal{G}\right)$, donde $G$ es un subconjunto de Borel de $\left[0,1\right]$ y $\mathcal{G}$ son los subconjuntos de Borel de $G$.
\end{Def}

\begin{Note}
Cualquier espacio Polaco es un espacio est\'andar.
\end{Note}


\begin{Def}
Un proceso estoc\'astico con conjunto de \'indices $\mathbb{I}$ y espacio de estados $\left(E,\mathcal{E}\right)$ es una familia $Z=\left(\mathbb{Z}_{s}\right)_{s\in\mathbb{I}}$ donde $\mathbb{Z}_{s}$ son elementos aleatorios definidos en un espacio de probabilidad com\'un $\left(\Omega,\mathcal{F},\prob\right)$ y todos toman valores en $\left(E,\mathcal{E}\right)$.
\end{Def}

\begin{Def}
Un proceso estoc\'astico \textit{one-sided contiuous time} (\textbf{PEOSCT}) es un proceso estoc\'astico con conjunto de \'indices $\mathbb{I}=\left[0,\infty\right)$.
\end{Def}


Sea $\left(E^{\mathbb{I}},\mathcal{E}^{\mathbb{I}}\right)$ denota el espacio producto $\left(E^{\mathbb{I}},\mathcal{E}^{\mathbb{I}}\right):=\otimes_{s\in\mathbb{I}}\left(E,\mathcal{E}\right)$. Vamos a considerar $\mathbb{Z}$ como un mapeo aleatorio, es decir, como un elemento aleatorio en $\left(E^{\mathbb{I}},\mathcal{E}^{\mathbb{I}}\right)$ definido por $Z\left(w\right)=\left(Z_{s}\left(w\right)\right)_{s\in\mathbb{I}}$ y $w\in\Omega$.

\begin{Note}
La distribuci\'on de un proceso estoc\'astico $Z$ es la distribuci\'on de $Z$ como un elemento aleatorio en $\left(E^{\mathbb{I}},\mathcal{E}^{\mathbb{I}}\right)$. La distribuci\'on de $Z$ esta determinada de manera \'unica por las distribuciones finito dimensionales.
\end{Note}

\begin{Note}
En particular cuando $Z$ toma valores reales, es decir, $\left(E,\mathcal{E}\right)=\left(\mathbb{R},\mathcal{B}\right)$ las distribuciones finito dimensionales est\'an determinadas por las funciones de distribuci\'on finito dimensionales

\begin{eqnarray}
\prob\left(Z_{t_{1}}\leq x_{1},\ldots,Z_{t_{n}}\leq x_{n}\right),x_{1},\ldots,x_{n}\in\mathbb{R},t_{1},\ldots,t_{n}\in\mathbb{I},n\geq1.
\end{eqnarray}
\end{Note}

\begin{Note}
Para espacios polacos $\left(E,\mathcal{E}\right)$ el Teorema de Consistencia de Kolmogorov asegura que dada una colecci\'on de distribuciones finito dimensionales consistentes, siempre existe un proceso estoc\'astico que posee tales distribuciones finito dimensionales.
\end{Note}


\begin{Def}
Las trayectorias de $Z$ son las realizaciones $Z\left(w\right)$ para $w\in\Omega$ del mapeo aleatorio $Z$.
\end{Def}

\begin{Note}
Algunas restricciones se imponen sobre las trayectorias, por ejemplo que sean continuas por la derecha, o continuas por la derecha con l\'imites por la izquierda, o de manera m\'as general, se pedir\'a que caigan en alg\'un subconjunto $H$ de $E^{\mathbb{I}}$. En este caso es natural considerar a $Z$ como un elemento aleatorio que no est\'a en $\left(E^{\mathbb{I}},\mathcal{E}^{\mathbb{I}}\right)$ sino en $\left(H,\mathcal{H}\right)$, donde $\mathcal{H}$ es la $\sigma$-\'algebra generada por los mapeos proyecci\'on que toman a $z\in H$ a $z_{t}\in E$ para $t\in\mathbb{I}$. A $\mathcal{H}$ se le conoce como la traza de $H$ en $E^{\mathbb{I}}$, es decir,
\begin{eqnarray}
\mathcal{H}:=E^{\mathbb{I}}\cap H:=\left\{A\cap H:A\in E^{\mathbb{I}}\right\}.
\end{eqnarray}
\end{Note}


\begin{Note}
$Z$ tiene trayectorias con valores en $H$ y cada $Z_{t}$ es un mapeo medible de $\left(\Omega,\mathcal{F}\right)$ a $\left(H,\mathcal{H}\right)$. Cuando se considera un espacio de trayectorias en particular $H$, al espacio $\left(H,\mathcal{H}\right)$ se le llama el espacio de trayectorias de $Z$.
\end{Note}

\begin{Note}
La distribuci\'on del proceso estoc\'astico $Z$ con espacio de trayectorias $\left(H,\mathcal{H}\right)$ es la distribuci\'on de $Z$ como  un elemento aleatorio en $\left(H,\mathcal{H}\right)$. La distribuci\'on, nuevemente, est\'a determinada de manera \'unica por las distribuciones finito dimensionales.
\end{Note}


\begin{Def}
Sea $Z$ un PEOSCT  con espacio de estados $\left(E,\mathcal{E}\right)$ y sea $T$ un tiempo aleatorio en $\left[0,\infty\right)$. Por $Z_{T}$ se entiende el mapeo con valores en $E$ definido en $\Omega$ en la manera obvia:
\begin{eqnarray*}
Z_{T}\left(w\right):=Z_{T\left(w\right)}\left(w\right). w\in\Omega.
\end{eqnarray*}
\end{Def}

\begin{Def}
Un PEOSCT $Z$ es conjuntamente medible (\textbf{CM}) si el mapeo que toma $\left(w,t\right)\in\Omega\times\left[0,\infty\right)$ a $Z_{t}\left(w\right)\in E$ es $\mathcal{F}\otimes\mathcal{B}\left[0,\infty\right)/\mathcal{E}$ medible.
\end{Def}

\begin{Note}
Un PEOSCT-CM implica que el proceso es medible, dado que $Z_{T}$ es una composici\'on  de dos mapeos continuos: el primero que toma $w$ en $\left(w,T\left(w\right)\right)$ es $\mathcal{F}/\mathcal{F}\otimes\mathcal{B}\left[0,\infty\right)$ medible, mientras que el segundo toma $\left(w,T\left(w\right)\right)$ en $Z_{T\left(w\right)}\left(w\right)$ es $\mathcal{F}\otimes\mathcal{B}\left[0,\infty\right)/\mathcal{E}$ medible.
\end{Note}


\begin{Def}
Un PEOSCT con espacio de estados $\left(H,\mathcal{H}\right)$ es can\'onicamente conjuntamente medible (\textbf{CCM}) si el mapeo $\left(z,t\right)\in H\times\left[0,\infty\right)$ en $Z_{t}\in E$ es $\mathcal{H}\otimes\mathcal{B}\left[0,\infty\right)/\mathcal{E}$ medible.
\end{Def}

\begin{Note}
Un PEOSCTCCM implica que el proceso es CM, dado que un PECCM $Z$ es un mapeo de $\Omega\times\left[0,\infty\right)$ a $E$, es la composici\'on de dos mapeos medibles: el primero, toma $\left(w,t\right)$ en $\left(Z\left(w\right),t\right)$ es $\mathcal{F}\otimes\mathcal{B}\left[0,\infty\right)/\mathcal{H}\otimes\mathcal{B}\left[0,\infty\right)$ medible, y el segundo que toma $\left(Z\left(w\right),t\right)$  en $Z_{t}\left(w\right)$ es $\mathcal{H}\otimes\mathcal{B}\left[0,\infty\right)/\mathcal{E}$ medible. Por tanto CCM es una condici\'on m\'as fuerte que CM.
\end{Note}

\begin{Def}
Un conjunto de trayectorias $H$ de un PEOSCT $Z$, es internamente shift-invariante (\textbf{ISI}) si 
\begin{eqnarray*}
\left\{\left(z_{t+s}\right)_{s\in\left[0,\infty\right)}:z\in H\right\}=H\textrm{, }t\in\left[0,\infty\right).
\end{eqnarray*}
\end{Def}


\begin{Def}
Dado un PEOSCTISI, se define el mapeo-shift $\theta_{t}$, $t\in\left[0,\infty\right)$, de $H$ a $H$ por 
\begin{eqnarray*}
\theta_{t}z=\left(z_{t+s}\right)_{s\in\left[0,\infty\right)}\textrm{, }z\in H.
\end{eqnarray*}
\end{Def}

\begin{Def}
Se dice que un proceso $Z$ es shift-medible (\textbf{SM}) si $Z$ tiene un conjunto de trayectorias $H$ que es ISI y adem\'as el mapeo que toma $\left(z,t\right)\in H\times\left[0,\infty\right)$ en $\theta_{t}z\in H$ es $\mathcal{H}\otimes\mathcal{B}\left[0,\infty\right)/\mathcal{H}$ medible.
\end{Def}

\begin{Note}
Un proceso estoc\'astico con conjunto de trayectorias $H$ ISI es shift-medible si y s\'olo si es CCM
\end{Note}

\begin{Note}
\begin{itemize}
\item Dado el espacio polaco $\left(E,\mathcal{E}\right)$ se tiene el  conjunto de trayectorias $D_{E}\left[0,\infty\right)$ que es ISI, entonces cumpe con ser CCM.

\item Si $G$ es abierto, podemos cubrirlo por bolas abiertas cuay cerradura este contenida en $G$, y como $G$ es segundo numerable como subespacio de $E$, lo podemos cubrir por una cantidad numerable de bolas abiertas.

\end{itemize}
\end{Note}


\begin{Note}
Los procesos estoc\'asticos $Z$ a tiempo discreto con espacio de estados polaco, tambi\'en tiene un espacio de trayectorias polaco y por tanto tiene distribuciones condicionales regulares.
\end{Note}

\begin{Teo}
El producto numerable de espacios polacos es polaco.
\end{Teo}


\begin{Def}
Sea $\left(\Omega,\mathcal{F},\prob\right)$ espacio de probabilidad que soporta al proceso $Z=\left(Z_{s}\right)_{s\in\left[0,\infty\right)}$ y $S=\left(S_{k}\right)_{0}^{\infty}$ donde $Z$ es un PEOSCTM con espacio de estados $\left(E,\mathcal{E}\right)$  y espacio de trayectorias $\left(H,\mathcal{H}\right)$  y adem\'as $S$ es una sucesi\'on de tiempos aleatorios one-sided que satisfacen la condici\'on $0\leq S_{0}<S_{1}<\cdots\rightarrow\infty$. Considerando $S$ como un mapeo medible de $\left(\Omega,\mathcal{F}\right)$ al espacio sucesi\'on $\left(L,\mathcal{L}\right)$, donde 
\begin{eqnarray*}
L=\left\{\left(s_{k}\right)_{0}^{\infty}\in\left[0,\infty\right)^{\left\{0,1,\ldots\right\}}:s_{0}<s_{1}<\cdots\rightarrow\infty\right\},
\end{eqnarray*}
donde $\mathcal{L}$ son los subconjuntos de Borel de $L$, es decir, $\mathcal{L}=L\cap\mathcal{B}^{\left\{0,1,\ldots\right\}}$.

As\'i el par $\left(Z,S\right)$ es un mapeo medible de  $\left(\Omega,\mathcal{F}\right)$ en $\left(H\times L,\mathcal{H}\otimes\mathcal{L}\right)$. El par $\mathcal{H}\otimes\mathcal{L}^{+}$ denotar\'a la clase de todas las funciones medibles de $\left(H\times L,\mathcal{H}\otimes\mathcal{L}\right)$ en $\left(\left[0,\infty\right),\mathcal{B}\left[0,\infty\right)\right)$.
\end{Def}


\begin{Def}
Sea $\theta_{t}$ el mapeo-shift conjunto de $H\times L$ en $H\times L$ dado por
\begin{eqnarray*}
\theta_{t}\left(z,\left(s_{k}\right)_{0}^{\infty}\right)=\theta_{t}\left(z,\left(s_{n_{t-}+k}-t\right)_{0}^{\infty}\right)
\end{eqnarray*}
donde 
$n_{t-}=inf\left\{n\geq1:s_{n}\geq t\right\}$.
\end{Def}

\begin{Note}
Con la finalidad de poder realizar los shift's sin complicaciones de medibilidad, se supondr\'a que $Z$ es shit-medible, es decir, el conjunto de trayectorias $H$ es invariante bajo shifts del tiempo y el mapeo que toma $\left(z,t\right)\in H\times\left[0,\infty\right)$ en $z_{t}\in E$ es $\mathcal{H}\otimes\mathcal{B}\left[0,\infty\right)/\mathcal{E}$ medible.
\end{Note}

\begin{Def}
Dado un proceso \textbf{PEOSSM} (Proceso Estoc\'astico One Side Shift Medible) $Z$, se dice regenerativo cl\'asico con tiempos de regeneraci\'on $S$ si 

\begin{eqnarray*}
\theta_{S_{n}}\left(Z,S\right)=\left(Z^{0},S^{0}\right),n\geq0
\end{eqnarray*}
y adem\'as $\theta_{S_{n}}\left(Z,S\right)$ es independiente de $\left(\left(Z_{s}\right)s\in\left[0,S_{n}\right),S_{0},\ldots,S_{n}\right)$
Si lo anterior se cumple, al par $\left(Z,S\right)$ se le llama regenerativo cl\'asico.
\end{Def}

\begin{Note}
Si el par $\left(Z,S\right)$ es regenerativo cl\'asico, entonces las longitudes de los ciclos $X_{1},X_{2},\ldots,$ son i.i.d. e independientes de la longitud del retraso $S_{0}$, es decir, $S$ es un proceso de renovaci\'on. Las longitudes de los ciclos tambi\'en son llamados tiempos de inter-regeneraci\'on y tiempos de ocurrencia.

\end{Note}

\begin{Teo}
Sup\'ongase que el par $\left(Z,S\right)$ es regenerativo cl\'asico con $\esp\left[X_{1}\right]<\infty$. Entonces $\left(Z^{*},S^{*}\right)$ en el teorema 2.1 es una versi\'on estacionaria de $\left(Z,S\right)$. Adem\'as, si $X_{1}$ es lattice con span $d$, entonces $\left(Z^{**},S^{**}\right)$ en el teorema 2.2 es una versi\'on periodicamente estacionaria de $\left(Z,S\right)$ con periodo $d$.

\end{Teo}

\begin{Def}
Una variable aleatoria $X_{1}$ es \textit{spread out} si existe una $n\geq1$ y una  funci\'on $f\in\mathcal{B}^{+}$ tal que $\int_{\rea}f\left(x\right)dx>0$ con $X_{2},X_{3},\ldots,X_{n}$ copias i.i.d  de $X_{1}$, $$\prob\left(X_{1}+\cdots+X_{n}\in B\right)\geq\int_{B}f\left(x\right)dx$$ para $B\in\mathcal{B}$.

\end{Def}



\begin{Def}
Dado un proceso estoc\'astico $Z$ se le llama \textit{wide-sense regenerative} (\textbf{WSR}) con tiempos de regeneraci\'on $S$ si $\theta_{S_{n}}\left(Z,S\right)=\left(Z^{0},S^{0}\right)$ para $n\geq0$ en distribuci\'on y $\theta_{S_{n}}\left(Z,S\right)$ es independiente de $\left(S_{0},S_{1},\ldots,S_{n}\right)$ para $n\geq0$.
Se dice que el par $\left(Z,S\right)$ es WSR si lo anterior se cumple.
\end{Def}


\begin{Note}
\begin{itemize}
\item El proceso de trayectorias $\left(\theta_{s}Z\right)_{s\in\left[0,\infty\right)}$ es WSR con tiempos de regeneraci\'on $S$ pero no es regenerativo cl\'asico.

\item Si $Z$ es cualquier proceso estacionario y $S$ es un proceso de renovaci\'on que es independiente de $Z$, entonces $\left(Z,S\right)$ es WSR pero en general no es regenerativo cl\'asico

\end{itemize}

\end{Note}


\begin{Note}
Para cualquier proceso estoc\'astico $Z$, el proceso de trayectorias $\left(\theta_{s}Z\right)_{s\in\left[0,\infty\right)}$ es siempre un proceso de Markov.
\end{Note}



\begin{Teo}
Supongase que el par $\left(Z,S\right)$ es WSR con $\esp\left[X_{1}\right]<\infty$. Entonces $\left(Z^{*},S^{*}\right)$ en el teorema 2.1 es una versi\'on estacionaria de 
$\left(Z,S\right)$.
\end{Teo}


\begin{Teo}
Supongase que $\left(Z,S\right)$ es cycle-stationary con $\esp\left[X_{1}\right]<\infty$. Sea $U$ distribuida uniformemente en $\left[0,1\right)$ e independiente de $\left(Z^{0},S^{0}\right)$ y sea $\prob^{*}$ la medida de probabilidad en $\left(\Omega,\prob\right)$ definida por $$d\prob^{*}=\frac{X_{1}}{\esp\left[X_{1}\right]}d\prob$$. Sea $\left(Z^{*},S^{*}\right)$ con distribuci\'on $\prob^{*}\left(\theta_{UX_{1}}\left(Z^{0},S^{0}\right)\in\cdot\right)$. Entonces $\left(Z^{}*,S^{*}\right)$ es estacionario,
\begin{eqnarray*}
\esp\left[f\left(Z^{*},S^{*}\right)\right]=\esp\left[\int_{0}^{X_{1}}f\left(\theta_{s}\left(Z^{0},S^{0}\right)\right)ds\right]/\esp\left[X_{1}\right]
\end{eqnarray*}
$f\in\mathcal{H}\otimes\mathcal{L}^{+}$, and $S_{0}^{*}$ es continuo con funci\'on distribuci\'on $G_{\infty}$ definida por $$G_{\infty}\left(x\right):=\frac{\esp\left[X_{1}\right]\wedge x}{\esp\left[X_{1}\right]}$$ para $x\geq0$ y densidad $\prob\left[X_{1}>x\right]/\esp\left[X_{1}\right]$, con $x\geq0$.

\end{Teo}


\begin{Teo}
Sea $Z$ un Proceso Estoc\'astico un lado shift-medible \textit{one-sided shift-measurable stochastic process}, (PEOSSM),
y $S_{0}$ y $S_{1}$ tiempos aleatorios tales que $0\leq S_{0}<S_{1}$ y
\begin{equation}
\theta_{S_{1}}Z=\theta_{S_{0}}Z\textrm{ en distribuci\'on}.
\end{equation}

Entonces el espacio de probabilidad subyacente $\left(\Omega,\mathcal{F},\prob\right)$ puede extenderse para soportar una sucesi\'on de tiempos aleatorios $S$ tales que

\begin{eqnarray}
\theta_{S_{n}}\left(Z,S\right)=\left(Z^{0},S^{0}\right),n\geq0,\textrm{ en distribuci\'on},\\
\left(Z,S_{0},S_{1}\right)\textrm{ depende de }\left(X_{2},X_{3},\ldots\right)\textrm{ solamente a traves de }\theta_{S_{1}}Z.
\end{eqnarray}
\end{Teo}




\begin{Def}
Un elemento aleatorio en un espacio medible $\left(E,\mathcal{E}\right)$ en un espacio de probabilidad $\left(\Omega,\mathcal{F},\prob\right)$ a $\left(E,\mathcal{E}\right)$, es decir,
para $A\in \mathcal{E}$,  se tiene que $\left\{Y\in A\right\}\in\mathcal{F}$, donde $\left\{Y\in A\right\}:=\left\{w\in\Omega:Y\left(w\right)\in A\right\}=:Y^{-1}A$.
\end{Def}

\begin{Note}
Tambi\'en se dice que $Y$ est\'a soportado por el espacio de probabilidad $\left(\Omega,\mathcal{F},\prob\right)$ y que $Y$ es un mapeo medible de $\Omega$ en $E$, es decir, es $\mathcal{F}/\mathcal{E}$ medible.
\end{Note}

\begin{Def}
Para cada $i\in \mathbb{I}$ sea $P_{i}$ una medida de probabilidad en un espacio medible $\left(E_{i},\mathcal{E}_{i}\right)$. Se define el espacio producto
$\otimes_{i\in\mathbb{I}}\left(E_{i},\mathcal{E}_{i}\right):=\left(\prod_{i\in\mathbb{I}}E_{i},\otimes_{i\in\mathbb{I}}\mathcal{E}_{i}\right)$, donde $\prod_{i\in\mathbb{I}}E_{i}$ es el producto cartesiano de los $E_{i}$'s, y $\otimes_{i\in\mathbb{I}}\mathcal{E}_{i}$ es la $\sigma$-\'algebra producto, es decir, es la $\sigma$-\'algebra m\'as peque\~na en $\prod_{i\in\mathbb{I}}E_{i}$ que hace al $i$-\'esimo mapeo proyecci\'on en $E_{i}$ medible para toda $i\in\mathbb{I}$ es la $\sigma$-\'algebra inducida por los mapeos proyecci\'on. $$\otimes_{i\in\mathbb{I}}\mathcal{E}_{i}:=\sigma\left\{\left\{y:y_{i}\in A\right\}:i\in\mathbb{I}\textrm{ y }A\in\mathcal{E}_{i}\right\}.$$
\end{Def}

\begin{Def}
Un espacio de probabilidad $\left(\tilde{\Omega},\tilde{\mathcal{F}},\tilde{\prob}\right)$ es una extensi\'on de otro espacio de probabilidad $\left(\Omega,\mathcal{F},\prob\right)$ si $\left(\tilde{\Omega},\tilde{\mathcal{F}},\tilde{\prob}\right)$ soporta un elemento aleatorio $\xi\in\left(\Omega,\mathcal{F}\right)$ que tienen a $\prob$ como distribuci\'on.
\end{Def}

\begin{Teo}
Sea $\mathbb{I}$ un conjunto de \'indices arbitrario. Para cada $i\in\mathbb{I}$ sea $P_{i}$ una medida de probabilidad en un espacio medible $\left(E_{i},\mathcal{E}_{i}\right)$. Entonces existe una \'unica medida de probabilidad $\otimes_{i\in\mathbb{I}}P_{i}$ en $\otimes_{i\in\mathbb{I}}\left(E_{i},\mathcal{E}_{i}\right)$ tal que 

\begin{eqnarray*}
\otimes_{i\in\mathbb{I}}P_{i}\left(y\in\prod_{i\in\mathbb{I}}E_{i}:y_{i}\in A_{i_{1}},\ldots,y_{n}\in A_{i_{n}}\right)=P_{i_{1}}\left(A_{i_{n}}\right)\cdots P_{i_{n}}\left(A_{i_{n}}\right)
\end{eqnarray*}
para todos los enteros $n>0$, toda $i_{1},\ldots,i_{n}\in\mathbb{I}$ y todo $A_{i_{1}}\in\mathcal{E}_{i_{1}},\ldots,A_{i_{n}}\in\mathcal{E}_{i_{n}}$
\end{Teo}

La medida $\otimes_{i\in\mathbb{I}}P_{i}$ es llamada la medida producto y $\otimes_{i\in\mathbb{I}}\left(E_{i},\mathcal{E}_{i},P_{i}\right):=\left(\prod_{i\in\mathbb{I}},E_{i},\otimes_{i\in\mathbb{I}}\mathcal{E}_{i},\otimes_{i\in\mathbb{I}}P_{i}\right)$, es llamado espacio de probabilidad producto.


\begin{Def}
Un espacio medible $\left(E,\mathcal{E}\right)$ es \textit{Polaco} si existe una m\'etrica en $E$ tal que $E$ es completo, es decir cada sucesi\'on de Cauchy converge a un l\'imite en $E$, y \textit{separable}, $E$ tienen un subconjunto denso numerable, y tal que $\mathcal{E}$ es generado por conjuntos abiertos.
\end{Def}


\begin{Def}
Dos espacios medibles $\left(E,\mathcal{E}\right)$ y $\left(G,\mathcal{G}\right)$ son Borel equivalentes \textit{isomorfos} si existe una biyecci\'on $f:E\rightarrow G$ tal que $f$ es $\mathcal{E}/\mathcal{G}$ medible y su inversa $f^{-1}$ es $\mathcal{G}/\mathcal{E}$ medible. La biyecci\'on es una equivalencia de Borel.
\end{Def}

\begin{Def}
Un espacio medible  $\left(E,\mathcal{E}\right)$ es un \textit{espacio est\'andar} si es Borel equivalente a $\left(G,\mathcal{G}\right)$, donde $G$ es un subconjunto de Borel de $\left[0,1\right]$ y $\mathcal{G}$ son los subconjuntos de Borel de $G$.
\end{Def}

\begin{Note}
Cualquier espacio Polaco es un espacio est\'andar.
\end{Note}


\begin{Def}
Un proceso estoc\'astico con conjunto de \'indices $\mathbb{I}$ y espacio de estados $\left(E,\mathcal{E}\right)$ es una familia $Z=\left(\mathbb{Z}_{s}\right)_{s\in\mathbb{I}}$ donde $\mathbb{Z}_{s}$ son elementos aleatorios definidos en un espacio de probabilidad com\'un $\left(\Omega,\mathcal{F},\prob\right)$ y todos toman valores en $\left(E,\mathcal{E}\right)$.
\end{Def}

\begin{Def}
Un proceso estoc\'astico \textit{one-sided contiuous time} (\textbf{PEOSCT}) es un proceso estoc\'astico con conjunto de \'indices $\mathbb{I}=\left[0,\infty\right)$.
\end{Def}


Sea $\left(E^{\mathbb{I}},\mathcal{E}^{\mathbb{I}}\right)$ denota el espacio producto $\left(E^{\mathbb{I}},\mathcal{E}^{\mathbb{I}}\right):=\otimes_{s\in\mathbb{I}}\left(E,\mathcal{E}\right)$. Vamos a considerar $\mathbb{Z}$ como un mapeo aleatorio, es decir, como un elemento aleatorio en $\left(E^{\mathbb{I}},\mathcal{E}^{\mathbb{I}}\right)$ definido por $Z\left(w\right)=\left(Z_{s}\left(w\right)\right)_{s\in\mathbb{I}}$ y $w\in\Omega$.

\begin{Note}
La distribuci\'on de un proceso estoc\'astico $Z$ es la distribuci\'on de $Z$ como un elemento aleatorio en $\left(E^{\mathbb{I}},\mathcal{E}^{\mathbb{I}}\right)$. La distribuci\'on de $Z$ esta determinada de manera \'unica por las distribuciones finito dimensionales.
\end{Note}

\begin{Note}
En particular cuando $Z$ toma valores reales, es decir, $\left(E,\mathcal{E}\right)=\left(\mathbb{R},\mathcal{B}\right)$ las distribuciones finito dimensionales est\'an determinadas por las funciones de distribuci\'on finito dimensionales

\begin{eqnarray}
\prob\left(Z_{t_{1}}\leq x_{1},\ldots,Z_{t_{n}}\leq x_{n}\right),x_{1},\ldots,x_{n}\in\mathbb{R},t_{1},\ldots,t_{n}\in\mathbb{I},n\geq1.
\end{eqnarray}
\end{Note}

\begin{Note}
Para espacios polacos $\left(E,\mathcal{E}\right)$ el Teorema de Consistencia de Kolmogorov asegura que dada una colecci\'on de distribuciones finito dimensionales consistentes, siempre existe un proceso estoc\'astico que posee tales distribuciones finito dimensionales.
\end{Note}


\begin{Def}
Las trayectorias de $Z$ son las realizaciones $Z\left(w\right)$ para $w\in\Omega$ del mapeo aleatorio $Z$.
\end{Def}

\begin{Note}
Algunas restricciones se imponen sobre las trayectorias, por ejemplo que sean continuas por la derecha, o continuas por la derecha con l\'imites por la izquierda, o de manera m\'as general, se pedir\'a que caigan en alg\'un subconjunto $H$ de $E^{\mathbb{I}}$. En este caso es natural considerar a $Z$ como un elemento aleatorio que no est\'a en $\left(E^{\mathbb{I}},\mathcal{E}^{\mathbb{I}}\right)$ sino en $\left(H,\mathcal{H}\right)$, donde $\mathcal{H}$ es la $\sigma$-\'algebra generada por los mapeos proyecci\'on que toman a $z\in H$ a $z_{t}\in E$ para $t\in\mathbb{I}$. A $\mathcal{H}$ se le conoce como la traza de $H$ en $E^{\mathbb{I}}$, es decir,
\begin{eqnarray}
\mathcal{H}:=E^{\mathbb{I}}\cap H:=\left\{A\cap H:A\in E^{\mathbb{I}}\right\}.
\end{eqnarray}
\end{Note}


\begin{Note}
$Z$ tiene trayectorias con valores en $H$ y cada $Z_{t}$ es un mapeo medible de $\left(\Omega,\mathcal{F}\right)$ a $\left(H,\mathcal{H}\right)$. Cuando se considera un espacio de trayectorias en particular $H$, al espacio $\left(H,\mathcal{H}\right)$ se le llama el espacio de trayectorias de $Z$.
\end{Note}

\begin{Note}
La distribuci\'on del proceso estoc\'astico $Z$ con espacio de trayectorias $\left(H,\mathcal{H}\right)$ es la distribuci\'on de $Z$ como  un elemento aleatorio en $\left(H,\mathcal{H}\right)$. La distribuci\'on, nuevemente, est\'a determinada de manera \'unica por las distribuciones finito dimensionales.
\end{Note}


\begin{Def}
Sea $Z$ un PEOSCT  con espacio de estados $\left(E,\mathcal{E}\right)$ y sea $T$ un tiempo aleatorio en $\left[0,\infty\right)$. Por $Z_{T}$ se entiende el mapeo con valores en $E$ definido en $\Omega$ en la manera obvia:
\begin{eqnarray*}
Z_{T}\left(w\right):=Z_{T\left(w\right)}\left(w\right). w\in\Omega.
\end{eqnarray*}
\end{Def}

\begin{Def}
Un PEOSCT $Z$ es conjuntamente medible (\textbf{CM}) si el mapeo que toma $\left(w,t\right)\in\Omega\times\left[0,\infty\right)$ a $Z_{t}\left(w\right)\in E$ es $\mathcal{F}\otimes\mathcal{B}\left[0,\infty\right)/\mathcal{E}$ medible.
\end{Def}

\begin{Note}
Un PEOSCT-CM implica que el proceso es medible, dado que $Z_{T}$ es una composici\'on  de dos mapeos continuos: el primero que toma $w$ en $\left(w,T\left(w\right)\right)$ es $\mathcal{F}/\mathcal{F}\otimes\mathcal{B}\left[0,\infty\right)$ medible, mientras que el segundo toma $\left(w,T\left(w\right)\right)$ en $Z_{T\left(w\right)}\left(w\right)$ es $\mathcal{F}\otimes\mathcal{B}\left[0,\infty\right)/\mathcal{E}$ medible.
\end{Note}


\begin{Def}
Un PEOSCT con espacio de estados $\left(H,\mathcal{H}\right)$ es can\'onicamente conjuntamente medible (\textbf{CCM}) si el mapeo $\left(z,t\right)\in H\times\left[0,\infty\right)$ en $Z_{t}\in E$ es $\mathcal{H}\otimes\mathcal{B}\left[0,\infty\right)/\mathcal{E}$ medible.
\end{Def}

\begin{Note}
Un PEOSCTCCM implica que el proceso es CM, dado que un PECCM $Z$ es un mapeo de $\Omega\times\left[0,\infty\right)$ a $E$, es la composici\'on de dos mapeos medibles: el primero, toma $\left(w,t\right)$ en $\left(Z\left(w\right),t\right)$ es $\mathcal{F}\otimes\mathcal{B}\left[0,\infty\right)/\mathcal{H}\otimes\mathcal{B}\left[0,\infty\right)$ medible, y el segundo que toma $\left(Z\left(w\right),t\right)$  en $Z_{t}\left(w\right)$ es $\mathcal{H}\otimes\mathcal{B}\left[0,\infty\right)/\mathcal{E}$ medible. Por tanto CCM es una condici\'on m\'as fuerte que CM.
\end{Note}

\begin{Def}
Un conjunto de trayectorias $H$ de un PEOSCT $Z$, es internamente shift-invariante (\textbf{ISI}) si 
\begin{eqnarray*}
\left\{\left(z_{t+s}\right)_{s\in\left[0,\infty\right)}:z\in H\right\}=H\textrm{, }t\in\left[0,\infty\right).
\end{eqnarray*}
\end{Def}


\begin{Def}
Dado un PEOSCTISI, se define el mapeo-shift $\theta_{t}$, $t\in\left[0,\infty\right)$, de $H$ a $H$ por 
\begin{eqnarray*}
\theta_{t}z=\left(z_{t+s}\right)_{s\in\left[0,\infty\right)}\textrm{, }z\in H.
\end{eqnarray*}
\end{Def}

\begin{Def}
Se dice que un proceso $Z$ es shift-medible (\textbf{SM}) si $Z$ tiene un conjunto de trayectorias $H$ que es ISI y adem\'as el mapeo que toma $\left(z,t\right)\in H\times\left[0,\infty\right)$ en $\theta_{t}z\in H$ es $\mathcal{H}\otimes\mathcal{B}\left[0,\infty\right)/\mathcal{H}$ medible.
\end{Def}

\begin{Note}
Un proceso estoc\'astico con conjunto de trayectorias $H$ ISI es shift-medible si y s\'olo si es CCM
\end{Note}

\begin{Note}
\begin{itemize}
\item Dado el espacio polaco $\left(E,\mathcal{E}\right)$ se tiene el  conjunto de trayectorias $D_{E}\left[0,\infty\right)$ que es ISI, entonces cumpe con ser CCM.

\item Si $G$ es abierto, podemos cubrirlo por bolas abiertas cuay cerradura este contenida en $G$, y como $G$ es segundo numerable como subespacio de $E$, lo podemos cubrir por una cantidad numerable de bolas abiertas.

\end{itemize}
\end{Note}


\begin{Note}
Los procesos estoc\'asticos $Z$ a tiempo discreto con espacio de estados polaco, tambi\'en tiene un espacio de trayectorias polaco y por tanto tiene distribuciones condicionales regulares.
\end{Note}

\begin{Teo}
El producto numerable de espacios polacos es polaco.
\end{Teo}


\begin{Def}
Sea $\left(\Omega,\mathcal{F},\prob\right)$ espacio de probabilidad que soporta al proceso $Z=\left(Z_{s}\right)_{s\in\left[0,\infty\right)}$ y $S=\left(S_{k}\right)_{0}^{\infty}$ donde $Z$ es un PEOSCTM con espacio de estados $\left(E,\mathcal{E}\right)$  y espacio de trayectorias $\left(H,\mathcal{H}\right)$  y adem\'as $S$ es una sucesi\'on de tiempos aleatorios one-sided que satisfacen la condici\'on $0\leq S_{0}<S_{1}<\cdots\rightarrow\infty$. Considerando $S$ como un mapeo medible de $\left(\Omega,\mathcal{F}\right)$ al espacio sucesi\'on $\left(L,\mathcal{L}\right)$, donde 
\begin{eqnarray*}
L=\left\{\left(s_{k}\right)_{0}^{\infty}\in\left[0,\infty\right)^{\left\{0,1,\ldots\right\}}:s_{0}<s_{1}<\cdots\rightarrow\infty\right\},
\end{eqnarray*}
donde $\mathcal{L}$ son los subconjuntos de Borel de $L$, es decir, $\mathcal{L}=L\cap\mathcal{B}^{\left\{0,1,\ldots\right\}}$.

As\'i el par $\left(Z,S\right)$ es un mapeo medible de  $\left(\Omega,\mathcal{F}\right)$ en $\left(H\times L,\mathcal{H}\otimes\mathcal{L}\right)$. El par $\mathcal{H}\otimes\mathcal{L}^{+}$ denotar\'a la clase de todas las funciones medibles de $\left(H\times L,\mathcal{H}\otimes\mathcal{L}\right)$ en $\left(\left[0,\infty\right),\mathcal{B}\left[0,\infty\right)\right)$.
\end{Def}


\begin{Def}
Sea $\theta_{t}$ el mapeo-shift conjunto de $H\times L$ en $H\times L$ dado por
\begin{eqnarray*}
\theta_{t}\left(z,\left(s_{k}\right)_{0}^{\infty}\right)=\theta_{t}\left(z,\left(s_{n_{t-}+k}-t\right)_{0}^{\infty}\right)
\end{eqnarray*}
donde 
$n_{t-}=inf\left\{n\geq1:s_{n}\geq t\right\}$.
\end{Def}

\begin{Note}
Con la finalidad de poder realizar los shift's sin complicaciones de medibilidad, se supondr\'a que $Z$ es shit-medible, es decir, el conjunto de trayectorias $H$ es invariante bajo shifts del tiempo y el mapeo que toma $\left(z,t\right)\in H\times\left[0,\infty\right)$ en $z_{t}\in E$ es $\mathcal{H}\otimes\mathcal{B}\left[0,\infty\right)/\mathcal{E}$ medible.
\end{Note}

\begin{Def}
Dado un proceso \textbf{PEOSSM} (Proceso Estoc\'astico One Side Shift Medible) $Z$, se dice regenerativo cl\'asico con tiempos de regeneraci\'on $S$ si 

\begin{eqnarray*}
\theta_{S_{n}}\left(Z,S\right)=\left(Z^{0},S^{0}\right),n\geq0
\end{eqnarray*}
y adem\'as $\theta_{S_{n}}\left(Z,S\right)$ es independiente de $\left(\left(Z_{s}\right)s\in\left[0,S_{n}\right),S_{0},\ldots,S_{n}\right)$
Si lo anterior se cumple, al par $\left(Z,S\right)$ se le llama regenerativo cl\'asico.
\end{Def}

\begin{Note}
Si el par $\left(Z,S\right)$ es regenerativo cl\'asico, entonces las longitudes de los ciclos $X_{1},X_{2},\ldots,$ son i.i.d. e independientes de la longitud del retraso $S_{0}$, es decir, $S$ es un proceso de renovaci\'on. Las longitudes de los ciclos tambi\'en son llamados tiempos de inter-regeneraci\'on y tiempos de ocurrencia.

\end{Note}

\begin{Teo}
Sup\'ongase que el par $\left(Z,S\right)$ es regenerativo cl\'asico con $\esp\left[X_{1}\right]<\infty$. Entonces $\left(Z^{*},S^{*}\right)$ en el teorema 2.1 es una versi\'on estacionaria de $\left(Z,S\right)$. Adem\'as, si $X_{1}$ es lattice con span $d$, entonces $\left(Z^{**},S^{**}\right)$ en el teorema 2.2 es una versi\'on periodicamente estacionaria de $\left(Z,S\right)$ con periodo $d$.

\end{Teo}

\begin{Def}
Una variable aleatoria $X_{1}$ es \textit{spread out} si existe una $n\geq1$ y una  funci\'on $f\in\mathcal{B}^{+}$ tal que $\int_{\rea}f\left(x\right)dx>0$ con $X_{2},X_{3},\ldots,X_{n}$ copias i.i.d  de $X_{1}$, $$\prob\left(X_{1}+\cdots+X_{n}\in B\right)\geq\int_{B}f\left(x\right)dx$$ para $B\in\mathcal{B}$.

\end{Def}



\begin{Def}
Dado un proceso estoc\'astico $Z$ se le llama \textit{wide-sense regenerative} (\textbf{WSR}) con tiempos de regeneraci\'on $S$ si $\theta_{S_{n}}\left(Z,S\right)=\left(Z^{0},S^{0}\right)$ para $n\geq0$ en distribuci\'on y $\theta_{S_{n}}\left(Z,S\right)$ es independiente de $\left(S_{0},S_{1},\ldots,S_{n}\right)$ para $n\geq0$.
Se dice que el par $\left(Z,S\right)$ es WSR si lo anterior se cumple.
\end{Def}


\begin{Note}
\begin{itemize}
\item El proceso de trayectorias $\left(\theta_{s}Z\right)_{s\in\left[0,\infty\right)}$ es WSR con tiempos de regeneraci\'on $S$ pero no es regenerativo cl\'asico.

\item Si $Z$ es cualquier proceso estacionario y $S$ es un proceso de renovaci\'on que es independiente de $Z$, entonces $\left(Z,S\right)$ es WSR pero en general no es regenerativo cl\'asico

\end{itemize}

\end{Note}


\begin{Note}
Para cualquier proceso estoc\'astico $Z$, el proceso de trayectorias $\left(\theta_{s}Z\right)_{s\in\left[0,\infty\right)}$ es siempre un proceso de Markov.
\end{Note}



\begin{Teo}
Supongase que el par $\left(Z,S\right)$ es WSR con $\esp\left[X_{1}\right]<\infty$. Entonces $\left(Z^{*},S^{*}\right)$ en el teorema 2.1 es una versi\'on estacionaria de 
$\left(Z,S\right)$.
\end{Teo}


\begin{Teo}
Supongase que $\left(Z,S\right)$ es cycle-stationary con $\esp\left[X_{1}\right]<\infty$. Sea $U$ distribuida uniformemente en $\left[0,1\right)$ e independiente de $\left(Z^{0},S^{0}\right)$ y sea $\prob^{*}$ la medida de probabilidad en $\left(\Omega,\prob\right)$ definida por $$d\prob^{*}=\frac{X_{1}}{\esp\left[X_{1}\right]}d\prob$$. Sea $\left(Z^{*},S^{*}\right)$ con distribuci\'on $\prob^{*}\left(\theta_{UX_{1}}\left(Z^{0},S^{0}\right)\in\cdot\right)$. Entonces $\left(Z^{}*,S^{*}\right)$ es estacionario,
\begin{eqnarray*}
\esp\left[f\left(Z^{*},S^{*}\right)\right]=\esp\left[\int_{0}^{X_{1}}f\left(\theta_{s}\left(Z^{0},S^{0}\right)\right)ds\right]/\esp\left[X_{1}\right]
\end{eqnarray*}
$f\in\mathcal{H}\otimes\mathcal{L}^{+}$, and $S_{0}^{*}$ es continuo con funci\'on distribuci\'on $G_{\infty}$ definida por $$G_{\infty}\left(x\right):=\frac{\esp\left[X_{1}\right]\wedge x}{\esp\left[X_{1}\right]}$$ para $x\geq0$ y densidad $\prob\left[X_{1}>x\right]/\esp\left[X_{1}\right]$, con $x\geq0$.

\end{Teo}


\begin{Teo}
Sea $Z$ un Proceso Estoc\'astico un lado shift-medible \textit{one-sided shift-measurable stochastic process}, (PEOSSM),
y $S_{0}$ y $S_{1}$ tiempos aleatorios tales que $0\leq S_{0}<S_{1}$ y
\begin{equation}
\theta_{S_{1}}Z=\theta_{S_{0}}Z\textrm{ en distribuci\'on}.
\end{equation}

Entonces el espacio de probabilidad subyacente $\left(\Omega,\mathcal{F},\prob\right)$ puede extenderse para soportar una sucesi\'on de tiempos aleatorios $S$ tales que

\begin{eqnarray}
\theta_{S_{n}}\left(Z,S\right)=\left(Z^{0},S^{0}\right),n\geq0,\textrm{ en distribuci\'on},\\
\left(Z,S_{0},S_{1}\right)\textrm{ depende de }\left(X_{2},X_{3},\ldots\right)\textrm{ solamente a traves de }\theta_{S_{1}}Z.
\end{eqnarray}
\end{Teo}
%______________________________________________________________________


\subsection{Procesos Regenerativos}
%________________________________________________________________________

%________________________________________________________________________
\subsection{Procesos Regenerativos Sigman, Thorisson y Wolff \cite{Sigman2}}
%________________________________________________________________________


\begin{Def}[Definici\'on Cl\'asica]
Un proceso estoc\'astico $X=\left\{X\left(t\right):t\geq0\right\}$ es llamado regenerativo is existe una variable aleatoria $R_{1}>0$ tal que
\begin{itemize}
\item[i)] $\left\{X\left(t+R_{1}\right):t\geq0\right\}$ es independiente de $\left\{\left\{X\left(t\right):t<R_{1}\right\},\right\}$
\item[ii)] $\left\{X\left(t+R_{1}\right):t\geq0\right\}$ es estoc\'asticamente equivalente a $\left\{X\left(t\right):t>0\right\}$
\end{itemize}

Llamamos a $R_{1}$ tiempo de regeneraci\'on, y decimos que $X$ se regenera en este punto.
\end{Def}

$\left\{X\left(t+R_{1}\right)\right\}$ es regenerativo con tiempo de regeneraci\'on $R_{2}$, independiente de $R_{1}$ pero con la misma distribuci\'on que $R_{1}$. Procediendo de esta manera se obtiene una secuencia de variables aleatorias independientes e id\'enticamente distribuidas $\left\{R_{n}\right\}$ llamados longitudes de ciclo. Si definimos a $Z_{k}\equiv R_{1}+R_{2}+\cdots+R_{k}$, se tiene un proceso de renovaci\'on llamado proceso de renovaci\'on encajado para $X$.


\begin{Note}
La existencia de un primer tiempo de regeneraci\'on, $R_{1}$, implica la existencia de una sucesi\'on completa de estos tiempos $R_{1},R_{2}\ldots,$ que satisfacen la propiedad deseada \cite{Sigman2}.
\end{Note}


\begin{Note} Para la cola $GI/GI/1$ los usuarios arriban con tiempos $t_{n}$ y son atendidos con tiempos de servicio $S_{n}$, los tiempos de arribo forman un proceso de renovaci\'on  con tiempos entre arribos independientes e identicamente distribuidos (\texttt{i.i.d.})$T_{n}=t_{n}-t_{n-1}$, adem\'as los tiempos de servicio son \texttt{i.i.d.} e independientes de los procesos de arribo. Por \textit{estable} se entiende que $\esp S_{n}<\esp T_{n}<\infty$.
\end{Note}
 


\begin{Def}
Para $x$ fijo y para cada $t\geq0$, sea $I_{x}\left(t\right)=1$ si $X\left(t\right)\leq x$,  $I_{x}\left(t\right)=0$ en caso contrario, y def\'inanse los tiempos promedio
\begin{eqnarray*}
\overline{X}&=&lim_{t\rightarrow\infty}\frac{1}{t}\int_{0}^{\infty}X\left(u\right)du\\
\prob\left(X_{\infty}\leq x\right)&=&lim_{t\rightarrow\infty}\frac{1}{t}\int_{0}^{\infty}I_{x}\left(u\right)du,
\end{eqnarray*}
cuando estos l\'imites existan.
\end{Def}

Como consecuencia del teorema de Renovaci\'on-Recompensa, se tiene que el primer l\'imite  existe y es igual a la constante
\begin{eqnarray*}
\overline{X}&=&\frac{\esp\left[\int_{0}^{R_{1}}X\left(t\right)dt\right]}{\esp\left[R_{1}\right]},
\end{eqnarray*}
suponiendo que ambas esperanzas son finitas.
 
\begin{Note}
Funciones de procesos regenerativos son regenerativas, es decir, si $X\left(t\right)$ es regenerativo y se define el proceso $Y\left(t\right)$ por $Y\left(t\right)=f\left(X\left(t\right)\right)$ para alguna funci\'on Borel medible $f\left(\cdot\right)$. Adem\'as $Y$ es regenerativo con los mismos tiempos de renovaci\'on que $X$. 

En general, los tiempos de renovaci\'on, $Z_{k}$ de un proceso regenerativo no requieren ser tiempos de paro con respecto a la evoluci\'on de $X\left(t\right)$.
\end{Note} 

\begin{Note}
Una funci\'on de un proceso de Markov, usualmente no ser\'a un proceso de Markov, sin embargo ser\'a regenerativo si el proceso de Markov lo es.
\end{Note}

 
\begin{Note}
Un proceso regenerativo con media de la longitud de ciclo finita es llamado positivo recurrente.
\end{Note}


\begin{Note}
\begin{itemize}
\item[a)] Si el proceso regenerativo $X$ es positivo recurrente y tiene trayectorias muestrales no negativas, entonces la ecuaci\'on anterior es v\'alida.
\item[b)] Si $X$ es positivo recurrente regenerativo, podemos construir una \'unica versi\'on estacionaria de este proceso, $X_{e}=\left\{X_{e}\left(t\right)\right\}$, donde $X_{e}$ es un proceso estoc\'astico regenerativo y estrictamente estacionario, con distribuci\'on marginal distribuida como $X_{\infty}$
\end{itemize}
\end{Note}


%__________________________________________________________________________________________
\subsection{Procesos Regenerativos Estacionarios - Stidham \cite{Stidham}}
%__________________________________________________________________________________________


Un proceso estoc\'astico a tiempo continuo $\left\{V\left(t\right),t\geq0\right\}$ es un proceso regenerativo si existe una sucesi\'on de variables aleatorias independientes e id\'enticamente distribuidas $\left\{X_{1},X_{2},\ldots\right\}$, sucesi\'on de renovaci\'on, tal que para cualquier conjunto de Borel $A$, 

\begin{eqnarray*}
\prob\left\{V\left(t\right)\in A|X_{1}+X_{2}+\cdots+X_{R\left(t\right)}=s,\left\{V\left(\tau\right),\tau<s\right\}\right\}=\prob\left\{V\left(t-s\right)\in A|X_{1}>t-s\right\},
\end{eqnarray*}
para todo $0\leq s\leq t$, donde $R\left(t\right)=\max\left\{X_{1}+X_{2}+\cdots+X_{j}\leq t\right\}=$n\'umero de renovaciones ({\emph{puntos de regeneraci\'on}}) que ocurren en $\left[0,t\right]$. El intervalo $\left[0,X_{1}\right)$ es llamado {\emph{primer ciclo de regeneraci\'on}} de $\left\{V\left(t \right),t\geq0\right\}$, $\left[X_{1},X_{1}+X_{2}\right)$ el {\emph{segundo ciclo de regeneraci\'on}}, y as\'i sucesivamente.

Sea $X=X_{1}$ y sea $F$ la funci\'on de distrbuci\'on de $X$


\begin{Def}
Se define el proceso estacionario, $\left\{V^{*}\left(t\right),t\geq0\right\}$, para $\left\{V\left(t\right),t\geq0\right\}$ por

\begin{eqnarray*}
\prob\left\{V\left(t\right)\in A\right\}=\frac{1}{\esp\left[X\right]}\int_{0}^{\infty}\prob\left\{V\left(t+x\right)\in A|X>x\right\}\left(1-F\left(x\right)\right)dx,
\end{eqnarray*} 
para todo $t\geq0$ y todo conjunto de Borel $A$.
\end{Def}

\begin{Def}
Una distribuci\'on se dice que es {\emph{aritm\'etica}} si todos sus puntos de incremento son m\'ultiplos de la forma $0,\lambda, 2\lambda,\ldots$ para alguna $\lambda>0$ entera.
\end{Def}


\begin{Def}
Una modificaci\'on medible de un proceso $\left\{V\left(t\right),t\geq0\right\}$, es una versi\'on de este, $\left\{V\left(t,w\right)\right\}$ conjuntamente medible para $t\geq0$ y para $w\in S$, $S$ espacio de estados para $\left\{V\left(t\right),t\geq0\right\}$.
\end{Def}

\begin{Teo}
Sea $\left\{V\left(t\right),t\geq\right\}$ un proceso regenerativo no negativo con modificaci\'on medible. Sea $\esp\left[X\right]<\infty$. Entonces el proceso estacionario dado por la ecuaci\'on anterior est\'a bien definido y tiene funci\'on de distribuci\'on independiente de $t$, adem\'as
\begin{itemize}
\item[i)] \begin{eqnarray*}
\esp\left[V^{*}\left(0\right)\right]&=&\frac{\esp\left[\int_{0}^{X}V\left(s\right)ds\right]}{\esp\left[X\right]}\end{eqnarray*}
\item[ii)] Si $\esp\left[V^{*}\left(0\right)\right]<\infty$, equivalentemente, si $\esp\left[\int_{0}^{X}V\left(s\right)ds\right]<\infty$,entonces
\begin{eqnarray*}
\frac{\int_{0}^{t}V\left(s\right)ds}{t}\rightarrow\frac{\esp\left[\int_{0}^{X}V\left(s\right)ds\right]}{\esp\left[X\right]}
\end{eqnarray*}
con probabilidad 1 y en media, cuando $t\rightarrow\infty$.
\end{itemize}
\end{Teo}

\begin{Coro}
Sea $\left\{V\left(t\right),t\geq0\right\}$ un proceso regenerativo no negativo, con modificaci\'on medible. Si $\esp <\infty$, $F$ es no-aritm\'etica, y para todo $x\geq0$, $P\left\{V\left(t\right)\leq x,C>x\right\}$ es de variaci\'on acotada como funci\'on de $t$ en cada intervalo finito $\left[0,\tau\right]$, entonces $V\left(t\right)$ converge en distribuci\'on  cuando $t\rightarrow\infty$ y $$\esp V=\frac{\esp \int_{0}^{X}V\left(s\right)ds}{\esp X}$$
Donde $V$ tiene la distribuci\'on l\'imite de $V\left(t\right)$ cuando $t\rightarrow\infty$.

\end{Coro}

Para el caso discreto se tienen resultados similares.



%______________________________________________________________________
\subsection{Procesos de Renovaci\'on}
%______________________________________________________________________

\begin{Def}%\label{Def.Tn}
Sean $0\leq T_{1}\leq T_{2}\leq \ldots$ son tiempos aleatorios infinitos en los cuales ocurren ciertos eventos. El n\'umero de tiempos $T_{n}$ en el intervalo $\left[0,t\right)$ es

\begin{eqnarray}
N\left(t\right)=\sum_{n=1}^{\infty}\indora\left(T_{n}\leq t\right),
\end{eqnarray}
para $t\geq0$.
\end{Def}

Si se consideran los puntos $T_{n}$ como elementos de $\rea_{+}$, y $N\left(t\right)$ es el n\'umero de puntos en $\rea$. El proceso denotado por $\left\{N\left(t\right):t\geq0\right\}$, denotado por $N\left(t\right)$, es un proceso puntual en $\rea_{+}$. Los $T_{n}$ son los tiempos de ocurrencia, el proceso puntual $N\left(t\right)$ es simple si su n\'umero de ocurrencias son distintas: $0<T_{1}<T_{2}<\ldots$ casi seguramente.

\begin{Def}
Un proceso puntual $N\left(t\right)$ es un proceso de renovaci\'on si los tiempos de interocurrencia $\xi_{n}=T_{n}-T_{n-1}$, para $n\geq1$, son independientes e identicamente distribuidos con distribuci\'on $F$, donde $F\left(0\right)=0$ y $T_{0}=0$. Los $T_{n}$ son llamados tiempos de renovaci\'on, referente a la independencia o renovaci\'on de la informaci\'on estoc\'astica en estos tiempos. Los $\xi_{n}$ son los tiempos de inter-renovaci\'on, y $N\left(t\right)$ es el n\'umero de renovaciones en el intervalo $\left[0,t\right)$
\end{Def}


\begin{Note}
Para definir un proceso de renovaci\'on para cualquier contexto, solamente hay que especificar una distribuci\'on $F$, con $F\left(0\right)=0$, para los tiempos de inter-renovaci\'on. La funci\'on $F$ en turno degune las otra variables aleatorias. De manera formal, existe un espacio de probabilidad y una sucesi\'on de variables aleatorias $\xi_{1},\xi_{2},\ldots$ definidas en este con distribuci\'on $F$. Entonces las otras cantidades son $T_{n}=\sum_{k=1}^{n}\xi_{k}$ y $N\left(t\right)=\sum_{n=1}^{\infty}\indora\left(T_{n}\leq t\right)$, donde $T_{n}\rightarrow\infty$ casi seguramente por la Ley Fuerte de los Grandes Números.
\end{Note}

%___________________________________________________________________________________________
%
\subsection{Teorema Principal de Renovaci\'on}
%___________________________________________________________________________________________
%

\begin{Note} Una funci\'on $h:\rea_{+}\rightarrow\rea$ es Directamente Riemann Integrable en los siguientes casos:
\begin{itemize}
\item[a)] $h\left(t\right)\geq0$ es decreciente y Riemann Integrable.
\item[b)] $h$ es continua excepto posiblemente en un conjunto de Lebesgue de medida 0, y $|h\left(t\right)|\leq b\left(t\right)$, donde $b$ es DRI.
\end{itemize}
\end{Note}

\begin{Teo}[Teorema Principal de Renovaci\'on]
Si $F$ es no aritm\'etica y $h\left(t\right)$ es Directamente Riemann Integrable (DRI), entonces

\begin{eqnarray*}
lim_{t\rightarrow\infty}U\star h=\frac{1}{\mu}\int_{\rea_{+}}h\left(s\right)ds.
\end{eqnarray*}
\end{Teo}

\begin{Prop}
Cualquier funci\'on $H\left(t\right)$ acotada en intervalos finitos y que es 0 para $t<0$ puede expresarse como
\begin{eqnarray*}
H\left(t\right)=U\star h\left(t\right)\textrm{,  donde }h\left(t\right)=H\left(t\right)-F\star H\left(t\right)
\end{eqnarray*}
\end{Prop}

\begin{Def}
Un proceso estoc\'astico $X\left(t\right)$ es crudamente regenerativo en un tiempo aleatorio positivo $T$ si
\begin{eqnarray*}
\esp\left[X\left(T+t\right)|T\right]=\esp\left[X\left(t\right)\right]\textrm{, para }t\geq0,\end{eqnarray*}
y con las esperanzas anteriores finitas.
\end{Def}

\begin{Prop}
Sup\'ongase que $X\left(t\right)$ es un proceso crudamente regenerativo en $T$, que tiene distribuci\'on $F$. Si $\esp\left[X\left(t\right)\right]$ es acotado en intervalos finitos, entonces
\begin{eqnarray*}
\esp\left[X\left(t\right)\right]=U\star h\left(t\right)\textrm{,  donde }h\left(t\right)=\esp\left[X\left(t\right)\indora\left(T>t\right)\right].
\end{eqnarray*}
\end{Prop}

\begin{Teo}[Regeneraci\'on Cruda]
Sup\'ongase que $X\left(t\right)$ es un proceso con valores positivo crudamente regenerativo en $T$, y def\'inase $M=\sup\left\{|X\left(t\right)|:t\leq T\right\}$. Si $T$ es no aritm\'etico y $M$ y $MT$ tienen media finita, entonces
\begin{eqnarray*}
lim_{t\rightarrow\infty}\esp\left[X\left(t\right)\right]=\frac{1}{\mu}\int_{\rea_{+}}h\left(s\right)ds,
\end{eqnarray*}
donde $h\left(t\right)=\esp\left[X\left(t\right)\indora\left(T>t\right)\right]$.
\end{Teo}

%___________________________________________________________________________________________
%
\subsection{Propiedades de los Procesos de Renovaci\'on}
%___________________________________________________________________________________________
%

Los tiempos $T_{n}$ est\'an relacionados con los conteos de $N\left(t\right)$ por

\begin{eqnarray*}
\left\{N\left(t\right)\geq n\right\}&=&\left\{T_{n}\leq t\right\}\\
T_{N\left(t\right)}\leq &t&<T_{N\left(t\right)+1},
\end{eqnarray*}

adem\'as $N\left(T_{n}\right)=n$, y 

\begin{eqnarray*}
N\left(t\right)=\max\left\{n:T_{n}\leq t\right\}=\min\left\{n:T_{n+1}>t\right\}
\end{eqnarray*}

Por propiedades de la convoluci\'on se sabe que

\begin{eqnarray*}
P\left\{T_{n}\leq t\right\}=F^{n\star}\left(t\right)
\end{eqnarray*}
que es la $n$-\'esima convoluci\'on de $F$. Entonces 

\begin{eqnarray*}
\left\{N\left(t\right)\geq n\right\}&=&\left\{T_{n}\leq t\right\}\\
P\left\{N\left(t\right)\leq n\right\}&=&1-F^{\left(n+1\right)\star}\left(t\right)
\end{eqnarray*}

Adem\'as usando el hecho de que $\esp\left[N\left(t\right)\right]=\sum_{n=1}^{\infty}P\left\{N\left(t\right)\geq n\right\}$
se tiene que

\begin{eqnarray*}
\esp\left[N\left(t\right)\right]=\sum_{n=1}^{\infty}F^{n\star}\left(t\right)
\end{eqnarray*}

\begin{Prop}
Para cada $t\geq0$, la funci\'on generadora de momentos $\esp\left[e^{\alpha N\left(t\right)}\right]$ existe para alguna $\alpha$ en una vecindad del 0, y de aqu\'i que $\esp\left[N\left(t\right)^{m}\right]<\infty$, para $m\geq1$.
\end{Prop}


\begin{Note}
Si el primer tiempo de renovaci\'on $\xi_{1}$ no tiene la misma distribuci\'on que el resto de las $\xi_{n}$, para $n\geq2$, a $N\left(t\right)$ se le llama Proceso de Renovaci\'on retardado, donde si $\xi$ tiene distribuci\'on $G$, entonces el tiempo $T_{n}$ de la $n$-\'esima renovaci\'on tiene distribuci\'on $G\star F^{\left(n-1\right)\star}\left(t\right)$
\end{Note}


\begin{Teo}
Para una constante $\mu\leq\infty$ ( o variable aleatoria), las siguientes expresiones son equivalentes:

\begin{eqnarray}
lim_{n\rightarrow\infty}n^{-1}T_{n}&=&\mu,\textrm{ c.s.}\\
lim_{t\rightarrow\infty}t^{-1}N\left(t\right)&=&1/\mu,\textrm{ c.s.}
\end{eqnarray}
\end{Teo}


Es decir, $T_{n}$ satisface la Ley Fuerte de los Grandes N\'umeros s\'i y s\'olo s\'i $N\left/t\right)$ la cumple.


\begin{Coro}[Ley Fuerte de los Grandes N\'umeros para Procesos de Renovaci\'on]
Si $N\left(t\right)$ es un proceso de renovaci\'on cuyos tiempos de inter-renovaci\'on tienen media $\mu\leq\infty$, entonces
\begin{eqnarray}
t^{-1}N\left(t\right)\rightarrow 1/\mu,\textrm{ c.s. cuando }t\rightarrow\infty.
\end{eqnarray}

\end{Coro}


Considerar el proceso estoc\'astico de valores reales $\left\{Z\left(t\right):t\geq0\right\}$ en el mismo espacio de probabilidad que $N\left(t\right)$

\begin{Def}
Para el proceso $\left\{Z\left(t\right):t\geq0\right\}$ se define la fluctuaci\'on m\'axima de $Z\left(t\right)$ en el intervalo $\left(T_{n-1},T_{n}\right]$:
\begin{eqnarray*}
M_{n}=\sup_{T_{n-1}<t\leq T_{n}}|Z\left(t\right)-Z\left(T_{n-1}\right)|
\end{eqnarray*}
\end{Def}

\begin{Teo}
Sup\'ongase que $n^{-1}T_{n}\rightarrow\mu$ c.s. cuando $n\rightarrow\infty$, donde $\mu\leq\infty$ es una constante o variable aleatoria. Sea $a$ una constante o variable aleatoria que puede ser infinita cuando $\mu$ es finita, y considere las expresiones l\'imite:
\begin{eqnarray}
lim_{n\rightarrow\infty}n^{-1}Z\left(T_{n}\right)&=&a,\textrm{ c.s.}\\
lim_{t\rightarrow\infty}t^{-1}Z\left(t\right)&=&a/\mu,\textrm{ c.s.}
\end{eqnarray}
La segunda expresi\'on implica la primera. Conversamente, la primera implica la segunda si el proceso $Z\left(t\right)$ es creciente, o si $lim_{n\rightarrow\infty}n^{-1}M_{n}=0$ c.s.
\end{Teo}

\begin{Coro}
Si $N\left(t\right)$ es un proceso de renovaci\'on, y $\left(Z\left(T_{n}\right)-Z\left(T_{n-1}\right),M_{n}\right)$, para $n\geq1$, son variables aleatorias independientes e id\'enticamente distribuidas con media finita, entonces,
\begin{eqnarray}
lim_{t\rightarrow\infty}t^{-1}Z\left(t\right)\rightarrow\frac{\esp\left[Z\left(T_{1}\right)-Z\left(T_{0}\right)\right]}{\esp\left[T_{1}\right]},\textrm{ c.s. cuando  }t\rightarrow\infty.
\end{eqnarray}
\end{Coro}



%___________________________________________________________________________________________
%
\subsection{Propiedades de los Procesos de Renovaci\'on}
%___________________________________________________________________________________________
%

Los tiempos $T_{n}$ est\'an relacionados con los conteos de $N\left(t\right)$ por

\begin{eqnarray*}
\left\{N\left(t\right)\geq n\right\}&=&\left\{T_{n}\leq t\right\}\\
T_{N\left(t\right)}\leq &t&<T_{N\left(t\right)+1},
\end{eqnarray*}

adem\'as $N\left(T_{n}\right)=n$, y 

\begin{eqnarray*}
N\left(t\right)=\max\left\{n:T_{n}\leq t\right\}=\min\left\{n:T_{n+1}>t\right\}
\end{eqnarray*}

Por propiedades de la convoluci\'on se sabe que

\begin{eqnarray*}
P\left\{T_{n}\leq t\right\}=F^{n\star}\left(t\right)
\end{eqnarray*}
que es la $n$-\'esima convoluci\'on de $F$. Entonces 

\begin{eqnarray*}
\left\{N\left(t\right)\geq n\right\}&=&\left\{T_{n}\leq t\right\}\\
P\left\{N\left(t\right)\leq n\right\}&=&1-F^{\left(n+1\right)\star}\left(t\right)
\end{eqnarray*}

Adem\'as usando el hecho de que $\esp\left[N\left(t\right)\right]=\sum_{n=1}^{\infty}P\left\{N\left(t\right)\geq n\right\}$
se tiene que

\begin{eqnarray*}
\esp\left[N\left(t\right)\right]=\sum_{n=1}^{\infty}F^{n\star}\left(t\right)
\end{eqnarray*}

\begin{Prop}
Para cada $t\geq0$, la funci\'on generadora de momentos $\esp\left[e^{\alpha N\left(t\right)}\right]$ existe para alguna $\alpha$ en una vecindad del 0, y de aqu\'i que $\esp\left[N\left(t\right)^{m}\right]<\infty$, para $m\geq1$.
\end{Prop}


\begin{Note}
Si el primer tiempo de renovaci\'on $\xi_{1}$ no tiene la misma distribuci\'on que el resto de las $\xi_{n}$, para $n\geq2$, a $N\left(t\right)$ se le llama Proceso de Renovaci\'on retardado, donde si $\xi$ tiene distribuci\'on $G$, entonces el tiempo $T_{n}$ de la $n$-\'esima renovaci\'on tiene distribuci\'on $G\star F^{\left(n-1\right)\star}\left(t\right)$
\end{Note}


\begin{Teo}
Para una constante $\mu\leq\infty$ ( o variable aleatoria), las siguientes expresiones son equivalentes:

\begin{eqnarray}
lim_{n\rightarrow\infty}n^{-1}T_{n}&=&\mu,\textrm{ c.s.}\\
lim_{t\rightarrow\infty}t^{-1}N\left(t\right)&=&1/\mu,\textrm{ c.s.}
\end{eqnarray}
\end{Teo}


Es decir, $T_{n}$ satisface la Ley Fuerte de los Grandes N\'umeros s\'i y s\'olo s\'i $N\left/t\right)$ la cumple.


\begin{Coro}[Ley Fuerte de los Grandes N\'umeros para Procesos de Renovaci\'on]
Si $N\left(t\right)$ es un proceso de renovaci\'on cuyos tiempos de inter-renovaci\'on tienen media $\mu\leq\infty$, entonces
\begin{eqnarray}
t^{-1}N\left(t\right)\rightarrow 1/\mu,\textrm{ c.s. cuando }t\rightarrow\infty.
\end{eqnarray}

\end{Coro}


Considerar el proceso estoc\'astico de valores reales $\left\{Z\left(t\right):t\geq0\right\}$ en el mismo espacio de probabilidad que $N\left(t\right)$

\begin{Def}
Para el proceso $\left\{Z\left(t\right):t\geq0\right\}$ se define la fluctuaci\'on m\'axima de $Z\left(t\right)$ en el intervalo $\left(T_{n-1},T_{n}\right]$:
\begin{eqnarray*}
M_{n}=\sup_{T_{n-1}<t\leq T_{n}}|Z\left(t\right)-Z\left(T_{n-1}\right)|
\end{eqnarray*}
\end{Def}

\begin{Teo}
Sup\'ongase que $n^{-1}T_{n}\rightarrow\mu$ c.s. cuando $n\rightarrow\infty$, donde $\mu\leq\infty$ es una constante o variable aleatoria. Sea $a$ una constante o variable aleatoria que puede ser infinita cuando $\mu$ es finita, y considere las expresiones l\'imite:
\begin{eqnarray}
lim_{n\rightarrow\infty}n^{-1}Z\left(T_{n}\right)&=&a,\textrm{ c.s.}\\
lim_{t\rightarrow\infty}t^{-1}Z\left(t\right)&=&a/\mu,\textrm{ c.s.}
\end{eqnarray}
La segunda expresi\'on implica la primera. Conversamente, la primera implica la segunda si el proceso $Z\left(t\right)$ es creciente, o si $lim_{n\rightarrow\infty}n^{-1}M_{n}=0$ c.s.
\end{Teo}

\begin{Coro}
Si $N\left(t\right)$ es un proceso de renovaci\'on, y $\left(Z\left(T_{n}\right)-Z\left(T_{n-1}\right),M_{n}\right)$, para $n\geq1$, son variables aleatorias independientes e id\'enticamente distribuidas con media finita, entonces,
\begin{eqnarray}
lim_{t\rightarrow\infty}t^{-1}Z\left(t\right)\rightarrow\frac{\esp\left[Z\left(T_{1}\right)-Z\left(T_{0}\right)\right]}{\esp\left[T_{1}\right]},\textrm{ c.s. cuando  }t\rightarrow\infty.
\end{eqnarray}
\end{Coro}


%___________________________________________________________________________________________
%
\subsection{Propiedades de los Procesos de Renovaci\'on}
%___________________________________________________________________________________________
%

Los tiempos $T_{n}$ est\'an relacionados con los conteos de $N\left(t\right)$ por

\begin{eqnarray*}
\left\{N\left(t\right)\geq n\right\}&=&\left\{T_{n}\leq t\right\}\\
T_{N\left(t\right)}\leq &t&<T_{N\left(t\right)+1},
\end{eqnarray*}

adem\'as $N\left(T_{n}\right)=n$, y 

\begin{eqnarray*}
N\left(t\right)=\max\left\{n:T_{n}\leq t\right\}=\min\left\{n:T_{n+1}>t\right\}
\end{eqnarray*}

Por propiedades de la convoluci\'on se sabe que

\begin{eqnarray*}
P\left\{T_{n}\leq t\right\}=F^{n\star}\left(t\right)
\end{eqnarray*}
que es la $n$-\'esima convoluci\'on de $F$. Entonces 

\begin{eqnarray*}
\left\{N\left(t\right)\geq n\right\}&=&\left\{T_{n}\leq t\right\}\\
P\left\{N\left(t\right)\leq n\right\}&=&1-F^{\left(n+1\right)\star}\left(t\right)
\end{eqnarray*}

Adem\'as usando el hecho de que $\esp\left[N\left(t\right)\right]=\sum_{n=1}^{\infty}P\left\{N\left(t\right)\geq n\right\}$
se tiene que

\begin{eqnarray*}
\esp\left[N\left(t\right)\right]=\sum_{n=1}^{\infty}F^{n\star}\left(t\right)
\end{eqnarray*}

\begin{Prop}
Para cada $t\geq0$, la funci\'on generadora de momentos $\esp\left[e^{\alpha N\left(t\right)}\right]$ existe para alguna $\alpha$ en una vecindad del 0, y de aqu\'i que $\esp\left[N\left(t\right)^{m}\right]<\infty$, para $m\geq1$.
\end{Prop}


\begin{Note}
Si el primer tiempo de renovaci\'on $\xi_{1}$ no tiene la misma distribuci\'on que el resto de las $\xi_{n}$, para $n\geq2$, a $N\left(t\right)$ se le llama Proceso de Renovaci\'on retardado, donde si $\xi$ tiene distribuci\'on $G$, entonces el tiempo $T_{n}$ de la $n$-\'esima renovaci\'on tiene distribuci\'on $G\star F^{\left(n-1\right)\star}\left(t\right)$
\end{Note}


\begin{Teo}
Para una constante $\mu\leq\infty$ ( o variable aleatoria), las siguientes expresiones son equivalentes:

\begin{eqnarray}
lim_{n\rightarrow\infty}n^{-1}T_{n}&=&\mu,\textrm{ c.s.}\\
lim_{t\rightarrow\infty}t^{-1}N\left(t\right)&=&1/\mu,\textrm{ c.s.}
\end{eqnarray}
\end{Teo}


Es decir, $T_{n}$ satisface la Ley Fuerte de los Grandes N\'umeros s\'i y s\'olo s\'i $N\left/t\right)$ la cumple.


\begin{Coro}[Ley Fuerte de los Grandes N\'umeros para Procesos de Renovaci\'on]
Si $N\left(t\right)$ es un proceso de renovaci\'on cuyos tiempos de inter-renovaci\'on tienen media $\mu\leq\infty$, entonces
\begin{eqnarray}
t^{-1}N\left(t\right)\rightarrow 1/\mu,\textrm{ c.s. cuando }t\rightarrow\infty.
\end{eqnarray}

\end{Coro}


Considerar el proceso estoc\'astico de valores reales $\left\{Z\left(t\right):t\geq0\right\}$ en el mismo espacio de probabilidad que $N\left(t\right)$

\begin{Def}
Para el proceso $\left\{Z\left(t\right):t\geq0\right\}$ se define la fluctuaci\'on m\'axima de $Z\left(t\right)$ en el intervalo $\left(T_{n-1},T_{n}\right]$:
\begin{eqnarray*}
M_{n}=\sup_{T_{n-1}<t\leq T_{n}}|Z\left(t\right)-Z\left(T_{n-1}\right)|
\end{eqnarray*}
\end{Def}

\begin{Teo}
Sup\'ongase que $n^{-1}T_{n}\rightarrow\mu$ c.s. cuando $n\rightarrow\infty$, donde $\mu\leq\infty$ es una constante o variable aleatoria. Sea $a$ una constante o variable aleatoria que puede ser infinita cuando $\mu$ es finita, y considere las expresiones l\'imite:
\begin{eqnarray}
lim_{n\rightarrow\infty}n^{-1}Z\left(T_{n}\right)&=&a,\textrm{ c.s.}\\
lim_{t\rightarrow\infty}t^{-1}Z\left(t\right)&=&a/\mu,\textrm{ c.s.}
\end{eqnarray}
La segunda expresi\'on implica la primera. Conversamente, la primera implica la segunda si el proceso $Z\left(t\right)$ es creciente, o si $lim_{n\rightarrow\infty}n^{-1}M_{n}=0$ c.s.
\end{Teo}

\begin{Coro}
Si $N\left(t\right)$ es un proceso de renovaci\'on, y $\left(Z\left(T_{n}\right)-Z\left(T_{n-1}\right),M_{n}\right)$, para $n\geq1$, son variables aleatorias independientes e id\'enticamente distribuidas con media finita, entonces,
\begin{eqnarray}
lim_{t\rightarrow\infty}t^{-1}Z\left(t\right)\rightarrow\frac{\esp\left[Z\left(T_{1}\right)-Z\left(T_{0}\right)\right]}{\esp\left[T_{1}\right]},\textrm{ c.s. cuando  }t\rightarrow\infty.
\end{eqnarray}
\end{Coro}

%___________________________________________________________________________________________
%
\subsection{Propiedades de los Procesos de Renovaci\'on}
%___________________________________________________________________________________________
%

Los tiempos $T_{n}$ est\'an relacionados con los conteos de $N\left(t\right)$ por

\begin{eqnarray*}
\left\{N\left(t\right)\geq n\right\}&=&\left\{T_{n}\leq t\right\}\\
T_{N\left(t\right)}\leq &t&<T_{N\left(t\right)+1},
\end{eqnarray*}

adem\'as $N\left(T_{n}\right)=n$, y 

\begin{eqnarray*}
N\left(t\right)=\max\left\{n:T_{n}\leq t\right\}=\min\left\{n:T_{n+1}>t\right\}
\end{eqnarray*}

Por propiedades de la convoluci\'on se sabe que

\begin{eqnarray*}
P\left\{T_{n}\leq t\right\}=F^{n\star}\left(t\right)
\end{eqnarray*}
que es la $n$-\'esima convoluci\'on de $F$. Entonces 

\begin{eqnarray*}
\left\{N\left(t\right)\geq n\right\}&=&\left\{T_{n}\leq t\right\}\\
P\left\{N\left(t\right)\leq n\right\}&=&1-F^{\left(n+1\right)\star}\left(t\right)
\end{eqnarray*}

Adem\'as usando el hecho de que $\esp\left[N\left(t\right)\right]=\sum_{n=1}^{\infty}P\left\{N\left(t\right)\geq n\right\}$
se tiene que

\begin{eqnarray*}
\esp\left[N\left(t\right)\right]=\sum_{n=1}^{\infty}F^{n\star}\left(t\right)
\end{eqnarray*}

\begin{Prop}
Para cada $t\geq0$, la funci\'on generadora de momentos $\esp\left[e^{\alpha N\left(t\right)}\right]$ existe para alguna $\alpha$ en una vecindad del 0, y de aqu\'i que $\esp\left[N\left(t\right)^{m}\right]<\infty$, para $m\geq1$.
\end{Prop}


\begin{Note}
Si el primer tiempo de renovaci\'on $\xi_{1}$ no tiene la misma distribuci\'on que el resto de las $\xi_{n}$, para $n\geq2$, a $N\left(t\right)$ se le llama Proceso de Renovaci\'on retardado, donde si $\xi$ tiene distribuci\'on $G$, entonces el tiempo $T_{n}$ de la $n$-\'esima renovaci\'on tiene distribuci\'on $G\star F^{\left(n-1\right)\star}\left(t\right)$
\end{Note}


\begin{Teo}
Para una constante $\mu\leq\infty$ ( o variable aleatoria), las siguientes expresiones son equivalentes:

\begin{eqnarray}
lim_{n\rightarrow\infty}n^{-1}T_{n}&=&\mu,\textrm{ c.s.}\\
lim_{t\rightarrow\infty}t^{-1}N\left(t\right)&=&1/\mu,\textrm{ c.s.}
\end{eqnarray}
\end{Teo}


Es decir, $T_{n}$ satisface la Ley Fuerte de los Grandes N\'umeros s\'i y s\'olo s\'i $N\left/t\right)$ la cumple.


\begin{Coro}[Ley Fuerte de los Grandes N\'umeros para Procesos de Renovaci\'on]
Si $N\left(t\right)$ es un proceso de renovaci\'on cuyos tiempos de inter-renovaci\'on tienen media $\mu\leq\infty$, entonces
\begin{eqnarray}
t^{-1}N\left(t\right)\rightarrow 1/\mu,\textrm{ c.s. cuando }t\rightarrow\infty.
\end{eqnarray}

\end{Coro}


Considerar el proceso estoc\'astico de valores reales $\left\{Z\left(t\right):t\geq0\right\}$ en el mismo espacio de probabilidad que $N\left(t\right)$

\begin{Def}
Para el proceso $\left\{Z\left(t\right):t\geq0\right\}$ se define la fluctuaci\'on m\'axima de $Z\left(t\right)$ en el intervalo $\left(T_{n-1},T_{n}\right]$:
\begin{eqnarray*}
M_{n}=\sup_{T_{n-1}<t\leq T_{n}}|Z\left(t\right)-Z\left(T_{n-1}\right)|
\end{eqnarray*}
\end{Def}

\begin{Teo}
Sup\'ongase que $n^{-1}T_{n}\rightarrow\mu$ c.s. cuando $n\rightarrow\infty$, donde $\mu\leq\infty$ es una constante o variable aleatoria. Sea $a$ una constante o variable aleatoria que puede ser infinita cuando $\mu$ es finita, y considere las expresiones l\'imite:
\begin{eqnarray}
lim_{n\rightarrow\infty}n^{-1}Z\left(T_{n}\right)&=&a,\textrm{ c.s.}\\
lim_{t\rightarrow\infty}t^{-1}Z\left(t\right)&=&a/\mu,\textrm{ c.s.}
\end{eqnarray}
La segunda expresi\'on implica la primera. Conversamente, la primera implica la segunda si el proceso $Z\left(t\right)$ es creciente, o si $lim_{n\rightarrow\infty}n^{-1}M_{n}=0$ c.s.
\end{Teo}

\begin{Coro}
Si $N\left(t\right)$ es un proceso de renovaci\'on, y $\left(Z\left(T_{n}\right)-Z\left(T_{n-1}\right),M_{n}\right)$, para $n\geq1$, son variables aleatorias independientes e id\'enticamente distribuidas con media finita, entonces,
\begin{eqnarray}
lim_{t\rightarrow\infty}t^{-1}Z\left(t\right)\rightarrow\frac{\esp\left[Z\left(T_{1}\right)-Z\left(T_{0}\right)\right]}{\esp\left[T_{1}\right]},\textrm{ c.s. cuando  }t\rightarrow\infty.
\end{eqnarray}
\end{Coro}
%___________________________________________________________________________________________
%
\subsection{Propiedades de los Procesos de Renovaci\'on}
%___________________________________________________________________________________________
%

Los tiempos $T_{n}$ est\'an relacionados con los conteos de $N\left(t\right)$ por

\begin{eqnarray*}
\left\{N\left(t\right)\geq n\right\}&=&\left\{T_{n}\leq t\right\}\\
T_{N\left(t\right)}\leq &t&<T_{N\left(t\right)+1},
\end{eqnarray*}

adem\'as $N\left(T_{n}\right)=n$, y 

\begin{eqnarray*}
N\left(t\right)=\max\left\{n:T_{n}\leq t\right\}=\min\left\{n:T_{n+1}>t\right\}
\end{eqnarray*}

Por propiedades de la convoluci\'on se sabe que

\begin{eqnarray*}
P\left\{T_{n}\leq t\right\}=F^{n\star}\left(t\right)
\end{eqnarray*}
que es la $n$-\'esima convoluci\'on de $F$. Entonces 

\begin{eqnarray*}
\left\{N\left(t\right)\geq n\right\}&=&\left\{T_{n}\leq t\right\}\\
P\left\{N\left(t\right)\leq n\right\}&=&1-F^{\left(n+1\right)\star}\left(t\right)
\end{eqnarray*}

Adem\'as usando el hecho de que $\esp\left[N\left(t\right)\right]=\sum_{n=1}^{\infty}P\left\{N\left(t\right)\geq n\right\}$
se tiene que

\begin{eqnarray*}
\esp\left[N\left(t\right)\right]=\sum_{n=1}^{\infty}F^{n\star}\left(t\right)
\end{eqnarray*}

\begin{Prop}
Para cada $t\geq0$, la funci\'on generadora de momentos $\esp\left[e^{\alpha N\left(t\right)}\right]$ existe para alguna $\alpha$ en una vecindad del 0, y de aqu\'i que $\esp\left[N\left(t\right)^{m}\right]<\infty$, para $m\geq1$.
\end{Prop}


\begin{Note}
Si el primer tiempo de renovaci\'on $\xi_{1}$ no tiene la misma distribuci\'on que el resto de las $\xi_{n}$, para $n\geq2$, a $N\left(t\right)$ se le llama Proceso de Renovaci\'on retardado, donde si $\xi$ tiene distribuci\'on $G$, entonces el tiempo $T_{n}$ de la $n$-\'esima renovaci\'on tiene distribuci\'on $G\star F^{\left(n-1\right)\star}\left(t\right)$
\end{Note}


\begin{Teo}
Para una constante $\mu\leq\infty$ ( o variable aleatoria), las siguientes expresiones son equivalentes:

\begin{eqnarray}
lim_{n\rightarrow\infty}n^{-1}T_{n}&=&\mu,\textrm{ c.s.}\\
lim_{t\rightarrow\infty}t^{-1}N\left(t\right)&=&1/\mu,\textrm{ c.s.}
\end{eqnarray}
\end{Teo}


Es decir, $T_{n}$ satisface la Ley Fuerte de los Grandes N\'umeros s\'i y s\'olo s\'i $N\left/t\right)$ la cumple.


\begin{Coro}[Ley Fuerte de los Grandes N\'umeros para Procesos de Renovaci\'on]
Si $N\left(t\right)$ es un proceso de renovaci\'on cuyos tiempos de inter-renovaci\'on tienen media $\mu\leq\infty$, entonces
\begin{eqnarray}
t^{-1}N\left(t\right)\rightarrow 1/\mu,\textrm{ c.s. cuando }t\rightarrow\infty.
\end{eqnarray}

\end{Coro}


Considerar el proceso estoc\'astico de valores reales $\left\{Z\left(t\right):t\geq0\right\}$ en el mismo espacio de probabilidad que $N\left(t\right)$

\begin{Def}
Para el proceso $\left\{Z\left(t\right):t\geq0\right\}$ se define la fluctuaci\'on m\'axima de $Z\left(t\right)$ en el intervalo $\left(T_{n-1},T_{n}\right]$:
\begin{eqnarray*}
M_{n}=\sup_{T_{n-1}<t\leq T_{n}}|Z\left(t\right)-Z\left(T_{n-1}\right)|
\end{eqnarray*}
\end{Def}

\begin{Teo}
Sup\'ongase que $n^{-1}T_{n}\rightarrow\mu$ c.s. cuando $n\rightarrow\infty$, donde $\mu\leq\infty$ es una constante o variable aleatoria. Sea $a$ una constante o variable aleatoria que puede ser infinita cuando $\mu$ es finita, y considere las expresiones l\'imite:
\begin{eqnarray}
lim_{n\rightarrow\infty}n^{-1}Z\left(T_{n}\right)&=&a,\textrm{ c.s.}\\
lim_{t\rightarrow\infty}t^{-1}Z\left(t\right)&=&a/\mu,\textrm{ c.s.}
\end{eqnarray}
La segunda expresi\'on implica la primera. Conversamente, la primera implica la segunda si el proceso $Z\left(t\right)$ es creciente, o si $lim_{n\rightarrow\infty}n^{-1}M_{n}=0$ c.s.
\end{Teo}

\begin{Coro}
Si $N\left(t\right)$ es un proceso de renovaci\'on, y $\left(Z\left(T_{n}\right)-Z\left(T_{n-1}\right),M_{n}\right)$, para $n\geq1$, son variables aleatorias independientes e id\'enticamente distribuidas con media finita, entonces,
\begin{eqnarray}
lim_{t\rightarrow\infty}t^{-1}Z\left(t\right)\rightarrow\frac{\esp\left[Z\left(T_{1}\right)-Z\left(T_{0}\right)\right]}{\esp\left[T_{1}\right]},\textrm{ c.s. cuando  }t\rightarrow\infty.
\end{eqnarray}
\end{Coro}


%___________________________________________________________________________________________
%
\subsection{Funci\'on de Renovaci\'on}
%___________________________________________________________________________________________
%


\begin{Def}
Sea $h\left(t\right)$ funci\'on de valores reales en $\rea$ acotada en intervalos finitos e igual a cero para $t<0$ La ecuaci\'on de renovaci\'on para $h\left(t\right)$ y la distribuci\'on $F$ es

\begin{eqnarray}%\label{Ec.Renovacion}
H\left(t\right)=h\left(t\right)+\int_{\left[0,t\right]}H\left(t-s\right)dF\left(s\right)\textrm{,    }t\geq0,
\end{eqnarray}
donde $H\left(t\right)$ es una funci\'on de valores reales. Esto es $H=h+F\star H$. Decimos que $H\left(t\right)$ es soluci\'on de esta ecuaci\'on si satisface la ecuaci\'on, y es acotada en intervalos finitos e iguales a cero para $t<0$.
\end{Def}

\begin{Prop}
La funci\'on $U\star h\left(t\right)$ es la \'unica soluci\'on de la ecuaci\'on de renovaci\'on (\ref{Ec.Renovacion}).
\end{Prop}

\begin{Teo}[Teorema Renovaci\'on Elemental]
\begin{eqnarray*}
t^{-1}U\left(t\right)\rightarrow 1/\mu\textrm{,    cuando }t\rightarrow\infty.
\end{eqnarray*}
\end{Teo}

%___________________________________________________________________________________________
%
\subsection{Funci\'on de Renovaci\'on}
%___________________________________________________________________________________________
%


Sup\'ongase que $N\left(t\right)$ es un proceso de renovaci\'on con distribuci\'on $F$ con media finita $\mu$.

\begin{Def}
La funci\'on de renovaci\'on asociada con la distribuci\'on $F$, del proceso $N\left(t\right)$, es
\begin{eqnarray*}
U\left(t\right)=\sum_{n=1}^{\infty}F^{n\star}\left(t\right),\textrm{   }t\geq0,
\end{eqnarray*}
donde $F^{0\star}\left(t\right)=\indora\left(t\geq0\right)$.
\end{Def}


\begin{Prop}
Sup\'ongase que la distribuci\'on de inter-renovaci\'on $F$ tiene densidad $f$. Entonces $U\left(t\right)$ tambi\'en tiene densidad, para $t>0$, y es $U^{'}\left(t\right)=\sum_{n=0}^{\infty}f^{n\star}\left(t\right)$. Adem\'as
\begin{eqnarray*}
\prob\left\{N\left(t\right)>N\left(t-\right)\right\}=0\textrm{,   }t\geq0.
\end{eqnarray*}
\end{Prop}

\begin{Def}
La Transformada de Laplace-Stieljes de $F$ est\'a dada por

\begin{eqnarray*}
\hat{F}\left(\alpha\right)=\int_{\rea_{+}}e^{-\alpha t}dF\left(t\right)\textrm{,  }\alpha\geq0.
\end{eqnarray*}
\end{Def}

Entonces

\begin{eqnarray*}
\hat{U}\left(\alpha\right)=\sum_{n=0}^{\infty}\hat{F^{n\star}}\left(\alpha\right)=\sum_{n=0}^{\infty}\hat{F}\left(\alpha\right)^{n}=\frac{1}{1-\hat{F}\left(\alpha\right)}.
\end{eqnarray*}


\begin{Prop}
La Transformada de Laplace $\hat{U}\left(\alpha\right)$ y $\hat{F}\left(\alpha\right)$ determina una a la otra de manera \'unica por la relaci\'on $\hat{U}\left(\alpha\right)=\frac{1}{1-\hat{F}\left(\alpha\right)}$.
\end{Prop}


\begin{Note}
Un proceso de renovaci\'on $N\left(t\right)$ cuyos tiempos de inter-renovaci\'on tienen media finita, es un proceso Poisson con tasa $\lambda$ si y s\'olo s\'i $\esp\left[U\left(t\right)\right]=\lambda t$, para $t\geq0$.
\end{Note}


\begin{Teo}
Sea $N\left(t\right)$ un proceso puntual simple con puntos de localizaci\'on $T_{n}$ tal que $\eta\left(t\right)=\esp\left[N\left(\right)\right]$ es finita para cada $t$. Entonces para cualquier funci\'on $f:\rea_{+}\rightarrow\rea$,
\begin{eqnarray*}
\esp\left[\sum_{n=1}^{N\left(\right)}f\left(T_{n}\right)\right]=\int_{\left(0,t\right]}f\left(s\right)d\eta\left(s\right)\textrm{,  }t\geq0,
\end{eqnarray*}
suponiendo que la integral exista. Adem\'as si $X_{1},X_{2},\ldots$ son variables aleatorias definidas en el mismo espacio de probabilidad que el proceso $N\left(t\right)$ tal que $\esp\left[X_{n}|T_{n}=s\right]=f\left(s\right)$, independiente de $n$. Entonces
\begin{eqnarray*}
\esp\left[\sum_{n=1}^{N\left(t\right)}X_{n}\right]=\int_{\left(0,t\right]}f\left(s\right)d\eta\left(s\right)\textrm{,  }t\geq0,
\end{eqnarray*} 
suponiendo que la integral exista. 
\end{Teo}

\begin{Coro}[Identidad de Wald para Renovaciones]
Para el proceso de renovaci\'on $N\left(t\right)$,
\begin{eqnarray*}
\esp\left[T_{N\left(t\right)+1}\right]=\mu\esp\left[N\left(t\right)+1\right]\textrm{,  }t\geq0,
\end{eqnarray*}  
\end{Coro}

%______________________________________________________________________
\subsection{Procesos de Renovaci\'on}
%______________________________________________________________________

\begin{Def}%\label{Def.Tn}
Sean $0\leq T_{1}\leq T_{2}\leq \ldots$ son tiempos aleatorios infinitos en los cuales ocurren ciertos eventos. El n\'umero de tiempos $T_{n}$ en el intervalo $\left[0,t\right)$ es

\begin{eqnarray}
N\left(t\right)=\sum_{n=1}^{\infty}\indora\left(T_{n}\leq t\right),
\end{eqnarray}
para $t\geq0$.
\end{Def}

Si se consideran los puntos $T_{n}$ como elementos de $\rea_{+}$, y $N\left(t\right)$ es el n\'umero de puntos en $\rea$. El proceso denotado por $\left\{N\left(t\right):t\geq0\right\}$, denotado por $N\left(t\right)$, es un proceso puntual en $\rea_{+}$. Los $T_{n}$ son los tiempos de ocurrencia, el proceso puntual $N\left(t\right)$ es simple si su n\'umero de ocurrencias son distintas: $0<T_{1}<T_{2}<\ldots$ casi seguramente.

\begin{Def}
Un proceso puntual $N\left(t\right)$ es un proceso de renovaci\'on si los tiempos de interocurrencia $\xi_{n}=T_{n}-T_{n-1}$, para $n\geq1$, son independientes e identicamente distribuidos con distribuci\'on $F$, donde $F\left(0\right)=0$ y $T_{0}=0$. Los $T_{n}$ son llamados tiempos de renovaci\'on, referente a la independencia o renovaci\'on de la informaci\'on estoc\'astica en estos tiempos. Los $\xi_{n}$ son los tiempos de inter-renovaci\'on, y $N\left(t\right)$ es el n\'umero de renovaciones en el intervalo $\left[0,t\right)$
\end{Def}


\begin{Note}
Para definir un proceso de renovaci\'on para cualquier contexto, solamente hay que especificar una distribuci\'on $F$, con $F\left(0\right)=0$, para los tiempos de inter-renovaci\'on. La funci\'on $F$ en turno degune las otra variables aleatorias. De manera formal, existe un espacio de probabilidad y una sucesi\'on de variables aleatorias $\xi_{1},\xi_{2},\ldots$ definidas en este con distribuci\'on $F$. Entonces las otras cantidades son $T_{n}=\sum_{k=1}^{n}\xi_{k}$ y $N\left(t\right)=\sum_{n=1}^{\infty}\indora\left(T_{n}\leq t\right)$, donde $T_{n}\rightarrow\infty$ casi seguramente por la Ley Fuerte de los Grandes Números.
\end{Note}

%___________________________________________________________________________________________
%
\subsection{Renewal and Regenerative Processes: Serfozo\cite{Serfozo}}
%___________________________________________________________________________________________
%
\begin{Def}%\label{Def.Tn}
Sean $0\leq T_{1}\leq T_{2}\leq \ldots$ son tiempos aleatorios infinitos en los cuales ocurren ciertos eventos. El n\'umero de tiempos $T_{n}$ en el intervalo $\left[0,t\right)$ es

\begin{eqnarray}
N\left(t\right)=\sum_{n=1}^{\infty}\indora\left(T_{n}\leq t\right),
\end{eqnarray}
para $t\geq0$.
\end{Def}

Si se consideran los puntos $T_{n}$ como elementos de $\rea_{+}$, y $N\left(t\right)$ es el n\'umero de puntos en $\rea$. El proceso denotado por $\left\{N\left(t\right):t\geq0\right\}$, denotado por $N\left(t\right)$, es un proceso puntual en $\rea_{+}$. Los $T_{n}$ son los tiempos de ocurrencia, el proceso puntual $N\left(t\right)$ es simple si su n\'umero de ocurrencias son distintas: $0<T_{1}<T_{2}<\ldots$ casi seguramente.

\begin{Def}
Un proceso puntual $N\left(t\right)$ es un proceso de renovaci\'on si los tiempos de interocurrencia $\xi_{n}=T_{n}-T_{n-1}$, para $n\geq1$, son independientes e identicamente distribuidos con distribuci\'on $F$, donde $F\left(0\right)=0$ y $T_{0}=0$. Los $T_{n}$ son llamados tiempos de renovaci\'on, referente a la independencia o renovaci\'on de la informaci\'on estoc\'astica en estos tiempos. Los $\xi_{n}$ son los tiempos de inter-renovaci\'on, y $N\left(t\right)$ es el n\'umero de renovaciones en el intervalo $\left[0,t\right)$
\end{Def}


\begin{Note}
Para definir un proceso de renovaci\'on para cualquier contexto, solamente hay que especificar una distribuci\'on $F$, con $F\left(0\right)=0$, para los tiempos de inter-renovaci\'on. La funci\'on $F$ en turno degune las otra variables aleatorias. De manera formal, existe un espacio de probabilidad y una sucesi\'on de variables aleatorias $\xi_{1},\xi_{2},\ldots$ definidas en este con distribuci\'on $F$. Entonces las otras cantidades son $T_{n}=\sum_{k=1}^{n}\xi_{k}$ y $N\left(t\right)=\sum_{n=1}^{\infty}\indora\left(T_{n}\leq t\right)$, donde $T_{n}\rightarrow\infty$ casi seguramente por la Ley Fuerte de los Grandes N\'umeros.
\end{Note}

Los tiempos $T_{n}$ est\'an relacionados con los conteos de $N\left(t\right)$ por

\begin{eqnarray*}
\left\{N\left(t\right)\geq n\right\}&=&\left\{T_{n}\leq t\right\}\\
T_{N\left(t\right)}\leq &t&<T_{N\left(t\right)+1},
\end{eqnarray*}

adem\'as $N\left(T_{n}\right)=n$, y 

\begin{eqnarray*}
N\left(t\right)=\max\left\{n:T_{n}\leq t\right\}=\min\left\{n:T_{n+1}>t\right\}
\end{eqnarray*}

Por propiedades de la convoluci\'on se sabe que

\begin{eqnarray*}
P\left\{T_{n}\leq t\right\}=F^{n\star}\left(t\right)
\end{eqnarray*}
que es la $n$-\'esima convoluci\'on de $F$. Entonces 

\begin{eqnarray*}
\left\{N\left(t\right)\geq n\right\}&=&\left\{T_{n}\leq t\right\}\\
P\left\{N\left(t\right)\leq n\right\}&=&1-F^{\left(n+1\right)\star}\left(t\right)
\end{eqnarray*}

Adem\'as usando el hecho de que $\esp\left[N\left(t\right)\right]=\sum_{n=1}^{\infty}P\left\{N\left(t\right)\geq n\right\}$
se tiene que

\begin{eqnarray*}
\esp\left[N\left(t\right)\right]=\sum_{n=1}^{\infty}F^{n\star}\left(t\right)
\end{eqnarray*}

\begin{Prop}
Para cada $t\geq0$, la funci\'on generadora de momentos $\esp\left[e^{\alpha N\left(t\right)}\right]$ existe para alguna $\alpha$ en una vecindad del 0, y de aqu\'i que $\esp\left[N\left(t\right)^{m}\right]<\infty$, para $m\geq1$.
\end{Prop}

\begin{Ejem}[\textbf{Proceso Poisson}]

Suponga que se tienen tiempos de inter-renovaci\'on \textit{i.i.d.} del proceso de renovaci\'on $N\left(t\right)$ tienen distribuci\'on exponencial $F\left(t\right)=q-e^{-\lambda t}$ con tasa $\lambda$. Entonces $N\left(t\right)$ es un proceso Poisson con tasa $\lambda$.

\end{Ejem}


\begin{Note}
Si el primer tiempo de renovaci\'on $\xi_{1}$ no tiene la misma distribuci\'on que el resto de las $\xi_{n}$, para $n\geq2$, a $N\left(t\right)$ se le llama Proceso de Renovaci\'on retardado, donde si $\xi$ tiene distribuci\'on $G$, entonces el tiempo $T_{n}$ de la $n$-\'esima renovaci\'on tiene distribuci\'on $G\star F^{\left(n-1\right)\star}\left(t\right)$
\end{Note}


\begin{Teo}
Para una constante $\mu\leq\infty$ ( o variable aleatoria), las siguientes expresiones son equivalentes:

\begin{eqnarray}
lim_{n\rightarrow\infty}n^{-1}T_{n}&=&\mu,\textrm{ c.s.}\\
lim_{t\rightarrow\infty}t^{-1}N\left(t\right)&=&1/\mu,\textrm{ c.s.}
\end{eqnarray}
\end{Teo}


Es decir, $T_{n}$ satisface la Ley Fuerte de los Grandes N\'umeros s\'i y s\'olo s\'i $N\left/t\right)$ la cumple.


\begin{Coro}[Ley Fuerte de los Grandes N\'umeros para Procesos de Renovaci\'on]
Si $N\left(t\right)$ es un proceso de renovaci\'on cuyos tiempos de inter-renovaci\'on tienen media $\mu\leq\infty$, entonces
\begin{eqnarray}
t^{-1}N\left(t\right)\rightarrow 1/\mu,\textrm{ c.s. cuando }t\rightarrow\infty.
\end{eqnarray}

\end{Coro}


Considerar el proceso estoc\'astico de valores reales $\left\{Z\left(t\right):t\geq0\right\}$ en el mismo espacio de probabilidad que $N\left(t\right)$

\begin{Def}
Para el proceso $\left\{Z\left(t\right):t\geq0\right\}$ se define la fluctuaci\'on m\'axima de $Z\left(t\right)$ en el intervalo $\left(T_{n-1},T_{n}\right]$:
\begin{eqnarray*}
M_{n}=\sup_{T_{n-1}<t\leq T_{n}}|Z\left(t\right)-Z\left(T_{n-1}\right)|
\end{eqnarray*}
\end{Def}

\begin{Teo}
Sup\'ongase que $n^{-1}T_{n}\rightarrow\mu$ c.s. cuando $n\rightarrow\infty$, donde $\mu\leq\infty$ es una constante o variable aleatoria. Sea $a$ una constante o variable aleatoria que puede ser infinita cuando $\mu$ es finita, y considere las expresiones l\'imite:
\begin{eqnarray}
lim_{n\rightarrow\infty}n^{-1}Z\left(T_{n}\right)&=&a,\textrm{ c.s.}\\
lim_{t\rightarrow\infty}t^{-1}Z\left(t\right)&=&a/\mu,\textrm{ c.s.}
\end{eqnarray}
La segunda expresi\'on implica la primera. Conversamente, la primera implica la segunda si el proceso $Z\left(t\right)$ es creciente, o si $lim_{n\rightarrow\infty}n^{-1}M_{n}=0$ c.s.
\end{Teo}

\begin{Coro}
Si $N\left(t\right)$ es un proceso de renovaci\'on, y $\left(Z\left(T_{n}\right)-Z\left(T_{n-1}\right),M_{n}\right)$, para $n\geq1$, son variables aleatorias independientes e id\'enticamente distribuidas con media finita, entonces,
\begin{eqnarray}
lim_{t\rightarrow\infty}t^{-1}Z\left(t\right)\rightarrow\frac{\esp\left[Z\left(T_{1}\right)-Z\left(T_{0}\right)\right]}{\esp\left[T_{1}\right]},\textrm{ c.s. cuando  }t\rightarrow\infty.
\end{eqnarray}
\end{Coro}


Sup\'ongase que $N\left(t\right)$ es un proceso de renovaci\'on con distribuci\'on $F$ con media finita $\mu$.

\begin{Def}
La funci\'on de renovaci\'on asociada con la distribuci\'on $F$, del proceso $N\left(t\right)$, es
\begin{eqnarray*}
U\left(t\right)=\sum_{n=1}^{\infty}F^{n\star}\left(t\right),\textrm{   }t\geq0,
\end{eqnarray*}
donde $F^{0\star}\left(t\right)=\indora\left(t\geq0\right)$.
\end{Def}


\begin{Prop}
Sup\'ongase que la distribuci\'on de inter-renovaci\'on $F$ tiene densidad $f$. Entonces $U\left(t\right)$ tambi\'en tiene densidad, para $t>0$, y es $U^{'}\left(t\right)=\sum_{n=0}^{\infty}f^{n\star}\left(t\right)$. Adem\'as
\begin{eqnarray*}
\prob\left\{N\left(t\right)>N\left(t-\right)\right\}=0\textrm{,   }t\geq0.
\end{eqnarray*}
\end{Prop}

\begin{Def}
La Transformada de Laplace-Stieljes de $F$ est\'a dada por

\begin{eqnarray*}
\hat{F}\left(\alpha\right)=\int_{\rea_{+}}e^{-\alpha t}dF\left(t\right)\textrm{,  }\alpha\geq0.
\end{eqnarray*}
\end{Def}

Entonces

\begin{eqnarray*}
\hat{U}\left(\alpha\right)=\sum_{n=0}^{\infty}\hat{F^{n\star}}\left(\alpha\right)=\sum_{n=0}^{\infty}\hat{F}\left(\alpha\right)^{n}=\frac{1}{1-\hat{F}\left(\alpha\right)}.
\end{eqnarray*}


\begin{Prop}
La Transformada de Laplace $\hat{U}\left(\alpha\right)$ y $\hat{F}\left(\alpha\right)$ determina una a la otra de manera \'unica por la relaci\'on $\hat{U}\left(\alpha\right)=\frac{1}{1-\hat{F}\left(\alpha\right)}$.
\end{Prop}


\begin{Note}
Un proceso de renovaci\'on $N\left(t\right)$ cuyos tiempos de inter-renovaci\'on tienen media finita, es un proceso Poisson con tasa $\lambda$ si y s\'olo s\'i $\esp\left[U\left(t\right)\right]=\lambda t$, para $t\geq0$.
\end{Note}


\begin{Teo}
Sea $N\left(t\right)$ un proceso puntual simple con puntos de localizaci\'on $T_{n}$ tal que $\eta\left(t\right)=\esp\left[N\left(\right)\right]$ es finita para cada $t$. Entonces para cualquier funci\'on $f:\rea_{+}\rightarrow\rea$,
\begin{eqnarray*}
\esp\left[\sum_{n=1}^{N\left(\right)}f\left(T_{n}\right)\right]=\int_{\left(0,t\right]}f\left(s\right)d\eta\left(s\right)\textrm{,  }t\geq0,
\end{eqnarray*}
suponiendo que la integral exista. Adem\'as si $X_{1},X_{2},\ldots$ son variables aleatorias definidas en el mismo espacio de probabilidad que el proceso $N\left(t\right)$ tal que $\esp\left[X_{n}|T_{n}=s\right]=f\left(s\right)$, independiente de $n$. Entonces
\begin{eqnarray*}
\esp\left[\sum_{n=1}^{N\left(t\right)}X_{n}\right]=\int_{\left(0,t\right]}f\left(s\right)d\eta\left(s\right)\textrm{,  }t\geq0,
\end{eqnarray*} 
suponiendo que la integral exista. 
\end{Teo}

\begin{Coro}[Identidad de Wald para Renovaciones]
Para el proceso de renovaci\'on $N\left(t\right)$,
\begin{eqnarray*}
\esp\left[T_{N\left(t\right)+1}\right]=\mu\esp\left[N\left(t\right)+1\right]\textrm{,  }t\geq0,
\end{eqnarray*}  
\end{Coro}


\begin{Def}
Sea $h\left(t\right)$ funci\'on de valores reales en $\rea$ acotada en intervalos finitos e igual a cero para $t<0$ La ecuaci\'on de renovaci\'on para $h\left(t\right)$ y la distribuci\'on $F$ es

\begin{eqnarray}%\label{Ec.Renovacion}
H\left(t\right)=h\left(t\right)+\int_{\left[0,t\right]}H\left(t-s\right)dF\left(s\right)\textrm{,    }t\geq0,
\end{eqnarray}
donde $H\left(t\right)$ es una funci\'on de valores reales. Esto es $H=h+F\star H$. Decimos que $H\left(t\right)$ es soluci\'on de esta ecuaci\'on si satisface la ecuaci\'on, y es acotada en intervalos finitos e iguales a cero para $t<0$.
\end{Def}

\begin{Prop}
La funci\'on $U\star h\left(t\right)$ es la \'unica soluci\'on de la ecuaci\'on de renovaci\'on (\ref{Ec.Renovacion}).
\end{Prop}

\begin{Teo}[Teorema Renovaci\'on Elemental]
\begin{eqnarray*}
t^{-1}U\left(t\right)\rightarrow 1/\mu\textrm{,    cuando }t\rightarrow\infty.
\end{eqnarray*}
\end{Teo}



Sup\'ongase que $N\left(t\right)$ es un proceso de renovaci\'on con distribuci\'on $F$ con media finita $\mu$.

\begin{Def}
La funci\'on de renovaci\'on asociada con la distribuci\'on $F$, del proceso $N\left(t\right)$, es
\begin{eqnarray*}
U\left(t\right)=\sum_{n=1}^{\infty}F^{n\star}\left(t\right),\textrm{   }t\geq0,
\end{eqnarray*}
donde $F^{0\star}\left(t\right)=\indora\left(t\geq0\right)$.
\end{Def}


\begin{Prop}
Sup\'ongase que la distribuci\'on de inter-renovaci\'on $F$ tiene densidad $f$. Entonces $U\left(t\right)$ tambi\'en tiene densidad, para $t>0$, y es $U^{'}\left(t\right)=\sum_{n=0}^{\infty}f^{n\star}\left(t\right)$. Adem\'as
\begin{eqnarray*}
\prob\left\{N\left(t\right)>N\left(t-\right)\right\}=0\textrm{,   }t\geq0.
\end{eqnarray*}
\end{Prop}

\begin{Def}
La Transformada de Laplace-Stieljes de $F$ est\'a dada por

\begin{eqnarray*}
\hat{F}\left(\alpha\right)=\int_{\rea_{+}}e^{-\alpha t}dF\left(t\right)\textrm{,  }\alpha\geq0.
\end{eqnarray*}
\end{Def}

Entonces

\begin{eqnarray*}
\hat{U}\left(\alpha\right)=\sum_{n=0}^{\infty}\hat{F^{n\star}}\left(\alpha\right)=\sum_{n=0}^{\infty}\hat{F}\left(\alpha\right)^{n}=\frac{1}{1-\hat{F}\left(\alpha\right)}.
\end{eqnarray*}


\begin{Prop}
La Transformada de Laplace $\hat{U}\left(\alpha\right)$ y $\hat{F}\left(\alpha\right)$ determina una a la otra de manera \'unica por la relaci\'on $\hat{U}\left(\alpha\right)=\frac{1}{1-\hat{F}\left(\alpha\right)}$.
\end{Prop}


\begin{Note}
Un proceso de renovaci\'on $N\left(t\right)$ cuyos tiempos de inter-renovaci\'on tienen media finita, es un proceso Poisson con tasa $\lambda$ si y s\'olo s\'i $\esp\left[U\left(t\right)\right]=\lambda t$, para $t\geq0$.
\end{Note}


\begin{Teo}
Sea $N\left(t\right)$ un proceso puntual simple con puntos de localizaci\'on $T_{n}$ tal que $\eta\left(t\right)=\esp\left[N\left(\right)\right]$ es finita para cada $t$. Entonces para cualquier funci\'on $f:\rea_{+}\rightarrow\rea$,
\begin{eqnarray*}
\esp\left[\sum_{n=1}^{N\left(\right)}f\left(T_{n}\right)\right]=\int_{\left(0,t\right]}f\left(s\right)d\eta\left(s\right)\textrm{,  }t\geq0,
\end{eqnarray*}
suponiendo que la integral exista. Adem\'as si $X_{1},X_{2},\ldots$ son variables aleatorias definidas en el mismo espacio de probabilidad que el proceso $N\left(t\right)$ tal que $\esp\left[X_{n}|T_{n}=s\right]=f\left(s\right)$, independiente de $n$. Entonces
\begin{eqnarray*}
\esp\left[\sum_{n=1}^{N\left(t\right)}X_{n}\right]=\int_{\left(0,t\right]}f\left(s\right)d\eta\left(s\right)\textrm{,  }t\geq0,
\end{eqnarray*} 
suponiendo que la integral exista. 
\end{Teo}

\begin{Coro}[Identidad de Wald para Renovaciones]
Para el proceso de renovaci\'on $N\left(t\right)$,
\begin{eqnarray*}
\esp\left[T_{N\left(t\right)+1}\right]=\mu\esp\left[N\left(t\right)+1\right]\textrm{,  }t\geq0,
\end{eqnarray*}  
\end{Coro}


\begin{Def}
Sea $h\left(t\right)$ funci\'on de valores reales en $\rea$ acotada en intervalos finitos e igual a cero para $t<0$ La ecuaci\'on de renovaci\'on para $h\left(t\right)$ y la distribuci\'on $F$ es

\begin{eqnarray}%\label{Ec.Renovacion}
H\left(t\right)=h\left(t\right)+\int_{\left[0,t\right]}H\left(t-s\right)dF\left(s\right)\textrm{,    }t\geq0,
\end{eqnarray}
donde $H\left(t\right)$ es una funci\'on de valores reales. Esto es $H=h+F\star H$. Decimos que $H\left(t\right)$ es soluci\'on de esta ecuaci\'on si satisface la ecuaci\'on, y es acotada en intervalos finitos e iguales a cero para $t<0$.
\end{Def}

\begin{Prop}
La funci\'on $U\star h\left(t\right)$ es la \'unica soluci\'on de la ecuaci\'on de renovaci\'on (\ref{Ec.Renovacion}).
\end{Prop}

\begin{Teo}[Teorema Renovaci\'on Elemental]
\begin{eqnarray*}
t^{-1}U\left(t\right)\rightarrow 1/\mu\textrm{,    cuando }t\rightarrow\infty.
\end{eqnarray*}
\end{Teo}


\begin{Note} Una funci\'on $h:\rea_{+}\rightarrow\rea$ es Directamente Riemann Integrable en los siguientes casos:
\begin{itemize}
\item[a)] $h\left(t\right)\geq0$ es decreciente y Riemann Integrable.
\item[b)] $h$ es continua excepto posiblemente en un conjunto de Lebesgue de medida 0, y $|h\left(t\right)|\leq b\left(t\right)$, donde $b$ es DRI.
\end{itemize}
\end{Note}

\begin{Teo}[Teorema Principal de Renovaci\'on]
Si $F$ es no aritm\'etica y $h\left(t\right)$ es Directamente Riemann Integrable (DRI), entonces

\begin{eqnarray*}
lim_{t\rightarrow\infty}U\star h=\frac{1}{\mu}\int_{\rea_{+}}h\left(s\right)ds.
\end{eqnarray*}
\end{Teo}

\begin{Prop}
Cualquier funci\'on $H\left(t\right)$ acotada en intervalos finitos y que es 0 para $t<0$ puede expresarse como
\begin{eqnarray*}
H\left(t\right)=U\star h\left(t\right)\textrm{,  donde }h\left(t\right)=H\left(t\right)-F\star H\left(t\right)
\end{eqnarray*}
\end{Prop}

\begin{Def}
Un proceso estoc\'astico $X\left(t\right)$ es crudamente regenerativo en un tiempo aleatorio positivo $T$ si
\begin{eqnarray*}
\esp\left[X\left(T+t\right)|T\right]=\esp\left[X\left(t\right)\right]\textrm{, para }t\geq0,\end{eqnarray*}
y con las esperanzas anteriores finitas.
\end{Def}

\begin{Prop}
Sup\'ongase que $X\left(t\right)$ es un proceso crudamente regenerativo en $T$, que tiene distribuci\'on $F$. Si $\esp\left[X\left(t\right)\right]$ es acotado en intervalos finitos, entonces
\begin{eqnarray*}
\esp\left[X\left(t\right)\right]=U\star h\left(t\right)\textrm{,  donde }h\left(t\right)=\esp\left[X\left(t\right)\indora\left(T>t\right)\right].
\end{eqnarray*}
\end{Prop}

\begin{Teo}[Regeneraci\'on Cruda]
Sup\'ongase que $X\left(t\right)$ es un proceso con valores positivo crudamente regenerativo en $T$, y def\'inase $M=\sup\left\{|X\left(t\right)|:t\leq T\right\}$. Si $T$ es no aritm\'etico y $M$ y $MT$ tienen media finita, entonces
\begin{eqnarray*}
lim_{t\rightarrow\infty}\esp\left[X\left(t\right)\right]=\frac{1}{\mu}\int_{\rea_{+}}h\left(s\right)ds,
\end{eqnarray*}
donde $h\left(t\right)=\esp\left[X\left(t\right)\indora\left(T>t\right)\right]$.
\end{Teo}


\begin{Note} Una funci\'on $h:\rea_{+}\rightarrow\rea$ es Directamente Riemann Integrable en los siguientes casos:
\begin{itemize}
\item[a)] $h\left(t\right)\geq0$ es decreciente y Riemann Integrable.
\item[b)] $h$ es continua excepto posiblemente en un conjunto de Lebesgue de medida 0, y $|h\left(t\right)|\leq b\left(t\right)$, donde $b$ es DRI.
\end{itemize}
\end{Note}

\begin{Teo}[Teorema Principal de Renovaci\'on]
Si $F$ es no aritm\'etica y $h\left(t\right)$ es Directamente Riemann Integrable (DRI), entonces

\begin{eqnarray*}
lim_{t\rightarrow\infty}U\star h=\frac{1}{\mu}\int_{\rea_{+}}h\left(s\right)ds.
\end{eqnarray*}
\end{Teo}

\begin{Prop}
Cualquier funci\'on $H\left(t\right)$ acotada en intervalos finitos y que es 0 para $t<0$ puede expresarse como
\begin{eqnarray*}
H\left(t\right)=U\star h\left(t\right)\textrm{,  donde }h\left(t\right)=H\left(t\right)-F\star H\left(t\right)
\end{eqnarray*}
\end{Prop}

\begin{Def}
Un proceso estoc\'astico $X\left(t\right)$ es crudamente regenerativo en un tiempo aleatorio positivo $T$ si
\begin{eqnarray*}
\esp\left[X\left(T+t\right)|T\right]=\esp\left[X\left(t\right)\right]\textrm{, para }t\geq0,\end{eqnarray*}
y con las esperanzas anteriores finitas.
\end{Def}

\begin{Prop}
Sup\'ongase que $X\left(t\right)$ es un proceso crudamente regenerativo en $T$, que tiene distribuci\'on $F$. Si $\esp\left[X\left(t\right)\right]$ es acotado en intervalos finitos, entonces
\begin{eqnarray*}
\esp\left[X\left(t\right)\right]=U\star h\left(t\right)\textrm{,  donde }h\left(t\right)=\esp\left[X\left(t\right)\indora\left(T>t\right)\right].
\end{eqnarray*}
\end{Prop}

\begin{Teo}[Regeneraci\'on Cruda]
Sup\'ongase que $X\left(t\right)$ es un proceso con valores positivo crudamente regenerativo en $T$, y def\'inase $M=\sup\left\{|X\left(t\right)|:t\leq T\right\}$. Si $T$ es no aritm\'etico y $M$ y $MT$ tienen media finita, entonces
\begin{eqnarray*}
lim_{t\rightarrow\infty}\esp\left[X\left(t\right)\right]=\frac{1}{\mu}\int_{\rea_{+}}h\left(s\right)ds,
\end{eqnarray*}
donde $h\left(t\right)=\esp\left[X\left(t\right)\indora\left(T>t\right)\right]$.
\end{Teo}

\begin{Def}
Para el proceso $\left\{\left(N\left(t\right),X\left(t\right)\right):t\geq0\right\}$, sus trayectoria muestrales en el intervalo de tiempo $\left[T_{n-1},T_{n}\right)$ est\'an descritas por
\begin{eqnarray*}
\zeta_{n}=\left(\xi_{n},\left\{X\left(T_{n-1}+t\right):0\leq t<\xi_{n}\right\}\right)
\end{eqnarray*}
Este $\zeta_{n}$ es el $n$-\'esimo segmento del proceso. El proceso es regenerativo sobre los tiempos $T_{n}$ si sus segmentos $\zeta_{n}$ son independientes e id\'enticamennte distribuidos.
\end{Def}


\begin{Note}
Si $\tilde{X}\left(t\right)$ con espacio de estados $\tilde{S}$ es regenerativo sobre $T_{n}$, entonces $X\left(t\right)=f\left(\tilde{X}\left(t\right)\right)$ tambi\'en es regenerativo sobre $T_{n}$, para cualquier funci\'on $f:\tilde{S}\rightarrow S$.
\end{Note}

\begin{Note}
Los procesos regenerativos son crudamente regenerativos, pero no al rev\'es.
\end{Note}


\begin{Note}
Un proceso estoc\'astico a tiempo continuo o discreto es regenerativo si existe un proceso de renovaci\'on  tal que los segmentos del proceso entre tiempos de renovaci\'on sucesivos son i.i.d., es decir, para $\left\{X\left(t\right):t\geq0\right\}$ proceso estoc\'astico a tiempo continuo con espacio de estados $S$, espacio m\'etrico.
\end{Note}

Para $\left\{X\left(t\right):t\geq0\right\}$ Proceso Estoc\'astico a tiempo continuo con estado de espacios $S$, que es un espacio m\'etrico, con trayectorias continuas por la derecha y con l\'imites por la izquierda c.s. Sea $N\left(t\right)$ un proceso de renovaci\'on en $\rea_{+}$ definido en el mismo espacio de probabilidad que $X\left(t\right)$, con tiempos de renovaci\'on $T$ y tiempos de inter-renovaci\'on $\xi_{n}=T_{n}-T_{n-1}$, con misma distribuci\'on $F$ de media finita $\mu$.



\begin{Def}
Para el proceso $\left\{\left(N\left(t\right),X\left(t\right)\right):t\geq0\right\}$, sus trayectoria muestrales en el intervalo de tiempo $\left[T_{n-1},T_{n}\right)$ est\'an descritas por
\begin{eqnarray*}
\zeta_{n}=\left(\xi_{n},\left\{X\left(T_{n-1}+t\right):0\leq t<\xi_{n}\right\}\right)
\end{eqnarray*}
Este $\zeta_{n}$ es el $n$-\'esimo segmento del proceso. El proceso es regenerativo sobre los tiempos $T_{n}$ si sus segmentos $\zeta_{n}$ son independientes e id\'enticamennte distribuidos.
\end{Def}

\begin{Note}
Un proceso regenerativo con media de la longitud de ciclo finita es llamado positivo recurrente.
\end{Note}

\begin{Teo}[Procesos Regenerativos]
Suponga que el proceso
\end{Teo}


\begin{Def}[Renewal Process Trinity]
Para un proceso de renovaci\'on $N\left(t\right)$, los siguientes procesos proveen de informaci\'on sobre los tiempos de renovaci\'on.
\begin{itemize}
\item $A\left(t\right)=t-T_{N\left(t\right)}$, el tiempo de recurrencia hacia atr\'as al tiempo $t$, que es el tiempo desde la \'ultima renovaci\'on para $t$.

\item $B\left(t\right)=T_{N\left(t\right)+1}-t$, el tiempo de recurrencia hacia adelante al tiempo $t$, residual del tiempo de renovaci\'on, que es el tiempo para la pr\'oxima renovaci\'on despu\'es de $t$.

\item $L\left(t\right)=\xi_{N\left(t\right)+1}=A\left(t\right)+B\left(t\right)$, la longitud del intervalo de renovaci\'on que contiene a $t$.
\end{itemize}
\end{Def}

\begin{Note}
El proceso tridimensional $\left(A\left(t\right),B\left(t\right),L\left(t\right)\right)$ es regenerativo sobre $T_{n}$, y por ende cada proceso lo es. Cada proceso $A\left(t\right)$ y $B\left(t\right)$ son procesos de MArkov a tiempo continuo con trayectorias continuas por partes en el espacio de estados $\rea_{+}$. Una expresi\'on conveniente para su distribuci\'on conjunta es, para $0\leq x<t,y\geq0$
\begin{equation}\label{NoRenovacion}
P\left\{A\left(t\right)>x,B\left(t\right)>y\right\}=
P\left\{N\left(t+y\right)-N\left((t-x)\right)=0\right\}
\end{equation}
\end{Note}

\begin{Ejem}[Tiempos de recurrencia Poisson]
Si $N\left(t\right)$ es un proceso Poisson con tasa $\lambda$, entonces de la expresi\'on (\ref{NoRenovacion}) se tiene que

\begin{eqnarray*}
\begin{array}{lc}
P\left\{A\left(t\right)>x,B\left(t\right)>y\right\}=e^{-\lambda\left(x+y\right)},&0\leq x<t,y\geq0,
\end{array}
\end{eqnarray*}
que es la probabilidad Poisson de no renovaciones en un intervalo de longitud $x+y$.

\end{Ejem}

%\begin{Note}
Una cadena de Markov erg\'odica tiene la propiedad de ser estacionaria si la distribuci\'on de su estado al tiempo $0$ es su distribuci\'on estacionaria.
%\end{Note}


\begin{Def}
Un proceso estoc\'astico a tiempo continuo $\left\{X\left(t\right):t\geq0\right\}$ en un espacio general es estacionario si sus distribuciones finito dimensionales son invariantes bajo cualquier  traslado: para cada $0\leq s_{1}<s_{2}<\cdots<s_{k}$ y $t\geq0$,
\begin{eqnarray*}
\left(X\left(s_{1}+t\right),\ldots,X\left(s_{k}+t\right)\right)=_{d}\left(X\left(s_{1}\right),\ldots,X\left(s_{k}\right)\right).
\end{eqnarray*}
\end{Def}

\begin{Note}
Un proceso de Markov es estacionario si $X\left(t\right)=_{d}X\left(0\right)$, $t\geq0$.
\end{Note}

Considerese el proceso $N\left(t\right)=\sum_{n}\indora\left(\tau_{n}\leq t\right)$ en $\rea_{+}$, con puntos $0<\tau_{1}<\tau_{2}<\cdots$.

\begin{Prop}
Si $N$ es un proceso puntual estacionario y $\esp\left[N\left(1\right)\right]<\infty$, entonces $\esp\left[N\left(t\right)\right]=t\esp\left[N\left(1\right)\right]$, $t\geq0$

\end{Prop}

\begin{Teo}
Los siguientes enunciados son equivalentes
\begin{itemize}
\item[i)] El proceso retardado de renovaci\'on $N$ es estacionario.

\item[ii)] EL proceso de tiempos de recurrencia hacia adelante $B\left(t\right)$ es estacionario.


\item[iii)] $\esp\left[N\left(t\right)\right]=t/\mu$,


\item[iv)] $G\left(t\right)=F_{e}\left(t\right)=\frac{1}{\mu}\int_{0}^{t}\left[1-F\left(s\right)\right]ds$
\end{itemize}
Cuando estos enunciados son ciertos, $P\left\{B\left(t\right)\leq x\right\}=F_{e}\left(x\right)$, para $t,x\geq0$.

\end{Teo}

\begin{Note}
Una consecuencia del teorema anterior es que el Proceso Poisson es el \'unico proceso sin retardo que es estacionario.
\end{Note}

\begin{Coro}
El proceso de renovaci\'on $N\left(t\right)$ sin retardo, y cuyos tiempos de inter renonaci\'on tienen media finita, es estacionario si y s\'olo si es un proceso Poisson.

\end{Coro}


%________________________________________________________________________
\subsection{Procesos Regenerativos}
%________________________________________________________________________



\begin{Note}
Si $\tilde{X}\left(t\right)$ con espacio de estados $\tilde{S}$ es regenerativo sobre $T_{n}$, entonces $X\left(t\right)=f\left(\tilde{X}\left(t\right)\right)$ tambi\'en es regenerativo sobre $T_{n}$, para cualquier funci\'on $f:\tilde{S}\rightarrow S$.
\end{Note}

\begin{Note}
Los procesos regenerativos son crudamente regenerativos, pero no al rev\'es.
\end{Note}
%\subsection*{Procesos Regenerativos: Sigman\cite{Sigman1}}
\begin{Def}[Definici\'on Cl\'asica]
Un proceso estoc\'astico $X=\left\{X\left(t\right):t\geq0\right\}$ es llamado regenerativo is existe una variable aleatoria $R_{1}>0$ tal que
\begin{itemize}
\item[i)] $\left\{X\left(t+R_{1}\right):t\geq0\right\}$ es independiente de $\left\{\left\{X\left(t\right):t<R_{1}\right\},\right\}$
\item[ii)] $\left\{X\left(t+R_{1}\right):t\geq0\right\}$ es estoc\'asticamente equivalente a $\left\{X\left(t\right):t>0\right\}$
\end{itemize}

Llamamos a $R_{1}$ tiempo de regeneraci\'on, y decimos que $X$ se regenera en este punto.
\end{Def}

$\left\{X\left(t+R_{1}\right)\right\}$ es regenerativo con tiempo de regeneraci\'on $R_{2}$, independiente de $R_{1}$ pero con la misma distribuci\'on que $R_{1}$. Procediendo de esta manera se obtiene una secuencia de variables aleatorias independientes e id\'enticamente distribuidas $\left\{R_{n}\right\}$ llamados longitudes de ciclo. Si definimos a $Z_{k}\equiv R_{1}+R_{2}+\cdots+R_{k}$, se tiene un proceso de renovaci\'on llamado proceso de renovaci\'on encajado para $X$.




\begin{Def}
Para $x$ fijo y para cada $t\geq0$, sea $I_{x}\left(t\right)=1$ si $X\left(t\right)\leq x$,  $I_{x}\left(t\right)=0$ en caso contrario, y def\'inanse los tiempos promedio
\begin{eqnarray*}
\overline{X}&=&lim_{t\rightarrow\infty}\frac{1}{t}\int_{0}^{\infty}X\left(u\right)du\\
\prob\left(X_{\infty}\leq x\right)&=&lim_{t\rightarrow\infty}\frac{1}{t}\int_{0}^{\infty}I_{x}\left(u\right)du,
\end{eqnarray*}
cuando estos l\'imites existan.
\end{Def}

Como consecuencia del teorema de Renovaci\'on-Recompensa, se tiene que el primer l\'imite  existe y es igual a la constante
\begin{eqnarray*}
\overline{X}&=&\frac{\esp\left[\int_{0}^{R_{1}}X\left(t\right)dt\right]}{\esp\left[R_{1}\right]},
\end{eqnarray*}
suponiendo que ambas esperanzas son finitas.

\begin{Note}
\begin{itemize}
\item[a)] Si el proceso regenerativo $X$ es positivo recurrente y tiene trayectorias muestrales no negativas, entonces la ecuaci\'on anterior es v\'alida.
\item[b)] Si $X$ es positivo recurrente regenerativo, podemos construir una \'unica versi\'on estacionaria de este proceso, $X_{e}=\left\{X_{e}\left(t\right)\right\}$, donde $X_{e}$ es un proceso estoc\'astico regenerativo y estrictamente estacionario, con distribuci\'on marginal distribuida como $X_{\infty}$
\end{itemize}
\end{Note}

%________________________________________________________________________
%\subsection{Procesos Regenerativos}
%________________________________________________________________________

Para $\left\{X\left(t\right):t\geq0\right\}$ Proceso Estoc\'astico a tiempo continuo con estado de espacios $S$, que es un espacio m\'etrico, con trayectorias continuas por la derecha y con l\'imites por la izquierda c.s. Sea $N\left(t\right)$ un proceso de renovaci\'on en $\rea_{+}$ definido en el mismo espacio de probabilidad que $X\left(t\right)$, con tiempos de renovaci\'on $T$ y tiempos de inter-renovaci\'on $\xi_{n}=T_{n}-T_{n-1}$, con misma distribuci\'on $F$ de media finita $\mu$.



\begin{Def}
Para el proceso $\left\{\left(N\left(t\right),X\left(t\right)\right):t\geq0\right\}$, sus trayectoria muestrales en el intervalo de tiempo $\left[T_{n-1},T_{n}\right)$ est\'an descritas por
\begin{eqnarray*}
\zeta_{n}=\left(\xi_{n},\left\{X\left(T_{n-1}+t\right):0\leq t<\xi_{n}\right\}\right)
\end{eqnarray*}
Este $\zeta_{n}$ es el $n$-\'esimo segmento del proceso. El proceso es regenerativo sobre los tiempos $T_{n}$ si sus segmentos $\zeta_{n}$ son independientes e id\'enticamennte distribuidos.
\end{Def}


\begin{Note}
Si $\tilde{X}\left(t\right)$ con espacio de estados $\tilde{S}$ es regenerativo sobre $T_{n}$, entonces $X\left(t\right)=f\left(\tilde{X}\left(t\right)\right)$ tambi\'en es regenerativo sobre $T_{n}$, para cualquier funci\'on $f:\tilde{S}\rightarrow S$.
\end{Note}

\begin{Note}
Los procesos regenerativos son crudamente regenerativos, pero no al rev\'es.
\end{Note}

\begin{Def}[Definici\'on Cl\'asica]
Un proceso estoc\'astico $X=\left\{X\left(t\right):t\geq0\right\}$ es llamado regenerativo is existe una variable aleatoria $R_{1}>0$ tal que
\begin{itemize}
\item[i)] $\left\{X\left(t+R_{1}\right):t\geq0\right\}$ es independiente de $\left\{\left\{X\left(t\right):t<R_{1}\right\},\right\}$
\item[ii)] $\left\{X\left(t+R_{1}\right):t\geq0\right\}$ es estoc\'asticamente equivalente a $\left\{X\left(t\right):t>0\right\}$
\end{itemize}

Llamamos a $R_{1}$ tiempo de regeneraci\'on, y decimos que $X$ se regenera en este punto.
\end{Def}

$\left\{X\left(t+R_{1}\right)\right\}$ es regenerativo con tiempo de regeneraci\'on $R_{2}$, independiente de $R_{1}$ pero con la misma distribuci\'on que $R_{1}$. Procediendo de esta manera se obtiene una secuencia de variables aleatorias independientes e id\'enticamente distribuidas $\left\{R_{n}\right\}$ llamados longitudes de ciclo. Si definimos a $Z_{k}\equiv R_{1}+R_{2}+\cdots+R_{k}$, se tiene un proceso de renovaci\'on llamado proceso de renovaci\'on encajado para $X$.

\begin{Note}
Un proceso regenerativo con media de la longitud de ciclo finita es llamado positivo recurrente.
\end{Note}


\begin{Def}
Para $x$ fijo y para cada $t\geq0$, sea $I_{x}\left(t\right)=1$ si $X\left(t\right)\leq x$,  $I_{x}\left(t\right)=0$ en caso contrario, y def\'inanse los tiempos promedio
\begin{eqnarray*}
\overline{X}&=&lim_{t\rightarrow\infty}\frac{1}{t}\int_{0}^{\infty}X\left(u\right)du\\
\prob\left(X_{\infty}\leq x\right)&=&lim_{t\rightarrow\infty}\frac{1}{t}\int_{0}^{\infty}I_{x}\left(u\right)du,
\end{eqnarray*}
cuando estos l\'imites existan.
\end{Def}

Como consecuencia del teorema de Renovaci\'on-Recompensa, se tiene que el primer l\'imite  existe y es igual a la constante
\begin{eqnarray*}
\overline{X}&=&\frac{\esp\left[\int_{0}^{R_{1}}X\left(t\right)dt\right]}{\esp\left[R_{1}\right]},
\end{eqnarray*}
suponiendo que ambas esperanzas son finitas.

\begin{Note}
\begin{itemize}
\item[a)] Si el proceso regenerativo $X$ es positivo recurrente y tiene trayectorias muestrales no negativas, entonces la ecuaci\'on anterior es v\'alida.
\item[b)] Si $X$ es positivo recurrente regenerativo, podemos construir una \'unica versi\'on estacionaria de este proceso, $X_{e}=\left\{X_{e}\left(t\right)\right\}$, donde $X_{e}$ es un proceso estoc\'astico regenerativo y estrictamente estacionario, con distribuci\'on marginal distribuida como $X_{\infty}$
\end{itemize}
\end{Note}

%__________________________________________________________________________________________
%\subsection{Procesos Regenerativos Estacionarios - Stidham \cite{Stidham}}
%__________________________________________________________________________________________


Un proceso estoc\'astico a tiempo continuo $\left\{V\left(t\right),t\geq0\right\}$ es un proceso regenerativo si existe una sucesi\'on de variables aleatorias independientes e id\'enticamente distribuidas $\left\{X_{1},X_{2},\ldots\right\}$, sucesi\'on de renovaci\'on, tal que para cualquier conjunto de Borel $A$, 

\begin{eqnarray*}
\prob\left\{V\left(t\right)\in A|X_{1}+X_{2}+\cdots+X_{R\left(t\right)}=s,\left\{V\left(\tau\right),\tau<s\right\}\right\}=\prob\left\{V\left(t-s\right)\in A|X_{1}>t-s\right\},
\end{eqnarray*}
para todo $0\leq s\leq t$, donde $R\left(t\right)=\max\left\{X_{1}+X_{2}+\cdots+X_{j}\leq t\right\}=$n\'umero de renovaciones ({\emph{puntos de regeneraci\'on}}) que ocurren en $\left[0,t\right]$. El intervalo $\left[0,X_{1}\right)$ es llamado {\emph{primer ciclo de regeneraci\'on}} de $\left\{V\left(t \right),t\geq0\right\}$, $\left[X_{1},X_{1}+X_{2}\right)$ el {\emph{segundo ciclo de regeneraci\'on}}, y as\'i sucesivamente.

Sea $X=X_{1}$ y sea $F$ la funci\'on de distrbuci\'on de $X$


\begin{Def}
Se define el proceso estacionario, $\left\{V^{*}\left(t\right),t\geq0\right\}$, para $\left\{V\left(t\right),t\geq0\right\}$ por

\begin{eqnarray*}
\prob\left\{V\left(t\right)\in A\right\}=\frac{1}{\esp\left[X\right]}\int_{0}^{\infty}\prob\left\{V\left(t+x\right)\in A|X>x\right\}\left(1-F\left(x\right)\right)dx,
\end{eqnarray*} 
para todo $t\geq0$ y todo conjunto de Borel $A$.
\end{Def}

\begin{Def}
Una distribuci\'on se dice que es {\emph{aritm\'etica}} si todos sus puntos de incremento son m\'ultiplos de la forma $0,\lambda, 2\lambda,\ldots$ para alguna $\lambda>0$ entera.
\end{Def}


\begin{Def}
Una modificaci\'on medible de un proceso $\left\{V\left(t\right),t\geq0\right\}$, es una versi\'on de este, $\left\{V\left(t,w\right)\right\}$ conjuntamente medible para $t\geq0$ y para $w\in S$, $S$ espacio de estados para $\left\{V\left(t\right),t\geq0\right\}$.
\end{Def}

\begin{Teo}
Sea $\left\{V\left(t\right),t\geq\right\}$ un proceso regenerativo no negativo con modificaci\'on medible. Sea $\esp\left[X\right]<\infty$. Entonces el proceso estacionario dado por la ecuaci\'on anterior est\'a bien definido y tiene funci\'on de distribuci\'on independiente de $t$, adem\'as
\begin{itemize}
\item[i)] \begin{eqnarray*}
\esp\left[V^{*}\left(0\right)\right]&=&\frac{\esp\left[\int_{0}^{X}V\left(s\right)ds\right]}{\esp\left[X\right]}\end{eqnarray*}
\item[ii)] Si $\esp\left[V^{*}\left(0\right)\right]<\infty$, equivalentemente, si $\esp\left[\int_{0}^{X}V\left(s\right)ds\right]<\infty$,entonces
\begin{eqnarray*}
\frac{\int_{0}^{t}V\left(s\right)ds}{t}\rightarrow\frac{\esp\left[\int_{0}^{X}V\left(s\right)ds\right]}{\esp\left[X\right]}
\end{eqnarray*}
con probabilidad 1 y en media, cuando $t\rightarrow\infty$.
\end{itemize}
\end{Teo}

%__________________________________________________________________________________________
%\subsection{Procesos Regenerativos Estacionarios - Stidham \cite{Stidham}}
%__________________________________________________________________________________________


Un proceso estoc\'astico a tiempo continuo $\left\{V\left(t\right),t\geq0\right\}$ es un proceso regenerativo si existe una sucesi\'on de variables aleatorias independientes e id\'enticamente distribuidas $\left\{X_{1},X_{2},\ldots\right\}$, sucesi\'on de renovaci\'on, tal que para cualquier conjunto de Borel $A$, 

\begin{eqnarray*}
\prob\left\{V\left(t\right)\in A|X_{1}+X_{2}+\cdots+X_{R\left(t\right)}=s,\left\{V\left(\tau\right),\tau<s\right\}\right\}=\prob\left\{V\left(t-s\right)\in A|X_{1}>t-s\right\},
\end{eqnarray*}
para todo $0\leq s\leq t$, donde $R\left(t\right)=\max\left\{X_{1}+X_{2}+\cdots+X_{j}\leq t\right\}=$n\'umero de renovaciones ({\emph{puntos de regeneraci\'on}}) que ocurren en $\left[0,t\right]$. El intervalo $\left[0,X_{1}\right)$ es llamado {\emph{primer ciclo de regeneraci\'on}} de $\left\{V\left(t \right),t\geq0\right\}$, $\left[X_{1},X_{1}+X_{2}\right)$ el {\emph{segundo ciclo de regeneraci\'on}}, y as\'i sucesivamente.

Sea $X=X_{1}$ y sea $F$ la funci\'on de distrbuci\'on de $X$


\begin{Def}
Se define el proceso estacionario, $\left\{V^{*}\left(t\right),t\geq0\right\}$, para $\left\{V\left(t\right),t\geq0\right\}$ por

\begin{eqnarray*}
\prob\left\{V\left(t\right)\in A\right\}=\frac{1}{\esp\left[X\right]}\int_{0}^{\infty}\prob\left\{V\left(t+x\right)\in A|X>x\right\}\left(1-F\left(x\right)\right)dx,
\end{eqnarray*} 
para todo $t\geq0$ y todo conjunto de Borel $A$.
\end{Def}

\begin{Def}
Una distribuci\'on se dice que es {\emph{aritm\'etica}} si todos sus puntos de incremento son m\'ultiplos de la forma $0,\lambda, 2\lambda,\ldots$ para alguna $\lambda>0$ entera.
\end{Def}


\begin{Def}
Una modificaci\'on medible de un proceso $\left\{V\left(t\right),t\geq0\right\}$, es una versi\'on de este, $\left\{V\left(t,w\right)\right\}$ conjuntamente medible para $t\geq0$ y para $w\in S$, $S$ espacio de estados para $\left\{V\left(t\right),t\geq0\right\}$.
\end{Def}

\begin{Teo}
Sea $\left\{V\left(t\right),t\geq\right\}$ un proceso regenerativo no negativo con modificaci\'on medible. Sea $\esp\left[X\right]<\infty$. Entonces el proceso estacionario dado por la ecuaci\'on anterior est\'a bien definido y tiene funci\'on de distribuci\'on independiente de $t$, adem\'as
\begin{itemize}
\item[i)] \begin{eqnarray*}
\esp\left[V^{*}\left(0\right)\right]&=&\frac{\esp\left[\int_{0}^{X}V\left(s\right)ds\right]}{\esp\left[X\right]}\end{eqnarray*}
\item[ii)] Si $\esp\left[V^{*}\left(0\right)\right]<\infty$, equivalentemente, si $\esp\left[\int_{0}^{X}V\left(s\right)ds\right]<\infty$,entonces
\begin{eqnarray*}
\frac{\int_{0}^{t}V\left(s\right)ds}{t}\rightarrow\frac{\esp\left[\int_{0}^{X}V\left(s\right)ds\right]}{\esp\left[X\right]}
\end{eqnarray*}
con probabilidad 1 y en media, cuando $t\rightarrow\infty$.
\end{itemize}
\end{Teo}

Para $\left\{X\left(t\right):t\geq0\right\}$ Proceso Estoc\'astico a tiempo continuo con estado de espacios $S$, que es un espacio m\'etrico, con trayectorias continuas por la derecha y con l\'imites por la izquierda c.s. Sea $N\left(t\right)$ un proceso de renovaci\'on en $\rea_{+}$ definido en el mismo espacio de probabilidad que $X\left(t\right)$, con tiempos de renovaci\'on $T$ y tiempos de inter-renovaci\'on $\xi_{n}=T_{n}-T_{n-1}$, con misma distribuci\'on $F$ de media finita $\mu$.
%_____________________________________________________
\subsection{Puntos de Renovaci\'on}
%_____________________________________________________

Para cada cola $Q_{i}$ se tienen los procesos de arribo a la cola, para estas, los tiempos de arribo est\'an dados por $$\left\{T_{1}^{i},T_{2}^{i},\ldots,T_{k}^{i},\ldots\right\},$$ entonces, consideremos solamente los primeros tiempos de arribo a cada una de las colas, es decir, $$\left\{T_{1}^{1},T_{1}^{2},T_{1}^{3},T_{1}^{4}\right\},$$ se sabe que cada uno de estos tiempos se distribuye de manera exponencial con par\'ametro $1/mu_{i}$. Adem\'as se sabe que para $$T^{*}=\min\left\{T_{1}^{1},T_{1}^{2},T_{1}^{3},T_{1}^{4}\right\},$$ $T^{*}$ se distribuye de manera exponencial con par\'ametro $$\mu^{*}=\sum_{i=1}^{4}\mu_{i}.$$ Ahora, dado que 
\begin{center}
\begin{tabular}{lcl}
$\tilde{r}=r_{1}+r_{2}$ & y &$\hat{r}=r_{3}+r_{4}.$
\end{tabular}
\end{center}


Supongamos que $$\tilde{r},\hat{r}<\mu^{*},$$ entonces si tomamos $$r^{*}=\min\left\{\tilde{r},\hat{r}\right\},$$ se tiene que para  $$t^{*}\in\left(0,r^{*}\right)$$ se cumple que 
\begin{center}
\begin{tabular}{lcl}
$\tau_{1}\left(1\right)=0$ & y por tanto & $\overline{\tau}_{1}=0,$
\end{tabular}
\end{center}
entonces para la segunda cola en este primer ciclo se cumple que $$\tau_{2}=\overline{\tau}_{1}+r_{1}=r_{1}<\mu^{*},$$ y por tanto se tiene que  $$\overline{\tau}_{2}=\tau_{2}.$$ Por lo tanto, nuevamente para la primer cola en el segundo ciclo $$\tau_{1}\left(2\right)=\tau_{2}\left(1\right)+r_{2}=\tilde{r}<\mu^{*}.$$ An\'alogamente para el segundo sistema se tiene que ambas colas est\'an vac\'ias, es decir, existe un valor $t^{*}$ tal que en el intervalo $\left(0,t^{*}\right)$ no ha llegado ning\'un usuario, es decir, $$L_{i}\left(t^{*}\right)=0$$ para $i=1,2,3,4$.

\subsection{Resultados para Procesos de Salida}




%________________________________________________________________________
\subsection{Procesos Regenerativos}
%________________________________________________________________________

Para $\left\{X\left(t\right):t\geq0\right\}$ Proceso Estoc\'astico a tiempo continuo con estado de espacios $S$, que es un espacio m\'etrico, con trayectorias continuas por la derecha y con l\'imites por la izquierda c.s. Sea $N\left(t\right)$ un proceso de renovaci\'on en $\rea_{+}$ definido en el mismo espacio de probabilidad que $X\left(t\right)$, con tiempos de renovaci\'on $T$ y tiempos de inter-renovaci\'on $\xi_{n}=T_{n}-T_{n-1}$, con misma distribuci\'on $F$ de media finita $\mu$.



\begin{Def}
Para el proceso $\left\{\left(N\left(t\right),X\left(t\right)\right):t\geq0\right\}$, sus trayectoria muestrales en el intervalo de tiempo $\left[T_{n-1},T_{n}\right)$ est\'an descritas por
\begin{eqnarray*}
\zeta_{n}=\left(\xi_{n},\left\{X\left(T_{n-1}+t\right):0\leq t<\xi_{n}\right\}\right)
\end{eqnarray*}
Este $\zeta_{n}$ es el $n$-\'esimo segmento del proceso. El proceso es regenerativo sobre los tiempos $T_{n}$ si sus segmentos $\zeta_{n}$ son independientes e id\'enticamennte distribuidos.
\end{Def}


\begin{Obs}
Si $\tilde{X}\left(t\right)$ con espacio de estados $\tilde{S}$ es regenerativo sobre $T_{n}$, entonces $X\left(t\right)=f\left(\tilde{X}\left(t\right)\right)$ tambi\'en es regenerativo sobre $T_{n}$, para cualquier funci\'on $f:\tilde{S}\rightarrow S$.
\end{Obs}

\begin{Obs}
Los procesos regenerativos son crudamente regenerativos, pero no al rev\'es.
\end{Obs}

\begin{Def}[Definici\'on Cl\'asica]
Un proceso estoc\'astico $X=\left\{X\left(t\right):t\geq0\right\}$ es llamado regenerativo is existe una variable aleatoria $R_{1}>0$ tal que
\begin{itemize}
\item[i)] $\left\{X\left(t+R_{1}\right):t\geq0\right\}$ es independiente de $\left\{\left\{X\left(t\right):t<R_{1}\right\},\right\}$
\item[ii)] $\left\{X\left(t+R_{1}\right):t\geq0\right\}$ es estoc\'asticamente equivalente a $\left\{X\left(t\right):t>0\right\}$
\end{itemize}

Llamamos a $R_{1}$ tiempo de regeneraci\'on, y decimos que $X$ se regenera en este punto.
\end{Def}

$\left\{X\left(t+R_{1}\right)\right\}$ es regenerativo con tiempo de regeneraci\'on $R_{2}$, independiente de $R_{1}$ pero con la misma distribuci\'on que $R_{1}$. Procediendo de esta manera se obtiene una secuencia de variables aleatorias independientes e id\'enticamente distribuidas $\left\{R_{n}\right\}$ llamados longitudes de ciclo. Si definimos a $Z_{k}\equiv R_{1}+R_{2}+\cdots+R_{k}$, se tiene un proceso de renovaci\'on llamado proceso de renovaci\'on encajado para $X$.

\begin{Note}
Un proceso regenerativo con media de la longitud de ciclo finita es llamado positivo recurrente.
\end{Note}


\begin{Def}
Para $x$ fijo y para cada $t\geq0$, sea $I_{x}\left(t\right)=1$ si $X\left(t\right)\leq x$,  $I_{x}\left(t\right)=0$ en caso contrario, y def\'inanse los tiempos promedio
\begin{eqnarray*}
\overline{X}&=&lim_{t\rightarrow\infty}\frac{1}{t}\int_{0}^{\infty}X\left(u\right)du\\
\prob\left(X_{\infty}\leq x\right)&=&lim_{t\rightarrow\infty}\frac{1}{t}\int_{0}^{\infty}I_{x}\left(u\right)du,
\end{eqnarray*}
cuando estos l\'imites existan.
\end{Def}

Como consecuencia del teorema de Renovaci\'on-Recompensa, se tiene que el primer l\'imite  existe y es igual a la constante
\begin{eqnarray*}
\overline{X}&=&\frac{\esp\left[\int_{0}^{R_{1}}X\left(t\right)dt\right]}{\esp\left[R_{1}\right]},
\end{eqnarray*}
suponiendo que ambas esperanzas son finitas.

\begin{Note}
\begin{itemize}
\item[a)] Si el proceso regenerativo $X$ es positivo recurrente y tiene trayectorias muestrales no negativas, entonces la ecuaci\'on anterior es v\'alida.
\item[b)] Si $X$ es positivo recurrente regenerativo, podemos construir una \'unica versi\'on estacionaria de este proceso, $X_{e}=\left\{X_{e}\left(t\right)\right\}$, donde $X_{e}$ es un proceso estoc\'astico regenerativo y estrictamente estacionario, con distribuci\'on marginal distribuida como $X_{\infty}$
\end{itemize}
\end{Note}

\subsection{Renewal and Regenerative Processes: Serfozo\cite{Serfozo}}
\begin{Def}\label{Def.Tn}
Sean $0\leq T_{1}\leq T_{2}\leq \ldots$ son tiempos aleatorios infinitos en los cuales ocurren ciertos eventos. El n\'umero de tiempos $T_{n}$ en el intervalo $\left[0,t\right)$ es

\begin{eqnarray}
N\left(t\right)=\sum_{n=1}^{\infty}\indora\left(T_{n}\leq t\right),
\end{eqnarray}
para $t\geq0$.
\end{Def}

Si se consideran los puntos $T_{n}$ como elementos de $\rea_{+}$, y $N\left(t\right)$ es el n\'umero de puntos en $\rea$. El proceso denotado por $\left\{N\left(t\right):t\geq0\right\}$, denotado por $N\left(t\right)$, es un proceso puntual en $\rea_{+}$. Los $T_{n}$ son los tiempos de ocurrencia, el proceso puntual $N\left(t\right)$ es simple si su n\'umero de ocurrencias son distintas: $0<T_{1}<T_{2}<\ldots$ casi seguramente.

\begin{Def}
Un proceso puntual $N\left(t\right)$ es un proceso de renovaci\'on si los tiempos de interocurrencia $\xi_{n}=T_{n}-T_{n-1}$, para $n\geq1$, son independientes e identicamente distribuidos con distribuci\'on $F$, donde $F\left(0\right)=0$ y $T_{0}=0$. Los $T_{n}$ son llamados tiempos de renovaci\'on, referente a la independencia o renovaci\'on de la informaci\'on estoc\'astica en estos tiempos. Los $\xi_{n}$ son los tiempos de inter-renovaci\'on, y $N\left(t\right)$ es el n\'umero de renovaciones en el intervalo $\left[0,t\right)$
\end{Def}


\begin{Note}
Para definir un proceso de renovaci\'on para cualquier contexto, solamente hay que especificar una distribuci\'on $F$, con $F\left(0\right)=0$, para los tiempos de inter-renovaci\'on. La funci\'on $F$ en turno degune las otra variables aleatorias. De manera formal, existe un espacio de probabilidad y una sucesi\'on de variables aleatorias $\xi_{1},\xi_{2},\ldots$ definidas en este con distribuci\'on $F$. Entonces las otras cantidades son $T_{n}=\sum_{k=1}^{n}\xi_{k}$ y $N\left(t\right)=\sum_{n=1}^{\infty}\indora\left(T_{n}\leq t\right)$, donde $T_{n}\rightarrow\infty$ casi seguramente por la Ley Fuerte de los Grandes N\'umeros.
\end{Note}


Los tiempos $T_{n}$ est\'an relacionados con los conteos de $N\left(t\right)$ por

\begin{eqnarray*}
\left\{N\left(t\right)\geq n\right\}&=&\left\{T_{n}\leq t\right\}\\
T_{N\left(t\right)}\leq &t&<T_{N\left(t\right)+1},
\end{eqnarray*}

adem\'as $N\left(T_{n}\right)=n$, y 

\begin{eqnarray*}
N\left(t\right)=\max\left\{n:T_{n}\leq t\right\}=\min\left\{n:T_{n+1}>t\right\}
\end{eqnarray*}

Por propiedades de la convoluci\'on se sabe que

\begin{eqnarray*}
P\left\{T_{n}\leq t\right\}=F^{n\star}\left(t\right)
\end{eqnarray*}
que es la $n$-\'esima convoluci\'on de $F$. Entonces 

\begin{eqnarray*}
\left\{N\left(t\right)\geq n\right\}&=&\left\{T_{n}\leq t\right\}\\
P\left\{N\left(t\right)\leq n\right\}&=&1-F^{\left(n+1\right)\star}\left(t\right)
\end{eqnarray*}

Adem\'as usando el hecho de que $\esp\left[N\left(t\right)\right]=\sum_{n=1}^{\infty}P\left\{N\left(t\right)\geq n\right\}$
se tiene que

\begin{eqnarray*}
\esp\left[N\left(t\right)\right]=\sum_{n=1}^{\infty}F^{n\star}\left(t\right)
\end{eqnarray*}

\begin{Prop}
Para cada $t\geq0$, la funci\'on generadora de momentos $\esp\left[e^{\alpha N\left(t\right)}\right]$ existe para alguna $\alpha$ en una vecindad del 0, y de aqu\'i que $\esp\left[N\left(t\right)^{m}\right]<\infty$, para $m\geq1$.
\end{Prop}


\begin{Note}
Si el primer tiempo de renovaci\'on $\xi_{1}$ no tiene la misma distribuci\'on que el resto de las $\xi_{n}$, para $n\geq2$, a $N\left(t\right)$ se le llama Proceso de Renovaci\'on retardado, donde si $\xi$ tiene distribuci\'on $G$, entonces el tiempo $T_{n}$ de la $n$-\'esima renovaci\'on tiene distribuci\'on $G\star F^{\left(n-1\right)\star}\left(t\right)$
\end{Note}


\begin{Teo}
Para una constante $\mu\leq\infty$ ( o variable aleatoria), las siguientes expresiones son equivalentes:

\begin{eqnarray}
lim_{n\rightarrow\infty}n^{-1}T_{n}&=&\mu,\textrm{ c.s.}\\
lim_{t\rightarrow\infty}t^{-1}N\left(t\right)&=&1/\mu,\textrm{ c.s.}
\end{eqnarray}
\end{Teo}


Es decir, $T_{n}$ satisface la Ley Fuerte de los Grandes N\'umeros s\'i y s\'olo s\'i $N\left/t\right)$ la cumple.


\begin{Coro}[Ley Fuerte de los Grandes N\'umeros para Procesos de Renovaci\'on]
Si $N\left(t\right)$ es un proceso de renovaci\'on cuyos tiempos de inter-renovaci\'on tienen media $\mu\leq\infty$, entonces
\begin{eqnarray}
t^{-1}N\left(t\right)\rightarrow 1/\mu,\textrm{ c.s. cuando }t\rightarrow\infty.
\end{eqnarray}

\end{Coro}


Considerar el proceso estoc\'astico de valores reales $\left\{Z\left(t\right):t\geq0\right\}$ en el mismo espacio de probabilidad que $N\left(t\right)$

\begin{Def}
Para el proceso $\left\{Z\left(t\right):t\geq0\right\}$ se define la fluctuaci\'on m\'axima de $Z\left(t\right)$ en el intervalo $\left(T_{n-1},T_{n}\right]$:
\begin{eqnarray*}
M_{n}=\sup_{T_{n-1}<t\leq T_{n}}|Z\left(t\right)-Z\left(T_{n-1}\right)|
\end{eqnarray*}
\end{Def}

\begin{Teo}
Sup\'ongase que $n^{-1}T_{n}\rightarrow\mu$ c.s. cuando $n\rightarrow\infty$, donde $\mu\leq\infty$ es una constante o variable aleatoria. Sea $a$ una constante o variable aleatoria que puede ser infinita cuando $\mu$ es finita, y considere las expresiones l\'imite:
\begin{eqnarray}
lim_{n\rightarrow\infty}n^{-1}Z\left(T_{n}\right)&=&a,\textrm{ c.s.}\\
lim_{t\rightarrow\infty}t^{-1}Z\left(t\right)&=&a/\mu,\textrm{ c.s.}
\end{eqnarray}
La segunda expresi\'on implica la primera. Conversamente, la primera implica la segunda si el proceso $Z\left(t\right)$ es creciente, o si $lim_{n\rightarrow\infty}n^{-1}M_{n}=0$ c.s.
\end{Teo}

\begin{Coro}
Si $N\left(t\right)$ es un proceso de renovaci\'on, y $\left(Z\left(T_{n}\right)-Z\left(T_{n-1}\right),M_{n}\right)$, para $n\geq1$, son variables aleatorias independientes e id\'enticamente distribuidas con media finita, entonces,
\begin{eqnarray}
lim_{t\rightarrow\infty}t^{-1}Z\left(t\right)\rightarrow\frac{\esp\left[Z\left(T_{1}\right)-Z\left(T_{0}\right)\right]}{\esp\left[T_{1}\right]},\textrm{ c.s. cuando  }t\rightarrow\infty.
\end{eqnarray}
\end{Coro}


Sup\'ongase que $N\left(t\right)$ es un proceso de renovaci\'on con distribuci\'on $F$ con media finita $\mu$.

\begin{Def}
La funci\'on de renovaci\'on asociada con la distribuci\'on $F$, del proceso $N\left(t\right)$, es
\begin{eqnarray*}
U\left(t\right)=\sum_{n=1}^{\infty}F^{n\star}\left(t\right),\textrm{   }t\geq0,
\end{eqnarray*}
donde $F^{0\star}\left(t\right)=\indora\left(t\geq0\right)$.
\end{Def}


\begin{Prop}
Sup\'ongase que la distribuci\'on de inter-renovaci\'on $F$ tiene densidad $f$. Entonces $U\left(t\right)$ tambi\'en tiene densidad, para $t>0$, y es $U^{'}\left(t\right)=\sum_{n=0}^{\infty}f^{n\star}\left(t\right)$. Adem\'as
\begin{eqnarray*}
\prob\left\{N\left(t\right)>N\left(t-\right)\right\}=0\textrm{,   }t\geq0.
\end{eqnarray*}
\end{Prop}

\begin{Def}
La Transformada de Laplace-Stieljes de $F$ est\'a dada por

\begin{eqnarray*}
\hat{F}\left(\alpha\right)=\int_{\rea_{+}}e^{-\alpha t}dF\left(t\right)\textrm{,  }\alpha\geq0.
\end{eqnarray*}
\end{Def}

Entonces

\begin{eqnarray*}
\hat{U}\left(\alpha\right)=\sum_{n=0}^{\infty}\hat{F^{n\star}}\left(\alpha\right)=\sum_{n=0}^{\infty}\hat{F}\left(\alpha\right)^{n}=\frac{1}{1-\hat{F}\left(\alpha\right)}.
\end{eqnarray*}


\begin{Prop}
La Transformada de Laplace $\hat{U}\left(\alpha\right)$ y $\hat{F}\left(\alpha\right)$ determina una a la otra de manera \'unica por la relaci\'on $\hat{U}\left(\alpha\right)=\frac{1}{1-\hat{F}\left(\alpha\right)}$.
\end{Prop}


\begin{Note}
Un proceso de renovaci\'on $N\left(t\right)$ cuyos tiempos de inter-renovaci\'on tienen media finita, es un proceso Poisson con tasa $\lambda$ si y s\'olo s\'i $\esp\left[U\left(t\right)\right]=\lambda t$, para $t\geq0$.
\end{Note}


\begin{Teo}
Sea $N\left(t\right)$ un proceso puntual simple con puntos de localizaci\'on $T_{n}$ tal que $\eta\left(t\right)=\esp\left[N\left(\right)\right]$ es finita para cada $t$. Entonces para cualquier funci\'on $f:\rea_{+}\rightarrow\rea$,
\begin{eqnarray*}
\esp\left[\sum_{n=1}^{N\left(\right)}f\left(T_{n}\right)\right]=\int_{\left(0,t\right]}f\left(s\right)d\eta\left(s\right)\textrm{,  }t\geq0,
\end{eqnarray*}
suponiendo que la integral exista. Adem\'as si $X_{1},X_{2},\ldots$ son variables aleatorias definidas en el mismo espacio de probabilidad que el proceso $N\left(t\right)$ tal que $\esp\left[X_{n}|T_{n}=s\right]=f\left(s\right)$, independiente de $n$. Entonces
\begin{eqnarray*}
\esp\left[\sum_{n=1}^{N\left(t\right)}X_{n}\right]=\int_{\left(0,t\right]}f\left(s\right)d\eta\left(s\right)\textrm{,  }t\geq0,
\end{eqnarray*} 
suponiendo que la integral exista. 
\end{Teo}

\begin{Coro}[Identidad de Wald para Renovaciones]
Para el proceso de renovaci\'on $N\left(t\right)$,
\begin{eqnarray*}
\esp\left[T_{N\left(t\right)+1}\right]=\mu\esp\left[N\left(t\right)+1\right]\textrm{,  }t\geq0,
\end{eqnarray*}  
\end{Coro}


\begin{Def}
Sea $h\left(t\right)$ funci\'on de valores reales en $\rea$ acotada en intervalos finitos e igual a cero para $t<0$ La ecuaci\'on de renovaci\'on para $h\left(t\right)$ y la distribuci\'on $F$ es

\begin{eqnarray}\label{Ec.Renovacion}
H\left(t\right)=h\left(t\right)+\int_{\left[0,t\right]}H\left(t-s\right)dF\left(s\right)\textrm{,    }t\geq0,
\end{eqnarray}
donde $H\left(t\right)$ es una funci\'on de valores reales. Esto es $H=h+F\star H$. Decimos que $H\left(t\right)$ es soluci\'on de esta ecuaci\'on si satisface la ecuaci\'on, y es acotada en intervalos finitos e iguales a cero para $t<0$.
\end{Def}

\begin{Prop}
La funci\'on $U\star h\left(t\right)$ es la \'unica soluci\'on de la ecuaci\'on de renovaci\'on (\ref{Ec.Renovacion}).
\end{Prop}

\begin{Teo}[Teorema Renovaci\'on Elemental]
\begin{eqnarray*}
t^{-1}U\left(t\right)\rightarrow 1/\mu\textrm{,    cuando }t\rightarrow\infty.
\end{eqnarray*}
\end{Teo}



Sup\'ongase que $N\left(t\right)$ es un proceso de renovaci\'on con distribuci\'on $F$ con media finita $\mu$.

\begin{Def}
La funci\'on de renovaci\'on asociada con la distribuci\'on $F$, del proceso $N\left(t\right)$, es
\begin{eqnarray*}
U\left(t\right)=\sum_{n=1}^{\infty}F^{n\star}\left(t\right),\textrm{   }t\geq0,
\end{eqnarray*}
donde $F^{0\star}\left(t\right)=\indora\left(t\geq0\right)$.
\end{Def}


\begin{Prop}
Sup\'ongase que la distribuci\'on de inter-renovaci\'on $F$ tiene densidad $f$. Entonces $U\left(t\right)$ tambi\'en tiene densidad, para $t>0$, y es $U^{'}\left(t\right)=\sum_{n=0}^{\infty}f^{n\star}\left(t\right)$. Adem\'as
\begin{eqnarray*}
\prob\left\{N\left(t\right)>N\left(t-\right)\right\}=0\textrm{,   }t\geq0.
\end{eqnarray*}
\end{Prop}

\begin{Def}
La Transformada de Laplace-Stieljes de $F$ est\'a dada por

\begin{eqnarray*}
\hat{F}\left(\alpha\right)=\int_{\rea_{+}}e^{-\alpha t}dF\left(t\right)\textrm{,  }\alpha\geq0.
\end{eqnarray*}
\end{Def}

Entonces

\begin{eqnarray*}
\hat{U}\left(\alpha\right)=\sum_{n=0}^{\infty}\hat{F^{n\star}}\left(\alpha\right)=\sum_{n=0}^{\infty}\hat{F}\left(\alpha\right)^{n}=\frac{1}{1-\hat{F}\left(\alpha\right)}.
\end{eqnarray*}


\begin{Prop}
La Transformada de Laplace $\hat{U}\left(\alpha\right)$ y $\hat{F}\left(\alpha\right)$ determina una a la otra de manera \'unica por la relaci\'on $\hat{U}\left(\alpha\right)=\frac{1}{1-\hat{F}\left(\alpha\right)}$.
\end{Prop}


\begin{Note}
Un proceso de renovaci\'on $N\left(t\right)$ cuyos tiempos de inter-renovaci\'on tienen media finita, es un proceso Poisson con tasa $\lambda$ si y s\'olo s\'i $\esp\left[U\left(t\right)\right]=\lambda t$, para $t\geq0$.
\end{Note}


\begin{Teo}
Sea $N\left(t\right)$ un proceso puntual simple con puntos de localizaci\'on $T_{n}$ tal que $\eta\left(t\right)=\esp\left[N\left(\right)\right]$ es finita para cada $t$. Entonces para cualquier funci\'on $f:\rea_{+}\rightarrow\rea$,
\begin{eqnarray*}
\esp\left[\sum_{n=1}^{N\left(\right)}f\left(T_{n}\right)\right]=\int_{\left(0,t\right]}f\left(s\right)d\eta\left(s\right)\textrm{,  }t\geq0,
\end{eqnarray*}
suponiendo que la integral exista. Adem\'as si $X_{1},X_{2},\ldots$ son variables aleatorias definidas en el mismo espacio de probabilidad que el proceso $N\left(t\right)$ tal que $\esp\left[X_{n}|T_{n}=s\right]=f\left(s\right)$, independiente de $n$. Entonces
\begin{eqnarray*}
\esp\left[\sum_{n=1}^{N\left(t\right)}X_{n}\right]=\int_{\left(0,t\right]}f\left(s\right)d\eta\left(s\right)\textrm{,  }t\geq0,
\end{eqnarray*} 
suponiendo que la integral exista. 
\end{Teo}

\begin{Coro}[Identidad de Wald para Renovaciones]
Para el proceso de renovaci\'on $N\left(t\right)$,
\begin{eqnarray*}
\esp\left[T_{N\left(t\right)+1}\right]=\mu\esp\left[N\left(t\right)+1\right]\textrm{,  }t\geq0,
\end{eqnarray*}  
\end{Coro}


\begin{Def}
Sea $h\left(t\right)$ funci\'on de valores reales en $\rea$ acotada en intervalos finitos e igual a cero para $t<0$ La ecuaci\'on de renovaci\'on para $h\left(t\right)$ y la distribuci\'on $F$ es

\begin{eqnarray}\label{Ec.Renovacion}
H\left(t\right)=h\left(t\right)+\int_{\left[0,t\right]}H\left(t-s\right)dF\left(s\right)\textrm{,    }t\geq0,
\end{eqnarray}
donde $H\left(t\right)$ es una funci\'on de valores reales. Esto es $H=h+F\star H$. Decimos que $H\left(t\right)$ es soluci\'on de esta ecuaci\'on si satisface la ecuaci\'on, y es acotada en intervalos finitos e iguales a cero para $t<0$.
\end{Def}

\begin{Prop}
La funci\'on $U\star h\left(t\right)$ es la \'unica soluci\'on de la ecuaci\'on de renovaci\'on (\ref{Ec.Renovacion}).
\end{Prop}

\begin{Teo}[Teorema Renovaci\'on Elemental]
\begin{eqnarray*}
t^{-1}U\left(t\right)\rightarrow 1/\mu\textrm{,    cuando }t\rightarrow\infty.
\end{eqnarray*}
\end{Teo}


\begin{Note} Una funci\'on $h:\rea_{+}\rightarrow\rea$ es Directamente Riemann Integrable en los siguientes casos:
\begin{itemize}
\item[a)] $h\left(t\right)\geq0$ es decreciente y Riemann Integrable.
\item[b)] $h$ es continua excepto posiblemente en un conjunto de Lebesgue de medida 0, y $|h\left(t\right)|\leq b\left(t\right)$, donde $b$ es DRI.
\end{itemize}
\end{Note}

\begin{Teo}[Teorema Principal de Renovaci\'on]
Si $F$ es no aritm\'etica y $h\left(t\right)$ es Directamente Riemann Integrable (DRI), entonces

\begin{eqnarray*}
lim_{t\rightarrow\infty}U\star h=\frac{1}{\mu}\int_{\rea_{+}}h\left(s\right)ds.
\end{eqnarray*}
\end{Teo}

\begin{Prop}
Cualquier funci\'on $H\left(t\right)$ acotada en intervalos finitos y que es 0 para $t<0$ puede expresarse como
\begin{eqnarray*}
H\left(t\right)=U\star h\left(t\right)\textrm{,  donde }h\left(t\right)=H\left(t\right)-F\star H\left(t\right)
\end{eqnarray*}
\end{Prop}

\begin{Def}
Un proceso estoc\'astico $X\left(t\right)$ es crudamente regenerativo en un tiempo aleatorio positivo $T$ si
\begin{eqnarray*}
\esp\left[X\left(T+t\right)|T\right]=\esp\left[X\left(t\right)\right]\textrm{, para }t\geq0,\end{eqnarray*}
y con las esperanzas anteriores finitas.
\end{Def}

\begin{Prop}
Sup\'ongase que $X\left(t\right)$ es un proceso crudamente regenerativo en $T$, que tiene distribuci\'on $F$. Si $\esp\left[X\left(t\right)\right]$ es acotado en intervalos finitos, entonces
\begin{eqnarray*}
\esp\left[X\left(t\right)\right]=U\star h\left(t\right)\textrm{,  donde }h\left(t\right)=\esp\left[X\left(t\right)\indora\left(T>t\right)\right].
\end{eqnarray*}
\end{Prop}

\begin{Teo}[Regeneraci\'on Cruda]
Sup\'ongase que $X\left(t\right)$ es un proceso con valores positivo crudamente regenerativo en $T$, y def\'inase $M=\sup\left\{|X\left(t\right)|:t\leq T\right\}$. Si $T$ es no aritm\'etico y $M$ y $MT$ tienen media finita, entonces
\begin{eqnarray*}
lim_{t\rightarrow\infty}\esp\left[X\left(t\right)\right]=\frac{1}{\mu}\int_{\rea_{+}}h\left(s\right)ds,
\end{eqnarray*}
donde $h\left(t\right)=\esp\left[X\left(t\right)\indora\left(T>t\right)\right]$.
\end{Teo}


\begin{Note} Una funci\'on $h:\rea_{+}\rightarrow\rea$ es Directamente Riemann Integrable en los siguientes casos:
\begin{itemize}
\item[a)] $h\left(t\right)\geq0$ es decreciente y Riemann Integrable.
\item[b)] $h$ es continua excepto posiblemente en un conjunto de Lebesgue de medida 0, y $|h\left(t\right)|\leq b\left(t\right)$, donde $b$ es DRI.
\end{itemize}
\end{Note}

\begin{Teo}[Teorema Principal de Renovaci\'on]
Si $F$ es no aritm\'etica y $h\left(t\right)$ es Directamente Riemann Integrable (DRI), entonces

\begin{eqnarray*}
lim_{t\rightarrow\infty}U\star h=\frac{1}{\mu}\int_{\rea_{+}}h\left(s\right)ds.
\end{eqnarray*}
\end{Teo}

\begin{Prop}
Cualquier funci\'on $H\left(t\right)$ acotada en intervalos finitos y que es 0 para $t<0$ puede expresarse como
\begin{eqnarray*}
H\left(t\right)=U\star h\left(t\right)\textrm{,  donde }h\left(t\right)=H\left(t\right)-F\star H\left(t\right)
\end{eqnarray*}
\end{Prop}

\begin{Def}
Un proceso estoc\'astico $X\left(t\right)$ es crudamente regenerativo en un tiempo aleatorio positivo $T$ si
\begin{eqnarray*}
\esp\left[X\left(T+t\right)|T\right]=\esp\left[X\left(t\right)\right]\textrm{, para }t\geq0,\end{eqnarray*}
y con las esperanzas anteriores finitas.
\end{Def}

\begin{Prop}
Sup\'ongase que $X\left(t\right)$ es un proceso crudamente regenerativo en $T$, que tiene distribuci\'on $F$. Si $\esp\left[X\left(t\right)\right]$ es acotado en intervalos finitos, entonces
\begin{eqnarray*}
\esp\left[X\left(t\right)\right]=U\star h\left(t\right)\textrm{,  donde }h\left(t\right)=\esp\left[X\left(t\right)\indora\left(T>t\right)\right].
\end{eqnarray*}
\end{Prop}

\begin{Teo}[Regeneraci\'on Cruda]
Sup\'ongase que $X\left(t\right)$ es un proceso con valores positivo crudamente regenerativo en $T$, y def\'inase $M=\sup\left\{|X\left(t\right)|:t\leq T\right\}$. Si $T$ es no aritm\'etico y $M$ y $MT$ tienen media finita, entonces
\begin{eqnarray*}
lim_{t\rightarrow\infty}\esp\left[X\left(t\right)\right]=\frac{1}{\mu}\int_{\rea_{+}}h\left(s\right)ds,
\end{eqnarray*}
donde $h\left(t\right)=\esp\left[X\left(t\right)\indora\left(T>t\right)\right]$.
\end{Teo}

%________________________________________________________________________
\subsection{Procesos Regenerativos}
%________________________________________________________________________

Para $\left\{X\left(t\right):t\geq0\right\}$ Proceso Estoc\'astico a tiempo continuo con estado de espacios $S$, que es un espacio m\'etrico, con trayectorias continuas por la derecha y con l\'imites por la izquierda c.s. Sea $N\left(t\right)$ un proceso de renovaci\'on en $\rea_{+}$ definido en el mismo espacio de probabilidad que $X\left(t\right)$, con tiempos de renovaci\'on $T$ y tiempos de inter-renovaci\'on $\xi_{n}=T_{n}-T_{n-1}$, con misma distribuci\'on $F$ de media finita $\mu$.



\begin{Def}
Para el proceso $\left\{\left(N\left(t\right),X\left(t\right)\right):t\geq0\right\}$, sus trayectoria muestrales en el intervalo de tiempo $\left[T_{n-1},T_{n}\right)$ est\'an descritas por
\begin{eqnarray*}
\zeta_{n}=\left(\xi_{n},\left\{X\left(T_{n-1}+t\right):0\leq t<\xi_{n}\right\}\right)
\end{eqnarray*}
Este $\zeta_{n}$ es el $n$-\'esimo segmento del proceso. El proceso es regenerativo sobre los tiempos $T_{n}$ si sus segmentos $\zeta_{n}$ son independientes e id\'enticamennte distribuidos.
\end{Def}


\begin{Obs}
Si $\tilde{X}\left(t\right)$ con espacio de estados $\tilde{S}$ es regenerativo sobre $T_{n}$, entonces $X\left(t\right)=f\left(\tilde{X}\left(t\right)\right)$ tambi\'en es regenerativo sobre $T_{n}$, para cualquier funci\'on $f:\tilde{S}\rightarrow S$.
\end{Obs}

\begin{Obs}
Los procesos regenerativos son crudamente regenerativos, pero no al rev\'es.
\end{Obs}

\begin{Def}[Definici\'on Cl\'asica]
Un proceso estoc\'astico $X=\left\{X\left(t\right):t\geq0\right\}$ es llamado regenerativo is existe una variable aleatoria $R_{1}>0$ tal que
\begin{itemize}
\item[i)] $\left\{X\left(t+R_{1}\right):t\geq0\right\}$ es independiente de $\left\{\left\{X\left(t\right):t<R_{1}\right\},\right\}$
\item[ii)] $\left\{X\left(t+R_{1}\right):t\geq0\right\}$ es estoc\'asticamente equivalente a $\left\{X\left(t\right):t>0\right\}$
\end{itemize}

Llamamos a $R_{1}$ tiempo de regeneraci\'on, y decimos que $X$ se regenera en este punto.
\end{Def}

$\left\{X\left(t+R_{1}\right)\right\}$ es regenerativo con tiempo de regeneraci\'on $R_{2}$, independiente de $R_{1}$ pero con la misma distribuci\'on que $R_{1}$. Procediendo de esta manera se obtiene una secuencia de variables aleatorias independientes e id\'enticamente distribuidas $\left\{R_{n}\right\}$ llamados longitudes de ciclo. Si definimos a $Z_{k}\equiv R_{1}+R_{2}+\cdots+R_{k}$, se tiene un proceso de renovaci\'on llamado proceso de renovaci\'on encajado para $X$.

\begin{Note}
Un proceso regenerativo con media de la longitud de ciclo finita es llamado positivo recurrente.
\end{Note}


\begin{Def}
Para $x$ fijo y para cada $t\geq0$, sea $I_{x}\left(t\right)=1$ si $X\left(t\right)\leq x$,  $I_{x}\left(t\right)=0$ en caso contrario, y def\'inanse los tiempos promedio
\begin{eqnarray*}
\overline{X}&=&lim_{t\rightarrow\infty}\frac{1}{t}\int_{0}^{\infty}X\left(u\right)du\\
\prob\left(X_{\infty}\leq x\right)&=&lim_{t\rightarrow\infty}\frac{1}{t}\int_{0}^{\infty}I_{x}\left(u\right)du,
\end{eqnarray*}
cuando estos l\'imites existan.
\end{Def}

Como consecuencia del teorema de Renovaci\'on-Recompensa, se tiene que el primer l\'imite  existe y es igual a la constante
\begin{eqnarray*}
\overline{X}&=&\frac{\esp\left[\int_{0}^{R_{1}}X\left(t\right)dt\right]}{\esp\left[R_{1}\right]},
\end{eqnarray*}
suponiendo que ambas esperanzas son finitas.

\begin{Note}
\begin{itemize}
\item[a)] Si el proceso regenerativo $X$ es positivo recurrente y tiene trayectorias muestrales no negativas, entonces la ecuaci\'on anterior es v\'alida.
\item[b)] Si $X$ es positivo recurrente regenerativo, podemos construir una \'unica versi\'on estacionaria de este proceso, $X_{e}=\left\{X_{e}\left(t\right)\right\}$, donde $X_{e}$ es un proceso estoc\'astico regenerativo y estrictamente estacionario, con distribuci\'on marginal distribuida como $X_{\infty}$
\end{itemize}
\end{Note}

%________________________________________________________________________
\subsection{Procesos Regenerativos}
%________________________________________________________________________

Para $\left\{X\left(t\right):t\geq0\right\}$ Proceso Estoc\'astico a tiempo continuo con estado de espacios $S$, que es un espacio m\'etrico, con trayectorias continuas por la derecha y con l\'imites por la izquierda c.s. Sea $N\left(t\right)$ un proceso de renovaci\'on en $\rea_{+}$ definido en el mismo espacio de probabilidad que $X\left(t\right)$, con tiempos de renovaci\'on $T$ y tiempos de inter-renovaci\'on $\xi_{n}=T_{n}-T_{n-1}$, con misma distribuci\'on $F$ de media finita $\mu$.



\begin{Def}
Para el proceso $\left\{\left(N\left(t\right),X\left(t\right)\right):t\geq0\right\}$, sus trayectoria muestrales en el intervalo de tiempo $\left[T_{n-1},T_{n}\right)$ est\'an descritas por
\begin{eqnarray*}
\zeta_{n}=\left(\xi_{n},\left\{X\left(T_{n-1}+t\right):0\leq t<\xi_{n}\right\}\right)
\end{eqnarray*}
Este $\zeta_{n}$ es el $n$-\'esimo segmento del proceso. El proceso es regenerativo sobre los tiempos $T_{n}$ si sus segmentos $\zeta_{n}$ son independientes e id\'enticamennte distribuidos.
\end{Def}


\begin{Obs}
Si $\tilde{X}\left(t\right)$ con espacio de estados $\tilde{S}$ es regenerativo sobre $T_{n}$, entonces $X\left(t\right)=f\left(\tilde{X}\left(t\right)\right)$ tambi\'en es regenerativo sobre $T_{n}$, para cualquier funci\'on $f:\tilde{S}\rightarrow S$.
\end{Obs}

\begin{Obs}
Los procesos regenerativos son crudamente regenerativos, pero no al rev\'es.
\end{Obs}

\begin{Def}[Definici\'on Cl\'asica]
Un proceso estoc\'astico $X=\left\{X\left(t\right):t\geq0\right\}$ es llamado regenerativo is existe una variable aleatoria $R_{1}>0$ tal que
\begin{itemize}
\item[i)] $\left\{X\left(t+R_{1}\right):t\geq0\right\}$ es independiente de $\left\{\left\{X\left(t\right):t<R_{1}\right\},\right\}$
\item[ii)] $\left\{X\left(t+R_{1}\right):t\geq0\right\}$ es estoc\'asticamente equivalente a $\left\{X\left(t\right):t>0\right\}$
\end{itemize}

Llamamos a $R_{1}$ tiempo de regeneraci\'on, y decimos que $X$ se regenera en este punto.
\end{Def}

$\left\{X\left(t+R_{1}\right)\right\}$ es regenerativo con tiempo de regeneraci\'on $R_{2}$, independiente de $R_{1}$ pero con la misma distribuci\'on que $R_{1}$. Procediendo de esta manera se obtiene una secuencia de variables aleatorias independientes e id\'enticamente distribuidas $\left\{R_{n}\right\}$ llamados longitudes de ciclo. Si definimos a $Z_{k}\equiv R_{1}+R_{2}+\cdots+R_{k}$, se tiene un proceso de renovaci\'on llamado proceso de renovaci\'on encajado para $X$.

\begin{Note}
Un proceso regenerativo con media de la longitud de ciclo finita es llamado positivo recurrente.
\end{Note}


\begin{Def}
Para $x$ fijo y para cada $t\geq0$, sea $I_{x}\left(t\right)=1$ si $X\left(t\right)\leq x$,  $I_{x}\left(t\right)=0$ en caso contrario, y def\'inanse los tiempos promedio
\begin{eqnarray*}
\overline{X}&=&lim_{t\rightarrow\infty}\frac{1}{t}\int_{0}^{\infty}X\left(u\right)du\\
\prob\left(X_{\infty}\leq x\right)&=&lim_{t\rightarrow\infty}\frac{1}{t}\int_{0}^{\infty}I_{x}\left(u\right)du,
\end{eqnarray*}
cuando estos l\'imites existan.
\end{Def}

Como consecuencia del teorema de Renovaci\'on-Recompensa, se tiene que el primer l\'imite  existe y es igual a la constante
\begin{eqnarray*}
\overline{X}&=&\frac{\esp\left[\int_{0}^{R_{1}}X\left(t\right)dt\right]}{\esp\left[R_{1}\right]},
\end{eqnarray*}
suponiendo que ambas esperanzas son finitas.

\begin{Note}
\begin{itemize}
\item[a)] Si el proceso regenerativo $X$ es positivo recurrente y tiene trayectorias muestrales no negativas, entonces la ecuaci\'on anterior es v\'alida.
\item[b)] Si $X$ es positivo recurrente regenerativo, podemos construir una \'unica versi\'on estacionaria de este proceso, $X_{e}=\left\{X_{e}\left(t\right)\right\}$, donde $X_{e}$ es un proceso estoc\'astico regenerativo y estrictamente estacionario, con distribuci\'on marginal distribuida como $X_{\infty}$
\end{itemize}
\end{Note}
%__________________________________________________________________________________________
\subsection{Procesos Regenerativos Estacionarios - Stidham \cite{Stidham}}
%__________________________________________________________________________________________


Un proceso estoc\'astico a tiempo continuo $\left\{V\left(t\right),t\geq0\right\}$ es un proceso regenerativo si existe una sucesi\'on de variables aleatorias independientes e id\'enticamente distribuidas $\left\{X_{1},X_{2},\ldots\right\}$, sucesi\'on de renovaci\'on, tal que para cualquier conjunto de Borel $A$, 

\begin{eqnarray*}
\prob\left\{V\left(t\right)\in A|X_{1}+X_{2}+\cdots+X_{R\left(t\right)}=s,\left\{V\left(\tau\right),\tau<s\right\}\right\}=\prob\left\{V\left(t-s\right)\in A|X_{1}>t-s\right\},
\end{eqnarray*}
para todo $0\leq s\leq t$, donde $R\left(t\right)=\max\left\{X_{1}+X_{2}+\cdots+X_{j}\leq t\right\}=$n\'umero de renovaciones ({\emph{puntos de regeneraci\'on}}) que ocurren en $\left[0,t\right]$. El intervalo $\left[0,X_{1}\right)$ es llamado {\emph{primer ciclo de regeneraci\'on}} de $\left\{V\left(t \right),t\geq0\right\}$, $\left[X_{1},X_{1}+X_{2}\right)$ el {\emph{segundo ciclo de regeneraci\'on}}, y as\'i sucesivamente.

Sea $X=X_{1}$ y sea $F$ la funci\'on de distrbuci\'on de $X$


\begin{Def}
Se define el proceso estacionario, $\left\{V^{*}\left(t\right),t\geq0\right\}$, para $\left\{V\left(t\right),t\geq0\right\}$ por

\begin{eqnarray*}
\prob\left\{V\left(t\right)\in A\right\}=\frac{1}{\esp\left[X\right]}\int_{0}^{\infty}\prob\left\{V\left(t+x\right)\in A|X>x\right\}\left(1-F\left(x\right)\right)dx,
\end{eqnarray*} 
para todo $t\geq0$ y todo conjunto de Borel $A$.
\end{Def}

\begin{Def}
Una distribuci\'on se dice que es {\emph{aritm\'etica}} si todos sus puntos de incremento son m\'ultiplos de la forma $0,\lambda, 2\lambda,\ldots$ para alguna $\lambda>0$ entera.
\end{Def}


\begin{Def}
Una modificaci\'on medible de un proceso $\left\{V\left(t\right),t\geq0\right\}$, es una versi\'on de este, $\left\{V\left(t,w\right)\right\}$ conjuntamente medible para $t\geq0$ y para $w\in S$, $S$ espacio de estados para $\left\{V\left(t\right),t\geq0\right\}$.
\end{Def}

\begin{Teo}
Sea $\left\{V\left(t\right),t\geq\right\}$ un proceso regenerativo no negativo con modificaci\'on medible. Sea $\esp\left[X\right]<\infty$. Entonces el proceso estacionario dado por la ecuaci\'on anterior est\'a bien definido y tiene funci\'on de distribuci\'on independiente de $t$, adem\'as
\begin{itemize}
\item[i)] \begin{eqnarray*}
\esp\left[V^{*}\left(0\right)\right]&=&\frac{\esp\left[\int_{0}^{X}V\left(s\right)ds\right]}{\esp\left[X\right]}\end{eqnarray*}
\item[ii)] Si $\esp\left[V^{*}\left(0\right)\right]<\infty$, equivalentemente, si $\esp\left[\int_{0}^{X}V\left(s\right)ds\right]<\infty$,entonces
\begin{eqnarray*}
\frac{\int_{0}^{t}V\left(s\right)ds}{t}\rightarrow\frac{\esp\left[\int_{0}^{X}V\left(s\right)ds\right]}{\esp\left[X\right]}
\end{eqnarray*}
con probabilidad 1 y en media, cuando $t\rightarrow\infty$.
\end{itemize}
\end{Teo}


%__________________________________________________________________________________________
\subsection{Procesos Regenerativos Estacionarios - Stidham \cite{Stidham}}
%__________________________________________________________________________________________


Un proceso estoc\'astico a tiempo continuo $\left\{V\left(t\right),t\geq0\right\}$ es un proceso regenerativo si existe una sucesi\'on de variables aleatorias independientes e id\'enticamente distribuidas $\left\{X_{1},X_{2},\ldots\right\}$, sucesi\'on de renovaci\'on, tal que para cualquier conjunto de Borel $A$, 

\begin{eqnarray*}
\prob\left\{V\left(t\right)\in A|X_{1}+X_{2}+\cdots+X_{R\left(t\right)}=s,\left\{V\left(\tau\right),\tau<s\right\}\right\}=\prob\left\{V\left(t-s\right)\in A|X_{1}>t-s\right\},
\end{eqnarray*}
para todo $0\leq s\leq t$, donde $R\left(t\right)=\max\left\{X_{1}+X_{2}+\cdots+X_{j}\leq t\right\}=$n\'umero de renovaciones ({\emph{puntos de regeneraci\'on}}) que ocurren en $\left[0,t\right]$. El intervalo $\left[0,X_{1}\right)$ es llamado {\emph{primer ciclo de regeneraci\'on}} de $\left\{V\left(t \right),t\geq0\right\}$, $\left[X_{1},X_{1}+X_{2}\right)$ el {\emph{segundo ciclo de regeneraci\'on}}, y as\'i sucesivamente.

Sea $X=X_{1}$ y sea $F$ la funci\'on de distrbuci\'on de $X$


\begin{Def}
Se define el proceso estacionario, $\left\{V^{*}\left(t\right),t\geq0\right\}$, para $\left\{V\left(t\right),t\geq0\right\}$ por

\begin{eqnarray*}
\prob\left\{V\left(t\right)\in A\right\}=\frac{1}{\esp\left[X\right]}\int_{0}^{\infty}\prob\left\{V\left(t+x\right)\in A|X>x\right\}\left(1-F\left(x\right)\right)dx,
\end{eqnarray*} 
para todo $t\geq0$ y todo conjunto de Borel $A$.
\end{Def}

\begin{Def}
Una distribuci\'on se dice que es {\emph{aritm\'etica}} si todos sus puntos de incremento son m\'ultiplos de la forma $0,\lambda, 2\lambda,\ldots$ para alguna $\lambda>0$ entera.
\end{Def}


\begin{Def}
Una modificaci\'on medible de un proceso $\left\{V\left(t\right),t\geq0\right\}$, es una versi\'on de este, $\left\{V\left(t,w\right)\right\}$ conjuntamente medible para $t\geq0$ y para $w\in S$, $S$ espacio de estados para $\left\{V\left(t\right),t\geq0\right\}$.
\end{Def}

\begin{Teo}
Sea $\left\{V\left(t\right),t\geq\right\}$ un proceso regenerativo no negativo con modificaci\'on medible. Sea $\esp\left[X\right]<\infty$. Entonces el proceso estacionario dado por la ecuaci\'on anterior est\'a bien definido y tiene funci\'on de distribuci\'on independiente de $t$, adem\'as
\begin{itemize}
\item[i)] \begin{eqnarray*}
\esp\left[V^{*}\left(0\right)\right]&=&\frac{\esp\left[\int_{0}^{X}V\left(s\right)ds\right]}{\esp\left[X\right]}\end{eqnarray*}
\item[ii)] Si $\esp\left[V^{*}\left(0\right)\right]<\infty$, equivalentemente, si $\esp\left[\int_{0}^{X}V\left(s\right)ds\right]<\infty$,entonces
\begin{eqnarray*}
\frac{\int_{0}^{t}V\left(s\right)ds}{t}\rightarrow\frac{\esp\left[\int_{0}^{X}V\left(s\right)ds\right]}{\esp\left[X\right]}
\end{eqnarray*}
con probabilidad 1 y en media, cuando $t\rightarrow\infty$.
\end{itemize}
\end{Teo}
%___________________________________________________________________________________________
%
\subsection{Propiedades de los Procesos de Renovaci\'on}
%___________________________________________________________________________________________
%

Los tiempos $T_{n}$ est\'an relacionados con los conteos de $N\left(t\right)$ por

\begin{eqnarray*}
\left\{N\left(t\right)\geq n\right\}&=&\left\{T_{n}\leq t\right\}\\
T_{N\left(t\right)}\leq &t&<T_{N\left(t\right)+1},
\end{eqnarray*}

adem\'as $N\left(T_{n}\right)=n$, y 

\begin{eqnarray*}
N\left(t\right)=\max\left\{n:T_{n}\leq t\right\}=\min\left\{n:T_{n+1}>t\right\}
\end{eqnarray*}

Por propiedades de la convoluci\'on se sabe que

\begin{eqnarray*}
P\left\{T_{n}\leq t\right\}=F^{n\star}\left(t\right)
\end{eqnarray*}
que es la $n$-\'esima convoluci\'on de $F$. Entonces 

\begin{eqnarray*}
\left\{N\left(t\right)\geq n\right\}&=&\left\{T_{n}\leq t\right\}\\
P\left\{N\left(t\right)\leq n\right\}&=&1-F^{\left(n+1\right)\star}\left(t\right)
\end{eqnarray*}

Adem\'as usando el hecho de que $\esp\left[N\left(t\right)\right]=\sum_{n=1}^{\infty}P\left\{N\left(t\right)\geq n\right\}$
se tiene que

\begin{eqnarray*}
\esp\left[N\left(t\right)\right]=\sum_{n=1}^{\infty}F^{n\star}\left(t\right)
\end{eqnarray*}

\begin{Prop}
Para cada $t\geq0$, la funci\'on generadora de momentos $\esp\left[e^{\alpha N\left(t\right)}\right]$ existe para alguna $\alpha$ en una vecindad del 0, y de aqu\'i que $\esp\left[N\left(t\right)^{m}\right]<\infty$, para $m\geq1$.
\end{Prop}


\begin{Note}
Si el primer tiempo de renovaci\'on $\xi_{1}$ no tiene la misma distribuci\'on que el resto de las $\xi_{n}$, para $n\geq2$, a $N\left(t\right)$ se le llama Proceso de Renovaci\'on retardado, donde si $\xi$ tiene distribuci\'on $G$, entonces el tiempo $T_{n}$ de la $n$-\'esima renovaci\'on tiene distribuci\'on $G\star F^{\left(n-1\right)\star}\left(t\right)$
\end{Note}


\begin{Teo}
Para una constante $\mu\leq\infty$ ( o variable aleatoria), las siguientes expresiones son equivalentes:

\begin{eqnarray}
lim_{n\rightarrow\infty}n^{-1}T_{n}&=&\mu,\textrm{ c.s.}\\
lim_{t\rightarrow\infty}t^{-1}N\left(t\right)&=&1/\mu,\textrm{ c.s.}
\end{eqnarray}
\end{Teo}


Es decir, $T_{n}$ satisface la Ley Fuerte de los Grandes N\'umeros s\'i y s\'olo s\'i $N\left/t\right)$ la cumple.


\begin{Coro}[Ley Fuerte de los Grandes N\'umeros para Procesos de Renovaci\'on]
Si $N\left(t\right)$ es un proceso de renovaci\'on cuyos tiempos de inter-renovaci\'on tienen media $\mu\leq\infty$, entonces
\begin{eqnarray}
t^{-1}N\left(t\right)\rightarrow 1/\mu,\textrm{ c.s. cuando }t\rightarrow\infty.
\end{eqnarray}

\end{Coro}


Considerar el proceso estoc\'astico de valores reales $\left\{Z\left(t\right):t\geq0\right\}$ en el mismo espacio de probabilidad que $N\left(t\right)$

\begin{Def}
Para el proceso $\left\{Z\left(t\right):t\geq0\right\}$ se define la fluctuaci\'on m\'axima de $Z\left(t\right)$ en el intervalo $\left(T_{n-1},T_{n}\right]$:
\begin{eqnarray*}
M_{n}=\sup_{T_{n-1}<t\leq T_{n}}|Z\left(t\right)-Z\left(T_{n-1}\right)|
\end{eqnarray*}
\end{Def}

\begin{Teo}
Sup\'ongase que $n^{-1}T_{n}\rightarrow\mu$ c.s. cuando $n\rightarrow\infty$, donde $\mu\leq\infty$ es una constante o variable aleatoria. Sea $a$ una constante o variable aleatoria que puede ser infinita cuando $\mu$ es finita, y considere las expresiones l\'imite:
\begin{eqnarray}
lim_{n\rightarrow\infty}n^{-1}Z\left(T_{n}\right)&=&a,\textrm{ c.s.}\\
lim_{t\rightarrow\infty}t^{-1}Z\left(t\right)&=&a/\mu,\textrm{ c.s.}
\end{eqnarray}
La segunda expresi\'on implica la primera. Conversamente, la primera implica la segunda si el proceso $Z\left(t\right)$ es creciente, o si $lim_{n\rightarrow\infty}n^{-1}M_{n}=0$ c.s.
\end{Teo}

\begin{Coro}
Si $N\left(t\right)$ es un proceso de renovaci\'on, y $\left(Z\left(T_{n}\right)-Z\left(T_{n-1}\right),M_{n}\right)$, para $n\geq1$, son variables aleatorias independientes e id\'enticamente distribuidas con media finita, entonces,
\begin{eqnarray}
lim_{t\rightarrow\infty}t^{-1}Z\left(t\right)\rightarrow\frac{\esp\left[Z\left(T_{1}\right)-Z\left(T_{0}\right)\right]}{\esp\left[T_{1}\right]},\textrm{ c.s. cuando  }t\rightarrow\infty.
\end{eqnarray}
\end{Coro}


%___________________________________________________________________________________________
%
\subsection{Propiedades de los Procesos de Renovaci\'on}
%___________________________________________________________________________________________
%

Los tiempos $T_{n}$ est\'an relacionados con los conteos de $N\left(t\right)$ por

\begin{eqnarray*}
\left\{N\left(t\right)\geq n\right\}&=&\left\{T_{n}\leq t\right\}\\
T_{N\left(t\right)}\leq &t&<T_{N\left(t\right)+1},
\end{eqnarray*}

adem\'as $N\left(T_{n}\right)=n$, y 

\begin{eqnarray*}
N\left(t\right)=\max\left\{n:T_{n}\leq t\right\}=\min\left\{n:T_{n+1}>t\right\}
\end{eqnarray*}

Por propiedades de la convoluci\'on se sabe que

\begin{eqnarray*}
P\left\{T_{n}\leq t\right\}=F^{n\star}\left(t\right)
\end{eqnarray*}
que es la $n$-\'esima convoluci\'on de $F$. Entonces 

\begin{eqnarray*}
\left\{N\left(t\right)\geq n\right\}&=&\left\{T_{n}\leq t\right\}\\
P\left\{N\left(t\right)\leq n\right\}&=&1-F^{\left(n+1\right)\star}\left(t\right)
\end{eqnarray*}

Adem\'as usando el hecho de que $\esp\left[N\left(t\right)\right]=\sum_{n=1}^{\infty}P\left\{N\left(t\right)\geq n\right\}$
se tiene que

\begin{eqnarray*}
\esp\left[N\left(t\right)\right]=\sum_{n=1}^{\infty}F^{n\star}\left(t\right)
\end{eqnarray*}

\begin{Prop}
Para cada $t\geq0$, la funci\'on generadora de momentos $\esp\left[e^{\alpha N\left(t\right)}\right]$ existe para alguna $\alpha$ en una vecindad del 0, y de aqu\'i que $\esp\left[N\left(t\right)^{m}\right]<\infty$, para $m\geq1$.
\end{Prop}


\begin{Note}
Si el primer tiempo de renovaci\'on $\xi_{1}$ no tiene la misma distribuci\'on que el resto de las $\xi_{n}$, para $n\geq2$, a $N\left(t\right)$ se le llama Proceso de Renovaci\'on retardado, donde si $\xi$ tiene distribuci\'on $G$, entonces el tiempo $T_{n}$ de la $n$-\'esima renovaci\'on tiene distribuci\'on $G\star F^{\left(n-1\right)\star}\left(t\right)$
\end{Note}


\begin{Teo}
Para una constante $\mu\leq\infty$ ( o variable aleatoria), las siguientes expresiones son equivalentes:

\begin{eqnarray}
lim_{n\rightarrow\infty}n^{-1}T_{n}&=&\mu,\textrm{ c.s.}\\
lim_{t\rightarrow\infty}t^{-1}N\left(t\right)&=&1/\mu,\textrm{ c.s.}
\end{eqnarray}
\end{Teo}


Es decir, $T_{n}$ satisface la Ley Fuerte de los Grandes N\'umeros s\'i y s\'olo s\'i $N\left/t\right)$ la cumple.


\begin{Coro}[Ley Fuerte de los Grandes N\'umeros para Procesos de Renovaci\'on]
Si $N\left(t\right)$ es un proceso de renovaci\'on cuyos tiempos de inter-renovaci\'on tienen media $\mu\leq\infty$, entonces
\begin{eqnarray}
t^{-1}N\left(t\right)\rightarrow 1/\mu,\textrm{ c.s. cuando }t\rightarrow\infty.
\end{eqnarray}

\end{Coro}


Considerar el proceso estoc\'astico de valores reales $\left\{Z\left(t\right):t\geq0\right\}$ en el mismo espacio de probabilidad que $N\left(t\right)$

\begin{Def}
Para el proceso $\left\{Z\left(t\right):t\geq0\right\}$ se define la fluctuaci\'on m\'axima de $Z\left(t\right)$ en el intervalo $\left(T_{n-1},T_{n}\right]$:
\begin{eqnarray*}
M_{n}=\sup_{T_{n-1}<t\leq T_{n}}|Z\left(t\right)-Z\left(T_{n-1}\right)|
\end{eqnarray*}
\end{Def}

\begin{Teo}
Sup\'ongase que $n^{-1}T_{n}\rightarrow\mu$ c.s. cuando $n\rightarrow\infty$, donde $\mu\leq\infty$ es una constante o variable aleatoria. Sea $a$ una constante o variable aleatoria que puede ser infinita cuando $\mu$ es finita, y considere las expresiones l\'imite:
\begin{eqnarray}
lim_{n\rightarrow\infty}n^{-1}Z\left(T_{n}\right)&=&a,\textrm{ c.s.}\\
lim_{t\rightarrow\infty}t^{-1}Z\left(t\right)&=&a/\mu,\textrm{ c.s.}
\end{eqnarray}
La segunda expresi\'on implica la primera. Conversamente, la primera implica la segunda si el proceso $Z\left(t\right)$ es creciente, o si $lim_{n\rightarrow\infty}n^{-1}M_{n}=0$ c.s.
\end{Teo}

\begin{Coro}
Si $N\left(t\right)$ es un proceso de renovaci\'on, y $\left(Z\left(T_{n}\right)-Z\left(T_{n-1}\right),M_{n}\right)$, para $n\geq1$, son variables aleatorias independientes e id\'enticamente distribuidas con media finita, entonces,
\begin{eqnarray}
lim_{t\rightarrow\infty}t^{-1}Z\left(t\right)\rightarrow\frac{\esp\left[Z\left(T_{1}\right)-Z\left(T_{0}\right)\right]}{\esp\left[T_{1}\right]},\textrm{ c.s. cuando  }t\rightarrow\infty.
\end{eqnarray}
\end{Coro}

%___________________________________________________________________________________________
%
\subsection{Propiedades de los Procesos de Renovaci\'on}
%___________________________________________________________________________________________
%

Los tiempos $T_{n}$ est\'an relacionados con los conteos de $N\left(t\right)$ por

\begin{eqnarray*}
\left\{N\left(t\right)\geq n\right\}&=&\left\{T_{n}\leq t\right\}\\
T_{N\left(t\right)}\leq &t&<T_{N\left(t\right)+1},
\end{eqnarray*}

adem\'as $N\left(T_{n}\right)=n$, y 

\begin{eqnarray*}
N\left(t\right)=\max\left\{n:T_{n}\leq t\right\}=\min\left\{n:T_{n+1}>t\right\}
\end{eqnarray*}

Por propiedades de la convoluci\'on se sabe que

\begin{eqnarray*}
P\left\{T_{n}\leq t\right\}=F^{n\star}\left(t\right)
\end{eqnarray*}
que es la $n$-\'esima convoluci\'on de $F$. Entonces 

\begin{eqnarray*}
\left\{N\left(t\right)\geq n\right\}&=&\left\{T_{n}\leq t\right\}\\
P\left\{N\left(t\right)\leq n\right\}&=&1-F^{\left(n+1\right)\star}\left(t\right)
\end{eqnarray*}

Adem\'as usando el hecho de que $\esp\left[N\left(t\right)\right]=\sum_{n=1}^{\infty}P\left\{N\left(t\right)\geq n\right\}$
se tiene que

\begin{eqnarray*}
\esp\left[N\left(t\right)\right]=\sum_{n=1}^{\infty}F^{n\star}\left(t\right)
\end{eqnarray*}

\begin{Prop}
Para cada $t\geq0$, la funci\'on generadora de momentos $\esp\left[e^{\alpha N\left(t\right)}\right]$ existe para alguna $\alpha$ en una vecindad del 0, y de aqu\'i que $\esp\left[N\left(t\right)^{m}\right]<\infty$, para $m\geq1$.
\end{Prop}


\begin{Note}
Si el primer tiempo de renovaci\'on $\xi_{1}$ no tiene la misma distribuci\'on que el resto de las $\xi_{n}$, para $n\geq2$, a $N\left(t\right)$ se le llama Proceso de Renovaci\'on retardado, donde si $\xi$ tiene distribuci\'on $G$, entonces el tiempo $T_{n}$ de la $n$-\'esima renovaci\'on tiene distribuci\'on $G\star F^{\left(n-1\right)\star}\left(t\right)$
\end{Note}


\begin{Teo}
Para una constante $\mu\leq\infty$ ( o variable aleatoria), las siguientes expresiones son equivalentes:

\begin{eqnarray}
lim_{n\rightarrow\infty}n^{-1}T_{n}&=&\mu,\textrm{ c.s.}\\
lim_{t\rightarrow\infty}t^{-1}N\left(t\right)&=&1/\mu,\textrm{ c.s.}
\end{eqnarray}
\end{Teo}


Es decir, $T_{n}$ satisface la Ley Fuerte de los Grandes N\'umeros s\'i y s\'olo s\'i $N\left/t\right)$ la cumple.


\begin{Coro}[Ley Fuerte de los Grandes N\'umeros para Procesos de Renovaci\'on]
Si $N\left(t\right)$ es un proceso de renovaci\'on cuyos tiempos de inter-renovaci\'on tienen media $\mu\leq\infty$, entonces
\begin{eqnarray}
t^{-1}N\left(t\right)\rightarrow 1/\mu,\textrm{ c.s. cuando }t\rightarrow\infty.
\end{eqnarray}

\end{Coro}


Considerar el proceso estoc\'astico de valores reales $\left\{Z\left(t\right):t\geq0\right\}$ en el mismo espacio de probabilidad que $N\left(t\right)$

\begin{Def}
Para el proceso $\left\{Z\left(t\right):t\geq0\right\}$ se define la fluctuaci\'on m\'axima de $Z\left(t\right)$ en el intervalo $\left(T_{n-1},T_{n}\right]$:
\begin{eqnarray*}
M_{n}=\sup_{T_{n-1}<t\leq T_{n}}|Z\left(t\right)-Z\left(T_{n-1}\right)|
\end{eqnarray*}
\end{Def}

\begin{Teo}
Sup\'ongase que $n^{-1}T_{n}\rightarrow\mu$ c.s. cuando $n\rightarrow\infty$, donde $\mu\leq\infty$ es una constante o variable aleatoria. Sea $a$ una constante o variable aleatoria que puede ser infinita cuando $\mu$ es finita, y considere las expresiones l\'imite:
\begin{eqnarray}
lim_{n\rightarrow\infty}n^{-1}Z\left(T_{n}\right)&=&a,\textrm{ c.s.}\\
lim_{t\rightarrow\infty}t^{-1}Z\left(t\right)&=&a/\mu,\textrm{ c.s.}
\end{eqnarray}
La segunda expresi\'on implica la primera. Conversamente, la primera implica la segunda si el proceso $Z\left(t\right)$ es creciente, o si $lim_{n\rightarrow\infty}n^{-1}M_{n}=0$ c.s.
\end{Teo}

\begin{Coro}
Si $N\left(t\right)$ es un proceso de renovaci\'on, y $\left(Z\left(T_{n}\right)-Z\left(T_{n-1}\right),M_{n}\right)$, para $n\geq1$, son variables aleatorias independientes e id\'enticamente distribuidas con media finita, entonces,
\begin{eqnarray}
lim_{t\rightarrow\infty}t^{-1}Z\left(t\right)\rightarrow\frac{\esp\left[Z\left(T_{1}\right)-Z\left(T_{0}\right)\right]}{\esp\left[T_{1}\right]},\textrm{ c.s. cuando  }t\rightarrow\infty.
\end{eqnarray}
\end{Coro}

%___________________________________________________________________________________________
%
\subsection{Propiedades de los Procesos de Renovaci\'on}
%___________________________________________________________________________________________
%

Los tiempos $T_{n}$ est\'an relacionados con los conteos de $N\left(t\right)$ por

\begin{eqnarray*}
\left\{N\left(t\right)\geq n\right\}&=&\left\{T_{n}\leq t\right\}\\
T_{N\left(t\right)}\leq &t&<T_{N\left(t\right)+1},
\end{eqnarray*}

adem\'as $N\left(T_{n}\right)=n$, y 

\begin{eqnarray*}
N\left(t\right)=\max\left\{n:T_{n}\leq t\right\}=\min\left\{n:T_{n+1}>t\right\}
\end{eqnarray*}

Por propiedades de la convoluci\'on se sabe que

\begin{eqnarray*}
P\left\{T_{n}\leq t\right\}=F^{n\star}\left(t\right)
\end{eqnarray*}
que es la $n$-\'esima convoluci\'on de $F$. Entonces 

\begin{eqnarray*}
\left\{N\left(t\right)\geq n\right\}&=&\left\{T_{n}\leq t\right\}\\
P\left\{N\left(t\right)\leq n\right\}&=&1-F^{\left(n+1\right)\star}\left(t\right)
\end{eqnarray*}

Adem\'as usando el hecho de que $\esp\left[N\left(t\right)\right]=\sum_{n=1}^{\infty}P\left\{N\left(t\right)\geq n\right\}$
se tiene que

\begin{eqnarray*}
\esp\left[N\left(t\right)\right]=\sum_{n=1}^{\infty}F^{n\star}\left(t\right)
\end{eqnarray*}

\begin{Prop}
Para cada $t\geq0$, la funci\'on generadora de momentos $\esp\left[e^{\alpha N\left(t\right)}\right]$ existe para alguna $\alpha$ en una vecindad del 0, y de aqu\'i que $\esp\left[N\left(t\right)^{m}\right]<\infty$, para $m\geq1$.
\end{Prop}


\begin{Note}
Si el primer tiempo de renovaci\'on $\xi_{1}$ no tiene la misma distribuci\'on que el resto de las $\xi_{n}$, para $n\geq2$, a $N\left(t\right)$ se le llama Proceso de Renovaci\'on retardado, donde si $\xi$ tiene distribuci\'on $G$, entonces el tiempo $T_{n}$ de la $n$-\'esima renovaci\'on tiene distribuci\'on $G\star F^{\left(n-1\right)\star}\left(t\right)$
\end{Note}


\begin{Teo}
Para una constante $\mu\leq\infty$ ( o variable aleatoria), las siguientes expresiones son equivalentes:

\begin{eqnarray}
lim_{n\rightarrow\infty}n^{-1}T_{n}&=&\mu,\textrm{ c.s.}\\
lim_{t\rightarrow\infty}t^{-1}N\left(t\right)&=&1/\mu,\textrm{ c.s.}
\end{eqnarray}
\end{Teo}


Es decir, $T_{n}$ satisface la Ley Fuerte de los Grandes N\'umeros s\'i y s\'olo s\'i $N\left/t\right)$ la cumple.


\begin{Coro}[Ley Fuerte de los Grandes N\'umeros para Procesos de Renovaci\'on]
Si $N\left(t\right)$ es un proceso de renovaci\'on cuyos tiempos de inter-renovaci\'on tienen media $\mu\leq\infty$, entonces
\begin{eqnarray}
t^{-1}N\left(t\right)\rightarrow 1/\mu,\textrm{ c.s. cuando }t\rightarrow\infty.
\end{eqnarray}

\end{Coro}


Considerar el proceso estoc\'astico de valores reales $\left\{Z\left(t\right):t\geq0\right\}$ en el mismo espacio de probabilidad que $N\left(t\right)$

\begin{Def}
Para el proceso $\left\{Z\left(t\right):t\geq0\right\}$ se define la fluctuaci\'on m\'axima de $Z\left(t\right)$ en el intervalo $\left(T_{n-1},T_{n}\right]$:
\begin{eqnarray*}
M_{n}=\sup_{T_{n-1}<t\leq T_{n}}|Z\left(t\right)-Z\left(T_{n-1}\right)|
\end{eqnarray*}
\end{Def}

\begin{Teo}
Sup\'ongase que $n^{-1}T_{n}\rightarrow\mu$ c.s. cuando $n\rightarrow\infty$, donde $\mu\leq\infty$ es una constante o variable aleatoria. Sea $a$ una constante o variable aleatoria que puede ser infinita cuando $\mu$ es finita, y considere las expresiones l\'imite:
\begin{eqnarray}
lim_{n\rightarrow\infty}n^{-1}Z\left(T_{n}\right)&=&a,\textrm{ c.s.}\\
lim_{t\rightarrow\infty}t^{-1}Z\left(t\right)&=&a/\mu,\textrm{ c.s.}
\end{eqnarray}
La segunda expresi\'on implica la primera. Conversamente, la primera implica la segunda si el proceso $Z\left(t\right)$ es creciente, o si $lim_{n\rightarrow\infty}n^{-1}M_{n}=0$ c.s.
\end{Teo}

\begin{Coro}
Si $N\left(t\right)$ es un proceso de renovaci\'on, y $\left(Z\left(T_{n}\right)-Z\left(T_{n-1}\right),M_{n}\right)$, para $n\geq1$, son variables aleatorias independientes e id\'enticamente distribuidas con media finita, entonces,
\begin{eqnarray}
lim_{t\rightarrow\infty}t^{-1}Z\left(t\right)\rightarrow\frac{\esp\left[Z\left(T_{1}\right)-Z\left(T_{0}\right)\right]}{\esp\left[T_{1}\right]},\textrm{ c.s. cuando  }t\rightarrow\infty.
\end{eqnarray}
\end{Coro}


%__________________________________________________________________________________________
\subsection{Procesos Regenerativos Estacionarios - Stidham \cite{Stidham}}
%__________________________________________________________________________________________


Un proceso estoc\'astico a tiempo continuo $\left\{V\left(t\right),t\geq0\right\}$ es un proceso regenerativo si existe una sucesi\'on de variables aleatorias independientes e id\'enticamente distribuidas $\left\{X_{1},X_{2},\ldots\right\}$, sucesi\'on de renovaci\'on, tal que para cualquier conjunto de Borel $A$, 

\begin{eqnarray*}
\prob\left\{V\left(t\right)\in A|X_{1}+X_{2}+\cdots+X_{R\left(t\right)}=s,\left\{V\left(\tau\right),\tau<s\right\}\right\}=\prob\left\{V\left(t-s\right)\in A|X_{1}>t-s\right\},
\end{eqnarray*}
para todo $0\leq s\leq t$, donde $R\left(t\right)=\max\left\{X_{1}+X_{2}+\cdots+X_{j}\leq t\right\}=$n\'umero de renovaciones ({\emph{puntos de regeneraci\'on}}) que ocurren en $\left[0,t\right]$. El intervalo $\left[0,X_{1}\right)$ es llamado {\emph{primer ciclo de regeneraci\'on}} de $\left\{V\left(t \right),t\geq0\right\}$, $\left[X_{1},X_{1}+X_{2}\right)$ el {\emph{segundo ciclo de regeneraci\'on}}, y as\'i sucesivamente.

Sea $X=X_{1}$ y sea $F$ la funci\'on de distrbuci\'on de $X$


\begin{Def}
Se define el proceso estacionario, $\left\{V^{*}\left(t\right),t\geq0\right\}$, para $\left\{V\left(t\right),t\geq0\right\}$ por

\begin{eqnarray*}
\prob\left\{V\left(t\right)\in A\right\}=\frac{1}{\esp\left[X\right]}\int_{0}^{\infty}\prob\left\{V\left(t+x\right)\in A|X>x\right\}\left(1-F\left(x\right)\right)dx,
\end{eqnarray*} 
para todo $t\geq0$ y todo conjunto de Borel $A$.
\end{Def}

\begin{Def}
Una distribuci\'on se dice que es {\emph{aritm\'etica}} si todos sus puntos de incremento son m\'ultiplos de la forma $0,\lambda, 2\lambda,\ldots$ para alguna $\lambda>0$ entera.
\end{Def}


\begin{Def}
Una modificaci\'on medible de un proceso $\left\{V\left(t\right),t\geq0\right\}$, es una versi\'on de este, $\left\{V\left(t,w\right)\right\}$ conjuntamente medible para $t\geq0$ y para $w\in S$, $S$ espacio de estados para $\left\{V\left(t\right),t\geq0\right\}$.
\end{Def}

\begin{Teo}
Sea $\left\{V\left(t\right),t\geq\right\}$ un proceso regenerativo no negativo con modificaci\'on medible. Sea $\esp\left[X\right]<\infty$. Entonces el proceso estacionario dado por la ecuaci\'on anterior est\'a bien definido y tiene funci\'on de distribuci\'on independiente de $t$, adem\'as
\begin{itemize}
\item[i)] \begin{eqnarray*}
\esp\left[V^{*}\left(0\right)\right]&=&\frac{\esp\left[\int_{0}^{X}V\left(s\right)ds\right]}{\esp\left[X\right]}\end{eqnarray*}
\item[ii)] Si $\esp\left[V^{*}\left(0\right)\right]<\infty$, equivalentemente, si $\esp\left[\int_{0}^{X}V\left(s\right)ds\right]<\infty$,entonces
\begin{eqnarray*}
\frac{\int_{0}^{t}V\left(s\right)ds}{t}\rightarrow\frac{\esp\left[\int_{0}^{X}V\left(s\right)ds\right]}{\esp\left[X\right]}
\end{eqnarray*}
con probabilidad 1 y en media, cuando $t\rightarrow\infty$.
\end{itemize}
\end{Teo}

%______________________________________________________________________
\subsection{Procesos de Renovaci\'on}
%______________________________________________________________________

\begin{Def}\label{Def.Tn}
Sean $0\leq T_{1}\leq T_{2}\leq \ldots$ son tiempos aleatorios infinitos en los cuales ocurren ciertos eventos. El n\'umero de tiempos $T_{n}$ en el intervalo $\left[0,t\right)$ es

\begin{eqnarray}
N\left(t\right)=\sum_{n=1}^{\infty}\indora\left(T_{n}\leq t\right),
\end{eqnarray}
para $t\geq0$.
\end{Def}

Si se consideran los puntos $T_{n}$ como elementos de $\rea_{+}$, y $N\left(t\right)$ es el n\'umero de puntos en $\rea$. El proceso denotado por $\left\{N\left(t\right):t\geq0\right\}$, denotado por $N\left(t\right)$, es un proceso puntual en $\rea_{+}$. Los $T_{n}$ son los tiempos de ocurrencia, el proceso puntual $N\left(t\right)$ es simple si su n\'umero de ocurrencias son distintas: $0<T_{1}<T_{2}<\ldots$ casi seguramente.

\begin{Def}
Un proceso puntual $N\left(t\right)$ es un proceso de renovaci\'on si los tiempos de interocurrencia $\xi_{n}=T_{n}-T_{n-1}$, para $n\geq1$, son independientes e identicamente distribuidos con distribuci\'on $F$, donde $F\left(0\right)=0$ y $T_{0}=0$. Los $T_{n}$ son llamados tiempos de renovaci\'on, referente a la independencia o renovaci\'on de la informaci\'on estoc\'astica en estos tiempos. Los $\xi_{n}$ son los tiempos de inter-renovaci\'on, y $N\left(t\right)$ es el n\'umero de renovaciones en el intervalo $\left[0,t\right)$
\end{Def}


\begin{Note}
Para definir un proceso de renovaci\'on para cualquier contexto, solamente hay que especificar una distribuci\'on $F$, con $F\left(0\right)=0$, para los tiempos de inter-renovaci\'on. La funci\'on $F$ en turno degune las otra variables aleatorias. De manera formal, existe un espacio de probabilidad y una sucesi\'on de variables aleatorias $\xi_{1},\xi_{2},\ldots$ definidas en este con distribuci\'on $F$. Entonces las otras cantidades son $T_{n}=\sum_{k=1}^{n}\xi_{k}$ y $N\left(t\right)=\sum_{n=1}^{\infty}\indora\left(T_{n}\leq t\right)$, donde $T_{n}\rightarrow\infty$ casi seguramente por la Ley Fuerte de los Grandes Números.
\end{Note}

%___________________________________________________________________________________________
%
\subsection{Teorema Principal de Renovaci\'on}
%___________________________________________________________________________________________
%

\begin{Note} Una funci\'on $h:\rea_{+}\rightarrow\rea$ es Directamente Riemann Integrable en los siguientes casos:
\begin{itemize}
\item[a)] $h\left(t\right)\geq0$ es decreciente y Riemann Integrable.
\item[b)] $h$ es continua excepto posiblemente en un conjunto de Lebesgue de medida 0, y $|h\left(t\right)|\leq b\left(t\right)$, donde $b$ es DRI.
\end{itemize}
\end{Note}

\begin{Teo}[Teorema Principal de Renovaci\'on]
Si $F$ es no aritm\'etica y $h\left(t\right)$ es Directamente Riemann Integrable (DRI), entonces

\begin{eqnarray*}
lim_{t\rightarrow\infty}U\star h=\frac{1}{\mu}\int_{\rea_{+}}h\left(s\right)ds.
\end{eqnarray*}
\end{Teo}

\begin{Prop}
Cualquier funci\'on $H\left(t\right)$ acotada en intervalos finitos y que es 0 para $t<0$ puede expresarse como
\begin{eqnarray*}
H\left(t\right)=U\star h\left(t\right)\textrm{,  donde }h\left(t\right)=H\left(t\right)-F\star H\left(t\right)
\end{eqnarray*}
\end{Prop}

\begin{Def}
Un proceso estoc\'astico $X\left(t\right)$ es crudamente regenerativo en un tiempo aleatorio positivo $T$ si
\begin{eqnarray*}
\esp\left[X\left(T+t\right)|T\right]=\esp\left[X\left(t\right)\right]\textrm{, para }t\geq0,\end{eqnarray*}
y con las esperanzas anteriores finitas.
\end{Def}

\begin{Prop}
Sup\'ongase que $X\left(t\right)$ es un proceso crudamente regenerativo en $T$, que tiene distribuci\'on $F$. Si $\esp\left[X\left(t\right)\right]$ es acotado en intervalos finitos, entonces
\begin{eqnarray*}
\esp\left[X\left(t\right)\right]=U\star h\left(t\right)\textrm{,  donde }h\left(t\right)=\esp\left[X\left(t\right)\indora\left(T>t\right)\right].
\end{eqnarray*}
\end{Prop}

\begin{Teo}[Regeneraci\'on Cruda]
Sup\'ongase que $X\left(t\right)$ es un proceso con valores positivo crudamente regenerativo en $T$, y def\'inase $M=\sup\left\{|X\left(t\right)|:t\leq T\right\}$. Si $T$ es no aritm\'etico y $M$ y $MT$ tienen media finita, entonces
\begin{eqnarray*}
lim_{t\rightarrow\infty}\esp\left[X\left(t\right)\right]=\frac{1}{\mu}\int_{\rea_{+}}h\left(s\right)ds,
\end{eqnarray*}
donde $h\left(t\right)=\esp\left[X\left(t\right)\indora\left(T>t\right)\right]$.
\end{Teo}



%___________________________________________________________________________________________
%
\subsection{Funci\'on de Renovaci\'on}
%___________________________________________________________________________________________
%


\begin{Def}
Sea $h\left(t\right)$ funci\'on de valores reales en $\rea$ acotada en intervalos finitos e igual a cero para $t<0$ La ecuaci\'on de renovaci\'on para $h\left(t\right)$ y la distribuci\'on $F$ es

\begin{eqnarray}\label{Ec.Renovacion}
H\left(t\right)=h\left(t\right)+\int_{\left[0,t\right]}H\left(t-s\right)dF\left(s\right)\textrm{,    }t\geq0,
\end{eqnarray}
donde $H\left(t\right)$ es una funci\'on de valores reales. Esto es $H=h+F\star H$. Decimos que $H\left(t\right)$ es soluci\'on de esta ecuaci\'on si satisface la ecuaci\'on, y es acotada en intervalos finitos e iguales a cero para $t<0$.
\end{Def}

\begin{Prop}
La funci\'on $U\star h\left(t\right)$ es la \'unica soluci\'on de la ecuaci\'on de renovaci\'on (\ref{Ec.Renovacion}).
\end{Prop}

\begin{Teo}[Teorema Renovaci\'on Elemental]
\begin{eqnarray*}
t^{-1}U\left(t\right)\rightarrow 1/\mu\textrm{,    cuando }t\rightarrow\infty.
\end{eqnarray*}
\end{Teo}

%___________________________________________________________________________________________
%
\subsection{Propiedades de los Procesos de Renovaci\'on}
%___________________________________________________________________________________________
%

Los tiempos $T_{n}$ est\'an relacionados con los conteos de $N\left(t\right)$ por

\begin{eqnarray*}
\left\{N\left(t\right)\geq n\right\}&=&\left\{T_{n}\leq t\right\}\\
T_{N\left(t\right)}\leq &t&<T_{N\left(t\right)+1},
\end{eqnarray*}

adem\'as $N\left(T_{n}\right)=n$, y 

\begin{eqnarray*}
N\left(t\right)=\max\left\{n:T_{n}\leq t\right\}=\min\left\{n:T_{n+1}>t\right\}
\end{eqnarray*}

Por propiedades de la convoluci\'on se sabe que

\begin{eqnarray*}
P\left\{T_{n}\leq t\right\}=F^{n\star}\left(t\right)
\end{eqnarray*}
que es la $n$-\'esima convoluci\'on de $F$. Entonces 

\begin{eqnarray*}
\left\{N\left(t\right)\geq n\right\}&=&\left\{T_{n}\leq t\right\}\\
P\left\{N\left(t\right)\leq n\right\}&=&1-F^{\left(n+1\right)\star}\left(t\right)
\end{eqnarray*}

Adem\'as usando el hecho de que $\esp\left[N\left(t\right)\right]=\sum_{n=1}^{\infty}P\left\{N\left(t\right)\geq n\right\}$
se tiene que

\begin{eqnarray*}
\esp\left[N\left(t\right)\right]=\sum_{n=1}^{\infty}F^{n\star}\left(t\right)
\end{eqnarray*}

\begin{Prop}
Para cada $t\geq0$, la funci\'on generadora de momentos $\esp\left[e^{\alpha N\left(t\right)}\right]$ existe para alguna $\alpha$ en una vecindad del 0, y de aqu\'i que $\esp\left[N\left(t\right)^{m}\right]<\infty$, para $m\geq1$.
\end{Prop}


\begin{Note}
Si el primer tiempo de renovaci\'on $\xi_{1}$ no tiene la misma distribuci\'on que el resto de las $\xi_{n}$, para $n\geq2$, a $N\left(t\right)$ se le llama Proceso de Renovaci\'on retardado, donde si $\xi$ tiene distribuci\'on $G$, entonces el tiempo $T_{n}$ de la $n$-\'esima renovaci\'on tiene distribuci\'on $G\star F^{\left(n-1\right)\star}\left(t\right)$
\end{Note}


\begin{Teo}
Para una constante $\mu\leq\infty$ ( o variable aleatoria), las siguientes expresiones son equivalentes:

\begin{eqnarray}
lim_{n\rightarrow\infty}n^{-1}T_{n}&=&\mu,\textrm{ c.s.}\\
lim_{t\rightarrow\infty}t^{-1}N\left(t\right)&=&1/\mu,\textrm{ c.s.}
\end{eqnarray}
\end{Teo}


Es decir, $T_{n}$ satisface la Ley Fuerte de los Grandes N\'umeros s\'i y s\'olo s\'i $N\left/t\right)$ la cumple.


\begin{Coro}[Ley Fuerte de los Grandes N\'umeros para Procesos de Renovaci\'on]
Si $N\left(t\right)$ es un proceso de renovaci\'on cuyos tiempos de inter-renovaci\'on tienen media $\mu\leq\infty$, entonces
\begin{eqnarray}
t^{-1}N\left(t\right)\rightarrow 1/\mu,\textrm{ c.s. cuando }t\rightarrow\infty.
\end{eqnarray}

\end{Coro}


Considerar el proceso estoc\'astico de valores reales $\left\{Z\left(t\right):t\geq0\right\}$ en el mismo espacio de probabilidad que $N\left(t\right)$

\begin{Def}
Para el proceso $\left\{Z\left(t\right):t\geq0\right\}$ se define la fluctuaci\'on m\'axima de $Z\left(t\right)$ en el intervalo $\left(T_{n-1},T_{n}\right]$:
\begin{eqnarray*}
M_{n}=\sup_{T_{n-1}<t\leq T_{n}}|Z\left(t\right)-Z\left(T_{n-1}\right)|
\end{eqnarray*}
\end{Def}

\begin{Teo}
Sup\'ongase que $n^{-1}T_{n}\rightarrow\mu$ c.s. cuando $n\rightarrow\infty$, donde $\mu\leq\infty$ es una constante o variable aleatoria. Sea $a$ una constante o variable aleatoria que puede ser infinita cuando $\mu$ es finita, y considere las expresiones l\'imite:
\begin{eqnarray}
lim_{n\rightarrow\infty}n^{-1}Z\left(T_{n}\right)&=&a,\textrm{ c.s.}\\
lim_{t\rightarrow\infty}t^{-1}Z\left(t\right)&=&a/\mu,\textrm{ c.s.}
\end{eqnarray}
La segunda expresi\'on implica la primera. Conversamente, la primera implica la segunda si el proceso $Z\left(t\right)$ es creciente, o si $lim_{n\rightarrow\infty}n^{-1}M_{n}=0$ c.s.
\end{Teo}

\begin{Coro}
Si $N\left(t\right)$ es un proceso de renovaci\'on, y $\left(Z\left(T_{n}\right)-Z\left(T_{n-1}\right),M_{n}\right)$, para $n\geq1$, son variables aleatorias independientes e id\'enticamente distribuidas con media finita, entonces,
\begin{eqnarray}
lim_{t\rightarrow\infty}t^{-1}Z\left(t\right)\rightarrow\frac{\esp\left[Z\left(T_{1}\right)-Z\left(T_{0}\right)\right]}{\esp\left[T_{1}\right]},\textrm{ c.s. cuando  }t\rightarrow\infty.
\end{eqnarray}
\end{Coro}

%___________________________________________________________________________________________
%
\subsection{Funci\'on de Renovaci\'on}
%___________________________________________________________________________________________
%


Sup\'ongase que $N\left(t\right)$ es un proceso de renovaci\'on con distribuci\'on $F$ con media finita $\mu$.

\begin{Def}
La funci\'on de renovaci\'on asociada con la distribuci\'on $F$, del proceso $N\left(t\right)$, es
\begin{eqnarray*}
U\left(t\right)=\sum_{n=1}^{\infty}F^{n\star}\left(t\right),\textrm{   }t\geq0,
\end{eqnarray*}
donde $F^{0\star}\left(t\right)=\indora\left(t\geq0\right)$.
\end{Def}


\begin{Prop}
Sup\'ongase que la distribuci\'on de inter-renovaci\'on $F$ tiene densidad $f$. Entonces $U\left(t\right)$ tambi\'en tiene densidad, para $t>0$, y es $U^{'}\left(t\right)=\sum_{n=0}^{\infty}f^{n\star}\left(t\right)$. Adem\'as
\begin{eqnarray*}
\prob\left\{N\left(t\right)>N\left(t-\right)\right\}=0\textrm{,   }t\geq0.
\end{eqnarray*}
\end{Prop}

\begin{Def}
La Transformada de Laplace-Stieljes de $F$ est\'a dada por

\begin{eqnarray*}
\hat{F}\left(\alpha\right)=\int_{\rea_{+}}e^{-\alpha t}dF\left(t\right)\textrm{,  }\alpha\geq0.
\end{eqnarray*}
\end{Def}

Entonces

\begin{eqnarray*}
\hat{U}\left(\alpha\right)=\sum_{n=0}^{\infty}\hat{F^{n\star}}\left(\alpha\right)=\sum_{n=0}^{\infty}\hat{F}\left(\alpha\right)^{n}=\frac{1}{1-\hat{F}\left(\alpha\right)}.
\end{eqnarray*}


\begin{Prop}
La Transformada de Laplace $\hat{U}\left(\alpha\right)$ y $\hat{F}\left(\alpha\right)$ determina una a la otra de manera \'unica por la relaci\'on $\hat{U}\left(\alpha\right)=\frac{1}{1-\hat{F}\left(\alpha\right)}$.
\end{Prop}


\begin{Note}
Un proceso de renovaci\'on $N\left(t\right)$ cuyos tiempos de inter-renovaci\'on tienen media finita, es un proceso Poisson con tasa $\lambda$ si y s\'olo s\'i $\esp\left[U\left(t\right)\right]=\lambda t$, para $t\geq0$.
\end{Note}


\begin{Teo}
Sea $N\left(t\right)$ un proceso puntual simple con puntos de localizaci\'on $T_{n}$ tal que $\eta\left(t\right)=\esp\left[N\left(\right)\right]$ es finita para cada $t$. Entonces para cualquier funci\'on $f:\rea_{+}\rightarrow\rea$,
\begin{eqnarray*}
\esp\left[\sum_{n=1}^{N\left(\right)}f\left(T_{n}\right)\right]=\int_{\left(0,t\right]}f\left(s\right)d\eta\left(s\right)\textrm{,  }t\geq0,
\end{eqnarray*}
suponiendo que la integral exista. Adem\'as si $X_{1},X_{2},\ldots$ son variables aleatorias definidas en el mismo espacio de probabilidad que el proceso $N\left(t\right)$ tal que $\esp\left[X_{n}|T_{n}=s\right]=f\left(s\right)$, independiente de $n$. Entonces
\begin{eqnarray*}
\esp\left[\sum_{n=1}^{N\left(t\right)}X_{n}\right]=\int_{\left(0,t\right]}f\left(s\right)d\eta\left(s\right)\textrm{,  }t\geq0,
\end{eqnarray*} 
suponiendo que la integral exista. 
\end{Teo}

\begin{Coro}[Identidad de Wald para Renovaciones]
Para el proceso de renovaci\'on $N\left(t\right)$,
\begin{eqnarray*}
\esp\left[T_{N\left(t\right)+1}\right]=\mu\esp\left[N\left(t\right)+1\right]\textrm{,  }t\geq0,
\end{eqnarray*}  
\end{Coro}

%______________________________________________________________________
\subsection{Procesos de Renovaci\'on}
%______________________________________________________________________

\begin{Def}\label{Def.Tn}
Sean $0\leq T_{1}\leq T_{2}\leq \ldots$ son tiempos aleatorios infinitos en los cuales ocurren ciertos eventos. El n\'umero de tiempos $T_{n}$ en el intervalo $\left[0,t\right)$ es

\begin{eqnarray}
N\left(t\right)=\sum_{n=1}^{\infty}\indora\left(T_{n}\leq t\right),
\end{eqnarray}
para $t\geq0$.
\end{Def}

Si se consideran los puntos $T_{n}$ como elementos de $\rea_{+}$, y $N\left(t\right)$ es el n\'umero de puntos en $\rea$. El proceso denotado por $\left\{N\left(t\right):t\geq0\right\}$, denotado por $N\left(t\right)$, es un proceso puntual en $\rea_{+}$. Los $T_{n}$ son los tiempos de ocurrencia, el proceso puntual $N\left(t\right)$ es simple si su n\'umero de ocurrencias son distintas: $0<T_{1}<T_{2}<\ldots$ casi seguramente.

\begin{Def}
Un proceso puntual $N\left(t\right)$ es un proceso de renovaci\'on si los tiempos de interocurrencia $\xi_{n}=T_{n}-T_{n-1}$, para $n\geq1$, son independientes e identicamente distribuidos con distribuci\'on $F$, donde $F\left(0\right)=0$ y $T_{0}=0$. Los $T_{n}$ son llamados tiempos de renovaci\'on, referente a la independencia o renovaci\'on de la informaci\'on estoc\'astica en estos tiempos. Los $\xi_{n}$ son los tiempos de inter-renovaci\'on, y $N\left(t\right)$ es el n\'umero de renovaciones en el intervalo $\left[0,t\right)$
\end{Def}


\begin{Note}
Para definir un proceso de renovaci\'on para cualquier contexto, solamente hay que especificar una distribuci\'on $F$, con $F\left(0\right)=0$, para los tiempos de inter-renovaci\'on. La funci\'on $F$ en turno degune las otra variables aleatorias. De manera formal, existe un espacio de probabilidad y una sucesi\'on de variables aleatorias $\xi_{1},\xi_{2},\ldots$ definidas en este con distribuci\'on $F$. Entonces las otras cantidades son $T_{n}=\sum_{k=1}^{n}\xi_{k}$ y $N\left(t\right)=\sum_{n=1}^{\infty}\indora\left(T_{n}\leq t\right)$, donde $T_{n}\rightarrow\infty$ casi seguramente por la Ley Fuerte de los Grandes Números.
\end{Note}
%_____________________________________________________
\subsection{Puntos de Renovaci\'on}
%_____________________________________________________

Para cada cola $Q_{i}$ se tienen los procesos de arribo a la cola, para estas, los tiempos de arribo est\'an dados por $$\left\{T_{1}^{i},T_{2}^{i},\ldots,T_{k}^{i},\ldots\right\},$$ entonces, consideremos solamente los primeros tiempos de arribo a cada una de las colas, es decir, $$\left\{T_{1}^{1},T_{1}^{2},T_{1}^{3},T_{1}^{4}\right\},$$ se sabe que cada uno de estos tiempos se distribuye de manera exponencial con par\'ametro $1/mu_{i}$. Adem\'as se sabe que para $$T^{*}=\min\left\{T_{1}^{1},T_{1}^{2},T_{1}^{3},T_{1}^{4}\right\},$$ $T^{*}$ se distribuye de manera exponencial con par\'ametro $$\mu^{*}=\sum_{i=1}^{4}\mu_{i}.$$ Ahora, dado que 
\begin{center}
\begin{tabular}{lcl}
$\tilde{r}=r_{1}+r_{2}$ & y &$\hat{r}=r_{3}+r_{4}.$
\end{tabular}
\end{center}


Supongamos que $$\tilde{r},\hat{r}<\mu^{*},$$ entonces si tomamos $$r^{*}=\min\left\{\tilde{r},\hat{r}\right\},$$ se tiene que para  $$t^{*}\in\left(0,r^{*}\right)$$ se cumple que 
\begin{center}
\begin{tabular}{lcl}
$\tau_{1}\left(1\right)=0$ & y por tanto & $\overline{\tau}_{1}=0,$
\end{tabular}
\end{center}
entonces para la segunda cola en este primer ciclo se cumple que $$\tau_{2}=\overline{\tau}_{1}+r_{1}=r_{1}<\mu^{*},$$ y por tanto se tiene que  $$\overline{\tau}_{2}=\tau_{2}.$$ Por lo tanto, nuevamente para la primer cola en el segundo ciclo $$\tau_{1}\left(2\right)=\tau_{2}\left(1\right)+r_{2}=\tilde{r}<\mu^{*}.$$ An\'alogamente para el segundo sistema se tiene que ambas colas est\'an vac\'ias, es decir, existe un valor $t^{*}$ tal que en el intervalo $\left(0,t^{*}\right)$ no ha llegado ning\'un usuario, es decir, $$L_{i}\left(t^{*}\right)=0$$ para $i=1,2,3,4$.

\subsection{Resultados para Procesos de Salida}

En \cite{Sigman2} prueban que para la existencia de un una sucesi\'on infinita no decreciente de tiempos de regeneraci\'on $\tau_{1}\leq\tau_{2}\leq\cdots$ en los cuales el proceso se regenera, basta un tiempo de regeneraci\'on $R_{1}$, donde $R_{j}=\tau_{j}-\tau_{j-1}$. Para tal efecto se requiere la existencia de un espacio de probabilidad $\left(\Omega,\mathcal{F},\prob\right)$, y proceso estoc\'astico $\textit{X}=\left\{X\left(t\right):t\geq0\right\}$ con espacio de estados $\left(S,\mathcal{R}\right)$, con $\mathcal{R}$ $\sigma$-\'algebra.

\begin{Prop}
Si existe una variable aleatoria no negativa $R_{1}$ tal que $\theta_{R\footnotesize{1}}X=_{D}X$, entonces $\left(\Omega,\mathcal{F},\prob\right)$ puede extenderse para soportar una sucesi\'on estacionaria de variables aleatorias $R=\left\{R_{k}:k\geq1\right\}$, tal que para $k\geq1$,
\begin{eqnarray*}
\theta_{k}\left(X,R\right)=_{D}\left(X,R\right).
\end{eqnarray*}

Adem\'as, para $k\geq1$, $\theta_{k}R$ es condicionalmente independiente de $\left(X,R_{1},\ldots,R_{k}\right)$, dado $\theta_{\tau k}X$.

\end{Prop}


\begin{itemize}
\item Doob en 1953 demostr\'o que el estado estacionario de un proceso de partida en un sistema de espera $M/G/\infty$, es Poisson con la misma tasa que el proceso de arribos.

\item Burke en 1968, fue el primero en demostrar que el estado estacionario de un proceso de salida de una cola $M/M/s$ es un proceso Poisson.

\item Disney en 1973 obtuvo el siguiente resultado:

\begin{Teo}
Para el sistema de espera $M/G/1/L$ con disciplina FIFO, el proceso $\textbf{I}$ es un proceso de renovaci\'on si y s\'olo si el proceso denominado longitud de la cola es estacionario y se cumple cualquiera de los siguientes casos:

\begin{itemize}
\item[a)] Los tiempos de servicio son identicamente cero;
\item[b)] $L=0$, para cualquier proceso de servicio $S$;
\item[c)] $L=1$ y $G=D$;
\item[d)] $L=\infty$ y $G=M$.
\end{itemize}
En estos casos, respectivamente, las distribuciones de interpartida $P\left\{T_{n+1}-T_{n}\leq t\right\}$ son


\begin{itemize}
\item[a)] $1-e^{-\lambda t}$, $t\geq0$;
\item[b)] $1-e^{-\lambda t}*F\left(t\right)$, $t\geq0$;
\item[c)] $1-e^{-\lambda t}*\indora_{d}\left(t\right)$, $t\geq0$;
\item[d)] $1-e^{-\lambda t}*F\left(t\right)$, $t\geq0$.
\end{itemize}
\end{Teo}


\item Finch (1959) mostr\'o que para los sistemas $M/G/1/L$, con $1\leq L\leq \infty$ con distribuciones de servicio dos veces diferenciable, solamente el sistema $M/M/1/\infty$ tiene proceso de salida de renovaci\'on estacionario.

\item King (1971) demostro que un sistema de colas estacionario $M/G/1/1$ tiene sus tiempos de interpartida sucesivas $D_{n}$ y $D_{n+1}$ son independientes, si y s\'olo si, $G=D$, en cuyo caso le proceso de salida es de renovaci\'on.

\item Disney (1973) demostr\'o que el \'unico sistema estacionario $M/G/1/L$, que tiene proceso de salida de renovaci\'on  son los sistemas $M/M/1$ y $M/D/1/1$.



\item El siguiente resultado es de Disney y Koning (1985)
\begin{Teo}
En un sistema de espera $M/G/s$, el estado estacionario del proceso de salida es un proceso Poisson para cualquier distribuci\'on de los tiempos de servicio si el sistema tiene cualquiera de las siguientes cuatro propiedades.

\begin{itemize}
\item[a)] $s=\infty$
\item[b)] La disciplina de servicio es de procesador compartido.
\item[c)] La disciplina de servicio es LCFS y preemptive resume, esto se cumple para $L<\infty$
\item[d)] $G=M$.
\end{itemize}

\end{Teo}

\item El siguiente resultado es de Alamatsaz (1983)

\begin{Teo}
En cualquier sistema de colas $GI/G/1/L$ con $1\leq L<\infty$ y distribuci\'on de interarribos $A$ y distribuci\'on de los tiempos de servicio $B$, tal que $A\left(0\right)=0$, $A\left(t\right)\left(1-B\left(t\right)\right)>0$ para alguna $t>0$ y $B\left(t\right)$ para toda $t>0$, es imposible que el proceso de salida estacionario sea de renovaci\'on.
\end{Teo}

\end{itemize}

Estos resultados aparecen en Daley (1968) \cite{Daley68} para $\left\{T_{n}\right\}$ intervalos de inter-arribo, $\left\{D_{n}\right\}$ intervalos de inter-salida y $\left\{S_{n}\right\}$ tiempos de servicio.

\begin{itemize}
\item Si el proceso $\left\{T_{n}\right\}$ es Poisson, el proceso $\left\{D_{n}\right\}$ es no correlacionado si y s\'olo si es un proceso Poisso, lo cual ocurre si y s\'olo si $\left\{S_{n}\right\}$ son exponenciales negativas.

\item Si $\left\{S_{n}\right\}$ son exponenciales negativas, $\left\{D_{n}\right\}$ es un proceso de renovaci\'on  si y s\'olo si es un proceso Poisson, lo cual ocurre si y s\'olo si $\left\{T_{n}\right\}$ es un proceso Poisson.

\item $\esp\left(D_{n}\right)=\esp\left(T_{n}\right)$.

\item Para un sistema de visitas $GI/M/1$ se tiene el siguiente teorema:

\begin{Teo}
En un sistema estacionario $GI/M/1$ los intervalos de interpartida tienen
\begin{eqnarray*}
\esp\left(e^{-\theta D_{n}}\right)&=&\mu\left(\mu+\theta\right)^{-1}\left[\delta\theta
-\mu\left(1-\delta\right)\alpha\left(\theta\right)\right]
\left[\theta-\mu\left(1-\delta\right)^{-1}\right]\\
\alpha\left(\theta\right)&=&\esp\left[e^{-\theta T_{0}}\right]\\
var\left(D_{n}\right)&=&var\left(T_{0}\right)-\left(\tau^{-1}-\delta^{-1}\right)
2\delta\left(\esp\left(S_{0}\right)\right)^{2}\left(1-\delta\right)^{-1}.
\end{eqnarray*}
\end{Teo}



\begin{Teo}
El proceso de salida de un sistema de colas estacionario $GI/M/1$ es un proceso de renovaci\'on si y s\'olo si el proceso de entrada es un proceso Poisson, en cuyo caso el proceso de salida es un proceso Poisson.
\end{Teo}


\begin{Teo}
Los intervalos de interpartida $\left\{D_{n}\right\}$ de un sistema $M/G/1$ estacionario son no correlacionados si y s\'olo si la distribuci\'on de los tiempos de servicio es exponencial negativa, es decir, el sistema es de tipo  $M/M/1$.

\end{Teo}



\end{itemize}


%________________________________________________________________________
\subsection{Procesos Regenerativos}
%________________________________________________________________________

Para $\left\{X\left(t\right):t\geq0\right\}$ Proceso Estoc\'astico a tiempo continuo con estado de espacios $S$, que es un espacio m\'etrico, con trayectorias continuas por la derecha y con l\'imites por la izquierda c.s. Sea $N\left(t\right)$ un proceso de renovaci\'on en $\rea_{+}$ definido en el mismo espacio de probabilidad que $X\left(t\right)$, con tiempos de renovaci\'on $T$ y tiempos de inter-renovaci\'on $\xi_{n}=T_{n}-T_{n-1}$, con misma distribuci\'on $F$ de media finita $\mu$.



\begin{Def}
Para el proceso $\left\{\left(N\left(t\right),X\left(t\right)\right):t\geq0\right\}$, sus trayectoria muestrales en el intervalo de tiempo $\left[T_{n-1},T_{n}\right)$ est\'an descritas por
\begin{eqnarray*}
\zeta_{n}=\left(\xi_{n},\left\{X\left(T_{n-1}+t\right):0\leq t<\xi_{n}\right\}\right)
\end{eqnarray*}
Este $\zeta_{n}$ es el $n$-\'esimo segmento del proceso. El proceso es regenerativo sobre los tiempos $T_{n}$ si sus segmentos $\zeta_{n}$ son independientes e id\'enticamennte distribuidos.
\end{Def}


\begin{Obs}
Si $\tilde{X}\left(t\right)$ con espacio de estados $\tilde{S}$ es regenerativo sobre $T_{n}$, entonces $X\left(t\right)=f\left(\tilde{X}\left(t\right)\right)$ tambi\'en es regenerativo sobre $T_{n}$, para cualquier funci\'on $f:\tilde{S}\rightarrow S$.
\end{Obs}

\begin{Obs}
Los procesos regenerativos son crudamente regenerativos, pero no al rev\'es.
\end{Obs}

\begin{Def}[Definici\'on Cl\'asica]
Un proceso estoc\'astico $X=\left\{X\left(t\right):t\geq0\right\}$ es llamado regenerativo is existe una variable aleatoria $R_{1}>0$ tal que
\begin{itemize}
\item[i)] $\left\{X\left(t+R_{1}\right):t\geq0\right\}$ es independiente de $\left\{\left\{X\left(t\right):t<R_{1}\right\},\right\}$
\item[ii)] $\left\{X\left(t+R_{1}\right):t\geq0\right\}$ es estoc\'asticamente equivalente a $\left\{X\left(t\right):t>0\right\}$
\end{itemize}

Llamamos a $R_{1}$ tiempo de regeneraci\'on, y decimos que $X$ se regenera en este punto.
\end{Def}

$\left\{X\left(t+R_{1}\right)\right\}$ es regenerativo con tiempo de regeneraci\'on $R_{2}$, independiente de $R_{1}$ pero con la misma distribuci\'on que $R_{1}$. Procediendo de esta manera se obtiene una secuencia de variables aleatorias independientes e id\'enticamente distribuidas $\left\{R_{n}\right\}$ llamados longitudes de ciclo. Si definimos a $Z_{k}\equiv R_{1}+R_{2}+\cdots+R_{k}$, se tiene un proceso de renovaci\'on llamado proceso de renovaci\'on encajado para $X$.

\begin{Note}
Un proceso regenerativo con media de la longitud de ciclo finita es llamado positivo recurrente.
\end{Note}


\begin{Def}
Para $x$ fijo y para cada $t\geq0$, sea $I_{x}\left(t\right)=1$ si $X\left(t\right)\leq x$,  $I_{x}\left(t\right)=0$ en caso contrario, y def\'inanse los tiempos promedio
\begin{eqnarray*}
\overline{X}&=&lim_{t\rightarrow\infty}\frac{1}{t}\int_{0}^{\infty}X\left(u\right)du\\
\prob\left(X_{\infty}\leq x\right)&=&lim_{t\rightarrow\infty}\frac{1}{t}\int_{0}^{\infty}I_{x}\left(u\right)du,
\end{eqnarray*}
cuando estos l\'imites existan.
\end{Def}

Como consecuencia del teorema de Renovaci\'on-Recompensa, se tiene que el primer l\'imite  existe y es igual a la constante
\begin{eqnarray*}
\overline{X}&=&\frac{\esp\left[\int_{0}^{R_{1}}X\left(t\right)dt\right]}{\esp\left[R_{1}\right]},
\end{eqnarray*}
suponiendo que ambas esperanzas son finitas.

\begin{Note}
\begin{itemize}
\item[a)] Si el proceso regenerativo $X$ es positivo recurrente y tiene trayectorias muestrales no negativas, entonces la ecuaci\'on anterior es v\'alida.
\item[b)] Si $X$ es positivo recurrente regenerativo, podemos construir una \'unica versi\'on estacionaria de este proceso, $X_{e}=\left\{X_{e}\left(t\right)\right\}$, donde $X_{e}$ es un proceso estoc\'astico regenerativo y estrictamente estacionario, con distribuci\'on marginal distribuida como $X_{\infty}$
\end{itemize}
\end{Note}

\subsection{Renewal and Regenerative Processes: Serfozo\cite{Serfozo}}
\begin{Def}\label{Def.Tn}
Sean $0\leq T_{1}\leq T_{2}\leq \ldots$ son tiempos aleatorios infinitos en los cuales ocurren ciertos eventos. El n\'umero de tiempos $T_{n}$ en el intervalo $\left[0,t\right)$ es

\begin{eqnarray}
N\left(t\right)=\sum_{n=1}^{\infty}\indora\left(T_{n}\leq t\right),
\end{eqnarray}
para $t\geq0$.
\end{Def}

Si se consideran los puntos $T_{n}$ como elementos de $\rea_{+}$, y $N\left(t\right)$ es el n\'umero de puntos en $\rea$. El proceso denotado por $\left\{N\left(t\right):t\geq0\right\}$, denotado por $N\left(t\right)$, es un proceso puntual en $\rea_{+}$. Los $T_{n}$ son los tiempos de ocurrencia, el proceso puntual $N\left(t\right)$ es simple si su n\'umero de ocurrencias son distintas: $0<T_{1}<T_{2}<\ldots$ casi seguramente.

\begin{Def}
Un proceso puntual $N\left(t\right)$ es un proceso de renovaci\'on si los tiempos de interocurrencia $\xi_{n}=T_{n}-T_{n-1}$, para $n\geq1$, son independientes e identicamente distribuidos con distribuci\'on $F$, donde $F\left(0\right)=0$ y $T_{0}=0$. Los $T_{n}$ son llamados tiempos de renovaci\'on, referente a la independencia o renovaci\'on de la informaci\'on estoc\'astica en estos tiempos. Los $\xi_{n}$ son los tiempos de inter-renovaci\'on, y $N\left(t\right)$ es el n\'umero de renovaciones en el intervalo $\left[0,t\right)$
\end{Def}


\begin{Note}
Para definir un proceso de renovaci\'on para cualquier contexto, solamente hay que especificar una distribuci\'on $F$, con $F\left(0\right)=0$, para los tiempos de inter-renovaci\'on. La funci\'on $F$ en turno degune las otra variables aleatorias. De manera formal, existe un espacio de probabilidad y una sucesi\'on de variables aleatorias $\xi_{1},\xi_{2},\ldots$ definidas en este con distribuci\'on $F$. Entonces las otras cantidades son $T_{n}=\sum_{k=1}^{n}\xi_{k}$ y $N\left(t\right)=\sum_{n=1}^{\infty}\indora\left(T_{n}\leq t\right)$, donde $T_{n}\rightarrow\infty$ casi seguramente por la Ley Fuerte de los Grandes N\'umeros.
\end{Note}







Los tiempos $T_{n}$ est\'an relacionados con los conteos de $N\left(t\right)$ por

\begin{eqnarray*}
\left\{N\left(t\right)\geq n\right\}&=&\left\{T_{n}\leq t\right\}\\
T_{N\left(t\right)}\leq &t&<T_{N\left(t\right)+1},
\end{eqnarray*}

adem\'as $N\left(T_{n}\right)=n$, y 

\begin{eqnarray*}
N\left(t\right)=\max\left\{n:T_{n}\leq t\right\}=\min\left\{n:T_{n+1}>t\right\}
\end{eqnarray*}

Por propiedades de la convoluci\'on se sabe que

\begin{eqnarray*}
P\left\{T_{n}\leq t\right\}=F^{n\star}\left(t\right)
\end{eqnarray*}
que es la $n$-\'esima convoluci\'on de $F$. Entonces 

\begin{eqnarray*}
\left\{N\left(t\right)\geq n\right\}&=&\left\{T_{n}\leq t\right\}\\
P\left\{N\left(t\right)\leq n\right\}&=&1-F^{\left(n+1\right)\star}\left(t\right)
\end{eqnarray*}

Adem\'as usando el hecho de que $\esp\left[N\left(t\right)\right]=\sum_{n=1}^{\infty}P\left\{N\left(t\right)\geq n\right\}$
se tiene que

\begin{eqnarray*}
\esp\left[N\left(t\right)\right]=\sum_{n=1}^{\infty}F^{n\star}\left(t\right)
\end{eqnarray*}

\begin{Prop}
Para cada $t\geq0$, la funci\'on generadora de momentos $\esp\left[e^{\alpha N\left(t\right)}\right]$ existe para alguna $\alpha$ en una vecindad del 0, y de aqu\'i que $\esp\left[N\left(t\right)^{m}\right]<\infty$, para $m\geq1$.
\end{Prop}


\begin{Note}
Si el primer tiempo de renovaci\'on $\xi_{1}$ no tiene la misma distribuci\'on que el resto de las $\xi_{n}$, para $n\geq2$, a $N\left(t\right)$ se le llama Proceso de Renovaci\'on retardado, donde si $\xi$ tiene distribuci\'on $G$, entonces el tiempo $T_{n}$ de la $n$-\'esima renovaci\'on tiene distribuci\'on $G\star F^{\left(n-1\right)\star}\left(t\right)$
\end{Note}


\begin{Teo}
Para una constante $\mu\leq\infty$ ( o variable aleatoria), las siguientes expresiones son equivalentes:

\begin{eqnarray}
lim_{n\rightarrow\infty}n^{-1}T_{n}&=&\mu,\textrm{ c.s.}\\
lim_{t\rightarrow\infty}t^{-1}N\left(t\right)&=&1/\mu,\textrm{ c.s.}
\end{eqnarray}
\end{Teo}


Es decir, $T_{n}$ satisface la Ley Fuerte de los Grandes N\'umeros s\'i y s\'olo s\'i $N\left/t\right)$ la cumple.


\begin{Coro}[Ley Fuerte de los Grandes N\'umeros para Procesos de Renovaci\'on]
Si $N\left(t\right)$ es un proceso de renovaci\'on cuyos tiempos de inter-renovaci\'on tienen media $\mu\leq\infty$, entonces
\begin{eqnarray}
t^{-1}N\left(t\right)\rightarrow 1/\mu,\textrm{ c.s. cuando }t\rightarrow\infty.
\end{eqnarray}

\end{Coro}


Considerar el proceso estoc\'astico de valores reales $\left\{Z\left(t\right):t\geq0\right\}$ en el mismo espacio de probabilidad que $N\left(t\right)$

\begin{Def}
Para el proceso $\left\{Z\left(t\right):t\geq0\right\}$ se define la fluctuaci\'on m\'axima de $Z\left(t\right)$ en el intervalo $\left(T_{n-1},T_{n}\right]$:
\begin{eqnarray*}
M_{n}=\sup_{T_{n-1}<t\leq T_{n}}|Z\left(t\right)-Z\left(T_{n-1}\right)|
\end{eqnarray*}
\end{Def}

\begin{Teo}
Sup\'ongase que $n^{-1}T_{n}\rightarrow\mu$ c.s. cuando $n\rightarrow\infty$, donde $\mu\leq\infty$ es una constante o variable aleatoria. Sea $a$ una constante o variable aleatoria que puede ser infinita cuando $\mu$ es finita, y considere las expresiones l\'imite:
\begin{eqnarray}
lim_{n\rightarrow\infty}n^{-1}Z\left(T_{n}\right)&=&a,\textrm{ c.s.}\\
lim_{t\rightarrow\infty}t^{-1}Z\left(t\right)&=&a/\mu,\textrm{ c.s.}
\end{eqnarray}
La segunda expresi\'on implica la primera. Conversamente, la primera implica la segunda si el proceso $Z\left(t\right)$ es creciente, o si $lim_{n\rightarrow\infty}n^{-1}M_{n}=0$ c.s.
\end{Teo}

\begin{Coro}
Si $N\left(t\right)$ es un proceso de renovaci\'on, y $\left(Z\left(T_{n}\right)-Z\left(T_{n-1}\right),M_{n}\right)$, para $n\geq1$, son variables aleatorias independientes e id\'enticamente distribuidas con media finita, entonces,
\begin{eqnarray}
lim_{t\rightarrow\infty}t^{-1}Z\left(t\right)\rightarrow\frac{\esp\left[Z\left(T_{1}\right)-Z\left(T_{0}\right)\right]}{\esp\left[T_{1}\right]},\textrm{ c.s. cuando  }t\rightarrow\infty.
\end{eqnarray}
\end{Coro}


Sup\'ongase que $N\left(t\right)$ es un proceso de renovaci\'on con distribuci\'on $F$ con media finita $\mu$.

\begin{Def}
La funci\'on de renovaci\'on asociada con la distribuci\'on $F$, del proceso $N\left(t\right)$, es
\begin{eqnarray*}
U\left(t\right)=\sum_{n=1}^{\infty}F^{n\star}\left(t\right),\textrm{   }t\geq0,
\end{eqnarray*}
donde $F^{0\star}\left(t\right)=\indora\left(t\geq0\right)$.
\end{Def}


\begin{Prop}
Sup\'ongase que la distribuci\'on de inter-renovaci\'on $F$ tiene densidad $f$. Entonces $U\left(t\right)$ tambi\'en tiene densidad, para $t>0$, y es $U^{'}\left(t\right)=\sum_{n=0}^{\infty}f^{n\star}\left(t\right)$. Adem\'as
\begin{eqnarray*}
\prob\left\{N\left(t\right)>N\left(t-\right)\right\}=0\textrm{,   }t\geq0.
\end{eqnarray*}
\end{Prop}

\begin{Def}
La Transformada de Laplace-Stieljes de $F$ est\'a dada por

\begin{eqnarray*}
\hat{F}\left(\alpha\right)=\int_{\rea_{+}}e^{-\alpha t}dF\left(t\right)\textrm{,  }\alpha\geq0.
\end{eqnarray*}
\end{Def}

Entonces

\begin{eqnarray*}
\hat{U}\left(\alpha\right)=\sum_{n=0}^{\infty}\hat{F^{n\star}}\left(\alpha\right)=\sum_{n=0}^{\infty}\hat{F}\left(\alpha\right)^{n}=\frac{1}{1-\hat{F}\left(\alpha\right)}.
\end{eqnarray*}


\begin{Prop}
La Transformada de Laplace $\hat{U}\left(\alpha\right)$ y $\hat{F}\left(\alpha\right)$ determina una a la otra de manera \'unica por la relaci\'on $\hat{U}\left(\alpha\right)=\frac{1}{1-\hat{F}\left(\alpha\right)}$.
\end{Prop}


\begin{Note}
Un proceso de renovaci\'on $N\left(t\right)$ cuyos tiempos de inter-renovaci\'on tienen media finita, es un proceso Poisson con tasa $\lambda$ si y s\'olo s\'i $\esp\left[U\left(t\right)\right]=\lambda t$, para $t\geq0$.
\end{Note}


\begin{Teo}
Sea $N\left(t\right)$ un proceso puntual simple con puntos de localizaci\'on $T_{n}$ tal que $\eta\left(t\right)=\esp\left[N\left(\right)\right]$ es finita para cada $t$. Entonces para cualquier funci\'on $f:\rea_{+}\rightarrow\rea$,
\begin{eqnarray*}
\esp\left[\sum_{n=1}^{N\left(\right)}f\left(T_{n}\right)\right]=\int_{\left(0,t\right]}f\left(s\right)d\eta\left(s\right)\textrm{,  }t\geq0,
\end{eqnarray*}
suponiendo que la integral exista. Adem\'as si $X_{1},X_{2},\ldots$ son variables aleatorias definidas en el mismo espacio de probabilidad que el proceso $N\left(t\right)$ tal que $\esp\left[X_{n}|T_{n}=s\right]=f\left(s\right)$, independiente de $n$. Entonces
\begin{eqnarray*}
\esp\left[\sum_{n=1}^{N\left(t\right)}X_{n}\right]=\int_{\left(0,t\right]}f\left(s\right)d\eta\left(s\right)\textrm{,  }t\geq0,
\end{eqnarray*} 
suponiendo que la integral exista. 
\end{Teo}

\begin{Coro}[Identidad de Wald para Renovaciones]
Para el proceso de renovaci\'on $N\left(t\right)$,
\begin{eqnarray*}
\esp\left[T_{N\left(t\right)+1}\right]=\mu\esp\left[N\left(t\right)+1\right]\textrm{,  }t\geq0,
\end{eqnarray*}  
\end{Coro}


\begin{Def}
Sea $h\left(t\right)$ funci\'on de valores reales en $\rea$ acotada en intervalos finitos e igual a cero para $t<0$ La ecuaci\'on de renovaci\'on para $h\left(t\right)$ y la distribuci\'on $F$ es

\begin{eqnarray}\label{Ec.Renovacion}
H\left(t\right)=h\left(t\right)+\int_{\left[0,t\right]}H\left(t-s\right)dF\left(s\right)\textrm{,    }t\geq0,
\end{eqnarray}
donde $H\left(t\right)$ es una funci\'on de valores reales. Esto es $H=h+F\star H$. Decimos que $H\left(t\right)$ es soluci\'on de esta ecuaci\'on si satisface la ecuaci\'on, y es acotada en intervalos finitos e iguales a cero para $t<0$.
\end{Def}

\begin{Prop}
La funci\'on $U\star h\left(t\right)$ es la \'unica soluci\'on de la ecuaci\'on de renovaci\'on (\ref{Ec.Renovacion}).
\end{Prop}

\begin{Teo}[Teorema Renovaci\'on Elemental]
\begin{eqnarray*}
t^{-1}U\left(t\right)\rightarrow 1/\mu\textrm{,    cuando }t\rightarrow\infty.
\end{eqnarray*}
\end{Teo}



Sup\'ongase que $N\left(t\right)$ es un proceso de renovaci\'on con distribuci\'on $F$ con media finita $\mu$.

\begin{Def}
La funci\'on de renovaci\'on asociada con la distribuci\'on $F$, del proceso $N\left(t\right)$, es
\begin{eqnarray*}
U\left(t\right)=\sum_{n=1}^{\infty}F^{n\star}\left(t\right),\textrm{   }t\geq0,
\end{eqnarray*}
donde $F^{0\star}\left(t\right)=\indora\left(t\geq0\right)$.
\end{Def}


\begin{Prop}
Sup\'ongase que la distribuci\'on de inter-renovaci\'on $F$ tiene densidad $f$. Entonces $U\left(t\right)$ tambi\'en tiene densidad, para $t>0$, y es $U^{'}\left(t\right)=\sum_{n=0}^{\infty}f^{n\star}\left(t\right)$. Adem\'as
\begin{eqnarray*}
\prob\left\{N\left(t\right)>N\left(t-\right)\right\}=0\textrm{,   }t\geq0.
\end{eqnarray*}
\end{Prop}

\begin{Def}
La Transformada de Laplace-Stieljes de $F$ est\'a dada por

\begin{eqnarray*}
\hat{F}\left(\alpha\right)=\int_{\rea_{+}}e^{-\alpha t}dF\left(t\right)\textrm{,  }\alpha\geq0.
\end{eqnarray*}
\end{Def}

Entonces

\begin{eqnarray*}
\hat{U}\left(\alpha\right)=\sum_{n=0}^{\infty}\hat{F^{n\star}}\left(\alpha\right)=\sum_{n=0}^{\infty}\hat{F}\left(\alpha\right)^{n}=\frac{1}{1-\hat{F}\left(\alpha\right)}.
\end{eqnarray*}


\begin{Prop}
La Transformada de Laplace $\hat{U}\left(\alpha\right)$ y $\hat{F}\left(\alpha\right)$ determina una a la otra de manera \'unica por la relaci\'on $\hat{U}\left(\alpha\right)=\frac{1}{1-\hat{F}\left(\alpha\right)}$.
\end{Prop}


\begin{Note}
Un proceso de renovaci\'on $N\left(t\right)$ cuyos tiempos de inter-renovaci\'on tienen media finita, es un proceso Poisson con tasa $\lambda$ si y s\'olo s\'i $\esp\left[U\left(t\right)\right]=\lambda t$, para $t\geq0$.
\end{Note}


\begin{Teo}
Sea $N\left(t\right)$ un proceso puntual simple con puntos de localizaci\'on $T_{n}$ tal que $\eta\left(t\right)=\esp\left[N\left(\right)\right]$ es finita para cada $t$. Entonces para cualquier funci\'on $f:\rea_{+}\rightarrow\rea$,
\begin{eqnarray*}
\esp\left[\sum_{n=1}^{N\left(\right)}f\left(T_{n}\right)\right]=\int_{\left(0,t\right]}f\left(s\right)d\eta\left(s\right)\textrm{,  }t\geq0,
\end{eqnarray*}
suponiendo que la integral exista. Adem\'as si $X_{1},X_{2},\ldots$ son variables aleatorias definidas en el mismo espacio de probabilidad que el proceso $N\left(t\right)$ tal que $\esp\left[X_{n}|T_{n}=s\right]=f\left(s\right)$, independiente de $n$. Entonces
\begin{eqnarray*}
\esp\left[\sum_{n=1}^{N\left(t\right)}X_{n}\right]=\int_{\left(0,t\right]}f\left(s\right)d\eta\left(s\right)\textrm{,  }t\geq0,
\end{eqnarray*} 
suponiendo que la integral exista. 
\end{Teo}

\begin{Coro}[Identidad de Wald para Renovaciones]
Para el proceso de renovaci\'on $N\left(t\right)$,
\begin{eqnarray*}
\esp\left[T_{N\left(t\right)+1}\right]=\mu\esp\left[N\left(t\right)+1\right]\textrm{,  }t\geq0,
\end{eqnarray*}  
\end{Coro}


\begin{Def}
Sea $h\left(t\right)$ funci\'on de valores reales en $\rea$ acotada en intervalos finitos e igual a cero para $t<0$ La ecuaci\'on de renovaci\'on para $h\left(t\right)$ y la distribuci\'on $F$ es

\begin{eqnarray}\label{Ec.Renovacion}
H\left(t\right)=h\left(t\right)+\int_{\left[0,t\right]}H\left(t-s\right)dF\left(s\right)\textrm{,    }t\geq0,
\end{eqnarray}
donde $H\left(t\right)$ es una funci\'on de valores reales. Esto es $H=h+F\star H$. Decimos que $H\left(t\right)$ es soluci\'on de esta ecuaci\'on si satisface la ecuaci\'on, y es acotada en intervalos finitos e iguales a cero para $t<0$.
\end{Def}

\begin{Prop}
La funci\'on $U\star h\left(t\right)$ es la \'unica soluci\'on de la ecuaci\'on de renovaci\'on (\ref{Ec.Renovacion}).
\end{Prop}

\begin{Teo}[Teorema Renovaci\'on Elemental]
\begin{eqnarray*}
t^{-1}U\left(t\right)\rightarrow 1/\mu\textrm{,    cuando }t\rightarrow\infty.
\end{eqnarray*}
\end{Teo}


\begin{Note} Una funci\'on $h:\rea_{+}\rightarrow\rea$ es Directamente Riemann Integrable en los siguientes casos:
\begin{itemize}
\item[a)] $h\left(t\right)\geq0$ es decreciente y Riemann Integrable.
\item[b)] $h$ es continua excepto posiblemente en un conjunto de Lebesgue de medida 0, y $|h\left(t\right)|\leq b\left(t\right)$, donde $b$ es DRI.
\end{itemize}
\end{Note}

\begin{Teo}[Teorema Principal de Renovaci\'on]
Si $F$ es no aritm\'etica y $h\left(t\right)$ es Directamente Riemann Integrable (DRI), entonces

\begin{eqnarray*}
lim_{t\rightarrow\infty}U\star h=\frac{1}{\mu}\int_{\rea_{+}}h\left(s\right)ds.
\end{eqnarray*}
\end{Teo}

\begin{Prop}
Cualquier funci\'on $H\left(t\right)$ acotada en intervalos finitos y que es 0 para $t<0$ puede expresarse como
\begin{eqnarray*}
H\left(t\right)=U\star h\left(t\right)\textrm{,  donde }h\left(t\right)=H\left(t\right)-F\star H\left(t\right)
\end{eqnarray*}
\end{Prop}

\begin{Def}
Un proceso estoc\'astico $X\left(t\right)$ es crudamente regenerativo en un tiempo aleatorio positivo $T$ si
\begin{eqnarray*}
\esp\left[X\left(T+t\right)|T\right]=\esp\left[X\left(t\right)\right]\textrm{, para }t\geq0,\end{eqnarray*}
y con las esperanzas anteriores finitas.
\end{Def}

\begin{Prop}
Sup\'ongase que $X\left(t\right)$ es un proceso crudamente regenerativo en $T$, que tiene distribuci\'on $F$. Si $\esp\left[X\left(t\right)\right]$ es acotado en intervalos finitos, entonces
\begin{eqnarray*}
\esp\left[X\left(t\right)\right]=U\star h\left(t\right)\textrm{,  donde }h\left(t\right)=\esp\left[X\left(t\right)\indora\left(T>t\right)\right].
\end{eqnarray*}
\end{Prop}

\begin{Teo}[Regeneraci\'on Cruda]
Sup\'ongase que $X\left(t\right)$ es un proceso con valores positivo crudamente regenerativo en $T$, y def\'inase $M=\sup\left\{|X\left(t\right)|:t\leq T\right\}$. Si $T$ es no aritm\'etico y $M$ y $MT$ tienen media finita, entonces
\begin{eqnarray*}
lim_{t\rightarrow\infty}\esp\left[X\left(t\right)\right]=\frac{1}{\mu}\int_{\rea_{+}}h\left(s\right)ds,
\end{eqnarray*}
donde $h\left(t\right)=\esp\left[X\left(t\right)\indora\left(T>t\right)\right]$.
\end{Teo}


\begin{Note} Una funci\'on $h:\rea_{+}\rightarrow\rea$ es Directamente Riemann Integrable en los siguientes casos:
\begin{itemize}
\item[a)] $h\left(t\right)\geq0$ es decreciente y Riemann Integrable.
\item[b)] $h$ es continua excepto posiblemente en un conjunto de Lebesgue de medida 0, y $|h\left(t\right)|\leq b\left(t\right)$, donde $b$ es DRI.
\end{itemize}
\end{Note}

\begin{Teo}[Teorema Principal de Renovaci\'on]
Si $F$ es no aritm\'etica y $h\left(t\right)$ es Directamente Riemann Integrable (DRI), entonces

\begin{eqnarray*}
lim_{t\rightarrow\infty}U\star h=\frac{1}{\mu}\int_{\rea_{+}}h\left(s\right)ds.
\end{eqnarray*}
\end{Teo}

\begin{Prop}
Cualquier funci\'on $H\left(t\right)$ acotada en intervalos finitos y que es 0 para $t<0$ puede expresarse como
\begin{eqnarray*}
H\left(t\right)=U\star h\left(t\right)\textrm{,  donde }h\left(t\right)=H\left(t\right)-F\star H\left(t\right)
\end{eqnarray*}
\end{Prop}

\begin{Def}
Un proceso estoc\'astico $X\left(t\right)$ es crudamente regenerativo en un tiempo aleatorio positivo $T$ si
\begin{eqnarray*}
\esp\left[X\left(T+t\right)|T\right]=\esp\left[X\left(t\right)\right]\textrm{, para }t\geq0,\end{eqnarray*}
y con las esperanzas anteriores finitas.
\end{Def}

\begin{Prop}
Sup\'ongase que $X\left(t\right)$ es un proceso crudamente regenerativo en $T$, que tiene distribuci\'on $F$. Si $\esp\left[X\left(t\right)\right]$ es acotado en intervalos finitos, entonces
\begin{eqnarray*}
\esp\left[X\left(t\right)\right]=U\star h\left(t\right)\textrm{,  donde }h\left(t\right)=\esp\left[X\left(t\right)\indora\left(T>t\right)\right].
\end{eqnarray*}
\end{Prop}

\begin{Teo}[Regeneraci\'on Cruda]
Sup\'ongase que $X\left(t\right)$ es un proceso con valores positivo crudamente regenerativo en $T$, y def\'inase $M=\sup\left\{|X\left(t\right)|:t\leq T\right\}$. Si $T$ es no aritm\'etico y $M$ y $MT$ tienen media finita, entonces
\begin{eqnarray*}
lim_{t\rightarrow\infty}\esp\left[X\left(t\right)\right]=\frac{1}{\mu}\int_{\rea_{+}}h\left(s\right)ds,
\end{eqnarray*}
donde $h\left(t\right)=\esp\left[X\left(t\right)\indora\left(T>t\right)\right]$.
\end{Teo}

%________________________________________________________________________
\subsection{Procesos Regenerativos}
%________________________________________________________________________

Para $\left\{X\left(t\right):t\geq0\right\}$ Proceso Estoc\'astico a tiempo continuo con estado de espacios $S$, que es un espacio m\'etrico, con trayectorias continuas por la derecha y con l\'imites por la izquierda c.s. Sea $N\left(t\right)$ un proceso de renovaci\'on en $\rea_{+}$ definido en el mismo espacio de probabilidad que $X\left(t\right)$, con tiempos de renovaci\'on $T$ y tiempos de inter-renovaci\'on $\xi_{n}=T_{n}-T_{n-1}$, con misma distribuci\'on $F$ de media finita $\mu$.



\begin{Def}
Para el proceso $\left\{\left(N\left(t\right),X\left(t\right)\right):t\geq0\right\}$, sus trayectoria muestrales en el intervalo de tiempo $\left[T_{n-1},T_{n}\right)$ est\'an descritas por
\begin{eqnarray*}
\zeta_{n}=\left(\xi_{n},\left\{X\left(T_{n-1}+t\right):0\leq t<\xi_{n}\right\}\right)
\end{eqnarray*}
Este $\zeta_{n}$ es el $n$-\'esimo segmento del proceso. El proceso es regenerativo sobre los tiempos $T_{n}$ si sus segmentos $\zeta_{n}$ son independientes e id\'enticamennte distribuidos.
\end{Def}


\begin{Obs}
Si $\tilde{X}\left(t\right)$ con espacio de estados $\tilde{S}$ es regenerativo sobre $T_{n}$, entonces $X\left(t\right)=f\left(\tilde{X}\left(t\right)\right)$ tambi\'en es regenerativo sobre $T_{n}$, para cualquier funci\'on $f:\tilde{S}\rightarrow S$.
\end{Obs}

\begin{Obs}
Los procesos regenerativos son crudamente regenerativos, pero no al rev\'es.
\end{Obs}

\begin{Def}[Definici\'on Cl\'asica]
Un proceso estoc\'astico $X=\left\{X\left(t\right):t\geq0\right\}$ es llamado regenerativo is existe una variable aleatoria $R_{1}>0$ tal que
\begin{itemize}
\item[i)] $\left\{X\left(t+R_{1}\right):t\geq0\right\}$ es independiente de $\left\{\left\{X\left(t\right):t<R_{1}\right\},\right\}$
\item[ii)] $\left\{X\left(t+R_{1}\right):t\geq0\right\}$ es estoc\'asticamente equivalente a $\left\{X\left(t\right):t>0\right\}$
\end{itemize}

Llamamos a $R_{1}$ tiempo de regeneraci\'on, y decimos que $X$ se regenera en este punto.
\end{Def}

$\left\{X\left(t+R_{1}\right)\right\}$ es regenerativo con tiempo de regeneraci\'on $R_{2}$, independiente de $R_{1}$ pero con la misma distribuci\'on que $R_{1}$. Procediendo de esta manera se obtiene una secuencia de variables aleatorias independientes e id\'enticamente distribuidas $\left\{R_{n}\right\}$ llamados longitudes de ciclo. Si definimos a $Z_{k}\equiv R_{1}+R_{2}+\cdots+R_{k}$, se tiene un proceso de renovaci\'on llamado proceso de renovaci\'on encajado para $X$.

\begin{Note}
Un proceso regenerativo con media de la longitud de ciclo finita es llamado positivo recurrente.
\end{Note}


\begin{Def}
Para $x$ fijo y para cada $t\geq0$, sea $I_{x}\left(t\right)=1$ si $X\left(t\right)\leq x$,  $I_{x}\left(t\right)=0$ en caso contrario, y def\'inanse los tiempos promedio
\begin{eqnarray*}
\overline{X}&=&lim_{t\rightarrow\infty}\frac{1}{t}\int_{0}^{\infty}X\left(u\right)du\\
\prob\left(X_{\infty}\leq x\right)&=&lim_{t\rightarrow\infty}\frac{1}{t}\int_{0}^{\infty}I_{x}\left(u\right)du,
\end{eqnarray*}
cuando estos l\'imites existan.
\end{Def}

Como consecuencia del teorema de Renovaci\'on-Recompensa, se tiene que el primer l\'imite  existe y es igual a la constante
\begin{eqnarray*}
\overline{X}&=&\frac{\esp\left[\int_{0}^{R_{1}}X\left(t\right)dt\right]}{\esp\left[R_{1}\right]},
\end{eqnarray*}
suponiendo que ambas esperanzas son finitas.

\begin{Note}
\begin{itemize}
\item[a)] Si el proceso regenerativo $X$ es positivo recurrente y tiene trayectorias muestrales no negativas, entonces la ecuaci\'on anterior es v\'alida.
\item[b)] Si $X$ es positivo recurrente regenerativo, podemos construir una \'unica versi\'on estacionaria de este proceso, $X_{e}=\left\{X_{e}\left(t\right)\right\}$, donde $X_{e}$ es un proceso estoc\'astico regenerativo y estrictamente estacionario, con distribuci\'on marginal distribuida como $X_{\infty}$
\end{itemize}
\end{Note}

%________________________________________________________________________
\subsection{Procesos Regenerativos}
%________________________________________________________________________

Para $\left\{X\left(t\right):t\geq0\right\}$ Proceso Estoc\'astico a tiempo continuo con estado de espacios $S$, que es un espacio m\'etrico, con trayectorias continuas por la derecha y con l\'imites por la izquierda c.s. Sea $N\left(t\right)$ un proceso de renovaci\'on en $\rea_{+}$ definido en el mismo espacio de probabilidad que $X\left(t\right)$, con tiempos de renovaci\'on $T$ y tiempos de inter-renovaci\'on $\xi_{n}=T_{n}-T_{n-1}$, con misma distribuci\'on $F$ de media finita $\mu$.



\begin{Def}
Para el proceso $\left\{\left(N\left(t\right),X\left(t\right)\right):t\geq0\right\}$, sus trayectoria muestrales en el intervalo de tiempo $\left[T_{n-1},T_{n}\right)$ est\'an descritas por
\begin{eqnarray*}
\zeta_{n}=\left(\xi_{n},\left\{X\left(T_{n-1}+t\right):0\leq t<\xi_{n}\right\}\right)
\end{eqnarray*}
Este $\zeta_{n}$ es el $n$-\'esimo segmento del proceso. El proceso es regenerativo sobre los tiempos $T_{n}$ si sus segmentos $\zeta_{n}$ son independientes e id\'enticamennte distribuidos.
\end{Def}


\begin{Obs}
Si $\tilde{X}\left(t\right)$ con espacio de estados $\tilde{S}$ es regenerativo sobre $T_{n}$, entonces $X\left(t\right)=f\left(\tilde{X}\left(t\right)\right)$ tambi\'en es regenerativo sobre $T_{n}$, para cualquier funci\'on $f:\tilde{S}\rightarrow S$.
\end{Obs}

\begin{Obs}
Los procesos regenerativos son crudamente regenerativos, pero no al rev\'es.
\end{Obs}

\begin{Def}[Definici\'on Cl\'asica]
Un proceso estoc\'astico $X=\left\{X\left(t\right):t\geq0\right\}$ es llamado regenerativo is existe una variable aleatoria $R_{1}>0$ tal que
\begin{itemize}
\item[i)] $\left\{X\left(t+R_{1}\right):t\geq0\right\}$ es independiente de $\left\{\left\{X\left(t\right):t<R_{1}\right\},\right\}$
\item[ii)] $\left\{X\left(t+R_{1}\right):t\geq0\right\}$ es estoc\'asticamente equivalente a $\left\{X\left(t\right):t>0\right\}$
\end{itemize}

Llamamos a $R_{1}$ tiempo de regeneraci\'on, y decimos que $X$ se regenera en este punto.
\end{Def}

$\left\{X\left(t+R_{1}\right)\right\}$ es regenerativo con tiempo de regeneraci\'on $R_{2}$, independiente de $R_{1}$ pero con la misma distribuci\'on que $R_{1}$. Procediendo de esta manera se obtiene una secuencia de variables aleatorias independientes e id\'enticamente distribuidas $\left\{R_{n}\right\}$ llamados longitudes de ciclo. Si definimos a $Z_{k}\equiv R_{1}+R_{2}+\cdots+R_{k}$, se tiene un proceso de renovaci\'on llamado proceso de renovaci\'on encajado para $X$.

\begin{Note}
Un proceso regenerativo con media de la longitud de ciclo finita es llamado positivo recurrente.
\end{Note}


\begin{Def}
Para $x$ fijo y para cada $t\geq0$, sea $I_{x}\left(t\right)=1$ si $X\left(t\right)\leq x$,  $I_{x}\left(t\right)=0$ en caso contrario, y def\'inanse los tiempos promedio
\begin{eqnarray*}
\overline{X}&=&lim_{t\rightarrow\infty}\frac{1}{t}\int_{0}^{\infty}X\left(u\right)du\\
\prob\left(X_{\infty}\leq x\right)&=&lim_{t\rightarrow\infty}\frac{1}{t}\int_{0}^{\infty}I_{x}\left(u\right)du,
\end{eqnarray*}
cuando estos l\'imites existan.
\end{Def}

Como consecuencia del teorema de Renovaci\'on-Recompensa, se tiene que el primer l\'imite  existe y es igual a la constante
\begin{eqnarray*}
\overline{X}&=&\frac{\esp\left[\int_{0}^{R_{1}}X\left(t\right)dt\right]}{\esp\left[R_{1}\right]},
\end{eqnarray*}
suponiendo que ambas esperanzas son finitas.

\begin{Note}
\begin{itemize}
\item[a)] Si el proceso regenerativo $X$ es positivo recurrente y tiene trayectorias muestrales no negativas, entonces la ecuaci\'on anterior es v\'alida.
\item[b)] Si $X$ es positivo recurrente regenerativo, podemos construir una \'unica versi\'on estacionaria de este proceso, $X_{e}=\left\{X_{e}\left(t\right)\right\}$, donde $X_{e}$ es un proceso estoc\'astico regenerativo y estrictamente estacionario, con distribuci\'on marginal distribuida como $X_{\infty}$
\end{itemize}
\end{Note}

%
%___________________________________________________________________________________________
%\vspace{5.5cm}
%\chapter{Cadenas de Markov estacionarias}
%\vspace{-1.0cm}
%___________________________________________________________________________________________
%
\subsection{Propiedades de los Procesos de Renovaci\'on}
%___________________________________________________________________________________________
%

Los tiempos $T_{n}$ est\'an relacionados con los conteos de $N\left(t\right)$ por

\begin{eqnarray*}
\left\{N\left(t\right)\geq n\right\}&=&\left\{T_{n}\leq t\right\}\\
T_{N\left(t\right)}\leq &t&<T_{N\left(t\right)+1},
\end{eqnarray*}

adem\'as $N\left(T_{n}\right)=n$, y 

\begin{eqnarray*}
N\left(t\right)=\max\left\{n:T_{n}\leq t\right\}=\min\left\{n:T_{n+1}>t\right\}
\end{eqnarray*}

Por propiedades de la convoluci\'on se sabe que

\begin{eqnarray*}
P\left\{T_{n}\leq t\right\}=F^{n\star}\left(t\right)
\end{eqnarray*}
que es la $n$-\'esima convoluci\'on de $F$. Entonces 

\begin{eqnarray*}
\left\{N\left(t\right)\geq n\right\}&=&\left\{T_{n}\leq t\right\}\\
P\left\{N\left(t\right)\leq n\right\}&=&1-F^{\left(n+1\right)\star}\left(t\right)
\end{eqnarray*}

Adem\'as usando el hecho de que $\esp\left[N\left(t\right)\right]=\sum_{n=1}^{\infty}P\left\{N\left(t\right)\geq n\right\}$
se tiene que

\begin{eqnarray*}
\esp\left[N\left(t\right)\right]=\sum_{n=1}^{\infty}F^{n\star}\left(t\right)
\end{eqnarray*}

\begin{Prop}
Para cada $t\geq0$, la funci\'on generadora de momentos $\esp\left[e^{\alpha N\left(t\right)}\right]$ existe para alguna $\alpha$ en una vecindad del 0, y de aqu\'i que $\esp\left[N\left(t\right)^{m}\right]<\infty$, para $m\geq1$.
\end{Prop}


\begin{Note}
Si el primer tiempo de renovaci\'on $\xi_{1}$ no tiene la misma distribuci\'on que el resto de las $\xi_{n}$, para $n\geq2$, a $N\left(t\right)$ se le llama Proceso de Renovaci\'on retardado, donde si $\xi$ tiene distribuci\'on $G$, entonces el tiempo $T_{n}$ de la $n$-\'esima renovaci\'on tiene distribuci\'on $G\star F^{\left(n-1\right)\star}\left(t\right)$
\end{Note}


\begin{Teo}
Para una constante $\mu\leq\infty$ ( o variable aleatoria), las siguientes expresiones son equivalentes:

\begin{eqnarray}
lim_{n\rightarrow\infty}n^{-1}T_{n}&=&\mu,\textrm{ c.s.}\\
lim_{t\rightarrow\infty}t^{-1}N\left(t\right)&=&1/\mu,\textrm{ c.s.}
\end{eqnarray}
\end{Teo}


Es decir, $T_{n}$ satisface la Ley Fuerte de los Grandes N\'umeros s\'i y s\'olo s\'i $N\left/t\right)$ la cumple.


\begin{Coro}[Ley Fuerte de los Grandes N\'umeros para Procesos de Renovaci\'on]
Si $N\left(t\right)$ es un proceso de renovaci\'on cuyos tiempos de inter-renovaci\'on tienen media $\mu\leq\infty$, entonces
\begin{eqnarray}
t^{-1}N\left(t\right)\rightarrow 1/\mu,\textrm{ c.s. cuando }t\rightarrow\infty.
\end{eqnarray}

\end{Coro}


Considerar el proceso estoc\'astico de valores reales $\left\{Z\left(t\right):t\geq0\right\}$ en el mismo espacio de probabilidad que $N\left(t\right)$

\begin{Def}
Para el proceso $\left\{Z\left(t\right):t\geq0\right\}$ se define la fluctuaci\'on m\'axima de $Z\left(t\right)$ en el intervalo $\left(T_{n-1},T_{n}\right]$:
\begin{eqnarray*}
M_{n}=\sup_{T_{n-1}<t\leq T_{n}}|Z\left(t\right)-Z\left(T_{n-1}\right)|
\end{eqnarray*}
\end{Def}

\begin{Teo}
Sup\'ongase que $n^{-1}T_{n}\rightarrow\mu$ c.s. cuando $n\rightarrow\infty$, donde $\mu\leq\infty$ es una constante o variable aleatoria. Sea $a$ una constante o variable aleatoria que puede ser infinita cuando $\mu$ es finita, y considere las expresiones l\'imite:
\begin{eqnarray}
lim_{n\rightarrow\infty}n^{-1}Z\left(T_{n}\right)&=&a,\textrm{ c.s.}\\
lim_{t\rightarrow\infty}t^{-1}Z\left(t\right)&=&a/\mu,\textrm{ c.s.}
\end{eqnarray}
La segunda expresi\'on implica la primera. Conversamente, la primera implica la segunda si el proceso $Z\left(t\right)$ es creciente, o si $lim_{n\rightarrow\infty}n^{-1}M_{n}=0$ c.s.
\end{Teo}

\begin{Coro}
Si $N\left(t\right)$ es un proceso de renovaci\'on, y $\left(Z\left(T_{n}\right)-Z\left(T_{n-1}\right),M_{n}\right)$, para $n\geq1$, son variables aleatorias independientes e id\'enticamente distribuidas con media finita, entonces,
\begin{eqnarray}
lim_{t\rightarrow\infty}t^{-1}Z\left(t\right)\rightarrow\frac{\esp\left[Z\left(T_{1}\right)-Z\left(T_{0}\right)\right]}{\esp\left[T_{1}\right]},\textrm{ c.s. cuando  }t\rightarrow\infty.
\end{eqnarray}
\end{Coro}


%___________________________________________________________________________________________
%
%\subsection{Propiedades de los Procesos de Renovaci\'on}
%___________________________________________________________________________________________
%

Los tiempos $T_{n}$ est\'an relacionados con los conteos de $N\left(t\right)$ por

\begin{eqnarray*}
\left\{N\left(t\right)\geq n\right\}&=&\left\{T_{n}\leq t\right\}\\
T_{N\left(t\right)}\leq &t&<T_{N\left(t\right)+1},
\end{eqnarray*}

adem\'as $N\left(T_{n}\right)=n$, y 

\begin{eqnarray*}
N\left(t\right)=\max\left\{n:T_{n}\leq t\right\}=\min\left\{n:T_{n+1}>t\right\}
\end{eqnarray*}

Por propiedades de la convoluci\'on se sabe que

\begin{eqnarray*}
P\left\{T_{n}\leq t\right\}=F^{n\star}\left(t\right)
\end{eqnarray*}
que es la $n$-\'esima convoluci\'on de $F$. Entonces 

\begin{eqnarray*}
\left\{N\left(t\right)\geq n\right\}&=&\left\{T_{n}\leq t\right\}\\
P\left\{N\left(t\right)\leq n\right\}&=&1-F^{\left(n+1\right)\star}\left(t\right)
\end{eqnarray*}

Adem\'as usando el hecho de que $\esp\left[N\left(t\right)\right]=\sum_{n=1}^{\infty}P\left\{N\left(t\right)\geq n\right\}$
se tiene que

\begin{eqnarray*}
\esp\left[N\left(t\right)\right]=\sum_{n=1}^{\infty}F^{n\star}\left(t\right)
\end{eqnarray*}

\begin{Prop}
Para cada $t\geq0$, la funci\'on generadora de momentos $\esp\left[e^{\alpha N\left(t\right)}\right]$ existe para alguna $\alpha$ en una vecindad del 0, y de aqu\'i que $\esp\left[N\left(t\right)^{m}\right]<\infty$, para $m\geq1$.
\end{Prop}


\begin{Note}
Si el primer tiempo de renovaci\'on $\xi_{1}$ no tiene la misma distribuci\'on que el resto de las $\xi_{n}$, para $n\geq2$, a $N\left(t\right)$ se le llama Proceso de Renovaci\'on retardado, donde si $\xi$ tiene distribuci\'on $G$, entonces el tiempo $T_{n}$ de la $n$-\'esima renovaci\'on tiene distribuci\'on $G\star F^{\left(n-1\right)\star}\left(t\right)$
\end{Note}


\begin{Teo}
Para una constante $\mu\leq\infty$ ( o variable aleatoria), las siguientes expresiones son equivalentes:

\begin{eqnarray}
lim_{n\rightarrow\infty}n^{-1}T_{n}&=&\mu,\textrm{ c.s.}\\
lim_{t\rightarrow\infty}t^{-1}N\left(t\right)&=&1/\mu,\textrm{ c.s.}
\end{eqnarray}
\end{Teo}


Es decir, $T_{n}$ satisface la Ley Fuerte de los Grandes N\'umeros s\'i y s\'olo s\'i $N\left/t\right)$ la cumple.


\begin{Coro}[Ley Fuerte de los Grandes N\'umeros para Procesos de Renovaci\'on]
Si $N\left(t\right)$ es un proceso de renovaci\'on cuyos tiempos de inter-renovaci\'on tienen media $\mu\leq\infty$, entonces
\begin{eqnarray}
t^{-1}N\left(t\right)\rightarrow 1/\mu,\textrm{ c.s. cuando }t\rightarrow\infty.
\end{eqnarray}

\end{Coro}


Considerar el proceso estoc\'astico de valores reales $\left\{Z\left(t\right):t\geq0\right\}$ en el mismo espacio de probabilidad que $N\left(t\right)$

\begin{Def}
Para el proceso $\left\{Z\left(t\right):t\geq0\right\}$ se define la fluctuaci\'on m\'axima de $Z\left(t\right)$ en el intervalo $\left(T_{n-1},T_{n}\right]$:
\begin{eqnarray*}
M_{n}=\sup_{T_{n-1}<t\leq T_{n}}|Z\left(t\right)-Z\left(T_{n-1}\right)|
\end{eqnarray*}
\end{Def}

\begin{Teo}
Sup\'ongase que $n^{-1}T_{n}\rightarrow\mu$ c.s. cuando $n\rightarrow\infty$, donde $\mu\leq\infty$ es una constante o variable aleatoria. Sea $a$ una constante o variable aleatoria que puede ser infinita cuando $\mu$ es finita, y considere las expresiones l\'imite:
\begin{eqnarray}
lim_{n\rightarrow\infty}n^{-1}Z\left(T_{n}\right)&=&a,\textrm{ c.s.}\\
lim_{t\rightarrow\infty}t^{-1}Z\left(t\right)&=&a/\mu,\textrm{ c.s.}
\end{eqnarray}
La segunda expresi\'on implica la primera. Conversamente, la primera implica la segunda si el proceso $Z\left(t\right)$ es creciente, o si $lim_{n\rightarrow\infty}n^{-1}M_{n}=0$ c.s.
\end{Teo}

\begin{Coro}
Si $N\left(t\right)$ es un proceso de renovaci\'on, y $\left(Z\left(T_{n}\right)-Z\left(T_{n-1}\right),M_{n}\right)$, para $n\geq1$, son variables aleatorias independientes e id\'enticamente distribuidas con media finita, entonces,
\begin{eqnarray}
lim_{t\rightarrow\infty}t^{-1}Z\left(t\right)\rightarrow\frac{\esp\left[Z\left(T_{1}\right)-Z\left(T_{0}\right)\right]}{\esp\left[T_{1}\right]},\textrm{ c.s. cuando  }t\rightarrow\infty.
\end{eqnarray}
\end{Coro}

%___________________________________________________________________________________________
%
%\subsection{Propiedades de los Procesos de Renovaci\'on}
%___________________________________________________________________________________________
%

Los tiempos $T_{n}$ est\'an relacionados con los conteos de $N\left(t\right)$ por

\begin{eqnarray*}
\left\{N\left(t\right)\geq n\right\}&=&\left\{T_{n}\leq t\right\}\\
T_{N\left(t\right)}\leq &t&<T_{N\left(t\right)+1},
\end{eqnarray*}

adem\'as $N\left(T_{n}\right)=n$, y 

\begin{eqnarray*}
N\left(t\right)=\max\left\{n:T_{n}\leq t\right\}=\min\left\{n:T_{n+1}>t\right\}
\end{eqnarray*}

Por propiedades de la convoluci\'on se sabe que

\begin{eqnarray*}
P\left\{T_{n}\leq t\right\}=F^{n\star}\left(t\right)
\end{eqnarray*}
que es la $n$-\'esima convoluci\'on de $F$. Entonces 

\begin{eqnarray*}
\left\{N\left(t\right)\geq n\right\}&=&\left\{T_{n}\leq t\right\}\\
P\left\{N\left(t\right)\leq n\right\}&=&1-F^{\left(n+1\right)\star}\left(t\right)
\end{eqnarray*}

Adem\'as usando el hecho de que $\esp\left[N\left(t\right)\right]=\sum_{n=1}^{\infty}P\left\{N\left(t\right)\geq n\right\}$
se tiene que

\begin{eqnarray*}
\esp\left[N\left(t\right)\right]=\sum_{n=1}^{\infty}F^{n\star}\left(t\right)
\end{eqnarray*}

\begin{Prop}
Para cada $t\geq0$, la funci\'on generadora de momentos $\esp\left[e^{\alpha N\left(t\right)}\right]$ existe para alguna $\alpha$ en una vecindad del 0, y de aqu\'i que $\esp\left[N\left(t\right)^{m}\right]<\infty$, para $m\geq1$.
\end{Prop}


\begin{Note}
Si el primer tiempo de renovaci\'on $\xi_{1}$ no tiene la misma distribuci\'on que el resto de las $\xi_{n}$, para $n\geq2$, a $N\left(t\right)$ se le llama Proceso de Renovaci\'on retardado, donde si $\xi$ tiene distribuci\'on $G$, entonces el tiempo $T_{n}$ de la $n$-\'esima renovaci\'on tiene distribuci\'on $G\star F^{\left(n-1\right)\star}\left(t\right)$
\end{Note}


\begin{Teo}
Para una constante $\mu\leq\infty$ ( o variable aleatoria), las siguientes expresiones son equivalentes:

\begin{eqnarray}
lim_{n\rightarrow\infty}n^{-1}T_{n}&=&\mu,\textrm{ c.s.}\\
lim_{t\rightarrow\infty}t^{-1}N\left(t\right)&=&1/\mu,\textrm{ c.s.}
\end{eqnarray}
\end{Teo}


Es decir, $T_{n}$ satisface la Ley Fuerte de los Grandes N\'umeros s\'i y s\'olo s\'i $N\left/t\right)$ la cumple.


\begin{Coro}[Ley Fuerte de los Grandes N\'umeros para Procesos de Renovaci\'on]
Si $N\left(t\right)$ es un proceso de renovaci\'on cuyos tiempos de inter-renovaci\'on tienen media $\mu\leq\infty$, entonces
\begin{eqnarray}
t^{-1}N\left(t\right)\rightarrow 1/\mu,\textrm{ c.s. cuando }t\rightarrow\infty.
\end{eqnarray}

\end{Coro}


Considerar el proceso estoc\'astico de valores reales $\left\{Z\left(t\right):t\geq0\right\}$ en el mismo espacio de probabilidad que $N\left(t\right)$

\begin{Def}
Para el proceso $\left\{Z\left(t\right):t\geq0\right\}$ se define la fluctuaci\'on m\'axima de $Z\left(t\right)$ en el intervalo $\left(T_{n-1},T_{n}\right]$:
\begin{eqnarray*}
M_{n}=\sup_{T_{n-1}<t\leq T_{n}}|Z\left(t\right)-Z\left(T_{n-1}\right)|
\end{eqnarray*}
\end{Def}

\begin{Teo}
Sup\'ongase que $n^{-1}T_{n}\rightarrow\mu$ c.s. cuando $n\rightarrow\infty$, donde $\mu\leq\infty$ es una constante o variable aleatoria. Sea $a$ una constante o variable aleatoria que puede ser infinita cuando $\mu$ es finita, y considere las expresiones l\'imite:
\begin{eqnarray}
lim_{n\rightarrow\infty}n^{-1}Z\left(T_{n}\right)&=&a,\textrm{ c.s.}\\
lim_{t\rightarrow\infty}t^{-1}Z\left(t\right)&=&a/\mu,\textrm{ c.s.}
\end{eqnarray}
La segunda expresi\'on implica la primera. Conversamente, la primera implica la segunda si el proceso $Z\left(t\right)$ es creciente, o si $lim_{n\rightarrow\infty}n^{-1}M_{n}=0$ c.s.
\end{Teo}

\begin{Coro}
Si $N\left(t\right)$ es un proceso de renovaci\'on, y $\left(Z\left(T_{n}\right)-Z\left(T_{n-1}\right),M_{n}\right)$, para $n\geq1$, son variables aleatorias independientes e id\'enticamente distribuidas con media finita, entonces,
\begin{eqnarray}
lim_{t\rightarrow\infty}t^{-1}Z\left(t\right)\rightarrow\frac{\esp\left[Z\left(T_{1}\right)-Z\left(T_{0}\right)\right]}{\esp\left[T_{1}\right]},\textrm{ c.s. cuando  }t\rightarrow\infty.
\end{eqnarray}
\end{Coro}



%___________________________________________________________________________________________
%
\subsection{Propiedades de los Procesos de Renovaci\'on}
%___________________________________________________________________________________________
%

Los tiempos $T_{n}$ est\'an relacionados con los conteos de $N\left(t\right)$ por

\begin{eqnarray*}
\left\{N\left(t\right)\geq n\right\}&=&\left\{T_{n}\leq t\right\}\\
T_{N\left(t\right)}\leq &t&<T_{N\left(t\right)+1},
\end{eqnarray*}

adem\'as $N\left(T_{n}\right)=n$, y 

\begin{eqnarray*}
N\left(t\right)=\max\left\{n:T_{n}\leq t\right\}=\min\left\{n:T_{n+1}>t\right\}
\end{eqnarray*}

Por propiedades de la convoluci\'on se sabe que

\begin{eqnarray*}
P\left\{T_{n}\leq t\right\}=F^{n\star}\left(t\right)
\end{eqnarray*}
que es la $n$-\'esima convoluci\'on de $F$. Entonces 

\begin{eqnarray*}
\left\{N\left(t\right)\geq n\right\}&=&\left\{T_{n}\leq t\right\}\\
P\left\{N\left(t\right)\leq n\right\}&=&1-F^{\left(n+1\right)\star}\left(t\right)
\end{eqnarray*}

Adem\'as usando el hecho de que $\esp\left[N\left(t\right)\right]=\sum_{n=1}^{\infty}P\left\{N\left(t\right)\geq n\right\}$
se tiene que

\begin{eqnarray*}
\esp\left[N\left(t\right)\right]=\sum_{n=1}^{\infty}F^{n\star}\left(t\right)
\end{eqnarray*}

\begin{Prop}
Para cada $t\geq0$, la funci\'on generadora de momentos $\esp\left[e^{\alpha N\left(t\right)}\right]$ existe para alguna $\alpha$ en una vecindad del 0, y de aqu\'i que $\esp\left[N\left(t\right)^{m}\right]<\infty$, para $m\geq1$.
\end{Prop}


\begin{Note}
Si el primer tiempo de renovaci\'on $\xi_{1}$ no tiene la misma distribuci\'on que el resto de las $\xi_{n}$, para $n\geq2$, a $N\left(t\right)$ se le llama Proceso de Renovaci\'on retardado, donde si $\xi$ tiene distribuci\'on $G$, entonces el tiempo $T_{n}$ de la $n$-\'esima renovaci\'on tiene distribuci\'on $G\star F^{\left(n-1\right)\star}\left(t\right)$
\end{Note}


\begin{Teo}
Para una constante $\mu\leq\infty$ ( o variable aleatoria), las siguientes expresiones son equivalentes:

\begin{eqnarray}
lim_{n\rightarrow\infty}n^{-1}T_{n}&=&\mu,\textrm{ c.s.}\\
lim_{t\rightarrow\infty}t^{-1}N\left(t\right)&=&1/\mu,\textrm{ c.s.}
\end{eqnarray}
\end{Teo}


Es decir, $T_{n}$ satisface la Ley Fuerte de los Grandes N\'umeros s\'i y s\'olo s\'i $N\left/t\right)$ la cumple.


\begin{Coro}[Ley Fuerte de los Grandes N\'umeros para Procesos de Renovaci\'on]
Si $N\left(t\right)$ es un proceso de renovaci\'on cuyos tiempos de inter-renovaci\'on tienen media $\mu\leq\infty$, entonces
\begin{eqnarray}
t^{-1}N\left(t\right)\rightarrow 1/\mu,\textrm{ c.s. cuando }t\rightarrow\infty.
\end{eqnarray}

\end{Coro}


Considerar el proceso estoc\'astico de valores reales $\left\{Z\left(t\right):t\geq0\right\}$ en el mismo espacio de probabilidad que $N\left(t\right)$

\begin{Def}
Para el proceso $\left\{Z\left(t\right):t\geq0\right\}$ se define la fluctuaci\'on m\'axima de $Z\left(t\right)$ en el intervalo $\left(T_{n-1},T_{n}\right]$:
\begin{eqnarray*}
M_{n}=\sup_{T_{n-1}<t\leq T_{n}}|Z\left(t\right)-Z\left(T_{n-1}\right)|
\end{eqnarray*}
\end{Def}

\begin{Teo}
Sup\'ongase que $n^{-1}T_{n}\rightarrow\mu$ c.s. cuando $n\rightarrow\infty$, donde $\mu\leq\infty$ es una constante o variable aleatoria. Sea $a$ una constante o variable aleatoria que puede ser infinita cuando $\mu$ es finita, y considere las expresiones l\'imite:
\begin{eqnarray}
lim_{n\rightarrow\infty}n^{-1}Z\left(T_{n}\right)&=&a,\textrm{ c.s.}\\
lim_{t\rightarrow\infty}t^{-1}Z\left(t\right)&=&a/\mu,\textrm{ c.s.}
\end{eqnarray}
La segunda expresi\'on implica la primera. Conversamente, la primera implica la segunda si el proceso $Z\left(t\right)$ es creciente, o si $lim_{n\rightarrow\infty}n^{-1}M_{n}=0$ c.s.
\end{Teo}

\begin{Coro}
Si $N\left(t\right)$ es un proceso de renovaci\'on, y $\left(Z\left(T_{n}\right)-Z\left(T_{n-1}\right),M_{n}\right)$, para $n\geq1$, son variables aleatorias independientes e id\'enticamente distribuidas con media finita, entonces,
\begin{eqnarray}
lim_{t\rightarrow\infty}t^{-1}Z\left(t\right)\rightarrow\frac{\esp\left[Z\left(T_{1}\right)-Z\left(T_{0}\right)\right]}{\esp\left[T_{1}\right]},\textrm{ c.s. cuando  }t\rightarrow\infty.
\end{eqnarray}
\end{Coro}




%__________________________________________________________________________________________
\subsection{Procesos Regenerativos Estacionarios - Stidham \cite{Stidham}}
%__________________________________________________________________________________________


Un proceso estoc\'astico a tiempo continuo $\left\{V\left(t\right),t\geq0\right\}$ es un proceso regenerativo si existe una sucesi\'on de variables aleatorias independientes e id\'enticamente distribuidas $\left\{X_{1},X_{2},\ldots\right\}$, sucesi\'on de renovaci\'on, tal que para cualquier conjunto de Borel $A$, 

\begin{eqnarray*}
\prob\left\{V\left(t\right)\in A|X_{1}+X_{2}+\cdots+X_{R\left(t\right)}=s,\left\{V\left(\tau\right),\tau<s\right\}\right\}=\prob\left\{V\left(t-s\right)\in A|X_{1}>t-s\right\},
\end{eqnarray*}
para todo $0\leq s\leq t$, donde $R\left(t\right)=\max\left\{X_{1}+X_{2}+\cdots+X_{j}\leq t\right\}=$n\'umero de renovaciones ({\emph{puntos de regeneraci\'on}}) que ocurren en $\left[0,t\right]$. El intervalo $\left[0,X_{1}\right)$ es llamado {\emph{primer ciclo de regeneraci\'on}} de $\left\{V\left(t \right),t\geq0\right\}$, $\left[X_{1},X_{1}+X_{2}\right)$ el {\emph{segundo ciclo de regeneraci\'on}}, y as\'i sucesivamente.

Sea $X=X_{1}$ y sea $F$ la funci\'on de distrbuci\'on de $X$


\begin{Def}
Se define el proceso estacionario, $\left\{V^{*}\left(t\right),t\geq0\right\}$, para $\left\{V\left(t\right),t\geq0\right\}$ por

\begin{eqnarray*}
\prob\left\{V\left(t\right)\in A\right\}=\frac{1}{\esp\left[X\right]}\int_{0}^{\infty}\prob\left\{V\left(t+x\right)\in A|X>x\right\}\left(1-F\left(x\right)\right)dx,
\end{eqnarray*} 
para todo $t\geq0$ y todo conjunto de Borel $A$.
\end{Def}

\begin{Def}
Una distribuci\'on se dice que es {\emph{aritm\'etica}} si todos sus puntos de incremento son m\'ultiplos de la forma $0,\lambda, 2\lambda,\ldots$ para alguna $\lambda>0$ entera.
\end{Def}


\begin{Def}
Una modificaci\'on medible de un proceso $\left\{V\left(t\right),t\geq0\right\}$, es una versi\'on de este, $\left\{V\left(t,w\right)\right\}$ conjuntamente medible para $t\geq0$ y para $w\in S$, $S$ espacio de estados para $\left\{V\left(t\right),t\geq0\right\}$.
\end{Def}

\begin{Teo}
Sea $\left\{V\left(t\right),t\geq\right\}$ un proceso regenerativo no negativo con modificaci\'on medible. Sea $\esp\left[X\right]<\infty$. Entonces el proceso estacionario dado por la ecuaci\'on anterior est\'a bien definido y tiene funci\'on de distribuci\'on independiente de $t$, adem\'as
\begin{itemize}
\item[i)] \begin{eqnarray*}
\esp\left[V^{*}\left(0\right)\right]&=&\frac{\esp\left[\int_{0}^{X}V\left(s\right)ds\right]}{\esp\left[X\right]}\end{eqnarray*}
\item[ii)] Si $\esp\left[V^{*}\left(0\right)\right]<\infty$, equivalentemente, si $\esp\left[\int_{0}^{X}V\left(s\right)ds\right]<\infty$,entonces
\begin{eqnarray*}
\frac{\int_{0}^{t}V\left(s\right)ds}{t}\rightarrow\frac{\esp\left[\int_{0}^{X}V\left(s\right)ds\right]}{\esp\left[X\right]}
\end{eqnarray*}
con probabilidad 1 y en media, cuando $t\rightarrow\infty$.
\end{itemize}
\end{Teo}

%______________________________________________________________________
\subsection{Procesos de Renovaci\'on}
%______________________________________________________________________

\begin{Def}\label{Def.Tn}
Sean $0\leq T_{1}\leq T_{2}\leq \ldots$ son tiempos aleatorios infinitos en los cuales ocurren ciertos eventos. El n\'umero de tiempos $T_{n}$ en el intervalo $\left[0,t\right)$ es

\begin{eqnarray}
N\left(t\right)=\sum_{n=1}^{\infty}\indora\left(T_{n}\leq t\right),
\end{eqnarray}
para $t\geq0$.
\end{Def}

Si se consideran los puntos $T_{n}$ como elementos de $\rea_{+}$, y $N\left(t\right)$ es el n\'umero de puntos en $\rea$. El proceso denotado por $\left\{N\left(t\right):t\geq0\right\}$, denotado por $N\left(t\right)$, es un proceso puntual en $\rea_{+}$. Los $T_{n}$ son los tiempos de ocurrencia, el proceso puntual $N\left(t\right)$ es simple si su n\'umero de ocurrencias son distintas: $0<T_{1}<T_{2}<\ldots$ casi seguramente.

\begin{Def}
Un proceso puntual $N\left(t\right)$ es un proceso de renovaci\'on si los tiempos de interocurrencia $\xi_{n}=T_{n}-T_{n-1}$, para $n\geq1$, son independientes e identicamente distribuidos con distribuci\'on $F$, donde $F\left(0\right)=0$ y $T_{0}=0$. Los $T_{n}$ son llamados tiempos de renovaci\'on, referente a la independencia o renovaci\'on de la informaci\'on estoc\'astica en estos tiempos. Los $\xi_{n}$ son los tiempos de inter-renovaci\'on, y $N\left(t\right)$ es el n\'umero de renovaciones en el intervalo $\left[0,t\right)$
\end{Def}


\begin{Note}
Para definir un proceso de renovaci\'on para cualquier contexto, solamente hay que especificar una distribuci\'on $F$, con $F\left(0\right)=0$, para los tiempos de inter-renovaci\'on. La funci\'on $F$ en turno degune las otra variables aleatorias. De manera formal, existe un espacio de probabilidad y una sucesi\'on de variables aleatorias $\xi_{1},\xi_{2},\ldots$ definidas en este con distribuci\'on $F$. Entonces las otras cantidades son $T_{n}=\sum_{k=1}^{n}\xi_{k}$ y $N\left(t\right)=\sum_{n=1}^{\infty}\indora\left(T_{n}\leq t\right)$, donde $T_{n}\rightarrow\infty$ casi seguramente por la Ley Fuerte de los Grandes Números.
\end{Note}

%___________________________________________________________________________________________
%
\subsection{Teorema Principal de Renovaci\'on}
%___________________________________________________________________________________________
%

\begin{Note} Una funci\'on $h:\rea_{+}\rightarrow\rea$ es Directamente Riemann Integrable en los siguientes casos:
\begin{itemize}
\item[a)] $h\left(t\right)\geq0$ es decreciente y Riemann Integrable.
\item[b)] $h$ es continua excepto posiblemente en un conjunto de Lebesgue de medida 0, y $|h\left(t\right)|\leq b\left(t\right)$, donde $b$ es DRI.
\end{itemize}
\end{Note}

\begin{Teo}[Teorema Principal de Renovaci\'on]
Si $F$ es no aritm\'etica y $h\left(t\right)$ es Directamente Riemann Integrable (DRI), entonces

\begin{eqnarray*}
lim_{t\rightarrow\infty}U\star h=\frac{1}{\mu}\int_{\rea_{+}}h\left(s\right)ds.
\end{eqnarray*}
\end{Teo}

\begin{Prop}
Cualquier funci\'on $H\left(t\right)$ acotada en intervalos finitos y que es 0 para $t<0$ puede expresarse como
\begin{eqnarray*}
H\left(t\right)=U\star h\left(t\right)\textrm{,  donde }h\left(t\right)=H\left(t\right)-F\star H\left(t\right)
\end{eqnarray*}
\end{Prop}

\begin{Def}
Un proceso estoc\'astico $X\left(t\right)$ es crudamente regenerativo en un tiempo aleatorio positivo $T$ si
\begin{eqnarray*}
\esp\left[X\left(T+t\right)|T\right]=\esp\left[X\left(t\right)\right]\textrm{, para }t\geq0,\end{eqnarray*}
y con las esperanzas anteriores finitas.
\end{Def}

\begin{Prop}
Sup\'ongase que $X\left(t\right)$ es un proceso crudamente regenerativo en $T$, que tiene distribuci\'on $F$. Si $\esp\left[X\left(t\right)\right]$ es acotado en intervalos finitos, entonces
\begin{eqnarray*}
\esp\left[X\left(t\right)\right]=U\star h\left(t\right)\textrm{,  donde }h\left(t\right)=\esp\left[X\left(t\right)\indora\left(T>t\right)\right].
\end{eqnarray*}
\end{Prop}

\begin{Teo}[Regeneraci\'on Cruda]
Sup\'ongase que $X\left(t\right)$ es un proceso con valores positivo crudamente regenerativo en $T$, y def\'inase $M=\sup\left\{|X\left(t\right)|:t\leq T\right\}$. Si $T$ es no aritm\'etico y $M$ y $MT$ tienen media finita, entonces
\begin{eqnarray*}
lim_{t\rightarrow\infty}\esp\left[X\left(t\right)\right]=\frac{1}{\mu}\int_{\rea_{+}}h\left(s\right)ds,
\end{eqnarray*}
donde $h\left(t\right)=\esp\left[X\left(t\right)\indora\left(T>t\right)\right]$.
\end{Teo}



%___________________________________________________________________________________________
%
\subsection{Funci\'on de Renovaci\'on}
%___________________________________________________________________________________________
%


\begin{Def}
Sea $h\left(t\right)$ funci\'on de valores reales en $\rea$ acotada en intervalos finitos e igual a cero para $t<0$ La ecuaci\'on de renovaci\'on para $h\left(t\right)$ y la distribuci\'on $F$ es

\begin{eqnarray}\label{Ec.Renovacion}
H\left(t\right)=h\left(t\right)+\int_{\left[0,t\right]}H\left(t-s\right)dF\left(s\right)\textrm{,    }t\geq0,
\end{eqnarray}
donde $H\left(t\right)$ es una funci\'on de valores reales. Esto es $H=h+F\star H$. Decimos que $H\left(t\right)$ es soluci\'on de esta ecuaci\'on si satisface la ecuaci\'on, y es acotada en intervalos finitos e iguales a cero para $t<0$.
\end{Def}

\begin{Prop}
La funci\'on $U\star h\left(t\right)$ es la \'unica soluci\'on de la ecuaci\'on de renovaci\'on (\ref{Ec.Renovacion}).
\end{Prop}

\begin{Teo}[Teorema Renovaci\'on Elemental]
\begin{eqnarray*}
t^{-1}U\left(t\right)\rightarrow 1/\mu\textrm{,    cuando }t\rightarrow\infty.
\end{eqnarray*}
\end{Teo}

%___________________________________________________________________________________________
%
\subsection{Propiedades de los Procesos de Renovaci\'on}
%___________________________________________________________________________________________
%

Los tiempos $T_{n}$ est\'an relacionados con los conteos de $N\left(t\right)$ por

\begin{eqnarray*}
\left\{N\left(t\right)\geq n\right\}&=&\left\{T_{n}\leq t\right\}\\
T_{N\left(t\right)}\leq &t&<T_{N\left(t\right)+1},
\end{eqnarray*}

adem\'as $N\left(T_{n}\right)=n$, y 

\begin{eqnarray*}
N\left(t\right)=\max\left\{n:T_{n}\leq t\right\}=\min\left\{n:T_{n+1}>t\right\}
\end{eqnarray*}

Por propiedades de la convoluci\'on se sabe que

\begin{eqnarray*}
P\left\{T_{n}\leq t\right\}=F^{n\star}\left(t\right)
\end{eqnarray*}
que es la $n$-\'esima convoluci\'on de $F$. Entonces 

\begin{eqnarray*}
\left\{N\left(t\right)\geq n\right\}&=&\left\{T_{n}\leq t\right\}\\
P\left\{N\left(t\right)\leq n\right\}&=&1-F^{\left(n+1\right)\star}\left(t\right)
\end{eqnarray*}

Adem\'as usando el hecho de que $\esp\left[N\left(t\right)\right]=\sum_{n=1}^{\infty}P\left\{N\left(t\right)\geq n\right\}$
se tiene que

\begin{eqnarray*}
\esp\left[N\left(t\right)\right]=\sum_{n=1}^{\infty}F^{n\star}\left(t\right)
\end{eqnarray*}

\begin{Prop}
Para cada $t\geq0$, la funci\'on generadora de momentos $\esp\left[e^{\alpha N\left(t\right)}\right]$ existe para alguna $\alpha$ en una vecindad del 0, y de aqu\'i que $\esp\left[N\left(t\right)^{m}\right]<\infty$, para $m\geq1$.
\end{Prop}


\begin{Note}
Si el primer tiempo de renovaci\'on $\xi_{1}$ no tiene la misma distribuci\'on que el resto de las $\xi_{n}$, para $n\geq2$, a $N\left(t\right)$ se le llama Proceso de Renovaci\'on retardado, donde si $\xi$ tiene distribuci\'on $G$, entonces el tiempo $T_{n}$ de la $n$-\'esima renovaci\'on tiene distribuci\'on $G\star F^{\left(n-1\right)\star}\left(t\right)$
\end{Note}


\begin{Teo}
Para una constante $\mu\leq\infty$ ( o variable aleatoria), las siguientes expresiones son equivalentes:

\begin{eqnarray}
lim_{n\rightarrow\infty}n^{-1}T_{n}&=&\mu,\textrm{ c.s.}\\
lim_{t\rightarrow\infty}t^{-1}N\left(t\right)&=&1/\mu,\textrm{ c.s.}
\end{eqnarray}
\end{Teo}


Es decir, $T_{n}$ satisface la Ley Fuerte de los Grandes N\'umeros s\'i y s\'olo s\'i $N\left/t\right)$ la cumple.


\begin{Coro}[Ley Fuerte de los Grandes N\'umeros para Procesos de Renovaci\'on]
Si $N\left(t\right)$ es un proceso de renovaci\'on cuyos tiempos de inter-renovaci\'on tienen media $\mu\leq\infty$, entonces
\begin{eqnarray}
t^{-1}N\left(t\right)\rightarrow 1/\mu,\textrm{ c.s. cuando }t\rightarrow\infty.
\end{eqnarray}

\end{Coro}


Considerar el proceso estoc\'astico de valores reales $\left\{Z\left(t\right):t\geq0\right\}$ en el mismo espacio de probabilidad que $N\left(t\right)$

\begin{Def}
Para el proceso $\left\{Z\left(t\right):t\geq0\right\}$ se define la fluctuaci\'on m\'axima de $Z\left(t\right)$ en el intervalo $\left(T_{n-1},T_{n}\right]$:
\begin{eqnarray*}
M_{n}=\sup_{T_{n-1}<t\leq T_{n}}|Z\left(t\right)-Z\left(T_{n-1}\right)|
\end{eqnarray*}
\end{Def}

\begin{Teo}
Sup\'ongase que $n^{-1}T_{n}\rightarrow\mu$ c.s. cuando $n\rightarrow\infty$, donde $\mu\leq\infty$ es una constante o variable aleatoria. Sea $a$ una constante o variable aleatoria que puede ser infinita cuando $\mu$ es finita, y considere las expresiones l\'imite:
\begin{eqnarray}
lim_{n\rightarrow\infty}n^{-1}Z\left(T_{n}\right)&=&a,\textrm{ c.s.}\\
lim_{t\rightarrow\infty}t^{-1}Z\left(t\right)&=&a/\mu,\textrm{ c.s.}
\end{eqnarray}
La segunda expresi\'on implica la primera. Conversamente, la primera implica la segunda si el proceso $Z\left(t\right)$ es creciente, o si $lim_{n\rightarrow\infty}n^{-1}M_{n}=0$ c.s.
\end{Teo}

\begin{Coro}
Si $N\left(t\right)$ es un proceso de renovaci\'on, y $\left(Z\left(T_{n}\right)-Z\left(T_{n-1}\right),M_{n}\right)$, para $n\geq1$, son variables aleatorias independientes e id\'enticamente distribuidas con media finita, entonces,
\begin{eqnarray}
lim_{t\rightarrow\infty}t^{-1}Z\left(t\right)\rightarrow\frac{\esp\left[Z\left(T_{1}\right)-Z\left(T_{0}\right)\right]}{\esp\left[T_{1}\right]},\textrm{ c.s. cuando  }t\rightarrow\infty.
\end{eqnarray}
\end{Coro}

%___________________________________________________________________________________________
%
\subsection{Funci\'on de Renovaci\'on}
%___________________________________________________________________________________________
%


Sup\'ongase que $N\left(t\right)$ es un proceso de renovaci\'on con distribuci\'on $F$ con media finita $\mu$.

\begin{Def}
La funci\'on de renovaci\'on asociada con la distribuci\'on $F$, del proceso $N\left(t\right)$, es
\begin{eqnarray*}
U\left(t\right)=\sum_{n=1}^{\infty}F^{n\star}\left(t\right),\textrm{   }t\geq0,
\end{eqnarray*}
donde $F^{0\star}\left(t\right)=\indora\left(t\geq0\right)$.
\end{Def}


\begin{Prop}
Sup\'ongase que la distribuci\'on de inter-renovaci\'on $F$ tiene densidad $f$. Entonces $U\left(t\right)$ tambi\'en tiene densidad, para $t>0$, y es $U^{'}\left(t\right)=\sum_{n=0}^{\infty}f^{n\star}\left(t\right)$. Adem\'as
\begin{eqnarray*}
\prob\left\{N\left(t\right)>N\left(t-\right)\right\}=0\textrm{,   }t\geq0.
\end{eqnarray*}
\end{Prop}

\begin{Def}
La Transformada de Laplace-Stieljes de $F$ est\'a dada por

\begin{eqnarray*}
\hat{F}\left(\alpha\right)=\int_{\rea_{+}}e^{-\alpha t}dF\left(t\right)\textrm{,  }\alpha\geq0.
\end{eqnarray*}
\end{Def}

Entonces

\begin{eqnarray*}
\hat{U}\left(\alpha\right)=\sum_{n=0}^{\infty}\hat{F^{n\star}}\left(\alpha\right)=\sum_{n=0}^{\infty}\hat{F}\left(\alpha\right)^{n}=\frac{1}{1-\hat{F}\left(\alpha\right)}.
\end{eqnarray*}


\begin{Prop}
La Transformada de Laplace $\hat{U}\left(\alpha\right)$ y $\hat{F}\left(\alpha\right)$ determina una a la otra de manera \'unica por la relaci\'on $\hat{U}\left(\alpha\right)=\frac{1}{1-\hat{F}\left(\alpha\right)}$.
\end{Prop}


\begin{Note}
Un proceso de renovaci\'on $N\left(t\right)$ cuyos tiempos de inter-renovaci\'on tienen media finita, es un proceso Poisson con tasa $\lambda$ si y s\'olo s\'i $\esp\left[U\left(t\right)\right]=\lambda t$, para $t\geq0$.
\end{Note}


\begin{Teo}
Sea $N\left(t\right)$ un proceso puntual simple con puntos de localizaci\'on $T_{n}$ tal que $\eta\left(t\right)=\esp\left[N\left(\right)\right]$ es finita para cada $t$. Entonces para cualquier funci\'on $f:\rea_{+}\rightarrow\rea$,
\begin{eqnarray*}
\esp\left[\sum_{n=1}^{N\left(\right)}f\left(T_{n}\right)\right]=\int_{\left(0,t\right]}f\left(s\right)d\eta\left(s\right)\textrm{,  }t\geq0,
\end{eqnarray*}
suponiendo que la integral exista. Adem\'as si $X_{1},X_{2},\ldots$ son variables aleatorias definidas en el mismo espacio de probabilidad que el proceso $N\left(t\right)$ tal que $\esp\left[X_{n}|T_{n}=s\right]=f\left(s\right)$, independiente de $n$. Entonces
\begin{eqnarray*}
\esp\left[\sum_{n=1}^{N\left(t\right)}X_{n}\right]=\int_{\left(0,t\right]}f\left(s\right)d\eta\left(s\right)\textrm{,  }t\geq0,
\end{eqnarray*} 
suponiendo que la integral exista. 
\end{Teo}

\begin{Coro}[Identidad de Wald para Renovaciones]
Para el proceso de renovaci\'on $N\left(t\right)$,
\begin{eqnarray*}
\esp\left[T_{N\left(t\right)+1}\right]=\mu\esp\left[N\left(t\right)+1\right]\textrm{,  }t\geq0,
\end{eqnarray*}  
\end{Coro}

%______________________________________________________________________
\subsection{Procesos de Renovaci\'on}
%______________________________________________________________________

\begin{Def}\label{Def.Tn}
Sean $0\leq T_{1}\leq T_{2}\leq \ldots$ son tiempos aleatorios infinitos en los cuales ocurren ciertos eventos. El n\'umero de tiempos $T_{n}$ en el intervalo $\left[0,t\right)$ es

\begin{eqnarray}
N\left(t\right)=\sum_{n=1}^{\infty}\indora\left(T_{n}\leq t\right),
\end{eqnarray}
para $t\geq0$.
\end{Def}

Si se consideran los puntos $T_{n}$ como elementos de $\rea_{+}$, y $N\left(t\right)$ es el n\'umero de puntos en $\rea$. El proceso denotado por $\left\{N\left(t\right):t\geq0\right\}$, denotado por $N\left(t\right)$, es un proceso puntual en $\rea_{+}$. Los $T_{n}$ son los tiempos de ocurrencia, el proceso puntual $N\left(t\right)$ es simple si su n\'umero de ocurrencias son distintas: $0<T_{1}<T_{2}<\ldots$ casi seguramente.

\begin{Def}
Un proceso puntual $N\left(t\right)$ es un proceso de renovaci\'on si los tiempos de interocurrencia $\xi_{n}=T_{n}-T_{n-1}$, para $n\geq1$, son independientes e identicamente distribuidos con distribuci\'on $F$, donde $F\left(0\right)=0$ y $T_{0}=0$. Los $T_{n}$ son llamados tiempos de renovaci\'on, referente a la independencia o renovaci\'on de la informaci\'on estoc\'astica en estos tiempos. Los $\xi_{n}$ son los tiempos de inter-renovaci\'on, y $N\left(t\right)$ es el n\'umero de renovaciones en el intervalo $\left[0,t\right)$
\end{Def}


\begin{Note}
Para definir un proceso de renovaci\'on para cualquier contexto, solamente hay que especificar una distribuci\'on $F$, con $F\left(0\right)=0$, para los tiempos de inter-renovaci\'on. La funci\'on $F$ en turno degune las otra variables aleatorias. De manera formal, existe un espacio de probabilidad y una sucesi\'on de variables aleatorias $\xi_{1},\xi_{2},\ldots$ definidas en este con distribuci\'on $F$. Entonces las otras cantidades son $T_{n}=\sum_{k=1}^{n}\xi_{k}$ y $N\left(t\right)=\sum_{n=1}^{\infty}\indora\left(T_{n}\leq t\right)$, donde $T_{n}\rightarrow\infty$ casi seguramente por la Ley Fuerte de los Grandes Números.
\end{Note}
%_____________________________________________________
\subsection{Puntos de Renovaci\'on}
%_____________________________________________________

Para cada cola $Q_{i}$ se tienen los procesos de arribo a la cola, para estas, los tiempos de arribo est\'an dados por $$\left\{T_{1}^{i},T_{2}^{i},\ldots,T_{k}^{i},\ldots\right\},$$ entonces, consideremos solamente los primeros tiempos de arribo a cada una de las colas, es decir, $$\left\{T_{1}^{1},T_{1}^{2},T_{1}^{3},T_{1}^{4}\right\},$$ se sabe que cada uno de estos tiempos se distribuye de manera exponencial con par\'ametro $1/mu_{i}$. Adem\'as se sabe que para $$T^{*}=\min\left\{T_{1}^{1},T_{1}^{2},T_{1}^{3},T_{1}^{4}\right\},$$ $T^{*}$ se distribuye de manera exponencial con par\'ametro $$\mu^{*}=\sum_{i=1}^{4}\mu_{i}.$$ Ahora, dado que 
\begin{center}
\begin{tabular}{lcl}
$\tilde{r}=r_{1}+r_{2}$ & y &$\hat{r}=r_{3}+r_{4}.$
\end{tabular}
\end{center}


Supongamos que $$\tilde{r},\hat{r}<\mu^{*},$$ entonces si tomamos $$r^{*}=\min\left\{\tilde{r},\hat{r}\right\},$$ se tiene que para  $$t^{*}\in\left(0,r^{*}\right)$$ se cumple que 
\begin{center}
\begin{tabular}{lcl}
$\tau_{1}\left(1\right)=0$ & y por tanto & $\overline{\tau}_{1}=0,$
\end{tabular}
\end{center}
entonces para la segunda cola en este primer ciclo se cumple que $$\tau_{2}=\overline{\tau}_{1}+r_{1}=r_{1}<\mu^{*},$$ y por tanto se tiene que  $$\overline{\tau}_{2}=\tau_{2}.$$ Por lo tanto, nuevamente para la primer cola en el segundo ciclo $$\tau_{1}\left(2\right)=\tau_{2}\left(1\right)+r_{2}=\tilde{r}<\mu^{*}.$$ An\'alogamente para el segundo sistema se tiene que ambas colas est\'an vac\'ias, es decir, existe un valor $t^{*}$ tal que en el intervalo $\left(0,t^{*}\right)$ no ha llegado ning\'un usuario, es decir, $$L_{i}\left(t^{*}\right)=0$$ para $i=1,2,3,4$.

\subsection{Resultados para Procesos de Salida}

En \cite{Sigman2} prueban que para la existencia de un una sucesi\'on infinita no decreciente de tiempos de regeneraci\'on $\tau_{1}\leq\tau_{2}\leq\cdots$ en los cuales el proceso se regenera, basta un tiempo de regeneraci\'on $R_{1}$, donde $R_{j}=\tau_{j}-\tau_{j-1}$. Para tal efecto se requiere la existencia de un espacio de probabilidad $\left(\Omega,\mathcal{F},\prob\right)$, y proceso estoc\'astico $\textit{X}=\left\{X\left(t\right):t\geq0\right\}$ con espacio de estados $\left(S,\mathcal{R}\right)$, con $\mathcal{R}$ $\sigma$-\'algebra.

\begin{Prop}
Si existe una variable aleatoria no negativa $R_{1}$ tal que $\theta_{R\footnotesize{1}}X=_{D}X$, entonces $\left(\Omega,\mathcal{F},\prob\right)$ puede extenderse para soportar una sucesi\'on estacionaria de variables aleatorias $R=\left\{R_{k}:k\geq1\right\}$, tal que para $k\geq1$,
\begin{eqnarray*}
\theta_{k}\left(X,R\right)=_{D}\left(X,R\right).
\end{eqnarray*}

Adem\'as, para $k\geq1$, $\theta_{k}R$ es condicionalmente independiente de $\left(X,R_{1},\ldots,R_{k}\right)$, dado $\theta_{\tau k}X$.

\end{Prop}


\begin{itemize}
\item Doob en 1953 demostr\'o que el estado estacionario de un proceso de partida en un sistema de espera $M/G/\infty$, es Poisson con la misma tasa que el proceso de arribos.

\item Burke en 1968, fue el primero en demostrar que el estado estacionario de un proceso de salida de una cola $M/M/s$ es un proceso Poisson.

\item Disney en 1973 obtuvo el siguiente resultado:

\begin{Teo}
Para el sistema de espera $M/G/1/L$ con disciplina FIFO, el proceso $\textbf{I}$ es un proceso de renovaci\'on si y s\'olo si el proceso denominado longitud de la cola es estacionario y se cumple cualquiera de los siguientes casos:

\begin{itemize}
\item[a)] Los tiempos de servicio son identicamente cero;
\item[b)] $L=0$, para cualquier proceso de servicio $S$;
\item[c)] $L=1$ y $G=D$;
\item[d)] $L=\infty$ y $G=M$.
\end{itemize}
En estos casos, respectivamente, las distribuciones de interpartida $P\left\{T_{n+1}-T_{n}\leq t\right\}$ son


\begin{itemize}
\item[a)] $1-e^{-\lambda t}$, $t\geq0$;
\item[b)] $1-e^{-\lambda t}*F\left(t\right)$, $t\geq0$;
\item[c)] $1-e^{-\lambda t}*\indora_{d}\left(t\right)$, $t\geq0$;
\item[d)] $1-e^{-\lambda t}*F\left(t\right)$, $t\geq0$.
\end{itemize}
\end{Teo}


\item Finch (1959) mostr\'o que para los sistemas $M/G/1/L$, con $1\leq L\leq \infty$ con distribuciones de servicio dos veces diferenciable, solamente el sistema $M/M/1/\infty$ tiene proceso de salida de renovaci\'on estacionario.

\item King (1971) demostro que un sistema de colas estacionario $M/G/1/1$ tiene sus tiempos de interpartida sucesivas $D_{n}$ y $D_{n+1}$ son independientes, si y s\'olo si, $G=D$, en cuyo caso le proceso de salida es de renovaci\'on.

\item Disney (1973) demostr\'o que el \'unico sistema estacionario $M/G/1/L$, que tiene proceso de salida de renovaci\'on  son los sistemas $M/M/1$ y $M/D/1/1$.



\item El siguiente resultado es de Disney y Koning (1985)
\begin{Teo}
En un sistema de espera $M/G/s$, el estado estacionario del proceso de salida es un proceso Poisson para cualquier distribuci\'on de los tiempos de servicio si el sistema tiene cualquiera de las siguientes cuatro propiedades.

\begin{itemize}
\item[a)] $s=\infty$
\item[b)] La disciplina de servicio es de procesador compartido.
\item[c)] La disciplina de servicio es LCFS y preemptive resume, esto se cumple para $L<\infty$
\item[d)] $G=M$.
\end{itemize}

\end{Teo}

\item El siguiente resultado es de Alamatsaz (1983)

\begin{Teo}
En cualquier sistema de colas $GI/G/1/L$ con $1\leq L<\infty$ y distribuci\'on de interarribos $A$ y distribuci\'on de los tiempos de servicio $B$, tal que $A\left(0\right)=0$, $A\left(t\right)\left(1-B\left(t\right)\right)>0$ para alguna $t>0$ y $B\left(t\right)$ para toda $t>0$, es imposible que el proceso de salida estacionario sea de renovaci\'on.
\end{Teo}

\end{itemize}

Estos resultados aparecen en Daley (1968) \cite{Daley68} para $\left\{T_{n}\right\}$ intervalos de inter-arribo, $\left\{D_{n}\right\}$ intervalos de inter-salida y $\left\{S_{n}\right\}$ tiempos de servicio.

\begin{itemize}
\item Si el proceso $\left\{T_{n}\right\}$ es Poisson, el proceso $\left\{D_{n}\right\}$ es no correlacionado si y s\'olo si es un proceso Poisso, lo cual ocurre si y s\'olo si $\left\{S_{n}\right\}$ son exponenciales negativas.

\item Si $\left\{S_{n}\right\}$ son exponenciales negativas, $\left\{D_{n}\right\}$ es un proceso de renovaci\'on  si y s\'olo si es un proceso Poisson, lo cual ocurre si y s\'olo si $\left\{T_{n}\right\}$ es un proceso Poisson.

\item $\esp\left(D_{n}\right)=\esp\left(T_{n}\right)$.

\item Para un sistema de visitas $GI/M/1$ se tiene el siguiente teorema:

\begin{Teo}
En un sistema estacionario $GI/M/1$ los intervalos de interpartida tienen
\begin{eqnarray*}
\esp\left(e^{-\theta D_{n}}\right)&=&\mu\left(\mu+\theta\right)^{-1}\left[\delta\theta
-\mu\left(1-\delta\right)\alpha\left(\theta\right)\right]
\left[\theta-\mu\left(1-\delta\right)^{-1}\right]\\
\alpha\left(\theta\right)&=&\esp\left[e^{-\theta T_{0}}\right]\\
var\left(D_{n}\right)&=&var\left(T_{0}\right)-\left(\tau^{-1}-\delta^{-1}\right)
2\delta\left(\esp\left(S_{0}\right)\right)^{2}\left(1-\delta\right)^{-1}.
\end{eqnarray*}
\end{Teo}



\begin{Teo}
El proceso de salida de un sistema de colas estacionario $GI/M/1$ es un proceso de renovaci\'on si y s\'olo si el proceso de entrada es un proceso Poisson, en cuyo caso el proceso de salida es un proceso Poisson.
\end{Teo}


\begin{Teo}
Los intervalos de interpartida $\left\{D_{n}\right\}$ de un sistema $M/G/1$ estacionario son no correlacionados si y s\'olo si la distribuci\'on de los tiempos de servicio es exponencial negativa, es decir, el sistema es de tipo  $M/M/1$.

\end{Teo}



\end{itemize}





%________________________________________________________________________
\subsection{Procesos Regenerativos Sigman, Thorisson y Wolff \cite{Sigman1}}
%________________________________________________________________________


\begin{Def}[Definici\'on Cl\'asica]
Un proceso estoc\'astico $X=\left\{X\left(t\right):t\geq0\right\}$ es llamado regenerativo is existe una variable aleatoria $R_{1}>0$ tal que
\begin{itemize}
\item[i)] $\left\{X\left(t+R_{1}\right):t\geq0\right\}$ es independiente de $\left\{\left\{X\left(t\right):t<R_{1}\right\},\right\}$
\item[ii)] $\left\{X\left(t+R_{1}\right):t\geq0\right\}$ es estoc\'asticamente equivalente a $\left\{X\left(t\right):t>0\right\}$
\end{itemize}

Llamamos a $R_{1}$ tiempo de regeneraci\'on, y decimos que $X$ se regenera en este punto.
\end{Def}

$\left\{X\left(t+R_{1}\right)\right\}$ es regenerativo con tiempo de regeneraci\'on $R_{2}$, independiente de $R_{1}$ pero con la misma distribuci\'on que $R_{1}$. Procediendo de esta manera se obtiene una secuencia de variables aleatorias independientes e id\'enticamente distribuidas $\left\{R_{n}\right\}$ llamados longitudes de ciclo. Si definimos a $Z_{k}\equiv R_{1}+R_{2}+\cdots+R_{k}$, se tiene un proceso de renovaci\'on llamado proceso de renovaci\'on encajado para $X$.


\begin{Note}
La existencia de un primer tiempo de regeneraci\'on, $R_{1}$, implica la existencia de una sucesi\'on completa de estos tiempos $R_{1},R_{2}\ldots,$ que satisfacen la propiedad deseada \cite{Sigman2}.
\end{Note}


\begin{Note} Para la cola $GI/GI/1$ los usuarios arriban con tiempos $t_{n}$ y son atendidos con tiempos de servicio $S_{n}$, los tiempos de arribo forman un proceso de renovaci\'on  con tiempos entre arribos independientes e identicamente distribuidos (\texttt{i.i.d.})$T_{n}=t_{n}-t_{n-1}$, adem\'as los tiempos de servicio son \texttt{i.i.d.} e independientes de los procesos de arribo. Por \textit{estable} se entiende que $\esp S_{n}<\esp T_{n}<\infty$.
\end{Note}
 


\begin{Def}
Para $x$ fijo y para cada $t\geq0$, sea $I_{x}\left(t\right)=1$ si $X\left(t\right)\leq x$,  $I_{x}\left(t\right)=0$ en caso contrario, y def\'inanse los tiempos promedio
\begin{eqnarray*}
\overline{X}&=&lim_{t\rightarrow\infty}\frac{1}{t}\int_{0}^{\infty}X\left(u\right)du\\
\prob\left(X_{\infty}\leq x\right)&=&lim_{t\rightarrow\infty}\frac{1}{t}\int_{0}^{\infty}I_{x}\left(u\right)du,
\end{eqnarray*}
cuando estos l\'imites existan.
\end{Def}

Como consecuencia del teorema de Renovaci\'on-Recompensa, se tiene que el primer l\'imite  existe y es igual a la constante
\begin{eqnarray*}
\overline{X}&=&\frac{\esp\left[\int_{0}^{R_{1}}X\left(t\right)dt\right]}{\esp\left[R_{1}\right]},
\end{eqnarray*}
suponiendo que ambas esperanzas son finitas.
 
\begin{Note}
Funciones de procesos regenerativos son regenerativas, es decir, si $X\left(t\right)$ es regenerativo y se define el proceso $Y\left(t\right)$ por $Y\left(t\right)=f\left(X\left(t\right)\right)$ para alguna funci\'on Borel medible $f\left(\cdot\right)$. Adem\'as $Y$ es regenerativo con los mismos tiempos de renovaci\'on que $X$. 

En general, los tiempos de renovaci\'on, $Z_{k}$ de un proceso regenerativo no requieren ser tiempos de paro con respecto a la evoluci\'on de $X\left(t\right)$.
\end{Note} 

\begin{Note}
Una funci\'on de un proceso de Markov, usualmente no ser\'a un proceso de Markov, sin embargo ser\'a regenerativo si el proceso de Markov lo es.
\end{Note}

 
\begin{Note}
Un proceso regenerativo con media de la longitud de ciclo finita es llamado positivo recurrente.
\end{Note}


\begin{Note}
\begin{itemize}
\item[a)] Si el proceso regenerativo $X$ es positivo recurrente y tiene trayectorias muestrales no negativas, entonces la ecuaci\'on anterior es v\'alida.
\item[b)] Si $X$ es positivo recurrente regenerativo, podemos construir una \'unica versi\'on estacionaria de este proceso, $X_{e}=\left\{X_{e}\left(t\right)\right\}$, donde $X_{e}$ es un proceso estoc\'astico regenerativo y estrictamente estacionario, con distribuci\'on marginal distribuida como $X_{\infty}$
\end{itemize}
\end{Note}


%__________________________________________________________________________________________
%\subsection{Procesos Regenerativos Estacionarios - Stidham \cite{Stidham}}
%__________________________________________________________________________________________


Un proceso estoc\'astico a tiempo continuo $\left\{V\left(t\right),t\geq0\right\}$ es un proceso regenerativo si existe una sucesi\'on de variables aleatorias independientes e id\'enticamente distribuidas $\left\{X_{1},X_{2},\ldots\right\}$, sucesi\'on de renovaci\'on, tal que para cualquier conjunto de Borel $A$, 

\begin{eqnarray*}
\prob\left\{V\left(t\right)\in A|X_{1}+X_{2}+\cdots+X_{R\left(t\right)}=s,\left\{V\left(\tau\right),\tau<s\right\}\right\}=\prob\left\{V\left(t-s\right)\in A|X_{1}>t-s\right\},
\end{eqnarray*}
para todo $0\leq s\leq t$, donde $R\left(t\right)=\max\left\{X_{1}+X_{2}+\cdots+X_{j}\leq t\right\}=$n\'umero de renovaciones ({\emph{puntos de regeneraci\'on}}) que ocurren en $\left[0,t\right]$. El intervalo $\left[0,X_{1}\right)$ es llamado {\emph{primer ciclo de regeneraci\'on}} de $\left\{V\left(t \right),t\geq0\right\}$, $\left[X_{1},X_{1}+X_{2}\right)$ el {\emph{segundo ciclo de regeneraci\'on}}, y as\'i sucesivamente.

Sea $X=X_{1}$ y sea $F$ la funci\'on de distrbuci\'on de $X$


\begin{Def}
Se define el proceso estacionario, $\left\{V^{*}\left(t\right),t\geq0\right\}$, para $\left\{V\left(t\right),t\geq0\right\}$ por

\begin{eqnarray*}
\prob\left\{V\left(t\right)\in A\right\}=\frac{1}{\esp\left[X\right]}\int_{0}^{\infty}\prob\left\{V\left(t+x\right)\in A|X>x\right\}\left(1-F\left(x\right)\right)dx,
\end{eqnarray*} 
para todo $t\geq0$ y todo conjunto de Borel $A$.
\end{Def}

\begin{Def}
Una distribuci\'on se dice que es {\emph{aritm\'etica}} si todos sus puntos de incremento son m\'ultiplos de la forma $0,\lambda, 2\lambda,\ldots$ para alguna $\lambda>0$ entera.
\end{Def}


\begin{Def}
Una modificaci\'on medible de un proceso $\left\{V\left(t\right),t\geq0\right\}$, es una versi\'on de este, $\left\{V\left(t,w\right)\right\}$ conjuntamente medible para $t\geq0$ y para $w\in S$, $S$ espacio de estados para $\left\{V\left(t\right),t\geq0\right\}$.
\end{Def}

\begin{Teo}
Sea $\left\{V\left(t\right),t\geq\right\}$ un proceso regenerativo no negativo con modificaci\'on medible. Sea $\esp\left[X\right]<\infty$. Entonces el proceso estacionario dado por la ecuaci\'on anterior est\'a bien definido y tiene funci\'on de distribuci\'on independiente de $t$, adem\'as
\begin{itemize}
\item[i)] \begin{eqnarray*}
\esp\left[V^{*}\left(0\right)\right]&=&\frac{\esp\left[\int_{0}^{X}V\left(s\right)ds\right]}{\esp\left[X\right]}\end{eqnarray*}
\item[ii)] Si $\esp\left[V^{*}\left(0\right)\right]<\infty$, equivalentemente, si $\esp\left[\int_{0}^{X}V\left(s\right)ds\right]<\infty$,entonces
\begin{eqnarray*}
\frac{\int_{0}^{t}V\left(s\right)ds}{t}\rightarrow\frac{\esp\left[\int_{0}^{X}V\left(s\right)ds\right]}{\esp\left[X\right]}
\end{eqnarray*}
con probabilidad 1 y en media, cuando $t\rightarrow\infty$.
\end{itemize}
\end{Teo}

\begin{Coro}
Sea $\left\{V\left(t\right),t\geq0\right\}$ un proceso regenerativo no negativo, con modificaci\'on medible. Si $\esp <\infty$, $F$ es no-aritm\'etica, y para todo $x\geq0$, $P\left\{V\left(t\right)\leq x,C>x\right\}$ es de variaci\'on acotada como funci\'on de $t$ en cada intervalo finito $\left[0,\tau\right]$, entonces $V\left(t\right)$ converge en distribuci\'on  cuando $t\rightarrow\infty$ y $$\esp V=\frac{\esp \int_{0}^{X}V\left(s\right)ds}{\esp X}$$
Donde $V$ tiene la distribuci\'on l\'imite de $V\left(t\right)$ cuando $t\rightarrow\infty$.

\end{Coro}

Para el caso discreto se tienen resultados similares.



%______________________________________________________________________
%\subsection{Procesos de Renovaci\'on}
%______________________________________________________________________

\begin{Def}%\label{Def.Tn}
Sean $0\leq T_{1}\leq T_{2}\leq \ldots$ son tiempos aleatorios infinitos en los cuales ocurren ciertos eventos. El n\'umero de tiempos $T_{n}$ en el intervalo $\left[0,t\right)$ es

\begin{eqnarray}
N\left(t\right)=\sum_{n=1}^{\infty}\indora\left(T_{n}\leq t\right),
\end{eqnarray}
para $t\geq0$.
\end{Def}

Si se consideran los puntos $T_{n}$ como elementos de $\rea_{+}$, y $N\left(t\right)$ es el n\'umero de puntos en $\rea$. El proceso denotado por $\left\{N\left(t\right):t\geq0\right\}$, denotado por $N\left(t\right)$, es un proceso puntual en $\rea_{+}$. Los $T_{n}$ son los tiempos de ocurrencia, el proceso puntual $N\left(t\right)$ es simple si su n\'umero de ocurrencias son distintas: $0<T_{1}<T_{2}<\ldots$ casi seguramente.

\begin{Def}
Un proceso puntual $N\left(t\right)$ es un proceso de renovaci\'on si los tiempos de interocurrencia $\xi_{n}=T_{n}-T_{n-1}$, para $n\geq1$, son independientes e identicamente distribuidos con distribuci\'on $F$, donde $F\left(0\right)=0$ y $T_{0}=0$. Los $T_{n}$ son llamados tiempos de renovaci\'on, referente a la independencia o renovaci\'on de la informaci\'on estoc\'astica en estos tiempos. Los $\xi_{n}$ son los tiempos de inter-renovaci\'on, y $N\left(t\right)$ es el n\'umero de renovaciones en el intervalo $\left[0,t\right)$
\end{Def}


\begin{Note}
Para definir un proceso de renovaci\'on para cualquier contexto, solamente hay que especificar una distribuci\'on $F$, con $F\left(0\right)=0$, para los tiempos de inter-renovaci\'on. La funci\'on $F$ en turno degune las otra variables aleatorias. De manera formal, existe un espacio de probabilidad y una sucesi\'on de variables aleatorias $\xi_{1},\xi_{2},\ldots$ definidas en este con distribuci\'on $F$. Entonces las otras cantidades son $T_{n}=\sum_{k=1}^{n}\xi_{k}$ y $N\left(t\right)=\sum_{n=1}^{\infty}\indora\left(T_{n}\leq t\right)$, donde $T_{n}\rightarrow\infty$ casi seguramente por la Ley Fuerte de los Grandes Números.
\end{Note}

%___________________________________________________________________________________________
%
%\subsection{Teorema Principal de Renovaci\'on}
%___________________________________________________________________________________________
%

\begin{Note} Una funci\'on $h:\rea_{+}\rightarrow\rea$ es Directamente Riemann Integrable en los siguientes casos:
\begin{itemize}
\item[a)] $h\left(t\right)\geq0$ es decreciente y Riemann Integrable.
\item[b)] $h$ es continua excepto posiblemente en un conjunto de Lebesgue de medida 0, y $|h\left(t\right)|\leq b\left(t\right)$, donde $b$ es DRI.
\end{itemize}
\end{Note}

\begin{Teo}[Teorema Principal de Renovaci\'on]
Si $F$ es no aritm\'etica y $h\left(t\right)$ es Directamente Riemann Integrable (DRI), entonces

\begin{eqnarray*}
lim_{t\rightarrow\infty}U\star h=\frac{1}{\mu}\int_{\rea_{+}}h\left(s\right)ds.
\end{eqnarray*}
\end{Teo}

\begin{Prop}
Cualquier funci\'on $H\left(t\right)$ acotada en intervalos finitos y que es 0 para $t<0$ puede expresarse como
\begin{eqnarray*}
H\left(t\right)=U\star h\left(t\right)\textrm{,  donde }h\left(t\right)=H\left(t\right)-F\star H\left(t\right)
\end{eqnarray*}
\end{Prop}

\begin{Def}
Un proceso estoc\'astico $X\left(t\right)$ es crudamente regenerativo en un tiempo aleatorio positivo $T$ si
\begin{eqnarray*}
\esp\left[X\left(T+t\right)|T\right]=\esp\left[X\left(t\right)\right]\textrm{, para }t\geq0,\end{eqnarray*}
y con las esperanzas anteriores finitas.
\end{Def}

\begin{Prop}
Sup\'ongase que $X\left(t\right)$ es un proceso crudamente regenerativo en $T$, que tiene distribuci\'on $F$. Si $\esp\left[X\left(t\right)\right]$ es acotado en intervalos finitos, entonces
\begin{eqnarray*}
\esp\left[X\left(t\right)\right]=U\star h\left(t\right)\textrm{,  donde }h\left(t\right)=\esp\left[X\left(t\right)\indora\left(T>t\right)\right].
\end{eqnarray*}
\end{Prop}

\begin{Teo}[Regeneraci\'on Cruda]
Sup\'ongase que $X\left(t\right)$ es un proceso con valores positivo crudamente regenerativo en $T$, y def\'inase $M=\sup\left\{|X\left(t\right)|:t\leq T\right\}$. Si $T$ es no aritm\'etico y $M$ y $MT$ tienen media finita, entonces
\begin{eqnarray*}
lim_{t\rightarrow\infty}\esp\left[X\left(t\right)\right]=\frac{1}{\mu}\int_{\rea_{+}}h\left(s\right)ds,
\end{eqnarray*}
donde $h\left(t\right)=\esp\left[X\left(t\right)\indora\left(T>t\right)\right]$.
\end{Teo}

%___________________________________________________________________________________________
%
%\subsection{Propiedades de los Procesos de Renovaci\'on}
%___________________________________________________________________________________________
%

Los tiempos $T_{n}$ est\'an relacionados con los conteos de $N\left(t\right)$ por

\begin{eqnarray*}
\left\{N\left(t\right)\geq n\right\}&=&\left\{T_{n}\leq t\right\}\\
T_{N\left(t\right)}\leq &t&<T_{N\left(t\right)+1},
\end{eqnarray*}

adem\'as $N\left(T_{n}\right)=n$, y 

\begin{eqnarray*}
N\left(t\right)=\max\left\{n:T_{n}\leq t\right\}=\min\left\{n:T_{n+1}>t\right\}
\end{eqnarray*}

Por propiedades de la convoluci\'on se sabe que

\begin{eqnarray*}
P\left\{T_{n}\leq t\right\}=F^{n\star}\left(t\right)
\end{eqnarray*}
que es la $n$-\'esima convoluci\'on de $F$. Entonces 

\begin{eqnarray*}
\left\{N\left(t\right)\geq n\right\}&=&\left\{T_{n}\leq t\right\}\\
P\left\{N\left(t\right)\leq n\right\}&=&1-F^{\left(n+1\right)\star}\left(t\right)
\end{eqnarray*}

Adem\'as usando el hecho de que $\esp\left[N\left(t\right)\right]=\sum_{n=1}^{\infty}P\left\{N\left(t\right)\geq n\right\}$
se tiene que

\begin{eqnarray*}
\esp\left[N\left(t\right)\right]=\sum_{n=1}^{\infty}F^{n\star}\left(t\right)
\end{eqnarray*}

\begin{Prop}
Para cada $t\geq0$, la funci\'on generadora de momentos $\esp\left[e^{\alpha N\left(t\right)}\right]$ existe para alguna $\alpha$ en una vecindad del 0, y de aqu\'i que $\esp\left[N\left(t\right)^{m}\right]<\infty$, para $m\geq1$.
\end{Prop}


\begin{Note}
Si el primer tiempo de renovaci\'on $\xi_{1}$ no tiene la misma distribuci\'on que el resto de las $\xi_{n}$, para $n\geq2$, a $N\left(t\right)$ se le llama Proceso de Renovaci\'on retardado, donde si $\xi$ tiene distribuci\'on $G$, entonces el tiempo $T_{n}$ de la $n$-\'esima renovaci\'on tiene distribuci\'on $G\star F^{\left(n-1\right)\star}\left(t\right)$
\end{Note}


\begin{Teo}
Para una constante $\mu\leq\infty$ ( o variable aleatoria), las siguientes expresiones son equivalentes:

\begin{eqnarray}
lim_{n\rightarrow\infty}n^{-1}T_{n}&=&\mu,\textrm{ c.s.}\\
lim_{t\rightarrow\infty}t^{-1}N\left(t\right)&=&1/\mu,\textrm{ c.s.}
\end{eqnarray}
\end{Teo}


Es decir, $T_{n}$ satisface la Ley Fuerte de los Grandes N\'umeros s\'i y s\'olo s\'i $N\left/t\right)$ la cumple.


\begin{Coro}[Ley Fuerte de los Grandes N\'umeros para Procesos de Renovaci\'on]
Si $N\left(t\right)$ es un proceso de renovaci\'on cuyos tiempos de inter-renovaci\'on tienen media $\mu\leq\infty$, entonces
\begin{eqnarray}
t^{-1}N\left(t\right)\rightarrow 1/\mu,\textrm{ c.s. cuando }t\rightarrow\infty.
\end{eqnarray}

\end{Coro}


Considerar el proceso estoc\'astico de valores reales $\left\{Z\left(t\right):t\geq0\right\}$ en el mismo espacio de probabilidad que $N\left(t\right)$

\begin{Def}
Para el proceso $\left\{Z\left(t\right):t\geq0\right\}$ se define la fluctuaci\'on m\'axima de $Z\left(t\right)$ en el intervalo $\left(T_{n-1},T_{n}\right]$:
\begin{eqnarray*}
M_{n}=\sup_{T_{n-1}<t\leq T_{n}}|Z\left(t\right)-Z\left(T_{n-1}\right)|
\end{eqnarray*}
\end{Def}

\begin{Teo}
Sup\'ongase que $n^{-1}T_{n}\rightarrow\mu$ c.s. cuando $n\rightarrow\infty$, donde $\mu\leq\infty$ es una constante o variable aleatoria. Sea $a$ una constante o variable aleatoria que puede ser infinita cuando $\mu$ es finita, y considere las expresiones l\'imite:
\begin{eqnarray}
lim_{n\rightarrow\infty}n^{-1}Z\left(T_{n}\right)&=&a,\textrm{ c.s.}\\
lim_{t\rightarrow\infty}t^{-1}Z\left(t\right)&=&a/\mu,\textrm{ c.s.}
\end{eqnarray}
La segunda expresi\'on implica la primera. Conversamente, la primera implica la segunda si el proceso $Z\left(t\right)$ es creciente, o si $lim_{n\rightarrow\infty}n^{-1}M_{n}=0$ c.s.
\end{Teo}

\begin{Coro}
Si $N\left(t\right)$ es un proceso de renovaci\'on, y $\left(Z\left(T_{n}\right)-Z\left(T_{n-1}\right),M_{n}\right)$, para $n\geq1$, son variables aleatorias independientes e id\'enticamente distribuidas con media finita, entonces,
\begin{eqnarray}
lim_{t\rightarrow\infty}t^{-1}Z\left(t\right)\rightarrow\frac{\esp\left[Z\left(T_{1}\right)-Z\left(T_{0}\right)\right]}{\esp\left[T_{1}\right]},\textrm{ c.s. cuando  }t\rightarrow\infty.
\end{eqnarray}
\end{Coro}



%___________________________________________________________________________________________
%
%\subsection{Propiedades de los Procesos de Renovaci\'on}
%___________________________________________________________________________________________
%

Los tiempos $T_{n}$ est\'an relacionados con los conteos de $N\left(t\right)$ por

\begin{eqnarray*}
\left\{N\left(t\right)\geq n\right\}&=&\left\{T_{n}\leq t\right\}\\
T_{N\left(t\right)}\leq &t&<T_{N\left(t\right)+1},
\end{eqnarray*}

adem\'as $N\left(T_{n}\right)=n$, y 

\begin{eqnarray*}
N\left(t\right)=\max\left\{n:T_{n}\leq t\right\}=\min\left\{n:T_{n+1}>t\right\}
\end{eqnarray*}

Por propiedades de la convoluci\'on se sabe que

\begin{eqnarray*}
P\left\{T_{n}\leq t\right\}=F^{n\star}\left(t\right)
\end{eqnarray*}
que es la $n$-\'esima convoluci\'on de $F$. Entonces 

\begin{eqnarray*}
\left\{N\left(t\right)\geq n\right\}&=&\left\{T_{n}\leq t\right\}\\
P\left\{N\left(t\right)\leq n\right\}&=&1-F^{\left(n+1\right)\star}\left(t\right)
\end{eqnarray*}

Adem\'as usando el hecho de que $\esp\left[N\left(t\right)\right]=\sum_{n=1}^{\infty}P\left\{N\left(t\right)\geq n\right\}$
se tiene que

\begin{eqnarray*}
\esp\left[N\left(t\right)\right]=\sum_{n=1}^{\infty}F^{n\star}\left(t\right)
\end{eqnarray*}

\begin{Prop}
Para cada $t\geq0$, la funci\'on generadora de momentos $\esp\left[e^{\alpha N\left(t\right)}\right]$ existe para alguna $\alpha$ en una vecindad del 0, y de aqu\'i que $\esp\left[N\left(t\right)^{m}\right]<\infty$, para $m\geq1$.
\end{Prop}


\begin{Note}
Si el primer tiempo de renovaci\'on $\xi_{1}$ no tiene la misma distribuci\'on que el resto de las $\xi_{n}$, para $n\geq2$, a $N\left(t\right)$ se le llama Proceso de Renovaci\'on retardado, donde si $\xi$ tiene distribuci\'on $G$, entonces el tiempo $T_{n}$ de la $n$-\'esima renovaci\'on tiene distribuci\'on $G\star F^{\left(n-1\right)\star}\left(t\right)$
\end{Note}


\begin{Teo}
Para una constante $\mu\leq\infty$ ( o variable aleatoria), las siguientes expresiones son equivalentes:

\begin{eqnarray}
lim_{n\rightarrow\infty}n^{-1}T_{n}&=&\mu,\textrm{ c.s.}\\
lim_{t\rightarrow\infty}t^{-1}N\left(t\right)&=&1/\mu,\textrm{ c.s.}
\end{eqnarray}
\end{Teo}


Es decir, $T_{n}$ satisface la Ley Fuerte de los Grandes N\'umeros s\'i y s\'olo s\'i $N\left/t\right)$ la cumple.


\begin{Coro}[Ley Fuerte de los Grandes N\'umeros para Procesos de Renovaci\'on]
Si $N\left(t\right)$ es un proceso de renovaci\'on cuyos tiempos de inter-renovaci\'on tienen media $\mu\leq\infty$, entonces
\begin{eqnarray}
t^{-1}N\left(t\right)\rightarrow 1/\mu,\textrm{ c.s. cuando }t\rightarrow\infty.
\end{eqnarray}

\end{Coro}


Considerar el proceso estoc\'astico de valores reales $\left\{Z\left(t\right):t\geq0\right\}$ en el mismo espacio de probabilidad que $N\left(t\right)$

\begin{Def}
Para el proceso $\left\{Z\left(t\right):t\geq0\right\}$ se define la fluctuaci\'on m\'axima de $Z\left(t\right)$ en el intervalo $\left(T_{n-1},T_{n}\right]$:
\begin{eqnarray*}
M_{n}=\sup_{T_{n-1}<t\leq T_{n}}|Z\left(t\right)-Z\left(T_{n-1}\right)|
\end{eqnarray*}
\end{Def}

\begin{Teo}
Sup\'ongase que $n^{-1}T_{n}\rightarrow\mu$ c.s. cuando $n\rightarrow\infty$, donde $\mu\leq\infty$ es una constante o variable aleatoria. Sea $a$ una constante o variable aleatoria que puede ser infinita cuando $\mu$ es finita, y considere las expresiones l\'imite:
\begin{eqnarray}
lim_{n\rightarrow\infty}n^{-1}Z\left(T_{n}\right)&=&a,\textrm{ c.s.}\\
lim_{t\rightarrow\infty}t^{-1}Z\left(t\right)&=&a/\mu,\textrm{ c.s.}
\end{eqnarray}
La segunda expresi\'on implica la primera. Conversamente, la primera implica la segunda si el proceso $Z\left(t\right)$ es creciente, o si $lim_{n\rightarrow\infty}n^{-1}M_{n}=0$ c.s.
\end{Teo}

\begin{Coro}
Si $N\left(t\right)$ es un proceso de renovaci\'on, y $\left(Z\left(T_{n}\right)-Z\left(T_{n-1}\right),M_{n}\right)$, para $n\geq1$, son variables aleatorias independientes e id\'enticamente distribuidas con media finita, entonces,
\begin{eqnarray}
lim_{t\rightarrow\infty}t^{-1}Z\left(t\right)\rightarrow\frac{\esp\left[Z\left(T_{1}\right)-Z\left(T_{0}\right)\right]}{\esp\left[T_{1}\right]},\textrm{ c.s. cuando  }t\rightarrow\infty.
\end{eqnarray}
\end{Coro}


%___________________________________________________________________________________________
%
%\subsection{Propiedades de los Procesos de Renovaci\'on}
%___________________________________________________________________________________________
%

Los tiempos $T_{n}$ est\'an relacionados con los conteos de $N\left(t\right)$ por

\begin{eqnarray*}
\left\{N\left(t\right)\geq n\right\}&=&\left\{T_{n}\leq t\right\}\\
T_{N\left(t\right)}\leq &t&<T_{N\left(t\right)+1},
\end{eqnarray*}

adem\'as $N\left(T_{n}\right)=n$, y 

\begin{eqnarray*}
N\left(t\right)=\max\left\{n:T_{n}\leq t\right\}=\min\left\{n:T_{n+1}>t\right\}
\end{eqnarray*}

Por propiedades de la convoluci\'on se sabe que

\begin{eqnarray*}
P\left\{T_{n}\leq t\right\}=F^{n\star}\left(t\right)
\end{eqnarray*}
que es la $n$-\'esima convoluci\'on de $F$. Entonces 

\begin{eqnarray*}
\left\{N\left(t\right)\geq n\right\}&=&\left\{T_{n}\leq t\right\}\\
P\left\{N\left(t\right)\leq n\right\}&=&1-F^{\left(n+1\right)\star}\left(t\right)
\end{eqnarray*}

Adem\'as usando el hecho de que $\esp\left[N\left(t\right)\right]=\sum_{n=1}^{\infty}P\left\{N\left(t\right)\geq n\right\}$
se tiene que

\begin{eqnarray*}
\esp\left[N\left(t\right)\right]=\sum_{n=1}^{\infty}F^{n\star}\left(t\right)
\end{eqnarray*}

\begin{Prop}
Para cada $t\geq0$, la funci\'on generadora de momentos $\esp\left[e^{\alpha N\left(t\right)}\right]$ existe para alguna $\alpha$ en una vecindad del 0, y de aqu\'i que $\esp\left[N\left(t\right)^{m}\right]<\infty$, para $m\geq1$.
\end{Prop}


\begin{Note}
Si el primer tiempo de renovaci\'on $\xi_{1}$ no tiene la misma distribuci\'on que el resto de las $\xi_{n}$, para $n\geq2$, a $N\left(t\right)$ se le llama Proceso de Renovaci\'on retardado, donde si $\xi$ tiene distribuci\'on $G$, entonces el tiempo $T_{n}$ de la $n$-\'esima renovaci\'on tiene distribuci\'on $G\star F^{\left(n-1\right)\star}\left(t\right)$
\end{Note}


\begin{Teo}
Para una constante $\mu\leq\infty$ ( o variable aleatoria), las siguientes expresiones son equivalentes:

\begin{eqnarray}
lim_{n\rightarrow\infty}n^{-1}T_{n}&=&\mu,\textrm{ c.s.}\\
lim_{t\rightarrow\infty}t^{-1}N\left(t\right)&=&1/\mu,\textrm{ c.s.}
\end{eqnarray}
\end{Teo}


Es decir, $T_{n}$ satisface la Ley Fuerte de los Grandes N\'umeros s\'i y s\'olo s\'i $N\left/t\right)$ la cumple.


\begin{Coro}[Ley Fuerte de los Grandes N\'umeros para Procesos de Renovaci\'on]
Si $N\left(t\right)$ es un proceso de renovaci\'on cuyos tiempos de inter-renovaci\'on tienen media $\mu\leq\infty$, entonces
\begin{eqnarray}
t^{-1}N\left(t\right)\rightarrow 1/\mu,\textrm{ c.s. cuando }t\rightarrow\infty.
\end{eqnarray}

\end{Coro}


Considerar el proceso estoc\'astico de valores reales $\left\{Z\left(t\right):t\geq0\right\}$ en el mismo espacio de probabilidad que $N\left(t\right)$

\begin{Def}
Para el proceso $\left\{Z\left(t\right):t\geq0\right\}$ se define la fluctuaci\'on m\'axima de $Z\left(t\right)$ en el intervalo $\left(T_{n-1},T_{n}\right]$:
\begin{eqnarray*}
M_{n}=\sup_{T_{n-1}<t\leq T_{n}}|Z\left(t\right)-Z\left(T_{n-1}\right)|
\end{eqnarray*}
\end{Def}

\begin{Teo}
Sup\'ongase que $n^{-1}T_{n}\rightarrow\mu$ c.s. cuando $n\rightarrow\infty$, donde $\mu\leq\infty$ es una constante o variable aleatoria. Sea $a$ una constante o variable aleatoria que puede ser infinita cuando $\mu$ es finita, y considere las expresiones l\'imite:
\begin{eqnarray}
lim_{n\rightarrow\infty}n^{-1}Z\left(T_{n}\right)&=&a,\textrm{ c.s.}\\
lim_{t\rightarrow\infty}t^{-1}Z\left(t\right)&=&a/\mu,\textrm{ c.s.}
\end{eqnarray}
La segunda expresi\'on implica la primera. Conversamente, la primera implica la segunda si el proceso $Z\left(t\right)$ es creciente, o si $lim_{n\rightarrow\infty}n^{-1}M_{n}=0$ c.s.
\end{Teo}

\begin{Coro}
Si $N\left(t\right)$ es un proceso de renovaci\'on, y $\left(Z\left(T_{n}\right)-Z\left(T_{n-1}\right),M_{n}\right)$, para $n\geq1$, son variables aleatorias independientes e id\'enticamente distribuidas con media finita, entonces,
\begin{eqnarray}
lim_{t\rightarrow\infty}t^{-1}Z\left(t\right)\rightarrow\frac{\esp\left[Z\left(T_{1}\right)-Z\left(T_{0}\right)\right]}{\esp\left[T_{1}\right]},\textrm{ c.s. cuando  }t\rightarrow\infty.
\end{eqnarray}
\end{Coro}

%___________________________________________________________________________________________
%
%\subsection{Propiedades de los Procesos de Renovaci\'on}
%___________________________________________________________________________________________
%

Los tiempos $T_{n}$ est\'an relacionados con los conteos de $N\left(t\right)$ por

\begin{eqnarray*}
\left\{N\left(t\right)\geq n\right\}&=&\left\{T_{n}\leq t\right\}\\
T_{N\left(t\right)}\leq &t&<T_{N\left(t\right)+1},
\end{eqnarray*}

adem\'as $N\left(T_{n}\right)=n$, y 

\begin{eqnarray*}
N\left(t\right)=\max\left\{n:T_{n}\leq t\right\}=\min\left\{n:T_{n+1}>t\right\}
\end{eqnarray*}

Por propiedades de la convoluci\'on se sabe que

\begin{eqnarray*}
P\left\{T_{n}\leq t\right\}=F^{n\star}\left(t\right)
\end{eqnarray*}
que es la $n$-\'esima convoluci\'on de $F$. Entonces 

\begin{eqnarray*}
\left\{N\left(t\right)\geq n\right\}&=&\left\{T_{n}\leq t\right\}\\
P\left\{N\left(t\right)\leq n\right\}&=&1-F^{\left(n+1\right)\star}\left(t\right)
\end{eqnarray*}

Adem\'as usando el hecho de que $\esp\left[N\left(t\right)\right]=\sum_{n=1}^{\infty}P\left\{N\left(t\right)\geq n\right\}$
se tiene que

\begin{eqnarray*}
\esp\left[N\left(t\right)\right]=\sum_{n=1}^{\infty}F^{n\star}\left(t\right)
\end{eqnarray*}

\begin{Prop}
Para cada $t\geq0$, la funci\'on generadora de momentos $\esp\left[e^{\alpha N\left(t\right)}\right]$ existe para alguna $\alpha$ en una vecindad del 0, y de aqu\'i que $\esp\left[N\left(t\right)^{m}\right]<\infty$, para $m\geq1$.
\end{Prop}


\begin{Note}
Si el primer tiempo de renovaci\'on $\xi_{1}$ no tiene la misma distribuci\'on que el resto de las $\xi_{n}$, para $n\geq2$, a $N\left(t\right)$ se le llama Proceso de Renovaci\'on retardado, donde si $\xi$ tiene distribuci\'on $G$, entonces el tiempo $T_{n}$ de la $n$-\'esima renovaci\'on tiene distribuci\'on $G\star F^{\left(n-1\right)\star}\left(t\right)$
\end{Note}


\begin{Teo}
Para una constante $\mu\leq\infty$ ( o variable aleatoria), las siguientes expresiones son equivalentes:

\begin{eqnarray}
lim_{n\rightarrow\infty}n^{-1}T_{n}&=&\mu,\textrm{ c.s.}\\
lim_{t\rightarrow\infty}t^{-1}N\left(t\right)&=&1/\mu,\textrm{ c.s.}
\end{eqnarray}
\end{Teo}


Es decir, $T_{n}$ satisface la Ley Fuerte de los Grandes N\'umeros s\'i y s\'olo s\'i $N\left/t\right)$ la cumple.


\begin{Coro}[Ley Fuerte de los Grandes N\'umeros para Procesos de Renovaci\'on]
Si $N\left(t\right)$ es un proceso de renovaci\'on cuyos tiempos de inter-renovaci\'on tienen media $\mu\leq\infty$, entonces
\begin{eqnarray}
t^{-1}N\left(t\right)\rightarrow 1/\mu,\textrm{ c.s. cuando }t\rightarrow\infty.
\end{eqnarray}

\end{Coro}


Considerar el proceso estoc\'astico de valores reales $\left\{Z\left(t\right):t\geq0\right\}$ en el mismo espacio de probabilidad que $N\left(t\right)$

\begin{Def}
Para el proceso $\left\{Z\left(t\right):t\geq0\right\}$ se define la fluctuaci\'on m\'axima de $Z\left(t\right)$ en el intervalo $\left(T_{n-1},T_{n}\right]$:
\begin{eqnarray*}
M_{n}=\sup_{T_{n-1}<t\leq T_{n}}|Z\left(t\right)-Z\left(T_{n-1}\right)|
\end{eqnarray*}
\end{Def}

\begin{Teo}
Sup\'ongase que $n^{-1}T_{n}\rightarrow\mu$ c.s. cuando $n\rightarrow\infty$, donde $\mu\leq\infty$ es una constante o variable aleatoria. Sea $a$ una constante o variable aleatoria que puede ser infinita cuando $\mu$ es finita, y considere las expresiones l\'imite:
\begin{eqnarray}
lim_{n\rightarrow\infty}n^{-1}Z\left(T_{n}\right)&=&a,\textrm{ c.s.}\\
lim_{t\rightarrow\infty}t^{-1}Z\left(t\right)&=&a/\mu,\textrm{ c.s.}
\end{eqnarray}
La segunda expresi\'on implica la primera. Conversamente, la primera implica la segunda si el proceso $Z\left(t\right)$ es creciente, o si $lim_{n\rightarrow\infty}n^{-1}M_{n}=0$ c.s.
\end{Teo}

\begin{Coro}
Si $N\left(t\right)$ es un proceso de renovaci\'on, y $\left(Z\left(T_{n}\right)-Z\left(T_{n-1}\right),M_{n}\right)$, para $n\geq1$, son variables aleatorias independientes e id\'enticamente distribuidas con media finita, entonces,
\begin{eqnarray}
lim_{t\rightarrow\infty}t^{-1}Z\left(t\right)\rightarrow\frac{\esp\left[Z\left(T_{1}\right)-Z\left(T_{0}\right)\right]}{\esp\left[T_{1}\right]},\textrm{ c.s. cuando  }t\rightarrow\infty.
\end{eqnarray}
\end{Coro}
%___________________________________________________________________________________________
%
%\subsection{Propiedades de los Procesos de Renovaci\'on}
%___________________________________________________________________________________________
%

Los tiempos $T_{n}$ est\'an relacionados con los conteos de $N\left(t\right)$ por

\begin{eqnarray*}
\left\{N\left(t\right)\geq n\right\}&=&\left\{T_{n}\leq t\right\}\\
T_{N\left(t\right)}\leq &t&<T_{N\left(t\right)+1},
\end{eqnarray*}

adem\'as $N\left(T_{n}\right)=n$, y 

\begin{eqnarray*}
N\left(t\right)=\max\left\{n:T_{n}\leq t\right\}=\min\left\{n:T_{n+1}>t\right\}
\end{eqnarray*}

Por propiedades de la convoluci\'on se sabe que

\begin{eqnarray*}
P\left\{T_{n}\leq t\right\}=F^{n\star}\left(t\right)
\end{eqnarray*}
que es la $n$-\'esima convoluci\'on de $F$. Entonces 

\begin{eqnarray*}
\left\{N\left(t\right)\geq n\right\}&=&\left\{T_{n}\leq t\right\}\\
P\left\{N\left(t\right)\leq n\right\}&=&1-F^{\left(n+1\right)\star}\left(t\right)
\end{eqnarray*}

Adem\'as usando el hecho de que $\esp\left[N\left(t\right)\right]=\sum_{n=1}^{\infty}P\left\{N\left(t\right)\geq n\right\}$
se tiene que

\begin{eqnarray*}
\esp\left[N\left(t\right)\right]=\sum_{n=1}^{\infty}F^{n\star}\left(t\right)
\end{eqnarray*}

\begin{Prop}
Para cada $t\geq0$, la funci\'on generadora de momentos $\esp\left[e^{\alpha N\left(t\right)}\right]$ existe para alguna $\alpha$ en una vecindad del 0, y de aqu\'i que $\esp\left[N\left(t\right)^{m}\right]<\infty$, para $m\geq1$.
\end{Prop}


\begin{Note}
Si el primer tiempo de renovaci\'on $\xi_{1}$ no tiene la misma distribuci\'on que el resto de las $\xi_{n}$, para $n\geq2$, a $N\left(t\right)$ se le llama Proceso de Renovaci\'on retardado, donde si $\xi$ tiene distribuci\'on $G$, entonces el tiempo $T_{n}$ de la $n$-\'esima renovaci\'on tiene distribuci\'on $G\star F^{\left(n-1\right)\star}\left(t\right)$
\end{Note}


\begin{Teo}
Para una constante $\mu\leq\infty$ ( o variable aleatoria), las siguientes expresiones son equivalentes:

\begin{eqnarray}
lim_{n\rightarrow\infty}n^{-1}T_{n}&=&\mu,\textrm{ c.s.}\\
lim_{t\rightarrow\infty}t^{-1}N\left(t\right)&=&1/\mu,\textrm{ c.s.}
\end{eqnarray}
\end{Teo}


Es decir, $T_{n}$ satisface la Ley Fuerte de los Grandes N\'umeros s\'i y s\'olo s\'i $N\left/t\right)$ la cumple.


\begin{Coro}[Ley Fuerte de los Grandes N\'umeros para Procesos de Renovaci\'on]
Si $N\left(t\right)$ es un proceso de renovaci\'on cuyos tiempos de inter-renovaci\'on tienen media $\mu\leq\infty$, entonces
\begin{eqnarray}
t^{-1}N\left(t\right)\rightarrow 1/\mu,\textrm{ c.s. cuando }t\rightarrow\infty.
\end{eqnarray}

\end{Coro}


Considerar el proceso estoc\'astico de valores reales $\left\{Z\left(t\right):t\geq0\right\}$ en el mismo espacio de probabilidad que $N\left(t\right)$

\begin{Def}
Para el proceso $\left\{Z\left(t\right):t\geq0\right\}$ se define la fluctuaci\'on m\'axima de $Z\left(t\right)$ en el intervalo $\left(T_{n-1},T_{n}\right]$:
\begin{eqnarray*}
M_{n}=\sup_{T_{n-1}<t\leq T_{n}}|Z\left(t\right)-Z\left(T_{n-1}\right)|
\end{eqnarray*}
\end{Def}

\begin{Teo}
Sup\'ongase que $n^{-1}T_{n}\rightarrow\mu$ c.s. cuando $n\rightarrow\infty$, donde $\mu\leq\infty$ es una constante o variable aleatoria. Sea $a$ una constante o variable aleatoria que puede ser infinita cuando $\mu$ es finita, y considere las expresiones l\'imite:
\begin{eqnarray}
lim_{n\rightarrow\infty}n^{-1}Z\left(T_{n}\right)&=&a,\textrm{ c.s.}\\
lim_{t\rightarrow\infty}t^{-1}Z\left(t\right)&=&a/\mu,\textrm{ c.s.}
\end{eqnarray}
La segunda expresi\'on implica la primera. Conversamente, la primera implica la segunda si el proceso $Z\left(t\right)$ es creciente, o si $lim_{n\rightarrow\infty}n^{-1}M_{n}=0$ c.s.
\end{Teo}

\begin{Coro}
Si $N\left(t\right)$ es un proceso de renovaci\'on, y $\left(Z\left(T_{n}\right)-Z\left(T_{n-1}\right),M_{n}\right)$, para $n\geq1$, son variables aleatorias independientes e id\'enticamente distribuidas con media finita, entonces,
\begin{eqnarray}
lim_{t\rightarrow\infty}t^{-1}Z\left(t\right)\rightarrow\frac{\esp\left[Z\left(T_{1}\right)-Z\left(T_{0}\right)\right]}{\esp\left[T_{1}\right]},\textrm{ c.s. cuando  }t\rightarrow\infty.
\end{eqnarray}
\end{Coro}


%___________________________________________________________________________________________
%
%\subsection{Funci\'on de Renovaci\'on}
%___________________________________________________________________________________________
%


\begin{Def}
Sea $h\left(t\right)$ funci\'on de valores reales en $\rea$ acotada en intervalos finitos e igual a cero para $t<0$ La ecuaci\'on de renovaci\'on para $h\left(t\right)$ y la distribuci\'on $F$ es

\begin{eqnarray}%\label{Ec.Renovacion}
H\left(t\right)=h\left(t\right)+\int_{\left[0,t\right]}H\left(t-s\right)dF\left(s\right)\textrm{,    }t\geq0,
\end{eqnarray}
donde $H\left(t\right)$ es una funci\'on de valores reales. Esto es $H=h+F\star H$. Decimos que $H\left(t\right)$ es soluci\'on de esta ecuaci\'on si satisface la ecuaci\'on, y es acotada en intervalos finitos e iguales a cero para $t<0$.
\end{Def}

\begin{Prop}
La funci\'on $U\star h\left(t\right)$ es la \'unica soluci\'on de la ecuaci\'on de renovaci\'on (\ref{Ec.Renovacion}).
\end{Prop}

\begin{Teo}[Teorema Renovaci\'on Elemental]
\begin{eqnarray*}
t^{-1}U\left(t\right)\rightarrow 1/\mu\textrm{,    cuando }t\rightarrow\infty.
\end{eqnarray*}
\end{Teo}

%___________________________________________________________________________________________
%
%\subsection{Funci\'on de Renovaci\'on}
%___________________________________________________________________________________________
%


Sup\'ongase que $N\left(t\right)$ es un proceso de renovaci\'on con distribuci\'on $F$ con media finita $\mu$.

\begin{Def}
La funci\'on de renovaci\'on asociada con la distribuci\'on $F$, del proceso $N\left(t\right)$, es
\begin{eqnarray*}
U\left(t\right)=\sum_{n=1}^{\infty}F^{n\star}\left(t\right),\textrm{   }t\geq0,
\end{eqnarray*}
donde $F^{0\star}\left(t\right)=\indora\left(t\geq0\right)$.
\end{Def}


\begin{Prop}
Sup\'ongase que la distribuci\'on de inter-renovaci\'on $F$ tiene densidad $f$. Entonces $U\left(t\right)$ tambi\'en tiene densidad, para $t>0$, y es $U^{'}\left(t\right)=\sum_{n=0}^{\infty}f^{n\star}\left(t\right)$. Adem\'as
\begin{eqnarray*}
\prob\left\{N\left(t\right)>N\left(t-\right)\right\}=0\textrm{,   }t\geq0.
\end{eqnarray*}
\end{Prop}

\begin{Def}
La Transformada de Laplace-Stieljes de $F$ est\'a dada por

\begin{eqnarray*}
\hat{F}\left(\alpha\right)=\int_{\rea_{+}}e^{-\alpha t}dF\left(t\right)\textrm{,  }\alpha\geq0.
\end{eqnarray*}
\end{Def}

Entonces

\begin{eqnarray*}
\hat{U}\left(\alpha\right)=\sum_{n=0}^{\infty}\hat{F^{n\star}}\left(\alpha\right)=\sum_{n=0}^{\infty}\hat{F}\left(\alpha\right)^{n}=\frac{1}{1-\hat{F}\left(\alpha\right)}.
\end{eqnarray*}


\begin{Prop}
La Transformada de Laplace $\hat{U}\left(\alpha\right)$ y $\hat{F}\left(\alpha\right)$ determina una a la otra de manera \'unica por la relaci\'on $\hat{U}\left(\alpha\right)=\frac{1}{1-\hat{F}\left(\alpha\right)}$.
\end{Prop}


\begin{Note}
Un proceso de renovaci\'on $N\left(t\right)$ cuyos tiempos de inter-renovaci\'on tienen media finita, es un proceso Poisson con tasa $\lambda$ si y s\'olo s\'i $\esp\left[U\left(t\right)\right]=\lambda t$, para $t\geq0$.
\end{Note}


\begin{Teo}
Sea $N\left(t\right)$ un proceso puntual simple con puntos de localizaci\'on $T_{n}$ tal que $\eta\left(t\right)=\esp\left[N\left(\right)\right]$ es finita para cada $t$. Entonces para cualquier funci\'on $f:\rea_{+}\rightarrow\rea$,
\begin{eqnarray*}
\esp\left[\sum_{n=1}^{N\left(\right)}f\left(T_{n}\right)\right]=\int_{\left(0,t\right]}f\left(s\right)d\eta\left(s\right)\textrm{,  }t\geq0,
\end{eqnarray*}
suponiendo que la integral exista. Adem\'as si $X_{1},X_{2},\ldots$ son variables aleatorias definidas en el mismo espacio de probabilidad que el proceso $N\left(t\right)$ tal que $\esp\left[X_{n}|T_{n}=s\right]=f\left(s\right)$, independiente de $n$. Entonces
\begin{eqnarray*}
\esp\left[\sum_{n=1}^{N\left(t\right)}X_{n}\right]=\int_{\left(0,t\right]}f\left(s\right)d\eta\left(s\right)\textrm{,  }t\geq0,
\end{eqnarray*} 
suponiendo que la integral exista. 
\end{Teo}

\begin{Coro}[Identidad de Wald para Renovaciones]
Para el proceso de renovaci\'on $N\left(t\right)$,
\begin{eqnarray*}
\esp\left[T_{N\left(t\right)+1}\right]=\mu\esp\left[N\left(t\right)+1\right]\textrm{,  }t\geq0,
\end{eqnarray*}  
\end{Coro}

%______________________________________________________________________
%\subsection{Procesos de Renovaci\'on}
%______________________________________________________________________

\begin{Def}%\label{Def.Tn}
Sean $0\leq T_{1}\leq T_{2}\leq \ldots$ son tiempos aleatorios infinitos en los cuales ocurren ciertos eventos. El n\'umero de tiempos $T_{n}$ en el intervalo $\left[0,t\right)$ es

\begin{eqnarray}
N\left(t\right)=\sum_{n=1}^{\infty}\indora\left(T_{n}\leq t\right),
\end{eqnarray}
para $t\geq0$.
\end{Def}

Si se consideran los puntos $T_{n}$ como elementos de $\rea_{+}$, y $N\left(t\right)$ es el n\'umero de puntos en $\rea$. El proceso denotado por $\left\{N\left(t\right):t\geq0\right\}$, denotado por $N\left(t\right)$, es un proceso puntual en $\rea_{+}$. Los $T_{n}$ son los tiempos de ocurrencia, el proceso puntual $N\left(t\right)$ es simple si su n\'umero de ocurrencias son distintas: $0<T_{1}<T_{2}<\ldots$ casi seguramente.

\begin{Def}
Un proceso puntual $N\left(t\right)$ es un proceso de renovaci\'on si los tiempos de interocurrencia $\xi_{n}=T_{n}-T_{n-1}$, para $n\geq1$, son independientes e identicamente distribuidos con distribuci\'on $F$, donde $F\left(0\right)=0$ y $T_{0}=0$. Los $T_{n}$ son llamados tiempos de renovaci\'on, referente a la independencia o renovaci\'on de la informaci\'on estoc\'astica en estos tiempos. Los $\xi_{n}$ son los tiempos de inter-renovaci\'on, y $N\left(t\right)$ es el n\'umero de renovaciones en el intervalo $\left[0,t\right)$
\end{Def}


\begin{Note}
Para definir un proceso de renovaci\'on para cualquier contexto, solamente hay que especificar una distribuci\'on $F$, con $F\left(0\right)=0$, para los tiempos de inter-renovaci\'on. La funci\'on $F$ en turno degune las otra variables aleatorias. De manera formal, existe un espacio de probabilidad y una sucesi\'on de variables aleatorias $\xi_{1},\xi_{2},\ldots$ definidas en este con distribuci\'on $F$. Entonces las otras cantidades son $T_{n}=\sum_{k=1}^{n}\xi_{k}$ y $N\left(t\right)=\sum_{n=1}^{\infty}\indora\left(T_{n}\leq t\right)$, donde $T_{n}\rightarrow\infty$ casi seguramente por la Ley Fuerte de los Grandes Números.
\end{Note}

%___________________________________________________________________________________________
%
%\subsection{Renewal and Regenerative Processes: Serfozo\cite{Serfozo}}
%___________________________________________________________________________________________
%
\begin{Def}%\label{Def.Tn}
Sean $0\leq T_{1}\leq T_{2}\leq \ldots$ son tiempos aleatorios infinitos en los cuales ocurren ciertos eventos. El n\'umero de tiempos $T_{n}$ en el intervalo $\left[0,t\right)$ es

\begin{eqnarray}
N\left(t\right)=\sum_{n=1}^{\infty}\indora\left(T_{n}\leq t\right),
\end{eqnarray}
para $t\geq0$.
\end{Def}

Si se consideran los puntos $T_{n}$ como elementos de $\rea_{+}$, y $N\left(t\right)$ es el n\'umero de puntos en $\rea$. El proceso denotado por $\left\{N\left(t\right):t\geq0\right\}$, denotado por $N\left(t\right)$, es un proceso puntual en $\rea_{+}$. Los $T_{n}$ son los tiempos de ocurrencia, el proceso puntual $N\left(t\right)$ es simple si su n\'umero de ocurrencias son distintas: $0<T_{1}<T_{2}<\ldots$ casi seguramente.

\begin{Def}
Un proceso puntual $N\left(t\right)$ es un proceso de renovaci\'on si los tiempos de interocurrencia $\xi_{n}=T_{n}-T_{n-1}$, para $n\geq1$, son independientes e identicamente distribuidos con distribuci\'on $F$, donde $F\left(0\right)=0$ y $T_{0}=0$. Los $T_{n}$ son llamados tiempos de renovaci\'on, referente a la independencia o renovaci\'on de la informaci\'on estoc\'astica en estos tiempos. Los $\xi_{n}$ son los tiempos de inter-renovaci\'on, y $N\left(t\right)$ es el n\'umero de renovaciones en el intervalo $\left[0,t\right)$
\end{Def}


\begin{Note}
Para definir un proceso de renovaci\'on para cualquier contexto, solamente hay que especificar una distribuci\'on $F$, con $F\left(0\right)=0$, para los tiempos de inter-renovaci\'on. La funci\'on $F$ en turno degune las otra variables aleatorias. De manera formal, existe un espacio de probabilidad y una sucesi\'on de variables aleatorias $\xi_{1},\xi_{2},\ldots$ definidas en este con distribuci\'on $F$. Entonces las otras cantidades son $T_{n}=\sum_{k=1}^{n}\xi_{k}$ y $N\left(t\right)=\sum_{n=1}^{\infty}\indora\left(T_{n}\leq t\right)$, donde $T_{n}\rightarrow\infty$ casi seguramente por la Ley Fuerte de los Grandes N\'umeros.
\end{Note}







Los tiempos $T_{n}$ est\'an relacionados con los conteos de $N\left(t\right)$ por

\begin{eqnarray*}
\left\{N\left(t\right)\geq n\right\}&=&\left\{T_{n}\leq t\right\}\\
T_{N\left(t\right)}\leq &t&<T_{N\left(t\right)+1},
\end{eqnarray*}

adem\'as $N\left(T_{n}\right)=n$, y 

\begin{eqnarray*}
N\left(t\right)=\max\left\{n:T_{n}\leq t\right\}=\min\left\{n:T_{n+1}>t\right\}
\end{eqnarray*}

Por propiedades de la convoluci\'on se sabe que

\begin{eqnarray*}
P\left\{T_{n}\leq t\right\}=F^{n\star}\left(t\right)
\end{eqnarray*}
que es la $n$-\'esima convoluci\'on de $F$. Entonces 

\begin{eqnarray*}
\left\{N\left(t\right)\geq n\right\}&=&\left\{T_{n}\leq t\right\}\\
P\left\{N\left(t\right)\leq n\right\}&=&1-F^{\left(n+1\right)\star}\left(t\right)
\end{eqnarray*}

Adem\'as usando el hecho de que $\esp\left[N\left(t\right)\right]=\sum_{n=1}^{\infty}P\left\{N\left(t\right)\geq n\right\}$
se tiene que

\begin{eqnarray*}
\esp\left[N\left(t\right)\right]=\sum_{n=1}^{\infty}F^{n\star}\left(t\right)
\end{eqnarray*}

\begin{Prop}
Para cada $t\geq0$, la funci\'on generadora de momentos $\esp\left[e^{\alpha N\left(t\right)}\right]$ existe para alguna $\alpha$ en una vecindad del 0, y de aqu\'i que $\esp\left[N\left(t\right)^{m}\right]<\infty$, para $m\geq1$.
\end{Prop}

\begin{Ejem}[\textbf{Proceso Poisson}]

Suponga que se tienen tiempos de inter-renovaci\'on \textit{i.i.d.} del proceso de renovaci\'on $N\left(t\right)$ tienen distribuci\'on exponencial $F\left(t\right)=q-e^{-\lambda t}$ con tasa $\lambda$. Entonces $N\left(t\right)$ es un proceso Poisson con tasa $\lambda$.

\end{Ejem}


\begin{Note}
Si el primer tiempo de renovaci\'on $\xi_{1}$ no tiene la misma distribuci\'on que el resto de las $\xi_{n}$, para $n\geq2$, a $N\left(t\right)$ se le llama Proceso de Renovaci\'on retardado, donde si $\xi$ tiene distribuci\'on $G$, entonces el tiempo $T_{n}$ de la $n$-\'esima renovaci\'on tiene distribuci\'on $G\star F^{\left(n-1\right)\star}\left(t\right)$
\end{Note}


\begin{Teo}
Para una constante $\mu\leq\infty$ ( o variable aleatoria), las siguientes expresiones son equivalentes:

\begin{eqnarray}
lim_{n\rightarrow\infty}n^{-1}T_{n}&=&\mu,\textrm{ c.s.}\\
lim_{t\rightarrow\infty}t^{-1}N\left(t\right)&=&1/\mu,\textrm{ c.s.}
\end{eqnarray}
\end{Teo}


Es decir, $T_{n}$ satisface la Ley Fuerte de los Grandes N\'umeros s\'i y s\'olo s\'i $N\left/t\right)$ la cumple.


\begin{Coro}[Ley Fuerte de los Grandes N\'umeros para Procesos de Renovaci\'on]
Si $N\left(t\right)$ es un proceso de renovaci\'on cuyos tiempos de inter-renovaci\'on tienen media $\mu\leq\infty$, entonces
\begin{eqnarray}
t^{-1}N\left(t\right)\rightarrow 1/\mu,\textrm{ c.s. cuando }t\rightarrow\infty.
\end{eqnarray}

\end{Coro}


Considerar el proceso estoc\'astico de valores reales $\left\{Z\left(t\right):t\geq0\right\}$ en el mismo espacio de probabilidad que $N\left(t\right)$

\begin{Def}
Para el proceso $\left\{Z\left(t\right):t\geq0\right\}$ se define la fluctuaci\'on m\'axima de $Z\left(t\right)$ en el intervalo $\left(T_{n-1},T_{n}\right]$:
\begin{eqnarray*}
M_{n}=\sup_{T_{n-1}<t\leq T_{n}}|Z\left(t\right)-Z\left(T_{n-1}\right)|
\end{eqnarray*}
\end{Def}

\begin{Teo}
Sup\'ongase que $n^{-1}T_{n}\rightarrow\mu$ c.s. cuando $n\rightarrow\infty$, donde $\mu\leq\infty$ es una constante o variable aleatoria. Sea $a$ una constante o variable aleatoria que puede ser infinita cuando $\mu$ es finita, y considere las expresiones l\'imite:
\begin{eqnarray}
lim_{n\rightarrow\infty}n^{-1}Z\left(T_{n}\right)&=&a,\textrm{ c.s.}\\
lim_{t\rightarrow\infty}t^{-1}Z\left(t\right)&=&a/\mu,\textrm{ c.s.}
\end{eqnarray}
La segunda expresi\'on implica la primera. Conversamente, la primera implica la segunda si el proceso $Z\left(t\right)$ es creciente, o si $lim_{n\rightarrow\infty}n^{-1}M_{n}=0$ c.s.
\end{Teo}

\begin{Coro}
Si $N\left(t\right)$ es un proceso de renovaci\'on, y $\left(Z\left(T_{n}\right)-Z\left(T_{n-1}\right),M_{n}\right)$, para $n\geq1$, son variables aleatorias independientes e id\'enticamente distribuidas con media finita, entonces,
\begin{eqnarray}
lim_{t\rightarrow\infty}t^{-1}Z\left(t\right)\rightarrow\frac{\esp\left[Z\left(T_{1}\right)-Z\left(T_{0}\right)\right]}{\esp\left[T_{1}\right]},\textrm{ c.s. cuando  }t\rightarrow\infty.
\end{eqnarray}
\end{Coro}


Sup\'ongase que $N\left(t\right)$ es un proceso de renovaci\'on con distribuci\'on $F$ con media finita $\mu$.

\begin{Def}
La funci\'on de renovaci\'on asociada con la distribuci\'on $F$, del proceso $N\left(t\right)$, es
\begin{eqnarray*}
U\left(t\right)=\sum_{n=1}^{\infty}F^{n\star}\left(t\right),\textrm{   }t\geq0,
\end{eqnarray*}
donde $F^{0\star}\left(t\right)=\indora\left(t\geq0\right)$.
\end{Def}


\begin{Prop}
Sup\'ongase que la distribuci\'on de inter-renovaci\'on $F$ tiene densidad $f$. Entonces $U\left(t\right)$ tambi\'en tiene densidad, para $t>0$, y es $U^{'}\left(t\right)=\sum_{n=0}^{\infty}f^{n\star}\left(t\right)$. Adem\'as
\begin{eqnarray*}
\prob\left\{N\left(t\right)>N\left(t-\right)\right\}=0\textrm{,   }t\geq0.
\end{eqnarray*}
\end{Prop}

\begin{Def}
La Transformada de Laplace-Stieljes de $F$ est\'a dada por

\begin{eqnarray*}
\hat{F}\left(\alpha\right)=\int_{\rea_{+}}e^{-\alpha t}dF\left(t\right)\textrm{,  }\alpha\geq0.
\end{eqnarray*}
\end{Def}

Entonces

\begin{eqnarray*}
\hat{U}\left(\alpha\right)=\sum_{n=0}^{\infty}\hat{F^{n\star}}\left(\alpha\right)=\sum_{n=0}^{\infty}\hat{F}\left(\alpha\right)^{n}=\frac{1}{1-\hat{F}\left(\alpha\right)}.
\end{eqnarray*}


\begin{Prop}
La Transformada de Laplace $\hat{U}\left(\alpha\right)$ y $\hat{F}\left(\alpha\right)$ determina una a la otra de manera \'unica por la relaci\'on $\hat{U}\left(\alpha\right)=\frac{1}{1-\hat{F}\left(\alpha\right)}$.
\end{Prop}


\begin{Note}
Un proceso de renovaci\'on $N\left(t\right)$ cuyos tiempos de inter-renovaci\'on tienen media finita, es un proceso Poisson con tasa $\lambda$ si y s\'olo s\'i $\esp\left[U\left(t\right)\right]=\lambda t$, para $t\geq0$.
\end{Note}


\begin{Teo}
Sea $N\left(t\right)$ un proceso puntual simple con puntos de localizaci\'on $T_{n}$ tal que $\eta\left(t\right)=\esp\left[N\left(\right)\right]$ es finita para cada $t$. Entonces para cualquier funci\'on $f:\rea_{+}\rightarrow\rea$,
\begin{eqnarray*}
\esp\left[\sum_{n=1}^{N\left(\right)}f\left(T_{n}\right)\right]=\int_{\left(0,t\right]}f\left(s\right)d\eta\left(s\right)\textrm{,  }t\geq0,
\end{eqnarray*}
suponiendo que la integral exista. Adem\'as si $X_{1},X_{2},\ldots$ son variables aleatorias definidas en el mismo espacio de probabilidad que el proceso $N\left(t\right)$ tal que $\esp\left[X_{n}|T_{n}=s\right]=f\left(s\right)$, independiente de $n$. Entonces
\begin{eqnarray*}
\esp\left[\sum_{n=1}^{N\left(t\right)}X_{n}\right]=\int_{\left(0,t\right]}f\left(s\right)d\eta\left(s\right)\textrm{,  }t\geq0,
\end{eqnarray*} 
suponiendo que la integral exista. 
\end{Teo}

\begin{Coro}[Identidad de Wald para Renovaciones]
Para el proceso de renovaci\'on $N\left(t\right)$,
\begin{eqnarray*}
\esp\left[T_{N\left(t\right)+1}\right]=\mu\esp\left[N\left(t\right)+1\right]\textrm{,  }t\geq0,
\end{eqnarray*}  
\end{Coro}


\begin{Def}
Sea $h\left(t\right)$ funci\'on de valores reales en $\rea$ acotada en intervalos finitos e igual a cero para $t<0$ La ecuaci\'on de renovaci\'on para $h\left(t\right)$ y la distribuci\'on $F$ es

\begin{eqnarray}%\label{Ec.Renovacion}
H\left(t\right)=h\left(t\right)+\int_{\left[0,t\right]}H\left(t-s\right)dF\left(s\right)\textrm{,    }t\geq0,
\end{eqnarray}
donde $H\left(t\right)$ es una funci\'on de valores reales. Esto es $H=h+F\star H$. Decimos que $H\left(t\right)$ es soluci\'on de esta ecuaci\'on si satisface la ecuaci\'on, y es acotada en intervalos finitos e iguales a cero para $t<0$.
\end{Def}

\begin{Prop}
La funci\'on $U\star h\left(t\right)$ es la \'unica soluci\'on de la ecuaci\'on de renovaci\'on (\ref{Ec.Renovacion}).
\end{Prop}

\begin{Teo}[Teorema Renovaci\'on Elemental]
\begin{eqnarray*}
t^{-1}U\left(t\right)\rightarrow 1/\mu\textrm{,    cuando }t\rightarrow\infty.
\end{eqnarray*}
\end{Teo}



Sup\'ongase que $N\left(t\right)$ es un proceso de renovaci\'on con distribuci\'on $F$ con media finita $\mu$.

\begin{Def}
La funci\'on de renovaci\'on asociada con la distribuci\'on $F$, del proceso $N\left(t\right)$, es
\begin{eqnarray*}
U\left(t\right)=\sum_{n=1}^{\infty}F^{n\star}\left(t\right),\textrm{   }t\geq0,
\end{eqnarray*}
donde $F^{0\star}\left(t\right)=\indora\left(t\geq0\right)$.
\end{Def}


\begin{Prop}
Sup\'ongase que la distribuci\'on de inter-renovaci\'on $F$ tiene densidad $f$. Entonces $U\left(t\right)$ tambi\'en tiene densidad, para $t>0$, y es $U^{'}\left(t\right)=\sum_{n=0}^{\infty}f^{n\star}\left(t\right)$. Adem\'as
\begin{eqnarray*}
\prob\left\{N\left(t\right)>N\left(t-\right)\right\}=0\textrm{,   }t\geq0.
\end{eqnarray*}
\end{Prop}

\begin{Def}
La Transformada de Laplace-Stieljes de $F$ est\'a dada por

\begin{eqnarray*}
\hat{F}\left(\alpha\right)=\int_{\rea_{+}}e^{-\alpha t}dF\left(t\right)\textrm{,  }\alpha\geq0.
\end{eqnarray*}
\end{Def}

Entonces

\begin{eqnarray*}
\hat{U}\left(\alpha\right)=\sum_{n=0}^{\infty}\hat{F^{n\star}}\left(\alpha\right)=\sum_{n=0}^{\infty}\hat{F}\left(\alpha\right)^{n}=\frac{1}{1-\hat{F}\left(\alpha\right)}.
\end{eqnarray*}


\begin{Prop}
La Transformada de Laplace $\hat{U}\left(\alpha\right)$ y $\hat{F}\left(\alpha\right)$ determina una a la otra de manera \'unica por la relaci\'on $\hat{U}\left(\alpha\right)=\frac{1}{1-\hat{F}\left(\alpha\right)}$.
\end{Prop}


\begin{Note}
Un proceso de renovaci\'on $N\left(t\right)$ cuyos tiempos de inter-renovaci\'on tienen media finita, es un proceso Poisson con tasa $\lambda$ si y s\'olo s\'i $\esp\left[U\left(t\right)\right]=\lambda t$, para $t\geq0$.
\end{Note}


\begin{Teo}
Sea $N\left(t\right)$ un proceso puntual simple con puntos de localizaci\'on $T_{n}$ tal que $\eta\left(t\right)=\esp\left[N\left(\right)\right]$ es finita para cada $t$. Entonces para cualquier funci\'on $f:\rea_{+}\rightarrow\rea$,
\begin{eqnarray*}
\esp\left[\sum_{n=1}^{N\left(\right)}f\left(T_{n}\right)\right]=\int_{\left(0,t\right]}f\left(s\right)d\eta\left(s\right)\textrm{,  }t\geq0,
\end{eqnarray*}
suponiendo que la integral exista. Adem\'as si $X_{1},X_{2},\ldots$ son variables aleatorias definidas en el mismo espacio de probabilidad que el proceso $N\left(t\right)$ tal que $\esp\left[X_{n}|T_{n}=s\right]=f\left(s\right)$, independiente de $n$. Entonces
\begin{eqnarray*}
\esp\left[\sum_{n=1}^{N\left(t\right)}X_{n}\right]=\int_{\left(0,t\right]}f\left(s\right)d\eta\left(s\right)\textrm{,  }t\geq0,
\end{eqnarray*} 
suponiendo que la integral exista. 
\end{Teo}

\begin{Coro}[Identidad de Wald para Renovaciones]
Para el proceso de renovaci\'on $N\left(t\right)$,
\begin{eqnarray*}
\esp\left[T_{N\left(t\right)+1}\right]=\mu\esp\left[N\left(t\right)+1\right]\textrm{,  }t\geq0,
\end{eqnarray*}  
\end{Coro}


\begin{Def}
Sea $h\left(t\right)$ funci\'on de valores reales en $\rea$ acotada en intervalos finitos e igual a cero para $t<0$ La ecuaci\'on de renovaci\'on para $h\left(t\right)$ y la distribuci\'on $F$ es

\begin{eqnarray}%\label{Ec.Renovacion}
H\left(t\right)=h\left(t\right)+\int_{\left[0,t\right]}H\left(t-s\right)dF\left(s\right)\textrm{,    }t\geq0,
\end{eqnarray}
donde $H\left(t\right)$ es una funci\'on de valores reales. Esto es $H=h+F\star H$. Decimos que $H\left(t\right)$ es soluci\'on de esta ecuaci\'on si satisface la ecuaci\'on, y es acotada en intervalos finitos e iguales a cero para $t<0$.
\end{Def}

\begin{Prop}
La funci\'on $U\star h\left(t\right)$ es la \'unica soluci\'on de la ecuaci\'on de renovaci\'on (\ref{Ec.Renovacion}).
\end{Prop}

\begin{Teo}[Teorema Renovaci\'on Elemental]
\begin{eqnarray*}
t^{-1}U\left(t\right)\rightarrow 1/\mu\textrm{,    cuando }t\rightarrow\infty.
\end{eqnarray*}
\end{Teo}


\begin{Note} Una funci\'on $h:\rea_{+}\rightarrow\rea$ es Directamente Riemann Integrable en los siguientes casos:
\begin{itemize}
\item[a)] $h\left(t\right)\geq0$ es decreciente y Riemann Integrable.
\item[b)] $h$ es continua excepto posiblemente en un conjunto de Lebesgue de medida 0, y $|h\left(t\right)|\leq b\left(t\right)$, donde $b$ es DRI.
\end{itemize}
\end{Note}

\begin{Teo}[Teorema Principal de Renovaci\'on]
Si $F$ es no aritm\'etica y $h\left(t\right)$ es Directamente Riemann Integrable (DRI), entonces

\begin{eqnarray*}
lim_{t\rightarrow\infty}U\star h=\frac{1}{\mu}\int_{\rea_{+}}h\left(s\right)ds.
\end{eqnarray*}
\end{Teo}

\begin{Prop}
Cualquier funci\'on $H\left(t\right)$ acotada en intervalos finitos y que es 0 para $t<0$ puede expresarse como
\begin{eqnarray*}
H\left(t\right)=U\star h\left(t\right)\textrm{,  donde }h\left(t\right)=H\left(t\right)-F\star H\left(t\right)
\end{eqnarray*}
\end{Prop}

\begin{Def}
Un proceso estoc\'astico $X\left(t\right)$ es crudamente regenerativo en un tiempo aleatorio positivo $T$ si
\begin{eqnarray*}
\esp\left[X\left(T+t\right)|T\right]=\esp\left[X\left(t\right)\right]\textrm{, para }t\geq0,\end{eqnarray*}
y con las esperanzas anteriores finitas.
\end{Def}

\begin{Prop}
Sup\'ongase que $X\left(t\right)$ es un proceso crudamente regenerativo en $T$, que tiene distribuci\'on $F$. Si $\esp\left[X\left(t\right)\right]$ es acotado en intervalos finitos, entonces
\begin{eqnarray*}
\esp\left[X\left(t\right)\right]=U\star h\left(t\right)\textrm{,  donde }h\left(t\right)=\esp\left[X\left(t\right)\indora\left(T>t\right)\right].
\end{eqnarray*}
\end{Prop}

\begin{Teo}[Regeneraci\'on Cruda]
Sup\'ongase que $X\left(t\right)$ es un proceso con valores positivo crudamente regenerativo en $T$, y def\'inase $M=\sup\left\{|X\left(t\right)|:t\leq T\right\}$. Si $T$ es no aritm\'etico y $M$ y $MT$ tienen media finita, entonces
\begin{eqnarray*}
lim_{t\rightarrow\infty}\esp\left[X\left(t\right)\right]=\frac{1}{\mu}\int_{\rea_{+}}h\left(s\right)ds,
\end{eqnarray*}
donde $h\left(t\right)=\esp\left[X\left(t\right)\indora\left(T>t\right)\right]$.
\end{Teo}


\begin{Note} Una funci\'on $h:\rea_{+}\rightarrow\rea$ es Directamente Riemann Integrable en los siguientes casos:
\begin{itemize}
\item[a)] $h\left(t\right)\geq0$ es decreciente y Riemann Integrable.
\item[b)] $h$ es continua excepto posiblemente en un conjunto de Lebesgue de medida 0, y $|h\left(t\right)|\leq b\left(t\right)$, donde $b$ es DRI.
\end{itemize}
\end{Note}

\begin{Teo}[Teorema Principal de Renovaci\'on]
Si $F$ es no aritm\'etica y $h\left(t\right)$ es Directamente Riemann Integrable (DRI), entonces

\begin{eqnarray*}
lim_{t\rightarrow\infty}U\star h=\frac{1}{\mu}\int_{\rea_{+}}h\left(s\right)ds.
\end{eqnarray*}
\end{Teo}

\begin{Prop}
Cualquier funci\'on $H\left(t\right)$ acotada en intervalos finitos y que es 0 para $t<0$ puede expresarse como
\begin{eqnarray*}
H\left(t\right)=U\star h\left(t\right)\textrm{,  donde }h\left(t\right)=H\left(t\right)-F\star H\left(t\right)
\end{eqnarray*}
\end{Prop}

\begin{Def}
Un proceso estoc\'astico $X\left(t\right)$ es crudamente regenerativo en un tiempo aleatorio positivo $T$ si
\begin{eqnarray*}
\esp\left[X\left(T+t\right)|T\right]=\esp\left[X\left(t\right)\right]\textrm{, para }t\geq0,\end{eqnarray*}
y con las esperanzas anteriores finitas.
\end{Def}

\begin{Prop}
Sup\'ongase que $X\left(t\right)$ es un proceso crudamente regenerativo en $T$, que tiene distribuci\'on $F$. Si $\esp\left[X\left(t\right)\right]$ es acotado en intervalos finitos, entonces
\begin{eqnarray*}
\esp\left[X\left(t\right)\right]=U\star h\left(t\right)\textrm{,  donde }h\left(t\right)=\esp\left[X\left(t\right)\indora\left(T>t\right)\right].
\end{eqnarray*}
\end{Prop}

\begin{Teo}[Regeneraci\'on Cruda]
Sup\'ongase que $X\left(t\right)$ es un proceso con valores positivo crudamente regenerativo en $T$, y def\'inase $M=\sup\left\{|X\left(t\right)|:t\leq T\right\}$. Si $T$ es no aritm\'etico y $M$ y $MT$ tienen media finita, entonces
\begin{eqnarray*}
lim_{t\rightarrow\infty}\esp\left[X\left(t\right)\right]=\frac{1}{\mu}\int_{\rea_{+}}h\left(s\right)ds,
\end{eqnarray*}
donde $h\left(t\right)=\esp\left[X\left(t\right)\indora\left(T>t\right)\right]$.
\end{Teo}

\begin{Def}
Para el proceso $\left\{\left(N\left(t\right),X\left(t\right)\right):t\geq0\right\}$, sus trayectoria muestrales en el intervalo de tiempo $\left[T_{n-1},T_{n}\right)$ est\'an descritas por
\begin{eqnarray*}
\zeta_{n}=\left(\xi_{n},\left\{X\left(T_{n-1}+t\right):0\leq t<\xi_{n}\right\}\right)
\end{eqnarray*}
Este $\zeta_{n}$ es el $n$-\'esimo segmento del proceso. El proceso es regenerativo sobre los tiempos $T_{n}$ si sus segmentos $\zeta_{n}$ son independientes e id\'enticamennte distribuidos.
\end{Def}


\begin{Note}
Si $\tilde{X}\left(t\right)$ con espacio de estados $\tilde{S}$ es regenerativo sobre $T_{n}$, entonces $X\left(t\right)=f\left(\tilde{X}\left(t\right)\right)$ tambi\'en es regenerativo sobre $T_{n}$, para cualquier funci\'on $f:\tilde{S}\rightarrow S$.
\end{Note}

\begin{Note}
Los procesos regenerativos son crudamente regenerativos, pero no al rev\'es.
\end{Note}


\begin{Note}
Un proceso estoc\'astico a tiempo continuo o discreto es regenerativo si existe un proceso de renovaci\'on  tal que los segmentos del proceso entre tiempos de renovaci\'on sucesivos son i.i.d., es decir, para $\left\{X\left(t\right):t\geq0\right\}$ proceso estoc\'astico a tiempo continuo con espacio de estados $S$, espacio m\'etrico.
\end{Note}

Para $\left\{X\left(t\right):t\geq0\right\}$ Proceso Estoc\'astico a tiempo continuo con estado de espacios $S$, que es un espacio m\'etrico, con trayectorias continuas por la derecha y con l\'imites por la izquierda c.s. Sea $N\left(t\right)$ un proceso de renovaci\'on en $\rea_{+}$ definido en el mismo espacio de probabilidad que $X\left(t\right)$, con tiempos de renovaci\'on $T$ y tiempos de inter-renovaci\'on $\xi_{n}=T_{n}-T_{n-1}$, con misma distribuci\'on $F$ de media finita $\mu$.



\begin{Def}
Para el proceso $\left\{\left(N\left(t\right),X\left(t\right)\right):t\geq0\right\}$, sus trayectoria muestrales en el intervalo de tiempo $\left[T_{n-1},T_{n}\right)$ est\'an descritas por
\begin{eqnarray*}
\zeta_{n}=\left(\xi_{n},\left\{X\left(T_{n-1}+t\right):0\leq t<\xi_{n}\right\}\right)
\end{eqnarray*}
Este $\zeta_{n}$ es el $n$-\'esimo segmento del proceso. El proceso es regenerativo sobre los tiempos $T_{n}$ si sus segmentos $\zeta_{n}$ son independientes e id\'enticamennte distribuidos.
\end{Def}

\begin{Note}
Un proceso regenerativo con media de la longitud de ciclo finita es llamado positivo recurrente.
\end{Note}

\begin{Teo}[Procesos Regenerativos]
Suponga que el proceso
\end{Teo}


\begin{Def}[Renewal Process Trinity]
Para un proceso de renovaci\'on $N\left(t\right)$, los siguientes procesos proveen de informaci\'on sobre los tiempos de renovaci\'on.
\begin{itemize}
\item $A\left(t\right)=t-T_{N\left(t\right)}$, el tiempo de recurrencia hacia atr\'as al tiempo $t$, que es el tiempo desde la \'ultima renovaci\'on para $t$.

\item $B\left(t\right)=T_{N\left(t\right)+1}-t$, el tiempo de recurrencia hacia adelante al tiempo $t$, residual del tiempo de renovaci\'on, que es el tiempo para la pr\'oxima renovaci\'on despu\'es de $t$.

\item $L\left(t\right)=\xi_{N\left(t\right)+1}=A\left(t\right)+B\left(t\right)$, la longitud del intervalo de renovaci\'on que contiene a $t$.
\end{itemize}
\end{Def}

\begin{Note}
El proceso tridimensional $\left(A\left(t\right),B\left(t\right),L\left(t\right)\right)$ es regenerativo sobre $T_{n}$, y por ende cada proceso lo es. Cada proceso $A\left(t\right)$ y $B\left(t\right)$ son procesos de MArkov a tiempo continuo con trayectorias continuas por partes en el espacio de estados $\rea_{+}$. Una expresi\'on conveniente para su distribuci\'on conjunta es, para $0\leq x<t,y\geq0$
\begin{equation}\label{NoRenovacion}
P\left\{A\left(t\right)>x,B\left(t\right)>y\right\}=
P\left\{N\left(t+y\right)-N\left((t-x)\right)=0\right\}
\end{equation}
\end{Note}

\begin{Ejem}[Tiempos de recurrencia Poisson]
Si $N\left(t\right)$ es un proceso Poisson con tasa $\lambda$, entonces de la expresi\'on (\ref{NoRenovacion}) se tiene que

\begin{eqnarray*}
\begin{array}{lc}
P\left\{A\left(t\right)>x,B\left(t\right)>y\right\}=e^{-\lambda\left(x+y\right)},&0\leq x<t,y\geq0,
\end{array}
\end{eqnarray*}
que es la probabilidad Poisson de no renovaciones en un intervalo de longitud $x+y$.

\end{Ejem}

\begin{Note}
Una cadena de Markov erg\'odica tiene la propiedad de ser estacionaria si la distribuci\'on de su estado al tiempo $0$ es su distribuci\'on estacionaria.
\end{Note}


\begin{Def}
Un proceso estoc\'astico a tiempo continuo $\left\{X\left(t\right):t\geq0\right\}$ en un espacio general es estacionario si sus distribuciones finito dimensionales son invariantes bajo cualquier  traslado: para cada $0\leq s_{1}<s_{2}<\cdots<s_{k}$ y $t\geq0$,
\begin{eqnarray*}
\left(X\left(s_{1}+t\right),\ldots,X\left(s_{k}+t\right)\right)=_{d}\left(X\left(s_{1}\right),\ldots,X\left(s_{k}\right)\right).
\end{eqnarray*}
\end{Def}

\begin{Note}
Un proceso de Markov es estacionario si $X\left(t\right)=_{d}X\left(0\right)$, $t\geq0$.
\end{Note}

Considerese el proceso $N\left(t\right)=\sum_{n}\indora\left(\tau_{n}\leq t\right)$ en $\rea_{+}$, con puntos $0<\tau_{1}<\tau_{2}<\cdots$.

\begin{Prop}
Si $N$ es un proceso puntual estacionario y $\esp\left[N\left(1\right)\right]<\infty$, entonces $\esp\left[N\left(t\right)\right]=t\esp\left[N\left(1\right)\right]$, $t\geq0$

\end{Prop}

\begin{Teo}
Los siguientes enunciados son equivalentes
\begin{itemize}
\item[i)] El proceso retardado de renovaci\'on $N$ es estacionario.

\item[ii)] EL proceso de tiempos de recurrencia hacia adelante $B\left(t\right)$ es estacionario.


\item[iii)] $\esp\left[N\left(t\right)\right]=t/\mu$,


\item[iv)] $G\left(t\right)=F_{e}\left(t\right)=\frac{1}{\mu}\int_{0}^{t}\left[1-F\left(s\right)\right]ds$
\end{itemize}
Cuando estos enunciados son ciertos, $P\left\{B\left(t\right)\leq x\right\}=F_{e}\left(x\right)$, para $t,x\geq0$.

\end{Teo}

\begin{Note}
Una consecuencia del teorema anterior es que el Proceso Poisson es el \'unico proceso sin retardo que es estacionario.
\end{Note}

\begin{Coro}
El proceso de renovaci\'on $N\left(t\right)$ sin retardo, y cuyos tiempos de inter renonaci\'on tienen media finita, es estacionario si y s\'olo si es un proceso Poisson.

\end{Coro}


%________________________________________________________________________
%\subsection{Procesos Regenerativos}
%________________________________________________________________________



\begin{Note}
Si $\tilde{X}\left(t\right)$ con espacio de estados $\tilde{S}$ es regenerativo sobre $T_{n}$, entonces $X\left(t\right)=f\left(\tilde{X}\left(t\right)\right)$ tambi\'en es regenerativo sobre $T_{n}$, para cualquier funci\'on $f:\tilde{S}\rightarrow S$.
\end{Note}

\begin{Note}
Los procesos regenerativos son crudamente regenerativos, pero no al rev\'es.
\end{Note}
%\subsection*{Procesos Regenerativos: Sigman\cite{Sigman1}}
\begin{Def}[Definici\'on Cl\'asica]
Un proceso estoc\'astico $X=\left\{X\left(t\right):t\geq0\right\}$ es llamado regenerativo is existe una variable aleatoria $R_{1}>0$ tal que
\begin{itemize}
\item[i)] $\left\{X\left(t+R_{1}\right):t\geq0\right\}$ es independiente de $\left\{\left\{X\left(t\right):t<R_{1}\right\},\right\}$
\item[ii)] $\left\{X\left(t+R_{1}\right):t\geq0\right\}$ es estoc\'asticamente equivalente a $\left\{X\left(t\right):t>0\right\}$
\end{itemize}

Llamamos a $R_{1}$ tiempo de regeneraci\'on, y decimos que $X$ se regenera en este punto.
\end{Def}

$\left\{X\left(t+R_{1}\right)\right\}$ es regenerativo con tiempo de regeneraci\'on $R_{2}$, independiente de $R_{1}$ pero con la misma distribuci\'on que $R_{1}$. Procediendo de esta manera se obtiene una secuencia de variables aleatorias independientes e id\'enticamente distribuidas $\left\{R_{n}\right\}$ llamados longitudes de ciclo. Si definimos a $Z_{k}\equiv R_{1}+R_{2}+\cdots+R_{k}$, se tiene un proceso de renovaci\'on llamado proceso de renovaci\'on encajado para $X$.




\begin{Def}
Para $x$ fijo y para cada $t\geq0$, sea $I_{x}\left(t\right)=1$ si $X\left(t\right)\leq x$,  $I_{x}\left(t\right)=0$ en caso contrario, y def\'inanse los tiempos promedio
\begin{eqnarray*}
\overline{X}&=&lim_{t\rightarrow\infty}\frac{1}{t}\int_{0}^{\infty}X\left(u\right)du\\
\prob\left(X_{\infty}\leq x\right)&=&lim_{t\rightarrow\infty}\frac{1}{t}\int_{0}^{\infty}I_{x}\left(u\right)du,
\end{eqnarray*}
cuando estos l\'imites existan.
\end{Def}

Como consecuencia del teorema de Renovaci\'on-Recompensa, se tiene que el primer l\'imite  existe y es igual a la constante
\begin{eqnarray*}
\overline{X}&=&\frac{\esp\left[\int_{0}^{R_{1}}X\left(t\right)dt\right]}{\esp\left[R_{1}\right]},
\end{eqnarray*}
suponiendo que ambas esperanzas son finitas.

\begin{Note}
\begin{itemize}
\item[a)] Si el proceso regenerativo $X$ es positivo recurrente y tiene trayectorias muestrales no negativas, entonces la ecuaci\'on anterior es v\'alida.
\item[b)] Si $X$ es positivo recurrente regenerativo, podemos construir una \'unica versi\'on estacionaria de este proceso, $X_{e}=\left\{X_{e}\left(t\right)\right\}$, donde $X_{e}$ es un proceso estoc\'astico regenerativo y estrictamente estacionario, con distribuci\'on marginal distribuida como $X_{\infty}$
\end{itemize}
\end{Note}

%________________________________________________________________________
%\subsection{Procesos Regenerativos}
%________________________________________________________________________

Para $\left\{X\left(t\right):t\geq0\right\}$ Proceso Estoc\'astico a tiempo continuo con estado de espacios $S$, que es un espacio m\'etrico, con trayectorias continuas por la derecha y con l\'imites por la izquierda c.s. Sea $N\left(t\right)$ un proceso de renovaci\'on en $\rea_{+}$ definido en el mismo espacio de probabilidad que $X\left(t\right)$, con tiempos de renovaci\'on $T$ y tiempos de inter-renovaci\'on $\xi_{n}=T_{n}-T_{n-1}$, con misma distribuci\'on $F$ de media finita $\mu$.



\begin{Def}
Para el proceso $\left\{\left(N\left(t\right),X\left(t\right)\right):t\geq0\right\}$, sus trayectoria muestrales en el intervalo de tiempo $\left[T_{n-1},T_{n}\right)$ est\'an descritas por
\begin{eqnarray*}
\zeta_{n}=\left(\xi_{n},\left\{X\left(T_{n-1}+t\right):0\leq t<\xi_{n}\right\}\right)
\end{eqnarray*}
Este $\zeta_{n}$ es el $n$-\'esimo segmento del proceso. El proceso es regenerativo sobre los tiempos $T_{n}$ si sus segmentos $\zeta_{n}$ son independientes e id\'enticamennte distribuidos.
\end{Def}


\begin{Note}
Si $\tilde{X}\left(t\right)$ con espacio de estados $\tilde{S}$ es regenerativo sobre $T_{n}$, entonces $X\left(t\right)=f\left(\tilde{X}\left(t\right)\right)$ tambi\'en es regenerativo sobre $T_{n}$, para cualquier funci\'on $f:\tilde{S}\rightarrow S$.
\end{Note}

\begin{Note}
Los procesos regenerativos son crudamente regenerativos, pero no al rev\'es.
\end{Note}

\begin{Def}[Definici\'on Cl\'asica]
Un proceso estoc\'astico $X=\left\{X\left(t\right):t\geq0\right\}$ es llamado regenerativo is existe una variable aleatoria $R_{1}>0$ tal que
\begin{itemize}
\item[i)] $\left\{X\left(t+R_{1}\right):t\geq0\right\}$ es independiente de $\left\{\left\{X\left(t\right):t<R_{1}\right\},\right\}$
\item[ii)] $\left\{X\left(t+R_{1}\right):t\geq0\right\}$ es estoc\'asticamente equivalente a $\left\{X\left(t\right):t>0\right\}$
\end{itemize}

Llamamos a $R_{1}$ tiempo de regeneraci\'on, y decimos que $X$ se regenera en este punto.
\end{Def}

$\left\{X\left(t+R_{1}\right)\right\}$ es regenerativo con tiempo de regeneraci\'on $R_{2}$, independiente de $R_{1}$ pero con la misma distribuci\'on que $R_{1}$. Procediendo de esta manera se obtiene una secuencia de variables aleatorias independientes e id\'enticamente distribuidas $\left\{R_{n}\right\}$ llamados longitudes de ciclo. Si definimos a $Z_{k}\equiv R_{1}+R_{2}+\cdots+R_{k}$, se tiene un proceso de renovaci\'on llamado proceso de renovaci\'on encajado para $X$.

\begin{Note}
Un proceso regenerativo con media de la longitud de ciclo finita es llamado positivo recurrente.
\end{Note}


\begin{Def}
Para $x$ fijo y para cada $t\geq0$, sea $I_{x}\left(t\right)=1$ si $X\left(t\right)\leq x$,  $I_{x}\left(t\right)=0$ en caso contrario, y def\'inanse los tiempos promedio
\begin{eqnarray*}
\overline{X}&=&lim_{t\rightarrow\infty}\frac{1}{t}\int_{0}^{\infty}X\left(u\right)du\\
\prob\left(X_{\infty}\leq x\right)&=&lim_{t\rightarrow\infty}\frac{1}{t}\int_{0}^{\infty}I_{x}\left(u\right)du,
\end{eqnarray*}
cuando estos l\'imites existan.
\end{Def}

Como consecuencia del teorema de Renovaci\'on-Recompensa, se tiene que el primer l\'imite  existe y es igual a la constante
\begin{eqnarray*}
\overline{X}&=&\frac{\esp\left[\int_{0}^{R_{1}}X\left(t\right)dt\right]}{\esp\left[R_{1}\right]},
\end{eqnarray*}
suponiendo que ambas esperanzas son finitas.

\begin{Note}
\begin{itemize}
\item[a)] Si el proceso regenerativo $X$ es positivo recurrente y tiene trayectorias muestrales no negativas, entonces la ecuaci\'on anterior es v\'alida.
\item[b)] Si $X$ es positivo recurrente regenerativo, podemos construir una \'unica versi\'on estacionaria de este proceso, $X_{e}=\left\{X_{e}\left(t\right)\right\}$, donde $X_{e}$ es un proceso estoc\'astico regenerativo y estrictamente estacionario, con distribuci\'on marginal distribuida como $X_{\infty}$
\end{itemize}
\end{Note}

%__________________________________________________________________________________________
%\subsection{Procesos Regenerativos Estacionarios - Stidham \cite{Stidham}}
%__________________________________________________________________________________________


Un proceso estoc\'astico a tiempo continuo $\left\{V\left(t\right),t\geq0\right\}$ es un proceso regenerativo si existe una sucesi\'on de variables aleatorias independientes e id\'enticamente distribuidas $\left\{X_{1},X_{2},\ldots\right\}$, sucesi\'on de renovaci\'on, tal que para cualquier conjunto de Borel $A$, 

\begin{eqnarray*}
\prob\left\{V\left(t\right)\in A|X_{1}+X_{2}+\cdots+X_{R\left(t\right)}=s,\left\{V\left(\tau\right),\tau<s\right\}\right\}=\prob\left\{V\left(t-s\right)\in A|X_{1}>t-s\right\},
\end{eqnarray*}
para todo $0\leq s\leq t$, donde $R\left(t\right)=\max\left\{X_{1}+X_{2}+\cdots+X_{j}\leq t\right\}=$n\'umero de renovaciones ({\emph{puntos de regeneraci\'on}}) que ocurren en $\left[0,t\right]$. El intervalo $\left[0,X_{1}\right)$ es llamado {\emph{primer ciclo de regeneraci\'on}} de $\left\{V\left(t \right),t\geq0\right\}$, $\left[X_{1},X_{1}+X_{2}\right)$ el {\emph{segundo ciclo de regeneraci\'on}}, y as\'i sucesivamente.

Sea $X=X_{1}$ y sea $F$ la funci\'on de distrbuci\'on de $X$


\begin{Def}
Se define el proceso estacionario, $\left\{V^{*}\left(t\right),t\geq0\right\}$, para $\left\{V\left(t\right),t\geq0\right\}$ por

\begin{eqnarray*}
\prob\left\{V\left(t\right)\in A\right\}=\frac{1}{\esp\left[X\right]}\int_{0}^{\infty}\prob\left\{V\left(t+x\right)\in A|X>x\right\}\left(1-F\left(x\right)\right)dx,
\end{eqnarray*} 
para todo $t\geq0$ y todo conjunto de Borel $A$.
\end{Def}

\begin{Def}
Una distribuci\'on se dice que es {\emph{aritm\'etica}} si todos sus puntos de incremento son m\'ultiplos de la forma $0,\lambda, 2\lambda,\ldots$ para alguna $\lambda>0$ entera.
\end{Def}


\begin{Def}
Una modificaci\'on medible de un proceso $\left\{V\left(t\right),t\geq0\right\}$, es una versi\'on de este, $\left\{V\left(t,w\right)\right\}$ conjuntamente medible para $t\geq0$ y para $w\in S$, $S$ espacio de estados para $\left\{V\left(t\right),t\geq0\right\}$.
\end{Def}

\begin{Teo}
Sea $\left\{V\left(t\right),t\geq\right\}$ un proceso regenerativo no negativo con modificaci\'on medible. Sea $\esp\left[X\right]<\infty$. Entonces el proceso estacionario dado por la ecuaci\'on anterior est\'a bien definido y tiene funci\'on de distribuci\'on independiente de $t$, adem\'as
\begin{itemize}
\item[i)] \begin{eqnarray*}
\esp\left[V^{*}\left(0\right)\right]&=&\frac{\esp\left[\int_{0}^{X}V\left(s\right)ds\right]}{\esp\left[X\right]}\end{eqnarray*}
\item[ii)] Si $\esp\left[V^{*}\left(0\right)\right]<\infty$, equivalentemente, si $\esp\left[\int_{0}^{X}V\left(s\right)ds\right]<\infty$,entonces
\begin{eqnarray*}
\frac{\int_{0}^{t}V\left(s\right)ds}{t}\rightarrow\frac{\esp\left[\int_{0}^{X}V\left(s\right)ds\right]}{\esp\left[X\right]}
\end{eqnarray*}
con probabilidad 1 y en media, cuando $t\rightarrow\infty$.
\end{itemize}
\end{Teo}
%
%___________________________________________________________________________________________
%\vspace{5.5cm}
%\chapter{Cadenas de Markov estacionarias}
%\vspace{-1.0cm}


%__________________________________________________________________________________________
%\subsection{Procesos Regenerativos Estacionarios - Stidham \cite{Stidham}}
%__________________________________________________________________________________________


Un proceso estoc\'astico a tiempo continuo $\left\{V\left(t\right),t\geq0\right\}$ es un proceso regenerativo si existe una sucesi\'on de variables aleatorias independientes e id\'enticamente distribuidas $\left\{X_{1},X_{2},\ldots\right\}$, sucesi\'on de renovaci\'on, tal que para cualquier conjunto de Borel $A$, 

\begin{eqnarray*}
\prob\left\{V\left(t\right)\in A|X_{1}+X_{2}+\cdots+X_{R\left(t\right)}=s,\left\{V\left(\tau\right),\tau<s\right\}\right\}=\prob\left\{V\left(t-s\right)\in A|X_{1}>t-s\right\},
\end{eqnarray*}
para todo $0\leq s\leq t$, donde $R\left(t\right)=\max\left\{X_{1}+X_{2}+\cdots+X_{j}\leq t\right\}=$n\'umero de renovaciones ({\emph{puntos de regeneraci\'on}}) que ocurren en $\left[0,t\right]$. El intervalo $\left[0,X_{1}\right)$ es llamado {\emph{primer ciclo de regeneraci\'on}} de $\left\{V\left(t \right),t\geq0\right\}$, $\left[X_{1},X_{1}+X_{2}\right)$ el {\emph{segundo ciclo de regeneraci\'on}}, y as\'i sucesivamente.

Sea $X=X_{1}$ y sea $F$ la funci\'on de distrbuci\'on de $X$


\begin{Def}
Se define el proceso estacionario, $\left\{V^{*}\left(t\right),t\geq0\right\}$, para $\left\{V\left(t\right),t\geq0\right\}$ por

\begin{eqnarray*}
\prob\left\{V\left(t\right)\in A\right\}=\frac{1}{\esp\left[X\right]}\int_{0}^{\infty}\prob\left\{V\left(t+x\right)\in A|X>x\right\}\left(1-F\left(x\right)\right)dx,
\end{eqnarray*} 
para todo $t\geq0$ y todo conjunto de Borel $A$.
\end{Def}

\begin{Def}
Una distribuci\'on se dice que es {\emph{aritm\'etica}} si todos sus puntos de incremento son m\'ultiplos de la forma $0,\lambda, 2\lambda,\ldots$ para alguna $\lambda>0$ entera.
\end{Def}


\begin{Def}
Una modificaci\'on medible de un proceso $\left\{V\left(t\right),t\geq0\right\}$, es una versi\'on de este, $\left\{V\left(t,w\right)\right\}$ conjuntamente medible para $t\geq0$ y para $w\in S$, $S$ espacio de estados para $\left\{V\left(t\right),t\geq0\right\}$.
\end{Def}

\begin{Teo}
Sea $\left\{V\left(t\right),t\geq\right\}$ un proceso regenerativo no negativo con modificaci\'on medible. Sea $\esp\left[X\right]<\infty$. Entonces el proceso estacionario dado por la ecuaci\'on anterior est\'a bien definido y tiene funci\'on de distribuci\'on independiente de $t$, adem\'as
\begin{itemize}
\item[i)] \begin{eqnarray*}
\esp\left[V^{*}\left(0\right)\right]&=&\frac{\esp\left[\int_{0}^{X}V\left(s\right)ds\right]}{\esp\left[X\right]}\end{eqnarray*}
\item[ii)] Si $\esp\left[V^{*}\left(0\right)\right]<\infty$, equivalentemente, si $\esp\left[\int_{0}^{X}V\left(s\right)ds\right]<\infty$,entonces
\begin{eqnarray*}
\frac{\int_{0}^{t}V\left(s\right)ds}{t}\rightarrow\frac{\esp\left[\int_{0}^{X}V\left(s\right)ds\right]}{\esp\left[X\right]}
\end{eqnarray*}
con probabilidad 1 y en media, cuando $t\rightarrow\infty$.
\end{itemize}
\end{Teo}

Para $\left\{X\left(t\right):t\geq0\right\}$ Proceso Estoc\'astico a tiempo continuo con estado de espacios $S$, que es un espacio m\'etrico, con trayectorias continuas por la derecha y con l\'imites por la izquierda c.s. Sea $N\left(t\right)$ un proceso de renovaci\'on en $\rea_{+}$ definido en el mismo espacio de probabilidad que $X\left(t\right)$, con tiempos de renovaci\'on $T$ y tiempos de inter-renovaci\'on $\xi_{n}=T_{n}-T_{n-1}$, con misma distribuci\'on $F$ de media finita $\mu$.


%______________________________________________________________________
%\subsection{Ejemplos, Notas importantes}


Sean $T_{1},T_{2},\ldots$ los puntos donde las longitudes de las colas de la red de sistemas de visitas c\'iclicas son cero simult\'aneamente, cuando la cola $Q_{j}$ es visitada por el servidor para dar servicio, es decir, $L_{1}\left(T_{i}\right)=0,L_{2}\left(T_{i}\right)=0,\hat{L}_{1}\left(T_{i}\right)=0$ y $\hat{L}_{2}\left(T_{i}\right)=0$, a estos puntos se les denominar\'a puntos regenerativos. Sea la funci\'on generadora de momentos para $L_{i}$, el n\'umero de usuarios en la cola $Q_{i}\left(z\right)$ en cualquier momento, est\'a dada por el tiempo promedio de $z^{L_{i}\left(t\right)}$ sobre el ciclo regenerativo definido anteriormente:

\begin{eqnarray*}
Q_{i}\left(z\right)&=&\esp\left[z^{L_{i}\left(t\right)}\right]=\frac{\esp\left[\sum_{m=1}^{M_{i}}\sum_{t=\tau_{i}\left(m\right)}^{\tau_{i}\left(m+1\right)-1}z^{L_{i}\left(t\right)}\right]}{\esp\left[\sum_{m=1}^{M_{i}}\tau_{i}\left(m+1\right)-\tau_{i}\left(m\right)\right]}
\end{eqnarray*}

$M_{i}$ es un tiempo de paro en el proceso regenerativo con $\esp\left[M_{i}\right]<\infty$\footnote{En Stidham\cite{Stidham} y Heyman  se muestra que una condici\'on suficiente para que el proceso regenerativo 
estacionario sea un procesoo estacionario es que el valor esperado del tiempo del ciclo regenerativo sea finito, es decir: $\esp\left[\sum_{m=1}^{M_{i}}C_{i}^{(m)}\right]<\infty$, como cada $C_{i}^{(m)}$ contiene intervalos de r\'eplica positivos, se tiene que $\esp\left[M_{i}\right]<\infty$, adem\'as, como $M_{i}>0$, se tiene que la condici\'on anterior es equivalente a tener que $\esp\left[C_{i}\right]<\infty$,
por lo tanto una condici\'on suficiente para la existencia del proceso regenerativo est\'a dada por $\sum_{k=1}^{N}\mu_{k}<1.$}, se sigue del lema de Wald que:


\begin{eqnarray*}
\esp\left[\sum_{m=1}^{M_{i}}\sum_{t=\tau_{i}\left(m\right)}^{\tau_{i}\left(m+1\right)-1}z^{L_{i}\left(t\right)}\right]&=&\esp\left[M_{i}\right]\esp\left[\sum_{t=\tau_{i}\left(m\right)}^{\tau_{i}\left(m+1\right)-1}z^{L_{i}\left(t\right)}\right]\\
\esp\left[\sum_{m=1}^{M_{i}}\tau_{i}\left(m+1\right)-\tau_{i}\left(m\right)\right]&=&\esp\left[M_{i}\right]\esp\left[\tau_{i}\left(m+1\right)-\tau_{i}\left(m\right)\right]
\end{eqnarray*}

por tanto se tiene que


\begin{eqnarray*}
Q_{i}\left(z\right)&=&\frac{\esp\left[\sum_{t=\tau_{i}\left(m\right)}^{\tau_{i}\left(m+1\right)-1}z^{L_{i}\left(t\right)}\right]}{\esp\left[\tau_{i}\left(m+1\right)-\tau_{i}\left(m\right)\right]}
\end{eqnarray*}

observar que el denominador es simplemente la duraci\'on promedio del tiempo del ciclo.


Haciendo las siguientes sustituciones en la ecuaci\'on (\ref{Corolario2}): $n\rightarrow t-\tau_{i}\left(m\right)$, $T \rightarrow \overline{\tau}_{i}\left(m\right)-\tau_{i}\left(m\right)$, $L_{n}\rightarrow L_{i}\left(t\right)$ y $F\left(z\right)=\esp\left[z^{L_{0}}\right]\rightarrow F_{i}\left(z\right)=\esp\left[z^{L_{i}\tau_{i}\left(m\right)}\right]$, se puede ver que

\begin{eqnarray}\label{Eq.Arribos.Primera}
\esp\left[\sum_{n=0}^{T-1}z^{L_{n}}\right]=
\esp\left[\sum_{t=\tau_{i}\left(m\right)}^{\overline{\tau}_{i}\left(m\right)-1}z^{L_{i}\left(t\right)}\right]
=z\frac{F_{i}\left(z\right)-1}{z-P_{i}\left(z\right)}
\end{eqnarray}

Por otra parte durante el tiempo de intervisita para la cola $i$, $L_{i}\left(t\right)$ solamente se incrementa de manera que el incremento por intervalo de tiempo est\'a dado por la funci\'on generadora de probabilidades de $P_{i}\left(z\right)$, por tanto la suma sobre el tiempo de intervisita puede evaluarse como:

\begin{eqnarray*}
\esp\left[\sum_{t=\tau_{i}\left(m\right)}^{\tau_{i}\left(m+1\right)-1}z^{L_{i}\left(t\right)}\right]&=&\esp\left[\sum_{t=\tau_{i}\left(m\right)}^{\tau_{i}\left(m+1\right)-1}\left\{P_{i}\left(z\right)\right\}^{t-\overline{\tau}_{i}\left(m\right)}\right]=\frac{1-\esp\left[\left\{P_{i}\left(z\right)\right\}^{\tau_{i}\left(m+1\right)-\overline{\tau}_{i}\left(m\right)}\right]}{1-P_{i}\left(z\right)}\\
&=&\frac{1-I_{i}\left[P_{i}\left(z\right)\right]}{1-P_{i}\left(z\right)}
\end{eqnarray*}
por tanto

\begin{eqnarray*}
\esp\left[\sum_{t=\tau_{i}\left(m\right)}^{\tau_{i}\left(m+1\right)-1}z^{L_{i}\left(t\right)}\right]&=&
\frac{1-F_{i}\left(z\right)}{1-P_{i}\left(z\right)}
\end{eqnarray*}

Por lo tanto

\begin{eqnarray*}
Q_{i}\left(z\right)&=&\frac{\esp\left[\sum_{t=\tau_{i}\left(m\right)}^{\tau_{i}
\left(m+1\right)-1}z^{L_{i}\left(t\right)}\right]}{\esp\left[\tau_{i}\left(m+1\right)-\tau_{i}\left(m\right)\right]}\\
&=&\frac{1}{\esp\left[\tau_{i}\left(m+1\right)-\tau_{i}\left(m\right)\right]}
\left\{
\esp\left[\sum_{t=\tau_{i}\left(m\right)}^{\overline{\tau}_{i}\left(m\right)-1}
z^{L_{i}\left(t\right)}\right]
+\esp\left[\sum_{t=\overline{\tau}_{i}\left(m\right)}^{\tau_{i}\left(m+1\right)-1}
z^{L_{i}\left(t\right)}\right]\right\}\\
&=&\frac{1}{\esp\left[\tau_{i}\left(m+1\right)-\tau_{i}\left(m\right)\right]}
\left\{
z\frac{F_{i}\left(z\right)-1}{z-P_{i}\left(z\right)}+\frac{1-F_{i}\left(z\right)}
{1-P_{i}\left(z\right)}
\right\}
\end{eqnarray*}

es decir

\begin{equation}
Q_{i}\left(z\right)=\frac{1}{\esp\left[C_{i}\right]}\cdot\frac{1-F_{i}\left(z\right)}{P_{i}\left(z\right)-z}\cdot\frac{\left(1-z\right)P_{i}\left(z\right)}{1-P_{i}\left(z\right)}
\end{equation}

\begin{Teo}
Dada una Red de Sistemas de Visitas C\'iclicas (RSVC), conformada por dos Sistemas de Visitas C\'iclicas (SVC), donde cada uno de ellos consta de dos colas tipo $M/M/1$. Los dos sistemas est\'an comunicados entre s\'i por medio de la transferencia de usuarios entre las colas $Q_{1}\leftrightarrow Q_{3}$ y $Q_{2}\leftrightarrow Q_{4}$. Se definen los eventos para los procesos de arribos al tiempo $t$, $A_{j}\left(t\right)=\left\{0 \textrm{ arribos en }Q_{j}\textrm{ al tiempo }t\right\}$ para alg\'un tiempo $t\geq0$ y $Q_{j}$ la cola $j$-\'esima en la RSVC, para $j=1,2,3,4$.  Existe un intervalo $I\neq\emptyset$ tal que para $T^{*}\in I$, tal que $\prob\left\{A_{1}\left(T^{*}\right),A_{2}\left(Tt^{*}\right),
A_{3}\left(T^{*}\right),A_{4}\left(T^{*}\right)|T^{*}\in I\right\}>0$.
\end{Teo}

\begin{proof}
Sin p\'erdida de generalidad podemos considerar como base del an\'alisis a la cola $Q_{1}$ del primer sistema que conforma la RSVC.

Sea $n>0$, ciclo en el primer sistema en el que se sabe que $L_{j}\left(\overline{\tau}_{1}\left(n\right)\right)=0$, pues la pol\'itica de servicio con que atienden los servidores es la exhaustiva. Como es sabido, para trasladarse a la siguiente cola, el servidor incurre en un tiempo de traslado $r_{1}\left(n\right)>0$, entonces tenemos que $\tau_{2}\left(n\right)=\overline{\tau}_{1}\left(n\right)+r_{1}\left(n\right)$.


Definamos el intervalo $I_{1}\equiv\left[\overline{\tau}_{1}\left(n\right),\tau_{2}\left(n\right)\right]$ de longitud $\xi_{1}=r_{1}\left(n\right)$. Dado que los tiempos entre arribo son exponenciales con tasa $\tilde{\mu}_{1}=\mu_{1}+\hat{\mu}_{1}$ ($\mu_{1}$ son los arribos a $Q_{1}$ por primera vez al sistema, mientras que $\hat{\mu}_{1}$ son los arribos de traslado procedentes de $Q_{3}$) se tiene que la probabilidad del evento $A_{1}\left(t\right)$ est\'a dada por 

\begin{equation}
\prob\left\{A_{1}\left(t\right)|t\in I_{1}\left(n\right)\right\}=e^{-\tilde{\mu}_{1}\xi_{1}\left(n\right)}.
\end{equation} 

Por otra parte, para la cola $Q_{2}$, el tiempo $\overline{\tau}_{2}\left(n-1\right)$ es tal que $L_{2}\left(\overline{\tau}_{2}\left(n-1\right)\right)=0$, es decir, es el tiempo en que la cola queda totalmente vac\'ia en el ciclo anterior a $n$. Entonces tenemos un sgundo intervalo $I_{2}\equiv\left[\overline{\tau}_{2}\left(n-1\right),\tau_{2}\left(n\right)\right]$. Por lo tanto la probabilidad del evento $A_{2}\left(t\right)$ tiene probabilidad dada por

\begin{equation}
\prob\left\{A_{2}\left(t\right)|t\in I_{2}\left(n\right)\right\}=e^{-\tilde{\mu}_{2}\xi_{2}\left(n\right)},
\end{equation} 

donde $\xi_{2}\left(n\right)=\tau_{2}\left(n\right)-\overline{\tau}_{2}\left(n-1\right)$.



Entonces, se tiene que

\begin{eqnarray*}
\prob\left\{A_{1}\left(t\right),A_{2}\left(t\right)|t\in I_{1}\left(n\right)\right\}&=&
\prob\left\{A_{1}\left(t\right)|t\in I_{1}\left(n\right)\right\}
\prob\left\{A_{2}\left(t\right)|t\in I_{1}\left(n\right)\right\}\\
&\geq&
\prob\left\{A_{1}\left(t\right)|t\in I_{1}\left(n\right)\right\}
\prob\left\{A_{2}\left(t\right)|t\in I_{2}\left(n\right)\right\}\\
&=&e^{-\tilde{\mu}_{1}\xi_{1}\left(n\right)}e^{-\tilde{\mu}_{2}\xi_{2}\left(n\right)}
=e^{-\left[\tilde{\mu}_{1}\xi_{1}\left(n\right)+\tilde{\mu}_{2}\xi_{2}\left(n\right)\right]}.
\end{eqnarray*}


es decir, 

\begin{equation}
\prob\left\{A_{1}\left(t\right),A_{2}\left(t\right)|t\in I_{1}\left(n\right)\right\}
=e^{-\left[\tilde{\mu}_{1}\xi_{1}\left(n\right)+\tilde{\mu}_{2}\xi_{2}
\left(n\right)\right]}>0.
\end{equation}

En lo que respecta a la relaci\'on entre los dos SVC que conforman la RSVC, se afirma que existe $m>0$ tal que $\overline{\tau}_{3}\left(m\right)<\tau_{2}\left(n\right)<\tau_{4}\left(m\right)$.

Para $Q_{3}$ sea $I_{3}=\left[\overline{\tau}_{3}\left(m\right),\tau_{4}\left(m\right)\right]$ con longitud  $\xi_{3}\left(m\right)=r_{3}\left(m\right)$, entonces 

\begin{equation}
\prob\left\{A_{3}\left(t\right)|t\in I_{3}\left(n\right)\right\}=e^{-\tilde{\mu}_{3}\xi_{3}\left(n\right)}.
\end{equation} 

An\'alogamente que como se hizo para $Q_{2}$, tenemos que para $Q_{4}$ se tiene el intervalo $I_{4}=\left[\overline{\tau}_{4}\left(m-1\right),\tau_{4}\left(m\right)\right]$ con longitud $\xi_{4}\left(m\right)=\tau_{4}\left(m\right)-\overline{\tau}_{4}\left(m-1\right)$, entonces


\begin{equation}
\prob\left\{A_{4}\left(t\right)|t\in I_{4}\left(m\right)\right\}=e^{-\tilde{\mu}_{4}\xi_{4}\left(n\right)}.
\end{equation} 

Al igual que para el primer sistema, dado que $I_{3}\left(m\right)\subset I_{4}\left(m\right)$, se tiene que

\begin{eqnarray*}
\xi_{3}\left(m\right)\leq\xi_{4}\left(m\right)&\Leftrightarrow& -\xi_{3}\left(m\right)\geq-\xi_{4}\left(m\right)
\\
-\tilde{\mu}_{4}\xi_{3}\left(m\right)\geq-\tilde{\mu}_{4}\xi_{4}\left(m\right)&\Leftrightarrow&
e^{-\tilde{\mu}_{4}\xi_{3}\left(m\right)}\geq e^{-\tilde{\mu}_{4}\xi_{4}\left(m\right)}\\
\prob\left\{A_{4}\left(t\right)|t\in I_{3}\left(m\right)\right\}&\geq&
\prob\left\{A_{4}\left(t\right)|t\in I_{4}\left(m\right)\right\}
\end{eqnarray*}

Entonces, dado que los eventos $A_{3}$ y $A_{4}$ son independientes, se tiene que

\begin{eqnarray*}
\prob\left\{A_{3}\left(t\right),A_{4}\left(t\right)|t\in I_{3}\left(m\right)\right\}&=&
\prob\left\{A_{3}\left(t\right)|t\in I_{3}\left(m\right)\right\}
\prob\left\{A_{4}\left(t\right)|t\in I_{3}\left(m\right)\right\}\\
&\geq&
\prob\left\{A_{3}\left(t\right)|t\in I_{3}\left(n\right)\right\}
\prob\left\{A_{4}\left(t\right)|t\in I_{4}\left(n\right)\right\}\\
&=&e^{-\tilde{\mu}_{3}\xi_{3}\left(m\right)}e^{-\tilde{\mu}_{4}\xi_{4}
\left(m\right)}
=e^{-\left[\tilde{\mu}_{3}\xi_{3}\left(m\right)+\tilde{\mu}_{4}\xi_{4}
\left(m\right)\right]}.
\end{eqnarray*}


es decir, 

\begin{equation}
\prob\left\{A_{3}\left(t\right),A_{4}\left(t\right)|t\in I_{3}\left(m\right)\right\}
=e^{-\left[\tilde{\mu}_{3}\xi_{3}\left(m\right)+\tilde{\mu}_{4}\xi_{4}
\left(m\right)\right]}>0.
\end{equation}

Por construcci\'on se tiene que $I\left(n,m\right)\equiv I_{1}\left(n\right)\cap I_{3}\left(m\right)\neq\emptyset$,entonces en particular se tienen las contenciones $I\left(n,m\right)\subseteq I_{1}\left(n\right)$ y $I\left(n,m\right)\subseteq I_{3}\left(m\right)$, por lo tanto si definimos $\xi_{n,m}\equiv\ell\left(I\left(n,m\right)\right)$ tenemos que

\begin{eqnarray*}
\xi_{n,m}\leq\xi_{1}\left(n\right)\textrm{ y }\xi_{n,m}\leq\xi_{3}\left(m\right)\textrm{ entonces }
-\xi_{n,m}\geq-\xi_{1}\left(n\right)\textrm{ y }-\xi_{n,m}\leq-\xi_{3}\left(m\right)\\
\end{eqnarray*}
por lo tanto tenemos las desigualdades 



\begin{eqnarray*}
\begin{array}{ll}
-\tilde{\mu}_{1}\xi_{n,m}\geq-\tilde{\mu}_{1}\xi_{1}\left(n\right),&
-\tilde{\mu}_{2}\xi_{n,m}\geq-\tilde{\mu}_{2}\xi_{1}\left(n\right)
\geq-\tilde{\mu}_{2}\xi_{2}\left(n\right),\\
-\tilde{\mu}_{3}\xi_{n,m}\geq-\tilde{\mu}_{3}\xi_{3}\left(m\right),&
-\tilde{\mu}_{4}\xi_{n,m}\geq-\tilde{\mu}_{4}\xi_{3}\left(m\right)
\geq-\tilde{\mu}_{4}\xi_{4}\left(m\right).
\end{array}
\end{eqnarray*}

Sea $T^{*}\in I_{n,m}$, entonces dado que en particular $T^{*}\in I_{1}\left(n\right)$ se cumple con probabilidad positiva que no hay arribos a las colas $Q_{1}$ y $Q_{2}$, en consecuencia, tampoco hay usuarios de transferencia para $Q_{3}$ y $Q_{4}$, es decir, $\tilde{\mu}_{1}=\mu_{1}$, $\tilde{\mu}_{2}=\mu_{2}$, $\tilde{\mu}_{3}=\mu_{3}$, $\tilde{\mu}_{4}=\mu_{4}$, es decir, los eventos $Q_{1}$ y $Q_{3}$ son condicionalmente independientes en el intervalo $I_{n,m}$; lo mismo ocurre para las colas $Q_{2}$ y $Q_{4}$, por lo tanto tenemos que


\begin{eqnarray}
\begin{array}{l}
\prob\left\{A_{1}\left(T^{*}\right),A_{2}\left(T^{*}\right),
A_{3}\left(T^{*}\right),A_{4}\left(T^{*}\right)|T^{*}\in I_{n,m}\right\}
=\prod_{j=1}^{4}\prob\left\{A_{j}\left(T^{*}\right)|T^{*}\in I_{n,m}\right\}\\
\geq\prob\left\{A_{1}\left(T^{*}\right)|T^{*}\in I_{1}\left(n\right)\right\}
\prob\left\{A_{2}\left(T^{*}\right)|T^{*}\in I_{2}\left(n\right)\right\}
\prob\left\{A_{3}\left(T^{*}\right)|T^{*}\in I_{3}\left(m\right)\right\}
\prob\left\{A_{4}\left(T^{*}\right)|T^{*}\in I_{4}\left(m\right)\right\}\\
=e^{-\mu_{1}\xi_{1}\left(n\right)}
e^{-\mu_{2}\xi_{2}\left(n\right)}
e^{-\mu_{3}\xi_{3}\left(m\right)}
e^{-\mu_{4}\xi_{4}\left(m\right)}
=e^{-\left[\tilde{\mu}_{1}\xi_{1}\left(n\right)
+\tilde{\mu}_{2}\xi_{2}\left(n\right)
+\tilde{\mu}_{3}\xi_{3}\left(m\right)
+\tilde{\mu}_{4}\xi_{4}
\left(m\right)\right]}>0.
\end{array}
\end{eqnarray}
\end{proof}


Estos resultados aparecen en Daley (1968) \cite{Daley68} para $\left\{T_{n}\right\}$ intervalos de inter-arribo, $\left\{D_{n}\right\}$ intervalos de inter-salida y $\left\{S_{n}\right\}$ tiempos de servicio.

\begin{itemize}
\item Si el proceso $\left\{T_{n}\right\}$ es Poisson, el proceso $\left\{D_{n}\right\}$ es no correlacionado si y s\'olo si es un proceso Poisso, lo cual ocurre si y s\'olo si $\left\{S_{n}\right\}$ son exponenciales negativas.

\item Si $\left\{S_{n}\right\}$ son exponenciales negativas, $\left\{D_{n}\right\}$ es un proceso de renovaci\'on  si y s\'olo si es un proceso Poisson, lo cual ocurre si y s\'olo si $\left\{T_{n}\right\}$ es un proceso Poisson.

\item $\esp\left(D_{n}\right)=\esp\left(T_{n}\right)$.

\item Para un sistema de visitas $GI/M/1$ se tiene el siguiente teorema:

\begin{Teo}
En un sistema estacionario $GI/M/1$ los intervalos de interpartida tienen
\begin{eqnarray*}
\esp\left(e^{-\theta D_{n}}\right)&=&\mu\left(\mu+\theta\right)^{-1}\left[\delta\theta
-\mu\left(1-\delta\right)\alpha\left(\theta\right)\right]
\left[\theta-\mu\left(1-\delta\right)^{-1}\right]\\
\alpha\left(\theta\right)&=&\esp\left[e^{-\theta T_{0}}\right]\\
var\left(D_{n}\right)&=&var\left(T_{0}\right)-\left(\tau^{-1}-\delta^{-1}\right)
2\delta\left(\esp\left(S_{0}\right)\right)^{2}\left(1-\delta\right)^{-1}.
\end{eqnarray*}
\end{Teo}



\begin{Teo}
El proceso de salida de un sistema de colas estacionario $GI/M/1$ es un proceso de renovaci\'on si y s\'olo si el proceso de entrada es un proceso Poisson, en cuyo caso el proceso de salida es un proceso Poisson.
\end{Teo}


\begin{Teo}
Los intervalos de interpartida $\left\{D_{n}\right\}$ de un sistema $M/G/1$ estacionario son no correlacionados si y s\'olo si la distribuci\'on de los tiempos de servicio es exponencial negativa, es decir, el sistema es de tipo  $M/M/1$.

\end{Teo}



\end{itemize}


%\section{Resultados para Procesos de Salida}

En Sigman, Thorison y Wolff \cite{Sigman2} prueban que para la existencia de un una sucesi\'on infinita no decreciente de tiempos de regeneraci\'on $\tau_{1}\leq\tau_{2}\leq\cdots$ en los cuales el proceso se regenera, basta un tiempo de regeneraci\'on $R_{1}$, donde $R_{j}=\tau_{j}-\tau_{j-1}$. Para tal efecto se requiere la existencia de un espacio de probabilidad $\left(\Omega,\mathcal{F},\prob\right)$, y proceso estoc\'astico $\textit{X}=\left\{X\left(t\right):t\geq0\right\}$ con espacio de estados $\left(S,\mathcal{R}\right)$, con $\mathcal{R}$ $\sigma$-\'algebra.

\begin{Prop}
Si existe una variable aleatoria no negativa $R_{1}$ tal que $\theta_{R\footnotesize{1}}X=_{D}X$, entonces $\left(\Omega,\mathcal{F},\prob\right)$ puede extenderse para soportar una sucesi\'on estacionaria de variables aleatorias $R=\left\{R_{k}:k\geq1\right\}$, tal que para $k\geq1$,
\begin{eqnarray*}
\theta_{k}\left(X,R\right)=_{D}\left(X,R\right).
\end{eqnarray*}

Adem\'as, para $k\geq1$, $\theta_{k}R$ es condicionalmente independiente de $\left(X,R_{1},\ldots,R_{k}\right)$, dado $\theta_{\tau k}X$.

\end{Prop}


\begin{itemize}
\item Doob en 1953 demostr\'o que el estado estacionario de un proceso de partida en un sistema de espera $M/G/\infty$, es Poisson con la misma tasa que el proceso de arribos.

\item Burke en 1968, fue el primero en demostrar que el estado estacionario de un proceso de salida de una cola $M/M/s$ es un proceso Poisson.

\item Disney en 1973 obtuvo el siguiente resultado:

\begin{Teo}
Para el sistema de espera $M/G/1/L$ con disciplina FIFO, el proceso $\textbf{I}$ es un proceso de renovaci\'on si y s\'olo si el proceso denominado longitud de la cola es estacionario y se cumple cualquiera de los siguientes casos:

\begin{itemize}
\item[a)] Los tiempos de servicio son identicamente cero;
\item[b)] $L=0$, para cualquier proceso de servicio $S$;
\item[c)] $L=1$ y $G=D$;
\item[d)] $L=\infty$ y $G=M$.
\end{itemize}
En estos casos, respectivamente, las distribuciones de interpartida $P\left\{T_{n+1}-T_{n}\leq t\right\}$ son


\begin{itemize}
\item[a)] $1-e^{-\lambda t}$, $t\geq0$;
\item[b)] $1-e^{-\lambda t}*F\left(t\right)$, $t\geq0$;
\item[c)] $1-e^{-\lambda t}*\indora_{d}\left(t\right)$, $t\geq0$;
\item[d)] $1-e^{-\lambda t}*F\left(t\right)$, $t\geq0$.
\end{itemize}
\end{Teo}


\item Finch (1959) mostr\'o que para los sistemas $M/G/1/L$, con $1\leq L\leq \infty$ con distribuciones de servicio dos veces diferenciable, solamente el sistema $M/M/1/\infty$ tiene proceso de salida de renovaci\'on estacionario.

\item King (1971) demostro que un sistema de colas estacionario $M/G/1/1$ tiene sus tiempos de interpartida sucesivas $D_{n}$ y $D_{n+1}$ son independientes, si y s\'olo si, $G=D$, en cuyo caso le proceso de salida es de renovaci\'on.

\item Disney (1973) demostr\'o que el \'unico sistema estacionario $M/G/1/L$, que tiene proceso de salida de renovaci\'on  son los sistemas $M/M/1$ y $M/D/1/1$.



\item El siguiente resultado es de Disney y Koning (1985)
\begin{Teo}
En un sistema de espera $M/G/s$, el estado estacionario del proceso de salida es un proceso Poisson para cualquier distribuci\'on de los tiempos de servicio si el sistema tiene cualquiera de las siguientes cuatro propiedades.

\begin{itemize}
\item[a)] $s=\infty$
\item[b)] La disciplina de servicio es de procesador compartido.
\item[c)] La disciplina de servicio es LCFS y preemptive resume, esto se cumple para $L<\infty$
\item[d)] $G=M$.
\end{itemize}

\end{Teo}

\item El siguiente resultado es de Alamatsaz (1983)

\begin{Teo}
En cualquier sistema de colas $GI/G/1/L$ con $1\leq L<\infty$ y distribuci\'on de interarribos $A$ y distribuci\'on de los tiempos de servicio $B$, tal que $A\left(0\right)=0$, $A\left(t\right)\left(1-B\left(t\right)\right)>0$ para alguna $t>0$ y $B\left(t\right)$ para toda $t>0$, es imposible que el proceso de salida estacionario sea de renovaci\'on.
\end{Teo}

\end{itemize}

Estos resultados aparecen en Daley (1968) \cite{Daley68} para $\left\{T_{n}\right\}$ intervalos de inter-arribo, $\left\{D_{n}\right\}$ intervalos de inter-salida y $\left\{S_{n}\right\}$ tiempos de servicio.

\begin{itemize}
\item Si el proceso $\left\{T_{n}\right\}$ es Poisson, el proceso $\left\{D_{n}\right\}$ es no correlacionado si y s\'olo si es un proceso Poisso, lo cual ocurre si y s\'olo si $\left\{S_{n}\right\}$ son exponenciales negativas.

\item Si $\left\{S_{n}\right\}$ son exponenciales negativas, $\left\{D_{n}\right\}$ es un proceso de renovaci\'on  si y s\'olo si es un proceso Poisson, lo cual ocurre si y s\'olo si $\left\{T_{n}\right\}$ es un proceso Poisson.

\item $\esp\left(D_{n}\right)=\esp\left(T_{n}\right)$.

\item Para un sistema de visitas $GI/M/1$ se tiene el siguiente teorema:

\begin{Teo}
En un sistema estacionario $GI/M/1$ los intervalos de interpartida tienen
\begin{eqnarray*}
\esp\left(e^{-\theta D_{n}}\right)&=&\mu\left(\mu+\theta\right)^{-1}\left[\delta\theta
-\mu\left(1-\delta\right)\alpha\left(\theta\right)\right]
\left[\theta-\mu\left(1-\delta\right)^{-1}\right]\\
\alpha\left(\theta\right)&=&\esp\left[e^{-\theta T_{0}}\right]\\
var\left(D_{n}\right)&=&var\left(T_{0}\right)-\left(\tau^{-1}-\delta^{-1}\right)
2\delta\left(\esp\left(S_{0}\right)\right)^{2}\left(1-\delta\right)^{-1}.
\end{eqnarray*}
\end{Teo}



\begin{Teo}
El proceso de salida de un sistema de colas estacionario $GI/M/1$ es un proceso de renovaci\'on si y s\'olo si el proceso de entrada es un proceso Poisson, en cuyo caso el proceso de salida es un proceso Poisson.
\end{Teo}


\begin{Teo}
Los intervalos de interpartida $\left\{D_{n}\right\}$ de un sistema $M/G/1$ estacionario son no correlacionados si y s\'olo si la distribuci\'on de los tiempos de servicio es exponencial negativa, es decir, el sistema es de tipo  $M/M/1$.

\end{Teo}



\end{itemize}
%\newpage
%________________________________________________________________________
%\section{Redes de Sistemas de Visitas C\'iclicas}
%________________________________________________________________________

Sean $Q_{1},Q_{2},Q_{3}$ y $Q_{4}$ en una Red de Sistemas de Visitas C\'iclicas (RSVC). Supongamos que cada una de las colas es del tipo $M/M/1$ con tasa de arribo $\mu_{i}$ y que la transferencia de usuarios entre los dos sistemas ocurre entre $Q_{1}\leftrightarrow Q_{3}$ y $Q_{2}\leftrightarrow Q_{4}$ con respectiva tasa de arribo igual a la tasa de salida $\hat{\mu}_{i}=\mu_{i}$, esto se sabe por lo desarrollado en la secci\'on anterior.  

Consideremos, sin p\'erdida de generalidad como base del an\'alisis, la cola $Q_{1}$ adem\'as supongamos al servidor lo comenzamos a observar una vez que termina de atender a la misma para desplazarse y llegar a $Q_{2}$, es decir al tiempo $\tau_{2}$.

Sea $n\in\nat$, $n>0$, ciclo del servidor en que regresa a $Q_{1}$ para dar servicio y atender conforme a la pol\'itica exhaustiva, entonces se tiene que $\overline{\tau}_{1}\left(n\right)$ es el tiempo del servidor en el sistema 1 en que termina de dar servicio a todos los usuarios presentes en la cola, por lo tanto se cumple que $L_{1}\left(\overline{\tau}_{1}\left(n\right)\right)=0$, entonces el servidor para llegar a $Q_{2}$ incurre en un tiempo de traslado $r_{1}$ y por tanto se cumple que $\tau_{2}\left(n\right)=\overline{\tau}_{1}\left(n\right)+r_{1}$. Dado que los tiempos entre arribos son exponenciales se cumple que 

\begin{eqnarray*}
\prob\left\{0 \textrm{ arribos en }Q_{1}\textrm{ en el intervalo }\left[\overline{\tau}_{1}\left(n\right),\overline{\tau}_{1}\left(n\right)+r_{1}\right]\right\}=e^{-\tilde{\mu}_{1}r_{1}},\\
\prob\left\{0 \textrm{ arribos en }Q_{2}\textrm{ en el intervalo }\left[\overline{\tau}_{1}\left(n\right),\overline{\tau}_{1}\left(n\right)+r_{1}\right]\right\}=e^{-\tilde{\mu}_{2}r_{1}}.
\end{eqnarray*}

El evento que nos interesa consiste en que no haya arribos desde que el servidor abandon\'o $Q_{2}$ y regresa nuevamente para dar servicio, es decir en el intervalo de tiempo $\left[\overline{\tau}_{2}\left(n-1\right),\tau_{2}\left(n\right)\right]$. Entonces, si hacemos


\begin{eqnarray*}
\varphi_{1}\left(n\right)&\equiv&\overline{\tau}_{1}\left(n\right)+r_{1}=\overline{\tau}_{2}\left(n-1\right)+r_{1}+r_{2}+\overline{\tau}_{1}\left(n\right)-\tau_{1}\left(n\right)\\
&=&\overline{\tau}_{2}\left(n-1\right)+\overline{\tau}_{1}\left(n\right)-\tau_{1}\left(n\right)+r,,
\end{eqnarray*}

y longitud del intervalo

\begin{eqnarray*}
\xi&\equiv&\overline{\tau}_{1}\left(n\right)+r_{1}-\overline{\tau}_{2}\left(n-1\right)
=\overline{\tau}_{2}\left(n-1\right)+\overline{\tau}_{1}\left(n\right)-\tau_{1}\left(n\right)+r-\overline{\tau}_{2}\left(n-1\right)\\
&=&\overline{\tau}_{1}\left(n\right)-\tau_{1}\left(n\right)+r.
\end{eqnarray*}


Entonces, determinemos la probabilidad del evento no arribos a $Q_{2}$ en $\left[\overline{\tau}_{2}\left(n-1\right),\varphi_{1}\left(n\right)\right]$:

\begin{eqnarray}
\prob\left\{0 \textrm{ arribos en }Q_{2}\textrm{ en el intervalo }\left[\overline{\tau}_{2}\left(n-1\right),\varphi_{1}\left(n\right)\right]\right\}
=e^{-\tilde{\mu}_{2}\xi}.
\end{eqnarray}

De manera an\'aloga, tenemos que la probabilidad de no arribos a $Q_{1}$ en $\left[\overline{\tau}_{2}\left(n-1\right),\varphi_{1}\left(n\right)\right]$ esta dada por

\begin{eqnarray}
\prob\left\{0 \textrm{ arribos en }Q_{1}\textrm{ en el intervalo }\left[\overline{\tau}_{2}\left(n-1\right),\varphi_{1}\left(n\right)\right]\right\}
=e^{-\tilde{\mu}_{1}\xi},
\end{eqnarray}

\begin{eqnarray}
\prob\left\{0 \textrm{ arribos en }Q_{2}\textrm{ en el intervalo }\left[\overline{\tau}_{2}\left(n-1\right),\varphi_{1}\left(n\right)\right]\right\}
=e^{-\tilde{\mu}_{2}\xi}.
\end{eqnarray}

Por tanto 

\begin{eqnarray}
\begin{array}{l}
\prob\left\{0 \textrm{ arribos en }Q_{1}\textrm{ y }Q_{2}\textrm{ en el intervalo }\left[\overline{\tau}_{2}\left(n-1\right),\varphi_{1}\left(n\right)\right]\right\}\\
=\prob\left\{0 \textrm{ arribos en }Q_{1}\textrm{ en el intervalo }\left[\overline{\tau}_{2}\left(n-1\right),\varphi_{1}\left(n\right)\right]\right\}\\
\times
\prob\left\{0 \textrm{ arribos en }Q_{2}\textrm{ en el intervalo }\left[\overline{\tau}_{2}\left(n-1\right),\varphi_{1}\left(n\right)\right]\right\}=e^{-\tilde{\mu}_{1}\xi}e^{-\tilde{\mu}_{2}\xi}
=e^{-\tilde{\mu}\xi}.
\end{array}
\end{eqnarray}

Para el segundo sistema, consideremos nuevamente $\overline{\tau}_{1}\left(n\right)+r_{1}$, sin p\'erdida de generalidad podemos suponer que existe $m>0$ tal que $\overline{\tau}_{3}\left(m\right)<\overline{\tau}_{1}+r_{1}<\tau_{4}\left(m\right)$, entonces

\begin{eqnarray}
\prob\left\{0 \textrm{ arribos en }Q_{3}\textrm{ en el intervalo }\left[\overline{\tau}_{3}\left(m\right),\overline{\tau}_{1}\left(n\right)+r_{1}\right]\right\}
=e^{-\tilde{\mu}_{3}\xi_{3}},
\end{eqnarray}
donde 
\begin{eqnarray}
\xi_{3}=\overline{\tau}_{1}\left(n\right)+r_{1}-\overline{\tau}_{3}\left(m\right)=
\overline{\tau}_{1}\left(n\right)-\overline{\tau}_{3}\left(m\right)+r_{1},
\end{eqnarray}

mientras que para $Q_{4}$ al igual que con $Q_{2}$ escribiremos $\tau_{4}\left(m\right)$ en t\'erminos de $\overline{\tau}_{4}\left(m-1\right)$:

$\varphi_{2}\equiv\tau_{4}\left(m\right)=\overline{\tau}_{4}\left(m-1\right)+r_{4}+\overline{\tau}_{3}\left(m\right)
-\tau_{3}\left(m\right)+r_{3}=\overline{\tau}_{4}\left(m-1\right)+\overline{\tau}_{3}\left(m\right)
-\tau_{3}\left(m\right)+\hat{r}$, adem\'as,

$\xi_{2}\equiv\varphi_{2}\left(m\right)-\overline{\tau}_{1}-r_{1}=\overline{\tau}_{4}\left(m-1\right)+\overline{\tau}_{3}\left(m\right)
-\tau_{3}\left(m\right)-\overline{\tau}_{1}\left(n\right)+\hat{r}-r_{1}$. 

Entonces


\begin{eqnarray}
\prob\left\{0 \textrm{ arribos en }Q_{4}\textrm{ en el intervalo }\left[\overline{\tau}_{1}\left(n\right)+r_{1},\varphi_{2}\left(m\right)\right]\right\}
=e^{-\tilde{\mu}_{4}\xi_{2}},
\end{eqnarray}

mientras que para $Q_{3}$ se tiene que 

\begin{eqnarray}
\prob\left\{0 \textrm{ arribos en }Q_{3}\textrm{ en el intervalo }\left[\overline{\tau}_{1}\left(n\right)+r_{1},\varphi_{2}\left(m\right)\right]\right\}
=e^{-\tilde{\mu}_{3}\xi_{2}}
\end{eqnarray}

Por tanto

\begin{eqnarray}
\prob\left\{0 \textrm{ arribos en }Q_{3}\wedge Q_{4}\textrm{ en el intervalo }\left[\overline{\tau}_{1}\left(n\right)+r_{1},\varphi_{2}\left(m\right)\right]\right\}
=e^{-\hat{\mu}\xi_{2}}
\end{eqnarray}
donde $\hat{\mu}=\tilde{\mu}_{3}+\tilde{\mu}_{4}$.

Ahora, definamos los intervalos $\mathcal{I}_{1}=\left[\overline{\tau}_{1}\left(n\right)+r_{1},\varphi_{1}\left(n\right)\right]$  y $\mathcal{I}_{2}=\left[\overline{\tau}_{1}\left(n\right)+r_{1},\varphi_{2}\left(m\right)\right]$, entonces, sea $\mathcal{I}=\mathcal{I}_{1}\cap\mathcal{I}_{2}$ el intervalo donde cada una de las colas se encuentran vac\'ias, es decir, si tomamos $T^{*}\in\mathcal{I}$, entonces  $L_{1}\left(T^{*}\right)=L_{2}\left(T^{*}\right)=L_{3}\left(T^{*}\right)=L_{4}\left(T^{*}\right)=0$.

Ahora, dado que por construcci\'on $\mathcal{I}\neq\emptyset$ y que para $T^{*}\in\mathcal{I}$ en ninguna de las colas han llegado usuarios, se tiene que no hay transferencia entre las colas, por lo tanto, el sistema 1 y el sistema 2 son condicionalmente independientes en $\mathcal{I}$, es decir

\begin{eqnarray}
\prob\left\{L_{1}\left(T^{*}\right),L_{2}\left(T^{*}\right),L_{3}\left(T^{*}\right),L_{4}\left(T^{*}\right)|T^{*}\in\mathcal{I}\right\}=\prod_{j=1}^{4}\prob\left\{L_{j}\left(T^{*}\right)\right\},
\end{eqnarray}

para $T^{*}\in\mathcal{I}$. 

%\newpage























%________________________________________________________________________
%\section{Procesos Regenerativos}
%________________________________________________________________________

%________________________________________________________________________
%\subsection*{Procesos Regenerativos Sigman, Thorisson y Wolff \cite{Sigman1}}
%________________________________________________________________________


\begin{Def}[Definici\'on Cl\'asica]
Un proceso estoc\'astico $X=\left\{X\left(t\right):t\geq0\right\}$ es llamado regenerativo is existe una variable aleatoria $R_{1}>0$ tal que
\begin{itemize}
\item[i)] $\left\{X\left(t+R_{1}\right):t\geq0\right\}$ es independiente de $\left\{\left\{X\left(t\right):t<R_{1}\right\},\right\}$
\item[ii)] $\left\{X\left(t+R_{1}\right):t\geq0\right\}$ es estoc\'asticamente equivalente a $\left\{X\left(t\right):t>0\right\}$
\end{itemize}

Llamamos a $R_{1}$ tiempo de regeneraci\'on, y decimos que $X$ se regenera en este punto.
\end{Def}

$\left\{X\left(t+R_{1}\right)\right\}$ es regenerativo con tiempo de regeneraci\'on $R_{2}$, independiente de $R_{1}$ pero con la misma distribuci\'on que $R_{1}$. Procediendo de esta manera se obtiene una secuencia de variables aleatorias independientes e id\'enticamente distribuidas $\left\{R_{n}\right\}$ llamados longitudes de ciclo. Si definimos a $Z_{k}\equiv R_{1}+R_{2}+\cdots+R_{k}$, se tiene un proceso de renovaci\'on llamado proceso de renovaci\'on encajado para $X$.


\begin{Note}
La existencia de un primer tiempo de regeneraci\'on, $R_{1}$, implica la existencia de una sucesi\'on completa de estos tiempos $R_{1},R_{2}\ldots,$ que satisfacen la propiedad deseada \cite{Sigman2}.
\end{Note}


\begin{Note} Para la cola $GI/GI/1$ los usuarios arriban con tiempos $t_{n}$ y son atendidos con tiempos de servicio $S_{n}$, los tiempos de arribo forman un proceso de renovaci\'on  con tiempos entre arribos independientes e identicamente distribuidos (\texttt{i.i.d.})$T_{n}=t_{n}-t_{n-1}$, adem\'as los tiempos de servicio son \texttt{i.i.d.} e independientes de los procesos de arribo. Por \textit{estable} se entiende que $\esp S_{n}<\esp T_{n}<\infty$.
\end{Note}
 


\begin{Def}
Para $x$ fijo y para cada $t\geq0$, sea $I_{x}\left(t\right)=1$ si $X\left(t\right)\leq x$,  $I_{x}\left(t\right)=0$ en caso contrario, y def\'inanse los tiempos promedio
\begin{eqnarray*}
\overline{X}&=&lim_{t\rightarrow\infty}\frac{1}{t}\int_{0}^{\infty}X\left(u\right)du\\
\prob\left(X_{\infty}\leq x\right)&=&lim_{t\rightarrow\infty}\frac{1}{t}\int_{0}^{\infty}I_{x}\left(u\right)du,
\end{eqnarray*}
cuando estos l\'imites existan.
\end{Def}

Como consecuencia del teorema de Renovaci\'on-Recompensa, se tiene que el primer l\'imite  existe y es igual a la constante
\begin{eqnarray*}
\overline{X}&=&\frac{\esp\left[\int_{0}^{R_{1}}X\left(t\right)dt\right]}{\esp\left[R_{1}\right]},
\end{eqnarray*}
suponiendo que ambas esperanzas son finitas.
 
\begin{Note}
Funciones de procesos regenerativos son regenerativas, es decir, si $X\left(t\right)$ es regenerativo y se define el proceso $Y\left(t\right)$ por $Y\left(t\right)=f\left(X\left(t\right)\right)$ para alguna funci\'on Borel medible $f\left(\cdot\right)$. Adem\'as $Y$ es regenerativo con los mismos tiempos de renovaci\'on que $X$. 

En general, los tiempos de renovaci\'on, $Z_{k}$ de un proceso regenerativo no requieren ser tiempos de paro con respecto a la evoluci\'on de $X\left(t\right)$.
\end{Note} 

\begin{Note}
Una funci\'on de un proceso de Markov, usualmente no ser\'a un proceso de Markov, sin embargo ser\'a regenerativo si el proceso de Markov lo es.
\end{Note}

 
\begin{Note}
Un proceso regenerativo con media de la longitud de ciclo finita es llamado positivo recurrente.
\end{Note}


\begin{Note}
\begin{itemize}
\item[a)] Si el proceso regenerativo $X$ es positivo recurrente y tiene trayectorias muestrales no negativas, entonces la ecuaci\'on anterior es v\'alida.
\item[b)] Si $X$ es positivo recurrente regenerativo, podemos construir una \'unica versi\'on estacionaria de este proceso, $X_{e}=\left\{X_{e}\left(t\right)\right\}$, donde $X_{e}$ es un proceso estoc\'astico regenerativo y estrictamente estacionario, con distribuci\'on marginal distribuida como $X_{\infty}$
\end{itemize}
\end{Note}


%__________________________________________________________________________________________
%\subsection*{Procesos Regenerativos Estacionarios - Stidham \cite{Stidham}}
%__________________________________________________________________________________________


Un proceso estoc\'astico a tiempo continuo $\left\{V\left(t\right),t\geq0\right\}$ es un proceso regenerativo si existe una sucesi\'on de variables aleatorias independientes e id\'enticamente distribuidas $\left\{X_{1},X_{2},\ldots\right\}$, sucesi\'on de renovaci\'on, tal que para cualquier conjunto de Borel $A$, 

\begin{eqnarray*}
\prob\left\{V\left(t\right)\in A|X_{1}+X_{2}+\cdots+X_{R\left(t\right)}=s,\left\{V\left(\tau\right),\tau<s\right\}\right\}=\prob\left\{V\left(t-s\right)\in A|X_{1}>t-s\right\},
\end{eqnarray*}
para todo $0\leq s\leq t$, donde $R\left(t\right)=\max\left\{X_{1}+X_{2}+\cdots+X_{j}\leq t\right\}=$n\'umero de renovaciones ({\emph{puntos de regeneraci\'on}}) que ocurren en $\left[0,t\right]$. El intervalo $\left[0,X_{1}\right)$ es llamado {\emph{primer ciclo de regeneraci\'on}} de $\left\{V\left(t \right),t\geq0\right\}$, $\left[X_{1},X_{1}+X_{2}\right)$ el {\emph{segundo ciclo de regeneraci\'on}}, y as\'i sucesivamente.

Sea $X=X_{1}$ y sea $F$ la funci\'on de distrbuci\'on de $X$


\begin{Def}
Se define el proceso estacionario, $\left\{V^{*}\left(t\right),t\geq0\right\}$, para $\left\{V\left(t\right),t\geq0\right\}$ por

\begin{eqnarray*}
\prob\left\{V\left(t\right)\in A\right\}=\frac{1}{\esp\left[X\right]}\int_{0}^{\infty}\prob\left\{V\left(t+x\right)\in A|X>x\right\}\left(1-F\left(x\right)\right)dx,
\end{eqnarray*} 
para todo $t\geq0$ y todo conjunto de Borel $A$.
\end{Def}

\begin{Def}
Una distribuci\'on se dice que es {\emph{aritm\'etica}} si todos sus puntos de incremento son m\'ultiplos de la forma $0,\lambda, 2\lambda,\ldots$ para alguna $\lambda>0$ entera.
\end{Def}


\begin{Def}
Una modificaci\'on medible de un proceso $\left\{V\left(t\right),t\geq0\right\}$, es una versi\'on de este, $\left\{V\left(t,w\right)\right\}$ conjuntamente medible para $t\geq0$ y para $w\in S$, $S$ espacio de estados para $\left\{V\left(t\right),t\geq0\right\}$.
\end{Def}

\begin{Teo}
Sea $\left\{V\left(t\right),t\geq\right\}$ un proceso regenerativo no negativo con modificaci\'on medible. Sea $\esp\left[X\right]<\infty$. Entonces el proceso estacionario dado por la ecuaci\'on anterior est\'a bien definido y tiene funci\'on de distribuci\'on independiente de $t$, adem\'as
\begin{itemize}
\item[i)] \begin{eqnarray*}
\esp\left[V^{*}\left(0\right)\right]&=&\frac{\esp\left[\int_{0}^{X}V\left(s\right)ds\right]}{\esp\left[X\right]}\end{eqnarray*}
\item[ii)] Si $\esp\left[V^{*}\left(0\right)\right]<\infty$, equivalentemente, si $\esp\left[\int_{0}^{X}V\left(s\right)ds\right]<\infty$,entonces
\begin{eqnarray*}
\frac{\int_{0}^{t}V\left(s\right)ds}{t}\rightarrow\frac{\esp\left[\int_{0}^{X}V\left(s\right)ds\right]}{\esp\left[X\right]}
\end{eqnarray*}
con probabilidad 1 y en media, cuando $t\rightarrow\infty$.
\end{itemize}
\end{Teo}

\begin{Coro}
Sea $\left\{V\left(t\right),t\geq0\right\}$ un proceso regenerativo no negativo, con modificaci\'on medible. Si $\esp <\infty$, $F$ es no-aritm\'etica, y para todo $x\geq0$, $P\left\{V\left(t\right)\leq x,C>x\right\}$ es de variaci\'on acotada como funci\'on de $t$ en cada intervalo finito $\left[0,\tau\right]$, entonces $V\left(t\right)$ converge en distribuci\'on  cuando $t\rightarrow\infty$ y $$\esp V=\frac{\esp \int_{0}^{X}V\left(s\right)ds}{\esp X}$$
Donde $V$ tiene la distribuci\'on l\'imite de $V\left(t\right)$ cuando $t\rightarrow\infty$.

\end{Coro}

Para el caso discreto se tienen resultados similares.



%______________________________________________________________________
%\section{Procesos de Renovaci\'on}
%______________________________________________________________________

\begin{Def}\label{Def.Tn}
Sean $0\leq T_{1}\leq T_{2}\leq \ldots$ son tiempos aleatorios infinitos en los cuales ocurren ciertos eventos. El n\'umero de tiempos $T_{n}$ en el intervalo $\left[0,t\right)$ es

\begin{eqnarray}
N\left(t\right)=\sum_{n=1}^{\infty}\indora\left(T_{n}\leq t\right),
\end{eqnarray}
para $t\geq0$.
\end{Def}

Si se consideran los puntos $T_{n}$ como elementos de $\rea_{+}$, y $N\left(t\right)$ es el n\'umero de puntos en $\rea$. El proceso denotado por $\left\{N\left(t\right):t\geq0\right\}$, denotado por $N\left(t\right)$, es un proceso puntual en $\rea_{+}$. Los $T_{n}$ son los tiempos de ocurrencia, el proceso puntual $N\left(t\right)$ es simple si su n\'umero de ocurrencias son distintas: $0<T_{1}<T_{2}<\ldots$ casi seguramente.

\begin{Def}
Un proceso puntual $N\left(t\right)$ es un proceso de renovaci\'on si los tiempos de interocurrencia $\xi_{n}=T_{n}-T_{n-1}$, para $n\geq1$, son independientes e identicamente distribuidos con distribuci\'on $F$, donde $F\left(0\right)=0$ y $T_{0}=0$. Los $T_{n}$ son llamados tiempos de renovaci\'on, referente a la independencia o renovaci\'on de la informaci\'on estoc\'astica en estos tiempos. Los $\xi_{n}$ son los tiempos de inter-renovaci\'on, y $N\left(t\right)$ es el n\'umero de renovaciones en el intervalo $\left[0,t\right)$
\end{Def}


\begin{Note}
Para definir un proceso de renovaci\'on para cualquier contexto, solamente hay que especificar una distribuci\'on $F$, con $F\left(0\right)=0$, para los tiempos de inter-renovaci\'on. La funci\'on $F$ en turno degune las otra variables aleatorias. De manera formal, existe un espacio de probabilidad y una sucesi\'on de variables aleatorias $\xi_{1},\xi_{2},\ldots$ definidas en este con distribuci\'on $F$. Entonces las otras cantidades son $T_{n}=\sum_{k=1}^{n}\xi_{k}$ y $N\left(t\right)=\sum_{n=1}^{\infty}\indora\left(T_{n}\leq t\right)$, donde $T_{n}\rightarrow\infty$ casi seguramente por la Ley Fuerte de los Grandes Números.
\end{Note}

%___________________________________________________________________________________________
%
%\subsection*{Teorema Principal de Renovaci\'on}
%___________________________________________________________________________________________
%

\begin{Note} Una funci\'on $h:\rea_{+}\rightarrow\rea$ es Directamente Riemann Integrable en los siguientes casos:
\begin{itemize}
\item[a)] $h\left(t\right)\geq0$ es decreciente y Riemann Integrable.
\item[b)] $h$ es continua excepto posiblemente en un conjunto de Lebesgue de medida 0, y $|h\left(t\right)|\leq b\left(t\right)$, donde $b$ es DRI.
\end{itemize}
\end{Note}

\begin{Teo}[Teorema Principal de Renovaci\'on]
Si $F$ es no aritm\'etica y $h\left(t\right)$ es Directamente Riemann Integrable (DRI), entonces

\begin{eqnarray*}
lim_{t\rightarrow\infty}U\star h=\frac{1}{\mu}\int_{\rea_{+}}h\left(s\right)ds.
\end{eqnarray*}
\end{Teo}

\begin{Prop}
Cualquier funci\'on $H\left(t\right)$ acotada en intervalos finitos y que es 0 para $t<0$ puede expresarse como
\begin{eqnarray*}
H\left(t\right)=U\star h\left(t\right)\textrm{,  donde }h\left(t\right)=H\left(t\right)-F\star H\left(t\right)
\end{eqnarray*}
\end{Prop}

\begin{Def}
Un proceso estoc\'astico $X\left(t\right)$ es crudamente regenerativo en un tiempo aleatorio positivo $T$ si
\begin{eqnarray*}
\esp\left[X\left(T+t\right)|T\right]=\esp\left[X\left(t\right)\right]\textrm{, para }t\geq0,\end{eqnarray*}
y con las esperanzas anteriores finitas.
\end{Def}

\begin{Prop}
Sup\'ongase que $X\left(t\right)$ es un proceso crudamente regenerativo en $T$, que tiene distribuci\'on $F$. Si $\esp\left[X\left(t\right)\right]$ es acotado en intervalos finitos, entonces
\begin{eqnarray*}
\esp\left[X\left(t\right)\right]=U\star h\left(t\right)\textrm{,  donde }h\left(t\right)=\esp\left[X\left(t\right)\indora\left(T>t\right)\right].
\end{eqnarray*}
\end{Prop}

\begin{Teo}[Regeneraci\'on Cruda]
Sup\'ongase que $X\left(t\right)$ es un proceso con valores positivo crudamente regenerativo en $T$, y def\'inase $M=\sup\left\{|X\left(t\right)|:t\leq T\right\}$. Si $T$ es no aritm\'etico y $M$ y $MT$ tienen media finita, entonces
\begin{eqnarray*}
lim_{t\rightarrow\infty}\esp\left[X\left(t\right)\right]=\frac{1}{\mu}\int_{\rea_{+}}h\left(s\right)ds,
\end{eqnarray*}
donde $h\left(t\right)=\esp\left[X\left(t\right)\indora\left(T>t\right)\right]$.
\end{Teo}

%___________________________________________________________________________________________
%
%\subsection*{Propiedades de los Procesos de Renovaci\'on}
%___________________________________________________________________________________________
%

Los tiempos $T_{n}$ est\'an relacionados con los conteos de $N\left(t\right)$ por

\begin{eqnarray*}
\left\{N\left(t\right)\geq n\right\}&=&\left\{T_{n}\leq t\right\}\\
T_{N\left(t\right)}\leq &t&<T_{N\left(t\right)+1},
\end{eqnarray*}

adem\'as $N\left(T_{n}\right)=n$, y 

\begin{eqnarray*}
N\left(t\right)=\max\left\{n:T_{n}\leq t\right\}=\min\left\{n:T_{n+1}>t\right\}
\end{eqnarray*}

Por propiedades de la convoluci\'on se sabe que

\begin{eqnarray*}
P\left\{T_{n}\leq t\right\}=F^{n\star}\left(t\right)
\end{eqnarray*}
que es la $n$-\'esima convoluci\'on de $F$. Entonces 

\begin{eqnarray*}
\left\{N\left(t\right)\geq n\right\}&=&\left\{T_{n}\leq t\right\}\\
P\left\{N\left(t\right)\leq n\right\}&=&1-F^{\left(n+1\right)\star}\left(t\right)
\end{eqnarray*}

Adem\'as usando el hecho de que $\esp\left[N\left(t\right)\right]=\sum_{n=1}^{\infty}P\left\{N\left(t\right)\geq n\right\}$
se tiene que

\begin{eqnarray*}
\esp\left[N\left(t\right)\right]=\sum_{n=1}^{\infty}F^{n\star}\left(t\right)
\end{eqnarray*}

\begin{Prop}
Para cada $t\geq0$, la funci\'on generadora de momentos $\esp\left[e^{\alpha N\left(t\right)}\right]$ existe para alguna $\alpha$ en una vecindad del 0, y de aqu\'i que $\esp\left[N\left(t\right)^{m}\right]<\infty$, para $m\geq1$.
\end{Prop}


\begin{Note}
Si el primer tiempo de renovaci\'on $\xi_{1}$ no tiene la misma distribuci\'on que el resto de las $\xi_{n}$, para $n\geq2$, a $N\left(t\right)$ se le llama Proceso de Renovaci\'on retardado, donde si $\xi$ tiene distribuci\'on $G$, entonces el tiempo $T_{n}$ de la $n$-\'esima renovaci\'on tiene distribuci\'on $G\star F^{\left(n-1\right)\star}\left(t\right)$
\end{Note}


\begin{Teo}
Para una constante $\mu\leq\infty$ ( o variable aleatoria), las siguientes expresiones son equivalentes:

\begin{eqnarray}
lim_{n\rightarrow\infty}n^{-1}T_{n}&=&\mu,\textrm{ c.s.}\\
lim_{t\rightarrow\infty}t^{-1}N\left(t\right)&=&1/\mu,\textrm{ c.s.}
\end{eqnarray}
\end{Teo}


Es decir, $T_{n}$ satisface la Ley Fuerte de los Grandes N\'umeros s\'i y s\'olo s\'i $N\left/t\right)$ la cumple.


\begin{Coro}[Ley Fuerte de los Grandes N\'umeros para Procesos de Renovaci\'on]
Si $N\left(t\right)$ es un proceso de renovaci\'on cuyos tiempos de inter-renovaci\'on tienen media $\mu\leq\infty$, entonces
\begin{eqnarray}
t^{-1}N\left(t\right)\rightarrow 1/\mu,\textrm{ c.s. cuando }t\rightarrow\infty.
\end{eqnarray}

\end{Coro}


Considerar el proceso estoc\'astico de valores reales $\left\{Z\left(t\right):t\geq0\right\}$ en el mismo espacio de probabilidad que $N\left(t\right)$

\begin{Def}
Para el proceso $\left\{Z\left(t\right):t\geq0\right\}$ se define la fluctuaci\'on m\'axima de $Z\left(t\right)$ en el intervalo $\left(T_{n-1},T_{n}\right]$:
\begin{eqnarray*}
M_{n}=\sup_{T_{n-1}<t\leq T_{n}}|Z\left(t\right)-Z\left(T_{n-1}\right)|
\end{eqnarray*}
\end{Def}

\begin{Teo}
Sup\'ongase que $n^{-1}T_{n}\rightarrow\mu$ c.s. cuando $n\rightarrow\infty$, donde $\mu\leq\infty$ es una constante o variable aleatoria. Sea $a$ una constante o variable aleatoria que puede ser infinita cuando $\mu$ es finita, y considere las expresiones l\'imite:
\begin{eqnarray}
lim_{n\rightarrow\infty}n^{-1}Z\left(T_{n}\right)&=&a,\textrm{ c.s.}\\
lim_{t\rightarrow\infty}t^{-1}Z\left(t\right)&=&a/\mu,\textrm{ c.s.}
\end{eqnarray}
La segunda expresi\'on implica la primera. Conversamente, la primera implica la segunda si el proceso $Z\left(t\right)$ es creciente, o si $lim_{n\rightarrow\infty}n^{-1}M_{n}=0$ c.s.
\end{Teo}

\begin{Coro}
Si $N\left(t\right)$ es un proceso de renovaci\'on, y $\left(Z\left(T_{n}\right)-Z\left(T_{n-1}\right),M_{n}\right)$, para $n\geq1$, son variables aleatorias independientes e id\'enticamente distribuidas con media finita, entonces,
\begin{eqnarray}
lim_{t\rightarrow\infty}t^{-1}Z\left(t\right)\rightarrow\frac{\esp\left[Z\left(T_{1}\right)-Z\left(T_{0}\right)\right]}{\esp\left[T_{1}\right]},\textrm{ c.s. cuando  }t\rightarrow\infty.
\end{eqnarray}
\end{Coro}



%___________________________________________________________________________________________
%
%\subsection{Propiedades de los Procesos de Renovaci\'on}
%___________________________________________________________________________________________
%

Los tiempos $T_{n}$ est\'an relacionados con los conteos de $N\left(t\right)$ por

\begin{eqnarray*}
\left\{N\left(t\right)\geq n\right\}&=&\left\{T_{n}\leq t\right\}\\
T_{N\left(t\right)}\leq &t&<T_{N\left(t\right)+1},
\end{eqnarray*}

adem\'as $N\left(T_{n}\right)=n$, y 

\begin{eqnarray*}
N\left(t\right)=\max\left\{n:T_{n}\leq t\right\}=\min\left\{n:T_{n+1}>t\right\}
\end{eqnarray*}

Por propiedades de la convoluci\'on se sabe que

\begin{eqnarray*}
P\left\{T_{n}\leq t\right\}=F^{n\star}\left(t\right)
\end{eqnarray*}
que es la $n$-\'esima convoluci\'on de $F$. Entonces 

\begin{eqnarray*}
\left\{N\left(t\right)\geq n\right\}&=&\left\{T_{n}\leq t\right\}\\
P\left\{N\left(t\right)\leq n\right\}&=&1-F^{\left(n+1\right)\star}\left(t\right)
\end{eqnarray*}

Adem\'as usando el hecho de que $\esp\left[N\left(t\right)\right]=\sum_{n=1}^{\infty}P\left\{N\left(t\right)\geq n\right\}$
se tiene que

\begin{eqnarray*}
\esp\left[N\left(t\right)\right]=\sum_{n=1}^{\infty}F^{n\star}\left(t\right)
\end{eqnarray*}

\begin{Prop}
Para cada $t\geq0$, la funci\'on generadora de momentos $\esp\left[e^{\alpha N\left(t\right)}\right]$ existe para alguna $\alpha$ en una vecindad del 0, y de aqu\'i que $\esp\left[N\left(t\right)^{m}\right]<\infty$, para $m\geq1$.
\end{Prop}


\begin{Note}
Si el primer tiempo de renovaci\'on $\xi_{1}$ no tiene la misma distribuci\'on que el resto de las $\xi_{n}$, para $n\geq2$, a $N\left(t\right)$ se le llama Proceso de Renovaci\'on retardado, donde si $\xi$ tiene distribuci\'on $G$, entonces el tiempo $T_{n}$ de la $n$-\'esima renovaci\'on tiene distribuci\'on $G\star F^{\left(n-1\right)\star}\left(t\right)$
\end{Note}


\begin{Teo}
Para una constante $\mu\leq\infty$ ( o variable aleatoria), las siguientes expresiones son equivalentes:

\begin{eqnarray}
lim_{n\rightarrow\infty}n^{-1}T_{n}&=&\mu,\textrm{ c.s.}\\
lim_{t\rightarrow\infty}t^{-1}N\left(t\right)&=&1/\mu,\textrm{ c.s.}
\end{eqnarray}
\end{Teo}


Es decir, $T_{n}$ satisface la Ley Fuerte de los Grandes N\'umeros s\'i y s\'olo s\'i $N\left/t\right)$ la cumple.


\begin{Coro}[Ley Fuerte de los Grandes N\'umeros para Procesos de Renovaci\'on]
Si $N\left(t\right)$ es un proceso de renovaci\'on cuyos tiempos de inter-renovaci\'on tienen media $\mu\leq\infty$, entonces
\begin{eqnarray}
t^{-1}N\left(t\right)\rightarrow 1/\mu,\textrm{ c.s. cuando }t\rightarrow\infty.
\end{eqnarray}

\end{Coro}


Considerar el proceso estoc\'astico de valores reales $\left\{Z\left(t\right):t\geq0\right\}$ en el mismo espacio de probabilidad que $N\left(t\right)$

\begin{Def}
Para el proceso $\left\{Z\left(t\right):t\geq0\right\}$ se define la fluctuaci\'on m\'axima de $Z\left(t\right)$ en el intervalo $\left(T_{n-1},T_{n}\right]$:
\begin{eqnarray*}
M_{n}=\sup_{T_{n-1}<t\leq T_{n}}|Z\left(t\right)-Z\left(T_{n-1}\right)|
\end{eqnarray*}
\end{Def}

\begin{Teo}
Sup\'ongase que $n^{-1}T_{n}\rightarrow\mu$ c.s. cuando $n\rightarrow\infty$, donde $\mu\leq\infty$ es una constante o variable aleatoria. Sea $a$ una constante o variable aleatoria que puede ser infinita cuando $\mu$ es finita, y considere las expresiones l\'imite:
\begin{eqnarray}
lim_{n\rightarrow\infty}n^{-1}Z\left(T_{n}\right)&=&a,\textrm{ c.s.}\\
lim_{t\rightarrow\infty}t^{-1}Z\left(t\right)&=&a/\mu,\textrm{ c.s.}
\end{eqnarray}
La segunda expresi\'on implica la primera. Conversamente, la primera implica la segunda si el proceso $Z\left(t\right)$ es creciente, o si $lim_{n\rightarrow\infty}n^{-1}M_{n}=0$ c.s.
\end{Teo}

\begin{Coro}
Si $N\left(t\right)$ es un proceso de renovaci\'on, y $\left(Z\left(T_{n}\right)-Z\left(T_{n-1}\right),M_{n}\right)$, para $n\geq1$, son variables aleatorias independientes e id\'enticamente distribuidas con media finita, entonces,
\begin{eqnarray}
lim_{t\rightarrow\infty}t^{-1}Z\left(t\right)\rightarrow\frac{\esp\left[Z\left(T_{1}\right)-Z\left(T_{0}\right)\right]}{\esp\left[T_{1}\right]},\textrm{ c.s. cuando  }t\rightarrow\infty.
\end{eqnarray}
\end{Coro}


%___________________________________________________________________________________________
%
%\subsection{Propiedades de los Procesos de Renovaci\'on}
%___________________________________________________________________________________________
%

Los tiempos $T_{n}$ est\'an relacionados con los conteos de $N\left(t\right)$ por

\begin{eqnarray*}
\left\{N\left(t\right)\geq n\right\}&=&\left\{T_{n}\leq t\right\}\\
T_{N\left(t\right)}\leq &t&<T_{N\left(t\right)+1},
\end{eqnarray*}

adem\'as $N\left(T_{n}\right)=n$, y 

\begin{eqnarray*}
N\left(t\right)=\max\left\{n:T_{n}\leq t\right\}=\min\left\{n:T_{n+1}>t\right\}
\end{eqnarray*}

Por propiedades de la convoluci\'on se sabe que

\begin{eqnarray*}
P\left\{T_{n}\leq t\right\}=F^{n\star}\left(t\right)
\end{eqnarray*}
que es la $n$-\'esima convoluci\'on de $F$. Entonces 

\begin{eqnarray*}
\left\{N\left(t\right)\geq n\right\}&=&\left\{T_{n}\leq t\right\}\\
P\left\{N\left(t\right)\leq n\right\}&=&1-F^{\left(n+1\right)\star}\left(t\right)
\end{eqnarray*}

Adem\'as usando el hecho de que $\esp\left[N\left(t\right)\right]=\sum_{n=1}^{\infty}P\left\{N\left(t\right)\geq n\right\}$
se tiene que

\begin{eqnarray*}
\esp\left[N\left(t\right)\right]=\sum_{n=1}^{\infty}F^{n\star}\left(t\right)
\end{eqnarray*}

\begin{Prop}
Para cada $t\geq0$, la funci\'on generadora de momentos $\esp\left[e^{\alpha N\left(t\right)}\right]$ existe para alguna $\alpha$ en una vecindad del 0, y de aqu\'i que $\esp\left[N\left(t\right)^{m}\right]<\infty$, para $m\geq1$.
\end{Prop}


\begin{Note}
Si el primer tiempo de renovaci\'on $\xi_{1}$ no tiene la misma distribuci\'on que el resto de las $\xi_{n}$, para $n\geq2$, a $N\left(t\right)$ se le llama Proceso de Renovaci\'on retardado, donde si $\xi$ tiene distribuci\'on $G$, entonces el tiempo $T_{n}$ de la $n$-\'esima renovaci\'on tiene distribuci\'on $G\star F^{\left(n-1\right)\star}\left(t\right)$
\end{Note}


\begin{Teo}
Para una constante $\mu\leq\infty$ ( o variable aleatoria), las siguientes expresiones son equivalentes:

\begin{eqnarray}
lim_{n\rightarrow\infty}n^{-1}T_{n}&=&\mu,\textrm{ c.s.}\\
lim_{t\rightarrow\infty}t^{-1}N\left(t\right)&=&1/\mu,\textrm{ c.s.}
\end{eqnarray}
\end{Teo}


Es decir, $T_{n}$ satisface la Ley Fuerte de los Grandes N\'umeros s\'i y s\'olo s\'i $N\left/t\right)$ la cumple.


\begin{Coro}[Ley Fuerte de los Grandes N\'umeros para Procesos de Renovaci\'on]
Si $N\left(t\right)$ es un proceso de renovaci\'on cuyos tiempos de inter-renovaci\'on tienen media $\mu\leq\infty$, entonces
\begin{eqnarray}
t^{-1}N\left(t\right)\rightarrow 1/\mu,\textrm{ c.s. cuando }t\rightarrow\infty.
\end{eqnarray}

\end{Coro}


Considerar el proceso estoc\'astico de valores reales $\left\{Z\left(t\right):t\geq0\right\}$ en el mismo espacio de probabilidad que $N\left(t\right)$

\begin{Def}
Para el proceso $\left\{Z\left(t\right):t\geq0\right\}$ se define la fluctuaci\'on m\'axima de $Z\left(t\right)$ en el intervalo $\left(T_{n-1},T_{n}\right]$:
\begin{eqnarray*}
M_{n}=\sup_{T_{n-1}<t\leq T_{n}}|Z\left(t\right)-Z\left(T_{n-1}\right)|
\end{eqnarray*}
\end{Def}

\begin{Teo}
Sup\'ongase que $n^{-1}T_{n}\rightarrow\mu$ c.s. cuando $n\rightarrow\infty$, donde $\mu\leq\infty$ es una constante o variable aleatoria. Sea $a$ una constante o variable aleatoria que puede ser infinita cuando $\mu$ es finita, y considere las expresiones l\'imite:
\begin{eqnarray}
lim_{n\rightarrow\infty}n^{-1}Z\left(T_{n}\right)&=&a,\textrm{ c.s.}\\
lim_{t\rightarrow\infty}t^{-1}Z\left(t\right)&=&a/\mu,\textrm{ c.s.}
\end{eqnarray}
La segunda expresi\'on implica la primera. Conversamente, la primera implica la segunda si el proceso $Z\left(t\right)$ es creciente, o si $lim_{n\rightarrow\infty}n^{-1}M_{n}=0$ c.s.
\end{Teo}

\begin{Coro}
Si $N\left(t\right)$ es un proceso de renovaci\'on, y $\left(Z\left(T_{n}\right)-Z\left(T_{n-1}\right),M_{n}\right)$, para $n\geq1$, son variables aleatorias independientes e id\'enticamente distribuidas con media finita, entonces,
\begin{eqnarray}
lim_{t\rightarrow\infty}t^{-1}Z\left(t\right)\rightarrow\frac{\esp\left[Z\left(T_{1}\right)-Z\left(T_{0}\right)\right]}{\esp\left[T_{1}\right]},\textrm{ c.s. cuando  }t\rightarrow\infty.
\end{eqnarray}
\end{Coro}

%___________________________________________________________________________________________
%
%\subsection{Propiedades de los Procesos de Renovaci\'on}
%___________________________________________________________________________________________
%

Los tiempos $T_{n}$ est\'an relacionados con los conteos de $N\left(t\right)$ por

\begin{eqnarray*}
\left\{N\left(t\right)\geq n\right\}&=&\left\{T_{n}\leq t\right\}\\
T_{N\left(t\right)}\leq &t&<T_{N\left(t\right)+1},
\end{eqnarray*}

adem\'as $N\left(T_{n}\right)=n$, y 

\begin{eqnarray*}
N\left(t\right)=\max\left\{n:T_{n}\leq t\right\}=\min\left\{n:T_{n+1}>t\right\}
\end{eqnarray*}

Por propiedades de la convoluci\'on se sabe que

\begin{eqnarray*}
P\left\{T_{n}\leq t\right\}=F^{n\star}\left(t\right)
\end{eqnarray*}
que es la $n$-\'esima convoluci\'on de $F$. Entonces 

\begin{eqnarray*}
\left\{N\left(t\right)\geq n\right\}&=&\left\{T_{n}\leq t\right\}\\
P\left\{N\left(t\right)\leq n\right\}&=&1-F^{\left(n+1\right)\star}\left(t\right)
\end{eqnarray*}

Adem\'as usando el hecho de que $\esp\left[N\left(t\right)\right]=\sum_{n=1}^{\infty}P\left\{N\left(t\right)\geq n\right\}$
se tiene que

\begin{eqnarray*}
\esp\left[N\left(t\right)\right]=\sum_{n=1}^{\infty}F^{n\star}\left(t\right)
\end{eqnarray*}

\begin{Prop}
Para cada $t\geq0$, la funci\'on generadora de momentos $\esp\left[e^{\alpha N\left(t\right)}\right]$ existe para alguna $\alpha$ en una vecindad del 0, y de aqu\'i que $\esp\left[N\left(t\right)^{m}\right]<\infty$, para $m\geq1$.
\end{Prop}


\begin{Note}
Si el primer tiempo de renovaci\'on $\xi_{1}$ no tiene la misma distribuci\'on que el resto de las $\xi_{n}$, para $n\geq2$, a $N\left(t\right)$ se le llama Proceso de Renovaci\'on retardado, donde si $\xi$ tiene distribuci\'on $G$, entonces el tiempo $T_{n}$ de la $n$-\'esima renovaci\'on tiene distribuci\'on $G\star F^{\left(n-1\right)\star}\left(t\right)$
\end{Note}


\begin{Teo}
Para una constante $\mu\leq\infty$ ( o variable aleatoria), las siguientes expresiones son equivalentes:

\begin{eqnarray}
lim_{n\rightarrow\infty}n^{-1}T_{n}&=&\mu,\textrm{ c.s.}\\
lim_{t\rightarrow\infty}t^{-1}N\left(t\right)&=&1/\mu,\textrm{ c.s.}
\end{eqnarray}
\end{Teo}


Es decir, $T_{n}$ satisface la Ley Fuerte de los Grandes N\'umeros s\'i y s\'olo s\'i $N\left/t\right)$ la cumple.


\begin{Coro}[Ley Fuerte de los Grandes N\'umeros para Procesos de Renovaci\'on]
Si $N\left(t\right)$ es un proceso de renovaci\'on cuyos tiempos de inter-renovaci\'on tienen media $\mu\leq\infty$, entonces
\begin{eqnarray}
t^{-1}N\left(t\right)\rightarrow 1/\mu,\textrm{ c.s. cuando }t\rightarrow\infty.
\end{eqnarray}

\end{Coro}


Considerar el proceso estoc\'astico de valores reales $\left\{Z\left(t\right):t\geq0\right\}$ en el mismo espacio de probabilidad que $N\left(t\right)$

\begin{Def}
Para el proceso $\left\{Z\left(t\right):t\geq0\right\}$ se define la fluctuaci\'on m\'axima de $Z\left(t\right)$ en el intervalo $\left(T_{n-1},T_{n}\right]$:
\begin{eqnarray*}
M_{n}=\sup_{T_{n-1}<t\leq T_{n}}|Z\left(t\right)-Z\left(T_{n-1}\right)|
\end{eqnarray*}
\end{Def}

\begin{Teo}
Sup\'ongase que $n^{-1}T_{n}\rightarrow\mu$ c.s. cuando $n\rightarrow\infty$, donde $\mu\leq\infty$ es una constante o variable aleatoria. Sea $a$ una constante o variable aleatoria que puede ser infinita cuando $\mu$ es finita, y considere las expresiones l\'imite:
\begin{eqnarray}
lim_{n\rightarrow\infty}n^{-1}Z\left(T_{n}\right)&=&a,\textrm{ c.s.}\\
lim_{t\rightarrow\infty}t^{-1}Z\left(t\right)&=&a/\mu,\textrm{ c.s.}
\end{eqnarray}
La segunda expresi\'on implica la primera. Conversamente, la primera implica la segunda si el proceso $Z\left(t\right)$ es creciente, o si $lim_{n\rightarrow\infty}n^{-1}M_{n}=0$ c.s.
\end{Teo}

\begin{Coro}
Si $N\left(t\right)$ es un proceso de renovaci\'on, y $\left(Z\left(T_{n}\right)-Z\left(T_{n-1}\right),M_{n}\right)$, para $n\geq1$, son variables aleatorias independientes e id\'enticamente distribuidas con media finita, entonces,
\begin{eqnarray}
lim_{t\rightarrow\infty}t^{-1}Z\left(t\right)\rightarrow\frac{\esp\left[Z\left(T_{1}\right)-Z\left(T_{0}\right)\right]}{\esp\left[T_{1}\right]},\textrm{ c.s. cuando  }t\rightarrow\infty.
\end{eqnarray}
\end{Coro}
%___________________________________________________________________________________________
%
%\subsection{Propiedades de los Procesos de Renovaci\'on}
%___________________________________________________________________________________________
%

Los tiempos $T_{n}$ est\'an relacionados con los conteos de $N\left(t\right)$ por

\begin{eqnarray*}
\left\{N\left(t\right)\geq n\right\}&=&\left\{T_{n}\leq t\right\}\\
T_{N\left(t\right)}\leq &t&<T_{N\left(t\right)+1},
\end{eqnarray*}

adem\'as $N\left(T_{n}\right)=n$, y 

\begin{eqnarray*}
N\left(t\right)=\max\left\{n:T_{n}\leq t\right\}=\min\left\{n:T_{n+1}>t\right\}
\end{eqnarray*}

Por propiedades de la convoluci\'on se sabe que

\begin{eqnarray*}
P\left\{T_{n}\leq t\right\}=F^{n\star}\left(t\right)
\end{eqnarray*}
que es la $n$-\'esima convoluci\'on de $F$. Entonces 

\begin{eqnarray*}
\left\{N\left(t\right)\geq n\right\}&=&\left\{T_{n}\leq t\right\}\\
P\left\{N\left(t\right)\leq n\right\}&=&1-F^{\left(n+1\right)\star}\left(t\right)
\end{eqnarray*}

Adem\'as usando el hecho de que $\esp\left[N\left(t\right)\right]=\sum_{n=1}^{\infty}P\left\{N\left(t\right)\geq n\right\}$
se tiene que

\begin{eqnarray*}
\esp\left[N\left(t\right)\right]=\sum_{n=1}^{\infty}F^{n\star}\left(t\right)
\end{eqnarray*}

\begin{Prop}
Para cada $t\geq0$, la funci\'on generadora de momentos $\esp\left[e^{\alpha N\left(t\right)}\right]$ existe para alguna $\alpha$ en una vecindad del 0, y de aqu\'i que $\esp\left[N\left(t\right)^{m}\right]<\infty$, para $m\geq1$.
\end{Prop}


\begin{Note}
Si el primer tiempo de renovaci\'on $\xi_{1}$ no tiene la misma distribuci\'on que el resto de las $\xi_{n}$, para $n\geq2$, a $N\left(t\right)$ se le llama Proceso de Renovaci\'on retardado, donde si $\xi$ tiene distribuci\'on $G$, entonces el tiempo $T_{n}$ de la $n$-\'esima renovaci\'on tiene distribuci\'on $G\star F^{\left(n-1\right)\star}\left(t\right)$
\end{Note}


\begin{Teo}
Para una constante $\mu\leq\infty$ ( o variable aleatoria), las siguientes expresiones son equivalentes:

\begin{eqnarray}
lim_{n\rightarrow\infty}n^{-1}T_{n}&=&\mu,\textrm{ c.s.}\\
lim_{t\rightarrow\infty}t^{-1}N\left(t\right)&=&1/\mu,\textrm{ c.s.}
\end{eqnarray}
\end{Teo}


Es decir, $T_{n}$ satisface la Ley Fuerte de los Grandes N\'umeros s\'i y s\'olo s\'i $N\left/t\right)$ la cumple.


\begin{Coro}[Ley Fuerte de los Grandes N\'umeros para Procesos de Renovaci\'on]
Si $N\left(t\right)$ es un proceso de renovaci\'on cuyos tiempos de inter-renovaci\'on tienen media $\mu\leq\infty$, entonces
\begin{eqnarray}
t^{-1}N\left(t\right)\rightarrow 1/\mu,\textrm{ c.s. cuando }t\rightarrow\infty.
\end{eqnarray}

\end{Coro}


Considerar el proceso estoc\'astico de valores reales $\left\{Z\left(t\right):t\geq0\right\}$ en el mismo espacio de probabilidad que $N\left(t\right)$

\begin{Def}
Para el proceso $\left\{Z\left(t\right):t\geq0\right\}$ se define la fluctuaci\'on m\'axima de $Z\left(t\right)$ en el intervalo $\left(T_{n-1},T_{n}\right]$:
\begin{eqnarray*}
M_{n}=\sup_{T_{n-1}<t\leq T_{n}}|Z\left(t\right)-Z\left(T_{n-1}\right)|
\end{eqnarray*}
\end{Def}

\begin{Teo}
Sup\'ongase que $n^{-1}T_{n}\rightarrow\mu$ c.s. cuando $n\rightarrow\infty$, donde $\mu\leq\infty$ es una constante o variable aleatoria. Sea $a$ una constante o variable aleatoria que puede ser infinita cuando $\mu$ es finita, y considere las expresiones l\'imite:
\begin{eqnarray}
lim_{n\rightarrow\infty}n^{-1}Z\left(T_{n}\right)&=&a,\textrm{ c.s.}\\
lim_{t\rightarrow\infty}t^{-1}Z\left(t\right)&=&a/\mu,\textrm{ c.s.}
\end{eqnarray}
La segunda expresi\'on implica la primera. Conversamente, la primera implica la segunda si el proceso $Z\left(t\right)$ es creciente, o si $lim_{n\rightarrow\infty}n^{-1}M_{n}=0$ c.s.
\end{Teo}

\begin{Coro}
Si $N\left(t\right)$ es un proceso de renovaci\'on, y $\left(Z\left(T_{n}\right)-Z\left(T_{n-1}\right),M_{n}\right)$, para $n\geq1$, son variables aleatorias independientes e id\'enticamente distribuidas con media finita, entonces,
\begin{eqnarray}
lim_{t\rightarrow\infty}t^{-1}Z\left(t\right)\rightarrow\frac{\esp\left[Z\left(T_{1}\right)-Z\left(T_{0}\right)\right]}{\esp\left[T_{1}\right]},\textrm{ c.s. cuando  }t\rightarrow\infty.
\end{eqnarray}
\end{Coro}


%___________________________________________________________________________________________
%
%\subsection*{Funci\'on de Renovaci\'on}
%___________________________________________________________________________________________
%


\begin{Def}
Sea $h\left(t\right)$ funci\'on de valores reales en $\rea$ acotada en intervalos finitos e igual a cero para $t<0$ La ecuaci\'on de renovaci\'on para $h\left(t\right)$ y la distribuci\'on $F$ es

\begin{eqnarray}\label{Ec.Renovacion}
H\left(t\right)=h\left(t\right)+\int_{\left[0,t\right]}H\left(t-s\right)dF\left(s\right)\textrm{,    }t\geq0,
\end{eqnarray}
donde $H\left(t\right)$ es una funci\'on de valores reales. Esto es $H=h+F\star H$. Decimos que $H\left(t\right)$ es soluci\'on de esta ecuaci\'on si satisface la ecuaci\'on, y es acotada en intervalos finitos e iguales a cero para $t<0$.
\end{Def}

\begin{Prop}
La funci\'on $U\star h\left(t\right)$ es la \'unica soluci\'on de la ecuaci\'on de renovaci\'on (\ref{Ec.Renovacion}).
\end{Prop}

\begin{Teo}[Teorema Renovaci\'on Elemental]
\begin{eqnarray*}
t^{-1}U\left(t\right)\rightarrow 1/\mu\textrm{,    cuando }t\rightarrow\infty.
\end{eqnarray*}
\end{Teo}

%___________________________________________________________________________________________
%
%\subsection{Funci\'on de Renovaci\'on}
%___________________________________________________________________________________________
%


Sup\'ongase que $N\left(t\right)$ es un proceso de renovaci\'on con distribuci\'on $F$ con media finita $\mu$.

\begin{Def}
La funci\'on de renovaci\'on asociada con la distribuci\'on $F$, del proceso $N\left(t\right)$, es
\begin{eqnarray*}
U\left(t\right)=\sum_{n=1}^{\infty}F^{n\star}\left(t\right),\textrm{   }t\geq0,
\end{eqnarray*}
donde $F^{0\star}\left(t\right)=\indora\left(t\geq0\right)$.
\end{Def}


\begin{Prop}
Sup\'ongase que la distribuci\'on de inter-renovaci\'on $F$ tiene densidad $f$. Entonces $U\left(t\right)$ tambi\'en tiene densidad, para $t>0$, y es $U^{'}\left(t\right)=\sum_{n=0}^{\infty}f^{n\star}\left(t\right)$. Adem\'as
\begin{eqnarray*}
\prob\left\{N\left(t\right)>N\left(t-\right)\right\}=0\textrm{,   }t\geq0.
\end{eqnarray*}
\end{Prop}

\begin{Def}
La Transformada de Laplace-Stieljes de $F$ est\'a dada por

\begin{eqnarray*}
\hat{F}\left(\alpha\right)=\int_{\rea_{+}}e^{-\alpha t}dF\left(t\right)\textrm{,  }\alpha\geq0.
\end{eqnarray*}
\end{Def}

Entonces

\begin{eqnarray*}
\hat{U}\left(\alpha\right)=\sum_{n=0}^{\infty}\hat{F^{n\star}}\left(\alpha\right)=\sum_{n=0}^{\infty}\hat{F}\left(\alpha\right)^{n}=\frac{1}{1-\hat{F}\left(\alpha\right)}.
\end{eqnarray*}


\begin{Prop}
La Transformada de Laplace $\hat{U}\left(\alpha\right)$ y $\hat{F}\left(\alpha\right)$ determina una a la otra de manera \'unica por la relaci\'on $\hat{U}\left(\alpha\right)=\frac{1}{1-\hat{F}\left(\alpha\right)}$.
\end{Prop}


\begin{Note}
Un proceso de renovaci\'on $N\left(t\right)$ cuyos tiempos de inter-renovaci\'on tienen media finita, es un proceso Poisson con tasa $\lambda$ si y s\'olo s\'i $\esp\left[U\left(t\right)\right]=\lambda t$, para $t\geq0$.
\end{Note}


\begin{Teo}
Sea $N\left(t\right)$ un proceso puntual simple con puntos de localizaci\'on $T_{n}$ tal que $\eta\left(t\right)=\esp\left[N\left(\right)\right]$ es finita para cada $t$. Entonces para cualquier funci\'on $f:\rea_{+}\rightarrow\rea$,
\begin{eqnarray*}
\esp\left[\sum_{n=1}^{N\left(\right)}f\left(T_{n}\right)\right]=\int_{\left(0,t\right]}f\left(s\right)d\eta\left(s\right)\textrm{,  }t\geq0,
\end{eqnarray*}
suponiendo que la integral exista. Adem\'as si $X_{1},X_{2},\ldots$ son variables aleatorias definidas en el mismo espacio de probabilidad que el proceso $N\left(t\right)$ tal que $\esp\left[X_{n}|T_{n}=s\right]=f\left(s\right)$, independiente de $n$. Entonces
\begin{eqnarray*}
\esp\left[\sum_{n=1}^{N\left(t\right)}X_{n}\right]=\int_{\left(0,t\right]}f\left(s\right)d\eta\left(s\right)\textrm{,  }t\geq0,
\end{eqnarray*} 
suponiendo que la integral exista. 
\end{Teo}

\begin{Coro}[Identidad de Wald para Renovaciones]
Para el proceso de renovaci\'on $N\left(t\right)$,
\begin{eqnarray*}
\esp\left[T_{N\left(t\right)+1}\right]=\mu\esp\left[N\left(t\right)+1\right]\textrm{,  }t\geq0,
\end{eqnarray*}  
\end{Coro}

%______________________________________________________________________
%\subsection{Procesos de Renovaci\'on}
%______________________________________________________________________

\begin{Def}\label{Def.Tn}
Sean $0\leq T_{1}\leq T_{2}\leq \ldots$ son tiempos aleatorios infinitos en los cuales ocurren ciertos eventos. El n\'umero de tiempos $T_{n}$ en el intervalo $\left[0,t\right)$ es

\begin{eqnarray}
N\left(t\right)=\sum_{n=1}^{\infty}\indora\left(T_{n}\leq t\right),
\end{eqnarray}
para $t\geq0$.
\end{Def}

Si se consideran los puntos $T_{n}$ como elementos de $\rea_{+}$, y $N\left(t\right)$ es el n\'umero de puntos en $\rea$. El proceso denotado por $\left\{N\left(t\right):t\geq0\right\}$, denotado por $N\left(t\right)$, es un proceso puntual en $\rea_{+}$. Los $T_{n}$ son los tiempos de ocurrencia, el proceso puntual $N\left(t\right)$ es simple si su n\'umero de ocurrencias son distintas: $0<T_{1}<T_{2}<\ldots$ casi seguramente.

\begin{Def}
Un proceso puntual $N\left(t\right)$ es un proceso de renovaci\'on si los tiempos de interocurrencia $\xi_{n}=T_{n}-T_{n-1}$, para $n\geq1$, son independientes e identicamente distribuidos con distribuci\'on $F$, donde $F\left(0\right)=0$ y $T_{0}=0$. Los $T_{n}$ son llamados tiempos de renovaci\'on, referente a la independencia o renovaci\'on de la informaci\'on estoc\'astica en estos tiempos. Los $\xi_{n}$ son los tiempos de inter-renovaci\'on, y $N\left(t\right)$ es el n\'umero de renovaciones en el intervalo $\left[0,t\right)$
\end{Def}


\begin{Note}
Para definir un proceso de renovaci\'on para cualquier contexto, solamente hay que especificar una distribuci\'on $F$, con $F\left(0\right)=0$, para los tiempos de inter-renovaci\'on. La funci\'on $F$ en turno degune las otra variables aleatorias. De manera formal, existe un espacio de probabilidad y una sucesi\'on de variables aleatorias $\xi_{1},\xi_{2},\ldots$ definidas en este con distribuci\'on $F$. Entonces las otras cantidades son $T_{n}=\sum_{k=1}^{n}\xi_{k}$ y $N\left(t\right)=\sum_{n=1}^{\infty}\indora\left(T_{n}\leq t\right)$, donde $T_{n}\rightarrow\infty$ casi seguramente por la Ley Fuerte de los Grandes Números.
\end{Note}

\begin{Def}\label{Def.Tn}
Sean $0\leq T_{1}\leq T_{2}\leq \ldots$ son tiempos aleatorios infinitos en los cuales ocurren ciertos eventos. El n\'umero de tiempos $T_{n}$ en el intervalo $\left[0,t\right)$ es

\begin{eqnarray}
N\left(t\right)=\sum_{n=1}^{\infty}\indora\left(T_{n}\leq t\right),
\end{eqnarray}
para $t\geq0$.
\end{Def}

Si se consideran los puntos $T_{n}$ como elementos de $\rea_{+}$, y $N\left(t\right)$ es el n\'umero de puntos en $\rea$. El proceso denotado por $\left\{N\left(t\right):t\geq0\right\}$, denotado por $N\left(t\right)$, es un proceso puntual en $\rea_{+}$. Los $T_{n}$ son los tiempos de ocurrencia, el proceso puntual $N\left(t\right)$ es simple si su n\'umero de ocurrencias son distintas: $0<T_{1}<T_{2}<\ldots$ casi seguramente.

\begin{Def}
Un proceso puntual $N\left(t\right)$ es un proceso de renovaci\'on si los tiempos de interocurrencia $\xi_{n}=T_{n}-T_{n-1}$, para $n\geq1$, son independientes e identicamente distribuidos con distribuci\'on $F$, donde $F\left(0\right)=0$ y $T_{0}=0$. Los $T_{n}$ son llamados tiempos de renovaci\'on, referente a la independencia o renovaci\'on de la informaci\'on estoc\'astica en estos tiempos. Los $\xi_{n}$ son los tiempos de inter-renovaci\'on, y $N\left(t\right)$ es el n\'umero de renovaciones en el intervalo $\left[0,t\right)$
\end{Def}


\begin{Note}
Para definir un proceso de renovaci\'on para cualquier contexto, solamente hay que especificar una distribuci\'on $F$, con $F\left(0\right)=0$, para los tiempos de inter-renovaci\'on. La funci\'on $F$ en turno degune las otra variables aleatorias. De manera formal, existe un espacio de probabilidad y una sucesi\'on de variables aleatorias $\xi_{1},\xi_{2},\ldots$ definidas en este con distribuci\'on $F$. Entonces las otras cantidades son $T_{n}=\sum_{k=1}^{n}\xi_{k}$ y $N\left(t\right)=\sum_{n=1}^{\infty}\indora\left(T_{n}\leq t\right)$, donde $T_{n}\rightarrow\infty$ casi seguramente por la Ley Fuerte de los Grandes N\'umeros.
\end{Note}







Los tiempos $T_{n}$ est\'an relacionados con los conteos de $N\left(t\right)$ por

\begin{eqnarray*}
\left\{N\left(t\right)\geq n\right\}&=&\left\{T_{n}\leq t\right\}\\
T_{N\left(t\right)}\leq &t&<T_{N\left(t\right)+1},
\end{eqnarray*}

adem\'as $N\left(T_{n}\right)=n$, y 

\begin{eqnarray*}
N\left(t\right)=\max\left\{n:T_{n}\leq t\right\}=\min\left\{n:T_{n+1}>t\right\}
\end{eqnarray*}

Por propiedades de la convoluci\'on se sabe que

\begin{eqnarray*}
P\left\{T_{n}\leq t\right\}=F^{n\star}\left(t\right)
\end{eqnarray*}
que es la $n$-\'esima convoluci\'on de $F$. Entonces 

\begin{eqnarray*}
\left\{N\left(t\right)\geq n\right\}&=&\left\{T_{n}\leq t\right\}\\
P\left\{N\left(t\right)\leq n\right\}&=&1-F^{\left(n+1\right)\star}\left(t\right)
\end{eqnarray*}

Adem\'as usando el hecho de que $\esp\left[N\left(t\right)\right]=\sum_{n=1}^{\infty}P\left\{N\left(t\right)\geq n\right\}$
se tiene que

\begin{eqnarray*}
\esp\left[N\left(t\right)\right]=\sum_{n=1}^{\infty}F^{n\star}\left(t\right)
\end{eqnarray*}

\begin{Prop}
Para cada $t\geq0$, la funci\'on generadora de momentos $\esp\left[e^{\alpha N\left(t\right)}\right]$ existe para alguna $\alpha$ en una vecindad del 0, y de aqu\'i que $\esp\left[N\left(t\right)^{m}\right]<\infty$, para $m\geq1$.
\end{Prop}

\begin{Ejem}[\textbf{Proceso Poisson}]

Suponga que se tienen tiempos de inter-renovaci\'on \textit{i.i.d.} del proceso de renovaci\'on $N\left(t\right)$ tienen distribuci\'on exponencial $F\left(t\right)=q-e^{-\lambda t}$ con tasa $\lambda$. Entonces $N\left(t\right)$ es un proceso Poisson con tasa $\lambda$.

\end{Ejem}


\begin{Note}
Si el primer tiempo de renovaci\'on $\xi_{1}$ no tiene la misma distribuci\'on que el resto de las $\xi_{n}$, para $n\geq2$, a $N\left(t\right)$ se le llama Proceso de Renovaci\'on retardado, donde si $\xi$ tiene distribuci\'on $G$, entonces el tiempo $T_{n}$ de la $n$-\'esima renovaci\'on tiene distribuci\'on $G\star F^{\left(n-1\right)\star}\left(t\right)$
\end{Note}


\begin{Teo}
Para una constante $\mu\leq\infty$ ( o variable aleatoria), las siguientes expresiones son equivalentes:

\begin{eqnarray}
lim_{n\rightarrow\infty}n^{-1}T_{n}&=&\mu,\textrm{ c.s.}\\
lim_{t\rightarrow\infty}t^{-1}N\left(t\right)&=&1/\mu,\textrm{ c.s.}
\end{eqnarray}
\end{Teo}


Es decir, $T_{n}$ satisface la Ley Fuerte de los Grandes N\'umeros s\'i y s\'olo s\'i $N\left/t\right)$ la cumple.


\begin{Coro}[Ley Fuerte de los Grandes N\'umeros para Procesos de Renovaci\'on]
Si $N\left(t\right)$ es un proceso de renovaci\'on cuyos tiempos de inter-renovaci\'on tienen media $\mu\leq\infty$, entonces
\begin{eqnarray}
t^{-1}N\left(t\right)\rightarrow 1/\mu,\textrm{ c.s. cuando }t\rightarrow\infty.
\end{eqnarray}

\end{Coro}


Considerar el proceso estoc\'astico de valores reales $\left\{Z\left(t\right):t\geq0\right\}$ en el mismo espacio de probabilidad que $N\left(t\right)$

\begin{Def}
Para el proceso $\left\{Z\left(t\right):t\geq0\right\}$ se define la fluctuaci\'on m\'axima de $Z\left(t\right)$ en el intervalo $\left(T_{n-1},T_{n}\right]$:
\begin{eqnarray*}
M_{n}=\sup_{T_{n-1}<t\leq T_{n}}|Z\left(t\right)-Z\left(T_{n-1}\right)|
\end{eqnarray*}
\end{Def}

\begin{Teo}
Sup\'ongase que $n^{-1}T_{n}\rightarrow\mu$ c.s. cuando $n\rightarrow\infty$, donde $\mu\leq\infty$ es una constante o variable aleatoria. Sea $a$ una constante o variable aleatoria que puede ser infinita cuando $\mu$ es finita, y considere las expresiones l\'imite:
\begin{eqnarray}
lim_{n\rightarrow\infty}n^{-1}Z\left(T_{n}\right)&=&a,\textrm{ c.s.}\\
lim_{t\rightarrow\infty}t^{-1}Z\left(t\right)&=&a/\mu,\textrm{ c.s.}
\end{eqnarray}
La segunda expresi\'on implica la primera. Conversamente, la primera implica la segunda si el proceso $Z\left(t\right)$ es creciente, o si $lim_{n\rightarrow\infty}n^{-1}M_{n}=0$ c.s.
\end{Teo}

\begin{Coro}
Si $N\left(t\right)$ es un proceso de renovaci\'on, y $\left(Z\left(T_{n}\right)-Z\left(T_{n-1}\right),M_{n}\right)$, para $n\geq1$, son variables aleatorias independientes e id\'enticamente distribuidas con media finita, entonces,
\begin{eqnarray}
lim_{t\rightarrow\infty}t^{-1}Z\left(t\right)\rightarrow\frac{\esp\left[Z\left(T_{1}\right)-Z\left(T_{0}\right)\right]}{\esp\left[T_{1}\right]},\textrm{ c.s. cuando  }t\rightarrow\infty.
\end{eqnarray}
\end{Coro}


Sup\'ongase que $N\left(t\right)$ es un proceso de renovaci\'on con distribuci\'on $F$ con media finita $\mu$.

\begin{Def}
La funci\'on de renovaci\'on asociada con la distribuci\'on $F$, del proceso $N\left(t\right)$, es
\begin{eqnarray*}
U\left(t\right)=\sum_{n=1}^{\infty}F^{n\star}\left(t\right),\textrm{   }t\geq0,
\end{eqnarray*}
donde $F^{0\star}\left(t\right)=\indora\left(t\geq0\right)$.
\end{Def}


\begin{Prop}
Sup\'ongase que la distribuci\'on de inter-renovaci\'on $F$ tiene densidad $f$. Entonces $U\left(t\right)$ tambi\'en tiene densidad, para $t>0$, y es $U^{'}\left(t\right)=\sum_{n=0}^{\infty}f^{n\star}\left(t\right)$. Adem\'as
\begin{eqnarray*}
\prob\left\{N\left(t\right)>N\left(t-\right)\right\}=0\textrm{,   }t\geq0.
\end{eqnarray*}
\end{Prop}

\begin{Def}
La Transformada de Laplace-Stieljes de $F$ est\'a dada por

\begin{eqnarray*}
\hat{F}\left(\alpha\right)=\int_{\rea_{+}}e^{-\alpha t}dF\left(t\right)\textrm{,  }\alpha\geq0.
\end{eqnarray*}
\end{Def}

Entonces

\begin{eqnarray*}
\hat{U}\left(\alpha\right)=\sum_{n=0}^{\infty}\hat{F^{n\star}}\left(\alpha\right)=\sum_{n=0}^{\infty}\hat{F}\left(\alpha\right)^{n}=\frac{1}{1-\hat{F}\left(\alpha\right)}.
\end{eqnarray*}


\begin{Prop}
La Transformada de Laplace $\hat{U}\left(\alpha\right)$ y $\hat{F}\left(\alpha\right)$ determina una a la otra de manera \'unica por la relaci\'on $\hat{U}\left(\alpha\right)=\frac{1}{1-\hat{F}\left(\alpha\right)}$.
\end{Prop}


\begin{Note}
Un proceso de renovaci\'on $N\left(t\right)$ cuyos tiempos de inter-renovaci\'on tienen media finita, es un proceso Poisson con tasa $\lambda$ si y s\'olo s\'i $\esp\left[U\left(t\right)\right]=\lambda t$, para $t\geq0$.
\end{Note}


\begin{Teo}
Sea $N\left(t\right)$ un proceso puntual simple con puntos de localizaci\'on $T_{n}$ tal que $\eta\left(t\right)=\esp\left[N\left(\right)\right]$ es finita para cada $t$. Entonces para cualquier funci\'on $f:\rea_{+}\rightarrow\rea$,
\begin{eqnarray*}
\esp\left[\sum_{n=1}^{N\left(\right)}f\left(T_{n}\right)\right]=\int_{\left(0,t\right]}f\left(s\right)d\eta\left(s\right)\textrm{,  }t\geq0,
\end{eqnarray*}
suponiendo que la integral exista. Adem\'as si $X_{1},X_{2},\ldots$ son variables aleatorias definidas en el mismo espacio de probabilidad que el proceso $N\left(t\right)$ tal que $\esp\left[X_{n}|T_{n}=s\right]=f\left(s\right)$, independiente de $n$. Entonces
\begin{eqnarray*}
\esp\left[\sum_{n=1}^{N\left(t\right)}X_{n}\right]=\int_{\left(0,t\right]}f\left(s\right)d\eta\left(s\right)\textrm{,  }t\geq0,
\end{eqnarray*} 
suponiendo que la integral exista. 
\end{Teo}

\begin{Coro}[Identidad de Wald para Renovaciones]
Para el proceso de renovaci\'on $N\left(t\right)$,
\begin{eqnarray*}
\esp\left[T_{N\left(t\right)+1}\right]=\mu\esp\left[N\left(t\right)+1\right]\textrm{,  }t\geq0,
\end{eqnarray*}  
\end{Coro}


\begin{Def}
Sea $h\left(t\right)$ funci\'on de valores reales en $\rea$ acotada en intervalos finitos e igual a cero para $t<0$ La ecuaci\'on de renovaci\'on para $h\left(t\right)$ y la distribuci\'on $F$ es

\begin{eqnarray}\label{Ec.Renovacion}
H\left(t\right)=h\left(t\right)+\int_{\left[0,t\right]}H\left(t-s\right)dF\left(s\right)\textrm{,    }t\geq0,
\end{eqnarray}
donde $H\left(t\right)$ es una funci\'on de valores reales. Esto es $H=h+F\star H$. Decimos que $H\left(t\right)$ es soluci\'on de esta ecuaci\'on si satisface la ecuaci\'on, y es acotada en intervalos finitos e iguales a cero para $t<0$.
\end{Def}

\begin{Prop}
La funci\'on $U\star h\left(t\right)$ es la \'unica soluci\'on de la ecuaci\'on de renovaci\'on (\ref{Ec.Renovacion}).
\end{Prop}

\begin{Teo}[Teorema Renovaci\'on Elemental]
\begin{eqnarray*}
t^{-1}U\left(t\right)\rightarrow 1/\mu\textrm{,    cuando }t\rightarrow\infty.
\end{eqnarray*}
\end{Teo}



Sup\'ongase que $N\left(t\right)$ es un proceso de renovaci\'on con distribuci\'on $F$ con media finita $\mu$.

\begin{Def}
La funci\'on de renovaci\'on asociada con la distribuci\'on $F$, del proceso $N\left(t\right)$, es
\begin{eqnarray*}
U\left(t\right)=\sum_{n=1}^{\infty}F^{n\star}\left(t\right),\textrm{   }t\geq0,
\end{eqnarray*}
donde $F^{0\star}\left(t\right)=\indora\left(t\geq0\right)$.
\end{Def}


\begin{Prop}
Sup\'ongase que la distribuci\'on de inter-renovaci\'on $F$ tiene densidad $f$. Entonces $U\left(t\right)$ tambi\'en tiene densidad, para $t>0$, y es $U^{'}\left(t\right)=\sum_{n=0}^{\infty}f^{n\star}\left(t\right)$. Adem\'as
\begin{eqnarray*}
\prob\left\{N\left(t\right)>N\left(t-\right)\right\}=0\textrm{,   }t\geq0.
\end{eqnarray*}
\end{Prop}

\begin{Def}
La Transformada de Laplace-Stieljes de $F$ est\'a dada por

\begin{eqnarray*}
\hat{F}\left(\alpha\right)=\int_{\rea_{+}}e^{-\alpha t}dF\left(t\right)\textrm{,  }\alpha\geq0.
\end{eqnarray*}
\end{Def}

Entonces

\begin{eqnarray*}
\hat{U}\left(\alpha\right)=\sum_{n=0}^{\infty}\hat{F^{n\star}}\left(\alpha\right)=\sum_{n=0}^{\infty}\hat{F}\left(\alpha\right)^{n}=\frac{1}{1-\hat{F}\left(\alpha\right)}.
\end{eqnarray*}


\begin{Prop}
La Transformada de Laplace $\hat{U}\left(\alpha\right)$ y $\hat{F}\left(\alpha\right)$ determina una a la otra de manera \'unica por la relaci\'on $\hat{U}\left(\alpha\right)=\frac{1}{1-\hat{F}\left(\alpha\right)}$.
\end{Prop}


\begin{Note}
Un proceso de renovaci\'on $N\left(t\right)$ cuyos tiempos de inter-renovaci\'on tienen media finita, es un proceso Poisson con tasa $\lambda$ si y s\'olo s\'i $\esp\left[U\left(t\right)\right]=\lambda t$, para $t\geq0$.
\end{Note}


\begin{Teo}
Sea $N\left(t\right)$ un proceso puntual simple con puntos de localizaci\'on $T_{n}$ tal que $\eta\left(t\right)=\esp\left[N\left(\right)\right]$ es finita para cada $t$. Entonces para cualquier funci\'on $f:\rea_{+}\rightarrow\rea$,
\begin{eqnarray*}
\esp\left[\sum_{n=1}^{N\left(\right)}f\left(T_{n}\right)\right]=\int_{\left(0,t\right]}f\left(s\right)d\eta\left(s\right)\textrm{,  }t\geq0,
\end{eqnarray*}
suponiendo que la integral exista. Adem\'as si $X_{1},X_{2},\ldots$ son variables aleatorias definidas en el mismo espacio de probabilidad que el proceso $N\left(t\right)$ tal que $\esp\left[X_{n}|T_{n}=s\right]=f\left(s\right)$, independiente de $n$. Entonces
\begin{eqnarray*}
\esp\left[\sum_{n=1}^{N\left(t\right)}X_{n}\right]=\int_{\left(0,t\right]}f\left(s\right)d\eta\left(s\right)\textrm{,  }t\geq0,
\end{eqnarray*} 
suponiendo que la integral exista. 
\end{Teo}

\begin{Coro}[Identidad de Wald para Renovaciones]
Para el proceso de renovaci\'on $N\left(t\right)$,
\begin{eqnarray*}
\esp\left[T_{N\left(t\right)+1}\right]=\mu\esp\left[N\left(t\right)+1\right]\textrm{,  }t\geq0,
\end{eqnarray*}  
\end{Coro}


\begin{Def}
Sea $h\left(t\right)$ funci\'on de valores reales en $\rea$ acotada en intervalos finitos e igual a cero para $t<0$ La ecuaci\'on de renovaci\'on para $h\left(t\right)$ y la distribuci\'on $F$ es

\begin{eqnarray}\label{Ec.Renovacion}
H\left(t\right)=h\left(t\right)+\int_{\left[0,t\right]}H\left(t-s\right)dF\left(s\right)\textrm{,    }t\geq0,
\end{eqnarray}
donde $H\left(t\right)$ es una funci\'on de valores reales. Esto es $H=h+F\star H$. Decimos que $H\left(t\right)$ es soluci\'on de esta ecuaci\'on si satisface la ecuaci\'on, y es acotada en intervalos finitos e iguales a cero para $t<0$.
\end{Def}

\begin{Prop}
La funci\'on $U\star h\left(t\right)$ es la \'unica soluci\'on de la ecuaci\'on de renovaci\'on (\ref{Ec.Renovacion}).
\end{Prop}

\begin{Teo}[Teorema Renovaci\'on Elemental]
\begin{eqnarray*}
t^{-1}U\left(t\right)\rightarrow 1/\mu\textrm{,    cuando }t\rightarrow\infty.
\end{eqnarray*}
\end{Teo}


\begin{Note} Una funci\'on $h:\rea_{+}\rightarrow\rea$ es Directamente Riemann Integrable en los siguientes casos:
\begin{itemize}
\item[a)] $h\left(t\right)\geq0$ es decreciente y Riemann Integrable.
\item[b)] $h$ es continua excepto posiblemente en un conjunto de Lebesgue de medida 0, y $|h\left(t\right)|\leq b\left(t\right)$, donde $b$ es DRI.
\end{itemize}
\end{Note}

\begin{Teo}[Teorema Principal de Renovaci\'on]
Si $F$ es no aritm\'etica y $h\left(t\right)$ es Directamente Riemann Integrable (DRI), entonces

\begin{eqnarray*}
lim_{t\rightarrow\infty}U\star h=\frac{1}{\mu}\int_{\rea_{+}}h\left(s\right)ds.
\end{eqnarray*}
\end{Teo}

\begin{Prop}
Cualquier funci\'on $H\left(t\right)$ acotada en intervalos finitos y que es 0 para $t<0$ puede expresarse como
\begin{eqnarray*}
H\left(t\right)=U\star h\left(t\right)\textrm{,  donde }h\left(t\right)=H\left(t\right)-F\star H\left(t\right)
\end{eqnarray*}
\end{Prop}

\begin{Def}
Un proceso estoc\'astico $X\left(t\right)$ es crudamente regenerativo en un tiempo aleatorio positivo $T$ si
\begin{eqnarray*}
\esp\left[X\left(T+t\right)|T\right]=\esp\left[X\left(t\right)\right]\textrm{, para }t\geq0,\end{eqnarray*}
y con las esperanzas anteriores finitas.
\end{Def}

\begin{Prop}
Sup\'ongase que $X\left(t\right)$ es un proceso crudamente regenerativo en $T$, que tiene distribuci\'on $F$. Si $\esp\left[X\left(t\right)\right]$ es acotado en intervalos finitos, entonces
\begin{eqnarray*}
\esp\left[X\left(t\right)\right]=U\star h\left(t\right)\textrm{,  donde }h\left(t\right)=\esp\left[X\left(t\right)\indora\left(T>t\right)\right].
\end{eqnarray*}
\end{Prop}

\begin{Teo}[Regeneraci\'on Cruda]
Sup\'ongase que $X\left(t\right)$ es un proceso con valores positivo crudamente regenerativo en $T$, y def\'inase $M=\sup\left\{|X\left(t\right)|:t\leq T\right\}$. Si $T$ es no aritm\'etico y $M$ y $MT$ tienen media finita, entonces
\begin{eqnarray*}
lim_{t\rightarrow\infty}\esp\left[X\left(t\right)\right]=\frac{1}{\mu}\int_{\rea_{+}}h\left(s\right)ds,
\end{eqnarray*}
donde $h\left(t\right)=\esp\left[X\left(t\right)\indora\left(T>t\right)\right]$.
\end{Teo}


\begin{Note} Una funci\'on $h:\rea_{+}\rightarrow\rea$ es Directamente Riemann Integrable en los siguientes casos:
\begin{itemize}
\item[a)] $h\left(t\right)\geq0$ es decreciente y Riemann Integrable.
\item[b)] $h$ es continua excepto posiblemente en un conjunto de Lebesgue de medida 0, y $|h\left(t\right)|\leq b\left(t\right)$, donde $b$ es DRI.
\end{itemize}
\end{Note}

\begin{Teo}[Teorema Principal de Renovaci\'on]
Si $F$ es no aritm\'etica y $h\left(t\right)$ es Directamente Riemann Integrable (DRI), entonces

\begin{eqnarray*}
lim_{t\rightarrow\infty}U\star h=\frac{1}{\mu}\int_{\rea_{+}}h\left(s\right)ds.
\end{eqnarray*}
\end{Teo}

\begin{Prop}
Cualquier funci\'on $H\left(t\right)$ acotada en intervalos finitos y que es 0 para $t<0$ puede expresarse como
\begin{eqnarray*}
H\left(t\right)=U\star h\left(t\right)\textrm{,  donde }h\left(t\right)=H\left(t\right)-F\star H\left(t\right)
\end{eqnarray*}
\end{Prop}

\begin{Def}
Un proceso estoc\'astico $X\left(t\right)$ es crudamente regenerativo en un tiempo aleatorio positivo $T$ si
\begin{eqnarray*}
\esp\left[X\left(T+t\right)|T\right]=\esp\left[X\left(t\right)\right]\textrm{, para }t\geq0,\end{eqnarray*}
y con las esperanzas anteriores finitas.
\end{Def}

\begin{Prop}
Sup\'ongase que $X\left(t\right)$ es un proceso crudamente regenerativo en $T$, que tiene distribuci\'on $F$. Si $\esp\left[X\left(t\right)\right]$ es acotado en intervalos finitos, entonces
\begin{eqnarray*}
\esp\left[X\left(t\right)\right]=U\star h\left(t\right)\textrm{,  donde }h\left(t\right)=\esp\left[X\left(t\right)\indora\left(T>t\right)\right].
\end{eqnarray*}
\end{Prop}

\begin{Teo}[Regeneraci\'on Cruda]
Sup\'ongase que $X\left(t\right)$ es un proceso con valores positivo crudamente regenerativo en $T$, y def\'inase $M=\sup\left\{|X\left(t\right)|:t\leq T\right\}$. Si $T$ es no aritm\'etico y $M$ y $MT$ tienen media finita, entonces
\begin{eqnarray*}
lim_{t\rightarrow\infty}\esp\left[X\left(t\right)\right]=\frac{1}{\mu}\int_{\rea_{+}}h\left(s\right)ds,
\end{eqnarray*}
donde $h\left(t\right)=\esp\left[X\left(t\right)\indora\left(T>t\right)\right]$.
\end{Teo}

\begin{Def}
Para el proceso $\left\{\left(N\left(t\right),X\left(t\right)\right):t\geq0\right\}$, sus trayectoria muestrales en el intervalo de tiempo $\left[T_{n-1},T_{n}\right)$ est\'an descritas por
\begin{eqnarray*}
\zeta_{n}=\left(\xi_{n},\left\{X\left(T_{n-1}+t\right):0\leq t<\xi_{n}\right\}\right)
\end{eqnarray*}
Este $\zeta_{n}$ es el $n$-\'esimo segmento del proceso. El proceso es regenerativo sobre los tiempos $T_{n}$ si sus segmentos $\zeta_{n}$ son independientes e id\'enticamennte distribuidos.
\end{Def}


\begin{Note}
Si $\tilde{X}\left(t\right)$ con espacio de estados $\tilde{S}$ es regenerativo sobre $T_{n}$, entonces $X\left(t\right)=f\left(\tilde{X}\left(t\right)\right)$ tambi\'en es regenerativo sobre $T_{n}$, para cualquier funci\'on $f:\tilde{S}\rightarrow S$.
\end{Note}

\begin{Note}
Los procesos regenerativos son crudamente regenerativos, pero no al rev\'es.
\end{Note}


\begin{Note}
Un proceso estoc\'astico a tiempo continuo o discreto es regenerativo si existe un proceso de renovaci\'on  tal que los segmentos del proceso entre tiempos de renovaci\'on sucesivos son i.i.d., es decir, para $\left\{X\left(t\right):t\geq0\right\}$ proceso estoc\'astico a tiempo continuo con espacio de estados $S$, espacio m\'etrico.
\end{Note}

Para $\left\{X\left(t\right):t\geq0\right\}$ Proceso Estoc\'astico a tiempo continuo con estado de espacios $S$, que es un espacio m\'etrico, con trayectorias continuas por la derecha y con l\'imites por la izquierda c.s. Sea $N\left(t\right)$ un proceso de renovaci\'on en $\rea_{+}$ definido en el mismo espacio de probabilidad que $X\left(t\right)$, con tiempos de renovaci\'on $T$ y tiempos de inter-renovaci\'on $\xi_{n}=T_{n}-T_{n-1}$, con misma distribuci\'on $F$ de media finita $\mu$.



\begin{Def}
Para el proceso $\left\{\left(N\left(t\right),X\left(t\right)\right):t\geq0\right\}$, sus trayectoria muestrales en el intervalo de tiempo $\left[T_{n-1},T_{n}\right)$ est\'an descritas por
\begin{eqnarray*}
\zeta_{n}=\left(\xi_{n},\left\{X\left(T_{n-1}+t\right):0\leq t<\xi_{n}\right\}\right)
\end{eqnarray*}
Este $\zeta_{n}$ es el $n$-\'esimo segmento del proceso. El proceso es regenerativo sobre los tiempos $T_{n}$ si sus segmentos $\zeta_{n}$ son independientes e id\'enticamennte distribuidos.
\end{Def}

\begin{Note}
Un proceso regenerativo con media de la longitud de ciclo finita es llamado positivo recurrente.
\end{Note}

\begin{Teo}[Procesos Regenerativos]
Suponga que el proceso
\end{Teo}


\begin{Def}[Renewal Process Trinity]
Para un proceso de renovaci\'on $N\left(t\right)$, los siguientes procesos proveen de informaci\'on sobre los tiempos de renovaci\'on.
\begin{itemize}
\item $A\left(t\right)=t-T_{N\left(t\right)}$, el tiempo de recurrencia hacia atr\'as al tiempo $t$, que es el tiempo desde la \'ultima renovaci\'on para $t$.

\item $B\left(t\right)=T_{N\left(t\right)+1}-t$, el tiempo de recurrencia hacia adelante al tiempo $t$, residual del tiempo de renovaci\'on, que es el tiempo para la pr\'oxima renovaci\'on despu\'es de $t$.

\item $L\left(t\right)=\xi_{N\left(t\right)+1}=A\left(t\right)+B\left(t\right)$, la longitud del intervalo de renovaci\'on que contiene a $t$.
\end{itemize}
\end{Def}

\begin{Note}
El proceso tridimensional $\left(A\left(t\right),B\left(t\right),L\left(t\right)\right)$ es regenerativo sobre $T_{n}$, y por ende cada proceso lo es. Cada proceso $A\left(t\right)$ y $B\left(t\right)$ son procesos de MArkov a tiempo continuo con trayectorias continuas por partes en el espacio de estados $\rea_{+}$. Una expresi\'on conveniente para su distribuci\'on conjunta es, para $0\leq x<t,y\geq0$
\begin{equation}\label{NoRenovacion}
P\left\{A\left(t\right)>x,B\left(t\right)>y\right\}=
P\left\{N\left(t+y\right)-N\left((t-x)\right)=0\right\}
\end{equation}
\end{Note}

\begin{Ejem}[Tiempos de recurrencia Poisson]
Si $N\left(t\right)$ es un proceso Poisson con tasa $\lambda$, entonces de la expresi\'on (\ref{NoRenovacion}) se tiene que

\begin{eqnarray*}
\begin{array}{lc}
P\left\{A\left(t\right)>x,B\left(t\right)>y\right\}=e^{-\lambda\left(x+y\right)},&0\leq x<t,y\geq0,
\end{array}
\end{eqnarray*}
que es la probabilidad Poisson de no renovaciones en un intervalo de longitud $x+y$.

\end{Ejem}

\begin{Note}
Una cadena de Markov erg\'odica tiene la propiedad de ser estacionaria si la distribuci\'on de su estado al tiempo $0$ es su distribuci\'on estacionaria.
\end{Note}


\begin{Def}
Un proceso estoc\'astico a tiempo continuo $\left\{X\left(t\right):t\geq0\right\}$ en un espacio general es estacionario si sus distribuciones finito dimensionales son invariantes bajo cualquier  traslado: para cada $0\leq s_{1}<s_{2}<\cdots<s_{k}$ y $t\geq0$,
\begin{eqnarray*}
\left(X\left(s_{1}+t\right),\ldots,X\left(s_{k}+t\right)\right)=_{d}\left(X\left(s_{1}\right),\ldots,X\left(s_{k}\right)\right).
\end{eqnarray*}
\end{Def}

\begin{Note}
Un proceso de Markov es estacionario si $X\left(t\right)=_{d}X\left(0\right)$, $t\geq0$.
\end{Note}

Considerese el proceso $N\left(t\right)=\sum_{n}\indora\left(\tau_{n}\leq t\right)$ en $\rea_{+}$, con puntos $0<\tau_{1}<\tau_{2}<\cdots$.

\begin{Prop}
Si $N$ es un proceso puntual estacionario y $\esp\left[N\left(1\right)\right]<\infty$, entonces $\esp\left[N\left(t\right)\right]=t\esp\left[N\left(1\right)\right]$, $t\geq0$

\end{Prop}

\begin{Teo}
Los siguientes enunciados son equivalentes
\begin{itemize}
\item[i)] El proceso retardado de renovaci\'on $N$ es estacionario.

\item[ii)] EL proceso de tiempos de recurrencia hacia adelante $B\left(t\right)$ es estacionario.


\item[iii)] $\esp\left[N\left(t\right)\right]=t/\mu$,


\item[iv)] $G\left(t\right)=F_{e}\left(t\right)=\frac{1}{\mu}\int_{0}^{t}\left[1-F\left(s\right)\right]ds$
\end{itemize}
Cuando estos enunciados son ciertos, $P\left\{B\left(t\right)\leq x\right\}=F_{e}\left(x\right)$, para $t,x\geq0$.

\end{Teo}

\begin{Note}
Una consecuencia del teorema anterior es que el Proceso Poisson es el \'unico proceso sin retardo que es estacionario.
\end{Note}

\begin{Coro}
El proceso de renovaci\'on $N\left(t\right)$ sin retardo, y cuyos tiempos de inter renonaci\'on tienen media finita, es estacionario si y s\'olo si es un proceso Poisson.

\end{Coro}

%______________________________________________________________________

%\section{Ejemplos, Notas importantes}
%______________________________________________________________________
%\section*{Ap\'endice A}
%__________________________________________________________________

%________________________________________________________________________
%\subsection*{Procesos Regenerativos}
%________________________________________________________________________



\begin{Note}
Si $\tilde{X}\left(t\right)$ con espacio de estados $\tilde{S}$ es regenerativo sobre $T_{n}$, entonces $X\left(t\right)=f\left(\tilde{X}\left(t\right)\right)$ tambi\'en es regenerativo sobre $T_{n}$, para cualquier funci\'on $f:\tilde{S}\rightarrow S$.
\end{Note}

\begin{Note}
Los procesos regenerativos son crudamente regenerativos, pero no al rev\'es.
\end{Note}
%\subsection*{Procesos Regenerativos: Sigman\cite{Sigman1}}
\begin{Def}[Definici\'on Cl\'asica]
Un proceso estoc\'astico $X=\left\{X\left(t\right):t\geq0\right\}$ es llamado regenerativo is existe una variable aleatoria $R_{1}>0$ tal que
\begin{itemize}
\item[i)] $\left\{X\left(t+R_{1}\right):t\geq0\right\}$ es independiente de $\left\{\left\{X\left(t\right):t<R_{1}\right\},\right\}$
\item[ii)] $\left\{X\left(t+R_{1}\right):t\geq0\right\}$ es estoc\'asticamente equivalente a $\left\{X\left(t\right):t>0\right\}$
\end{itemize}

Llamamos a $R_{1}$ tiempo de regeneraci\'on, y decimos que $X$ se regenera en este punto.
\end{Def}

$\left\{X\left(t+R_{1}\right)\right\}$ es regenerativo con tiempo de regeneraci\'on $R_{2}$, independiente de $R_{1}$ pero con la misma distribuci\'on que $R_{1}$. Procediendo de esta manera se obtiene una secuencia de variables aleatorias independientes e id\'enticamente distribuidas $\left\{R_{n}\right\}$ llamados longitudes de ciclo. Si definimos a $Z_{k}\equiv R_{1}+R_{2}+\cdots+R_{k}$, se tiene un proceso de renovaci\'on llamado proceso de renovaci\'on encajado para $X$.




\begin{Def}
Para $x$ fijo y para cada $t\geq0$, sea $I_{x}\left(t\right)=1$ si $X\left(t\right)\leq x$,  $I_{x}\left(t\right)=0$ en caso contrario, y def\'inanse los tiempos promedio
\begin{eqnarray*}
\overline{X}&=&lim_{t\rightarrow\infty}\frac{1}{t}\int_{0}^{\infty}X\left(u\right)du\\
\prob\left(X_{\infty}\leq x\right)&=&lim_{t\rightarrow\infty}\frac{1}{t}\int_{0}^{\infty}I_{x}\left(u\right)du,
\end{eqnarray*}
cuando estos l\'imites existan.
\end{Def}

Como consecuencia del teorema de Renovaci\'on-Recompensa, se tiene que el primer l\'imite  existe y es igual a la constante
\begin{eqnarray*}
\overline{X}&=&\frac{\esp\left[\int_{0}^{R_{1}}X\left(t\right)dt\right]}{\esp\left[R_{1}\right]},
\end{eqnarray*}
suponiendo que ambas esperanzas son finitas.

\begin{Note}
\begin{itemize}
\item[a)] Si el proceso regenerativo $X$ es positivo recurrente y tiene trayectorias muestrales no negativas, entonces la ecuaci\'on anterior es v\'alida.
\item[b)] Si $X$ es positivo recurrente regenerativo, podemos construir una \'unica versi\'on estacionaria de este proceso, $X_{e}=\left\{X_{e}\left(t\right)\right\}$, donde $X_{e}$ es un proceso estoc\'astico regenerativo y estrictamente estacionario, con distribuci\'on marginal distribuida como $X_{\infty}$
\end{itemize}
\end{Note}

Para $\left\{X\left(t\right):t\geq0\right\}$ Proceso Estoc\'astico a tiempo continuo con estado de espacios $S$, que es un espacio m\'etrico, con trayectorias continuas por la derecha y con l\'imites por la izquierda c.s. Sea $N\left(t\right)$ un proceso de renovaci\'on en $\rea_{+}$ definido en el mismo espacio de probabilidad que $X\left(t\right)$, con tiempos de renovaci\'on $T$ y tiempos de inter-renovaci\'on $\xi_{n}=T_{n}-T_{n-1}$, con misma distribuci\'on $F$ de media finita $\mu$.


\begin{Def}
Para el proceso $\left\{\left(N\left(t\right),X\left(t\right)\right):t\geq0\right\}$, sus trayectoria muestrales en el intervalo de tiempo $\left[T_{n-1},T_{n}\right)$ est\'an descritas por
\begin{eqnarray*}
\zeta_{n}=\left(\xi_{n},\left\{X\left(T_{n-1}+t\right):0\leq t<\xi_{n}\right\}\right)
\end{eqnarray*}
Este $\zeta_{n}$ es el $n$-\'esimo segmento del proceso. El proceso es regenerativo sobre los tiempos $T_{n}$ si sus segmentos $\zeta_{n}$ son independientes e id\'enticamennte distribuidos.
\end{Def}


\begin{Note}
Si $\tilde{X}\left(t\right)$ con espacio de estados $\tilde{S}$ es regenerativo sobre $T_{n}$, entonces $X\left(t\right)=f\left(\tilde{X}\left(t\right)\right)$ tambi\'en es regenerativo sobre $T_{n}$, para cualquier funci\'on $f:\tilde{S}\rightarrow S$.
\end{Note}

\begin{Note}
Los procesos regenerativos son crudamente regenerativos, pero no al rev\'es.
\end{Note}

\begin{Def}[Definici\'on Cl\'asica]
Un proceso estoc\'astico $X=\left\{X\left(t\right):t\geq0\right\}$ es llamado regenerativo is existe una variable aleatoria $R_{1}>0$ tal que
\begin{itemize}
\item[i)] $\left\{X\left(t+R_{1}\right):t\geq0\right\}$ es independiente de $\left\{\left\{X\left(t\right):t<R_{1}\right\},\right\}$
\item[ii)] $\left\{X\left(t+R_{1}\right):t\geq0\right\}$ es estoc\'asticamente equivalente a $\left\{X\left(t\right):t>0\right\}$
\end{itemize}

Llamamos a $R_{1}$ tiempo de regeneraci\'on, y decimos que $X$ se regenera en este punto.
\end{Def}

$\left\{X\left(t+R_{1}\right)\right\}$ es regenerativo con tiempo de regeneraci\'on $R_{2}$, independiente de $R_{1}$ pero con la misma distribuci\'on que $R_{1}$. Procediendo de esta manera se obtiene una secuencia de variables aleatorias independientes e id\'enticamente distribuidas $\left\{R_{n}\right\}$ llamados longitudes de ciclo. Si definimos a $Z_{k}\equiv R_{1}+R_{2}+\cdots+R_{k}$, se tiene un proceso de renovaci\'on llamado proceso de renovaci\'on encajado para $X$.

\begin{Note}
Un proceso regenerativo con media de la longitud de ciclo finita es llamado positivo recurrente.
\end{Note}


\begin{Def}
Para $x$ fijo y para cada $t\geq0$, sea $I_{x}\left(t\right)=1$ si $X\left(t\right)\leq x$,  $I_{x}\left(t\right)=0$ en caso contrario, y def\'inanse los tiempos promedio
\begin{eqnarray*}
\overline{X}&=&lim_{t\rightarrow\infty}\frac{1}{t}\int_{0}^{\infty}X\left(u\right)du\\
\prob\left(X_{\infty}\leq x\right)&=&lim_{t\rightarrow\infty}\frac{1}{t}\int_{0}^{\infty}I_{x}\left(u\right)du,
\end{eqnarray*}
cuando estos l\'imites existan.
\end{Def}

Como consecuencia del teorema de Renovaci\'on-Recompensa, se tiene que el primer l\'imite  existe y es igual a la constante
\begin{eqnarray*}
\overline{X}&=&\frac{\esp\left[\int_{0}^{R_{1}}X\left(t\right)dt\right]}{\esp\left[R_{1}\right]},
\end{eqnarray*}
suponiendo que ambas esperanzas son finitas.

\begin{Note}
\begin{itemize}
\item[a)] Si el proceso regenerativo $X$ es positivo recurrente y tiene trayectorias muestrales no negativas, entonces la ecuaci\'on anterior es v\'alida.
\item[b)] Si $X$ es positivo recurrente regenerativo, podemos construir una \'unica versi\'on estacionaria de este proceso, $X_{e}=\left\{X_{e}\left(t\right)\right\}$, donde $X_{e}$ es un proceso estoc\'astico regenerativo y estrictamente estacionario, con distribuci\'on marginal distribuida como $X_{\infty}$
\end{itemize}
\end{Note}

%__________________________________________________________________________________________
%\subsection{Procesos Regenerativos Estacionarios - Stidham \cite{Stidham}}
%__________________________________________________________________________________________


Un proceso estoc\'astico a tiempo continuo $\left\{V\left(t\right),t\geq0\right\}$ es un proceso regenerativo si existe una sucesi\'on de variables aleatorias independientes e id\'enticamente distribuidas $\left\{X_{1},X_{2},\ldots\right\}$, sucesi\'on de renovaci\'on, tal que para cualquier conjunto de Borel $A$, 

\begin{eqnarray*}
\prob\left\{V\left(t\right)\in A|X_{1}+X_{2}+\cdots+X_{R\left(t\right)}=s,\left\{V\left(\tau\right),\tau<s\right\}\right\}=\prob\left\{V\left(t-s\right)\in A|X_{1}>t-s\right\},
\end{eqnarray*}
para todo $0\leq s\leq t$, donde $R\left(t\right)=\max\left\{X_{1}+X_{2}+\cdots+X_{j}\leq t\right\}=$n\'umero de renovaciones ({\emph{puntos de regeneraci\'on}}) que ocurren en $\left[0,t\right]$. El intervalo $\left[0,X_{1}\right)$ es llamado {\emph{primer ciclo de regeneraci\'on}} de $\left\{V\left(t \right),t\geq0\right\}$, $\left[X_{1},X_{1}+X_{2}\right)$ el {\emph{segundo ciclo de regeneraci\'on}}, y as\'i sucesivamente.

Sea $X=X_{1}$ y sea $F$ la funci\'on de distrbuci\'on de $X$


\begin{Def}
Se define el proceso estacionario, $\left\{V^{*}\left(t\right),t\geq0\right\}$, para $\left\{V\left(t\right),t\geq0\right\}$ por

\begin{eqnarray*}
\prob\left\{V\left(t\right)\in A\right\}=\frac{1}{\esp\left[X\right]}\int_{0}^{\infty}\prob\left\{V\left(t+x\right)\in A|X>x\right\}\left(1-F\left(x\right)\right)dx,
\end{eqnarray*} 
para todo $t\geq0$ y todo conjunto de Borel $A$.
\end{Def}

\begin{Def}
Una distribuci\'on se dice que es {\emph{aritm\'etica}} si todos sus puntos de incremento son m\'ultiplos de la forma $0,\lambda, 2\lambda,\ldots$ para alguna $\lambda>0$ entera.
\end{Def}


\begin{Def}
Una modificaci\'on medible de un proceso $\left\{V\left(t\right),t\geq0\right\}$, es una versi\'on de este, $\left\{V\left(t,w\right)\right\}$ conjuntamente medible para $t\geq0$ y para $w\in S$, $S$ espacio de estados para $\left\{V\left(t\right),t\geq0\right\}$.
\end{Def}

\begin{Teo}
Sea $\left\{V\left(t\right),t\geq\right\}$ un proceso regenerativo no negativo con modificaci\'on medible. Sea $\esp\left[X\right]<\infty$. Entonces el proceso estacionario dado por la ecuaci\'on anterior est\'a bien definido y tiene funci\'on de distribuci\'on independiente de $t$, adem\'as
\begin{itemize}
\item[i)] \begin{eqnarray*}
\esp\left[V^{*}\left(0\right)\right]&=&\frac{\esp\left[\int_{0}^{X}V\left(s\right)ds\right]}{\esp\left[X\right]}\end{eqnarray*}
\item[ii)] Si $\esp\left[V^{*}\left(0\right)\right]<\infty$, equivalentemente, si $\esp\left[\int_{0}^{X}V\left(s\right)ds\right]<\infty$,entonces
\begin{eqnarray*}
\frac{\int_{0}^{t}V\left(s\right)ds}{t}\rightarrow\frac{\esp\left[\int_{0}^{X}V\left(s\right)ds\right]}{\esp\left[X\right]}
\end{eqnarray*}
con probabilidad 1 y en media, cuando $t\rightarrow\infty$.
\end{itemize}
\end{Teo}
%
%___________________________________________________________________________________________
%\vspace{5.5cm}
%\chapter{Cadenas de Markov estacionarias}
%\vspace{-1.0cm}


%__________________________________________________________________________________________
%\subsection{Procesos Regenerativos Estacionarios - Stidham \cite{Stidham}}
%__________________________________________________________________________________________


Un proceso estoc\'astico a tiempo continuo $\left\{V\left(t\right),t\geq0\right\}$ es un proceso regenerativo si existe una sucesi\'on de variables aleatorias independientes e id\'enticamente distribuidas $\left\{X_{1},X_{2},\ldots\right\}$, sucesi\'on de renovaci\'on, tal que para cualquier conjunto de Borel $A$, 

\begin{eqnarray*}
\prob\left\{V\left(t\right)\in A|X_{1}+X_{2}+\cdots+X_{R\left(t\right)}=s,\left\{V\left(\tau\right),\tau<s\right\}\right\}=\prob\left\{V\left(t-s\right)\in A|X_{1}>t-s\right\},
\end{eqnarray*}
para todo $0\leq s\leq t$, donde $R\left(t\right)=\max\left\{X_{1}+X_{2}+\cdots+X_{j}\leq t\right\}=$n\'umero de renovaciones ({\emph{puntos de regeneraci\'on}}) que ocurren en $\left[0,t\right]$. El intervalo $\left[0,X_{1}\right)$ es llamado {\emph{primer ciclo de regeneraci\'on}} de $\left\{V\left(t \right),t\geq0\right\}$, $\left[X_{1},X_{1}+X_{2}\right)$ el {\emph{segundo ciclo de regeneraci\'on}}, y as\'i sucesivamente.

Sea $X=X_{1}$ y sea $F$ la funci\'on de distrbuci\'on de $X$


\begin{Def}
Se define el proceso estacionario, $\left\{V^{*}\left(t\right),t\geq0\right\}$, para $\left\{V\left(t\right),t\geq0\right\}$ por

\begin{eqnarray*}
\prob\left\{V\left(t\right)\in A\right\}=\frac{1}{\esp\left[X\right]}\int_{0}^{\infty}\prob\left\{V\left(t+x\right)\in A|X>x\right\}\left(1-F\left(x\right)\right)dx,
\end{eqnarray*} 
para todo $t\geq0$ y todo conjunto de Borel $A$.
\end{Def}

\begin{Def}
Una distribuci\'on se dice que es {\emph{aritm\'etica}} si todos sus puntos de incremento son m\'ultiplos de la forma $0,\lambda, 2\lambda,\ldots$ para alguna $\lambda>0$ entera.
\end{Def}


\begin{Def}
Una modificaci\'on medible de un proceso $\left\{V\left(t\right),t\geq0\right\}$, es una versi\'on de este, $\left\{V\left(t,w\right)\right\}$ conjuntamente medible para $t\geq0$ y para $w\in S$, $S$ espacio de estados para $\left\{V\left(t\right),t\geq0\right\}$.
\end{Def}

\begin{Teo}
Sea $\left\{V\left(t\right),t\geq\right\}$ un proceso regenerativo no negativo con modificaci\'on medible. Sea $\esp\left[X\right]<\infty$. Entonces el proceso estacionario dado por la ecuaci\'on anterior est\'a bien definido y tiene funci\'on de distribuci\'on independiente de $t$, adem\'as
\begin{itemize}
\item[i)] \begin{eqnarray*}
\esp\left[V^{*}\left(0\right)\right]&=&\frac{\esp\left[\int_{0}^{X}V\left(s\right)ds\right]}{\esp\left[X\right]}\end{eqnarray*}
\item[ii)] Si $\esp\left[V^{*}\left(0\right)\right]<\infty$, equivalentemente, si $\esp\left[\int_{0}^{X}V\left(s\right)ds\right]<\infty$,entonces
\begin{eqnarray*}
\frac{\int_{0}^{t}V\left(s\right)ds}{t}\rightarrow\frac{\esp\left[\int_{0}^{X}V\left(s\right)ds\right]}{\esp\left[X\right]}
\end{eqnarray*}
con probabilidad 1 y en media, cuando $t\rightarrow\infty$.
\end{itemize}
\end{Teo}

Sea la funci\'on generadora de momentos para $L_{i}$, el n\'umero de usuarios en la cola $Q_{i}\left(z\right)$ en cualquier momento, est\'a dada por el tiempo promedio de $z^{L_{i}\left(t\right)}$ sobre el ciclo regenerativo definido anteriormente. Entonces 



Es decir, es posible determinar las longitudes de las colas a cualquier tiempo $t$. Entonces, determinando el primer momento es posible ver que


\begin{Def}
El tiempo de Ciclo $C_{i}$ es el periodo de tiempo que comienza cuando la cola $i$ es visitada por primera vez en un ciclo, y termina cuando es visitado nuevamente en el pr\'oximo ciclo. La duraci\'on del mismo est\'a dada por $\tau_{i}\left(m+1\right)-\tau_{i}\left(m\right)$, o equivalentemente $\overline{\tau}_{i}\left(m+1\right)-\overline{\tau}_{i}\left(m\right)$ bajo condiciones de estabilidad.
\end{Def}


\begin{Def}
El tiempo de intervisita $I_{i}$ es el periodo de tiempo que comienza cuando se ha completado el servicio en un ciclo y termina cuando es visitada nuevamente en el pr\'oximo ciclo. Su  duraci\'on del mismo est\'a dada por $\tau_{i}\left(m+1\right)-\overline{\tau}_{i}\left(m\right)$.
\end{Def}

La duraci\'on del tiempo de intervisita es $\tau_{i}\left(m+1\right)-\overline{\tau}\left(m\right)$. Dado que el n\'umero de usuarios presentes en $Q_{i}$ al tiempo $t=\tau_{i}\left(m+1\right)$ es igual al n\'umero de arribos durante el intervalo de tiempo $\left[\overline{\tau}\left(m\right),\tau_{i}\left(m+1\right)\right]$ se tiene que


\begin{eqnarray*}
\esp\left[z_{i}^{L_{i}\left(\tau_{i}\left(m+1\right)\right)}\right]=\esp\left[\left\{P_{i}\left(z_{i}\right)\right\}^{\tau_{i}\left(m+1\right)-\overline{\tau}\left(m\right)}\right]
\end{eqnarray*}

entonces, si $I_{i}\left(z\right)=\esp\left[z^{\tau_{i}\left(m+1\right)-\overline{\tau}\left(m\right)}\right]$
se tiene que $F_{i}\left(z\right)=I_{i}\left[P_{i}\left(z\right)\right]$
para $i=1,2$.

Conforme a la definici\'on dada al principio del cap\'itulo, definici\'on (\ref{Def.Tn}), sean $T_{1},T_{2},\ldots$ los puntos donde las longitudes de las colas de la red de sistemas de visitas c\'iclicas son cero simult\'aneamente, cuando la cola $Q_{j}$ es visitada por el servidor para dar servicio, es decir, $L_{1}\left(T_{i}\right)=0,L_{2}\left(T_{i}\right)=0,\hat{L}_{1}\left(T_{i}\right)=0$ y $\hat{L}_{2}\left(T_{i}\right)=0$, a estos puntos se les denominar\'a puntos regenerativos. Entonces, 

\begin{Def}
Al intervalo de tiempo entre dos puntos regenerativos se le llamar\'a ciclo regenerativo.
\end{Def}

\begin{Def}
Para $T_{i}$ se define, $M_{i}$, el n\'umero de ciclos de visita a la cola $Q_{l}$, durante el ciclo regenerativo, es decir, $M_{i}$ es un proceso de renovaci\'on.
\end{Def}

\begin{Def}
Para cada uno de los $M_{i}$'s, se definen a su vez la duraci\'on de cada uno de estos ciclos de visita en el ciclo regenerativo, $C_{i}^{(m)}$, para $m=1,2,\ldots,M_{i}$, que a su vez, tambi\'en es n proceso de renovaci\'on.
\end{Def}

\footnote{In Stidham and  Heyman \cite{Stidham} shows that is sufficient for the regenerative process to be stationary that the mean regenerative cycle time is finite: $\esp\left[\sum_{m=1}^{M_{i}}C_{i}^{(m)}\right]<\infty$, 


 como cada $C_{i}^{(m)}$ contiene intervalos de r\'eplica positivos, se tiene que $\esp\left[M_{i}\right]<\infty$, adem\'as, como $M_{i}>0$, se tiene que la condici\'on anterior es equivalente a tener que $\esp\left[C_{i}\right]<\infty$,
por lo tanto una condici\'on suficiente para la existencia del proceso regenerativo est\'a dada por $\sum_{k=1}^{N}\mu_{k}<1.$}

Para $\left\{X\left(t\right):t\geq0\right\}$ Proceso Estoc\'astico a tiempo continuo con estado de espacios $S$, que es un espacio m\'etrico, con trayectorias continuas por la derecha y con l\'imites por la izquierda c.s. Sea $N\left(t\right)$ un proceso de renovaci\'on en $\rea_{+}$ definido en el mismo espacio de probabilidad que $X\left(t\right)$, con tiempos de renovaci\'on $T$ y tiempos de inter-renovaci\'on $\xi_{n}=T_{n}-T_{n-1}$, con misma distribuci\'on $F$ de media finita $\mu$.

\begin{Def}
Un elemento aleatorio en un espacio medible $\left(E,\mathcal{E}\right)$ en un espacio de probabilidad $\left(\Omega,\mathcal{F},\prob\right)$ a $\left(E,\mathcal{E}\right)$, es decir,
para $A\in \mathcal{E}$,  se tiene que $\left\{Y\in A\right\}\in\mathcal{F}$, donde $\left\{Y\in A\right\}:=\left\{w\in\Omega:Y\left(w\right)\in A\right\}=:Y^{-1}A$.
\end{Def}

\begin{Note}
Tambi\'en se dice que $Y$ est\'a soportado por el espacio de probabilidad $\left(\Omega,\mathcal{F},\prob\right)$ y que $Y$ es un mapeo medible de $\Omega$ en $E$, es decir, es $\mathcal{F}/\mathcal{E}$ medible.
\end{Note}

\begin{Def}
Para cada $i\in \mathbb{I}$ sea $P_{i}$ una medida de probabilidad en un espacio medible $\left(E_{i},\mathcal{E}_{i}\right)$. Se define el espacio producto
$\otimes_{i\in\mathbb{I}}\left(E_{i},\mathcal{E}_{i}\right):=\left(\prod_{i\in\mathbb{I}}E_{i},\otimes_{i\in\mathbb{I}}\mathcal{E}_{i}\right)$, donde $\prod_{i\in\mathbb{I}}E_{i}$ es el producto cartesiano de los $E_{i}$'s, y $\otimes_{i\in\mathbb{I}}\mathcal{E}_{i}$ es la $\sigma$-\'algebra producto, es decir, es la $\sigma$-\'algebra m\'as peque\~na en $\prod_{i\in\mathbb{I}}E_{i}$ que hace al $i$-\'esimo mapeo proyecci\'on en $E_{i}$ medible para toda $i\in\mathbb{I}$ es la $\sigma$-\'algebra inducida por los mapeos proyecci\'on. $$\otimes_{i\in\mathbb{I}}\mathcal{E}_{i}:=\sigma\left\{\left\{y:y_{i}\in A\right\}:i\in\mathbb{I}\textrm{ y }A\in\mathcal{E}_{i}\right\}.$$
\end{Def}

\begin{Def}
Un espacio de probabilidad $\left(\tilde{\Omega},\tilde{\mathcal{F}},\tilde{\prob}\right)$ es una extensi\'on de otro espacio de probabilidad $\left(\Omega,\mathcal{F},\prob\right)$ si $\left(\tilde{\Omega},\tilde{\mathcal{F}},\tilde{\prob}\right)$ soporta un elemento aleatorio $\xi\in\left(\Omega,\mathcal{F}\right)$ que tienen a $\prob$ como distribuci\'on.
\end{Def}

\begin{Teo}
Sea $\mathbb{I}$ un conjunto de \'indices arbitrario. Para cada $i\in\mathbb{I}$ sea $P_{i}$ una medida de probabilidad en un espacio medible $\left(E_{i},\mathcal{E}_{i}\right)$. Entonces existe una \'unica medida de probabilidad $\otimes_{i\in\mathbb{I}}P_{i}$ en $\otimes_{i\in\mathbb{I}}\left(E_{i},\mathcal{E}_{i}\right)$ tal que 

\begin{eqnarray*}
\otimes_{i\in\mathbb{I}}P_{i}\left(y\in\prod_{i\in\mathbb{I}}E_{i}:y_{i}\in A_{i_{1}},\ldots,y_{n}\in A_{i_{n}}\right)=P_{i_{1}}\left(A_{i_{n}}\right)\cdots P_{i_{n}}\left(A_{i_{n}}\right)
\end{eqnarray*}
para todos los enteros $n>0$, toda $i_{1},\ldots,i_{n}\in\mathbb{I}$ y todo $A_{i_{1}}\in\mathcal{E}_{i_{1}},\ldots,A_{i_{n}}\in\mathcal{E}_{i_{n}}$
\end{Teo}

La medida $\otimes_{i\in\mathbb{I}}P_{i}$ es llamada la medida producto y $\otimes_{i\in\mathbb{I}}\left(E_{i},\mathcal{E}_{i},P_{i}\right):=\left(\prod_{i\in\mathbb{I}},E_{i},\otimes_{i\in\mathbb{I}}\mathcal{E}_{i},\otimes_{i\in\mathbb{I}}P_{i}\right)$, es llamado espacio de probabilidad producto.


\begin{Def}
Un espacio medible $\left(E,\mathcal{E}\right)$ es \textit{Polaco} si existe una m\'etrica en $E$ tal que $E$ es completo, es decir cada sucesi\'on de Cauchy converge a un l\'imite en $E$, y \textit{separable}, $E$ tienen un subconjunto denso numerable, y tal que $\mathcal{E}$ es generado por conjuntos abiertos.
\end{Def}


\begin{Def}
Dos espacios medibles $\left(E,\mathcal{E}\right)$ y $\left(G,\mathcal{G}\right)$ son Borel equivalentes \textit{isomorfos} si existe una biyecci\'on $f:E\rightarrow G$ tal que $f$ es $\mathcal{E}/\mathcal{G}$ medible y su inversa $f^{-1}$ es $\mathcal{G}/\mathcal{E}$ medible. La biyecci\'on es una equivalencia de Borel.
\end{Def}

\begin{Def}
Un espacio medible  $\left(E,\mathcal{E}\right)$ es un \textit{espacio est\'andar} si es Borel equivalente a $\left(G,\mathcal{G}\right)$, donde $G$ es un subconjunto de Borel de $\left[0,1\right]$ y $\mathcal{G}$ son los subconjuntos de Borel de $G$.
\end{Def}

\begin{Note}
Cualquier espacio Polaco es un espacio est\'andar.
\end{Note}


\begin{Def}
Un proceso estoc\'astico con conjunto de \'indices $\mathbb{I}$ y espacio de estados $\left(E,\mathcal{E}\right)$ es una familia $Z=\left(\mathbb{Z}_{s}\right)_{s\in\mathbb{I}}$ donde $\mathbb{Z}_{s}$ son elementos aleatorios definidos en un espacio de probabilidad com\'un $\left(\Omega,\mathcal{F},\prob\right)$ y todos toman valores en $\left(E,\mathcal{E}\right)$.
\end{Def}

\begin{Def}
Un proceso estoc\'astico \textit{one-sided contiuous time} (\textbf{PEOSCT}) es un proceso estoc\'astico con conjunto de \'indices $\mathbb{I}=\left[0,\infty\right)$.
\end{Def}


Sea $\left(E^{\mathbb{I}},\mathcal{E}^{\mathbb{I}}\right)$ denota el espacio producto $\left(E^{\mathbb{I}},\mathcal{E}^{\mathbb{I}}\right):=\otimes_{s\in\mathbb{I}}\left(E,\mathcal{E}\right)$. Vamos a considerar $\mathbb{Z}$ como un mapeo aleatorio, es decir, como un elemento aleatorio en $\left(E^{\mathbb{I}},\mathcal{E}^{\mathbb{I}}\right)$ definido por $Z\left(w\right)=\left(Z_{s}\left(w\right)\right)_{s\in\mathbb{I}}$ y $w\in\Omega$.

\begin{Note}
La distribuci\'on de un proceso estoc\'astico $Z$ es la distribuci\'on de $Z$ como un elemento aleatorio en $\left(E^{\mathbb{I}},\mathcal{E}^{\mathbb{I}}\right)$. La distribuci\'on de $Z$ esta determinada de manera \'unica por las distribuciones finito dimensionales.
\end{Note}

\begin{Note}
En particular cuando $Z$ toma valores reales, es decir, $\left(E,\mathcal{E}\right)=\left(\mathbb{R},\mathcal{B}\right)$ las distribuciones finito dimensionales est\'an determinadas por las funciones de distribuci\'on finito dimensionales

\begin{eqnarray}
\prob\left(Z_{t_{1}}\leq x_{1},\ldots,Z_{t_{n}}\leq x_{n}\right),x_{1},\ldots,x_{n}\in\mathbb{R},t_{1},\ldots,t_{n}\in\mathbb{I},n\geq1.
\end{eqnarray}
\end{Note}

\begin{Note}
Para espacios polacos $\left(E,\mathcal{E}\right)$ el Teorema de Consistencia de Kolmogorov asegura que dada una colecci\'on de distribuciones finito dimensionales consistentes, siempre existe un proceso estoc\'astico que posee tales distribuciones finito dimensionales.
\end{Note}


\begin{Def}
Las trayectorias de $Z$ son las realizaciones $Z\left(w\right)$ para $w\in\Omega$ del mapeo aleatorio $Z$.
\end{Def}

\begin{Note}
Algunas restricciones se imponen sobre las trayectorias, por ejemplo que sean continuas por la derecha, o continuas por la derecha con l\'imites por la izquierda, o de manera m\'as general, se pedir\'a que caigan en alg\'un subconjunto $H$ de $E^{\mathbb{I}}$. En este caso es natural considerar a $Z$ como un elemento aleatorio que no est\'a en $\left(E^{\mathbb{I}},\mathcal{E}^{\mathbb{I}}\right)$ sino en $\left(H,\mathcal{H}\right)$, donde $\mathcal{H}$ es la $\sigma$-\'algebra generada por los mapeos proyecci\'on que toman a $z\in H$ a $z_{t}\in E$ para $t\in\mathbb{I}$. A $\mathcal{H}$ se le conoce como la traza de $H$ en $E^{\mathbb{I}}$, es decir,
\begin{eqnarray}
\mathcal{H}:=E^{\mathbb{I}}\cap H:=\left\{A\cap H:A\in E^{\mathbb{I}}\right\}.
\end{eqnarray}
\end{Note}


\begin{Note}
$Z$ tiene trayectorias con valores en $H$ y cada $Z_{t}$ es un mapeo medible de $\left(\Omega,\mathcal{F}\right)$ a $\left(H,\mathcal{H}\right)$. Cuando se considera un espacio de trayectorias en particular $H$, al espacio $\left(H,\mathcal{H}\right)$ se le llama el espacio de trayectorias de $Z$.
\end{Note}

\begin{Note}
La distribuci\'on del proceso estoc\'astico $Z$ con espacio de trayectorias $\left(H,\mathcal{H}\right)$ es la distribuci\'on de $Z$ como  un elemento aleatorio en $\left(H,\mathcal{H}\right)$. La distribuci\'on, nuevemente, est\'a determinada de manera \'unica por las distribuciones finito dimensionales.
\end{Note}


\begin{Def}
Sea $Z$ un PEOSCT  con espacio de estados $\left(E,\mathcal{E}\right)$ y sea $T$ un tiempo aleatorio en $\left[0,\infty\right)$. Por $Z_{T}$ se entiende el mapeo con valores en $E$ definido en $\Omega$ en la manera obvia:
\begin{eqnarray*}
Z_{T}\left(w\right):=Z_{T\left(w\right)}\left(w\right). w\in\Omega.
\end{eqnarray*}
\end{Def}

\begin{Def}
Un PEOSCT $Z$ es conjuntamente medible (\textbf{CM}) si el mapeo que toma $\left(w,t\right)\in\Omega\times\left[0,\infty\right)$ a $Z_{t}\left(w\right)\in E$ es $\mathcal{F}\otimes\mathcal{B}\left[0,\infty\right)/\mathcal{E}$ medible.
\end{Def}

\begin{Note}
Un PEOSCT-CM implica que el proceso es medible, dado que $Z_{T}$ es una composici\'on  de dos mapeos continuos: el primero que toma $w$ en $\left(w,T\left(w\right)\right)$ es $\mathcal{F}/\mathcal{F}\otimes\mathcal{B}\left[0,\infty\right)$ medible, mientras que el segundo toma $\left(w,T\left(w\right)\right)$ en $Z_{T\left(w\right)}\left(w\right)$ es $\mathcal{F}\otimes\mathcal{B}\left[0,\infty\right)/\mathcal{E}$ medible.
\end{Note}


\begin{Def}
Un PEOSCT con espacio de estados $\left(H,\mathcal{H}\right)$ es can\'onicamente conjuntamente medible (\textbf{CCM}) si el mapeo $\left(z,t\right)\in H\times\left[0,\infty\right)$ en $Z_{t}\in E$ es $\mathcal{H}\otimes\mathcal{B}\left[0,\infty\right)/\mathcal{E}$ medible.
\end{Def}

\begin{Note}
Un PEOSCTCCM implica que el proceso es CM, dado que un PECCM $Z$ es un mapeo de $\Omega\times\left[0,\infty\right)$ a $E$, es la composici\'on de dos mapeos medibles: el primero, toma $\left(w,t\right)$ en $\left(Z\left(w\right),t\right)$ es $\mathcal{F}\otimes\mathcal{B}\left[0,\infty\right)/\mathcal{H}\otimes\mathcal{B}\left[0,\infty\right)$ medible, y el segundo que toma $\left(Z\left(w\right),t\right)$  en $Z_{t}\left(w\right)$ es $\mathcal{H}\otimes\mathcal{B}\left[0,\infty\right)/\mathcal{E}$ medible. Por tanto CCM es una condici\'on m\'as fuerte que CM.
\end{Note}

\begin{Def}
Un conjunto de trayectorias $H$ de un PEOSCT $Z$, es internamente shift-invariante (\textbf{ISI}) si 
\begin{eqnarray*}
\left\{\left(z_{t+s}\right)_{s\in\left[0,\infty\right)}:z\in H\right\}=H\textrm{, }t\in\left[0,\infty\right).
\end{eqnarray*}
\end{Def}


\begin{Def}
Dado un PEOSCTISI, se define el mapeo-shift $\theta_{t}$, $t\in\left[0,\infty\right)$, de $H$ a $H$ por 
\begin{eqnarray*}
\theta_{t}z=\left(z_{t+s}\right)_{s\in\left[0,\infty\right)}\textrm{, }z\in H.
\end{eqnarray*}
\end{Def}

\begin{Def}
Se dice que un proceso $Z$ es shift-medible (\textbf{SM}) si $Z$ tiene un conjunto de trayectorias $H$ que es ISI y adem\'as el mapeo que toma $\left(z,t\right)\in H\times\left[0,\infty\right)$ en $\theta_{t}z\in H$ es $\mathcal{H}\otimes\mathcal{B}\left[0,\infty\right)/\mathcal{H}$ medible.
\end{Def}

\begin{Note}
Un proceso estoc\'astico con conjunto de trayectorias $H$ ISI es shift-medible si y s\'olo si es CCM
\end{Note}

\begin{Note}
\begin{itemize}
\item Dado el espacio polaco $\left(E,\mathcal{E}\right)$ se tiene el  conjunto de trayectorias $D_{E}\left[0,\infty\right)$ que es ISI, entonces cumpe con ser CCM.

\item Si $G$ es abierto, podemos cubrirlo por bolas abiertas cuay cerradura este contenida en $G$, y como $G$ es segundo numerable como subespacio de $E$, lo podemos cubrir por una cantidad numerable de bolas abiertas.

\end{itemize}
\end{Note}


\begin{Note}
Los procesos estoc\'asticos $Z$ a tiempo discreto con espacio de estados polaco, tambi\'en tiene un espacio de trayectorias polaco y por tanto tiene distribuciones condicionales regulares.
\end{Note}

\begin{Teo}
El producto numerable de espacios polacos es polaco.
\end{Teo}


\begin{Def}
Sea $\left(\Omega,\mathcal{F},\prob\right)$ espacio de probabilidad que soporta al proceso $Z=\left(Z_{s}\right)_{s\in\left[0,\infty\right)}$ y $S=\left(S_{k}\right)_{0}^{\infty}$ donde $Z$ es un PEOSCTM con espacio de estados $\left(E,\mathcal{E}\right)$  y espacio de trayectorias $\left(H,\mathcal{H}\right)$  y adem\'as $S$ es una sucesi\'on de tiempos aleatorios one-sided que satisfacen la condici\'on $0\leq S_{0}<S_{1}<\cdots\rightarrow\infty$. Considerando $S$ como un mapeo medible de $\left(\Omega,\mathcal{F}\right)$ al espacio sucesi\'on $\left(L,\mathcal{L}\right)$, donde 
\begin{eqnarray*}
L=\left\{\left(s_{k}\right)_{0}^{\infty}\in\left[0,\infty\right)^{\left\{0,1,\ldots\right\}}:s_{0}<s_{1}<\cdots\rightarrow\infty\right\},
\end{eqnarray*}
donde $\mathcal{L}$ son los subconjuntos de Borel de $L$, es decir, $\mathcal{L}=L\cap\mathcal{B}^{\left\{0,1,\ldots\right\}}$.

As\'i el par $\left(Z,S\right)$ es un mapeo medible de  $\left(\Omega,\mathcal{F}\right)$ en $\left(H\times L,\mathcal{H}\otimes\mathcal{L}\right)$. El par $\mathcal{H}\otimes\mathcal{L}^{+}$ denotar\'a la clase de todas las funciones medibles de $\left(H\times L,\mathcal{H}\otimes\mathcal{L}\right)$ en $\left(\left[0,\infty\right),\mathcal{B}\left[0,\infty\right)\right)$.
\end{Def}


\begin{Def}
Sea $\theta_{t}$ el mapeo-shift conjunto de $H\times L$ en $H\times L$ dado por
\begin{eqnarray*}
\theta_{t}\left(z,\left(s_{k}\right)_{0}^{\infty}\right)=\theta_{t}\left(z,\left(s_{n_{t-}+k}-t\right)_{0}^{\infty}\right)
\end{eqnarray*}
donde 
$n_{t-}=inf\left\{n\geq1:s_{n}\geq t\right\}$.
\end{Def}

\begin{Note}
Con la finalidad de poder realizar los shift's sin complicaciones de medibilidad, se supondr\'a que $Z$ es shit-medible, es decir, el conjunto de trayectorias $H$ es invariante bajo shifts del tiempo y el mapeo que toma $\left(z,t\right)\in H\times\left[0,\infty\right)$ en $z_{t}\in E$ es $\mathcal{H}\otimes\mathcal{B}\left[0,\infty\right)/\mathcal{E}$ medible.
\end{Note}

\begin{Def}
Dado un proceso \textbf{PEOSSM} (Proceso Estoc\'astico One Side Shift Medible) $Z$, se dice regenerativo cl\'asico con tiempos de regeneraci\'on $S$ si 

\begin{eqnarray*}
\theta_{S_{n}}\left(Z,S\right)=\left(Z^{0},S^{0}\right),n\geq0
\end{eqnarray*}
y adem\'as $\theta_{S_{n}}\left(Z,S\right)$ es independiente de $\left(\left(Z_{s}\right)s\in\left[0,S_{n}\right),S_{0},\ldots,S_{n}\right)$
Si lo anterior se cumple, al par $\left(Z,S\right)$ se le llama regenerativo cl\'asico.
\end{Def}

\begin{Note}
Si el par $\left(Z,S\right)$ es regenerativo cl\'asico, entonces las longitudes de los ciclos $X_{1},X_{2},\ldots,$ son i.i.d. e independientes de la longitud del retraso $S_{0}$, es decir, $S$ es un proceso de renovaci\'on. Las longitudes de los ciclos tambi\'en son llamados tiempos de inter-regeneraci\'on y tiempos de ocurrencia.

\end{Note}

\begin{Teo}
Sup\'ongase que el par $\left(Z,S\right)$ es regenerativo cl\'asico con $\esp\left[X_{1}\right]<\infty$. Entonces $\left(Z^{*},S^{*}\right)$ en el teorema 2.1 es una versi\'on estacionaria de $\left(Z,S\right)$. Adem\'as, si $X_{1}$ es lattice con span $d$, entonces $\left(Z^{**},S^{**}\right)$ en el teorema 2.2 es una versi\'on periodicamente estacionaria de $\left(Z,S\right)$ con periodo $d$.

\end{Teo}

\begin{Def}
Una variable aleatoria $X_{1}$ es \textit{spread out} si existe una $n\geq1$ y una  funci\'on $f\in\mathcal{B}^{+}$ tal que $\int_{\rea}f\left(x\right)dx>0$ con $X_{2},X_{3},\ldots,X_{n}$ copias i.i.d  de $X_{1}$, $$\prob\left(X_{1}+\cdots+X_{n}\in B\right)\geq\int_{B}f\left(x\right)dx$$ para $B\in\mathcal{B}$.

\end{Def}



\begin{Def}
Dado un proceso estoc\'astico $Z$ se le llama \textit{wide-sense regenerative} (\textbf{WSR}) con tiempos de regeneraci\'on $S$ si $\theta_{S_{n}}\left(Z,S\right)=\left(Z^{0},S^{0}\right)$ para $n\geq0$ en distribuci\'on y $\theta_{S_{n}}\left(Z,S\right)$ es independiente de $\left(S_{0},S_{1},\ldots,S_{n}\right)$ para $n\geq0$.
Se dice que el par $\left(Z,S\right)$ es WSR si lo anterior se cumple.
\end{Def}


\begin{Note}
\begin{itemize}
\item El proceso de trayectorias $\left(\theta_{s}Z\right)_{s\in\left[0,\infty\right)}$ es WSR con tiempos de regeneraci\'on $S$ pero no es regenerativo cl\'asico.

\item Si $Z$ es cualquier proceso estacionario y $S$ es un proceso de renovaci\'on que es independiente de $Z$, entonces $\left(Z,S\right)$ es WSR pero en general no es regenerativo cl\'asico

\end{itemize}

\end{Note}


\begin{Note}
Para cualquier proceso estoc\'astico $Z$, el proceso de trayectorias $\left(\theta_{s}Z\right)_{s\in\left[0,\infty\right)}$ es siempre un proceso de Markov.
\end{Note}



\begin{Teo}
Supongase que el par $\left(Z,S\right)$ es WSR con $\esp\left[X_{1}\right]<\infty$. Entonces $\left(Z^{*},S^{*}\right)$ en el teorema 2.1 es una versi\'on estacionaria de 
$\left(Z,S\right)$.
\end{Teo}


\begin{Teo}
Supongase que $\left(Z,S\right)$ es cycle-stationary con $\esp\left[X_{1}\right]<\infty$. Sea $U$ distribuida uniformemente en $\left[0,1\right)$ e independiente de $\left(Z^{0},S^{0}\right)$ y sea $\prob^{*}$ la medida de probabilidad en $\left(\Omega,\prob\right)$ definida por $$d\prob^{*}=\frac{X_{1}}{\esp\left[X_{1}\right]}d\prob$$. Sea $\left(Z^{*},S^{*}\right)$ con distribuci\'on $\prob^{*}\left(\theta_{UX_{1}}\left(Z^{0},S^{0}\right)\in\cdot\right)$. Entonces $\left(Z^{}*,S^{*}\right)$ es estacionario,
\begin{eqnarray*}
\esp\left[f\left(Z^{*},S^{*}\right)\right]=\esp\left[\int_{0}^{X_{1}}f\left(\theta_{s}\left(Z^{0},S^{0}\right)\right)ds\right]/\esp\left[X_{1}\right]
\end{eqnarray*}
$f\in\mathcal{H}\otimes\mathcal{L}^{+}$, and $S_{0}^{*}$ es continuo con funci\'on distribuci\'on $G_{\infty}$ definida por $$G_{\infty}\left(x\right):=\frac{\esp\left[X_{1}\right]\wedge x}{\esp\left[X_{1}\right]}$$ para $x\geq0$ y densidad $\prob\left[X_{1}>x\right]/\esp\left[X_{1}\right]$, con $x\geq0$.

\end{Teo}


\begin{Teo}
Sea $Z$ un Proceso Estoc\'astico un lado shift-medible \textit{one-sided shift-measurable stochastic process}, (PEOSSM),
y $S_{0}$ y $S_{1}$ tiempos aleatorios tales que $0\leq S_{0}<S_{1}$ y
\begin{equation}
\theta_{S_{1}}Z=\theta_{S_{0}}Z\textrm{ en distribuci\'on}.
\end{equation}

Entonces el espacio de probabilidad subyacente $\left(\Omega,\mathcal{F},\prob\right)$ puede extenderse para soportar una sucesi\'on de tiempos aleatorios $S$ tales que

\begin{eqnarray}
\theta_{S_{n}}\left(Z,S\right)=\left(Z^{0},S^{0}\right),n\geq0,\textrm{ en distribuci\'on},\\
\left(Z,S_{0},S_{1}\right)\textrm{ depende de }\left(X_{2},X_{3},\ldots\right)\textrm{ solamente a traves de }\theta_{S_{1}}Z.
\end{eqnarray}
\end{Teo}



%_________________________________________________________________________
%
\subsection{Output Process and Regenerative Processes}
%_________________________________________________________________________
%
En Sigman, Thorison y Wolff \cite{Sigman2} prueban que para la existencia de un una sucesi\'on infinita no decreciente de tiempos de regeneraci\'on $\tau_{1}\leq\tau_{2}\leq\cdots$ en los cuales el proceso se regenera, basta un tiempo de regeneraci\'on $R_{1}$, donde $R_{j}=\tau_{j}-\tau_{j-1}$. Para tal efecto se requiere la existencia de un espacio de probabilidad $\left(\Omega,\mathcal{F},\prob\right)$, y proceso estoc\'astico $\textit{X}=\left\{X\left(t\right):t\geq0\right\}$ con espacio de estados $\left(S,\mathcal{R}\right)$, con $\mathcal{R}$ $\sigma$-\'algebra.

\begin{Prop}
Si existe una variable aleatoria no negativa $R_{1}$ tal que $\theta_{R1}X=_{D}X$, entonces $\left(\Omega,\mathcal{F},\prob\right)$ puede extenderse para soportar una sucesi\'on estacionaria de variables aleatorias $R=\left\{R_{k}:k\geq1\right\}$, tal que para $k\geq1$,
\begin{eqnarray*}
\theta_{k}\left(X,R\right)=_{D}\left(X,R\right).
\end{eqnarray*}

Adem\'as, para $k\geq1$, $\theta_{k}R$ es condicionalmente independiente de $\left(X,R_{1},\ldots,R_{k}\right)$, dado $\theta_{\tau k}X$.

\end{Prop}


\begin{itemize}
\item Doob en 1953 demostr\'o que el estado estacionario de un proceso de partida en un sistema de espera $M/G/\infty$, es Poisson con la misma tasa que el proceso de arribos.

\item Burke en 1968, fue el primero en demostrar que el estado estacionario de un proceso de salida de una cola $M/M/s$ es un proceso Poisson.

\item Disney en 1973 obtuvo el siguiente resultado:

\begin{Teo}
Para el sistema de espera $M/G/1/L$ con disciplina FIFO, el proceso $\textbf{I}$ es un proceso de renovaci\'on si y s\'olo si el proceso denominado longitud de la cola es estacionario y se cumple cualquiera de los siguientes casos:

\begin{itemize}
\item[a)] Los tiempos de servicio son identicamente cero;
\item[b)] $L=0$, para cualquier proceso de servicio $S$;
\item[c)] $L=1$ y $G=D$;
\item[d)] $L=\infty$ y $G=M$.
\end{itemize}
En estos casos, respectivamente, las distribuciones de interpartida $P\left\{T_{n+1}-T_{n}\leq t\right\}$ son


\begin{itemize}
\item[a)] $1-e^{-\lambda t}$, $t\geq0$;
\item[b)] $1-e^{-\lambda t}*F\left(t\right)$, $t\geq0$;
\item[c)] $1-e^{-\lambda t}*\indora_{d}\left(t\right)$, $t\geq0$;
\item[d)] $1-e^{-\lambda t}*F\left(t\right)$, $t\geq0$.
\end{itemize}
\end{Teo}


\item Finch (1959) mostr\'o que para los sistemas $M/G/1/L$, con $1\leq L\leq \infty$ con distribuciones de servicio dos veces diferenciable, solamente el sistema $M/M/1/\infty$ tiene proceso de salida de renovaci\'on estacionario.

\item King (1971) demostro que un sistema de colas estacionario $M/G/1/1$ tiene sus tiempos de interpartida sucesivas $D_{n}$ y $D_{n+1}$ son independientes, si y s\'olo si, $G=D$, en cuyo caso le proceso de salida es de renovaci\'on.

\item Disney (1973) demostr\'o que el \'unico sistema estacionario $M/G/1/L$, que tiene proceso de salida de renovaci\'on  son los sistemas $M/M/1$ y $M/D/1/1$.



\item El siguiente resultado es de Disney y Koning (1985)
\begin{Teo}
En un sistema de espera $M/G/s$, el estado estacionario del proceso de salida es un proceso Poisson para cualquier distribuci\'on de los tiempos de servicio si el sistema tiene cualquiera de las siguientes cuatro propiedades.

\begin{itemize}
\item[a)] $s=\infty$
\item[b)] La disciplina de servicio es de procesador compartido.
\item[c)] La disciplina de servicio es LCFS y preemptive resume, esto se cumple para $L<\infty$
\item[d)] $G=M$.
\end{itemize}

\end{Teo}

\item El siguiente resultado es de Alamatsaz (1983)

\begin{Teo}
En cualquier sistema de colas $GI/G/1/L$ con $1\leq L<\infty$ y distribuci\'on de interarribos $A$ y distribuci\'on de los tiempos de servicio $B$, tal que $A\left(0\right)=0$, $A\left(t\right)\left(1-B\left(t\right)\right)>0$ para alguna $t>0$ y $B\left(t\right)$ para toda $t>0$, es imposible que el proceso de salida estacionario sea de renovaci\'on.
\end{Teo}

\end{itemize}



%________________________________________________________________________
%\subsection{Procesos Regenerativos Sigman, Thorisson y Wolff \cite{Sigman1}}
%________________________________________________________________________


\begin{Def}[Definici\'on Cl\'asica]
Un proceso estoc\'astico $X=\left\{X\left(t\right):t\geq0\right\}$ es llamado regenerativo is existe una variable aleatoria $R_{1}>0$ tal que
\begin{itemize}
\item[i)] $\left\{X\left(t+R_{1}\right):t\geq0\right\}$ es independiente de $\left\{\left\{X\left(t\right):t<R_{1}\right\},\right\}$
\item[ii)] $\left\{X\left(t+R_{1}\right):t\geq0\right\}$ es estoc\'asticamente equivalente a $\left\{X\left(t\right):t>0\right\}$
\end{itemize}

Llamamos a $R_{1}$ tiempo de regeneraci\'on, y decimos que $X$ se regenera en este punto.
\end{Def}

$\left\{X\left(t+R_{1}\right)\right\}$ es regenerativo con tiempo de regeneraci\'on $R_{2}$, independiente de $R_{1}$ pero con la misma distribuci\'on que $R_{1}$. Procediendo de esta manera se obtiene una secuencia de variables aleatorias independientes e id\'enticamente distribuidas $\left\{R_{n}\right\}$ llamados longitudes de ciclo. Si definimos a $Z_{k}\equiv R_{1}+R_{2}+\cdots+R_{k}$, se tiene un proceso de renovaci\'on llamado proceso de renovaci\'on encajado para $X$.


\begin{Note}
La existencia de un primer tiempo de regeneraci\'on, $R_{1}$, implica la existencia de una sucesi\'on completa de estos tiempos $R_{1},R_{2}\ldots,$ que satisfacen la propiedad deseada \cite{Sigman2}.
\end{Note}


\begin{Note} Para la cola $GI/GI/1$ los usuarios arriban con tiempos $t_{n}$ y son atendidos con tiempos de servicio $S_{n}$, los tiempos de arribo forman un proceso de renovaci\'on  con tiempos entre arribos independientes e identicamente distribuidos (\texttt{i.i.d.})$T_{n}=t_{n}-t_{n-1}$, adem\'as los tiempos de servicio son \texttt{i.i.d.} e independientes de los procesos de arribo. Por \textit{estable} se entiende que $\esp S_{n}<\esp T_{n}<\infty$.
\end{Note}
 


\begin{Def}
Para $x$ fijo y para cada $t\geq0$, sea $I_{x}\left(t\right)=1$ si $X\left(t\right)\leq x$,  $I_{x}\left(t\right)=0$ en caso contrario, y def\'inanse los tiempos promedio
\begin{eqnarray*}
\overline{X}&=&lim_{t\rightarrow\infty}\frac{1}{t}\int_{0}^{\infty}X\left(u\right)du\\
\prob\left(X_{\infty}\leq x\right)&=&lim_{t\rightarrow\infty}\frac{1}{t}\int_{0}^{\infty}I_{x}\left(u\right)du,
\end{eqnarray*}
cuando estos l\'imites existan.
\end{Def}

Como consecuencia del teorema de Renovaci\'on-Recompensa, se tiene que el primer l\'imite  existe y es igual a la constante
\begin{eqnarray*}
\overline{X}&=&\frac{\esp\left[\int_{0}^{R_{1}}X\left(t\right)dt\right]}{\esp\left[R_{1}\right]},
\end{eqnarray*}
suponiendo que ambas esperanzas son finitas.
 
\begin{Note}
Funciones de procesos regenerativos son regenerativas, es decir, si $X\left(t\right)$ es regenerativo y se define el proceso $Y\left(t\right)$ por $Y\left(t\right)=f\left(X\left(t\right)\right)$ para alguna funci\'on Borel medible $f\left(\cdot\right)$. Adem\'as $Y$ es regenerativo con los mismos tiempos de renovaci\'on que $X$. 

En general, los tiempos de renovaci\'on, $Z_{k}$ de un proceso regenerativo no requieren ser tiempos de paro con respecto a la evoluci\'on de $X\left(t\right)$.
\end{Note} 

\begin{Note}
Una funci\'on de un proceso de Markov, usualmente no ser\'a un proceso de Markov, sin embargo ser\'a regenerativo si el proceso de Markov lo es.
\end{Note}

 
\begin{Note}
Un proceso regenerativo con media de la longitud de ciclo finita es llamado positivo recurrente.
\end{Note}


\begin{Note}
\begin{itemize}
\item[a)] Si el proceso regenerativo $X$ es positivo recurrente y tiene trayectorias muestrales no negativas, entonces la ecuaci\'on anterior es v\'alida.
\item[b)] Si $X$ es positivo recurrente regenerativo, podemos construir una \'unica versi\'on estacionaria de este proceso, $X_{e}=\left\{X_{e}\left(t\right)\right\}$, donde $X_{e}$ es un proceso estoc\'astico regenerativo y estrictamente estacionario, con distribuci\'on marginal distribuida como $X_{\infty}$
\end{itemize}
\end{Note}


%__________________________________________________________________________________________
%\subsection{Procesos Regenerativos Estacionarios - Stidham \cite{Stidham}}
%__________________________________________________________________________________________


Un proceso estoc\'astico a tiempo continuo $\left\{V\left(t\right),t\geq0\right\}$ es un proceso regenerativo si existe una sucesi\'on de variables aleatorias independientes e id\'enticamente distribuidas $\left\{X_{1},X_{2},\ldots\right\}$, sucesi\'on de renovaci\'on, tal que para cualquier conjunto de Borel $A$, 

\begin{eqnarray*}
\prob\left\{V\left(t\right)\in A|X_{1}+X_{2}+\cdots+X_{R\left(t\right)}=s,\left\{V\left(\tau\right),\tau<s\right\}\right\}=\prob\left\{V\left(t-s\right)\in A|X_{1}>t-s\right\},
\end{eqnarray*}
para todo $0\leq s\leq t$, donde $R\left(t\right)=\max\left\{X_{1}+X_{2}+\cdots+X_{j}\leq t\right\}=$n\'umero de renovaciones ({\emph{puntos de regeneraci\'on}}) que ocurren en $\left[0,t\right]$. El intervalo $\left[0,X_{1}\right)$ es llamado {\emph{primer ciclo de regeneraci\'on}} de $\left\{V\left(t \right),t\geq0\right\}$, $\left[X_{1},X_{1}+X_{2}\right)$ el {\emph{segundo ciclo de regeneraci\'on}}, y as\'i sucesivamente.

Sea $X=X_{1}$ y sea $F$ la funci\'on de distrbuci\'on de $X$


\begin{Def}
Se define el proceso estacionario, $\left\{V^{*}\left(t\right),t\geq0\right\}$, para $\left\{V\left(t\right),t\geq0\right\}$ por

\begin{eqnarray*}
\prob\left\{V\left(t\right)\in A\right\}=\frac{1}{\esp\left[X\right]}\int_{0}^{\infty}\prob\left\{V\left(t+x\right)\in A|X>x\right\}\left(1-F\left(x\right)\right)dx,
\end{eqnarray*} 
para todo $t\geq0$ y todo conjunto de Borel $A$.
\end{Def}

\begin{Def}
Una distribuci\'on se dice que es {\emph{aritm\'etica}} si todos sus puntos de incremento son m\'ultiplos de la forma $0,\lambda, 2\lambda,\ldots$ para alguna $\lambda>0$ entera.
\end{Def}


\begin{Def}
Una modificaci\'on medible de un proceso $\left\{V\left(t\right),t\geq0\right\}$, es una versi\'on de este, $\left\{V\left(t,w\right)\right\}$ conjuntamente medible para $t\geq0$ y para $w\in S$, $S$ espacio de estados para $\left\{V\left(t\right),t\geq0\right\}$.
\end{Def}

\begin{Teo}
Sea $\left\{V\left(t\right),t\geq\right\}$ un proceso regenerativo no negativo con modificaci\'on medible. Sea $\esp\left[X\right]<\infty$. Entonces el proceso estacionario dado por la ecuaci\'on anterior est\'a bien definido y tiene funci\'on de distribuci\'on independiente de $t$, adem\'as
\begin{itemize}
\item[i)] \begin{eqnarray*}
\esp\left[V^{*}\left(0\right)\right]&=&\frac{\esp\left[\int_{0}^{X}V\left(s\right)ds\right]}{\esp\left[X\right]}\end{eqnarray*}
\item[ii)] Si $\esp\left[V^{*}\left(0\right)\right]<\infty$, equivalentemente, si $\esp\left[\int_{0}^{X}V\left(s\right)ds\right]<\infty$,entonces
\begin{eqnarray*}
\frac{\int_{0}^{t}V\left(s\right)ds}{t}\rightarrow\frac{\esp\left[\int_{0}^{X}V\left(s\right)ds\right]}{\esp\left[X\right]}
\end{eqnarray*}
con probabilidad 1 y en media, cuando $t\rightarrow\infty$.
\end{itemize}
\end{Teo}

\begin{Coro}
Sea $\left\{V\left(t\right),t\geq0\right\}$ un proceso regenerativo no negativo, con modificaci\'on medible. Si $\esp <\infty$, $F$ es no-aritm\'etica, y para todo $x\geq0$, $P\left\{V\left(t\right)\leq x,C>x\right\}$ es de variaci\'on acotada como funci\'on de $t$ en cada intervalo finito $\left[0,\tau\right]$, entonces $V\left(t\right)$ converge en distribuci\'on  cuando $t\rightarrow\infty$ y $$\esp V=\frac{\esp \int_{0}^{X}V\left(s\right)ds}{\esp X}$$
Donde $V$ tiene la distribuci\'on l\'imite de $V\left(t\right)$ cuando $t\rightarrow\infty$.

\end{Coro}

Para el caso discreto se tienen resultados similares.



%______________________________________________________________________
%\subsection{Procesos de Renovaci\'on}
%______________________________________________________________________

\begin{Def}%\label{Def.Tn}
Sean $0\leq T_{1}\leq T_{2}\leq \ldots$ son tiempos aleatorios infinitos en los cuales ocurren ciertos eventos. El n\'umero de tiempos $T_{n}$ en el intervalo $\left[0,t\right)$ es

\begin{eqnarray}
N\left(t\right)=\sum_{n=1}^{\infty}\indora\left(T_{n}\leq t\right),
\end{eqnarray}
para $t\geq0$.
\end{Def}

Si se consideran los puntos $T_{n}$ como elementos de $\rea_{+}$, y $N\left(t\right)$ es el n\'umero de puntos en $\rea$. El proceso denotado por $\left\{N\left(t\right):t\geq0\right\}$, denotado por $N\left(t\right)$, es un proceso puntual en $\rea_{+}$. Los $T_{n}$ son los tiempos de ocurrencia, el proceso puntual $N\left(t\right)$ es simple si su n\'umero de ocurrencias son distintas: $0<T_{1}<T_{2}<\ldots$ casi seguramente.

\begin{Def}
Un proceso puntual $N\left(t\right)$ es un proceso de renovaci\'on si los tiempos de interocurrencia $\xi_{n}=T_{n}-T_{n-1}$, para $n\geq1$, son independientes e identicamente distribuidos con distribuci\'on $F$, donde $F\left(0\right)=0$ y $T_{0}=0$. Los $T_{n}$ son llamados tiempos de renovaci\'on, referente a la independencia o renovaci\'on de la informaci\'on estoc\'astica en estos tiempos. Los $\xi_{n}$ son los tiempos de inter-renovaci\'on, y $N\left(t\right)$ es el n\'umero de renovaciones en el intervalo $\left[0,t\right)$
\end{Def}


\begin{Note}
Para definir un proceso de renovaci\'on para cualquier contexto, solamente hay que especificar una distribuci\'on $F$, con $F\left(0\right)=0$, para los tiempos de inter-renovaci\'on. La funci\'on $F$ en turno degune las otra variables aleatorias. De manera formal, existe un espacio de probabilidad y una sucesi\'on de variables aleatorias $\xi_{1},\xi_{2},\ldots$ definidas en este con distribuci\'on $F$. Entonces las otras cantidades son $T_{n}=\sum_{k=1}^{n}\xi_{k}$ y $N\left(t\right)=\sum_{n=1}^{\infty}\indora\left(T_{n}\leq t\right)$, donde $T_{n}\rightarrow\infty$ casi seguramente por la Ley Fuerte de los Grandes Números.
\end{Note}

%___________________________________________________________________________________________
%
%\subsection{Teorema Principal de Renovaci\'on}
%___________________________________________________________________________________________
%

\begin{Note} Una funci\'on $h:\rea_{+}\rightarrow\rea$ es Directamente Riemann Integrable en los siguientes casos:
\begin{itemize}
\item[a)] $h\left(t\right)\geq0$ es decreciente y Riemann Integrable.
\item[b)] $h$ es continua excepto posiblemente en un conjunto de Lebesgue de medida 0, y $|h\left(t\right)|\leq b\left(t\right)$, donde $b$ es DRI.
\end{itemize}
\end{Note}

\begin{Teo}[Teorema Principal de Renovaci\'on]
Si $F$ es no aritm\'etica y $h\left(t\right)$ es Directamente Riemann Integrable (DRI), entonces

\begin{eqnarray*}
lim_{t\rightarrow\infty}U\star h=\frac{1}{\mu}\int_{\rea_{+}}h\left(s\right)ds.
\end{eqnarray*}
\end{Teo}

\begin{Prop}
Cualquier funci\'on $H\left(t\right)$ acotada en intervalos finitos y que es 0 para $t<0$ puede expresarse como
\begin{eqnarray*}
H\left(t\right)=U\star h\left(t\right)\textrm{,  donde }h\left(t\right)=H\left(t\right)-F\star H\left(t\right)
\end{eqnarray*}
\end{Prop}

\begin{Def}
Un proceso estoc\'astico $X\left(t\right)$ es crudamente regenerativo en un tiempo aleatorio positivo $T$ si
\begin{eqnarray*}
\esp\left[X\left(T+t\right)|T\right]=\esp\left[X\left(t\right)\right]\textrm{, para }t\geq0,\end{eqnarray*}
y con las esperanzas anteriores finitas.
\end{Def}

\begin{Prop}
Sup\'ongase que $X\left(t\right)$ es un proceso crudamente regenerativo en $T$, que tiene distribuci\'on $F$. Si $\esp\left[X\left(t\right)\right]$ es acotado en intervalos finitos, entonces
\begin{eqnarray*}
\esp\left[X\left(t\right)\right]=U\star h\left(t\right)\textrm{,  donde }h\left(t\right)=\esp\left[X\left(t\right)\indora\left(T>t\right)\right].
\end{eqnarray*}
\end{Prop}

\begin{Teo}[Regeneraci\'on Cruda]
Sup\'ongase que $X\left(t\right)$ es un proceso con valores positivo crudamente regenerativo en $T$, y def\'inase $M=\sup\left\{|X\left(t\right)|:t\leq T\right\}$. Si $T$ es no aritm\'etico y $M$ y $MT$ tienen media finita, entonces
\begin{eqnarray*}
lim_{t\rightarrow\infty}\esp\left[X\left(t\right)\right]=\frac{1}{\mu}\int_{\rea_{+}}h\left(s\right)ds,
\end{eqnarray*}
donde $h\left(t\right)=\esp\left[X\left(t\right)\indora\left(T>t\right)\right]$.
\end{Teo}

%___________________________________________________________________________________________
%
%\subsection{Propiedades de los Procesos de Renovaci\'on}
%___________________________________________________________________________________________
%

Los tiempos $T_{n}$ est\'an relacionados con los conteos de $N\left(t\right)$ por

\begin{eqnarray*}
\left\{N\left(t\right)\geq n\right\}&=&\left\{T_{n}\leq t\right\}\\
T_{N\left(t\right)}\leq &t&<T_{N\left(t\right)+1},
\end{eqnarray*}

adem\'as $N\left(T_{n}\right)=n$, y 

\begin{eqnarray*}
N\left(t\right)=\max\left\{n:T_{n}\leq t\right\}=\min\left\{n:T_{n+1}>t\right\}
\end{eqnarray*}

Por propiedades de la convoluci\'on se sabe que

\begin{eqnarray*}
P\left\{T_{n}\leq t\right\}=F^{n\star}\left(t\right)
\end{eqnarray*}
que es la $n$-\'esima convoluci\'on de $F$. Entonces 

\begin{eqnarray*}
\left\{N\left(t\right)\geq n\right\}&=&\left\{T_{n}\leq t\right\}\\
P\left\{N\left(t\right)\leq n\right\}&=&1-F^{\left(n+1\right)\star}\left(t\right)
\end{eqnarray*}

Adem\'as usando el hecho de que $\esp\left[N\left(t\right)\right]=\sum_{n=1}^{\infty}P\left\{N\left(t\right)\geq n\right\}$
se tiene que

\begin{eqnarray*}
\esp\left[N\left(t\right)\right]=\sum_{n=1}^{\infty}F^{n\star}\left(t\right)
\end{eqnarray*}

\begin{Prop}
Para cada $t\geq0$, la funci\'on generadora de momentos $\esp\left[e^{\alpha N\left(t\right)}\right]$ existe para alguna $\alpha$ en una vecindad del 0, y de aqu\'i que $\esp\left[N\left(t\right)^{m}\right]<\infty$, para $m\geq1$.
\end{Prop}


\begin{Note}
Si el primer tiempo de renovaci\'on $\xi_{1}$ no tiene la misma distribuci\'on que el resto de las $\xi_{n}$, para $n\geq2$, a $N\left(t\right)$ se le llama Proceso de Renovaci\'on retardado, donde si $\xi$ tiene distribuci\'on $G$, entonces el tiempo $T_{n}$ de la $n$-\'esima renovaci\'on tiene distribuci\'on $G\star F^{\left(n-1\right)\star}\left(t\right)$
\end{Note}


\begin{Teo}
Para una constante $\mu\leq\infty$ ( o variable aleatoria), las siguientes expresiones son equivalentes:

\begin{eqnarray}
lim_{n\rightarrow\infty}n^{-1}T_{n}&=&\mu,\textrm{ c.s.}\\
lim_{t\rightarrow\infty}t^{-1}N\left(t\right)&=&1/\mu,\textrm{ c.s.}
\end{eqnarray}
\end{Teo}


Es decir, $T_{n}$ satisface la Ley Fuerte de los Grandes N\'umeros s\'i y s\'olo s\'i $N\left/t\right)$ la cumple.


\begin{Coro}[Ley Fuerte de los Grandes N\'umeros para Procesos de Renovaci\'on]
Si $N\left(t\right)$ es un proceso de renovaci\'on cuyos tiempos de inter-renovaci\'on tienen media $\mu\leq\infty$, entonces
\begin{eqnarray}
t^{-1}N\left(t\right)\rightarrow 1/\mu,\textrm{ c.s. cuando }t\rightarrow\infty.
\end{eqnarray}

\end{Coro}


Considerar el proceso estoc\'astico de valores reales $\left\{Z\left(t\right):t\geq0\right\}$ en el mismo espacio de probabilidad que $N\left(t\right)$

\begin{Def}
Para el proceso $\left\{Z\left(t\right):t\geq0\right\}$ se define la fluctuaci\'on m\'axima de $Z\left(t\right)$ en el intervalo $\left(T_{n-1},T_{n}\right]$:
\begin{eqnarray*}
M_{n}=\sup_{T_{n-1}<t\leq T_{n}}|Z\left(t\right)-Z\left(T_{n-1}\right)|
\end{eqnarray*}
\end{Def}

\begin{Teo}
Sup\'ongase que $n^{-1}T_{n}\rightarrow\mu$ c.s. cuando $n\rightarrow\infty$, donde $\mu\leq\infty$ es una constante o variable aleatoria. Sea $a$ una constante o variable aleatoria que puede ser infinita cuando $\mu$ es finita, y considere las expresiones l\'imite:
\begin{eqnarray}
lim_{n\rightarrow\infty}n^{-1}Z\left(T_{n}\right)&=&a,\textrm{ c.s.}\\
lim_{t\rightarrow\infty}t^{-1}Z\left(t\right)&=&a/\mu,\textrm{ c.s.}
\end{eqnarray}
La segunda expresi\'on implica la primera. Conversamente, la primera implica la segunda si el proceso $Z\left(t\right)$ es creciente, o si $lim_{n\rightarrow\infty}n^{-1}M_{n}=0$ c.s.
\end{Teo}

\begin{Coro}
Si $N\left(t\right)$ es un proceso de renovaci\'on, y $\left(Z\left(T_{n}\right)-Z\left(T_{n-1}\right),M_{n}\right)$, para $n\geq1$, son variables aleatorias independientes e id\'enticamente distribuidas con media finita, entonces,
\begin{eqnarray}
lim_{t\rightarrow\infty}t^{-1}Z\left(t\right)\rightarrow\frac{\esp\left[Z\left(T_{1}\right)-Z\left(T_{0}\right)\right]}{\esp\left[T_{1}\right]},\textrm{ c.s. cuando  }t\rightarrow\infty.
\end{eqnarray}
\end{Coro}



%___________________________________________________________________________________________
%
%\subsection{Propiedades de los Procesos de Renovaci\'on}
%___________________________________________________________________________________________
%

Los tiempos $T_{n}$ est\'an relacionados con los conteos de $N\left(t\right)$ por

\begin{eqnarray*}
\left\{N\left(t\right)\geq n\right\}&=&\left\{T_{n}\leq t\right\}\\
T_{N\left(t\right)}\leq &t&<T_{N\left(t\right)+1},
\end{eqnarray*}

adem\'as $N\left(T_{n}\right)=n$, y 

\begin{eqnarray*}
N\left(t\right)=\max\left\{n:T_{n}\leq t\right\}=\min\left\{n:T_{n+1}>t\right\}
\end{eqnarray*}

Por propiedades de la convoluci\'on se sabe que

\begin{eqnarray*}
P\left\{T_{n}\leq t\right\}=F^{n\star}\left(t\right)
\end{eqnarray*}
que es la $n$-\'esima convoluci\'on de $F$. Entonces 

\begin{eqnarray*}
\left\{N\left(t\right)\geq n\right\}&=&\left\{T_{n}\leq t\right\}\\
P\left\{N\left(t\right)\leq n\right\}&=&1-F^{\left(n+1\right)\star}\left(t\right)
\end{eqnarray*}

Adem\'as usando el hecho de que $\esp\left[N\left(t\right)\right]=\sum_{n=1}^{\infty}P\left\{N\left(t\right)\geq n\right\}$
se tiene que

\begin{eqnarray*}
\esp\left[N\left(t\right)\right]=\sum_{n=1}^{\infty}F^{n\star}\left(t\right)
\end{eqnarray*}

\begin{Prop}
Para cada $t\geq0$, la funci\'on generadora de momentos $\esp\left[e^{\alpha N\left(t\right)}\right]$ existe para alguna $\alpha$ en una vecindad del 0, y de aqu\'i que $\esp\left[N\left(t\right)^{m}\right]<\infty$, para $m\geq1$.
\end{Prop}


\begin{Note}
Si el primer tiempo de renovaci\'on $\xi_{1}$ no tiene la misma distribuci\'on que el resto de las $\xi_{n}$, para $n\geq2$, a $N\left(t\right)$ se le llama Proceso de Renovaci\'on retardado, donde si $\xi$ tiene distribuci\'on $G$, entonces el tiempo $T_{n}$ de la $n$-\'esima renovaci\'on tiene distribuci\'on $G\star F^{\left(n-1\right)\star}\left(t\right)$
\end{Note}


\begin{Teo}
Para una constante $\mu\leq\infty$ ( o variable aleatoria), las siguientes expresiones son equivalentes:

\begin{eqnarray}
lim_{n\rightarrow\infty}n^{-1}T_{n}&=&\mu,\textrm{ c.s.}\\
lim_{t\rightarrow\infty}t^{-1}N\left(t\right)&=&1/\mu,\textrm{ c.s.}
\end{eqnarray}
\end{Teo}


Es decir, $T_{n}$ satisface la Ley Fuerte de los Grandes N\'umeros s\'i y s\'olo s\'i $N\left/t\right)$ la cumple.


\begin{Coro}[Ley Fuerte de los Grandes N\'umeros para Procesos de Renovaci\'on]
Si $N\left(t\right)$ es un proceso de renovaci\'on cuyos tiempos de inter-renovaci\'on tienen media $\mu\leq\infty$, entonces
\begin{eqnarray}
t^{-1}N\left(t\right)\rightarrow 1/\mu,\textrm{ c.s. cuando }t\rightarrow\infty.
\end{eqnarray}

\end{Coro}


Considerar el proceso estoc\'astico de valores reales $\left\{Z\left(t\right):t\geq0\right\}$ en el mismo espacio de probabilidad que $N\left(t\right)$

\begin{Def}
Para el proceso $\left\{Z\left(t\right):t\geq0\right\}$ se define la fluctuaci\'on m\'axima de $Z\left(t\right)$ en el intervalo $\left(T_{n-1},T_{n}\right]$:
\begin{eqnarray*}
M_{n}=\sup_{T_{n-1}<t\leq T_{n}}|Z\left(t\right)-Z\left(T_{n-1}\right)|
\end{eqnarray*}
\end{Def}

\begin{Teo}
Sup\'ongase que $n^{-1}T_{n}\rightarrow\mu$ c.s. cuando $n\rightarrow\infty$, donde $\mu\leq\infty$ es una constante o variable aleatoria. Sea $a$ una constante o variable aleatoria que puede ser infinita cuando $\mu$ es finita, y considere las expresiones l\'imite:
\begin{eqnarray}
lim_{n\rightarrow\infty}n^{-1}Z\left(T_{n}\right)&=&a,\textrm{ c.s.}\\
lim_{t\rightarrow\infty}t^{-1}Z\left(t\right)&=&a/\mu,\textrm{ c.s.}
\end{eqnarray}
La segunda expresi\'on implica la primera. Conversamente, la primera implica la segunda si el proceso $Z\left(t\right)$ es creciente, o si $lim_{n\rightarrow\infty}n^{-1}M_{n}=0$ c.s.
\end{Teo}

\begin{Coro}
Si $N\left(t\right)$ es un proceso de renovaci\'on, y $\left(Z\left(T_{n}\right)-Z\left(T_{n-1}\right),M_{n}\right)$, para $n\geq1$, son variables aleatorias independientes e id\'enticamente distribuidas con media finita, entonces,
\begin{eqnarray}
lim_{t\rightarrow\infty}t^{-1}Z\left(t\right)\rightarrow\frac{\esp\left[Z\left(T_{1}\right)-Z\left(T_{0}\right)\right]}{\esp\left[T_{1}\right]},\textrm{ c.s. cuando  }t\rightarrow\infty.
\end{eqnarray}
\end{Coro}


%___________________________________________________________________________________________
%
%\subsection{Propiedades de los Procesos de Renovaci\'on}
%___________________________________________________________________________________________
%

Los tiempos $T_{n}$ est\'an relacionados con los conteos de $N\left(t\right)$ por

\begin{eqnarray*}
\left\{N\left(t\right)\geq n\right\}&=&\left\{T_{n}\leq t\right\}\\
T_{N\left(t\right)}\leq &t&<T_{N\left(t\right)+1},
\end{eqnarray*}

adem\'as $N\left(T_{n}\right)=n$, y 

\begin{eqnarray*}
N\left(t\right)=\max\left\{n:T_{n}\leq t\right\}=\min\left\{n:T_{n+1}>t\right\}
\end{eqnarray*}

Por propiedades de la convoluci\'on se sabe que

\begin{eqnarray*}
P\left\{T_{n}\leq t\right\}=F^{n\star}\left(t\right)
\end{eqnarray*}
que es la $n$-\'esima convoluci\'on de $F$. Entonces 

\begin{eqnarray*}
\left\{N\left(t\right)\geq n\right\}&=&\left\{T_{n}\leq t\right\}\\
P\left\{N\left(t\right)\leq n\right\}&=&1-F^{\left(n+1\right)\star}\left(t\right)
\end{eqnarray*}

Adem\'as usando el hecho de que $\esp\left[N\left(t\right)\right]=\sum_{n=1}^{\infty}P\left\{N\left(t\right)\geq n\right\}$
se tiene que

\begin{eqnarray*}
\esp\left[N\left(t\right)\right]=\sum_{n=1}^{\infty}F^{n\star}\left(t\right)
\end{eqnarray*}

\begin{Prop}
Para cada $t\geq0$, la funci\'on generadora de momentos $\esp\left[e^{\alpha N\left(t\right)}\right]$ existe para alguna $\alpha$ en una vecindad del 0, y de aqu\'i que $\esp\left[N\left(t\right)^{m}\right]<\infty$, para $m\geq1$.
\end{Prop}


\begin{Note}
Si el primer tiempo de renovaci\'on $\xi_{1}$ no tiene la misma distribuci\'on que el resto de las $\xi_{n}$, para $n\geq2$, a $N\left(t\right)$ se le llama Proceso de Renovaci\'on retardado, donde si $\xi$ tiene distribuci\'on $G$, entonces el tiempo $T_{n}$ de la $n$-\'esima renovaci\'on tiene distribuci\'on $G\star F^{\left(n-1\right)\star}\left(t\right)$
\end{Note}


\begin{Teo}
Para una constante $\mu\leq\infty$ ( o variable aleatoria), las siguientes expresiones son equivalentes:

\begin{eqnarray}
lim_{n\rightarrow\infty}n^{-1}T_{n}&=&\mu,\textrm{ c.s.}\\
lim_{t\rightarrow\infty}t^{-1}N\left(t\right)&=&1/\mu,\textrm{ c.s.}
\end{eqnarray}
\end{Teo}


Es decir, $T_{n}$ satisface la Ley Fuerte de los Grandes N\'umeros s\'i y s\'olo s\'i $N\left/t\right)$ la cumple.


\begin{Coro}[Ley Fuerte de los Grandes N\'umeros para Procesos de Renovaci\'on]
Si $N\left(t\right)$ es un proceso de renovaci\'on cuyos tiempos de inter-renovaci\'on tienen media $\mu\leq\infty$, entonces
\begin{eqnarray}
t^{-1}N\left(t\right)\rightarrow 1/\mu,\textrm{ c.s. cuando }t\rightarrow\infty.
\end{eqnarray}

\end{Coro}


Considerar el proceso estoc\'astico de valores reales $\left\{Z\left(t\right):t\geq0\right\}$ en el mismo espacio de probabilidad que $N\left(t\right)$

\begin{Def}
Para el proceso $\left\{Z\left(t\right):t\geq0\right\}$ se define la fluctuaci\'on m\'axima de $Z\left(t\right)$ en el intervalo $\left(T_{n-1},T_{n}\right]$:
\begin{eqnarray*}
M_{n}=\sup_{T_{n-1}<t\leq T_{n}}|Z\left(t\right)-Z\left(T_{n-1}\right)|
\end{eqnarray*}
\end{Def}

\begin{Teo}
Sup\'ongase que $n^{-1}T_{n}\rightarrow\mu$ c.s. cuando $n\rightarrow\infty$, donde $\mu\leq\infty$ es una constante o variable aleatoria. Sea $a$ una constante o variable aleatoria que puede ser infinita cuando $\mu$ es finita, y considere las expresiones l\'imite:
\begin{eqnarray}
lim_{n\rightarrow\infty}n^{-1}Z\left(T_{n}\right)&=&a,\textrm{ c.s.}\\
lim_{t\rightarrow\infty}t^{-1}Z\left(t\right)&=&a/\mu,\textrm{ c.s.}
\end{eqnarray}
La segunda expresi\'on implica la primera. Conversamente, la primera implica la segunda si el proceso $Z\left(t\right)$ es creciente, o si $lim_{n\rightarrow\infty}n^{-1}M_{n}=0$ c.s.
\end{Teo}

\begin{Coro}
Si $N\left(t\right)$ es un proceso de renovaci\'on, y $\left(Z\left(T_{n}\right)-Z\left(T_{n-1}\right),M_{n}\right)$, para $n\geq1$, son variables aleatorias independientes e id\'enticamente distribuidas con media finita, entonces,
\begin{eqnarray}
lim_{t\rightarrow\infty}t^{-1}Z\left(t\right)\rightarrow\frac{\esp\left[Z\left(T_{1}\right)-Z\left(T_{0}\right)\right]}{\esp\left[T_{1}\right]},\textrm{ c.s. cuando  }t\rightarrow\infty.
\end{eqnarray}
\end{Coro}

%___________________________________________________________________________________________
%
%\subsection{Propiedades de los Procesos de Renovaci\'on}
%___________________________________________________________________________________________
%

Los tiempos $T_{n}$ est\'an relacionados con los conteos de $N\left(t\right)$ por

\begin{eqnarray*}
\left\{N\left(t\right)\geq n\right\}&=&\left\{T_{n}\leq t\right\}\\
T_{N\left(t\right)}\leq &t&<T_{N\left(t\right)+1},
\end{eqnarray*}

adem\'as $N\left(T_{n}\right)=n$, y 

\begin{eqnarray*}
N\left(t\right)=\max\left\{n:T_{n}\leq t\right\}=\min\left\{n:T_{n+1}>t\right\}
\end{eqnarray*}

Por propiedades de la convoluci\'on se sabe que

\begin{eqnarray*}
P\left\{T_{n}\leq t\right\}=F^{n\star}\left(t\right)
\end{eqnarray*}
que es la $n$-\'esima convoluci\'on de $F$. Entonces 

\begin{eqnarray*}
\left\{N\left(t\right)\geq n\right\}&=&\left\{T_{n}\leq t\right\}\\
P\left\{N\left(t\right)\leq n\right\}&=&1-F^{\left(n+1\right)\star}\left(t\right)
\end{eqnarray*}

Adem\'as usando el hecho de que $\esp\left[N\left(t\right)\right]=\sum_{n=1}^{\infty}P\left\{N\left(t\right)\geq n\right\}$
se tiene que

\begin{eqnarray*}
\esp\left[N\left(t\right)\right]=\sum_{n=1}^{\infty}F^{n\star}\left(t\right)
\end{eqnarray*}

\begin{Prop}
Para cada $t\geq0$, la funci\'on generadora de momentos $\esp\left[e^{\alpha N\left(t\right)}\right]$ existe para alguna $\alpha$ en una vecindad del 0, y de aqu\'i que $\esp\left[N\left(t\right)^{m}\right]<\infty$, para $m\geq1$.
\end{Prop}


\begin{Note}
Si el primer tiempo de renovaci\'on $\xi_{1}$ no tiene la misma distribuci\'on que el resto de las $\xi_{n}$, para $n\geq2$, a $N\left(t\right)$ se le llama Proceso de Renovaci\'on retardado, donde si $\xi$ tiene distribuci\'on $G$, entonces el tiempo $T_{n}$ de la $n$-\'esima renovaci\'on tiene distribuci\'on $G\star F^{\left(n-1\right)\star}\left(t\right)$
\end{Note}


\begin{Teo}
Para una constante $\mu\leq\infty$ ( o variable aleatoria), las siguientes expresiones son equivalentes:

\begin{eqnarray}
lim_{n\rightarrow\infty}n^{-1}T_{n}&=&\mu,\textrm{ c.s.}\\
lim_{t\rightarrow\infty}t^{-1}N\left(t\right)&=&1/\mu,\textrm{ c.s.}
\end{eqnarray}
\end{Teo}


Es decir, $T_{n}$ satisface la Ley Fuerte de los Grandes N\'umeros s\'i y s\'olo s\'i $N\left/t\right)$ la cumple.


\begin{Coro}[Ley Fuerte de los Grandes N\'umeros para Procesos de Renovaci\'on]
Si $N\left(t\right)$ es un proceso de renovaci\'on cuyos tiempos de inter-renovaci\'on tienen media $\mu\leq\infty$, entonces
\begin{eqnarray}
t^{-1}N\left(t\right)\rightarrow 1/\mu,\textrm{ c.s. cuando }t\rightarrow\infty.
\end{eqnarray}

\end{Coro}


Considerar el proceso estoc\'astico de valores reales $\left\{Z\left(t\right):t\geq0\right\}$ en el mismo espacio de probabilidad que $N\left(t\right)$

\begin{Def}
Para el proceso $\left\{Z\left(t\right):t\geq0\right\}$ se define la fluctuaci\'on m\'axima de $Z\left(t\right)$ en el intervalo $\left(T_{n-1},T_{n}\right]$:
\begin{eqnarray*}
M_{n}=\sup_{T_{n-1}<t\leq T_{n}}|Z\left(t\right)-Z\left(T_{n-1}\right)|
\end{eqnarray*}
\end{Def}

\begin{Teo}
Sup\'ongase que $n^{-1}T_{n}\rightarrow\mu$ c.s. cuando $n\rightarrow\infty$, donde $\mu\leq\infty$ es una constante o variable aleatoria. Sea $a$ una constante o variable aleatoria que puede ser infinita cuando $\mu$ es finita, y considere las expresiones l\'imite:
\begin{eqnarray}
lim_{n\rightarrow\infty}n^{-1}Z\left(T_{n}\right)&=&a,\textrm{ c.s.}\\
lim_{t\rightarrow\infty}t^{-1}Z\left(t\right)&=&a/\mu,\textrm{ c.s.}
\end{eqnarray}
La segunda expresi\'on implica la primera. Conversamente, la primera implica la segunda si el proceso $Z\left(t\right)$ es creciente, o si $lim_{n\rightarrow\infty}n^{-1}M_{n}=0$ c.s.
\end{Teo}

\begin{Coro}
Si $N\left(t\right)$ es un proceso de renovaci\'on, y $\left(Z\left(T_{n}\right)-Z\left(T_{n-1}\right),M_{n}\right)$, para $n\geq1$, son variables aleatorias independientes e id\'enticamente distribuidas con media finita, entonces,
\begin{eqnarray}
lim_{t\rightarrow\infty}t^{-1}Z\left(t\right)\rightarrow\frac{\esp\left[Z\left(T_{1}\right)-Z\left(T_{0}\right)\right]}{\esp\left[T_{1}\right]},\textrm{ c.s. cuando  }t\rightarrow\infty.
\end{eqnarray}
\end{Coro}
%___________________________________________________________________________________________
%
%\subsection{Propiedades de los Procesos de Renovaci\'on}
%___________________________________________________________________________________________
%

Los tiempos $T_{n}$ est\'an relacionados con los conteos de $N\left(t\right)$ por

\begin{eqnarray*}
\left\{N\left(t\right)\geq n\right\}&=&\left\{T_{n}\leq t\right\}\\
T_{N\left(t\right)}\leq &t&<T_{N\left(t\right)+1},
\end{eqnarray*}

adem\'as $N\left(T_{n}\right)=n$, y 

\begin{eqnarray*}
N\left(t\right)=\max\left\{n:T_{n}\leq t\right\}=\min\left\{n:T_{n+1}>t\right\}
\end{eqnarray*}

Por propiedades de la convoluci\'on se sabe que

\begin{eqnarray*}
P\left\{T_{n}\leq t\right\}=F^{n\star}\left(t\right)
\end{eqnarray*}
que es la $n$-\'esima convoluci\'on de $F$. Entonces 

\begin{eqnarray*}
\left\{N\left(t\right)\geq n\right\}&=&\left\{T_{n}\leq t\right\}\\
P\left\{N\left(t\right)\leq n\right\}&=&1-F^{\left(n+1\right)\star}\left(t\right)
\end{eqnarray*}

Adem\'as usando el hecho de que $\esp\left[N\left(t\right)\right]=\sum_{n=1}^{\infty}P\left\{N\left(t\right)\geq n\right\}$
se tiene que

\begin{eqnarray*}
\esp\left[N\left(t\right)\right]=\sum_{n=1}^{\infty}F^{n\star}\left(t\right)
\end{eqnarray*}

\begin{Prop}
Para cada $t\geq0$, la funci\'on generadora de momentos $\esp\left[e^{\alpha N\left(t\right)}\right]$ existe para alguna $\alpha$ en una vecindad del 0, y de aqu\'i que $\esp\left[N\left(t\right)^{m}\right]<\infty$, para $m\geq1$.
\end{Prop}


\begin{Note}
Si el primer tiempo de renovaci\'on $\xi_{1}$ no tiene la misma distribuci\'on que el resto de las $\xi_{n}$, para $n\geq2$, a $N\left(t\right)$ se le llama Proceso de Renovaci\'on retardado, donde si $\xi$ tiene distribuci\'on $G$, entonces el tiempo $T_{n}$ de la $n$-\'esima renovaci\'on tiene distribuci\'on $G\star F^{\left(n-1\right)\star}\left(t\right)$
\end{Note}


\begin{Teo}
Para una constante $\mu\leq\infty$ ( o variable aleatoria), las siguientes expresiones son equivalentes:

\begin{eqnarray}
lim_{n\rightarrow\infty}n^{-1}T_{n}&=&\mu,\textrm{ c.s.}\\
lim_{t\rightarrow\infty}t^{-1}N\left(t\right)&=&1/\mu,\textrm{ c.s.}
\end{eqnarray}
\end{Teo}


Es decir, $T_{n}$ satisface la Ley Fuerte de los Grandes N\'umeros s\'i y s\'olo s\'i $N\left/t\right)$ la cumple.


\begin{Coro}[Ley Fuerte de los Grandes N\'umeros para Procesos de Renovaci\'on]
Si $N\left(t\right)$ es un proceso de renovaci\'on cuyos tiempos de inter-renovaci\'on tienen media $\mu\leq\infty$, entonces
\begin{eqnarray}
t^{-1}N\left(t\right)\rightarrow 1/\mu,\textrm{ c.s. cuando }t\rightarrow\infty.
\end{eqnarray}

\end{Coro}


Considerar el proceso estoc\'astico de valores reales $\left\{Z\left(t\right):t\geq0\right\}$ en el mismo espacio de probabilidad que $N\left(t\right)$

\begin{Def}
Para el proceso $\left\{Z\left(t\right):t\geq0\right\}$ se define la fluctuaci\'on m\'axima de $Z\left(t\right)$ en el intervalo $\left(T_{n-1},T_{n}\right]$:
\begin{eqnarray*}
M_{n}=\sup_{T_{n-1}<t\leq T_{n}}|Z\left(t\right)-Z\left(T_{n-1}\right)|
\end{eqnarray*}
\end{Def}

\begin{Teo}
Sup\'ongase que $n^{-1}T_{n}\rightarrow\mu$ c.s. cuando $n\rightarrow\infty$, donde $\mu\leq\infty$ es una constante o variable aleatoria. Sea $a$ una constante o variable aleatoria que puede ser infinita cuando $\mu$ es finita, y considere las expresiones l\'imite:
\begin{eqnarray}
lim_{n\rightarrow\infty}n^{-1}Z\left(T_{n}\right)&=&a,\textrm{ c.s.}\\
lim_{t\rightarrow\infty}t^{-1}Z\left(t\right)&=&a/\mu,\textrm{ c.s.}
\end{eqnarray}
La segunda expresi\'on implica la primera. Conversamente, la primera implica la segunda si el proceso $Z\left(t\right)$ es creciente, o si $lim_{n\rightarrow\infty}n^{-1}M_{n}=0$ c.s.
\end{Teo}

\begin{Coro}
Si $N\left(t\right)$ es un proceso de renovaci\'on, y $\left(Z\left(T_{n}\right)-Z\left(T_{n-1}\right),M_{n}\right)$, para $n\geq1$, son variables aleatorias independientes e id\'enticamente distribuidas con media finita, entonces,
\begin{eqnarray}
lim_{t\rightarrow\infty}t^{-1}Z\left(t\right)\rightarrow\frac{\esp\left[Z\left(T_{1}\right)-Z\left(T_{0}\right)\right]}{\esp\left[T_{1}\right]},\textrm{ c.s. cuando  }t\rightarrow\infty.
\end{eqnarray}
\end{Coro}


%___________________________________________________________________________________________
%
%\subsection{Funci\'on de Renovaci\'on}
%___________________________________________________________________________________________
%


\begin{Def}
Sea $h\left(t\right)$ funci\'on de valores reales en $\rea$ acotada en intervalos finitos e igual a cero para $t<0$ La ecuaci\'on de renovaci\'on para $h\left(t\right)$ y la distribuci\'on $F$ es

\begin{eqnarray}%\label{Ec.Renovacion}
H\left(t\right)=h\left(t\right)+\int_{\left[0,t\right]}H\left(t-s\right)dF\left(s\right)\textrm{,    }t\geq0,
\end{eqnarray}
donde $H\left(t\right)$ es una funci\'on de valores reales. Esto es $H=h+F\star H$. Decimos que $H\left(t\right)$ es soluci\'on de esta ecuaci\'on si satisface la ecuaci\'on, y es acotada en intervalos finitos e iguales a cero para $t<0$.
\end{Def}

\begin{Prop}
La funci\'on $U\star h\left(t\right)$ es la \'unica soluci\'on de la ecuaci\'on de renovaci\'on (\ref{Ec.Renovacion}).
\end{Prop}

\begin{Teo}[Teorema Renovaci\'on Elemental]
\begin{eqnarray*}
t^{-1}U\left(t\right)\rightarrow 1/\mu\textrm{,    cuando }t\rightarrow\infty.
\end{eqnarray*}
\end{Teo}

%___________________________________________________________________________________________
%
%\subsection{Funci\'on de Renovaci\'on}
%___________________________________________________________________________________________
%


Sup\'ongase que $N\left(t\right)$ es un proceso de renovaci\'on con distribuci\'on $F$ con media finita $\mu$.

\begin{Def}
La funci\'on de renovaci\'on asociada con la distribuci\'on $F$, del proceso $N\left(t\right)$, es
\begin{eqnarray*}
U\left(t\right)=\sum_{n=1}^{\infty}F^{n\star}\left(t\right),\textrm{   }t\geq0,
\end{eqnarray*}
donde $F^{0\star}\left(t\right)=\indora\left(t\geq0\right)$.
\end{Def}


\begin{Prop}
Sup\'ongase que la distribuci\'on de inter-renovaci\'on $F$ tiene densidad $f$. Entonces $U\left(t\right)$ tambi\'en tiene densidad, para $t>0$, y es $U^{'}\left(t\right)=\sum_{n=0}^{\infty}f^{n\star}\left(t\right)$. Adem\'as
\begin{eqnarray*}
\prob\left\{N\left(t\right)>N\left(t-\right)\right\}=0\textrm{,   }t\geq0.
\end{eqnarray*}
\end{Prop}

\begin{Def}
La Transformada de Laplace-Stieljes de $F$ est\'a dada por

\begin{eqnarray*}
\hat{F}\left(\alpha\right)=\int_{\rea_{+}}e^{-\alpha t}dF\left(t\right)\textrm{,  }\alpha\geq0.
\end{eqnarray*}
\end{Def}

Entonces

\begin{eqnarray*}
\hat{U}\left(\alpha\right)=\sum_{n=0}^{\infty}\hat{F^{n\star}}\left(\alpha\right)=\sum_{n=0}^{\infty}\hat{F}\left(\alpha\right)^{n}=\frac{1}{1-\hat{F}\left(\alpha\right)}.
\end{eqnarray*}


\begin{Prop}
La Transformada de Laplace $\hat{U}\left(\alpha\right)$ y $\hat{F}\left(\alpha\right)$ determina una a la otra de manera \'unica por la relaci\'on $\hat{U}\left(\alpha\right)=\frac{1}{1-\hat{F}\left(\alpha\right)}$.
\end{Prop}


\begin{Note}
Un proceso de renovaci\'on $N\left(t\right)$ cuyos tiempos de inter-renovaci\'on tienen media finita, es un proceso Poisson con tasa $\lambda$ si y s\'olo s\'i $\esp\left[U\left(t\right)\right]=\lambda t$, para $t\geq0$.
\end{Note}


\begin{Teo}
Sea $N\left(t\right)$ un proceso puntual simple con puntos de localizaci\'on $T_{n}$ tal que $\eta\left(t\right)=\esp\left[N\left(\right)\right]$ es finita para cada $t$. Entonces para cualquier funci\'on $f:\rea_{+}\rightarrow\rea$,
\begin{eqnarray*}
\esp\left[\sum_{n=1}^{N\left(\right)}f\left(T_{n}\right)\right]=\int_{\left(0,t\right]}f\left(s\right)d\eta\left(s\right)\textrm{,  }t\geq0,
\end{eqnarray*}
suponiendo que la integral exista. Adem\'as si $X_{1},X_{2},\ldots$ son variables aleatorias definidas en el mismo espacio de probabilidad que el proceso $N\left(t\right)$ tal que $\esp\left[X_{n}|T_{n}=s\right]=f\left(s\right)$, independiente de $n$. Entonces
\begin{eqnarray*}
\esp\left[\sum_{n=1}^{N\left(t\right)}X_{n}\right]=\int_{\left(0,t\right]}f\left(s\right)d\eta\left(s\right)\textrm{,  }t\geq0,
\end{eqnarray*} 
suponiendo que la integral exista. 
\end{Teo}

\begin{Coro}[Identidad de Wald para Renovaciones]
Para el proceso de renovaci\'on $N\left(t\right)$,
\begin{eqnarray*}
\esp\left[T_{N\left(t\right)+1}\right]=\mu\esp\left[N\left(t\right)+1\right]\textrm{,  }t\geq0,
\end{eqnarray*}  
\end{Coro}

%______________________________________________________________________
%\subsection{Procesos de Renovaci\'on}
%______________________________________________________________________

\begin{Def}%\label{Def.Tn}
Sean $0\leq T_{1}\leq T_{2}\leq \ldots$ son tiempos aleatorios infinitos en los cuales ocurren ciertos eventos. El n\'umero de tiempos $T_{n}$ en el intervalo $\left[0,t\right)$ es

\begin{eqnarray}
N\left(t\right)=\sum_{n=1}^{\infty}\indora\left(T_{n}\leq t\right),
\end{eqnarray}
para $t\geq0$.
\end{Def}

Si se consideran los puntos $T_{n}$ como elementos de $\rea_{+}$, y $N\left(t\right)$ es el n\'umero de puntos en $\rea$. El proceso denotado por $\left\{N\left(t\right):t\geq0\right\}$, denotado por $N\left(t\right)$, es un proceso puntual en $\rea_{+}$. Los $T_{n}$ son los tiempos de ocurrencia, el proceso puntual $N\left(t\right)$ es simple si su n\'umero de ocurrencias son distintas: $0<T_{1}<T_{2}<\ldots$ casi seguramente.

\begin{Def}
Un proceso puntual $N\left(t\right)$ es un proceso de renovaci\'on si los tiempos de interocurrencia $\xi_{n}=T_{n}-T_{n-1}$, para $n\geq1$, son independientes e identicamente distribuidos con distribuci\'on $F$, donde $F\left(0\right)=0$ y $T_{0}=0$. Los $T_{n}$ son llamados tiempos de renovaci\'on, referente a la independencia o renovaci\'on de la informaci\'on estoc\'astica en estos tiempos. Los $\xi_{n}$ son los tiempos de inter-renovaci\'on, y $N\left(t\right)$ es el n\'umero de renovaciones en el intervalo $\left[0,t\right)$
\end{Def}


\begin{Note}
Para definir un proceso de renovaci\'on para cualquier contexto, solamente hay que especificar una distribuci\'on $F$, con $F\left(0\right)=0$, para los tiempos de inter-renovaci\'on. La funci\'on $F$ en turno degune las otra variables aleatorias. De manera formal, existe un espacio de probabilidad y una sucesi\'on de variables aleatorias $\xi_{1},\xi_{2},\ldots$ definidas en este con distribuci\'on $F$. Entonces las otras cantidades son $T_{n}=\sum_{k=1}^{n}\xi_{k}$ y $N\left(t\right)=\sum_{n=1}^{\infty}\indora\left(T_{n}\leq t\right)$, donde $T_{n}\rightarrow\infty$ casi seguramente por la Ley Fuerte de los Grandes Números.
\end{Note}

%___________________________________________________________________________________________
%
%\subsection{Renewal and Regenerative Processes: Serfozo\cite{Serfozo}}
%___________________________________________________________________________________________
%
\begin{Def}%\label{Def.Tn}
Sean $0\leq T_{1}\leq T_{2}\leq \ldots$ son tiempos aleatorios infinitos en los cuales ocurren ciertos eventos. El n\'umero de tiempos $T_{n}$ en el intervalo $\left[0,t\right)$ es

\begin{eqnarray}
N\left(t\right)=\sum_{n=1}^{\infty}\indora\left(T_{n}\leq t\right),
\end{eqnarray}
para $t\geq0$.
\end{Def}

Si se consideran los puntos $T_{n}$ como elementos de $\rea_{+}$, y $N\left(t\right)$ es el n\'umero de puntos en $\rea$. El proceso denotado por $\left\{N\left(t\right):t\geq0\right\}$, denotado por $N\left(t\right)$, es un proceso puntual en $\rea_{+}$. Los $T_{n}$ son los tiempos de ocurrencia, el proceso puntual $N\left(t\right)$ es simple si su n\'umero de ocurrencias son distintas: $0<T_{1}<T_{2}<\ldots$ casi seguramente.

\begin{Def}
Un proceso puntual $N\left(t\right)$ es un proceso de renovaci\'on si los tiempos de interocurrencia $\xi_{n}=T_{n}-T_{n-1}$, para $n\geq1$, son independientes e identicamente distribuidos con distribuci\'on $F$, donde $F\left(0\right)=0$ y $T_{0}=0$. Los $T_{n}$ son llamados tiempos de renovaci\'on, referente a la independencia o renovaci\'on de la informaci\'on estoc\'astica en estos tiempos. Los $\xi_{n}$ son los tiempos de inter-renovaci\'on, y $N\left(t\right)$ es el n\'umero de renovaciones en el intervalo $\left[0,t\right)$
\end{Def}


\begin{Note}
Para definir un proceso de renovaci\'on para cualquier contexto, solamente hay que especificar una distribuci\'on $F$, con $F\left(0\right)=0$, para los tiempos de inter-renovaci\'on. La funci\'on $F$ en turno degune las otra variables aleatorias. De manera formal, existe un espacio de probabilidad y una sucesi\'on de variables aleatorias $\xi_{1},\xi_{2},\ldots$ definidas en este con distribuci\'on $F$. Entonces las otras cantidades son $T_{n}=\sum_{k=1}^{n}\xi_{k}$ y $N\left(t\right)=\sum_{n=1}^{\infty}\indora\left(T_{n}\leq t\right)$, donde $T_{n}\rightarrow\infty$ casi seguramente por la Ley Fuerte de los Grandes N\'umeros.
\end{Note}







Los tiempos $T_{n}$ est\'an relacionados con los conteos de $N\left(t\right)$ por

\begin{eqnarray*}
\left\{N\left(t\right)\geq n\right\}&=&\left\{T_{n}\leq t\right\}\\
T_{N\left(t\right)}\leq &t&<T_{N\left(t\right)+1},
\end{eqnarray*}

adem\'as $N\left(T_{n}\right)=n$, y 

\begin{eqnarray*}
N\left(t\right)=\max\left\{n:T_{n}\leq t\right\}=\min\left\{n:T_{n+1}>t\right\}
\end{eqnarray*}

Por propiedades de la convoluci\'on se sabe que

\begin{eqnarray*}
P\left\{T_{n}\leq t\right\}=F^{n\star}\left(t\right)
\end{eqnarray*}
que es la $n$-\'esima convoluci\'on de $F$. Entonces 

\begin{eqnarray*}
\left\{N\left(t\right)\geq n\right\}&=&\left\{T_{n}\leq t\right\}\\
P\left\{N\left(t\right)\leq n\right\}&=&1-F^{\left(n+1\right)\star}\left(t\right)
\end{eqnarray*}

Adem\'as usando el hecho de que $\esp\left[N\left(t\right)\right]=\sum_{n=1}^{\infty}P\left\{N\left(t\right)\geq n\right\}$
se tiene que

\begin{eqnarray*}
\esp\left[N\left(t\right)\right]=\sum_{n=1}^{\infty}F^{n\star}\left(t\right)
\end{eqnarray*}

\begin{Prop}
Para cada $t\geq0$, la funci\'on generadora de momentos $\esp\left[e^{\alpha N\left(t\right)}\right]$ existe para alguna $\alpha$ en una vecindad del 0, y de aqu\'i que $\esp\left[N\left(t\right)^{m}\right]<\infty$, para $m\geq1$.
\end{Prop}

\begin{Ejem}[\textbf{Proceso Poisson}]

Suponga que se tienen tiempos de inter-renovaci\'on \textit{i.i.d.} del proceso de renovaci\'on $N\left(t\right)$ tienen distribuci\'on exponencial $F\left(t\right)=q-e^{-\lambda t}$ con tasa $\lambda$. Entonces $N\left(t\right)$ es un proceso Poisson con tasa $\lambda$.

\end{Ejem}


\begin{Note}
Si el primer tiempo de renovaci\'on $\xi_{1}$ no tiene la misma distribuci\'on que el resto de las $\xi_{n}$, para $n\geq2$, a $N\left(t\right)$ se le llama Proceso de Renovaci\'on retardado, donde si $\xi$ tiene distribuci\'on $G$, entonces el tiempo $T_{n}$ de la $n$-\'esima renovaci\'on tiene distribuci\'on $G\star F^{\left(n-1\right)\star}\left(t\right)$
\end{Note}


\begin{Teo}
Para una constante $\mu\leq\infty$ ( o variable aleatoria), las siguientes expresiones son equivalentes:

\begin{eqnarray}
lim_{n\rightarrow\infty}n^{-1}T_{n}&=&\mu,\textrm{ c.s.}\\
lim_{t\rightarrow\infty}t^{-1}N\left(t\right)&=&1/\mu,\textrm{ c.s.}
\end{eqnarray}
\end{Teo}


Es decir, $T_{n}$ satisface la Ley Fuerte de los Grandes N\'umeros s\'i y s\'olo s\'i $N\left/t\right)$ la cumple.


\begin{Coro}[Ley Fuerte de los Grandes N\'umeros para Procesos de Renovaci\'on]
Si $N\left(t\right)$ es un proceso de renovaci\'on cuyos tiempos de inter-renovaci\'on tienen media $\mu\leq\infty$, entonces
\begin{eqnarray}
t^{-1}N\left(t\right)\rightarrow 1/\mu,\textrm{ c.s. cuando }t\rightarrow\infty.
\end{eqnarray}

\end{Coro}


Considerar el proceso estoc\'astico de valores reales $\left\{Z\left(t\right):t\geq0\right\}$ en el mismo espacio de probabilidad que $N\left(t\right)$

\begin{Def}
Para el proceso $\left\{Z\left(t\right):t\geq0\right\}$ se define la fluctuaci\'on m\'axima de $Z\left(t\right)$ en el intervalo $\left(T_{n-1},T_{n}\right]$:
\begin{eqnarray*}
M_{n}=\sup_{T_{n-1}<t\leq T_{n}}|Z\left(t\right)-Z\left(T_{n-1}\right)|
\end{eqnarray*}
\end{Def}

\begin{Teo}
Sup\'ongase que $n^{-1}T_{n}\rightarrow\mu$ c.s. cuando $n\rightarrow\infty$, donde $\mu\leq\infty$ es una constante o variable aleatoria. Sea $a$ una constante o variable aleatoria que puede ser infinita cuando $\mu$ es finita, y considere las expresiones l\'imite:
\begin{eqnarray}
lim_{n\rightarrow\infty}n^{-1}Z\left(T_{n}\right)&=&a,\textrm{ c.s.}\\
lim_{t\rightarrow\infty}t^{-1}Z\left(t\right)&=&a/\mu,\textrm{ c.s.}
\end{eqnarray}
La segunda expresi\'on implica la primera. Conversamente, la primera implica la segunda si el proceso $Z\left(t\right)$ es creciente, o si $lim_{n\rightarrow\infty}n^{-1}M_{n}=0$ c.s.
\end{Teo}

\begin{Coro}
Si $N\left(t\right)$ es un proceso de renovaci\'on, y $\left(Z\left(T_{n}\right)-Z\left(T_{n-1}\right),M_{n}\right)$, para $n\geq1$, son variables aleatorias independientes e id\'enticamente distribuidas con media finita, entonces,
\begin{eqnarray}
lim_{t\rightarrow\infty}t^{-1}Z\left(t\right)\rightarrow\frac{\esp\left[Z\left(T_{1}\right)-Z\left(T_{0}\right)\right]}{\esp\left[T_{1}\right]},\textrm{ c.s. cuando  }t\rightarrow\infty.
\end{eqnarray}
\end{Coro}


Sup\'ongase que $N\left(t\right)$ es un proceso de renovaci\'on con distribuci\'on $F$ con media finita $\mu$.

\begin{Def}
La funci\'on de renovaci\'on asociada con la distribuci\'on $F$, del proceso $N\left(t\right)$, es
\begin{eqnarray*}
U\left(t\right)=\sum_{n=1}^{\infty}F^{n\star}\left(t\right),\textrm{   }t\geq0,
\end{eqnarray*}
donde $F^{0\star}\left(t\right)=\indora\left(t\geq0\right)$.
\end{Def}


\begin{Prop}
Sup\'ongase que la distribuci\'on de inter-renovaci\'on $F$ tiene densidad $f$. Entonces $U\left(t\right)$ tambi\'en tiene densidad, para $t>0$, y es $U^{'}\left(t\right)=\sum_{n=0}^{\infty}f^{n\star}\left(t\right)$. Adem\'as
\begin{eqnarray*}
\prob\left\{N\left(t\right)>N\left(t-\right)\right\}=0\textrm{,   }t\geq0.
\end{eqnarray*}
\end{Prop}

\begin{Def}
La Transformada de Laplace-Stieljes de $F$ est\'a dada por

\begin{eqnarray*}
\hat{F}\left(\alpha\right)=\int_{\rea_{+}}e^{-\alpha t}dF\left(t\right)\textrm{,  }\alpha\geq0.
\end{eqnarray*}
\end{Def}

Entonces

\begin{eqnarray*}
\hat{U}\left(\alpha\right)=\sum_{n=0}^{\infty}\hat{F^{n\star}}\left(\alpha\right)=\sum_{n=0}^{\infty}\hat{F}\left(\alpha\right)^{n}=\frac{1}{1-\hat{F}\left(\alpha\right)}.
\end{eqnarray*}


\begin{Prop}
La Transformada de Laplace $\hat{U}\left(\alpha\right)$ y $\hat{F}\left(\alpha\right)$ determina una a la otra de manera \'unica por la relaci\'on $\hat{U}\left(\alpha\right)=\frac{1}{1-\hat{F}\left(\alpha\right)}$.
\end{Prop}


\begin{Note}
Un proceso de renovaci\'on $N\left(t\right)$ cuyos tiempos de inter-renovaci\'on tienen media finita, es un proceso Poisson con tasa $\lambda$ si y s\'olo s\'i $\esp\left[U\left(t\right)\right]=\lambda t$, para $t\geq0$.
\end{Note}


\begin{Teo}
Sea $N\left(t\right)$ un proceso puntual simple con puntos de localizaci\'on $T_{n}$ tal que $\eta\left(t\right)=\esp\left[N\left(\right)\right]$ es finita para cada $t$. Entonces para cualquier funci\'on $f:\rea_{+}\rightarrow\rea$,
\begin{eqnarray*}
\esp\left[\sum_{n=1}^{N\left(\right)}f\left(T_{n}\right)\right]=\int_{\left(0,t\right]}f\left(s\right)d\eta\left(s\right)\textrm{,  }t\geq0,
\end{eqnarray*}
suponiendo que la integral exista. Adem\'as si $X_{1},X_{2},\ldots$ son variables aleatorias definidas en el mismo espacio de probabilidad que el proceso $N\left(t\right)$ tal que $\esp\left[X_{n}|T_{n}=s\right]=f\left(s\right)$, independiente de $n$. Entonces
\begin{eqnarray*}
\esp\left[\sum_{n=1}^{N\left(t\right)}X_{n}\right]=\int_{\left(0,t\right]}f\left(s\right)d\eta\left(s\right)\textrm{,  }t\geq0,
\end{eqnarray*} 
suponiendo que la integral exista. 
\end{Teo}

\begin{Coro}[Identidad de Wald para Renovaciones]
Para el proceso de renovaci\'on $N\left(t\right)$,
\begin{eqnarray*}
\esp\left[T_{N\left(t\right)+1}\right]=\mu\esp\left[N\left(t\right)+1\right]\textrm{,  }t\geq0,
\end{eqnarray*}  
\end{Coro}


\begin{Def}
Sea $h\left(t\right)$ funci\'on de valores reales en $\rea$ acotada en intervalos finitos e igual a cero para $t<0$ La ecuaci\'on de renovaci\'on para $h\left(t\right)$ y la distribuci\'on $F$ es

\begin{eqnarray}%\label{Ec.Renovacion}
H\left(t\right)=h\left(t\right)+\int_{\left[0,t\right]}H\left(t-s\right)dF\left(s\right)\textrm{,    }t\geq0,
\end{eqnarray}
donde $H\left(t\right)$ es una funci\'on de valores reales. Esto es $H=h+F\star H$. Decimos que $H\left(t\right)$ es soluci\'on de esta ecuaci\'on si satisface la ecuaci\'on, y es acotada en intervalos finitos e iguales a cero para $t<0$.
\end{Def}

\begin{Prop}
La funci\'on $U\star h\left(t\right)$ es la \'unica soluci\'on de la ecuaci\'on de renovaci\'on (\ref{Ec.Renovacion}).
\end{Prop}

\begin{Teo}[Teorema Renovaci\'on Elemental]
\begin{eqnarray*}
t^{-1}U\left(t\right)\rightarrow 1/\mu\textrm{,    cuando }t\rightarrow\infty.
\end{eqnarray*}
\end{Teo}



Sup\'ongase que $N\left(t\right)$ es un proceso de renovaci\'on con distribuci\'on $F$ con media finita $\mu$.

\begin{Def}
La funci\'on de renovaci\'on asociada con la distribuci\'on $F$, del proceso $N\left(t\right)$, es
\begin{eqnarray*}
U\left(t\right)=\sum_{n=1}^{\infty}F^{n\star}\left(t\right),\textrm{   }t\geq0,
\end{eqnarray*}
donde $F^{0\star}\left(t\right)=\indora\left(t\geq0\right)$.
\end{Def}


\begin{Prop}
Sup\'ongase que la distribuci\'on de inter-renovaci\'on $F$ tiene densidad $f$. Entonces $U\left(t\right)$ tambi\'en tiene densidad, para $t>0$, y es $U^{'}\left(t\right)=\sum_{n=0}^{\infty}f^{n\star}\left(t\right)$. Adem\'as
\begin{eqnarray*}
\prob\left\{N\left(t\right)>N\left(t-\right)\right\}=0\textrm{,   }t\geq0.
\end{eqnarray*}
\end{Prop}

\begin{Def}
La Transformada de Laplace-Stieljes de $F$ est\'a dada por

\begin{eqnarray*}
\hat{F}\left(\alpha\right)=\int_{\rea_{+}}e^{-\alpha t}dF\left(t\right)\textrm{,  }\alpha\geq0.
\end{eqnarray*}
\end{Def}

Entonces

\begin{eqnarray*}
\hat{U}\left(\alpha\right)=\sum_{n=0}^{\infty}\hat{F^{n\star}}\left(\alpha\right)=\sum_{n=0}^{\infty}\hat{F}\left(\alpha\right)^{n}=\frac{1}{1-\hat{F}\left(\alpha\right)}.
\end{eqnarray*}


\begin{Prop}
La Transformada de Laplace $\hat{U}\left(\alpha\right)$ y $\hat{F}\left(\alpha\right)$ determina una a la otra de manera \'unica por la relaci\'on $\hat{U}\left(\alpha\right)=\frac{1}{1-\hat{F}\left(\alpha\right)}$.
\end{Prop}


\begin{Note}
Un proceso de renovaci\'on $N\left(t\right)$ cuyos tiempos de inter-renovaci\'on tienen media finita, es un proceso Poisson con tasa $\lambda$ si y s\'olo s\'i $\esp\left[U\left(t\right)\right]=\lambda t$, para $t\geq0$.
\end{Note}


\begin{Teo}
Sea $N\left(t\right)$ un proceso puntual simple con puntos de localizaci\'on $T_{n}$ tal que $\eta\left(t\right)=\esp\left[N\left(\right)\right]$ es finita para cada $t$. Entonces para cualquier funci\'on $f:\rea_{+}\rightarrow\rea$,
\begin{eqnarray*}
\esp\left[\sum_{n=1}^{N\left(\right)}f\left(T_{n}\right)\right]=\int_{\left(0,t\right]}f\left(s\right)d\eta\left(s\right)\textrm{,  }t\geq0,
\end{eqnarray*}
suponiendo que la integral exista. Adem\'as si $X_{1},X_{2},\ldots$ son variables aleatorias definidas en el mismo espacio de probabilidad que el proceso $N\left(t\right)$ tal que $\esp\left[X_{n}|T_{n}=s\right]=f\left(s\right)$, independiente de $n$. Entonces
\begin{eqnarray*}
\esp\left[\sum_{n=1}^{N\left(t\right)}X_{n}\right]=\int_{\left(0,t\right]}f\left(s\right)d\eta\left(s\right)\textrm{,  }t\geq0,
\end{eqnarray*} 
suponiendo que la integral exista. 
\end{Teo}

\begin{Coro}[Identidad de Wald para Renovaciones]
Para el proceso de renovaci\'on $N\left(t\right)$,
\begin{eqnarray*}
\esp\left[T_{N\left(t\right)+1}\right]=\mu\esp\left[N\left(t\right)+1\right]\textrm{,  }t\geq0,
\end{eqnarray*}  
\end{Coro}


\begin{Def}
Sea $h\left(t\right)$ funci\'on de valores reales en $\rea$ acotada en intervalos finitos e igual a cero para $t<0$ La ecuaci\'on de renovaci\'on para $h\left(t\right)$ y la distribuci\'on $F$ es

\begin{eqnarray}%\label{Ec.Renovacion}
H\left(t\right)=h\left(t\right)+\int_{\left[0,t\right]}H\left(t-s\right)dF\left(s\right)\textrm{,    }t\geq0,
\end{eqnarray}
donde $H\left(t\right)$ es una funci\'on de valores reales. Esto es $H=h+F\star H$. Decimos que $H\left(t\right)$ es soluci\'on de esta ecuaci\'on si satisface la ecuaci\'on, y es acotada en intervalos finitos e iguales a cero para $t<0$.
\end{Def}

\begin{Prop}
La funci\'on $U\star h\left(t\right)$ es la \'unica soluci\'on de la ecuaci\'on de renovaci\'on (\ref{Ec.Renovacion}).
\end{Prop}

\begin{Teo}[Teorema Renovaci\'on Elemental]
\begin{eqnarray*}
t^{-1}U\left(t\right)\rightarrow 1/\mu\textrm{,    cuando }t\rightarrow\infty.
\end{eqnarray*}
\end{Teo}


\begin{Note} Una funci\'on $h:\rea_{+}\rightarrow\rea$ es Directamente Riemann Integrable en los siguientes casos:
\begin{itemize}
\item[a)] $h\left(t\right)\geq0$ es decreciente y Riemann Integrable.
\item[b)] $h$ es continua excepto posiblemente en un conjunto de Lebesgue de medida 0, y $|h\left(t\right)|\leq b\left(t\right)$, donde $b$ es DRI.
\end{itemize}
\end{Note}

\begin{Teo}[Teorema Principal de Renovaci\'on]
Si $F$ es no aritm\'etica y $h\left(t\right)$ es Directamente Riemann Integrable (DRI), entonces

\begin{eqnarray*}
lim_{t\rightarrow\infty}U\star h=\frac{1}{\mu}\int_{\rea_{+}}h\left(s\right)ds.
\end{eqnarray*}
\end{Teo}

\begin{Prop}
Cualquier funci\'on $H\left(t\right)$ acotada en intervalos finitos y que es 0 para $t<0$ puede expresarse como
\begin{eqnarray*}
H\left(t\right)=U\star h\left(t\right)\textrm{,  donde }h\left(t\right)=H\left(t\right)-F\star H\left(t\right)
\end{eqnarray*}
\end{Prop}

\begin{Def}
Un proceso estoc\'astico $X\left(t\right)$ es crudamente regenerativo en un tiempo aleatorio positivo $T$ si
\begin{eqnarray*}
\esp\left[X\left(T+t\right)|T\right]=\esp\left[X\left(t\right)\right]\textrm{, para }t\geq0,\end{eqnarray*}
y con las esperanzas anteriores finitas.
\end{Def}

\begin{Prop}
Sup\'ongase que $X\left(t\right)$ es un proceso crudamente regenerativo en $T$, que tiene distribuci\'on $F$. Si $\esp\left[X\left(t\right)\right]$ es acotado en intervalos finitos, entonces
\begin{eqnarray*}
\esp\left[X\left(t\right)\right]=U\star h\left(t\right)\textrm{,  donde }h\left(t\right)=\esp\left[X\left(t\right)\indora\left(T>t\right)\right].
\end{eqnarray*}
\end{Prop}

\begin{Teo}[Regeneraci\'on Cruda]
Sup\'ongase que $X\left(t\right)$ es un proceso con valores positivo crudamente regenerativo en $T$, y def\'inase $M=\sup\left\{|X\left(t\right)|:t\leq T\right\}$. Si $T$ es no aritm\'etico y $M$ y $MT$ tienen media finita, entonces
\begin{eqnarray*}
lim_{t\rightarrow\infty}\esp\left[X\left(t\right)\right]=\frac{1}{\mu}\int_{\rea_{+}}h\left(s\right)ds,
\end{eqnarray*}
donde $h\left(t\right)=\esp\left[X\left(t\right)\indora\left(T>t\right)\right]$.
\end{Teo}


\begin{Note} Una funci\'on $h:\rea_{+}\rightarrow\rea$ es Directamente Riemann Integrable en los siguientes casos:
\begin{itemize}
\item[a)] $h\left(t\right)\geq0$ es decreciente y Riemann Integrable.
\item[b)] $h$ es continua excepto posiblemente en un conjunto de Lebesgue de medida 0, y $|h\left(t\right)|\leq b\left(t\right)$, donde $b$ es DRI.
\end{itemize}
\end{Note}

\begin{Teo}[Teorema Principal de Renovaci\'on]
Si $F$ es no aritm\'etica y $h\left(t\right)$ es Directamente Riemann Integrable (DRI), entonces

\begin{eqnarray*}
lim_{t\rightarrow\infty}U\star h=\frac{1}{\mu}\int_{\rea_{+}}h\left(s\right)ds.
\end{eqnarray*}
\end{Teo}

\begin{Prop}
Cualquier funci\'on $H\left(t\right)$ acotada en intervalos finitos y que es 0 para $t<0$ puede expresarse como
\begin{eqnarray*}
H\left(t\right)=U\star h\left(t\right)\textrm{,  donde }h\left(t\right)=H\left(t\right)-F\star H\left(t\right)
\end{eqnarray*}
\end{Prop}

\begin{Def}
Un proceso estoc\'astico $X\left(t\right)$ es crudamente regenerativo en un tiempo aleatorio positivo $T$ si
\begin{eqnarray*}
\esp\left[X\left(T+t\right)|T\right]=\esp\left[X\left(t\right)\right]\textrm{, para }t\geq0,\end{eqnarray*}
y con las esperanzas anteriores finitas.
\end{Def}

\begin{Prop}
Sup\'ongase que $X\left(t\right)$ es un proceso crudamente regenerativo en $T$, que tiene distribuci\'on $F$. Si $\esp\left[X\left(t\right)\right]$ es acotado en intervalos finitos, entonces
\begin{eqnarray*}
\esp\left[X\left(t\right)\right]=U\star h\left(t\right)\textrm{,  donde }h\left(t\right)=\esp\left[X\left(t\right)\indora\left(T>t\right)\right].
\end{eqnarray*}
\end{Prop}

\begin{Teo}[Regeneraci\'on Cruda]
Sup\'ongase que $X\left(t\right)$ es un proceso con valores positivo crudamente regenerativo en $T$, y def\'inase $M=\sup\left\{|X\left(t\right)|:t\leq T\right\}$. Si $T$ es no aritm\'etico y $M$ y $MT$ tienen media finita, entonces
\begin{eqnarray*}
lim_{t\rightarrow\infty}\esp\left[X\left(t\right)\right]=\frac{1}{\mu}\int_{\rea_{+}}h\left(s\right)ds,
\end{eqnarray*}
donde $h\left(t\right)=\esp\left[X\left(t\right)\indora\left(T>t\right)\right]$.
\end{Teo}

\begin{Def}
Para el proceso $\left\{\left(N\left(t\right),X\left(t\right)\right):t\geq0\right\}$, sus trayectoria muestrales en el intervalo de tiempo $\left[T_{n-1},T_{n}\right)$ est\'an descritas por
\begin{eqnarray*}
\zeta_{n}=\left(\xi_{n},\left\{X\left(T_{n-1}+t\right):0\leq t<\xi_{n}\right\}\right)
\end{eqnarray*}
Este $\zeta_{n}$ es el $n$-\'esimo segmento del proceso. El proceso es regenerativo sobre los tiempos $T_{n}$ si sus segmentos $\zeta_{n}$ son independientes e id\'enticamennte distribuidos.
\end{Def}


\begin{Note}
Si $\tilde{X}\left(t\right)$ con espacio de estados $\tilde{S}$ es regenerativo sobre $T_{n}$, entonces $X\left(t\right)=f\left(\tilde{X}\left(t\right)\right)$ tambi\'en es regenerativo sobre $T_{n}$, para cualquier funci\'on $f:\tilde{S}\rightarrow S$.
\end{Note}

\begin{Note}
Los procesos regenerativos son crudamente regenerativos, pero no al rev\'es.
\end{Note}


\begin{Note}
Un proceso estoc\'astico a tiempo continuo o discreto es regenerativo si existe un proceso de renovaci\'on  tal que los segmentos del proceso entre tiempos de renovaci\'on sucesivos son i.i.d., es decir, para $\left\{X\left(t\right):t\geq0\right\}$ proceso estoc\'astico a tiempo continuo con espacio de estados $S$, espacio m\'etrico.
\end{Note}

Para $\left\{X\left(t\right):t\geq0\right\}$ Proceso Estoc\'astico a tiempo continuo con estado de espacios $S$, que es un espacio m\'etrico, con trayectorias continuas por la derecha y con l\'imites por la izquierda c.s. Sea $N\left(t\right)$ un proceso de renovaci\'on en $\rea_{+}$ definido en el mismo espacio de probabilidad que $X\left(t\right)$, con tiempos de renovaci\'on $T$ y tiempos de inter-renovaci\'on $\xi_{n}=T_{n}-T_{n-1}$, con misma distribuci\'on $F$ de media finita $\mu$.



\begin{Def}
Para el proceso $\left\{\left(N\left(t\right),X\left(t\right)\right):t\geq0\right\}$, sus trayectoria muestrales en el intervalo de tiempo $\left[T_{n-1},T_{n}\right)$ est\'an descritas por
\begin{eqnarray*}
\zeta_{n}=\left(\xi_{n},\left\{X\left(T_{n-1}+t\right):0\leq t<\xi_{n}\right\}\right)
\end{eqnarray*}
Este $\zeta_{n}$ es el $n$-\'esimo segmento del proceso. El proceso es regenerativo sobre los tiempos $T_{n}$ si sus segmentos $\zeta_{n}$ son independientes e id\'enticamennte distribuidos.
\end{Def}

\begin{Note}
Un proceso regenerativo con media de la longitud de ciclo finita es llamado positivo recurrente.
\end{Note}

\begin{Teo}[Procesos Regenerativos]
Suponga que el proceso
\end{Teo}


\begin{Def}[Renewal Process Trinity]
Para un proceso de renovaci\'on $N\left(t\right)$, los siguientes procesos proveen de informaci\'on sobre los tiempos de renovaci\'on.
\begin{itemize}
\item $A\left(t\right)=t-T_{N\left(t\right)}$, el tiempo de recurrencia hacia atr\'as al tiempo $t$, que es el tiempo desde la \'ultima renovaci\'on para $t$.

\item $B\left(t\right)=T_{N\left(t\right)+1}-t$, el tiempo de recurrencia hacia adelante al tiempo $t$, residual del tiempo de renovaci\'on, que es el tiempo para la pr\'oxima renovaci\'on despu\'es de $t$.

\item $L\left(t\right)=\xi_{N\left(t\right)+1}=A\left(t\right)+B\left(t\right)$, la longitud del intervalo de renovaci\'on que contiene a $t$.
\end{itemize}
\end{Def}

\begin{Note}
El proceso tridimensional $\left(A\left(t\right),B\left(t\right),L\left(t\right)\right)$ es regenerativo sobre $T_{n}$, y por ende cada proceso lo es. Cada proceso $A\left(t\right)$ y $B\left(t\right)$ son procesos de MArkov a tiempo continuo con trayectorias continuas por partes en el espacio de estados $\rea_{+}$. Una expresi\'on conveniente para su distribuci\'on conjunta es, para $0\leq x<t,y\geq0$
\begin{equation}\label{NoRenovacion}
P\left\{A\left(t\right)>x,B\left(t\right)>y\right\}=
P\left\{N\left(t+y\right)-N\left((t-x)\right)=0\right\}
\end{equation}
\end{Note}

\begin{Ejem}[Tiempos de recurrencia Poisson]
Si $N\left(t\right)$ es un proceso Poisson con tasa $\lambda$, entonces de la expresi\'on (\ref{NoRenovacion}) se tiene que

\begin{eqnarray*}
\begin{array}{lc}
P\left\{A\left(t\right)>x,B\left(t\right)>y\right\}=e^{-\lambda\left(x+y\right)},&0\leq x<t,y\geq0,
\end{array}
\end{eqnarray*}
que es la probabilidad Poisson de no renovaciones en un intervalo de longitud $x+y$.

\end{Ejem}

\begin{Note}
Una cadena de Markov erg\'odica tiene la propiedad de ser estacionaria si la distribuci\'on de su estado al tiempo $0$ es su distribuci\'on estacionaria.
\end{Note}


\begin{Def}
Un proceso estoc\'astico a tiempo continuo $\left\{X\left(t\right):t\geq0\right\}$ en un espacio general es estacionario si sus distribuciones finito dimensionales son invariantes bajo cualquier  traslado: para cada $0\leq s_{1}<s_{2}<\cdots<s_{k}$ y $t\geq0$,
\begin{eqnarray*}
\left(X\left(s_{1}+t\right),\ldots,X\left(s_{k}+t\right)\right)=_{d}\left(X\left(s_{1}\right),\ldots,X\left(s_{k}\right)\right).
\end{eqnarray*}
\end{Def}

\begin{Note}
Un proceso de Markov es estacionario si $X\left(t\right)=_{d}X\left(0\right)$, $t\geq0$.
\end{Note}

Considerese el proceso $N\left(t\right)=\sum_{n}\indora\left(\tau_{n}\leq t\right)$ en $\rea_{+}$, con puntos $0<\tau_{1}<\tau_{2}<\cdots$.

\begin{Prop}
Si $N$ es un proceso puntual estacionario y $\esp\left[N\left(1\right)\right]<\infty$, entonces $\esp\left[N\left(t\right)\right]=t\esp\left[N\left(1\right)\right]$, $t\geq0$

\end{Prop}

\begin{Teo}
Los siguientes enunciados son equivalentes
\begin{itemize}
\item[i)] El proceso retardado de renovaci\'on $N$ es estacionario.

\item[ii)] EL proceso de tiempos de recurrencia hacia adelante $B\left(t\right)$ es estacionario.


\item[iii)] $\esp\left[N\left(t\right)\right]=t/\mu$,


\item[iv)] $G\left(t\right)=F_{e}\left(t\right)=\frac{1}{\mu}\int_{0}^{t}\left[1-F\left(s\right)\right]ds$
\end{itemize}
Cuando estos enunciados son ciertos, $P\left\{B\left(t\right)\leq x\right\}=F_{e}\left(x\right)$, para $t,x\geq0$.

\end{Teo}

\begin{Note}
Una consecuencia del teorema anterior es que el Proceso Poisson es el \'unico proceso sin retardo que es estacionario.
\end{Note}

\begin{Coro}
El proceso de renovaci\'on $N\left(t\right)$ sin retardo, y cuyos tiempos de inter renonaci\'on tienen media finita, es estacionario si y s\'olo si es un proceso Poisson.

\end{Coro}


%________________________________________________________________________
%\subsection{Procesos Regenerativos}
%________________________________________________________________________



\begin{Note}
Si $\tilde{X}\left(t\right)$ con espacio de estados $\tilde{S}$ es regenerativo sobre $T_{n}$, entonces $X\left(t\right)=f\left(\tilde{X}\left(t\right)\right)$ tambi\'en es regenerativo sobre $T_{n}$, para cualquier funci\'on $f:\tilde{S}\rightarrow S$.
\end{Note}

\begin{Note}
Los procesos regenerativos son crudamente regenerativos, pero no al rev\'es.
\end{Note}
%\subsection*{Procesos Regenerativos: Sigman\cite{Sigman1}}
\begin{Def}[Definici\'on Cl\'asica]
Un proceso estoc\'astico $X=\left\{X\left(t\right):t\geq0\right\}$ es llamado regenerativo is existe una variable aleatoria $R_{1}>0$ tal que
\begin{itemize}
\item[i)] $\left\{X\left(t+R_{1}\right):t\geq0\right\}$ es independiente de $\left\{\left\{X\left(t\right):t<R_{1}\right\},\right\}$
\item[ii)] $\left\{X\left(t+R_{1}\right):t\geq0\right\}$ es estoc\'asticamente equivalente a $\left\{X\left(t\right):t>0\right\}$
\end{itemize}

Llamamos a $R_{1}$ tiempo de regeneraci\'on, y decimos que $X$ se regenera en este punto.
\end{Def}

$\left\{X\left(t+R_{1}\right)\right\}$ es regenerativo con tiempo de regeneraci\'on $R_{2}$, independiente de $R_{1}$ pero con la misma distribuci\'on que $R_{1}$. Procediendo de esta manera se obtiene una secuencia de variables aleatorias independientes e id\'enticamente distribuidas $\left\{R_{n}\right\}$ llamados longitudes de ciclo. Si definimos a $Z_{k}\equiv R_{1}+R_{2}+\cdots+R_{k}$, se tiene un proceso de renovaci\'on llamado proceso de renovaci\'on encajado para $X$.




\begin{Def}
Para $x$ fijo y para cada $t\geq0$, sea $I_{x}\left(t\right)=1$ si $X\left(t\right)\leq x$,  $I_{x}\left(t\right)=0$ en caso contrario, y def\'inanse los tiempos promedio
\begin{eqnarray*}
\overline{X}&=&lim_{t\rightarrow\infty}\frac{1}{t}\int_{0}^{\infty}X\left(u\right)du\\
\prob\left(X_{\infty}\leq x\right)&=&lim_{t\rightarrow\infty}\frac{1}{t}\int_{0}^{\infty}I_{x}\left(u\right)du,
\end{eqnarray*}
cuando estos l\'imites existan.
\end{Def}

Como consecuencia del teorema de Renovaci\'on-Recompensa, se tiene que el primer l\'imite  existe y es igual a la constante
\begin{eqnarray*}
\overline{X}&=&\frac{\esp\left[\int_{0}^{R_{1}}X\left(t\right)dt\right]}{\esp\left[R_{1}\right]},
\end{eqnarray*}
suponiendo que ambas esperanzas son finitas.

\begin{Note}
\begin{itemize}
\item[a)] Si el proceso regenerativo $X$ es positivo recurrente y tiene trayectorias muestrales no negativas, entonces la ecuaci\'on anterior es v\'alida.
\item[b)] Si $X$ es positivo recurrente regenerativo, podemos construir una \'unica versi\'on estacionaria de este proceso, $X_{e}=\left\{X_{e}\left(t\right)\right\}$, donde $X_{e}$ es un proceso estoc\'astico regenerativo y estrictamente estacionario, con distribuci\'on marginal distribuida como $X_{\infty}$
\end{itemize}
\end{Note}

%________________________________________________________________________
%\subsection{Procesos Regenerativos}
%________________________________________________________________________

Para $\left\{X\left(t\right):t\geq0\right\}$ Proceso Estoc\'astico a tiempo continuo con estado de espacios $S$, que es un espacio m\'etrico, con trayectorias continuas por la derecha y con l\'imites por la izquierda c.s. Sea $N\left(t\right)$ un proceso de renovaci\'on en $\rea_{+}$ definido en el mismo espacio de probabilidad que $X\left(t\right)$, con tiempos de renovaci\'on $T$ y tiempos de inter-renovaci\'on $\xi_{n}=T_{n}-T_{n-1}$, con misma distribuci\'on $F$ de media finita $\mu$.



\begin{Def}
Para el proceso $\left\{\left(N\left(t\right),X\left(t\right)\right):t\geq0\right\}$, sus trayectoria muestrales en el intervalo de tiempo $\left[T_{n-1},T_{n}\right)$ est\'an descritas por
\begin{eqnarray*}
\zeta_{n}=\left(\xi_{n},\left\{X\left(T_{n-1}+t\right):0\leq t<\xi_{n}\right\}\right)
\end{eqnarray*}
Este $\zeta_{n}$ es el $n$-\'esimo segmento del proceso. El proceso es regenerativo sobre los tiempos $T_{n}$ si sus segmentos $\zeta_{n}$ son independientes e id\'enticamennte distribuidos.
\end{Def}


\begin{Note}
Si $\tilde{X}\left(t\right)$ con espacio de estados $\tilde{S}$ es regenerativo sobre $T_{n}$, entonces $X\left(t\right)=f\left(\tilde{X}\left(t\right)\right)$ tambi\'en es regenerativo sobre $T_{n}$, para cualquier funci\'on $f:\tilde{S}\rightarrow S$.
\end{Note}

\begin{Note}
Los procesos regenerativos son crudamente regenerativos, pero no al rev\'es.
\end{Note}

\begin{Def}[Definici\'on Cl\'asica]
Un proceso estoc\'astico $X=\left\{X\left(t\right):t\geq0\right\}$ es llamado regenerativo is existe una variable aleatoria $R_{1}>0$ tal que
\begin{itemize}
\item[i)] $\left\{X\left(t+R_{1}\right):t\geq0\right\}$ es independiente de $\left\{\left\{X\left(t\right):t<R_{1}\right\},\right\}$
\item[ii)] $\left\{X\left(t+R_{1}\right):t\geq0\right\}$ es estoc\'asticamente equivalente a $\left\{X\left(t\right):t>0\right\}$
\end{itemize}

Llamamos a $R_{1}$ tiempo de regeneraci\'on, y decimos que $X$ se regenera en este punto.
\end{Def}

$\left\{X\left(t+R_{1}\right)\right\}$ es regenerativo con tiempo de regeneraci\'on $R_{2}$, independiente de $R_{1}$ pero con la misma distribuci\'on que $R_{1}$. Procediendo de esta manera se obtiene una secuencia de variables aleatorias independientes e id\'enticamente distribuidas $\left\{R_{n}\right\}$ llamados longitudes de ciclo. Si definimos a $Z_{k}\equiv R_{1}+R_{2}+\cdots+R_{k}$, se tiene un proceso de renovaci\'on llamado proceso de renovaci\'on encajado para $X$.

\begin{Note}
Un proceso regenerativo con media de la longitud de ciclo finita es llamado positivo recurrente.
\end{Note}


\begin{Def}
Para $x$ fijo y para cada $t\geq0$, sea $I_{x}\left(t\right)=1$ si $X\left(t\right)\leq x$,  $I_{x}\left(t\right)=0$ en caso contrario, y def\'inanse los tiempos promedio
\begin{eqnarray*}
\overline{X}&=&lim_{t\rightarrow\infty}\frac{1}{t}\int_{0}^{\infty}X\left(u\right)du\\
\prob\left(X_{\infty}\leq x\right)&=&lim_{t\rightarrow\infty}\frac{1}{t}\int_{0}^{\infty}I_{x}\left(u\right)du,
\end{eqnarray*}
cuando estos l\'imites existan.
\end{Def}

Como consecuencia del teorema de Renovaci\'on-Recompensa, se tiene que el primer l\'imite  existe y es igual a la constante
\begin{eqnarray*}
\overline{X}&=&\frac{\esp\left[\int_{0}^{R_{1}}X\left(t\right)dt\right]}{\esp\left[R_{1}\right]},
\end{eqnarray*}
suponiendo que ambas esperanzas son finitas.

\begin{Note}
\begin{itemize}
\item[a)] Si el proceso regenerativo $X$ es positivo recurrente y tiene trayectorias muestrales no negativas, entonces la ecuaci\'on anterior es v\'alida.
\item[b)] Si $X$ es positivo recurrente regenerativo, podemos construir una \'unica versi\'on estacionaria de este proceso, $X_{e}=\left\{X_{e}\left(t\right)\right\}$, donde $X_{e}$ es un proceso estoc\'astico regenerativo y estrictamente estacionario, con distribuci\'on marginal distribuida como $X_{\infty}$
\end{itemize}
\end{Note}

%__________________________________________________________________________________________
%\subsection{Procesos Regenerativos Estacionarios - Stidham \cite{Stidham}}
%__________________________________________________________________________________________


Un proceso estoc\'astico a tiempo continuo $\left\{V\left(t\right),t\geq0\right\}$ es un proceso regenerativo si existe una sucesi\'on de variables aleatorias independientes e id\'enticamente distribuidas $\left\{X_{1},X_{2},\ldots\right\}$, sucesi\'on de renovaci\'on, tal que para cualquier conjunto de Borel $A$, 

\begin{eqnarray*}
\prob\left\{V\left(t\right)\in A|X_{1}+X_{2}+\cdots+X_{R\left(t\right)}=s,\left\{V\left(\tau\right),\tau<s\right\}\right\}=\prob\left\{V\left(t-s\right)\in A|X_{1}>t-s\right\},
\end{eqnarray*}
para todo $0\leq s\leq t$, donde $R\left(t\right)=\max\left\{X_{1}+X_{2}+\cdots+X_{j}\leq t\right\}=$n\'umero de renovaciones ({\emph{puntos de regeneraci\'on}}) que ocurren en $\left[0,t\right]$. El intervalo $\left[0,X_{1}\right)$ es llamado {\emph{primer ciclo de regeneraci\'on}} de $\left\{V\left(t \right),t\geq0\right\}$, $\left[X_{1},X_{1}+X_{2}\right)$ el {\emph{segundo ciclo de regeneraci\'on}}, y as\'i sucesivamente.

Sea $X=X_{1}$ y sea $F$ la funci\'on de distrbuci\'on de $X$


\begin{Def}
Se define el proceso estacionario, $\left\{V^{*}\left(t\right),t\geq0\right\}$, para $\left\{V\left(t\right),t\geq0\right\}$ por

\begin{eqnarray*}
\prob\left\{V\left(t\right)\in A\right\}=\frac{1}{\esp\left[X\right]}\int_{0}^{\infty}\prob\left\{V\left(t+x\right)\in A|X>x\right\}\left(1-F\left(x\right)\right)dx,
\end{eqnarray*} 
para todo $t\geq0$ y todo conjunto de Borel $A$.
\end{Def}

\begin{Def}
Una distribuci\'on se dice que es {\emph{aritm\'etica}} si todos sus puntos de incremento son m\'ultiplos de la forma $0,\lambda, 2\lambda,\ldots$ para alguna $\lambda>0$ entera.
\end{Def}


\begin{Def}
Una modificaci\'on medible de un proceso $\left\{V\left(t\right),t\geq0\right\}$, es una versi\'on de este, $\left\{V\left(t,w\right)\right\}$ conjuntamente medible para $t\geq0$ y para $w\in S$, $S$ espacio de estados para $\left\{V\left(t\right),t\geq0\right\}$.
\end{Def}

\begin{Teo}
Sea $\left\{V\left(t\right),t\geq\right\}$ un proceso regenerativo no negativo con modificaci\'on medible. Sea $\esp\left[X\right]<\infty$. Entonces el proceso estacionario dado por la ecuaci\'on anterior est\'a bien definido y tiene funci\'on de distribuci\'on independiente de $t$, adem\'as
\begin{itemize}
\item[i)] \begin{eqnarray*}
\esp\left[V^{*}\left(0\right)\right]&=&\frac{\esp\left[\int_{0}^{X}V\left(s\right)ds\right]}{\esp\left[X\right]}\end{eqnarray*}
\item[ii)] Si $\esp\left[V^{*}\left(0\right)\right]<\infty$, equivalentemente, si $\esp\left[\int_{0}^{X}V\left(s\right)ds\right]<\infty$,entonces
\begin{eqnarray*}
\frac{\int_{0}^{t}V\left(s\right)ds}{t}\rightarrow\frac{\esp\left[\int_{0}^{X}V\left(s\right)ds\right]}{\esp\left[X\right]}
\end{eqnarray*}
con probabilidad 1 y en media, cuando $t\rightarrow\infty$.
\end{itemize}
\end{Teo}
%
%___________________________________________________________________________________________
%\vspace{5.5cm}
%\chapter{Cadenas de Markov estacionarias}
%\vspace{-1.0cm}


%__________________________________________________________________________________________
%\subsection{Procesos Regenerativos Estacionarios - Stidham \cite{Stidham}}
%__________________________________________________________________________________________


Un proceso estoc\'astico a tiempo continuo $\left\{V\left(t\right),t\geq0\right\}$ es un proceso regenerativo si existe una sucesi\'on de variables aleatorias independientes e id\'enticamente distribuidas $\left\{X_{1},X_{2},\ldots\right\}$, sucesi\'on de renovaci\'on, tal que para cualquier conjunto de Borel $A$, 

\begin{eqnarray*}
\prob\left\{V\left(t\right)\in A|X_{1}+X_{2}+\cdots+X_{R\left(t\right)}=s,\left\{V\left(\tau\right),\tau<s\right\}\right\}=\prob\left\{V\left(t-s\right)\in A|X_{1}>t-s\right\},
\end{eqnarray*}
para todo $0\leq s\leq t$, donde $R\left(t\right)=\max\left\{X_{1}+X_{2}+\cdots+X_{j}\leq t\right\}=$n\'umero de renovaciones ({\emph{puntos de regeneraci\'on}}) que ocurren en $\left[0,t\right]$. El intervalo $\left[0,X_{1}\right)$ es llamado {\emph{primer ciclo de regeneraci\'on}} de $\left\{V\left(t \right),t\geq0\right\}$, $\left[X_{1},X_{1}+X_{2}\right)$ el {\emph{segundo ciclo de regeneraci\'on}}, y as\'i sucesivamente.

Sea $X=X_{1}$ y sea $F$ la funci\'on de distrbuci\'on de $X$


\begin{Def}
Se define el proceso estacionario, $\left\{V^{*}\left(t\right),t\geq0\right\}$, para $\left\{V\left(t\right),t\geq0\right\}$ por

\begin{eqnarray*}
\prob\left\{V\left(t\right)\in A\right\}=\frac{1}{\esp\left[X\right]}\int_{0}^{\infty}\prob\left\{V\left(t+x\right)\in A|X>x\right\}\left(1-F\left(x\right)\right)dx,
\end{eqnarray*} 
para todo $t\geq0$ y todo conjunto de Borel $A$.
\end{Def}

\begin{Def}
Una distribuci\'on se dice que es {\emph{aritm\'etica}} si todos sus puntos de incremento son m\'ultiplos de la forma $0,\lambda, 2\lambda,\ldots$ para alguna $\lambda>0$ entera.
\end{Def}


\begin{Def}
Una modificaci\'on medible de un proceso $\left\{V\left(t\right),t\geq0\right\}$, es una versi\'on de este, $\left\{V\left(t,w\right)\right\}$ conjuntamente medible para $t\geq0$ y para $w\in S$, $S$ espacio de estados para $\left\{V\left(t\right),t\geq0\right\}$.
\end{Def}

\begin{Teo}
Sea $\left\{V\left(t\right),t\geq\right\}$ un proceso regenerativo no negativo con modificaci\'on medible. Sea $\esp\left[X\right]<\infty$. Entonces el proceso estacionario dado por la ecuaci\'on anterior est\'a bien definido y tiene funci\'on de distribuci\'on independiente de $t$, adem\'as
\begin{itemize}
\item[i)] \begin{eqnarray*}
\esp\left[V^{*}\left(0\right)\right]&=&\frac{\esp\left[\int_{0}^{X}V\left(s\right)ds\right]}{\esp\left[X\right]}\end{eqnarray*}
\item[ii)] Si $\esp\left[V^{*}\left(0\right)\right]<\infty$, equivalentemente, si $\esp\left[\int_{0}^{X}V\left(s\right)ds\right]<\infty$,entonces
\begin{eqnarray*}
\frac{\int_{0}^{t}V\left(s\right)ds}{t}\rightarrow\frac{\esp\left[\int_{0}^{X}V\left(s\right)ds\right]}{\esp\left[X\right]}
\end{eqnarray*}
con probabilidad 1 y en media, cuando $t\rightarrow\infty$.
\end{itemize}
\end{Teo}

Para $\left\{X\left(t\right):t\geq0\right\}$ Proceso Estoc\'astico a tiempo continuo con estado de espacios $S$, que es un espacio m\'etrico, con trayectorias continuas por la derecha y con l\'imites por la izquierda c.s. Sea $N\left(t\right)$ un proceso de renovaci\'on en $\rea_{+}$ definido en el mismo espacio de probabilidad que $X\left(t\right)$, con tiempos de renovaci\'on $T$ y tiempos de inter-renovaci\'on $\xi_{n}=T_{n}-T_{n-1}$, con misma distribuci\'on $F$ de media finita $\mu$.

%________________________________________________________________________
\subsection{Procesos Regenerativos}
%________________________________________________________________________

%________________________________________________________________________
\subsection{Procesos Regenerativos Sigman, Thorisson y Wolff \cite{Sigman1}}
%________________________________________________________________________


\begin{Def}[Definici\'on Cl\'asica]
Un proceso estoc\'astico $X=\left\{X\left(t\right):t\geq0\right\}$ es llamado regenerativo is existe una variable aleatoria $R_{1}>0$ tal que
\begin{itemize}
\item[i)] $\left\{X\left(t+R_{1}\right):t\geq0\right\}$ es independiente de $\left\{\left\{X\left(t\right):t<R_{1}\right\},\right\}$
\item[ii)] $\left\{X\left(t+R_{1}\right):t\geq0\right\}$ es estoc\'asticamente equivalente a $\left\{X\left(t\right):t>0\right\}$
\end{itemize}

Llamamos a $R_{1}$ tiempo de regeneraci\'on, y decimos que $X$ se regenera en este punto.
\end{Def}

$\left\{X\left(t+R_{1}\right)\right\}$ es regenerativo con tiempo de regeneraci\'on $R_{2}$, independiente de $R_{1}$ pero con la misma distribuci\'on que $R_{1}$. Procediendo de esta manera se obtiene una secuencia de variables aleatorias independientes e id\'enticamente distribuidas $\left\{R_{n}\right\}$ llamados longitudes de ciclo. Si definimos a $Z_{k}\equiv R_{1}+R_{2}+\cdots+R_{k}$, se tiene un proceso de renovaci\'on llamado proceso de renovaci\'on encajado para $X$.


\begin{Note}
La existencia de un primer tiempo de regeneraci\'on, $R_{1}$, implica la existencia de una sucesi\'on completa de estos tiempos $R_{1},R_{2}\ldots,$ que satisfacen la propiedad deseada \cite{Sigman2}.
\end{Note}


\begin{Note} Para la cola $GI/GI/1$ los usuarios arriban con tiempos $t_{n}$ y son atendidos con tiempos de servicio $S_{n}$, los tiempos de arribo forman un proceso de renovaci\'on  con tiempos entre arribos independientes e identicamente distribuidos (\texttt{i.i.d.})$T_{n}=t_{n}-t_{n-1}$, adem\'as los tiempos de servicio son \texttt{i.i.d.} e independientes de los procesos de arribo. Por \textit{estable} se entiende que $\esp S_{n}<\esp T_{n}<\infty$.
\end{Note}
 


\begin{Def}
Para $x$ fijo y para cada $t\geq0$, sea $I_{x}\left(t\right)=1$ si $X\left(t\right)\leq x$,  $I_{x}\left(t\right)=0$ en caso contrario, y def\'inanse los tiempos promedio
\begin{eqnarray*}
\overline{X}&=&lim_{t\rightarrow\infty}\frac{1}{t}\int_{0}^{\infty}X\left(u\right)du\\
\prob\left(X_{\infty}\leq x\right)&=&lim_{t\rightarrow\infty}\frac{1}{t}\int_{0}^{\infty}I_{x}\left(u\right)du,
\end{eqnarray*}
cuando estos l\'imites existan.
\end{Def}

Como consecuencia del teorema de Renovaci\'on-Recompensa, se tiene que el primer l\'imite  existe y es igual a la constante
\begin{eqnarray*}
\overline{X}&=&\frac{\esp\left[\int_{0}^{R_{1}}X\left(t\right)dt\right]}{\esp\left[R_{1}\right]},
\end{eqnarray*}
suponiendo que ambas esperanzas son finitas.
 
\begin{Note}
Funciones de procesos regenerativos son regenerativas, es decir, si $X\left(t\right)$ es regenerativo y se define el proceso $Y\left(t\right)$ por $Y\left(t\right)=f\left(X\left(t\right)\right)$ para alguna funci\'on Borel medible $f\left(\cdot\right)$. Adem\'as $Y$ es regenerativo con los mismos tiempos de renovaci\'on que $X$. 

En general, los tiempos de renovaci\'on, $Z_{k}$ de un proceso regenerativo no requieren ser tiempos de paro con respecto a la evoluci\'on de $X\left(t\right)$.
\end{Note} 

\begin{Note}
Una funci\'on de un proceso de Markov, usualmente no ser\'a un proceso de Markov, sin embargo ser\'a regenerativo si el proceso de Markov lo es.
\end{Note}

 
\begin{Note}
Un proceso regenerativo con media de la longitud de ciclo finita es llamado positivo recurrente.
\end{Note}


\begin{Note}
\begin{itemize}
\item[a)] Si el proceso regenerativo $X$ es positivo recurrente y tiene trayectorias muestrales no negativas, entonces la ecuaci\'on anterior es v\'alida.
\item[b)] Si $X$ es positivo recurrente regenerativo, podemos construir una \'unica versi\'on estacionaria de este proceso, $X_{e}=\left\{X_{e}\left(t\right)\right\}$, donde $X_{e}$ es un proceso estoc\'astico regenerativo y estrictamente estacionario, con distribuci\'on marginal distribuida como $X_{\infty}$
\end{itemize}
\end{Note}


%__________________________________________________________________________________________
%\subsection*{Procesos Regenerativos Estacionarios - Stidham \cite{Stidham}}
%__________________________________________________________________________________________


Un proceso estoc\'astico a tiempo continuo $\left\{V\left(t\right),t\geq0\right\}$ es un proceso regenerativo si existe una sucesi\'on de variables aleatorias independientes e id\'enticamente distribuidas $\left\{X_{1},X_{2},\ldots\right\}$, sucesi\'on de renovaci\'on, tal que para cualquier conjunto de Borel $A$, 

\begin{eqnarray*}
\prob\left\{V\left(t\right)\in A|X_{1}+X_{2}+\cdots+X_{R\left(t\right)}=s,\left\{V\left(\tau\right),\tau<s\right\}\right\}=\prob\left\{V\left(t-s\right)\in A|X_{1}>t-s\right\},
\end{eqnarray*}
para todo $0\leq s\leq t$, donde $R\left(t\right)=\max\left\{X_{1}+X_{2}+\cdots+X_{j}\leq t\right\}=$n\'umero de renovaciones ({\emph{puntos de regeneraci\'on}}) que ocurren en $\left[0,t\right]$. El intervalo $\left[0,X_{1}\right)$ es llamado {\emph{primer ciclo de regeneraci\'on}} de $\left\{V\left(t \right),t\geq0\right\}$, $\left[X_{1},X_{1}+X_{2}\right)$ el {\emph{segundo ciclo de regeneraci\'on}}, y as\'i sucesivamente.

Sea $X=X_{1}$ y sea $F$ la funci\'on de distrbuci\'on de $X$


\begin{Def}
Se define el proceso estacionario, $\left\{V^{*}\left(t\right),t\geq0\right\}$, para $\left\{V\left(t\right),t\geq0\right\}$ por

\begin{eqnarray*}
\prob\left\{V\left(t\right)\in A\right\}=\frac{1}{\esp\left[X\right]}\int_{0}^{\infty}\prob\left\{V\left(t+x\right)\in A|X>x\right\}\left(1-F\left(x\right)\right)dx,
\end{eqnarray*} 
para todo $t\geq0$ y todo conjunto de Borel $A$.
\end{Def}

\begin{Def}
Una distribuci\'on se dice que es {\emph{aritm\'etica}} si todos sus puntos de incremento son m\'ultiplos de la forma $0,\lambda, 2\lambda,\ldots$ para alguna $\lambda>0$ entera.
\end{Def}


\begin{Def}
Una modificaci\'on medible de un proceso $\left\{V\left(t\right),t\geq0\right\}$, es una versi\'on de este, $\left\{V\left(t,w\right)\right\}$ conjuntamente medible para $t\geq0$ y para $w\in S$, $S$ espacio de estados para $\left\{V\left(t\right),t\geq0\right\}$.
\end{Def}

\begin{Teo}
Sea $\left\{V\left(t\right),t\geq\right\}$ un proceso regenerativo no negativo con modificaci\'on medible. Sea $\esp\left[X\right]<\infty$. Entonces el proceso estacionario dado por la ecuaci\'on anterior est\'a bien definido y tiene funci\'on de distribuci\'on independiente de $t$, adem\'as
\begin{itemize}
\item[i)] \begin{eqnarray*}
\esp\left[V^{*}\left(0\right)\right]&=&\frac{\esp\left[\int_{0}^{X}V\left(s\right)ds\right]}{\esp\left[X\right]}\end{eqnarray*}
\item[ii)] Si $\esp\left[V^{*}\left(0\right)\right]<\infty$, equivalentemente, si $\esp\left[\int_{0}^{X}V\left(s\right)ds\right]<\infty$,entonces
\begin{eqnarray*}
\frac{\int_{0}^{t}V\left(s\right)ds}{t}\rightarrow\frac{\esp\left[\int_{0}^{X}V\left(s\right)ds\right]}{\esp\left[X\right]}
\end{eqnarray*}
con probabilidad 1 y en media, cuando $t\rightarrow\infty$.
\end{itemize}
\end{Teo}

\begin{Coro}
Sea $\left\{V\left(t\right),t\geq0\right\}$ un proceso regenerativo no negativo, con modificaci\'on medible. Si $\esp <\infty$, $F$ es no-aritm\'etica, y para todo $x\geq0$, $P\left\{V\left(t\right)\leq x,C>x\right\}$ es de variaci\'on acotada como funci\'on de $t$ en cada intervalo finito $\left[0,\tau\right]$, entonces $V\left(t\right)$ converge en distribuci\'on  cuando $t\rightarrow\infty$ y $$\esp V=\frac{\esp \int_{0}^{X}V\left(s\right)ds}{\esp X}$$
Donde $V$ tiene la distribuci\'on l\'imite de $V\left(t\right)$ cuando $t\rightarrow\infty$.

\end{Coro}

Para el caso discreto se tienen resultados similares.



%______________________________________________________________________
%\section{Procesos de Renovaci\'on}
%______________________________________________________________________

\begin{Def}\label{Def.Tn}
Sean $0\leq T_{1}\leq T_{2}\leq \ldots$ son tiempos aleatorios infinitos en los cuales ocurren ciertos eventos. El n\'umero de tiempos $T_{n}$ en el intervalo $\left[0,t\right)$ es

\begin{eqnarray}
N\left(t\right)=\sum_{n=1}^{\infty}\indora\left(T_{n}\leq t\right),
\end{eqnarray}
para $t\geq0$.
\end{Def}

Si se consideran los puntos $T_{n}$ como elementos de $\rea_{+}$, y $N\left(t\right)$ es el n\'umero de puntos en $\rea$. El proceso denotado por $\left\{N\left(t\right):t\geq0\right\}$, denotado por $N\left(t\right)$, es un proceso puntual en $\rea_{+}$. Los $T_{n}$ son los tiempos de ocurrencia, el proceso puntual $N\left(t\right)$ es simple si su n\'umero de ocurrencias son distintas: $0<T_{1}<T_{2}<\ldots$ casi seguramente.

\begin{Def}
Un proceso puntual $N\left(t\right)$ es un proceso de renovaci\'on si los tiempos de interocurrencia $\xi_{n}=T_{n}-T_{n-1}$, para $n\geq1$, son independientes e identicamente distribuidos con distribuci\'on $F$, donde $F\left(0\right)=0$ y $T_{0}=0$. Los $T_{n}$ son llamados tiempos de renovaci\'on, referente a la independencia o renovaci\'on de la informaci\'on estoc\'astica en estos tiempos. Los $\xi_{n}$ son los tiempos de inter-renovaci\'on, y $N\left(t\right)$ es el n\'umero de renovaciones en el intervalo $\left[0,t\right)$
\end{Def}


\begin{Note}
Para definir un proceso de renovaci\'on para cualquier contexto, solamente hay que especificar una distribuci\'on $F$, con $F\left(0\right)=0$, para los tiempos de inter-renovaci\'on. La funci\'on $F$ en turno degune las otra variables aleatorias. De manera formal, existe un espacio de probabilidad y una sucesi\'on de variables aleatorias $\xi_{1},\xi_{2},\ldots$ definidas en este con distribuci\'on $F$. Entonces las otras cantidades son $T_{n}=\sum_{k=1}^{n}\xi_{k}$ y $N\left(t\right)=\sum_{n=1}^{\infty}\indora\left(T_{n}\leq t\right)$, donde $T_{n}\rightarrow\infty$ casi seguramente por la Ley Fuerte de los Grandes Números.
\end{Note}

%___________________________________________________________________________________________
%
%\subsection*{Teorema Principal de Renovaci\'on}
%___________________________________________________________________________________________
%

\begin{Note} Una funci\'on $h:\rea_{+}\rightarrow\rea$ es Directamente Riemann Integrable en los siguientes casos:
\begin{itemize}
\item[a)] $h\left(t\right)\geq0$ es decreciente y Riemann Integrable.
\item[b)] $h$ es continua excepto posiblemente en un conjunto de Lebesgue de medida 0, y $|h\left(t\right)|\leq b\left(t\right)$, donde $b$ es DRI.
\end{itemize}
\end{Note}

\begin{Teo}[Teorema Principal de Renovaci\'on]
Si $F$ es no aritm\'etica y $h\left(t\right)$ es Directamente Riemann Integrable (DRI), entonces

\begin{eqnarray*}
lim_{t\rightarrow\infty}U\star h=\frac{1}{\mu}\int_{\rea_{+}}h\left(s\right)ds.
\end{eqnarray*}
\end{Teo}

\begin{Prop}
Cualquier funci\'on $H\left(t\right)$ acotada en intervalos finitos y que es 0 para $t<0$ puede expresarse como
\begin{eqnarray*}
H\left(t\right)=U\star h\left(t\right)\textrm{,  donde }h\left(t\right)=H\left(t\right)-F\star H\left(t\right)
\end{eqnarray*}
\end{Prop}

\begin{Def}
Un proceso estoc\'astico $X\left(t\right)$ es crudamente regenerativo en un tiempo aleatorio positivo $T$ si
\begin{eqnarray*}
\esp\left[X\left(T+t\right)|T\right]=\esp\left[X\left(t\right)\right]\textrm{, para }t\geq0,\end{eqnarray*}
y con las esperanzas anteriores finitas.
\end{Def}

\begin{Prop}
Sup\'ongase que $X\left(t\right)$ es un proceso crudamente regenerativo en $T$, que tiene distribuci\'on $F$. Si $\esp\left[X\left(t\right)\right]$ es acotado en intervalos finitos, entonces
\begin{eqnarray*}
\esp\left[X\left(t\right)\right]=U\star h\left(t\right)\textrm{,  donde }h\left(t\right)=\esp\left[X\left(t\right)\indora\left(T>t\right)\right].
\end{eqnarray*}
\end{Prop}

\begin{Teo}[Regeneraci\'on Cruda]
Sup\'ongase que $X\left(t\right)$ es un proceso con valores positivo crudamente regenerativo en $T$, y def\'inase $M=\sup\left\{|X\left(t\right)|:t\leq T\right\}$. Si $T$ es no aritm\'etico y $M$ y $MT$ tienen media finita, entonces
\begin{eqnarray*}
lim_{t\rightarrow\infty}\esp\left[X\left(t\right)\right]=\frac{1}{\mu}\int_{\rea_{+}}h\left(s\right)ds,
\end{eqnarray*}
donde $h\left(t\right)=\esp\left[X\left(t\right)\indora\left(T>t\right)\right]$.
\end{Teo}

%___________________________________________________________________________________________
%
%\subsection*{Propiedades de los Procesos de Renovaci\'on}
%___________________________________________________________________________________________
%

Los tiempos $T_{n}$ est\'an relacionados con los conteos de $N\left(t\right)$ por

\begin{eqnarray*}
\left\{N\left(t\right)\geq n\right\}&=&\left\{T_{n}\leq t\right\}\\
T_{N\left(t\right)}\leq &t&<T_{N\left(t\right)+1},
\end{eqnarray*}

adem\'as $N\left(T_{n}\right)=n$, y 

\begin{eqnarray*}
N\left(t\right)=\max\left\{n:T_{n}\leq t\right\}=\min\left\{n:T_{n+1}>t\right\}
\end{eqnarray*}

Por propiedades de la convoluci\'on se sabe que

\begin{eqnarray*}
P\left\{T_{n}\leq t\right\}=F^{n\star}\left(t\right)
\end{eqnarray*}
que es la $n$-\'esima convoluci\'on de $F$. Entonces 

\begin{eqnarray*}
\left\{N\left(t\right)\geq n\right\}&=&\left\{T_{n}\leq t\right\}\\
P\left\{N\left(t\right)\leq n\right\}&=&1-F^{\left(n+1\right)\star}\left(t\right)
\end{eqnarray*}

Adem\'as usando el hecho de que $\esp\left[N\left(t\right)\right]=\sum_{n=1}^{\infty}P\left\{N\left(t\right)\geq n\right\}$
se tiene que

\begin{eqnarray*}
\esp\left[N\left(t\right)\right]=\sum_{n=1}^{\infty}F^{n\star}\left(t\right)
\end{eqnarray*}

\begin{Prop}
Para cada $t\geq0$, la funci\'on generadora de momentos $\esp\left[e^{\alpha N\left(t\right)}\right]$ existe para alguna $\alpha$ en una vecindad del 0, y de aqu\'i que $\esp\left[N\left(t\right)^{m}\right]<\infty$, para $m\geq1$.
\end{Prop}


\begin{Note}
Si el primer tiempo de renovaci\'on $\xi_{1}$ no tiene la misma distribuci\'on que el resto de las $\xi_{n}$, para $n\geq2$, a $N\left(t\right)$ se le llama Proceso de Renovaci\'on retardado, donde si $\xi$ tiene distribuci\'on $G$, entonces el tiempo $T_{n}$ de la $n$-\'esima renovaci\'on tiene distribuci\'on $G\star F^{\left(n-1\right)\star}\left(t\right)$
\end{Note}


\begin{Teo}
Para una constante $\mu\leq\infty$ ( o variable aleatoria), las siguientes expresiones son equivalentes:

\begin{eqnarray}
lim_{n\rightarrow\infty}n^{-1}T_{n}&=&\mu,\textrm{ c.s.}\\
lim_{t\rightarrow\infty}t^{-1}N\left(t\right)&=&1/\mu,\textrm{ c.s.}
\end{eqnarray}
\end{Teo}


Es decir, $T_{n}$ satisface la Ley Fuerte de los Grandes N\'umeros s\'i y s\'olo s\'i $N\left/t\right)$ la cumple.


\begin{Coro}[Ley Fuerte de los Grandes N\'umeros para Procesos de Renovaci\'on]
Si $N\left(t\right)$ es un proceso de renovaci\'on cuyos tiempos de inter-renovaci\'on tienen media $\mu\leq\infty$, entonces
\begin{eqnarray}
t^{-1}N\left(t\right)\rightarrow 1/\mu,\textrm{ c.s. cuando }t\rightarrow\infty.
\end{eqnarray}

\end{Coro}


Considerar el proceso estoc\'astico de valores reales $\left\{Z\left(t\right):t\geq0\right\}$ en el mismo espacio de probabilidad que $N\left(t\right)$

\begin{Def}
Para el proceso $\left\{Z\left(t\right):t\geq0\right\}$ se define la fluctuaci\'on m\'axima de $Z\left(t\right)$ en el intervalo $\left(T_{n-1},T_{n}\right]$:
\begin{eqnarray*}
M_{n}=\sup_{T_{n-1}<t\leq T_{n}}|Z\left(t\right)-Z\left(T_{n-1}\right)|
\end{eqnarray*}
\end{Def}

\begin{Teo}
Sup\'ongase que $n^{-1}T_{n}\rightarrow\mu$ c.s. cuando $n\rightarrow\infty$, donde $\mu\leq\infty$ es una constante o variable aleatoria. Sea $a$ una constante o variable aleatoria que puede ser infinita cuando $\mu$ es finita, y considere las expresiones l\'imite:
\begin{eqnarray}
lim_{n\rightarrow\infty}n^{-1}Z\left(T_{n}\right)&=&a,\textrm{ c.s.}\\
lim_{t\rightarrow\infty}t^{-1}Z\left(t\right)&=&a/\mu,\textrm{ c.s.}
\end{eqnarray}
La segunda expresi\'on implica la primera. Conversamente, la primera implica la segunda si el proceso $Z\left(t\right)$ es creciente, o si $lim_{n\rightarrow\infty}n^{-1}M_{n}=0$ c.s.
\end{Teo}

\begin{Coro}
Si $N\left(t\right)$ es un proceso de renovaci\'on, y $\left(Z\left(T_{n}\right)-Z\left(T_{n-1}\right),M_{n}\right)$, para $n\geq1$, son variables aleatorias independientes e id\'enticamente distribuidas con media finita, entonces,
\begin{eqnarray}
lim_{t\rightarrow\infty}t^{-1}Z\left(t\right)\rightarrow\frac{\esp\left[Z\left(T_{1}\right)-Z\left(T_{0}\right)\right]}{\esp\left[T_{1}\right]},\textrm{ c.s. cuando  }t\rightarrow\infty.
\end{eqnarray}
\end{Coro}



%___________________________________________________________________________________________
%
%\subsection{Propiedades de los Procesos de Renovaci\'on}
%___________________________________________________________________________________________
%

Los tiempos $T_{n}$ est\'an relacionados con los conteos de $N\left(t\right)$ por

\begin{eqnarray*}
\left\{N\left(t\right)\geq n\right\}&=&\left\{T_{n}\leq t\right\}\\
T_{N\left(t\right)}\leq &t&<T_{N\left(t\right)+1},
\end{eqnarray*}

adem\'as $N\left(T_{n}\right)=n$, y 

\begin{eqnarray*}
N\left(t\right)=\max\left\{n:T_{n}\leq t\right\}=\min\left\{n:T_{n+1}>t\right\}
\end{eqnarray*}

Por propiedades de la convoluci\'on se sabe que

\begin{eqnarray*}
P\left\{T_{n}\leq t\right\}=F^{n\star}\left(t\right)
\end{eqnarray*}
que es la $n$-\'esima convoluci\'on de $F$. Entonces 

\begin{eqnarray*}
\left\{N\left(t\right)\geq n\right\}&=&\left\{T_{n}\leq t\right\}\\
P\left\{N\left(t\right)\leq n\right\}&=&1-F^{\left(n+1\right)\star}\left(t\right)
\end{eqnarray*}

Adem\'as usando el hecho de que $\esp\left[N\left(t\right)\right]=\sum_{n=1}^{\infty}P\left\{N\left(t\right)\geq n\right\}$
se tiene que

\begin{eqnarray*}
\esp\left[N\left(t\right)\right]=\sum_{n=1}^{\infty}F^{n\star}\left(t\right)
\end{eqnarray*}

\begin{Prop}
Para cada $t\geq0$, la funci\'on generadora de momentos $\esp\left[e^{\alpha N\left(t\right)}\right]$ existe para alguna $\alpha$ en una vecindad del 0, y de aqu\'i que $\esp\left[N\left(t\right)^{m}\right]<\infty$, para $m\geq1$.
\end{Prop}


\begin{Note}
Si el primer tiempo de renovaci\'on $\xi_{1}$ no tiene la misma distribuci\'on que el resto de las $\xi_{n}$, para $n\geq2$, a $N\left(t\right)$ se le llama Proceso de Renovaci\'on retardado, donde si $\xi$ tiene distribuci\'on $G$, entonces el tiempo $T_{n}$ de la $n$-\'esima renovaci\'on tiene distribuci\'on $G\star F^{\left(n-1\right)\star}\left(t\right)$
\end{Note}


\begin{Teo}
Para una constante $\mu\leq\infty$ ( o variable aleatoria), las siguientes expresiones son equivalentes:

\begin{eqnarray}
lim_{n\rightarrow\infty}n^{-1}T_{n}&=&\mu,\textrm{ c.s.}\\
lim_{t\rightarrow\infty}t^{-1}N\left(t\right)&=&1/\mu,\textrm{ c.s.}
\end{eqnarray}
\end{Teo}


Es decir, $T_{n}$ satisface la Ley Fuerte de los Grandes N\'umeros s\'i y s\'olo s\'i $N\left/t\right)$ la cumple.


\begin{Coro}[Ley Fuerte de los Grandes N\'umeros para Procesos de Renovaci\'on]
Si $N\left(t\right)$ es un proceso de renovaci\'on cuyos tiempos de inter-renovaci\'on tienen media $\mu\leq\infty$, entonces
\begin{eqnarray}
t^{-1}N\left(t\right)\rightarrow 1/\mu,\textrm{ c.s. cuando }t\rightarrow\infty.
\end{eqnarray}

\end{Coro}


Considerar el proceso estoc\'astico de valores reales $\left\{Z\left(t\right):t\geq0\right\}$ en el mismo espacio de probabilidad que $N\left(t\right)$

\begin{Def}
Para el proceso $\left\{Z\left(t\right):t\geq0\right\}$ se define la fluctuaci\'on m\'axima de $Z\left(t\right)$ en el intervalo $\left(T_{n-1},T_{n}\right]$:
\begin{eqnarray*}
M_{n}=\sup_{T_{n-1}<t\leq T_{n}}|Z\left(t\right)-Z\left(T_{n-1}\right)|
\end{eqnarray*}
\end{Def}

\begin{Teo}
Sup\'ongase que $n^{-1}T_{n}\rightarrow\mu$ c.s. cuando $n\rightarrow\infty$, donde $\mu\leq\infty$ es una constante o variable aleatoria. Sea $a$ una constante o variable aleatoria que puede ser infinita cuando $\mu$ es finita, y considere las expresiones l\'imite:
\begin{eqnarray}
lim_{n\rightarrow\infty}n^{-1}Z\left(T_{n}\right)&=&a,\textrm{ c.s.}\\
lim_{t\rightarrow\infty}t^{-1}Z\left(t\right)&=&a/\mu,\textrm{ c.s.}
\end{eqnarray}
La segunda expresi\'on implica la primera. Conversamente, la primera implica la segunda si el proceso $Z\left(t\right)$ es creciente, o si $lim_{n\rightarrow\infty}n^{-1}M_{n}=0$ c.s.
\end{Teo}

\begin{Coro}
Si $N\left(t\right)$ es un proceso de renovaci\'on, y $\left(Z\left(T_{n}\right)-Z\left(T_{n-1}\right),M_{n}\right)$, para $n\geq1$, son variables aleatorias independientes e id\'enticamente distribuidas con media finita, entonces,
\begin{eqnarray}
lim_{t\rightarrow\infty}t^{-1}Z\left(t\right)\rightarrow\frac{\esp\left[Z\left(T_{1}\right)-Z\left(T_{0}\right)\right]}{\esp\left[T_{1}\right]},\textrm{ c.s. cuando  }t\rightarrow\infty.
\end{eqnarray}
\end{Coro}


%___________________________________________________________________________________________
%
%\subsection{Propiedades de los Procesos de Renovaci\'on}
%___________________________________________________________________________________________
%

Los tiempos $T_{n}$ est\'an relacionados con los conteos de $N\left(t\right)$ por

\begin{eqnarray*}
\left\{N\left(t\right)\geq n\right\}&=&\left\{T_{n}\leq t\right\}\\
T_{N\left(t\right)}\leq &t&<T_{N\left(t\right)+1},
\end{eqnarray*}

adem\'as $N\left(T_{n}\right)=n$, y 

\begin{eqnarray*}
N\left(t\right)=\max\left\{n:T_{n}\leq t\right\}=\min\left\{n:T_{n+1}>t\right\}
\end{eqnarray*}

Por propiedades de la convoluci\'on se sabe que

\begin{eqnarray*}
P\left\{T_{n}\leq t\right\}=F^{n\star}\left(t\right)
\end{eqnarray*}
que es la $n$-\'esima convoluci\'on de $F$. Entonces 

\begin{eqnarray*}
\left\{N\left(t\right)\geq n\right\}&=&\left\{T_{n}\leq t\right\}\\
P\left\{N\left(t\right)\leq n\right\}&=&1-F^{\left(n+1\right)\star}\left(t\right)
\end{eqnarray*}

Adem\'as usando el hecho de que $\esp\left[N\left(t\right)\right]=\sum_{n=1}^{\infty}P\left\{N\left(t\right)\geq n\right\}$
se tiene que

\begin{eqnarray*}
\esp\left[N\left(t\right)\right]=\sum_{n=1}^{\infty}F^{n\star}\left(t\right)
\end{eqnarray*}

\begin{Prop}
Para cada $t\geq0$, la funci\'on generadora de momentos $\esp\left[e^{\alpha N\left(t\right)}\right]$ existe para alguna $\alpha$ en una vecindad del 0, y de aqu\'i que $\esp\left[N\left(t\right)^{m}\right]<\infty$, para $m\geq1$.
\end{Prop}


\begin{Note}
Si el primer tiempo de renovaci\'on $\xi_{1}$ no tiene la misma distribuci\'on que el resto de las $\xi_{n}$, para $n\geq2$, a $N\left(t\right)$ se le llama Proceso de Renovaci\'on retardado, donde si $\xi$ tiene distribuci\'on $G$, entonces el tiempo $T_{n}$ de la $n$-\'esima renovaci\'on tiene distribuci\'on $G\star F^{\left(n-1\right)\star}\left(t\right)$
\end{Note}


\begin{Teo}
Para una constante $\mu\leq\infty$ ( o variable aleatoria), las siguientes expresiones son equivalentes:

\begin{eqnarray}
lim_{n\rightarrow\infty}n^{-1}T_{n}&=&\mu,\textrm{ c.s.}\\
lim_{t\rightarrow\infty}t^{-1}N\left(t\right)&=&1/\mu,\textrm{ c.s.}
\end{eqnarray}
\end{Teo}


Es decir, $T_{n}$ satisface la Ley Fuerte de los Grandes N\'umeros s\'i y s\'olo s\'i $N\left/t\right)$ la cumple.


\begin{Coro}[Ley Fuerte de los Grandes N\'umeros para Procesos de Renovaci\'on]
Si $N\left(t\right)$ es un proceso de renovaci\'on cuyos tiempos de inter-renovaci\'on tienen media $\mu\leq\infty$, entonces
\begin{eqnarray}
t^{-1}N\left(t\right)\rightarrow 1/\mu,\textrm{ c.s. cuando }t\rightarrow\infty.
\end{eqnarray}

\end{Coro}


Considerar el proceso estoc\'astico de valores reales $\left\{Z\left(t\right):t\geq0\right\}$ en el mismo espacio de probabilidad que $N\left(t\right)$

\begin{Def}
Para el proceso $\left\{Z\left(t\right):t\geq0\right\}$ se define la fluctuaci\'on m\'axima de $Z\left(t\right)$ en el intervalo $\left(T_{n-1},T_{n}\right]$:
\begin{eqnarray*}
M_{n}=\sup_{T_{n-1}<t\leq T_{n}}|Z\left(t\right)-Z\left(T_{n-1}\right)|
\end{eqnarray*}
\end{Def}

\begin{Teo}
Sup\'ongase que $n^{-1}T_{n}\rightarrow\mu$ c.s. cuando $n\rightarrow\infty$, donde $\mu\leq\infty$ es una constante o variable aleatoria. Sea $a$ una constante o variable aleatoria que puede ser infinita cuando $\mu$ es finita, y considere las expresiones l\'imite:
\begin{eqnarray}
lim_{n\rightarrow\infty}n^{-1}Z\left(T_{n}\right)&=&a,\textrm{ c.s.}\\
lim_{t\rightarrow\infty}t^{-1}Z\left(t\right)&=&a/\mu,\textrm{ c.s.}
\end{eqnarray}
La segunda expresi\'on implica la primera. Conversamente, la primera implica la segunda si el proceso $Z\left(t\right)$ es creciente, o si $lim_{n\rightarrow\infty}n^{-1}M_{n}=0$ c.s.
\end{Teo}

\begin{Coro}
Si $N\left(t\right)$ es un proceso de renovaci\'on, y $\left(Z\left(T_{n}\right)-Z\left(T_{n-1}\right),M_{n}\right)$, para $n\geq1$, son variables aleatorias independientes e id\'enticamente distribuidas con media finita, entonces,
\begin{eqnarray}
lim_{t\rightarrow\infty}t^{-1}Z\left(t\right)\rightarrow\frac{\esp\left[Z\left(T_{1}\right)-Z\left(T_{0}\right)\right]}{\esp\left[T_{1}\right]},\textrm{ c.s. cuando  }t\rightarrow\infty.
\end{eqnarray}
\end{Coro}

%___________________________________________________________________________________________
%
%\subsection{Propiedades de los Procesos de Renovaci\'on}
%___________________________________________________________________________________________
%

Los tiempos $T_{n}$ est\'an relacionados con los conteos de $N\left(t\right)$ por

\begin{eqnarray*}
\left\{N\left(t\right)\geq n\right\}&=&\left\{T_{n}\leq t\right\}\\
T_{N\left(t\right)}\leq &t&<T_{N\left(t\right)+1},
\end{eqnarray*}

adem\'as $N\left(T_{n}\right)=n$, y 

\begin{eqnarray*}
N\left(t\right)=\max\left\{n:T_{n}\leq t\right\}=\min\left\{n:T_{n+1}>t\right\}
\end{eqnarray*}

Por propiedades de la convoluci\'on se sabe que

\begin{eqnarray*}
P\left\{T_{n}\leq t\right\}=F^{n\star}\left(t\right)
\end{eqnarray*}
que es la $n$-\'esima convoluci\'on de $F$. Entonces 

\begin{eqnarray*}
\left\{N\left(t\right)\geq n\right\}&=&\left\{T_{n}\leq t\right\}\\
P\left\{N\left(t\right)\leq n\right\}&=&1-F^{\left(n+1\right)\star}\left(t\right)
\end{eqnarray*}

Adem\'as usando el hecho de que $\esp\left[N\left(t\right)\right]=\sum_{n=1}^{\infty}P\left\{N\left(t\right)\geq n\right\}$
se tiene que

\begin{eqnarray*}
\esp\left[N\left(t\right)\right]=\sum_{n=1}^{\infty}F^{n\star}\left(t\right)
\end{eqnarray*}

\begin{Prop}
Para cada $t\geq0$, la funci\'on generadora de momentos $\esp\left[e^{\alpha N\left(t\right)}\right]$ existe para alguna $\alpha$ en una vecindad del 0, y de aqu\'i que $\esp\left[N\left(t\right)^{m}\right]<\infty$, para $m\geq1$.
\end{Prop}


\begin{Note}
Si el primer tiempo de renovaci\'on $\xi_{1}$ no tiene la misma distribuci\'on que el resto de las $\xi_{n}$, para $n\geq2$, a $N\left(t\right)$ se le llama Proceso de Renovaci\'on retardado, donde si $\xi$ tiene distribuci\'on $G$, entonces el tiempo $T_{n}$ de la $n$-\'esima renovaci\'on tiene distribuci\'on $G\star F^{\left(n-1\right)\star}\left(t\right)$
\end{Note}


\begin{Teo}
Para una constante $\mu\leq\infty$ ( o variable aleatoria), las siguientes expresiones son equivalentes:

\begin{eqnarray}
lim_{n\rightarrow\infty}n^{-1}T_{n}&=&\mu,\textrm{ c.s.}\\
lim_{t\rightarrow\infty}t^{-1}N\left(t\right)&=&1/\mu,\textrm{ c.s.}
\end{eqnarray}
\end{Teo}


Es decir, $T_{n}$ satisface la Ley Fuerte de los Grandes N\'umeros s\'i y s\'olo s\'i $N\left/t\right)$ la cumple.


\begin{Coro}[Ley Fuerte de los Grandes N\'umeros para Procesos de Renovaci\'on]
Si $N\left(t\right)$ es un proceso de renovaci\'on cuyos tiempos de inter-renovaci\'on tienen media $\mu\leq\infty$, entonces
\begin{eqnarray}
t^{-1}N\left(t\right)\rightarrow 1/\mu,\textrm{ c.s. cuando }t\rightarrow\infty.
\end{eqnarray}

\end{Coro}


Considerar el proceso estoc\'astico de valores reales $\left\{Z\left(t\right):t\geq0\right\}$ en el mismo espacio de probabilidad que $N\left(t\right)$

\begin{Def}
Para el proceso $\left\{Z\left(t\right):t\geq0\right\}$ se define la fluctuaci\'on m\'axima de $Z\left(t\right)$ en el intervalo $\left(T_{n-1},T_{n}\right]$:
\begin{eqnarray*}
M_{n}=\sup_{T_{n-1}<t\leq T_{n}}|Z\left(t\right)-Z\left(T_{n-1}\right)|
\end{eqnarray*}
\end{Def}

\begin{Teo}
Sup\'ongase que $n^{-1}T_{n}\rightarrow\mu$ c.s. cuando $n\rightarrow\infty$, donde $\mu\leq\infty$ es una constante o variable aleatoria. Sea $a$ una constante o variable aleatoria que puede ser infinita cuando $\mu$ es finita, y considere las expresiones l\'imite:
\begin{eqnarray}
lim_{n\rightarrow\infty}n^{-1}Z\left(T_{n}\right)&=&a,\textrm{ c.s.}\\
lim_{t\rightarrow\infty}t^{-1}Z\left(t\right)&=&a/\mu,\textrm{ c.s.}
\end{eqnarray}
La segunda expresi\'on implica la primera. Conversamente, la primera implica la segunda si el proceso $Z\left(t\right)$ es creciente, o si $lim_{n\rightarrow\infty}n^{-1}M_{n}=0$ c.s.
\end{Teo}

\begin{Coro}
Si $N\left(t\right)$ es un proceso de renovaci\'on, y $\left(Z\left(T_{n}\right)-Z\left(T_{n-1}\right),M_{n}\right)$, para $n\geq1$, son variables aleatorias independientes e id\'enticamente distribuidas con media finita, entonces,
\begin{eqnarray}
lim_{t\rightarrow\infty}t^{-1}Z\left(t\right)\rightarrow\frac{\esp\left[Z\left(T_{1}\right)-Z\left(T_{0}\right)\right]}{\esp\left[T_{1}\right]},\textrm{ c.s. cuando  }t\rightarrow\infty.
\end{eqnarray}
\end{Coro}
%___________________________________________________________________________________________
%
%\subsection{Propiedades de los Procesos de Renovaci\'on}
%___________________________________________________________________________________________
%

Los tiempos $T_{n}$ est\'an relacionados con los conteos de $N\left(t\right)$ por

\begin{eqnarray*}
\left\{N\left(t\right)\geq n\right\}&=&\left\{T_{n}\leq t\right\}\\
T_{N\left(t\right)}\leq &t&<T_{N\left(t\right)+1},
\end{eqnarray*}

adem\'as $N\left(T_{n}\right)=n$, y 

\begin{eqnarray*}
N\left(t\right)=\max\left\{n:T_{n}\leq t\right\}=\min\left\{n:T_{n+1}>t\right\}
\end{eqnarray*}

Por propiedades de la convoluci\'on se sabe que

\begin{eqnarray*}
P\left\{T_{n}\leq t\right\}=F^{n\star}\left(t\right)
\end{eqnarray*}
que es la $n$-\'esima convoluci\'on de $F$. Entonces 

\begin{eqnarray*}
\left\{N\left(t\right)\geq n\right\}&=&\left\{T_{n}\leq t\right\}\\
P\left\{N\left(t\right)\leq n\right\}&=&1-F^{\left(n+1\right)\star}\left(t\right)
\end{eqnarray*}

Adem\'as usando el hecho de que $\esp\left[N\left(t\right)\right]=\sum_{n=1}^{\infty}P\left\{N\left(t\right)\geq n\right\}$
se tiene que

\begin{eqnarray*}
\esp\left[N\left(t\right)\right]=\sum_{n=1}^{\infty}F^{n\star}\left(t\right)
\end{eqnarray*}

\begin{Prop}
Para cada $t\geq0$, la funci\'on generadora de momentos $\esp\left[e^{\alpha N\left(t\right)}\right]$ existe para alguna $\alpha$ en una vecindad del 0, y de aqu\'i que $\esp\left[N\left(t\right)^{m}\right]<\infty$, para $m\geq1$.
\end{Prop}


\begin{Note}
Si el primer tiempo de renovaci\'on $\xi_{1}$ no tiene la misma distribuci\'on que el resto de las $\xi_{n}$, para $n\geq2$, a $N\left(t\right)$ se le llama Proceso de Renovaci\'on retardado, donde si $\xi$ tiene distribuci\'on $G$, entonces el tiempo $T_{n}$ de la $n$-\'esima renovaci\'on tiene distribuci\'on $G\star F^{\left(n-1\right)\star}\left(t\right)$
\end{Note}


\begin{Teo}
Para una constante $\mu\leq\infty$ ( o variable aleatoria), las siguientes expresiones son equivalentes:

\begin{eqnarray}
lim_{n\rightarrow\infty}n^{-1}T_{n}&=&\mu,\textrm{ c.s.}\\
lim_{t\rightarrow\infty}t^{-1}N\left(t\right)&=&1/\mu,\textrm{ c.s.}
\end{eqnarray}
\end{Teo}


Es decir, $T_{n}$ satisface la Ley Fuerte de los Grandes N\'umeros s\'i y s\'olo s\'i $N\left/t\right)$ la cumple.


\begin{Coro}[Ley Fuerte de los Grandes N\'umeros para Procesos de Renovaci\'on]
Si $N\left(t\right)$ es un proceso de renovaci\'on cuyos tiempos de inter-renovaci\'on tienen media $\mu\leq\infty$, entonces
\begin{eqnarray}
t^{-1}N\left(t\right)\rightarrow 1/\mu,\textrm{ c.s. cuando }t\rightarrow\infty.
\end{eqnarray}

\end{Coro}


Considerar el proceso estoc\'astico de valores reales $\left\{Z\left(t\right):t\geq0\right\}$ en el mismo espacio de probabilidad que $N\left(t\right)$

\begin{Def}
Para el proceso $\left\{Z\left(t\right):t\geq0\right\}$ se define la fluctuaci\'on m\'axima de $Z\left(t\right)$ en el intervalo $\left(T_{n-1},T_{n}\right]$:
\begin{eqnarray*}
M_{n}=\sup_{T_{n-1}<t\leq T_{n}}|Z\left(t\right)-Z\left(T_{n-1}\right)|
\end{eqnarray*}
\end{Def}

\begin{Teo}
Sup\'ongase que $n^{-1}T_{n}\rightarrow\mu$ c.s. cuando $n\rightarrow\infty$, donde $\mu\leq\infty$ es una constante o variable aleatoria. Sea $a$ una constante o variable aleatoria que puede ser infinita cuando $\mu$ es finita, y considere las expresiones l\'imite:
\begin{eqnarray}
lim_{n\rightarrow\infty}n^{-1}Z\left(T_{n}\right)&=&a,\textrm{ c.s.}\\
lim_{t\rightarrow\infty}t^{-1}Z\left(t\right)&=&a/\mu,\textrm{ c.s.}
\end{eqnarray}
La segunda expresi\'on implica la primera. Conversamente, la primera implica la segunda si el proceso $Z\left(t\right)$ es creciente, o si $lim_{n\rightarrow\infty}n^{-1}M_{n}=0$ c.s.
\end{Teo}

\begin{Coro}
Si $N\left(t\right)$ es un proceso de renovaci\'on, y $\left(Z\left(T_{n}\right)-Z\left(T_{n-1}\right),M_{n}\right)$, para $n\geq1$, son variables aleatorias independientes e id\'enticamente distribuidas con media finita, entonces,
\begin{eqnarray}
lim_{t\rightarrow\infty}t^{-1}Z\left(t\right)\rightarrow\frac{\esp\left[Z\left(T_{1}\right)-Z\left(T_{0}\right)\right]}{\esp\left[T_{1}\right]},\textrm{ c.s. cuando  }t\rightarrow\infty.
\end{eqnarray}
\end{Coro}


%___________________________________________________________________________________________
%
%\subsection*{Funci\'on de Renovaci\'on}
%___________________________________________________________________________________________
%


\begin{Def}
Sea $h\left(t\right)$ funci\'on de valores reales en $\rea$ acotada en intervalos finitos e igual a cero para $t<0$ La ecuaci\'on de renovaci\'on para $h\left(t\right)$ y la distribuci\'on $F$ es

\begin{eqnarray}\label{Ec.Renovacion}
H\left(t\right)=h\left(t\right)+\int_{\left[0,t\right]}H\left(t-s\right)dF\left(s\right)\textrm{,    }t\geq0,
\end{eqnarray}
donde $H\left(t\right)$ es una funci\'on de valores reales. Esto es $H=h+F\star H$. Decimos que $H\left(t\right)$ es soluci\'on de esta ecuaci\'on si satisface la ecuaci\'on, y es acotada en intervalos finitos e iguales a cero para $t<0$.
\end{Def}

\begin{Prop}
La funci\'on $U\star h\left(t\right)$ es la \'unica soluci\'on de la ecuaci\'on de renovaci\'on (\ref{Ec.Renovacion}).
\end{Prop}

\begin{Teo}[Teorema Renovaci\'on Elemental]
\begin{eqnarray*}
t^{-1}U\left(t\right)\rightarrow 1/\mu\textrm{,    cuando }t\rightarrow\infty.
\end{eqnarray*}
\end{Teo}

%___________________________________________________________________________________________
%
%\subsection{Funci\'on de Renovaci\'on}
%___________________________________________________________________________________________
%


Sup\'ongase que $N\left(t\right)$ es un proceso de renovaci\'on con distribuci\'on $F$ con media finita $\mu$.

\begin{Def}
La funci\'on de renovaci\'on asociada con la distribuci\'on $F$, del proceso $N\left(t\right)$, es
\begin{eqnarray*}
U\left(t\right)=\sum_{n=1}^{\infty}F^{n\star}\left(t\right),\textrm{   }t\geq0,
\end{eqnarray*}
donde $F^{0\star}\left(t\right)=\indora\left(t\geq0\right)$.
\end{Def}


\begin{Prop}
Sup\'ongase que la distribuci\'on de inter-renovaci\'on $F$ tiene densidad $f$. Entonces $U\left(t\right)$ tambi\'en tiene densidad, para $t>0$, y es $U^{'}\left(t\right)=\sum_{n=0}^{\infty}f^{n\star}\left(t\right)$. Adem\'as
\begin{eqnarray*}
\prob\left\{N\left(t\right)>N\left(t-\right)\right\}=0\textrm{,   }t\geq0.
\end{eqnarray*}
\end{Prop}

\begin{Def}
La Transformada de Laplace-Stieljes de $F$ est\'a dada por

\begin{eqnarray*}
\hat{F}\left(\alpha\right)=\int_{\rea_{+}}e^{-\alpha t}dF\left(t\right)\textrm{,  }\alpha\geq0.
\end{eqnarray*}
\end{Def}

Entonces

\begin{eqnarray*}
\hat{U}\left(\alpha\right)=\sum_{n=0}^{\infty}\hat{F^{n\star}}\left(\alpha\right)=\sum_{n=0}^{\infty}\hat{F}\left(\alpha\right)^{n}=\frac{1}{1-\hat{F}\left(\alpha\right)}.
\end{eqnarray*}


\begin{Prop}
La Transformada de Laplace $\hat{U}\left(\alpha\right)$ y $\hat{F}\left(\alpha\right)$ determina una a la otra de manera \'unica por la relaci\'on $\hat{U}\left(\alpha\right)=\frac{1}{1-\hat{F}\left(\alpha\right)}$.
\end{Prop}


\begin{Note}
Un proceso de renovaci\'on $N\left(t\right)$ cuyos tiempos de inter-renovaci\'on tienen media finita, es un proceso Poisson con tasa $\lambda$ si y s\'olo s\'i $\esp\left[U\left(t\right)\right]=\lambda t$, para $t\geq0$.
\end{Note}


\begin{Teo}
Sea $N\left(t\right)$ un proceso puntual simple con puntos de localizaci\'on $T_{n}$ tal que $\eta\left(t\right)=\esp\left[N\left(\right)\right]$ es finita para cada $t$. Entonces para cualquier funci\'on $f:\rea_{+}\rightarrow\rea$,
\begin{eqnarray*}
\esp\left[\sum_{n=1}^{N\left(\right)}f\left(T_{n}\right)\right]=\int_{\left(0,t\right]}f\left(s\right)d\eta\left(s\right)\textrm{,  }t\geq0,
\end{eqnarray*}
suponiendo que la integral exista. Adem\'as si $X_{1},X_{2},\ldots$ son variables aleatorias definidas en el mismo espacio de probabilidad que el proceso $N\left(t\right)$ tal que $\esp\left[X_{n}|T_{n}=s\right]=f\left(s\right)$, independiente de $n$. Entonces
\begin{eqnarray*}
\esp\left[\sum_{n=1}^{N\left(t\right)}X_{n}\right]=\int_{\left(0,t\right]}f\left(s\right)d\eta\left(s\right)\textrm{,  }t\geq0,
\end{eqnarray*} 
suponiendo que la integral exista. 
\end{Teo}

\begin{Coro}[Identidad de Wald para Renovaciones]
Para el proceso de renovaci\'on $N\left(t\right)$,
\begin{eqnarray*}
\esp\left[T_{N\left(t\right)+1}\right]=\mu\esp\left[N\left(t\right)+1\right]\textrm{,  }t\geq0,
\end{eqnarray*}  
\end{Coro}

%______________________________________________________________________
%\subsection{Procesos de Renovaci\'on}
%______________________________________________________________________

\begin{Def}\label{Def.Tn}
Sean $0\leq T_{1}\leq T_{2}\leq \ldots$ son tiempos aleatorios infinitos en los cuales ocurren ciertos eventos. El n\'umero de tiempos $T_{n}$ en el intervalo $\left[0,t\right)$ es

\begin{eqnarray}
N\left(t\right)=\sum_{n=1}^{\infty}\indora\left(T_{n}\leq t\right),
\end{eqnarray}
para $t\geq0$.
\end{Def}

Si se consideran los puntos $T_{n}$ como elementos de $\rea_{+}$, y $N\left(t\right)$ es el n\'umero de puntos en $\rea$. El proceso denotado por $\left\{N\left(t\right):t\geq0\right\}$, denotado por $N\left(t\right)$, es un proceso puntual en $\rea_{+}$. Los $T_{n}$ son los tiempos de ocurrencia, el proceso puntual $N\left(t\right)$ es simple si su n\'umero de ocurrencias son distintas: $0<T_{1}<T_{2}<\ldots$ casi seguramente.

\begin{Def}
Un proceso puntual $N\left(t\right)$ es un proceso de renovaci\'on si los tiempos de interocurrencia $\xi_{n}=T_{n}-T_{n-1}$, para $n\geq1$, son independientes e identicamente distribuidos con distribuci\'on $F$, donde $F\left(0\right)=0$ y $T_{0}=0$. Los $T_{n}$ son llamados tiempos de renovaci\'on, referente a la independencia o renovaci\'on de la informaci\'on estoc\'astica en estos tiempos. Los $\xi_{n}$ son los tiempos de inter-renovaci\'on, y $N\left(t\right)$ es el n\'umero de renovaciones en el intervalo $\left[0,t\right)$
\end{Def}


\begin{Note}
Para definir un proceso de renovaci\'on para cualquier contexto, solamente hay que especificar una distribuci\'on $F$, con $F\left(0\right)=0$, para los tiempos de inter-renovaci\'on. La funci\'on $F$ en turno degune las otra variables aleatorias. De manera formal, existe un espacio de probabilidad y una sucesi\'on de variables aleatorias $\xi_{1},\xi_{2},\ldots$ definidas en este con distribuci\'on $F$. Entonces las otras cantidades son $T_{n}=\sum_{k=1}^{n}\xi_{k}$ y $N\left(t\right)=\sum_{n=1}^{\infty}\indora\left(T_{n}\leq t\right)$, donde $T_{n}\rightarrow\infty$ casi seguramente por la Ley Fuerte de los Grandes Números.
\end{Note}

%___________________________________________________________________________________________
%
\subsection{Renewal and Regenerative Processes: Serfozo\cite{Serfozo}}
%___________________________________________________________________________________________
%
\begin{Def}\label{Def.Tn}
Sean $0\leq T_{1}\leq T_{2}\leq \ldots$ son tiempos aleatorios infinitos en los cuales ocurren ciertos eventos. El n\'umero de tiempos $T_{n}$ en el intervalo $\left[0,t\right)$ es

\begin{eqnarray}
N\left(t\right)=\sum_{n=1}^{\infty}\indora\left(T_{n}\leq t\right),
\end{eqnarray}
para $t\geq0$.
\end{Def}

Si se consideran los puntos $T_{n}$ como elementos de $\rea_{+}$, y $N\left(t\right)$ es el n\'umero de puntos en $\rea$. El proceso denotado por $\left\{N\left(t\right):t\geq0\right\}$, denotado por $N\left(t\right)$, es un proceso puntual en $\rea_{+}$. Los $T_{n}$ son los tiempos de ocurrencia, el proceso puntual $N\left(t\right)$ es simple si su n\'umero de ocurrencias son distintas: $0<T_{1}<T_{2}<\ldots$ casi seguramente.

\begin{Def}
Un proceso puntual $N\left(t\right)$ es un proceso de renovaci\'on si los tiempos de interocurrencia $\xi_{n}=T_{n}-T_{n-1}$, para $n\geq1$, son independientes e identicamente distribuidos con distribuci\'on $F$, donde $F\left(0\right)=0$ y $T_{0}=0$. Los $T_{n}$ son llamados tiempos de renovaci\'on, referente a la independencia o renovaci\'on de la informaci\'on estoc\'astica en estos tiempos. Los $\xi_{n}$ son los tiempos de inter-renovaci\'on, y $N\left(t\right)$ es el n\'umero de renovaciones en el intervalo $\left[0,t\right)$
\end{Def}


\begin{Note}
Para definir un proceso de renovaci\'on para cualquier contexto, solamente hay que especificar una distribuci\'on $F$, con $F\left(0\right)=0$, para los tiempos de inter-renovaci\'on. La funci\'on $F$ en turno degune las otra variables aleatorias. De manera formal, existe un espacio de probabilidad y una sucesi\'on de variables aleatorias $\xi_{1},\xi_{2},\ldots$ definidas en este con distribuci\'on $F$. Entonces las otras cantidades son $T_{n}=\sum_{k=1}^{n}\xi_{k}$ y $N\left(t\right)=\sum_{n=1}^{\infty}\indora\left(T_{n}\leq t\right)$, donde $T_{n}\rightarrow\infty$ casi seguramente por la Ley Fuerte de los Grandes N\'umeros.
\end{Note}







Los tiempos $T_{n}$ est\'an relacionados con los conteos de $N\left(t\right)$ por

\begin{eqnarray*}
\left\{N\left(t\right)\geq n\right\}&=&\left\{T_{n}\leq t\right\}\\
T_{N\left(t\right)}\leq &t&<T_{N\left(t\right)+1},
\end{eqnarray*}

adem\'as $N\left(T_{n}\right)=n$, y 

\begin{eqnarray*}
N\left(t\right)=\max\left\{n:T_{n}\leq t\right\}=\min\left\{n:T_{n+1}>t\right\}
\end{eqnarray*}

Por propiedades de la convoluci\'on se sabe que

\begin{eqnarray*}
P\left\{T_{n}\leq t\right\}=F^{n\star}\left(t\right)
\end{eqnarray*}
que es la $n$-\'esima convoluci\'on de $F$. Entonces 

\begin{eqnarray*}
\left\{N\left(t\right)\geq n\right\}&=&\left\{T_{n}\leq t\right\}\\
P\left\{N\left(t\right)\leq n\right\}&=&1-F^{\left(n+1\right)\star}\left(t\right)
\end{eqnarray*}

Adem\'as usando el hecho de que $\esp\left[N\left(t\right)\right]=\sum_{n=1}^{\infty}P\left\{N\left(t\right)\geq n\right\}$
se tiene que

\begin{eqnarray*}
\esp\left[N\left(t\right)\right]=\sum_{n=1}^{\infty}F^{n\star}\left(t\right)
\end{eqnarray*}

\begin{Prop}
Para cada $t\geq0$, la funci\'on generadora de momentos $\esp\left[e^{\alpha N\left(t\right)}\right]$ existe para alguna $\alpha$ en una vecindad del 0, y de aqu\'i que $\esp\left[N\left(t\right)^{m}\right]<\infty$, para $m\geq1$.
\end{Prop}

\begin{Ejem}[\textbf{Proceso Poisson}]

Suponga que se tienen tiempos de inter-renovaci\'on \textit{i.i.d.} del proceso de renovaci\'on $N\left(t\right)$ tienen distribuci\'on exponencial $F\left(t\right)=q-e^{-\lambda t}$ con tasa $\lambda$. Entonces $N\left(t\right)$ es un proceso Poisson con tasa $\lambda$.

\end{Ejem}


\begin{Note}
Si el primer tiempo de renovaci\'on $\xi_{1}$ no tiene la misma distribuci\'on que el resto de las $\xi_{n}$, para $n\geq2$, a $N\left(t\right)$ se le llama Proceso de Renovaci\'on retardado, donde si $\xi$ tiene distribuci\'on $G$, entonces el tiempo $T_{n}$ de la $n$-\'esima renovaci\'on tiene distribuci\'on $G\star F^{\left(n-1\right)\star}\left(t\right)$
\end{Note}


\begin{Teo}
Para una constante $\mu\leq\infty$ ( o variable aleatoria), las siguientes expresiones son equivalentes:

\begin{eqnarray}
lim_{n\rightarrow\infty}n^{-1}T_{n}&=&\mu,\textrm{ c.s.}\\
lim_{t\rightarrow\infty}t^{-1}N\left(t\right)&=&1/\mu,\textrm{ c.s.}
\end{eqnarray}
\end{Teo}


Es decir, $T_{n}$ satisface la Ley Fuerte de los Grandes N\'umeros s\'i y s\'olo s\'i $N\left/t\right)$ la cumple.


\begin{Coro}[Ley Fuerte de los Grandes N\'umeros para Procesos de Renovaci\'on]
Si $N\left(t\right)$ es un proceso de renovaci\'on cuyos tiempos de inter-renovaci\'on tienen media $\mu\leq\infty$, entonces
\begin{eqnarray}
t^{-1}N\left(t\right)\rightarrow 1/\mu,\textrm{ c.s. cuando }t\rightarrow\infty.
\end{eqnarray}

\end{Coro}


Considerar el proceso estoc\'astico de valores reales $\left\{Z\left(t\right):t\geq0\right\}$ en el mismo espacio de probabilidad que $N\left(t\right)$

\begin{Def}
Para el proceso $\left\{Z\left(t\right):t\geq0\right\}$ se define la fluctuaci\'on m\'axima de $Z\left(t\right)$ en el intervalo $\left(T_{n-1},T_{n}\right]$:
\begin{eqnarray*}
M_{n}=\sup_{T_{n-1}<t\leq T_{n}}|Z\left(t\right)-Z\left(T_{n-1}\right)|
\end{eqnarray*}
\end{Def}

\begin{Teo}
Sup\'ongase que $n^{-1}T_{n}\rightarrow\mu$ c.s. cuando $n\rightarrow\infty$, donde $\mu\leq\infty$ es una constante o variable aleatoria. Sea $a$ una constante o variable aleatoria que puede ser infinita cuando $\mu$ es finita, y considere las expresiones l\'imite:
\begin{eqnarray}
lim_{n\rightarrow\infty}n^{-1}Z\left(T_{n}\right)&=&a,\textrm{ c.s.}\\
lim_{t\rightarrow\infty}t^{-1}Z\left(t\right)&=&a/\mu,\textrm{ c.s.}
\end{eqnarray}
La segunda expresi\'on implica la primera. Conversamente, la primera implica la segunda si el proceso $Z\left(t\right)$ es creciente, o si $lim_{n\rightarrow\infty}n^{-1}M_{n}=0$ c.s.
\end{Teo}

\begin{Coro}
Si $N\left(t\right)$ es un proceso de renovaci\'on, y $\left(Z\left(T_{n}\right)-Z\left(T_{n-1}\right),M_{n}\right)$, para $n\geq1$, son variables aleatorias independientes e id\'enticamente distribuidas con media finita, entonces,
\begin{eqnarray}
lim_{t\rightarrow\infty}t^{-1}Z\left(t\right)\rightarrow\frac{\esp\left[Z\left(T_{1}\right)-Z\left(T_{0}\right)\right]}{\esp\left[T_{1}\right]},\textrm{ c.s. cuando  }t\rightarrow\infty.
\end{eqnarray}
\end{Coro}


Sup\'ongase que $N\left(t\right)$ es un proceso de renovaci\'on con distribuci\'on $F$ con media finita $\mu$.

\begin{Def}
La funci\'on de renovaci\'on asociada con la distribuci\'on $F$, del proceso $N\left(t\right)$, es
\begin{eqnarray*}
U\left(t\right)=\sum_{n=1}^{\infty}F^{n\star}\left(t\right),\textrm{   }t\geq0,
\end{eqnarray*}
donde $F^{0\star}\left(t\right)=\indora\left(t\geq0\right)$.
\end{Def}


\begin{Prop}
Sup\'ongase que la distribuci\'on de inter-renovaci\'on $F$ tiene densidad $f$. Entonces $U\left(t\right)$ tambi\'en tiene densidad, para $t>0$, y es $U^{'}\left(t\right)=\sum_{n=0}^{\infty}f^{n\star}\left(t\right)$. Adem\'as
\begin{eqnarray*}
\prob\left\{N\left(t\right)>N\left(t-\right)\right\}=0\textrm{,   }t\geq0.
\end{eqnarray*}
\end{Prop}

\begin{Def}
La Transformada de Laplace-Stieljes de $F$ est\'a dada por

\begin{eqnarray*}
\hat{F}\left(\alpha\right)=\int_{\rea_{+}}e^{-\alpha t}dF\left(t\right)\textrm{,  }\alpha\geq0.
\end{eqnarray*}
\end{Def}

Entonces

\begin{eqnarray*}
\hat{U}\left(\alpha\right)=\sum_{n=0}^{\infty}\hat{F^{n\star}}\left(\alpha\right)=\sum_{n=0}^{\infty}\hat{F}\left(\alpha\right)^{n}=\frac{1}{1-\hat{F}\left(\alpha\right)}.
\end{eqnarray*}


\begin{Prop}
La Transformada de Laplace $\hat{U}\left(\alpha\right)$ y $\hat{F}\left(\alpha\right)$ determina una a la otra de manera \'unica por la relaci\'on $\hat{U}\left(\alpha\right)=\frac{1}{1-\hat{F}\left(\alpha\right)}$.
\end{Prop}


\begin{Note}
Un proceso de renovaci\'on $N\left(t\right)$ cuyos tiempos de inter-renovaci\'on tienen media finita, es un proceso Poisson con tasa $\lambda$ si y s\'olo s\'i $\esp\left[U\left(t\right)\right]=\lambda t$, para $t\geq0$.
\end{Note}


\begin{Teo}
Sea $N\left(t\right)$ un proceso puntual simple con puntos de localizaci\'on $T_{n}$ tal que $\eta\left(t\right)=\esp\left[N\left(\right)\right]$ es finita para cada $t$. Entonces para cualquier funci\'on $f:\rea_{+}\rightarrow\rea$,
\begin{eqnarray*}
\esp\left[\sum_{n=1}^{N\left(\right)}f\left(T_{n}\right)\right]=\int_{\left(0,t\right]}f\left(s\right)d\eta\left(s\right)\textrm{,  }t\geq0,
\end{eqnarray*}
suponiendo que la integral exista. Adem\'as si $X_{1},X_{2},\ldots$ son variables aleatorias definidas en el mismo espacio de probabilidad que el proceso $N\left(t\right)$ tal que $\esp\left[X_{n}|T_{n}=s\right]=f\left(s\right)$, independiente de $n$. Entonces
\begin{eqnarray*}
\esp\left[\sum_{n=1}^{N\left(t\right)}X_{n}\right]=\int_{\left(0,t\right]}f\left(s\right)d\eta\left(s\right)\textrm{,  }t\geq0,
\end{eqnarray*} 
suponiendo que la integral exista. 
\end{Teo}

\begin{Coro}[Identidad de Wald para Renovaciones]
Para el proceso de renovaci\'on $N\left(t\right)$,
\begin{eqnarray*}
\esp\left[T_{N\left(t\right)+1}\right]=\mu\esp\left[N\left(t\right)+1\right]\textrm{,  }t\geq0,
\end{eqnarray*}  
\end{Coro}


\begin{Def}
Sea $h\left(t\right)$ funci\'on de valores reales en $\rea$ acotada en intervalos finitos e igual a cero para $t<0$ La ecuaci\'on de renovaci\'on para $h\left(t\right)$ y la distribuci\'on $F$ es

\begin{eqnarray}\label{Ec.Renovacion}
H\left(t\right)=h\left(t\right)+\int_{\left[0,t\right]}H\left(t-s\right)dF\left(s\right)\textrm{,    }t\geq0,
\end{eqnarray}
donde $H\left(t\right)$ es una funci\'on de valores reales. Esto es $H=h+F\star H$. Decimos que $H\left(t\right)$ es soluci\'on de esta ecuaci\'on si satisface la ecuaci\'on, y es acotada en intervalos finitos e iguales a cero para $t<0$.
\end{Def}

\begin{Prop}
La funci\'on $U\star h\left(t\right)$ es la \'unica soluci\'on de la ecuaci\'on de renovaci\'on (\ref{Ec.Renovacion}).
\end{Prop}

\begin{Teo}[Teorema Renovaci\'on Elemental]
\begin{eqnarray*}
t^{-1}U\left(t\right)\rightarrow 1/\mu\textrm{,    cuando }t\rightarrow\infty.
\end{eqnarray*}
\end{Teo}



Sup\'ongase que $N\left(t\right)$ es un proceso de renovaci\'on con distribuci\'on $F$ con media finita $\mu$.

\begin{Def}
La funci\'on de renovaci\'on asociada con la distribuci\'on $F$, del proceso $N\left(t\right)$, es
\begin{eqnarray*}
U\left(t\right)=\sum_{n=1}^{\infty}F^{n\star}\left(t\right),\textrm{   }t\geq0,
\end{eqnarray*}
donde $F^{0\star}\left(t\right)=\indora\left(t\geq0\right)$.
\end{Def}


\begin{Prop}
Sup\'ongase que la distribuci\'on de inter-renovaci\'on $F$ tiene densidad $f$. Entonces $U\left(t\right)$ tambi\'en tiene densidad, para $t>0$, y es $U^{'}\left(t\right)=\sum_{n=0}^{\infty}f^{n\star}\left(t\right)$. Adem\'as
\begin{eqnarray*}
\prob\left\{N\left(t\right)>N\left(t-\right)\right\}=0\textrm{,   }t\geq0.
\end{eqnarray*}
\end{Prop}

\begin{Def}
La Transformada de Laplace-Stieljes de $F$ est\'a dada por

\begin{eqnarray*}
\hat{F}\left(\alpha\right)=\int_{\rea_{+}}e^{-\alpha t}dF\left(t\right)\textrm{,  }\alpha\geq0.
\end{eqnarray*}
\end{Def}

Entonces

\begin{eqnarray*}
\hat{U}\left(\alpha\right)=\sum_{n=0}^{\infty}\hat{F^{n\star}}\left(\alpha\right)=\sum_{n=0}^{\infty}\hat{F}\left(\alpha\right)^{n}=\frac{1}{1-\hat{F}\left(\alpha\right)}.
\end{eqnarray*}


\begin{Prop}
La Transformada de Laplace $\hat{U}\left(\alpha\right)$ y $\hat{F}\left(\alpha\right)$ determina una a la otra de manera \'unica por la relaci\'on $\hat{U}\left(\alpha\right)=\frac{1}{1-\hat{F}\left(\alpha\right)}$.
\end{Prop}


\begin{Note}
Un proceso de renovaci\'on $N\left(t\right)$ cuyos tiempos de inter-renovaci\'on tienen media finita, es un proceso Poisson con tasa $\lambda$ si y s\'olo s\'i $\esp\left[U\left(t\right)\right]=\lambda t$, para $t\geq0$.
\end{Note}


\begin{Teo}
Sea $N\left(t\right)$ un proceso puntual simple con puntos de localizaci\'on $T_{n}$ tal que $\eta\left(t\right)=\esp\left[N\left(\right)\right]$ es finita para cada $t$. Entonces para cualquier funci\'on $f:\rea_{+}\rightarrow\rea$,
\begin{eqnarray*}
\esp\left[\sum_{n=1}^{N\left(\right)}f\left(T_{n}\right)\right]=\int_{\left(0,t\right]}f\left(s\right)d\eta\left(s\right)\textrm{,  }t\geq0,
\end{eqnarray*}
suponiendo que la integral exista. Adem\'as si $X_{1},X_{2},\ldots$ son variables aleatorias definidas en el mismo espacio de probabilidad que el proceso $N\left(t\right)$ tal que $\esp\left[X_{n}|T_{n}=s\right]=f\left(s\right)$, independiente de $n$. Entonces
\begin{eqnarray*}
\esp\left[\sum_{n=1}^{N\left(t\right)}X_{n}\right]=\int_{\left(0,t\right]}f\left(s\right)d\eta\left(s\right)\textrm{,  }t\geq0,
\end{eqnarray*} 
suponiendo que la integral exista. 
\end{Teo}

\begin{Coro}[Identidad de Wald para Renovaciones]
Para el proceso de renovaci\'on $N\left(t\right)$,
\begin{eqnarray*}
\esp\left[T_{N\left(t\right)+1}\right]=\mu\esp\left[N\left(t\right)+1\right]\textrm{,  }t\geq0,
\end{eqnarray*}  
\end{Coro}


\begin{Def}
Sea $h\left(t\right)$ funci\'on de valores reales en $\rea$ acotada en intervalos finitos e igual a cero para $t<0$ La ecuaci\'on de renovaci\'on para $h\left(t\right)$ y la distribuci\'on $F$ es

\begin{eqnarray}\label{Ec.Renovacion}
H\left(t\right)=h\left(t\right)+\int_{\left[0,t\right]}H\left(t-s\right)dF\left(s\right)\textrm{,    }t\geq0,
\end{eqnarray}
donde $H\left(t\right)$ es una funci\'on de valores reales. Esto es $H=h+F\star H$. Decimos que $H\left(t\right)$ es soluci\'on de esta ecuaci\'on si satisface la ecuaci\'on, y es acotada en intervalos finitos e iguales a cero para $t<0$.
\end{Def}

\begin{Prop}
La funci\'on $U\star h\left(t\right)$ es la \'unica soluci\'on de la ecuaci\'on de renovaci\'on (\ref{Ec.Renovacion}).
\end{Prop}

\begin{Teo}[Teorema Renovaci\'on Elemental]
\begin{eqnarray*}
t^{-1}U\left(t\right)\rightarrow 1/\mu\textrm{,    cuando }t\rightarrow\infty.
\end{eqnarray*}
\end{Teo}


\begin{Note} Una funci\'on $h:\rea_{+}\rightarrow\rea$ es Directamente Riemann Integrable en los siguientes casos:
\begin{itemize}
\item[a)] $h\left(t\right)\geq0$ es decreciente y Riemann Integrable.
\item[b)] $h$ es continua excepto posiblemente en un conjunto de Lebesgue de medida 0, y $|h\left(t\right)|\leq b\left(t\right)$, donde $b$ es DRI.
\end{itemize}
\end{Note}

\begin{Teo}[Teorema Principal de Renovaci\'on]
Si $F$ es no aritm\'etica y $h\left(t\right)$ es Directamente Riemann Integrable (DRI), entonces

\begin{eqnarray*}
lim_{t\rightarrow\infty}U\star h=\frac{1}{\mu}\int_{\rea_{+}}h\left(s\right)ds.
\end{eqnarray*}
\end{Teo}

\begin{Prop}
Cualquier funci\'on $H\left(t\right)$ acotada en intervalos finitos y que es 0 para $t<0$ puede expresarse como
\begin{eqnarray*}
H\left(t\right)=U\star h\left(t\right)\textrm{,  donde }h\left(t\right)=H\left(t\right)-F\star H\left(t\right)
\end{eqnarray*}
\end{Prop}

\begin{Def}
Un proceso estoc\'astico $X\left(t\right)$ es crudamente regenerativo en un tiempo aleatorio positivo $T$ si
\begin{eqnarray*}
\esp\left[X\left(T+t\right)|T\right]=\esp\left[X\left(t\right)\right]\textrm{, para }t\geq0,\end{eqnarray*}
y con las esperanzas anteriores finitas.
\end{Def}

\begin{Prop}
Sup\'ongase que $X\left(t\right)$ es un proceso crudamente regenerativo en $T$, que tiene distribuci\'on $F$. Si $\esp\left[X\left(t\right)\right]$ es acotado en intervalos finitos, entonces
\begin{eqnarray*}
\esp\left[X\left(t\right)\right]=U\star h\left(t\right)\textrm{,  donde }h\left(t\right)=\esp\left[X\left(t\right)\indora\left(T>t\right)\right].
\end{eqnarray*}
\end{Prop}

\begin{Teo}[Regeneraci\'on Cruda]
Sup\'ongase que $X\left(t\right)$ es un proceso con valores positivo crudamente regenerativo en $T$, y def\'inase $M=\sup\left\{|X\left(t\right)|:t\leq T\right\}$. Si $T$ es no aritm\'etico y $M$ y $MT$ tienen media finita, entonces
\begin{eqnarray*}
lim_{t\rightarrow\infty}\esp\left[X\left(t\right)\right]=\frac{1}{\mu}\int_{\rea_{+}}h\left(s\right)ds,
\end{eqnarray*}
donde $h\left(t\right)=\esp\left[X\left(t\right)\indora\left(T>t\right)\right]$.
\end{Teo}


\begin{Note} Una funci\'on $h:\rea_{+}\rightarrow\rea$ es Directamente Riemann Integrable en los siguientes casos:
\begin{itemize}
\item[a)] $h\left(t\right)\geq0$ es decreciente y Riemann Integrable.
\item[b)] $h$ es continua excepto posiblemente en un conjunto de Lebesgue de medida 0, y $|h\left(t\right)|\leq b\left(t\right)$, donde $b$ es DRI.
\end{itemize}
\end{Note}

\begin{Teo}[Teorema Principal de Renovaci\'on]
Si $F$ es no aritm\'etica y $h\left(t\right)$ es Directamente Riemann Integrable (DRI), entonces

\begin{eqnarray*}
lim_{t\rightarrow\infty}U\star h=\frac{1}{\mu}\int_{\rea_{+}}h\left(s\right)ds.
\end{eqnarray*}
\end{Teo}

\begin{Prop}
Cualquier funci\'on $H\left(t\right)$ acotada en intervalos finitos y que es 0 para $t<0$ puede expresarse como
\begin{eqnarray*}
H\left(t\right)=U\star h\left(t\right)\textrm{,  donde }h\left(t\right)=H\left(t\right)-F\star H\left(t\right)
\end{eqnarray*}
\end{Prop}

\begin{Def}
Un proceso estoc\'astico $X\left(t\right)$ es crudamente regenerativo en un tiempo aleatorio positivo $T$ si
\begin{eqnarray*}
\esp\left[X\left(T+t\right)|T\right]=\esp\left[X\left(t\right)\right]\textrm{, para }t\geq0,\end{eqnarray*}
y con las esperanzas anteriores finitas.
\end{Def}

\begin{Prop}
Sup\'ongase que $X\left(t\right)$ es un proceso crudamente regenerativo en $T$, que tiene distribuci\'on $F$. Si $\esp\left[X\left(t\right)\right]$ es acotado en intervalos finitos, entonces
\begin{eqnarray*}
\esp\left[X\left(t\right)\right]=U\star h\left(t\right)\textrm{,  donde }h\left(t\right)=\esp\left[X\left(t\right)\indora\left(T>t\right)\right].
\end{eqnarray*}
\end{Prop}

\begin{Teo}[Regeneraci\'on Cruda]
Sup\'ongase que $X\left(t\right)$ es un proceso con valores positivo crudamente regenerativo en $T$, y def\'inase $M=\sup\left\{|X\left(t\right)|:t\leq T\right\}$. Si $T$ es no aritm\'etico y $M$ y $MT$ tienen media finita, entonces
\begin{eqnarray*}
lim_{t\rightarrow\infty}\esp\left[X\left(t\right)\right]=\frac{1}{\mu}\int_{\rea_{+}}h\left(s\right)ds,
\end{eqnarray*}
donde $h\left(t\right)=\esp\left[X\left(t\right)\indora\left(T>t\right)\right]$.
\end{Teo}

\begin{Def}
Para el proceso $\left\{\left(N\left(t\right),X\left(t\right)\right):t\geq0\right\}$, sus trayectoria muestrales en el intervalo de tiempo $\left[T_{n-1},T_{n}\right)$ est\'an descritas por
\begin{eqnarray*}
\zeta_{n}=\left(\xi_{n},\left\{X\left(T_{n-1}+t\right):0\leq t<\xi_{n}\right\}\right)
\end{eqnarray*}
Este $\zeta_{n}$ es el $n$-\'esimo segmento del proceso. El proceso es regenerativo sobre los tiempos $T_{n}$ si sus segmentos $\zeta_{n}$ son independientes e id\'enticamennte distribuidos.
\end{Def}


\begin{Note}
Si $\tilde{X}\left(t\right)$ con espacio de estados $\tilde{S}$ es regenerativo sobre $T_{n}$, entonces $X\left(t\right)=f\left(\tilde{X}\left(t\right)\right)$ tambi\'en es regenerativo sobre $T_{n}$, para cualquier funci\'on $f:\tilde{S}\rightarrow S$.
\end{Note}

\begin{Note}
Los procesos regenerativos son crudamente regenerativos, pero no al rev\'es.
\end{Note}


\begin{Note}
Un proceso estoc\'astico a tiempo continuo o discreto es regenerativo si existe un proceso de renovaci\'on  tal que los segmentos del proceso entre tiempos de renovaci\'on sucesivos son i.i.d., es decir, para $\left\{X\left(t\right):t\geq0\right\}$ proceso estoc\'astico a tiempo continuo con espacio de estados $S$, espacio m\'etrico.
\end{Note}

Para $\left\{X\left(t\right):t\geq0\right\}$ Proceso Estoc\'astico a tiempo continuo con estado de espacios $S$, que es un espacio m\'etrico, con trayectorias continuas por la derecha y con l\'imites por la izquierda c.s. Sea $N\left(t\right)$ un proceso de renovaci\'on en $\rea_{+}$ definido en el mismo espacio de probabilidad que $X\left(t\right)$, con tiempos de renovaci\'on $T$ y tiempos de inter-renovaci\'on $\xi_{n}=T_{n}-T_{n-1}$, con misma distribuci\'on $F$ de media finita $\mu$.



\begin{Def}
Para el proceso $\left\{\left(N\left(t\right),X\left(t\right)\right):t\geq0\right\}$, sus trayectoria muestrales en el intervalo de tiempo $\left[T_{n-1},T_{n}\right)$ est\'an descritas por
\begin{eqnarray*}
\zeta_{n}=\left(\xi_{n},\left\{X\left(T_{n-1}+t\right):0\leq t<\xi_{n}\right\}\right)
\end{eqnarray*}
Este $\zeta_{n}$ es el $n$-\'esimo segmento del proceso. El proceso es regenerativo sobre los tiempos $T_{n}$ si sus segmentos $\zeta_{n}$ son independientes e id\'enticamennte distribuidos.
\end{Def}

\begin{Note}
Un proceso regenerativo con media de la longitud de ciclo finita es llamado positivo recurrente.
\end{Note}

\begin{Teo}[Procesos Regenerativos]
Suponga que el proceso
\end{Teo}


\begin{Def}[Renewal Process Trinity]
Para un proceso de renovaci\'on $N\left(t\right)$, los siguientes procesos proveen de informaci\'on sobre los tiempos de renovaci\'on.
\begin{itemize}
\item $A\left(t\right)=t-T_{N\left(t\right)}$, el tiempo de recurrencia hacia atr\'as al tiempo $t$, que es el tiempo desde la \'ultima renovaci\'on para $t$.

\item $B\left(t\right)=T_{N\left(t\right)+1}-t$, el tiempo de recurrencia hacia adelante al tiempo $t$, residual del tiempo de renovaci\'on, que es el tiempo para la pr\'oxima renovaci\'on despu\'es de $t$.

\item $L\left(t\right)=\xi_{N\left(t\right)+1}=A\left(t\right)+B\left(t\right)$, la longitud del intervalo de renovaci\'on que contiene a $t$.
\end{itemize}
\end{Def}

\begin{Note}
El proceso tridimensional $\left(A\left(t\right),B\left(t\right),L\left(t\right)\right)$ es regenerativo sobre $T_{n}$, y por ende cada proceso lo es. Cada proceso $A\left(t\right)$ y $B\left(t\right)$ son procesos de MArkov a tiempo continuo con trayectorias continuas por partes en el espacio de estados $\rea_{+}$. Una expresi\'on conveniente para su distribuci\'on conjunta es, para $0\leq x<t,y\geq0$
\begin{equation}\label{NoRenovacion}
P\left\{A\left(t\right)>x,B\left(t\right)>y\right\}=
P\left\{N\left(t+y\right)-N\left((t-x)\right)=0\right\}
\end{equation}
\end{Note}

\begin{Ejem}[Tiempos de recurrencia Poisson]
Si $N\left(t\right)$ es un proceso Poisson con tasa $\lambda$, entonces de la expresi\'on (\ref{NoRenovacion}) se tiene que

\begin{eqnarray*}
\begin{array}{lc}
P\left\{A\left(t\right)>x,B\left(t\right)>y\right\}=e^{-\lambda\left(x+y\right)},&0\leq x<t,y\geq0,
\end{array}
\end{eqnarray*}
que es la probabilidad Poisson de no renovaciones en un intervalo de longitud $x+y$.

\end{Ejem}

\begin{Note}
Una cadena de Markov erg\'odica tiene la propiedad de ser estacionaria si la distribuci\'on de su estado al tiempo $0$ es su distribuci\'on estacionaria.
\end{Note}


\begin{Def}
Un proceso estoc\'astico a tiempo continuo $\left\{X\left(t\right):t\geq0\right\}$ en un espacio general es estacionario si sus distribuciones finito dimensionales son invariantes bajo cualquier  traslado: para cada $0\leq s_{1}<s_{2}<\cdots<s_{k}$ y $t\geq0$,
\begin{eqnarray*}
\left(X\left(s_{1}+t\right),\ldots,X\left(s_{k}+t\right)\right)=_{d}\left(X\left(s_{1}\right),\ldots,X\left(s_{k}\right)\right).
\end{eqnarray*}
\end{Def}

\begin{Note}
Un proceso de Markov es estacionario si $X\left(t\right)=_{d}X\left(0\right)$, $t\geq0$.
\end{Note}

Considerese el proceso $N\left(t\right)=\sum_{n}\indora\left(\tau_{n}\leq t\right)$ en $\rea_{+}$, con puntos $0<\tau_{1}<\tau_{2}<\cdots$.

\begin{Prop}
Si $N$ es un proceso puntual estacionario y $\esp\left[N\left(1\right)\right]<\infty$, entonces $\esp\left[N\left(t\right)\right]=t\esp\left[N\left(1\right)\right]$, $t\geq0$

\end{Prop}

\begin{Teo}
Los siguientes enunciados son equivalentes
\begin{itemize}
\item[i)] El proceso retardado de renovaci\'on $N$ es estacionario.

\item[ii)] EL proceso de tiempos de recurrencia hacia adelante $B\left(t\right)$ es estacionario.


\item[iii)] $\esp\left[N\left(t\right)\right]=t/\mu$,


\item[iv)] $G\left(t\right)=F_{e}\left(t\right)=\frac{1}{\mu}\int_{0}^{t}\left[1-F\left(s\right)\right]ds$
\end{itemize}
Cuando estos enunciados son ciertos, $P\left\{B\left(t\right)\leq x\right\}=F_{e}\left(x\right)$, para $t,x\geq0$.

\end{Teo}

\begin{Note}
Una consecuencia del teorema anterior es que el Proceso Poisson es el \'unico proceso sin retardo que es estacionario.
\end{Note}

\begin{Coro}
El proceso de renovaci\'on $N\left(t\right)$ sin retardo, y cuyos tiempos de inter renonaci\'on tienen media finita, es estacionario si y s\'olo si es un proceso Poisson.

\end{Coro}

%______________________________________________________________________

%\section{Ejemplos, Notas importantes}
%______________________________________________________________________
%\section*{Ap\'endice A}
%__________________________________________________________________

%________________________________________________________________________
%\subsection*{Procesos Regenerativos}
%________________________________________________________________________



\begin{Note}
Si $\tilde{X}\left(t\right)$ con espacio de estados $\tilde{S}$ es regenerativo sobre $T_{n}$, entonces $X\left(t\right)=f\left(\tilde{X}\left(t\right)\right)$ tambi\'en es regenerativo sobre $T_{n}$, para cualquier funci\'on $f:\tilde{S}\rightarrow S$.
\end{Note}

\begin{Note}
Los procesos regenerativos son crudamente regenerativos, pero no al rev\'es.
\end{Note}
%\subsection*{Procesos Regenerativos: Sigman\cite{Sigman1}}
\begin{Def}[Definici\'on Cl\'asica]
Un proceso estoc\'astico $X=\left\{X\left(t\right):t\geq0\right\}$ es llamado regenerativo is existe una variable aleatoria $R_{1}>0$ tal que
\begin{itemize}
\item[i)] $\left\{X\left(t+R_{1}\right):t\geq0\right\}$ es independiente de $\left\{\left\{X\left(t\right):t<R_{1}\right\},\right\}$
\item[ii)] $\left\{X\left(t+R_{1}\right):t\geq0\right\}$ es estoc\'asticamente equivalente a $\left\{X\left(t\right):t>0\right\}$
\end{itemize}

Llamamos a $R_{1}$ tiempo de regeneraci\'on, y decimos que $X$ se regenera en este punto.
\end{Def}

$\left\{X\left(t+R_{1}\right)\right\}$ es regenerativo con tiempo de regeneraci\'on $R_{2}$, independiente de $R_{1}$ pero con la misma distribuci\'on que $R_{1}$. Procediendo de esta manera se obtiene una secuencia de variables aleatorias independientes e id\'enticamente distribuidas $\left\{R_{n}\right\}$ llamados longitudes de ciclo. Si definimos a $Z_{k}\equiv R_{1}+R_{2}+\cdots+R_{k}$, se tiene un proceso de renovaci\'on llamado proceso de renovaci\'on encajado para $X$.




\begin{Def}
Para $x$ fijo y para cada $t\geq0$, sea $I_{x}\left(t\right)=1$ si $X\left(t\right)\leq x$,  $I_{x}\left(t\right)=0$ en caso contrario, y def\'inanse los tiempos promedio
\begin{eqnarray*}
\overline{X}&=&lim_{t\rightarrow\infty}\frac{1}{t}\int_{0}^{\infty}X\left(u\right)du\\
\prob\left(X_{\infty}\leq x\right)&=&lim_{t\rightarrow\infty}\frac{1}{t}\int_{0}^{\infty}I_{x}\left(u\right)du,
\end{eqnarray*}
cuando estos l\'imites existan.
\end{Def}

Como consecuencia del teorema de Renovaci\'on-Recompensa, se tiene que el primer l\'imite  existe y es igual a la constante
\begin{eqnarray*}
\overline{X}&=&\frac{\esp\left[\int_{0}^{R_{1}}X\left(t\right)dt\right]}{\esp\left[R_{1}\right]},
\end{eqnarray*}
suponiendo que ambas esperanzas son finitas.

\begin{Note}
\begin{itemize}
\item[a)] Si el proceso regenerativo $X$ es positivo recurrente y tiene trayectorias muestrales no negativas, entonces la ecuaci\'on anterior es v\'alida.
\item[b)] Si $X$ es positivo recurrente regenerativo, podemos construir una \'unica versi\'on estacionaria de este proceso, $X_{e}=\left\{X_{e}\left(t\right)\right\}$, donde $X_{e}$ es un proceso estoc\'astico regenerativo y estrictamente estacionario, con distribuci\'on marginal distribuida como $X_{\infty}$
\end{itemize}
\end{Note}

Para $\left\{X\left(t\right):t\geq0\right\}$ Proceso Estoc\'astico a tiempo continuo con estado de espacios $S$, que es un espacio m\'etrico, con trayectorias continuas por la derecha y con l\'imites por la izquierda c.s. Sea $N\left(t\right)$ un proceso de renovaci\'on en $\rea_{+}$ definido en el mismo espacio de probabilidad que $X\left(t\right)$, con tiempos de renovaci\'on $T$ y tiempos de inter-renovaci\'on $\xi_{n}=T_{n}-T_{n-1}$, con misma distribuci\'on $F$ de media finita $\mu$.


\begin{Def}
Para el proceso $\left\{\left(N\left(t\right),X\left(t\right)\right):t\geq0\right\}$, sus trayectoria muestrales en el intervalo de tiempo $\left[T_{n-1},T_{n}\right)$ est\'an descritas por
\begin{eqnarray*}
\zeta_{n}=\left(\xi_{n},\left\{X\left(T_{n-1}+t\right):0\leq t<\xi_{n}\right\}\right)
\end{eqnarray*}
Este $\zeta_{n}$ es el $n$-\'esimo segmento del proceso. El proceso es regenerativo sobre los tiempos $T_{n}$ si sus segmentos $\zeta_{n}$ son independientes e id\'enticamennte distribuidos.
\end{Def}


\begin{Note}
Si $\tilde{X}\left(t\right)$ con espacio de estados $\tilde{S}$ es regenerativo sobre $T_{n}$, entonces $X\left(t\right)=f\left(\tilde{X}\left(t\right)\right)$ tambi\'en es regenerativo sobre $T_{n}$, para cualquier funci\'on $f:\tilde{S}\rightarrow S$.
\end{Note}

\begin{Note}
Los procesos regenerativos son crudamente regenerativos, pero no al rev\'es.
\end{Note}

\begin{Def}[Definici\'on Cl\'asica]
Un proceso estoc\'astico $X=\left\{X\left(t\right):t\geq0\right\}$ es llamado regenerativo is existe una variable aleatoria $R_{1}>0$ tal que
\begin{itemize}
\item[i)] $\left\{X\left(t+R_{1}\right):t\geq0\right\}$ es independiente de $\left\{\left\{X\left(t\right):t<R_{1}\right\},\right\}$
\item[ii)] $\left\{X\left(t+R_{1}\right):t\geq0\right\}$ es estoc\'asticamente equivalente a $\left\{X\left(t\right):t>0\right\}$
\end{itemize}

Llamamos a $R_{1}$ tiempo de regeneraci\'on, y decimos que $X$ se regenera en este punto.
\end{Def}

$\left\{X\left(t+R_{1}\right)\right\}$ es regenerativo con tiempo de regeneraci\'on $R_{2}$, independiente de $R_{1}$ pero con la misma distribuci\'on que $R_{1}$. Procediendo de esta manera se obtiene una secuencia de variables aleatorias independientes e id\'enticamente distribuidas $\left\{R_{n}\right\}$ llamados longitudes de ciclo. Si definimos a $Z_{k}\equiv R_{1}+R_{2}+\cdots+R_{k}$, se tiene un proceso de renovaci\'on llamado proceso de renovaci\'on encajado para $X$.

\begin{Note}
Un proceso regenerativo con media de la longitud de ciclo finita es llamado positivo recurrente.
\end{Note}


\begin{Def}
Para $x$ fijo y para cada $t\geq0$, sea $I_{x}\left(t\right)=1$ si $X\left(t\right)\leq x$,  $I_{x}\left(t\right)=0$ en caso contrario, y def\'inanse los tiempos promedio
\begin{eqnarray*}
\overline{X}&=&lim_{t\rightarrow\infty}\frac{1}{t}\int_{0}^{\infty}X\left(u\right)du\\
\prob\left(X_{\infty}\leq x\right)&=&lim_{t\rightarrow\infty}\frac{1}{t}\int_{0}^{\infty}I_{x}\left(u\right)du,
\end{eqnarray*}
cuando estos l\'imites existan.
\end{Def}

Como consecuencia del teorema de Renovaci\'on-Recompensa, se tiene que el primer l\'imite  existe y es igual a la constante
\begin{eqnarray*}
\overline{X}&=&\frac{\esp\left[\int_{0}^{R_{1}}X\left(t\right)dt\right]}{\esp\left[R_{1}\right]},
\end{eqnarray*}
suponiendo que ambas esperanzas son finitas.

\begin{Note}
\begin{itemize}
\item[a)] Si el proceso regenerativo $X$ es positivo recurrente y tiene trayectorias muestrales no negativas, entonces la ecuaci\'on anterior es v\'alida.
\item[b)] Si $X$ es positivo recurrente regenerativo, podemos construir una \'unica versi\'on estacionaria de este proceso, $X_{e}=\left\{X_{e}\left(t\right)\right\}$, donde $X_{e}$ es un proceso estoc\'astico regenerativo y estrictamente estacionario, con distribuci\'on marginal distribuida como $X_{\infty}$
\end{itemize}
\end{Note}

%__________________________________________________________________________________________
%\subsection{Procesos Regenerativos Estacionarios - Stidham \cite{Stidham}}
%__________________________________________________________________________________________


Un proceso estoc\'astico a tiempo continuo $\left\{V\left(t\right),t\geq0\right\}$ es un proceso regenerativo si existe una sucesi\'on de variables aleatorias independientes e id\'enticamente distribuidas $\left\{X_{1},X_{2},\ldots\right\}$, sucesi\'on de renovaci\'on, tal que para cualquier conjunto de Borel $A$, 

\begin{eqnarray*}
\prob\left\{V\left(t\right)\in A|X_{1}+X_{2}+\cdots+X_{R\left(t\right)}=s,\left\{V\left(\tau\right),\tau<s\right\}\right\}=\prob\left\{V\left(t-s\right)\in A|X_{1}>t-s\right\},
\end{eqnarray*}
para todo $0\leq s\leq t$, donde $R\left(t\right)=\max\left\{X_{1}+X_{2}+\cdots+X_{j}\leq t\right\}=$n\'umero de renovaciones ({\emph{puntos de regeneraci\'on}}) que ocurren en $\left[0,t\right]$. El intervalo $\left[0,X_{1}\right)$ es llamado {\emph{primer ciclo de regeneraci\'on}} de $\left\{V\left(t \right),t\geq0\right\}$, $\left[X_{1},X_{1}+X_{2}\right)$ el {\emph{segundo ciclo de regeneraci\'on}}, y as\'i sucesivamente.

Sea $X=X_{1}$ y sea $F$ la funci\'on de distrbuci\'on de $X$


\begin{Def}
Se define el proceso estacionario, $\left\{V^{*}\left(t\right),t\geq0\right\}$, para $\left\{V\left(t\right),t\geq0\right\}$ por

\begin{eqnarray*}
\prob\left\{V\left(t\right)\in A\right\}=\frac{1}{\esp\left[X\right]}\int_{0}^{\infty}\prob\left\{V\left(t+x\right)\in A|X>x\right\}\left(1-F\left(x\right)\right)dx,
\end{eqnarray*} 
para todo $t\geq0$ y todo conjunto de Borel $A$.
\end{Def}

\begin{Def}
Una distribuci\'on se dice que es {\emph{aritm\'etica}} si todos sus puntos de incremento son m\'ultiplos de la forma $0,\lambda, 2\lambda,\ldots$ para alguna $\lambda>0$ entera.
\end{Def}


\begin{Def}
Una modificaci\'on medible de un proceso $\left\{V\left(t\right),t\geq0\right\}$, es una versi\'on de este, $\left\{V\left(t,w\right)\right\}$ conjuntamente medible para $t\geq0$ y para $w\in S$, $S$ espacio de estados para $\left\{V\left(t\right),t\geq0\right\}$.
\end{Def}

\begin{Teo}
Sea $\left\{V\left(t\right),t\geq\right\}$ un proceso regenerativo no negativo con modificaci\'on medible. Sea $\esp\left[X\right]<\infty$. Entonces el proceso estacionario dado por la ecuaci\'on anterior est\'a bien definido y tiene funci\'on de distribuci\'on independiente de $t$, adem\'as
\begin{itemize}
\item[i)] \begin{eqnarray*}
\esp\left[V^{*}\left(0\right)\right]&=&\frac{\esp\left[\int_{0}^{X}V\left(s\right)ds\right]}{\esp\left[X\right]}\end{eqnarray*}
\item[ii)] Si $\esp\left[V^{*}\left(0\right)\right]<\infty$, equivalentemente, si $\esp\left[\int_{0}^{X}V\left(s\right)ds\right]<\infty$,entonces
\begin{eqnarray*}
\frac{\int_{0}^{t}V\left(s\right)ds}{t}\rightarrow\frac{\esp\left[\int_{0}^{X}V\left(s\right)ds\right]}{\esp\left[X\right]}
\end{eqnarray*}
con probabilidad 1 y en media, cuando $t\rightarrow\infty$.
\end{itemize}
\end{Teo}
%________________________________________________________________________
\section{Procesos Regenerativos Sigman, Thorisson y Wolff \cite{Sigman1}}
%________________________________________________________________________


\begin{Def}[Definici\'on Cl\'asica]
Un proceso estoc\'astico $X=\left\{X\left(t\right):t\geq0\right\}$ es llamado regenerativo is existe una variable aleatoria $R_{1}>0$ tal que
\begin{itemize}
\item[i)] $\left\{X\left(t+R_{1}\right):t\geq0\right\}$ es independiente de $\left\{\left\{X\left(t\right):t<R_{1}\right\},\right\}$
\item[ii)] $\left\{X\left(t+R_{1}\right):t\geq0\right\}$ es estoc\'asticamente equivalente a $\left\{X\left(t\right):t>0\right\}$
\end{itemize}

Llamamos a $R_{1}$ tiempo de regeneraci\'on, y decimos que $X$ se regenera en este punto.
\end{Def}

$\left\{X\left(t+R_{1}\right)\right\}$ es regenerativo con tiempo de regeneraci\'on $R_{2}$, independiente de $R_{1}$ pero con la misma distribuci\'on que $R_{1}$. Procediendo de esta manera se obtiene una secuencia de variables aleatorias independientes e id\'enticamente distribuidas $\left\{R_{n}\right\}$ llamados longitudes de ciclo. Si definimos a $Z_{k}\equiv R_{1}+R_{2}+\cdots+R_{k}$, se tiene un proceso de renovaci\'on llamado proceso de renovaci\'on encajado para $X$.


\begin{Note}
La existencia de un primer tiempo de regeneraci\'on, $R_{1}$, implica la existencia de una sucesi\'on completa de estos tiempos $R_{1},R_{2}\ldots,$ que satisfacen la propiedad deseada \cite{Sigman2}.
\end{Note}


\begin{Note} Para la cola $GI/GI/1$ los usuarios arriban con tiempos $t_{n}$ y son atendidos con tiempos de servicio $S_{n}$, los tiempos de arribo forman un proceso de renovaci\'on  con tiempos entre arribos independientes e identicamente distribuidos (\texttt{i.i.d.})$T_{n}=t_{n}-t_{n-1}$, adem\'as los tiempos de servicio son \texttt{i.i.d.} e independientes de los procesos de arribo. Por \textit{estable} se entiende que $\esp S_{n}<\esp T_{n}<\infty$.
\end{Note}
 


\begin{Def}
Para $x$ fijo y para cada $t\geq0$, sea $I_{x}\left(t\right)=1$ si $X\left(t\right)\leq x$,  $I_{x}\left(t\right)=0$ en caso contrario, y def\'inanse los tiempos promedio
\begin{eqnarray*}
\overline{X}&=&lim_{t\rightarrow\infty}\frac{1}{t}\int_{0}^{\infty}X\left(u\right)du\\
\prob\left(X_{\infty}\leq x\right)&=&lim_{t\rightarrow\infty}\frac{1}{t}\int_{0}^{\infty}I_{x}\left(u\right)du,
\end{eqnarray*}
cuando estos l\'imites existan.
\end{Def}

Como consecuencia del teorema de Renovaci\'on-Recompensa, se tiene que el primer l\'imite  existe y es igual a la constante
\begin{eqnarray*}
\overline{X}&=&\frac{\esp\left[\int_{0}^{R_{1}}X\left(t\right)dt\right]}{\esp\left[R_{1}\right]},
\end{eqnarray*}
suponiendo que ambas esperanzas son finitas.
 
\begin{Note}
Funciones de procesos regenerativos son regenerativas, es decir, si $X\left(t\right)$ es regenerativo y se define el proceso $Y\left(t\right)$ por $Y\left(t\right)=f\left(X\left(t\right)\right)$ para alguna funci\'on Borel medible $f\left(\cdot\right)$. Adem\'as $Y$ es regenerativo con los mismos tiempos de renovaci\'on que $X$. 

En general, los tiempos de renovaci\'on, $Z_{k}$ de un proceso regenerativo no requieren ser tiempos de paro con respecto a la evoluci\'on de $X\left(t\right)$.
\end{Note} 

\begin{Note}
Una funci\'on de un proceso de Markov, usualmente no ser\'a un proceso de Markov, sin embargo ser\'a regenerativo si el proceso de Markov lo es.
\end{Note}

 
\begin{Note}
Un proceso regenerativo con media de la longitud de ciclo finita es llamado positivo recurrente.
\end{Note}


\begin{Note}
\begin{itemize}
\item[a)] Si el proceso regenerativo $X$ es positivo recurrente y tiene trayectorias muestrales no negativas, entonces la ecuaci\'on anterior es v\'alida.
\item[b)] Si $X$ es positivo recurrente regenerativo, podemos construir una \'unica versi\'on estacionaria de este proceso, $X_{e}=\left\{X_{e}\left(t\right)\right\}$, donde $X_{e}$ es un proceso estoc\'astico regenerativo y estrictamente estacionario, con distribuci\'on marginal distribuida como $X_{\infty}$
\end{itemize}
\end{Note}

\begin{Def}%\label{Def.Tn}
Sean $0\leq T_{1}\leq T_{2}\leq \ldots$ son tiempos aleatorios infinitos en los cuales ocurren ciertos eventos. El n\'umero de tiempos $T_{n}$ en el intervalo $\left[0,t\right)$ es

\begin{eqnarray}
N\left(t\right)=\sum_{n=1}^{\infty}\indora\left(T_{n}\leq t\right),
\end{eqnarray}
para $t\geq0$.
\end{Def}

Si se consideran los puntos $T_{n}$ como elementos de $\rea_{+}$, y $N\left(t\right)$ es el n\'umero de puntos en $\rea$. El proceso denotado por $\left\{N\left(t\right):t\geq0\right\}$, denotado por $N\left(t\right)$, es un proceso puntual en $\rea_{+}$. Los $T_{n}$ son los tiempos de ocurrencia, el proceso puntual $N\left(t\right)$ es simple si su n\'umero de ocurrencias son distintas: $0<T_{1}<T_{2}<\ldots$ casi seguramente.

\begin{Def}
Un proceso puntual $N\left(t\right)$ es un proceso de renovaci\'on si los tiempos de interocurrencia $\xi_{n}=T_{n}-T_{n-1}$, para $n\geq1$, son independientes e identicamente distribuidos con distribuci\'on $F$, donde $F\left(0\right)=0$ y $T_{0}=0$. Los $T_{n}$ son llamados tiempos de renovaci\'on, referente a la independencia o renovaci\'on de la informaci\'on estoc\'astica en estos tiempos. Los $\xi_{n}$ son los tiempos de inter-renovaci\'on, y $N\left(t\right)$ es el n\'umero de renovaciones en el intervalo $\left[0,t\right)$
\end{Def}


\begin{Note}
Para definir un proceso de renovaci\'on para cualquier contexto, solamente hay que especificar una distribuci\'on $F$, con $F\left(0\right)=0$, para los tiempos de inter-renovaci\'on. La funci\'on $F$ en turno degune las otra variables aleatorias. De manera formal, existe un espacio de probabilidad y una sucesi\'on de variables aleatorias $\xi_{1},\xi_{2},\ldots$ definidas en este con distribuci\'on $F$. Entonces las otras cantidades son $T_{n}=\sum_{k=1}^{n}\xi_{k}$ y $N\left(t\right)=\sum_{n=1}^{\infty}\indora\left(T_{n}\leq t\right)$, donde $T_{n}\rightarrow\infty$ casi seguramente por la Ley Fuerte de los Grandes Números.
\end{Note}

%___________________________________________________________________________________________
%
\subsection{Teorema Principal de Renovaci\'on}
%___________________________________________________________________________________________
%

\begin{Note} Una funci\'on $h:\rea_{+}\rightarrow\rea$ es Directamente Riemann Integrable en los siguientes casos:
\begin{itemize}
\item[a)] $h\left(t\right)\geq0$ es decreciente y Riemann Integrable.
\item[b)] $h$ es continua excepto posiblemente en un conjunto de Lebesgue de medida 0, y $|h\left(t\right)|\leq b\left(t\right)$, donde $b$ es DRI.
\end{itemize}
\end{Note}

\begin{Teo}[Teorema Principal de Renovaci\'on]
Si $F$ es no aritm\'etica y $h\left(t\right)$ es Directamente Riemann Integrable (DRI), entonces

\begin{eqnarray*}
lim_{t\rightarrow\infty}U\star h=\frac{1}{\mu}\int_{\rea_{+}}h\left(s\right)ds.
\end{eqnarray*}
\end{Teo}

\begin{Prop}
Cualquier funci\'on $H\left(t\right)$ acotada en intervalos finitos y que es 0 para $t<0$ puede expresarse como
\begin{eqnarray*}
H\left(t\right)=U\star h\left(t\right)\textrm{,  donde }h\left(t\right)=H\left(t\right)-F\star H\left(t\right)
\end{eqnarray*}
\end{Prop}

\begin{Def}
Un proceso estoc\'astico $X\left(t\right)$ es crudamente regenerativo en un tiempo aleatorio positivo $T$ si
\begin{eqnarray*}
\esp\left[X\left(T+t\right)|T\right]=\esp\left[X\left(t\right)\right]\textrm{, para }t\geq0,\end{eqnarray*}
y con las esperanzas anteriores finitas.
\end{Def}

\begin{Prop}
Sup\'ongase que $X\left(t\right)$ es un proceso crudamente regenerativo en $T$, que tiene distribuci\'on $F$. Si $\esp\left[X\left(t\right)\right]$ es acotado en intervalos finitos, entonces
\begin{eqnarray*}
\esp\left[X\left(t\right)\right]=U\star h\left(t\right)\textrm{,  donde }h\left(t\right)=\esp\left[X\left(t\right)\indora\left(T>t\right)\right].
\end{eqnarray*}
\end{Prop}

\begin{Teo}[Regeneraci\'on Cruda]
Sup\'ongase que $X\left(t\right)$ es un proceso con valores positivo crudamente regenerativo en $T$, y def\'inase $M=\sup\left\{|X\left(t\right)|:t\leq T\right\}$. Si $T$ es no aritm\'etico y $M$ y $MT$ tienen media finita, entonces
\begin{eqnarray*}
lim_{t\rightarrow\infty}\esp\left[X\left(t\right)\right]=\frac{1}{\mu}\int_{\rea_{+}}h\left(s\right)ds,
\end{eqnarray*}
donde $h\left(t\right)=\esp\left[X\left(t\right)\indora\left(T>t\right)\right]$.
\end{Teo}

%___________________________________________________________________________________________
%
\subsection{Propiedades de los Procesos de Renovaci\'on}
%___________________________________________________________________________________________
%

Los tiempos $T_{n}$ est\'an relacionados con los conteos de $N\left(t\right)$ por

\begin{eqnarray*}
\left\{N\left(t\right)\geq n\right\}&=&\left\{T_{n}\leq t\right\}\\
T_{N\left(t\right)}\leq &t&<T_{N\left(t\right)+1},
\end{eqnarray*}

adem\'as $N\left(T_{n}\right)=n$, y 

\begin{eqnarray*}
N\left(t\right)=\max\left\{n:T_{n}\leq t\right\}=\min\left\{n:T_{n+1}>t\right\}
\end{eqnarray*}

Por propiedades de la convoluci\'on se sabe que

\begin{eqnarray*}
P\left\{T_{n}\leq t\right\}=F^{n\star}\left(t\right)
\end{eqnarray*}
que es la $n$-\'esima convoluci\'on de $F$. Entonces 

\begin{eqnarray*}
\left\{N\left(t\right)\geq n\right\}&=&\left\{T_{n}\leq t\right\}\\
P\left\{N\left(t\right)\leq n\right\}&=&1-F^{\left(n+1\right)\star}\left(t\right)
\end{eqnarray*}

Adem\'as usando el hecho de que $\esp\left[N\left(t\right)\right]=\sum_{n=1}^{\infty}P\left\{N\left(t\right)\geq n\right\}$
se tiene que

\begin{eqnarray*}
\esp\left[N\left(t\right)\right]=\sum_{n=1}^{\infty}F^{n\star}\left(t\right)
\end{eqnarray*}

\begin{Prop}
Para cada $t\geq0$, la funci\'on generadora de momentos $\esp\left[e^{\alpha N\left(t\right)}\right]$ existe para alguna $\alpha$ en una vecindad del 0, y de aqu\'i que $\esp\left[N\left(t\right)^{m}\right]<\infty$, para $m\geq1$.
\end{Prop}


\begin{Note}
Si el primer tiempo de renovaci\'on $\xi_{1}$ no tiene la misma distribuci\'on que el resto de las $\xi_{n}$, para $n\geq2$, a $N\left(t\right)$ se le llama Proceso de Renovaci\'on retardado, donde si $\xi$ tiene distribuci\'on $G$, entonces el tiempo $T_{n}$ de la $n$-\'esima renovaci\'on tiene distribuci\'on $G\star F^{\left(n-1\right)\star}\left(t\right)$
\end{Note}


\begin{Teo}
Para una constante $\mu\leq\infty$ ( o variable aleatoria), las siguientes expresiones son equivalentes:

\begin{eqnarray}
lim_{n\rightarrow\infty}n^{-1}T_{n}&=&\mu,\textrm{ c.s.}\\
lim_{t\rightarrow\infty}t^{-1}N\left(t\right)&=&1/\mu,\textrm{ c.s.}
\end{eqnarray}
\end{Teo}


Es decir, $T_{n}$ satisface la Ley Fuerte de los Grandes N\'umeros s\'i y s\'olo s\'i $N\left/t\right)$ la cumple.


\begin{Coro}[Ley Fuerte de los Grandes N\'umeros para Procesos de Renovaci\'on]
Si $N\left(t\right)$ es un proceso de renovaci\'on cuyos tiempos de inter-renovaci\'on tienen media $\mu\leq\infty$, entonces
\begin{eqnarray}
t^{-1}N\left(t\right)\rightarrow 1/\mu,\textrm{ c.s. cuando }t\rightarrow\infty.
\end{eqnarray}

\end{Coro}


Considerar el proceso estoc\'astico de valores reales $\left\{Z\left(t\right):t\geq0\right\}$ en el mismo espacio de probabilidad que $N\left(t\right)$

\begin{Def}
Para el proceso $\left\{Z\left(t\right):t\geq0\right\}$ se define la fluctuaci\'on m\'axima de $Z\left(t\right)$ en el intervalo $\left(T_{n-1},T_{n}\right]$:
\begin{eqnarray*}
M_{n}=\sup_{T_{n-1}<t\leq T_{n}}|Z\left(t\right)-Z\left(T_{n-1}\right)|
\end{eqnarray*}
\end{Def}

\begin{Teo}
Sup\'ongase que $n^{-1}T_{n}\rightarrow\mu$ c.s. cuando $n\rightarrow\infty$, donde $\mu\leq\infty$ es una constante o variable aleatoria. Sea $a$ una constante o variable aleatoria que puede ser infinita cuando $\mu$ es finita, y considere las expresiones l\'imite:
\begin{eqnarray}
lim_{n\rightarrow\infty}n^{-1}Z\left(T_{n}\right)&=&a,\textrm{ c.s.}\\
lim_{t\rightarrow\infty}t^{-1}Z\left(t\right)&=&a/\mu,\textrm{ c.s.}
\end{eqnarray}
La segunda expresi\'on implica la primera. Conversamente, la primera implica la segunda si el proceso $Z\left(t\right)$ es creciente, o si $lim_{n\rightarrow\infty}n^{-1}M_{n}=0$ c.s.
\end{Teo}

\begin{Coro}
Si $N\left(t\right)$ es un proceso de renovaci\'on, y $\left(Z\left(T_{n}\right)-Z\left(T_{n-1}\right),M_{n}\right)$, para $n\geq1$, son variables aleatorias independientes e id\'enticamente distribuidas con media finita, entonces,
\begin{eqnarray}
lim_{t\rightarrow\infty}t^{-1}Z\left(t\right)\rightarrow\frac{\esp\left[Z\left(T_{1}\right)-Z\left(T_{0}\right)\right]}{\esp\left[T_{1}\right]},\textrm{ c.s. cuando  }t\rightarrow\infty.
\end{eqnarray}
\end{Coro}



%___________________________________________________________________________________________
%
\subsection{Funci\'on de Renovaci\'on}
%___________________________________________________________________________________________
%


\begin{Def}
Sea $h\left(t\right)$ funci\'on de valores reales en $\rea$ acotada en intervalos finitos e igual a cero para $t<0$ La ecuaci\'on de renovaci\'on para $h\left(t\right)$ y la distribuci\'on $F$ es

\begin{eqnarray}%\label{Ec.Renovacion}
H\left(t\right)=h\left(t\right)+\int_{\left[0,t\right]}H\left(t-s\right)dF\left(s\right)\textrm{,    }t\geq0,
\end{eqnarray}
donde $H\left(t\right)$ es una funci\'on de valores reales. Esto es $H=h+F\star H$. Decimos que $H\left(t\right)$ es soluci\'on de esta ecuaci\'on si satisface la ecuaci\'on, y es acotada en intervalos finitos e iguales a cero para $t<0$.
\end{Def}

\begin{Prop}
La funci\'on $U\star h\left(t\right)$ es la \'unica soluci\'on de la ecuaci\'on de renovaci\'on (\ref{Ec.Renovacion}).
\end{Prop}

\begin{Teo}[Teorema Renovaci\'on Elemental]
\begin{eqnarray*}
t^{-1}U\left(t\right)\rightarrow 1/\mu\textrm{,    cuando }t\rightarrow\infty.
\end{eqnarray*}
\end{Teo}

%______________________________________________________________________
\subsection{Procesos de Renovaci\'on}
%______________________________________________________________________

\begin{Def}%\label{Def.Tn}
Sean $0\leq T_{1}\leq T_{2}\leq \ldots$ son tiempos aleatorios infinitos en los cuales ocurren ciertos eventos. El n\'umero de tiempos $T_{n}$ en el intervalo $\left[0,t\right)$ es

\begin{eqnarray}
N\left(t\right)=\sum_{n=1}^{\infty}\indora\left(T_{n}\leq t\right),
\end{eqnarray}
para $t\geq0$.
\end{Def}

Si se consideran los puntos $T_{n}$ como elementos de $\rea_{+}$, y $N\left(t\right)$ es el n\'umero de puntos en $\rea$. El proceso denotado por $\left\{N\left(t\right):t\geq0\right\}$, denotado por $N\left(t\right)$, es un proceso puntual en $\rea_{+}$. Los $T_{n}$ son los tiempos de ocurrencia, el proceso puntual $N\left(t\right)$ es simple si su n\'umero de ocurrencias son distintas: $0<T_{1}<T_{2}<\ldots$ casi seguramente.

\begin{Def}
Un proceso puntual $N\left(t\right)$ es un proceso de renovaci\'on si los tiempos de interocurrencia $\xi_{n}=T_{n}-T_{n-1}$, para $n\geq1$, son independientes e identicamente distribuidos con distribuci\'on $F$, donde $F\left(0\right)=0$ y $T_{0}=0$. Los $T_{n}$ son llamados tiempos de renovaci\'on, referente a la independencia o renovaci\'on de la informaci\'on estoc\'astica en estos tiempos. Los $\xi_{n}$ son los tiempos de inter-renovaci\'on, y $N\left(t\right)$ es el n\'umero de renovaciones en el intervalo $\left[0,t\right)$
\end{Def}


\begin{Note}
Para definir un proceso de renovaci\'on para cualquier contexto, solamente hay que especificar una distribuci\'on $F$, con $F\left(0\right)=0$, para los tiempos de inter-renovaci\'on. La funci\'on $F$ en turno degune las otra variables aleatorias. De manera formal, existe un espacio de probabilidad y una sucesi\'on de variables aleatorias $\xi_{1},\xi_{2},\ldots$ definidas en este con distribuci\'on $F$. Entonces las otras cantidades son $T_{n}=\sum_{k=1}^{n}\xi_{k}$ y $N\left(t\right)=\sum_{n=1}^{\infty}\indora\left(T_{n}\leq t\right)$, donde $T_{n}\rightarrow\infty$ casi seguramente por la Ley Fuerte de los Grandes Números.
\end{Note}

%___________________________________________________________________________________________
%
\subsection{Renewal and Regenerative Processes: Serfozo\cite{Serfozo}}
%___________________________________________________________________________________________
%
\begin{Def}%\label{Def.Tn}
Sean $0\leq T_{1}\leq T_{2}\leq \ldots$ son tiempos aleatorios infinitos en los cuales ocurren ciertos eventos. El n\'umero de tiempos $T_{n}$ en el intervalo $\left[0,t\right)$ es

\begin{eqnarray}
N\left(t\right)=\sum_{n=1}^{\infty}\indora\left(T_{n}\leq t\right),
\end{eqnarray}
para $t\geq0$.
\end{Def}

Si se consideran los puntos $T_{n}$ como elementos de $\rea_{+}$, y $N\left(t\right)$ es el n\'umero de puntos en $\rea$. El proceso denotado por $\left\{N\left(t\right):t\geq0\right\}$, denotado por $N\left(t\right)$, es un proceso puntual en $\rea_{+}$. Los $T_{n}$ son los tiempos de ocurrencia, el proceso puntual $N\left(t\right)$ es simple si su n\'umero de ocurrencias son distintas: $0<T_{1}<T_{2}<\ldots$ casi seguramente.

\begin{Def}
Un proceso puntual $N\left(t\right)$ es un proceso de renovaci\'on si los tiempos de interocurrencia $\xi_{n}=T_{n}-T_{n-1}$, para $n\geq1$, son independientes e identicamente distribuidos con distribuci\'on $F$, donde $F\left(0\right)=0$ y $T_{0}=0$. Los $T_{n}$ son llamados tiempos de renovaci\'on, referente a la independencia o renovaci\'on de la informaci\'on estoc\'astica en estos tiempos. Los $\xi_{n}$ son los tiempos de inter-renovaci\'on, y $N\left(t\right)$ es el n\'umero de renovaciones en el intervalo $\left[0,t\right)$
\end{Def}


\begin{Note}
Para definir un proceso de renovaci\'on para cualquier contexto, solamente hay que especificar una distribuci\'on $F$, con $F\left(0\right)=0$, para los tiempos de inter-renovaci\'on. La funci\'on $F$ en turno degune las otra variables aleatorias. De manera formal, existe un espacio de probabilidad y una sucesi\'on de variables aleatorias $\xi_{1},\xi_{2},\ldots$ definidas en este con distribuci\'on $F$. Entonces las otras cantidades son $T_{n}=\sum_{k=1}^{n}\xi_{k}$ y $N\left(t\right)=\sum_{n=1}^{\infty}\indora\left(T_{n}\leq t\right)$, donde $T_{n}\rightarrow\infty$ casi seguramente por la Ley Fuerte de los Grandes N\'umeros.
\end{Note}







Los tiempos $T_{n}$ est\'an relacionados con los conteos de $N\left(t\right)$ por

\begin{eqnarray*}
\left\{N\left(t\right)\geq n\right\}&=&\left\{T_{n}\leq t\right\}\\
T_{N\left(t\right)}\leq &t&<T_{N\left(t\right)+1},
\end{eqnarray*}

adem\'as $N\left(T_{n}\right)=n$, y 

\begin{eqnarray*}
N\left(t\right)=\max\left\{n:T_{n}\leq t\right\}=\min\left\{n:T_{n+1}>t\right\}
\end{eqnarray*}

Por propiedades de la convoluci\'on se sabe que

\begin{eqnarray*}
P\left\{T_{n}\leq t\right\}=F^{n\star}\left(t\right)
\end{eqnarray*}
que es la $n$-\'esima convoluci\'on de $F$. Entonces 

\begin{eqnarray*}
\left\{N\left(t\right)\geq n\right\}&=&\left\{T_{n}\leq t\right\}\\
P\left\{N\left(t\right)\leq n\right\}&=&1-F^{\left(n+1\right)\star}\left(t\right)
\end{eqnarray*}

Adem\'as usando el hecho de que $\esp\left[N\left(t\right)\right]=\sum_{n=1}^{\infty}P\left\{N\left(t\right)\geq n\right\}$
se tiene que

\begin{eqnarray*}
\esp\left[N\left(t\right)\right]=\sum_{n=1}^{\infty}F^{n\star}\left(t\right)
\end{eqnarray*}

\begin{Prop}
Para cada $t\geq0$, la funci\'on generadora de momentos $\esp\left[e^{\alpha N\left(t\right)}\right]$ existe para alguna $\alpha$ en una vecindad del 0, y de aqu\'i que $\esp\left[N\left(t\right)^{m}\right]<\infty$, para $m\geq1$.
\end{Prop}

\begin{Ejem}[\textbf{Proceso Poisson}]

Suponga que se tienen tiempos de inter-renovaci\'on \textit{i.i.d.} del proceso de renovaci\'on $N\left(t\right)$ tienen distribuci\'on exponencial $F\left(t\right)=q-e^{-\lambda t}$ con tasa $\lambda$. Entonces $N\left(t\right)$ es un proceso Poisson con tasa $\lambda$.

\end{Ejem}


\begin{Note}
Si el primer tiempo de renovaci\'on $\xi_{1}$ no tiene la misma distribuci\'on que el resto de las $\xi_{n}$, para $n\geq2$, a $N\left(t\right)$ se le llama Proceso de Renovaci\'on retardado, donde si $\xi$ tiene distribuci\'on $G$, entonces el tiempo $T_{n}$ de la $n$-\'esima renovaci\'on tiene distribuci\'on $G\star F^{\left(n-1\right)\star}\left(t\right)$
\end{Note}


\begin{Teo}
Para una constante $\mu\leq\infty$ ( o variable aleatoria), las siguientes expresiones son equivalentes:

\begin{eqnarray}
lim_{n\rightarrow\infty}n^{-1}T_{n}&=&\mu,\textrm{ c.s.}\\
lim_{t\rightarrow\infty}t^{-1}N\left(t\right)&=&1/\mu,\textrm{ c.s.}
\end{eqnarray}
\end{Teo}


Es decir, $T_{n}$ satisface la Ley Fuerte de los Grandes N\'umeros s\'i y s\'olo s\'i $N\left/t\right)$ la cumple.


\begin{Coro}[Ley Fuerte de los Grandes N\'umeros para Procesos de Renovaci\'on]
Si $N\left(t\right)$ es un proceso de renovaci\'on cuyos tiempos de inter-renovaci\'on tienen media $\mu\leq\infty$, entonces
\begin{eqnarray}
t^{-1}N\left(t\right)\rightarrow 1/\mu,\textrm{ c.s. cuando }t\rightarrow\infty.
\end{eqnarray}

\end{Coro}


Considerar el proceso estoc\'astico de valores reales $\left\{Z\left(t\right):t\geq0\right\}$ en el mismo espacio de probabilidad que $N\left(t\right)$

\begin{Def}
Para el proceso $\left\{Z\left(t\right):t\geq0\right\}$ se define la fluctuaci\'on m\'axima de $Z\left(t\right)$ en el intervalo $\left(T_{n-1},T_{n}\right]$:
\begin{eqnarray*}
M_{n}=\sup_{T_{n-1}<t\leq T_{n}}|Z\left(t\right)-Z\left(T_{n-1}\right)|
\end{eqnarray*}
\end{Def}

\begin{Teo}
Sup\'ongase que $n^{-1}T_{n}\rightarrow\mu$ c.s. cuando $n\rightarrow\infty$, donde $\mu\leq\infty$ es una constante o variable aleatoria. Sea $a$ una constante o variable aleatoria que puede ser infinita cuando $\mu$ es finita, y considere las expresiones l\'imite:
\begin{eqnarray}
lim_{n\rightarrow\infty}n^{-1}Z\left(T_{n}\right)&=&a,\textrm{ c.s.}\\
lim_{t\rightarrow\infty}t^{-1}Z\left(t\right)&=&a/\mu,\textrm{ c.s.}
\end{eqnarray}
La segunda expresi\'on implica la primera. Conversamente, la primera implica la segunda si el proceso $Z\left(t\right)$ es creciente, o si $lim_{n\rightarrow\infty}n^{-1}M_{n}=0$ c.s.
\end{Teo}

\begin{Coro}
Si $N\left(t\right)$ es un proceso de renovaci\'on, y $\left(Z\left(T_{n}\right)-Z\left(T_{n-1}\right),M_{n}\right)$, para $n\geq1$, son variables aleatorias independientes e id\'enticamente distribuidas con media finita, entonces,
\begin{eqnarray}
lim_{t\rightarrow\infty}t^{-1}Z\left(t\right)\rightarrow\frac{\esp\left[Z\left(T_{1}\right)-Z\left(T_{0}\right)\right]}{\esp\left[T_{1}\right]},\textrm{ c.s. cuando  }t\rightarrow\infty.
\end{eqnarray}
\end{Coro}


Sup\'ongase que $N\left(t\right)$ es un proceso de renovaci\'on con distribuci\'on $F$ con media finita $\mu$.

\begin{Def}
La funci\'on de renovaci\'on asociada con la distribuci\'on $F$, del proceso $N\left(t\right)$, es
\begin{eqnarray*}
U\left(t\right)=\sum_{n=1}^{\infty}F^{n\star}\left(t\right),\textrm{   }t\geq0,
\end{eqnarray*}
donde $F^{0\star}\left(t\right)=\indora\left(t\geq0\right)$.
\end{Def}


\begin{Prop}
Sup\'ongase que la distribuci\'on de inter-renovaci\'on $F$ tiene densidad $f$. Entonces $U\left(t\right)$ tambi\'en tiene densidad, para $t>0$, y es $U^{'}\left(t\right)=\sum_{n=0}^{\infty}f^{n\star}\left(t\right)$. Adem\'as
\begin{eqnarray*}
\prob\left\{N\left(t\right)>N\left(t-\right)\right\}=0\textrm{,   }t\geq0.
\end{eqnarray*}
\end{Prop}

\begin{Def}
La Transformada de Laplace-Stieljes de $F$ est\'a dada por

\begin{eqnarray*}
\hat{F}\left(\alpha\right)=\int_{\rea_{+}}e^{-\alpha t}dF\left(t\right)\textrm{,  }\alpha\geq0.
\end{eqnarray*}
\end{Def}

Entonces

\begin{eqnarray*}
\hat{U}\left(\alpha\right)=\sum_{n=0}^{\infty}\hat{F^{n\star}}\left(\alpha\right)=\sum_{n=0}^{\infty}\hat{F}\left(\alpha\right)^{n}=\frac{1}{1-\hat{F}\left(\alpha\right)}.
\end{eqnarray*}


\begin{Prop}
La Transformada de Laplace $\hat{U}\left(\alpha\right)$ y $\hat{F}\left(\alpha\right)$ determina una a la otra de manera \'unica por la relaci\'on $\hat{U}\left(\alpha\right)=\frac{1}{1-\hat{F}\left(\alpha\right)}$.
\end{Prop}


\begin{Note}
Un proceso de renovaci\'on $N\left(t\right)$ cuyos tiempos de inter-renovaci\'on tienen media finita, es un proceso Poisson con tasa $\lambda$ si y s\'olo s\'i $\esp\left[U\left(t\right)\right]=\lambda t$, para $t\geq0$.
\end{Note}


\begin{Teo}
Sea $N\left(t\right)$ un proceso puntual simple con puntos de localizaci\'on $T_{n}$ tal que $\eta\left(t\right)=\esp\left[N\left(\right)\right]$ es finita para cada $t$. Entonces para cualquier funci\'on $f:\rea_{+}\rightarrow\rea$,
\begin{eqnarray*}
\esp\left[\sum_{n=1}^{N\left(\right)}f\left(T_{n}\right)\right]=\int_{\left(0,t\right]}f\left(s\right)d\eta\left(s\right)\textrm{,  }t\geq0,
\end{eqnarray*}
suponiendo que la integral exista. Adem\'as si $X_{1},X_{2},\ldots$ son variables aleatorias definidas en el mismo espacio de probabilidad que el proceso $N\left(t\right)$ tal que $\esp\left[X_{n}|T_{n}=s\right]=f\left(s\right)$, independiente de $n$. Entonces
\begin{eqnarray*}
\esp\left[\sum_{n=1}^{N\left(t\right)}X_{n}\right]=\int_{\left(0,t\right]}f\left(s\right)d\eta\left(s\right)\textrm{,  }t\geq0,
\end{eqnarray*} 
suponiendo que la integral exista. 
\end{Teo}

\begin{Coro}[Identidad de Wald para Renovaciones]
Para el proceso de renovaci\'on $N\left(t\right)$,
\begin{eqnarray*}
\esp\left[T_{N\left(t\right)+1}\right]=\mu\esp\left[N\left(t\right)+1\right]\textrm{,  }t\geq0,
\end{eqnarray*}  
\end{Coro}


\begin{Def}
Sea $h\left(t\right)$ funci\'on de valores reales en $\rea$ acotada en intervalos finitos e igual a cero para $t<0$ La ecuaci\'on de renovaci\'on para $h\left(t\right)$ y la distribuci\'on $F$ es

\begin{eqnarray}%\label{Ec.Renovacion}
H\left(t\right)=h\left(t\right)+\int_{\left[0,t\right]}H\left(t-s\right)dF\left(s\right)\textrm{,    }t\geq0,
\end{eqnarray}
donde $H\left(t\right)$ es una funci\'on de valores reales. Esto es $H=h+F\star H$. Decimos que $H\left(t\right)$ es soluci\'on de esta ecuaci\'on si satisface la ecuaci\'on, y es acotada en intervalos finitos e iguales a cero para $t<0$.
\end{Def}

\begin{Prop}
La funci\'on $U\star h\left(t\right)$ es la \'unica soluci\'on de la ecuaci\'on de renovaci\'on (\ref{Ec.Renovacion}).
\end{Prop}

\begin{Teo}[Teorema Renovaci\'on Elemental]
\begin{eqnarray*}
t^{-1}U\left(t\right)\rightarrow 1/\mu\textrm{,    cuando }t\rightarrow\infty.
\end{eqnarray*}
\end{Teo}



Sup\'ongase que $N\left(t\right)$ es un proceso de renovaci\'on con distribuci\'on $F$ con media finita $\mu$.

\begin{Def}
La funci\'on de renovaci\'on asociada con la distribuci\'on $F$, del proceso $N\left(t\right)$, es
\begin{eqnarray*}
U\left(t\right)=\sum_{n=1}^{\infty}F^{n\star}\left(t\right),\textrm{   }t\geq0,
\end{eqnarray*}
donde $F^{0\star}\left(t\right)=\indora\left(t\geq0\right)$.
\end{Def}


\begin{Prop}
Sup\'ongase que la distribuci\'on de inter-renovaci\'on $F$ tiene densidad $f$. Entonces $U\left(t\right)$ tambi\'en tiene densidad, para $t>0$, y es $U^{'}\left(t\right)=\sum_{n=0}^{\infty}f^{n\star}\left(t\right)$. Adem\'as
\begin{eqnarray*}
\prob\left\{N\left(t\right)>N\left(t-\right)\right\}=0\textrm{,   }t\geq0.
\end{eqnarray*}
\end{Prop}

\begin{Def}
La Transformada de Laplace-Stieljes de $F$ est\'a dada por

\begin{eqnarray*}
\hat{F}\left(\alpha\right)=\int_{\rea_{+}}e^{-\alpha t}dF\left(t\right)\textrm{,  }\alpha\geq0.
\end{eqnarray*}
\end{Def}

Entonces

\begin{eqnarray*}
\hat{U}\left(\alpha\right)=\sum_{n=0}^{\infty}\hat{F^{n\star}}\left(\alpha\right)=\sum_{n=0}^{\infty}\hat{F}\left(\alpha\right)^{n}=\frac{1}{1-\hat{F}\left(\alpha\right)}.
\end{eqnarray*}


\begin{Prop}
La Transformada de Laplace $\hat{U}\left(\alpha\right)$ y $\hat{F}\left(\alpha\right)$ determina una a la otra de manera \'unica por la relaci\'on $\hat{U}\left(\alpha\right)=\frac{1}{1-\hat{F}\left(\alpha\right)}$.
\end{Prop}


\begin{Note}
Un proceso de renovaci\'on $N\left(t\right)$ cuyos tiempos de inter-renovaci\'on tienen media finita, es un proceso Poisson con tasa $\lambda$ si y s\'olo s\'i $\esp\left[U\left(t\right)\right]=\lambda t$, para $t\geq0$.
\end{Note}


\begin{Teo}
Sea $N\left(t\right)$ un proceso puntual simple con puntos de localizaci\'on $T_{n}$ tal que $\eta\left(t\right)=\esp\left[N\left(\right)\right]$ es finita para cada $t$. Entonces para cualquier funci\'on $f:\rea_{+}\rightarrow\rea$,
\begin{eqnarray*}
\esp\left[\sum_{n=1}^{N\left(\right)}f\left(T_{n}\right)\right]=\int_{\left(0,t\right]}f\left(s\right)d\eta\left(s\right)\textrm{,  }t\geq0,
\end{eqnarray*}
suponiendo que la integral exista. Adem\'as si $X_{1},X_{2},\ldots$ son variables aleatorias definidas en el mismo espacio de probabilidad que el proceso $N\left(t\right)$ tal que $\esp\left[X_{n}|T_{n}=s\right]=f\left(s\right)$, independiente de $n$. Entonces
\begin{eqnarray*}
\esp\left[\sum_{n=1}^{N\left(t\right)}X_{n}\right]=\int_{\left(0,t\right]}f\left(s\right)d\eta\left(s\right)\textrm{,  }t\geq0,
\end{eqnarray*} 
suponiendo que la integral exista. 
\end{Teo}

\begin{Coro}[Identidad de Wald para Renovaciones]
Para el proceso de renovaci\'on $N\left(t\right)$,
\begin{eqnarray*}
\esp\left[T_{N\left(t\right)+1}\right]=\mu\esp\left[N\left(t\right)+1\right]\textrm{,  }t\geq0,
\end{eqnarray*}  
\end{Coro}


\begin{Def}
Sea $h\left(t\right)$ funci\'on de valores reales en $\rea$ acotada en intervalos finitos e igual a cero para $t<0$ La ecuaci\'on de renovaci\'on para $h\left(t\right)$ y la distribuci\'on $F$ es

\begin{eqnarray}%\label{Ec.Renovacion}
H\left(t\right)=h\left(t\right)+\int_{\left[0,t\right]}H\left(t-s\right)dF\left(s\right)\textrm{,    }t\geq0,
\end{eqnarray}
donde $H\left(t\right)$ es una funci\'on de valores reales. Esto es $H=h+F\star H$. Decimos que $H\left(t\right)$ es soluci\'on de esta ecuaci\'on si satisface la ecuaci\'on, y es acotada en intervalos finitos e iguales a cero para $t<0$.
\end{Def}

\begin{Prop}
La funci\'on $U\star h\left(t\right)$ es la \'unica soluci\'on de la ecuaci\'on de renovaci\'on (\ref{Ec.Renovacion}).
\end{Prop}

\begin{Teo}[Teorema Renovaci\'on Elemental]
\begin{eqnarray*}
t^{-1}U\left(t\right)\rightarrow 1/\mu\textrm{,    cuando }t\rightarrow\infty.
\end{eqnarray*}
\end{Teo}


\begin{Note} Una funci\'on $h:\rea_{+}\rightarrow\rea$ es Directamente Riemann Integrable en los siguientes casos:
\begin{itemize}
\item[a)] $h\left(t\right)\geq0$ es decreciente y Riemann Integrable.
\item[b)] $h$ es continua excepto posiblemente en un conjunto de Lebesgue de medida 0, y $|h\left(t\right)|\leq b\left(t\right)$, donde $b$ es DRI.
\end{itemize}
\end{Note}

\begin{Teo}[Teorema Principal de Renovaci\'on]
Si $F$ es no aritm\'etica y $h\left(t\right)$ es Directamente Riemann Integrable (DRI), entonces

\begin{eqnarray*}
lim_{t\rightarrow\infty}U\star h=\frac{1}{\mu}\int_{\rea_{+}}h\left(s\right)ds.
\end{eqnarray*}
\end{Teo}

\begin{Prop}
Cualquier funci\'on $H\left(t\right)$ acotada en intervalos finitos y que es 0 para $t<0$ puede expresarse como
\begin{eqnarray*}
H\left(t\right)=U\star h\left(t\right)\textrm{,  donde }h\left(t\right)=H\left(t\right)-F\star H\left(t\right)
\end{eqnarray*}
\end{Prop}

\begin{Def}
Un proceso estoc\'astico $X\left(t\right)$ es crudamente regenerativo en un tiempo aleatorio positivo $T$ si
\begin{eqnarray*}
\esp\left[X\left(T+t\right)|T\right]=\esp\left[X\left(t\right)\right]\textrm{, para }t\geq0,\end{eqnarray*}
y con las esperanzas anteriores finitas.
\end{Def}

\begin{Prop}
Sup\'ongase que $X\left(t\right)$ es un proceso crudamente regenerativo en $T$, que tiene distribuci\'on $F$. Si $\esp\left[X\left(t\right)\right]$ es acotado en intervalos finitos, entonces
\begin{eqnarray*}
\esp\left[X\left(t\right)\right]=U\star h\left(t\right)\textrm{,  donde }h\left(t\right)=\esp\left[X\left(t\right)\indora\left(T>t\right)\right].
\end{eqnarray*}
\end{Prop}

\begin{Teo}[Regeneraci\'on Cruda]
Sup\'ongase que $X\left(t\right)$ es un proceso con valores positivo crudamente regenerativo en $T$, y def\'inase $M=\sup\left\{|X\left(t\right)|:t\leq T\right\}$. Si $T$ es no aritm\'etico y $M$ y $MT$ tienen media finita, entonces
\begin{eqnarray*}
lim_{t\rightarrow\infty}\esp\left[X\left(t\right)\right]=\frac{1}{\mu}\int_{\rea_{+}}h\left(s\right)ds,
\end{eqnarray*}
donde $h\left(t\right)=\esp\left[X\left(t\right)\indora\left(T>t\right)\right]$.
\end{Teo}


\begin{Note} Una funci\'on $h:\rea_{+}\rightarrow\rea$ es Directamente Riemann Integrable en los siguientes casos:
\begin{itemize}
\item[a)] $h\left(t\right)\geq0$ es decreciente y Riemann Integrable.
\item[b)] $h$ es continua excepto posiblemente en un conjunto de Lebesgue de medida 0, y $|h\left(t\right)|\leq b\left(t\right)$, donde $b$ es DRI.
\end{itemize}
\end{Note}

\begin{Teo}[Teorema Principal de Renovaci\'on]
Si $F$ es no aritm\'etica y $h\left(t\right)$ es Directamente Riemann Integrable (DRI), entonces

\begin{eqnarray*}
lim_{t\rightarrow\infty}U\star h=\frac{1}{\mu}\int_{\rea_{+}}h\left(s\right)ds.
\end{eqnarray*}
\end{Teo}

\begin{Prop}
Cualquier funci\'on $H\left(t\right)$ acotada en intervalos finitos y que es 0 para $t<0$ puede expresarse como
\begin{eqnarray*}
H\left(t\right)=U\star h\left(t\right)\textrm{,  donde }h\left(t\right)=H\left(t\right)-F\star H\left(t\right)
\end{eqnarray*}
\end{Prop}

\begin{Def}
Un proceso estoc\'astico $X\left(t\right)$ es crudamente regenerativo en un tiempo aleatorio positivo $T$ si
\begin{eqnarray*}
\esp\left[X\left(T+t\right)|T\right]=\esp\left[X\left(t\right)\right]\textrm{, para }t\geq0,\end{eqnarray*}
y con las esperanzas anteriores finitas.
\end{Def}

\begin{Prop}
Sup\'ongase que $X\left(t\right)$ es un proceso crudamente regenerativo en $T$, que tiene distribuci\'on $F$. Si $\esp\left[X\left(t\right)\right]$ es acotado en intervalos finitos, entonces
\begin{eqnarray*}
\esp\left[X\left(t\right)\right]=U\star h\left(t\right)\textrm{,  donde }h\left(t\right)=\esp\left[X\left(t\right)\indora\left(T>t\right)\right].
\end{eqnarray*}
\end{Prop}

\begin{Teo}[Regeneraci\'on Cruda]
Sup\'ongase que $X\left(t\right)$ es un proceso con valores positivo crudamente regenerativo en $T$, y def\'inase $M=\sup\left\{|X\left(t\right)|:t\leq T\right\}$. Si $T$ es no aritm\'etico y $M$ y $MT$ tienen media finita, entonces
\begin{eqnarray*}
lim_{t\rightarrow\infty}\esp\left[X\left(t\right)\right]=\frac{1}{\mu}\int_{\rea_{+}}h\left(s\right)ds,
\end{eqnarray*}
donde $h\left(t\right)=\esp\left[X\left(t\right)\indora\left(T>t\right)\right]$.
\end{Teo}

\begin{Def}
Para el proceso $\left\{\left(N\left(t\right),X\left(t\right)\right):t\geq0\right\}$, sus trayectoria muestrales en el intervalo de tiempo $\left[T_{n-1},T_{n}\right)$ est\'an descritas por
\begin{eqnarray*}
\zeta_{n}=\left(\xi_{n},\left\{X\left(T_{n-1}+t\right):0\leq t<\xi_{n}\right\}\right)
\end{eqnarray*}
Este $\zeta_{n}$ es el $n$-\'esimo segmento del proceso. El proceso es regenerativo sobre los tiempos $T_{n}$ si sus segmentos $\zeta_{n}$ son independientes e id\'enticamennte distribuidos.
\end{Def}


\begin{Note}
Si $\tilde{X}\left(t\right)$ con espacio de estados $\tilde{S}$ es regenerativo sobre $T_{n}$, entonces $X\left(t\right)=f\left(\tilde{X}\left(t\right)\right)$ tambi\'en es regenerativo sobre $T_{n}$, para cualquier funci\'on $f:\tilde{S}\rightarrow S$.
\end{Note}

\begin{Note}
Los procesos regenerativos son crudamente regenerativos, pero no al rev\'es.
\end{Note}


\begin{Note}
Un proceso estoc\'astico a tiempo continuo o discreto es regenerativo si existe un proceso de renovaci\'on  tal que los segmentos del proceso entre tiempos de renovaci\'on sucesivos son i.i.d., es decir, para $\left\{X\left(t\right):t\geq0\right\}$ proceso estoc\'astico a tiempo continuo con espacio de estados $S$, espacio m\'etrico.
\end{Note}

Para $\left\{X\left(t\right):t\geq0\right\}$ Proceso Estoc\'astico a tiempo continuo con estado de espacios $S$, que es un espacio m\'etrico, con trayectorias continuas por la derecha y con l\'imites por la izquierda c.s. Sea $N\left(t\right)$ un proceso de renovaci\'on en $\rea_{+}$ definido en el mismo espacio de probabilidad que $X\left(t\right)$, con tiempos de renovaci\'on $T$ y tiempos de inter-renovaci\'on $\xi_{n}=T_{n}-T_{n-1}$, con misma distribuci\'on $F$ de media finita $\mu$.



\begin{Def}
Para el proceso $\left\{\left(N\left(t\right),X\left(t\right)\right):t\geq0\right\}$, sus trayectoria muestrales en el intervalo de tiempo $\left[T_{n-1},T_{n}\right)$ est\'an descritas por
\begin{eqnarray*}
\zeta_{n}=\left(\xi_{n},\left\{X\left(T_{n-1}+t\right):0\leq t<\xi_{n}\right\}\right)
\end{eqnarray*}
Este $\zeta_{n}$ es el $n$-\'esimo segmento del proceso. El proceso es regenerativo sobre los tiempos $T_{n}$ si sus segmentos $\zeta_{n}$ son independientes e id\'enticamennte distribuidos.
\end{Def}

\begin{Note}
Un proceso regenerativo con media de la longitud de ciclo finita es llamado positivo recurrente.
\end{Note}

\begin{Teo}[Procesos Regenerativos]
Suponga que el proceso
\end{Teo}


\begin{Def}[Renewal Process Trinity]
Para un proceso de renovaci\'on $N\left(t\right)$, los siguientes procesos proveen de informaci\'on sobre los tiempos de renovaci\'on.
\begin{itemize}
\item $A\left(t\right)=t-T_{N\left(t\right)}$, el tiempo de recurrencia hacia atr\'as al tiempo $t$, que es el tiempo desde la \'ultima renovaci\'on para $t$.

\item $B\left(t\right)=T_{N\left(t\right)+1}-t$, el tiempo de recurrencia hacia adelante al tiempo $t$, residual del tiempo de renovaci\'on, que es el tiempo para la pr\'oxima renovaci\'on despu\'es de $t$.

\item $L\left(t\right)=\xi_{N\left(t\right)+1}=A\left(t\right)+B\left(t\right)$, la longitud del intervalo de renovaci\'on que contiene a $t$.
\end{itemize}
\end{Def}

\begin{Note}
El proceso tridimensional $\left(A\left(t\right),B\left(t\right),L\left(t\right)\right)$ es regenerativo sobre $T_{n}$, y por ende cada proceso lo es. Cada proceso $A\left(t\right)$ y $B\left(t\right)$ son procesos de MArkov a tiempo continuo con trayectorias continuas por partes en el espacio de estados $\rea_{+}$. Una expresi\'on conveniente para su distribuci\'on conjunta es, para $0\leq x<t,y\geq0$
\begin{equation}\label{NoRenovacion}
P\left\{A\left(t\right)>x,B\left(t\right)>y\right\}=
P\left\{N\left(t+y\right)-N\left((t-x)\right)=0\right\}
\end{equation}
\end{Note}

\begin{Ejem}[Tiempos de recurrencia Poisson]
Si $N\left(t\right)$ es un proceso Poisson con tasa $\lambda$, entonces de la expresi\'on (\ref{NoRenovacion}) se tiene que

\begin{eqnarray*}
\begin{array}{lc}
P\left\{A\left(t\right)>x,B\left(t\right)>y\right\}=e^{-\lambda\left(x+y\right)},&0\leq x<t,y\geq0,
\end{array}
\end{eqnarray*}
que es la probabilidad Poisson de no renovaciones en un intervalo de longitud $x+y$.

\end{Ejem}

\begin{Note}
Una cadena de Markov erg\'odica tiene la propiedad de ser estacionaria si la distribuci\'on de su estado al tiempo $0$ es su distribuci\'on estacionaria.
\end{Note}


\begin{Def}
Un proceso estoc\'astico a tiempo continuo $\left\{X\left(t\right):t\geq0\right\}$ en un espacio general es estacionario si sus distribuciones finito dimensionales son invariantes bajo cualquier  traslado: para cada $0\leq s_{1}<s_{2}<\cdots<s_{k}$ y $t\geq0$,
\begin{eqnarray*}
\left(X\left(s_{1}+t\right),\ldots,X\left(s_{k}+t\right)\right)=_{d}\left(X\left(s_{1}\right),\ldots,X\left(s_{k}\right)\right).
\end{eqnarray*}
\end{Def}

\begin{Note}
Un proceso de Markov es estacionario si $X\left(t\right)=_{d}X\left(0\right)$, $t\geq0$.
\end{Note}

Considerese el proceso $N\left(t\right)=\sum_{n}\indora\left(\tau_{n}\leq t\right)$ en $\rea_{+}$, con puntos $0<\tau_{1}<\tau_{2}<\cdots$.

\begin{Prop}
Si $N$ es un proceso puntual estacionario y $\esp\left[N\left(1\right)\right]<\infty$, entonces $\esp\left[N\left(t\right)\right]=t\esp\left[N\left(1\right)\right]$, $t\geq0$

\end{Prop}

\begin{Teo}
Los siguientes enunciados son equivalentes
\begin{itemize}
\item[i)] El proceso retardado de renovaci\'on $N$ es estacionario.

\item[ii)] EL proceso de tiempos de recurrencia hacia adelante $B\left(t\right)$ es estacionario.


\item[iii)] $\esp\left[N\left(t\right)\right]=t/\mu$,


\item[iv)] $G\left(t\right)=F_{e}\left(t\right)=\frac{1}{\mu}\int_{0}^{t}\left[1-F\left(s\right)\right]ds$
\end{itemize}
Cuando estos enunciados son ciertos, $P\left\{B\left(t\right)\leq x\right\}=F_{e}\left(x\right)$, para $t,x\geq0$.

\end{Teo}

\begin{Note}
Una consecuencia del teorema anterior es que el Proceso Poisson es el \'unico proceso sin retardo que es estacionario.
\end{Note}

\begin{Coro}
El proceso de renovaci\'on $N\left(t\right)$ sin retardo, y cuyos tiempos de inter renonaci\'on tienen media finita, es estacionario si y s\'olo si es un proceso Poisson.

\end{Coro}


%________________________________________________________________________
\subsection{Procesos Regenerativos}
%________________________________________________________________________



\begin{Note}
Si $\tilde{X}\left(t\right)$ con espacio de estados $\tilde{S}$ es regenerativo sobre $T_{n}$, entonces $X\left(t\right)=f\left(\tilde{X}\left(t\right)\right)$ tambi\'en es regenerativo sobre $T_{n}$, para cualquier funci\'on $f:\tilde{S}\rightarrow S$.
\end{Note}

\begin{Note}
Los procesos regenerativos son crudamente regenerativos, pero no al rev\'es.
\end{Note}
%\subsection*{Procesos Regenerativos: Sigman\cite{Sigman1}}
\begin{Def}[Definici\'on Cl\'asica]
Un proceso estoc\'astico $X=\left\{X\left(t\right):t\geq0\right\}$ es llamado regenerativo is existe una variable aleatoria $R_{1}>0$ tal que
\begin{itemize}
\item[i)] $\left\{X\left(t+R_{1}\right):t\geq0\right\}$ es independiente de $\left\{\left\{X\left(t\right):t<R_{1}\right\},\right\}$
\item[ii)] $\left\{X\left(t+R_{1}\right):t\geq0\right\}$ es estoc\'asticamente equivalente a $\left\{X\left(t\right):t>0\right\}$
\end{itemize}

Llamamos a $R_{1}$ tiempo de regeneraci\'on, y decimos que $X$ se regenera en este punto.
\end{Def}

$\left\{X\left(t+R_{1}\right)\right\}$ es regenerativo con tiempo de regeneraci\'on $R_{2}$, independiente de $R_{1}$ pero con la misma distribuci\'on que $R_{1}$. Procediendo de esta manera se obtiene una secuencia de variables aleatorias independientes e id\'enticamente distribuidas $\left\{R_{n}\right\}$ llamados longitudes de ciclo. Si definimos a $Z_{k}\equiv R_{1}+R_{2}+\cdots+R_{k}$, se tiene un proceso de renovaci\'on llamado proceso de renovaci\'on encajado para $X$.




\begin{Def}
Para $x$ fijo y para cada $t\geq0$, sea $I_{x}\left(t\right)=1$ si $X\left(t\right)\leq x$,  $I_{x}\left(t\right)=0$ en caso contrario, y def\'inanse los tiempos promedio
\begin{eqnarray*}
\overline{X}&=&lim_{t\rightarrow\infty}\frac{1}{t}\int_{0}^{\infty}X\left(u\right)du\\
\prob\left(X_{\infty}\leq x\right)&=&lim_{t\rightarrow\infty}\frac{1}{t}\int_{0}^{\infty}I_{x}\left(u\right)du,
\end{eqnarray*}
cuando estos l\'imites existan.
\end{Def}

Como consecuencia del teorema de Renovaci\'on-Recompensa, se tiene que el primer l\'imite  existe y es igual a la constante
\begin{eqnarray*}
\overline{X}&=&\frac{\esp\left[\int_{0}^{R_{1}}X\left(t\right)dt\right]}{\esp\left[R_{1}\right]},
\end{eqnarray*}
suponiendo que ambas esperanzas son finitas.

\begin{Note}
\begin{itemize}
\item[a)] Si el proceso regenerativo $X$ es positivo recurrente y tiene trayectorias muestrales no negativas, entonces la ecuaci\'on anterior es v\'alida.
\item[b)] Si $X$ es positivo recurrente regenerativo, podemos construir una \'unica versi\'on estacionaria de este proceso, $X_{e}=\left\{X_{e}\left(t\right)\right\}$, donde $X_{e}$ es un proceso estoc\'astico regenerativo y estrictamente estacionario, con distribuci\'on marginal distribuida como $X_{\infty}$
\end{itemize}
\end{Note}

%________________________________________________________________________
\subsection{Procesos Regenerativos}
%________________________________________________________________________

Para $\left\{X\left(t\right):t\geq0\right\}$ Proceso Estoc\'astico a tiempo continuo con estado de espacios $S$, que es un espacio m\'etrico, con trayectorias continuas por la derecha y con l\'imites por la izquierda c.s. Sea $N\left(t\right)$ un proceso de renovaci\'on en $\rea_{+}$ definido en el mismo espacio de probabilidad que $X\left(t\right)$, con tiempos de renovaci\'on $T$ y tiempos de inter-renovaci\'on $\xi_{n}=T_{n}-T_{n-1}$, con misma distribuci\'on $F$ de media finita $\mu$.



\begin{Def}
Para el proceso $\left\{\left(N\left(t\right),X\left(t\right)\right):t\geq0\right\}$, sus trayectoria muestrales en el intervalo de tiempo $\left[T_{n-1},T_{n}\right)$ est\'an descritas por
\begin{eqnarray*}
\zeta_{n}=\left(\xi_{n},\left\{X\left(T_{n-1}+t\right):0\leq t<\xi_{n}\right\}\right)
\end{eqnarray*}
Este $\zeta_{n}$ es el $n$-\'esimo segmento del proceso. El proceso es regenerativo sobre los tiempos $T_{n}$ si sus segmentos $\zeta_{n}$ son independientes e id\'enticamennte distribuidos.
\end{Def}


\begin{Note}
Si $\tilde{X}\left(t\right)$ con espacio de estados $\tilde{S}$ es regenerativo sobre $T_{n}$, entonces $X\left(t\right)=f\left(\tilde{X}\left(t\right)\right)$ tambi\'en es regenerativo sobre $T_{n}$, para cualquier funci\'on $f:\tilde{S}\rightarrow S$.
\end{Note}

\begin{Note}
Los procesos regenerativos son crudamente regenerativos, pero no al rev\'es.
\end{Note}

\begin{Def}[Definici\'on Cl\'asica]
Un proceso estoc\'astico $X=\left\{X\left(t\right):t\geq0\right\}$ es llamado regenerativo is existe una variable aleatoria $R_{1}>0$ tal que
\begin{itemize}
\item[i)] $\left\{X\left(t+R_{1}\right):t\geq0\right\}$ es independiente de $\left\{\left\{X\left(t\right):t<R_{1}\right\},\right\}$
\item[ii)] $\left\{X\left(t+R_{1}\right):t\geq0\right\}$ es estoc\'asticamente equivalente a $\left\{X\left(t\right):t>0\right\}$
\end{itemize}

Llamamos a $R_{1}$ tiempo de regeneraci\'on, y decimos que $X$ se regenera en este punto.
\end{Def}

$\left\{X\left(t+R_{1}\right)\right\}$ es regenerativo con tiempo de regeneraci\'on $R_{2}$, independiente de $R_{1}$ pero con la misma distribuci\'on que $R_{1}$. Procediendo de esta manera se obtiene una secuencia de variables aleatorias independientes e id\'enticamente distribuidas $\left\{R_{n}\right\}$ llamados longitudes de ciclo. Si definimos a $Z_{k}\equiv R_{1}+R_{2}+\cdots+R_{k}$, se tiene un proceso de renovaci\'on llamado proceso de renovaci\'on encajado para $X$.

\begin{Note}
Un proceso regenerativo con media de la longitud de ciclo finita es llamado positivo recurrente.
\end{Note}


\begin{Def}
Para $x$ fijo y para cada $t\geq0$, sea $I_{x}\left(t\right)=1$ si $X\left(t\right)\leq x$,  $I_{x}\left(t\right)=0$ en caso contrario, y def\'inanse los tiempos promedio
\begin{eqnarray*}
\overline{X}&=&lim_{t\rightarrow\infty}\frac{1}{t}\int_{0}^{\infty}X\left(u\right)du\\
\prob\left(X_{\infty}\leq x\right)&=&lim_{t\rightarrow\infty}\frac{1}{t}\int_{0}^{\infty}I_{x}\left(u\right)du,
\end{eqnarray*}
cuando estos l\'imites existan.
\end{Def}

Como consecuencia del teorema de Renovaci\'on-Recompensa, se tiene que el primer l\'imite  existe y es igual a la constante
\begin{eqnarray*}
\overline{X}&=&\frac{\esp\left[\int_{0}^{R_{1}}X\left(t\right)dt\right]}{\esp\left[R_{1}\right]},
\end{eqnarray*}
suponiendo que ambas esperanzas son finitas.

\begin{Note}
\begin{itemize}
\item[a)] Si el proceso regenerativo $X$ es positivo recurrente y tiene trayectorias muestrales no negativas, entonces la ecuaci\'on anterior es v\'alida.
\item[b)] Si $X$ es positivo recurrente regenerativo, podemos construir una \'unica versi\'on estacionaria de este proceso, $X_{e}=\left\{X_{e}\left(t\right)\right\}$, donde $X_{e}$ es un proceso estoc\'astico regenerativo y estrictamente estacionario, con distribuci\'on marginal distribuida como $X_{\infty}$
\end{itemize}
\end{Note}

%__________________________________________________________________________________________
\subsection{Procesos Regenerativos Estacionarios - Stidham \cite{Stidham}}
%__________________________________________________________________________________________


Un proceso estoc\'astico a tiempo continuo $\left\{V\left(t\right),t\geq0\right\}$ es un proceso regenerativo si existe una sucesi\'on de variables aleatorias independientes e id\'enticamente distribuidas $\left\{X_{1},X_{2},\ldots\right\}$, sucesi\'on de renovaci\'on, tal que para cualquier conjunto de Borel $A$, 

\begin{eqnarray*}
\prob\left\{V\left(t\right)\in A|X_{1}+X_{2}+\cdots+X_{R\left(t\right)}=s,\left\{V\left(\tau\right),\tau<s\right\}\right\}=\prob\left\{V\left(t-s\right)\in A|X_{1}>t-s\right\},
\end{eqnarray*}
para todo $0\leq s\leq t$, donde $R\left(t\right)=\max\left\{X_{1}+X_{2}+\cdots+X_{j}\leq t\right\}=$n\'umero de renovaciones ({\emph{puntos de regeneraci\'on}}) que ocurren en $\left[0,t\right]$. El intervalo $\left[0,X_{1}\right)$ es llamado {\emph{primer ciclo de regeneraci\'on}} de $\left\{V\left(t \right),t\geq0\right\}$, $\left[X_{1},X_{1}+X_{2}\right)$ el {\emph{segundo ciclo de regeneraci\'on}}, y as\'i sucesivamente.

Sea $X=X_{1}$ y sea $F$ la funci\'on de distrbuci\'on de $X$


\begin{Def}
Se define el proceso estacionario, $\left\{V^{*}\left(t\right),t\geq0\right\}$, para $\left\{V\left(t\right),t\geq0\right\}$ por

\begin{eqnarray*}
\prob\left\{V\left(t\right)\in A\right\}=\frac{1}{\esp\left[X\right]}\int_{0}^{\infty}\prob\left\{V\left(t+x\right)\in A|X>x\right\}\left(1-F\left(x\right)\right)dx,
\end{eqnarray*} 
para todo $t\geq0$ y todo conjunto de Borel $A$.
\end{Def}

\begin{Def}
Una distribuci\'on se dice que es {\emph{aritm\'etica}} si todos sus puntos de incremento son m\'ultiplos de la forma $0,\lambda, 2\lambda,\ldots$ para alguna $\lambda>0$ entera.
\end{Def}


\begin{Def}
Una modificaci\'on medible de un proceso $\left\{V\left(t\right),t\geq0\right\}$, es una versi\'on de este, $\left\{V\left(t,w\right)\right\}$ conjuntamente medible para $t\geq0$ y para $w\in S$, $S$ espacio de estados para $\left\{V\left(t\right),t\geq0\right\}$.
\end{Def}

\begin{Teo}
Sea $\left\{V\left(t\right),t\geq\right\}$ un proceso regenerativo no negativo con modificaci\'on medible. Sea $\esp\left[X\right]<\infty$. Entonces el proceso estacionario dado por la ecuaci\'on anterior est\'a bien definido y tiene funci\'on de distribuci\'on independiente de $t$, adem\'as
\begin{itemize}
\item[i)] \begin{eqnarray*}
\esp\left[V^{*}\left(0\right)\right]&=&\frac{\esp\left[\int_{0}^{X}V\left(s\right)ds\right]}{\esp\left[X\right]}\end{eqnarray*}
\item[ii)] Si $\esp\left[V^{*}\left(0\right)\right]<\infty$, equivalentemente, si $\esp\left[\int_{0}^{X}V\left(s\right)ds\right]<\infty$,entonces
\begin{eqnarray*}
\frac{\int_{0}^{t}V\left(s\right)ds}{t}\rightarrow\frac{\esp\left[\int_{0}^{X}V\left(s\right)ds\right]}{\esp\left[X\right]}
\end{eqnarray*}
con probabilidad 1 y en media, cuando $t\rightarrow\infty$.
\end{itemize}
\end{Teo}
%
%___________________________________________________________________________________________
%\vspace{5.5cm}
%\chapter{Cadenas de Markov estacionarias}
%\vspace{-1.0cm}


%__________________________________________________________________________________________
\subsection{Procesos Regenerativos Estacionarios - Stidham \cite{Stidham}}
%__________________________________________________________________________________________


Un proceso estoc\'astico a tiempo continuo $\left\{V\left(t\right),t\geq0\right\}$ es un proceso regenerativo si existe una sucesi\'on de variables aleatorias independientes e id\'enticamente distribuidas $\left\{X_{1},X_{2},\ldots\right\}$, sucesi\'on de renovaci\'on, tal que para cualquier conjunto de Borel $A$, 

\begin{eqnarray*}
\prob\left\{V\left(t\right)\in A|X_{1}+X_{2}+\cdots+X_{R\left(t\right)}=s,\left\{V\left(\tau\right),\tau<s\right\}\right\}=\prob\left\{V\left(t-s\right)\in A|X_{1}>t-s\right\},
\end{eqnarray*}
para todo $0\leq s\leq t$, donde $R\left(t\right)=\max\left\{X_{1}+X_{2}+\cdots+X_{j}\leq t\right\}=$n\'umero de renovaciones ({\emph{puntos de regeneraci\'on}}) que ocurren en $\left[0,t\right]$. El intervalo $\left[0,X_{1}\right)$ es llamado {\emph{primer ciclo de regeneraci\'on}} de $\left\{V\left(t \right),t\geq0\right\}$, $\left[X_{1},X_{1}+X_{2}\right)$ el {\emph{segundo ciclo de regeneraci\'on}}, y as\'i sucesivamente.

Sea $X=X_{1}$ y sea $F$ la funci\'on de distrbuci\'on de $X$


\begin{Def}
Se define el proceso estacionario, $\left\{V^{*}\left(t\right),t\geq0\right\}$, para $\left\{V\left(t\right),t\geq0\right\}$ por

\begin{eqnarray*}
\prob\left\{V\left(t\right)\in A\right\}=\frac{1}{\esp\left[X\right]}\int_{0}^{\infty}\prob\left\{V\left(t+x\right)\in A|X>x\right\}\left(1-F\left(x\right)\right)dx,
\end{eqnarray*} 
para todo $t\geq0$ y todo conjunto de Borel $A$.
\end{Def}

\begin{Def}
Una distribuci\'on se dice que es {\emph{aritm\'etica}} si todos sus puntos de incremento son m\'ultiplos de la forma $0,\lambda, 2\lambda,\ldots$ para alguna $\lambda>0$ entera.
\end{Def}


\begin{Def}
Una modificaci\'on medible de un proceso $\left\{V\left(t\right),t\geq0\right\}$, es una versi\'on de este, $\left\{V\left(t,w\right)\right\}$ conjuntamente medible para $t\geq0$ y para $w\in S$, $S$ espacio de estados para $\left\{V\left(t\right),t\geq0\right\}$.
\end{Def}

\begin{Teo}
Sea $\left\{V\left(t\right),t\geq\right\}$ un proceso regenerativo no negativo con modificaci\'on medible. Sea $\esp\left[X\right]<\infty$. Entonces el proceso estacionario dado por la ecuaci\'on anterior est\'a bien definido y tiene funci\'on de distribuci\'on independiente de $t$, adem\'as
\begin{itemize}
\item[i)] \begin{eqnarray*}
\esp\left[V^{*}\left(0\right)\right]&=&\frac{\esp\left[\int_{0}^{X}V\left(s\right)ds\right]}{\esp\left[X\right]}\end{eqnarray*}
\item[ii)] Si $\esp\left[V^{*}\left(0\right)\right]<\infty$, equivalentemente, si $\esp\left[\int_{0}^{X}V\left(s\right)ds\right]<\infty$,entonces
\begin{eqnarray*}
\frac{\int_{0}^{t}V\left(s\right)ds}{t}\rightarrow\frac{\esp\left[\int_{0}^{X}V\left(s\right)ds\right]}{\esp\left[X\right]}
\end{eqnarray*}
con probabilidad 1 y en media, cuando $t\rightarrow\infty$.
\end{itemize}
\end{Teo}

Para $\left\{X\left(t\right):t\geq0\right\}$ Proceso Estoc\'astico a tiempo continuo con estado de espacios $S$, que es un espacio m\'etrico, con trayectorias continuas por la derecha y con l\'imites por la izquierda c.s. Sea $N\left(t\right)$ un proceso de renovaci\'on en $\rea_{+}$ definido en el mismo espacio de probabilidad que $X\left(t\right)$, con tiempos de renovaci\'on $T$ y tiempos de inter-renovaci\'on $\xi_{n}=T_{n}-T_{n-1}$, con misma distribuci\'on $F$ de media finita $\mu$.



%________________________________________________________________________
\subsection{Procesos Regenerativos}
%________________________________________________________________________

%________________________________________________________________________
\subsection{Procesos Regenerativos Sigman, Thorisson y Wolff \cite{Sigman1}}
%________________________________________________________________________


\begin{Def}[Definici\'on Cl\'asica]
Un proceso estoc\'astico $X=\left\{X\left(t\right):t\geq0\right\}$ es llamado regenerativo is existe una variable aleatoria $R_{1}>0$ tal que
\begin{itemize}
\item[i)] $\left\{X\left(t+R_{1}\right):t\geq0\right\}$ es independiente de $\left\{\left\{X\left(t\right):t<R_{1}\right\},\right\}$
\item[ii)] $\left\{X\left(t+R_{1}\right):t\geq0\right\}$ es estoc\'asticamente equivalente a $\left\{X\left(t\right):t>0\right\}$
\end{itemize}

Llamamos a $R_{1}$ tiempo de regeneraci\'on, y decimos que $X$ se regenera en este punto.
\end{Def}

$\left\{X\left(t+R_{1}\right)\right\}$ es regenerativo con tiempo de regeneraci\'on $R_{2}$, independiente de $R_{1}$ pero con la misma distribuci\'on que $R_{1}$. Procediendo de esta manera se obtiene una secuencia de variables aleatorias independientes e id\'enticamente distribuidas $\left\{R_{n}\right\}$ llamados longitudes de ciclo. Si definimos a $Z_{k}\equiv R_{1}+R_{2}+\cdots+R_{k}$, se tiene un proceso de renovaci\'on llamado proceso de renovaci\'on encajado para $X$.


\begin{Note}
La existencia de un primer tiempo de regeneraci\'on, $R_{1}$, implica la existencia de una sucesi\'on completa de estos tiempos $R_{1},R_{2}\ldots,$ que satisfacen la propiedad deseada \cite{Sigman2}.
\end{Note}


\begin{Note} Para la cola $GI/GI/1$ los usuarios arriban con tiempos $t_{n}$ y son atendidos con tiempos de servicio $S_{n}$, los tiempos de arribo forman un proceso de renovaci\'on  con tiempos entre arribos independientes e identicamente distribuidos (\texttt{i.i.d.})$T_{n}=t_{n}-t_{n-1}$, adem\'as los tiempos de servicio son \texttt{i.i.d.} e independientes de los procesos de arribo. Por \textit{estable} se entiende que $\esp S_{n}<\esp T_{n}<\infty$.
\end{Note}
 


\begin{Def}
Para $x$ fijo y para cada $t\geq0$, sea $I_{x}\left(t\right)=1$ si $X\left(t\right)\leq x$,  $I_{x}\left(t\right)=0$ en caso contrario, y def\'inanse los tiempos promedio
\begin{eqnarray*}
\overline{X}&=&lim_{t\rightarrow\infty}\frac{1}{t}\int_{0}^{\infty}X\left(u\right)du\\
\prob\left(X_{\infty}\leq x\right)&=&lim_{t\rightarrow\infty}\frac{1}{t}\int_{0}^{\infty}I_{x}\left(u\right)du,
\end{eqnarray*}
cuando estos l\'imites existan.
\end{Def}

Como consecuencia del teorema de Renovaci\'on-Recompensa, se tiene que el primer l\'imite  existe y es igual a la constante
\begin{eqnarray*}
\overline{X}&=&\frac{\esp\left[\int_{0}^{R_{1}}X\left(t\right)dt\right]}{\esp\left[R_{1}\right]},
\end{eqnarray*}
suponiendo que ambas esperanzas son finitas.
 
\begin{Note}
Funciones de procesos regenerativos son regenerativas, es decir, si $X\left(t\right)$ es regenerativo y se define el proceso $Y\left(t\right)$ por $Y\left(t\right)=f\left(X\left(t\right)\right)$ para alguna funci\'on Borel medible $f\left(\cdot\right)$. Adem\'as $Y$ es regenerativo con los mismos tiempos de renovaci\'on que $X$. 

En general, los tiempos de renovaci\'on, $Z_{k}$ de un proceso regenerativo no requieren ser tiempos de paro con respecto a la evoluci\'on de $X\left(t\right)$.
\end{Note} 

\begin{Note}
Una funci\'on de un proceso de Markov, usualmente no ser\'a un proceso de Markov, sin embargo ser\'a regenerativo si el proceso de Markov lo es.
\end{Note}

 
\begin{Note}
Un proceso regenerativo con media de la longitud de ciclo finita es llamado positivo recurrente.
\end{Note}


\begin{Note}
\begin{itemize}
\item[a)] Si el proceso regenerativo $X$ es positivo recurrente y tiene trayectorias muestrales no negativas, entonces la ecuaci\'on anterior es v\'alida.
\item[b)] Si $X$ es positivo recurrente regenerativo, podemos construir una \'unica versi\'on estacionaria de este proceso, $X_{e}=\left\{X_{e}\left(t\right)\right\}$, donde $X_{e}$ es un proceso estoc\'astico regenerativo y estrictamente estacionario, con distribuci\'on marginal distribuida como $X_{\infty}$
\end{itemize}
\end{Note}


%__________________________________________________________________________________________
\subsection{Procesos Regenerativos Estacionarios - Stidham \cite{Stidham}}
%__________________________________________________________________________________________


Un proceso estoc\'astico a tiempo continuo $\left\{V\left(t\right),t\geq0\right\}$ es un proceso regenerativo si existe una sucesi\'on de variables aleatorias independientes e id\'enticamente distribuidas $\left\{X_{1},X_{2},\ldots\right\}$, sucesi\'on de renovaci\'on, tal que para cualquier conjunto de Borel $A$, 

\begin{eqnarray*}
\prob\left\{V\left(t\right)\in A|X_{1}+X_{2}+\cdots+X_{R\left(t\right)}=s,\left\{V\left(\tau\right),\tau<s\right\}\right\}=\prob\left\{V\left(t-s\right)\in A|X_{1}>t-s\right\},
\end{eqnarray*}
para todo $0\leq s\leq t$, donde $R\left(t\right)=\max\left\{X_{1}+X_{2}+\cdots+X_{j}\leq t\right\}=$n\'umero de renovaciones ({\emph{puntos de regeneraci\'on}}) que ocurren en $\left[0,t\right]$. El intervalo $\left[0,X_{1}\right)$ es llamado {\emph{primer ciclo de regeneraci\'on}} de $\left\{V\left(t \right),t\geq0\right\}$, $\left[X_{1},X_{1}+X_{2}\right)$ el {\emph{segundo ciclo de regeneraci\'on}}, y as\'i sucesivamente.

Sea $X=X_{1}$ y sea $F$ la funci\'on de distrbuci\'on de $X$


\begin{Def}
Se define el proceso estacionario, $\left\{V^{*}\left(t\right),t\geq0\right\}$, para $\left\{V\left(t\right),t\geq0\right\}$ por

\begin{eqnarray*}
\prob\left\{V\left(t\right)\in A\right\}=\frac{1}{\esp\left[X\right]}\int_{0}^{\infty}\prob\left\{V\left(t+x\right)\in A|X>x\right\}\left(1-F\left(x\right)\right)dx,
\end{eqnarray*} 
para todo $t\geq0$ y todo conjunto de Borel $A$.
\end{Def}

\begin{Def}
Una distribuci\'on se dice que es {\emph{aritm\'etica}} si todos sus puntos de incremento son m\'ultiplos de la forma $0,\lambda, 2\lambda,\ldots$ para alguna $\lambda>0$ entera.
\end{Def}


\begin{Def}
Una modificaci\'on medible de un proceso $\left\{V\left(t\right),t\geq0\right\}$, es una versi\'on de este, $\left\{V\left(t,w\right)\right\}$ conjuntamente medible para $t\geq0$ y para $w\in S$, $S$ espacio de estados para $\left\{V\left(t\right),t\geq0\right\}$.
\end{Def}

\begin{Teo}
Sea $\left\{V\left(t\right),t\geq\right\}$ un proceso regenerativo no negativo con modificaci\'on medible. Sea $\esp\left[X\right]<\infty$. Entonces el proceso estacionario dado por la ecuaci\'on anterior est\'a bien definido y tiene funci\'on de distribuci\'on independiente de $t$, adem\'as
\begin{itemize}
\item[i)] \begin{eqnarray*}
\esp\left[V^{*}\left(0\right)\right]&=&\frac{\esp\left[\int_{0}^{X}V\left(s\right)ds\right]}{\esp\left[X\right]}\end{eqnarray*}
\item[ii)] Si $\esp\left[V^{*}\left(0\right)\right]<\infty$, equivalentemente, si $\esp\left[\int_{0}^{X}V\left(s\right)ds\right]<\infty$,entonces
\begin{eqnarray*}
\frac{\int_{0}^{t}V\left(s\right)ds}{t}\rightarrow\frac{\esp\left[\int_{0}^{X}V\left(s\right)ds\right]}{\esp\left[X\right]}
\end{eqnarray*}
con probabilidad 1 y en media, cuando $t\rightarrow\infty$.
\end{itemize}
\end{Teo}

\begin{Coro}
Sea $\left\{V\left(t\right),t\geq0\right\}$ un proceso regenerativo no negativo, con modificaci\'on medible. Si $\esp <\infty$, $F$ es no-aritm\'etica, y para todo $x\geq0$, $P\left\{V\left(t\right)\leq x,C>x\right\}$ es de variaci\'on acotada como funci\'on de $t$ en cada intervalo finito $\left[0,\tau\right]$, entonces $V\left(t\right)$ converge en distribuci\'on  cuando $t\rightarrow\infty$ y $$\esp V=\frac{\esp \int_{0}^{X}V\left(s\right)ds}{\esp X}$$
Donde $V$ tiene la distribuci\'on l\'imite de $V\left(t\right)$ cuando $t\rightarrow\infty$.

\end{Coro}

Para el caso discreto se tienen resultados similares.




%__________________________________________________________________________________________
\subsection{Procesos Regenerativos Estacionarios - Stidham \cite{Stidham}}
%__________________________________________________________________________________________


Un proceso estoc\'astico a tiempo continuo $\left\{V\left(t\right),t\geq0\right\}$ es un proceso regenerativo si existe una sucesi\'on de variables aleatorias independientes e id\'enticamente distribuidas $\left\{X_{1},X_{2},\ldots\right\}$, sucesi\'on de renovaci\'on, tal que para cualquier conjunto de Borel $A$, 

\begin{eqnarray*}
\prob\left\{V\left(t\right)\in A|X_{1}+X_{2}+\cdots+X_{R\left(t\right)}=s,\left\{V\left(\tau\right),\tau<s\right\}\right\}=\prob\left\{V\left(t-s\right)\in A|X_{1}>t-s\right\},
\end{eqnarray*}
para todo $0\leq s\leq t$, donde $R\left(t\right)=\max\left\{X_{1}+X_{2}+\cdots+X_{j}\leq t\right\}=$n\'umero de renovaciones ({\emph{puntos de regeneraci\'on}}) que ocurren en $\left[0,t\right]$. El intervalo $\left[0,X_{1}\right)$ es llamado {\emph{primer ciclo de regeneraci\'on}} de $\left\{V\left(t \right),t\geq0\right\}$, $\left[X_{1},X_{1}+X_{2}\right)$ el {\emph{segundo ciclo de regeneraci\'on}}, y as\'i sucesivamente.

Sea $X=X_{1}$ y sea $F$ la funci\'on de distrbuci\'on de $X$


\begin{Def}
Se define el proceso estacionario, $\left\{V^{*}\left(t\right),t\geq0\right\}$, para $\left\{V\left(t\right),t\geq0\right\}$ por

\begin{eqnarray*}
\prob\left\{V\left(t\right)\in A\right\}=\frac{1}{\esp\left[X\right]}\int_{0}^{\infty}\prob\left\{V\left(t+x\right)\in A|X>x\right\}\left(1-F\left(x\right)\right)dx,
\end{eqnarray*} 
para todo $t\geq0$ y todo conjunto de Borel $A$.
\end{Def}

\begin{Def}
Una distribuci\'on se dice que es {\emph{aritm\'etica}} si todos sus puntos de incremento son m\'ultiplos de la forma $0,\lambda, 2\lambda,\ldots$ para alguna $\lambda>0$ entera.
\end{Def}


\begin{Def}
Una modificaci\'on medible de un proceso $\left\{V\left(t\right),t\geq0\right\}$, es una versi\'on de este, $\left\{V\left(t,w\right)\right\}$ conjuntamente medible para $t\geq0$ y para $w\in S$, $S$ espacio de estados para $\left\{V\left(t\right),t\geq0\right\}$.
\end{Def}

\begin{Teo}
Sea $\left\{V\left(t\right),t\geq\right\}$ un proceso regenerativo no negativo con modificaci\'on medible. Sea $\esp\left[X\right]<\infty$. Entonces el proceso estacionario dado por la ecuaci\'on anterior est\'a bien definido y tiene funci\'on de distribuci\'on independiente de $t$, adem\'as
\begin{itemize}
\item[i)] \begin{eqnarray*}
\esp\left[V^{*}\left(0\right)\right]&=&\frac{\esp\left[\int_{0}^{X}V\left(s\right)ds\right]}{\esp\left[X\right]}\end{eqnarray*}
\item[ii)] Si $\esp\left[V^{*}\left(0\right)\right]<\infty$, equivalentemente, si $\esp\left[\int_{0}^{X}V\left(s\right)ds\right]<\infty$,entonces
\begin{eqnarray*}
\frac{\int_{0}^{t}V\left(s\right)ds}{t}\rightarrow\frac{\esp\left[\int_{0}^{X}V\left(s\right)ds\right]}{\esp\left[X\right]}
\end{eqnarray*}
con probabilidad 1 y en media, cuando $t\rightarrow\infty$.
\end{itemize}
\end{Teo}

\begin{Coro}
Sea $\left\{V\left(t\right),t\geq0\right\}$ un proceso regenerativo no negativo, con modificaci\'on medible. Si $\esp <\infty$, $F$ es no-aritm\'etica, y para todo $x\geq0$, $P\left\{V\left(t\right)\leq x,C>x\right\}$ es de variaci\'on acotada como funci\'on de $t$ en cada intervalo finito $\left[0,\tau\right]$, entonces $V\left(t\right)$ converge en distribuci\'on  cuando $t\rightarrow\infty$ y $$\esp V=\frac{\esp \int_{0}^{X}V\left(s\right)ds}{\esp X}$$
Donde $V$ tiene la distribuci\'on l\'imite de $V\left(t\right)$ cuando $t\rightarrow\infty$.

\end{Coro}

Para el caso discreto se tienen resultados similares.



\chapter{Teorema de Down}
\section{Teorema de Down}



En este ap\'endice enunciaremos una serie de resultados que son
necesarios para la demostraci\'on as\'i como su demostraci\'on del
Teorema de Down \ref{Tma2.1.Down}, adem\'as de un teorema
referente a las propiedades que cumple el Modelo de Flujo.\\


Dado el proceso $X=\left\{X\left(t\right),t\geq0\right\}$ definido
en (\ref{Esp.Edos.Down}) que describe la din\'amica del sistema de
visitas c\'iclicas, si $U\left(t\right)$ es el residual de los
tiempos de llegada al tiempo $t$ entre dos usuarios consecutivos y
$V\left(t\right)$ es el residual de los tiempos de servicio al
tiempo $t$ para el usuario que est\'as siendo atendido por el
servidor. Sea $\mathbb{X}$ el espacio de estados que puede tomar
el proceso $X$.


\begin{Lema}[Lema 4.3, Dai\cite{Dai}]\label{Lema.4.3}
Sea $\left\{x_{n}\right\}\subset \mathbf{X}$ con
$|x_{n}|\rightarrow\infty$, conforme $n\rightarrow\infty$. Suponga
que
\[lim_{n\rightarrow\infty}\frac{1}{|x_{n}|}U\left(0\right)=\overline{U}_{k},\]
y
\[lim_{n\rightarrow\infty}\frac{1}{|x_{n}|}V\left(0\right)=\overline{V}_{k}.\]
\begin{itemize}
\item[a)] Conforme $n\rightarrow\infty$ casi seguramente,
\[lim_{n\rightarrow\infty}\frac{1}{|x_{n}|}U^{x_{n}}_{k}\left(|x_{n}|t\right)=\left(\overline{U}_{k}-t\right)^{+}\textrm{, u.o.c.}\]
y
\[lim_{n\rightarrow\infty}\frac{1}{|x_{n}|}V^{x_{n}}_{k}\left(|x_{n}|t\right)=\left(\overline{V}_{k}-t\right)^{+}.\]

\item[b)] Para cada $t\geq0$ fijo,
\[\left\{\frac{1}{|x_{n}|}U^{x_{n}}_{k}\left(|x_{n}|t\right),|x_{n}|\geq1\right\}\]
y
\[\left\{\frac{1}{|x_{n}|}V^{x_{n}}_{k}\left(|x_{n}|t\right),|x_{n}|\geq1\right\}\]
\end{itemize}
son uniformemente convergentes.
\end{Lema}

Sea $e$ es un vector de unos, $C$ es la matriz definida por
\[C_{ik}=\left\{\begin{array}{cc}
1,& S\left(k\right)=i,\\
0,& \textrm{ en otro caso}.\\
\end{array}\right.
\]
Es necesario enunciar el siguiente Teorema que se utilizar\'a para
el Teorema (\ref{Tma.4.2.Dai}):
\begin{Teo}[Teorema 4.1, Dai \cite{Dai}]
Considere una disciplina que cumpla la ley de conservaci\'on, para
casi todas las trayectorias muestrales $\omega$ y cualquier
sucesi\'on de estados iniciales $\left\{x_{n}\right\}\subset
\mathbf{X}$, con $|x_{n}|\rightarrow\infty$, existe una
subsucesi\'on $\left\{x_{n_{j}}\right\}$ con
$|x_{n_{j}}|\rightarrow\infty$ tal que
\begin{equation}\label{Eq.4.15}
\frac{1}{|x_{n_{j}}|}\left(Q^{x_{n_{j}}}\left(0\right),U^{x_{n_{j}}}\left(0\right),V^{x_{n_{j}}}\left(0\right)\right)\rightarrow\left(\overline{Q}\left(0\right),\overline{U},\overline{V}\right),
\end{equation}

\begin{equation}\label{Eq.4.16}
\frac{1}{|x_{n_{j}}|}\left(Q^{x_{n_{j}}}\left(|x_{n_{j}}|t\right),T^{x_{n_{j}}}\left(|x_{n_{j}}|t\right)\right)\rightarrow\left(\overline{Q}\left(t\right),\overline{T}\left(t\right)\right)\textrm{
u.o.c.}
\end{equation}

Adem\'as,
$\left(\overline{Q}\left(t\right),\overline{T}\left(t\right)\right)$
satisface las siguientes ecuaciones:
\begin{equation}\label{Eq.MF.1.3a}
\overline{Q}\left(t\right)=Q\left(0\right)+\left(\alpha
t-\overline{U}\right)^{+}-\left(I-P\right)^{'}M^{-1}\left(\overline{T}\left(t\right)-\overline{V}\right)^{+},
\end{equation}

\begin{equation}\label{Eq.MF.2.3a}
\overline{Q}\left(t\right)\geq0,\\
\end{equation}

\begin{equation}\label{Eq.MF.3.3a}
\overline{T}\left(t\right)\textrm{ es no decreciente y comienza en cero},\\
\end{equation}

\begin{equation}\label{Eq.MF.4.3a}
\overline{I}\left(t\right)=et-C\overline{T}\left(t\right)\textrm{
es no decreciente,}\\
\end{equation}

\begin{equation}\label{Eq.MF.5.3a}
\int_{0}^{\infty}\left(C\overline{Q}\left(t\right)\right)d\overline{I}\left(t\right)=0,\\
\end{equation}

\begin{equation}\label{Eq.MF.6.3a}
\textrm{Condiciones en
}\left(\overline{Q}\left(\cdot\right),\overline{T}\left(\cdot\right)\right)\textrm{
espec\'ificas de la disciplina de la cola,}
\end{equation}
\end{Teo}


Propiedades importantes para el modelo de flujo retrasado:

\begin{Prop}[Proposici\'on 4.2, Dai \cite{Dai}]
 Sea $\left(\overline{Q},\overline{T},\overline{T}^{0}\right)$ un flujo l\'imite de \ref{Eq.Punto.Limite}
 y suponga que cuando $x\rightarrow\infty$ a lo largo de una subsucesi\'on
\[\left(\frac{1}{|x|}Q_{k}^{x}\left(0\right),\frac{1}{|x|}A_{k}^{x}\left(0\right),\frac{1}{|x|}B_{k}^{x}\left(0\right),\frac{1}{|x|}B_{k}^{x,0}\left(0\right)\right)\rightarrow\left(\overline{Q}_{k}\left(0\right),0,0,0\right)\]
para $k=1,\ldots,K$. El flujo l\'imite tiene las siguientes
propiedades, donde las propiedades de la derivada se cumplen donde
la derivada exista:
\begin{itemize}
 \item[i)] Los vectores de tiempo ocupado $\overline{T}\left(t\right)$ y $\overline{T}^{0}\left(t\right)$ son crecientes y continuas con
$\overline{T}\left(0\right)=\overline{T}^{0}\left(0\right)=0$.
\item[ii)] Para todo $t\geq0$
\[\sum_{k=1}^{K}\left[\overline{T}_{k}\left(t\right)+\overline{T}_{k}^{0}\left(t\right)\right]=t.\]
\item[iii)] Para todo $1\leq k\leq K$
\[\overline{Q}_{k}\left(t\right)=\overline{Q}_{k}\left(0\right)+\alpha_{k}t-\mu_{k}\overline{T}_{k}\left(t\right).\]
\item[iv)]  Para todo $1\leq k\leq K$
\[\dot{{\overline{T}}}_{k}\left(t\right)=\rho_{k}\] para $\overline{Q}_{k}\left(t\right)=0$.
\item[v)] Para todo $k,j$
\[\mu_{k}^{0}\overline{T}_{k}^{0}\left(t\right)=\mu_{j}^{0}\overline{T}_{j}^{0}\left(t\right).\]
\item[vi)]  Para todo $1\leq k\leq K$
\[\mu_{k}\dot{{\overline{T}}}_{k}\left(t\right)=l_{k}\mu_{k}^{0}\dot{{\overline{T}}}_{k}^{0}\left(t\right),\] para $\overline{Q}_{k}\left(t\right)>0$.
\end{itemize}
\end{Prop}

\begin{Lema}[Lema 3.1, Chen \cite{Chen}]\label{Lema3.1}
Si el modelo de flujo es estable, definido por las ecuaciones
(3.8)-(3.13), entonces el modelo de flujo retrasado tambi\'en es
estable.
\end{Lema}

\begin{Lema}[Lema 5.2, Gut \cite{Gut}]\label{Lema.5.2.Gut}
Sea $\left\{\xi\left(k\right):k\in\ent\right\}$ sucesi\'on de
variables aleatorias i.i.d. con valores en
$\left(0,\infty\right)$, y sea $E\left(t\right)$ el proceso de
conteo
\[E\left(t\right)=max\left\{n\geq1:\xi\left(1\right)+\cdots+\xi\left(n-1\right)\leq t\right\}.\]
Si $E\left[\xi\left(1\right)\right]<\infty$, entonces para
cualquier entero $r\geq1$
\begin{equation}
lim_{t\rightarrow\infty}\esp\left[\left(\frac{E\left(t\right)}{t}\right)^{r}\right]=\left(\frac{1}{E\left[\xi_{1}\right]}\right)^{r},
\end{equation}
de aqu\'i, bajo estas condiciones
\begin{itemize}
\item[a)] Para cualquier $t>0$,
$sup_{t\geq\delta}\esp\left[\left(\frac{E\left(t\right)}{t}\right)^{r}\right]<\infty$.

\item[b)] Las variables aleatorias
$\left\{\left(\frac{E\left(t\right)}{t}\right)^{r}:t\geq1\right\}$
son uniformemente integrables.
\end{itemize}
\end{Lema}

\begin{Teo}[Teorema 5.1: Ley Fuerte para Procesos de Conteo, Gut
\cite{Gut}]\label{Tma.5.1.Gut} Sea
$0<\mu<\esp\left(X_{1}\right]\leq\infty$. entonces

\begin{itemize}
\item[a)] $\frac{N\left(t\right)}{t}\rightarrow\frac{1}{\mu}$
a.s., cuando $t\rightarrow\infty$.


\item[b)]$\esp\left[\frac{N\left(t\right)}{t}\right]^{r}\rightarrow\frac{1}{\mu^{r}}$,
cuando $t\rightarrow\infty$ para todo $r>0$.
\end{itemize}
\end{Teo}


\begin{Prop}[Proposici\'on 5.1, Dai y Sean \cite{DaiSean}]\label{Prop.5.1}
Suponga que los supuestos (A1) y (A2) se cumplen, adem\'as suponga
que el modelo de flujo es estable. Entonces existe $t_{0}>0$ tal
que
\begin{equation}\label{Eq.Prop.5.1}
lim_{|x|\rightarrow\infty}\frac{1}{|x|^{p+1}}\esp_{x}\left[|X\left(t_{0}|x|\right)|^{p+1}\right]=0.
\end{equation}

\end{Prop}


\begin{Prop}[Proposici\'on 5.3, Dai y Sean \cite{DaiSean}]\label{Prop.5.3.DaiSean}
Sea $X$ proceso de estados para la red de colas, y suponga que se
cumplen los supuestos (A1) y (A2), entonces para alguna constante
positiva $C_{p+1}<\infty$, $\delta>0$ y un conjunto compacto
$C\subset X$.

\begin{equation}\label{Eq.5.4}
\esp_{x}\left[\int_{0}^{\tau_{C}\left(\delta\right)}\left(1+|X\left(t\right)|^{p}\right)dt\right]\leq
C_{p+1}\left(1+|x|^{p+1}\right).
\end{equation}
\end{Prop}

\begin{Prop}[Proposici\'on 5.4, Dai y Sean \cite{DaiSean}]\label{Prop.5.4.DaiSean}
Sea $X$ un proceso de Markov Borel Derecho en $X$, sea
$f:X\leftarrow\rea_{+}$ y defina para alguna $\delta>0$, y un
conjunto cerrado $C\subset X$
\[V\left(x\right):=\esp_{x}\left[\int_{0}^{\tau_{C}\left(\delta\right)}f\left(X\left(t\right)\right)dt\right],\]
para $x\in X$. Si $V$ es finito en todas partes y uniformemente
acotada en $C$, entonces existe $k<\infty$ tal que
\begin{equation}\label{Eq.5.11}
\frac{1}{t}\esp_{x}\left[V\left(x\right)\right]+\frac{1}{t}\int_{0}^{t}\esp_{x}\left[f\left(X\left(s\right)\right)ds\right]\leq\frac{1}{t}V\left(x\right)+k,
\end{equation}
para $x\in X$ y $t>0$.
\end{Prop}


\begin{Teo}[Teorema 5.5, Dai y Sean  \cite{DaiSean}]
Suponga que se cumplen (A1) y (A2), adem\'as suponga que el modelo
de flujo es estable. Entonces existe una constante $k_{p}<\infty$
tal que
\begin{equation}\label{Eq.5.13}
\frac{1}{t}\int_{0}^{t}\esp_{x}\left[|Q\left(s\right)|^{p}\right]ds\leq
k_{p}\left\{\frac{1}{t}|x|^{p+1}+1\right\},
\end{equation}
para $t\geq0$, $x\in X$. En particular para cada condici\'on
inicial
\begin{equation}\label{Eq.5.14}
\limsup_{t\rightarrow\infty}\frac{1}{t}\int_{0}^{t}\esp_{x}\left[|Q\left(s\right)|^{p}\right]ds\leq
k_{p}.
\end{equation}
\end{Teo}

\begin{Teo}[Teorema 6.2 Dai y Sean \cite{DaiSean}]\label{Tma.6.2}
Suponga que se cumplen los supuestos (A1)-(A3) y que el modelo de
flujo es estable, entonces se tiene que
\[\parallel P^{t}\left(x,\cdot\right)-\pi\left(\cdot\right)\parallel_{f_{p}}\rightarrow0,\]
para $t\rightarrow\infty$ y $x\in X$. En particular para cada
condici\'on inicial
\[lim_{t\rightarrow\infty}\esp_{x}\left[\left|Q_{t}\right|^{p}\right]=\esp_{\pi}\left[\left|Q_{0}\right|^{p}\right]<\infty,\]
\end{Teo}

donde

\begin{eqnarray*}
\parallel
P^{t}\left(c,\cdot\right)-\pi\left(\cdot\right)\parallel_{f}=sup_{|g\leq
f|}|\int\pi\left(dy\right)g\left(y\right)-\int
P^{t}\left(x,dy\right)g\left(y\right)|,
\end{eqnarray*}
para $x\in\mathbb{X}$.

\begin{Teo}[Teorema 6.3, Dai y Sean \cite{DaiSean}]\label{Tma.6.3}
Suponga que se cumplen los supuestos (A1)-(A3) y que el modelo de
flujo es estable, entonces con
$f\left(x\right)=f_{1}\left(x\right)$, se tiene que
\[lim_{t\rightarrow\infty}t^{(p-1)}\left|P^{t}\left(c,\cdot\right)-\pi\left(\cdot\right)\right|_{f}=0,\]
para $x\in X$. En particular, para cada condici\'on inicial
\[lim_{t\rightarrow\infty}t^{(p-1)}\left|\esp_{x}\left[Q_{t}\right]-\esp_{\pi}\left[Q_{0}\right]\right|=0.\]
\end{Teo}



\begin{Prop}[Proposici\'on 5.1, Dai y Meyn \cite{DaiSean}]\label{Prop.5.1.DaiSean}
Suponga que los supuestos A1) y A2) son ciertos y que el modelo de
flujo es estable. Entonces existe $t_{0}>0$ tal que
\begin{equation}
lim_{|x|\rightarrow\infty}\frac{1}{|x|^{p+1}}\esp_{x}\left[|X\left(t_{0}|x|\right)|^{p+1}\right]=0.
\end{equation}
\end{Prop}


\begin{Teo}[Teorema 5.5, Dai y Meyn \cite{DaiSean}]\label{Tma.5.5.DaiSean}
Suponga que los supuestos A1) y A2) se cumplen y que el modelo de
flujo es estable. Entonces existe una constante $\kappa_{p}$ tal
que
\begin{equation}
\frac{1}{t}\int_{0}^{t}\esp_{x}\left[|Q\left(s\right)|^{p}\right]ds\leq\kappa_{p}\left\{\frac{1}{t}|x|^{p+1}+1\right\},
\end{equation}
para $t>0$ y $x\in X$. En particular, para cada condici\'on
inicial
\begin{eqnarray*}
\limsup_{t\rightarrow\infty}\frac{1}{t}\int_{0}^{t}\esp_{x}\left[|Q\left(s\right)|^{p}\right]ds\leq\kappa_{p}.
\end{eqnarray*}
\end{Teo}


\begin{Teo}[Teorema 6.4, Dai y Meyn \cite{DaiSean}]\label{Tma.6.4.DaiSean}
Suponga que se cumplen los supuestos A1), A2) y A3) y que el
modelo de flujo es estable. Sea $\nu$ cualquier distribuci\'on de
probabilidad en
$\left(\mathbb{X},\mathcal{B}_{\mathbb{X}}\right)$, y $\pi$ la
distribuci\'on estacionaria de $X$.
\begin{itemize}
\item[i)] Para cualquier $f:X\leftarrow\rea_{+}$
\begin{equation}
\lim_{t\rightarrow\infty}\frac{1}{t}\int_{o}^{t}f\left(X\left(s\right)\right)ds=\pi\left(f\right):=\int
f\left(x\right)\pi\left(dx\right),
\end{equation}
$\prob$-c.s.

\item[ii)] Para cualquier $f:X\leftarrow\rea_{+}$ con
$\pi\left(|f|\right)<\infty$, la ecuaci\'on anterior se cumple.
\end{itemize}
\end{Teo}

\begin{Teo}[Teorema 2.2, Down \cite{Down}]\label{Tma2.2.Down}
Suponga que el fluido modelo es inestable en el sentido de que
para alguna $\epsilon_{0},c_{0}\geq0$,
\begin{equation}\label{Eq.Inestability}
|Q\left(T\right)|\geq\epsilon_{0}T-c_{0}\textrm{,   }T\geq0,
\end{equation}
para cualquier condici\'on inicial $Q\left(0\right)$, con
$|Q\left(0\right)|=1$. Entonces para cualquier $0<q\leq1$, existe
$B<0$ tal que para cualquier $|x|\geq B$,
\begin{equation}
\prob_{x}\left\{\mathbb{X}\rightarrow\infty\right\}\geq q.
\end{equation}
\end{Teo}

\begin{Dem}[Teorema \ref{Tma2.1.Down}] La demostraci\'on de este
teorema se da a continuaci\'on:\\
\begin{itemize}
\item[i)] Utilizando la proposici\'on \ref{Prop.5.3.DaiSean} se
tiene que la proposici\'on \ref{Prop.5.4.DaiSean} es cierta para
$f\left(x\right)=1+|x|^{p}$.

\item[i)] es consecuencia directa del Teorema \ref{Tma.6.2}.

\item[iii)] ver la demostraci\'on dada en Dai y Sean
\cite{DaiSean} p\'aginas 1901-1902.

\item[iv)] ver Dai y Sean \cite{DaiSean} p\'aginas 1902-1903 \'o
\cite{MeynTweedie2}.
\end{itemize}
\end{Dem}
\newpage
%_________________________________________________________________________
%\subsection{AP\'ENDICE B}\label{apend.B}
%_________________________________________________________________________


Con la finalidad de ejemplificar la simulaci\'on y el an\'alisis
num\'erico en los sistemas de visitas c\'iclicas revisaremos el
ejemplo presentado en Roubos \cite{TesisRoubos} donde se presenta
un sistema conformado por tres colas, en las cuales los tiempos de
arribo ocurren conforme a un proceso Poisson con tasas de arribo
$\lambda_{1}=0.3$, $\lambda_{2}=0.4$ y $\lambda_{3}=0.2$ para las
colas 1,2 y 3 respectivamente. Los tiempos de servicio para cada
una de las colas se distribuyen de manera exponencial con media 1,
uniforme sobre el intervalo $\left[0,1\right]$ y gamma con
par\'ametro de forma 1 y de escala 2, respectivamente. Finalmente
se esta considerando que los tiempos de traslado entre las colas
se distribuyen de manera exponencial con media 1, 2 y 3 para ir de
la cola 1 a 2, de 2 a 3 y de 3 a 1, respectivamente.\\

Entonces, conforme a lo descrito en la secci\'on 2.4, los
resultados obtenidos para los tiempos de espera en cada una de las
colas para las pol\'iticas exhaustiva y cerrada considerando diez
mil trayectorias son:

Resultados num\'ericos
\begin{eqnarray*}
\esp W_{1}&\cong&18.71\\
\esp W_{2}&\cong&22.38\\
\esp W_{3}&\cong&15.84\\
\end{eqnarray*}

\begin{center}
\begin{table}[!ht]\caption{{\small Se muestran los resultados obtenidos v\'ia
sumulaci\'on de Monte Carlo para la cola 1 y valores grandes de
$T$ considerando la pol\'itica Exhaustiva}}\label{Exhaustiva}
%\centering
\begin{tabular}{|c||c|c|c||c|c||}
  \hline
  & \multicolumn{5}{c}{Cola 1}\vline \\
  \hline
  $T$ & Lim Inf & $\mu$ & Lim Sup & Var & Error \\\hline
1000 &   15.8083 &  15.9713 &  16.1342 &  0.0831 & 0.0052\\
5000 &   18.0915 &  18.2002 &  18.3090 &  0.0555 & 0.0030\\
10000 &  18.3118 &  18.3926 &  18.4734 &  0.0412 & 0.0020\\
\hline
\end{tabular}\end{table}
\end{center}

\begin{center}
\begin{table}[!ht]\caption{{\small Resultados obtenidos para la cola 2}}\label{Exhaustiva} %\centering
\begin{tabular}{|c||c|c|c||c|c||}
 \hline
  & \multicolumn{5}{c}{Cola 2}\vline \\
\hline
  $T$ & Lim Inf & $\mu$ & Lim Sup & Var & Error \\\hline
1000 &  18.9229 &  19.1150 &  19.3070  &  0.0980 &  0.0051\\
5000 &  21.6480 &  21.7753 &  21.9026  &  0.0649 &  0.0030\\
10000 & 21.9448 &  22.0398 &  22.1348  &  0.0485 &  0.0022\\
\hline \hline
\end{tabular}\end{table}
\end{center}

\begin{center}
\begin{table}[!ht]\caption{{\small Resultados obtenidos para
la cola 3}}\label{Exhaustiva} %\centering
\begin{tabular}{|c||c|c|c||c|c||}
 \hline
  & \multicolumn{5}{c}{Cola 3}\vline \\
\hline
  $T$ & Lim Inf & $\mu$ & Lim Sup & Var & Error \\\hline
1000 &  13.3704 &  13.5026 &  13.6348 & 0.0674 & 0.0050\\
5000 &  15.3073 &  15.3962 &  15.4851 & 0.0454 & 0.0029\\
10000 & 15.5141 &  15.5804 &  15.6467 & 0.0338 & 0.0022\\
\hline
\end{tabular}\end{table}
\end{center}


En la figura \ref{GrafExhaustiva10M} se muestran los tiempos de
espera, los errores relativos para cada una de las colas;

Resultados num\'ericos
\begin{eqnarray*}
\esp W_{1}&\cong&23.51\\
\esp W_{2}&\cong&21.75\\
\esp W_{3}&\cong&25.84\\
\end{eqnarray*}

\begin{center}
\begin{table}[!ht]\caption{{\small Se muestran los resultados obtenidos
 v\'ia sumulaci\'on de Monte Carlo para las cola 1 y valores grandes de
$T$ considerando la pol\'itica Cerrada}}\label{CerradaC1}
%\centering
\begin{tabular}{|c||c|c|c||c|c||}
  \hline
  & \multicolumn{5}{c}{Cola 1}\vline \\
  \hline
  $T$ & Lim Inf & $\mu$ & Lim Sup & Var & Error \\\hline
1000 &  19.4470 &  19.6232 &  19.7994 &   0.0899 &   0.0046\\
5000 &  22.5228 &  22.6496 &  22.7764 &   0.0647 &   0.0029\\
10000 &  23.0238 &  23.1182 &  23.2126 &   0.0482 &   0.00231\\
\hline \hline
\end{tabular}\end{table}
\end{center}

\begin{center}
\begin{table}[!ht]\caption{{\small Resultados obtenidos para
la cola 2}}\label{CerradaC2} %\centering
\begin{tabular}{|c||c|c|c||c|c||}
 \hline
  & \multicolumn{5}{c}{Cola 2}\vline \\
\hline
  $T$ & Lim Inf & $\mu$ & Lim Sup & Var & Error \\\hline
1000 &  17.9870 &  18.1421 &  18.2972  &  0.0791 &   0.0044\\
5000 &  20.8343 &  20.9473 &  21.062  &  0.0576 &   0.0028\\
10000 &  21.3078 &  21.3924 &  21.4771  &  0.0432 & 0.0020\\\hline
\hline
\end{tabular}\end{table}
\end{center}

\begin{center}
\begin{table}[!ht]\caption{{\small Resultados obtenidos para
la cola 3}}\label{CerradaC3} %\centering
\begin{tabular}{|c||c|c|c||c|c||}
 \hline
  & \multicolumn{5}{c}{Cola 3}\vline \\
\hline
  $T$ & Lim Inf & $\mu$ & Lim Sup & Var & Error \\\hline
1000 &  21.3555 &  21.5625 &  21.7695 &   0.1056  &  0.0049\\
5000 &  24.7102 &  24.8675 &  25.0031 &   0.0747  &  0.0030\\
10000 &  25.3103 &  25.4197 &  25.5291 &   0.0558  &  0.0022\\
\hline
\end{tabular}\end{table}
\end{center}

En la figura \ref{GrafCerrada10M} se muestran los tiempos de
espera, los errores relativos para cada una de las colas as\'i
como el tiempo que requiere el procesador para realizar la
simulaci\'on de MonteCarlo.



En este ap\'endice enunciaremos una serie de resultados que son
necesarios para la demostraci\'on as\'i como su demostraci\'on del
Teorema de Down \ref{Tma2.1.Down}, adem\'as de un teorema
referente a las propiedades que cumple el Modelo de Flujo.\\


Dado el proceso $X=\left\{X\left(t\right),t\geq0\right\}$ definido
en (\ref{Esp.Edos.Down}) que describe la din\'amica del sistema de
visitas c\'iclicas, si $U\left(t\right)$ es el residual de los
tiempos de llegada al tiempo $t$ entre dos usuarios consecutivos y
$V\left(t\right)$ es el residual de los tiempos de servicio al
tiempo $t$ para el usuario que est\'as siendo atendido por el
servidor. Sea $\mathbb{X}$ el espacio de estados que puede tomar
el proceso $X$.


\begin{Lema}[Lema 4.3, Dai\cite{Dai}]\label{Lema.4.3}
Sea $\left\{x_{n}\right\}\subset \mathbf{X}$ con
$|x_{n}|\rightarrow\infty$, conforme $n\rightarrow\infty$. Suponga
que
\[lim_{n\rightarrow\infty}\frac{1}{|x_{n}|}U\left(0\right)=\overline{U}_{k},\]
y
\[lim_{n\rightarrow\infty}\frac{1}{|x_{n}|}V\left(0\right)=\overline{V}_{k}.\]
\begin{itemize}
\item[a)] Conforme $n\rightarrow\infty$ casi seguramente,
\[lim_{n\rightarrow\infty}\frac{1}{|x_{n}|}U^{x_{n}}_{k}\left(|x_{n}|t\right)=\left(\overline{U}_{k}-t\right)^{+}\textrm{, u.o.c.}\]
y
\[lim_{n\rightarrow\infty}\frac{1}{|x_{n}|}V^{x_{n}}_{k}\left(|x_{n}|t\right)=\left(\overline{V}_{k}-t\right)^{+}.\]

\item[b)] Para cada $t\geq0$ fijo,
\[\left\{\frac{1}{|x_{n}|}U^{x_{n}}_{k}\left(|x_{n}|t\right),|x_{n}|\geq1\right\}\]
y
\[\left\{\frac{1}{|x_{n}|}V^{x_{n}}_{k}\left(|x_{n}|t\right),|x_{n}|\geq1\right\}\]
\end{itemize}
son uniformemente convergentes.
\end{Lema}

Sea $e$ es un vector de unos, $C$ es la matriz definida por
\[C_{ik}=\left\{\begin{array}{cc}
1,& S\left(k\right)=i,\\
0,& \textrm{ en otro caso}.\\
\end{array}\right.
\]
Es necesario enunciar el siguiente Teorema que se utilizar\'a para
el Teorema (\ref{Tma.4.2.Dai}):
\begin{Teo}[Teorema 4.1, Dai \cite{Dai}]
Considere una disciplina que cumpla la ley de conservaci\'on, para
casi todas las trayectorias muestrales $\omega$ y cualquier
sucesi\'on de estados iniciales $\left\{x_{n}\right\}\subset
\mathbf{X}$, con $|x_{n}|\rightarrow\infty$, existe una
subsucesi\'on $\left\{x_{n_{j}}\right\}$ con
$|x_{n_{j}}|\rightarrow\infty$ tal que
\begin{equation}\label{Eq.4.15}
\frac{1}{|x_{n_{j}}|}\left(Q^{x_{n_{j}}}\left(0\right),U^{x_{n_{j}}}\left(0\right),V^{x_{n_{j}}}\left(0\right)\right)\rightarrow\left(\overline{Q}\left(0\right),\overline{U},\overline{V}\right),
\end{equation}

\begin{equation}\label{Eq.4.16}
\frac{1}{|x_{n_{j}}|}\left(Q^{x_{n_{j}}}\left(|x_{n_{j}}|t\right),T^{x_{n_{j}}}\left(|x_{n_{j}}|t\right)\right)\rightarrow\left(\overline{Q}\left(t\right),\overline{T}\left(t\right)\right)\textrm{
u.o.c.}
\end{equation}

Adem\'as,
$\left(\overline{Q}\left(t\right),\overline{T}\left(t\right)\right)$
satisface las siguientes ecuaciones:
\begin{equation}\label{Eq.MF.1.3a}
\overline{Q}\left(t\right)=Q\left(0\right)+\left(\alpha
t-\overline{U}\right)^{+}-\left(I-P\right)^{'}M^{-1}\left(\overline{T}\left(t\right)-\overline{V}\right)^{+},
\end{equation}

\begin{equation}\label{Eq.MF.2.3a}
\overline{Q}\left(t\right)\geq0,\\
\end{equation}

\begin{equation}\label{Eq.MF.3.3a}
\overline{T}\left(t\right)\textrm{ es no decreciente y comienza en cero},\\
\end{equation}

\begin{equation}\label{Eq.MF.4.3a}
\overline{I}\left(t\right)=et-C\overline{T}\left(t\right)\textrm{
es no decreciente,}\\
\end{equation}

\begin{equation}\label{Eq.MF.5.3a}
\int_{0}^{\infty}\left(C\overline{Q}\left(t\right)\right)d\overline{I}\left(t\right)=0,\\
\end{equation}

\begin{equation}\label{Eq.MF.6.3a}
\textrm{Condiciones en
}\left(\overline{Q}\left(\cdot\right),\overline{T}\left(\cdot\right)\right)\textrm{
espec\'ificas de la disciplina de la cola,}
\end{equation}
\end{Teo}


Propiedades importantes para el modelo de flujo retrasado:

\begin{Prop}[Proposici\'on 4.2, Dai \cite{Dai}]
 Sea $\left(\overline{Q},\overline{T},\overline{T}^{0}\right)$ un flujo l\'imite de \ref{Eq.Punto.Limite}
 y suponga que cuando $x\rightarrow\infty$ a lo largo de una subsucesi\'on
\[\left(\frac{1}{|x|}Q_{k}^{x}\left(0\right),\frac{1}{|x|}A_{k}^{x}\left(0\right),\frac{1}{|x|}B_{k}^{x}\left(0\right),\frac{1}{|x|}B_{k}^{x,0}\left(0\right)\right)\rightarrow\left(\overline{Q}_{k}\left(0\right),0,0,0\right)\]
para $k=1,\ldots,K$. El flujo l\'imite tiene las siguientes
propiedades, donde las propiedades de la derivada se cumplen donde
la derivada exista:
\begin{itemize}
 \item[i)] Los vectores de tiempo ocupado $\overline{T}\left(t\right)$ y $\overline{T}^{0}\left(t\right)$ son crecientes y continuas con
$\overline{T}\left(0\right)=\overline{T}^{0}\left(0\right)=0$.
\item[ii)] Para todo $t\geq0$
\[\sum_{k=1}^{K}\left[\overline{T}_{k}\left(t\right)+\overline{T}_{k}^{0}\left(t\right)\right]=t.\]
\item[iii)] Para todo $1\leq k\leq K$
\[\overline{Q}_{k}\left(t\right)=\overline{Q}_{k}\left(0\right)+\alpha_{k}t-\mu_{k}\overline{T}_{k}\left(t\right).\]
\item[iv)]  Para todo $1\leq k\leq K$
\[\dot{{\overline{T}}}_{k}\left(t\right)=\rho_{k}\] para $\overline{Q}_{k}\left(t\right)=0$.
\item[v)] Para todo $k,j$
\[\mu_{k}^{0}\overline{T}_{k}^{0}\left(t\right)=\mu_{j}^{0}\overline{T}_{j}^{0}\left(t\right).\]
\item[vi)]  Para todo $1\leq k\leq K$
\[\mu_{k}\dot{{\overline{T}}}_{k}\left(t\right)=l_{k}\mu_{k}^{0}\dot{{\overline{T}}}_{k}^{0}\left(t\right),\] para $\overline{Q}_{k}\left(t\right)>0$.
\end{itemize}
\end{Prop}

\begin{Lema}[Lema 3.1, Chen \cite{Chen}]\label{Lema3.1}
Si el modelo de flujo es estable, definido por las ecuaciones
(3.8)-(3.13), entonces el modelo de flujo retrasado tambi\'en es
estable.
\end{Lema}

\begin{Lema}[Lema 5.2, Gut \cite{Gut}]\label{Lema.5.2.Gut}
Sea $\left\{\xi\left(k\right):k\in\ent\right\}$ sucesi\'on de
variables aleatorias i.i.d. con valores en
$\left(0,\infty\right)$, y sea $E\left(t\right)$ el proceso de
conteo
\[E\left(t\right)=max\left\{n\geq1:\xi\left(1\right)+\cdots+\xi\left(n-1\right)\leq t\right\}.\]
Si $E\left[\xi\left(1\right)\right]<\infty$, entonces para
cualquier entero $r\geq1$
\begin{equation}
lim_{t\rightarrow\infty}\esp\left[\left(\frac{E\left(t\right)}{t}\right)^{r}\right]=\left(\frac{1}{E\left[\xi_{1}\right]}\right)^{r},
\end{equation}
de aqu\'i, bajo estas condiciones
\begin{itemize}
\item[a)] Para cualquier $t>0$,
$sup_{t\geq\delta}\esp\left[\left(\frac{E\left(t\right)}{t}\right)^{r}\right]<\infty$.

\item[b)] Las variables aleatorias
$\left\{\left(\frac{E\left(t\right)}{t}\right)^{r}:t\geq1\right\}$
son uniformemente integrables.
\end{itemize}
\end{Lema}

\begin{Teo}[Teorema 5.1: Ley Fuerte para Procesos de Conteo, Gut
\cite{Gut}]\label{Tma.5.1.Gut} Sea
$0<\mu<\esp\left(X_{1}\right]\leq\infty$. entonces

\begin{itemize}
\item[a)] $\frac{N\left(t\right)}{t}\rightarrow\frac{1}{\mu}$
a.s., cuando $t\rightarrow\infty$.


\item[b)]$\esp\left[\frac{N\left(t\right)}{t}\right]^{r}\rightarrow\frac{1}{\mu^{r}}$,
cuando $t\rightarrow\infty$ para todo $r>0$.
\end{itemize}
\end{Teo}


\begin{Prop}[Proposici\'on 5.1, Dai y Sean \cite{DaiSean}]\label{Prop.5.1}
Suponga que los supuestos (A1) y (A2) se cumplen, adem\'as suponga
que el modelo de flujo es estable. Entonces existe $t_{0}>0$ tal
que
\begin{equation}\label{Eq.Prop.5.1}
lim_{|x|\rightarrow\infty}\frac{1}{|x|^{p+1}}\esp_{x}\left[|X\left(t_{0}|x|\right)|^{p+1}\right]=0.
\end{equation}

\end{Prop}


\begin{Prop}[Proposici\'on 5.3, Dai y Sean \cite{DaiSean}]\label{Prop.5.3.DaiSean}
Sea $X$ proceso de estados para la red de colas, y suponga que se
cumplen los supuestos (A1) y (A2), entonces para alguna constante
positiva $C_{p+1}<\infty$, $\delta>0$ y un conjunto compacto
$C\subset X$.

\begin{equation}\label{Eq.5.4}
\esp_{x}\left[\int_{0}^{\tau_{C}\left(\delta\right)}\left(1+|X\left(t\right)|^{p}\right)dt\right]\leq
C_{p+1}\left(1+|x|^{p+1}\right).
\end{equation}
\end{Prop}

\begin{Prop}[Proposici\'on 5.4, Dai y Sean \cite{DaiSean}]\label{Prop.5.4.DaiSean}
Sea $X$ un proceso de Markov Borel Derecho en $X$, sea
$f:X\leftarrow\rea_{+}$ y defina para alguna $\delta>0$, y un
conjunto cerrado $C\subset X$
\[V\left(x\right):=\esp_{x}\left[\int_{0}^{\tau_{C}\left(\delta\right)}f\left(X\left(t\right)\right)dt\right],\]
para $x\in X$. Si $V$ es finito en todas partes y uniformemente
acotada en $C$, entonces existe $k<\infty$ tal que
\begin{equation}\label{Eq.5.11}
\frac{1}{t}\esp_{x}\left[V\left(x\right)\right]+\frac{1}{t}\int_{0}^{t}\esp_{x}\left[f\left(X\left(s\right)\right)ds\right]\leq\frac{1}{t}V\left(x\right)+k,
\end{equation}
para $x\in X$ y $t>0$.
\end{Prop}


\begin{Teo}[Teorema 5.5, Dai y Sean  \cite{DaiSean}]
Suponga que se cumplen (A1) y (A2), adem\'as suponga que el modelo
de flujo es estable. Entonces existe una constante $k_{p}<\infty$
tal que
\begin{equation}\label{Eq.5.13}
\frac{1}{t}\int_{0}^{t}\esp_{x}\left[|Q\left(s\right)|^{p}\right]ds\leq
k_{p}\left\{\frac{1}{t}|x|^{p+1}+1\right\},
\end{equation}
para $t\geq0$, $x\in X$. En particular para cada condici\'on
inicial
\begin{equation}\label{Eq.5.14}
\limsup_{t\rightarrow\infty}\frac{1}{t}\int_{0}^{t}\esp_{x}\left[|Q\left(s\right)|^{p}\right]ds\leq
k_{p}.
\end{equation}
\end{Teo}

\begin{Teo}[Teorema 6.2 Dai y Sean \cite{DaiSean}]\label{Tma.6.2}
Suponga que se cumplen los supuestos (A1)-(A3) y que el modelo de
flujo es estable, entonces se tiene que
\[\parallel P^{t}\left(x,\cdot\right)-\pi\left(\cdot\right)\parallel_{f_{p}}\rightarrow0,\]
para $t\rightarrow\infty$ y $x\in X$. En particular para cada
condici\'on inicial
\[lim_{t\rightarrow\infty}\esp_{x}\left[\left|Q_{t}\right|^{p}\right]=\esp_{\pi}\left[\left|Q_{0}\right|^{p}\right]<\infty,\]
\end{Teo}

donde

\begin{eqnarray*}
\parallel
P^{t}\left(c,\cdot\right)-\pi\left(\cdot\right)\parallel_{f}=sup_{|g\leq
f|}|\int\pi\left(dy\right)g\left(y\right)-\int
P^{t}\left(x,dy\right)g\left(y\right)|,
\end{eqnarray*}
para $x\in\mathbb{X}$.

\begin{Teo}[Teorema 6.3, Dai y Sean \cite{DaiSean}]\label{Tma.6.3}
Suponga que se cumplen los supuestos (A1)-(A3) y que el modelo de
flujo es estable, entonces con
$f\left(x\right)=f_{1}\left(x\right)$, se tiene que
\[lim_{t\rightarrow\infty}t^{(p-1)}\left|P^{t}\left(c,\cdot\right)-\pi\left(\cdot\right)\right|_{f}=0,\]
para $x\in X$. En particular, para cada condici\'on inicial
\[lim_{t\rightarrow\infty}t^{(p-1)}\left|\esp_{x}\left[Q_{t}\right]-\esp_{\pi}\left[Q_{0}\right]\right|=0.\]
\end{Teo}



\begin{Prop}[Proposici\'on 5.1, Dai y Meyn \cite{DaiSean}]\label{Prop.5.1.DaiSean}
Suponga que los supuestos A1) y A2) son ciertos y que el modelo de
flujo es estable. Entonces existe $t_{0}>0$ tal que
\begin{equation}
lim_{|x|\rightarrow\infty}\frac{1}{|x|^{p+1}}\esp_{x}\left[|X\left(t_{0}|x|\right)|^{p+1}\right]=0.
\end{equation}
\end{Prop}


\begin{Teo}[Teorema 5.5, Dai y Meyn \cite{DaiSean}]\label{Tma.5.5.DaiSean}
Suponga que los supuestos A1) y A2) se cumplen y que el modelo de
flujo es estable. Entonces existe una constante $\kappa_{p}$ tal
que
\begin{equation}
\frac{1}{t}\int_{0}^{t}\esp_{x}\left[|Q\left(s\right)|^{p}\right]ds\leq\kappa_{p}\left\{\frac{1}{t}|x|^{p+1}+1\right\},
\end{equation}
para $t>0$ y $x\in X$. En particular, para cada condici\'on
inicial
\begin{eqnarray*}
\limsup_{t\rightarrow\infty}\frac{1}{t}\int_{0}^{t}\esp_{x}\left[|Q\left(s\right)|^{p}\right]ds\leq\kappa_{p}.
\end{eqnarray*}
\end{Teo}


\begin{Teo}[Teorema 6.4, Dai y Meyn \cite{DaiSean}]\label{Tma.6.4.DaiSean}
Suponga que se cumplen los supuestos A1), A2) y A3) y que el
modelo de flujo es estable. Sea $\nu$ cualquier distribuci\'on de
probabilidad en
$\left(\mathbb{X},\mathcal{B}_{\mathbb{X}}\right)$, y $\pi$ la
distribuci\'on estacionaria de $X$.
\begin{itemize}
\item[i)] Para cualquier $f:X\leftarrow\rea_{+}$
\begin{equation}
\lim_{t\rightarrow\infty}\frac{1}{t}\int_{o}^{t}f\left(X\left(s\right)\right)ds=\pi\left(f\right):=\int
f\left(x\right)\pi\left(dx\right),
\end{equation}
$\prob$-c.s.

\item[ii)] Para cualquier $f:X\leftarrow\rea_{+}$ con
$\pi\left(|f|\right)<\infty$, la ecuaci\'on anterior se cumple.
\end{itemize}
\end{Teo}

\begin{Teo}[Teorema 2.2, Down \cite{Down}]\label{Tma2.2.Down}
Suponga que el fluido modelo es inestable en el sentido de que
para alguna $\epsilon_{0},c_{0}\geq0$,
\begin{equation}\label{Eq.Inestability}
|Q\left(T\right)|\geq\epsilon_{0}T-c_{0}\textrm{,   }T\geq0,
\end{equation}
para cualquier condici\'on inicial $Q\left(0\right)$, con
$|Q\left(0\right)|=1$. Entonces para cualquier $0<q\leq1$, existe
$B<0$ tal que para cualquier $|x|\geq B$,
\begin{equation}
\prob_{x}\left\{\mathbb{X}\rightarrow\infty\right\}\geq q.
\end{equation}
\end{Teo}

\begin{Dem}[Teorema \ref{Tma2.1.Down}] La demostraci\'on de este
teorema se da a continuaci\'on:\\
\begin{itemize}
\item[i)] Utilizando la proposici\'on \ref{Prop.5.3.DaiSean} se
tiene que la proposici\'on \ref{Prop.5.4.DaiSean} es cierta para
$f\left(x\right)=1+|x|^{p}$.

\item[i)] es consecuencia directa del Teorema \ref{Tma.6.2}.

\item[iii)] ver la demostraci\'on dada en Dai y Sean
\cite{DaiSean} p\'aginas 1901-1902.

\item[iv)] ver Dai y Sean \cite{DaiSean} p\'aginas 1902-1903 \'o
\cite{MeynTweedie2}.
\end{itemize}
\end{Dem}
\newpage
%_________________________________________________________________________
%\subsection{AP\'ENDICE B}
%_________________________________________________________________________
%\numberwithin{equation}{section}

\begin{Teo}[Teorema de Continuidad]
Sup\'ongase que $\left\{X_{n},n=1,2,3,\ldots\right\}$ son
variables aleatorias finitas, no negativas con valores enteros
tales que $P\left(X_{n}=k\right)=p_{k}^{(n)}$, para
$n=1,2,3,\ldots$, $k=0,1,2,\ldots$, con
$\sum_{k=0}^{\infty}p_{k}^{(n)}=1$, para $n=1,2,3,\ldots$. Sea
$g_{n}$ la PGF para la variable aleatoria $X_{n}$. Entonces existe
una sucesi\'on $\left\{p_{k}\right\}$ tal que \begin{eqnarray*}
lim_{n\rightarrow\infty}p_{k}^{(n)}=p_{k}\textrm{ para }0<s<1.
\end{eqnarray*}
En este caso, $g\left(s\right)=\sum_{k=0}^{\infty}s^{k}p_{k}$.
Adem\'as
\begin{eqnarray*}
\sum_{k=0}^{\infty}p_{k}=1\textrm{ si y s\'olo si
}lim_{s\uparrow1}g\left(s\right)=1
\end{eqnarray*}
\end{Teo}

\begin{Teo}
Sea $N$ una variable aleatoria con valores enteros no negativos
finita tal que $P\left(N=k\right)=p_{k}$, para $k=0,1,2,\ldots$, y
$\sum_{k=0}^{\infty}p_{k}=P\left(N<\infty\right)=1$. Sea $\Phi$ la
PGF de $N$ tal que
$g\left(s\right)=\esp\left[s^{N}\right]=\sum_{k=0}^{\infty}s^{k}p_{k}$
con $g\left(1\right)=1$. Si $0\leq p_{1}\leq1$ y
$\esp\left[N\right]=g^{'}\left(1\right)\leq1$, entonces no existe
soluci\'on  de la ecuaci\'on $g\left(s\right)=s$ en el intervalo
$\left[0,1\right)$. Si $\esp\left[N\right]=g^{'}\left(1\right)>1$,
lo cual implica que $0\leq p_{1}<1$, entonces existe una \'unica
soluci\'on de la ecuaci\'on $g\left(s\right)=s$ en el intervalo
$\left[0,1\right)$.
\end{Teo}


\begin{Teo}
Si $X$ y $Y$ tienen PGF $G_{X}$ y $G_{Y}$ respectivamente,
entonces,\[G_{X}\left(s\right)=G_{Y}\left(s\right)\] para toda
$s$, si y s\'olo si \[P\left(X=k\right))=P\left(Y=k\right)\] para
toda $k=0,1,\ldots,$., es decir, si y s\'olo si $X$ y $Y$ tienen
la misma distribuci\'on de probabilidad.
\end{Teo}


\begin{Teo}
Para cada $n$ fijo, sea la sucesi\'oin de probabilidades
$\left\{a_{0,n},a_{1,n},\ldots,\right\}$, tales que $a_{k,n}\geq0$
para toda $k=0,1,2,\ldots,$ y $\sum_{k\geq0}a_{k,n}=1$, y sea
$G_{n}\left(s\right)$ la correspondiente funci\'on generadora,
$G_{n}\left(s\right)=\sum_{k\geq0}a_{k,n}s^{k}$. De modo que para
cada valor fijo de $k$
\begin{eqnarray*}
lim_{n\rightarrow\infty}a_{k,n}=a_{k},
\end{eqnarray*}
es decir converge en distribuci\'on, es necesario y suficiente que
para cada valor fijo $s\in\left[0,\right)$,
\begin{eqnarray*}
lim_{n\rightarrow\infty}G_{n}\left(s\right)=G\left(s\right),
\end{eqnarray*}
donde $G\left(s\right)=\sum_{k\geq0}p_{k}s^{k}$, para cualquier

la funci\'on generadora del l\'imite de la sucesi\'on.
\end{Teo}

\begin{Teo}[Teorema de Abel]
Sea $G\left(s\right)=\sum_{k\geq0}a_{k}s^{k}$ para cualquier
$\left\{p_{0},p_{1},\ldots,\right\}$, tales que $p_{k}\geq0$ para
toda $k=0,1,2,\ldots,$. Entonces $G\left(s\right)$ es continua por
la derecha en $s=1$, es decir
\begin{eqnarray*}
lim_{s\uparrow1}G\left(s\right)=\sum_{k\geq0}p_{k}=G\left(\right),
\end{eqnarray*}
sin importar si la suma es finita o no.
\end{Teo}
\begin{Note}
El radio de Convergencia para cualquier PGF es $R\geq1$, entonces,
el Teorema de Abel nos dice que a\'un en el peor escenario, cuando
$R=1$, a\'un se puede confiar en que la PGF ser\'a continua en
$s=1$, en contraste, no se puede asegurar que la PGF ser\'a
continua en el l\'imite inferior $-R$, puesto que la PGF es
sim\'etrica alrededor del cero: la PGF converge para todo
$s\in\left(-R,R\right)$, y no lo hace para $s<-R$ o $s>R$.
Adem\'as nos dice que podemos escribir $G_{X}\left(1\right)$ como
una abreviaci\'on de $lim_{s\uparrow1}G_{X}\left(s\right)$.
\end{Note}

Entonces si suponemos que la diferenciaci\'on t\'ermino a
t\'ermino est\'a permitida, entonces

\begin{eqnarray*}
G_{X}^{'}\left(s\right)&=&\sum_{x=1}^{\infty}xs^{x-1}p_{x}
\end{eqnarray*}

el Teorema de Abel nos dice que
\begin{eqnarray*}
\esp\left(X\right]&=&\lim_{s\uparrow1}G_{X}^{'}\left(s\right):\\
\esp\left[X\right]&=&=\sum_{x=1}^{\infty}xp_{x}=G_{X}^{'}\left(1\right)\\
&=&\lim_{s\uparrow1}G_{X}^{'}\left(s\right),
\end{eqnarray*}
dado que el Teorema de Abel se aplica a
\begin{eqnarray*}
G_{X}^{'}\left(s\right)&=&\sum_{x=1}^{\infty}xs^{x-1}p_{x},
\end{eqnarray*}
estableciendo as\'i que $G_{X}^{'}\left(s\right)$ es continua en
$s=1$. Sin el Teorema de Abel no se podr\'ia asegurar que el
l\'imite de $G_{X}^{'}\left(s\right)$ conforme $s\uparrow1$ sea la
respuesta correcta para $\esp\left[X\right]$.

\begin{Note}
La PGF converge para todo $|s|<R$, para alg\'un $R$. De hecho la
PGF converge absolutamente si $|s|<R$. La PGF adem\'as converge
uniformemente en conjuntos de la forma
$\left\{s:|s|<R^{'}\right\}$, donde $R^{'}<R$, es decir,
$\forall\epsilon>0, \exists n_{0}\in\ent$ tal que $\forall s$, con
$|s|<R^{'}$, y $\forall n\geq n_{0}$,
\begin{eqnarray*}
|\sum_{x=0}^{n}s^{x}\prob\left(X=x\right)-G_{X}\left(s\right)|<\epsilon.
\end{eqnarray*}
De hecho, la convergencia uniforme es la que nos permite
diferenciar t\'ermino a t\'ermino:
\begin{eqnarray*}
G_{X}\left(s\right)=\esp\left[s^{X}\right]=\sum_{x=0}^{\infty}s^{x}\prob\left(X=x\right),
\end{eqnarray*}
y sea $s<R$.
\begin{enumerate}
\item
\begin{eqnarray*}
G_{X}^{'}\left(s\right)&=&\frac{d}{ds}\left(\sum_{x=0}^{\infty}s^{x}\prob\left(X=x\right)\right)=\sum_{x=0}^{\infty}\frac{d}{ds}\left(s^{x}\prob\left(X=x\right)\right)\\
&=&\sum_{x=0}^{n}xs^{x-1}\prob\left(X=x\right).
\end{eqnarray*}

\item\begin{eqnarray*}
\int_{a}^{b}G_{X}\left(s\right)ds&=&\int_{a}^{b}\left(\sum_{x=0}^{\infty}s^{x}\prob\left(X=x\right)\right)ds=\sum_{x=0}^{\infty}\left(\int_{a}^{b}s^{x}\prob\left(X=x\right)ds\right)\\
&=&\sum_{x=0}^{\infty}\frac{s^{x+1}}{x+1}\prob\left(X=x\right),
\end{eqnarray*}
para $-R<a<b<R$.
\end{enumerate}
\end{Note}

\begin{Teo}[Teorema de Convergencia Mon\'otona para PGF]
Sean $X$ y $X_{n}$ variables aleatorias no negativas, con valores
en los enteros, finitas, tales que
\begin{eqnarray*}
lim_{n\rightarrow\infty}G_{X_{n}}\left(s\right)&=&G_{X}\left(s\right)
\end{eqnarray*}
para $0\leq s\leq1$, entonces
\begin{eqnarray*}
lim_{n\rightarrow\infty}P\left(X_{n}=k\right)=P\left(X=k\right),
\end{eqnarray*}
para $k=0,1,2,\ldots.$
\end{Teo}

El teorema anterior requiere del siguiente lema

\begin{Lemma}
Sean $a_{n,k}\in\ent^{+}$, $n\in\nat$ constantes no negativas con
$\sum_{k\geq0}a_{k,n}\leq1$. Sup\'ongase que para $0\leq s\leq1$,
se tiene

\begin{eqnarray*}
a_{n}\left(s\right)&=&\sum_{k=0}^{\infty}a_{k,n}s^{k}\rightarrow
a\left(s\right)=\sum_{k=0}^{\infty}a_{k}s^{k}.
\end{eqnarray*}
Entonces
\begin{eqnarray*}
a_{0,n}\rightarrow a_{0}.
\end{eqnarray*}
\end{Lemma}


En este ap\'endice enunciaremos una serie de resultados que son
necesarios para la demostraci\'on as\'i como su demostraci\'on del
Teorema de Down \ref{Tma2.1.Down}, adem\'as de un teorema
referente a las propiedades que cumple el Modelo de Flujo.\\


Dado el proceso $X=\left\{X\left(t\right),t\geq0\right\}$ definido
en (\ref{Esp.Edos.Down}) que describe la din\'amica del sistema de
visitas c\'iclicas, si $U\left(t\right)$ es el residual de los
tiempos de llegada al tiempo $t$ entre dos usuarios consecutivos y
$V\left(t\right)$ es el residual de los tiempos de servicio al
tiempo $t$ para el usuario que est\'as siendo atendido por el
servidor. Sea $\mathbb{X}$ el espacio de estados que puede tomar
el proceso $X$.


\begin{Lema}[Lema 4.3, Dai\cite{Dai}]\label{Lema.4.3}
Sea $\left\{x_{n}\right\}\subset \mathbf{X}$ con
$|x_{n}|\rightarrow\infty$, conforme $n\rightarrow\infty$. Suponga
que
\[lim_{n\rightarrow\infty}\frac{1}{|x_{n}|}U\left(0\right)=\overline{U}_{k},\]
y
\[lim_{n\rightarrow\infty}\frac{1}{|x_{n}|}V\left(0\right)=\overline{V}_{k}.\]
\begin{itemize}
\item[a)] Conforme $n\rightarrow\infty$ casi seguramente,
\[lim_{n\rightarrow\infty}\frac{1}{|x_{n}|}U^{x_{n}}_{k}\left(|x_{n}|t\right)=\left(\overline{U}_{k}-t\right)^{+}\textrm{, u.o.c.}\]
y
\[lim_{n\rightarrow\infty}\frac{1}{|x_{n}|}V^{x_{n}}_{k}\left(|x_{n}|t\right)=\left(\overline{V}_{k}-t\right)^{+}.\]

\item[b)] Para cada $t\geq0$ fijo,
\[\left\{\frac{1}{|x_{n}|}U^{x_{n}}_{k}\left(|x_{n}|t\right),|x_{n}|\geq1\right\}\]
y
\[\left\{\frac{1}{|x_{n}|}V^{x_{n}}_{k}\left(|x_{n}|t\right),|x_{n}|\geq1\right\}\]
\end{itemize}
son uniformemente convergentes.
\end{Lema}

Sea $e$ es un vector de unos, $C$ es la matriz definida por
\[C_{ik}=\left\{\begin{array}{cc}
1,& S\left(k\right)=i,\\
0,& \textrm{ en otro caso}.\\
\end{array}\right.
\]
Es necesario enunciar el siguiente Teorema que se utilizar\'a para
el Teorema (\ref{Tma.4.2.Dai}):
\begin{Teo}[Teorema 4.1, Dai \cite{Dai}]
Considere una disciplina que cumpla la ley de conservaci\'on, para
casi todas las trayectorias muestrales $\omega$ y cualquier
sucesi\'on de estados iniciales $\left\{x_{n}\right\}\subset
\mathbf{X}$, con $|x_{n}|\rightarrow\infty$, existe una
subsucesi\'on $\left\{x_{n_{j}}\right\}$ con
$|x_{n_{j}}|\rightarrow\infty$ tal que
\begin{equation}\label{Eq.4.15}
\frac{1}{|x_{n_{j}}|}\left(Q^{x_{n_{j}}}\left(0\right),U^{x_{n_{j}}}\left(0\right),V^{x_{n_{j}}}\left(0\right)\right)\rightarrow\left(\overline{Q}\left(0\right),\overline{U},\overline{V}\right),
\end{equation}

\begin{equation}\label{Eq.4.16}
\frac{1}{|x_{n_{j}}|}\left(Q^{x_{n_{j}}}\left(|x_{n_{j}}|t\right),T^{x_{n_{j}}}\left(|x_{n_{j}}|t\right)\right)\rightarrow\left(\overline{Q}\left(t\right),\overline{T}\left(t\right)\right)\textrm{
u.o.c.}
\end{equation}

Adem\'as,
$\left(\overline{Q}\left(t\right),\overline{T}\left(t\right)\right)$
satisface las siguientes ecuaciones:
\begin{equation}\label{Eq.MF.1.3a}
\overline{Q}\left(t\right)=Q\left(0\right)+\left(\alpha
t-\overline{U}\right)^{+}-\left(I-P\right)^{'}M^{-1}\left(\overline{T}\left(t\right)-\overline{V}\right)^{+},
\end{equation}

\begin{equation}\label{Eq.MF.2.3a}
\overline{Q}\left(t\right)\geq0,\\
\end{equation}

\begin{equation}\label{Eq.MF.3.3a}
\overline{T}\left(t\right)\textrm{ es no decreciente y comienza en cero},\\
\end{equation}

\begin{equation}\label{Eq.MF.4.3a}
\overline{I}\left(t\right)=et-C\overline{T}\left(t\right)\textrm{
es no decreciente,}\\
\end{equation}

\begin{equation}\label{Eq.MF.5.3a}
\int_{0}^{\infty}\left(C\overline{Q}\left(t\right)\right)d\overline{I}\left(t\right)=0,\\
\end{equation}

\begin{equation}\label{Eq.MF.6.3a}
\textrm{Condiciones en
}\left(\overline{Q}\left(\cdot\right),\overline{T}\left(\cdot\right)\right)\textrm{
espec\'ificas de la disciplina de la cola,}
\end{equation}
\end{Teo}


Propiedades importantes para el modelo de flujo retrasado:

\begin{Prop}[Proposici\'on 4.2, Dai \cite{Dai}]
 Sea $\left(\overline{Q},\overline{T},\overline{T}^{0}\right)$ un flujo l\'imite de \ref{Eq.Punto.Limite}
 y suponga que cuando $x\rightarrow\infty$ a lo largo de una subsucesi\'on
\[\left(\frac{1}{|x|}Q_{k}^{x}\left(0\right),\frac{1}{|x|}A_{k}^{x}\left(0\right),\frac{1}{|x|}B_{k}^{x}\left(0\right),\frac{1}{|x|}B_{k}^{x,0}\left(0\right)\right)\rightarrow\left(\overline{Q}_{k}\left(0\right),0,0,0\right)\]
para $k=1,\ldots,K$. El flujo l\'imite tiene las siguientes
propiedades, donde las propiedades de la derivada se cumplen donde
la derivada exista:
\begin{itemize}
 \item[i)] Los vectores de tiempo ocupado $\overline{T}\left(t\right)$ y $\overline{T}^{0}\left(t\right)$ son crecientes y continuas con
$\overline{T}\left(0\right)=\overline{T}^{0}\left(0\right)=0$.
\item[ii)] Para todo $t\geq0$
\[\sum_{k=1}^{K}\left[\overline{T}_{k}\left(t\right)+\overline{T}_{k}^{0}\left(t\right)\right]=t.\]
\item[iii)] Para todo $1\leq k\leq K$
\[\overline{Q}_{k}\left(t\right)=\overline{Q}_{k}\left(0\right)+\alpha_{k}t-\mu_{k}\overline{T}_{k}\left(t\right).\]
\item[iv)]  Para todo $1\leq k\leq K$
\[\dot{{\overline{T}}}_{k}\left(t\right)=\rho_{k}\] para $\overline{Q}_{k}\left(t\right)=0$.
\item[v)] Para todo $k,j$
\[\mu_{k}^{0}\overline{T}_{k}^{0}\left(t\right)=\mu_{j}^{0}\overline{T}_{j}^{0}\left(t\right).\]
\item[vi)]  Para todo $1\leq k\leq K$
\[\mu_{k}\dot{{\overline{T}}}_{k}\left(t\right)=l_{k}\mu_{k}^{0}\dot{{\overline{T}}}_{k}^{0}\left(t\right),\] para $\overline{Q}_{k}\left(t\right)>0$.
\end{itemize}
\end{Prop}

\begin{Lema}[Lema 3.1, Chen \cite{Chen}]\label{Lema3.1}
Si el modelo de flujo es estable, definido por las ecuaciones
(3.8)-(3.13), entonces el modelo de flujo retrasado tambi\'en es
estable.
\end{Lema}

\begin{Lema}[Lema 5.2, Gut \cite{Gut}]\label{Lema.5.2.Gut}
Sea $\left\{\xi\left(k\right):k\in\ent\right\}$ sucesi\'on de
variables aleatorias i.i.d. con valores en
$\left(0,\infty\right)$, y sea $E\left(t\right)$ el proceso de
conteo
\[E\left(t\right)=max\left\{n\geq1:\xi\left(1\right)+\cdots+\xi\left(n-1\right)\leq t\right\}.\]
Si $E\left[\xi\left(1\right)\right]<\infty$, entonces para
cualquier entero $r\geq1$
\begin{equation}
lim_{t\rightarrow\infty}\esp\left[\left(\frac{E\left(t\right)}{t}\right)^{r}\right]=\left(\frac{1}{E\left[\xi_{1}\right]}\right)^{r},
\end{equation}
de aqu\'i, bajo estas condiciones
\begin{itemize}
\item[a)] Para cualquier $t>0$,
$sup_{t\geq\delta}\esp\left[\left(\frac{E\left(t\right)}{t}\right)^{r}\right]<\infty$.

\item[b)] Las variables aleatorias
$\left\{\left(\frac{E\left(t\right)}{t}\right)^{r}:t\geq1\right\}$
son uniformemente integrables.
\end{itemize}
\end{Lema}

\begin{Teo}[Teorema 5.1: Ley Fuerte para Procesos de Conteo, Gut
\cite{Gut}]\label{Tma.5.1.Gut} Sea
$0<\mu<\esp\left(X_{1}\right]\leq\infty$. entonces

\begin{itemize}
\item[a)] $\frac{N\left(t\right)}{t}\rightarrow\frac{1}{\mu}$
a.s., cuando $t\rightarrow\infty$.


\item[b)]$\esp\left[\frac{N\left(t\right)}{t}\right]^{r}\rightarrow\frac{1}{\mu^{r}}$,
cuando $t\rightarrow\infty$ para todo $r>0$.
\end{itemize}
\end{Teo}


\begin{Prop}[Proposici\'on 5.1, Dai y Sean \cite{DaiSean}]\label{Prop.5.1}
Suponga que los supuestos (A1) y (A2) se cumplen, adem\'as suponga
que el modelo de flujo es estable. Entonces existe $t_{0}>0$ tal
que
\begin{equation}\label{Eq.Prop.5.1}
lim_{|x|\rightarrow\infty}\frac{1}{|x|^{p+1}}\esp_{x}\left[|X\left(t_{0}|x|\right)|^{p+1}\right]=0.
\end{equation}

\end{Prop}


\begin{Prop}[Proposici\'on 5.3, Dai y Sean \cite{DaiSean}]\label{Prop.5.3.DaiSean}
Sea $X$ proceso de estados para la red de colas, y suponga que se
cumplen los supuestos (A1) y (A2), entonces para alguna constante
positiva $C_{p+1}<\infty$, $\delta>0$ y un conjunto compacto
$C\subset X$.

\begin{equation}\label{Eq.5.4}
\esp_{x}\left[\int_{0}^{\tau_{C}\left(\delta\right)}\left(1+|X\left(t\right)|^{p}\right)dt\right]\leq
C_{p+1}\left(1+|x|^{p+1}\right).
\end{equation}
\end{Prop}

\begin{Prop}[Proposici\'on 5.4, Dai y Sean \cite{DaiSean}]\label{Prop.5.4.DaiSean}
Sea $X$ un proceso de Markov Borel Derecho en $X$, sea
$f:X\leftarrow\rea_{+}$ y defina para alguna $\delta>0$, y un
conjunto cerrado $C\subset X$
\[V\left(x\right):=\esp_{x}\left[\int_{0}^{\tau_{C}\left(\delta\right)}f\left(X\left(t\right)\right)dt\right],\]
para $x\in X$. Si $V$ es finito en todas partes y uniformemente
acotada en $C$, entonces existe $k<\infty$ tal que
\begin{equation}\label{Eq.5.11}
\frac{1}{t}\esp_{x}\left[V\left(x\right)\right]+\frac{1}{t}\int_{0}^{t}\esp_{x}\left[f\left(X\left(s\right)\right)ds\right]\leq\frac{1}{t}V\left(x\right)+k,
\end{equation}
para $x\in X$ y $t>0$.
\end{Prop}


\begin{Teo}[Teorema 5.5, Dai y Sean  \cite{DaiSean}]
Suponga que se cumplen (A1) y (A2), adem\'as suponga que el modelo
de flujo es estable. Entonces existe una constante $k_{p}<\infty$
tal que
\begin{equation}\label{Eq.5.13}
\frac{1}{t}\int_{0}^{t}\esp_{x}\left[|Q\left(s\right)|^{p}\right]ds\leq
k_{p}\left\{\frac{1}{t}|x|^{p+1}+1\right\},
\end{equation}
para $t\geq0$, $x\in X$. En particular para cada condici\'on
inicial
\begin{equation}\label{Eq.5.14}
\limsup_{t\rightarrow\infty}\frac{1}{t}\int_{0}^{t}\esp_{x}\left[|Q\left(s\right)|^{p}\right]ds\leq
k_{p}.
\end{equation}
\end{Teo}

\begin{Teo}[Teorema 6.2 Dai y Sean \cite{DaiSean}]\label{Tma.6.2}
Suponga que se cumplen los supuestos (A1)-(A3) y que el modelo de
flujo es estable, entonces se tiene que
\[\parallel P^{t}\left(x,\cdot\right)-\pi\left(\cdot\right)\parallel_{f_{p}}\rightarrow0,\]
para $t\rightarrow\infty$ y $x\in X$. En particular para cada
condici\'on inicial
\[lim_{t\rightarrow\infty}\esp_{x}\left[\left|Q_{t}\right|^{p}\right]=\esp_{\pi}\left[\left|Q_{0}\right|^{p}\right]<\infty,\]
\end{Teo}

donde

\begin{eqnarray*}
\parallel
P^{t}\left(c,\cdot\right)-\pi\left(\cdot\right)\parallel_{f}=sup_{|g\leq
f|}|\int\pi\left(dy\right)g\left(y\right)-\int
P^{t}\left(x,dy\right)g\left(y\right)|,
\end{eqnarray*}
para $x\in\mathbb{X}$.

\begin{Teo}[Teorema 6.3, Dai y Sean \cite{DaiSean}]\label{Tma.6.3}
Suponga que se cumplen los supuestos (A1)-(A3) y que el modelo de
flujo es estable, entonces con
$f\left(x\right)=f_{1}\left(x\right)$, se tiene que
\[lim_{t\rightarrow\infty}t^{(p-1)}\left|P^{t}\left(c,\cdot\right)-\pi\left(\cdot\right)\right|_{f}=0,\]
para $x\in X$. En particular, para cada condici\'on inicial
\[lim_{t\rightarrow\infty}t^{(p-1)}\left|\esp_{x}\left[Q_{t}\right]-\esp_{\pi}\left[Q_{0}\right]\right|=0.\]
\end{Teo}



\begin{Prop}[Proposici\'on 5.1, Dai y Meyn \cite{DaiSean}]\label{Prop.5.1.DaiSean}
Suponga que los supuestos A1) y A2) son ciertos y que el modelo de
flujo es estable. Entonces existe $t_{0}>0$ tal que
\begin{equation}
lim_{|x|\rightarrow\infty}\frac{1}{|x|^{p+1}}\esp_{x}\left[|X\left(t_{0}|x|\right)|^{p+1}\right]=0.
\end{equation}
\end{Prop}


\begin{Teo}[Teorema 5.5, Dai y Meyn \cite{DaiSean}]\label{Tma.5.5.DaiSean}
Suponga que los supuestos A1) y A2) se cumplen y que el modelo de
flujo es estable. Entonces existe una constante $\kappa_{p}$ tal
que
\begin{equation}
\frac{1}{t}\int_{0}^{t}\esp_{x}\left[|Q\left(s\right)|^{p}\right]ds\leq\kappa_{p}\left\{\frac{1}{t}|x|^{p+1}+1\right\},
\end{equation}
para $t>0$ y $x\in X$. En particular, para cada condici\'on
inicial
\begin{eqnarray*}
\limsup_{t\rightarrow\infty}\frac{1}{t}\int_{0}^{t}\esp_{x}\left[|Q\left(s\right)|^{p}\right]ds\leq\kappa_{p}.
\end{eqnarray*}
\end{Teo}


\begin{Teo}[Teorema 6.4, Dai y Meyn \cite{DaiSean}]\label{Tma.6.4.DaiSean}
Suponga que se cumplen los supuestos A1), A2) y A3) y que el
modelo de flujo es estable. Sea $\nu$ cualquier distribuci\'on de
probabilidad en
$\left(\mathbb{X},\mathcal{B}_{\mathbb{X}}\right)$, y $\pi$ la
distribuci\'on estacionaria de $X$.
\begin{itemize}
\item[i)] Para cualquier $f:X\leftarrow\rea_{+}$
\begin{equation}
\lim_{t\rightarrow\infty}\frac{1}{t}\int_{o}^{t}f\left(X\left(s\right)\right)ds=\pi\left(f\right):=\int
f\left(x\right)\pi\left(dx\right),
\end{equation}
$\prob$-c.s.

\item[ii)] Para cualquier $f:X\leftarrow\rea_{+}$ con
$\pi\left(|f|\right)<\infty$, la ecuaci\'on anterior se cumple.
\end{itemize}
\end{Teo}

\begin{Teo}[Teorema 2.2, Down \cite{Down}]\label{Tma2.2.Down}
Suponga que el fluido modelo es inestable en el sentido de que
para alguna $\epsilon_{0},c_{0}\geq0$,
\begin{equation}\label{Eq.Inestability}
|Q\left(T\right)|\geq\epsilon_{0}T-c_{0}\textrm{,   }T\geq0,
\end{equation}
para cualquier condici\'on inicial $Q\left(0\right)$, con
$|Q\left(0\right)|=1$. Entonces para cualquier $0<q\leq1$, existe
$B<0$ tal que para cualquier $|x|\geq B$,
\begin{equation}
\prob_{x}\left\{\mathbb{X}\rightarrow\infty\right\}\geq q.
\end{equation}
\end{Teo}

\begin{Dem}[Teorema \ref{Tma2.1.Down}] La demostraci\'on de este
teorema se da a continuaci\'on:\\
\begin{itemize}
\item[i)] Utilizando la proposici\'on \ref{Prop.5.3.DaiSean} se
tiene que la proposici\'on \ref{Prop.5.4.DaiSean} es cierta para
$f\left(x\right)=1+|x|^{p}$.

\item[i)] es consecuencia directa del Teorema \ref{Tma.6.2}.

\item[iii)] ver la demostraci\'on dada en Dai y Sean
\cite{DaiSean} p\'aginas 1901-1902.

\item[iv)] ver Dai y Sean \cite{DaiSean} p\'aginas 1902-1903 \'o
\cite{MeynTweedie2}.
\end{itemize}
\end{Dem}
\newpage
%_________________________________________________________________________
%\subsection{AP\'ENDICE B}

\begin{Teo}[Teorema de Continuidad]
Sup\'ongase que $\left\{X_{n},n=1,2,3,\ldots\right\}$ son
variables aleatorias finitas, no negativas con valores enteros
tales que $P\left(X_{n}=k\right)=p_{k}^{(n)}$, para
$n=1,2,3,\ldots$, $k=0,1,2,\ldots$, con
$\sum_{k=0}^{\infty}p_{k}^{(n)}=1$, para $n=1,2,3,\ldots$. Sea
$g_{n}$ la PGF para la variable aleatoria $X_{n}$. Entonces existe
una sucesi\'on $\left\{p_{k}\right\}$ tal que \begin{eqnarray*}
lim_{n\rightarrow\infty}p_{k}^{(n)}=p_{k}\textrm{ para }0<s<1.
\end{eqnarray*}
En este caso, $g\left(s\right)=\sum_{k=0}^{\infty}s^{k}p_{k}$.
Adem\'as
\begin{eqnarray*}
\sum_{k=0}^{\infty}p_{k}=1\textrm{ si y s\'olo si
}lim_{s\uparrow1}g\left(s\right)=1
\end{eqnarray*}
\end{Teo}

\begin{Teo}
Sea $N$ una variable aleatoria con valores enteros no negativos
finita tal que $P\left(N=k\right)=p_{k}$, para $k=0,1,2,\ldots$, y
$\sum_{k=0}^{\infty}p_{k}=P\left(N<\infty\right)=1$. Sea $\Phi$ la
PGF de $N$ tal que
$g\left(s\right)=\esp\left[s^{N}\right]=\sum_{k=0}^{\infty}s^{k}p_{k}$
con $g\left(1\right)=1$. Si $0\leq p_{1}\leq1$ y
$\esp\left[N\right]=g^{'}\left(1\right)\leq1$, entonces no existe
soluci\'on  de la ecuaci\'on $g\left(s\right)=s$ en el intervalo
$\left[0,1\right)$. Si $\esp\left[N\right]=g^{'}\left(1\right)>1$,
lo cual implica que $0\leq p_{1}<1$, entonces existe una \'unica
soluci\'on de la ecuaci\'on $g\left(s\right)=s$ en el intervalo
$\left[0,1\right)$.
\end{Teo}


\begin{Teo}
Si $X$ y $Y$ tienen PGF $G_{X}$ y $G_{Y}$ respectivamente,
entonces,\[G_{X}\left(s\right)=G_{Y}\left(s\right)\] para toda
$s$, si y s\'olo si \[P\left(X=k\right))=P\left(Y=k\right)\] para
toda $k=0,1,\ldots,$., es decir, si y s\'olo si $X$ y $Y$ tienen
la misma distribuci\'on de probabilidad.
\end{Teo}


\begin{Teo}
Para cada $n$ fijo, sea la sucesi\'oin de probabilidades
$\left\{a_{0,n},a_{1,n},\ldots,\right\}$, tales que $a_{k,n}\geq0$
para toda $k=0,1,2,\ldots,$ y $\sum_{k\geq0}a_{k,n}=1$, y sea
$G_{n}\left(s\right)$ la correspondiente funci\'on generadora,
$G_{n}\left(s\right)=\sum_{k\geq0}a_{k,n}s^{k}$. De modo que para
cada valor fijo de $k$
\begin{eqnarray*}
lim_{n\rightarrow\infty}a_{k,n}=a_{k},
\end{eqnarray*}
es decir converge en distribuci\'on, es necesario y suficiente que
para cada valor fijo $s\in\left[0,\right)$,
\begin{eqnarray*}
lim_{n\rightarrow\infty}G_{n}\left(s\right)=G\left(s\right),
\end{eqnarray*}
donde $G\left(s\right)=\sum_{k\geq0}p_{k}s^{k}$, para cualquier

la funci\'on generadora del l\'imite de la sucesi\'on.
\end{Teo}

\begin{Teo}[Teorema de Abel]
Sea $G\left(s\right)=\sum_{k\geq0}a_{k}s^{k}$ para cualquier
$\left\{p_{0},p_{1},\ldots,\right\}$, tales que $p_{k}\geq0$ para
toda $k=0,1,2,\ldots,$. Entonces $G\left(s\right)$ es continua por
la derecha en $s=1$, es decir
\begin{eqnarray*}
lim_{s\uparrow1}G\left(s\right)=\sum_{k\geq0}p_{k}=G\left(\right),
\end{eqnarray*}
sin importar si la suma es finita o no.
\end{Teo}
\begin{Note}
El radio de Convergencia para cualquier PGF es $R\geq1$, entonces,
el Teorema de Abel nos dice que a\'un en el peor escenario, cuando
$R=1$, a\'un se puede confiar en que la PGF ser\'a continua en
$s=1$, en contraste, no se puede asegurar que la PGF ser\'a
continua en el l\'imite inferior $-R$, puesto que la PGF es
sim\'etrica alrededor del cero: la PGF converge para todo
$s\in\left(-R,R\right)$, y no lo hace para $s<-R$ o $s>R$.
Adem\'as nos dice que podemos escribir $G_{X}\left(1\right)$ como
una abreviaci\'on de $lim_{s\uparrow1}G_{X}\left(s\right)$.
\end{Note}

Entonces si suponemos que la diferenciaci\'on t\'ermino a
t\'ermino est\'a permitida, entonces

\begin{eqnarray*}
G_{X}^{'}\left(s\right)&=&\sum_{x=1}^{\infty}xs^{x-1}p_{x}
\end{eqnarray*}

el Teorema de Abel nos dice que
\begin{eqnarray*}
\esp\left(X\right]&=&\lim_{s\uparrow1}G_{X}^{'}\left(s\right):\\
\esp\left[X\right]&=&=\sum_{x=1}^{\infty}xp_{x}=G_{X}^{'}\left(1\right)\\
&=&\lim_{s\uparrow1}G_{X}^{'}\left(s\right),
\end{eqnarray*}
dado que el Teorema de Abel se aplica a
\begin{eqnarray*}
G_{X}^{'}\left(s\right)&=&\sum_{x=1}^{\infty}xs^{x-1}p_{x},
\end{eqnarray*}
estableciendo as\'i que $G_{X}^{'}\left(s\right)$ es continua en
$s=1$. Sin el Teorema de Abel no se podr\'ia asegurar que el
l\'imite de $G_{X}^{'}\left(s\right)$ conforme $s\uparrow1$ sea la
respuesta correcta para $\esp\left[X\right]$.

\begin{Note}
La PGF converge para todo $|s|<R$, para alg\'un $R$. De hecho la
PGF converge absolutamente si $|s|<R$. La PGF adem\'as converge
uniformemente en conjuntos de la forma
$\left\{s:|s|<R^{'}\right\}$, donde $R^{'}<R$, es decir,
$\forall\epsilon>0, \exists n_{0}\in\ent$ tal que $\forall s$, con
$|s|<R^{'}$, y $\forall n\geq n_{0}$,
\begin{eqnarray*}
|\sum_{x=0}^{n}s^{x}\prob\left(X=x\right)-G_{X}\left(s\right)|<\epsilon.
\end{eqnarray*}
De hecho, la convergencia uniforme es la que nos permite
diferenciar t\'ermino a t\'ermino:
\begin{eqnarray*}
G_{X}\left(s\right)=\esp\left[s^{X}\right]=\sum_{x=0}^{\infty}s^{x}\prob\left(X=x\right),
\end{eqnarray*}
y sea $s<R$.
\begin{enumerate}
\item
\begin{eqnarray*}
G_{X}^{'}\left(s\right)&=&\frac{d}{ds}\left(\sum_{x=0}^{\infty}s^{x}\prob\left(X=x\right)\right)=\sum_{x=0}^{\infty}\frac{d}{ds}\left(s^{x}\prob\left(X=x\right)\right)\\
&=&\sum_{x=0}^{n}xs^{x-1}\prob\left(X=x\right).
\end{eqnarray*}

\item\begin{eqnarray*}
\int_{a}^{b}G_{X}\left(s\right)ds&=&\int_{a}^{b}\left(\sum_{x=0}^{\infty}s^{x}\prob\left(X=x\right)\right)ds=\sum_{x=0}^{\infty}\left(\int_{a}^{b}s^{x}\prob\left(X=x\right)ds\right)\\
&=&\sum_{x=0}^{\infty}\frac{s^{x+1}}{x+1}\prob\left(X=x\right),
\end{eqnarray*}
para $-R<a<b<R$.
\end{enumerate}
\end{Note}

\begin{Teo}[Teorema de Convergencia Mon\'otona para PGF]
Sean $X$ y $X_{n}$ variables aleatorias no negativas, con valores
en los enteros, finitas, tales que
\begin{eqnarray*}
lim_{n\rightarrow\infty}G_{X_{n}}\left(s\right)&=&G_{X}\left(s\right)
\end{eqnarray*}
para $0\leq s\leq1$, entonces
\begin{eqnarray*}
lim_{n\rightarrow\infty}P\left(X_{n}=k\right)=P\left(X=k\right),
\end{eqnarray*}
para $k=0,1,2,\ldots.$
\end{Teo}

El teorema anterior requiere del siguiente lema

\begin{Lemma}
Sean $a_{n,k}\in\ent^{+}$, $n\in\nat$ constantes no negativas con
$\sum_{k\geq0}a_{k,n}\leq1$. Sup\'ongase que para $0\leq s\leq1$,
se tiene

\begin{eqnarray*}
a_{n}\left(s\right)&=&\sum_{k=0}^{\infty}a_{k,n}s^{k}\rightarrow
a\left(s\right)=\sum_{k=0}^{\infty}a_{k}s^{k}.
\end{eqnarray*}
Entonces
\begin{eqnarray*}
a_{0,n}\rightarrow a_{0}.
\end{eqnarray*}
\end{Lemma}


%_____________________________________________________________________________________
%
\subsubsection{Teorema de Estabilidad}
%_____________________________________________________________________________________
%

\begin{itemize}
\item[(A1.)] Para $j,j+1\in\left\{1,2,\ldots,K\right\}$
\begin{eqnarray}\label{A1}
\xi_{1},\xi_{2},\ldots,\xi_{K}\textrm{ ,
}\eta_{1},\eta_{2},\ldots,\eta_{K}\textrm{ , }\delta_{j,j+1},
\end{eqnarray}
son mutuamente independientes y sucesiones independientes e
id\'enticamente distribuidas.

\item[(A2.)] Para alg\'un entero $p\geq1$,
\begin{eqnarray}\label{A2}
\esp\left[\xi_{k}^{p+1}\left(1\right)\right]<\infty\textrm{ ,
}\esp\left[\eta_{k}^{p+1}\left(1\right)\right]<\infty\textrm{ y
}\esp\left[\delta_{j,j+1}^{p+1}\left(1\right)\right]<\infty
\end{eqnarray}
para $k=1,\ldots,K$ y para $j,j+1\in\left\{1,2,\ldots,K\right\}$.
\end{itemize}
En el caso particular de un modelo con un solo servidor, $M=1$, se
tiene que si se define

\begin{Def}\label{Def.Ro}
\begin{equation}\label{RoM1}
\rho=\sum_{k=1}^{K}\rho_{k}+max_{1\leq j\leq
K}\left(\frac{\lambda_{j}}{p_{j}\overline{N}}\right)\delta^{*}.
\end{equation}
\end{Def}

\begin{Teo}[Teorema 2.1 \cite{Down}]
Si $\rho<1$, entonces:
\begin{itemize}
\item[i)] Para alguna constante $\kappa_{p}$, y para cada
condici{\'o}n inicial $x\in X$
\begin{equation}\label{Estability.Eq1}
lim_{t\rightarrow\infty}\sup\frac{1}{t}\int_{0}^{t}\esp_{x}\left[|Q\left(s\right)|^{p}\right]ds\leq\kappa_{p},
\end{equation}
donde $p$ es el entero dado en (\ref{A2}).

 \item[ii)]
 \begin{equation}\label{Estability.Eq2}
lim_{t\rightarrow\infty}\esp_{x}\left[Q_{k}\left(t\right)^{r}\right]=\esp\left[Q_{k}\left(0\right)^{r}\right],
\end{equation}
para $r=1,2,\ldots,p$ y $k=1,2,\ldots,K$.
  \item[iii)]  El primer momento converge con raz{\'o}n $t^{p-1}$:
  \begin{equation}\label{Estability.Eq3}
lim_{t\rightarrow\infty}t^{p-1}|\esp_{x}\left[Q_{k}\left(t\right)\right]-\esp\left[Q\left(0\right)\right]|=0.
\end{equation}
\item[iv)] La {\em Ley Fuerte de los grandes n{\'u}meros} se
cumple:
\begin{equation}\label{Estability.Eq4}
lim_{t\rightarrow\infty}\frac{1}{t}\int_{0}^{t}Q_{k}^{r}ds=\esp\left[Q_{k}\left(0\right)^{r}\right],\textrm{
}\prob\textrm{-c.s.}
\end{equation}
para $r=1,2,\ldots,p$ y $k=1,2,\ldots,K$.
\end{itemize}
\end{Teo}


\begin{Def}
El flujo modelo se dice {\em estable} si existe un tiempo fijo
$t_{0}$ tal que $\overline{Q}\left(t\right)=0$, con $t\geq t_{0}$,
para cualquier fluido l{\'\i}mite.
\end{Def}

\begin{Teo}\label{Teorema.2.1}
Suponga que el fluido modelo es estable, y suponga que los
supuestos (\ref{A1}) y (\ref{A2}). Entonces

\begin{itemize}
\item[i)] Para alguna constante $\kappa_{p}$, y para cada
condici\'on inicial $x\in X$
\begin{equation}\label{Estability.Eq1}
lim_{t\rightarrow\infty}\sup\frac{1}{t}\int_{0}^{t}\esp_{x}\left[|Q|^{p}\right]ds\leq\kappa_{p},
\end{equation}
donde $p$ es el entero dado en (\ref{A2}). Si adem\'as se cumple
la condici\'on (\ref{A3}), entonces para cada condici\'on inicial

 \item[ii)] Los momentos transitorios convergen a su estado estacionario:
 \begin{equation}\label{Estability.Eq2}
lim_{t\rightarrow\infty}\esp_{x}\left[Q_{k}\left(t\right)^{r}\right]=\esp_{\pi}\left[Q_{k}\left(0\right)^{r}\right],
\end{equation}
para $r=1,2,\ldots,p$ y $k=1,2,\ldots,K$. Donde $\pi$ es la
probabilidad invariante para $\mathbb{X}$.

  \item[iii)]  El primer momento converge con raz\'on $t^{p-1}$:
  \begin{equation}\label{Estability.Eq3}
lim_{t\rightarrow\infty}t^{p-1}|\esp_{x}\left[Q_{k}\left(t\right)\right]-\esp_{\pi}\left[Q\left(0\right)\right]=0.
\end{equation}

\item[iv)] La {\em Ley Fuerte de los grandes n\'umeros} se cumple:
\begin{equation}\label{Estability.Eq4}
lim_{t\rightarrow\infty}\frac{1}{t}\int_{0}^{t}Q_{k}^{r}ds=\esp_{\pi}\left[Q_{k}\left(0\right)^{r}\right],\textrm{
}\prob\textrm{-c.s.}
\end{equation}
para $r=1,2,\ldots,p$ y $k=1,2,\ldots,K$.
\end{itemize}
\end{Teo}
\begin{Teo}\label{Teorema2.2}
Suponga que el fluido modelo es inestable en el sentido de que
para alguna $\epsilon_{0},c_{0}\geq0$,
\begin{equation}\label{Eq.Inestability}
|Q\left(T\right)|\geq\epsilon_{0}T-c_{0}\textrm{,   }T\geq0,
\end{equation}
para cualquier condici\'on inicial $Q\left(0\right)$, con
$|Q\left(0\right)|=1$. Entonces para cualquier $0<q\leq1$, existe
$B<0$ tal que para cualquier $|x|\geq B$,
\begin{equation}
\prob_{x}\left\{\mathbb{X}\rightarrow\infty\right\}\geq q.
\end{equation}
\end{Teo}

\begin{Def}
El flujo modelo se dice {\em estable} si existe un tiempo fijo
$t_{0}$ tal que $\overline{Q}\left(t\right)=0$, con $t\geq t_{0}$,
para cualquier fluido l{\'\i}mite.
\end{Def}

\begin{Teo}\label{Teorema.2.1}
Suponga que el fluido modelo es estable, y suponga que los
supuestos (\ref{A1}) y (\ref{A2}). Entonces

\begin{itemize}
\item[i)] Para alguna constante $\kappa_{p}$, y para cada
condici\'on inicial $x\in X$
\begin{equation}\label{Estability.Eq1}
lim_{t\rightarrow\infty}\sup\frac{1}{t}\int_{0}^{t}\esp_{x}\left[|Q|^{p}\right]ds\leq\kappa_{p},
\end{equation}
donde $p$ es el entero dado en (\ref{A2}). Si además se cumple la
condici\'on (\ref{A3}), entonces para cada condici\'on inicial

 \item[ii)] Los momentos transitorios convergen a su estado estacionario:
 \begin{equation}\label{Estability.Eq2}
lim_{t\rightarrow\infty}\esp_{x}\left[Q_{k}\left(t\right)^{r}\right]=\esp_{\pi}\left[Q_{k}\left(0\right)^{r}\right],
\end{equation}
para $r=1,2,\ldots,p$ y $k=1,2,\ldots,K$. Donde $\pi$ es la
probabilidad invariante para $\mathbb{X}$.

  \item[iii)]  El primer momento converge con raz\'on $t^{p-1}$:
  \begin{equation}\label{Estability.Eq3}
lim_{t\rightarrow\infty}t^{p-1}|\esp_{x}\left[Q_{k}\left(t\right)\right]-\esp_{\pi}\left[Q\left(0\right)\right]=0.
\end{equation}

\item[iv)] La {\em Ley Fuerte de los grandes números} se cumple:
\begin{equation}\label{Estability.Eq4}
lim_{t\rightarrow\infty}\frac{1}{t}\int_{0}^{t}Q_{k}^{r}ds=\esp_{\pi}\left[Q_{k}\left(0\right)^{r}\right],\textrm{
}\prob\textrm{-c.s.}
\end{equation}
para $r=1,2,\ldots,p$ y $k=1,2,\ldots,K$.
\end{itemize}
\end{Teo}

\begin{Teo}\label{Teorema2.2}
Suponga que el fluido modelo es inestable en el sentido de que
para alguna $\epsilon_{0},c_{0}\geq0$,
\begin{equation}\label{Eq.Inestability}
|Q\left(T\right)|\geq\epsilon_{0}T-c_{0}\textrm{,   }T\geq0,
\end{equation}
para cualquier condici\'on inicial $Q\left(0\right)$, con
$|Q\left(0\right)|=1$. Entonces para cualquier $0<q\leq1$, existe
$B<0$ tal que para cualquier $|x|\geq B$,
\begin{equation}
\prob_{x}\left\{\mathbb{X}\rightarrow\infty\right\}\geq q.
\end{equation}
\end{Teo}


%_____________________________________________________________________________________
%
\subsubsection{Consecuencias del Teorema}
%_____________________________________________________________________________________
%
En el caso particular de un modelo con un solo servidor, $M=1$, se
tiene que si se define
\begin{Def}\label{Def.Ro}
\begin{equation}\label{RoM1}
\rho=\sum_{k=1}^{K}\rho_{k}+\max_{1\leq j\leq
K}\left(\frac{\lambda_{j}}{p_{j}\overline{N}}\right)\delta^{*}.
\end{equation}
\end{Def}
donde si
\begin{itemize}
\item Si $\rho<1$, entonces la red es estable, es decir el teorema
(\ref{Teorema.2.1}) se cumple. \item De lo contrario, es decir, si
$\rho>1$ entonces la red es inestable, es decir, el teorema
(\ref{Teorema.2.2}).
\end{itemize}

En el caso particular de un modelo con un solo servidor, $M=1$, se
tiene que si se define
\begin{Def}\label{Def.Ro}
\begin{equation}\label{RoM1}
\rho=\sum_{k=1}^{K}\rho_{k}+\max_{1\leq j\leq
K}\left(\frac{\lambda_{j}}{p_{j}\overline{N}}\right)\delta^{*}.
\end{equation}
\end{Def}
donde si
\begin{itemize}
\item Si $\rho<1$, entonces la red es estable, es decir el teorema
(\ref{Teorema.2.1}) se cumple. \item De lo contrario, es decir, si
$\rho>1$ entonces la red es inestable, es decir, el teorema
(\ref{Teorema.2.2}).
\end{itemize}

%_____________________________________________________________________
%\subsection{AP\'ENDICE: Simulaci\'on Fantasma}
%_____________________________________________________________________

El metro de la Ciudad de M\'exico es uno de los sistemas m\'as
grandes del mundo en cuanto al n\'umero de pasajeros que
transporta, como a la longitud del mismo. El sistema es similar a
los de otros pa\'ises, en los cuales por una v\'ia circula un
\'unico tren cuyo or\'igen y destino siempre el mismo. Una l\'inea
consiste de dos v\'ias, una en sentido opuesto a la otra.

El metro de la Ciudad de M\'exico cuenta actualmente con 11
l\'ineas. Cada l\'inea tiene tres distintos tipos de estaciones:
Terminales, Normales y de Correspondencia. El n\'umero de
estaciones por l\'inea es variable: la l\'inea dos es la m\'as
grande con 24 estaciones, mientras que la l\'inea $A$ s\'olo
cuenta con 10 estaciones.  La mayor\'ia de las estaciones de
correspondencia unen a dos l\'ineas, pero hay algunas, como la de
Pantitl\'an, que son estaci\'on terminal de 4 distintas.

Actualmente se cuenta con 355 trenes de 6 o 9 vagones, cuya
capacidad m\'axima de pasajeros es de 1020 y 1530 respectivamente.
Cada l\'inea tiene asignado en principio un n\'umero fijo de
trenes: la l\'inea 3 cuenta con 58, mientras que la l\'inea 4
dispone solamente de 13.

Dependiendo de la afluencia se divide el intervalo de tiempo en el
que funciona el metro en cuatro intervalos distintos, dos para las
horas pico y dos para las horas con menor demanda.

Para cada intervalo, cada l\'inea tiene definida la frecuencia con
la que circulan los trenes, es decir, la cantidad de trenes por
hora.

%_____________________________________________________________________
\subsubsection{Notaci\'on}
%_____________________________________________________________________
\begin{itemize}
\item El metro cuenta con un total de $\mathcal{L}$ l\'ineas.

\item Cada d\'ia se divide en $s$ periodos de tiempo, la longitud
de cada segmento es $T_{s}$ y
$\tau_{s}=T_{1}+T_{2}+\cdots+T_{s-1}$.

\item Cada l\'inea del metro tiene asociada un tiempo establecido
entre dos trenes consecutivos, variables de control, los cuales se
denotar\'an por $\mu_{l,s}$,
$\mu=\left(\mu_{l,s}\right)_{l,s=1}^{\mathcal{M},S}$.

\item El par $\left(o,d\right)$ denotar\'a una \'unica sucesi\'on
de plataformas por origen-destino.

\item Denotaremos por $p$ cualquier plataforma de la $l$-\'esima
l\'inea, $L_{l}$. \item Para la l\'inea $l_{l}$ sea $\mathcal{M}$
el n\'umero total de l\'ineas con las que se intersectan con la
l\'inea $L_{l}$ en la plataforma $p$.

\item $\left(o,d\right)$ es el viaje de un pasajero de la
plataforma $o$, a la plataforma $d$.

\item El t\'ermino $\lambda\left(o,d\right)$ es la tasa demanda
para el recorrido $\left(o,d\right)$.

\item $V_{k}\left(p\right)$ es el tiempo de salida del $k$-\'esimo
tren de la plataforma $p$.

\item $S_{k}^{m}$ es el tiempo de llegada del $k$-\'esimo tren de
la plataforma $p_{m}$.

\item $\left\{\delta_{i}\right\}_{i\geq1}$ es una sucesi\'on de
variables aleatorias independientes e id\'enticamente
distribuidas, con media y varianza conocidas, que modelar\'an los
tiempos de espera del tren en la plataforma ocasionado por el
conductor, provocada por los tiempos de viaje m\'as la apertura y
cierre de puertas que son manejadas por el conductor del tren.

\item $P_{k}^{m}\left(o,d\right)$ es el n\'umero de pasajeros de
transferencia en el $k$-\'esimo tren de la l\'inea $m$, para
cualquier plataforma $o\in L_{m}$.

\item Dado $m\geq1$, $P_{k}^{m}\left(o_{m},d\right)$ es el
n\'umero de pasajeros de transferencia en el $k$-\'esimo tren de
la l\'inea $m$, para la plataforma $o_{m}\in L_{m}$.

\item $\hat{D}\left(p_{k},p_{k+1}\right)$ es la distancia entre
las plataformas $k$ y $k+1$.

\item $v$ es la velocidad promedio del tren, es la misma para
todas las l\'ineas en cualquier intervalo de tiempo $s$.

\item $D_{o}\left(p_{m},p\right)$ es el conjunto de los destinos a
partir del origen $o$ que requieren un transbordo de $p_{1}$ a
$p$.

 \item $\eta_{m}\left(\cdot\right)$ es el proceso de conteo de
llegadas de los trenes de la l\'inea $L_{m}$ en la estaci\'on
donde est\'a ubicada la plataforma $p_{m}$.

\item $N_{o}\left(\cdot\right)$ es un Proceso Poisson acumulado,
de todas las llegadas del or\'igen $p$ con tiempos de llegada
$S_{k}^{o}$.

\end{itemize}

%_____________________________________________________________________
\subsubsection{Supuestos Te\'oricos}
%_____________________________________________________________________

\begin{Sup}\label{Sup1} Los pasajeros con destino la plataforma $d$, lleg\'an al anden origen, $o$, conforme a un proceso Poisson con par\'ametro $\lambda\left(o,d \right)$. Todos los procesos de llegada Poisson son independientes entre s\'i e independientes de los procesos de salida de los trenes en la plataforma.
\end{Sup}
\begin{Sup}\label{Sup2} Para cada plataforma $p\in L_{l}$ y para cada l\'inea $L_{l}$, el tiempo de salida del tren en la plataforma inicial $p_{1}\in L_{l}$, sigue la siguiente forma recursiva:
\begin{equation}\label{Eq.TiemposSalida}
V_{j}\left(p_{1}\right)=V_{j-1}\left(p_{1}\right)+\mu_{l}\left(1+\delta_{j}\left(p\right)\right)
\end{equation}
\end{Sup}
Para $k=0,1,2,\ldots$ y plataformas $p_{k}$ y $p_{k+1}$, el tiempo
de salida del $j$-\'esimo tren de la plataforma $p_{k+1}$ est\'a
dada por
\begin{equation}\label{Eq.DepTimePlatform}
V_{j}\left(p_{k+1}\right)=V_{j}\left(p_{k}\right)+\frac{D_{k}\left(p_{k},p_{k+1}\right)}{v}+\mu_{l}\delta_{j}\left(p_{k+1}\right)
\end{equation}
donde  $\left\{\delta_{i}\right\}_{i\geq1}$ es una sucesi\'on de
variables aleatorias independientes e id\'enticamente
distribuidas, con media y varianza conocidas. Las
$\left\{\delta_{j}\right\}_{j\geq1}$ modelan las fluctuaciones en
los tiempos de interarribo provocada por los tiempos de viaje
m\'as los tiempos provocados por el apertura y cierre de puertas
que son manejadas por el conductor del tren.

Los tiempos de llegada de los pasajeros en cualquier plataforma
est\'an completamente determinadas por las llegadas de los
pasajeros, conforme a su correspondiente proceso Poisson, desde
fuera, y su traslado a lo largo de la red para lograr su destino.

Para un valor fijo de la frecuencia con la que salen los trenes en
las plataformas, la distribuci\'on de los tiempos de salida en el
resto de las plataformas queda completamente determinada por los
tiempos de salida en la estaci\'on inicial, la sucesi\'on de
tiempos de viaje de plataforma a plataforma, as\'i como del
n\'umero de las mismas para cada l\'inea $L_{l}$.

De igual manera los tiempos de llegada de los pasajeros a
cualquier plataforma est\'an completamente determinados por el
flujo, de pasajeros, que se comporta como un proceso Poisson con
par\'ametro $\lambda\left(o,d\right)$, provenientes de fuera del
sistema, adem\'as de su movimiento a lo largo de la red para
alcanzar sus destinos.

Si $p_{i}\in L_{l}$ y $p_{k}\notin L_{l}$, entonces los procesos
de salida correspondientes se asume que son independientes bajo el
supuesto de no sincronizaci\'on.

Los procesos de llegada de los pasajeros en cada plataforma
est\'an compuestos por un proceso Poisson de pasajeros que abordan
desde fuera del sistema, m\'as los que llegan de otras l\'i­neas
de transferencia.

Los pasajeros de primer orden son aquellos que provienen de la
l\'inea $L_{m}$, en el $k$-\'esimo tren, que transbordan por
primera vez en la plataforma $p\in L_{l}$, es decir
\begin{eqnarray*}
\sum_{o\in L_{m}}P_{k}^{m}\left(o,d\right)\indora_{\left\{d\in
D_{o}\left(p_{m},p\right)\right\}}
\end{eqnarray*}

Para un valor fijo de la frecuencia con la que salen los trenes en
las plataformas, $\mu_{l}$, la distribuci\'on de los tiempos de
salida en el resto de las plataformas queda completamente
determinada por los tiempos de salida en la estaci\'on inicial, la
sucesi\'on de tiempos de viaje de plataforma a plataforma, as\'i
como del n\'umero de las mismas para cada l\'inea $L_{l}$.

%_____________________________________________________________________
\subsubsection{Proposiciones}
%__________________________________________________________________________________

\begin{Prop}
Sea $p$ plataforma en $L_{l}$, ambas fijas. Sea
$\left(L_{m}\right)_{m\geq1}$ colecci\'on de l\'ineas con
correspondencia en $p$. Para cada $m$, sea $p_{m}$ una plataforma
de correspondencia con la l\'inea $L_{l}$ en la estaci\'on donde
est\'a ubicada la plataforma $p$. Para cada $p_{m}$ los tiempos de
llegada entre dos trenes consecutivos est\'an dados por
$T_{k}\left(p_{m}\right)=S_{k}\left(p_{m}\right)-S_{k-1}\left(p_{m}\right)$,
que se abreviar\'a por $T_{k}^{m}=S_{k}^{m}-S_{k-1}^{m}$. El
n\'umero de pasajeros de transferencia de primer orden en la
plataforma $p$, procedentes de $p_{m}$, con tiempos de llegada
$S_{k}^{m}$, cumplen con la condici\'on

\begin{equation}\label{Eq.EspCond2}
\left.\esp\left[P_{k}^{m}\right|T_{k}^{m}\right]=\sum_{n=1}^{L}\sum_{o\in
L_{n}}\lambda\left(o,d\right)\mu_{n}\left[\esp\left[\eta_{n}\left(V_{j}\right)\right]-\esp\left[\eta_{n}\left(V_{j-1}\right)\right]\right]\indora_{d\in
D_{o}\left(p_{m},p\right)}\textrm{.}
\end{equation}
donde $D_{o}\left(p_{m},p\right)$ es el conjunto de los destinos a
partir del origen $o$ que requieren un transbordo de $p_{m}\in
L_{m}$ a $p\in L_{l}$ y $\eta_{m}\left(\cdot\right)$ es el proceso
de conteo de llegadas de los trenes de la l\'inea $L_{m}$ en la
estaci\'on donde est\'a ubicada la plataforma $p_{m}$.
\end{Prop}
\begin{Dem}


Sea $m\geq1$ fija y sea $p_{m}$ plataforma en $L_{m}$ que hace
correspondencia con $p$ en la estaci\'on de transferencia donde
tambi\'en est\'a ubicada la plataforma $p\in L_{l}$. Se sabe que
para cualquier $o\in L_{m}$ los tiempos de salida del $k$-\'esimo
tren en la misma, son de la forma

\begin{eqnarray*}
V_{k}\left(o\right)&=&V_{k}\left(p_{m}\right)-\frac{\hat{D}\left(o,p_{m}\right)}{v}-\mu_{m}\delta_{k}\left(o,p_{m}\right)\\
&=&S_{k}\left(p_{m}\right)-\frac{\hat{D}\left(o,p_{m}\right)}{v}-\mu_{m}\delta_{k}\left(o,p_{m}\right)\textrm{.}
\end{eqnarray*}

Dado que el n\'umero de pasajeros que abordan el $k$-\'esimo tren
en $o\in L_{m}$ con destino $d\in D_{o}\left(p_{m},p\right)$ es el
n\'umero de arribos Poisson con intensidad
$\lambda\left(o,d\right)$ en
$\left(V_{k-1}\left(o\right),V_{k}\left(o\right)\right]$, entonces
dado $o_{m}\in L_{m}$ se tiene que:
\begin{eqnarray*}
V_{k}\left(p_{m}\right)&=&V_{k}\left(p_{m-1}\right)+\frac{\hat{D}\left(p_{m-1},p_{m}\right)}{v}+\mu_{m}\delta_{k}\left(p_{m}\right)\\
&=&\left[V_{k}\left(p_{m-2}\right)+\frac{\hat{D}\left(p_{m-2},p_{m-1}\right)}{v}+\mu_{m}\delta_{k}\left(p_{m-1}\right)\right]\\
&+&\frac{D\left(p_{m-1},p_{m}\right)}{v}+\mu_{m}\delta_{k}\left(p_{m}\right)\\
&=&V_{k}\left(p_{m-2}\right)+\frac{\hat{D}\left(p_{m-2},p_{m}\right)}{v}+\mu_{m}\delta_{k}\left(p_{m-1},p_{m}\right)\\
&=&\left[V_{k}\left(p_{m-3}\right)+\frac{\hat{D}\left(p_{m-3},p_{m-2}\right)}{v}+\mu_{m}\delta_{k}\left(p_{m-2}\right)\right]\\
&+&\frac{\hat{D}\left(p_{m-2},p_{m}\right)}{v}+\mu_{m}\delta_{k}\left(p_{m-1},p_{m}\right)\\
&=&V_{k}\left(p_{m-3}\right)+\frac{\hat{D}\left(p_{m-3},p_{m}\right)}{v}+\mu_{m}\delta_{k}\left(p_{m-2},p_{m}\right)\\
&\vdots&\\
&=&V_{k}\left(p_{1}\right)+\frac{\hat{D}\left(p_{1},p_{m}\right)}{v}+\mu_{m}\delta_{k}\left(p_{1},p_{m}\right)\\
&=&V_{k}\left(o\right)+\frac{\hat{D}\left(o_{m},p_{m}\right)}{v}+\mu_{m}\delta_{k}\left(o_{m},p_{m}\right)
\end{eqnarray*}
Entonces se tiene que
\begin{equation}
V_{k}\left(o_{m}\right)=V_{k}\left(p_{m}\right)-\frac{\hat{D}\left(o_{m},p_{m}\right)}{v}-\mu_{m}\delta_{k}\left(o_{m},p_{m}\right)
\end{equation}

y como $V_{k}\left(p_{m}\right)=S_{k}^{m}$
\begin{equation}
V_{k}\left(o_{m}\right)=S_{k}^{m}-\frac{\hat{D}\left(o_{m},p_{m}\right)}{v}-\mu_{m}\delta_{k}\left(o_{m},p_{m}\right)
\end{equation}

Por el supuesto (\ref{Sup1}) los pasajeros con destino la
plataforma $d$, lleg\'an a la plataforma $o_{m}$, conforme a un
proceso Poisson con par\'ametro $\lambda\left(o,d \right)$.
\begin{eqnarray*}
&&\esp\left[P_{k}^{m}\left(o_{m},d\right)\right]=\lambda\left(o_{m},d\right)\esp\left[V_{k}\left(o_{m}\right)-V_{k-1}\left(o_{m}\right)\right]\\
&=&\lambda\left(o_{m},d\right)\esp\left[\left(V_{k}\left(p_{m}\right)-V_{k-1}\left(p_{m}\right)\right)+\mu_{m}\left[\delta_{k-1}\left(o_{m},p_{m}\right)-\delta_{k}\left(o_{m},p_{m}\right)\right]\right]\\
&=&\lambda\left(o_{m},d\right)\esp\left[T_{k}\left(p_{m}\right)+\mu_{m}\left[\delta_{k-1}\left(o_{m},p_{m}\right)-\delta_{k}\left(o_{m},p_{m}\right)\right]\right]
\end{eqnarray*}
es decir
\begin{equation}
\esp\left[P_{k}^{m}\left(o_{m},d\right)\right]=\lambda\left(o_{m},d\right)\esp\left[T_{k}\left(p_{m}\right)+\mu_{m}\left[\delta_{k-1}\left(o_{m},p_{m}\right)-\delta_{k}\left(o_{m},p_{m}\right)\right]\right]\textrm{.}
\end{equation}
Lo que se quiere determinar es el valor esperado del n\'umero de
pasajeros que llegar\'an a la plataforma $p_{m}$ entre dos trenes
consecutivos, entonces lo que hacemos es utilizar una propiedad de
la esperanza condicional

\begin{eqnarray*}
&&\esp\left[P_{k}^{m}\left(o_{m},d\right)\right]=\esp\left.\left[\esp\left[P_{k}^{m}\left(o_{m},d\right)\right|T_{k}^{m}\right]\right]\\
&=&\esp\left.\left[\esp\left[\lambda\left(o_{m},d\right)\left\{T_{k}^{m}+\mu_{m}\left[\delta_{k-1}\left(o_{m},p_{m}\right)-\delta_{k}\left(o_{m},p_{m}\right)\right]\right\}\right|T_{k}^{m}\right]\right]\\
&=&\esp\left.\left[\lambda\left(o_{m},d\right)\esp\left[T_{k}^{m}+\mu_{m}\left[\delta_{k-1}\left(o_{m},p_{m}\right)-\delta_{k}\left(o_{m},p_{m}\right)\right]\right|T_{k}^{m}\right]\right]\\
&=&\esp\left.\left.\left[\lambda\left(o_{m},d\right)\esp\left[T_{k}^{m}\right|T_{k}^{m}\right]+\lambda\left(o_{m},d\right)\mu_{m}\esp\left[\delta_{k}\left(o_{m},p_{m}\right)-\delta_{k-1}\left(o_{m},p_{m}\right)\right]\right|T_{k}^{m}\right]\\
&=&\esp\left[\lambda\left(o_{m},d\right)T_{k}^{m}+\lambda\left(o_{m},d\right)\mu_{m}\esp\left[\delta_{k}\left(o_{m},p_{m}\right)-\delta_{k-1}\left(o_{m},p_{m}\right)\right]\right]\\
&=&\esp\left[\lambda\left(o_{m},d\right)T_{k}^{m}\right]
\end{eqnarray*}
por tanto
\begin{equation}
\esp\left[P_{k}^{m}\left(o_{m},d\right)\right]=\lambda\left(o_{m},d\right)\mu_{m}
\end{equation}

para todos los pasajeros que tienen como origen una plataforma en
la l\'inea $L_{m}$ y un destido $d$ que requiere hacer una
transferencia de $p_{m}$ a $p$. La pen\'ultima igualdad es cierta
dado que los procesos Poisson con intensidad
$\lambda\left(o_{m},d\right)$ en el nodo or\'igen, $o_{m}\in
L_{m}$ son independientes de los tiempos de salida de los trenes,
adem\'as del hecho de que las $\delta_{k}\left(o_{m},p_{m}\right)$
son variables aleatorias independientes e id\'enticamente
distribuidas.

Por el supuesto (\ref{Sup1}) los pasajeros con destino la
plataforma $d$, lleg\'an a la plataforma $o_{m}$, conforme a un
proceso Poisson con par\'ametro $\lambda\left(o,d \right)$.
\begin{eqnarray*}
&&\esp\left[P_{k}^{m}\left(o_{m},d\right)\right]=\lambda\left(o_{m},d\right)\esp\left[V_{k}\left(o_{m}\right)-V_{k-1}\left(o_{m}\right)\right]\\
&=&\lambda\left(o_{m},d\right)\esp\left[\left(V_{k}\left(p_{m}\right)-V_{k-1}\left(p_{m}\right)\right)+\mu_{m}\left[\delta_{k-1}\left(o_{m},p_{m}\right)-\delta_{k}\left(o_{m},p_{m}\right)\right]\right]\\
&=&\lambda\left(o_{m},d\right)\esp\left[T_{k}\left(p_{m}\right)+\mu_{m}\left[\delta_{k-1}\left(o_{m},p_{m}\right)-\delta_{k}\left(o_{m},p_{m}\right)\right]\right]
\end{eqnarray*}
es decir
\begin{equation}
\esp\left[P_{k}^{m}\left(o_{m},d\right)\right]=\lambda\left(o_{m},d\right)\esp\left[T_{k}\left(p_{m}\right)+\mu_{m}\left[\delta_{k-1}\left(o_{m},p_{m}\right)-\delta_{k}\left(o_{m},p_{m}\right)\right]\right]\textrm{.}
\end{equation}
Lo que se quiere determinar es el valor esperado del n\'umero de
pasajeros que llegar\'an a la plataforma $p_{m}$ entre dos trenes
consecutivos, entonces lo que hacemos es utilizar una propiedad de
la esperanza condicional
\begin{eqnarray*}
&&\esp\left[P_{k}^{m}\left(o_{m},d\right)\right]=\esp\left.\left[\esp\left[P_{k}^{m}\left(o_{m},d\right)\right|T_{k}^{m}\right]\right]\\
&=&\esp\left.\left[\esp\left[\lambda\left(o_{m},d\right)\left\{T_{k}^{m}+\mu_{m}\left[\delta_{k-1}\left(o_{m},p_{m}\right)-\delta_{k}\left(o_{m},p_{m}\right)\right]\right\}\right|T_{k}^{m}\right]\right]\\
&=&\esp\left.\left[\lambda\left(o_{m},d\right)\esp\left[T_{k}^{m}+\mu_{m}\left[\delta_{k-1}\left(o_{m},p_{m}\right)-\delta_{k}\left(o_{m},p_{m}\right)\right]\right|T_{k}^{m}\right]\right]\\
&=&\esp\left.\left.\left[\lambda\left(o_{m},d\right)\esp\left[T_{k}^{m}\right|T_{k}^{m}\right]+\lambda\left(o_{m},d\right)\mu_{m}\esp\left[\delta_{k}\left(o_{m},p_{m}\right)-\delta_{k-1}\left(o_{m},p_{m}\right)\right]\right|T_{k}^{m}\right]\\
&=&\esp\left[\lambda\left(o_{m},d\right)T_{k}^{m}+\lambda\left(o_{m},d\right)\mu_{m}\esp\left[\delta_{k}\left(o_{m},p_{m}\right)-\delta_{k-1}\left(o_{m},p_{m}\right)\right]\right]\\
&=&\esp\left[\lambda\left(o_{m},d\right)T_{k}^{m}\right]
\end{eqnarray*}
por tanto
\begin{equation}
\esp\left[P_{k}^{m}\left(o_{m},d\right)\right]=\lambda\left(o_{m},d\right)\mu_{m}
\end{equation}
para todos los pasajeros que tienen como origen una plataforma en la l\'inea $L_{m}$ y un destido $d$ que requiere hacer una transferencia de $p_{m}$ a $p$. La pen\'ultima igualdad es cierta dado que los procesos Poisson con intensidad $\lambda\left(o_{m},d\right)$ en el nodo or\'igen, $o_{m}\in L_{m}$ son independientes de los tiempos de salida de los trenes, adem\'as del hecho de que las $\delta_{k}\left(o_{m},p_{m}\right)$ son variables aleatorias independientes e id\'enticamente distribuidas.%_____________________________________________________________________

Sea $L_{m}$ cualquier l\'inea de correspondencia con $L_{l}$ en la
misma estaci\'on donde est\'a ubicada $p$.

Sea $o$ plataforma cualesquiera en $L_{m}$ fija. El n\'umero de
pasajeros que abordar\'an el $j$-\'esimo tren de la l\'inea
$L_{l}$ en la plataforma $p$, es el total de arribos Poisson de
pasajeros que lleguen a la misma entre el $j-1$ y el $j$-\'esimo
tren.

Tales pasajeros vienen en el $k$-\'esimo tren de la l\'inea
$L_{m}$ que cambian a $L_{l}$ en la plataforma $p_{m}$. El total
de trenes que llegan a $p_{m}$ para subirse al $j$-\'esimo tren en
$p$ se estima de la siguiente manera:

Si definimos $\eta_{m}\left(\cdot\right)$ como el proceso de
conteo de llegadas de los trenes de la l\'inea $L_{m}$ en la
estaci\'on donde est\'a ubicada la plataforma $p_{m}$.

Entonces $\eta_{m}\left(V_{j-1}\left(p_{m}\right)\right)$ es el
n\'umero de trenes que llegan a $p_{m}$ al tiempo $V_{j-1}$,
an\'alogamente para
$\eta_{m}\left(V_{j}\left(p_{m}\right)\right)$. Entonces el total
de estos pasajeros, $\hat{P}_{j}^{m}\left(o,d\right)$ se puede
estimar por:
\begin{equation}
\hat{P}_{j}^{m}\left(o,d\right)=\sum_{k=\eta_{m}\left(V_{j-1}\left(p_{m}\right)\right)+1}^{\eta_{m}\left(V_{j}\left(p_{m}\right)\right)}P_{k}^{m}\left(o,d\right).
\end{equation}

Sea
$Y_{j}\left(p_{m}\right)=V_{j}\left(p_{m}\right)-V_{j-1}\left(p_{m}\right)$,
y  entonces

\begin{eqnarray*}
&&\left.\esp\left[\hat{P}_{j}^{m}\left(o,d\right)\right|Y_{j}\left(p\right)\right]=\esp\left[\left.\sum_{k=\eta_{m}\left(V_{j-1}\left(p_{m}\right)\right)+1}^{\eta_{m}\left(V_{j}\left(p_{m}\right)\right)}P_{k}^{m}\left(o,d\right)\right|Y_{j}\left(p\right)\right]\\
&=&\esp\left[\left.\sum_{k=\eta_{m}\left(V_{j-1}\left(p_{m}\right)\right)+1}^{\eta_{m}\left(V_{j}\left(p_{m}\right)\right)}P_{k}^{m}\left(o,d\right)\right|Y_{j}\left(p_{m}\right)\right]\\
&=&\esp\left[\left.\left(\sum_{k=1}^{\eta_{m}\left(V_{j}\left(p_{m}\right)\right)}P_{k}^{m}\left(o,d\right)-\sum_{k=1}^{\eta_{m}\left(V_{j-1}\left(p_{m}\right)\right)}P_{k}^{m}\left(o,d\right)\right)\right|Y_{j}\left(p_{m}\right)\right]
\end{eqnarray*}

\begin{eqnarray}
\left.\esp\left[\hat{P}_{j}^{m}\left(o,d\right)\right|Y_{j}\left(p\right)\right]&=&\esp\left[\left.\sum_{k=1}^{\eta_{m}\left(V_{j}\left(p_{m}\right)\right)}P_{k}^{m}\left(o,d\right)\right|Y_{j}\left(p_{m}\right)\right]\\
&-&\esp\left[\left.\sum_{k=1}^{\eta_{m}\left(V_{j-1}\left(p_{m}\right)\right)+1}P_{k}^{m}\left(o,d\right)\right|Y_{j}\left(p_{m}\right)\right]\textrm{.}
\end{eqnarray}

Dado que
$\sigma\left(Y_{j}\left(p_{m}\right)\right)\subset\sigma\left(\eta_{m}\left(V_{j}\left(p_{m}\right)\right)\right)$,
entonces se tiene que

\begin{eqnarray*}
&&\esp\left[\left.\sum_{k=1}^{\eta_{m}\left(V_{j}\left(p_{m}\right)\right)}P_{k}^{m}\left(o,d\right)\right|Y_{j}\left(p_{m}\right)\right]=\esp\left.\left[\esp\left[\left.\sum_{k=1}^{\eta_{m}\left(V_{j}\left(p_{m}\right)\right)}P_{k}^{m}\left(o,d\right)\right|Y_{j}\left(p_{m}\right)\right]\right|\eta_{m}\left(V_{j}\right)\right]\\
&=&\esp\left.\left[\esp\left[\left.\sum_{k=1}^{\eta_{m}\left(V_{j}\left(p_{m}\right)\right)}P_{k}^{m}\left(o,d\right)\right|\eta_{m}\left(V_{j}\right)\right]\right|Y_{j}\left(p_{m}\right)\right]\\
&=&\esp\left.\left[\left.\sum_{l\geq1}\esp\left[\sum_{k=1}^{\eta_{m}\left(V_{j}\left(p_{m}\right)\right)}P_{k}^{m}\left(o,d\right)\right|\eta_{m}\left(V_{j}\left(p_{m}\right)\right)=l\right]\prob\left[\eta_{m}\left(V_{j}\left(p_{m}\right)\right)=l\right]\right|Y_{j}\left(p_{m}\right)\right]\\
&=&\esp\left.\left[\sum_{l\geq1}\esp\left[\sum_{k=1}^{l}P_{k}^{m}\left(o,d\right)\right]\prob\left[\eta_{m}\left(V_{j}\right)=l\right]\right|Y_{j}\left(p_{m}\right)\right]\\
&=&\esp\left.\left[\sum_{l\geq1}l\esp\left[P_{k}^{m}\left(o,d\right)\right]\prob\left[\eta_{m}\left(V_{j}\right)=l\right]\right|Y_{j}\left(p_{m}\right)\right]\\
&=&\esp\left.\left[\esp\left[P_{k}^{m}\left(o,d\right)\right]\sum_{l\geq1}l\prob\left[\eta_{m}\left(V_{j}\right)=l\right]\right|Y_{j}\left(p_{m}\right)\right]\\
&=&\esp\left.\left[\esp\left[P_{k}^{m}\left(o,d\right)\right]\esp\left[\eta_{m}\left(V_{j}\right)\right]\right|Y_{j}\left(p_{m}\right)\right]\\
&=&\esp\left[P_{k}^{m}\left(o,d\right)\right]\esp\left[\eta_{m}\left(V_{j}\right)\right]
\end{eqnarray*}

Es decir
\begin{equation}\label{EqEsp1Op3}
\esp\left.\left[\sum_{k=1}^{\eta_{m}\left(V_{j}\left(p_{m}\right)\right)}P_{k}^{m}\left(o,d\right)\right|Y_{j}\left(p\right)\right]=\esp\left[P_{k}^{m}\left(o,d\right)\right]\esp\left[\eta_{m}\left(V_{j}\right)\right]
\end{equation}
entonces procediendo de manera an\'aloga para
$\eta_{m}\left(V_{j-1}\right)$, se tiene que

\begin{equation}\label{EqEsp2Op3}
\esp\left.\left[\sum_{k=1}^{\eta_{m}\left(V_{j-1}\right)}P_{k}^{m}\left(o,d\right)\right|Y_{j}\left(p\right)\right]=\esp\left[P_{k}^{m}\left(o,d\right)\right]\esp\left[\eta_{m}\left(V_{j-1}\right)\right]\textrm{.}
\end{equation}

De las ecuaciones (\ref{EqEsp1Op3}) y (\ref{EqEsp2Op3}) se tiene
que:
\begin{eqnarray*}
\esp\left.\left[P_{j}^{m}\left(o,d\right)\right|Y_{j}\left(p\right)\right]&=&\esp\left[P_{k}^{m}\left(o,d\right)\right]\esp\left[\eta_{m}\left(V_{j}\right)\right]-\esp\left[P_{k}^{m}\left(o,d\right)\right]\esp\left[\eta_{m}\left(V_{j-1}\right)\right]\\
&=&\esp\left[P_{k}^{m}\left(o,d\right)\right]\left(\esp\left[\eta_{m}\left(V_{j}\right)\right]-\esp\left[\eta_{m}\left(V_{j-1}\right)\right]\right)
\end{eqnarray*}


por lo tanto, se tiene que
\begin{eqnarray*}
\esp\left.\left[P_{j}^{m}\left(o,d\right)\right|Y_{j}\left(p\right)\right]&=&\esp\left.\left[\esp\left[P_{k}^{m}\left(o,d\right)\right|T_{k}\left(p_{m}\right)\right]\right]\left(\esp\left[\eta_{m}\left(V_{j}\right)\right]-\esp\left[\eta_{m}\left(V_{j-1}\right)\right]\right)\\
&=&\esp\left[\lambda\left(o,d\right)T_{k}\left(p_{m}\right)\right]\left(\esp\left[\eta_{m}\left(V_{j}\right)\right]-\esp\left[\eta_{m}\left(V_{j-1}\right)\right]\right)\\
&=&\lambda\left(o,d\right)\esp\left[T_{k}\left(p_{m}\right)\right]\left(\esp\left[\eta_{m}\left(V_{j}\right)\right]-\esp\left[\eta_{m}\left(V_{j-1}\right)\right]\right)\\
\end{eqnarray*}

es decir,

\begin{equation}\label{EqFinalPPO}
\esp\left.\left[P_{j}^{m}\left(o,d\right)\right|Y_{j}\left(p\right)\right]=\lambda\left(o,d\right)\mu_{m}\left(\esp\left[\eta_{m}\left(V_{j}\right)\right]-\esp\left[\eta_{m}\left(V_{j-1}\right)\right]\right)
\end{equation}


Sea $r\geq1$ fija, y sea $L_{r}$ l\'inea tal que $L_{r}$ y $L_{l}$
no tienen plataforma en com\'un, de modo tal que la \'unica manera
de dirigirse de $o\in L_{r}$ a $d\in L_{l}$ es haciendo por lo
menos un cambio de l\'inea.

Sea $q\in L_{m}$ tal que $L_{r}$ hace correspondencia con $L_{m}$
en la estaci\'on donde est\'a ubicada la plataforma $q$ y la
plataforma que llamaremos $p_{r}\in L_{r}$.

Consideremos de momento solamente las plataformas $q\in L_{m}$ y
$p_{r}\in L_{r}$. Siguiendo el razonamiento dado con anterioridad,
los pasajeros de primer orden que llegan a $p_{r}$ para hacer
cambio de l\'inea y abordar el $k$-\'esimo tren en $q$ es:

\begin{equation}
\hat{P}_{k}^{r}\left(o,d\right)=\sum_{n=\eta_{r}\left(V_{i-1}\left(p_{r}\right)\right)+1}^{\eta_{r}\left(V_{i}\left(p_{r}\right)\right)}P_{i}^{r}\left(o,d\right).
\end{equation}

Al igual que antes, consideremos
$Y_{i}\left(p_{r}\right)=V_{i}\left(p_{r}\right)-V_{i-1}\left(p_{r}\right)$,
entonces

\begin{equation}\label{EqFinalPSO}
\esp\left.\left[P_{i}^{r}\left(o,d\right)\right|Y_{i}\left(q\right)\right]=\lambda\left(o,d\right)\mu_{r}\left(\esp\left[\eta_{r}\left(V_{i}\right)\right]-\esp\left[\eta_{r}\left(V_{i-1}\right)\right]\right)
\end{equation}

Entonces el total de pasajeros que llegan a $p$ para abordar el
$i$-\'esimo tren, son los que provienen de cualquier $o\in L_{m}$,
m\'as los pasajeros de primer orden en $q$ procedentes de
cualquier $o\in L_{r}$, es decir

\begin{eqnarray*}
\esp\left.\left[P_{i}^{m}\left(o,d\right)\right|Y_{j}\left(p\right)\right]&=&\lambda\left(o,d\right)\mu_{m}\left(\esp\left[\eta_{m}\left(V_{j}\right)\right]-\esp\left[\eta_{m}\left(V_{j-1}\right)\right]\right)\\
&+&\lambda\left(o,d\right)\mu_{r}\left(\esp\left[\eta_{r}\left(V_{i}\right)\right]-\esp\left[\eta_{r}\left(V_{i-1}\right)\right]\right)
\end{eqnarray*}

Utilizando el mismo argumento para los pasajeros de orden superior
se tiene

\begin{eqnarray*}
\esp\left[P_{j}^{m}|Y_{j}\left(p\right)\right]=\sum_{n=1}^{L}\sum_{o\in
L_{n}}\lambda\left(o,d\right)\mu_{n}\left[\esp\left[\eta_{n}\left(V_{i}\right)\right]-\esp\left[\eta_{n}\left(V_{i-1}\right)\right]\right]\indora_{d\in
D_{o}\left(p_{m},p\right)}\textrm{.}
\end{eqnarray*}
con $\eta_{m}\left(V_{i}\right)$ el proceso de conteo de los
trenes que llegan a la plataforma $p_{n}$.

\begin{Prop}
Bajo los supuestos (\ref{Sup1}) y (\ref{Sup2}), los tiempos de
salida entre dos trenes consecutivos, $Y_{j}$ en la plataforma
$p\in L_{l}$ sigue una distribuci\'on $G_{\mu}$, donde
$\esp\left[Y_{j}\right]=\mu$, par\'ametro de escala de $G_{\mu}$.
Espec\'ificamente, si la l\'inea est\'a dada por la sucesi\'on de
plataformas $L_{1}=\left(p_{1},p_{2},\ldots,p_{L}\right)$,
entonces para $q\in\left\{1,2,\ldots,L\right\}$ los tiempos de
intersalida en la plataforma $p_{q}$ satisfacen
\begin{equation}
Var\left[Y_{j}\right]=\mu^{2}\left\{Var\left[\delta_{j}\left(p_{1}\right)\right]-2\left(q-1\right)Var\left[\delta_{j}\left(p\right)\right]\right\}
\end{equation}
\end{Prop}

A saber, los tiempos de salida del tren $V_{j}$ en la plataforma
incial $p_{1}\in L_{l}$ sigue la recursi\'on
\begin{eqnarray*}
V_{j}\left(p_{1}\right)&=&V_{j-1}\left(p_{1}\right)+\mu_{l}\left(1+\delta_{j}\left(p_{1}\right)\right)
\end{eqnarray*}
entonces para $j=1,2,\ldots$

\begin{eqnarray*}
Y_{j}\left(p_{1}\right)&=&V_{j}\left(p_{1}\right)-V_{j-1}\left(p_{1}\right)=\mu_{l}\left(1+\delta_{j}\left(p_{1}\right)\right)
\end{eqnarray*}
para la siguiente plataforma
\begin{eqnarray*}
V_{j}\left(p_{2}\right)&=&V_{j}\left(p_{1}\right)+\frac{D\left(p_{1},p_{2}\right)}{v}+\mu_{l}\delta_{j}\left(p_{2}\right)\\
V_{j+1}\left(p_{2}\right)&=&V_{j+1}\left(p_{1}\right)+\frac{D\left(p_{1},p_{2}\right)}{v}+\mu_{l}\delta_{j+1}\left(p_{2}\right)\\
\end{eqnarray*}

entonces
\begin{eqnarray*}
Y_{j+1}\left(p_{2}\right)&=&V_{j+1}\left(p_{2}\right)-V_{j}\left(p_{2}\right)\\
&=&V_{j+1}\left(p_{1}\right)+\mu_{l}\delta_{j+1}\left(p_{2}\right)-V_{j}\left(p_{1}\right)-\mu_{l}\delta_{j}\left(p_{2}\right)\\
&=&Y_{j+1}\left(p_{1}\right)+\mu_{l}\left[\delta_{j+1}\left(p_{2}\right)-\delta_{j}\left(p_{2}\right)\right]\\
&=&\mu_{l}\left(1+\delta_{j}\left(p_{1}\right)\right)+\mu_{l}\left[\delta_{j+1}\left(p_{2}\right)-\delta_{j}\left(p_{2}\right)\right]\\
&=&\mu_{l}\left[\left(1+\delta_{j}\left(p_{1}\right)\right)+\delta_{j+1}\left(p_{2}\right)-\delta_{j}\left(p_{2}\right)\right]
\end{eqnarray*}

en t\'erminos de  la $q$-\'esima plataforma y $p_{n}$
\begin{eqnarray*}
Y_{j+1}\left(p_{q}\right)=\mu_{l}\left[\left(1+\delta_{j+1}\left(p_{1}\right)\right)+\sum_{n=2}^{q}\left(\delta_{j+1}\left(p_{n}\right)-\delta_{j}\left(p_{n}\right)\right)\right]
\end{eqnarray*}

recordando que la sucesi\'on de variables aleatorias
independientes e id\'enticamente distribuidas
$\left\{\delta_{j}\right\}$ tienen esperanza cero, se sigue que
\begin{eqnarray*}
\esp\left[Y_{j+1}\left(p_{q}\right)\right]=\mu
\end{eqnarray*}
y entonces por propiedades de la varianza se tiene la igualdad que
falta.
\begin{eqnarray*}
Var\left[Y_{j}\right]=\mu^{2}\left\{Var\left[\delta_{j}\left(p_{1}\right)\right]-2\left(q-1\right)Var\left[\delta_{j}\left(p\right)\right]\right\}
\end{eqnarray*}

Utilizando el mismo argumento para los pasajeros de orden superior
se tiene

\begin{eqnarray*}
\esp\left[P_{j}^{m}|Y_{j}\left(p\right)\right]=\sum_{n=1}^{L}\sum_{o\in
L_{n}}\lambda\left(o,d\right)\mu_{n}\left[\esp\left[\eta_{n}\left(V_{i}\right)\right]-\esp\left[\eta_{n}\left(V_{i-1}\right)\right]\right]\indora_{d\in
D_{o}\left(p_{m},p\right)}\textrm{.}
\end{eqnarray*}
con $\eta_{m}\left(V_{i}\right)$ el proceso de conteo de los
trenes que llegan a la plataforma $p_{n}$.
\end{Dem}

\begin{Prop}
Bajo los supuestos (\ref{Sup1}) y (\ref{Sup2}), los tiempos de
salida entre dos trenes consecutivos, $Y_{j}$ en la plataforma
$p\in L_{l}$ sigue una distribuci\'on $G_{\mu}$, donde
$\esp\left[Y_{j}\right]=\mu$, par\'ametro de escala de $G_{\mu}$.
Espec\'ificamente, si la l\'inea est\'a dada por la sucesi\'on de
plataformas $L_{1}=\left(p_{1},p_{2},\ldots,p_{L}\right)$,
entonces para $q\in\left\{1,2,\ldots,L\right\}$ los tiempos de
intersalida en la plataforma $p_{q}$ satisfacen
\begin{equation}
Var\left[Y_{j}\right]=\mu^{2}\left\{Var\left[\delta_{j}\left(p_{1}\right)\right]-2\left(q-1\right)Var\left[\delta_{j}\left(p\right)\right]\right\}
\end{equation}
\end{Prop}

A saber, los tiempos de salida del tren $V_{j}$ en la plataforma
incial $p_{1}\in L_{l}$ sigue la recursi\'on
\begin{eqnarray*}
V_{j}\left(p_{1}\right)&=&V_{j-1}\left(p_{1}\right)+\mu_{l}\left(1+\delta_{j}\left(p_{1}\right)\right)
\end{eqnarray*}
entonces para $j=1,2,\ldots$

\begin{eqnarray*}
Y_{j}\left(p_{1}\right)&=&V_{j}\left(p_{1}\right)-V_{j-1}\left(p_{1}\right)=\mu_{l}\left(1+\delta_{j}\left(p_{1}\right)\right)
\end{eqnarray*}
para la siguiente plataforma
\begin{eqnarray*}
V_{j}\left(p_{2}\right)&=&V_{j}\left(p_{1}\right)+\frac{D\left(p_{1},p_{2}\right)}{v}+\mu_{l}\delta_{j}\left(p_{2}\right)\\
V_{j+1}\left(p_{2}\right)&=&V_{j+1}\left(p_{1}\right)+\frac{D\left(p_{1},p_{2}\right)}{v}+\mu_{l}\delta_{j+1}\left(p_{2}\right)\\
\end{eqnarray*}
entonces
\begin{eqnarray*}
Y_{j+1}\left(p_{2}\right)&=&V_{j+1}\left(p_{2}\right)-V_{j}\left(p_{2}\right)\\
&=&V_{j+1}\left(p_{1}\right)+\mu_{l}\delta_{j+1}\left(p_{2}\right)-V_{j}\left(p_{1}\right)-\mu_{l}\delta_{j}\left(p_{2}\right)\\
&=&Y_{j+1}\left(p_{1}\right)+\mu_{l}\left[\delta_{j+1}\left(p_{2}\right)-\delta_{j}\left(p_{2}\right)\right]\\
&=&\mu_{l}\left(1+\delta_{j}\left(p_{1}\right)\right)+\mu_{l}\left[\delta_{j+1}\left(p_{2}\right)-\delta_{j}\left(p_{2}\right)\right]\\
&=&\mu_{l}\left[\left(1+\delta_{j}\left(p_{1}\right)\right)+\delta_{j+1}\left(p_{2}\right)-\delta_{j}\left(p_{2}\right)\right]
\end{eqnarray*}

en t\'erminos de  la $q$-\'esima plataforma y $p_{n}$
\begin{eqnarray*}
Y_{j+1}\left(p_{q}\right)=\mu_{l}\left[\left(1+\delta_{j+1}\left(p_{1}\right)\right)+\sum_{n=2}^{q}\left(\delta_{j+1}\left(p_{n}\right)-\delta_{j}\left(p_{n}\right)\right)\right]
\end{eqnarray*}
recordando que la sucesi\'on de variables aleatorias
independientes e id\'enticamente distribuidas
$\left\{\delta_{j}\right\}$ tienen esperanza cero, se sigue que
\begin{eqnarray*}
\esp\left[Y_{j+1}\left(p_{q}\right)\right]=\mu
\end{eqnarray*}
y entonces por propiedades de la varianza se tiene la igualdad que
falta.
\begin{eqnarray*}
Var\left[Y_{j}\right]=\mu^{2}\left\{Var\left[\delta_{j}\left(p_{1}\right)\right]-2\left(q-1\right)Var\left[\delta_{j}\left(p\right)\right]\right\}
\end{eqnarray*}
%_____________________________________________________________________
%
\subsubsection{El Modelo Global}
%_____________________________________________________________________
El costo de operaci\'on promedio por d\'ia  es
$K\left(\overline{\mu}\right)$, el costo $k_{l}$ por viaje en
cualquier tren de la l\'inea $L_{l}$ est\'a dado por:

\begin{equation}
K\left(\overline{\mu}\right)=\sum_{s=1}^{S}k_{l}\frac{T_{s}}{\mu_{l,s}}
\end{equation}

donde $s$ es un segmento del d\'ia correspondiente a las distintas
demandas del servicio a lo largo de las horas que presta servicio
el metro. $T_{s}$ es la longitud del segmento $s$, y $\mu_{l,s}$
corresponde al valor del par\'ametro de control para la l\'inea
$l$ en el segmento $s$.

El tiempo de espera acumulado de los pasajeros es el tiempo total
que los pasajeros tienen que esperar en las plataformas, el cu\'al
representa el {\em costo social}. Los pasajeros esperan en
cualquier plataforma $p\in L_{l}$ sin importar de donde vengan,
entre dos salidas de trenes consecutivos en $p$.

A este tiempo total de espera se le denotar\'a por
$W_{j}\left(p\right)$, que es el tiempo total de espera de los
pasajeros en la plataforma $p$ de la l\'inea $L_{l}$ para tomar el
$j$-\'esimo tren.

Si $M_{p}\left(\cdot\right)$ denota el proceso de conteo de salida
de los trenes de la plataforma en $p$, entonces

\begin{equation}\label{Eq.FuncionObjetivo}
F\left(\overline{\mu}\right)=K\left(\overline{\mu}\right)+\sum_{s=1}^{S}\sum_{l=1}^{L}\sum_{p\in
L_{l}}\esp\left[\sum_{j=M_{p}\left(\tau_{s}\right)+1}^{M_{p}\left(\tau_{s}+T_{s}\right)}
W_{j}\left(p\right)\right]
\end{equation}

Los procesos de llegada de los pasajeros de transferencia de la
l\'inea $L_{m}\neq L_{l}$ en la estaci\'on donde est\'a la
plataforma $p\in L_{l}$ ubicada, dependen de los procesos de
salida en las plataformas anteriores a la plataforma de
transferencia en la l\'inea $L_{m}$, as\'i como de  posibles
transferencias.

El costo esperado, $F\left(\mu\right)$, de la red, es el costo
esperado de operaci\'on por d\'ia, m\'as el tiempo de espera total
acumulado de los pasajeros en las plataformas. Calcular
directamente el gradiente del costo esperado por d\'ia puede ser
altamente complicado en t\'erminos computacionales.

Es por medio de t\'ecnicas de simulaci\'on que el c\'alculo de las
derivadas se puede simplificar haciendo estos c\'alculos por
plataforma, es decir, el modelo global lo estudiamos localmente.

%_____________________________________________________________________
%
\subsubsection{Modelo Fantasma}
%_____________________________________________________________________
Consideremos cualquier plataforma $p$ en una l\'inea $L_{l}$, y
supongamos que el segmento del d\'ia $s$ se alarga al infinito, es
decir, los procesos Poisson de llegada se asumen estacionarios con
par\'ametros constantes, adem\'as las frecuencias se consideran
invariantes con respecto al tiempo.

El tiempo total de espera para los pasajeros que abordan el tren
$j$ en la plataforma $p$ se puede escribir como

\begin{equation}\label{Eq.TTTlEsp}
W_{j}\left(p\right)=\sum_{n=N_{o}\left(V_{j-1}\right)+1}^{N_{o}\left(V_{j}\right)}\left(V_{j}-S_{n}^{o}\right)+\sum_{m=1}^{M}\sum_{k=N_{m}\left(V_{j-1}\right)+1}^{N_{m}\left(V_{j}\right)}P_{k}^{(m)}\left(V_{j}-S_{k}^{m}\right)
\end{equation}

donde
\begin{itemize}
\item $N_{o}\left(\cdot\right)$ es un Proceso Poisson acumulado,
de todas las llegadas del or\'igen $p$ con tiempos de llegada
$S_{k}^{o}$. \item $M$ son las distintas l\'ineas que pueden
transferir pasajeros en esta estaci\'on a traves de la plataforma
$p$.

 \item $N_{m}\left(\cdot\right)$ son los procesos de llegada
de los trenes de la l\'inea $L_{m}$ a la estaci\'on donde est\'a
ubicada la plataforma $p$, con correspondientes tiempos de llegada
$S_{k}^{m}$.

\item $P_{k}^{m}$ es el n\'umero de pasajeros en transferencia en
el tren de llegada $k$, para la l\'inea $L_{m}$.
\end{itemize}

Sean las plataformas $p\in L_{l}$ y $p_{m}\in L_{m}$ en la misma
estaci\'on. Los trenes que parten de $p_{m}$ en tiempos
$V_{k}\left(p_{m}\right)$, corresponden a los procesos de llegada
de transferencia $N_{m}\left(\cdot\right)$ en la plataforma $p\in
L_{l}$ y a los pasajeros que se mover\'an de $p_{m}$ a $p$.

Si $p_{i}\in L_{l}$ y $p_{k}\notin L_{l}$, entonces los procesos
de salida correspondientes se asume que son independientes bajo el
supuesto de no sincronizaci\'on.

Los procesos de llegada de los pasajeros en cada plataforma
est\'an compuestos por un proceso Poisson de pasajeros, que
abordan desde fuera del sistema, m\'as los que llegan de otras
l\'ineas de transferencia.
%_____________________________________________________________________
%
\subsubsection{La funci\'on susituta}
%_____________________________________________________________________
Recordemos las expresiones referentes tanto al modelo global del
sistema de transporte colectivo (\ref{Eq.FuncionObjetivo})

\begin{eqnarray*}
F\left(\overline{\mu}\right)=K\left(\overline{\mu}\right)+\sum_{s=1}^{S}\sum_{l=1}^{L}\sum_{p\in
L_{l}}\esp\left[\sum_{j=M_{p}\left(\tau_{s}\right)+1}^{M_{p}\left(\tau_{s}+T_{s}\right)}\textrm{,}
W_{j}\left(p\right)\right]
\end{eqnarray*}
como la del tiempo total de espera de los pasajeros para abordar
el tren $j$ en la plataforma $p$, (\ref{Eq.TTTlEsp}),
\begin{eqnarray*}
W_{j}\left(p\right)=\sum_{n=N_{o}\left(V_{j-1}\right)+1}^{N_{o}\left(V_{j}\right)}\left(V_{j}-S_{n}^{o}\right)+\sum_{m=1}^{\mathcal{M}}\sum_{k=N_{m}\left(V_{j-1}\right)+1}^{N_{m}\left(V_{j}\right)}P_{k}^{m}\left(V_{j}-S_{k}^{m}\right)\textrm{.}
\end{eqnarray*}

Dada la alta interdependencia, es dif\'icil estimar la funci\'on
costo as\'i como el c\'alculo de las derivadas con respecto al
par\'ametro de control $\mu$, recordemos que lo que se quiere es
optimizar la funci\'on objetivo (\ref{Eq.FuncionObjetivo})
entonces lo que se propone es utilizar una funci\'on sustituta que
se pueda simular, dicha funci\'on es:
\begin{equation}\label{Eq.FuncSustituta}
\Phi\left(\mu\right)=\sum_{j=1}^{M\left(T\right)}\sum_{m=0}^{\mathcal{M}}\rho_{m}\sum_{k=N_{m}\left(V_{j-1}\right)+1}^{N_{m}\left(V_{j}\right)}T_{k}\left(V_{j}-S_{k}^{m}\right)
\end{equation}

Calculemos la esperanza de $\Phi$:
\begin{eqnarray*}
\esp\left[\Phi\left(\mu\right)\right]=\esp\left[\esp\left[\Phi\left(\mu\right)|T_{k}^{m}\right]\right]
\end{eqnarray*}
De momento supongamos que $M\left(T\right)$ no es aleatorio, y
adem\'as consideremos solamente la suma a partir de $1$, entonces,

\begin{eqnarray*}
\esp\left[\Phi\left(\mu\right)\right]&\approx&\esp\left[\sum_{j=1}^{M\left(T\right)}\sum_{m=1}^{\mathcal{M}}\esp\left[\sum_{k=N_{m}\left(V_{j-1}\right)+1}^{N_{m}}\esp\left[P_{k}^{m}|T_{k}^{m}\right]\left(V_{j}-S_{k}^{m}\right)|T_{k}^{m}\right]\right]\\
&=&\esp\left[\sum_{j=1}^{M\left(T\right)}\sum_{m=1}^{\mathcal{M}}\sum_{k=N_{m}\left(V_{j-1}\right)+1}^{N_{m}}P_{k}^{m}\left(V_{j}-S_{k}^{m}\right)\right]\\
&\approx&\esp\left[\sum_{j=1}^{M\left(T\right)}W_{j}\left(p\right)\right]
\end{eqnarray*}


%_______________________________________________________________________________
%\subsection{AP\'ENDICE A}


En este ap\'endice enunciaremos una serie de resultados que son
necesarios para la demostraci\'on as\'i como su demostraci\'on del
Teorema de Down \ref{Tma2.1.Down}, adem\'as de un teorema
referente a las propiedades que cumple el Modelo de Flujo.\\


Dado el proceso $X=\left\{X\left(t\right),t\geq0\right\}$ definido
en (\ref{Esp.Edos.Down}) que describe la din\'amica del sistema de
visitas c\'iclicas, si $U\left(t\right)$ es el residual de los
tiempos de llegada al tiempo $t$ entre dos usuarios consecutivos y
$V\left(t\right)$ es el residual de los tiempos de servicio al
tiempo $t$ para el usuario que est\'as siendo atendido por el
servidor. Sea $\mathbb{X}$ el espacio de estados que puede tomar
el proceso $X$.


\begin{Lema}[Lema 4.3, Dai\cite{Dai}]\label{Lema.4.3}
Sea $\left\{x_{n}\right\}\subset \mathbf{X}$ con
$|x_{n}|\rightarrow\infty$, conforme $n\rightarrow\infty$. Suponga
que
\[lim_{n\rightarrow\infty}\frac{1}{|x_{n}|}U\left(0\right)=\overline{U}_{k},\]
y
\[lim_{n\rightarrow\infty}\frac{1}{|x_{n}|}V\left(0\right)=\overline{V}_{k}.\]
\begin{itemize}
\item[a)] Conforme $n\rightarrow\infty$ casi seguramente,
\[lim_{n\rightarrow\infty}\frac{1}{|x_{n}|}U^{x_{n}}_{k}\left(|x_{n}|t\right)=\left(\overline{U}_{k}-t\right)^{+}\textrm{, u.o.c.}\]
y
\[lim_{n\rightarrow\infty}\frac{1}{|x_{n}|}V^{x_{n}}_{k}\left(|x_{n}|t\right)=\left(\overline{V}_{k}-t\right)^{+}.\]

\item[b)] Para cada $t\geq0$ fijo,
\[\left\{\frac{1}{|x_{n}|}U^{x_{n}}_{k}\left(|x_{n}|t\right),|x_{n}|\geq1\right\}\]
y
\[\left\{\frac{1}{|x_{n}|}V^{x_{n}}_{k}\left(|x_{n}|t\right),|x_{n}|\geq1\right\}\]
\end{itemize}
son uniformemente convergentes.
\end{Lema}

Sea $e$ es un vector de unos, $C$ es la matriz definida por
\[C_{ik}=\left\{\begin{array}{cc}
1,& S\left(k\right)=i,\\
0,& \textrm{ en otro caso}.\\
\end{array}\right.
\]
Es necesario enunciar el siguiente Teorema que se utilizar\'a para
el Teorema (\ref{Tma.4.2.Dai}):
\begin{Teo}[Teorema 4.1, Dai \cite{Dai}]
Considere una disciplina que cumpla la ley de conservaci\'on, para
casi todas las trayectorias muestrales $\omega$ y cualquier
sucesi\'on de estados iniciales $\left\{x_{n}\right\}\subset
\mathbf{X}$, con $|x_{n}|\rightarrow\infty$, existe una
subsucesi\'on $\left\{x_{n_{j}}\right\}$ con
$|x_{n_{j}}|\rightarrow\infty$ tal que
\begin{equation}\label{Eq.4.15}
\frac{1}{|x_{n_{j}}|}\left(Q^{x_{n_{j}}}\left(0\right),U^{x_{n_{j}}}\left(0\right),V^{x_{n_{j}}}\left(0\right)\right)\rightarrow\left(\overline{Q}\left(0\right),\overline{U},\overline{V}\right),
\end{equation}

\begin{equation}\label{Eq.4.16}
\frac{1}{|x_{n_{j}}|}\left(Q^{x_{n_{j}}}\left(|x_{n_{j}}|t\right),T^{x_{n_{j}}}\left(|x_{n_{j}}|t\right)\right)\rightarrow\left(\overline{Q}\left(t\right),\overline{T}\left(t\right)\right)\textrm{
u.o.c.}
\end{equation}

Adem\'as,
$\left(\overline{Q}\left(t\right),\overline{T}\left(t\right)\right)$
satisface las siguientes ecuaciones:
\begin{equation}\label{Eq.MF.1.3a}
\overline{Q}\left(t\right)=Q\left(0\right)+\left(\alpha
t-\overline{U}\right)^{+}-\left(I-P\right)^{'}M^{-1}\left(\overline{T}\left(t\right)-\overline{V}\right)^{+},
\end{equation}

\begin{equation}\label{Eq.MF.2.3a}
\overline{Q}\left(t\right)\geq0,\\
\end{equation}

\begin{equation}\label{Eq.MF.3.3a}
\overline{T}\left(t\right)\textrm{ es no decreciente y comienza en cero},\\
\end{equation}

\begin{equation}\label{Eq.MF.4.3a}
\overline{I}\left(t\right)=et-C\overline{T}\left(t\right)\textrm{
es no decreciente,}\\
\end{equation}

\begin{equation}\label{Eq.MF.5.3a}
\int_{0}^{\infty}\left(C\overline{Q}\left(t\right)\right)d\overline{I}\left(t\right)=0,\\
\end{equation}

\begin{equation}\label{Eq.MF.6.3a}
\textrm{Condiciones en
}\left(\overline{Q}\left(\cdot\right),\overline{T}\left(\cdot\right)\right)\textrm{
espec\'ificas de la disciplina de la cola,}
\end{equation}
\end{Teo}


Propiedades importantes para el modelo de flujo retrasado:

\begin{Prop}[Proposici\'on 4.2, Dai \cite{Dai}]
 Sea $\left(\overline{Q},\overline{T},\overline{T}^{0}\right)$ un flujo l\'imite de \ref{Eq.Punto.Limite}
 y suponga que cuando $x\rightarrow\infty$ a lo largo de una subsucesi\'on
\[\left(\frac{1}{|x|}Q_{k}^{x}\left(0\right),\frac{1}{|x|}A_{k}^{x}\left(0\right),\frac{1}{|x|}B_{k}^{x}\left(0\right),\frac{1}{|x|}B_{k}^{x,0}\left(0\right)\right)\rightarrow\left(\overline{Q}_{k}\left(0\right),0,0,0\right)\]
para $k=1,\ldots,K$. El flujo l\'imite tiene las siguientes
propiedades, donde las propiedades de la derivada se cumplen donde
la derivada exista:
\begin{itemize}
 \item[i)] Los vectores de tiempo ocupado $\overline{T}\left(t\right)$ y $\overline{T}^{0}\left(t\right)$ son crecientes y continuas con
$\overline{T}\left(0\right)=\overline{T}^{0}\left(0\right)=0$.
\item[ii)] Para todo $t\geq0$
\[\sum_{k=1}^{K}\left[\overline{T}_{k}\left(t\right)+\overline{T}_{k}^{0}\left(t\right)\right]=t.\]
\item[iii)] Para todo $1\leq k\leq K$
\[\overline{Q}_{k}\left(t\right)=\overline{Q}_{k}\left(0\right)+\alpha_{k}t-\mu_{k}\overline{T}_{k}\left(t\right).\]
\item[iv)]  Para todo $1\leq k\leq K$
\[\dot{{\overline{T}}}_{k}\left(t\right)=\rho_{k}\] para $\overline{Q}_{k}\left(t\right)=0$.
\item[v)] Para todo $k,j$
\[\mu_{k}^{0}\overline{T}_{k}^{0}\left(t\right)=\mu_{j}^{0}\overline{T}_{j}^{0}\left(t\right).\]
\item[vi)]  Para todo $1\leq k\leq K$
\[\mu_{k}\dot{{\overline{T}}}_{k}\left(t\right)=l_{k}\mu_{k}^{0}\dot{{\overline{T}}}_{k}^{0}\left(t\right),\] para $\overline{Q}_{k}\left(t\right)>0$.
\end{itemize}
\end{Prop}

\begin{Lema}[Lema 3.1, Chen \cite{Chen}]\label{Lema3.1}
Si el modelo de flujo es estable, definido por las ecuaciones
(3.8)-(3.13), entonces el modelo de flujo retrasado tambi\'en es
estable.
\end{Lema}

\begin{Lema}[Lema 5.2, Gut \cite{Gut}]\label{Lema.5.2.Gut}
Sea $\left\{\xi\left(k\right):k\in\ent\right\}$ sucesi\'on de
variables aleatorias i.i.d. con valores en
$\left(0,\infty\right)$, y sea $E\left(t\right)$ el proceso de
conteo
\[E\left(t\right)=max\left\{n\geq1:\xi\left(1\right)+\cdots+\xi\left(n-1\right)\leq t\right\}.\]
Si $E\left[\xi\left(1\right)\right]<\infty$, entonces para
cualquier entero $r\geq1$
\begin{equation}
lim_{t\rightarrow\infty}\esp\left[\left(\frac{E\left(t\right)}{t}\right)^{r}\right]=\left(\frac{1}{E\left[\xi_{1}\right]}\right)^{r},
\end{equation}
de aqu\'i, bajo estas condiciones
\begin{itemize}
\item[a)] Para cualquier $t>0$,
$sup_{t\geq\delta}\esp\left[\left(\frac{E\left(t\right)}{t}\right)^{r}\right]<\infty$.

\item[b)] Las variables aleatorias
$\left\{\left(\frac{E\left(t\right)}{t}\right)^{r}:t\geq1\right\}$
son uniformemente integrables.
\end{itemize}
\end{Lema}

\begin{Teo}[Teorema 5.1: Ley Fuerte para Procesos de Conteo, Gut
\cite{Gut}]\label{Tma.5.1.Gut} Sea
$0<\mu<\esp\left(X_{1}\right]\leq\infty$. entonces

\begin{itemize}
\item[a)] $\frac{N\left(t\right)}{t}\rightarrow\frac{1}{\mu}$
a.s., cuando $t\rightarrow\infty$.


\item[b)]$\esp\left[\frac{N\left(t\right)}{t}\right]^{r}\rightarrow\frac{1}{\mu^{r}}$,
cuando $t\rightarrow\infty$ para todo $r>0$.
\end{itemize}
\end{Teo}


\begin{Prop}[Proposici\'on 5.1, Dai y Sean \cite{DaiSean}]\label{Prop.5.1}
Suponga que los supuestos (A1) y (A2) se cumplen, adem\'as suponga
que el modelo de flujo es estable. Entonces existe $t_{0}>0$ tal
que
\begin{equation}\label{Eq.Prop.5.1}
lim_{|x|\rightarrow\infty}\frac{1}{|x|^{p+1}}\esp_{x}\left[|X\left(t_{0}|x|\right)|^{p+1}\right]=0.
\end{equation}

\end{Prop}


\begin{Prop}[Proposici\'on 5.3, Dai y Sean \cite{DaiSean}]\label{Prop.5.3.DaiSean}
Sea $X$ proceso de estados para la red de colas, y suponga que se
cumplen los supuestos (A1) y (A2), entonces para alguna constante
positiva $C_{p+1}<\infty$, $\delta>0$ y un conjunto compacto
$C\subset X$.

\begin{equation}\label{Eq.5.4}
\esp_{x}\left[\int_{0}^{\tau_{C}\left(\delta\right)}\left(1+|X\left(t\right)|^{p}\right)dt\right]\leq
C_{p+1}\left(1+|x|^{p+1}\right).
\end{equation}
\end{Prop}

\begin{Prop}[Proposici\'on 5.4, Dai y Sean \cite{DaiSean}]\label{Prop.5.4.DaiSean}
Sea $X$ un proceso de Markov Borel Derecho en $X$, sea
$f:X\leftarrow\rea_{+}$ y defina para alguna $\delta>0$, y un
conjunto cerrado $C\subset X$
\[V\left(x\right):=\esp_{x}\left[\int_{0}^{\tau_{C}\left(\delta\right)}f\left(X\left(t\right)\right)dt\right],\]
para $x\in X$. Si $V$ es finito en todas partes y uniformemente
acotada en $C$, entonces existe $k<\infty$ tal que
\begin{equation}\label{Eq.5.11}
\frac{1}{t}\esp_{x}\left[V\left(x\right)\right]+\frac{1}{t}\int_{0}^{t}\esp_{x}\left[f\left(X\left(s\right)\right)ds\right]\leq\frac{1}{t}V\left(x\right)+k,
\end{equation}
para $x\in X$ y $t>0$.
\end{Prop}


\begin{Teo}[Teorema 5.5, Dai y Sean  \cite{DaiSean}]
Suponga que se cumplen (A1) y (A2), adem\'as suponga que el modelo
de flujo es estable. Entonces existe una constante $k_{p}<\infty$
tal que
\begin{equation}\label{Eq.5.13}
\frac{1}{t}\int_{0}^{t}\esp_{x}\left[|Q\left(s\right)|^{p}\right]ds\leq
k_{p}\left\{\frac{1}{t}|x|^{p+1}+1\right\},
\end{equation}
para $t\geq0$, $x\in X$. En particular para cada condici\'on
inicial
\begin{equation}\label{Eq.5.14}
\limsup_{t\rightarrow\infty}\frac{1}{t}\int_{0}^{t}\esp_{x}\left[|Q\left(s\right)|^{p}\right]ds\leq
k_{p}.
\end{equation}
\end{Teo}

\begin{Teo}[Teorema 6.2 Dai y Sean \cite{DaiSean}]\label{Tma.6.2}
Suponga que se cumplen los supuestos (A1)-(A3) y que el modelo de
flujo es estable, entonces se tiene que
\[\parallel P^{t}\left(x,\cdot\right)-\pi\left(\cdot\right)\parallel_{f_{p}}\rightarrow0,\]
para $t\rightarrow\infty$ y $x\in X$. En particular para cada
condici\'on inicial
\[lim_{t\rightarrow\infty}\esp_{x}\left[\left|Q_{t}\right|^{p}\right]=\esp_{\pi}\left[\left|Q_{0}\right|^{p}\right]<\infty,\]
\end{Teo}

donde

\begin{eqnarray*}
\parallel
P^{t}\left(c,\cdot\right)-\pi\left(\cdot\right)\parallel_{f}=sup_{|g\leq
f|}|\int\pi\left(dy\right)g\left(y\right)-\int
P^{t}\left(x,dy\right)g\left(y\right)|,
\end{eqnarray*}
para $x\in\mathbb{X}$.

\begin{Teo}[Teorema 6.3, Dai y Sean \cite{DaiSean}]\label{Tma.6.3}
Suponga que se cumplen los supuestos (A1)-(A3) y que el modelo de
flujo es estable, entonces con
$f\left(x\right)=f_{1}\left(x\right)$, se tiene que
\[lim_{t\rightarrow\infty}t^{(p-1)}\left|P^{t}\left(c,\cdot\right)-\pi\left(\cdot\right)\right|_{f}=0,\]
para $x\in X$. En particular, para cada condici\'on inicial
\[lim_{t\rightarrow\infty}t^{(p-1)}\left|\esp_{x}\left[Q_{t}\right]-\esp_{\pi}\left[Q_{0}\right]\right|=0.\]
\end{Teo}



\begin{Prop}[Proposici\'on 5.1, Dai y Meyn \cite{DaiSean}]\label{Prop.5.1.DaiSean}
Suponga que los supuestos A1) y A2) son ciertos y que el modelo de
flujo es estable. Entonces existe $t_{0}>0$ tal que
\begin{equation}
lim_{|x|\rightarrow\infty}\frac{1}{|x|^{p+1}}\esp_{x}\left[|X\left(t_{0}|x|\right)|^{p+1}\right]=0.
\end{equation}
\end{Prop}


\begin{Teo}[Teorema 5.5, Dai y Meyn \cite{DaiSean}]\label{Tma.5.5.DaiSean}
Suponga que los supuestos A1) y A2) se cumplen y que el modelo de
flujo es estable. Entonces existe una constante $\kappa_{p}$ tal
que
\begin{equation}
\frac{1}{t}\int_{0}^{t}\esp_{x}\left[|Q\left(s\right)|^{p}\right]ds\leq\kappa_{p}\left\{\frac{1}{t}|x|^{p+1}+1\right\},
\end{equation}
para $t>0$ y $x\in X$. En particular, para cada condici\'on
inicial
\begin{eqnarray*}
\limsup_{t\rightarrow\infty}\frac{1}{t}\int_{0}^{t}\esp_{x}\left[|Q\left(s\right)|^{p}\right]ds\leq\kappa_{p}.
\end{eqnarray*}
\end{Teo}


\begin{Teo}[Teorema 6.4, Dai y Meyn \cite{DaiSean}]\label{Tma.6.4.DaiSean}
Suponga que se cumplen los supuestos A1), A2) y A3) y que el
modelo de flujo es estable. Sea $\nu$ cualquier distribuci\'on de
probabilidad en
$\left(\mathbb{X},\mathcal{B}_{\mathbb{X}}\right)$, y $\pi$ la
distribuci\'on estacionaria de $X$.
\begin{itemize}
\item[i)] Para cualquier $f:X\leftarrow\rea_{+}$
\begin{equation}
\lim_{t\rightarrow\infty}\frac{1}{t}\int_{o}^{t}f\left(X\left(s\right)\right)ds=\pi\left(f\right):=\int
f\left(x\right)\pi\left(dx\right),
\end{equation}
$\prob$-c.s.

\item[ii)] Para cualquier $f:X\leftarrow\rea_{+}$ con
$\pi\left(|f|\right)<\infty$, la ecuaci\'on anterior se cumple.
\end{itemize}
\end{Teo}

\begin{Teo}[Teorema 2.2, Down \cite{Down}]\label{Tma2.2.Down}
Suponga que el fluido modelo es inestable en el sentido de que
para alguna $\epsilon_{0},c_{0}\geq0$,
\begin{equation}\label{Eq.Inestability}
|Q\left(T\right)|\geq\epsilon_{0}T-c_{0}\textrm{,   }T\geq0,
\end{equation}
para cualquier condici\'on inicial $Q\left(0\right)$, con
$|Q\left(0\right)|=1$. Entonces para cualquier $0<q\leq1$, existe
$B<0$ tal que para cualquier $|x|\geq B$,
\begin{equation}
\prob_{x}\left\{\mathbb{X}\rightarrow\infty\right\}\geq q.
\end{equation}
\end{Teo}

\begin{Dem}[Teorema \ref{Tma2.1.Down}] La demostraci\'on de este
teorema se da a continuaci\'on:\\
\begin{itemize}
\item[i)] Utilizando la proposici\'on \ref{Prop.5.3.DaiSean} se
tiene que la proposici\'on \ref{Prop.5.4.DaiSean} es cierta para
$f\left(x\right)=1+|x|^{p}$.

\item[i)] es consecuencia directa del Teorema \ref{Tma.6.2}.

\item[iii)] ver la demostraci\'on dada en Dai y Sean
\cite{DaiSean} p\'aginas 1901-1902.

\item[iv)] ver Dai y Sean \cite{DaiSean} p\'aginas 1902-1903 \'o
\cite{MeynTweedie2}.
\end{itemize}
\end{Dem}
%\newpage
%_________________________________________________________________________
%\subsection{AP\'ENDICE B}
%_________________________________________________________________________
%\numberwithin{equation}{section}


%_______________________________________________________________________________
%\subsection{AP\'ENDICE A}


En este ap\'endice enunciaremos una serie de resultados que son
necesarios para la demostraci\'on as\'i como su demostraci\'on del
Teorema de Down \ref{Tma2.1.Down}, adem\'as de un teorema
referente a las propiedades que cumple el Modelo de Flujo.\\


Dado el proceso $X=\left\{X\left(t\right),t\geq0\right\}$ definido
en (\ref{Esp.Edos.Down}) que describe la din\'amica del sistema de
visitas c\'iclicas, si $U\left(t\right)$ es el residual de los
tiempos de llegada al tiempo $t$ entre dos usuarios consecutivos y
$V\left(t\right)$ es el residual de los tiempos de servicio al
tiempo $t$ para el usuario que est\'as siendo atendido por el
servidor. Sea $\mathbb{X}$ el espacio de estados que puede tomar
el proceso $X$.


\begin{Lema}[Lema 4.3, Dai\cite{Dai}]\label{Lema.4.3}
Sea $\left\{x_{n}\right\}\subset \mathbf{X}$ con
$|x_{n}|\rightarrow\infty$, conforme $n\rightarrow\infty$. Suponga
que
\[lim_{n\rightarrow\infty}\frac{1}{|x_{n}|}U\left(0\right)=\overline{U}_{k},\]
y
\[lim_{n\rightarrow\infty}\frac{1}{|x_{n}|}V\left(0\right)=\overline{V}_{k}.\]
\begin{itemize}
\item[a)] Conforme $n\rightarrow\infty$ casi seguramente,
\[lim_{n\rightarrow\infty}\frac{1}{|x_{n}|}U^{x_{n}}_{k}\left(|x_{n}|t\right)=\left(\overline{U}_{k}-t\right)^{+}\textrm{, u.o.c.}\]
y
\[lim_{n\rightarrow\infty}\frac{1}{|x_{n}|}V^{x_{n}}_{k}\left(|x_{n}|t\right)=\left(\overline{V}_{k}-t\right)^{+}.\]

\item[b)] Para cada $t\geq0$ fijo,
\[\left\{\frac{1}{|x_{n}|}U^{x_{n}}_{k}\left(|x_{n}|t\right),|x_{n}|\geq1\right\}\]
y
\[\left\{\frac{1}{|x_{n}|}V^{x_{n}}_{k}\left(|x_{n}|t\right),|x_{n}|\geq1\right\}\]
\end{itemize}
son uniformemente convergentes.
\end{Lema}

Sea $e$ es un vector de unos, $C$ es la matriz definida por
\[C_{ik}=\left\{\begin{array}{cc}
1,& S\left(k\right)=i,\\
0,& \textrm{ en otro caso}.\\
\end{array}\right.
\]
Es necesario enunciar el siguiente Teorema que se utilizar\'a para
el Teorema (\ref{Tma.4.2.Dai}):
\begin{Teo}[Teorema 4.1, Dai \cite{Dai}]
Considere una disciplina que cumpla la ley de conservaci\'on, para
casi todas las trayectorias muestrales $\omega$ y cualquier
sucesi\'on de estados iniciales $\left\{x_{n}\right\}\subset
\mathbf{X}$, con $|x_{n}|\rightarrow\infty$, existe una
subsucesi\'on $\left\{x_{n_{j}}\right\}$ con
$|x_{n_{j}}|\rightarrow\infty$ tal que
\begin{equation}\label{Eq.4.15}
\frac{1}{|x_{n_{j}}|}\left(Q^{x_{n_{j}}}\left(0\right),U^{x_{n_{j}}}\left(0\right),V^{x_{n_{j}}}\left(0\right)\right)\rightarrow\left(\overline{Q}\left(0\right),\overline{U},\overline{V}\right),
\end{equation}

\begin{equation}\label{Eq.4.16}
\frac{1}{|x_{n_{j}}|}\left(Q^{x_{n_{j}}}\left(|x_{n_{j}}|t\right),T^{x_{n_{j}}}\left(|x_{n_{j}}|t\right)\right)\rightarrow\left(\overline{Q}\left(t\right),\overline{T}\left(t\right)\right)\textrm{
u.o.c.}
\end{equation}

Adem\'as,
$\left(\overline{Q}\left(t\right),\overline{T}\left(t\right)\right)$
satisface las siguientes ecuaciones:
\begin{equation}\label{Eq.MF.1.3a}
\overline{Q}\left(t\right)=Q\left(0\right)+\left(\alpha
t-\overline{U}\right)^{+}-\left(I-P\right)^{'}M^{-1}\left(\overline{T}\left(t\right)-\overline{V}\right)^{+},
\end{equation}

\begin{equation}\label{Eq.MF.2.3a}
\overline{Q}\left(t\right)\geq0,\\
\end{equation}

\begin{equation}\label{Eq.MF.3.3a}
\overline{T}\left(t\right)\textrm{ es no decreciente y comienza en cero},\\
\end{equation}

\begin{equation}\label{Eq.MF.4.3a}
\overline{I}\left(t\right)=et-C\overline{T}\left(t\right)\textrm{
es no decreciente,}\\
\end{equation}

\begin{equation}\label{Eq.MF.5.3a}
\int_{0}^{\infty}\left(C\overline{Q}\left(t\right)\right)d\overline{I}\left(t\right)=0,\\
\end{equation}

\begin{equation}\label{Eq.MF.6.3a}
\textrm{Condiciones en
}\left(\overline{Q}\left(\cdot\right),\overline{T}\left(\cdot\right)\right)\textrm{
espec\'ificas de la disciplina de la cola,}
\end{equation}
\end{Teo}


Propiedades importantes para el modelo de flujo retrasado:

\begin{Prop}[Proposici\'on 4.2, Dai \cite{Dai}]
 Sea $\left(\overline{Q},\overline{T},\overline{T}^{0}\right)$ un flujo l\'imite de \ref{Eq.Punto.Limite}
 y suponga que cuando $x\rightarrow\infty$ a lo largo de una subsucesi\'on
\[\left(\frac{1}{|x|}Q_{k}^{x}\left(0\right),\frac{1}{|x|}A_{k}^{x}\left(0\right),\frac{1}{|x|}B_{k}^{x}\left(0\right),\frac{1}{|x|}B_{k}^{x,0}\left(0\right)\right)\rightarrow\left(\overline{Q}_{k}\left(0\right),0,0,0\right)\]
para $k=1,\ldots,K$. El flujo l\'imite tiene las siguientes
propiedades, donde las propiedades de la derivada se cumplen donde
la derivada exista:
\begin{itemize}
 \item[i)] Los vectores de tiempo ocupado $\overline{T}\left(t\right)$ y $\overline{T}^{0}\left(t\right)$ son crecientes y continuas con
$\overline{T}\left(0\right)=\overline{T}^{0}\left(0\right)=0$.
\item[ii)] Para todo $t\geq0$
\[\sum_{k=1}^{K}\left[\overline{T}_{k}\left(t\right)+\overline{T}_{k}^{0}\left(t\right)\right]=t.\]
\item[iii)] Para todo $1\leq k\leq K$
\[\overline{Q}_{k}\left(t\right)=\overline{Q}_{k}\left(0\right)+\alpha_{k}t-\mu_{k}\overline{T}_{k}\left(t\right).\]
\item[iv)]  Para todo $1\leq k\leq K$
\[\dot{{\overline{T}}}_{k}\left(t\right)=\rho_{k}\] para $\overline{Q}_{k}\left(t\right)=0$.
\item[v)] Para todo $k,j$
\[\mu_{k}^{0}\overline{T}_{k}^{0}\left(t\right)=\mu_{j}^{0}\overline{T}_{j}^{0}\left(t\right).\]
\item[vi)]  Para todo $1\leq k\leq K$
\[\mu_{k}\dot{{\overline{T}}}_{k}\left(t\right)=l_{k}\mu_{k}^{0}\dot{{\overline{T}}}_{k}^{0}\left(t\right),\] para $\overline{Q}_{k}\left(t\right)>0$.
\end{itemize}
\end{Prop}

\begin{Lema}[Lema 3.1, Chen \cite{Chen}]\label{Lema3.1}
Si el modelo de flujo es estable, definido por las ecuaciones
(3.8)-(3.13), entonces el modelo de flujo retrasado tambi\'en es
estable.
\end{Lema}

\begin{Lema}[Lema 5.2, Gut \cite{Gut}]\label{Lema.5.2.Gut}
Sea $\left\{\xi\left(k\right):k\in\ent\right\}$ sucesi\'on de
variables aleatorias i.i.d. con valores en
$\left(0,\infty\right)$, y sea $E\left(t\right)$ el proceso de
conteo
\[E\left(t\right)=max\left\{n\geq1:\xi\left(1\right)+\cdots+\xi\left(n-1\right)\leq t\right\}.\]
Si $E\left[\xi\left(1\right)\right]<\infty$, entonces para
cualquier entero $r\geq1$
\begin{equation}
lim_{t\rightarrow\infty}\esp\left[\left(\frac{E\left(t\right)}{t}\right)^{r}\right]=\left(\frac{1}{E\left[\xi_{1}\right]}\right)^{r},
\end{equation}
de aqu\'i, bajo estas condiciones
\begin{itemize}
\item[a)] Para cualquier $t>0$,
$sup_{t\geq\delta}\esp\left[\left(\frac{E\left(t\right)}{t}\right)^{r}\right]<\infty$.

\item[b)] Las variables aleatorias
$\left\{\left(\frac{E\left(t\right)}{t}\right)^{r}:t\geq1\right\}$
son uniformemente integrables.
\end{itemize}
\end{Lema}

\begin{Teo}[Teorema 5.1: Ley Fuerte para Procesos de Conteo, Gut
\cite{Gut}]\label{Tma.5.1.Gut} Sea
$0<\mu<\esp\left(X_{1}\right]\leq\infty$. entonces

\begin{itemize}
\item[a)] $\frac{N\left(t\right)}{t}\rightarrow\frac{1}{\mu}$
a.s., cuando $t\rightarrow\infty$.


\item[b)]$\esp\left[\frac{N\left(t\right)}{t}\right]^{r}\rightarrow\frac{1}{\mu^{r}}$,
cuando $t\rightarrow\infty$ para todo $r>0$.
\end{itemize}
\end{Teo}


\begin{Prop}[Proposici\'on 5.1, Dai y Sean \cite{DaiSean}]\label{Prop.5.1}
Suponga que los supuestos (A1) y (A2) se cumplen, adem\'as suponga
que el modelo de flujo es estable. Entonces existe $t_{0}>0$ tal
que
\begin{equation}\label{Eq.Prop.5.1}
lim_{|x|\rightarrow\infty}\frac{1}{|x|^{p+1}}\esp_{x}\left[|X\left(t_{0}|x|\right)|^{p+1}\right]=0.
\end{equation}

\end{Prop}


\begin{Prop}[Proposici\'on 5.3, Dai y Sean \cite{DaiSean}]\label{Prop.5.3.DaiSean}
Sea $X$ proceso de estados para la red de colas, y suponga que se
cumplen los supuestos (A1) y (A2), entonces para alguna constante
positiva $C_{p+1}<\infty$, $\delta>0$ y un conjunto compacto
$C\subset X$.

\begin{equation}\label{Eq.5.4}
\esp_{x}\left[\int_{0}^{\tau_{C}\left(\delta\right)}\left(1+|X\left(t\right)|^{p}\right)dt\right]\leq
C_{p+1}\left(1+|x|^{p+1}\right).
\end{equation}
\end{Prop}

\begin{Prop}[Proposici\'on 5.4, Dai y Sean \cite{DaiSean}]\label{Prop.5.4.DaiSean}
Sea $X$ un proceso de Markov Borel Derecho en $X$, sea
$f:X\leftarrow\rea_{+}$ y defina para alguna $\delta>0$, y un
conjunto cerrado $C\subset X$
\[V\left(x\right):=\esp_{x}\left[\int_{0}^{\tau_{C}\left(\delta\right)}f\left(X\left(t\right)\right)dt\right],\]
para $x\in X$. Si $V$ es finito en todas partes y uniformemente
acotada en $C$, entonces existe $k<\infty$ tal que
\begin{equation}\label{Eq.5.11}
\frac{1}{t}\esp_{x}\left[V\left(x\right)\right]+\frac{1}{t}\int_{0}^{t}\esp_{x}\left[f\left(X\left(s\right)\right)ds\right]\leq\frac{1}{t}V\left(x\right)+k,
\end{equation}
para $x\in X$ y $t>0$.
\end{Prop}


\begin{Teo}[Teorema 5.5, Dai y Sean  \cite{DaiSean}]
Suponga que se cumplen (A1) y (A2), adem\'as suponga que el modelo
de flujo es estable. Entonces existe una constante $k_{p}<\infty$
tal que
\begin{equation}\label{Eq.5.13}
\frac{1}{t}\int_{0}^{t}\esp_{x}\left[|Q\left(s\right)|^{p}\right]ds\leq
k_{p}\left\{\frac{1}{t}|x|^{p+1}+1\right\},
\end{equation}
para $t\geq0$, $x\in X$. En particular para cada condici\'on
inicial
\begin{equation}\label{Eq.5.14}
\limsup_{t\rightarrow\infty}\frac{1}{t}\int_{0}^{t}\esp_{x}\left[|Q\left(s\right)|^{p}\right]ds\leq
k_{p}.
\end{equation}
\end{Teo}

\begin{Teo}[Teorema 6.2 Dai y Sean \cite{DaiSean}]\label{Tma.6.2}
Suponga que se cumplen los supuestos (A1)-(A3) y que el modelo de
flujo es estable, entonces se tiene que
\[\parallel P^{t}\left(x,\cdot\right)-\pi\left(\cdot\right)\parallel_{f_{p}}\rightarrow0,\]
para $t\rightarrow\infty$ y $x\in X$. En particular para cada
condici\'on inicial
\[lim_{t\rightarrow\infty}\esp_{x}\left[\left|Q_{t}\right|^{p}\right]=\esp_{\pi}\left[\left|Q_{0}\right|^{p}\right]<\infty,\]
\end{Teo}

donde

\begin{eqnarray*}
\parallel
P^{t}\left(c,\cdot\right)-\pi\left(\cdot\right)\parallel_{f}=sup_{|g\leq
f|}|\int\pi\left(dy\right)g\left(y\right)-\int
P^{t}\left(x,dy\right)g\left(y\right)|,
\end{eqnarray*}
para $x\in\mathbb{X}$.

\begin{Teo}[Teorema 6.3, Dai y Sean \cite{DaiSean}]\label{Tma.6.3}
Suponga que se cumplen los supuestos (A1)-(A3) y que el modelo de
flujo es estable, entonces con
$f\left(x\right)=f_{1}\left(x\right)$, se tiene que
\[lim_{t\rightarrow\infty}t^{(p-1)}\left|P^{t}\left(c,\cdot\right)-\pi\left(\cdot\right)\right|_{f}=0,\]
para $x\in X$. En particular, para cada condici\'on inicial
\[lim_{t\rightarrow\infty}t^{(p-1)}\left|\esp_{x}\left[Q_{t}\right]-\esp_{\pi}\left[Q_{0}\right]\right|=0.\]
\end{Teo}



\begin{Prop}[Proposici\'on 5.1, Dai y Meyn \cite{DaiSean}]\label{Prop.5.1.DaiSean}
Suponga que los supuestos A1) y A2) son ciertos y que el modelo de
flujo es estable. Entonces existe $t_{0}>0$ tal que
\begin{equation}
lim_{|x|\rightarrow\infty}\frac{1}{|x|^{p+1}}\esp_{x}\left[|X\left(t_{0}|x|\right)|^{p+1}\right]=0.
\end{equation}
\end{Prop}


\begin{Teo}[Teorema 5.5, Dai y Meyn \cite{DaiSean}]\label{Tma.5.5.DaiSean}
Suponga que los supuestos A1) y A2) se cumplen y que el modelo de
flujo es estable. Entonces existe una constante $\kappa_{p}$ tal
que
\begin{equation}
\frac{1}{t}\int_{0}^{t}\esp_{x}\left[|Q\left(s\right)|^{p}\right]ds\leq\kappa_{p}\left\{\frac{1}{t}|x|^{p+1}+1\right\},
\end{equation}
para $t>0$ y $x\in X$. En particular, para cada condici\'on
inicial
\begin{eqnarray*}
\limsup_{t\rightarrow\infty}\frac{1}{t}\int_{0}^{t}\esp_{x}\left[|Q\left(s\right)|^{p}\right]ds\leq\kappa_{p}.
\end{eqnarray*}
\end{Teo}


\begin{Teo}[Teorema 6.4, Dai y Meyn \cite{DaiSean}]\label{Tma.6.4.DaiSean}
Suponga que se cumplen los supuestos A1), A2) y A3) y que el
modelo de flujo es estable. Sea $\nu$ cualquier distribuci\'on de
probabilidad en
$\left(\mathbb{X},\mathcal{B}_{\mathbb{X}}\right)$, y $\pi$ la
distribuci\'on estacionaria de $X$.
\begin{itemize}
\item[i)] Para cualquier $f:X\leftarrow\rea_{+}$
\begin{equation}
\lim_{t\rightarrow\infty}\frac{1}{t}\int_{o}^{t}f\left(X\left(s\right)\right)ds=\pi\left(f\right):=\int
f\left(x\right)\pi\left(dx\right),
\end{equation}
$\prob$-c.s.

\item[ii)] Para cualquier $f:X\leftarrow\rea_{+}$ con
$\pi\left(|f|\right)<\infty$, la ecuaci\'on anterior se cumple.
\end{itemize}
\end{Teo}

\begin{Teo}[Teorema 2.2, Down \cite{Down}]\label{Tma2.2.Down}
Suponga que el fluido modelo es inestable en el sentido de que
para alguna $\epsilon_{0},c_{0}\geq0$,
\begin{equation}\label{Eq.Inestability}
|Q\left(T\right)|\geq\epsilon_{0}T-c_{0}\textrm{,   }T\geq0,
\end{equation}
para cualquier condici\'on inicial $Q\left(0\right)$, con
$|Q\left(0\right)|=1$. Entonces para cualquier $0<q\leq1$, existe
$B<0$ tal que para cualquier $|x|\geq B$,
\begin{equation}
\prob_{x}\left\{\mathbb{X}\rightarrow\infty\right\}\geq q.
\end{equation}
\end{Teo}

\begin{Dem}[Teorema \ref{Tma2.1.Down}] La demostraci\'on de este
teorema se da a continuaci\'on:\\
\begin{itemize}
\item[i)] Utilizando la proposici\'on \ref{Prop.5.3.DaiSean} se
tiene que la proposici\'on \ref{Prop.5.4.DaiSean} es cierta para
$f\left(x\right)=1+|x|^{p}$.

\item[i)] es consecuencia directa del Teorema \ref{Tma.6.2}.

\item[iii)] ver la demostraci\'on dada en Dai y Sean
\cite{DaiSean} p\'aginas 1901-1902.

\item[iv)] ver Dai y Sean \cite{DaiSean} p\'aginas 1902-1903 \'o
\cite{MeynTweedie2}.
\end{itemize}
\end{Dem}
%\newpage
%_________________________________________________________________________
%\subsection{AP\'ENDICE B}




\begin{Assumption}
\label{A:PD}

\begin{enumerate}
\item[a)] $c$ is lower semicontinuous, and inf-compact on $\mathbb{K}$ (i.e.
for every $x\in X$ and $r\in \mathbb{R}$ the set $\{a \in A(x):c(x,a) \leq  r
\}$ is compact).

\item[b)] The transition law $Q$ is strongly continuous, i.e. $u(x,a)=\int
u(y)Q(dy|x,a)$, $(x,a)\in\mathbb{K}$ is continuous and bounded on $\mathbb{K}$, for every
measurable bounded function $u$ on $X$.

\item[c)] There exists a policy $\pi$ such that $V(\pi,x)<\infty$, for each $%
x \in X$.
\end{enumerate}
\end{Assumption}

\begin{Remark}
\label{R:BT}

The following consequences of Assumption \ref{A:PD} are well-known (see
Theorem 4.2.3 and Lemma 4.2.8 in \cite{Hernandez}):

\begin{enumerate}
\item[a)] The optimal value function $V^{\ast}$ is the solution of the
\textit{Optimality Equation} (OE), i.e. for all $x \in X$,
\begin{equation*}
V^{\ast}(x)=\underset{a\in A(x)}{\min }\left\{ c(x,a)+\alpha \int
V^{\ast}(y)Q(dy|x,a)\right\} \text{.}
\end{equation*}

There is also $f^{\ast}\in \mathbb{F}$ such that:
\begin{equation}
V^{\ast}(x)= c(x,f^{\ast}(x))+\alpha \int V^{\ast}(y)Q(dy|x,f^{\ast}(x)), \label{2.1}
\end{equation}
$ x\in X$, and $f^{\ast}$ is optimal.

\item[b)] For every $x \in X$, $v_{n}(x)\uparrow V^{\ast}$, with $v_{n}$
defined as
\begin{equation*}
v_{n}(x)=\underset{a\in A(x)}{\min }\left\{ c(x,a)+\alpha \int
v_{n-1}(y)Q(dy| x,a)\right\},
\end{equation*}
 $x\in X, n=1,2,\cdots $, and $v_{0}(x)=0$. Moreover, for each $n$, there is $%
f_{n}\in \mathbb{F}$ such that, for each $x\in X$,
\begin{equation}
\underset{a\in A(x)}{\min }\left\{ c(x,a)+\alpha \int
v_{n-1}(y)Q(dy|x,a)\right\}= c(x,f_{n}(x))+\alpha \int
v_{n-1}(y)Q(dy|x,f_{n}(x)).  \label{2.2}
\end{equation}
\end{enumerate}
\end{Remark}

Let $(X,A,\{A(x):x\in X\},Q,c)$ be a fixed Markov control model. Take $M$ as the MDP with the Markov control model $(X,A,\{A(x):x\in
X\},Q,c)$. The optimal value function, the optimal policy which comes from (%
\ref{2.1}), and the minimizers in (\ref{2.2}) will be denoted for $M$ by $%
V^{\ast}$, $f^{\ast}$, and $f_{n}$ , $n=1,2,\cdots $, respectively. Also let
$v_{n}$, $n=1,2,\cdots $, be the value iteration functions for $M$. Let $%
G(x,a):=c(x,a)+\alpha \int V^{\ast}(y)Q(dy|x,a)$, $(x,a)\in \mathbb{K}$.

It will be also supposed that the MDPs taken into account satisfy one of the
following Assumptions \ref{A:2} or \ref{A:3}.

\begin{Assumption}
\label{A:2}

\begin{enumerate}
\item[a)] $X$ and $A$ are convex;

\item[b)] $(1- \lambda)a+a^{\prime }\in A((1- \lambda)x+x^{\prime })$ for
all $x$, $x^{\prime }\in X$, $a\in A(x)$, $a^{\prime }\in A(x^{\prime })$
and $\lambda \in [0,1]$. Besides it is assumed that: if $x$ and $y\in X$, $x <
y $, then $A(y)\subseteq A(x)$, and $A(x)$ are convex for each $x \in X$;

\item[c)] $Q$ is induced by a difference equation $x_{t+1}=F(x_{t},a_{t},%
\xi_{t})$, with $t=0,1,\cdots $, where $F:X\times A\times S \rightarrow X$
is a measurable function and $\{\xi_{t}\}$ is a sequence of independent and
identically distributed (i.i.d.) random variables with values in $S \subseteq
\mathbb{R}$, and with a common density $\Delta$. In addition, we suppose
that $F(\cdot,\cdot,s)$ is a convex function on $\mathbb{K}$, for each $s\in
S$; and if $x$ and $y\in X$, $x < y$, then $F(x,a,s)\leq F (y,a,s)$ for each
$a\in A(y)$ and $s\in S$;

\item[d)] $c$ is convex on $\mathbb{K}$, and if  $x$ and $y\in X$, $x < y$,
then $c(x,a)\leq c(y,a)$, for each $a\in A(y)$.
\end{enumerate}
\end{Assumption}

\begin{Assumption}
\label{A:3}

\begin{enumerate}
\item[a)] Same as Assumption \ref{A:2} (a);

\item[b)] $(1- \lambda)a+a^{\prime }\in A((1- \lambda)x+x^{\prime })$ for
all $x$, $x^{\prime }\in X$, $a\in A(x)$, $a^{\prime }\in A(x^{\prime })$
and $\lambda\in [0,1]$. Besides $A(x)$ is assumed to be convex for each $x
\in X$;

\item[c)] $Q$ is given by the relation $x_{t+1}=\gamma x_{t}+\delta
a_{t}+\xi_{t}$, $t=0,1,\cdots $, where $\{\xi_{t}\}$ are i.i.d. random
variables taking values in $S\subseteq \mathbb{R}$ with the density $\Delta$%
, $\gamma$ and $\delta$ are real numbers;

\item[d)] $c$ is convex on $\mathbb{K}$.
\end{enumerate}
\end{Assumption}

\begin{Remark}
\label{R:2} Assumptions \ref{A:2} and \ref{A:3} are essentially presented in
Conditions C1 and C2 in \cite{DRS}, but changing a strictly convex $c(\cdot,
\cdot)$ by a convex $c(\cdot, \cdot)$. (In fact, in \cite{DRS}, Conditions C1
and C2 take into account the more general situation in which both $X$ and $A$
are subsets of Euclidean spaces of the dimension greater than one.)
Also note that it is possible to obtain that each of Assumptions \ref{A:2}
and \ref{A:3} implies that, for each $x\in X$, $G(x,\cdot)$ is convex but
not necessarily strictly convex (hence, $M$ does not necessarily have a
unique optimal policy). The proof of this fact is a direct consequence of
the convexity of the cost function $c$, and of the proof of Lemma 6.2 in
\cite{DRS}.
\end{Remark}




\begin{Assumption}
\label{A:4} There is a policy $\phi$ such that $E_{x}^{\phi }\left[ \text{$\sum\limits_{t=0}^{\infty }$}\alpha
^{t}c^*(x_{t},a_{t})\right] \text{}<\infty$%
, for each $x\in X$.
\end{Assumption}

\begin{Remark}
\label{R:3} Suppose that, for M, Assumption 2.1 holds. Then, it is direct to verify that if $M_{\epsilon}$ satisfies Assumption \ref{A:4}, then it also
satisfies Assumption \ref{A:PD}.
\end{Remark}

\begin{Condition}
\label{C:1} There exists a measurable function $Z:X\rightarrow \mathbb{R}$,
which may depend on $\alpha$, such that $c^{%
\ast}(x,a)-c(x,a)=\epsilon a^{2}\leq\epsilon Z(x)$, and $\int
Z(y)Q(dy|x,a)\leq Z(x)$ for each $x\in X$ and $a\in B(x)$.
\end{Condition}

\begin{Theorem}
\label{T:1} Suppose that Assumptions \ref{A:PD} and \ref{A:4} hold, and
that, for $M$, one of Assumptions \ref{A:2} or \ref{A:3} holds. Let $%
\epsilon $ be a positive number. Then,

\begin{enumerate}
\item[a)] If $A$ is compact, $|W^{\ast}(x)-V^{\ast}(x)|\leq \epsilon K^{2}/(1-\alpha)$%
, $x\in X$, where $K$ is the diameter of a compact set $D$ such that $0\in D$
and $A\subseteq D$.

\item[b)] Under Condition \ref{C:1}, $|W^{\ast}(x) - V^{\ast}(x)|\leq
\epsilon Z(x)/(1- \alpha)$, $x\in X$.
\end{enumerate}
\end{Theorem}

\begin{proof}
The proof of case (a) follows from the proof of case (b) given that $Z(x)=K^{2}$, $x\in X$. (Observe that in this case, if $a\in A$,
then $a^{2}=(a-0)^{2} \leq K^{2}$.)

\textbf{(b)} Assume that $M$ satisfies Assumption \ref{A:2}. (The proof for
the case in which $M$ satisfies Assumption \ref{A:3} is similar.)

\end{proof}

The following Corollary  is immediate.

\begin{Corollary}\label{Co:1}
Suppose that Assumptions \ref{A:PD} and \ref{A:4} hold. Suppose
that for $M$ one of Assumptions \ref{A:2} or \ref{A:3} holds (hence $M$
does not necessarily have a unique optimal policy). Let $\epsilon $ be a
positive number. If $A$ is compact or Condition \ref{C:1} holds, then there
exists an MDP $M_{\epsilon }$ with a unique optimal policy $g^{\ast }$, such
that inequalities in Theorem 3.7 (a) or (b) hold, respectively.
\end{Corollary}

\begin{Example}\label{E:1}
Ejemplo1
\end{Example}

\begin{Lemma}\label{L:1}
Lema1
\end{Lemma}

\begin{proof}
Assumption \ref{A:PD} (a) trivially holds. The proof of the strong continuity of $Q$

\end{proof}





%______________________________________________________________________
\section{Preliminaries: }
%______________________________________________________________________

Consider a Network consisting in two cyclic polling systems with two queues each other, $Q_{1}, Q_{2}$ for the first system and $\hat{Q}_{1},\hat{Q}_{2}$ for the second one, each with infinite-sized buffer. In each system a single server visits the queues in cyclic order, where he applies the exhaustive policy, i.e., when the server polls a queue, he serves all the customers present until the queue becomes empty.


At the second system the customers at queue 2 moves to the first system's queue 2, we assume that the network is open; that is, all customers eventually leave the network. As usually in Polling Systems Theory we assume the arrivals in each queue the arrival processes are Poisson whit i.i.d. interarrival times, their service times are also i.i.d. and finally upon completion of a visit at any queue, the servers incurs in a random switchover time according to an arbitray distribution.  We define a cycle to be the time interval between two consecutive polling instants, the time period in a cycle during which the server is serving a queue is called a service period. The queues are attended in cyclic order.

Time is slotted with slot size equal to the service time of a fixed costumer, we call the time interval $\left[t,t+1\right]$ the $t$-th slot. The arrival processes are denoted by $X_{1}\left(t\right),X_{2}\left(t\right)$ for the first system and $\hat{X}_{1}\left(t\right)$ ,$\hat{X}_{2}\left(t\right)$ for the second, the arrival rate at $Q_{i}$ and $\hat{Q}_{i}$ is denoted by $\mu_{i}$ and $\hat{\mu}_{i}$ respectively, with the condition $\mu_{i}<1$ and $\hat{\mu}_{i}<1$. The users arrives in a independent form at each of the queues.

We define the process $Y_{2}$ to consider the costumers who pass from system 2, to system 1, with arrival rate $\tilde{\mu}_{2}$. The service time customers of queue $i$ is a random variable $\tau_{i}$ with process defined by $S_{i}$. In similar manner the switchover period following the service of queue $i$ is an independent random variable $R_{i}$ with general distribution. To determine the length of the queues, i.e., the number of users in the queue at the moment the server arrives we define the process $L_{i}$ and $\hat{L}_{i}$ for the first and second system respectively. In the sequel, we use the buffer occupancy method to obtain the generating function, first and second moments of queue size distributions at polling instants.

At each of the queues in the network the number of users is the number of users at the time the server arrives plus the numbers of arrivals during the service time.

%____________________________________________________________________________________________________
%\subsection{Probability Generating Functions}
%____________________________________________________________________________________________________

In order to obtain the joint probability generating function (PGF) for the number or users residing in queue $i$ when the queue is polled in the NCPS, we define for each of the arrival processes $X_{i},\hat{X}_{i}$, $i=1,2$,  $Y_{2}$ and $\tilde{X}_{2}$ with $\tilde{X}_{2}=X_{2}+Y_{2}$, their PGF $P_{i}\left(z_{i}\right)=\esp\left[z_{i}^{X_{i}\left(t\right)}\right],\hat{P}_{i}\left(w_{i}\right)=\esp\left[w_{i}^{\hat{X}_{i}\left(t\right)}\right]$, for $i=1,2$, and $\check{P}_{2}\left(z_{2}\right)=\esp\left[z_{2}^{Y_{2}\left(t\right)}\right], \tilde{P}_{2}\left(z_{2}\right)=\esp\left[z_{2}^{\tilde{X}_{2}\left(t\right)}\right]$ , with first moment given by $\mu_{i}=\esp\left[X_{i}\left(t\right)\right]=P_{i}^{(1)}\left(1\right), \hat{\mu}_{i}=\esp\left[\hat{X}_{i}\left(t\right)\right]=\hat{P}_{i}^{(1)}\left(1\right)$, for $i=1,2$, while $\check{\mu}_{2}=\esp\left[Y_{2}\left(t\right)\right]=\check{P}_{2}^{(1)}\left(1\right),\tilde{\mu}_{2}=\esp\left[\tilde{X}_{2}\left(t\right)\right]=\tilde{P}_{2}^{(1)}\left(1\right)$.

The PGF For the service time is defined by: $S_{i}\left(z_{i}\right)=\esp\left[z_{i}^{\overline{\tau}_{i}-\tau_{i}}\right]$ y $\hat{S}_{i}\left(w_{i}\right)=\esp\left[w_{i}^{\overline{\zeta}_{i}-\zeta_{i}}\right]$, with first moment $s_{i}=\esp\left[\overline{\tau}_{i}-\tau_{i}\right]$ y $\hat{s}_{i}=\esp\left[\overline{\zeta}_{i}-\zeta_{i}\right]$, for $i=1,2$.

In a similar manner the PGF for the switchover time of the server from the moment it ends to attend a queue to the time of arrival to the next queue are given by $R_{i}\left(z_{i}\right)=\esp\left[z_{1}^{\tau_{i+1}-\overline{\tau}_{i}}\right]$ and $\hat{R}_{i}\left(w_{i}\right)=\esp\left[w_{i}^{\zeta_{i+1}-\overline{\zeta}_{i}}\right]$ with first moment $r_{i}=R_{i}^{(1)}\left(1\right)=\esp\left[\tau_{i+1}-\overline{\tau}_{i}\right]$ and $\hat{r}_{i}=\hat{R}_{i}^{(1)}\left(1\right)=\esp\left[\zeta_{i+1}-\overline{\zeta}_{i}\right]$ with $i=1,2$.

The number of users in the queue at time $\overline{\tau}_{1},\overline{\tau}_{2}, \overline{\zeta}_{1},\overline{\zeta}_{2}$, it's zero, i.e.,
 $L_{i}\left(\overline{\tau_{i}}\right)=0,$ and $\hat{L}_{i}\left(\overline{\zeta_{i}}\right)=0$ for $i=1,2$. Then the number of users in the queue of the second system at the moment the server ends attending in the queue is given by the number of users present at the moment it arrives plus the number of arrivals during the service time, i.e., $\hat{L}_{i}\left(\overline{\tau}_{j}\right)=\hat{L}_{i}\left(\tau_{j}\right)+\hat{X}_{i}\left(\overline{\tau}_{j}-\tau_{j}\right)$, for $i,j=1,2$, meanwhile for the first system : $L_{1}\left(\overline{\tau}_{j}\right)=L_{1}\left(\tau_{j}\right)+X_{1}\left(\overline{\tau}_{j}-\tau_{j}\right)$. Specifically for the second queue of the first system we need to consider the users of transfer becoming from the second queue in the second system while the server it's in the other queue attending, it means that this users have been aready attended by the server before they can go to the first system:

\begin{equation}\label{Eq.UsuariosTotalesZ2}
L_{2}\left(\overline{\tau}_{1}\right)=L_{2}\left(\tau_{1}\right)+X_{2}\left(\overline{\tau}_{1}-\tau_{1}\right)+Y_{2}\left(\overline{\tau}_{1}-\tau_{1}\right).
\end{equation}

%_________________________________________________________________________
%\subsection{Gambler's ruin problem}
%_________________________________________________________________________

As is know the gambler's ruin problem can be used to model the server's busy period in a Cyclic Polling System, so let $\tilde{L}_{0}\geq0$ the number of users present at the moment the server arrives to start serving, also let $T$ be the time the server need to attend the users in the queue starting with $\tilde{L}_{0}$ users.


Suppose the gambler has two simultaneous, independent and simultaneous moves, such events are independent and identical to each other for each realization. The gain on the $n$-th game is $\tilde{X}_{n}=X_{n}+Y_{n}$ units from which is substracted a playing fee of 1 unit for each move. His PGF is given by $F\left(z\right)=\esp\left[z^{\tilde{L}_{0}}\right]$, futhermore
$$\tilde{P}\left(z\right)=\esp\left[z^{\tilde{X}_{n}}\right]=\esp\left[z^{X_{n}+Y_{n}}\right]=\esp\left[z^{X_{n}}z^{Y_{n}}\right]=\esp\left[z^{X_{n}}\right]\esp\left[z^{Y_{n}}\right]=P\left(z\right)\check{P}\left(z\right),$$

with $\tilde{\mu}=\esp\left[\tilde{X}_{n}\right]=\tilde{P}\left[z\right]<1$. If  $\tilde{L}_{n}$ denotes the capital remaining after the $n$-th game, then $$\tilde{L}_{n}=\tilde{L}_{0}+\tilde{X}_{1}+\tilde{X}_{2}+\cdots+\tilde{X}_{n}-2n.$$

The result that relates the gambler's ruin problem with the busy period of the serverit's a generalization of the result given in Takagi \cite{Takagi} chapter 3.


\textbf{Proposition} \ref{Prop.1.1.2Sa}
Let's $G_{n}\left(z\right)$ and $G\left(z,w\right)$ defined as in
(\ref{Eq.3.16.a.2SA}), then

\begin{eqnarray*}%\label{Eq.Pag.45}
G_{n}\left(z\right)=\frac{1}{z}\left[G_{n-1}\left(z\right)-G_{n-1}\left(0\right)\right]\tilde{P}\left(z\right).
\end{eqnarray*}

Futhermore

\begin{eqnarray*}%\label{Eq.Pag.46}
G\left(z,w\right)=\frac{zF\left(z\right)-wP\left(z\right)G\left(0,w\right)}{z-wR\left(z\right)},
\end{eqnarray*}

with a unique pole in the unit circle, also the pole is of the form $z=\theta\left(w\right)$ and satisfies

\begin{enumerate}
\item[i)]$\tilde{\theta}\left(1\right)=1$,

\item[ii)] $\tilde{\theta}^{(1)}\left(1\right)=\frac{1}{1-\tilde{\mu}}$,

\item[iii)]
$\tilde{\theta}^{(2)}\left(1\right)=\frac{\tilde{\mu}}{\left(1-\tilde{\mu}\right)^{2}}+\frac{\tilde{\sigma}}{\left(1-\tilde{\mu}\right)^{3}}$.
\end{enumerate}

Finally the following satisfies $\esp\left[w^{T}\right]=G\left(0,w\right)=F\left[\tilde{\theta}\left(w\right)\right].$
%\end{Prop}

\textbf{Corollary} \ref{Corolario1.A} The first and second moments for the gambler's ruin are

\begin{eqnarray*}
\begin{array}{ll}
\esp\left[T\right]=\frac{\esp\left[\tilde{L}_{0}\right]}{1-\tilde{\mu}},&
Var\left[T\right]=\frac{Var\left[\tilde{L}_{0}\right]}{\left(1-\tilde{\mu}\right)^{2}}+\frac{\sigma^{2}\esp\left[\tilde{L}_{0}\right]}{\left(1-\tilde{\mu}\right)^{3}}.
\end{array}
\end{eqnarray*}
%_____________________________________________________________________
%__________________________________________________________________________
%\subsection{Arrival Processes in the Queues for NCPS}
%__________________________________________________________________________

In order to model the network of cyclic polling system it's necessary to define the arrival processes for the queues belonging to the system that the server doesn't correspond. In the case of the first system and the server arrive to a queue in the second one:$F_{i,j}\left(z_{i};\zeta_{j}\right)=\esp\left[z_{i}^{L_{i}\left(\zeta_{j}\right)}\right]=
\sum_{k=0}^{\infty}\prob\left[L_{i}\left(\zeta_{j}\right)=k\right]z_{i}^{k}$for $i,j=1,2$. For the second system and the server arrives to a queue in the first system $\hat{F}_{i,j}\left(w_{i};\tau_{j}\right)=\esp\left[w_{i}^{\hat{L}_{i}\left(\tau_{j}\right)}\right] =\sum_{k=0}^{\infty}\prob\left[\hat{L}_{i}\left(\tau_{j}\right)=k\right]w_{i}^{k}$ for $i,j=1,2$. With the developed we can define the joint PGF for the second system:


\begin{eqnarray*}
\esp\left[w_{1}^{\hat{L}_{1}\left(\tau_{j}\right)}w_{2}^{\hat{L}_{2}\left(\tau_{j}\right)}\right]
&=&\esp\left[w_{1}^{\hat{L}_{1}\left(\tau_{j}\right)}\right]
\esp\left[w_{2}^{\hat{L}_{2}\left(\tau_{j}\right)}\right]=\hat{F}_{1,j}\left(w_{1};\tau_{j}\right)\hat{F}_{2,j}\left(w_{2};\tau_{j}\right)=\hat{F}_{j}\left(w_{1},w_{2};\tau_{j}\right).
\end{eqnarray*}

In a similar manner we defin the joint PGF for the first system, and the second system's server

\begin{eqnarray*}
\esp\left[z_{1}^{L_{1}\left(\zeta_{j}\right)}z_{2}^{L_{2}\left(\zeta_{j}\right)}\right]
&=&\esp\left[z_{1}^{L_{1}\left(\zeta_{j}\right)}\right]
\esp\left[z_{2}^{L_{2}\left(\zeta_{j}\right)}\right]=F_{1,j}\left(z_{1};\zeta_{j}\right)F_{2,j}\left(z_{2};\zeta_{j}\right)=F_{j}\left(z_{1},z_{2};\zeta_{j}\right).
\end{eqnarray*}

Now we proceed to determine the joint PGF for the times that the server visit each queue in each system, i.e., $t=\left\{\tau_{1},\tau_{2},\zeta_{1},\zeta_{2}\right\}$:

\begin{eqnarray}\label{Eq.Conjuntas}
\begin{array}{ll}
F_{j}\left(z_{1},z_{2},w_{1},w_{2}\right)=\esp\left[\prod_{i=1}^{2}z_{i}^{L_{i}\left(\tau_{j}
\right)}\prod_{i=1}^{2}w_{i}^{\hat{L}_{i}\left(\tau_{j}\right)}\right],&
\hat{F}_{j}\left(z_{1},z_{2},w_{1},w_{2}\right)=\esp\left[\prod_{i=1}^{2}z_{i}^{L_{i}
\left(\zeta_{j}\right)}\prod_{i=1}^{2}w_{i}^{\hat{L}_{i}\left(\zeta_{j}\right)}\right]
\end{array}
\end{eqnarray}
for $j=1,2$. Then with the purpose of find the number of users present in the netwotk when the server ends attending one of the queues in any of the systems

\begin{eqnarray*}
&&\esp\left[z_{1}^{L_{1}\left(\overline{\tau}_{1}\right)}z_{2}^{L_{2}\left(\overline{\tau}_{1}\right)}w_{1}^{\hat{L}_{1}\left(\overline{\tau}_{1}\right)}w_{2}^{\hat{L}_{2}\left(\overline{\tau}_{1}\right)}\right]
=\esp\left[z_{2}^{L_{2}\left(\overline{\tau}_{1}\right)}w_{1}^{\hat{L}_{1}\left(\overline{\tau}_{1}
\right)}w_{2}^{\hat{L}_{2}\left(\overline{\tau}_{1}\right)}\right]\\
&=&\esp\left[z_{2}^{L_{2}\left(\tau_{1}\right)+X_{2}\left(\overline{\tau}_{1}-\tau_{1}\right)+Y_{2}\left(\overline{\tau}_{1}-\tau_{1}\right)}w_{1}^{\hat{L}_{1}\left(\tau_{1}\right)+\hat{X}_{1}\left(\overline{\tau}_{1}-\tau_{1}\right)}w_{2}^{\hat{L}_{2}\left(\tau_{1}\right)+\hat{X}_{2}\left(\overline{\tau}_{1}-\tau_{1}\right)}\right]
\end{eqnarray*}

using the equation(\ref{Eq.UsuariosTotalesZ2}) we have


\begin{eqnarray*}
&=&\esp\left[z_{2}^{L_{2}\left(\tau_{1}\right)}z_{2}^{X_{2}\left(\overline{\tau}_{1}-\tau_{1}\right)}z_{2}^{Y_{2}\left(\overline{\tau}_{1}-\tau_{1}\right)}w_{1}^{\hat{L}_{1}\left(\tau_{1}\right)}w_{1}^{\hat{X}_{1}\left(\overline{\tau}_{1}-\tau_{1}\right)}w_{2}^{\hat{L}_{2}\left(\tau_{1}\right)}w_{2}^{\hat{X}_{2}\left(\overline{\tau}_{1}-\tau_{1}\right)}\right]\\
&=&\esp\left[z_{2}^{L_{2}\left(\tau_{1}\right)}\left\{w_{1}^{\hat{L}_{1}\left(\tau_{1}\right)}w_{2}^{\hat{L}_{2}\left(\tau_{1}\right)}\right\}\left\{z_{2}^{X_{2}\left(\overline{\tau}_{1}-\tau_{1}\right)}
z_{2}^{Y_{2}\left(\overline{\tau}_{1}-\tau_{1}\right)}w_{1}^{\hat{X}_{1}\left(\overline{\tau}_{1}-\tau_{1}\right)}w_{2}^{\hat{X}_{2}\left(\overline{\tau}_{1}-\tau_{1}\right)}\right\}\right]
\end{eqnarray*}

applying the fact that the arrivals processes in the queues in each systems are independent:

\begin{eqnarray*}
&=&\esp\left[z_{2}^{L_{2}\left(\tau_{1}\right)}\left\{z_{2}^{X_{2}\left(\overline{\tau}_{1}-\tau_{1}\right)}z_{2}^{Y_{2}\left(\overline{\tau}_{1}-\tau_{1}\right)}w_{1}^{\hat{X}_{1}\left(\overline{\tau}_{1}-\tau_{1}\right)}w_{2}^{\hat{X}_{2}\left(\overline{\tau}_{1}-\tau_{1}\right)}\right\}\right]\esp\left[w_{1}^{\hat{L}_{1}\left(\tau_{1}\right)}w_{2}^{\hat{L}_{2}\left(\tau_{1}\right)}\right]
\end{eqnarray*}

given that the arrival processes in the queues are independent, it's possible to separate the expectation for the arrival processes in $Q_{1}$ and $Q_{2}$ at time $\tau_{1}$, which is the time the server visits $Q_{1}$. Considering
$\tilde{X}_{2}\left(z_{2}\right)=X_{2}\left(z_{2}\right)+Y_{2}\left(z_{2}\right)$ we have


\begin{eqnarray*}
&=&\esp\left[z_{2}^{L_{2}\left(\tau_{1}\right)}\left\{z_{2}^{\tilde{X}_{2}\left(\overline{\tau}_{1}-\tau_{1}\right)}w_{1}^{\hat{X}_{1}\left(\overline{\tau}_{1}-\tau_{1}\right)}w_{2}^{\hat{X}_{2}\left(\overline{\tau}_{1}-\tau_{1}\right)}\right\}\right]\esp\left[w_{1}^{\hat{L}_{1}\left(\tau_{1}\right)}w_{2}^{\hat{L}_{2}\left(\tau_{1}\right)}\right]=\esp\left[z_{2}^{L_{2}\left(\tau_{1}\right)}\left\{\tilde{P}_{2}\left(z_{2}\right)^{\overline{\tau}_{1}-\tau_{1}}\hat{P}_{1}\left(w_{1}\right)^{\overline{\tau}_{1}-\tau_{1}}\right.\right.\\
&&\left.\left.\hat{P}_{2}\left(w_{2}\right)^{\overline{\tau}_{1}-\tau_{1}}\right\}\right]\esp\left[w_{1}^{\hat{L}_{1}\left(\tau_{1}\right)}w_{2}^{\hat{L}_{2}\left(\tau_{1}\right)}\right]
=\esp\left[z_{2}^{L_{2}\left(\tau_{1}\right)}\left\{\tilde{P}_{2}\left(z_{2}\right)\hat{P}_{1}\left(w_{1}\right)\hat{P}_{2}\left(w_{2}\right)\right\}^{\overline{\tau}_{1}-\tau_{1}}\right]\esp\left[w_{1}^{\hat{L}_{1}\left(\tau_{1}\right)}w_{2}^{\hat{L}_{2}\left(\tau_{1}\right)}\right]\\
&=&\esp\left[z_{2}^{L_{2}\left(\tau_{1}\right)}\theta_{1}\left(\tilde{P}_{2}\left(z_{2}\right)\hat{P}_{1}\left(w_{1}\right)\hat{P}_{2}\left(w_{2}\right)\right)^{L_{1}\left(\tau_{1}\right)}\right]\esp\left[w_{1}^{\hat{L}_{1}\left(\tau_{1}\right)}w_{2}^{\hat{L}_{2}\left(\tau_{1}\right)}\right]
=F_{1}\left(\theta_{1}\left(\tilde{P}_{2}\left(z_{2}\right)\hat{P}_{1}\left(w_{1}\right)\hat{P}_{2}\left(w_{2}\right)\right),z{2}\right)\\
&&\cdot\hat{F}_{1}\left(w_{1},w_{2};\tau_{1}\right)\equiv
F_{1}\left(\theta_{1}\left(\tilde{P}_{2}\left(z_{2}\right)\hat{P}_{1}\left(w_{1}\right)\hat{P}_{2}\left(w_{2}\right)\right),z_{2},w_{1},w_{2}\right).
\end{eqnarray*}

The last equalities  are true because the number of arrivals to $\hat{Q}_{2}$
during the time interval $\left[\tau_{1},\overline{\tau}_{1}\right]$ still haven't been attended by the server in the system 2, then the users can't pass to the first system through the queue $Q_{2}$. Therefore the number of users switching from $\hat{Q}_{2}$ to $Q_{2}$ during the time interval $\left[\tau_{1},\overline{\tau}_{1}\right]$ depends on the policy of transfer between the two systems, according to the last section

\begin{eqnarray*}\label{Eq.Fs}
\begin{array}{l}
\esp\left[z_{1}^{L_{1}\left(\overline{\tau}_{1}\right)}z_{2}^{L_{2}\left(\overline{\tau}_{1}
\right)}w_{1}^{\hat{L}_{1}\left(\overline{\tau}_{1}\right)}w_{2}^{\hat{L}_{2}\left(
\overline{\tau}_{1}\right)}\right]=F_{1}\left(\theta_{1}\left(\tilde{P}_{2}\left(z_{2}\right)
\hat{P}_{1}\left(w_{1}\right)\hat{P}_{2}\left(w_{2}\right)\right),z_{2},w_{1},w_{2}\right)\\
=F_{1}\left(\theta_{1}\left(\tilde{P}_{2}\left(z_{2}\right)\hat{P}_{1}\left(w_{1}\right)\hat{P}_{2}\left(w_{2}\right)\right),z_{2}\right)\hat{F}_{1}\left(w_{1},w_{2};\tau_{1}\right)
\end{array}
\end{eqnarray*}

Using reasoning similar for the rest of the server's arrival times

\begin{eqnarray*}
\esp\left[z_{1}^{L_{1}\left(\overline{\tau}_{2}\right)}z_{2}^{L_{2}\left(\overline{\tau}_{2}\right)}w_{1}^{\hat{L}_{1}\left(\overline{\tau}_{2}\right)}w_{2}^{\hat{L}_{2}\left(\overline{\tau}_{2}\right)}\right]&=&F_{2}\left(z_{1},\tilde{\theta}_{2}\left(P_{1}\left(z_{1}\right)\hat{P}_{1}\left(w_{1}\right)\hat{P}_{2}\left(w_{2}\right)\right)\right)
\hat{F}_{2}\left(w_{1},w_{2};\tau_{2}\right)\\
\esp\left[z_{1}^{L_{1}\left(\overline{\zeta}_{1}\right)}z_{2}^{L_{2}\left(\overline{\zeta}_{1}
\right)}w_{1}^{\hat{L}_{1}\left(\overline{\zeta}_{1}\right)}w_{2}^{\hat{L}_{2}\left(
\overline{\zeta}_{1}\right)}\right]
&=&F_{1}\left(z_{1},z_{2};\zeta_{1}\right)\hat{F}_{1}\left(\hat{\theta}_{1}\left(P_{1}\left(z_{1}\right)\tilde{P}_{2}\left(z_{2}\right)\hat{P}_{2}\left(w_{2}\right)\right),w_{2}\right),\\
\esp\left[z_{1}^{L_{1}\left(\overline{\zeta}_{2}\right)}z_{2}^{L_{2}\left(\overline{\zeta}_{2}\right)}w_{1}^{\hat{L}_{1}\left(\overline{\zeta}_{2}\right)}w_{2}^{\hat{L}_{2}\left(\overline{\zeta}_{2}\right)}\right]
&=&F_{2}\left(z_{1},z_{2};\zeta_{2}\right)\hat{F}_{2}\left(w_{1},\hat{\theta}_{2}\left(P_{1}\left(z_{1}\right)\tilde{P}_{2}\left(z_{2}\right)\hat{P}_{1}\left(w_{1}
\right)\right)\right).
\end{eqnarray*}
%__________________________________________________________________________
%\subsection{Recursive equations for the NCPS}
%__________________________________________________________________________
Now we are in conditions to obtain the recursive equations that model the NCPS we need to consider the swithcover times that the server ocuppies to translate from one queue to another and, the number or user presents in the system at the time the server leaves to queue to start attending the next. Thus far developed, we can find that for the NCPS:

\begin{eqnarray}\label{Recursive.Equations.First.Casse}
\begin{array}{l}
F_{2}\left(z_{1},z_{2},w_{1},w_{2}\right)=R_{1}\left(P_{1}\left(z_{1}\right)\tilde{P}_{2}\left(z_{2}\right)\prod_{i=1}^{2}
\hat{P}_{i}\left(w_{i}\right)\right)F_{1}\left(\theta_{1}\left(\tilde{P}_{2}\left(z_{2}\right)\hat{P}_{1}\left(w_{1}\right)\hat{P}_{2}\left(w_{2}\right)\right),z_{2}\right)\hat{F}_{1}\left(w_{1},w_{2};\tau_{1}\right),\\
F_{1}\left(z_{1},z_{2},w_{1},w_{2}\right)=R_{2}\left(P_{1}\left(z_{1}\right)\tilde{P}_{2}\left(z_{2}\right)\prod_{i=1}^{2}
\hat{P}_{i}\left(w_{i}\right)\right)F_{2}\left(z_{1},\tilde{\theta}_{2}\left(P_{1}\left(z_{1}\right)\hat{P}_{1}\left(w_{1}\right)\hat{P}_{2}\left(w_{2}\right)\right)\right)
\hat{F}_{2}\left(w_{1},w_{2};\tau_{2}\right),\\
\hat{F}_{2}\left(z_{1},z_{2},w_{1},w_{2}\right)=\hat{R}_{1}\left(P_{1}\left(z_{1}\right)\tilde{P}_{2}\left(z_{2}\right)\prod_{i=1}^{2}
\hat{P}_{i}\left(w_{i}\right)\right)F_{1}\left(z_{1},z_{2};\zeta_{1}\right)\hat{F}_{1}\left(\hat{\theta}_{1}\left(P_{1}\left(z_{1}\right)\tilde{P}_{2}\left(z_{2}\right)\hat{P}_{2}\left(w_{2}\right)\right),w_{2}\right),\\
\hat{F}_{1}\left(z_{1},z_{2},w_{1},w_{2}\right)=\hat{R}_{2}\left(P_{1}\left(z_{1}\right)\tilde{P}_{2}\left(z_{2}\right)\prod_{i=1}^{2}
\hat{P}_{i}\left(w_{i}\right)\right)F_{2}\left(z_{1},z_{2};\zeta_{2}\right)\hat{F}_{2}\left(w_{1},\hat{\theta}_{2}\left(P_{1}\left(z_{1}\right)\tilde{P}_{2}\left(z_{2}\right)\hat{P}_{1}\left(w_{1}
\right)\right)\right).
\end{array}
\end{eqnarray}


%______________________________________________________________________
\section{Main Result and An Example}
%______________________________________________________________________
%\begin{figure}[H]\caption{Network of Cyclic Polling System with double bidirectional transfer}
%\centering
%%\includegraphics[width=9cm]{Grafica4.jpg}
%%\end{figure}\label{FigureRSVC3}


%_____________________________________________________
%\subsubsection{Server Switchover times}
%_____________________________________________________
It's necessary to give an step ahead, considering the case illustrated in \texttt{Figure 1}, where just like before, the server's switchover times are given by the generals equations
$R_{i}\left(\mathbf{z,w}\right)=R_{i}\left(\tilde{P}_{1}\left(z_{1}\right)
\tilde{P}_{2}\left(z_{2}\right)\tilde{P}_{3}\left(z_{3}\right)
\tilde{P}_{4}\left(z_{4}\right)\right)$, with first order derivatives given by $D_{i}R_{i}=r_{i}\tilde{\mu}_{i}$, and second order partial derivatives $D_{j}D_{i}R_{k}=R_{k}^{(2)}\tilde{\mu}_{i}\tilde{\mu}_{j}+\indora_{i=j}r_{k}P_{i}^{(2)}+\indora_{i=j}r_{k}\tilde{\mu}_{i}\tilde{\mu}_{j}$ for any $i,j,k$. According to the equations given before, the queue lengths for the other sytem's server times, we can obtain general expressions, so for
$F_{1}\left(z_{1},z_{2};\tau_{3}\right)$, $F_{2}\left(z_{1},z_{2};\tau_{4}\right)$, $F_{3}\left(z_{3},z_{4};\tau_{1}\right)$, $F_{4}\left(z_{3},z_{4};\tau_{2}\right)$, we can obtain general expressions,

\begin{eqnarray}\label{Ec.Gral.Primer.Momento.Ind.Exh}
\begin{array}{ll}
D_{j}F_{i}\left(z_{1},z_{2};\tau_{i+2}\right)=\indora_{j\leq2}F_{j,i+2}^{(1)},&
D_{j}F_{i}\left(z_{3},z_{4};\tau_{i-2}\right)=\indora_{j\geq3}F_{j,i-2}^{(1)}
\end{array}
\end{eqnarray}

for $i=1,2,3,4$ and $j=1,2,3,4$. With second order derivatives given by.

\begin{eqnarray}\label{Ec.Gral.Segundo.Momento.Ind.Exh}
\begin{array}{l}
D_{j}D_{i}F_{k}\left(z_{1},z_{2};\tau_{k+2}\right)=\indora_{i\geq3}\indora_{j=i}F_{i,k+2}^{(2)}+\indora_{i\geq 3}\indora_{j\neq i}F_{j,k-2}^{(1)}F_{i,k+2}^{(1)}\\
D_{j}D_{i}F_{k}\left(z_{3},z_{4};\tau_{k-2}\right)=\indora_{i\geq3}\indora_{j=i}F_{i,k-2}^{(2)}+\indora_{i\geq 3}\indora_{j\neq i}F_{j,k-2}^{(1)}F_{i,k-2}^{(1)}
\end{array}
\end{eqnarray}


 According with the developed at the moment, we can get the recursive equations which are of the following form

\begin{eqnarray}\label{General.System.Double.Transfer}
\begin{array}{l}
F_{1}\left(z_{1},z_{2},z_{3},z_{4}\right)=R_{2}\left(\prod_{i=1}^{4}\tilde{P}_{i}\left(z_{i}\right)\right)F_{2}\left(z_{1},\tilde{\theta}_{2}\left(\tilde{P}_{1}\left(z_{1}\right)\tilde{P}_{3}\left(z_{3}\right)\tilde{P}_{4}\left(z_{4}\right)\right)\right)
F_{4}\left(z_{3},z_{4};\tau_{2}\right),\\
F_{2}\left(z_{1},z_{2},z_{3},z_{4}\right)=R_{1}\left(\prod_{i=1}^{4}\tilde{P}_{i}\left(z_{i}\right)\right)
F_{1}\left(\tilde{\theta}_{1}\left(\tilde{P}_{2}\left(z_{2}\right)\tilde{P}_{3}\left(z_{3}\right)\tilde{P}_{4}\left(z_{4}\right)\right),z_{2}\right)
F_{3}\left(z_{3},z_{4};\tau_{1}\right),\\
F_{3}\left(z_{1},z_{2},z_{3},z_{4}\right)=R_{4}\left(\prod_{i=1}^{4}\tilde{P}_{i}\left(z_{i}\right)\right)
F_{4}\left(z_{3},\tilde{\theta}_{4}\left(\tilde{P}_{1}\left(z_{1}\right)\tilde{P}_{2}\left(z_{2}\right)\tilde{P}_{3}\left(z_{3}\right)
\right)\right)
F_{2}\left(z_{1},z_{2};\tau_{4}\right),\\
F_{4}\left(z_{1},z_{2},z_{3},z_{4}\right)=R_{3}\left(\prod_{i=1}^{4}\tilde{P}_{i}\left(z_{i}\right)\right)
F_{3}\left(\tilde{\theta}_{3}\left(\tilde{P}_{1}\left(z_{1}\right)\tilde{P}_{2}\left(z_{2}\right)\tilde{P}_{4}\left(z_{4}
\right)\right),z_{4}\right)
F_{1}\left(z_{1},z_{2};\tau_{3}\right),
\end{array}
\end{eqnarray}

So we have the first theorem

\begin{Teo}
Suppose  $\tilde{\mu}=\tilde{\mu}_{1}+\tilde{\mu}_{2}<1$, $\hat{\mu}=\tilde{\mu}_{3}+\tilde{\mu}_{4}<1$, then the number of users en the queues conforming the network of cyclic polling system, (\ref{General.System.Double.Transfer}), when the server visit a queue can be found solving the linear system given by equations (\ref{Ec.Primer.Orden.General.Impar}) and (\ref{Ec.Primer.Orden.General.Par}),

\begin{eqnarray}\label{Ec.Primer.Orden.General.Impar}
\begin{array}{l}
f_{j}\left(i\right)=r_{j+1}\tilde{\mu}_{i}
+\indora_{i\neq j+1}f_{j+1}\left(j+1\right)\frac{\tilde{\mu}_{i}}{1-\tilde{\mu}_{j+1}}
+\indora_{i=j}f_{j+1}\left(i\right)
+\indora_{j=1}\indora_{i\geq3}F_{i,j+1}^{(1)}
+\indora_{j=3}\indora_{i\leq2}F_{i,j+1}^{(1)}
\end{array}
\end{eqnarray}
$j=1,3$ and $i=1,2,3,4$

\begin{eqnarray}\label{Ec.Primer.Orden.General.Par}
\begin{array}{l}
f_{j}\left(i\right)=r_{j-1}\tilde{\mu}_{i}
+\indora_{i\neq j-1}f_{j-1}\left(j-1\right)\frac{\tilde{\mu}_{i}}{1-\tilde{\mu}_{j-1}}
+\indora_{i=j}f_{j-1}\left(i\right)
+\indora_{j=2}\indora_{i\geq3}F_{i,j-1}^{(1)}
+\indora_{j=4}\indora_{i\leq2}F_{i,j-1}^{(1)}
\end{array}
\end{eqnarray}
$j=2,4$ and $i=1,2,3,4$.


whose solutions are:
%{\footnotesize{

\begin{eqnarray}
\begin{array}{lll}
f_{2}\left(1\right)=r_{1}\tilde{\mu}_{1},&
f_{1}\left(2\right)=r_{2}\tilde{\mu}_{2},&
f_{3}\left(4\right)=r_{4}\tilde{\mu}_{4},\\
f_{4}\left(3\right)=r_{3}\tilde{\mu}_{3},&
f_{1}\left(1\right)=r\frac{\tilde{\mu}_{1}\left(1-\tilde{\mu}_{1}\right)}{1-\tilde{\mu}},&
f_{2}\left(2\right)=r\frac{\tilde{\mu}_{2}\left(1-\tilde{\mu}_{2}\right)}{1-\tilde{\mu}},\\
f_{1}\left(3\right)=\tilde{\mu}_{3}\left(r_{2}+\frac{r\tilde{\mu}_{2}}{1-\tilde{\mu}}\right)+F_{3,2}^{(1)}\left(1\right),&
f_{1}\left(4\right)=\tilde{\mu}_{4}\left(r_{2}+\frac{r\tilde{\mu}_{2}}{1-\tilde{\mu}}\right)+F_{4,2}^{(1)}\left(1\right),&
f_{2}\left(3\right)=\tilde{\mu}_{3}\left(r_{1}+\frac{r\tilde{\mu}_{1}}{1-\tilde{\mu}}\right)+F_{3,1}^{(1)}\left(1\right),\\
f_{2}\left(4\right)=\tilde{\mu}_{4}\left(r_{1}+\frac{r\tilde{\mu}_{1}}{1-\tilde{\mu}}\right)+F_{4,1}^{(1)}\left(1\right),&
f_{3}\left(1\right)=\tilde{\mu}_{1}\left(r_{4}+\frac{\hat{r}\tilde{\mu}_{4}}{1-\hat{\mu}}\right)+F_{1,4}^{(1)}\left(1\right),&
f_{3}\left(2\right)=\tilde{\mu}_{2}\left(r_{4}+\frac{\hat{r}\tilde{\mu}_{4}}{1-\hat{\mu}}\right)+F_{2,4}^{(1)}\left(1\right),\\
f_{3}\left(3\right)=\hat{r}\frac{\tilde{\mu}_{3}\left(1-\tilde{\mu}_{3}\right)}{1-\hat{\mu}},&
f_{4}\left(1\right)=\tilde{\mu}_{1}\left(r_{3}+\frac{\hat{r}\tilde{\mu}_{3}}{1-\hat{\mu}}\right)+F_{1,3}^{(1)}\left(1\right),&
f_{4}\left(2\right)=\tilde{\mu}_{2}\left(r_{3}+\frac{\hat{r}\tilde{\mu}_{3}}{1-\hat{\mu}}\right)+F_{2,3}^{(1)}\left(1\right),\\
&f_{4}\left(4\right)=\hat{r}\frac{\tilde{\mu}_{4}\left(1-\tilde{\mu}_{4}\right)}{1-\hat{\mu}}&
\end{array}
\end{eqnarray}
\end{Teo}
%______________________________________________________________________

\begin{Teo}
For the system given by \ref{General.System.Double.Transfer} we have that the second moments are in their general form
{\small{
\begin{eqnarray}\label{Eq.Sdo.Orden.Exh}
\begin{array}{l}
f_{1}\left(i,j\right)=\indora_{i=1}f_{2}\left(1,1\right)
+\left[\left(1-\indora_{i=j=3}\right)\indora_{i+j\leq6}\indora_{i\leq j}\frac{\mu_{j}}{1-\tilde{\mu}_{2}}
+\left(1-\indora_{i=j=3}\right)\indora_{i+j\leq6}\indora_{i>j}\frac{\mu_{i}}{1-\tilde{\mu}_{2}}
+\indora_{i=1}\frac{\mu_{i}}{1-\tilde{\mu}_{2}}\right]f_{2}\left(1,2\right)\\
+
\indora_{i,j\neq2}\left(\frac{1}{1-\tilde{\mu}_{2}}\right)^{2}\mu_{i}\mu_{j}f_{2}\left(2,2\right)
+\left[\indora_{i,j\neq2}\tilde{\theta}_{2}^{(2)}\tilde{\mu}_{i}\tilde{\mu}_{j}
+\indora_{i,j\neq2}\indora_{i=j}\frac{\tilde{P}_{i}^{(2)}}{1-\tilde{\mu}_{2}}
+\indora_{i,j\neq2}\indora_{i\neq j}\frac{\tilde{\mu}_{i}\tilde{\mu}_{j}}{1-\tilde{\mu}_{2}}\right]f_{2}\left(2\right)\\
+\left[r_{2}\tilde{\mu}_{i}
+\indora_{i\geq3}F_{i,2}^{(1)}\right]f_{2}\left(j\right)
+\left[r_{2}\tilde{\mu}_{j}
+\indora_{j\geq3}F_{j,2}^{(1)}\right]f_{2}\left(i\right)
+\left[R_{2}^{(2)}
+\indora_{i=j}r_{2}\right]\tilde{\mu}_{i}\mu_{j}\\
+\indora_{j\geq3}F_{j,2}^{(1)}\left[\indora_{j\neq i}F_{i,2}^{(1)}
+r_{2}\tilde{\mu}_{i}\right]
+r_{2}\left[\indora_{i=j}P_{i}^{(2)}
+\indora_{i\geq3}F_{i,2}^{(1)}\tilde{\mu}_{j}\right]
+\indora_{i\geq3}\indora_{j=i}F_{i,2}^{(2)}\\
f_{2}\left(i,j\right)=
\indora_{i,j\neq1}\left(\frac{1}{1-\tilde{\mu}_{1}}\right)^{2}\tilde{\mu}_{i}\tilde{\mu}_{j}f_{1}\left(1,1\right)
+\left[\left(1-\indora_{i=j=3}\right)\indora_{i+j\leq6}\indora_{i\leq j}\frac{\tilde{\mu}_{j}}{1-\tilde{\mu}_{1}}
+\left(1-\indora_{i=j=3}\right)\indora_{i+j\leq6}\indora_{i>j}\frac{\tilde{\mu}_{i}}{1-\tilde{\mu}_{1}}\right.
\\
+\left.\indora_{i=2}\frac{\tilde{\mu}_{i}}{1-\tilde{\mu}_{1}}\right]f_{1}\left(1,2\right)
+\indora_{i=2}f_{1}\left(2,2\right)
+\left[\indora_{i,j\neq1}\tilde{\theta}_{1}^{(2)}\tilde{\mu}_{i}\tilde{\mu}_{j}
+\indora_{i,j\neq1}\indora_{i\neq j}\frac{\tilde{\mu}_{i}\tilde{\mu}_{j}}{1-\tilde{\mu}_{1}}
+\indora_{i,j\neq1}\indora_{i=j}\frac{\tilde{P}_{i}^{(2)}}{1-\tilde{\mu}_{1}}\right]f_{1}\left(1\right)\\
+\left[r_{1}\mu_{i}+\indora_{i\geq3}F_{i,1}^{(1)}\right]f_{1}\left(j\right)
+\left[\indora_{j\geq3}F_{j,1}^{(1)}+r_{1}\mu_{j}\right]f_{1}\left(i\right)
+\left[R_{1}^{(2)}+\indora_{i=j}\right]\tilde{\mu}_{i}\tilde{\mu}_{j}
+\indora_{i\geq3}F_{i,1}^{(1)}\left[r_{1}\mu_{j}
+\indora_{j\neq i}F_{j,1}^{(1)}\right]\\
+r_{1}\left[\indora_{j\geq3}\mu_{i}F_{j,1}^{(1)}
+\indora_{i=j}P_{i}^{(2)}\right]
+\indora_{i\geq3}\indora_{j=i}F_{i,1}^{(2)}\\
f_{3}\left(i,j\right)=
\indora_{i=3}f_{4}\left(3,3\right)
+\left[\left(1-\indora_{i=j=2}\right)\indora_{i+j\geq4}\indora_{i\leq j}\frac{\tilde{\mu}_{i}}{1-\tilde{\mu}_{4}}
+\left(1-\indora_{i=j=2}\right)\indora_{i+j\geq4}\indora_{i>j}\frac{\tilde{\mu}_{j}}{1-\tilde{\mu}_{4}}
+\indora_{i=3}\frac{\tilde{\mu}_{i}}{1-\tilde{\mu}_{4}}\right]f_{4}\left(3,4\right)\\
+\indora_{i,j\neq4}f_{4}\left(4,4\right)\left(\frac{1}{1-\tilde{\mu}_{4}}\right)^{2}\tilde{\mu}_{i}\tilde{\mu}_{j}
+\left[\indora_{i,j\neq4}\tilde{\theta}_{4}^{(2)}\tilde{\mu}_{i}\tilde{\mu}_{j}
+\indora_{i,j\neq4}\indora_{i=j}\frac{\tilde{P}_{i}^{(2)}}{1-\tilde{\mu}_{4}}
+\indora_{i,j\neq4}\indora_{i\neq j}\frac{\tilde{\mu}_{i}\tilde{\mu}_{j}}{1-\tilde{\mu}_{4}}\right]f_{4}\left(4\right)\\
+\left[r_{4}\tilde{\mu}_{i}+\indora_{i\leq2}F_{i,4}^{(1)}\right]f_{4}\left(j\right)
+\left[r_{4}\tilde{\mu}_{j}+\indora_{j\leq2}F_{j,4}^{(1)}\right]f_{4}\left(i\right)
+\left[R_{4}^{(2)}+\indora_{i=j}r_{4}\right]\tilde{\mu}_{i}\tilde{\mu}_{j}\\
+   \indora_{i\leq2}F_{i,4}^{(1)}\left[r_{4}\tilde{\mu}_{j}
+\indora_{j\neq i}F_{j,4}^{(1)}\right]
+r_{4}\left[\indora_{i=j}P_{i}^{(2)}+\indora_{j\leq2}\tilde{\mu}_{i}F_{j,4}^{(1)}\right]
+\indora_{i\leq2}\indora_{j=i}F_{i,4}^{(2)}\\
f_{4}\left(i,j\right)=
\indora_{i,j\neq3}f_{3}\left(3,3\right)\left(\frac{1}{1-\tilde{\mu}_{3}}\right)^{2}\tilde{\mu}_{i}\tilde{\mu}_{j}
+\left[\left(1-\indora_{i=j=2}\right)\indora_{i+j\geq5}\indora_{i\leq j}\frac{\tilde{\mu}_{i}}{1-\tilde{\mu}_{3}}
+\left(1-\indora_{i=j=2}\right)\indora_{i+j\geq5}\indora_{i>j}\frac{\tilde{\mu}_{j}}{1-\tilde{\mu}_{3}}\right.\\
+\left.\indora_{i=4}\frac{\tilde{\mu}_{i}}{1-\tilde{\mu}_{3}}\right]f_{3}\left(3,4\right)
+\indora_{i=4}f_{3}\left(4,4\right)
+\left[\indora_{i,j\neq3}\tilde{\theta}_{3}^{(2)}\tilde{\mu}_{i}\tilde{\mu}_{j}
+\indora_{i,j\neq3}\indora_{i=j}\frac{\tilde{P}_{i}^{(2)}}{1-\tilde{\mu}_{3}}
+\indora_{i,j\neq3}\indora_{i\neq j}\frac{\tilde{\mu}_{i}\tilde{\mu}_{j}}{1-\tilde{\mu}_{3}}\right]f_{3}\left(3\right)\\
+\left[r_{3}\tilde{\mu}_{i}+\indora_{i\leq2}F_{i,3}^{(1)}\right]f_{3}\left(j\right)
+\left[r_{3}\tilde{\mu}_{j}+\indora_{j\leq2}F_{j,3}^{(1)}\right]f_{3}\left(i\right)
+\left[R_{3}^{(2)}+\indora_{i=j}r_{3}\right]\tilde{\mu}_{i}\tilde{\mu}_{j}\\
+\indora_{i\leq2}F_{i,3}^{(1)}\left[r_{3}\tilde{\mu}_{j}+\indora_{j\neq i}F_{j,3}^{(1)}\right]
+r_{3}\left[\indora_{i=j}P_{i}^{(2)}+\indora_{j\leq2}\tilde{\mu}_{i}F_{j,3}^{(1)}\right]
+\indora_{i\leq2}\indora_{j=i}F_{i,3}^{(2)}
\end{array}
\end{eqnarray}}}
\end{Teo}


\begin{Coro}
Conforming the equations given in \ref{Eq.Sdo.Orden.Exh} the second order moments are obtained solving the system


\begin{eqnarray*}\label{System.Second.Order.Moments}
\begin{array}{ll}
f_{1}\left(1,1\right)=a_{1}f_{2}\left(2,2\right)
+a_{2}f_{2}\left(2,1\right)
+a_{3}f_{2}\left(1,1\right)
+K_{1},&
f_{1}\left(1,2\right)=K_{2}\\
f_{1}\left(1,3\right)=a_{4}f_{2}\left(2,2\right)+a_{5}f\left(2,1\right)+K_{3},&
f_{1}\left(1,4\right)=a_{6}f_{2}\left(2,2\right)+a_{7}f_{2}\left(2,1\right)+K_{4}\\
f_{1}\left(2,2\right)=K_{5},&
f_{1}\left(2,3\right)=K_{6}\\
f_{1}\left(2,4\right)=K_{7},&
f_{1}\left(3,3\right)=a_{8}f_{2}\left(2,2\right)+K_{8}\\
f_{1}\left(3,4\right)=a_{9}f_{2}\left(2,2\right)+K_{9},&
f_{1}\left(4,4\right)=a_{10}f_{2}\left(2,2\right)+K_{10}\\
f_{2}\left(1,1\right)=K_{11},&
f_{2}\left(1,2\right)=K_{12}\\
f_{2}\left(1,3\right)=K_{13},&
f_{2}\left(1,4\right)=K_{14}\\
f_{2}\left(2,2\right)=a_{11}f_{1}\left(1,1\right)
+a_{12}f_{1}\left(1,2\right)+a_{13}f_{1}\left(2,2\right)+K_{15},&
f_{2}\left(2,3\right)=a_{14}f_{1}\left(1,1\right)+a_{15}f_{1}\left(1,2\right)+K_{16}\\
f_{2}\left(2,4\right)=a_{16}f_{1}\left(1,1\right)+a_{17}f_{1}\left(1,2\right)+K_{17},&
f_{2}\left(3,3\right)=a_{18}f_{1}\left(1,1\right)+K_{18}\\
f_{2}\left(3,4\right)=a_{19}f_{1}\left(1,1\right)+K_{19},&
f_{2}\left(4,4\right)=a_{20}f_{1}\left(1,1\right)+K_{20}\\
f_{3}\left(1,1\right)=a_{21}f_{4} \left(4,4\right)+K_{21},&
f_{3}\left(1,2\right)=a_{22}f_{4}\left(4,4\right)+K_{22}\\
f_{3}\left(1,3\right)=a_{23}f_{4}\left(4,4\right)+a_{24}f_{4}\left(4,3\right)+K_{23},&
f_{3}\left(1,4\right)=K_{24}\\
f_{3}\left(2,2\right)=a_{25}f_{4}\left(4,4\right)+K_{25},&
f_{3}\left(2,3\right)=a_{26}f_{4}\left(4,4\right)+a_{27}f_{4}\left(4,3\right)+K_{26}\\
f_{3}\left(2,4\right)=K_{27},&
f_{3}\left(3,3\right)=a_{28}f_{4}\left(4,4\right)+a_{29}f_{4}\left(4,3\right)+a_{30}f_{4}\left(3,3\right)+K_{28}\\
f_{3}\left(3,4\right)=K_{29},&
f_{3}\left(4,4\right)=K_{30}\\
f_{4}\left(1,1\right)=a_{31}f_{3}\left(3,3\right)+K_{31},&
f_{4}\left(1,2\right)=a_{32}f_{3}\left(3,3\right)+K_{32}\\
F_{4}\left(1,3\right)=K_{33},&
f_{4}\left(1,4\right)=a_{33}f_{3}\left(3,3\right)+a_{34}f_{3}\left(3,4\right)+K_{34}\\
f_{4}\left(2,2\right)=a_{35}f_{3}\left(3,3\right)+K_{35},&
f_{4}\left(2,3\right)=K_{36}\\
f_{4}\left(2,4\right)=a_{36}f_{3}\left(3,3\right)+a_{37}f_{3}\left(3,4\right)+K_{37},&
f_{4}\left(3,3\right)=K_{38}\\
f_{4}\left(3,4\right)=K_{39},&
f_{4}\left(4,4\right)=a_{38}f_{3}\left(3,3\right)+a_{39}f_{3}\left(3,4\right)+a_{40}f_{3}\left(4,4\right)+K_{40}
\end{array}
\end{eqnarray*}



The system solutions are given by


\begin{eqnarray*}
\begin{array}{lll}
f_{1}\left(1,1\right)=b_{3},&
f_{2}\left(2,2\right)=\frac{b_{2}}{1-b_{1}},&
f_{1}\left(1,3\right)=a_{4}\left(\frac{b_{2}}{1-b_{1}}\right)+a_{5}K_{12}+K_{3},\\
f_{1}\left(1,4\right)=a_{6}\left(\frac{b_{2}}{1-b_{1}}\right)+a_{7}K_{12}+K_{4},&
f_{1}\left(3,3\right)=a_{8}\left(\frac{b_{2}}{1-b_{1}}\right)+K_{8},&
f_{1}\left(3,4\right)=a_{9}\left(\frac{b_{2}}{1-b_{1}}\right)+K_{9}\\
f_{1}\left(4,4\right)=a_{10}\left(\frac{b_{2}}{1-b_{1}}\right)+a_{5}K_{12}+K_{10},&
f_{2}\left(2,3\right)=a_{14}b_{3}+a_{15}K_{2}+K_{16},&
f_{2}\left(2,4\right)=a_{16}b_{3}+a_{17}K_{2}+K_{17},\\
f_{2}\left(3,3\right)=a_{18}b_{3}+K_{18},&
f_{2}\left(3,4\right)=a_{19}b_{3}+K_{19},&
f_{2}\left(4,4\right)=a_{20}b_{3}+K_{20}\\
f_{3}\left(3,3\right)=\frac{b_{5}}{1-b_{4}},&
f_{4}\left(2,2\right)=b_{6},&
f_{3}\left(1,1\right)=a_{21}b_{6}+K_{21},\\
f_{3}\left(1,2\right)=a_{22}b_{6}+K_{22},&
f_{3}\left(1,3\right)=a_{23}b_{6}+a_{24}K_{39}+K_{23},&
f_{3}\left(2,2\right)=a_{25}b_{6}+K_{25}\\
f_{3}\left(2,3\right)=a_{26}b_{6}+a_{27}K_{39}+K_{26},&
f_{4}\left(1,1\right)=a_{31}\left(\frac{b_{5}}{1-b_{4}}\right)+K_{31},&
f_{4}\left(1,2\right)=a_{32}\left(\frac{b_{5}}{1-b_{4}}\right)+K_{32},\\
f_{4}\left(1,4\right)=a_{33}\left(\frac{b_{5}}{1-b_{4}}\right)+a_{34}K_{29}+K_{31},&
f_{4}\left(2,2\right)=a_{35}\left(\frac{b_{5}}{1-b_{4}}\right)+K_{35},&
f_{4}\left(2,4\right)=a_{36}\left(\frac{b_{5}}{1-b_{4}}\right)+a_{37}K_{29}+K_{37}
\end{array}
\end{eqnarray*}


where
\begin{eqnarray*}
\begin{array}{lll}
N_{1}=a_{2}K_{12}+a_{3}K_{11}+K_{1},&
N_{2}=a_{12}K_{2}+a_{13}K_{5}+K_{15},&
b_{1}=a_{1}a_{11}\\
b_{2}=a_{11}N_{1}+N_{2},&
b_{3}=a_{1}\left(\frac{b_{2}}{1-b_{1}}\right)+N_{1},&
N_{3}=a_{29}K_{39}+a_{30}K_{38}+K_{28}\\
N_{4}=a_{39}K_{29}+a_{40}K_{30}+K_{40},&
b_{4}=a_{28}a_{38},&
b_{5}=a_{28}N_{4}+N_{3}\\
&b_{6}=a_{38}\left(\frac{b_{5}}{1-b_{4}}\right)+N_{4}&
\end{array}
\end{eqnarray*}

\end{Coro}

the values for the $a_{i}$'s and $K_{i}$ can be found in \textit{Appendix B} %(\ref{Secc.Append.B}).




%______________________________________________________________________
\section{Concluding Remarks}
%______________________________________________________________________

Using a similar reasoning it's possible to find de first and second moments for the queue lengths of the CPSN. We have the following theorem

\begin{Teo}
Given a CPSN attended by a single server who attends conforming to the gated policy and suppose  $\tilde{\mu}=\tilde{\mu}_{1}+\tilde{\mu}_{2}<1$, $\hat{\mu}=\tilde{\mu}_{3}+\tilde{\mu}_{4}<1$, then the number of users en the queues conforming the network of cyclic polling system, when the server visit a queue can be found solving the linear system given by equations (\ref{Ec.Primer.Orden.General.Impar.Gated}) and (\ref{Ec.Primer.Orden.General.Par.Gated}),

\begin{eqnarray}\label{Ec.Primer.Orden.General.Impar.Gated}
\begin{array}{l}
f_{j}\left(i\right)=r_{j+1}\tilde{\mu}_{i}
+f_{j+1}\left(j+1\right)\tilde{\mu}_{i}
+\indora_{i=j}f_{j+1}\left(i\right)
+\indora_{j=1}\indora_{i\geq3}F_{i,j+1}^{(1)}
+\indora_{j=3}\indora_{i\leq2}F_{i,j+1}^{(1)}
\end{array}
\end{eqnarray}
$j=1,3$ and $i=1,2,3,4$

\begin{eqnarray}\label{Ec.Primer.Orden.General.Par.Gated}
\begin{array}{l}
f_{j}\left(i\right)=r_{j-1}\tilde{\mu}_{i}
+f_{j-1}\left(j-1\right)\tilde{\mu}_{i}
+\indora_{i=j}f_{j-1}\left(i\right)
+\indora_{j=2}\indora_{i\geq3}F_{i,j-1}^{(1)}
+\indora_{j=4}\indora_{i\leq2}F_{i,j-1}^{(1)}
\end{array}
\end{eqnarray}
$j=2,4$ and $i=1,2,3,4$.


whose solutions are:

\begin{eqnarray}\label{Sol.Sist.Ec.Lineales.Gated}
\begin{array}{lll}
f_{1}\left(1\right)=\frac{r\tilde{\mu}_{1}}{1-\tilde{\mu}},&
f_{1}\left(2\right)=\tilde{\mu}_{2}\frac{r_{2}\left(1-\tilde{\mu}_{1}\right)+r_{1}\tilde{\mu}_{2}}{1-\tilde{\mu}},&
f_{1}\left(3\right)=\tilde{\mu}_{3}\left[r_{2}+\frac{r\tilde{\mu}_{2}}{1-\tilde{\mu}}\right]+F_{3,2}^{(1)},\\
f_{1}\left(4\right)=\tilde{\mu}_{4}\left[r_{2}+\frac{r\tilde{\mu}_{2}}{1-\tilde{\mu}}\right]+F_{4,2}^{(1)},&
f_{2}\left(1\right)=\tilde{\mu}_{1}\frac{r_{1}\left(1-\tilde{\mu}_{2}\right)+r_{2}\tilde{\mu}_{1}}{1-\tilde{\mu}},&
f_{2}\left(2\right)=r\frac{\tilde{\mu}_{2}}{1-\tilde{\mu}},\\
f_{2}\left(3\right)=\tilde{\mu}_{3}\frac{r_{1}\left(1-\tilde{\mu}_{2}\right)+r_{2}\tilde{\mu}_{1}}{1-\tilde{\mu}}+F_{3,1}^{(1)},&
f_{2}\left(4\right)=\tilde{\mu}_{4}\frac{r_{1}\left(1-\tilde{\mu}_{2}\right)+r_{2}\tilde{\mu}_{1}}{1-\tilde{\mu}}+F_{4,1}^{(1)},&
f_{3}\left(1\right)=\tilde{\mu}_{1}\left[r_{4}+\frac{\hat{r}\mu_{4}}{1-\hat{\mu}}\right]+F_{1,4}^{(1)},\\
f_{3}\left(2\right)=\tilde{\mu}_{2}\left[r_{4}+\frac{\hat{r}\mu_{4}}{1-\hat{\mu}}\right]+F_{2,4}^{(1)},&
f_{3}\left(3\right)=\frac{\hat{r}\tilde{\mu}_{3}}{1-\hat{\mu}},&
f_{3}\left(4\right)=\tilde{\mu}_{4}\frac{r_{4}\left(1-\tilde{\mu}_{3}\right)+r_{3}\tilde{\mu}_{4}}{1-\hat{\mu}},\\
f_{4}\left(1\right)=\tilde{\mu}_{1}\frac{r_{3}\left(1-\tilde{\mu}_{4}\right)+r_{4}\tilde{\mu}_{3}}{1-\hat{\mu}}+F_{1,3}^{(1)},&
f_{4}\left(2\right)=\tilde{\mu}_{2}\frac{r_{3}\left(1-\tilde{\mu}_{4}\right)+r_{4}\tilde{\mu}_{3}}{1-\hat{\mu}}+F_{2,3}^{(1)},&
f_{4}\left(3\right)=\tilde{\mu}_{3}\frac{r_{3}\left(1-\tilde{\mu}_{4}\right)+r_{4}\tilde{\mu}_{3}}{1-\hat{\mu}},\\
&f_{4}\left(4\right)=\frac{\hat{r}\tilde{\mu}_{4}}{1-\hat{\mu}}.&
\end{array}
\end{eqnarray}
\end{Teo}

for the second moments

\begin{Teo}
Given a CPSN attended by a single server who attends conforming to the gated policy and suppose  $\tilde{\mu}=\tilde{\mu}_{1}+\tilde{\mu}_{2}<1$, $\hat{\mu}=\tilde{\mu}_{3}+\tilde{\mu}_{4}<1$, we have that the second moments are in their general form

{\small{
\begin{eqnarray}\label{Eq.Sdo.Orden.Gated}
\begin{array}{l}
f_{1}\left(i,k\right)=
\indora_{k=1}\indora_{i=k}\tilde{\mu}_{i}f_{2}\left(1,1\right)
+\left[\indora_{k=1}\tilde{\mu}_{1}+\indora_{i=1}\tilde{\mu}_{k}\right]f_{2}\left(1,2\right)
+\tilde{\mu}_{i}\tilde{\mu}_{k}f_{2}\left(2,2\right)
+\left[\indora_{i=k}\tilde{P}_{i}^{(2)}
+\indora_{i\neq k}\tilde{\mu}_{i}\tilde{\mu}_{k}\right]f_{2}\left(2\right)\\
+\left[r_{2}\tilde{\mu}_{i}+\indora_{i\geq3}F_{i,2}^{(1)}\right]f_{2}\left(k\right)
+\left[r_{2}\tilde{\mu}_{k}+\indora_{k\geq3}F_{k,2}^{(1)}\right]f_{2}\left(i\right)
+\left[R_{2}^{(2)}+\indora_{i=k}r_{2}\right]\tilde{\mu}_{i}\tilde{\mu}_{k}\\
+\left[\indora_{k\geq3}\tilde{\mu}_{i}F_{k,2}^{(1)}+\indora_{i=k}P_{i}^{(2)}\right]r_{2}
+\left[\indora_{i\geq3}\indora_{k\neq i}F_{k,2}^{(1)}+\indora_{i\geq3}r_{2}\tilde{\mu}_{k}\right]F_{i,2}^{(1)}
+\indora_{i\geq3}\indora_{k=i}F_{i,2}^{(2)}\\
f_{2}\left(i,k\right)=\tilde{\mu}_{i}\tilde{\mu}_{k}f_{1}\left(1,1\right)
+\left[\indora_{k=2}\tilde{\mu}_{i}
+\indora_{i=2}\tilde{\mu}_{k}\right]f_{1}\left(1,2\right)
+\indora_{k=2}\indora_{i=k}\tilde{\mu}_{i}f_{1}\left(2,2\right)
+\left[\indora_{i=k}\tilde{P}_{i}^{(2)}
+\indora_{i\neq k}\tilde{\mu}_{i}\tilde{\mu}_{k}\right]f_{1}\left(1\right)\\
+\left[r_{1}\tilde{\mu}_{i}+\indora_{i\geq3}F_{i,1}^{(1)}\right]f_{1}\left(k\right)
+\left[r_{1}\tilde{\mu}_{k}+\indora_{k\geq3}F_{k,1}^{(1)}\right]f_{1}\left(i\right)
+\left[R_{1}^{(2)}+\indora_{i=k}r_{1}\right]\tilde{\mu}_{i}\tilde{\mu}_{k}\\
+\left[\indora_{i\geq3}\indora_{k\neq i}F_{i,1}^{(1)}+\indora_{k\geq3}r_{1}\tilde{\mu}_{i}\right]F_{k,1}^{(1)}
+\left[\indora_{i=k}P_{i}^{(2)}+\indora_{i\geq3}F_{i,1}^{(1)}\tilde{\mu}_{k}\right]r_{1}
+\indora_{i\geq3}\indora_{k=i}F_{i,1}^{(2)}\\
f_{3}\left(i,k\right)=\indora_{k=3}\indora_{i=k}\tilde{\mu}_{i}f_{4}\left(3,3\right)
+\left[\indora_{k=3}\tilde{\mu}_{i}+\indora_{i=3}\tilde{\mu}_{k}\right]f_{4}\left(3,4\right)
+\tilde{\mu}_{i}\tilde{\mu}_{k}f_{4}\left(4,4\right)
+\left[\indora_{i=k}\tilde{P}_{i}^{(2)}+\indora_{i\neq k}\tilde{\mu}_{i}\tilde{\mu}_{k}\right]f_{4}\left(4\right)\\
+\left[r_{4}\tilde{\mu}_{i}+\indora_{i\leq2}F_{i,4}^{(1)}\right]f_{4}\left(k\right)
+\left[r_{4}\tilde{\mu}_{k}+\indora_{k\leq2}F_{k,4}^{(1)}\right]f_{4}\left(i\right)
+\left[R_{4}^{(2)}+\indora_{i=k}r_{4}\right]\tilde{\mu}_{i}\tilde{\mu}_{k}\\
+\left[\indora_{i=k}P_{i}^{(2)}+\indora_{k\leq2}\tilde{\mu}_{i}F_{k,4}^{(1)}\right]r_{4}
+\left[\indora_{i\leq2}\indora_{k\neq i}F_{k,4}^{(1)}+\indora_{i\leq2}r_{4}\tilde{\mu}_{k}\right]F_{i,4}^{(1)}
+\indora_{i\leq2}\indora_{k=i}F_{i,4}^{(2)}\\
f_{4}\left(i,k\right)=\tilde{\mu}_{i}\tilde{\mu}_{k}f_{3}\left(3,3\right)
+\left[\indora_{k=4}\tilde{\mu}_{i}+\indora_{i=4}\tilde{\mu}_{k}\right]f_{3}\left(3,4\right)
+\indora_{k=4}\indora_{i=k}\tilde{\mu}_{i}f_{3}\left(4,4\right)
+\left[\indora_{i=k}\tilde{P}_{i}^{(2)}
+\indora_{i\neq k}\tilde{\mu}_{i}\tilde{\mu}_{k}\right]f_{3}\left(3\right)\\
+\left[r_{3}\tilde{\mu}_{i}+\indora_{i\leq2}F_{i,3}^{(1)}\right]f_{3}\left(k\right)
+\left[r_{3}\tilde{\mu}_{k}+\indora_{k\leq2}F_{k,3}^{(1)}\right]f_{3}\left(i\right)
+\left[R_{3}^{(2)}+\indora_{i=k}r_{3}\right]\tilde{\mu}_{i}\tilde{\mu}_{k}\\
+\left[\indora_{i\leq2}\indora_{k\neq i}F_{k,3}^{(1)}+\indora_{i\leq2}r_{3}\tilde{\mu}_{k}\right]F_{i,3}^{(1)}
+\left[\indora_{k\leq2}\tilde{\mu}_{i}F_{k,3}^{(1)}+\indora_{i=k}P_{i}^{(2)}\right]r_{3}
+\indora_{i\leq2}\indora_{k=i}F_{i,3}^{(2)}
\end{array}
\end{eqnarray}}}
\end{Teo}

\begin{Coro}
Conforming the equations given in \ref{Eq.Sdo.Orden.Gated} the second order moments are obtained solving the system
\end{Coro}







%______________________________________________________________________
\section{General Case Calculations Exhaustive Policy}\label{Secc.Append.B}
%______________________________________________________________________

%_______________________________________________________________
%\subsection{Calculations}
%_______________________________________________________________


Remember the equations given in equations (\ref{Ec.Gral.Primer.Momento.Ind.Exh}) and (\ref{Ec.Gral.Segundo.Momento.Ind.Exh}) which can be written in particular cases like



\newpage




%______________________________________________________________________
\section{Descripci\'on}
%______________________________________________________________________


%___________________________________________________________________
Let's define the
probability of the event no ruin before the $n$-th period begining with $\tilde{L}_{0}$ users, $g_{n,k}$ considering a capital equal to $k$ units after $n-1$ events, i.e.,  given $n\in\left\{1,2,\ldots\right\}$ y $k\in\left\{0,1,2,\ldots\right\}$ $g_{n,k}:=P\left\{\tilde{L}_{j}>0, j=1,\ldots,n,\tilde{L}_{n}=k\right\}$, which can be written as:



\begin{Assumption}
\label{A:PD}

\begin{enumerate}
\item[a)] $c$ is lower semicontinuous, and inf-compact on $\mathbb{K}$ (i.e.
for every $x\in X$ and $r\in \mathbb{R}$ the set $\{a \in A(x):c(x,a) \leq  r
\}$ is compact).

\item[b)] The transition law $Q$ is strongly continuous, i.e. $u(x,a)=\int
u(y)Q(dy|x,a)$, $(x,a)\in\mathbb{K}$ is continuous and bounded on $\mathbb{K}$, for every
measurable bounded function $u$ on $X$.

\item[c)] There exists a policy $\pi$ such that $V(\pi,x)<\infty$, for each $%
x \in X$.
\end{enumerate}
\end{Assumption}

\begin{Remark}
\label{R:BT}

The following consequences of Assumption \ref{A:PD} are well-known (see
Theorem 4.2.3 and Lemma 4.2.8 in \cite{Hernandez}):

\begin{enumerate}
\item[a)] The optimal value function $V^{\ast}$ is the solution of the
\textit{Optimality Equation} (OE), i.e. for all $x \in X$,
\begin{equation*}
V^{\ast}(x)=\underset{a\in A(x)}{\min }\left\{ c(x,a)+\alpha \int
V^{\ast}(y)Q(dy|x,a)\right\} \text{.}
\end{equation*}

There is also $f^{\ast}\in \mathbb{F}$ such that:
\begin{equation}
V^{\ast}(x)= c(x,f^{\ast}(x))+\alpha \int V^{\ast}(y)Q(dy|x,f^{\ast}(x)), \label{2.1}
\end{equation}
$ x\in X$, and $f^{\ast}$ is optimal.

\item[b)] For every $x \in X$, $v_{n}(x)\uparrow V^{\ast}$, with $v_{n}$
defined as
\begin{equation*}
v_{n}(x)=\underset{a\in A(x)}{\min }\left\{ c(x,a)+\alpha \int
v_{n-1}(y)Q(dy| x,a)\right\},
\end{equation*}
 $x\in X, n=1,2,\cdots $, and $v_{0}(x)=0$. Moreover, for each $n$, there is $%
f_{n}\in \mathbb{F}$ such that, for each $x\in X$,
\begin{equation}
\underset{a\in A(x)}{\min }\left\{ c(x,a)+\alpha \int
v_{n-1}(y)Q(dy|x,a)\right\}= c(x,f_{n}(x))+\alpha \int
v_{n-1}(y)Q(dy|x,f_{n}(x)).  \label{2.2}
\end{equation}
\end{enumerate}
\end{Remark}

Let $(X,A,\{A(x):x\in X\},Q,c)$ be a fixed Markov control model. Take $M$ as the MDP with the Markov control model $(X,A,\{A(x):x\in
X\},Q,c)$. The optimal value function, the optimal policy which comes from (%
\ref{2.1}), and the minimizers in (\ref{2.2}) will be denoted for $M$ by $%
V^{\ast}$, $f^{\ast}$, and $f_{n}$ , $n=1,2,\cdots $, respectively. Also let
$v_{n}$, $n=1,2,\cdots $, be the value iteration functions for $M$. Let $%
G(x,a):=c(x,a)+\alpha \int V^{\ast}(y)Q(dy|x,a)$, $(x,a)\in \mathbb{K}$.

It will be also supposed that the MDPs taken into account satisfy one of the
following Assumptions \ref{A:2} or \ref{A:3}.

\begin{Assumption}
\label{A:2}

\begin{enumerate}
\item[a)] $X$ and $A$ are convex;

\item[b)] $(1- \lambda)a+a^{\prime }\in A((1- \lambda)x+x^{\prime })$ for
all $x$, $x^{\prime }\in X$, $a\in A(x)$, $a^{\prime }\in A(x^{\prime })$
and $\lambda \in [0,1]$. Besides it is assumed that: if $x$ and $y\in X$, $x <
y $, then $A(y)\subseteq A(x)$, and $A(x)$ are convex for each $x \in X$;

\item[c)] $Q$ is induced by a difference equation $x_{t+1}=F(x_{t},a_{t},%
\xi_{t})$, with $t=0,1,\cdots $, where $F:X\times A\times S \rightarrow X$
is a measurable function and $\{\xi_{t}\}$ is a sequence of independent and
identically distributed (i.i.d.) random variables with values in $S \subseteq
\mathbb{R}$, and with a common density $\Delta$. In addition, we suppose
that $F(\cdot,\cdot,s)$ is a convex function on $\mathbb{K}$, for each $s\in
S$; and if $x$ and $y\in X$, $x < y$, then $F(x,a,s)\leq F (y,a,s)$ for each
$a\in A(y)$ and $s\in S$;

\item[d)] $c$ is convex on $\mathbb{K}$, and if  $x$ and $y\in X$, $x < y$,
then $c(x,a)\leq c(y,a)$, for each $a\in A(y)$.
\end{enumerate}
\end{Assumption}

\begin{Assumption}
\label{A:3}

\begin{enumerate}
\item[a)] Same as Assumption \ref{A:2} (a);

\item[b)] $(1- \lambda)a+a^{\prime }\in A((1- \lambda)x+x^{\prime })$ for
all $x$, $x^{\prime }\in X$, $a\in A(x)$, $a^{\prime }\in A(x^{\prime })$
and $\lambda\in [0,1]$. Besides $A(x)$ is assumed to be convex for each $x
\in X$;

\item[c)] $Q$ is given by the relation $x_{t+1}=\gamma x_{t}+\delta
a_{t}+\xi_{t}$, $t=0,1,\cdots $, where $\{\xi_{t}\}$ are i.i.d. random
variables taking values in $S\subseteq \mathbb{R}$ with the density $\Delta$%
, $\gamma$ and $\delta$ are real numbers;

\item[d)] $c$ is convex on $\mathbb{K}$.
\end{enumerate}
\end{Assumption}

\begin{Remark}
\label{R:2} Assumptions \ref{A:2} and \ref{A:3} are essentially presented in
Conditions C1 and C2 in \cite{DRS}, but changing a strictly convex $c(\cdot,
\cdot)$ by a convex $c(\cdot, \cdot)$. (In fact, in \cite{DRS}, Conditions C1
and C2 take into account the more general situation in which both $X$ and $A$
are subsets of Euclidean spaces of the dimension greater than one.)
Also note that it is possible to obtain that each of Assumptions \ref{A:2}
and \ref{A:3} implies that, for each $x\in X$, $G(x,\cdot)$ is convex but
not necessarily strictly convex (hence, $M$ does not necessarily have a
unique optimal policy). The proof of this fact is a direct consequence of
the convexity of the cost function $c$, and of the proof of Lemma 6.2 in
\cite{DRS}.
\end{Remark}





\begin{Assumption}
\label{A:4} There is a policy $\phi$ such that $E_{x}^{\phi }\left[ \text{$\sum\limits_{t=0}^{\infty }$}\alpha
^{t}c^*(x_{t},a_{t})\right] \text{}<\infty$%
, for each $x\in X$.
\end{Assumption}

\begin{Remark}
\label{R:3} Suppose that, for M, Assumption 2.1 holds. Then, it is direct to verify that if $M_{\epsilon}$ satisfies Assumption \ref{A:4}, then it also
satisfies Assumption \ref{A:PD}.
\end{Remark}

\begin{Condition}
\label{C:1} There exists a measurable function $Z:X\rightarrow \mathbb{R}$,
which may depend on $\alpha$, such that $c^{%
\ast}(x,a)-c(x,a)=\epsilon a^{2}\leq\epsilon Z(x)$, and $\int
Z(y)Q(dy|x,a)\leq Z(x)$ for each $x\in X$ and $a\in B(x)$.
\end{Condition}

\begin{Theorem}
\label{T:1} Suppose that Assumptions \ref{A:PD} and \ref{A:4} hold, and
that, for $M$, one of Assumptions \ref{A:2} or \ref{A:3} holds. Let $%
\epsilon $ be a positive number. Then,

\begin{enumerate}
\item[a)] If $A$ is compact, $|W^{\ast}(x)-V^{\ast}(x)|\leq \epsilon K^{2}/(1-\alpha)$%
, $x\in X$, where $K$ is the diameter of a compact set $D$ such that $0\in D$
and $A\subseteq D$.

\item[b)] Under Condition \ref{C:1}, $|W^{\ast}(x) - V^{\ast}(x)|\leq
\epsilon Z(x)/(1- \alpha)$, $x\in X$.
\end{enumerate}
\end{Theorem}

\begin{proof}
The proof of case (a) follows from the proof of case (b) given that $Z(x)=K^{2}$, $x\in X$. (Observe that in this case, if $a\in A$,
then $a^{2}=(a-0)^{2} \leq K^{2}$.)

\textbf{(b)} Assume that $M$ satisfies Assumption \ref{A:2}. (The proof for
the case in which $M$ satisfies Assumption \ref{A:3} is similar.)

\end{proof}

The following Corollary  is immediate.

\begin{Corollary}\label{Co:1}
Suppose that Assumptions \ref{A:PD} and \ref{A:4} hold. Suppose
that for $M$ one of Assumptions \ref{A:2} or \ref{A:3} holds (hence $M$
does not necessarily have a unique optimal policy). Let $\epsilon $ be a
positive number. If $A$ is compact or Condition \ref{C:1} holds, then there
exists an MDP $M_{\epsilon }$ with a unique optimal policy $g^{\ast }$, such
that inequalities in Theorem 3.7 (a) or (b) hold, respectively.
\end{Corollary}

\begin{Example}\label{E:1}
Ejemplo1
\end{Example}

\begin{Lemma}\label{L:1}
Lema1
\end{Lemma}

\begin{proof}
Assumption \ref{A:PD} (a) trivially holds. The proof of the strong continuity of $Q$

\end{proof}





\section{Descripci\'on de una Red de Sistemas de Visitas C\'iclicas}

Consideremos una red de sistema de visitas c\'iclicas conformada por dos sistemas de visitas c\'iclicas, cada una con dos colas independientes, donde adem\'as se permite el intercambio de usuarios entre los dos sistemas en la segunda cola de cada uno de ellos.

%____________________________________________________________________
\subsection*{Supuestos sobre la Red de Sistemas de Visitas C\'iclicas}
%____________________________________________________________________

\begin{itemize}
\item Los arribos de los usuarios ocurren conforme a un proceso de conteo general con tasa de llegada $\mu_{1}$ y $\mu_{2}$ para el sistema 1, mientras que para el sistema 2, lo hacen conforme a un proceso Poisson con tasa $\hat{\mu}_{1},\hat{\mu}_{2}$ respectivamente.



\item Se considerar\'an intervalos de tiempo de la forma
$\left[t,t+1\right]$. Los usuarios arriban de manera independiente del resto de las colas. Se define el grupo de
usuarios que llegan a cada una de las colas del sistema 1,
caracterizadas por $Q_{1}$ y $Q_{2}$ respectivamente, en el
intervalo de tiempo $\left[t,t+1\right]$ por
$X_{1}\left(t\right),X_{2}\left(t\right)$.


\item Se definen los procesos
$\hat{X}_{1}\left(t\right),\hat{X}_{2}\left(t\right)$ para las
colas del sistema 2, denotadas por $\hat{Q}_{1}$ y $\hat{Q}_{2}$
respectivamente. Donde adem\'as se supone que $\mu_{i}<1$ y $\hat{\mu}_{i}<1$ para $i=1,2$.


\item Se define el proceso $Y_{2}\left(t\right)$ para el n\'umero de usuarios que se trasladan del sistema 2 al sistema 1 en el intervalo de tiempo $\left[t,t+1\right]$, este proceso tiene par\'ametro $\check{\mu}_{2}$.% El traslado de un sistema a otro ocurre de manera tal que el proceso de llegadas a $Q_{2}$ es un proceso Poisson con par\'ametro $\tilde{\mu}_{2}=\mu_{2}+\check{\mu}_{2}<1$.


\item En lo que respecta al servidor, en t\'erminos de los tiempos de
visita a cada una de las colas, se definen las variables
aleatorias $\tau_{i},$ para $Q_{i}$, para $i=1,2$, respectivamente;
y $\zeta_{i}$ para $\hat{Q}_{i}$,  $i=1,2$,  del sistema
2 respectivamente. A los tiempos en que el servidor termina de atender en las colas $Q_{i},\hat{Q}_{i}$, se les denotar\'a por
$\overline{\tau}_{i},\overline{\zeta}_{i}$ para  $i=1,2$,
respectivamente.

\item Los tiempos de traslado del servidor desde el momento en que termina de atender a una cola y llega a la siguiente para comenzar a dar servicio est\'an dados por
$\tau_{i+1}-\overline{\tau}_{i}$ y
$\zeta_{i+1}-\overline{\zeta}_{i}$,  $i=1,2$, para el sistema 1 y el sistema 2, respectivamente.

\end{itemize}




%\begin{figure}[H]
%\centering
%%%\includegraphics[width=5cm]{RedSistemasVisitasCiclicas.jpg}
%%\end{figure}\label{RSVC}

El uso de la FGP nos permite determinar las funciones de distribuci\'on de probabilidades conjunta de manera indirecta, sin necesidad de hacer uso de las propiedades de las distribuciones de probabilidad de cada uno de los procesos que intervienen en la RSVC. Para cada una de las colas en cada sistema, el n\'umero de usuarios al tiempo en que llega el servidor a dar servicio est\'a
dado por el n\'umero de usuarios presentes en la cola al tiempo
$t$, m\'as el n\'umero de usuarios que llegan a la cola en el intervalo de tiempo $\left[\tau_{i},\overline{\tau}_{i}\right]$. Una vez definidas las FGP's conjuntas, se construyen las ecuaciones recursivas que permiten obtener la informaci\'on sobre la longitud de cada una de las colas al momento en que uno de los servidores llega a una de ellas para dar servicio.\smallskip

%__________________________________________________________________________
\subsection{Funciones Generadoras de Probabilidades}
%__________________________________________________________________________


Para cada uno de los procesos de llegada a las colas $X_{i},\hat{X}_{i}$,  $i=1,2$,  y $Y_{2}$, con $\tilde{X}_{2}=X_{2}+Y_{2}$ se define FGP: $P_{i}\left(z_{i}\right)=\esp\left[z_{i}^{X_{i}\left(t\right)}\right],\hat{P}_{i}\left(w_{i}\right)=\esp\left[w_{i}^{\hat{X}_{i}\left(t\right)}\right]$, para
$i=1,2$, y $\check{P}_{2}\left(z_{2}\right)=\esp\left[z_{2}^{Y_{2}\left(t\right)}\right], \tilde{P}_{2}\left(z_{2}\right)=\esp\left[z_{2}^{\tilde{X}_{2}\left(t\right)}\right]$ , con primer momento definidos por $\mu_{i}=\esp\left[X_{i}\left(t\right)\right]=P_{i}^{(1)}\left(1\right), \hat{\mu}_{i}=\esp\left[\hat{X}_{i}\left(t\right)\right]=\hat{P}_{i}^{(1)}\left(1\right)$, para $i=1,2$, y por otra parte
$\check{\mu}_{2}=\esp\left[Y_{2}\left(t\right)\right]=\check{P}_{2}^{(1)}\left(1\right),\tilde{\mu}_{2}=\esp\left[\tilde{X}_{2}\left(t\right)\right]=\tilde{P}_{2}^{(1)}\left(1\right)$.

Sus procesos se definen por: $S_{i}\left(z_{i}\right)=\esp\left[z_{i}^{\overline{\tau}_{i}-\tau_{i}}\right]$ y $\hat{S}_{i}\left(w_{i}\right)=\esp\left[w_{i}^{\overline{\zeta}_{i}-\zeta_{i}}\right]$, con primer momento dado por: $s_{i}=\esp\left[\overline{\tau}_{i}-\tau_{i}\right]$ y $\hat{s}_{i}=\esp\left[\overline{\zeta}_{i}-\zeta_{i}\right]$, para $i=1,2$. An\'alogamente los tiempos de traslado del servidor desde el momento en que termina de atender a una cola y llega a la
siguiente para comenzar a dar servicio est\'an dados por
$\tau_{i+1}-\overline{\tau}_{i}$ y
$\zeta_{i+1}-\overline{\zeta}_{i}$ para el sistema 1 y el sistema 2, respectivamente, con $i=1,2$.

La FGP para estos tiempos de traslado est\'an dados por $R_{i}\left(z_{i}\right)=\esp\left[z_{1}^{\tau_{i+1}-\overline{\tau}_{i}}\right]$ y $\hat{R}_{i}\left(w_{i}\right)=\esp\left[w_{i}^{\zeta_{i+1}-\overline{\zeta}_{i}}\right]$ y al igual que como se hizo con anterioridad, se tienen los primeros momentos de estos procesos de traslado del servidor entre las colas de cada uno de los sistemas que conforman la red de sistemas de visitas c\'iclicas: $r_{i}=R_{i}^{(1)}\left(1\right)=\esp\left[\tau_{i+1}-\overline{\tau}_{i}\right]$ y $\hat{r}_{i}=\hat{R}_{i}^{(1)}\left(1\right)=\esp\left[\zeta_{i+1}-\overline{\zeta}_{i}\right]$ para $i=1,2$.

Para el proceso $L_{i}\left(t\right)$ que determina el n\'umero de usuarios presentes en cada una de las colas al tiempo $t$, se define su FGP, $H_{i}\left(t\right)$, correspondiente al sistema 1,  mientras que para el segundo sistema el proceso correspondiente est\'a dado por $\hat{L}_{i}\left(t\right)$, con FGP $\hat{H}_{i}\left(t\right)$, es decir $H_{i}\left(t\right)=\esp\left[z_{i}^{L_{i}\left(t\right)}\right]$ y $\hat{H}_{i}\left(t\right)=\esp\left[w_{i}^{\hat{L}_{i}\left(t\right)}\right]$ para el sistema 1 y 2 respectivamente. Con lo dicho hasta ahora, se tiene que el n\'umero de usuarios
presentes en los tiempos $\overline{\tau}_{1},\overline{\tau}_{2},
\overline{\zeta}_{1},\overline{\zeta}_{2}$, es cero, es decir,
 $L_{i}\left(\overline{\tau_{i}}\right)=0,$ y
$\hat{L}_{i}\left(\overline{\zeta_{i}}\right)=0$ para i=1,2 para
cada uno de los dos sistemas.

Para cada una de las colas en la RSVC, el n\'umero de
usuarios al tiempo en que llega el servidor a una de ellas est\'a
dado por el n\'umero de usuarios presentes en la cola al tiempo
$t=\tau_{i},\zeta_{i}$, m\'as el n\'umero de usuarios que llegan a
la cola en el intervalo de tiempo
$\left[\tau_{i},\overline{\tau}_{i}\right],\left[\zeta_{i},\overline{\zeta}_{i}\right]$,
es decir $\hat{L}_{i}\left(\overline{\tau}_{j}\right)=\hat{L}_{i}\left(\tau_{j}\right)+\hat{X}_{i}\left(\overline{\tau}_{j}-\tau_{j}\right)$, para $i,j=1,2$, mientras que para el primer sistema: $L_{1}\left(\overline{\tau}_{j}\right)=L_{1}\left(\tau_{j}\right)+X_{1}\left(\overline{\tau}_{j}-\tau_{j}\right)$.

En el caso espec\'ifico de $Q_{2}$, adem\'as, hay que considerar
el n\'umero de usuarios que pasan del sistema 2 al sistema 1, a
traves de $\hat{Q}_{2}$ mientras el servidor en $Q_{2}$ est\'a
ausente, es decir, una vez que son atendidos en $\hat{Q}_{2}$:

\begin{equation}\label{Eq.UsuariosTotalesZ2}
L_{2}\left(\overline{\tau}_{1}\right)=L_{2}\left(\tau_{1}\right)+X_{2}\left(\overline{\tau}_{1}-\tau_{1}\right)+Y_{2}\left(\overline{\tau}_{1}-\tau_{1}\right).
\end{equation}

%_________________________________________________________________________
\subsection{El problema de la ruina del jugador}
%_________________________________________________________________________

Sea $\tilde{L}_{0}$ el n\'umero de usuarios presentes en la cola al momento en que el servidor llega para dar servicio. Sea $T$ el tiempo que requiere el servidor para atender a todos los usuarios presentes en la cola comenzando con $\tilde{L}_{0}$ usuarios. Supongamos que se tiene un jugador que cuenta con un capital inicial de $\tilde{L}_{0}\geq0$ unidades, esta persona realiza una
serie de dos juegos simult\'aneos e independientes de manera sucesiva, dichos eventos son independientes e id\'enticos entre
s\'i para cada realizaci\'on. La ganancia en el $n$-\'esimo juego es $\tilde{X}_{n}=X_{n}+Y_{n}$ unidades de las cuales se resta una cuota de 1 unidad por cada juego simult\'aneo, es decir, se restan dos unidades por cada juego realizado. En el contexto de teor\'ia de colas este proceso se puede pensar como el n\'umero de usuarios que llegan a una cola v\'ia dos procesos de arribo distintos e independientes entre s\'i. Su FGP est\'a dada por $F\left(z\right)=\esp\left[z^{\tilde{L}_{0}}\right]$, adem\'as
$$\tilde{P}\left(z\right)=\esp\left[z^{\tilde{X}_{n}}\right]=\esp\left[z^{X_{n}+Y_{n}}\right]=\esp\left[z^{X_{n}}z^{Y_{n}}\right]=\esp\left[z^{X_{n}}\right]\esp\left[z^{Y_{n}}\right]=P\left(z\right)\check{P}\left(z\right),$$

con $\tilde{\mu}=\esp\left[\tilde{X}_{n}\right]=\tilde{P}\left[z\right]<1$. Sea $\tilde{L}_{n}$ el capital remanente despu\'es del $n$-\'esimo
juego. Entonces

$$\tilde{L}_{n}=\tilde{L}_{0}+\tilde{X}_{1}+\tilde{X}_{2}+\cdots+\tilde{X}_{n}-2n.$$

La ruina del jugador ocurre despu\'es del $n$-\'esimo juego, es decir, la cola se vac\'ia despu\'es del $n$-\'esimo juego. Sea $g_{n,k}$ la probabilidad del evento de que el jugador no caiga en ruina antes del $n$-\'esimo juego, y que adem\'as tenga un capital de $k$ unidades antes del $n$-\'esimo juego, es decir, dada $n\in\left\{1,2,\ldots\right\}$ y $k\in\left\{0,1,2,\ldots\right\}$ $g_{n,k}:=P\left\{\tilde{L}_{j}>0, j=1,\ldots,n,\tilde{L}_{n}=k\right\}$, la cual adem\'as se puede escribir como:

\begin{eqnarray}\label{Eq.Gnk.2S}
g_{n,k}=\sum_{j=1}^{k+1}\sum_{l=1}^{j}g_{n-1,j}P\left\{X_{n}=k-j-l+1\right\}P\left\{Y_{n}=l\right\}.
\end{eqnarray}

Se definen las siguientes FGP:
\begin{equation}\label{Eq.3.16.a.2S}
G_{n}\left(z\right)=\sum_{k=0}^{\infty}g_{n,k}z^{k},\textrm{ para
}n=0,1,\ldots,
\end{equation}

\begin{equation}\label{Eq.3.16.b.2S}
G\left(z,w\right)=\sum_{n=0}^{\infty}G_{n}\left(z\right)w^{n}.
\end{equation}



%__________________________________________________________________________________
% INICIA LA PROPOSICIÓN
%__________________________________________________________________________________


\begin{Prop}\label{Prop.1.1.2S}
Sean $G_{n}\left(z\right)$ y $G\left(z,w\right)$ definidas como en
(\ref{Eq.3.16.a.2S}) y (\ref{Eq.3.16.b.2S}) respectivamente,
entonces
\begin{equation}\label{Eq.Pag.45}
G_{n}\left(z\right)=\frac{1}{z}\left[G_{n-1}\left(z\right)-G_{n-1}\left(0\right)\right]\tilde{P}\left(z\right).
\end{equation}

Adem\'as


\begin{equation}\label{Eq.Pag.46}
G\left(z,w\right)=\frac{zF\left(z\right)-wP\left(z\right)G\left(0,w\right)}{z-wR\left(z\right)},
\end{equation}

con un \'unico polo en el c\'irculo unitario, adem\'as, el polo es
de la forma $z=\theta\left(w\right)$ y satisface que

\begin{enumerate}
\item[i)]$\tilde{\theta}\left(1\right)=1$,

\item[ii)] $\tilde{\theta}^{(1)}\left(1\right)=\frac{1}{1-\tilde{\mu}}$,

\item[iii)]
$\tilde{\theta}^{(2)}\left(1\right)=\frac{\tilde{\mu}}{\left(1-\tilde{\mu}\right)^{2}}+\frac{\tilde{\sigma}}{\left(1-\tilde{\mu}\right)^{3}}$.
\end{enumerate}

Finalmente, adem\'as se cumple que
\begin{equation}
\esp\left[w^{T}\right]=G\left(0,w\right)=F\left[\tilde{\theta}\left(w\right)\right].
\end{equation}
\end{Prop}
%__________________________________________________________________________________
% TERMINA LA PROPOSICIÓN E INICIA LA DEMOSTRACI\'ON
%__________________________________________________________________________________

\begin{Coro}
El tiempo de ruina del jugador tiene primer y segundo momento
dados por

\begin{eqnarray}
\esp\left[T\right]&=&\frac{\esp\left[\tilde{L}_{0}\right]}{1-\tilde{\mu}}\\
Var\left[T\right]&=&\frac{Var\left[\tilde{L}_{0}\right]}{\left(1-\tilde{\mu}\right)^{2}}+\frac{\sigma^{2}\esp\left[\tilde{L}_{0}\right]}{\left(1-\tilde{\mu}\right)^{3}}.
\end{eqnarray}
\end{Coro}



%__________________________________________________________________________
\section{Procesos de Llegadas a las colas en la RSVC}
%__________________________________________________________________________

Se definen los procesos de llegada de los usuarios a cada una de
las colas dependiendo de la llegada del servidor pero del sistema
al cu\'al no pertenece la cola en cuesti\'on:

Para el sistema 1 y el servidor del segundo sistema

\begin{eqnarray*}
F_{i,j}\left(z_{i};\zeta_{j}\right)=\esp\left[z_{i}^{L_{i}\left(\zeta_{j}\right)}\right]=
\sum_{k=0}^{\infty}\prob\left[L_{i}\left(\zeta_{j}\right)=k\right]z_{i}^{k}\textrm{, para }i,j=1,2.
%F_{1,1}\left(z_{1};\zeta_{1}\right)&=&\esp\left[z_{1}^{L_{1}\left(\zeta_{1}\right)}\right]=
%\sum_{k=0}^{\infty}\prob\left[L_{1}\left(\zeta_{1}\right)=k\right]z_{1}^{k};\\
%F_{2,1}\left(z_{2};\zeta_{1}\right)&=&\esp\left[z_{2}^{L_{2}\left(\zeta_{1}\right)}\right]=
%\sum_{k=0}^{\infty}\prob\left[L_{2}\left(\zeta_{1}\right)=k\right]z_{2}^{k};\\
%F_{1,2}\left(z_{1};\zeta_{2}\right)&=&\esp\left[z_{1}^{L_{1}\left(\zeta_{2}\right)}\right]=
%\sum_{k=0}^{\infty}\prob\left[L_{1}\left(\zeta_{2}\right)=k\right]z_{1}^{k};\\
%F_{2,2}\left(z_{2};\zeta_{2}\right)&=&\esp\left[z_{2}^{L_{2}\left(\zeta_{2}\right)}\right]=
%\sum_{k=0}^{\infty}\prob\left[L_{2}\left(\zeta_{2}\right)=k\right]z_{2}^{k}.\\
\end{eqnarray*}

Para el segundo sistema y el servidor del primero


\begin{eqnarray*}
\hat{F}_{i,j}\left(w_{i};\tau_{j}\right)&=&\esp\left[w_{i}^{\hat{L}_{i}\left(\tau_{j}\right)}\right] =\sum_{k=0}^{\infty}\prob\left[\hat{L}_{i}\left(\tau_{j}\right)=k\right]w_{i}^{k}\textrm{, para }i,j=1,2.
%\hat{F}_{1,1}\left(w_{1};\tau_{1}\right)&=&\esp\left[w_{1}^{\hat{L}_{1}\left(\tau_{1}\right)}\right] =\sum_{k=0}^{\infty}\prob\left[\hat{L}_{1}\left(\tau_{1}\right)=k\right]w_{1}^{k}\\
%\hat{F}_{2,1}\left(w_{2};\tau_{1}\right)&=&\esp\left[w_{2}^{\hat{L}_{2}\left(\tau_{1}\right)}\right] =\sum_{k=0}^{\infty}\prob\left[\hat{L}_{2}\left(\tau_{1}\right)=k\right]w_{2}^{k}\\
%\hat{F}_{1,2}\left(w_{1};\tau_{2}\right)&=&\esp\left[w_{1}^{\hat{L}_{1}\left(\tau_{2}\right)}\right]
%=\sum_{k=0}^{\infty}\prob\left[\hat{L}_{1}\left(\tau_{2}\right)=k\right]w_{1}^{k}\\
%\hat{F}_{2,2}\left(w_{2};\tau_{2}\right)&=&\esp\left[w_{2}^{\hat{L}_{2}\left(\tau_{2}\right)}\right]
%=\sum_{k=0}^{\infty}\prob\left[\hat{L}_{2}\left(\tau_{2}\right)=k\right]w_{2}^{k}\\
\end{eqnarray*}


Ahora, con lo anterior definamos la FGP conjunta para el segundo sistema;% y $\tau_{1}$:


\begin{eqnarray*}
\esp\left[w_{1}^{\hat{L}_{1}\left(\tau_{j}\right)}w_{2}^{\hat{L}_{2}\left(\tau_{j}\right)}\right]
&=&\esp\left[w_{1}^{\hat{L}_{1}\left(\tau_{j}\right)}\right]
\esp\left[w_{2}^{\hat{L}_{2}\left(\tau_{j}\right)}\right]=\hat{F}_{1,j}\left(w_{1};\tau_{j}\right)\hat{F}_{2,j}\left(w_{2};\tau_{j}\right)=\hat{F}_{j}\left(w_{1},w_{2};\tau_{j}\right).\\
%\esp\left[w_{1}^{\hat{L}_{1}\left(\tau_{1}\right)}w_{2}^{\hat{L}_{2}\left(\tau_{1}\right)}\right]
%&=&\esp\left[w_{1}^{\hat{L}_{1}\left(\tau_{1}\right)}\right]
%\esp\left[w_{2}^{\hat{L}_{2}\left(\tau_{1}\right)}\right]=\hat{F}_{1,1}\left(w_{1};\tau_{1}\right)\hat{F}_{2,1}\left(w_{2};\tau_{1}\right)=\hat{F}_{1}\left(w_{1},w_{2};\tau_{1}\right)\\
%\esp\left[w_{1}^{\hat{L}_{1}\left(\tau_{2}\right)}w_{2}^{\hat{L}_{2}\left(\tau_{2}\right)}\right]
%&=&\esp\left[w_{1}^{\hat{L}_{1}\left(\tau_{2}\right)}\right]
%   \esp\left[w_{2}^{\hat{L}_{2}\left(\tau_{2}\right)}\right]=\hat{F}_{1,2}\left(w_{1};\tau_{2}\right)\hat{F}_{2,2}\left(w_{2};\tau_{2}\right)=\hat{F}_{2}\left(w_{1},w_{2};\tau_{2}\right).
\end{eqnarray*}

Con respecto al sistema 1 se tiene la FGP conjunta con respecto al servidor del otro sistema:
\begin{eqnarray*}
\esp\left[z_{1}^{L_{1}\left(\zeta_{j}\right)}z_{2}^{L_{2}\left(\zeta_{j}\right)}\right]
&=&\esp\left[z_{1}^{L_{1}\left(\zeta_{j}\right)}\right]
\esp\left[z_{2}^{L_{2}\left(\zeta_{j}\right)}\right]=F_{1,j}\left(z_{1};\zeta_{j}\right)F_{2,j}\left(z_{2};\zeta_{j}\right)=F_{j}\left(z_{1},z_{2};\zeta_{j}\right).
%\esp\left[z_{1}^{L_{1}\left(\zeta_{1}\right)}z_{2}^{L_{2}\left(\zeta_{1}\right)}\right]
%&=&\esp\left[z_{1}^{L_{1}\left(\zeta_{1}\right)}\right]
%\esp\left[z_{2}^{L_{2}\left(\zeta_{1}\right)}\right]=F_{1,1}\left(z_{1};\zeta_{1}\right)F_{2,1}\left(z_{2};\zeta_{1}\right)=F_{1}\left(z_{1},z_{2};\zeta_{1}\right)\\
%\esp\left[z_{1}^{L_{1}\left(\zeta_{2}\right)}z_{2}^{L_{2}\left(\zeta_{2}\right)}\right]
%&=&\esp\left[z_{1}^{L_{1}\left(\zeta_{2}\right)}\right]
%\esp\left[z_{2}^{L_{2}\left(\zeta_{2}\right)}\right]=F_{1,2}\left(z_{1};\zeta_{2}\right)F_{2,2}\left(z_{2};\zeta_{2}\right)=F_{2}\left(z_{1},z_{2};\zeta_{2}\right).
\end{eqnarray*}

Ahora analicemos la Red de Sistemas de Visitas C\'iclicas, se define la PGF conjunta al tiempo $t$ para los tiempos de visita del servidor en cada una de las colas, para comenzar a dar servicio, definidos anteriormente al tiempo
$t=\left\{\tau_{1},\tau_{2},\zeta_{1},\zeta_{2}\right\}$:

\begin{eqnarray}\label{Eq.Conjuntas}
F_{j}\left(z_{1},z_{2},w_{1},w_{2}\right)&=&\esp\left[\prod_{i=1}^{2}z_{i}^{L_{i}\left(\tau_{j}
\right)}\prod_{i=1}^{2}w_{i}^{\hat{L}_{i}\left(\tau_{j}\right)}\right]\\
\hat{F}_{j}\left(z_{1},z_{2},w_{1},w_{2}\right)&=&\esp\left[\prod_{i=1}^{2}z_{i}^{L_{i}
\left(\zeta_{j}\right)}\prod_{i=1}^{2}w_{i}^{\hat{L}_{i}\left(\zeta_{j}\right)}\right]
\end{eqnarray}
para $j=1,2$. Entonces, con la finalidad de encontrar el n\'umero de usuarios presentes en el sistema cuando el servidor termina de atender una de las colas de cualquier sistema se tiene lo siguiente


\begin{eqnarray*}
&&\esp\left[z_{1}^{L_{1}\left(\overline{\tau}_{1}\right)}z_{2}^{L_{2}\left(\overline{\tau}_{1}\right)}w_{1}^{\hat{L}_{1}\left(\overline{\tau}_{1}\right)}w_{2}^{\hat{L}_{2}\left(\overline{\tau}_{1}\right)}\right]=
\esp\left[z_{2}^{L_{2}\left(\overline{\tau}_{1}\right)}w_{1}^{\hat{L}_{1}\left(\overline{\tau}_{1}
\right)}w_{2}^{\hat{L}_{2}\left(\overline{\tau}_{1}\right)}\right]\\
&=&\esp\left[z_{2}^{L_{2}\left(\tau_{1}\right)+X_{2}\left(\overline{\tau}_{1}-\tau_{1}\right)+Y_{2}\left(\overline{\tau}_{1}-\tau_{1}\right)}w_{1}^{\hat{L}_{1}\left(\tau_{1}\right)+\hat{X}_{1}\left(\overline{\tau}_{1}-\tau_{1}\right)}w_{2}^{\hat{L}_{2}\left(\tau_{1}\right)+\hat{X}_{2}\left(\overline{\tau}_{1}-\tau_{1}\right)}\right]
\end{eqnarray*}
utilizando la (\ref{Eq.UsuariosTotalesZ2}), se tiene que


\begin{eqnarray*}
&=&\esp\left[z_{2}^{L_{2}\left(\tau_{1}\right)}z_{2}^{X_{2}\left(\overline{\tau}_{1}-\tau_{1}\right)}z_{2}^{Y_{2}\left(\overline{\tau}_{1}-\tau_{1}\right)}w_{1}^{\hat{L}_{1}\left(\tau_{1}\right)}w_{1}^{\hat{X}_{1}\left(\overline{\tau}_{1}-\tau_{1}\right)}w_{2}^{\hat{L}_{2}\left(\tau_{1}\right)}w_{2}^{\hat{X}_{2}\left(\overline{\tau}_{1}-\tau_{1}\right)}\right]\\
&=&\esp\left[z_{2}^{L_{2}\left(\tau_{1}\right)}\left\{w_{1}^{\hat{L}_{1}\left(\tau_{1}\right)}w_{2}^{\hat{L}_{2}\left(\tau_{1}\right)}\right\}\left\{z_{2}^{X_{2}\left(\overline{\tau}_{1}-\tau_{1}\right)}
z_{2}^{Y_{2}\left(\overline{\tau}_{1}-\tau_{1}\right)}w_{1}^{\hat{X}_{1}\left(\overline{\tau}_{1}-\tau_{1}\right)}w_{2}^{\hat{X}_{2}\left(\overline{\tau}_{1}-\tau_{1}\right)}\right\}\right]\\
\end{eqnarray*}
aplicando el hecho de que el n\'umero de usuarios que llegan a cada una de las colas del segundo sistema es independiente de las llegadas a las colas del primer sistema:

\begin{eqnarray*}
&=&\esp\left[z_{2}^{L_{2}\left(\tau_{1}\right)}\left\{z_{2}^{X_{2}\left(\overline{\tau}_{1}-\tau_{1}\right)}z_{2}^{Y_{2}\left(\overline{\tau}_{1}-\tau_{1}\right)}w_{1}^{\hat{X}_{1}\left(\overline{\tau}_{1}-\tau_{1}\right)}w_{2}^{\hat{X}_{2}\left(\overline{\tau}_{1}-\tau_{1}\right)}\right\}\right]\esp\left[w_{1}^{\hat{L}_{1}\left(\tau_{1}\right)}w_{2}^{\hat{L}_{2}\left(\tau_{1}\right)}\right]
\end{eqnarray*}
dado que los arribos a cada una de las colas son independientes, podemos separar la esperanza para los procesos de llegada a $Q_{1}$ y $Q_{2}$ al tiempo $\tau_{1}$, que es el tiempo en que el servidor visita a $Q_{1}$. Recordando que $\tilde{X}_{2}\left(z_{2}\right)=X_{2}\left(z_{2}\right)+Y_{2}\left(z_{2}\right)$ se tiene


\begin{eqnarray*}
&=&\esp\left[z_{2}^{L_{2}\left(\tau_{1}\right)}\left\{z_{2}^{\tilde{X}_{2}\left(\overline{\tau}_{1}-\tau_{1}\right)}w_{1}^{\hat{X}_{1}\left(\overline{\tau}_{1}-\tau_{1}\right)}w_{2}^{\hat{X}_{2}\left(\overline{\tau}_{1}-\tau_{1}\right)}\right\}\right]\esp\left[w_{1}^{\hat{L}_{1}\left(\tau_{1}\right)}w_{2}^{\hat{L}_{2}\left(\tau_{1}\right)}\right]\\
&=&\esp\left[z_{2}^{L_{2}\left(\tau_{1}\right)}\left\{\tilde{P}_{2}\left(z_{2}\right)^{\overline{\tau}_{1}-\tau_{1}}\hat{P}_{1}\left(w_{1}\right)^{\overline{\tau}_{1}-\tau_{1}}\hat{P}_{2}\left(w_{2}\right)^{\overline{\tau}_{1}-\tau_{1}}\right\}\right]\esp\left[w_{1}^{\hat{L}_{1}\left(\tau_{1}\right)}w_{2}^{\hat{L}_{2}\left(\tau_{1}\right)}\right]\\
&=&\esp\left[z_{2}^{L_{2}\left(\tau_{1}\right)}\left\{\tilde{P}_{2}\left(z_{2}\right)\hat{P}_{1}\left(w_{1}\right)\hat{P}_{2}\left(w_{2}\right)\right\}^{\overline{\tau}_{1}-\tau_{1}}\right]\esp\left[w_{1}^{\hat{L}_{1}\left(\tau_{1}\right)}w_{2}^{\hat{L}_{2}\left(\tau_{1}\right)}\right]\\
&=&\esp\left[z_{2}^{L_{2}\left(\tau_{1}\right)}\theta_{1}\left(\tilde{P}_{2}\left(z_{2}\right)\hat{P}_{1}\left(w_{1}\right)\hat{P}_{2}\left(w_{2}\right)\right)^{L_{1}\left(\tau_{1}\right)}\right]\esp\left[w_{1}^{\hat{L}_{1}\left(\tau_{1}\right)}w_{2}^{\hat{L}_{2}\left(\tau_{1}\right)}\right]\\
&=&F_{1}\left(\theta_{1}\left(\tilde{P}_{2}\left(z_{2}\right)\hat{P}_{1}\left(w_{1}\right)\hat{P}_{2}\left(w_{2}\right)\right),z{2}\right)\hat{F}_{1}\left(w_{1},w_{2};\tau_{1}\right)\equiv
F_{1}\left(\theta_{1}\left(\tilde{P}_{2}\left(z_{2}\right)\hat{P}_{1}\left(w_{1}\right)\hat{P}_{2}\left(w_{2}\right)\right),z_{2},w_{1},w_{2}\right).
\end{eqnarray*}

Las igualdades anteriores son ciertas pues el n\'umero de usuarios
que llegan a $\hat{Q}_{2}$ durante el intervalo
$\left[\tau_{1},\overline{\tau}_{1}\right]$ a\'un no han sido
atendidos por el servidor del sistema $2$ y por tanto a\'un no
pueden pasar al sistema $1$ a traves de $Q_{2}$. Por tanto el n\'umero de
usuarios que pasan de $\hat{Q}_{2}$ a $Q_{2}$ en el intervalo de
tiempo $\left[\tau_{1},\overline{\tau}_{1}\right]$ depende de la
pol\'itica de traslado entre los dos sistemas, conforme a la
secci\'on anterior.\smallskip

Por lo tanto
\begin{eqnarray}\label{Eq.Fs}
\esp\left[z_{1}^{L_{1}\left(\overline{\tau}_{1}\right)}z_{2}^{L_{2}\left(\overline{\tau}_{1}
\right)}w_{1}^{\hat{L}_{1}\left(\overline{\tau}_{1}\right)}w_{2}^{\hat{L}_{2}\left(
\overline{\tau}_{1}\right)}\right]&=&F_{1}\left(\theta_{1}\left(\tilde{P}_{2}\left(z_{2}\right)
\hat{P}_{1}\left(w_{1}\right)\hat{P}_{2}\left(w_{2}\right)\right),z_{2},w_{1},w_{2}\right)\\
&=&F_{1}\left(\theta_{1}\left(\tilde{P}_{2}\left(z_{2}\right)\hat{P}_{1}\left(w_{1}\right)\hat{P}_{2}\left(w_{2}\right)\right),z_{2}\right)\hat{F}_{1}\left(w_{1},w_{2};\tau_{1}\right)
\end{eqnarray}


Utilizando un razonamiento an\'alogo para $\overline{\tau}_{2}$ y la proposici\'on (\ref{Prop.1.1.2S}) referente al problema de la ruina del jugador obtenemos:

\begin{eqnarray*}
&&\esp\left[z_{1}^{L_{1}\left(\overline{\tau}_{2}\right)}z_{2}^{L_{2}\left(\overline{\tau}_{2}\right)}w_{1}^{\hat{L}_{1}\left(\overline{\tau}_{2}\right)}w_{2}^{\hat{L}_{2}\left(\overline{\tau}_{2}\right)}\right]=
\esp\left[z_{1}^{L_{1}\left(\overline{\tau}_{2}\right)}w_{1}^{\hat{L}_{1}\left(\overline{\tau}_{2}\right)}w_{2}^{\hat{L}_{2}\left(\overline{\tau}_{2}\right)}\right]\\
&=&\esp\left[z_{1}^{L_{1}\left(\tau_{2}\right)}\left\{P_{1}\left(z_{1}\right)\hat{P}_{1}\left(w_{1}\right)\hat{P}_{2}\left(w_{2}\right)\right\}^{\overline{\tau}_{2}-\tau_{2}}\right]
\esp\left[w_{1}^{\hat{L}_{1}\left(\tau_{2}\right)}w_{2}^{\hat{L}_{2}\left(\tau_{2}\right)}\right]\\
&=&\esp\left[z_{1}^{L_{1}\left(\tau_{2}\right)}\tilde{\theta}_{2}\left(P_{1}\left(z_{1}\right)\hat{P}_{1}\left(w_{1}\right)\hat{P}_{2}\left(w_{2}\right)\right)^{L_{2}\left(\tau_{2}\right)}\right]
\esp\left[w_{1}^{\hat{L}_{1}\left(\tau_{2}\right)}w_{2}^{\hat{L}_{2}\left(\tau_{2}\right)}\right]\\
&=&F_{2}\left(z_{1},\tilde{\theta}_{2}\left(P_{1}\left(z_{1}\right)\hat{P}_{1}\left(w_{1}\right)\hat{P}_{2}\left(w_{2}\right)\right)\right)
\hat{F}_{2}\left(w_{1},w_{2};\tau_{2}\right)\\
\end{eqnarray*}


entonces se define
\begin{eqnarray}
\esp\left[z_{1}^{L_{1}\left(\overline{\tau}_{2}\right)}z_{2}^{L_{2}\left(\overline{\tau}_{2}\right)}w_{1}^{\hat{L}_{1}\left(\overline{\tau}_{2}\right)}w_{2}^{\hat{L}_{2}\left(\overline{\tau}_{2}\right)}\right]=F_{2}\left(z_{1},\tilde{\theta}_{2}\left(P_{1}\left(z_{1}\right)\hat{P}_{1}\left(w_{1}\right)\hat{P}_{2}\left(w_{2}\right)\right),w_{1},w_{2}\right)\\
\equiv F_{2}\left(z_{1},\tilde{\theta}_{2}\left(P_{1}\left(z_{1}\right)\hat{P}_{1}\left(w_{1}\right)\hat{P}_{2}\left(w_{2}\right)\right)\right)
\hat{F}_{2}\left(w_{1},w_{2};\tau_{2}\right)
\end{eqnarray}

Para $\overline{\zeta}_{1}$ obtenemos una expresi\'on similar

\begin{eqnarray}
\esp\left[z_{1}^{L_{1}\left(\overline{\zeta}_{1}\right)}z_{2}^{L_{2}\left(\overline{\zeta}_{1}
\right)}w_{1}^{\hat{L}_{1}\left(\overline{\zeta}_{1}\right)}w_{2}^{\hat{L}_{2}\left(
\overline{\zeta}_{1}\right)}\right]&=&\hat{F}_{1}\left(z_{1},z_{2},\hat{\theta}_{1}\left(P_{1}\left(z_{1}\right)\tilde{P}_{2}\left(z_{2}\right)\hat{P}_{2}\left(w_{2}\right)\right),w_{2}\right)\\
&=&F_{1}\left(z_{1},z_{2};\zeta_{1}\right)\hat{F}_{1}\left(\hat{\theta}_{1}\left(P_{1}\left(z_{1}\right)\tilde{P}_{2}\left(z_{2}\right)\hat{P}_{2}\left(w_{2}\right)\right),w_{2}\right).
\end{eqnarray}


Finalmente para $\overline{\zeta}_{2}$
\begin{eqnarray}
\esp\left[z_{1}^{L_{1}\left(\overline{\zeta}_{2}\right)}z_{2}^{L_{2}\left(\overline{\zeta}_{2}\right)}w_{1}^{\hat{L}_{1}\left(\overline{\zeta}_{2}\right)}w_{2}^{\hat{L}_{2}\left(\overline{\zeta}_{2}\right)}\right]&=&\hat{F}_{2}\left(z_{1},z_{2},w_{1},\hat{\theta}_{2}\left(P_{1}\left(z_{1}\right)\tilde{P}_{2}\left(z_{2}\right)\hat{P}_{1}\left(w_{1}\right)\right)\right)\\
&=&F_{2}\left(z_{1},z_{2};\zeta_{2}\right)\hat{F}_{2}\left(w_{1},\hat{\theta}_{2}\left(P_{1}\left(z_{1}\right)\tilde{P}_{2}\left(z_{2}\right)\hat{P}_{1}\left(w_{1}
\right)\right)\right)
\end{eqnarray}
%__________________________________________________________________________
\section{Ecuaciones Recursivas para la RSVC}
%__________________________________________________________________________

Con lo desarrollado hasta ahora podemos encontrar las ecuaciones
recursivas que modelan la RSVC:

\begin{eqnarray*}
F_{2}\left(z_{1},z_{2},w_{1},w_{2}\right)&=&R_{1}\left(P_{1}\left(z_{1}\right)\tilde{P}_{2}\left(z_{2}\right)\prod_{i=1}^{2}
\hat{P}_{i}\left(w_{i}\right)\right)F_{1}\left(\theta_{1}\left(\tilde{P}_{2}\left(z_{2}\right)\hat{P}_{1}\left(w_{1}\right)\hat{P}_{2}\left(w_{2}\right)\right),z_{2}\right)\hat{F}_{1}\left(w_{1},w_{2};\tau_{1}\right),
\end{eqnarray*}


\begin{eqnarray*}
F_{1}\left(z_{1},z_{2},w_{1},w_{2}\right)&=&R_{2}\left(P_{1}\left(z_{1}\right)\tilde{P}_{2}\left(z_{2}\right)\prod_{i=1}^{2}
\hat{P}_{i}\left(w_{i}\right)\right)F_{2}\left(z_{1},\tilde{\theta}_{2}\left(P_{1}\left(z_{1}\right)\hat{P}_{1}\left(w_{1}\right)\hat{P}_{2}\left(w_{2}\right)\right)\right)
\hat{F}_{2}\left(w_{1},w_{2};\tau_{2}\right),
\end{eqnarray*}

\begin{eqnarray*}
\hat{F}_{2}\left(z_{1},z_{2},w_{1},w_{2}\right)&=&\hat{R}_{1}\left(P_{1}\left(z_{1}\right)\tilde{P}_{2}\left(z_{2}\right)\prod_{i=1}^{2}
\hat{P}_{i}\left(w_{i}\right)\right)F_{1}\left(z_{1},z_{2};\zeta_{1}\right)\hat{F}_{1}\left(\hat{\theta}_{1}\left(P_{1}\left(z_{1}\right)\tilde{P}_{2}\left(z_{2}\right)\hat{P}_{2}\left(w_{2}\right)\right),w_{2}\right),
\end{eqnarray*}

\begin{eqnarray*}
\hat{F}_{1}\left(z_{1},z_{2},w_{1},w_{2}\right)&=&\hat{R}_{2}\left(P_{1}\left(z_{1}\right)\tilde{P}_{2}\left(z_{2}\right)\prod_{i=1}^{2}
\hat{P}_{i}\left(w_{i}\right)\right)F_{2}\left(z_{1},z_{2};\zeta_{2}\right)\hat{F}_{2}\left(w_{1},\hat{\theta}_{2}\left(P_{1}\left(z_{1}\right)\tilde{P}_{2}\left(z_{2}\right)\hat{P}_{1}\left(w_{1}
\right)\right)\right),
\end{eqnarray*}


Con la finalidad de facilitar los c\'alculos para determinar los primeros y segundos momentos de los procesos involucrados en la RSVC, es conveniente utilizar la notaci\'on propuesta por Lang \cite{Lang}, es por eso que requerimos definir el operador diferencial $D_{i}$, $i=1,2,3,4$, donde $D_{1}f$ denota la derivad parcial de $f$ con respecto a $z_{1}$, $D_{3}f$ es la derivada parcial de $f$ con respecto a $w_{1}$ y $D_{4}f$ es la derivada parcial de $f$ con respecto a $w_{2}$. Otra consideraci\'on de gran utilidad es que la expresi\'on expresada, es obtenida como consecuencia de aplicar el operador diferencial y adem\'as evaluarla en $z_{1}=1,z_{2}=1,w_{1}=1$ y $w_{2}=1$. En este sentido, la expresi\'ion $F_{2}\left(z_{1},z_{2},w_{1},w_{2}\right)=R_{1}\left(P_{1}\left(z_{1}\right)\tilde{P}_{2}\left(z_{2}\right)\prod_{i=1}^{2}
\hat{P}_{i}\left(w_{i}\right)\right)F_{1}\left(\theta_{1}\left(\tilde{P}_{2}\left(z_{2}\right)\hat{P}_{1}\left(w_{1}\right)\hat{P}_{2}\left(w_{2}\right)\right),z_{2}\right)\hat{F}_{1}\left(w_{1},w_{2};\tau_{1}\right)$ ser\'a representada por su versi\'on simplificada $F_{2}=R_{1}F_{1}\hat{F}_{3}$. Por otra parte $D_{1}\left[R_{1}F_{1}\right]=D_{1}R_{1}\left(F_{1}\right)+R_{1}D_{1}F_{1}$, se tomar\'a simplemente como $D_{1}\left[R_{1}F_{1}\right]=D_{1}R_{1}+D_{1}F_{1}$.

%_________________________________________________________________________________________________
\subsection{Tiempos de Traslado del Servidor}
%_________________________________________________________________________________________________

Recordemos que los tiempos de traslado del servidor para cualquiera de las colas del sistema 1 est\'an dados por la expresi\'on:

\begin{eqnarray}\label{Ec.Ri}
R_{i}\left(\mathbf{z,w}\right)=R_{i}\left(P_{1}\left(z_{1}\right)\tilde{P}_{2}\left(z_{2}\right)\hat{P}_{1}\left(w_{1}\right)\hat{P}_{2}\left(w_{2}\right)\right)
\end{eqnarray}

entonces, las derivadas parciales con respecto a cada uno de los argumentos $z_{1},z_{2},w_{1}$ y $w_{2}$ son de la forma

\begin{eqnarray}\label{Ec.Derivada.Ri}
D_{i}R_{i}&=&DR_{i}D_{i}P_{i}
\end{eqnarray}
donde se hacen las siguientes convenciones:

\begin{eqnarray*}
\begin{array}{llll}
D_{2}P_{2}\equiv D_{2}\tilde{P}_{2}, & D_{3}P_{3}\equiv D_{3}\hat{P}_{1}, &D_{4}P_{4}\equiv D_{4}\hat{P}_{2},
\end{array}
\end{eqnarray*}

%_________________________________________________________________________________________________
\subsection{Longitudes de la Cola en tiempos del servidor del otro sistema}
%_________________________________________________________________________________________________


Recordemos que  $F_{1,2}\left(z_{1};\zeta_{2}\right)F_{2,2}\left(z_{2};\zeta_{2}\right)=F_{2}\left(z_{1},z_{2};\zeta_{2}\right)$, entonces

\begin{eqnarray*}
D_{1}F_{2}\left(z_{1},z_{2};\zeta_{2}\right)&=&D_{1}\left[F_{1,2}\left(z_{1};\zeta_{2}\right)F_{2,2}\left(z_{2};\zeta_{2}\right)\right]
=F_{2,2}\left(z_{2};\zeta_{2}\right)D_{1}F_{1,2}\left(z_{1};\zeta_{2}\right)=F_{1,2}^{(1)}\left(1\right)
\end{eqnarray*}

es decir, $D_{1}F_{2}=F_{1,2}^{(1)}(1)$; de manera an\'aloga se puede ver que $D_{2}F_{2}=F_{2,2}^{(1)}\left(1\right)$, mientras que $D_{3}F_{2}=D_{4}F_{2}=0$. Es decir, las expresiones resultantes pueden expresarse de manera general como:

%\begin{eqnarray*}
%\begin{array}{llll}
%D_{1}F_{1}=F_{1,1}^{(1)}\left(1\right),&D_{2}F_{1}=F_{2,1}^{(1)}\left(1\right), & D_{3}F_{1}=0, & D_{4}F_{1}=0,\\
%D_{1}F_{2}=F_{1,2}^{(1)}\left(1\right),&D_{2}F_{2}=F_{2,2}^{(1)}\left(1\right), & D_{3}F_{2}=0, & D_{4}F_{2}=0,\\
%D_{1}\hat{F}_{1}=0,&D_{2}\hat{F}_{1}=0,&D_{3}=\hat{F}_{1,1}^{(1)}\left(1\right),&D_{4}\hat{F}_{1}=\hat{F}_{2,1}^{(1)}\left(1\right)\\
%D_{1}\hat{F}_{2}=0,&D_{2}\hat{F}_{2}=0,&D_{3}\hat{F}_{2}=\hat{F}_{1,2}^{(1)}\left(1\right),&D_{4}\hat{F}_{2}=\hat{F}_{2,2}^{(1)}\left(1\right)\\
%\end{array}
%\end{eqnarray*}

%que en general pueden escribirse como

\begin{eqnarray*}
\begin{array}{ccc}
D_{i}F_{j}=\indora_{i\leq2}F_{i,j}^{(1)}\left(1\right),& \textrm{ y } &D_{i}\hat{F}_{j}=\indora_{i\geq2}F_{i,j}^{(1)}\left(1\right)
\end{array}
\end{eqnarray*}

%_________________________________________________________________________________________________
\subsection{Usuarios presentes en la cola en tiempos del servidor de sus sistema}
%_________________________________________________________________________________________________
Recordemos la expresi\'on obtenida para las longitudes de la cola para cada uno de los sistemas considerando que los tiempos del servidor correspondiente al mismo sistema: $F_{1}\left(\theta_{1}\left(\tilde{P}_{2}\left(z_{2}\right)\hat{P}_{1}\left(w_{1}
\right)\hat{P}_{2}\left(w_{2}\right)\right),z_{2}\right)$. Al igual que antes, podemos obtener las expresiones resultantes de aplicar el operador diferencial con respecto a cada uno de los argumentos:

$D_{1}F_{1}=0$, $D_{2}F_{1}=D_{1}F_{1}D\theta_{1}D_{2}\tilde{P}_{2}+D_{2}F_{1}$, $D_{3}F_{1}=D_{1}F_{1}D\theta_{1}D_{3}\hat{P}_{1}+D_{3}\hat{F}_{1}$ y finalmente
$D_{4}F_{1}=D_{1}F_{1}D\theta_{1}D_{4}\hat{P}_{2}+D_{4}\hat{F}_{1}$, en t\'erminos generales:

\begin{eqnarray*}
\begin{array}{ll}
D_{i}F_{1}=\indora_{i\neq1}D_{1}F_{1}D\theta_{1}D_{i}P_{i}+\indora_{i=2}D_{i}F_{1}, & D_{i}F_{2}=\indora_{i\neq2}D_{2}F_{2}D\tilde{\theta}_{2}D_{i}P_{i}+\indora_{i=1}D_{i}F_{2}\\
D_{i}\hat{F}_{1}=\indora_{i\neq3}D_{3}\hat{F}_{1}D\hat{\theta}_{1}D_{i}P_{i}+\indora_{i=4}D_{i}\hat{F}_{1},& D_{i}\hat{F}_{2}=\indora_{i\neq4}D_{4}\hat{F}_{2}D\hat{\theta}_{2}D_{i}P_{i}+\indora_{i=3}D_{i}\hat{F}_{2}.
\end{array}
\end{eqnarray*}

\begin{eqnarray}
D_{i}F_{1}&=&\indora_{i\neq1}D_{1}F_{1}D\theta_{1}D_{i}P_{i}+\indora_{i=2}D_{i}F_{1},\\ D_{i}F_{2}&=&\indora_{i\neq2}D_{2}F_{2}D\tilde{\theta}_{2}D_{i}P_{i}+\indora_{i=1}D_{i}F_{2}\\
D_{i}\hat{F}_{1}&=&\indora_{i\neq3}D_{3}\hat{F}_{1}D\hat{\theta}_{1}D_{i}P_{i}+\indora_{i=4}D_{i}\hat{F}_{1},\\
D_{i}\hat{F}_{2}&=&\indora_{i\neq4}D_{4}\hat{F}_{2}D\hat{\theta}_{2}D_{i}P_{i}+\indora_{i=3}D_{i}\hat{F}_{2}.
\end{eqnarray}


%_________________________________________________________________________________________________
\subsection{Usuarios presentes en la RSVC}
%_________________________________________________________________________________________________

Hagamos lo correspondiente para las longitudes de las colas de la RSVC utilizando las expresiones obtenidas en las secciones anteriores, recordemos que

\begin{eqnarray*}
\mathbf{F}_{1}\left(\theta_{1}\left(\tilde{P}_{2}\left(z_{2}\right)\hat{P}_{1}\left(w_{1}\right)
\hat{P}_{2}\left(w_{2}\right)\right),z_{2},w_{1},w_{2}\right)=
F_{1}\left(\theta_{1}\left(\tilde{P}_{2}\left(z_{2}\right)\hat{P}_{1}\left(w_{1}
\right)\hat{P}_{2}\left(w_{2}\right)\right),z_{2}\right)
\hat{F}_{1}\left(w_{1},w_{2};\tau_{1}\right)\\
\end{eqnarray*}

entonces



\begin{eqnarray*}
D_{1}\mathbf{F}_{1}&=& 0\\
D_{2}\mathbf{F}_{1}&=&f_{1}\left(1\right)\left(\frac{1}{1-\mu_{1}}\right)\tilde{\mu}_{2}+f_{1}\left(2\right)\\
D_{3}\mathbf{F}_{1}&=&f_{1}\left(1\right)\left(\frac{1}{1-\mu_{1}}\right)\hat{\mu}_{1}+\hat{F}_{1,1}^{(1)}\left(1\right)\\
D_{4}\mathbf{F}_{1}&=&f_{1}\left(1\right)\left(\frac{1}{1-\mu_{1}}\right)\hat{\mu}_{2}+\hat{F}_{2,1}^{(1)}\left(1\right)
\end{eqnarray*}


para $\tau_{2}$:

\begin{eqnarray*}
\mathbf{F}_{2}\left(z_{1},\tilde{\theta}_{2}\left(P_{1}\left(z_{1}\right)\hat{P}_{1}\left(w_{1}\right)\hat{P}_{2}\left(w_{2}\right)\right),
w_{1},w_{2}\right)=F_{2}\left(z_{1},\tilde{\theta}_{2}\left(P_{1}\left(z_{1}\right)\hat{P}_{1}\left(w_{1}\right)
\hat{P}_{2}\left(w_{2}\right)\right)\right)\hat{F}_{2}\left(w_{1},w_{2};\tau_{2}\right)
\end{eqnarray*}
se tiene que

\begin{eqnarray*}
D_{1}\mathbf{F}_{2}&=&f_{2}\left(2\right)\left(\frac{1}{1-\tilde{\mu}_{2}}\right)\mu_{1}+f_{2}\left(1\right)\\
D_{2}\mathbf{F}_{2}&=&0\\
D_{3}\mathbf{F}_{2}&=&f_{2}\left(2\right)\left(\frac{1}{1-\tilde{\mu}_{2}}\right)\hat{\mu}_{1}+\hat{F}_{2,1}^{(1)}\left(1\right)\\
D_{4}\mathbf{F}_{2}&=&f_{2}\left(2\right)\left(\frac{1}{1-\tilde{\mu}_{2}}\right)\hat{\mu}_{2}+\hat{F}_{2,2}^{(1)}\left(1\right)\\
\end{eqnarray*}



Ahora para el segundo sistema

\begin{eqnarray*}\hat{\mathbf{F}}_{1}\left(z_{1},z_{2},\hat{\theta}_{1}\left(P_{1}\left(z_{1}\right)\tilde{P}_{2}\left(z_{2}\right)\hat{P}_{2}\left(w_{2}\right)\right),
w_{2}\right)=F_{1}\left(z_{1},z_{2};\zeta_{1}\right)\hat{F}_{1}\left(\hat{\theta}_{1}\left(P_{1}\left(z_{1}\right)\tilde{P}_{2}\left(z_{2}\right)
\hat{P}_{2}\left(w_{2}\right)\right),w_{2}\right)
\end{eqnarray*}
entonces

\begin{eqnarray*}
D_{1}\hat{\mathbf{F}}_{1}&=&\hat{f}_{1}\left(1\right)\left(\frac{1}{1-\hat{\mu}_{1}}\right)\mu_{1}+F_{1,1}^{(1)}\left(1\right)\\
D_{2}\hat{\mathbf{F}}_{1}&=&\hat{f}_{1}\left(1\right)\left(\frac{1}{1-\hat{\mu}_{1}}\right)\tilde{\mu}_{2}+F_{2,1}^{(1)}\left(1\right)\\
D_{3}\hat{\mathbf{F}}_{1}&=&0\\
D_{4}\hat{\mathbf{F}}_{1}&=&\hat{f}_{1}\left(1\right)\left(\frac{1}{1-\hat{\mu}_{1}}\right)\hat{\mu}_{2}+\hat{f}_{1}\left(2\right)\\
\end{eqnarray*}




Finalmente para $\zeta_{2}$

\begin{eqnarray*}\hat{\mathbf{F}}_{2}\left(z_{1},z_{2},w_{1},\hat{\theta}_{2}\left(P_{1}\left(z_{1}\right)\tilde{P}_{2}\left(z_{2}\right)\hat{P}_{1}\left(w_{1}\right)\right)\right)&=&F_{2}\left(z_{1},z_{2};\zeta_{2}\right)\hat{F}_{2}\left(w_{1},\hat{\theta}_{2}\left(P_{1}\left(z_{1}\right)\tilde{P}_{2}\left(z_{2}\right)\hat{P}_{1}\left(w_{1}\right)\right)\right]
\end{eqnarray*}
por tanto:


\begin{eqnarray*}
D_{1}\hat{\mathbf{F}}_{2}&=&\hat{f}_{2}\left(1\right)\left(\frac{1}{1-\hat{\mu}_{2}}\right)\mu_{1}+F_{1,2}^{(1)}\left(1\right)\\
D_{2}\hat{\mathbf{F}}_{2}&=&\hat{f}_{2}\left(1\right)\left(\frac{1}{1-\hat{\mu}_{2}}\right)\tilde{\mu}_{2}+F_{2,2}^{(1)}\left(1\right)\\
D_{3}\hat{\mathbf{F}}_{2}&=&\hat{f}_{2}\left(1\right)\left(\frac{1}{1-\hat{\mu}_{2}}\right)\hat{\mu}_{1}+\hat{f}_{2}\left(1\right)\\
D_{4}\hat{\mathbf{F}}_{2}&=&0\\
\end{eqnarray*}


%_________________________________________________________________________________________________
\subsection{Ecuaciones Recursivas}
%_________________________________________________________________________________________________

Entonces, de todo lo desarrollado hasta ahora se tienen las siguientes ecuaciones:

%Para $$, se tiene que


\begin{eqnarray}\label{Ec.Primeras.Derivadas.Parciales}
\begin{array}{ll}
\mathbf{F}_{1}=R_{2}F_{2}\hat{F}_{2}, & D_{i}\mathbf{F}_{1}=D_{i}\left(R_{2}+F_{2}+\indora_{i\geq3}\hat{F}_{2}\right)\\
\mathbf{F}_{2}=R_{1}F_{1}\hat{F}_{1}, & D_{i}\mathbf{F}_{2}=D_{i}\left(R_{1}+F_{1}+\indora_{i\geq3}\hat{F}_{1}\right)\\
\hat{\mathbf{F}}_{1}=\hat{R}_{2}\hat{F}_{2}F_{2}, & D_{i}\hat{\mathbf{F}}_{1}=D_{i}\left(\hat{R}_{2}+\hat{F}_{2}+\indora_{i\leq2}F_{2}\right)\\
\hat{\mathbf{F}}_{2}=\hat{R}_{1}\hat{F}_{1}F_{1}, & D_{i}\hat{\mathbf{F}}_{2}=D_{i}\left(\hat{R}_{1}+\hat{F}_{1}+\indora_{i\leq2}F_{1}\right)
\end{array}
\end{eqnarray}

cuyas expresiones son de la forma:


\begin{eqnarray*}
\begin{array}{ll}
D_{1}\mathbf{F}_{2}=r_{1}\mu_{1},&
D_{2}\mathbf{F}_{2}=r_{1}\tilde{\mu}_{2}+f_{1}\left(1\right)\left(\frac{1}{1-\mu_{1}}\right)\tilde{\mu}_{2}+f_{1}\left(2\right),\\
D_{3}\mathbf{F}_{2}=r_{1}\hat{\mu}_{1}+f_{1}\left(1\right)\left(\frac{1}{1-\mu_{1}}\right)\hat{\mu}_{1}+\hat{F}_{1,1}^{(1)}\left(1\right),&
D_{4}\mathbf{F}_{2}=r_{1}\hat{\mu}_{2}+f_{1}\left(1\right)\left(\frac{1}{1-\mu_{1}}\right)\hat{\mu}_{2}+\hat{F}_{2,1}^{(1)}\left(1\right),\\
D_{1}\mathbf{F}_{1}=r_{2}\mu_{1}+f_{2}\left(2\right)\left(\frac{1}{1-\tilde{\mu}_{2}}\right)\mu_{1}+f_{2}\left(1\right),&
D_{2}\mathbf{F}_{1}=r_{2}\tilde{\mu}_{2},\\
D_{3}\mathbf{F}_{1}=r_{2}\hat{\mu}_{1}+f_{2}\left(2\right)\left(\frac{1}{1-\tilde{\mu}_{2}}\right)\hat{\mu}_{1}+\hat{F}_{2,1}^{(1)}\left(1\right),&
D_{4}\mathbf{F}_{1}=r_{2}\hat{\mu}_{2}+f_{2}\left(2\right)\left(\frac{1}{1-\tilde{\mu}_{2}}\right)\hat{\mu}_{2}+\hat{F}_{2,2}^{(1)}\left(1\right),\\
D_{1}\hat{\mathbf{F}}_{2}=\hat{r}_{1}\mu_{1}+\hat{f}_{1}\left(1\right)\left(\frac{1}{1-\hat{\mu}_{1}}\right)\mu_{1}+F_{1,1}^{(1)}\left(1\right),&
D_{2}\hat{\mathbf{F}}_{2}=\hat{r}_{1}\mu_{2}+\hat{f}_{1}\left(1\right)\left(\frac{1}{1-\hat{\mu}_{1}}\right)\tilde{\mu}_{2}+F_{2,1}^{(1)}\left(1\right),\\
D_{3}\hat{\mathbf{F}}_{2}=\hat{r}_{1}\hat{\mu}_{1},&
D_{4}\hat{\mathbf{F}}_{2}=\hat{r}_{1}\hat{\mu}_{2}+\hat{f}_{1}\left(1\right)\left(\frac{1}{1-\hat{\mu}_{1}}\right)\hat{\mu}_{2}+\hat{f}_{1}\left(2\right),\\
D_{1}\hat{\mathbf{F}}_{1}=\hat{r}_{2}\mu_{1}+\hat{f}_{2}\left(1\right)\left(\frac{1}{1-\hat{\mu}_{2}}\right)\mu_{1}+F_{1,2}^{(1)}\left(1\right),&
D_{2}\hat{\mathbf{F}}_{1}=\hat{r}_{2}\tilde{\mu}_{2}+\hat{f}_{2}\left(2\right)\left(\frac{1}{1-\hat{\mu}_{2}}\right)\tilde{\mu}_{2}+F_{2,2}^{(1)}\left(1\right),\\
D_{3}\hat{\mathbf{F}}_{1}=\hat{r}_{2}\hat{\mu}_{1}+\hat{f}_{2}\left(2\right)\left(\frac{1}{1-\hat{\mu}_{2}}\right)\hat{\mu}_{1}+\hat{f}_{2}\left(1\right),&
D_{4}\hat{\mathbf{F}}_{1}=\hat{r}_{2}\hat{\mu}_{2}
\end{array}
\end{eqnarray*}


de las cuales resulta

\begin{eqnarray*}
\begin{array}{llll}
f_{2}\left(1\right)=r_{1}\mu_{1},&
f_{1}\left(2\right)=r_{2}\tilde{\mu}_{2},&
\hat{f}_{1}\left(4\right)=\hat{r}_{2}\hat{\mu}_{2},&
\hat{f}_{2}\left(3\right)=\hat{r}_{1}\hat{\mu}_{1}
\end{array}
\end{eqnarray*}

\begin{eqnarray*}
f_{1}\left(1\right)&=&r_{2}\mu_{1}+\mu_{1}\left(\frac{f_{2}\left(2\right)}{1-\tilde{\mu}_{2}}\right)+r_{1}\mu_{1}=\mu_{1}\left(r_{1}+r_{2}+\frac{f_{2}\left(2\right)}{1-\tilde{\mu}_{2}}\right)=\mu_{1}\left(r+\frac{f_{2}\left(2\right)}{1-\tilde{\mu}_{2}}\right),\\
f_{1}\left(3\right)&=&r_{2}\hat{\mu}_{1}+\hat{\mu}_{1}\left(\frac{f_{2}\left(2\right)}{1-\tilde{\mu}_{2}}\right)+\hat{F}^{(1)}_{1,2}\left(1\right)=\hat{\mu}_{1}\left(r_{2}+\frac{f_{2}\left(2\right)}{1-\tilde{\mu}_{2}}\right)+\hat{F}_{1,2}^{(1)}\left(1\right),\end{eqnarray*}

utilizando un razonamiento an\'alogo a los anteriores se puede verificar que

\begin{eqnarray*}
\begin{array}{ll}
f_{1}\left(4\right)=\hat{\mu}_{2}\left(r_{2}+\frac{f_{2}\left(2\right)}{1-\tilde{\mu}_{2}}\right)+\hat{F}_{2,2}^{(1)}\left(1\right),&
f_{2}\left(2\right)=\left(r+\frac{f_{1}\left(1\right)}{1-\mu_{1}}\right)\tilde{\mu}_{2},\\
f_{2}\left(3\right)=\hat{\mu}_{1}\left(r_{1}+\frac{f_{1}\left(1\right)}{1-\mu_{1}}\right)+\hat{F}_{1,1}^{(1)}\left(1\right),&
f_{2}\left(4\right)=\hat{\mu}_{2}\left(r_{1}+\frac{f_{1}\left(1\right)}{1-\mu_{1}}\right)+\hat{F}_{2,1}^{(1)}\left(1\right),
\end{array}
\end{eqnarray*}


\begin{eqnarray*}
\begin{array}{ll}
\hat{f}_{1}\left(1\right)=\left(\hat{r}_{2}+\frac{\hat{f}_{2}\left(4\right)}{1-\hat{\mu}_{2}}\right)\mu_{1}+F_{1,2}^{(1)}\left(1\right),&
\hat{f}_{1}\left(2\right)=\left(\hat{r}_{2}+\frac{\hat{f}_{2}\left(4\right)}{1-\hat{\mu}_{2}}\right)\tilde{\mu}_{2}+F_{2,2}^{(1)}\left(1\right),\\
\hat{f}_{1}\left(3\right)=\left(\hat{r}+\frac{\hat{f}_{2}\left(4\right)}{1-\hat{\mu}_{2}}\right)\hat{\mu}_{1},&
\hat{f}_{2}\left(1\right)=\left(\hat{r}_{1}+\frac{\hat{f}_{1}\left(3\right)}{1-\hat{\mu}_{1}}\right)\mu_{1}+F_{1,1}^{(1)}\left(1\right),\\
\hat{f}_{2}\left(2\right)=\left(\hat{r}_{1}+\frac{\hat{f}_{1}\left(3\right)}{1-\hat{\mu}_{1}}\right)\tilde{\mu}_{2}+F_{2,1}^{(1)}\left(1\right),&
\hat{f}_{2}\left(4\right)=\left(\hat{r}+\frac{\hat{f}_{1}\left(3\right)}{1-\hat{\mu}_{1}}\right)\hat{\mu}_{2},\\
\end{array}
\end{eqnarray*}


%_______________________________________________________________________________________________
\subsection{Soluci\'on del Sistema de Ecuaciones Lineales}
%_________________________________________________________________________________________________

Si $\mu=\mu_{1}+\tilde{\mu}_{2}$, $\hat{\mu}=\hat{\mu}_{1}+\hat{\mu}_{2}$, $r=r_{1}+r_{2}$ y $\hat{r}=\hat{r}_{1}+\hat{r}_{2}$ la soluci\'on del sistema de
ecuaciones est\'a dada por


\begin{eqnarray*}
f_{1}\left(1\right)&=&\mu_{1}\left(r_{2}+\frac{f_{2}\left(2\right)}{1-\tilde{\mu}_{2}}\right)+\hat{F}_{2,1}^{(1)}\left(1\right)=\mu_{1}\left(r_{2}+\frac{r\frac{\tilde{\mu}_{2}\left(1-\tilde{\mu}_{2}\right)}{1-\mu}}{1-\tilde{\mu}_{2}}\right)+\hat{F}_{2,1}^{(1)}\left(1\right)=\mu_{1}\left(r_{2}+\frac{r\tilde{\mu}_{2}}{1-\mu}\right)+\hat{F}_{2,1}^{(1)}\left(1\right),
\end{eqnarray*}

de manera an\'aloga se obtiene lo siguiente:


\begin{eqnarray*}
\begin{array}{ll}
f_{1}\left(3\right)=\hat{\mu}_{1}\left(r_{2}+\frac{r\tilde{\mu}_{2}}{1-\mu}\right)+\hat{F}_{2,1}^{(1)}\left(1\right),&
f_{1}\left(4\right)=\hat{\mu}_{2}\left(r_{2}+\frac{r\tilde{\mu}_{2}}{1-\mu}\right)+\hat{F}_{2,2}^{(1)}\left(1\right),\\
f_{2}\left(3\right)=\hat{\mu}_{1}\left(r_{1}+\frac{r\mu_{1}}{1-\mu}\right)+\hat{F}_{1,1}^{(1)}\left(1\right),&
f_{2}\left(4\right)=\hat{\mu}_{2}\left(r_{1}+\frac{r\mu_{1}}{1-\mu}\right)+\hat{F}_{2,1}^{(1)}\left(1\right),\\
\hat{f}_{1}\left(1\right)=\mu_{1}\left(\hat{r}_{2}+\frac{\hat{r}\hat{\mu}_{2}}{1-\hat{\mu}}\right)+F_{1,2}^{(1)}\left(1\right),&
\hat{f}_{1}\left(2\right)=\tilde{\mu}_{2}\left(\hat{r}_{2}+\frac{\hat{r}\hat{\mu}_{2}}{1-\hat{\mu}}\right)+F_{2,2}^{(1)}\left(1\right),\\
\hat{f}_{2}\left(1\right)=\mu_{1}\left(\hat{r}_{1}+\frac{\hat{r}\hat{\mu}_{1}}{1-\hat{\mu}}\right)+F_{1,1}^{(1)}\left(1\right),&
\hat{f}_{2}\left(2\right)=\tilde{\mu}_{2}\left(\hat{r}_{1}+\frac{\hat{r}\hat{\mu}_{1}}{1-\hat{\mu}}\right)+F_{2,1}^{(1)}\left(1\right)
\end{array}
\end{eqnarray*}


%\begin{eqnarray*}
%\end{eqnarray*}

%----------------------------------------------------------------------------------------
\section{Resultado Principal}
%----------------------------------------------------------------------------------------
Sean $\mu_{1},\mu_{2},\check{\mu}_{2},\hat{\mu}_{1},\hat{\mu}_{2}$ y $\tilde{\mu}_{2}=\mu_{2}+\check{\mu}_{2}$ los valores esperados de los proceso definidos anteriormente, y sean $r_{1},r_{2}, \hat{r}_{1}$ y $\hat{r}_{2}$ los valores esperado s de los tiempos de traslado del servidor entre las colas para cada uno de los sistemas de visitas c\'iclicas. Si se definen $\mu=\mu_{1}+\tilde{\mu}_{2}$, $\hat{\mu}=\hat{\mu}_{1}+\hat{\mu}_{2}$, y $r=r_{1}+r_{2}$ y  $\hat{r}=\hat{r}_{1}+\hat{r}_{2}$, entonces se tiene el siguiente resultado.

\begin{Teo}
Supongamos que $\mu<1$, $\hat{\mu}<1$, entonces, el n\'umero de usuarios presentes en cada una de las colas que conforman la RSVC cuando uno de los servidores visita a alguna de ellas est\'a dada por la soluci\'on del Sistema de Ecuaciones Lineales presentados arriba cuyas expresiones damos a continuaci\'on:
%{\footnotesize{


\begin{eqnarray*}
\begin{array}{lll}
f_{1}\left(1\right)=\mu_{1}\left(r_{2}+\frac{r\tilde{\mu}_{2}}{1-\mu}\right)+\hat{F}_{2,1}^{(1)}\left(1\right),&f_{1}\left(2\right)=r_{2}\tilde{\mu}_{2},&f_{1}\left(3\right)=\hat{\mu}_{1}\left(r_{2}+\frac{r\tilde{\mu}_{2}}{1-\mu}\right)+\hat{F}_{2,1}^{(1)}\left(1\right),\\
f_{1}\left(4\right)=\hat{\mu}_{2}\left(r_{2}+\frac{r\tilde{\mu}_{2}}{1-\mu}\right)+\hat{F}_{2,2}^{(1)}\left(1\right),&f_{2}\left(1\right)=r_{1}\mu_{1},&f_{2}\left(2\right)=r\frac{\tilde{\mu}_{2}\left(1-\tilde{\mu}_{2}\right)}{1-\mu},\\
f_{2}\left(3\right)=\hat{\mu}_{1}\left(r_{1}+\frac{r\mu_{1}}{1-\mu}\right)+\hat{F}_{1,1}^{(1)}\left(1\right),&f_{2}\left(4\right)=\hat{\mu}_{2}\left(r_{1}+\frac{r\mu_{1}}{1-\mu}\right)+\hat{F}_{2,1}^{(1)}\left(1\right),&\hat{f}_{1}\left(1\right)=\mu_{1}\left(\hat{r}_{2}+\frac{\hat{r}\hat{\mu}_{2}}{1-\hat{\mu}}\right)+F_{1,2}^{(1)}\left(1\right),\\
\hat{f}_{1}\left(2\right)=\tilde{\mu}_{2}\left(\hat{r}_{2}+\frac{\hat{r}\hat{\mu}_{2}}{1-\hat{\mu}}\right)+F_{2,2}^{(1)}\left(1\right),&\hat{f}_{1}\left(3\right)=\hat{r}\frac{\hat{\mu}_{1}\left(1-\hat{\mu}_{1}\right)}{1-\hat{\mu}},&\hat{f}_{1}\left(4\right)=\hat{r}_{2}\hat{\mu}_{2},\\
\hat{f}_{2}\left(1\right)=\mu_{1}\left(\hat{r}_{1}+\frac{\hat{r}\hat{\mu}_{1}}{1-\hat{\mu}}\right)+F_{1,1}^{(1)}\left(1\right),&\hat{f}_{2}\left(2\right)=\tilde{\mu}_{2}\left(\hat{r}_{1}+\frac{\hat{r}\hat{\mu}_{1}}{1-\hat{\mu}}\right)+F_{2,1}^{(1)}\left(1\right),&\hat{f}_{2}\left(3\right)=\hat{r}_{1}\hat{\mu}_{1},\\
&\hat{f}_{2}\left(4\right)=\hat{r}\frac{\hat{\mu}_{2}\left(1-\hat{\mu}_{2}\right)}{1-\hat{\mu}}.&\\
\end{array}
\end{eqnarray*} %}}
\end{Teo}
%\newpage
%___________________________________________________________________________________________
%
\section{Derivadas de Orden Superior}
%___________________________________________________________________________________________
%
Si tomamos la derivada de segundo orden con respecto a las ecuaciones dadas en (\ref{Ec.Primeras.Derivadas.Parciales}) obtenemos

\small{
\begin{eqnarray*}\label{Ec.Derivadas.Segundo.Orden}
D_{k}D_{i}F_{1}&=&D_{k}D_{i}\left(R_{2}+F_{2}+\indora_{i\geq3}\hat{F}_{4}\right)+D_{i}R_{2}D_{k}\left(F_{2}+\indora_{k\geq3}\hat{F}_{4}\right)+D_{i}F_{2}D_{k}\left(R_{2}+\indora_{k\geq3}\hat{F}_{4}\right)+\indora_{i\geq3}D_{i}\hat{F}_{4}D_{k}\left(R_{}+F_{2}\right)\\
D_{k}D_{i}F_{2}&=&D_{k}D_{i}\left(R_{1}+F_{1}+\indora_{i\geq3}\hat{F}_{3}\right)+D_{i}R_{1}D_{k}\left(F_{1}+\indora_{k\geq3}\hat{F}_{3}\right)+D_{i}F_{1}D_{k}\left(R_{1}+\indora_{k\geq3}\hat{F}_{3}\right)+\indora_{i\geq3}D_{i}\hat{F}_{3}D_{k}\left(R_{1}+F_{1}\right)\\
D_{k}D_{i}\hat{F}_{3}&=&D_{k}D_{i}\left(\hat{R}_{4}+\indora_{i\leq2}F_{2}+\hat{F}_{4}\right)+D_{i}\hat{R}_{4}D_{k}\left(\indora_{k\leq2}F_{2}+\hat{F}_{4}\right)+D_{i}\hat{F}_{4}D_{k}\left(\hat{R}_{4}+\indora_{k\leq2}F_{2}\right)+\indora_{i\leq2}D_{i}F_{2}D_{k}\left(\hat{R}_{4}+\hat{F}_{4}\right)\\
D_{k}D_{i}\hat{F}_{4}&=&D_{k}D_{i}\left(\hat{R}_{3}+\indora_{i\leq2}F_{1}+\hat{F}_{3}\right)+D_{i}\hat{R}_{3}D_{k}\left(\indora_{k\leq2}F_{1}+\hat{F}_{3}\right)+D_{i}\hat{F}_{3}D_{k}\left(\hat{R}_{3}+\indora_{k\leq2}F_{1}\right)+\indora_{i\leq2}D_{i}F_{1}D_{k}\left(\hat{R}_{3}+\hat{F}_{3}\right)
\end{eqnarray*}}
para $i,k=1,\ldots,4$. Es necesario determinar las derivadas de segundo orden para las expresiones de la forma $D_{k}D_{i}\left(R_{2}+F_{2}+\indora_{i\geq3}\hat{F}_{4}\right)$

%_________________________________________________________________________________________________________
\subsection{Derivadas de Segundo Orden: Tiempos de Traslado del Servidor}
%_________________________________________________________________________________________________________

A saber, $R_{i}\left(z_{1},z_{2},w_{1},w_{2}\right)=R_{i}\left(P_{1}\left(z_{1}\right)\tilde{P}_{2}\left(z_{2}\right)
\hat{P}_{1}\left(w_{1}\right)\hat{P}_{2}\left(w_{2}\right)\right)$, la denotaremos por la expresi\'on $R_{i}=R_{i}\left(
P_{1}\tilde{P}_{2}\hat{P}_{1}\hat{P}_{2}\right)$, donde al igual que antes, utilizando la notaci\'on dada en \cite{Lang} se tiene   que

\begin{eqnarray}
D_{i}D_{i}R_{k}=D^{2}R_{k}\left(D_{i}P_{i}\right)^{2}+DR_{k}D_{i}D_{i}P_{i}
\end{eqnarray}

mientras que para $i\neq j$

\begin{eqnarray}
D_{i}D_{j}R_{k}=D^{2}R_{k}D_{i}P_{i}D_{j}P_{j}+DR_{k}D_{j}P_{j}D_{i}P_{i}
\end{eqnarray}

%_________________________________________________________________________________________________________
\subsection{Derivadas de Segundo Orden: Longitudes de las Colas}
%_________________________________________________________________________________________________________

Recordemos la expresi\'on $F_{1}\left(\theta_{1}\left(\tilde{P}_{2}\left(z_{2}\right)\hat{P}_{1}\left(w_{1}\right)\hat{P}_{2}\left(w_{2}\right)\right),
z_{2}\right)$, que denotaremos por $F_{1}\left(\theta_{1}\left(\tilde{P}_{2}\hat{P}_{1}\hat{P}_{2}\right),z_{2}\right)$, entonces las derivadas parciales mixtas son:

\begin{eqnarray*}
D_{i}F_{1}=\indora_{i\geq2}D_{i}F_{1}D\theta_{1}D_{i}P_{i}+\indora_{i=2} D_{i}F_{1},
\end{eqnarray*}

entonces para
$F_{1}\left(\theta_{1}\left(\tilde{P}_{2}\hat{P}_{1}\hat{P}_{2}\right),z_{2}\right)$

$$D_{2}F_{1}=D_{1}F_{1}D_{1}\theta_{1}D_{2}\tilde{P}_{2}\left\{\hat{P}_{1}\hat{P}_{2}\right\}+D_{2}F_{1}$$

\begin{eqnarray*}
D_{1}D_{1}F_{1}&=&0\\
D_{2}D_{1}F_{1}&=&0\\
D_{3}D_{1}F_{1}&=&0\\
D_{4}D_{1}F_{1}&=&0\\
D_{1}D_{2}F_{1}&=&0\\
D_{2}D_{2}F_{1}&=&D_{1}D_{1}F_{1}D\theta_{1}D_{2}\tilde{P}_{2}D\theta_{1}D_{2}\tilde{P}_{2}
+D_{1}F_{1}DD\theta_{1}D_{2}D_{2}\tilde{P}_{2}
+D_{1}F_{1}D\theta_{1}D_{2}D_{2}\tilde{P}_{2}
+D_{1}D_{2}F_{1}D\theta_{1}D_{2}\tilde{P}_{2}\\
&+&D_{1}D_{2}F_{1}D\theta_{1}D_{2}\tilde{P}_{2}+D_{2}D_{2}F_{1}\\
D_{3}D_{2}F_{1}&=&D_{1}D_{1}F_{1}D\theta_{1}D_{3}\hat{P}_{1}D\theta_{1}D_{2}\tilde{P}_{2}+D_{1}F_{1}DD\theta_{1}D_{3}\hat{P}_{1}D_{2}\tilde{P}_{2}+D_{1}F_{1}D\theta_{1}D_{2}\tilde{P}_{2}D_{3}\hat{P}_{1}+D_{1}D_{2}F_{1}D\theta_{1}D_{3}\hat{P}_{1}\\
D_{4}D_{2}F_{1}&=&D_{1}D_{1}F_{1}D\theta_{1}D_{4}\hat{P}_{2}D\theta_{1}D_{2}\tilde{P}_{2}+D_{1}F_{1}DD\theta_{1}D_{4}\hat{P}_{2}D_{2}\tilde{P}_{2}+D_{1}F_{1}D\theta_{1}D_{2}\tilde{P}_{2}D_{4}\hat{P}_{2}+D_{1}D_{2}F_{1}D\theta_{1}D_{4}\hat{P}_{2}\\
D_{1}D_{3}F_{1}&=&0\\
D_{2}D_{3}F_{1}&=&
D_{1}D_{1}F_{1}D\theta_{1}D_{2}\tilde{P}_{2}D\theta_{1}D_{3}\hat{P}_{1}+
D_{2}D_{1}F_{1}D\theta_{1}D_{3}\hat{P}_{1}+
D_{1}F_{1}DD\theta_{1}D_{2}\tilde{P}_{2}D_{3}\hat{P}_{1}+
D_{1}F_{1}D\theta_{1}D_{3}\hat{P}_{1}D_{2}\tilde{P}_{2}\\
D_{3}D_{3}F_{1}&=&D_{1}D_{1}F_{1}D\theta_{1}D_{3}\hat{P}_{1}D\theta_{1}D_{3}\hat{P}_{1}+D_{1}F_{1}DD\theta_{1}D_{3}\hat{P}_{1}D_{3}\hat{P}_{1}+D_{1}F_{1}D\theta_{1}D_{3}D_{3}\hat{P}_{1}\\
D_{4}D_{3}F_{1}&=&D_{1}D_{1}F_{1}D\theta_{1}D_{4}\hat{P}_{2}D\theta_{1}D_{3}\hat{P}_{1}+D_{1}F_{1}DD\theta_{1}D_{4}\hat{P}_{2}D_{3}\hat{P}_{1}+D_{1}F_{1}D\theta_{1}D_{3}\hat{P}_{1}D_{4}\hat{P}_{2}\\
D_{1}D_{4}F_{1}&=&0\\
D_{2}D_{4}F_{1}&=&D_{1}D_{1}F_{1}D\theta_{1}D_{2}\tilde{P}_{2}D\theta_{1}D_{4}\hat{P}_{2}+D_{1}F_{1}DD\theta_{1}D_{2}\tilde{P}_{2}D_{4}\hat{P}_{2}+D_{1}F_{1}D\theta_{1}D_{4}\hat{P}_{2}D_{2}\tilde{P}_{2}+D_{2}D_{1}F_{1}D\theta_{1}D_{4}\hat{P}_{2}\\
D_{3}D_{4}F_{1}&=&D_{1}D_{1}F_{1}D\theta_{1}D_{3}\hat{P}_{1}D\theta_{1}D_{4}\hat{P}_{2}+D_{1}F_{1}DD\theta_{1}D_{3}\hat{P}_{1}D_{4}\hat{P}_{2}+D_{1}F_{1}D\theta_{1}D_{4}\hat{P}_{2}D_{3}\hat{P}_{1}\\
D_{4}D_{4}F_{1}&=&D_{1}D_{1}F_{1}D\theta_{1}D_{4}\hat{P}_{2}D\theta_{1}D_{4}\hat{P}_{2}+D_{1}F_{1}DD\theta_{1}D_{4}\hat{P}_{2}D_{4}\hat{P}_{2}+D_{1}F_{1}D\theta_{1}D_{4}D_{4}\hat{P}_{2}
\end{eqnarray*}


%\newpage

Para $F_{2}\left(z_{1},\tilde{\theta}_{2}\left(P_{1}\hat{P}_{1}\hat{P}_{2}\right)\right)$

\begin{eqnarray*}
D_{i}F_{2}=\indora_{i\neq2}D_{2}F_{2}D\tilde{\theta}_{2}D_{i}P_{i}+\indora_{i=1} D_{i}F_{2},
\end{eqnarray*}



\begin{eqnarray*}
D_{1}D_{1}F_{2}&=&\left(D_{2}D_{2}F_{2}D_{1}\tilde{\theta}_{2}D_{1}P_{1}+D_{1}D_{2}F_{2}\right)D_{2}\tilde{\theta}_{2}D_{1}P_{1}+D_{2}F_{2}D_{1}D_{2}\tilde{\theta}_{2}D_{1}P_{1}+D_{2}F_{2}D_{2}\tilde{\theta}_{2}D_{1}D_{1}P_{1}+D_{1}D_{1}F_{2}\\
D_{2}D_{1}F_{2}&=&0\\
D_{3}D_{1}F_{2}&=&D_{2}D_{1}F_{2}D_{3}\tilde{\theta}_{2}D_{3}\hat{P}_{1}+D_{2}D_{2}F_{2}D_{3}\tilde{\theta}_{2}D_{3}P_{1}D_{2}\tilde{\theta}_{2}D_{1}P_{1}+D_{2}F_{2}D_{3}D_{2}\tilde{\theta}_{2}D_{3}\hat{P}_{1}D_{1}P_{1}+D_{2}F_{2}D_{2}\tilde{\theta}_{2}D_{1}P_{1}D_{3}\hat{P}_{1}\\
D_{4}D_{1}F_{2}&=&D_{2}D_{1}F_{2}D_{4}\tilde{\theta}_{2}D_{4}\hat{P}_{2}+D_{2}D_{2}F_{2}D_{4}\tilde{\theta}_{2}D_{4}P_{2}D_{4}\tilde{\theta}_{2}D_{1}P_{1}+D_{2}F_{2}D_{4}D_{2}\tilde{\theta}_{2}D_{4}\hat{P}_{2}D_{1}P_{1}+D_{2}F_{2}D_{2}\tilde{\theta}_{2}D_{1}P_{1}D_{4}\hat{P}_{2}\\
D_{1}D_{3}F_{2}&=&\left(D_{2}D_{2}F_{2}D_{1}\tilde{\theta}_{2}D_{1}P_{1}+D_{1}D_{2}F_{2}\right)D_{3}\tilde{\theta}_{2}D_{3}\hat{P}_{1}+D_{2}F_{2}D_{1}D_{3}\tilde{\theta}_{2}D_{1}P_{1}D_{3}\hat{P}_{1}+D_{2}F_{2}D_{3}\tilde{\theta}_{2}D_{3}\hat{P}_{1}D_{1}P_{1}\\
D_{2}D_{3}F_{3}&=&0\\
D_{3}D_{3}F_{2}&=&D_{2}D_{2}F_{2}D_{3}\tilde{\theta}_{2}D_{3}\hat{P}_{1}D_{3}\tilde{\theta}_{2}D_{3}\hat{P}_{1}+D_{2}F_{2}D_{3}D_{3}\tilde{\theta}_{2}D_{3}\hat{P}_{1}D_{3}\hat{P}_{1}+D_{2}F_{2}D_{3}\tilde{\theta}_{2}D_{3}D_{3}\hat{P}_{1}\\
D_{4}D_{3}F_{2}&=&D_{2}D_{2}F_{2}D_{4}\tilde{\theta}_{2}D_{4}\hat{P}_{2}D_{3}\tilde{\theta}_{2}D_{3}\hat{P}_{1}+D_{2}F_{2}D_{4}D_{3}\tilde{\theta}_{2}D_{4}\hat{P}_{2}D_{3}\hat{P}_{1}+D_{2}F_{2}D_{3}\tilde{\theta}_{2}D_{3}\hat{P}_{1}D_{4}\hat{P}_{2}\\
D_{1}D_{4}F_{2}&=&\left(D_{2}D_{2}F_{2}D_{4}\tilde{\theta}_{2}D_{1}P_{1}+D_{1}D_{2}F_{2}\right)D_{4}\tilde{\theta}_{2}D_{4}\hat{P}_{2}+D_{2}F_{2}D_{1}D_{4}\tilde{\theta}_{2}D_{1}P_{1}D_{4}\hat{P}_{2}+D_{2}F_{2}D_{4}\tilde{\theta}_{2}D_{4}\hat{P}_{2}D_{1}P_{1}\\
D_{2}D_{4}F_{2}&=&0\\
D_{3}D_{4}F_{2}&=&D_{2}F_{2}D_{4}\tilde{\theta}_{2}D_{4}\hat{P}_{2}D_{3}\hat{P}_{1}+D_{2}F_{2}D_{3}D_{4}\tilde{\theta}_{2}D_{4}\hat{P}_{2}D_{3}\hat{P}_{1}+D_{2}F_{2}D_{4}\tilde{\theta}_{2}D_{4}\hat{P}_{2}D_{3}\hat{P}_{1}\\
D_{4}D_{4}F_{2}&=&D_{2}F_{2}D_{4}\tilde{\theta}_{2}D_{4}D_{4}\hat{P}_{2}+D_{2}F_{2}D_{4}D_{4}\tilde{\theta}_{2}D_{4}\hat{P}_{2}D_{4}\hat{P}_{2}+D_{2}F_{2}D_{4}\tilde{\theta}_{2}D_{4}\hat{P}_{2}D_{4}\hat{P}_{2}\\
\end{eqnarray*}


%\newpage



%\newpage

para $\hat{F}_{1}\left(\hat{\theta}_{1}\left(P_{1}\tilde{P}_{2}\hat{P}_{2}\right),w_{2}\right)$

\begin{eqnarray*}
D_{i}\hat{F}_{1}=\indora_{i\neq3}D_{3}\hat{F}_{1}D\hat{\theta}_{1}D_{i}P_{i}+\indora_{i=4}D_{i}\hat{F}_{1},
\end{eqnarray*}


\begin{eqnarray*}
D_{1}D_{1}\hat{F}_{1}&=&D_{1}\hat{\theta}_{1}D_{1}D_{1}P_{1}D_{1}\hat{F}_{1}
+D_{1}P_{1}D_{1}P_{1}D_{1}D_{1}\hat{\theta}_{1}D_{1}\hat{F}_{1}+
D_{1}P_{1}D_{1}P_{1}D_{1}\hat{\theta}_{1}D_{1}\hat{\theta}_{1}
D_{1}D_{1}\hat{F}_{1}\\
D_{1}D_{1}\hat{F}_{1}&=&D_{1}P_{1}D_{2}P_{1}D\hat{\theta}_{1}D_{1}\hat{F}_{1}+
D_{1}P_{1}D_{2}P_{1}DD\hat{\theta}_{1}D_{1}\hat{F}_{1}+
D_{1}P_{1}D_{2}P_{1}D\hat{\theta}_{1}D\hat{\theta}_{1}D_{1}D_{1}\hat{\theta}_{1}\\
D_{3}D_{1}\hat{F}_{1}&=&0\\
D_{4}D_{1}\hat{F}_{1}&=&D_{1}P_{1}D_{4}\hat{P}_{2}D\hat{\theta}_{1}D_{1}\hat{F}_{1}
+D_{1}D_{4}\hat{P}_{2}DD\hat{\theta}_{1}D_{1}\hat{F}_{1}
+D_{1}D\hat{\theta}_{1}\left(D_{2}D{1}\hat{F}_{1}
+D_{4}P_{2}D\hat{\theta}_{1}D_{1}D_{1}\hat{F}_{1}\right)\\
D_{1}D_{2}\hat{F}_{1}&=&D_{1}P_{1}D_{2}P_{2}D\hat{\theta}_{1}D_{1}\hat{F}_{1}+
D_{1}P_{1}D_{2}P_{2}DD\hat{\theta}_{1}D_{1}\hat{F}_{1}+
D_{1}P_{1}D_{2}P_{2}D\hat{\theta}_{1}D\hat{\theta}_{1}D_{1}D_{1}\hat{F}_{1}\\
D_{2}D_{2}\hat{F}_{1}&=&D\hat{\theta}_{1}D_{2}D_{2}P_{2}D_{1}\hat{F}_{1}+ D_{2}P_{2}D_{2}P_{2}DD\hat{\theta}_{1}D_{1}\hat{F}_{1}+
D_{2}P_{2}D_{2}P_{2}D\hat{\theta}_{1}D\hat{\theta}_{1}
D_{1}D_{1}\hat{F}_{1}\\
D_{3}D_{2}\hat{F}_{1}&=&0\\
D_{4}D_{2}\hat{F}_{1}&=&D_{2}P_{2}D_{4}\hat{P}_{2}D\hat{\theta} _{1}D\hat{F}_{1}+D_{2}P_{2}D_{4}\hat{P}_{2}DD\hat{\theta}_{1}D_{1}\hat{F}_{1} +D_{2}P_{2}D\hat{\theta}_{1}\left(D_{2}D_{1}\hat{F}_{1}+ D_{2}\hat{P}_{2}D\hat{\theta}_{1}D_{1}D_{1}\hat{F}_{1}\right)\\
D_{1}D_{3}\hat{F}_{1}&=&0\\
D_{2}D_{3}\hat{F}_{1}&=&0\\
D_{3}D_{3}\hat{F}_{1}&=&0\\
D_{4}D_{3}\hat{F}_{1}&=&0\\
D_{1}D_{4}\hat{F}_{1}&=&D_{1}P_{1}D_{4}\hat{F}_{2}D\hat{\theta}_{1}D_{1}
\hat{F}_{1}+D_{1}P_{1}D_{4}\hat{P}_{2}DD\hat{\theta}_{1}D_{1}\hat{F}_{1}+D_{1}P_{1}D\hat{\theta}_{1}D_{2}D_{1}\hat{F}_{1}+ D_{1}P_{1}D_{4}\hat{P}_{2}D\hat{\theta}_{1}D\hat{\theta}_{1}D_{1}D_{1}
\hat{F}_{1}\\
D_{2}D_{4}\hat{F}_{1}&=&D_{2}P_{2}D_{4}\hat{P}_{2}D\hat{\theta}_{1}D_{1}
\hat{F}_{1}+D_{2}P_{2}D_{4}\hat{P}_{2}DD\hat{\theta}_{1}D_{1}\hat{F}_{1}+D_{2}P_{2}D\hat{\theta}_{1}D_{2}D_{1}\hat{F}_{1}+
D_{2}P_{2}D_{4}\hat{P}_{2}D\hat{\theta}_{1}D\hat{\theta}_{1}D_{1}D_{1}\hat{F}_{1}\\
D_{3}D_{4}\hat{F}_{1}&=&0\\
D_{4}D_{4}\hat{F}_{1}&=&D_{2}D_{2}\hat{F}_{1}+D\hat{\theta}_{1}D_{4}D_{4}\hat{F}_{2}+ D_{1}\hat{F}_{1}+
D_{4}\hat{P}_{2}D_{4}\hat{P}_{2}DD\hat{\theta}_{1}D_{1}\hat{F}_{1}+
D_{4}\hat{P}_{2}D\hat{\theta}_{1}D_{2}D_{1}\hat{F}_{1}\\
&+&D_{4}\hat{P}_{2}D\hat{\theta}_{1}\left(D_{2}D_{1}\hat{F}_{1}+ D_{4}\hat{P}_{2}D\hat{\theta}_{1}D_{1}D_{1}\hat{F}_{1}\right)\\
\end{eqnarray*}




%\newpage
finalmente, para $\hat{F}_{2}\left(w_{1},\hat{\theta}_{2}\left(P_{1}\tilde{P}_{2}\hat{P}_{1}\right)\right)$

\begin{eqnarray*}
D_{i}\hat{F}_{2}=\indora_{i\neq4}D_{4}\hat{F}_{2}D\hat{\theta}_{2}D_{i}P_{i}+\indora_{i=3}D_{i}\hat{F}_{2},
\end{eqnarray*}

\begin{eqnarray*}
D_{1}D_{1}\hat{F}_{2}&=&D_{1}\hat{\theta}_{2}D_{2}D_{2}P_{1}D_{2}
\hat{F}_{2}
+D_{1}P_{1}D_{1}P_{1}D_{1}D_{1}\hat{\theta}_{2}D_{2}\hat{F}_{2}+
D_{1}P_{1}D_{1}P_{1}D_{1}\hat{\theta}_{2}D_{1}\hat{\theta}_{2}
D_{1}D_{1}\hat{F}_{2}\\
D_{2}D_{1}\hat{F}_{2}&=&D_{1}P_{1}D_{2}P_{2}D\hat{\theta}_{2}D_{2}
\hat{F}_{2}+
D_{1}P_{1}D_{2}P_{2}DD\hat{\theta}_{2}D_{2}\hat{F}_{2}+
D_{1}P_{1}D_{2}P_{2}D\hat{\theta}_{2}D\hat{\theta}_{2}D_{2}
D_{2}\hat{\theta}_{2}\\
D_{3}D_{1}\hat{F}_{2}&=&D_{1}P_{1}D_{3}\hat{P}_{1}D\hat{\theta}_{2}
D_{2}\hat{F}_{2}
+D_{1}P_{1}D_{3}\hat{P}_{1}DD\hat{\theta}_{2}D_{2}\hat{F}_{2}
+D_{1}P_{1}D\hat{\theta}_{2}\left(D_{2}D{1}\hat{F}_{2}
+D_{3}\hat{P}_{1}D\hat{\theta}_{2}D_{2}D_{2}\hat{F}_{2}\right)\\
D_{4}D_{1}\hat{F}_{2}&=&0\\
D_{1}D_{2}\hat{F}_{2}&=&D_{1}P_{1}D_{2}P_{2}D\hat{\theta}_{2}D_{2}\hat{F}_{2}+
D_{1}P_{1}D_{2}P_{2}DD\hat{\theta}_{2}D_{2}\hat{F}_{2}+
D_{1}P_{1}D_{2}P_{2}D\hat{\theta}_{2}D\hat{\theta}_{2}D_{2}D_{2}\hat{F}_{2}\\
D_{2}D_{2}\hat{F}_{2}&=&DD\hat{\theta}_{2}D_{2}D_{2}P_{2}D_{2}\hat{F}_{2}+ D_{2}P_{2}D_{2}P_{2}DD\hat{\theta}_{2}D_{2}\hat{F}_{2}+
D_{2}P_{2}D_{2}P_{2}D\hat{\theta}_{2}D\hat{\theta}_{2} D_{2}D_{2}\hat{F}_{2}\\
D_{3}D_{2}\hat{F}_{2}&=&D_{2}P_{2}D_{3}\hat{P}_{1}D\hat{\theta} _{2}D_{2}\hat{F}_{2}+D_{2}P_{2}D_{3}\hat{P}_{1}DD\hat{\theta}_{2}
D_{2}\hat{F}_{2}
+D_{2}P_{2}D\hat{\theta}_{2}\left(D_{2}D_{1}\hat{F}_{2}+ D_{3}\hat{P}_{1}D\hat{\theta}_{2}D_{2}D_{2}\hat{F}_{2}\right)\\
D_{4}D_{2}\hat{F}_{2}&=&0\\
D_{1}D_{3}\hat{F}_{2}&=&
D_{1}P_{1}D_{3}\hat{P}_{1}D\hat{\theta}_{2}D_{2}\hat{F}_{2}
+D_{1}P_{1}D_{3}\hat{P}_{1}DD\hat{\theta}_{2}D_{2}\hat{F}_{2}
+D_{1}P_{1}D\hat{\theta}_{2}D\hat{\theta}_{2}D_{2}D_{2}\hat{F}_{2}
+D_{1}P_{1}D\hat{\theta}_{1}D_{2}D_{1}\hat{F}_{2}\\
D_{2}D_{3}\hat{F}_{2}&=&
D_{2}P_{2}D_{3}\hat{P}_{1}D\hat{\theta}_{2}D_{2}\hat{F}_{2}
+D_{2}P_{2}D_{3}\hat{P}_{1}DD\hat{\theta}_{2}D_{2}\hat{F}_{2}
+D_{2}P_{2}D_{3}\hat{P}_{1}D\hat{\theta}_{2}D_{2}D_{2}\hat{F}_{2}
+D_{2}P_{2}D\hat{\theta}_{2}D\hat{\theta}_{2}D_{1}D_{2}\hat{F}_{2}\\
D_{4}D_{3}\hat{F}_{2}&=&
D_{3}D_{3}\hat{P}_{1}D\hat{\theta}_{2}D_{2}\hat{F}_{2}
+D_{3}\hat{P}_{1}D_{3}\hat{P}_{1}DD\hat{\theta}_{2}D_{2}\hat{F}_{2}
+D_{3}\hat{P}_{1}D\hat{\theta}_{2}D_{1}D_{2}\hat{F}_{2}
+D_{3}\hat{P}_{1}D\hat{\theta}_{2}\left(D_{3}\hat{P}_{1}D\hat{\theta}_{2}
D_{2}D_{2}\hat{F}_{2}+D_{1}D_{2}\hat{F}_{2}\right)\\
D_{4}D_{3}\hat{F}_{2}&=&0\\
D_{1}D_{4}\hat{F}_{2}&=&0\\
D_{2}D_{4}\hat{F}_{2}&=&0\\
D_{3}D_{4}\hat{F}_{2}&=&0\\
D_{4}D_{4}\hat{F}_{2}&=&0\\
\end{eqnarray*}

%__________________________________________________________________
\section{Aplicaciones}
%__________________________________________________________________

%__________________________________________________________________
\subsection{Ejemplo 1: Automatizaci\'on en dos l\'ineas de trabajo}
%__________________________________________________________________
Consideremos dos l\'ineas de producci\'on atendidas cada una de ellas por un robot, en las que en una de ellas un robot realiza la misma actividad en dos estaciones distintas, una vez que termina de realizar una actividad en una de las colas, se desplaza a la siguiente para hacer lo correspondientes con los materiales presentes en la estaci\'on. Una vez que las piezas son liberadas por el robot se desplazan al otro sistema en donde son objeto del terminado de la pieza para su almacenamiento. En este caso el sistema 1 consta de una sola cola de tipo $M/M/1$ y el sistema 2 es un sistema de visitas c\'iclicas conformado por dos colas id\'enticas, donde al igual que antes, el traslado de un sistema a otro se realiza de la cola $\hat{Q}_{2}$ a la \'unica cola $Q_{1}$ del sistema 1.

%\begin{figure}[H]
%\centering
%%%\includegraphics[width=9cm]{Grafica1.jpg}
%%\end{figure}\label{RSVC1}



El n\'umero de usuarios presentes en el sistema 1 se sigue modelando conforme a un SVC, mientras que para es sistema 1, $Q_{1}$ se comporta como una Red de Jackson, una red conformada por $\hat{Q}_{2}$ y $Q_{1}$, donde el n\'umero de usuarios que llegan a $Q_{1}$ lo hacen de acuerdo a su propio proceso de arribos m\'as los que provienen del sistema 2, los tiempos entre arribos de los usuarios procedentes del sistema 2, lo hacen conforme a una distribuci\'on exponencial.

Las ecuaciones recursivas son


\begin{eqnarray*}
F_{1}\left(z_{1},w_{1},w_{2}\right)&=&R\left(\tilde{P}_{2}\left(z_{2}\right)\prod_{i=1}^{2}
\hat{P}_{i}\left(w_{i}\right)\right)F_{2}\left(\tilde{\theta}_{2}\left(\hat{P}_{1}\left(w_{1}\right)\hat{P}_{2}\left(w_{2}\right)\right)\right)
\hat{F}_{2}\left(w_{1},w_{2};\tau_{2}\right),
\end{eqnarray*}

\begin{eqnarray*}
\hat{F}_{1}\left(z_{1},w_{1},w_{2}\right)&=&\hat{R}_{2}\left(\tilde{P}_{2}\left(z_{2}\right)\prod_{i=1}^{2}
\hat{P}_{i}\left(w_{i}\right)\right)F_{2}\left(z_{1};\zeta_{2}\right)\hat{F}_{2}\left(w_{1},\hat{\theta}_{2}\left(\tilde{P}_{2}\left(z_{2}\right)\hat{P}_{1}\left(w_{1}
\right)\right)\right),
\end{eqnarray*}


\begin{eqnarray*}
\hat{F}_{2}\left(z_{1},w_{1},w_{2}\right)&=&\hat{R}_{1}\left(\tilde{P}_{2}\left(z_{2}\right)\prod_{i=1}^{2}
\hat{P}_{i}\left(w_{i}\right)\right)F_{1}\left(z_{1};\zeta_{1}\right)\hat{F}_{1}\left(\hat{\theta}_{1}\left(\tilde{P}_{2}\left(z_{2}\right)\hat{P}_{2}\left(w_{2}\right)\right),w_{2}\right),
\end{eqnarray*}




%__________________________________________________________________
\subsection{Ejemplo 2: Sistema de Salud P\'ublica}
%__________________________________________________________________

Consideremos un hospital en el \'area de urgencias, donde hay una ventanilla a la cu\'al van llegando todos los posibles pacientes para su valoraci\'on, despu\'es de la cual pueden o ser canalizados a un \'area de atenci\'on que requiera de atenci\'on sin llegar a ser urgencia, o puede abandonar el sistema dependiendo de la valoraci\'on hecha por el m\'edico en turno. Por otra parte, hay una secci\'on del hospital en la que son atendidas las personas sin necesidad de pasar por la ventanilla de valoraci\'on, es decir, son atenciones de urgencia. Las personas que despu\'es de la valoraci\'on son turnadas al \'area de atenci\'on deben de esperar su turno pues a esta secci\'on tambi\'en llegan pacientes provenientes de otras \'areas del hospital. Para este caso, el sistema 1 est\'a conformado por una \'unica cola $Q_{1}$ que podemos asumir sin p\'erdida de generalidad que es de tipo $M/M/1$, mientras que el sistema 2 es un SVC como los hasta ahora estudiados. Es decir, en este caso en particular el servidor del sistema 1 da servicio de manera ininterrumpida en $Q_{1}$ en tanto no se vac\'ie la cola.




%\begin{figure}[H]
%\centering
%%%\includegraphics[width=9cm]{Grafica2.jpg}
%%\end{figure}\label{RSVC2}

Las ecuaciones recursivas son de la forma


\begin{eqnarray*}
F_{1}\left(z_{1},z_{2},w_{1}\right)&=&R_{2}\left(P_{1}\left(z_{1}\right)\tilde{P}_{2}\left(z_{2}\right)
\hat{P}_{1}\left(w_{1}\right)\right)F_{2}\left(z_{1},\tilde{\theta}_{2}\left(P_{1}\left(z_{1}\right)\hat{P}_{1}\left(w_{1}\right)\right)\right)
\hat{F}_{2}\left(w_{1};\tau_{2}\right),
\end{eqnarray*}


\begin{eqnarray*}
F_{2}\left(z_{1},z_{2},w_{1}\right)&=&R_{1}\left(P_{1}\left(z_{1}\right)\tilde{P}_{2}\left(z_{2}\right)
\hat{P}_{1}\left(w_{1}\right)\right)F_{1}\left(\theta_{1}\left(\hat{P}_{1}\left(w_{1}\right)\hat{P}_{2}\left(w_{2}\right)\right),z_{2}\right)\hat{F}_{1}\left(w_{1};\tau_{1}\right),
\end{eqnarray*}



\begin{eqnarray*}
\hat{F}_{1}\left(z_{1},z_{2},w_{1}\right)&=&\hat{R}_{2}\left(P_{1}\left(z_{1}\right)\tilde{P}_{2}\left(z_{2}\right)
\hat{P}_{1}\left(w_{1}\right)\right)F_{2}\left(z_{1},z_{2};\zeta_{2}\right)\hat{F}_{}\left(\hat{\theta}_{1}\left(P_{1}\left(z_{1}\right)\tilde{P}_{2}\left(z_{2}\right)
\right)\right),
\end{eqnarray*}


%__________________________________________________________________
\subsection{Ejemplo 3: RSVC con dos conexiones}
%__________________________________________________________________

Al igual que antes consideremos una RSVC conformada por dos SVC que se comunican entre s\'i en $\hat{Q}_{2}$ y $Q_{2}$, permitiendo el paso de los usuarios del sistema 2 hacia el sistema 1. Ahora supongamos que tambi\'en se permite el paso en $\hat{Q}_{1}$ hacia $Q_{1}$.

%\begin{figure}[H]
%\centering
%%%\includegraphics[width=9cm]{Grafica3.jpg}
%%\end{figure}\label{RSVC3}


Cuyas ecuaciones recursivas son de la forma


\begin{eqnarray*}
F_{1}\left(z_{1},z_{2},w_{1},w_{2}\right)&=&R_{2}\left(\tilde{P}_{1}\left(z_{1}\right)\tilde{P}_{2}\left(z_{2}\right)\prod_{i=1}^{2}
\hat{P}_{i}\left(w_{i}\right)\right)F_{2}\left(z_{1},\tilde{\theta}_{2}\left(\tilde{P}_{1}\left(z_{1}\right)\hat{P}_{1}\left(w_{1}\right)\hat{P}_{2}\left(w_{2}\right)\right)\right)
\hat{F}_{2}\left(w_{1},w_{2};\tau_{2}\right),
\end{eqnarray*}

\begin{eqnarray*}
F_{2}\left(z_{1},z_{2},w_{1},w_{2}\right)&=&R_{1}\left(\tilde{P}_{1}\left(z_{1}\right)\tilde{P}_{2}\left(z_{2}\right)\prod_{i=1}^{2}
\hat{P}_{i}\left(w_{i}\right)\right)F_{1}\left(\tilde{\theta}_{1}\left(\tilde{P}_{2}\left(z_{2}\right)\hat{P}_{1}\left(w_{1}\right)\hat{P}_{2}\left(w_{2}\right)\right),z_{2}\right)\hat{F}_{1}\left(w_{1},w_{2};\tau_{1}\right),
\end{eqnarray*}


\begin{eqnarray*}
\hat{F}_{1}\left(z_{1},z_{2},w_{1},w_{2}\right)&=&\hat{R}_{2}\left(\tilde{P}_{1}\left(z_{1}\right)\tilde{P}_{2}\left(z_{2}\right)\prod_{i=1}^{2}
\hat{P}_{i}\left(w_{i}\right)\right)F_{2}\left(z_{1},z_{2};\zeta_{2}\right)\hat{F}_{2}\left(w_{1},\hat{\theta}_{2}\left(\tilde{P}_{1}\left(z_{1}\right)\tilde{P}_{2}\left(z_{2}\right)\hat{P}_{1}\left(w_{1}
\right)\right)\right),
\end{eqnarray*}


\begin{eqnarray*}
\hat{F}_{2}\left(z_{1},z_{2},w_{1},w_{2}\right)&=&\hat{R}_{1}\left(\tilde{P}_{1}\left(z_{1}\right)\tilde{P}_{2}\left(z_{2}\right)\prod_{i=1}^{2}
\hat{P}_{i}\left(w_{i}\right)\right)F_{1}\left(z_{1},z_{2};\zeta_{1}\right)\hat{F}_{1}\left(\hat{\theta}_{1}\left(\tilde{P}_{1}\left(z_{1}\right)\tilde{P}_{2}\left(z_{2}\right)\hat{P}_{2}\left(w_{2}\right)\right),w_{2}\right),
\end{eqnarray*}



\section*{Objetivos Principales}

\begin{itemize}
%\item Generalizar los principales resultados existentes para Sistemas de Visitas C\'iclicas para el caso en el que se tienen dos Sistemas de Visitas C\'iclicas con propiedades similares.

\item Encontrar las ecuaciones que modelan el comportamiento de una RSVC con propiedades similares.

\item Encontrar expresiones anal\'iticas para las longitudes de las colas al momento en que el servidor llega a una de ellas para comenzar a dar servicio, as\'i como de sus segundos momentos.

\item Determinar las principales medidas de desempe\~no para la RSVC tales como: N\'umero de usuarios presentes en cada una de las colas del sistema cuando uno de los servidores est\'a presente atendiendo, Tiempos que transcurre entre las visitas del servidor a la misma cola.


\end{itemize}


%_________________________________________________________________________
%\section{Sistemas de Visitas C\'iclicas}
%_________________________________________________________________________
\numberwithin{equation}{section}%
%__________________________________________________________________________




%\section*{Introducci\'on}




%__________________________________________________________________________
%\subsection{Definiciones}
%__________________________________________________________________________


\section{Descripci\'on de una Red de Sistemas de Visitas C\'iclicas}

Consideremos una red de sistema de visitas c\'iclicas conformada por dos sistemas de visitas c\'iclicas, cada una con dos colas independientes, donde adem\'as se permite el intercambio de usuarios entre los dos sistemas en la segunda cola de cada uno de ellos.

%____________________________________________________________________
\subsection*{Supuestos sobre la Red de Sistemas de Visitas C\'iclicas}
%____________________________________________________________________

\begin{itemize}
\item Los arribos de los usuarios ocurren conforme a un proceso de conteo general con tasa de llegada $\mu_{1}$ y $\mu_{2}$ para el sistema 1, mientras que para el sistema 2, lo hacen conforme a un proceso Poisson con tasa $\hat{\mu}_{1},\hat{\mu}_{2}$ respectivamente.



\item Se considerar\'an intervalos de tiempo de la forma
$\left[t,t+1\right]$. Los usuarios arriban de manera independiente del resto de las colas. Se define el grupo de
usuarios que llegan a cada una de las colas del sistema 1,
caracterizadas por $Q_{1}$ y $Q_{2}$ respectivamente, en el
intervalo de tiempo $\left[t,t+1\right]$ por
$X_{1}\left(t\right),X_{2}\left(t\right)$.


\item Se definen los procesos
$\hat{X}_{1}\left(t\right),\hat{X}_{2}\left(t\right)$ para las
colas del sistema 2, denotadas por $\hat{Q}_{1}$ y $\hat{Q}_{2}$
respectivamente. Donde adem\'as se supone que $\mu_{i}<1$ y $\hat{\mu}_{i}<1$ para $i=1,2$.


\item Se define el proceso $Y_{2}\left(t\right)$ para el n\'umero de usuarios que se trasladan del sistema 2 al sistema 1 en el intervalo de tiempo $\left[t,t+1\right]$, este proceso tiene par\'ametro $\check{\mu}_{2}$.% El traslado de un sistema a otro ocurre de manera tal que el proceso de llegadas a $Q_{2}$ es un proceso Poisson con par\'ametro $\tilde{\mu}_{2}=\mu_{2}+\check{\mu}_{2}<1$.


\item En lo que respecta al servidor, en t\'erminos de los tiempos de
visita a cada una de las colas, se definen las variables
aleatorias $\tau_{i},$ para $Q_{i}$, para $i=1,2$, respectivamente;
y $\zeta_{i}$ para $\hat{Q}_{i}$,  $i=1,2$,  del sistema
2 respectivamente. A los tiempos en que el servidor termina de atender en las colas $Q_{i},\hat{Q}_{i}$, se les denotar\'a por
$\overline{\tau}_{i},\overline{\zeta}_{i}$ para  $i=1,2$,
respectivamente.

\item Los tiempos de traslado del servidor desde el momento en que termina de atender a una cola y llega a la siguiente para comenzar a dar servicio est\'an dados por
$\tau_{i+1}-\overline{\tau}_{i}$ y
$\zeta_{i+1}-\overline{\zeta}_{i}$,  $i=1,2$, para el sistema 1 y el sistema 2, respectivamente.

\end{itemize}




%\begin{figure}[H]
%\centering
%%%\includegraphics[width=5cm]{RedSistemasVisitasCiclicas.jpg}
%%\end{figure}\label{RSVC}

El uso de la FGP nos permite determinar las funciones de distribuci\'on de probabilidades conjunta de manera indirecta, sin necesidad de hacer uso de las propiedades de las distribuciones de probabilidad de cada uno de los procesos que intervienen en la RSVC. Para cada una de las colas en cada sistema, el n\'umero de usuarios al tiempo en que llega el servidor a dar servicio est\'a
dado por el n\'umero de usuarios presentes en la cola al tiempo
$t$, m\'as el n\'umero de usuarios que llegan a la cola en el intervalo de tiempo $\left[\tau_{i},\overline{\tau}_{i}\right]$. Una vez definidas las FGP's conjuntas, se construyen las ecuaciones recursivas que permiten obtener la informaci\'on sobre la longitud de cada una de las colas al momento en que uno de los servidores llega a una de ellas para dar servicio.\smallskip

%__________________________________________________________________________
\subsection{Funciones Generadoras de Probabilidades}
%__________________________________________________________________________


Para cada uno de los procesos de llegada a las colas $X_{i},\hat{X}_{i}$,  $i=1,2$,  y $Y_{2}$, con $\tilde{X}_{2}=X_{2}+Y_{2}$ se define FGP: $P_{i}\left(z_{i}\right)=\esp\left[z_{i}^{X_{i}\left(t\right)}\right],\hat{P}_{i}\left(w_{i}\right)=\esp\left[w_{i}^{\hat{X}_{i}\left(t\right)}\right]$, para
$i=1,2$, y $\check{P}_{2}\left(z_{2}\right)=\esp\left[z_{2}^{Y_{2}\left(t\right)}\right], \tilde{P}_{2}\left(z_{2}\right)=\esp\left[z_{2}^{\tilde{X}_{2}\left(t\right)}\right]$ , con primer momento definidos por $\mu_{i}=\esp\left[X_{i}\left(t\right)\right]=P_{i}^{(1)}\left(1\right), \hat{\mu}_{i}=\esp\left[\hat{X}_{i}\left(t\right)\right]=\hat{P}_{i}^{(1)}\left(1\right)$, para $i=1,2$, y por otra parte
$\check{\mu}_{2}=\esp\left[Y_{2}\left(t\right)\right]=\check{P}_{2}^{(1)}\left(1\right),\tilde{\mu}_{2}=\esp\left[\tilde{X}_{2}\left(t\right)\right]=\tilde{P}_{2}^{(1)}\left(1\right)$.

Sus procesos se definen por: $S_{i}\left(z_{i}\right)=\esp\left[z_{i}^{\overline{\tau}_{i}-\tau_{i}}\right]$ y $\hat{S}_{i}\left(w_{i}\right)=\esp\left[w_{i}^{\overline{\zeta}_{i}-\zeta_{i}}\right]$, con primer momento dado por: $s_{i}=\esp\left[\overline{\tau}_{i}-\tau_{i}\right]$ y $\hat{s}_{i}=\esp\left[\overline{\zeta}_{i}-\zeta_{i}\right]$, para $i=1,2$. An\'alogamente los tiempos de traslado del servidor desde el momento en que termina de atender a una cola y llega a la
siguiente para comenzar a dar servicio est\'an dados por
$\tau_{i+1}-\overline{\tau}_{i}$ y
$\zeta_{i+1}-\overline{\zeta}_{i}$ para el sistema 1 y el sistema 2, respectivamente, con $i=1,2$.

La FGP para estos tiempos de traslado est\'an dados por $R_{i}\left(z_{i}\right)=\esp\left[z_{1}^{\tau_{i+1}-\overline{\tau}_{i}}\right]$ y $\hat{R}_{i}\left(w_{i}\right)=\esp\left[w_{i}^{\zeta_{i+1}-\overline{\zeta}_{i}}\right]$ y al igual que como se hizo con anterioridad, se tienen los primeros momentos de estos procesos de traslado del servidor entre las colas de cada uno de los sistemas que conforman la red de sistemas de visitas c\'iclicas: $r_{i}=R_{i}^{(1)}\left(1\right)=\esp\left[\tau_{i+1}-\overline{\tau}_{i}\right]$ y $\hat{r}_{i}=\hat{R}_{i}^{(1)}\left(1\right)=\esp\left[\zeta_{i+1}-\overline{\zeta}_{i}\right]$ para $i=1,2$.

Para el proceso $L_{i}\left(t\right)$ que determina el n\'umero de usuarios presentes en cada una de las colas al tiempo $t$, se define su FGP, $H_{i}\left(t\right)$, correspondiente al sistema 1,  mientras que para el segundo sistema el proceso correspondiente est\'a dado por $\hat{L}_{i}\left(t\right)$, con FGP $\hat{H}_{i}\left(t\right)$, es decir $H_{i}\left(t\right)=\esp\left[z_{i}^{L_{i}\left(t\right)}\right]$ y $\hat{H}_{i}\left(t\right)=\esp\left[w_{i}^{\hat{L}_{i}\left(t\right)}\right]$ para el sistema 1 y 2 respectivamente. Con lo dicho hasta ahora, se tiene que el n\'umero de usuarios
presentes en los tiempos $\overline{\tau}_{1},\overline{\tau}_{2},
\overline{\zeta}_{1},\overline{\zeta}_{2}$, es cero, es decir,
 $L_{i}\left(\overline{\tau_{i}}\right)=0,$ y
$\hat{L}_{i}\left(\overline{\zeta_{i}}\right)=0$ para i=1,2 para
cada uno de los dos sistemas.

Para cada una de las colas en la RSVC, el n\'umero de
usuarios al tiempo en que llega el servidor a una de ellas est\'a
dado por el n\'umero de usuarios presentes en la cola al tiempo
$t=\tau_{i},\zeta_{i}$, m\'as el n\'umero de usuarios que llegan a
la cola en el intervalo de tiempo
$\left[\tau_{i},\overline{\tau}_{i}\right],\left[\zeta_{i},\overline{\zeta}_{i}\right]$,
es decir $\hat{L}_{i}\left(\overline{\tau}_{j}\right)=\hat{L}_{i}\left(\tau_{j}\right)+\hat{X}_{i}\left(\overline{\tau}_{j}-\tau_{j}\right)$, para $i,j=1,2$, mientras que para el primer sistema: $L_{1}\left(\overline{\tau}_{j}\right)=L_{1}\left(\tau_{j}\right)+X_{1}\left(\overline{\tau}_{j}-\tau_{j}\right)$.

En el caso espec\'ifico de $Q_{2}$, adem\'as, hay que considerar
el n\'umero de usuarios que pasan del sistema 2 al sistema 1, a
traves de $\hat{Q}_{2}$ mientras el servidor en $Q_{2}$ est\'a
ausente, es decir, una vez que son atendidos en $\hat{Q}_{2}$:

\begin{equation}\label{Eq.UsuariosTotalesZ2}
L_{2}\left(\overline{\tau}_{1}\right)=L_{2}\left(\tau_{1}\right)+X_{2}\left(\overline{\tau}_{1}-\tau_{1}\right)+Y_{2}\left(\overline{\tau}_{1}-\tau_{1}\right).
\end{equation}

%_________________________________________________________________________
\subsection{El problema de la ruina del jugador}
%_________________________________________________________________________

Sea $\tilde{L}_{0}$ el n\'umero de usuarios presentes en la cola al momento en que el servidor llega para dar servicio. Sea $T$ el tiempo que requiere el servidor para atender a todos los usuarios presentes en la cola comenzando con $\tilde{L}_{0}$ usuarios. Supongamos que se tiene un jugador que cuenta con un capital inicial de $\tilde{L}_{0}\geq0$ unidades, esta persona realiza una
serie de dos juegos simult\'aneos e independientes de manera sucesiva, dichos eventos son independientes e id\'enticos entre
s\'i para cada realizaci\'on. La ganancia en el $n$-\'esimo juego es $\tilde{X}_{n}=X_{n}+Y_{n}$ unidades de las cuales se resta una cuota de 1 unidad por cada juego simult\'aneo, es decir, se restan dos unidades por cada juego realizado. En el contexto de teor\'ia de colas este proceso se puede pensar como el n\'umero de usuarios que llegan a una cola v\'ia dos procesos de arribo distintos e independientes entre s\'i. Su FGP est\'a dada por $F\left(z\right)=\esp\left[z^{\tilde{L}_{0}}\right]$, adem\'as
$$\tilde{P}\left(z\right)=\esp\left[z^{\tilde{X}_{n}}\right]=\esp\left[z^{X_{n}+Y_{n}}\right]=\esp\left[z^{X_{n}}z^{Y_{n}}\right]=\esp\left[z^{X_{n}}\right]\esp\left[z^{Y_{n}}\right]=P\left(z\right)\check{P}\left(z\right),$$

con $\tilde{\mu}=\esp\left[\tilde{X}_{n}\right]=\tilde{P}\left[z\right]<1$. Sea $\tilde{L}_{n}$ el capital remanente despu\'es del $n$-\'esimo
juego. Entonces

$$\tilde{L}_{n}=\tilde{L}_{0}+\tilde{X}_{1}+\tilde{X}_{2}+\cdots+\tilde{X}_{n}-2n.$$

La ruina del jugador ocurre despu\'es del $n$-\'esimo juego, es decir, la cola se vac\'ia despu\'es del $n$-\'esimo juego. Sea $g_{n,k}$ la probabilidad del evento de que el jugador no caiga en ruina antes del $n$-\'esimo juego, y que adem\'as tenga un capital de $k$ unidades antes del $n$-\'esimo juego, es decir, dada $n\in\left\{1,2,\ldots\right\}$ y $k\in\left\{0,1,2,\ldots\right\}$ $g_{n,k}:=P\left\{\tilde{L}_{j}>0, j=1,\ldots,n,\tilde{L}_{n}=k\right\}$, la cual adem\'as se puede escribir como:

\begin{eqnarray}\label{Eq.Gnk.2S}
g_{n,k}=\sum_{j=1}^{k+1}\sum_{l=1}^{j}g_{n-1,j}P\left\{X_{n}=k-j-l+1\right\}P\left\{Y_{n}=l\right\}.
\end{eqnarray}

Se definen las siguientes FGP:
\begin{equation}\label{Eq.3.16.a.2S}
G_{n}\left(z\right)=\sum_{k=0}^{\infty}g_{n,k}z^{k},\textrm{ para
}n=0,1,\ldots,
\end{equation}

\begin{equation}\label{Eq.3.16.b.2S}
G\left(z,w\right)=\sum_{n=0}^{\infty}G_{n}\left(z\right)w^{n}.
\end{equation}



%__________________________________________________________________________________
% INICIA LA PROPOSICIÓN
%__________________________________________________________________________________


\begin{Prop}\label{Prop.1.1.2S}
Sean $G_{n}\left(z\right)$ y $G\left(z,w\right)$ definidas como en
(\ref{Eq.3.16.a.2S}) y (\ref{Eq.3.16.b.2S}) respectivamente,
entonces
\begin{equation}\label{Eq.Pag.45}
G_{n}\left(z\right)=\frac{1}{z}\left[G_{n-1}\left(z\right)-G_{n-1}\left(0\right)\right]\tilde{P}\left(z\right).
\end{equation}

Adem\'as


\begin{equation}\label{Eq.Pag.46}
G\left(z,w\right)=\frac{zF\left(z\right)-wP\left(z\right)G\left(0,w\right)}{z-wR\left(z\right)},
\end{equation}

con un \'unico polo en el c\'irculo unitario, adem\'as, el polo es
de la forma $z=\theta\left(w\right)$ y satisface que

\begin{enumerate}
\item[i)]$\tilde{\theta}\left(1\right)=1$,

\item[ii)] $\tilde{\theta}^{(1)}\left(1\right)=\frac{1}{1-\tilde{\mu}}$,

\item[iii)]
$\tilde{\theta}^{(2)}\left(1\right)=\frac{\tilde{\mu}}{\left(1-\tilde{\mu}\right)^{2}}+\frac{\tilde{\sigma}}{\left(1-\tilde{\mu}\right)^{3}}$.
\end{enumerate}

Finalmente, adem\'as se cumple que
\begin{equation}
\esp\left[w^{T}\right]=G\left(0,w\right)=F\left[\tilde{\theta}\left(w\right)\right].
\end{equation}
\end{Prop}
%__________________________________________________________________________________
% TERMINA LA PROPOSICIÓN E INICIA LA DEMOSTRACI\'ON
%__________________________________________________________________________________

\begin{Coro}
El tiempo de ruina del jugador tiene primer y segundo momento
dados por

\begin{eqnarray}
\esp\left[T\right]&=&\frac{\esp\left[\tilde{L}_{0}\right]}{1-\tilde{\mu}}\\
Var\left[T\right]&=&\frac{Var\left[\tilde{L}_{0}\right]}{\left(1-\tilde{\mu}\right)^{2}}+\frac{\sigma^{2}\esp\left[\tilde{L}_{0}\right]}{\left(1-\tilde{\mu}\right)^{3}}.
\end{eqnarray}
\end{Coro}



%__________________________________________________________________________
\section{Procesos de Llegadas a las colas en la RSVC}
%__________________________________________________________________________

Se definen los procesos de llegada de los usuarios a cada una de
las colas dependiendo de la llegada del servidor pero del sistema
al cu\'al no pertenece la cola en cuesti\'on:

Para el sistema 1 y el servidor del segundo sistema

\begin{eqnarray*}
F_{i,j}\left(z_{i};\zeta_{j}\right)=\esp\left[z_{i}^{L_{i}\left(\zeta_{j}\right)}\right]=
\sum_{k=0}^{\infty}\prob\left[L_{i}\left(\zeta_{j}\right)=k\right]z_{i}^{k}\textrm{, para }i,j=1,2.
%F_{1,1}\left(z_{1};\zeta_{1}\right)&=&\esp\left[z_{1}^{L_{1}\left(\zeta_{1}\right)}\right]=
%\sum_{k=0}^{\infty}\prob\left[L_{1}\left(\zeta_{1}\right)=k\right]z_{1}^{k};\\
%F_{2,1}\left(z_{2};\zeta_{1}\right)&=&\esp\left[z_{2}^{L_{2}\left(\zeta_{1}\right)}\right]=
%\sum_{k=0}^{\infty}\prob\left[L_{2}\left(\zeta_{1}\right)=k\right]z_{2}^{k};\\
%F_{1,2}\left(z_{1};\zeta_{2}\right)&=&\esp\left[z_{1}^{L_{1}\left(\zeta_{2}\right)}\right]=
%\sum_{k=0}^{\infty}\prob\left[L_{1}\left(\zeta_{2}\right)=k\right]z_{1}^{k};\\
%F_{2,2}\left(z_{2};\zeta_{2}\right)&=&\esp\left[z_{2}^{L_{2}\left(\zeta_{2}\right)}\right]=
%\sum_{k=0}^{\infty}\prob\left[L_{2}\left(\zeta_{2}\right)=k\right]z_{2}^{k}.\\
\end{eqnarray*}

Para el segundo sistema y el servidor del primero


\begin{eqnarray*}
\hat{F}_{i,j}\left(w_{i};\tau_{j}\right)&=&\esp\left[w_{i}^{\hat{L}_{i}\left(\tau_{j}\right)}\right] =\sum_{k=0}^{\infty}\prob\left[\hat{L}_{i}\left(\tau_{j}\right)=k\right]w_{i}^{k}\textrm{, para }i,j=1,2.
%\hat{F}_{1,1}\left(w_{1};\tau_{1}\right)&=&\esp\left[w_{1}^{\hat{L}_{1}\left(\tau_{1}\right)}\right] =\sum_{k=0}^{\infty}\prob\left[\hat{L}_{1}\left(\tau_{1}\right)=k\right]w_{1}^{k}\\
%\hat{F}_{2,1}\left(w_{2};\tau_{1}\right)&=&\esp\left[w_{2}^{\hat{L}_{2}\left(\tau_{1}\right)}\right] =\sum_{k=0}^{\infty}\prob\left[\hat{L}_{2}\left(\tau_{1}\right)=k\right]w_{2}^{k}\\
%\hat{F}_{1,2}\left(w_{1};\tau_{2}\right)&=&\esp\left[w_{1}^{\hat{L}_{1}\left(\tau_{2}\right)}\right]
%=\sum_{k=0}^{\infty}\prob\left[\hat{L}_{1}\left(\tau_{2}\right)=k\right]w_{1}^{k}\\
%\hat{F}_{2,2}\left(w_{2};\tau_{2}\right)&=&\esp\left[w_{2}^{\hat{L}_{2}\left(\tau_{2}\right)}\right]
%=\sum_{k=0}^{\infty}\prob\left[\hat{L}_{2}\left(\tau_{2}\right)=k\right]w_{2}^{k}\\
\end{eqnarray*}


Ahora, con lo anterior definamos la FGP conjunta para el segundo sistema;% y $\tau_{1}$:


\begin{eqnarray*}
\esp\left[w_{1}^{\hat{L}_{1}\left(\tau_{j}\right)}w_{2}^{\hat{L}_{2}\left(\tau_{j}\right)}\right]
&=&\esp\left[w_{1}^{\hat{L}_{1}\left(\tau_{j}\right)}\right]
\esp\left[w_{2}^{\hat{L}_{2}\left(\tau_{j}\right)}\right]=\hat{F}_{1,j}\left(w_{1};\tau_{j}\right)\hat{F}_{2,j}\left(w_{2};\tau_{j}\right)=\hat{F}_{j}\left(w_{1},w_{2};\tau_{j}\right).\\
%\esp\left[w_{1}^{\hat{L}_{1}\left(\tau_{1}\right)}w_{2}^{\hat{L}_{2}\left(\tau_{1}\right)}\right]
%&=&\esp\left[w_{1}^{\hat{L}_{1}\left(\tau_{1}\right)}\right]
%\esp\left[w_{2}^{\hat{L}_{2}\left(\tau_{1}\right)}\right]=\hat{F}_{1,1}\left(w_{1};\tau_{1}\right)\hat{F}_{2,1}\left(w_{2};\tau_{1}\right)=\hat{F}_{1}\left(w_{1},w_{2};\tau_{1}\right)\\
%\esp\left[w_{1}^{\hat{L}_{1}\left(\tau_{2}\right)}w_{2}^{\hat{L}_{2}\left(\tau_{2}\right)}\right]
%&=&\esp\left[w_{1}^{\hat{L}_{1}\left(\tau_{2}\right)}\right]
%   \esp\left[w_{2}^{\hat{L}_{2}\left(\tau_{2}\right)}\right]=\hat{F}_{1,2}\left(w_{1};\tau_{2}\right)\hat{F}_{2,2}\left(w_{2};\tau_{2}\right)=\hat{F}_{2}\left(w_{1},w_{2};\tau_{2}\right).
\end{eqnarray*}

Con respecto al sistema 1 se tiene la FGP conjunta con respecto al servidor del otro sistema:
\begin{eqnarray*}
\esp\left[z_{1}^{L_{1}\left(\zeta_{j}\right)}z_{2}^{L_{2}\left(\zeta_{j}\right)}\right]
&=&\esp\left[z_{1}^{L_{1}\left(\zeta_{j}\right)}\right]
\esp\left[z_{2}^{L_{2}\left(\zeta_{j}\right)}\right]=F_{1,j}\left(z_{1};\zeta_{j}\right)F_{2,j}\left(z_{2};\zeta_{j}\right)=F_{j}\left(z_{1},z_{2};\zeta_{j}\right).
%\esp\left[z_{1}^{L_{1}\left(\zeta_{1}\right)}z_{2}^{L_{2}\left(\zeta_{1}\right)}\right]
%&=&\esp\left[z_{1}^{L_{1}\left(\zeta_{1}\right)}\right]
%\esp\left[z_{2}^{L_{2}\left(\zeta_{1}\right)}\right]=F_{1,1}\left(z_{1};\zeta_{1}\right)F_{2,1}\left(z_{2};\zeta_{1}\right)=F_{1}\left(z_{1},z_{2};\zeta_{1}\right)\\
%\esp\left[z_{1}^{L_{1}\left(\zeta_{2}\right)}z_{2}^{L_{2}\left(\zeta_{2}\right)}\right]
%&=&\esp\left[z_{1}^{L_{1}\left(\zeta_{2}\right)}\right]
%\esp\left[z_{2}^{L_{2}\left(\zeta_{2}\right)}\right]=F_{1,2}\left(z_{1};\zeta_{2}\right)F_{2,2}\left(z_{2};\zeta_{2}\right)=F_{2}\left(z_{1},z_{2};\zeta_{2}\right).
\end{eqnarray*}

Ahora analicemos la Red de Sistemas de Visitas C\'iclicas, se define la PGF conjunta al tiempo $t$ para los tiempos de visita del servidor en cada una de las colas, para comenzar a dar servicio, definidos anteriormente al tiempo
$t=\left\{\tau_{1},\tau_{2},\zeta_{1},\zeta_{2}\right\}$:

\begin{eqnarray}\label{Eq.Conjuntas}
F_{j}\left(z_{1},z_{2},w_{1},w_{2}\right)&=&\esp\left[\prod_{i=1}^{2}z_{i}^{L_{i}\left(\tau_{j}
\right)}\prod_{i=1}^{2}w_{i}^{\hat{L}_{i}\left(\tau_{j}\right)}\right]\\
\hat{F}_{j}\left(z_{1},z_{2},w_{1},w_{2}\right)&=&\esp\left[\prod_{i=1}^{2}z_{i}^{L_{i}
\left(\zeta_{j}\right)}\prod_{i=1}^{2}w_{i}^{\hat{L}_{i}\left(\zeta_{j}\right)}\right]
\end{eqnarray}
para $j=1,2$. Entonces, con la finalidad de encontrar el n\'umero de usuarios presentes en el sistema cuando el servidor termina de atender una de las colas de cualquier sistema se tiene lo siguiente


\begin{eqnarray*}
&&\esp\left[z_{1}^{L_{1}\left(\overline{\tau}_{1}\right)}z_{2}^{L_{2}\left(\overline{\tau}_{1}\right)}w_{1}^{\hat{L}_{1}\left(\overline{\tau}_{1}\right)}w_{2}^{\hat{L}_{2}\left(\overline{\tau}_{1}\right)}\right]=
\esp\left[z_{2}^{L_{2}\left(\overline{\tau}_{1}\right)}w_{1}^{\hat{L}_{1}\left(\overline{\tau}_{1}
\right)}w_{2}^{\hat{L}_{2}\left(\overline{\tau}_{1}\right)}\right]\\
&=&\esp\left[z_{2}^{L_{2}\left(\tau_{1}\right)+X_{2}\left(\overline{\tau}_{1}-\tau_{1}\right)+Y_{2}\left(\overline{\tau}_{1}-\tau_{1}\right)}w_{1}^{\hat{L}_{1}\left(\tau_{1}\right)+\hat{X}_{1}\left(\overline{\tau}_{1}-\tau_{1}\right)}w_{2}^{\hat{L}_{2}\left(\tau_{1}\right)+\hat{X}_{2}\left(\overline{\tau}_{1}-\tau_{1}\right)}\right]
\end{eqnarray*}
utilizando la (\ref{Eq.UsuariosTotalesZ2}), se tiene que


\begin{eqnarray*}
&=&\esp\left[z_{2}^{L_{2}\left(\tau_{1}\right)}z_{2}^{X_{2}\left(\overline{\tau}_{1}-\tau_{1}\right)}z_{2}^{Y_{2}\left(\overline{\tau}_{1}-\tau_{1}\right)}w_{1}^{\hat{L}_{1}\left(\tau_{1}\right)}w_{1}^{\hat{X}_{1}\left(\overline{\tau}_{1}-\tau_{1}\right)}w_{2}^{\hat{L}_{2}\left(\tau_{1}\right)}w_{2}^{\hat{X}_{2}\left(\overline{\tau}_{1}-\tau_{1}\right)}\right]\\
&=&\esp\left[z_{2}^{L_{2}\left(\tau_{1}\right)}\left\{w_{1}^{\hat{L}_{1}\left(\tau_{1}\right)}w_{2}^{\hat{L}_{2}\left(\tau_{1}\right)}\right\}\left\{z_{2}^{X_{2}\left(\overline{\tau}_{1}-\tau_{1}\right)}
z_{2}^{Y_{2}\left(\overline{\tau}_{1}-\tau_{1}\right)}w_{1}^{\hat{X}_{1}\left(\overline{\tau}_{1}-\tau_{1}\right)}w_{2}^{\hat{X}_{2}\left(\overline{\tau}_{1}-\tau_{1}\right)}\right\}\right]\\
\end{eqnarray*}
aplicando el hecho de que el n\'umero de usuarios que llegan a cada una de las colas del segundo sistema es independiente de las llegadas a las colas del primer sistema:

\begin{eqnarray*}
&=&\esp\left[z_{2}^{L_{2}\left(\tau_{1}\right)}\left\{z_{2}^{X_{2}\left(\overline{\tau}_{1}-\tau_{1}\right)}z_{2}^{Y_{2}\left(\overline{\tau}_{1}-\tau_{1}\right)}w_{1}^{\hat{X}_{1}\left(\overline{\tau}_{1}-\tau_{1}\right)}w_{2}^{\hat{X}_{2}\left(\overline{\tau}_{1}-\tau_{1}\right)}\right\}\right]\esp\left[w_{1}^{\hat{L}_{1}\left(\tau_{1}\right)}w_{2}^{\hat{L}_{2}\left(\tau_{1}\right)}\right]
\end{eqnarray*}
dado que los arribos a cada una de las colas son independientes, podemos separar la esperanza para los procesos de llegada a $Q_{1}$ y $Q_{2}$ al tiempo $\tau_{1}$, que es el tiempo en que el servidor visita a $Q_{1}$. Recordando que $\tilde{X}_{2}\left(z_{2}\right)=X_{2}\left(z_{2}\right)+Y_{2}\left(z_{2}\right)$ se tiene


\begin{eqnarray*}
&=&\esp\left[z_{2}^{L_{2}\left(\tau_{1}\right)}\left\{z_{2}^{\tilde{X}_{2}\left(\overline{\tau}_{1}-\tau_{1}\right)}w_{1}^{\hat{X}_{1}\left(\overline{\tau}_{1}-\tau_{1}\right)}w_{2}^{\hat{X}_{2}\left(\overline{\tau}_{1}-\tau_{1}\right)}\right\}\right]\esp\left[w_{1}^{\hat{L}_{1}\left(\tau_{1}\right)}w_{2}^{\hat{L}_{2}\left(\tau_{1}\right)}\right]\\
&=&\esp\left[z_{2}^{L_{2}\left(\tau_{1}\right)}\left\{\tilde{P}_{2}\left(z_{2}\right)^{\overline{\tau}_{1}-\tau_{1}}\hat{P}_{1}\left(w_{1}\right)^{\overline{\tau}_{1}-\tau_{1}}\hat{P}_{2}\left(w_{2}\right)^{\overline{\tau}_{1}-\tau_{1}}\right\}\right]\esp\left[w_{1}^{\hat{L}_{1}\left(\tau_{1}\right)}w_{2}^{\hat{L}_{2}\left(\tau_{1}\right)}\right]\\
&=&\esp\left[z_{2}^{L_{2}\left(\tau_{1}\right)}\left\{\tilde{P}_{2}\left(z_{2}\right)\hat{P}_{1}\left(w_{1}\right)\hat{P}_{2}\left(w_{2}\right)\right\}^{\overline{\tau}_{1}-\tau_{1}}\right]\esp\left[w_{1}^{\hat{L}_{1}\left(\tau_{1}\right)}w_{2}^{\hat{L}_{2}\left(\tau_{1}\right)}\right]\\
&=&\esp\left[z_{2}^{L_{2}\left(\tau_{1}\right)}\theta_{1}\left(\tilde{P}_{2}\left(z_{2}\right)\hat{P}_{1}\left(w_{1}\right)\hat{P}_{2}\left(w_{2}\right)\right)^{L_{1}\left(\tau_{1}\right)}\right]\esp\left[w_{1}^{\hat{L}_{1}\left(\tau_{1}\right)}w_{2}^{\hat{L}_{2}\left(\tau_{1}\right)}\right]\\
&=&F_{1}\left(\theta_{1}\left(\tilde{P}_{2}\left(z_{2}\right)\hat{P}_{1}\left(w_{1}\right)\hat{P}_{2}\left(w_{2}\right)\right),z{2}\right)\hat{F}_{1}\left(w_{1},w_{2};\tau_{1}\right)\equiv
F_{1}\left(\theta_{1}\left(\tilde{P}_{2}\left(z_{2}\right)\hat{P}_{1}\left(w_{1}\right)\hat{P}_{2}\left(w_{2}\right)\right),z_{2},w_{1},w_{2}\right).
\end{eqnarray*}

Las igualdades anteriores son ciertas pues el n\'umero de usuarios
que llegan a $\hat{Q}_{2}$ durante el intervalo
$\left[\tau_{1},\overline{\tau}_{1}\right]$ a\'un no han sido
atendidos por el servidor del sistema $2$ y por tanto a\'un no
pueden pasar al sistema $1$ a traves de $Q_{2}$. Por tanto el n\'umero de
usuarios que pasan de $\hat{Q}_{2}$ a $Q_{2}$ en el intervalo de
tiempo $\left[\tau_{1},\overline{\tau}_{1}\right]$ depende de la
pol\'itica de traslado entre los dos sistemas, conforme a la
secci\'on anterior.\smallskip

Por lo tanto
\begin{eqnarray}\label{Eq.Fs}
\esp\left[z_{1}^{L_{1}\left(\overline{\tau}_{1}\right)}z_{2}^{L_{2}\left(\overline{\tau}_{1}
\right)}w_{1}^{\hat{L}_{1}\left(\overline{\tau}_{1}\right)}w_{2}^{\hat{L}_{2}\left(
\overline{\tau}_{1}\right)}\right]&=&F_{1}\left(\theta_{1}\left(\tilde{P}_{2}\left(z_{2}\right)
\hat{P}_{1}\left(w_{1}\right)\hat{P}_{2}\left(w_{2}\right)\right),z_{2},w_{1},w_{2}\right)\\
&=&F_{1}\left(\theta_{1}\left(\tilde{P}_{2}\left(z_{2}\right)\hat{P}_{1}\left(w_{1}\right)\hat{P}_{2}\left(w_{2}\right)\right),z_{2}\right)\hat{F}_{1}\left(w_{1},w_{2};\tau_{1}\right)
\end{eqnarray}


Utilizando un razonamiento an\'alogo para $\overline{\tau}_{2}$ y la proposici\'on (\ref{Prop.1.1.2S}) referente al problema de la ruina del jugador obtenemos:

\begin{eqnarray*}
&&\esp\left[z_{1}^{L_{1}\left(\overline{\tau}_{2}\right)}z_{2}^{L_{2}\left(\overline{\tau}_{2}\right)}w_{1}^{\hat{L}_{1}\left(\overline{\tau}_{2}\right)}w_{2}^{\hat{L}_{2}\left(\overline{\tau}_{2}\right)}\right]=
\esp\left[z_{1}^{L_{1}\left(\overline{\tau}_{2}\right)}w_{1}^{\hat{L}_{1}\left(\overline{\tau}_{2}\right)}w_{2}^{\hat{L}_{2}\left(\overline{\tau}_{2}\right)}\right]\\
&=&\esp\left[z_{1}^{L_{1}\left(\tau_{2}\right)}\left\{P_{1}\left(z_{1}\right)\hat{P}_{1}\left(w_{1}\right)\hat{P}_{2}\left(w_{2}\right)\right\}^{\overline{\tau}_{2}-\tau_{2}}\right]
\esp\left[w_{1}^{\hat{L}_{1}\left(\tau_{2}\right)}w_{2}^{\hat{L}_{2}\left(\tau_{2}\right)}\right]\\
&=&\esp\left[z_{1}^{L_{1}\left(\tau_{2}\right)}\tilde{\theta}_{2}\left(P_{1}\left(z_{1}\right)\hat{P}_{1}\left(w_{1}\right)\hat{P}_{2}\left(w_{2}\right)\right)^{L_{2}\left(\tau_{2}\right)}\right]
\esp\left[w_{1}^{\hat{L}_{1}\left(\tau_{2}\right)}w_{2}^{\hat{L}_{2}\left(\tau_{2}\right)}\right]\\
&=&F_{2}\left(z_{1},\tilde{\theta}_{2}\left(P_{1}\left(z_{1}\right)\hat{P}_{1}\left(w_{1}\right)\hat{P}_{2}\left(w_{2}\right)\right)\right)
\hat{F}_{2}\left(w_{1},w_{2};\tau_{2}\right)\\
\end{eqnarray*}


entonces se define
\begin{eqnarray}
\esp\left[z_{1}^{L_{1}\left(\overline{\tau}_{2}\right)}z_{2}^{L_{2}\left(\overline{\tau}_{2}\right)}w_{1}^{\hat{L}_{1}\left(\overline{\tau}_{2}\right)}w_{2}^{\hat{L}_{2}\left(\overline{\tau}_{2}\right)}\right]=F_{2}\left(z_{1},\tilde{\theta}_{2}\left(P_{1}\left(z_{1}\right)\hat{P}_{1}\left(w_{1}\right)\hat{P}_{2}\left(w_{2}\right)\right),w_{1},w_{2}\right)\\
\equiv F_{2}\left(z_{1},\tilde{\theta}_{2}\left(P_{1}\left(z_{1}\right)\hat{P}_{1}\left(w_{1}\right)\hat{P}_{2}\left(w_{2}\right)\right)\right)
\hat{F}_{2}\left(w_{1},w_{2};\tau_{2}\right)
\end{eqnarray}

Para $\overline{\zeta}_{1}$ obtenemos una expresi\'on similar

\begin{eqnarray}
\esp\left[z_{1}^{L_{1}\left(\overline{\zeta}_{1}\right)}z_{2}^{L_{2}\left(\overline{\zeta}_{1}
\right)}w_{1}^{\hat{L}_{1}\left(\overline{\zeta}_{1}\right)}w_{2}^{\hat{L}_{2}\left(
\overline{\zeta}_{1}\right)}\right]&=&\hat{F}_{1}\left(z_{1},z_{2},\hat{\theta}_{1}\left(P_{1}\left(z_{1}\right)\tilde{P}_{2}\left(z_{2}\right)\hat{P}_{2}\left(w_{2}\right)\right),w_{2}\right)\\
&=&F_{1}\left(z_{1},z_{2};\zeta_{1}\right)\hat{F}_{1}\left(\hat{\theta}_{1}\left(P_{1}\left(z_{1}\right)\tilde{P}_{2}\left(z_{2}\right)\hat{P}_{2}\left(w_{2}\right)\right),w_{2}\right).
\end{eqnarray}


Finalmente para $\overline{\zeta}_{2}$
\begin{eqnarray}
\esp\left[z_{1}^{L_{1}\left(\overline{\zeta}_{2}\right)}z_{2}^{L_{2}\left(\overline{\zeta}_{2}\right)}w_{1}^{\hat{L}_{1}\left(\overline{\zeta}_{2}\right)}w_{2}^{\hat{L}_{2}\left(\overline{\zeta}_{2}\right)}\right]&=&\hat{F}_{2}\left(z_{1},z_{2},w_{1},\hat{\theta}_{2}\left(P_{1}\left(z_{1}\right)\tilde{P}_{2}\left(z_{2}\right)\hat{P}_{1}\left(w_{1}\right)\right)\right)\\
&=&F_{2}\left(z_{1},z_{2};\zeta_{2}\right)\hat{F}_{2}\left(w_{1},\hat{\theta}_{2}\left(P_{1}\left(z_{1}\right)\tilde{P}_{2}\left(z_{2}\right)\hat{P}_{1}\left(w_{1}
\right)\right)\right)
\end{eqnarray}
%__________________________________________________________________________
\section{Ecuaciones Recursivas para la RSVC}
%__________________________________________________________________________

Con lo desarrollado hasta ahora podemos encontrar las ecuaciones
recursivas que modelan la RSVC:

\begin{eqnarray*}
F_{2}\left(z_{1},z_{2},w_{1},w_{2}\right)&=&R_{1}\left(P_{1}\left(z_{1}\right)\tilde{P}_{2}\left(z_{2}\right)\prod_{i=1}^{2}
\hat{P}_{i}\left(w_{i}\right)\right)F_{1}\left(\theta_{1}\left(\tilde{P}_{2}\left(z_{2}\right)\hat{P}_{1}\left(w_{1}\right)\hat{P}_{2}\left(w_{2}\right)\right),z_{2}\right)\hat{F}_{1}\left(w_{1},w_{2};\tau_{1}\right),
\end{eqnarray*}


\begin{eqnarray*}
F_{1}\left(z_{1},z_{2},w_{1},w_{2}\right)&=&R_{2}\left(P_{1}\left(z_{1}\right)\tilde{P}_{2}\left(z_{2}\right)\prod_{i=1}^{2}
\hat{P}_{i}\left(w_{i}\right)\right)F_{2}\left(z_{1},\tilde{\theta}_{2}\left(P_{1}\left(z_{1}\right)\hat{P}_{1}\left(w_{1}\right)\hat{P}_{2}\left(w_{2}\right)\right)\right)
\hat{F}_{2}\left(w_{1},w_{2};\tau_{2}\right),
\end{eqnarray*}

\begin{eqnarray*}
\hat{F}_{2}\left(z_{1},z_{2},w_{1},w_{2}\right)&=&\hat{R}_{1}\left(P_{1}\left(z_{1}\right)\tilde{P}_{2}\left(z_{2}\right)\prod_{i=1}^{2}
\hat{P}_{i}\left(w_{i}\right)\right)F_{1}\left(z_{1},z_{2};\zeta_{1}\right)\hat{F}_{1}\left(\hat{\theta}_{1}\left(P_{1}\left(z_{1}\right)\tilde{P}_{2}\left(z_{2}\right)\hat{P}_{2}\left(w_{2}\right)\right),w_{2}\right),
\end{eqnarray*}

\begin{eqnarray*}
\hat{F}_{1}\left(z_{1},z_{2},w_{1},w_{2}\right)&=&\hat{R}_{2}\left(P_{1}\left(z_{1}\right)\tilde{P}_{2}\left(z_{2}\right)\prod_{i=1}^{2}
\hat{P}_{i}\left(w_{i}\right)\right)F_{2}\left(z_{1},z_{2};\zeta_{2}\right)\hat{F}_{2}\left(w_{1},\hat{\theta}_{2}\left(P_{1}\left(z_{1}\right)\tilde{P}_{2}\left(z_{2}\right)\hat{P}_{1}\left(w_{1}
\right)\right)\right),
\end{eqnarray*}


Con la finalidad de facilitar los c\'alculos para determinar los primeros y segundos momentos de los procesos involucrados en la RSVC, es conveniente utilizar la notaci\'on propuesta por Lang \cite{Lang}, es por eso que requerimos definir el operador diferencial $D_{i}$, $i=1,2,3,4$, donde $D_{1}f$ denota la derivad parcial de $f$ con respecto a $z_{1}$, $D_{3}f$ es la derivada parcial de $f$ con respecto a $w_{1}$ y $D_{4}f$ es la derivada parcial de $f$ con respecto a $w_{2}$. Otra consideraci\'on de gran utilidad es que la expresi\'on expresada, es obtenida como consecuencia de aplicar el operador diferencial y adem\'as evaluarla en $z_{1}=1,z_{2}=1,w_{1}=1$ y $w_{2}=1$. En este sentido, la expresi\'ion $F_{2}\left(z_{1},z_{2},w_{1},w_{2}\right)=R_{1}\left(P_{1}\left(z_{1}\right)\tilde{P}_{2}\left(z_{2}\right)\prod_{i=1}^{2}
\hat{P}_{i}\left(w_{i}\right)\right)F_{1}\left(\theta_{1}\left(\tilde{P}_{2}\left(z_{2}\right)\hat{P}_{1}\left(w_{1}\right)\hat{P}_{2}\left(w_{2}\right)\right),z_{2}\right)\hat{F}_{1}\left(w_{1},w_{2};\tau_{1}\right)$ ser\'a representada por su versi\'on simplificada $F_{2}=R_{1}F_{1}\hat{F}_{3}$. Por otra parte $D_{1}\left[R_{1}F_{1}\right]=D_{1}R_{1}\left(F_{1}\right)+R_{1}D_{1}F_{1}$, se tomar\'a simplemente como $D_{1}\left[R_{1}F_{1}\right]=D_{1}R_{1}+D_{1}F_{1}$.

%_________________________________________________________________________________________________
\subsection{Tiempos de Traslado del Servidor}
%_________________________________________________________________________________________________

Recordemos que los tiempos de traslado del servidor para cualquiera de las colas del sistema 1 est\'an dados por la expresi\'on:

\begin{eqnarray}\label{Ec.Ri}
R_{i}\left(\mathbf{z,w}\right)=R_{i}\left(P_{1}\left(z_{1}\right)\tilde{P}_{2}\left(z_{2}\right)\hat{P}_{1}\left(w_{1}\right)\hat{P}_{2}\left(w_{2}\right)\right)
\end{eqnarray}

entonces, las derivadas parciales con respecto a cada uno de los argumentos $z_{1},z_{2},w_{1}$ y $w_{2}$ son de la forma

\begin{eqnarray}\label{Ec.Derivada.Ri}
D_{i}R_{i}&=&DR_{i}D_{i}P_{i}
\end{eqnarray}
donde se hacen las siguientes convenciones:

\begin{eqnarray*}
\begin{array}{llll}
D_{2}P_{2}\equiv D_{2}\tilde{P}_{2}, & D_{3}P_{3}\equiv D_{3}\hat{P}_{1}, &D_{4}P_{4}\equiv D_{4}\hat{P}_{2},
\end{array}
\end{eqnarray*}

%_________________________________________________________________________________________________
\subsection{Longitudes de la Cola en tiempos del servidor del otro sistema}
%_________________________________________________________________________________________________


Recordemos que  $F_{1,2}\left(z_{1};\zeta_{2}\right)F_{2,2}\left(z_{2};\zeta_{2}\right)=F_{2}\left(z_{1},z_{2};\zeta_{2}\right)$, entonces

\begin{eqnarray*}
D_{1}F_{2}\left(z_{1},z_{2};\zeta_{2}\right)&=&D_{1}\left[F_{1,2}\left(z_{1};\zeta_{2}\right)F_{2,2}\left(z_{2};\zeta_{2}\right)\right]
=F_{2,2}\left(z_{2};\zeta_{2}\right)D_{1}F_{1,2}\left(z_{1};\zeta_{2}\right)=F_{1,2}^{(1)}\left(1\right)
\end{eqnarray*}

es decir, $D_{1}F_{2}=F_{1,2}^{(1)}(1)$; de manera an\'aloga se puede ver que $D_{2}F_{2}=F_{2,2}^{(1)}\left(1\right)$, mientras que $D_{3}F_{2}=D_{4}F_{2}=0$. Es decir, las expresiones resultantes pueden expresarse de manera general como:

%\begin{eqnarray*}
%\begin{array}{llll}
%D_{1}F_{1}=F_{1,1}^{(1)}\left(1\right),&D_{2}F_{1}=F_{2,1}^{(1)}\left(1\right), & D_{3}F_{1}=0, & D_{4}F_{1}=0,\\
%D_{1}F_{2}=F_{1,2}^{(1)}\left(1\right),&D_{2}F_{2}=F_{2,2}^{(1)}\left(1\right), & D_{3}F_{2}=0, & D_{4}F_{2}=0,\\
%D_{1}\hat{F}_{1}=0,&D_{2}\hat{F}_{1}=0,&D_{3}=\hat{F}_{1,1}^{(1)}\left(1\right),&D_{4}\hat{F}_{1}=\hat{F}_{2,1}^{(1)}\left(1\right)\\
%D_{1}\hat{F}_{2}=0,&D_{2}\hat{F}_{2}=0,&D_{3}\hat{F}_{2}=\hat{F}_{1,2}^{(1)}\left(1\right),&D_{4}\hat{F}_{2}=\hat{F}_{2,2}^{(1)}\left(1\right)\\
%\end{array}
%\end{eqnarray*}

%que en general pueden escribirse como

\begin{eqnarray*}
\begin{array}{ccc}
D_{i}F_{j}=\indora_{i\leq2}F_{i,j}^{(1)}\left(1\right),& \textrm{ y } &D_{i}\hat{F}_{j}=\indora_{i\geq2}F_{i,j}^{(1)}\left(1\right)
\end{array}
\end{eqnarray*}

%_________________________________________________________________________________________________
\subsection{Usuarios presentes en la cola en tiempos del servidor de sus sistema}
%_________________________________________________________________________________________________
Recordemos la expresi\'on obtenida para las longitudes de la cola para cada uno de los sistemas considerando que los tiempos del servidor correspondiente al mismo sistema: $F_{1}\left(\theta_{1}\left(\tilde{P}_{2}\left(z_{2}\right)\hat{P}_{1}\left(w_{1}
\right)\hat{P}_{2}\left(w_{2}\right)\right),z_{2}\right)$. Al igual que antes, podemos obtener las expresiones resultantes de aplicar el operador diferencial con respecto a cada uno de los argumentos:

$D_{1}F_{1}=0$, $D_{2}F_{1}=D_{1}F_{1}D\theta_{1}D_{2}\tilde{P}_{2}+D_{2}F_{1}$, $D_{3}F_{1}=D_{1}F_{1}D\theta_{1}D_{3}\hat{P}_{1}+D_{3}\hat{F}_{1}$ y finalmente
$D_{4}F_{1}=D_{1}F_{1}D\theta_{1}D_{4}\hat{P}_{2}+D_{4}\hat{F}_{1}$, en t\'erminos generales:

\begin{eqnarray*}
\begin{array}{ll}
D_{i}F_{1}=\indora_{i\neq1}D_{1}F_{1}D\theta_{1}D_{i}P_{i}+\indora_{i=2}D_{i}F_{1}, & D_{i}F_{2}=\indora_{i\neq2}D_{2}F_{2}D\tilde{\theta}_{2}D_{i}P_{i}+\indora_{i=1}D_{i}F_{2}\\
D_{i}\hat{F}_{1}=\indora_{i\neq3}D_{3}\hat{F}_{1}D\hat{\theta}_{1}D_{i}P_{i}+\indora_{i=4}D_{i}\hat{F}_{1},& D_{i}\hat{F}_{2}=\indora_{i\neq4}D_{4}\hat{F}_{2}D\hat{\theta}_{2}D_{i}P_{i}+\indora_{i=3}D_{i}\hat{F}_{2}.
\end{array}
\end{eqnarray*}

\begin{eqnarray}
D_{i}F_{1}&=&\indora_{i\neq1}D_{1}F_{1}D\theta_{1}D_{i}P_{i}+\indora_{i=2}D_{i}F_{1},\\ D_{i}F_{2}&=&\indora_{i\neq2}D_{2}F_{2}D\tilde{\theta}_{2}D_{i}P_{i}+\indora_{i=1}D_{i}F_{2}\\
D_{i}\hat{F}_{1}&=&\indora_{i\neq3}D_{3}\hat{F}_{1}D\hat{\theta}_{1}D_{i}P_{i}+\indora_{i=4}D_{i}\hat{F}_{1},\\
D_{i}\hat{F}_{2}&=&\indora_{i\neq4}D_{4}\hat{F}_{2}D\hat{\theta}_{2}D_{i}P_{i}+\indora_{i=3}D_{i}\hat{F}_{2}.
\end{eqnarray}


%_________________________________________________________________________________________________
\subsection{Usuarios presentes en la RSVC}
%_________________________________________________________________________________________________

Hagamos lo correspondiente para las longitudes de las colas de la RSVC utilizando las expresiones obtenidas en las secciones anteriores, recordemos que

\begin{eqnarray*}
\mathbf{F}_{1}\left(\theta_{1}\left(\tilde{P}_{2}\left(z_{2}\right)\hat{P}_{1}\left(w_{1}\right)
\hat{P}_{2}\left(w_{2}\right)\right),z_{2},w_{1},w_{2}\right)=
F_{1}\left(\theta_{1}\left(\tilde{P}_{2}\left(z_{2}\right)\hat{P}_{1}\left(w_{1}
\right)\hat{P}_{2}\left(w_{2}\right)\right),z_{2}\right)
\hat{F}_{1}\left(w_{1},w_{2};\tau_{1}\right)\\
\end{eqnarray*}

entonces



\begin{eqnarray*}
D_{1}\mathbf{F}_{1}&=& 0\\
D_{2}\mathbf{F}_{1}&=&f_{1}\left(1\right)\left(\frac{1}{1-\mu_{1}}\right)\tilde{\mu}_{2}+f_{1}\left(2\right)\\
D_{3}\mathbf{F}_{1}&=&f_{1}\left(1\right)\left(\frac{1}{1-\mu_{1}}\right)\hat{\mu}_{1}+\hat{F}_{1,1}^{(1)}\left(1\right)\\
D_{4}\mathbf{F}_{1}&=&f_{1}\left(1\right)\left(\frac{1}{1-\mu_{1}}\right)\hat{\mu}_{2}+\hat{F}_{2,1}^{(1)}\left(1\right)
\end{eqnarray*}


para $\tau_{2}$:

\begin{eqnarray*}
\mathbf{F}_{2}\left(z_{1},\tilde{\theta}_{2}\left(P_{1}\left(z_{1}\right)\hat{P}_{1}\left(w_{1}\right)\hat{P}_{2}\left(w_{2}\right)\right),
w_{1},w_{2}\right)=F_{2}\left(z_{1},\tilde{\theta}_{2}\left(P_{1}\left(z_{1}\right)\hat{P}_{1}\left(w_{1}\right)
\hat{P}_{2}\left(w_{2}\right)\right)\right)\hat{F}_{2}\left(w_{1},w_{2};\tau_{2}\right)
\end{eqnarray*}
se tiene que

\begin{eqnarray*}
D_{1}\mathbf{F}_{2}&=&f_{2}\left(2\right)\left(\frac{1}{1-\tilde{\mu}_{2}}\right)\mu_{1}+f_{2}\left(1\right)\\
D_{2}\mathbf{F}_{2}&=&0\\
D_{3}\mathbf{F}_{2}&=&f_{2}\left(2\right)\left(\frac{1}{1-\tilde{\mu}_{2}}\right)\hat{\mu}_{1}+\hat{F}_{2,1}^{(1)}\left(1\right)\\
D_{4}\mathbf{F}_{2}&=&f_{2}\left(2\right)\left(\frac{1}{1-\tilde{\mu}_{2}}\right)\hat{\mu}_{2}+\hat{F}_{2,2}^{(1)}\left(1\right)\\
\end{eqnarray*}



Ahora para el segundo sistema

\begin{eqnarray*}\hat{\mathbf{F}}_{1}\left(z_{1},z_{2},\hat{\theta}_{1}\left(P_{1}\left(z_{1}\right)\tilde{P}_{2}\left(z_{2}\right)\hat{P}_{2}\left(w_{2}\right)\right),
w_{2}\right)=F_{1}\left(z_{1},z_{2};\zeta_{1}\right)\hat{F}_{1}\left(\hat{\theta}_{1}\left(P_{1}\left(z_{1}\right)\tilde{P}_{2}\left(z_{2}\right)
\hat{P}_{2}\left(w_{2}\right)\right),w_{2}\right)
\end{eqnarray*}
entonces

\begin{eqnarray*}
D_{1}\hat{\mathbf{F}}_{1}&=&\hat{f}_{1}\left(1\right)\left(\frac{1}{1-\hat{\mu}_{1}}\right)\mu_{1}+F_{1,1}^{(1)}\left(1\right)\\
D_{2}\hat{\mathbf{F}}_{1}&=&\hat{f}_{1}\left(1\right)\left(\frac{1}{1-\hat{\mu}_{1}}\right)\tilde{\mu}_{2}+F_{2,1}^{(1)}\left(1\right)\\
D_{3}\hat{\mathbf{F}}_{1}&=&0\\
D_{4}\hat{\mathbf{F}}_{1}&=&\hat{f}_{1}\left(1\right)\left(\frac{1}{1-\hat{\mu}_{1}}\right)\hat{\mu}_{2}+\hat{f}_{1}\left(2\right)\\
\end{eqnarray*}




Finalmente para $\zeta_{2}$

\begin{eqnarray*}\hat{\mathbf{F}}_{2}\left(z_{1},z_{2},w_{1},\hat{\theta}_{2}\left(P_{1}\left(z_{1}\right)\tilde{P}_{2}\left(z_{2}\right)\hat{P}_{1}\left(w_{1}\right)\right)\right)&=&F_{2}\left(z_{1},z_{2};\zeta_{2}\right)\hat{F}_{2}\left(w_{1},\hat{\theta}_{2}\left(P_{1}\left(z_{1}\right)\tilde{P}_{2}\left(z_{2}\right)\hat{P}_{1}\left(w_{1}\right)\right)\right]
\end{eqnarray*}
por tanto:


\begin{eqnarray*}
D_{1}\hat{\mathbf{F}}_{2}&=&\hat{f}_{2}\left(1\right)\left(\frac{1}{1-\hat{\mu}_{2}}\right)\mu_{1}+F_{1,2}^{(1)}\left(1\right)\\
D_{2}\hat{\mathbf{F}}_{2}&=&\hat{f}_{2}\left(1\right)\left(\frac{1}{1-\hat{\mu}_{2}}\right)\tilde{\mu}_{2}+F_{2,2}^{(1)}\left(1\right)\\
D_{3}\hat{\mathbf{F}}_{2}&=&\hat{f}_{2}\left(1\right)\left(\frac{1}{1-\hat{\mu}_{2}}\right)\hat{\mu}_{1}+\hat{f}_{2}\left(1\right)\\
D_{4}\hat{\mathbf{F}}_{2}&=&0\\
\end{eqnarray*}


%_________________________________________________________________________________________________
\subsection{Ecuaciones Recursivas}
%_________________________________________________________________________________________________

Entonces, de todo lo desarrollado hasta ahora se tienen las siguientes ecuaciones:

%Para $$, se tiene que


\begin{eqnarray}\label{Ec.Primeras.Derivadas.Parciales}
\begin{array}{ll}
\mathbf{F}_{1}=R_{2}F_{2}\hat{F}_{2}, & D_{i}\mathbf{F}_{1}=D_{i}\left(R_{2}+F_{2}+\indora_{i\geq3}\hat{F}_{2}\right)\\
\mathbf{F}_{2}=R_{1}F_{1}\hat{F}_{1}, & D_{i}\mathbf{F}_{2}=D_{i}\left(R_{1}+F_{1}+\indora_{i\geq3}\hat{F}_{1}\right)\\
\hat{\mathbf{F}}_{1}=\hat{R}_{2}\hat{F}_{2}F_{2}, & D_{i}\hat{\mathbf{F}}_{1}=D_{i}\left(\hat{R}_{2}+\hat{F}_{2}+\indora_{i\leq2}F_{2}\right)\\
\hat{\mathbf{F}}_{2}=\hat{R}_{1}\hat{F}_{1}F_{1}, & D_{i}\hat{\mathbf{F}}_{2}=D_{i}\left(\hat{R}_{1}+\hat{F}_{1}+\indora_{i\leq2}F_{1}\right)
\end{array}
\end{eqnarray}

cuyas expresiones son de la forma:


\begin{eqnarray*}
\begin{array}{ll}
D_{1}\mathbf{F}_{2}=r_{1}\mu_{1},&
D_{2}\mathbf{F}_{2}=r_{1}\tilde{\mu}_{2}+f_{1}\left(1\right)\left(\frac{1}{1-\mu_{1}}\right)\tilde{\mu}_{2}+f_{1}\left(2\right),\\
D_{3}\mathbf{F}_{2}=r_{1}\hat{\mu}_{1}+f_{1}\left(1\right)\left(\frac{1}{1-\mu_{1}}\right)\hat{\mu}_{1}+\hat{F}_{1,1}^{(1)}\left(1\right),&
D_{4}\mathbf{F}_{2}=r_{1}\hat{\mu}_{2}+f_{1}\left(1\right)\left(\frac{1}{1-\mu_{1}}\right)\hat{\mu}_{2}+\hat{F}_{2,1}^{(1)}\left(1\right),\\
D_{1}\mathbf{F}_{1}=r_{2}\mu_{1}+f_{2}\left(2\right)\left(\frac{1}{1-\tilde{\mu}_{2}}\right)\mu_{1}+f_{2}\left(1\right),&
D_{2}\mathbf{F}_{1}=r_{2}\tilde{\mu}_{2},\\
D_{3}\mathbf{F}_{1}=r_{2}\hat{\mu}_{1}+f_{2}\left(2\right)\left(\frac{1}{1-\tilde{\mu}_{2}}\right)\hat{\mu}_{1}+\hat{F}_{2,1}^{(1)}\left(1\right),&
D_{4}\mathbf{F}_{1}=r_{2}\hat{\mu}_{2}+f_{2}\left(2\right)\left(\frac{1}{1-\tilde{\mu}_{2}}\right)\hat{\mu}_{2}+\hat{F}_{2,2}^{(1)}\left(1\right),\\
D_{1}\hat{\mathbf{F}}_{2}=\hat{r}_{1}\mu_{1}+\hat{f}_{1}\left(1\right)\left(\frac{1}{1-\hat{\mu}_{1}}\right)\mu_{1}+F_{1,1}^{(1)}\left(1\right),&
D_{2}\hat{\mathbf{F}}_{2}=\hat{r}_{1}\mu_{2}+\hat{f}_{1}\left(1\right)\left(\frac{1}{1-\hat{\mu}_{1}}\right)\tilde{\mu}_{2}+F_{2,1}^{(1)}\left(1\right),\\
D_{3}\hat{\mathbf{F}}_{2}=\hat{r}_{1}\hat{\mu}_{1},&
D_{4}\hat{\mathbf{F}}_{2}=\hat{r}_{1}\hat{\mu}_{2}+\hat{f}_{1}\left(1\right)\left(\frac{1}{1-\hat{\mu}_{1}}\right)\hat{\mu}_{2}+\hat{f}_{1}\left(2\right),\\
D_{1}\hat{\mathbf{F}}_{1}=\hat{r}_{2}\mu_{1}+\hat{f}_{2}\left(1\right)\left(\frac{1}{1-\hat{\mu}_{2}}\right)\mu_{1}+F_{1,2}^{(1)}\left(1\right),&
D_{2}\hat{\mathbf{F}}_{1}=\hat{r}_{2}\tilde{\mu}_{2}+\hat{f}_{2}\left(2\right)\left(\frac{1}{1-\hat{\mu}_{2}}\right)\tilde{\mu}_{2}+F_{2,2}^{(1)}\left(1\right),\\
D_{3}\hat{\mathbf{F}}_{1}=\hat{r}_{2}\hat{\mu}_{1}+\hat{f}_{2}\left(2\right)\left(\frac{1}{1-\hat{\mu}_{2}}\right)\hat{\mu}_{1}+\hat{f}_{2}\left(1\right),&
D_{4}\hat{\mathbf{F}}_{1}=\hat{r}_{2}\hat{\mu}_{2}
\end{array}
\end{eqnarray*}


de las cuales resulta

\begin{eqnarray*}
\begin{array}{llll}
f_{2}\left(1\right)=r_{1}\mu_{1},&
f_{1}\left(2\right)=r_{2}\tilde{\mu}_{2},&
\hat{f}_{1}\left(4\right)=\hat{r}_{2}\hat{\mu}_{2},&
\hat{f}_{2}\left(3\right)=\hat{r}_{1}\hat{\mu}_{1}
\end{array}
\end{eqnarray*}

\begin{eqnarray*}
f_{1}\left(1\right)&=&r_{2}\mu_{1}+\mu_{1}\left(\frac{f_{2}\left(2\right)}{1-\tilde{\mu}_{2}}\right)+r_{1}\mu_{1}=\mu_{1}\left(r_{1}+r_{2}+\frac{f_{2}\left(2\right)}{1-\tilde{\mu}_{2}}\right)=\mu_{1}\left(r+\frac{f_{2}\left(2\right)}{1-\tilde{\mu}_{2}}\right),\\
f_{1}\left(3\right)&=&r_{2}\hat{\mu}_{1}+\hat{\mu}_{1}\left(\frac{f_{2}\left(2\right)}{1-\tilde{\mu}_{2}}\right)+\hat{F}^{(1)}_{1,2}\left(1\right)=\hat{\mu}_{1}\left(r_{2}+\frac{f_{2}\left(2\right)}{1-\tilde{\mu}_{2}}\right)+\hat{F}_{1,2}^{(1)}\left(1\right),\end{eqnarray*}

utilizando un razonamiento an\'alogo a los anteriores se puede verificar que

\begin{eqnarray*}
\begin{array}{ll}
f_{1}\left(4\right)=\hat{\mu}_{2}\left(r_{2}+\frac{f_{2}\left(2\right)}{1-\tilde{\mu}_{2}}\right)+\hat{F}_{2,2}^{(1)}\left(1\right),&
f_{2}\left(2\right)=\left(r+\frac{f_{1}\left(1\right)}{1-\mu_{1}}\right)\tilde{\mu}_{2},\\
f_{2}\left(3\right)=\hat{\mu}_{1}\left(r_{1}+\frac{f_{1}\left(1\right)}{1-\mu_{1}}\right)+\hat{F}_{1,1}^{(1)}\left(1\right),&
f_{2}\left(4\right)=\hat{\mu}_{2}\left(r_{1}+\frac{f_{1}\left(1\right)}{1-\mu_{1}}\right)+\hat{F}_{2,1}^{(1)}\left(1\right),
\end{array}
\end{eqnarray*}


\begin{eqnarray*}
\begin{array}{ll}
\hat{f}_{1}\left(1\right)=\left(\hat{r}_{2}+\frac{\hat{f}_{2}\left(4\right)}{1-\hat{\mu}_{2}}\right)\mu_{1}+F_{1,2}^{(1)}\left(1\right),&
\hat{f}_{1}\left(2\right)=\left(\hat{r}_{2}+\frac{\hat{f}_{2}\left(4\right)}{1-\hat{\mu}_{2}}\right)\tilde{\mu}_{2}+F_{2,2}^{(1)}\left(1\right),\\
\hat{f}_{1}\left(3\right)=\left(\hat{r}+\frac{\hat{f}_{2}\left(4\right)}{1-\hat{\mu}_{2}}\right)\hat{\mu}_{1},&
\hat{f}_{2}\left(1\right)=\left(\hat{r}_{1}+\frac{\hat{f}_{1}\left(3\right)}{1-\hat{\mu}_{1}}\right)\mu_{1}+F_{1,1}^{(1)}\left(1\right),\\
\hat{f}_{2}\left(2\right)=\left(\hat{r}_{1}+\frac{\hat{f}_{1}\left(3\right)}{1-\hat{\mu}_{1}}\right)\tilde{\mu}_{2}+F_{2,1}^{(1)}\left(1\right),&
\hat{f}_{2}\left(4\right)=\left(\hat{r}+\frac{\hat{f}_{1}\left(3\right)}{1-\hat{\mu}_{1}}\right)\hat{\mu}_{2},\\
\end{array}
\end{eqnarray*}


%_______________________________________________________________________________________________
\subsection{Soluci\'on del Sistema de Ecuaciones Lineales}
%_________________________________________________________________________________________________

Si $\mu=\mu_{1}+\tilde{\mu}_{2}$, $\hat{\mu}=\hat{\mu}_{1}+\hat{\mu}_{2}$, $r=r_{1}+r_{2}$ y $\hat{r}=\hat{r}_{1}+\hat{r}_{2}$ la soluci\'on del sistema de
ecuaciones est\'a dada por


\begin{eqnarray*}
f_{1}\left(1\right)&=&\mu_{1}\left(r_{2}+\frac{f_{2}\left(2\right)}{1-\tilde{\mu}_{2}}\right)+\hat{F}_{2,1}^{(1)}\left(1\right)=\mu_{1}\left(r_{2}+\frac{r\frac{\tilde{\mu}_{2}\left(1-\tilde{\mu}_{2}\right)}{1-\mu}}{1-\tilde{\mu}_{2}}\right)+\hat{F}_{2,1}^{(1)}\left(1\right)=\mu_{1}\left(r_{2}+\frac{r\tilde{\mu}_{2}}{1-\mu}\right)+\hat{F}_{2,1}^{(1)}\left(1\right),
\end{eqnarray*}

de manera an\'aloga se obtiene lo siguiente:


\begin{eqnarray*}
\begin{array}{ll}
f_{1}\left(3\right)=\hat{\mu}_{1}\left(r_{2}+\frac{r\tilde{\mu}_{2}}{1-\mu}\right)+\hat{F}_{2,1}^{(1)}\left(1\right),&
f_{1}\left(4\right)=\hat{\mu}_{2}\left(r_{2}+\frac{r\tilde{\mu}_{2}}{1-\mu}\right)+\hat{F}_{2,2}^{(1)}\left(1\right),\\
f_{2}\left(3\right)=\hat{\mu}_{1}\left(r_{1}+\frac{r\mu_{1}}{1-\mu}\right)+\hat{F}_{1,1}^{(1)}\left(1\right),&
f_{2}\left(4\right)=\hat{\mu}_{2}\left(r_{1}+\frac{r\mu_{1}}{1-\mu}\right)+\hat{F}_{2,1}^{(1)}\left(1\right),\\
\hat{f}_{1}\left(1\right)=\mu_{1}\left(\hat{r}_{2}+\frac{\hat{r}\hat{\mu}_{2}}{1-\hat{\mu}}\right)+F_{1,2}^{(1)}\left(1\right),&
\hat{f}_{1}\left(2\right)=\tilde{\mu}_{2}\left(\hat{r}_{2}+\frac{\hat{r}\hat{\mu}_{2}}{1-\hat{\mu}}\right)+F_{2,2}^{(1)}\left(1\right),\\
\hat{f}_{2}\left(1\right)=\mu_{1}\left(\hat{r}_{1}+\frac{\hat{r}\hat{\mu}_{1}}{1-\hat{\mu}}\right)+F_{1,1}^{(1)}\left(1\right),&
\hat{f}_{2}\left(2\right)=\tilde{\mu}_{2}\left(\hat{r}_{1}+\frac{\hat{r}\hat{\mu}_{1}}{1-\hat{\mu}}\right)+F_{2,1}^{(1)}\left(1\right)
\end{array}
\end{eqnarray*}


%\begin{eqnarray*}
%\end{eqnarray*}

%----------------------------------------------------------------------------------------
\section{Resultado Principal}
%----------------------------------------------------------------------------------------
Sean $\mu_{1},\mu_{2},\check{\mu}_{2},\hat{\mu}_{1},\hat{\mu}_{2}$ y $\tilde{\mu}_{2}=\mu_{2}+\check{\mu}_{2}$ los valores esperados de los proceso definidos anteriormente, y sean $r_{1},r_{2}, \hat{r}_{1}$ y $\hat{r}_{2}$ los valores esperado s de los tiempos de traslado del servidor entre las colas para cada uno de los sistemas de visitas c\'iclicas. Si se definen $\mu=\mu_{1}+\tilde{\mu}_{2}$, $\hat{\mu}=\hat{\mu}_{1}+\hat{\mu}_{2}$, y $r=r_{1}+r_{2}$ y  $\hat{r}=\hat{r}_{1}+\hat{r}_{2}$, entonces se tiene el siguiente resultado.

\begin{Teo}
Supongamos que $\mu<1$, $\hat{\mu}<1$, entonces, el n\'umero de usuarios presentes en cada una de las colas que conforman la RSVC cuando uno de los servidores visita a alguna de ellas est\'a dada por la soluci\'on del Sistema de Ecuaciones Lineales presentados arriba cuyas expresiones damos a continuaci\'on:
%{\footnotesize{


\begin{eqnarray*}
\begin{array}{lll}
f_{1}\left(1\right)=\mu_{1}\left(r_{2}+\frac{r\tilde{\mu}_{2}}{1-\mu}\right)+\hat{F}_{2,1}^{(1)}\left(1\right),&f_{1}\left(2\right)=r_{2}\tilde{\mu}_{2},&f_{1}\left(3\right)=\hat{\mu}_{1}\left(r_{2}+\frac{r\tilde{\mu}_{2}}{1-\mu}\right)+\hat{F}_{2,1}^{(1)}\left(1\right),\\
f_{1}\left(4\right)=\hat{\mu}_{2}\left(r_{2}+\frac{r\tilde{\mu}_{2}}{1-\mu}\right)+\hat{F}_{2,2}^{(1)}\left(1\right),&f_{2}\left(1\right)=r_{1}\mu_{1},&f_{2}\left(2\right)=r\frac{\tilde{\mu}_{2}\left(1-\tilde{\mu}_{2}\right)}{1-\mu},\\
f_{2}\left(3\right)=\hat{\mu}_{1}\left(r_{1}+\frac{r\mu_{1}}{1-\mu}\right)+\hat{F}_{1,1}^{(1)}\left(1\right),&f_{2}\left(4\right)=\hat{\mu}_{2}\left(r_{1}+\frac{r\mu_{1}}{1-\mu}\right)+\hat{F}_{2,1}^{(1)}\left(1\right),&\hat{f}_{1}\left(1\right)=\mu_{1}\left(\hat{r}_{2}+\frac{\hat{r}\hat{\mu}_{2}}{1-\hat{\mu}}\right)+F_{1,2}^{(1)}\left(1\right),\\
\hat{f}_{1}\left(2\right)=\tilde{\mu}_{2}\left(\hat{r}_{2}+\frac{\hat{r}\hat{\mu}_{2}}{1-\hat{\mu}}\right)+F_{2,2}^{(1)}\left(1\right),&\hat{f}_{1}\left(3\right)=\hat{r}\frac{\hat{\mu}_{1}\left(1-\hat{\mu}_{1}\right)}{1-\hat{\mu}},&\hat{f}_{1}\left(4\right)=\hat{r}_{2}\hat{\mu}_{2},\\
\hat{f}_{2}\left(1\right)=\mu_{1}\left(\hat{r}_{1}+\frac{\hat{r}\hat{\mu}_{1}}{1-\hat{\mu}}\right)+F_{1,1}^{(1)}\left(1\right),&\hat{f}_{2}\left(2\right)=\tilde{\mu}_{2}\left(\hat{r}_{1}+\frac{\hat{r}\hat{\mu}_{1}}{1-\hat{\mu}}\right)+F_{2,1}^{(1)}\left(1\right),&\hat{f}_{2}\left(3\right)=\hat{r}_{1}\hat{\mu}_{1},\\
&\hat{f}_{2}\left(4\right)=\hat{r}\frac{\hat{\mu}_{2}\left(1-\hat{\mu}_{2}\right)}{1-\hat{\mu}}.&\\
\end{array}
\end{eqnarray*} %}}
\end{Teo}
%\newpage
%___________________________________________________________________________________________
%
\section{Derivadas de Orden Superior}
%___________________________________________________________________________________________
%
Si tomamos la derivada de segundo orden con respecto a las ecuaciones dadas en (\ref{Ec.Primeras.Derivadas.Parciales}) obtenemos

\small{
\begin{eqnarray*}\label{Ec.Derivadas.Segundo.Orden}
D_{k}D_{i}F_{1}&=&D_{k}D_{i}\left(R_{2}+F_{2}+\indora_{i\geq3}\hat{F}_{4}\right)+D_{i}R_{2}D_{k}\left(F_{2}+\indora_{k\geq3}\hat{F}_{4}\right)+D_{i}F_{2}D_{k}\left(R_{2}+\indora_{k\geq3}\hat{F}_{4}\right)+\indora_{i\geq3}D_{i}\hat{F}_{4}D_{k}\left(R_{}+F_{2}\right)\\
D_{k}D_{i}F_{2}&=&D_{k}D_{i}\left(R_{1}+F_{1}+\indora_{i\geq3}\hat{F}_{3}\right)+D_{i}R_{1}D_{k}\left(F_{1}+\indora_{k\geq3}\hat{F}_{3}\right)+D_{i}F_{1}D_{k}\left(R_{1}+\indora_{k\geq3}\hat{F}_{3}\right)+\indora_{i\geq3}D_{i}\hat{F}_{3}D_{k}\left(R_{1}+F_{1}\right)\\
D_{k}D_{i}\hat{F}_{3}&=&D_{k}D_{i}\left(\hat{R}_{4}+\indora_{i\leq2}F_{2}+\hat{F}_{4}\right)+D_{i}\hat{R}_{4}D_{k}\left(\indora_{k\leq2}F_{2}+\hat{F}_{4}\right)+D_{i}\hat{F}_{4}D_{k}\left(\hat{R}_{4}+\indora_{k\leq2}F_{2}\right)+\indora_{i\leq2}D_{i}F_{2}D_{k}\left(\hat{R}_{4}+\hat{F}_{4}\right)\\
D_{k}D_{i}\hat{F}_{4}&=&D_{k}D_{i}\left(\hat{R}_{3}+\indora_{i\leq2}F_{1}+\hat{F}_{3}\right)+D_{i}\hat{R}_{3}D_{k}\left(\indora_{k\leq2}F_{1}+\hat{F}_{3}\right)+D_{i}\hat{F}_{3}D_{k}\left(\hat{R}_{3}+\indora_{k\leq2}F_{1}\right)+\indora_{i\leq2}D_{i}F_{1}D_{k}\left(\hat{R}_{3}+\hat{F}_{3}\right)
\end{eqnarray*}}
para $i,k=1,\ldots,4$. Es necesario determinar las derivadas de segundo orden para las expresiones de la forma $D_{k}D_{i}\left(R_{2}+F_{2}+\indora_{i\geq3}\hat{F}_{4}\right)$

%_________________________________________________________________________________________________________
\subsection{Derivadas de Segundo Orden: Tiempos de Traslado del Servidor}
%_________________________________________________________________________________________________________

A saber, $R_{i}\left(z_{1},z_{2},w_{1},w_{2}\right)=R_{i}\left(P_{1}\left(z_{1}\right)\tilde{P}_{2}\left(z_{2}\right)
\hat{P}_{1}\left(w_{1}\right)\hat{P}_{2}\left(w_{2}\right)\right)$, la denotaremos por la expresi\'on $R_{i}=R_{i}\left(
P_{1}\tilde{P}_{2}\hat{P}_{1}\hat{P}_{2}\right)$, donde al igual que antes, utilizando la notaci\'on dada en \cite{Lang} se tiene   que

\begin{eqnarray}
D_{i}D_{i}R_{k}=D^{2}R_{k}\left(D_{i}P_{i}\right)^{2}+DR_{k}D_{i}D_{i}P_{i}
\end{eqnarray}

mientras que para $i\neq j$

\begin{eqnarray}
D_{i}D_{j}R_{k}=D^{2}R_{k}D_{i}P_{i}D_{j}P_{j}+DR_{k}D_{j}P_{j}D_{i}P_{i}
\end{eqnarray}

%_________________________________________________________________________________________________________
\subsection{Derivadas de Segundo Orden: Longitudes de las Colas}
%_________________________________________________________________________________________________________

Recordemos la expresi\'on $F_{1}\left(\theta_{1}\left(\tilde{P}_{2}\left(z_{2}\right)\hat{P}_{1}\left(w_{1}\right)\hat{P}_{2}\left(w_{2}\right)\right),
z_{2}\right)$, que denotaremos por $F_{1}\left(\theta_{1}\left(\tilde{P}_{2}\hat{P}_{1}\hat{P}_{2}\right),z_{2}\right)$, entonces las derivadas parciales mixtas son:

\begin{eqnarray*}
D_{i}F_{1}=\indora_{i\geq2}D_{i}F_{1}D\theta_{1}D_{i}P_{i}+\indora_{i=2} D_{i}F_{1},
\end{eqnarray*}

entonces para
$F_{1}\left(\theta_{1}\left(\tilde{P}_{2}\hat{P}_{1}\hat{P}_{2}\right),z_{2}\right)$

$$D_{2}F_{1}=D_{1}F_{1}D_{1}\theta_{1}D_{2}\tilde{P}_{2}\left\{\hat{P}_{1}\hat{P}_{2}\right\}+D_{2}F_{1}$$

\begin{eqnarray*}
D_{1}D_{1}F_{1}&=&0\\
D_{2}D_{1}F_{1}&=&0\\
D_{3}D_{1}F_{1}&=&0\\
D_{4}D_{1}F_{1}&=&0\\
D_{1}D_{2}F_{1}&=&0\\
D_{2}D_{2}F_{1}&=&D_{1}D_{1}F_{1}D_{1}\theta_{1}D_{2}\tilde{P}_{2}D_{1}\theta_{1}D_{2}\tilde{P}_{2}+D_{2}D_{1}F_{1}D_{1}\theta_{1}D_{2}\tilde{P}_{2}
+D_{1}F_{1}D_{1}\theta_{1}D_{2}\tilde{P}_{2}+D_{1}F_{1}D_{1}\theta_{1}D_{2}D_{2}\tilde{P}_{2}+D_{2}D_{2}F_{1}\\
&+&D_{1}D_{2}F_{1}D\theta_{1}D_{2}\tilde{P}_{2}\\
D_{3}D_{2}F_{1}&=&D_{1}D_{1}F_{1}D_{3}\theta_{1}D_{3}\hat{P}_{1}D_{2}\theta_{1}D_{2}\tilde{P}_{2}+D_{1}F_{1}D_{3}D_{2}\theta_{1}D_{3}\hat{P}_{1}D_{2}\tilde{P}_{2}+D_{1}F_{1}D_{2}\theta_{1}D_{2}\tilde{P}_{2}D_{3}\hat{P}_{1}+D_{3}D_{2}F_{1}D_{3}\theta_{1}D_{3}\hat{P}_{1}\\
D_{4}D_{2}F_{1}&=&D_{1}D_{1}F_{1}D_{4}\theta_{1}D_{4}\hat{P}_{2}D_{2}\theta_{1}D_{2}\tilde{P}_{2}+D_{1}F_{1}D_{4}D_{2}\theta_{1}D_{4}\hat{P}_{2}D_{2}\tilde{P}_{2}+D_{1}F_{1}D_{2}\theta_{1}D_{2}\tilde{P}_{2}D_{4}\hat{P}_{2}+D_{4}D_{2}F_{1}D_{4}\theta_{1}D_{4}\hat{P}_{2}\\
D_{1}D_{3}F_{1}&=&0\\
D_{2}D_{3}F_{1}&=&D_{1}D_{1}F_{1}D_{2}\theta_{1}D_{2}\tilde{P}_{2}D_{3}\theta_{1}D_{3}\hat{P}_{1}+D_{2}D_{1}F_{1}D_{3}\theta_{1}D_{3}\hat{P}_{1}+D_{1}F_{1}D_{2}D_{3}\theta_{1}D_{2}\tilde{P}_{2}D_{3}\hat{P}_{1}+D_{1}F_{1}D_{3}\theta_{1}D_{3}\hat{P}_{1}D_{2}\tilde{P}_{2}\\
D_{3}D_{3}F_{1}&=&D_{1}D_{1}F_{1}D_{3}\theta_{1}D_{3}\hat{P}_{1}D_{3}\theta_{1}D_{3}\hat{P}_{1}+D_{1}F_{1}D_{3}D_{3}\theta_{1}D_{3}\hat{P}_{1}D_{3}\hat{P}_{1}+D_{1}F_{1}D_{3}\theta_{1}D_{3}D_{3}\hat{P}_{1}\\
D_{4}D_{3}F_{1}&=&D_{1}D_{1}F_{1}D_{4}\theta_{1}D_{4}\hat{P}_{2}D_{3}\theta_{1}D_{3}\hat{P}_{1}+D_{1}F_{1}D_{4}D_{3}\theta_{1}D_{4}\hat{P}_{2}D_{3}\hat{P}_{1}+D_{1}F_{1}D_{3}\theta_{1}D_{3}\hat{P}_{1}D_{4}\hat{P}_{2}\\
D_{1}D_{4}F_{1}&=&0\\
D_{2}D_{4}F_{1}&=&D_{1}D_{1}F_{1}D_{2}\theta_{1}D_{2}\tilde{P}_{2}D_{4}\theta_{1}D_{4}\hat{P}_{2}+D_{2}D_{1}F_{1}D_{4}\theta_{1}D_{4}\hat{P}_{2}+D_{1}F_{1}D_{2}D_{4}\theta_{1}D_{2}\tilde{P}_{2}D_{4}\hat{P}_{2}+D_{1}F_{1}D_{4}\theta_{1}D_{4}\hat{P}_{2}D_{2}\tilde{P}_{2}\\
D_{3}D_{4}F_{1}&=&D_{1}D_{1}F_{1}D_{3}\theta_{1}D_{3}\hat{P}_{1}D_{4}\theta_{1}D_{4}\hat{P}_{2}+D_{1}F_{1}D_{3}D_{4}\theta_{1}D_{3}\hat{P}_{1}D_{4}\hat{P}_{2}+D_{1}F_{1}D_{4}\theta_{1}D_{4}\hat{P}_{2}D_{3}\hat{P}_{1}\\
D_{4}D_{4}F_{1}&=&D_{1}D_{1}F_{1}D_{4}\theta_{1}D_{4}\hat{P}_{2}D_{4}\theta_{1}D_{4}\hat{P}_{2}+D_{1}F_{1}D_{4}D_{4}\theta_{1}D_{4}\hat{P}_{2}D_{4}\hat{P}_{2}+D_{1}F_{1}D_{4}\theta_{1}D_{4}D_{4}\hat{P}_{2}
\end{eqnarray*}


%\newpage

Para $F_{2}\left(z_{1},\tilde{\theta}_{2}\left(P_{1}\hat{P}_{1}\hat{P}_{2}\right)\right)$

\begin{eqnarray*}
D_{i}F_{2}=\indora_{i\neq2}D_{2}F_{2}D\tilde{\theta}_{2}D_{i}P_{i}+\indora_{i=1} D_{i}F_{2},
\end{eqnarray*}



\begin{eqnarray*}
D_{1}D_{1}F_{2}&=&\left(D_{2}D_{2}F_{2}D_{1}\tilde{\theta}_{2}D_{1}P_{1}+D_{1}D_{2}F_{2}\right)D_{2}\tilde{\theta}_{2}D_{1}P_{1}+D_{2}F_{2}D_{1}D_{2}\tilde{\theta}_{2}D_{1}P_{1}+D_{2}F_{2}D_{2}\tilde{\theta}_{2}D_{1}D_{1}P_{1}+D_{1}D_{1}F_{2}\\
D_{2}D_{1}F_{2}&=&0\\
D_{3}D_{1}F_{2}&=&D_{2}D_{1}F_{2}D_{3}\tilde{\theta}_{2}D_{3}\hat{P}_{1}+D_{2}D_{2}F_{2}D_{3}\tilde{\theta}_{2}D_{3}P_{1}D_{2}\tilde{\theta}_{2}D_{1}P_{1}+D_{2}F_{2}D_{3}D_{2}\tilde{\theta}_{2}D_{3}\hat{P}_{1}D_{1}P_{1}+D_{2}F_{2}D_{2}\tilde{\theta}_{2}D_{1}P_{1}D_{3}\hat{P}_{1}\\
D_{4}D_{1}F_{2}&=&D_{2}D_{1}F_{2}D_{4}\tilde{\theta}_{2}D_{4}\hat{P}_{2}+D_{2}D_{2}F_{2}D_{4}\tilde{\theta}_{2}D_{4}P_{2}D_{4}\tilde{\theta}_{2}D_{1}P_{1}+D_{2}F_{2}D_{4}D_{2}\tilde{\theta}_{2}D_{4}\hat{P}_{2}D_{1}P_{1}+D_{2}F_{2}D_{2}\tilde{\theta}_{2}D_{1}P_{1}D_{4}\hat{P}_{2}\\
D_{1}D_{3}F_{2}&=&\left(D_{2}D_{2}F_{2}D_{1}\tilde{\theta}_{2}D_{1}P_{1}+D_{1}D_{2}F_{2}\right)D_{3}\tilde{\theta}_{2}D_{3}\hat{P}_{1}+D_{2}F_{2}D_{1}D_{3}\tilde{\theta}_{2}D_{1}P_{1}D_{3}\hat{P}_{1}+D_{2}F_{2}D_{3}\tilde{\theta}_{2}D_{3}\hat{P}_{1}D_{1}P_{1}\\
D_{2}D_{3}F_{3}&=&0\\
D_{3}D_{3}F_{2}&=&D_{2}D_{2}F_{2}D_{3}\tilde{\theta}_{2}D_{3}\hat{P}_{1}D_{3}\tilde{\theta}_{2}D_{3}\hat{P}_{1}+D_{2}F_{2}D_{3}D_{3}\tilde{\theta}_{2}D_{3}\hat{P}_{1}D_{3}\hat{P}_{1}+D_{2}F_{2}D_{3}\tilde{\theta}_{2}D_{3}D_{3}\hat{P}_{1}\\
D_{4}D_{3}F_{2}&=&D_{2}D_{2}F_{2}D_{4}\tilde{\theta}_{2}D_{4}\hat{P}_{2}D_{3}\tilde{\theta}_{2}D_{3}\hat{P}_{1}+D_{2}F_{2}D_{4}D_{3}\tilde{\theta}_{2}D_{4}\hat{P}_{2}D_{3}\hat{P}_{1}+D_{2}F_{2}D_{3}\tilde{\theta}_{2}D_{3}\hat{P}_{1}D_{4}\hat{P}_{2}\\
D_{1}D_{4}F_{2}&=&\left(D_{2}D_{2}F_{2}D_{4}\tilde{\theta}_{2}D_{1}P_{1}+D_{1}D_{2}F_{2}\right)D_{4}\tilde{\theta}_{2}D_{4}\hat{P}_{2}+D_{2}F_{2}D_{1}D_{4}\tilde{\theta}_{2}D_{1}P_{1}D_{4}\hat{P}_{2}+D_{2}F_{2}D_{4}\tilde{\theta}_{2}D_{4}\hat{P}_{2}D_{1}P_{1}\\
D_{2}D_{4}F_{2}&=&0\\
D_{3}D_{4}F_{2}&=&D_{2}F_{2}D_{4}\tilde{\theta}_{2}D_{4}\hat{P}_{2}D_{3}\hat{P}_{1}+D_{2}F_{2}D_{3}D_{4}\tilde{\theta}_{2}D_{4}\hat{P}_{2}D_{3}\hat{P}_{1}+D_{2}F_{2}D_{4}\tilde{\theta}_{2}D_{4}\hat{P}_{2}D_{3}\hat{P}_{1}\\
D_{4}D_{4}F_{2}&=&D_{2}F_{2}D_{4}\tilde{\theta}_{2}D_{4}D_{4}\hat{P}_{2}+D_{2}F_{2}D_{4}D_{4}\tilde{\theta}_{2}D_{4}\hat{P}_{2}D_{4}\hat{P}_{2}+D_{2}F_{2}D_{4}\tilde{\theta}_{2}D_{4}\hat{P}_{2}D_{4}\hat{P}_{2}\\
\end{eqnarray*}


%\newpage



%\newpage

para $\hat{F}_{1}\left(\hat{\theta}_{1}\left(P_{1}\tilde{P}_{2}\hat{P}_{2}\right),w_{2}\right)$

\begin{eqnarray*}
D_{i}\hat{F}_{1}=\indora_{i\neq3}D_{3}\hat{F}_{1}D\hat{\theta}_{1}D_{i}P_{i}+\indora_{i=4}D_{i}\hat{F}_{1},
\end{eqnarray*}


\begin{eqnarray*}
D_{1}D_{1}\hat{F}_{1}&=&D_{1}\hat{\theta}_{1}D_{1}D_{1}P_{1}D_{1}\hat{F}_{1}
+D_{1}P_{1}D_{1}P_{1}D_{1}D_{1}\hat{\theta}_{1}D_{1}\hat{F}_{1}+
D_{1}P_{1}D_{1}P_{1}D_{1}\hat{\theta}_{1}D_{1}\hat{\theta}_{1}
D_{1}D_{1}\hat{F}_{1}\\
D_{1}D_{1}\hat{F}_{1}&=&D_{1}P_{1}D_{2}P_{1}D\hat{\theta}_{1}D_{1}\hat{F}_{1}+
D_{1}P_{1}D_{2}P_{1}DD\hat{\theta}_{1}D_{1}\hat{F}_{1}+
D_{1}P_{1}D_{2}P_{1}D\hat{\theta}_{1}D\hat{\theta}_{1}D_{1}D_{1}\hat{\theta}_{1}\\
D_{3}D_{1}\hat{F}_{1}&=&0\\
D_{4}D_{1}\hat{F}_{1}&=&D_{1}P_{1}D_{4}\hat{P}_{2}D\hat{\theta}_{1}D_{1}\hat{F}_{1}
+D_{1}D_{4}\hat{P}_{2}DD\hat{\theta}_{1}D_{1}\hat{F}_{1}
+D_{1}D\hat{\theta}_{1}\left(D_{2}D{1}\hat{F}_{1}
+D_{4}P_{2}D\hat{\theta}_{1}D_{1}D_{1}\hat{F}_{1}\right)\\
D_{1}D_{2}\hat{F}_{1}&=&D_{1}P_{1}D_{2}P_{2}D\hat{\theta}_{1}D_{1}\hat{F}_{1}+
D_{1}P_{1}D_{2}P_{2}DD\hat{\theta}_{1}D_{1}\hat{F}_{1}+
D_{1}P_{1}D_{2}P_{2}D\hat{\theta}_{1}D\hat{\theta}_{1}D_{1}D_{1}\hat{F}_{1}\\
D_{2}D_{2}\hat{F}_{1}&=&D\hat{\theta}_{1}D_{2}D_{2}P_{2}D_{1}\hat{F}_{1}+ D_{2}P_{2}D_{2}P_{2}DD\hat{\theta}_{1}D_{1}\hat{F}_{1}+
D_{2}P_{2}D_{2}P_{2}D\hat{\theta}_{1}D\hat{\theta}_{1}
D_{1}D_{1}\hat{F}_{1}\\
D_{3}D_{2}\hat{F}_{1}&=&0\\
D_{4}D_{2}\hat{F}_{1}&=&D_{2}P_{2}D_{4}\hat{P}_{2}D\hat{\theta} _{1}D\hat{F}_{1}+D_{2}P_{2}D_{4}\hat{P}_{2}DD\hat{\theta}_{1}D_{1}\hat{F}_{1} +D_{2}P_{2}D\hat{\theta}_{1}\left(D_{2}D_{1}\hat{F}_{1}+ D_{2}\hat{P}_{2}D\hat{\theta}_{1}D_{1}D_{1}\hat{F}_{1}\right)\\
D_{1}D_{3}\hat{F}_{1}&=&0\\
D_{2}D_{3}\hat{F}_{1}&=&0\\
D_{3}D_{3}\hat{F}_{1}&=&0\\
D_{4}D_{3}\hat{F}_{1}&=&0\\
D_{1}D_{4}\hat{F}_{1}&=&D_{1}P_{1}D_{4}\hat{F}_{2}D\hat{\theta}_{1}D_{1}
\hat{F}_{1}+D_{1}P_{1}D_{4}\hat{P}_{2}DD\hat{\theta}_{1}D_{1}\hat{F}_{1}+D_{1}P_{1}D\hat{\theta}_{1}D_{2}D_{1}\hat{F}_{1}+ D_{1}P_{1}D_{4}\hat{P}_{2}D\hat{\theta}_{1}D\hat{\theta}_{1}D_{1}D_{1}
\hat{F}_{1}\\
D_{2}D_{4}\hat{F}_{1}&=&D_{2}P_{2}D_{4}\hat{P}_{2}D\hat{\theta}_{1}D_{1}
\hat{F}_{1}+D_{2}P_{2}D_{4}\hat{P}_{2}DD\hat{\theta}_{1}D_{1}\hat{F}_{1}+D_{2}P_{2}D\hat{\theta}_{1}D_{2}D_{1}\hat{F}_{1}+
D_{2}P_{2}D_{4}\hat{P}_{2}D\hat{\theta}_{1}D\hat{\theta}_{1}D_{1}D_{1}\hat{F}_{1}\\
D_{3}D_{4}\hat{F}_{1}&=&0\\
D_{4}D_{4}\hat{F}_{1}&=&D_{2}D_{2}\hat{F}_{1}+D\hat{\theta}_{1}D_{4}D_{4}\hat{F}_{2}+ D_{1}\hat{F}_{1}+
D_{4}\hat{P}_{2}D_{4}\hat{P}_{2}DD\hat{\theta}_{1}D_{1}\hat{F}_{1}+
D_{4}\hat{P}_{2}D\hat{\theta}_{1}D_{2}D_{1}\hat{F}_{1}\\
&+&D_{4}\hat{P}_{2}D\hat{\theta}_{1}\left(D_{2}D_{1}\hat{F}_{1}+ D_{4}\hat{P}_{2}D\hat{\theta}_{1}D_{1}D_{1}\hat{F}_{1}\right)\\
\end{eqnarray*}




%\newpage
finalmente, para $\hat{F}_{2}\left(w_{1},\hat{\theta}_{2}\left(P_{1}\tilde{P}_{2}\hat{P}_{1}\right)\right)$

\begin{eqnarray*}
D_{i}\hat{F}_{2}=\indora_{i\neq4}D_{4}\hat{F}_{2}D\hat{\theta}_{2}D_{i}P_{i}+\indora_{i=3}D_{i}\hat{F}_{2},
\end{eqnarray*}

\begin{eqnarray*}
D_{1}D_{1}\hat{F}_{2}&=&D_{1}\hat{\theta}_{2}D_{2}D_{2}P_{1}D_{2}
\hat{F}_{2}
+D_{1}P_{1}D_{1}P_{1}D_{1}D_{1}\hat{\theta}_{2}D_{2}\hat{F}_{2}+
D_{1}P_{1}D_{1}P_{1}D_{1}\hat{\theta}_{2}D_{1}\hat{\theta}_{2}
D_{1}D_{1}\hat{F}_{2}\\
D_{2}D_{1}\hat{F}_{2}&=&D_{1}P_{1}D_{2}P_{2}D\hat{\theta}_{2}D_{2}
\hat{F}_{2}+
D_{1}P_{1}D_{2}P_{2}DD\hat{\theta}_{2}D_{2}\hat{F}_{2}+
D_{1}P_{1}D_{2}P_{2}D\hat{\theta}_{2}D\hat{\theta}_{2}D_{2}
D_{2}\hat{\theta}_{2}\\
D_{3}D_{1}\hat{F}_{2}&=&D_{1}P_{1}D_{3}\hat{P}_{1}D\hat{\theta}_{2}
D_{2}\hat{F}_{2}
+D_{1}P_{1}D_{3}\hat{P}_{1}DD\hat{\theta}_{2}D_{2}\hat{F}_{2}
+D_{1}P_{1}D\hat{\theta}_{2}\left(D_{2}D{1}\hat{F}_{2}
+D_{3}\hat{P}_{1}D\hat{\theta}_{2}D_{2}D_{2}\hat{F}_{2}\right)\\
D_{4}D_{1}\hat{F}_{2}&=&0\\
D_{1}D_{2}\hat{F}_{2}&=&D_{1}P_{1}D_{2}P_{2}D\hat{\theta}_{2}D_{2}\hat{F}_{2}+
D_{1}P_{1}D_{2}P_{2}DD\hat{\theta}_{2}D_{2}\hat{F}_{2}+
D_{1}P_{1}D_{2}P_{2}D\hat{\theta}_{2}D\hat{\theta}_{2}D_{2}D_{2}\hat{F}_{2}\\
D_{2}D_{2}\hat{F}_{2}&=&DD\hat{\theta}_{2}D_{2}D_{2}P_{2}D_{2}\hat{F}_{2}+ D_{2}P_{2}D_{2}P_{2}DD\hat{\theta}_{2}D_{2}\hat{F}_{2}+
D_{2}P_{2}D_{2}P_{2}D\hat{\theta}_{2}D\hat{\theta}_{2} D_{2}D_{2}\hat{F}_{2}\\
D_{3}D_{2}\hat{F}_{2}&=&D_{2}P_{2}D_{3}\hat{P}_{1}D\hat{\theta} _{2}D_{2}\hat{F}_{2}+D_{2}P_{2}D_{3}\hat{P}_{1}DD\hat{\theta}_{2}
D_{2}\hat{F}_{2}
+D_{2}P_{2}D\hat{\theta}_{2}\left(D_{2}D_{1}\hat{F}_{2}+ D_{3}\hat{P}_{1}D\hat{\theta}_{2}D_{2}D_{2}\hat{F}_{2}\right)\\
D_{4}D_{2}\hat{F}_{2}&=&0\\
D_{1}D_{3}\hat{F}_{2}&=&
D_{1}P_{1}D_{3}\hat{P}_{1}D\hat{\theta}_{2}D_{2}\hat{F}_{2}
+D_{1}P_{1}D_{3}\hat{P}_{1}DD\hat{\theta}_{2}D_{2}\hat{F}_{2}
+D_{1}P_{1}D\hat{\theta}_{2}D\hat{\theta}_{2}D_{2}D_{2}\hat{F}_{2}
+D_{1}P_{1}D\hat{\theta}_{1}D_{2}D_{1}\hat{F}_{2}\\
D_{2}D_{3}\hat{F}_{2}&=&
D_{2}P_{2}D_{3}\hat{P}_{1}D\hat{\theta}_{2}D_{2}\hat{F}_{2}
+D_{2}P_{2}D_{3}\hat{P}_{1}DD\hat{\theta}_{2}D_{2}\hat{F}_{2}
+D_{2}P_{2}D_{3}\hat{P}_{1}D\hat{\theta}_{2}D_{2}D_{2}\hat{F}_{2}
+D_{2}P_{2}D\hat{\theta}_{2}D\hat{\theta}_{2}D_{1}D_{2}\hat{F}_{2}\\
D_{4}D_{3}\hat{F}_{2}&=&
D_{3}D_{3}\hat{P}_{1}D\hat{\theta}_{2}D_{2}\hat{F}_{2}
+D_{3}\hat{P}_{1}D_{3}\hat{P}_{1}DD\hat{\theta}_{2}D_{2}\hat{F}_{2}
+D_{3}\hat{P}_{1}D\hat{\theta}_{2}D_{1}D_{2}\hat{F}_{2}
+D_{3}\hat{P}_{1}D\hat{\theta}_{2}\left(D_{3}\hat{P}_{1}D\hat{\theta}_{2}
D_{2}D_{2}\hat{F}_{2}+D_{1}D_{2}\hat{F}_{2}\right)\\
D_{4}D_{3}\hat{F}_{2}&=&0\\
D_{1}D_{4}\hat{F}_{2}&=&0\\
D_{2}D_{4}\hat{F}_{2}&=&0\\
D_{3}D_{4}\hat{F}_{2}&=&0\\
D_{4}D_{4}\hat{F}_{2}&=&0\\
\end{eqnarray*}

%__________________________________________________________________
\section{Aplicaciones}
%__________________________________________________________________

%__________________________________________________________________
\subsection{Ejemplo 1: Automatizaci\'on en dos l\'ineas de trabajo}
%__________________________________________________________________
Consideremos dos l\'ineas de producci\'on atendidas cada una de ellas por un robot, en las que en una de ellas un robot realiza la misma actividad en dos estaciones distintas, una vez que termina de realizar una actividad en una de las colas, se desplaza a la siguiente para hacer lo correspondientes con los materiales presentes en la estaci\'on. Una vez que las piezas son liberadas por el robot se desplazan al otro sistema en donde son objeto del terminado de la pieza para su almacenamiento. En este caso el sistema 1 consta de una sola cola de tipo $M/M/1$ y el sistema 2 es un sistema de visitas c\'iclicas conformado por dos colas id\'enticas, donde al igual que antes, el traslado de un sistema a otro se realiza de la cola $\hat{Q}_{2}$ a la \'unica cola $Q_{1}$ del sistema 1.

%\begin{figure}[H]
%\centering
%%%\includegraphics[width=9cm]{Grafica1.jpg}
%%\end{figure}\label{RSVC1}



El n\'umero de usuarios presentes en el sistema 1 se sigue modelando conforme a un SVC, mientras que para es sistema 1, $Q_{1}$ se comporta como una Red de Jackson, una red conformada por $\hat{Q}_{2}$ y $Q_{1}$, donde el n\'umero de usuarios que llegan a $Q_{1}$ lo hacen de acuerdo a su propio proceso de arribos m\'as los que provienen del sistema 2, los tiempos entre arribos de los usuarios procedentes del sistema 2, lo hacen conforme a una distribuci\'on exponencial.

Las ecuaciones recursivas son


\begin{eqnarray*}
F_{1}\left(z_{1},w_{1},w_{2}\right)&=&R\left(\tilde{P}_{2}\left(z_{2}\right)\prod_{i=1}^{2}
\hat{P}_{i}\left(w_{i}\right)\right)F_{2}\left(\tilde{\theta}_{2}\left(\hat{P}_{1}\left(w_{1}\right)\hat{P}_{2}\left(w_{2}\right)\right)\right)
\hat{F}_{2}\left(w_{1},w_{2};\tau_{2}\right),
\end{eqnarray*}

\begin{eqnarray*}
\hat{F}_{1}\left(z_{1},w_{1},w_{2}\right)&=&\hat{R}_{2}\left(\tilde{P}_{2}\left(z_{2}\right)\prod_{i=1}^{2}
\hat{P}_{i}\left(w_{i}\right)\right)F_{2}\left(z_{1};\zeta_{2}\right)\hat{F}_{2}\left(w_{1},\hat{\theta}_{2}\left(\tilde{P}_{2}\left(z_{2}\right)\hat{P}_{1}\left(w_{1}
\right)\right)\right),
\end{eqnarray*}


\begin{eqnarray*}
\hat{F}_{2}\left(z_{1},w_{1},w_{2}\right)&=&\hat{R}_{1}\left(\tilde{P}_{2}\left(z_{2}\right)\prod_{i=1}^{2}
\hat{P}_{i}\left(w_{i}\right)\right)F_{1}\left(z_{1};\zeta_{1}\right)\hat{F}_{1}\left(\hat{\theta}_{1}\left(\tilde{P}_{2}\left(z_{2}\right)\hat{P}_{2}\left(w_{2}\right)\right),w_{2}\right),
\end{eqnarray*}




%__________________________________________________________________
\subsection{Ejemplo 2: Sistema de Salud P\'ublica}
%__________________________________________________________________

Consideremos un hospital en el \'area de urgencias, donde hay una ventanilla a la cu\'al van llegando todos los posibles pacientes para su valoraci\'on, despu\'es de la cual pueden o ser canalizados a un \'area de atenci\'on que requiera de atenci\'on sin llegar a ser urgencia, o puede abandonar el sistema dependiendo de la valoraci\'on hecha por el m\'edico en turno. Por otra parte, hay una secci\'on del hospital en la que son atendidas las personas sin necesidad de pasar por la ventanilla de valoraci\'on, es decir, son atenciones de urgencia. Las personas que despu\'es de la valoraci\'on son turnadas al \'area de atenci\'on deben de esperar su turno pues a esta secci\'on tambi\'en llegan pacientes provenientes de otras \'areas del hospital. Para este caso, el sistema 1 est\'a conformado por una \'unica cola $Q_{1}$ que podemos asumir sin p\'erdida de generalidad que es de tipo $M/M/1$, mientras que el sistema 2 es un SVC como los hasta ahora estudiados. Es decir, en este caso en particular el servidor del sistema 1 da servicio de manera ininterrumpida en $Q_{1}$ en tanto no se vac\'ie la cola.




%\begin{figure}[H]
%\centering
%%%\includegraphics[width=9cm]{Grafica2.jpg}
%%\end{figure}\label{RSVC2}

Las ecuaciones recursivas son de la forma


\begin{eqnarray*}
F_{1}\left(z_{1},z_{2},w_{1}\right)&=&R_{2}\left(P_{1}\left(z_{1}\right)\tilde{P}_{2}\left(z_{2}\right)
\hat{P}_{1}\left(w_{1}\right)\right)F_{2}\left(z_{1},\tilde{\theta}_{2}\left(P_{1}\left(z_{1}\right)\hat{P}_{1}\left(w_{1}\right)\right)\right)
\hat{F}_{2}\left(w_{1};\tau_{2}\right),
\end{eqnarray*}


\begin{eqnarray*}
F_{2}\left(z_{1},z_{2},w_{1}\right)&=&R_{1}\left(P_{1}\left(z_{1}\right)\tilde{P}_{2}\left(z_{2}\right)
\hat{P}_{1}\left(w_{1}\right)\right)F_{1}\left(\theta_{1}\left(\hat{P}_{1}\left(w_{1}\right)\hat{P}_{2}\left(w_{2}\right)\right),z_{2}\right)\hat{F}_{1}\left(w_{1};\tau_{1}\right),
\end{eqnarray*}



\begin{eqnarray*}
\hat{F}_{1}\left(z_{1},z_{2},w_{1}\right)&=&\hat{R}_{2}\left(P_{1}\left(z_{1}\right)\tilde{P}_{2}\left(z_{2}\right)
\hat{P}_{1}\left(w_{1}\right)\right)F_{2}\left(z_{1},z_{2};\zeta_{2}\right)\hat{F}_{}\left(\hat{\theta}_{1}\left(P_{1}\left(z_{1}\right)\tilde{P}_{2}\left(z_{2}\right)
\right)\right),
\end{eqnarray*}


%__________________________________________________________________
\subsection{Ejemplo 3: RSVC con dos conexiones}
%__________________________________________________________________

Al igual que antes consideremos una RSVC conformada por dos SVC que se comunican entre s\'i en $\hat{Q}_{2}$ y $Q_{2}$, permitiendo el paso de los usuarios del sistema 2 hacia el sistema 1. Ahora supongamos que tambi\'en se permite el paso en $\hat{Q}_{1}$ hacia $Q_{1}$.

%\begin{figure}[H]
%\centering
%%%\includegraphics[width=9cm]{Grafica3.jpg}
%%\end{figure}\label{RSVC3}


Cuyas ecuaciones recursivas son de la forma


\begin{eqnarray*}
F_{1}\left(z_{1},z_{2},w_{1},w_{2}\right)&=&R_{2}\left(\tilde{P}_{1}\left(z_{1}\right)\tilde{P}_{2}\left(z_{2}\right)\prod_{i=1}^{2}
\hat{P}_{i}\left(w_{i}\right)\right)F_{2}\left(z_{1},\tilde{\theta}_{2}\left(\tilde{P}_{1}\left(z_{1}\right)\hat{P}_{1}\left(w_{1}\right)\hat{P}_{2}\left(w_{2}\right)\right)\right)
\hat{F}_{2}\left(w_{1},w_{2};\tau_{2}\right),
\end{eqnarray*}

\begin{eqnarray*}
F_{2}\left(z_{1},z_{2},w_{1},w_{2}\right)&=&R_{1}\left(\tilde{P}_{1}\left(z_{1}\right)\tilde{P}_{2}\left(z_{2}\right)\prod_{i=1}^{2}
\hat{P}_{i}\left(w_{i}\right)\right)F_{1}\left(\tilde{\theta}_{1}\left(\tilde{P}_{2}\left(z_{2}\right)\hat{P}_{1}\left(w_{1}\right)\hat{P}_{2}\left(w_{2}\right)\right),z_{2}\right)\hat{F}_{1}\left(w_{1},w_{2};\tau_{1}\right),
\end{eqnarray*}


\begin{eqnarray*}
\hat{F}_{1}\left(z_{1},z_{2},w_{1},w_{2}\right)&=&\hat{R}_{2}\left(\tilde{P}_{1}\left(z_{1}\right)\tilde{P}_{2}\left(z_{2}\right)\prod_{i=1}^{2}
\hat{P}_{i}\left(w_{i}\right)\right)F_{2}\left(z_{1},z_{2};\zeta_{2}\right)\hat{F}_{2}\left(w_{1},\hat{\theta}_{2}\left(\tilde{P}_{1}\left(z_{1}\right)\tilde{P}_{2}\left(z_{2}\right)\hat{P}_{1}\left(w_{1}
\right)\right)\right),
\end{eqnarray*}


\begin{eqnarray*}
\hat{F}_{2}\left(z_{1},z_{2},w_{1},w_{2}\right)&=&\hat{R}_{1}\left(\tilde{P}_{1}\left(z_{1}\right)\tilde{P}_{2}\left(z_{2}\right)\prod_{i=1}^{2}
\hat{P}_{i}\left(w_{i}\right)\right)F_{1}\left(z_{1},z_{2};\zeta_{1}\right)\hat{F}_{1}\left(\hat{\theta}_{1}\left(\tilde{P}_{1}\left(z_{1}\right)\tilde{P}_{2}\left(z_{2}\right)\hat{P}_{2}\left(w_{2}\right)\right),w_{2}\right),
\end{eqnarray*}




%_____________________________________________________
\subsubsection{Queue lengths for server times in the System}
%_____________________________________________________

Now, we obtain the first moments equations for the queue lengths as before for the times the server arrives to the queue to start attending



Remember that


\begin{eqnarray*}
F_{2}\left(z_{1},z_{2},w_{1},w_{2}\right)&=&R_{1}\left(\prod_{i=1}^{4}\tilde{P}_{i}\left(z_{i}\right)\right)
F_{1}\left(\tilde{\theta}_{1}\left(\tilde{P}_{2}\left(z_{2}\right)\tilde{P}_{3}\left(z_{3}\right)\tilde{P}_{4}\left(z_{4}\right)\right),z_{2}\right)
F_{3}\left(z_{3},z_{4};\tau_{1}\right),
\end{eqnarray*}

where


\begin{eqnarray*}
F_{1}\left(\tilde{\theta}_{1}\left(\tilde{P}_{2}\tilde{P}_{3}\tilde{P}_{4}\right),z_{2}\right)
\end{eqnarray*}

so

\begin{eqnarray}
D_{i}F_{1}&=&\indora_{i\neq1}D_{1}F_{1}D\tilde{\theta}_{1}D_{i}\tilde{P}_{i}+\indora_{i=2}D_{i}F_{1},
\end{eqnarray}

then


\begin{eqnarray*}
\begin{array}{ll}
D_{1}F_{1}=0,&
D_{2}F_{1}=D_{1}F_{1}D\tilde{\theta}_{1}D_{2}\tilde{P}_{2}+D_{2}F_{1}
=f_{1}\left(1\right)\frac{1}{1-\tilde{\mu}_{1}}\tilde{\mu}_{2}+f_{1}\left(2\right),\\
D_{3}F_{1}=D_{1}F_{1}D\tilde{\theta}_{1}D_{3}\tilde{P}_{3}
=f_{1}\left(1\right)\frac{1}{1-\tilde{\mu}_{1}}\tilde{\mu}_{3},&
D_{4}F_{1}=D_{1}F_{1}D\tilde{\theta}_{1}D_{4}\tilde{P}_{4}
=f_{1}\left(1\right)\frac{1}{1-\tilde{\mu}_{1}}\tilde{\mu}_{4}

\end{array}
\end{eqnarray*}


\begin{eqnarray}
D_{i}F_{2}&=&\indora_{i\neq2}D_{2}F_{2}D\tilde{\theta}_{2}D_{i}\tilde{P}_{i}
+\indora_{i=1}D_{i}F_{2}
\end{eqnarray}

\begin{eqnarray*}
\begin{array}{ll}
D_{1}F_{2}=D_{2}F_{2}D\tilde{\theta}_{2}D_{1}\tilde{P}_{1}
+D_{1}F_{2}=f_{2}\left(2\right)\frac{1}{1-\tilde{\mu}_{2}}\tilde{\mu}_{1},&
D_{2}F_{2}=0\\
D_{3}F_{2}=D_{2}F_{2}D\tilde{\theta}_{2}D_{3}\tilde{P}_{3}
=f_{2}\left(2\right)\frac{1}{1-\tilde{\mu}_{2}}\tilde{\mu}_{3},&
D_{4}F_{2}=D_{2}F_{2}D\tilde{\theta}_{2}D_{4}\tilde{P}_{4}
=f_{2}\left(2\right)\frac{1}{1-\tilde{\mu}_{2}}\tilde{\mu}_{4}
\end{array}
\end{eqnarray*}



\begin{eqnarray}
D_{i}F_{3}&=&\indora_{i\neq3}D_{3}F_{3}D\hat{\theta}_{1}D_{i}\tilde{P}_{i}+\indora_{i=4}D_{i}F_{3},
\end{eqnarray}

\begin{eqnarray*}
\begin{array}{ll}
D_{1}F_{3}=D_{3}F_{3}D\tilde{\theta}_{3}D_{1}\tilde{P}_{1}=F_{3}\left(3\right)\frac{1}{1-\tilde{\mu}_{3}}\tilde{\mu}_{1},&
D_{2}F_{3}=D_{3}F_{3}D\tilde{\theta}_{3}D_{2}\tilde{P}_{2}
=F_{3}\left(3\right)\frac{1}{1-\tilde{\mu}_{3}}\tilde{\mu}_{2}\\
D_{3}F_{3}=0,&
D_{4}F_{3}=D_{3}F_{3}D\tilde{\theta}_{3}D_{4}\tilde{P}_{4}
+D_{4}F_{3}
=F_{3}\left(3\right)\frac{1}{1-\tilde{\mu}_{3}}\tilde{\mu}_{4}+F_{3}\left(2\right),

\end{array}
\end{eqnarray*}


\begin{eqnarray}
D_{i}F_{4}&=&\indora_{i\neq4}D_{4}F_{4}D\tilde{\theta}_{4}D_{i}\tilde{P}_{i}+\indora_{i=3}D_{i}F_{4}.
\end{eqnarray}

\begin{eqnarray*}
\begin{array}{ll}
D_{1}F_{4}=D_{4}F_{4}D\tilde{\theta}_{4}D_{1}\tilde{P}_{1}
=F_{4}\left(4\right)\frac{1}{1-\tilde{\mu}_{4}}\tilde{\mu}_{1},&
D_{2}F_{4}=D_{4}F_{4}D\tilde{\theta}_{4}D_{2}\tilde{P}_{2}
=F_{4}\left(4\right)\frac{1}{1-\tilde{\mu}_{4}}\tilde{\mu}_{2},\\
D_{3}F_{4}=D_{4}F_{4}D\tilde{\theta}_{4}D_{3}\tilde{P}_{3}+D_{3}F_{4}
=F_{4}\left(4\right)\frac{1}{1-\tilde{\mu}_{4}}\tilde{\mu}_{3}+F_{4}\left(4\right)\\
D_{4}F_{4}=0

\end{array}
\end{eqnarray*}



%_____________________________________________________
\subsubsection{Recursive Equations for the Cyclic Polling System}
%_____________________________________________________

Then, now we can obtain the linear system of equations in order to obtain the first moments of the lengths of the queues:



For $\mathbf{F}_{1}=R_{2}F_{2}F_{4}$ we get the general equations

\begin{eqnarray}
D_{i}\mathbf{F}_{1}=D_{i}\left(R_{2}+F_{2}+\indora_{i\geq3}F_{4}\right)
\end{eqnarray}

So

\begin{eqnarray*}
D_{1}\mathbf{F}_{1}&=&D_{1}R_{2}+D_{1}F_{2}
=r_{1}\tilde{\mu}_{1}+f_{2}\left(2\right)\frac{1}{1-\tilde{\mu}_{2}}\tilde{\mu}_{1}\\
D_{2}\mathbf{F}_{1}&=&D_{2}\left(R_{2}+F_{2}\right)
=r_{2}\tilde{\mu}_{1}\\
D_{3}\mathbf{F}_{1}&=&D_{3}\left(R_{2}+F_{2}+F_{4}\right)
=r_{1}\tilde{\mu}_{3}+f_{2}\left(2\right)\frac{1}{1-\tilde{\mu}_{2}}\tilde{\mu}_{3}+F_{3,2}^{(1)}\left(1\right)\\
D_{4}\mathbf{F}_{1}&=&D_{4}\left(R_{2}+F_{2}+F_{4}\right)
=r_{2}\tilde{\mu}_{4}+f_{2}\left(2\right)\frac{1}{1-\tilde{\mu}_{2}}\tilde{\mu}_{4}
+F_{4,2}^{(1)}\left(1\right)
\end{eqnarray*}

it means

\begin{eqnarray*}
\begin{array}{ll}
D_{1}\mathbf{F}_{1}=r_{2}\tilde{\mu}_{3}+f_{2}\left(2\right)\left(\frac{1}{1-\tilde{\mu}_{2}}\right)\tilde{\mu}_{1}+f_{2}\left(1\right),&
D_{2}\mathbf{F}_{1}=r_{2}\tilde{\mu}_{2},\\
D_{3}\mathbf{F}_{1}=r_{2}\tilde{\mu}_{3}+f_{2}\left(2\right)\left(\frac{1}{1-\tilde{\mu}_{2}}\right)\tilde{\mu}_{3}+F_{3,2}^{(1)}\left(1\right),&
D_{4}\mathbf{F}_{1}=r_{2}\tilde{\mu}_{4}+f_{2}\left(2\right)\left(\frac{1}{1-\tilde{\mu}_{2}}\right)\tilde{\mu}_{4}+F_{4,2}^{(1)}\left(1\right),\end{array}
\end{eqnarray*}


\begin{eqnarray}
\begin{array}{ll}
\mathbf{F}_{2}=R_{1}F_{1}F_{3}, & D_{i}\mathbf{F}_{2}=D_{i}\left(R_{1}+F_{1}+\indora_{i\geq3}F_{3}\right)\\
\end{array}
\end{eqnarray}



equivalently


\begin{eqnarray*}
\begin{array}{ll}
D_{1}\mathbf{F}_{2}=r_{1}\tilde{\mu}_{1},&
D_{2}\mathbf{F}_{2}=r_{1}\tilde{\mu}_{2}+f_{1}\left(1\right)\left(\frac{1}{1-\tilde{\mu}_{1}}\right)\tilde{\mu}_{2}+f_{1}\left(2\right),\\
D_{3}\mathbf{F}_{2}=r_{1}\tilde{\mu}_{3}+f_{1}\left(1\right)\left(\frac{1}{1-\tilde{\mu}_{1}}\right)\tilde{\mu}_{3}+F_{3,1}^{(1)}\left(1\right),&
D_{4}\mathbf{F}_{2}=r_{1}\tilde{\mu}_{4}+f_{1}\left(1\right)\left(\frac{1}{1-\tilde{\mu}_{1}}\right)\tilde{\mu}_{4}+F_{4,,1}^{(1)}\left(1\right),\\
\end{array}
\end{eqnarray*}



\begin{eqnarray}
\begin{array}{ll}
\hat{\mathbf{F}}_{1}=\hat{R}_{2}F_{4}F_{2}, & D_{i}\hat{\mathbf{F}}_{1}=D_{i}\left(\hat{R}_{2}+F_{4}+\indora_{i\leq2}F_{2}\right)\\
\end{array}
\end{eqnarray}


equivalently


\begin{eqnarray*}
\begin{array}{ll}
D_{1}\hat{\mathbf{F}}_{1}=\hat{r}_{2}\tilde{\mu}_{1}+F_{4}\left(2\right)\left(\frac{1}{1-\tilde{\mu}_{4}}\right)\tilde{\mu}_{1}+F_{1,4}^{(1)}\left(1\right),&
D_{2}\hat{\mathbf{F}}_{1}=\hat{r}_{2}\tilde{\mu}_{2}+F_{4}\left(2\right)\left(\frac{1}{1-\tilde{\mu}_{4}}\right)\tilde{\mu}_{2}+F_{2,4}^{(1)}\left(1\right),\\
D_{3}\hat{\mathbf{F}}_{1}=\hat{r}_{2}\tilde{\mu}_{3}+F_{4}\left(2\right)\left(\frac{1}{1-\tilde{\mu}_{4}}\right)\tilde{\mu}_{3}+F_{4}\left(1\right),&
D_{4}\hat{\mathbf{F}}_{1}=\hat{r}_{2}\tilde{\mu}_{4}
\end{array}
\end{eqnarray*}



\begin{eqnarray}
\begin{array}{ll}
\hat{\mathbf{F}}_{2}=\hat{R}_{1}F_{3}F_{1}, & D_{i}\hat{\mathbf{F}}_{2}=D_{i}\left(\hat{R}_{1}+F_{3}+\indora_{i\leq2}F_{1}\right)
\end{array}
\end{eqnarray}



equivalently


\begin{eqnarray*}
\begin{array}{ll}
D_{1}\hat{\mathbf{F}}_{2}=\hat{r}_{1}\tilde{\mu}_{1}+F_{3}\left(1\right)\left(\frac{1}{1-\tilde{\mu}_{3}}\right)\tilde{\mu}_{1}+F_{1,3}^{(1)}\left(1\right),&
D_{2}\hat{\mathbf{F}}_{2}=\hat{r}_{1}\mu_{2}+F_{3}\left(1\right)\left(\frac{1}{1-\tilde{\mu}_{3}}\right)\tilde{\mu}_{2}+F_{2,3}^{(1)}\left(1\right),\\
D_{3}\hat{\mathbf{F}}_{2}=\hat{r}_{1}\tilde{\mu}_{3},&
D_{4}\hat{\mathbf{F}}_{2}=\hat{r}_{1}\tilde{\mu}_{4}+F_{3}\left(1\right)\left(\frac{1}{1-\tilde{\mu}_{3}}\right)\tilde{\mu}_{4}+F_{3}\left(2\right),\\
\end{array}
\end{eqnarray*}





Then we have that if $\mu=\tilde{\mu}_{1}+\tilde{\mu}_{2}$, $\hat{\mu}=\tilde{\mu}_{3}+\tilde{\mu}_{4}$, $r=r_{1}+r_{2}$ and $\hat{r}=\hat{r}_{1}+\hat{r}_{2}$  the system's solution is given by

\begin{eqnarray*}
\begin{array}{llll}
f_{2}\left(1\right)=r_{1}\tilde{\mu}_{1},&
f_{1}\left(2\right)=r_{2}\tilde{\mu}_{2},&
F_{3}\left(4\right)=\hat{r}_{2}\tilde{\mu}_{4},&
F_{4}\left(3\right)=\hat{r}_{1}\tilde{\mu}_{3}
\end{array}
\end{eqnarray*}



it's easy to verify that

\begin{eqnarray}\label{Sist.Ec.Lineales.Doble.Traslado}
\begin{array}{ll}
f_{1}\left(1\right)=\tilde{\mu}_{1}\left(r+\frac{f_{2}\left(2\right)}{1-\tilde{\mu}_{2}}\right),& f_{1}\left(3\right)=\tilde{\mu}_{3}\left(r_{2}+\frac{f_{2}\left(2\right)}{1-\tilde{\mu}_{2}}\right)+F_{3,2}^{(1)}\left(1\right)\\
f_{1}\left(4\right)=\tilde{\mu}_{4}\left(r_{2}+\frac{f_{2}\left(2\right)}{1-\tilde{\mu}_{2}}\right)+F_{4,2}^{(1)}\left(1\right),&
f_{2}\left(2\right)=\left(r+\frac{f_{1}\left(1\right)}{1-\mu_{1}}\right)\tilde{\mu}_{2},\\
f_{2}\left(3\right)=\tilde{\mu}_{3}\left(r_{1}+\frac{f_{1}\left(1\right)}{1-\tilde{\mu}_{1}}\right)+F_{3,1}^{(1)}\left(1\right),&
f_{2}\left(4\right)=\tilde{\mu}_{4}\left(r_{1}+\frac{f_{1}\left(1\right)}{1-\mu_{1}}\right)+F_{4,,1}^{(1)}\left(1\right),\\
F_{3}\left(1\right)=\left(\hat{r}_{2}+\frac{F_{4}\left(4\right)}{1-\tilde{\mu}_{4}}\right)\tilde{\mu}_{1}+F_{1,4}^{(1)}\left(1\right),&
F_{3}\left(2\right)=\left(\hat{r}_{2}+\frac{F_{4}\left(4\right)}{1-\tilde{\mu}_{4}}\right)\tilde{\mu}_{2}+F_{2,4}^{(1)}\left(1\right),\\
F_{3}\left(3\right)=\left(\hat{r}+\frac{F_{4}\left(4\right)}{1-\tilde{\mu}_{4}}\right)\tilde{\mu}_{3},&
F_{4}\left(1\right)=\left(\hat{r}_{1}+\frac{F_{3}\left(3\right)}{1-\tilde{\mu}_{3}}\right)\mu_{1}+F_{1,3}^{(1)}\left(1\right),\\
F_{4}\left(2\right)=\left(\hat{r}_{1}+\frac{F_{3}\left(3\right)}{1-\tilde{\mu}_{3}}\right)\tilde{\mu}_{2}+F_{2,3}^{(1)}\left(1\right),&
F_{4}\left(4\right)=\left(\hat{r}+\frac{F_{3}\left(3\right)}{1-\tilde{\mu}_{3}}\right)\tilde{\mu}_{4},\\
\end{array}
\end{eqnarray}

with system's solutions given by

\begin{eqnarray}
\begin{array}{ll}
f_{1}\left(1\right)=r\frac{\mu_{1}\left(1-\mu_{1}\right)}{1-\mu},&
f_{2}\left(2\right)=r\frac{\tilde{\mu}_{2}\left(1-\tilde{\mu}_{2}\right)}{1-\mu},\\
f_{1}\left(3\right)=\tilde{\mu}_{3}\left(r_{2}+\frac{r\tilde{\mu}_{2}}{1-\mu}\right)+F_{3,2}^{(1)}\left(1\right),&
f_{1}\left(4\right)=\tilde{\mu}_{4}\left(r_{2}+\frac{r\tilde{\mu}_{2}}{1-\mu}\right)+F_{4,2}^{(1)}\left(1\right),\\
f_{2}\left(3\right)=\tilde{\mu}_{3}\left(r_{1}+\frac{r\mu_{1}}{1-\mu}\right)+F_{3,1}^{(1)}\left(1\right),&
f_{2}\left(4\right)=\tilde{\mu}_{4}\left(r_{1}+\frac{r\mu_{1}}{1-\mu}\right)+F_{4,,1}^{(1)}\left(1\right),\\
F_{3}\left(1\right)=\tilde{\mu}_{1}\left(\hat{r}_{2}+\frac{\hat{r}\tilde{\mu}_{4}}{1-\hat{\mu}}\right)+F_{1,4}^{(1)}\left(1\right),&
F_{3}\left(2\right)=\tilde{\mu}_{2}\left(\hat{r}_{2}+\frac{\hat{r}\tilde{\mu}_{4}}{1-\hat{\mu}}\right)+F_{2,4}^{(1)}\left(1\right),\\
F_{4}\left(1\right)=\tilde{\mu}_{1}\left(\hat{r}_{1}+\frac{\hat{r}\tilde{\mu}_{3}}{1-\hat{\mu}}\right)+F_{1,3}^{(1)}\left(1\right),&
F_{4}\left(2\right)=\tilde{\mu}_{2}\left(\hat{r}_{1}+\frac{\hat{r}\tilde{\mu}_{3}}{1-\hat{\mu}}\right)+F_{2,3}^{(1)}\left(1\right)
\end{array}
\end{eqnarray}

%_________________________________________________________________________________________________________
\subsection{General Second Order Derivatives}
%_________________________________________________________________________________________________________


Now, taking the second order derivative with respect to the equations given in (\ref{Sist.Ec.Lineales.Doble.Traslado}) we obtain it in their general form

\small{
\begin{eqnarray*}\label{Ec.Derivadas.Segundo.Orden.Doble.Transferencia}
D_{k}D_{i}F_{1}&=&D_{k}D_{i}\left(R_{2}+F_{2}+\indora_{i\geq3}F_{4}\right)+D_{i}R_{2}D_{k}\left(F_{2}+\indora_{k\geq3}F_{4}\right)+D_{i}F_{2}D_{k}\left(R_{2}+\indora_{k\geq3}F_{4}\right)+\indora_{i\geq3}D_{i}\hat{F}_{4}D_{k}\left(R_{2}+F_{2}\right)\\
D_{k}D_{i}F_{2}&=&D_{k}D_{i}\left(R_{1}+F_{1}+\indora_{i\geq3}F_{3}\right)+D_{i}R_{1}D_{k}\left(F_{1}+\indora_{k\geq3}F_{3}\right)+D_{i}F_{1}D_{k}\left(R_{1}+\indora_{k\geq3}F_{3}\right)+\indora_{i\geq3}D_{i}\hat{F}_{3}D_{k}\left(R_{1}+F_{1}\right)\\
D_{k}D_{i}F_{3}&=&D_{k}D_{i}\left(\hat{R}_{4}+\indora_{i\leq2}F_{2}+F_{4}\right)+D_{i}\hat{R}_{4}D_{k}\left(\indora_{k\leq2}F_{2}+F_{4}\right)+D_{i}\hat{F}_{4}D_{k}\left(\hat{R}_{4}+\indora_{k\leq2}F_{2}\right)+\indora_{i\leq2}D_{i}F_{2}D_{k}\left(\hat{R}_{4}+F_{4}\right)\\
D_{k}D_{i}F_{4}&=&D_{k}D_{i}\left(\hat{R}_{3}+\indora_{i\leq2}F_{1}+F_{3}\right)+D_{i}\hat{R}_{3}D_{k}\left(\indora_{k\leq2}F_{1}+F_{3}\right)+D_{i}\hat{F}_{3}D_{k}\left(\hat{R}_{3}+\indora_{k\leq2}F_{1}\right)+\indora_{i\leq2}D_{i}F_{1}D_{k}\left(\hat{R}_{3}+F_{3}\right)
\end{eqnarray*}}
for $i,k=1,\ldots,4$. In order to have it in an specific way we need to compute the expressions $D_{k}D_{i}\left(R_{2}+F_{2}+\indora_{i\geq3}F_{4}\right)$

%_________________________________________________________________________________________________________
\subsubsection{Second Order Derivatives: Serve's Switchover Times}
%_________________________________________________________________________________________________________

Remind $R_{i}\left(z_{1},z_{2},w_{1},w_{2}\right)=R_{i}\left(\tilde{P}_{1}\left(z_{1}\right)\tilde{P}_{2}\left(z_{2}\right)
\tilde{P}_{3}\left(w_{1}\right)\tilde{P}_{4}\left(w_{2}\right)\right)$,  which we will write in his reduced form $R_{i}=R_{i}\left(
\tilde{P}_{1}\tilde{P}_{2}\tilde{P}_{3}\tilde{P}_{4}\right)$, and according to the notation given in \cite{Lang} we obtain

\begin{eqnarray}
D_{i}D_{i}R_{k}=D^{2}R_{k}\left(D_{i}\tilde{P}_{i}\right)^{2}+DR_{k}D_{i}D_{i}\tilde{P}_{i}
\end{eqnarray}

whereas for $i\neq j$

\begin{eqnarray}
D_{i}D_{j}R_{k}=D^{2}R_{k}D_{i}\tilde{P}_{i}D_{j}\tilde{P}_{j}+DR_{k}D_{j}\tilde{P}_{j}D_{i}\tilde{P}_{i}
\end{eqnarray}

%_________________________________________________________________________________________________________
\subsubsection{Second Order Derivatives: Queue Lengths}
%_________________________________________________________________________________________________________

Just like before the expression $F_{1}\left(\tilde{\theta}_{1}\left(\tilde{P}_{2}\left(z_{2}\right)\tilde{P}_{3}\left(w_{1}\right)\tilde{P}_{4}\left(w_{2}\right)\right),
z_{2}\right)$, will be denoted by $F_{1}\left(\tilde{\theta}_{1}\left(\tilde{P}_{2}\tilde{P}_{3}\tilde{P}_{4}\right),z_{2}\right)$, then the mixed partial derivatives are:

\begin{eqnarray*}
D_{j}D_{i}F_{1}&=&\indora_{i,j\neq1}D_{1}D_{1}F_{1}\left(D\tilde{\theta}_{1}\right)^{2}D_{i}\tilde{P}_{i}D_{j}\tilde{P}_{j}
+\indora_{i,j\neq1}D_{1}F_{1}D^{2}\tilde{\theta}_{1}D_{i}\tilde{P}_{i}D_{j}\tilde{P}_{j}
+\indora_{i,j\neq1}D_{1}F_{1}D\tilde{\theta}_{1}\left(\indora_{i=j}D_{i}^{2}\tilde{P}_{i}+\indora_{i\neq j}D_{i}\tilde{P}_{i}D_{j}\tilde{P}_{j}\right)\\
&+&\left(1-\indora_{i=j=3}\right)\indora_{i+j\leq6}D_{1}D_{2}F_{1}D\tilde{\theta}_{1}\left(\indora_{i\leq j}D_{j}\tilde{P}_{j}+\indora_{i>j}D_{i}\tilde{P}_{i}\right)
+\indora_{i=2}\left(D_{1}D_{2}F_{1}D\tilde{\theta}_{1}D_{i}\tilde{P}_{i}+D_{i}^{2}F_{1}\right)
\end{eqnarray*}


Recall the expression for $F_{1}\left(\tilde{\theta}_{1}\left(\tilde{P}_{2}\left(z_{2}\right)\tilde{P}_{3}\left(w_{1}\right)\tilde{P}_{4}\left(w_{2}\right)\right),
z_{2}\right)$, which is denoted by $F_{1}\left(\tilde{\theta}_{1}\left(\tilde{P}_{2}\tilde{P}_{3}\tilde{P}_{4}\right),z_{2}\right)$, then the mixed partial derivatives are given by

\begin{eqnarray*}
\begin{array}{llll}
D_{1}D_{1}F_{1}=0,&
D_{2}D_{1}F_{1}=0,&
D_{3}D_{1}F_{1}=0,&
D_{4}D_{1}F_{1}=0,\\
D_{1}D_{2}F_{1}=0,&
D_{1}D_{3}F_{1}=0,&
D_{1}D_{4}F_{1}=0,&
\end{array}
\end{eqnarray*}

\begin{eqnarray*}
D_{2}D_{2}F_{1}&=&D_{1}^{2}F_{1}\left(D\tilde{\theta}_{1}\right)^{2}\left(D_{2}\tilde{P}_{2}\right)^{2}
+D_{1}F_{1}D^{2}\tilde{\theta}_{1}\left(D_{2}\tilde{P}_{2}\right)^{2}
+D_{1}F_{1}D\tilde{\theta}_{1}D_{2}^{2}\tilde{P}_{2}
+D_{1}D_{2}F_{1}D\tilde{\theta}_{1}D_{2}\tilde{P}_{2}\\
&+&D_{1}D_{2}F_{1}D\tilde{\theta}_{1}D_{2}\tilde{P}_{2}+D_{2}D_{2}F_{1}\\
&=&f_{1}\left(1,1\right)\left(\frac{\tilde{\mu}_{2}}{1-\tilde{\mu}_{1}}\right)^{2}
+f_{1}\left(1\right)\tilde{\theta}_{1}^(2)\tilde{\mu}_{2}^{(2)}
+f_{1}\left(1\right)\frac{1}{1-\tilde{\mu}_{1}}\tilde{P}_{2}^{(2)}+f_{1}\left(1,2\right)\frac{\tilde{\mu}_{2}}{1-\tilde{\mu}_{1}}+f_{1}\left(1,2\right)\frac{\tilde{\mu}_{2}}{1-\tilde{\mu}_{1}}+f_{1}\left(2,2\right)
\end{eqnarray*}

\begin{eqnarray*}
D_{3}D_{2}F_{1}&=&D_{1}^{2}F_{1}\left(D\tilde{\theta}_{1}\right)^{2}D_{3}\tilde{P}_{3}D_{2}\tilde{P}_{2}+D_{1}F_{1}D^{2}\tilde{\theta}_{1}D_{3}\tilde{P}_{3}D_{2}\tilde{P}_{2}+D_{1}F_{1}D\tilde{\theta}_{1}D_{2}\tilde{P}_{2}D_{3}\tilde{P}_{3}+D_{1}D_{2}F_{1}D\tilde{\theta}_{1}D_{3}\tilde{P}_{3}\\
&=&f_{1}\left(1,1\right)\left(\frac{1}{1-\tilde{\mu}_{1}}\right)^{2}\tilde{\mu}_{2}\tilde{\mu}_{3}+f_{1}\left(1\right)\tilde{\theta}_{1}^{(2)}\tilde{\mu}_{2}\tilde{\mu}_{3}+f_{1}\left(1\right)\frac{\tilde{\mu}_{2}\tilde{\mu}_{3}}{1-\tilde{\mu}_{1}}+f_{1}\left(1,2\right)\frac{\tilde{\mu}_{3}}{1-\tilde{\mu}_{1}}
\end{eqnarray*}

\begin{eqnarray*}
D_{4}D_{2}F_{1}&=&D_{1}^{2}F_{1}\left(D\tilde{\theta}_{1}\right)^{2}D_{4}\tilde{P}_{4}D_{2}\tilde{P}_{2}+D_{1}F_{1}D^{2}\tilde{\theta}_{1}D_{4}\tilde{P}_{4}D_{2}\tilde{P}_{2}+D_{1}F_{1}D\tilde{\theta}_{1}D_{2}\tilde{P}_{2}D_{4}\tilde{P}_{4}+D_{1}D_{2}F_{1}D\tilde{\theta}_{1}D_{4}\tilde{P}_{4}\\
&=&f_{1}\left(1,1\right)\left(\frac{1}{1-\tilde{\mu}_{1}}\right)^{2}\tilde{\mu}_{2}\tilde{\mu}_{4}+f_{1}\left(1\right)\tilde{\theta}_{1}^{(2)}\tilde{\mu}_{2}\tilde{\mu}_{4}+f_{1}\left(1\right)\frac{\tilde{\mu}_{2}\tilde{\mu}_{4}}{1-\tilde{\mu}_{1}}+f_{1}\left(1,2\right)\frac{\tilde{\mu}_{4}}{1-\tilde{\mu}_{1}}
\end{eqnarray*}

\begin{eqnarray*}
D_{2}D_{3}F_{1}&=&
D_{1}^{2}F_{1}\left(D\tilde{\theta}_{1}\right)^{2}D_{2}\tilde{P}_{2}D_{3}\tilde{P}_{3}
+D_{1}F_{1}D^{2}\tilde{\theta}_{1}D_{2}\tilde{P}_{2}D_{3}\tilde{P}_{3}+
D_{1}F_{1}D\tilde{\theta}_{1}D_{3}\tilde{P}_{3}D_{2}\tilde{P}_{2}
+D_{1}D_{2}F_{1}D\tilde{\theta}_{1}D_{3}\tilde{P}_{3}\\
&=&f_{1}\left(1,1\right)\left(\frac{1}{1-\tilde{\mu}_{1}}\right)^{2}\tilde{\mu}_{2}\tilde{\mu}_{3}+f_{1}\left(1\right)\tilde{\theta}_{1}^{(2)}\tilde{\mu}_{2}\tilde{\mu}_{3}+f_{1}\left(1\right)\frac{\tilde{\mu}_{2}\tilde{\mu}_{3}}{1-\tilde{\mu}_{1}}+f_{1}\left(1,2\right)\frac{\tilde{\mu}_{3}}{1-\tilde{\mu}_{1}}
\end{eqnarray*}

\begin{eqnarray*}
D_{3}D_{3}F_{1}&=&D_{1}^{2}F_{1}\left(D\tilde{\theta}_{1}\right)^{2}\left(D_{3}\tilde{P}_{3}\right)^{2}+D_{1}F_{1}D^{2}\tilde{\theta}_{1}\left(D_{3}\tilde{P}_{3}\right)^{2}+D_{1}F_{1}D\tilde{\theta}_{1}D_{3}^{2}\tilde{P}_{3}\\
&=&f_{1}\left(1,1\right)\left(\frac{\tilde{\mu}_{3}}{1-\tilde{\mu}_{1}}\right)^{2}+f_{1}\left(1\right)\tilde{\theta}_{1}^{(2)}\tilde{\mu}_{3}^{2}+f_{1}\left(1\right)\frac{\tilde{\mu}_{3}^{2}}{1-\tilde{\mu}_{1}}
\end{eqnarray*}

\begin{eqnarray*}
D_{4}D_{3}F_{1}&=&D_{1}^{2}F_{1}\left(D\tilde{\theta}_{1}\right)^{2}D_{4}\tilde{P}_{4}D_{3}\tilde{P}_{3}+D_{1}F_{1}D^{2}\tilde{\theta}_{1}D_{4}\tilde{P}_{4}D_{3}\tilde{P}_{3}+D_{1}F_{1}D\tilde{\theta}_{1}D_{3}\tilde{P}_{3}D_{4}\tilde{P}_{4}\\
&=&f_{1}\left(1,1\right)\left(\frac{1}{1-\tilde{\mu}_{1}}\right)^{2}\tilde{\mu}_{3}\tilde{\mu}_{4}
+f_{1}\left(1\right)\tilde{\theta}_{1}^{2}\tilde{\mu}_{4}\tilde{\mu}_{3}
+f_{1}\left(1\right)\frac{\tilde{\mu}_{4}\tilde{\mu}_{3}}{1-\tilde{\mu}_{1}}
\end{eqnarray*}

\begin{eqnarray*}
D_{2}D_{4}F_{1}&=&D_{1}^{2}F_{1}\left(D\tilde{\theta}_{1}\right)^{2}D_{2}\tilde{P}_{2}D_{4}\tilde{P}_{4}+D_{1}F_{1}D^{2}\tilde{\theta}_{1}D_{2}\tilde{P}_{2}D_{4}\tilde{P}_{4}+D_{1}F_{1}D\tilde{\theta}_{1}D_{4}\tilde{P}_{4}D_{2}\tilde{P}_{2}+D_{1}D_{2}F_{1}D\tilde{\theta}_{1}D_{4}\tilde{P}_{4}\\
&=&f_{1}\left(1,1\right)\left(\frac{1}{1-\tilde{\mu}_{1}}\right)^{2}\tilde{\mu}_{4}\tilde{\mu}_{2}
+f_{1}\left(1\right)\tilde{\theta}_{1}^{(2)}\tilde{\mu}_{4}\tilde{\mu}_{2}
+f_{1}\left(1\right)\frac{\tilde{\mu}_{4}\tilde{\mu}_{2}}{1-\tilde{\mu}_{1}}+f_{1}\left(1,2\right)\frac{\tilde{\mu}_{4}}{1-\tilde{\mu}_{1}}
\end{eqnarray*}

\begin{eqnarray*}
D_{3}D_{4}F_{1}&=&D_{1}^{2}F_{1}\left(D\tilde{\theta}_{1}\right)^{2}D_{3}\tilde{P}_{3}D_{4}\tilde{P}_{4}+D_{1}F_{1}D^{2}\tilde{\theta}_{1}D_{3}\tilde{P}_{3}D_{4}\tilde{P}_{4}+D_{1}F_{1}D\tilde{\theta}_{1}D_{4}\tilde{P}_{4}D_{3}\tilde{P}_{3}\\
&=&f_{1}\left(1,1\right)\left(\frac{1}{1-\tilde{\mu}_{1}}\right)^{2}\tilde{\mu}_{3}\tilde{\mu}_{4}+f_{1}\left(1\right)\tilde{\theta}_{1}^{(2)}\tilde{\mu}_{3}\tilde{\mu}_{4}+f_{1}\left(1\right)\frac{\tilde{\mu}_{3}\tilde{\mu}_{4}}{1-\tilde{\mu}_{1}}
\end{eqnarray*}

\begin{eqnarray*}
D_{4}D_{4}F_{1}&=&D_{1}^{2}F_{1}\left(D\tilde{\theta}_{1}\right)^{2}\left(D_{4}\tilde{P}_{4}\right)^{2}+D_{1}F_{1}D^{2}\tilde{\theta}_{1}\left(D_{4}\tilde{P}_{4}\right)^{2}+D_{1}F_{1}D\tilde{\theta}_{1}D_{4}^{2}\tilde{P}_{4}\\
&=&f_{1}\left(1,1\right)\left(\frac{\tilde{\mu}_{4}}{1-\tilde{\mu}_{1}}\right)^{2}+f_{1}\left(1\right)\tilde{\theta}_{1}^{(2)}\tilde{\mu}_{4}^{2}+f_{1}\left(1\right)\frac{1}{1-\tilde{\mu}_{1}}\tilde{P}_{4}^{(2)}
\end{eqnarray*}



Meanwhile for  $F_{2}\left(z_{1},\tilde{\theta}_{2}\left(\tilde{P}_{1}\tilde{P}_{3}\tilde{P}_{4}\right)\right)$

\begin{eqnarray*}
D_{j}D_{i}F_{2}&=&\indora_{i,j\neq2}D_{2}D_{2}F_{2}\left(D\theta_{2}\right)^{2}D_{i}\tilde{P}_{i}D_{j}\tilde{P}_{j}+\indora_{i,j\neq2}D_{2}F_{2}D^{2}\theta_{2}D_{i}\tilde{P}_{i}D_{j}\tilde{P}_{j}\\
&+&\indora_{i,j\neq2}D_{2}F_{2}D\theta_{2}\left(\indora_{i=j}D_{i}^{2}\tilde{P}_{i}
+\indora_{i\neq j}D_{i}\tilde{P}_{i}D_{j}\tilde{P}_{j}\right)\\
&+&\left(1-\indora_{i=j=3}\right)\indora_{i+j\leq6}D_{2}D_{1}F_{2}D\theta_{2}\left(\indora_{i\leq j}D_{j}\tilde{P}_{j}+\indora_{i>j}D_{i}\tilde{P}_{i}\right)
+\indora_{i=1}\left(D_{2}D_{1}F_{2}D\theta_{2}D_{i}\tilde{P}_{i}+D_{i}^{2}F_{2}\right)
\end{eqnarray*}

\begin{eqnarray*}
\begin{array}{llll}
D_{2}D_{1}F_{2}=0,&
D_{2}D_{3}F_{3}=0,&
D_{2}D_{4}F_{2}=0,&\\
D_{1}D_{2}F_{2}=0,&
D_{2}D_{2}F_{2}=0,&
D_{3}D_{2}F_{2}=0,&
D_{4}D_{2}F_{2}=0\\
\end{array}
\end{eqnarray*}


\begin{eqnarray*}
D_{1}D_{1}F_{2}&=&
\left(D_{1}\tilde{P}_{1}\right)^{2}\left(D\tilde{\theta}_{2}\right)^{2}D_{2}^{2}F_{2}
+\left(D_{1}\tilde{P}_{1}\right)^{2}D^{2}\tilde{\theta}_{2}D_{2}F_{2}
+D_{1}^{2}\tilde{P}_{1}D\tilde{\theta}_{2}D_{2}F_{2}
+D_{1}\tilde{P}_{1}D\tilde{\theta}_{2}D_{2}D_{1}F_{2}\\
&+&D_{2}D_{1}F_{2}D\tilde{\theta}_{2}D_{1}\tilde{P}_{1}+
D_{1}^{2}F_{2}\\
&=&f_{2}\left(2\right)\frac{\tilde{P}_{1}^{(2)}}{1-\tilde{\mu}_{2}}
+f_{2}\left(2\right)\theta_{2}^{(2)}\tilde{\mu}_{1}^{2}
+f_{2}\left(2,1\right)\frac{\tilde{\mu}_{1}}{1-\tilde{\mu}_{2}}
+\left(\frac{\tilde{\mu}_{1}}{1-\tilde{\mu}_{2}}\right)^{2}f_{2}\left(2,2\right)
+\frac{\tilde{\mu}_{1}}{1-\tilde{\mu}_{2}}f_{2}\left(2,1\right)+f_{2}\left(1,1\right)
\end{eqnarray*}


\begin{eqnarray*}
D_{3}D_{1}F_{2}&=&D_{2}D_{1}F_{2}D\tilde{\theta}_{2}D_{3}\tilde{P}_{3}
+D_{2}^{2}F_{2}\left(D\tilde{\theta}_{2}\right)^{2}D_{3}\tilde{P}_{1}D_{1}\tilde{P}_{1}
+D_{2}F_{2}D^{2}\tilde{\theta}_{2}D_{3}\tilde{P}_{3}D_{1}\tilde{P}_{1}
+D_{2}F_{2}D\tilde{\theta}_{2}D_{1}\tilde{P}_{1}D_{3}\tilde{P}_{3}\\
&=&f_{2}\left(2,1\right)\frac{\tilde{\mu}_{3}}{1-\tilde{\mu}_{2}}
+f_{2}\left(2,2\right)\left(\frac{1}{1-\tilde{\mu}_{2}}\right)^{2}\tilde{\mu}_{1}\tilde{\mu}_{3}
+f_{2}\left(2\right)\tilde{\theta}_{2}^{(2)}\tilde{\mu}_{1}\tilde{\mu}_{3}
+f_{2}\left(2\right)\frac{\tilde{\mu}_{1}\tilde{\mu}_{3}}{1-\tilde{\mu}_{2}}
\end{eqnarray*}


\begin{eqnarray*}
D_{4}D_{1}F_{2}&=&D_{2}^{2}F_{2}\left(D\tilde{\theta}_{2}\right)^{2}D_{4}\tilde{P}_{2}D_{1}\tilde{P}_{1}+D_{2}F_{2}D^{2}\tilde{\theta}_{2}D_{4}\tilde{P}_{4}D_{1}\tilde{P}_{1}
+D_{2}F_{2}D\tilde{\theta}_{2}D_{1}\tilde{P}_{1}D_{4}\tilde{P}_{4}+D_{2}D_{1}F_{2}D\tilde{\theta}_{2}D_{4}\tilde{P}_{4}\\
&=&f_{2}\left(2,2\right)\left(\frac{1}{1-\tilde{\mu}_{2}}\right)^{2}\tilde{\mu}_{1}\tilde{\mu}_{4}
+f_{2}\left(2\right)\tilde{\theta}_{2}^{(2)}\tilde{\mu}_{1}\tilde{\mu}_{4}
+f_{2}\left(2\right)\frac{\tilde{\mu}_{1}\tilde{\mu}_{4}}{1-\tilde{\mu}_{2}}
+f_{2}\left(2,1\right)\frac{\tilde{\mu}_{4}}{1-\tilde{\mu}_{2}}
\end{eqnarray*}


\begin{eqnarray*}
D_{1}D_{3}F_{2}&=&D_{2}^{2}F_{2}\left(D\tilde{\theta}_{2}\right)^{2}D_{1}\tilde{P}_{1}D_{3}\tilde{P}_{3}
+D_{2}F_{2}D^{2}\tilde{\theta}_{2}D_{1}\tilde{P}_{1}D_{3}\tilde{P}_{3}
+D_{2}F_{2}D\tilde{\theta}_{2}D_{3}\tilde{P}_{3}D_{1}\tilde{P}_{1}
+D_{2}D_{1}F_{2}D\tilde{\theta}_{2}D_{3}\tilde{P}_{3}\\
&=&f_{2}\left(2,2\right)\left(\frac{1}{1-\tilde{\mu}_{2}}\right)^{2}\tilde{\mu}_{1}\tilde{\mu}_{3}
+f_{2}\left(2\right)\tilde{\theta}_{2}^{(2)}\tilde{\mu}_{1}\tilde{\mu}_{3}
+f_{2}\left(2\right)\frac{\tilde{\mu}_{1}\tilde{\mu}_{3}}{1-\tilde{\mu}_{2}}
+f_{2}\left(2,1\right)\frac{\tilde{\mu}_{3}}{1-\tilde{\mu}_{2}}
\end{eqnarray*}


\begin{eqnarray*}
D_{3}D_{3}F_{2}&=&D_{2}^{2}F_{2}\left(D\tilde{\theta}_{2}\right)^{2}\left(D_{3}\tilde{P}_{3}\right)^{2}
+D_{2}F_{2}\left(D_{3}\tilde{P}_{3}\right)^{2}D^{2}\tilde{\theta}_{2}
+D_{2}F_{2}D\tilde{\theta}_{2}D_{3}^{2}\tilde{P}_{3}\\
&=&f_{2}\left(2,2\right)\left(\frac{1}{1-\tilde{\mu}_{2}}\right)^{2}\tilde{\mu}_{3}^{2}
+f_{2}\left(2\right)\tilde{\theta}_{2}^{(2)}\tilde{\mu}_{3}^{2}
+f_{2}\left(2\right)\frac{\tilde{P}_{3}^{(2)}}{1-\tilde{\mu}_{2}}
\end{eqnarray*}


\begin{eqnarray*}
D_{4}D_{3}F_{2}&=&D_{2}^{2}F_{2}\left(D\tilde{\theta}_{2}\right)^{2}D_{4}\tilde{P}_{4}D_{3}\tilde{P}_{3}
+D_{2}F_{2}D^{2}\tilde{\theta}_{2}D_{4}\tilde{P}_{4}D_{3}\tilde{P}_{3}
+D_{2}F_{2}D\tilde{\theta}_{2}D_{3}\tilde{P}_{3}D_{4}\tilde{P}_{4}\\
&=&f_{2}\left(2,2\right)\left(\frac{1}{1-\tilde{\mu}_{2}}\right)^{2}\tilde{\mu}_{3}\tilde{\mu}_{4}
+f_{2}\left(2\right)\tilde{\theta}_{2}^{(2)}\tilde{\mu}_{3}\tilde{\mu}_{4}
+f_{2}\left(2\right)\frac{\tilde{\mu}_{3}\tilde{\mu}_{4}}{1-\tilde{\mu}_{2}}
\end{eqnarray*}


\begin{eqnarray*}
D_{1}D_{4}F_{2}&=&D_{2}^{2}F_{2}\left(D\tilde{\theta}_{2}\right)^{2}D_{1}\tilde{P}_{1}D_{4}\tilde{P}_{4}
+D_{2}F_{2}D^{2}\tilde{\theta}_{2}D_{1}\tilde{P}_{1}D_{4}\tilde{P}_{4}
+D_{2}F_{2}D\tilde{\theta}_{2}D_{4}\tilde{P}_{4}D_{1}\tilde{P}_{1}
+D_{2}D_{1}F_{2}D\tilde{\theta}_{2}D_{4}\tilde{P}_{4}\\
&=&f_{2}\left(2,2\right)\left(\frac{1}{1-\tilde{\mu}_{2}}\right)^{2}\tilde{\mu}_{1}\tilde{\mu}_{4}
+f_{2}\left(2\right)\tilde{\theta}_{2}^{(2)}\tilde{\mu}_{1}\tilde{\mu}_{4}
+f_{2}\left(2\right)\frac{\tilde{\mu}_{1}\tilde{\mu}_{4}}{1-\tilde{\mu}_{2}}
+f_{2}\left(2,1\right)\frac{\tilde{\mu}_{4}}{1-\tilde{\mu}_{2}}
\end{eqnarray*}


\begin{eqnarray*}
D_{3}D_{4}F_{2}&=&
D_{2}^{2}F_{2}\left(D\tilde{\theta}_{2}\right)^{2}D_{4}\tilde{P}_{4}D_{3}\tilde{P}_{3}
+D_{2}F_{2}D^{2}\tilde{\theta}_{2}D_{4}\tilde{P}_{4}D_{3}\tilde{P}_{3}
+D_{2}F_{2}D\tilde{\theta}_{2}D_{4}\tilde{P}_{4}D_{3}\tilde{P}_{3}\\
&=&f_{2}\left(2,2\right)\left(\frac{1}{1-\tilde{\mu}_{2}}\right)^{2}\tilde{\mu}_{3}\tilde{\mu}_{4}
+f_{2}\left(2\right)\tilde{\theta}_{2}^{(2)}\tilde{\mu}_{3}\tilde{\mu}_{4}
+f_{2}\left(2\right)\frac{\tilde{\mu}_{3}\tilde{\mu}_{4}}{1-\tilde{\mu}_{2}}
\end{eqnarray*}


\begin{eqnarray*}
D_{4}D_{4}F_{2}&=&D_{2}F_{2}D\tilde{\theta}_{2}D_{4}^{2}\tilde{P}_{4}
+D_{2}F_{2}D^{2}\tilde{\theta}_{2}\left(D_{4}\tilde{P}_{4}\right)^{2}
+D_{2}^{2}F_{2}\left(D\tilde{\theta}_{2}\right)^{2}\left(D_{4}\tilde{P}_{4}\right)^{2}\\
&=&f_{2}\left(2,2\right)\left(\frac{\tilde{\mu}_{4}}{1-\tilde{\mu}_{2}}\right)^{2}
+f_{2}\left(2\right)\tilde{\theta}_{2}^{(2)}\tilde{\mu}_{4}^{2}
+f_{2}\left(2\right)\frac{\tilde{P}_{4}^{(2)}}{1-\tilde{\mu}_{2}}
\end{eqnarray*}


%\newpage



%\newpage

For $F_{3}\left(\hat{\theta}_{1}\left(\tilde{P}_{1}\tilde{P}_{2}\tilde{P}_{4}\right),w_{2}\right)$



\begin{eqnarray*}
D_{j}D_{i}F_{3}&=&\indora_{i,j\neq3}D_{3}D_{3}F_{3}\left(D\hat{\theta}_{1}\right)^{2}D_{i}\tilde{P}_{i}D_{j}\tilde{P}_{j}
+\indora_{i,j\neq3}D_{3}F_{3}D^{2}\tilde{\theta}_{1}D_{i}\tilde{P}_{i}D_{j}\tilde{P}_{j}
+\indora_{i,j\neq3}D_{3}F_{3}D\tilde{\theta}_{1}\left(\indora_{i=j}D_{i}^{2}\tilde{P}_{i}+\indora_{i\neq j}D_{i}\tilde{P}_{i}D_{j}\tilde{P}_{j}\right)\\
&+&\indora_{i+j\geq5}D_{3}D_{4}F_{3}D\tilde{\theta}_{1}\left(\indora_{i\leq j}D_{i}\tilde{P}_{i}+\indora_{i>j}D_{j}\tilde{P}_{j}\right)
+\indora_{i=4}\left(D_{3}D_{4}F_{3}D\tilde{\theta}_{1}D_{i}\tilde{P}_{i}+D_{i}^{2}F_{3}\right)
\end{eqnarray*}


\begin{eqnarray*}
\begin{array}{llll}
D_{3}D_{1}F_{3}=0,&
D_{3}D_{2}F_{3}=0,&
D_{1}D_{3}F_{3}=0,&
D_{2}D_{3}F_{3}=0\\
D_{3}D_{3}F_{3}=0,&
D_{4}D_{3}F_{3}=0,&
D_{3}D_{4}F_{3}=0,&
\end{array}
\end{eqnarray*}


\begin{eqnarray*}
D_{1}D_{1}F_{3}&=&
D_{3}^{2}F_{3}\left(D\tilde{\theta}_{1}\right)^{2}\left(D_{1}\tilde{P}_{1}\right)^{2}
+D_{3}F_{3}D^{2}\tilde{\theta}_{1}\left(D_{1}\tilde{P}_{1}\right)^{2}
+D_{3}F_{3}D\tilde{\theta}_{1}D_{1}^{2}\tilde{P}_{1}\\
&=&F_{3}\left(3,3\right)\left(\frac{\tilde{\mu}_{1}}{1-\tilde{\mu}_{4}}\right)^{2}
+F_{3}\left(3\right)\frac{\tilde{P}_{1}^{(2)}}{1-\tilde{\mu}_{3}}
+F_{3}\left(3\right)\tilde{\theta}_{1}^{(2)}\tilde{\mu}_{1}^{2}
\end{eqnarray*}


\begin{eqnarray*}
D_{2}D_{1}F_{3}&=&
D_{3}^{2}F_{3}\left(D\tilde{\theta}_{1}\right)^{2}D_{1}\tilde{P}_{1}D_{2}\tilde{P}_{1}+
D_{3}F_{3}D^{2}\tilde{\theta}_{1}D_{1}\tilde{P}_{1}D_{2}\tilde{P}_{2}+
D_{3}F_{3}D\tilde{\theta}_{1}D_{1}\tilde{P}_{1}D_{2}\tilde{P}_{2}\\
&=&F_{3}\left(3,3\right)\left(\frac{1}{1-\tilde{\mu}_{3}}\right)^{2}\tilde{\mu}_{1}\tilde{\mu}_{2}
+F_{3}\left(3\right)\tilde{\mu}_{1}\tilde{\mu}_{2}\tilde{\theta}_{1}^{(2)}
+F_{3}\left(3\right)\frac{\tilde{\mu}_{1}\tilde{\mu}_{2}}{1-\tilde{\mu}_{3}}
\end{eqnarray*}


\begin{eqnarray*}
D_{4}D_{1}F_{3}&=&
D_{3}D_{3}F_{3}\left(D\tilde{\theta}_{1}\right)^{2}D_{4}\tilde{P}_{4}D_{1}\tilde{P}_{1}
+D_{3}F_{3}D^{2}\tilde{\theta}_{1}D_{1}\tilde{P}_{1}D_{4}\tilde{P}_{4}
+D_{3}F_{3}D\tilde{\theta}_{1}D_{1}\tilde{P}_{1}D_{4}\tilde{P}_{4}
+D_{3}D_{4}F_{3}D\tilde{\theta}_{1}D_{1}\tilde{P}_{1}\\
&=&F_{3}\left(3,3\right)\left(\frac{1}{1-\tilde{\mu}_{3}}\right)^{2}\tilde{\mu}_{1}\tilde{\mu}_{3}
+F_{3}\left(3\right)\tilde{\theta}_{1}^{(2)}\tilde{\mu}_{1}\tilde{\mu}_{4}
+F_{3}\left(3\right)\frac{\tilde{\mu}_{1}\tilde{\mu}_{4}}{1-\tilde{\mu}_{3}}
+F_{3}\left(3,4\right)\frac{\tilde{\mu}_{1}}{1-\tilde{\mu}_{3}}
\end{eqnarray*}


\begin{eqnarray*}
D_{1}D_{2}F_{3}&=&
D_{3}^{2}F_{3}\left(D\tilde{\theta}_{1}\right)^{2}D_{1}\tilde{P}_{1}D_{2}\tilde{P}_{2}
+D_{3}F_{3}D^{2}\tilde{\theta}_{1}D_{1}\tilde{P}_{1}D_{2}\tilde{P}_{2}+
D_{3}F_{3}D\tilde{\theta}_{1}D_{1}\tilde{P}_{1}D_{2}\tilde{P}_{2}\\
&=&F_{3}\left(3,3\right)\left(\frac{1}{1-\tilde{\mu}_{3}}\right)^{2}\tilde{\mu}_{1}\tilde{\mu}_{2}
+F_{3}\left(3\right)\tilde{\theta}_{1}^{(2)}\tilde{\mu}_{1}\tilde{\mu}_{2}
+F_{3}\left(3\right)\frac{\tilde{\mu}_{1}\tilde{\mu}_{2}}{1-\tilde{\mu}_{3}}
\end{eqnarray*}


\begin{eqnarray*}
D_{2}D_{2}F_{3}&=&
D_{3}^{2}F_{3}\left(D\tilde{\theta}_{1}\right)^{2}\left(D_{2}\tilde{P}_{2}\right)^{2}
+D_{3}F_{3}D^{2}\tilde{\theta}_{1}\left(D_{2}\tilde{P}_{2}\right)^{2}+
D_{3}F_{3}D\tilde{\theta}_{1}D_{2}^{2}\tilde{P}_{2}\\
&=&F_{3}\left(3,3\right)\left(\frac{\tilde{\mu}_{2}}{1-\tilde{\mu}_{3}}\right)^{2}
+F_{3}\left(3\right)\tilde{\theta}_{1}^{(2)}\tilde{\mu}_{2}^{2}
+F_{3}\left(3\right)\tilde{P}_{2}^{(2)}\frac{1}{1-\tilde{\mu}_{3}}
\end{eqnarray*}


\begin{eqnarray*}
D_{4}D_{2}F_{3}&=&
D_{3}^{2}F_{3}\left(D\tilde{\theta}_{1}\right)^{2}D_{4}\tilde{P}_{4}D_{2}\tilde{P}_{2}
+D_{3}F_{3}D^{2}\tilde{\theta}_{1}D_{2}\tilde{P}_{2}D_{4}\tilde{P}_{4}
+D_{3}F_{3}D\tilde{\theta}_{1}D_{2}\tilde{P}_{2}D_{4}\tilde{P}_{4}
+D_{3}D_{4}F_{3}D\tilde{\theta}_{1}D_{2}\tilde{P}_{2}\\
&=&F_{3}\left(3,3\right)\left(\frac{1}{1-\tilde{\mu}_{3}}\right)^{2}\tilde{\mu}_{2}\tilde{\mu}_{4}
+F_{3}\left(3\right)\tilde{\theta}_{1}^{(2)}\tilde{\mu}_{2}\tilde{\mu}_{4}
+F_{3}\left(3\right)\frac{\tilde{\mu}_{2}\tilde{\mu}_{4}}{1-\tilde{\mu}_{3}}
+F_{3}\left(3,4\right)\frac{\tilde{\mu}_{2}}{1-\tilde{\mu}_{3}}
\end{eqnarray*}



\begin{eqnarray*}
D_{1}D_{4}F_{3}&=&
D_{3}D_{3}F_{3}\left(D\tilde{\theta}_{1}\right)^{2}D_{1}\tilde{P}_{1}D_{4}\tilde{P}_{4}
+D_{3}F_{3}D^{2}\tilde{\theta}_{1}D_{1}\tilde{P}_{1}D_{4}\tilde{P}_{4}
+D_{3}F_{3}D\tilde{\theta}_{1}D_{1}\tilde{P}_{1}D_{4}\tilde{P}_{4}
+D_{3}D_{4}F_{3}D\tilde{\theta}_{1}D_{1}\tilde{P}_{1}\\
&=&F_{3}\left(3,3\right)\left(\frac{1}{1-\tilde{\mu}_{3}}\right)^{2}\tilde{\mu}_{1}\tilde{\mu}_{4}
+F_{3}\left(3\right)\tilde{\theta}_{1}^{(2)}\tilde{\mu}_{1}\tilde{\mu}_{4}
+F_{3}\left(3\right)\frac{\tilde{\mu}_{1}\tilde{\mu}_{4}}{1-\tilde{\mu}_{3}}
+F_{3}\left(3,4\right)\frac{\tilde{\mu}_{1}}{1-\tilde{\mu}_{3}}
\end{eqnarray*}


\begin{eqnarray*}
D_{2}D_{4}F_{3}&=&
D_{3}^{2}F_{3}\left(D\tilde{\theta}_{1}\right)^{2}D_{2}\tilde{P}_{2}D_{4}\tilde{P}_{4}
+D_{3}F_{3}D^{2}\tilde{\theta}_{1}D_{2}\tilde{P}_{2}D_{4}\tilde{P}_{4}
+D_{3}F_{3}D\tilde{\theta}_{1}D_{2}\tilde{P}_{2}D_{4}\tilde{P}_{4}
+D_{3}D_{4}F_{3}D\tilde{\theta}_{1}D_{2}\tilde{P}_{2}\\
&=&F_{3}\left(3,3\right)\left(\frac{1}{1-\tilde{\mu}_{3}}\right)^{2}\tilde{\mu}_{2}\tilde{\mu}_{4}
+F_{3}\left(3\right)\tilde{\theta}_{1}^{(2)}\tilde{\mu}_{2}\tilde{\mu}_{4}
+F_{3}\left(3\right)\frac{\tilde{\mu}_{2}\tilde{\mu}_{4}}{1-\tilde{\mu}_{3}}
+F_{3}\left(3,4\right)\frac{\tilde{\mu}_{2}}{1-\tilde{\mu}_{3}}
\end{eqnarray*}



\begin{eqnarray*}
D_{4}D_{4}F_{3}&=&
D_{3}^{2}F_{3}\left(D\tilde{\theta}_{1}\right)^{2}\left(D_{4}\tilde{P}_{4}\right)^{2}
+D_{3}F_{3}D^{2}\tilde{\theta}_{1}\left(D_{4}\tilde{P}_{4}\right)^{2}
+D_{3}F_{3}D\tilde{\theta}_{1}D_{4}^{2}\tilde{P}_{4}
+D_{3}D_{4}F_{3}D\tilde{\theta}_{1}D_{4}\tilde{P}_{4}\\
&+&D_{3}D_{4}F_{3}D\tilde{\theta}_{1}D_{4}\tilde{P}_{4}
+D_{4}D_{4}F_{3}\\
&=&F_{3}\left(3,3\right)\left(\frac{\tilde{\mu}_{4}}{1-\tilde{\mu}_{3}}\right)^{2}
+F_{3}\left(3\right)\tilde{\theta}_{1}^{(2)}\tilde{\mu}_{4}^{2}
+F_{3}\left(3\right)\frac{\tilde{P}_{4}^{(2)}}{1-\tilde{\mu}_{3}}
+F_{3}\left(3,4\right)\frac{\tilde{\mu}_{4}}{1-\tilde{\mu}_{3}}
+F_{3}\left(3,4\right)\frac{\tilde{\mu}_{4}}{1-\tilde{\mu}_{3}}
+F_{3}\left(4,4\right)
\end{eqnarray*}




Finally for $F_{4}\left(w_{1},\tilde{\theta}_{2}\left(\tilde{P}_{1}\tilde{P}_{2}\tilde{P}_{3}\right)\right)$

\begin{eqnarray*}
D_{j}D_{i}F_{4}&=&\indora_{i,j\neq4}D_{4}D_{4}F_{4}\left(D\tilde{\theta}_{2}\right)^{2}D_{i}\tilde{P}_{i}D_{j}\tilde{P}_{j}
+\indora_{i,j\neq4}D_{4}F_{4}D^{2}\tilde{\theta}_{2}D_{i}\tilde{P}_{i}D_{j}\tilde{P}_{j}
+\indora_{i,j\neq4}D_{4}F_{4}D\tilde{\theta}_{2}\left(\indora_{i=j}D_{i}^{2}\tilde{P}_{i}+\indora_{i\neq j}D_{i}\tilde{P}_{i}D_{j}\tilde{P}_{j}\right)\\
&+&\left(1-\indora_{i=j=2}\right)\indora_{i+j\geq4}D_{4}D_{3}F_{4}D\tilde{\theta}_{2}\left(\indora_{i\leq j}D_{i}\tilde{P}_{i}+\indora_{i>j}D_{j}\tilde{P}_{j}\right)
+\indora_{i=3}\left(D_{4}D_{3}F_{4}D\tilde{\theta}_{2}D_{i}\tilde{P}_{i}+D_{i}^{2}F_{4}\right)
\end{eqnarray*}



\begin{eqnarray*}
\begin{array}{llll}
D_{4}D_{1}F_{4}=0,&
D_{4}D_{2}F_{4}=0,&
D_{4}D_{3}F_{4}=0,&
D_{1}D_{4}F_{4}=0\\
D_{2}D_{4}F_{4}=0,&
D_{3}D_{4}F_{4}=0,&
D_{4}D_{4}F_{4}=0,&
\end{array}
\end{eqnarray*}


\begin{eqnarray*}
D_{1}D_{1}F_{4}&=&
D_{4}^{2}F_{4}\left(D\tilde{\theta}_{2}\right)^{2}\left(D_{1}\tilde{P}_{1}\right)^{2}
+D_{4}F_{4}\tilde{\theta}_{2}\left(D_{1}\tilde{P}_{1}\right)^{2}D^{2}+
D_{4}F_{4}D\tilde{\theta}_{2}D_{1}^{2}\tilde{P}_{1}\\
&=&F_{4}\left(4,4\right)\left(\frac{\tilde{\mu}_{1}}{1-\tilde{\mu}_{4}}\right)^{2}
+F_{4}\left(4\right)\tilde{\theta}_{2}^{(2)}\tilde{\mu}_{1}^{2}
+F_{4}\left(4\right)\frac{\tilde{P}_{1}^{(2)}}{1-\tilde{\mu}_{2}}
\end{eqnarray*}



\begin{eqnarray*}
D_{2}D_{1}F_{4}&=&
D_{4}^{2}F_{4}\left(D\tilde{\theta}_{2}\right)^{2}D_{1}\tilde{P}_{1}D_{2}\tilde{P}_{2}
+D_{4}F_{4}D^{2}\tilde{\theta}_{2}D_{1}\tilde{P}_{1}D_{2}\tilde{P}_{2}
+D_{4}F_{4}D\tilde{\theta}_{2}D_{1}\tilde{P}_{1}D_{2}\tilde{P}_{2}\\
&=&F_{4}\left(4,4\right)\left(\frac{1}{1-\tilde{\mu}_{4}}\right)^{2}\tilde{\mu}_{1}\tilde{\mu}_{2}
+F_{4}\left(4\right)\tilde{\theta}_{2}^{(2)}\tilde{\mu}_{1}\tilde{\mu}_{2}
+F_{4}\left(4\right)\frac{\tilde{\mu}_{1}\tilde{\mu}_{2}}{1-\tilde{\mu}_{2}}
\end{eqnarray*}



\begin{eqnarray*}
D_{3}D_{1}F_{4}&=&
D_{4}^{2}F_{4}\left(D\tilde{\theta}_{2}\right)^{2}D_{1}\tilde{P}_{1}D_{3}\tilde{P}_{3}
+D_{4}F_{4}D^{2}\tilde{\theta}_{2}D_{1}\tilde{P}_{1}D_{3}\tilde{P}_{3}
+D_{4}F_{4}D\tilde{\theta}_{2}D_{1}\tilde{P}_{1}D_{3}\tilde{P}_{3}
+D_{4}D_{3}F_{4}D\tilde{\theta}_{2}D_{1}\tilde{P}_{1}\\
&=&F_{4}\left(4,4\right)\left(\frac{1}{1-\tilde{\mu}_{4}}\right)^{2}\tilde{\mu}_{1}\tilde{\mu}_{3}
+F_{4}\left(4\right)\tilde{\theta}_{2}^{(2)}\tilde{\mu}_{1}\tilde{\mu}_{3}
+F_{4}\left(4\right)\frac{\tilde{\mu}_{1}\tilde{\mu}_{3}}{1-\tilde{\mu}_{4}}
+F_{4}\left(4,3\right)\frac{\tilde{\mu}_{1}}{1-\tilde{\mu}_{4}}
\end{eqnarray*}



\begin{eqnarray*}
D_{1}D_{2}F_{4}&=&
D_{4}D_{4}F_{4}\left(D\tilde{\theta}_{2}\right)^{2}D_{1}\tilde{P}_{1}D_{2}\tilde{P}_{2}
+D_{4}F_{4}D^{2}\tilde{\theta}_{2}D_{1}\tilde{P}_{1}D_{2}\tilde{P}_{2}
+D_{4}F_{4}D\tilde{\theta}_{2}D_{1}\tilde{P}_{1}D_{2}\tilde{P}_{2}
\\
&=&
F_{4}\left(4,4\right)\left(\frac{1}{1-\tilde{\mu}_{4}}\right)^{2}\tilde{\mu}_{1}\tilde{\mu}_{2}
+F_{4}\left(4\right)\tilde{\theta}_{2}^{(2)}\tilde{\mu}_{1}\tilde{\mu}_{2}
+F_{4}\left(4\right)\frac{\tilde{\mu}_{1}\tilde{\mu}_{2}}{1-\tilde{\mu}_{2}}
\end{eqnarray*}



\begin{eqnarray*}
D_{2}D_{2}F_{4}&=&
D_{4}^{2}F_{4}\left(D\tilde{\theta}_{2}\right)^{2}\left(D_{2}\tilde{P}_{2}\right)^{2}
+D_{4}F_{4}D^{2}\tilde{\theta}_{2}\left(D_{2}\tilde{P}_{2}\right)^{2}
+D_{4}F_{4}D\tilde{\theta}_{2}D_{2}^{2}\tilde{P}_{2}
\\
&=&F_{4}\left(4,4\right)\left(\frac{\tilde{\mu}_{2}}{1-\tilde{\mu}_{4}}\right)^{2}
+F_{4}\left(4\right)\tilde{\theta}_{2}^{(2)}\tilde{\mu}_{2}^{2}
+F_{4}\left(4\right)\frac{\tilde{P}_{2}^{(2)}}{1-\tilde{\mu}_{4}}
\end{eqnarray*}



\begin{eqnarray*}
D_{3}D_{2}F_{4}&=&
D_{4}^{2}F_{4}\left(D\tilde{\theta}_{2}\right)^{2}D_{2}\tilde{P}_{2}D_{3}\tilde{P}_{3}
+D_{4}F_{4} D^{2}\tilde{\theta}_{2}D_{2}\tilde{P}_{2}D_{3}\tilde{P}_{3}
+D_{4}F_{4}D\tilde{\theta} _{2}D_{2}\tilde{P}_{2}D_{3}\tilde{P}_{3}
+D_{4}D_{3}F_{4}D\tilde{\theta}_{2}D_{2}\tilde{P}_{2}\\
&=&
F_{4}\left(4,4\right)\left(\frac{1}{1-\tilde{\mu}_{4}}\right)^{2}\tilde{\mu}_{2}\tilde{\mu}_{3}
+F_{4}\left(4\right)\tilde{\theta}_{2}^{(2)}\tilde{\mu}_{2}\tilde{\mu}_{3}
+F_{4}\left(4\right)\frac{\tilde{\mu}_{2}\tilde{\mu}_{3}}{1-\tilde{\mu}_{4}}
+F_{4}\left(4,3\right)\frac{\tilde{\mu}_{2}}{1-\tilde{\mu}_{4}}
\end{eqnarray*}



\begin{eqnarray*}
D_{1}D_{3}F_{4}&=&
D_{4}D_{4}F_{4}\left(D\tilde{\theta}_{2}\right)^{2}D_{1}\tilde{P}_{1}D_{3}\tilde{P}_{3}
+D_{4}F_{4}D^{2}\tilde{\theta}_{2}D_{1}\tilde{P}_{1}D_{3}\tilde{P}_{3}
+D_{4}F_{4}D\tilde{\theta}_{2}D_{1}\tilde{P}_{1}D_{3}\tilde{P}_{3}
+D_{4}D_{3}F_{4}D\tilde{\theta}_{2}D_{1}\tilde{P}_{1}\\
&=&
F_{4}\left(4,4\right)\left(\frac{1}{1-\tilde{\mu}_{4}}\right)^{2}\tilde{\mu}_{1}\tilde{\mu}_{3}
+F_{4}\left(4\right)\tilde{\theta}_{2}^{(2)}\tilde{\mu}_{1}\tilde{\mu}_{3}
+F_{4}\left(4\right)\frac{\tilde{\mu}_{1}\tilde{\mu}_{3}}{1-\tilde{\mu}_{4}}
+F_{4}\left(4,3\right)\frac{\tilde{\mu}_{1}}{1-\tilde{\mu}_{4}}
\end{eqnarray*}



\begin{eqnarray*}
D_{2}D_{3}F_{4}&=&
D_{4}^{2}F_{4}\left(D\tilde{\theta}_{2}\right)^{2}D_{2}\tilde{P}_{2}D_{3}\tilde{P}_{3}
+D_{4}F_{4}D^{2}\tilde{\theta}_{2}D_{2}\tilde{P}_{2}D_{3}\tilde{P}_{3}
+D_{4}F_{4}D\tilde{\theta}_{2}D_{2}\tilde{P}_{2}D_{3}\tilde{P}_{3}
+D_{4}D_{3}F_{4}D\tilde{\theta}_{2}D_{2}\tilde{P}_{2}\\
&=&
F_{4}\left(4,4\right)\left(\frac{1}{1-\tilde{\mu}_{4}}\right)^{2}\tilde{\mu}_{2}\tilde{\mu}_{3}
+F_{4}\left(4\right)\tilde{\theta}_{2}^{(2)}\tilde{\mu}_{2}\tilde{\mu}_{3}
+F_{4}\left(4\right)\frac{\tilde{\mu}_{2}\tilde{\mu}_{3}}{1-\tilde{\mu}_{4}}
+F_{4}\left(4,3\right)\frac{\tilde{\mu}_{2}}{1-\tilde{\mu}_{4}}
\end{eqnarray*}



\begin{eqnarray*}
D_{3}D_{3}F_{4}&=&
D_{4}^{2}F_{4}\left(D\tilde{\theta}_{2}\right)^{2}\left(D_{3}\tilde{P}_{3}\right)^{2}
+D_{4}F_{4}D^{2}\tilde{\theta}_{2}\left(D_{3}\tilde{P}_{3}\right)^{2}
+D_{4}F_{4}D\tilde{\theta}_{2}D_{3}^{2}\tilde{P}_{3}
+D_{4}D_{3}F_{4}D\tilde{\theta}_{2}D_{3}\tilde{P}_{3}\\
&+&D_{4}D_{3}F_{4}D\tilde{\theta}_{2}D_{3}\tilde{P}_{3}
+D_{3}^{2}F_{4}\\
&=&
F_{4}\left(4,4\right)\left(\frac{\tilde{\mu}_{3}}{1-\tilde{\mu}_{4}}\right)^{2}
+F_{4}\left(4\right)\tilde{\theta}_{2}^{(2)}\tilde{\mu}_{3}^{2}
+F_{4}\left(4\right)\frac{\tilde{P}_{3}^{(2)}}{1-\tilde{\mu}_{4}}
+F_{4}\left(4,3\right)\frac{\tilde{\mu}_{3}}{1-\tilde{\mu}_{4}}
+F_{4}\left(4,3\right)\frac{\tilde{\mu}_{1}}{1-\tilde{\mu}_{4}}
+F_{4}\left(3,3\right)
\end{eqnarray*}

%_____________________________________________________________
\subsection{Second Grade Derivative Recursive Equations}
%_____________________________________________________________


Then according to the equations given at the beginning of this section, we have

\begin{eqnarray*}
D_{k}D_{i}F_{1}&=&D_{k}D_{i}\left(R_{2}+F_{2}+\indora_{i\geq3}F_{4}\right)+D_{i}R_{2}D_{k}\left(F_{2}+\indora_{k\geq3}F_{4}\right)\\&+&D_{i}F_{2}D_{k}\left(R_{2}+\indora_{k\geq3}F_{4}\right)+\indora_{i\geq3}D_{i}F_{4}D_{k}\left(R_{2}+F_{2}\right)
\end{eqnarray*}
%_____________________________________________________________
\subsection*{$F_{1}$}
%_____________________________________________________________
%_____________________________________________________________
\subsubsection*{$F_{1}$ and $i=1$}
%_____________________________________________________________

for $i=1$, and $k=1$

\begin{eqnarray*}
D_{1}D_{1}F_{1}&=&D_{1}D_{1}\left(R_{2}+F_{2}\right)+D_{1}R_{2}D_{1}F_{2}
+D_{1}F_{2}D_{1}R_{2}
=D_{1}^{2}R_{2}
+D_{1}^{2}F_{2}
+D_{1}R_{2}D_{1}F_{2}
+D_{1}F_{2}D_{1}R_{2}\\
&=&R_{2}^{(2)}\tilde{\mu}_{1}+r_{2}\tilde{P}_{1}^{(2)}
+D_{1}^{2}F_{2}
+2r_{2}\tilde{\mu}_{1}f_{2}\left(1\right)
\end{eqnarray*}

$k=2$
\begin{eqnarray*}
D_{2}D_{i}F_{1}&=&D_{2}D_{1}\left(R_{2}+F_{2}\right)
+D_{1}R_{2}D_{2}F_{2}+D_{1}F_{2}D_{2}R_{2}
=D_{2}D_{1}R_{2}
+D_{2}D_{1}F_{2}
+D_{1}R_{2}D_{2}F_{2}
+D_{1}F_{2}D_{2}R_{2}\\
&=&R_{2}^{(2)}\tilde{\mu}_{1}\tilde{\mu}_{2}+r_{2}\tilde{\mu}_{1}\tilde{\mu}_{2}
+D_{2}D_{1}F_{2}
+r_{2}\tilde{\mu}_{1}f_{2}\left(2\right)
+r_{2}\tilde{\mu}_{2}f_{2}\left(1\right)
\end{eqnarray*}

$k=3$
\begin{eqnarray*}
D_{3}D_{1}F_{1}&=&D_{3}D_{1}\left(R_{2}+F_{2}\right)
+D_{1}R_{2}D_{3}\left(F_{2}+F_{4}\right)
+D_{1}F_{2}D_{3}\left(R_{2}+F_{4}\right)\\
&=&D_{3}D_{1}R_{2}+D_{3}D_{1}F_{2}
+D_{1}R_{2}D_{3}F_{2}+D_{1}R_{2}D_{3}F_{4}
+D_{1}F_{2}D_{3}R_{2}+D_{1}F_{2}D_{3}F_{4}\\
&=&R_{2}^{(2)}\tilde{\mu}_{1}\tilde{\mu}_{3}+r_{2}\tilde{\mu}_{1}\tilde{\mu}_{3}
+D_{3}D_{1}F_{2}
+r_{2}\tilde{\mu}_{1}f_{2}\left(3\right)
+r_{2}\tilde{\mu}_{1}D_{3}F_{4}
+r_{2}\tilde{\mu}_{3}f_{2}\left(1\right)
+D_{3}F_{4}f_{2}\left(1\right)
\end{eqnarray*}

$k=4$
\begin{eqnarray*}
D_{4}D_{1}F_{1}&=&D_{4}D_{1}\left(R_{2}+F_{2}\right)
+D_{1}R_{2}D_{4}\left(F_{2}+F_{4}\right)
+D_{1}F_{2}D_{4}\left(R_{2}+F_{4}\right)\\
&=&D_{4}D_{1}R_{2}+D_{4}D_{1}F_{2}
+D_{1}R_{2}D_{4}F_{2}+D_{1}R_{2}D_{4}F_{4}
+D_{1}F_{2}D_{4}R_{2}+D_{1}F_{2}D_{4}F_{4}\\
&=&R_{2}^{(2)}\tilde{\mu}_{1}\tilde{\mu}_{4}+r_{2}\tilde{\mu}_{1}\tilde{\mu}_{4}
+D_{4}D_{1}F_{2}
+r_{2}\tilde{\mu}_{1}f_{2}\left(4\right)
+r_{2}\tilde{\mu}_{1}D_{4}F_{4}
+r_{2}\tilde{\mu}_{4}f_{2}\left(1\right)
+f_{2}\left(1\right)D_{4}F_{4}
\end{eqnarray*}


%_____________________________________________________________
\subsubsection*{$F_{1}$ and $i=2$}
%_____________________________________________________________

for $i=2$,

$k=2$
\begin{eqnarray*}
D_{2}D_{2}F_{1}&=&D_{2}D_{2}\left(R_{2}+F_{2}\right)
+D_{2}R_{2}D_{2}F_{2}+D_{2}F_{2}D_{2}R_{2}
=D_{2}D_{2}R_{2}+D_{2}D_{2}F_{2}+D_{2}R_{2}D_{2}F_{2}+D_{2}F_{2}D_{2}R_{2}\\
&=&R_{2}^{(2)}\tilde{\mu}_{2}^{2}+r_{2}\tilde{P}_{2}^{(2)}
+D_{2}D_{2}F_{2}
+2r_{2}\tilde{\mu}_{2}f_{2}\left(2\right)
\end{eqnarray*}

$k=3$
\begin{eqnarray*}
D_{3}D_{2}F_{1}&=&D_{3}D_{2}\left(R_{2}+F_{2}\right)
+D_{2}R_{2}D_{3}\left(F_{2}+F_{4}\right)
+D_{2}F_{2}D_{3}\left(R_{2}+F_{4}\right)\\
&=&D_{3}D_{2}R_{2}+D_{3}D_{2}F_{2}
+D_{2}R_{2}D_{3}F_{2}+D_{2}R_{2}D_{3}F_{4}
+D_{2}F_{2}D_{3}R_{2}+D_{2}F_{2}D_{3}F_{4}\\
&=&R_{2}^{(2)}\tilde{\mu}_{2}\tilde{\mu}_{3}+r_{2}\tilde{\mu}_{2}\tilde{\mu}_{3}
+D_{3}D_{2}F_{2}
+r_{2}\tilde{\mu}_{2}f_{2}\left(3\right)
+r_{2}\tilde{\mu}_{2}D_{3}F_{4}
+r_{2}\tilde{\mu}_{3}f_{2}\left(2\right)
+f_{2}\left(2\right)D_{3}F_{4}
\end{eqnarray*}

$k=4$
\begin{eqnarray*}
D_{4}D_{2}F_{1}&=&D_{4}D_{2}\left(R_{2}+F_{2}\right)
+D_{2}R_{2}D_{4}\left(F_{2}+F_{4}\right)
+D_{2}F_{2}D_{4}\left(R_{2}+F_{4}\right)\\
&=&D_{4}D_{2}R_{2}+D_{4}D_{2}F_{2}
+D_{2}R_{2}D_{4}F_{2}+D_{2}R_{2}D_{4}F_{4}
+D_{2}F_{2}D_{4}R_{2}+D_{2}F_{2}D_{4}F_{4}\\
&=&R_{2}^{(2)}\tilde{\mu}_{2}\tilde{\mu}_{4}+r_{2}\tilde{\mu}_{2}\tilde{\mu}_{4}
+D_{4}D_{2}F_{2}
+r_{2}\tilde{\mu}_{2}f_{2}\left(4\right)
+r_{2}\tilde{\mu}_{2}D_{4}F_{4}
+r_{2}\tilde{\mu}_{4}f_{2}\left(2\right)
+f_{2}\left(2\right)D_{4}F_{4}
\end{eqnarray*}

%_____________________________________________________________
\subsubsection*{$F_{1}$ and $i=3$}
%_____________________________________________________________
for $i=3$, and $k=3$
\begin{eqnarray*}
D_{3}D_{3}F_{1}&=&D_{3}D_{3}\left(R_{2}+F_{2}+F_{4}\right)
+D_{3}R_{2}D_{3}\left(F_{2}+F_{4}\right)
+D_{3}F_{2}D_{3}\left(R_{2}+F_{4}\right)
+D_{3}F_{4}D_{3}\left(R_{2}+F_{2}\right)\\
&=&D_{3}D_{3}R_{2}+D_{3}D_{3}F_{2}+D_{3}D_{3}F_{4}
+D_{3}R_{2}D_{3}F_{2}+D_{3}R_{2}D_{3}F_{4}\\
&+&D_{3}F_{2}D_{3}R_{2}+D_{3}F_{2}D_{3}F_{4}
+D_{3}F_{4}D_{3}R_{2}+D_{3}F_{4}D_{3}F_{2}\\
&=&R_{2}^{(2)}\tilde{\mu}_{3}^{2}+r_{2}\tilde{P}_{3}^{(2)}
+D_{3}D_{3}F_{2}
+D_{3}D_{3}F_{4}
+r_{2}\tilde{\mu}_{3}f_{2}\left(3\right)
+r_{2}\tilde{\mu}_{3}D_{3}F_{4}\\
&+&r_{2}\tilde{\mu}_{3}f_{2}\left(3\right)
+f_{2}\left(3\right)D_{3}F_{4}
+r_{2}\tilde{\mu}_{3}D_{3}F_{4}
+f_{2}\left(3\right)D_{3}F_{4}
\end{eqnarray*}

$k=4$
\begin{eqnarray*}
D_{4}D_{3}F_{1}&=&D_{4}D_{3}\left(R_{2}+F_{2}+F_{4}\right)
+D_{3}R_{2}D_{4}\left(F_{2}+F_{4}\right)
+D_{3}F_{2}D_{4}\left(R_{2}+F_{4}\right)
+D_{3}F_{4}D_{4}\left(R_{2}+F_{2}\right)\\
&=&D_{4}D_{3}R_{2}+D_{4}D_{3}F_{2}+D_{4}D_{3}F_{4}
+D_{3}R_{2}D_{4}F_{2}+D_{3}R_{2}D_{4}F_{4}\\
&+&D_{3}F_{2}D_{4}R_{2}+D_{3}F_{2}D_{4}F_{4}
+D_{3}F_{4}D_{4}R_{2}+D_{3}F_{4}D_{4}F_{2}\\
&=&R_{2}^{(2)}\tilde{\mu}_{3}\tilde{\mu}_{4}+r_{2}\tilde{\mu}_{3}\tilde{\mu}_{4}
+D_{4}D_{3}F_{2}
+D_{4}D_{3}F_{4}
+r_{2}\tilde{\mu}_{3}f_{2}\left(4\right)
+r_{2}\tilde{\mu}_{3}D_{4}F_{4}\\
&+&r_{2}\tilde{\mu}_{4}f_{2}\left(3\right)
+D_{4}F_{4}f_{2}\left(3\right)
+D_{3}F_{4}r_{2}\tilde{\mu}_{4}
+D_{3}F_{4}f_{2}\left(4\right)
\end{eqnarray*}

%_____________________________________________________________
\subsubsection*{$F_{1}$ and $i=4$}
%_____________________________________________________________

for $i=4$, $k=4$
\begin{eqnarray*}
D_{4}D_{4}F_{1}&=&D_{4}D_{4}\left(R_{2}+F_{2}+F_{4}\right)
+D_{4}R_{2}D_{4}\left(F_{2}+F_{4}\right)
+D_{4}F_{2}D_{4}\left(R_{2}+F_{4}\right)
+D_{4}F_{4}D_{4}\left(R_{2}+F_{2}\right)\\
&=&D_{4}D_{4}R_{2}+D_{4}D_{4}F_{2}+D_{4}D_{4}F_{4}
+D_{4}R_{2}D_{4}F_{2}+D_{4}R_{2}D_{4}F_{4}\\
&+&D_{4}F_{2}D_{4}R_{2}+D_{4}F_{2}D_{4}F_{4}
+D_{4}F_{4}D_{4}R_{2}+D_{4}F_{4}D_{4}F_{2}\\
&=&R_{2}^{(2)}\tilde{\mu}_{4}^{2}+r_{2}\tilde{P}_{4}^{(2)}
+D_{4}D_{4}F_{2}
+D_{4}D_{4}F_{4}
+r_{2}\tilde{\mu}_{4}f_{2}\left(4\right)
+r_{2}\tilde{\mu}_{4}D_{4}F_{4}\\
&+&r_{2}\tilde{\mu}_{4}f_{2}\left(4\right)
+D_{4}F_{4}f_{2}\left(4\right)
+D_{4}F_{4}r_{2}\tilde{\mu}_{4}
+D_{4}F_{4}f_{2}\left(4\right)
\end{eqnarray*}

%__________________________________________________________________________________________
%_____________________________________________________________
\subsection*{$F_{2}$}
%_____________________________________________________________
\begin{eqnarray}
D_{k}D_{i}F_{2}&=&D_{k}D_{i}\left(R_{1}+F_{1}+\indora_{i\geq3}F_{3}\right)+D_{i}R_{1}D_{k}\left(F_{1}+\indora_{k\geq3}F_{3}\right)+D_{i}F_{1}D_{k}\left(R_{1}+\indora_{k\geq3}F_{3}\right)+\indora_{i\geq3}D_{i}\hat{F}_{3}D_{k}\left(R_{1}+F_{1}\right)
\end{eqnarray}
%_____________________________________________________________
\subsubsection*{$F_{2}$ and $i=1$}
%_____________________________________________________________
$i=1$, $k=1$
\begin{eqnarray*}
D_{1}D_{1}F_{2}&=&D_{1}D_{1}\left(R_{1}+F_{1}\right)
+D_{1}R_{1}D_{1}F_{1}
+D_{1}F_{1}D_{1}R_{1}
=D_{1}^{2}R_{1}
+D_{1}^{2}F_{1}
+D_{1}R_{1}D_{1}F_{1}
+D_{1}F_{1}D_{1}R_{1}\\
&=&R_{1}^{2}\tilde{\mu}_{1}^{2}+r_{1}\tilde{P}_{1}^{(2)}
+D_{1}^{2}F_{1}
+2r_{1}\tilde{\mu}_{1}f_{1}\left(1\right)
\end{eqnarray*}

$k=2$
\begin{eqnarray*}
D_{2}D_{1}F_{2}&=&D_{2}D_{1}\left(R_{1}+F_{1}\right)+D_{1}R_{1}D_{2}F_{1}+D_{1}F_{1}D_{2}R_{1}=
D_{2}D_{1}R_{1}+D_{2}D_{1}F_{1}+D_{1}R_{1}D_{2}F_{1}+D_{1}F_{1}D_{2}R_{1}\\
&=&R_{1}^{(2)}\tilde{\mu}_{1}\tilde{\mu}_{2}+r_{1}\tilde{\mu}_{1}\tilde{\mu}_{2}
+D_{2}D_{1}F_{1}
+r_{1}\tilde{\mu}_{1}f_{1}\left(2\right)
+r_{1}\tilde{\mu}_{2}f_{1}\left(1\right)
\end{eqnarray*}

$k=3$
\begin{eqnarray*}
D_{3}D_{1}F_{2}&=&D_{3}D_{1}\left(R_{1}+F_{1}\right)+D_{1}R_{1}D_{3}\left(F_{1}+F_{3}\right)+D_{1}F_{1}D_{3}\left(R_{1}+F_{3}\right)\\
&=&D_{3}D_{1}R_{1}+D_{3}D_{1}F_{1}+D_{1}R_{1}D_{3}F_{1}+D_{1}R_{1}D_{3}F_{3}+D_{1}F_{1}D_{3}R_{1}+D_{1}F_{1}D_{3}F_{3}\\
&=&R_{1}^{(2)}\tilde{\mu}_{1}\tilde{\mu}_{3}+r_{1}\tilde{\mu}_{1}\tilde{\mu}_{3}
+D_{3}D_{1}F_{1}
+r_{1}\tilde{\mu}_{1}f_{1}\left(3\right)
+r_{1}\tilde{\mu}_{1}D_{3}F_{3}
+r_{1}\tilde{\mu}_{3}f_{1}\left(1\right)
+D_{3}\hat{F}_{3}f_{1}\left(1\right)
\end{eqnarray*}

$k=4$
\begin{eqnarray*}
D_{4}D_{1}F_{2}&=&D_{4}D_{1}\left(R_{1}+F_{1}\right)+D_{1}R_{1}D_{4}\left(F_{1}+F_{3}\right)+D_{1}F_{1}D_{4}\left(R_{1}+F_{3}\right)\\
&=&D_{4}D_{1}R_{1}+D_{4}D_{1}F_{1}+D_{1}R_{1}D_{4}F_{1}+D_{1}R_{1}D_{4}F_{3}
+D_{1}F_{1}D_{4}R_{1}+D_{1}F_{1}D_{4}F_{3}\\
&=&R_{1}^{(2)}\tilde{\mu}_{1}\tilde{\mu}_{4}+r_{1}\tilde{\mu}_{1}\tilde{\mu}_{4}
+D_{4}D_{1}F_{1}
+r_{1}\tilde{\mu}_{1}f_{1}\left(4\right)
+\tilde{\mu}_{1}D_{4}f_{3}\left(4\right)
+\tilde{\mu}_{1}\tilde{\mu}_{4}f_{1}\left(1\right)
+f_{1}\left(1\right)D_{4}F_{4}
\end{eqnarray*}
%_____________________________________________________________
\subsubsection*{$F_{2}$ and $i=2$}
%_____________________________________________________________
%__________________________________________________________________________________________
$i=2$
%__________________________________________________________________________________________
$k=2$
\begin{eqnarray*}
D_{2}D_{2}F_{2}&=&D_{2}D_{2}\left(R_{1}+F_{1}\right)+D_{2}R_{1}D_{2}F_{1}+D_{2}F_{1}D_{2}R_{1}
=D_{2}D_{2}R_{1}+D_{2}D_{2}F_{1}+D_{2}R_{1}D_{2}F_{1}+D_{2}F_{1}D_{2}R_{1}\\
&=&R_{1}^{(2)}\tilde{\mu}_{2}^{2}+r_{1}\tilde{P}_{2}^{(2)}
+D_{2}D_{2}F_{1}
2r_{1}\tilde{\mu}_{2}f_{1}\left(2\right)
\end{eqnarray*}

$k=3$
\begin{eqnarray*}
D_{3}D_{2}F_{2}&=&D_{3}D_{2}\left(R_{1}+F_{1}\right)+D_{2}R_{1}D_{3}\left(F_{1}+F_{3}\right)+D_{2}F_{1}D_{3}\left(R_{1}+F_{3}\right)\\
&=&D_{3}D_{2}R_{1}+D_{3}D_{2}F_{1}
+D_{2}R_{1}D_{3}F_{1}+D_{2}R_{1}D_{3}F_{3}
+D_{2}F_{1}D_{3}R_{1}+D_{2}F_{1}D_{3}F_{3}\\
&=&R_{1}^{(2)}\tilde{\mu}_{2}\tilde{\mu}_{3}+r_{1}\tilde{\mu}_{2}\tilde{\mu}_{3}
+D_{3}D_{2}F_{1}
+r_{1}\tilde{\mu}_{2}f_{1}\left(3\right)
+r_{1}\tilde{\mu}_{2}D_{3}F_{3}
+r_{1}\tilde{\mu}_{3}f_{1}\left(2\right)
+D_{3}\hat{F}_{3}f_{1}\left(2\right)
\end{eqnarray*}

$k=4$
\begin{eqnarray*}
D_{4}D_{2}F_{2}&=&D_{4}D_{2}\left(R_{1}+F_{1}\right)+D_{2}R_{1}D_{4}\left(F_{1}+F_{3}\right)+D_{2}F_{1}D_{4}\left(R_{1}+F_{3}\right)\\
&=&D_{4}D_{2}R_{1}+D_{4}D_{2}F_{1}
+D_{2}R_{1}D_{4}F_{1}+D_{2}R_{1}D_{4}F_{3}
+D_{2}F_{1}D_{4}R_{1}+D_{2}F_{1}D_{4}F_{3}\\
&=&R_{1}^{(2)}\tilde{\mu}_{2}\tilde{\mu}_{4}+r_{1}\tilde{\mu}_{2}\tilde{\mu}_{4}
+D_{4}D_{2}F_{1}
+r_{1}\tilde{\mu}_{2}f_{1}\left(4\right)
+r_{1}\tilde{\mu}_{2}D_{4}F_{3}
+r_{1}\tilde{\mu}_{4}f_{1}\left(2\right)
+D_{4}\hat{F}_{3}f_{1}\left(2\right)
\end{eqnarray*}

%_____________________________________________________________
\subsubsection*{$F_{2}$ and $i=3$}
%_____________________________________________________________
%__________________________________________________________________________________________
$i=3$
%__________________________________________________________________________________________
$k=3$
\begin{eqnarray*}
D_{3}D_{3}F_{2}&=&D_{3}D_{3}\left(R_{1}+F_{1}+F_{3}\right)
+D_{3}R_{1}D_{3}\left(F_{1}+F_{3}\right)
+D_{3}F_{1}D_{3}\left(R_{1}+F_{3}\right)
+D_{3}\hat{F}_{3}D_{3}\left(R_{1}+F_{1}\right)\\
&=&D_{3}D_{3}R_{1}+D_{3}D_{3}F_{1}+D_{3}D_{3}F_{3}
+D_{3}R_{1}D_{3}F_{1}+D_{3}R_{1}D_{3}F_{3}\\
&+&D_{3}F_{1}D_{3}R_{1}+D_{3}F_{1}D_{3}F_{3}
+D_{3}\hat{F}_{3}D_{3}R_{1}+D_{3}\hat{F}_{3}D_{3}F_{1}\\
&=&R_{1}^{(2)}\tilde{\mu}_{3}^{2}+r_{1}\tilde{P}_{3}^{(2)}
+D_{3}D_{3}F_{1}
+D_{3}D_{3}F_{3}
+r_{1}\tilde{\mu}_{3}f_{1}\left(3\right)
+r_{1}\tilde{\mu}_{3}f_{3}\left(3\right)\\
&+&r_{1}\tilde{\mu}_{3}f_{1}\left(3\right)
+D_{3}\hat{F}_{3}f_{1}\left(3\right)
+D_{3}\hat{F}_{3}r_{1}\tilde{\mu}_{3}
+D_{3}\hat{F}_{3}f_{1}\left(3\right)
\end{eqnarray*}

$k=4$
\begin{eqnarray*}
D_{4}D_{3}F_{2}&=&D_{4}D_{3}\left(R_{1}+F_{1}+F_{3}\right)
+D_{3}R_{1}D_{4}\left(F_{1}+F_{3}\right)
+D_{3}F_{1}D_{4}\left(R_{1}+F_{3}\right)
+D_{3}\hat{F}_{3}D_{4}\left(R_{1}+F_{1}\right)\\
&=&D_{4}D_{3}R_{1}+D_{4}D_{3}F_{1}+D_{4}D_{3}F_{3}
+D_{3}R_{1}D_{4}F_{1}+D_{3}R_{1}D_{4}F_{3}\\
&+&D_{3}F_{1}D_{4}R_{1}+D_{3}F_{1}D_{4}F_{3}
+D_{3}\hat{F}_{3}D_{4}R_{1}+D_{3}\hat{F}_{3}D_{4}F_{1}\\
&=&R_{1}^{(2)}\tilde{\mu}_{3}\tilde{\mu}_{4}+r_{1}\tilde{\mu}_{3}\tilde{\mu}_{4}
+D_{4}D_{3}F_{1}
+D_{4}D_{3}F_{3}
+r_{1}\tilde{\mu}_{3}f_{1}\left(4\right)
+r_{1}\tilde{\mu}_{3}D_{4}F_{3}\\
&+&r_{1}\tilde{\mu}_{4}f_{1}\left(3\right)
+D_{4}\hat{F}_{3}f_{1}\left(3\right)
+r_{1}\tilde{\mu}_{4}D_{3}F_{3}
+D_{3}\hat{F}_{3}f_{1}\left(4\right)
\end{eqnarray*}
%_____________________________________________________________
\subsubsection*{$F_{2}$ and $i=4$}
%_____________________________________________________________%__________________________________________________________________________________________
$i=4$ and $k=4$
\begin{eqnarray*}
D_{4}D_{4}F_{2}&=&D_{4}D_{4}\left(R_{1}+F_{1}+F_{3}\right)
+D_{4}R_{1}D_{4}\left(F_{1}+F_{3}\right)
+D_{4}F_{1}D_{4}\left(R_{1}+F_{3}\right)
+D_{4}\hat{F}_{3}D_{4}\left(R_{1}+F_{1}\right)\\
&=&D_{4}D_{4}R_{1}+D_{4}D_{4}F_{1}+D_{4}D_{4}F_{3}
+D_{4}R_{1}D_{4}F_{1}+D_{4}R_{1}D_{4}F_{3}\\
&+&D_{4}F_{1}D_{4}R_{1}+D_{4}F_{1}D_{4}F_{3}
+D_{4}\hat{F}_{3}D_{4}R_{1}+D_{4}\hat{F}_{3}D_{4}F_{1}\\
&=&R_{1}^{(2)}\tilde{\mu}_{4}^{2}+r_{1}\tilde{P}_{4}^{(2)}
+D_{4}D_{4}F_{1}
+D_{4}D_{4}F_{3}
+f_{1}\left(4\right)r_{1}\tilde{\mu}_{4}
+r_{1}\tilde{\mu}_{4}D_{4}F_{3}\\
&+&r_{1}\tilde{\mu}_{4}f_{1}\left(4\right)
+D_{4}\hat{F}_{3}f_{1}\left(4\right)
+D_{4}\hat{F}_{3}r_{1}\tilde{\mu}_{4}
+D_{4}\hat{F}_{3}f_{1}\left(4\right)
\end{eqnarray*}
%__________________________________________________________________________________________
\subsection*{$F_{3}$}
%__________________________________________________________________________________________

\begin{eqnarray}
D_{k}D_{i}F_{3}&=&D_{k}D_{i}\left(\hat{R}_{4}+\indora_{i\leq2}F_{2}+F_{4}\right)+D_{i}\hat{R}_{4}D_{k}\left(\indora_{k\leq2}F_{2}+F_{4}\right)+D_{i}F_{4}D_{k}\left(\hat{R}_{4}+\indora_{k\leq2}F_{2}\right)+\indora_{i\leq2}D_{i}F_{2}D_{k}\left(\hat{R}_{4}+F_{4}\right)
\end{eqnarray}
%__________________________________________________________________________________________
\subsubsection*{$F_{3}$, $i=1$}
%__________________________________________________________________________________________

%__________________________________________________________________________________________
$i=1$ and $k=1$
\begin{eqnarray*}
D_{1}D_{1}F_{3}&=&D_{1}D_{1}\left(\hat{R}_{4}+F_{2}+F_{4}\right)
+D_{1}\hat{R}_{4}D_{1}\left(F_{2}+F_{4}\right)
+D_{1}F_{4}D_{1}\left(\hat{R}_{4}+F_{2}\right)
+D_{1}F_{2}D_{1}\left(\hat{R}_{4}+F_{4}\right)\\
&=&D_{1}^{2}\hat{R}_{4}+D_{1}^{2}F_{2}+D_{1}^{2}F_{4}
+D_{1}\hat{R}_{4}D_{1}F_{2}+D_{1}\hat{R}_{4}D_{1}F_{4}
+D_{1}F_{4}D_{1}\hat{R}_{4}+D_{1}F_{4}D_{1}F_{2}
+D_{1}F_{2}D_{1}\hat{R}_{4}+D_{1}F_{2}D_{1}F_{4}\\
&=&\hat{R}_{2}^{(2)}\tilde{\mu}_{1}^{2}+\hat{r}_{2}\tilde{P}_{1}^{(2)}
+D_{1}^{2}F_{2}
+D_{1}^{2}F_{4}
+\hat{r}_{2}\tilde{\mu}_{1}D_{1}F_{2}\\
&+&\hat{r}_{2}\tilde{\mu}_{1}F_{4}\left(1\right)
+F_{4}\left(1\right)\hat{r}_{2}\tilde{\mu}_{1}
+F_{4}\left(1\right)D_{1}F_{2}
+D_{1}F_{2}\hat{r}_{2}\tilde{\mu}_{1}
+D_{1}F_{2}F_{4}\left(1\right)
\end{eqnarray*}

$k=2$
\begin{eqnarray*}
D_{2}D_{1}F_{3}&=&D_{2}D_{1}\left(\hat{R}_{4}+F_{2}+F_{4}\right)
+D_{1}\hat{R}_{4}D_{2}\left(F_{2}+F_{4}\right)
+D_{1}F_{4}D_{2}\left(\hat{R}_{4}+F_{2}\right)
+D_{1}F_{2}D_{2}\left(\hat{R}_{4}+F_{4}\right)\\
&=&D_{2}D_{1}\hat{R}_{4}+D_{2}D_{1}F_{2}+D_{2}D_{1}F_{4}
+D_{1}\hat{R}_{4}D_{2}F_{2}+D_{1}\hat{R}_{4}D_{2}F_{4}\\
&+&D_{1}F_{4}D_{2}\hat{R}_{4}+D_{1}F_{4}D_{2}F_{2}
+D_{1}F_{2}D_{2}\hat{R}_{4}+D_{1}F_{2}D_{2}F_{4}\\
&=&\hat{R}_{2}^{(2)}\tilde{\mu}_{1}\tilde{\mu}_{2}+\hat{r}_{2}\tilde{\mu}_{1}\tilde{\mu}_{2}
+D_{2}D_{1}F_{2}
+D_{2}D_{1}F_{4}
+\hat{r}_{2}\tilde{\mu}_{1}D_{2}F_{2}
+\hat{r}_{2}\tilde{\mu}_{1}F_{4}\left(2\right)\\
&+&\hat{r}_{2}\tilde{\mu}_{2}F_{4}\left(1\right)
+F_{4}\left(1\right)D_{2}F_{2}
+\hat{r}_{2}\tilde{\mu}_{2}D_{1}F_{2}
+D_{1}F_{2}F_{4}\left(2\right)
\end{eqnarray*}

$k=3$
\begin{eqnarray*}
D_{3}D_{1}F_{3}&=&D_{3}D_{1}\left(\hat{R}_{4}+F_{2}+F_{4}\right)
+D_{1}\hat{R}_{4}D_{3}\left(F_{4}\right)
+D_{1}F_{4}D_{3}\hat{R}_{4}
+D_{1}F_{2}D_{3}\left(\hat{R}_{4}+F_{4}\right)\\
&=&D_{3}D_{1}\hat{R}_{4}+D_{3}D_{1}F_{2}+D_{3}D_{1}F_{4}
+D_{1}\hat{R}_{4}D_{3}F_{4}
+D_{1}F_{4}D_{3}\hat{R}_{4}
+D_{1}F_{2}D_{3}\hat{R}_{4}+D_{1}F_{2}D_{3}F_{4}\\
&=&\hat{R}_{2}^{(2)}\tilde{\mu}_{1}\tilde{\mu}_{3}+\hat{r}_{2}\tilde{\mu}_{1}\tilde{\mu}_{3}
+D_{3}D_{1}F_{2}
+D_{3}D_{1}F_{4}
+\hat{r}_{2}\tilde{\mu}_{1}F_{4}\left(3\right)
+F_{4}\left(1\right)\hat{r}_{2}\tilde{\mu}_{3}
+D_{1}F_{2}\hat{r}_{2}\tilde{\mu}_{3}
+D_{1}F_{2}F_{4}\left(3\right)
\end{eqnarray*}

$k=4$
\begin{eqnarray*}
D_{4}D_{1}F_{3}&=&D_{4}D_{1}\left(\hat{R}_{4}+F_{2}+F_{4}\right)
+D_{1}\hat{R}_{4}D_{4}F_{4}
+D_{1}F_{4}D_{4}\hat{R}_{4}
+D_{1}F_{2}D_{4}\left(\hat{R}_{4}+F_{4}\right)\\
&=&D_{4}D_{1}\hat{R}_{4}+D_{4}D_{1}F_{2}+D_{4}D_{1}F_{4}
+D_{1}\hat{R}_{4}D_{4}F_{4}
+D_{1}F_{4}D_{4}\hat{R}_{4}
+D_{1}F_{2}D_{4}\hat{R}_{4}+D_{1}F_{2}D_{4}F_{4}\\
&=&\hat{R}_{2}^{(2)}\tilde{\mu}_{1}\tilde{\mu}_{4}+\hat{r}_{2}\tilde{\mu}_{1}\tilde{\mu}_{4}
+D_{4}D_{1}F_{2}
+D_{4}D_{1}F_{4}
+\hat{r}_{2}\tilde{\mu}_{1}F_{4}\left(4\right)
+F_{4}\left(1\right)\hat{r}_{2}\tilde{\mu}_{4}
+D_{1}F_{2}\hat{r}_{2}\tilde{\mu}_{4}
+D_{1}F_{2}F_{4}\left(4\right)
\end{eqnarray*}

%__________________________________________________________________________________________
\subsubsection*{$F_{3}$, $i=2$}
%__________________________________________________________________________________________

%__________________________________________________________________________________________
$i=2$ and $k=2$
\begin{eqnarray*}
D_{2}D_{2}F_{3}&=&D_{2}D_{2}\left(\hat{R}_{4}+F_{2}+F_{4}\right)
+D_{2}\hat{R}_{4}D_{2}\left(F_{2}+F_{4}\right)
+D_{2}F_{4}D_{2}\left(\hat{R}_{4}+F_{2}\right)
+D_{2}F_{2}D_{2}\left(\hat{R}_{4}+F_{4}\right)\\
&=&D_{2}D_{2}\hat{R}_{4}+D_{2}D_{2}F_{2}+D_{2}D_{2}F_{4}
+D_{2}\hat{R}_{4}D_{2}F_{2}+D_{2}\hat{R}_{4}D_{2}F_{4}\\
&+&D_{2}F_{4}D_{2}\hat{R}_{4}+D_{2}F_{4}D_{2}F_{2}
+D_{2}F_{2}D_{2}\hat{R}_{4}+D_{2}F_{2}D_{2}F_{4}\\
&=&\hat{R}_{2}^{(2)}\tilde{\mu}_{2}^{2}+\hat{r}_{2}\tilde{P}_{1}^{(2)}
+D_{2}D_{2}F_{2}
+D_{2}D_{2}F_{4}
+\hat{r}_{2}\tilde{\mu}_{2}D_{2}F_{2}
+\hat{r}_{2}\tilde{\mu}_{2}F_{4}\left(4\right)\\
&+&F_{4}\left(4\right)\hat{r}_{2}\tilde{\mu}_{2}
+F_{4}\left(4\right)D_{2}F_{2}
+D_{2}F_{2}\hat{r}_{2}\tilde{\mu}_{2}
+D_{2}F_{2}F_{4}\left(4\right)
\end{eqnarray*}

$k=3$
\begin{eqnarray*}
D_{3}D_{2}F_{3}&=&D_{3}D_{2}\left(\hat{R}_{4}+F_{2}+F_{4}\right)
+D_{2}\hat{R}_{4}D_{3}F_{4}
+D_{2}F_{4}D_{3}\hat{R}_{4}
+D_{2}F_{2}D_{3}\left(\hat{R}_{4}+F_{4}\right)\\
&=&D_{3}D_{2}\hat{R}_{4}+D_{3}D_{2}F_{2}+D_{3}D_{2}F_{4}
+D_{2}\hat{R}_{4}D_{3}F_{4}
+D_{2}F_{4}D_{3}\hat{R}_{4}
+D_{2}F_{2}D_{3}\hat{R}_{4}+D_{2}F_{2}D_{3}F_{4}\\
&=&\hat{R}_{2}^{(2)}\tilde{\mu}_{2}\tilde{\mu}_{3}+\hat{r}_{2}\tilde{\mu}_{2}\tilde{\mu}_{3}
+D_{3}D_{2}F_{2}
+D_{3}D_{2}F_{4}+\hat{r}_{2}\tilde{\mu}_{2}F_{4}\left(3\right)
+F_{4}\left(4\right)\hat{r}_{2}\tilde{\mu}_{3}
+\hat{r}_{2}\tilde{\mu}_{3}D_{2}F_{2}
+D_{2}F_{2}F_{4}\left(3\right)
\end{eqnarray*}

$k=4$
\begin{eqnarray*}
D_{4}D_{2}F_{3}&=&D_{4}D_{2}\left(\hat{R}_{4}+F_{2}+F_{4}\right)
+D_{2}\hat{R}_{4}D_{4}F_{4}
+D_{2}F_{4}D_{4}\hat{R}_{4}
+D_{2}F_{2}D_{4}\left(\hat{R}_{4}+F_{4}\right)\\
&=&D_{4}D_{2}\hat{R}_{4}+D_{4}D_{2}F_{2}+D_{4}D_{2}F_{4}
+D_{2}\hat{R}_{4}D_{4}F_{4}
+D_{2}F_{4}D_{4}\hat{R}_{4}
+D_{2}F_{2}D_{4}\hat{R}_{4}+D_{2}F_{2}D_{4}F_{4}\\
&=&\hat{R}_{2}^{(2)}\tilde{\mu}_{2}\tilde{\mu}_{4}+\hat{r}_{2}\tilde{\mu}_{2}\tilde{\mu}_{4}
+D_{4}D_{2}F_{2}
+D_{4}D_{2}F_{4}
+\hat{r}_{2}\tilde{\mu}_{2}F_{4}\left(4\right)
+F_{4}\left(4\right)\hat{r}_{2}\tilde{\mu}_{4}
+D_{2}F_{2}\hat{r}_{2}\tilde{\mu}_{4}
+D_{2}F_{2}F_{4}\left(4\right)
\end{eqnarray*}
%__________________________________________________________________________________________
\subsubsection*{$F_{3}$, $i=3$}
%__________________________________________________________________________________________

$k=3$
\begin{eqnarray*}
D_{3}D_{3}F_{3}&=&D_{3}D_{3}\left(\hat{R}_{4}+F_{4}\right)
+D_{3}\hat{R}_{4}D_{3}F_{4}
+D_{3}F_{4}D_{3}\hat{R}_{4}=D_{3}^{2}\hat{R}_{4}+D_{3}^{2}F_{4}
+D_{3}\hat{R}_{4}D_{3}F_{4}
+D_{3}F_{4}D_{3}\hat{R}_{4}\\
&=&\hat{R}_{2}^{(2)}\tilde{\mu}_{3}^{2}+\hat{r}_{2}\tilde{P}_{3}^{(2)}
+D_{3}^{2}F_{4}
+\hat{r}_{2}\tilde{\mu}_{3}F_{4}\left(4\right)
+\hat{r}_{2}\tilde{\mu}_{3}F_{4}\left(3\right)
\end{eqnarray*}

$k=4$
\begin{eqnarray*}
D_{4}D_{3}F_{3}&=&D_{4}D_{3}\left(\hat{R}_{4}+F_{4}\right)
+D_{3}\hat{R}_{4}D_{4}F_{4}
+D_{3}F_{4}D_{4}\hat{R}_{4}=D_{4}D_{3}\hat{R}_{4}+D_{4}D_{3}F_{4}
+D_{3}\hat{R}_{4}D_{4}F_{4}
+D_{3}F_{4}D_{4}\hat{R}_{4}\\
&=&\hat{R}_{2}^{(2)}\tilde{\mu}_{3}\tilde{\mu}_{4}+\hat{r}_{2}\tilde{\mu}_{3}\tilde{\mu}_{4}
+D_{4}D_{3}F_{4}
+\hat{r}_{2}\tilde{\mu}_{3}F_{4}\left(4\right)
+\hat{r}_{2}\tilde{\mu}_{4}F_{4}\left(3\right)
\end{eqnarray*}
%__________________________________________________________________________________________
\subsubsection*{$F_{3}$, $i=4$}
%__________________________________________________________________________________________

$k=4$
\begin{eqnarray*}
D_{4}D_{4}F_{3}&=&D_{4}D_{4}\left(\hat{R}_{4}+F_{4}\right)
+D_{4}\hat{R}_{4}D_{4}F_{4}
+D_{4}F_{4}D_{4}\hat{R}_{4}=D_{4}^{2}\hat{R}_{4}+D_{4}^{2}F_{4}
+D_{4}\hat{R}_{4}D_{4}F_{4}
+D_{4}F_{4}D_{4}\hat{R}_{4}\\
&=&\hat{R}_{2}^{(2)}\tilde{\mu}_{4}^{2}+\hat{r}_{2}\tilde{P}_{4}^{(2)}+D_{4}^{2}F_{4}
+2\hat{r}_{2}\tilde{\mu}_{4}F_{4}\left(4\right)
\end{eqnarray*}
%__________________________________________________________________________________________
%
%__________________________________________________________________________________________
\subsection*{$F_{4}$}
%__________________________________________________________________________________________
for $F_{4}$
%__________________________________________________________________________________________
%
%__________________________________________________________________________________________

\begin{eqnarray}
D_{k}D_{i}F_{4}&=&D_{k}D_{i}\left(\hat{R}_{3}+\indora_{i\leq2}F_{1}+F_{3}\right)+D_{i}\hat{R}_{3}D_{k}\left(\indora_{k\leq2}F_{1}+F_{3}\right)+D_{i}\hat{F}_{3}D_{k}\left(\hat{R}_{3}+\indora_{k\leq2}F_{1}\right)+\indora_{i\leq2}D_{i}F_{1}D_{k}\left(\hat{R}_{3}+F_{3}\right)\\
&=&
\end{eqnarray}
%__________________________________________________________________________________________
\subsubsection*{$F_{4}$, $i=1$}
%__________________________________________________________________________________________

$k=1$
\begin{eqnarray*}
D_{1}D_{1}F_{4}&=&D_{1}^{2}\left(\hat{R}_{3}+F_{1}+F_{3}\right)
+D_{1}\hat{R}_{3}D_{1}\left(F_{1}+F_{3}\right)
+D_{1}\hat{F}_{3}D_{1}\left(\hat{R}_{3}+F_{1}\right)
+D_{1}F_{1}D_{1}\left(\hat{R}_{3}+F_{3}\right)\\
&=&D_{1}^{2}\hat{R}_{3}+D_{1}^{2}F_{1}+D_{1}^{2}F_{3}
+D_{1}\hat{R}_{3}D_{1}F_{1}+D_{1}\hat{R}_{3}D_{1}F_{3}
+D_{1}\hat{F}_{3}D_{1}\hat{R}_{3}+D_{1}\hat{F}_{3}D_{1}F_{1}
+D_{1}F_{1}D_{1}\hat{R}_{3}+D_{1}F_{1}D_{1}F_{3}\\
&=&
\hat{R}_{1}^{(2)}\tilde{\mu}_{1}^{2}+\hat{r}_{1}\tilde{P}_{2}^{(2)}
+D_{1}^{2}F_{1}
+D_{1}^{2}F_{3}
+D_{1}F_{1}\hat{r}_{1}\tilde{\mu}_{1}\\
&+&\hat{r}_{1}\tilde{\mu}_{1}F_{3}\left(1\right)
+\hat{r}_{1}\tilde{\mu}_{1}F_{3}\left(1\right)
+D_{1}F_{1}F_{3}\left(1\right)
+D_{1}F_{1}\hat{r}_{1}\tilde{\mu}_{1}
+D_{1}F_{1}F_{3}\left(1\right)
\end{eqnarray*}

$k=2$
\begin{eqnarray*}
D_{2}D_{1}F_{4}&=&D_{2}D_{1}\left(\hat{R}_{3}+F_{1}+F_{3}\right)
+D_{1}\hat{R}_{3}D_{2}\left(F_{1}+F_{3}\right)
+D_{1}\hat{F}_{3}D_{2}\left(\hat{R}_{3}+F_{1}\right)
+D_{1}F_{1}D_{2}\left(\hat{R}_{3}+F_{3}\right)\\
&=&D_{2}D_{1}\hat{R}_{3}+D_{2}D_{1}F_{1}+D_{2}D_{1}F_{3}
+D_{1}\hat{R}_{3}D_{2}F_{1}+D_{1}\hat{R}_{3}D_{2}F_{3}\\
&+&D_{1}\hat{F}_{3}D_{2}\hat{R}_{3}+D_{1}\hat{F}_{3}D_{2}F_{1}
+D_{1}F_{1}D_{2}\hat{R}_{3}+D_{1}F_{1}D_{2}F_{3}\\
&=&\hat{R}_{1}^{(2)}\tilde{\mu}_{1}\tilde{\mu}_{2}+\hat{r}_{1}\tilde{\mu}_{1}\tilde{\mu}_{2}
+D_{2}D_{1}F_{1}
+D_{2}D_{1}F_{3}
+\hat{r}_{1}\tilde{\mu}_{1}D_{2}F_{1}
+\hat{r}_{1}\tilde{\mu}_{1}F_{3}\left(2\right)\\
&+&F_{3}\left(1\right)\hat{r}_{1}\tilde{\mu}_{2}
+\hat{r}_{1}\tilde{\mu}_{1}D_{2}F_{1}
+D_{1}F_{1}\hat{r}_{1}\tilde{\mu}_{2}
+D_{1}F_{1}F_{3}\left(2\right)
\end{eqnarray*}

$k=3$
\begin{eqnarray*}
D_{3}D_{1}F_{4}&=&D_{3}D_{1}\left(\hat{R}_{3}+F_{1}+F_{3}\right)
+D_{1}\hat{R}_{3}D_{3}F_{3}
+D_{1}\hat{F}_{3}D_{3}\hat{R}_{3}
+D_{1}F_{1}D_{3}\left(\hat{R}_{3}+F_{3}\right)\\
&=&D_{3}D_{1}\hat{R}_{3}+D_{3}D_{1}F_{1}+D_{3}D_{1}F_{3}
+D_{1}\hat{R}_{3}D_{3}F_{3}
+D_{1}\hat{F}_{3}D_{3}\hat{R}_{3}
+D_{1}F_{1}D_{3}\hat{R}_{3}+D_{1}F_{1}D_{3}F_{3}\\
&=&\hat{R}_{1}^{(2)}\tilde{\mu}_{1}\tilde{\mu}_{3}+\hat{r}_{1}\tilde{\mu}_{1}\tilde{\mu}_{3}
+D_{3}D_{1}F_{1}
+D_{3}D_{1}F_{3}
+\hat{r}_{1}\tilde{\mu}_{1}F_{3}\left(3\right)
+\hat{r}_{1}\tilde{\mu}_{3}F_{3}\left(1\right)
+\hat{r}_{1}\tilde{\mu}_{3}D_{1}F_{1}
+D_{1}F_{1}F_{3}\left(3\right)
\end{eqnarray*}

$k=4$
\begin{eqnarray*}
D_{4}D_{1}F_{4}&=&D_{4}D_{1}\left(\hat{R}_{3}+F_{1}+F_{3}\right)
+D_{1}\hat{R}_{3}D_{4}F_{3}
+D_{1}\hat{F}_{3}D_{4}\hat{R}_{3}
+D_{1}F_{1}D_{4}\left(\hat{R}_{3}+F_{3}\right)\\
&=&D_{4}D_{1}\hat{R}_{3}+D_{4}D_{1}F_{1}+D_{4}D_{1}F_{3}
+D_{1}\hat{R}_{3}D_{4}F_{3}
+D_{1}\hat{F}_{3}D_{4}\hat{R}_{3}
+D_{1}F_{1}D_{4}\hat{R}_{3}+D_{1}F_{1}D_{4}F_{3}\\
&=&\hat{R}_{1}^{(2)}\tilde{\mu}_{1}\tilde{\mu}_{4}+\hat{r}_{1}\tilde{\mu}_{1}\tilde{\mu}_{4}
+D_{4}D_{1}F_{1}
+D_{4}D_{1}F_{3}
+F_{3}\left(4\right)\hat{r}_{1}\tilde{\mu}_{1}
+F_{3}\left(3\right)\hat{r}_{1}\tilde{\mu}_{4}
+D_{1}F_{1}\hat{r}_{1}\tilde{\mu}_{4}
+D_{1}F_{1}F_{3}\left(4\right)
\end{eqnarray*}
%__________________________________________________________________________________________
\subsubsection*{$F_{4}$, $i=2$}
%__________________________________________________________________________________________


$k=2$
\begin{eqnarray*}
D_{2}D_{2}F_{4}&=&D_{2}D_{2}\left(\hat{R}_{3}+F_{1}+F_{3}\right)
+D_{2}\hat{R}_{3}D_{2}\left(F_{1}+F_{3}\right)
+D_{2}\hat{F}_{3}D_{2}\left(\hat{R}_{3}+F_{1}\right)
+D_{2}F_{1}D_{2}\left(\hat{R}_{3}+F_{3}\right)\\
&=&D_{2}^{2}\hat{R}_{3}+D_{2}^{2}F_{1}+D_{2}^{2}F_{3}
+D_{2}\hat{R}_{3}D_{2}F_{1}+D_{2}\hat{R}_{3}D_{2}F_{3}
+D_{2}\hat{F}_{3}D_{2}\hat{R}_{3}+D_{2}\hat{F}_{3}D_{2}F_{1}
+D_{2}F_{1}D_{2}\hat{R}_{3}+D_{2}F_{1}D_{2}F_{3}\\
&=&\hat{R}_{1}^{(2)}\tilde{\mu}_{2}^{2}+\hat{r}_{1}\tilde{P}_{2}^{(2)}
+D_{2}^{2}F_{1}
+D_{2}^{2}F_{3}
+\hat{r}_{1}\tilde{\mu}_{2}D_{2}F_{1}\\
&+&\hat{r}_{1}\tilde{\mu}_{2}F_{3}\left(2\right)
+\hat{r}_{1}\tilde{\mu}_{2}F_{3}\left(2\right)
+F_{3}\left(1\right)D_{2}F_{1}
+\hat{r}_{1}\tilde{\mu}_{2}D_{2}F_{1}
+F_{3}\left(3\right)D_{2}F_{1}
\end{eqnarray*}

$k=3$
\begin{eqnarray*}
D_{3}D_{2}F_{4}&=&D_{3}D_{2}\left(\hat{R}_{3}+F_{1}+F_{3}\right)
+D_{2}\hat{R}_{3}D_{3}F_{3}
+D_{2}\hat{F}_{3}D_{3}\hat{R}_{3}
+D_{2}F_{1}D_{3}\left(\hat{R}_{3}+F_{3}\right)\\
&=&D_{3}D_{2}\hat{R}_{3}+D_{3}D_{2}F_{1}+D_{3}D_{2}F_{3}
+D_{2}\hat{R}_{3}D_{3}F_{3}
+D_{2}\hat{F}_{3}D_{3}\hat{R}_{3}
+D_{2}F_{1}D_{3}\hat{R}_{3}+D_{2}F_{1}D_{3}F_{3}\\
&=&\hat{R}_{1}^{(2)}\tilde{\mu}_{2}\tilde{\mu}_{3}+\hat{r}_{1}\tilde{\mu}_{2}\tilde{\mu}_{3}
+D_{3}D_{2}F_{1}
+D_{3}D_{2}F_{3}
+\hat{r}_{1}\tilde{\mu}_{2}F_{3}\left(3\right)
+\hat{r}_{1}\tilde{\mu}_{3}F_{3}\left(2\right)
+\hat{r}_{1}\tilde{\mu}_{3}D_{2}F_{1}
+F_{3}\left(3\right)D_{2}F_{1}
\end{eqnarray*}

$k=4$
\begin{eqnarray*}
D_{4}D_{2}F_{4}&=&D_{4}D_{2}\left(\hat{R}_{3}+F_{1}+F_{3}\right)
+D_{2}\hat{R}_{3}D_{4}F_{3}
+D_{2}\hat{F}_{3}D_{4}\hat{R}_{3}
+D_{2}F_{1}D_{4}\left(\hat{R}_{3}+F_{3}\right)\\
&=&D_{4}D_{2}\hat{R}_{3}+D_{4}D_{2}F_{1}+F_{3}
+D_{2}\hat{R}_{3}D_{4}F_{3}
+D_{2}\hat{F}_{3}D_{4}\hat{R}_{3}
+D_{2}F_{1}D_{4}\hat{R}_{3}+D_{2}F_{1}D_{4}F_{3}\\
&=&\hat{R}_{1}^{(2)}\tilde{\mu}_{2}\tilde{\mu}_{4}+\hat{r}_{1}\tilde{\mu}_{2}\tilde{\mu}_{4}
+D_{4}D_{2}F_{1}
+D_{4}D_{2}F_{3}
+\hat{r}_{1}\tilde{\mu}_{2}F_{3}\left(4\right)
+\hat{r}_{1}\tilde{\mu}_{4}F_{3}\left(2\right)
+\hat{r}_{1}\tilde{\mu}_{4}D_{2}F_{1}
+F_{3}\left(4\right)D_{2}F_{1}
\end{eqnarray*}
%__________________________________________________________________________________________
\subsubsection*{$F_{4}$, $i=3$}
%__________________________________________________________________________________________

$k=3$
\begin{eqnarray*}
D_{3}D_{3}F_{4}&=&D_{3}D_{3}\left(\hat{R}_{3}+F_{3}\right)
+D_{3}\hat{R}_{3}D_{3}F_{3}
+D_{3}\hat{F}_{3}D_{3}\hat{R}_{3}=D_{3}^{2}\hat{R}_{3}+D_{3}^{2}F_{3}
+D_{3}\hat{R}_{3}D_{3}F_{3}
+D_{3}\hat{F}_{3}D_{3}\hat{R}_{3}\\
&=&\hat{R}_{1}^{(2)}\tilde{\mu}_{3}^{2}+\hat{r}_{1}\tilde{P}_{3}^{(2)}
+D_{3}^{2}F_{3}
+\hat{r}_{1}\tilde{\mu}_{3}F_{3}\left(3\right)
+\hat{r}_{1}\tilde{\mu}_{3}F_{3}\left(3\right)
\end{eqnarray*}

$k=4$
\begin{eqnarray*}
D_{4}D_{3}F_{4}&=&D_{4}D_{3}\left(\hat{R}_{3}+F_{3}\right)
+D_{3}\hat{R}_{3}D_{4}F_{3}
+D_{3}\hat{F}_{3}D_{4}\hat{R}_{3}=D_{4}D_{3}\hat{R}_{3}+D_{4}D_{3}F_{3}
+D_{3}\hat{R}_{3}D_{4}F_{3}
+D_{3}\hat{F}_{3}D_{4}\hat{R}_{3}\\
&=&\hat{R}_{1}^{(2)}\tilde{\mu}_{3}\tilde{\mu}_{4}+\hat{r}_{1}\tilde{\mu}_{3}\tilde{\mu}_{4}
+D_{4}D_{3}F_{3}
+\hat{r}_{1}\tilde{\mu}_{3}F_{3}\left(4\right)
+\hat{r}_{1}\tilde{\mu}_{4}F_{3}\left(3\right)
\end{eqnarray*}
%__________________________________________________________________________________________
$i=4$
%__________________________________________________________________________________________

$k=4$
\begin{eqnarray*}
D_{4}D_{4}F_{4}&=&D_{4}^{2}\left(\hat{R}_{3}+F_{3}\right)
+D_{4}\hat{R}_{3}D_{4}F_{3}
+D_{4}\hat{F}_{3}D_{4}\hat{R}_{3}=D_{4}^{2}\hat{R}_{3}+D_{4}^{2}F_{3}
+D_{4}\hat{R}_{3}D_{4}F_{3}
+D_{4}\hat{F}_{3}D_{4}\hat{R}_{3}\\
&=&\hat{R}_{1}^{(2)}\tilde{\mu}_{4}^{2}+\hat{r}_{1}\tilde{P}_{4}^{(2)}
+D_{4}^{2}F_{3}
+\hat{r}_{1}\tilde{\mu}_{4}F_{3}\left(4\right)
\end{eqnarray*}
%__________________________________________________________________________________________
%

%_____________________________________________________________________________________
\newpage

%__________________________________________________________________
\section{Generalizaciones}
%__________________________________________________________________
\subsection{RSVC con dos conexiones}
%__________________________________________________________________

%\begin{figure}[H]
%\centering
%%%\includegraphics[width=9cm]{Grafica3.jpg}
%%\end{figure}\label{RSVC3}


Sus ecuaciones recursivas son de la forma

\begin{eqnarray*}
F_{1}\left(z_{1},z_{2},z_{3},z_{4}\right)&=&R_{2}\left(\prod_{i=1}^{4}\tilde{P}_{i}\left(z_{i}\right)\right)F_{2}\left(z_{1},\tilde{\theta}_{2}\left(\tilde{P}_{1}\left(z_{1}\right)\tilde{P}_{3}\left(z_{3}\right)\tilde{P}_{4}\left(z_{4}\right)\right)\right)
F_{4}\left(z_{3},z_{4};\tau_{2}\right),
\end{eqnarray*}

\begin{eqnarray*}
F_{2}\left(z_{1},z_{2},z_{3},z_{4}\right)&=&R_{1}\left(\prod_{i=1}^{4}\tilde{P}_{i}\left(z_{i}\right)\right)
F_{1}\left(\tilde{\theta}_{1}\left(\tilde{P}_{2}\left(z_{2}\right)\tilde{P}_{1}\left(z_{3}\right)\tilde{P}_{4}\left(z_{4}\right)\right),z_{2}\right)
F_{3}\left(z_{3},z_{4};\tau_{1}\right),
\end{eqnarray*}


\begin{eqnarray*}
F_{3}\left(z_{1},z_{2},z_{3},z_{4}\right)&=&R_{4}\left(\prod_{i=1}^{4}\tilde{P}_{i}\left(z_{i}\right)\right)
F_{4}\left(z_{3},\tilde{\theta}_{4}\left(\tilde{P}_{1}\left(z_{1}\right)\tilde{P}_{2}\left(z_{2}\right)\tilde{P}_{3}\left(z_{3}\right)
\right)\right)
F_{2}\left(z_{1},z_{2};\zeta_{4}\right),
\end{eqnarray*}

\begin{eqnarray*}
F_{4}\left(z_{1},z_{2},z_{3},z_{4}\right)&=&R_{3}\left(\prod_{i=1}^{4}\tilde{P}_{i}\left(z_{i}\right)\right)
F_{3}\left(\tilde{\theta}_{3}\left(\tilde{P}_{1}\left(z_{1}\right)\tilde{P}_{2}\left(z_{2}\right)\tilde{P}_{4}\left(z_{4}
\right)\right),z_{4}\right)
F_{1}\left(z_{1},z_{2};\zeta_{3}\right),
\end{eqnarray*}


\begin{eqnarray*}
F_{1}\left(z_{1},z_{2},w_{1},w_{2}\right)&=&R_{2}\left(\prod_{i=1}^{2}\tilde{P}_{i}\left(z_{i}\right)\prod_{i=1}^{2}
\hat{P}_{i}\left(w_{i}\right)\right)F_{2}\left(z_{1},\tilde{\theta}_{2}\left(\tilde{P}_{1}\left(z_{1}\right)\hat{P}_{1}\left(w_{1}\right)\hat{P}_{2}\left(w_{2}\right)\right)\right)
\hat{F}_{2}\left(w_{1},w_{2};\tau_{2}\right),
\end{eqnarray*}

\begin{eqnarray*}
F_{2}\left(z_{1},z_{2},w_{1},w_{2}\right)&=&R_{1}\left(\prod_{i=1}^{2}\tilde{P}_{i}\left(z_{i}\right)\prod_{i=1}^{2}
\hat{P}_{i}\left(w_{i}\right)\right)F_{1}\left(\tilde{\theta}_{1}\left(\tilde{P}_{2}\left(z_{2}\right)\hat{P}_{1}\left(w_{1}\right)\hat{P}_{2}\left(w_{2}\right)\right),z_{2}\right)\hat{F}_{1}\left(w_{1},w_{2};\tau_{1}\right),
\end{eqnarray*}


\begin{eqnarray*}
\hat{F}_{1}\left(z_{1},z_{2},w_{1},w_{2}\right)&=&\hat{R}_{2}\left(\prod_{i=1}^{2}\tilde{P}_{i}\left(z_{i}\right)\prod_{i=1}^{2}
\hat{P}_{i}\left(w_{i}\right)\right)F_{2}\left(z_{1},z_{2};\zeta_{2}\right)\hat{F}_{2}\left(w_{1},\hat{\theta}_{2}\left(\tilde{P}_{1}\left(z_{1}\right)\tilde{P}_{2}\left(z_{2}\right)\hat{P}_{1}\left(w_{1}
\right)\right)\right),
\end{eqnarray*}


\begin{eqnarray*}
\hat{F}_{2}\left(z_{1},z_{2},w_{1},w_{2}\right)&=&\hat{R}_{1}\left(\prod_{i=1}^{2}\tilde{P}_{i}\left(z_{i}\right)\prod_{i=1}^{2}
\hat{P}_{i}\left(w_{i}\right)\right)F_{1}\left(z_{1},z_{2};\zeta_{1}\right)\hat{F}_{1}\left(\hat{\theta}_{1}\left(\tilde{P}_{1}\left(z_{1}\right)\tilde{P}_{2}\left(z_{2}\right)\hat{P}_{2}\left(w_{2}\right)\right),w_{2}\right),
\end{eqnarray*}

%_____________________________________________________
\subsection{First Moments of the Queue Lengths}
%_____________________________________________________


The server's switchover times are given by the general equation

\begin{eqnarray}\label{Ec.Ri}
R_{i}\left(\mathbf{z,w}\right)=R_{i}\left(\tilde{P}_{1}\left(z_{1}\right)
\tilde{P}_{2}\left(z_{2}\right)\tilde{P}_{3}\left(z_{3}\right)
\tilde{P}_{4}\left(z_{4}\right)\right)
\end{eqnarray}

with
\begin{eqnarray}\label{Ec.Derivada.Ri}
D_{i}R_{i}&=&DR_{i}D_{i}\tilde{P}_{i}
\end{eqnarray}

also we need to recall that


$F_{1}\left(z_{1},z_{2};\tau_{3}\right)=F_{1,1}\left(z_{1};\tau_{3}\right)F_{2,1}\left(z_{2};\tau_{3}\right)$
then


$D_{1}F_{1}\left(z_{1},z_{2};\tau_{3}\right)=D_{1}F_{1,1}\left(z_{1};\tau_{3}\right)=F_{1,1}^{(1)}\left(1\right)$, and
$D_{2}F_{1}\left(z_{1},z_{2};\tau_{3}\right)=D_{2}F_{2,1}\left(z_{1};\tau_{3}\right)=F_{2,1}^{(1)}\left(1\right)$, with second order derivatives given by

\begin{eqnarray*}
D_{1}^{2}F_{1}\left(z_{1},z_{2};\tau_{3}\right)&=&D_{1}^{2}F_{1,1}\left(z_{1};\tau_{3}\right)=F_{1,1}^{(2)}\left(1\right)\\
D_{2}D_{1}F_{1}\left(z_{1},z_{2};\tau_{3}\right)&=&D_{2}F_{2,1}\left(z_{2};\tau_{3}\right)D_{1}F_{1,1}\left(z_{1};\tau_{3}\right)=F_{1,1}^{(1)}\left(1\right)F_{1,1}^{(1)}\left(1\right)\\
D_{2}^{2}F_{1}\left(z_{1},z_{2};\tau_{3}\right)&=&D_{2}^{2}F_{2,1}\left(z_{2};\tau_{3}\right)=F_{2,1}^{(2)}\left(1\right)
\end{eqnarray*}

in a similar manner we can obtain the following for
\begin{itemize}
\item $F_{2}\left(z_{1},z_{2};\tau_{4}\right)=F_{1,2}\left(z_{1};\tau_{4}\right)F_{2,2}\left(z_{2};\tau_{4}\right)$ with


$D_{1}F_{2}\left(z_{1},z_{2};\tau_{4}\right)=D_{1}F_{1,2}\left(z_{1};\tau_{4}\right)=F_{1,2}^{(1)}\left(1\right)$, and
$D_{2}F_{2}\left(z_{1},z_{2};\tau_{4}\right)=D_{2}F_{2,2}\left(z_{1};\tau_{4}\right)=F_{2,2}^{(1)}\left(1\right)$, with

\begin{eqnarray*}
D_{1}^{2}F_{2}\left(z_{1},z_{2};\tau_{4}\right)&=&D_{1}^{2}F_{1,2}\left(z_{1};\tau_{4}\right)=F_{1,2}^{(2)}\left(1\right)\\
D_{2}D_{1}F_{2}\left(z_{1},z_{2};\tau_{4}\right)&=&D_{2}F_{2,1}\left(z_{2};\tau_{4}\right)D_{1}F_{1,2}\left(z_{1};\tau_{4}\right)=F_{2,2}^{(1)}\left(1\right)F_{1,2}^{(1)}\left(1\right)\\
D_{2}^{2}F_{2}\left(z_{1},z_{2};\tau_{4}\right)&=&D_{2}^{2}F_{2,2}\left(z_{2};\tau_{4}\right)=F_{2,2}^{(2)}\left(1\right)
\end{eqnarray*}

\item $F_{3}\left(z_{3},z_{3};\zeta_{2}\right)=F_{1,2}\left(z_{1};\zeta_{2}\right)F_{2,2}\left(z_{2};\zeta_{2}\right)$ with


$D_{1}F_{2}\left(z_{1},z_{2};\zeta_{2}\right)=D_{1}F_{1,2}\left(z_{1};\zeta_{2}\right)=F_{1,2}^{(1)}\left(1\right)$, and
$D_{2}F_{2}\left(z_{1},z_{2};\zeta_{2}\right)=D_{2}F_{2,2}\left(z_{1};\zeta_{2}\right)=F_{2,2}^{(1)}\left(1\right)$, with

\begin{eqnarray*}
D_{1}^{2}F_{2}\left(z_{1},z_{2};\zeta_{2}\right)&=&D_{1}^{2}F_{1,2}\left(z_{1};\zeta_{2}\right)=F_{1,2}^{(2)}\left(1\right)\\
D_{2}D_{1}F_{2}\left(z_{1},z_{2};\zeta_{2}\right)&=&D_{2}F_{2,1}\left(z_{2};\zeta_{1}\right)D_{1}F_{1,2}\left(z_{1};\zeta_{2}\right)=F_{2,2}^{(1)}\left(1\right)F_{1,2}^{(1)}\left(1\right)\\
D_{2}^{2}F_{2}\left(z_{1},z_{2};\zeta_{2}\right)&=&D_{2}^{2}F_{2,2}\left(z_{2};\zeta_{2}\right)=F_{2,2}^{(2)}\left(1\right)
\end{eqnarray*}



\end{itemize}































also we need to remind $F_{1,2}\left(z_{1};\zeta_{2}\right)F_{2,2}\left(z_{2};\zeta_{2}\right)=F_{2}\left(z_{1},z_{2};\zeta_{2}\right)$, therefore

\begin{eqnarray*}
D_{1}F_{2}\left(z_{1},z_{2};\zeta_{2}\right)&=&D_{1}\left[F_{1,2}\left(z_{1};\zeta_{2}\right)F_{2,2}\left(z_{2};\zeta_{2}\right)\right]
=F_{2,2}\left(z_{2};\zeta_{2}\right)D_{1}F_{1,2}\left(z_{1};\zeta_{2}\right)=F_{1,2}^{(1)}\left(1\right)
\end{eqnarray*}

i.e., $D_{1}F_{2}=F_{1,2}^{(1)}(1)$; $D_{2}F_{2}=F_{2,2}^{(1)}\left(1\right)$, whereas that $D_{3}F_{2}=D_{4}F_{2}=0$, then

\begin{eqnarray}
\begin{array}{ccc}
D_{i}F_{j}=\indora_{i\leq2}F_{i,j}^{(1)}\left(1\right),& \textrm{ and } &D_{i}\hat{F}_{j}=\indora_{i\geq2}F_{i,j}^{(1)}\left(1\right).
\end{array}
\end{eqnarray}

Now, we obtain the first moments equations for the queue lengths as before for the times the server arrives to the queue to start attending



Remember that


\begin{eqnarray*}
F_{2}\left(z_{1},z_{2},w_{1},w_{2}\right)&=&R_{1}\left(\prod_{i=1}^{2}\tilde{P}_{i}\left(z_{i}\right)\prod_{i=1}^{2}
\hat{P}_{i}\left(w_{i}\right)\right)F_{1}\left(\tilde{\theta}_{1}\left(\tilde{P}_{2}\left(z_{2}\right)\hat{P}_{1}\left(w_{1}\right)\hat{P}_{2}\left(w_{2}\right)\right),z_{2}\right)\hat{F}_{1}\left(w_{1},w_{2};\tau_{1}\right),
\end{eqnarray*}

where


\begin{eqnarray*}
F_{1}\left(\tilde{\theta}_{1}\left(\tilde{P}_{2}\hat{P}_{1}\hat{P}_{2}\right),z_{2}\right)
\end{eqnarray*}

so

\begin{eqnarray}
D_{i}F_{1}&=&\indora_{i\neq1}D_{1}F_{1}D\tilde{\theta}_{1}D_{i}P_{i}+\indora_{i=2}D_{i}F_{1},
\end{eqnarray}

then


\begin{eqnarray*}
\begin{array}{ll}
D_{1}F_{1}=0,&
D_{2}F_{1}=D_{1}F_{1}D\tilde{\theta}_{1}D_{2}P_{2}+D_{2}F_{1}
=f_{1}\left(1\right)\frac{1}{1-\tilde{\mu}_{1}}\tilde{\mu}_{2}+f_{1}\left(2\right),\\
D_{3}F_{1}=D_{1}F_{1}D\tilde{\theta}_{1}D_{3}P_{3}
=f_{1}\left(1\right)\frac{1}{1-\tilde{\mu}_{1}}\hat{\mu}_{1},&
D_{4}F_{1}=D_{1}F_{1}D\tilde{\theta}_{1}D_{4}P_{4}
=f_{1}\left(1\right)\frac{1}{1-\tilde{\mu}_{1}}\hat{\mu}_{2}

\end{array}
\end{eqnarray*}


\begin{eqnarray}
D_{i}F_{2}&=&\indora_{i\neq2}D_{2}F_{2}D\tilde{\theta}_{2}D_{i}P_{i}
+\indora_{i=1}D_{i}F_{2}
\end{eqnarray}

\begin{eqnarray*}
\begin{array}{ll}
D_{1}F_{2}=D_{2}F_{2}D\tilde{\theta}_{2}D_{1}P_{1}
+D_{1}F_{2}=f_{2}\left(2\right)\frac{1}{1-\tilde{\mu}_{2}}\tilde{\mu}_{1},&
D_{2}F_{2}=0\\
D_{3}F_{2}=D_{2}F_{2}D\tilde{\theta}_{2}D_{3}P_{3}
=f_{2}\left(2\right)\frac{1}{1-\tilde{\mu}_{2}}\hat{\mu}_{1},&
D_{4}F_{2}=D_{2}F_{2}D\tilde{\theta}_{2}D_{4}P_{4}
=f_{2}\left(2\right)\frac{1}{1-\tilde{\mu}_{2}}\hat{\mu}_{2}
\end{array}
\end{eqnarray*}



\begin{eqnarray}
D_{i}\hat{F}_{1}&=&\indora_{i\neq3}D_{3}\hat{F}_{1}D\hat{\theta}_{1}D_{i}P_{i}+\indora_{i=4}D_{i}\hat{F}_{1},
\end{eqnarray}

\begin{eqnarray*}
\begin{array}{ll}
D_{1}\hat{F}_{1}=D_{3}\hat{F}_{1}D\hat{\theta}_{1}D_{1}P_{1}=\hat{f}_{1}\left(3\right)\frac{1}{1-\hat{\mu}_{1}}\tilde{\mu}_{1},&
D_{2}\hat{F}_{1}=D_{3}\hat{F}_{1}D\hat{\theta}_{1}D_{2}P_{2}
=\hat{f}_{1}\left(3\right)\frac{1}{1-\hat{\mu}_{1}}\tilde{\mu}_{2}\\
D_{3}\hat{F}_{1}=0,&
D_{4}\hat{F}_{1}=D_{3}\hat{F}_{1}D\hat{\theta}_{1}D_{4}P_{4}
+D_{4}\hat{F}_{1}
=\hat{f}_{1}\left(3\right)\frac{1}{1-\hat{\mu}_{1}}\hat{\mu}_{2}+\hat{f}_{1}\left(2\right),

\end{array}
\end{eqnarray*}


\begin{eqnarray}
D_{i}\hat{F}_{2}&=&\indora_{i\neq4}D_{4}\hat{F}_{2}D\hat{\theta}_{2}D_{i}P_{i}+\indora_{i=3}D_{i}\hat{F}_{2}.
\end{eqnarray}

\begin{eqnarray*}
\begin{array}{ll}
D_{1}\hat{F}_{2}=D_{4}\hat{F}_{2}D\hat{\theta}_{2}D_{1}P_{1}
=\hat{f}_{2}\left(4\right)\frac{1}{1-\hat{\mu}_{2}}\tilde{\mu}_{1},&
D_{2}\hat{F}_{2}=D_{4}\hat{F}_{2}D\hat{\theta}_{2}D_{2}P_{2}
=\hat{f}_{2}\left(4\right)\frac{1}{1-\hat{\mu}_{2}}\tilde{\mu}_{2},\\
D_{3}\hat{F}_{2}=D_{4}\hat{F}_{2}D\hat{\theta}_{2}D_{3}P_{3}+D_{3}\hat{F}_{2}
=\hat{f}_{2}\left(4\right)\frac{1}{1-\hat{\mu}_{2}}\hat{\mu}_{1}+\hat{f}_{2}\left(4\right)\\
D_{4}\hat{F}_{2}=0

\end{array}
\end{eqnarray*}
Then, now we can obtain the linear system of equations in order to obtain the first moments of the lengths of the queues:



For $\mathbf{F}_{1}=R_{2}F_{2}\hat{F}_{2}$ we get the general equations

\begin{eqnarray}
D_{i}\mathbf{F}_{1}=D_{i}\left(R_{2}+F_{2}+\indora_{i\geq3}\hat{F}_{2}\right)
\end{eqnarray}

So

\begin{eqnarray*}
D_{1}\mathbf{F}_{1}&=&D_{1}R_{2}+D_{1}F_{2}
=r_{1}\tilde{\mu}_{1}+f_{2}\left(2\right)\frac{1}{1-\tilde{\mu}_{2}}\tilde{\mu}_{1}\\
D_{2}\mathbf{F}_{1}&=&D_{2}\left(R_{2}+F_{2}\right)
=r_{2}\tilde{\mu}_{1}\\
D_{3}\mathbf{F}_{1}&=&D_{3}\left(R_{2}+F_{2}+\hat{F}_{2}\right)
=r_{1}\hat{\mu}_{1}+f_{2}\left(2\right)\frac{1}{1-\tilde{\mu}_{2}}\hat{\mu}_{1}+\hat{F}_{1,2}^{(1)}\left(1\right)\\
D_{4}\mathbf{F}_{1}&=&D_{4}\left(R_{2}+F_{2}+\hat{F}_{2}\right)
=r_{2}\hat{\mu}_{2}+f_{2}\left(2\right)\frac{1}{1-\tilde{\mu}_{2}}\hat{\mu}_{2}
+\hat{F}_{2,2}^{(1)}\left(1\right)
\end{eqnarray*}

it means

\begin{eqnarray*}
\begin{array}{ll}
D_{1}\mathbf{F}_{1}=r_{2}\hat{\mu}_{1}+f_{2}\left(2\right)\left(\frac{1}{1-\tilde{\mu}_{2}}\right)\tilde{\mu}_{1}+f_{2}\left(1\right),&
D_{2}\mathbf{F}_{1}=r_{2}\tilde{\mu}_{2},\\
D_{3}\mathbf{F}_{1}=r_{2}\hat{\mu}_{1}+f_{2}\left(2\right)\left(\frac{1}{1-\tilde{\mu}_{2}}\right)\hat{\mu}_{1}+\hat{F}_{1,2}^{(1)}\left(1\right),&
D_{4}\mathbf{F}_{1}=r_{2}\hat{\mu}_{2}+f_{2}\left(2\right)\left(\frac{1}{1-\tilde{\mu}_{2}}\right)\hat{\mu}_{2}+\hat{F}_{2,2}^{(1)}\left(1\right),\end{array}
\end{eqnarray*}


\begin{eqnarray}
\begin{array}{ll}
\mathbf{F}_{2}=R_{1}F_{1}\hat{F}_{1}, & D_{i}\mathbf{F}_{2}=D_{i}\left(R_{1}+F_{1}+\indora_{i\geq3}\hat{F}_{1}\right)\\
\end{array}
\end{eqnarray}



equivalently


\begin{eqnarray*}
\begin{array}{ll}
D_{1}\mathbf{F}_{2}=r_{1}\tilde{\mu}_{1},&
D_{2}\mathbf{F}_{2}=r_{1}\tilde{\mu}_{2}+f_{1}\left(1\right)\left(\frac{1}{1-\tilde{\mu}_{1}}\right)\tilde{\mu}_{2}+f_{1}\left(2\right),\\
D_{3}\mathbf{F}_{2}=r_{1}\hat{\mu}_{1}+f_{1}\left(1\right)\left(\frac{1}{1-\tilde{\mu}_{1}}\right)\hat{\mu}_{1}+\hat{F}_{1,1}^{(1)}\left(1\right),&
D_{4}\mathbf{F}_{2}=r_{1}\hat{\mu}_{2}+f_{1}\left(1\right)\left(\frac{1}{1-\tilde{\mu}_{1}}\right)\hat{\mu}_{2}+\hat{F}_{2,1}^{(1)}\left(1\right),\\
\end{array}
\end{eqnarray*}



\begin{eqnarray}
\begin{array}{ll}
\hat{\mathbf{F}}_{1}=\hat{R}_{2}\hat{F}_{2}F_{2}, & D_{i}\hat{\mathbf{F}}_{1}=D_{i}\left(\hat{R}_{2}+\hat{F}_{2}+\indora_{i\leq2}F_{2}\right)\\
\end{array}
\end{eqnarray}


equivalently


\begin{eqnarray*}
\begin{array}{ll}
D_{1}\hat{\mathbf{F}}_{1}=\hat{r}_{2}\tilde{\mu}_{1}+\hat{f}_{2}\left(2\right)\left(\frac{1}{1-\hat{\mu}_{2}}\right)\tilde{\mu}_{1}+F_{1,2}^{(1)}\left(1\right),&
D_{2}\hat{\mathbf{F}}_{1}=\hat{r}_{2}\tilde{\mu}_{2}+\hat{f}_{2}\left(2\right)\left(\frac{1}{1-\hat{\mu}_{2}}\right)\tilde{\mu}_{2}+F_{2,2}^{(1)}\left(1\right),\\
D_{3}\hat{\mathbf{F}}_{1}=\hat{r}_{2}\hat{\mu}_{1}+\hat{f}_{2}\left(2\right)\left(\frac{1}{1-\hat{\mu}_{2}}\right)\hat{\mu}_{1}+\hat{f}_{2}\left(1\right),&
D_{4}\hat{\mathbf{F}}_{1}=\hat{r}_{2}\hat{\mu}_{2}
\end{array}
\end{eqnarray*}



\begin{eqnarray}
\begin{array}{ll}
\hat{\mathbf{F}}_{2}=\hat{R}_{1}\hat{F}_{1}F_{1}, & D_{i}\hat{\mathbf{F}}_{2}=D_{i}\left(\hat{R}_{1}+\hat{F}_{1}+\indora_{i\leq2}F_{1}\right)
\end{array}
\end{eqnarray}



equivalently


\begin{eqnarray*}
\begin{array}{ll}
D_{1}\hat{\mathbf{F}}_{2}=\hat{r}_{1}\tilde{\mu}_{1}+\hat{f}_{1}\left(1\right)\left(\frac{1}{1-\hat{\mu}_{1}}\right)\tilde{\mu}_{1}+F_{1,1}^{(1)}\left(1\right),&
D_{2}\hat{\mathbf{F}}_{2}=\hat{r}_{1}\mu_{2}+\hat{f}_{1}\left(1\right)\left(\frac{1}{1-\hat{\mu}_{1}}\right)\tilde{\mu}_{2}+F_{2,1}^{(1)}\left(1\right),\\
D_{3}\hat{\mathbf{F}}_{2}=\hat{r}_{1}\hat{\mu}_{1},&
D_{4}\hat{\mathbf{F}}_{2}=\hat{r}_{1}\hat{\mu}_{2}+\hat{f}_{1}\left(1\right)\left(\frac{1}{1-\hat{\mu}_{1}}\right)\hat{\mu}_{2}+\hat{f}_{1}\left(2\right),\\
\end{array}
\end{eqnarray*}





Then we have that if $\mu=\tilde{\mu}_{1}+\tilde{\mu}_{2}$, $\hat{\mu}=\hat{\mu}_{1}+\hat{\mu}_{2}$, $r=r_{1}+r_{2}$ and $\hat{r}=\hat{r}_{1}+\hat{r}_{2}$  the system's solution is given by

\begin{eqnarray*}
\begin{array}{llll}
f_{2}\left(1\right)=r_{1}\tilde{\mu}_{1},&
f_{1}\left(2\right)=r_{2}\tilde{\mu}_{2},&
\hat{f}_{1}\left(4\right)=\hat{r}_{2}\hat{\mu}_{2},&
\hat{f}_{2}\left(3\right)=\hat{r}_{1}\hat{\mu}_{1}
\end{array}
\end{eqnarray*}



it's easy to verify that

\begin{eqnarray}\label{Sist.Ec.Lineales.Doble.Traslado}
\begin{array}{ll}
f_{1}\left(1\right)=\tilde{\mu}_{1}\left(r+\frac{f_{2}\left(2\right)}{1-\tilde{\mu}_{2}}\right),& f_{1}\left(3\right)=\hat{\mu}_{1}\left(r_{2}+\frac{f_{2}\left(2\right)}{1-\tilde{\mu}_{2}}\right)+\hat{F}_{1,2}^{(1)}\left(1\right)\\
f_{1}\left(4\right)=\hat{\mu}_{2}\left(r_{2}+\frac{f_{2}\left(2\right)}{1-\tilde{\mu}_{2}}\right)+\hat{F}_{2,2}^{(1)}\left(1\right),&
f_{2}\left(2\right)=\left(r+\frac{f_{1}\left(1\right)}{1-\mu_{1}}\right)\tilde{\mu}_{2},\\
f_{2}\left(3\right)=\hat{\mu}_{1}\left(r_{1}+\frac{f_{1}\left(1\right)}{1-\tilde{\mu}_{1}}\right)+\hat{F}_{1,1}^{(1)}\left(1\right),&
f_{2}\left(4\right)=\hat{\mu}_{2}\left(r_{1}+\frac{f_{1}\left(1\right)}{1-\mu_{1}}\right)+\hat{F}_{2,1}^{(1)}\left(1\right),\\
\hat{f}_{1}\left(1\right)=\left(\hat{r}_{2}+\frac{\hat{f}_{2}\left(4\right)}{1-\hat{\mu}_{2}}\right)\tilde{\mu}_{1}+F_{1,2}^{(1)}\left(1\right),&
\hat{f}_{1}\left(2\right)=\left(\hat{r}_{2}+\frac{\hat{f}_{2}\left(4\right)}{1-\hat{\mu}_{2}}\right)\tilde{\mu}_{2}+F_{2,2}^{(1)}\left(1\right),\\
\hat{f}_{1}\left(3\right)=\left(\hat{r}+\frac{\hat{f}_{2}\left(4\right)}{1-\hat{\mu}_{2}}\right)\hat{\mu}_{1},&
\hat{f}_{2}\left(1\right)=\left(\hat{r}_{1}+\frac{\hat{f}_{1}\left(3\right)}{1-\hat{\mu}_{1}}\right)\mu_{1}+F_{1,1}^{(1)}\left(1\right),\\
\hat{f}_{2}\left(2\right)=\left(\hat{r}_{1}+\frac{\hat{f}_{1}\left(3\right)}{1-\hat{\mu}_{1}}\right)\tilde{\mu}_{2}+F_{2,1}^{(1)}\left(1\right),&
\hat{f}_{2}\left(4\right)=\left(\hat{r}+\frac{\hat{f}_{1}\left(3\right)}{1-\hat{\mu}_{1}}\right)\hat{\mu}_{2},\\
\end{array}
\end{eqnarray}

with system's solutions given by

\begin{eqnarray}
\begin{array}{ll}
f_{1}\left(1\right)=r\frac{\mu_{1}\left(1-\mu_{1}\right)}{1-\mu},&
f_{2}\left(2\right)=r\frac{\tilde{\mu}_{2}\left(1-\tilde{\mu}_{2}\right)}{1-\mu},\\
f_{1}\left(3\right)=\hat{\mu}_{1}\left(r_{2}+\frac{r\tilde{\mu}_{2}}{1-\mu}\right)+\hat{F}_{1,2}^{(1)}\left(1\right),&
f_{1}\left(4\right)=\hat{\mu}_{2}\left(r_{2}+\frac{r\tilde{\mu}_{2}}{1-\mu}\right)+\hat{F}_{2,2}^{(1)}\left(1\right),\\
f_{2}\left(3\right)=\hat{\mu}_{1}\left(r_{1}+\frac{r\mu_{1}}{1-\mu}\right)+\hat{F}_{1,1}^{(1)}\left(1\right),&
f_{2}\left(4\right)=\hat{\mu}_{2}\left(r_{1}+\frac{r\mu_{1}}{1-\mu}\right)+\hat{F}_{2,1}^{(1)}\left(1\right),\\
\hat{f}_{1}\left(1\right)=\tilde{\mu}_{1}\left(\hat{r}_{2}+\frac{\hat{r}\hat{\mu}_{2}}{1-\hat{\mu}}\right)+F_{1,2}^{(1)}\left(1\right),&
\hat{f}_{1}\left(2\right)=\tilde{\mu}_{2}\left(\hat{r}_{2}+\frac{\hat{r}\hat{\mu}_{2}}{1-\hat{\mu}}\right)+F_{2,2}^{(1)}\left(1\right),\\
\hat{f}_{2}\left(1\right)=\tilde{\mu}_{1}\left(\hat{r}_{1}+\frac{\hat{r}\hat{\mu}_{1}}{1-\hat{\mu}}\right)+F_{1,1}^{(1)}\left(1\right),&
\hat{f}_{2}\left(2\right)=\tilde{\mu}_{2}\left(\hat{r}_{1}+\frac{\hat{r}\hat{\mu}_{1}}{1-\hat{\mu}}\right)+F_{2,1}^{(1)}\left(1\right)
\end{array}
\end{eqnarray}

%_________________________________________________________________________________________________________
\subsection{General Second Order Derivatives}
%_________________________________________________________________________________________________________


Now, taking the second order derivative with respect to the equations given in (\ref{Sist.Ec.Lineales.Doble.Traslado}) we obtain it in their general form

\small{
\begin{eqnarray*}\label{Ec.Derivadas.Segundo.Orden.Doble.Transferencia}
D_{k}D_{i}F_{1}&=&D_{k}D_{i}\left(R_{2}+F_{2}+\indora_{i\geq3}\hat{F}_{4}\right)+D_{i}R_{2}D_{k}\left(F_{2}+\indora_{k\geq3}\hat{F}_{4}\right)+D_{i}F_{2}D_{k}\left(R_{2}+\indora_{k\geq3}\hat{F}_{4}\right)+\indora_{i\geq3}D_{i}\hat{F}_{4}D_{k}\left(R_{2}+F_{2}\right)\\
D_{k}D_{i}F_{2}&=&D_{k}D_{i}\left(R_{1}+F_{1}+\indora_{i\geq3}\hat{F}_{3}\right)+D_{i}R_{1}D_{k}\left(F_{1}+\indora_{k\geq3}\hat{F}_{3}\right)+D_{i}F_{1}D_{k}\left(R_{1}+\indora_{k\geq3}\hat{F}_{3}\right)+\indora_{i\geq3}D_{i}\hat{F}_{3}D_{k}\left(R_{1}+F_{1}\right)\\
D_{k}D_{i}\hat{F}_{3}&=&D_{k}D_{i}\left(\hat{R}_{4}+\indora_{i\leq2}F_{2}+\hat{F}_{4}\right)+D_{i}\hat{R}_{4}D_{k}\left(\indora_{k\leq2}F_{2}+\hat{F}_{4}\right)+D_{i}\hat{F}_{4}D_{k}\left(\hat{R}_{4}+\indora_{k\leq2}F_{2}\right)+\indora_{i\leq2}D_{i}F_{2}D_{k}\left(\hat{R}_{4}+\hat{F}_{4}\right)\\
D_{k}D_{i}\hat{F}_{4}&=&D_{k}D_{i}\left(\hat{R}_{3}+\indora_{i\leq2}F_{1}+\hat{F}_{3}\right)+D_{i}\hat{R}_{3}D_{k}\left(\indora_{k\leq2}F_{1}+\hat{F}_{3}\right)+D_{i}\hat{F}_{3}D_{k}\left(\hat{R}_{3}+\indora_{k\leq2}F_{1}\right)+\indora_{i\leq2}D_{i}F_{1}D_{k}\left(\hat{R}_{3}+\hat{F}_{3}\right)
\end{eqnarray*}}
for $i,k=1,\ldots,4$. In order to have it in an specific way we need to compute the expressions $D_{k}D_{i}\left(R_{2}+F_{2}+\indora_{i\geq3}\hat{F}_{4}\right)$

%_________________________________________________________________________________________________________
\subsubsection{Second Order Derivatives: Serve's Switchover Times}
%_________________________________________________________________________________________________________

Remind $R_{i}\left(z_{1},z_{2},w_{1},w_{2}\right)=R_{i}\left(P_{1}\left(z_{1}\right)\tilde{P}_{2}\left(z_{2}\right)
\hat{P}_{1}\left(w_{1}\right)\hat{P}_{2}\left(w_{2}\right)\right)$,  which we will write in his reduced form $R_{i}=R_{i}\left(
P_{1}\tilde{P}_{2}\hat{P}_{1}\hat{P}_{2}\right)$, and according to the notation given in \cite{Lang} we obtain

\begin{eqnarray}
D_{i}D_{i}R_{k}=D^{2}R_{k}\left(D_{i}P_{i}\right)^{2}+DR_{k}D_{i}D_{i}P_{i}
\end{eqnarray}

whereas for $i\neq j$

\begin{eqnarray}
D_{i}D_{j}R_{k}=D^{2}R_{k}D_{i}P_{i}D_{j}P_{j}+DR_{k}D_{j}P_{j}D_{i}P_{i}
\end{eqnarray}

%_________________________________________________________________________________________________________
\subsubsection{Second Order Derivatives: Queue Lengths}
%_________________________________________________________________________________________________________

Just like before the expression $F_{1}\left(\tilde{\theta}_{1}\left(\tilde{P}_{2}\left(z_{2}\right)\hat{P}_{1}\left(w_{1}\right)\hat{P}_{2}\left(w_{2}\right)\right),
z_{2}\right)$, will be denoted by $F_{1}\left(\tilde{\theta}_{1}\left(\tilde{P}_{2}\hat{P}_{1}\hat{P}_{2}\right),z_{2}\right)$, then the mixed partial derivatives are:

\begin{eqnarray*}
D_{j}D_{i}F_{1}&=&\indora_{i,j\neq1}D_{1}D_{1}F_{1}\left(D\tilde{\theta}_{1}\right)^{2}D_{i}P_{i}D_{j}P_{j}
+\indora_{i,j\neq1}D_{1}F_{1}D^{2}\tilde{\theta}_{1}D_{i}P_{i}D_{j}P_{j}
+\indora_{i,j\neq1}D_{1}F_{1}D\tilde{\theta}_{1}\left(\indora_{i=j}D_{i}^{2}P_{i}+\indora_{i\neq j}D_{i}P_{i}D_{j}P_{j}\right)\\
&+&\left(1-\indora_{i=j=3}\right)\indora_{i+j\leq6}D_{1}D_{2}F_{1}D\tilde{\theta}_{1}\left(\indora_{i\leq j}D_{j}P_{j}+\indora_{i>j}D_{i}P_{i}\right)
+\indora_{i=2}\left(D_{1}D_{2}F_{1}D\tilde{\theta}_{1}D_{i}P_{i}+D_{i}^{2}F_{1}\right)
\end{eqnarray*}


Recall the expression for $F_{1}\left(\tilde{\theta}_{1}\left(\tilde{P}_{2}\left(z_{2}\right)\hat{P}_{1}\left(w_{1}\right)\hat{P}_{2}\left(w_{2}\right)\right),
z_{2}\right)$, which is denoted by $F_{1}\left(\tilde{\theta}_{1}\left(\tilde{P}_{2}\hat{P}_{1}\hat{P}_{2}\right),z_{2}\right)$, then the mixed partial derivatives are given by

\begin{eqnarray*}
\begin{array}{llll}
D_{1}D_{1}F_{1}=0,&
D_{2}D_{1}F_{1}=0,&
D_{3}D_{1}F_{1}=0,&
D_{4}D_{1}F_{1}=0,\\
D_{1}D_{2}F_{1}=0,&
D_{1}D_{3}F_{1}=0,&
D_{1}D_{4}F_{1}=0,&
\end{array}
\end{eqnarray*}

\begin{eqnarray*}
D_{2}D_{2}F_{1}&=&D_{1}^{2}F_{1}\left(D\tilde{\theta}_{1}\right)^{2}\left(D_{2}\tilde{P}_{2}\right)^{2}
+D_{1}F_{1}D^{2}\tilde{\theta}_{1}\left(D_{2}\tilde{P}_{2}\right)^{2}
+D_{1}F_{1}D\tilde{\theta}_{1}D_{2}^{2}\tilde{P}_{2}
+D_{1}D_{2}F_{1}D\tilde{\theta}_{1}D_{2}\tilde{P}_{2}\\
&+&D_{1}D_{2}F_{1}D\tilde{\theta}_{1}D_{2}\tilde{P}_{2}+D_{2}D_{2}F_{1}\\
&=&f_{1}\left(1,1\right)\left(\frac{\tilde{\mu}_{2}}{1-\tilde{\mu}_{1}}\right)^{2}
+f_{1}\left(1\right)\tilde{\theta}_{1}^(2)\tilde{\mu}_{2}^{(2)}
+f_{1}\left(1\right)\frac{1}{1-\tilde{\mu}_{1}}\tilde{P}_{2}^{(2)}+f_{1}\left(1,2\right)\frac{\tilde{\mu}_{2}}{1-\tilde{\mu}_{1}}+f_{1}\left(1,2\right)\frac{\tilde{\mu}_{2}}{1-\tilde{\mu}_{1}}+f_{1}\left(2,2\right)
\end{eqnarray*}

\begin{eqnarray*}
D_{3}D_{2}F_{1}&=&D_{1}^{2}F_{1}\left(D\tilde{\theta}_{1}\right)^{2}D_{3}\hat{P}_{1}D_{2}\tilde{P}_{2}+D_{1}F_{1}D^{2}\tilde{\theta}_{1}D_{3}\hat{P}_{1}D_{2}\tilde{P}_{2}+D_{1}F_{1}D\tilde{\theta}_{1}D_{2}\tilde{P}_{2}D_{3}\hat{P}_{1}+D_{1}D_{2}F_{1}D\tilde{\theta}_{1}D_{3}\hat{P}_{1}\\
&=&f_{1}\left(1,1\right)\left(\frac{1}{1-\tilde{\mu}_{1}}\right)^{2}\tilde{\mu}_{2}\hat{\mu}_{1}+f_{1}\left(1\right)\tilde{\theta}_{1}^{(2)}\tilde{\mu}_{2}\hat{\mu}_{1}+f_{1}\left(1\right)\frac{\tilde{\mu}_{2}\hat{\mu}_{1}}{1-\tilde{\mu}_{1}}+f_{1}\left(1,2\right)\frac{\hat{\mu}_{1}}{1-\tilde{\mu}_{1}}
\end{eqnarray*}

\begin{eqnarray*}
D_{4}D_{2}F_{1}&=&D_{1}^{2}F_{1}\left(D\tilde{\theta}_{1}\right)^{2}D_{4}\hat{P}_{2}D_{2}\tilde{P}_{2}+D_{1}F_{1}D^{2}\tilde{\theta}_{1}D_{4}\hat{P}_{2}D_{2}\tilde{P}_{2}+D_{1}F_{1}D\tilde{\theta}_{1}D_{2}\tilde{P}_{2}D_{4}\hat{P}_{2}+D_{1}D_{2}F_{1}D\tilde{\theta}_{1}D_{4}\hat{P}_{2}\\
&=&f_{1}\left(1,1\right)\left(\frac{1}{1-\tilde{\mu}_{1}}\right)^{2}\tilde{\mu}_{2}\hat{\mu}_{2}+f_{1}\left(1\right)\tilde{\theta}_{1}^{(2)}\tilde{\mu}_{2}\hat{\mu}_{2}+f_{1}\left(1\right)\frac{\tilde{\mu}_{2}\hat{\mu}_{2}}{1-\tilde{\mu}_{1}}+f_{1}\left(1,2\right)\frac{\hat{\mu}_{2}}{1-\tilde{\mu}_{1}}
\end{eqnarray*}

\begin{eqnarray*}
D_{2}D_{3}F_{1}&=&
D_{1}^{2}F_{1}\left(D\tilde{\theta}_{1}\right)^{2}D_{2}\tilde{P}_{2}D_{3}\hat{P}_{1}
+D_{1}F_{1}D^{2}\tilde{\theta}_{1}D_{2}\tilde{P}_{2}D_{3}\hat{P}_{1}+
D_{1}F_{1}D\tilde{\theta}_{1}D_{3}\hat{P}_{1}D_{2}\tilde{P}_{2}
+D_{1}D_{2}F_{1}D\tilde{\theta}_{1}D_{3}\hat{P}_{1}\\
&=&f_{1}\left(1,1\right)\left(\frac{1}{1-\tilde{\mu}_{1}}\right)^{2}\tilde{\mu}_{2}\hat{\mu}_{1}+f_{1}\left(1\right)\tilde{\theta}_{1}^{(2)}\tilde{\mu}_{2}\hat{\mu}_{1}+f_{1}\left(1\right)\frac{\tilde{\mu}_{2}\hat{\mu}_{1}}{1-\tilde{\mu}_{1}}+f_{1}\left(1,2\right)\frac{\hat{\mu}_{1}}{1-\tilde{\mu}_{1}}
\end{eqnarray*}

\begin{eqnarray*}
D_{3}D_{3}F_{1}&=&D_{1}^{2}F_{1}\left(D\tilde{\theta}_{1}\right)^{2}\left(D_{3}\hat{P}_{1}\right)^{2}+D_{1}F_{1}D^{2}\tilde{\theta}_{1}\left(D_{3}\hat{P}_{1}\right)^{2}+D_{1}F_{1}D\tilde{\theta}_{1}D_{3}^{2}\hat{P}_{1}\\
&=&f_{1}\left(1,1\right)\left(\frac{\hat{\mu}_{1}}{1-\tilde{\mu}_{1}}\right)^{2}+f_{1}\left(1\right)\tilde{\theta}_{1}^{(2)}\hat{\mu}_{1}^{2}+f_{1}\left(1\right)\frac{\hat{\mu}_{1}^{2}}{1-\tilde{\mu}_{1}}
\end{eqnarray*}

\begin{eqnarray*}
D_{4}D_{3}F_{1}&=&D_{1}^{2}F_{1}\left(D\tilde{\theta}_{1}\right)^{2}D_{4}\hat{P}_{2}D_{3}\hat{P}_{1}+D_{1}F_{1}D^{2}\tilde{\theta}_{1}D_{4}\hat{P}_{2}D_{3}\hat{P}_{1}+D_{1}F_{1}D\tilde{\theta}_{1}D_{3}\hat{P}_{1}D_{4}\hat{P}_{2}\\
&=&f_{1}\left(1,1\right)\left(\frac{1}{1-\tilde{\mu}_{1}}\right)^{2}\hat{\mu}_{1}\hat{\mu}_{2}
+f_{1}\left(1\right)\tilde{\theta}_{1}^{2}\hat{\mu}_{2}\hat{\mu}_{1}
+f_{1}\left(1\right)\frac{\hat{\mu}_{2}\hat{\mu}_{1}}{1-\tilde{\mu}_{1}}
\end{eqnarray*}

\begin{eqnarray*}
D_{2}D_{4}F_{1}&=&D_{1}^{2}F_{1}\left(D\tilde{\theta}_{1}\right)^{2}D_{2}\tilde{P}_{2}D_{4}\hat{P}_{2}+D_{1}F_{1}D^{2}\tilde{\theta}_{1}D_{2}\tilde{P}_{2}D_{4}\hat{P}_{2}+D_{1}F_{1}D\tilde{\theta}_{1}D_{4}\hat{P}_{2}D_{2}\tilde{P}_{2}+D_{1}D_{2}F_{1}D\tilde{\theta}_{1}D_{4}\hat{P}_{2}\\
&=&f_{1}\left(1,1\right)\left(\frac{1}{1-\tilde{\mu}_{1}}\right)^{2}\hat{\mu}_{2}\tilde{\mu}_{2}
+f_{1}\left(1\right)\tilde{\theta}_{1}^{(2)}\hat{\mu}_{2}\tilde{\mu}_{2}
+f_{1}\left(1\right)\frac{\hat{\mu}_{2}\tilde{\mu}_{2}}{1-\tilde{\mu}_{1}}+f_{1}\left(1,2\right)\frac{\hat{\mu}_{2}}{1-\tilde{\mu}_{1}}
\end{eqnarray*}

\begin{eqnarray*}
D_{3}D_{4}F_{1}&=&D_{1}^{2}F_{1}\left(D\tilde{\theta}_{1}\right)^{2}D_{3}\hat{P}_{1}D_{4}\hat{P}_{2}+D_{1}F_{1}D^{2}\tilde{\theta}_{1}D_{3}\hat{P}_{1}D_{4}\hat{P}_{2}+D_{1}F_{1}D\tilde{\theta}_{1}D_{4}\hat{P}_{2}D_{3}\hat{P}_{1}\\
&=&f_{1}\left(1,1\right)\left(\frac{1}{1-\tilde{\mu}_{1}}\right)^{2}\hat{\mu}_{1}\hat{\mu}_{2}+f_{1}\left(1\right)\tilde{\theta}_{1}^{(2)}\hat{\mu}_{1}\hat{\mu}_{2}+f_{1}\left(1\right)\frac{\hat{\mu}_{1}\hat{\mu}_{2}}{1-\tilde{\mu}_{1}}
\end{eqnarray*}

\begin{eqnarray*}
D_{4}D_{4}F_{1}&=&D_{1}^{2}F_{1}\left(D\tilde{\theta}_{1}\right)^{2}\left(D_{4}\hat{P}_{2}\right)^{2}+D_{1}F_{1}D^{2}\tilde{\theta}_{1}\left(D_{4}\hat{P}_{2}\right)^{2}+D_{1}F_{1}D\tilde{\theta}_{1}D_{4}^{2}\hat{P}_{2}\\
&=&f_{1}\left(1,1\right)\left(\frac{\hat{\mu}_{2}}{1-\tilde{\mu}_{1}}\right)^{2}+f_{1}\left(1\right)\tilde{\theta}_{1}^{(2)}\hat{\mu}_{2}^{2}+f_{1}\left(1\right)\frac{1}{1-\tilde{\mu}_{1}}\hat{P}_{2}^{(2)}
\end{eqnarray*}



Meanwhile for  $F_{2}\left(z_{1},\tilde{\theta}_{2}\left(P_{1}\hat{P}_{1}\hat{P}_{2}\right)\right)$

\begin{eqnarray*}
D_{j}D_{i}F_{2}&=&\indora_{i,j\neq2}D_{2}D_{2}F_{2}\left(D\theta_{2}\right)^{2}D_{i}P_{i}D_{j}P_{j}+\indora_{i,j\neq2}D_{2}F_{2}D^{2}\theta_{2}D_{i}P_{i}D_{j}P_{j}\\
&+&\indora_{i,j\neq2}D_{2}F_{2}D\theta_{2}\left(\indora_{i=j}D_{i}^{2}P_{i}
+\indora_{i\neq j}D_{i}P_{i}D_{j}P_{j}\right)\\
&+&\left(1-\indora_{i=j=3}\right)\indora_{i+j\leq6}D_{2}D_{1}F_{2}D\theta_{2}\left(\indora_{i\leq j}D_{j}P_{j}+\indora_{i>j}D_{i}P_{i}\right)
+\indora_{i=1}\left(D_{2}D_{1}F_{2}D\theta_{2}D_{i}P_{i}+D_{i}^{2}F_{2}\right)
\end{eqnarray*}

\begin{eqnarray*}
\begin{array}{llll}
D_{2}D_{1}F_{2}=0,&
D_{2}D_{3}F_{3}=0,&
D_{2}D_{4}F_{2}=0,&\\
D_{1}D_{2}F_{2}=0,&
D_{2}D_{2}F_{2}=0,&
D_{3}D_{2}F_{2}=0,&
D_{4}D_{2}F_{2}=0\\
\end{array}
\end{eqnarray*}


\begin{eqnarray*}
D_{1}D_{1}F_{2}&=&
\left(D_{1}P_{1}\right)^{2}\left(D\tilde{\theta}_{2}\right)^{2}D_{2}^{2}F_{2}
+\left(D_{1}P_{1}\right)^{2}D^{2}\tilde{\theta}_{2}D_{2}F_{2}
+D_{1}^{2}P_{1}D\tilde{\theta}_{2}D_{2}F_{2}
+D_{1}P_{1}D\tilde{\theta}_{2}D_{2}D_{1}F_{2}\\
&+&D_{2}D_{1}F_{2}D\tilde{\theta}_{2}D_{1}P_{1}+
D_{1}^{2}F_{2}\\
&=&f_{2}\left(2\right)\frac{\tilde{P}_{1}^{(2)}}{1-\tilde{\mu}_{2}}
+f_{2}\left(2\right)\theta_{2}^{(2)}\tilde{\mu}_{1}^{2}
+f_{2}\left(2,1\right)\frac{\tilde{\mu}_{1}}{1-\tilde{\mu}_{2}}
+\left(\frac{\tilde{\mu}_{1}}{1-\tilde{\mu}_{2}}\right)^{2}f_{2}\left(2,2\right)
+\frac{\tilde{\mu}_{1}}{1-\tilde{\mu}_{2}}f_{2}\left(2,1\right)+f_{2}\left(1,1\right)
\end{eqnarray*}


\begin{eqnarray*}
D_{3}D_{1}F_{2}&=&D_{2}D_{1}F_{2}D\tilde{\theta}_{2}D_{3}\hat{P}_{1}
+D_{2}^{2}F_{2}\left(D\tilde{\theta}_{2}\right)^{2}D_{3}P_{1}D_{1}P_{1}
+D_{2}F_{2}D^{2}\tilde{\theta}_{2}D_{3}\hat{P}_{1}D_{1}P_{1}
+D_{2}F_{2}D\tilde{\theta}_{2}D_{1}P_{1}D_{3}\hat{P}_{1}\\
&=&f_{2}\left(2,1\right)\frac{\hat{\mu}_{1}}{1-\tilde{\mu}_{2}}
+f_{2}\left(2,2\right)\left(\frac{1}{1-\tilde{\mu}_{2}}\right)^{2}\tilde{\mu}_{1}\hat{\mu}_{1}
+f_{2}\left(2\right)\tilde{\theta}_{2}^{(2)}\tilde{\mu}_{1}\hat{\mu}_{1}
+f_{2}\left(2\right)\frac{\tilde{\mu}_{1}\hat{\mu}_{1}}{1-\tilde{\mu}_{2}}
\end{eqnarray*}


\begin{eqnarray*}
D_{4}D_{1}F_{2}&=&D_{2}^{2}F_{2}\left(D\tilde{\theta}_{2}\right)^{2}D_{4}P_{2}D_{1}P_{1}+D_{2}F_{2}D^{2}\tilde{\theta}_{2}D_{4}\hat{P}_{2}D_{1}P_{1}
+D_{2}F_{2}D\tilde{\theta}_{2}D_{1}P_{1}D_{4}\hat{P}_{2}+D_{2}D_{1}F_{2}D\tilde{\theta}_{2}D_{4}\hat{P}_{2}\\
&=&f_{2}\left(2,2\right)\left(\frac{1}{1-\tilde{\mu}_{2}}\right)^{2}\tilde{\mu}_{1}\hat{\mu}_{2}
+f_{2}\left(2\right)\tilde{\theta}_{2}^{(2)}\tilde{\mu}_{1}\hat{\mu}_{2}
+f_{2}\left(2\right)\frac{\tilde{\mu}_{1}\hat{\mu}_{2}}{1-\tilde{\mu}_{2}}
+f_{2}\left(2,1\right)\frac{\hat{\mu}_{2}}{1-\tilde{\mu}_{2}}
\end{eqnarray*}


\begin{eqnarray*}
D_{1}D_{3}F_{2}&=&D_{2}^{2}F_{2}\left(D\tilde{\theta}_{2}\right)^{2}D_{1}P_{1}D_{3}\hat{P}_{1}
+D_{2}F_{2}D^{2}\tilde{\theta}_{2}D_{1}P_{1}D_{3}\hat{P}_{1}
+D_{2}F_{2}D\tilde{\theta}_{2}D_{3}\hat{P}_{1}D_{1}P_{1}
+D_{2}D_{1}F_{2}D\tilde{\theta}_{2}D_{3}\hat{P}_{1}\\
&=&f_{2}\left(2,2\right)\left(\frac{1}{1-\tilde{\mu}_{2}}\right)^{2}\tilde{\mu}_{1}\hat{\mu}_{1}
+f_{2}\left(2\right)\tilde{\theta}_{2}^{(2)}\tilde{\mu}_{1}\hat{\mu}_{1}
+f_{2}\left(2\right)\frac{\tilde{\mu}_{1}\hat{\mu}_{1}}{1-\tilde{\mu}_{2}}
+f_{2}\left(2,1\right)\frac{\hat{\mu}_{1}}{1-\tilde{\mu}_{2}}
\end{eqnarray*}


\begin{eqnarray*}
D_{3}D_{3}F_{2}&=&D_{2}^{2}F_{2}\left(D\tilde{\theta}_{2}\right)^{2}\left(D_{3}\hat{P}_{1}\right)^{2}
+D_{2}F_{2}\left(D_{3}\hat{P}_{1}\right)^{2}D^{2}\tilde{\theta}_{2}
+D_{2}F_{2}D\tilde{\theta}_{2}D_{3}^{2}\hat{P}_{1}\\
&=&f_{2}\left(2,2\right)\left(\frac{1}{1-\tilde{\mu}_{2}}\right)^{2}\hat{\mu}_{1}^{2}
+f_{2}\left(2\right)\tilde{\theta}_{2}^{(2)}\hat{\mu}_{1}^{2}
+f_{2}\left(2\right)\frac{\hat{P}_{1}^{(2)}}{1-\tilde{\mu}_{2}}
\end{eqnarray*}


\begin{eqnarray*}
D_{4}D_{3}F_{2}&=&D_{2}^{2}F_{2}\left(D\tilde{\theta}_{2}\right)^{2}D_{4}\hat{P}_{2}D_{3}\hat{P}_{1}
+D_{2}F_{2}D^{2}\tilde{\theta}_{2}D_{4}\hat{P}_{2}D_{3}\hat{P}_{1}
+D_{2}F_{2}D\tilde{\theta}_{2}D_{3}\hat{P}_{1}D_{4}\hat{P}_{2}\\
&=&f_{2}\left(2,2\right)\left(\frac{1}{1-\tilde{\mu}_{2}}\right)^{2}\hat{\mu}_{1}\hat{\mu}_{2}
+f_{2}\left(2\right)\tilde{\theta}_{2}^{(2)}\hat{\mu}_{1}\hat{\mu}_{2}
+f_{2}\left(2\right)\frac{\hat{\mu}_{1}\hat{\mu}_{2}}{1-\tilde{\mu}_{2}}
\end{eqnarray*}


\begin{eqnarray*}
D_{1}D_{4}F_{2}&=&D_{2}^{2}F_{2}\left(D\tilde{\theta}_{2}\right)^{2}D_{1}P_{1}D_{4}\hat{P}_{2}
+D_{2}F_{2}D^{2}\tilde{\theta}_{2}D_{1}P_{1}D_{4}\hat{P}_{2}
+D_{2}F_{2}D\tilde{\theta}_{2}D_{4}\hat{P}_{2}D_{1}P_{1}
+D_{2}D_{1}F_{2}D\tilde{\theta}_{2}D_{4}\hat{P}_{2}\\
&=&f_{2}\left(2,2\right)\left(\frac{1}{1-\tilde{\mu}_{2}}\right)^{2}\tilde{\mu}_{1}\hat{\mu}_{2}
+f_{2}\left(2\right)\tilde{\theta}_{2}^{(2)}\tilde{\mu}_{1}\hat{\mu}_{2}
+f_{2}\left(2\right)\frac{\tilde{\mu}_{1}\hat{\mu}_{2}}{1-\tilde{\mu}_{2}}
+f_{2}\left(2,1\right)\frac{\hat{\mu}_{2}}{1-\tilde{\mu}_{2}}
\end{eqnarray*}


\begin{eqnarray*}
D_{3}D_{4}F_{2}&=&
D_{2}^{2}F_{2}\left(D\tilde{\theta}_{2}\right)^{2}D_{4}\hat{P}_{2}D_{3}\hat{P}_{1}
+D_{2}F_{2}D^{2}\tilde{\theta}_{2}D_{4}\hat{P}_{2}D_{3}\hat{P}_{1}
+D_{2}F_{2}D\tilde{\theta}_{2}D_{4}\hat{P}_{2}D_{3}\hat{P}_{1}\\
&=&f_{2}\left(2,2\right)\left(\frac{1}{1-\tilde{\mu}_{2}}\right)^{2}\hat{\mu}_{1}\hat{\mu}_{2}
+f_{2}\left(2\right)\tilde{\theta}_{2}^{(2)}\hat{\mu}_{1}\hat{\mu}_{2}
+f_{2}\left(2\right)\frac{\hat{\mu}_{1}\hat{\mu}_{2}}{1-\tilde{\mu}_{2}}
\end{eqnarray*}


\begin{eqnarray*}
D_{4}D_{4}F_{2}&=&D_{2}F_{2}D\tilde{\theta}_{2}D_{4}^{2}\hat{P}_{2}
+D_{2}F_{2}D^{2}\tilde{\theta}_{2}\left(D_{4}\hat{P}_{2}\right)^{2}
+D_{2}^{2}F_{2}\left(D\tilde{\theta}_{2}\right)^{2}\left(D_{4}\hat{P}_{2}\right)^{2}\\
&=&f_{2}\left(2,2\right)\left(\frac{\hat{\mu}_{2}}{1-\tilde{\mu}_{2}}\right)^{2}
+f_{2}\left(2\right)\tilde{\theta}_{2}^{(2)}\hat{\mu}_{2}^{2}
+f_{2}\left(2\right)\frac{\hat{P}_{2}^{(2)}}{1-\tilde{\mu}_{2}}
\end{eqnarray*}


%\newpage



%\newpage

For $\hat{F}_{1}\left(\hat{\theta}_{1}\left(P_{1}\tilde{P}_{2}\hat{P}_{2}\right),w_{2}\right)$



\begin{eqnarray*}
D_{j}D_{i}\hat{F}_{1}&=&\indora_{i,j\neq3}D_{3}D_{3}\hat{F}_{1}\left(D\hat{\theta}_{1}\right)^{2}D_{i}P_{i}D_{j}P_{j}
+\indora_{i,j\neq3}D_{3}\hat{F}_{1}D^{2}\hat{\theta}_{1}D_{i}P_{i}D_{j}P_{j}
+\indora_{i,j\neq3}D_{3}\hat{F}_{1}D\hat{\theta}_{1}\left(\indora_{i=j}D_{i}^{2}P_{i}+\indora_{i\neq j}D_{i}P_{i}D_{j}P_{j}\right)\\
&+&\indora_{i+j\geq5}D_{3}D_{4}\hat{F}_{1}D\hat{\theta}_{1}\left(\indora_{i\leq j}D_{i}P_{i}+\indora_{i>j}D_{j}P_{j}\right)
+\indora_{i=4}\left(D_{3}D_{4}\hat{F}_{1}D\hat{\theta}_{1}D_{i}P_{i}+D_{i}^{2}\hat{F}_{1}\right)
\end{eqnarray*}


\begin{eqnarray*}
\begin{array}{llll}
D_{3}D_{1}\hat{F}_{1}=0,&
D_{3}D_{2}\hat{F}_{1}=0,&
D_{1}D_{3}\hat{F}_{1}=0,&
D_{2}D_{3}\hat{F}_{1}=0\\
D_{3}D_{3}\hat{F}_{1}=0,&
D_{4}D_{3}\hat{F}_{1}=0,&
D_{3}D_{4}\hat{F}_{1}=0,&
\end{array}
\end{eqnarray*}


\begin{eqnarray*}
D_{1}D_{1}\hat{F}_{1}&=&
D_{3}^{2}\hat{F}_{1}\left(D\hat{\theta}_{1}\right)^{2}\left(D_{1}P_{1}\right)^{2}
+D_{3}\hat{F}_{1}D^{2}\hat{\theta}_{1}\left(D_{1}P_{1}\right)^{2}
+D_{3}\hat{F}_{1}D\hat{\theta}_{1}D_{1}^{2}P_{1}\\
&=&\hat{f}_{1}\left(3,3\right)\left(\frac{\tilde{\mu}_{1}}{1-\hat{\mu}_{2}}\right)^{2}
+\hat{f}_{1}\left(3\right)\frac{P_{1}^{(2)}}{1-\hat{\mu}_{1}}
+\hat{f}_{1}\left(3\right)\hat{\theta}_{1}^{(2)}\tilde{\mu}_{1}^{2}
\end{eqnarray*}


\begin{eqnarray*}
D_{2}D_{1}\hat{F}_{1}&=&
D_{3}^{2}\hat{F}_{1}\left(D\hat{\theta}_{1}\right)^{2}D_{1}P_{1}D_{2}P_{1}+
D_{3}\hat{F}_{1}D^{2}\hat{\theta}_{1}D_{1}P_{1}D_{2}P_{2}+
D_{3}\hat{F}_{1}D\hat{\theta}_{1}D_{1}P_{1}D_{2}P_{2}\\
&=&\hat{f}_{1}\left(3,3\right)\left(\frac{1}{1-\hat{\mu}_{1}}\right)^{2}\tilde{\mu}_{1}\tilde{\mu}_{2}
+\hat{f}_{1}\left(3\right)\tilde{\mu}_{1}\tilde{\mu}_{2}\hat{\theta}_{1}^{(2)}
+\hat{f}_{1}\left(3\right)\frac{\tilde{\mu}_{1}\tilde{\mu}_{2}}{1-\hat{\mu}_{1}}
\end{eqnarray*}


\begin{eqnarray*}
D_{4}D_{1}\hat{F}_{1}&=&
D_{3}D_{3}\hat{F}_{1}\left(D\hat{\theta}_{1}\right)^{2}D_{4}\hat{P}_{2}D_{1}P_{1}
+D_{3}\hat{F}_{1}D^{2}\hat{\theta}_{1}D_{1}P_{1}D_{4}\hat{P}_{2}
+D_{3}\hat{F}_{1}D\hat{\theta}_{1}D_{1}P_{1}D_{4}\hat{P}_{2}
+D_{3}D_{4}\hat{F}_{1}D\hat{\theta}_{1}D_{1}P_{1}\\
&=&\hat{f}_{1}\left(3,3\right)\left(\frac{1}{1-\hat{\mu}_{1}}\right)^{2}\tilde{\mu}_{1}\hat{\mu}_{1}
+\hat{f}_{1}\left(3\right)\hat{\theta}_{1}^{(2)}\tilde{\mu}_{1}\hat{\mu}_{2}
+\hat{f}_{1}\left(3\right)\frac{\tilde{\mu}_{1}\hat{\mu}_{2}}{1-\hat{\mu}_{1}}
+\hat{f}_{1}\left(3,4\right)\frac{\tilde{\mu}_{1}}{1-\hat{\mu}_{1}}
\end{eqnarray*}


\begin{eqnarray*}
D_{1}D_{2}\hat{F}_{1}&=&
D_{3}^{2}\hat{F}_{1}\left(D\hat{\theta}_{1}\right)^{2}D_{1}P_{1}D_{2}P_{2}
+D_{3}\hat{F}_{1}D^{2}\hat{\theta}_{1}D_{1}P_{1}D_{2}P_{2}+
D_{3}\hat{F}_{1}D\hat{\theta}_{1}D_{1}P_{1}D_{2}P_{2}\\
&=&\hat{f}_{1}\left(3,3\right)\left(\frac{1}{1-\hat{\mu}_{1}}\right)^{2}\tilde{\mu}_{1}\tilde{\mu}_{2}
+\hat{f}_{1}\left(3\right)\hat{\theta}_{1}^{(2)}\tilde{\mu}_{1}\tilde{\mu}_{2}
+\hat{f}_{1}\left(3\right)\frac{\tilde{\mu}_{1}\tilde{\mu}_{2}}{1-\hat{\mu}_{1}}
\end{eqnarray*}


\begin{eqnarray*}
D_{2}D_{2}\hat{F}_{1}&=&
D_{3}^{2}\hat{F}_{1}\left(D\hat{\theta}_{1}\right)^{2}\left(D_{2}P_{2}\right)^{2}
+D_{3}\hat{F}_{1}D^{2}\hat{\theta}_{1}\left(D_{2}P_{2}\right)^{2}+
D_{3}\hat{F}_{1}D\hat{\theta}_{1}D_{2}^{2}P_{2}\\
&=&\hat{f}_{1}\left(3,3\right)\left(\frac{\tilde{\mu}_{2}}{1-\hat{\mu}_{1}}\right)^{2}
+\hat{f}_{1}\left(3\right)\hat{\theta}_{1}^{(2)}\tilde{\mu}_{2}^{2}
+\hat{f}_{1}\left(3\right)\tilde{P}_{2}^{(2)}\frac{1}{1-\hat{\mu}_{1}}
\end{eqnarray*}


\begin{eqnarray*}
D_{4}D_{2}\hat{F}_{1}&=&
D_{3}^{2}\hat{F}_{1}\left(D\hat{\theta}_{1}\right)^{2}D_{4}\hat{P}_{2}D_{2}P_{2}
+D_{3}\hat{F}_{1}D^{2}\hat{\theta}_{1}D_{2}P_{2}D_{4}\hat{P}_{2}
+D_{3}\hat{F}_{1}D\hat{\theta}_{1}D_{2}P_{2}D_{4}\hat{P}_{2}
+D_{3}D_{4}\hat{F}_{1}D\hat{\theta}_{1}D_{2}P_{2}\\
&=&\hat{f}_{1}\left(3,3\right)\left(\frac{1}{1-\hat{\mu}_{1}}\right)^{2}\tilde{\mu}_{2}\hat{\mu}_{2}
+\hat{f}_{1}\left(3\right)\hat{\theta}_{1}^{(2)}\tilde{\mu}_{2}\hat{\mu}_{2}
+\hat{f}_{1}\left(3\right)\frac{\tilde{\mu}_{2}\hat{\mu}_{2}}{1-\hat{\mu}_{1}}
+\hat{f}_{1}\left(3,4\right)\frac{\tilde{\mu}_{2}}{1-\hat{\mu}_{1}}
\end{eqnarray*}



\begin{eqnarray*}
D_{1}D_{4}\hat{F}_{1}&=&
D_{3}D_{3}\hat{F}_{1}\left(D\hat{\theta}_{1}\right)^{2}D_{1}P_{1}D_{4}\hat{P}_{2}
+D_{3}\hat{F}_{1}D^{2}\hat{\theta}_{1}D_{1}P_{1}D_{4}\hat{P}_{2}
+D_{3}\hat{F}_{1}D\hat{\theta}_{1}D_{1}P_{1}D_{4}\hat{P}_{2}
+D_{3}D_{4}\hat{F}_{1}D\hat{\theta}_{1}D_{1}P_{1}\\
&=&\hat{f}_{1}\left(3,3\right)\left(\frac{1}{1-\hat{\mu}_{1}}\right)^{2}\tilde{\mu}_{1}\hat{\mu}_{2}
+\hat{f}_{1}\left(3\right)\hat{\theta}_{1}^{(2)}\tilde{\mu}_{1}\hat{\mu}_{2}
+\hat{f}_{1}\left(3\right)\frac{\tilde{\mu}_{1}\hat{\mu}_{2}}{1-\hat{\mu}_{1}}
+\hat{f}_{1}\left(3,4\right)\frac{\tilde{\mu}_{1}}{1-\hat{\mu}_{1}}
\end{eqnarray*}


\begin{eqnarray*}
D_{2}D_{4}\hat{F}_{1}&=&
D_{3}^{2}\hat{F}_{1}\left(D\hat{\theta}_{1}\right)^{2}D_{2}P_{2}D_{4}\hat{P}_{2}
+D_{3}\hat{F}_{1}D^{2}\hat{\theta}_{1}D_{2}P_{2}D_{4}\hat{P}_{2}
+D_{3}\hat{F}_{1}D\hat{\theta}_{1}D_{2}P_{2}D_{4}\hat{P}_{2}
+D_{3}D_{4}\hat{F}_{1}D\hat{\theta}_{1}D_{2}P_{2}\\
&=&\hat{f}_{1}\left(3,3\right)\left(\frac{1}{1-\hat{\mu}_{1}}\right)^{2}\tilde{\mu}_{2}\hat{\mu}_{2}
+\hat{f}_{1}\left(3\right)\hat{\theta}_{1}^{(2)}\tilde{\mu}_{2}\hat{\mu}_{2}
+\hat{f}_{1}\left(3\right)\frac{\tilde{\mu}_{2}\hat{\mu}_{2}}{1-\hat{\mu}_{1}}
+\hat{f}_{1}\left(3,4\right)\frac{\tilde{\mu}_{2}}{1-\hat{\mu}_{1}}
\end{eqnarray*}



\begin{eqnarray*}
D_{4}D_{4}\hat{F}_{1}&=&
D_{3}^{2}\hat{F}_{1}\left(D\hat{\theta}_{1}\right)^{2}\left(D_{4}\hat{P}_{2}\right)^{2}
+D_{3}\hat{F}_{1}D^{2}\hat{\theta}_{1}\left(D_{4}\hat{P}_{2}\right)^{2}
+D_{3}\hat{F}_{1}D\hat{\theta}_{1}D_{4}^{2}\hat{P}_{2}
+D_{3}D_{4}\hat{F}_{1}D\hat{\theta}_{1}D_{4}\hat{P}_{2}\\
&+&D_{3}D_{4}\hat{F}_{1}D\hat{\theta}_{1}D_{4}\hat{P}_{2}
+D_{4}D_{4}\hat{F}_{1}\\
&=&\hat{f}_{1}\left(3,3\right)\left(\frac{\hat{\mu}_{2}}{1-\hat{\mu}_{1}}\right)^{2}
+\hat{f}_{1}\left(3\right)\hat{\theta}_{1}^{(2)}\hat{\mu}_{2}^{2}
+\hat{f}_{1}\left(3\right)\frac{\hat{P}_{2}^{(2)}}{1-\hat{\mu}_{1}}
+\hat{f}_{1}\left(3,4\right)\frac{\hat{\mu}_{2}}{1-\hat{\mu}_{1}}
+\hat{f}_{1}\left(3,4\right)\frac{\hat{\mu}_{2}}{1-\hat{\mu}_{1}}
+\hat{f}_{1}\left(4,4\right)
\end{eqnarray*}




Finally for $\hat{F}_{2}\left(w_{1},\hat{\theta}_{2}\left(P_{1}\tilde{P}_{2}\hat{P}_{1}\right)\right)$

\begin{eqnarray*}
D_{j}D_{i}\hat{F}_{2}&=&\indora_{i,j\neq4}D_{4}D_{4}\hat{F}_{2}\left(D\hat{\theta}_{2}\right)^{2}D_{i}P_{i}D_{j}P_{j}
+\indora_{i,j\neq4}D_{4}\hat{F}_{2}D^{2}\hat{\theta}_{2}D_{i}P_{i}D_{j}P_{j}
+\indora_{i,j\neq4}D_{4}\hat{F}_{2}D\hat{\theta}_{2}\left(\indora_{i=j}D_{i}^{2}P_{i}+\indora_{i\neq j}D_{i}P_{i}D_{j}P_{j}\right)\\
&+&\left(1-\indora_{i=j=2}\right)\indora_{i+j\geq4}D_{4}D_{3}\hat{F}_{2}D\hat{\theta}_{2}\left(\indora_{i\leq j}D_{i}P_{i}+\indora_{i>j}D_{j}P_{j}\right)
+\indora_{i=3}\left(D_{4}D_{3}\hat{F}_{2}D\hat{\theta}_{2}D_{i}P_{i}+D_{i}^{2}\hat{F}_{2}\right)
\end{eqnarray*}



\begin{eqnarray*}
\begin{array}{llll}
D_{4}D_{1}\hat{F}_{2}=0,&
D_{4}D_{2}\hat{F}_{2}=0,&
D_{4}D_{3}\hat{F}_{2}=0,&
D_{1}D_{4}\hat{F}_{2}=0\\
D_{2}D_{4}\hat{F}_{2}=0,&
D_{3}D_{4}\hat{F}_{2}=0,&
D_{4}D_{4}\hat{F}_{2}=0,&
\end{array}
\end{eqnarray*}


\begin{eqnarray*}
D_{1}D_{1}\hat{F}_{2}&=&
D_{4}^{2}\hat{F}_{2}\left(D\hat{\theta}_{2}\right)^{2}\left(D_{1}P_{1}\right)^{2}
+D_{4}\hat{F}_{2}\hat{\theta}_{2}\left(D_{1}P_{1}\right)^{2}D^{2}+
D_{4}\hat{F}_{2}D\hat{\theta}_{2}D_{1}^{2}P_{1}\\
&=&\hat{f}_{2}\left(4,4\right)\left(\frac{\tilde{\mu}_{1}}{1-\hat{\mu}_{2}}\right)^{2}
+\hat{f}_{2}\left(4\right)\hat{\theta}_{2}^{(2)}\tilde{\mu}_{1}^{2}
+\hat{f}_{2}\left(4\right)\frac{\tilde{P}_{1}^{(2)}}{1-\tilde{\mu}_{2}}
\end{eqnarray*}



\begin{eqnarray*}
D_{2}D_{1}\hat{F}_{2}&=&
D_{4}^{2}\hat{F}_{2}\left(D\hat{\theta}_{2}\right)^{2}D_{1}P_{1}D_{2}P_{2}
+D_{4}\hat{F}_{2}D^{2}\hat{\theta}_{2}D_{1}P_{1}D_{2}P_{2}
+D_{4}\hat{F}_{2}D\hat{\theta}_{2}D_{1}P_{1}D_{2}P_{2}\\
&=&\hat{f}_{2}\left(4,4\right)\left(\frac{1}{1-\hat{\mu}_{2}}\right)^{2}\tilde{\mu}_{1}\tilde{\mu}_{2}
+\hat{f}_{2}\left(4\right)\hat{\theta}_{2}^{(2)}\tilde{\mu}_{1}\tilde{\mu}_{2}
+\hat{f}_{2}\left(4\right)\frac{\tilde{\mu}_{1}\tilde{\mu}_{2}}{1-\tilde{\mu}_{2}}
\end{eqnarray*}



\begin{eqnarray*}
D_{3}D_{1}\hat{F}_{2}&=&
D_{4}^{2}\hat{F}_{2}\left(D\hat{\theta}_{2}\right)^{2}D_{1}P_{1}D_{3}\hat{P}_{1}
+D_{4}\hat{F}_{2}D^{2}\hat{\theta}_{2}D_{1}P_{1}D_{3}\hat{P}_{1}
+D_{4}\hat{F}_{2}D\hat{\theta}_{2}D_{1}P_{1}D_{3}\hat{P}_{1}
+D_{4}D_{3}\hat{F}_{2}D\hat{\theta}_{2}D_{1}P_{1}\\
&=&\hat{f}_{2}\left(4,4\right)\left(\frac{1}{1-\hat{\mu}_{2}}\right)^{2}\tilde{\mu}_{1}\hat{\mu}_{1}
+\hat{f}_{2}\left(4\right)\hat{\theta}_{2}^{(2)}\tilde{\mu}_{1}\hat{\mu}_{1}
+\hat{f}_{2}\left(4\right)\frac{\tilde{\mu}_{1}\hat{\mu}_{1}}{1-\hat{\mu}_{2}}
+\hat{f}_{2}\left(4,3\right)\frac{\tilde{\mu}_{1}}{1-\hat{\mu}_{2}}
\end{eqnarray*}



\begin{eqnarray*}
D_{1}D_{2}\hat{F}_{2}&=&
D_{4}D_{4}\hat{F}_{2}\left(D\hat{\theta}_{2}\right)^{2}D_{1}P_{1}D_{2}P_{2}
+D_{4}\hat{F}_{2}D^{2}\hat{\theta}_{2}D_{1}P_{1}D_{2}P_{2}
+D_{4}\hat{F}_{2}D\hat{\theta}_{2}D_{1}P_{1}D_{2}P_{2}
\\
&=&
\hat{f}_{2}\left(4,4\right)\left(\frac{1}{1-\hat{\mu}_{2}}\right)^{2}\tilde{\mu}_{1}\tilde{\mu}_{2}
+\hat{f}_{2}\left(4\right)\hat{\theta}_{2}^{(2)}\tilde{\mu}_{1}\tilde{\mu}_{2}
+\hat{f}_{2}\left(4\right)\frac{\tilde{\mu}_{1}\tilde{\mu}_{2}}{1-\tilde{\mu}_{2}}
\end{eqnarray*}



\begin{eqnarray*}
D_{2}D_{2}\hat{F}_{2}&=&
D_{4}^{2}\hat{F}_{2}\left(D\hat{\theta}_{2}\right)^{2}\left(D_{2}P_{2}\right)^{2}
+D_{4}\hat{F}_{2}D^{2}\hat{\theta}_{2}\left(D_{2}P_{2}\right)^{2}
+D_{4}\hat{F}_{2}D\hat{\theta}_{2}D_{2}^{2}P_{2}
\\
&=&\hat{f}_{2}\left(4,4\right)\left(\frac{\tilde{\mu}_{2}}{1-\hat{\mu}_{2}}\right)^{2}
+\hat{f}_{2}\left(4\right)\hat{\theta}_{2}^{(2)}\tilde{\mu}_{2}^{2}
+\hat{f}_{2}\left(4\right)\frac{\tilde{P}_{2}^{(2)}}{1-\hat{\mu}_{2}}
\end{eqnarray*}



\begin{eqnarray*}
D_{3}D_{2}\hat{F}_{2}&=&
D_{4}^{2}\hat{F}_{2}\left(D\hat{\theta}_{2}\right)^{2}D_{2}P_{2}D_{3}\hat{P}_{1}
+D_{4}\hat{F}_{2} D^{2}\hat{\theta}_{2}D_{2}P_{2}D_{3}\hat{P}_{1}
+D_{4}\hat{F}_{2}D\hat{\theta} _{2}D_{2}P_{2}D_{3}\hat{P}_{1}
+D_{4}D_{3}\hat{F}_{2}D\hat{\theta}_{2}D_{2}P_{2}\\
&=&
\hat{f}_{2}\left(4,4\right)\left(\frac{1}{1-\hat{\mu}_{2}}\right)^{2}\tilde{\mu}_{2}\hat{\mu}_{1}
+\hat{f}_{2}\left(4\right)\hat{\theta}_{2}^{(2)}\tilde{\mu}_{2}\hat{\mu}_{1}
+\hat{f}_{2}\left(4\right)\frac{\tilde{\mu}_{2}\hat{\mu}_{1}}{1-\hat{\mu}_{2}}
+\hat{f}_{2}\left(4,3\right)\frac{\tilde{\mu}_{2}}{1-\hat{\mu}_{2}}
\end{eqnarray*}



\begin{eqnarray*}
D_{1}D_{3}\hat{F}_{2}&=&
D_{4}D_{4}\hat{F}_{2}\left(D\hat{\theta}_{2}\right)^{2}D_{1}P_{1}D_{3}\hat{P}_{1}
+D_{4}\hat{F}_{2}D^{2}\hat{\theta}_{2}D_{1}P_{1}D_{3}\hat{P}_{1}
+D_{4}\hat{F}_{2}D\hat{\theta}_{2}D_{1}P_{1}D_{3}\hat{P}_{1}
+D_{4}D_{3}\hat{F}_{2}D\hat{\theta}_{2}D_{1}P_{1}\\
&=&
\hat{f}_{2}\left(4,4\right)\left(\frac{1}{1-\hat{\mu}_{2}}\right)^{2}\tilde{\mu}_{1}\hat{\mu}_{1}
+\hat{f}_{2}\left(4\right)\hat{\theta}_{2}^{(2)}\tilde{\mu}_{1}\hat{\mu}_{1}
+\hat{f}_{2}\left(4\right)\frac{\tilde{\mu}_{1}\hat{\mu}_{1}}{1-\hat{\mu}_{2}}
+\hat{f}_{2}\left(4,3\right)\frac{\tilde{\mu}_{1}}{1-\hat{\mu}_{2}}
\end{eqnarray*}



\begin{eqnarray*}
D_{2}D_{3}\hat{F}_{2}&=&
D_{4}^{2}\hat{F}_{2}\left(D\hat{\theta}_{2}\right)^{2}D_{2}P_{2}D_{3}\hat{P}_{1}
+D_{4}\hat{F}_{2}D^{2}\hat{\theta}_{2}D_{2}P_{2}D_{3}\hat{P}_{1}
+D_{4}\hat{F}_{2}D\hat{\theta}_{2}D_{2}P_{2}D_{3}\hat{P}_{1}
+D_{4}D_{3}\hat{F}_{2}D\hat{\theta}_{2}D_{2}P_{2}\\
&=&
\hat{f}_{2}\left(4,4\right)\left(\frac{1}{1-\hat{\mu}_{2}}\right)^{2}\tilde{\mu}_{2}\hat{\mu}_{1}
+\hat{f}_{2}\left(4\right)\hat{\theta}_{2}^{(2)}\tilde{\mu}_{2}\hat{\mu}_{1}
+\hat{f}_{2}\left(4\right)\frac{\tilde{\mu}_{2}\hat{\mu}_{1}}{1-\hat{\mu}_{2}}
+\hat{f}_{2}\left(4,3\right)\frac{\tilde{\mu}_{2}}{1-\hat{\mu}_{2}}
\end{eqnarray*}



\begin{eqnarray*}
D_{3}D_{3}\hat{F}_{2}&=&
D_{4}^{2}\hat{F}_{2}\left(D\hat{\theta}_{2}\right)^{2}\left(D_{3}\hat{P}_{1}\right)^{2}
+D_{4}\hat{F}_{2}D^{2}\hat{\theta}_{2}\left(D_{3}\hat{P}_{1}\right)^{2}
+D_{4}\hat{F}_{2}D\hat{\theta}_{2}D_{3}^{2}\hat{P}_{1}
+D_{4}D_{3}\hat{F}_{2}D\hat{\theta}_{2}D_{3}\hat{P}_{1}\\
&+&D_{4}D_{3}\hat{f}_{2}D\hat{\theta}_{2}D_{3}\hat{P}_{1}
+D_{3}^{2}\hat{F}_{2}\\
&=&
\hat{f}_{2}\left(4,4\right)\left(\frac{\hat{\mu}_{1}}{1-\hat{\mu}_{2}}\right)^{2}
+\hat{f}_{2}\left(4\right)\hat{\theta}_{2}^{(2)}\hat{\mu}_{1}^{2}
+\hat{f}_{2}\left(4\right)\frac{\hat{P}_{1}^{(2)}}{1-\hat{\mu}_{2}}
+\hat{f}_{2}\left(4,3\right)\frac{\hat{\mu}_{1}}{1-\hat{\mu}_{2}}
+\hat{f}_{2}\left(4,3\right)\frac{\tilde{\mu}_{1}}{1-\hat{\mu}_{2}}
+\hat{f}_{2}\left(3,3\right)
\end{eqnarray*}

%_____________________________________________________________
\subsection{Second Grade Derivative Recursive Equations}
%_____________________________________________________________


Then according to the equations given at the beginning of this section, we have

\begin{eqnarray*}
D_{k}D_{i}F_{1}&=&D_{k}D_{i}\left(R_{2}+F_{2}+\indora_{i\geq3}\hat{F}_{4}\right)+D_{i}R_{2}D_{k}\left(F_{2}+\indora_{k\geq3}\hat{F}_{4}\right)\\&+&D_{i}F_{2}D_{k}\left(R_{2}+\indora_{k\geq3}\hat{F}_{4}\right)+\indora_{i\geq3}D_{i}\hat{F}_{4}D_{k}\left(R_{2}+F_{2}\right)
\end{eqnarray*}
%_____________________________________________________________
\subsection*{$F_{1}$}
%_____________________________________________________________
%_____________________________________________________________
\subsubsection*{$F_{1}$ and $i=1$}
%_____________________________________________________________

for $i=1$, and $k=1$

\begin{eqnarray*}
D_{1}D_{1}F_{1}&=&D_{1}D_{1}\left(R_{2}+F_{2}\right)+D_{1}R_{2}D_{1}F_{2}
+D_{1}F_{2}D_{1}R_{2}
=D_{1}^{2}R_{2}
+D_{1}^{2}F_{2}
+D_{1}R_{2}D_{1}F_{2}
+D_{1}F_{2}D_{1}R_{2}\\
&=&R_{2}^{(2)}\tilde{\mu}_{1}+r_{2}\tilde{P}_{1}^{(2)}
+D_{1}^{2}F_{2}
+2r_{2}\tilde{\mu}_{1}f_{2}\left(1\right)
\end{eqnarray*}

$k=2$
\begin{eqnarray*}
D_{2}D_{i}F_{1}&=&D_{2}D_{1}\left(R_{2}+F_{2}\right)
+D_{1}R_{2}D_{2}F_{2}+D_{1}F_{2}D_{2}R_{2}
=D_{2}D_{1}R_{2}
+D_{2}D_{1}F_{2}
+D_{1}R_{2}D_{2}F_{2}
+D_{1}F_{2}D_{2}R_{2}\\
&=&R_{2}^{(2)}\tilde{\mu}_{1}\tilde{\mu}_{2}+r_{2}\tilde{\mu}_{1}\tilde{\mu}_{2}
+D_{2}D_{1}F_{2}
+r_{2}\tilde{\mu}_{1}f_{2}\left(2\right)
+r_{2}\tilde{\mu}_{2}f_{2}\left(1\right)
\end{eqnarray*}

$k=3$
\begin{eqnarray*}
D_{3}D_{1}F_{1}&=&D_{3}D_{1}\left(R_{2}+F_{2}\right)
+D_{1}R_{2}D_{3}\left(F_{2}+\hat{F}_{4}\right)
+D_{1}F_{2}D_{3}\left(R_{2}+\hat{F}_{4}\right)\\
&=&D_{3}D_{1}R_{2}+D_{3}D_{1}F_{2}
+D_{1}R_{2}D_{3}F_{2}+D_{1}R_{2}D_{3}\hat{F}_{4}
+D_{1}F_{2}D_{3}R_{2}+D_{1}F_{2}D_{3}\hat{F}_{4}\\
&=&R_{2}^{(2)}\tilde{\mu}_{1}\hat{\mu}_{1}+r_{2}\tilde{\mu}_{1}\hat{\mu}_{1}
+D_{3}D_{1}F_{2}
+r_{2}\tilde{\mu}_{1}f_{2}\left(3\right)
+r_{2}\tilde{\mu}_{1}D_{3}\hat{F}_{4}
+r_{2}\hat{\mu}_{1}f_{2}\left(1\right)
+D_{3}\hat{F}_{4}f_{2}\left(1\right)
\end{eqnarray*}

$k=4$
\begin{eqnarray*}
D_{4}D_{1}F_{1}&=&D_{4}D_{1}\left(R_{2}+F_{2}\right)
+D_{1}R_{2}D_{4}\left(F_{2}+\hat{F}_{4}\right)
+D_{1}F_{2}D_{4}\left(R_{2}+\hat{F}_{4}\right)\\
&=&D_{4}D_{1}R_{2}+D_{4}D_{1}F_{2}
+D_{1}R_{2}D_{4}F_{2}+D_{1}R_{2}D_{4}\hat{F}_{4}
+D_{1}F_{2}D_{4}R_{2}+D_{1}F_{2}D_{4}\hat{F}_{4}\\
&=&R_{2}^{(2)}\tilde{\mu}_{1}\hat{\mu}_{2}+r_{2}\tilde{\mu}_{1}\hat{\mu}_{2}
+D_{4}D_{1}F_{2}
+r_{2}\tilde{\mu}_{1}f_{2}\left(4\right)
+r_{2}\tilde{\mu}_{1}D_{4}\hat{F}_{4}
+r_{2}\hat{\mu}_{2}f_{2}\left(1\right)
+f_{2}\left(1\right)D_{4}\hat{F}_{4}
\end{eqnarray*}


%_____________________________________________________________
\subsubsection*{$F_{1}$ and $i=2$}
%_____________________________________________________________

for $i=2$,

$k=2$
\begin{eqnarray*}
D_{2}D_{2}F_{1}&=&D_{2}D_{2}\left(R_{2}+F_{2}\right)
+D_{2}R_{2}D_{2}F_{2}+D_{2}F_{2}D_{2}R_{2}
=D_{2}D_{2}R_{2}+D_{2}D_{2}F_{2}+D_{2}R_{2}D_{2}F_{2}+D_{2}F_{2}D_{2}R_{2}\\
&=&R_{2}^{(2)}\tilde{\mu}_{2}^{2}+r_{2}\tilde{P}_{2}^{(2)}
+D_{2}D_{2}F_{2}
+2r_{2}\tilde{\mu}_{2}f_{2}\left(2\right)
\end{eqnarray*}

$k=3$
\begin{eqnarray*}
D_{3}D_{2}F_{1}&=&D_{3}D_{2}\left(R_{2}+F_{2}\right)
+D_{2}R_{2}D_{3}\left(F_{2}+\hat{F}_{4}\right)
+D_{2}F_{2}D_{3}\left(R_{2}+\hat{F}_{4}\right)\\
&=&D_{3}D_{2}R_{2}+D_{3}D_{2}F_{2}
+D_{2}R_{2}D_{3}F_{2}+D_{2}R_{2}D_{3}\hat{F}_{4}
+D_{2}F_{2}D_{3}R_{2}+D_{2}F_{2}D_{3}\hat{F}_{4}\\
&=&R_{2}^{(2)}\tilde{\mu}_{2}\hat{\mu}_{1}+r_{2}\tilde{\mu}_{2}\hat{\mu}_{1}
+D_{3}D_{2}F_{2}
+r_{2}\tilde{\mu}_{2}f_{2}\left(3\right)
+r_{2}\tilde{\mu}_{2}D_{3}\hat{F}_{4}
+r_{2}\hat{\mu}_{1}f_{2}\left(2\right)
+f_{2}\left(2\right)D_{3}\hat{F}_{4}
\end{eqnarray*}

$k=4$
\begin{eqnarray*}
D_{4}D_{2}F_{1}&=&D_{4}D_{2}\left(R_{2}+F_{2}\right)
+D_{2}R_{2}D_{4}\left(F_{2}+\hat{F}_{4}\right)
+D_{2}F_{2}D_{4}\left(R_{2}+\hat{F}_{4}\right)\\
&=&D_{4}D_{2}R_{2}+D_{4}D_{2}F_{2}
+D_{2}R_{2}D_{4}F_{2}+D_{2}R_{2}D_{4}\hat{F}_{4}
+D_{2}F_{2}D_{4}R_{2}+D_{2}F_{2}D_{4}\hat{F}_{4}\\
&=&R_{2}^{(2)}\tilde{\mu}_{2}\hat{\mu}_{2}+r_{2}\tilde{\mu}_{2}\hat{\mu}_{2}
+D_{4}D_{2}F_{2}
+r_{2}\tilde{\mu}_{2}f_{2}\left(4\right)
+r_{2}\tilde{\mu}_{2}D_{4}\hat{F}_{4}
+r_{2}\hat{\mu}_{2}f_{2}\left(2\right)
+f_{2}\left(2\right)D_{4}\hat{F}_{4}
\end{eqnarray*}

%_____________________________________________________________
\subsubsection*{$F_{1}$ and $i=3$}
%_____________________________________________________________
for $i=3$, and $k=3$
\begin{eqnarray*}
D_{3}D_{3}F_{1}&=&D_{3}D_{3}\left(R_{2}+F_{2}+\hat{F}_{4}\right)
+D_{3}R_{2}D_{3}\left(F_{2}+\hat{F}_{4}\right)
+D_{3}F_{2}D_{3}\left(R_{2}+\hat{F}_{4}\right)
+D_{3}\hat{F}_{4}D_{3}\left(R_{2}+F_{2}\right)\\
&=&D_{3}D_{3}R_{2}+D_{3}D_{3}F_{2}+D_{3}D_{3}\hat{F}_{4}
+D_{3}R_{2}D_{3}F_{2}+D_{3}R_{2}D_{3}\hat{F}_{4}\\
&+&D_{3}F_{2}D_{3}R_{2}+D_{3}F_{2}D_{3}\hat{F}_{4}
+D_{3}\hat{F}_{4}D_{3}R_{2}+D_{3}\hat{F}_{4}D_{3}F_{2}\\
&=&R_{2}^{(2)}\hat{\mu}_{1}^{2}+r_{2}\hat{P}_{1}^{(2)}
+D_{3}D_{3}F_{2}
+D_{3}D_{3}\hat{F}_{4}
+r_{2}\hat{\mu}_{1}f_{2}\left(3\right)
+r_{2}\hat{\mu}_{1}D_{3}\hat{F}_{4}\\
&+&r_{2}\hat{\mu}_{1}f_{2}\left(3\right)
+f_{2}\left(3\right)D_{3}\hat{F}_{4}
+r_{2}\hat{\mu}_{1}D_{3}\hat{F}_{4}
+f_{2}\left(3\right)D_{3}\hat{F}_{4}
\end{eqnarray*}

$k=4$
\begin{eqnarray*}
D_{4}D_{3}F_{1}&=&D_{4}D_{3}\left(R_{2}+F_{2}+\hat{F}_{4}\right)
+D_{3}R_{2}D_{4}\left(F_{2}+\hat{F}_{4}\right)
+D_{3}F_{2}D_{4}\left(R_{2}+\hat{F}_{4}\right)
+D_{3}\hat{F}_{4}D_{4}\left(R_{2}+F_{2}\right)\\
&=&D_{4}D_{3}R_{2}+D_{4}D_{3}F_{2}+D_{4}D_{3}\hat{F}_{4}
+D_{3}R_{2}D_{4}F_{2}+D_{3}R_{2}D_{4}\hat{F}_{4}\\
&+&D_{3}F_{2}D_{4}R_{2}+D_{3}F_{2}D_{4}\hat{F}_{4}
+D_{3}\hat{F}_{4}D_{4}R_{2}+D_{3}\hat{F}_{4}D_{4}F_{2}\\
&=&R_{2}^{(2)}\hat{\mu}_{1}\hat{\mu}_{2}+r_{2}\hat{\mu}_{1}\hat{\mu}_{2}
+D_{4}D_{3}F_{2}
+D_{4}D_{3}\hat{F}_{4}
+r_{2}\hat{\mu}_{1}f_{2}\left(4\right)
+r_{2}\hat{\mu}_{1}D_{4}\hat{F}_{4}\\
&+&r_{2}\hat{\mu}_{2}f_{2}\left(3\right)
+D_{4}\hat{F}_{4}f_{2}\left(3\right)
+D_{3}\hat{F}_{4}r_{2}\hat{\mu}_{2}
+D_{3}\hat{F}_{4}f_{2}\left(4\right)
\end{eqnarray*}

%_____________________________________________________________
\subsubsection*{$F_{1}$ and $i=4$}
%_____________________________________________________________

for $i=4$, $k=4$
\begin{eqnarray*}
D_{4}D_{4}F_{1}&=&D_{4}D_{4}\left(R_{2}+F_{2}+\hat{F}_{4}\right)
+D_{4}R_{2}D_{4}\left(F_{2}+\hat{F}_{4}\right)
+D_{4}F_{2}D_{4}\left(R_{2}+\hat{F}_{4}\right)
+D_{4}\hat{F}_{4}D_{4}\left(R_{2}+F_{2}\right)\\
&=&D_{4}D_{4}R_{2}+D_{4}D_{4}F_{2}+D_{4}D_{4}\hat{F}_{4}
+D_{4}R_{2}D_{4}F_{2}+D_{4}R_{2}D_{4}\hat{F}_{4}\\
&+&D_{4}F_{2}D_{4}R_{2}+D_{4}F_{2}D_{4}\hat{F}_{4}
+D_{4}\hat{F}_{4}D_{4}R_{2}+D_{4}\hat{F}_{4}D_{4}F_{2}\\
&=&R_{2}^{(2)}\hat{\mu}_{2}^{2}+r_{2}\hat{P}_{2}^{(2)}
+D_{4}D_{4}F_{2}
+D_{4}D_{4}\hat{F}_{4}
+r_{2}\hat{\mu}_{2}f_{2}\left(4\right)
+r_{2}\hat{\mu}_{2}D_{4}\hat{F}_{4}\\
&+&r_{2}\hat{\mu}_{2}f_{2}\left(4\right)
+D_{4}\hat{F}_{4}f_{2}\left(4\right)
+D_{4}\hat{F}_{4}r_{2}\hat{\mu}_{2}
+D_{4}\hat{F}_{4}f_{2}\left(4\right)
\end{eqnarray*}

%__________________________________________________________________________________________
%_____________________________________________________________
\subsection*{$F_{2}$}
%_____________________________________________________________
\begin{eqnarray}
D_{k}D_{i}F_{2}&=&D_{k}D_{i}\left(R_{1}+F_{1}+\indora_{i\geq3}\hat{F}_{3}\right)+D_{i}R_{1}D_{k}\left(F_{1}+\indora_{k\geq3}\hat{F}_{3}\right)+D_{i}F_{1}D_{k}\left(R_{1}+\indora_{k\geq3}\hat{F}_{3}\right)+\indora_{i\geq3}D_{i}\hat{F}_{3}D_{k}\left(R_{1}+F_{1}\right)
\end{eqnarray}
%_____________________________________________________________
\subsubsection*{$F_{2}$ and $i=1$}
%_____________________________________________________________
$i=1$, $k=1$
\begin{eqnarray*}
D_{1}D_{1}F_{2}&=&D_{1}D_{1}\left(R_{1}+F_{1}\right)
+D_{1}R_{1}D_{1}F_{1}
+D_{1}F_{1}D_{1}R_{1}
=D_{1}^{2}R_{1}
+D_{1}^{2}F_{1}
+D_{1}R_{1}D_{1}F_{1}
+D_{1}F_{1}D_{1}R_{1}\\
&=&R_{1}^{2}\tilde{\mu}_{1}^{2}+r_{1}\tilde{P}_{1}^{(2)}
+D_{1}^{2}F_{1}
+2r_{1}\tilde{\mu}_{1}f_{1}\left(1\right)
\end{eqnarray*}

$k=2$
\begin{eqnarray*}
D_{2}D_{1}F_{2}&=&D_{2}D_{1}\left(R_{1}+F_{1}\right)+D_{1}R_{1}D_{2}F_{1}+D_{1}F_{1}D_{2}R_{1}=
D_{2}D_{1}R_{1}+D_{2}D_{1}F_{1}+D_{1}R_{1}D_{2}F_{1}+D_{1}F_{1}D_{2}R_{1}\\
&=&R_{1}^{(2)}\tilde{\mu}_{1}\tilde{\mu}_{2}+r_{1}\tilde{\mu}_{1}\tilde{\mu}_{2}
+D_{2}D_{1}F_{1}
+r_{1}\tilde{\mu}_{1}f_{1}\left(2\right)
+r_{1}\tilde{\mu}_{2}f_{1}\left(1\right)
\end{eqnarray*}

$k=3$
\begin{eqnarray*}
D_{3}D_{1}F_{2}&=&D_{3}D_{1}\left(R_{1}+F_{1}\right)+D_{1}R_{1}D_{3}\left(F_{1}+\hat{F}_{3}\right)+D_{1}F_{1}D_{3}\left(R_{1}+\hat{F}_{3}\right)\\
&=&D_{3}D_{1}R_{1}+D_{3}D_{1}F_{1}+D_{1}R_{1}D_{3}F_{1}+D_{1}R_{1}D_{3}\hat{F}_{3}+D_{1}F_{1}D_{3}R_{1}+D_{1}F_{1}D_{3}\hat{F}_{3}\\
&=&R_{1}^{(2)}\tilde{\mu}_{1}\hat{\mu}_{1}+r_{1}\tilde{\mu}_{1}\hat{\mu}_{1}
+D_{3}D_{1}F_{1}
+r_{1}\tilde{\mu}_{1}f_{1}\left(3\right)
+r_{1}\tilde{\mu}_{1}D_{3}\hat{F}_{3}
+r_{1}\hat{\mu}_{1}f_{1}\left(1\right)
+D_{3}\hat{F}_{3}f_{1}\left(1\right)
\end{eqnarray*}

$k=4$
\begin{eqnarray*}
D_{4}D_{1}F_{2}&=&D_{4}D_{1}\left(R_{1}+F_{1}\right)+D_{1}R_{1}D_{4}\left(F_{1}+\hat{F}_{3}\right)+D_{1}F_{1}D_{4}\left(R_{1}+\hat{F}_{3}\right)\\
&=&D_{4}D_{1}R_{1}+D_{4}D_{1}F_{1}+D_{1}R_{1}D_{4}F_{1}+D_{1}R_{1}D_{4}\hat{F}_{3}
+D_{1}F_{1}D_{4}R_{1}+D_{1}F_{1}D_{4}\hat{F}_{3}\\
&=&R_{1}^{(2)}\tilde{\mu}_{1}\hat{\mu}_{2}+r_{1}\tilde{\mu}_{1}\hat{\mu}_{2}
+D_{4}D_{1}F_{1}
+r_{1}\tilde{\mu}_{1}f_{1}\left(4\right)
+\tilde{\mu}_{1}D_{4}f_{3}\left(4\right)
+\tilde{\mu}_{1}\hat{\mu}_{2}f_{1}\left(1\right)
+f_{1}\left(1\right)D_{4}F_{4}
\end{eqnarray*}
%_____________________________________________________________
\subsubsection*{$F_{2}$ and $i=2$}
%_____________________________________________________________
%__________________________________________________________________________________________
$i=2$
%__________________________________________________________________________________________
$k=2$
\begin{eqnarray*}
D_{2}D_{2}F_{2}&=&D_{2}D_{2}\left(R_{1}+F_{1}\right)+D_{2}R_{1}D_{2}F_{1}+D_{2}F_{1}D_{2}R_{1}
=D_{2}D_{2}R_{1}+D_{2}D_{2}F_{1}+D_{2}R_{1}D_{2}F_{1}+D_{2}F_{1}D_{2}R_{1}\\
&=&R_{1}^{(2)}\tilde{\mu}_{2}^{2}+r_{1}\tilde{P}_{2}^{(2)}
+D_{2}D_{2}F_{1}
2r_{1}\tilde{\mu}_{2}f_{1}\left(2\right)
\end{eqnarray*}

$k=3$
\begin{eqnarray*}
D_{3}D_{2}F_{2}&=&D_{3}D_{2}\left(R_{1}+F_{1}\right)+D_{2}R_{1}D_{3}\left(F_{1}+\hat{F}_{3}\right)+D_{2}F_{1}D_{3}\left(R_{1}+\hat{F}_{3}\right)\\
&=&D_{3}D_{2}R_{1}+D_{3}D_{2}F_{1}
+D_{2}R_{1}D_{3}F_{1}+D_{2}R_{1}D_{3}\hat{F}_{3}
+D_{2}F_{1}D_{3}R_{1}+D_{2}F_{1}D_{3}\hat{F}_{3}\\
&=&R_{1}^{(2)}\tilde{\mu}_{2}\hat{\mu}_{1}+r_{1}\tilde{\mu}_{2}\hat{\mu}_{1}
+D_{3}D_{2}F_{1}
+r_{1}\tilde{\mu}_{2}f_{1}\left(3\right)
+r_{1}\tilde{\mu}_{2}D_{3}\hat{F}_{3}
+r_{1}\hat{\mu}_{1}f_{1}\left(2\right)
+D_{3}\hat{F}_{3}f_{1}\left(2\right)
\end{eqnarray*}

$k=4$
\begin{eqnarray*}
D_{4}D_{2}F_{2}&=&D_{4}D_{2}\left(R_{1}+F_{1}\right)+D_{2}R_{1}D_{4}\left(F_{1}+\hat{F}_{3}\right)+D_{2}F_{1}D_{4}\left(R_{1}+\hat{F}_{3}\right)\\
&=&D_{4}D_{2}R_{1}+D_{4}D_{2}F_{1}
+D_{2}R_{1}D_{4}F_{1}+D_{2}R_{1}D_{4}\hat{F}_{3}
+D_{2}F_{1}D_{4}R_{1}+D_{2}F_{1}D_{4}\hat{F}_{3}\\
&=&R_{1}^{(2)}\tilde{\mu}_{2}\hat{\mu}_{2}+r_{1}\tilde{\mu}_{2}\hat{\mu}_{2}
+D_{4}D_{2}F_{1}
+r_{1}\tilde{\mu}_{2}f_{1}\left(4\right)
+r_{1}\tilde{\mu}_{2}D_{4}\hat{F}_{3}
+r_{1}\hat{\mu}_{2}f_{1}\left(2\right)
+D_{4}\hat{F}_{3}f_{1}\left(2\right)
\end{eqnarray*}

%_____________________________________________________________
\subsubsection*{$F_{2}$ and $i=3$}
%_____________________________________________________________
%__________________________________________________________________________________________
$i=3$
%__________________________________________________________________________________________
$k=3$
\begin{eqnarray*}
D_{3}D_{3}F_{2}&=&D_{3}D_{3}\left(R_{1}+F_{1}+\hat{F}_{3}\right)
+D_{3}R_{1}D_{3}\left(F_{1}+\hat{F}_{3}\right)
+D_{3}F_{1}D_{3}\left(R_{1}+\hat{F}_{3}\right)
+D_{3}\hat{F}_{3}D_{3}\left(R_{1}+F_{1}\right)\\
&=&D_{3}D_{3}R_{1}+D_{3}D_{3}F_{1}+D_{3}D_{3}\hat{F}_{3}
+D_{3}R_{1}D_{3}F_{1}+D_{3}R_{1}D_{3}\hat{F}_{3}\\
&+&D_{3}F_{1}D_{3}R_{1}+D_{3}F_{1}D_{3}\hat{F}_{3}
+D_{3}\hat{F}_{3}D_{3}R_{1}+D_{3}\hat{F}_{3}D_{3}F_{1}\\
&=&R_{1}^{(2)}\hat{\mu}_{1}^{2}+r_{1}\hat{P}_{1}^{(2)}
+D_{3}D_{3}F_{1}
+D_{3}D_{3}\hat{F}_{3}
+r_{1}\hat{\mu}_{1}f_{1}\left(3\right)
+r_{1}\hat{\mu}_{1}f_{3}\left(3\right)\\
&+&r_{1}\hat{\mu}_{1}f_{1}\left(3\right)
+D_{3}\hat{F}_{3}f_{1}\left(3\right)
+D_{3}\hat{F}_{3}r_{1}\hat{\mu}_{1}
+D_{3}\hat{F}_{3}f_{1}\left(3\right)
\end{eqnarray*}

$k=4$
\begin{eqnarray*}
D_{4}D_{3}F_{2}&=&D_{4}D_{3}\left(R_{1}+F_{1}+\hat{F}_{3}\right)
+D_{3}R_{1}D_{4}\left(F_{1}+\hat{F}_{3}\right)
+D_{3}F_{1}D_{4}\left(R_{1}+\hat{F}_{3}\right)
+D_{3}\hat{F}_{3}D_{4}\left(R_{1}+F_{1}\right)\\
&=&D_{4}D_{3}R_{1}+D_{4}D_{3}F_{1}+D_{4}D_{3}\hat{F}_{3}
+D_{3}R_{1}D_{4}F_{1}+D_{3}R_{1}D_{4}\hat{F}_{3}\\
&+&D_{3}F_{1}D_{4}R_{1}+D_{3}F_{1}D_{4}\hat{F}_{3}
+D_{3}\hat{F}_{3}D_{4}R_{1}+D_{3}\hat{F}_{3}D_{4}F_{1}\\
&=&R_{1}^{(2)}\hat{\mu}_{1}\hat{\mu}_{2}+r_{1}\hat{\mu}_{1}\hat{\mu}_{2}
+D_{4}D_{3}F_{1}
+D_{4}D_{3}\hat{F}_{3}
+r_{1}\hat{\mu}_{1}f_{1}\left(4\right)
+r_{1}\hat{\mu}_{1}D_{4}\hat{F}_{3}\\
&+&r_{1}\hat{\mu}_{2}f_{1}\left(3\right)
+D_{4}\hat{F}_{3}f_{1}\left(3\right)
+r_{1}\hat{\mu}_{2}D_{3}\hat{F}_{3}
+D_{3}\hat{F}_{3}f_{1}\left(4\right)
\end{eqnarray*}
%_____________________________________________________________
\subsubsection*{$F_{2}$ and $i=4$}
%_____________________________________________________________%__________________________________________________________________________________________
$i=4$ and $k=4$
\begin{eqnarray*}
D_{4}D_{4}F_{2}&=&D_{4}D_{4}\left(R_{1}+F_{1}+\hat{F}_{3}\right)
+D_{4}R_{1}D_{4}\left(F_{1}+\hat{F}_{3}\right)
+D_{4}F_{1}D_{4}\left(R_{1}+\hat{F}_{3}\right)
+D_{4}\hat{F}_{3}D_{4}\left(R_{1}+F_{1}\right)\\
&=&D_{4}D_{4}R_{1}+D_{4}D_{4}F_{1}+D_{4}D_{4}\hat{F}_{3}
+D_{4}R_{1}D_{4}F_{1}+D_{4}R_{1}D_{4}\hat{F}_{3}\\
&+&D_{4}F_{1}D_{4}R_{1}+D_{4}F_{1}D_{4}\hat{F}_{3}
+D_{4}\hat{F}_{3}D_{4}R_{1}+D_{4}\hat{F}_{3}D_{4}F_{1}\\
&=&R_{1}^{(2)}\hat{\mu}_{2}^{2}+r_{1}\hat{P}_{2}^{(2)}
+D_{4}D_{4}F_{1}
+D_{4}D_{4}\hat{F}_{3}
+f_{1}\left(4\right)r_{1}\hat{\mu}_{2}
+r_{1}\hat{\mu}_{2}D_{4}\hat{F}_{3}\\
&+&r_{1}\hat{\mu}_{2}f_{1}\left(4\right)
+D_{4}\hat{F}_{3}f_{1}\left(4\right)
+D_{4}\hat{F}_{3}r_{1}\hat{\mu}_{2}
+D_{4}\hat{F}_{3}f_{1}\left(4\right)
\end{eqnarray*}
%__________________________________________________________________________________________
\subsection*{$\hat{F}_{1}$}
%__________________________________________________________________________________________

\begin{eqnarray}
D_{k}D_{i}\hat{F}_{1}&=&D_{k}D_{i}\left(\hat{R}_{4}+\indora_{i\leq2}F_{2}+\hat{F}_{4}\right)+D_{i}\hat{R}_{4}D_{k}\left(\indora_{k\leq2}F_{2}+\hat{F}_{4}\right)+D_{i}\hat{F}_{4}D_{k}\left(\hat{R}_{4}+\indora_{k\leq2}F_{2}\right)+\indora_{i\leq2}D_{i}F_{2}D_{k}\left(\hat{R}_{4}+\hat{F}_{4}\right)
\end{eqnarray}
%__________________________________________________________________________________________
\subsubsection*{$\hat{F}_{1}$, $i=1$}
%__________________________________________________________________________________________

%__________________________________________________________________________________________
$i=1$ and $k=1$
\begin{eqnarray*}
D_{1}D_{1}\hat{F}_{1}&=&D_{1}D_{1}\left(\hat{R}_{4}+F_{2}+\hat{F}_{4}\right)
+D_{1}\hat{R}_{4}D_{1}\left(F_{2}+\hat{F}_{4}\right)
+D_{1}\hat{F}_{4}D_{1}\left(\hat{R}_{4}+F_{2}\right)
+D_{1}F_{2}D_{1}\left(\hat{R}_{4}+\hat{F}_{4}\right)\\
&=&D_{1}^{2}\hat{R}_{4}+D_{1}^{2}F_{2}+D_{1}^{2}\hat{F}_{4}
+D_{1}\hat{R}_{4}D_{1}F_{2}+D_{1}\hat{R}_{4}D_{1}\hat{F}_{4}
+D_{1}\hat{F}_{4}D_{1}\hat{R}_{4}+D_{1}\hat{F}_{4}D_{1}F_{2}
+D_{1}F_{2}D_{1}\hat{R}_{4}+D_{1}F_{2}D_{1}\hat{F}_{4}\\
&=&\hat{R}_{2}^{(2)}\tilde{\mu}_{1}^{2}+\hat{r}_{2}\tilde{P}_{1}^{(2)}
+D_{1}^{2}F_{2}
+D_{1}^{2}\hat{F}_{4}
+\hat{r}_{2}\tilde{\mu}_{1}D_{1}F_{2}\\
&+&\hat{r}_{2}\tilde{\mu}_{1}\hat{f}_{2}\left(1\right)
+\hat{f}_{2}\left(1\right)\hat{r}_{2}\tilde{\mu}_{1}
+\hat{f}_{2}\left(1\right)D_{1}F_{2}
+D_{1}F_{2}\hat{r}_{2}\tilde{\mu}_{1}
+D_{1}F_{2}\hat{f}_{2}\left(1\right)
\end{eqnarray*}

$k=2$
\begin{eqnarray*}
D_{2}D_{1}\hat{F}_{1}&=&D_{2}D_{1}\left(\hat{R}_{4}+F_{2}+\hat{F}_{4}\right)
+D_{1}\hat{R}_{4}D_{2}\left(F_{2}+\hat{F}_{4}\right)
+D_{1}\hat{F}_{4}D_{2}\left(\hat{R}_{4}+F_{2}\right)
+D_{1}F_{2}D_{2}\left(\hat{R}_{4}+\hat{F}_{4}\right)\\
&=&D_{2}D_{1}\hat{R}_{4}+D_{2}D_{1}F_{2}+D_{2}D_{1}\hat{F}_{4}
+D_{1}\hat{R}_{4}D_{2}F_{2}+D_{1}\hat{R}_{4}D_{2}\hat{F}_{4}\\
&+&D_{1}\hat{F}_{4}D_{2}\hat{R}_{4}+D_{1}\hat{F}_{4}D_{2}F_{2}
+D_{1}F_{2}D_{2}\hat{R}_{4}+D_{1}F_{2}D_{2}\hat{F}_{4}\\
&=&\hat{R}_{2}^{(2)}\tilde{\mu}_{1}\tilde{\mu}_{2}+\hat{r}_{2}\tilde{\mu}_{1}\tilde{\mu}_{2}
+D_{2}D_{1}F_{2}
+D_{2}D_{1}\hat{F}_{4}
+\hat{r}_{2}\tilde{\mu}_{1}D_{2}F_{2}
+\hat{r}_{2}\tilde{\mu}_{1}\hat{f}_{2}\left(2\right)\\
&+&\hat{r}_{2}\tilde{\mu}_{2}\hat{f}_{2}\left(1\right)
+\hat{f}_{2}\left(1\right)D_{2}F_{2}
+\hat{r}_{2}\tilde{\mu}_{2}D_{1}F_{2}
+D_{1}F_{2}\hat{f}_{2}\left(2\right)
\end{eqnarray*}

$k=3$
\begin{eqnarray*}
D_{3}D_{1}\hat{F}_{1}&=&D_{3}D_{1}\left(\hat{R}_{4}+F_{2}+\hat{F}_{4}\right)
+D_{1}\hat{R}_{4}D_{3}\left(\hat{F}_{4}\right)
+D_{1}\hat{F}_{4}D_{3}\hat{R}_{4}
+D_{1}F_{2}D_{3}\left(\hat{R}_{4}+\hat{F}_{4}\right)\\
&=&D_{3}D_{1}\hat{R}_{4}+D_{3}D_{1}F_{2}+D_{3}D_{1}\hat{F}_{4}
+D_{1}\hat{R}_{4}D_{3}\hat{F}_{4}
+D_{1}\hat{F}_{4}D_{3}\hat{R}_{4}
+D_{1}F_{2}D_{3}\hat{R}_{4}+D_{1}F_{2}D_{3}\hat{F}_{4}\\
&=&\hat{R}_{2}^{(2)}\tilde{\mu}_{1}\hat{\mu}_{1}+\hat{r}_{2}\tilde{\mu}_{1}\hat{\mu}_{1}
+D_{3}D_{1}F_{2}
+D_{3}D_{1}\hat{F}_{4}
+\hat{r}_{2}\tilde{\mu}_{1}\hat{f}_{2}\left(3\right)
+\hat{f}_{2}\left(1\right)\hat{r}_{2}\hat{\mu}_{1}
+D_{1}F_{2}\hat{r}_{2}\hat{\mu}_{1}
+D_{1}F_{2}\hat{f}_{2}\left(3\right)
\end{eqnarray*}

$k=4$
\begin{eqnarray*}
D_{4}D_{1}\hat{F}_{1}&=&D_{4}D_{1}\left(\hat{R}_{4}+F_{2}+\hat{F}_{4}\right)
+D_{1}\hat{R}_{4}D_{4}\hat{F}_{4}
+D_{1}\hat{F}_{4}D_{4}\hat{R}_{4}
+D_{1}F_{2}D_{4}\left(\hat{R}_{4}+\hat{F}_{4}\right)\\
&=&D_{4}D_{1}\hat{R}_{4}+D_{4}D_{1}F_{2}+D_{4}D_{1}\hat{F}_{4}
+D_{1}\hat{R}_{4}D_{4}\hat{F}_{4}
+D_{1}\hat{F}_{4}D_{4}\hat{R}_{4}
+D_{1}F_{2}D_{4}\hat{R}_{4}+D_{1}F_{2}D_{4}\hat{F}_{4}\\
&=&\hat{R}_{2}^{(2)}\tilde{\mu}_{1}\hat{\mu}_{2}+\hat{r}_{2}\tilde{\mu}_{1}\hat{\mu}_{2}
+D_{4}D_{1}F_{2}
+D_{4}D_{1}\hat{F}_{4}
+\hat{r}_{2}\tilde{\mu}_{1}\hat{f}_{2}\left(4\right)
+\hat{f}_{2}\left(1\right)\hat{r}_{2}\hat{\mu}_{2}
+D_{1}F_{2}\hat{r}_{2}\hat{\mu}_{2}
+D_{1}F_{2}\hat{f}_{2}\left(4\right)
\end{eqnarray*}

%__________________________________________________________________________________________
\subsubsection*{$\hat{F}_{1}$, $i=2$}
%__________________________________________________________________________________________

%__________________________________________________________________________________________
$i=2$ and $k=2$
\begin{eqnarray*}
D_{2}D_{2}\hat{F}_{1}&=&D_{2}D_{2}\left(\hat{R}_{4}+F_{2}+\hat{F}_{4}\right)
+D_{2}\hat{R}_{4}D_{2}\left(F_{2}+\hat{F}_{4}\right)
+D_{2}\hat{F}_{4}D_{2}\left(\hat{R}_{4}+F_{2}\right)
+D_{2}F_{2}D_{2}\left(\hat{R}_{4}+\hat{F}_{4}\right)\\
&=&D_{2}D_{2}\hat{R}_{4}+D_{2}D_{2}F_{2}+D_{2}D_{2}\hat{F}_{4}
+D_{2}\hat{R}_{4}D_{2}F_{2}+D_{2}\hat{R}_{4}D_{2}\hat{F}_{4}\\
&+&D_{2}\hat{F}_{4}D_{2}\hat{R}_{4}+D_{2}\hat{F}_{4}D_{2}F_{2}
+D_{2}F_{2}D_{2}\hat{R}_{4}+D_{2}F_{2}D_{2}\hat{F}_{4}\\
&=&\hat{R}_{2}^{(2)}\tilde{\mu}_{2}^{2}+\hat{r}_{2}\tilde{P}_{1}^{(2)}
+D_{2}D_{2}F_{2}
+D_{2}D_{2}\hat{F}_{4}
+\hat{r}_{2}\tilde{\mu}_{2}D_{2}F_{2}
+\hat{r}_{2}\tilde{\mu}_{2}\hat{f}_{2}\left(4\right)\\
&+&\hat{f}_{2}\left(4\right)\hat{r}_{2}\tilde{\mu}_{2}
+\hat{f}_{2}\left(4\right)D_{2}F_{2}
+D_{2}F_{2}\hat{r}_{2}\tilde{\mu}_{2}
+D_{2}F_{2}\hat{f}_{2}\left(4\right)
\end{eqnarray*}

$k=3$
\begin{eqnarray*}
D_{3}D_{2}\hat{F}_{1}&=&D_{3}D_{2}\left(\hat{R}_{4}+F_{2}+\hat{F}_{4}\right)
+D_{2}\hat{R}_{4}D_{3}\hat{F}_{4}
+D_{2}\hat{F}_{4}D_{3}\hat{R}_{4}
+D_{2}F_{2}D_{3}\left(\hat{R}_{4}+\hat{F}_{4}\right)\\
&=&D_{3}D_{2}\hat{R}_{4}+D_{3}D_{2}F_{2}+D_{3}D_{2}\hat{F}_{4}
+D_{2}\hat{R}_{4}D_{3}\hat{F}_{4}
+D_{2}\hat{F}_{4}D_{3}\hat{R}_{4}
+D_{2}F_{2}D_{3}\hat{R}_{4}+D_{2}F_{2}D_{3}\hat{F}_{4}\\
&=&\hat{R}_{2}^{(2)}\tilde{\mu}_{2}\hat{\mu}_{1}+\hat{r}_{2}\tilde{\mu}_{2}\hat{\mu}_{1}
+D_{3}D_{2}F_{2}
+D_{3}D_{2}\hat{F}_{4}+\hat{r}_{2}\tilde{\mu}_{2}\hat{f}_{2}\left(3\right)
+\hat{f}_{2}\left(4\right)\hat{r}_{2}\hat{\mu}_{1}
+\hat{r}_{2}\hat{\mu}_{1}D_{2}F_{2}
+D_{2}F_{2}\hat{f}_{2}\left(3\right)
\end{eqnarray*}

$k=4$
\begin{eqnarray*}
D_{4}D_{2}\hat{F}_{1}&=&D_{4}D_{2}\left(\hat{R}_{4}+F_{2}+\hat{F}_{4}\right)
+D_{2}\hat{R}_{4}D_{4}\hat{F}_{4}
+D_{2}\hat{F}_{4}D_{4}\hat{R}_{4}
+D_{2}F_{2}D_{4}\left(\hat{R}_{4}+\hat{F}_{4}\right)\\
&=&D_{4}D_{2}\hat{R}_{4}+D_{4}D_{2}F_{2}+D_{4}D_{2}\hat{F}_{4}
+D_{2}\hat{R}_{4}D_{4}\hat{F}_{4}
+D_{2}\hat{F}_{4}D_{4}\hat{R}_{4}
+D_{2}F_{2}D_{4}\hat{R}_{4}+D_{2}F_{2}D_{4}\hat{F}_{4}\\
&=&\hat{R}_{2}^{(2)}\tilde{\mu}_{2}\hat{\mu}_{2}+\hat{r}_{2}\tilde{\mu}_{2}\hat{\mu}_{2}
+D_{4}D_{2}F_{2}
+D_{4}D_{2}\hat{F}_{4}
+\hat{r}_{2}\tilde{\mu}_{2}\hat{f}_{2}\left(4\right)
+\hat{f}_{2}\left(4\right)\hat{r}_{2}\hat{\mu}_{2}
+D_{2}F_{2}\hat{r}_{2}\hat{\mu}_{2}
+D_{2}F_{2}\hat{f}_{2}\left(4\right)
\end{eqnarray*}
%__________________________________________________________________________________________
\subsubsection*{$\hat{F}_{1}$, $i=3$}
%__________________________________________________________________________________________

$k=3$
\begin{eqnarray*}
D_{3}D_{3}\hat{F}_{1}&=&D_{3}D_{3}\left(\hat{R}_{4}+\hat{F}_{4}\right)
+D_{3}\hat{R}_{4}D_{3}\hat{F}_{4}
+D_{3}\hat{F}_{4}D_{3}\hat{R}_{4}=D_{3}^{2}\hat{R}_{4}+D_{3}^{2}\hat{F}_{4}
+D_{3}\hat{R}_{4}D_{3}\hat{F}_{4}
+D_{3}\hat{F}_{4}D_{3}\hat{R}_{4}\\
&=&\hat{R}_{2}^{(2)}\hat{\mu}_{1}^{2}+\hat{r}_{2}\hat{P}_{1}^{(2)}
+D_{3}^{2}\hat{F}_{4}
+\hat{r}_{2}\hat{\mu}_{1}\hat{f}_{2}\left(4\right)
+\hat{r}_{2}\hat{\mu}_{1}\hat{f}_{2}\left(3\right)
\end{eqnarray*}

$k=4$
\begin{eqnarray*}
D_{4}D_{3}\hat{F}_{1}&=&D_{4}D_{3}\left(\hat{R}_{4}+\hat{F}_{4}\right)
+D_{3}\hat{R}_{4}D_{4}\hat{F}_{4}
+D_{3}\hat{F}_{4}D_{4}\hat{R}_{4}=D_{4}D_{3}\hat{R}_{4}+D_{4}D_{3}\hat{F}_{4}
+D_{3}\hat{R}_{4}D_{4}\hat{F}_{4}
+D_{3}\hat{F}_{4}D_{4}\hat{R}_{4}\\
&=&\hat{R}_{2}^{(2)}\hat{\mu}_{1}\hat{\mu}_{2}+\hat{r}_{2}\hat{\mu}_{1}\hat{\mu}_{2}
+D_{4}D_{3}\hat{F}_{4}
+\hat{r}_{2}\hat{\mu}_{1}\hat{f}_{2}\left(4\right)
+\hat{r}_{2}\hat{\mu}_{2}\hat{f}_{2}\left(3\right)
\end{eqnarray*}
%__________________________________________________________________________________________
\subsubsection*{$\hat{F}_{1}$, $i=4$}
%__________________________________________________________________________________________

$k=4$
\begin{eqnarray*}
D_{4}D_{4}\hat{F}_{1}&=&D_{4}D_{4}\left(\hat{R}_{4}+\hat{F}_{4}\right)
+D_{4}\hat{R}_{4}D_{4}\hat{F}_{4}
+D_{4}\hat{F}_{4}D_{4}\hat{R}_{4}=D_{4}^{2}\hat{R}_{4}+D_{4}^{2}\hat{F}_{4}
+D_{4}\hat{R}_{4}D_{4}\hat{F}_{4}
+D_{4}\hat{F}_{4}D_{4}\hat{R}_{4}\\
&=&\hat{R}_{2}^{(2)}\hat{\mu}_{2}^{2}+\hat{r}_{2}\hat{P}_{2}^{(2)}+D_{4}^{2}\hat{F}_{4}
+2\hat{r}_{2}\hat{\mu}_{2}\hat{f}_{2}\left(4\right)
\end{eqnarray*}
%__________________________________________________________________________________________
%
%__________________________________________________________________________________________
\subsection*{$\hat{F}_{2}$}
%__________________________________________________________________________________________
for $\hat{F}_{2}$
%__________________________________________________________________________________________
%
%__________________________________________________________________________________________

\begin{eqnarray}
D_{k}D_{i}\hat{F}_{2}&=&D_{k}D_{i}\left(\hat{R}_{3}+\indora_{i\leq2}F_{1}+\hat{F}_{3}\right)+D_{i}\hat{R}_{3}D_{k}\left(\indora_{k\leq2}F_{1}+\hat{F}_{3}\right)+D_{i}\hat{F}_{3}D_{k}\left(\hat{R}_{3}+\indora_{k\leq2}F_{1}\right)+\indora_{i\leq2}D_{i}F_{1}D_{k}\left(\hat{R}_{3}+\hat{F}_{3}\right)\\
&=&
\end{eqnarray}
%__________________________________________________________________________________________
\subsubsection*{$\hat{F}_{2}$, $i=1$}
%__________________________________________________________________________________________

$k=1$
\begin{eqnarray*}
D_{1}D_{1}\hat{F}_{2}&=&D_{1}^{2}\left(\hat{R}_{3}+F_{1}+\hat{F}_{3}\right)
+D_{1}\hat{R}_{3}D_{1}\left(F_{1}+\hat{F}_{3}\right)
+D_{1}\hat{F}_{3}D_{1}\left(\hat{R}_{3}+F_{1}\right)
+D_{1}F_{1}D_{1}\left(\hat{R}_{3}+\hat{F}_{3}\right)\\
&=&D_{1}^{2}\hat{R}_{3}+D_{1}^{2}F_{1}+D_{1}^{2}\hat{F}_{3}
+D_{1}\hat{R}_{3}D_{1}F_{1}+D_{1}\hat{R}_{3}D_{1}\hat{F}_{3}
+D_{1}\hat{F}_{3}D_{1}\hat{R}_{3}+D_{1}\hat{F}_{3}D_{1}F_{1}
+D_{1}F_{1}D_{1}\hat{R}_{3}+D_{1}F_{1}D_{1}\hat{F}_{3}\\
&=&
\hat{R}_{1}^{(2)}\tilde{\mu}_{1}^{2}+\hat{r}_{1}\tilde{P}_{2}^{(2)}
+D_{1}^{2}F_{1}
+D_{1}^{2}\hat{F}_{3}
+D_{1}F_{1}\hat{r}_{1}\tilde{\mu}_{1}\\
&+&\hat{r}_{1}\tilde{\mu}_{1}\hat{f}_{1}\left(1\right)
+\hat{r}_{1}\tilde{\mu}_{1}\hat{f}_{1}\left(1\right)
+D_{1}F_{1}\hat{f}_{1}\left(1\right)
+D_{1}F_{1}\hat{r}_{1}\tilde{\mu}_{1}
+D_{1}F_{1}\hat{f}_{1}\left(1\right)
\end{eqnarray*}

$k=2$
\begin{eqnarray*}
D_{2}D_{1}\hat{F}_{2}&=&D_{2}D_{1}\left(\hat{R}_{3}+F_{1}+\hat{F}_{3}\right)
+D_{1}\hat{R}_{3}D_{2}\left(F_{1}+\hat{F}_{3}\right)
+D_{1}\hat{F}_{3}D_{2}\left(\hat{R}_{3}+F_{1}\right)
+D_{1}F_{1}D_{2}\left(\hat{R}_{3}+\hat{F}_{3}\right)\\
&=&D_{2}D_{1}\hat{R}_{3}+D_{2}D_{1}F_{1}+D_{2}D_{1}\hat{F}_{3}
+D_{1}\hat{R}_{3}D_{2}F_{1}+D_{1}\hat{R}_{3}D_{2}\hat{F}_{3}\\
&+&D_{1}\hat{F}_{3}D_{2}\hat{R}_{3}+D_{1}\hat{F}_{3}D_{2}F_{1}
+D_{1}F_{1}D_{2}\hat{R}_{3}+D_{1}F_{1}D_{2}\hat{F}_{3}\\
&=&\hat{R}_{1}^{(2)}\tilde{\mu}_{1}\tilde{\mu}_{2}+\hat{r}_{1}\tilde{\mu}_{1}\tilde{\mu}_{2}
+D_{2}D_{1}F_{1}
+D_{2}D_{1}\hat{F}_{3}
+\hat{r}_{1}\tilde{\mu}_{1}D_{2}F_{1}
+\hat{r}_{1}\tilde{\mu}_{1}\hat{f}_{1}\left(2\right)\\
&+&\hat{f}_{1}\left(1\right)\hat{r}_{1}\tilde{\mu}_{2}
+\hat{r}_{1}\tilde{\mu}_{1}D_{2}F_{1}
+D_{1}F_{1}\hat{r}_{1}\tilde{\mu}_{2}
+D_{1}F_{1}\hat{f}_{1}\left(2\right)
\end{eqnarray*}

$k=3$
\begin{eqnarray*}
D_{3}D_{1}\hat{F}_{2}&=&D_{3}D_{1}\left(\hat{R}_{3}+F_{1}+\hat{F}_{3}\right)
+D_{1}\hat{R}_{3}D_{3}\hat{F}_{3}
+D_{1}\hat{F}_{3}D_{3}\hat{R}_{3}
+D_{1}F_{1}D_{3}\left(\hat{R}_{3}+\hat{F}_{3}\right)\\
&=&D_{3}D_{1}\hat{R}_{3}+D_{3}D_{1}F_{1}+D_{3}D_{1}\hat{F}_{3}
+D_{1}\hat{R}_{3}D_{3}\hat{F}_{3}
+D_{1}\hat{F}_{3}D_{3}\hat{R}_{3}
+D_{1}F_{1}D_{3}\hat{R}_{3}+D_{1}F_{1}D_{3}\hat{F}_{3}\\
&=&\hat{R}_{1}^{(2)}\tilde{\mu}_{1}\hat{\mu}_{1}+\hat{r}_{1}\tilde{\mu}_{1}\hat{\mu}_{1}
+D_{3}D_{1}F_{1}
+D_{3}D_{1}\hat{F}_{3}
+\hat{r}_{1}\tilde{\mu}_{1}\hat{f}_{1}\left(3\right)
+\hat{r}_{1}\hat{\mu}_{1}\hat{f}_{1}\left(1\right)
+\hat{r}_{1}\hat{\mu}_{1}D_{1}F_{1}
+D_{1}F_{1}\hat{f}_{1}\left(3\right)
\end{eqnarray*}

$k=4$
\begin{eqnarray*}
D_{4}D_{1}\hat{F}_{2}&=&D_{4}D_{1}\left(\hat{R}_{3}+F_{1}+\hat{F}_{3}\right)
+D_{1}\hat{R}_{3}D_{4}\hat{F}_{3}
+D_{1}\hat{F}_{3}D_{4}\hat{R}_{3}
+D_{1}F_{1}D_{4}\left(\hat{R}_{3}+\hat{F}_{3}\right)\\
&=&D_{4}D_{1}\hat{R}_{3}+D_{4}D_{1}F_{1}+D_{4}D_{1}\hat{F}_{3}
+D_{1}\hat{R}_{3}D_{4}\hat{F}_{3}
+D_{1}\hat{F}_{3}D_{4}\hat{R}_{3}
+D_{1}F_{1}D_{4}\hat{R}_{3}+D_{1}F_{1}D_{4}\hat{F}_{3}\\
&=&\hat{R}_{1}^{(2)}\tilde{\mu}_{1}\hat{\mu}_{2}+\hat{r}_{1}\tilde{\mu}_{1}\hat{\mu}_{2}
+D_{4}D_{1}F_{1}
+D_{4}D_{1}\hat{F}_{3}
+\hat{f}_{1}\left(4\right)\hat{r}_{1}\tilde{\mu}_{1}
+\hat{f}_{1}\left(3\right)\hat{r}_{1}\hat{\mu}_{2}
+D_{1}F_{1}\hat{r}_{1}\hat{\mu}_{2}
+D_{1}F_{1}\hat{f}_{1}\left(4\right)
\end{eqnarray*}
%__________________________________________________________________________________________
\subsubsection*{$\hat{F}_{2}$, $i=2$}
%__________________________________________________________________________________________


$k=2$
\begin{eqnarray*}
D_{2}D_{2}\hat{F}_{2}&=&D_{2}D_{2}\left(\hat{R}_{3}+F_{1}+\hat{F}_{3}\right)
+D_{2}\hat{R}_{3}D_{2}\left(F_{1}+\hat{F}_{3}\right)
+D_{2}\hat{F}_{3}D_{2}\left(\hat{R}_{3}+F_{1}\right)
+D_{2}F_{1}D_{2}\left(\hat{R}_{3}+\hat{F}_{3}\right)\\
&=&D_{2}^{2}\hat{R}_{3}+D_{2}^{2}F_{1}+D_{2}^{2}\hat{F}_{3}
+D_{2}\hat{R}_{3}D_{2}F_{1}+D_{2}\hat{R}_{3}D_{2}\hat{F}_{3}
+D_{2}\hat{F}_{3}D_{2}\hat{R}_{3}+D_{2}\hat{F}_{3}D_{2}F_{1}
+D_{2}F_{1}D_{2}\hat{R}_{3}+D_{2}F_{1}D_{2}\hat{F}_{3}\\
&=&\hat{R}_{1}^{(2)}\tilde{\mu}_{2}^{2}+\hat{r}_{1}\tilde{P}_{2}^{(2)}
+D_{2}^{2}F_{1}
+D_{2}^{2}\hat{F}_{3}
+\hat{r}_{1}\tilde{\mu}_{2}D_{2}F_{1}\\
&+&\hat{r}_{1}\tilde{\mu}_{2}\hat{f}_{1}\left(2\right)
+\hat{r}_{1}\tilde{\mu}_{2}\hat{f}_{1}\left(2\right)
+\hat{f}_{1}\left(1\right)D_{2}F_{1}
+\hat{r}_{1}\tilde{\mu}_{2}D_{2}F_{1}
+\hat{f}_{1}\left(3\right)D_{2}F_{1}
\end{eqnarray*}

$k=3$
\begin{eqnarray*}
D_{3}D_{2}\hat{F}_{2}&=&D_{3}D_{2}\left(\hat{R}_{3}+F_{1}+\hat{F}_{3}\right)
+D_{2}\hat{R}_{3}D_{3}\hat{F}_{3}
+D_{2}\hat{F}_{3}D_{3}\hat{R}_{3}
+D_{2}F_{1}D_{3}\left(\hat{R}_{3}+\hat{F}_{3}\right)\\
&=&D_{3}D_{2}\hat{R}_{3}+D_{3}D_{2}F_{1}+D_{3}D_{2}\hat{F}_{3}
+D_{2}\hat{R}_{3}D_{3}\hat{F}_{3}
+D_{2}\hat{F}_{3}D_{3}\hat{R}_{3}
+D_{2}F_{1}D_{3}\hat{R}_{3}+D_{2}F_{1}D_{3}\hat{F}_{3}\\
&=&\hat{R}_{1}^{(2)}\tilde{\mu}_{2}\hat{\mu}_{1}+\hat{r}_{1}\tilde{\mu}_{2}\hat{\mu}_{1}
+D_{3}D_{2}F_{1}
+D_{3}D_{2}\hat{F}_{3}
+\hat{r}_{1}\tilde{\mu}_{2}\hat{f}_{1}\left(3\right)
+\hat{r}_{1}\hat{\mu}_{1}\hat{f}_{1}\left(2\right)
+\hat{r}_{1}\hat{\mu}_{1}D_{2}F_{1}
+\hat{f}_{1}\left(3\right)D_{2}F_{1}
\end{eqnarray*}

$k=4$
\begin{eqnarray*}
D_{4}D_{2}\hat{F}_{2}&=&D_{4}D_{2}\left(\hat{R}_{3}+F_{1}+\hat{F}_{3}\right)
+D_{2}\hat{R}_{3}D_{4}\hat{F}_{3}
+D_{2}\hat{F}_{3}D_{4}\hat{R}_{3}
+D_{2}F_{1}D_{4}\left(\hat{R}_{3}+\hat{F}_{3}\right)\\
&=&D_{4}D_{2}\hat{R}_{3}+D_{4}D_{2}F_{1}+\hat{F}_{3}
+D_{2}\hat{R}_{3}D_{4}\hat{F}_{3}
+D_{2}\hat{F}_{3}D_{4}\hat{R}_{3}
+D_{2}F_{1}D_{4}\hat{R}_{3}+D_{2}F_{1}D_{4}\hat{F}_{3}\\
&=&\hat{R}_{1}^{(2)}\tilde{\mu}_{2}\hat{\mu}_{2}+\hat{r}_{1}\tilde{\mu}_{2}\hat{\mu}_{2}
+D_{4}D_{2}F_{1}
+D_{4}D_{2}\hat{F}_{3}
+\hat{r}_{1}\tilde{\mu}_{2}\hat{f}_{1}\left(4\right)
+\hat{r}_{1}\hat{\mu}_{2}\hat{f}_{1}\left(2\right)
+\hat{r}_{1}\hat{\mu}_{2}D_{2}F_{1}
+\hat{f}_{1}\left(4\right)D_{2}F_{1}
\end{eqnarray*}
%__________________________________________________________________________________________
\subsubsection*{$\hat{F}_{2}$, $i=3$}
%__________________________________________________________________________________________

$k=3$
\begin{eqnarray*}
D_{3}D_{3}\hat{F}_{2}&=&D_{3}D_{3}\left(\hat{R}_{3}+\hat{F}_{3}\right)
+D_{3}\hat{R}_{3}D_{3}\hat{F}_{3}
+D_{3}\hat{F}_{3}D_{3}\hat{R}_{3}=D_{3}^{2}\hat{R}_{3}+D_{3}^{2}\hat{F}_{3}
+D_{3}\hat{R}_{3}D_{3}\hat{F}_{3}
+D_{3}\hat{F}_{3}D_{3}\hat{R}_{3}\\
&=&\hat{R}_{1}^{(2)}\hat{\mu}_{1}^{2}+\hat{r}_{1}\hat{P}_{1}^{(2)}
+D_{3}^{2}\hat{F}_{3}
+\hat{r}_{1}\hat{\mu}_{1}\hat{f}_{1}\left(3\right)
+\hat{r}_{1}\hat{\mu}_{1}\hat{f}_{1}\left(3\right)
\end{eqnarray*}

$k=4$
\begin{eqnarray*}
D_{4}D_{3}\hat{F}_{2}&=&D_{4}D_{3}\left(\hat{R}_{3}+\hat{F}_{3}\right)
+D_{3}\hat{R}_{3}D_{4}\hat{F}_{3}
+D_{3}\hat{F}_{3}D_{4}\hat{R}_{3}=D_{4}D_{3}\hat{R}_{3}+D_{4}D_{3}\hat{F}_{3}
+D_{3}\hat{R}_{3}D_{4}\hat{F}_{3}
+D_{3}\hat{F}_{3}D_{4}\hat{R}_{3}\\
&=&\hat{R}_{1}^{(2)}\hat{\mu}_{1}\hat{\mu}_{2}+\hat{r}_{1}\hat{\mu}_{1}\hat{\mu}_{2}
+D_{4}D_{3}\hat{F}_{3}
+\hat{r}_{1}\hat{\mu}_{1}\hat{f}_{1}\left(4\right)
+\hat{r}_{1}\hat{\mu}_{2}\hat{f}_{1}\left(3\right)
\end{eqnarray*}
%__________________________________________________________________________________________
$i=4$
%__________________________________________________________________________________________

$k=4$
\begin{eqnarray*}
D_{4}D_{4}\hat{F}_{2}&=&D_{4}^{2}\left(\hat{R}_{3}+\hat{F}_{3}\right)
+D_{4}\hat{R}_{3}D_{4}\hat{F}_{3}
+D_{4}\hat{F}_{3}D_{4}\hat{R}_{3}=D_{4}^{2}\hat{R}_{3}+D_{4}^{2}\hat{F}_{3}
+D_{4}\hat{R}_{3}D_{4}\hat{F}_{3}
+D_{4}\hat{F}_{3}D_{4}\hat{R}_{3}\\
&=&\hat{R}_{1}^{(2)}\hat{\mu}_{2}^{2}+\hat{r}_{1}\hat{P}_{2}^{(2)}
+D_{4}^{2}\hat{F}_{3}
+\hat{r}_{1}\hat{\mu}_{2}\hat{f}_{1}\left(4\right)
\end{eqnarray*}
%__________________________________________________________________________________________
%

%_____________________________________________________________________________________
\newpage


%__________________________________________________________________
\section{Generalizaciones}
%__________________________________________________________________
\subsection{RSVC con dos conexiones}
%__________________________________________________________________

%\begin{figure}[H]
%\centering
%%%\includegraphics[width=9cm]{Grafica3.jpg}
%%\end{figure}\label{RSVC3}


Sus ecuaciones recursivas son de la forma


\begin{eqnarray*}
F_{1}\left(z_{1},z_{2},w_{1},w_{2}\right)&=&R_{2}\left(\prod_{i=1}^{2}\tilde{P}_{i}\left(z_{i}\right)\prod_{i=1}^{2}
\hat{P}_{i}\left(w_{i}\right)\right)F_{2}\left(z_{1},\tilde{\theta}_{2}\left(\tilde{P}_{1}\left(z_{1}\right)\hat{P}_{1}\left(w_{1}\right)\hat{P}_{2}\left(w_{2}\right)\right)\right)
\hat{F}_{2}\left(w_{1},w_{2};\tau_{2}\right),
\end{eqnarray*}

\begin{eqnarray*}
F_{2}\left(z_{1},z_{2},w_{1},w_{2}\right)&=&R_{1}\left(\prod_{i=1}^{2}\tilde{P}_{i}\left(z_{i}\right)\prod_{i=1}^{2}
\hat{P}_{i}\left(w_{i}\right)\right)F_{1}\left(\tilde{\theta}_{1}\left(\tilde{P}_{2}\left(z_{2}\right)\hat{P}_{1}\left(w_{1}\right)\hat{P}_{2}\left(w_{2}\right)\right),z_{2}\right)\hat{F}_{1}\left(w_{1},w_{2};\tau_{1}\right),
\end{eqnarray*}


\begin{eqnarray*}
\hat{F}_{1}\left(z_{1},z_{2},w_{1},w_{2}\right)&=&\hat{R}_{2}\left(\prod_{i=1}^{2}\tilde{P}_{i}\left(z_{i}\right)\prod_{i=1}^{2}
\hat{P}_{i}\left(w_{i}\right)\right)F_{2}\left(z_{1},z_{2};\zeta_{2}\right)\hat{F}_{2}\left(w_{1},\hat{\theta}_{2}\left(\tilde{P}_{1}\left(z_{1}\right)\tilde{P}_{2}\left(z_{2}\right)\hat{P}_{1}\left(w_{1}
\right)\right)\right),
\end{eqnarray*}


\begin{eqnarray*}
\hat{F}_{2}\left(z_{1},z_{2},w_{1},w_{2}\right)&=&\hat{R}_{1}\left(\prod_{i=1}^{2}\tilde{P}_{i}\left(z_{i}\right)\prod_{i=1}^{2}
\hat{P}_{i}\left(w_{i}\right)\right)F_{1}\left(z_{1},z_{2};\zeta_{1}\right)\hat{F}_{1}\left(\hat{\theta}_{1}\left(\tilde{P}_{1}\left(z_{1}\right)\tilde{P}_{2}\left(z_{2}\right)\hat{P}_{2}\left(w_{2}\right)\right),w_{2}\right),
\end{eqnarray*}

%_____________________________________________________
\subsection{First Moments of the Queue Lengths}
%_____________________________________________________


The server's switchover times are given by the general equation

\begin{eqnarray}\label{Ec.Ri}
R_{i}\left(\mathbf{z,w}\right)=R_{i}\left(\tilde{P}_{1}\left(z_{1}\right)\tilde{P}_{2}\left(z_{2}\right)\hat{P}_{1}\left(w_{1}\right)\hat{P}_{2}\left(w_{2}\right)\right)
\end{eqnarray}

with
\begin{eqnarray}\label{Ec.Derivada.Ri}
D_{i}R_{i}&=&DR_{i}D_{i}P_{i}
\end{eqnarray}
the following notation is considered

\begin{eqnarray*}
\begin{array}{llll}
D_{1}P_{1}\equiv D_{1}\tilde{P}_{1}, & D_{2}P_{2}\equiv D_{2}\tilde{P}_{2}, & D_{3}P_{3}\equiv D_{3}\hat{P}_{1}, &D_{4}P_{4}\equiv D_{4}\hat{P}_{2},
\end{array}
\end{eqnarray*}

also we need to remind $F_{1,2}\left(z_{1};\zeta_{2}\right)F_{2,2}\left(z_{2};\zeta_{2}\right)=F_{2}\left(z_{1},z_{2};\zeta_{2}\right)$, therefore

\begin{eqnarray*}
D_{1}F_{2}\left(z_{1},z_{2};\zeta_{2}\right)&=&D_{1}\left[F_{1,2}\left(z_{1};\zeta_{2}\right)F_{2,2}\left(z_{2};\zeta_{2}\right)\right]
=F_{2,2}\left(z_{2};\zeta_{2}\right)D_{1}F_{1,2}\left(z_{1};\zeta_{2}\right)=F_{1,2}^{(1)}\left(1\right)
\end{eqnarray*}

i.e., $D_{1}F_{2}=F_{1,2}^{(1)}(1)$; $D_{2}F_{2}=F_{2,2}^{(1)}\left(1\right)$, whereas that $D_{3}F_{2}=D_{4}F_{2}=0$, then

\begin{eqnarray}
\begin{array}{ccc}
D_{i}F_{j}=\indora_{i\leq2}F_{i,j}^{(1)}\left(1\right),& \textrm{ and } &D_{i}\hat{F}_{j}=\indora_{i\geq2}F_{i,j}^{(1)}\left(1\right).
\end{array}
\end{eqnarray}

Now, we obtain the first moments equations for the queue lengths as before for the times the server arrives to the queue to start attending



Remember that


\begin{eqnarray*}
F_{2}\left(z_{1},z_{2},w_{1},w_{2}\right)&=&R_{1}\left(\prod_{i=1}^{2}\tilde{P}_{i}\left(z_{i}\right)\prod_{i=1}^{2}
\hat{P}_{i}\left(w_{i}\right)\right)F_{1}\left(\tilde{\theta}_{1}\left(\tilde{P}_{2}\left(z_{2}\right)\hat{P}_{1}\left(w_{1}\right)\hat{P}_{2}\left(w_{2}\right)\right),z_{2}\right)\hat{F}_{1}\left(w_{1},w_{2};\tau_{1}\right),
\end{eqnarray*}

where


\begin{eqnarray*}
F_{1}\left(\tilde{\theta}_{1}\left(\tilde{P}_{2}\hat{P}_{1}\hat{P}_{2}\right),z_{2}\right)
\end{eqnarray*}

so

\begin{eqnarray}
D_{i}F_{1}&=&\indora_{i\neq1}D_{1}F_{1}D\tilde{\theta}_{1}D_{i}P_{i}+\indora_{i=2}D_{i}F_{1},
\end{eqnarray}

then


\begin{eqnarray*}
\begin{array}{ll}
D_{1}F_{1}=0,&
D_{2}F_{1}=D_{1}F_{1}D\tilde{\theta}_{1}D_{2}P_{2}+D_{2}F_{1}
=f_{1}\left(1\right)\frac{1}{1-\tilde{\mu}_{1}}\tilde{\mu}_{2}+f_{1}\left(2\right),\\
D_{3}F_{1}=D_{1}F_{1}D\tilde{\theta}_{1}D_{3}P_{3}
=f_{1}\left(1\right)\frac{1}{1-\tilde{\mu}_{1}}\hat{\mu}_{1},&
D_{4}F_{1}=D_{1}F_{1}D\tilde{\theta}_{1}D_{4}P_{4}
=f_{1}\left(1\right)\frac{1}{1-\tilde{\mu}_{1}}\hat{\mu}_{2}

\end{array}
\end{eqnarray*}


\begin{eqnarray}
D_{i}F_{2}&=&\indora_{i\neq2}D_{2}F_{2}D\tilde{\theta}_{2}D_{i}P_{i}
+\indora_{i=1}D_{i}F_{2}
\end{eqnarray}

\begin{eqnarray*}
\begin{array}{ll}
D_{1}F_{2}=D_{2}F_{2}D\tilde{\theta}_{2}D_{1}P_{1}
+D_{1}F_{2}=f_{2}\left(2\right)\frac{1}{1-\tilde{\mu}_{2}}\tilde{\mu}_{1},&
D_{2}F_{2}=0\\
D_{3}F_{2}=D_{2}F_{2}D\tilde{\theta}_{2}D_{3}P_{3}
=f_{2}\left(2\right)\frac{1}{1-\tilde{\mu}_{2}}\hat{\mu}_{1},&
D_{4}F_{2}=D_{2}F_{2}D\tilde{\theta}_{2}D_{4}P_{4}
=f_{2}\left(2\right)\frac{1}{1-\tilde{\mu}_{2}}\hat{\mu}_{2}
\end{array}
\end{eqnarray*}



\begin{eqnarray}
D_{i}\hat{F}_{1}&=&\indora_{i\neq3}D_{3}\hat{F}_{1}D\hat{\theta}_{1}D_{i}P_{i}+\indora_{i=4}D_{i}\hat{F}_{1},
\end{eqnarray}

\begin{eqnarray*}
\begin{array}{ll}
D_{1}\hat{F}_{1}=D_{3}\hat{F}_{1}D\hat{\theta}_{1}D_{1}P_{1}=\hat{f}_{1}\left(3\right)\frac{1}{1-\hat{\mu}_{1}}\tilde{\mu}_{1},&
D_{2}\hat{F}_{1}=D_{3}\hat{F}_{1}D\hat{\theta}_{1}D_{2}P_{2}
=\hat{f}_{1}\left(3\right)\frac{1}{1-\hat{\mu}_{1}}\tilde{\mu}_{2}\\
D_{3}\hat{F}_{1}=0,&
D_{4}\hat{F}_{1}=D_{3}\hat{F}_{1}D\hat{\theta}_{1}D_{4}P_{4}
+D_{4}\hat{F}_{1}
=\hat{f}_{1}\left(3\right)\frac{1}{1-\hat{\mu}_{1}}\hat{\mu}_{2}+\hat{f}_{1}\left(2\right),

\end{array}
\end{eqnarray*}


\begin{eqnarray}
D_{i}\hat{F}_{2}&=&\indora_{i\neq4}D_{4}\hat{F}_{2}D\hat{\theta}_{2}D_{i}P_{i}+\indora_{i=3}D_{i}\hat{F}_{2}.
\end{eqnarray}

\begin{eqnarray*}
\begin{array}{ll}
D_{1}\hat{F}_{2}=D_{4}\hat{F}_{2}D\hat{\theta}_{2}D_{1}P_{1}
=\hat{f}_{2}\left(4\right)\frac{1}{1-\hat{\mu}_{2}}\tilde{\mu}_{1},&
D_{2}\hat{F}_{2}=D_{4}\hat{F}_{2}D\hat{\theta}_{2}D_{2}P_{2}
=\hat{f}_{2}\left(4\right)\frac{1}{1-\hat{\mu}_{2}}\tilde{\mu}_{2},\\
D_{3}\hat{F}_{2}=D_{4}\hat{F}_{2}D\hat{\theta}_{2}D_{3}P_{3}+D_{3}\hat{F}_{2}
=\hat{f}_{2}\left(4\right)\frac{1}{1-\hat{\mu}_{2}}\hat{\mu}_{1}+\hat{f}_{2}\left(4\right)\\
D_{4}\hat{F}_{2}=0

\end{array}
\end{eqnarray*}
Then, now we can obtain the linear system of equations in order to obtain the first moments of the lengths of the queues:



For $\mathbf{F}_{1}=R_{2}F_{2}\hat{F}_{2}$ we get the general equations

\begin{eqnarray}
D_{i}\mathbf{F}_{1}=D_{i}\left(R_{2}+F_{2}+\indora_{i\geq3}\hat{F}_{2}\right)
\end{eqnarray}

So

\begin{eqnarray*}
D_{1}\mathbf{F}_{1}&=&D_{1}R_{2}+D_{1}F_{2}
=r_{1}\tilde{\mu}_{1}+f_{2}\left(2\right)\frac{1}{1-\tilde{\mu}_{2}}\tilde{\mu}_{1}\\
D_{2}\mathbf{F}_{1}&=&D_{2}\left(R_{2}+F_{2}\right)
=r_{2}\tilde{\mu}_{1}\\
D_{3}\mathbf{F}_{1}&=&D_{3}\left(R_{2}+F_{2}+\hat{F}_{2}\right)
=r_{1}\hat{\mu}_{1}+f_{2}\left(2\right)\frac{1}{1-\tilde{\mu}_{2}}\hat{\mu}_{1}+\hat{F}_{1,2}^{(1)}\left(1\right)\\
D_{4}\mathbf{F}_{1}&=&D_{4}\left(R_{2}+F_{2}+\hat{F}_{2}\right)
=r_{2}\hat{\mu}_{2}+f_{2}\left(2\right)\frac{1}{1-\tilde{\mu}_{2}}\hat{\mu}_{2}
+\hat{F}_{2,2}^{(1)}\left(1\right)
\end{eqnarray*}

it means

\begin{eqnarray*}
\begin{array}{ll}
D_{1}\mathbf{F}_{1}=r_{2}\hat{\mu}_{1}+f_{2}\left(2\right)\left(\frac{1}{1-\tilde{\mu}_{2}}\right)\tilde{\mu}_{1}+f_{2}\left(1\right),&
D_{2}\mathbf{F}_{1}=r_{2}\tilde{\mu}_{2},\\
D_{3}\mathbf{F}_{1}=r_{2}\hat{\mu}_{1}+f_{2}\left(2\right)\left(\frac{1}{1-\tilde{\mu}_{2}}\right)\hat{\mu}_{1}+\hat{F}_{1,2}^{(1)}\left(1\right),&
D_{4}\mathbf{F}_{1}=r_{2}\hat{\mu}_{2}+f_{2}\left(2\right)\left(\frac{1}{1-\tilde{\mu}_{2}}\right)\hat{\mu}_{2}+\hat{F}_{2,2}^{(1)}\left(1\right),\end{array}
\end{eqnarray*}


\begin{eqnarray}
\begin{array}{ll}
\mathbf{F}_{2}=R_{1}F_{1}\hat{F}_{1}, & D_{i}\mathbf{F}_{2}=D_{i}\left(R_{1}+F_{1}+\indora_{i\geq3}\hat{F}_{1}\right)\\
\end{array}
\end{eqnarray}



equivalently


\begin{eqnarray*}
\begin{array}{ll}
D_{1}\mathbf{F}_{2}=r_{1}\tilde{\mu}_{1},&
D_{2}\mathbf{F}_{2}=r_{1}\tilde{\mu}_{2}+f_{1}\left(1\right)\left(\frac{1}{1-\tilde{\mu}_{1}}\right)\tilde{\mu}_{2}+f_{1}\left(2\right),\\
D_{3}\mathbf{F}_{2}=r_{1}\hat{\mu}_{1}+f_{1}\left(1\right)\left(\frac{1}{1-\tilde{\mu}_{1}}\right)\hat{\mu}_{1}+\hat{F}_{1,1}^{(1)}\left(1\right),&
D_{4}\mathbf{F}_{2}=r_{1}\hat{\mu}_{2}+f_{1}\left(1\right)\left(\frac{1}{1-\tilde{\mu}_{1}}\right)\hat{\mu}_{2}+\hat{F}_{2,1}^{(1)}\left(1\right),\\
\end{array}
\end{eqnarray*}



\begin{eqnarray}
\begin{array}{ll}
\hat{\mathbf{F}}_{1}=\hat{R}_{2}\hat{F}_{2}F_{2}, & D_{i}\hat{\mathbf{F}}_{1}=D_{i}\left(\hat{R}_{2}+\hat{F}_{2}+\indora_{i\leq2}F_{2}\right)\\
\end{array}
\end{eqnarray}


equivalently


\begin{eqnarray*}
\begin{array}{ll}
D_{1}\hat{\mathbf{F}}_{1}=\hat{r}_{2}\tilde{\mu}_{1}+\hat{f}_{2}\left(2\right)\left(\frac{1}{1-\hat{\mu}_{2}}\right)\tilde{\mu}_{1}+F_{1,2}^{(1)}\left(1\right),&
D_{2}\hat{\mathbf{F}}_{1}=\hat{r}_{2}\tilde{\mu}_{2}+\hat{f}_{2}\left(2\right)\left(\frac{1}{1-\hat{\mu}_{2}}\right)\tilde{\mu}_{2}+F_{2,2}^{(1)}\left(1\right),\\
D_{3}\hat{\mathbf{F}}_{1}=\hat{r}_{2}\hat{\mu}_{1}+\hat{f}_{2}\left(2\right)\left(\frac{1}{1-\hat{\mu}_{2}}\right)\hat{\mu}_{1}+\hat{f}_{2}\left(1\right),&
D_{4}\hat{\mathbf{F}}_{1}=\hat{r}_{2}\hat{\mu}_{2}
\end{array}
\end{eqnarray*}



\begin{eqnarray}
\begin{array}{ll}
\hat{\mathbf{F}}_{2}=\hat{R}_{1}\hat{F}_{1}F_{1}, & D_{i}\hat{\mathbf{F}}_{2}=D_{i}\left(\hat{R}_{1}+\hat{F}_{1}+\indora_{i\leq2}F_{1}\right)
\end{array}
\end{eqnarray}



equivalently


\begin{eqnarray*}
\begin{array}{ll}
D_{1}\hat{\mathbf{F}}_{2}=\hat{r}_{1}\tilde{\mu}_{1}+\hat{f}_{1}\left(1\right)\left(\frac{1}{1-\hat{\mu}_{1}}\right)\tilde{\mu}_{1}+F_{1,1}^{(1)}\left(1\right),&
D_{2}\hat{\mathbf{F}}_{2}=\hat{r}_{1}\mu_{2}+\hat{f}_{1}\left(1\right)\left(\frac{1}{1-\hat{\mu}_{1}}\right)\tilde{\mu}_{2}+F_{2,1}^{(1)}\left(1\right),\\
D_{3}\hat{\mathbf{F}}_{2}=\hat{r}_{1}\hat{\mu}_{1},&
D_{4}\hat{\mathbf{F}}_{2}=\hat{r}_{1}\hat{\mu}_{2}+\hat{f}_{1}\left(1\right)\left(\frac{1}{1-\hat{\mu}_{1}}\right)\hat{\mu}_{2}+\hat{f}_{1}\left(2\right),\\
\end{array}
\end{eqnarray*}





Then we have that if $\mu=\tilde{\mu}_{1}+\tilde{\mu}_{2}$, $\hat{\mu}=\hat{\mu}_{1}+\hat{\mu}_{2}$, $r=r_{1}+r_{2}$ and $\hat{r}=\hat{r}_{1}+\hat{r}_{2}$  the system's solution is given by

\begin{eqnarray*}
\begin{array}{llll}
f_{2}\left(1\right)=r_{1}\tilde{\mu}_{1},&
f_{1}\left(2\right)=r_{2}\tilde{\mu}_{2},&
\hat{f}_{1}\left(4\right)=\hat{r}_{2}\hat{\mu}_{2},&
\hat{f}_{2}\left(3\right)=\hat{r}_{1}\hat{\mu}_{1}
\end{array}
\end{eqnarray*}



it's easy to verify that

\begin{eqnarray}\label{Sist.Ec.Lineales.Doble.Traslado}
\begin{array}{ll}
f_{1}\left(1\right)=\tilde{\mu}_{1}\left(r+\frac{f_{2}\left(2\right)}{1-\tilde{\mu}_{2}}\right),& f_{1}\left(3\right)=\hat{\mu}_{1}\left(r_{2}+\frac{f_{2}\left(2\right)}{1-\tilde{\mu}_{2}}\right)+\hat{F}_{1,2}^{(1)}\left(1\right)\\
f_{1}\left(4\right)=\hat{\mu}_{2}\left(r_{2}+\frac{f_{2}\left(2\right)}{1-\tilde{\mu}_{2}}\right)+\hat{F}_{2,2}^{(1)}\left(1\right),&
f_{2}\left(2\right)=\left(r+\frac{f_{1}\left(1\right)}{1-\mu_{1}}\right)\tilde{\mu}_{2},\\
f_{2}\left(3\right)=\hat{\mu}_{1}\left(r_{1}+\frac{f_{1}\left(1\right)}{1-\tilde{\mu}_{1}}\right)+\hat{F}_{1,1}^{(1)}\left(1\right),&
f_{2}\left(4\right)=\hat{\mu}_{2}\left(r_{1}+\frac{f_{1}\left(1\right)}{1-\mu_{1}}\right)+\hat{F}_{2,1}^{(1)}\left(1\right),\\
\hat{f}_{1}\left(1\right)=\left(\hat{r}_{2}+\frac{\hat{f}_{2}\left(4\right)}{1-\hat{\mu}_{2}}\right)\tilde{\mu}_{1}+F_{1,2}^{(1)}\left(1\right),&
\hat{f}_{1}\left(2\right)=\left(\hat{r}_{2}+\frac{\hat{f}_{2}\left(4\right)}{1-\hat{\mu}_{2}}\right)\tilde{\mu}_{2}+F_{2,2}^{(1)}\left(1\right),\\
\hat{f}_{1}\left(3\right)=\left(\hat{r}+\frac{\hat{f}_{2}\left(4\right)}{1-\hat{\mu}_{2}}\right)\hat{\mu}_{1},&
\hat{f}_{2}\left(1\right)=\left(\hat{r}_{1}+\frac{\hat{f}_{1}\left(3\right)}{1-\hat{\mu}_{1}}\right)\mu_{1}+F_{1,1}^{(1)}\left(1\right),\\
\hat{f}_{2}\left(2\right)=\left(\hat{r}_{1}+\frac{\hat{f}_{1}\left(3\right)}{1-\hat{\mu}_{1}}\right)\tilde{\mu}_{2}+F_{2,1}^{(1)}\left(1\right),&
\hat{f}_{2}\left(4\right)=\left(\hat{r}+\frac{\hat{f}_{1}\left(3\right)}{1-\hat{\mu}_{1}}\right)\hat{\mu}_{2},\\
\end{array}
\end{eqnarray}

with system's solutions given by

\begin{eqnarray}
\begin{array}{ll}
f_{1}\left(1\right)=r\frac{\mu_{1}\left(1-\mu_{1}\right)}{1-\mu},&
f_{2}\left(2\right)=r\frac{\tilde{\mu}_{2}\left(1-\tilde{\mu}_{2}\right)}{1-\mu},\\
f_{1}\left(3\right)=\hat{\mu}_{1}\left(r_{2}+\frac{r\tilde{\mu}_{2}}{1-\mu}\right)+\hat{F}_{1,2}^{(1)}\left(1\right),&
f_{1}\left(4\right)=\hat{\mu}_{2}\left(r_{2}+\frac{r\tilde{\mu}_{2}}{1-\mu}\right)+\hat{F}_{2,2}^{(1)}\left(1\right),\\
f_{2}\left(3\right)=\hat{\mu}_{1}\left(r_{1}+\frac{r\mu_{1}}{1-\mu}\right)+\hat{F}_{1,1}^{(1)}\left(1\right),&
f_{2}\left(4\right)=\hat{\mu}_{2}\left(r_{1}+\frac{r\mu_{1}}{1-\mu}\right)+\hat{F}_{2,1}^{(1)}\left(1\right),\\
\hat{f}_{1}\left(1\right)=\tilde{\mu}_{1}\left(\hat{r}_{2}+\frac{\hat{r}\hat{\mu}_{2}}{1-\hat{\mu}}\right)+F_{1,2}^{(1)}\left(1\right),&
\hat{f}_{1}\left(2\right)=\tilde{\mu}_{2}\left(\hat{r}_{2}+\frac{\hat{r}\hat{\mu}_{2}}{1-\hat{\mu}}\right)+F_{2,2}^{(1)}\left(1\right),\\
\hat{f}_{2}\left(1\right)=\tilde{\mu}_{1}\left(\hat{r}_{1}+\frac{\hat{r}\hat{\mu}_{1}}{1-\hat{\mu}}\right)+F_{1,1}^{(1)}\left(1\right),&
\hat{f}_{2}\left(2\right)=\tilde{\mu}_{2}\left(\hat{r}_{1}+\frac{\hat{r}\hat{\mu}_{1}}{1-\hat{\mu}}\right)+F_{2,1}^{(1)}\left(1\right)
\end{array}
\end{eqnarray}

%_________________________________________________________________________________________________________
\subsection{General Second Order Derivatives}
%_________________________________________________________________________________________________________


Now, taking the second order derivative with respect to the equations given in (\ref{Sist.Ec.Lineales.Doble.Traslado}) we obtain it in their general form

\small{
\begin{eqnarray*}\label{Ec.Derivadas.Segundo.Orden.Doble.Transferencia}
D_{k}D_{i}F_{1}&=&D_{k}D_{i}\left(R_{2}+F_{2}+\indora_{i\geq3}\hat{F}_{4}\right)+D_{i}R_{2}D_{k}\left(F_{2}+\indora_{k\geq3}\hat{F}_{4}\right)+D_{i}F_{2}D_{k}\left(R_{2}+\indora_{k\geq3}\hat{F}_{4}\right)+\indora_{i\geq3}D_{i}\hat{F}_{4}D_{k}\left(R_{2}+F_{2}\right)\\
D_{k}D_{i}F_{2}&=&D_{k}D_{i}\left(R_{1}+F_{1}+\indora_{i\geq3}\hat{F}_{3}\right)+D_{i}R_{1}D_{k}\left(F_{1}+\indora_{k\geq3}\hat{F}_{3}\right)+D_{i}F_{1}D_{k}\left(R_{1}+\indora_{k\geq3}\hat{F}_{3}\right)+\indora_{i\geq3}D_{i}\hat{F}_{3}D_{k}\left(R_{1}+F_{1}\right)\\
D_{k}D_{i}\hat{F}_{3}&=&D_{k}D_{i}\left(\hat{R}_{4}+\indora_{i\leq2}F_{2}+\hat{F}_{4}\right)+D_{i}\hat{R}_{4}D_{k}\left(\indora_{k\leq2}F_{2}+\hat{F}_{4}\right)+D_{i}\hat{F}_{4}D_{k}\left(\hat{R}_{4}+\indora_{k\leq2}F_{2}\right)+\indora_{i\leq2}D_{i}F_{2}D_{k}\left(\hat{R}_{4}+\hat{F}_{4}\right)\\
D_{k}D_{i}\hat{F}_{4}&=&D_{k}D_{i}\left(\hat{R}_{3}+\indora_{i\leq2}F_{1}+\hat{F}_{3}\right)+D_{i}\hat{R}_{3}D_{k}\left(\indora_{k\leq2}F_{1}+\hat{F}_{3}\right)+D_{i}\hat{F}_{3}D_{k}\left(\hat{R}_{3}+\indora_{k\leq2}F_{1}\right)+\indora_{i\leq2}D_{i}F_{1}D_{k}\left(\hat{R}_{3}+\hat{F}_{3}\right)
\end{eqnarray*}}
for $i,k=1,\ldots,4$. In order to have it in an specific way we need to compute the expressions $D_{k}D_{i}\left(R_{2}+F_{2}+\indora_{i\geq3}\hat{F}_{4}\right)$

%_________________________________________________________________________________________________________
\subsubsection{Second Order Derivatives: Serve's Switchover Times}
%_________________________________________________________________________________________________________

Remind $R_{i}\left(z_{1},z_{2},w_{1},w_{2}\right)=R_{i}\left(P_{1}\left(z_{1}\right)\tilde{P}_{2}\left(z_{2}\right)
\hat{P}_{1}\left(w_{1}\right)\hat{P}_{2}\left(w_{2}\right)\right)$,  which we will write in his reduced form $R_{i}=R_{i}\left(
P_{1}\tilde{P}_{2}\hat{P}_{1}\hat{P}_{2}\right)$, and according to the notation given in \cite{Lang} we obtain

\begin{eqnarray}
D_{i}D_{i}R_{k}=D^{2}R_{k}\left(D_{i}P_{i}\right)^{2}+DR_{k}D_{i}D_{i}P_{i}
\end{eqnarray}

whereas for $i\neq j$

\begin{eqnarray}
D_{i}D_{j}R_{k}=D^{2}R_{k}D_{i}P_{i}D_{j}P_{j}+DR_{k}D_{j}P_{j}D_{i}P_{i}
\end{eqnarray}

%_________________________________________________________________________________________________________
\subsubsection{Second Order Derivatives: Queue Lengths}
%_________________________________________________________________________________________________________

Just like before the expression $F_{1}\left(\tilde{\theta}_{1}\left(\tilde{P}_{2}\left(z_{2}\right)\hat{P}_{1}\left(w_{1}\right)\hat{P}_{2}\left(w_{2}\right)\right),
z_{2}\right)$, will be denoted by $F_{1}\left(\tilde{\theta}_{1}\left(\tilde{P}_{2}\hat{P}_{1}\hat{P}_{2}\right),z_{2}\right)$, then the mixed partial derivatives are:

\begin{eqnarray*}
D_{j}D_{i}F_{1}&=&\indora_{i,j\neq1}D_{1}D_{1}F_{1}\left(D\tilde{\theta}_{1}\right)^{2}D_{i}P_{i}D_{j}P_{j}
+\indora_{i,j\neq1}D_{1}F_{1}D^{2}\tilde{\theta}_{1}D_{i}P_{i}D_{j}P_{j}
+\indora_{i,j\neq1}D_{1}F_{1}D\tilde{\theta}_{1}\left(\indora_{i=j}D_{i}^{2}P_{i}+\indora_{i\neq j}D_{i}P_{i}D_{j}P_{j}\right)\\
&+&\left(1-\indora_{i=j=3}\right)\indora_{i+j\leq6}D_{1}D_{2}F_{1}D\tilde{\theta}_{1}\left(\indora_{i\leq j}D_{j}P_{j}+\indora_{i>j}D_{i}P_{i}\right)
+\indora_{i=2}\left(D_{1}D_{2}F_{1}D\tilde{\theta}_{1}D_{i}P_{i}+D_{i}^{2}F_{1}\right)
\end{eqnarray*}


Recall the expression for $F_{1}\left(\tilde{\theta}_{1}\left(\tilde{P}_{2}\left(z_{2}\right)\hat{P}_{1}\left(w_{1}\right)\hat{P}_{2}\left(w_{2}\right)\right),
z_{2}\right)$, which is denoted by $F_{1}\left(\tilde{\theta}_{1}\left(\tilde{P}_{2}\hat{P}_{1}\hat{P}_{2}\right),z_{2}\right)$, then the mixed partial derivatives are given by

\begin{eqnarray*}
\begin{array}{llll}
D_{1}D_{1}F_{1}=0,&
D_{2}D_{1}F_{1}=0,&
D_{3}D_{1}F_{1}=0,&
D_{4}D_{1}F_{1}=0,\\
D_{1}D_{2}F_{1}=0,&
D_{1}D_{3}F_{1}=0,&
D_{1}D_{4}F_{1}=0,&
\end{array}
\end{eqnarray*}

\begin{eqnarray*}
D_{2}D_{2}F_{1}&=&D_{1}^{2}F_{1}\left(D\tilde{\theta}_{1}\right)^{2}\left(D_{2}\tilde{P}_{2}\right)^{2}
+D_{1}F_{1}D^{2}\tilde{\theta}_{1}\left(D_{2}\tilde{P}_{2}\right)^{2}
+D_{1}F_{1}D\tilde{\theta}_{1}D_{2}^{2}\tilde{P}_{2}
+D_{1}D_{2}F_{1}D\tilde{\theta}_{1}D_{2}\tilde{P}_{2}\\
&+&D_{1}D_{2}F_{1}D\tilde{\theta}_{1}D_{2}\tilde{P}_{2}+D_{2}D_{2}F_{1}\\
&=&f_{1}\left(1,1\right)\left(\frac{\tilde{\mu}_{2}}{1-\tilde{\mu}_{1}}\right)^{2}
+f_{1}\left(1\right)\tilde{\theta}_{1}^(2)\tilde{\mu}_{2}^{(2)}
+f_{1}\left(1\right)\frac{1}{1-\tilde{\mu}_{1}}\tilde{P}_{2}^{(2)}+f_{1}\left(1,2\right)\frac{\tilde{\mu}_{2}}{1-\tilde{\mu}_{1}}+f_{1}\left(1,2\right)\frac{\tilde{\mu}_{2}}{1-\tilde{\mu}_{1}}+f_{1}\left(2,2\right)
\end{eqnarray*}

\begin{eqnarray*}
D_{3}D_{2}F_{1}&=&D_{1}^{2}F_{1}\left(D\tilde{\theta}_{1}\right)^{2}D_{3}\hat{P}_{1}D_{2}\tilde{P}_{2}+D_{1}F_{1}D^{2}\tilde{\theta}_{1}D_{3}\hat{P}_{1}D_{2}\tilde{P}_{2}+D_{1}F_{1}D\tilde{\theta}_{1}D_{2}\tilde{P}_{2}D_{3}\hat{P}_{1}+D_{1}D_{2}F_{1}D\tilde{\theta}_{1}D_{3}\hat{P}_{1}\\
&=&f_{1}\left(1,1\right)\left(\frac{1}{1-\tilde{\mu}_{1}}\right)^{2}\tilde{\mu}_{2}\hat{\mu}_{1}+f_{1}\left(1\right)\tilde{\theta}_{1}^{(2)}\tilde{\mu}_{2}\hat{\mu}_{1}+f_{1}\left(1\right)\frac{\tilde{\mu}_{2}\hat{\mu}_{1}}{1-\tilde{\mu}_{1}}+f_{1}\left(1,2\right)\frac{\hat{\mu}_{1}}{1-\tilde{\mu}_{1}}
\end{eqnarray*}

\begin{eqnarray*}
D_{4}D_{2}F_{1}&=&D_{1}^{2}F_{1}\left(D\tilde{\theta}_{1}\right)^{2}D_{4}\hat{P}_{2}D_{2}\tilde{P}_{2}+D_{1}F_{1}D^{2}\tilde{\theta}_{1}D_{4}\hat{P}_{2}D_{2}\tilde{P}_{2}+D_{1}F_{1}D\tilde{\theta}_{1}D_{2}\tilde{P}_{2}D_{4}\hat{P}_{2}+D_{1}D_{2}F_{1}D\tilde{\theta}_{1}D_{4}\hat{P}_{2}\\
&=&f_{1}\left(1,1\right)\left(\frac{1}{1-\tilde{\mu}_{1}}\right)^{2}\tilde{\mu}_{2}\hat{\mu}_{2}+f_{1}\left(1\right)\tilde{\theta}_{1}^{(2)}\tilde{\mu}_{2}\hat{\mu}_{2}+f_{1}\left(1\right)\frac{\tilde{\mu}_{2}\hat{\mu}_{2}}{1-\tilde{\mu}_{1}}+f_{1}\left(1,2\right)\frac{\hat{\mu}_{2}}{1-\tilde{\mu}_{1}}
\end{eqnarray*}

\begin{eqnarray*}
D_{2}D_{3}F_{1}&=&
D_{1}^{2}F_{1}\left(D\tilde{\theta}_{1}\right)^{2}D_{2}\tilde{P}_{2}D_{3}\hat{P}_{1}
+D_{1}F_{1}D^{2}\tilde{\theta}_{1}D_{2}\tilde{P}_{2}D_{3}\hat{P}_{1}+
D_{1}F_{1}D\tilde{\theta}_{1}D_{3}\hat{P}_{1}D_{2}\tilde{P}_{2}
+D_{1}D_{2}F_{1}D\tilde{\theta}_{1}D_{3}\hat{P}_{1}\\
&=&f_{1}\left(1,1\right)\left(\frac{1}{1-\tilde{\mu}_{1}}\right)^{2}\tilde{\mu}_{2}\hat{\mu}_{1}+f_{1}\left(1\right)\tilde{\theta}_{1}^{(2)}\tilde{\mu}_{2}\hat{\mu}_{1}+f_{1}\left(1\right)\frac{\tilde{\mu}_{2}\hat{\mu}_{1}}{1-\tilde{\mu}_{1}}+f_{1}\left(1,2\right)\frac{\hat{\mu}_{1}}{1-\tilde{\mu}_{1}}
\end{eqnarray*}

\begin{eqnarray*}
D_{3}D_{3}F_{1}&=&D_{1}^{2}F_{1}\left(D\tilde{\theta}_{1}\right)^{2}\left(D_{3}\hat{P}_{1}\right)^{2}+D_{1}F_{1}D^{2}\tilde{\theta}_{1}\left(D_{3}\hat{P}_{1}\right)^{2}+D_{1}F_{1}D\tilde{\theta}_{1}D_{3}^{2}\hat{P}_{1}\\
&=&f_{1}\left(1,1\right)\left(\frac{\hat{\mu}_{1}}{1-\tilde{\mu}_{1}}\right)^{2}+f_{1}\left(1\right)\tilde{\theta}_{1}^{(2)}\hat{\mu}_{1}^{2}+f_{1}\left(1\right)\frac{\hat{\mu}_{1}^{2}}{1-\tilde{\mu}_{1}}
\end{eqnarray*}

\begin{eqnarray*}
D_{4}D_{3}F_{1}&=&D_{1}^{2}F_{1}\left(D\tilde{\theta}_{1}\right)^{2}D_{4}\hat{P}_{2}D_{3}\hat{P}_{1}+D_{1}F_{1}D^{2}\tilde{\theta}_{1}D_{4}\hat{P}_{2}D_{3}\hat{P}_{1}+D_{1}F_{1}D\tilde{\theta}_{1}D_{3}\hat{P}_{1}D_{4}\hat{P}_{2}\\
&=&f_{1}\left(1,1\right)\left(\frac{1}{1-\tilde{\mu}_{1}}\right)^{2}\hat{\mu}_{1}\hat{\mu}_{2}
+f_{1}\left(1\right)\tilde{\theta}_{1}^{2}\hat{\mu}_{2}\hat{\mu}_{1}
+f_{1}\left(1\right)\frac{\hat{\mu}_{2}\hat{\mu}_{1}}{1-\tilde{\mu}_{1}}
\end{eqnarray*}

\begin{eqnarray*}
D_{2}D_{4}F_{1}&=&D_{1}^{2}F_{1}\left(D\tilde{\theta}_{1}\right)^{2}D_{2}\tilde{P}_{2}D_{4}\hat{P}_{2}+D_{1}F_{1}D^{2}\tilde{\theta}_{1}D_{2}\tilde{P}_{2}D_{4}\hat{P}_{2}+D_{1}F_{1}D\tilde{\theta}_{1}D_{4}\hat{P}_{2}D_{2}\tilde{P}_{2}+D_{1}D_{2}F_{1}D\tilde{\theta}_{1}D_{4}\hat{P}_{2}\\
&=&f_{1}\left(1,1\right)\left(\frac{1}{1-\tilde{\mu}_{1}}\right)^{2}\hat{\mu}_{2}\tilde{\mu}_{2}
+f_{1}\left(1\right)\tilde{\theta}_{1}^{(2)}\hat{\mu}_{2}\tilde{\mu}_{2}
+f_{1}\left(1\right)\frac{\hat{\mu}_{2}\tilde{\mu}_{2}}{1-\tilde{\mu}_{1}}+f_{1}\left(1,2\right)\frac{\hat{\mu}_{2}}{1-\tilde{\mu}_{1}}
\end{eqnarray*}

\begin{eqnarray*}
D_{3}D_{4}F_{1}&=&D_{1}^{2}F_{1}\left(D\tilde{\theta}_{1}\right)^{2}D_{3}\hat{P}_{1}D_{4}\hat{P}_{2}+D_{1}F_{1}D^{2}\tilde{\theta}_{1}D_{3}\hat{P}_{1}D_{4}\hat{P}_{2}+D_{1}F_{1}D\tilde{\theta}_{1}D_{4}\hat{P}_{2}D_{3}\hat{P}_{1}\\
&=&f_{1}\left(1,1\right)\left(\frac{1}{1-\tilde{\mu}_{1}}\right)^{2}\hat{\mu}_{1}\hat{\mu}_{2}+f_{1}\left(1\right)\tilde{\theta}_{1}^{(2)}\hat{\mu}_{1}\hat{\mu}_{2}+f_{1}\left(1\right)\frac{\hat{\mu}_{1}\hat{\mu}_{2}}{1-\tilde{\mu}_{1}}
\end{eqnarray*}

\begin{eqnarray*}
D_{4}D_{4}F_{1}&=&D_{1}^{2}F_{1}\left(D\tilde{\theta}_{1}\right)^{2}\left(D_{4}\hat{P}_{2}\right)^{2}+D_{1}F_{1}D^{2}\tilde{\theta}_{1}\left(D_{4}\hat{P}_{2}\right)^{2}+D_{1}F_{1}D\tilde{\theta}_{1}D_{4}^{2}\hat{P}_{2}\\
&=&f_{1}\left(1,1\right)\left(\frac{\hat{\mu}_{2}}{1-\tilde{\mu}_{1}}\right)^{2}+f_{1}\left(1\right)\tilde{\theta}_{1}^{(2)}\hat{\mu}_{2}^{2}+f_{1}\left(1\right)\frac{1}{1-\tilde{\mu}_{1}}\hat{P}_{2}^{(2)}
\end{eqnarray*}



Meanwhile for  $F_{2}\left(z_{1},\tilde{\theta}_{2}\left(P_{1}\hat{P}_{1}\hat{P}_{2}\right)\right)$

\begin{eqnarray*}
D_{j}D_{i}F_{2}&=&\indora_{i,j\neq2}D_{2}D_{2}F_{2}\left(D\theta_{2}\right)^{2}D_{i}P_{i}D_{j}P_{j}+\indora_{i,j\neq2}D_{2}F_{2}D^{2}\theta_{2}D_{i}P_{i}D_{j}P_{j}\\
&+&\indora_{i,j\neq2}D_{2}F_{2}D\theta_{2}\left(\indora_{i=j}D_{i}^{2}P_{i}
+\indora_{i\neq j}D_{i}P_{i}D_{j}P_{j}\right)\\
&+&\left(1-\indora_{i=j=3}\right)\indora_{i+j\leq6}D_{2}D_{1}F_{2}D\theta_{2}\left(\indora_{i\leq j}D_{j}P_{j}+\indora_{i>j}D_{i}P_{i}\right)
+\indora_{i=1}\left(D_{2}D_{1}F_{2}D\theta_{2}D_{i}P_{i}+D_{i}^{2}F_{2}\right)
\end{eqnarray*}

\begin{eqnarray*}
\begin{array}{llll}
D_{2}D_{1}F_{2}=0,&
D_{2}D_{3}F_{3}=0,&
D_{2}D_{4}F_{2}=0,&\\
D_{1}D_{2}F_{2}=0,&
D_{2}D_{2}F_{2}=0,&
D_{3}D_{2}F_{2}=0,&
D_{4}D_{2}F_{2}=0\\
\end{array}
\end{eqnarray*}


\begin{eqnarray*}
D_{1}D_{1}F_{2}&=&
\left(D_{1}P_{1}\right)^{2}\left(D\tilde{\theta}_{2}\right)^{2}D_{2}^{2}F_{2}
+\left(D_{1}P_{1}\right)^{2}D^{2}\tilde{\theta}_{2}D_{2}F_{2}
+D_{1}^{2}P_{1}D\tilde{\theta}_{2}D_{2}F_{2}
+D_{1}P_{1}D\tilde{\theta}_{2}D_{2}D_{1}F_{2}\\
&+&D_{2}D_{1}F_{2}D\tilde{\theta}_{2}D_{1}P_{1}+
D_{1}^{2}F_{2}\\
&=&f_{2}\left(2\right)\frac{\tilde{P}_{1}^{(2)}}{1-\tilde{\mu}_{2}}
+f_{2}\left(2\right)\theta_{2}^{(2)}\tilde{\mu}_{1}^{2}
+f_{2}\left(2,1\right)\frac{\tilde{\mu}_{1}}{1-\tilde{\mu}_{2}}
+\left(\frac{\tilde{\mu}_{1}}{1-\tilde{\mu}_{2}}\right)^{2}f_{2}\left(2,2\right)
+\frac{\tilde{\mu}_{1}}{1-\tilde{\mu}_{2}}f_{2}\left(2,1\right)+f_{2}\left(1,1\right)
\end{eqnarray*}


\begin{eqnarray*}
D_{3}D_{1}F_{2}&=&D_{2}D_{1}F_{2}D\tilde{\theta}_{2}D_{3}\hat{P}_{1}
+D_{2}^{2}F_{2}\left(D\tilde{\theta}_{2}\right)^{2}D_{3}P_{1}D_{1}P_{1}
+D_{2}F_{2}D^{2}\tilde{\theta}_{2}D_{3}\hat{P}_{1}D_{1}P_{1}
+D_{2}F_{2}D\tilde{\theta}_{2}D_{1}P_{1}D_{3}\hat{P}_{1}\\
&=&f_{2}\left(2,1\right)\frac{\hat{\mu}_{1}}{1-\tilde{\mu}_{2}}
+f_{2}\left(2,2\right)\left(\frac{1}{1-\tilde{\mu}_{2}}\right)^{2}\tilde{\mu}_{1}\hat{\mu}_{1}
+f_{2}\left(2\right)\tilde{\theta}_{2}^{(2)}\tilde{\mu}_{1}\hat{\mu}_{1}
+f_{2}\left(2\right)\frac{\tilde{\mu}_{1}\hat{\mu}_{1}}{1-\tilde{\mu}_{2}}
\end{eqnarray*}


\begin{eqnarray*}
D_{4}D_{1}F_{2}&=&D_{2}^{2}F_{2}\left(D\tilde{\theta}_{2}\right)^{2}D_{4}P_{2}D_{1}P_{1}+D_{2}F_{2}D^{2}\tilde{\theta}_{2}D_{4}\hat{P}_{2}D_{1}P_{1}
+D_{2}F_{2}D\tilde{\theta}_{2}D_{1}P_{1}D_{4}\hat{P}_{2}+D_{2}D_{1}F_{2}D\tilde{\theta}_{2}D_{4}\hat{P}_{2}\\
&=&f_{2}\left(2,2\right)\left(\frac{1}{1-\tilde{\mu}_{2}}\right)^{2}\tilde{\mu}_{1}\hat{\mu}_{2}
+f_{2}\left(2\right)\tilde{\theta}_{2}^{(2)}\tilde{\mu}_{1}\hat{\mu}_{2}
+f_{2}\left(2\right)\frac{\tilde{\mu}_{1}\hat{\mu}_{2}}{1-\tilde{\mu}_{2}}
+f_{2}\left(2,1\right)\frac{\hat{\mu}_{2}}{1-\tilde{\mu}_{2}}
\end{eqnarray*}


\begin{eqnarray*}
D_{1}D_{3}F_{2}&=&D_{2}^{2}F_{2}\left(D\tilde{\theta}_{2}\right)^{2}D_{1}P_{1}D_{3}\hat{P}_{1}
+D_{2}F_{2}D^{2}\tilde{\theta}_{2}D_{1}P_{1}D_{3}\hat{P}_{1}
+D_{2}F_{2}D\tilde{\theta}_{2}D_{3}\hat{P}_{1}D_{1}P_{1}
+D_{2}D_{1}F_{2}D\tilde{\theta}_{2}D_{3}\hat{P}_{1}\\
&=&f_{2}\left(2,2\right)\left(\frac{1}{1-\tilde{\mu}_{2}}\right)^{2}\tilde{\mu}_{1}\hat{\mu}_{1}
+f_{2}\left(2\right)\tilde{\theta}_{2}^{(2)}\tilde{\mu}_{1}\hat{\mu}_{1}
+f_{2}\left(2\right)\frac{\tilde{\mu}_{1}\hat{\mu}_{1}}{1-\tilde{\mu}_{2}}
+f_{2}\left(2,1\right)\frac{\hat{\mu}_{1}}{1-\tilde{\mu}_{2}}
\end{eqnarray*}


\begin{eqnarray*}
D_{3}D_{3}F_{2}&=&D_{2}^{2}F_{2}\left(D\tilde{\theta}_{2}\right)^{2}\left(D_{3}\hat{P}_{1}\right)^{2}
+D_{2}F_{2}\left(D_{3}\hat{P}_{1}\right)^{2}D^{2}\tilde{\theta}_{2}
+D_{2}F_{2}D\tilde{\theta}_{2}D_{3}^{2}\hat{P}_{1}\\
&=&f_{2}\left(2,2\right)\left(\frac{1}{1-\tilde{\mu}_{2}}\right)^{2}\hat{\mu}_{1}^{2}
+f_{2}\left(2\right)\tilde{\theta}_{2}^{(2)}\hat{\mu}_{1}^{2}
+f_{2}\left(2\right)\frac{\hat{P}_{1}^{(2)}}{1-\tilde{\mu}_{2}}
\end{eqnarray*}


\begin{eqnarray*}
D_{4}D_{3}F_{2}&=&D_{2}^{2}F_{2}\left(D\tilde{\theta}_{2}\right)^{2}D_{4}\hat{P}_{2}D_{3}\hat{P}_{1}
+D_{2}F_{2}D^{2}\tilde{\theta}_{2}D_{4}\hat{P}_{2}D_{3}\hat{P}_{1}
+D_{2}F_{2}D\tilde{\theta}_{2}D_{3}\hat{P}_{1}D_{4}\hat{P}_{2}\\
&=&f_{2}\left(2,2\right)\left(\frac{1}{1-\tilde{\mu}_{2}}\right)^{2}\hat{\mu}_{1}\hat{\mu}_{2}
+f_{2}\left(2\right)\tilde{\theta}_{2}^{(2)}\hat{\mu}_{1}\hat{\mu}_{2}
+f_{2}\left(2\right)\frac{\hat{\mu}_{1}\hat{\mu}_{2}}{1-\tilde{\mu}_{2}}
\end{eqnarray*}


\begin{eqnarray*}
D_{1}D_{4}F_{2}&=&D_{2}^{2}F_{2}\left(D\tilde{\theta}_{2}\right)^{2}D_{1}P_{1}D_{4}\hat{P}_{2}
+D_{2}F_{2}D^{2}\tilde{\theta}_{2}D_{1}P_{1}D_{4}\hat{P}_{2}
+D_{2}F_{2}D\tilde{\theta}_{2}D_{4}\hat{P}_{2}D_{1}P_{1}
+D_{2}D_{1}F_{2}D\tilde{\theta}_{2}D_{4}\hat{P}_{2}\\
&=&f_{2}\left(2,2\right)\left(\frac{1}{1-\tilde{\mu}_{2}}\right)^{2}\tilde{\mu}_{1}\hat{\mu}_{2}
+f_{2}\left(2\right)\tilde{\theta}_{2}^{(2)}\tilde{\mu}_{1}\hat{\mu}_{2}
+f_{2}\left(2\right)\frac{\tilde{\mu}_{1}\hat{\mu}_{2}}{1-\tilde{\mu}_{2}}
+f_{2}\left(2,1\right)\frac{\hat{\mu}_{2}}{1-\tilde{\mu}_{2}}
\end{eqnarray*}


\begin{eqnarray*}
D_{3}D_{4}F_{2}&=&
D_{2}^{2}F_{2}\left(D\tilde{\theta}_{2}\right)^{2}D_{4}\hat{P}_{2}D_{3}\hat{P}_{1}
+D_{2}F_{2}D^{2}\tilde{\theta}_{2}D_{4}\hat{P}_{2}D_{3}\hat{P}_{1}
+D_{2}F_{2}D\tilde{\theta}_{2}D_{4}\hat{P}_{2}D_{3}\hat{P}_{1}\\
&=&f_{2}\left(2,2\right)\left(\frac{1}{1-\tilde{\mu}_{2}}\right)^{2}\hat{\mu}_{1}\hat{\mu}_{2}
+f_{2}\left(2\right)\tilde{\theta}_{2}^{(2)}\hat{\mu}_{1}\hat{\mu}_{2}
+f_{2}\left(2\right)\frac{\hat{\mu}_{1}\hat{\mu}_{2}}{1-\tilde{\mu}_{2}}
\end{eqnarray*}


\begin{eqnarray*}
D_{4}D_{4}F_{2}&=&D_{2}F_{2}D\tilde{\theta}_{2}D_{4}^{2}\hat{P}_{2}
+D_{2}F_{2}D^{2}\tilde{\theta}_{2}\left(D_{4}\hat{P}_{2}\right)^{2}
+D_{2}^{2}F_{2}\left(D\tilde{\theta}_{2}\right)^{2}\left(D_{4}\hat{P}_{2}\right)^{2}\\
&=&f_{2}\left(2,2\right)\left(\frac{\hat{\mu}_{2}}{1-\tilde{\mu}_{2}}\right)^{2}
+f_{2}\left(2\right)\tilde{\theta}_{2}^{(2)}\hat{\mu}_{2}^{2}
+f_{2}\left(2\right)\frac{\hat{P}_{2}^{(2)}}{1-\tilde{\mu}_{2}}
\end{eqnarray*}


%\newpage



%\newpage

For $\hat{F}_{1}\left(\hat{\theta}_{1}\left(P_{1}\tilde{P}_{2}\hat{P}_{2}\right),w_{2}\right)$



\begin{eqnarray*}
D_{j}D_{i}\hat{F}_{1}&=&\indora_{i,j\neq3}D_{3}D_{3}\hat{F}_{1}\left(D\hat{\theta}_{1}\right)^{2}D_{i}P_{i}D_{j}P_{j}
+\indora_{i,j\neq3}D_{3}\hat{F}_{1}D^{2}\hat{\theta}_{1}D_{i}P_{i}D_{j}P_{j}
+\indora_{i,j\neq3}D_{3}\hat{F}_{1}D\hat{\theta}_{1}\left(\indora_{i=j}D_{i}^{2}P_{i}+\indora_{i\neq j}D_{i}P_{i}D_{j}P_{j}\right)\\
&+&\indora_{i+j\geq5}D_{3}D_{4}\hat{F}_{1}D\hat{\theta}_{1}\left(\indora_{i\leq j}D_{i}P_{i}+\indora_{i>j}D_{j}P_{j}\right)
+\indora_{i=4}\left(D_{3}D_{4}\hat{F}_{1}D\hat{\theta}_{1}D_{i}P_{i}+D_{i}^{2}\hat{F}_{1}\right)
\end{eqnarray*}


\begin{eqnarray*}
\begin{array}{llll}
D_{3}D_{1}\hat{F}_{1}=0,&
D_{3}D_{2}\hat{F}_{1}=0,&
D_{1}D_{3}\hat{F}_{1}=0,&
D_{2}D_{3}\hat{F}_{1}=0\\
D_{3}D_{3}\hat{F}_{1}=0,&
D_{4}D_{3}\hat{F}_{1}=0,&
D_{3}D_{4}\hat{F}_{1}=0,&
\end{array}
\end{eqnarray*}


\begin{eqnarray*}
D_{1}D_{1}\hat{F}_{1}&=&
D_{3}^{2}\hat{F}_{1}\left(D\hat{\theta}_{1}\right)^{2}\left(D_{1}P_{1}\right)^{2}
+D_{3}\hat{F}_{1}D^{2}\hat{\theta}_{1}\left(D_{1}P_{1}\right)^{2}
+D_{3}\hat{F}_{1}D\hat{\theta}_{1}D_{1}^{2}P_{1}\\
&=&\hat{f}_{1}\left(3,3\right)\left(\frac{\tilde{\mu}_{1}}{1-\hat{\mu}_{2}}\right)^{2}
+\hat{f}_{1}\left(3\right)\frac{P_{1}^{(2)}}{1-\hat{\mu}_{1}}
+\hat{f}_{1}\left(3\right)\hat{\theta}_{1}^{(2)}\tilde{\mu}_{1}^{2}
\end{eqnarray*}


\begin{eqnarray*}
D_{2}D_{1}\hat{F}_{1}&=&
D_{3}^{2}\hat{F}_{1}\left(D\hat{\theta}_{1}\right)^{2}D_{1}P_{1}D_{2}P_{1}+
D_{3}\hat{F}_{1}D^{2}\hat{\theta}_{1}D_{1}P_{1}D_{2}P_{2}+
D_{3}\hat{F}_{1}D\hat{\theta}_{1}D_{1}P_{1}D_{2}P_{2}\\
&=&\hat{f}_{1}\left(3,3\right)\left(\frac{1}{1-\hat{\mu}_{1}}\right)^{2}\tilde{\mu}_{1}\tilde{\mu}_{2}
+\hat{f}_{1}\left(3\right)\tilde{\mu}_{1}\tilde{\mu}_{2}\hat{\theta}_{1}^{(2)}
+\hat{f}_{1}\left(3\right)\frac{\tilde{\mu}_{1}\tilde{\mu}_{2}}{1-\hat{\mu}_{1}}
\end{eqnarray*}


\begin{eqnarray*}
D_{4}D_{1}\hat{F}_{1}&=&
D_{3}D_{3}\hat{F}_{1}\left(D\hat{\theta}_{1}\right)^{2}D_{4}\hat{P}_{2}D_{1}P_{1}
+D_{3}\hat{F}_{1}D^{2}\hat{\theta}_{1}D_{1}P_{1}D_{4}\hat{P}_{2}
+D_{3}\hat{F}_{1}D\hat{\theta}_{1}D_{1}P_{1}D_{4}\hat{P}_{2}
+D_{3}D_{4}\hat{F}_{1}D\hat{\theta}_{1}D_{1}P_{1}\\
&=&\hat{f}_{1}\left(3,3\right)\left(\frac{1}{1-\hat{\mu}_{1}}\right)^{2}\tilde{\mu}_{1}\hat{\mu}_{1}
+\hat{f}_{1}\left(3\right)\hat{\theta}_{1}^{(2)}\tilde{\mu}_{1}\hat{\mu}_{2}
+\hat{f}_{1}\left(3\right)\frac{\tilde{\mu}_{1}\hat{\mu}_{2}}{1-\hat{\mu}_{1}}
+\hat{f}_{1}\left(3,4\right)\frac{\tilde{\mu}_{1}}{1-\hat{\mu}_{1}}
\end{eqnarray*}


\begin{eqnarray*}
D_{1}D_{2}\hat{F}_{1}&=&
D_{3}^{2}\hat{F}_{1}\left(D\hat{\theta}_{1}\right)^{2}D_{1}P_{1}D_{2}P_{2}
+D_{3}\hat{F}_{1}D^{2}\hat{\theta}_{1}D_{1}P_{1}D_{2}P_{2}+
D_{3}\hat{F}_{1}D\hat{\theta}_{1}D_{1}P_{1}D_{2}P_{2}\\
&=&\hat{f}_{1}\left(3,3\right)\left(\frac{1}{1-\hat{\mu}_{1}}\right)^{2}\tilde{\mu}_{1}\tilde{\mu}_{2}
+\hat{f}_{1}\left(3\right)\hat{\theta}_{1}^{(2)}\tilde{\mu}_{1}\tilde{\mu}_{2}
+\hat{f}_{1}\left(3\right)\frac{\tilde{\mu}_{1}\tilde{\mu}_{2}}{1-\hat{\mu}_{1}}
\end{eqnarray*}


\begin{eqnarray*}
D_{2}D_{2}\hat{F}_{1}&=&
D_{3}^{2}\hat{F}_{1}\left(D\hat{\theta}_{1}\right)^{2}\left(D_{2}P_{2}\right)^{2}
+D_{3}\hat{F}_{1}D^{2}\hat{\theta}_{1}\left(D_{2}P_{2}\right)^{2}+
D_{3}\hat{F}_{1}D\hat{\theta}_{1}D_{2}^{2}P_{2}\\
&=&\hat{f}_{1}\left(3,3\right)\left(\frac{\tilde{\mu}_{2}}{1-\hat{\mu}_{1}}\right)^{2}
+\hat{f}_{1}\left(3\right)\hat{\theta}_{1}^{(2)}\tilde{\mu}_{2}^{2}
+\hat{f}_{1}\left(3\right)\tilde{P}_{2}^{(2)}\frac{1}{1-\hat{\mu}_{1}}
\end{eqnarray*}


\begin{eqnarray*}
D_{4}D_{2}\hat{F}_{1}&=&
D_{3}^{2}\hat{F}_{1}\left(D\hat{\theta}_{1}\right)^{2}D_{4}\hat{P}_{2}D_{2}P_{2}
+D_{3}\hat{F}_{1}D^{2}\hat{\theta}_{1}D_{2}P_{2}D_{4}\hat{P}_{2}
+D_{3}\hat{F}_{1}D\hat{\theta}_{1}D_{2}P_{2}D_{4}\hat{P}_{2}
+D_{3}D_{4}\hat{F}_{1}D\hat{\theta}_{1}D_{2}P_{2}\\
&=&\hat{f}_{1}\left(3,3\right)\left(\frac{1}{1-\hat{\mu}_{1}}\right)^{2}\tilde{\mu}_{2}\hat{\mu}_{2}
+\hat{f}_{1}\left(3\right)\hat{\theta}_{1}^{(2)}\tilde{\mu}_{2}\hat{\mu}_{2}
+\hat{f}_{1}\left(3\right)\frac{\tilde{\mu}_{2}\hat{\mu}_{2}}{1-\hat{\mu}_{1}}
+\hat{f}_{1}\left(3,4\right)\frac{\tilde{\mu}_{2}}{1-\hat{\mu}_{1}}
\end{eqnarray*}



\begin{eqnarray*}
D_{1}D_{4}\hat{F}_{1}&=&
D_{3}D_{3}\hat{F}_{1}\left(D\hat{\theta}_{1}\right)^{2}D_{1}P_{1}D_{4}\hat{P}_{2}
+D_{3}\hat{F}_{1}D^{2}\hat{\theta}_{1}D_{1}P_{1}D_{4}\hat{P}_{2}
+D_{3}\hat{F}_{1}D\hat{\theta}_{1}D_{1}P_{1}D_{4}\hat{P}_{2}
+D_{3}D_{4}\hat{F}_{1}D\hat{\theta}_{1}D_{1}P_{1}\\
&=&\hat{f}_{1}\left(3,3\right)\left(\frac{1}{1-\hat{\mu}_{1}}\right)^{2}\tilde{\mu}_{1}\hat{\mu}_{2}
+\hat{f}_{1}\left(3\right)\hat{\theta}_{1}^{(2)}\tilde{\mu}_{1}\hat{\mu}_{2}
+\hat{f}_{1}\left(3\right)\frac{\tilde{\mu}_{1}\hat{\mu}_{2}}{1-\hat{\mu}_{1}}
+\hat{f}_{1}\left(3,4\right)\frac{\tilde{\mu}_{1}}{1-\hat{\mu}_{1}}
\end{eqnarray*}


\begin{eqnarray*}
D_{2}D_{4}\hat{F}_{1}&=&
D_{3}^{2}\hat{F}_{1}\left(D\hat{\theta}_{1}\right)^{2}D_{2}P_{2}D_{4}\hat{P}_{2}
+D_{3}\hat{F}_{1}D^{2}\hat{\theta}_{1}D_{2}P_{2}D_{4}\hat{P}_{2}
+D_{3}\hat{F}_{1}D\hat{\theta}_{1}D_{2}P_{2}D_{4}\hat{P}_{2}
+D_{3}D_{4}\hat{F}_{1}D\hat{\theta}_{1}D_{2}P_{2}\\
&=&\hat{f}_{1}\left(3,3\right)\left(\frac{1}{1-\hat{\mu}_{1}}\right)^{2}\tilde{\mu}_{2}\hat{\mu}_{2}
+\hat{f}_{1}\left(3\right)\hat{\theta}_{1}^{(2)}\tilde{\mu}_{2}\hat{\mu}_{2}
+\hat{f}_{1}\left(3\right)\frac{\tilde{\mu}_{2}\hat{\mu}_{2}}{1-\hat{\mu}_{1}}
+\hat{f}_{1}\left(3,4\right)\frac{\tilde{\mu}_{2}}{1-\hat{\mu}_{1}}
\end{eqnarray*}



\begin{eqnarray*}
D_{4}D_{4}\hat{F}_{1}&=&
D_{3}^{2}\hat{F}_{1}\left(D\hat{\theta}_{1}\right)^{2}\left(D_{4}\hat{P}_{2}\right)^{2}
+D_{3}\hat{F}_{1}D^{2}\hat{\theta}_{1}\left(D_{4}\hat{P}_{2}\right)^{2}
+D_{3}\hat{F}_{1}D\hat{\theta}_{1}D_{4}^{2}\hat{P}_{2}
+D_{3}D_{4}\hat{F}_{1}D\hat{\theta}_{1}D_{4}\hat{P}_{2}\\
&+&D_{3}D_{4}\hat{F}_{1}D\hat{\theta}_{1}D_{4}\hat{P}_{2}
+D_{4}D_{4}\hat{F}_{1}\\
&=&\hat{f}_{1}\left(3,3\right)\left(\frac{\hat{\mu}_{2}}{1-\hat{\mu}_{1}}\right)^{2}
+\hat{f}_{1}\left(3\right)\hat{\theta}_{1}^{(2)}\hat{\mu}_{2}^{2}
+\hat{f}_{1}\left(3\right)\frac{\hat{P}_{2}^{(2)}}{1-\hat{\mu}_{1}}
+\hat{f}_{1}\left(3,4\right)\frac{\hat{\mu}_{2}}{1-\hat{\mu}_{1}}
+\hat{f}_{1}\left(3,4\right)\frac{\hat{\mu}_{2}}{1-\hat{\mu}_{1}}
+\hat{f}_{1}\left(4,4\right)
\end{eqnarray*}




Finally for $\hat{F}_{2}\left(w_{1},\hat{\theta}_{2}\left(P_{1}\tilde{P}_{2}\hat{P}_{1}\right)\right)$

\begin{eqnarray*}
D_{j}D_{i}\hat{F}_{2}&=&\indora_{i,j\neq4}D_{4}D_{4}\hat{F}_{2}\left(D\hat{\theta}_{2}\right)^{2}D_{i}P_{i}D_{j}P_{j}
+\indora_{i,j\neq4}D_{4}\hat{F}_{2}D^{2}\hat{\theta}_{2}D_{i}P_{i}D_{j}P_{j}
+\indora_{i,j\neq4}D_{4}\hat{F}_{2}D\hat{\theta}_{2}\left(\indora_{i=j}D_{i}^{2}P_{i}+\indora_{i\neq j}D_{i}P_{i}D_{j}P_{j}\right)\\
&+&\left(1-\indora_{i=j=2}\right)\indora_{i+j\geq4}D_{4}D_{3}\hat{F}_{2}D\hat{\theta}_{2}\left(\indora_{i\leq j}D_{i}P_{i}+\indora_{i>j}D_{j}P_{j}\right)
+\indora_{i=3}\left(D_{4}D_{3}\hat{F}_{2}D\hat{\theta}_{2}D_{i}P_{i}+D_{i}^{2}\hat{F}_{2}\right)
\end{eqnarray*}



\begin{eqnarray*}
\begin{array}{llll}
D_{4}D_{1}\hat{F}_{2}=0,&
D_{4}D_{2}\hat{F}_{2}=0,&
D_{4}D_{3}\hat{F}_{2}=0,&
D_{1}D_{4}\hat{F}_{2}=0\\
D_{2}D_{4}\hat{F}_{2}=0,&
D_{3}D_{4}\hat{F}_{2}=0,&
D_{4}D_{4}\hat{F}_{2}=0,&
\end{array}
\end{eqnarray*}


\begin{eqnarray*}
D_{1}D_{1}\hat{F}_{2}&=&
D_{4}^{2}\hat{F}_{2}\left(D\hat{\theta}_{2}\right)^{2}\left(D_{1}P_{1}\right)^{2}
+D_{4}\hat{F}_{2}\hat{\theta}_{2}\left(D_{1}P_{1}\right)^{2}D^{2}+
D_{4}\hat{F}_{2}D\hat{\theta}_{2}D_{1}^{2}P_{1}\\
&=&\hat{f}_{2}\left(4,4\right)\left(\frac{\tilde{\mu}_{1}}{1-\hat{\mu}_{2}}\right)^{2}
+\hat{f}_{2}\left(4\right)\hat{\theta}_{2}^{(2)}\tilde{\mu}_{1}^{2}
+\hat{f}_{2}\left(4\right)\frac{\tilde{P}_{1}^{(2)}}{1-\tilde{\mu}_{2}}
\end{eqnarray*}



\begin{eqnarray*}
D_{2}D_{1}\hat{F}_{2}&=&
D_{4}^{2}\hat{F}_{2}\left(D\hat{\theta}_{2}\right)^{2}D_{1}P_{1}D_{2}P_{2}
+D_{4}\hat{F}_{2}D^{2}\hat{\theta}_{2}D_{1}P_{1}D_{2}P_{2}
+D_{4}\hat{F}_{2}D\hat{\theta}_{2}D_{1}P_{1}D_{2}P_{2}\\
&=&\hat{f}_{2}\left(4,4\right)\left(\frac{1}{1-\hat{\mu}_{2}}\right)^{2}\tilde{\mu}_{1}\tilde{\mu}_{2}
+\hat{f}_{2}\left(4\right)\hat{\theta}_{2}^{(2)}\tilde{\mu}_{1}\tilde{\mu}_{2}
+\hat{f}_{2}\left(4\right)\frac{\tilde{\mu}_{1}\tilde{\mu}_{2}}{1-\tilde{\mu}_{2}}
\end{eqnarray*}



\begin{eqnarray*}
D_{3}D_{1}\hat{F}_{2}&=&
D_{4}^{2}\hat{F}_{2}\left(D\hat{\theta}_{2}\right)^{2}D_{1}P_{1}D_{3}\hat{P}_{1}
+D_{4}\hat{F}_{2}D^{2}\hat{\theta}_{2}D_{1}P_{1}D_{3}\hat{P}_{1}
+D_{4}\hat{F}_{2}D\hat{\theta}_{2}D_{1}P_{1}D_{3}\hat{P}_{1}
+D_{4}D_{3}\hat{F}_{2}D\hat{\theta}_{2}D_{1}P_{1}\\
&=&\hat{f}_{2}\left(4,4\right)\left(\frac{1}{1-\hat{\mu}_{2}}\right)^{2}\tilde{\mu}_{1}\hat{\mu}_{1}
+\hat{f}_{2}\left(4\right)\hat{\theta}_{2}^{(2)}\tilde{\mu}_{1}\hat{\mu}_{1}
+\hat{f}_{2}\left(4\right)\frac{\tilde{\mu}_{1}\hat{\mu}_{1}}{1-\hat{\mu}_{2}}
+\hat{f}_{2}\left(4,3\right)\frac{\tilde{\mu}_{1}}{1-\hat{\mu}_{2}}
\end{eqnarray*}



\begin{eqnarray*}
D_{1}D_{2}\hat{F}_{2}&=&
D_{4}D_{4}\hat{F}_{2}\left(D\hat{\theta}_{2}\right)^{2}D_{1}P_{1}D_{2}P_{2}
+D_{4}\hat{F}_{2}D^{2}\hat{\theta}_{2}D_{1}P_{1}D_{2}P_{2}
+D_{4}\hat{F}_{2}D\hat{\theta}_{2}D_{1}P_{1}D_{2}P_{2}
\\
&=&
\hat{f}_{2}\left(4,4\right)\left(\frac{1}{1-\hat{\mu}_{2}}\right)^{2}\tilde{\mu}_{1}\tilde{\mu}_{2}
+\hat{f}_{2}\left(4\right)\hat{\theta}_{2}^{(2)}\tilde{\mu}_{1}\tilde{\mu}_{2}
+\hat{f}_{2}\left(4\right)\frac{\tilde{\mu}_{1}\tilde{\mu}_{2}}{1-\tilde{\mu}_{2}}
\end{eqnarray*}



\begin{eqnarray*}
D_{2}D_{2}\hat{F}_{2}&=&
D_{4}^{2}\hat{F}_{2}\left(D\hat{\theta}_{2}\right)^{2}\left(D_{2}P_{2}\right)^{2}
+D_{4}\hat{F}_{2}D^{2}\hat{\theta}_{2}\left(D_{2}P_{2}\right)^{2}
+D_{4}\hat{F}_{2}D\hat{\theta}_{2}D_{2}^{2}P_{2}
\\
&=&\hat{f}_{2}\left(4,4\right)\left(\frac{\tilde{\mu}_{2}}{1-\hat{\mu}_{2}}\right)^{2}
+\hat{f}_{2}\left(4\right)\hat{\theta}_{2}^{(2)}\tilde{\mu}_{2}^{2}
+\hat{f}_{2}\left(4\right)\frac{\tilde{P}_{2}^{(2)}}{1-\hat{\mu}_{2}}
\end{eqnarray*}



\begin{eqnarray*}
D_{3}D_{2}\hat{F}_{2}&=&
D_{4}^{2}\hat{F}_{2}\left(D\hat{\theta}_{2}\right)^{2}D_{2}P_{2}D_{3}\hat{P}_{1}
+D_{4}\hat{F}_{2} D^{2}\hat{\theta}_{2}D_{2}P_{2}D_{3}\hat{P}_{1}
+D_{4}\hat{F}_{2}D\hat{\theta} _{2}D_{2}P_{2}D_{3}\hat{P}_{1}
+D_{4}D_{3}\hat{F}_{2}D\hat{\theta}_{2}D_{2}P_{2}\\
&=&
\hat{f}_{2}\left(4,4\right)\left(\frac{1}{1-\hat{\mu}_{2}}\right)^{2}\tilde{\mu}_{2}\hat{\mu}_{1}
+\hat{f}_{2}\left(4\right)\hat{\theta}_{2}^{(2)}\tilde{\mu}_{2}\hat{\mu}_{1}
+\hat{f}_{2}\left(4\right)\frac{\tilde{\mu}_{2}\hat{\mu}_{1}}{1-\hat{\mu}_{2}}
+\hat{f}_{2}\left(4,3\right)\frac{\tilde{\mu}_{2}}{1-\hat{\mu}_{2}}
\end{eqnarray*}



\begin{eqnarray*}
D_{1}D_{3}\hat{F}_{2}&=&
D_{4}D_{4}\hat{F}_{2}\left(D\hat{\theta}_{2}\right)^{2}D_{1}P_{1}D_{3}\hat{P}_{1}
+D_{4}\hat{F}_{2}D^{2}\hat{\theta}_{2}D_{1}P_{1}D_{3}\hat{P}_{1}
+D_{4}\hat{F}_{2}D\hat{\theta}_{2}D_{1}P_{1}D_{3}\hat{P}_{1}
+D_{4}D_{3}\hat{F}_{2}D\hat{\theta}_{2}D_{1}P_{1}\\
&=&
\hat{f}_{2}\left(4,4\right)\left(\frac{1}{1-\hat{\mu}_{2}}\right)^{2}\tilde{\mu}_{1}\hat{\mu}_{1}
+\hat{f}_{2}\left(4\right)\hat{\theta}_{2}^{(2)}\tilde{\mu}_{1}\hat{\mu}_{1}
+\hat{f}_{2}\left(4\right)\frac{\tilde{\mu}_{1}\hat{\mu}_{1}}{1-\hat{\mu}_{2}}
+\hat{f}_{2}\left(4,3\right)\frac{\tilde{\mu}_{1}}{1-\hat{\mu}_{2}}
\end{eqnarray*}



\begin{eqnarray*}
D_{2}D_{3}\hat{F}_{2}&=&
D_{4}^{2}\hat{F}_{2}\left(D\hat{\theta}_{2}\right)^{2}D_{2}P_{2}D_{3}\hat{P}_{1}
+D_{4}\hat{F}_{2}D^{2}\hat{\theta}_{2}D_{2}P_{2}D_{3}\hat{P}_{1}
+D_{4}\hat{F}_{2}D\hat{\theta}_{2}D_{2}P_{2}D_{3}\hat{P}_{1}
+D_{4}D_{3}\hat{F}_{2}D\hat{\theta}_{2}D_{2}P_{2}\\
&=&
\hat{f}_{2}\left(4,4\right)\left(\frac{1}{1-\hat{\mu}_{2}}\right)^{2}\tilde{\mu}_{2}\hat{\mu}_{1}
+\hat{f}_{2}\left(4\right)\hat{\theta}_{2}^{(2)}\tilde{\mu}_{2}\hat{\mu}_{1}
+\hat{f}_{2}\left(4\right)\frac{\tilde{\mu}_{2}\hat{\mu}_{1}}{1-\hat{\mu}_{2}}
+\hat{f}_{2}\left(4,3\right)\frac{\tilde{\mu}_{2}}{1-\hat{\mu}_{2}}
\end{eqnarray*}



\begin{eqnarray*}
D_{3}D_{3}\hat{F}_{2}&=&
D_{4}^{2}\hat{F}_{2}\left(D\hat{\theta}_{2}\right)^{2}\left(D_{3}\hat{P}_{1}\right)^{2}
+D_{4}\hat{F}_{2}D^{2}\hat{\theta}_{2}\left(D_{3}\hat{P}_{1}\right)^{2}
+D_{4}\hat{F}_{2}D\hat{\theta}_{2}D_{3}^{2}\hat{P}_{1}
+D_{4}D_{3}\hat{F}_{2}D\hat{\theta}_{2}D_{3}\hat{P}_{1}\\
&+&D_{4}D_{3}\hat{f}_{2}D\hat{\theta}_{2}D_{3}\hat{P}_{1}
+D_{3}^{2}\hat{F}_{2}\\
&=&
\hat{f}_{2}\left(4,4\right)\left(\frac{\hat{\mu}_{1}}{1-\hat{\mu}_{2}}\right)^{2}
+\hat{f}_{2}\left(4\right)\hat{\theta}_{2}^{(2)}\hat{\mu}_{1}^{2}
+\hat{f}_{2}\left(4\right)\frac{\hat{P}_{1}^{(2)}}{1-\hat{\mu}_{2}}
+\hat{f}_{2}\left(4,3\right)\frac{\hat{\mu}_{1}}{1-\hat{\mu}_{2}}
+\hat{f}_{2}\left(4,3\right)\frac{\tilde{\mu}_{1}}{1-\hat{\mu}_{2}}
+\hat{f}_{2}\left(3,3\right)
\end{eqnarray*}

%_____________________________________________________________
\subsection*{Second Grade Derivative Recursive Equations}
%_____________________________________________________________


Then according to the equations given at the beginning of this section, we have

\begin{eqnarray*}
D_{k}D_{i}F_{1}&=&D_{k}D_{i}\left(R_{2}+F_{2}+\indora_{i\geq3}\hat{F}_{4}\right)+D_{i}R_{2}D_{k}\left(F_{2}+\indora_{k\geq3}\hat{F}_{4}\right)\\&+&D_{i}F_{2}D_{k}\left(R_{2}+\indora_{k\geq3}\hat{F}_{4}\right)+\indora_{i\geq3}D_{i}\hat{F}_{4}D_{k}\left(R_{2}+F_{2}\right)
\end{eqnarray*}


%_____________________________________________________________
\subsubsection*{$F_{1}$ and $i=1$}
%_____________________________________________________________

for $i=1$, and $k=1$

\begin{eqnarray*}
D_{1}D_{1}F_{1}&=&D_{1}D_{1}\left(R_{2}+F_{2}\right)+D_{1}R_{2}D_{1}F_{2}
+D_{1}F_{2}D_{1}R_{2}
=D_{1}^{2}R_{2}
+D_{1}^{2}F_{2}
+D_{1}R_{2}D_{1}F_{2}
+D_{1}F_{2}D_{1}R_{2}\\
&=&R_{2}^{(2)}\tilde{\mu}_{1}+r_{2}\tilde{P}_{1}^{(2)}
+D_{1}^{2}F_{2}
+2r_{2}\tilde{\mu}_{1}f_{2}\left(1\right)
\end{eqnarray*}

$k=2$
\begin{eqnarray*}
D_{2}D_{i}F_{1}&=&D_{2}D_{1}\left(R_{2}+F_{2}\right)
+D_{1}R_{2}D_{2}F_{2}+D_{1}F_{2}D_{2}R_{2}
=D_{2}D_{1}R_{2}
+D_{2}D_{1}F_{2}
+D_{1}R_{2}D_{2}F_{2}
+D_{1}F_{2}D_{2}R_{2}\\
&=&R_{2}^{(2)}\tilde{\mu}_{1}\tilde{\mu}_{2}+r_{2}\tilde{\mu}_{1}\tilde{\mu}_{2}
+D_{2}D_{1}F_{2}
+r_{2}\tilde{\mu}_{1}f_{2}\left(2\right)
+r_{2}\tilde{\mu}_{2}f_{2}\left(1\right)
\end{eqnarray*}

$k=3$
\begin{eqnarray*}
D_{3}D_{1}F_{1}&=&D_{3}D_{1}\left(R_{2}+F_{2}\right)
+D_{1}R_{2}D_{3}\left(F_{2}+\hat{F}_{4}\right)
+D_{1}F_{2}D_{3}\left(R_{2}+\hat{F}_{4}\right)\\
&=&D_{3}D_{1}R_{2}+D_{3}D_{1}F_{2}
+D_{1}R_{2}D_{3}F_{2}+D_{1}R_{2}D_{3}\hat{F}_{4}
+D_{1}F_{2}D_{3}R_{2}+D_{1}F_{2}D_{3}\hat{F}_{4}\\
&=&R_{2}^{(2)}\tilde{\mu}_{1}\hat{\mu}_{1}+r_{2}\tilde{\mu}_{1}\hat{\mu}_{1}
+D_{3}D_{1}F_{2}
+r_{2}\tilde{\mu}_{1}f_{2}\left(3\right)
+r_{2}\tilde{\mu}_{1}D_{3}\hat{F}_{4}
+r_{2}\hat{\mu}_{1}f_{2}\left(1\right)
+D_{3}\hat{F}_{4}f_{2}\left(1\right)
\end{eqnarray*}

$k=4$
\begin{eqnarray*}
D_{4}D_{1}F_{1}&=&D_{4}D_{1}\left(R_{2}+F_{2}\right)
+D_{1}R_{2}D_{4}\left(F_{2}+\hat{F}_{4}\right)
+D_{1}F_{2}D_{4}\left(R_{2}+\hat{F}_{4}\right)\\
&=&D_{4}D_{1}R_{2}+D_{4}D_{1}F_{2}
+D_{1}R_{2}D_{4}F_{2}+D_{1}R_{2}D_{4}\hat{F}_{4}
+D_{1}F_{2}D_{4}R_{2}+D_{1}F_{2}D_{4}\hat{F}_{4}\\
&=&R_{2}^{(2)}\tilde{\mu}_{1}\hat{\mu}_{2}+r_{2}\tilde{\mu}_{1}\hat{\mu}_{2}
+D_{4}D_{1}F_{2}
+r_{2}\tilde{\mu}_{1}f_{2}\left(4\right)
+r_{2}\tilde{\mu}_{1}D_{4}\hat{F}_{4}
+r_{2}\hat{\mu}_{2}f_{2}\left(1\right)
+f_{2}\left(1\right)D_{4}\hat{F}_{4}
\end{eqnarray*}


%_____________________________________________________________
\subsubsection*{$F_{1}$ and $i=2$}
%_____________________________________________________________

for $i=2$, and $k=1$

\begin{eqnarray*}
D_{1}D_{2}F_{1}&=&D_{1}D_{2}\left(R_{2}+F_{2}\right)
+D_{2}R_{2}D_{1}F_{2}+D_{2}F_{2}D_{1}R_{2}=
D_{1}D_{2}R_{2}+D_{1}D_{2}F_{2}
+D_{2}R_{2}D_{1}F_{2}+D_{2}F_{2}D_{1}R_{2}\\
&=&R_{2}^{(2)}\tilde{\mu}_{1}\tilde{\mu}_{2}+r_{2}\tilde{\mu}_{1}\tilde{\mu}_{2}
+D_{1}D_{2}F_{2}
+r_{2}\tilde{\mu}_{2}f_{2}\left(1\right)
+r_{2}\tilde{\mu}_{1}f_{2}\left(2\right)
\end{eqnarray*}

$k=2$
\begin{eqnarray*}
D_{2}D_{2}F_{1}&=&D_{2}D_{2}\left(R_{2}+F_{2}\right)
+D_{2}R_{2}D_{2}F_{2}+D_{2}F_{2}D_{2}R_{2}
=D_{2}D_{2}R_{2}+D_{2}D_{2}F_{2}+D_{2}R_{2}D_{2}F_{2}+D_{2}F_{2}D_{2}R_{2}\\
&=&R_{2}^{(2)}\tilde{\mu}_{2}^{2}+r_{2}\tilde{P}_{2}^{(2)}
+D_{2}D_{2}F_{2}
+2r_{2}\tilde{\mu}_{2}f_{2}\left(2\right)
\end{eqnarray*}

$k=3$
\begin{eqnarray*}
D_{3}D_{2}F_{1}&=&D_{3}D_{2}\left(R_{2}+F_{2}\right)
+D_{2}R_{2}D_{3}\left(F_{2}+\hat{F}_{4}\right)
+D_{2}F_{2}D_{3}\left(R_{2}+\hat{F}_{4}\right)\\
&=&D_{3}D_{2}R_{2}+D_{3}D_{2}F_{2}
+D_{2}R_{2}D_{3}F_{2}+D_{2}R_{2}D_{3}\hat{F}_{4}
+D_{2}F_{2}D_{3}R_{2}+D_{2}F_{2}D_{3}\hat{F}_{4}\\
&=&R_{2}^{(2)}\tilde{\mu}_{2}\hat{\mu}_{1}+r_{2}\tilde{\mu}_{2}\hat{\mu}_{1}
+D_{3}D_{2}F_{2}
+r_{2}\tilde{\mu}_{2}f_{2}\left(3\right)
+r_{2}\tilde{\mu}_{2}D_{3}\hat{F}_{4}
+r_{2}\hat{\mu}_{1}f_{2}\left(2\right)
+f_{2}\left(2\right)D_{3}\hat{F}_{4}
\end{eqnarray*}

$k=4$
\begin{eqnarray*}
D_{4}D_{2}F_{1}&=&D_{4}D_{2}\left(R_{2}+F_{2}\right)
+D_{2}R_{2}D_{4}\left(F_{2}+\hat{F}_{4}\right)
+D_{2}F_{2}D_{4}\left(R_{2}+\hat{F}_{4}\right)\\
&=&D_{4}D_{2}R_{2}+D_{4}D_{2}F_{2}
+D_{2}R_{2}D_{4}F_{2}+D_{2}R_{2}D_{4}\hat{F}_{4}
+D_{2}F_{2}D_{4}R_{2}+D_{2}F_{2}D_{4}\hat{F}_{4}\\
&=&R_{2}^{(2)}\tilde{\mu}_{2}\hat{\mu}_{2}+r_{2}\tilde{\mu}_{2}\hat{\mu}_{2}
+D_{4}D_{2}F_{2}
+r_{2}\tilde{\mu}_{2}f_{2}\left(4\right)
+r_{2}\tilde{\mu}_{2}D_{4}\hat{F}_{4}
+r_{2}\hat{\mu}_{2}f_{2}\left(2\right)
+f_{2}\left(2\right)D_{4}\hat{F}_{4}
\end{eqnarray*}

%_____________________________________________________________
\subsubsection*{$F_{1}$ and $i=3$}
%_____________________________________________________________
for $i=3$, and $k=1$

\begin{eqnarray*}
D_{1}D_{3}F_{1}&=&D_{1}D_{3}\left(R_{2}+F_{2}+\hat{F}_{4}\right)
+D_{3}R_{2}D_{1}F_{2}+D_{3}F_{2}D_{1}R_{2}
+D_{3}\hat{F}_{4}D_{1}\left(R_{2}+F_{2}\right)\\
&=&D_{1}D_{3}R_{2}+D_{1}D_{3}F_{2}+D_{1}D_{3}\hat{F}_{4}
+D_{3}R_{2}D_{1}F_{2}+D_{3}F_{2}D_{1}R_{2}
+D_{3}\hat{F}_{4}D_{1}R_{2}+D_{3}\hat{F}_{4}D_{1}F_{2}\\
&=&R_{2}^{(2)}\tilde{\mu}_{1}\hat{\mu}_{1}+r_{2}\tilde{\mu}_{1}\hat{\mu}_{1}
+D_{1}D_{3}F_{2}
+D_{1}D_{3}\hat{F}_{4}
+r_{2}\hat{\mu}_{1}f_{2}\left(1\right)
+r_{2}\tilde{\mu}_{1}f_{2}\left(3\right)
+r_{2}\tilde{\mu}_{1}D_{3}\hat{F}_{4}
+D_{3}\hat{F}_{4}f_{2}\left(1\right)
\end{eqnarray*}

$k=2$
\begin{eqnarray*}
D_{2}D_{3}F_{1}&=&D_{2}D_{3}\left(R_{2}+F_{2}+\hat{F}_{4}\right)
+D_{3}R_{2}D_{2}F_{2}
+D_{3}F_{2}D_{2}R_{2}
+D_{3}\hat{F}_{4}D_{2}\left(R_{2}+F_{2}\right)\\
&=&D_{2}D_{3}R_{2}+D_{2}D_{3}F_{2}+D_{2}D_{3}\hat{F}_{4}
+D_{3}R_{2}D_{2}F_{2}+D_{3}F_{2}D_{2}R_{2}
+D_{3}\hat{F}_{4}D_{2}R_{2}+D_{3}\hat{F}_{4}D_{2}F_{2}\\
&=&R_{2}^{(2)}\tilde{\mu}_{2}\hat{\mu}_{1}+r_{2}\tilde{\mu}_{2}\hat{\mu}_{1}
+D_{2}D_{3}F_{2}
+D_{2}D_{3}\hat{F}_{4}
+r_{2}\hat{\mu}_{1}f_{2}\left(2\right)
+r_{2}\tilde{\mu}_{2}f_{2}\left(3\right)
+r_{2}\tilde{\mu}_{2}D_{3}\hat{F}_{4}
+f_{2}\left(4\right)D_{3}\hat{F}_{4}
\end{eqnarray*}

$k=3$
\begin{eqnarray*}
D_{3}D_{3}F_{1}&=&D_{3}D_{3}\left(R_{2}+F_{2}+\hat{F}_{4}\right)
+D_{3}R_{2}D_{3}\left(F_{2}+\hat{F}_{4}\right)
+D_{3}F_{2}D_{3}\left(R_{2}+\hat{F}_{4}\right)
+D_{3}\hat{F}_{4}D_{3}\left(R_{2}+F_{2}\right)\\
&=&D_{3}D_{3}R_{2}+D_{3}D_{3}F_{2}+D_{3}D_{3}\hat{F}_{4}
+D_{3}R_{2}D_{3}F_{2}+D_{3}R_{2}D_{3}\hat{F}_{4}\\
&+&D_{3}F_{2}D_{3}R_{2}+D_{3}F_{2}D_{3}\hat{F}_{4}
+D_{3}\hat{F}_{4}D_{3}R_{2}+D_{3}\hat{F}_{4}D_{3}F_{2}\\
&=&R_{2}^{(2)}\hat{\mu}_{1}^{2}+r_{2}\hat{P}_{1}^{(2)}
+D_{3}D_{3}F_{2}
+D_{3}D_{3}\hat{F}_{4}
+r_{2}\hat{\mu}_{1}f_{2}\left(3\right)
+r_{2}\hat{\mu}_{1}D_{3}\hat{F}_{4}\\
&+&r_{2}\hat{\mu}_{1}f_{2}\left(3\right)
+f_{2}\left(3\right)D_{3}\hat{F}_{4}
+r_{2}\hat{\mu}_{1}D_{3}\hat{F}_{4}
+f_{2}\left(3\right)D_{3}\hat{F}_{4}
\end{eqnarray*}

$k=4$
\begin{eqnarray*}
D_{4}D_{3}F_{1}&=&D_{4}D_{3}\left(R_{2}+F_{2}+\hat{F}_{4}\right)
+D_{3}R_{2}D_{4}\left(F_{2}+\hat{F}_{4}\right)
+D_{3}F_{2}D_{4}\left(R_{2}+\hat{F}_{4}\right)
+D_{3}\hat{F}_{4}D_{4}\left(R_{2}+F_{2}\right)\\
&=&D_{4}D_{3}R_{2}+D_{4}D_{3}F_{2}+D_{4}D_{3}\hat{F}_{4}
+D_{3}R_{2}D_{4}F_{2}+D_{3}R_{2}D_{4}\hat{F}_{4}\\
&+&D_{3}F_{2}D_{4}R_{2}+D_{3}F_{2}D_{4}\hat{F}_{4}
+D_{3}\hat{F}_{4}D_{4}R_{2}+D_{3}\hat{F}_{4}D_{4}F_{2}\\
&=&R_{2}^{(2)}\hat{\mu}_{1}\hat{\mu}_{2}+r_{2}\hat{\mu}_{1}\hat{\mu}_{2}
+D_{4}D_{3}F_{2}
+D_{4}D_{3}\hat{F}_{4}
+r_{2}\hat{\mu}_{1}f_{2}\left(4\right)
+r_{2}\hat{\mu}_{1}D_{4}\hat{F}_{4}\\
&+&r_{2}\hat{\mu}_{2}f_{2}\left(3\right)
+D_{4}\hat{F}_{4}f_{2}\left(3\right)
+D_{3}\hat{F}_{4}r_{2}\hat{\mu}_{2}
+D_{3}\hat{F}_{4}f_{2}\left(4\right)
\end{eqnarray*}

%_____________________________________________________________
\subsubsection*{$F_{1}$ and $i=4$}
%_____________________________________________________________

for $i=4$, and $k=1$


\begin{eqnarray*}
D_{1}D_{4}F_{1}&=&D_{1}D_{4}\left(R_{2}+F_{2}+\hat{F}_{4}\right)
+D_{4}R_{2}D_{1}F_{2}
+D_{4}F_{2}D_{1}R_{2}
+D_{4}\hat{F}_{4}D_{1}\left(R_{2}+F_{2}\right)\\
&=&D_{1}D_{4}R_{2}+D_{1}D_{4}F_{2}+D_{1}D_{4}\hat{F}_{4}
+D_{4}R_{2}D_{1}F_{2}+D_{4}F_{2}D_{1}R_{2}
+D_{4}\hat{F}_{4}D_{1}R_{2}+D_{4}\hat{F}_{4}D_{1}F_{2}\\
\end{eqnarray*}

$k=2$
\begin{eqnarray*}
D_{2}D_{4}F_{1}&=&D_{2}D_{4}\left(R_{2}+F_{2}+\hat{F}_{4}\right)
+D_{4}R_{2}D_{2}F_{2}+D_{4}F_{2}D_{2}R_{2}
+D_{4}\hat{F}_{4}D_{2}\left(R_{2}+F_{2}\right)\\
&=&D_{2}D_{4}R_{2}+D_{2}D_{4}F_{2}+D_{2}D_{4}\hat{F}_{4}
+D_{4}R_{2}D_{2}F_{2}+D_{4}F_{2}D_{2}R_{2}
+D_{4}\hat{F}_{4}D_{2}R_{2}+D_{4}\hat{F}_{4}D_{2}F_{2}
\end{eqnarray*}

$k=3$
\begin{eqnarray*}
D_{3}D_{4}F_{1}&=&D_{3}D_{4}\left(R_{2}+F_{2}+\hat{F}_{4}\right)
+D_{4}R_{2}D_{3}\left(F_{2}+\hat{F}_{4}\right)
+D_{4}F_{2}D_{3}\left(R_{2}+\hat{F}_{4}\right)
+D_{4}\hat{F}_{4}D_{3}\left(R_{2}+F_{2}\right)\\
&=&D_{3}D_{4}R_{2}+D_{3}D_{4}F_{2}+D_{3}D_{4}\hat{F}_{4}
+D_{4}R_{2}D_{3}F_{2}+D_{4}R_{2}D_{3}\hat{F}_{4}\\
&+&D_{4}F_{2}D_{3}R_{2}+D_{4}F_{2}D_{3}\hat{F}_{4}
+D_{4}\hat{F}_{4}D_{3}R_{2}+D_{4}\hat{F}_{4}D_{3}F_{2}
\end{eqnarray*}


$k=4$
\begin{eqnarray*}
D_{4}D_{4}F_{1}&=&D_{4}D_{4}\left(R_{2}+F_{2}+\hat{F}_{4}\right)
+D_{4}R_{2}D_{4}\left(F_{2}+\hat{F}_{4}\right)
+D_{4}F_{2}D_{4}\left(R_{2}+\hat{F}_{4}\right)
+D_{4}\hat{F}_{4}D_{4}\left(R_{2}+F_{2}\right)\\
&=&D_{4}D_{4}R_{2}+D_{4}D_{4}F_{2}+D_{4}D_{4}\hat{F}_{4}
+D_{4}R_{2}D_{4}F_{2}+D_{4}R_{2}D_{4}\hat{F}_{4}\\
&+&D_{4}F_{2}D_{4}R_{2}+D_{4}F_{2}D_{4}\hat{F}_{4}
+D_{4}\hat{F}_{4}D_{4}R_{2}+D_{4}\hat{F}_{4}D_{4}F_{2}
\end{eqnarray*}

%__________________________________________________________________________________________
%
%__________________________________________________________________________________________

\begin{eqnarray}
D_{k}D_{i}F_{2}&=&D_{k}D_{i}\left(R_{1}+F_{1}+\indora_{i\geq3}\hat{F}_{3}\right)+D_{i}R_{1}D_{k}\left(F_{1}+\indora_{k\geq3}\hat{F}_{3}\right)+D_{i}F_{1}D_{k}\left(R_{1}+\indora_{k\geq3}\hat{F}_{3}\right)+\indora_{i\geq3}D_{i}\hat{F}_{3}D_{k}\left(R_{1}+F_{1}\right)
\end{eqnarray}

%__________________________________________________________________________________________
$i=1$
%__________________________________________________________________________________________
$k=1$
\begin{eqnarray*}
D_{1}D_{1}F_{2}&=&D_{1}D_{1}\left(R_{1}+F_{1}\right)+D_{1}R_{1}D_{1}F_{1}+D_{1}F_{1}D_{1}R_{1}=D_{1}^{2}R_{1}+D_{1}^{2}F_{1}+D_{1}R_{1}D_{1}F_{1}+D_{1}F_{1}D_{1}R_{1}
\end{eqnarray*}

$k=2$
\begin{eqnarray*}
D_{2}D_{1}F_{2}&=&D_{2}D_{1}\left(R_{1}+F_{1}\right)+D_{1}R_{1}D_{2}F_{1}+D_{1}F_{1}D_{2}R_{1}=
D_{2}D_{1}R_{1}+D_{2}D_{1}F_{1}+D_{1}R_{1}D_{2}F_{1}+D_{1}F_{1}D_{2}R_{1}
\end{eqnarray*}

$k=3$
\begin{eqnarray*}
D_{3}D_{1}F_{2}&=&D_{3}D_{1}\left(R_{1}+F_{1}\right)+D_{1}R_{1}D_{3}\left(F_{1}+\hat{F}_{3}\right)+D_{1}F_{1}D_{3}\left(R_{1}+\hat{F}_{3}\right)\\
&=&D_{3}D_{1}R_{1}+D_{3}D_{1}F_{1}+D_{1}R_{1}D_{3}F_{1}+D_{1}R_{1}D_{3}\hat{F}_{3}+D_{1}F_{1}D_{3}R_{1}+D_{1}F_{1}D_{3}\hat{F}_{3}
\end{eqnarray*}

$k=4$
\begin{eqnarray*}
D_{4}D_{1}F_{2}&=&D_{4}D_{1}\left(R_{1}+F_{1}\right)+D_{1}R_{1}D_{4}\left(F_{1}+\hat{F}_{3}\right)+D_{1}F_{1}D_{4}\left(R_{1}+\hat{F}_{3}\right)\\
&=&D_{4}D_{1}R_{1}+D_{4}D_{1}F_{1}+D_{1}R_{1}D_{4}F_{1}+D_{1}R_{1}D_{4}\hat{F}_{3}
+D_{1}F_{1}D_{4}R_{1}+D_{1}F_{1}D_{4}\hat{F}_{3}
\end{eqnarray*}

%__________________________________________________________________________________________
$i=2$
%__________________________________________________________________________________________
$k=1$
\begin{eqnarray*}
D_{1}D_{2}F_{2}&=&D_{1}D_{2}\left(R_{1}+F_{1}\right)+D_{2}R_{1}D_{1}F_{1}+D_{2}F_{1}D_{1}R_{1}
=D_{1}D_{2}R_{1}+D_{1}D_{2}F_{1}+D_{2}R_{1}D_{1}F_{1}+D_{2}F_{1}D_{1}R_{1}
\end{eqnarray*}
$k=2$
\begin{eqnarray*}
D_{3}D_{2}F_{2}&=&D_{3}D_{2}\left(R_{1}+F_{1}\right)+D_{2}R_{1}D_{2}F_{1}+D_{2}F_{1}D_{2}R_{1}
=D_{3}D_{2}R_{1}+D_{3}D_{2}F_{1}+D_{2}R_{1}D_{2}F_{1}+D_{2}F_{1}D_{2}R_{1}
\end{eqnarray*}

$k=3$
\begin{eqnarray*}
D_{3}D_{2}F_{2}&=&D_{3}D_{2}\left(R_{1}+F_{1}\right)+D_{2}R_{1}D_{3}\left(F_{1}+\hat{F}_{3}\right)+D_{2}F_{1}D_{3}\left(R_{1}+\hat{F}_{3}\right)\\
&=&D_{3}D_{2}R_{1}+D_{3}D_{2}F_{1}+D_{2}R_{1}D_{3}F_{1}+D_{2}R_{1}D_{3}\hat{F}_{3}
+D_{2}F_{1}D_{3}R_{1}+D_{2}F_{1}D_{3}\hat{F}_{3}
\end{eqnarray*}

$k=4$
\begin{eqnarray*}
D_{4}D_{2}F_{2}&=&D_{4}D_{2}\left(R_{1}+F_{1}+\hat{F}_{3}\right)+D_{2}R_{1}D_{4}\left(F_{1}+\hat{F}_{3}\right)+D_{2}F_{1}D_{4}\left(R_{1}+\hat{F}_{3}\right)\\
&=&D_{4}D_{2}R_{1}+D_{4}D_{2}F_{1}+D_{4}D_{2}\hat{F}_{3}
+D_{2}R_{1}D_{4}F_{1}+D_{2}R_{1}D_{4}\hat{F}_{3}
+D_{2}F_{1}D_{4}R_{1}+D_{2}F_{1}D_{4}\hat{F}_{3}
\end{eqnarray*}

%__________________________________________________________________________________________
$i=3$
%__________________________________________________________________________________________
$k=1$
\begin{eqnarray*}
D_{1}D_{3}F_{2}&=&D_{1}D_{3}\left(R_{1}+F_{1}+\hat{F}_{3}\right)+D_{3}R_{1}D_{1}F_{1}+D_{3}F_{1}D_{1}R_{1}+D_{3}\hat{F}_{3}D_{1}\left(R_{1}+F_{1}\right)\\
&=&D_{1}D_{3}R_{1}+D_{1}D_{3}F_{1}+D_{1}D_{3}\hat{F}_{3}
+D_{3}R_{1}D_{1}F_{1}+D_{3}F_{1}D_{1}R_{1}
+D_{3}\hat{F}_{3}D_{1}R_{1}+D_{3}\hat{F}_{3}D_{1}F_{1}
\end{eqnarray*}
$k=2$
\begin{eqnarray*}
D_{2}D_{3}F_{2}&=&D_{2}D_{3}\left(R_{1}+F_{1}+\hat{F}_{3}\right)
+D_{3}R_{1}D_{2}F_{1}+D_{3}F_{1}D_{2}R_{1}
+D_{3}\hat{F}_{3}D_{2}\left(R_{1}+F_{1}\right)\\
&=&D_{2}D_{3}R_{1}+D_{2}D_{3}F_{1}+D_{2}D_{3}\hat{F}_{3}
+D_{3}R_{1}D_{2}F_{1}+D_{3}F_{1}D_{2}R_{1}
+D_{3}\hat{F}_{3}D_{2}R_{1}+D_{3}\hat{F}_{3}D_{2}F_{1}
\end{eqnarray*}

$k=3$
\begin{eqnarray*}
D_{3}D_{3}F_{2}&=&D_{3}D_{3}\left(R_{1}+F_{1}+\hat{F}_{3}\right)
+D_{3}R_{1}D_{3}\left(F_{1}+\hat{F}_{3}\right)
+D_{3}F_{1}D_{3}\left(R_{1}+\hat{F}_{3}\right)
+D_{3}\hat{F}_{3}D_{3}\left(R_{1}+F_{1}\right)\\
&=&D_{3}D_{3}R_{1}+D_{3}D_{3}F_{1}+D_{3}D_{3}\hat{F}_{3}
+D_{3}R_{1}D_{3}F_{1}+D_{3}R_{1}D_{3}\hat{F}_{3}\\
&+&D_{3}F_{1}D_{3}R_{1}+D_{3}F_{1}D_{3}\hat{F}_{3}
+D_{3}\hat{F}_{3}D_{3}R_{1}+D_{3}\hat{F}_{3}D_{3}F_{1}
\end{eqnarray*}

$k=4$
\begin{eqnarray*}
D_{4}D_{3}F_{2}&=&D_{4}D_{3}\left(R_{1}+F_{1}+\hat{F}_{3}\right)
+D_{3}R_{1}D_{4}\left(F_{1}+\hat{F}_{3}\right)
+D_{3}F_{1}D_{4}\left(R_{1}+\hat{F}_{3}\right)
+D_{3}\hat{F}_{3}D_{4}\left(R_{1}+F_{1}\right)\\
&=&D_{4}D_{3}R_{1}+D_{4}D_{3}F_{1}+D_{4}D_{3}\hat{F}_{3}
+D_{3}R_{1}D_{4}F_{1}+D_{3}R_{1}D_{4}\hat{F}_{3}\\
&+&D_{3}F_{1}D_{4}R_{1}+D_{3}F_{1}D_{4}\hat{F}_{3}
+D_{3}\hat{F}_{3}D_{4}R_{1}+D_{3}\hat{F}_{3}D_{4}F_{1}
\end{eqnarray*}
%__________________________________________________________________________________________
$i=4$
%__________________________________________________________________________________________
$k=1$
\begin{eqnarray*}
D_{1}D_{4}F_{2}&=&D_{1}D_{4}\left(R_{1}+F_{1}+\hat{F}_{3}\right)
+D_{4}R_{1}D_{1}F_{1}+D_{4}F_{1}D_{1}R_{1}
+D_{4}\hat{F}_{3}D_{1}\left(R_{1}+F_{1}\right)\\
&=&D_{1}D_{4}R_{1}+D_{1}D_{4}F_{1}+D_{1}D_{4}\hat{F}_{3}
+D_{4}R_{1}D_{1}F_{1}+D_{4}F_{1}D_{1}R_{1}
+D_{4}\hat{F}_{3}D_{1}R_{1}+D_{4}\hat{F}_{3}D_{1}F_{1}
\end{eqnarray*}
$k=2$
\begin{eqnarray*}
D_{2}D_{4}F_{2}&=&D_{2}D_{4}\left(R_{1}+F_{1}+\hat{F}_{3}\right)
+D_{4}R_{1}D_{2}F_{1}+D_{4}F_{1}D_{2}R_{1}
+D_{4}\hat{F}_{3}D_{2}\left(R_{1}+F_{1}\right)\\
&=&D_{2}D_{4}R_{1}+D_{2}D_{4}F_{1}+D_{2}D_{4}\hat{F}_{3}
+D_{4}R_{1}D_{2}F_{1}+D_{4}F_{1}D_{2}R_{1}
+D_{4}\hat{F}_{3}D_{2}R_{1}+D_{4}\hat{F}_{3}D_{2}F_{1}
\end{eqnarray*}

$k=3$
\begin{eqnarray*}
D_{3}D_{4}F_{2}&=&D_{3}D_{4}\left(R_{1}+F_{1}+\hat{F}_{3}\right)
+D_{4}R_{1}D_{3}\left(F_{1}+\hat{F}_{3}\right)
+D_{4}F_{1}D_{3}\left(R_{1}+\hat{F}_{3}\right)
+D_{4}\hat{F}_{3}D_{3}\left(R_{1}+F_{1}\right)\\
&=&D_{3}D_{4}R_{1}+D_{3}D_{4}F_{1}+D_{3}D_{4}\hat{F}_{3}
+D_{4}R_{1}D_{3}F_{1}+D_{4}R_{1}D_{3}\hat{F}_{3}\\
&+&D_{4}F_{1}D_{3}R_{1}+D_{4}F_{1}D_{3}\hat{F}_{3}
+D_{4}\hat{F}_{3}D_{3}R_{1}+D_{4}\hat{F}_{3}D_{3}F_{1}
\end{eqnarray*}

$k=4$
\begin{eqnarray*}
D_{4}D_{4}F_{2}&=&D_{4}D_{4}\left(R_{1}+F_{1}+\hat{F}_{3}\right)
+D_{4}R_{1}D_{4}\left(F_{1}+\hat{F}_{3}\right)
+D_{4}F_{1}D_{4}\left(R_{1}+\hat{F}_{3}\right)
+D_{4}\hat{F}_{3}D_{4}\left(R_{1}+F_{1}\right)\\
&=&D_{4}D_{4}R_{1}+D_{4}D_{4}F_{1}+D_{4}D_{4}\hat{F}_{3}
+D_{4}R_{1}D_{4}F_{1}+D_{4}R_{1}D_{4}\hat{F}_{3}\\
&+&D_{4}F_{1}D_{4}R_{1}+D_{4}F_{1}D_{4}\hat{F}_{3}
+D_{4}\hat{F}_{3}D_{4}R_{1}+D_{4}\hat{F}_{3}D_{4}F_{1}
\end{eqnarray*}
%__________________________________________________________________________________________
%
%__________________________________________________________________________________________
%
%__________________________________________________________________________________________

\begin{eqnarray}
D_{k}D_{i}\hat{F}_{1}&=&D_{k}D_{i}\left(\hat{R}_{4}+\indora_{i\leq2}F_{2}+\hat{F}_{4}\right)+D_{i}\hat{R}_{4}D_{k}\left(\indora_{k\leq2}F_{2}+\hat{F}_{4}\right)+D_{i}\hat{F}_{4}D_{k}\left(\hat{R}_{4}+\indora_{k\leq2}F_{2}\right)+\indora_{i\leq2}D_{i}F_{2}D_{k}\left(\hat{R}_{4}+\hat{F}_{4}\right)
\end{eqnarray}
%__________________________________________________________________________________________
$i=1$
%__________________________________________________________________________________________
$k=1$
\begin{eqnarray*}
D_{1}D_{1}\hat{F}_{1}&=&D_{1}D_{1}\left(\hat{R}_{4}+F_{2}+\hat{F}_{4}\right)
+D_{1}\hat{R}_{4}D_{1}\left(F_{2}+\hat{F}_{4}\right)
+D_{1}\hat{F}_{4}D_{1}\left(\hat{R}_{4}+F_{2}\right)
+D_{1}F_{2}D_{1}\left(\hat{R}_{4}+\hat{F}_{4}\right)\\
&=&D_{1}^{2}\hat{R}_{4}+D_{1}^{2}F_{2}+D_{1}^{2}\hat{F}_{4}
+D_{1}\hat{R}_{4}D_{1}F_{2}+D_{1}\hat{R}_{4}D_{1}\hat{F}_{4}
+D_{1}\hat{F}_{4}D_{1}\hat{R}_{4}+D_{1}\hat{F}_{4}D_{1}F_{2}
+D_{1}F_{2}D_{1}\hat{R}_{4}+D_{1}F_{2}D_{1}\hat{F}_{4}
\end{eqnarray*}

$k=2$
\begin{eqnarray*}
D_{2}D_{1}\hat{F}_{1}&=&D_{2}D_{1}\left(\hat{R}_{4}+F_{2}+\hat{F}_{4}\right)
+D_{1}\hat{R}_{4}D_{2}\left(F_{2}+\hat{F}_{4}\right)
+D_{1}\hat{F}_{4}D_{2}\left(\hat{R}_{4}+F_{2}\right)
+D_{1}F_{2}D_{2}\left(\hat{R}_{4}+\hat{F}_{4}\right)\\
&=&D_{2}D_{1}\hat{R}_{4}+D_{2}D_{1}F_{2}+D_{2}D_{1}\hat{F}_{4}
+D_{1}\hat{R}_{4}D_{2}F_{2}+D_{1}\hat{R}_{4}D_{2}\hat{F}_{4}\\
&+&D_{1}\hat{F}_{4}D_{2}\hat{R}_{4}+D_{1}\hat{F}_{4}D_{2}F_{2}
+D_{1}F_{2}D_{2}\hat{R}_{4}+D_{1}F_{2}D_{2}\hat{F}_{4}
\end{eqnarray*}

$k=3$
\begin{eqnarray*}
D_{3}D_{1}\hat{F}_{1}&=&D_{3}D_{1}\left(\hat{R}_{4}+F_{2}+\hat{F}_{4}\right)
+D_{1}\hat{R}_{4}D_{3}\left(\hat{F}_{4}\right)
+D_{1}\hat{F}_{4}D_{3}\hat{R}_{4}
+D_{1}F_{2}D_{3}\left(\hat{R}_{4}+\hat{F}_{4}\right)\\
&=&D_{3}D_{1}\hat{R}_{4}+D_{3}D_{1}F_{2}+D_{3}D_{1}\hat{F}_{4}
+D_{1}\hat{R}_{4}D_{3}\hat{F}_{4}
+D_{1}\hat{F}_{4}D_{3}\hat{R}_{4}
+D_{1}F_{2}D_{3}\hat{R}_{4}+D_{1}F_{2}D_{3}\hat{F}_{4}
\end{eqnarray*}

$k=4$
\begin{eqnarray*}
D_{4}D_{1}\hat{F}_{1}&=&D_{4}D_{1}\left(\hat{R}_{4}+F_{2}+\hat{F}_{4}\right)
+D_{1}\hat{R}_{4}D_{4}\hat{F}_{4}
+D_{1}\hat{F}_{4}D_{4}\hat{R}_{4}
+D_{1}F_{2}D_{4}\left(\hat{R}_{4}+\hat{F}_{4}\right)\\
&=&D_{4}D_{1}\hat{R}_{4}+D_{4}D_{1}F_{2}+D_{4}D_{1}\hat{F}_{4}
+D_{1}\hat{R}_{4}D_{4}\hat{F}_{4}
+D_{1}\hat{F}_{4}D_{4}\hat{R}_{4}
+D_{1}F_{2}D_{4}\hat{R}_{4}+D_{1}F_{2}D_{4}\hat{F}_{4}
\end{eqnarray*}

%__________________________________________________________________________________________
$i=2$
%__________________________________________________________________________________________
$k=1$
\begin{eqnarray*}
D_{1}D_{2}\hat{F}_{1}&=&D_{1}D_{2}\left(\hat{R}_{4}+F_{2}+\hat{F}_{4}\right)
+D_{2}\hat{R}_{4}D_{1}\left(F_{2}+\hat{F}_{4}\right)
+D_{2}\hat{F}_{4}D_{2}\left(\hat{R}_{4}+F_{2}\right)
+D_{2}F_{2}D_{1}\left(\hat{R}_{4}+\hat{F}_{4}\right)\\
&=&D_{1}D_{2}\hat{R}_{4}+D_{1}D_{2}F_{2}+D_{1}D_{2}\hat{F}_{4}
+D_{2}\hat{R}_{4}D_{1}F_{2}+D_{2}\hat{R}_{4}D_{1}\hat{F}_{4}\\
&+&D_{2}\hat{F}_{4}D_{2}\hat{R}_{4}+D_{2}\hat{F}_{4}D_{2}F_{2}
+D_{2}F_{2}D_{1}\hat{R}_{4}+D_{2}F_{2}D_{1}\hat{F}_{4}
\end{eqnarray*}
$k=2$
\begin{eqnarray*}
D_{2}D_{2}\hat{F}_{1}&=&D_{2}D_{2}\left(\hat{R}_{4}+F_{2}+\hat{F}_{4}\right)
+D_{2}\hat{R}_{4}D_{2}\left(F_{2}+\hat{F}_{4}\right)
+D_{2}\hat{F}_{4}D_{2}\left(\hat{R}_{4}+F_{2}\right)
+D_{2}F_{2}D_{2}\left(\hat{R}_{4}+\hat{F}_{4}\right)\\
&=&D_{2}D_{2}\left(\hat{R}_{4}+F_{2}+\hat{F}_{4}\right)
+D_{2}\hat{R}_{4}D_{2}\left(F_{2}+\hat{F}_{4}\right)
+D_{2}\hat{F}_{4}D_{2}\left(\hat{R}_{4}+F_{2}\right)
+D_{2}F_{2}D_{2}\left(\hat{R}_{4}+\hat{F}_{4}\right)
\end{eqnarray*}

$k=3$
\begin{eqnarray*}
D_{3}D_{2}\hat{F}_{1}&=&D_{3}D_{2}\left(\hat{R}_{4}+F_{2}+\hat{F}_{4}\right)
+D_{2}\hat{R}_{4}D_{3}\hat{F}_{4}
+D_{2}\hat{F}_{4}D_{3}\hat{R}_{4}
+D_{2}F_{2}D_{3}\left(\hat{R}_{4}+\hat{F}_{4}\right)\\
&=&D_{3}D_{2}\hat{R}_{4}+D_{3}D_{2}F_{2}+D_{3}D_{2}\hat{F}_{4}
+D_{2}\hat{R}_{4}D_{3}\hat{F}_{4}
+D_{2}\hat{F}_{4}D_{3}\hat{R}_{4}
+D_{2}F_{2}D_{3}\hat{R}_{4}+D_{2}F_{2}D_{3}\hat{F}_{4}
\end{eqnarray*}

$k=4$
\begin{eqnarray*}
D_{4}D_{2}\hat{F}_{1}&=&D_{4}D_{2}\left(\hat{R}_{4}+F_{2}+\hat{F}_{4}\right)
+D_{2}\hat{R}_{4}D_{4}\hat{F}_{4}
+D_{2}\hat{F}_{4}D_{4}\hat{R}_{4}
+D_{2}F_{2}D_{4}\left(\hat{R}_{4}+\hat{F}_{4}\right)\\
&=&D_{4}D_{2}\hat{R}_{4}+D_{4}D_{2}F_{2}+D_{4}D_{2}\hat{F}_{4}
+D_{2}\hat{R}_{4}D_{4}\hat{F}_{4}
+D_{2}\hat{F}_{4}D_{4}\hat{R}_{4}
+D_{2}F_{2}D_{4}\hat{R}_{4}+D_{2}F_{2}D_{4}\hat{F}_{4}
\end{eqnarray*}

%__________________________________________________________________________________________
$i=3$
%__________________________________________________________________________________________
$k=1$
\begin{eqnarray*}
D_{1}D_{3}\hat{F}_{1}&=&D_{1}D_{3}\left(\hat{R}_{4}+\hat{F}_{4}\right)
+D_{3}\hat{R}_{4}D_{1}\left(F_{2}+\hat{F}_{4}\right)
+D_{3}\hat{F}_{4}D_{1}\left(\hat{R}_{4}+F_{2}\right)\\
&=&D_{1}D_{3}\hat{R}_{4}+D_{1}D_{3}\hat{F}_{4}
+D_{3}\hat{R}_{4}D_{1}F_{2}+D_{3}\hat{R}_{4}D_{1}\hat{F}_{4}
+D_{3}\hat{F}_{4}D_{1}\hat{R}_{4}+D_{3}\hat{F}_{4}D_{1}F_{2}
\end{eqnarray*}
$k=2$
\begin{eqnarray*}
D_{2}D_{3}\hat{F}_{1}&=&D_{2}D_{3}\left(\hat{R}_{4}+\hat{F}_{4}\right)
+D_{3}\hat{R}_{4}D_{2}\left(F_{2}+\hat{F}_{4}\right)
+D_{3}\hat{F}_{4}D_{2}\left(\hat{R}_{4}+F_{2}\right)\\
&=&D_{2}D_{3}\hat{R}_{4}+D_{2}D_{3}\hat{F}_{4}
+D_{3}\hat{R}_{4}D_{2}F_{2}+D_{3}\hat{R}_{4}D_{2}\hat{F}_{4}
+D_{3}\hat{F}_{4}D_{2}\hat{R}_{4}+D_{3}\hat{F}_{4}D_{2}F_{2}
\end{eqnarray*}

$k=3$
\begin{eqnarray*}
D_{3}D_{3}\hat{F}_{1}&=&D_{3}D_{3}\left(\hat{R}_{4}+\hat{F}_{4}\right)
+D_{3}\hat{R}_{4}D_{3}\hat{F}_{4}
+D_{3}\hat{F}_{4}D_{3}\hat{R}_{4}\\
&=&D_{3}^{2}\hat{R}_{4}+D_{3}^{2}\hat{F}_{4}
+D_{3}\hat{R}_{4}D_{3}\hat{F}_{4}
+D_{3}\hat{F}_{4}D_{3}\hat{R}_{4}
\end{eqnarray*}

$k=4$
\begin{eqnarray*}
D_{4}D_{3}\hat{F}_{1}&=&D_{4}D_{3}\left(\hat{R}_{4}+\hat{F}_{4}\right)
+D_{3}\hat{R}_{4}D_{4}\hat{F}_{4}
+D_{3}\hat{F}_{4}D_{4}\hat{R}_{4}\\
&=&D_{4}D_{3}\hat{R}_{4}+D_{4}D_{3}\hat{F}_{4}
+D_{3}\hat{R}_{4}D_{4}\hat{F}_{4}
+D_{3}\hat{F}_{4}D_{4}\hat{R}_{4}
\end{eqnarray*}

%__________________________________________________________________________________________
$i=4$
%__________________________________________________________________________________________
$k=1$
\begin{eqnarray*}
D_{1}D_{4}\hat{F}_{1}&=&D_{1}D_{4}\left(\hat{R}_{4}+\hat{F}_{4}\right)
+D_{4}\hat{R}_{4}D_{1}\left(F_{2}+\hat{F}_{4}\right)
+D_{4}\hat{F}_{4}D_{1}\left(\hat{R}_{4}+F_{2}\right)\\
&=&D_{1}D_{4}\hat{R}_{4}+D_{1}D_{4}\hat{F}_{4}
+D_{4}\hat{R}_{4}D_{1}F_{2}+D_{4}\hat{R}_{4}D_{1}\hat{F}_{4}
+D_{4}\hat{F}_{4}D_{1}\hat{R}_{4}+D_{4}\hat{F}_{4}D_{1}F_{2}
\end{eqnarray*}
$k=2$
\begin{eqnarray*}
D_{2}D_{4}\hat{F}_{1}&=&D_{2}D_{4}\left(\hat{R}_{4}+\hat{F}_{4}\right)
+D_{4}\hat{R}_{4}D_{2}\left(F_{2}+\hat{F}_{4}\right)
+D_{4}\hat{F}_{4}D_{2}\left(\hat{R}_{4}+F_{2}\right)\\
&=&D_{2}D_{4}\hat{R}_{4}+D_{2}D_{4}\hat{F}_{4}
+D_{4}\hat{R}_{4}D_{2}F_{2}+D_{4}\hat{R}_{4}D_{2}\hat{F}_{4}
+D_{4}\hat{F}_{4}D_{2}\hat{R}_{4}+D_{4}\hat{F}_{4}D_{2}F_{2}
\end{eqnarray*}

$k=3$
\begin{eqnarray*}
D_{3}D_{4}\hat{F}_{1}&=&D_{3}D_{4}\left(\hat{R}_{4}+\hat{F}_{4}\right)
+D_{4}\hat{R}_{4}D_{3}\hat{F}_{4}
+D_{4}\hat{F}_{4}D_{3}\hat{R}_{4}\\
&=&D_{3}D_{4}\hat{R}_{4}+D_{3}D_{4}\hat{F}_{4}
+D_{4}\hat{R}_{4}D_{3}\hat{F}_{4}
+D_{4}\hat{F}_{4}D_{3}\hat{R}_{4}
\end{eqnarray*}

$k=4$
\begin{eqnarray*}
D_{4}D_{4}\hat{F}_{1}&=&D_{4}D_{4}\left(\hat{R}_{4}+\hat{F}_{4}\right)
+D_{4}\hat{R}_{4}D_{4}\hat{F}_{4}
+D_{4}\hat{F}_{4}D_{4}\hat{R}_{4}\\
&=&D_{4}^{2}\hat{R}_{4}+D_{4}^{2}\hat{F}_{4}
+D_{4}\hat{R}_{4}D_{4}\hat{F}_{4}
+D_{4}\hat{F}_{4}D_{4}\hat{R}_{4}
\end{eqnarray*}
%__________________________________________________________________________________________
%
for $\hat{F}_{2}$
%__________________________________________________________________________________________
%
%__________________________________________________________________________________________

\begin{eqnarray}
D_{k}D_{i}\hat{F}_{2}&=&D_{k}D_{i}\left(\hat{R}_{3}+\indora_{i\leq2}F_{1}+\hat{F}_{3}\right)+D_{i}\hat{R}_{3}D_{k}\left(\indora_{k\leq2}F_{1}+\hat{F}_{3}\right)+D_{i}\hat{F}_{3}D_{k}\left(\hat{R}_{3}+\indora_{k\leq2}F_{1}\right)+\indora_{i\leq2}D_{i}F_{1}D_{k}\left(\hat{R}_{3}+\hat{F}_{3}\right)\\
&=&
\end{eqnarray}
%__________________________________________________________________________________________
$i=1$
%__________________________________________________________________________________________
$k=1$
\begin{eqnarray*}
D_{1}D_{1}\hat{F}_{2}&=&D_{1}^{2}\left(\hat{R}_{3}+F_{1}+\hat{F}_{3}\right)
+D_{1}\hat{R}_{3}D_{1}\left(F_{1}+\hat{F}_{3}\right)
+D_{1}\hat{F}_{3}D_{1}\left(\hat{R}_{3}+F_{1}\right)
+D_{1}F_{1}D_{1}\left(\hat{R}_{3}+\hat{F}_{3}\right)\\
&=&D_{1}^{2}\hat{R}_{3}+D_{1}^{2}F_{1}+D_{1}^{2}\hat{F}_{3}
+D_{1}\hat{R}_{3}D_{1}F_{1}+D_{1}\hat{R}_{3}D_{1}\hat{F}_{3}
+D_{1}\hat{F}_{3}D_{1}\hat{R}_{3}+D_{1}\hat{F}_{3}D_{1}F_{1}
+D_{1}F_{1}D_{1}\hat{R}_{3}+D_{1}F_{1}D_{1}\hat{F}_{3}
\end{eqnarray*}

$k=2$
\begin{eqnarray*}
D_{2}D_{1}\hat{F}_{2}&=&D_{2}D_{1}\left(\hat{R}_{3}+F_{1}+\hat{F}_{3}\right)
+D_{1}\hat{R}_{3}D_{2}\left(F_{1}+\hat{F}_{3}\right)
+D_{1}\hat{F}_{3}D_{2}\left(\hat{R}_{3}+F_{1}\right)
+D_{1}F_{1}D_{2}\left(\hat{R}_{3}+\hat{F}_{3}\right)\\
&=&D_{2}D_{1}\hat{R}_{3}+D_{2}D_{1}F_{1}+D_{2}D_{1}\hat{F}_{3}
+D_{1}\hat{R}_{3}D_{2}F_{1}+D_{1}\hat{R}_{3}D_{2}\hat{F}_{3}\\
&+&D_{1}\hat{F}_{3}D_{2}\hat{R}_{3}+D_{1}\hat{F}_{3}D_{2}F_{1}
+D_{1}F_{1}D_{2}\hat{R}_{3}+D_{1}F_{1}D_{2}\hat{F}_{3}
\end{eqnarray*}

$k=3$
\begin{eqnarray*}
D_{3}D_{1}\hat{F}_{2}&=&D_{3}D_{1}\left(\hat{R}_{3}+F_{1}+\hat{F}_{3}\right)
+D_{1}\hat{R}_{3}D_{3}\hat{F}_{3}
+D_{1}\hat{F}_{3}D_{3}\hat{R}_{3}
+D_{1}F_{1}D_{3}\left(\hat{R}_{3}+\hat{F}_{3}\right)\\
&=&D_{3}D_{1}\hat{R}_{3}+D_{3}D_{1}F_{1}+D_{3}D_{1}\hat{F}_{3}
+D_{1}\hat{R}_{3}D_{3}\hat{F}_{3}
+D_{1}\hat{F}_{3}D_{3}\hat{R}_{3}
+D_{1}F_{1}D_{3}\hat{R}_{3}+D_{1}F_{1}D_{3}\hat{F}_{3}
\end{eqnarray*}

$k=4$
\begin{eqnarray*}
D_{4}D_{1}\hat{F}_{2}&=&D_{4}D_{1}\left(\hat{R}_{3}+F_{1}+\hat{F}_{3}\right)
+D_{1}\hat{R}_{3}D_{4}\hat{F}_{3}
+D_{1}\hat{F}_{3}D_{4}\hat{R}_{3}
+D_{1}F_{1}D_{4}\left(\hat{R}_{3}+\hat{F}_{3}\right)\\
&=&D_{4}D_{1}\hat{R}_{3}+D_{4}D_{1}F_{1}+D_{4}D_{1}\hat{F}_{3}
+D_{1}\hat{R}_{3}D_{4}\hat{F}_{3}
+D_{1}\hat{F}_{3}D_{4}\hat{R}_{3}
+D_{1}F_{1}D_{4}\hat{R}_{3}+D_{1}F_{1}D_{4}\hat{F}_{3}
\end{eqnarray*}

%__________________________________________________________________________________________
$i=2$
%__________________________________________________________________________________________
$k=1$
\begin{eqnarray*}
D_{1}D_{2}\hat{F}_{2}&=&D_{1}D_{2}\left(\hat{R}_{3}+F_{1}+\hat{F}_{3}\right)
+D_{2}\hat{R}_{3}D_{1}\left(F_{1}+\hat{F}_{3}\right)
+D_{2}\hat{F}_{3}D_{1}\left(\hat{R}_{3}+F_{1}\right)
+D_{2}F_{1}D_{1}\left(\hat{R}_{3}+\hat{F}_{3}\right)\\
&=&D_{1}D_{2}\hat{R}_{3}+D_{1}D_{2}F_{1}+D_{1}D_{2}\hat{F}_{3}
+D_{2}\hat{R}_{3}D_{1}F_{1}+D_{2}\hat{R}_{3}D_{1}\hat{F}_{3}\\
&+&D_{2}\hat{F}_{3}D_{1}\hat{R}_{3}+D_{2}\hat{F}_{3}D_{1}F_{1}
+D_{2}F_{1}D_{1}\hat{R}_{3}+D_{2}F_{1}D_{1}\hat{F}_{3}
\end{eqnarray*}

$k=2$
\begin{eqnarray*}
D_{2}D_{2}\hat{F}_{2}&=&D_{2}D_{2}\left(\hat{R}_{3}+F_{1}+\hat{F}_{3}\right)
+D_{2}\hat{R}_{3}D_{2}\left(F_{1}+\hat{F}_{3}\right)
+D_{2}\hat{F}_{3}D_{2}\left(\hat{R}_{3}+F_{1}\right)
+D_{2}F_{1}D_{2}\left(\hat{R}_{3}+\hat{F}_{3}\right)\\
&=&D_{2}^{2}\hat{R}_{3}+D_{2}^{2}F_{1}+D_{2}^{2}\hat{F}_{3}
+D_{2}\hat{R}_{3}D_{2}F_{1}+D_{2}\hat{R}_{3}D_{2}\hat{F}_{3}
+D_{2}\hat{F}_{3}D_{2}\hat{R}_{3}+D_{2}\hat{F}_{3}D_{2}F_{1}
+D_{2}F_{1}D_{2}\hat{R}_{3}+D_{2}F_{1}D_{2}\hat{F}_{3}
\end{eqnarray*}

$k=3$
\begin{eqnarray*}
D_{3}D_{2}\hat{F}_{2}&=&D_{3}D_{2}\left(\hat{R}_{3}+F_{1}+\hat{F}_{3}\right)
+D_{2}\hat{R}_{3}D_{3}\hat{F}_{3}
+D_{2}\hat{F}_{3}D_{3}\hat{R}_{3}
+D_{2}F_{1}D_{3}\left(\hat{R}_{3}+\hat{F}_{3}\right)\\
&=&D_{3}D_{2}\hat{R}_{3}+D_{3}D_{2}F_{1}+D_{3}D_{2}\hat{F}_{3}
+D_{2}\hat{R}_{3}D_{3}\hat{F}_{3}
+D_{2}\hat{F}_{3}D_{3}\hat{R}_{3}
+D_{2}F_{1}D_{3}\hat{R}_{3}+D_{2}F_{1}D_{3}\hat{F}_{3}
\end{eqnarray*}

$k=4$
\begin{eqnarray*}
D_{4}D_{2}\hat{F}_{2}&=&D_{4}D_{2}\left(\hat{R}_{3}+F_{1}+\hat{F}_{3}\right)
+D_{2}\hat{R}_{3}D_{4}\hat{F}_{3}
+D_{2}\hat{F}_{3}D_{4}\hat{R}_{3}
+D_{2}F_{1}D_{4}\left(\hat{R}_{3}+\hat{F}_{3}\right)\\
&=&D_{4}D_{2}\hat{R}_{3}+D_{4}D_{2}F_{1}+\hat{F}_{3}
+D_{2}\hat{R}_{3}D_{4}\hat{F}_{3}
+D_{2}\hat{F}_{3}D_{4}\hat{R}_{3}
+D_{2}F_{1}D_{4}\hat{R}_{3}+D_{2}F_{1}D_{4}\hat{F}_{3}
\end{eqnarray*}
%__________________________________________________________________________________________
$i=3$
%__________________________________________________________________________________________
$k=1$
\begin{eqnarray*}
D_{1}D_{3}\hat{F}_{2}&=&D_{1}D_{3}\left(\hat{R}_{3}+\hat{F}_{3}\right)
+D_{3}\hat{R}_{3}D_{1}\left(F_{1}+\hat{F}_{3}\right)
+D_{3}\hat{F}_{3}D_{1}\left(\hat{R}_{3}+F_{1}\right)\\
&=&D_{1}D_{3}\hat{R}_{3}+D_{1}D_{3}\hat{F}_{3}
+D_{3}\hat{R}_{3}D_{1}F_{1}+D_{3}\hat{R}_{3}D_{1}\hat{F}_{3}
+D_{3}\hat{F}_{3}D_{1}\hat{R}_{3}+D_{3}\hat{F}_{3}D_{1}F_{1}
\end{eqnarray*}

$k=2$
\begin{eqnarray*}
D_{2}D_{3}\hat{F}_{2}&=&D_{2}D_{3}\left(\hat{R}_{3}+\hat{F}_{3}\right)
+D_{3}\hat{R}_{3}D_{2}\left(F_{1}+\hat{F}_{3}\right)
+D_{3}\hat{F}_{3}D_{2}\left(\hat{R}_{3}+F_{1}\right)\\
&=&D_{2}D_{3}\hat{R}_{3}+D_{2}D_{3}\hat{F}_{3}
+D_{3}\hat{R}_{3}D_{2}F_{1}+D_{3}\hat{R}_{3}D_{2}\hat{F}_{3}
+D_{3}\hat{F}_{3}D_{2}\hat{R}_{3}+D_{3}\hat{F}_{3}D_{2}F_{1}
\end{eqnarray*}

$k=3$
\begin{eqnarray*}
D_{3}D_{3}\hat{F}_{2}&=&D_{3}D_{3}\left(\hat{R}_{3}+\hat{F}_{3}\right)
+D_{3}\hat{R}_{3}D_{3}\hat{F}_{3}
+D_{3}\hat{F}_{3}D_{3}\hat{R}_{3}\\
&=&D_{3}^{2}\hat{R}_{3}+D_{3}^{2}\hat{F}_{3}
+D_{3}\hat{R}_{3}D_{3}\hat{F}_{3}
+D_{3}\hat{F}_{3}D_{3}\hat{R}_{3}
\end{eqnarray*}

$k=4$
\begin{eqnarray*}
D_{4}D_{3}\hat{F}_{2}&=&D_{4}D_{3}\left(\hat{R}_{3}+\hat{F}_{3}\right)
+D_{3}\hat{R}_{3}D_{4}\hat{F}_{3}
+D_{3}\hat{F}_{3}D_{4}\hat{R}_{3}\\
&=&D_{4}D_{3}\hat{R}_{3}+D_{4}D_{3}\hat{F}_{3}
+D_{3}\hat{R}_{3}D_{4}\hat{F}_{3}
+D_{3}\hat{F}_{3}D_{4}\hat{R}_{3}
\end{eqnarray*}
%__________________________________________________________________________________________
$i=4$
%__________________________________________________________________________________________
$k=1$
\begin{eqnarray*}
D_{1}D_{4}\hat{F}_{2}&=&D_{1}D_{4}\left(\hat{R}_{3}+\hat{F}_{3}\right)
+D_{4}\hat{R}_{3}D_{1}\left(F_{1}+\hat{F}_{3}\right)
+D_{4}\hat{F}_{3}D_{4}\left(\hat{R}_{3}+F_{1}\right)\\
&=&D_{1}D_{4}\hat{R}_{3}+D_{1}D_{4}\hat{F}_{3}
+D_{4}\hat{R}_{3}D_{1}F_{1}+D_{4}\hat{R}_{3}D_{1}\hat{F}_{3}
+D_{4}\hat{F}_{3}D_{4}\hat{R}_{3}+D_{4}\hat{F}_{3}D_{4}F_{1}
\end{eqnarray*}

$k=2$
\begin{eqnarray*}
D_{2}D_{4}\hat{F}_{2}&=&D_{2}D_{4}\left(\hat{R}_{3}+\hat{F}_{3}\right)
+D_{4}\hat{R}_{3}D_{2}\left(F_{1}+\hat{F}_{3}\right)
+D_{4}\hat{F}_{3}D_{2}\left(\hat{R}_{3}+F_{1}\right)\\
&=&D_{2}D_{4}\hat{R}_{3}+D_{2}D_{4}\hat{F}_{3}
+D_{4}\hat{R}_{3}D_{2}F_{1}+D_{4}\hat{R}_{3}D_{2}\hat{F}_{3}
+D_{4}\hat{F}_{3}D_{2}\hat{R}_{3}+D_{4}\hat{F}_{3}D_{2}F_{1}
\end{eqnarray*}

$k=3$
\begin{eqnarray*}
D_{3}D_{4}\hat{F}_{2}&=&D_{3}D_{4}\left(\hat{R}_{3}+\hat{F}_{3}\right)
+D_{4}\hat{R}_{3}D_{3}\hat{F}_{3}
+D_{4}\hat{F}_{3}D_{3}\hat{R}_{3}\\
&=&D_{3}D_{4}\hat{R}_{3}+D_{3}D_{4}\hat{F}_{3}
+D_{4}\hat{R}_{3}D_{3}\hat{F}_{3}
+D_{4}\hat{F}_{3}D_{3}\hat{R}_{3}
\end{eqnarray*}

$k=4$
\begin{eqnarray*}
D_{4}D_{4}\hat{F}_{2}&=&D_{4}^{2}\left(\hat{R}_{3}+\hat{F}_{3}\right)
+D_{4}\hat{R}_{3}D_{4}\hat{F}_{3}
+D_{4}\hat{F}_{3}D_{4}\hat{R}_{3}\\
&=&D_{4}^{2}\hat{R}_{3}+D_{4}^{2}\hat{F}_{3}
+D_{4}\hat{R}_{3}D_{4}\hat{F}_{3}
+D_{4}\hat{F}_{3}D_{4}\hat{R}_{3}
\end{eqnarray*}
%__________________________________________________________________________________________
%

%_____________________________________________________________________________________
\newpage

%__________________________________________________________________
\section{Generalizaciones}
%__________________________________________________________________
\subsection{RSVC con dos conexiones}
%__________________________________________________________________

%\begin{figure}[H]
%\centering
%%%\includegraphics[width=9cm]{Grafica3.jpg}
%%\end{figure}\label{RSVC3}


Sus ecuaciones recursivas son de la forma


\begin{eqnarray*}
F_{1}\left(z_{1},z_{2},w_{1},w_{2}\right)&=&R_{2}\left(\prod_{i=1}^{2}\tilde{P}_{i}\left(z_{i}\right)\prod_{i=1}^{2}
\hat{P}_{i}\left(w_{i}\right)\right)F_{2}\left(z_{1},\tilde{\theta}_{2}\left(\tilde{P}_{1}\left(z_{1}\right)\hat{P}_{1}\left(w_{1}\right)\hat{P}_{2}\left(w_{2}\right)\right)\right)
\hat{F}_{2}\left(w_{1},w_{2};\tau_{2}\right),
\end{eqnarray*}

\begin{eqnarray*}
F_{2}\left(z_{1},z_{2},w_{1},w_{2}\right)&=&R_{1}\left(\prod_{i=1}^{2}\tilde{P}_{i}\left(z_{i}\right)\prod_{i=1}^{2}
\hat{P}_{i}\left(w_{i}\right)\right)F_{1}\left(\tilde{\theta}_{1}\left(\tilde{P}_{2}\left(z_{2}\right)\hat{P}_{1}\left(w_{1}\right)\hat{P}_{2}\left(w_{2}\right)\right),z_{2}\right)\hat{F}_{1}\left(w_{1},w_{2};\tau_{1}\right),
\end{eqnarray*}


\begin{eqnarray*}
\hat{F}_{1}\left(z_{1},z_{2},w_{1},w_{2}\right)&=&\hat{R}_{2}\left(\prod_{i=1}^{2}\tilde{P}_{i}\left(z_{i}\right)\prod_{i=1}^{2}
\hat{P}_{i}\left(w_{i}\right)\right)F_{2}\left(z_{1},z_{2};\zeta_{2}\right)\hat{F}_{2}\left(w_{1},\hat{\theta}_{2}\left(\tilde{P}_{1}\left(z_{1}\right)\tilde{P}_{2}\left(z_{2}\right)\hat{P}_{1}\left(w_{1}
\right)\right)\right),
\end{eqnarray*}


\begin{eqnarray*}
\hat{F}_{2}\left(z_{1},z_{2},w_{1},w_{2}\right)&=&\hat{R}_{1}\left(\prod_{i=1}^{2}\tilde{P}_{i}\left(z_{i}\right)\prod_{i=1}^{2}
\hat{P}_{i}\left(w_{i}\right)\right)F_{1}\left(z_{1},z_{2};\zeta_{1}\right)\hat{F}_{1}\left(\hat{\theta}_{1}\left(\tilde{P}_{1}\left(z_{1}\right)\tilde{P}_{2}\left(z_{2}\right)\hat{P}_{2}\left(w_{2}\right)\right),w_{2}\right),
\end{eqnarray*}

%_____________________________________________________
\subsection{First Moments of the Queue Lengths}
%_____________________________________________________


The server's switchover times are given by the general equation

\begin{eqnarray}\label{Ec.Ri}
R_{i}\left(\mathbf{z,w}\right)=R_{i}\left(\tilde{P}_{1}\left(z_{1}\right)\tilde{P}_{2}\left(z_{2}\right)\hat{P}_{1}\left(w_{1}\right)\hat{P}_{2}\left(w_{2}\right)\right)
\end{eqnarray}

with
\begin{eqnarray}\label{Ec.Derivada.Ri}
D_{i}R_{i}&=&DR_{i}D_{i}P_{i}
\end{eqnarray}
the following notation is considered

\begin{eqnarray*}
\begin{array}{llll}
D_{1}P_{1}\equiv D_{1}\tilde{P}_{1}, & D_{2}P_{2}\equiv D_{2}\tilde{P}_{2}, & D_{3}P_{3}\equiv D_{3}\hat{P}_{1}, &D_{4}P_{4}\equiv D_{4}\hat{P}_{2},
\end{array}
\end{eqnarray*}

also we need to remind $F_{1,2}\left(z_{1};\zeta_{2}\right)F_{2,2}\left(z_{2};\zeta_{2}\right)=F_{2}\left(z_{1},z_{2};\zeta_{2}\right)$, therefore

\begin{eqnarray*}
D_{1}F_{2}\left(z_{1},z_{2};\zeta_{2}\right)&=&D_{1}\left[F_{1,2}\left(z_{1};\zeta_{2}\right)F_{2,2}\left(z_{2};\zeta_{2}\right)\right]
=F_{2,2}\left(z_{2};\zeta_{2}\right)D_{1}F_{1,2}\left(z_{1};\zeta_{2}\right)=F_{1,2}^{(1)}\left(1\right)
\end{eqnarray*}

i.e., $D_{1}F_{2}=F_{1,2}^{(1)}(1)$; $D_{2}F_{2}=F_{2,2}^{(1)}\left(1\right)$, whereas that $D_{3}F_{2}=D_{4}F_{2}=0$, then

\begin{eqnarray}
\begin{array}{ccc}
D_{i}F_{j}=\indora_{i\leq2}F_{i,j}^{(1)}\left(1\right),& \textrm{ and } &D_{i}\hat{F}_{j}=\indora_{i\geq2}F_{i,j}^{(1)}\left(1\right).
\end{array}
\end{eqnarray}

Now, we obtain the first moments equations for the queue lengths as before for the times the server arrives to the queue to start attending



Remember that


\begin{eqnarray*}
F_{2}\left(z_{1},z_{2},w_{1},w_{2}\right)&=&R_{1}\left(\prod_{i=1}^{2}\tilde{P}_{i}\left(z_{i}\right)\prod_{i=1}^{2}
\hat{P}_{i}\left(w_{i}\right)\right)F_{1}\left(\tilde{\theta}_{1}\left(\tilde{P}_{2}\left(z_{2}\right)\hat{P}_{1}\left(w_{1}\right)\hat{P}_{2}\left(w_{2}\right)\right),z_{2}\right)\hat{F}_{1}\left(w_{1},w_{2};\tau_{1}\right),
\end{eqnarray*}

where


\begin{eqnarray*}
F_{1}\left(\tilde{\theta}_{1}\left(\tilde{P}_{2}\hat{P}_{1}\hat{P}_{2}\right),z_{2}\right)
\end{eqnarray*}

so

\begin{eqnarray}
D_{i}F_{1}&=&\indora_{i\neq1}D_{1}F_{1}D\tilde{\theta}_{1}D_{i}P_{i}+\indora_{i=2}D_{i}F_{1},
\end{eqnarray}

then


\begin{eqnarray*}
\begin{array}{ll}
D_{1}F_{1}=0,&
D_{2}F_{1}=D_{1}F_{1}D\tilde{\theta}_{1}D_{2}P_{2}+D_{2}F_{1}
=f_{1}\left(1\right)\frac{1}{1-\tilde{\mu}_{1}}\tilde{\mu}_{2}+f_{1}\left(2\right),\\
D_{3}F_{1}=D_{1}F_{1}D\tilde{\theta}_{1}D_{3}P_{3}
=f_{1}\left(1\right)\frac{1}{1-\tilde{\mu}_{1}}\hat{\mu}_{1},&
D_{4}F_{1}=D_{1}F_{1}D\tilde{\theta}_{1}D_{4}P_{4}
=f_{1}\left(1\right)\frac{1}{1-\tilde{\mu}_{1}}\hat{\mu}_{2}

\end{array}
\end{eqnarray*}


\begin{eqnarray}
D_{i}F_{2}&=&\indora_{i\neq2}D_{2}F_{2}D\tilde{\theta}_{2}D_{i}P_{i}
+\indora_{i=1}D_{i}F_{2}
\end{eqnarray}

\begin{eqnarray*}
\begin{array}{ll}
D_{1}F_{2}=D_{2}F_{2}D\tilde{\theta}_{2}D_{1}P_{1}
+D_{1}F_{2}=f_{2}\left(2\right)\frac{1}{1-\tilde{\mu}_{2}}\tilde{\mu}_{1},&
D_{2}F_{2}=0\\
D_{3}F_{2}=D_{2}F_{2}D\tilde{\theta}_{2}D_{3}P_{3}
=f_{2}\left(2\right)\frac{1}{1-\tilde{\mu}_{2}}\hat{\mu}_{1},&
D_{4}F_{2}=D_{2}F_{2}D\tilde{\theta}_{2}D_{4}P_{4}
=f_{2}\left(2\right)\frac{1}{1-\tilde{\mu}_{2}}\hat{\mu}_{2}
\end{array}
\end{eqnarray*}



\begin{eqnarray}
D_{i}\hat{F}_{1}&=&\indora_{i\neq3}D_{3}\hat{F}_{1}D\hat{\theta}_{1}D_{i}P_{i}+\indora_{i=4}D_{i}\hat{F}_{1},
\end{eqnarray}

\begin{eqnarray*}
\begin{array}{ll}
D_{1}\hat{F}_{1}=D_{3}\hat{F}_{1}D\hat{\theta}_{1}D_{1}P_{1}=\hat{f}_{1}\left(3\right)\frac{1}{1-\hat{\mu}_{1}}\tilde{\mu}_{1},&
D_{2}\hat{F}_{1}=D_{3}\hat{F}_{1}D\hat{\theta}_{1}D_{2}P_{2}
=\hat{f}_{1}\left(3\right)\frac{1}{1-\hat{\mu}_{1}}\tilde{\mu}_{2}\\
D_{3}\hat{F}_{1}=0,&
D_{4}\hat{F}_{1}=D_{3}\hat{F}_{1}D\hat{\theta}_{1}D_{4}P_{4}
+D_{4}\hat{F}_{1}
=\hat{f}_{1}\left(3\right)\frac{1}{1-\hat{\mu}_{1}}\hat{\mu}_{2}+\hat{f}_{1}\left(2\right),

\end{array}
\end{eqnarray*}


\begin{eqnarray}
D_{i}\hat{F}_{2}&=&\indora_{i\neq4}D_{4}\hat{F}_{2}D\hat{\theta}_{2}D_{i}P_{i}+\indora_{i=3}D_{i}\hat{F}_{2}.
\end{eqnarray}

\begin{eqnarray*}
\begin{array}{ll}
D_{1}\hat{F}_{2}=D_{4}\hat{F}_{2}D\hat{\theta}_{2}D_{1}P_{1}
=\hat{f}_{2}\left(4\right)\frac{1}{1-\hat{\mu}_{2}}\tilde{\mu}_{1},&
D_{2}\hat{F}_{2}=D_{4}\hat{F}_{2}D\hat{\theta}_{2}D_{2}P_{2}
=\hat{f}_{2}\left(4\right)\frac{1}{1-\hat{\mu}_{2}}\tilde{\mu}_{2},\\
D_{3}\hat{F}_{2}=D_{4}\hat{F}_{2}D\hat{\theta}_{2}D_{3}P_{3}+D_{3}\hat{F}_{2}
=\hat{f}_{2}\left(4\right)\frac{1}{1-\hat{\mu}_{2}}\hat{\mu}_{1}+\hat{f}_{2}\left(4\right)\\
D_{4}\hat{F}_{2}=0

\end{array}
\end{eqnarray*}
Then, now we can obtain the linear system of equations in order to obtain the first moments of the lengths of the queues:



For $\mathbf{F}_{1}=R_{2}F_{2}\hat{F}_{2}$ we get the general equations

\begin{eqnarray}
D_{i}\mathbf{F}_{1}=D_{i}\left(R_{2}+F_{2}+\indora_{i\geq3}\hat{F}_{2}\right)
\end{eqnarray}

So

\begin{eqnarray*}
D_{1}\mathbf{F}_{1}&=&D_{1}R_{2}+D_{1}F_{2}
=r_{1}\tilde{\mu}_{1}+f_{2}\left(2\right)\frac{1}{1-\tilde{\mu}_{2}}\tilde{\mu}_{1}\\
D_{2}\mathbf{F}_{1}&=&D_{2}\left(R_{2}+F_{2}\right)
=r_{2}\tilde{\mu}_{1}\\
D_{3}\mathbf{F}_{1}&=&D_{3}\left(R_{2}+F_{2}+\hat{F}_{2}\right)
=r_{1}\hat{\mu}_{1}+f_{2}\left(2\right)\frac{1}{1-\tilde{\mu}_{2}}\hat{\mu}_{1}+\hat{F}_{1,2}^{(1)}\left(1\right)\\
D_{4}\mathbf{F}_{1}&=&D_{4}\left(R_{2}+F_{2}+\hat{F}_{2}\right)
=r_{2}\hat{\mu}_{2}+f_{2}\left(2\right)\frac{1}{1-\tilde{\mu}_{2}}\hat{\mu}_{2}
+\hat{F}_{2,2}^{(1)}\left(1\right)
\end{eqnarray*}

it means

\begin{eqnarray*}
\begin{array}{ll}
D_{1}\mathbf{F}_{1}=r_{2}\hat{\mu}_{1}+f_{2}\left(2\right)\left(\frac{1}{1-\tilde{\mu}_{2}}\right)\tilde{\mu}_{1}+f_{2}\left(1\right),&
D_{2}\mathbf{F}_{1}=r_{2}\tilde{\mu}_{2},\\
D_{3}\mathbf{F}_{1}=r_{2}\hat{\mu}_{1}+f_{2}\left(2\right)\left(\frac{1}{1-\tilde{\mu}_{2}}\right)\hat{\mu}_{1}+\hat{F}_{1,2}^{(1)}\left(1\right),&
D_{4}\mathbf{F}_{1}=r_{2}\hat{\mu}_{2}+f_{2}\left(2\right)\left(\frac{1}{1-\tilde{\mu}_{2}}\right)\hat{\mu}_{2}+\hat{F}_{2,2}^{(1)}\left(1\right),\end{array}
\end{eqnarray*}


\begin{eqnarray}
\begin{array}{ll}
\mathbf{F}_{2}=R_{1}F_{1}\hat{F}_{1}, & D_{i}\mathbf{F}_{2}=D_{i}\left(R_{1}+F_{1}+\indora_{i\geq3}\hat{F}_{1}\right)\\
\end{array}
\end{eqnarray}



equivalently


\begin{eqnarray*}
\begin{array}{ll}
D_{1}\mathbf{F}_{2}=r_{1}\tilde{\mu}_{1},&
D_{2}\mathbf{F}_{2}=r_{1}\tilde{\mu}_{2}+f_{1}\left(1\right)\left(\frac{1}{1-\tilde{\mu}_{1}}\right)\tilde{\mu}_{2}+f_{1}\left(2\right),\\
D_{3}\mathbf{F}_{2}=r_{1}\hat{\mu}_{1}+f_{1}\left(1\right)\left(\frac{1}{1-\tilde{\mu}_{1}}\right)\hat{\mu}_{1}+\hat{F}_{1,1}^{(1)}\left(1\right),&
D_{4}\mathbf{F}_{2}=r_{1}\hat{\mu}_{2}+f_{1}\left(1\right)\left(\frac{1}{1-\tilde{\mu}_{1}}\right)\hat{\mu}_{2}+\hat{F}_{2,1}^{(1)}\left(1\right),\\
\end{array}
\end{eqnarray*}



\begin{eqnarray}
\begin{array}{ll}
\hat{\mathbf{F}}_{1}=\hat{R}_{2}\hat{F}_{2}F_{2}, & D_{i}\hat{\mathbf{F}}_{1}=D_{i}\left(\hat{R}_{2}+\hat{F}_{2}+\indora_{i\leq2}F_{2}\right)\\
\end{array}
\end{eqnarray}


equivalently


\begin{eqnarray*}
\begin{array}{ll}
D_{1}\hat{\mathbf{F}}_{1}=\hat{r}_{2}\tilde{\mu}_{1}+\hat{f}_{2}\left(2\right)\left(\frac{1}{1-\hat{\mu}_{2}}\right)\tilde{\mu}_{1}+F_{1,2}^{(1)}\left(1\right),&
D_{2}\hat{\mathbf{F}}_{1}=\hat{r}_{2}\tilde{\mu}_{2}+\hat{f}_{2}\left(2\right)\left(\frac{1}{1-\hat{\mu}_{2}}\right)\tilde{\mu}_{2}+F_{2,2}^{(1)}\left(1\right),\\
D_{3}\hat{\mathbf{F}}_{1}=\hat{r}_{2}\hat{\mu}_{1}+\hat{f}_{2}\left(2\right)\left(\frac{1}{1-\hat{\mu}_{2}}\right)\hat{\mu}_{1}+\hat{f}_{2}\left(1\right),&
D_{4}\hat{\mathbf{F}}_{1}=\hat{r}_{2}\hat{\mu}_{2}
\end{array}
\end{eqnarray*}



\begin{eqnarray}
\begin{array}{ll}
\hat{\mathbf{F}}_{2}=\hat{R}_{1}\hat{F}_{1}F_{1}, & D_{i}\hat{\mathbf{F}}_{2}=D_{i}\left(\hat{R}_{1}+\hat{F}_{1}+\indora_{i\leq2}F_{1}\right)
\end{array}
\end{eqnarray}



equivalently


\begin{eqnarray*}
\begin{array}{ll}
D_{1}\hat{\mathbf{F}}_{2}=\hat{r}_{1}\tilde{\mu}_{1}+\hat{f}_{1}\left(1\right)\left(\frac{1}{1-\hat{\mu}_{1}}\right)\tilde{\mu}_{1}+F_{1,1}^{(1)}\left(1\right),&
D_{2}\hat{\mathbf{F}}_{2}=\hat{r}_{1}\mu_{2}+\hat{f}_{1}\left(1\right)\left(\frac{1}{1-\hat{\mu}_{1}}\right)\tilde{\mu}_{2}+F_{2,1}^{(1)}\left(1\right),\\
D_{3}\hat{\mathbf{F}}_{2}=\hat{r}_{1}\hat{\mu}_{1},&
D_{4}\hat{\mathbf{F}}_{2}=\hat{r}_{1}\hat{\mu}_{2}+\hat{f}_{1}\left(1\right)\left(\frac{1}{1-\hat{\mu}_{1}}\right)\hat{\mu}_{2}+\hat{f}_{1}\left(2\right),\\
\end{array}
\end{eqnarray*}





Then we have that if $\mu=\tilde{\mu}_{1}+\tilde{\mu}_{2}$, $\hat{\mu}=\hat{\mu}_{1}+\hat{\mu}_{2}$, $r=r_{1}+r_{2}$ and $\hat{r}=\hat{r}_{1}+\hat{r}_{2}$  the system's solution is given by

\begin{eqnarray*}
\begin{array}{llll}
f_{2}\left(1\right)=r_{1}\tilde{\mu}_{1},&
f_{1}\left(2\right)=r_{2}\tilde{\mu}_{2},&
\hat{f}_{1}\left(4\right)=\hat{r}_{2}\hat{\mu}_{2},&
\hat{f}_{2}\left(3\right)=\hat{r}_{1}\hat{\mu}_{1}
\end{array}
\end{eqnarray*}



it's easy to verify that

\begin{eqnarray}\label{Sist.Ec.Lineales.Doble.Traslado}
\begin{array}{ll}
f_{1}\left(1\right)=\tilde{\mu}_{1}\left(r+\frac{f_{2}\left(2\right)}{1-\tilde{\mu}_{2}}\right),& f_{1}\left(3\right)=\hat{\mu}_{1}\left(r_{2}+\frac{f_{2}\left(2\right)}{1-\tilde{\mu}_{2}}\right)+\hat{F}_{1,2}^{(1)}\left(1\right)\\
f_{1}\left(4\right)=\hat{\mu}_{2}\left(r_{2}+\frac{f_{2}\left(2\right)}{1-\tilde{\mu}_{2}}\right)+\hat{F}_{2,2}^{(1)}\left(1\right),&
f_{2}\left(2\right)=\left(r+\frac{f_{1}\left(1\right)}{1-\mu_{1}}\right)\tilde{\mu}_{2},\\
f_{2}\left(3\right)=\hat{\mu}_{1}\left(r_{1}+\frac{f_{1}\left(1\right)}{1-\tilde{\mu}_{1}}\right)+\hat{F}_{1,1}^{(1)}\left(1\right),&
f_{2}\left(4\right)=\hat{\mu}_{2}\left(r_{1}+\frac{f_{1}\left(1\right)}{1-\mu_{1}}\right)+\hat{F}_{2,1}^{(1)}\left(1\right),\\
\hat{f}_{1}\left(1\right)=\left(\hat{r}_{2}+\frac{\hat{f}_{2}\left(4\right)}{1-\hat{\mu}_{2}}\right)\tilde{\mu}_{1}+F_{1,2}^{(1)}\left(1\right),&
\hat{f}_{1}\left(2\right)=\left(\hat{r}_{2}+\frac{\hat{f}_{2}\left(4\right)}{1-\hat{\mu}_{2}}\right)\tilde{\mu}_{2}+F_{2,2}^{(1)}\left(1\right),\\
\hat{f}_{1}\left(3\right)=\left(\hat{r}+\frac{\hat{f}_{2}\left(4\right)}{1-\hat{\mu}_{2}}\right)\hat{\mu}_{1},&
\hat{f}_{2}\left(1\right)=\left(\hat{r}_{1}+\frac{\hat{f}_{1}\left(3\right)}{1-\hat{\mu}_{1}}\right)\mu_{1}+F_{1,1}^{(1)}\left(1\right),\\
\hat{f}_{2}\left(2\right)=\left(\hat{r}_{1}+\frac{\hat{f}_{1}\left(3\right)}{1-\hat{\mu}_{1}}\right)\tilde{\mu}_{2}+F_{2,1}^{(1)}\left(1\right),&
\hat{f}_{2}\left(4\right)=\left(\hat{r}+\frac{\hat{f}_{1}\left(3\right)}{1-\hat{\mu}_{1}}\right)\hat{\mu}_{2},\\
\end{array}
\end{eqnarray}

with system's solutions given by

\begin{eqnarray}
\begin{array}{ll}
f_{1}\left(1\right)=r\frac{\mu_{1}\left(1-\mu_{1}\right)}{1-\mu},&
f_{2}\left(2\right)=r\frac{\tilde{\mu}_{2}\left(1-\tilde{\mu}_{2}\right)}{1-\mu},\\
f_{1}\left(3\right)=\hat{\mu}_{1}\left(r_{2}+\frac{r\tilde{\mu}_{2}}{1-\mu}\right)+\hat{F}_{1,2}^{(1)}\left(1\right),&
f_{1}\left(4\right)=\hat{\mu}_{2}\left(r_{2}+\frac{r\tilde{\mu}_{2}}{1-\mu}\right)+\hat{F}_{2,2}^{(1)}\left(1\right),\\
f_{2}\left(3\right)=\hat{\mu}_{1}\left(r_{1}+\frac{r\mu_{1}}{1-\mu}\right)+\hat{F}_{1,1}^{(1)}\left(1\right),&
f_{2}\left(4\right)=\hat{\mu}_{2}\left(r_{1}+\frac{r\mu_{1}}{1-\mu}\right)+\hat{F}_{2,1}^{(1)}\left(1\right),\\
\hat{f}_{1}\left(1\right)=\tilde{\mu}_{1}\left(\hat{r}_{2}+\frac{\hat{r}\hat{\mu}_{2}}{1-\hat{\mu}}\right)+F_{1,2}^{(1)}\left(1\right),&
\hat{f}_{1}\left(2\right)=\tilde{\mu}_{2}\left(\hat{r}_{2}+\frac{\hat{r}\hat{\mu}_{2}}{1-\hat{\mu}}\right)+F_{2,2}^{(1)}\left(1\right),\\
\hat{f}_{2}\left(1\right)=\tilde{\mu}_{1}\left(\hat{r}_{1}+\frac{\hat{r}\hat{\mu}_{1}}{1-\hat{\mu}}\right)+F_{1,1}^{(1)}\left(1\right),&
\hat{f}_{2}\left(2\right)=\tilde{\mu}_{2}\left(\hat{r}_{1}+\frac{\hat{r}\hat{\mu}_{1}}{1-\hat{\mu}}\right)+F_{2,1}^{(1)}\left(1\right)
\end{array}
\end{eqnarray}

%_________________________________________________________________________________________________________
\subsection{General Second Order Derivatives}
%_________________________________________________________________________________________________________


Now, taking the second order derivative with respect to the equations given in (\ref{Sist.Ec.Lineales.Doble.Traslado}) we obtain it in their general form

\small{
\begin{eqnarray*}\label{Ec.Derivadas.Segundo.Orden.Doble.Transferencia}
D_{k}D_{i}F_{1}&=&D_{k}D_{i}\left(R_{2}+F_{2}+\indora_{i\geq3}\hat{F}_{4}\right)+D_{i}R_{2}D_{k}\left(F_{2}+\indora_{k\geq3}\hat{F}_{4}\right)+D_{i}F_{2}D_{k}\left(R_{2}+\indora_{k\geq3}\hat{F}_{4}\right)+\indora_{i\geq3}D_{i}\hat{F}_{4}D_{k}\left(R_{2}+F_{2}\right)\\
D_{k}D_{i}F_{2}&=&D_{k}D_{i}\left(R_{1}+F_{1}+\indora_{i\geq3}\hat{F}_{3}\right)+D_{i}R_{1}D_{k}\left(F_{1}+\indora_{k\geq3}\hat{F}_{3}\right)+D_{i}F_{1}D_{k}\left(R_{1}+\indora_{k\geq3}\hat{F}_{3}\right)+\indora_{i\geq3}D_{i}\hat{F}_{3}D_{k}\left(R_{1}+F_{1}\right)\\
D_{k}D_{i}\hat{F}_{3}&=&D_{k}D_{i}\left(\hat{R}_{4}+\indora_{i\leq2}F_{2}+\hat{F}_{4}\right)+D_{i}\hat{R}_{4}D_{k}\left(\indora_{k\leq2}F_{2}+\hat{F}_{4}\right)+D_{i}\hat{F}_{4}D_{k}\left(\hat{R}_{4}+\indora_{k\leq2}F_{2}\right)+\indora_{i\leq2}D_{i}F_{2}D_{k}\left(\hat{R}_{4}+\hat{F}_{4}\right)\\
D_{k}D_{i}\hat{F}_{4}&=&D_{k}D_{i}\left(\hat{R}_{3}+\indora_{i\leq2}F_{1}+\hat{F}_{3}\right)+D_{i}\hat{R}_{3}D_{k}\left(\indora_{k\leq2}F_{1}+\hat{F}_{3}\right)+D_{i}\hat{F}_{3}D_{k}\left(\hat{R}_{3}+\indora_{k\leq2}F_{1}\right)+\indora_{i\leq2}D_{i}F_{1}D_{k}\left(\hat{R}_{3}+\hat{F}_{3}\right)
\end{eqnarray*}}
for $i,k=1,\ldots,4$. In order to have it in an specific way we need to compute the expressions $D_{k}D_{i}\left(R_{2}+F_{2}+\indora_{i\geq3}\hat{F}_{4}\right)$

%_________________________________________________________________________________________________________
\subsubsection{Second Order Derivatives: Serve's Switchover Times}
%_________________________________________________________________________________________________________

Remind $R_{i}\left(z_{1},z_{2},w_{1},w_{2}\right)=R_{i}\left(P_{1}\left(z_{1}\right)\tilde{P}_{2}\left(z_{2}\right)
\hat{P}_{1}\left(w_{1}\right)\hat{P}_{2}\left(w_{2}\right)\right)$,  which we will write in his reduced form $R_{i}=R_{i}\left(
P_{1}\tilde{P}_{2}\hat{P}_{1}\hat{P}_{2}\right)$, and according to the notation given in \cite{Lang} we obtain

\begin{eqnarray}
D_{i}D_{i}R_{k}=D^{2}R_{k}\left(D_{i}P_{i}\right)^{2}+DR_{k}D_{i}D_{i}P_{i}
\end{eqnarray}

whereas for $i\neq j$

\begin{eqnarray}
D_{i}D_{j}R_{k}=D^{2}R_{k}D_{i}P_{i}D_{j}P_{j}+DR_{k}D_{j}P_{j}D_{i}P_{i}
\end{eqnarray}

%_________________________________________________________________________________________________________
\subsubsection{Second Order Derivatives: Queue Lengths}
%_________________________________________________________________________________________________________

Just like before the expression $F_{1}\left(\tilde{\theta}_{1}\left(\tilde{P}_{2}\left(z_{2}\right)\hat{P}_{1}\left(w_{1}\right)\hat{P}_{2}\left(w_{2}\right)\right),
z_{2}\right)$, will be denoted by $F_{1}\left(\tilde{\theta}_{1}\left(\tilde{P}_{2}\hat{P}_{1}\hat{P}_{2}\right),z_{2}\right)$, then the mixed partial derivatives are:

\begin{eqnarray*}
D_{j}D_{i}F_{1}&=&\indora_{i,j\neq1}D_{1}D_{1}F_{1}\left(D\tilde{\theta}_{1}\right)^{2}D_{i}P_{i}D_{j}P_{j}
+\indora_{i,j\neq1}D_{1}F_{1}D^{2}\tilde{\theta}_{1}D_{i}P_{i}D_{j}P_{j}
+\indora_{i,j\neq1}D_{1}F_{1}D\tilde{\theta}_{1}\left(\indora_{i=j}D_{i}^{2}P_{i}+\indora_{i\neq j}D_{i}P_{i}D_{j}P_{j}\right)\\
&+&\left(1-\indora_{i=j=3}\right)\indora_{i+j\leq6}D_{1}D_{2}F_{1}D\tilde{\theta}_{1}\left(\indora_{i\leq j}D_{j}P_{j}+\indora_{i>j}D_{i}P_{i}\right)
+\indora_{i=2}\left(D_{1}D_{2}F_{1}D\tilde{\theta}_{1}D_{i}P_{i}+D_{i}^{2}F_{1}\right)
\end{eqnarray*}


Recall the expression for $F_{1}\left(\tilde{\theta}_{1}\left(\tilde{P}_{2}\left(z_{2}\right)\hat{P}_{1}\left(w_{1}\right)\hat{P}_{2}\left(w_{2}\right)\right),
z_{2}\right)$, which is denoted by $F_{1}\left(\tilde{\theta}_{1}\left(\tilde{P}_{2}\hat{P}_{1}\hat{P}_{2}\right),z_{2}\right)$, then the mixed partial derivatives are given by

\begin{eqnarray*}
\begin{array}{llll}
D_{1}D_{1}F_{1}=0,&
D_{2}D_{1}F_{1}=0,&
D_{3}D_{1}F_{1}=0,&
D_{4}D_{1}F_{1}=0,\\
D_{1}D_{2}F_{1}=0,&
D_{1}D_{3}F_{1}=0,&
D_{1}D_{4}F_{1}=0,&
\end{array}
\end{eqnarray*}

%_____________________________________________________________________________________
\newpage
%__________________________________________________________________
\section{Generalizaciones}
%__________________________________________________________________
\subsection{RSVC con dos conexiones}
%__________________________________________________________________

%\begin{figure}[H]
%\centering
%%%\includegraphics[width=9cm]{Grafica3.jpg}
%%\end{figure}\label{RSVC3}


Sus ecuaciones recursivas son de la forma


\begin{eqnarray*}
F_{1}\left(z_{1},z_{2},w_{1},w_{2}\right)&=&R_{2}\left(\prod_{i=1}^{2}\tilde{P}_{i}\left(z_{i}\right)\prod_{i=1}^{2}
\hat{P}_{i}\left(w_{i}\right)\right)F_{2}\left(z_{1},\tilde{\theta}_{2}\left(\tilde{P}_{1}\left(z_{1}\right)\hat{P}_{1}\left(w_{1}\right)\hat{P}_{2}\left(w_{2}\right)\right)\right)
\hat{F}_{2}\left(w_{1},w_{2};\tau_{2}\right),
\end{eqnarray*}

\begin{eqnarray*}
F_{2}\left(z_{1},z_{2},w_{1},w_{2}\right)&=&R_{1}\left(\prod_{i=1}^{2}\tilde{P}_{i}\left(z_{i}\right)\prod_{i=1}^{2}
\hat{P}_{i}\left(w_{i}\right)\right)F_{1}\left(\tilde{\theta}_{1}\left(\tilde{P}_{2}\left(z_{2}\right)\hat{P}_{1}\left(w_{1}\right)\hat{P}_{2}\left(w_{2}\right)\right),z_{2}\right)\hat{F}_{1}\left(w_{1},w_{2};\tau_{1}\right),
\end{eqnarray*}


\begin{eqnarray*}
\hat{F}_{1}\left(z_{1},z_{2},w_{1},w_{2}\right)&=&\hat{R}_{2}\left(\prod_{i=1}^{2}\tilde{P}_{i}\left(z_{i}\right)\prod_{i=1}^{2}
\hat{P}_{i}\left(w_{i}\right)\right)F_{2}\left(z_{1},z_{2};\zeta_{2}\right)\hat{F}_{2}\left(w_{1},\hat{\theta}_{2}\left(\tilde{P}_{1}\left(z_{1}\right)\tilde{P}_{2}\left(z_{2}\right)\hat{P}_{1}\left(w_{1}
\right)\right)\right),
\end{eqnarray*}


\begin{eqnarray*}
\hat{F}_{2}\left(z_{1},z_{2},w_{1},w_{2}\right)&=&\hat{R}_{1}\left(\prod_{i=1}^{2}\tilde{P}_{i}\left(z_{i}\right)\prod_{i=1}^{2}
\hat{P}_{i}\left(w_{i}\right)\right)F_{1}\left(z_{1},z_{2};\zeta_{1}\right)\hat{F}_{1}\left(\hat{\theta}_{1}\left(\tilde{P}_{1}\left(z_{1}\right)\tilde{P}_{2}\left(z_{2}\right)\hat{P}_{2}\left(w_{2}\right)\right),w_{2}\right),
\end{eqnarray*}

%_____________________________________________________
\subsection{First Moments of the Queue Lengths}
%_____________________________________________________


The server's switchover times are given by the general equation

\begin{eqnarray}\label{Ec.Ri}
R_{i}\left(\mathbf{z,w}\right)=R_{i}\left(\tilde{P}_{1}\left(z_{1}\right)\tilde{P}_{2}\left(z_{2}\right)\hat{P}_{1}\left(w_{1}\right)\hat{P}_{2}\left(w_{2}\right)\right)
\end{eqnarray}

with
\begin{eqnarray}\label{Ec.Derivada.Ri}
D_{i}R_{i}&=&DR_{i}D_{i}P_{i}
\end{eqnarray}
the following notation is considered

\begin{eqnarray*}
\begin{array}{llll}
D_{1}P_{1}\equiv D_{1}\tilde{P}_{1}, & D_{2}P_{2}\equiv D_{2}\tilde{P}_{2}, & D_{3}P_{3}\equiv D_{3}\hat{P}_{1}, &D_{4}P_{4}\equiv D_{4}\hat{P}_{2},
\end{array}
\end{eqnarray*}

also we need to remind $F_{1,2}\left(z_{1};\zeta_{2}\right)F_{2,2}\left(z_{2};\zeta_{2}\right)=F_{2}\left(z_{1},z_{2};\zeta_{2}\right)$, therefore

\begin{eqnarray*}
D_{1}F_{2}\left(z_{1},z_{2};\zeta_{2}\right)&=&D_{1}\left[F_{1,2}\left(z_{1};\zeta_{2}\right)F_{2,2}\left(z_{2};\zeta_{2}\right)\right]
=F_{2,2}\left(z_{2};\zeta_{2}\right)D_{1}F_{1,2}\left(z_{1};\zeta_{2}\right)=F_{1,2}^{(1)}\left(1\right)
\end{eqnarray*}

i.e., $D_{1}F_{2}=F_{1,2}^{(1)}(1)$; $D_{2}F_{2}=F_{2,2}^{(1)}\left(1\right)$, whereas that $D_{3}F_{2}=D_{4}F_{2}=0$, then

\begin{eqnarray}
\begin{array}{ccc}
D_{i}F_{j}=\indora_{i\leq2}F_{i,j}^{(1)}\left(1\right),& \textrm{ and } &D_{i}\hat{F}_{j}=\indora_{i\geq2}F_{i,j}^{(1)}\left(1\right).
\end{array}
\end{eqnarray}

Now, we obtain the first moments equations for the queue lengths as before for the times the server arrives to the queue to start attending



Remember that


\begin{eqnarray*}
F_{2}\left(z_{1},z_{2},w_{1},w_{2}\right)&=&R_{1}\left(\prod_{i=1}^{2}\tilde{P}_{i}\left(z_{i}\right)\prod_{i=1}^{2}
\hat{P}_{i}\left(w_{i}\right)\right)F_{1}\left(\tilde{\theta}_{1}\left(\tilde{P}_{2}\left(z_{2}\right)\hat{P}_{1}\left(w_{1}\right)\hat{P}_{2}\left(w_{2}\right)\right),z_{2}\right)\hat{F}_{1}\left(w_{1},w_{2};\tau_{1}\right),
\end{eqnarray*}

where


\begin{eqnarray*}
F_{1}\left(\tilde{\theta}_{1}\left(\tilde{P}_{2}\hat{P}_{1}\hat{P}_{2}\right),z_{2}\right)
\end{eqnarray*}

so

\begin{eqnarray}
D_{i}F_{1}&=&\indora_{i\neq1}D_{1}F_{1}D\tilde{\theta}_{1}D_{i}P_{i}+\indora_{i=2}D_{i}F_{1},
\end{eqnarray}

then


\begin{eqnarray*}
\begin{array}{ll}
D_{1}F_{1}=0,&
D_{2}F_{1}=D_{1}F_{1}D\tilde{\theta}_{1}D_{2}P_{2}+D_{2}F_{1}
=f_{1}\left(1\right)\frac{1}{1-\tilde{\mu}_{1}}\tilde{\mu}_{2}+f_{1}\left(2\right),\\
D_{3}F_{1}=D_{1}F_{1}D\tilde{\theta}_{1}D_{3}P_{3}
=f_{1}\left(1\right)\frac{1}{1-\tilde{\mu}_{1}}\hat{\mu}_{1},&
D_{4}F_{1}=D_{1}F_{1}D\tilde{\theta}_{1}D_{4}P_{4}
=f_{1}\left(1\right)\frac{1}{1-\tilde{\mu}_{1}}\hat{\mu}_{2}

\end{array}
\end{eqnarray*}


\begin{eqnarray}
D_{i}F_{2}&=&\indora_{i\neq2}D_{2}F_{2}D\tilde{\theta}_{2}D_{i}P_{i}
+\indora_{i=1}D_{i}F_{2}
\end{eqnarray}

\begin{eqnarray*}
\begin{array}{ll}
D_{1}F_{2}=D_{2}F_{2}D\tilde{\theta}_{2}D_{1}P_{1}
+D_{1}F_{2}=f_{2}\left(2\right)\frac{1}{1-\tilde{\mu}_{2}}\tilde{\mu}_{1},&
D_{2}F_{2}=0\\
D_{3}F_{2}=D_{2}F_{2}D\tilde{\theta}_{2}D_{3}P_{3}
=f_{2}\left(2\right)\frac{1}{1-\tilde{\mu}_{2}}\hat{\mu}_{1},&
D_{4}F_{2}=D_{2}F_{2}D\tilde{\theta}_{2}D_{4}P_{4}
=f_{2}\left(2\right)\frac{1}{1-\tilde{\mu}_{2}}\hat{\mu}_{2}
\end{array}
\end{eqnarray*}



\begin{eqnarray}
D_{i}\hat{F}_{1}&=&\indora_{i\neq3}D_{3}\hat{F}_{1}D\hat{\theta}_{1}D_{i}P_{i}+\indora_{i=4}D_{i}\hat{F}_{1},
\end{eqnarray}

\begin{eqnarray*}
\begin{array}{ll}
D_{1}\hat{F}_{1}=D_{3}\hat{F}_{1}D\hat{\theta}_{1}D_{1}P_{1}=\hat{f}_{1}\left(3\right)\frac{1}{1-\hat{\mu}_{1}}\tilde{\mu}_{1},&
D_{2}\hat{F}_{1}=D_{3}\hat{F}_{1}D\hat{\theta}_{1}D_{2}P_{2}
=\hat{f}_{1}\left(3\right)\frac{1}{1-\hat{\mu}_{1}}\tilde{\mu}_{2}\\
D_{3}\hat{F}_{1}=0,&
D_{4}\hat{F}_{1}=D_{3}\hat{F}_{1}D\hat{\theta}_{1}D_{4}P_{4}
+D_{4}\hat{F}_{1}
=\hat{f}_{1}\left(3\right)\frac{1}{1-\hat{\mu}_{1}}\hat{\mu}_{2}+\hat{f}_{1}\left(2\right),

\end{array}
\end{eqnarray*}


\begin{eqnarray}
D_{i}\hat{F}_{2}&=&\indora_{i\neq4}D_{4}\hat{F}_{2}D\hat{\theta}_{2}D_{i}P_{i}+\indora_{i=3}D_{i}\hat{F}_{2}.
\end{eqnarray}

\begin{eqnarray*}
\begin{array}{ll}
D_{1}\hat{F}_{2}=D_{4}\hat{F}_{2}D\hat{\theta}_{2}D_{1}P_{1}
=\hat{f}_{2}\left(4\right)\frac{1}{1-\hat{\mu}_{2}}\tilde{\mu}_{1},&
D_{2}\hat{F}_{2}=D_{4}\hat{F}_{2}D\hat{\theta}_{2}D_{2}P_{2}
=\hat{f}_{2}\left(4\right)\frac{1}{1-\hat{\mu}_{2}}\tilde{\mu}_{2},\\
D_{3}\hat{F}_{2}=D_{4}\hat{F}_{2}D\hat{\theta}_{2}D_{3}P_{3}+D_{3}\hat{F}_{2}
=\hat{f}_{2}\left(4\right)\frac{1}{1-\hat{\mu}_{2}}\hat{\mu}_{1}+\hat{f}_{2}\left(4\right)\\
D_{4}\hat{F}_{2}=0

\end{array}
\end{eqnarray*}
Then, now we can obtain the linear system of equations in order to obtain the first moments of the lengths of the queues:



For $\mathbf{F}_{1}=R_{2}F_{2}\hat{F}_{2}$ we get the general equations

\begin{eqnarray}
D_{i}\mathbf{F}_{1}=D_{i}\left(R_{2}+F_{2}+\indora_{i\geq3}\hat{F}_{2}\right)
\end{eqnarray}

So

\begin{eqnarray*}
D_{1}\mathbf{F}_{1}&=&D_{1}R_{2}+D_{1}F_{2}
=r_{1}\tilde{\mu}_{1}+f_{2}\left(2\right)\frac{1}{1-\tilde{\mu}_{2}}\tilde{\mu}_{1}\\
D_{2}\mathbf{F}_{1}&=&D_{2}\left(R_{2}+F_{2}\right)
=r_{2}\tilde{\mu}_{1}\\
D_{3}\mathbf{F}_{1}&=&D_{3}\left(R_{2}+F_{2}+\hat{F}_{2}\right)
=r_{1}\hat{\mu}_{1}+f_{2}\left(2\right)\frac{1}{1-\tilde{\mu}_{2}}\hat{\mu}_{1}+\hat{F}_{1,2}^{(1)}\left(1\right)\\
D_{4}\mathbf{F}_{1}&=&D_{4}\left(R_{2}+F_{2}+\hat{F}_{2}\right)
=r_{2}\hat{\mu}_{2}+f_{2}\left(2\right)\frac{1}{1-\tilde{\mu}_{2}}\hat{\mu}_{2}
+\hat{F}_{2,2}^{(1)}\left(1\right)
\end{eqnarray*}

it means

\begin{eqnarray*}
\begin{array}{ll}
D_{1}\mathbf{F}_{1}=r_{2}\hat{\mu}_{1}+f_{2}\left(2\right)\left(\frac{1}{1-\tilde{\mu}_{2}}\right)\tilde{\mu}_{1}+f_{2}\left(1\right),&
D_{2}\mathbf{F}_{1}=r_{2}\tilde{\mu}_{2},\\
D_{3}\mathbf{F}_{1}=r_{2}\hat{\mu}_{1}+f_{2}\left(2\right)\left(\frac{1}{1-\tilde{\mu}_{2}}\right)\hat{\mu}_{1}+\hat{F}_{1,2}^{(1)}\left(1\right),&
D_{4}\mathbf{F}_{1}=r_{2}\hat{\mu}_{2}+f_{2}\left(2\right)\left(\frac{1}{1-\tilde{\mu}_{2}}\right)\hat{\mu}_{2}+\hat{F}_{2,2}^{(1)}\left(1\right),\end{array}
\end{eqnarray*}


\begin{eqnarray}
\begin{array}{ll}
\mathbf{F}_{2}=R_{1}F_{1}\hat{F}_{1}, & D_{i}\mathbf{F}_{2}=D_{i}\left(R_{1}+F_{1}+\indora_{i\geq3}\hat{F}_{1}\right)\\
\end{array}
\end{eqnarray}



equivalently


\begin{eqnarray*}
\begin{array}{ll}
D_{1}\mathbf{F}_{2}=r_{1}\tilde{\mu}_{1},&
D_{2}\mathbf{F}_{2}=r_{1}\tilde{\mu}_{2}+f_{1}\left(1\right)\left(\frac{1}{1-\tilde{\mu}_{1}}\right)\tilde{\mu}_{2}+f_{1}\left(2\right),\\
D_{3}\mathbf{F}_{2}=r_{1}\hat{\mu}_{1}+f_{1}\left(1\right)\left(\frac{1}{1-\tilde{\mu}_{1}}\right)\hat{\mu}_{1}+\hat{F}_{1,1}^{(1)}\left(1\right),&
D_{4}\mathbf{F}_{2}=r_{1}\hat{\mu}_{2}+f_{1}\left(1\right)\left(\frac{1}{1-\tilde{\mu}_{1}}\right)\hat{\mu}_{2}+\hat{F}_{2,1}^{(1)}\left(1\right),\\
\end{array}
\end{eqnarray*}



\begin{eqnarray}
\begin{array}{ll}
\hat{\mathbf{F}}_{1}=\hat{R}_{2}\hat{F}_{2}F_{2}, & D_{i}\hat{\mathbf{F}}_{1}=D_{i}\left(\hat{R}_{2}+\hat{F}_{2}+\indora_{i\leq2}F_{2}\right)\\
\end{array}
\end{eqnarray}


equivalently


\begin{eqnarray*}
\begin{array}{ll}
D_{1}\hat{\mathbf{F}}_{1}=\hat{r}_{2}\tilde{\mu}_{1}+\hat{f}_{2}\left(2\right)\left(\frac{1}{1-\hat{\mu}_{2}}\right)\tilde{\mu}_{1}+F_{1,2}^{(1)}\left(1\right),&
D_{2}\hat{\mathbf{F}}_{1}=\hat{r}_{2}\tilde{\mu}_{2}+\hat{f}_{2}\left(2\right)\left(\frac{1}{1-\hat{\mu}_{2}}\right)\tilde{\mu}_{2}+F_{2,2}^{(1)}\left(1\right),\\
D_{3}\hat{\mathbf{F}}_{1}=\hat{r}_{2}\hat{\mu}_{1}+\hat{f}_{2}\left(2\right)\left(\frac{1}{1-\hat{\mu}_{2}}\right)\hat{\mu}_{1}+\hat{f}_{2}\left(1\right),&
D_{4}\hat{\mathbf{F}}_{1}=\hat{r}_{2}\hat{\mu}_{2}
\end{array}
\end{eqnarray*}



\begin{eqnarray}
\begin{array}{ll}
\hat{\mathbf{F}}_{2}=\hat{R}_{1}\hat{F}_{1}F_{1}, & D_{i}\hat{\mathbf{F}}_{2}=D_{i}\left(\hat{R}_{1}+\hat{F}_{1}+\indora_{i\leq2}F_{1}\right)
\end{array}
\end{eqnarray}



equivalently


\begin{eqnarray*}
\begin{array}{ll}
D_{1}\hat{\mathbf{F}}_{2}=\hat{r}_{1}\tilde{\mu}_{1}+\hat{f}_{1}\left(1\right)\left(\frac{1}{1-\hat{\mu}_{1}}\right)\tilde{\mu}_{1}+F_{1,1}^{(1)}\left(1\right),&
D_{2}\hat{\mathbf{F}}_{2}=\hat{r}_{1}\mu_{2}+\hat{f}_{1}\left(1\right)\left(\frac{1}{1-\hat{\mu}_{1}}\right)\tilde{\mu}_{2}+F_{2,1}^{(1)}\left(1\right),\\
D_{3}\hat{\mathbf{F}}_{2}=\hat{r}_{1}\hat{\mu}_{1},&
D_{4}\hat{\mathbf{F}}_{2}=\hat{r}_{1}\hat{\mu}_{2}+\hat{f}_{1}\left(1\right)\left(\frac{1}{1-\hat{\mu}_{1}}\right)\hat{\mu}_{2}+\hat{f}_{1}\left(2\right),\\
\end{array}
\end{eqnarray*}





Then we have that if $\mu=\tilde{\mu}_{1}+\tilde{\mu}_{2}$, $\hat{\mu}=\hat{\mu}_{1}+\hat{\mu}_{2}$, $r=r_{1}+r_{2}$ and $\hat{r}=\hat{r}_{1}+\hat{r}_{2}$  the system's solution is given by

\begin{eqnarray*}
\begin{array}{llll}
f_{2}\left(1\right)=r_{1}\tilde{\mu}_{1},&
f_{1}\left(2\right)=r_{2}\tilde{\mu}_{2},&
\hat{f}_{1}\left(4\right)=\hat{r}_{2}\hat{\mu}_{2},&
\hat{f}_{2}\left(3\right)=\hat{r}_{1}\hat{\mu}_{1}
\end{array}
\end{eqnarray*}



it's easy to verify that

\begin{eqnarray}\label{Sist.Ec.Lineales.Doble.Traslado}
\begin{array}{ll}
f_{1}\left(1\right)=\tilde{\mu}_{1}\left(r+\frac{f_{2}\left(2\right)}{1-\tilde{\mu}_{2}}\right),& f_{1}\left(3\right)=\hat{\mu}_{1}\left(r_{2}+\frac{f_{2}\left(2\right)}{1-\tilde{\mu}_{2}}\right)+\hat{F}_{1,2}^{(1)}\left(1\right)\\
f_{1}\left(4\right)=\hat{\mu}_{2}\left(r_{2}+\frac{f_{2}\left(2\right)}{1-\tilde{\mu}_{2}}\right)+\hat{F}_{2,2}^{(1)}\left(1\right),&
f_{2}\left(2\right)=\left(r+\frac{f_{1}\left(1\right)}{1-\mu_{1}}\right)\tilde{\mu}_{2},\\
f_{2}\left(3\right)=\hat{\mu}_{1}\left(r_{1}+\frac{f_{1}\left(1\right)}{1-\tilde{\mu}_{1}}\right)+\hat{F}_{1,1}^{(1)}\left(1\right),&
f_{2}\left(4\right)=\hat{\mu}_{2}\left(r_{1}+\frac{f_{1}\left(1\right)}{1-\mu_{1}}\right)+\hat{F}_{2,1}^{(1)}\left(1\right),\\
\hat{f}_{1}\left(1\right)=\left(\hat{r}_{2}+\frac{\hat{f}_{2}\left(4\right)}{1-\hat{\mu}_{2}}\right)\tilde{\mu}_{1}+F_{1,2}^{(1)}\left(1\right),&
\hat{f}_{1}\left(2\right)=\left(\hat{r}_{2}+\frac{\hat{f}_{2}\left(4\right)}{1-\hat{\mu}_{2}}\right)\tilde{\mu}_{2}+F_{2,2}^{(1)}\left(1\right),\\
\hat{f}_{1}\left(3\right)=\left(\hat{r}+\frac{\hat{f}_{2}\left(4\right)}{1-\hat{\mu}_{2}}\right)\hat{\mu}_{1},&
\hat{f}_{2}\left(1\right)=\left(\hat{r}_{1}+\frac{\hat{f}_{1}\left(3\right)}{1-\hat{\mu}_{1}}\right)\mu_{1}+F_{1,1}^{(1)}\left(1\right),\\
\hat{f}_{2}\left(2\right)=\left(\hat{r}_{1}+\frac{\hat{f}_{1}\left(3\right)}{1-\hat{\mu}_{1}}\right)\tilde{\mu}_{2}+F_{2,1}^{(1)}\left(1\right),&
\hat{f}_{2}\left(4\right)=\left(\hat{r}+\frac{\hat{f}_{1}\left(3\right)}{1-\hat{\mu}_{1}}\right)\hat{\mu}_{2},\\
\end{array}
\end{eqnarray}

with system's solutions given by

\begin{eqnarray}
\begin{array}{ll}
f_{1}\left(1\right)=r\frac{\mu_{1}\left(1-\mu_{1}\right)}{1-\mu},&
f_{2}\left(2\right)=r\frac{\tilde{\mu}_{2}\left(1-\tilde{\mu}_{2}\right)}{1-\mu},\\
f_{1}\left(3\right)=\hat{\mu}_{1}\left(r_{2}+\frac{r\tilde{\mu}_{2}}{1-\mu}\right)+\hat{F}_{1,2}^{(1)}\left(1\right),&
f_{1}\left(4\right)=\hat{\mu}_{2}\left(r_{2}+\frac{r\tilde{\mu}_{2}}{1-\mu}\right)+\hat{F}_{2,2}^{(1)}\left(1\right),\\
f_{2}\left(3\right)=\hat{\mu}_{1}\left(r_{1}+\frac{r\mu_{1}}{1-\mu}\right)+\hat{F}_{1,1}^{(1)}\left(1\right),&
f_{2}\left(4\right)=\hat{\mu}_{2}\left(r_{1}+\frac{r\mu_{1}}{1-\mu}\right)+\hat{F}_{2,1}^{(1)}\left(1\right),\\
\hat{f}_{1}\left(1\right)=\tilde{\mu}_{1}\left(\hat{r}_{2}+\frac{\hat{r}\hat{\mu}_{2}}{1-\hat{\mu}}\right)+F_{1,2}^{(1)}\left(1\right),&
\hat{f}_{1}\left(2\right)=\tilde{\mu}_{2}\left(\hat{r}_{2}+\frac{\hat{r}\hat{\mu}_{2}}{1-\hat{\mu}}\right)+F_{2,2}^{(1)}\left(1\right),\\
\hat{f}_{2}\left(1\right)=\tilde{\mu}_{1}\left(\hat{r}_{1}+\frac{\hat{r}\hat{\mu}_{1}}{1-\hat{\mu}}\right)+F_{1,1}^{(1)}\left(1\right),&
\hat{f}_{2}\left(2\right)=\tilde{\mu}_{2}\left(\hat{r}_{1}+\frac{\hat{r}\hat{\mu}_{1}}{1-\hat{\mu}}\right)+F_{2,1}^{(1)}\left(1\right)
\end{array}
\end{eqnarray}

%_________________________________________________________________________________________________________
\subsection{General Second Order Derivatives}
%_________________________________________________________________________________________________________


Now, taking the second order derivative with respect to the equations given in (\ref{Sist.Ec.Lineales.Doble.Traslado}) we obtain it in their general form

\small{
\begin{eqnarray*}\label{Ec.Derivadas.Segundo.Orden.Doble.Transferencia}
D_{k}D_{i}F_{1}&=&D_{k}D_{i}\left(R_{2}+F_{2}+\indora_{i\geq3}\hat{F}_{4}\right)+D_{i}R_{2}D_{k}\left(F_{2}+\indora_{k\geq3}\hat{F}_{4}\right)+D_{i}F_{2}D_{k}\left(R_{2}+\indora_{k\geq3}\hat{F}_{4}\right)+\indora_{i\geq3}D_{i}\hat{F}_{4}D_{k}\left(R_{2}+F_{2}\right)\\
D_{k}D_{i}F_{2}&=&D_{k}D_{i}\left(R_{1}+F_{1}+\indora_{i\geq3}\hat{F}_{3}\right)+D_{i}R_{1}D_{k}\left(F_{1}+\indora_{k\geq3}\hat{F}_{3}\right)+D_{i}F_{1}D_{k}\left(R_{1}+\indora_{k\geq3}\hat{F}_{3}\right)+\indora_{i\geq3}D_{i}\hat{F}_{3}D_{k}\left(R_{1}+F_{1}\right)\\
D_{k}D_{i}\hat{F}_{3}&=&D_{k}D_{i}\left(\hat{R}_{4}+\indora_{i\leq2}F_{2}+\hat{F}_{4}\right)+D_{i}\hat{R}_{4}D_{k}\left(\indora_{k\leq2}F_{2}+\hat{F}_{4}\right)+D_{i}\hat{F}_{4}D_{k}\left(\hat{R}_{4}+\indora_{k\leq2}F_{2}\right)+\indora_{i\leq2}D_{i}F_{2}D_{k}\left(\hat{R}_{4}+\hat{F}_{4}\right)\\
D_{k}D_{i}\hat{F}_{4}&=&D_{k}D_{i}\left(\hat{R}_{3}+\indora_{i\leq2}F_{1}+\hat{F}_{3}\right)+D_{i}\hat{R}_{3}D_{k}\left(\indora_{k\leq2}F_{1}+\hat{F}_{3}\right)+D_{i}\hat{F}_{3}D_{k}\left(\hat{R}_{3}+\indora_{k\leq2}F_{1}\right)+\indora_{i\leq2}D_{i}F_{1}D_{k}\left(\hat{R}_{3}+\hat{F}_{3}\right)
\end{eqnarray*}}
for $i,k=1,\ldots,4$. In order to have it in an specific way we need to compute the expressions $D_{k}D_{i}\left(R_{2}+F_{2}+\indora_{i\geq3}\hat{F}_{4}\right)$

%_________________________________________________________________________________________________________
\subsubsection{Second Order Derivatives: Serve's Switchover Times}
%_________________________________________________________________________________________________________

Remind $R_{i}\left(z_{1},z_{2},w_{1},w_{2}\right)=R_{i}\left(P_{1}\left(z_{1}\right)\tilde{P}_{2}\left(z_{2}\right)
\hat{P}_{1}\left(w_{1}\right)\hat{P}_{2}\left(w_{2}\right)\right)$,  which we will write in his reduced form $R_{i}=R_{i}\left(
P_{1}\tilde{P}_{2}\hat{P}_{1}\hat{P}_{2}\right)$, and according to the notation given in \cite{Lang} we obtain

\begin{eqnarray}
D_{i}D_{i}R_{k}=D^{2}R_{k}\left(D_{i}P_{i}\right)^{2}+DR_{k}D_{i}D_{i}P_{i}
\end{eqnarray}

whereas for $i\neq j$

\begin{eqnarray}
D_{i}D_{j}R_{k}=D^{2}R_{k}D_{i}P_{i}D_{j}P_{j}+DR_{k}D_{j}P_{j}D_{i}P_{i}
\end{eqnarray}

%_________________________________________________________________________________________________________
\subsubsection{Second Order Derivatives: Queue Lengths}
%_________________________________________________________________________________________________________

Just like before the expression $F_{1}\left(\tilde{\theta}_{1}\left(\tilde{P}_{2}\left(z_{2}\right)\hat{P}_{1}\left(w_{1}\right)\hat{P}_{2}\left(w_{2}\right)\right),
z_{2}\right)$, will be denoted by $F_{1}\left(\tilde{\theta}_{1}\left(\tilde{P}_{2}\hat{P}_{1}\hat{P}_{2}\right),z_{2}\right)$, then the mixed partial derivatives are:

\begin{eqnarray*}
D_{j}D_{i}F_{1}&=&\indora_{i,j\neq1}D_{1}D_{1}F_{1}\left(D\tilde{\theta}_{1}\right)^{2}D_{i}P_{i}D_{j}P_{j}
+\indora_{i,j\neq1}D_{1}F_{1}D^{2}\tilde{\theta}_{1}D_{i}P_{i}D_{j}P_{j}
+\indora_{i,j\neq1}D_{1}F_{1}D\tilde{\theta}_{1}\left(\indora_{i=j}D_{i}^{2}P_{i}+\indora_{i\neq j}D_{i}P_{i}D_{j}P_{j}\right)\\
&+&\left(1-\indora_{i=j=3}\right)\indora_{i+j\leq6}D_{1}D_{2}F_{1}D\tilde{\theta}_{1}\left(\indora_{i\leq j}D_{j}P_{j}+\indora_{i>j}D_{i}P_{i}\right)
+\indora_{i=2}\left(D_{1}D_{2}F_{1}D\tilde{\theta}_{1}D_{i}P_{i}+D_{i}^{2}F_{1}\right)
\end{eqnarray*}


Recall the expression for $F_{1}\left(\tilde{\theta}_{1}\left(\tilde{P}_{2}\left(z_{2}\right)\hat{P}_{1}\left(w_{1}\right)\hat{P}_{2}\left(w_{2}\right)\right),
z_{2}\right)$, which is denoted by $F_{1}\left(\tilde{\theta}_{1}\left(\tilde{P}_{2}\hat{P}_{1}\hat{P}_{2}\right),z_{2}\right)$, then the mixed partial derivatives are given by

\begin{eqnarray*}
\begin{array}{llll}
D_{1}D_{1}F_{1}=0,&
D_{2}D_{1}F_{1}=0,&
D_{3}D_{1}F_{1}=0,&
D_{4}D_{1}F_{1}=0,\\
D_{1}D_{2}F_{1}=0,&
D_{1}D_{3}F_{1}=0,&
D_{1}D_{4}F_{1}=0,&
\end{array}
\end{eqnarray*}

\begin{eqnarray*}
D_{2}D_{2}F_{1}&=&D_{1}^{2}F_{1}\left(D\tilde{\theta}_{1}\right)^{2}\left(D_{2}\tilde{P}_{2}\right)^{2}
+D_{1}F_{1}D^{2}\tilde{\theta}_{1}\left(D_{2}\tilde{P}_{2}\right)^{2}
+D_{1}F_{1}D\tilde{\theta}_{1}D_{2}^{2}\tilde{P}_{2}
+D_{1}D_{2}F_{1}D\tilde{\theta}_{1}D_{2}\tilde{P}_{2}\\
&+&D_{1}D_{2}F_{1}D\tilde{\theta}_{1}D_{2}\tilde{P}_{2}+D_{2}D_{2}F_{1}\\
&=&f_{1}\left(1,1\right)\left(\frac{\tilde{\mu}_{2}}{1-\tilde{\mu}_{1}}\right)^{2}
+f_{1}\left(1\right)\tilde{\theta}_{1}^(2)\tilde{\mu}_{2}^{(2)}
+f_{1}\left(1\right)\frac{1}{1-\tilde{\mu}_{1}}\tilde{P}_{2}^{(2)}+f_{1}\left(1,2\right)\frac{\tilde{\mu}_{2}}{1-\tilde{\mu}_{1}}+f_{1}\left(1,2\right)\frac{\tilde{\mu}_{2}}{1-\tilde{\mu}_{1}}+f_{1}\left(2,2\right)
\end{eqnarray*}

\begin{eqnarray*}
D_{3}D_{2}F_{1}&=&D_{1}^{2}F_{1}\left(D\tilde{\theta}_{1}\right)^{2}D_{3}\hat{P}_{1}D_{2}\tilde{P}_{2}+D_{1}F_{1}D^{2}\tilde{\theta}_{1}D_{3}\hat{P}_{1}D_{2}\tilde{P}_{2}+D_{1}F_{1}D\tilde{\theta}_{1}D_{2}\tilde{P}_{2}D_{3}\hat{P}_{1}+D_{1}D_{2}F_{1}D\tilde{\theta}_{1}D_{3}\hat{P}_{1}\\
&=&f_{1}\left(1,1\right)\left(\frac{1}{1-\tilde{\mu}_{1}}\right)^{2}\tilde{\mu}_{2}\hat{\mu}_{1}+f_{1}\left(1\right)\tilde{\theta}_{1}^{(2)}\tilde{\mu}_{2}\hat{\mu}_{1}+f_{1}\left(1\right)\frac{\tilde{\mu}_{2}\hat{\mu}_{1}}{1-\tilde{\mu}_{1}}+f_{1}\left(1,2\right)\frac{\hat{\mu}_{1}}{1-\tilde{\mu}_{1}}
\end{eqnarray*}

\begin{eqnarray*}
D_{4}D_{2}F_{1}&=&D_{1}^{2}F_{1}\left(D\tilde{\theta}_{1}\right)^{2}D_{4}\hat{P}_{2}D_{2}\tilde{P}_{2}+D_{1}F_{1}D^{2}\tilde{\theta}_{1}D_{4}\hat{P}_{2}D_{2}\tilde{P}_{2}+D_{1}F_{1}D\tilde{\theta}_{1}D_{2}\tilde{P}_{2}D_{4}\hat{P}_{2}+D_{1}D_{2}F_{1}D\tilde{\theta}_{1}D_{4}\hat{P}_{2}\\
&=&f_{1}\left(1,1\right)\left(\frac{1}{1-\tilde{\mu}_{1}}\right)^{2}\tilde{\mu}_{2}\hat{\mu}_{2}+f_{1}\left(1\right)\tilde{\theta}_{1}^{(2)}\tilde{\mu}_{2}\hat{\mu}_{2}+f_{1}\left(1\right)\frac{\tilde{\mu}_{2}\hat{\mu}_{2}}{1-\tilde{\mu}_{1}}+f_{1}\left(1,2\right)\frac{\hat{\mu}_{2}}{1-\tilde{\mu}_{1}}
\end{eqnarray*}

\begin{eqnarray*}
D_{2}D_{3}F_{1}&=&
D_{1}^{2}F_{1}\left(D\tilde{\theta}_{1}\right)^{2}D_{2}\tilde{P}_{2}D_{3}\hat{P}_{1}
+D_{1}F_{1}D^{2}\tilde{\theta}_{1}D_{2}\tilde{P}_{2}D_{3}\hat{P}_{1}+
D_{1}F_{1}D\tilde{\theta}_{1}D_{3}\hat{P}_{1}D_{2}\tilde{P}_{2}
+D_{1}D_{2}F_{1}D\tilde{\theta}_{1}D_{3}\hat{P}_{1}\\
&=&f_{1}\left(1,1\right)\left(\frac{1}{1-\tilde{\mu}_{1}}\right)^{2}\tilde{\mu}_{2}\hat{\mu}_{1}+f_{1}\left(1\right)\tilde{\theta}_{1}^{(2)}\tilde{\mu}_{2}\hat{\mu}_{1}+f_{1}\left(1\right)\frac{\tilde{\mu}_{2}\hat{\mu}_{1}}{1-\tilde{\mu}_{1}}+f_{1}\left(1,2\right)\frac{\hat{\mu}_{1}}{1-\tilde{\mu}_{1}}
\end{eqnarray*}

\begin{eqnarray*}
D_{3}D_{3}F_{1}&=&D_{1}^{2}F_{1}\left(D\tilde{\theta}_{1}\right)^{2}\left(D_{3}\hat{P}_{1}\right)^{2}+D_{1}F_{1}D^{2}\tilde{\theta}_{1}\left(D_{3}\hat{P}_{1}\right)^{2}+D_{1}F_{1}D\tilde{\theta}_{1}D_{3}^{2}\hat{P}_{1}\\
&=&f_{1}\left(1,1\right)\left(\frac{\hat{\mu}_{1}}{1-\tilde{\mu}_{1}}\right)^{2}+f_{1}\left(1\right)\tilde{\theta}_{1}^{(2)}\hat{\mu}_{1}^{2}+f_{1}\left(1\right)\frac{\hat{\mu}_{1}^{2}}{1-\tilde{\mu}_{1}}
\end{eqnarray*}

\begin{eqnarray*}
D_{4}D_{3}F_{1}&=&D_{1}^{2}F_{1}\left(D\tilde{\theta}_{1}\right)^{2}D_{4}\hat{P}_{2}D_{3}\hat{P}_{1}+D_{1}F_{1}D^{2}\tilde{\theta}_{1}D_{4}\hat{P}_{2}D_{3}\hat{P}_{1}+D_{1}F_{1}D\tilde{\theta}_{1}D_{3}\hat{P}_{1}D_{4}\hat{P}_{2}\\
&=&f_{1}\left(1,1\right)\left(\frac{1}{1-\tilde{\mu}_{1}}\right)^{2}\hat{\mu}_{1}\hat{\mu}_{2}
+f_{1}\left(1\right)\tilde{\theta}_{1}^{2}\hat{\mu}_{2}\hat{\mu}_{1}
+f_{1}\left(1\right)\frac{\hat{\mu}_{2}\hat{\mu}_{1}}{1-\tilde{\mu}_{1}}
\end{eqnarray*}

\begin{eqnarray*}
D_{2}D_{4}F_{1}&=&D_{1}^{2}F_{1}\left(D\tilde{\theta}_{1}\right)^{2}D_{2}\tilde{P}_{2}D_{4}\hat{P}_{2}+D_{1}F_{1}D^{2}\tilde{\theta}_{1}D_{2}\tilde{P}_{2}D_{4}\hat{P}_{2}+D_{1}F_{1}D\tilde{\theta}_{1}D_{4}\hat{P}_{2}D_{2}\tilde{P}_{2}+D_{1}D_{2}F_{1}D\tilde{\theta}_{1}D_{4}\hat{P}_{2}\\
&=&f_{1}\left(1,1\right)\left(\frac{1}{1-\tilde{\mu}_{1}}\right)^{2}\hat{\mu}_{2}\tilde{\mu}_{2}
+f_{1}\left(1\right)\tilde{\theta}_{1}^{(2)}\hat{\mu}_{2}\tilde{\mu}_{2}
+f_{1}\left(1\right)\frac{\hat{\mu}_{2}\tilde{\mu}_{2}}{1-\tilde{\mu}_{1}}+f_{1}\left(1,2\right)\frac{\hat{\mu}_{2}}{1-\tilde{\mu}_{1}}
\end{eqnarray*}

\begin{eqnarray*}
D_{3}D_{4}F_{1}&=&D_{1}^{2}F_{1}\left(D\tilde{\theta}_{1}\right)^{2}D_{3}\hat{P}_{1}D_{4}\hat{P}_{2}+D_{1}F_{1}D^{2}\tilde{\theta}_{1}D_{3}\hat{P}_{1}D_{4}\hat{P}_{2}+D_{1}F_{1}D\tilde{\theta}_{1}D_{4}\hat{P}_{2}D_{3}\hat{P}_{1}\\
&=&f_{1}\left(1,1\right)\left(\frac{1}{1-\tilde{\mu}_{1}}\right)^{2}\hat{\mu}_{1}\hat{\mu}_{2}+f_{1}\left(1\right)\tilde{\theta}_{1}^{(2)}\hat{\mu}_{1}\hat{\mu}_{2}+f_{1}\left(1\right)\frac{\hat{\mu}_{1}\hat{\mu}_{2}}{1-\tilde{\mu}_{1}}
\end{eqnarray*}

\begin{eqnarray*}
D_{4}D_{4}F_{1}&=&D_{1}^{2}F_{1}\left(D\tilde{\theta}_{1}\right)^{2}\left(D_{4}\hat{P}_{2}\right)^{2}+D_{1}F_{1}D^{2}\tilde{\theta}_{1}\left(D_{4}\hat{P}_{2}\right)^{2}+D_{1}F_{1}D\tilde{\theta}_{1}D_{4}^{2}\hat{P}_{2}\\
&=&f_{1}\left(1,1\right)\left(\frac{\hat{\mu}_{2}}{1-\tilde{\mu}_{1}}\right)^{2}+f_{1}\left(1\right)\tilde{\theta}_{1}^{(2)}\hat{\mu}_{2}^{2}+f_{1}\left(1\right)\frac{1}{1-\tilde{\mu}_{1}}\hat{P}_{2}^{(2)}
\end{eqnarray*}



Meanwhile for  $F_{2}\left(z_{1},\tilde{\theta}_{2}\left(P_{1}\hat{P}_{1}\hat{P}_{2}\right)\right)$

\begin{eqnarray*}
D_{j}D_{i}F_{2}&=&\indora_{i,j\neq2}D_{2}D_{2}F_{2}\left(D\theta_{2}\right)^{2}D_{i}P_{i}D_{j}P_{j}+\indora_{i,j\neq2}D_{2}F_{2}D^{2}\theta_{2}D_{i}P_{i}D_{j}P_{j}\\
&+&\indora_{i,j\neq2}D_{2}F_{2}D\theta_{2}\left(\indora_{i=j}D_{i}^{2}P_{i}
+\indora_{i\neq j}D_{i}P_{i}D_{j}P_{j}\right)\\
&+&\left(1-\indora_{i=j=3}\right)\indora_{i+j\leq6}D_{2}D_{1}F_{2}D\theta_{2}\left(\indora_{i\leq j}D_{j}P_{j}+\indora_{i>j}D_{i}P_{i}\right)
+\indora_{i=1}\left(D_{2}D_{1}F_{2}D\theta_{2}D_{i}P_{i}+D_{i}^{2}F_{2}\right)
\end{eqnarray*}

\begin{eqnarray*}
\begin{array}{llll}
D_{2}D_{1}F_{2}=0,&
D_{2}D_{3}F_{3}=0,&
D_{2}D_{4}F_{2}=0,&\\
D_{1}D_{2}F_{2}=0,&
D_{2}D_{2}F_{2}=0,&
D_{3}D_{2}F_{2}=0,&
D_{4}D_{2}F_{2}=0\\
\end{array}
\end{eqnarray*}


\begin{eqnarray*}
D_{1}D_{1}F_{2}&=&
\left(D_{1}P_{1}\right)^{2}\left(D\tilde{\theta}_{2}\right)^{2}D_{2}^{2}F_{2}
+\left(D_{1}P_{1}\right)^{2}D^{2}\tilde{\theta}_{2}D_{2}F_{2}
+D_{1}^{2}P_{1}D\tilde{\theta}_{2}D_{2}F_{2}
+D_{1}P_{1}D\tilde{\theta}_{2}D_{2}D_{1}F_{2}\\
&+&D_{2}D_{1}F_{2}D\tilde{\theta}_{2}D_{1}P_{1}+
D_{1}^{2}F_{2}\\
&=&f_{2}\left(2\right)\frac{\tilde{P}_{1}^{(2)}}{1-\tilde{\mu}_{2}}
+f_{2}\left(2\right)\theta_{2}^{(2)}\tilde{\mu}_{1}^{2}
+f_{2}\left(2,1\right)\frac{\tilde{\mu}_{1}}{1-\tilde{\mu}_{2}}
+\left(\frac{\tilde{\mu}_{1}}{1-\tilde{\mu}_{2}}\right)^{2}f_{2}\left(2,2\right)
+\frac{\tilde{\mu}_{1}}{1-\tilde{\mu}_{2}}f_{2}\left(2,1\right)+f_{2}\left(1,1\right)
\end{eqnarray*}


\begin{eqnarray*}
D_{3}D_{1}F_{2}&=&D_{2}D_{1}F_{2}D\tilde{\theta}_{2}D_{3}\hat{P}_{1}
+D_{2}^{2}F_{2}\left(D\tilde{\theta}_{2}\right)^{2}D_{3}P_{1}D_{1}P_{1}
+D_{2}F_{2}D^{2}\tilde{\theta}_{2}D_{3}\hat{P}_{1}D_{1}P_{1}
+D_{2}F_{2}D\tilde{\theta}_{2}D_{1}P_{1}D_{3}\hat{P}_{1}\\
&=&f_{2}\left(2,1\right)\frac{\hat{\mu}_{1}}{1-\tilde{\mu}_{2}}
+f_{2}\left(2,2\right)\left(\frac{1}{1-\tilde{\mu}_{2}}\right)^{2}\tilde{\mu}_{1}\hat{\mu}_{1}
+f_{2}\left(2\right)\tilde{\theta}_{2}^{(2)}\tilde{\mu}_{1}\hat{\mu}_{1}
+f_{2}\left(2\right)\frac{\tilde{\mu}_{1}\hat{\mu}_{1}}{1-\tilde{\mu}_{2}}
\end{eqnarray*}


\begin{eqnarray*}
D_{4}D_{1}F_{2}&=&D_{2}^{2}F_{2}\left(D\tilde{\theta}_{2}\right)^{2}D_{4}P_{2}D_{1}P_{1}+D_{2}F_{2}D^{2}\tilde{\theta}_{2}D_{4}\hat{P}_{2}D_{1}P_{1}
+D_{2}F_{2}D\tilde{\theta}_{2}D_{1}P_{1}D_{4}\hat{P}_{2}+D_{2}D_{1}F_{2}D\tilde{\theta}_{2}D_{4}\hat{P}_{2}\\
&=&f_{2}\left(2,2\right)\left(\frac{1}{1-\tilde{\mu}_{2}}\right)^{2}\tilde{\mu}_{1}\hat{\mu}_{2}
+f_{2}\left(2\right)\tilde{\theta}_{2}^{(2)}\tilde{\mu}_{1}\hat{\mu}_{2}
+f_{2}\left(2\right)\frac{\tilde{\mu}_{1}\hat{\mu}_{2}}{1-\tilde{\mu}_{2}}
+f_{2}\left(2,1\right)\frac{\hat{\mu}_{2}}{1-\tilde{\mu}_{2}}
\end{eqnarray*}


\begin{eqnarray*}
D_{1}D_{3}F_{2}&=&D_{2}^{2}F_{2}\left(D\tilde{\theta}_{2}\right)^{2}D_{1}P_{1}D_{3}\hat{P}_{1}
+D_{2}F_{2}D^{2}\tilde{\theta}_{2}D_{1}P_{1}D_{3}\hat{P}_{1}
+D_{2}F_{2}D\tilde{\theta}_{2}D_{3}\hat{P}_{1}D_{1}P_{1}
+D_{2}D_{1}F_{2}D\tilde{\theta}_{2}D_{3}\hat{P}_{1}\\
&=&f_{2}\left(2,2\right)\left(\frac{1}{1-\tilde{\mu}_{2}}\right)^{2}\tilde{\mu}_{1}\hat{\mu}_{1}
+f_{2}\left(2\right)\tilde{\theta}_{2}^{(2)}\tilde{\mu}_{1}\hat{\mu}_{1}
+f_{2}\left(2\right)\frac{\tilde{\mu}_{1}\hat{\mu}_{1}}{1-\tilde{\mu}_{2}}
+f_{2}\left(2,1\right)\frac{\hat{\mu}_{1}}{1-\tilde{\mu}_{2}}
\end{eqnarray*}


\begin{eqnarray*}
D_{3}D_{3}F_{2}&=&D_{2}^{2}F_{2}\left(D\tilde{\theta}_{2}\right)^{2}\left(D_{3}\hat{P}_{1}\right)^{2}
+D_{2}F_{2}\left(D_{3}\hat{P}_{1}\right)^{2}D^{2}\tilde{\theta}_{2}
+D_{2}F_{2}D\tilde{\theta}_{2}D_{3}^{2}\hat{P}_{1}\\
&=&f_{2}\left(2,2\right)\left(\frac{1}{1-\tilde{\mu}_{2}}\right)^{2}\hat{\mu}_{1}^{2}
+f_{2}\left(2\right)\tilde{\theta}_{2}^{(2)}\hat{\mu}_{1}^{2}
+f_{2}\left(2\right)\frac{\hat{P}_{1}^{(2)}}{1-\tilde{\mu}_{2}}
\end{eqnarray*}


\begin{eqnarray*}
D_{4}D_{3}F_{2}&=&D_{2}^{2}F_{2}\left(D\tilde{\theta}_{2}\right)^{2}D_{4}\hat{P}_{2}D_{3}\hat{P}_{1}
+D_{2}F_{2}D^{2}\tilde{\theta}_{2}D_{4}\hat{P}_{2}D_{3}\hat{P}_{1}
+D_{2}F_{2}D\tilde{\theta}_{2}D_{3}\hat{P}_{1}D_{4}\hat{P}_{2}\\
&=&f_{2}\left(2,2\right)\left(\frac{1}{1-\tilde{\mu}_{2}}\right)^{2}\hat{\mu}_{1}\hat{\mu}_{2}
+f_{2}\left(2\right)\tilde{\theta}_{2}^{(2)}\hat{\mu}_{1}\hat{\mu}_{2}
+f_{2}\left(2\right)\frac{\hat{\mu}_{1}\hat{\mu}_{2}}{1-\tilde{\mu}_{2}}
\end{eqnarray*}


\begin{eqnarray*}
D_{1}D_{4}F_{2}&=&D_{2}^{2}F_{2}\left(D\tilde{\theta}_{2}\right)^{2}D_{1}P_{1}D_{4}\hat{P}_{2}
+D_{2}F_{2}D^{2}\tilde{\theta}_{2}D_{1}P_{1}D_{4}\hat{P}_{2}
+D_{2}F_{2}D\tilde{\theta}_{2}D_{4}\hat{P}_{2}D_{1}P_{1}
+D_{2}D_{1}F_{2}D\tilde{\theta}_{2}D_{4}\hat{P}_{2}\\
&=&f_{2}\left(2,2\right)\left(\frac{1}{1-\tilde{\mu}_{2}}\right)^{2}\tilde{\mu}_{1}\hat{\mu}_{2}
+f_{2}\left(2\right)\tilde{\theta}_{2}^{(2)}\tilde{\mu}_{1}\hat{\mu}_{2}
+f_{2}\left(2\right)\frac{\tilde{\mu}_{1}\hat{\mu}_{2}}{1-\tilde{\mu}_{2}}
+f_{2}\left(2,1\right)\frac{\hat{\mu}_{2}}{1-\tilde{\mu}_{2}}
\end{eqnarray*}


\begin{eqnarray*}
D_{3}D_{4}F_{2}&=&
D_{2}^{2}F_{2}\left(D\tilde{\theta}_{2}\right)^{2}D_{4}\hat{P}_{2}D_{3}\hat{P}_{1}
+D_{2}F_{2}D^{2}\tilde{\theta}_{2}D_{4}\hat{P}_{2}D_{3}\hat{P}_{1}
+D_{2}F_{2}D\tilde{\theta}_{2}D_{4}\hat{P}_{2}D_{3}\hat{P}_{1}\\
&=&f_{2}\left(2,2\right)\left(\frac{1}{1-\tilde{\mu}_{2}}\right)^{2}\hat{\mu}_{1}\hat{\mu}_{2}
+f_{2}\left(2\right)\tilde{\theta}_{2}^{(2)}\hat{\mu}_{1}\hat{\mu}_{2}
+f_{2}\left(2\right)\frac{\hat{\mu}_{1}\hat{\mu}_{2}}{1-\tilde{\mu}_{2}}
\end{eqnarray*}


\begin{eqnarray*}
D_{4}D_{4}F_{2}&=&D_{2}F_{2}D\tilde{\theta}_{2}D_{4}^{2}\hat{P}_{2}
+D_{2}F_{2}D^{2}\tilde{\theta}_{2}\left(D_{4}\hat{P}_{2}\right)^{2}
+D_{2}^{2}F_{2}\left(D\tilde{\theta}_{2}\right)^{2}\left(D_{4}\hat{P}_{2}\right)^{2}\\
&=&f_{2}\left(2,2\right)\left(\frac{\hat{\mu}_{2}}{1-\tilde{\mu}_{2}}\right)^{2}
+f_{2}\left(2\right)\tilde{\theta}_{2}^{(2)}\hat{\mu}_{2}^{2}
+f_{2}\left(2\right)\frac{\hat{P}_{2}^{(2)}}{1-\tilde{\mu}_{2}}
\end{eqnarray*}


%\newpage



%\newpage

For $\hat{F}_{1}\left(\hat{\theta}_{1}\left(P_{1}\tilde{P}_{2}\hat{P}_{2}\right),w_{2}\right)$



\begin{eqnarray*}
D_{j}D_{i}\hat{F}_{1}&=&\indora_{i,j\neq3}D_{3}D_{3}\hat{F}_{1}\left(D\hat{\theta}_{1}\right)^{2}D_{i}P_{i}D_{j}P_{j}
+\indora_{i,j\neq3}D_{3}\hat{F}_{1}D^{2}\hat{\theta}_{1}D_{i}P_{i}D_{j}P_{j}
+\indora_{i,j\neq3}D_{3}\hat{F}_{1}D\hat{\theta}_{1}\left(\indora_{i=j}D_{i}^{2}P_{i}+\indora_{i\neq j}D_{i}P_{i}D_{j}P_{j}\right)\\
&+&\indora_{i+j\geq5}D_{3}D_{4}\hat{F}_{1}D\hat{\theta}_{1}\left(\indora_{i\leq j}D_{i}P_{i}+\indora_{i>j}D_{j}P_{j}\right)
+\indora_{i=4}\left(D_{3}D_{4}\hat{F}_{1}D\hat{\theta}_{1}D_{i}P_{i}+D_{i}^{2}\hat{F}_{1}\right)
\end{eqnarray*}


\begin{eqnarray*}
\begin{array}{llll}
D_{3}D_{1}\hat{F}_{1}=0,&
D_{3}D_{2}\hat{F}_{1}=0,&
D_{1}D_{3}\hat{F}_{1}=0,&
D_{2}D_{3}\hat{F}_{1}=0\\
D_{3}D_{3}\hat{F}_{1}=0,&
D_{4}D_{3}\hat{F}_{1}=0,&
D_{3}D_{4}\hat{F}_{1}=0,&
\end{array}
\end{eqnarray*}


\begin{eqnarray*}
D_{1}D_{1}\hat{F}_{1}&=&
D_{3}^{2}\hat{F}_{1}\left(D\hat{\theta}_{1}\right)^{2}\left(D_{1}P_{1}\right)^{2}
+D_{3}\hat{F}_{1}D^{2}\hat{\theta}_{1}\left(D_{1}P_{1}\right)^{2}
+D_{3}\hat{F}_{1}D\hat{\theta}_{1}D_{1}^{2}P_{1}\\
&=&\hat{f}_{1}\left(3,3\right)\left(\frac{\tilde{\mu}_{1}}{1-\hat{\mu}_{2}}\right)^{2}
+\hat{f}_{1}\left(3\right)\frac{P_{1}^{(2)}}{1-\hat{\mu}_{1}}
+\hat{f}_{1}\left(3\right)\hat{\theta}_{1}^{(2)}\tilde{\mu}_{1}^{2}
\end{eqnarray*}


\begin{eqnarray*}
D_{2}D_{1}\hat{F}_{1}&=&
D_{3}^{2}\hat{F}_{1}\left(D\hat{\theta}_{1}\right)^{2}D_{1}P_{1}D_{2}P_{1}+
D_{3}\hat{F}_{1}D^{2}\hat{\theta}_{1}D_{1}P_{1}D_{2}P_{2}+
D_{3}\hat{F}_{1}D\hat{\theta}_{1}D_{1}P_{1}D_{2}P_{2}\\
&=&\hat{f}_{1}\left(3,3\right)\left(\frac{1}{1-\hat{\mu}_{1}}\right)^{2}\tilde{\mu}_{1}\tilde{\mu}_{2}
+\hat{f}_{1}\left(3\right)\tilde{\mu}_{1}\tilde{\mu}_{2}\hat{\theta}_{1}^{(2)}
+\hat{f}_{1}\left(3\right)\frac{\tilde{\mu}_{1}\tilde{\mu}_{2}}{1-\hat{\mu}_{1}}
\end{eqnarray*}


\begin{eqnarray*}
D_{4}D_{1}\hat{F}_{1}&=&
D_{3}D_{3}\hat{F}_{1}\left(D\hat{\theta}_{1}\right)^{2}D_{4}\hat{P}_{2}D_{1}P_{1}
+D_{3}\hat{F}_{1}D^{2}\hat{\theta}_{1}D_{1}P_{1}D_{4}\hat{P}_{2}
+D_{3}\hat{F}_{1}D\hat{\theta}_{1}D_{1}P_{1}D_{4}\hat{P}_{2}
+D_{3}D_{4}\hat{F}_{1}D\hat{\theta}_{1}D_{1}P_{1}\\
&=&\hat{f}_{1}\left(3,3\right)\left(\frac{1}{1-\hat{\mu}_{1}}\right)^{2}\tilde{\mu}_{1}\hat{\mu}_{1}
+\hat{f}_{1}\left(3\right)\hat{\theta}_{1}^{(2)}\tilde{\mu}_{1}\hat{\mu}_{2}
+\hat{f}_{1}\left(3\right)\frac{\tilde{\mu}_{1}\hat{\mu}_{2}}{1-\hat{\mu}_{1}}
+\hat{f}_{1}\left(3,4\right)\frac{\tilde{\mu}_{1}}{1-\hat{\mu}_{1}}
\end{eqnarray*}


\begin{eqnarray*}
D_{1}D_{2}\hat{F}_{1}&=&
D_{3}^{2}\hat{F}_{1}\left(D\hat{\theta}_{1}\right)^{2}D_{1}P_{1}D_{2}P_{2}
+D_{3}\hat{F}_{1}D^{2}\hat{\theta}_{1}D_{1}P_{1}D_{2}P_{2}+
D_{3}\hat{F}_{1}D\hat{\theta}_{1}D_{1}P_{1}D_{2}P_{2}\\
&=&\hat{f}_{1}\left(3,3\right)\left(\frac{1}{1-\hat{\mu}_{1}}\right)^{2}\tilde{\mu}_{1}\tilde{\mu}_{2}
+\hat{f}_{1}\left(3\right)\hat{\theta}_{1}^{(2)}\tilde{\mu}_{1}\tilde{\mu}_{2}
+\hat{f}_{1}\left(3\right)\frac{\tilde{\mu}_{1}\tilde{\mu}_{2}}{1-\hat{\mu}_{1}}
\end{eqnarray*}


\begin{eqnarray*}
D_{2}D_{2}\hat{F}_{1}&=&
D_{3}^{2}\hat{F}_{1}\left(D\hat{\theta}_{1}\right)^{2}\left(D_{2}P_{2}\right)^{2}
+D_{3}\hat{F}_{1}D^{2}\hat{\theta}_{1}\left(D_{2}P_{2}\right)^{2}+
D_{3}\hat{F}_{1}D\hat{\theta}_{1}D_{2}^{2}P_{2}\\
&=&\hat{f}_{1}\left(3,3\right)\left(\frac{\tilde{\mu}_{2}}{1-\hat{\mu}_{1}}\right)^{2}
+\hat{f}_{1}\left(3\right)\hat{\theta}_{1}^{(2)}\tilde{\mu}_{2}^{2}
+\hat{f}_{1}\left(3\right)\tilde{P}_{2}^{(2)}\frac{1}{1-\hat{\mu}_{1}}
\end{eqnarray*}


\begin{eqnarray*}
D_{4}D_{2}\hat{F}_{1}&=&D_{2}P_{2}D_{4}\hat{P}_{2}D\hat{\theta}_{1}D_{3}\hat{F}_{1}
+D_{2}P_{2}D_{4}\hat{P}_{2}D^{2}\hat{\theta}_{1}D_{3}\hat{F}_{1}
+D_{2}P_{2}D\hat{\theta}_{1}D_{4}D_{3}\hat{F}_{1}
+D_{2}P_{2}\left(D\hat{\theta}_{1}\right)^{2}D_{4}\hat{P}_{2}D_{3}^{2}\hat{F}_{1}\\
&=&\hat{f}_{1}\left(3\right)\frac{\tilde{\mu}_{2}\hat{\mu}_{2}}{1-\hat{\mu}_{1}}
+\hat{f}_{1}\left(3\right)\hat{\theta}_{1}^{(2)}\tilde{\mu}_{2}\hat{\mu}_{2}
+\hat{f}_{1}\left(4,3\right)\frac{\tilde{\mu}_{2}}{1-\hat{\mu}_{1}}
+\hat{f}_{1}\left(3,3\right)\left(\frac{1}{1-\hat{\mu}_{1}}\right)^{2}\tilde{\mu}_{2}\hat{\mu}_{2}
\end{eqnarray*}



\begin{eqnarray*}
D_{1}D_{4}\hat{F}_{1}&=&D_{1}P_{1}D_{4}\hat{P}_{2}D\hat{\theta}_{1}D_{3}\hat{F}_{1}
+D_{1}P_{1}D_{4}\hat{P}_{2}D^{2}\hat{\theta}_{1}D_{3}\hat{F}_{1}
+D_{1}P_{1}D\hat{\theta}_{1}D_{3}D_{4}\hat{F}_{1}
+ D_{1}P_{1}D_{4}\hat{P}_{2}\left(D\hat{\theta}_{1}\right)^{2}D_{3}D_{3}
\hat{F}_{1}\\
&=&\hat{f}_{1}\left(3\right)\frac{\tilde{\mu}_{1}\hat{\mu}_{2}}{1-\hat{\mu}_{1}}
+\hat{f}_{1}\left(3\right)\hat{\theta}_{1}^{(2)}\tilde{\mu}_{1}\hat{\mu}_{2}
+\hat{f}_{1}\left(3,4\right)\frac{\tilde{\mu}_{1}}{1-\hat{\mu}_{1}}
+\hat{f}_{1}\left(3,3\right)\left(\frac{1}{1-\hat{\mu}_{1}}\right)^{2}\tilde{\mu}_{1}\hat{\mu}_{2}
\end{eqnarray*}


\begin{eqnarray*}
D_{2}D_{4}\hat{F}_{1}&=&D_{2}P_{2}D_{4}\hat{P}_{2}D\hat{\theta}_{1}D_{3}
\hat{F}_{1}
+D_{2}P_{2}D_{4}\hat{P}_{2}D^{2}\hat{\theta}_{1}D_{3}\hat{F}_{1}
+D_{2}P_{2}D\hat{\theta}_{1}D_{3}D_{4}\hat{F}_{1}+
D_{2}P_{2}D_{4}\hat{P}_{2}\left(D\hat{\theta}_{1}\right)^{2}D_{3}^{2}\hat{F}_{1}\\
&=&\hat{f}_{1}\left(3\right)\frac{\tilde{\mu}_{2}\hat{\mu}_{2}}{1-\hat{\mu}_{1}}
+\hat{f}_{1}\left(3\right)\hat{\theta}_{1}^{(2)}\tilde{\mu}_{2}\hat{\mu}_{2}
+\hat{f}_{1}\left(3,4\right)\frac{\tilde{\mu}_{2}}{1-\hat{\mu}_{1}}
+\hat{f}_{1}\left(3,3\right)\left(\frac{1}{1-\hat{\mu}_{1}}\right)^{2}\tilde{\mu}_{2}\hat{\mu}_{2}
\end{eqnarray*}



\begin{eqnarray*}
D_{4}D_{4}\hat{F}_{1}&=&D_{4}D_{4}\hat{F}_{1}+D\hat{\theta}_{1}D_{4}^{2}\hat{P}_{2}D_{3}\hat{F}_{1}
+\left(D_{4}\hat{P}_{2}\right)^{2}D^{2}\hat{\theta}_{1}D_{3}\hat{F}_{1}+
D_{4}\hat{P}_{2}D\hat{\theta}_{1}D_{3}D_{4}\hat{F}_{1}\\
&+&\left(D_{4}\hat{P}_{2}\right)^{2}\left(D\hat{\theta}_{1}\right)^{2}D_{3}^{2}\hat{F}_{1}
+D_{3}D_{4}\hat{F}_{1}D\hat{\theta}_{1}D_{4}\hat{P}_{2}\\
&=&\hat{f}_{1}\left(4,4\right)
+\hat{f}_{1}\left(3\right)\frac{\hat{P}_{2}^{(2)}}{1-\hat{\mu}_{1}}
+\hat{f}_{1}\left(3\right)\hat{\theta}_{1}^{(2)}\hat{\mu}_{2}^{2}
+\hat{f}_{1}\left(3,4\right)\frac{\hat{\mu}_{2}}{1-\hat{\mu}_{1}}
+\hat{f}_{1}\left(3,3\right)\left(\frac{\hat{\mu}_{2}}{1-\hat{\mu}_{1}}\right)^{2}
+\hat{f}_{1}\left(3,4\right)\frac{\hat{\mu}_{2}}{1-\hat{\mu}_{1}}
\end{eqnarray*}




Finally for $\hat{F}_{2}\left(w_{1},\hat{\theta}_{2}\left(P_{1}\tilde{P}_{2}\hat{P}_{1}\right)\right)$

\begin{eqnarray*}
D_{j}D_{i}\hat{F}_{2}&=&\indora_{i,j\neq4}D_{4}D_{4}\hat{F}_{2}\left(D\hat{\theta}_{2}\right)^{2}D_{i}P_{i}D_{j}P_{j}
+\indora_{i,j\neq4}D_{4}\hat{F}_{2}D^{2}\hat{\theta}_{2}D_{i}P_{i}D_{j}P_{j}
+\indora_{i,j\neq4}D_{4}\hat{F}_{2}D\hat{\theta}_{2}\left(\indora_{i=j}D_{i}^{2}P_{i}+\indora_{i\neq j}D_{i}P_{i}D_{j}P_{j}\right)\\
&+&\left(1-\indora_{i=j=2}\right)\indora_{i+j\geq4}D_{4}D_{3}\hat{F}_{2}D\hat{\theta}_{2}\left(\indora_{i\leq j}D_{i}P_{i}+\indora_{i>j}D_{j}P_{j}\right)
+\indora_{i=3}\left(D_{4}D_{3}\hat{F}_{2}D\hat{\theta}_{2}D_{i}P_{i}+D_{i}^{2}\hat{F}_{2}\right)
\end{eqnarray*}



\begin{eqnarray*}
\begin{array}{llll}
D_{4}D_{1}\hat{F}_{2}=0,&
D_{4}D_{2}\hat{F}_{2}=0,&
D_{4}D_{3}\hat{F}_{2}=0,&
D_{1}D_{4}\hat{F}_{2}=0\\
D_{2}D_{4}\hat{F}_{2}=0,&
D_{3}D_{4}\hat{F}_{2}=0,&
D_{4}D_{4}\hat{F}_{2}=0,&
\end{array}
\end{eqnarray*}


\begin{eqnarray*}
D_{1}D_{1}\hat{F}_{2}&=&D\hat{\theta}_{2}D_{1}^{2}P_{1}D_{4}\hat{F}_{2}
+\left(D_{1}P_{1}\right)^{2}D^{2}\hat{\theta}_{2}D_{4}\hat{F}_{2}+
\left(D_{1}P_{1}\right)^{2}\left(D\hat{\theta}_{2}\right)^{2}D_{4}^{2}\hat{F}_{2}\\
&=&\hat{f}_{2}\left(4\right)\frac{\tilde{P}_{1}^{(2)}}{1-\tilde{\mu}_{2}}
+\hat{f}_{2}\left(4\right)\hat{\theta}_{2}^{(2)}\tilde{\mu}_{1}^{2}
+\hat{f}_{2}\left(4,4\right)\left(\frac{\tilde{\mu}_{1}}{1-\hat{\mu}_{2}}\right)^{2}
\end{eqnarray*}



\begin{eqnarray*}
D_{2}D_{1}\hat{F}_{2}&=&D_{1}P_{1}D_{2}P_{2}D\hat{\theta}_{2}D_{4}\hat{F}_{2}+
D_{1}P_{1}D_{2}P_{2}D^{2}\hat{\theta}_{2}D_{4}\hat{F}_{2}+
D_{1}P_{1}D_{2}P_{2}\left(D\hat{\theta}_{2}\right)^{2}D_{4}^{2}\hat{F}_{2}\\
&=&\hat{f}_{2}\left(4\right)\frac{\tilde{\mu}_{1}\tilde{\mu}_{2}}{1-\tilde{\mu}_{2}}
+\hat{f}_{2}\left(4\right)\hat{\theta}_{2}^{(2)}\tilde{\mu}_{1}\tilde{\mu}_{2}
+\hat{f}_{2}\left(4,4\right)\left(\frac{1}{1-\hat{\mu}_{2}}\right)^{2}\tilde{\mu}_{1}\tilde{\mu}_{2}
\end{eqnarray*}



\begin{eqnarray*}
D_{3}D_{1}\hat{F}_{2}&=&
D_{1}P_{1}D_{3}\hat{P}_{1}D\hat{\theta}_{2}D_{4}\hat{F}_{2}
+D_{1}P_{1}D_{3}\hat{P}_{1}D^{2}\hat{\theta}_{2}D_{4}\hat{F}_{2}
+D_{1}P_{1}D_{3}\hat{P}_{1}\left(D\hat{\theta}_{2}\right)^{2}D_{4}^{2}\hat{F}_{2}
+D_{1}P_{1}D\hat{\theta}_{2}D_{4}D_{3}\hat{F}_{2}\\
&=&\hat{f}_{2}\left(4\right)\frac{\tilde{\mu}_{1}\hat{\mu}_{1}}{1-\hat{\mu}_{2}}
+\hat{f}_{2}\left(4\right)\hat{\theta}_{2}^{(2)}\tilde{\mu}_{1}\hat{\mu}_{1}
+\hat{f}_{2}\left(4,4\right)\left(\frac{1}{1-\hat{\mu}_{2}}\right)^{2}\tilde{\mu}_{1}\hat{\mu}_{1}
+\hat{f}_{2}\left(4,3\right)\frac{\tilde{\mu}_{1}}{1-\hat{\mu}_{2}}
\end{eqnarray*}



\begin{eqnarray*}
D_{1}D_{2}\hat{F}_{2}&=&
D_{1}P_{1}D_{2}P_{2}D\hat{\theta}_{2}D_{4}\hat{F}_{2}+
D_{1}P_{1}D_{2}P_{2}D^{2}\hat{\theta}_{2}D_{4}\hat{F}_{2}+
D_{1}P_{1}D_{2}P_{2}\left(D\hat{\theta}_{2}\right)^{2}D_{4}D_{4}\hat{F}_{2}\\
&=&\hat{f}_{2}\left(4\right)\frac{\tilde{\mu}_{1}\tilde{\mu}_{2}}{1-\tilde{\mu}_{2}}
+\hat{f}_{2}\left(4\right)\hat{\theta}_{2}^{(2)}\tilde{\mu}_{1}\tilde{\mu}_{2}
+\hat{f}_{2}\left(4,4\right)\left(\frac{1}{1-\hat{\mu}_{2}}\right)^{2}\tilde{\mu}_{1}\tilde{\mu}_{2}
\end{eqnarray*}



\begin{eqnarray*}
D_{2}D_{2}\hat{F}_{2}&=&
D\hat{\theta}_{2}D_{2}^{2}P_{2}D_{4}\hat{F}_{2}+
\left(D_{2}P_{2}\right)^{2}D^{2}\hat{\theta}_{2}D_{4}\hat{F}_{2}+
\left(D_{2}P_{2}\right)^{2}\left(D\hat{\theta}_{2}\right)^{2}D_{4}^{2}\hat{F}_{2}\\
&=&\hat{f}_{2}\left(4\right)\frac{\tilde{P}_{2}^{(2)}}{1-\hat{\mu}_{2}}
+\hat{f}_{2}\left(4\right)\hat{\theta}_{2}^{(2)}\tilde{\mu}_{2}^{2}
+\hat{f}_{2}\left(4,4\right)\left(\frac{\tilde{\mu}_{2}}{1-\hat{\mu}_{2}}\right)^{2}
\end{eqnarray*}



\begin{eqnarray*}
D_{3}D_{2}\hat{F}_{2}&=&
D_{2}P_{2}D_{3}\hat{P}_{1}D\hat{\theta} _{2}D_{4}\hat{F}_{2}
+D_{2}P_{2}D_{3}\hat{P}_{1}D^{2}\hat{\theta}_{2}D_{4}\hat{F}_{2}
+D_{2}P_{2}D_{3}\hat{P}_{1}\left(D\hat{\theta}_{2}\right)^{2}D_{4}^{2}\hat{F}_{2}
+D_{2}P_{2}D\hat{\theta}_{2}D_{3}D_{4}\hat{F}_{2}\\
&=&\hat{f}_{2}\left(4\right)\frac{\tilde{\mu}_{2}\hat{\mu}_{1}}{1-\hat{\mu}_{2}}
+\hat{f}_{2}\left(4\right)\hat{\theta}_{2}^{(2)}\tilde{\mu}_{2}\hat{\mu}_{1}
+\hat{f}_{2}\left(4,4\right)\left(\frac{1}{1-\hat{\mu}_{2}}\right)^{2}\tilde{\mu}_{2}\hat{\mu}_{1}
+\hat{f}_{2}\left(3,4\right)\frac{\tilde{\mu}_{2}}{1-\hat{\mu}_{2}}
\end{eqnarray*}



\begin{eqnarray*}
D_{1}D_{3}\hat{F}_{2}&=&
D_{1}P_{1}D_{3}\hat{P}_{1}D\hat{\theta}_{2}D_{4}\hat{F}_{2}
+D_{1}P_{1}D_{3}\hat{P}_{1}D^{2}\hat{\theta}_{2}D_{4}\hat{F}_{2}
+D_{1}P_{1}D_{3}\hat{P}_{1}\left(D\hat{\theta}_{2}\right)^{2}D_{4}D_{4}\hat{F}_{2}
+D_{1}P_{1}D\hat{\theta}_{2}D_{4}D_{3}\hat{F}_{2}\\
&=&\hat{f}_{2}\left(4\right)\frac{\tilde{\mu}_{1}\hat{\mu}_{1}}{1-\hat{\mu}_{2}}
+\hat{f}_{2}\left(4\right)\hat{\theta}_{2}^{(2)}\tilde{\mu}_{1}\hat{\mu}_{1}
+\hat{f}_{2}\left(4,4\right)\left(\frac{1}{1-\hat{\mu}_{2}}\right)^{2}\tilde{\mu}_{1}\hat{\mu}_{1}
+\hat{f}_{2}\left(4,3\right)\frac{\tilde{\mu}_{1}}{1-\hat{\mu}_{2}}
\end{eqnarray*}



\begin{eqnarray*}
D_{2}D_{3}\hat{F}_{2}&=&
D_{2}P_{2}D_{3}\hat{P}_{1}D\hat{\theta}_{2}D_{4}\hat{F}_{2}
+D_{2}P_{2}D_{3}\hat{P}_{1}D^{2}\hat{\theta}_{2}D_{4}\hat{F}_{2}
+D_{2}P_{2}D_{3}\hat{P}_{1}\left(D\hat{\theta}_{2}\right)^{2}D_{4}^{2}\hat{F}_{2}
+D_{2}P_{2}D\hat{\theta}_{2}D_{4}D_{3}\hat{F}_{2}\\
&=&\hat{f}_{2}\left(4\right)\frac{\tilde{\mu}_{2}\hat{\mu}_{1}}{1-\hat{\mu}_{2}}
+\hat{f}_{2}\left(4\right)\hat{\theta}_{2}^{(2)}\tilde{\mu}_{2}\hat{\mu}_{1}
+\hat{f}_{2}\left(4,4\right)\left(\frac{1}{1-\hat{\mu}_{2}}\right)^{2}\tilde{\mu}_{2}\hat{\mu}_{1}
+\hat{f}_{2}\left(4,3\right)\frac{\tilde{\mu}_{2}}{1-\hat{\mu}_{2}}
\end{eqnarray*}



\begin{eqnarray*}
D_{3}D_{3}\hat{F}_{2}&=&
D_{3}^{2}\hat{P}_{1}D\hat{\theta}_{2}D_{4}\hat{F}_{2}
+\left(D_{3}\hat{P}_{1}\right)^{2}D^{2}\hat{\theta}_{2}D_{4}\hat{F}_{2}
+D_{3}\hat{P}_{1}D\hat{\theta}_{2}D_{4}D_{3}\hat{F}_{2}
+ \left(D_{3}\hat{P}_{1}\right)^{2}\left(D\hat{\theta}_{2}\right)^{2}
D_{4}^{2}\hat{F}_{2}+D_{3}^{2}\hat{F}_{2}
+D_{4}D_{3}\hat{f}_{2}D\hat{\theta}_{2}D_{3}\hat{P}_{1}\\
&=&\hat{f}_{2}\left(4\right)\frac{\hat{P}_{1}^{(2)}}{1-\hat{\mu}_{2}}
+\hat{f}_{2}\left(4\right)\hat{\theta}_{2}^{(2)}\hat{\mu}_{1}^{2}
+\hat{f}_{2}\left(4,3\right)\frac{\hat{\mu}_{1}}{1-\hat{\mu}_{2}}
+\hat{f}_{2}\left(4,4\right)\left(\frac{\hat{\mu}_{1}}{1-\hat{\mu}_{2}}\right)^{2}
+\hat{f}_{2}\left(3,3\right)
+\hat{f}_{2}\left(4,3\right)\frac{\tilde{\mu}_{1}}{1-\hat{\mu}_{2}}
\end{eqnarray*}

Then according to the equations given at the beginning of this section, we have

\begin{eqnarray*}
D_{k}D_{i}F_{1}&=&D_{k}D_{i}\left(R_{2}+F_{2}+\indora_{i\geq3}\hat{F}_{4}\right)+D_{i}R_{2}D_{k}\left(F_{2}+\indora_{k\geq3}\hat{F}_{4}\right)\\&+&D_{i}F_{2}D_{k}\left(R_{2}+\indora_{k\geq3}\hat{F}_{4}\right)+\indora_{i\geq3}D_{i}\hat{F}_{4}D_{k}\left(R_{2}+F_{2}\right)
\end{eqnarray*}

for $i=1$, and $k=1$

\begin{eqnarray*}
D_{1}D_{1}F_{1}&=&D_{1}D_{1}\left(R_{2}+F_{2}\right)
+D_{1}R_{2}D_{1}F_{2}
+D_{1}F_{2}D_{1}R_{2}=D_{1}^{2}R_{2}+D_{1}^{2}F_{2}+D_{1}R_{2}D_{1}F_{2}
+D_{1}F_{2}D_{1}R_{2}
\end{eqnarray*}

$k=2$
\begin{eqnarray*}
D_{2}D_{i}F_{1}&=&D_{2}D_{1}\left(R_{2}+F_{2}\right)
+D_{1}R_{2}D_{2}F_{2}+D_{1}F_{2}D_{2}R_{2}=D_{2}D_{1}R_{2}+D_{2}D_{1}F_{2}+D_{1}R_{2}D_{2}F_{2}+D_{1}F_{2}D_{2}R_{2}
\end{eqnarray*}

$k=3$
\begin{eqnarray*}
D_{3}D_{1}F_{1}&=&D_{3}D_{1}\left(R_{2}+F_{2}\right)
+D_{1}R_{2}D_{3}\left(F_{2}+\hat{F}_{4}\right)
+D_{1}F_{2}D_{3}\left(R_{2}+\hat{F}_{4}\right)\\
&=&D_{3}D_{1}R_{2}+D_{3}D_{1}F_{2}
+D_{1}R_{2}D_{3}F_{2}+D_{1}R_{2}D_{3}\hat{F}_{4}
+D_{1}F_{2}D_{3}R_{2}+D_{1}F_{2}D_{3}\hat{F}_{4}
\end{eqnarray*}

$k=4$
\begin{eqnarray*}
D_{4}D_{1}F_{1}&=&D_{4}D_{1}\left(R_{2}+F_{2}\right)
+D_{1}R_{2}D_{4}\left(F_{2}+\hat{F}_{4}\right)
+D_{1}F_{2}D_{4}\left(R_{2}+\hat{F}_{4}\right)\\
&=&D_{4}D_{1}R_{2}+D_{4}D_{1}F_{2}
+D_{1}R_{2}D_{4}F_{2}+D_{1}R_{2}D_{4}\hat{F}_{4}
+D_{1}F_{2}D_{4}R_{2}+D_{1}F_{2}D_{4}\hat{F}_{4}
\end{eqnarray*}

for $i=2$, and $k=1$

\begin{eqnarray*}
D_{1}D_{2}F_{1}&=&D_{1}D_{2}\left(R_{2}+F_{2}\right)
+D_{2}R_{2}D_{1}F_{2}+D_{2}F_{2}D_{1}R_{2}=
D_{1}D_{2}R_{2}+D_{1}D_{2}F_{2}
+D_{2}R_{2}D_{1}F_{2}+D_{2}F_{2}D_{1}R_{2}
\end{eqnarray*}

$k=2$
\begin{eqnarray*}
D_{2}D_{2}F_{1}&=&D_{2}D_{2}\left(R_{2}+F_{2}\right)
+D_{2}R_{2}D_{2}F_{2}+D_{2}F_{2}D_{2}R_{2}
=D_{2}D_{2}R_{2}+D_{2}D_{2}F_{2}+D_{2}R_{2}D_{2}F_{2}+D_{2}F_{2}D_{2}R_{2}
\end{eqnarray*}

$k=3$
\begin{eqnarray*}
D_{3}D_{2}F_{1}&=&D_{3}D_{2}\left(R_{2}+F_{2}\right)
+D_{2}R_{2}D_{3}\left(F_{2}+\hat{F}_{4}\right)
+D_{2}F_{2}D_{3}\left(R_{2}+\hat{F}_{4}\right)\\
&=&D_{3}D_{2}R_{2}+D_{3}D_{2}F_{2}
+D_{2}R_{2}D_{3}F_{2}+D_{2}R_{2}D_{3}\hat{F}_{4}
+D_{2}F_{2}D_{3}R_{2}+D_{2}F_{2}D_{3}\hat{F}_{4}
\end{eqnarray*}

$k=4$
\begin{eqnarray*}
D_{4}D_{2}F_{1}&=&D_{4}D_{2}\left(R_{2}+F_{2}\right)
+D_{2}R_{2}D_{4}\left(F_{2}+\hat{F}_{4}\right)
+D_{2}F_{2}D_{4}\left(R_{2}+\hat{F}_{4}\right)\\
&=&D_{4}D_{2}R_{2}+D_{4}D_{2}F_{2}
+D_{2}R_{2}D_{4}F_{2}+D_{2}R_{2}D_{4}\hat{F}_{4}
+D_{2}F_{2}D_{4}R_{2}+D_{2}F_{2}D_{4}\hat{F}_{4}
\end{eqnarray*}

for $i=3$, and $k=1$

\begin{eqnarray*}
D_{1}D_{3}F_{1}&=&D_{1}D_{3}\left(R_{2}+F_{2}+\hat{F}_{4}\right)
+D_{3}R_{2}D_{1}F_{2}+D_{3}F_{2}D_{1}R_{2}
+D_{3}\hat{F}_{4}D_{1}\left(R_{2}+F_{2}\right)\\
&=&D_{1}D_{3}R_{2}+D_{1}D_{3}F_{2}+D_{1}D_{3}\hat{F}_{4}
+D_{3}R_{2}D_{1}F_{2}+D_{3}F_{2}D_{1}R_{2}
+D_{3}\hat{F}_{4}D_{1}R_{2}+D_{3}\hat{F}_{4}D_{1}F_{2}
\end{eqnarray*}

$k=2$
\begin{eqnarray*}
D_{2}D_{3}F_{1}&=&D_{2}D_{3}\left(R_{2}+F_{2}+\hat{F}_{4}\right)
+D_{3}R_{2}D_{2}F_{2}
+D_{3}F_{2}D_{2}R_{2}
+D_{3}\hat{F}_{4}D_{2}\left(R_{2}+F_{2}\right)\\
&=&D_{2}D_{3}R_{2}+D_{2}D_{3}F_{2}+D_{2}D_{3}\hat{F}_{4}
+D_{3}R_{2}D_{2}F_{2}+D_{3}F_{2}D_{2}R_{2}
+D_{3}\hat{F}_{4}D_{2}R_{2}+D_{3}\hat{F}_{4}D_{2}F_{2}
\end{eqnarray*}

$k=3$
\begin{eqnarray*}
D_{3}D_{3}F_{1}&=&D_{3}D_{3}\left(R_{2}+F_{2}+\hat{F}_{4}\right)
+D_{3}R_{2}D_{3}\left(F_{2}+\hat{F}_{4}\right)
+D_{3}F_{2}D_{3}\left(R_{2}+\hat{F}_{4}\right)
+D_{3}\hat{F}_{4}D_{3}\left(R_{2}+F_{2}\right)\\
&=&D_{3}D_{3}R_{2}+D_{3}D_{3}F_{2}+D_{3}D_{3}\hat{F}_{4}
+D_{3}R_{2}D_{3}F_{2}+D_{3}R_{2}D_{3}\hat{F}_{4}\\
&+&D_{3}F_{2}D_{3}R_{2}+D_{3}F_{2}D_{3}\hat{F}_{4}
+D_{3}\hat{F}_{4}D_{3}R_{2}+D_{3}\hat{F}_{4}D_{3}F_{2}
\end{eqnarray*}

$k=4$
\begin{eqnarray*}
D_{4}D_{3}F_{1}&=&D_{4}D_{3}\left(R_{2}+F_{2}+\hat{F}_{4}\right)
+D_{3}R_{2}D_{4}\left(F_{2}+\hat{F}_{4}\right)
+D_{3}F_{2}D_{4}\left(R_{2}+\hat{F}_{4}\right)
+D_{3}\hat{F}_{4}D_{4}\left(R_{2}+F_{2}\right)\\
&=&D_{4}D_{3}R_{2}+D_{4}D_{3}F_{2}+D_{4}D_{3}\hat{F}_{4}
+D_{3}R_{2}D_{4}F_{2}+D_{3}R_{2}D_{4}\hat{F}_{4}\\
&+&D_{3}F_{2}D_{4}R_{2}+D_{3}F_{2}D_{4}\hat{F}_{4}
+D_{3}\hat{F}_{4}D_{4}R_{2}+D_{3}\hat{F}_{4}D_{4}F_{2}
\end{eqnarray*}

for $i=4$, and $k=1$


\begin{eqnarray*}
D_{1}D_{4}F_{1}&=&D_{1}D_{4}\left(R_{2}+F_{2}+\hat{F}_{4}\right)
+D_{4}R_{2}D_{1}F_{2}
+D_{4}F_{2}D_{1}R_{2}
+D_{4}\hat{F}_{4}D_{1}\left(R_{2}+F_{2}\right)\\
&=&D_{1}D_{4}R_{2}+D_{1}D_{4}F_{2}+D_{1}D_{4}\hat{F}_{4}
+D_{4}R_{2}D_{1}F_{2}+D_{4}F_{2}D_{1}R_{2}
+D_{4}\hat{F}_{4}D_{1}R_{2}+D_{4}\hat{F}_{4}D_{1}F_{2}
\end{eqnarray*}

$k=2$
\begin{eqnarray*}
D_{2}D_{4}F_{1}&=&D_{2}D_{4}\left(R_{2}+F_{2}+\hat{F}_{4}\right)
+D_{4}R_{2}D_{2}F_{2}+D_{4}F_{2}D_{2}R_{2}
+D_{4}\hat{F}_{4}D_{2}\left(R_{2}+F_{2}\right)\\
&=&D_{2}D_{4}R_{2}+D_{2}D_{4}F_{2}+D_{2}D_{4}\hat{F}_{4}
+D_{4}R_{2}D_{2}F_{2}+D_{4}F_{2}D_{2}R_{2}
+D_{4}\hat{F}_{4}D_{2}R_{2}+D_{4}\hat{F}_{4}D_{2}F_{2}
\end{eqnarray*}

$k=3$
\begin{eqnarray*}
D_{3}D_{4}F_{1}&=&D_{3}D_{4}\left(R_{2}+F_{2}+\hat{F}_{4}\right)
+D_{4}R_{2}D_{3}\left(F_{2}+\hat{F}_{4}\right)
+D_{4}F_{2}D_{3}\left(R_{2}+\hat{F}_{4}\right)
+D_{4}\hat{F}_{4}D_{3}\left(R_{2}+F_{2}\right)\\
&=&D_{3}D_{4}R_{2}+D_{3}D_{4}F_{2}+D_{3}D_{4}\hat{F}_{4}
+D_{4}R_{2}D_{3}F_{2}+D_{4}R_{2}D_{3}\hat{F}_{4}\\
&+&D_{4}F_{2}D_{3}R_{2}+D_{4}F_{2}D_{3}\hat{F}_{4}
+D_{4}\hat{F}_{4}D_{3}R_{2}+D_{4}\hat{F}_{4}D_{3}F_{2}
\end{eqnarray*}


$k=4$
\begin{eqnarray*}
D_{4}D_{4}F_{1}&=&D_{4}D_{4}\left(R_{2}+F_{2}+\hat{F}_{4}\right)
+D_{4}R_{2}D_{4}\left(F_{2}+\hat{F}_{4}\right)
+D_{4}F_{2}D_{4}\left(R_{2}+\hat{F}_{4}\right)
+D_{4}\hat{F}_{4}D_{4}\left(R_{2}+F_{2}\right)\\
&=&D_{4}D_{4}R_{2}+D_{4}D_{4}F_{2}+D_{4}D_{4}\hat{F}_{4}
+D_{4}R_{2}D_{4}F_{2}+D_{4}R_{2}D_{4}\hat{F}_{4}\\
&+&D_{4}F_{2}D_{4}R_{2}+D_{4}F_{2}D_{4}\hat{F}_{4}
+D_{4}\hat{F}_{4}D_{4}R_{2}+D_{4}\hat{F}_{4}D_{4}F_{2}
\end{eqnarray*}


%_____________________________________________________________________________________
\newpage

%__________________________________________________________________
\section{Generalizaciones}
%__________________________________________________________________
\subsection{RSVC con dos conexiones}
%__________________________________________________________________

%\begin{figure}[H]
%\centering
%%%\includegraphics[width=9cm]{Grafica3.jpg}
%%\end{figure}\label{RSVC3}


Sus ecuaciones recursivas son de la forma


\begin{eqnarray*}
F_{1}\left(z_{1},z_{2},w_{1},w_{2}\right)&=&R_{2}\left(\prod_{i=1}^{2}\tilde{P}_{i}\left(z_{i}\right)\prod_{i=1}^{2}
\hat{P}_{i}\left(w_{i}\right)\right)F_{2}\left(z_{1},\tilde{\theta}_{2}\left(\tilde{P}_{1}\left(z_{1}\right)\hat{P}_{1}\left(w_{1}\right)\hat{P}_{2}\left(w_{2}\right)\right)\right)
\hat{F}_{2}\left(w_{1},w_{2};\tau_{2}\right),
\end{eqnarray*}

\begin{eqnarray*}
F_{2}\left(z_{1},z_{2},w_{1},w_{2}\right)&=&R_{1}\left(\prod_{i=1}^{2}\tilde{P}_{i}\left(z_{i}\right)\prod_{i=1}^{2}
\hat{P}_{i}\left(w_{i}\right)\right)F_{1}\left(\tilde{\theta}_{1}\left(\tilde{P}_{2}\left(z_{2}\right)\hat{P}_{1}\left(w_{1}\right)\hat{P}_{2}\left(w_{2}\right)\right),z_{2}\right)\hat{F}_{1}\left(w_{1},w_{2};\tau_{1}\right),
\end{eqnarray*}


\begin{eqnarray*}
\hat{F}_{1}\left(z_{1},z_{2},w_{1},w_{2}\right)&=&\hat{R}_{2}\left(\prod_{i=1}^{2}\tilde{P}_{i}\left(z_{i}\right)\prod_{i=1}^{2}
\hat{P}_{i}\left(w_{i}\right)\right)F_{2}\left(z_{1},z_{2};\zeta_{2}\right)\hat{F}_{2}\left(w_{1},\hat{\theta}_{2}\left(\tilde{P}_{1}\left(z_{1}\right)\tilde{P}_{2}\left(z_{2}\right)\hat{P}_{1}\left(w_{1}
\right)\right)\right),
\end{eqnarray*}


\begin{eqnarray*}
\hat{F}_{2}\left(z_{1},z_{2},w_{1},w_{2}\right)&=&\hat{R}_{1}\left(\prod_{i=1}^{2}\tilde{P}_{i}\left(z_{i}\right)\prod_{i=1}^{2}
\hat{P}_{i}\left(w_{i}\right)\right)F_{1}\left(z_{1},z_{2};\zeta_{1}\right)\hat{F}_{1}\left(\hat{\theta}_{1}\left(\tilde{P}_{1}\left(z_{1}\right)\tilde{P}_{2}\left(z_{2}\right)\hat{P}_{2}\left(w_{2}\right)\right),w_{2}\right),
\end{eqnarray*}

%_____________________________________________________
\subsection{First Moments of the Queue Lengths}
%_____________________________________________________


The server's switchover times are given by the general equation

\begin{eqnarray}\label{Ec.Ri}
R_{i}\left(\mathbf{z,w}\right)=R_{i}\left(\tilde{P}_{1}\left(z_{1}\right)\tilde{P}_{2}\left(z_{2}\right)\hat{P}_{1}\left(w_{1}\right)\hat{P}_{2}\left(w_{2}\right)\right)
\end{eqnarray}

with
\begin{eqnarray}\label{Ec.Derivada.Ri}
D_{i}R_{i}&=&DR_{i}D_{i}P_{i}
\end{eqnarray}
the following notation is considered

\begin{eqnarray*}
\begin{array}{llll}
D_{1}P_{1}\equiv D_{1}\tilde{P}_{1}, & D_{2}P_{2}\equiv D_{2}\tilde{P}_{2}, & D_{3}P_{3}\equiv D_{3}\hat{P}_{1}, &D_{4}P_{4}\equiv D_{4}\hat{P}_{2},
\end{array}
\end{eqnarray*}

also we need to remind $F_{1,2}\left(z_{1};\zeta_{2}\right)F_{2,2}\left(z_{2};\zeta_{2}\right)=F_{2}\left(z_{1},z_{2};\zeta_{2}\right)$, therefore

\begin{eqnarray*}
D_{1}F_{2}\left(z_{1},z_{2};\zeta_{2}\right)&=&D_{1}\left[F_{1,2}\left(z_{1};\zeta_{2}\right)F_{2,2}\left(z_{2};\zeta_{2}\right)\right]
=F_{2,2}\left(z_{2};\zeta_{2}\right)D_{1}F_{1,2}\left(z_{1};\zeta_{2}\right)=F_{1,2}^{(1)}\left(1\right)
\end{eqnarray*}

i.e., $D_{1}F_{2}=F_{1,2}^{(1)}(1)$; $D_{2}F_{2}=F_{2,2}^{(1)}\left(1\right)$, whereas that $D_{3}F_{2}=D_{4}F_{2}=0$, then

\begin{eqnarray}
\begin{array}{ccc}
D_{i}F_{j}=\indora_{i\leq2}F_{i,j}^{(1)}\left(1\right),& \textrm{ and } &D_{i}\hat{F}_{j}=\indora_{i\geq2}F_{i,j}^{(1)}\left(1\right).
\end{array}
\end{eqnarray}

Now, we obtain the first moments equations for the queue lengths as before for the times the server arrives to the queue to start attending



Remember that


\begin{eqnarray*}
F_{2}\left(z_{1},z_{2},w_{1},w_{2}\right)&=&R_{1}\left(\prod_{i=1}^{2}\tilde{P}_{i}\left(z_{i}\right)\prod_{i=1}^{2}
\hat{P}_{i}\left(w_{i}\right)\right)F_{1}\left(\tilde{\theta}_{1}\left(\tilde{P}_{2}\left(z_{2}\right)\hat{P}_{1}\left(w_{1}\right)\hat{P}_{2}\left(w_{2}\right)\right),z_{2}\right)\hat{F}_{1}\left(w_{1},w_{2};\tau_{1}\right),
\end{eqnarray*}

where


\begin{eqnarray*}
F_{1}\left(\tilde{\theta}_{1}\left(\tilde{P}_{2}\hat{P}_{1}\hat{P}_{2}\right),z_{2}\right)
\end{eqnarray*}

so

\begin{eqnarray}
D_{i}F_{1}&=&\indora_{i\neq1}D_{1}F_{1}D\tilde{\theta}_{1}D_{i}P_{i}+\indora_{i=2}D_{i}F_{1},
\end{eqnarray}

then


\begin{eqnarray*}
\begin{array}{ll}
D_{1}F_{1}=0,&
D_{2}F_{1}=D_{1}F_{1}D\tilde{\theta}_{1}D_{2}P_{2}+D_{2}F_{1}
=f_{1}\left(1\right)\frac{1}{1-\tilde{\mu}_{1}}\tilde{\mu}_{2}+f_{1}\left(2\right),\\
D_{3}F_{1}=D_{1}F_{1}D\tilde{\theta}_{1}D_{3}P_{3}
=f_{1}\left(1\right)\frac{1}{1-\tilde{\mu}_{1}}\hat{\mu}_{1},&
D_{4}F_{1}=D_{1}F_{1}D\tilde{\theta}_{1}D_{4}P_{4}
=f_{1}\left(1\right)\frac{1}{1-\tilde{\mu}_{1}}\hat{\mu}_{2}

\end{array}
\end{eqnarray*}


\begin{eqnarray}
D_{i}F_{2}&=&\indora_{i\neq2}D_{2}F_{2}D\tilde{\theta}_{2}D_{i}P_{i}
+\indora_{i=1}D_{i}F_{2}
\end{eqnarray}

\begin{eqnarray*}
\begin{array}{ll}
D_{1}F_{2}=D_{2}F_{2}D\tilde{\theta}_{2}D_{1}P_{1}
+D_{1}F_{2}=f_{2}\left(2\right)\frac{1}{1-\tilde{\mu}_{2}}\tilde{\mu}_{1},&
D_{2}F_{2}=0\\
D_{3}F_{2}=D_{2}F_{2}D\tilde{\theta}_{2}D_{3}P_{3}
=f_{2}\left(2\right)\frac{1}{1-\tilde{\mu}_{2}}\hat{\mu}_{1},&
D_{4}F_{2}=D_{2}F_{2}D\tilde{\theta}_{2}D_{4}P_{4}
=f_{2}\left(2\right)\frac{1}{1-\tilde{\mu}_{2}}\hat{\mu}_{2}
\end{array}
\end{eqnarray*}



\begin{eqnarray}
D_{i}\hat{F}_{1}&=&\indora_{i\neq3}D_{3}\hat{F}_{1}D\hat{\theta}_{1}D_{i}P_{i}+\indora_{i=4}D_{i}\hat{F}_{1},
\end{eqnarray}

\begin{eqnarray*}
\begin{array}{ll}
D_{1}\hat{F}_{1}=D_{3}\hat{F}_{1}D\hat{\theta}_{1}D_{1}P_{1}=\hat{f}_{1}\left(3\right)\frac{1}{1-\hat{\mu}_{1}}\tilde{\mu}_{1},&
D_{2}\hat{F}_{1}=D_{3}\hat{F}_{1}D\hat{\theta}_{1}D_{2}P_{2}
=\hat{f}_{1}\left(3\right)\frac{1}{1-\hat{\mu}_{1}}\tilde{\mu}_{2}\\
D_{3}\hat{F}_{1}=0,&
D_{4}\hat{F}_{1}=D_{3}\hat{F}_{1}D\hat{\theta}_{1}D_{4}P_{4}
+D_{4}\hat{F}_{1}
=\hat{f}_{1}\left(3\right)\frac{1}{1-\hat{\mu}_{1}}\hat{\mu}_{2}+\hat{f}_{1}\left(2\right),

\end{array}
\end{eqnarray*}


\begin{eqnarray}
D_{i}\hat{F}_{2}&=&\indora_{i\neq4}D_{4}\hat{F}_{2}D\hat{\theta}_{2}D_{i}P_{i}+\indora_{i=3}D_{i}\hat{F}_{2}.
\end{eqnarray}

\begin{eqnarray*}
\begin{array}{ll}
D_{1}\hat{F}_{2}=D_{4}\hat{F}_{2}D\hat{\theta}_{2}D_{1}P_{1}
=\hat{f}_{2}\left(4\right)\frac{1}{1-\hat{\mu}_{2}}\tilde{\mu}_{1},&
D_{2}\hat{F}_{2}=D_{4}\hat{F}_{2}D\hat{\theta}_{2}D_{2}P_{2}
=\hat{f}_{2}\left(4\right)\frac{1}{1-\hat{\mu}_{2}}\tilde{\mu}_{2},\\
D_{3}\hat{F}_{2}=D_{4}\hat{F}_{2}D\hat{\theta}_{2}D_{3}P_{3}+D_{3}\hat{F}_{2}
=\hat{f}_{2}\left(4\right)\frac{1}{1-\hat{\mu}_{2}}\hat{\mu}_{1}+\hat{f}_{2}\left(4\right)\\
D_{4}\hat{F}_{2}=0

\end{array}
\end{eqnarray*}
Then, now we can obtain the linear system of equations in order to obtain the first moments of the lengths of the queues:



For $\mathbf{F}_{1}=R_{2}F_{2}\hat{F}_{2}$ we get the general equations

\begin{eqnarray}
D_{i}\mathbf{F}_{1}=D_{i}\left(R_{2}+F_{2}+\indora_{i\geq3}\hat{F}_{2}\right)
\end{eqnarray}

So

\begin{eqnarray*}
D_{1}\mathbf{F}_{1}&=&D_{1}R_{2}+D_{1}F_{2}
=r_{1}\tilde{\mu}_{1}+f_{2}\left(2\right)\frac{1}{1-\tilde{\mu}_{2}}\tilde{\mu}_{1}\\
D_{2}\mathbf{F}_{1}&=&D_{2}\left(R_{2}+F_{2}\right)
=r_{2}\tilde{\mu}_{1}\\
D_{3}\mathbf{F}_{1}&=&D_{3}\left(R_{2}+F_{2}+\hat{F}_{2}\right)
=r_{1}\hat{\mu}_{1}+f_{2}\left(2\right)\frac{1}{1-\tilde{\mu}_{2}}\hat{\mu}_{1}+\hat{F}_{1,2}^{(1)}\left(1\right)\\
D_{4}\mathbf{F}_{1}&=&D_{4}\left(R_{2}+F_{2}+\hat{F}_{2}\right)
=r_{2}\hat{\mu}_{2}+f_{2}\left(2\right)\frac{1}{1-\tilde{\mu}_{2}}\hat{\mu}_{2}
+\hat{F}_{2,2}^{(1)}\left(1\right)
\end{eqnarray*}

it means

\begin{eqnarray*}
\begin{array}{ll}
D_{1}\mathbf{F}_{1}=r_{2}\hat{\mu}_{1}+f_{2}\left(2\right)\left(\frac{1}{1-\tilde{\mu}_{2}}\right)\tilde{\mu}_{1}+f_{2}\left(1\right),&
D_{2}\mathbf{F}_{1}=r_{2}\tilde{\mu}_{2},\\
D_{3}\mathbf{F}_{1}=r_{2}\hat{\mu}_{1}+f_{2}\left(2\right)\left(\frac{1}{1-\tilde{\mu}_{2}}\right)\hat{\mu}_{1}+\hat{F}_{1,2}^{(1)}\left(1\right),&
D_{4}\mathbf{F}_{1}=r_{2}\hat{\mu}_{2}+f_{2}\left(2\right)\left(\frac{1}{1-\tilde{\mu}_{2}}\right)\hat{\mu}_{2}+\hat{F}_{2,2}^{(1)}\left(1\right),\end{array}
\end{eqnarray*}


\begin{eqnarray}
\begin{array}{ll}
\mathbf{F}_{2}=R_{1}F_{1}\hat{F}_{1}, & D_{i}\mathbf{F}_{2}=D_{i}\left(R_{1}+F_{1}+\indora_{i\geq3}\hat{F}_{1}\right)\\
\end{array}
\end{eqnarray}



equivalently


\begin{eqnarray*}
\begin{array}{ll}
D_{1}\mathbf{F}_{2}=r_{1}\tilde{\mu}_{1},&
D_{2}\mathbf{F}_{2}=r_{1}\tilde{\mu}_{2}+f_{1}\left(1\right)\left(\frac{1}{1-\tilde{\mu}_{1}}\right)\tilde{\mu}_{2}+f_{1}\left(2\right),\\
D_{3}\mathbf{F}_{2}=r_{1}\hat{\mu}_{1}+f_{1}\left(1\right)\left(\frac{1}{1-\tilde{\mu}_{1}}\right)\hat{\mu}_{1}+\hat{F}_{1,1}^{(1)}\left(1\right),&
D_{4}\mathbf{F}_{2}=r_{1}\hat{\mu}_{2}+f_{1}\left(1\right)\left(\frac{1}{1-\tilde{\mu}_{1}}\right)\hat{\mu}_{2}+\hat{F}_{2,1}^{(1)}\left(1\right),\\
\end{array}
\end{eqnarray*}



\begin{eqnarray}
\begin{array}{ll}
\hat{\mathbf{F}}_{1}=\hat{R}_{2}\hat{F}_{2}F_{2}, & D_{i}\hat{\mathbf{F}}_{1}=D_{i}\left(\hat{R}_{2}+\hat{F}_{2}+\indora_{i\leq2}F_{2}\right)\\
\end{array}
\end{eqnarray}


equivalently


\begin{eqnarray*}
\begin{array}{ll}
D_{1}\hat{\mathbf{F}}_{1}=\hat{r}_{2}\tilde{\mu}_{1}+\hat{f}_{2}\left(2\right)\left(\frac{1}{1-\hat{\mu}_{2}}\right)\tilde{\mu}_{1}+F_{1,2}^{(1)}\left(1\right),&
D_{2}\hat{\mathbf{F}}_{1}=\hat{r}_{2}\tilde{\mu}_{2}+\hat{f}_{2}\left(2\right)\left(\frac{1}{1-\hat{\mu}_{2}}\right)\tilde{\mu}_{2}+F_{2,2}^{(1)}\left(1\right),\\
D_{3}\hat{\mathbf{F}}_{1}=\hat{r}_{2}\hat{\mu}_{1}+\hat{f}_{2}\left(2\right)\left(\frac{1}{1-\hat{\mu}_{2}}\right)\hat{\mu}_{1}+\hat{f}_{2}\left(1\right),&
D_{4}\hat{\mathbf{F}}_{1}=\hat{r}_{2}\hat{\mu}_{2}
\end{array}
\end{eqnarray*}



\begin{eqnarray}
\begin{array}{ll}
\hat{\mathbf{F}}_{2}=\hat{R}_{1}\hat{F}_{1}F_{1}, & D_{i}\hat{\mathbf{F}}_{2}=D_{i}\left(\hat{R}_{1}+\hat{F}_{1}+\indora_{i\leq2}F_{1}\right)
\end{array}
\end{eqnarray}



equivalently


\begin{eqnarray*}
\begin{array}{ll}
D_{1}\hat{\mathbf{F}}_{2}=\hat{r}_{1}\tilde{\mu}_{1}+\hat{f}_{1}\left(1\right)\left(\frac{1}{1-\hat{\mu}_{1}}\right)\tilde{\mu}_{1}+F_{1,1}^{(1)}\left(1\right),&
D_{2}\hat{\mathbf{F}}_{2}=\hat{r}_{1}\mu_{2}+\hat{f}_{1}\left(1\right)\left(\frac{1}{1-\hat{\mu}_{1}}\right)\tilde{\mu}_{2}+F_{2,1}^{(1)}\left(1\right),\\
D_{3}\hat{\mathbf{F}}_{2}=\hat{r}_{1}\hat{\mu}_{1},&
D_{4}\hat{\mathbf{F}}_{2}=\hat{r}_{1}\hat{\mu}_{2}+\hat{f}_{1}\left(1\right)\left(\frac{1}{1-\hat{\mu}_{1}}\right)\hat{\mu}_{2}+\hat{f}_{1}\left(2\right),\\
\end{array}
\end{eqnarray*}





Then we have that if $\mu=\tilde{\mu}_{1}+\tilde{\mu}_{2}$, $\hat{\mu}=\hat{\mu}_{1}+\hat{\mu}_{2}$, $r=r_{1}+r_{2}$ and $\hat{r}=\hat{r}_{1}+\hat{r}_{2}$  the system's solution is given by

\begin{eqnarray*}
\begin{array}{llll}
f_{2}\left(1\right)=r_{1}\tilde{\mu}_{1},&
f_{1}\left(2\right)=r_{2}\tilde{\mu}_{2},&
\hat{f}_{1}\left(4\right)=\hat{r}_{2}\hat{\mu}_{2},&
\hat{f}_{2}\left(3\right)=\hat{r}_{1}\hat{\mu}_{1}
\end{array}
\end{eqnarray*}



it's easy to verify that

\begin{eqnarray}\label{Sist.Ec.Lineales.Doble.Traslado}
\begin{array}{ll}
f_{1}\left(1\right)=\tilde{\mu}_{1}\left(r+\frac{f_{2}\left(2\right)}{1-\tilde{\mu}_{2}}\right),& f_{1}\left(3\right)=\hat{\mu}_{1}\left(r_{2}+\frac{f_{2}\left(2\right)}{1-\tilde{\mu}_{2}}\right)+\hat{F}_{1,2}^{(1)}\left(1\right)\\
f_{1}\left(4\right)=\hat{\mu}_{2}\left(r_{2}+\frac{f_{2}\left(2\right)}{1-\tilde{\mu}_{2}}\right)+\hat{F}_{2,2}^{(1)}\left(1\right),&
f_{2}\left(2\right)=\left(r+\frac{f_{1}\left(1\right)}{1-\mu_{1}}\right)\tilde{\mu}_{2},\\
f_{2}\left(3\right)=\hat{\mu}_{1}\left(r_{1}+\frac{f_{1}\left(1\right)}{1-\tilde{\mu}_{1}}\right)+\hat{F}_{1,1}^{(1)}\left(1\right),&
f_{2}\left(4\right)=\hat{\mu}_{2}\left(r_{1}+\frac{f_{1}\left(1\right)}{1-\mu_{1}}\right)+\hat{F}_{2,1}^{(1)}\left(1\right),\\
\hat{f}_{1}\left(1\right)=\left(\hat{r}_{2}+\frac{\hat{f}_{2}\left(4\right)}{1-\hat{\mu}_{2}}\right)\tilde{\mu}_{1}+F_{1,2}^{(1)}\left(1\right),&
\hat{f}_{1}\left(2\right)=\left(\hat{r}_{2}+\frac{\hat{f}_{2}\left(4\right)}{1-\hat{\mu}_{2}}\right)\tilde{\mu}_{2}+F_{2,2}^{(1)}\left(1\right),\\
\hat{f}_{1}\left(3\right)=\left(\hat{r}+\frac{\hat{f}_{2}\left(4\right)}{1-\hat{\mu}_{2}}\right)\hat{\mu}_{1},&
\hat{f}_{2}\left(1\right)=\left(\hat{r}_{1}+\frac{\hat{f}_{1}\left(3\right)}{1-\hat{\mu}_{1}}\right)\mu_{1}+F_{1,1}^{(1)}\left(1\right),\\
\hat{f}_{2}\left(2\right)=\left(\hat{r}_{1}+\frac{\hat{f}_{1}\left(3\right)}{1-\hat{\mu}_{1}}\right)\tilde{\mu}_{2}+F_{2,1}^{(1)}\left(1\right),&
\hat{f}_{2}\left(4\right)=\left(\hat{r}+\frac{\hat{f}_{1}\left(3\right)}{1-\hat{\mu}_{1}}\right)\hat{\mu}_{2},\\
\end{array}
\end{eqnarray}

with system's solutions given by

\begin{eqnarray}
\begin{array}{ll}
f_{1}\left(1\right)=r\frac{\mu_{1}\left(1-\mu_{1}\right)}{1-\mu},&
f_{2}\left(2\right)=r\frac{\tilde{\mu}_{2}\left(1-\tilde{\mu}_{2}\right)}{1-\mu},\\
f_{1}\left(3\right)=\hat{\mu}_{1}\left(r_{2}+\frac{r\tilde{\mu}_{2}}{1-\mu}\right)+\hat{F}_{1,2}^{(1)}\left(1\right),&
f_{1}\left(4\right)=\hat{\mu}_{2}\left(r_{2}+\frac{r\tilde{\mu}_{2}}{1-\mu}\right)+\hat{F}_{2,2}^{(1)}\left(1\right),\\
f_{2}\left(3\right)=\hat{\mu}_{1}\left(r_{1}+\frac{r\mu_{1}}{1-\mu}\right)+\hat{F}_{1,1}^{(1)}\left(1\right),&
f_{2}\left(4\right)=\hat{\mu}_{2}\left(r_{1}+\frac{r\mu_{1}}{1-\mu}\right)+\hat{F}_{2,1}^{(1)}\left(1\right),\\
\hat{f}_{1}\left(1\right)=\tilde{\mu}_{1}\left(\hat{r}_{2}+\frac{\hat{r}\hat{\mu}_{2}}{1-\hat{\mu}}\right)+F_{1,2}^{(1)}\left(1\right),&
\hat{f}_{1}\left(2\right)=\tilde{\mu}_{2}\left(\hat{r}_{2}+\frac{\hat{r}\hat{\mu}_{2}}{1-\hat{\mu}}\right)+F_{2,2}^{(1)}\left(1\right),\\
\hat{f}_{2}\left(1\right)=\tilde{\mu}_{1}\left(\hat{r}_{1}+\frac{\hat{r}\hat{\mu}_{1}}{1-\hat{\mu}}\right)+F_{1,1}^{(1)}\left(1\right),&
\hat{f}_{2}\left(2\right)=\tilde{\mu}_{2}\left(\hat{r}_{1}+\frac{\hat{r}\hat{\mu}_{1}}{1-\hat{\mu}}\right)+F_{2,1}^{(1)}\left(1\right)
\end{array}
\end{eqnarray}

%_________________________________________________________________________________________________________
\subsection*{General Second Order Derivatives}
%_________________________________________________________________________________________________________


Now, taking the second order derivative with respect to the equations given in (\ref{Sist.Ec.Lineales.Doble.Traslado}) we obtain it in their general form

\small{
\begin{eqnarray*}\label{Ec.Derivadas.Segundo.Orden.Doble.Transferencia}
D_{k}D_{i}F_{1}&=&D_{k}D_{i}\left(R_{2}+F_{2}+\indora_{i\geq3}\hat{F}_{4}\right)+D_{i}R_{2}D_{k}\left(F_{2}+\indora_{k\geq3}\hat{F}_{4}\right)+D_{i}F_{2}D_{k}\left(R_{2}+\indora_{k\geq3}\hat{F}_{4}\right)+\indora_{i\geq3}D_{i}\hat{F}_{4}D_{k}\left(R_{}+F_{2}\right)\\
D_{k}D_{i}F_{2}&=&D_{k}D_{i}\left(R_{1}+F_{1}+\indora_{i\geq3}\hat{F}_{3}\right)+D_{i}R_{1}D_{k}\left(F_{1}+\indora_{k\geq3}\hat{F}_{3}\right)+D_{i}F_{1}D_{k}\left(R_{1}+\indora_{k\geq3}\hat{F}_{3}\right)+\indora_{i\geq3}D_{i}\hat{F}_{3}D_{k}\left(R_{1}+F_{1}\right)\\
D_{k}D_{i}\hat{F}_{3}&=&D_{k}D_{i}\left(\hat{R}_{4}+\indora_{i\leq2}F_{2}+\hat{F}_{4}\right)+D_{i}\hat{R}_{4}D_{k}\left(\indora_{k\leq2}F_{2}+\hat{F}_{4}\right)+D_{i}\hat{F}_{4}D_{k}\left(\hat{R}_{4}+\indora_{k\leq2}F_{2}\right)+\indora_{i\leq2}D_{i}F_{2}D_{k}\left(\hat{R}_{4}+\hat{F}_{4}\right)\\
D_{k}D_{i}\hat{F}_{4}&=&D_{k}D_{i}\left(\hat{R}_{3}+\indora_{i\leq2}F_{1}+\hat{F}_{3}\right)+D_{i}\hat{R}_{3}D_{k}\left(\indora_{k\leq2}F_{1}+\hat{F}_{3}\right)+D_{i}\hat{F}_{3}D_{k}\left(\hat{R}_{3}+\indora_{k\leq2}F_{1}\right)+\indora_{i\leq2}D_{i}F_{1}D_{k}\left(\hat{R}_{3}+\hat{F}_{3}\right)
\end{eqnarray*}}
for $i,k=1,\ldots,4$. In order to have it in an specific way we need to compute the expressions $D_{k}D_{i}\left(R_{2}+F_{2}+\indora_{i\geq3}\hat{F}_{4}\right)$

%_________________________________________________________________________________________________________
\subsubsection*{Second Order Derivatives: Serve's Switchover Times}
%_________________________________________________________________________________________________________

Remind $R_{i}\left(z_{1},z_{2},w_{1},w_{2}\right)=R_{i}\left(P_{1}\left(z_{1}\right)\tilde{P}_{2}\left(z_{2}\right)
\hat{P}_{1}\left(w_{1}\right)\hat{P}_{2}\left(w_{2}\right)\right)$,  which we will write in his reduced form $R_{i}=R_{i}\left(
P_{1}\tilde{P}_{2}\hat{P}_{1}\hat{P}_{2}\right)$, and according to the notation given in \cite{Lang} we obtain

\begin{eqnarray}
D_{i}D_{i}R_{k}=D^{2}R_{k}\left(D_{i}P_{i}\right)^{2}+DR_{k}D_{i}D_{i}P_{i}
\end{eqnarray}

whereas for $i\neq j$

\begin{eqnarray}
D_{i}D_{j}R_{k}=D^{2}R_{k}D_{i}P_{i}D_{j}P_{j}+DR_{k}D_{j}P_{j}D_{i}P_{i}
\end{eqnarray}

%_________________________________________________________________________________________________________
\subsubsection*{Second Order Derivatives: Queue Lengths}
%_________________________________________________________________________________________________________

Just like before the expression $F_{1}\left(\tilde{\theta}_{1}\left(\tilde{P}_{2}\left(z_{2}\right)\hat{P}_{1}\left(w_{1}\right)\hat{P}_{2}\left(w_{2}\right)\right),
z_{2}\right)$, will be denoted by $F_{1}\left(\tilde{\theta}_{1}\left(\tilde{P}_{2}\hat{P}_{1}\hat{P}_{2}\right),z_{2}\right)$, then the mixed partial derivatives are:

\begin{eqnarray*}
D_{j}D_{i}F_{1}&=&\indora_{i,j\neq1}D_{1}D_{1}F_{1}\left(D\tilde{\theta}_{1}\right)^{2}D_{i}P_{i}D_{j}P_{j}
+\indora_{i,j\neq1}D_{1}F_{1}D^{2}\tilde{\theta}_{1}D_{i}P_{i}D_{j}P_{j}
+\indora_{i,j\neq1}D_{1}F_{1}D\tilde{\theta}_{1}\left(\indora_{i=j}D_{i}^{2}P_{i}+\indora_{i\neq j}D_{i}P_{i}D_{j}P_{j}\right)\\
&+&\left(1-\indora_{i=j=3}\right)\indora_{i+j\leq6}D_{1}D_{2}F_{1}D\tilde{\theta}_{1}\left(\indora_{i<j}D_{j}P_{j}+\indora_{i>j}D_{i}P_{i}\right)
+\indora_{i=2}\left(D_{1}D_{2}F_{1}D\tilde{\theta}_{1}D_{i}P_{i}+D_{i}^{2}F_{1}\right)
\end{eqnarray*}


Recall the expression for $F_{1}\left(\tilde{\theta}_{1}\left(\tilde{P}_{2}\left(z_{2}\right)\hat{P}_{1}\left(w_{1}\right)\hat{P}_{2}\left(w_{2}\right)\right),
z_{2}\right)$, which is denoted by $F_{1}\left(\tilde{\theta}_{1}\left(\tilde{P}_{2}\hat{P}_{1}\hat{P}_{2}\right),z_{2}\right)$, then the mixed partial derivatives are given by

\begin{eqnarray*}
\begin{array}{llll}
D_{1}D_{1}F_{1}=0,&
D_{2}D_{1}F_{1}=0,&
D_{3}D_{1}F_{1}=0,&
D_{4}D_{1}F_{1}=0,\\
D_{1}D_{2}F_{1}=0,&
D_{1}D_{3}F_{1}=0,&
D_{1}D_{4}F_{1}=0,&
\end{array}
\end{eqnarray*}

\begin{eqnarray*}
D_{2}D_{2}F_{1}&=&D_{1}^{2}F_{1}\left(D\tilde{\theta}_{1}\right)^{2}\left(D_{2}\tilde{P}_{2}\right)^{2}
+D_{1}F_{1}D^{2}\tilde{\theta}_{1}\left(D_{2}\tilde{P}_{2}\right)^{2}
+D_{1}F_{1}D\tilde{\theta}_{1}D_{2}^{2}\tilde{P}_{2}
+D_{1}D_{2}F_{1}D\tilde{\theta}_{1}D_{2}\tilde{P}_{2}\\
&+&D_{1}D_{2}F_{1}D\tilde{\theta}_{1}D_{2}\tilde{P}_{2}+D_{2}D_{2}F_{1}\\
&=&f_{1}\left(1,1\right)\left(\frac{\tilde{\mu}_{2}}{1-\tilde{\mu}_{1}}\right)^{2}
+f_{1}\left(1\right)\tilde{\theta}_{1}^(2)\tilde{\mu}_{2}^{(2)}
+f_{1}\left(1\right)\frac{1}{1-\tilde{\mu}_{1}}\tilde{P}_{2}^{(2)}+f_{1}\left(1,2\right)\frac{\tilde{\mu}_{2}}{1-\tilde{\mu}_{1}}+f_{1}\left(1,2\right)\frac{\tilde{\mu}_{2}}{1-\tilde{\mu}_{1}}+f_{1}\left(2,2\right)
\end{eqnarray*}

\begin{eqnarray*}
D_{3}D_{2}F_{1}&=&D_{1}^{2}F_{1}\left(D\tilde{\theta}_{1}\right)^{2}D_{3}\hat{P}_{1}D_{2}\tilde{P}_{2}+D_{1}F_{1}D^{2}\tilde{\theta}_{1}D_{3}\hat{P}_{1}D_{2}\tilde{P}_{2}+D_{1}F_{1}D\tilde{\theta}_{1}D_{2}\tilde{P}_{2}D_{3}\hat{P}_{1}+D_{1}D_{2}F_{1}D\tilde{\theta}_{1}D_{3}\hat{P}_{1}\\
&=&f_{1}\left(1,1\right)\left(\frac{1}{1-\tilde{\mu}_{1}}\right)^{2}\tilde{\mu}_{2}\hat{\mu}_{1}+f_{1}\left(1\right)\tilde{\theta}_{1}^{(2)}\tilde{\mu}_{2}\hat{\mu}_{1}+f_{1}\left(1\right)\frac{\tilde{\mu}_{2}\hat{\mu}_{1}}{1-\tilde{\mu}_{1}}+f_{1}\left(1,2\right)\frac{\hat{\mu}_{1}}{1-\tilde{\mu}_{1}}
\end{eqnarray*}

\begin{eqnarray*}
D_{4}D_{2}F_{1}&=&D_{1}^{2}F_{1}\left(D\tilde{\theta}_{1}\right)^{2}D_{4}\hat{P}_{2}D_{2}\tilde{P}_{2}+D_{1}F_{1}D^{2}\tilde{\theta}_{1}D_{4}\hat{P}_{2}D_{2}\tilde{P}_{2}+D_{1}F_{1}D\tilde{\theta}_{1}D_{2}\tilde{P}_{2}D_{4}\hat{P}_{2}+D_{1}D_{2}F_{1}D\tilde{\theta}_{1}D_{4}\hat{P}_{2}\\
&=&f_{1}\left(1,1\right)\left(\frac{1}{1-\tilde{\mu}_{1}}\right)^{2}\tilde{\mu}_{2}\hat{\mu}_{2}+f_{1}\left(1\right)\tilde{\theta}_{1}^{(2)}\tilde{\mu}_{2}\hat{\mu}_{2}+f_{1}\left(1\right)\frac{\tilde{\mu}_{2}\hat{\mu}_{2}}{1-\tilde{\mu}_{1}}+f_{1}\left(1,2\right)\frac{\hat{\mu}_{2}}{1-\tilde{\mu}_{1}}
\end{eqnarray*}

\begin{eqnarray*}
D_{2}D_{3}F_{1}&=&
D_{1}^{2}F_{1}\left(D\tilde{\theta}_{1}\right)^{2}D_{2}\tilde{P}_{2}D_{3}\hat{P}_{1}+
D_{2}D_{1}F_{1}D\tilde{\theta}_{1}D_{3}\hat{P}_{1}+
D_{1}F_{1}D^{2}\tilde{\theta}_{1}D_{2}\tilde{P}_{2}D_{3}\hat{P}_{1}+
D_{1}F_{1}D\tilde{\theta}_{1}D_{3}\hat{P}_{1}D_{2}\tilde{P}_{2}\\
&=&f_{1}\left(1,1\right)\left(\frac{1}{1-\tilde{\mu}_{1}}\right)^{2}\tilde{\mu}_{2}\hat{\mu}_{1}+f_{1}\left(1\right)\tilde{\theta}_{1}^{(2)}\tilde{\mu}_{2}\hat{\mu}_{1}+f_{1}\left(1\right)\frac{\tilde{\mu}_{2}\hat{\mu}_{1}}{1-\tilde{\mu}_{1}}+f_{1}\left(1,2\right)\frac{\hat{\mu}_{1}}{1-\tilde{\mu}_{1}}
\end{eqnarray*}

\begin{eqnarray*}
D_{3}D_{3}F_{1}&=&D_{1}^{2}F_{1}\left(D\tilde{\theta}_{1}\right)^{2}\left(D_{3}\hat{P}_{1}\right)^{2}+D_{1}F_{1}D^{2}\tilde{\theta}_{1}\left(D_{3}\hat{P}_{1}\right)^{2}+D_{1}F_{1}D\tilde{\theta}_{1}D_{3}^{2}\hat{P}_{1}\\
&=&f_{1}\left(1,1\right)\left(\frac{\hat{\mu}_{1}}{1-\tilde{\mu}_{1}}\right)^{2}+f_{1}\left(1\right)\tilde{\theta}_{1}^{(2)}\hat{\mu}_{1}^{2}+f_{1}\left(1\right)\frac{\hat{\mu}_{1}^{2}}{1-\tilde{\mu}_{1}}
\end{eqnarray*}

\begin{eqnarray*}
D_{4}D_{3}F_{1}&=&D_{1}^{2}F_{1}\left(D\tilde{\theta}_{1}\right)^{2}D_{4}\hat{P}_{2}D_{3}\hat{P}_{1}+D_{1}F_{1}D^{2}\tilde{\theta}_{1}D_{4}\hat{P}_{2}D_{3}\hat{P}_{1}+D_{1}F_{1}D\tilde{\theta}_{1}D_{3}\hat{P}_{1}D_{4}\hat{P}_{2}\\
&=&f_{1}\left(1,1\right)\left(\frac{1}{1-\tilde{\mu}_{1}}\right)^{2}\hat{\mu}_{1}\hat{\mu}_{2}
+f_{1}\left(1\right)\tilde{\theta}_{1}^{2}\hat{\mu}_{2}\hat{\mu}_{1}
+f_{1}\left(1\right)\frac{\hat{\mu}_{2}\hat{\mu}_{1}}{1-\tilde{\mu}_{1}}
\end{eqnarray*}

\begin{eqnarray*}
D_{2}D_{4}F_{1}&=&D_{1}^{2}F_{1}\left(D\tilde{\theta}_{1}\right)^{2}D_{2}\tilde{P}_{2}D_{4}\hat{P}_{2}+D_{1}F_{1}D^{2}\tilde{\theta}_{1}D_{2}\tilde{P}_{2}D_{4}\hat{P}_{2}+D_{1}F_{1}D\tilde{\theta}_{1}D_{4}\hat{P}_{2}D_{2}\tilde{P}_{2}+D_{1}D_{2}F_{1}D\tilde{\theta}_{1}D_{4}\hat{P}_{2}\\
&=&f_{1}\left(1,1\right)\left(\frac{1}{1-\tilde{\mu}_{1}}\right)^{2}\hat{\mu}_{2}\tilde{\mu}_{2}
+f_{1}\left(1\right)\tilde{\theta}_{1}^{(2)}\hat{\mu}_{2}\tilde{\mu}_{2}
+f_{1}\left(1\right)\frac{\hat{\mu}_{2}\tilde{\mu}_{2}}{1-\tilde{\mu}_{1}}+f_{1}\left(1,2\right)\frac{\hat{\mu}_{2}}{1-\tilde{\mu}_{1}}
\end{eqnarray*}

\begin{eqnarray*}
D_{3}D_{4}F_{1}&=&D_{1}^{2}F_{1}\left(D\tilde{\theta}_{1}\right)^{2}D_{3}\hat{P}_{1}D_{4}\hat{P}_{2}+D_{1}F_{1}D^{2}\tilde{\theta}_{1}D_{3}\hat{P}_{1}D_{4}\hat{P}_{2}+D_{1}F_{1}D\tilde{\theta}_{1}D_{4}\hat{P}_{2}D_{3}\hat{P}_{1}\\
&=&f_{1}\left(1,1\right)\left(\frac{1}{1-\tilde{\mu}_{1}}\right)^{2}\hat{\mu}_{1}\hat{\mu}_{2}+f_{1}\left(1\right)\tilde{\theta}_{1}^{(2)}\hat{\mu}_{1}\hat{\mu}_{2}+f_{1}\left(1\right)\frac{\hat{\mu}_{1}\hat{\mu}_{2}}{1-\tilde{\mu}_{1}}
\end{eqnarray*}

\begin{eqnarray*}
D_{4}D_{4}F_{1}&=&D_{1}^{2}F_{1}\left(D\tilde{\theta}_{1}\right)^{2}\left(D_{4}\hat{P}_{2}\right)^{2}+D_{1}F_{1}D^{2}\tilde{\theta}_{1}\left(D_{4}\hat{P}_{2}\right)^{2}+D_{1}F_{1}D\tilde{\theta}_{1}D_{4}^{2}\hat{P}_{2}\\
&=&f_{1}\left(1,1\right)\left(\frac{\hat{\mu}_{2}}{1-\tilde{\mu}_{1}}\right)^{2}+f_{1}\left(1\right)\tilde{\theta}_{1}^{(2)}\hat{\mu}_{2}^{2}+f_{1}\left(1\right)\frac{1}{1-\tilde{\mu}_{1}}\hat{P}_{2}^{(2)}
\end{eqnarray*}



Meanwhile for  $F_{2}\left(z_{1},\tilde{\theta}_{2}\left(P_{1}\hat{P}_{1}\hat{P}_{2}\right)\right)$

\begin{eqnarray*}
D_{j}D_{i}F_{2}&=&\indora_{i,j\neq2}D_{2}D_{2}F_{2}\left(D\theta_{2}\right)^{2}D_{i}P_{i}D_{j}P_{j}+\indora_{i,j\neq2}D_{2}F_{2}D^{2}\theta_{2}D_{i}P_{i}D_{j}P_{j}\\
&+&\indora_{i,j\neq2}D_{2}F_{2}D\theta_{2}\left(\indora_{i=j}D_{i}^{2}P_{i}
+\indora_{i\neq j}D_{i}P_{i}D_{j}P_{j}\right)\\
&+&\left(1-\indora_{i=j=3}\right)\indora_{i,j\leq6}D_{2}D_{1}F_{2}D\theta_{2}\left(\indora_{i<j}D_{j}P_{j}+\indora_{i>j}D_{i}P_{i}\right)
+\indora_{i=1}\left(D_{2}D_{1}F_{2}D\theta_{2}D_{i}P_{i}+D_{i}^{2}F_{2}\right)
\end{eqnarray*}

\begin{eqnarray*}
\begin{array}{llll}
D_{2}D_{1}F_{2}=0,&
D_{2}D_{3}F_{3}=0,&
D_{2}D_{4}F_{2}=0,&\\
D_{1}D_{2}F_{2}=0,&
D_{2}D_{2}F_{2}=0,&
D_{3}D_{2}F_{2}=0,&
D_{4}D_{2}F_{2}=0\\
\end{array}
\end{eqnarray*}


\begin{eqnarray*}
D_{1}D_{1}F_{2}&=&
D_{1}^{2}P_{1}D\tilde{\theta}_{2}D_{2}F_{2}+
\left(D_{1}P_{1}\right)^{2}D^{2}\tilde{\theta}_{2}D_{2}F_{2}+
D_{1}P_{1}D\tilde{\theta}_{2}D_{2}D_{1}F_{2}+
\left(D_{1}P_{1}\right)^{2}\left(D\tilde{\theta}_{2}\right)^{2}D_{2}^{2}F_{2}\\
&+&D_{1}P_{1}D\tilde{\theta}_{2}D_{2}D_{1}F_{2}+
D_{1}^{2}F_{2}\\
&=&f_{2}\left(2\right)\frac{\tilde{P}_{1}^{(2)}}{1-\tilde{\mu}_{2}}
+f_{2}\left(2\right)\theta_{2}^{(2)}\tilde{\mu}_{1}^{2}
+f_{2}\left(2,1\right)\frac{\tilde{\mu}_{1}}{1-\tilde{\mu}_{2}}
+\left(\frac{\tilde{\mu}_{1}}{1-\tilde{\mu}_{2}}\right)^{2}f_{2}\left(2,2\right)
+\frac{\tilde{\mu}_{1}}{1-\tilde{\mu}_{2}}f_{2}\left(2,1\right)+f_{2}\left(1,1\right)
\end{eqnarray*}


\begin{eqnarray*}
D_{3}D_{1}F_{2}&=&D_{2}D_{1}F_{2}D\tilde{\theta}_{2}D_{3}\hat{P}_{1}
+D_{2}^{2}F_{2}\left(D\tilde{\theta}_{2}\right)^{2}D_{3}P_{1}D_{1}P_{1}
+D_{2}F_{2}D^{2}\tilde{\theta}_{2}D_{3}\hat{P}_{1}D_{1}P_{1}
+D_{2}F_{2}D\tilde{\theta}_{2}D_{1}P_{1}D_{3}\hat{P}_{1}\\
&=&f_{2}\left(2,1\right)\frac{\hat{\mu}_{1}}{1-\tilde{\mu}_{2}}
+f_{2}\left(2,2\right)\left(\frac{1}{1-\tilde{\mu}_{2}}\right)^{2}\tilde{\mu}_{1}\hat{\mu}_{1}
+f_{2}\left(2\right)\tilde{\theta}_{2}^{(2)}\tilde{\mu}_{1}\hat{\mu}_{1}
+f_{2}\left(2\right)\frac{\tilde{\mu}_{1}\hat{\mu}_{1}}{1-\tilde{\mu}_{2}}
\end{eqnarray*}


\begin{eqnarray*}
D_{4}D_{1}F_{2}&=&D_{2}D_{1}F_{2}D\tilde{\theta}_{2}D_{4}\hat{P}_{2}
+D_{2}^{2}F_{2}\left(D\tilde{\theta}_{2}\right)^{2}D_{4}P_{2}D_{1}P_{1}
+D_{2}F_{2}D^{2}\tilde{\theta}_{2}D_{4}\hat{P}_{2}D_{1}P_{1}
+D_{2}F_{2}D\tilde{\theta}_{2}D_{1}P_{1}D_{4}\hat{P}_{2}\\
&=&f_{2}\left(2,1\right)\frac{\hat{\mu}_{2}}{1-\tilde{\mu}_{2}}
+f_{2}\left(2,2\right)\left(\frac{1}{1-\tilde{\mu}_{2}}\right)^{2}\tilde{\mu}_{1}\hat{\mu}_{2}
+f_{2}\left(2\right)\tilde{\theta}_{2}^{(2)}\tilde{\mu}_{1}\hat{\mu}_{2}
+f_{2}\left(2\right)\frac{\tilde{\mu}_{1}\hat{\mu}_{2}}{1-\tilde{\mu}_{2}}
\end{eqnarray*}


\begin{eqnarray*}
D_{1}D_{3}F_{2}&=&D_{2}^{2}F_{2}\left(D\tilde{\theta}_{2}\right)^{2}D_{1}P_{1}D_{3}\hat{P}_{1}
+D_{2}D_{1}F_{2}D\tilde{\theta}_{2}D_{3}\hat{P}_{1}
+D_{2}F_{2}D^{2}\tilde{\theta}_{2}D_{1}P_{1}D_{3}\hat{P}_{1}
+D_{2}F_{2}D\tilde{\theta}_{2}D_{3}\hat{P}_{1}D_{1}P_{1}\\
&=&f_{2}\left(2,2\right)\left(\frac{1}{1-\tilde{\mu}_{2}}\right)^{2}\tilde{\mu}_{1}\hat{\mu}_{1}
+f_{2}\left(2,1\right)\frac{\hat{\mu}_{1}}{1-\tilde{\mu}_{2}}
+f_{2}\left(2\right)\tilde{\theta}_{2}^{(2)}\tilde{\mu}_{1}\hat{\mu}_{1}
+f_{2}\left(2\right)\frac{\tilde{\mu}_{1}\hat{\mu}_{1}}{1-\tilde{\mu}_{2}}
\end{eqnarray*}


\begin{eqnarray*}
D_{3}D_{3}F_{2}&=&D_{2}^{2}F_{2}\left(D\tilde{\theta}_{2}\right)^{2}\left(D_{3}\hat{P}_{1}\right)^{2}
+D_{2}F_{2}\left(D_{3}\hat{P}_{1}\right)^{2}D^{2}\tilde{\theta}_{2}
+D_{2}F_{2}D\tilde{\theta}_{2}D_{3}^{2}\hat{P}_{1}\\
&=&f_{2}\left(2,2\right)\left(\frac{1}{1-\tilde{\mu}_{2}}\right)^{2}\hat{\mu}_{1}^{2}
+f_{2}\left(2\right)\tilde{\theta}_{2}^{(2)}\hat{\mu}_{1}^{2}
+f_{2}\left(2\right)\frac{\hat{P}_{1}^{(2)}}{1-\tilde{\mu}_{2}}
\end{eqnarray*}


\begin{eqnarray*}
D_{4}D_{3}F_{2}&=&D_{2}^{2}F_{2}\left(D\tilde{\theta}_{2}\right)^{2}D_{4}\hat{P}_{2}D_{3}\hat{P}_{1}
+D_{2}F_{2}D^{2}\tilde{\theta}_{2}D_{4}\hat{P}_{2}D_{3}\hat{P}_{1}
+D_{2}F_{2}D\tilde{\theta}_{2}D_{3}\hat{P}_{1}D_{4}\hat{P}_{2}\\
&=&f_{2}\left(2,2\right)\left(\frac{1}{1-\tilde{\mu}_{2}}\right)^{2}\hat{\mu}_{1}\hat{\mu}_{2}
+f_{2}\left(2\right)\tilde{\theta}_{2}^{(2)}\hat{\mu}_{1}\hat{\mu}_{2}
+f_{2}\left(2\right)\frac{\hat{\mu}_{1}\hat{\mu}_{2}}{1-\tilde{\mu}_{2}}
\end{eqnarray*}


\begin{eqnarray*}
D_{1}D_{4}F_{2}&=&D_{2}^{2}F_{2}\left(D\tilde{\theta}_{2}\right)^{2}D_{1}P_{1}D_{4}\hat{P}_{2}
+D_{2}D_{1}F_{2}D\tilde{\theta}_{2}D_{4}\hat{P}_{2}
+D_{2}F_{2}D^{2}\tilde{\theta}_{2}D_{1}P_{1}D_{4}\hat{P}_{2}
+D_{2}F_{2}D\tilde{\theta}_{2}D_{4}\hat{P}_{2}D_{1}P_{1}\\
&=&f_{2}\left(2,2\right)\left(\frac{1}{1-\tilde{\mu}_{2}}\right)^{2}\tilde{\mu}_{1}\hat{\mu}_{2}
+f_{2}\left(2,1\right)\frac{\hat{\mu}_{2}}{1-\tilde{\mu}_{2}}
+f_{2}\left(2\right)\tilde{\theta}_{2}^{(2)}\tilde{\mu}_{1}\hat{\mu}_{2}
+f_{2}\left(2\right)\frac{\tilde{\mu}_{1}\hat{\mu}_{2}}{1-\tilde{\mu}_{2}}
\end{eqnarray*}


\begin{eqnarray*}
D_{3}D_{4}F_{2}&=&
D_{2}F_{2}D\tilde{\theta}_{2}D_{4}\hat{P}_{2}D_{3}\hat{P}_{1}
+D_{2}F_{2}D^{2}\tilde{\theta}_{2}D_{4}\hat{P}_{2}D_{3}\hat{P}_{1}
+D_{2}^{2}F_{2}\left(D\tilde{\theta}_{2}\right)^{2}D_{4}\hat{P}_{2}D_{3}\hat{P}_{1}\\
&=&f_{2}\left(2,2\right)\left(\frac{1}{1-\tilde{\mu}_{2}}\right)^{2}\hat{\mu}_{1}\hat{\mu}_{2}
+f_{2}\left(2\right)\tilde{\theta}_{2}^{(2)}\hat{\mu}_{1}\hat{\mu}_{2}
+f_{2}\left(2\right)\frac{\hat{\mu}_{1}\hat{\mu}_{2}}{1-\tilde{\mu}_{2}}
\end{eqnarray*}


\begin{eqnarray*}
D_{4}D_{4}F_{2}&=&D_{2}F_{2}D\tilde{\theta}_{2}D_{4}^{2}\hat{P}_{2}
+D_{2}F_{2}D^{2}\tilde{\theta}_{2}\left(D_{4}\hat{P}_{2}\right)^{2}
+D_{2}^{2}F_{2}\left(D\tilde{\theta}_{2}\right)^{2}\left(D_{4}\hat{P}_{2}\right)^{2}\\
&=&f_{2}\left(2,2\right)\left(\frac{\hat{\mu}_{2}}{1-\tilde{\mu}_{2}}\right)^{2}
+f_{2}\left(2\right)\tilde{\theta}_{2}^{(2)}\hat{\mu}_{2}^{2}
+f_{2}\left(2\right)\frac{\hat{P}_{2}^{(2)}}{1-\tilde{\mu}_{2}}
\end{eqnarray*}


%\newpage



%\newpage

For $\hat{F}_{1}\left(\hat{\theta}_{1}\left(P_{1}\tilde{P}_{2}\hat{P}_{2}\right),w_{2}\right)$



\begin{eqnarray*}
D_{j}D_{i}\hat{F}_{1}&=&\indora_{i,j\neq3}D_{3}D_{3}\hat{F}_{1}\left(D\hat{\theta}_{1}\right)^{2}D_{i}P_{i}D_{j}P_{j}
+\indora_{i,j\neq3}D_{3}\hat{F}_{1}D^{2}\hat{\theta}_{1}D_{i}P_{i}D_{j}P_{j}
+\indora_{i,j\neq3}D_{3}\hat{F}_{1}D\hat{\theta}_{1}\left(\indora_{i=j}D_{i}^{2}P_{i}+\indora_{i\neq j}D_{i}P_{i}D_{j}P_{j}\right)\\
&+&\indora_{i+j\geq5}D_{3}D_{4}\hat{F}_{1}D\hat{\theta}_{1}\left(\indora_{i<j}D_{i}P_{i}+\indora_{i>j}D_{j}P_{j}\right)
+\indora_{i=4}\left(D_{3}D_{4}\hat{F}_{1}D\hat{\theta}_{1}D_{i}P_{i}+D_{i}^{2}\hat{F}_{1}\right)
\end{eqnarray*}


\begin{eqnarray*}
\begin{array}{llll}
D_{3}D_{1}\hat{F}_{1}=0,&
D_{3}D_{2}\hat{F}_{1}=0,&
D_{1}D_{3}\hat{F}_{1}=0,&
D_{2}D_{3}\hat{F}_{1}=0\\
D_{3}D_{3}\hat{F}_{1}=0,&
D_{4}D_{3}\hat{F}_{1}=0,&
D_{3}D_{4}\hat{F}_{1}=0,&
\end{array}
\end{eqnarray*}


\begin{eqnarray*}
D_{1}D_{1}\hat{F}_{1}&=&
D\hat{\theta}_{1}D_{1}^{2}P_{1}D_{3}\hat{F}_{1}
+\left(D_{1}P_{1}\right)^{2}D^{2}\hat{\theta}_{1}D_{3}\hat{F}_{1}
+\left(D_{1}P_{1}\right)^{2}\left(D\hat{\theta}_{1}\right)^{2}D_{3}^{2}\hat{F}_{1}\\
&=&\hat{f}_{1}\left(3,3\right)\left(\frac{\tilde{\mu}_{1}}{1-\hat{\mu}_{2}}\right)^{2}
+\hat{f}_{1}\left(3\right)\frac{P_{1}^{(2)}}{1-\hat{\mu}_{1}}
+\hat{f}_{1}\left(3\right)\hat{\theta}_{1}^{(2)}\tilde{\mu}_{1}^{2}
\end{eqnarray*}


\begin{eqnarray*}
D_{2}D_{1}\hat{F}_{1}&=&D_{1}P_{1}D_{2}P_{2}D\hat{\theta}_{1}D_{3}\hat{F}_{1}+
D_{1}P_{1}D_{2}P_{2}D^{2}\hat{\theta}_{1}D_{3}\hat{F}_{1}+
D_{1}P_{1}D_{2}P_{1}\left(D\hat{\theta}_{1}\right)^{2}D_{3}^{2}\hat{F}_{1}\\
&=&\hat{f}_{1}\left(3\right)\frac{\tilde{\mu}_{1}\tilde{\mu}_{2}}{1-\hat{\mu}_{1}}
+\hat{f}_{1}\left(3\right)\tilde{\mu}_{1}\tilde{\mu}_{2}\hat{\theta}_{1}^{(2)}
+\hat{f}_{1}\left(3,3\right)\left(\frac{1}{1-\hat{\mu}_{1}}\right)^{2}\tilde{\mu}_{1}\tilde{\mu}_{2}
\end{eqnarray*}


\begin{eqnarray*}
D_{4}D_{1}\hat{F}_{1}&=&D_{1}P_{1}D_{4}\hat{P}_{2}D\hat{\theta}_{1}D_{3}\hat{F}_{1}
+D_{1}P_{1}D_{4}\hat{P}_{2}D^{2}\hat{\theta}_{1}D_{3}\hat{F}_{1}
+D_{1}P_{1}D\hat{\theta}_{1}D_{3}D_{4}\hat{F}_{1}
+D_{4}\hat{P}_{2}D_{1}P_{1}\left(D\hat{\theta}_{1}\right)^{2}D_{3}D_{3}\hat{F}_{1}\\
&=&\hat{f}_{1}\left(3\right)\frac{\tilde{\mu}_{1}\hat{\mu}_{2}}{1-\hat{\mu}_{1}}
+\hat{f}_{1}\left(3\right)\hat{\theta}_{1}^{(2)}\tilde{\mu}_{1}\hat{\mu}_{2}
+\hat{f}_{1}\left(3,4\right)\frac{\tilde{\mu}_{1}}{1-\hat{\mu}_{1}}
+\hat{f}_{1}\left(3,3\right)\left(\frac{1}{1-\hat{\mu}_{1}}\right)^{2}\tilde{\mu}_{1}\hat{\mu}_{1}
\end{eqnarray*}


\begin{eqnarray*}
D_{1}D_{2}\hat{F}_{1}&=&D_{1}P_{1}D_{2}P_{2}D\hat{\theta}_{1}D_{3}\hat{F}_{1}+
D_{1}P_{1}D_{2}P_{2}D^{2}\hat{\theta}_{1}D_{3}\hat{F}_{1}+
D_{1}P_{1}D_{2}P_{2}\left(D\hat{\theta}_{1}\right)^{2}D_{3}^{2}\hat{F}_{1}\\
&=&\hat{f}_{1}\left(3\right)\frac{\tilde{\mu}_{1}\tilde{\mu}_{2}}{1-\hat{\mu}_{1}}
+\hat{f}_{1}\left(3\right)\hat{\theta}_{1}^{(2)}\tilde{\mu}_{1}\tilde{\mu}_{2}
+\hat{f}_{1}\left(3,3\right)\left(\frac{1}{1-\hat{\mu}_{1}}\right)^{2}\tilde{\mu}_{1}\tilde{\mu}_{2}
\end{eqnarray*}


\begin{eqnarray*}
D_{2}D_{2}\hat{F}_{1}&=&
D\hat{\theta}_{1}D_{2}^{2}P_{2}D_{3}\hat{F}_{1}+
 \left(D_{2}P_{2}\right)^{2}D^{2}\hat{\theta}_{1}D_{3}\hat{F}_{1}+
\left(D_{2}P_{2}\right)^{2}\left(D\hat{\theta}_{1}\right)^{2}D_{3}^{2}\hat{F}_{1}\\
&=&\hat{f}_{1}\left(3\right)\tilde{P}_{2}^{(2)}\frac{1}{1-\hat{\mu}_{1}}
+\hat{f}_{1}\left(3\right)\hat{\theta}_{1}^{(2)}\tilde{\mu}_{2}^{2}
+\hat{f}_{1}\left(3,3\right)\left(\frac{\tilde{\mu}_{2}}{1-\hat{\mu}_{1}}\right)^{2}
\end{eqnarray*}


\begin{eqnarray*}
D_{4}D_{2}\hat{F}_{1}&=&D_{2}P_{2}D_{4}\hat{P}_{2}D\hat{\theta}_{1}D_{3}\hat{F}_{1}
+D_{2}P_{2}D_{4}\hat{P}_{2}D^{2}\hat{\theta}_{1}D_{3}\hat{F}_{1}
+D_{2}P_{2}D\hat{\theta}_{1}D_{4}D_{3}\hat{F}_{1}
+D_{2}P_{2}\left(D\hat{\theta}_{1}\right)^{2}D_{4}\hat{P}_{2}D_{3}^{2}\hat{F}_{1}\\
&=&\hat{f}_{1}\left(3\right)\frac{\tilde{\mu}_{2}\hat{\mu}_{2}}{1-\hat{\mu}_{1}}
+\hat{f}_{1}\left(3\right)\hat{\theta}_{1}^{(2)}\tilde{\mu}_{2}\hat{\mu}_{2}
+\hat{f}_{1}\left(4,3\right)\frac{\tilde{\mu}_{2}}{1-\hat{\mu}_{1}}
+\hat{f}_{1}\left(3,3\right)\left(\frac{1}{1-\hat{\mu}_{1}}\right)^{2}\tilde{\mu}_{2}\hat{\mu}_{2}
\end{eqnarray*}



\begin{eqnarray*}
D_{1}D_{4}\hat{F}_{1}&=&D_{1}P_{1}D_{4}\hat{P}_{2}D\hat{\theta}_{1}D_{3}\hat{F}_{1}
+D_{1}P_{1}D_{4}\hat{P}_{2}D^{2}\hat{\theta}_{1}D_{3}\hat{F}_{1}
+D_{1}P_{1}D\hat{\theta}_{1}D_{3}D_{4}\hat{F}_{1}
+ D_{1}P_{1}D_{4}\hat{P}_{2}\left(D\hat{\theta}_{1}\right)^{2}D_{3}D_{3}
\hat{F}_{1}\\
&=&\hat{f}_{1}\left(3\right)\frac{\tilde{\mu}_{1}\hat{\mu}_{2}}{1-\hat{\mu}_{1}}
+\hat{f}_{1}\left(3\right)\hat{\theta}_{1}^{(2)}\tilde{\mu}_{1}\hat{\mu}_{2}
+\hat{f}_{1}\left(3,4\right)\frac{\tilde{\mu}_{1}}{1-\hat{\mu}_{1}}
+\hat{f}_{1}\left(3,3\right)\left(\frac{1}{1-\hat{\mu}_{1}}\right)^{2}\tilde{\mu}_{1}\hat{\mu}_{2}
\end{eqnarray*}


\begin{eqnarray*}
D_{2}D_{4}\hat{F}_{1}&=&D_{2}P_{2}D_{4}\hat{P}_{2}D\hat{\theta}_{1}D_{3}
\hat{F}_{1}
+D_{2}P_{2}D_{4}\hat{P}_{2}D^{2}\hat{\theta}_{1}D_{3}\hat{F}_{1}
+D_{2}P_{2}D\hat{\theta}_{1}D_{3}D_{4}\hat{F}_{1}+
D_{2}P_{2}D_{4}\hat{P}_{2}\left(D\hat{\theta}_{1}\right)^{2}D_{3}^{2}\hat{F}_{1}\\
&=&\hat{f}_{1}\left(3\right)\frac{\tilde{\mu}_{2}\hat{\mu}_{2}}{1-\hat{\mu}_{1}}
+\hat{f}_{1}\left(3\right)\hat{\theta}_{1}^{(2)}\tilde{\mu}_{2}\hat{\mu}_{2}
+\hat{f}_{1}\left(3,4\right)\frac{\tilde{\mu}_{2}}{1-\hat{\mu}_{1}}
+\hat{f}_{1}\left(3,3\right)\left(\frac{1}{1-\hat{\mu}_{1}}\right)^{2}\tilde{\mu}_{2}\hat{\mu}_{2}
\end{eqnarray*}



\begin{eqnarray*}
D_{4}D_{4}\hat{F}_{1}&=&D_{4}D_{4}\hat{F}_{1}+D\hat{\theta}_{1}D_{4}^{2}\hat{P}_{2}D_{3}\hat{F}_{1}
+\left(D_{4}\hat{P}_{2}\right)^{2}D^{2}\hat{\theta}_{1}D_{3}\hat{F}_{1}+
D_{4}\hat{P}_{2}D\hat{\theta}_{1}D_{3}D_{4}\hat{F}_{1}\\
&+&\left(D_{4}\hat{P}_{2}\right)^{2}\left(D\hat{\theta}_{1}\right)^{2}D_{3}^{2}\hat{F}_{1}
+D_{3}D_{4}\hat{F}_{1}D\hat{\theta}_{1}D_{4}\hat{P}_{2}\\
&=&\hat{f}_{1}\left(4,4\right)
+\hat{f}_{1}\left(3\right)\frac{\hat{P}_{2}^{(2)}}{1-\hat{\mu}_{1}}
+\hat{f}_{1}\left(3\right)\hat{\theta}_{1}^{(2)}\hat{\mu}_{2}^{2}
+\hat{f}_{1}\left(3,4\right)\frac{\hat{\mu}_{2}}{1-\hat{\mu}_{1}}
+\hat{f}_{1}\left(3,3\right)\left(\frac{\hat{\mu}_{2}}{1-\hat{\mu}_{1}}\right)^{2}
+\hat{f}_{1}\left(3,4\right)\frac{\hat{\mu}_{2}}{1-\hat{\mu}_{1}}
\end{eqnarray*}




Finally for $\hat{F}_{2}\left(w_{1},\hat{\theta}_{2}\left(P_{1}\tilde{P}_{2}\hat{P}_{1}\right)\right)$

\begin{eqnarray*}
D_{j}D_{i}\hat{F}_{2}&=&\indora_{i,j\neq4}D_{4}D_{4}\hat{F}_{2}\left(D\hat{\theta}_{2}\right)^{2}D_{i}P_{i}D_{j}P_{j}
+\indora_{i,j\neq4}D_{4}\hat{F}_{2}D^{2}\hat{\theta}_{2}D_{i}P_{i}D_{j}P_{j}
+\indora_{i,j\neq4}D_{4}\hat{F}_{2}D\hat{\theta}_{2}\left(\indora_{i=j}D_{i}^{2}P_{i}+\indora_{i\neq j}D_{i}P_{i}D_{j}P_{j}\right)\\
&+&\left(1-\indora_{i=j=2}\right)\indora_{i+j\geq4}D_{4}D_{3}\hat{F}_{2}D\hat{\theta}_{2}\left(\indora_{i\leq j}D_{i}P_{i}+\indora_{i>j}D_{j}P_{j}\right)
+\indora_{i=3}\left(D_{4}D_{3}\hat{F}_{2}D\hat{\theta}_{2}D_{i}P_{i}+D_{i}^{2}\hat{F}_{2}\right)
\end{eqnarray*}



\begin{eqnarray*}
\begin{array}{llll}
D_{4}D_{1}\hat{F}_{2}=0,&
D_{4}D_{2}\hat{F}_{2}=0,&
D_{4}D_{3}\hat{F}_{2}=0,&
D_{1}D_{4}\hat{F}_{2}=0\\
D_{2}D_{4}\hat{F}_{2}=0,&
D_{3}D_{4}\hat{F}_{2}=0,&
D_{4}D_{4}\hat{F}_{2}=0,&
\end{array}
\end{eqnarray*}


\begin{eqnarray*}
D_{1}D_{1}\hat{F}_{2}&=&D\hat{\theta}_{2}D_{1}^{2}P_{1}D_{4}\hat{F}_{2}
+\left(D_{1}P_{1}\right)^{2}D^{2}\hat{\theta}_{2}D_{4}\hat{F}_{2}+
\left(D_{1}P_{1}\right)^{2}\left(D\hat{\theta}_{2}\right)^{2}D_{4}^{2}\hat{F}_{2}\\
&=&\hat{f}_{2}\left(4\right)\frac{\tilde{P}_{1}^{(2)}}{1-\tilde{\mu}_{2}}
+\hat{f}_{2}\left(4\right)\hat{\theta}_{2}^{(2)}\tilde{\mu}_{1}^{2}
+\hat{f}_{2}\left(4,4\right)\left(\frac{\tilde{\mu}_{1}}{1-\hat{\mu}_{2}}\right)^{2}
\end{eqnarray*}



\begin{eqnarray*}
D_{2}D_{1}\hat{F}_{2}&=&D_{1}P_{1}D_{2}P_{2}D\hat{\theta}_{2}D_{4}\hat{F}_{2}+
D_{1}P_{1}D_{2}P_{2}D^{2}\hat{\theta}_{2}D_{4}\hat{F}_{2}+
D_{1}P_{1}D_{2}P_{2}\left(D\hat{\theta}_{2}\right)^{2}D_{4}^{2}\hat{F}_{2}\\
&=&\hat{f}_{2}\left(4\right)\frac{\tilde{\mu}_{1}\tilde{\mu}_{2}}{1-\tilde{\mu}_{2}}
+\hat{f}_{2}\left(4\right)\hat{\theta}_{2}^{(2)}\tilde{\mu}_{1}\tilde{\mu}_{2}
+\hat{f}_{2}\left(4,4\right)\left(\frac{1}{1-\hat{\mu}_{2}}\right)^{2}\tilde{\mu}_{1}\tilde{\mu}_{2}
\end{eqnarray*}



\begin{eqnarray*}
D_{3}D_{1}\hat{F}_{2}&=&
D_{1}P_{1}D_{3}\hat{P}_{1}D\hat{\theta}_{2}D_{4}\hat{F}_{2}
+D_{1}P_{1}D_{3}\hat{P}_{1}D^{2}\hat{\theta}_{2}D_{4}\hat{F}_{2}
+D_{1}P_{1}D_{3}\hat{P}_{1}\left(D\hat{\theta}_{2}\right)^{2}D_{4}^{2}\hat{F}_{2}
+D_{1}P_{1}D\hat{\theta}_{2}D_{4}D_{3}\hat{F}_{2}\\
&=&\hat{f}_{2}\left(4\right)\frac{\tilde{\mu}_{1}\hat{\mu}_{1}}{1-\hat{\mu}_{2}}
+\hat{f}_{2}\left(4\right)\hat{\theta}_{2}^{(2)}\tilde{\mu}_{1}\hat{\mu}_{1}
+\hat{f}_{2}\left(4,4\right)\left(\frac{1}{1-\hat{\mu}_{2}}\right)^{2}\tilde{\mu}_{1}\hat{\mu}_{1}
+\hat{f}_{2}\left(4,3\right)\frac{\tilde{\mu}_{1}}{1-\hat{\mu}_{2}}
\end{eqnarray*}



\begin{eqnarray*}
D_{1}D_{2}\hat{F}_{2}&=&
D_{1}P_{1}D_{2}P_{2}D\hat{\theta}_{2}D_{4}\hat{F}_{2}+
D_{1}P_{1}D_{2}P_{2}D^{2}\hat{\theta}_{2}D_{4}\hat{F}_{2}+
D_{1}P_{1}D_{2}P_{2}\left(D\hat{\theta}_{2}\right)^{2}D_{4}D_{4}\hat{F}_{2}\\
&=&\hat{f}_{2}\left(4\right)\frac{\tilde{\mu}_{1}\tilde{\mu}_{2}}{1-\tilde{\mu}_{2}}
+\hat{f}_{2}\left(4\right)\hat{\theta}_{2}^{(2)}\tilde{\mu}_{1}\tilde{\mu}_{2}
+\hat{f}_{2}\left(4,4\right)\left(\frac{1}{1-\hat{\mu}_{2}}\right)^{2}\tilde{\mu}_{1}\tilde{\mu}_{2}
\end{eqnarray*}



\begin{eqnarray*}
D_{2}D_{2}\hat{F}_{2}&=&
D\hat{\theta}_{2}D_{2}^{2}P_{2}D_{4}\hat{F}_{2}+
\left(D_{2}P_{2}\right)^{2}D^{2}\hat{\theta}_{2}D_{4}\hat{F}_{2}+
\left(D_{2}P_{2}\right)^{2}\left(D\hat{\theta}_{2}\right)^{2}D_{4}^{2}\hat{F}_{2}\\
&=&\hat{f}_{2}\left(4\right)\frac{\tilde{P}_{2}^{(2)}}{1-\hat{\mu}_{2}}
+\hat{f}_{2}\left(4\right)\hat{\theta}_{2}^{(2)}\tilde{\mu}_{2}^{2}
+\hat{f}_{2}\left(4,4\right)\left(\frac{\tilde{\mu}_{2}}{1-\hat{\mu}_{2}}\right)^{2}
\end{eqnarray*}



\begin{eqnarray*}
D_{3}D_{2}\hat{F}_{2}&=&
D_{2}P_{2}D_{3}\hat{P}_{1}D\hat{\theta} _{2}D_{4}\hat{F}_{2}
+D_{2}P_{2}D_{3}\hat{P}_{1}D^{2}\hat{\theta}_{2}D_{4}\hat{F}_{2}
+D_{2}P_{2}D_{3}\hat{P}_{1}\left(D\hat{\theta}_{2}\right)^{2}D_{4}^{2}\hat{F}_{2}
+D_{2}P_{2}D\hat{\theta}_{2}D_{3}D_{4}\hat{F}_{2}\\
&=&\hat{f}_{2}\left(4\right)\frac{\tilde{\mu}_{2}\hat{\mu}_{1}}{1-\hat{\mu}_{2}}
+\hat{f}_{2}\left(4\right)\hat{\theta}_{2}^{(2)}\tilde{\mu}_{2}\hat{\mu}_{1}
+\hat{f}_{2}\left(4,4\right)\left(\frac{1}{1-\hat{\mu}_{2}}\right)^{2}\tilde{\mu}_{2}\hat{\mu}_{1}
+\hat{f}_{2}\left(3,4\right)\frac{\tilde{\mu}_{2}}{1-\hat{\mu}_{2}}
\end{eqnarray*}



\begin{eqnarray*}
D_{1}D_{3}\hat{F}_{2}&=&
D_{1}P_{1}D_{3}\hat{P}_{1}D\hat{\theta}_{2}D_{4}\hat{F}_{2}
+D_{1}P_{1}D_{3}\hat{P}_{1}D^{2}\hat{\theta}_{2}D_{4}\hat{F}_{2}
+D_{1}P_{1}D_{3}\hat{P}_{1}\left(D\hat{\theta}_{2}\right)^{2}D_{4}D_{4}\hat{F}_{2}
+D_{1}P_{1}D\hat{\theta}_{2}D_{4}D_{3}\hat{F}_{2}\\
&=&\hat{f}_{2}\left(4\right)\frac{\tilde{\mu}_{1}\hat{\mu}_{1}}{1-\hat{\mu}_{2}}
+\hat{f}_{2}\left(4\right)\hat{\theta}_{2}^{(2)}\tilde{\mu}_{1}\hat{\mu}_{1}
+\hat{f}_{2}\left(4,4\right)\left(\frac{1}{1-\hat{\mu}_{2}}\right)^{2}\tilde{\mu}_{1}\hat{\mu}_{1}
+\hat{f}_{2}\left(4,3\right)\frac{\tilde{\mu}_{1}}{1-\hat{\mu}_{2}}
\end{eqnarray*}



\begin{eqnarray*}
D_{2}D_{3}\hat{F}_{2}&=&
D_{2}P_{2}D_{3}\hat{P}_{1}D\hat{\theta}_{2}D_{4}\hat{F}_{2}
+D_{2}P_{2}D_{3}\hat{P}_{1}D^{2}\hat{\theta}_{2}D_{4}\hat{F}_{2}
+D_{2}P_{2}D_{3}\hat{P}_{1}\left(D\hat{\theta}_{2}\right)^{2}D_{4}^{2}\hat{F}_{2}
+D_{2}P_{2}D\hat{\theta}_{2}D_{4}D_{3}\hat{F}_{2}\\
&=&\hat{f}_{2}\left(4\right)\frac{\tilde{\mu}_{2}\hat{\mu}_{1}}{1-\hat{\mu}_{2}}
+\hat{f}_{2}\left(4\right)\hat{\theta}_{2}^{(2)}\tilde{\mu}_{2}\hat{\mu}_{1}
+\hat{f}_{2}\left(4,4\right)\left(\frac{1}{1-\hat{\mu}_{2}}\right)^{2}\tilde{\mu}_{2}\hat{\mu}_{1}
+\hat{f}_{2}\left(4,3\right)\frac{\tilde{\mu}_{2}}{1-\hat{\mu}_{2}}
\end{eqnarray*}



\begin{eqnarray*}
D_{3}D_{3}\hat{F}_{2}&=&
D_{3}^{2}\hat{P}_{1}D\hat{\theta}_{2}D_{4}\hat{F}_{2}
+\left(D_{3}\hat{P}_{1}\right)^{2}D^{2}\hat{\theta}_{2}D_{4}\hat{F}_{2}
+D_{3}\hat{P}_{1}D\hat{\theta}_{2}D_{4}D_{3}\hat{F}_{2}
+ \left(D_{3}\hat{P}_{1}\right)^{2}\left(D\hat{\theta}_{2}\right)^{2}
D_{4}^{2}\hat{F}_{2}+D_{3}^{2}\hat{F}_{2}
+D_{4}D_{3}\hat{f}_{2}D\hat{\theta}_{2}D_{3}\hat{P}_{1}\\
&=&\hat{f}_{2}\left(4\right)\frac{\hat{P}_{1}^{(2)}}{1-\hat{\mu}_{2}}
+\hat{f}_{2}\left(4\right)\hat{\theta}_{2}^{(2)}\hat{\mu}_{1}^{2}
+\hat{f}_{2}\left(4,3\right)\frac{\hat{\mu}_{1}}{1-\hat{\mu}_{2}}
+\hat{f}_{2}\left(4,4\right)\left(\frac{\hat{\mu}_{1}}{1-\hat{\mu}_{2}}\right)^{2}
+\hat{f}_{2}\left(3,3\right)
+\hat{f}_{2}\left(4,3\right)\frac{\tilde{\mu}_{1}}{1-\hat{\mu}_{2}}
\end{eqnarray*}




%_____________________________________________________________________________________
\newpage

%__________________________________________________________________
\section{Generalizaciones}
%__________________________________________________________________
\subsection{RSVC con dos conexiones}
%__________________________________________________________________

%\begin{figure}[H]
%\centering
%%%\includegraphics[width=9cm]{Grafica3.jpg}
%%\end{figure}\label{RSVC3}


Sus ecuaciones recursivas son de la forma


\begin{eqnarray*}
F_{1}\left(z_{1},z_{2},w_{1},w_{2}\right)&=&R_{2}\left(\prod_{i=1}^{2}\tilde{P}_{i}\left(z_{i}\right)\prod_{i=1}^{2}
\hat{P}_{i}\left(w_{i}\right)\right)F_{2}\left(z_{1},\tilde{\theta}_{2}\left(\tilde{P}_{1}\left(z_{1}\right)\hat{P}_{1}\left(w_{1}\right)\hat{P}_{2}\left(w_{2}\right)\right)\right)
\hat{F}_{2}\left(w_{1},w_{2};\tau_{2}\right),
\end{eqnarray*}

\begin{eqnarray*}
F_{2}\left(z_{1},z_{2},w_{1},w_{2}\right)&=&R_{1}\left(\prod_{i=1}^{2}\tilde{P}_{i}\left(z_{i}\right)\prod_{i=1}^{2}
\hat{P}_{i}\left(w_{i}\right)\right)F_{1}\left(\tilde{\theta}_{1}\left(\tilde{P}_{2}\left(z_{2}\right)\hat{P}_{1}\left(w_{1}\right)\hat{P}_{2}\left(w_{2}\right)\right),z_{2}\right)\hat{F}_{1}\left(w_{1},w_{2};\tau_{1}\right),
\end{eqnarray*}


\begin{eqnarray*}
\hat{F}_{1}\left(z_{1},z_{2},w_{1},w_{2}\right)&=&\hat{R}_{2}\left(\prod_{i=1}^{2}\tilde{P}_{i}\left(z_{i}\right)\prod_{i=1}^{2}
\hat{P}_{i}\left(w_{i}\right)\right)F_{2}\left(z_{1},z_{2};\zeta_{2}\right)\hat{F}_{2}\left(w_{1},\hat{\theta}_{2}\left(\tilde{P}_{1}\left(z_{1}\right)\tilde{P}_{2}\left(z_{2}\right)\hat{P}_{1}\left(w_{1}
\right)\right)\right),
\end{eqnarray*}


\begin{eqnarray*}
\hat{F}_{2}\left(z_{1},z_{2},w_{1},w_{2}\right)&=&\hat{R}_{1}\left(\prod_{i=1}^{2}\tilde{P}_{i}\left(z_{i}\right)\prod_{i=1}^{2}
\hat{P}_{i}\left(w_{i}\right)\right)F_{1}\left(z_{1},z_{2};\zeta_{1}\right)\hat{F}_{1}\left(\hat{\theta}_{1}\left(\tilde{P}_{1}\left(z_{1}\right)\tilde{P}_{2}\left(z_{2}\right)\hat{P}_{2}\left(w_{2}\right)\right),w_{2}\right),
\end{eqnarray*}

%_____________________________________________________
\subsection{First Moments of the Queue Lengths}
%_____________________________________________________


The server's switchover times are given by the general equation

\begin{eqnarray}\label{Ec.Ri}
R_{i}\left(\mathbf{z,w}\right)=R_{i}\left(\tilde{P}_{1}\left(z_{1}\right)\tilde{P}_{2}\left(z_{2}\right)\hat{P}_{1}\left(w_{1}\right)\hat{P}_{2}\left(w_{2}\right)\right)
\end{eqnarray}

with
\begin{eqnarray}\label{Ec.Derivada.Ri}
D_{i}R_{i}&=&DR_{i}D_{i}P_{i}
\end{eqnarray}
the following notation is considered

\begin{eqnarray*}
\begin{array}{llll}
D_{1}P_{1}\equiv D_{1}\tilde{P}_{1}, & D_{2}P_{2}\equiv D_{2}\tilde{P}_{2}, & D_{3}P_{3}\equiv D_{3}\hat{P}_{1}, &D_{4}P_{4}\equiv D_{4}\hat{P}_{2},
\end{array}
\end{eqnarray*}

also we need to remind $F_{1,2}\left(z_{1};\zeta_{2}\right)F_{2,2}\left(z_{2};\zeta_{2}\right)=F_{2}\left(z_{1},z_{2};\zeta_{2}\right)$, therefore

\begin{eqnarray*}
D_{1}F_{2}\left(z_{1},z_{2};\zeta_{2}\right)&=&D_{1}\left[F_{1,2}\left(z_{1};\zeta_{2}\right)F_{2,2}\left(z_{2};\zeta_{2}\right)\right]
=F_{2,2}\left(z_{2};\zeta_{2}\right)D_{1}F_{1,2}\left(z_{1};\zeta_{2}\right)=F_{1,2}^{(1)}\left(1\right)
\end{eqnarray*}

i.e., $D_{1}F_{2}=F_{1,2}^{(1)}(1)$; $D_{2}F_{2}=F_{2,2}^{(1)}\left(1\right)$, whereas that $D_{3}F_{2}=D_{4}F_{2}=0$, then

\begin{eqnarray}
\begin{array}{ccc}
D_{i}F_{j}=\indora_{i\leq2}F_{i,j}^{(1)}\left(1\right),& \textrm{ and } &D_{i}\hat{F}_{j}=\indora_{i\geq2}F_{i,j}^{(1)}\left(1\right).
\end{array}
\end{eqnarray}

Now, we obtain the first moments equations for the queue lengths as before for the times the server arrives to the queue to start attending



Remember that


\begin{eqnarray*}
F_{2}\left(z_{1},z_{2},w_{1},w_{2}\right)&=&R_{1}\left(\prod_{i=1}^{2}\tilde{P}_{i}\left(z_{i}\right)\prod_{i=1}^{2}
\hat{P}_{i}\left(w_{i}\right)\right)F_{1}\left(\tilde{\theta}_{1}\left(\tilde{P}_{2}\left(z_{2}\right)\hat{P}_{1}\left(w_{1}\right)\hat{P}_{2}\left(w_{2}\right)\right),z_{2}\right)\hat{F}_{1}\left(w_{1},w_{2};\tau_{1}\right),
\end{eqnarray*}

where


\begin{eqnarray*}
F_{1}\left(\tilde{\theta}_{1}\left(\tilde{P}_{2}\hat{P}_{1}\hat{P}_{2}\right),z_{2}\right)
\end{eqnarray*}

so

\begin{eqnarray}
D_{i}F_{1}&=&\indora_{i\neq1}D_{1}F_{1}D\tilde{\theta}_{1}D_{i}P_{i}+\indora_{i=2}D_{i}F_{1},
\end{eqnarray}

then


\begin{eqnarray*}
\begin{array}{ll}
D_{1}F_{1}=0,&
D_{2}F_{1}=D_{1}F_{1}D\tilde{\theta}_{1}D_{2}P_{2}+D_{2}F_{1}
=f_{1}\left(1\right)\frac{1}{1-\tilde{\mu}_{1}}\tilde{\mu}_{2}+f_{1}\left(2\right),\\
D_{3}F_{1}=D_{1}F_{1}D\tilde{\theta}_{1}D_{3}P_{3}
=f_{1}\left(1\right)\frac{1}{1-\tilde{\mu}_{1}}\hat{\mu}_{1},&
D_{4}F_{1}=D_{1}F_{1}D\tilde{\theta}_{1}D_{4}P_{4}
=f_{1}\left(1\right)\frac{1}{1-\tilde{\mu}_{1}}\hat{\mu}_{2}

\end{array}
\end{eqnarray*}


\begin{eqnarray}
D_{i}F_{2}&=&\indora_{i\neq2}D_{2}F_{2}D\tilde{\theta}_{2}D_{i}P_{i}
+\indora_{i=1}D_{i}F_{2}
\end{eqnarray}

\begin{eqnarray*}
\begin{array}{ll}
D_{1}F_{2}=D_{2}F_{2}D\tilde{\theta}_{2}D_{1}P_{1}
+D_{1}F_{2}=f_{2}\left(2\right)\frac{1}{1-\tilde{\mu}_{2}}\tilde{\mu}_{1},&
D_{2}F_{2}=0\\
D_{3}F_{2}=D_{2}F_{2}D\tilde{\theta}_{2}D_{3}P_{3}
=f_{2}\left(2\right)\frac{1}{1-\tilde{\mu}_{2}}\hat{\mu}_{1},&
D_{4}F_{2}=D_{2}F_{2}D\tilde{\theta}_{2}D_{4}P_{4}
=f_{2}\left(2\right)\frac{1}{1-\tilde{\mu}_{2}}\hat{\mu}_{2}
\end{array}
\end{eqnarray*}



\begin{eqnarray}
D_{i}\hat{F}_{1}&=&\indora_{i\neq3}D_{3}\hat{F}_{1}D\hat{\theta}_{1}D_{i}P_{i}+\indora_{i=4}D_{i}\hat{F}_{1},
\end{eqnarray}

\begin{eqnarray*}
\begin{array}{ll}
D_{1}\hat{F}_{1}=D_{3}\hat{F}_{1}D\hat{\theta}_{1}D_{1}P_{1}=\hat{f}_{1}\left(3\right)\frac{1}{1-\hat{\mu}_{1}}\tilde{\mu}_{1},&
D_{2}\hat{F}_{1}=D_{3}\hat{F}_{1}D\hat{\theta}_{1}D_{2}P_{2}
=\hat{f}_{1}\left(3\right)\frac{1}{1-\hat{\mu}_{1}}\tilde{\mu}_{2}\\
D_{3}\hat{F}_{1}=0,&
D_{4}\hat{F}_{1}=D_{3}\hat{F}_{1}D\hat{\theta}_{1}D_{4}P_{4}
+D_{4}\hat{F}_{1}
=\hat{f}_{1}\left(3\right)\frac{1}{1-\hat{\mu}_{1}}\hat{\mu}_{2}+\hat{f}_{1}\left(2\right),

\end{array}
\end{eqnarray*}


\begin{eqnarray}
D_{i}\hat{F}_{2}&=&\indora_{i\neq4}D_{4}\hat{F}_{2}D\hat{\theta}_{2}D_{i}P_{i}+\indora_{i=3}D_{i}\hat{F}_{2}.
\end{eqnarray}

\begin{eqnarray*}
\begin{array}{ll}
D_{1}\hat{F}_{2}=D_{4}\hat{F}_{2}D\hat{\theta}_{2}D_{1}P_{1}
=\hat{f}_{2}\left(4\right)\frac{1}{1-\hat{\mu}_{2}}\tilde{\mu}_{1},&
D_{2}\hat{F}_{2}=D_{4}\hat{F}_{2}D\hat{\theta}_{2}D_{2}P_{2}
=\hat{f}_{2}\left(4\right)\frac{1}{1-\hat{\mu}_{2}}\tilde{\mu}_{2},\\
D_{3}\hat{F}_{2}=D_{4}\hat{F}_{2}D\hat{\theta}_{2}D_{3}P_{3}+D_{3}\hat{F}_{2}
=\hat{f}_{2}\left(4\right)\frac{1}{1-\hat{\mu}_{2}}\hat{\mu}_{1}+\hat{f}_{2}\left(4\right)\\
D_{4}\hat{F}_{2}=0

\end{array}
\end{eqnarray*}
Then, now we can obtain the linear system of equations in order to obtain the first moments of the lengths of the queues:



For $\mathbf{F}_{1}=R_{2}F_{2}\hat{F}_{2}$ we get the general equations

\begin{eqnarray}
D_{i}\mathbf{F}_{1}=D_{i}\left(R_{2}+F_{2}+\indora_{i\geq3}\hat{F}_{2}\right)
\end{eqnarray}

So

\begin{eqnarray*}
D_{1}\mathbf{F}_{1}&=&D_{1}R_{2}+D_{1}F_{2}
=r_{1}\tilde{\mu}_{1}+f_{2}\left(2\right)\frac{1}{1-\tilde{\mu}_{2}}\tilde{\mu}_{1}\\
D_{2}\mathbf{F}_{1}&=&D_{2}\left(R_{2}+F_{2}\right)
=r_{2}\tilde{\mu}_{1}\\
D_{3}\mathbf{F}_{1}&=&D_{3}\left(R_{2}+F_{2}+\hat{F}_{2}\right)
=r_{1}\hat{\mu}_{1}+f_{2}\left(2\right)\frac{1}{1-\tilde{\mu}_{2}}\hat{\mu}_{1}+\hat{F}_{1,2}^{(1)}\left(1\right)\\
D_{4}\mathbf{F}_{1}&=&D_{4}\left(R_{2}+F_{2}+\hat{F}_{2}\right)
=r_{2}\hat{\mu}_{2}+f_{2}\left(2\right)\frac{1}{1-\tilde{\mu}_{2}}\hat{\mu}_{2}
+\hat{F}_{2,2}^{(1)}\left(1\right)
\end{eqnarray*}

it means

\begin{eqnarray*}
\begin{array}{ll}
D_{1}\mathbf{F}_{1}=r_{2}\hat{\mu}_{1}+f_{2}\left(2\right)\left(\frac{1}{1-\tilde{\mu}_{2}}\right)\tilde{\mu}_{1}+f_{2}\left(1\right),&
D_{2}\mathbf{F}_{1}=r_{2}\tilde{\mu}_{2},\\
D_{3}\mathbf{F}_{1}=r_{2}\hat{\mu}_{1}+f_{2}\left(2\right)\left(\frac{1}{1-\tilde{\mu}_{2}}\right)\hat{\mu}_{1}+\hat{F}_{1,2}^{(1)}\left(1\right),&
D_{4}\mathbf{F}_{1}=r_{2}\hat{\mu}_{2}+f_{2}\left(2\right)\left(\frac{1}{1-\tilde{\mu}_{2}}\right)\hat{\mu}_{2}+\hat{F}_{2,2}^{(1)}\left(1\right),\end{array}
\end{eqnarray*}


\begin{eqnarray}
\begin{array}{ll}
\mathbf{F}_{2}=R_{1}F_{1}\hat{F}_{1}, & D_{i}\mathbf{F}_{2}=D_{i}\left(R_{1}+F_{1}+\indora_{i\geq3}\hat{F}_{1}\right)\\
\end{array}
\end{eqnarray}



equivalently


\begin{eqnarray*}
\begin{array}{ll}
D_{1}\mathbf{F}_{2}=r_{1}\tilde{\mu}_{1},&
D_{2}\mathbf{F}_{2}=r_{1}\tilde{\mu}_{2}+f_{1}\left(1\right)\left(\frac{1}{1-\tilde{\mu}_{1}}\right)\tilde{\mu}_{2}+f_{1}\left(2\right),\\
D_{3}\mathbf{F}_{2}=r_{1}\hat{\mu}_{1}+f_{1}\left(1\right)\left(\frac{1}{1-\tilde{\mu}_{1}}\right)\hat{\mu}_{1}+\hat{F}_{1,1}^{(1)}\left(1\right),&
D_{4}\mathbf{F}_{2}=r_{1}\hat{\mu}_{2}+f_{1}\left(1\right)\left(\frac{1}{1-\tilde{\mu}_{1}}\right)\hat{\mu}_{2}+\hat{F}_{2,1}^{(1)}\left(1\right),\\
\end{array}
\end{eqnarray*}



\begin{eqnarray}
\begin{array}{ll}
\hat{\mathbf{F}}_{1}=\hat{R}_{2}\hat{F}_{2}F_{2}, & D_{i}\hat{\mathbf{F}}_{1}=D_{i}\left(\hat{R}_{2}+\hat{F}_{2}+\indora_{i\leq2}F_{2}\right)\\
\end{array}
\end{eqnarray}


equivalently


\begin{eqnarray*}
\begin{array}{ll}
D_{1}\hat{\mathbf{F}}_{1}=\hat{r}_{2}\tilde{\mu}_{1}+\hat{f}_{2}\left(2\right)\left(\frac{1}{1-\hat{\mu}_{2}}\right)\tilde{\mu}_{1}+F_{1,2}^{(1)}\left(1\right),&
D_{2}\hat{\mathbf{F}}_{1}=\hat{r}_{2}\tilde{\mu}_{2}+\hat{f}_{2}\left(2\right)\left(\frac{1}{1-\hat{\mu}_{2}}\right)\tilde{\mu}_{2}+F_{2,2}^{(1)}\left(1\right),\\
D_{3}\hat{\mathbf{F}}_{1}=\hat{r}_{2}\hat{\mu}_{1}+\hat{f}_{2}\left(2\right)\left(\frac{1}{1-\hat{\mu}_{2}}\right)\hat{\mu}_{1}+\hat{f}_{2}\left(1\right),&
D_{4}\hat{\mathbf{F}}_{1}=\hat{r}_{2}\hat{\mu}_{2}
\end{array}
\end{eqnarray*}



\begin{eqnarray}
\begin{array}{ll}
\hat{\mathbf{F}}_{2}=\hat{R}_{1}\hat{F}_{1}F_{1}, & D_{i}\hat{\mathbf{F}}_{2}=D_{i}\left(\hat{R}_{1}+\hat{F}_{1}+\indora_{i\leq2}F_{1}\right)
\end{array}
\end{eqnarray}



equivalently


\begin{eqnarray*}
\begin{array}{ll}
D_{1}\hat{\mathbf{F}}_{2}=\hat{r}_{1}\tilde{\mu}_{1}+\hat{f}_{1}\left(1\right)\left(\frac{1}{1-\hat{\mu}_{1}}\right)\tilde{\mu}_{1}+F_{1,1}^{(1)}\left(1\right),&
D_{2}\hat{\mathbf{F}}_{2}=\hat{r}_{1}\mu_{2}+\hat{f}_{1}\left(1\right)\left(\frac{1}{1-\hat{\mu}_{1}}\right)\tilde{\mu}_{2}+F_{2,1}^{(1)}\left(1\right),\\
D_{3}\hat{\mathbf{F}}_{2}=\hat{r}_{1}\hat{\mu}_{1},&
D_{4}\hat{\mathbf{F}}_{2}=\hat{r}_{1}\hat{\mu}_{2}+\hat{f}_{1}\left(1\right)\left(\frac{1}{1-\hat{\mu}_{1}}\right)\hat{\mu}_{2}+\hat{f}_{1}\left(2\right),\\
\end{array}
\end{eqnarray*}





Then we have that if $\mu=\tilde{\mu}_{1}+\tilde{\mu}_{2}$, $\hat{\mu}=\hat{\mu}_{1}+\hat{\mu}_{2}$, $r=r_{1}+r_{2}$ and $\hat{r}=\hat{r}_{1}+\hat{r}_{2}$  the system's solution is given by

\begin{eqnarray*}
\begin{array}{llll}
f_{2}\left(1\right)=r_{1}\tilde{\mu}_{1},&
f_{1}\left(2\right)=r_{2}\tilde{\mu}_{2},&
\hat{f}_{1}\left(4\right)=\hat{r}_{2}\hat{\mu}_{2},&
\hat{f}_{2}\left(3\right)=\hat{r}_{1}\hat{\mu}_{1}
\end{array}
\end{eqnarray*}



it's easy to verify that

\begin{eqnarray}\label{Sist.Ec.Lineales.Doble.Traslado}
\begin{array}{ll}
f_{1}\left(1\right)=\tilde{\mu}_{1}\left(r+\frac{f_{2}\left(2\right)}{1-\tilde{\mu}_{2}}\right),& f_{1}\left(3\right)=\hat{\mu}_{1}\left(r_{2}+\frac{f_{2}\left(2\right)}{1-\tilde{\mu}_{2}}\right)+\hat{F}_{1,2}^{(1)}\left(1\right)\\
f_{1}\left(4\right)=\hat{\mu}_{2}\left(r_{2}+\frac{f_{2}\left(2\right)}{1-\tilde{\mu}_{2}}\right)+\hat{F}_{2,2}^{(1)}\left(1\right),&
f_{2}\left(2\right)=\left(r+\frac{f_{1}\left(1\right)}{1-\mu_{1}}\right)\tilde{\mu}_{2},\\
f_{2}\left(3\right)=\hat{\mu}_{1}\left(r_{1}+\frac{f_{1}\left(1\right)}{1-\tilde{\mu}_{1}}\right)+\hat{F}_{1,1}^{(1)}\left(1\right),&
f_{2}\left(4\right)=\hat{\mu}_{2}\left(r_{1}+\frac{f_{1}\left(1\right)}{1-\mu_{1}}\right)+\hat{F}_{2,1}^{(1)}\left(1\right),\\
\hat{f}_{1}\left(1\right)=\left(\hat{r}_{2}+\frac{\hat{f}_{2}\left(4\right)}{1-\hat{\mu}_{2}}\right)\tilde{\mu}_{1}+F_{1,2}^{(1)}\left(1\right),&
\hat{f}_{1}\left(2\right)=\left(\hat{r}_{2}+\frac{\hat{f}_{2}\left(4\right)}{1-\hat{\mu}_{2}}\right)\tilde{\mu}_{2}+F_{2,2}^{(1)}\left(1\right),\\
\hat{f}_{1}\left(3\right)=\left(\hat{r}+\frac{\hat{f}_{2}\left(4\right)}{1-\hat{\mu}_{2}}\right)\hat{\mu}_{1},&
\hat{f}_{2}\left(1\right)=\left(\hat{r}_{1}+\frac{\hat{f}_{1}\left(3\right)}{1-\hat{\mu}_{1}}\right)\mu_{1}+F_{1,1}^{(1)}\left(1\right),\\
\hat{f}_{2}\left(2\right)=\left(\hat{r}_{1}+\frac{\hat{f}_{1}\left(3\right)}{1-\hat{\mu}_{1}}\right)\tilde{\mu}_{2}+F_{2,1}^{(1)}\left(1\right),&
\hat{f}_{2}\left(4\right)=\left(\hat{r}+\frac{\hat{f}_{1}\left(3\right)}{1-\hat{\mu}_{1}}\right)\hat{\mu}_{2},\\
\end{array}
\end{eqnarray}

with system's solutions given by

\begin{eqnarray}
\begin{array}{ll}
f_{1}\left(1\right)=r\frac{\mu_{1}\left(1-\mu_{1}\right)}{1-\mu},&
f_{2}\left(2\right)=r\frac{\tilde{\mu}_{2}\left(1-\tilde{\mu}_{2}\right)}{1-\mu},\\
f_{1}\left(3\right)=\hat{\mu}_{1}\left(r_{2}+\frac{r\tilde{\mu}_{2}}{1-\mu}\right)+\hat{F}_{1,2}^{(1)}\left(1\right),&
f_{1}\left(4\right)=\hat{\mu}_{2}\left(r_{2}+\frac{r\tilde{\mu}_{2}}{1-\mu}\right)+\hat{F}_{2,2}^{(1)}\left(1\right),\\
f_{2}\left(3\right)=\hat{\mu}_{1}\left(r_{1}+\frac{r\mu_{1}}{1-\mu}\right)+\hat{F}_{1,1}^{(1)}\left(1\right),&
f_{2}\left(4\right)=\hat{\mu}_{2}\left(r_{1}+\frac{r\mu_{1}}{1-\mu}\right)+\hat{F}_{2,1}^{(1)}\left(1\right),\\
\hat{f}_{1}\left(1\right)=\tilde{\mu}_{1}\left(\hat{r}_{2}+\frac{\hat{r}\hat{\mu}_{2}}{1-\hat{\mu}}\right)+F_{1,2}^{(1)}\left(1\right),&
\hat{f}_{1}\left(2\right)=\tilde{\mu}_{2}\left(\hat{r}_{2}+\frac{\hat{r}\hat{\mu}_{2}}{1-\hat{\mu}}\right)+F_{2,2}^{(1)}\left(1\right),\\
\hat{f}_{2}\left(1\right)=\tilde{\mu}_{1}\left(\hat{r}_{1}+\frac{\hat{r}\hat{\mu}_{1}}{1-\hat{\mu}}\right)+F_{1,1}^{(1)}\left(1\right),&
\hat{f}_{2}\left(2\right)=\tilde{\mu}_{2}\left(\hat{r}_{1}+\frac{\hat{r}\hat{\mu}_{1}}{1-\hat{\mu}}\right)+F_{2,1}^{(1)}\left(1\right)
\end{array}
\end{eqnarray}

%_________________________________________________________________________________________________________
\subsection*{General Second Order Derivatives}
%_________________________________________________________________________________________________________


Now, taking the second order derivative with respect to the equations given in (\ref{Sist.Ec.Lineales.Doble.Traslado}) we obtain it in their general form

\small{
\begin{eqnarray*}\label{Ec.Derivadas.Segundo.Orden.Doble.Transferencia}
D_{k}D_{i}F_{1}&=&D_{k}D_{i}\left(R_{2}+F_{2}+\indora_{i\geq3}\hat{F}_{4}\right)+D_{i}R_{2}D_{k}\left(F_{2}+\indora_{k\geq3}\hat{F}_{4}\right)+D_{i}F_{2}D_{k}\left(R_{2}+\indora_{k\geq3}\hat{F}_{4}\right)+\indora_{i\geq3}D_{i}\hat{F}_{4}D_{k}\left(R_{}+F_{2}\right)\\
D_{k}D_{i}F_{2}&=&D_{k}D_{i}\left(R_{1}+F_{1}+\indora_{i\geq3}\hat{F}_{3}\right)+D_{i}R_{1}D_{k}\left(F_{1}+\indora_{k\geq3}\hat{F}_{3}\right)+D_{i}F_{1}D_{k}\left(R_{1}+\indora_{k\geq3}\hat{F}_{3}\right)+\indora_{i\geq3}D_{i}\hat{F}_{3}D_{k}\left(R_{1}+F_{1}\right)\\
D_{k}D_{i}\hat{F}_{3}&=&D_{k}D_{i}\left(\hat{R}_{4}+\indora_{i\leq2}F_{2}+\hat{F}_{4}\right)+D_{i}\hat{R}_{4}D_{k}\left(\indora_{k\leq2}F_{2}+\hat{F}_{4}\right)+D_{i}\hat{F}_{4}D_{k}\left(\hat{R}_{4}+\indora_{k\leq2}F_{2}\right)+\indora_{i\leq2}D_{i}F_{2}D_{k}\left(\hat{R}_{4}+\hat{F}_{4}\right)\\
D_{k}D_{i}\hat{F}_{4}&=&D_{k}D_{i}\left(\hat{R}_{3}+\indora_{i\leq2}F_{1}+\hat{F}_{3}\right)+D_{i}\hat{R}_{3}D_{k}\left(\indora_{k\leq2}F_{1}+\hat{F}_{3}\right)+D_{i}\hat{F}_{3}D_{k}\left(\hat{R}_{3}+\indora_{k\leq2}F_{1}\right)+\indora_{i\leq2}D_{i}F_{1}D_{k}\left(\hat{R}_{3}+\hat{F}_{3}\right)
\end{eqnarray*}}
for $i,k=1,\ldots,4$. In order to have it in an specific way we need to compute the expressions $D_{k}D_{i}\left(R_{2}+F_{2}+\indora_{i\geq3}\hat{F}_{4}\right)$

%_________________________________________________________________________________________________________
\subsubsection*{Second Order Derivatives: Serve's Switchover Times}
%_________________________________________________________________________________________________________

Remind $R_{i}\left(z_{1},z_{2},w_{1},w_{2}\right)=R_{i}\left(P_{1}\left(z_{1}\right)\tilde{P}_{2}\left(z_{2}\right)
\hat{P}_{1}\left(w_{1}\right)\hat{P}_{2}\left(w_{2}\right)\right)$,  which we will write in his reduced form $R_{i}=R_{i}\left(
P_{1}\tilde{P}_{2}\hat{P}_{1}\hat{P}_{2}\right)$, and according to the notation given in \cite{Lang} we obtain

\begin{eqnarray}
D_{i}D_{i}R_{k}=D^{2}R_{k}\left(D_{i}P_{i}\right)^{2}+DR_{k}D_{i}D_{i}P_{i}
\end{eqnarray}

whereas for $i\neq j$

\begin{eqnarray}
D_{i}D_{j}R_{k}=D^{2}R_{k}D_{i}P_{i}D_{j}P_{j}+DR_{k}D_{j}P_{j}D_{i}P_{i}
\end{eqnarray}

%_________________________________________________________________________________________________________
\subsubsection*{Second Order Derivatives: Queue Lengths}
%_________________________________________________________________________________________________________

Just like before the expression $F_{1}\left(\tilde{\theta}_{1}\left(\tilde{P}_{2}\left(z_{2}\right)\hat{P}_{1}\left(w_{1}\right)\hat{P}_{2}\left(w_{2}\right)\right),
z_{2}\right)$, will be denoted by $F_{1}\left(\tilde{\theta}_{1}\left(\tilde{P}_{2}\hat{P}_{1}\hat{P}_{2}\right),z_{2}\right)$, then the mixed partial derivatives are:

\begin{eqnarray*}
D_{j}D_{i}F_{1}&=&\indora_{i,j\neq1}D_{1}D_{1}F_{1}\left(D\tilde{\theta}_{1}\right)^{2}D_{i}P_{i}D_{j}P_{j}
+\indora_{i,j\neq1}D_{1}F_{1}D^{2}\tilde{\theta}_{1}D_{i}P_{i}D_{j}P_{j}
+\indora_{i,j\neq1}D_{1}F_{1}D\tilde{\theta}_{1}\left(\indora_{i=j}D_{i}^{2}P_{i}+\indora_{i\neq j}D_{i}P_{i}D_{j}P_{j}\right)\\
&+&\indora_{i+j\leq6}D_{1}D_{2}F_{1}D\tilde{\theta}_{1}D_{i}P_{i}
+\indora_{i=2}\left(D_{1}D_{2}F_{1}D\tilde{\theta}_{1}D_{i}P_{i}+D_{i}^{2}F_{1}\right)
\end{eqnarray*}


Recall the expression for $F_{1}\left(\tilde{\theta}_{1}\left(\tilde{P}_{2}\left(z_{2}\right)\hat{P}_{1}\left(w_{1}\right)\hat{P}_{2}\left(w_{2}\right)\right),
z_{2}\right)$, which is denoted by $F_{1}\left(\tilde{\theta}_{1}\left(\tilde{P}_{2}\hat{P}_{1}\hat{P}_{2}\right),z_{2}\right)$, then the mixed partial derivatives are given by

\begin{eqnarray*}
\begin{array}{llll}
D_{1}D_{1}F_{1}=0,&
D_{2}D_{1}F_{1}=0,&
D_{3}D_{1}F_{1}=0,&
D_{4}D_{1}F_{1}=0,\\
D_{1}D_{2}F_{1}=0,&
D_{1}D_{3}F_{1}=0,&
D_{1}D_{4}F_{1}=0,&
\end{array}
\end{eqnarray*}

\begin{eqnarray*}
D_{2}D_{2}F_{1}&=&D_{1}^{2}F_{1}\left(D\tilde{\theta}_{1}\right)^{2}\left(D_{2}\tilde{P}_{2}\right)^{2}
+D_{1}F_{1}D^{2}\tilde{\theta}_{1}\left(D_{2}\tilde{P}_{2}\right)^{2}
+D_{1}F_{1}D\tilde{\theta}_{1}D_{2}^{2}\tilde{P}_{2}
+D_{1}D_{2}F_{1}D\tilde{\theta}_{1}D_{2}\tilde{P}_{2}\\
&+&D_{1}D_{2}F_{1}D\tilde{\theta}_{1}D_{2}\tilde{P}_{2}+D_{2}D_{2}F_{1}\\
&=&f_{1}\left(1,1\right)\left(\frac{\tilde{\mu}_{2}}{1-\tilde{\mu}_{1}}\right)^{2}
+f_{1}\left(1\right)\tilde{\theta}_{1}^{(2)}\tilde{\mu}_{2}^{(2)}
+f_{1}\left(1\right)\frac{1}{1-\tilde{\mu}_{1}}\tilde{P}_{2}^{(2)}+f_{1}\left(1,2\right)\frac{\tilde{\mu}_{2}}{1-\tilde{\mu}_{1}}+f_{1}\left(1,2\right)\frac{\tilde{\mu}_{2}}{1-\tilde{\mu}_{1}}+f_{1}\left(2,2\right)
\end{eqnarray*}

\begin{eqnarray*}
D_{3}D_{2}F_{1}&=&D_{1}^{2}F_{1}\left(D\tilde{\theta}_{1}\right)^{2}D_{3}\hat{P}_{1}D_{2}\tilde{P}_{2}+D_{1}F_{1}D^{2}\tilde{\theta}_{1}D_{3}\hat{P}_{1}D_{2}\tilde{P}_{2}+D_{1}F_{1}D\tilde{\theta}_{1}D_{2}\tilde{P}_{2}D_{3}\hat{P}_{1}+D_{1}D_{2}F_{1}D\tilde{\theta}_{1}D_{3}\hat{P}_{1}\\
&=&f_{1}\left(1,1\right)\left(\frac{1}{1-\tilde{\mu}_{1}}\right)^{2}\tilde{\mu}_{2}\hat{\mu}_{1}+f_{1}\left(1\right)\tilde{\theta}_{1}^{(2)}\tilde{\mu}_{2}\hat{\mu}_{1}+f_{1}\left(1\right)\frac{\tilde{\mu}_{2}\hat{\mu}_{1}}{1-\tilde{\mu}_{1}}+f_{1}\left(1,2\right)\frac{\hat{\mu}_{1}}{1-\tilde{\mu}_{1}}
\end{eqnarray*}

\begin{eqnarray*}
D_{4}D_{2}F_{1}&=&D_{1}^{2}F_{1}\left(D\tilde{\theta}_{1}\right)^{2}D_{4}\hat{P}_{2}D_{2}\tilde{P}_{2}+D_{1}F_{1}D^{2}\tilde{\theta}_{1}D_{4}\hat{P}_{2}D_{2}\tilde{P}_{2}+D_{1}F_{1}D\tilde{\theta}_{1}D_{2}\tilde{P}_{2}D_{4}\hat{P}_{2}+D_{1}D_{2}F_{1}D\tilde{\theta}_{1}D_{4}\hat{P}_{2}\\
&=&f_{1}\left(1,1\right)\left(\frac{1}{1-\tilde{\mu}_{1}}\right)^{2}\tilde{\mu}_{2}\hat{\mu}_{2}+f_{1}\left(1\right)\tilde{\theta}_{1}^{(2)}\tilde{\mu}_{2}\hat{\mu}_{2}+f_{1}\left(1\right)\frac{\tilde{\mu}_{2}\hat{\mu}_{2}}{1-\tilde{\mu}_{1}}+f_{1}\left(1,2\right)\frac{\hat{\mu}_{2}}{1-\tilde{\mu}_{1}}
\end{eqnarray*}

\begin{eqnarray*}
D_{2}D_{3}F_{1}&=&
D_{1}^{2}F_{1}\left(D\tilde{\theta}_{1}\right)^{2}D_{2}\tilde{P}_{2}D_{3}\hat{P}_{1}+
D_{2}D_{1}F_{1}D\tilde{\theta}_{1}D_{3}\hat{P}_{1}+
D_{1}F_{1}D^{2}\tilde{\theta}_{1}D_{2}\tilde{P}_{2}D_{3}\hat{P}_{1}+
D_{1}F_{1}D\tilde{\theta}_{1}D_{3}\hat{P}_{1}D_{2}\tilde{P}_{2}\\
&=&f_{1}\left(1,1\right)\left(\frac{1}{1-\tilde{\mu}_{1}}\right)^{2}\tilde{\mu}_{2}\hat{\mu}_{1}+f_{1}\left(1\right)\tilde{\theta}_{1}^{(2)}\tilde{\mu}_{2}\hat{\mu}_{1}+f_{1}\left(1\right)\frac{\tilde{\mu}_{2}\hat{\mu}_{1}}{1-\tilde{\mu}_{1}}+f_{1}\left(1,2\right)\frac{\hat{\mu}_{1}}{1-\tilde{\mu}_{1}}
\end{eqnarray*}

\begin{eqnarray*}
D_{3}D_{3}F_{1}&=&D_{1}^{2}F_{1}\left(D\tilde{\theta}_{1}\right)^{2}\left(D_{3}\hat{P}_{1}\right)^{2}+D_{1}F_{1}D^{2}\tilde{\theta}_{1}\left(D_{3}\hat{P}_{1}\right)^{2}+D_{1}F_{1}D\tilde{\theta}_{1}D_{3}^{2}\hat{P}_{1}\\
&=&f_{1}\left(1,1\right)\left(\frac{\hat{\mu}_{1}}{1-\tilde{\mu}_{1}}\right)^{2}+f_{1}\left(1\right)\tilde{\theta}_{1}^{(2)}\hat{\mu}_{1}^{2}+f_{1}\left(1\right)\frac{\hat{\mu}_{1}^{2}}{1-\tilde{\mu}_{1}}
\end{eqnarray*}

\begin{eqnarray*}
D_{4}D_{3}F_{1}&=&D_{1}^{2}F_{1}\left(D\tilde{\theta}_{1}\right)^{2}D_{4}\hat{P}_{2}D_{3}\hat{P}_{1}+D_{1}F_{1}D^{2}\tilde{\theta}_{1}D_{4}\hat{P}_{2}D_{3}\hat{P}_{1}+D_{1}F_{1}D\tilde{\theta}_{1}D_{3}\hat{P}_{1}D_{4}\hat{P}_{2}\\
&=&f_{1}\left(1,1\right)\left(\frac{1}{1-\tilde{\mu}_{1}}\right)^{2}\hat{\mu}_{1}\hat{\mu}_{2}+f_{1}\left(1\right)\left(\tilde{\theta}_{1}\right)^{2}\hat{\mu}_{2}\hat{\mu}_{1}+f_{1}\left(1\right)\frac{\hat{\mu}_{2}\hat{\mu}_{1}}{1-\tilde{\mu}_{1}}
\end{eqnarray*}

\begin{eqnarray*}
D_{2}D_{4}F_{1}&=&D_{1}^{2}F_{1}\left(D\tilde{\theta}_{1}\right)^{2}D_{2}\tilde{P}_{2}D_{4}\hat{P}_{2}+D_{1}F_{1}D^{2}\tilde{\theta}_{1}D_{2}\tilde{P}_{2}D_{4}\hat{P}_{2}+D_{1}F_{1}D\tilde{\theta}_{1}D_{4}\hat{P}_{2}D_{2}\tilde{P}_{2}+D_{2}D_{1}F_{1}D\tilde{\theta}_{1}D_{4}\hat{P}_{2}\\
&=&f_{1}\left(1,1\right)\left(\frac{1}{1-\tilde{\mu}_{1}}\right)^{2}\hat{\mu}_{2}\tilde{\mu}_{2}
+f_{1}\left(1\right)\tilde{\theta}_{1}^{(2)}\hat{\mu}_{2}\tilde{\mu}_{2}
+f_{1}\left(1\right)\frac{\hat{\mu}_{2}\tilde{\mu}_{2}}{1-\tilde{\mu}_{1}}+f_{1}\left(1,2\right)\frac{\hat{\mu}_{2}}{1-\tilde{\mu}_{1}}
\end{eqnarray*}

\begin{eqnarray*}
D_{3}D_{4}F_{1}&=&D_{1}^{2}F_{1}\left(D\tilde{\theta}_{1}\right)^{2}D_{3}\hat{P}_{1}D_{4}\hat{P}_{2}+D_{1}F_{1}D^{2}\tilde{\theta}_{1}D_{3}\hat{P}_{1}D_{4}\hat{P}_{2}+D_{1}F_{1}D\tilde{\theta}_{1}D_{4}\hat{P}_{2}D_{3}\hat{P}_{1}\\
&=&f_{1}\left(1,1\right)\left(\frac{1}{1-\tilde{\mu}_{1}}\right)^{2}\hat{\mu}_{1}\hat{\mu}_{2}+f_{1}\left(1\right)\tilde{\theta}_{1}^{(2)}\hat{\mu}_{1}\hat{\mu}_{2}+f_{1}\left(1\right)\frac{\hat{\mu}_{1}\hat{\mu}_{2}}{1-\tilde{\mu}_{1}}
\end{eqnarray*}

\begin{eqnarray*}
D_{4}D_{4}F_{1}&=&D_{1}^{2}F_{1}\left(D\tilde{\theta}_{1}\right)^{2}\left(D_{4}\hat{P}_{2}\right)^{2}+D_{1}F_{1}D^{2}\tilde{\theta}_{1}\left(D_{4}\hat{P}_{2}\right)^{2}+D_{1}F_{1}D\tilde{\theta}_{1}D_{4}^{2}\hat{P}_{2}\\
&=&f_{1}\left(1,1\right)\left(\frac{\hat{\mu}_{2}}{1-\tilde{\mu}_{1}}\right)^{2}+f_{1}\left(1\right)\tilde{\theta}_{1}^{(2)}\left(\hat{\mu}_{2}\right)^{2}+f_{1}\left(1\right)\frac{1}{1-\tilde{\mu}_{1}}\hat{P}_{2}^{(2)}
\end{eqnarray*}



Meanwhile for  $F_{2}\left(z_{1},\tilde{\theta}_{2}\left(P_{1}\hat{P}_{1}\hat{P}_{2}\right)\right)$

\begin{eqnarray*}
D_{j}D_{i}F_{2}&=&\indora_{i,j\neq2}D_{2}D_{2}F_{2}\left(D\theta_{2}\right)^{2}D_{i}P_{i}D_{j}P_{j}+\indora_{i,j\neq2}D_{2}F_{2}D^{2}\theta_{2}D_{i}P_{i}D_{j}P_{j}\\
&+&\indora_{i,j\neq2}D_{2}F_{2}D\theta_{2}\left(\indora_{i=j}D_{i}^{2}P_{i}
+\indora_{i\neq j}D_{i}P_{i}D_{j}P_{j}\right)\\
&+&\indora_{i,j\leq6}D_{2}D_{1}F_{2}D\theta_{2}D_{i}P_{i}
+\indora_{i=1}\left(D_{2}D_{1}F_{2}D\theta_{2}D_{i}P_{i}+D_{i}^{2}F_{2}\right)
\end{eqnarray*}

\begin{eqnarray*}
\begin{array}{llll}
D_{2}D_{1}F_{2}=0,&
D_{2}D_{3}F_{3}=0,&
D_{2}D_{4}F_{2}=0,&\\
D_{1}D_{2}F_{2}=0,&
D_{2}D_{2}F_{2}=0,&
D_{3}D_{2}F_{2}=0,&
D_{4}D_{2}F_{2}=0\\
\end{array}
\end{eqnarray*}


\begin{eqnarray*}
D_{1}D_{1}F_{2}&=&
D_{1}^{2}P_{1}D\tilde{\theta}_{2}D_{2}F_{2}+
\left(D_{1}P_{1}\right)^{2}D^{2}\tilde{\theta}_{2}D_{2}F_{2}+
D_{1}P_{1}D\tilde{\theta}_{2}D_{1}D_{2}F_{2}+
\left(D_{1}P_{1}\right)^{2}\left(D\tilde{\theta}_{2}\right)^{2}D_{2}^{2}F_{2}\\
&+&D_{1}P_{1}D\tilde{\theta}_{2}D_{2}D_{1}F_{2}+
D_{1}^{2}F_{2}\\
&=&f_{2}\left(2\right)\frac{\tilde{P}_{1}^{(2)}}{1-\tilde{\mu}_{2}}
+f_{2}\left(2\right)\theta_{2}^{(2)}\tilde{\mu}_{1}^{2}
+f_{2}\left(2,1\right)\frac{\tilde{\mu}_{1}}{1-\tilde{\mu}_{2}}
+\left(\frac{\tilde{\mu}_{1}}{1-\tilde{\mu}_{2}}\right)^{2}f_{2}\left(2,2\right)
+\frac{\tilde{\mu}_{1}}{1-\tilde{\mu}_{2}}f_{2}\left(1,2\right)+f_{2}\left(1,1\right)
\end{eqnarray*}


\begin{eqnarray*}
D_{3}D_{1}F_{2}&=&D_{2}D_{1}F_{2}D\tilde{\theta}_{2}D_{3}\hat{P}_{1}
+D_{2}^{2}F_{2}\left(D\tilde{\theta}_{2}\right)^{2}D_{3}P_{1}D_{1}P_{1}
+D_{2}F_{2}D^{2}\tilde{\theta}_{2}D_{3}\hat{P}_{1}D_{1}P_{1}
+D_{2}F_{2}D\tilde{\theta}_{2}D_{1}P_{1}D_{3}\hat{P}_{1}\\
&=&f_{2}\left(1,2\right)\frac{\hat{\mu}_{1}}{1-\tilde{\mu}_{2}}
+f_{2}\left(2,2\right)\left(\frac{1}{1-\tilde{\mu}_{2}}\right)^{2}\tilde{\mu}_{1}\hat{\mu}_{1}
+f_{2}\left(2\right)\tilde{\theta}_{2}^{(2)}\tilde{\mu}_{1}\hat{\mu}_{1}
+f_{2}\left(2\right)\frac{\tilde{\mu}_{1}\hat{\mu}_{1}}{1-\tilde{\mu}_{2}}
\end{eqnarray*}


\begin{eqnarray*}
D_{4}D_{1}F_{2}&=&D_{1}D_{2}F_{2}D\tilde{\theta}_{2}D_{4}\hat{P}_{2}
+D_{2}^{2}F_{2}\left(D\tilde{\theta}_{2}\right)^{2}D_{4}P_{2}D_{1}P_{1}
+D_{2}F_{2}D^{2}\tilde{\theta}_{2}D_{4}\hat{P}_{2}D_{1}P_{1}
+D_{2}F_{2}D\tilde{\theta}_{2}D_{1}P_{1}D_{4}\hat{P}_{2}\\
&=&f_{2}\left(1,2\right)\frac{\hat{\mu}_{2}}{1-\tilde{\mu}_{2}}
+f_{2}\left(2,2\right)\left(\frac{1}{1-\tilde{\mu}_{2}}\right)^{2}\tilde{\mu}_{1}\hat{\mu}_{2}
+f_{2}\left(2\right)\tilde{\theta}_{2}^{(2)}\tilde{\mu}_{1}\hat{\mu}_{2}
+f_{2}\left(2\right)\frac{\tilde{\mu}_{1}\hat{\mu}_{2}}{1-\tilde{\mu}_{2}}
\end{eqnarray*}


\begin{eqnarray*}
D_{1}D_{3}F_{2}&=&D_{2}^{2}F_{2}\left(D\tilde{\theta}_{2}\right)^{2}D_{1}P_{1}D_{3}\hat{P}_{1}
+D_{2}D_{1}F_{2}D\tilde{\theta}_{2}D_{3}\hat{P}_{1}
+D_{2}F_{2}D^{2}\tilde{\theta}_{2}D_{1}P_{1}D_{3}\hat{P}_{1}
+D_{2}F_{2}D\tilde{\theta}_{2}D_{3}\hat{P}_{1}D_{1}P_{1}\\
&=&f_{2}\left(2,2\right)\left(\frac{1}{1-\tilde{\mu}_{2}}\right)^{2}\tilde{\mu}_{1}\hat{\mu}_{1}
+f_{2}\left(2,1\right)\frac{\hat{\mu}_{1}}{1-\tilde{\mu}_{2}}
+f_{2}\left(2\right)\tilde{\theta}_{2}^{(2)}\tilde{\mu}_{1}\hat{\mu}_{1}
+f_{2}\left(2\right)\frac{\tilde{\mu}_{1}\hat{\mu}_{1}}{1-\tilde{\mu}_{2}}
\end{eqnarray*}


\begin{eqnarray*}
D_{3}D_{3}F_{2}&=&D_{2}^{2}F_{2}\left(D\tilde{\theta}_{2}\right)^{2}\left(D_{3}\hat{P}_{1}\right)^{2}
+D_{2}F_{2}\left(D_{3}\hat{P}_{1}\right)^{2}D^{2}\tilde{\theta}_{2}
+D_{2}F_{2}D\tilde{\theta}_{2}D_{3}^{2}\hat{P}_{1}\\
&=&f_{2}\left(2,2\right)\left(\frac{1}{1-\tilde{\mu}_{2}}\right)^{2}\hat{\mu}_{1}^{2}
+f_{2}\left(2\right)\tilde{\theta}_{2}^{(2)}\hat{\mu}_{1}^{2}
+f_{2}\left(2\right)\frac{\hat{P}_{1}^{(2)}}{1-\tilde{\mu}_{2}}
\end{eqnarray*}


\begin{eqnarray*}
D_{4}D_{3}F_{2}&=&D_{2}^{2}F_{2}\left(D\tilde{\theta}_{2}\right)^{2}D_{4}\hat{P}_{2}D_{3}\hat{P}_{1}
+D_{2}F_{2}D^{2}\tilde{\theta}_{2}D_{4}\hat{P}_{2}D_{3}\hat{P}_{1}
+D_{2}F_{2}D\tilde{\theta}_{2}D_{3}\hat{P}_{1}D_{4}\hat{P}_{2}\\
&=&f_{2}\left(2,2\right)\left(\frac{1}{1-\tilde{\mu}_{2}}\right)^{2}\hat{\mu}_{1}\hat{\mu}_{2}
+f_{2}\left(2\right)\tilde{\theta}_{2}^{(2)}\hat{\mu}_{1}\hat{\mu}_{2}
+f_{2}\left(2\right)\frac{\hat{\mu}_{1}\hat{\mu}_{2}}{1-\tilde{\mu}_{2}}
\end{eqnarray*}


\begin{eqnarray*}
D_{1}D_{4}F_{2}&=&D_{2}^{2}F_{2}\left(D\tilde{\theta}_{2}\right)^{2}D_{1}P_{1}D_{4}\hat{P}_{2}
+D_{1}D_{2}F_{2}D\tilde{\theta}_{2}D_{4}\hat{P}_{2}
+D_{2}F_{2}D^{2}\tilde{\theta}_{2}D_{1}P_{1}D_{4}\hat{P}_{2}
+D_{2}F_{2}D\tilde{\theta}_{2}D_{4}\hat{P}_{2}D_{1}P_{1}\\
&=&f_{2}\left(2,2\right)\left(\frac{1}{1-\tilde{\mu}_{2}}\right)^{2}\tilde{\mu}_{1}\hat{\mu}_{2}
+f_{2}\left(1,2\right)\frac{\hat{\mu}_{2}}{1-\tilde{\mu}_{2}}
+f_{2}\left(2\right)\tilde{\theta}_{2}^{(2)}\tilde{\mu}_{1}\hat{\mu}_{2}
+f_{2}\left(2\right)\frac{\tilde{\mu}_{1}\hat{\mu}_{2}}{1-\tilde{\mu}_{2}}
\end{eqnarray*}


\begin{eqnarray*}
D_{3}D_{4}F_{2}&=&
D_{2}F_{2}D\tilde{\theta}_{2}D_{4}\hat{P}_{2}D_{3}\hat{P}_{1}
+D_{2}F_{2}D^{2}\tilde{\theta}_{2}D_{4}\hat{P}_{2}D_{3}\hat{P}_{1}
+D_{2}^{2}F_{2}\left(D\tilde{\theta}_{2}\right)^{2}D_{4}\hat{P}_{2}D_{3}\hat{P}_{1}\\
&=&f_{2}\left(2,2\right)\left(\frac{1}{1-\tilde{\mu}_{2}}\right)^{2}\hat{\mu}_{1}\hat{\mu}_{2}
+f_{2}\left(2\right)\tilde{\theta}_{2}^{(2)}\hat{\mu}_{1}\hat{\mu}_{2}
+f_{2}\left(2\right)\frac{\hat{\mu}_{1}\hat{\mu}_{2}}{1-\tilde{\mu}_{2}}
\end{eqnarray*}


\begin{eqnarray*}
D_{4}D_{4}F_{2}&=&D_{2}F_{2}D\tilde{\theta}_{2}D_{4}^{2}\hat{P}_{2}
+D_{2}F_{2}D^{2}\tilde{\theta}_{2}\left(D_{4}\hat{P}_{2}\right)^{2}
+D_{2}^{2}F_{2}\left(D\tilde{\theta}_{2}\right)^{2}\left(D_{4}\hat{P}_{2}\right)^{2}\\
&=&f_{2}\left(2,2\right)\left(\frac{1}{1-\tilde{\mu}_{2}}\right)^{2}\hat{\mu}_{2}^{2}
+f_{2}\left(2\right)\tilde{\theta}_{2}^{(2)}\hat{\mu}_{2}^{2}
+f_{2}\left(2\right)\frac{\hat{P}_{2}^{(2)}}{1-\tilde{\mu}_{2}}
\end{eqnarray*}


%\newpage



%\newpage

For $\hat{F}_{1}\left(\hat{\theta}_{1}\left(P_{1}\tilde{P}_{2}\hat{P}_{2}\right),w_{2}\right)$



\begin{eqnarray*}
D_{j}D_{i}\hat{F}_{1}&=&\indora_{i,j\neq3}D_{3}D_{3}\hat{F}_{1}\left(D\hat{\theta}_{1}\right)^{2}D_{i}P_{i}D_{j}P_{j}
+\indora_{i,j\neq3}D_{3}\hat{F}_{1}D^{2}\hat{\theta}_{1}D_{i}P_{i}D_{j}P_{j}
+\indora_{i,j\neq3}D_{3}\hat{F}_{1}D\hat{\theta}_{1}\left(\indora_{i=j}D_{i}^{2}P_{i}+\indora_{i\neq j}D_{i}P_{i}D_{j}P_{j}\right)\\
&+&\indora_{i+j\geq5}D_{3}D_{4}\hat{F}_{1}D\hat{\theta}_{1}D_{i}P_{i}
+\indora_{i=4}\left(D_{3}D_{4}\hat{F}_{1}D\hat{\theta}_{1}D_{i}P_{i}+D_{i}^{2}\hat{F}_{1}\right)
\end{eqnarray*}


\begin{eqnarray*}
\begin{array}{llll}
D_{3}D_{1}\hat{F}_{1}=0,&
D_{3}D_{2}\hat{F}_{1}=0,&
D_{1}D_{3}\hat{F}_{1}=0,&
D_{2}D_{3}\hat{F}_{1}=0\\
D_{3}D_{3}\hat{F}_{1}=0,&
D_{4}D_{3}\hat{F}_{1}=0,&
D_{3}D_{4}\hat{F}_{1}=0,&
\end{array}
\end{eqnarray*}


\begin{eqnarray*}
D_{1}D_{1}\hat{F}_{1}&=&
D\hat{\theta}_{1}D_{1}^{2}P_{1}D_{3}\hat{F}_{1}
+\left(D_{1}P_{1}\right)^{2}D^{2}\hat{\theta}_{1}D_{3}\hat{F}_{1}
+\left(D_{1}P_{1}\right)^{2}\left(D\hat{\theta}_{1}\right)^{2}D_{3}^{2}\hat{F}_{1}\\
&=&\hat{f}_{1}\left(3,3\right)\left(\frac{\tilde{\mu}_{1}}{1-\hat{\mu}_{2}}\right)^{2}
+\hat{f}_{1}\left(3\right)\frac{P_{1}^{(2)}}{1-\hat{\mu}_{1}}
+\hat{f}_{1}\left(3\right)\hat{\theta}_{1}^{(2)}\tilde{\mu}_{1}^{2}
\end{eqnarray*}


\begin{eqnarray*}
D_{2}D_{1}\hat{F}_{1}&=&D_{1}P_{1}D_{2}P_{2}D\hat{\theta}_{1}D_{3}\hat{F}_{1}+
D_{1}P_{1}D_{2}P_{2}D^{2}\hat{\theta}_{1}D_{3}\hat{F}_{1}+
D_{1}P_{1}D_{2}P_{1}\left(D\hat{\theta}_{1}\right)^{2}D_{3}^{2}\hat{f}_{1}\\
&=&\hat{f}_{1}\left(3\right)\frac{\tilde{\mu}_{1}\tilde{\mu}_{2}}{1-\hat{\mu}_{1}}
+\hat{f}_{1}\left(3\right)\tilde{\mu}_{1}\tilde{\mu}_{2}\hat{\theta}_{1}^{(2)}
+\hat{f}_{1}\left(3,3\right)\left(\frac{1}{1-\hat{\mu}_{1}}\right)^{2}\tilde{\mu}_{1}\tilde{\mu}_{2}
\end{eqnarray*}


\begin{eqnarray*}
D_{4}D_{1}\hat{F}_{1}&=&D_{1}P_{1}D_{4}\hat{P}_{2}D\hat{\theta}_{1}D_{3}\hat{F}_{1}
+D_{1}P_{1}D_{4}\hat{P}_{2}D^{2}\hat{\theta}_{1}D_{3}\hat{F}_{1}
+D_{1}P_{1}D\hat{\theta}_{1}D_{2}D_{1}\hat{F}_{1}
+D_{4}\hat{P}_{2}D_{1}P_{1}\left(D\hat{\theta}_{1}\right)^{2}D_{3}D_{3}\hat{F}_{1}\\
&=&\hat{f}_{1}\left(3\right)\frac{\tilde{\mu}_{1}\hat{\mu}_{2}}{1-\hat{\mu}_{1}}
+\hat{f}_{1}\left(3\right)\hat{\theta}_{1}^{(2)}\tilde{\mu}_{1}\hat{\mu}_{2}
+\hat{f}_{1}\left(3,4\right)\frac{\tilde{\mu}_{1}}{1-\hat{\mu}_{1}}
+\hat{f}_{1}\left(3,3\right)\left(\frac{1}{1-\hat{\mu}_{1}}\right)^{2}\tilde{\mu}_{1}\hat{\mu}_{1}
\end{eqnarray*}


\begin{eqnarray*}
D_{1}D_{2}\hat{F}_{1}&=&D_{1}P_{1}D_{2}P_{2}D\hat{\theta}_{1}D_{3}\hat{F}_{1}+
D_{1}P_{1}D_{2}P_{2}D^{2}\hat{\theta}_{1}D_{3}\hat{F}_{1}+
D_{1}P_{1}D_{2}P_{2}\left(D\hat{\theta}_{1}\right)^{2}D_{3}^{2}\hat{F}_{1}\\
&=&\hat{f}_{1}\left(3\right)\frac{\tilde{\mu}_{1}\tilde{\mu}_{2}}{1-\hat{\mu}_{1}}
+\hat{f}_{1}\left(3\right)\hat{\theta}_{1}^{(2)}\tilde{\mu}_{1}\tilde{\mu}_{2}
+\hat{f}_{1}\left(3,3\right)\left(\frac{1}{1-\hat{\mu}_{1}}\right)^{2}\tilde{\mu}_{1}\tilde{\mu}_{2}
\end{eqnarray*}


\begin{eqnarray*}
D_{2}D_{2}\hat{F}_{1}&=&
D\hat{\theta}_{1}D_{2}^{2}P_{2}D_{3}\hat{F}_{1}+
 \left(D_{2}P_{2}\right)^{2}D^{2}\hat{\theta}_{1}D_{3}\hat{F}_{1}+
\left(D_{2}P_{2}\right)^{2}\left(D\hat{\theta}_{1}\right)^{2}D_{3}^{2}\hat{F}_{1}\\
&=&\hat{f}_{1}\left(3\right)\tilde{P}_{2}^{(2)}\frac{1}{1-\hat{\mu}_{1}}
+\hat{f}_{1}\left(3\right)\hat{\theta}_{1}^{(2)}\tilde{\mu}_{2}^{2}
+\hat{f}_{1}\left(3,3\right)\left(\frac{\tilde{\mu}_{2}}{1-\hat{\mu}_{1}}\right)^{2}
\end{eqnarray*}


\begin{eqnarray*}
D_{4}D_{2}\hat{F}_{1}&=&D_{2}P_{2}D_{4}\hat{P}_{2}D\hat{\theta}_{1}D_{3}\hat{F}_{1}
+D_{2}P_{2}D_{4}\hat{P}_{2}D^{2}\hat{\theta}_{1}D_{3}\hat{F}_{1}
+D_{2}P_{2}D\hat{\theta}_{1}D_{4}D_{3}\hat{F}_{1}
+D_{2}P_{2}\left(D\hat{\theta}_{1}\right)^{2}D_{4}\hat{P}_{2}D_{3}^{2}\hat{F}_{1}\\
&=&\hat{f}_{1}\left(3\right)\frac{\tilde{\mu}_{2}\hat{\mu}_{2}}{1-\hat{\mu}_{1}}
+\hat{f}_{1}\left(3\right)\hat{\theta}_{1}^{(2)}\tilde{\mu}_{2}\hat{\mu}_{2}
+\hat{f}_{1}\left(3,4\right)\frac{\tilde{\mu}_{2}}{1-\hat{\mu}_{1}}
+\hat{f}_{1}\left(3,3\right)\left(\frac{1}{1-\hat{\mu}_{1}}\right)^{2}\tilde{\mu}_{2}\hat{\mu}_{2}
\end{eqnarray*}



\begin{eqnarray*}
D_{1}D_{4}\hat{F}_{1}&=&D_{1}P_{1}D_{4}\hat{F}_{2}D\hat{\theta}_{1}D_{1}\hat{F}_{1}
+D_{1}P_{1}D_{4}\hat{P}_{2}D^{2}\hat{\theta}_{1}D_{1}\hat{F}_{1}
+D_{1}P_{1}D\hat{\theta}_{1}D_{2}D_{1}\hat{F}_{1}
+ D_{1}P_{1}D_{4}\hat{P}_{2}\left(D\hat{\theta}_{1}\right)^{2}D_{1}D_{1}
\hat{F}_{1}\\
&=&\hat{f}_{1}\left(1\right)\frac{\tilde{\mu}_{1}\hat{\mu}_{2}}{1-\hat{\mu}_{1}}
+\hat{f}_{1}\left(1\right)\hat{\theta}_{1}^{(2)}\tilde{\mu}_{1}\hat{\mu}_{2}
+\hat{f}_{1}\left(1,2\right)\frac{\tilde{\mu}_{1}}{1-\hat{\mu}_{1}}
+\hat{f}_{1}\left(1,1\right)\left(\frac{1}{1-\hat{\mu}_{1}}\right)^{2}\tilde{\mu}_{1}\hat{\mu}_{2}
\end{eqnarray*}


\begin{eqnarray*}
D_{2}D_{4}\hat{F}_{1}&=&D_{2}P_{2}D_{4}\hat{P}_{2}D\hat{\theta}_{1}D_{3}
\hat{F}_{1}
+D_{2}P_{2}D_{4}\hat{P}_{2}D^{2}\hat{\theta}_{1}D_{3}\hat{F}_{1}
+D_{2}P_{2}D\hat{\theta}_{1}D_{3}D_{4}\hat{F}_{1}+
D_{2}P_{2}D_{4}\hat{P}_{2}\left(D\hat{\theta}_{1}\right)^{2}D_{3}^{2}\hat{F}_{1}\\
&=&\hat{f}_{1}\left(3\right)\frac{\tilde{\mu}_{2}\hat{\mu}_{2}}{1-\hat{\mu}_{1}}
+\hat{f}_{1}\left(3\right)\hat{\theta}_{1}^{(2)}\tilde{\mu}_{2}\hat{\mu}_{2}
+\hat{f}_{1}\left(3,4\right)\frac{\tilde{\mu}_{2}}{1-\hat{\mu}_{1}}
+\hat{f}_{1}\left(3,3\right)\left(\frac{1}{1-\hat{\mu}_{1}}\right)^{2}\tilde{\mu}_{2}\hat{\mu}_{2}
\end{eqnarray*}



\begin{eqnarray*}
D_{4}D_{4}\hat{F}_{1}&=&D_{4}D_{4}\hat{F}_{1}+D\hat{\theta}_{1}D_{4}^{2}\hat{P}_{2}D_{3}\hat{F}_{1}
+\left(D_{4}\hat{P}_{2}\right)^{2}D^{2}\hat{\theta}_{1}D_{3}\hat{F}_{1}+
D_{4}\hat{P}_{2}D\hat{\theta}_{1}D_{3}D_{4}\hat{F}_{1}\\
&+&\left(D_{4}\hat{P}_{2}\right)^{2}\left(D\hat{\theta}_{1}\right)^{2}D_{3}^{2}\hat{F}_{1}
+D_{3}D_{4}\hat{F}_{1}D\hat{\theta}_{1}D_{4}\hat{P}_{2}\\
&=&\hat{f}_{1}\left(4,4\right)
+\hat{f}_{1}\left(3\right)\frac{\hat{P}_{2}^{(2)}}{1-\hat{\mu}_{1}}
+\hat{f}_{1}\left(3\right)\hat{\theta}_{1}^{(2)}\hat{\mu}_{2}^{2}
+\hat{f}_{1}\left(3,4\right)\frac{\hat{\mu}_{2}}{1-\hat{\mu}_{1}}
+\hat{f}_{1}\left(1,1\right)\left(\frac{\hat{\mu}_{2}}{1-\hat{\mu}_{1}}\right)^{2}
+\hat{f}_{1}\left(3,4\right)\frac{\hat{\mu}_{2}}{1-\hat{\mu}_{1}}
\end{eqnarray*}




Finally for $\hat{F}_{2}\left(w_{1},\hat{\theta}_{2}\left(P_{1}\tilde{P}_{2}\hat{P}_{1}\right)\right)$

\begin{eqnarray*}
D_{j}D_{i}\hat{F}_{2}&=&\indora_{i,j\neq4}D_{4}D_{4}\hat{F}_{2}\left(D\hat{\theta}_{2}\right)^{2}D_{i}P_{i}D_{j}P_{j}
+\indora_{i,j\neq4}D_{4}\hat{F}_{2}D^{2}\hat{\theta}_{2}D_{i}P_{i}D_{j}P_{j}
+\indora_{i,j\neq4}D_{4}\hat{F}_{2}D\hat{\theta}_{2}\left(\indora_{i=j}D_{i}^{2}P_{i}+\indora_{i\neq j}D_{i}P_{i}D_{j}P_{j}\right)\\
&+&\indora_{i+j\geq5}D_{4}D_{3}\hat{F}_{2}D\hat{\theta}_{2}D_{i}P_{i}
+\indora_{i=3}\left(D_{4}D_{3}\hat{F}_{2}D\hat{\theta}_{2}D_{i}P_{i}+D_{i}^{2}\hat{F}_{2}\right)
\end{eqnarray*}



\begin{eqnarray*}
\begin{array}{llll}
D_{4}D_{1}\hat{F}_{2}=0,&
D_{4}D_{2}\hat{F}_{2}=0,&
D_{4}D_{3}\hat{F}_{2}=0,&
D_{1}D_{4}\hat{F}_{2}=0\\
D_{2}D_{4}\hat{F}_{2}=0,&
D_{3}D_{4}\hat{F}_{2}=0,&
D_{4}D_{4}\hat{F}_{2}=0,&
\end{array}
\end{eqnarray*}


\begin{eqnarray*}
D_{1}D_{1}\hat{F}_{2}&=&D\hat{\theta}_{2}D_{1}^{2}P_{1}D_{4}\hat{F}_{2}
+\left(D_{1}P_{1}\right)^{2}D^{2}\hat{\theta}_{2}D_{4}\hat{F}_{2}+
\left(D_{1}P_{1}\right)^{2}\left(D\hat{\theta}_{2}\right)^{2}D_{4}^{2}\hat{F}_{2}\\
&=&\hat{f}_{2}\left(4\right)\frac{\tilde{P}_{1}^{(2)}}{1-\tilde{\theta}_{2}}
+\hat{f}_{2}\left(4\right)\hat{\theta}_{2}^{(2)}\tilde{\mu}_{1}^{2}
+\hat{f}_{2}\left(4,4\right)\left(\frac{\tilde{\mu}_{1}}{1-\hat{\mu}_{2}}\right)^{2}
\end{eqnarray*}



\begin{eqnarray*}
D_{2}D_{1}\hat{F}_{2}&=&D_{1}P_{1}D_{2}P_{2}D\hat{\theta}_{2}D_{4}\hat{F}_{2}+
D_{1}P_{1}D_{2}P_{2}D^{2}\hat{\theta}_{2}D_{4}\hat{F}_{2}+
D_{1}P_{1}D_{2}P_{2}\left(D\hat{\theta}_{2}\right)^{2}D_{4}^{2}\hat{F}_{2}\\
&=&\hat{f}_{2}\left(4\right)\frac{\tilde{\mu}_{1}\tilde{\mu}_{2}}{1-\tilde{\mu}_{2}}
+\hat{f}_{2}\left(4\right)\hat{\theta}_{2}^{(2)}\tilde{\mu}_{1}\tilde{\mu}_{2}
+\hat{f}_{2}\left(4,4\right)\left(\frac{1}{1-\hat{\mu}_{2}}\right)^{2}\tilde{\mu}_{1}\tilde{\mu}_{2}
\end{eqnarray*}



\begin{eqnarray*}
D_{3}D_{1}\hat{F}_{2}&=&
D_{1}P_{1}D_{3}\hat{P}_{1}D\hat{\theta}_{2}D_{4}\hat{F}_{2}
+D_{1}P_{1}D_{3}\hat{P}_{1}D^{2}\hat{\theta}_{2}D_{4}\hat{F}_{2}
+D_{1}P_{1}D_{3}\hat{P}_{1}\left(D\hat{\theta}_{2}\right)^{2}D_{4}^{2}\hat{F}_{2}
+D_{1}P_{1}D\hat{\theta}_{2}D_{4}D_{3}\hat{F}_{2}\\
&=&\hat{f}_{2}\left(4\right)\frac{\tilde{\mu}_{1}\hat{\mu}_{1}}{1-\hat{\mu}_{2}}
+\hat{f}_{2}\left(4\right)\hat{\theta}_{2}^{(2)}\tilde{\mu}_{1}\hat{\mu}_{1}
+\hat{f}_{2}\left(4,4\right)\left(\frac{1}{1-\hat{\mu}_{2}}\right)^{2}\tilde{\mu}_{1}\hat{\mu}_{1}
+\hat{f}_{2}\left(4,3\right)\frac{\tilde{\mu}_{1}}{1-\hat{\mu}_{2}}
\end{eqnarray*}



\begin{eqnarray*}
D_{1}D_{2}\hat{F}_{2}&=&
D_{1}P_{1}D_{2}P_{2}D\hat{\theta}_{2}D_{4}\hat{F}_{2}+
D_{1}P_{1}D_{2}P_{2}D^{2}\hat{\theta}_{2}D_{4}\hat{F}_{2}+
D_{1}P_{1}D_{2}P_{2}\left(D\hat{\theta}_{2}\right)^{2}D_{4}D_{4}\hat{F}_{2}\\
&=&\hat{f}_{2}\left(4\right)\frac{\tilde{\mu}_{1}\tilde{\mu}_{2}}{1-\tilde{\theta}_{2}}
+\hat{f}_{2}\left(4\right)\hat{\theta}_{2}^{(2)}\tilde{\mu}_{1}\tilde{\mu}_{2}
+\hat{f}_{2}\left(4,4\right)\left(\frac{1}{1-\hat{\mu}_{2}}\right)^{2}\tilde{\mu}_{1}\tilde{\mu}_{2}
\end{eqnarray*}



\begin{eqnarray*}
D_{2}D_{2}\hat{F}_{2}&=&
D\hat{\theta}_{2}D_{2}^{2}P_{2}D_{4}\hat{F}_{2}+
\left(D_{2}P_{2}\right)^{2}D^{2}\hat{\theta}_{2}D_{4}\hat{F}_{2}+
\left(D_{2}P_{2}\right)^{2}\left(D\hat{\theta}_{2}\right)^{2}D_{4}^{2}\hat{F}_{2}\\
&=&\hat{f}_{2}\left(4\right)\frac{\tilde{P}_{2}^{(2)}}{1-\hat{\mu}_{2}}
+\hat{f}_{2}\left(4\right)\hat{\theta}_{2}^{(2)}\tilde{\mu}_{2}^{2}
+\hat{f}_{2}\left(4,4\right)\left(\frac{\tilde{\mu}_{2}}{1-\hat{\mu}_{2}}\right)^{2}
\end{eqnarray*}



\begin{eqnarray*}
D_{3}D_{2}\hat{F}_{2}&=&
D_{2}P_{2}D_{3}\hat{P}_{1}D\hat{\theta} _{2}D_{2}\hat{F}_{2}
+D_{2}P_{2}D_{3}\hat{P}_{1}D^{2}\hat{\theta}_{2}D_{2}\hat{F}_{2}
+D_{2}P_{2}D_{3}\hat{P}_{1}\left(D\hat{\theta}_{2}\right)^{2}D_{2}^{2}\hat{F}_{2}
+D_{2}P_{2}D\hat{\theta}_{2}D_{1}D_{2}\hat{F}_{2}\\
&=&\hat{f}_{2}\left(2\right)\frac{\tilde{\mu}_{2}\hat{\mu}_{1}}{1-\hat{\mu}_{2}}
+\hat{f}_{2}\left(2\right)\hat{\theta}_{2}^{(2)}\tilde{\mu}_{2}\hat{\mu}_{1}
+\hat{f}_{2}\left(2,2\right)\left(\frac{1}{1-\hat{\mu}_{2}}\right)^{2}\tilde{\mu}_{2}\hat{\mu}_{1}
+\hat{f}_{2}\left(1,2\right)\frac{\tilde{\mu}_{2}}{1-\hat{\mu}_{2}}
\end{eqnarray*}



\begin{eqnarray*}
D_{1}D_{3}\hat{F}_{2}&=&
D_{1}P_{1}D_{3}\hat{P}_{1}D\hat{\theta}_{2}D_{4}\hat{F}_{2}
+D_{1}P_{1}D_{3}\hat{P}_{1}D^{2}\hat{\theta}_{2}D_{4}\hat{F}_{2}
+D_{1}P_{1}D_{3}\hat{P}_{1}\left(D\hat{\theta}_{2}\right)^{2}D_{4}D_{4}\hat{F}_{2}
+D_{1}P_{1}D\hat{\theta}_{2}D_{4}D_{3}\hat{F}_{2}\\
&=&\hat{f}_{2}\left(4\right)\frac{\tilde{\mu}_{1}\hat{\mu}_{1}}{1-\hat{\mu}_{2}}
+\hat{f}_{2}\left(4\right)\hat{\theta}_{2}^{(2)}\tilde{\mu}_{1}\hat{\mu}_{1}
+\hat{f}_{2}\left(4,4\right)\left(\frac{1}{1-\hat{\mu}_{2}}\right)^{2}\tilde{\mu}_{1}\hat{\mu}_{1}
+\hat{f}_{2}\left(4,3\right)\frac{\tilde{\mu}_{1}}{1-\hat{\mu}_{2}}
\end{eqnarray*}



\begin{eqnarray*}
D_{2}D_{3}\hat{F}_{2}&=&
D_{2}P_{2}D_{3}\hat{P}_{1}D\hat{\theta}_{2}D_{4}\hat{F}_{2}
+D_{2}P_{2}D_{3}\hat{P}_{1}D^{2}\hat{\theta}_{2}D_{4}\hat{F}_{2}
+D_{2}P_{2}D_{3}\hat{P}_{1}\left(D\hat{\theta}_{2}\right)^{2}D_{4}^{2}\hat{F}_{2}
+D_{2}P_{2}D\hat{\theta}_{2}D_{4}D_{3}\hat{F}_{2}\\
&=&\hat{f}_{2}\left(4\right)\frac{\tilde{\mu}_{2}\hat{\mu}_{1}}{1-\hat{\mu}_{2}}
+\hat{f}_{2}\left(4\right)\hat{\theta}_{2}^{(2)}\tilde{\mu}_{2}\hat{\mu}_{1}
+\hat{f}_{2}\left(4,4\right)\left(\frac{1}{1-\hat{\mu}_{2}}\right)^{2}\tilde{\mu}_{2}\hat{\mu}_{1}
+\hat{f}_{2}\left(4,3\right)\frac{\tilde{\mu}_{2}}{1-\hat{\mu}_{2}}
\end{eqnarray*}



\begin{eqnarray*}
D_{3}D_{3}\hat{F}_{2}&=&
D_{3}^{2}\hat{P}_{1}D\hat{\theta}_{2}D_{4}\hat{F}_{2}
+\left(D_{3}\hat{P}_{1}\right)^{2}D^{2}\hat{\theta}_{2}D_{4}\hat{F}_{2}
+D_{3}\hat{P}_{1}D\hat{\theta}_{2}D_{4}D_{3}\hat{F}_{2}
+ \left(D_{3}\hat{P}_{1}\right)^{2}\left(D\hat{\theta}_{2}\right)^{2}
D_{4}^{2}\hat{F}_{2}+D_{3}^{2}\hat{F}_{2}
+D_{4}D_{3}\hat{f}_{2}D\hat{\theta}_{2}D_{3}\hat{P}_{1}\\
&=&\hat{f}_{2}\left(4\right)\frac{\hat{P}_{1}^{(2)}}{1-\hat{\mu}_{2}}
+\hat{f}_{2}\left(4\right)\hat{\theta}_{2}^{(2)}\hat{\mu}_{1}^{2}
+\hat{f}_{2}\left(4,3\right)\frac{\hat{\mu}_{1}}{1-\hat{\mu}_{2}}
+\hat{f}_{2}\left(4,4\right)\left(\frac{\hat{\mu}_{1}}{1-\hat{\mu}_{2}}\right)^{2}
+\hat{f}_{2}\left(3,3\right)
+\hat{f}_{2}\left(4,3\right)\frac{\tilde{\mu}_{1}}{1-\hat{\mu}_{2}}
\end{eqnarray*}




%_____________________________________________________________________________________
\newpage


%__________________________________________________________________
\section{Generalizaciones}
%__________________________________________________________________
\subsection{RSVC con dos conexiones}
%__________________________________________________________________

%\begin{figure}[H]
%\centering
%%%\includegraphics[width=9cm]{Grafica3.jpg}
%%\end{figure}\label{RSVC3}


Sus ecuaciones recursivas son de la forma


\begin{eqnarray*}
F_{1}\left(z_{1},z_{2},w_{1},w_{2}\right)&=&R_{2}\left(\prod_{i=1}^{2}\tilde{P}_{i}\left(z_{i}\right)\prod_{i=1}^{2}
\hat{P}_{i}\left(w_{i}\right)\right)F_{2}\left(z_{1},\tilde{\theta}_{2}\left(\tilde{P}_{1}\left(z_{1}\right)\hat{P}_{1}\left(w_{1}\right)\hat{P}_{2}\left(w_{2}\right)\right)\right)
\hat{F}_{2}\left(w_{1},w_{2};\tau_{2}\right),
\end{eqnarray*}

\begin{eqnarray*}
F_{2}\left(z_{1},z_{2},w_{1},w_{2}\right)&=&R_{1}\left(\prod_{i=1}^{2}\tilde{P}_{i}\left(z_{i}\right)\prod_{i=1}^{2}
\hat{P}_{i}\left(w_{i}\right)\right)F_{1}\left(\tilde{\theta}_{1}\left(\tilde{P}_{2}\left(z_{2}\right)\hat{P}_{1}\left(w_{1}\right)\hat{P}_{2}\left(w_{2}\right)\right),z_{2}\right)\hat{F}_{1}\left(w_{1},w_{2};\tau_{1}\right),
\end{eqnarray*}


\begin{eqnarray*}
\hat{F}_{1}\left(z_{1},z_{2},w_{1},w_{2}\right)&=&\hat{R}_{2}\left(\prod_{i=1}^{2}\tilde{P}_{i}\left(z_{i}\right)\prod_{i=1}^{2}
\hat{P}_{i}\left(w_{i}\right)\right)F_{2}\left(z_{1},z_{2};\zeta_{2}\right)\hat{F}_{2}\left(w_{1},\hat{\theta}_{2}\left(\tilde{P}_{1}\left(z_{1}\right)\tilde{P}_{2}\left(z_{2}\right)\hat{P}_{1}\left(w_{1}
\right)\right)\right),
\end{eqnarray*}


\begin{eqnarray*}
\hat{F}_{2}\left(z_{1},z_{2},w_{1},w_{2}\right)&=&\hat{R}_{1}\left(\prod_{i=1}^{2}\tilde{P}_{i}\left(z_{i}\right)\prod_{i=1}^{2}
\hat{P}_{i}\left(w_{i}\right)\right)F_{1}\left(z_{1},z_{2};\zeta_{1}\right)\hat{F}_{1}\left(\hat{\theta}_{1}\left(\tilde{P}_{1}\left(z_{1}\right)\tilde{P}_{2}\left(z_{2}\right)\hat{P}_{2}\left(w_{2}\right)\right),w_{2}\right),
\end{eqnarray*}

%_____________________________________________________
\subsection{First Moments of the Queue Lengths}
%_____________________________________________________


The server's switchover times are given by the general equation

\begin{eqnarray}\label{Ec.Ri}
R_{i}\left(\mathbf{z,w}\right)=R_{i}\left(\tilde{P}_{1}\left(z_{1}\right)\tilde{P}_{2}\left(z_{2}\right)\hat{P}_{1}\left(w_{1}\right)\hat{P}_{2}\left(w_{2}\right)\right)
\end{eqnarray}

with
\begin{eqnarray}\label{Ec.Derivada.Ri}
D_{i}R_{i}&=&DR_{i}D_{i}P_{i}
\end{eqnarray}
the following notation is considered

\begin{eqnarray*}
\begin{array}{llll}
D_{1}P_{1}\equiv D_{1}\tilde{P}_{1}, & D_{2}P_{2}\equiv D_{2}\tilde{P}_{2}, & D_{3}P_{3}\equiv D_{3}\hat{P}_{1}, &D_{4}P_{4}\equiv D_{4}\hat{P}_{2},
\end{array}
\end{eqnarray*}

also we need to remind $F_{1,2}\left(z_{1};\zeta_{2}\right)F_{2,2}\left(z_{2};\zeta_{2}\right)=F_{2}\left(z_{1},z_{2};\zeta_{2}\right)$, therefore

\begin{eqnarray*}
D_{1}F_{2}\left(z_{1},z_{2};\zeta_{2}\right)&=&D_{1}\left[F_{1,2}\left(z_{1};\zeta_{2}\right)F_{2,2}\left(z_{2};\zeta_{2}\right)\right]
=F_{2,2}\left(z_{2};\zeta_{2}\right)D_{1}F_{1,2}\left(z_{1};\zeta_{2}\right)=F_{1,2}^{(1)}\left(1\right)
\end{eqnarray*}

i.e., $D_{1}F_{2}=F_{1,2}^{(1)}(1)$; $D_{2}F_{2}=F_{2,2}^{(1)}\left(1\right)$, whereas that $D_{3}F_{2}=D_{4}F_{2}=0$, then

\begin{eqnarray}
\begin{array}{ccc}
D_{i}F_{j}=\indora_{i\leq2}F_{i,j}^{(1)}\left(1\right),& \textrm{ and } &D_{i}\hat{F}_{j}=\indora_{i\geq2}F_{i,j}^{(1)}\left(1\right).
\end{array}
\end{eqnarray}

Now, we obtain the first moments equations for the queue lengths as before for the times the server arrives to the queue to start attending



Remember that


\begin{eqnarray*}
F_{2}\left(z_{1},z_{2},w_{1},w_{2}\right)&=&R_{1}\left(\prod_{i=1}^{2}\tilde{P}_{i}\left(z_{i}\right)\prod_{i=1}^{2}
\hat{P}_{i}\left(w_{i}\right)\right)F_{1}\left(\tilde{\theta}_{1}\left(\tilde{P}_{2}\left(z_{2}\right)\hat{P}_{1}\left(w_{1}\right)\hat{P}_{2}\left(w_{2}\right)\right),z_{2}\right)\hat{F}_{1}\left(w_{1},w_{2};\tau_{1}\right),
\end{eqnarray*}

where


\begin{eqnarray*}
F_{1}\left(\tilde{\theta}_{1}\left(\tilde{P}_{2}\hat{P}_{1}\hat{P}_{2}\right),z_{2}\right)
\end{eqnarray*}

so

\begin{eqnarray}
D_{i}F_{1}&=&\indora_{i\neq1}D_{1}F_{1}D\tilde{\theta}_{1}D_{i}P_{i}+\indora_{i=2}D_{i}F_{1},
\end{eqnarray}

then


\begin{eqnarray*}
\begin{array}{ll}
D_{1}F_{1}=0,&
D_{2}F_{1}=D_{1}F_{1}D\tilde{\theta}_{1}D_{2}P_{2}+D_{2}F_{1}
=f_{1}\left(1\right)\frac{1}{1-\tilde{\mu}_{1}}\tilde{\mu}_{2}+f_{1}\left(2\right),\\
D_{3}F_{1}=D_{1}F_{1}D\tilde{\theta}_{1}D_{3}P_{3}
=f_{1}\left(1\right)\frac{1}{1-\tilde{\mu}_{1}}\hat{\mu}_{1},&
D_{4}F_{1}=D_{1}F_{1}D\tilde{\theta}_{1}D_{4}P_{4}
=f_{1}\left(1\right)\frac{1}{1-\tilde{\mu}_{1}}\hat{\mu}_{2}

\end{array}
\end{eqnarray*}


\begin{eqnarray}
D_{i}F_{2}&=&\indora_{i\neq2}D_{2}F_{2}D\tilde{\theta}_{2}D_{i}P_{i}
+\indora_{i=1}D_{i}F_{2}
\end{eqnarray}

\begin{eqnarray*}
\begin{array}{ll}
D_{1}F_{2}=D_{2}F_{2}D\tilde{\theta}_{2}D_{1}P_{1}
+D_{1}F_{2}=f_{2}\left(2\right)\frac{1}{1-\tilde{\mu}_{2}}\tilde{\mu}_{1},&
D_{2}F_{2}=0\\
D_{3}F_{2}=D_{2}F_{2}D\tilde{\theta}_{2}D_{3}P_{3}
=f_{2}\left(2\right)\frac{1}{1-\tilde{\mu}_{2}}\hat{\mu}_{1},&
D_{4}F_{2}=D_{2}F_{2}D\tilde{\theta}_{2}D_{4}P_{4}
=f_{2}\left(2\right)\frac{1}{1-\tilde{\mu}_{2}}\hat{\mu}_{2}
\end{array}
\end{eqnarray*}



\begin{eqnarray}
D_{i}\hat{F}_{1}&=&\indora_{i\neq3}D_{3}\hat{F}_{1}D\hat{\theta}_{1}D_{i}P_{i}+\indora_{i=4}D_{i}\hat{F}_{1},
\end{eqnarray}

\begin{eqnarray*}
\begin{array}{ll}
D_{1}\hat{F}_{1}=D_{3}\hat{F}_{1}D\hat{\theta}_{1}D_{1}P_{1}=\hat{f}_{1}\left(3\right)\frac{1}{1-\hat{\mu}_{1}}\tilde{\mu}_{1},&
D_{2}\hat{F}_{1}=D_{3}\hat{F}_{1}D\hat{\theta}_{1}D_{2}P_{2}
=\hat{f}_{1}\left(3\right)\frac{1}{1-\hat{\mu}_{1}}\tilde{\mu}_{2}\\
D_{3}\hat{F}_{1}=0,&
D_{4}\hat{F}_{1}=D_{3}\hat{F}_{1}D\hat{\theta}_{1}D_{4}P_{4}
+D_{4}\hat{F}_{1}
=\hat{f}_{1}\left(3\right)\frac{1}{1-\hat{\mu}_{1}}\hat{\mu}_{2}+\hat{f}_{1}\left(2\right),

\end{array}
\end{eqnarray*}


\begin{eqnarray}
D_{i}\hat{F}_{2}&=&\indora_{i\neq4}D_{4}\hat{F}_{2}D\hat{\theta}_{2}D_{i}P_{i}+\indora_{i=3}D_{i}\hat{F}_{2}.
\end{eqnarray}

\begin{eqnarray*}
\begin{array}{ll}
D_{1}\hat{F}_{2}=D_{4}\hat{F}_{2}D\hat{\theta}_{2}D_{1}P_{1}
=\hat{f}_{2}\left(4\right)\frac{1}{1-\hat{\mu}_{2}}\tilde{\mu}_{1},&
D_{2}\hat{F}_{2}=D_{4}\hat{F}_{2}D\hat{\theta}_{2}D_{2}P_{2}
=\hat{f}_{2}\left(4\right)\frac{1}{1-\hat{\mu}_{2}}\tilde{\mu}_{2},\\
D_{3}\hat{F}_{2}=D_{4}\hat{F}_{2}D\hat{\theta}_{2}D_{3}P_{3}+D_{3}\hat{F}_{2}
=\hat{f}_{2}\left(4\right)\frac{1}{1-\hat{\mu}_{2}}\hat{\mu}_{1}+\hat{f}_{2}\left(4\right)\\
D_{4}\hat{F}_{2}=0

\end{array}
\end{eqnarray*}
Then, now we can obtain the linear system of equations in order to obtain the first moments of the lengths of the queues:



For $\mathbf{F}_{1}=R_{2}F_{2}\hat{F}_{2}$ we get the general equations

\begin{eqnarray}
D_{i}\mathbf{F}_{1}=D_{i}\left(R_{2}+F_{2}+\indora_{i\geq3}\hat{F}_{2}\right)
\end{eqnarray}

So

\begin{eqnarray*}
D_{1}\mathbf{F}_{1}&=&D_{1}R_{2}+D_{1}F_{2}
=r_{1}\tilde{\mu}_{1}+f_{2}\left(2\right)\frac{1}{1-\tilde{\mu}_{2}}\tilde{\mu}_{1}\\
D_{2}\mathbf{F}_{1}&=&D_{2}\left(R_{2}+F_{2}\right)
=r_{2}\tilde{\mu}_{1}\\
D_{3}\mathbf{F}_{1}&=&D_{3}\left(R_{2}+F_{2}+\hat{F}_{2}\right)
=r_{1}\hat{\mu}_{1}+f_{2}\left(2\right)\frac{1}{1-\tilde{\mu}_{2}}\hat{\mu}_{1}+\hat{F}_{1,2}^{(1)}\left(1\right)\\
D_{4}\mathbf{F}_{1}&=&D_{4}\left(R_{2}+F_{2}+\hat{F}_{2}\right)
=r_{2}\hat{\mu}_{2}+f_{2}\left(2\right)\frac{1}{1-\tilde{\mu}_{2}}\hat{\mu}_{2}
+\hat{F}_{2,2}^{(1)}\left(1\right)
\end{eqnarray*}

it means

\begin{eqnarray*}
\begin{array}{ll}
D_{1}\mathbf{F}_{1}=r_{2}\hat{\mu}_{1}+f_{2}\left(2\right)\left(\frac{1}{1-\tilde{\mu}_{2}}\right)\tilde{\mu}_{1}+f_{2}\left(1\right),&
D_{2}\mathbf{F}_{1}=r_{2}\tilde{\mu}_{2},\\
D_{3}\mathbf{F}_{1}=r_{2}\hat{\mu}_{1}+f_{2}\left(2\right)\left(\frac{1}{1-\tilde{\mu}_{2}}\right)\hat{\mu}_{1}+\hat{F}_{1,2}^{(1)}\left(1\right),&
D_{4}\mathbf{F}_{1}=r_{2}\hat{\mu}_{2}+f_{2}\left(2\right)\left(\frac{1}{1-\tilde{\mu}_{2}}\right)\hat{\mu}_{2}+\hat{F}_{2,2}^{(1)}\left(1\right),\end{array}
\end{eqnarray*}


\begin{eqnarray}
\begin{array}{ll}
\mathbf{F}_{2}=R_{1}F_{1}\hat{F}_{1}, & D_{i}\mathbf{F}_{2}=D_{i}\left(R_{1}+F_{1}+\indora_{i\geq3}\hat{F}_{1}\right)\\
\end{array}
\end{eqnarray}



equivalently


\begin{eqnarray*}
\begin{array}{ll}
D_{1}\mathbf{F}_{2}=r_{1}\tilde{\mu}_{1},&
D_{2}\mathbf{F}_{2}=r_{1}\tilde{\mu}_{2}+f_{1}\left(1\right)\left(\frac{1}{1-\tilde{\mu}_{1}}\right)\tilde{\mu}_{2}+f_{1}\left(2\right),\\
D_{3}\mathbf{F}_{2}=r_{1}\hat{\mu}_{1}+f_{1}\left(1\right)\left(\frac{1}{1-\tilde{\mu}_{1}}\right)\hat{\mu}_{1}+\hat{F}_{1,1}^{(1)}\left(1\right),&
D_{4}\mathbf{F}_{2}=r_{1}\hat{\mu}_{2}+f_{1}\left(1\right)\left(\frac{1}{1-\tilde{\mu}_{1}}\right)\hat{\mu}_{2}+\hat{F}_{2,1}^{(1)}\left(1\right),\\
\end{array}
\end{eqnarray*}



\begin{eqnarray}
\begin{array}{ll}
\hat{\mathbf{F}}_{1}=\hat{R}_{2}\hat{F}_{2}F_{2}, & D_{i}\hat{\mathbf{F}}_{1}=D_{i}\left(\hat{R}_{2}+\hat{F}_{2}+\indora_{i\leq2}F_{2}\right)\\
\end{array}
\end{eqnarray}


equivalently


\begin{eqnarray*}
\begin{array}{ll}
D_{1}\hat{\mathbf{F}}_{1}=\hat{r}_{2}\tilde{\mu}_{1}+\hat{f}_{2}\left(2\right)\left(\frac{1}{1-\hat{\mu}_{2}}\right)\tilde{\mu}_{1}+F_{1,2}^{(1)}\left(1\right),&
D_{2}\hat{\mathbf{F}}_{1}=\hat{r}_{2}\tilde{\mu}_{2}+\hat{f}_{2}\left(2\right)\left(\frac{1}{1-\hat{\mu}_{2}}\right)\tilde{\mu}_{2}+F_{2,2}^{(1)}\left(1\right),\\
D_{3}\hat{\mathbf{F}}_{1}=\hat{r}_{2}\hat{\mu}_{1}+\hat{f}_{2}\left(2\right)\left(\frac{1}{1-\hat{\mu}_{2}}\right)\hat{\mu}_{1}+\hat{f}_{2}\left(1\right),&
D_{4}\hat{\mathbf{F}}_{1}=\hat{r}_{2}\hat{\mu}_{2}
\end{array}
\end{eqnarray*}



\begin{eqnarray}
\begin{array}{ll}
\hat{\mathbf{F}}_{2}=\hat{R}_{1}\hat{F}_{1}F_{1}, & D_{i}\hat{\mathbf{F}}_{2}=D_{i}\left(\hat{R}_{1}+\hat{F}_{1}+\indora_{i\leq2}F_{1}\right)
\end{array}
\end{eqnarray}



equivalently


\begin{eqnarray*}
\begin{array}{ll}
D_{1}\hat{\mathbf{F}}_{2}=\hat{r}_{1}\tilde{\mu}_{1}+\hat{f}_{1}\left(1\right)\left(\frac{1}{1-\hat{\mu}_{1}}\right)\tilde{\mu}_{1}+F_{1,1}^{(1)}\left(1\right),&
D_{2}\hat{\mathbf{F}}_{2}=\hat{r}_{1}\mu_{2}+\hat{f}_{1}\left(1\right)\left(\frac{1}{1-\hat{\mu}_{1}}\right)\tilde{\mu}_{2}+F_{2,1}^{(1)}\left(1\right),\\
D_{3}\hat{\mathbf{F}}_{2}=\hat{r}_{1}\hat{\mu}_{1},&
D_{4}\hat{\mathbf{F}}_{2}=\hat{r}_{1}\hat{\mu}_{2}+\hat{f}_{1}\left(1\right)\left(\frac{1}{1-\hat{\mu}_{1}}\right)\hat{\mu}_{2}+\hat{f}_{1}\left(2\right),\\
\end{array}
\end{eqnarray*}





Then we have that if $\mu=\tilde{\mu}_{1}+\tilde{\mu}_{2}$, $\hat{\mu}=\hat{\mu}_{1}+\hat{\mu}_{2}$, $r=r_{1}+r_{2}$ and $\hat{r}=\hat{r}_{1}+\hat{r}_{2}$  the system's solution is given by

\begin{eqnarray*}
\begin{array}{llll}
f_{2}\left(1\right)=r_{1}\tilde{\mu}_{1},&
f_{1}\left(2\right)=r_{2}\tilde{\mu}_{2},&
\hat{f}_{1}\left(4\right)=\hat{r}_{2}\hat{\mu}_{2},&
\hat{f}_{2}\left(3\right)=\hat{r}_{1}\hat{\mu}_{1}
\end{array}
\end{eqnarray*}



it's easy to verify that

\begin{eqnarray}\label{Sist.Ec.Lineales.Doble.Traslado}
\begin{array}{ll}
f_{1}\left(1\right)=\tilde{\mu}_{1}\left(r+\frac{f_{2}\left(2\right)}{1-\tilde{\mu}_{2}}\right),& f_{1}\left(3\right)=\hat{\mu}_{1}\left(r_{2}+\frac{f_{2}\left(2\right)}{1-\tilde{\mu}_{2}}\right)+\hat{F}_{1,2}^{(1)}\left(1\right)\\
f_{1}\left(4\right)=\hat{\mu}_{2}\left(r_{2}+\frac{f_{2}\left(2\right)}{1-\tilde{\mu}_{2}}\right)+\hat{F}_{2,2}^{(1)}\left(1\right),&
f_{2}\left(2\right)=\left(r+\frac{f_{1}\left(1\right)}{1-\mu_{1}}\right)\tilde{\mu}_{2},\\
f_{2}\left(3\right)=\hat{\mu}_{1}\left(r_{1}+\frac{f_{1}\left(1\right)}{1-\tilde{\mu}_{1}}\right)+\hat{F}_{1,1}^{(1)}\left(1\right),&
f_{2}\left(4\right)=\hat{\mu}_{2}\left(r_{1}+\frac{f_{1}\left(1\right)}{1-\mu_{1}}\right)+\hat{F}_{2,1}^{(1)}\left(1\right),\\
\hat{f}_{1}\left(1\right)=\left(\hat{r}_{2}+\frac{\hat{f}_{2}\left(4\right)}{1-\hat{\mu}_{2}}\right)\tilde{\mu}_{1}+F_{1,2}^{(1)}\left(1\right),&
\hat{f}_{1}\left(2\right)=\left(\hat{r}_{2}+\frac{\hat{f}_{2}\left(4\right)}{1-\hat{\mu}_{2}}\right)\tilde{\mu}_{2}+F_{2,2}^{(1)}\left(1\right),\\
\hat{f}_{1}\left(3\right)=\left(\hat{r}+\frac{\hat{f}_{2}\left(4\right)}{1-\hat{\mu}_{2}}\right)\hat{\mu}_{1},&
\hat{f}_{2}\left(1\right)=\left(\hat{r}_{1}+\frac{\hat{f}_{1}\left(3\right)}{1-\hat{\mu}_{1}}\right)\mu_{1}+F_{1,1}^{(1)}\left(1\right),\\
\hat{f}_{2}\left(2\right)=\left(\hat{r}_{1}+\frac{\hat{f}_{1}\left(3\right)}{1-\hat{\mu}_{1}}\right)\tilde{\mu}_{2}+F_{2,1}^{(1)}\left(1\right),&
\hat{f}_{2}\left(4\right)=\left(\hat{r}+\frac{\hat{f}_{1}\left(3\right)}{1-\hat{\mu}_{1}}\right)\hat{\mu}_{2},\\
\end{array}
\end{eqnarray}

with system's solutions given by

\begin{eqnarray}
\begin{array}{ll}
f_{1}\left(1\right)=r\frac{\mu_{1}\left(1-\mu_{1}\right)}{1-\mu},&
f_{2}\left(2\right)=r\frac{\tilde{\mu}_{2}\left(1-\tilde{\mu}_{2}\right)}{1-\mu},\\
f_{1}\left(3\right)=\hat{\mu}_{1}\left(r_{2}+\frac{r\tilde{\mu}_{2}}{1-\mu}\right)+\hat{F}_{1,2}^{(1)}\left(1\right),&
f_{1}\left(4\right)=\hat{\mu}_{2}\left(r_{2}+\frac{r\tilde{\mu}_{2}}{1-\mu}\right)+\hat{F}_{2,2}^{(1)}\left(1\right),\\
f_{2}\left(3\right)=\hat{\mu}_{1}\left(r_{1}+\frac{r\mu_{1}}{1-\mu}\right)+\hat{F}_{1,1}^{(1)}\left(1\right),&
f_{2}\left(4\right)=\hat{\mu}_{2}\left(r_{1}+\frac{r\mu_{1}}{1-\mu}\right)+\hat{F}_{2,1}^{(1)}\left(1\right),\\
\hat{f}_{1}\left(1\right)=\tilde{\mu}_{1}\left(\hat{r}_{2}+\frac{\hat{r}\hat{\mu}_{2}}{1-\hat{\mu}}\right)+F_{1,2}^{(1)}\left(1\right),&
\hat{f}_{1}\left(2\right)=\tilde{\mu}_{2}\left(\hat{r}_{2}+\frac{\hat{r}\hat{\mu}_{2}}{1-\hat{\mu}}\right)+F_{2,2}^{(1)}\left(1\right),\\
\hat{f}_{2}\left(1\right)=\tilde{\mu}_{1}\left(\hat{r}_{1}+\frac{\hat{r}\hat{\mu}_{1}}{1-\hat{\mu}}\right)+F_{1,1}^{(1)}\left(1\right),&
\hat{f}_{2}\left(2\right)=\tilde{\mu}_{2}\left(\hat{r}_{1}+\frac{\hat{r}\hat{\mu}_{1}}{1-\hat{\mu}}\right)+F_{2,1}^{(1)}\left(1\right)
\end{array}
\end{eqnarray}

%_________________________________________________________________________________________________________
\subsection{General Second Order Derivatives}
%_________________________________________________________________________________________________________


Now, taking the second order derivative with respect to the equations given in (\ref{Sist.Ec.Lineales.Doble.Traslado}) we obtain it in their general form

\small{
\begin{eqnarray*}\label{Ec.Derivadas.Segundo.Orden.Doble.Transferencia}
D_{k}D_{i}F_{1}&=&D_{k}D_{i}\left(R_{2}+F_{2}+\indora_{i\geq3}\hat{F}_{4}\right)+D_{i}R_{2}D_{k}\left(F_{2}+\indora_{k\geq3}\hat{F}_{4}\right)+D_{i}F_{2}D_{k}\left(R_{2}+\indora_{k\geq3}\hat{F}_{4}\right)+\indora_{i\geq3}D_{i}\hat{F}_{4}D_{k}\left(R_{2}+F_{2}\right)\\
D_{k}D_{i}F_{2}&=&D_{k}D_{i}\left(R_{1}+F_{1}+\indora_{i\geq3}\hat{F}_{3}\right)+D_{i}R_{1}D_{k}\left(F_{1}+\indora_{k\geq3}\hat{F}_{3}\right)+D_{i}F_{1}D_{k}\left(R_{1}+\indora_{k\geq3}\hat{F}_{3}\right)+\indora_{i\geq3}D_{i}\hat{F}_{3}D_{k}\left(R_{1}+F_{1}\right)\\
D_{k}D_{i}\hat{F}_{3}&=&D_{k}D_{i}\left(\hat{R}_{4}+\indora_{i\leq2}F_{2}+\hat{F}_{4}\right)+D_{i}\hat{R}_{4}D_{k}\left(\indora_{k\leq2}F_{2}+\hat{F}_{4}\right)+D_{i}\hat{F}_{4}D_{k}\left(\hat{R}_{4}+\indora_{k\leq2}F_{2}\right)+\indora_{i\leq2}D_{i}F_{2}D_{k}\left(\hat{R}_{4}+\hat{F}_{4}\right)\\
D_{k}D_{i}\hat{F}_{4}&=&D_{k}D_{i}\left(\hat{R}_{3}+\indora_{i\leq2}F_{1}+\hat{F}_{3}\right)+D_{i}\hat{R}_{3}D_{k}\left(\indora_{k\leq2}F_{1}+\hat{F}_{3}\right)+D_{i}\hat{F}_{3}D_{k}\left(\hat{R}_{3}+\indora_{k\leq2}F_{1}\right)+\indora_{i\leq2}D_{i}F_{1}D_{k}\left(\hat{R}_{3}+\hat{F}_{3}\right)
\end{eqnarray*}}
for $i,k=1,\ldots,4$. In order to have it in an specific way we need to compute the expressions $D_{k}D_{i}\left(R_{2}+F_{2}+\indora_{i\geq3}\hat{F}_{4}\right)$

%_________________________________________________________________________________________________________
\subsubsection{Second Order Derivatives: Serve's Switchover Times}
%_________________________________________________________________________________________________________

Remind $R_{i}\left(z_{1},z_{2},w_{1},w_{2}\right)=R_{i}\left(P_{1}\left(z_{1}\right)\tilde{P}_{2}\left(z_{2}\right)
\hat{P}_{1}\left(w_{1}\right)\hat{P}_{2}\left(w_{2}\right)\right)$,  which we will write in his reduced form $R_{i}=R_{i}\left(
P_{1}\tilde{P}_{2}\hat{P}_{1}\hat{P}_{2}\right)$, and according to the notation given in \cite{Lang} we obtain

\begin{eqnarray}
D_{i}D_{i}R_{k}=D^{2}R_{k}\left(D_{i}P_{i}\right)^{2}+DR_{k}D_{i}D_{i}P_{i}
\end{eqnarray}

whereas for $i\neq j$

\begin{eqnarray}
D_{i}D_{j}R_{k}=D^{2}R_{k}D_{i}P_{i}D_{j}P_{j}+DR_{k}D_{j}P_{j}D_{i}P_{i}
\end{eqnarray}

%_________________________________________________________________________________________________________
\subsubsection{Second Order Derivatives: Queue Lengths}
%_________________________________________________________________________________________________________

Just like before the expression $F_{1}\left(\tilde{\theta}_{1}\left(\tilde{P}_{2}\left(z_{2}\right)\hat{P}_{1}\left(w_{1}\right)\hat{P}_{2}\left(w_{2}\right)\right),
z_{2}\right)$, will be denoted by $F_{1}\left(\tilde{\theta}_{1}\left(\tilde{P}_{2}\hat{P}_{1}\hat{P}_{2}\right),z_{2}\right)$, then the mixed partial derivatives are:

\begin{eqnarray*}
D_{j}D_{i}F_{1}&=&\indora_{i,j\neq1}D_{1}D_{1}F_{1}\left(D\tilde{\theta}_{1}\right)^{2}D_{i}P_{i}D_{j}P_{j}
+\indora_{i,j\neq1}D_{1}F_{1}D^{2}\tilde{\theta}_{1}D_{i}P_{i}D_{j}P_{j}
+\indora_{i,j\neq1}D_{1}F_{1}D\tilde{\theta}_{1}\left(\indora_{i=j}D_{i}^{2}P_{i}+\indora_{i\neq j}D_{i}P_{i}D_{j}P_{j}\right)\\
&+&\left(1-\indora_{i=j=3}\right)\indora_{i+j\leq6}D_{1}D_{2}F_{1}D\tilde{\theta}_{1}\left(\indora_{i\leq j}D_{j}P_{j}+\indora_{i>j}D_{i}P_{i}\right)
+\indora_{i=2}\left(D_{1}D_{2}F_{1}D\tilde{\theta}_{1}D_{i}P_{i}+D_{i}^{2}F_{1}\right)
\end{eqnarray*}


Recall the expression for $F_{1}\left(\tilde{\theta}_{1}\left(\tilde{P}_{2}\left(z_{2}\right)\hat{P}_{1}\left(w_{1}\right)\hat{P}_{2}\left(w_{2}\right)\right),
z_{2}\right)$, which is denoted by $F_{1}\left(\tilde{\theta}_{1}\left(\tilde{P}_{2}\hat{P}_{1}\hat{P}_{2}\right),z_{2}\right)$, then the mixed partial derivatives are given by

\begin{eqnarray*}
\begin{array}{llll}
D_{1}D_{1}F_{1}=0,&
D_{2}D_{1}F_{1}=0,&
D_{3}D_{1}F_{1}=0,&
D_{4}D_{1}F_{1}=0,\\
D_{1}D_{2}F_{1}=0,&
D_{1}D_{3}F_{1}=0,&
D_{1}D_{4}F_{1}=0,&
\end{array}
\end{eqnarray*}

\begin{eqnarray*}
D_{2}D_{2}F_{1}&=&D_{1}^{2}F_{1}\left(D\tilde{\theta}_{1}\right)^{2}\left(D_{2}\tilde{P}_{2}\right)^{2}
+D_{1}F_{1}D^{2}\tilde{\theta}_{1}\left(D_{2}\tilde{P}_{2}\right)^{2}
+D_{1}F_{1}D\tilde{\theta}_{1}D_{2}^{2}\tilde{P}_{2}
+D_{1}D_{2}F_{1}D\tilde{\theta}_{1}D_{2}\tilde{P}_{2}\\
&+&D_{1}D_{2}F_{1}D\tilde{\theta}_{1}D_{2}\tilde{P}_{2}+D_{2}D_{2}F_{1}\\
&=&f_{1}\left(1,1\right)\left(\frac{\tilde{\mu}_{2}}{1-\tilde{\mu}_{1}}\right)^{2}
+f_{1}\left(1\right)\tilde{\theta}_{1}^(2)\tilde{\mu}_{2}^{(2)}
+f_{1}\left(1\right)\frac{1}{1-\tilde{\mu}_{1}}\tilde{P}_{2}^{(2)}+f_{1}\left(1,2\right)\frac{\tilde{\mu}_{2}}{1-\tilde{\mu}_{1}}+f_{1}\left(1,2\right)\frac{\tilde{\mu}_{2}}{1-\tilde{\mu}_{1}}+f_{1}\left(2,2\right)
\end{eqnarray*}

\begin{eqnarray*}
D_{3}D_{2}F_{1}&=&D_{1}^{2}F_{1}\left(D\tilde{\theta}_{1}\right)^{2}D_{3}\hat{P}_{1}D_{2}\tilde{P}_{2}+D_{1}F_{1}D^{2}\tilde{\theta}_{1}D_{3}\hat{P}_{1}D_{2}\tilde{P}_{2}+D_{1}F_{1}D\tilde{\theta}_{1}D_{2}\tilde{P}_{2}D_{3}\hat{P}_{1}+D_{1}D_{2}F_{1}D\tilde{\theta}_{1}D_{3}\hat{P}_{1}\\
&=&f_{1}\left(1,1\right)\left(\frac{1}{1-\tilde{\mu}_{1}}\right)^{2}\tilde{\mu}_{2}\hat{\mu}_{1}+f_{1}\left(1\right)\tilde{\theta}_{1}^{(2)}\tilde{\mu}_{2}\hat{\mu}_{1}+f_{1}\left(1\right)\frac{\tilde{\mu}_{2}\hat{\mu}_{1}}{1-\tilde{\mu}_{1}}+f_{1}\left(1,2\right)\frac{\hat{\mu}_{1}}{1-\tilde{\mu}_{1}}
\end{eqnarray*}

\begin{eqnarray*}
D_{4}D_{2}F_{1}&=&D_{1}^{2}F_{1}\left(D\tilde{\theta}_{1}\right)^{2}D_{4}\hat{P}_{2}D_{2}\tilde{P}_{2}+D_{1}F_{1}D^{2}\tilde{\theta}_{1}D_{4}\hat{P}_{2}D_{2}\tilde{P}_{2}+D_{1}F_{1}D\tilde{\theta}_{1}D_{2}\tilde{P}_{2}D_{4}\hat{P}_{2}+D_{1}D_{2}F_{1}D\tilde{\theta}_{1}D_{4}\hat{P}_{2}\\
&=&f_{1}\left(1,1\right)\left(\frac{1}{1-\tilde{\mu}_{1}}\right)^{2}\tilde{\mu}_{2}\hat{\mu}_{2}+f_{1}\left(1\right)\tilde{\theta}_{1}^{(2)}\tilde{\mu}_{2}\hat{\mu}_{2}+f_{1}\left(1\right)\frac{\tilde{\mu}_{2}\hat{\mu}_{2}}{1-\tilde{\mu}_{1}}+f_{1}\left(1,2\right)\frac{\hat{\mu}_{2}}{1-\tilde{\mu}_{1}}
\end{eqnarray*}

\begin{eqnarray*}
D_{2}D_{3}F_{1}&=&
D_{1}^{2}F_{1}\left(D\tilde{\theta}_{1}\right)^{2}D_{2}\tilde{P}_{2}D_{3}\hat{P}_{1}
+D_{1}F_{1}D^{2}\tilde{\theta}_{1}D_{2}\tilde{P}_{2}D_{3}\hat{P}_{1}+
D_{1}F_{1}D\tilde{\theta}_{1}D_{3}\hat{P}_{1}D_{2}\tilde{P}_{2}
+D_{1}D_{2}F_{1}D\tilde{\theta}_{1}D_{3}\hat{P}_{1}\\
&=&f_{1}\left(1,1\right)\left(\frac{1}{1-\tilde{\mu}_{1}}\right)^{2}\tilde{\mu}_{2}\hat{\mu}_{1}+f_{1}\left(1\right)\tilde{\theta}_{1}^{(2)}\tilde{\mu}_{2}\hat{\mu}_{1}+f_{1}\left(1\right)\frac{\tilde{\mu}_{2}\hat{\mu}_{1}}{1-\tilde{\mu}_{1}}+f_{1}\left(1,2\right)\frac{\hat{\mu}_{1}}{1-\tilde{\mu}_{1}}
\end{eqnarray*}

\begin{eqnarray*}
D_{3}D_{3}F_{1}&=&D_{1}^{2}F_{1}\left(D\tilde{\theta}_{1}\right)^{2}\left(D_{3}\hat{P}_{1}\right)^{2}+D_{1}F_{1}D^{2}\tilde{\theta}_{1}\left(D_{3}\hat{P}_{1}\right)^{2}+D_{1}F_{1}D\tilde{\theta}_{1}D_{3}^{2}\hat{P}_{1}\\
&=&f_{1}\left(1,1\right)\left(\frac{\hat{\mu}_{1}}{1-\tilde{\mu}_{1}}\right)^{2}+f_{1}\left(1\right)\tilde{\theta}_{1}^{(2)}\hat{\mu}_{1}^{2}+f_{1}\left(1\right)\frac{\hat{\mu}_{1}^{2}}{1-\tilde{\mu}_{1}}
\end{eqnarray*}

\begin{eqnarray*}
D_{4}D_{3}F_{1}&=&D_{1}^{2}F_{1}\left(D\tilde{\theta}_{1}\right)^{2}D_{4}\hat{P}_{2}D_{3}\hat{P}_{1}+D_{1}F_{1}D^{2}\tilde{\theta}_{1}D_{4}\hat{P}_{2}D_{3}\hat{P}_{1}+D_{1}F_{1}D\tilde{\theta}_{1}D_{3}\hat{P}_{1}D_{4}\hat{P}_{2}\\
&=&f_{1}\left(1,1\right)\left(\frac{1}{1-\tilde{\mu}_{1}}\right)^{2}\hat{\mu}_{1}\hat{\mu}_{2}
+f_{1}\left(1\right)\tilde{\theta}_{1}^{2}\hat{\mu}_{2}\hat{\mu}_{1}
+f_{1}\left(1\right)\frac{\hat{\mu}_{2}\hat{\mu}_{1}}{1-\tilde{\mu}_{1}}
\end{eqnarray*}

\begin{eqnarray*}
D_{2}D_{4}F_{1}&=&D_{1}^{2}F_{1}\left(D\tilde{\theta}_{1}\right)^{2}D_{2}\tilde{P}_{2}D_{4}\hat{P}_{2}+D_{1}F_{1}D^{2}\tilde{\theta}_{1}D_{2}\tilde{P}_{2}D_{4}\hat{P}_{2}+D_{1}F_{1}D\tilde{\theta}_{1}D_{4}\hat{P}_{2}D_{2}\tilde{P}_{2}+D_{1}D_{2}F_{1}D\tilde{\theta}_{1}D_{4}\hat{P}_{2}\\
&=&f_{1}\left(1,1\right)\left(\frac{1}{1-\tilde{\mu}_{1}}\right)^{2}\hat{\mu}_{2}\tilde{\mu}_{2}
+f_{1}\left(1\right)\tilde{\theta}_{1}^{(2)}\hat{\mu}_{2}\tilde{\mu}_{2}
+f_{1}\left(1\right)\frac{\hat{\mu}_{2}\tilde{\mu}_{2}}{1-\tilde{\mu}_{1}}+f_{1}\left(1,2\right)\frac{\hat{\mu}_{2}}{1-\tilde{\mu}_{1}}
\end{eqnarray*}

\begin{eqnarray*}
D_{3}D_{4}F_{1}&=&D_{1}^{2}F_{1}\left(D\tilde{\theta}_{1}\right)^{2}D_{3}\hat{P}_{1}D_{4}\hat{P}_{2}+D_{1}F_{1}D^{2}\tilde{\theta}_{1}D_{3}\hat{P}_{1}D_{4}\hat{P}_{2}+D_{1}F_{1}D\tilde{\theta}_{1}D_{4}\hat{P}_{2}D_{3}\hat{P}_{1}\\
&=&f_{1}\left(1,1\right)\left(\frac{1}{1-\tilde{\mu}_{1}}\right)^{2}\hat{\mu}_{1}\hat{\mu}_{2}+f_{1}\left(1\right)\tilde{\theta}_{1}^{(2)}\hat{\mu}_{1}\hat{\mu}_{2}+f_{1}\left(1\right)\frac{\hat{\mu}_{1}\hat{\mu}_{2}}{1-\tilde{\mu}_{1}}
\end{eqnarray*}

\begin{eqnarray*}
D_{4}D_{4}F_{1}&=&D_{1}^{2}F_{1}\left(D\tilde{\theta}_{1}\right)^{2}\left(D_{4}\hat{P}_{2}\right)^{2}+D_{1}F_{1}D^{2}\tilde{\theta}_{1}\left(D_{4}\hat{P}_{2}\right)^{2}+D_{1}F_{1}D\tilde{\theta}_{1}D_{4}^{2}\hat{P}_{2}\\
&=&f_{1}\left(1,1\right)\left(\frac{\hat{\mu}_{2}}{1-\tilde{\mu}_{1}}\right)^{2}+f_{1}\left(1\right)\tilde{\theta}_{1}^{(2)}\hat{\mu}_{2}^{2}+f_{1}\left(1\right)\frac{1}{1-\tilde{\mu}_{1}}\hat{P}_{2}^{(2)}
\end{eqnarray*}



Meanwhile for  $F_{2}\left(z_{1},\tilde{\theta}_{2}\left(P_{1}\hat{P}_{1}\hat{P}_{2}\right)\right)$

\begin{eqnarray*}
D_{j}D_{i}F_{2}&=&\indora_{i,j\neq2}D_{2}D_{2}F_{2}\left(D\theta_{2}\right)^{2}D_{i}P_{i}D_{j}P_{j}+\indora_{i,j\neq2}D_{2}F_{2}D^{2}\theta_{2}D_{i}P_{i}D_{j}P_{j}\\
&+&\indora_{i,j\neq2}D_{2}F_{2}D\theta_{2}\left(\indora_{i=j}D_{i}^{2}P_{i}
+\indora_{i\neq j}D_{i}P_{i}D_{j}P_{j}\right)\\
&+&\left(1-\indora_{i=j=3}\right)\indora_{i+j\leq6}D_{2}D_{1}F_{2}D\theta_{2}\left(\indora_{i\leq j}D_{j}P_{j}+\indora_{i>j}D_{i}P_{i}\right)
+\indora_{i=1}\left(D_{2}D_{1}F_{2}D\theta_{2}D_{i}P_{i}+D_{i}^{2}F_{2}\right)
\end{eqnarray*}

\begin{eqnarray*}
\begin{array}{llll}
D_{2}D_{1}F_{2}=0,&
D_{2}D_{3}F_{3}=0,&
D_{2}D_{4}F_{2}=0,&\\
D_{1}D_{2}F_{2}=0,&
D_{2}D_{2}F_{2}=0,&
D_{3}D_{2}F_{2}=0,&
D_{4}D_{2}F_{2}=0\\
\end{array}
\end{eqnarray*}


\begin{eqnarray*}
D_{1}D_{1}F_{2}&=&
\left(D_{1}P_{1}\right)^{2}\left(D\tilde{\theta}_{2}\right)^{2}D_{2}^{2}F_{2}
+\left(D_{1}P_{1}\right)^{2}D^{2}\tilde{\theta}_{2}D_{2}F_{2}
+D_{1}^{2}P_{1}D\tilde{\theta}_{2}D_{2}F_{2}
+D_{1}P_{1}D\tilde{\theta}_{2}D_{2}D_{1}F_{2}\\
&+&D_{2}D_{1}F_{2}D\tilde{\theta}_{2}D_{1}P_{1}+
D_{1}^{2}F_{2}\\
&=&f_{2}\left(2\right)\frac{\tilde{P}_{1}^{(2)}}{1-\tilde{\mu}_{2}}
+f_{2}\left(2\right)\theta_{2}^{(2)}\tilde{\mu}_{1}^{2}
+f_{2}\left(2,1\right)\frac{\tilde{\mu}_{1}}{1-\tilde{\mu}_{2}}
+\left(\frac{\tilde{\mu}_{1}}{1-\tilde{\mu}_{2}}\right)^{2}f_{2}\left(2,2\right)
+\frac{\tilde{\mu}_{1}}{1-\tilde{\mu}_{2}}f_{2}\left(2,1\right)+f_{2}\left(1,1\right)
\end{eqnarray*}


\begin{eqnarray*}
D_{3}D_{1}F_{2}&=&D_{2}D_{1}F_{2}D\tilde{\theta}_{2}D_{3}\hat{P}_{1}
+D_{2}^{2}F_{2}\left(D\tilde{\theta}_{2}\right)^{2}D_{3}P_{1}D_{1}P_{1}
+D_{2}F_{2}D^{2}\tilde{\theta}_{2}D_{3}\hat{P}_{1}D_{1}P_{1}
+D_{2}F_{2}D\tilde{\theta}_{2}D_{1}P_{1}D_{3}\hat{P}_{1}\\
&=&f_{2}\left(2,1\right)\frac{\hat{\mu}_{1}}{1-\tilde{\mu}_{2}}
+f_{2}\left(2,2\right)\left(\frac{1}{1-\tilde{\mu}_{2}}\right)^{2}\tilde{\mu}_{1}\hat{\mu}_{1}
+f_{2}\left(2\right)\tilde{\theta}_{2}^{(2)}\tilde{\mu}_{1}\hat{\mu}_{1}
+f_{2}\left(2\right)\frac{\tilde{\mu}_{1}\hat{\mu}_{1}}{1-\tilde{\mu}_{2}}
\end{eqnarray*}


\begin{eqnarray*}
D_{4}D_{1}F_{2}&=&D_{2}^{2}F_{2}\left(D\tilde{\theta}_{2}\right)^{2}D_{4}P_{2}D_{1}P_{1}+D_{2}F_{2}D^{2}\tilde{\theta}_{2}D_{4}\hat{P}_{2}D_{1}P_{1}
+D_{2}F_{2}D\tilde{\theta}_{2}D_{1}P_{1}D_{4}\hat{P}_{2}+D_{2}D_{1}F_{2}D\tilde{\theta}_{2}D_{4}\hat{P}_{2}\\
&=&f_{2}\left(2,2\right)\left(\frac{1}{1-\tilde{\mu}_{2}}\right)^{2}\tilde{\mu}_{1}\hat{\mu}_{2}
+f_{2}\left(2\right)\tilde{\theta}_{2}^{(2)}\tilde{\mu}_{1}\hat{\mu}_{2}
+f_{2}\left(2\right)\frac{\tilde{\mu}_{1}\hat{\mu}_{2}}{1-\tilde{\mu}_{2}}
+f_{2}\left(2,1\right)\frac{\hat{\mu}_{2}}{1-\tilde{\mu}_{2}}
\end{eqnarray*}


\begin{eqnarray*}
D_{1}D_{3}F_{2}&=&D_{2}^{2}F_{2}\left(D\tilde{\theta}_{2}\right)^{2}D_{1}P_{1}D_{3}\hat{P}_{1}
+D_{2}F_{2}D^{2}\tilde{\theta}_{2}D_{1}P_{1}D_{3}\hat{P}_{1}
+D_{2}F_{2}D\tilde{\theta}_{2}D_{3}\hat{P}_{1}D_{1}P_{1}
+D_{2}D_{1}F_{2}D\tilde{\theta}_{2}D_{3}\hat{P}_{1}\\
&=&f_{2}\left(2,2\right)\left(\frac{1}{1-\tilde{\mu}_{2}}\right)^{2}\tilde{\mu}_{1}\hat{\mu}_{1}
+f_{2}\left(2\right)\tilde{\theta}_{2}^{(2)}\tilde{\mu}_{1}\hat{\mu}_{1}
+f_{2}\left(2\right)\frac{\tilde{\mu}_{1}\hat{\mu}_{1}}{1-\tilde{\mu}_{2}}
+f_{2}\left(2,1\right)\frac{\hat{\mu}_{1}}{1-\tilde{\mu}_{2}}
\end{eqnarray*}


\begin{eqnarray*}
D_{3}D_{3}F_{2}&=&D_{2}^{2}F_{2}\left(D\tilde{\theta}_{2}\right)^{2}\left(D_{3}\hat{P}_{1}\right)^{2}
+D_{2}F_{2}\left(D_{3}\hat{P}_{1}\right)^{2}D^{2}\tilde{\theta}_{2}
+D_{2}F_{2}D\tilde{\theta}_{2}D_{3}^{2}\hat{P}_{1}\\
&=&f_{2}\left(2,2\right)\left(\frac{1}{1-\tilde{\mu}_{2}}\right)^{2}\hat{\mu}_{1}^{2}
+f_{2}\left(2\right)\tilde{\theta}_{2}^{(2)}\hat{\mu}_{1}^{2}
+f_{2}\left(2\right)\frac{\hat{P}_{1}^{(2)}}{1-\tilde{\mu}_{2}}
\end{eqnarray*}


\begin{eqnarray*}
D_{4}D_{3}F_{2}&=&D_{2}^{2}F_{2}\left(D\tilde{\theta}_{2}\right)^{2}D_{4}\hat{P}_{2}D_{3}\hat{P}_{1}
+D_{2}F_{2}D^{2}\tilde{\theta}_{2}D_{4}\hat{P}_{2}D_{3}\hat{P}_{1}
+D_{2}F_{2}D\tilde{\theta}_{2}D_{3}\hat{P}_{1}D_{4}\hat{P}_{2}\\
&=&f_{2}\left(2,2\right)\left(\frac{1}{1-\tilde{\mu}_{2}}\right)^{2}\hat{\mu}_{1}\hat{\mu}_{2}
+f_{2}\left(2\right)\tilde{\theta}_{2}^{(2)}\hat{\mu}_{1}\hat{\mu}_{2}
+f_{2}\left(2\right)\frac{\hat{\mu}_{1}\hat{\mu}_{2}}{1-\tilde{\mu}_{2}}
\end{eqnarray*}


\begin{eqnarray*}
D_{1}D_{4}F_{2}&=&D_{2}^{2}F_{2}\left(D\tilde{\theta}_{2}\right)^{2}D_{1}P_{1}D_{4}\hat{P}_{2}
+D_{2}F_{2}D^{2}\tilde{\theta}_{2}D_{1}P_{1}D_{4}\hat{P}_{2}
+D_{2}F_{2}D\tilde{\theta}_{2}D_{4}\hat{P}_{2}D_{1}P_{1}
+D_{2}D_{1}F_{2}D\tilde{\theta}_{2}D_{4}\hat{P}_{2}\\
&=&f_{2}\left(2,2\right)\left(\frac{1}{1-\tilde{\mu}_{2}}\right)^{2}\tilde{\mu}_{1}\hat{\mu}_{2}
+f_{2}\left(2\right)\tilde{\theta}_{2}^{(2)}\tilde{\mu}_{1}\hat{\mu}_{2}
+f_{2}\left(2\right)\frac{\tilde{\mu}_{1}\hat{\mu}_{2}}{1-\tilde{\mu}_{2}}
+f_{2}\left(2,1\right)\frac{\hat{\mu}_{2}}{1-\tilde{\mu}_{2}}
\end{eqnarray*}


\begin{eqnarray*}
D_{3}D_{4}F_{2}&=&
D_{2}^{2}F_{2}\left(D\tilde{\theta}_{2}\right)^{2}D_{4}\hat{P}_{2}D_{3}\hat{P}_{1}
+D_{2}F_{2}D^{2}\tilde{\theta}_{2}D_{4}\hat{P}_{2}D_{3}\hat{P}_{1}
+D_{2}F_{2}D\tilde{\theta}_{2}D_{4}\hat{P}_{2}D_{3}\hat{P}_{1}\\
&=&f_{2}\left(2,2\right)\left(\frac{1}{1-\tilde{\mu}_{2}}\right)^{2}\hat{\mu}_{1}\hat{\mu}_{2}
+f_{2}\left(2\right)\tilde{\theta}_{2}^{(2)}\hat{\mu}_{1}\hat{\mu}_{2}
+f_{2}\left(2\right)\frac{\hat{\mu}_{1}\hat{\mu}_{2}}{1-\tilde{\mu}_{2}}
\end{eqnarray*}


\begin{eqnarray*}
D_{4}D_{4}F_{2}&=&D_{2}F_{2}D\tilde{\theta}_{2}D_{4}^{2}\hat{P}_{2}
+D_{2}F_{2}D^{2}\tilde{\theta}_{2}\left(D_{4}\hat{P}_{2}\right)^{2}
+D_{2}^{2}F_{2}\left(D\tilde{\theta}_{2}\right)^{2}\left(D_{4}\hat{P}_{2}\right)^{2}\\
&=&f_{2}\left(2,2\right)\left(\frac{\hat{\mu}_{2}}{1-\tilde{\mu}_{2}}\right)^{2}
+f_{2}\left(2\right)\tilde{\theta}_{2}^{(2)}\hat{\mu}_{2}^{2}
+f_{2}\left(2\right)\frac{\hat{P}_{2}^{(2)}}{1-\tilde{\mu}_{2}}
\end{eqnarray*}


%\newpage



%\newpage

For $\hat{F}_{1}\left(\hat{\theta}_{1}\left(P_{1}\tilde{P}_{2}\hat{P}_{2}\right),w_{2}\right)$



\begin{eqnarray*}
D_{j}D_{i}\hat{F}_{1}&=&\indora_{i,j\neq3}D_{3}D_{3}\hat{F}_{1}\left(D\hat{\theta}_{1}\right)^{2}D_{i}P_{i}D_{j}P_{j}
+\indora_{i,j\neq3}D_{3}\hat{F}_{1}D^{2}\hat{\theta}_{1}D_{i}P_{i}D_{j}P_{j}
+\indora_{i,j\neq3}D_{3}\hat{F}_{1}D\hat{\theta}_{1}\left(\indora_{i=j}D_{i}^{2}P_{i}+\indora_{i\neq j}D_{i}P_{i}D_{j}P_{j}\right)\\
&+&\indora_{i+j\geq5}D_{3}D_{4}\hat{F}_{1}D\hat{\theta}_{1}\left(\indora_{i\leq j}D_{i}P_{i}+\indora_{i>j}D_{j}P_{j}\right)
+\indora_{i=4}\left(D_{3}D_{4}\hat{F}_{1}D\hat{\theta}_{1}D_{i}P_{i}+D_{i}^{2}\hat{F}_{1}\right)
\end{eqnarray*}


\begin{eqnarray*}
\begin{array}{llll}
D_{3}D_{1}\hat{F}_{1}=0,&
D_{3}D_{2}\hat{F}_{1}=0,&
D_{1}D_{3}\hat{F}_{1}=0,&
D_{2}D_{3}\hat{F}_{1}=0\\
D_{3}D_{3}\hat{F}_{1}=0,&
D_{4}D_{3}\hat{F}_{1}=0,&
D_{3}D_{4}\hat{F}_{1}=0,&
\end{array}
\end{eqnarray*}


\begin{eqnarray*}
D_{1}D_{1}\hat{F}_{1}&=&
D_{3}^{2}\hat{F}_{1}\left(D\hat{\theta}_{1}\right)^{2}\left(D_{1}P_{1}\right)^{2}
+D_{3}\hat{F}_{1}D^{2}\hat{\theta}_{1}\left(D_{1}P_{1}\right)^{2}
+D_{3}\hat{F}_{1}D\hat{\theta}_{1}D_{1}^{2}P_{1}\\
&=&\hat{f}_{1}\left(3,3\right)\left(\frac{\tilde{\mu}_{1}}{1-\hat{\mu}_{2}}\right)^{2}
+\hat{f}_{1}\left(3\right)\frac{P_{1}^{(2)}}{1-\hat{\mu}_{1}}
+\hat{f}_{1}\left(3\right)\hat{\theta}_{1}^{(2)}\tilde{\mu}_{1}^{2}
\end{eqnarray*}


\begin{eqnarray*}
D_{2}D_{1}\hat{F}_{1}&=&
D_{3}^{2}\hat{F}_{1}\left(D\hat{\theta}_{1}\right)^{2}D_{1}P_{1}D_{2}P_{1}+
D_{3}\hat{F}_{1}D^{2}\hat{\theta}_{1}D_{1}P_{1}D_{2}P_{2}+
D_{3}\hat{F}_{1}D\hat{\theta}_{1}D_{1}P_{1}D_{2}P_{2}\\
&=&\hat{f}_{1}\left(3,3\right)\left(\frac{1}{1-\hat{\mu}_{1}}\right)^{2}\tilde{\mu}_{1}\tilde{\mu}_{2}
+\hat{f}_{1}\left(3\right)\tilde{\mu}_{1}\tilde{\mu}_{2}\hat{\theta}_{1}^{(2)}
+\hat{f}_{1}\left(3\right)\frac{\tilde{\mu}_{1}\tilde{\mu}_{2}}{1-\hat{\mu}_{1}}
\end{eqnarray*}


\begin{eqnarray*}
D_{4}D_{1}\hat{F}_{1}&=&
D_{3}D_{3}\hat{F}_{1}\left(D\hat{\theta}_{1}\right)^{2}D_{4}\hat{P}_{2}D_{1}P_{1}
+D_{3}\hat{F}_{1}D^{2}\hat{\theta}_{1}D_{1}P_{1}D_{4}\hat{P}_{2}
+D_{3}\hat{F}_{1}D\hat{\theta}_{1}D_{1}P_{1}D_{4}\hat{P}_{2}
+D_{3}D_{4}\hat{F}_{1}D\hat{\theta}_{1}D_{1}P_{1}\\
&=&\hat{f}_{1}\left(3,3\right)\left(\frac{1}{1-\hat{\mu}_{1}}\right)^{2}\tilde{\mu}_{1}\hat{\mu}_{1}
+\hat{f}_{1}\left(3\right)\hat{\theta}_{1}^{(2)}\tilde{\mu}_{1}\hat{\mu}_{2}
+\hat{f}_{1}\left(3\right)\frac{\tilde{\mu}_{1}\hat{\mu}_{2}}{1-\hat{\mu}_{1}}
+\hat{f}_{1}\left(3,4\right)\frac{\tilde{\mu}_{1}}{1-\hat{\mu}_{1}}
\end{eqnarray*}


\begin{eqnarray*}
D_{1}D_{2}\hat{F}_{1}&=&
D_{3}^{2}\hat{F}_{1}\left(D\hat{\theta}_{1}\right)^{2}D_{1}P_{1}D_{2}P_{2}
+D_{3}\hat{F}_{1}D^{2}\hat{\theta}_{1}D_{1}P_{1}D_{2}P_{2}+
D_{3}\hat{F}_{1}D\hat{\theta}_{1}D_{1}P_{1}D_{2}P_{2}\\
&=&\hat{f}_{1}\left(3,3\right)\left(\frac{1}{1-\hat{\mu}_{1}}\right)^{2}\tilde{\mu}_{1}\tilde{\mu}_{2}
+\hat{f}_{1}\left(3\right)\hat{\theta}_{1}^{(2)}\tilde{\mu}_{1}\tilde{\mu}_{2}
+\hat{f}_{1}\left(3\right)\frac{\tilde{\mu}_{1}\tilde{\mu}_{2}}{1-\hat{\mu}_{1}}
\end{eqnarray*}


\begin{eqnarray*}
D_{2}D_{2}\hat{F}_{1}&=&
D_{3}^{2}\hat{F}_{1}\left(D\hat{\theta}_{1}\right)^{2}\left(D_{2}P_{2}\right)^{2}
+D_{3}\hat{F}_{1}D^{2}\hat{\theta}_{1}\left(D_{2}P_{2}\right)^{2}+
D_{3}\hat{F}_{1}D\hat{\theta}_{1}D_{2}^{2}P_{2}\\
&=&\hat{f}_{1}\left(3,3\right)\left(\frac{\tilde{\mu}_{2}}{1-\hat{\mu}_{1}}\right)^{2}
+\hat{f}_{1}\left(3\right)\hat{\theta}_{1}^{(2)}\tilde{\mu}_{2}^{2}
+\hat{f}_{1}\left(3\right)\tilde{P}_{2}^{(2)}\frac{1}{1-\hat{\mu}_{1}}
\end{eqnarray*}


\begin{eqnarray*}
D_{4}D_{2}\hat{F}_{1}&=&
D_{3}^{2}\hat{F}_{1}\left(D\hat{\theta}_{1}\right)^{2}D_{4}\hat{P}_{2}D_{2}P_{2}
+D_{3}\hat{F}_{1}D^{2}\hat{\theta}_{1}D_{2}P_{2}D_{4}\hat{P}_{2}
+D_{3}\hat{F}_{1}D\hat{\theta}_{1}D_{2}P_{2}D_{4}\hat{P}_{2}
+D_{3}D_{4}\hat{F}_{1}D\hat{\theta}_{1}D_{2}P_{2}\\
&=&\hat{f}_{1}\left(3,3\right)\left(\frac{1}{1-\hat{\mu}_{1}}\right)^{2}\tilde{\mu}_{2}\hat{\mu}_{2}
+\hat{f}_{1}\left(3\right)\hat{\theta}_{1}^{(2)}\tilde{\mu}_{2}\hat{\mu}_{2}
+\hat{f}_{1}\left(3\right)\frac{\tilde{\mu}_{2}\hat{\mu}_{2}}{1-\hat{\mu}_{1}}
+\hat{f}_{1}\left(3,4\right)\frac{\tilde{\mu}_{2}}{1-\hat{\mu}_{1}}
\end{eqnarray*}



\begin{eqnarray*}
D_{1}D_{4}\hat{F}_{1}&=&
D_{3}D_{3}\hat{F}_{1}\left(D\hat{\theta}_{1}\right)^{2}D_{1}P_{1}D_{4}\hat{P}_{2}
+D_{3}\hat{F}_{1}D^{2}\hat{\theta}_{1}D_{1}P_{1}D_{4}\hat{P}_{2}
+D_{3}\hat{F}_{1}D\hat{\theta}_{1}D_{1}P_{1}D_{4}\hat{P}_{2}
+D_{3}D_{4}\hat{F}_{1}D\hat{\theta}_{1}D_{1}P_{1}\\
&=&\hat{f}_{1}\left(3,3\right)\left(\frac{1}{1-\hat{\mu}_{1}}\right)^{2}\tilde{\mu}_{1}\hat{\mu}_{2}
+\hat{f}_{1}\left(3\right)\hat{\theta}_{1}^{(2)}\tilde{\mu}_{1}\hat{\mu}_{2}
+\hat{f}_{1}\left(3\right)\frac{\tilde{\mu}_{1}\hat{\mu}_{2}}{1-\hat{\mu}_{1}}
+\hat{f}_{1}\left(3,4\right)\frac{\tilde{\mu}_{1}}{1-\hat{\mu}_{1}}
\end{eqnarray*}


\begin{eqnarray*}
D_{2}D_{4}\hat{F}_{1}&=&
D_{3}^{2}\hat{F}_{1}\left(D\hat{\theta}_{1}\right)^{2}D_{2}P_{2}D_{4}\hat{P}_{2}
+D_{3}\hat{F}_{1}D^{2}\hat{\theta}_{1}D_{2}P_{2}D_{4}\hat{P}_{2}
+D_{3}\hat{F}_{1}D\hat{\theta}_{1}D_{2}P_{2}D_{4}\hat{P}_{2}
+D_{3}D_{4}\hat{F}_{1}D\hat{\theta}_{1}D_{2}P_{2}\\
&=&\hat{f}_{1}\left(3,3\right)\left(\frac{1}{1-\hat{\mu}_{1}}\right)^{2}\tilde{\mu}_{2}\hat{\mu}_{2}
+\hat{f}_{1}\left(3\right)\hat{\theta}_{1}^{(2)}\tilde{\mu}_{2}\hat{\mu}_{2}
+\hat{f}_{1}\left(3\right)\frac{\tilde{\mu}_{2}\hat{\mu}_{2}}{1-\hat{\mu}_{1}}
+\hat{f}_{1}\left(3,4\right)\frac{\tilde{\mu}_{2}}{1-\hat{\mu}_{1}}
\end{eqnarray*}



\begin{eqnarray*}
D_{4}D_{4}\hat{F}_{1}&=&
D_{3}^{2}\hat{F}_{1}\left(D\hat{\theta}_{1}\right)^{2}\left(D_{4}\hat{P}_{2}\right)^{2}
+D_{3}\hat{F}_{1}D^{2}\hat{\theta}_{1}\left(D_{4}\hat{P}_{2}\right)^{2}
+D_{3}\hat{F}_{1}D\hat{\theta}_{1}D_{4}^{2}\hat{P}_{2}
+D_{3}D_{4}\hat{F}_{1}D\hat{\theta}_{1}D_{4}\hat{P}_{2}\\
&+&D_{3}D_{4}\hat{F}_{1}D\hat{\theta}_{1}D_{4}\hat{P}_{2}
+D_{4}D_{4}\hat{F}_{1}\\
&=&\hat{f}_{1}\left(3,3\right)\left(\frac{\hat{\mu}_{2}}{1-\hat{\mu}_{1}}\right)^{2}
+\hat{f}_{1}\left(3\right)\hat{\theta}_{1}^{(2)}\hat{\mu}_{2}^{2}
+\hat{f}_{1}\left(3\right)\frac{\hat{P}_{2}^{(2)}}{1-\hat{\mu}_{1}}
+\hat{f}_{1}\left(3,4\right)\frac{\hat{\mu}_{2}}{1-\hat{\mu}_{1}}
+\hat{f}_{1}\left(3,4\right)\frac{\hat{\mu}_{2}}{1-\hat{\mu}_{1}}
+\hat{f}_{1}\left(4,4\right)
\end{eqnarray*}




Finally for $\hat{F}_{2}\left(w_{1},\hat{\theta}_{2}\left(P_{1}\tilde{P}_{2}\hat{P}_{1}\right)\right)$

\begin{eqnarray*}
D_{j}D_{i}\hat{F}_{2}&=&\indora_{i,j\neq4}D_{4}D_{4}\hat{F}_{2}\left(D\hat{\theta}_{2}\right)^{2}D_{i}P_{i}D_{j}P_{j}
+\indora_{i,j\neq4}D_{4}\hat{F}_{2}D^{2}\hat{\theta}_{2}D_{i}P_{i}D_{j}P_{j}
+\indora_{i,j\neq4}D_{4}\hat{F}_{2}D\hat{\theta}_{2}\left(\indora_{i=j}D_{i}^{2}P_{i}+\indora_{i\neq j}D_{i}P_{i}D_{j}P_{j}\right)\\
&+&\left(1-\indora_{i=j=2}\right)\indora_{i+j\geq4}D_{4}D_{3}\hat{F}_{2}D\hat{\theta}_{2}\left(\indora_{i\leq j}D_{i}P_{i}+\indora_{i>j}D_{j}P_{j}\right)
+\indora_{i=3}\left(D_{4}D_{3}\hat{F}_{2}D\hat{\theta}_{2}D_{i}P_{i}+D_{i}^{2}\hat{F}_{2}\right)
\end{eqnarray*}



\begin{eqnarray*}
\begin{array}{llll}
D_{4}D_{1}\hat{F}_{2}=0,&
D_{4}D_{2}\hat{F}_{2}=0,&
D_{4}D_{3}\hat{F}_{2}=0,&
D_{1}D_{4}\hat{F}_{2}=0\\
D_{2}D_{4}\hat{F}_{2}=0,&
D_{3}D_{4}\hat{F}_{2}=0,&
D_{4}D_{4}\hat{F}_{2}=0,&
\end{array}
\end{eqnarray*}


\begin{eqnarray*}
D_{1}D_{1}\hat{F}_{2}&=&
D_{4}^{2}\hat{F}_{2}\left(D\hat{\theta}_{2}\right)^{2}\left(D_{1}P_{1}\right)^{2}
+D_{4}\hat{F}_{2}\hat{\theta}_{2}\left(D_{1}P_{1}\right)^{2}D^{2}+
D_{4}\hat{F}_{2}D\hat{\theta}_{2}D_{1}^{2}P_{1}\\
&=&\hat{f}_{2}\left(4,4\right)\left(\frac{\tilde{\mu}_{1}}{1-\hat{\mu}_{2}}\right)^{2}
+\hat{f}_{2}\left(4\right)\hat{\theta}_{2}^{(2)}\tilde{\mu}_{1}^{2}
+\hat{f}_{2}\left(4\right)\frac{\tilde{P}_{1}^{(2)}}{1-\tilde{\mu}_{2}}
\end{eqnarray*}



\begin{eqnarray*}
D_{2}D_{1}\hat{F}_{2}&=&
D_{4}^{2}\hat{F}_{2}\left(D\hat{\theta}_{2}\right)^{2}D_{1}P_{1}D_{2}P_{2}
+D_{4}\hat{F}_{2}D^{2}\hat{\theta}_{2}D_{1}P_{1}D_{2}P_{2}
+D_{4}\hat{F}_{2}D\hat{\theta}_{2}D_{1}P_{1}D_{2}P_{2}\\
&=&\hat{f}_{2}\left(4,4\right)\left(\frac{1}{1-\hat{\mu}_{2}}\right)^{2}\tilde{\mu}_{1}\tilde{\mu}_{2}
+\hat{f}_{2}\left(4\right)\hat{\theta}_{2}^{(2)}\tilde{\mu}_{1}\tilde{\mu}_{2}
+\hat{f}_{2}\left(4\right)\frac{\tilde{\mu}_{1}\tilde{\mu}_{2}}{1-\tilde{\mu}_{2}}
\end{eqnarray*}



\begin{eqnarray*}
D_{3}D_{1}\hat{F}_{2}&=&
D_{4}^{2}\hat{F}_{2}\left(D\hat{\theta}_{2}\right)^{2}D_{1}P_{1}D_{3}\hat{P}_{1}
+D_{4}\hat{F}_{2}D^{2}\hat{\theta}_{2}D_{1}P_{1}D_{3}\hat{P}_{1}
+D_{4}\hat{F}_{2}D\hat{\theta}_{2}D_{1}P_{1}D_{3}\hat{P}_{1}
+D_{4}D_{3}\hat{F}_{2}D\hat{\theta}_{2}D_{1}P_{1}\\
&=&\hat{f}_{2}\left(4,4\right)\left(\frac{1}{1-\hat{\mu}_{2}}\right)^{2}\tilde{\mu}_{1}\hat{\mu}_{1}
+\hat{f}_{2}\left(4\right)\hat{\theta}_{2}^{(2)}\tilde{\mu}_{1}\hat{\mu}_{1}
+\hat{f}_{2}\left(4\right)\frac{\tilde{\mu}_{1}\hat{\mu}_{1}}{1-\hat{\mu}_{2}}
+\hat{f}_{2}\left(4,3\right)\frac{\tilde{\mu}_{1}}{1-\hat{\mu}_{2}}
\end{eqnarray*}



\begin{eqnarray*}
D_{1}D_{2}\hat{F}_{2}&=&
D_{4}D_{4}\hat{F}_{2}\left(D\hat{\theta}_{2}\right)^{2}D_{1}P_{1}D_{2}P_{2}
+D_{4}\hat{F}_{2}D^{2}\hat{\theta}_{2}D_{1}P_{1}D_{2}P_{2}
+D_{4}\hat{F}_{2}D\hat{\theta}_{2}D_{1}P_{1}D_{2}P_{2}
\\
&=&
\hat{f}_{2}\left(4,4\right)\left(\frac{1}{1-\hat{\mu}_{2}}\right)^{2}\tilde{\mu}_{1}\tilde{\mu}_{2}
+\hat{f}_{2}\left(4\right)\hat{\theta}_{2}^{(2)}\tilde{\mu}_{1}\tilde{\mu}_{2}
+\hat{f}_{2}\left(4\right)\frac{\tilde{\mu}_{1}\tilde{\mu}_{2}}{1-\tilde{\mu}_{2}}
\end{eqnarray*}



\begin{eqnarray*}
D_{2}D_{2}\hat{F}_{2}&=&
D_{4}^{2}\hat{F}_{2}\left(D\hat{\theta}_{2}\right)^{2}\left(D_{2}P_{2}\right)^{2}
+D_{4}\hat{F}_{2}D^{2}\hat{\theta}_{2}\left(D_{2}P_{2}\right)^{2}
+D_{4}\hat{F}_{2}D\hat{\theta}_{2}D_{2}^{2}P_{2}
\\
&=&\hat{f}_{2}\left(4,4\right)\left(\frac{\tilde{\mu}_{2}}{1-\hat{\mu}_{2}}\right)^{2}
+\hat{f}_{2}\left(4\right)\hat{\theta}_{2}^{(2)}\tilde{\mu}_{2}^{2}
+\hat{f}_{2}\left(4\right)\frac{\tilde{P}_{2}^{(2)}}{1-\hat{\mu}_{2}}
\end{eqnarray*}



\begin{eqnarray*}
D_{3}D_{2}\hat{F}_{2}&=&
D_{4}^{2}\hat{F}_{2}\left(D\hat{\theta}_{2}\right)^{2}D_{2}P_{2}D_{3}\hat{P}_{1}
+D_{4}\hat{F}_{2} D^{2}\hat{\theta}_{2}D_{2}P_{2}D_{3}\hat{P}_{1}
+D_{4}\hat{F}_{2}D\hat{\theta} _{2}D_{2}P_{2}D_{3}\hat{P}_{1}
+D_{4}D_{3}\hat{F}_{2}D\hat{\theta}_{2}D_{2}P_{2}\\
&=&
\hat{f}_{2}\left(4,4\right)\left(\frac{1}{1-\hat{\mu}_{2}}\right)^{2}\tilde{\mu}_{2}\hat{\mu}_{1}
+\hat{f}_{2}\left(4\right)\hat{\theta}_{2}^{(2)}\tilde{\mu}_{2}\hat{\mu}_{1}
+\hat{f}_{2}\left(4\right)\frac{\tilde{\mu}_{2}\hat{\mu}_{1}}{1-\hat{\mu}_{2}}
+\hat{f}_{2}\left(4,3\right)\frac{\tilde{\mu}_{2}}{1-\hat{\mu}_{2}}
\end{eqnarray*}



\begin{eqnarray*}
D_{1}D_{3}\hat{F}_{2}&=&
D_{4}D_{4}\hat{F}_{2}\left(D\hat{\theta}_{2}\right)^{2}D_{1}P_{1}D_{3}\hat{P}_{1}
+D_{4}\hat{F}_{2}D^{2}\hat{\theta}_{2}D_{1}P_{1}D_{3}\hat{P}_{1}
+D_{4}\hat{F}_{2}D\hat{\theta}_{2}D_{1}P_{1}D_{3}\hat{P}_{1}
+D_{4}D_{3}\hat{F}_{2}D\hat{\theta}_{2}D_{1}P_{1}\\
&=&
\hat{f}_{2}\left(4,4\right)\left(\frac{1}{1-\hat{\mu}_{2}}\right)^{2}\tilde{\mu}_{1}\hat{\mu}_{1}
+\hat{f}_{2}\left(4\right)\hat{\theta}_{2}^{(2)}\tilde{\mu}_{1}\hat{\mu}_{1}
+\hat{f}_{2}\left(4\right)\frac{\tilde{\mu}_{1}\hat{\mu}_{1}}{1-\hat{\mu}_{2}}
+\hat{f}_{2}\left(4,3\right)\frac{\tilde{\mu}_{1}}{1-\hat{\mu}_{2}}
\end{eqnarray*}



\begin{eqnarray*}
D_{2}D_{3}\hat{F}_{2}&=&
D_{4}^{2}\hat{F}_{2}\left(D\hat{\theta}_{2}\right)^{2}D_{2}P_{2}D_{3}\hat{P}_{1}
+D_{4}\hat{F}_{2}D^{2}\hat{\theta}_{2}D_{2}P_{2}D_{3}\hat{P}_{1}
+D_{4}\hat{F}_{2}D\hat{\theta}_{2}D_{2}P_{2}D_{3}\hat{P}_{1}
+D_{4}D_{3}\hat{F}_{2}D\hat{\theta}_{2}D_{2}P_{2}\\
&=&
\hat{f}_{2}\left(4,4\right)\left(\frac{1}{1-\hat{\mu}_{2}}\right)^{2}\tilde{\mu}_{2}\hat{\mu}_{1}
+\hat{f}_{2}\left(4\right)\hat{\theta}_{2}^{(2)}\tilde{\mu}_{2}\hat{\mu}_{1}
+\hat{f}_{2}\left(4\right)\frac{\tilde{\mu}_{2}\hat{\mu}_{1}}{1-\hat{\mu}_{2}}
+\hat{f}_{2}\left(4,3\right)\frac{\tilde{\mu}_{2}}{1-\hat{\mu}_{2}}
\end{eqnarray*}



\begin{eqnarray*}
D_{3}D_{3}\hat{F}_{2}&=&
D_{4}^{2}\hat{F}_{2}\left(D\hat{\theta}_{2}\right)^{2}\left(D_{3}\hat{P}_{1}\right)^{2}
+D_{4}\hat{F}_{2}D^{2}\hat{\theta}_{2}\left(D_{3}\hat{P}_{1}\right)^{2}
+D_{4}\hat{F}_{2}D\hat{\theta}_{2}D_{3}^{2}\hat{P}_{1}
+D_{4}D_{3}\hat{F}_{2}D\hat{\theta}_{2}D_{3}\hat{P}_{1}\\
&+&D_{4}D_{3}\hat{f}_{2}D\hat{\theta}_{2}D_{3}\hat{P}_{1}
+D_{3}^{2}\hat{F}_{2}\\
&=&
\hat{f}_{2}\left(4,4\right)\left(\frac{\hat{\mu}_{1}}{1-\hat{\mu}_{2}}\right)^{2}
+\hat{f}_{2}\left(4\right)\hat{\theta}_{2}^{(2)}\hat{\mu}_{1}^{2}
+\hat{f}_{2}\left(4\right)\frac{\hat{P}_{1}^{(2)}}{1-\hat{\mu}_{2}}
+\hat{f}_{2}\left(4,3\right)\frac{\hat{\mu}_{1}}{1-\hat{\mu}_{2}}
+\hat{f}_{2}\left(4,3\right)\frac{\tilde{\mu}_{1}}{1-\hat{\mu}_{2}}
+\hat{f}_{2}\left(3,3\right)
\end{eqnarray*}

%_____________________________________________________________
\subsection{Second Grade Derivative Recursive Equations}
%_____________________________________________________________


Then according to the equations given at the beginning of this section, we have

\begin{eqnarray*}
D_{k}D_{i}F_{1}&=&D_{k}D_{i}\left(R_{2}+F_{2}+\indora_{i\geq3}\hat{F}_{4}\right)+D_{i}R_{2}D_{k}\left(F_{2}+\indora_{k\geq3}\hat{F}_{4}\right)\\&+&D_{i}F_{2}D_{k}\left(R_{2}+\indora_{k\geq3}\hat{F}_{4}\right)+\indora_{i\geq3}D_{i}\hat{F}_{4}D_{k}\left(R_{2}+F_{2}\right)
\end{eqnarray*}
%_____________________________________________________________
\subsection*{$F_{1}$}
%_____________________________________________________________
%_____________________________________________________________
\subsubsection*{$F_{1}$ and $i=1$}
%_____________________________________________________________

for $i=1$, and $k=1$

\begin{eqnarray*}
D_{1}D_{1}F_{1}&=&D_{1}D_{1}\left(R_{2}+F_{2}\right)+D_{1}R_{2}D_{1}F_{2}
+D_{1}F_{2}D_{1}R_{2}
=D_{1}^{2}R_{2}
+D_{1}^{2}F_{2}
+D_{1}R_{2}D_{1}F_{2}
+D_{1}F_{2}D_{1}R_{2}\\
&=&R_{2}^{(2)}\tilde{\mu}_{1}+r_{2}\tilde{P}_{1}^{(2)}
+D_{1}^{2}F_{2}
+2r_{2}\tilde{\mu}_{1}f_{2}\left(1\right)
\end{eqnarray*}

$k=2$
\begin{eqnarray*}
D_{2}D_{i}F_{1}&=&D_{2}D_{1}\left(R_{2}+F_{2}\right)
+D_{1}R_{2}D_{2}F_{2}+D_{1}F_{2}D_{2}R_{2}
=D_{2}D_{1}R_{2}
+D_{2}D_{1}F_{2}
+D_{1}R_{2}D_{2}F_{2}
+D_{1}F_{2}D_{2}R_{2}\\
&=&R_{2}^{(2)}\tilde{\mu}_{1}\tilde{\mu}_{2}+r_{2}\tilde{\mu}_{1}\tilde{\mu}_{2}
+D_{2}D_{1}F_{2}
+r_{2}\tilde{\mu}_{1}f_{2}\left(2\right)
+r_{2}\tilde{\mu}_{2}f_{2}\left(1\right)
\end{eqnarray*}

$k=3$
\begin{eqnarray*}
D_{3}D_{1}F_{1}&=&D_{3}D_{1}\left(R_{2}+F_{2}\right)
+D_{1}R_{2}D_{3}\left(F_{2}+\hat{F}_{4}\right)
+D_{1}F_{2}D_{3}\left(R_{2}+\hat{F}_{4}\right)\\
&=&D_{3}D_{1}R_{2}+D_{3}D_{1}F_{2}
+D_{1}R_{2}D_{3}F_{2}+D_{1}R_{2}D_{3}\hat{F}_{4}
+D_{1}F_{2}D_{3}R_{2}+D_{1}F_{2}D_{3}\hat{F}_{4}\\
&=&R_{2}^{(2)}\tilde{\mu}_{1}\hat{\mu}_{1}+r_{2}\tilde{\mu}_{1}\hat{\mu}_{1}
+D_{3}D_{1}F_{2}
+r_{2}\tilde{\mu}_{1}f_{2}\left(3\right)
+r_{2}\tilde{\mu}_{1}D_{3}\hat{F}_{4}
+r_{2}\hat{\mu}_{1}f_{2}\left(1\right)
+D_{3}\hat{F}_{4}f_{2}\left(1\right)
\end{eqnarray*}

$k=4$
\begin{eqnarray*}
D_{4}D_{1}F_{1}&=&D_{4}D_{1}\left(R_{2}+F_{2}\right)
+D_{1}R_{2}D_{4}\left(F_{2}+\hat{F}_{4}\right)
+D_{1}F_{2}D_{4}\left(R_{2}+\hat{F}_{4}\right)\\
&=&D_{4}D_{1}R_{2}+D_{4}D_{1}F_{2}
+D_{1}R_{2}D_{4}F_{2}+D_{1}R_{2}D_{4}\hat{F}_{4}
+D_{1}F_{2}D_{4}R_{2}+D_{1}F_{2}D_{4}\hat{F}_{4}\\
&=&R_{2}^{(2)}\tilde{\mu}_{1}\hat{\mu}_{2}+r_{2}\tilde{\mu}_{1}\hat{\mu}_{2}
+D_{4}D_{1}F_{2}
+r_{2}\tilde{\mu}_{1}f_{2}\left(4\right)
+r_{2}\tilde{\mu}_{1}D_{4}\hat{F}_{4}
+r_{2}\hat{\mu}_{2}f_{2}\left(1\right)
+f_{2}\left(1\right)D_{4}\hat{F}_{4}
\end{eqnarray*}


%_____________________________________________________________
\subsubsection*{$F_{1}$ and $i=2$}
%_____________________________________________________________

for $i=2$,

$k=2$
\begin{eqnarray*}
D_{2}D_{2}F_{1}&=&D_{2}D_{2}\left(R_{2}+F_{2}\right)
+D_{2}R_{2}D_{2}F_{2}+D_{2}F_{2}D_{2}R_{2}
=D_{2}D_{2}R_{2}+D_{2}D_{2}F_{2}+D_{2}R_{2}D_{2}F_{2}+D_{2}F_{2}D_{2}R_{2}\\
&=&R_{2}^{(2)}\tilde{\mu}_{2}^{2}+r_{2}\tilde{P}_{2}^{(2)}
+D_{2}D_{2}F_{2}
+2r_{2}\tilde{\mu}_{2}f_{2}\left(2\right)
\end{eqnarray*}

$k=3$
\begin{eqnarray*}
D_{3}D_{2}F_{1}&=&D_{3}D_{2}\left(R_{2}+F_{2}\right)
+D_{2}R_{2}D_{3}\left(F_{2}+\hat{F}_{4}\right)
+D_{2}F_{2}D_{3}\left(R_{2}+\hat{F}_{4}\right)\\
&=&D_{3}D_{2}R_{2}+D_{3}D_{2}F_{2}
+D_{2}R_{2}D_{3}F_{2}+D_{2}R_{2}D_{3}\hat{F}_{4}
+D_{2}F_{2}D_{3}R_{2}+D_{2}F_{2}D_{3}\hat{F}_{4}\\
&=&R_{2}^{(2)}\tilde{\mu}_{2}\hat{\mu}_{1}+r_{2}\tilde{\mu}_{2}\hat{\mu}_{1}
+D_{3}D_{2}F_{2}
+r_{2}\tilde{\mu}_{2}f_{2}\left(3\right)
+r_{2}\tilde{\mu}_{2}D_{3}\hat{F}_{4}
+r_{2}\hat{\mu}_{1}f_{2}\left(2\right)
+f_{2}\left(2\right)D_{3}\hat{F}_{4}
\end{eqnarray*}

$k=4$
\begin{eqnarray*}
D_{4}D_{2}F_{1}&=&D_{4}D_{2}\left(R_{2}+F_{2}\right)
+D_{2}R_{2}D_{4}\left(F_{2}+\hat{F}_{4}\right)
+D_{2}F_{2}D_{4}\left(R_{2}+\hat{F}_{4}\right)\\
&=&D_{4}D_{2}R_{2}+D_{4}D_{2}F_{2}
+D_{2}R_{2}D_{4}F_{2}+D_{2}R_{2}D_{4}\hat{F}_{4}
+D_{2}F_{2}D_{4}R_{2}+D_{2}F_{2}D_{4}\hat{F}_{4}\\
&=&R_{2}^{(2)}\tilde{\mu}_{2}\hat{\mu}_{2}+r_{2}\tilde{\mu}_{2}\hat{\mu}_{2}
+D_{4}D_{2}F_{2}
+r_{2}\tilde{\mu}_{2}f_{2}\left(4\right)
+r_{2}\tilde{\mu}_{2}D_{4}\hat{F}_{4}
+r_{2}\hat{\mu}_{2}f_{2}\left(2\right)
+f_{2}\left(2\right)D_{4}\hat{F}_{4}
\end{eqnarray*}

%_____________________________________________________________
\subsubsection*{$F_{1}$ and $i=3$}
%_____________________________________________________________
for $i=3$, and $k=3$
\begin{eqnarray*}
D_{3}D_{3}F_{1}&=&D_{3}D_{3}\left(R_{2}+F_{2}+\hat{F}_{4}\right)
+D_{3}R_{2}D_{3}\left(F_{2}+\hat{F}_{4}\right)
+D_{3}F_{2}D_{3}\left(R_{2}+\hat{F}_{4}\right)
+D_{3}\hat{F}_{4}D_{3}\left(R_{2}+F_{2}\right)\\
&=&D_{3}D_{3}R_{2}+D_{3}D_{3}F_{2}+D_{3}D_{3}\hat{F}_{4}
+D_{3}R_{2}D_{3}F_{2}+D_{3}R_{2}D_{3}\hat{F}_{4}\\
&+&D_{3}F_{2}D_{3}R_{2}+D_{3}F_{2}D_{3}\hat{F}_{4}
+D_{3}\hat{F}_{4}D_{3}R_{2}+D_{3}\hat{F}_{4}D_{3}F_{2}\\
&=&R_{2}^{(2)}\hat{\mu}_{1}^{2}+r_{2}\hat{P}_{1}^{(2)}
+D_{3}D_{3}F_{2}
+D_{3}D_{3}\hat{F}_{4}
+r_{2}\hat{\mu}_{1}f_{2}\left(3\right)
+r_{2}\hat{\mu}_{1}D_{3}\hat{F}_{4}\\
&+&r_{2}\hat{\mu}_{1}f_{2}\left(3\right)
+f_{2}\left(3\right)D_{3}\hat{F}_{4}
+r_{2}\hat{\mu}_{1}D_{3}\hat{F}_{4}
+f_{2}\left(3\right)D_{3}\hat{F}_{4}
\end{eqnarray*}

$k=4$
\begin{eqnarray*}
D_{4}D_{3}F_{1}&=&D_{4}D_{3}\left(R_{2}+F_{2}+\hat{F}_{4}\right)
+D_{3}R_{2}D_{4}\left(F_{2}+\hat{F}_{4}\right)
+D_{3}F_{2}D_{4}\left(R_{2}+\hat{F}_{4}\right)
+D_{3}\hat{F}_{4}D_{4}\left(R_{2}+F_{2}\right)\\
&=&D_{4}D_{3}R_{2}+D_{4}D_{3}F_{2}+D_{4}D_{3}\hat{F}_{4}
+D_{3}R_{2}D_{4}F_{2}+D_{3}R_{2}D_{4}\hat{F}_{4}\\
&+&D_{3}F_{2}D_{4}R_{2}+D_{3}F_{2}D_{4}\hat{F}_{4}
+D_{3}\hat{F}_{4}D_{4}R_{2}+D_{3}\hat{F}_{4}D_{4}F_{2}\\
&=&R_{2}^{(2)}\hat{\mu}_{1}\hat{\mu}_{2}+r_{2}\hat{\mu}_{1}\hat{\mu}_{2}
+D_{4}D_{3}F_{2}
+D_{4}D_{3}\hat{F}_{4}
+r_{2}\hat{\mu}_{1}f_{2}\left(4\right)
+r_{2}\hat{\mu}_{1}D_{4}\hat{F}_{4}\\
&+&r_{2}\hat{\mu}_{2}f_{2}\left(3\right)
+D_{4}\hat{F}_{4}f_{2}\left(3\right)
+D_{3}\hat{F}_{4}r_{2}\hat{\mu}_{2}
+D_{3}\hat{F}_{4}f_{2}\left(4\right)
\end{eqnarray*}

%_____________________________________________________________
\subsubsection*{$F_{1}$ and $i=4$}
%_____________________________________________________________

for $i=4$, $k=4$
\begin{eqnarray*}
D_{4}D_{4}F_{1}&=&D_{4}D_{4}\left(R_{2}+F_{2}+\hat{F}_{4}\right)
+D_{4}R_{2}D_{4}\left(F_{2}+\hat{F}_{4}\right)
+D_{4}F_{2}D_{4}\left(R_{2}+\hat{F}_{4}\right)
+D_{4}\hat{F}_{4}D_{4}\left(R_{2}+F_{2}\right)\\
&=&D_{4}D_{4}R_{2}+D_{4}D_{4}F_{2}+D_{4}D_{4}\hat{F}_{4}
+D_{4}R_{2}D_{4}F_{2}+D_{4}R_{2}D_{4}\hat{F}_{4}\\
&+&D_{4}F_{2}D_{4}R_{2}+D_{4}F_{2}D_{4}\hat{F}_{4}
+D_{4}\hat{F}_{4}D_{4}R_{2}+D_{4}\hat{F}_{4}D_{4}F_{2}\\
&=&R_{2}^{(2)}\hat{\mu}_{2}^{2}+r_{2}\hat{P}_{2}^{(2)}
+D_{4}D_{4}F_{2}
+D_{4}D_{4}\hat{F}_{4}
+r_{2}\hat{\mu}_{2}f_{2}\left(4\right)
+r_{2}\hat{\mu}_{2}D_{4}\hat{F}_{4}\\
&+&r_{2}\hat{\mu}_{2}f_{2}\left(4\right)
+D_{4}\hat{F}_{4}f_{2}\left(4\right)
+D_{4}\hat{F}_{4}r_{2}\hat{\mu}_{2}
+D_{4}\hat{F}_{4}f_{2}\left(4\right)
\end{eqnarray*}

%__________________________________________________________________________________________
%_____________________________________________________________
\subsection*{$F_{2}$}
%_____________________________________________________________
\begin{eqnarray}
D_{k}D_{i}F_{2}&=&D_{k}D_{i}\left(R_{1}+F_{1}+\indora_{i\geq3}\hat{F}_{3}\right)+D_{i}R_{1}D_{k}\left(F_{1}+\indora_{k\geq3}\hat{F}_{3}\right)+D_{i}F_{1}D_{k}\left(R_{1}+\indora_{k\geq3}\hat{F}_{3}\right)+\indora_{i\geq3}D_{i}\hat{F}_{3}D_{k}\left(R_{1}+F_{1}\right)
\end{eqnarray}
%_____________________________________________________________
\subsubsection*{$F_{2}$ and $i=1$}
%_____________________________________________________________
$i=1$, $k=1$
\begin{eqnarray*}
D_{1}D_{1}F_{2}&=&D_{1}D_{1}\left(R_{1}+F_{1}\right)
+D_{1}R_{1}D_{1}F_{1}
+D_{1}F_{1}D_{1}R_{1}
=D_{1}^{2}R_{1}
+D_{1}^{2}F_{1}
+D_{1}R_{1}D_{1}F_{1}
+D_{1}F_{1}D_{1}R_{1}\\
&=&R_{1}^{2}\tilde{\mu}_{1}^{2}+r_{1}\tilde{P}_{1}^{(2)}
+D_{1}^{2}F_{1}
+2r_{1}\tilde{\mu}_{1}f_{1}\left(1\right)
\end{eqnarray*}

$k=2$
\begin{eqnarray*}
D_{2}D_{1}F_{2}&=&D_{2}D_{1}\left(R_{1}+F_{1}\right)+D_{1}R_{1}D_{2}F_{1}+D_{1}F_{1}D_{2}R_{1}=
D_{2}D_{1}R_{1}+D_{2}D_{1}F_{1}+D_{1}R_{1}D_{2}F_{1}+D_{1}F_{1}D_{2}R_{1}\\
&=&R_{1}^{(2)}\tilde{\mu}_{1}\tilde{\mu}_{2}+r_{1}\tilde{\mu}_{1}\tilde{\mu}_{2}
+D_{2}D_{1}F_{1}
+r_{1}\tilde{\mu}_{1}f_{1}\left(2\right)
+r_{1}\tilde{\mu}_{2}f_{1}\left(1\right)
\end{eqnarray*}

$k=3$
\begin{eqnarray*}
D_{3}D_{1}F_{2}&=&D_{3}D_{1}\left(R_{1}+F_{1}\right)+D_{1}R_{1}D_{3}\left(F_{1}+\hat{F}_{3}\right)+D_{1}F_{1}D_{3}\left(R_{1}+\hat{F}_{3}\right)\\
&=&D_{3}D_{1}R_{1}+D_{3}D_{1}F_{1}+D_{1}R_{1}D_{3}F_{1}+D_{1}R_{1}D_{3}\hat{F}_{3}+D_{1}F_{1}D_{3}R_{1}+D_{1}F_{1}D_{3}\hat{F}_{3}\\
&=&R_{1}^{(2)}\tilde{\mu}_{1}\hat{\mu}_{1}+r_{1}\tilde{\mu}_{1}\hat{\mu}_{1}
+D_{3}D_{1}F_{1}
+r_{1}\tilde{\mu}_{1}f_{1}\left(3\right)
+r_{1}\tilde{\mu}_{1}D_{3}\hat{F}_{3}
+r_{1}\hat{\mu}_{1}f_{1}\left(1\right)
+D_{3}\hat{F}_{3}f_{1}\left(1\right)
\end{eqnarray*}

$k=4$
\begin{eqnarray*}
D_{4}D_{1}F_{2}&=&D_{4}D_{1}\left(R_{1}+F_{1}\right)+D_{1}R_{1}D_{4}\left(F_{1}+\hat{F}_{3}\right)+D_{1}F_{1}D_{4}\left(R_{1}+\hat{F}_{3}\right)\\
&=&D_{4}D_{1}R_{1}+D_{4}D_{1}F_{1}+D_{1}R_{1}D_{4}F_{1}+D_{1}R_{1}D_{4}\hat{F}_{3}
+D_{1}F_{1}D_{4}R_{1}+D_{1}F_{1}D_{4}\hat{F}_{3}\\
&=&R_{1}^{(2)}\tilde{\mu}_{1}\hat{\mu}_{2}+r_{1}\tilde{\mu}_{1}\hat{\mu}_{2}
+D_{4}D_{1}F_{1}
+r_{1}\tilde{\mu}_{1}f_{1}\left(4\right)
+\tilde{\mu}_{1}D_{4}f_{3}\left(4\right)
+\tilde{\mu}_{1}\hat{\mu}_{2}f_{1}\left(1\right)
+f_{1}\left(1\right)D_{4}F_{4}
\end{eqnarray*}
%_____________________________________________________________
\subsubsection*{$F_{2}$ and $i=2$}
%_____________________________________________________________
%__________________________________________________________________________________________
$i=2$
%__________________________________________________________________________________________
$k=2$
\begin{eqnarray*}
D_{2}D_{2}F_{2}&=&D_{2}D_{2}\left(R_{1}+F_{1}\right)+D_{2}R_{1}D_{2}F_{1}+D_{2}F_{1}D_{2}R_{1}
=D_{2}D_{2}R_{1}+D_{2}D_{2}F_{1}+D_{2}R_{1}D_{2}F_{1}+D_{2}F_{1}D_{2}R_{1}\\
&=&R_{1}^{(2)}\tilde{\mu}_{2}^{2}+r_{1}\tilde{P}_{2}^{(2)}
+D_{2}D_{2}F_{1}
2r_{1}\tilde{\mu}_{2}f_{1}\left(2\right)
\end{eqnarray*}

$k=3$
\begin{eqnarray*}
D_{3}D_{2}F_{2}&=&D_{3}D_{2}\left(R_{1}+F_{1}\right)+D_{2}R_{1}D_{3}\left(F_{1}+\hat{F}_{3}\right)+D_{2}F_{1}D_{3}\left(R_{1}+\hat{F}_{3}\right)\\
&=&D_{3}D_{2}R_{1}+D_{3}D_{2}F_{1}
+D_{2}R_{1}D_{3}F_{1}+D_{2}R_{1}D_{3}\hat{F}_{3}
+D_{2}F_{1}D_{3}R_{1}+D_{2}F_{1}D_{3}\hat{F}_{3}\\
&=&R_{1}^{(2)}\tilde{\mu}_{2}\hat{\mu}_{1}+r_{1}\tilde{\mu}_{2}\hat{\mu}_{1}
+D_{3}D_{2}F_{1}
+r_{1}\tilde{\mu}_{2}f_{1}\left(3\right)
+r_{1}\tilde{\mu}_{2}D_{3}\hat{F}_{3}
+r_{1}\hat{\mu}_{1}f_{1}\left(2\right)
+D_{3}\hat{F}_{3}f_{1}\left(2\right)
\end{eqnarray*}

$k=4$
\begin{eqnarray*}
D_{4}D_{2}F_{2}&=&D_{4}D_{2}\left(R_{1}+F_{1}\right)+D_{2}R_{1}D_{4}\left(F_{1}+\hat{F}_{3}\right)+D_{2}F_{1}D_{4}\left(R_{1}+\hat{F}_{3}\right)\\
&=&D_{4}D_{2}R_{1}+D_{4}D_{2}F_{1}
+D_{2}R_{1}D_{4}F_{1}+D_{2}R_{1}D_{4}\hat{F}_{3}
+D_{2}F_{1}D_{4}R_{1}+D_{2}F_{1}D_{4}\hat{F}_{3}\\
&=&R_{1}^{(2)}\tilde{\mu}_{2}\hat{\mu}_{2}+r_{1}\tilde{\mu}_{2}\hat{\mu}_{2}
+D_{4}D_{2}F_{1}
+r_{1}\tilde{\mu}_{2}f_{1}\left(4\right)
+r_{1}\tilde{\mu}_{2}D_{4}\hat{F}_{3}
+r_{1}\hat{\mu}_{2}f_{1}\left(2\right)
+D_{4}\hat{F}_{3}f_{1}\left(2\right)
\end{eqnarray*}

%_____________________________________________________________
\subsubsection*{$F_{2}$ and $i=3$}
%_____________________________________________________________
%__________________________________________________________________________________________
$i=3$
%__________________________________________________________________________________________
$k=3$
\begin{eqnarray*}
D_{3}D_{3}F_{2}&=&D_{3}D_{3}\left(R_{1}+F_{1}+\hat{F}_{3}\right)
+D_{3}R_{1}D_{3}\left(F_{1}+\hat{F}_{3}\right)
+D_{3}F_{1}D_{3}\left(R_{1}+\hat{F}_{3}\right)
+D_{3}\hat{F}_{3}D_{3}\left(R_{1}+F_{1}\right)\\
&=&D_{3}D_{3}R_{1}+D_{3}D_{3}F_{1}+D_{3}D_{3}\hat{F}_{3}
+D_{3}R_{1}D_{3}F_{1}+D_{3}R_{1}D_{3}\hat{F}_{3}\\
&+&D_{3}F_{1}D_{3}R_{1}+D_{3}F_{1}D_{3}\hat{F}_{3}
+D_{3}\hat{F}_{3}D_{3}R_{1}+D_{3}\hat{F}_{3}D_{3}F_{1}\\
&=&R_{1}^{(2)}\hat{\mu}_{1}^{2}+r_{1}\hat{P}_{1}^{(2)}
+D_{3}D_{3}F_{1}
+D_{3}D_{3}\hat{F}_{3}
+r_{1}\hat{\mu}_{1}f_{1}\left(3\right)
+r_{1}\hat{\mu}_{1}f_{3}\left(3\right)\\
&+&r_{1}\hat{\mu}_{1}f_{1}\left(3\right)
+D_{3}\hat{F}_{3}f_{1}\left(3\right)
+D_{3}\hat{F}_{3}r_{1}\hat{\mu}_{1}
+D_{3}\hat{F}_{3}f_{1}\left(3\right)
\end{eqnarray*}

$k=4$
\begin{eqnarray*}
D_{4}D_{3}F_{2}&=&D_{4}D_{3}\left(R_{1}+F_{1}+\hat{F}_{3}\right)
+D_{3}R_{1}D_{4}\left(F_{1}+\hat{F}_{3}\right)
+D_{3}F_{1}D_{4}\left(R_{1}+\hat{F}_{3}\right)
+D_{3}\hat{F}_{3}D_{4}\left(R_{1}+F_{1}\right)\\
&=&D_{4}D_{3}R_{1}+D_{4}D_{3}F_{1}+D_{4}D_{3}\hat{F}_{3}
+D_{3}R_{1}D_{4}F_{1}+D_{3}R_{1}D_{4}\hat{F}_{3}\\
&+&D_{3}F_{1}D_{4}R_{1}+D_{3}F_{1}D_{4}\hat{F}_{3}
+D_{3}\hat{F}_{3}D_{4}R_{1}+D_{3}\hat{F}_{3}D_{4}F_{1}\\
&=&R_{1}^{(2)}\hat{\mu}_{1}\hat{\mu}_{2}+r_{1}\hat{\mu}_{1}\hat{\mu}_{2}
+D_{4}D_{3}F_{1}
+D_{4}D_{3}\hat{F}_{3}
+r_{1}\hat{\mu}_{1}f_{1}\left(4\right)
+r_{1}\hat{\mu}_{1}D_{4}\hat{F}_{3}\\
&+&r_{1}\hat{\mu}_{2}f_{1}\left(3\right)
+D_{4}\hat{F}_{3}f_{1}\left(3\right)
+r_{1}\hat{\mu}_{2}D_{3}\hat{F}_{3}
+D_{3}\hat{F}_{3}f_{1}\left(4\right)
\end{eqnarray*}
%_____________________________________________________________
\subsubsection*{$F_{2}$ and $i=4$}
%_____________________________________________________________%__________________________________________________________________________________________
$i=4$ and $k=4$
\begin{eqnarray*}
D_{4}D_{4}F_{2}&=&D_{4}D_{4}\left(R_{1}+F_{1}+\hat{F}_{3}\right)
+D_{4}R_{1}D_{4}\left(F_{1}+\hat{F}_{3}\right)
+D_{4}F_{1}D_{4}\left(R_{1}+\hat{F}_{3}\right)
+D_{4}\hat{F}_{3}D_{4}\left(R_{1}+F_{1}\right)\\
&=&D_{4}D_{4}R_{1}+D_{4}D_{4}F_{1}+D_{4}D_{4}\hat{F}_{3}
+D_{4}R_{1}D_{4}F_{1}+D_{4}R_{1}D_{4}\hat{F}_{3}\\
&+&D_{4}F_{1}D_{4}R_{1}+D_{4}F_{1}D_{4}\hat{F}_{3}
+D_{4}\hat{F}_{3}D_{4}R_{1}+D_{4}\hat{F}_{3}D_{4}F_{1}\\
&=&R_{1}^{(2)}\hat{\mu}_{2}^{2}+r_{1}\hat{P}_{2}^{(2)}
+D_{4}D_{4}F_{1}
+D_{4}D_{4}\hat{F}_{3}
+f_{1}\left(4\right)r_{1}\hat{\mu}_{2}
+r_{1}\hat{\mu}_{2}D_{4}\hat{F}_{3}\\
&+&r_{1}\hat{\mu}_{2}f_{1}\left(4\right)
+D_{4}\hat{F}_{3}f_{1}\left(4\right)
+D_{4}\hat{F}_{3}r_{1}\hat{\mu}_{2}
+D_{4}\hat{F}_{3}f_{1}\left(4\right)
\end{eqnarray*}
%__________________________________________________________________________________________
\subsection*{$\hat{F}_{1}$}
%__________________________________________________________________________________________

\begin{eqnarray}
D_{k}D_{i}\hat{F}_{1}&=&D_{k}D_{i}\left(\hat{R}_{4}+\indora_{i\leq2}F_{2}+\hat{F}_{4}\right)+D_{i}\hat{R}_{4}D_{k}\left(\indora_{k\leq2}F_{2}+\hat{F}_{4}\right)+D_{i}\hat{F}_{4}D_{k}\left(\hat{R}_{4}+\indora_{k\leq2}F_{2}\right)+\indora_{i\leq2}D_{i}F_{2}D_{k}\left(\hat{R}_{4}+\hat{F}_{4}\right)
\end{eqnarray}
%__________________________________________________________________________________________
\subsubsection*{$\hat{F}_{1}$, $i=1$}
%__________________________________________________________________________________________

%__________________________________________________________________________________________
$i=1$ and $k=1$
\begin{eqnarray*}
D_{1}D_{1}\hat{F}_{1}&=&D_{1}D_{1}\left(\hat{R}_{4}+F_{2}+\hat{F}_{4}\right)
+D_{1}\hat{R}_{4}D_{1}\left(F_{2}+\hat{F}_{4}\right)
+D_{1}\hat{F}_{4}D_{1}\left(\hat{R}_{4}+F_{2}\right)
+D_{1}F_{2}D_{1}\left(\hat{R}_{4}+\hat{F}_{4}\right)\\
&=&D_{1}^{2}\hat{R}_{4}+D_{1}^{2}F_{2}+D_{1}^{2}\hat{F}_{4}
+D_{1}\hat{R}_{4}D_{1}F_{2}+D_{1}\hat{R}_{4}D_{1}\hat{F}_{4}
+D_{1}\hat{F}_{4}D_{1}\hat{R}_{4}+D_{1}\hat{F}_{4}D_{1}F_{2}
+D_{1}F_{2}D_{1}\hat{R}_{4}+D_{1}F_{2}D_{1}\hat{F}_{4}\\
&=&\hat{R}_{2}^{(2)}\tilde{\mu}_{1}^{2}+\hat{r}_{2}\tilde{P}_{1}^{(2)}
+D_{1}^{2}F_{2}
+D_{1}^{2}\hat{F}_{4}
+\hat{r}_{2}\tilde{\mu}_{1}D_{1}F_{2}\\
&+&\hat{r}_{2}\tilde{\mu}_{1}\hat{f}_{2}\left(1\right)
+\hat{f}_{2}\left(1\right)\hat{r}_{2}\tilde{\mu}_{1}
+\hat{f}_{2}\left(1\right)D_{1}F_{2}
+D_{1}F_{2}\hat{r}_{2}\tilde{\mu}_{1}
+D_{1}F_{2}\hat{f}_{2}\left(1\right)
\end{eqnarray*}

$k=2$
\begin{eqnarray*}
D_{2}D_{1}\hat{F}_{1}&=&D_{2}D_{1}\left(\hat{R}_{4}+F_{2}+\hat{F}_{4}\right)
+D_{1}\hat{R}_{4}D_{2}\left(F_{2}+\hat{F}_{4}\right)
+D_{1}\hat{F}_{4}D_{2}\left(\hat{R}_{4}+F_{2}\right)
+D_{1}F_{2}D_{2}\left(\hat{R}_{4}+\hat{F}_{4}\right)\\
&=&D_{2}D_{1}\hat{R}_{4}+D_{2}D_{1}F_{2}+D_{2}D_{1}\hat{F}_{4}
+D_{1}\hat{R}_{4}D_{2}F_{2}+D_{1}\hat{R}_{4}D_{2}\hat{F}_{4}\\
&+&D_{1}\hat{F}_{4}D_{2}\hat{R}_{4}+D_{1}\hat{F}_{4}D_{2}F_{2}
+D_{1}F_{2}D_{2}\hat{R}_{4}+D_{1}F_{2}D_{2}\hat{F}_{4}\\
&=&\hat{R}_{2}^{(2)}\tilde{\mu}_{1}\tilde{\mu}_{2}+\hat{r}_{2}\tilde{\mu}_{1}\tilde{\mu}_{2}
+D_{2}D_{1}F_{2}
+D_{2}D_{1}\hat{F}_{4}
+\hat{r}_{2}\tilde{\mu}_{1}D_{2}F_{2}
+\hat{r}_{2}\tilde{\mu}_{1}\hat{f}_{2}\left(2\right)\\
&+&\hat{r}_{2}\tilde{\mu}_{2}\hat{f}_{2}\left(1\right)
+\hat{f}_{2}\left(1\right)D_{2}F_{2}
+\hat{r}_{2}\tilde{\mu}_{2}D_{1}F_{2}
+D_{1}F_{2}\hat{f}_{2}\left(2\right)
\end{eqnarray*}

$k=3$
\begin{eqnarray*}
D_{3}D_{1}\hat{F}_{1}&=&D_{3}D_{1}\left(\hat{R}_{4}+F_{2}+\hat{F}_{4}\right)
+D_{1}\hat{R}_{4}D_{3}\left(\hat{F}_{4}\right)
+D_{1}\hat{F}_{4}D_{3}\hat{R}_{4}
+D_{1}F_{2}D_{3}\left(\hat{R}_{4}+\hat{F}_{4}\right)\\
&=&D_{3}D_{1}\hat{R}_{4}+D_{3}D_{1}F_{2}+D_{3}D_{1}\hat{F}_{4}
+D_{1}\hat{R}_{4}D_{3}\hat{F}_{4}
+D_{1}\hat{F}_{4}D_{3}\hat{R}_{4}
+D_{1}F_{2}D_{3}\hat{R}_{4}+D_{1}F_{2}D_{3}\hat{F}_{4}\\
&=&\hat{R}_{2}^{(2)}\tilde{\mu}_{1}\hat{\mu}_{1}+\hat{r}_{2}\tilde{\mu}_{1}\hat{\mu}_{1}
+D_{3}D_{1}F_{2}
+D_{3}D_{1}\hat{F}_{4}
+\hat{r}_{2}\tilde{\mu}_{1}\hat{f}_{2}\left(3\right)
+\hat{f}_{2}\left(1\right)\hat{r}_{2}\hat{\mu}_{1}
+D_{1}F_{2}\hat{r}_{2}\hat{\mu}_{1}
+D_{1}F_{2}\hat{f}_{2}\left(3\right)
\end{eqnarray*}

$k=4$
\begin{eqnarray*}
D_{4}D_{1}\hat{F}_{1}&=&D_{4}D_{1}\left(\hat{R}_{4}+F_{2}+\hat{F}_{4}\right)
+D_{1}\hat{R}_{4}D_{4}\hat{F}_{4}
+D_{1}\hat{F}_{4}D_{4}\hat{R}_{4}
+D_{1}F_{2}D_{4}\left(\hat{R}_{4}+\hat{F}_{4}\right)\\
&=&D_{4}D_{1}\hat{R}_{4}+D_{4}D_{1}F_{2}+D_{4}D_{1}\hat{F}_{4}
+D_{1}\hat{R}_{4}D_{4}\hat{F}_{4}
+D_{1}\hat{F}_{4}D_{4}\hat{R}_{4}
+D_{1}F_{2}D_{4}\hat{R}_{4}+D_{1}F_{2}D_{4}\hat{F}_{4}\\
&=&\hat{R}_{2}^{(2)}\tilde{\mu}_{1}\hat{\mu}_{2}+\hat{r}_{2}\tilde{\mu}_{1}\hat{\mu}_{2}
+D_{4}D_{1}F_{2}
+D_{4}D_{1}\hat{F}_{4}
+\hat{r}_{2}\tilde{\mu}_{1}\hat{f}_{2}\left(4\right)
+\hat{f}_{2}\left(1\right)\hat{r}_{2}\hat{\mu}_{2}
+D_{1}F_{2}\hat{r}_{2}\hat{\mu}_{2}
+D_{1}F_{2}\hat{f}_{2}\left(4\right)
\end{eqnarray*}

%__________________________________________________________________________________________
\subsubsection*{$\hat{F}_{1}$, $i=2$}
%__________________________________________________________________________________________

%__________________________________________________________________________________________
$i=2$ and $k=2$
\begin{eqnarray*}
D_{2}D_{2}\hat{F}_{1}&=&D_{2}D_{2}\left(\hat{R}_{4}+F_{2}+\hat{F}_{4}\right)
+D_{2}\hat{R}_{4}D_{2}\left(F_{2}+\hat{F}_{4}\right)
+D_{2}\hat{F}_{4}D_{2}\left(\hat{R}_{4}+F_{2}\right)
+D_{2}F_{2}D_{2}\left(\hat{R}_{4}+\hat{F}_{4}\right)\\
&=&D_{2}D_{2}\hat{R}_{4}+D_{2}D_{2}F_{2}+D_{2}D_{2}\hat{F}_{4}
+D_{2}\hat{R}_{4}D_{2}F_{2}+D_{2}\hat{R}_{4}D_{2}\hat{F}_{4}\\
&+&D_{2}\hat{F}_{4}D_{2}\hat{R}_{4}+D_{2}\hat{F}_{4}D_{2}F_{2}
+D_{2}F_{2}D_{2}\hat{R}_{4}+D_{2}F_{2}D_{2}\hat{F}_{4}\\
&=&\hat{R}_{2}^{(2)}\tilde{\mu}_{2}^{2}+\hat{r}_{2}\tilde{P}_{1}^{(2)}
+D_{2}D_{2}F_{2}
+D_{2}D_{2}\hat{F}_{4}
+\hat{r}_{2}\tilde{\mu}_{2}D_{2}F_{2}
+\hat{r}_{2}\tilde{\mu}_{2}\hat{f}_{2}\left(4\right)\\
&+&\hat{f}_{2}\left(4\right)\hat{r}_{2}\tilde{\mu}_{2}
+\hat{f}_{2}\left(4\right)D_{2}F_{2}
+D_{2}F_{2}\hat{r}_{2}\tilde{\mu}_{2}
+D_{2}F_{2}\hat{f}_{2}\left(4\right)
\end{eqnarray*}

$k=3$
\begin{eqnarray*}
D_{3}D_{2}\hat{F}_{1}&=&D_{3}D_{2}\left(\hat{R}_{4}+F_{2}+\hat{F}_{4}\right)
+D_{2}\hat{R}_{4}D_{3}\hat{F}_{4}
+D_{2}\hat{F}_{4}D_{3}\hat{R}_{4}
+D_{2}F_{2}D_{3}\left(\hat{R}_{4}+\hat{F}_{4}\right)\\
&=&D_{3}D_{2}\hat{R}_{4}+D_{3}D_{2}F_{2}+D_{3}D_{2}\hat{F}_{4}
+D_{2}\hat{R}_{4}D_{3}\hat{F}_{4}
+D_{2}\hat{F}_{4}D_{3}\hat{R}_{4}
+D_{2}F_{2}D_{3}\hat{R}_{4}+D_{2}F_{2}D_{3}\hat{F}_{4}\\
&=&\hat{R}_{2}^{(2)}\tilde{\mu}_{2}\hat{\mu}_{1}+\hat{r}_{2}\tilde{\mu}_{2}\hat{\mu}_{1}
+D_{3}D_{2}F_{2}
+D_{3}D_{2}\hat{F}_{4}+\hat{r}_{2}\tilde{\mu}_{2}\hat{f}_{2}\left(3\right)
+\hat{f}_{2}\left(4\right)\hat{r}_{2}\hat{\mu}_{1}
+\hat{r}_{2}\hat{\mu}_{1}D_{2}F_{2}
+D_{2}F_{2}\hat{f}_{2}\left(3\right)
\end{eqnarray*}

$k=4$
\begin{eqnarray*}
D_{4}D_{2}\hat{F}_{1}&=&D_{4}D_{2}\left(\hat{R}_{4}+F_{2}+\hat{F}_{4}\right)
+D_{2}\hat{R}_{4}D_{4}\hat{F}_{4}
+D_{2}\hat{F}_{4}D_{4}\hat{R}_{4}
+D_{2}F_{2}D_{4}\left(\hat{R}_{4}+\hat{F}_{4}\right)\\
&=&D_{4}D_{2}\hat{R}_{4}+D_{4}D_{2}F_{2}+D_{4}D_{2}\hat{F}_{4}
+D_{2}\hat{R}_{4}D_{4}\hat{F}_{4}
+D_{2}\hat{F}_{4}D_{4}\hat{R}_{4}
+D_{2}F_{2}D_{4}\hat{R}_{4}+D_{2}F_{2}D_{4}\hat{F}_{4}\\
&=&\hat{R}_{2}^{(2)}\tilde{\mu}_{2}\hat{\mu}_{2}+\hat{r}_{2}\tilde{\mu}_{2}\hat{\mu}_{2}
+D_{4}D_{2}F_{2}
+D_{4}D_{2}\hat{F}_{4}
+\hat{r}_{2}\tilde{\mu}_{2}\hat{f}_{2}\left(4\right)
+\hat{f}_{2}\left(4\right)\hat{r}_{2}\hat{\mu}_{2}
+D_{2}F_{2}\hat{r}_{2}\hat{\mu}_{2}
+D_{2}F_{2}\hat{f}_{2}\left(4\right)
\end{eqnarray*}
%__________________________________________________________________________________________
\subsubsection*{$\hat{F}_{1}$, $i=3$}
%__________________________________________________________________________________________

$k=3$
\begin{eqnarray*}
D_{3}D_{3}\hat{F}_{1}&=&D_{3}D_{3}\left(\hat{R}_{4}+\hat{F}_{4}\right)
+D_{3}\hat{R}_{4}D_{3}\hat{F}_{4}
+D_{3}\hat{F}_{4}D_{3}\hat{R}_{4}=D_{3}^{2}\hat{R}_{4}+D_{3}^{2}\hat{F}_{4}
+D_{3}\hat{R}_{4}D_{3}\hat{F}_{4}
+D_{3}\hat{F}_{4}D_{3}\hat{R}_{4}\\
&=&\hat{R}_{2}^{(2)}\hat{\mu}_{1}^{2}+\hat{r}_{2}\hat{P}_{1}^{(2)}
+D_{3}^{2}\hat{F}_{4}
+\hat{r}_{2}\hat{\mu}_{1}\hat{f}_{2}\left(4\right)
+\hat{r}_{2}\hat{\mu}_{1}\hat{f}_{2}\left(3\right)
\end{eqnarray*}

$k=4$
\begin{eqnarray*}
D_{4}D_{3}\hat{F}_{1}&=&D_{4}D_{3}\left(\hat{R}_{4}+\hat{F}_{4}\right)
+D_{3}\hat{R}_{4}D_{4}\hat{F}_{4}
+D_{3}\hat{F}_{4}D_{4}\hat{R}_{4}=D_{4}D_{3}\hat{R}_{4}+D_{4}D_{3}\hat{F}_{4}
+D_{3}\hat{R}_{4}D_{4}\hat{F}_{4}
+D_{3}\hat{F}_{4}D_{4}\hat{R}_{4}\\
&=&\hat{R}_{2}^{(2)}\hat{\mu}_{1}\hat{\mu}_{2}+\hat{r}_{2}\hat{\mu}_{1}\hat{\mu}_{2}
+D_{4}D_{3}\hat{F}_{4}
+\hat{r}_{2}\hat{\mu}_{1}\hat{f}_{2}\left(4\right)
+\hat{r}_{2}\hat{\mu}_{2}\hat{f}_{2}\left(3\right)
\end{eqnarray*}
%__________________________________________________________________________________________
\subsubsection*{$\hat{F}_{1}$, $i=4$}
%__________________________________________________________________________________________

$k=4$
\begin{eqnarray*}
D_{4}D_{4}\hat{F}_{1}&=&D_{4}D_{4}\left(\hat{R}_{4}+\hat{F}_{4}\right)
+D_{4}\hat{R}_{4}D_{4}\hat{F}_{4}
+D_{4}\hat{F}_{4}D_{4}\hat{R}_{4}=D_{4}^{2}\hat{R}_{4}+D_{4}^{2}\hat{F}_{4}
+D_{4}\hat{R}_{4}D_{4}\hat{F}_{4}
+D_{4}\hat{F}_{4}D_{4}\hat{R}_{4}\\
&=&\hat{R}_{2}^{(2)}\hat{\mu}_{2}^{2}+\hat{r}_{2}\hat{P}_{2}^{(2)}+D_{4}^{2}\hat{F}_{4}
+2\hat{r}_{2}\hat{\mu}_{2}\hat{f}_{2}\left(4\right)
\end{eqnarray*}
%__________________________________________________________________________________________
%
%__________________________________________________________________________________________
\subsection*{$\hat{F}_{2}$}
%__________________________________________________________________________________________
for $\hat{F}_{2}$
%__________________________________________________________________________________________
%
%__________________________________________________________________________________________

\begin{eqnarray}
D_{k}D_{i}\hat{F}_{2}&=&D_{k}D_{i}\left(\hat{R}_{3}+\indora_{i\leq2}F_{1}+\hat{F}_{3}\right)+D_{i}\hat{R}_{3}D_{k}\left(\indora_{k\leq2}F_{1}+\hat{F}_{3}\right)+D_{i}\hat{F}_{3}D_{k}\left(\hat{R}_{3}+\indora_{k\leq2}F_{1}\right)+\indora_{i\leq2}D_{i}F_{1}D_{k}\left(\hat{R}_{3}+\hat{F}_{3}\right)\\
&=&
\end{eqnarray}
%__________________________________________________________________________________________
\subsubsection*{$\hat{F}_{2}$, $i=1$}
%__________________________________________________________________________________________

$k=1$
\begin{eqnarray*}
D_{1}D_{1}\hat{F}_{2}&=&D_{1}^{2}\left(\hat{R}_{3}+F_{1}+\hat{F}_{3}\right)
+D_{1}\hat{R}_{3}D_{1}\left(F_{1}+\hat{F}_{3}\right)
+D_{1}\hat{F}_{3}D_{1}\left(\hat{R}_{3}+F_{1}\right)
+D_{1}F_{1}D_{1}\left(\hat{R}_{3}+\hat{F}_{3}\right)\\
&=&D_{1}^{2}\hat{R}_{3}+D_{1}^{2}F_{1}+D_{1}^{2}\hat{F}_{3}
+D_{1}\hat{R}_{3}D_{1}F_{1}+D_{1}\hat{R}_{3}D_{1}\hat{F}_{3}
+D_{1}\hat{F}_{3}D_{1}\hat{R}_{3}+D_{1}\hat{F}_{3}D_{1}F_{1}
+D_{1}F_{1}D_{1}\hat{R}_{3}+D_{1}F_{1}D_{1}\hat{F}_{3}\\
&=&
\hat{R}_{1}^{(2)}\tilde{\mu}_{1}^{2}+\hat{r}_{1}\tilde{P}_{2}^{(2)}
+D_{1}^{2}F_{1}
+D_{1}^{2}\hat{F}_{3}
+D_{1}F_{1}\hat{r}_{1}\tilde{\mu}_{1}\\
&+&\hat{r}_{1}\tilde{\mu}_{1}\hat{f}_{1}\left(1\right)
+\hat{r}_{1}\tilde{\mu}_{1}\hat{f}_{1}\left(1\right)
+D_{1}F_{1}\hat{f}_{1}\left(1\right)
+D_{1}F_{1}\hat{r}_{1}\tilde{\mu}_{1}
+D_{1}F_{1}\hat{f}_{1}\left(1\right)
\end{eqnarray*}

$k=2$
\begin{eqnarray*}
D_{2}D_{1}\hat{F}_{2}&=&D_{2}D_{1}\left(\hat{R}_{3}+F_{1}+\hat{F}_{3}\right)
+D_{1}\hat{R}_{3}D_{2}\left(F_{1}+\hat{F}_{3}\right)
+D_{1}\hat{F}_{3}D_{2}\left(\hat{R}_{3}+F_{1}\right)
+D_{1}F_{1}D_{2}\left(\hat{R}_{3}+\hat{F}_{3}\right)\\
&=&D_{2}D_{1}\hat{R}_{3}+D_{2}D_{1}F_{1}+D_{2}D_{1}\hat{F}_{3}
+D_{1}\hat{R}_{3}D_{2}F_{1}+D_{1}\hat{R}_{3}D_{2}\hat{F}_{3}\\
&+&D_{1}\hat{F}_{3}D_{2}\hat{R}_{3}+D_{1}\hat{F}_{3}D_{2}F_{1}
+D_{1}F_{1}D_{2}\hat{R}_{3}+D_{1}F_{1}D_{2}\hat{F}_{3}\\
&=&\hat{R}_{1}^{(2)}\tilde{\mu}_{1}\tilde{\mu}_{2}+\hat{r}_{1}\tilde{\mu}_{1}\tilde{\mu}_{2}
+D_{2}D_{1}F_{1}
+D_{2}D_{1}\hat{F}_{3}
+\hat{r}_{1}\tilde{\mu}_{1}D_{2}F_{1}
+\hat{r}_{1}\tilde{\mu}_{1}\hat{f}_{1}\left(2\right)\\
&+&\hat{f}_{1}\left(1\right)\hat{r}_{1}\tilde{\mu}_{2}
+\hat{r}_{1}\tilde{\mu}_{1}D_{2}F_{1}
+D_{1}F_{1}\hat{r}_{1}\tilde{\mu}_{2}
+D_{1}F_{1}\hat{f}_{1}\left(2\right)
\end{eqnarray*}

$k=3$
\begin{eqnarray*}
D_{3}D_{1}\hat{F}_{2}&=&D_{3}D_{1}\left(\hat{R}_{3}+F_{1}+\hat{F}_{3}\right)
+D_{1}\hat{R}_{3}D_{3}\hat{F}_{3}
+D_{1}\hat{F}_{3}D_{3}\hat{R}_{3}
+D_{1}F_{1}D_{3}\left(\hat{R}_{3}+\hat{F}_{3}\right)\\
&=&D_{3}D_{1}\hat{R}_{3}+D_{3}D_{1}F_{1}+D_{3}D_{1}\hat{F}_{3}
+D_{1}\hat{R}_{3}D_{3}\hat{F}_{3}
+D_{1}\hat{F}_{3}D_{3}\hat{R}_{3}
+D_{1}F_{1}D_{3}\hat{R}_{3}+D_{1}F_{1}D_{3}\hat{F}_{3}\\
&=&\hat{R}_{1}^{(2)}\tilde{\mu}_{1}\hat{\mu}_{1}+\hat{r}_{1}\tilde{\mu}_{1}\hat{\mu}_{1}
+D_{3}D_{1}F_{1}
+D_{3}D_{1}\hat{F}_{3}
+\hat{r}_{1}\tilde{\mu}_{1}\hat{f}_{1}\left(3\right)
+\hat{r}_{1}\hat{\mu}_{1}\hat{f}_{1}\left(1\right)
+\hat{r}_{1}\hat{\mu}_{1}D_{1}F_{1}
+D_{1}F_{1}\hat{f}_{1}\left(3\right)
\end{eqnarray*}

$k=4$
\begin{eqnarray*}
D_{4}D_{1}\hat{F}_{2}&=&D_{4}D_{1}\left(\hat{R}_{3}+F_{1}+\hat{F}_{3}\right)
+D_{1}\hat{R}_{3}D_{4}\hat{F}_{3}
+D_{1}\hat{F}_{3}D_{4}\hat{R}_{3}
+D_{1}F_{1}D_{4}\left(\hat{R}_{3}+\hat{F}_{3}\right)\\
&=&D_{4}D_{1}\hat{R}_{3}+D_{4}D_{1}F_{1}+D_{4}D_{1}\hat{F}_{3}
+D_{1}\hat{R}_{3}D_{4}\hat{F}_{3}
+D_{1}\hat{F}_{3}D_{4}\hat{R}_{3}
+D_{1}F_{1}D_{4}\hat{R}_{3}+D_{1}F_{1}D_{4}\hat{F}_{3}\\
&=&\hat{R}_{1}^{(2)}\tilde{\mu}_{1}\hat{\mu}_{2}+\hat{r}_{1}\tilde{\mu}_{1}\hat{\mu}_{2}
+D_{4}D_{1}F_{1}
+D_{4}D_{1}\hat{F}_{3}
+\hat{f}_{1}\left(4\right)\hat{r}_{1}\tilde{\mu}_{1}
+\hat{f}_{1}\left(3\right)\hat{r}_{1}\hat{\mu}_{2}
+D_{1}F_{1}\hat{r}_{1}\hat{\mu}_{2}
+D_{1}F_{1}\hat{f}_{1}\left(4\right)
\end{eqnarray*}
%__________________________________________________________________________________________
\subsubsection*{$\hat{F}_{2}$, $i=2$}
%__________________________________________________________________________________________


$k=2$
\begin{eqnarray*}
D_{2}D_{2}\hat{F}_{2}&=&D_{2}D_{2}\left(\hat{R}_{3}+F_{1}+\hat{F}_{3}\right)
+D_{2}\hat{R}_{3}D_{2}\left(F_{1}+\hat{F}_{3}\right)
+D_{2}\hat{F}_{3}D_{2}\left(\hat{R}_{3}+F_{1}\right)
+D_{2}F_{1}D_{2}\left(\hat{R}_{3}+\hat{F}_{3}\right)\\
&=&D_{2}^{2}\hat{R}_{3}+D_{2}^{2}F_{1}+D_{2}^{2}\hat{F}_{3}
+D_{2}\hat{R}_{3}D_{2}F_{1}+D_{2}\hat{R}_{3}D_{2}\hat{F}_{3}
+D_{2}\hat{F}_{3}D_{2}\hat{R}_{3}+D_{2}\hat{F}_{3}D_{2}F_{1}
+D_{2}F_{1}D_{2}\hat{R}_{3}+D_{2}F_{1}D_{2}\hat{F}_{3}\\
&=&\hat{R}_{1}^{(2)}\tilde{\mu}_{2}^{2}+\hat{r}_{1}\tilde{P}_{2}^{(2)}
+D_{2}^{2}F_{1}
+D_{2}^{2}\hat{F}_{3}
+\hat{r}_{1}\tilde{\mu}_{2}D_{2}F_{1}\\
&+&\hat{r}_{1}\tilde{\mu}_{2}\hat{f}_{1}\left(2\right)
+\hat{r}_{1}\tilde{\mu}_{2}\hat{f}_{1}\left(2\right)
+\hat{f}_{1}\left(1\right)D_{2}F_{1}
+\hat{r}_{1}\tilde{\mu}_{2}D_{2}F_{1}
+\hat{f}_{1}\left(3\right)D_{2}F_{1}
\end{eqnarray*}

$k=3$
\begin{eqnarray*}
D_{3}D_{2}\hat{F}_{2}&=&D_{3}D_{2}\left(\hat{R}_{3}+F_{1}+\hat{F}_{3}\right)
+D_{2}\hat{R}_{3}D_{3}\hat{F}_{3}
+D_{2}\hat{F}_{3}D_{3}\hat{R}_{3}
+D_{2}F_{1}D_{3}\left(\hat{R}_{3}+\hat{F}_{3}\right)\\
&=&D_{3}D_{2}\hat{R}_{3}+D_{3}D_{2}F_{1}+D_{3}D_{2}\hat{F}_{3}
+D_{2}\hat{R}_{3}D_{3}\hat{F}_{3}
+D_{2}\hat{F}_{3}D_{3}\hat{R}_{3}
+D_{2}F_{1}D_{3}\hat{R}_{3}+D_{2}F_{1}D_{3}\hat{F}_{3}\\
&=&\hat{R}_{1}^{(2)}\tilde{\mu}_{2}\hat{\mu}_{1}+\hat{r}_{1}\tilde{\mu}_{2}\hat{\mu}_{1}
+D_{3}D_{2}F_{1}
+D_{3}D_{2}\hat{F}_{3}
+\hat{r}_{1}\tilde{\mu}_{2}\hat{f}_{1}\left(3\right)
+\hat{r}_{1}\hat{\mu}_{1}\hat{f}_{1}\left(2\right)
+\hat{r}_{1}\hat{\mu}_{1}D_{2}F_{1}
+\hat{f}_{1}\left(3\right)D_{2}F_{1}
\end{eqnarray*}

$k=4$
\begin{eqnarray*}
D_{4}D_{2}\hat{F}_{2}&=&D_{4}D_{2}\left(\hat{R}_{3}+F_{1}+\hat{F}_{3}\right)
+D_{2}\hat{R}_{3}D_{4}\hat{F}_{3}
+D_{2}\hat{F}_{3}D_{4}\hat{R}_{3}
+D_{2}F_{1}D_{4}\left(\hat{R}_{3}+\hat{F}_{3}\right)\\
&=&D_{4}D_{2}\hat{R}_{3}+D_{4}D_{2}F_{1}+\hat{F}_{3}
+D_{2}\hat{R}_{3}D_{4}\hat{F}_{3}
+D_{2}\hat{F}_{3}D_{4}\hat{R}_{3}
+D_{2}F_{1}D_{4}\hat{R}_{3}+D_{2}F_{1}D_{4}\hat{F}_{3}\\
&=&\hat{R}_{1}^{(2)}\tilde{\mu}_{2}\hat{\mu}_{2}+\hat{r}_{1}\tilde{\mu}_{2}\hat{\mu}_{2}
+D_{4}D_{2}F_{1}
+D_{4}D_{2}\hat{F}_{3}
+\hat{r}_{1}\tilde{\mu}_{2}\hat{f}_{1}\left(4\right)
+\hat{r}_{1}\hat{\mu}_{2}\hat{f}_{1}\left(2\right)
+\hat{r}_{1}\hat{\mu}_{2}D_{2}F_{1}
+\hat{f}_{1}\left(4\right)D_{2}F_{1}
\end{eqnarray*}
%__________________________________________________________________________________________
\subsubsection*{$\hat{F}_{2}$, $i=3$}
%__________________________________________________________________________________________

$k=3$
\begin{eqnarray*}
D_{3}D_{3}\hat{F}_{2}&=&D_{3}D_{3}\left(\hat{R}_{3}+\hat{F}_{3}\right)
+D_{3}\hat{R}_{3}D_{3}\hat{F}_{3}
+D_{3}\hat{F}_{3}D_{3}\hat{R}_{3}=D_{3}^{2}\hat{R}_{3}+D_{3}^{2}\hat{F}_{3}
+D_{3}\hat{R}_{3}D_{3}\hat{F}_{3}
+D_{3}\hat{F}_{3}D_{3}\hat{R}_{3}\\
&=&\hat{R}_{1}^{(2)}\hat{\mu}_{1}^{2}+\hat{r}_{1}\hat{P}_{1}^{(2)}
+D_{3}^{2}\hat{F}_{3}
+\hat{r}_{1}\hat{\mu}_{1}\hat{f}_{1}\left(3\right)
+\hat{r}_{1}\hat{\mu}_{1}\hat{f}_{1}\left(3\right)
\end{eqnarray*}

$k=4$
\begin{eqnarray*}
D_{4}D_{3}\hat{F}_{2}&=&D_{4}D_{3}\left(\hat{R}_{3}+\hat{F}_{3}\right)
+D_{3}\hat{R}_{3}D_{4}\hat{F}_{3}
+D_{3}\hat{F}_{3}D_{4}\hat{R}_{3}=D_{4}D_{3}\hat{R}_{3}+D_{4}D_{3}\hat{F}_{3}
+D_{3}\hat{R}_{3}D_{4}\hat{F}_{3}
+D_{3}\hat{F}_{3}D_{4}\hat{R}_{3}\\
&=&\hat{R}_{1}^{(2)}\hat{\mu}_{1}\hat{\mu}_{2}+\hat{r}_{1}\hat{\mu}_{1}\hat{\mu}_{2}
+D_{4}D_{3}\hat{F}_{3}
+\hat{r}_{1}\hat{\mu}_{1}\hat{f}_{1}\left(4\right)
+\hat{r}_{1}\hat{\mu}_{2}\hat{f}_{1}\left(3\right)
\end{eqnarray*}
%__________________________________________________________________________________________
$i=4$
%__________________________________________________________________________________________

$k=4$
\begin{eqnarray*}
D_{4}D_{4}\hat{F}_{2}&=&D_{4}^{2}\left(\hat{R}_{3}+\hat{F}_{3}\right)
+D_{4}\hat{R}_{3}D_{4}\hat{F}_{3}
+D_{4}\hat{F}_{3}D_{4}\hat{R}_{3}=D_{4}^{2}\hat{R}_{3}+D_{4}^{2}\hat{F}_{3}
+D_{4}\hat{R}_{3}D_{4}\hat{F}_{3}
+D_{4}\hat{F}_{3}D_{4}\hat{R}_{3}\\
&=&\hat{R}_{1}^{(2)}\hat{\mu}_{2}^{2}+\hat{r}_{1}\hat{P}_{2}^{(2)}
+D_{4}^{2}\hat{F}_{3}
+\hat{r}_{1}\hat{\mu}_{2}\hat{f}_{1}\left(4\right)
\end{eqnarray*}
%__________________________________________________________________________________________
%

%_____________________________________________________________________________________
\newpage


%__________________________________________________________________
\section{Generalizaciones}
%__________________________________________________________________
\subsection{RSVC con dos conexiones}
%__________________________________________________________________

%\begin{figure}[H]
%\centering
%%%\includegraphics[width=9cm]{Grafica3.jpg}
%%\end{figure}\label{RSVC3}


Sus ecuaciones recursivas son de la forma


\begin{eqnarray*}
F_{1}\left(z_{1},z_{2},w_{1},w_{2}\right)&=&R_{2}\left(\prod_{i=1}^{2}\tilde{P}_{i}\left(z_{i}\right)\prod_{i=1}^{2}
\hat{P}_{i}\left(w_{i}\right)\right)F_{2}\left(z_{1},\tilde{\theta}_{2}\left(\tilde{P}_{1}\left(z_{1}\right)\hat{P}_{1}\left(w_{1}\right)\hat{P}_{2}\left(w_{2}\right)\right)\right)
\hat{F}_{2}\left(w_{1},w_{2};\tau_{2}\right),
\end{eqnarray*}

\begin{eqnarray*}
F_{2}\left(z_{1},z_{2},w_{1},w_{2}\right)&=&R_{1}\left(\prod_{i=1}^{2}\tilde{P}_{i}\left(z_{i}\right)\prod_{i=1}^{2}
\hat{P}_{i}\left(w_{i}\right)\right)F_{1}\left(\tilde{\theta}_{1}\left(\tilde{P}_{2}\left(z_{2}\right)\hat{P}_{1}\left(w_{1}\right)\hat{P}_{2}\left(w_{2}\right)\right),z_{2}\right)\hat{F}_{1}\left(w_{1},w_{2};\tau_{1}\right),
\end{eqnarray*}


\begin{eqnarray*}
\hat{F}_{1}\left(z_{1},z_{2},w_{1},w_{2}\right)&=&\hat{R}_{2}\left(\prod_{i=1}^{2}\tilde{P}_{i}\left(z_{i}\right)\prod_{i=1}^{2}
\hat{P}_{i}\left(w_{i}\right)\right)F_{2}\left(z_{1},z_{2};\zeta_{2}\right)\hat{F}_{2}\left(w_{1},\hat{\theta}_{2}\left(\tilde{P}_{1}\left(z_{1}\right)\tilde{P}_{2}\left(z_{2}\right)\hat{P}_{1}\left(w_{1}
\right)\right)\right),
\end{eqnarray*}


\begin{eqnarray*}
\hat{F}_{2}\left(z_{1},z_{2},w_{1},w_{2}\right)&=&\hat{R}_{1}\left(\prod_{i=1}^{2}\tilde{P}_{i}\left(z_{i}\right)\prod_{i=1}^{2}
\hat{P}_{i}\left(w_{i}\right)\right)F_{1}\left(z_{1},z_{2};\zeta_{1}\right)\hat{F}_{1}\left(\hat{\theta}_{1}\left(\tilde{P}_{1}\left(z_{1}\right)\tilde{P}_{2}\left(z_{2}\right)\hat{P}_{2}\left(w_{2}\right)\right),w_{2}\right),
\end{eqnarray*}

%_____________________________________________________
\subsection{First Moments of the Queue Lengths}
%_____________________________________________________


The server's switchover times are given by the general equation

\begin{eqnarray}\label{Ec.Ri}
R_{i}\left(\mathbf{z,w}\right)=R_{i}\left(\tilde{P}_{1}\left(z_{1}\right)\tilde{P}_{2}\left(z_{2}\right)\hat{P}_{1}\left(w_{1}\right)\hat{P}_{2}\left(w_{2}\right)\right)
\end{eqnarray}

with
\begin{eqnarray}\label{Ec.Derivada.Ri}
D_{i}R_{i}&=&DR_{i}D_{i}P_{i}
\end{eqnarray}
the following notation is considered

\begin{eqnarray*}
\begin{array}{llll}
D_{1}P_{1}\equiv D_{1}\tilde{P}_{1}, & D_{2}P_{2}\equiv D_{2}\tilde{P}_{2}, & D_{3}P_{3}\equiv D_{3}\hat{P}_{1}, &D_{4}P_{4}\equiv D_{4}\hat{P}_{2},
\end{array}
\end{eqnarray*}

also we need to remind $F_{1,2}\left(z_{1};\zeta_{2}\right)F_{2,2}\left(z_{2};\zeta_{2}\right)=F_{2}\left(z_{1},z_{2};\zeta_{2}\right)$, therefore

\begin{eqnarray*}
D_{1}F_{2}\left(z_{1},z_{2};\zeta_{2}\right)&=&D_{1}\left[F_{1,2}\left(z_{1};\zeta_{2}\right)F_{2,2}\left(z_{2};\zeta_{2}\right)\right]
=F_{2,2}\left(z_{2};\zeta_{2}\right)D_{1}F_{1,2}\left(z_{1};\zeta_{2}\right)=F_{1,2}^{(1)}\left(1\right)
\end{eqnarray*}

i.e., $D_{1}F_{2}=F_{1,2}^{(1)}(1)$; $D_{2}F_{2}=F_{2,2}^{(1)}\left(1\right)$, whereas that $D_{3}F_{2}=D_{4}F_{2}=0$, then

\begin{eqnarray*}
\begin{array}{ccc}
D_{i}F_{j}=\indora_{i\leq2}F_{i,j}^{(1)}\left(1\right),& \textrm{ y } &D_{i}\hat{F}_{j}=\indora_{i\geq2}F_{i,j}^{(1)}\left(1\right).
\end{array}
\end{eqnarray*}

Now, we obtain the first moments equations for the queue lengths as before for the times the server arrives to the queue to start attending



Remember that


\begin{eqnarray*}
F_{2}\left(z_{1},z_{2},w_{1},w_{2}\right)&=&R_{1}\left(\prod_{i=1}^{2}\tilde{P}_{i}\left(z_{i}\right)\prod_{i=1}^{2}
\hat{P}_{i}\left(w_{i}\right)\right)F_{1}\left(\tilde{\theta}_{1}\left(\tilde{P}_{2}\left(z_{2}\right)\hat{P}_{1}\left(w_{1}\right)\hat{P}_{2}\left(w_{2}\right)\right),z_{2}\right)\hat{F}_{1}\left(w_{1},w_{2};\tau_{1}\right),
\end{eqnarray*}

where


\begin{eqnarray*}
F_{1}\left(\tilde{\theta}_{1}\left(\tilde{P}_{2}\hat{P}_{1}\hat{P}_{2}\right),z_{2}\right)
\end{eqnarray*}

so

\begin{eqnarray*}
D_{i}F_{1}&=&\indora_{i\neq1}D_{1}F_{1}D\tilde{\theta}_{1}D_{i}P_{i}+\indora_{i=2}D_{i}F_{1},
\end{eqnarray*}

then


\begin{eqnarray}
D_{1}F_{1}&=&0,\\
D_{2}F_{1}&=&D_{1}F_{1}D\tilde{\theta}_{1}D_{2}P_{2}+D_{2}F_{1}
=f_{1}\left(1\right)\frac{1}{1-\tilde{\mu}_{1}}\tilde{\mu}_{2}+f_{1}\left(2\right),\\
D_{3}F_{1}&=&D_{1}F_{1}D\tilde{\theta}_{1}D_{3}P_{3}
=f_{1}\left(1\right)\frac{1}{1-\tilde{\mu}_{1}}\hat{\mu}_{1}\\
D_{4}F_{1}&=&D_{1}F_{1}D\tilde{\theta}_{1}D_{4}P_{4}
=f_{1}\left(1\right)\frac{1}{1-\tilde{\mu}_{1}}\hat{\mu}_{2}
\end{eqnarray}


\begin{eqnarray*}
D_{i}F_{2}&=&\indora_{i\neq2}D_{2}F_{2}D\tilde{\theta}_{2}D_{i}P_{i}
+\indora_{i=1}D_{i}F_{2}
\end{eqnarray*}

\begin{eqnarray}
D_{1}F_{2}&=&D_{2}F_{2}D\tilde{\theta}_{2}D_{1}P_{1}
+D_{1}F_{2}=f_{2}\left(2\right)\frac{1}{1-\tilde{\mu}_{2}}\tilde{\mu}_{1}\\
D_{2}F_{2}&=&0\\
D_{3}F_{2}&=&D_{2}F_{2}D\tilde{\theta}_{2}D_{3}P_{3}
=f_{2}\left(2\right)\frac{1}{1-\tilde{\mu}_{2}}\hat{\mu}_{1}\\
D_{4}F_{2}&=&D_{2}F_{2}D\tilde{\theta}_{2}D_{4}P_{4}
=f_{2}\left(2\right)\frac{1}{1-\tilde{\mu}_{2}}\hat{\mu}_{2}
\end{eqnarray}



\begin{eqnarray*}
D_{i}\hat{F}_{1}&=&\indora_{i\neq3}D_{3}\hat{F}_{1}D\hat{\theta}_{1}D_{i}P_{i}+\indora_{i=4}D_{i}\hat{F}_{1},
\end{eqnarray*}

\begin{eqnarray}
D_{1}\hat{F}_{1}&=&D_{3}\hat{F}_{1}D\hat{\theta}_{1}D_{1}P_{1}=\hat{f}_{1}\left(3\right)\frac{1}{1-\hat{\mu}_{1}}\tilde{\mu}_{1}
=\\
D_{2}\hat{F}_{1}&=&D_{3}\hat{F}_{1}D\hat{\theta}_{1}D_{2}P_{2}
=\hat{f}_{1}\left(3\right)\frac{1}{1-\hat{\mu}_{1}}\tilde{\mu}_{2}\\
D_{3}\hat{F}_{1}&=&0\\
D_{4}\hat{F}_{1}&=&D_{3}\hat{F}_{1}D\hat{\theta}_{1}D_{4}P_{4}
+D_{4}\hat{F}_{1}
=\hat{f}_{1}\left(3\right)\frac{1}{1-\hat{\mu}_{1}}\hat{\mu}_{2}+\hat{f}_{1}\left(2\right),
\end{eqnarray}


\begin{eqnarray*}
D_{i}\hat{F}_{2}&=&\indora_{i\neq4}D_{4}\hat{F}_{2}D\hat{\theta}_{2}D_{i}P_{i}+\indora_{i=3}D_{i}\hat{F}_{2}.
\end{eqnarray*}

\begin{eqnarray}
D_{1}\hat{F}_{2}&=&D_{4}\hat{F}_{2}D\hat{\theta}_{2}D_{1}P_{1}
=\hat{f}_{2}\left(4\right)\frac{1}{1-\hat{\mu}_{2}}\tilde{\mu}_{1}\\
D_{2}\hat{F}_{2}&=&D_{4}\hat{F}_{2}D\hat{\theta}_{2}D_{2}P_{2}
=\hat{f}_{2}\left(4\right)\frac{1}{1-\hat{\mu}_{2}}\tilde{\mu}_{2}\\
D_{3}\hat{F}_{2}&=&D_{4}\hat{F}_{2}D\hat{\theta}_{2}D_{3}P_{3}+D_{3}\hat{F}_{2}
=\hat{f}_{2}\left(4\right)\frac{1}{1-\hat{\mu}_{2}}\hat{\mu}_{1}+\hat{f}_{2}\left(4\right)\\
D_{4}\hat{F}_{2}&=&0
\end{eqnarray}
Then, now we can obtain the linear system of equations in order to obtain the first moments of the lengths of the queues:



For $\mathbf{F}_{1}=R_{2}F_{2}\hat{F}_{2}$ we get the general equations

\begin{eqnarray*}
D_{i}\mathbf{F}_{1}=D_{i}\left(R_{2}+F_{2}+\indora_{i\geq3}\hat{F}_{2}\right)
\end{eqnarray*}

So

\begin{eqnarray*}
D_{1}\mathbf{F}_{1}&=&D_{1}R_{2}+D_{1}F_{2}
=r_{1}\tilde{\mu}_{1}+f_{2}\left(2\right)\frac{1}{1-\tilde{\mu}_{2}}\tilde{\mu}_{1}\\
D_{2}\mathbf{F}_{1}&=&D_{2}\left(R_{2}+F_{2}\right)
=r_{2}\tilde{\mu}_{1}\\
\end{eqnarray*}


\begin{eqnarray*}
D_{3}\mathbf{F}_{1}&=&D_{3}\left(R_{2}+F_{2}+\hat{F}_{2}\right)
=r_{1}\hat{\mu}_{1}+f_{2}\left(2\right)\frac{1}{1-\tilde{\mu}_{2}}\hat{\mu}_{1}+\hat{F}_{1,2}^{(1)}\left(1\right)
\end{eqnarray*}


\begin{eqnarray*}
D_{4}\mathbf{F}_{1}&=&D_{4}\left(R_{2}+F_{2}+\hat{F}_{2}\right)
\end{eqnarray*}





\begin{eqnarray}
\begin{array}{ll}
\mathbf{F}_{2}=R_{1}F_{1}\hat{F}_{1}, & D_{i}\mathbf{F}_{2}=D_{i}\left(R_{1}+F_{1}+\indora_{i\geq3}\hat{F}_{1}\right)\\
\hat{\mathbf{F}}_{1}=\hat{R}_{2}\hat{F}_{2}F_{2}, & D_{i}\hat{\mathbf{F}}_{1}=D_{i}\left(\hat{R}_{2}+\hat{F}_{2}+\indora_{i\leq2}F_{2}\right)\\
\hat{\mathbf{F}}_{2}=\hat{R}_{1}\hat{F}_{1}F_{1}, & D_{i}\hat{\mathbf{F}}_{2}=D_{i}\left(\hat{R}_{1}+\hat{F}_{1}+\indora_{i\leq2}F_{1}\right)
\end{array}
\end{eqnarray}



equivalently


\begin{eqnarray*}
\begin{array}{ll}
D_{1}\mathbf{F}_{2}=r_{1}\tilde{\mu}_{1},&
D_{2}\mathbf{F}_{2}=r_{1}\tilde{\mu}_{2}+f_{1}\left(1\right)\left(\frac{1}{1-\tilde{\mu}_{1}}\right)\tilde{\mu}_{2}+f_{1}\left(2\right),\\
D_{3}\mathbf{F}_{2}=r_{1}\hat{\mu}_{1}+f_{1}\left(1\right)\left(\frac{1}{1-\tilde{\mu}_{1}}\right)\hat{\mu}_{1}+\hat{F}_{1,1}^{(1)}\left(1\right),&
D_{4}\mathbf{F}_{2}=r_{1}\hat{\mu}_{2}+f_{1}\left(1\right)\left(\frac{1}{1-\tilde{\mu}_{1}}\right)\hat{\mu}_{2}+\hat{F}_{2,1}^{(1)}\left(1\right),\\
D_{1}\mathbf{F}_{1}=r_{2}\hat{\mu}_{1}+f_{2}\left(2\right)\left(\frac{1}{1-\tilde{\mu}_{2}}\right)\tilde{\mu}_{1}+f_{2}\left(1\right),&
D_{2}\mathbf{F}_{1}=r_{2}\tilde{\mu}_{2},\\
D_{3}\mathbf{F}_{1}=r_{2}\hat{\mu}_{1}+f_{2}\left(2\right)\left(\frac{1}{1-\tilde{\mu}_{2}}\right)\hat{\mu}_{1}+\hat{F}_{1,2}^{(1)}\left(1\right),&
D_{4}\mathbf{F}_{1}=r_{2}\hat{\mu}_{2}+f_{2}\left(2\right)\left(\frac{1}{1-\tilde{\mu}_{2}}\right)\hat{\mu}_{2}+\hat{F}_{2,2}^{(1)}\left(1\right),\\
D_{1}\hat{\mathbf{F}}_{2}=\hat{r}_{1}\tilde{\mu}_{1}+\hat{f}_{1}\left(1\right)\left(\frac{1}{1-\hat{\mu}_{1}}\right)\tilde{\mu}_{1}+F_{1,1}^{(1)}\left(1\right),&
D_{2}\hat{\mathbf{F}}_{2}=\hat{r}_{1}\mu_{2}+\hat{f}_{1}\left(1\right)\left(\frac{1}{1-\hat{\mu}_{1}}\right)\tilde{\mu}_{2}+F_{2,1}^{(1)}\left(1\right),\\
D_{3}\hat{\mathbf{F}}_{2}=\hat{r}_{1}\hat{\mu}_{1},&
D_{4}\hat{\mathbf{F}}_{2}=\hat{r}_{1}\hat{\mu}_{2}+\hat{f}_{1}\left(1\right)\left(\frac{1}{1-\hat{\mu}_{1}}\right)\hat{\mu}_{2}+\hat{f}_{1}\left(2\right),\\
D_{1}\hat{\mathbf{F}}_{1}=\hat{r}_{2}\tilde{\mu}_{1}+\hat{f}_{2}\left(2\right)\left(\frac{1}{1-\hat{\mu}_{2}}\right)\tilde{\mu}_{1}+F_{1,2}^{(1)}\left(1\right),&
D_{2}\hat{\mathbf{F}}_{1}=\hat{r}_{2}\tilde{\mu}_{2}+\hat{f}_{2}\left(2\right)\left(\frac{1}{1-\hat{\mu}_{2}}\right)\tilde{\mu}_{2}+F_{2,2}^{(1)}\left(1\right),\\
D_{3}\hat{\mathbf{F}}_{1}=\hat{r}_{2}\hat{\mu}_{1}+\hat{f}_{2}\left(2\right)\left(\frac{1}{1-\hat{\mu}_{2}}\right)\hat{\mu}_{1}+\hat{f}_{2}\left(1\right),&
D_{4}\hat{\mathbf{F}}_{1}=\hat{r}_{2}\hat{\mu}_{2}
\end{array}
\end{eqnarray*}


Then we have that if $\mu=\tilde{\mu}_{1}+\tilde{\mu}_{2}$, $\hat{\mu}=\hat{\mu}_{1}+\hat{\mu}_{2}$, $r=r_{1}+r_{2}$ and $\hat{r}=\hat{r}_{1}+\hat{r}_{2}$  the system's solution is given by

\begin{eqnarray*}
\begin{array}{llll}
f_{2}\left(1\right)=r_{1}\tilde{\mu}_{1},&
f_{1}\left(2\right)=r_{2}\tilde{\mu}_{2},&
\hat{f}_{1}\left(4\right)=\hat{r}_{2}\hat{\mu}_{2},&
\hat{f}_{2}\left(3\right)=\hat{r}_{1}\hat{\mu}_{1}
\end{array}
\end{eqnarray*}



it's easy to verify that

\begin{eqnarray*}\label{Sist.Ec.Lineales.Doble.Traslado}
\begin{array}{ll}
f_{1}\left(1\right)=\tilde{\mu}_{1}\left(r+\frac{f_{2}\left(2\right)}{1-\tilde{\mu}_{2}}\right),& f_{1}\left(3\right)=\hat{\mu}_{1}\left(r_{2}+\frac{f_{2}\left(2\right)}{1-\tilde{\mu}_{2}}\right)+\hat{F}_{1,2}^{(1)}\left(1\right)\\
f_{1}\left(4\right)=\hat{\mu}_{2}\left(r_{2}+\frac{f_{2}\left(2\right)}{1-\tilde{\mu}_{2}}\right)+\hat{F}_{2,2}^{(1)}\left(1\right),&
f_{2}\left(2\right)=\left(r+\frac{f_{1}\left(1\right)}{1-\mu_{1}}\right)\tilde{\mu}_{2},\\
f_{2}\left(3\right)=\hat{\mu}_{1}\left(r_{1}+\frac{f_{1}\left(1\right)}{1-\tilde{\mu}_{1}}\right)+\hat{F}_{1,1}^{(1)}\left(1\right),&
f_{2}\left(4\right)=\hat{\mu}_{2}\left(r_{1}+\frac{f_{1}\left(1\right)}{1-\mu_{1}}\right)+\hat{F}_{2,1}^{(1)}\left(1\right),\\
\hat{f}_{1}\left(1\right)=\left(\hat{r}_{2}+\frac{\hat{f}_{2}\left(4\right)}{1-\hat{\mu}_{2}}\right)\tilde{\mu}_{1}+F_{1,2}^{(1)}\left(1\right),&
\hat{f}_{1}\left(2\right)=\left(\hat{r}_{2}+\frac{\hat{f}_{2}\left(4\right)}{1-\hat{\mu}_{2}}\right)\tilde{\mu}_{2}+F_{2,2}^{(1)}\left(1\right),\\
\hat{f}_{1}\left(3\right)=\left(\hat{r}+\frac{\hat{f}_{2}\left(4\right)}{1-\hat{\mu}_{2}}\right)\hat{\mu}_{1},&
\hat{f}_{2}\left(1\right)=\left(\hat{r}_{1}+\frac{\hat{f}_{1}\left(3\right)}{1-\hat{\mu}_{1}}\right)\mu_{1}+F_{1,1}^{(1)}\left(1\right),\\
\hat{f}_{2}\left(2\right)=\left(\hat{r}_{1}+\frac{\hat{f}_{1}\left(3\right)}{1-\hat{\mu}_{1}}\right)\tilde{\mu}_{2}+F_{2,1}^{(1)}\left(1\right),&
\hat{f}_{2}\left(4\right)=\left(\hat{r}+\frac{\hat{f}_{1}\left(3\right)}{1-\hat{\mu}_{1}}\right)\hat{\mu}_{2},\\
\end{array}
\end{eqnarray*}

with system's solutions given by

\begin{eqnarray*}
\begin{array}{ll}
f_{1}\left(1\right)=r\frac{\mu_{1}\left(1-\mu_{1}\right)}{1-\mu},&
f_{2}\left(2\right)=r\frac{\tilde{\mu}_{2}\left(1-\tilde{\mu}_{2}\right)}{1-\mu},\\
f_{1}\left(3\right)=\hat{\mu}_{1}\left(r_{2}+\frac{r\tilde{\mu}_{2}}{1-\mu}\right)+\hat{F}_{1,2}^{(1)}\left(1\right),&
f_{1}\left(4\right)=\hat{\mu}_{2}\left(r_{2}+\frac{r\tilde{\mu}_{2}}{1-\mu}\right)+\hat{F}_{2,2}^{(1)}\left(1\right),\\
f_{2}\left(3\right)=\hat{\mu}_{1}\left(r_{1}+\frac{r\mu_{1}}{1-\mu}\right)+\hat{F}_{1,1}^{(1)}\left(1\right),&
f_{2}\left(4\right)=\hat{\mu}_{2}\left(r_{1}+\frac{r\mu_{1}}{1-\mu}\right)+\hat{F}_{2,1}^{(1)}\left(1\right),\\
\hat{f}_{1}\left(1\right)=\tilde{\mu}_{1}\left(\hat{r}_{2}+\frac{\hat{r}\hat{\mu}_{2}}{1-\hat{\mu}}\right)+F_{1,2}^{(1)}\left(1\right),&
\hat{f}_{1}\left(2\right)=\tilde{\mu}_{2}\left(\hat{r}_{2}+\frac{\hat{r}\hat{\mu}_{2}}{1-\hat{\mu}}\right)+F_{2,2}^{(1)}\left(1\right),\\
\hat{f}_{2}\left(1\right)=\tilde{\mu}_{1}\left(\hat{r}_{1}+\frac{\hat{r}\hat{\mu}_{1}}{1-\hat{\mu}}\right)+F_{1,1}^{(1)}\left(1\right),&
\hat{f}_{2}\left(2\right)=\tilde{\mu}_{2}\left(\hat{r}_{1}+\frac{\hat{r}\hat{\mu}_{1}}{1-\hat{\mu}}\right)+F_{2,1}^{(1)}\left(1\right)
\end{array}
\end{eqnarray*}

%_________________________________________________________________________________________________________
\subsection*{General Second Order Derivatives}
%_________________________________________________________________________________________________________


Now, taking the second order derivative with respect to the equations given in (\ref{Sist.Ec.Lineales.Doble.Traslado}) we obtain it in their general form

\small{
\begin{eqnarray*}\label{Ec.Derivadas.Segundo.Orden.Doble.Transferencia}
D_{k}D_{i}F_{1}&=&D_{k}D_{i}\left(R_{2}+F_{2}+\indora_{i\geq3}\hat{F}_{4}\right)+D_{i}R_{2}D_{k}\left(F_{2}+\indora_{k\geq3}\hat{F}_{4}\right)+D_{i}F_{2}D_{k}\left(R_{2}+\indora_{k\geq3}\hat{F}_{4}\right)+\indora_{i\geq3}D_{i}\hat{F}_{4}D_{k}\left(R_{}+F_{2}\right)\\
D_{k}D_{i}F_{2}&=&D_{k}D_{i}\left(R_{1}+F_{1}+\indora_{i\geq3}\hat{F}_{3}\right)+D_{i}R_{1}D_{k}\left(F_{1}+\indora_{k\geq3}\hat{F}_{3}\right)+D_{i}F_{1}D_{k}\left(R_{1}+\indora_{k\geq3}\hat{F}_{3}\right)+\indora_{i\geq3}D_{i}\hat{F}_{3}D_{k}\left(R_{1}+F_{1}\right)\\
D_{k}D_{i}\hat{F}_{3}&=&D_{k}D_{i}\left(\hat{R}_{4}+\indora_{i\leq2}F_{2}+\hat{F}_{4}\right)+D_{i}\hat{R}_{4}D_{k}\left(\indora_{k\leq2}F_{2}+\hat{F}_{4}\right)+D_{i}\hat{F}_{4}D_{k}\left(\hat{R}_{4}+\indora_{k\leq2}F_{2}\right)+\indora_{i\leq2}D_{i}F_{2}D_{k}\left(\hat{R}_{4}+\hat{F}_{4}\right)\\
D_{k}D_{i}\hat{F}_{4}&=&D_{k}D_{i}\left(\hat{R}_{3}+\indora_{i\leq2}F_{1}+\hat{F}_{3}\right)+D_{i}\hat{R}_{3}D_{k}\left(\indora_{k\leq2}F_{1}+\hat{F}_{3}\right)+D_{i}\hat{F}_{3}D_{k}\left(\hat{R}_{3}+\indora_{k\leq2}F_{1}\right)+\indora_{i\leq2}D_{i}F_{1}D_{k}\left(\hat{R}_{3}+\hat{F}_{3}\right)
\end{eqnarray*}}
for $i,k=1,\ldots,4$. In order to have it in an specific way we need to compute the expressions $D_{k}D_{i}\left(R_{2}+F_{2}+\indora_{i\geq3}\hat{F}_{4}\right)$

%_________________________________________________________________________________________________________
\subsubsection*{Second Order Derivatives: Serve's Switchover Times}
%_________________________________________________________________________________________________________

Remind $R_{i}\left(z_{1},z_{2},w_{1},w_{2}\right)=R_{i}\left(P_{1}\left(z_{1}\right)\tilde{P}_{2}\left(z_{2}\right)
\hat{P}_{1}\left(w_{1}\right)\hat{P}_{2}\left(w_{2}\right)\right)$,  which we will write in his reduced form $R_{i}=R_{i}\left(
P_{1}\tilde{P}_{2}\hat{P}_{1}\hat{P}_{2}\right)$, and according to the notation given in \cite{Lang} we obtain

\begin{eqnarray}
D_{i}D_{i}R_{k}=D^{2}R_{k}\left(D_{i}P_{i}\right)^{2}+DR_{k}D_{i}D_{i}P_{i}
\end{eqnarray}

whereas for $i\neq j$

\begin{eqnarray}
D_{i}D_{j}R_{k}=D^{2}R_{k}D_{i}P_{i}D_{j}P_{j}+DR_{k}D_{j}P_{j}D_{i}P_{i}
\end{eqnarray}

%_________________________________________________________________________________________________________
\subsubsection*{Second Order Derivatives: Queue Lengths}
%_________________________________________________________________________________________________________

Just like before the expression $F_{1}\left(\tilde{\theta}_{1}\left(\tilde{P}_{2}\left(z_{2}\right)\hat{P}_{1}\left(w_{1}\right)\hat{P}_{2}\left(w_{2}\right)\right),
z_{2}\right)$, will be denoted by $F_{1}\left(\tilde{\theta}_{1}\left(\tilde{P}_{2}\hat{P}_{1}\hat{P}_{2}\right),z_{2}\right)$, then the mixed partial derivatives are:

\begin{eqnarray*}
D_{i}F_{1}=\indora_{i\geq2}D_{i}F_{1}D\tilde{\theta}_{1}D_{i}P_{i}+\indora_{i=2} D_{i}F_{1},
\end{eqnarray*}

then for
$F_{1}\left(\tilde{\theta}_{1}\left(\tilde{P}_{2}\hat{P}_{1}\hat{P}_{2}\right),z_{2}\right)$

\begin{eqnarray*}
D_{j}D_{i}F_{1}&=&\indora_{i,j\neq1}D_{1}D_{1}F_{1}\left(D\tilde{\theta}_{1}\right)^{2}D_{i}P_{i}D_{j}P_{j}+\indora_{i,j\neq1}D_{1}F_{1}D^{2}\tilde{\theta}_{1}D_{i}P_{i}D_{j}P_{j}+\indora_{i,j\neq1}D_{1}F_{1}D\tilde{\theta}_{1}\left(\indora_{i=j}D_{i}^{2}P_{i}+\indora_{i\neq j}D_{i}P_{i}D_{j}P_{j}\right)\\
&+&\indora_{i,j\neq1}D_{1}D_{2}F_{1}D\tilde{\theta}_{1}D_{i}P_{i}+\indora_{i=2}\left(D_{1}D_{2}F_{1}D\tilde{\theta}_{1}D_{i}P_{i}+D_{i}^{2}F_{1}\right)
\end{eqnarray*}


Recall the expression for $F_{1}\left(\tilde{\theta}_{1}\left(\tilde{P}_{2}\left(z_{2}\right)\hat{P}_{1}\left(w_{1}\right)\hat{P}_{2}\left(w_{2}\right)\right),
z_{2}\right)$, which is denoted by $F_{1}\left(\tilde{\theta}_{1}\left(\tilde{P}_{2}\hat{P}_{1}\hat{P}_{2}\right),z_{2}\right)$, then the mixed partial derivatives are given by

\begin{eqnarray*}
\begin{array}{llll}
D_{1}D_{1}F_{1}=0,&
D_{2}D_{1}F_{1}=0,&
D_{3}D_{1}F_{1}=0,&
D_{4}D_{1}F_{1}=0,\\
D_{1}D_{2}F_{1}=0,&
D_{1}D_{3}F_{1}=0,&
D_{1}D_{4}F_{1}=0,&
\end{array}
\end{eqnarray*}

\begin{eqnarray*}
D_{2}D_{2}F_{1}&=&D_{1}^{2}F_{1}\left(D\tilde{\theta}_{1}\right)^{2}\left(D_{2}\tilde{P}_{2}\right)^{2}
+D_{1}F_{1}D^{2}\tilde{\theta}_{1}D_{2}^{2}\tilde{P}_{2}
+D_{1}F_{1}D\tilde{\theta}_{1}D_{2}^{2}\tilde{P}_{2}
+D_{1}D_{2}F_{1}D\tilde{\theta}_{1}D_{2}\tilde{P}_{2}\\
&+&D_{1}D_{2}F_{1}D\tilde{\theta}_{1}D_{2}\tilde{P}_{2}+D_{2}D_{2}F_{1}\\
&=&f_{1}\left(1,1\right)\left(\frac{\tilde{\mu}_{2}}{1-\tilde{\mu}_{1}}\right)^{2}+f_{1}\left(1\right)\tilde{\theta}_{1}^{(2)}\tilde{P}_{2}^{(2)}+f_{1}\left(1\right)\frac{1}{1-\tilde{\mu}_{1}}\tilde{P}_{2}^{(2)}+f_{1}\left(1,2\right)\frac{\tilde{\mu}_{2}}{1-\tilde{\mu}_{1}}+f_{1}\left(1,2\right)\frac{\tilde{\mu}_{2}}{1-\tilde{\mu}_{1}}+f_{1}\left(2,2\right)
\end{eqnarray*}

\begin{eqnarray*}
D_{3}D_{2}F_{1}&=&D_{1}^{2}F_{1}\left(D\tilde{\theta}_{1}\right)^{2}D_{3}\hat{P}_{1}D_{2}\tilde{P}_{2}+D_{1}F_{1}D^{2}\tilde{\theta}_{1}D_{3}\hat{P}_{1}D_{2}\tilde{P}_{2}+D_{1}F_{1}D\tilde{\theta}_{1}D_{2}\tilde{P}_{2}D_{3}\hat{P}_{1}+D_{1}D_{2}F_{1}D\tilde{\theta}_{1}D_{3}\hat{P}_{1}\\
&=&f_{1}\left(1,1\right)\left(\frac{1}{1-\tilde{\mu}_{1}}\right)^{2}\tilde{\mu}_{2}\hat{\mu}_{1}+f_{1}\left(1\right)\tilde{\theta}_{1}^{(2)}\tilde{\mu}_{2}\hat{\mu}_{1}+f_{1}\left(1\right)\frac{\tilde{\mu}_{2}\hat{\mu}_{1}}{1-\tilde{\mu}_{1}}+f_{1}\left(1,2\right)\frac{\hat{\mu}_{1}}{1-\tilde{\mu}_{1}}
\end{eqnarray*}

\begin{eqnarray*}
D_{4}D_{2}F_{1}&=&D_{1}^{2}F_{1}\left(D\tilde{\theta}_{1}\right)^{2}D_{4}\hat{P}_{2}D_{2}\tilde{P}_{2}+D_{1}F_{1}D^{2}\tilde{\theta}_{1}D_{4}\hat{P}_{2}D_{2}\tilde{P}_{2}+D_{1}F_{1}D\tilde{\theta}_{1}D_{2}\tilde{P}_{2}D_{4}\hat{P}_{2}+D_{1}D_{2}F_{1}D\tilde{\theta}_{1}D_{4}\hat{P}_{2}\\
&=&f_{1}\left(1,1\right)\left(\frac{1}{1-\tilde{\mu}_{1}}\right)^{2}\tilde{\mu}_{2}\hat{\mu}_{2}+f_{1}\left(1\right)\tilde{\theta}_{1}^{(2)}\tilde{\mu}_{2}\hat{\mu}_{2}+f_{1}\left(1\right)\frac{\tilde{\mu}_{2}\hat{\mu}_{2}}{1-\tilde{\mu}_{1}}+f_{1}\left(1,2\right)\frac{\hat{\mu}_{2}}{1-\tilde{\mu}_{1}}
\end{eqnarray*}

\begin{eqnarray*}
D_{2}D_{3}F_{1}&=&
D_{1}^{2}F_{1}\left(D\tilde{\theta}_{1}\right)^{2}D_{2}\tilde{P}_{2}D_{3}\hat{P}_{1}+
D_{2}D_{1}F_{1}D\tilde{\theta}_{1}D_{3}\hat{P}_{1}+
D_{1}F_{1}D^{2}\tilde{\theta}_{1}D_{2}\tilde{P}_{2}D_{3}\hat{P}_{1}+
D_{1}F_{1}D\tilde{\theta}_{1}D_{3}\hat{P}_{1}D_{2}\tilde{P}_{2}\\
&=&f_{1}\left(1,1\right)\left(\frac{1}{1-\tilde{\mu}_{1}}\right)^{2}\tilde{\mu}_{2}\hat{\mu}_{1}+f_{1}\left(1\right)\tilde{\theta}_{1}^{(2)}\tilde{\mu}_{2}\hat{\mu}_{1}+f_{1}\left(1\right)\frac{\tilde{\mu}_{2}\hat{\mu}_{1}}{1-\tilde{\mu}_{1}}+f_{1}\left(1,2\right)\frac{\hat{\mu}_{1}}{1-\tilde{\mu}_{1}}
\end{eqnarray*}

\begin{eqnarray*}
D_{3}D_{3}F_{1}&=&D_{1}^{2}F_{1}\left(D\tilde{\theta}_{1}\right)^{2}\left(D_{3}\hat{P}_{1}\right)^{2}+D_{1}F_{1}D^{2}\tilde{\theta}_{1}\left(D_{3}\hat{P}_{1}\right)^{2}+D_{1}F_{1}D\tilde{\theta}_{1}D_{3}^{2}\hat{P}_{1}\\
&=&f_{1}\left(1,1\right)\left(\frac{\hat{\mu}_{1}}{1-\tilde{\mu}_{1}}\right)^{2}+f_{1}\left(1\right)\tilde{\theta}_{1}^{(2)}\hat{\mu}_{1}^{2}+f_{1}\left(1\right)\frac{\hat{\mu}_{1}^{2}}{1-\tilde{\mu}_{1}}
\end{eqnarray*}

\begin{eqnarray*}
D_{4}D_{3}F_{1}&=&D_{1}^{2}F_{1}\left(D\tilde{\theta}_{1}\right)^{2}D_{4}\hat{P}_{2}D_{3}\hat{P}_{1}+D_{1}F_{1}D^{2}\tilde{\theta}_{1}D_{4}\hat{P}_{2}D_{3}\hat{P}_{1}+D_{1}F_{1}D\tilde{\theta}_{1}D_{3}\hat{P}_{1}D_{4}\hat{P}_{2}\\
&=&f_{1}\left(1,1\right)\left(\frac{1}{1-\tilde{\mu}_{1}}\right)^{2}\hat{\mu}_{1}\hat{\mu}_{2}+f_{1}\left(1\right)\left(\tilde{\theta}_{1}\right)^{2}\hat{\mu}_{2}\hat{\mu}_{1}+f_{1}\left(1\right)\frac{\hat{\mu}_{2}\hat{\mu}_{1}}{1-\tilde{\mu}_{1}}
\end{eqnarray*}

\begin{eqnarray*}
D_{2}D_{4}F_{1}&=&D_{1}^{2}F_{1}\left(D\tilde{\theta}_{1}\right)^{2}D_{2}\tilde{P}_{2}D_{4}\hat{P}_{2}+D_{1}F_{1}D^{2}\tilde{\theta}_{1}D_{2}\tilde{P}_{2}D_{4}\hat{P}_{2}+D_{1}F_{1}D\tilde{\theta}_{1}D_{4}\hat{P}_{2}D_{2}\tilde{P}_{2}+D_{2}D_{1}F_{1}D\tilde{\theta}_{1}D_{4}\hat{P}_{2}\\
&=&f_{1}\left(1,1\right)\left(\frac{1}{1-\tilde{\mu}_{1}}\right)^{2}\hat{\mu}_{2}\tilde{\mu}_{2}+f_{1}\left(1,1\right)\tilde{\mu}_{1}^{(2)}\hat{\mu}_{2}\tilde{\mu}_{2}+f_{1}\left(1\right)\frac{\hat{\mu}_{2}\tilde{\mu}_{2}}{1-\tilde{\theta}_{1}}+f_{1}\left(1,2\right)\frac{\hat{\mu}_{2}}{1-\tilde{\mu}_{1}}
\end{eqnarray*}

\begin{eqnarray*}
D_{3}D_{4}F_{1}&=&D_{1}^{2}F_{1}\left(D\tilde{\theta}_{1}\right)^{2}D_{3}\hat{P}_{1}D_{4}\hat{P}_{2}+D_{1}F_{1}D^{2}\tilde{\theta}_{1}D_{3}\hat{P}_{1}D_{4}\hat{P}_{2}+D_{1}F_{1}D\tilde{\theta}_{1}D_{4}\hat{P}_{2}D_{3}\hat{P}_{1}\\
&=&f_{1}\left(1,1\right)\left(\frac{1}{1-\tilde{\mu}_{1}}\right)^{2}\hat{\mu}_{1}\hat{\mu}_{2}+f_{1}\left(1\right)\tilde{\theta}_{1}^{(2)}\hat{\mu}_{1}\hat{\mu}_{2}+f_{1}\left(1\right)\frac{\hat{\mu}_{1}\hat{\mu}_{2}}{1-\tilde{\mu}_{1}}
\end{eqnarray*}

\begin{eqnarray*}
D_{4}D_{4}F_{1}&=&D_{1}^{2}F_{1}\left(D\tilde{\theta}_{1}\right)^{2}\left(D_{4}\hat{P}_{2}\right)^{2}+D_{1}F_{1}D^{2}\tilde{\theta}_{1}\left(D_{4}\hat{P}_{2}\right)^{2}+D_{1}F_{1}D\tilde{\theta}_{1}D_{4}^{2}\hat{P}_{2}\\
&=&f_{1}\left(1,1\right)\left(\frac{\hat{\mu}_{2}}{1-\tilde{\mu}_{1}}\right)^{2}+f_{1}\left(1\right)\tilde{\theta}_{1}^{(2)}\left(\hat{\mu}_{2}\right)^{2}+f_{1}\left(1\right)\frac{1}{1-\tilde{\mu}_{1}}\hat{P}_{2}^{(2)}
\end{eqnarray*}



Para $F_{2}\left(z_{1},\tilde{\theta}_{2}\left(P_{1}\hat{P}_{1}\hat{P}_{2}\right)\right)$

\begin{eqnarray*}
D_{j}D_{i}F_{2}&=&\indora_{i,j\neq2}D_{2}D_{21}F_{2}\left(D\theta_{2}\right)^{2}D_{i}P_{i}D_{j}P_{j}+\indora_{i,j\neq2}D_{2}F_{2}D^{2}\theta_{2}D_{i}P_{i}D_{j}P_{j}\\
&+&\indora_{i,j\neq2}D_{2}F_{2}D\theta_{2}\left(\indora_{i=j}D_{i}^{2}P_{i}+\indora_{i\neq j}D_{i}P_{i}D_{j}P_{j}\right)\\
&+&\indora_{i,j\neq2}D_{2}D_{1}F_{2}D\theta_{2}D_{i}P_{i}+\indora_{i=2}\left(D_{2}D_{1}F_{2}D\theta_{2}D_{i}P_{i}+D_{i}^{2}F_{2}\right)
\end{eqnarray*}

\begin{eqnarray*}
\begin{array}{llll}
D_{2}D_{1}F_{2}=0,&
D_{2}D_{3}F_{3}=0,&
D_{2}D_{4}F_{2}=0,&\\
D_{1}D_{2}F_{2}=0,&
D_{2}D_{2}F_{2}=0,&
D_{3}D_{2}F_{2}=0,&
D_{4}D_{2}F_{2}=0\\
\end{array}
\end{eqnarray*}


\begin{eqnarray*}
D_{1}D_{1}F_{2}&=&
D_{1}^{2}P_{1}D\tilde{\theta}_{2}D_{2}F_{2}+
\left(D_{1}P_{1}\right)^{2}D^{2}\tilde{\theta}_{2}D_{2}F_{2}+
D_{1}P_{1}D\tilde{\theta}_{2}D_{1}D_{2}F_{2}+
\left(D_{1}P_{1}\right)^{2}\left(D\tilde{\theta}_{2}\right)^{2}D_{2}^{2}F_{2}+
D_{1}P_{1}D\tilde{\theta}_{2}D_{1}D_{2}F_{2}+
D_{1}^{2}F_{2}\\
D_{3}D_{1}F_{2}&=&D_{2}D_{1}F_{2}D\tilde{\theta}_{2}D_{3}\hat{P}_{1}
+D_{2}^{2}F_{2}\left(D\tilde{\theta}_{2}\right)^{2}D_{3}P_{1}D_{1}P_{1}
+D_{2}F_{2}D^{2}\tilde{\theta}_{2}D_{3}\hat{P}_{1}D_{1}P_{1}
+D_{2}F_{2}D\tilde{\theta}_{2}D_{1}P_{1}D_{3}\hat{P}_{1}\\
D_{4}D_{1}F_{2}&=&D_{1}D_{2}F_{2}D\tilde{\theta}_{2}D_{4}\hat{P}_{2}
+D_{2}^{2}F_{2}\left(D\tilde{\theta}_{2}\right)^{2}D_{4}P_{2}D_{1}P_{1}
+D_{2}F_{2}D^{2}\tilde{\theta}_{2}D_{4}\hat{P}_{2}D_{1}P_{1}
+D_{2}F_{2}D\tilde{\theta}_{2}D_{1}P_{1}D_{4}\hat{P}_{2}\\
D_{1}D_{3}F_{2}&=&D_{2}^{2}F_{2}\left(D\tilde{\theta}_{2}\right)^{2}D_{1}P_{1}D_{3}\hat{P}_{1}
+D_{2}D_{1}F_{2}D\tilde{\theta}_{2}D_{3}\hat{P}_{1}
+D_{2}F_{2}D^{2}\tilde{\theta}_{2}D_{1}P_{1}D_{3}\hat{P}_{1}
+D_{2}F_{2}D\tilde{\theta}_{2}D_{3}\hat{P}_{1}D_{1}P_{1}\\
D_{3}D_{3}F_{2}&=&D_{2}^{2}F_{2}\left(D\tilde{\theta}_{2}\right)^{2}\left(D_{3}\hat{P}_{1}\right)^{2}
+D_{2}F_{2}\left(D_{3}\hat{P}_{1}\right)^{2}D^{2}\tilde{\theta}_{2}
+D_{2}F_{2}D\tilde{\theta}_{2}D_{3}^{2}\hat{P}_{1}\\
D_{4}D_{3}F_{2}&=&D_{2}^{2}F_{2}\left(D\tilde{\theta}_{2}\right)^{2}D_{4}\hat{P}_{2}D_{3}\hat{P}_{1}
+D_{2}F_{2}D^{2}\tilde{\theta}_{2}D_{4}\hat{P}_{2}D_{3}\hat{P}_{1}
+D_{2}F_{2}D\tilde{\theta}_{2}D_{3}\hat{P}_{1}D_{4}\hat{P}_{2}\\
D_{1}D_{4}F_{2}&=&D_{2}^{2}F_{2}\left(D\tilde{\theta}_{2}\right)^{2}D_{1}P_{1}D_{4}\hat{P}_{2}
+D_{1}D_{2}F_{2}D\tilde{\theta}_{2}D_{4}\hat{P}_{2}
+D_{2}F_{2}D^{2}\tilde{\theta}_{2}D_{1}P_{1}D_{4}\hat{P}_{2}
+D_{2}F_{2}D\tilde{\theta}_{2}D_{4}\hat{P}_{2}D_{1}P_{1}\\
D_{3}D_{4}F_{2}&=&
D_{2}F_{2}D\tilde{\theta}_{2}D_{4}\hat{P}_{2}D_{3}\hat{P}_{1}
+D_{2}F_{2}D^{2}\tilde{\theta}_{2}D_{4}\hat{P}_{2}D_{3}\hat{P}_{1}
+D_{2}^{2}F_{2}\left(D\tilde{\theta}_{2}\right)^{2}D_{4}\hat{P}_{2}D_{3}\hat{P}_{1}\\
D_{4}D_{4}F_{2}&=&D_{2}F_{2}D\tilde{\theta}_{2}D_{4}^{2}\hat{P}_{2}
+D_{2}F_{2}D^{2}\tilde{\theta}_{2}\left(D_{4}\hat{P}_{2}\right)^{2}
+D_{2}^{2}F_{2}\left(D\tilde{\theta}_{2}\right)^{2}\left(D_{4}\hat{P}_{2}\right)^{2}\\
\end{eqnarray*}


%\newpage



%\newpage

para $\hat{F}_{1}\left(\hat{\theta}_{1}\left(P_{1}\tilde{P}_{2}\hat{P}_{2}\right),w_{2}\right)$

\begin{eqnarray*}
D_{i}\hat{F}_{1}=\indora_{i\neq3}D_{3}\hat{F}_{1}D\hat{\theta}_{1}D_{i}P_{i}+\indora_{i=4}D_{i}\hat{F}_{1},
\end{eqnarray*}


\begin{eqnarray*}
D_{1}D_{1}\hat{F}_{1}&=&
D\hat{\theta}_{1}D_{1}^{2}P_{1}D_{1}\hat{F}_{1}
+\left(D_{1}P_{1}\right)^{2}D^{2}\hat{\theta}_{1}D_{1}\hat{F}_{1}
+\left(D_{1}P_{1}\right)^{2}\left(D\hat{\theta}_{1}\right)^{2}D_{1}^{2}\hat{F}_{1}\\
D_{2}D_{1}\hat{F}_{1}&=&D_{1}P_{1}D_{2}P_{2}D\hat{\theta}_{1}D_{1}\hat{F}_{1}+
D_{1}P_{1}D_{2}P_{2}D^{2}\hat{\theta}_{1}D_{1}\hat{F}_{1}+
D_{1}P_{1}D_{2}P_{1}\left(D\hat{\theta}_{1}\right)^{2}D_{1}^{2}\hat{\theta}_{1}\\
D_{3}D_{1}\hat{F}_{1}&=&0\\
D_{4}D_{1}\hat{F}_{1}&=&D_{1}P_{1}D_{4}\hat{P}_{2}D\hat{\theta}_{1}D_{1}\hat{F}_{1}
+D_{1}P_{1}D_{4}\hat{P}_{2}D^{2}\hat{\theta}_{1}D_{1}\hat{F}_{1}
+D_{1}P_{1}D\hat{\theta}_{1}D_{2}D{1}\hat{F}_{1}
+D_{1}P_{1}D\hat{\theta}_{1}D_{1}D_{1}\hat{F}_{1}\\
D_{1}D_{2}\hat{F}_{1}&=&D_{1}P_{1}D_{2}P_{2}D\hat{\theta}_{1}D_{1}\hat{F}_{1}+
D_{1}P_{1}D_{2}P_{2}D^{2}\hat{\theta}_{1}D_{1}\hat{F}_{1}+
D_{1}P_{1}D_{2}P_{2}\left(D\hat{\theta}_{1}\right)^{2}D_{1}^{2}\hat{F}_{1}\\
D_{2}D_{2}\hat{F}_{1}&=&
D\hat{\theta}_{1}D_{2}^{2}P_{2}D_{1}\hat{F}_{1}+
 \left(D_{2}P_{2}\right)^{2}D^{2}\hat{\theta}_{1}D_{1}\hat{F}_{1}+
\left(D_{2}P_{2}\right)^{2}\left(D\hat{\theta}_{1}\right)^{2}D_{1}^{2}\hat{F}_{1}\\
D_{3}D_{2}\hat{F}_{1}&=&0\\
D_{4}D_{2}\hat{F}_{1}&=&D_{2}P_{2}D_{4}\hat{P}_{2}D\hat{\theta}_{1}D\hat{F}_{1}
+D_{2}P_{2}D_{4}\hat{P}_{2}D^{2}\hat{\theta}_{1}D_{1}\hat{F}_{1}
+D_{2}P_{2}D\hat{\theta}_{1}D_{2}D_{1}\hat{F}_{1}
+D_{2}P_{2}\left(D\hat{\theta}_{1}\right)^{2}D_{4}\hat{P}_{2}D_{1}^{2}\hat{F}_{1}\\
D_{1}D_{3}\hat{F}_{1}&=&0\\
D_{2}D_{3}\hat{F}_{1}&=&0\\
D_{3}D_{3}\hat{F}_{1}&=&0\\
D_{4}D_{3}\hat{F}_{1}&=&0\\
D_{1}D_{4}\hat{F}_{1}&=&D_{1}P_{1}D_{4}\hat{F}_{2}D\hat{\theta}_{1}D_{1}\hat{F}_{1}
+D_{1}P_{1}D_{4}\hat{P}_{2}D^{2}\hat{\theta}_{1}D_{1}\hat{F}_{1}
+D_{1}P_{1}D\hat{\theta}_{1}D_{2}D_{1}\hat{F}_{1}
+ D_{1}P_{1}D_{4}\hat{P}_{2}\left(D\hat{\theta}_{1}\right)^{2}D_{1}D_{1}\hat{F}_{1}\\
D_{2}D_{4}\hat{F}_{1}&=&D_{2}P_{2}D_{4}\hat{P}_{2}D\hat{\theta}_{1}D_{1}
\hat{F}_{1}
+D_{2}P_{2}D_{4}\hat{P}_{2}D^{2}\hat{\theta}_{1}D_{1}\hat{F}_{1}
+D_{2}P_{2}D\hat{\theta}_{1}D_{2}D_{1}\hat{F}_{1}+
D_{2}P_{2}D_{4}\hat{P}_{2}\left(D\hat{\theta}_{1}\right)^{2}D_{1}^{2}\hat{F}_{1}\\
D_{3}D_{4}\hat{F}_{1}&=&0\\
D_{4}D_{4}\hat{F}_{1}&=&D_{2}D_{2}\hat{F}_{1}+D\hat{\theta}_{1}D_{4}^{2}\hat{P}_{2}D_{1}\hat{F}_{1}
+\left(D_{4}\hat{P}_{2}\right)^{2}D^{2}\hat{\theta}_{1}D_{1}\hat{F}_{1}+
D_{4}\hat{P}_{2}D\hat{\theta}_{1}D_{2}D_{1}\hat{F}_{1}\\
&+&D_{4}\hat{P}_{2}D\hat{\theta}_{1}D_{2}D_{1}\hat{F}_{1}+ \left(D_{4}\hat{P}_{2}\right)^{2}D\hat{\theta}_{1}D\hat{\theta}_{1}D_{1}^{2}\hat{F}_{1}\\
\end{eqnarray*}




%\newpage
finalmente, para $\hat{F}_{2}\left(w_{1},\hat{\theta}_{2}\left(P_{1}\tilde{P}_{2}\hat{P}_{1}\right)\right)$

\begin{eqnarray*}
D_{i}\hat{F}_{2}=\indora_{i\neq4}D_{4}\hat{F}_{2}D\hat{\theta}_{2}D_{i}P_{i}+\indora_{i=3}D_{i}\hat{F}_{2},
\end{eqnarray*}

\begin{eqnarray*}
D_{1}D_{1}\hat{F}_{2}&=&D_{1}\hat{\theta}_{2}D_{2}^{2}P_{1}D_{2}\hat{F}_{2}
+\left(D_{1}P_{1}\right)^{2}D_{1}^{2}\hat{\theta}_{2}D_{2}\hat{F}_{2}+
\left(D_{1}P_{1}\right)^{2}\left(D\hat{\theta}_{2}\right)^{2}D_{1}^{2}\hat{F}_{2}\\
D_{2}D_{1}\hat{F}_{2}&=&D_{1}P_{1}D_{2}P_{2}D\hat{\theta}_{2}D_{2}\hat{F}_{2}+
D_{1}P_{1}D_{2}P_{2}D^{2}\hat{\theta}_{2}D_{2}\hat{F}_{2}+
D_{1}P_{1}D_{2}P_{2}\left(D\hat{\theta}_{2}\right)^{2}D_{2}^{2}\hat{F}_{2}\\
D_{3}D_{1}\hat{F}_{2}&=&
D_{1}P_{1}D_{3}\hat{P}_{1}D\hat{\theta}_{2}D_{2}\hat{F}_{2}
+D_{1}P_{1}D_{3}\hat{P}_{1}D^{2}\hat{\theta}_{2}D_{2}\hat{F}_{2}
+D_{1}P_{1}D_{3}\hat{P}_{1}\left(D\hat{\theta}_{2}\right)^{2}D_{2}^{2}\hat{F}_{2}
+D_{1}P_{1}D\hat{\theta}_{2}D_{1}D_{2}\hat{F}_{2}\\
D_{4}D_{1}\hat{F}_{2}&=&0\\
D_{1}D_{2}\hat{F}_{2}&=&
D_{1}P_{1}D_{2}P_{2}D\hat{\theta}_{2}D_{2}\hat{F}_{2}+
D_{1}P_{1}D_{2}P_{2}D^{2}\hat{\theta}_{2}D_{2}\hat{F}_{2}+
D_{1}P_{1}D_{2}P_{2}\left(D\hat{\theta}_{2}\right)^{2}D_{2}D_{2}\hat{F}_{2}\\
D_{2}D_{2}\hat{F}_{2}&=&
D\hat{\theta}_{2}D_{2}^{2}P_{2}D_{2}\hat{F}_{2}+
\left(D_{2}P_{2}\right)^{2}D^{2}\hat{\theta}_{2}D_{2}\hat{F}_{2}+
\left(D_{2}P_{2}\right)^{2}\left(D\hat{\theta}_{2}\right)^{2}D_{2}^{2}\hat{F}_{2}\\
D_{3}D_{2}\hat{F}_{2}&=&
D_{2}P_{2}D_{3}\hat{P}_{1}D\hat{\theta} _{2}D_{2}\hat{F}_{2}
+D_{2}P_{2}D_{3}\hat{P}_{1}D^{2}\hat{\theta}_{2}D_{2}\hat{F}_{2}
+D_{2}P_{2}D_{3}\hat{P}_{1}\left(D\hat{\theta}_{2}\right)^{2}D_{2}^{2}\hat{F}_{2}
+D_{2}P_{2}D\hat{\theta}_{2}D_{1}D_{2}\hat{F}_{2}\\
D_{4}D_{2}\hat{F}_{2}&=&0\\
D_{1}D_{3}\hat{F}_{2}&=&
D_{1}P_{1}D_{3}\hat{P}_{1}D\hat{\theta}_{2}D_{2}\hat{F}_{2}
+D_{1}P_{1}D_{3}\hat{P}_{1}D^{2}\hat{\theta}_{2}D_{2}\hat{F}_{2}
+D_{1}P_{1}D_{3}\hat{P}_{1}\left(D\hat{\theta}_{2}\right)^{2}D_{2}D_{2}\hat{F}_{2}
+D_{1}P_{1}D\hat{\theta}_{2}D_{2}D_{1}\hat{F}_{2}\\
D_{2}D_{3}\hat{F}_{2}&=&
D_{2}P_{2}D_{3}\hat{P}_{1}D\hat{\theta}_{2}D_{2}\hat{F}_{2}
+D_{2}P_{2}D_{3}\hat{P}_{1}D^{2}\hat{\theta}_{2}D_{2}\hat{F}_{2}
+D_{2}P_{2}D_{3}\hat{P}_{1}\left(D\hat{\theta}_{2}\right)^{2}D_{2}^{2}\hat{F}_{2}
+D_{2}P_{2}D\hat{\theta}_{2}D_{1}D_{2}\hat{F}_{2}\\
D_{3}D_{3}\hat{F}_{2}&=&
D_{3}^{2}\hat{P}_{1}D\hat{\theta}_{2}D_{2}\hat{F}_{2}
+\left(D_{3}\hat{P}_{1}\right)^{2}D^{2}\hat{\theta}_{2}D_{2}\hat{F}_{2}
+D_{3}\hat{P}_{1}D\hat{\theta}_{2}D_{1}D_{2}\hat{F}_{2}
+ \left(D_{3}\hat{P}_{1}\right)^{2}\left(D\hat{\theta}_{2}\right)^{2}
D_{2}^{2}\hat{F}_{2}\\
&+&D_{3}\hat{P}_{1}D\hat{\theta}_{2}D_{1}D_{2}\hat{F}_{2}
+D_{1}^{2}\hat{F}_{2}\\
D_{4}D_{3}\hat{F}_{2}&=&0\\
D_{1}D_{4}\hat{F}_{2}&=&0\\
D_{2}D_{4}\hat{F}_{2}&=&0\\
D_{3}D_{4}\hat{F}_{2}&=&0\\
D_{4}D_{4}\hat{F}_{2}&=&0\\
\end{eqnarray*}
%_____________________________________________________________________________________
\newpage
%__________________________________________________________________
\section{Generalizaciones}
%__________________________________________________________________
\subsection{RSVC con dos conexiones}
%__________________________________________________________________

%\begin{figure}[H]
%\centering
%%%\includegraphics[width=9cm]{Grafica3.jpg}
%%\end{figure}\label{RSVC3}


Sus ecuaciones recursivas son de la forma


\begin{eqnarray*}
F_{1}\left(z_{1},z_{2},w_{1},w_{2}\right)&=&R_{2}\left(\prod_{i=1}^{2}\tilde{P}_{i}\left(z_{i}\right)\prod_{i=1}^{2}
\hat{P}_{i}\left(w_{i}\right)\right)F_{2}\left(z_{1},\tilde{\theta}_{2}\left(\tilde{P}_{1}\left(z_{1}\right)\hat{P}_{1}\left(w_{1}\right)\hat{P}_{2}\left(w_{2}\right)\right)\right)
\hat{F}_{2}\left(w_{1},w_{2};\tau_{2}\right),
\end{eqnarray*}

\begin{eqnarray*}
F_{2}\left(z_{1},z_{2},w_{1},w_{2}\right)&=&R_{1}\left(\prod_{i=1}^{2}\tilde{P}_{i}\left(z_{i}\right)\prod_{i=1}^{2}
\hat{P}_{i}\left(w_{i}\right)\right)F_{1}\left(\tilde{\theta}_{1}\left(\tilde{P}_{2}\left(z_{2}\right)\hat{P}_{1}\left(w_{1}\right)\hat{P}_{2}\left(w_{2}\right)\right),z_{2}\right)\hat{F}_{1}\left(w_{1},w_{2};\tau_{1}\right),
\end{eqnarray*}


\begin{eqnarray*}
\hat{F}_{1}\left(z_{1},z_{2},w_{1},w_{2}\right)&=&\hat{R}_{2}\left(\prod_{i=1}^{2}\tilde{P}_{i}\left(z_{i}\right)\prod_{i=1}^{2}
\hat{P}_{i}\left(w_{i}\right)\right)F_{2}\left(z_{1},z_{2};\zeta_{2}\right)\hat{F}_{2}\left(w_{1},\hat{\theta}_{2}\left(\tilde{P}_{1}\left(z_{1}\right)\tilde{P}_{2}\left(z_{2}\right)\hat{P}_{1}\left(w_{1}
\right)\right)\right),
\end{eqnarray*}


\begin{eqnarray*}
\hat{F}_{2}\left(z_{1},z_{2},w_{1},w_{2}\right)&=&\hat{R}_{1}\left(\prod_{i=1}^{2}\tilde{P}_{i}\left(z_{i}\right)\prod_{i=1}^{2}
\hat{P}_{i}\left(w_{i}\right)\right)F_{1}\left(z_{1},z_{2};\zeta_{1}\right)\hat{F}_{1}\left(\hat{\theta}_{1}\left(\tilde{P}_{1}\left(z_{1}\right)\tilde{P}_{2}\left(z_{2}\right)\hat{P}_{2}\left(w_{2}\right)\right),w_{2}\right),
\end{eqnarray*}

the server's switchover times are given by the general equation

\begin{eqnarray}\label{Ec.Ri}
R_{i}\left(\mathbf{z,w}\right)=R_{i}\left(\tilde{P}_{1}\left(z_{1}\right)\tilde{P}_{2}\left(z_{2}\right)\hat{P}_{1}\left(w_{1}\right)\hat{P}_{2}\left(w_{2}\right)\right)
\end{eqnarray}

with
\begin{eqnarray}\label{Ec.Derivada.Ri}
D_{i}R_{i}&=&DR_{i}D_{i}P_{i}
\end{eqnarray}
the following notation is considered

\begin{eqnarray*}
\begin{array}{llll}
D_{1}P_{1}\equiv D_{1}\tilde{P}_{1}, & D_{2}P_{2}\equiv D_{2}\tilde{P}_{2}, & D_{3}P_{3}\equiv D_{3}\hat{P}_{1}, &D_{4}P_{4}\equiv D_{4}\hat{P}_{2},
\end{array}
\end{eqnarray*}

also we need to remind $F_{1,2}\left(z_{1};\zeta_{2}\right)F_{2,2}\left(z_{2};\zeta_{2}\right)=F_{2}\left(z_{1},z_{2};\zeta_{2}\right)$, therefore

\begin{eqnarray*}
D_{1}F_{2}\left(z_{1},z_{2};\zeta_{2}\right)&=&D_{1}\left[F_{1,2}\left(z_{1};\zeta_{2}\right)F_{2,2}\left(z_{2};\zeta_{2}\right)\right]
=F_{2,2}\left(z_{2};\zeta_{2}\right)D_{1}F_{1,2}\left(z_{1};\zeta_{2}\right)=F_{1,2}^{(1)}\left(1\right)
\end{eqnarray*}

i.e., $D_{1}F_{2}=F_{1,2}^{(1)}(1)$; $D_{2}F_{2}=F_{2,2}^{(1)}\left(1\right)$, whereas that $D_{3}F_{2}=D_{4}F_{2}=0$, then

\begin{eqnarray*}
\begin{array}{ccc}
D_{i}F_{j}=\indora_{i\leq2}F_{i,j}^{(1)}\left(1\right),& \textrm{ y } &D_{i}\hat{F}_{j}=\indora_{i\geq2}F_{i,j}^{(1)}\left(1\right).
\end{array}
\end{eqnarray*}

Now, we obtain the first moments equations for the queue lengths as before for the times the server arrives to the queue to start attending



Remember that


\begin{eqnarray*}
F_{2}\left(z_{1},z_{2},w_{1},w_{2}\right)&=&R_{1}\left(\prod_{i=1}^{2}\tilde{P}_{i}\left(z_{i}\right)\prod_{i=1}^{2}
\hat{P}_{i}\left(w_{i}\right)\right)F_{1}\left(\tilde{\theta}_{1}\left(\tilde{P}_{2}\left(z_{2}\right)\hat{P}_{1}\left(w_{1}\right)\hat{P}_{2}\left(w_{2}\right)\right),z_{2}\right)\hat{F}_{1}\left(w_{1},w_{2};\tau_{1}\right),
\end{eqnarray*}

where


\begin{eqnarray*}
F_{1}\left(\tilde{\theta}_{1}\left(\tilde{P}_{2}\hat{P}_{1}\hat{P}_{2}\right),z_{2}\right)
\end{eqnarray*}

so

\begin{eqnarray*}
D_{i}F_{1}&=&\indora_{i\neq1}D_{1}F_{1}D\tilde{\theta}_{1}D_{i}P_{i}+\indora_{i=2}D_{i}F_{1},
\end{eqnarray*}

then


\begin{eqnarray}
D_{1}F_{1}&=&0,\\
D_{2}F_{1}&=&D_{1}F_{1}D\tilde{\theta}_{1}D_{2}P_{2}+D_{2}F_{1}
=f_{1}\left(1\right)\frac{1}{1-\tilde{\mu}_{1}}\tilde{\mu}_{2}+f_{1}\left(2\right),\\
D_{3}F_{1}&=&D_{1}F_{1}D\tilde{\theta}_{1}D_{3}P_{3}
=f_{1}\left(1\right)\frac{1}{1-\tilde{\mu}_{1}}\hat{\mu}_{1}\\
D_{4}F_{1}&=&D_{1}F_{1}D\tilde{\theta}_{1}D_{4}P_{4}
=f_{1}\left(1\right)\frac{1}{1-\tilde{\mu}_{1}}\hat{\mu}_{2}
\end{eqnarray}


\begin{eqnarray*}
D_{i}F_{2}&=&\indora_{i\neq2}D_{2}F_{2}D\tilde{\theta}_{2}D_{i}P_{i}
+\indora_{i=1}D_{i}F_{2}
\end{eqnarray*}

\begin{eqnarray}
D_{1}F_{2}&=&D_{2}F_{2}D\tilde{\theta}_{2}D_{1}P_{1}
+D_{1}F_{2}=f_{2}\left(2\right)\frac{1}{1-\tilde{\mu}_{2}}\tilde{\mu}_{1}\\
D_{2}F_{2}&=&0\\
D_{3}F_{2}&=&D_{2}F_{2}D\tilde{\theta}_{2}D_{3}P_{3}
=f_{2}\left(2\right)\frac{1}{1-\tilde{\mu}_{2}}\hat{\mu}_{1}\\
D_{4}F_{2}&=&D_{2}F_{2}D\tilde{\theta}_{2}D_{4}P_{4}
=f_{2}\left(2\right)\frac{1}{1-\tilde{\mu}_{2}}\hat{\mu}_{2}
\end{eqnarray}



\begin{eqnarray*}
D_{i}\hat{F}_{1}&=&\indora_{i\neq3}D_{3}\hat{F}_{1}D\hat{\theta}_{1}D_{i}P_{i}+\indora_{i=4}D_{i}\hat{F}_{1},
\end{eqnarray*}

\begin{eqnarray}
D_{1}\hat{F}_{1}&=&D_{3}\hat{F}_{1}D\hat{\theta}_{1}D_{1}P_{1}=\hat{f}_{1}\left(3\right)\frac{1}{1-\hat{\mu}_{1}}\tilde{\mu}_{1}
=\\
D_{2}\hat{F}_{1}&=&D_{3}\hat{F}_{1}D\hat{\theta}_{1}D_{2}P_{2}
=\hat{f}_{1}\left(3\right)\frac{1}{1-\hat{\mu}_{1}}\tilde{\mu}_{2}\\
D_{3}\hat{F}_{1}&=&0\\
D_{4}\hat{F}_{1}&=&D_{3}\hat{F}_{1}D\hat{\theta}_{1}D_{4}P_{4}
+D_{4}\hat{F}_{1}
=\hat{f}_{1}\left(3\right)\frac{1}{1-\hat{\mu}_{1}}\hat{\mu}_{2}+\hat{f}_{1}\left(2\right),
\end{eqnarray}


\begin{eqnarray*}
D_{i}\hat{F}_{2}&=&\indora_{i\neq4}D_{4}\hat{F}_{2}D\hat{\theta}_{2}D_{i}P_{i}+\indora_{i=3}D_{i}\hat{F}_{2}.
\end{eqnarray*}

\begin{eqnarray}
D_{1}\hat{F}_{2}&=&D_{4}\hat{F}_{2}D\hat{\theta}_{2}D_{1}P_{1}
=\hat{f}_{2}\left(4\right)\frac{1}{1-\hat{\mu}_{2}}\tilde{\mu}_{1}\\
D_{2}\hat{F}_{2}&=&D_{4}\hat{F}_{2}D\hat{\theta}_{2}D_{2}P_{2}
=\hat{f}_{2}\left(4\right)\frac{1}{1-\hat{\mu}_{2}}\tilde{\mu}_{2}\\
D_{3}\hat{F}_{2}&=&D_{4}\hat{F}_{2}D\hat{\theta}_{2}D_{3}P_{3}+D_{3}\hat{F}_{2}
=\hat{f}_{2}\left(4\right)\frac{1}{1-\hat{\mu}_{2}}\hat{\mu}_{1}+\hat{f}_{2}\left(4\right)\\
D_{4}\hat{F}_{2}&=&0
\end{eqnarray}
Then, now we can obtain the linear system of equations in order to obtain the first moments of the lengths of the queues:



For $\mathbf{F}_{1}=R_{2}F_{2}\hat{F}_{2}$ we get the general equations

\begin{eqnarray*}
D_{i}\mathbf{F}_{1}=D_{i}\left(R_{2}+F_{2}+\indora_{i\geq3}\hat{F}_{2}\right)
\end{eqnarray*}

So

\begin{eqnarray*}
D_{1}\mathbf{F}_{1}&=&D_{1}R_{2}+D_{1}F_{2}
=r_{1}\tilde{\mu}_{1}+f_{2}\left(2\right)\frac{1}{1-\tilde{\mu}_{2}}\tilde{\mu}_{1}\\
D_{2}\mathbf{F}_{1}&=&D_{2}\left(R_{2}+F_{2}\right)
=r_{2}\tilde{\mu}_{1}\\
\end{eqnarray*}


\begin{eqnarray*}
D_{3}\mathbf{F}_{1}&=&D_{3}\left(R_{2}+F_{2}+\hat{F}_{2}\right)
=r_{1}\hat{\mu}_{1}+f_{2}\left(2\right)\frac{1}{1-\tilde{\mu}_{2}}\hat{\mu}_{1}+\hat{F}_{1,2}^{(1)}\left(1\right)
\end{eqnarray*}


\begin{eqnarray*}
D_{4}\mathbf{F}_{1}&=&D_{4}\left(R_{2}+F_{2}+\hat{F}_{2}\right)
\end{eqnarray*}





\begin{eqnarray}\label{Ec.Primeras.Derivadas.Parciales}
\begin{array}{ll}
\mathbf{F}_{2}=R_{1}F_{1}\hat{F}_{1}, & D_{i}\mathbf{F}_{2}=D_{i}\left(R_{1}+F_{1}+\indora_{i\geq3}\hat{F}_{1}\right)\\
\hat{\mathbf{F}}_{1}=\hat{R}_{2}\hat{F}_{2}F_{2}, & D_{i}\hat{\mathbf{F}}_{1}=D_{i}\left(\hat{R}_{2}+\hat{F}_{2}+\indora_{i\leq2}F_{2}\right)\\
\hat{\mathbf{F}}_{2}=\hat{R}_{1}\hat{F}_{1}F_{1}, & D_{i}\hat{\mathbf{F}}_{2}=D_{i}\left(\hat{R}_{1}+\hat{F}_{1}+\indora_{i\leq2}F_{1}\right)
\end{array}
\end{eqnarray}
%___________________________________________________________________________________________
%
\subsection{Derivadas de Orden Superior}
%___________________________________________________________________________________________
%
\small{
\begin{eqnarray*}\label{Ec.Derivadas.Segundo.Orden}
D_{k}D_{i}F_{1}&=&D_{k}D_{i}\left(R_{2}+F_{2}+\indora_{i\geq3}\hat{F}_{4}\right)+D_{i}R_{2}D_{k}\left(F_{2}+\indora_{k\geq3}\hat{F}_{4}\right)+D_{i}F_{2}D_{k}\left(R_{2}+\indora_{k\geq3}\hat{F}_{4}\right)+\indora_{i\geq3}D_{i}\hat{F}_{4}D_{k}\left(R_{}+F_{2}\right)\\
D_{k}D_{i}F_{2}&=&D_{k}D_{i}\left(R_{1}+F_{1}+\indora_{i\geq3}\hat{F}_{3}\right)+D_{i}R_{1}D_{k}\left(F_{1}+\indora_{k\geq3}\hat{F}_{3}\right)+D_{i}F_{1}D_{k}\left(R_{1}+\indora_{k\geq3}\hat{F}_{3}\right)+\indora_{i\geq3}D_{i}\hat{F}_{3}D_{k}\left(R_{1}+F_{1}\right)\\
D_{k}D_{i}\hat{F}_{3}&=&D_{k}D_{i}\left(\hat{R}_{4}+\indora_{i\leq2}F_{2}+\hat{F}_{4}\right)+D_{i}\hat{R}_{4}D_{k}\left(\indora_{k\leq2}F_{2}+\hat{F}_{4}\right)+D_{i}\hat{F}_{4}D_{k}\left(\hat{R}_{4}+\indora_{k\leq2}F_{2}\right)+\indora_{i\leq2}D_{i}F_{2}D_{k}\left(\hat{R}_{4}+\hat{F}_{4}\right)\\
D_{k}D_{i}\hat{F}_{4}&=&D_{k}D_{i}\left(\hat{R}_{3}+\indora_{i\leq2}F_{1}+\hat{F}_{3}\right)+D_{i}\hat{R}_{3}D_{k}\left(\indora_{k\leq2}F_{1}+\hat{F}_{3}\right)+D_{i}\hat{F}_{3}D_{k}\left(\hat{R}_{3}+\indora_{k\leq2}F_{1}\right)+\indora_{i\leq2}D_{i}F_{1}D_{k}\left(\hat{R}_{3}+\hat{F}_{3}\right)
\end{eqnarray*}}
para $i,k=1,\ldots,4$. Es necesario determinar las derivadas de segundo orden para las expresiones de la forma $D_{k}D_{i}\left(R_{2}+F_{2}+\indora_{i\geq3}\hat{F}_{4}\right)$

A saber, $R_{i}\left(z_{1},z_{2},w_{1},w_{2}\right)=R_{i}\left(P_{1}\left(z_{1}\right)\tilde{P}_{2}\left(z_{2}\right)
\hat{P}_{1}\left(w_{1}\right)\hat{P}_{2}\left(w_{2}\right)\right)$, la denotaremos por la expresi\'on $R_{i}=R_{i}\left(
P_{1}\tilde{P}_{2}\hat{P}_{1}\hat{P}_{2}\right)$, donde al igual que antes, utilizando la notaci\'on dada en \cite{Lang} se tiene   que

\begin{eqnarray}
D_{i}D_{i}R_{k}=D^{2}R_{k}\left(D_{i}P_{i}\right)^{2}+DR_{k}D_{i}D_{i}P_{i}
\end{eqnarray}

mientras que para $i\neq j$

\begin{eqnarray}
D_{i}D_{j}R_{k}=D^{2}R_{k}D_{i}P_{i}D_{j}P_{j}+DR_{k}D_{j}P_{j}D_{i}P_{i}
\end{eqnarray}

Recordemos la expresi\'on $F_{1}\left(\theta_{1}\left(\tilde{P}_{2}\left(z_{2}\right)\hat{P}_{1}\left(w_{1}\right)\hat{P}_{2}\left(w_{2}\right)\right),
z_{2}\right)$, que denotaremos por $F_{1}\left(\theta_{1}\left(\tilde{P}_{2}\hat{P}_{1}\hat{P}_{2}\right),z_{2}\right)$, entonces las derivadas parciales mixtas son:

\begin{eqnarray*}
D_{i}F_{1}=\indora_{i\geq2}D_{i}F_{1}D\theta_{1}D_{i}P_{i}+\indora_{i=2} D_{i}F_{1},
\end{eqnarray*}

entonces para
$F_{1}\left(\theta_{1}\left(\tilde{P}_{2}\hat{P}_{1}\hat{P}_{2}\right),z_{2}\right)$

$$D_{2}F_{1}=D_{1}F_{1}D_{1}\theta_{1}D_{2}\tilde{P}_{2}\left\{\hat{P}_{1}\hat{P}_{2}\right\}+D_{2}F_{1}$$

\begin{eqnarray*}
D_{j}D_{i}F_{1}&=&\indora_{i,j\neq1}D_{1}D_{1}F_{1}\left(D\theta_{1}\right)^{2}D_{i}P_{i}D_{j}P_{j}+\indora_{i,j\neq1}D_{1}F_{1}D^{2}\theta_{1}D_{i}P_{i}D_{j}P_{j}\\
&+&\indora_{i,j\neq1}D_{1}F_{1}D\theta_{1}\left(\indora_{i=j}D_{i}^{2}P_{i}+\indora_{i\neq j}D_{i}P_{i}D_{j}P_{j}\right)\\
&+&\indora_{i,j\neq1}D_{1}D_{2}F_{1}D\theta_{1}D_{i}P_{i}+\indora_{i=2}\left(D_{1}D_{2}F_{1}D\theta_{1}D_{i}P_{i}+D_{i}^{2}F_{1}\right)
\end{eqnarray*}


Para $F_{2}\left(z_{1},\tilde{\theta}_{2}\left(P_{1}\hat{P}_{1}\hat{P}_{2}\right)\right)$

\begin{eqnarray*}
D_{i}F_{2}=\indora_{i\neq2}D_{2}F_{2}D\tilde{\theta}_{2}D_{i}P_{i}+\indora_{i=1} D_{i}F_{2},
\end{eqnarray*}


%\begin{eqnarray*}
%D_{j}D_{i}F_{2}&=&
%\indora_{i,j\neq1}D_{2}^{2}F_{2}\left(D\tilde{\theta}_{2}\right)^{2}_{i}P_{i}D_{j}P_{j}+\indora_{i,j\neq2}D_{2}F_{2}D^{2}\tilde{\theta}_{2}D_{i}P_{i}D_{j}P_{j}\\
%&+&\indora_{i,j\neq2}D_{2}F_{2}D\tilde{\theta}_{2}D_{i}P_{i}D_{j}P_{j}+\indora_{i=j}D_{i}P_{i}D_{j}P_{j}\left(\indora_{i=j}D_{i}^{2}P_{i}+\indora_{i\neq j}D_{i}P_{i}D_{j}P_{j}\right)\\
%&+&\indora_{i,j\neq1}D_{1}D_{2}F_{1}D\theta_{1}D_{i}P_{i}+\indora_{i=2}\left(D_{1}D_{2}F_{1}D\theta_{1}D_{i}P_{i}+D_{i}^{2}F_{1}\right)
%\end{eqnarray*}


\begin{eqnarray*}
D_{j}D_{i}F_{2}&=&\indora_{i,j\neq2}D_{2}D_{21}F_{2}\left(D\theta_{2}\right)^{2}D_{i}P_{i}D_{j}P_{j}+\indora_{i,j\neq2}D_{2}F_{2}D^{2}\theta_{2}D_{i}P_{i}D_{j}P_{j}\\
&+&\indora_{i,j\neq2}D_{2}F_{2}D\theta_{2}\left(\indora_{i=j}D_{i}^{2}P_{i}+\indora_{i\neq j}D_{i}P_{i}D_{j}P_{j}\right)\\
&+&\indora_{i,j\neq2}D_{2}D_{1}F_{2}D\theta_{2}D_{i}P_{i}+\indora_{i=2}\left(D_{2}D_{1}F_{2}D\theta_{2}D_{i}P_{i}+D_{i}^{2}F_{2}\right)
\end{eqnarray*}



\begin{eqnarray*}
D_{1}D_{1}F_{2}&=&
D_{1}^{2}P_{1}D\tilde{\theta}_{2}D_{2}F_{2}+
\left(D_{1}P_{1}\right)^{2}D^{2}\tilde{\theta}_{2}D_{2}F_{2}+
D_{1}P_{1}D\tilde{\theta}_{2}D_{1}D_{2}F_{2}+
\left(D_{1}P_{1}\right)^{2}\left(D\tilde{\theta}_{2}\right)^{2}D_{2}^{2}F_{2}+
D_{1}P_{1}D\tilde{\theta}_{2}D_{1}D_{2}F_{2}+
D_{1}^{2}F_{2}\\
D_{2}D_{1}F_{2}&=&0\\
D_{3}D_{1}F_{2}&=&D_{2}D_{1}F_{2}D\tilde{\theta}_{2}D_{3}\hat{P}_{1}
+D_{2}^{2}F_{2}\left(D\tilde{\theta}_{2}\right)^{2}D_{3}P_{1}D_{1}P_{1}
+D_{2}F_{2}D^{2}\tilde{\theta}_{2}D_{3}\hat{P}_{1}D_{1}P_{1}
+D_{2}F_{2}D\tilde{\theta}_{2}D_{1}P_{1}D_{3}\hat{P}_{1}\\
D_{4}D_{1}F_{2}&=&D_{1}D_{2}F_{2}D\tilde{\theta}_{2}D_{4}\hat{P}_{2}
+D_{2}^{2}F_{2}\left(D\tilde{\theta}_{2}\right)^{2}D_{4}P_{2}D_{1}P_{1}
+D_{2}F_{2}D^{2}\tilde{\theta}_{2}D_{4}\hat{P}_{2}D_{1}P_{1}
+D_{2}F_{2}D\tilde{\theta}_{2}D_{1}P_{1}D_{4}\hat{P}_{2}\\
D_{1}D_{3}F_{2}&=&D_{2}^{2}F_{2}\left(D\tilde{\theta}_{2}\right)^{2}D_{1}P_{1}D_{3}\hat{P}_{1}
+D_{2}D_{1}F_{2}D\tilde{\theta}_{2}D_{3}\hat{P}_{1}
+D_{2}F_{2}D^{2}\tilde{\theta}_{2}D_{1}P_{1}D_{3}\hat{P}_{1}
+D_{2}F_{2}D\tilde{\theta}_{2}D_{3}\hat{P}_{1}D_{1}P_{1}\\
D_{2}D_{3}F_{3}&=&0\\
D_{3}D_{3}F_{2}&=&D_{2}^{2}F_{2}\left(D\tilde{\theta}_{2}\right)^{2}\left(D_{3}\hat{P}_{1}\right)^{2}
+D_{2}F_{2}\left(D_{3}\hat{P}_{1}\right)^{2}D^{2}\tilde{\theta}_{2}
+D_{2}F_{2}D\tilde{\theta}_{2}D_{3}^{2}\hat{P}_{1}\\
D_{4}D_{3}F_{2}&=&D_{2}^{2}F_{2}\left(D\tilde{\theta}_{2}\right)^{2}D_{4}\hat{P}_{2}D_{3}\hat{P}_{1}
+D_{2}F_{2}D^{2}\tilde{\theta}_{2}D_{4}\hat{P}_{2}D_{3}\hat{P}_{1}
+D_{2}F_{2}D\tilde{\theta}_{2}D_{3}\hat{P}_{1}D_{4}\hat{P}_{2}\\
D_{1}D_{4}F_{2}&=&D_{2}^{2}F_{2}\left(D\tilde{\theta}_{2}\right)^{2}D_{1}P_{1}D_{4}\hat{P}_{2}
+D_{1}D_{2}F_{2}D\tilde{\theta}_{2}D_{4}\hat{P}_{2}
+D_{2}F_{2}D^{2}\tilde{\theta}_{2}D_{1}P_{1}D_{4}\hat{P}_{2}
+D_{2}F_{2}D\tilde{\theta}_{2}D_{4}\hat{P}_{2}D_{1}P_{1}\\
D_{2}D_{4}F_{2}&=&0\\
D_{3}D_{4}F_{2}&=&
D_{2}F_{2}D\tilde{\theta}_{2}D_{4}\hat{P}_{2}D_{3}\hat{P}_{1}
+D_{2}F_{2}D^{2}\tilde{\theta}_{2}D_{4}\hat{P}_{2}D_{3}\hat{P}_{1}
+D_{2}^{2}F_{2}\left(D\tilde{\theta}_{2}\right)^{2}D_{4}\hat{P}_{2}D_{3}\hat{P}_{1}\\
D_{4}D_{4}F_{2}&=&D_{2}F_{2}D\tilde{\theta}_{2}D_{4}^{2}\hat{P}_{2}
+D_{2}F_{2}D^{2}\tilde{\theta}_{2}\left(D_{4}\hat{P}_{2}\right)^{2}
+D_{2}^{2}F_{2}\left(D\tilde{\theta}_{2}\right)^{2}\left(D_{4}\hat{P}_{2}\right)^{2}\\
\end{eqnarray*}


%\newpage



%\newpage

para $\hat{F}_{1}\left(\hat{\theta}_{1}\left(P_{1}\tilde{P}_{2}\hat{P}_{2}\right),w_{2}\right)$

\begin{eqnarray*}
D_{i}\hat{F}_{1}=\indora_{i\neq3}D_{3}\hat{F}_{1}D\hat{\theta}_{1}D_{i}P_{i}+\indora_{i=4}D_{i}\hat{F}_{1},
\end{eqnarray*}


\begin{eqnarray*}
D_{1}D_{1}\hat{F}_{1}&=&
D\hat{\theta}_{1}D_{1}^{2}P_{1}D_{1}\hat{F}_{1}
+\left(D_{1}P_{1}\right)^{2}D^{2}\hat{\theta}_{1}D_{1}\hat{F}_{1}
+\left(D_{1}P_{1}\right)^{2}\left(D\hat{\theta}_{1}\right)^{2}D_{1}^{2}\hat{F}_{1}\\
D_{2}D_{1}\hat{F}_{1}&=&D_{1}P_{1}D_{2}P_{2}D\hat{\theta}_{1}D_{1}\hat{F}_{1}+
D_{1}P_{1}D_{2}P_{2}D^{2}\hat{\theta}_{1}D_{1}\hat{F}_{1}+
D_{1}P_{1}D_{2}P_{1}\left(D\hat{\theta}_{1}\right)^{2}D_{1}^{2}\hat{\theta}_{1}\\
D_{3}D_{1}\hat{F}_{1}&=&0\\
D_{4}D_{1}\hat{F}_{1}&=&D_{1}P_{1}D_{4}\hat{P}_{2}D\hat{\theta}_{1}D_{1}\hat{F}_{1}
+D_{1}P_{1}D_{4}\hat{P}_{2}D^{2}\hat{\theta}_{1}D_{1}\hat{F}_{1}
+D_{1}P_{1}D\hat{\theta}_{1}D_{2}D{1}\hat{F}_{1}
+D_{1}P_{1}D\hat{\theta}_{1}D_{1}D_{1}\hat{F}_{1}\\
D_{1}D_{2}\hat{F}_{1}&=&D_{1}P_{1}D_{2}P_{2}D\hat{\theta}_{1}D_{1}\hat{F}_{1}+
D_{1}P_{1}D_{2}P_{2}D^{2}\hat{\theta}_{1}D_{1}\hat{F}_{1}+
D_{1}P_{1}D_{2}P_{2}\left(D\hat{\theta}_{1}\right)^{2}D_{1}^{2}\hat{F}_{1}\\
D_{2}D_{2}\hat{F}_{1}&=&
D\hat{\theta}_{1}D_{2}^{2}P_{2}D_{1}\hat{F}_{1}+
 \left(D_{2}P_{2}\right)^{2}D^{2}\hat{\theta}_{1}D_{1}\hat{F}_{1}+
\left(D_{2}P_{2}\right)^{2}\left(D\hat{\theta}_{1}\right)^{2}D_{1}^{2}\hat{F}_{1}\\
D_{3}D_{2}\hat{F}_{1}&=&0\\
D_{4}D_{2}\hat{F}_{1}&=&D_{2}P_{2}D_{4}\hat{P}_{2}D\hat{\theta}_{1}D\hat{F}_{1}
+D_{2}P_{2}D_{4}\hat{P}_{2}D^{2}\hat{\theta}_{1}D_{1}\hat{F}_{1}
+D_{2}P_{2}D\hat{\theta}_{1}D_{2}D_{1}\hat{F}_{1}
+D_{2}P_{2}\left(D\hat{\theta}_{1}\right)^{2}D_{4}\hat{P}_{2}D_{1}^{2}\hat{F}_{1}\\
D_{1}D_{3}\hat{F}_{1}&=&0\\
D_{2}D_{3}\hat{F}_{1}&=&0\\
D_{3}D_{3}\hat{F}_{1}&=&0\\
D_{4}D_{3}\hat{F}_{1}&=&0\\
D_{1}D_{4}\hat{F}_{1}&=&D_{1}P_{1}D_{4}\hat{F}_{2}D\hat{\theta}_{1}D_{1}\hat{F}_{1}
+D_{1}P_{1}D_{4}\hat{P}_{2}D^{2}\hat{\theta}_{1}D_{1}\hat{F}_{1}
+D_{1}P_{1}D\hat{\theta}_{1}D_{2}D_{1}\hat{F}_{1}
+ D_{1}P_{1}D_{4}\hat{P}_{2}\left(D\hat{\theta}_{1}\right)^{2}D_{1}D_{1}\hat{F}_{1}\\
D_{2}D_{4}\hat{F}_{1}&=&D_{2}P_{2}D_{4}\hat{P}_{2}D\hat{\theta}_{1}D_{1}
\hat{F}_{1}
+D_{2}P_{2}D_{4}\hat{P}_{2}D^{2}\hat{\theta}_{1}D_{1}\hat{F}_{1}
+D_{2}P_{2}D\hat{\theta}_{1}D_{2}D_{1}\hat{F}_{1}+
D_{2}P_{2}D_{4}\hat{P}_{2}\left(D\hat{\theta}_{1}\right)^{2}D_{1}^{2}\hat{F}_{1}\\
D_{3}D_{4}\hat{F}_{1}&=&0\\
D_{4}D_{4}\hat{F}_{1}&=&D_{2}D_{2}\hat{F}_{1}+D\hat{\theta}_{1}D_{4}^{2}\hat{P}_{2}D_{1}\hat{F}_{1}
+\left(D_{4}\hat{P}_{2}\right)^{2}D^{2}\hat{\theta}_{1}D_{1}\hat{F}_{1}+
D_{4}\hat{P}_{2}D\hat{\theta}_{1}D_{2}D_{1}\hat{F}_{1}\\
&+&D_{4}\hat{P}_{2}D\hat{\theta}_{1}D_{2}D_{1}\hat{F}_{1}+ \left(D_{4}\hat{P}_{2}\right)^{2}D\hat{\theta}_{1}D\hat{\theta}_{1}D_{1}^{2}\hat{F}_{1}\\
\end{eqnarray*}




%\newpage
finalmente, para $\hat{F}_{2}\left(w_{1},\hat{\theta}_{2}\left(P_{1}\tilde{P}_{2}\hat{P}_{1}\right)\right)$

\begin{eqnarray*}
D_{i}\hat{F}_{2}=\indora_{i\neq4}D_{4}\hat{F}_{2}D\hat{\theta}_{2}D_{i}P_{i}+\indora_{i=3}D_{i}\hat{F}_{2},
\end{eqnarray*}

\begin{eqnarray*}
D_{1}D_{1}\hat{F}_{2}&=&D_{1}\hat{\theta}_{2}D_{2}^{2}P_{1}D_{2}\hat{F}_{2}
+\left(D_{1}P_{1}\right)^{2}D_{1}^{2}\hat{\theta}_{2}D_{2}\hat{F}_{2}+
\left(D_{1}P_{1}\right)^{2}\left(D\hat{\theta}_{2}\right)^{2}D_{1}^{2}\hat{F}_{2}\\
D_{2}D_{1}\hat{F}_{2}&=&D_{1}P_{1}D_{2}P_{2}D\hat{\theta}_{2}D_{2}\hat{F}_{2}+
D_{1}P_{1}D_{2}P_{2}D^{2}\hat{\theta}_{2}D_{2}\hat{F}_{2}+
D_{1}P_{1}D_{2}P_{2}\left(D\hat{\theta}_{2}\right)^{2}D_{2}^{2}\hat{F}_{2}\\
D_{3}D_{1}\hat{F}_{2}&=&
D_{1}P_{1}D_{3}\hat{P}_{1}D\hat{\theta}_{2}D_{2}\hat{F}_{2}
+D_{1}P_{1}D_{3}\hat{P}_{1}D^{2}\hat{\theta}_{2}D_{2}\hat{F}_{2}
+D_{1}P_{1}D_{3}\hat{P}_{1}\left(D\hat{\theta}_{2}\right)^{2}D_{2}^{2}\hat{F}_{2}
+D_{1}P_{1}D\hat{\theta}_{2}D_{1}D_{2}\hat{F}_{2}\\
D_{4}D_{1}\hat{F}_{2}&=&0\\
D_{1}D_{2}\hat{F}_{2}&=&
D_{1}P_{1}D_{2}P_{2}D\hat{\theta}_{2}D_{2}\hat{F}_{2}+
D_{1}P_{1}D_{2}P_{2}D^{2}\hat{\theta}_{2}D_{2}\hat{F}_{2}+
D_{1}P_{1}D_{2}P_{2}\left(D\hat{\theta}_{2}\right)^{2}D_{2}D_{2}\hat{F}_{2}\\
D_{2}D_{2}\hat{F}_{2}&=&
D\hat{\theta}_{2}D_{2}^{2}P_{2}D_{2}\hat{F}_{2}+
\left(D_{2}P_{2}\right)^{2}D^{2}\hat{\theta}_{2}D_{2}\hat{F}_{2}+
\left(D_{2}P_{2}\right)^{2}\left(D\hat{\theta}_{2}\right)^{2}D_{2}^{2}\hat{F}_{2}\\
D_{3}D_{2}\hat{F}_{2}&=&
D_{2}P_{2}D_{3}\hat{P}_{1}D\hat{\theta} _{2}D_{2}\hat{F}_{2}
+D_{2}P_{2}D_{3}\hat{P}_{1}D^{2}\hat{\theta}_{2}D_{2}\hat{F}_{2}
+D_{2}P_{2}D_{3}\hat{P}_{1}\left(D\hat{\theta}_{2}\right)^{2}D_{2}^{2}\hat{F}_{2}
+D_{2}P_{2}D\hat{\theta}_{2}D_{1}D_{2}\hat{F}_{2}\\
D_{4}D_{2}\hat{F}_{2}&=&0\\
D_{1}D_{3}\hat{F}_{2}&=&
D_{1}P_{1}D_{3}\hat{P}_{1}D\hat{\theta}_{2}D_{2}\hat{F}_{2}
+D_{1}P_{1}D_{3}\hat{P}_{1}D^{2}\hat{\theta}_{2}D_{2}\hat{F}_{2}
+D_{1}P_{1}D_{3}\hat{P}_{1}\left(D\hat{\theta}_{2}\right)^{2}D_{2}D_{2}\hat{F}_{2}
+D_{1}P_{1}D\hat{\theta}_{2}D_{2}D_{1}\hat{F}_{2}\\
D_{2}D_{3}\hat{F}_{2}&=&
D_{2}P_{2}D_{3}\hat{P}_{1}D\hat{\theta}_{2}D_{2}\hat{F}_{2}
+D_{2}P_{2}D_{3}\hat{P}_{1}D^{2}\hat{\theta}_{2}D_{2}\hat{F}_{2}
+D_{2}P_{2}D_{3}\hat{P}_{1}\left(D\hat{\theta}_{2}\right)^{2}D_{2}^{2}\hat{F}_{2}
+D_{2}P_{2}D\hat{\theta}_{2}D_{1}D_{2}\hat{F}_{2}\\
D_{3}D_{3}\hat{F}_{2}&=&
D_{3}^{2}\hat{P}_{1}D\hat{\theta}_{2}D_{2}\hat{F}_{2}
+\left(D_{3}\hat{P}_{1}\right)^{2}D^{2}\hat{\theta}_{2}D_{2}\hat{F}_{2}
+D_{3}\hat{P}_{1}D\hat{\theta}_{2}D_{1}D_{2}\hat{F}_{2}
+ \left(D_{3}\hat{P}_{1}\right)^{2}\left(D\hat{\theta}_{2}\right)^{2}
D_{2}^{2}\hat{F}_{2}\\
&+&D_{3}\hat{P}_{1}D\hat{\theta}_{2}D_{1}D_{2}\hat{F}_{2}
+D_{1}^{2}\hat{F}_{2}\\
D_{4}D_{3}\hat{F}_{2}&=&0\\
D_{1}D_{4}\hat{F}_{2}&=&0\\
D_{2}D_{4}\hat{F}_{2}&=&0\\
D_{3}D_{4}\hat{F}_{2}&=&0\\
D_{4}D_{4}\hat{F}_{2}&=&0\\
\end{eqnarray*}


%__________________________________________________________________________
\section{Teor\'ia General}
%__________________________________________________________________________


%__________________________________________________________________________
\subsection{Ecuaciones Recursivas para la RSVC}
%__________________________________________________________________________

Recordemos las ecuaciones recursivas que modelan la RSVC:

\begin{eqnarray*}
F_{2}\left(z_{1},z_{2},w_{1},w_{2}\right)&=&R_{1}\left(P_{1}\left(z_{1}\right)\tilde{P}_{2}\left(z_{2}\right)\prod_{i=1}^{2}
\hat{P}_{i}\left(w_{i}\right)\right)F_{1}\left(\theta_{1}\left(\tilde{P}_{2}\left(z_{2}\right)\hat{P}_{1}\left(w_{1}\right)\hat{P}_{2}\left(w_{2}\right)\right),z_{2}\right)\hat{F}_{1}\left(w_{1},w_{2};\tau_{1}\right),
\end{eqnarray*}


\begin{eqnarray*}
F_{1}\left(z_{1},z_{2},w_{1},w_{2}\right)&=&R_{2}\left(P_{1}\left(z_{1}\right)\tilde{P}_{2}\left(z_{2}\right)\prod_{i=1}^{2}
\hat{P}_{i}\left(w_{i}\right)\right)F_{2}\left(z_{1},\tilde{\theta}_{2}\left(P_{1}\left(z_{1}\right)\hat{P}_{1}\left(w_{1}\right)\hat{P}_{2}\left(w_{2}\right)\right)\right)
\hat{F}_{2}\left(w_{1},w_{2};\tau_{2}\right),
\end{eqnarray*}

\begin{eqnarray*}
\hat{F}_{2}\left(z_{1},z_{2},w_{1},w_{2}\right)&=&\hat{R}_{1}\left(P_{1}\left(z_{1}\right)\tilde{P}_{2}\left(z_{2}\right)\prod_{i=1}^{2}
\hat{P}_{i}\left(w_{i}\right)\right)F_{1}\left(z_{1},z_{2};\zeta_{1}\right)\hat{F}_{1}\left(\hat{\theta}_{1}\left(P_{1}\left(z_{1}\right)\tilde{P}_{2}\left(z_{2}\right)\hat{P}_{2}\left(w_{2}\right)\right),w_{2}\right),
\end{eqnarray*}

\begin{eqnarray*}
\hat{F}_{1}\left(z_{1},z_{2},w_{1},w_{2}\right)&=&\hat{R}_{2}\left(P_{1}\left(z_{1}\right)\tilde{P}_{2}\left(z_{2}\right)\prod_{i=1}^{2}
\hat{P}_{i}\left(w_{i}\right)\right)F_{2}\left(z_{1},z_{2};\zeta_{2}\right)\hat{F}_{2}\left(w_{1},\hat{\theta}_{2}\left(P_{1}\left(z_{1}\right)\tilde{P}_{2}\left(z_{2}\right)\hat{P}_{1}\left(w_{1}
\right)\right)\right),
\end{eqnarray*}

donde :

\begin{eqnarray}\label{Ec.Ri}
R_{i}\left(\mathbf{z,w}\right)=R_{i}\left(P_{1}\left(z_{1}\right)\tilde{P}_{2}\left(z_{2}\right)\hat{P}_{1}\left(w_{1}\right)\hat{P}_{2}\left(w_{2}\right)\right)
\end{eqnarray}

con
\begin{eqnarray}\label{Ec.Derivada.Ri}
D_{i}R_{i}&=&DR_{i}D_{i}P_{i}
\end{eqnarray}
y convenciones:

\begin{eqnarray*}
\begin{array}{llll}
D_{2}P_{2}\equiv D_{2}\tilde{P}_{2}, & D_{3}P_{3}\equiv D_{3}\hat{P}_{1}, &D_{4}P_{4}\equiv D_{4}\hat{P}_{2},
\end{array}
\end{eqnarray*}

Tambi\'en recordemos que  $F_{1,2}\left(z_{1};\zeta_{2}\right)F_{2,2}\left(z_{2};\zeta_{2}\right)=F_{2}\left(z_{1},z_{2};\zeta_{2}\right)$, entonces

\begin{eqnarray*}
D_{1}F_{2}\left(z_{1},z_{2};\zeta_{2}\right)&=&D_{1}\left[F_{1,2}\left(z_{1};\zeta_{2}\right)F_{2,2}\left(z_{2};\zeta_{2}\right)\right]
=F_{2,2}\left(z_{2};\zeta_{2}\right)D_{1}F_{1,2}\left(z_{1};\zeta_{2}\right)=F_{1,2}^{(1)}\left(1\right)
\end{eqnarray*}

es decir, $D_{1}F_{2}=F_{1,2}^{(1)}(1)$; $D_{2}F_{2}=F_{2,2}^{(1)}\left(1\right)$, mientras que $D_{3}F_{2}=D_{4}F_{2}=0$, es decir,

\begin{eqnarray*}
\begin{array}{ccc}
D_{i}F_{j}=\indora_{i\leq2}F_{i,j}^{(1)}\left(1\right),& \textrm{ y } &D_{i}\hat{F}_{j}=\indora_{i\geq2}F_{i,j}^{(1)}\left(1\right)
\end{array}
\end{eqnarray*}

$D_{4}F_{1}=D_{1}F_{1}D\theta_{1}D_{4}\hat{P}_{2}+D_{4}\hat{F}_{1}$, en t\'erminos generales:

\begin{eqnarray*}
\begin{array}{ll}
D_{i}F_{1}=\indora_{i\neq1}D_{1}F_{1}D\theta_{1}D_{i}P_{i}+\indora_{i=2}D_{i}F_{1}, & D_{i}F_{2}=\indora_{i\neq2}D_{2}F_{2}D\tilde{\theta}_{2}D_{i}P_{i}+\indora_{i=1}D_{i}F_{2}\\
D_{i}\hat{F}_{1}=\indora_{i\neq3}D_{3}\hat{F}_{1}D\hat{\theta}_{1}D_{i}P_{i}+\indora_{i=4}D_{i}\hat{F}_{1},& D_{i}\hat{F}_{2}=\indora_{i\neq4}D_{4}\hat{F}_{2}D\hat{\theta}_{2}D_{i}P_{i}+\indora_{i=3}D_{i}\hat{F}_{2}.
\end{array}
\end{eqnarray*}

\begin{eqnarray}
D_{i}F_{1}&=&\indora_{i\neq1}D_{1}F_{1}D\theta_{1}D_{i}P_{i}+\indora_{i=2}D_{i}F_{1},\\ D_{i}F_{2}&=&\indora_{i\neq2}D_{2}F_{2}D\tilde{\theta}_{2}D_{i}P_{i}+\indora_{i=1}D_{i}F_{2}\\
D_{i}\hat{F}_{1}&=&\indora_{i\neq3}D_{3}\hat{F}_{1}D\hat{\theta}_{1}D_{i}P_{i}+\indora_{i=4}D_{i}\hat{F}_{1},\\
D_{i}\hat{F}_{2}&=&\indora_{i\neq4}D_{4}\hat{F}_{2}D\hat{\theta}_{2}D_{i}P_{i}+\indora_{i=3}D_{i}\hat{F}_{2}.
\end{eqnarray}

Hagamos lo correspondiente para las longitudes de las colas de la RSVC utilizando las expresiones obtenidas en las secciones anteriores, recordemos que

\begin{eqnarray*}
\mathbf{F}_{1}\left(\theta_{1}\left(\tilde{P}_{2}\left(z_{2}\right)\hat{P}_{1}\left(w_{1}\right)
\hat{P}_{2}\left(w_{2}\right)\right),z_{2},w_{1},w_{2}\right)=
F_{1}\left(\theta_{1}\left(\tilde{P}_{2}\left(z_{2}\right)\hat{P}_{1}\left(w_{1}
\right)\hat{P}_{2}\left(w_{2}\right)\right),z_{2}\right)
\hat{F}_{1}\left(w_{1},w_{2};\tau_{1}\right)\\
\end{eqnarray*}

entonces



\begin{eqnarray*}
D_{1}\mathbf{F}_{1}&=& 0\\
D_{2}\mathbf{F}_{1}&=&f_{1}\left(1\right)\left(\frac{1}{1-\mu_{1}}\right)\tilde{\mu}_{2}+f_{1}\left(2\right)\\
D_{3}\mathbf{F}_{1}&=&f_{1}\left(1\right)\left(\frac{1}{1-\mu_{1}}\right)\hat{\mu}_{1}+\hat{F}_{1,1}^{(1)}\left(1\right)\\
D_{4}\mathbf{F}_{1}&=&f_{1}\left(1\right)\left(\frac{1}{1-\mu_{1}}\right)\hat{\mu}_{2}+\hat{F}_{2,1}^{(1)}\left(1\right)
\end{eqnarray*}


para $\tau_{2}$:

\begin{eqnarray*}
\mathbf{F}_{2}\left(z_{1},\tilde{\theta}_{2}\left(P_{1}\left(z_{1}\right)\hat{P}_{1}\left(w_{1}\right)\hat{P}_{2}\left(w_{2}\right)\right),
w_{1},w_{2}\right)=F_{2}\left(z_{1},\tilde{\theta}_{2}\left(P_{1}\left(z_{1}\right)\hat{P}_{1}\left(w_{1}\right)
\hat{P}_{2}\left(w_{2}\right)\right)\right)\hat{F}_{2}\left(w_{1},w_{2};\tau_{2}\right)
\end{eqnarray*}
se tiene que

\begin{eqnarray*}
D_{1}\mathbf{F}_{2}&=&f_{2}\left(2\right)\left(\frac{1}{1-\tilde{\mu}_{2}}\right)\mu_{1}+f_{2}\left(1\right)\\
D_{2}\mathbf{F}_{2}&=&0\\
D_{3}\mathbf{F}_{2}&=&f_{2}\left(2\right)\left(\frac{1}{1-\tilde{\mu}_{2}}\right)\hat{\mu}_{1}+\hat{F}_{2,1}^{(1)}\left(1\right)\\
D_{4}\mathbf{F}_{2}&=&f_{2}\left(2\right)\left(\frac{1}{1-\tilde{\mu}_{2}}\right)\hat{\mu}_{2}+\hat{F}_{2,2}^{(1)}\left(1\right)\\
\end{eqnarray*}



Ahora para el segundo sistema

\begin{eqnarray*}\hat{\mathbf{F}}_{1}\left(z_{1},z_{2},\hat{\theta}_{1}\left(P_{1}\left(z_{1}\right)\tilde{P}_{2}\left(z_{2}\right)\hat{P}_{2}\left(w_{2}\right)\right),
w_{2}\right)=F_{1}\left(z_{1},z_{2};\zeta_{1}\right)\hat{F}_{1}\left(\hat{\theta}_{1}\left(P_{1}\left(z_{1}\right)\tilde{P}_{2}\left(z_{2}\right)
\hat{P}_{2}\left(w_{2}\right)\right),w_{2}\right)
\end{eqnarray*}
entonces

\begin{eqnarray*}
D_{1}\hat{\mathbf{F}}_{1}&=&\hat{f}_{1}\left(1\right)\left(\frac{1}{1-\hat{\mu}_{1}}\right)\mu_{1}+F_{1,1}^{(1)}\left(1\right)\\
D_{2}\hat{\mathbf{F}}_{1}&=&\hat{f}_{1}\left(1\right)\left(\frac{1}{1-\hat{\mu}_{1}}\right)\tilde{\mu}_{2}+F_{2,1}^{(1)}\left(1\right)\\
D_{3}\hat{\mathbf{F}}_{1}&=&0\\
D_{4}\hat{\mathbf{F}}_{1}&=&\hat{f}_{1}\left(1\right)\left(\frac{1}{1-\hat{\mu}_{1}}\right)\hat{\mu}_{2}+\hat{f}_{1}\left(2\right)\\
\end{eqnarray*}




Finalmente para $\zeta_{2}$

\begin{eqnarray*}\hat{\mathbf{F}}_{2}\left(z_{1},z_{2},w_{1},\hat{\theta}_{2}\left(P_{1}\left(z_{1}\right)\tilde{P}_{2}\left(z_{2}\right)\hat{P}_{1}\left(w_{1}\right)\right)\right)&=&F_{2}\left(z_{1},z_{2};\zeta_{2}\right)\hat{F}_{2}\left(w_{1},\hat{\theta}_{2}\left(P_{1}\left(z_{1}\right)\tilde{P}_{2}\left(z_{2}\right)\hat{P}_{1}\left(w_{1}\right)\right)\right]
\end{eqnarray*}
por tanto:


\begin{eqnarray*}
D_{1}\hat{\mathbf{F}}_{2}&=&\hat{f}_{2}\left(1\right)\left(\frac{1}{1-\hat{\mu}_{2}}\right)\mu_{1}+F_{1,2}^{(1)}\left(1\right)\\
D_{2}\hat{\mathbf{F}}_{2}&=&\hat{f}_{2}\left(1\right)\left(\frac{1}{1-\hat{\mu}_{2}}\right)\tilde{\mu}_{2}+F_{2,2}^{(1)}\left(1\right)\\
D_{3}\hat{\mathbf{F}}_{2}&=&\hat{f}_{2}\left(1\right)\left(\frac{1}{1-\hat{\mu}_{2}}\right)\hat{\mu}_{1}+\hat{f}_{2}\left(1\right)\\
D_{4}\hat{\mathbf{F}}_{2}&=&0\\
\end{eqnarray*}


Entonces, de todo lo desarrollado hasta ahora se tienen las siguientes ecuaciones:

%Para $$, se tiene que


\begin{eqnarray}\label{Ec.Primeras.Derivadas.Parciales}
\begin{array}{ll}
\mathbf{F}_{1}=R_{2}F_{2}\hat{F}_{2}, & D_{i}\mathbf{F}_{1}=D_{i}\left(R_{2}+F_{2}+\indora_{i\geq3}\hat{F}_{2}\right)\\
\mathbf{F}_{2}=R_{1}F_{1}\hat{F}_{1}, & D_{i}\mathbf{F}_{2}=D_{i}\left(R_{1}+F_{1}+\indora_{i\geq3}\hat{F}_{1}\right)\\
\hat{\mathbf{F}}_{1}=\hat{R}_{2}\hat{F}_{2}F_{2}, & D_{i}\hat{\mathbf{F}}_{1}=D_{i}\left(\hat{R}_{2}+\hat{F}_{2}+\indora_{i\leq2}F_{2}\right)\\
\hat{\mathbf{F}}_{2}=\hat{R}_{1}\hat{F}_{1}F_{1}, & D_{i}\hat{\mathbf{F}}_{2}=D_{i}\left(\hat{R}_{1}+\hat{F}_{1}+\indora_{i\leq2}F_{1}\right)
\end{array}
\end{eqnarray}
%___________________________________________________________________________________________
%
\subsection{Derivadas de Orden Superior}
%___________________________________________________________________________________________
%
\small{
\begin{eqnarray*}\label{Ec.Derivadas.Segundo.Orden}
D_{k}D_{i}F_{1}&=&D_{k}D_{i}\left(R_{2}+F_{2}+\indora_{i\geq3}\hat{F}_{4}\right)+D_{i}R_{2}D_{k}\left(F_{2}+\indora_{k\geq3}\hat{F}_{4}\right)+D_{i}F_{2}D_{k}\left(R_{2}+\indora_{k\geq3}\hat{F}_{4}\right)+\indora_{i\geq3}D_{i}\hat{F}_{4}D_{k}\left(R_{}+F_{2}\right)\\
D_{k}D_{i}F_{2}&=&D_{k}D_{i}\left(R_{1}+F_{1}+\indora_{i\geq3}\hat{F}_{3}\right)+D_{i}R_{1}D_{k}\left(F_{1}+\indora_{k\geq3}\hat{F}_{3}\right)+D_{i}F_{1}D_{k}\left(R_{1}+\indora_{k\geq3}\hat{F}_{3}\right)+\indora_{i\geq3}D_{i}\hat{F}_{3}D_{k}\left(R_{1}+F_{1}\right)\\
D_{k}D_{i}\hat{F}_{3}&=&D_{k}D_{i}\left(\hat{R}_{4}+\indora_{i\leq2}F_{2}+\hat{F}_{4}\right)+D_{i}\hat{R}_{4}D_{k}\left(\indora_{k\leq2}F_{2}+\hat{F}_{4}\right)+D_{i}\hat{F}_{4}D_{k}\left(\hat{R}_{4}+\indora_{k\leq2}F_{2}\right)+\indora_{i\leq2}D_{i}F_{2}D_{k}\left(\hat{R}_{4}+\hat{F}_{4}\right)\\
D_{k}D_{i}\hat{F}_{4}&=&D_{k}D_{i}\left(\hat{R}_{3}+\indora_{i\leq2}F_{1}+\hat{F}_{3}\right)+D_{i}\hat{R}_{3}D_{k}\left(\indora_{k\leq2}F_{1}+\hat{F}_{3}\right)+D_{i}\hat{F}_{3}D_{k}\left(\hat{R}_{3}+\indora_{k\leq2}F_{1}\right)+\indora_{i\leq2}D_{i}F_{1}D_{k}\left(\hat{R}_{3}+\hat{F}_{3}\right)
\end{eqnarray*}}
para $i,k=1,\ldots,4$. Es necesario determinar las derivadas de segundo orden para las expresiones de la forma $D_{k}D_{i}\left(R_{2}+F_{2}+\indora_{i\geq3}\hat{F}_{4}\right)$

A saber, $R_{i}\left(z_{1},z_{2},w_{1},w_{2}\right)=R_{i}\left(P_{1}\left(z_{1}\right)\tilde{P}_{2}\left(z_{2}\right)
\hat{P}_{1}\left(w_{1}\right)\hat{P}_{2}\left(w_{2}\right)\right)$, la denotaremos por la expresi\'on $R_{i}=R_{i}\left(
P_{1}\tilde{P}_{2}\hat{P}_{1}\hat{P}_{2}\right)$, donde al igual que antes, utilizando la notaci\'on dada en \cite{Lang} se tiene   que

\begin{eqnarray}
D_{i}D_{i}R_{k}=D^{2}R_{k}\left(D_{i}P_{i}\right)^{2}+DR_{k}D_{i}D_{i}P_{i}
\end{eqnarray}

mientras que para $i\neq j$

\begin{eqnarray}
D_{i}D_{j}R_{k}=D^{2}R_{k}D_{i}P_{i}D_{j}P_{j}+DR_{k}D_{j}P_{j}D_{i}P_{i}
\end{eqnarray}

Recordemos la expresi\'on $F_{1}\left(\theta_{1}\left(\tilde{P}_{2}\left(z_{2}\right)\hat{P}_{1}\left(w_{1}\right)\hat{P}_{2}\left(w_{2}\right)\right),
z_{2}\right)$, que denotaremos por $F_{1}\left(\theta_{1}\left(\tilde{P}_{2}\hat{P}_{1}\hat{P}_{2}\right),z_{2}\right)$, entonces las derivadas parciales mixtas son:

\begin{eqnarray*}
D_{i}F_{1}=\indora_{i\geq2}D_{i}F_{1}D\theta_{1}D_{i}P_{i}+\indora_{i=2} D_{i}F_{1},
\end{eqnarray*}

entonces para
$F_{1}\left(\theta_{1}\left(\tilde{P}_{2}\hat{P}_{1}\hat{P}_{2}\right),z_{2}\right)$

$$D_{2}F_{1}=D_{1}F_{1}D_{1}\theta_{1}D_{2}\tilde{P}_{2}\left\{\hat{P}_{1}\hat{P}_{2}\right\}+D_{2}F_{1}$$

\begin{eqnarray*}
D_{j}D_{i}F_{1}&=&\indora_{i,j\neq1}D_{1}D_{1}F_{1}\left(D\theta_{1}\right)^{2}D_{i}P_{i}D_{j}P_{j}+\indora_{i,j\neq1}D_{1}F_{1}D^{2}\theta_{1}D_{i}P_{i}D_{j}P_{j}\\
&+&\indora_{i,j\neq1}D_{1}F_{1}D\theta_{1}\left(\indora_{i=j}D_{i}^{2}P_{i}+\indora_{i\neq j}D_{i}P_{i}D_{j}P_{j}\right)\\
&+&\indora_{i,j\neq1}D_{1}D_{2}F_{1}D\theta_{1}D_{i}P_{i}+\indora_{i=2}\left(D_{1}D_{2}F_{1}D\theta_{1}D_{i}P_{i}+D_{i}^{2}F_{1}\right)
\end{eqnarray*}


Para $F_{2}\left(z_{1},\tilde{\theta}_{2}\left(P_{1}\hat{P}_{1}\hat{P}_{2}\right)\right)$

\begin{eqnarray*}
D_{i}F_{2}=\indora_{i\neq2}D_{2}F_{2}D\tilde{\theta}_{2}D_{i}P_{i}+\indora_{i=1} D_{i}F_{2},
\end{eqnarray*}


%\begin{eqnarray*}
%D_{j}D_{i}F_{2}&=&
%\indora_{i,j\neq1}D_{2}^{2}F_{2}\left(D\tilde{\theta}_{2}\right)^{2}_{i}P_{i}D_{j}P_{j}+\indora_{i,j\neq2}D_{2}F_{2}D^{2}\tilde{\theta}_{2}D_{i}P_{i}D_{j}P_{j}\\
%&+&\indora_{i,j\neq2}D_{2}F_{2}D\tilde{\theta}_{2}D_{i}P_{i}D_{j}P_{j}+\indora_{i=j}D_{i}P_{i}D_{j}P_{j}\left(\indora_{i=j}D_{i}^{2}P_{i}+\indora_{i\neq j}D_{i}P_{i}D_{j}P_{j}\right)\\
%&+&\indora_{i,j\neq1}D_{1}D_{2}F_{1}D\theta_{1}D_{i}P_{i}+\indora_{i=2}\left(D_{1}D_{2}F_{1}D\theta_{1}D_{i}P_{i}+D_{i}^{2}F_{1}\right)
%\end{eqnarray*}


\begin{eqnarray*}
D_{j}D_{i}F_{2}&=&\indora_{i,j\neq2}D_{2}D_{21}F_{2}\left(D\theta_{2}\right)^{2}D_{i}P_{i}D_{j}P_{j}+\indora_{i,j\neq2}D_{2}F_{2}D^{2}\theta_{2}D_{i}P_{i}D_{j}P_{j}\\
&+&\indora_{i,j\neq2}D_{2}F_{2}D\theta_{2}\left(\indora_{i=j}D_{i}^{2}P_{i}+\indora_{i\neq j}D_{i}P_{i}D_{j}P_{j}\right)\\
&+&\indora_{i,j\neq2}D_{2}D_{1}F_{2}D\theta_{2}D_{i}P_{i}+\indora_{i=2}\left(D_{2}D_{1}F_{2}D\theta_{2}D_{i}P_{i}+D_{i}^{2}F_{2}\right)
\end{eqnarray*}



\begin{eqnarray*}
D_{1}D_{1}F_{2}&=&
D_{1}^{2}P_{1}D\tilde{\theta}_{2}D_{2}F_{2}+
\left(D_{1}P_{1}\right)^{2}D^{2}\tilde{\theta}_{2}D_{2}F_{2}+
D_{1}P_{1}D\tilde{\theta}_{2}D_{1}D_{2}F_{2}+
\left(D_{1}P_{1}\right)^{2}\left(D\tilde{\theta}_{2}\right)^{2}D_{2}^{2}F_{2}+
D_{1}P_{1}D\tilde{\theta}_{2}D_{1}D_{2}F_{2}+
D_{1}^{2}F_{2}\\
D_{2}D_{1}F_{2}&=&0\\
D_{3}D_{1}F_{2}&=&D_{2}D_{1}F_{2}D\tilde{\theta}_{2}D_{3}\hat{P}_{1}
+D_{2}^{2}F_{2}\left(D\tilde{\theta}_{2}\right)^{2}D_{3}P_{1}D_{1}P_{1}
+D_{2}F_{2}D^{2}\tilde{\theta}_{2}D_{3}\hat{P}_{1}D_{1}P_{1}
+D_{2}F_{2}D\tilde{\theta}_{2}D_{1}P_{1}D_{3}\hat{P}_{1}\\
D_{4}D_{1}F_{2}&=&D_{1}D_{2}F_{2}D\tilde{\theta}_{2}D_{4}\hat{P}_{2}
+D_{2}^{2}F_{2}\left(D\tilde{\theta}_{2}\right)^{2}D_{4}P_{2}D_{1}P_{1}
+D_{2}F_{2}D^{2}\tilde{\theta}_{2}D_{4}\hat{P}_{2}D_{1}P_{1}
+D_{2}F_{2}D\tilde{\theta}_{2}D_{1}P_{1}D_{4}\hat{P}_{2}\\
D_{1}D_{3}F_{2}&=&D_{2}^{2}F_{2}\left(D\tilde{\theta}_{2}\right)^{2}D_{1}P_{1}D_{3}\hat{P}_{1}
+D_{2}D_{1}F_{2}D\tilde{\theta}_{2}D_{3}\hat{P}_{1}
+D_{2}F_{2}D^{2}\tilde{\theta}_{2}D_{1}P_{1}D_{3}\hat{P}_{1}
+D_{2}F_{2}D\tilde{\theta}_{2}D_{3}\hat{P}_{1}D_{1}P_{1}\\
D_{2}D_{3}F_{3}&=&0\\
D_{3}D_{3}F_{2}&=&D_{2}^{2}F_{2}\left(D\tilde{\theta}_{2}\right)^{2}\left(D_{3}\hat{P}_{1}\right)^{2}
+D_{2}F_{2}\left(D_{3}\hat{P}_{1}\right)^{2}D^{2}\tilde{\theta}_{2}
+D_{2}F_{2}D\tilde{\theta}_{2}D_{3}^{2}\hat{P}_{1}\\
D_{4}D_{3}F_{2}&=&D_{2}^{2}F_{2}\left(D\tilde{\theta}_{2}\right)^{2}D_{4}\hat{P}_{2}D_{3}\hat{P}_{1}
+D_{2}F_{2}D^{2}\tilde{\theta}_{2}D_{4}\hat{P}_{2}D_{3}\hat{P}_{1}
+D_{2}F_{2}D\tilde{\theta}_{2}D_{3}\hat{P}_{1}D_{4}\hat{P}_{2}\\
D_{1}D_{4}F_{2}&=&D_{2}^{2}F_{2}\left(D\tilde{\theta}_{2}\right)^{2}D_{1}P_{1}D_{4}\hat{P}_{2}
+D_{1}D_{2}F_{2}D\tilde{\theta}_{2}D_{4}\hat{P}_{2}
+D_{2}F_{2}D^{2}\tilde{\theta}_{2}D_{1}P_{1}D_{4}\hat{P}_{2}
+D_{2}F_{2}D\tilde{\theta}_{2}D_{4}\hat{P}_{2}D_{1}P_{1}\\
D_{2}D_{4}F_{2}&=&0\\
D_{3}D_{4}F_{2}&=&
D_{2}F_{2}D\tilde{\theta}_{2}D_{4}\hat{P}_{2}D_{3}\hat{P}_{1}
+D_{2}F_{2}D^{2}\tilde{\theta}_{2}D_{4}\hat{P}_{2}D_{3}\hat{P}_{1}
+D_{2}^{2}F_{2}\left(D\tilde{\theta}_{2}\right)^{2}D_{4}\hat{P}_{2}D_{3}\hat{P}_{1}\\
D_{4}D_{4}F_{2}&=&D_{2}F_{2}D\tilde{\theta}_{2}D_{4}^{2}\hat{P}_{2}
+D_{2}F_{2}D^{2}\tilde{\theta}_{2}\left(D_{4}\hat{P}_{2}\right)^{2}
+D_{2}^{2}F_{2}\left(D\tilde{\theta}_{2}\right)^{2}\left(D_{4}\hat{P}_{2}\right)^{2}\\
\end{eqnarray*}


%\newpage



%\newpage

para $\hat{F}_{1}\left(\hat{\theta}_{1}\left(P_{1}\tilde{P}_{2}\hat{P}_{2}\right),w_{2}\right)$

\begin{eqnarray*}
D_{i}\hat{F}_{1}=\indora_{i\neq3}D_{3}\hat{F}_{1}D\hat{\theta}_{1}D_{i}P_{i}+\indora_{i=4}D_{i}\hat{F}_{1},
\end{eqnarray*}


\begin{eqnarray*}
D_{1}D_{1}\hat{F}_{1}&=&
D\hat{\theta}_{1}D_{1}^{2}P_{1}D_{1}\hat{F}_{1}
+\left(D_{1}P_{1}\right)^{2}D^{2}\hat{\theta}_{1}D_{1}\hat{F}_{1}
+\left(D_{1}P_{1}\right)^{2}\left(D\hat{\theta}_{1}\right)^{2}D_{1}^{2}\hat{F}_{1}\\
D_{2}D_{1}\hat{F}_{1}&=&D_{1}P_{1}D_{2}P_{2}D\hat{\theta}_{1}D_{1}\hat{F}_{1}+
D_{1}P_{1}D_{2}P_{2}D^{2}\hat{\theta}_{1}D_{1}\hat{F}_{1}+
D_{1}P_{1}D_{2}P_{1}\left(D\hat{\theta}_{1}\right)^{2}D_{1}^{2}\hat{\theta}_{1}\\
D_{3}D_{1}\hat{F}_{1}&=&0\\
D_{4}D_{1}\hat{F}_{1}&=&D_{1}P_{1}D_{4}\hat{P}_{2}D\hat{\theta}_{1}D_{1}\hat{F}_{1}
+D_{1}P_{1}D_{4}\hat{P}_{2}D^{2}\hat{\theta}_{1}D_{1}\hat{F}_{1}
+D_{1}P_{1}D\hat{\theta}_{1}D_{2}D{1}\hat{F}_{1}
+D_{1}P_{1}D\hat{\theta}_{1}D_{1}D_{1}\hat{F}_{1}\\
D_{1}D_{2}\hat{F}_{1}&=&D_{1}P_{1}D_{2}P_{2}D\hat{\theta}_{1}D_{1}\hat{F}_{1}+
D_{1}P_{1}D_{2}P_{2}D^{2}\hat{\theta}_{1}D_{1}\hat{F}_{1}+
D_{1}P_{1}D_{2}P_{2}\left(D\hat{\theta}_{1}\right)^{2}D_{1}^{2}\hat{F}_{1}\\
D_{2}D_{2}\hat{F}_{1}&=&
D\hat{\theta}_{1}D_{2}^{2}P_{2}D_{1}\hat{F}_{1}+
 \left(D_{2}P_{2}\right)^{2}D^{2}\hat{\theta}_{1}D_{1}\hat{F}_{1}+
\left(D_{2}P_{2}\right)^{2}\left(D\hat{\theta}_{1}\right)^{2}D_{1}^{2}\hat{F}_{1}\\
D_{3}D_{2}\hat{F}_{1}&=&0\\
D_{4}D_{2}\hat{F}_{1}&=&D_{2}P_{2}D_{4}\hat{P}_{2}D\hat{\theta}_{1}D\hat{F}_{1}
+D_{2}P_{2}D_{4}\hat{P}_{2}D^{2}\hat{\theta}_{1}D_{1}\hat{F}_{1}
+D_{2}P_{2}D\hat{\theta}_{1}D_{2}D_{1}\hat{F}_{1}
+D_{2}P_{2}\left(D\hat{\theta}_{1}\right)^{2}D_{4}\hat{P}_{2}D_{1}^{2}\hat{F}_{1}\\
D_{1}D_{3}\hat{F}_{1}&=&0\\
D_{2}D_{3}\hat{F}_{1}&=&0\\
D_{3}D_{3}\hat{F}_{1}&=&0\\
D_{4}D_{3}\hat{F}_{1}&=&0\\
D_{1}D_{4}\hat{F}_{1}&=&D_{1}P_{1}D_{4}\hat{F}_{2}D\hat{\theta}_{1}D_{1}\hat{F}_{1}
+D_{1}P_{1}D_{4}\hat{P}_{2}D^{2}\hat{\theta}_{1}D_{1}\hat{F}_{1}
+D_{1}P_{1}D\hat{\theta}_{1}D_{2}D_{1}\hat{F}_{1}
+ D_{1}P_{1}D_{4}\hat{P}_{2}\left(D\hat{\theta}_{1}\right)^{2}D_{1}D_{1}\hat{F}_{1}\\
D_{2}D_{4}\hat{F}_{1}&=&D_{2}P_{2}D_{4}\hat{P}_{2}D\hat{\theta}_{1}D_{1}
\hat{F}_{1}
+D_{2}P_{2}D_{4}\hat{P}_{2}D^{2}\hat{\theta}_{1}D_{1}\hat{F}_{1}
+D_{2}P_{2}D\hat{\theta}_{1}D_{2}D_{1}\hat{F}_{1}+
D_{2}P_{2}D_{4}\hat{P}_{2}\left(D\hat{\theta}_{1}\right)^{2}D_{1}^{2}\hat{F}_{1}\\
D_{3}D_{4}\hat{F}_{1}&=&0\\
D_{4}D_{4}\hat{F}_{1}&=&D_{2}D_{2}\hat{F}_{1}+D\hat{\theta}_{1}D_{4}^{2}\hat{P}_{2}D_{1}\hat{F}_{1}
+\left(D_{4}\hat{P}_{2}\right)^{2}D^{2}\hat{\theta}_{1}D_{1}\hat{F}_{1}+
D_{4}\hat{P}_{2}D\hat{\theta}_{1}D_{2}D_{1}\hat{F}_{1}\\
&+&D_{4}\hat{P}_{2}D\hat{\theta}_{1}D_{2}D_{1}\hat{F}_{1}+ \left(D_{4}\hat{P}_{2}\right)^{2}D\hat{\theta}_{1}D\hat{\theta}_{1}D_{1}^{2}\hat{F}_{1}\\
\end{eqnarray*}




%\newpage
finalmente, para $\hat{F}_{2}\left(w_{1},\hat{\theta}_{2}\left(P_{1}\tilde{P}_{2}\hat{P}_{1}\right)\right)$

\begin{eqnarray*}
D_{i}\hat{F}_{2}=\indora_{i\neq4}D_{4}\hat{F}_{2}D\hat{\theta}_{2}D_{i}P_{i}+\indora_{i=3}D_{i}\hat{F}_{2},
\end{eqnarray*}

\begin{eqnarray*}
D_{1}D_{1}\hat{F}_{2}&=&D_{1}\hat{\theta}_{2}D_{2}^{2}P_{1}D_{2}\hat{F}_{2}
+\left(D_{1}P_{1}\right)^{2}D_{1}^{2}\hat{\theta}_{2}D_{2}\hat{F}_{2}+
\left(D_{1}P_{1}\right)^{2}\left(D\hat{\theta}_{2}\right)^{2}D_{1}^{2}\hat{F}_{2}\\
D_{2}D_{1}\hat{F}_{2}&=&D_{1}P_{1}D_{2}P_{2}D\hat{\theta}_{2}D_{2}\hat{F}_{2}+
D_{1}P_{1}D_{2}P_{2}D^{2}\hat{\theta}_{2}D_{2}\hat{F}_{2}+
D_{1}P_{1}D_{2}P_{2}\left(D\hat{\theta}_{2}\right)^{2}D_{2}^{2}\hat{F}_{2}\\
D_{3}D_{1}\hat{F}_{2}&=&
D_{1}P_{1}D_{3}\hat{P}_{1}D\hat{\theta}_{2}D_{2}\hat{F}_{2}
+D_{1}P_{1}D_{3}\hat{P}_{1}D^{2}\hat{\theta}_{2}D_{2}\hat{F}_{2}
+D_{1}P_{1}D_{3}\hat{P}_{1}\left(D\hat{\theta}_{2}\right)^{2}D_{2}^{2}\hat{F}_{2}
+D_{1}P_{1}D\hat{\theta}_{2}D_{1}D_{2}\hat{F}_{2}\\
D_{4}D_{1}\hat{F}_{2}&=&0\\
D_{1}D_{2}\hat{F}_{2}&=&
D_{1}P_{1}D_{2}P_{2}D\hat{\theta}_{2}D_{2}\hat{F}_{2}+
D_{1}P_{1}D_{2}P_{2}D^{2}\hat{\theta}_{2}D_{2}\hat{F}_{2}+
D_{1}P_{1}D_{2}P_{2}\left(D\hat{\theta}_{2}\right)^{2}D_{2}D_{2}\hat{F}_{2}\\
D_{2}D_{2}\hat{F}_{2}&=&
D\hat{\theta}_{2}D_{2}^{2}P_{2}D_{2}\hat{F}_{2}+
\left(D_{2}P_{2}\right)^{2}D^{2}\hat{\theta}_{2}D_{2}\hat{F}_{2}+
\left(D_{2}P_{2}\right)^{2}\left(D\hat{\theta}_{2}\right)^{2}D_{2}^{2}\hat{F}_{2}\\
D_{3}D_{2}\hat{F}_{2}&=&
D_{2}P_{2}D_{3}\hat{P}_{1}D\hat{\theta} _{2}D_{2}\hat{F}_{2}
+D_{2}P_{2}D_{3}\hat{P}_{1}D^{2}\hat{\theta}_{2}D_{2}\hat{F}_{2}
+D_{2}P_{2}D_{3}\hat{P}_{1}\left(D\hat{\theta}_{2}\right)^{2}D_{2}^{2}\hat{F}_{2}
+D_{2}P_{2}D\hat{\theta}_{2}D_{1}D_{2}\hat{F}_{2}\\
D_{4}D_{2}\hat{F}_{2}&=&0\\
D_{1}D_{3}\hat{F}_{2}&=&
D_{1}P_{1}D_{3}\hat{P}_{1}D\hat{\theta}_{2}D_{2}\hat{F}_{2}
+D_{1}P_{1}D_{3}\hat{P}_{1}D^{2}\hat{\theta}_{2}D_{2}\hat{F}_{2}
+D_{1}P_{1}D_{3}\hat{P}_{1}\left(D\hat{\theta}_{2}\right)^{2}D_{2}D_{2}\hat{F}_{2}
+D_{1}P_{1}D\hat{\theta}_{2}D_{2}D_{1}\hat{F}_{2}\\
D_{2}D_{3}\hat{F}_{2}&=&
D_{2}P_{2}D_{3}\hat{P}_{1}D\hat{\theta}_{2}D_{2}\hat{F}_{2}
+D_{2}P_{2}D_{3}\hat{P}_{1}D^{2}\hat{\theta}_{2}D_{2}\hat{F}_{2}
+D_{2}P_{2}D_{3}\hat{P}_{1}\left(D\hat{\theta}_{2}\right)^{2}D_{2}^{2}\hat{F}_{2}
+D_{2}P_{2}D\hat{\theta}_{2}D_{1}D_{2}\hat{F}_{2}\\
D_{3}D_{3}\hat{F}_{2}&=&
D_{3}^{2}\hat{P}_{1}D\hat{\theta}_{2}D_{2}\hat{F}_{2}
+\left(D_{3}\hat{P}_{1}\right)^{2}D^{2}\hat{\theta}_{2}D_{2}\hat{F}_{2}
+D_{3}\hat{P}_{1}D\hat{\theta}_{2}D_{1}D_{2}\hat{F}_{2}
+ \left(D_{3}\hat{P}_{1}\right)^{2}\left(D\hat{\theta}_{2}\right)^{2}
D_{2}^{2}\hat{F}_{2}\\
&+&D_{3}\hat{P}_{1}D\hat{\theta}_{2}D_{1}D_{2}\hat{F}_{2}
+D_{1}^{2}\hat{F}_{2}\\
D_{4}D_{3}\hat{F}_{2}&=&0\\
D_{1}D_{4}\hat{F}_{2}&=&0\\
D_{2}D_{4}\hat{F}_{2}&=&0\\
D_{3}D_{4}\hat{F}_{2}&=&0\\
D_{4}D_{4}\hat{F}_{2}&=&0\\
\end{eqnarray*}

%__________________________________________________________________
\section{Ejemplos Particulares}
%__________________________________________________________________

%__________________________________________________________________
\subsection{Automatizaci\'on en dos l\'ineas de trabajo}
%__________________________________________________________________
%\begin{figure}[H]
%\centering
%%%\includegraphics[width=9cm]{Grafica1.jpg}
%%\end{figure}\label{RSVC1}

Las ecuaciones recursivas son


\begin{eqnarray*}
F_{1}\left(z_{1},w_{1},w_{2}\right)&=&R_{1}\left(\tilde{P}_{1}\left(z_{1}\right)\prod_{i=1}^{2}
\hat{P}_{i}\left(w_{i}\right)\right)F\left(\tilde{\theta}_{2}\left(\hat{P}_{1}\left(w_{1}\right)\hat{P}_{2}\left(w_{2}\right)\right)\right)
\hat{F}_{2}\left(w_{1},w_{2};\tau\right),\\
\hat{F}_{1}\left(z_{1},w_{1},w_{2}\right)&=&\hat{R}_{2}\left(\tilde{P}_{1}\left(z_{2}\right)\prod_{i=1}^{2}
\hat{P}_{i}\left(w_{i}\right)\right)F\left(z_{1};\zeta_{2}\right)\hat{F}_{2}\left(w_{1},\hat{\theta}_{2}\left(\tilde{P}_{2}\left(z_{2}\right)\hat{P}_{1}\left(w_{1}
\right)\right)\right),\\
\hat{F}_{2}\left(z_{1},w_{1},w_{2}\right)&=&\hat{R}_{1}\left(\tilde{P}_{1}\left(z_{1}\right)\prod_{i=1}^{2}
\hat{P}_{i}\left(w_{i}\right)\right)F\left(z_{1};\zeta_{1}\right)\hat{F}_{1}\left(\hat{\theta}_{1}\left(\tilde{P}_{2}\left(z_{2}\right)\hat{P}_{2}\left(w_{2}\right)\right),w_{2}\right),
\end{eqnarray*}

De la primera ecuaci\'on
\begin{eqnarray*}
F_{1}\left(z_{1},w_{1},w_{2}\right)&=&R\left(\tilde{P}_{1}\left(z_{1}\right)\prod_{i=1}^{2}
\hat{P}_{i}\left(w_{i}\right)\right)F\left(\tilde{\theta}_{2}\left(\hat{P}_{1}\left(w_{1}\right)\hat{P}_{2}\left(w_{2}\right)\right)\right)
\hat{F}_{2}\left(w_{1},w_{2};\tau\right),
\end{eqnarray*}
se desprende

$D_{1}R_{1}=r_{1}\tilde{\mu}_{1}$


Entonces

\begin{eqnarray*}
\begin{array}{ll}
f_{1}\left(1\right)=r_{1}\mu_{1},&f_{1}\left(3\right)=r_{1}\hat{\mu}_{1}+f_{1}\left(1\right)\frac{1}{1-\tilde{\mu}_{1}}\hat{\mu}_{1}+\hat{F}_{2,1}^{(1)}\left(1\right)\\
f_{1}\left(4\right)=r_{1}\hat{\mu}_{2}+f_{1}\left(1\right)\frac{1}{1-\tilde{\mu}_{1}}\hat{\mu}_{2}+\hat{F}_{2,2}^{(1)}\left(1\right),&
\hat{f}_{1}\left(1\right)=\hat{r}_{2}\mu_{1}+\hat{F}_{1,2}^{(1)}\left(1\right)
+\hat{f}_{2}\left(1\right)\frac{1}{1-\hat{\mu}_{2}}\mu_{1}\\
\hat{f}_{1}\left(3\right)=\hat{r}_{2}\hat{\mu}_{1}+\hat{f}_{2}\left(2\right)\frac{1}{1-\hat{\mu}_{2}}\hat{\mu}_{1}+\hat{f}_{2}\left(1\right),&
\hat{f}_{1}\left(4\right)=\hat{r}_{2}\hat{\mu}_{2}\\
\hat{f}_{2}\left(1\right)=\hat{r}_{1}\mu_{1}+\hat{F}_{1,1}^{(1)}\left(1\right)
+\hat{f}_{1}\left(1\right)\frac{1}{1-\hat{\mu}_{1}}\mu_{1},&
\hat{f}_{1}\left(3\right)=\hat{r}_{1}\hat{\mu}_{1}\\
\hat{f}_{1}\left(4\right)=\hat{r}_{1}\hat{\mu}_{2}+\hat{f}_{1}\left(1\right)\frac{1}{1-\hat{\mu}_{1}}\hat{\mu}_{1}+\hat{f}_{1}\left(2\right)\\
\end{array}
\end{eqnarray*}




%__________________________________________________________________
\subsection{Sistema de Salud P\'ublica}
%__________________________________________________________________

%\begin{figure}[H]
%\centering
%%%\includegraphics[width=9cm]{Grafica2.jpg}
%%\end{figure}\label{RSVC2}

Las ecuaciones recursivas son de la forma


\begin{eqnarray*}
F_{1}\left(z_{1},z_{2},w_{1}\right)&=&R_{2}\left(P_{1}\left(z_{1}\right)\tilde{P}_{2}\left(z_{2}\right)
\hat{P}_{1}\left(w_{1}\right)\right)F_{2}\left(z_{1},\tilde{\theta}_{2}\left(P_{1}\left(z_{1}\right)\hat{P}_{1}\left(w_{1}\right)\right)\right)
\hat{F}_{2}\left(w_{1};\tau_{2}\right),
\end{eqnarray*}


\begin{eqnarray*}
F_{2}\left(z_{1},z_{2},w_{1}\right)&=&R_{1}\left(P_{1}\left(z_{1}\right)\tilde{P}_{2}\left(z_{2}\right)
\hat{P}_{1}\left(w_{1}\right)\right)F_{1}\left(\theta_{1}\left(\hat{P}_{1}\left(w_{1}\right)\hat{P}_{2}\left(w_{2}\right)\right),z_{2}\right)\hat{F}_{1}\left(w_{1};\tau_{1}\right),
\end{eqnarray*}



\begin{eqnarray*}
\hat{F}_{1}\left(z_{1},z_{2},w_{1}\right)&=&\hat{R}_{2}\left(P_{1}\left(z_{1}\right)\tilde{P}_{2}\left(z_{2}\right)
\hat{P}_{1}\left(w_{1}\right)\right)F_{2}\left(z_{1},z_{2};\zeta_{2}\right)\hat{F}_{}\left(\hat{\theta}_{1}\left(P_{1}\left(z_{1}\right)\tilde{P}_{2}\left(z_{2}\right)
\right)\right),
\end{eqnarray*}


%__________________________________________________________________
\subsection{RSVC con dos conexiones}
%__________________________________________________________________

%\begin{figure}[H]
%\centering
%%%\includegraphics[width=9cm]{Grafica3.jpg}
%%\end{figure}\label{RSVC3}


Cuyas ecuaciones recursivas son de la forma


\begin{eqnarray*}
F_{1}\left(z_{1},z_{2},w_{1},w_{2}\right)&=&R_{2}\left(\tilde{P}_{1}\left(z_{1}\right)\tilde{P}_{2}\left(z_{2}\right)\prod_{i=1}^{2}
\hat{P}_{i}\left(w_{i}\right)\right)F_{2}\left(z_{1},\tilde{\theta}_{2}\left(\tilde{P}_{1}\left(z_{1}\right)\hat{P}_{1}\left(w_{1}\right)\hat{P}_{2}\left(w_{2}\right)\right)\right)
\hat{F}_{2}\left(w_{1},w_{2};\tau_{2}\right),
\end{eqnarray*}

\begin{eqnarray*}
F_{2}\left(z_{1},z_{2},w_{1},w_{2}\right)&=&R_{1}\left(\tilde{P}_{1}\left(z_{1}\right)\tilde{P}_{2}\left(z_{2}\right)\prod_{i=1}^{2}
\hat{P}_{i}\left(w_{i}\right)\right)F_{1}\left(\tilde{\theta}_{1}\left(\tilde{P}_{2}\left(z_{2}\right)\hat{P}_{1}\left(w_{1}\right)\hat{P}_{2}\left(w_{2}\right)\right),z_{2}\right)\hat{F}_{1}\left(w_{1},w_{2};\tau_{1}\right),
\end{eqnarray*}


\begin{eqnarray*}
\hat{F}_{1}\left(z_{1},z_{2},w_{1},w_{2}\right)&=&\hat{R}_{2}\left(\tilde{P}_{1}\left(z_{1}\right)\tilde{P}_{2}\left(z_{2}\right)\prod_{i=1}^{2}
\hat{P}_{i}\left(w_{i}\right)\right)F_{2}\left(z_{1},z_{2};\zeta_{2}\right)\hat{F}_{2}\left(w_{1},\hat{\theta}_{2}\left(\tilde{P}_{1}\left(z_{1}\right)\tilde{P}_{2}\left(z_{2}\right)\hat{P}_{1}\left(w_{1}
\right)\right)\right),
\end{eqnarray*}


\begin{eqnarray*}
\hat{F}_{2}\left(z_{1},z_{2},w_{1},w_{2}\right)&=&\hat{R}_{1}\left(\tilde{P}_{1}\left(z_{1}\right)\tilde{P}_{2}\left(z_{2}\right)\prod_{i=1}^{2}
\hat{P}_{i}\left(w_{i}\right)\right)F_{1}\left(z_{1},z_{2};\zeta_{1}\right)\hat{F}_{1}\left(\hat{\theta}_{1}\left(\tilde{P}_{1}\left(z_{1}\right)\tilde{P}_{2}\left(z_{2}\right)\hat{P}_{2}\left(w_{2}\right)\right),w_{2}\right),
\end{eqnarray*}



%______________________________________________________________________
\section{Preliminaries: }
%______________________________________________________________________

Consider a Network consisting in two cyclic polling systems with two queues each other, $Q_{1}, Q_{2}$ for the first system and $\hat{Q}_{1},\hat{Q}_{2}$ for the second one, each with infinite-sized buffer. In each system a single server visits the queues in cyclic order, where he applies the exhaustive policy, i.e., when the server polls a queue, he serves all the customers present until the queue becomes empty.


At the second system the customers at queue 2 moves to the first system's queue 2, we assume that the network is open; that is, all customers eventually leave the network. As usually in Polling Systems Theory we assume the arrivals in each queue the arrival processes are Poisson whit i.i.d. interarrival times, their service times are also i.i.d. and finally upon completion of a visit at any queue, the servers incurs in a random switchover time according to an arbitray distribution.  We define a cycle to be the time interval between two consecutive polling instants, the time period in a cycle during which the server is serving a queue is called a service period. The queues are attended in cyclic order.

Time is slotted with slot size equal to the service time of a fixed costumer, we call the time interval $\left[t,t+1\right]$ the $t$-th slot. The arrival processes are denoted by $X_{1}\left(t\right),X_{2}\left(t\right)$ for the first system and $\hat{X}_{1}\left(t\right)$ ,$\hat{X}_{2}\left(t\right)$ for the second, the arrival rate at $Q_{i}$ and $\hat{Q}_{i}$ is denoted by $\mu_{i}$ and $\hat{\mu}_{i}$ respectively, with the condition $\mu_{i}<1$ and $\hat{\mu}_{i}<1$. The users arrives in a independent form at each of the queues. We define the process $Y_{2}$ to consider the costumers who pass from system 2, to system 1, with arrival rate $\tilde{\mu}_{2}$. The service time customers of queue $i$ is a random variable $\tau_{i}$ with process defined by $S_{i}$. In similar manner the switchover period following the service of queue $i$ is an independent random variable $R_{i}$ with general distribution. To determine the length of the queues, i.e., the number of users in the queue at the moment the server arrives we define the process $L_{i}$ and $\hat{L}_{i}$ for the first and second system respectively. In the sequel, we use the buffer occupancy method to obtain the generating function, first and second moments of queue size distributions at polling instants. At each of the queues in the network the number of users is the number of users at the time the server arrives plus the numbers of arrivals during the service time. In order to obtain the joint probability generating function (PGF) for the number or users residing in queue $i$ when the queue is polled in the NCPS, we define for each of the arrival processes $X_{i},\hat{X}_{i}$, $i=1,2$,  $Y_{2}$ and $\tilde{X}_{2}$ with $\tilde{X}_{2}=X_{2}+Y_{2}$, their PGF $P_{i}\left(z_{i}\right)=\esp\left[z_{i}^{X_{i}\left(t\right)}\right],\hat{P}_{i}\left(w_{i}\right)=\esp\left[w_{i}^{\hat{X}_{i}\left(t\right)}\right]$, for $i=1,2$, and $\check{P}_{2}\left(z_{2}\right)=\esp\left[z_{2}^{Y_{2}\left(t\right)}\right], \tilde{P}_{2}\left(z_{2}\right)=\esp\left[z_{2}^{\tilde{X}_{2}\left(t\right)}\right]$ , with first moment given by $\mu_{i}=\esp\left[X_{i}\left(t\right)\right]=P_{i}^{(1)}\left(1\right), \hat{\mu}_{i}=\esp\left[\hat{X}_{i}\left(t\right)\right]=\hat{P}_{i}^{(1)}\left(1\right)$, for $i=1,2$, while $\check{\mu}_{2}=\esp\left[Y_{2}\left(t\right)\right]=\check{P}_{2}^{(1)}\left(1\right),\tilde{\mu}_{2}=\esp\left[\tilde{X}_{2}\left(t\right)\right]=\tilde{P}_{2}^{(1)}\left(1\right)$. The PGF For the service time is defined by: $S_{i}\left(z_{i}\right)=\esp\left[z_{i}^{\overline{\tau}_{i}-\tau_{i}}\right]$ y $\hat{S}_{i}\left(w_{i}\right)=\esp\left[w_{i}^{\overline{\zeta}_{i}-\zeta_{i}}\right]$, with first moment $s_{i}=\esp\left[\overline{\tau}_{i}-\tau_{i}\right]$ y $\hat{s}_{i}=\esp\left[\overline{\zeta}_{i}-\zeta_{i}\right]$, for $i=1,2$. In a similar manner the PGF for the switchover time of the server from the moment it ends to attend a queue to the time of arrival to the next queue are given by $R_{i}\left(z_{i}\right)=\esp\left[z_{1}^{\tau_{i+1}-\overline{\tau}_{i}}\right]$ and $\hat{R}_{i}\left(w_{i}\right)=\esp\left[w_{i}^{\zeta_{i+1}-\overline{\zeta}_{i}}\right]$ with first moment $r_{i}=R_{i}^{(1)}\left(1\right)=\esp\left[\tau_{i+1}-\overline{\tau}_{i}\right]$ and $\hat{r}_{i}=\hat{R}_{i}^{(1)}\left(1\right)=\esp\left[\zeta_{i+1}-\overline{\zeta}_{i}\right]$ with $i=1,2$. The number of users in the queue at time $\overline{\tau}_{1},\overline{\tau}_{2}, \overline{\zeta}_{1},\overline{\zeta}_{2}$, it's zero, i.e.,
 $L_{i}\left(\overline{\tau_{i}}\right)=0,$ and $\hat{L}_{i}\left(\overline{\zeta_{i}}\right)=0$ for $i=1,2$. Then the number of users in the queue of the second system at the moment the server ends attending in the queue is given by the number of users present at the moment it arrives plus the number of arrivals during the service time, i.e., $\hat{L}_{i}\left(\overline{\tau}_{j}\right)=\hat{L}_{i}\left(\tau_{j}\right)+\hat{X}_{i}\left(\overline{\tau}_{j}-\tau_{j}\right)$, for $i,j=1,2$, meanwhile for the first system : $L_{1}\left(\overline{\tau}_{j}\right)=L_{1}\left(\tau_{j}\right)+X_{1}\left(\overline{\tau}_{j}-\tau_{j}\right)$. Specifically for the second queue of the first system we need to consider the users of transfer becoming from the second queue in the second system while the server it's in the other queue attending, it means that this users have been aready attended by the server before they can go to the first system:

\begin{equation}\label{Eq.UsuariosTotalesZ2}
L_{2}\left(\overline{\tau}_{1}\right)=L_{2}\left(\tau_{1}\right)+X_{2}\left(\overline{\tau}_{1}-\tau_{1}\right)+Y_{2}\left(\overline{\tau}_{1}-\tau_{1}\right).
\end{equation}

%_________________________________________________________________________
%\subsection{Gambler's ruin problem}
%_________________________________________________________________________

As is know the gambler's ruin problem can be used to model the server's busy period in a Cyclic Polling System, so let $\tilde{L}_{0}\geq0$ the number of users present at the moment the server arrives to start serving, also let $T$ be the time the server need to attend the users in the queue starting with $\tilde{L}_{0}$ users. Suppose the gambler has two simultaneous, independent and simultaneous moves, such events are independent and identical to each other for each realization. The gain on the $n$-th game is $\tilde{X}_{n}=X_{n}+Y_{n}$ units from which is substracted a playing fee of 1 unit for each move. His PGF is given by $F\left(z\right)=\esp\left[z^{\tilde{L}_{0}}\right]$, futhermore
$$\tilde{P}\left(z\right)=\esp\left[z^{\tilde{X}_{n}}\right]=\esp\left[z^{X_{n}+Y_{n}}\right]=\esp\left[z^{X_{n}}z^{Y_{n}}\right]=\esp\left[z^{X_{n}}\right]\esp\left[z^{Y_{n}}\right]=P\left(z\right)\check{P}\left(z\right),$$

with $\tilde{\mu}=\esp\left[\tilde{X}_{n}\right]=\tilde{P}\left[z\right]<1$. If  $\tilde{L}_{n}$ denotes the capital remaining after the $n$-th game, then $$\tilde{L}_{n}=\tilde{L}_{0}+\tilde{X}_{1}+\tilde{X}_{2}+\cdots+\tilde{X}_{n}-2n.$$

The result that relates the gambler's ruin problem with the busy period of the serverit's a generalization of the result given in Takagi \cite{Takagi} chapter 3.


\textbf{Proposition} \ref{Prop.1.1.2Sa}
Let's $G_{n}\left(z\right)$ and $G\left(z,w\right)$ defined as in
(\ref{Eq.3.16.a.2SA}), then

\begin{eqnarray*}%\label{Eq.Pag.45}
G_{n}\left(z\right)=\frac{1}{z}\left[G_{n-1}\left(z\right)-G_{n-1}\left(0\right)\right]\tilde{P}\left(z\right).
\end{eqnarray*}

Futhermore

\begin{eqnarray*}%\label{Eq.Pag.46}
G\left(z,w\right)=\frac{zF\left(z\right)-wP\left(z\right)G\left(0,w\right)}{z-wR\left(z\right)},
\end{eqnarray*}

with a unique pole in the unit circle, also the pole is of the form $z=\theta\left(w\right)$ and satisfies

\begin{enumerate}
\item[i)]$\tilde{\theta}\left(1\right)=1$,

\item[ii)] $\tilde{\theta}^{(1)}\left(1\right)=\frac{1}{1-\tilde{\mu}}$,

\item[iii)]
$\tilde{\theta}^{(2)}\left(1\right)=\frac{\tilde{\mu}}{\left(1-\tilde{\mu}\right)^{2}}+\frac{\tilde{\sigma}}{\left(1-\tilde{\mu}\right)^{3}}$.
\end{enumerate}

Finally the following satisfies $\esp\left[w^{T}\right]=G\left(0,w\right)=F\left[\tilde{\theta}\left(w\right)\right].$
%\end{Prop}

\textbf{Corollary} \ref{Corolario1.A} The first and second moments for the gambler's ruin are

\begin{eqnarray*}
\begin{array}{ll}
\esp\left[T\right]=\frac{\esp\left[\tilde{L}_{0}\right]}{1-\tilde{\mu}},&
Var\left[T\right]=\frac{Var\left[\tilde{L}_{0}\right]}{\left(1-\tilde{\mu}\right)^{2}}+\frac{\sigma^{2}\esp\left[\tilde{L}_{0}\right]}{\left(1-\tilde{\mu}\right)^{3}}.
\end{array}
\end{eqnarray*}
%_____________________________________________________________________
%__________________________________________________________________________
%\subsection{Arrival Processes in the Queues for NCPS}
%__________________________________________________________________________

In order to model the network of cyclic polling system it's necessary to define the arrival processes for the queues belonging to the system that the server doesn't correspond. In the case of the first system and the server arrive to a queue in the second one:$F_{i,j}\left(z_{i};\zeta_{j}\right)=\esp\left[z_{i}^{L_{i}\left(\zeta_{j}\right)}\right]=
\sum_{k=0}^{\infty}\prob\left[L_{i}\left(\zeta_{j}\right)=k\right]z_{i}^{k}$for $i,j=1,2$. For the second system and the server arrives to a queue in the first system $\hat{F}_{i,j}\left(w_{i};\tau_{j}\right)=\esp\left[w_{i}^{\hat{L}_{i}\left(\tau_{j}\right)}\right] =\sum_{k=0}^{\infty}\prob\left[\hat{L}_{i}\left(\tau_{j}\right)=k\right]w_{i}^{k}$ for $i,j=1,2$. With the developed we can define the joint PGF for the second system:


\begin{eqnarray*}
\esp\left[w_{1}^{\hat{L}_{1}\left(\tau_{j}\right)}w_{2}^{\hat{L}_{2}\left(\tau_{j}\right)}\right]
&=&\esp\left[w_{1}^{\hat{L}_{1}\left(\tau_{j}\right)}\right]
\esp\left[w_{2}^{\hat{L}_{2}\left(\tau_{j}\right)}\right]=\hat{F}_{1,j}\left(w_{1};\tau_{j}\right)\hat{F}_{2,j}\left(w_{2};\tau_{j}\right)=\hat{F}_{j}\left(w_{1},w_{2};\tau_{j}\right).
\end{eqnarray*}

In a similar manner we defin the joint PGF for the first system, and the second system's server

\begin{eqnarray*}
\esp\left[z_{1}^{L_{1}\left(\zeta_{j}\right)}z_{2}^{L_{2}\left(\zeta_{j}\right)}\right]
&=&\esp\left[z_{1}^{L_{1}\left(\zeta_{j}\right)}\right]
\esp\left[z_{2}^{L_{2}\left(\zeta_{j}\right)}\right]=F_{1,j}\left(z_{1};\zeta_{j}\right)F_{2,j}\left(z_{2};\zeta_{j}\right)=F_{j}\left(z_{1},z_{2};\zeta_{j}\right).
\end{eqnarray*}

Now we proceed to determine the joint PGF for the times that the server visit each queue in each system, i.e., $t=\left\{\tau_{1},\tau_{2},\zeta_{1},\zeta_{2}\right\}$:

\begin{eqnarray}\label{Eq.Conjuntas}
\begin{array}{ll}
F_{j}\left(z_{1},z_{2},w_{1},w_{2}\right)=\esp\left[\prod_{i=1}^{2}z_{i}^{L_{i}\left(\tau_{j}
\right)}\prod_{i=1}^{2}w_{i}^{\hat{L}_{i}\left(\tau_{j}\right)}\right],&
\hat{F}_{j}\left(z_{1},z_{2},w_{1},w_{2}\right)=\esp\left[\prod_{i=1}^{2}z_{i}^{L_{i}
\left(\zeta_{j}\right)}\prod_{i=1}^{2}w_{i}^{\hat{L}_{i}\left(\zeta_{j}\right)}\right]
\end{array}
\end{eqnarray}
for $j=1,2$. Then with the purpose of find the number of users present in the netwotk when the server ends attending one of the queues in any of the systems

\begin{eqnarray*}
&&\esp\left[z_{1}^{L_{1}\left(\overline{\tau}_{1}\right)}z_{2}^{L_{2}\left(\overline{\tau}_{1}\right)}w_{1}^{\hat{L}_{1}\left(\overline{\tau}_{1}\right)}w_{2}^{\hat{L}_{2}\left(\overline{\tau}_{1}\right)}\right]
=\esp\left[z_{2}^{L_{2}\left(\overline{\tau}_{1}\right)}w_{1}^{\hat{L}_{1}\left(\overline{\tau}_{1}
\right)}w_{2}^{\hat{L}_{2}\left(\overline{\tau}_{1}\right)}\right]\\
&=&\esp\left[z_{2}^{L_{2}\left(\tau_{1}\right)+X_{2}\left(\overline{\tau}_{1}-\tau_{1}\right)+Y_{2}\left(\overline{\tau}_{1}-\tau_{1}\right)}w_{1}^{\hat{L}_{1}\left(\tau_{1}\right)+\hat{X}_{1}\left(\overline{\tau}_{1}-\tau_{1}\right)}w_{2}^{\hat{L}_{2}\left(\tau_{1}\right)+\hat{X}_{2}\left(\overline{\tau}_{1}-\tau_{1}\right)}\right]
\end{eqnarray*}

using the equation (\ref{Eq.UsuariosTotalesZ2}) we have


\begin{eqnarray*}
&=&\esp\left[z_{2}^{L_{2}\left(\tau_{1}\right)}z_{2}^{X_{2}\left(\overline{\tau}_{1}-\tau_{1}\right)}z_{2}^{Y_{2}\left(\overline{\tau}_{1}-\tau_{1}\right)}w_{1}^{\hat{L}_{1}\left(\tau_{1}\right)}w_{1}^{\hat{X}_{1}\left(\overline{\tau}_{1}-\tau_{1}\right)}w_{2}^{\hat{L}_{2}\left(\tau_{1}\right)}w_{2}^{\hat{X}_{2}\left(\overline{\tau}_{1}-\tau_{1}\right)}\right]\\
&=&\esp\left[z_{2}^{L_{2}\left(\tau_{1}\right)}\left\{w_{1}^{\hat{L}_{1}\left(\tau_{1}\right)}w_{2}^{\hat{L}_{2}\left(\tau_{1}\right)}\right\}\left\{z_{2}^{X_{2}\left(\overline{\tau}_{1}-\tau_{1}\right)}
z_{2}^{Y_{2}\left(\overline{\tau}_{1}-\tau_{1}\right)}w_{1}^{\hat{X}_{1}\left(\overline{\tau}_{1}-\tau_{1}\right)}w_{2}^{\hat{X}_{2}\left(\overline{\tau}_{1}-\tau_{1}\right)}\right\}\right]
\end{eqnarray*}

applying the fact that the arrivals processes in the queues in each systems are independent:

\begin{eqnarray*}
&=&\esp\left[z_{2}^{L_{2}\left(\tau_{1}\right)}\left\{z_{2}^{X_{2}\left(\overline{\tau}_{1}-\tau_{1}\right)}z_{2}^{Y_{2}\left(\overline{\tau}_{1}-\tau_{1}\right)}w_{1}^{\hat{X}_{1}\left(\overline{\tau}_{1}-\tau_{1}\right)}w_{2}^{\hat{X}_{2}\left(\overline{\tau}_{1}-\tau_{1}\right)}\right\}\right]\esp\left[w_{1}^{\hat{L}_{1}\left(\tau_{1}\right)}w_{2}^{\hat{L}_{2}\left(\tau_{1}\right)}\right]
\end{eqnarray*}

given that the arrival processes in the queues are independent, it's possible to separate the expectation for the arrival processes in $Q_{1}$ and $Q_{2}$ at time $\tau_{1}$, which is the time the server visits $Q_{1}$. Considering
$\tilde{X}_{2}\left(z_{2}\right)=X_{2}\left(z_{2}\right)+Y_{2}\left(z_{2}\right)$ we have


\begin{eqnarray*}
&=&\esp\left[z_{2}^{L_{2}\left(\tau_{1}\right)}\left\{z_{2}^{\tilde{X}_{2}\left(\overline{\tau}_{1}-\tau_{1}\right)}w_{1}^{\hat{X}_{1}\left(\overline{\tau}_{1}-\tau_{1}\right)}w_{2}^{\hat{X}_{2}\left(\overline{\tau}_{1}-\tau_{1}\right)}\right\}\right]\esp\left[w_{1}^{\hat{L}_{1}\left(\tau_{1}\right)}w_{2}^{\hat{L}_{2}\left(\tau_{1}\right)}\right]=\esp\left[z_{2}^{L_{2}\left(\tau_{1}\right)}\left\{\tilde{P}_{2}\left(z_{2}\right)^{\overline{\tau}_{1}-\tau_{1}}\hat{P}_{1}\left(w_{1}\right)^{\overline{\tau}_{1}-\tau_{1}}\right.\right.\\
&&\left.\left.\hat{P}_{2}\left(w_{2}\right)^{\overline{\tau}_{1}-\tau_{1}}\right\}\right]\esp\left[w_{1}^{\hat{L}_{1}\left(\tau_{1}\right)}w_{2}^{\hat{L}_{2}\left(\tau_{1}\right)}\right]
=\esp\left[z_{2}^{L_{2}\left(\tau_{1}\right)}\left\{\tilde{P}_{2}\left(z_{2}\right)\hat{P}_{1}\left(w_{1}\right)\hat{P}_{2}\left(w_{2}\right)\right\}^{\overline{\tau}_{1}-\tau_{1}}\right]\esp\left[w_{1}^{\hat{L}_{1}\left(\tau_{1}\right)}w_{2}^{\hat{L}_{2}\left(\tau_{1}\right)}\right]\\
&=&\esp\left[z_{2}^{L_{2}\left(\tau_{1}\right)}\theta_{1}\left(\tilde{P}_{2}\left(z_{2}\right)\hat{P}_{1}\left(w_{1}\right)\hat{P}_{2}\left(w_{2}\right)\right)^{L_{1}\left(\tau_{1}\right)}\right]\esp\left[w_{1}^{\hat{L}_{1}\left(\tau_{1}\right)}w_{2}^{\hat{L}_{2}\left(\tau_{1}\right)}\right]
=F_{1}\left(\theta_{1}\left(\tilde{P}_{2}\left(z_{2}\right)\hat{P}_{1}\left(w_{1}\right)\hat{P}_{2}\left(w_{2}\right)\right),z{2}\right)\\
&&\cdot\hat{F}_{1}\left(w_{1},w_{2};\tau_{1}\right)\equiv
F_{1}\left(\theta_{1}\left(\tilde{P}_{2}\left(z_{2}\right)\hat{P}_{1}\left(w_{1}\right)\hat{P}_{2}\left(w_{2}\right)\right),z_{2},w_{1},w_{2}\right).
\end{eqnarray*}

The last equalities  are true because the number of arrivals to $\hat{Q}_{2}$
during the time interval $\left[\tau_{1},\overline{\tau}_{1}\right]$ still haven't been attended by the server in the system 2, then the users can't pass to the first system through the queue $Q_{2}$. Therefore the number of users switching from $\hat{Q}_{2}$ to $Q_{2}$ during the time interval $\left[\tau_{1},\overline{\tau}_{1}\right]$ depends on the policy of transfer between the two systems, according to the last section
%{\small{
\begin{eqnarray*}\label{Eq.Fs}
\begin{array}{l}
\esp\left[z_{1}^{L_{1}\left(\overline{\tau}_{1}\right)}z_{2}^{L_{2}\left(\overline{\tau}_{1}
\right)}w_{1}^{\hat{L}_{1}\left(\overline{\tau}_{1}\right)}w_{2}^{\hat{L}_{2}\left(
\overline{\tau}_{1}\right)}\right]
=F_{1}\left(\theta_{1}\left(\tilde{P}_{2}\left(z_{2}\right)
\hat{P}_{1}\left(w_{1}\right)\hat{P}_{2}\left(w_{2}\right)\right),z_{2},w_{1},w_{2}\right)\\
=F_{1}\left(\theta_{1}\left(\tilde{P}_{2}\left(z_{2}\right)\hat{P}_{1}\left(w_{1}\right)\hat{P}_{2}\left(w_{2}\right)\right),z_{2}\right)\hat{F}_{1}\left(w_{1},w_{2};\tau_{1}\right)
\end{array}
\end{eqnarray*}%}}

Using reasoning similar for the rest of the server's arrival times we have that

\begin{eqnarray*}
\esp\left[z_{1}^{L_{1}\left(\overline{\tau}_{2}\right)}z_{2}^{L_{2}\left(\overline{\tau}_{2}\right)}w_{1}^{\hat{L}_{1}\left(\overline{\tau}_{2}\right)}w_{2}^{\hat{L}_{2}\left(\overline{\tau}_{2}\right)}\right]&=&F_{2}\left(z_{1},\tilde{\theta}_{2}\left(P_{1}\left(z_{1}\right)\hat{P}_{1}\left(w_{1}\right)\hat{P}_{2}\left(w_{2}\right)\right)\right)
\hat{F}_{2}\left(w_{1},w_{2};\tau_{2}\right)\\
\esp\left[z_{1}^{L_{1}\left(\overline{\zeta}_{1}\right)}z_{2}^{L_{2}\left(\overline{\zeta}_{1}
\right)}w_{1}^{\hat{L}_{1}\left(\overline{\zeta}_{1}\right)}w_{2}^{\hat{L}_{2}\left(
\overline{\zeta}_{1}\right)}\right]
&=&F_{1}\left(z_{1},z_{2};\zeta_{1}\right)\hat{F}_{1}\left(\hat{\theta}_{1}\left(P_{1}\left(z_{1}\right)\tilde{P}_{2}\left(z_{2}\right)\hat{P}_{2}\left(w_{2}\right)\right),w_{2}\right),\\
\esp\left[z_{1}^{L_{1}\left(\overline{\zeta}_{2}\right)}z_{2}^{L_{2}\left(\overline{\zeta}_{2}\right)}w_{1}^{\hat{L}_{1}\left(\overline{\zeta}_{2}\right)}w_{2}^{\hat{L}_{2}\left(\overline{\zeta}_{2}\right)}\right]
&=&F_{2}\left(z_{1},z_{2};\zeta_{2}\right)\hat{F}_{2}\left(w_{1},\hat{\theta}_{2}\left(P_{1}\left(z_{1}\right)\tilde{P}_{2}\left(z_{2}\right)\hat{P}_{1}\left(w_{1}
\right)\right)\right).
\end{eqnarray*}
%__________________________________________________________________________
%\subsection{Recursive equations for the NCPS}
%__________________________________________________________________________
Now we are in conditions to obtain the recursive equations that model the NCPS we need to consider the switchover times that the server ocuppies to translate from one queue to another and, the number or user presents in the system at the time the server leaves to queue to start attending the next. Thus far developed, we can find that for the NCPS:

\begin{eqnarray}\label{Recursive.Equations.First.Casse}
\begin{array}{l}
F_{2}\left(z_{1},z_{2},w_{1},w_{2}\right)=R_{1}\left(P_{1}\left(z_{1}\right)\tilde{P}_{2}\left(z_{2}\right)\prod_{i=1}^{2}
\hat{P}_{i}\left(w_{i}\right)\right)F_{1}\left(\theta_{1}\left(\tilde{P}_{2}\left(z_{2}\right)\hat{P}_{1}\left(w_{1}\right)\hat{P}_{2}\left(w_{2}\right)\right),z_{2}\right)\hat{F}_{1}\left(w_{1},w_{2};\tau_{1}\right),\\
F_{1}\left(z_{1},z_{2},w_{1},w_{2}\right)=R_{2}\left(P_{1}\left(z_{1}\right)\tilde{P}_{2}\left(z_{2}\right)\prod_{i=1}^{2}
\hat{P}_{i}\left(w_{i}\right)\right)F_{2}\left(z_{1},\tilde{\theta}_{2}\left(P_{1}\left(z_{1}\right)\hat{P}_{1}\left(w_{1}\right)\hat{P}_{2}\left(w_{2}\right)\right)\right)
\hat{F}_{2}\left(w_{1},w_{2};\tau_{2}\right),\\
\hat{F}_{2}\left(z_{1},z_{2},w_{1},w_{2}\right)=\hat{R}_{1}\left(P_{1}\left(z_{1}\right)\tilde{P}_{2}\left(z_{2}\right)\prod_{i=1}^{2}
\hat{P}_{i}\left(w_{i}\right)\right)F_{1}\left(z_{1},z_{2};\zeta_{1}\right)\hat{F}_{1}\left(\hat{\theta}_{1}\left(P_{1}\left(z_{1}\right)\tilde{P}_{2}\left(z_{2}\right)\hat{P}_{2}\left(w_{2}\right)\right),w_{2}\right),\\
\hat{F}_{1}\left(z_{1},z_{2},w_{1},w_{2}\right)=\hat{R}_{2}\left(P_{1}\left(z_{1}\right)\tilde{P}_{2}\left(z_{2}\right)\prod_{i=1}^{2}
\hat{P}_{i}\left(w_{i}\right)\right)F_{2}\left(z_{1},z_{2};\zeta_{2}\right)\hat{F}_{2}\left(w_{1},\hat{\theta}_{2}\left(P_{1}\left(z_{1}\right)\tilde{P}_{2}\left(z_{2}\right)\hat{P}_{1}\left(w_{1}
\right)\right)\right).
\end{array}
\end{eqnarray}


%______________________________________________________________________
\section{Main Result and An Example}
%______________________________________________________________________
%\begin{figure}[H]\caption{Network of Cyclic Polling System with double bidirectional transfer}
%\centering
%%%\includegraphics[width=9cm]{Grafica4.jpg}
%%\end{figure}\label{FigureRSVC3}


%_____________________________________________________
%\subsubsection{Server Switchover times}
%_____________________________________________________
It's necessary to give an step ahead, considering the case illustrated in \texttt{Figure 1}, where just like before, the server's switchover times are given by the generals equations
$R_{i}\left(\mathbf{z,w}\right)=R_{i}\left(\tilde{P}_{1}\left(z_{1}\right)
\tilde{P}_{2}\left(z_{2}\right)\tilde{P}_{3}\left(z_{3}\right)
\tilde{P}_{4}\left(z_{4}\right)\right)$, with first order derivatives given by $D_{i}R_{i}=r_{i}\tilde{\mu}_{i}$, and second order partial derivatives $D_{j}D_{i}R_{k}=R_{k}^{(2)}\tilde{\mu}_{i}\tilde{\mu}_{j}+\indora_{i=j}r_{k}P_{i}^{(2)}+\indora_{i\neq j}r_{k}\tilde{\mu}_{i}\tilde{\mu}_{j}$ for any $i,j,k$. According to the equations given before, the queue lengths for the other sytem's server times, we can obtain general expressions, so for
$F_{1}\left(z_{1},z_{2};\tau_{3}\right)$, $F_{2}\left(z_{1},z_{2};\tau_{4}\right)$, $F_{3}\left(z_{3},z_{4};\tau_{1}\right)$ and $F_{4}\left(z_{3},z_{4};\tau_{2}\right)$, we can obtain general expressions,

\begin{eqnarray}\label{Ec.Gral.Primer.Momento.Ind.Exh}
\begin{array}{ll}
D_{j}F_{i}\left(z_{1},z_{2};\tau_{i+2}\right)=\indora_{j\leq2}F_{j,i+2}^{(1)},&
D_{j}F_{i}\left(z_{3},z_{4};\tau_{i-2}\right)=\indora_{j\geq3}F_{j,i-2}^{(1)},
\end{array}
\end{eqnarray}

for $i=1,2,3,4$ and $j=1,2,3,4$. With second order derivatives given by

\begin{eqnarray}\label{Ec.Gral.Segundo.Momento.Ind.Exh}
\begin{array}{l}
D_{j}D_{i}F_{k}\left(z_{1},z_{2};\tau_{k+2}\right)=\indora_{i\geq3}\indora_{j=i}F_{i,k+2}^{(2)}+\indora_{i\geq 3}\indora_{j\neq i}F_{j,k-2}^{(1)}F_{i,k+2}^{(1)},\\
D_{j}D_{i}F_{k}\left(z_{3},z_{4};\tau_{k-2}\right)=\indora_{i\geq3}\indora_{j=i}F_{i,k-2}^{(2)}+\indora_{i\geq 3}\indora_{j\neq i}F_{j,k-2}^{(1)}F_{i,k-2}^{(1)}.
\end{array}
\end{eqnarray}


 According with the developed at the moment, we can get the recursive equations which are of the following form

\begin{eqnarray}\label{General.System.Double.Transfer}
\begin{array}{l}
F_{1}\left(z_{1},z_{2},z_{3},z_{4}\right)=R_{2}\left(\prod_{i=1}^{4}\tilde{P}_{i}\left(z_{i}\right)\right)F_{2}\left(z_{1},\tilde{\theta}_{2}\left(\tilde{P}_{1}\left(z_{1}\right)\tilde{P}_{3}\left(z_{3}\right)\tilde{P}_{4}\left(z_{4}\right)\right)\right)
F_{4}\left(z_{3},z_{4};\tau_{2}\right),\\
F_{2}\left(z_{1},z_{2},z_{3},z_{4}\right)=R_{1}\left(\prod_{i=1}^{4}\tilde{P}_{i}\left(z_{i}\right)\right)
F_{1}\left(\tilde{\theta}_{1}\left(\tilde{P}_{2}\left(z_{2}\right)\tilde{P}_{3}\left(z_{3}\right)\tilde{P}_{4}\left(z_{4}\right)\right),z_{2}\right)
F_{3}\left(z_{3},z_{4};\tau_{1}\right),\\
F_{3}\left(z_{1},z_{2},z_{3},z_{4}\right)=R_{4}\left(\prod_{i=1}^{4}\tilde{P}_{i}\left(z_{i}\right)\right)
F_{4}\left(z_{3},\tilde{\theta}_{4}\left(\tilde{P}_{1}\left(z_{1}\right)\tilde{P}_{2}\left(z_{2}\right)\tilde{P}_{3}\left(z_{3}\right)
\right)\right)
F_{2}\left(z_{1},z_{2};\tau_{4}\right),\\
F_{4}\left(z_{1},z_{2},z_{3},z_{4}\right)=R_{3}\left(\prod_{i=1}^{4}\tilde{P}_{i}\left(z_{i}\right)\right)
F_{3}\left(\tilde{\theta}_{3}\left(\tilde{P}_{1}\left(z_{1}\right)\tilde{P}_{2}\left(z_{2}\right)\tilde{P}_{4}\left(z_{4}
\right)\right),z_{4}\right)
F_{1}\left(z_{1},z_{2};\tau_{3}\right).
\end{array}
\end{eqnarray}

So we have the first theorem

\begin{Teo}
Suppose  $\tilde{\mu}=\tilde{\mu}_{1}+\tilde{\mu}_{2}<1$, $\hat{\mu}=\tilde{\mu}_{3}+\tilde{\mu}_{4}<1$, then the number of users en the queues conforming the network of cyclic polling system, (\ref{General.System.Double.Transfer}), when the server visit a queue can be found solving the linear system given by equations (\ref{Ec.Primer.Orden.General.Impar}) and (\ref{Ec.Primer.Orden.General.Par}),

\begin{eqnarray}\label{Ec.Primer.Orden.General.Impar}
\begin{array}{l}
f_{j}\left(i\right)=r_{j+1}\tilde{\mu}_{i}
+\indora_{i\neq j+1}f_{j+1}\left(j+1\right)\frac{\tilde{\mu}_{i}}{1-\tilde{\mu}_{j+1}}
+\indora_{i=j}f_{j+1}\left(i\right)
+\indora_{j=1}\indora_{i\geq3}F_{i,j+1}^{(1)}
+\indora_{j=3}\indora_{i\leq2}F_{i,j+1}^{(1)}
\end{array}
\end{eqnarray}
$j=1,3$ and $i=1,2,3,4$.

\begin{eqnarray}\label{Ec.Primer.Orden.General.Par}
\begin{array}{l}
f_{j}\left(i\right)=r_{j-1}\tilde{\mu}_{i}
+\indora_{i\neq j-1}f_{j-1}\left(j-1\right)\frac{\tilde{\mu}_{i}}{1-\tilde{\mu}_{j-1}}
+\indora_{i=j}f_{j-1}\left(i\right)
+\indora_{j=2}\indora_{i\geq3}F_{i,j-1}^{(1)}
+\indora_{j=4}\indora_{i\leq2}F_{i,j-1}^{(1)}
\end{array}
\end{eqnarray}
$j=2,4$ and $i=1,2,3,4$, whose solutions are:
%{\footnotesize{


\begin{eqnarray}
\begin{array}{l}
f_{i}\left(j\right)=\left(\indora_{j=i-1}+\indora_{j=i+1}\right)r_{j}\tilde{\mu}_{j}+\indora_{i=j}\left(\indora_{i\leq2}\frac{r\tilde{\mu}_{i}\left(1-\tilde{\mu}_{i}\right)}{1-\tilde{\mu}}+\indora_{i\geq2}\frac{\hat{r}\tilde{\mu}_{i}\left(1-\tilde{\mu}_{i}\right)}{1-\hat{\mu}}\right)
+\indora_{i=1}\indora_{j\geq3}\left(\tilde{\mu}_{j}\left(r_{i+1}+\frac{r\tilde{\mu}_{i+1}}{1-\tilde{\mu}}\right)\right.\\
\left.+F_{j,i+1}^{(1)}\right)
+\indora_{i=3}\indora_{j\geq3}\left(\tilde{\mu}_{j}\left(r_{i+1}+\frac{\hat{r}\tilde{\mu}_{i+1}}{1-\hat{\mu}}\right)+F_{j,i+1}^{(1)}\right)
+\indora_{i=2}\indora_{j\leq2}\left(\tilde{\mu}_{j}\left(r_{i-1}+\frac{r\tilde{\mu}_{i-1}}{1-\tilde{\mu}}\right)+F_{j,i-1}^{(1)}\right)\\
+\indora_{i=4}\indora_{j\leq2}\left(\tilde{\mu}_{j}\left(r_{i-1}+\frac{\hat{r}\tilde{\mu}_{i-1}}{1-\hat{\mu}}\right)+F_{j,i-1}^{(1)}\right)
\end{array}
\end{eqnarray}


%\begin{eqnarray}
%\begin{array}{lll}
%f_{1}\left(1\right)=r\frac{\tilde{\mu}_{1}\left(1-\tilde{\mu}_{1}\right)}{1-\tilde{\mu}},&
%f_{1}\left(2\right)=r_{2}\tilde{\mu}_{2},&
%f_{1}\left(3\right)=\tilde{\mu}_{3}\left(r_{2}+\frac{r\tilde{\mu}_{2}}{1-\tilde{\mu}}\right)+F_{3,2}^{(1)}\left(1\right),\\
%f_{1}\left(4\right)=\tilde{\mu}_{4}\left(r_{2}+\frac{r\tilde{\mu}_{2}}{1-\tilde{\mu}}\right)+F_{4,2}^{(1)}\left(1\right),&
%f_{2}\left(1\right)=r_{1}\tilde{\mu}_{1},&
%f_{2}\left(2\right)=r\frac{\tilde{\mu}_{2}\left(1-\tilde{\mu}_{2}\right)}{1-\tilde{\mu}},\\
%f_{2}\left(3\right)=\tilde{\mu}_{3}\left(r_{1}+\frac{r\tilde{\mu}_{1}}{1-\tilde{\mu}}\right)+F_{3,1}^{(1)}\left(1\right),&
%f_{2}\left(4\right)=\tilde{\mu}_{4}\left(r_{1}+\frac{r\tilde{\mu}_{1}}{1-\tilde{\mu}}\right)+F_{4,1}^{(1)}\left(1\right),&
%f_{3}\left(1\right)=\tilde{\mu}_{1}\left(r_{4}+\frac{\hat{r}\tilde{\mu}_{4}}{1-\hat{\mu}}\right)+F_{1,4}^{(1)}\left(1\right),\\
%f_{3}\left(2\right)=\tilde{\mu}_{2}\left(r_{4}+\frac{\hat{r}\tilde{\mu}_{4}}{1-\hat{\mu}}\right)+F_{2,4}^{(1)}\left(1\right),&
%f_{3}\left(3\right)=\hat{r}\frac{\tilde{\mu}_{3}\left(1-\tilde{\mu}_{3}\right)}{1-\hat{\mu}},&
%f_{3}\left(4\right)=r_{4}\tilde{\mu}_{4},\\
%f_{4}\left(1\right)=\tilde{\mu}_{1}\left(r_{3}+\frac{\hat{r}\tilde{\mu}_{3}}{1-\hat{\mu}}\right)+F_{1,3}^{(1)}\left(1\right),&
%f_{4}\left(2\right)=\tilde{\mu}_{2}\left(r_{3}+\frac{\hat{r}\tilde{\mu}_{3}}{1-\hat{\mu}}\right)+F_{2,3}^{(1)}\left(1\right),&
%f_{4}\left(3\right)=r_{3}\tilde{\mu}_{3},\\
%&f_{4}\left(4\right)=\hat{r}\frac{\tilde{\mu}_{4}\left(1-\tilde{\mu}_{4}\right)}{1-\hat{\mu}}&
%\end{array}
%\end{eqnarray}
\end{Teo}
%______________________________________________________________________

\begin{Teo}
For the system given by (\ref{General.System.Double.Transfer}) we have that the second moments are in their general form

{\small{
\begin{eqnarray}\label{Eq.Gral.Second.Order.Exhaustive}
\begin{array}{l}
f_{1}\left(i,k\right)=D_{k}D_{i}\left(R_{2}+F_{2}+\indora_{i\geq3}F_{4}\right)
+D_{i}R_{2}D_{k}\left(F_{2}+\indora_{k\geq3}F_{4}\right)
+D_{i}F_{2}D_{k}\left(R_{2}+\indora_{k\geq3}F_{4}\right)
+\indora_{i\geq3}D_{i}F_{4}D_{k}\left(R_{2}+F_{2}\right)\\
f_{2}\left(i,k\right)=D_{k}D_{i}\left(R_{1}+F_{1}+\indora_{i\geq3}F_{3}\right)+D_{i}R_{1}D_{k}\left(F_{1}+\indora_{k\geq3}F_{3}\right)+D_{i}F_{1}D_{k}\left(R_{1}+\indora_{k\geq3}F_{3}\right)
+\indora_{i\geq3}D_{i}\tilde{F}_{3}D_{k}\left(R_{1}+F_{1}\right)\\
f_{3}\left(i,k\right)=D_{k}D_{i}\left(\tilde{R}_{4}+\indora_{i\leq2}F_{2}+F_{4}\right)+D_{i}\tilde{R}_{4}D_{k}\left(\indora_{k\leq2}F_{2}+F_{4}\right)+D_{i}F_{4}D_{k}\left(R_{4}+\indora_{k\leq2}F_{2}\right)
+\indora_{i\leq2}D_{i}F_{2}D_{k}\left(R_{4}+F_{4}\right)\\
f_{4}\left(i,k\right)=D_{k}D_{i}\left(R_{3}+\indora_{i\leq2}F_{1}+F_{3}\right)+D_{i}R_{3}D_{k}\left(\indora_{k\leq2}F_{1}+F_{3}\right)+D_{i}F_{3}D_{k}\left(R_{3}+\indora_{k\leq2}F_{1}\right)
+\indora_{i\leq2}D_{i}F_{1}D_{k}\left(R_{3}+F_{3}\right)
\end{array}
\end{eqnarray}}}

\end{Teo}


\begin{Coro}
Conforming the equations given in (\ref{Eq.Gral.Second.Order.Exhaustive}) the second order moments are obtained solving the linear systems given by  (\ref{System.Second.Order.Moments.uno}) and (\ref{System.Second.Order.Moments.dos}). These solutions are


\begin{eqnarray}\label{Sol.System.Second.Order.Exhaustive}
\begin{array}{lll}
f_{1}\left(1,1\right)=b_{3},&
f_{2}\left(2,2\right)=\frac{b_{2}}{1-b_{1}},&
f_{1}\left(1,3\right)=a_{4}\left(\frac{b_{2}}{1-b_{1}}\right)+a_{5}K_{12}+K_{3},\\
f_{1}\left(1,4\right)=a_{6}\left(\frac{b_{2}}{1-b_{1}}\right)+a_{7}K_{12}+K_{4},&
f_{1}\left(3,3\right)=a_{8}\left(\frac{b_{2}}{1-b_{1}}\right)+K_{8},&
f_{1}\left(3,4\right)=a_{9}\left(\frac{b_{2}}{1-b_{1}}\right)+K_{9}\\
f_{1}\left(4,4\right)=a_{10}\left(\frac{b_{2}}{1-b_{1}}\right)+a_{5}K_{12}+K_{10},&
f_{2}\left(2,3\right)=a_{14}b_{3}+a_{15}K_{2}+K_{16},&
f_{2}\left(2,4\right)=a_{16}b_{3}+a_{17}K_{2}+K_{17},\\
f_{2}\left(3,3\right)=a_{18}b_{3}+K_{18},&
f_{2}\left(3,4\right)=a_{19}b_{3}+K_{19},&
f_{2}\left(4,4\right)=a_{20}b_{3}+K_{20}\\
f_{3}\left(3,3\right)=\frac{b_{5}}{1-b_{4}},&
f_{4}\left(2,2\right)=b_{6},&
f_{3}\left(1,1\right)=a_{21}b_{6}+K_{21},\\
f_{3}\left(1,2\right)=a_{22}b_{6}+K_{22},&
f_{3}\left(1,3\right)=a_{23}b_{6}+a_{24}K_{39}+K_{23},&
f_{3}\left(2,2\right)=a_{25}b_{6}+K_{25}\\
f_{3}\left(2,3\right)=a_{26}b_{6}+a_{27}K_{39}+K_{26},&
f_{4}\left(1,1\right)=a_{31}\left(\frac{b_{5}}{1-b_{4}}\right)+K_{31},&
f_{4}\left(1,2\right)=a_{32}\left(\frac{b_{5}}{1-b_{4}}\right)+K_{32},\\
f_{4}\left(1,4\right)=a_{33}\left(\frac{b_{5}}{1-b_{4}}\right)+a_{34}K_{29}+K_{31},&
f_{4}\left(2,2\right)=a_{35}\left(\frac{b_{5}}{1-b_{4}}\right)+K_{35},&
f_{4}\left(2,4\right)=a_{36}\left(\frac{b_{5}}{1-b_{4}}\right)+a_{37}K_{29}+K_{37}
\end{array}
\end{eqnarray}


where
\begin{eqnarray*}
\begin{array}{lll}
N_{1}=a_{2}K_{12}+a_{3}K_{11}+K_{1},&
N_{2}=a_{12}K_{2}+a_{13}K_{5}+K_{15},&
b_{1}=a_{1}a_{11}\\
b_{2}=a_{11}N_{1}+N_{2},&
b_{3}=a_{1}\left(\frac{b_{2}}{1-b_{1}}\right)+N_{1},&
N_{3}=a_{29}K_{39}+a_{30}K_{38}+K_{28}\\
N_{4}=a_{39}K_{29}+a_{40}K_{30}+K_{40},&
b_{4}=a_{28}a_{38},&
b_{5}=a_{28}N_{4}+N_{3}\\
&b_{6}=a_{38}\left(\frac{b_{5}}{1-b_{4}}\right)+N_{4}&
\end{array}
\end{eqnarray*}

\end{Coro}

the values for the $a_{i}$'s and $K_{i}$ can be found in \textit{Appendix B} in equations (\ref{Coefficients.Ais.Exh}, \ref{Coefficients.kis.Exh.uno}, \ref{Coefficients.kis.Exh.dos}, \ref{Coefficients.kis.Exh.tres}) and (\ref{Coefficients.kis.Exh.cuatro}).


%______________________________________________________________________
\section{Concluding Remarks}
%______________________________________________________________________

Using a similar reasoning it's possible to find de first and second moments for the queue lengths of the CPSN. We have the following theorem

\begin{Teo}
Given a CPSN attended by a single server who attends conforming to the gated policy and suppose  $\tilde{\mu}=\tilde{\mu}_{1}+\tilde{\mu}_{2}<1$, $\hat{\mu}=\tilde{\mu}_{3}+\tilde{\mu}_{4}<1$, then the number of users en the queues conforming the network of cyclic polling system, when the server visit a queue can be found solving the linear system given by equations (\ref{Ec.Primer.Orden.General.Impar.Gated}) and (\ref{Ec.Primer.Orden.General.Par.Gated}),

\begin{eqnarray}\label{Ec.Primer.Orden.General.Impar.Gated}
\begin{array}{l}
f_{j}\left(i\right)=r_{j+1}\tilde{\mu}_{i}
+f_{j+1}\left(j+1\right)\tilde{\mu}_{i}
+\indora_{i=j}f_{j+1}\left(i\right)
+\indora_{j=1}\indora_{i\geq3}F_{i,j+1}^{(1)}
+\indora_{j=3}\indora_{i\leq2}F_{i,j+1}^{(1)}
\end{array}
\end{eqnarray}
for $j=1,3$ and $i=1,2,3,4$.

\begin{eqnarray}\label{Ec.Primer.Orden.General.Par.Gated}
\begin{array}{l}
f_{j}\left(i\right)=r_{j-1}\tilde{\mu}_{i}
+f_{j-1}\left(j-1\right)\tilde{\mu}_{i}
+\indora_{i=j}f_{j-1}\left(i\right)
+\indora_{j=2}\indora_{i\geq3}F_{i,j-1}^{(1)}
+\indora_{j=4}\indora_{i\leq2}F_{i,j-1}^{(1)}
\end{array}
\end{eqnarray}
for $j=2,4$ and $i=1,2,3,4$, whose solutions are of he form

%\begin{eqnarray}
%f_{i}\left(j\right)=\indora_{i=1,3}\left(\tilde{\mu}_{j}\frac{r_{i+1}\left(1-\tilde{\mu}_{i}\right)+r_{i}\tilde{\mu}_{i+1}}{1-\indora_{i=1}\tilde{\mu}-\indora_{i=3}\hat{\mu}}+\indora_{i=1}\indora_{j\geq3}F_{j,i+1}^{(1)}
%+\indora_{i=3}\indora_{j\leq2}F_{j,i+1}^{(1)}\right)\\
%f_{i}\left(j\right)=\indora_{i=2,4}\left(\tilde{\mu}_{j}\frac{r_{i-1}\left(1-\tilde{\mu}_{i}\right)+r_{i}\tilde{\mu}_{i-1}}{1-\indora_{i=2}\tilde{\mu}-\indora_{i=4}\hat{\mu}}+\indora_{i=2}\indora_{j\geq3}F_{j,i-1}^{(1)}
%+\indora_{i=4}\indora_{j\leq2}F_{j,i-1}^{(1)}\right)
%\end{eqnarray}



\begin{eqnarray}\label{Sol.Sist.Ec.Lineales.Gated}
\begin{array}{l}
f_{i}\left(j\right)=\indora_{i=j}\left(\indora_{i\leq2}\frac{r\tilde{\mu}_{i}\left(1-\tilde{\mu}_{i}\right)}{1-\tilde{\mu}}+\indora_{i\geq3}\frac{\hat{r}\tilde{\mu}_{i}\left(1-\tilde{\mu}_{i}\right)}{1-\hat{\mu}}\right)
+\indora_{i=1,3}\left(\tilde{\mu}_{j}\frac{r_{i+1}\left(1-\tilde{\mu}_{i}\right)+r_{i}\tilde{\mu}_{i+1}}{1-\indora_{i=1}\tilde{\mu}-\indora_{i=3}\hat{\mu}}+\indora_{i=1}\indora_{j\geq3}F_{j,i+1}^{(1)}\right.\\
\left.+\indora_{i=3}\indora_{j\leq2}F_{j,i+1}^{(1)}\right)
+\indora_{i=2,4}\left(\tilde{\mu}_{j}\frac{r_{i-1}\left(1-\tilde{\mu}_{i}\right)+r_{i}\tilde{\mu}_{i-1}}{1-\indora_{i=2}\tilde{\mu}-\indora_{i=4}\hat{\mu}}+\indora_{i=2}\indora_{j\geq3}F_{j,i-1}^{(1)}
+\indora_{i=4}\indora_{j\leq2}F_{j,i-1}^{(1)}\right).
\end{array}
\end{eqnarray}


%+\indora_{j\leq2}+\indora_{i=1}\indora_{j\geq3}+\indora_{i=1}\indora_{j\leq2}\right)\\
%+\indora_{i=2,4}\left(\indora_{j\geq3}+\indora_{j\leq2}+\indora_{i=1}\indora_{j\geq3}+\indora_{i=1}\indora_{j\leq2}\right)

%\begin{eqnarray}\label{Sol.Sist.Ec.Lineales.Gated}
%\begin{array}{lll}
%5f_{1}\left(1\right)=\frac{r\tilde{\mu}_{1}}{1-\tilde{\mu}},&
%f_{1}\left(2\right)=\tilde{\mu}_{2}\frac{r_{2}\left(1-\tilde{\mu}_{1}\right)+r_{1}\tilde{\mu}_{2}}{1-\tilde{\mu}},&
%f_{1}\left(3\right)=\tilde{\mu}_{3}\frac{r_{2}\left(1-\tilde{\mu}_{1}\right)+r_{1}\tilde{\mu}_{2}}{1-\tilde{\mu}}+F_{3,2}^{(1)},\\
%f_{1}\left(4\right)=\tilde{\mu}_{4}\frac{r_{2}\left(1-\tilde{\mu}_{1}\right)+r_{1}\tilde{\mu}_{2}}{1-\tilde{\mu}}+F_{4,2}^{(1)},&
%f_{2}\left(1\right)=\tilde{\mu}_{1}\frac{r_{1}\left(1-\tilde{\mu}_{2}\right)+r_{2}\tilde{\mu}_{1}}{1-\tilde{\mu}},&
%f_{2}\left(2\right)=r\frac{\tilde{\mu}_{2}}{1-\tilde{\mu}},\\
%f_{2}\left(3\right)=\tilde{\mu}_{3}\frac{r_{1}\left(1-\tilde{\mu}_{2}\right)+r_{2}\tilde{\mu}_{1}}{1-\tilde{\mu}}+F_{3,1}^{(1)},&
%f_{2}\left(4\right)=\tilde{\mu}_{4}\frac{r_{1}\left(1-\tilde{\mu}_{2}\right)+r_{2}\tilde{\mu}_{1}}{1-\tilde{\mu}}+F_{4,1}^{(1)},&
%f_{3}\left(1\right)=\tilde{\mu}_{1}\frac{r_{4}\left(1-\tilde{\mu}_{3}\right)+r_{3}\tilde{\mu}_{4}}{1-\hat{\mu}}+F_{1,4}^{(1)},\\
%f_{3}\left(2\right)=\tilde{\mu}_{2}\frac{r_{4}\left(1-\tilde{\mu}_{3}\right)+r_{3}\tilde{\mu}_{4}}{1-\hat{\mu}}+F_{2,4}^{(1)},&
%f_{3}\left(3\right)=\frac{\hat{r}\tilde{\mu}_{3}}{1-\hat{\mu}},&
%f_{3}\left(4\right)=\tilde{\mu}_{4}\frac{r_{4}\left(1-\tilde{\mu}_{3}\right)+r_{3}\tilde{\mu}_{4}}{1-\hat{\mu}},\\
%f_{4}\left(1\right)=\tilde{\mu}_{1}\frac{r_{3}\left(1-\tilde{\mu}_{4}\right)+r_{4}\tilde{\mu}_{3}}{1-\hat{\mu}}+F_{1,3}^{(1)},&
%f_{4}\left(2\right)=\tilde{\mu}_{2}\frac{r_{3}\left(1-\tilde{\mu}_{4}\right)+r_{4}\tilde{\mu}_{3}}{1-\hat{\mu}}+F_{2,3}^{(1)},&
%f_{4}\left(3\right)=\tilde{\mu}_{3}\frac{r_{3}\left(1-\tilde{\mu}_{4}\right)+r_{4}\tilde{\mu}_{3}}{1-\hat{\mu}},\\
%&f_{4}\left(4\right)=\frac{\hat{r}\tilde{\mu}_{4}}{1-\hat{\mu}}.&
%\end{array}
%\end{eqnarray}
\end{Teo}

The second order moments can be obtained by direct operations according to theorem (\ref{Eq.Gral.Second.Order.Exhaustive})



\begin{Coro}
Conforming the equations given in (\ref{Eq.Sdo.Orden.Gated}) the second order moments are obtained solving the system
\end{Coro}





%______________________________________________________________________
\section{Appendix A: Gambler's ruin problem Proof}
%_______________________________________________________________________
Let's define the probability of the event no ruin before the $n$-th period begining with $\tilde{L}_{0}$ users, $g_{n,k}$ considering a capital equal to $k$ units after $n-1$ events, i.e.,  given $n\in\left\{1,2,\ldots\right\}$ y $k\in\left\{0,1,2,\ldots\right\}$ $g_{n,k}:=P\left\{\tilde{L}_{j}>0, j=1,\ldots,n,\tilde{L}_{n}=k\right\}$, which can be written as:

\begin{eqnarray*}
g_{n,k}&=&P\left\{\tilde{L}_{j}>0, j=1,\ldots,n,
\tilde{L}_{n}=k\right\}=\sum_{j=1}^{k+1}g_{n-1,j}P\left\{\tilde{X}_{n}=k-j+1\right\}\\
&=&\sum_{j=1}^{k+1}g_{n-1,j}P\left\{X_{n}+Y_{n}=k-j+1\right\}
=\sum_{j=1}^{k+1}\sum_{l=1}^{j}g_{n-1,j}P\left\{X_{n}+Y_{n}=k-j+1,Y_{n}=l\right\}\\
&=&\sum_{j=1}^{k+1}\sum_{l=1}^{j}g_{n-1,j}P\left\{X_{n}+Y_{n}=k-j+1|Y_{n}=l\right\}P\left\{Y_{n}=l\right\}\\
&=&\sum_{j=1}^{k+1}\sum_{l=1}^{j}g_{n-1,j}P\left\{X_{n}=k-j-l+1\right\}P\left\{Y_{n}=l\right\}\\
\end{eqnarray*}

so we have the following
\begin{eqnarray}\label{Eq.Gnk.2SA}
g_{n,k}=\sum_{j=1}^{k+1}\sum_{l=1}^{j}g_{n-1,j}P\left\{X_{n}=k-j-l+1\right\}P\left\{Y_{n}=l\right\}.
\end{eqnarray}
so

\begin{eqnarray}\label{Eq.3.16.a.2SA}
\begin{array}{ll}
G_{n}\left(z\right)=\sum_{k=0}^{\infty}g_{n,k}z^{k},\textrm{ para
}n=0,1,\ldots,\textrm{ and }&
G\left(z,w\right)=\sum_{n=0}^{\infty}G_{n}\left(z\right)w^{n}.
\end{array}
\end{eqnarray}

where we have that
\begin{equation}\label{Eq.L02SA}
g_{0,k}=P\left\{\tilde{L}_{0}=k\right\}.
\end{equation}

In particular for $k=0$, $g_{n,0}=G_{n}\left(0\right)=P\left\{\tilde{L}_{j}>0,\tilde{L}_{n}=0\right\}=P\left\{T=n\right\}$, for $j<n$. Futhermore $
G\left(0,w\right)=\sum_{n=0}^{\infty}G_{n}\left(0\right)w^{n}=\sum_{n=0}^{\infty}P\left\{T=n\right\}w^{n}
=\esp\left[w^{T}\right]$ which becomes the PGF for the ruin time $T$. The gambler's ruin occurs after the $n$-th game, i.e., the queue empty after $n$ steps starting with $\tilde{L}_{0}$ users.


\begin{Prop}\label{Prop.1.1.2Sa}
Let's $G_{n}\left(z\right)$ and $G\left(z,w\right)$ defined as in
(\ref{Eq.3.16.a.2SA}), then $G_{n}\left(z\right)=\frac{1}{z}\left[G_{n-1}\left(z\right)-G_{n-1}\left(0\right)\right]\tilde{P}\left(z\right)$. Futhermore $G\left(z,w\right)=\frac{zF\left(z\right)-wP\left(z\right)G\left(0,w\right)}{z-wR\left(z\right)}$, with a unique pole in the unit circle, also the pole is of the form $z=\theta\left(w\right)$ and satisfies

\begin{enumerate}
\item[i)]$\tilde{\theta}\left(1\right)=1$,
\item[ii)] $\tilde{\theta}^{(1)}\left(1\right)=\frac{1}{1-\tilde{\mu}}$,
\item[iii)]
$\tilde{\theta}^{(2)}\left(1\right)=\frac{\tilde{\mu}}{\left(1-\tilde{\mu}\right)^{2}}+\frac{\tilde{\sigma}}{\left(1-\tilde{\mu}\right)^{3}}$.
\end{enumerate}

Finally the following satisfies $\esp\left[w^{T}\right]=G\left(0,w\right)=F\left[\tilde{\theta}\left(w\right)\right].$
\end{Prop}
\begin{proof}
Multiplying equations (\ref{Eq.Gnk.2SA}) and (\ref{Eq.L02SA})
by $z^{k}$ we have that $g_{n,k}z^{k}=\sum_{j=1}^{k+1}\sum_{l=1}^{j}g_{n-1,j}P\left\{X_{n}=k-j-l+1\right\}P\left\{Y_{n}=l\right\}z^{k}$, $g_{0,k}z^{k}=P\left\{\tilde{L}_{0}=k\right\}z^{k}$, summing over $k$

\begin{eqnarray*}
\sum_{k=0}^{\infty}g_{n,k}z^{k}&=&\sum_{k=0}^{\infty}\sum_{j=1}^{k+1}\sum_{l=1}^{j}g_{n-1,j}P\left\{X_{n}=k-j-l+1\right\}P\left\{Y_{n}=l\right\}z^{k}\\
&=&\sum_{k=0}^{\infty}z^{k}\sum_{j=1}^{k+1}\sum_{l=1}^{j}g_{n-1,j}P\left\{X_{n}=k-\left(j+l
-1\right)\right\}P\left\{Y_{n}=l\right\}\\
&=&\sum_{k=0}^{\infty}z^{k+\left(j+l-1\right)-\left(j+l-1\right)}\sum_{j=1}^{k+1}\sum_{l=1}^{j}g_{n-1,j}P\left\{X_{n}=k-
\left(j+l-1\right)\right\}P\left\{Y_{n}=l\right\}\\
&=&\sum_{k=0}^{\infty}\sum_{j=1}^{k+1}\sum_{l=1}^{j}g_{n-1,j}z^{j-1}P\left\{X_{n}=k-
\left(j+l-1\right)\right\}z^{k-\left(j+l-1\right)}P\left\{Y_{n}=l\right\}z^{l}\\
&=&\sum_{j=1}^{\infty}\sum_{l=1}^{j}g_{n-1,j}z^{j-1}\sum_{k=j+l-1}^{\infty}P\left\{X_{n}=k-
\left(j+l-1\right)\right\}z^{k-\left(j+l-1\right)}P\left\{Y_{n}=l\right\}z^{l}
\end{eqnarray*}
\begin{eqnarray*}
&=&\sum_{j=1}^{\infty}g_{n-1,j}z^{j-1}\sum_{l=1}^{j}\sum_{k=j+l-1}^{\infty}P\left\{X_{n}=k-
\left(j+l-1\right)\right\}z^{k-\left(j+l-1\right)}P\left\{Y_{n}=l\right\}z^{l}\\
&=&\sum_{j=1}^{\infty}g_{n-1,j}z^{j-1}\sum_{k=j+l-1}^{\infty}\sum_{l=1}^{j}P\left\{X_{n}=k-
\left(j+l-1\right)\right\}z^{k-\left(j+l-1\right)}P\left\{Y_{n}=l\right\}z^{l}\\
&=&\sum_{j=1}^{\infty}g_{n-1,j}z^{j-1}\sum_{k=j+l-1}^{\infty}\sum_{l=1}^{j}P\left\{X_{n}=k-
\left(j+l-1\right)\right\}z^{k-\left(j+l-1\right)}\sum_{l=1}^{j}P
\left\{Y_{n}=l\right\}z^{l}\\
&=&\sum_{j=1}^{\infty}g_{n-1,j}z^{j-1}\sum_{l=1}^{\infty}P\left\{Y_{n}=l\right\}z^{l}
\sum_{k=j+l-1}^{\infty}\sum_{l=1}^{j}
P\left\{X_{n}=k-\left(j+l-1\right)\right\}z^{k-\left(j+l-1\right)}\\
&=&\frac{1}{z}\left[G_{n-1}\left(z\right)-G_{n-1}\left(0\right)\right]\tilde{P}\left(z\right)
\sum_{k=j+l-1}^{\infty}\sum_{l=1}^{j}
P\left\{X_{n}=k-\left(j+l-1\right)\right\}z^{k-\left(j+l-1\right)}\\
&=&\frac{1}{z}\left[G_{n-1}\left(z\right)-G_{n-1}\left(0\right)\right]\tilde{P}\left(z\right)P\left(z\right)=\frac{1}{z}\left[G_{n-1}\left(z\right)-G_{n-1}\left(0\right)\right]\tilde{P}\left(z\right),
\end{eqnarray*}

so (\ref{Eq.3.16.a.2SA})  can be rewritten as
\begin{equation}\label{Eq.3.16.a.2Sbis}
G_{n}\left(z\right)=\frac{1}{z}\left[G_{n-1}\left(z\right)-G_{n-1}\left(0\right)\right]\tilde{P}\left(z\right).
\end{equation}

then $\frac{G_{n}\left(z\right)}{z}=\sum_{k=1}^{\infty}g_{n,k}z^{k-1}$, therefore using (\ref{Eq.3.16.a.2Sbis}):

\begin{eqnarray*}
G\left(z,w\right)&=&\sum_{n=0}^{\infty}G_{n}\left(z\right)w^{n}=G_{0}\left(z\right)+
\sum_{n=1}^{\infty}G_{n}\left(z\right)w^{n}
=F\left(z\right)+\sum_{n=0}^{\infty}\left[G_{n}\left(z\right)-G_{n}\left(0\right)\right]w^{n}\frac{\tilde{P}\left(z\right)}{z}\\
&=&F\left(z\right)+\frac{w}{z}\sum_{n=0}^{\infty}\left[G_{n}\left(z\right)-G_{n}\left(0\right)\right]w^{n-1}\tilde{P}\left(z\right),
\end{eqnarray*}

it means that $G\left(z,w\right)=F\left(z\right)+\frac{w}{z}\left[G\left(z,w\right)-G\left(0,w\right)\right]\tilde{P}\left(z\right)$,
then $G\left(z,w\right)=F\left(z\right)+\frac{w}{z}\left[G\left(z,w\right)-G\left(0,w\right)\right]\tilde{P}\left(z\right)
=F\left(z\right)+\frac{w}{z}\tilde{P}\left(z\right)G\left(z,w\right)-\frac{w}{z}\tilde{P}\left(z\right)G\left(0,w\right)$
which is equivalent to
$G\left(z,w\right)\left\{1-\frac{w}{z}\tilde{P}\left(z\right)\right\}=F\left(z\right)-\frac{w}{z}\tilde{P}\left(z\right)G\left(0,w\right)$, therfore, $G\left(z,w\right)=\frac{zF\left(z\right)-w\tilde{P}\left(z\right)G\left(0,w\right)}{1-w\tilde{P}\left(z\right)}$. $G\left(z,w\right)$ is analytic in $|z|=1$, let's $z,w$ such that $|z|=1$ and $|w|\leq1$, given that $\tilde{P}\left(z\right)$is a PGF $|z-\left(z-w\tilde{P}\left(z\right)\right)|<|z|\Leftrightarrow|w\tilde{P}\left(z\right)|<|z|$, it means that Rouche's Theorem conditios are satisfied, the $z$ and  $z-w\tilde{P}\left(z\right)$ has the same number of zeros in $|z|=1$. Let $z=\tilde{\theta}\left(w\right)$ be the unique solution of
$z-w\tilde{P}\left(z\right)$, it means

\begin{equation}\label{Eq.Theta.w}
\tilde{\theta}\left(w\right)-w\tilde{P}\left(\tilde{\theta}\left(w\right)\right)=0,
\end{equation}
 with  $|\tilde{\theta}\left(w\right)|<1$. It's important to mention that $\tilde{\theta}\left(w\right)$ is the PGF for the gambler's ruin time when $\tilde{L}_{0}=1$. Considering the equation (\ref{Eq.Theta.w})

\begin{eqnarray*}
0&=&\frac{\partial}{\partial w}\tilde{\theta}\left(w\right)|_{w=1}-\frac{\partial}{\partial w}\left\{w\tilde{P}\left(\tilde{\theta}\left(w\right)\right)\right\}|_{w=1}
=\tilde{\theta}^{(1)}\left(w\right)|_{w=1}-\frac{\partial}{\partial w}w\left\{\tilde{P}\left(\tilde{\theta}\left(w\right)\right)\right\}|_{w=1}-w\frac{\partial}{\partial w}\tilde{P}\left(\tilde{\theta}\left(w\right)\right)|_{w=1}\\
&=&\tilde{\theta}^{(1)}\left(1\right)-\tilde{P}\left(\tilde{\theta}\left(1\right)\right)-w\left\{\frac{\partial \tilde{P}\left(\tilde{\theta}\left(w\right)\right)}{\partial \tilde{\theta}\left(w\right)}\cdot\frac{\partial\tilde{\theta}\left(w\right)}{\partial w}|_{w=1}\right\}
=\tilde{\theta}^{(1)}\left(1\right)-\tilde{P}\left(\tilde{\theta}\left(1\right)
\right)-\tilde{P}^{(1)}\left(\tilde{\theta}\left(1\right)\right)\cdot\tilde{\theta}^{(1)}\left(1\right),
\end{eqnarray*}
therefore, $\tilde{P}\left(\tilde{\theta}\left(1\right)\right)=\tilde{\theta}^{(1)}\left(1\right)-\tilde{P}^{(1)}\left(\tilde{\theta}\left(1\right)\right)\cdot
\tilde{\theta}^{(1)}\left(1\right)
=\tilde{\theta}^{(1)}\left(1\right)\left(1-\tilde{P}^{(1)}\left(\tilde{\theta}\left(1\right)\right)\right)$ then
$\tilde{\theta}^{(1)}\left(1\right)=\frac{\tilde{P}\left(\tilde{\theta}\left(1\right)\right)}{\left(1-\tilde{P}^{(1)}\left(\tilde{\theta}\left(1\right)\right)\right)}=\frac{1}{1-\tilde{\mu}}$. Now let's determine the second order moment for $\tilde{\theta}\left(w\right)$, consider again the equation (\ref{Eq.Theta.w}):

\begin{eqnarray*}
0&=&\tilde{\theta}\left(w\right)-w\tilde{P}\left(\tilde{\theta}\left(w\right)\right)
=\frac{\partial}{\partial w}\left\{\tilde{\theta}\left(w\right)-w\tilde{P}\left(\tilde{\theta}\left(w\right)\right)\right\}
=\frac{\partial}{\partial w}\left\{\frac{\partial}{\partial w}\left\{\tilde{\theta}\left(w\right)-w\tilde{P}\left(\tilde{\theta}\left(w\right)\right)\right\}\right\}\\
&=&\frac{\partial}{\partial w}\left\{\frac{\partial}{\partial w}\tilde{\theta}\left(w\right)-\frac{\partial}{\partial w}\left[w\tilde{P}\left(\tilde{\theta}\left(w\right)\right)\right]\right\}
=\frac{\partial}{\partial w}\left\{\frac{\partial}{\partial w}\tilde{\theta}\left(w\right)-\frac{\partial}{\partial w}\left[w\tilde{P}\left(\tilde{\theta}\left(w\right)\right)\right]\right\}\\
&=&\frac{\partial}{\partial w}\left\{\frac{\partial \tilde{\theta}\left(w\right)}{\partial w}-\left[\tilde{P}\left(\tilde{\theta}\left(w\right)\right)+w\frac{\partial}{\partial w}R\left(\tilde{\theta}\left(w\right)\right)\right]\right\}
=\frac{\partial}{\partial w}\left\{\frac{\partial \tilde{\theta}\left(w\right)}{\partial w}-\left[\tilde{P}\left(\tilde{\theta}\left(w\right)\right)+w\frac{\partial \tilde{P}\left(\tilde{\theta}\left(w\right)\right)}{\partial w}\frac{\partial \tilde{\theta}\left(w\right)}{\partial w}\right]\right\}\\
&=&\frac{\partial}{\partial w}\left\{\tilde{\theta}^{(1)}\left(w\right)-\tilde{P}\left(\tilde{\theta}\left(w\right)\right)-w\tilde{P}^{(1)}\left(\tilde{\theta}\left(w\right)\right)\tilde{\theta}^{(1)}\left(w\right)\right\}
=\frac{\partial}{\partial w}\tilde{\theta}^{(1)}\left(w\right)-\frac{\partial}{\partial w}\tilde{P}\left(\tilde{\theta}\left(w\right)\right)\\
&-&\frac{\partial}{\partial w}\left[w\tilde{P}^{(1)}\left(\tilde{\theta}\left(w\right)\right)\tilde{\theta}^{(1)}\left(w\right)\right]
=\frac{\partial}{\partial
w}\tilde{\theta}^{(1)}\left(w\right)-\frac{\partial
\tilde{P}\left(\tilde{\theta}\left(w\right)\right)}{\partial
\tilde{\theta}\left(w\right)}\frac{\partial \tilde{\theta}\left(w\right)}{\partial
w}-\tilde{P}^{(1)}\left(\tilde{\theta}\left(w\right)\right)\tilde{\theta}^{(1)}\left(w\right)\\
&-&w\frac{\partial
\tilde{P}^{(1)}\left(\tilde{\theta}\left(w\right)\right)}{\partial
w}\tilde{\theta}^{(1)}\left(w\right)-w\tilde{P}^{(1)}\left(\tilde{\theta}\left(w\right)\right)\frac{\partial
\tilde{\theta}^{(1)}\left(w\right)}{\partial w}
=\tilde{\theta}^{(2)}\left(w\right)-\tilde{P}^{(1)}\left(\tilde{\theta}\left(w\right)\right)\tilde{\theta}^{(1)}\left(w\right)\\
&-&\tilde{P}^{(1)}\left(\tilde{\theta}\left(w\right)\right)\tilde{\theta}^{(1)}\left(w\right)
-w\tilde{P}^{(2)}\left(\tilde{\theta}\left(w\right)\right)\left(\tilde{\theta}^{(1)}\left(w\right)\right)^{2}-w\tilde{P}^{(1)}\left(\tilde{\theta}\left(w\right)\right)\tilde{\theta}^{(2)}\left(w\right)=\tilde{\theta}^{(2)}\left(w\right)\\
&-&2\tilde{P}^{(1)}\left(\tilde{\theta}\left(w\right)\right)\tilde{\theta}^{(1)}\left(w\right)
-w\tilde{P}^{(2)}\left(\tilde{\theta}\left(w\right)\right)\left(\tilde{\theta}^{(1)}\left(w\right)\right)^{2}-w\tilde{P}^{(1)}\left(\tilde{\theta}\left(w\right)\right)\tilde{\theta}^{(2)}\left(w\right)\\
&=&\tilde{\theta}^{(2)}\left(w\right)\left[1-w\tilde{P}^{(1)}\left(\tilde{\theta}\left(w\right)\right)\right]-
\tilde{\theta}^{(1)}\left(w\right)\left[w\tilde{\theta}^{(1)}\left(w\right)\tilde{P}^{(2)}\left(\tilde{\theta}\left(w\right)\right)+2\tilde{P}^{(1)}\left(\tilde{\theta}\left(w\right)\right)\right],
\end{eqnarray*}


therefore $0=\tilde{\theta}^{(2)}\left(w\right)\left[1-w\tilde{P}^{(1)}\left(\tilde{\theta}\left(w\right)\right)\right]-\tilde{\theta}^{(1)}\left(w\right)\left[w\tilde{\theta}^{(1)}\left(w\right)\tilde{P}^{(2)}\left(\tilde{\theta}\left(w\right)\right)
+2\tilde{P}^{(1)}\left(\tilde{\theta}\left(w\right)\right)\right]$,


\begin{eqnarray*}
\tilde{\theta}^{(2)}\left(w\right)&=&\frac{\tilde{\theta}^{(1)}\left(w\right)\left[w\tilde{\theta}^{(1)}\left(w\right)\tilde{P}^{(2)}\left(\tilde{\theta}\left(w\right)\right)+2R^{(1)}\left(\tilde{\theta}\left(w\right)\right)\right]}{1-w\tilde{P}^{(1)}\left(\tilde{\theta}\left(w\right)\right)}
=\frac{\tilde{\theta}^{(1)}\left(w\right)w\tilde{\theta}^{(1)}\left(w\right)\tilde{P}^{(2)}\left(\tilde{\theta}\left(w\right)\right)}{1-w\tilde{P}^{(1)}\left(\tilde{\theta}\left(w\right)\right)}\\
&+&\frac{2\tilde{\theta}^{(1)}\left(w\right)\tilde{P}^{(1)}\left(\tilde{\theta}\left(w\right)\right)}{1-w\tilde{P}^{(1)}\left(\tilde{\theta}\left(w\right)\right)}
\end{eqnarray*}
evaluating the last expression in $w=1$:
\begin{eqnarray*}
\tilde{\theta}^{(2)}\left(1\right)&=&\frac{\left(\tilde{\theta}^{(1)}\left(1\right)\right)^{2}\tilde{P}^{(2)}\left(\tilde{\theta}\left(1\right)\right)}{1-\tilde{P}^{(1)}\left(\tilde{\theta}\left(1\right)\right)}+\frac{2\tilde{\theta}^{(1)}\left(1\right)\tilde{P}^{(1)}\left(\tilde{\theta}\left(1\right)\right)}{1-\tilde{P}^{(1)}\left(\tilde{\theta}\left(1\right)\right)}
=\frac{\left(\tilde{\theta}^{(1)}\left(1\right)\right)^{2}\tilde{P}^{(2)}\left(1\right)}{1-\tilde{P}^{(1)}\left(1\right)}+\frac{2\tilde{\theta}^{(1)}\left(1\right)\tilde{P}^{(1)}\left(1\right)}{1-\tilde{P}^{(1)}\left(1\right)}\\
&=&\frac{\left(\frac{1}{1-\tilde{\mu}}\right)^{2}\tilde{P}^{(2)}\left(1\right)}{1-\tilde{\mu}}+\frac{2\left(\frac{1}{1-\tilde{\mu}}\right)\tilde{\mu}}{1-\tilde{\mu}}=\frac{\tilde{P}^{(2)}\left(1\right)}{\left(1-\tilde{\mu}\right)^{3}}+\frac{2\tilde{\mu}}{\left(1-\tilde{\mu}\right)^{2}}
=\frac{\sigma^{2}-\tilde{\mu}+\tilde{\mu}^{2}}{\left(1-\tilde{\mu}\right)^{3}}+\frac{2\tilde{\mu}}{\left(1-\tilde{\mu}\right)^{2}}\\
&=&\frac{\sigma^{2}-\tilde{\mu}+\tilde{\mu}^{2}+2\tilde{\mu}\left(1-\tilde{\mu}\right)}{\left(1-\tilde{\mu}\right)^{3}}
=\frac{\sigma^{2}+\tilde{\mu}-\tilde{\mu}^{2}}{\left(1-\tilde{\mu}\right)^{3}}=\frac{\sigma^{2}}{\left(1-\tilde{\mu}\right)^{3}}+\frac{\tilde{\mu}\left(1-\tilde{\mu}\right)}{\left(1-\tilde{\mu}\right)^{3}}
=\frac{\sigma^{2}}{\left(1-\tilde{\mu}\right)^{3}}+\frac{\tilde{\mu}}{\left(1-\tilde{\mu}\right)^{2}}.
\end{eqnarray*}
\end{proof}

\begin{Coro}\label{Corolario1.A}

The first and second moments for the gambler's ruin are

\begin{eqnarray}\label{Second.Order.Gamblers.Ruin}
\begin{array}{ll}
\esp\left[T\right]=\frac{\esp\left[\tilde{L}_{0}\right]}{1-\tilde{\mu}},&
Var\left[T\right]=\frac{Var\left[\tilde{L}_{0}\right]}{\left(1-\tilde{\mu}\right)^{2}}+\frac{\sigma^{2}\esp\left[\tilde{L}_{0}\right]}{\left(1-\tilde{\mu}\right)^{3}}.
\end{array}
\end{eqnarray}
\end{Coro}



%______________________________________________________________________
\section{Appendix B: General Case Calculations Exhaustive Policy}\label{Secc.Append.B}
%______________________________________________________________________

%_______________________________________________________________
%\subsection{Calculations}
%_______________________________________________________________


Remember the equations given in equations (\ref{Ec.Gral.Primer.Momento.Ind.Exh}) and (\ref{Ec.Gral.Segundo.Momento.Ind.Exh}) for the first and second order partial derivatives respectively. The first moments equations for the queue lengths as before for the times the server arrives to the queue to start attending are


\begin{eqnarray}
\begin{array}{ll}
D_{i}F_{1}=\indora_{i\neq1}D_{1}F_{1}D\tilde{\theta}_{1}D_{i}\tilde{P}_{i}+\indora_{i=2}D_{2}F_{1},&
D_{i}F_{2}=\indora_{i\neq2}D_{2}F_{2}D\tilde{\theta}_{2}D_{i}\tilde{P}_{i}+\indora_{i=1}D_{1}F_{2}\\
D_{i}F_{3}=\indora_{i\neq3}D_{3}F_{3}D\tilde{\theta}_{3}D_{i}\tilde{P}_{i}+\indora_{i=4}D_{4}F_{3},&
D_{i}F_{4}=\indora_{i\neq4}D_{4}F_{4}D\tilde{\theta}_{4}D_{i}\tilde{P}_{i}+\indora_{i=3}D_{3}F_{4}.
\end{array}
\end{eqnarray}
We can obtain the linear system of equations: $f_{1}\left(i\right)=D_{i}R_{2}+D_{i}F_{2}+\indora_{i\geq3}D_{i}F_{4}$, so

\begin{eqnarray*}
\begin{array}{ll}
f_{1}\left(1\right)=r_{2}\tilde{\mu}_{1}+\frac{\tilde{\mu}_{1}}{1-\tilde{\mu}_{2}}f_{2}\left(2\right)+f_{2}\left(1\right),&
f_{1}\left(2\right)=r_{2}\tilde{\mu}_{2},\\
f_{1}\left(3\right)=r_{2}\tilde{\mu}_{3}+\frac{\tilde{\mu}_{3}}{1-\tilde{\mu}_{2}}f_{2}\left(2\right)+F_{3,2}^{(1)}\left(1\right),&
f_{1}\left(4\right)=r_{2}\tilde{\mu}_{4}+\frac{\tilde{\mu}_{4}}{1-\tilde{\mu}_{2}}f_{2}\left(2\right)+F_{4,2}^{(1)}\left(1\right),\end{array}
\end{eqnarray*}

for the rest of the queues we have that $f_{2}\left(i\right)=D_{i}\left(R_{1}+F_{1}+\indora_{i\geq3}F_{3}\right)$, $f_{3}\left(i\right)=D_{i}\left(R_{4}+F_{4}+\indora_{i\leq2}F_{2}\right)$ and $f_{4}\left(i\right)=D_{i}\left(R_{3}+F_{3}+\indora_{i\leq2}F_{1}\right)$, equivalently

\begin{eqnarray*}
\begin{array}{ll}
f_{2}\left(1\right)=r_{1}\tilde{\mu}_{1},&
f_{2}\left(2\right)=r_{1}\tilde{\mu}_{2}+\frac{\tilde{\mu}_{2}}{1-\tilde{\mu}_{1}}f_{1}\left(1\right)+f_{1}\left(2\right),\\
f_{2}\left(3\right)=r_{1}\tilde{\mu}_{3}+\frac{\tilde{\mu}_{3}}{1-\tilde{\mu}_{1}}f_{1}\left(1\right)+F_{3,1}^{(1)}\left(1\right),&
f_{2}\left(4\right)=r_{1}\tilde{\mu}_{4}+\frac{\tilde{\mu}_{4}}{1-\tilde{\mu}_{1}}f_{1}\left(1\right)+F_{4,1}^{(1)}\left(1\right),\\
f_{3}\left(1\right)=\tilde{r}_{4}\tilde{\mu}_{1}+\frac{\tilde{\mu}_{1}}{1-\tilde{\mu}_{4}}f_{4}\left(4\right)+F_{1,4}^{(1)}\left(1\right),&
f_{3}\left(2\right)=\tilde{r}_{4}\tilde{\mu}_{2}+\frac{\tilde{\mu}_{2}}{1-\tilde{\mu}_{4}}f_{4}\left(4\right)+F_{2,4}^{(1)}\left(1\right),\\
f_{3}\left(3\right)=\tilde{r}_{4}\tilde{\mu}_{3}+\frac{\tilde{\mu}_{3}}{1-\tilde{\mu}_{4}}f_{4}\left(4\right)+f_{4}\left(3\right),&
f_{3}\left(4\right)=\tilde{r}_{4}\tilde{\mu}_{4}\\
f_{4}\left(1\right)=\tilde{r}_{3}\tilde{\mu}_{1}+\frac{\tilde{\mu}_{1}}{1-\tilde{\mu}_{3}}f_{3}\left(3\right)+F_{1,3}^{(1)}\left(1\right),&
f_{4}\left(2\right)=\tilde{r}_{3}\mu_{2}+\frac{\tilde{\mu}_{2}}{1-\tilde{\mu}_{3}}f_{3}\left(3\right)+F_{2,3}^{(1)}\left(1\right),\\
f_{4}\left(3\right)=\tilde{r}_{3}\tilde{\mu}_{3},&
f_{4}\left(4\right)=\tilde{r}_{3}\tilde{\mu}_{4}+\frac{\tilde{\mu}_{4}}{1-\tilde{\mu}_{3}}f_{3}\left(3\right)+f_{3}\left(4\right),\\
\end{array}
\end{eqnarray*}

Then we have that if $\mu=\tilde{\mu}_{1}+\tilde{\mu}_{2}<1$, $\hat{\mu}=\tilde{\mu}_{3}+\tilde{\mu}_{4}<1$, $r=r_{1}+r_{2}$ and $\hat{r}=\tilde{r}_{3}+\tilde{r}_{4}$  the system's solution is given by

\begin{eqnarray*}
\begin{array}{lll}
f_{2}\left(1\right)=r_{1}\tilde{\mu}_{1},&
f_{1}\left(2\right)=r_{2}\tilde{\mu}_{2},&
f_{3}\left(4\right)=r_{4}\tilde{\mu}_{4},\\
f_{4}\left(3\right)=r_{3}\tilde{\mu}_{3},&
f_{1}\left(1\right)=r\frac{\tilde{\mu}_{1}\left(1-\tilde{\mu}_{1}\right)}{1-\mu},&
f_{2}\left(2\right)=r\frac{\tilde{\mu}_{2}\left(1-\tilde{\mu}_{2}\right)}{1-\mu},\\
f_{1}\left(3\right)=\tilde{\mu}_{3}\left(r_{2}+\frac{r\tilde{\mu}_{2}}{1-\mu}\right)+F_{3,2}^{(1)}\left(1\right),&
f_{1}\left(4\right)=\tilde{\mu}_{4}\left(r_{2}+\frac{r\tilde{\mu}_{2}}{1-\mu}\right)+F_{4,2}^{(1)}\left(1\right),&
f_{2}\left(3\right)=\tilde{\mu}_{3}\left(r_{1}+\frac{r\tilde{\mu}_{1}}{1-\tilde{\mu}}\right)+F_{3,1}^{(1)}\left(1\right),\\
f_{2}\left(4\right)=\tilde{\mu}_{4}\left(r_{1}+\frac{r\tilde{\mu}_{1}}{1-\mu}\right)+F_{4,,1}^{(1)}\left(1\right),&
f_{3}\left(1\right)=\tilde{\mu}_{1}\left(r_{4}+\frac{\hat{r}\tilde{\mu}_{4}}{1-\hat{\mu}}\right)+F_{1,4}^{(1)}\left(1\right),&
f_{3}\left(2\right)=\tilde{\mu}_{2}\left(r_{4}+\frac{\hat{r}\tilde{\mu}_{4}}{1-\hat{\mu}}\right)+F_{2,4}^{(1)}\left(1\right),\\
f_{3}\left(3\right)=\hat{r}\frac{\tilde{\mu}_{3}\left(1-\tilde{\mu}_{3}\right)}{1-\hat{\mu}},&
f_{4}\left(1\right)=\tilde{\mu}_{1}\left(r_{3}+\frac{\hat{r}\tilde{\mu}_{3}}{1-\hat{\mu}}\right)+F_{1,3}^{(1)}\left(1\right),&
f_{4}\left(2\right)=\tilde{\mu}_{2}\left(r_{3}+\frac{\hat{r}\tilde{\mu}_{3}}{1-\hat{\mu}}\right)+F_{2,3}^{(1)}\left(1\right),\\
&
f_{4}\left(4\right)=\hat{r}\frac{\tilde{\mu}_{4}\left(1-\tilde{\mu}_{4}\right)}{1-\hat{\mu}}.&
\end{array}
\end{eqnarray*}

Now, developing the equations given in (\ref{Eq.Gral.Second.Order.Exhaustive})we obtain for instance $f_{1}\left(1,1\right)=\left(\frac{\tilde{\mu}_{1}}{1-\tilde{\mu}_{2}}\right)^{2}f_{2}\left(2,2\right)
+2\frac{\tilde{\mu}_{1}}{1-\tilde{\mu}_{2}}f_{2}\left(2,1\right)
+f_{2}\left(1,1\right)
+\tilde{\mu}_{1}^{2}\left(R_{2}^{(2)}+f_{2}\left(2\right)\theta_{2}^{(2)}\right)
+\tilde{P}_{1}^{(2)}\left(\frac{f_{2}\left(2\right)}{1-\tilde{\mu}_{2}}+r_{2}\right)+2r_{2}\tilde{\mu}_{2}f_{2}\left(1\right)$, proceeding in a similar manner, we have the following general expressions

{\small{
\begin{eqnarray}\label{Eq.Sdo.Orden.Exh}
\begin{array}{l}
f_{1}\left(i,j\right)=\indora_{i=1}f_{2}\left(1,1\right)
+\left[\left(1-\indora_{i=j=3}\right)\indora_{i+j\leq6}\indora_{i\leq j}\frac{\mu_{j}}{1-\tilde{\mu}_{2}}
+\left(1-\indora_{i=j=3}\right)\indora_{i+j\leq6}\indora_{i>j}\frac{\mu_{i}}{1-\tilde{\mu}_{2}}
+\indora_{i=1}\frac{\mu_{i}}{1-\tilde{\mu}_{2}}\right]f_{2}\left(1,2\right)\\
+
\indora_{i,j\neq2}\left(\frac{1}{1-\tilde{\mu}_{2}}\right)^{2}\mu_{i}\mu_{j}f_{2}\left(2,2\right)
+\left[\indora_{i,j\neq2}\tilde{\theta}_{2}^{(2)}\tilde{\mu}_{i}\tilde{\mu}_{j}
+\indora_{i,j\neq2}\indora_{i=j}\frac{\tilde{P}_{i}^{(2)}}{1-\tilde{\mu}_{2}}
+\indora_{i,j\neq2}\indora_{i\neq j}\frac{\tilde{\mu}_{i}\tilde{\mu}_{j}}{1-\tilde{\mu}_{2}}\right]f_{2}\left(2\right)\\
+\left[r_{2}\tilde{\mu}_{i}
+\indora_{i\geq3}F_{i,2}^{(1)}\right]f_{2}\left(j\right)
+\left[r_{2}\tilde{\mu}_{j}
+\indora_{j\geq3}F_{j,2}^{(1)}\right]f_{2}\left(i\right)
+\left[R_{2}^{(2)}
+\indora_{i=j}r_{2}\right]\tilde{\mu}_{i}\mu_{j}\\
+\indora_{j\geq3}F_{j,2}^{(1)}\left[\indora_{j\neq i}F_{i,2}^{(1)}
+r_{2}\tilde{\mu}_{i}\right]
+r_{2}\left[\indora_{i=j}P_{i}^{(2)}
+\indora_{i\geq3}F_{i,2}^{(1)}\tilde{\mu}_{j}\right]
+\indora_{i\geq3}\indora_{j=i}F_{i,2}^{(2)}\\
f_{2}\left(i,j\right)=
\indora_{i,j\neq1}\left(\frac{1}{1-\tilde{\mu}_{1}}\right)^{2}\tilde{\mu}_{i}\tilde{\mu}_{j}f_{1}\left(1,1\right)
+\left[\left(1-\indora_{i=j=3}\right)\indora_{i+j\leq6}\indora_{i\leq j}\frac{\tilde{\mu}_{j}}{1-\tilde{\mu}_{1}}
+\left(1-\indora_{i=j=3}\right)\indora_{i+j\leq6}\indora_{i>j}\frac{\tilde{\mu}_{i}}{1-\tilde{\mu}_{1}}\right.
\\
+\left.\indora_{i=2}\frac{\tilde{\mu}_{i}}{1-\tilde{\mu}_{1}}\right]f_{1}\left(1,2\right)
+\indora_{i=2}f_{1}\left(2,2\right)
+\left[\indora_{i,j\neq1}\tilde{\theta}_{1}^{(2)}\tilde{\mu}_{i}\tilde{\mu}_{j}
+\indora_{i,j\neq1}\indora_{i\neq j}\frac{\tilde{\mu}_{i}\tilde{\mu}_{j}}{1-\tilde{\mu}_{1}}
+\indora_{i,j\neq1}\indora_{i=j}\frac{\tilde{P}_{i}^{(2)}}{1-\tilde{\mu}_{1}}\right]f_{1}\left(1\right)\\
+\left[r_{1}\mu_{i}+\indora_{i\geq3}F_{i,1}^{(1)}\right]f_{1}\left(j\right)
+\left[\indora_{j\geq3}F_{j,1}^{(1)}+r_{1}\mu_{j}\right]f_{1}\left(i\right)
+\left[R_{1}^{(2)}+\indora_{i=j}\right]\tilde{\mu}_{i}\tilde{\mu}_{j}
+\indora_{i\geq3}F_{i,1}^{(1)}\left[r_{1}\mu_{j}
+\indora_{j\neq i}F_{j,1}^{(1)}\right]\\
+r_{1}\left[\indora_{j\geq3}\mu_{i}F_{j,1}^{(1)}
+\indora_{i=j}P_{i}^{(2)}\right]
+\indora_{i\geq3}\indora_{j=i}F_{i,1}^{(2)}\\
f_{3}\left(i,j\right)=
\indora_{i=3}f_{4}\left(3,3\right)
+\left[\left(1-\indora_{i=j=2}\right)\indora_{i+j\geq4}\indora_{i\leq j}\frac{\tilde{\mu}_{i}}{1-\tilde{\mu}_{4}}
+\left(1-\indora_{i=j=2}\right)\indora_{i+j\geq4}\indora_{i>j}\frac{\tilde{\mu}_{j}}{1-\tilde{\mu}_{4}}
+\indora_{i=3}\frac{\tilde{\mu}_{i}}{1-\tilde{\mu}_{4}}\right]f_{4}\left(3,4\right)\\
+\indora_{i,j\neq4}f_{4}\left(4,4\right)\left(\frac{1}{1-\tilde{\mu}_{4}}\right)^{2}\tilde{\mu}_{i}\tilde{\mu}_{j}
+\left[\indora_{i,j\neq4}\tilde{\theta}_{4}^{(2)}\tilde{\mu}_{i}\tilde{\mu}_{j}
+\indora_{i,j\neq4}\indora_{i=j}\frac{\tilde{P}_{i}^{(2)}}{1-\tilde{\mu}_{4}}
+\indora_{i,j\neq4}\indora_{i\neq j}\frac{\tilde{\mu}_{i}\tilde{\mu}_{j}}{1-\tilde{\mu}_{4}}\right]f_{4}\left(4\right)\\
+\left[r_{4}\tilde{\mu}_{i}+\indora_{i\leq2}F_{i,4}^{(1)}\right]f_{4}\left(j\right)
+\left[r_{4}\tilde{\mu}_{j}+\indora_{j\leq2}F_{j,4}^{(1)}\right]f_{4}\left(i\right)
+\left[R_{4}^{(2)}+\indora_{i=j}r_{4}\right]\tilde{\mu}_{i}\tilde{\mu}_{j}\\
+   \indora_{i\leq2}F_{i,4}^{(1)}\left[r_{4}\tilde{\mu}_{j}
+\indora_{j\neq i}F_{j,4}^{(1)}\right]
+r_{4}\left[\indora_{i=j}P_{i}^{(2)}+\indora_{j\leq2}\tilde{\mu}_{i}F_{j,4}^{(1)}\right]
+\indora_{i\leq2}\indora_{j=i}F_{i,4}^{(2)}\\
f_{4}\left(i,j\right)=
\indora_{i,j\neq3}f_{3}\left(3,3\right)\left(\frac{1}{1-\tilde{\mu}_{3}}\right)^{2}\tilde{\mu}_{i}\tilde{\mu}_{j}
+\left[\left(1-\indora_{i=j=2}\right)\indora_{i+j\geq5}\indora_{i\leq j}\frac{\tilde{\mu}_{i}}{1-\tilde{\mu}_{3}}
+\left(1-\indora_{i=j=2}\right)\indora_{i+j\geq5}\indora_{i>j}\frac{\tilde{\mu}_{j}}{1-\tilde{\mu}_{3}}\right.\\
+\left.\indora_{i=4}\frac{\tilde{\mu}_{i}}{1-\tilde{\mu}_{3}}\right]f_{3}\left(3,4\right)
+\indora_{i=4}f_{3}\left(4,4\right)
+\left[\indora_{i,j\neq3}\tilde{\theta}_{3}^{(2)}\tilde{\mu}_{i}\tilde{\mu}_{j}
+\indora_{i,j\neq3}\indora_{i=j}\frac{\tilde{P}_{i}^{(2)}}{1-\tilde{\mu}_{3}}
+\indora_{i,j\neq3}\indora_{i\neq j}\frac{\tilde{\mu}_{i}\tilde{\mu}_{j}}{1-\tilde{\mu}_{3}}\right]f_{3}\left(3\right)\\
+\left[r_{3}\tilde{\mu}_{i}+\indora_{i\leq2}F_{i,3}^{(1)}\right]f_{3}\left(j\right)
+\left[r_{3}\tilde{\mu}_{j}+\indora_{j\leq2}F_{j,3}^{(1)}\right]f_{3}\left(i\right)
+\left[R_{3}^{(2)}+\indora_{i=j}r_{3}\right]\tilde{\mu}_{i}\tilde{\mu}_{j}\\
+\indora_{i\leq2}F_{i,3}^{(1)}\left[r_{3}\tilde{\mu}_{j}+\indora_{j\neq i}F_{j,3}^{(1)}\right]
+r_{3}\left[\indora_{i=j}P_{i}^{(2)}+\indora_{j\leq2}\tilde{\mu}_{i}F_{j,3}^{(1)}\right]
+\indora_{i\leq2}\indora_{j=i}F_{i,3}^{(2)}
\end{array}
\end{eqnarray}}}
from which we obtain the linear equations systems
\begin{eqnarray}\label{System.Second.Order.Moments.uno}
\begin{array}{ll}
f_{1}\left(1,1\right)=a_{1}f_{2}\left(2,2\right)
+a_{2}f_{2}\left(2,1\right)
+a_{3}f_{2}\left(1,1\right)
+K_{1},&
f_{1}\left(1,2\right)=K_{2}\\
f_{1}\left(1,3\right)=a_{4}f_{2}\left(2,2\right)+a_{5}f\left(2,1\right)+K_{3},&
f_{1}\left(1,4\right)=a_{6}f_{2}\left(2,2\right)+a_{7}f_{2}\left(2,1\right)+K_{4}\\
f_{1}\left(2,2\right)=K_{5},&
f_{1}\left(2,3\right)=K_{6}\\
f_{1}\left(2,4\right)=K_{7},&
f_{1}\left(3,3\right)=a_{8}f_{2}\left(2,2\right)+K_{8}\\
f_{1}\left(3,4\right)=a_{9}f_{2}\left(2,2\right)+K_{9},&
f_{1}\left(4,4\right)=a_{10}f_{2}\left(2,2\right)+K_{10}\\
f_{2}\left(1,1\right)=K_{11},&
f_{2}\left(1,2\right)=K_{12}\\
f_{2}\left(1,3\right)=K_{13},&
f_{2}\left(1,4\right)=K_{14}\\
f_{2}\left(2,2\right)=a_{11}f_{1}\left(1,1\right)
+a_{12}f_{1}\left(1,2\right)+a_{13}f_{1}\left(2,2\right)+K_{15},&
f_{2}\left(2,3\right)=a_{14}f_{1}\left(1,1\right)+a_{15}f_{1}\left(1,2\right)+K_{16}\\
f_{2}\left(2,4\right)=a_{16}f_{1}\left(1,1\right)+a_{17}f_{1}\left(1,2\right)+K_{17},&
f_{2}\left(3,3\right)=a_{18}f_{1}\left(1,1\right)+K_{18}\\
f_{2}\left(3,4\right)=a_{19}f_{1}\left(1,1\right)+K_{19},&
f_{2}\left(4,4\right)=a_{20}f_{1}\left(1,1\right)+K_{20}
\end{array}
\end{eqnarray}



\begin{eqnarray}\label{System.Second.Order.Moments.dos}
\begin{array}{ll}
f_{3}\left(1,1\right)=a_{21}f_{4} \left(4,4\right)+K_{21},&
f_{3}\left(1,2\right)=a_{22}f_{4}\left(4,4\right)+K_{22}\\
f_{3}\left(1,3\right)=a_{23}f_{4}\left(4,4\right)+a_{24}f_{4}\left(4,3\right)+K_{23},&
f_{3}\left(1,4\right)=K_{24}\\
f_{3}\left(2,2\right)=a_{25}f_{4}\left(4,4\right)+K_{25},&
f_{3}\left(2,3\right)=a_{26}f_{4}\left(4,4\right)+a_{27}f_{4}\left(4,3\right)+K_{26}\\
f_{3}\left(2,4\right)=K_{27},&
f_{3}\left(3,3\right)=a_{28}f_{4}\left(4,4\right)+a_{29}f_{4}\left(4,3\right)+a_{30}f_{4}\left(3,3\right)+K_{28}\\
f_{3}\left(3,4\right)=K_{29},&
f_{3}\left(4,4\right)=K_{30}\\
f_{4}\left(1,1\right)=a_{31}f_{3}\left(3,3\right)+K_{31},&
f_{4}\left(1,2\right)=a_{32}f_{3}\left(3,3\right)+K_{32}\\
F_{4}\left(1,3\right)=K_{33},&
f_{4}\left(1,4\right)=a_{33}f_{3}\left(3,3\right)+a_{34}f_{3}\left(3,4\right)+K_{34}\\
f_{4}\left(2,2\right)=a_{35}f_{3}\left(3,3\right)+K_{35},&
f_{4}\left(2,3\right)=K_{36}\\
f_{4}\left(2,4\right)=a_{36}f_{3}\left(3,3\right)+a_{37}f_{3}\left(3,4\right)+K_{37},&
f_{4}\left(3,3\right)=K_{38}\\
f_{4}\left(3,4\right)=K_{39},&
f_{4}\left(4,4\right)=a_{38}f_{3}\left(3,3\right)+a_{39}f_{3}\left(3,4\right)+a_{40}f_{3}\left(4,4\right)+K_{40}
\end{array}
\end{eqnarray}



%Which can be reduced to solve the system given in (\ref{System.Second.Order.Moments.uno}) and (\ref{System.Second.Order.Moments.dos}).

with values for $a_{i}$ and $K_{i}$
%{\small{
\begin{eqnarray}\label{Coefficients.Ais.Exh}
\begin{array}{llll}
a_{1}=\left(\frac{\tilde{\mu}_{1}}{1-\tilde{\mu}_{2}}\right)^{2},&
a_{2}=2\frac{\tilde{\mu}_{1}}{1-\tilde{\mu}_{2}},&
a_{3}=1,&
a_{4}=\left(\frac{1}{1-\tilde{\mu}_{2}}\right)^{2}\tilde{\mu}_{1}\tilde{\mu}_{3},\\
a_{5}=\frac{\tilde{\mu}_{3}}{1-\tilde{\mu}_{2}},&
a_{6}=\left(\frac{1}{1-\tilde{\mu}_{2}}\right)^{2}\tilde{\mu}_{1}\tilde{\mu}_{4},&
a_{7}=\frac{\tilde{\mu}_{4}}{1-\tilde{\mu}_{2}},&
a_{8}=\left(\frac{1}{1-\tilde{\mu}_{2}}\right)^{2}\tilde{\mu}_{3}^{2},\\
a_{9}=\left(\frac{1}{1-\tilde{\mu}_{2}}\right)^{2}\tilde{\mu}_{3}\tilde{\mu}_{4},&
a_{10}=\left(\frac{\tilde{\mu}_{4}}{1-\tilde{\mu}_{2}}\right)^{2}&
a_{11}=\left(\frac{\tilde{\mu}_{2}}{1-\tilde{\mu}_{1}}\right)^{2}&
a_{12}=2\frac{\tilde{\mu}_{2}}{1-\tilde{\mu}_{1}}\\
a_{13}=1&
a_{14}=\left(\frac{1}{1-\tilde{\mu}_{1}}\right)^{2}\tilde{\mu}_{2}\tilde{\mu}_{3}&
a_{15}=\frac{\tilde{\mu}_{3}}{1-\tilde{\mu}_{1}},&
a_{16}=\left(\frac{1}{1-\tilde{\mu}_{1}}\right)^{2}\tilde{\mu}_{2}\tilde{\mu}_{4},\\
a_{17}=\frac{\tilde{\mu}_{4}}{1-\tilde{\mu}_{1}}&
a_{18}=\left(\frac{\tilde{\mu}_{3}}{1-\tilde{\mu}_{1}}\right)^{2},&
a_{19}=\left(\frac{1}{1-\tilde{\mu}_{1}}\right)^{2}\tilde{\mu}_{3}\tilde{\mu}_{4}&
a_{20}=\left(\frac{\tilde{\mu}_{4}}{1-\tilde{\mu}_{1}}\right)^{2}\\
a_{21}=\left(\frac{\tilde{\mu}_{1}}{1-\tilde{\mu}_{4}}\right)^{2},&
a_{22}=\left(\frac{1}{1-\tilde{\mu}_{4}}\right)^{2}\tilde{\mu}_{1}\tilde{\mu}_{2}&
a_{23}=\left(\frac{1}{1-\tilde{\mu}_{4}}\right)^{2}\tilde{\mu}_{1}\tilde{\mu}_{3}&
a_{24}=\frac{\tilde{\mu}_{1}}{1-\tilde{\mu}_{4}}f_{4}\left(4,3\right)\\
a_{25}=\left(\frac{\tilde{\mu}_{2}}{1-\tilde{\mu}_{4}}\right)^{2}&
a_{26}=\left(\frac{1}{1-\tilde{\mu}_{4}}\right)^{2}\tilde{\mu}_{2}\tilde{\mu}_{3}&
a_{27}=\frac{\tilde{\mu}_{2}}{1-\tilde{\mu}_{4}},&
a_{28}=\left(\frac{\tilde{\mu}_{3}}{1-\tilde{\mu}_{4}}\right)^{2}\\
a_{29}=2\frac{\tilde{\mu}_{3}}{1-\tilde{\mu}_{4}}&
a_{30}=1&
a_{31}=\left(\frac{\tilde{\mu}_{3}}{1-\tilde{\mu}_{4}}\right)^{2}&
a_{32}=\left(\frac{1}{1-\tilde{\mu}_{3}}\right)^{2}\tilde{\mu}_{1}\tilde{\mu}_{2}\\
a_{33}=\left(\frac{1}{1-\tilde{\mu}_{3}}\right)^{2}\tilde{\mu}_{1}\tilde{\mu}_{3}&
a_{34}=\frac{\tilde{\mu}_{1}}{1-\tilde{\mu}_{3}}&
a_{35}=\left(\frac{\tilde{\mu}_{2}}{1-\tilde{\mu}_{3}}\right)^{2}&
a_{36}=\left(\frac{1}{1-\tilde{\mu}_{3}}\right)^{2}\tilde{\mu}_{2}\tilde{\mu}_{4}\\
a_{37}=\frac{\tilde{\mu}_{2}}{1-\tilde{\mu}_{3}}&
a_{38}=\left(\frac{\tilde{\mu}_{4}}{1-\tilde{\mu}_{3}}\right)^{2}&
a_{39}=2\frac{\tilde{\mu}_{4}}{1-\tilde{\mu}_{3}},&
a_{40}=1
\end{array}
\end{eqnarray}%}}





\begin{eqnarray}\label{Coefficients.kis.Exh.uno}
\begin{array}{l}
K_{1}=\tilde{\mu}_{1}^{2}\left(R_{2}^{(2)}+f_{2}\left(2\right)\theta_{2}^{(2)}\right)
+\tilde{P}_{1}^{(2)}\left(\frac{f_{2}\left(2\right)}{1-\tilde{\mu}_{2}}+r_{2}\right)
+2r_{2}\tilde{\mu}_{2}f_{2}\left(1\right)\\
K_{2}=\tilde{\mu}_{1}\tilde{\mu}_{2}\left[R_{2}^{(2)}
+r_{2}\right]
+r_{2}\left[\tilde{\mu}_{1}f_{2}\left(2\right)
+\tilde{\mu}_{2}f_{2}\left(1\right)\right],\\
K_{3}=\tilde{\mu}_{1}\tilde{\mu}_{3}\left[R_{2}^{(2)}+r_{2}+f_{2}\left(2\right)\left(\tilde{\theta}_{2}^{(2)}+\frac{1}{1-\tilde{\mu}_{2}}\right)\right]
+r_{2}\tilde{\mu}_{1}\left[F_{3,2}^{(1)}+f_{2}\left(1\right)\right]
+\left[r_{2}\tilde{\mu}_{3}+F_{3,2}^{(1)}\right]f_{2}\left(1\right)\\
K_{4}=\tilde{\mu}_{1}\tilde{\mu}_{4}\left[R_{2}^{(2)}
+r_{2}+f_{2}\left(2\right)\left(\tilde{\theta}_{2}^{(2)}
+\frac{1}{1-\tilde{\mu}_{2}}\right)\right]
+r_{2}\tilde{\mu}_{1}\left[f_{2}\left(4\right)+F_{4,2}^{(1)}\right]
+f_{2}\left(1\right)\left[r_{2}\tilde{\mu}_{4}+F_{4,2}^{(1)}\right]\\
K_{5}=\tilde{\mu}_{2}^{2}\left[R_{2}^{(2)}+2r_{2}\frac{r\left(1-\tilde{\mu}_{2}\right)}{1-\mu}\right]+r_{2}\tilde{P}_{2}^{(2)}\\
K_{6}=\tilde{\mu}_{2}\tilde{\mu}_{3}\left[R_{2}^{(2)}
+r_{2}\right]
+r_{2}\tilde{\mu}_{2}\left[f_{2}\left(3\right)+F_{3,2}^{(1)}\right]
+f_{2}\left(2\right)\left[r_{2}\tilde{\mu}_{3}+F_{3,2}^{(1)}\right]\\
K_{7}=\tilde{\mu}_{2}\tilde{\mu}_{4}\left[R_{2}^{(2)}+r_{2}\right]
+r_{2}\tilde{\mu}_{2}\left[f_{2}\left(4\right)+F_{4,2}^{(1)}\right]
+f_{2}\left(2\right)\left[r_{2}\tilde{\mu}_{4}+F_{4,2}^{(1)}\right]\\
K_{8}=\tilde{\mu}_{3}^{2}\left[R_{2}^{(2)}+
+f_{2}\left(2\right)\tilde{\theta}_{2}^{(2)}\right]
+\tilde{P}_{3}^{(2)}\left[\frac{f_{2}\left(2\right)}{1-\tilde{\mu}_{2}}
+r_{2}\right]
+2r_{2}\tilde{\mu}_{3}\left[f_{2}\left(3\right)+F_{3,2}^{(1)}\right]
+2f_{2}\left(3\right)F_{3,2}^{(1)}+F_{3,2}^{(2)}\\
K_{9}=\tilde{\mu}_{3}\tilde{\mu}_{4}\left[R_{2}^{(2)}
+r_{2}
+\left(\tilde{\theta}_{2}^{(2)}+\frac{1}{1-\tilde{\mu}_{2}}\right)f_{2}\left(2\right)\right]+r_{2}\tilde{\mu}_{3}\left(f_{2}\left(4\right)+F_{4,2}^{(1)}\right)
+r_{2}\tilde{\mu}_{4}\left(f_{2}\left(3\right)+F_{3,2}^{(1)}\right)\\
+F_{4,2}^{(1)}\left(f_{2}\left(3\right)+F_{3,2}^{(1)}\right)
+F_{3,2}^{(1)}f_{2}\left(4\right)\\
K_{10}=\tilde{\mu}_{4}^{2}\left[R_{2}^{(2)}+f_{2}\left(2\right)\tilde{\theta}_{2}^{(2)}\right]
+\tilde{P}_{4}^{(2)}\left[r_{2}+\frac{f_{2}\left(2\right)}{1-\tilde{\mu}_{2}}\right]
+2r_{2}\tilde{\mu}_{4}\left[f_{2}\left(4\right)+F_{4,2}^{(1)}\right]
+2F_{4,2}^{(1)}f_{2}\left(4\right)
\end{array}
\end{eqnarray}
\begin{eqnarray}\label{Coefficients.kis.Exh.dos}
\begin{array}{l}
K_{11}=R_{1}^{2}\tilde{\mu}_{1}^{2}+r_{1}\tilde{P}_{1}^{(2)}
+2r_{1}\tilde{\mu}_{1}f_{1}\left(1\right)\\
K_{12}=\tilde{\mu}_{1}\tilde{\mu}_{2}\left[R_{1}^{(2)}+r_{1}\right]
+r_{1}\left[\tilde{\mu}_{1}f_{1}\left(2\right)+\tilde{\mu}_{2}f_{1}\left(1\right)\right]\\
K_{13}=\tilde{\mu}_{1}\tilde{\mu}_{3}\left[R_{1}^{(2)}+r_{1}\right]
+r_{1}\tilde{\mu}_{1}\left[f_{1}\left(3\right)+F_{3,1}^{(1)}\right]
+f_{1}\left(1\right)\left[r_{1}\tilde{\mu}_{3}+F_{3,1}^{(1)}\right]\\
K_{14}=\tilde{\mu}_{1}\tilde{\mu}_{4}\left[R_{1}^{(2)}+r_{1}\right]
+r_{1}\tilde{\mu}_{1}\left[f_{1}\left(4\right)+F_{4,1}^{(1)}\right]
+f_{1}\left(1\right)\left[r_{1}\tilde{\mu}_{4}+F_{4,1}^{(1)}\right]\\
K_{15}=\tilde{\mu}_{2}^{2}\left[R_{1}^{(2)}+f_{1}\left(1\right)\tilde{\theta}_{1}^{(2)}\right]
+\tilde{P}_{2}^{(2)}\left[r_{1}+\frac{f_{1}\left(1\right)}{1-\tilde{\mu}_{1}}\right]
+2r_{1}\tilde{\mu}_{2}f_{1}\left(2\right)\\
K_{16}=\tilde{\mu}_{2}\tilde{\mu}_{3}\left[R_{1}^{(2)}
+r_{1}+f_{1}\left(1\right)\left(\tilde{\theta}_{1}^{(2)}+\frac{1}{1-\tilde{\mu}_{1}}\right)\right]
+r_{1}\tilde{\mu}_{2}\left[f_{1}\left(3\right)+F_{3,1}^{(1)}\right]
+f_{1}\left(2\right)\left[r_{1}\tilde{\mu}_{3}+F_{3,1}^{(1)}\right]\\
K_{17}=\tilde{\mu}_{2}\tilde{\mu}_{4}\left[R_{1}^{(2)}+r_{1}
+f_{1}\left(1\right)\left(\tilde{\theta}_{1}^{(2)}+\frac{1}{1-\tilde{\mu}_{1}}\right)\right]+r_{1}\tilde{\mu}_{2}\left[f_{1}\left(4\right)
+\tilde{\mu}_{2}F_{4,1}^{(1)}\right]
+f_{1}\left(2\right)\left[r_{1}\tilde{\mu}_{4}
+F_{4,1}^{(1)}\right]\\
K_{18}=\tilde{\mu}_{3}^{2}\left[R_{1}^{(2)}
+f_{1}\left(1\right)\tilde{\theta}_{1}^{(2)}\right]
+\tilde{P}_{3}^{(2)}\left[r_{1}+\frac{f_{1}\left(1\right)}{1-\tilde{\mu}_{1}}\right]
+2r_{1}\tilde{\mu}_{3}\left[f_{1}\left(3\right)+F_{3,1}^{(1)}\right]
+F_{3,1}^{(2)}+2F_{3,1}^{(1)}f_{1}\left(3\right)\\
K_{19}=\tilde{\mu}_{3}\tilde{\mu}_{4}\left[R_{1}^{(2)}+r_{1}
+f_{1}\left(1\right)\left(\tilde{\theta}_{1}^{2}
+\frac{1}{1-\tilde{\mu}_{1}}\right)\right]
+r_{1}\tilde{\mu}_{3}\left[f_{1}\left(4\right)+F_{4,1}^{(1)}\right]
+f_{1}\left(3\right)\left[r_{1}\tilde{\mu}_{4}+F_{4,1}^{(1)}\right]\\
+F_{3,1}^{(1)}\left[r_{1}\tilde{\mu}_{4}+F_{4,1}^{(1)}+f_{1}\left(4\right)\right]\\
K_{20}=\tilde{\mu}_{4}^{2}\left[R_{1}^{(2)}+f_{1}\left(1\right)\tilde{\theta}_{1}^{(2)}\right]
+\tilde{P}_{4}^{(2)}\left[r_{1}+\frac{f_{1}\left(1\right)}{1-\tilde{\mu}_{1}}\right]
+f_{1}\left(4\right)\left[2r_{1}\tilde{\mu}_{4}+2F_{4,1}^{(1)}\right]
+F_{4,1}^{(2)}+2F_{4,1}^{(1)}r_{1}\tilde{\mu}_{4}\\
\end{array}
\end{eqnarray}

\begin{eqnarray}\label{Coefficients.kis.Exh.tres}
\begin{array}{l}
K_{21}=\tilde{\mu}_{1}^{2}\left[R_{2}^{(2)}+f_{4}\left(4\right)\tilde{\theta}_{4}^{(2)}\right]
+2r_{4}\tilde{\mu}_{1}\left[F_{1,4}^{(1)}+f_{4}\left(1\right)\right]+\tilde{P}_{1}^{(2)}\left[r_{4}++\frac{f_{4}\left(4\right)}{1-\tilde{\mu}_{2}}\right]
+\left[F_{1,4}^{(2)}+2f_{4}\left(1\right)F_{1,4}^{(1)}\right]\\
K_{22}=\tilde{\mu}_{1}\tilde{\mu}_{2}\left[
R_{4}^{(2)}+r_{4}+f_{4}\left(4\right)\left(\tilde{\theta}_{4}^{(2)}+\frac{1}{1-\tilde{\mu}_{2}}\right)\right]+r_{4}\tilde{\mu}_{1} \left(F_{2,4}^{(1)}+f_{4}\left(2\right)\right)+r_{4}\tilde{\mu}_{2}\left(f_{4}\left(1\right)+F_{1,4}^{(1)}\right)\\
+\left[f_{4}\left(2\right)F_{1,4}^{(1)}
+f_{4}\left(1\right)F_{2,4}^{(1)}+F_{2,4}^{(1)}F_{1,4}^{(1)}\right]\\
K_{23}=\tilde{\mu}_{1}\tilde{\mu}_{3}\left[R_{4}^{(2)}+r_{4}+f_{4}\left(4\right)\left(\tilde{\theta}_{4}^{(2)}
+\frac{1}{1-\tilde{\mu}_{4}}\right)\right]+\tilde{\mu}_{3}\left[r_{4}\left(f_{4}\left(1\right)
+F_{1,4}^{(1)}\right)+r_{3}F_{1,4}^{(1)}\right]+r_{4}\tilde{\mu}_{1}f_{4}\left(3\right)\\
K_{24}=\tilde{\mu}_{1}\tilde{\mu}_{4}\left(
R_{4}^{(2)}+r_{4}\right)
+r_{4}\left[\tilde{\mu}_{1}f_{4}\left(4\right)
+\tilde{\mu}_{4}\left(f_{4}\left(1\right)+F_{1,4}^{(1)}
\right)\right]
+f_{4}\left(4\right)F_{1,4}^{(1)}\\
K_{25}=\tilde{\mu}_{2}^{2}\left[R_{4}^{(2)}+f_{4}\left(4\right)\tilde{\theta}_{4}^{(2)}\right]
+2r_{4}\tilde{\mu}_{2}\left[F_{2,4}^{(1)}
+f_{4}\left(2\right)\right]
+\tilde{P}_{2}^{(2)}\left[\frac{f_{4}\left(4\right)}{1-\tilde{\mu}_{4}}
+r_{4}\right]
+\left[2f_{4}\left(2\right)F_{2,4}^{(1)}
+F_{2,4}^{(2)}\right]\\
K_{26}=\tilde{\mu}_{2}\tilde{\mu}_{3}\left[
R_{4}^{(2)}
+r_{4}
+f_{4}\left(4\right)\left(\tilde{\theta}_{4}^{(2)}
+\frac{1}{1-\tilde{\mu}_{4}}\right)\right]
+r_{4}\tilde{\mu}_{3}\left[F_{2,4}^{(1)}
+f_{4}\left(2\right)\right]
+\left[r_{4}\tilde{\mu}_{2}
+F_{2,4}^{(1)}\right]f_{4}\left(3\right)\\
K_{27}=\tilde{\mu}_{2}\tilde{\mu}_{4}\left[
R_{4}^{(2)}+r_{4}\right]+r_{4}\tilde{\mu}_{4}\left[f_{4}\left(4\right)
+F_{2,4}^{(2)}\right]+\left[r_{4}\tilde{\mu}_{2}+F_{2,4}^{(2)}\right]f_{4}\left(4\right)\\
K_{28}=\tilde{\mu}_{3}^{2}\left[R_{4}^{(2)}
+f_{4}\left(4\right)\tilde{\theta}_{4}^{(2)}\right]
+\tilde{P}_{3}^{(2)}\left[r_{4}+\frac{f_{4}\left(4\right)}{1-\tilde{\mu}_{4}}\right]
+2r_{4}\tilde{\mu}_{3}f_{4}\left(4\right)\\
K_{29}=\tilde{\mu}_{3}\tilde{\mu}_{4}\left[R_{4}^{(2)}+r_{4}\right]+r_{4}\left[\tilde{\mu}_{3}f_{4}\left(4\right)
+\tilde{\mu}_{4}f_{4}\left(3\right)\right]\\
K_{30}=R_{4}^{(2)}\tilde{\mu}_{4}^{2}+r_{4}\tilde{P}_{4}^{(2)}+2r_{4}\tilde{\mu}_{4}f_{4}\left(4\right)\\
\end{array}
\end{eqnarray}

\begin{eqnarray}\label{Coefficients.kis.Exh.cuatro}
\begin{array}{l}
K_{31}=\tilde{\mu}_{1}^{2}\left[R_{3}^{(2)}
+\tilde{\theta}_{3}^{(2)}f_{3}\left(3\right)\right]
+\tilde{P}_{2}^{(2)}\left[r_{3}+\frac{f_{3}\left(3\right)}{1-\tilde{\mu}_{3}}\right]
+2r_{3}\tilde{\mu}_{1}\left[F_{1,3}^{(1)}
+f_{3}\left(1\right)\right]
+\left[2F_{1,3}^{(1)}f_{3}\left(1\right)+F_{1,3}^{(2)}\right]\\
K_{32}=\tilde{\mu}_{1}\tilde{\mu}_{2}\left[
R_{3}^{(2)}+r_{3}+\tilde{\theta}_{3}^{(2)}f_{3}\left(3\right)
+\frac{1}{1-\tilde{\mu}_{3}}f_{3}\left(3\right)\right]
+r_{3}\tilde{\mu}_{1}\left[f_{3}\left(2\right)+F_{2,3}^{(1)}\right]
+f_{3}\left(1\right)\left[F_{2,3}^{(1)}+r_{3}\tilde{\mu}_{2}\right]\\
+F_{1,3}^{(1)}\left[r_{3}\tilde{\mu}_{2}+f_{3}\left(2\right)\right]
+F_{2,3}^{(1)}F_{1,3}^{(1)}\\
K_{33}=\tilde{\mu}_{1}\tilde{\mu}_{3}\left[R_{3}^{(2)}
+r_{3}\right]
+r_{3}\tilde{\mu}_{3}\left[f_{3}\left(1\right)
+F_{1,3}^{(1)}\right]
+f_{3}\left(3\right)\left[r_{3}\tilde{\mu}_{1}+F_{1,3}^{(1)}\right]\\
K_{34}=\tilde{\mu}_{1}\tilde{\mu}_{4}\left[f_{3}\left(3\right)\left(\tilde{\theta}_{3}^{(2)}+\frac{1}{1-\tilde{\mu}_{3}}\right)
+r_{3}+R_{3}^{(2)}\right]
+r_{3}\tilde{\mu}_{4}\left[f_{3}\left(3\right)+F_{1,3}^{(1)}\right]
+f_{3}\left(4\right)\left[r_{3}\tilde{\mu}_{1}+F_{1,3}^{(1)}\right]\\
K_{35}=\tilde{\mu}_{2}^{2}\left[R_{3}^{(2)}
+f_{3}\left(3\right)\tilde{\theta}_{3}^{(2)}\right]+2r_{3}\tilde{\mu}_{2}\left[f_{3}\left(2\right)+F_{2,3}^{(1)}\right]
+\tilde{P}_{2}^{(2)}\left[f_{3}\left(3\right)\frac{1}{1-\tilde{\mu}_{3}}
+r_{3}\right]+\left[F_{2,3}^{(2)}
+2f_{3}\left(2\right)F_{2,3}^{(1)}\right]\\
K_{36}=\tilde{\mu}_{2}\tilde{\mu}_{3}\left[R_{3}^{(2)}+r_{3}\right]
+r_{3}\tilde{\mu}_{3}\left[f_{3}\left(2\right)+F_{2,3}^{(1)}\right]
+\left[r_{3}\tilde{\mu}_{2}+F_{2,3}^{(1)}\right]f_{3}\left(3\right)\\
K_{37}=\tilde{\mu}_{2}\tilde{\mu}_{4}\left[R_{3}^{(2)}
+r_{3}+f_{3}\left(3\right)\left(\tilde{\theta}_{3}^{(2)}
+\frac{1}{1-\tilde{\mu}_{3}}\right)\right]
+r_{3}\tilde{\mu}_{4}\left[f_{3}\left(2\right)
+F_{2,3}^{(1)}\right]+\left[r_{3}\tilde{\mu}_{2}+F_{2,3}^{(1)}\right]f_{3}\left(4\right)\\
K_{38}=R_{3}^{(2)}\tilde{\mu}_{3}^{2}+r_{3}\tilde{P}_{3}^{(2)}
+2r_{3}\tilde{\mu}_{3}f_{3}\left(3\right)\\
K_{39}=\tilde{\mu}_{3}\tilde{\mu}_{4}\left[R_{3}^{(2)}+r_{3}\right]
+r_{3}\left[\tilde{\mu}_{3}f_{3}\left(4\right)
+\tilde{\mu}_{4}f_{3}\left(3\right)\right]\\
K_{40}=\tilde{\mu}_{4}^{2}\left[R_{3}^{(2)}
+f_{3}\left(3\right)\tilde{\theta}_{3}^{(2)}\right]
+\tilde{P}_{4}^{(2)}\left[f_{3}\left(3\right)\frac{1}{1-\tilde{\mu}_{3}}+r_{3}\right]
+2r_{3}\tilde{\mu}_{4}f_{3}\left(4\right)
\end{array}
\end{eqnarray}
%\newpage

%______________________________________________________________________
\section{Appendix C: General Case Calculations Gated Policy}
%______________________________________________________________________


De acuerdo a la pol\'itica de servicio Cerrada, el n\'umero de usuarios presentes en la cola al momento en que el servidor termina de atender a todos los que estaban presentes cuando este llega para dar servicio, est\'a dada de la siguiente manera


Para cada una de las colas en cada sistema, el n\'umero de
usuarios al tiempo en que llega el servidor a dar servicio est\'a
dado por el n\'umero de usuarios presentes en la cola al tiempo
$t=\tau_{i},\zeta_{i}$, m\'as el n\'umero de usuarios que llegaron a
la cola en el intervalo de tiempo
$\left[\tau_{i},\overline{\tau}_{i}\right],\left[\zeta_{i},\overline{\zeta}_{i}\right]$,
es decir

Just like before we have that

\begin{eqnarray}%\label{Eq.TiemposLlegada.Cerrada}
L_{i}\left(\overline{\tau}_{1}\right)&=&L_{i}\left(\tau_{1}\right)+X_{i}\left(\overline{\tau}_{1}-\tau_{1}\right)+Y_{i}\left(\overline{\tau}_{1}-\tau_{1}\right)
\end{eqnarray}

Then at the moment the server ends attending the users in the queue at the moment the server arrives, so the number of users at the queue during the service time $\overline{\tau}_{1}-\tau_{1}$, so we have that


\begin{eqnarray*}
&&\esp\left[z_{1}^{L_{1}\left(\overline{\tau}_{1}\right)}z_{2}^{L_{2}\left(\overline{\tau}_{1}\right)}z_{3}^{L_{3}\left(\overline{\tau}_{1}\right)}z_{4}^{L_{4}\left(\overline{\tau}_{1}\right)}\right]
=\esp\left[z_{1}^{X_{1}\left(\overline{\tau}_{1}-\tau_{1}\right)}z_{2}^{L_{2}\left(\tau_{1}\right)+X_{2}\left(\overline{\tau}_{1}-\tau_{1}\right)+Y_{2}\left(\overline{\tau}_{1}-\tau_{1}\right)}z_{3}^{L_{3}\left(\tau_{1}\right)+X_{3}\left(\overline{\tau}_{1}-\tau_{1}\right)}z_{4}^{L_{4}\left(\tau_{1}\right)+X_{4}\left(\overline{\tau}_{1}-\tau_{1}\right)}\right]\\
&=&\esp\left[z_{1}^{X_{1}\left(\overline{\tau}_{1}-\tau_{1}\right)}z_{2}^{L_{2}\left(\tau_{1}\right)}z_{2}^{X_{2}\left(\overline{\tau}_{1}-\tau_{1}\right)+Y_{2}\left(\overline{\tau}_{1}-\tau_{1}\right)}z_{3}^{\hat{L}_{1}\left(\tau_{1}\right)}z_{3}^{X_{3}\left(\overline{\tau}_{1}-\tau_{1}\right)}z_{4}^{L_{4}\left(\tau_{1}\right)}z_{4}^{X_{4}\left(\overline{\tau}_{1}-\tau_{1}\right)}\right]\\
&=&\esp\left[z_{1}^{X_{1}\left(\overline{\tau}_{1}-\tau_{1}\right)}z_{2}^{L_{2}\left(\tau_{1}\right)}z_{2}^{\tilde{X}_{2}\left(\overline{\tau}_{1}-\tau_{1}\right)}z_{3}^{L_{3}\left(\tau_{1}\right)}z_{3}^{X_{3}\left(\overline{\tau}_{1}-\tau_{1}\right)}z_{4}^{L_{4}\left(\tau_{1}\right)}z_{4}^{X_{4}\left(\overline{\tau}_{1}-\tau_{1}\right)}\right]\\
&=&\esp\left[\left\{z_{2}^{L_{2}\left(\tau_{1}\right)}
z_{3}^{L_{3}\left(\tau_{1}\right)}
z_{4}^{L_{4}\left(\tau_{1}\right)}\right\}
\left\{z_{1}^{\tilde{X}_{1}\left(\overline{\tau}_{1}-\tau_{1}\right)}
z_{2}^{\tilde{X}_{2}\left(\overline{\tau}_{1}-\tau_{1}\right)}
z_{3}^{\tilde{X}_{3}\left(\overline{\tau}_{1}-\tau_{1}\right)}
z_{4}^{\tilde{X}_{4}\left(\overline{\tau}_{1}-\tau_{1}\right)}\right\}\right]\\
&=&\esp\left[\left\{z_{2}^{L_{2}\left(\tau_{1}\right)}
z_{3}^{L_{3}\left(\tau_{1}\right)}
z_{4}^{L_{4}\left(\tau_{1}\right)}\right\}
\left\{\left\{\tilde{P}_{1}\left(z_{1}\right)\right\}^{\overline{\tau}_{1}-\tau_{1}}
\tilde{P}_{2}\left(z_{2}\right\}^{\overline{\tau}_{1}-\tau_{1}}
\tilde{P}_{3}\left(z_{3}\right\}^{\overline{\tau}_{1}-\tau_{1}}
\tilde{P}_{4}\left(z_{4}\right\}^{\overline{\tau}_{1}-\tau_{1}}\right\}\right]\\
&=&\esp\left[\left\{z_{2}^{L_{2}\left(\tau_{1}\right)}
z_{3}^{L_{3}\left(\tau_{1}\right)}
z_{4}^{L_{4}\left(\tau_{1}\right)}\right\}
\left\{\tilde{P}_{1}\left(z_{1}\right)
\tilde{P}_{2}\left(z_{2}\right)
\tilde{P}_{3}\left(z_{3}\right)
\tilde{P}_{4}\left(z_{4}\right)\right\}^{\overline{\tau}_{1}-\tau_{1}}\right]\\
&=&\esp\left[\left\{z_{2}^{L_{2}\left(\tau_{1}\right)}
z_{3}^{L_{3}\left(\tau_{1}\right)}
z_{4}^{L_{4}\left(\tau_{1}\right)}\right\}
\left\{\tilde{P}_{1}\left(z_{1}\right)
\tilde{P}_{2}\left(z_{2}\right)
\tilde{P}_{3}\left(z_{3}\right)
\tilde{P}_{4}\left(z_{4}\right)\right\}^{L_{1}\left(\tau_{1}\right)}\right]\\
&=&F_{1}\left(\tilde{P}_{1}\left(z_{1}\right)
\tilde{P}_{2}\left(z_{2}\right)
\tilde{P}_{3}\left(z_{3}\right)
\tilde{P}_{4}\left(z_{4}\right),z_{2},z_{3},z_{4}\right]\\
&=&F_{1}\left(\prod_{i=1}^{4}\tilde{P}_{i}\left(z_{i}\right),z_{2},z_{3},z_{4}\right)
\end{eqnarray*}

In an analogous manner we have for the rest of the queues that conform the NCPS:

\begin{eqnarray*}
\esp\left[z_{1}^{L_{1}\left(\overline{\tau}_{2}\right)}z_{2}^{L_{2}\left(\overline{\tau}_{2}\right)}z_{3}^{L_{3}\left(\overline{\tau}_{2}\right)}z_{4}^{L_{4}\left(\overline{\tau}_{2}\right)}\right]
&=&F_{2}\left(z_{1},\prod_{i=1}^{4}\tilde{P}_{i}\left(z_{i}\right),z_{3},z_{4}\right)\\
\esp\left[z_{1}^{L_{1}\left(\overline{\tau}_{3}\right)}z_{2}^{L_{2}\left(\overline{\tau}_{3}\right)}z_{3}^{L_{3}\left(\overline{\tau}_{3}\right)}z_{4}^{L_{4}\left(\overline{\tau}_{3}\right)}\right]
&=&F_{3}\left(z_{1},z_{2},\prod_{i=1}^{4}\tilde{P}_{i}\left(z_{i}\right),z_{4}\right)\\
\esp\left[z_{1}^{L_{1}\left(\overline{\tau}_{4}\right)}z_{2}^{L_{2}\left(\overline{\tau}_{4}\right)}z_{3}^{L_{3}\left(\overline{\tau}_{4}\right)}z_{4}^{L_{4}\left(\overline{\tau}_{4}\right)}\right]
&=&F_{4}\left(z_{1},z_{2},z_{3},\prod_{i=1}^{4}\tilde{P}_{i}\left(z_{i}\right)\right)
\end{eqnarray*}


therefore, the recursive equations are of the form


\begin{eqnarray}%\label{Ec.Recursivas.Gated}
\begin{array}{l}
F_{1}\left(z_{1},z_{2},z_{3},z_{4}\right)=R_{2}\left(\prod_{i=1}^{4}\tilde{P}_{i}\left(z_{i}\right)\right)F_{2}\left(z_{1},\prod_{i=1}^{4}\tilde{P}_{i}\left(z_{i}\right),z_{3},z_{4}\right)\\
F_{2}\left(z_{1},z_{2},z_{3},z_{4}\right)=R_{1}\left(\prod_{i=1}^{4}\tilde{P}_{i}\left(z_{i}\right)\right)F_{1}\left(\prod_{i=1}^{4}\tilde{P}_{i}\left(z_{i}\right),z_{2},z_{3},z_{4}\right)\\
F_{3}\left(z_{1},z_{2},z_{3},z_{4}\right)=R_{4}\left(\prod_{i=1}^{4}\tilde{P}_{i}\left(z_{i}\right)\right)F_{4}\left(z_{1},z_{2},z_{3},\prod_{i=1}^{4}\tilde{P}_{i}\left(z_{i}\right)\right)\\
F_{4}\left(z_{1},z_{2},z_{3},z_{4}\right)=R_{3}\left(\prod_{i=1}^{4}\tilde{P}_{i}\left(z_{i}\right)\right)F_{3}\left(z_{1},z_{2},\prod_{i=1}^{4}\tilde{P}_{i}\left(z_{i}\right),z_{4}\right)
\end{array}
\end{eqnarray}

So conforming with the developed and proceeding in a similar manner for the rest of the queues, we can see that the following can be obtained

\begin{eqnarray*}%\label{Sist.Ec.Lineales.Gated}
\begin{array}{lll}
f_{1}\left(1\right)=r_{2}\tilde{\mu}_{1}+f_{2}\left(2\right)\tilde{\mu}_{1}+f_{2}\left(1\right),&
f_{1}\left(2\right)=r_{2}\tilde{\mu}_{2}+f_{2}\left(2\right)\tilde{\mu}_{2},&
f_{1}\left(3\right)=r_{2}\tilde{\mu}_{3}+f_{2}\left(2\right)\tilde{\mu}_{3},\\
f_{1}\left(4\right)=r_{2}\tilde{\mu}_{4}+f_{2}\left(2\right)\tilde{\mu}_{4},&
f_{2}\left(1\right)=r_{1}\tilde{\mu}_{1}+f_{1}\left(1\right)\tilde{\mu}_{1},&
f_{2}\left(2\right)=r_{1}\tilde{\mu}_{2}+f_{1}\left(1\right)\tilde{\mu}_{2}+f_{1}\left(2\right)\\
f_{2}\left(3\right)=r_{1}\tilde{\mu}_{3}+f_{1}\left(1\right)\tilde{\mu}_{3},&
f_{2}\left(4\right)=r_{1}\tilde{\mu}_{4}+f_{1}\left(1\right)\tilde{\mu}_{4},&
f_{3}\left(1\right)=r_{4}\tilde{\mu}_{1}+f_{4}\left(4\right)\tilde{\mu}_{1},\\
f_{3}\left(2\right)=r_{4}\tilde{\mu}_{2}+f_{4}\left(4\right)\tilde{\mu}_{2},&
f_{3}\left(3\right)=r_{4}\tilde{\mu}_{3}+f_{4}\left(4\right)\tilde{\mu}_{3}+f_{4}\left(3\right),&
f_{3}\left(4\right)=r_{4}\tilde{\mu}_{4}+f_{4}\left(4\right)\tilde{\mu}_{4},\\
f_{4}\left(1\right)=r_{3}\tilde{\mu}_{1}+f_{3}\left(3\right)\tilde{\mu}_{1},&
f_{4}\left(2\right)=r_{3}\tilde{\mu}_{2}+f_{3}\left(3\right)\tilde{\mu}_{2},&
f_{4}\left(3\right)=r_{3}\tilde{\mu}_{3}+f_{3}\left(3\right)\tilde{\mu}_{3},\\
&f_{4}\left(4\right)=r_{3}\tilde{\mu}_{4}+f_{3}\left(3\right)\tilde{\mu}_{4}+f_{3}\left(4\right).&
\end{array}
\end{eqnarray*}

whose solutions are
\begin{eqnarray*}
\begin{array}{llll}
f_{1}\left(1\right)=\frac{r\mu_{1}}{1-\mu},&
f_{1}\left(2\right)=\frac{\tilde{\mu}_{2}\left(r_{2}\left(1-\mu_{1}\right)+r_{1}\tilde{\mu}_{2}\right)}{1-\mu},&
f_{1}\left(3\right)=\frac{\mu_{3}\left(r_{2}\left(1-\mu_{1}\right)+r_{1}\tilde{\mu}_{2}\right)}{1-\mu},&
f_{1}\left(4\right)=\frac{\mu_{4}\left(r_{2}\left(1-\mu_{1}\right)+r_{1}\tilde{\mu}_{2}\right)}{1-\mu},\\
f_{2}\left(1\right)=\frac{\mu_{1}\left(r_{1}\left(1-\tilde{\mu}_{2}\right)+r_{2}\mu_{1}\right)}{1-\mu},&
f_{2}\left(2\right)=\frac{r\tilde{\mu}_{2}}{1-\mu},&
f_{2}\left(3\right)=\frac{\mu_{3}\left(r_{1}\left(1-\tilde{\mu}_{2}\right)+r_{2}\mu_{1}\right)}{1-\mu},&
f_{2}\left(4\right)=\frac{\mu_{4}\left(r_{1}\left(1-\tilde{\mu}_{2}\right)+r_{2}\mu_{1}\right)}{1-\mu},\\
f_{3}\left(1\right)=\frac{\mu_{1}\left(r_{4}\left(1-\mu_{3}\right)+r_{3}\mu_{4}\right)}{1-\hat{\mu}},&
f_{3}\left(2\right)=\frac{\tilde{\mu}_{2}\left(r_{4}\left(1-\mu_{3}\right)+r_{3}\mu_{4}\right)}{1-\hat{\mu}},&
f_{3}\left(3\right)=\frac{\hat{r}\mu_{3}}{1-\hat{\mu}},&
f_{3}\left(4\right)=\frac{\mu_{4}\left(r_{4}\left(1-\mu_{3}\right)+r_{3}\mu_{4}\right)}{1-\hat{\mu}},\\
f_{4}\left(1\right)=\frac{\mu_{1}\left(r_{3}\left(1-\mu_{4}\right)+r_{4}\mu_{3}\right)}{1-\hat{\mu}},&
f_{4}\left(2\right)=\frac{\tilde{\mu}_{2}\left(r_{3}\left(1-\mu_{4}\right)+r_{4}\mu_{3}\right)}{1-\hat{\mu}},&
f_{4}\left(3\right)=\frac{\hat{r}\mu_{4}}{1-\hat{\mu}},&
f_{4}\left(4\right)=\frac{\mu_{3}\left(r_{3}\left(1-\mu_{4}\right)+r_{4}\mu_{3}\right)}{1-\hat{\mu}},
\end{array}
\end{eqnarray*}

Also, according with the theorem (\ref{Eq.Gral.Second.Order.Exhaustive}) we have that for the gated policy $f_{1}\left(1,1\right)=f_{2}\left(1,1\right)
+2\tilde{\mu}_{1}f_{2}\left(1,2\right)
+\tilde{\mu}_{1}^{2}f_{2}\left(2,2\right)
+P_{1}^{(2)}\left[r_{2}
+f_{2}\left(2\right)\right]
+2r_{2}\tilde{\mu}_{1}f_{2}\left(1\right)
R_{2}^{(2)}\tilde{\mu}_{1}$, in general


{\small{
\begin{eqnarray}\label{Eq.Sdo.Orden.Gated}
\begin{array}{l}
f_{1}\left(i,k\right)=
\indora_{k=1}\indora_{i=k}\tilde{\mu}_{i}f_{2}\left(1,1\right)
+\left[\indora_{k=1}\tilde{\mu}_{1}+\indora_{i=1}\tilde{\mu}_{k}\right]f_{2}\left(1,2\right)
+\tilde{\mu}_{i}\tilde{\mu}_{k}f_{2}\left(2,2\right)
+\left[\indora_{i=k}\tilde{P}_{i}^{(2)}
+\indora_{i\neq k}\tilde{\mu}_{i}\tilde{\mu}_{k}\right]f_{2}\left(2\right)\\
+\left[r_{2}\tilde{\mu}_{i}+\indora_{i\geq3}F_{i,2}^{(1)}\right]f_{2}\left(k\right)
+\left[r_{2}\tilde{\mu}_{k}+\indora_{k\geq3}F_{k,2}^{(1)}\right]f_{2}\left(i\right)
+\left[R_{2}^{(2)}+\indora_{i=k}r_{2}\right]\tilde{\mu}_{i}\tilde{\mu}_{k}\\
+\left[\indora_{k\geq3}\tilde{\mu}_{i}F_{k,2}^{(1)}+\indora_{i=k}P_{i}^{(2)}\right]r_{2}
+\left[\indora_{i\geq3}\indora_{k\neq i}F_{k,2}^{(1)}+\indora_{i\geq3}r_{2}\tilde{\mu}_{k}\right]F_{i,2}^{(1)}
+\indora_{i\geq3}\indora_{k=i}F_{i,2}^{(2)}\\
f_{2}\left(i,k\right)=\tilde{\mu}_{i}\tilde{\mu}_{k}f_{1}\left(1,1\right)
+\left[\indora_{k=2}\tilde{\mu}_{i}
+\indora_{i=2}\tilde{\mu}_{k}\right]f_{1}\left(1,2\right)
+\indora_{k=2}\indora_{i=k}\tilde{\mu}_{i}f_{1}\left(2,2\right)
+\left[\indora_{i=k}\tilde{P}_{i}^{(2)}
+\indora_{i\neq k}\tilde{\mu}_{i}\tilde{\mu}_{k}\right]f_{1}\left(1\right)\\
+\left[r_{1}\tilde{\mu}_{i}+\indora_{i\geq3}F_{i,1}^{(1)}\right]f_{1}\left(k\right)
+\left[r_{1}\tilde{\mu}_{k}+\indora_{k\geq3}F_{k,1}^{(1)}\right]f_{1}\left(i\right)
+\left[R_{1}^{(2)}+\indora_{i=k}r_{1}\right]\tilde{\mu}_{i}\tilde{\mu}_{k}\\
+\left[\indora_{i\geq3}\indora_{k\neq i}F_{i,1}^{(1)}+\indora_{k\geq3}r_{1}\tilde{\mu}_{i}\right]F_{k,1}^{(1)}
+\left[\indora_{i=k}P_{i}^{(2)}+\indora_{i\geq3}F_{i,1}^{(1)}\tilde{\mu}_{k}\right]r_{1}
+\indora_{i\geq3}\indora_{k=i}F_{i,1}^{(2)}\\
f_{3}\left(i,k\right)=\indora_{k=3}\indora_{i=k}\tilde{\mu}_{i}f_{4}\left(3,3\right)
+\left[\indora_{k=3}\tilde{\mu}_{i}+\indora_{i=3}\tilde{\mu}_{k}\right]f_{4}\left(3,4\right)
+\tilde{\mu}_{i}\tilde{\mu}_{k}f_{4}\left(4,4\right)
+\left[\indora_{i=k}\tilde{P}_{i}^{(2)}+\indora_{i\neq k}\tilde{\mu}_{i}\tilde{\mu}_{k}\right]f_{4}\left(4\right)\\
+\left[r_{4}\tilde{\mu}_{i}+\indora_{i\leq2}F_{i,4}^{(1)}\right]f_{4}\left(k\right)
+\left[r_{4}\tilde{\mu}_{k}+\indora_{k\leq2}F_{k,4}^{(1)}\right]f_{4}\left(i\right)
+\left[R_{4}^{(2)}+\indora_{i=k}r_{4}\right]\tilde{\mu}_{i}\tilde{\mu}_{k}\\
+\left[\indora_{i=k}P_{i}^{(2)}+\indora_{k\leq2}\tilde{\mu}_{i}F_{k,4}^{(1)}\right]r_{4}
+\left[\indora_{i\leq2}\indora_{k\neq i}F_{k,4}^{(1)}+\indora_{i\leq2}r_{4}\tilde{\mu}_{k}\right]F_{i,4}^{(1)}
+\indora_{i\leq2}\indora_{k=i}F_{i,4}^{(2)}\\
f_{4}\left(i,k\right)=\tilde{\mu}_{i}\tilde{\mu}_{k}f_{3}\left(3,3\right)
+\left[\indora_{k=4}\tilde{\mu}_{i}+\indora_{i=4}\tilde{\mu}_{k}\right]f_{3}\left(3,4\right)
+\indora_{k=4}\indora_{i=k}\tilde{\mu}_{i}f_{3}\left(4,4\right)
+\left[\indora_{i=k}\tilde{P}_{i}^{(2)}
+\indora_{i\neq k}\tilde{\mu}_{i}\tilde{\mu}_{k}\right]f_{3}\left(3\right)\\
+\left[r_{3}\tilde{\mu}_{i}+\indora_{i\leq2}F_{i,3}^{(1)}\right]f_{3}\left(k\right)
+\left[r_{3}\tilde{\mu}_{k}+\indora_{k\leq2}F_{k,3}^{(1)}\right]f_{3}\left(i\right)
+\left[R_{3}^{(2)}+\indora_{i=k}r_{3}\right]\tilde{\mu}_{i}\tilde{\mu}_{k}\\
+\left[\indora_{i\leq2}\indora_{k\neq i}F_{k,3}^{(1)}+\indora_{i\leq2}r_{3}\tilde{\mu}_{k}\right]F_{i,3}^{(1)}
+\left[\indora_{k\leq2}\tilde{\mu}_{i}F_{k,3}^{(1)}+\indora_{i=k}P_{i}^{(2)}\right]r_{3}
+\indora_{i\leq2}\indora_{k=i}F_{i,3}^{(2)}
\end{array}
\end{eqnarray}}}

So, the linear system equatiosn is given by


\begin{eqnarray*}
\begin{array}{ll}
f_{1}\left(1,1\right)=a_{1}f_{2}\left(1,1\right)
+a_{2}f_{2}\left(1,2\right)
+a_{3}f_{2}\left(2,2\right)
+K_{1},&
f_{1}\left(1,2\right)=a_{4}f_{2}\left(1,2\right)+a_{5}f_{1}\left(2,2\right)+K_{2},\\
f_{1}\left(1,3\right)=a_{6}f_{2}\left(2,1\right)
+a_{7}f_{2}\left(2,2\right)
+K_{3},&
f_{1}\left(1,4\right)=a_{8}f_{2}\left(2,1\right)+a_{9}f_{2}\left(2,2\right)
+K_{4},\\
f_{1}\left(2,2\right)=a_{10}f_{2}\left(2,2\right)+K_{5},&
f_{1}\left(2,3\right)=a_{11}f_{2}\left(2,2\right)+K_{6},\\
f_{2}\left(2,4\right)=a_{12}f_{2}\left(2,2\right)+K_{7},&
f_{1}\left(3,3\right)=a_{13}f_{2}\left(2,2\right)+K_{8},\\
f_{1}\left(3,4\right)=a_{14}f_{2}\left(2,2\right)+K_{9},&
f_{1}\left(4,4\right)=a_{15}f_{2}\left(2,2\right)+K_{10}\\
f_{2}\left(1,1\right)=a_{16}f_{1}\left(1,1\right)+K_{11},&
f_{2}\left(1,2\right)=a_{17}f_{1}\left(1,1\right)+a_{18}f_{1}\left(1,2\right)+K_{12},\\
f_{2}\left(1,3\right)=a_{19}f_{1}\left(1,1\right)+K_{13},&
f_{2}\left(1,4\right)=a_{20}f_{1}\left(1,1\right)+K_{14},\\
f_{2}\left(2,2\right)=a_{21}f_{1}\left(1,1\right)+a_{22}f_{1}\left(1,2\right)+a_{23}f_{1}\left(2,2\right)+K_{15},&
f_{2}\left(2,3\right)=a_{24}f_{1}\left(1,1\right)
+a_{25}f_{1}\left(1,2\right)
+K_{16},\\
f_{2}\left(2,4\right)=a_{26}f_{1}\left(1,1\right)+a_{27}f_{1}\left(1,2\right)+K_{17},&
f_{2}\left(3,3\right)=a_{28}f_{1}\left(1,1\right)+K_{18},\\
f_{2}\left(3,4\right)=a_{29}f_{1}\left(1,1\right)+K_{19},&
f_{2}\left(4,4\right)=a_{30}f_{1}\left(1,1\right)+K_{20}
\end{array}
\end{eqnarray*}


\begin{eqnarray*}
\begin{array}{ll}
f_{3}\left(1,1\right)=a_{31}f_{4}\left(4,4\right)+K_{21},&
f_{3}\left(1,2\right)=a_{32}f_{4}\left(4,4\right)+K_{22},\\
f_{3}\left(1,3\right)=a_{33}f_{4}\left(4,4\right)+a_{34}f_{4}\left(3,4\right)+K_{23},&
f_{3}\left(1,4\right)=a_{35}f_{4}\left(4,4\right)+K_{24},\\
f_{3}\left(2,2\right)=a_{36}f_{4}\left(4,4\right)+K_{25},&
f_{3}\left(2,3\right)=a_{37}f_{4}\left(3,4\right)+a_{38}f_{4}\left(4,4\right)+K_{26},\\
f_{3}\left(2,4\right)=a_{39}f_{4}\left(4,4\right)+K_{27},&
f_{3}\left(3,3\right)=a_{39}f_{4}\left(3,3\right)+a_{40}f_{4}\left(3,4\right)+a_{41}f_{4}\left(4,4\right)+K_{28},\\
f_{3}\left(3,4\right)=a_{42}f_{4}\left(3,4\right)+a_{43}f_{4}\left(4,4\right)+K_{29},&
f_{3}\left(4,4\right)=a_{44}f_{4}\left(4,4\right)+K_{30}\\
f_{4}\left(1,1\right)=a_{45}f_{3}\left(3,3\right)+K_{31},&
f_{4}\left(1,2\right)=a_{46}f_{3}\left(3,3\right)+K_{32},\\
f_{4}\left(1,3\right)=a_{47}f_{3}\left(3,3\right)+K_{33},&
f_{4}\left(1,4\right)=a_{48}f_{3}\left(3,3\right)+a_{49}f_{3}\left(3,4\right)+K_{34},\\
f_{4}\left(2,2\right)=a_{50}f_{3}\left(3,3\right)+K_{35},&
f_{4}\left(2,3\right)=a_{51}f_{3}\left(3,3\right)+K_{36},\\
f_{4}\left(2,4\right)=a_{52}f_{3}\left(3,3\right)+a_{53}f_{3}\left(3,4\right)+K_{37},&
f_{4}\left(3,3\right)=a_{54}f_{3}\left(3,3\right)+K_{38},\\
f_{4}\left(3,4\right)=a_{55}f_{3}\left(3,3\right)+a_{56}f_{3}\left(3,4\right)+K_{39},&
f_{4}\left(4,4\right)=
a_{57}f_{3}\left(3,3\right)
+a_{58}f_{3}\left(3,4\right)
+a_{59}f_{3}\left(4,4\right)
+K_{40}
\end{array}
\end{eqnarray*}


with constants

%______________________________________________________________________
%\section{Appendix D}
%______________________________________________________________________

\begin{eqnarray}
\begin{array}{llllll}
a_{1}=1,&
a_{2}=2\mu_{1},&
a_{3}=\mu_{1}^{2},&
a_{4}=\tilde{\mu}_{2},&
a_{5}=\mu_{1}\tilde{\mu}_{2} ,&
a_{6}=\mu_{3},\\
a_{7}=\mu_{1}\mu_{3},&
a_{8}=\mu_{4},&
a_{9}=\mu_{1}\mu_{4},&
a_{10}=\tilde{\mu}_{2}^{2},&
a_{11}=\tilde{\mu}_{2}\mu_{3},&
a_{12}=\tilde{\mu}_{2}\mu_{4},\\
a_{13}=\mu_{3}^{2},&
a_{14}=\mu_{3}\mu_{4},&
a_{15}=\mu_{4}^{2},&
a_{16}=\mu_{1}^{2},&
a_{17}=\mu_{1}\tilde{\mu}_{2},&
a_{18}=\mu_{1},\\
a_{19}=\mu_{1}\mu_{3},&
a_{20}=\mu_{1}\mu_{4},&
a_{21}=\tilde{\mu}_{2}^{2},&
a_{22}=2\tilde{\mu}_{2},&
a_{23}=1,&
a_{24}=\tilde{\mu}_{2}\mu_{3},\\
a_{25}=\mu_{3},&
a_{26}=\tilde{\mu}_{2}\mu_{4},&
a_{27}=\mu_{4},&
a_{28}=\mu_{3}^{2},&
a_{29}=\mu_{3}\mu_{4},&
a_{30}=\mu_{4}^{2},\\
a_{31}=\mu_{1}^{2},&
a_{32}=\mu_{1}\tilde{\mu}_{2},&
a_{33}=\mu_{1},&
a_{34}=\mu_{1}\mu_{3},&
a_{35}=\mu_{1}\mu_{4},&
a_{36}=\tilde{\mu}_{2}^{2},\\
a_{37}=\tilde{\mu}_{2},&
a_{38}=\tilde{\mu}_{2}\mu_{3},&
a_{39}=\tilde{\mu}_{2}\mu_{4},&
a_{40}=1,&
a_{41}=2\mu_{3},&
a_{42}=\mu_{3}^{2},\\
a_{43}=\mu_{4},&
a_{44}=\mu_{3}\mu_{4},&
a_{45}=\mu_{4}^{2},&
a_{46}=\mu_{1}^{2},&
a_{47}=\mu_{1}\tilde{\mu}_{2},&
a_{48}=\mu_{1}\mu_{3},\\
a_{49}=\mu_{1}\mu_{4},&
a_{50}=\mu_{1},&
a_{51}=\tilde{\mu}_{2}^{2},&
a_{52}=\mu_{3}\tilde{\mu}_{2},&
a_{53}=\mu_{4}\tilde{\mu}_{2},&
a_{54}=\tilde{\mu}_{2},\\
a_{55}=\mu_{3}^{2},&
a_{56}=\mu_{3}\mu_{4},&
a_{57}=\mu_{3},&
a_{58}=\mu_{4}^{2},&
a_{59}=2\mu_{4},&
a_{60}=1.
\end{array}
\end{eqnarray}


\begin{eqnarray}
\begin{array}{l}
K_{1}=\tilde{P}_{1}^{(2)}\left[r_{2}
+f_{2}\left(2\right)\right]
+2r_{2}\tilde{\mu}_{1}f_{2}\left(1\right)
R_{2}^{(2)}\tilde{\mu}_{1}\\
K_{2}=R_{2}^{(2)}\tilde{\mu}_{1}\tilde{\mu}_{2}
+r_{2}\tilde{\mu}_{1}\tilde{\mu}_{2}
+f_{2}\left(2\right)\tilde{\mu}_{2}\mu_{1}
+r_{2}\tilde{\mu}_{1}f_{2}\left(2\right)
+r_{2}\tilde{\mu}_{2}f_{2}\left(1\right)\\
K_{3}=\mu_{1}\mu_{3}\left[R_{2}^{(2)}
+r_{2}
+f_{2}\left(2\right)\right]
+r_{2}\mu_{1}\left[f_{2}\left(3\right)
+F_{3,2}^{(1)}\right]
+f_{2}\left(1\right)\left[r_{2}\mu_{3}
+F_{3,2}^{(1)}\right]\\
K_{4}=\tilde{\mu}_{1}\tilde{\mu}_{4}\left[R_{2}^{(2)}
+r_{2}
+f_{2}\left(2\right)\right]
+r_{2}\tilde{\mu}_{1}\left[f_{2}\left(4\right)
+F_{4,2}^{(1)}\right]
+f_{2}\left(1\right)\left[r_{2}\tilde{\mu}_{4}
+F_{4,2}^{(1)}\right]\\
K_{5}=\tilde{P}_{2}^{(2)}\left[r_{2}+f_{2}\left(2\right)\right]
+\tilde{\mu}_{2}\left[R_{2}^{(2)}\tilde{\mu}_{2}
+2r_{2}f_{2}\left(2\right)\right]\\
K_{6}=\tilde{\mu}_{2}\mu_{3}\left[R_{2}^{(2)}
+r_{2}
+f_{2}\left(2\right)\right]
+r_{2}\tilde{\mu}_{2}\left[f_{2}\left(3\right)
+F_{3,1}^{(1)}\left(1\right)\right]
+f_{2}\left(2\right)\left[r_{2}\mu_{3}
+F_{3,1}^{(1)}\left(1\right)\right]\\
K_{7}=\tilde{\mu}_{2}\mu_{4}\left[R_{2}^{(2)}
+r_{2}
+f_{2}\left(2\right)\right]
+r_{2}\tilde{\mu}_{2}\left[f_{2}\left(4\right)
+F_{4,2}^{(1)}\right]
+f_{2}\left(2\right)\left[r_{2}\tilde{\mu}_{4}
+F_{4,2}^{(1)}\right]\\
K_{8}=\tilde{P}_{3}^{(2)}\left[r_{2}
+f_{2}\left(2\right)\right]
+r_{2}\tilde{\mu}_{3}\left[f_{2}\left(3\right)
+F_{3,2}^{(1)}
+f_{2}\left(3\right)\right]+F_{3,2}^{(1)}\left[2f_{2}\left(3\right)
+r_{2}\tilde{\mu}_{3}\right]
+F_{3,2}^{(2)}
+R_{2}^{(2)}\tilde{\mu}_{3}^{2}\\
K_{9}=\mu_{3}\mu_{4}\left[R_{2}^{(2)}
+r_{2}
+f_{2}\left(2\right)\right]
+r_{2}\tilde{\mu}_{3}\left[f_{2}\left(4\right)
+F_{4,2}^{(1)}\right]
+r_{2}\tilde{\mu}_{4}\left[f_{2}\left(3\right)
+F_{3,2}^{(1)}\right]
+F_{4,2}^{(1)}\left[f_{2}\left(3\right)
+F_{3,2}^{(1)}\right]
+F_{3,2}^{(1)}f_{2}\left(4\right)\\
K_{10}=P_{4}^{(2)}\left[r_{2}
+f_{2}\left(2\right)\right]
+2F_{4,2}^{(1)}\left[r_{2}\tilde{\mu}_{4}
+f_{2}\left(4\right)\right]
+\tilde{\mu}_{4}\left[\tilde{\mu}_{4}R_{2}^{(2)}
+2r_{2}f_{2}\left(4\right)\right]
+F_{4,2}^{(2)}
\end{array}
\end{eqnarray}
\begin{eqnarray}
\begin{array}{l}
K_{11}=f_{1}\left(1\right)\left[P_{1}^{(2)}
+2r_{1}\tilde{\mu}_{1}\right]
+R_{1}^{2}\tilde{\mu}_{1}^{2}
+r_{1}\tilde{P}_{1}^{(2)}\\
K_{12}=\mu_{1}\tilde{\mu}_{2}\left[R_{1}^{(2)}
+r_{1}+f_{1}\left(1\right)\right]
+r_{1}\left[\tilde{\mu}_{1}f_{1}\left(2\right)
+\tilde{\mu}_{2}f_{1}\left(1\right)\right]\\
K_{13}=\tilde{\mu}_{1}\tilde{\mu}_{3}\left[R_{1}^{(2)}
+r_{1}
+f_{1}\left(1\right)\right]
+r_{1}\tilde{\mu}_{1}\left[f_{1}\left(3\right)
+F_{3,1}^{(1)}\right]
+f_{1}\left(1\right)\left[r_{1}\tilde{\mu}_{3}
+F_{3,1}^{(1)}\right]\\
K_{14}=\mu_{1}\mu_{4}\left[R_{1}^{(2)}
+r_{1}+f_{1}\left(1\right)\right]
+r_{1}\mu_{1}\left[f_{1}\left(4\right)
+F_{4,1}^{(1)}\right]
+f_{1}\left(1\right)\left[r_{1}\mu_{4}
+F_{4,1}^{(1)}\right]\\
K_{15}=\tilde{P}_{2}^{(2)}\left[r_{1}+f_{1}\left(1\right)\right]
+\tilde{\mu}_{2}\left[\tilde{\mu}_{2}R_{1}^{(2)}
+2r_{1}f_{1}\left(2\right)\right]\\
K_{16}=\tilde{\mu}_{2}\mu_{3}\left[
R_{1}^{(2)}
+r_{1}
+f_{1}\left(1\right)\right]
+r_{1}\tilde{\mu}_{2}\left[f_{1}\left(3\right)
+F_{3,1}^{(1)}\right]
+f_{1}\left(2\right)\left[r_{1}\tilde{\mu}_{3}
+F_{3,1}^{(1)}\right]\\
K_{17}=\tilde{\mu}_{2}\mu_{4}\left[
+R_{1}^{(2)}
+r_{1}
+f_{1}\left(1\right)\right]
+r_{1}\tilde{\mu}_{2}\left[f_{1}\left(4\right)
+F_{4,1}^{(1)}\right]
+f_{1}\left(2\right)\left[r_{1}\tilde{\mu}_{4}
+F_{4,1}^{(1)}\right]\\
K_{18}=P_{3}^{(2)}\left[r_{1}+f_{1}\left(1\right)\right]
+2r_{1}\tilde{\mu}_{3}\left[f_{1}\left(3\right)
+F_{3,1}^{(1)}\right]
+\left[R_{1}^{(2)}\tilde{\mu}_{3}^{2}
+F_{3,1}^{(2)}
+2F_{3,1}^{(1)}f_{1}\left(3\right)\right]\\
K_{19}=\mu_{3}\mu_{4}\left[R_{1}^{(2)}
+r_{1}
+f_{1}\left(1\right)\right]
+r_{1}\mu_{3}\left[f_{1}\left(4\right)
+F_{4,1}^{(1)}\right]
+f_{1}\left(3\right)\left[r_{1}\tilde{\mu}_{4}
+F_{4,1}^{(1)}\right]
+F_{3,1}^{(1)}\left[r_{1}\tilde{\mu}_{4}
+f_{1}\left(4\right)\right]
+F_{4,2}^{(1)}F_{3,2}^{(1)}\\
K_{20}=P_{4}^{(2)}\left[r_{1}
+f_{1}\left(1\right)\right]
+\tilde{\mu}_{4}\left[2r_{1}f_{1}\left(4\right)
+R_{1}^{(2)}\tilde{\mu}_{4}\right]
+2F_{4,1}^{(1)}\left[f_{1}\left(4\right)
+r_{1}\tilde{\mu}_{4}\right]
+F_{4,1}^{(2)}
\end{array}
\end{eqnarray}
\begin{eqnarray}
\begin{array}{l}
K_{21}=P_{1}^{(2)}\left[r_{4}
+f_{4}\left(4\right)\right]
+2r_{4}\tilde{\mu}_{1}\left[F_{1,4}^{(1)}
+f_{4}\left(1\right)\right]
+\left[2f_{4}\left(1\right)F_{1,4}^{(1)}
+F_{1,4}^{(2)}
+R_{2}^{(2)}\tilde{\mu}_{1}^{2}\right]\\
K_{22}=\mu_{1}\tilde{\mu}_{2}\left[R_{4}^{(2)}
+r_{4}
+f_{4}\left(4\right)\right]
+r_{4}\tilde{\mu}_{1}\left[F_{2,4}^{(1)}
+f_{4}\left(2\right)\right]+
f_{4}\left(1\right)\left[r_{4}\tilde{\mu}_{2}
+F_{2,4}^{(1)}\right]
+F_{1,4}^{(1)}\left[r_{4}\tilde{\mu}_{2}
+f_{4}\left(2\right)+F_{2,4}^{(1)}\right]\\
K_{23}=\mu_{1}\mu_{3}\left[R_{4}^{(2)}
+r_{4}
+f_{4}\left(4\right)\right]
+r_{4}\tilde{\mu}_{3}\left[F_{1,4}^{(1)}
+f_{4}\left(1\right)\right]
+f_{4}\left(3\right)\left[F_{1,4}^{(1)}
+r_{4}\tilde{\mu}_{1}\right]\\
K_{24}=\mu_{1}\mu_{4}\left[R_{4}^{(2)}
+r_{4}
+f_{4}\left(4\right)\right]
+f_{4}\left(4\right)\left[r_{4}\tilde{\mu}_{1}
+F_{1,4}^{(1)}\right]
+r_{4}\tilde{\mu}_{4}\left[f_{4}\left(1\right)
+F_{1,4}^{(1)}\right]\\
K_{25}=R_{4}^{(2)}\tilde{\mu}_{2}^{2}
+\tilde{P}_{2}^{(2)}\left[r_{4}
+f_{4}\left(4\right)\right]
+2r_{4}\tilde{\mu}_{2}\left[F_{2,4}^{(1)}
+f_{4}\left(2\right)\right]
+\left[2f_{4}\left(2\right)F_{2,4}^{(1)}
+F_{2,4}^{(2)}\right]\\
K_{26}=\tilde{\mu}_{2}\mu_{3}\left[R_{4}^{(2)}
+r_{4}
+f_{4}\left(4\right)\right]
+r_{4}\tilde{\mu}_{3}\left[f_{4}\left(2\right)
+F_{2,4}^{(1)}\right]
+f_{4}\left(3\right)\left[r_{4}\tilde{\mu}_{2}+F_{2,4}^{(1)}\right]\\
K_{27}=\tilde{\mu}_{2}\mu_{4}\left[R_{4}^{(2)}
+r_{4}
+f_{4}\left(4\right)\right]
+r_{4}\tilde{\mu}_{4}\left[f_{4}\left(4\right)
+F_{2,4}^{(1)}\right]
+f_{4}\left(4\right)\left[r_{4}\tilde{\mu}_{2}
+F_{2,4}^{(1)}\right]\\
K_{28}=P_{3}^{(2)}\left[r_{4}
+f_{4}\left(4\right)\right]
+\tilde{\mu}_{3}\left[R_{4}^{(2)}\tilde{\mu}_{3}
+2r_{4}f_{4}\left(4\right)\right]\\
K_{29}=\mu_{3}\mu_{4}\left[R_{4}^{(2)}
+r_{4}
+f_{4}\left(4\right)\right]
+r_{4}\left[\tilde{\mu}_{3}f_{4}\left(4\right)
+\tilde{\mu}_{4}f_{4}\left(3\right)\right]\\
K_{30}=P_{4}^{(2)}\left[r_{4}
+f_{4}\left(4\right)\right]
+\tilde{\mu}_{4}\left[R_{4}^{(2)}\tilde{\mu}_{4}
+2r_{4}f_{4}\left(4\right)\right]
\end{array}
\end{eqnarray}

\begin{eqnarray}
\begin{array}{l}
K_{31}=P_{1}^{(2)}\left[r_{3}
+f_{3}\left(3\right)\right]
+2f_{3}\left(1\right)\left[r_{3}\tilde{\mu}_{1}
+F_{1,3}^{(1)}\right]
+\tilde{\mu}_{1}\left[R_{3}^{(2)}\tilde{\mu}_{1}
+2F_{1,3}^{(1)}r_{3}\right]
+F_{1,3}^{(2)}\\
K_{32}=\mu_{1}\tilde{\mu}_{2}\left[R_{3}^{(2)}
+r_{3}
+f_{3}\left(3\right)\right]
+r_{3}\tilde{\mu}_{1}\left[F_{2,3}^{(1)}
+f_{3}\left(2\right)\right]
+f_{3}\left(1\right)\left[r_{3}\tilde{\mu}_{2}
+F_{2,3}^{(1)}\right]
+F_{1,3}^{(1)}\left[r_{3}\tilde{\mu}_{2}
+f_{3}\left(2\right)+F_{2,3}^{(1)}\right]\\
K_{33}=\mu_{1}\mu_{3}\left[R_{3}^{(2)}+r_{3}
+f_{3}\left(3\right)\right]
+r_{3}\tilde{\mu}_{3}\left[f_{3}\left(1\right)
+F_{1,3}^{(1)}\right]
+f_{3}\left(3\right)\left[F_{1,3}^{(1)}+r_{3}\tilde{\mu}_{1}\right]\\
K_{34}=\mu_{1}\mu_{4}\left[R_{3}^{(2)}
+r_{3}
+f_{3}\left(3\right)\right]
+r_{3}\tilde{\mu}_{4}\left[f_{3}\left(1\right)
+F_{1,3}^{(1)}\right]
+f_{3}\left(4\right)\left[r_{3}\tilde{\mu}_{1}+F_{1,3}^{(1)}\right]\\
K_{35}=P_{2}^{(2)}\left[r_{3}
+f_{3}\left(3\right)\right]
+2r_{3}\tilde{\mu}_{2}\left[F_{2,3}^{(1)}
+f_{3}\left(2\right)\right]
+\left[R_{3}^{(2)}\tilde{\mu}_{2}^{2}
+F_{2,3}^{(2)}
+2f_{3}\left(2\right)F_{2,3}^{(1)}\right]\\
K_{36}=\mu_{3}\tilde{\mu}_{2}\left[R_{3}^{(2)}
+r_{3}
+f_{3}\left(3\right)\right]
+r_{3}\tilde{\mu}_{3}\left[f_{3}\left(2\right)
+F_{2,3}^{(1)}\right]
+f_{3}\left(3\right)\left[r_{3}\tilde{\mu}_{2}
+F_{2,3}^{(1)}\right]\\
K_{37}=\mu_{4}\tilde{\mu}_{2}\left[R_{3}^{(2)}
+r_{3}
+f_{3}\left(3\right)\right]
+f_{3}\left(4\right)\left[r_{3}\tilde{\mu}_{2}
+F_{2,3}^{(1)}\right]
+r_{3}\tilde{\mu}_{4}\left[f_{3}\left(2\right)
+F_{2,3}^{(1)}\right]\\
K_{38}=P_{3}^{(2)}\left[r_{3}
+f_{3}\left(3\right)\right]
+\tilde{\mu}_{3}\left[R_{3}^{(2)}\tilde{\mu}_{3}
+2r_{3}f_{3}\left(3\right)\right]\\
K_{39}=\mu_{4}\mu_{3}\left[R_{3}^{(2)}
+r_{3}
+f_{3}\left(3\right)\right]
+r_{3}\left[\tilde{\mu}_{3}f_{3}\left(4\right)
+\tilde{\mu}_{4}f_{3}\left(3\right)\right]\\
K_{40}=\tilde{\mu}_{4}\left[2r_{3}f_{3}\left(4\right)
+R_{3}^{(2)}\tilde{\mu}_{4}\right]
+P_{4}^{(2)}\left[r_{3}
+f_{3}\left(3\right)\right]
\end{array}
\end{eqnarray}



%___________________________________________________________________
Let's define the
probability of the event no ruin before the $n$-th period begining with $\tilde{L}_{0}$ users, $g_{n,k}$ considering a capital equal to $k$ units after $n-1$ events, i.e.,  given $n\in\left\{1,2,\ldots\right\}$ y $k\in\left\{0,1,2,\ldots\right\}$ $g_{n,k}:=P\left\{\tilde{L}_{j}>0, j=1,\ldots,n,\tilde{L}_{n}=k\right\}$, which can be written as:



\begin{Assumption}
\label{A:PD}

\begin{enumerate}
\item[a)] $c$ is lower semicontinuous, and inf-compact on $\mathbb{K}$ (i.e.
for every $x\in X$ and $r\in \mathbb{R}$ the set $\{a \in A(x):c(x,a) \leq  r
\}$ is compact).

\item[b)] The transition law $Q$ is strongly continuous, i.e. $u(x,a)=\int
u(y)Q(dy|x,a)$, $(x,a)\in\mathbb{K}$ is continuous and bounded on $\mathbb{K}$, for every
measurable bounded function $u$ on $X$.

\item[c)] There exists a policy $\pi$ such that $V(\pi,x)<\infty$, for each $%
x \in X$.
\end{enumerate}
\end{Assumption}

\begin{Remark}
\label{R:BT}

The following consequences of Assumption \ref{A:PD} are well-known (see
Theorem 4.2.3 and Lemma 4.2.8 in \cite{Hernandez}):

\begin{enumerate}
\item[a)] The optimal value function $V^{\ast}$ is the solution of the
\textit{Optimality Equation} (OE), i.e. for all $x \in X$,
\begin{equation*}
V^{\ast}(x)=\underset{a\in A(x)}{\min }\left\{ c(x,a)+\alpha \int
V^{\ast}(y)Q(dy|x,a)\right\} \text{.}
\end{equation*}

There is also $f^{\ast}\in \mathbb{F}$ such that:
\begin{equation}
V^{\ast}(x)= c(x,f^{\ast}(x))+\alpha \int V^{\ast}(y)Q(dy|x,f^{\ast}(x)), \label{2.1}
\end{equation}
$ x\in X$, and $f^{\ast}$ is optimal.

\item[b)] For every $x \in X$, $v_{n}(x)\uparrow V^{\ast}$, with $v_{n}$
defined as
\begin{equation*}
v_{n}(x)=\underset{a\in A(x)}{\min }\left\{ c(x,a)+\alpha \int
v_{n-1}(y)Q(dy| x,a)\right\},
\end{equation*}
 $x\in X, n=1,2,\cdots $, and $v_{0}(x)=0$. Moreover, for each $n$, there is $%
f_{n}\in \mathbb{F}$ such that, for each $x\in X$,
\begin{equation}
\underset{a\in A(x)}{\min }\left\{ c(x,a)+\alpha \int
v_{n-1}(y)Q(dy|x,a)\right\}= c(x,f_{n}(x))+\alpha \int
v_{n-1}(y)Q(dy|x,f_{n}(x)).  \label{2.2}
\end{equation}
\end{enumerate}
\end{Remark}

Let $(X,A,\{A(x):x\in X\},Q,c)$ be a fixed Markov control model. Take $M$ as the MDP with the Markov control model $(X,A,\{A(x):x\in
X\},Q,c)$. The optimal value function, the optimal policy which comes from (%
\ref{2.1}), and the minimizers in (\ref{2.2}) will be denoted for $M$ by $%
V^{\ast}$, $f^{\ast}$, and $f_{n}$ , $n=1,2,\cdots $, respectively. Also let
$v_{n}$, $n=1,2,\cdots $, be the value iteration functions for $M$. Let $%
G(x,a):=c(x,a)+\alpha \int V^{\ast}(y)Q(dy|x,a)$, $(x,a)\in \mathbb{K}$.

It will be also supposed that the MDPs taken into account satisfy one of the
following Assumptions \ref{A:2} or \ref{A:3}.

\begin{Assumption}
\label{A:2}

\begin{enumerate}
\item[a)] $X$ and $A$ are convex;

\item[b)] $(1- \lambda)a+a^{\prime }\in A((1- \lambda)x+x^{\prime })$ for
all $x$, $x^{\prime }\in X$, $a\in A(x)$, $a^{\prime }\in A(x^{\prime })$
and $\lambda \in [0,1]$. Besides it is assumed that: if $x$ and $y\in X$, $x <
y $, then $A(y)\subseteq A(x)$, and $A(x)$ are convex for each $x \in X$;

\item[c)] $Q$ is induced by a difference equation $x_{t+1}=F(x_{t},a_{t},%
\xi_{t})$, with $t=0,1,\cdots $, where $F:X\times A\times S \rightarrow X$
is a measurable function and $\{\xi_{t}\}$ is a sequence of independent and
identically distributed (i.i.d.) random variables with values in $S \subseteq
\mathbb{R}$, and with a common density $\Delta$. In addition, we suppose
that $F(\cdot,\cdot,s)$ is a convex function on $\mathbb{K}$, for each $s\in
S$; and if $x$ and $y\in X$, $x < y$, then $F(x,a,s)\leq F (y,a,s)$ for each
$a\in A(y)$ and $s\in S$;

\item[d)] $c$ is convex on $\mathbb{K}$, and if  $x$ and $y\in X$, $x < y$,
then $c(x,a)\leq c(y,a)$, for each $a\in A(y)$.
\end{enumerate}
\end{Assumption}

\begin{Assumption}
\label{A:3}

\begin{enumerate}
\item[a)] Same as Assumption \ref{A:2} (a);

\item[b)] $(1- \lambda)a+a^{\prime }\in A((1- \lambda)x+x^{\prime })$ for
all $x$, $x^{\prime }\in X$, $a\in A(x)$, $a^{\prime }\in A(x^{\prime })$
and $\lambda\in [0,1]$. Besides $A(x)$ is assumed to be convex for each $x
\in X$;

\item[c)] $Q$ is given by the relation $x_{t+1}=\gamma x_{t}+\delta
a_{t}+\xi_{t}$, $t=0,1,\cdots $, where $\{\xi_{t}\}$ are i.i.d. random
variables taking values in $S\subseteq \mathbb{R}$ with the density $\Delta$%
, $\gamma$ and $\delta$ are real numbers;

\item[d)] $c$ is convex on $\mathbb{K}$.
\end{enumerate}
\end{Assumption}

\begin{Remark}
\label{R:2} Assumptions \ref{A:2} and \ref{A:3} are essentially presented in
Conditions C1 and C2 in \cite{DRS}, but changing a strictly convex $c(\cdot,
\cdot)$ by a convex $c(\cdot, \cdot)$. (In fact, in \cite{DRS}, Conditions C1
and C2 take into account the more general situation in which both $X$ and $A$
are subsets of Euclidean spaces of the dimension greater than one.)
Also note that it is possible to obtain that each of Assumptions \ref{A:2}
and \ref{A:3} implies that, for each $x\in X$, $G(x,\cdot)$ is convex but
not necessarily strictly convex (hence, $M$ does not necessarily have a
unique optimal policy). The proof of this fact is a direct consequence of
the convexity of the cost function $c$, and of the proof of Lemma 6.2 in
\cite{DRS}.
\end{Remark}


\begin{eqnarray*}
\begin{array}{llll}
D_{1}F_{1}\left(z_{1},z_{2};\tau_{3}\right)=F_{1,3}^{(1)}\left(1\right),&
D_{2}F_{1}\left(z_{1},z_{2};\tau_{3}\right)=F_{2,3}^{(1)}\left(1\right),&
D_{1}F_{2}\left(z_{1},z_{2};\tau_{4}\right)=F_{1,4}^{(1)}\left(1\right),&
D_{2}F_{2}\left(z_{1},z_{2};\tau_{4}\right)=F_{2,4}^{(1)}\left(1\right)\\
D_{3}F_{3}\left(z_{3},z_{4};\tau_{1}\right)=F_{3,1}^{(1)}\left(1\right),&
D_{4}F_{3}\left(z_{3},z_{4};\tau_{1}\right)=F_{4,1}^{(1)}\left(1\right),&
D_{3}F_{4}\left(z_{3},z_{4};\tau_{2}\right)=F_{3,2}^{(1)}\left(1\right),&
D_{4}F_{4}\left(z_{3},z_{4};\tau_{2}\right)=F_{4,2}^{(1)}\left(1\right)
\end{array}
\end{eqnarray*}

\begin{eqnarray*}
\begin{array}{lll}
D_{1}^{2}F_{1}\left(z_{1},z_{2};\tau_{3}\right)=F_{1,3}^{(2)},&
D_{2}D_{1}F_{1}\left(z_{1},z_{2};\tau_{3}\right)=F_{2,3}^{(1)}\left(1\right)F_{1,3}^{(1)},&
D_{2}^{2}F_{1}\left(z_{1},z_{2};\tau_{3}\right)=F_{2,3}^{(2)}\\
D_{1}^{2}F_{2}\left(z_{1},z_{2};\tau_{4}\right)=F_{1,4}^{(2)},&
D_{2}D_{1}F_{2}\left(z_{1},z_{2};\tau_{4}\right)=F_{2,4}^{(1)}\left(1\right)F_{1,4}^{(1)},&
D_{2}^{2}F_{2}\left(z_{1},z_{2};\tau_{4}\right)=F_{2,4}^{(2)}\\
D_{3}^{2}F_{3}\left(z_{3},z_{4};\tau_{1}\right)=F_{3,1}^{(2)},&
D_{4}D_{3}F_{3}\left(z_{3},z_{4};\tau_{1}\right)=F_{3,1}^{(1)}\left(1\right)F_{4,1}^{(1)},&
D_{4}^{2}F_{3}\left(z_{3},z_{4};\tau_{1}\right)=F_{4,1}^{(2)}\\
D_{3}^{2}F_{4}\left(z_{3},z_{4};\tau_{2}\right)=F_{3,2}^{(2)},&
D_{4}D_{3}F_{4}\left(z_{3},z_{4};\tau_{2}\right)=F_{3,2}^{(1)}\left(1\right)F_{4,2}^{(1)},&
D_{4}^{2}F_{4}\left(z_{3},z_{4};\tau_{2}\right)=F_{4,2}^{(2)}
\end{array}
\end{eqnarray*}

 The second order partial derivatives can be obtained in the following
\begin{eqnarray}
\begin{array}{ll}
D_{i}^{2}F_{k}\left(z_{1},z_{2};\tau_{2k+1}\right)=F_{i,2k+1}^{(2)},&
D_{i}^{2}F_{k}\left(z_{3},z_{4};\tau_{2k+1}\right)=F_{i,2k+1}^{(2)}\\
D_{j}D_{i}F_{k}\left(z_{1},z_{2};\tau_{2k+1}\right)=F_{j,2k+1}^{(1)}F_{i,2k+1}^{(1)},&
D_{j}D_{i}F_{k}\left(z_{3},z_{4};\tau_{2k+1}\right)=F_{j,2k+1}^{(1)}F_{i,2k+1}^{(1)}.
\end{array}
\end{eqnarray}

\begin{eqnarray*}
D_{j}D_{i}F_{k}\left(z_{1},z_{2};\tau_{k-2}\right)&=&\indora_{i\geq3}\indora_{j=i}F_{i,k-2}^{(2)}+\indora_{i\geq 3}\indora_{j\neq i}F_{j,k-2}^{(1)}F_{i,k+2}^{(1)}\\
D_{j}D_{i}F_{k}\left(z_{3},z_{4};\tau_{k+2}\right)&=&\indora_{i\geq3}\indora_{j=i}F_{i,k+2}^{(2)}+\indora_{i\geq 3}\indora_{j\neq i}F_{j,k+2}^{(1)}F_{i,k+2}^{(1)}\\
D_{j}D_{i}F_{3}\left(z_{1},z_{2};\tau_{1}\right)&=&\indora_{i\geq3}\indora_{j=i}F_{i,1}^{(2)}+\indora_{i\geq 3}\indora_{j\neq i}F_{j,1}^{(1)}F_{i,1}^{(1)}\\
D_{j}D_{i}F_{4}\left(z_{1},z_{2};\tau_{2}\right)&=&\indora_{i\geq3}\indora_{j=i}F_{i,2}^{(2)}+\indora_{i\geq 3}\indora_{j\neq i}F_{j,2}^{(1)}F_{i,1}^{(1)}\\
D_{j}D_{i}F_{1}\left(z_{3},z_{4};\tau_{3}\right)&=&\indora_{i\leq2}\indora_{j=i}F_{i,3}^{(2)}+\indora_{i\leq 2}\indora_{j\neq i}F_{j,3}^{(1)}F_{i,3}^{(1)}\\
D_{j}D_{i}F_{2}\left(z_{3},z_{4};\tau_{4}\right)&=&\indora_{i\leq2}\indora_{j=i}F_{i,4}^{(2)}+\indora_{i\leq 2}\indora_{j\neq i}F_{j,4}^{(1)}F_{i,4}^{(1)}
\end{eqnarray*}










 with second order partial derivatives

\begin{eqnarray}
\begin{array}{ll}
D_{i}^{2}F_{k}\left(z_{1},z_{2};\tau_{2k+1}\right)=F_{i,2k+1}^{(2)},&
D_{i}^{2}F_{k}\left(z_{3},z_{4};\tau_{2k+1}\right)=F_{i,2k+1}^{(2)}\\
D_{j}D_{i}F_{k}\left(z_{1},z_{2};\tau_{2k+1}\right)=F_{j,2k+1}^{(1)}F_{i,2k+1}^{(1)},&
D_{j}D_{i}F_{k}\left(z_{3},z_{4};\tau_{2k+1}\right)=F_{j,2k+1}^{(1)}F_{i,2k+1}^{(1)}.
\end{array}
\end{eqnarray}



for $i,j,k=1,2,3,4$.




\begin{Assumption}
\label{A:4} There is a policy $\phi$ such that $E_{x}^{\phi }\left[ \text{$\sum\limits_{t=0}^{\infty }$}\alpha
^{t}c^*(x_{t},a_{t})\right] \text{}<\infty$%
, for each $x\in X$.
\end{Assumption}

\begin{Remark}
\label{R:3} Suppose that, for M, Assumption 2.1 holds. Then, it is direct to verify that if $M_{\epsilon}$ satisfies Assumption \ref{A:4}, then it also
satisfies Assumption \ref{A:PD}.
\end{Remark}

\begin{Condition}
\label{C:1} There exists a measurable function $Z:X\rightarrow \mathbb{R}$,
which may depend on $\alpha$, such that $c^{%
\ast}(x,a)-c(x,a)=\epsilon a^{2}\leq\epsilon Z(x)$, and $\int
Z(y)Q(dy|x,a)\leq Z(x)$ for each $x\in X$ and $a\in B(x)$.
\end{Condition}

\begin{Theorem}
\label{T:1} Suppose that Assumptions \ref{A:PD} and \ref{A:4} hold, and
that, for $M$, one of Assumptions \ref{A:2} or \ref{A:3} holds. Let $%
\epsilon $ be a positive number. Then,

\begin{enumerate}
\item[a)] If $A$ is compact, $|W^{\ast}(x)-V^{\ast}(x)|\leq \epsilon K^{2}/(1-\alpha)$%
, $x\in X$, where $K$ is the diameter of a compact set $D$ such that $0\in D$
and $A\subseteq D$.

\item[b)] Under Condition \ref{C:1}, $|W^{\ast}(x) - V^{\ast}(x)|\leq
\epsilon Z(x)/(1- \alpha)$, $x\in X$.
\end{enumerate}
\end{Theorem}

\begin{proof}
The proof of case (a) follows from the proof of case (b) given that $Z(x)=K^{2}$, $x\in X$. (Observe that in this case, if $a\in A$,
then $a^{2}=(a-0)^{2} \leq K^{2}$.)

\textbf{(b)} Assume that $M$ satisfies Assumption \ref{A:2}. (The proof for
the case in which $M$ satisfies Assumption \ref{A:3} is similar.)

\end{proof}

The following Corollary  is immediate.

\begin{Corollary}\label{Co:1}
Suppose that Assumptions \ref{A:PD} and \ref{A:4} hold. Suppose
that for $M$ one of Assumptions \ref{A:2} or \ref{A:3} holds (hence $M$
does not necessarily have a unique optimal policy). Let $\epsilon $ be a
positive number. If $A$ is compact or Condition \ref{C:1} holds, then there
exists an MDP $M_{\epsilon }$ with a unique optimal policy $g^{\ast }$, such
that inequalities in Theorem 3.7 (a) or (b) hold, respectively.
\end{Corollary}

\begin{Example}\label{E:1}
Ejemplo1
\end{Example}

\begin{Lemma}\label{L:1}
Lema1
\end{Lemma}

\begin{proof}
Assumption \ref{A:PD} (a) trivially holds. The proof of the strong continuity of $Q$

\end{proof}





The second order partial derivatives are
According to the notation given in \cite{Lang} we obtain

\begin{eqnarray}
D_{i}D_{i}R_{k}=D^{2}R_{k}\left(D_{i}\tilde{P}_{i}\right)^{2}+DR_{k}D_{i}^{2}\tilde{P}_{i}
\end{eqnarray}

whereas for $i\neq j$

\begin{eqnarray}
D_{i}D_{j}R_{k}=D^{2}R_{k}D_{i}\tilde{P}_{i}D_{j}\tilde{P}_{j}+DR_{k}D_{j}\tilde{P}_{j}D_{i}\tilde{P}_{i}
\end{eqnarray}
while the mixed partial derivatives are:

\begin{eqnarray}
\begin{array}{ll}
D_{i}D_{i}R_{k}=D^{2}R_{k}\left(D_{i}P_{i}\right)^{2}+DR_{k}D_{i}D_{i}P_{i},&i=j\\
D_{i}D_{j}R_{k}=D^{2}R_{k}D_{i}P_{i}D_{j}P_{j}+DR_{k}D_{j}P_{j}D_{i}P_{i},&i\neq j
\end{array}
\end{eqnarray}



\begin{eqnarray}
D_{j}D_{i}R_{k}=R_{k}^{(2)}\mu_{i}\mu_{j}+\indora_{i=j}r_{k}P_{i}^{(2)}+\indora_{i=j}r_{k}\mu_{i}\mu_{j}
\end{eqnarray}
for any $i,j,k$.



\begin{eqnarray*}
\begin{array}{llll}
D_{1}F_{1}\left(z_{1},z_{2};\tau_{3}\right)=F_{1,3}^{(1)}\left(1\right),&
D_{2}F_{1}\left(z_{1},z_{2};\tau_{3}\right)=F_{2,3}^{(1)}\left(1\right),&
D_{1}F_{2}\left(z_{1},z_{2};\tau_{4}\right)=F_{1,4}^{(1)}\left(1\right),&
D_{2}F_{2}\left(z_{1},z_{2};\tau_{4}\right)=F_{2,4}^{(1)}\left(1\right)\\
D_{3}F_{3}\left(z_{3},z_{4};\tau_{1}\right)=F_{3,1}^{(1)}\left(1\right),&
D_{4}F_{3}\left(z_{3},z_{4};\tau_{1}\right)=F_{4,1}^{(1)}\left(1\right),&
D_{3}F_{4}\left(z_{3},z_{4};\tau_{2}\right)=F_{3,2}^{(1)}\left(1\right),&
D_{4}F_{4}\left(z_{3},z_{4};\tau_{2}\right)=F_{4,2}^{(1)}\left(1\right)
\end{array}
\end{eqnarray*}

with second order derivatives given by

\begin{eqnarray*}
\begin{array}{lll}
D_{1}^{2}F_{1}\left(z_{1},z_{2};\tau_{3}\right)=F_{1,3}^{(2)}\left(1\right),&
D_{2}D_{1}F_{1}\left(z_{1},z_{2};\tau_{3}\right)=F_{2,3}^{(1)}\left(1\right)F_{1,3}^{(1)}\left(1\right),&
D_{2}^{2}F_{1}\left(z_{1},z_{2};\tau_{3}\right)=F_{2,3}^{(2)}\left(1\right)\\
D_{1}^{2}F_{2}\left(z_{1},z_{2};\tau_{4}\right)=F_{1,4}^{(2)}\left(1\right),&
D_{2}D_{1}F_{2}\left(z_{1},z_{2};\tau_{4}\right)=F_{2,4}^{(1)}\left(1\right)F_{1,4}^{(1)}\left(1\right),&
D_{2}^{2}F_{2}\left(z_{1},z_{2};\tau_{4}\right)=F_{2,4}^{(2)}\left(1\right)\\
D_{3}^{2}F_{3}\left(z_{3},z_{4};\tau_{1}\right)=F_{3,1}^{(2)}\left(1\right),&
D_{4}D_{3}F_{3}\left(z_{3},z_{4};\tau_{1}\right)=F_{3,1}^{(1)}\left(1\right)F_{4,1}^{(1)}\left(1\right),&
D_{4}^{2}F_{3}\left(z_{3},z_{4};\tau_{1}\right)=F_{4,1}^{(2)}\left(1\right)\\
D_{3}^{2}F_{4}\left(z_{3},z_{4};\tau_{2}\right)=F_{3,2}^{(2)}\left(1\right),&
D_{4}D_{3}F_{4}\left(z_{3},z_{4};\tau_{2}\right)=F_{3,2}^{(1)}\left(1\right)F_{4,2}^{(1)}\left(1\right),&
D_{4}^{2}F_{4}\left(z_{3},z_{4};\tau_{2}\right)=F_{4,2}^{(2)}\left(1\right)
\end{array}
\end{eqnarray*}

According to the equations given before, we can obtain general expressions, so for
$F_{1}\left(z_{1},z_{2};\tau_{3}\right)$ and $F_{2}\left(z_{1},z_{2};\tau_{4}\right)$ we have

\begin{eqnarray}%\label{Ec.Gral.Primer.Momento.Ind.Exh}
\begin{array}{ll}
D_{j}F_{i}\left(z_{1},z_{2};\tau_{2i+1}\right)=\indora_{j\leq2}F_{j,2i+1}^{(1)},&
D_{j}F_{i}\left(z_{3},z_{4};\tau_{2i+1}\right)=\indora_{j\geq3}F_{j,2i+1}^{(1)}
\end{array}
\end{eqnarray}

for $i=1,2$ and $j=1,2,3,4$. The second order partial derivatives can be obtained in the following


\begin{eqnarray}
\begin{array}{ll}
D_{i}^{2}F_{k}\left(z_{1},z_{2};\tau_{2k+1}\right)=F_{i,2k+1}^{(2)},&
D_{i}^{2}F_{k}\left(z_{3},z_{4};\tau_{2k+1}\right)=F_{i,2k+1}^{(2)}\\
D_{j}D_{i}F_{k}\left(z_{1},z_{2};\tau_{2k+1}\right)=F_{j,2k+1}^{(1)}F_{i,2k+1}^{(1)},&
D_{j}D_{i}F_{k}\left(z_{3},z_{4};\tau_{2k+1}\right)=F_{j,2k+1}^{(1)}F_{i,2k+1}^{(1)}.
\end{array}
\end{eqnarray}

\begin{eqnarray*}
%D_{j}D_{i}F_{k}\left(z_{1},z_{2};\tau_{2k+1}\right)&=&\indora_{i\geq3}\indora_{j=i}F_{i,2k+1}^{(2)}+\indora_{i\geq 3}\indora_{j\neq i}F_{j,2k+1}^{(1)}F_{i,2k+1}^{(1)}\\
D_{j}D_{i}F_{3}\left(z_{1},z_{2};\tau_{1}\right)&=&\indora_{i\geq3}\indora_{j=i}F_{i,1}^{(2)}+\indora_{i\geq 3}\indora_{j\neq i}F_{j,1}^{(1)}F_{i,1}^{(1)}\\
D_{j}D_{i}F_{4}\left(z_{1},z_{2};\tau_{2}\right)&=&\indora_{i\geq3}\indora_{j=i}F_{i,2}^{(2)}+\indora_{i\geq 3}\indora_{j\neq i}F_{j,2}^{(1)}F_{i,1}^{(1)}\\
D_{j}D_{i}F_{1}\left(z_{3},z_{4};\tau_{3}\right)&=&\indora_{i\leq2}\indora_{j=i}F_{i,3}^{(2)}+\indora_{i\leq 2}\indora_{j\neq i}F_{j,3}^{(1)}F_{i,3}^{(1)}\\
D_{j}D_{i}F_{2}\left(z_{3},z_{4};\tau_{4}\right)&=&\indora_{i\leq2}\indora_{j=i}F_{i,4}^{(2)}+\indora_{i\leq 2}\indora_{j\neq i}F_{j,4}^{(1)}F_{i,4}^{(1)}
\end{eqnarray*}



The second order partial derivatives are
{\footnotesize{
\begin{eqnarray*}
\begin{array}{l}
D_{j}D_{i}F_{1}=\indora_{i,j\neq1}D_{1}D_{1}F_{1}\left(D\tilde{\theta}_{1}\right)^{2}D_{i}\tilde{P}_{i}D_{j}\tilde{P}_{j}
+\indora_{i,j\neq1}D_{1}F_{1}D^{2}\tilde{\theta}_{1}D_{i}\tilde{P}_{i}D_{j}\tilde{P}_{j}
+\indora_{i,j\neq1}D_{1}F_{1}D\tilde{\theta}_{1}\left(\indora_{i=j}D_{i}^{2}\tilde{P}_{i}+\indora_{i\neq j}D_{i}\tilde{P}_{i}D_{j}\tilde{P}_{j}\right)\\
+\left(1-\indora_{i=j=3}\right)\indora_{i+j\leq6}D_{1}D_{2}F_{1}D\tilde{\theta}_{1}\left(\indora_{i\leq j}D_{j}\tilde{P}_{j}+\indora_{i>j}D_{i}\tilde{P}_{i}\right)
+\indora_{i=2}\left(D_{1}D_{2}F_{1}D\tilde{\theta}_{1}D_{i}\tilde{P}_{i}+D_{i}^{2}F_{1}\right)\\
D_{j}D_{i}F_{2}=\indora_{i,j\neq2}D_{2}D_{2}F_{2}\left(D\tilde{\theta}_{2}\right)^{2}D_{i}\tilde{P}_{i}D_{j}\tilde{P}_{j}
+\indora_{i,j\neq2}D_{2}F_{2}D^{2}\tilde{\theta}_{2}D_{i}\tilde{P}_{i}D_{j}\tilde{P}_{j}+\indora_{i,j\neq2}D_{2}F_{2}D\tilde{\theta}_{2}\left(\indora_{i=j}D_{i}^{2}\tilde{P}_{i}
+\indora_{i\neq j}D_{i}\tilde{P}_{i}D_{j}\tilde{P}_{j}\right)\\
+\left(1-\indora_{i=j=3}\right)\indora_{i+j\leq6}D_{2}D_{1}F_{2}D\tilde{\theta}_{2}\left(\indora_{i\leq j}D_{j}\tilde{P}_{j}+\indora_{i>j}D_{i}\tilde{P}_{i}\right)
+\indora_{i=1}\left(D_{2}D_{1}F_{2}D\tilde{\theta}_{2}D_{i}\tilde{P}_{i}+D_{i}^{2}F_{2}\right)\\
D_{j}D_{i}F_{3}=\indora_{i,j\neq3}D_{3}D_{3}F_{3}\left(D\tilde{\theta}_{3}\right)^{2}D_{i}\tilde{P}_{i}D_{j}\tilde{P}_{j}
+\indora_{i,j\neq3}D_{3}F_{3}D^{2}\tilde{\theta}_{3}D_{i}\tilde{P}_{i}D_{j}\tilde{P}_{j}
+\indora_{i,j\neq3}D_{3}F_{3}D\tilde{\theta}_{3}\left(\indora_{i=j}D_{i}^{2}\tilde{P}_{i}+\indora_{i\neq j}D_{i}\tilde{P}_{i}D_{j}\tilde{P}_{j}\right)\\
+\indora_{i+j\geq5}D_{3}D_{4}F_{3}D\tilde{\theta}_{3}\left(\indora_{i\leq j}D_{i}\tilde{P}_{i}+\indora_{i>j}D_{j}\tilde{P}_{j}\right)
+\indora_{i=4}\left(D_{3}D_{4}F_{3}D\tilde{\theta}_{3}D_{i}\tilde{P}_{i}+D_{i}^{2}F_{3}\right)\\
D_{j}D_{i}F_{4}=\indora_{i,j\neq4}D_{4}D_{4}F_{4}\left(D\tilde{\theta}_{4}\right)^{2}D_{i}\tilde{P}_{i}D_{j}\tilde{P}_{j}
+\indora_{i,j\neq4}D_{4}F_{4}D^{2}\tilde{\theta}_{4}D_{i}\tilde{P}_{i}D_{j}\tilde{P}_{j}
+\indora_{i,j\neq4}D_{4}F_{4}D\tilde{\theta}_{4}\left(\indora_{i=j}D_{i}^{2}\tilde{P}_{i}+\indora_{i\neq j}D_{i}\tilde{P}_{i}D_{j}\tilde{P}_{j}\right)\\
+\left(1-\indora_{i=j=2}\right)\indora_{i+j\geq4}D_{4}D_{3}F_{4}D\tilde{\theta}_{4}\left(\indora_{i\leq j}D_{i}\tilde{P}_{i}+\indora_{i>j}D_{j}\tilde{P}_{j}\right)
+\indora_{i=3}\left(D_{4}D_{3}F_{4}D\tilde{\theta}_{4}D_{i}\tilde{P}_{i}+D_{i}^{2}F_{4}\right)
\end{array}
\end{eqnarray*}}}


then the mixed partial derivatives are:

{\footnotesize{
\begin{eqnarray*}
D_{k}D_{i}F_{1}&=&D_{k}D_{i}\left(R_{2}+F_{2}+\indora_{i\geq3}F_{4}\right)+D_{i}R_{2}D_{k}\left(F_{2}+\indora_{k\geq3}F_{4}\right)
+D_{i}F_{2}D_{k}\left(R_{2}+\indora_{k\geq3}F_{4}\right)+\indora_{i\geq3}D_{i}F_{4}D_{k}\left(R_{2}+F_{2}\right)\\
D_{k}D_{i}F_{2}&=&D_{k}D_{i}\left(R_{1}+F_{1}+\indora_{i\geq3}F_{3}\right)+D_{i}R_{1}D_{k}\left(F_{1}
+\indora_{k\geq3}F_{3}\right)+D_{i}F_{1}D_{k}\left(R_{1}+\indora_{k\geq3}F_{3}\right)+\indora_{i\geq3}D_{i}F_{3}D_{k}\left(R_{1}+F_{1}\right)\\
D_{k}D_{i}F_{3}&=&D_{k}D_{i}\left(R_{4}+\indora_{k\leq2}F_{2}+F_{4}\right)+D_{i}R_{4}D_{k}\left(\indora_{k\leq2}F_{2}
+F_{4}\right)+D_{i}F_{4}D_{k}\left(R_{4}+\indora_{k\leq2}F_{2}\right)+\indora_{i\leq2}D_{i}F_{2}D_{k}\left(R_{4}+F_{4}\right)\\
D_{k}D_{i}F_{4}&=&D_{k}D_{i}\left(R_{3}+\indora_{k\leq2}F_{1}+F_{3}\right)+D_{i}R_{3}D_{k}\left(\indora_{k\leq2}F_{1}
+F_{3}\right)+D_{i}F_{3}D_{k}\left(R_{3}+\indora_{k\leq2}F_{1}\right)+\indora_{i\leq2}D_{i}F_{1}D_{k}\left(R_{3}+F_{3}\right)
\end{eqnarray*}}}
for $i,k=1,\ldots,4$.

\begin{eqnarray*}
D_{j}D_{i}F_{1}&=&
D_{j}D_{i}\left(R_{2}+F_{2}+\indora_{i\geq3}F_{4}\right)+D_{i}R_{2}D_{j}\left(F_{2}+\indora_{j\geq3}F_{4}\right)
+D_{i}F_{2}D_{j}\left(R_{2}+\indora_{j\geq3}F_{4}\right)+\indora_{i\geq3}D_{i}F_{4}D_{j}\left(R_{2}+F_{2}\right)\\
&=&D_{j}D_{i}R_{2}+D_{j}D_{i}F_{2}+\indora_{i\geq3}D_{j}D_{i}F_{4}
+D_{i}R_{2}D_{j}F_{2}+\indora_{j\geq3}D_{i}R_{2}D_{j}F_{4}
+D_{i}F_{2}D_{j}R_{2}+\indora_{j\geq3}D_{i}F_{2}D_{j}F_{4}\\
&+&\indora_{i\geq3}D_{i}F_{4}D_{j}R_{2}
+\indora_{i\geq3}D_{i}F_{4}D_{j}F_{2}\\
&=&D_{j}D_{i}R_{2}
+\indora_{i,j\neq2}D_{2}D_{2}F_{2}\left(D\tilde{\theta}_{2}\right)^{2}D_{i}\tilde{P}_{i}D_{j}\tilde{P}_{j}
+\indora_{i,j\neq2}D_{2}F_{2}D^{2}\tilde{\theta}_{2}D_{i}\tilde{P}_{i}D_{j}\tilde{P}_{j}
+\indora_{i,j\neq2}D_{2}F_{2}D\tilde{\theta}_{2}\indora_{i=j}D_{i}^{2}\tilde{P}_{i}\\
&+&\indora_{i,j\neq2}D_{2}F_{2}D\tilde{\theta}_{2}\indora_{i\neq j}D_{i}\tilde{P}_{i}D_{j}\tilde{P}_{j}
+\left(1-\indora_{i=j=3}\right)\indora_{i+j\leq6}D_{2}D_{1}F_{2}D\tilde{\theta}_{2}\indora_{i\leq j}D_{j}\tilde{P}_{j}\\
&+&\left(1-\indora_{i=j=3}\right)\indora_{i+j\leq6}D_{2}D_{1}F_{2}D\tilde{\theta}_{2}\indora_{i>j}D_{i}\tilde{P}_{i}
+\indora_{i=1}D_{2}D_{1}F_{2}D\tilde{\theta}_{2}D_{i}\tilde{P}_{i}+\indora_{i=1}D_{i}^{2}F_{2}\\
&+&\indora_{i\geq3}\indora_{j=i}F_{i,2}^{(2)}+\indora_{i\geq 3}\indora_{j\neq i}F_{j,2}^{(1)}F_{i,2}^{(1)}\\
&+&D_{i}R_{2}D_{j}F_{2}+\indora_{j\geq3}D_{i}R_{2}D_{j}F_{4}
+D_{i}F_{2}D_{j}R_{2}+\indora_{j\geq3}D_{i}F_{2}D_{j}F_{4}
+\indora_{i\geq3}D_{i}F_{4}D_{j}R_{2}+\indora_{i\geq3}D_{i}F_{4}D_{j}F_{2}\\
&=&
D_{j}D_{i}R_{2}
+\indora_{i,j\neq2}D_{2}D_{2}F_{2}\left(D\tilde{\theta}_{2}\right)^{2}D_{i}\tilde{P}_{i}D_{j}\tilde{P}_{j}
+\indora_{i,j\neq2}D_{2}F_{2}D^{2}\tilde{\theta}_{2}D_{i}\tilde{P}_{i}D_{j}\tilde{P}_{j}\\
&+&\indora_{i,j\neq2}D_{2}F_{2}D\tilde{\theta}_{2}\indora_{i=j}D_{i}^{2}\tilde{P}_{i}
+\indora_{i,j\neq2}D_{2}F_{2}D\tilde{\theta}_{2}\indora_{i\neq j}D_{i}\tilde{P}_{i}D_{j}\tilde{P}_{j}\\
&+&\left(1-\indora_{i=j=3}\right)\indora_{i+j\leq6}D_{2}D_{1}F_{2}D\tilde{\theta}_{2}\indora_{i\leq j}D_{j}\tilde{P}_{j}+\left(1-\indora_{i=j=3}\right)\indora_{i+j\leq6}D_{2}D_{1}F_{2}D\tilde{\theta}_{2}\indora_{i>j}D_{i}\tilde{P}_{i}\\
&+&\indora_{i=1}D_{2}D_{1}F_{2}D\tilde{\theta}_{2}D_{i}\tilde{P}_{i}+\indora_{i=1}D_{i}^{2}F_{2}\\
&+&\indora_{i\geq3}\indora_{j=i}F_{i,2}^{(2)}+\indora_{i\geq 3}\indora_{j\neq i}F_{j,2}^{(1)}F_{i,2}^{(1)}\\
&+&D_{i}R_{2}D_{j}F_{2}+\indora_{j\geq3}D_{i}R_{2}D_{j}F_{4}\\
&+&D_{i}F_{2}D_{j}R_{2}+\indora_{j\geq3}D_{i}F_{2}D_{j}F_{4}
+\indora_{i\geq3}D_{i}F_{4}D_{j}R_{2}+\indora_{i\geq3}D_{i}F_{4}D_{j}F_{2}\\
&=&\indora_{i,j\neq2}D_{2}^{2}F_{2}\left(D\tilde{\theta}_{2}\right)^{2}D_{i}\tilde{P}_{i}D_{j}\tilde{P}_{j}
+\left(1-\indora_{i=j=3}\right)\indora_{i+j\leq6}D_{2}D_{1}F_{2}D\tilde{\theta}_{2}\indora_{i\leq j}D_{j}\tilde{P}_{j}\\
&+&\left(1-\indora_{i=j=3}\right)\indora_{i+j\leq6}D_{2}D_{1}F_{2}D\tilde{\theta}_{2}\indora_{i>j}D_{i}\tilde{P}_{i}
+\indora_{i=1}D_{2}D_{1}F_{2}D\tilde{\theta}_{2}D_{i}\tilde{P}_{i}+\indora_{i=1}D_{i}^{2}F_{2}\\
&+&\indora_{i,j\neq2}D_{2}F_{2}D^{2}\tilde{\theta}_{2}D_{i}\tilde{P}_{i}D_{j}\tilde{P}_{j}
+\indora_{i,j\neq2}D_{2}F_{2}D\tilde{\theta}_{2}\indora_{i=j}D_{i}^{2}\tilde{P}_{i}
+\indora_{i,j\neq2}D_{2}F_{2}D\tilde{\theta}_{2}\indora_{i\neq j}D_{i}\tilde{P}_{i}D_{j}\tilde{P}_{j}\\
&+&\indora_{i\geq3}\indora_{j=i}F_{i,2}^{(2)}+\indora_{i\geq 3}\indora_{j\neq i}F_{j,2}^{(1)}F_{i,2}^{(1)}\\
&+&D_{i}R_{2}D_{j}F_{2}+\indora_{j\geq3}D_{i}R_{2}D_{j}F_{4}
+D_{i}F_{2}D_{j}R_{2}+\indora_{j\geq3}D_{i}F_{2}D_{j}F_{4}
+\indora_{i\geq3}D_{i}F_{4}D_{j}R_{2}\\
&+&\indora_{i\geq3}D_{i}F_{4}D_{j}F_{2}
+R_{2}^{(2)}\mu_{i}\mu_{j}+\indora_{i=j}r_{2}P_{i}^{(2)}+\indora_{i=j}r_{2}\mu_{i}\mu_{j}\\
D_{j}D_{i}F_{1}&=&
\indora_{i,j\neq2}D_{2}^{2}F_{2}\left(D\tilde{\theta}_{2}\right)^{2}D_{i}\tilde{P}_{i}D_{j}\tilde{P}_{j}
+\left(1-\indora_{i=j=3}\right)\indora_{i+j\leq6}D_{2}D_{1}F_{2}D\tilde{\theta}_{2}\indora_{i\leq j}D_{j}\tilde{P}_{j}\\
&+&\left(1-\indora_{i=j=3}\right)\indora_{i+j\leq6}D_{2}D_{1}F_{2}D\tilde{\theta}_{2}\indora_{i>j}D_{i}\tilde{P}_{i}
+\indora_{i=1}D_{2}D_{1}F_{2}D\tilde{\theta}_{2}D_{i}\tilde{P}_{i}+\indora_{i=1}D_{i}^{2}F_{2}\\
&+&\indora_{i,j\neq2}D_{2}F_{2}D^{2}\tilde{\theta}_{2}D_{i}\tilde{P}_{i}D_{j}\tilde{P}_{j}
+\indora_{i,j\neq2}D_{2}F_{2}D\tilde{\theta}_{2}\indora_{i=j}D_{i}^{2}\tilde{P}_{i}
+\indora_{i,j\neq2}D_{2}F_{2}D\tilde{\theta}_{2}\indora_{i\neq j}D_{i}\tilde{P}_{i}D_{j}\tilde{P}_{j}\\
&+&\indora_{i\geq3}\indora_{j=i}F_{i,2}^{(2)}+\indora_{i\geq 3}\indora_{j\neq i}F_{j,2}^{(1)}F_{i,2}^{(1)}\\
&+&D_{i}R_{2}D_{j}F_{2}+\indora_{j\geq3}D_{i}R_{2}D_{j}F_{4}
+D_{i}F_{2}D_{j}R_{2}+\indora_{j\geq3}D_{i}F_{2}D_{j}F_{4}
+\indora_{i\geq3}D_{i}F_{4}D_{j}R_{2}\\
&+&\indora_{i\geq3}D_{i}F_{4}D_{j}F_{2}
+R_{2}^{(2)}\mu_{i}\mu_{j}+\indora_{i=j}r_{2}P_{i}^{(2)}+\indora_{i=j}r_{2}\mu_{i}\mu_{j}
\end{eqnarray*}


\begin{eqnarray*}
f_{1}\left(i,j\right)&=&
\indora_{i,j\neq2}f_{2}\left(2,2\right)\left(\frac{1}{1-\tilde{\mu}_{2}}\right)^{2}\mu_{i}\mu_{j}
+\left(1-\indora_{i=j=3}\right)\indora_{i+j\leq6}\indora_{i\leq j}f_{2}\left(1,2\right)\frac{1}{1-\tilde{\mu}_{2}}\mu_{j}\\
&+&\left(1-\indora_{i=j=3}\right)\indora_{i+j\leq6}\indora_{i>j}f_{2}\left(1,2\right)\frac{1}{1-\tilde{\mu}_{2}}\mu_{i}
+\indora_{i=1}f_{2}\left(1,2\right)\frac{1}{1-\tilde{\mu}_{2}}\mu_{i}+\indora_{i=1}f_{2}\left(1,1\right)\\
&+&\indora_{i,j\neq2}f_{2}\left(2\right)\tilde{\theta}_{2}^{(2)}\tilde{\mu}_{i}\tilde{\mu}_{j}
+\indora_{i,j\neq2}\indora_{i=j}f_{2}\left(2\right)\frac{1}{1-\tilde{\mu}_{2}}\tilde{P}_{i}^{(2)}
+\indora_{i,j\neq2}\indora_{i\neq j}f_{2}\left(2\right)\frac{1}{1-\tilde{\mu}_{2}}\tilde{\mu}_{i}\tilde{\mu}_{j}\\
&+&\indora_{i\geq3}\indora_{j=i}F_{i,2}^{(2)}+\indora_{i\geq 3}\indora_{j\neq i}F_{j,2}^{(1)}F_{i,2}^{(1)}\\
&+&r_{2}\tilde{\mu}_{i}f_{2}\left(j\right)
+\indora_{j\geq3}r_{2}\tilde{\mu}_{i}F_{j,2}^{(1)}
+f_{2}\left(i\right)r_{2}\tilde{\mu}_{j}
+\indora_{j\geq3}f_{2}\left(i\right)F_{j,2}^{(1)}
+\indora_{i\geq3}F_{i,2}^{(1)}r_{2}\tilde{\mu}_{j}\\
&+&\indora_{i\geq3}F_{i,2}^{(1)}f_{2}\left(j\right)
+R_{2}^{(2)}\mu_{i}\mu_{j}+\indora_{i=j}r_{2}P_{i}^{(2)}+\indora_{i=j}r_{2}\tilde{\mu}_{i}\mu_{j}\\
&=&\indora_{i=1}f_{2}\left(1,1\right)
+\left[\left(1-\indora_{i=j=3}\right)\indora_{i+j\leq6}\indora_{i\leq j}\frac{1}{1-\tilde{\mu}_{2}}\mu_{j}
+\left(1-\indora_{i=j=3}\right)\indora_{i+j\leq6}\indora_{i>j}\frac{1}{1-\tilde{\mu}_{2}}\mu_{i}\right.\\
&+&\left.\indora_{i=1}\frac{1}{1-\tilde{\mu}_{2}}\mu_{i}\right]f_{2}\left(1,2\right)
+\indora_{i,j\neq2}\left(\frac{1}{1-\tilde{\mu}_{2}}\right)^{2}\mu_{i}\mu_{j}f_{2}\left(2,2\right)\\
&+&\left[\indora_{i,j\neq2}\tilde{\theta}_{2}^{(2)}\tilde{\mu}_{i}\tilde{\mu}_{j}
+\indora_{i,j\neq2}\indora_{i=j}\frac{1}{1-\tilde{\mu}_{2}}\tilde{P}_{i}^{(2)}
+\indora_{i,j\neq2}\indora_{i\neq j}\frac{1}{1-\tilde{\mu}_{2}}\tilde{\mu}_{i}\tilde{\mu}_{j}\right]f_{2}\left(2\right)\\
&+&\left[r_{2}\tilde{\mu}_{i}
+\indora_{i\geq3}F_{i,2}^{(1)}\right]f_{2}\left(j\right)
+\left[r_{2}\tilde{\mu}_{j}
+\indora_{j\geq3}F_{j,2}^{(1)}\right]f_{2}\left(i\right)
\\
&+&\left[R_{2}^{(2)}
+\indora_{i=j}r_{2}\right]\tilde{\mu}_{i}\mu_{j}\\
&+&\indora_{j\geq3}F_{j,2}^{(1)}\left[\indora_{j\neq i}F_{i,2}^{(1)}
+r_{2}\tilde{\mu}_{i}\right]+\indora_{i\geq3}\indora_{j=i}F_{i,2}^{(2)}\\
&+&r_{2}\left[\indora_{i=j}P_{i}^{(2)}
+\indora_{i\geq3}F_{i,2}^{(1)}\tilde{\mu}_{j}\right]
\end{eqnarray*}


\begin{eqnarray*}
f_{1}\left(i,j\right)&=&\indora_{i=1}f_{2}\left(1,1\right)
+\left[\left(1-\indora_{i=j=3}\right)\indora_{i+j\leq6}\indora_{i\leq j}\frac{\mu_{j}}{1-\tilde{\mu}_{2}}
+\left(1-\indora_{i=j=3}\right)\indora_{i+j\leq6}\indora_{i>j}\frac{\mu_{i}}{1-\tilde{\mu}_{2}}
+\indora_{i=1}\frac{\mu_{i}}{1-\tilde{\mu}_{2}}\right]f_{2}\left(1,2\right)\\
&+&
\indora_{i,j\neq2}\left(\frac{1}{1-\tilde{\mu}_{2}}\right)^{2}\mu_{i}\mu_{j}f_{2}\left(2,2\right)
+\left[\indora_{i,j\neq2}\tilde{\theta}_{2}^{(2)}\tilde{\mu}_{i}\tilde{\mu}_{j}
+\indora_{i,j\neq2}\indora_{i=j}\frac{\tilde{P}_{i}^{(2)}}{1-\tilde{\mu}_{2}}
+\indora_{i,j\neq2}\indora_{i\neq j}\frac{\tilde{\mu}_{i}\tilde{\mu}_{j}}{1-\tilde{\mu}_{2}}\right]f_{2}\left(2\right)\\
&+&\left[r_{2}\tilde{\mu}_{i}
+\indora_{i\geq3}F_{i,2}^{(1)}\right]f_{2}\left(j\right)
+\left[r_{2}\tilde{\mu}_{j}
+\indora_{j\geq3}F_{j,2}^{(1)}\right]f_{2}\left(i\right)
+\left[R_{2}^{(2)}
+\indora_{i=j}r_{2}\right]\tilde{\mu}_{i}\mu_{j}\\
&+&\indora_{j\geq3}F_{j,2}^{(1)}\left[\indora_{j\neq i}F_{i,2}^{(1)}
+r_{2}\tilde{\mu}_{i}\right]
+r_{2}\left[\indora_{i=j}P_{i}^{(2)}
+\indora_{i\geq3}F_{i,2}^{(1)}\tilde{\mu}_{j}\right]
+\indora_{i\geq3}\indora_{j=i}F_{i,2}^{(2)}
\end{eqnarray*}
%D_{j}F_{4}=F_{j,2}^{(1)}
%D_{i}F_{4}=F_{i,2}^{(1)}
%D_{j}D_{i}F_{4}=\indora_{j\geq3}\indora_{i\geq3}F_{i,2}^{(1)}F_{j,2}^{(1)}

\newpage


\begin{eqnarray*}
D_{j}D_{i}F_{3}\left(z_{1},z_{2};\tau_{1}\right)&=&\indora_{i\geq3}\indora_{j=i}F_{i,1}^{(2)}+\indora_{i\geq 3}\indora_{j\neq i}F_{j,1}^{(1)}F_{i,1}^{(1)}
\end{eqnarray*}


\begin{eqnarray*}
D_{j}D_{i}F_{2}&=&
D_{j}D_{i}R_{1}
+D_{j}D_{i}F_{1}
+\indora_{i\geq3}D_{j}D_{i}F_{3}
+D_{i}R_{1}D_{j}F_{1}
+\indora_{j\geq3}D_{i}R_{1}D_{j}F_{3}
+D_{i}F_{1}D_{j}R_{1}\\
&+&\indora_{j\geq3}D_{i}F_{1}D_{j}F_{3}
+\indora_{i\geq3}D_{i}F_{3}D_{j}R_{1}+\indora_{i\geq3}D_{i}F_{3}D_{j}F_{1}\\
&=&R_{1}^{(2)}\mu_{i}\mu_{j}+\indora_{i=j}r_{1}P_{i}^{(2)}+\indora_{i=j}r_{1}\mu_{i}\mu_{j}\\
&+&\indora_{i,j\neq1}D_{1}D_{1}F_{1}\left(D\tilde{\theta}_{1}\right)^{2}D_{i}\tilde{P}_{i}D_{j}\tilde{P}_{j}
+\indora_{i,j\neq1}D_{1}F_{1}D^{2}\tilde{\theta}_{1}D_{i}\tilde{P}_{i}D_{j}\tilde{P}_{j}\\
&+&\indora_{i,j\neq1}D_{1}F_{1}D\tilde{\theta}_{1}\indora_{i=j}D_{i}^{2}\tilde{P}_{i}
+\indora_{i,j\neq1}D_{1}F_{1}D\tilde{\theta}_{1}\indora_{i\neq j}D_{i}\tilde{P}_{i}D_{j}\tilde{P}_{j}\\
&+&\left(1-\indora_{i=j=3}\right)\indora_{i+j\leq6}D_{1}D_{2}F_{1}D\tilde{\theta}_{1}\indora_{i\leq j}D_{j}\tilde{P}_{j}
+\left(1-\indora_{i=j=3}\right)\indora_{i+j\leq6}D_{1}D_{2}F_{1}D\tilde{\theta}_{1}\indora_{i>j}D_{i}\tilde{P}_{i}\\
&+&\indora_{i=2}D_{1}D_{2}F_{1}D\tilde{\theta}_{1}D_{i}\tilde{P}_{i}+\indora_{i=2}D_{i}^{2}F_{1}\\
&+&\indora_{i\geq3}\indora_{j=i}F_{i,1}^{(2)}+\indora_{i\geq 3}\indora_{j\neq i}F_{j,1}^{(1)}F_{i,1}^{(1)}\\
&+&D_{i}R_{1}D_{j}F_{1}
+\indora_{j\geq3}D_{i}R_{1}D_{j}F_{3}\\
&+&D_{i}F_{1}D_{j}R_{1}+\indora_{j\geq3}D_{i}F_{1}D_{j}F_{3}
+\indora_{i\geq3}D_{i}F_{3}D_{j}R_{1}+\indora_{i\geq3}D_{i}F_{3}D_{j}F_{1}\\
&=&\indora_{i,j\neq1}D_{1}^{2}F_{1}\left(D\tilde{\theta}_{1}\right)^{2}D_{i}\tilde{P}_{i}D_{j}\tilde{P}_{j}
+\left(1-\indora_{i=j=3}\right)\indora_{i+j\leq6}D_{1}D_{2}F_{1}D\tilde{\theta}_{1}\indora_{i\leq j}D_{j}\tilde{P}_{j}\\
&+&\left(1-\indora_{i=j=3}\right)\indora_{i+j\leq6}D_{1}D_{2}F_{1}D\tilde{\theta}_{1}\indora_{i>j}D_{i}\tilde{P}_{i}
+\indora_{i=2}D_{1}D_{2}F_{1}D\tilde{\theta}_{1}D_{i}\tilde{P}_{i}+\indora_{i=2}D_{i}^{2}F_{1}\\
&+&\indora_{i,j\neq1}D_{1}F_{1}D^{2}\tilde{\theta}_{1}D_{i}\tilde{P}_{i}D_{j}\tilde{P}_{j}
+\indora_{i,j\neq1}D_{1}F_{1}D\tilde{\theta}_{1}\indora_{i\neq j}D_{i}\tilde{P}_{i}D_{j}\tilde{P}_{j}
+\indora_{i,j\neq1}D_{1}F_{1}D\tilde{\theta}_{1}\indora_{i=j}D_{i}^{2}\tilde{P}_{i}\\
&+&R_{1}^{(2)}\mu_{i}\mu_{j}+\indora_{i=j}r_{1}P_{i}^{(2)}+\indora_{i=j}r_{1}\mu_{i}\mu_{j}+D_{i}R_{1}D_{j}F_{1}\\
&+&\indora_{i\geq3}\indora_{j=i}F_{i,1}^{(2)}+\indora_{i\geq 3}\indora_{j\neq i}F_{j,1}^{(1)}F_{i,1}^{(1)}\\
&+&\indora_{j\geq3}D_{i}R_{1}D_{j}F_{3}
+D_{i}F_{1}D_{j}R_{1}\\
&+&\indora_{j\geq3}D_{i}F_{1}D_{j}F_{3}
+\indora_{i\geq3}D_{i}F_{3}D_{j}R_{1}+\indora_{i\geq3}D_{i}F_{3}D_{j}F_{1}
\end{eqnarray*}



\begin{eqnarray*}
D_{j}D_{i}F_{2}&=&
\indora_{i,j\neq1}D_{1}^{2}F_{1}\left(D\tilde{\theta}_{1}\right)^{2}D_{i}\tilde{P}_{i}D_{j}\tilde{P}_{j}
+\left(1-\indora_{i=j=3}\right)\indora_{i+j\leq6}D_{1}D_{2}F_{1}D\tilde{\theta}_{1}\indora_{i\leq j}D_{j}\tilde{P}_{j}
+\indora_{i=2}D_{i}^{2}F_{1}\\
&+&\left(1-\indora_{i=j=3}\right)\indora_{i+j\leq6}D_{1}D_{2}F_{1}D\tilde{\theta}_{1}\indora_{i>j}D_{i}\tilde{P}_{i}
+\indora_{i=2}D_{1}D_{2}F_{1}D\tilde{\theta}_{1}D_{i}\tilde{P}_{i}
+\indora_{i,j\neq1}D_{1}F_{1}D^{2}\tilde{\theta}_{1}D_{i}\tilde{P}_{i}D_{j}\tilde{P}_{j}\\
&+&\indora_{i,j\neq1}D_{1}F_{1}D\tilde{\theta}_{1}\indora_{i\neq j}D_{i}\tilde{P}_{i}D_{j}\tilde{P}_{j}
+\indora_{i,j\neq1}D_{1}F_{1}D\tilde{\theta}_{1}\indora_{i=j}D_{i}^{2}\tilde{P}_{i}
+R_{1}^{(2)}\mu_{i}\mu_{j}+\indora_{i=j}r_{1}P_{i}^{(2)}+\indora_{i=j}r_{1}\mu_{i}\mu_{j}\\
&+&D_{i}R_{1}D_{j}F_{1}\\
&+&\indora_{i\geq3}\indora_{j=i}F_{i,1}^{(2)}+\indora_{i\geq 3}\indora_{j\neq i}F_{j,1}^{(1)}F_{i,1}^{(1)}\\
&+&\indora_{j\geq3}D_{i}R_{1}D_{j}F_{3}
+D_{i}F_{1}D_{j}R_{1}+\indora_{j\geq3}D_{i}F_{1}D_{j}F_{3}\\
&+&\indora_{i\geq3}D_{i}F_{3}D_{j}R_{1}+\indora_{i\geq3}D_{i}F_{3}D_{j}F_{1}
\end{eqnarray*}

\begin{eqnarray*}
f_{2}\left(i,j\right)&=&
\indora_{i,j\neq1}\left(\frac{1}{1-\tilde{\mu}_{1}}\right)^{2}\tilde{\mu}_{i}\tilde{\mu}_{j}f_{1}\left(1,1\right)
+\left[\left(1-\indora_{i=j=3}\right)\indora_{i+j\leq6}\indora_{i\leq j}\frac{\tilde{\mu}_{j}}{1-\tilde{\mu}_{1}}
+\left(1-\indora_{i=j=3}\right)\indora_{i+j\leq6}\indora_{i>j}\frac{\tilde{\mu}_{i}}{1-\tilde{\mu}_{1}}\right.
\\
&+&\left.\indora_{i=2}\frac{\tilde{\mu}_{i}}{1-\tilde{\mu}_{1}}\right]f_{1}\left(1,2\right)
+\indora_{i=2}f_{1}\left(2,2\right)
+\left[\indora_{i,j\neq1}\tilde{\theta}_{1}^{(2)}\tilde{\mu}_{i}\tilde{\mu}_{j}
+\indora_{i,j\neq1}\indora_{i\neq j}\frac{\tilde{\mu}_{i}\tilde{\mu}_{j}}{1-\tilde{\mu}_{1}}
+\indora_{i,j\neq1}\indora_{i=j}\frac{\tilde{P}_{i}^{(2)}}{1-\tilde{\mu}_{1}}\right]f_{1}\left(1\right)\\
&+&\left[r_{1}\mu_{i}+\indora_{i\geq3}F_{i,1}^{(1)}\right]f_{1}\left(j\right)
+\left[\indora_{j\geq3}F_{j,1}^{(1)}+r_{1}\mu_{j}\right]f_{1}\left(i\right)
+\left[R_{1}^{(2)}+\indora_{i=j}\right]\tilde{\mu}_{i}\tilde{\mu}_{j}
+\indora_{i\geq3}F_{i,1}^{(1)}\left[r_{1}\mu_{j}
+\indora_{j\neq i}F_{j,1}^{(1)}\right]\\
&+&r_{1}\left[\indora_{j\geq3}\mu_{i}F_{j,1}^{(1)}
+\indora_{i=j}P_{i}^{(2)}\right]
+\indora_{i\geq3}\indora_{j=i}F_{i,1}^{(2)}
\\
\end{eqnarray*}

%\frac{1}{1-\tilde{\mu}_{1}}
%D_{j}F_{3}=F_{j,1}^{(1)}
%D_{i}F_{3}=F_{i,1}^{(1)}
%D_{j}D_{i}F_{3}=\indora_{j\geq3}\indora_{i\geq3}F_{i,1}^{(1)}F_{j,1}^{(1)}


\begin{eqnarray*}
D_{j}D_{i}F_{2}\left(z_{3},z_{4};\tau_{4}\right)&=&\indora_{i\leq2}\indora_{j=i}F_{i,4}^{(2)}+\indora_{i\leq 2}\indora_{j\neq i}F_{j,4}^{(1)}F_{i,4}^{(1)}
\end{eqnarray*}

\begin{eqnarray*}%\label{Ec.Derivadas.Segundo.Orden.Doble.Transferencia}
D_{j}D_{i}F_{3}&=&R_{3}^{(2)}\mu_{i}\mu_{j}+\indora_{i=j}r_{3}P_{i}^{(2)}+\indora_{i=j}r_{3}\mu_{i}\mu_{j}\\
&+&\indora_{i\leq2}\indora_{j=i}F_{i,4}^{(2)}+\indora_{i\leq 2}\indora_{j\neq i}F_{j,4}^{(1)}F_{i,4}^{(1)}\\
&+&\indora_{i,j\neq4}D_{4}D_{4}F_{4}\left(D\tilde{\theta}_{4}\right)^{2}D_{i}\tilde{P}_{i}D_{j}\tilde{P}_{j}\\
&+&\indora_{i,j\neq4}D_{4}F_{4}D^{2}\tilde{\theta}_{4}D_{i}\tilde{P}_{i}D_{j}\tilde{P}_{j}
+\indora_{i,j\neq4}D_{4}F_{4}D\tilde{\theta}_{4}\indora_{i=j}D_{i}^{2}\tilde{P}_{i}
+\indora_{i,j\neq4}D_{4}F_{4}D\tilde{\theta}_{4}\indora_{i\neq j}D_{i}\tilde{P}_{i}D_{j}\tilde{P}_{j}\\
&+&\left(1-\indora_{i=j=2}\right)\indora_{i+j\geq4}D_{4}D_{3}F_{4}D\tilde{\theta}_{4}\indora_{i\leq j}D_{i}\tilde{P}_{i}
+\left(1-\indora_{i=j=2}\right)\indora_{i+j\geq4}D_{4}D_{3}F_{4}D\tilde{\theta}_{4}\indora_{i>j}D_{j}\tilde{P}_{j}\\
&+&\indora_{i=3}D_{4}D_{3}F_{4}D\tilde{\theta}_{4}D_{i}\tilde{P}_{i}+\indora_{i=3}D_{i}^{2}F_{4}
+\indora_{j\leq2}D_{i}R_{4}D_{j}F_{2}+D_{i}R_{4}D_{j}F_{4}
+D_{i}F_{4}D_{j}R_{4}+\indora_{j\leq2}D_{i}F_{4}D_{j}F_{2}\\
&+&\indora_{i\leq2}D_{i}F_{2}D_{j}R_{4}+\indora_{i\leq2}D_{i}F_{2}D_{j}F_{4}\\
&=&
\indora_{i,j\neq4}D_{4}^{2}F_{4}\left(D\tilde{\theta}_{4}\right)^{2}D_{i}\tilde{P}_{i}D_{j}\tilde{P}_{j}
+\left(1-\indora_{i=j=2}\right)\indora_{i+j\geq4}D_{4}D_{3}F_{4}D\tilde{\theta}_{4}\indora_{i\leq j}D_{i}\tilde{P}_{i}
\\
&+&\left(1-\indora_{i=j=2}\right)\indora_{i+j\geq4}D_{4}D_{3}F_{4}D\tilde{\theta}_{4}\indora_{i>j}D_{j}\tilde{P}_{j}+\indora_{i=3}D_{4}D_{3}F_{4}D\tilde{\theta}_{4}D_{i}\tilde{P}_{i}+\indora_{i=3}D_{i}^{2}F_{4}\\
&+&\indora_{i,j\neq4}D_{4}F_{4}D^{2}\tilde{\theta}_{4}D_{i}\tilde{P}_{i}D_{j}\tilde{P}_{j}
+\indora_{i,j\neq4}D_{4}F_{4}D\tilde{\theta}_{4}\indora_{i=j}D_{i}^{2}\tilde{P}_{i}
+\indora_{i,j\neq4}D_{4}F_{4}D\tilde{\theta}_{4}\indora_{i\neq j}D_{i}\tilde{P}_{i}D_{j}\tilde{P}_{j}\\
&+&D_{i}R_{4}D_{j}F_{4}
+D_{i}F_{4}D_{j}R_{4}+\indora_{j\leq2}D_{i}F_{4}D_{j}F_{2}+\indora_{j\leq2}D_{i}R_{4}D_{j}F_{2}\\
&+&\indora_{i\leq2}D_{i}F_{2}D_{j}R_{4}+\indora_{i\leq2}D_{i}F_{2}D_{j}F_{4}
+R_{3}^{(2)}\mu_{i}\mu_{j}+\indora_{i=j}r_{3}P_{i}^{(2)}+\indora_{i=j}r_{3}\mu_{i}\mu_{j}\\
&+&\indora_{i\leq2}\indora_{j=i}F_{i,4}^{(2)}+\indora_{i\leq 2}\indora_{j\neq i}F_{j,4}^{(1)}F_{i,4}^{(1)}
\end{eqnarray*}




\begin{eqnarray*}%\label{Ec.Derivadas.Segundo.Orden.Doble.Transferencia}
D_{j}D_{i}F_{3}&=&\indora_{i,j\neq4}D_{4}^{2}F_{4}\left(D\tilde{\theta}_{4}\right)^{2}D_{i}\tilde{P}_{i}D_{j}\tilde{P}_{j}
+\left(1-\indora_{i=j=2}\right)\indora_{i+j\geq4}D_{4}D_{3}F_{4}D\tilde{\theta}_{4}\indora_{i\leq j}D_{i}\tilde{P}_{i}
\\
&+&\left(1-\indora_{i=j=2}\right)\indora_{i+j\geq4}D_{4}D_{3}F_{4}D\tilde{\theta}_{4}\indora_{i>j}D_{j}\tilde{P}_{j}+\indora_{i=3}D_{4}D_{3}F_{4}D\tilde{\theta}_{4}D_{i}\tilde{P}_{i}+\indora_{i=3}D_{i}^{2}F_{4}\\
&+&\indora_{i,j\neq4}D_{4}F_{4}D^{2}\tilde{\theta}_{4}D_{i}\tilde{P}_{i}D_{j}\tilde{P}_{j}
+\indora_{i,j\neq4}D_{4}F_{4}D\tilde{\theta}_{4}\indora_{i=j}D_{i}^{2}\tilde{P}_{i}
+\indora_{i,j\neq4}D_{4}F_{4}D\tilde{\theta}_{4}\indora_{i\neq j}D_{i}\tilde{P}_{i}D_{j}\tilde{P}_{j}\\
&+&D_{i}R_{4}D_{j}F_{4}
+D_{i}F_{4}D_{j}R_{4}+\indora_{j\leq2}D_{i}F_{4}D_{j}F_{2}+\indora_{j\leq2}D_{i}R_{4}D_{j}F_{2}\\
&+&\indora_{i\leq2}D_{i}F_{2}D_{j}R_{4}+\indora_{i\leq2}D_{i}F_{2}D_{j}F_{4}
+R_{3}^{(2)}\mu_{i}\mu_{j}+\indora_{i=j}r_{3}P_{i}^{(2)}+\indora_{i=j}r_{3}\mu_{i}\mu_{j}\\
&+&\indora_{i\leq2}\indora_{j=i}F_{i,4}^{(2)}+\indora_{i\leq 2}\indora_{j\neq i}F_{j,4}^{(1)}F_{i,4}^{(1)}
\end{eqnarray*}


\begin{eqnarray*}%\label{Ec.Derivadas.Segundo.Orden.Doble.Transferencia}
f_{3}\left(i,j\right)&=&
\indora_{i=3}f_{4}\left(3,3\right)
+\left[\left(1-\indora_{i=j=2}\right)\indora_{i+j\geq4}\indora_{i\leq j}\frac{\tilde{\mu}_{i}}{1-\tilde{\mu}_{4}}
+\left(1-\indora_{i=j=2}\right)\indora_{i+j\geq4}\indora_{i>j}\frac{\tilde{\mu}_{j}}{1-\tilde{\mu}_{4}}
+\indora_{i=3}\frac{\tilde{\mu}_{i}}{1-\tilde{\mu}_{4}}\right]f_{4}\left(3,4\right)\\
&+&\indora_{i,j\neq4}f_{4}\left(4,4\right)\left(\frac{1}{1-\tilde{\mu}_{4}}\right)^{2}\tilde{\mu}_{i}\tilde{\mu}_{j}
+\left[\indora_{i,j\neq4}\tilde{\theta}_{4}^{(2)}\tilde{\mu}_{i}\tilde{\mu}_{j}
+\indora_{i,j\neq4}\indora_{i=j}\frac{\tilde{P}_{i}^{(2)}}{1-\tilde{\mu}_{4}}
+\indora_{i,j\neq4}\indora_{i\neq j}\frac{\tilde{\mu}_{i}\tilde{\mu}_{j}}{1-\tilde{\mu}_{4}}\right]f_{4}\left(4\right)\\
&+&\left[r_{4}\tilde{\mu}_{i}+\indora_{i\leq2}F_{i,4}^{(1)}\right]f_{4}\left(j\right)
+\left[r_{4}\tilde{\mu}_{j}+\indora_{j\leq2}F_{j,4}^{(1)}\right]f_{4}\left(i\right)
+\left[R_{4}^{(2)}+\indora_{i=j}r_{4}\right]\tilde{\mu}_{i}\tilde{\mu}_{j}\\
&+& \indora_{i\leq2}F_{i,4}^{(1)}\left[r_{4}\tilde{\mu}_{j}
+\indora_{j\neq i}F_{j,4}^{(1)}\right]
+r_{4}\left[\indora_{i=j}P_{i}^{(2)}+\indora_{j\leq2}\tilde{\mu}_{i}F_{j,4}^{(1)}\right]
+\indora_{i\leq2}\indora_{j=i}F_{i,4}^{(2)}
\end{eqnarray*}



\begin{eqnarray*}
D_{j}D_{i}F_{1}\left(z_{3},z_{4};\tau_{3}\right)&=&\indora_{i\leq2}\indora_{j=i}F_{i,3}^{(2)}+\indora_{i\leq 2}\indora_{j\neq i}F_{j,3}^{(1)}F_{i,3}^{(1)}
\end{eqnarray*}



\begin{eqnarray*}%\label{Ec.Derivadas.Segundo.Orden.Doble.Transferencia}
D_{j}D_{i}F_{4}&=&D_{j}D_{i}R_{3}+\indora_{i\leq2}D_{j}D_{i}F_{1}
+\indora_{i=4}D_{3}D_{4}F_{3}D\tilde{\theta}_{3}D_{i}\tilde{P}_{i}+\indora_{i=4}D_{i}^{2}F_{3}\\
&+&\indora_{i,j\neq3}D_{3}D_{3}F_{3}\left(D\tilde{\theta}_{3}\right)^{2}D_{i}\tilde{P}_{i}D_{j}\tilde{P}_{j}
+\indora_{i,j\neq3}D_{3}F_{3}D^{2}\tilde{\theta}_{3}D_{i}\tilde{P}_{i}D_{j}\tilde{P}_{j}
+\indora_{i,j\neq3}D_{3}F_{3}D\tilde{\theta}_{3}\indora_{i=j}D_{i}^{2}\tilde{P}_{i}\\
&+&\indora_{i,j\neq3}D_{3}F_{3}D\tilde{\theta}_{3}\indora_{i\neq j}D_{i}\tilde{P}_{i}D_{j}\tilde{P}_{j}
+\indora_{i+j\geq5}D_{3}D_{4}F_{3}D\tilde{\theta}_{3}\indora_{i\leq j}D_{i}\tilde{P}_{i}
+\indora_{i+j\geq5}D_{3}D_{4}F_{3}D\tilde{\theta}_{3}\indora_{i>j}D_{j}\tilde{P}_{j}\\
&+&\indora_{j\leq2}D_{i}R_{3}D_{j}F_{1}+D_{i}R_{3}D_{j}F_{3}
+D_{i}F_{3}D_{j}R_{3}+\indora_{j\leq2}D_{i}F_{3}D_{j}F_{1}
+\indora_{i\leq2}D_{i}F_{1}D_{j}R_{3}+\indora_{i\leq2}D_{i}F_{1}D_{j}F_{3}\\
&=&\indora_{i,j\neq3}D_{3}D_{3}F_{3}\left(D\tilde{\theta}_{3}\right)^{2}D_{i}\tilde{P}_{i}D_{j}\tilde{P}_{j}
+\indora_{i=4}D_{3}D_{4}F_{3}D\tilde{\theta}_{3}D_{i}\tilde{P}_{i}+\indora_{i=4}D_{i}^{2}F_{3}
+\indora_{i+j\geq5}D_{3}D_{4}F_{3}D\tilde{\theta}_{3}\indora_{i>j}D_{j}\tilde{P}_{j}\\
&+&\indora_{i+j\geq5}D_{3}D_{4}F_{3}D\tilde{\theta}_{3}\indora_{i\leq j}D_{i}\tilde{P}_{i}
+\indora_{i,j\neq3}D_{3}F_{3}D^{2}\tilde{\theta}_{3}D_{i}\tilde{P}_{i}D_{j}\tilde{P}_{j}
+\indora_{i,j\neq3}D_{3}F_{3}D\tilde{\theta}_{3}\indora_{i=j}D_{i}^{2}\tilde{P}_{i}\\
&+&\indora_{i,j\neq3}D_{3}F_{3}D\tilde{\theta}_{3}\indora_{i\neq j}D_{i}\tilde{P}_{i}D_{j}\tilde{P}_{j}
+\indora_{i\leq2}D_{i}F_{1}D_{j}R_{3}+\indora_{i\leq2}D_{i}F_{1}D_{j}F_{3}\\
&+&\indora_{j\leq2}D_{i}R_{3}D_{j}F_{1}+D_{i}R_{3}D_{j}F_{3}
+D_{i}F_{3}D_{j}R_{3}+\indora_{j\leq2}D_{i}F_{3}D_{j}F_{1}
+D_{j}D_{i}R_{3}\\
&+&\indora_{i\leq2}\indora_{j=i}F_{i,3}^{(2)}+\indora_{i\leq 2}\indora_{j\neq i}F_{j,3}^{(1)}F_{i,3}^{(1)}
\end{eqnarray*}

\begin{eqnarray*}
D_{j}D_{i}F_{4}&=&\indora_{i,j\neq3}D_{3}D_{3}F_{3}\left(D\tilde{\theta}_{3}\right)^{2}D_{i}\tilde{P}_{i}D_{j}\tilde{P}_{j}
+\indora_{i=4}D_{3}D_{4}F_{3}D\tilde{\theta}_{3}D_{i}\tilde{P}_{i}+\indora_{i=4}D_{i}^{2}F_{3}
+\indora_{i+j\geq5}D_{3}D_{4}F_{3}D\tilde{\theta}_{3}\indora_{i>j}D_{j}\tilde{P}_{j}\\
&+&\indora_{i+j\geq5}D_{3}D_{4}F_{3}D\tilde{\theta}_{3}\indora_{i\leq j}D_{i}\tilde{P}_{i}
+\indora_{i,j\neq3}D_{3}F_{3}D^{2}\tilde{\theta}_{3}D_{i}\tilde{P}_{i}D_{j}\tilde{P}_{j}
+\indora_{i,j\neq3}D_{3}F_{3}D\tilde{\theta}_{3}\indora_{i=j}D_{i}^{2}\tilde{P}_{i}\\
&+&\indora_{i,j\neq3}D_{3}F_{3}D\tilde{\theta}_{3}\indora_{i\neq j}D_{i}\tilde{P}_{i}D_{j}\tilde{P}_{j}
+\indora_{i\leq2}D_{i}F_{1}D_{j}R_{3}+\indora_{i\leq2}D_{i}F_{1}D_{j}F_{3}\\
&+&\indora_{j\leq2}D_{i}R_{3}D_{j}F_{1}+D_{i}R_{3}D_{j}F_{3}
+D_{i}F_{3}D_{j}R_{3}+\indora_{j\leq2}D_{i}F_{3}D_{j}F_{1}
+D_{j}D_{i}R_{3}\\
&+&\indora_{i\leq2}\indora_{j=i}F_{i,3}^{(2)}+\indora_{i\leq 2}\indora_{j\neq i}F_{j,3}^{(1)}F_{i,3}^{(1)}
\end{eqnarray*}

\begin{eqnarray*}
f_{4}\left(i,j\right)&=&
\indora_{i,j\neq3}f_{3}\left(3,3\right)\left(\frac{1}{1-\tilde{\mu}_{3}}\right)^{2}\tilde{\mu}_{i}\tilde{\mu}_{j}
+\left[\left(1-\indora_{i=j=2}\right)\indora_{i+j\geq5}\indora_{i\leq j}\frac{\tilde{\mu}_{i}}{1-\tilde{\mu}_{3}}
+\left(1-\indora_{i=j=2}\right)\indora_{i+j\geq5}\indora_{i>j}\frac{\tilde{\mu}_{j}}{1-\tilde{\mu}_{3}}\right.\\
&+&\left.\indora_{i=4}\frac{\tilde{\mu}_{i}}{1-\tilde{\mu}_{3}}\right]f_{3}\left(3,4\right)
+\indora_{i=4}f_{3}\left(4,4\right)
+\left[\indora_{i,j\neq3}\tilde{\theta}_{3}^{(2)}\tilde{\mu}_{i}\tilde{\mu}_{j}
+\indora_{i,j\neq3}\indora_{i=j}\frac{\tilde{P}_{i}^{(2)}}{1-\tilde{\mu}_{3}}
+\indora_{i,j\neq3}\indora_{i\neq j}\frac{\tilde{\mu}_{i}\tilde{\mu}_{j}}{1-\tilde{\mu}_{3}}\right]f_{3}\left(3\right)\\
&+&\left[r_{3}\tilde{\mu}_{i}+\indora_{i\leq2}F_{i,3}^{(1)}\right]f_{3}\left(j\right)
+\left[r_{3}\tilde{\mu}_{j}+\indora_{j\leq2}F_{j,3}^{(1)}\right]f_{3}\left(i\right)
+\left[R_{3}^{(2)}+\indora_{i=j}r_{3}\right]\tilde{\mu}_{i}\tilde{\mu}_{j}\\
&+&\indora_{i\leq2}F_{i,3}^{(1)}\left[r_{3}\tilde{\mu}_{j}+\indora_{j\neq i}F_{j,3}^{(1)}\right]
+r_{3}\left[\indora_{i=j}P_{i}^{(2)}+\indora_{j\leq2}\tilde{\mu}_{i}F_{j,3}^{(1)}\right]
+\indora_{i\leq2}\indora_{j=i}F_{i,3}^{(2)}
\end{eqnarray*}


\begin{eqnarray*}
f_{1}\left(i,j\right)&=&\indora_{i=1}f_{2}\left(1,1\right)
+\left[\left(1-\indora_{i=j=3}\right)\indora_{i+j\leq6}\indora_{i\leq j}\frac{\mu_{j}}{1-\tilde{\mu}_{2}}
+\left(1-\indora_{i=j=3}\right)\indora_{i+j\leq6}\indora_{i>j}\frac{\mu_{i}}{1-\tilde{\mu}_{2}}
+\indora_{i=1}\frac{\mu_{i}}{1-\tilde{\mu}_{2}}\right]f_{2}\left(1,2\right)\\
&+&
\indora_{i,j\neq2}\left(\frac{1}{1-\tilde{\mu}_{2}}\right)^{2}\mu_{i}\mu_{j}f_{2}\left(2,2\right)
+\left[\indora_{i,j\neq2}\tilde{\theta}_{2}^{(2)}\tilde{\mu}_{i}\tilde{\mu}_{j}
+\indora_{i,j\neq2}\indora_{i=j}\frac{\tilde{P}_{i}^{(2)}}{1-\tilde{\mu}_{2}}
+\indora_{i,j\neq2}\indora_{i\neq j}\frac{\tilde{\mu}_{i}\tilde{\mu}_{j}}{1-\tilde{\mu}_{2}}\right]f_{2}\left(2\right)\\
&+&\left[r_{2}\tilde{\mu}_{i}
+\indora_{i\geq3}F_{i,2}^{(1)}\right]f_{2}\left(j\right)
+\left[r_{2}\tilde{\mu}_{j}
+\indora_{j\geq3}F_{j,2}^{(1)}\right]f_{2}\left(i\right)
+\left[R_{2}^{(2)}
+\indora_{i=j}r_{2}\right]\tilde{\mu}_{i}\mu_{j}\\
&+&\indora_{j\geq3}F_{j,2}^{(1)}\left[\indora_{j\neq i}F_{i,2}^{(1)}
+r_{2}\tilde{\mu}_{i}\right]
+r_{2}\left[\indora_{i=j}P_{i}^{(2)}
+\indora_{i\geq3}F_{i,2}^{(1)}\tilde{\mu}_{j}\right]
+\indora_{i\geq3}\indora_{j=i}F_{i,2}^{(2)}\\
f_{2}\left(i,j\right)&=&
\indora_{i,j\neq1}\left(\frac{1}{1-\tilde{\mu}_{1}}\right)^{2}\tilde{\mu}_{i}\tilde{\mu}_{j}f_{1}\left(1,1\right)
+\left[\left(1-\indora_{i=j=3}\right)\indora_{i+j\leq6}\indora_{i\leq j}\frac{\tilde{\mu}_{j}}{1-\tilde{\mu}_{1}}
+\left(1-\indora_{i=j=3}\right)\indora_{i+j\leq6}\indora_{i>j}\frac{\tilde{\mu}_{i}}{1-\tilde{\mu}_{1}}\right.
\\
&+&\left.\indora_{i=2}\frac{\tilde{\mu}_{i}}{1-\tilde{\mu}_{1}}\right]f_{1}\left(1,2\right)
+\indora_{i=2}f_{1}\left(2,2\right)
+\left[\indora_{i,j\neq1}\tilde{\theta}_{1}^{(2)}\tilde{\mu}_{i}\tilde{\mu}_{j}
+\indora_{i,j\neq1}\indora_{i\neq j}\frac{\tilde{\mu}_{i}\tilde{\mu}_{j}}{1-\tilde{\mu}_{1}}
+\indora_{i,j\neq1}\indora_{i=j}\frac{\tilde{P}_{i}^{(2)}}{1-\tilde{\mu}_{1}}\right]f_{1}\left(1\right)\\
&+&\left[r_{1}\mu_{i}+\indora_{i\geq3}F_{i,1}^{(1)}\right]f_{1}\left(j\right)
+\left[\indora_{j\geq3}F_{j,1}^{(1)}+r_{1}\mu_{j}\right]f_{1}\left(i\right)
+\left[R_{1}^{(2)}+\indora_{i=j}\right]\tilde{\mu}_{i}\tilde{\mu}_{j}
+\indora_{i\geq3}F_{i,1}^{(1)}\left[r_{1}\mu_{j}
+\indora_{j\neq i}F_{j,1}^{(1)}\right]\\
&+&r_{1}\left[\indora_{j\geq3}\mu_{i}F_{j,1}^{(1)}
+\indora_{i=j}P_{i}^{(2)}\right]
+\indora_{i\geq3}\indora_{j=i}F_{i,1}^{(2)}\\
f_{3}\left(i,j\right)&=&
\indora_{i=3}f_{4}\left(3,3\right)
+\left[\left(1-\indora_{i=j=2}\right)\indora_{i+j\geq4}\indora_{i\leq j}\frac{\tilde{\mu}_{i}}{1-\tilde{\mu}_{4}}
+\left(1-\indora_{i=j=2}\right)\indora_{i+j\geq4}\indora_{i>j}\frac{\tilde{\mu}_{j}}{1-\tilde{\mu}_{4}}
+\indora_{i=3}\frac{\tilde{\mu}_{i}}{1-\tilde{\mu}_{4}}\right]f_{4}\left(3,4\right)\\
&+&\indora_{i,j\neq4}f_{4}\left(4,4\right)\left(\frac{1}{1-\tilde{\mu}_{4}}\right)^{2}\tilde{\mu}_{i}\tilde{\mu}_{j}
+\left[\indora_{i,j\neq4}\tilde{\theta}_{4}^{(2)}\tilde{\mu}_{i}\tilde{\mu}_{j}
+\indora_{i,j\neq4}\indora_{i=j}\frac{\tilde{P}_{i}^{(2)}}{1-\tilde{\mu}_{4}}
+\indora_{i,j\neq4}\indora_{i\neq j}\frac{\tilde{\mu}_{i}\tilde{\mu}_{j}}{1-\tilde{\mu}_{4}}\right]f_{4}\left(4\right)\\
&+&\left[r_{4}\tilde{\mu}_{i}+\indora_{i\leq2}F_{i,4}^{(1)}\right]f_{4}\left(j\right)
+\left[r_{4}\tilde{\mu}_{j}+\indora_{j\leq2}F_{j,4}^{(1)}\right]f_{4}\left(i\right)
+\left[R_{4}^{(2)}+\indora_{i=j}r_{4}\right]\tilde{\mu}_{i}\tilde{\mu}_{j}\\
&+& \indora_{i\leq2}F_{i,4}^{(1)}\left[r_{4}\tilde{\mu}_{j}
+\indora_{j\neq i}F_{j,4}^{(1)}\right]
+r_{4}\left[\indora_{i=j}P_{i}^{(2)}+\indora_{j\leq2}\tilde{\mu}_{i}F_{j,4}^{(1)}\right]
+\indora_{i\leq2}\indora_{j=i}F_{i,4}^{(2)}\\
f_{4}\left(i,j\right)&=&
\indora_{i,j\neq3}f_{3}\left(3,3\right)\left(\frac{1}{1-\tilde{\mu}_{3}}\right)^{2}\tilde{\mu}_{i}\tilde{\mu}_{j}
+\left[\left(1-\indora_{i=j=2}\right)\indora_{i+j\geq5}\indora_{i\leq j}\frac{\tilde{\mu}_{i}}{1-\tilde{\mu}_{3}}
+\left(1-\indora_{i=j=2}\right)\indora_{i+j\geq5}\indora_{i>j}\frac{\tilde{\mu}_{j}}{1-\tilde{\mu}_{3}}\right.\\
&+&\left.\indora_{i=4}\frac{\tilde{\mu}_{i}}{1-\tilde{\mu}_{3}}\right]f_{3}\left(3,4\right)
+\indora_{i=4}f_{3}\left(4,4\right)
+\left[\indora_{i,j\neq3}\tilde{\theta}_{3}^{(2)}\tilde{\mu}_{i}\tilde{\mu}_{j}
+\indora_{i,j\neq3}\indora_{i=j}\frac{\tilde{P}_{i}^{(2)}}{1-\tilde{\mu}_{3}}
+\indora_{i,j\neq3}\indora_{i\neq j}\frac{\tilde{\mu}_{i}\tilde{\mu}_{j}}{1-\tilde{\mu}_{3}}\right]f_{3}\left(3\right)\\
&+&\left[r_{3}\tilde{\mu}_{i}+\indora_{i\leq2}F_{i,3}^{(1)}\right]f_{3}\left(j\right)
+\left[r_{3}\tilde{\mu}_{j}+\indora_{j\leq2}F_{j,3}^{(1)}\right]f_{3}\left(i\right)
+\left[R_{3}^{(2)}+\indora_{i=j}r_{3}\right]\tilde{\mu}_{i}\tilde{\mu}_{j}\\
&+&\indora_{i\leq2}F_{i,3}^{(1)}\left[r_{3}\tilde{\mu}_{j}+\indora_{j\neq i}F_{j,3}^{(1)}\right]
+r_{3}\left[\indora_{i=j}P_{i}^{(2)}+\indora_{j\leq2}\tilde{\mu}_{i}F_{j,3}^{(1)}\right]
+\indora_{i\leq2}\indora_{j=i}F_{i,3}^{(2)}
\end{eqnarray*}

\begin{eqnarray*}
f_{1}\left(1,1\right)&=&
f_{2}\left(1,1\right)
+2\mu_{1}f_{2}\left(1,2\right)
+\mu_{1}\tilde{\mu}_{2}f_{2}\left(2,2\right)
+P_{1}^{(2)}\left[r_{2}
+f_{2}\left(2\right)\right]
+2r_{2}f_{2}\left(1\right)
+R_{2}^{(2)}\mu_{1}\\
f_{1}\left(1,2\right)&=&
\tilde{\mu}_{2}f_{2}\left(1,2\right)
+\mu_{1}\tilde{\mu}_{2}f_{2}\left(2,2\right)
+\tilde{\mu}_{2}\mu_{1}\left[R_{2}^{(2)}
+r_{2}
+f_{2}\left(2\right)\right]
+r_{2}\tilde{\mu}_{1}f_{2}\left(2\right)
+r_{2}\tilde{\mu}_{2}f_{2}\left(1\right)\\
f_{1}\left(1,3\right)
&=&\mu_{3}f_{2}\left(1,2\right)
+\mu_{1}\mu_{3}f_{2}\left(2,2\right)
+\mu_{1}\mu_{3}\left[R_{2}^{(2)}
+r_{2}
+f_{2}\left(2\right)\right]
+r_{2}\mu_{1}\left[f_{2}\left(3\right)
+F_{3,2}^{(1)}\right]
+f_{2}\left(1\right)\left[r_{2}\mu_{3}
+F_{3,2}^{(1)}\right]\\
f_{1}\left(1,4\right)&=&\mu_{4}f_{2}\left(1,2\right)
+\mu_{1}\mu_{4}f_{2}\left(2,2\right)
+\tilde{\mu}_{1}\tilde{\mu}_{4}\left[R_{2}^{(2)}
+r_{2}
+f_{2}\left(2\right)\right]
+r_{2}\tilde{\mu}_{1}\left[f_{2}\left(4\right)
+F_{4,2}^{(1)}\right]
+f_{2}\left(1\right)\left[r_{2}\tilde{\mu}_{4}
+F_{4,2}^{(1)}\right]\\
f_{1}\left(2,2\right)&=&R_{2}^{(2)}\tilde{\mu}_{2}^{2}+r_{2}\tilde{P}_{2}^{(2)}
+f_{2}\left(2,2\right)\tilde{\mu}_{2}^{2}
+f_{2}\left(2\right)P_{2}^{(2)}
+2r_{2}\tilde{\mu}_{2}f_{2}\left(2\right)=\tilde{\mu}_{2}^{2}f_{2}\left(2,2\right)
+\tilde{P}_{2}^{(2)}\left[r_{2}+f_{2}\left(2\right)\right]
+\tilde{\mu}_{2}\left[R_{2}^{(2)}\tilde{\mu}_{2}
+2r_{2}f_{2}\left(2\right)\right]\\
f_{1}\left(2,3\right)&=&\tilde{\mu}_{2}\mu_{3}f_{2}\left(2,2\right)
+\tilde{\mu}_{2}\mu_{3}\left[R_{2}^{(2)}
+r_{2}
+f_{2}\left(2\right)\right]
+r_{2}\tilde{\mu}_{2}\left[f_{2}\left(3\right)
+F_{3,1}^{(1)}\left(1\right)\right]
+f_{2}\left(2\right)\left[r_{2}\mu_{3}
+F_{3,1}^{(1)}\left(1\right)\right]\\
f_{1}\left(2,4\right)&=&
\tilde{\mu}_{2}\mu_{4}f_{2}\left(2,2\right)
+\tilde{\mu}_{2}\mu_{4}\left[R_{2}^{(2)}
+r_{2}
+f_{2}\left(2\right)\right]
+r_{2}\tilde{\mu}_{2}\left[f_{2}\left(4\right)
+F_{4,2}^{(1)}\right]
+f_{2}\left(2\right)\left[r_{2}\tilde{\mu}_{4}
+F_{4,2}^{(1)}\right]\\
f_{1}\left(3,3\right)&=&\mu_{3}^{2}f_{2}\left(2,2\right)
+\tilde{P}_{3}^{(2)}\left[r_{2}
+f_{2}\left(2\right)\right]
+r_{2}\tilde{\mu}_{3}\left[f_{2}\left(3\right)
+F_{3,2}^{(1)}
+f_{2}\left(3\right)\right]+F_{3,2}^{(1)}\left[2f_{2}\left(3\right)
+r_{2}\tilde{\mu}_{3}\right]
+F_{3,2}^{(2)}
+R_{2}^{(2)}\tilde{\mu}_{3}^{2}\\
f_{1}\left(3,4\right)&=&
\mu_{3}\mu_{4}f_{2}\left(2,2\right)
+\mu_{3}\mu_{4}\left[R_{2}^{(2)}
+r_{2}
+f_{2}\left(2\right)\right]
+r_{2}\tilde{\mu}_{3}\left[f_{2}\left(4\right)
+F_{4,2}^{(1)}\right]
+r_{2}\tilde{\mu}_{4}\left[f_{2}\left(3\right)
+F_{3,2}^{(1)}\right]
+F_{4,2}^{(1)}\left[f_{2}\left(3\right)
+F_{3,2}^{(1)}\right]
+F_{3,2}^{(1)}f_{2}\left(4\right)\\
f_{1}\left(4,4\right)&=&
f_{2}\left(2,2\right)\mu_{4}^{2}
+P_{4}^{(2)}\left[r_{2}
+f_{2}\left(2\right)\right]
+2F_{4,2}^{(1)}\left[r_{2}\tilde{\mu}_{4}
+f_{2}\left(4\right)\right]
+\tilde{\mu}_{4}\left[R_{2}^{(2)}\tilde{\mu}_{4}
+2r_{2}f_{2}\left(4\right)\right]
+F_{4,2}^{(2)}
\end{eqnarray*}

\begin{eqnarray*}
f_{2}\left(1,1\right)
&=&
f_{1}\left(1,1\right)\mu_{1}^{2}
+f_{1}\left(1\right)\left[P_{1}^{(2)}
+2r_{1}\tilde{\mu}_{1}\right]
+R_{1}^{2}\tilde{\mu}_{1}^{2}
+r_{1}\tilde{P}_{1}^{(2)}\\
f_{2}\left(1,2\right)&=&
\mu_{1}\tilde{\mu}_{2}f_{1}\left(1,1\right)
+\mu_{1}f_{1}\left(1,2\right)
+\tilde{\mu}_{1}\tilde{\mu}_{2}\left[R_{1}^{(2)}
+r_{1}+f_{1}\left(1\right)\right]
+r_{1}\left[\tilde{\mu}_{1}f_{1}\left(2\right)
+\tilde{\mu}_{2}f_{1}\left(1\right)\right]\\
f_{2}\left(1,3\right)&=&\mu_{1}\mu_{3}f_{1}\left(1,1\right)
+\tilde{\mu}_{1}\tilde{\mu}_{3}\left[R_{1}^{(2)}
+r_{1}
+f_{1}\left(1\right)\right]
+r_{1}\tilde{\mu}_{1}\left[f_{1}\left(3\right)
+F_{3,1}^{(1)}\right]
+f_{1}\left(1\right)\left[r_{1}\tilde{\mu}_{3}
+F_{3,1}^{(1)}\right]\\
f_{2}\left(1,4\right)&=&\mu_{1}\mu_{4}f_{1}\left(1,1\right)
+\mu_{1}\mu_{4}\left[R_{1}^{(2)}
+r_{1}+f_{1}\left(1\right)\right]
+r_{1}\mu_{1}\left[f_{1}\left(4\right)
+F_{4,1}^{(1)}\right]
+f_{1}\left(1\right)\left[r_{1}\mu_{4}
+F_{4,1}^{(1)}\right]\\
f_{2}\left(2,2\right)&=&
\tilde{\mu}_{2}^{2}f_{1}\left(1,1\right)
+2\tilde{\mu}_{2}f_{1}\left(1,2\right)
+\tilde{\mu}_{2}f_{1}\left(2,2\right)
+\tilde{P}_{2}^{(2)}\left[r_{1}+f_{1}\left(1\right)\right]
+\tilde{\mu}_{2}\left[\tilde{\mu}_{2}R_{1}^{(2)}
+2r_{1}f_{1}\left(2\right)\right]\\
f_{2}\left(2,3\right)&=&
\tilde{\mu}_{2}\mu_{3}f_{1}\left(1,1\right)
+\mu_{3}f_{1}\left(1,2\right)
+\tilde{\mu}_{2}\mu_{3}\left[
R_{1}^{(2)}
+r_{1}
+f_{1}\left(1\right)\right]
+r_{1}\tilde{\mu}_{2}\left[f_{1}\left(3\right)
+F_{3,1}^{(1)}\right]
+f_{1}\left(2\right)\left[r_{1}\tilde{\mu}_{3}
+F_{3,1}^{(1)}\right]\\
f_{2}\left(2,4\right)&=&
\tilde{\mu}_{2}\mu_{4}f_{1}\left(1,1\right)
+\mu_{4}f_{1}\left(1,2\right)
+\tilde{\mu}_{2}\mu_{4}\left[
+R_{1}^{(2)}
+r_{1}
+f_{1}\left(1\right)\right]
+r_{1}\tilde{\mu}_{2}\left[f_{1}\left(4\right)
+F_{4,1}^{(1)}\right]
+f_{1}\left(2\right)\left[r_{1}\tilde{\mu}_{4}
+F_{4,1}^{(1)}\right]\\
f_{2}\left(3,3\right)&=&\mu_{3}^{2}f_{1}\left(1,1\right)
+P_{3}^{(2)}\left[r_{1}+f_{1}\left(1\right)\right]
+2r_{1}\tilde{\mu}_{3}\left[f_{1}\left(3\right)
+F_{3,1}^{(1)}\right]
+\left[R_{1}^{(2)}\tilde{\mu}_{3}^{2}
+F_{3,1}^{(2)}
+2F_{3,1}^{(1)}f_{1}\left(3\right)\right]\\
f_{2}\left(3,4\right)&=&
\mu_{3}\mu_{4}f_{1}\left(1,1\right)
+\mu_{3}\mu_{4}\left[R_{1}^{(2)}
+r_{1}
+f_{1}\left(1\right)\right]
+r_{1}\mu_{3}\left[f_{1}\left(4\right)
+F_{4,1}^{(1)}\right]
+
f_{1}\left(3\right)\left[r_{1}\tilde{\mu}_{4}
+F_{4,1}^{(1)}\right]
+F_{3,1}^{(1)}\left[r_{1}\tilde{\mu}_{4}
+f_{1}\left(4\right)\right]
+F_{4,2}^{(1)}F_{3,2}^{(1)}\\
f_{2}\left(4,4\right)&=&
\mu_{4}^{2}f_{1}\left(1,1\right)
+P_{4}^{(2)}\left[r_{1}
+f_{1}\left(1\right)\right]
+\tilde{\mu}_{4}\left[2r_{1}f_{1}\left(4\right)
+R_{1}^{(2)}\tilde{\mu}_{4}\right]
+2F_{4,1}^{(1)}\left[f_{1}\left(4\right)
+r_{1}\tilde{\mu}_{4}\right]
+F_{4,1}^{(2)}\\
\end{eqnarray*}

\begin{eqnarray*}
f_{3}\left(1,1\right)&=&
\mu_{1}^{2}f_{4}\left(4,4\right)
+P_{1}^{(2)}\left[r_{4}
+f_{4}\left(4\right)\right]
+2r_{4}\tilde{\mu}_{1}\left[F_{1,4}^{(1)}
+f_{4}\left(1\right)\right]
+\left[2f_{4}\left(1\right)F_{1,4}^{(1)}
+F_{1,4}^{(2)}
+R_{2}^{(2)}\tilde{\mu}_{1}^{2}\right]\\
f_{3}\left(1,2\right)&=&
\mu_{1}\tilde{\mu}_{2}f_{4}\left(4,4\right)
+\mu_{1}\tilde{\mu}_{2}\left[R_{4}^{(2)}
+r_{4}
+f_{4}\left(4\right)\right]
+r_{4}\tilde{\mu}_{1}\left[F_{2,4}^{(1)}
+f_{4}\left(2\right)\right]+
f_{4}\left(1\right)\left[r_{4}\tilde{\mu}_{2}
+F_{2,4}^{(1)}\right]
+F_{1,4}^{(1)}\left[r_{4}\tilde{\mu}_{2}
+f_{4}\left(2\right)+F_{2,4}^{(1)}\right]\\
f_{3}\left(1,3\right)
&=&\mu_{1}\mu_{3}f_{4}\left(4,4\right)
+\mu_{1}f_{4}\left(3,4\right)
+\mu_{1}\mu_{3}\left[R_{4}^{(2)}
+r_{4}
+f_{4}\left(4\right)\right]
+r_{4}\tilde{\mu}_{3}\left[F_{1,4}^{(1)}
+f_{4}\left(1\right)\right]
+f_{4}\left(3\right)\left[F_{1,4}^{(1)}
+r_{4}\tilde{\mu}_{1}\right]\\
f_{3}\left(1,4\right)
&=&
\mu_{1}\mu_{4}f_{4}\left(4,4\right)
+\mu_{1}\mu_{4}\left[R_{4}^{(2)}
+r_{4}
+f_{4}\left(4\right)\right]
+f_{4}\left(4\right)\left[r_{4}\tilde{\mu}_{1}
+F_{1,4}^{(1)}\right]
+r_{4}\tilde{\mu}_{4}\left[f_{4}\left(1\right)
+F_{1,4}^{(1)}\right]\\
f_{3}\left(2,2\right)&=&
\tilde{\mu}_{2}^{2}f_{4}\left(4,4\right)
+R_{4}^{(2)}\tilde{\mu}_{2}^{2}
+\tilde{P}_{2}^{(2)}\left[r_{4}
+f_{4}\left(4\right)\right]
+2r_{4}\tilde{\mu}_{2}\left[F_{2,4}^{(1)}
+f_{4}\left(2\right)\right]
+\left[2f_{4}\left(2\right)F_{2,4}^{(1)}
+F_{2,4}^{(2)}\right]\\
f_{3}\left(2,3\right)&=&
\tilde{\mu}_{2}\mu_{3}f_{4}\left(4,4\right)
+\tilde{\mu}_{2}f_{4}\left(3,4\right)
+\tilde{\mu}_{2}\mu_{3}\left[R_{4}^{(2)}
+r_{4}
+f_{4}\left(4\right)\right]
+r_{4}\tilde{\mu}_{3}\left[f_{4}\left(2\right)
+F_{2,4}^{(1)}\right]
+f_{4}\left(3\right)\left[r_{4}\tilde{\mu}_{2}+F_{2,4}^{(1)}\right]\\
f_{3}\left(2,4\right)&=&
\tilde{\mu}_{2}\mu_{4}f_{4}\left(4,4\right)
+\tilde{\mu}_{2}\mu_{4}\left[R_{4}^{(2)}
+r_{4}
+f_{4}\left(4\right)\right]
+r_{4}\tilde{\mu}_{4}\left[f_{4}\left(4\right)
+F_{2,4}^{(1)}\right]
+f_{4}\left(4\right)\left[r_{4}\tilde{\mu}_{2}
+F_{2,4}^{(1)}\right]\\
f_{3}\left(3,3\right)&=&
\mu_{3}^{2}f_{4}\left(4,4\right)
+2\mu_{3}f_{4}\left(3,4\right)
+f_{4}\left(3,3\right)
+P_{3}^{(2)}\left[r_{4}
+f_{4}\left(4\right)\right]
+\tilde{\mu}_{3}\left[R_{4}^{(2)}\tilde{\mu}_{3}
+2r_{4}f_{4}\left(4\right)\right]\\
f_{3}\left(3,4\right)&=&
\mu_{4}f_{4}\left(3,4\right)
+\mu_{3}\mu_{4}f_{4}\left(4,4\right)
+\mu_{3}\mu_{4}\left[R_{4}^{(2)}
+r_{4}
+f_{4}\left(4\right)\right]
+r_{4}\left[\tilde{\mu}_{3}f_{4}\left(4\right)
+\tilde{\mu}_{4}f_{4}\left(3\right)\right]\\
f_{3}\left(4,4\right)
&=&
\mu_{4}^{2}f_{4}\left(4,4\right)
+P_{4}^{(2)}\left[r_{4}
+f_{4}\left(4\right)\right]
+\tilde{\mu}_{4}\left[R_{4}^{(2)}\tilde{\mu}_{4}
+2r_{4}f_{4}\left(4\right)\right]
\end{eqnarray*}



\begin{eqnarray*}
f_{4}\left(1,1\right)
&=&
\mu_{1}^{2}f_{3}\left(3,3\right)
+P_{1}^{(2)}\left[r_{3}
+f_{3}\left(3\right)\right]
+2f_{3}\left(1\right)\left[r_{3}\tilde{\mu}_{1}
+F_{1,3}^{(1)}\right]
+\tilde{\mu}_{1}\left[R_{3}^{(2)}\tilde{\mu}_{1}
+2F_{1,3}^{(1)}r_{3}\right]
+F_{1,3}^{(2)}\\
f_{4}\left(1,2\right)&=&
\mu_{1}\tilde{\mu}_{2}f_{3}\left(3,3\right)
+\mu_{1}\tilde{\mu}_{2}\left[R_{3}^{(2)}
+r_{3}
+f_{3}\left(3\right)\right]
+r_{3}\tilde{\mu}_{1}\left[F_{2,3}^{(1)}
+f_{3}\left(2\right)\right]
+f_{3}\left(1\right)\left[r_{3}\tilde{\mu}_{2}
+F_{2,3}^{(1)}\right]
+F_{1,3}^{(1)}\left[r_{3}\tilde{\mu}_{2}
+f_{3}\left(2\right)+F_{2,3}^{(1)}\right]\\
f_{4}\left(1,3\right)&=&
\mu_{1}\mu_{3}f_{3}\left(3,3\right)
+\mu_{1}\mu_{3}\left[R_{3}^{(2)}+r_{3}
+f_{3}\left(3\right)\right]
+r_{3}\tilde{\mu}_{3}\left[f_{3}\left(1\right)
+F_{1,3}^{(1)}\right]
+f_{3}\left(3\right)\left[F_{1,3}^{(1)}+r_{3}\tilde{\mu}_{1}\right]\\
f_{4}\left(1,4\right)&=&
\mu_{4}\mu_{1}f_{3}\left(3,3\right)
+\mu_{1}f_{3}\left(3,4\right)
+\mu_{1}\mu_{4}\left[R_{3}^{(2)}
+r_{3}\tilde{\mu}_{1}
+f_{3}\left(3\right)\right]
+r_{3}\tilde{\mu}_{4}\left[f_{3}\left(3\right)
+F_{1,3}^{(1)}\right]
+f_{3}\left(4\right)\left[r_{3}\tilde{\mu}_{1}+F_{1,3}^{(1)}\right]\\
f_{4}\left(2,2\right)&=&
\tilde{\mu}_{2}^{2}f_{3}\left(3,3\right)
+P_{2}^{(2)}\left[r_{3}
+f_{3}\left(3\right)\right]
+2r_{3}\tilde{\mu}_{2}\left[F_{2,3}^{(1)}
+f_{3}\left(2\right)\right]
+\left[R_{3}^{(2)}\tilde{\mu}_{2}^{2}
+F_{2,3}^{(2)}
+2f_{3}\left(2\right)F_{2,3}^{(1)}\right]\\
f_{4}\left(2,3\right)&=&
\mu_{3}\tilde{\mu}_{2}f_{3}\left(3,3\right)
+\mu_{3}\tilde{\mu}_{2}\left[R_{3}^{(2)}
+r_{3}
+f_{3}\left(3\right)\right]
+r_{3}\tilde{\mu}_{3}\left[f_{3}\left(2\right)
+F_{2,3}^{(1)}\right]
+f_{3}\left(3\right)\left[r_{3}\tilde{\mu}_{2}
+F_{2,3}^{(1)}\right]\\
f_{4}\left(2,4\right)&=&
\tilde{\mu}_{2}\mu_{4}f_{3}\left(3,3\right)
+\tilde{\mu}_{2}f_{3}\left(3,4\right)
+\mu_{4}\tilde{\mu}_{2}\left[R_{3}^{(2)}
+r_{3}
+f_{3}\left(3\right)\right]
+f_{3}\left(4\right)\left[r_{3}\tilde{\mu}_{2}
+F_{2,3}^{(1)}\right]
+r_{3}\tilde{\mu}_{4}\left[f_{3}\left(2\right)
+F_{2,3}^{(1)}\right]\\
f_{4}\left(3,3\right)&=&
\mu_{3}^{2}f_{3}\left(3,3\right)
+P_{3}^{(2)}\left[r_{3}
+f_{3}\left(3\right)\right]
+\tilde{\mu}_{3}\left[R_{3}^{(2)}\tilde{\mu}_{3}
+2r_{3}f_{3}\left(3\right)\right]\\
f_{4}\left(3,4\right)&=&\mu_{3}\mu_{4}f_{3}\left(3,3\right)
+\mu_{3}f_{3}\left(3,4\right)
+\mu_{4}\mu_{3}\left[R_{3}^{(2)}
+r_{3}
+f_{3}\left(3\right)\right]
+r_{3}\left[\tilde{\mu}_{3}f_{3}\left(4\right)
+\tilde{\mu}_{4}f_{3}\left(3\right)\right]\\
f_{4}\left(4,4\right)&=&\mu_{4}^{2}f_{3}\left(3,3\right)
+2\mu_{4}f_{3}\left(3,4\right)
+\mu_{4}f_{3}\left(4,4\right)
+\tilde{\mu}_{4}\left[2r_{3}f_{3}\left(4\right)
+R_{3}^{(2)}\tilde{\mu}_{4}\right]
+P_{4}^{(2)}\left[r_{3}
+f_{3}\left(3\right)\right]\\
\end{eqnarray*}



\begin{eqnarray*}
\begin{array}{lllllll}
D_{1}D_{1}F_{1}=0,&
D_{2}D_{1}F_{1}=0,&
D_{3}D_{1}F_{1}=0,&
D_{4}D_{1}F_{1}=0,&
D_{1}D_{2}F_{1}=0,&
D_{1}D_{3}F_{1}=0,&
D_{1}D_{4}F_{1}=0,\\
D_{2}D_{1}F_{2}=0,&
D_{2}D_{3}F_{3}=0,&
D_{2}D_{4}F_{2}=0,&
D_{1}D_{2}F_{2}=0,&
D_{2}D_{2}F_{2}=0,&
D_{3}D_{2}F_{2}=0,&
D_{4}D_{2}F_{2}=0,\\
D_{3}D_{1}F_{3}=0,&
D_{3}D_{2}F_{3}=0,&
D_{1}D_{3}F_{3}=0,&
D_{2}D_{3}F_{3}=0,&
D_{3}D_{3}F_{3}=0,&
D_{4}D_{3}F_{3}=0,&
D_{3}D_{4}F_{3}=0,\\
D_{4}D_{1}F_{4}=0,&
D_{4}D_{2}F_{4}=0,&
D_{4}D_{3}F_{4}=0,&
D_{1}D_{4}F_{4}=0,&
D_{2}D_{4}F_{4}=0,&
D_{3}D_{4}F_{4}=0,&
D_{4}D_{4}F_{4}=0.
\end{array}
\end{eqnarray*}

\begin{eqnarray*}
D_{2}D_{2}F_{1}&=&f_{1}\left(1,1\right)\left(\frac{\tilde{\mu}_{2}}{1-\tilde{\mu}_{1}}\right)^{2}
+f_{1}\left(1\right)\tilde{\theta}_{1}^{(2)}\tilde{\mu}_{2}^{2}
+f_{1}\left(1\right)\frac{\tilde{P}_{2}^{(2)}}{1-\tilde{\mu}_{1}}+f_{1}\left(1,2\right)\frac{\tilde{\mu}_{2}}{1-\tilde{\mu}_{1}}+f_{1}\left(1,2\right)\frac{\tilde{\mu}_{2}}{1-\tilde{\mu}_{1}}+f_{1}\left(2,2\right)\\
D_{3}D_{2}F_{1}&=&f_{1}\left(1,1\right)\left(\frac{1}{1-\tilde{\mu}_{1}}\right)^{2}\tilde{\mu}_{2}\tilde{\mu}_{3}+f_{1}\left(1\right)\tilde{\theta}_{1}^{(2)}\tilde{\mu}_{2}\tilde{\mu}_{3}+f_{1}\left(1\right)\frac{\tilde{\mu}_{2}\tilde{\mu}_{3}}{1-\tilde{\mu}_{1}}+f_{1}\left(1,2\right)\frac{\tilde{\mu}_{3}}{1-\tilde{\mu}_{1}}\\
D_{4}D_{2}F_{1}&=&f_{1}\left(1,1\right)\left(\frac{1}{1-\tilde{\mu}_{1}}\right)^{2}\tilde{\mu}_{2}\tilde{\mu}_{4}+f_{1}\left(1\right)\tilde{\theta}_{1}^{(2)}\tilde{\mu}_{2}\tilde{\mu}_{4}+f_{1}\left(1\right)\frac{\tilde{\mu}_{2}\tilde{\mu}_{4}}{1-\tilde{\mu}_{1}}+f_{1}\left(1,2\right)\frac{\tilde{\mu}_{4}}{1-\tilde{\mu}_{1}}\\
D_{3}D_{3}F_{1}&=&f_{1}\left(1,1\right)\left(\frac{\tilde{\mu}_{3}}{1-\tilde{\mu}_{1}}\right)^{2}+f_{1}\left(1\right)\tilde{\theta}_{1}^{(2)}\tilde{\mu}_{3}^{2}+f_{1}\left(1\right)\frac{\tilde{P}_{3}^{2}}{1-\tilde{\mu}_{1}}\\
D_{4}D_{3}F_{1}&=&f_{1}\left(1,1\right)\left(\frac{1}{1-\tilde{\mu}_{1}}\right)^{2}\tilde{\mu}_{3}\tilde{\mu}_{4}
+f_{1}\left(1\right)\tilde{\theta}_{1}^{2}\tilde{\mu}_{4}\tilde{\mu}_{3}
+f_{1}\left(1\right)\frac{\tilde{\mu}_{4}\tilde{\mu}_{3}}{1-\tilde{\mu}_{1}}\\
D_{4}D_{4}F_{1}&=&f_{1}\left(1,1\right)\left(\frac{\tilde{\mu}_{4}}{1-\tilde{\mu}_{1}}\right)^{2}+f_{1}\left(1\right)\tilde{\theta}_{1}^{(2)}\tilde{\mu}_{4}^{2}+f_{1}\left(1\right)\frac{1}{1-\tilde{\mu}_{1}}\tilde{P}_{4}^{(2)}
\end{eqnarray*}


\begin{eqnarray*}
D_{1}D_{1}F_{2}&=&f_{2}\left(2\right)\frac{\tilde{P}_{1}^{(2)}}{1-\tilde{\mu}_{2}}
+f_{2}\left(2\right)\theta_{2}^{(2)}\tilde{\mu}_{1}^{2}
+f_{2}\left(2,1\right)\frac{\tilde{\mu}_{1}}{1-\tilde{\mu}_{2}}
+\left(\frac{\tilde{\mu}_{1}}{1-\tilde{\mu}_{2}}\right)^{2}f_{2}\left(2,2\right)
+\frac{\tilde{\mu}_{1}}{1-\tilde{\mu}_{2}}f_{2}\left(2,1\right)+f_{2}\left(1,1\right)\\
D_{3}D_{1}F_{2}&=&f_{2}\left(2,1\right)\frac{\tilde{\mu}_{3}}{1-\tilde{\mu}_{2}}
+f_{2}\left(2,2\right)\left(\frac{1}{1-\tilde{\mu}_{2}}\right)^{2}\tilde{\mu}_{1}\tilde{\mu}_{3}
+f_{2}\left(2\right)\tilde{\theta}_{2}^{(2)}\tilde{\mu}_{1}\tilde{\mu}_{3}
+f_{2}\left(2\right)\frac{\tilde{\mu}_{1}\tilde{\mu}_{3}}{1-\tilde{\mu}_{2}}\\
D_{4}D_{1}F_{2}&=&f_{2}\left(2,2\right)\left(\frac{1}{1-\tilde{\mu}_{2}}\right)^{2}\tilde{\mu}_{1}\tilde{\mu}_{4}
+f_{2}\left(2\right)\tilde{\theta}_{2}^{(2)}\tilde{\mu}_{1}\tilde{\mu}_{4}
+f_{2}\left(2\right)\frac{\tilde{\mu}_{1}\tilde{\mu}_{4}}{1-\tilde{\mu}_{2}}
+f_{2}\left(2,1\right)\frac{\tilde{\mu}_{4}}{1-\tilde{\mu}_{2}}\\
D_{3}D_{3}F_{2}&=&f_{2}\left(2,2\right)\left(\frac{1}{1-\tilde{\mu}_{2}}\right)^{2}\tilde{\mu}_{3}^{2}
+f_{2}\left(2\right)\tilde{\theta}_{2}^{(2)}\tilde{\mu}_{3}^{2}
+f_{2}\left(2\right)\frac{\tilde{P}_{3}^{(2)}}{1-\tilde{\mu}_{2}}\\
D_{4}D_{3}F_{2}&=&f_{2}\left(2,2\right)\left(\frac{1}{1-\tilde{\mu}_{2}}\right)^{2}\tilde{\mu}_{3}\tilde{\mu}_{4}
+f_{2}\left(2\right)\tilde{\theta}_{2}^{(2)}\tilde{\mu}_{3}\tilde{\mu}_{4}
+f_{2}\left(2\right)\frac{\tilde{\mu}_{3}\tilde{\mu}_{4}}{1-\tilde{\mu}_{2}}\\
D_{4}D_{4}F_{2}&=&f_{2}\left(2,2\right)\left(\frac{\tilde{\mu}_{4}}{1-\tilde{\mu}_{2}}\right)^{2}
+f_{2}\left(2\right)\tilde{\theta}_{2}^{(2)}\tilde{\mu}_{4}^{2}
+f_{2}\left(2\right)\frac{\tilde{P}_{4}^{(2)}}{1-\tilde{\mu}_{2}}
\end{eqnarray*}


\begin{eqnarray*}
D_{1}D_{1}F_{3}&=&f_{3}\left(3,3\right)\left(\frac{\tilde{\mu}_{3}}{1-\tilde{\mu}_{4}}\right)^{2}
+f_{3}\left(3\right)\frac{\tilde{P}_{1}^{(2)}}{1-\tilde{\mu}_{3}}
+f_{3}\left(3\right)\tilde{\theta}_{3}^{(2)}\tilde{\mu}_{1}^{2}\\
D_{2}D_{1}F_{3}&=&f_{3}\left(3,3\right)\left(\frac{1}{1-\tilde{\mu}_{3}}\right)^{2}\tilde{\mu}_{1}\tilde{\mu}_{2}
+f_{3}\left(3\right)\tilde{\mu}_{1}\tilde{\mu}_{2}\tilde{\theta}_{3}^{(2)}
+f_{3}\left(3\right)\frac{\tilde{\mu}_{1}\tilde{\mu}_{2}}{1-\tilde{\mu}_{3}}\\
D_{4}D_{1}F_{3}&=&f_{3}\left(3,3\right)\left(\frac{1}{1-\tilde{\mu}_{3}}\right)^{2}\tilde{\mu}_{1}\tilde{\mu}_{4}
+f_{3}\left(3\right)\tilde{\theta}_{3}^{(2)}\tilde{\mu}_{1}\tilde{\mu}_{4}
+f_{3}\left(3\right)\frac{\tilde{\mu}_{1}\tilde{\mu}_{4}}{1-\tilde{\mu}_{3}}
+f_{3}\left(3,4\right)\frac{\tilde{\mu}_{1}}{1-\tilde{\mu}_{3}}\\
D_{2}D_{2}F_{3}&=&f_{3}\left(3,3\right)\left(\frac{\tilde{\mu}_{2}}{1-\tilde{\mu}_{3}}\right)^{2}+f_{3}\left(3\right)\tilde{\theta}_{3}^{(2)}\tilde{\mu}_{2}^{2}
+f_{3}\left(3\right)\frac{\tilde{P}_{2}^{(2)}}{1-\tilde{\mu}_{3}}\\
D_{4}D_{2}F_{3}&=&f_{3}\left(3,3\right)\left(\frac{1}{1-\tilde{\mu}_{3}}\right)^{2}\tilde{\mu}_{2}\tilde{\mu}_{4}
+f_{3}\left(3\right)\tilde{\theta}_{3}^{(2)}\tilde{\mu}_{2}\tilde{\mu}_{4}
+f_{3}\left(3\right)\frac{\tilde{\mu}_{2}\tilde{\mu}_{4}}{1-\tilde{\mu}_{3}}
+f_{3}\left(3,4\right)\frac{\tilde{\mu}_{2}}{1-\tilde{\mu}_{3}}\\
D_{4}D_{4}F_{3}&=&f_{3}\left(3,3\right)\left(\frac{\tilde{\mu}_{4}}{1-\tilde{\mu}_{3}}\right)^{2}+f_{3}\left(3\right)\tilde{\theta}_{3}^{(2)}\tilde{\mu}_{4}^{2}
+f_{3}\left(3\right)\frac{\tilde{P}_{4}^{(2)}}{1-\tilde{\mu}_{3}}
+2f_{3}\left(3,4\right)\frac{\tilde{\mu}_{4}}{1-\tilde{\mu}_{3}}
+f_{3}\left(4,4\right)
\end{eqnarray*}



\begin{eqnarray*}
D_{1}D_{1}F_{4}&=&f_{4}\left(4,4\right)\left(\frac{\tilde{\mu}_{1}}{1-\tilde{\mu}_{4}}\right)^{2}
+f_{4}\left(4\right)\tilde{\theta}_{4}^{(2)}\tilde{\mu}_{1}^{2}
+f_{4}\left(4\right)\frac{\tilde{P}_{1}^{(2)}}{1-\tilde{\mu}_{4}}\\
D_{2}D_{1}F_{4}&=&f_{4}\left(4,4\right)\left(\frac{1}{1-\tilde{\mu}_{4}}\right)^{2}\tilde{\mu}_{1}\tilde{\mu}_{2}
+f_{4}\left(4\right)\tilde{\theta}_{4}^{(2)}\tilde{\mu}_{1}\tilde{\mu}_{2}
+f_{4}\left(4\right)\frac{\tilde{\mu}_{1}\tilde{\mu}_{2}}{1-\tilde{\mu}_{4}}\\
D_{3}D_{1}F_{4}&=&f_{4}\left(4,4\right)\left(\frac{1}{1-\tilde{\mu}_{4}}\right)^{2}\tilde{\mu}_{1}\tilde{\mu}_{3}
+f_{4}\left(4\right)\tilde{\theta}_{4}^{(2)}\tilde{\mu}_{1}\tilde{\mu}_{3}
+f_{4}\left(4\right)\frac{\tilde{\mu}_{1}\tilde{\mu}_{3}}{1-\tilde{\mu}_{4}}
+f_{4}\left(4,3\right)\frac{\tilde{\mu}_{1}}{1-\tilde{\mu}_{4}}\\
D_{2}D_{2}F_{4}&=&f_{4}\left(4,4\right)\left(\frac{\tilde{\mu}_{2}}{1-\tilde{\mu}_{4}}\right)^{2}
+f_{4}\left(4\right)\tilde{\theta}_{4}^{(2)}\tilde{\mu}_{2}^{2}
+f_{4}\left(4\right)\frac{\tilde{P}_{2}^{(2)}}{1-\tilde{\mu}_{4}}\\
D_{3}D_{2}F_{4}&=&f_{4}\left(4,4\right)\left(\frac{1}{1-\tilde{\mu}_{4}}\right)^{2}\tilde{\mu}_{2}\tilde{\mu}_{3}
+f_{4}\left(4\right)\tilde{\theta}_{4}^{(2)}\tilde{\mu}_{2}\tilde{\mu}_{3}
+f_{4}\left(4\right)\frac{\tilde{\mu}_{2}\tilde{\mu}_{3}}{1-\tilde{\mu}_{4}}
+f_{4}\left(4,3\right)\frac{\tilde{\mu}_{2}}{1-\tilde{\mu}_{4}}\\
D_{3}D_{3}F_{4}&=&f_{4}\left(4,4\right)\left(\frac{\tilde{\mu}_{3}}{1-\tilde{\mu}_{4}}\right)^{2}
+f_{4}\left(4\right)\tilde{\theta}_{4}^{(2)}\tilde{\mu}_{3}^{2}
+f_{4}\left(4\right)\frac{\tilde{P}_{3}^{(2)}}{1-\tilde{\mu}_{4}}
+2f_{4}\left(4,3\right)\frac{\tilde{\mu}_{3}}{1-\tilde{\mu}_{4}}
+f_{4}\left(3,3\right)
\end{eqnarray*}


Then according to the equations in \ref{Sist.Ec.Lineales.Doble.Traslado.App.B}, we have




\begin{eqnarray*}
f_{1}\left(i,k\right)&=&D_{k}D_{i}\left(R_{2}+F_{2}+\indora_{i\geq3}F_{4}\right)
+D_{i}R_{2}D_{k}\left(F_{2}+\indora_{k\geq3}F_{4}\right)
+D_{i}F_{2}D_{k}\left(R_{2}+\indora_{k\geq3}F_{4}\right)\\
&+&\indora_{i\geq3}D_{i}F_{4}D_{k}\left(R_{2}+F_{2}\right)
\end{eqnarray*}
%_____________________________________________________________
%\subsubsection*{$F_{1}$ }
%_____________________________________________________________

\begin{eqnarray*}
f_{1}\left(1,1\right)&=&R_{2}^{(2)}\tilde{\mu}_{1}^{2}+r_{2}\tilde{P}_{1}^{(2)}+f_{2}\left(2\right)\frac{\tilde{P}_{1}^{(2)}}{1-\tilde{\mu}_{2}}
+f_{2}\left(2\right)\theta_{2}^{(2)}\tilde{\mu}_{1}^{2}
+f_{2}\left(2,1\right)\frac{\tilde{\mu}_{1}}{1-\tilde{\mu}_{2}}
+\left(\frac{\tilde{\mu}_{1}}{1-\tilde{\mu}_{2}}\right)^{2}f_{2}\left(2,2\right)\\
&+&\frac{\tilde{\mu}_{1}}{1-\tilde{\mu}_{2}}f_{2}\left(2,1\right)+f_{2}\left(1,1\right)
+2r_{2}\tilde{\mu}_{2}f_{2}\left(1\right)=\left[\left(\frac{\tilde{\mu}_{1}}{1-\tilde{\mu}_{2}}\right)^{2}f_{2}\left(2,2\right)
+2\frac{\tilde{\mu}_{1}}{1-\tilde{\mu}_{2}}f_{2}\left(2,1\right)
+f_{2}\left(1,1\right)\right]\\
&+&\left[\tilde{\mu}_{1}^{2}\left(R_{2}^{(2)}+f_{2}\left(2\right)\theta_{2}^{(2)}\right)
+\tilde{P}_{1}^{(2)}\left(\frac{f_{2}\left(2\right)}{1-\tilde{\mu}_{2}}+r_{2}\right)
+2r_{2}\tilde{\mu}_{2}f_{2}\left(1\right)\right],\\
&=&a_{1}f_{2}\left(2,2\right)
+a_{2}f_{2}\left(2,1\right)
+a_{3}f_{2}\left(1,1\right)
+K_{1}
\end{eqnarray*}

\begin{eqnarray*}
\begin{array}{llll}
a_{1}=\left(\frac{\tilde{\mu}_{1}}{1-\tilde{\mu}_{2}}\right)^{2},&
a_{2}=2\frac{\tilde{\mu}_{1}}{1-\tilde{\mu}_{2}},&
a_{3}=1,&
K_{1}=\tilde{\mu}_{1}^{2}\left(R_{2}^{(2)}+f_{2}\left(2\right)\theta_{2}^{(2)}\right)
+\tilde{P}_{1}^{(2)}\left(\frac{f_{2}\left(2\right)}{1-\tilde{\mu}_{2}}+r_{2}\right)
+2r_{2}\tilde{\mu}_{2}f_{2}\left(1\right)
\end{array}
\end{eqnarray*}


\begin{eqnarray*}
f_{1}\left(1,2\right)&=&D_{2}D_{1}R_{2}
+D_{2}D_{1}F_{2}
+D_{1}R_{2}D_{2}F_{2}
+D_{1}F_{2}D_{2}R_{2}\\
&=&R_{2}^{(2)}\tilde{\mu}_{1}\tilde{\mu}_{2}+r_{2}\tilde{\mu}_{1}\tilde{\mu}_{2}
+D_{2}D_{1}F_{2}
+r_{2}\tilde{\mu}_{1}f_{2}\left(2\right)
+r_{2}\tilde{\mu}_{2}f_{2}\left(1\right)\\
&=&\tilde{\mu}_{1}\tilde{\mu}_{2}\left[R_{2}^{(2)}+
+r_{2}\right]
+r_{2}\left[\tilde{\mu}_{1}f_{2}\left(2\right)
+\tilde{\mu}_{2}f_{2}\left(1\right)\right]\\
&=&K_{2}
\end{eqnarray*}

\begin{eqnarray*}
K_{2}&=&\tilde{\mu}_{1}\tilde{\mu}_{2}\left[R_{2}^{(2)}+
+r_{2}\right]
+r_{2}\left[\tilde{\mu}_{1}f_{2}\left(2\right)
+\tilde{\mu}_{2}f_{2}\left(1\right)\right],
\end{eqnarray*}


\begin{eqnarray*}
f_{1}\left(1,3\right)&=&D_{3}D_{1}R_{2}+D_{3}D_{1}F_{2}
+D_{1}R_{2}D_{3}F_{2}+D_{1}R_{2}D_{3}F_{4}
+D_{1}F_{2}D_{3}R_{2}+D_{1}F_{2}D_{3}F_{4}\\
&=&R_{2}^{(2)}\tilde{\mu}_{1}\tilde{\mu}_{3}+r_{2}\tilde{\mu}_{1}\tilde{\mu}_{3}
+D_{3}D_{1}F_{2}
+r_{2}\tilde{\mu}_{1}f_{2}\left(3\right)
+r_{2}\tilde{\mu}_{1}D_{3}F_{4}
+r_{2}\tilde{\mu}_{3}f_{2}\left(1\right)
+D_{3}F_{4}f_{2}\left(1\right)\\
&=&R_{2}^{(2)}\tilde{\mu}_{1}\tilde{\mu}_{3}+r_{2}\tilde{\mu}_{1}\tilde{\mu}_{3}
+f_{2}\left(2,1\right)\frac{\tilde{\mu}_{3}}{1-\tilde{\mu}_{2}}
+f_{2}\left(2,2\right)\left(\frac{1}{1-\tilde{\mu}_{2}}\right)^{2}\tilde{\mu}_{1}\tilde{\mu}_{3}
+f_{2}\left(2\right)\tilde{\theta}_{2}^{(2)}\tilde{\mu}_{1}\tilde{\mu}_{3}
+f_{2}\left(2\right)\frac{\tilde{\mu}_{1}\tilde{\mu}_{3}}{1-\tilde{\mu}_{2}}\\
&+&r_{2}\tilde{\mu}_{1}\tilde{\mu}_{3}\left(r_{1}+\frac{r\tilde{\mu}_{1}}{1-\tilde{\mu}}\right)+F_{3,1}^{(1)}\left(1\right)
+r_{2}\tilde{\mu}_{1}F_{3,2}^{(1)}
+r_{2}\tilde{\mu}_{3}r_{1}\tilde{\mu}_{1}
+F_{3,2}^{(1)}r_{1}\tilde{\mu}_{1}=\left(\frac{1}{1-\tilde{\mu}_{2}}\right)^{2}\tilde{\mu}_{1}\tilde{\mu}_{3}f_{2}\left(2,2\right)\\
&+&\frac{\tilde{\mu}_{3}}{1-\tilde{\mu}_{2}}f_{2}\left(2,1\right)
+\left[\tilde{\mu}_{1}\tilde{\mu}_{3}\left[R_{2}^{(2)}+r_{2}+f_{2}\left(2\right)\left(\tilde{\theta}_{2}^{(2)}+\frac{1}{1-\tilde{\mu}_{2}}\right)\right]
+r_{2}\tilde{\mu}_{1}\left[F_{3,2}^{(1)}+f_{2}\left(1\right)\right]\right.\\
&+&\left.\left[r_{2}\tilde{\mu}_{3}+F_{3,2}^{(1)}\right]f_{2}\left(1\right)\right]\\
&=&a_{4}f_{2}\left(2,2\right)+a_{5}f\left(2,1\right)+K_{3},
\end{eqnarray*}


\begin{eqnarray*}
\begin{array}{ll}
a_{4}=\left(\frac{1}{1-\tilde{\mu}_{2}}\right)^{2}\tilde{\mu}_{1}\tilde{\mu}_{3},&
a_{5}=\frac{\tilde{\mu}_{3}}{1-\tilde{\mu}_{2}},
\end{array}
\end{eqnarray*}

\begin{eqnarray*}
K_{3}&=&\tilde{\mu}_{1}\tilde{\mu}_{3}\left[R_{2}^{(2)}+r_{2}+f_{2}\left(2\right)\left(\tilde{\theta}_{2}^{(2)}+\frac{1}{1-\tilde{\mu}_{2}}\right)\right]
+r_{2}\tilde{\mu}_{1}\left[F_{3,2}^{(1)}+f_{2}\left(1\right)\right]
+\left[r_{2}\tilde{\mu}_{3}+F_{3,2}^{(1)}\right]f_{2}\left(1\right),
\end{eqnarray*}

\begin{eqnarray*}
f_{1}\left(1,4\right)&=&D_{4}D_{1}R_{2}+D_{4}D_{1}F_{2}
+D_{1}R_{2}D_{4}F_{2}+D_{1}R_{2}D_{4}F_{4}
+D_{1}F_{2}D_{4}R_{2}+D_{1}F_{2}D_{4}F_{4}\\
&=&R_{2}^{(2)}\tilde{\mu}_{1}\tilde{\mu}_{4}+r_{2}\tilde{\mu}_{1}\tilde{\mu}_{4}
+D_{4}D_{1}F_{2}
+r_{2}\tilde{\mu}_{1}f_{2}\left(4\right)
+r_{2}\tilde{\mu}_{1}D_{4}F_{4}
+r_{2}\tilde{\mu}_{4}f_{2}\left(1\right)
+f_{2}\left(1\right)D_{4}F_{4}\\
&=&\left(\frac{1}{1-\tilde{\mu}_{2}}\right)^{2}\tilde{\mu}_{1}\tilde{\mu}_{4}f_{2}\left(2,2\right)
+\frac{\tilde{\mu}_{4}}{1-\tilde{\mu}_{2}}f_{2}\left(2,1\right)
+\tilde{\mu}_{1}\tilde{\mu}_{4}\left[R_{2}^{(2)}
+r_{2}+f_{2}\left(2\right)\left(\tilde{\theta}_{2}^{(2)}
+\frac{1}{1-\tilde{\mu}_{2}}\right)\right]\\
&+&\left[r_{2}\tilde{\mu}_{1}\left[f_{2}\left(4\right)+F_{4,2}^{(1)}\right]
+f_{2}\left(1\right)\left[r_{2}\tilde{\mu}_{4}+F_{4,2}^{(1)}\right]\right]\\
&=&a_{6}f_{2}\left(2,2\right)+a_{7}f_{2}\left(2,1\right)+K_{4}
\end{eqnarray*}

\begin{eqnarray*}
\begin{array}{lll}
a_{6}=\left(\frac{1}{1-\tilde{\mu}_{2}}\right)^{2}\tilde{\mu}_{1}\tilde{\mu}_{4},&
a_{7}=\frac{\tilde{\mu}_{4}}{1-\tilde{\mu}_{2}},&
\end{array}
\end{eqnarray*}

\begin{eqnarray*}
K_{4}=\tilde{\mu}_{1}\tilde{\mu}_{4}\left[R_{2}^{(2)}
+r_{2}+f_{2}\left(2\right)\left(\tilde{\theta}_{2}^{(2)}
+\frac{1}{1-\tilde{\mu}_{2}}\right)\right]
+r_{2}\tilde{\mu}_{1}\left[f_{2}\left(4\right)+F_{4,2}^{(1)}\right]
+f_{2}\left(1\right)\left[r_{2}\tilde{\mu}_{4}+F_{4,2}^{(1)}\right]
\end{eqnarray*}


\begin{eqnarray*}
f_{1}\left(2,2\right)&=&D_{2}D_{2}\left(R_{2}+F_{2}\right)
+D_{2}R_{2}D_{2}F_{2}+D_{2}F_{2}D_{2}R_{2}
=D_{2}D_{2}R_{2}+D_{2}D_{2}F_{2}+D_{2}R_{2}D_{2}F_{2}+D_{2}F_{2}D_{2}R_{2}\\
&=&R_{2}^{(2)}\tilde{\mu}_{2}^{2}+r_{2}\tilde{P}_{2}^{(2)}
+D_{2}D_{2}F_{2}
+2r_{2}\tilde{\mu}_{2}f_{2}\left(2\right)=R_{2}^{(2)}\tilde{\mu}_{2}^{2}+r_{2}\tilde{P}_{2}^{(2)}
+2r_{2}\tilde{\mu}_{2}f_{2}\left(2\right)\\
&=&\tilde{\mu}_{2}^{2}\left[R_{2}^{(2)}+2r_{2}\frac{r\left(1-\tilde{\mu}_{2}\right)}{1-\mu}\right]+r_{2}\tilde{P}_{2}^{(2)}=K_{5}\\
\end{eqnarray*}

\begin{eqnarray*}
K_{5}=\tilde{\mu}_{2}^{2}\left[R_{2}^{(2)}+2r_{2}\frac{r\left(1-\tilde{\mu}_{2}\right)}{1-\mu}\right]+r_{2}\tilde{P}_{2}^{(2)}
\end{eqnarray*}

\begin{eqnarray*}
f_{1}\left(2,3\right)&=&D_{3}D_{2}R_{2}+D_{3}D_{2}F_{2}
+D_{2}R_{2}D_{3}F_{2}+D_{2}R_{2}D_{3}F_{4}
+D_{2}F_{2}D_{3}R_{2}+D_{2}F_{2}D_{3}F_{4}\\
&=&R_{2}^{(2)}\tilde{\mu}_{2}\tilde{\mu}_{3}+r_{2}\tilde{\mu}_{2}\tilde{\mu}_{3}
+D_{3}D_{2}F_{2}
+r_{2}\tilde{\mu}_{2}f_{2}\left(3\right)
+r_{2}\tilde{\mu}_{2}D_{3}F_{4}
+r_{2}\tilde{\mu}_{3}f_{2}\left(2\right)
+f_{2}\left(2\right)D_{3}F_{4}\\
&=&R_{2}^{(2)}\tilde{\mu}_{2}\tilde{\mu}_{3}
+r_{2}\tilde{\mu}_{2}\tilde{\mu}_{3}
+r_{2}\tilde{\mu}_{2}f_{2}\left(3\right)+r_{2}\tilde{\mu}_{2}F_{3,2}^{(1)}
+r_{2}\tilde{\mu}_{3}f_{2}\left(2\right)
+f_{2}\left(2\right)F_{3,2}^{(1)}\\
&=&\tilde{\mu}_{2}\tilde{\mu}_{3}\left[R_{2}^{(2)}
+r_{2}\right]
+r_{2}\tilde{\mu}_{2}\left[f_{2}\left(3\right)+F_{3,2}^{(1)}\right]
+f_{2}\left(2\right)\left[r_{2}\tilde{\mu}_{3}+F_{3,2}^{(1)}\right]=K_{6}
\end{eqnarray*}

\begin{eqnarray*}
K_{6}=\tilde{\mu}_{2}\tilde{\mu}_{3}\left[R_{2}^{(2)}
+r_{2}\right]
+r_{2}\tilde{\mu}_{2}\left[f_{2}\left(3\right)+F_{3,2}^{(1)}\right]
+f_{2}\left(2\right)\left[r_{2}\tilde{\mu}_{3}+F_{3,2}^{(1)}\right]
\end{eqnarray*}



\begin{eqnarray*}
f_{1}\left(2,4\right)&=&D_{4}D_{2}R_{2}+D_{4}D_{2}F_{2}
+D_{2}R_{2}D_{4}F_{2}+D_{2}R_{2}D_{4}F_{4}
+D_{2}F_{2}D_{4}R_{2}+D_{2}F_{2}D_{4}F_{4}\\
&=&R_{2}^{(2)}\tilde{\mu}_{2}\tilde{\mu}_{4}+r_{2}\tilde{\mu}_{2}\tilde{\mu}_{4}
+D_{4}D_{2}F_{2}
+r_{2}\tilde{\mu}_{2}f_{2}\left(4\right)
+r_{2}\tilde{\mu}_{2}D_{4}F_{4}
+r_{2}\tilde{\mu}_{4}f_{2}\left(2\right)
+f_{2}\left(2\right)D_{4}F_{4}\\
&=&R_{2}^{(2)}\tilde{\mu}_{2}\tilde{\mu}_{4}+r_{2}\tilde{\mu}_{2}\tilde{\mu}_{4}
+r_{2}\tilde{\mu}_{2}f_{2}\left(4\right)
+r_{2}\tilde{\mu}_{2}F_{4,2}^{(1)}+r_{2}\tilde{\mu}_{4}f_{2}\left(2\right)
+f_{2}\left(2\right)F_{4,2}^{(1)}\\
&=&\tilde{\mu}_{2}\tilde{\mu}_{4}\left[R_{2}^{(2)}+r_{2}\right]
+r_{2}\tilde{\mu}_{2}\left[f_{2}\left(4\right)+F_{4,2}^{(1)}\right]
+f_{2}\left(2\right)\left[r_{2}\tilde{\mu}_{4}+F_{4,2}^{(1)}\right]\\
&=&K_{7}
\end{eqnarray*}

\begin{eqnarray*}
K_{7}=\tilde{\mu}_{2}\tilde{\mu}_{4}\left[R_{2}^{(2)}+r_{2}\right]
+r_{2}\tilde{\mu}_{2}\left[f_{2}\left(4\right)+F_{4,2}^{(1)}\right]
+f_{2}\left(2\right)\left[r_{2}\tilde{\mu}_{4}+F_{4,2}^{(1)}\right]
\end{eqnarray*}



\begin{eqnarray*}
f_{1}\left(3,3\right)&=&D_{3}D_{3}R_{2}+D_{3}D_{3}F_{2}+D_{3}D_{3}F_{4}
+D_{3}R_{2}D_{3}F_{2}+D_{3}R_{2}D_{3}F_{4}\\
&+&D_{3}F_{2}D_{3}R_{2}+D_{3}F_{2}D_{3}F_{4}
+D_{3}F_{4}D_{3}R_{2}+D_{3}F_{4}D_{3}F_{2}\\
&=&R_{2}^{(2)}\tilde{\mu}_{3}^{2}+r_{2}\tilde{P}_{3}^{(2)}
+D_{3}D_{3}F_{2}
+D_{3}D_{3}F_{4}
+r_{2}\tilde{\mu}_{3}f_{2}\left(3\right)
+r_{2}\tilde{\mu}_{3}D_{3}F_{4}\\
&+&r_{2}\tilde{\mu}_{3}f_{2}\left(3\right)
+f_{2}\left(3\right)D_{3}F_{4}
+r_{2}\tilde{\mu}_{3}D_{3}F_{4}
+f_{2}\left(3\right)D_{3}F_{4}\\
&=&R_{2}^{(2)}\tilde{\mu}_{3}^{2}+r_{2}\tilde{P}_{3}^{(2)}
+f_{2}\left(2,2\right)\left(\frac{1}{1-\tilde{\mu}_{2}}\right)^{2}\tilde{\mu}_{3}^{2}
+f_{2}\left(2\right)\tilde{\theta}_{2}^{(2)}\tilde{\mu}_{3}^{2}
+f_{2}\left(2\right)\frac{\tilde{P}_{3}^{(2)}}{1-\tilde{\mu}_{2}}
+F_{3,2}^{(2)}\\
&+&2r_{2}\tilde{\mu}_{3}f_{2}\left(3\right)
+2r_{2}\tilde{\mu}_{3}F_{3,2}^{(1)}
+2f_{2}\left(3\right)F_{3,2}^{(1)}=f_{2}\left(2,2\right)\left(\frac{1}{1-\tilde{\mu}_{2}}\right)^{2}\tilde{\mu}_{3}^{2}\\
&+&\tilde{\mu}_{3}^{2}\left[R_{2}^{(2)}+
+f_{2}\left(2\right)\tilde{\theta}_{2}^{(2)}\right]
+\tilde{P}_{3}^{(2)}\left[\frac{f_{2}\left(2\right)}{1-\tilde{\mu}_{2}}
+r_{2}\right]
+2r_{2}\tilde{\mu}_{3}\left[f_{2}\left(3\right)+F_{3,2}^{(1)}\right]
+2f_{2}\left(3\right)F_{3,2}^{(1)}+F_{3,2}^{(2)}\\
&=&a_{8}f_{2}\left(2,2\right)+K_{8}
\end{eqnarray*}


\begin{eqnarray*}
a_{8}&=&\left(\frac{1}{1-\tilde{\mu}_{2}}\right)^{2}\tilde{\mu}_{3}^{2}\\
K_{8}&=&\tilde{\mu}_{3}^{2}\left[R_{2}^{(2)}+
+f_{2}\left(2\right)\tilde{\theta}_{2}^{(2)}\right]
+\tilde{P}_{3}^{(2)}\left[\frac{f_{2}\left(2\right)}{1-\tilde{\mu}_{2}}
+r_{2}\right]
+2r_{2}\tilde{\mu}_{3}\left[f_{2}\left(3\right)+F_{3,2}^{(1)}\right]
+2f_{2}\left(3\right)F_{3,2}^{(1)}+F_{3,2}^{(2)}
\end{eqnarray*}


\begin{eqnarray*}
f_{1}\left(3,4\right)&=&D_{4}D_{3}R_{2}+D_{4}D_{3}F_{2}+D_{4}D_{3}F_{4}
+D_{3}R_{2}D_{4}F_{2}+D_{3}R_{2}D_{4}F_{4}\\
&+&D_{3}F_{2}D_{4}R_{2}+D_{3}F_{2}D_{4}F_{4}
+D_{3}F_{4}D_{4}R_{2}+D_{3}F_{4}D_{4}F_{2}\\
&=&R_{2}^{(2)}\tilde{\mu}_{3}\tilde{\mu}_{4}+r_{2}\tilde{\mu}_{3}\tilde{\mu}_{4}
+D_{4}D_{3}F_{2}
+D_{4}D_{3}F_{4}
+r_{2}\tilde{\mu}_{3}f_{2}\left(4\right)
+r_{2}\tilde{\mu}_{3}D_{4}F_{4}\\
&+&r_{2}\tilde{\mu}_{4}f_{2}\left(3\right)
+D_{4}F_{4}f_{2}\left(3\right)
+D_{3}F_{4}r_{2}\tilde{\mu}_{4}
+D_{3}F_{4}f_{2}\left(4\right)\\
&=&R_{2}^{(2)}\tilde{\mu}_{3}\tilde{\mu}_{4}+r_{2}\tilde{\mu}_{3}\tilde{\mu}_{4}
+f_{2}\left(2,2\right)\left(\frac{1}{1-\tilde{\mu}_{2}}\right)^{2}\tilde{\mu}_{3}\tilde{\mu}_{4}
+f_{2}\left(2\right)\tilde{\theta}_{2}^{(2)}\tilde{\mu}_{3}\tilde{\mu}_{4}
+f_{2}\left(2\right)\frac{\tilde{\mu}_{3}\tilde{\mu}_{4}}{1-\tilde{\mu}_{2}}
+F_{4,2}^{(1)}F_{3,2}^{(1)}\\
&+&r_{2}\tilde{\mu}_{3}f_{2}\left(4\right)
+r_{2}\tilde{\mu}_{3}F_{4,2}^{(1)}
+r_{2}\tilde{\mu}_{4}f_{2}\left(3\right)
+F_{4,2}^{(1)}f_{2}\left(3\right)
+F_{3,2}^{(1)}r_{2}\tilde{\mu}_{4}
+F_{3,2}^{(1)}f_{2}\left(4\right)\\
&=&\left(\frac{1}{1-\tilde{\mu}_{2}}\right)^{2}\tilde{\mu}_{3}\tilde{\mu}_{4}f_{2}\left(2,2\right)+
\tilde{\mu}_{3}\tilde{\mu}_{4}\left[R_{2}^{(2)}
+r_{2}
+\left(\tilde{\theta}_{2}^{(2)}+\frac{1}{1-\tilde{\mu}_{2}}\right)f_{2}\left(2\right)\right]
\\
&+&r_{2}\tilde{\mu}_{3}\left(f_{2}\left(4\right)+F_{4,2}^{(1)}\right)
+r_{2}\tilde{\mu}_{4}\left(f_{2}\left(3\right)+F_{3,2}^{(1)}\right)
+F_{4,2}^{(1)}\left(f_{2}\left(3\right)+F_{3,2}^{(1)}\right)
+F_{3,2}^{(1)}f_{2}\left(4\right)
\\
&=&a_{9}f_{2}\left(2,2\right)+K_{9}
\end{eqnarray*}
\begin{eqnarray*}
a_{9}&=&\left(\frac{1}{1-\tilde{\mu}_{2}}\right)^{2}\tilde{\mu}_{3}\tilde{\mu}_{4}\\
K_{9}&=&\tilde{\mu}_{3}\tilde{\mu}_{4}\left[R_{2}^{(2)}
+r_{2}
+\left(\tilde{\theta}_{2}^{(2)}+\frac{1}{1-\tilde{\mu}_{2}}\right)f_{2}\left(2\right)\right]
+r_{2}\tilde{\mu}_{3}\left(f_{2}\left(4\right)+F_{4,2}^{(1)}\right)
+r_{2}\tilde{\mu}_{4}\left(f_{2}\left(3\right)+F_{3,2}^{(1)}\right)\\
&+&F_{4,2}^{(1)}\left(f_{2}\left(3\right)+F_{3,2}^{(1)}\right)
+F_{3,2}^{(1)}f_{2}\left(4\right)
\end{eqnarray*}



\begin{eqnarray*}
f_{1}\left(4,4\right)&=&D_{4}D_{4}R_{2}+D_{4}D_{4}F_{2}+D_{4}D_{4}F_{4}
+D_{4}R_{2}D_{4}F_{2}+D_{4}R_{2}D_{4}F_{4}+D_{4}F_{2}D_{4}R_{2}+D_{4}F_{2}D_{4}F_{4}\\
&+&D_{4}F_{4}D_{4}R_{2}+D_{4}F_{4}D_{4}F_{2}
=R_{2}^{(2)}\tilde{\mu}_{4}^{2}+r_{2}\tilde{P}_{4}^{(2)}
+D_{4}D_{4}F_{2}
+D_{4}D_{4}F_{4}
+2r_{2}\tilde{\mu}_{4}f_{2}\left(4\right)\\
&+&2r_{2}\tilde{\mu}_{4}D_{4}F_{4}
+2D_{4}F_{4}f_{2}\left(4\right)
=R_{2}^{(2)}\tilde{\mu}_{4}^{2}+r_{2}\tilde{P}_{4}^{(2)}
+f_{2}\left(2,2\right)\left(\frac{\tilde{\mu}_{4}}{1-\tilde{\mu}_{2}}\right)^{2}
+f_{2}\left(2\right)\tilde{\theta}_{2}^{(2)}\tilde{\mu}_{4}^{2}
+f_{2}\left(2\right)\frac{\tilde{P}_{4}^{(2)}}{1-\tilde{\mu}_{2}}\\
&+&F_{4,2}^{(2)}+2r_{2}\tilde{\mu}_{4}f_{2}\left(4\right)
+2r_{2}\tilde{\mu}_{4}F_{4,2}^{(1)}
+2F_{4,2}^{(1)}f_{2}\left(4\right)
=\left(\frac{\tilde{\mu}_{4}}{1-\tilde{\mu}_{2}}\right)^{2}f_{2}\left(2,2\right)+\tilde{\mu}_{4}^{2}\left[R_{2}^{(2)}+f_{2}\left(2\right)\tilde{\theta}_{2}^{(2)}\right]\\
&+&\tilde{P}_{4}^{(2)}\left[r_{2}+\frac{f_{2}\left(2\right)}{1-\tilde{\mu}_{2}}\right]
+2r_{2}\tilde{\mu}_{4}\left[f_{2}\left(4\right)+F_{4,2}^{(1)}\right]
+2F_{4,2}^{(1)}f_{2}\left(4\right)=a_{10}f_{2}\left(2,2\right)+K_{10}
\end{eqnarray*}

\begin{eqnarray*}
a_{10}&=&\left(\frac{\tilde{\mu}_{4}}{1-\tilde{\mu}_{2}}\right)^{2}\\
K_{10}&=&\tilde{\mu}_{4}^{2}\left[R_{2}^{(2)}+f_{2}\left(2\right)\tilde{\theta}_{2}^{(2)}\right]
+\tilde{P}_{4}^{(2)}\left[r_{2}+\frac{f_{2}\left(2\right)}{1-\tilde{\mu}_{2}}\right]
+2r_{2}\tilde{\mu}_{4}\left[f_{2}\left(4\right)+F_{4,2}^{(1)}\right]
+2F_{4,2}^{(1)}f_{2}\left(4\right)
\end{eqnarray*}

So, following this procedure we have similar expressions for the rest of the elements


\begin{eqnarray*}
f_{2}\left(i,k\right)&=&D_{k}D_{i}\left(R_{1}+F_{1}+\indora_{i\geq3}F_{3}\right)+D_{i}R_{1}D_{k}\left(F_{1}+\indora_{k\geq3}F_{3}\right)+D_{i}F_{1}D_{k}\left(R_{1}+\indora_{k\geq3}F_{3}\right)\\
&+&\indora_{i\geq3}D_{i}\tilde{F}_{3}D_{k}\left(R_{1}+F_{1}\right)
\end{eqnarray*}
% $k=1$
\begin{eqnarray*}
f_{2}\left(1,1\right)&=&R_{1}^{2}\tilde{\mu}_{1}^{2}+r_{1}\tilde{P}_{1}^{(2)}
+2r_{1}\tilde{\mu}_{1}f_{1}\left(1\right)=K_{11}\\
f_{2}\left(1,2\right)&=&\tilde{\mu}_{1}\tilde{\mu}_{2}\left[R_{1}^{(2)}+r_{1}\right]
+r_{1}\left[\tilde{\mu}_{1}f_{1}\left(2\right)+\tilde{\mu}_{2}f_{1}\left(1\right)\right]=K_{12}
\end{eqnarray*}

\begin{eqnarray*}
f_{2}\left(1,3\right)&=&\tilde{\mu}_{1}\tilde{\mu}_{3}\left[R_{1}^{(2)}+r_{1}\right]
+r_{1}\tilde{\mu}_{1}\left[f_{1}\left(3\right)+F_{3,1}^{(1)}\right]
+f_{1}\left(1\right)\left[r_{1}\tilde{\mu}_{3}+F_{3,1}^{(1)}\right]=K_{13}\\
f_{2}\left(1,4\right)&=&\tilde{\mu}_{1}\tilde{\mu}_{4}\left[R_{1}^{(2)}+r_{1}\right]
+r_{1}\tilde{\mu}_{1}\left[f_{1}\left(4\right)+F_{4,1}^{(1)}\right]
+f_{1}\left(1\right)\left[r_{1}\tilde{\mu}_{4}+F_{4,1}^{(1)}\right]=K_{14}
\end{eqnarray*}

\begin{eqnarray*}
f_{2}\left(2,2\right)
&=&f_{1}\left(1,1\right)\left(\frac{\tilde{\mu}_{2}}{1-\tilde{\mu}_{1}}\right)^{2}
+2\frac{\tilde{\mu}_{2}}{1-\tilde{\mu}_{1}}f_{1}\left(1,2\right)
+f_{1}\left(2,2\right)+\tilde{\mu}_{2}^{2}\left[R_{1}^{(2)}+f_{1}\left(1\right)\tilde{\theta}_{1}^{(2)}\right]
+\tilde{P}_{2}^{(2)}\left[r_{1}+\frac{f_{1}\left(1\right)}{1-\tilde{\mu}_{1}}\right]\\
&+&2r_{1}\tilde{\mu}_{2}f_{1}\left(2\right)\\
&=&a_{11}f_{1}\left(1,1\right)
+a_{12}f_{1}\left(1,2\right)+a_{13}f_{1}\left(2,2\right)+K_{15}
\end{eqnarray*}

\begin{eqnarray*}
\begin{array}{ll}
a_{11}=\left(\frac{\tilde{\mu}_{2}}{1-\tilde{\mu}_{1}}\right)^{2}
a_{12}=2\frac{\tilde{\mu}_{2}}{1-\tilde{\mu}_{1}}
a_{13}=1
\end{array}
\end{eqnarray*}

\begin{eqnarray*}
K_{15}&=&\tilde{\mu}_{2}^{2}\left[R_{1}^{(2)}+f_{1}\left(1\right)\tilde{\theta}_{1}^{(2)}\right]
+\tilde{P}_{2}^{(2)}\left[r_{1}+\frac{f_{1}\left(1\right)}{1-\tilde{\mu}_{1}}\right]
+2r_{1}\tilde{\mu}_{2}f_{1}\left(2\right)
\end{eqnarray*}

\begin{eqnarray*}
f_{2}\left(2,3\right)&=&\left(\frac{1}{1-\tilde{\mu}_{1}}\right)^{2}\tilde{\mu}_{2}\tilde{\mu}_{3}f_{1}\left(1,1\right)
+\frac{\tilde{\mu}_{3}}{1-\tilde{\mu}_{1}}f_{1}\left(1,2\right)+
\tilde{\mu}_{2}\tilde{\mu}_{3}\left[R_{1}^{(2)}
+r_{1}+f_{1}\left(1\right)\left(\tilde{\theta}_{1}^{(2)}+\frac{1}{1-\tilde{\mu}_{1}}\right)\right]\\
&+&r_{1}\tilde{\mu}_{2}\left[f_{1}\left(3\right)+F_{3,1}^{(1)}\right]
+f_{1}\left(2\right)\left[r_{1}\tilde{\mu}_{3}+F_{3,1}^{(1)}\right]
=a_{14}f_{1}\left(1,1\right)+a_{15}f_{1}\left(1,2\right)+K_{16}
\end{eqnarray*}

\begin{eqnarray*}
\begin{array}{ll}
a_{14}=\left(\frac{1}{1-\tilde{\mu}_{1}}\right)^{2}\tilde{\mu}_{2}\tilde{\mu}_{3},&
a_{15}=\frac{\tilde{\mu}_{3}}{1-\tilde{\mu}_{1}}
\end{array}
\end{eqnarray*}

\begin{eqnarray*}
K_{16}=\tilde{\mu}_{2}\tilde{\mu}_{3}\left[R_{1}^{(2)}
+r_{1}+f_{1}\left(1\right)\left(\tilde{\theta}_{1}^{(2)}+\frac{1}{1-\tilde{\mu}_{1}}\right)\right]+r_{1}\tilde{\mu}_{2}\left[f_{1}\left(3\right)+F_{3,1}^{(1)}\right]
+f_{1}\left(2\right)\left[r_{1}\tilde{\mu}_{3}+F_{3,1}^{(1)}\right]
\end{eqnarray*}


%D_{4}D_{3}F_{4}=F_{4,2}^{(1)}F_{3,2}^{(1)
\begin{eqnarray*}
f_{2}\left(2,4\right)&=&
\left(\frac{1}{1-\tilde{\mu}_{1}}\right)^{2}\tilde{\mu}_{2}\tilde{\mu}_{4}f_{1}\left(1,1\right)
+\frac{\tilde{\mu}_{4}}{1-\tilde{\mu}_{1}}f_{1}\left(1,2\right)
+\tilde{\mu}_{2}\tilde{\mu}_{4}\left[R_{1}^{(2)}+r_{1}
+f_{1}\left(1\right)\left(\tilde{\theta}_{1}^{(2)}+\frac{1}{1-\tilde{\mu}_{1}}\right)\right]\\
&+&r_{1}\tilde{\mu}_{2}\left[f_{1}\left(4\right)
+\tilde{\mu}_{2}F_{4,1}^{(1)}\right]
+f_{1}\left(2\right)\left[r_{1}\tilde{\mu}_{4}
+F_{4,1}^{(1)}\right]=a_{16}f_{1}\left(1,1\right)+a_{17}f_{1}\left(1,2\right)+K_{17}
\end{eqnarray*}


\begin{eqnarray*}
\begin{array}{ll}
a_{16}=\left(\frac{1}{1-\tilde{\mu}_{1}}\right)^{2}\tilde{\mu}_{2}\tilde{\mu}_{4},&
a_{17}=\frac{\tilde{\mu}_{4}}{1-\tilde{\mu}_{1}}
\end{array}
\end{eqnarray*}

\begin{eqnarray*}
K_{17}=\tilde{\mu}_{2}\tilde{\mu}_{4}\left[R_{1}^{(2)}+r_{1}
+f_{1}\left(1\right)\left(\tilde{\theta}_{1}^{(2)}+\frac{1}{1-\tilde{\mu}_{1}}\right)\right]+r_{1}\tilde{\mu}_{2}\left[f_{1}\left(4\right)
+\tilde{\mu}_{2}F_{4,1}^{(1)}\right]
+f_{1}\left(2\right)\left[r_{1}\tilde{\mu}_{4}
+F_{4,1}^{(1)}\right]
\end{eqnarray*}


\begin{eqnarray*}
f_{2}\left(3,3\right)
&=&\left(\frac{\tilde{\mu}_{3}}{1-\tilde{\mu}_{1}}\right)^{2}f_{1}\left(1,1\right)
+\tilde{\mu}_{3}^{2}\left[R_{1}^{(2)}
+f_{1}\left(1\right)\tilde{\theta}_{1}^{(2)}\right]
+\tilde{P}_{3}^{(2)}\left[r_{1}+\frac{f_{1}\left(1\right)}{1-\tilde{\mu}_{1}}\right]
+2r_{1}\tilde{\mu}_{3}\left[f_{1}\left(3\right)+F_{3,1}^{(1)}\right]\\
&+&F_{3,1}^{(2)}+2F_{3,1}^{(1)}f_{1}\left(3\right)\\
&=&a_{18}f_{1}\left(1,1\right)+K_{18}
\end{eqnarray*}

\begin{eqnarray*}
a_{18}&=&\left(\frac{\tilde{\mu}_{3}}{1-\tilde{\mu}_{1}}\right)^{2}\\
K_{18}&=&\tilde{\mu}_{3}^{2}\left[R_{1}^{(2)}
+f_{1}\left(1\right)\tilde{\theta}_{1}^{(2)}\right]
+\tilde{P}_{3}^{(2)}\left[r_{1}+\frac{f_{1}\left(1\right)}{1-\tilde{\mu}_{1}}\right]
+2r_{1}\tilde{\mu}_{3}\left[f_{1}\left(3\right)+F_{3,1}^{(1)}\right]
+F_{3,1}^{(2)}+2F_{3,1}^{(1)}f_{1}\left(3\right)
\end{eqnarray*}


\begin{eqnarray*}
f_{2}\left(3,4\right)
&=&\left(\frac{1}{1-\tilde{\mu}_{1}}\right)^{2}\tilde{\mu}_{3}\tilde{\mu}_{4}f_{1}\left(1,1\right)+
\tilde{\mu}_{3}\tilde{\mu}_{4}\left[R_{1}^{(2)}+r_{1}
+f_{1}\left(1\right)\left(\tilde{\theta}_{1}^{2}
+\frac{1}{1-\tilde{\mu}_{1}}\right)\right]
+r_{1}\tilde{\mu}_{3}\left[f_{1}\left(4\right)+F_{4,1}^{(1)}\right]\\
&+&f_{1}\left(3\right)\left[r_{1}\tilde{\mu}_{4}+F_{4,1}^{(1)}\right]
+F_{3,1}^{(1)}\left[r_{1}\tilde{\mu}_{4}+F_{4,1}^{(1)}+f_{1}\left(4\right)\right]
=a_{19}f_{1}\left(1,1\right)+K_{19}
\end{eqnarray*}

\begin{eqnarray*}
a_{19}&=&\left(\frac{1}{1-\tilde{\mu}_{1}}\right)^{2}\tilde{\mu}_{3}\tilde{\mu}_{4}\\
K_{19}&=&\tilde{\mu}_{3}\tilde{\mu}_{4}\left[R_{1}^{(2)}+r_{1}
+f_{1}\left(1\right)\left(\tilde{\theta}_{1}^{2}
+\frac{1}{1-\tilde{\mu}_{1}}\right)\right]
+r_{1}\tilde{\mu}_{3}\left[f_{1}\left(4\right)+F_{4,1}^{(1)}\right]+f_{1}\left(3\right)\left[r_{1}\tilde{\mu}_{4}+F_{4,1}^{(1)}\right]\\
&+&F_{3,1}^{(1)}\left[r_{1}\tilde{\mu}_{4}+F_{4,1}^{(1)}+f_{1}\left(4\right)\right]
\end{eqnarray*}


\begin{eqnarray*}
f_{2}\left(4,4\right)
&=&\left(\frac{\tilde{\mu}_{4}}{1-\tilde{\mu}_{1}}\right)^{2}f_{1}\left(1,1\right)
+\tilde{\mu}_{4}^{2}\left[R_{1}^{(2)}+f_{1}\left(1\right)\tilde{\theta}_{1}^{(2)}\right]
+\tilde{P}_{4}^{(2)}\left[r_{1}+\frac{f_{1}\left(1\right)}{1-\tilde{\mu}_{1}}\right]
+f_{1}\left(4\right)\left[2r_{1}\tilde{\mu}_{4}+2F_{4,1}^{(1)}\right]\\
&+&F_{4,1}^{(2)}+2F_{4,1}^{(1)}r_{1}\tilde{\mu}_{4}
=a_{20}f_{1}\left(1,1\right)+K_{20}
\end{eqnarray*}

\begin{eqnarray*}
a_{20}&=&\left(\frac{\tilde{\mu}_{4}}{1-\tilde{\mu}_{1}}\right)^{2}\\
K_{20}&=&\tilde{\mu}_{4}^{2}\left[R_{1}^{(2)}+f_{1}\left(1\right)\tilde{\theta}_{1}^{(2)}\right]
+\tilde{P}_{4}^{(2)}\left[r_{1}+\frac{f_{1}\left(1\right)}{1-\tilde{\mu}_{1}}\right]
+f_{1}\left(4\right)\left[2r_{1}\tilde{\mu}_{4}+2F_{4,1}^{(1)}\right]+F_{4,1}^{(2)}+2F_{4,1}^{(1)}r_{1}\tilde{\mu}_{4}
\end{eqnarray*}


\begin{eqnarray*}
f_{3}\left(i,k\right)&=&D_{k}D_{i}\left(\tilde{R}_{4}+\indora_{k\leq2}F_{2}+F_{4}\right)+D_{i}\tilde{R}_{4}D_{k}\left(\indora_{k\leq2}F_{2}+F_{4}\right)+D_{i}F_{4}D_{k}\left(\tilde{R}_{4}+\indora_{k\leq2}F_{2}\right)\\
&+&\indora_{i\leq2}D_{i}F_{2}D_{k}\left(\tilde{R}_{4}+F_{4}\right)
\end{eqnarray*}

\begin{eqnarray*}
f_{3}\left(1,1\right)&=&f_{4}\left(4,4\right)\left(\frac{\tilde{\mu}_{1}}{1-\tilde{\mu}_{4}}\right)^{2}+
\tilde{\mu}_{1}^{2}\left[R_{2}^{(2)}+f_{4}\left(4\right)\tilde{\theta}_{4}^{(2)}\right]
+2r_{4}\tilde{\mu}_{1}\left[F_{1,4}^{(1)}+f_{4}\left(1\right)\right]
+\tilde{P}_{1}^{(2)}\left[r_{4}++\frac{f_{4}\left(4\right)}{1-\tilde{\mu}_{2}}\right]\\
&+&\left[F_{1,4}^{(2)}+2f_{4}\left(1\right)F_{1,4}^{(1)}\right]
=a_{21}f_{4} \left(4,4\right)+K_{21}
\end{eqnarray*}

\begin{eqnarray*}
a_{21}&=&\left(\frac{\tilde{\mu}_{1}}{1-\tilde{\mu}_{4}}\right)^{2}\\
K_{22}&=&\tilde{\mu}_{1}^{2}\left[R_{2}^{(2)}+f_{4}\left(4\right)\tilde{\theta}_{4}^{(2)}\right]
+2r_{4}\tilde{\mu}_{1}\left[F_{1,4}^{(1)}+f_{4}\left(1\right)\right]
+\tilde{P}_{1}^{(2)}\left[r_{4}++\frac{f_{4}\left(4\right)}{1-\tilde{\mu}_{2}}\right]
+\left[F_{1,4}^{(2)}+2f_{4}\left(1\right)F_{1,4}^{(1)}\right]
\end{eqnarray*}



\begin{eqnarray*}
f_{3}\left(1,2\right)
&=&\left(\frac{1}{1-\tilde{\mu}_{4}}\right)^{2}\tilde{\mu}_{1}\tilde{\mu}_{2}f_{4}\left(4,4\right)
+\tilde{\mu}_{1}\tilde{\mu}_{2}\left[
R_{4}^{(2)}+r_{4}+f_{4}\left(4\right)\left(\tilde{\theta}_{4}^{(2)}+\frac{1}{1-\tilde{\mu}_{2}}\right)\right]+r_{4}\tilde{\mu}_{1} \left(F_{2,4}^{(1)}+f_{4}\left(2\right)\right)\\
&+&r_{4}\tilde{\mu}_{2}\left(f_{4}\left(1\right)+F_{1,4}^{(1)}\right)+\left[f_{4}\left(2\right)F_{1,4}^{(1)}
+f_{4}\left(1\right)F_{2,4}^{(1)}+F_{2,4}^{(1)}F_{1,4}^{(1)}\right]=a_{22}f_{4}\left(4,4\right)+K_{22}
\end{eqnarray*}

\begin{eqnarray*}
a_{22}&=&\left(\frac{1}{1-\tilde{\mu}_{4}}\right)^{2}\tilde{\mu}_{1}\tilde{\mu}_{2}\\
K_{22}&=&\tilde{\mu}_{1}\tilde{\mu}_{2}\left[
R_{4}^{(2)}+r_{4}+f_{4}\left(4\right)\left(\tilde{\theta}_{4}^{(2)}+\frac{1}{1-\tilde{\mu}_{2}}\right)\right]+r_{4}\tilde{\mu}_{1} \left(F_{2,4}^{(1)}+f_{4}\left(2\right)\right)+r_{4}\tilde{\mu}_{2}\left(f_{4}\left(1\right)+F_{1,4}^{(1)}\right)\\
&+&\left[f_{4}\left(2\right)F_{1,4}^{(1)}
+f_{4}\left(1\right)F_{2,4}^{(1)}+F_{2,4}^{(1)}F_{1,4}^{(1)}\right]
\end{eqnarray*}


\begin{eqnarray*}
f_{3}\left(1,3\right)
&=&\left(\frac{1}{1-\tilde{\mu}_{4}}\right)^{2}\tilde{\mu}_{1}\tilde{\mu}_{3}f_{4}\left(4,4\right)
+\frac{\tilde{\mu}_{1}}{1-\tilde{\mu}_{4}}f_{4}\left(4,3\right)
+\tilde{\mu}_{1}\tilde{\mu}_{3}\left[R_{4}^{(2)}+r_{4}+f_{4}\left(4\right)\left(\tilde{\theta}_{4}^{(2)}
+\frac{1}{1-\tilde{\mu}_{4}}\right)\right]\\
&+&\tilde{\mu}_{3}\left[r_{4}\left(f_{4}\left(1\right)
+F_{1,4}^{(1)}\right)+r_{3}F_{1,4}^{(1)}\right]+r_{4}\tilde{\mu}_{1}f_{4}\left(3\right)\\
&=&a_{23}f_{4}\left(4,4\right)+a_{24}f_{4}\left(4,3\right)+K_{23}
\end{eqnarray*}

\begin{eqnarray*}
\begin{array}{ll}
a_{22}=\left(\frac{1}{1-\tilde{\mu}_{4}}\right)^{2}\tilde{\mu}_{1}\tilde{\mu}_{3},&
a_{23}=\frac{\tilde{\mu}_{1}}{1-\tilde{\mu}_{4}}
\end{array}
\end{eqnarray*}


\begin{eqnarray*}
K_{23}&=&\tilde{\mu}_{1}\tilde{\mu}_{3}\left[R_{4}^{(2)}+r_{4}+f_{4}\left(4\right)\left(\tilde{\theta}_{4}^{(2)}
+\frac{1}{1-\tilde{\mu}_{4}}\right)\right]
+\tilde{\mu}_{3}\left[r_{4}\left(f_{4}\left(1\right)
+F_{1,4}^{(1)}\right)+r_{3}F_{1,4}^{(1)}\right]+r_{4}\tilde{\mu}_{1}f_{4}\left(3\right)
\end{eqnarray*}


\begin{eqnarray*}
f_{3}\left(1,4\right)
&=&\tilde{\mu}_{1}\tilde{\mu}_{4}\left(
R_{4}^{(2)}+r_{4}\right)
+r_{4}\left[\tilde{\mu}_{1}f_{4}\left(4\right)
+\tilde{\mu}_{4}\left(f_{4}\left(1\right)+F_{1,4}^{(1)}
\right)\right]
+f_{4}\left(4\right)F_{1,4}^{(1)}=K_{24}\\
f_{3}\left(2,2\right)
&=&f_{4}\left(4,4\right)\left(\frac{\tilde{\mu}_{2}}{1-\tilde{\mu}_{4}}\right)^{2}
+\tilde{\mu}_{2}^{2}\left[R_{4}^{(2)}+f_{4}\left(4\right)\tilde{\theta}_{4}^{(2)}\right]
+2r_{4}\tilde{\mu}_{2}\left[F_{2,4}^{(1)}
+f_{4}\left(2\right)\right]+\tilde{P}_{2}^{(2)}\left[\frac{f_{4}\left(4\right)}{1-\tilde{\mu}_{4}}
+r_{4}\right]\\
&+&\left[2f_{4}\left(2\right)F_{2,4}^{(1)}
+F_{2,4}^{(2)}\right]=a_{25}f_{4}\left(4,4\right)+K_{25}
\end{eqnarray*}

\begin{eqnarray*}
a_{25}&=&\left(\frac{\tilde{\mu}_{2}}{1-\tilde{\mu}_{4}}\right)^{2}\\
K_{25}&=&\tilde{\mu}_{2}^{2}\left[R_{4}^{(2)}+f_{4}\left(4\right)\tilde{\theta}_{4}^{(2)}\right]
+2r_{4}\tilde{\mu}_{2}\left[F_{2,4}^{(1)}
+f_{4}\left(2\right)\right]+\tilde{P}_{2}^{(2)}\left[\frac{f_{4}\left(4\right)}{1-\tilde{\mu}_{4}}
+r_{4}\right]+\left[2f_{4}\left(2\right)F_{2,4}^{(1)}
+F_{2,4}^{(2)}\right]
\end{eqnarray*}



\begin{eqnarray*}
f_{3}\left(2,3\right)
&=&f_{4}\left(4,4\right)\left(\frac{1}{1-\tilde{\mu}_{4}}\right)^{2}\tilde{\mu}_{2}\tilde{\mu}_{3}
+f_{4}\left(4,3\right)\frac{\tilde{\mu}_{2}}{1-\tilde{\mu}_{4}}+
\tilde{\mu}_{2}\tilde{\mu}_{3}\left[
R_{4}^{(2)}
+r_{4}
+f_{4}\left(4\right)\left(\tilde{\theta}_{4}^{(2)}
+\frac{1}{1-\tilde{\mu}_{4}}\right)\right]\\
&+&r_{4}\tilde{\mu}_{3}\left[F_{2,4}^{(1)}
+f_{4}\left(2\right)\right]
+\left[r_{4}\tilde{\mu}_{2}
+F_{2,4}^{(1)}\right]f_{4}\left(3\right)=a_{26}f_{4}\left(4,4\right)+a_{27}f_{4}\left(4,3\right)+K_{26}
\end{eqnarray*}

\begin{eqnarray*}
\begin{array}{ll}
a_{26}=\left(\frac{1}{1-\tilde{\mu}_{4}}\right)^{2}\tilde{\mu}_{2}\tilde{\mu}_{3},&
a_{27}=\frac{\tilde{\mu}_{2}}{1-\tilde{\mu}_{4}}
\end{array}
\end{eqnarray*}


\begin{eqnarray*}
K_{26}&=&\tilde{\mu}_{2}\tilde{\mu}_{3}\left[
R_{4}^{(2)}
+r_{4}
+f_{4}\left(4\right)\left(\tilde{\theta}_{4}^{(2)}
+\frac{1}{1-\tilde{\mu}_{4}}\right)\right]
+r_{4}\tilde{\mu}_{3}\left[F_{2,4}^{(1)}
+f_{4}\left(2\right)\right]
+\left[r_{4}\tilde{\mu}_{2}
+F_{2,4}^{(1)}\right]f_{4}\left(3\right)
\end{eqnarray*}



\begin{eqnarray*}
f_{3}\left(2,4\right)
&=&\tilde{\mu}_{2}\tilde{\mu}_{4}\left[
R_{4}^{(2)}+r_{4}\right]+r_{4}\tilde{\mu}_{4}\left[f_{4}\left(4\right)
+F_{2,4}^{(2)}\right]+\left[r_{4}\tilde{\mu}_{2}+F_{2,4}^{(2)}\right]f_{4}\left(4\right)
=K_{27}
\end{eqnarray*}



\begin{eqnarray*}
f_{3}\left(3,3\right)
&=&f_{4}\left(4,4\right)\left(\frac{\tilde{\mu}_{3}}{1-\tilde{\mu}_{4}}\right)^{2}
+2f_{4}\left(4,3\right)\frac{\tilde{\mu}_{3}}{1-\tilde{\mu}_{4}}
+f_{4}\left(3,3\right)+\tilde{\mu}_{3}^{2}\left[R_{4}^{(2)}
+f_{4}\left(4\right)\tilde{\theta}_{4}^{(2)}\right]+\tilde{P}_{3}^{(2)}\left[r_{4}+\frac{f_{4}\left(4\right)}{1-\tilde{\mu}_{4}}\right]
\\
&+&2r_{4}\tilde{\mu}_{3}f_{4}\left(4\right)=a_{28}f_{4}\left(4,4\right)+a_{29}f_{4}\left(4,3\right)
+a_{30}f_{4}\left(3,3\right)+K_{28}
\end{eqnarray*}

\begin{eqnarray*}
\begin{array}{lll}
a_{28}=\left(\frac{\tilde{\mu}_{3}}{1-\tilde{\mu}_{4}}\right)^{2},&
a_{29}=2\frac{\tilde{\mu}_{3}}{1-\tilde{\mu}_{4}},&
a_{30}=1
\end{array}
\end{eqnarray*}

\begin{eqnarray*}
K_{28}=\tilde{\mu}_{3}^{2}\left[R_{4}^{(2)}
+f_{4}\left(4\right)\tilde{\theta}_{4}^{(2)}\right]
+\tilde{P}_{3}^{(2)}\left[r_{4}+\frac{f_{4}\left(4\right)}{1-\tilde{\mu}_{4}}\right]
+2r_{4}\tilde{\mu}_{3}f_{4}\left(4\right)
\end{eqnarray*}



\begin{eqnarray*}
f_{3}\left(3,4\right)
&=&\tilde{\mu}_{3}\tilde{\mu}_{4}\left[R_{4}^{(2)}+r_{4}\right]+r_{4}\left[\tilde{\mu}_{3}f_{4}\left(4\right)
+\tilde{\mu}_{4}f_{4}\left(3\right)\right]=K_{29}\\
f_{3}\left(4,4\right)&=&R_{4}^{(2)}\tilde{\mu}_{4}^{2}+r_{4}\tilde{P}_{4}^{(2)}+2r_{4}\tilde{\mu}_{4}f_{4}\left(4\right)=K_{30}
\end{eqnarray*}


\begin{eqnarray*}
f_{4}\left(i,k\right)&=&D_{k}D_{i}\left(R_{3}+\indora_{i\leq2}F_{1}+F_{3}\right)+D_{i}R_{3}D_{k}\left(\indora_{k\leq2}F_{1}+F_{3}\right)+D_{i}F_{3}D_{k}\left(R_{3}+\indora_{k\leq2}F_{1}\right)\\
&+&\indora_{i\leq2}D_{i}F_{1}D_{k}\left(R_{3}+F_{3}\right)
\end{eqnarray*}


\begin{eqnarray*}
f_{4}\left(1,1\right)
&=&f_{3}\left(3,3\right)\left(\frac{\tilde{\mu}_{3}}{1-\tilde{\mu}_{4}}\right)^{2}
+\tilde{\mu}_{1}^{2}\left[R_{3}^{(2)}
+\tilde{\theta}_{3}^{(2)}f_{3}\left(3\right)\right]
+\tilde{P}_{2}^{(2)}\left[r_{3}+\frac{f_{3}\left(3\right)}{1-\tilde{\mu}_{3}}\right]
+2r_{3}\tilde{\mu}_{1}\left[F_{1,3}^{(1)}
+f_{3}\left(1\right)\right]\\
&+&\left[2F_{1,3}^{(1)}f_{3}\left(1\right)+F_{1,3}^{(2)}\right]
=a_{31}f_{3}\left(3,3\right)+K_{31}
\end{eqnarray*}

\begin{eqnarray*}
a_{31}&=&\left(\frac{\tilde{\mu}_{3}}{1-\tilde{\mu}_{4}}\right)^{2}\\
K_{31}&=&\tilde{\mu}_{1}^{2}\left[R_{3}^{(2)}
+\tilde{\theta}_{3}^{(2)}f_{3}\left(3\right)\right]
+\tilde{P}_{2}^{(2)}\left[r_{3}+\frac{f_{3}\left(3\right)}{1-\tilde{\mu}_{3}}\right]
+2r_{3}\tilde{\mu}_{1}\left[F_{1,3}^{(1)}
+f_{3}\left(1\right)\right]+\left[2F_{1,3}^{(1)}f_{3}\left(1\right)+F_{1,3}^{(2)}\right]
\end{eqnarray*}



\begin{eqnarray*}
f_{4}\left(1,2\right)
&=&f_{3}\left(3,3\right)\left(\frac{1}{1-\tilde{\mu}_{3}}\right)^{2}\tilde{\mu}_{1}\tilde{\mu}_{2}
+\tilde{\mu}_{1}\tilde{\mu}_{2}\left[
R_{3}^{(2)}+r_{3}+\tilde{\theta}_{3}^{(2)}f_{3}\left(3\right)
+\frac{1}{1-\tilde{\mu}_{3}}f_{3}\left(3\right)\right]
+r_{3}\tilde{\mu}_{1}\left[f_{3}\left(2\right)+F_{2,3}^{(1)}\right]\\
&+&f_{3}\left(1\right)\left[F_{2,3}^{(1)}+r_{3}\tilde{\mu}_{2}\right]
+F_{1,3}^{(1)}\left[r_{3}\tilde{\mu}_{2}+f_{3}\left(2\right)\right]
+F_{2,3}^{(1)}F_{1,3}^{(1)}=a_{32}f_{3}\left(3,3\right)+K_{32}
\end{eqnarray*}


\begin{eqnarray*}
a_{32}&=&\left(\frac{1}{1-\tilde{\mu}_{3}}\right)^{2}\tilde{\mu}_{1}\tilde{\mu}_{2}\\
K_{32}&=&\tilde{\mu}_{1}\tilde{\mu}_{2}\left[
R_{3}^{(2)}+r_{3}+\tilde{\theta}_{3}^{(2)}f_{3}\left(3\right)
+\frac{1}{1-\tilde{\mu}_{3}}f_{3}\left(3\right)\right]
+r_{3}\tilde{\mu}_{1}\left[f_{3}\left(2\right)+F_{2,3}^{(1)}\right]
+f_{3}\left(1\right)\left[F_{2,3}^{(1)}+r_{3}\tilde{\mu}_{2}\right]
\\
&+&F_{1,3}^{(1)}\left[r_{3}\tilde{\mu}_{2}+f_{3}\left(2\right)\right]
+F_{2,3}^{(1)}F_{1,3}^{(1)}
\end{eqnarray*}




\begin{eqnarray*}
f_{4}\left(1,3\right)&=&\tilde{\mu}_{1}\tilde{\mu}_{3}\left[R_{3}^{(2)}
+r_{3}\right]
+r_{3}\tilde{\mu}_{3}\left[f_{3}\left(1\right)
+F_{1,3}^{(1)}\right]
+f_{3}\left(3\right)\left[r_{3}\tilde{\mu}_{1}+F_{1,3}^{(1)}\right]
=K_{33}\\
f_{4}\left(1,4\right)
&=&f_{3}\left(3,3\right)\left(\frac{1}{1-\tilde{\mu}_{3}}\right)^{2}\tilde{\mu}_{1}\tilde{\mu}_{3}
+f_{3}\left(3,4\right)\frac{\tilde{\mu}_{1}}{1-\tilde{\mu}_{3}}
+\tilde{\mu}_{1}\tilde{\mu}_{4}\left[f_{3}\left(3\right)\left(\tilde{\theta}_{3}^{(2)}+\frac{1}{1-\tilde{\mu}_{3}}\right)
+r_{3}+R_{3}^{(2)}\right]\\
&+&r_{3}\tilde{\mu}_{4}\left[f_{3}\left(3\right)+F_{1,3}^{(1)}\right]
+f_{3}\left(4\right)\left[r_{3}\tilde{\mu}_{1}+F_{1,3}^{(1)}\right]=a_{33}f_{3}\left(3,3\right)+a_{34}f_{3}\left(3,4\right)+K_{34}
\end{eqnarray*}



\begin{eqnarray*}
\begin{array}{ll}
a_{33}=\left(\frac{1}{1-\tilde{\mu}_{3}}\right)^{2}\tilde{\mu}_{1}\tilde{\mu}_{3},&
a_{34}=\frac{\tilde{\mu}_{1}}{1-\tilde{\mu}_{3}}
\end{array}
\end{eqnarray*}

\begin{eqnarray*}
K_{34}&=&\tilde{\mu}_{1}\tilde{\mu}_{4}\left[f_{3}\left(3\right)\left(\tilde{\theta}_{3}^{(2)}+\frac{1}{1-\tilde{\mu}_{3}}\right)
+r_{3}+R_{3}^{(2)}\right]+r_{3}\tilde{\mu}_{4}\left[f_{3}\left(3\right)+F_{1,3}^{(1)}\right]
+f_{3}\left(4\right)\left[r_{3}\tilde{\mu}_{1}+F_{1,3}^{(1)}\right]
\end{eqnarray*}

\begin{eqnarray*}
f_{4}\left(2,2\right)
&=&f_{3}\left(3,3\right)\left(\frac{\tilde{\mu}_{2}}{1-\tilde{\mu}_{3}}\right)^{2}
+\tilde{\mu}_{2}^{2}\left[R_{3}^{(2)}
+f_{3}\left(3\right)\tilde{\theta}_{3}^{(2)}\right]+2r_{3}\tilde{\mu}_{2}\left[f_{3}\left(2\right)+F_{2,3}^{(1)}\right]
+\tilde{P}_{2}^{(2)}\left[f_{3}\left(3\right)\frac{1}{1-\tilde{\mu}_{3}}
+r_{3}\right]\\
&+&\left[F_{2,3}^{(2)}
+2f_{3}\left(2\right)F_{2,3}^{(1)}\right]
=a_{35}f_{3}\left(3,3\right)+K_{35}
\end{eqnarray*}

\begin{eqnarray*}
a_{35}&=&\left(\frac{\tilde{\mu}_{2}}{1-\tilde{\mu}_{3}}\right)^{2}\\
K_{35}&=&\tilde{\mu}_{2}^{2}\left[R_{3}^{(2)}
+f_{3}\left(3\right)\tilde{\theta}_{3}^{(2)}\right]+2r_{3}\tilde{\mu}_{2}\left[f_{3}\left(2\right)+F_{2,3}^{(1)}\right]+\tilde{P}_{2}^{(2)}\left[f_{3}\left(3\right)\frac{1}{1-\tilde{\mu}_{3}}
+r_{3}\right]+\left[F_{2,3}^{(2)}
+2f_{3}\left(2\right)F_{2,3}^{(1)}\right]
\end{eqnarray*}


\begin{eqnarray*}
f_{4}\left(2,3\right)
&=&\tilde{\mu}_{2}\tilde{\mu}_{3}\left[R_{3}^{(2)}+r_{3}\right]
+r_{3}\tilde{\mu}_{3}\left[f_{3}\left(2\right)+F_{2,3}^{(1)}\right]
+\left[r_{3}\tilde{\mu}_{2}+F_{2,3}^{(1)}\right]f_{3}\left(3\right)
=K_{36}\\
f_{4}\left(2,4\right)
&=&f_{3}\left(3,3\right)\left(\frac{1}{1-\tilde{\mu}_{3}}\right)^{2}\tilde{\mu}_{2}\tilde{\mu}_{4}
+f_{3}\left(3,4\right)\frac{\tilde{\mu}_{2}}{1-\tilde{\mu}_{3}}
+\tilde{\mu}_{2}\tilde{\mu}_{4}\left[R_{3}^{(2)}
+r_{3}+f_{3}\left(3\right)\left(\tilde{\theta}_{3}^{(2)}
+\frac{1}{1-\tilde{\mu}_{3}}\right)\right]\\
&+&r_{3}\tilde{\mu}_{4}\left[f_{3}\left(2\right)
+F_{2,3}^{(1)}\right]+\left[r_{3}\tilde{\mu}_{2}+F_{2,3}^{(1)}\right]f_{3}\left(4\right)
=a_{36}f_{3}\left(3,3\right)+a_{37}f_{3}\left(3,4\right)+K_{37}
\end{eqnarray*}

\begin{eqnarray*}
\begin{array}{ll}
a_{36}=\left(\frac{1}{1-\tilde{\mu}_{3}}\right)^{2}\tilde{\mu}_{2}\tilde{\mu}_{4},&
a_{37}=\frac{\tilde{\mu}_{2}}{1-\tilde{\mu}_{3}}
\end{array}
\end{eqnarray*}

\begin{eqnarray*}
K_{37}&=&\tilde{\mu}_{2}\tilde{\mu}_{4}\left[R_{3}^{(2)}
+r_{3}+f_{3}\left(3\right)\left(\tilde{\theta}_{3}^{(2)}
+\frac{1}{1-\tilde{\mu}_{3}}\right)\right]
+r_{3}\tilde{\mu}_{4}\left[f_{3}\left(2\right)
+F_{2,3}^{(1)}\right]+\left[r_{3}\tilde{\mu}_{2}+F_{2,3}^{(1)}\right]f_{3}\left(4\right)
\end{eqnarray*}


\begin{eqnarray*}
f_{4}\left(3,3\right)
&=&R_{3}^{(2)}\tilde{\mu}_{3}^{2}+r_{3}\tilde{P}_{3}^{(2)}
+2r_{3}\tilde{\mu}_{3}f_{3}\left(3\right)=K_{38}
\end{eqnarray*}





\begin{eqnarray*}
f_{4}\left(3,4\right)
&=&\tilde{\mu}_{3}\tilde{\mu}_{4}\left[R_{3}^{(2)}+r_{3}\right]
+r_{3}\left[\tilde{\mu}_{3}f_{3}\left(4\right)
+\tilde{\mu}_{4}f_{3}\left(3\right)\right]
=K_{39}
\end{eqnarray*}


\begin{eqnarray*}
f_{4}\left(4,4\right)
&=&f_{3}\left(3,3\right)\left(\frac{\tilde{\mu}_{4}}{1-\tilde{\mu}_{3}}\right)^{2}
+2f_{3}\left(3,4\right)\frac{\tilde{\mu}_{4}}{1-\tilde{\mu}_{3}}
+f_{3}\left(4,4\right)+\tilde{\mu}_{4}^{2}\left[R_{3}^{(2)}
+f_{3}\left(3\right)\tilde{\theta}_{3}^{(2)}\right]\\
&+&\tilde{P}_{4}^{(2)}\left[f_{3}\left(3\right)\frac{1}{1-\tilde{\mu}_{3}}+r_{3}\right]
+2r_{3}\tilde{\mu}_{4}f_{3}\left(4\right)
=a_{38}f_{3}\left(3,3\right)+a_{39}f_{3}\left(3,4\right)
+a_{40}f_{3}\left(4,4\right)+K_{40}
\end{eqnarray*}

\begin{eqnarray*}
\begin{array}{lll}
a_{38}=\left(\frac{\tilde{\mu}_{4}}{1-\tilde{\mu}_{3}}\right)^{2},&
a_{39}=2\frac{\tilde{\mu}_{4}}{1-\tilde{\mu}_{3}},&
a_{40}=1
\end{array}
\end{eqnarray*}

\begin{eqnarray*}
K_{40}&=&\tilde{\mu}_{4}^{2}\left[R_{3}^{(2)}
+f_{3}\left(3\right)\tilde{\theta}_{3}^{(2)}\right]
+\tilde{P}_{4}^{(2)}\left[f_{3}\left(3\right)\frac{1}{1-\tilde{\mu}_{3}}+r_{3}\right]
+2r_{3}\tilde{\mu}_{4}f_{3}\left(4\right)
\end{eqnarray*}

so we have


%__________________________________________________________
\section{Resultados Necesarios}
%__________________________________________________________


\begin{Prop}
Sea $f\left(g\left(x\right)h\left(y\right)\right)$ funci\'on continua y con derivadas mixtas de segundo orden, entonces se tiene lo siguiente:

\begin{eqnarray*}
\frac{\partial}{\partial x}f\left(g\left(x\right)h\left(y\right)\right)=\frac{\partial f\left(g\left(x\right)h\left(y\right)\right)}{\partial x}\cdot \frac{\partial g\left(x\right)}{\partial x}\cdot h\left(y\right)
\end{eqnarray*}

por tanto


\begin{eqnarray*}
\frac{\partial}{\partial x}\frac{\partial}{\partial x}f\left(g\left(x\right)h\left(y\right)\right)&=&\frac{\partial}{\partial x}\left\{\frac{\partial f\left(g\left(x\right)h\left(y\right)\right)}{\partial x}\cdot \frac{\partial g\left(x\right)}{\partial x}\cdot h\left(y\right)\right\}\\
&=&\frac{\partial}{\partial x}\left\{\frac{\partial}{\partial x}f\left(g\left(x\right)h\left(y\right)\right)\right\}\cdot \frac{\partial g\left(x\right)}{\partial x}\cdot h\left(y\right)+\frac{\partial}{\partial x}f\left(g\left(x\right)h\left(y\right)\right)\cdot \frac{\partial g^{2}\left(x\right)}{\partial x^{2}}\cdot h\left(y\right)\\
&=&\frac{\partial^{2}}{\partial x}f\left(g\left(x\right)h\left(y\right)\right)\cdot \frac{\partial g\left(x\right)}{\partial x}\cdot h\left(y\right)\cdot \frac{\partial g\left(x\right)}{\partial x}\cdot h\left(y\right)+\frac{\partial}{\partial x}f\left(g\left(x\right)h\left(y\right)\right)\cdot \frac{\partial g^{2}\left(x\right)}{\partial x^{2}}\cdot h\left(y\right)\\
&=&\frac{\partial^{2}}{\partial x}f\left(g\left(x\right)h\left(y\right)\right)\cdot \left(\frac{\partial g\left(x\right)}{\partial x}\right)^{2}\cdot h^{2}\left(y\right)+\frac{\partial}{\partial x}f\left(g\left(x\right)h\left(y\right)\right)\cdot \frac{\partial g^{2}\left(x\right)}{\partial x^{2}}\cdot h\left(y\right)
\end{eqnarray*}


\end{Prop}



%___________________________________________________________________________________________
%
\subsubsection{Expresion de las Parciales mixtas para $F_{1}$ y $F_{2}$}
%___________________________________________________________________________________________
\begin{enumerate}

%1

\item \begin{eqnarray*}
\frac{\partial}{\partial z_{1}}\frac{\partial}{\partial z_{1}}F_{1}\left(\theta_{1}\left(\tilde{P}_{2}\left(z_{2}\right)\hat{P}_{1}\left(w_{1}\right)
\hat{P}_{2}\left(w_{2}\right),z_{2}\right)\right)|_{\mathbf{z,w}=1}&=&0\\
\end{eqnarray*}

%2

\item
\begin{eqnarray*}
\frac{\partial}{\partial z_{2}}\frac{\partial}{\partial z_{1}}F_{1}\left(\theta_{1}\left(\tilde{P}_{2}\left(z_{2}\right)\hat{P}_{1}\left(w_{1}\right)
\hat{P}_{2}\left(w_{2}\right),z_{2}\right)\right)|_{\mathbf{z,w}=1}&=&0\\
\end{eqnarray*}

%3

\item
\begin{eqnarray*}
\frac{\partial}{\partial w_{1}}\frac{\partial}{\partial z_{1}}F_{1}\left(\theta_{1}\left(\tilde{P}_{2}\left(z_{2}\right)\hat{P}_{1}\left(w_{1}\right)
\hat{P}_{2}\left(w_{2}\right),z_{2}\right)\right)|_{\mathbf{z,w}=1}&=&0\\
\end{eqnarray*}

%4

\item
\begin{eqnarray*}
\frac{\partial}{\partial w_{2}}\frac{\partial}{\partial z_{1}}F_{1}\left(\theta_{1}\left(\tilde{P}_{2}\left(z_{2}\right)\hat{P}_{1}\left(w_{1}\right)
\hat{P}_{2}\left(w_{2}\right),z_{2}\right)\right)|_{\mathbf{z,w}=1}&=&0
\end{eqnarray*}

%5

\item
\begin{eqnarray*}
\frac{\partial}{\partial z_{1}}\frac{\partial}{\partial z_{2}}F_{1}\left(\theta_{1}\left(\tilde{P}_{2}\left(z_{2}\right)\hat{P}_{1}\left(w_{1}\right)
\hat{P}_{2}\left(w_{2}\right),z_{2}\right)\right)|_{\mathbf{z,w}=1}&=&0
\end{eqnarray*}

%6

\item
\begin{eqnarray*}
&&\frac{\partial}{\partial z_{2}}\frac{\partial}{\partial z_{2}}F_{1}\left(\theta_{1}\left(\tilde{P}_{2}\left(z_{2}\right)\hat{P}_{1}\left(w_{1}\right)
\hat{P}_{2}\left(w_{2}\right)\right),z_{2}\right)|_{\mathbf{z,w}=1}=f_{1}\left(2,2\right)+\frac{1}{1-\mu_{1}}\tilde{P}_{2}^{(2)}\left(1\right)f_{1}\left(1\right)\\
&+&\tilde{\mu}_{2}^{2}\theta_{1}^{(2)}\left(1\right)f_{1}\left(1\right)+2\frac{\tilde{\mu}_{2}}{1-\mu_{1}}f_{1}\left(1,2\right)+\left(\frac{\tilde{\mu}_{2}}{1-\mu_{1}}\right)^{2}f_{1}\left(1,1\right)
\end{eqnarray*}

%7

\item
\begin{eqnarray*}
&&\frac{\partial}{\partial w_{1}}\frac{\partial}{\partial z_{2}}F_{1}\left(\theta_{1}\left(\tilde{P}_{2}\left(z_{2}\right)\hat{P}_{1}\left(w_{1}\right)
\hat{P}_{2}\left(w_{2}\right),z_{2}\right)\right)|_{\mathbf{z,w}=1}=\frac{\tilde{\mu}_{2}\hat{\mu}_{1}}{1-\mu_{1}}f_{1}\left(1\right)\\
&+&\tilde{\mu}_{2}\hat{\mu}_{1}\theta_{1}^{(2)}\left(1\right)f_{1}\left(1\right)+\frac{\hat{\mu}_{1}}{1-\mu_{1}}f_{1}\left(1,2\right)+\tilde{\mu}_{2}\hat{\mu}_{1}\left(\frac{1}{1-\mu_{1}}\right)^{2}f_{1}\left(1,1\right)
\end{eqnarray*}

%8

\item \begin{eqnarray*}
&&\frac{\partial}{\partial w_{2}}\frac{\partial}{\partial z_{2}}F_{1}\left(\theta_{1}\left(\tilde{P}_{2}\left(z_{2}\right)\hat{P}_{1}\left(w_{1}\right)
\hat{P}_{2}\left(w_{2}\right),z_{2}\right)\right)|_{\mathbf{z,w}=1}=\frac{\tilde{\mu}_{2}\hat{\mu}_{2}}{1-\mu_{1}}f_{1}\left(1\right)\\
&+&\tilde{\mu}_{2}\hat{\mu}_{2}\theta_{1}^{(2)}\left(1\right)f_{1}\left(1\right)+\frac{\hat{\mu}_{2}}{1-\mu_{1}}f_{1}\left(1,2\right)+\tilde{\mu}_{2}\hat{\mu}_{2}\left(\frac{1}{1-\mu_{1}}\right)^{2}f_{1}\left(1,1\right)
\end{eqnarray*}

%9

\item \begin{eqnarray*}
\frac{\partial}{\partial z_{1}}\frac{\partial}{\partial w_{1}}F_{1}\left(\theta_{1}\left(\tilde{P}_{2}\left(z_{2}\right)\hat{P}_{1}\left(w_{1}\right)
\hat{P}_{2}\left(w_{2}\right),z_{2}\right)\right)|_{\mathbf{z,w}=1}&=&0
\end{eqnarray*}

%10

\item \begin{eqnarray*}
&&\frac{\partial}{\partial z_{2}}\frac{\partial}{\partial w_{1}}F_{1}\left(\theta_{1}\left(\tilde{P}_{2}\left(z_{2}\right)\hat{P}_{1}\left(w_{1}\right)
\hat{P}_{2}\left(w_{2}\right),z_{2}\right)\right)|_{\mathbf{z,w}=1}=\frac{\tilde{\mu}_{2}\hat{\mu}_{1}}{1-\mu_{1}}f_{1}\left(2\right)\\
&+&\tilde{\mu}_{2}\hat{\mu}_{1}\theta_{1}^{(2)}\left(1\right)f_{1}\left(2\right)+\frac{\hat{\mu}_{1}}{1-\mu_{1}}f_{1}\left(2,1\right)+\tilde{\mu}_{2}\hat{\mu}_{1}\left(\frac{1}{1-\mu_{1}}\right)^{2}f_{1}\left(1,1\right)
\end{eqnarray*}

%11

\item
\begin{eqnarray*}
&&\frac{\partial}{\partial w_{1}}\frac{\partial}{\partial w_{1}}F_{1}\left(\theta_{1}\left(\tilde{P}_{2}\left(z_{2}\right)\hat{P}_{1}\left(w_{1}\right)
\hat{P}_{2}\left(w_{2}\right),z_{2}\right)\right)|_{\mathbf{z,w}=1}=\frac{1}{1-\mu_{1}} \hat{P}_{1}^{(2)}\left(1\right)f_{1}\left(1\right)\\
&+&\hat{\mu}_{1}\theta_{1}^{(2)}\left(1\right)f_{1}\left(1\right)+\left(\frac{\hat{\mu}_{1}}{1-\mu_{1}}\right)^{2}f_{1}\left(1,1\right)
\end{eqnarray*}

%12

\item
\begin{eqnarray*}
&&\frac{\partial}{\partial w_{2}}\frac{\partial}{\partial w_{1}}F_{1}\left(\theta_{1}\left(\tilde{P}_{2}\left(z_{2}\right)\hat{P}_{1}\left(w_{1}\right)
\hat{P}_{2}\left(w_{2}\right),z_{2}\right)\right)|_{\mathbf{z,w}=1}=\hat{\mu}_{1}\hat{\mu}_{2}f_{1}\left(1\right)\\
&+&\frac{\hat{\mu}_{1}\hat{\mu}_{2}}{1-\mu_{1}}f_{1}\left(1\right)+\hat{\mu}_{1}\hat{\mu}_{2}\theta_{1}^{(2)}\left(1\right)f_{1}\left(1\right)+\hat{\mu}_{1}\hat{\mu}_{2}\left(\frac{1}{1-\mu_{1}}\right)^{2}f_{1}\left(1,1\right)
\end{eqnarray*}

%13

\item \begin{eqnarray*}
\frac{\partial}{\partial z_{1}}\frac{\partial}{\partial w_{2}}F_{1}\left(\theta_{1}\left(\tilde{P}_{2}\left(z_{2}\right)\hat{P}_{1}\left(w_{1}\right)
\hat{P}_{2}\left(w_{2}\right),z_{2}\right)\right)|_{\mathbf{z,w}=1}&=&0
\end{eqnarray*}

%14

\item \begin{eqnarray*}
&&\frac{\partial}{\partial z_{2}}\frac{\partial}{\partial w_{2}}F_{1}\left(\theta_{1}\left(\tilde{P}_{2}\left(z_{2}\right)\hat{P}_{1}\left(w_{1}\right)
\hat{P}_{2}\left(w_{2}\right),z_{2}\right)\right)|_{\mathbf{z,w}=1}=\frac{\tilde{\mu}_{2}\hat{\mu}_{2}}{1-\mu_{1}}f_{1}\left(1\right)\\
&+&\tilde{\mu}_{2}\hat{\mu}_{2}\theta_{1}^{(2)}\left(1\right)f_{1}\left(1\right)+\frac{\hat{\mu}_{2}}{1-\mu_{1}}f_{1}\left(2,1\right)+\tilde{\mu}_{2}\hat{\mu}_{2}\left(\frac{1}{1-\mu_{1}}\right)^{2}f_{1}\left(2,2\right)
\end{eqnarray*}

%15

\item \begin{eqnarray*}
&&\frac{\partial}{\partial w_{1}}\frac{\partial}{\partial w_{2}}F_{1}\left(\theta_{1}\left(\tilde{P}_{2}\left(z_{2}\right)\hat{P}_{1}\left(w_{1}\right)
\hat{P}_{2}\left(w_{2}\right),z_{2}\right)\right)|_{\mathbf{z,w}=1}=\frac{\hat{\mu}_{1}\hat{\mu}_{2}}{1-\mu_{1}}f_{1}\left(1\right)\\
&+&\hat{\mu}_{1}\hat{\mu}_{2}\theta_{1}^{(2)}\left(1\right)f_{1}\left(1\right)+\hat{\mu}_{1}\hat{\mu}_{2}\left(\frac{1}{1-\mu_{1}}\right)^{2}f_{1}\left(1,1\right)
\end{eqnarray*}

%16

\item
\begin{eqnarray*}
&&\frac{\partial}{\partial w_{2}}\frac{\partial}{\partial w_{2}}F_{1}\left(\theta_{1}\left(\tilde{P}_{2}\left(z_{2}\right)\hat{P}_{1}\left(w_{1}\right)
\hat{P}_{2}\left(w_{2}\right),z_{2}\right)\right)|_{\mathbf{z,w}=1}=\frac{1}{1-\mu_{1}}\hat{P}_{2}^{(2)}\left(w_{2}\right)f_{1}\left(1\right)\\
&+&\hat{\mu}_{2}^{2}\theta_{1}^{(2)}\left(1\right)f_{1}\left(1\right)+\left(\hat{\mu}_{2}\frac{1}{1-\mu_{1}}\right)^{2}f_{1}\left(1,1\right)
\end{eqnarray*}

%17

\item
\begin{eqnarray*}
&&\frac{\partial}{\partial z_{1}}\frac{\partial}{\partial z_{1}}F_{2}\left(z_{1},\tilde{\theta}_{2}\left(P_{1}\left(z_{1}\right)\hat{P}_{1}\left(w_{1}\right)
\hat{P}_{2}\left(w_{2}\right)\right)\right)|_{\mathbf{z,w}=1}=\frac{1}{1-\tilde{\mu}_{2}}P_{1}^{(2)}\left(1\right)
f_{2}\left(2\right)+f_{2}\left(1,1\right)\\
&+&\mu_{1}^{2}\tilde{\theta}_{2}^{(2)}\left(1\right)f_{2}\left(2\right)+\mu_{1}\frac{1}{1-\tilde{\mu}_{2}}f_{2}\left(1,2\right)+\left(\mu_{1}\frac{1}{1-\tilde{\mu}_{2}}\right)^{2}f_{2}\left(2,2\right)+\frac{\mu_{1}}{1-\tilde{\mu}_{2}}f_{2}\left(1,2\right)\\
\end{eqnarray*}

%18

\item \begin{eqnarray*}
\frac{\partial}{\partial z_{2}}\frac{\partial}{\partial z_{1}}F_{2}\left(z_{1},\tilde{\theta}_{2}\left(P_{1}\left(z_{1}\right)\hat{P}_{1}\left(w_{1}\right)
\hat{P}_{2}\left(w_{2}\right)\right)\right)|_{\mathbf{z,w}=1}&=&0
\end{eqnarray*}

%19

\item \begin{eqnarray*}
&&\frac{\partial}{\partial w_{1}}\frac{\partial}{\partial z_{1}}F_{2}\left(z_{1},\tilde{\theta}_{2}\left(P_{1}\left(z_{1}\right)\hat{P}_{1}\left(w_{1}\right)
\hat{P}_{2}\left(w_{2}\right)\right)\right)|_{\mathbf{z,w}=1}=\frac{\mu_{1}\hat{\mu}_{1}}{1-\tilde{\mu}_{2}}f_{2}\left(2\right)\\
&+&\mu_{1}\hat{\mu}_{1}\tilde{\theta}_{2}^{(2)}\left(1\right)f_{2}\left(2\right)+\mu_{1}\hat{\mu}_{1}\left(\frac{1}{1-\tilde{\mu}_{2}}\right)^{2}f_{2}\left(2,2\right)+\frac{\hat{\mu}_{1}}{1-\tilde{\mu}_{2}}f_{2}\left(1,2\right)\end{eqnarray*}

%20

\item \begin{eqnarray*}
&&\frac{\partial}{\partial w_{2}}\frac{\partial}{\partial z_{1}}F_{2}\left(z_{1},\tilde{\theta}_{2}\left(P_{1}\left(z_{1}\right)\hat{P}_{1}\left(w_{1}\right)
\hat{P}_{2}\left(w_{2}\right)\right)\right)|_{\mathbf{z,w}=1}=\frac{\mu_{1}\hat{\mu}_{2}}{1-\tilde{\mu}_{2}}f_{2}\left(2\right)\\
&+&\mu_{1}\hat{\mu}_{2}\tilde{\theta}_{2}^{(2)}\left(1\right)f_{2}\left(2\right)+\mu_{1}\hat{\mu}_{2}
\left(\frac{1}{1-\tilde{\mu}_{2}}\right)^{2}f_{2}\left(2,2\right)+\frac{\hat{\mu}_{2}}{1-\tilde{\mu}_{2}}f_{2}\left(1,2\right)\end{eqnarray*}
%___________________________________________________________________________________________


%\newpage

%___________________________________________________________________________________________
%
%\section{Parciales mixtas de $F_{2}$ para $z_{2}$}
%___________________________________________________________________________________________
%___________________________________________________________________________________________
\item
\begin{eqnarray*}
\frac{\partial}{\partial z_{1}}\frac{\partial}{\partial z_{2}}F_{2}\left(z_{1},\tilde{\theta}_{2}\left(P_{1}\left(z_{1}\right)\hat{P}_{1}\left(w_{1}\right)
\hat{P}_{2}\left(w_{2}\right)\right)\right)|_{\mathbf{z,w}=1}&=&0;\\
\end{eqnarray*}
\item
\begin{eqnarray*}
\frac{\partial}{\partial z_{2}}\frac{\partial}{\partial z_{2}}F_{2}\left(z_{1},\tilde{\theta}_{2}\left(P_{1}\left(z_{1}\right)\hat{P}_{1}\left(w_{1}\right)
\hat{P}_{2}\left(w_{2}\right)\right)\right)|_{\mathbf{z,w}=1}&=&0\\
\end{eqnarray*}
\item
\begin{eqnarray*}\frac{\partial}{\partial w_{1}}\frac{\partial}{\partial z_{2}}F_{2}\left(z_{1},\tilde{\theta}_{2}\left(P_{1}\left(z_{1}\right)\hat{P}_{1}\left(w_{1}\right)
\hat{P}_{2}\left(w_{2}\right)\right)\right)|_{\mathbf{z,w}=1}&=&0\\
\end{eqnarray*}
\item
\begin{eqnarray*}\frac{\partial}{\partial w_{2}}\frac{\partial}{\partial z_{2}}F_{2}\left(z_{1},\tilde{\theta}_{2}\left(P_{1}\left(z_{1}\right)\hat{P}_{1}\left(w_{1}\right)
\hat{P}_{2}\left(w_{2}\right)\right)\right)|_{\mathbf{z,w}=1}&=&0
\end{eqnarray*}
%___________________________________________________________________________________________

%\newpage

%___________________________________________________________________________________________
%
%\section{Parciales mixtas de $F_{2}$ para $w_{1}$}
%___________________________________________________________________________________________
\item
\begin{eqnarray*}
\frac{\partial}{\partial z_{1}}\frac{\partial}{\partial w_{1}}F_{2}\left(z_{1},\tilde{\theta}_{2}\left(P_{1}\left(z_{1}\right)\hat{P}_{1}\left(w_{1}\right)
\hat{P}_{2}\left(w_{2}\right)\right)\right)|_{\mathbf{z,w}=1}&=&\frac{1}{1-\tilde{\mu}_{2}}P_{1}^{(2)}\left(1\right)\frac{\partial}{\partial
z_{2}}F_{2}\left(1,1\right)+\mu_{1}^{2}\tilde{\theta}_{2}^{(2)}\left(1\right)\frac{\partial}{\partial
z_{2}}F_{2}\left(1,1\right)\\
&+&\mu_{1}\frac{1}{1-\tilde{\mu}_{2}}f_{2}\left(1,2\right)+\left(\mu_{1}\frac{1}{1-\tilde{\mu}_{2}}\right)^{2}f_{2}\left(2,2\right)\\
&+&\mu_{1}\frac{1}{1-\tilde{\mu}_{2}}f_{2}\left(1,2\right)+f_{2}\left(1,1\right)
\end{eqnarray*}
%___________________________________________________________________________________________
%___________________________________________________________________________________________
\item \begin{eqnarray*}
\frac{\partial}{\partial z_{2}}\frac{\partial}{\partial w_{1}}F_{2}\left(z_{1},\tilde{\theta}_{2}\left(P_{1}\left(z_{1}\right)\hat{P}_{1}\left(w_{1}\right)
\hat{P}_{2}\left(w_{2}\right)\right)\right)|_{\mathbf{z,w}=1}&=&0
\end{eqnarray*}
%___________________________________________________________________________________________
\item
\begin{eqnarray*}
\frac{\partial}{\partial w_{1}}\frac{\partial}{\partial w_{1}}F_{2}\left(z_{1},\tilde{\theta}_{2}\left(P_{1}\left(z_{1}\right)\hat{P}_{1}\left(w_{1}\right)
\hat{P}_{2}\left(w_{2}\right)\right)\right)|_{\mathbf{z,w}=1}&=&\mu_{1}\hat{\mu}_{1}\frac{1}{1-\tilde{\mu}_{2}}\frac{\partial}{\partial
z_{2}}F_{2}\left(1,1\right)+\mu_{1}\hat{\mu}_{1}\left(\frac{1}{1-\tilde{\mu}_{2}}\right)^{2}\frac{\partial}{\partial
z_{2}}F_{2}\left(1,1\right)\\
&+&\mu_{1}\hat{\mu}_{1}
\left(\frac{1}{1-\tilde{\mu}_{2}}\right)^{2}\frac{\partial}{\partial
z_{2}}F_{2}\left(1,1\right)+\hat{\mu}_{1}\frac{1}{1-\tilde{\mu}_{2}}f_{2}\left(1,2\right)\end{eqnarray*}
\item
\begin{eqnarray*}
\frac{\partial}{\partial w_{2}}\frac{\partial}{\partial w_{1}}F_{2}\left(z_{1},\tilde{\theta}_{2}\left(P_{1}\left(z_{1}\right)\hat{P}_{1}\left(w_{1}\right)
\hat{P}_{2}\left(w_{2}\right)\right)\right)|_{\mathbf{z,w}=1}&=&\hat{\mu}_{1}\hat{\mu}_{2}\frac{1}{1-\tilde{\mu}_{2}}\frac{\partial}{\partial
z_{2}}F_{2}\left(1,1\right)+\hat{\mu}_{1}\hat{\mu}_{2}\tilde{\theta}_{2}^{(2)}\left(1\right)\frac{\partial}{\partial
z_{2}}F_{2}\left(1,1\right)\\
&+&\hat{\mu}_{1}\hat{\mu}_{2}\left(\frac{1}{1-\tilde{\mu}_{2}}\right)^{2}f_{2}\left(2,2\right)\end{eqnarray*}
%___________________________________________________________________________________________

%\newpage

%___________________________________________________________________________________________
%
%\section{Parciales mixtas de $F_{2}$ para $w_{2}$}
%___________________________________________________________________________________________
%___________________________________________________________________________________________
\item \begin{eqnarray*}
\frac{\partial}{\partial z_{1}}\frac{\partial}{\partial w_{2}}F_{2}\left(z_{1},\tilde{\theta}_{2}\left(P_{1}\left(z_{1}\right)\hat{P}_{1}\left(w_{1}\right)
\hat{P}_{2}\left(w_{2}\right)\right)\right)|_{\mathbf{z,w}=1}&=&\mu_{1}\hat{\mu}_{2}\frac{1}{1-\tilde{\mu}_{2}}\frac{\partial}{\partial
z_{1}}F_{2}\left(1\right)+\mu_{1}\hat{\mu}_{2}\tilde{\theta}_{2}^{(2)}\left(1\right)\frac{\partial}{\partial
z_{2}}F_{2}\left(1,1\right)\\
&+&\hat{\mu}_{2}\mu_{1}\left(\frac{1}{1-\tilde{\mu}_{2}}\right)^{2}f_{2}\left(2,2\right)+\hat{\mu}_{2}\frac{1}{1-\tilde{\mu}_{2}}f_{2}\left(1,2\right)\end{eqnarray*}
\item
\begin{eqnarray*}
\frac{\partial}{\partial z_{2}}\frac{\partial}{\partial w_{2}}F_{2}\left(z_{1},\tilde{\theta}_{2}\left(P_{1}\left(z_{1}\right)\hat{P}_{1}\left(w_{1}\right)
\hat{P}_{2}\left(w_{2}\right)\right)\right)|_{\mathbf{z,w}=1}&=&0
\end{eqnarray*}
\item
\begin{eqnarray*}
\frac{\partial}{\partial w_{1}}\frac{\partial}{\partial w_{2}}F_{2}\left(z_{1},\tilde{\theta}_{2}\left(P_{1}\left(z_{1}\right)\hat{P}_{1}\left(w_{1}\right)
\hat{P}_{2}\left(w_{2}\right)\right)\right)|_{\mathbf{z,w}=1}&=&\hat{\mu}_{1}\hat{\mu}_{2}\frac{1}{1-\tilde{\mu}_{2}}\frac{\partial}{\partial
z_{2}}F_{2}\left(1,1\right)+\hat{\mu}_{1}\hat{\mu}_{2}\tilde{\theta}_{2}^{(2)}\left(1\right)\frac{\partial}{\partial
z_{2}}F_{2}\left(1,1\right)\\
&+&\hat{\mu}_{1}\hat{\mu}_{2}\left(\frac{1}{1-\tilde{\mu}_{2}}\right)^{2}f_{2}\left(2,2\right)\end{eqnarray*}
\item
\begin{eqnarray*}
\frac{\partial}{\partial w_{2}}\frac{\partial}{\partial w_{2}}F_{2}\left(z_{1},\tilde{\theta}_{2}\left(P_{1}\left(z_{1}\right)\hat{P}_{1}\left(w_{1}\right)
\hat{P}_{2}\left(w_{2}\right)\right)\right)|_{\mathbf{z,w}=1}&=&\hat{P}_{2}^{(2)}\left(1\right)\frac{1}{1-\tilde{\mu}_{2}}\frac{\partial}{\partial
z_{2}}F_{2}\left(1,1\right)+\hat{\mu}_{2}^{2}\tilde{\theta}_{2}^{(2)}\left(1\right)\frac{\partial}{\partial
z_{2}}F_{2}\left(1,1\right)\\
&+&\left(\hat{\mu}_{2}\frac{1}{1-\tilde{\mu}_{2}}\right)^{2}f_{2}\left(2,2\right)
\end{eqnarray*}
%___________________________________________________________________________________________




%\newpage
%___________________________________________________________________________________________
%
%\section{Parciales mixtas de $\hat{F}_{1}$ para $z_{1}$}
%___________________________________________________________________________________________
\item \begin{eqnarray*}
\frac{\partial}{\partial z_{1}}\frac{\partial}{\partial z_{1}}\hat{F}_{1}\left(\hat{\theta}_{1}\left(P_{1}\left(z_{1}\right)\tilde{P}_{2}\left(z_{2}\right)
\hat{P}_{2}\left(w_{2}\right)\right),w_{2}\right)|_{\mathbf{z,w}=1}&=&\frac{1}{1-\hat{\mu}_{1}}P_{1}^{(2)}\frac{\partial}{\partial w_{1}}\hat{F}_{1}\left(1,1\right)+\mu_{1}^2\hat{\theta}_{1}^{(2)}\left(1\right)\frac{\partial}{\partial w_{1}}\hat{F}_{1}\left(1,1\right)\\
&+&\mu_{1}^2\left(\frac{1}{1- \hat{\mu}_{1}}\right)^2\hat{f}_{1}\left(1,1\right)
\end{eqnarray*}
%___________________________________________________________________________________________

%___________________________________________________________________________________________
\item
\begin{eqnarray*}
\frac{\partial}{\partial z_{2}}\frac{\partial}{\partial z_{1}}\hat{F}_{1}\left(\hat{\theta}_{1}\left(P_{1}\left(z_{1}\right)\tilde{P}_{2}\left(z_{2}\right)
\hat{P}_{2}\left(w_{2}\right)\right),w_{2}\right)|_{\mathbf{z,w}=1}&=&\mu_{1}\frac{1}{1-\hat{\mu}_{1}}\tilde{\mu}_{2}\frac{\partial}{\partial w_{1}}\hat{F}_{1}\left(1,1\right)\\
&+&\mu_{1}\tilde{\mu}_{2}\hat{\theta
}_{1}^{(2)}\left(1\right)\frac{\partial}{\partial w_{1}}\hat{F}_{1}\left(1,1\right)\\
&+&\mu_{1}\left(\frac{1}{1-\hat{\mu}_{1}}\right)^2\tilde{\mu}_{2}\hat{f}_{1}\left(1,1\right)
\end{eqnarray*}
%___________________________________________________________________________________________

%___________________________________________________________________________________________
\item \begin{eqnarray*}
\frac{\partial}{\partial w_{1}}\frac{\partial}{\partial z_{1}}\hat{F}_{1}\left(\hat{\theta}_{1}\left(P_{1}\left(z_{1}\right)\tilde{P}_{2}\left(z_{2}\right)
\hat{P}_{2}\left(w_{2}\right)\right),w_{2}\right)|_{\mathbf{z,w}=1}&=&0
\end{eqnarray*}
%___________________________________________________________________________________________

%___________________________________________________________________________________________
\item
\begin{eqnarray*}
\frac{\partial}{\partial w_{2}}\frac{\partial}{\partial z_{1}}\hat{F}_{1}\left(\hat{\theta}_{1}\left(P_{1}\left(z_{1}\right)\tilde{P}_{2}\left(z_{2}\right)
\hat{P}_{2}\left(w_{2}\right)\right),w_{2}\right)|_{\mathbf{z,w}=1}&=&\mu_{1}
\hat{\mu}_{2}\frac{1}{1-\hat{\mu
}_{1}}\frac{\partial}{\partial w_{1}}\hat{F}_{1}\left(1,1\right)+\mu_{1}\hat{\mu}_{2} \hat{\theta
}_{1}^{(2)}\left(1\right)\frac{\partial}{\partial w_{1}}\hat{F}_{1}\left(1,1\right)\\
&+&\mu_{1}\frac{1}{1-\hat{\mu}_{1}}f_{1}\left(1,2\right)+\mu_{1}\hat{\mu}_{2}\left(\frac{1}{1-\hat{\mu}_{1}}\right)^{2}\hat{f}_{1}\left(1,1\right)
\end{eqnarray*}
%___________________________________________________________________________________________


%___________________________________________________________________________________________
%
%\section{Parciales mixtas de $\hat{F}_{1}$ para $z_{2}$}
%___________________________________________________________________________________________
\item
\begin{eqnarray*}
\frac{\partial}{\partial z_{1}}\frac{\partial}{\partial z_{2}}\hat{F}_{1}\left(\hat{\theta}_{1}\left(P_{1}\left(z_{1}\right)\tilde{P}_{2}\left(z_{2}\right)
\hat{P}_{2}\left(w_{2}\right)\right),w_{2}\right)|_{\mathbf{z,w}=1}&=&\mu_{1}\tilde{\mu}_{2}\frac{1}{1-\hat{\mu}_{1}}\frac{\partial}{\partial w_{1}}
\hat{F}_{1}\left(1,1\right)+\mu_{1}\tilde{\mu}_{2}\hat{\theta
}_{1}^{(2)}\left(1\right)\frac{\partial}{\partial w_{1}}\hat{F}_{1}\left(1,1\right)\\
&+&\mu_{1}\tilde{\mu}_{2}\left(\frac{1}{1-\hat{\mu}_{1}}\right)^{2}\hat{f}_{1}\left(1,1\right)
\end{eqnarray*}
%___________________________________________________________________________________________

%___________________________________________________________________________________________
\item
\begin{eqnarray*}
\frac{\partial}{\partial z_{2}}\frac{\partial}{\partial z_{2}}\hat{F}_{1}\left(\hat{\theta}_{1}\left(P_{1}\left(z_{1}\right)\tilde{P}_{2}\left(z_{2}\right)
\hat{P}_{2}\left(w_{2}\right)\right),w_{2}\right)|_{\mathbf{z,w}=1}&=&\tilde{\mu}_{2}^{2}\hat{\theta
}_{1}^{(2)}\left(1\right)\frac{\partial}{\partial w_{1}}\hat{F}_{1}\left(1,1\right)+\frac{1}{1-\hat{\mu}_{1}}\tilde{P}_{2}^{(2)}\frac{\partial}{\partial w_{1}}\hat{F}_{1}\left(1,1\right)\\
&+&\tilde{\mu}_{2}^{2}\left(\frac{1}{1-\hat{\mu}_{1}}\right)^{2}\hat{f}_{1}\left(1,1\right)
\end{eqnarray*}
%___________________________________________________________________________________________

%___________________________________________________________________________________________
\item \begin{eqnarray*}
\frac{\partial}{\partial w_{1}}\frac{\partial}{\partial z_{2}}\hat{F}_{1}\left(\hat{\theta}_{1}\left(P_{1}\left(z_{1}\right)\tilde{P}_{2}\left(z_{2}\right)
\hat{P}_{2}\left(w_{2}\right)\right),w_{2}\right)|_{\mathbf{z,w}=1}&=&0
\end{eqnarray*}
%___________________________________________________________________________________________
%___________________________________________________________________________________________
\item
\begin{eqnarray*}
\frac{\partial}{\partial w_{2}}\frac{\partial}{\partial z_{2}}\hat{F}_{1}\left(\hat{\theta}_{1}\left(P_{1}\left(z_{1}\right)\tilde{P}_{2}\left(z_{2}\right)
\hat{P}_{2}\left(w_{2}\right)\right),w_{2}\right)|_{\mathbf{z,w}=1}&=&\hat{\mu}_{2}\tilde{\mu}_{2}\frac{1}{1-\hat{\mu}_{1}}
\frac{\partial}{\partial w_{1}}\hat{F}_{1}\left(1,1\right)+\hat{\mu}_{2}\tilde{\mu}_{2}\hat{\theta
}_{1}^{(2)}\left(1\right)\frac{\partial}{\partial w_{1}}\hat{F}_{1}\left(1,1\right)\\
&+&\frac{1}{1-\hat{\mu
}_{1}}\tilde{\mu}_{2}\hat{f}_{1}\left(1,2\right)+\tilde{\mu}_{2}\hat{\mu}_{2}\left(\frac{1}{1-\hat{\mu}_{1}}\right)^{2}\hat{f}_{1}\left(1,1\right)
\end{eqnarray*}
%___________________________________________________________________________________________

%\newpage

%___________________________________________________________________________________________
%
%\section{Parciales mixtas de $\hat{F}_{1}$ para $w_{1}$}
%___________________________________________________________________________________________
%___________________________________________________________________________________________
\item \begin{eqnarray*}
\frac{\partial}{\partial z_{1}}\frac{\partial}{\partial w_{1}}\hat{F}_{1}\left(\hat{\theta}_{1}\left(P_{1}\left(z_{1}\right)\tilde{P}_{2}\left(z_{2}\right)
\hat{P}_{2}\left(w_{2}\right)\right),w_{2}\right)|_{\mathbf{z,w}=1}&=&0
\end{eqnarray*}
%___________________________________________________________________________________________

%___________________________________________________________________________________________
\item
\begin{eqnarray*}
\frac{\partial}{\partial z_{2}}\frac{\partial}{\partial w_{1}}\hat{F}_{1}\left(\hat{\theta}_{1}\left(P_{1}\left(z_{1}\right)\tilde{P}_{2}\left(z_{2}\right)
\hat{P}_{2}\left(w_{2}\right)\right),w_{2}\right)|_{\mathbf{z,w}=1}&=&0
\end{eqnarray*}
%___________________________________________________________________________________________

%___________________________________________________________________________________________
\item
\begin{eqnarray*}
\frac{\partial}{\partial w_{1}}\frac{\partial}{\partial w_{1}}\hat{F}_{1}\left(\hat{\theta}_{1}\left(P_{1}\left(z_{1}\right)\tilde{P}_{2}\left(z_{2}\right)
\hat{P}_{2}\left(w_{2}\right)\right),w_{2}\right)|_{\mathbf{z,w}=1}&=&0
\end{eqnarray*}
%___________________________________________________________________________________________

%___________________________________________________________________________________________
\item
\begin{eqnarray*}
\frac{\partial}{\partial w_{2}}\frac{\partial}{\partial w_{1}}\hat{F}_{1}\left(\hat{\theta}_{1}\left(P_{1}\left(z_{1}\right)\tilde{P}_{2}\left(z_{2}\right)
\hat{P}_{2}\left(w_{2}\right)\right),w_{2}\right)|_{\mathbf{z,w}=1}&=&0
\end{eqnarray*}
%___________________________________________________________________________________________


%\newpage
%___________________________________________________________________________________________
%
%\section{Parciales mixtas de $\hat{F}_{1}$ para $w_{2}$}
%___________________________________________________________________________________________
%___________________________________________________________________________________________
\item \begin{eqnarray*}
\frac{\partial}{\partial z_{1}}\frac{\partial}{\partial w_{2}}\hat{F}_{1}\left(\hat{\theta}_{1}\left(P_{1}\left(z_{1}\right)\tilde{P}_{2}\left(z_{2}\right)
\hat{P}_{2}\left(w_{2}\right)\right),w_{2}\right)|_{\mathbf{z,w}=1}&=&\mu_{1}\hat{\mu}_{2}\frac{1}{1-\hat{\mu}_{1}}\frac{\partial}{\partial w_{1}}\hat{F}_{1}\left(1,1\right)+\mu_{1}\hat{\mu}_{2}\hat{\theta
}_{1}^{(2)}\frac{\partial}{\partial w_{1}}\hat{F}_{1}\left(1,1\right)\\
&+&\mu_{1}\frac{1}{1-\hat{\mu}_{1}}\hat{f}_{1}\left(1,2\right)+\mu_{1}\hat{\mu}_{2}\left(\frac{1}{1-\hat{\mu}_{1}}\right)^{2}\hat{f}_1\left(1,1\right)
\end{eqnarray*}
%___________________________________________________________________________________________

%___________________________________________________________________________________________
\begin{eqnarray*}
&&\frac{\partial}{\partial z_{2}}\frac{\partial}{\partial w_{2}}\hat{F}_{1}\left(\hat{\theta}_{1}\left(P_{1}\left(z_{1}\right)\tilde{P}_{2}\left(z_{2}\right)
\hat{P}_{2}\left(w_{2}\right)\right),w_{2}\right)|_{\mathbf{z,w}=1}\\
&=&P_1\left(z_1\right) \hat{P}_2'\left(w_2\right)
\hat{\theta }_1'\left(P_1\left(z_1\right)
\hat{P}_2\left(w_2\right) \tilde{P}_2\left(z_2\right)\right)
\tilde{P}_2'\left(z_2\right)\hat{F}_1^{(1,0)}\left(\hat{\theta }_1\left(P_1\left(z_1\right)
\hat{P}_2\left(w_2\right)
\tilde{P}_2\left(z_2\right)\right),w_2\right)\\
&+&P_1\left(z_1\right)^2
\hat{P}_2\left(w_2\right)\tilde{P}_2\left(z_2\right) \hat{P}_2'\left(w_2\right)
\tilde{P}_2'\left(z_2\right) \hat{\theta
}_1''\left(P_1\left(z_1\right) \hat{P}_2\left(w_2\right)
\tilde{P}_2\left(z_2\right)\right)\hat{F}_1^{(1,0)}\left(\hat{\theta }_1\left(P_1\left(z_1\right) \hat{P}_2\left(w_2\right) \tilde{P}_2\left(z_2\right)\right),w_2\right)\\
&+&P_1\left(z_1\right) \hat{P}_2\left(w_2\right) \hat{\theta
}_1'\left(P_1\left(z_1\right) \hat{P}_2\left(w_2\right)
\tilde{P}_2\left(z_2\right)\right)
\tilde{P}_2'\left(z_2\right)\hat{F}_1^{(1,1)}\left(\hat{\theta }_1\left(P_1\left(z_1\right) \hat{P}_2\left(w_2\right) \tilde{P}_2\left(z_2\right)\right),w_2\right)\\
&+&P_1\left(z_1\right)^2 \hat{P}_2\left(w_2\right)
\tilde{P}_2\left(z_2\right) \hat{P}_2'\left(w_2\right) \hat{\theta
}_1'\left(P_1\left(z_1\right)
\hat{P}_2\left(w_2\right) \tilde{P}_2\left(z_2\right)\right)^2\tilde{P}_2'\left(z_2\right) \hat{F}_1^{(2,0)}\left(\hat{\theta
}_1\left(P_1\left(z_1\right) \hat{P}_2\left(w_2\right)
\tilde{P}_2\left(z_2\right)\right),w_2\right)
\end{eqnarray*}
%___________________________________________________________________________________________

%___________________________________________________________________________________________
\begin{eqnarray*}
\frac{\partial}{\partial w_{1}}\frac{\partial}{\partial w_{2}}\hat{F}_{1}\left(\hat{\theta}_{1}\left(P_{1}\left(z_{1}\right)\tilde{P}_{2}\left(z_{2}\right)
\hat{P}_{2}\left(w_{2}\right)\right),w_{2}\right)|_{\mathbf{z,w}=1}&=&0
\end{eqnarray*}
%___________________________________________________________________________________________

%___________________________________________________________________________________________
\begin{eqnarray*}
&&\frac{\partial}{\partial w_{2}}\frac{\partial}{\partial w_{2}}\hat{F}_{1}\left(\hat{\theta}_{1}\left(P_{1}\left(z_{1}\right)\tilde{P}_{2}\left(z_{2}\right)
\hat{P}_{2}\left(w_{2}\right)\right),w_{2}\right)|_{\mathbf{z,w}=1}\\
&=&\hat{F}_1^{(0,2)}\left(\hat{\theta }_1\left(P_1\left(z_1\right) \hat{P}_2\left(w_2\right) \tilde{P}_2\left(z_2\right)\right),w_2\right)\\
&+&P_1\left(z_1\right) \tilde{P}_2\left(z_2\right) \hat{\theta
}_1'\left(P_1\left(z_1\right) \hat{P}_2\left(w_2\right)
\tilde{P}_2\left(z_2\right)\right)\hat{P}_2''\left(w_2\right) \hat{F}_1^{(1,0)}\left(\hat{\theta }_1\left(P_1\left(z_1\right) \hat{P}_2\left(w_2\right) \tilde{P}_2\left(z_2\right)\right),w_2\right)\\
&+&P_1\left(z_1\right)^2 \tilde{P}_2\left(z_2\right)^2
\hat{P}_2'\left(w_2\right)^2 \hat{\theta
}_1''\left(P_1\left(z_1\right) \hat{P}_2\left(w_2\right)
\tilde{P}_2\left(z_2\right)\right)\hat{F}_1^{(1,0)}\left(\hat{\theta }_1\left(P_1\left(z_1\right) \hat{P}_2\left(w_2\right) \tilde{P}_2\left(z_2\right)\right),w_2\right)\\
&+&P_1\left(z_1\right) \tilde{P}_2\left(z_2\right)
\hat{P}_2'\left(w_2\right) \hat{\theta
}_1'\left(P_1\left(z_1\right) \hat{P}_2\left(w_2\right)
\tilde{P}_2\left(z_2\right)\right)\\
&+&P_1\left(z_1\right) \tilde{P}_2\left(z_2\right)
\hat{P}_2'\left(w_2\right) \hat{\theta
}_1'\left(P_1\left(z_1\right) \hat{P}_2\left(w_2\right)
\tilde{P}_2\left(z_2\right)\right)\hat{F}_1^{(1,1)}\left(\hat{\theta }_1\left(P_1\left(z_1\right) \hat{P}_2\left(w_2\right) \tilde{P}_2\left(z_2\right)\right),w_2\right)\\
&+&P_1\left(z_1\right) \tilde{P}_2\left(z_2\right)
\hat{P}_2'\left(w_2\right) \hat{\theta
}_1'\left(P_1\left(z_1\right) \hat{P}_2\left(w_2\right)
\tilde{P}_2\left(z_2\right)\right)
P_1\left(z_1\right) \tilde{P}_2\left(z_2\right)
\hat{P}_2'\left(w_2\right) \hat{\theta
}_1'\left(P_1\left(z_1\right) \hat{P}_2\left(w_2\right)
\tilde{P}_2\left(z_2\right)\right)
\\
&&\left.\hat{F}_1^{(2,0)}\left(\hat{\theta
}_1\left(P_1\left(z_1\right) \hat{P}_2\left(w_2\right)
\tilde{P}_2\left(z_2\right)\right),w_2\right)\right)
\end{eqnarray*}
%___________________________________________________________________________________________


%___________________________________________________________________________________________
%
%\section{Parciales mixtas de $\hat{F}_{2}$ para $z_{1}$}
%___________________________________________________________________________________________
%___________________________________________________________________________________________
\begin{eqnarray*}
&&\frac{\partial}{\partial z_{1}}\frac{\partial}{\partial z_{1}}\hat{F}_{2}\left(w_{1},\hat{\theta}_{2}\left(P_{1}\left(z_{1}\right)\tilde{P}_{2}\left(z_{2}\right)
\hat{P}_{1}\left(w_{1}\right)\right)\right)|_{\mathbf{z,w}=1}\\
&=&P_1\left(w_1\right) \tilde{P}_2\left(z_2\right)
\hat{\theta }_2'\left(P_1\left(w_1\right) P_1\left(z_1\right)
\tilde{P}_2\left(z_2\right)\right)P_1''\left(z_1\right) \hat{F}_2^{(0,1)}\left(w_1,\hat{\theta }_2\left(P_1\left(w_1\right) P_1\left(z_1\right) \tilde{P}_2\left(z_2\right)\right)\right)\\
&+&P_1\left(w_1\right)^2 \tilde{P}_2\left(z_2\right)^2
P_1'\left(z_1\right)^2 \hat{\theta }_2''\left(P_1\left(w_1\right)
P_1\left(z_1\right) \tilde{P}_2\left(z_2\right)\right)\hat{F}_2^{(0,1)}\left(w_1,\hat{\theta }_2\left(P_1\left(w_1\right) P_1\left(z_1\right) \tilde{P}_2\left(z_2\right)\right)\right)\\
&+&P_1\left(w_1\right)^2 \tilde{P}_2\left(z_2\right)^2
P_1'\left(z_1\right)^2 \hat{\theta }_2'\left(P_1\left(w_1\right)
P_1\left(z_1\right) \tilde{P}_2\left(z_2\right)\right)^2\hat{F}_2^{(0,2)}\left(w_1,\hat{\theta
}_2\left(P_1\left(w_1\right) P_1\left(z_1\right)
\tilde{P}_2\left(z_2\right)\right)\right)
\end{eqnarray*}
%___________________________________________________________________________________________


%___________________________________________________________________________________________
\begin{eqnarray*}
&&\frac{\partial}{\partial z_{2}}\frac{\partial}{\partial z_{1}}\hat{F}_{2}\left(w_{1},\hat{\theta}_{2}\left(P_{1}\left(z_{1}\right)\tilde{P}_{2}\left(z_{2}\right)
\hat{P}_{1}\left(w_{1}\right)\right)\right)|_{\mathbf{z,w}=1}\\
&=&P_1\left(w_1\right) P_1'\left(z_1\right) \hat{\theta
}_2'\left(P_1\left(w_1\right) P_1\left(z_1\right)
\tilde{P}_2\left(z_2\right)\right)
\tilde{P}_2'\left(z_2\right)\hat{F}_2^{(0,1)}\left(w_1,\hat{\theta
}_2\left(P_1\left(w_1\right) P_1\left(z_1\right)
\tilde{P}_2\left(z_2\right)\right)\right)\\
&+&P_1\left(w_1\right)^2 P_1\left(z_1\right)\tilde{P}_2\left(z_2\right) P_1'\left(z_1\right)\tilde{P}_2'\left(z_2\right) \hat{\theta
}_2''\left(P_1\left(w_1\right) P_1\left(z_1\right)
\tilde{P}_2\left(z_2\right)\right)\hat{F}_2^{(0,1)}\left(w_1,\hat{\theta }_2\left(P_1\left(w_1\right) P_1\left(z_1\right) \tilde{P}_2\left(z_2\right)\right)\right)\\
&+&P_1\left(w_1\right)^2 P_1\left(z_1\right)
\tilde{P}_2\left(z_2\right) P_1'\left(z_1\right) \hat{\theta
}_2'\left(P_1\left(w_1\right) P_1\left(z_1\right)
\tilde{P}_2\left(z_2\right)\right)^2 \tilde{P}_2'\left(z_2\right)
\hat{F}_2^{(0,2)}\left(w_1,\hat{\theta
}_2\left(P_1\left(w_1\right) P_1\left(z_1\right)
\tilde{P}_2\left(z_2\right)\right)\right)
\end{eqnarray*}
%___________________________________________________________________________________________

%___________________________________________________________________________________________
\begin{eqnarray*}
&&\frac{\partial}{\partial w_{1}}\frac{\partial}{\partial z_{1}}\hat{F}_{2}\left(w_{1},\hat{\theta}_{2}\left(P_{1}\left(z_{1}\right)\tilde{P}_{2}\left(z_{2}\right)
\hat{P}_{1}\left(w_{1}\right)\right)\right)|_{\mathbf{z,w}=1}\\
&=&\tilde{P}_2\left(z_2\right) P_1'\left(w_1\right)
P_1'\left(z_1\right) \hat{\theta }_2'\left(P_1\left(w_1\right)
P_1\left(z_1\right) \tilde{P}_2\left(z_2\right)\right)\hat{F}_2^{(0,1)}\left(w_1,\hat{\theta
}_2\left(P_1\left(w_1\right) P_1\left(z_1\right)
\tilde{P}_2\left(z_2\right)\right)\right)\\
&+&P_1\left(w_1\right)P_1\left(z_1\right)\tilde{P}_2\left(z_2\right)^2 P_1'\left(w_1\right)P_1'\left(z_1\right) \hat{\theta }_2''\left(P_1\left(w_1\right)P_1\left(z_1\right) \tilde{P}_2\left(z_2\right)\right)\hat{F}_2^{(0,1)}\left(w_1,\hat{\theta }_2\left(P_1\left(w_1\right) P_1\left(z_1\right) \tilde{P}_2\left(z_2\right)\right)\right)\\
&+&P_1\left(w_1\right) \tilde{P}_2\left(z_2\right)
P_1'\left(z_1\right) \hat{\theta }_2'\left(P_1\left(w_1\right)
P_1\left(z_1\right) \tilde{P}_2\left(z_2\right)\right)P_1\left(z_1\right) \tilde{P}_2\left(z_2\right)
P_1'\left(w_1\right) \hat{\theta }_2'\left(P_1\left(w_1\right)
P_1\left(z_1\right) \tilde{P}_2\left(z_2\right)\right)\\
&&\hat{F}_2^{(0,2)}\left(w_1,\hat{\theta }_2\left(P_1\left(w_1\right) P_1\left(z_1\right) \tilde{P}_2\left(z_2\right)\right)\right)\\
&+&P_1\left(w_1\right) \tilde{P}_2\left(z_2\right)
P_1'\left(z_1\right) \hat{\theta }_2'\left(P_1\left(w_1\right)
P_1\left(z_1\right) \tilde{P}_2\left(z_2\right)\right)\hat{F}_2^{(1,1)}\left(w_1,\hat{\theta
}_2\left(P_1\left(w_1\right) P_1\left(z_1\right)
\tilde{P}_2\left(z_2\right)\right)\right)
\end{eqnarray*}
%___________________________________________________________________________________________


%___________________________________________________________________________________________
\begin{eqnarray*}
\frac{\partial}{\partial w_{2}}\frac{\partial}{\partial z_{1}}\hat{F}_{2}\left(w_{1},\hat{\theta}_{2}\left(P_{1}\left(z_{1}\right)\tilde{P}_{2}\left(z_{2}\right)
\hat{P}_{1}\left(w_{1}\right)\right)\right)|_{\mathbf{z,w}=1}&=&0
\end{eqnarray*}
%___________________________________________________________________________________________

%___________________________________________________________________________________________
%
%\section{Parciales mixtas de $\hat{F}_{2}$ para $z_{2}$}
%___________________________________________________________________________________________
%___________________________________________________________________________________________
\begin{eqnarray*}
&&\frac{\partial}{\partial z_{1}}\frac{\partial}{\partial z_{2}}\hat{F}_{2}\left(w_{1},\hat{\theta}_{2}\left(P_{1}\left(z_{1}\right)\tilde{P}_{2}\left(z_{2}\right)
\hat{P}_{1}\left(w_{1}\right)\right)\right)|_{\mathbf{z,w}=1}\\
&=&P_1\left(w_1\right) P_1'\left(z_1\right) \hat{\theta
}_2'\left(P_1\left(w_1\right) P_1\left(z_1\right)
\tilde{P}_2\left(z_2\right)\right)
\tilde{P}_2'\left(z_2\right)\hat{F}_2^{(0,1)}\left(w_1,\hat{\theta
}_2\left(P_1\left(w_1\right) P_1\left(z_1\right)
\tilde{P}_2\left(z_2\right)\right)\right)\\
&+&P_1\left(w_1\right)^2
P_1\left(z_1\right)\tilde{P}_2\left(z_2\right) P_1'\left(z_1\right)
\tilde{P}_2'\left(z_2\right) \hat{\theta
}_2''\left(P_1\left(w_1\right) P_1\left(z_1\right)
\tilde{P}_2\left(z_2\right)\right)\hat{F}_2^{(0,1)}\left(w_1,\hat{\theta }_2\left(P_1\left(w_1\right) P_1\left(z_1\right) \tilde{P}_2\left(z_2\right)\right)\right)\\
&+&P_1\left(w_1\right)^2 P_1\left(z_1\right)
\tilde{P}_2\left(z_2\right) P_1'\left(z_1\right) \hat{\theta
}_2'\left(P_1\left(w_1\right) P_1\left(z_1\right)
\tilde{P}_2\left(z_2\right)\right)^2\tilde{P}_2'\left(z_2\right)
\hat{F}_2^{(0,2)}\left(w_1,\hat{\theta
}_2\left(P_1\left(w_1\right) P_1\left(z_1\right)
\tilde{P}_2\left(z_2\right)\right)\right)
\end{eqnarray*}
%___________________________________________________________________________________________

%___________________________________________________________________________________________
\begin{eqnarray*}
&&\frac{\partial}{\partial z_{2}}\frac{\partial}{\partial z_{2}}\hat{F}_{2}\left(w_{1},\hat{\theta}_{2}\left(P_{1}\left(z_{1}\right)\tilde{P}_{2}\left(z_{2}\right)
\hat{P}_{1}\left(w_{1}\right)\right)\right)|_{\mathbf{z,w}=1}\\
&=&P_1\left(w_1\right)^2 P_1\left(z_1\right)^2
\tilde{P}_2'\left(z_2\right)^2 \hat{\theta
}_2''\left(P_1\left(w_1\right) P_1\left(z_1\right)
\tilde{P}_2\left(z_2\right)\right)\hat{F}_2^{(0,1)}\left(w_1,\hat{\theta }_2\left(P_1\left(w_1\right) P_1\left(z_1\right) \tilde{P}_2\left(z_2\right)\right)\right)\\
&+&P_1\left(w_1\right) P_1\left(z_1\right) \hat{\theta
}_2'\left(P_1\left(w_1\right) P_1\left(z_1\right)
\tilde{P}_2\left(z_2\right)\right) \tilde{P}_2''\left(z_2\right)\hat{F}_2^{(0,1)}\left(w_1,\hat{\theta }_2\left(P_1\left(w_1\right) P_1\left(z_1\right) \tilde{P}_2\left(z_2\right)\right)\right)\\
&+&P_1\left(w_1\right)^2 P_1\left(z_1\right)^2 \hat{\theta }_2'\left(P_1\left(w_1\right) P_1\left(z_1\right) \tilde{P}_2\left(z_2\right)\right)^2\tilde{P}_2'\left(z_2\right)^2
\hat{F}_2^{(0,2)}\left(w_1,\hat{\theta
}_2\left(P_1\left(w_1\right) P_1\left(z_1\right)
\tilde{P}_2\left(z_2\right)\right)\right)
\end{eqnarray*}
%___________________________________________________________________________________________

%___________________________________________________________________________________________
\begin{eqnarray*}
&&\frac{\partial}{\partial w_{1}}\frac{\partial}{\partial z_{2}}\hat{F}_{2}\left(w_{1},\hat{\theta}_{2}\left(P_{1}\left(z_{1}\right)\tilde{P}_{2}\left(z_{2}\right)
\hat{P}_{1}\left(w_{1}\right)\right)\right)|_{\mathbf{z,w}=1}\\
&=&P_1\left(z_1\right) P_1'\left(w_1\right) \hat{\theta
}_2'\left(P_1\left(w_1\right) P_1\left(z_1\right)
\tilde{P}_2\left(z_2\right)\right)
\tilde{P}_2'\left(z_2\right)\hat{F}_2^{(0,1)}\left(w_1,\hat{\theta
}_2\left(P_1\left(w_1\right) P_1\left(z_1\right)
\tilde{P}_2\left(z_2\right)\right)\right)\\
&+&P_1\left(w_1\right)P_1\left(z_1\right)^2\tilde{P}_2\left(z_2\right) P_1'\left(w_1\right)\tilde{P}_2'\left(z_2\right) \hat{\theta
}_2''\left(P_1\left(w_1\right) P_1\left(z_1\right)
\tilde{P}_2\left(z_2\right)\right)\hat{F}_2^{(0,1)}\left(w_1,\hat{\theta }_2\left(P_1\left(w_1\right) P_1\left(z_1\right) \tilde{P}_2\left(z_2\right)\right)\right)\\
&+&P_1\left(w_1\right) P_1\left(z_1\right) \hat{\theta
}_2'\left(P_1\left(w_1\right) P_1\left(z_1\right)
\tilde{P}_2\left(z_2\right)\right) \tilde{P}_2'\left(z_2\right)P_1\left(z_1\right) \tilde{P}_2\left(z_2\right)
P_1'\left(w_1\right) \hat{\theta }_2'\left(P_1\left(w_1\right)
P_1\left(z_1\right) \tilde{P}_2\left(z_2\right)\right)\\
&&\hat{F}_2^{(0,2)}\left(w_1,\hat{\theta }_2\left(P_1\left(w_1\right) P_1\left(z_1\right) \tilde{P}_2\left(z_2\right)\right)\right)\\
&+&P_1\left(w_1\right) P_1\left(z_1\right) \hat{\theta
}_2'\left(P_1\left(w_1\right) P_1\left(z_1\right)
\tilde{P}_2\left(z_2\right)\right) \tilde{P}_2'\left(z_2\right)
\hat{F}_2^{(1,1)}\left(w_1,\hat{\theta
}_2\left(P_1\left(w_1\right) P_1\left(z_1\right)
\tilde{P}_2\left(z_2\right)\right)\right)
\end{eqnarray*}
%___________________________________________________________________________________________

%___________________________________________________________________________________________
\begin{eqnarray*}
\frac{\partial}{\partial w_{2}}\frac{\partial}{\partial z_{2}}\hat{F}_{2}\left(w_{1},\hat{\theta}_{2}\left(P_{1}\left(z_{1}\right)\tilde{P}_{2}\left(z_{2}\right)
\hat{P}_{1}\left(w_{1}\right)\right)\right)|_{\mathbf{z,w}=1}&=&0
\end{eqnarray*}
%___________________________________________________________________________________________


%___________________________________________________________________________________________
%
%\section{Parciales mixtas de $\hat{F}_{2}$ para $w_{1}$}
%___________________________________________________________________________________________
%___________________________________________________________________________________________
\begin{eqnarray*}
&&\frac{\partial}{\partial z_{1}}\frac{\partial}{\partial w_{1}}\hat{F}_{2}\left(w_{1},\hat{\theta}_{2}\left(P_{1}\left(z_{1}\right)\tilde{P}_{2}\left(z_{2}\right)
\hat{P}_{1}\left(w_{1}\right)\right)\right)|_{\mathbf{z,w}=1}\\
&=&\tilde{P}_2\left(z_2\right) P_1'\left(w_1\right)
P_1'\left(z_1\right) \hat{\theta }_2'\left(P_1\left(w_1\right)
P_1\left(z_1\right) \tilde{P}_2\left(z_2\right)\right)\hat{F}_2^{(0,1)}\left(w_1,\hat{\theta
}_2\left(P_1\left(w_1\right) P_1\left(z_1\right)
\tilde{P}_2\left(z_2\right)\right)\right)\\
&+&P_1\left(w_1\right)P_1\left(z_1\right)
\tilde{P}_2\left(z_2\right)^2 P_1'\left(w_1\right)
P_1'\left(z_1\right) \hat{\theta }_2''\left(P_1\left(w_1\right)
P_1\left(z_1\right) \tilde{P}_2\left(z_2\right)\right)\hat{F}_2^{(0,1)}\left(w_1,\hat{\theta
}_2\left(P_1\left(w_1\right) P_1\left(z_1\right)
\tilde{P}_2\left(z_2\right)\right)\right)\\
&+&P_1\left(w_1\right)P_1\left(z_1\right)
\tilde{P}_2\left(z_2\right)^2 P_1'\left(w_1\right)
P_1'\left(z_1\right) \hat{\theta }_2'\left(P_1\left(w_1\right)
P_1\left(z_1\right) \tilde{P}_2\left(z_2\right)\right)^2\hat{F}_2^{(0,2)}\left(w_1,\hat{\theta }_2\left(P_1\left(w_1\right) P_1\left(z_1\right) \tilde{P}_2\left(z_2\right)\right)\right)\\
&+&P_1\left(w_1\right) \tilde{P}_2\left(z_2\right)
P_1'\left(z_1\right) \hat{\theta }_2'\left(P_1\left(w_1\right)
P_1\left(z_1\right) \tilde{P}_2\left(z_2\right)\right)\hat{F}_2^{(1,1)}\left(w_1,\hat{\theta
}_2\left(P_1\left(w_1\right) P_1\left(z_1\right)
\tilde{P}_2\left(z_2\right)\right)\right)
\end{eqnarray*}
%___________________________________________________________________________________________

%___________________________________________________________________________________________
\begin{eqnarray*}
&&\frac{\partial}{\partial z_{2}}\frac{\partial}{\partial w_{1}}\hat{F}_{2}\left(w_{1},\hat{\theta}_{2}\left(P_{1}\left(z_{1}\right)\tilde{P}_{2}\left(z_{2}\right)
\hat{P}_{1}\left(w_{1}\right)\right)\right)|_{\mathbf{z,w}=1}\\
&=&P_1\left(z_1\right) P_1'\left(w_1\right) \hat{\theta
}_2'\left(P_1\left(w_1\right) P_1\left(z_1\right)
\tilde{P}_2\left(z_2\right)\right)
\tilde{P}_2'\left(z_2\right)\hat{F}_2^{(0,1)}\left(w_1,\hat{\theta
}_2\left(P_1\left(w_1\right) P_1\left(z_1\right)
\tilde{P}_2\left(z_2\right)\right)\right)\\
&+&P_1\left(w_1\right)P_1\left(z_1\right)^2
\tilde{P}_2\left(z_2\right) P_1'\left(w_1\right)
\tilde{P}_2'\left(z_2\right) \hat{\theta
}_2''\left(P_1\left(w_1\right) P_1\left(z_1\right)
\tilde{P}_2\left(z_2\right)\right)\hat{F}_2^{(0,1)}\left(w_1,\hat{\theta }_2\left(P_1\left(w_1\right) P_1\left(z_1\right) \tilde{P}_2\left(z_2\right)\right)\right)\\
&+&P_1\left(w_1\right) P_1\left(z_1\right)^2
\tilde{P}_2\left(z_2\right) P_1'\left(w_1\right) \hat{\theta
}_2'\left(P_1\left(w_1\right) P_1\left(z_1\right)
\tilde{P}_2\left(z_2\right)\right)^2 \tilde{P}_2'\left(z_2\right) \hat{F}_2^{(0,2)}\left(w_1,\hat{\theta }_2\left(P_1\left(w_1\right) P_1\left(z_1\right) \tilde{P}_2\left(z_2\right)\right)\right)\\
&+&P_1\left(w_1\right) P_1\left(z_1\right) \hat{\theta
}_2'\left(P_1\left(w_1\right) P_1\left(z_1\right)
\tilde{P}_2\left(z_2\right)\right) \tilde{P}_2'\left(z_2\right)\hat{F}_2^{(1,1)}\left(w_1,\hat{\theta
}_2\left(P_1\left(w_1\right) P_1\left(z_1\right)
\tilde{P}_2\left(z_2\right)\right)\right)
\end{eqnarray*}
%___________________________________________________________________________________________

\begin{eqnarray*}
&&\frac{\partial}{\partial w_{1}}\frac{\partial}{\partial w_{1}}\hat{F}_{2}\left(w_{1},\hat{\theta}_{2}\left(P_{1}\left(z_{1}\right)\tilde{P}_{2}\left(z_{2}\right)
\hat{P}_{1}\left(w_{1}\right)\right)\right)|_{\mathbf{z,w}=1}\\
&=&P_1\left(z_1\right) \tilde{P}_2\left(z_2\right)
\hat{\theta }_2'\left(P_1\left(w_1\right) P_1\left(z_1\right)
\tilde{P}_2\left(z_2\right)\right)P_1''\left(w_1\right) \hat{F}_2^{(0,1)}\left(w_1,\hat{\theta }_2\left(P_1\left(w_1\right) P_1\left(z_1\right) \tilde{P}_2\left(z_2\right)\right)\right)\\
&+&P_1\left(z_1\right)^2 \tilde{P}_2\left(z_2\right)^2
P_1'\left(w_1\right)^2 \hat{\theta }_2''\left(P_1\left(w_1\right)
P_1\left(z_1\right) \tilde{P}_2\left(z_2\right)\right)\hat{F}_2^{(0,1)}\left(w_1,\hat{\theta }_2\left(P_1\left(w_1\right) P_1\left(z_1\right) \tilde{P}_2\left(z_2\right)\right)\right)\\
&+&P_1\left(z_1\right) \tilde{P}_2\left(z_2\right)
P_1'\left(w_1\right) \hat{\theta }_2'\left(P_1\left(w_1\right)
P_1\left(z_1\right) \tilde{P}_2\left(z_2\right)\right)\hat{F}_2^{(1,1)}\left(w_1,\hat{\theta }_2\left(P_1\left(w_1\right) P_1\left(z_1\right) \tilde{P}_2\left(z_2\right)\right)\right)\\
&+&P_1\left(z_1\right) \tilde{P}_2\left(z_2\right)
P_1'\left(w_1\right) \hat{\theta }_2'\left(P_1\left(w_1\right)
P_1\left(z_1\right) \tilde{P}_2\left(z_2\right)\right)P_1\left(z_1\right) \tilde{P}_2\left(z_2\right)
P_1'\left(w_1\right) \hat{\theta }_2'\left(P_1\left(w_1\right)
P_1\left(z_1\right) \tilde{P}_2\left(z_2\right)\right)\\
&&\hat{F}_2^{(0,2)}\left(w_1,\hat{\theta }_2\left(P_1\left(w_1\right) P_1\left(z_1\right) \tilde{P}_2\left(z_2\right)\right)\right)\\
&+&P_1\left(z_1\right) \tilde{P}_2\left(z_2\right)
P_1'\left(w_1\right) \hat{\theta }_2'\left(P_1\left(w_1\right)
P_1\left(z_1\right) \tilde{P}_2\left(z_2\right)\right)\hat{F}_2^{(1,1)}\left(w_1,\hat{\theta }_2\left(P_1\left(w_1\right) P_1\left(z_1\right) \tilde{P}_2\left(z_2\right)\right)\right)\\
&+&\hat{F}_2^{(2,0)}\left(w_1,\hat{\theta
}_2\left(P_1\left(w_1\right) P_1\left(z_1\right)
\tilde{P}_2\left(z_2\right)\right)\right)
\end{eqnarray*}



\begin{eqnarray*}
\frac{\partial}{\partial w_{2}}\frac{\partial}{\partial w_{1}}\hat{F}_{2}\left(w_{1},\hat{\theta}_{2}\left(P_{1}\left(z_{1}\right)\tilde{P}_{2}\left(z_{2}\right)
\hat{P}_{1}\left(w_{1}\right)\right)\right)|_{\mathbf{z,w}=1}&=&0
\end{eqnarray*}

%___________________________________________________________________________________________
%
%\section{Parciales mixtas de $\hat{F}_{2}$ para $w_{2}$}
%___________________________________________________________________________________________
\begin{eqnarray*}
\frac{\partial}{\partial z_{1}}\frac{\partial}{\partial w_{2}}\hat{F}_{2}\left(w_{1},\hat{\theta}_{2}\left(P_{1}\left(z_{1}\right)\tilde{P}_{2}\left(z_{2}\right)
\hat{P}_{1}\left(w_{1}\right)\right)\right)|_{\mathbf{z,w}=1}&=&0
\end{eqnarray*}

%___________________________________________________________________________________________
\begin{eqnarray*}
\frac{\partial}{\partial z_{2}}\frac{\partial}{\partial w_{2}}\hat{F}_{2}\left(w_{1},\hat{\theta}_{2}\left(P_{1}\left(z_{1}\right)\tilde{P}_{2}\left(z_{2}\right)
\hat{P}_{1}\left(w_{1}\right)\right)\right)|_{\mathbf{z,w}=1}&=&0
\end{eqnarray*}

%___________________________________________________________________________________________

%___________________________________________________________________________________________
\begin{eqnarray*}
\frac{\partial}{\partial w_{1}}\frac{\partial}{\partial w_{2}}\hat{F}_{2}\left(w_{1},\hat{\theta}_{2}\left(P_{1}\left(z_{1}\right)\tilde{P}_{2}\left(z_{2}\right)
\hat{P}_{1}\left(w_{1}\right)\right)\right)|_{\mathbf{z,w}=1}&=&0
\end{eqnarray*}

%___________________________________________________________________________________________

%___________________________________________________________________________________________
\begin{eqnarray*}
\frac{\partial}{\partial w_{2}}\frac{\partial}{\partial w_{2}}\hat{F}_{2}\left(w_{1},\hat{\theta}_{2}\left(P_{1}\left(z_{1}\right)\tilde{P}_{2}\left(z_{2}\right)
\hat{P}_{1}\left(w_{1}\right)\right)\right)|_{\mathbf{z,w}=1}&=&0
\end{eqnarray*}
\end{enumerate}




%___________________________________________________________________________________________
%
\subsection{Derivadas de Segundo Orden para $F_{1}$}
%___________________________________________________________________________________________

\subsubsection{Mixtas para $z_{1}$:}
%___________________________________________________________________________________________
\begin{enumerate}

%1/1/1
\item \begin{eqnarray*}
&&\frac{\partial}{\partial z_1}\frac{\partial}{\partial z_1}\left(R_2\left(P_1\left(z_1\right)\bar{P}_2\left(z_2\right)\hat{P}_1\left(w_1\right)\hat{P}_2\left(w_2\right)\right)F_2\left(z_1,\theta
_2\left(P_1\left(z_1\right)\hat{P}_1\left(w_1\right)\hat{P}_2\left(w_2\right)\right)\right)\hat{F}_2\left(w_1,w_2\right)\right)\\
&=&r_{2}P_{1}^{(2)}\left(1\right)+\mu_{1}^{2}R_{2}^{(2)}\left(1\right)+2\mu_{1}r_{2}\left(\frac{\mu_{1}}{1-\tilde{\mu}_{2}}F_{2}^{(0,1)}+F_{2}^{1,0)}\right)+\frac{1}{1-\tilde{\mu}_{2}}P_{1}^{(2)}F_{2}^{(0,1)}+\mu_{1}^{2}\tilde{\theta}_{2}^{(2)}\left(1\right)F_{2}^{(0,1)}\\
&+&\frac{\mu_{1}}{1-\tilde{\mu}_{2}}F_{2}^{(1,1)}+\frac{\mu_{1}}{1-\tilde{\mu}_{2}}\left(\frac{\mu_{1}}{1-\tilde{\mu}_{2}}F_{2}^{(0,2)}+F_{2}^{(1,1)}\right)+F_{2}^{(2,0)}.
\end{eqnarray*}

%2/2/1

\item \begin{eqnarray*}
&&\frac{\partial}{\partial z_2}\frac{\partial}{\partial z_1}\left(R_2\left(P_1\left(z_1\right)\bar{P}_2\left(z_2\right)\hat{P}_1\left(w_1\right)\hat{P}_2\left(w_2\right)\right)F_2\left(z_1,\theta
_2\left(P_1\left(z_1\right)\hat{P}_1\left(w_1\right)\hat{P}_2\left(w_2\right)\right)\right)\hat{F}_2\left(w_1,w_2\right)\right)\\
&=&\mu_{1}r_{2}\tilde{\mu}_{2}+\mu_{1}\tilde{\mu}_{2}R_{2}^{(2)}\left(1\right)+r_{2}\tilde{\mu}_{2}\left(\frac{\mu_{1}}{1-\tilde{\mu}_{2}}F_{2}^{(0,1)}+F_{2}^{(1,0)}\right).
\end{eqnarray*}
%3/3/1
\item \begin{eqnarray*}
&&\frac{\partial}{\partial w_1}\frac{\partial}{\partial z_1}\left(R_2\left(P_1\left(z_1\right)\bar{P}_2\left(z_2\right)\hat{P}_1\left(w_1\right)\hat{P}_2\left(w_2\right)\right)F_2\left(z_1,\theta
_2\left(P_1\left(z_1\right)\hat{P}_1\left(w_1\right)\hat{P}_2\left(w_2\right)\right)\right)\hat{F}_2\left(w_1,w_2\right)\right)\\
&=&\mu_{1}\hat{\mu}_{1}r_{2}+\mu_{1}\hat{\mu}_{1}R_{2}^{(2)}\left(1\right)+r_{2}\frac{\mu_{1}}{1-\tilde{\mu}_{2}}F_{2}^{(0,1)}+r_{2}\hat{\mu}_{1}\left(\frac{\mu_{1}}{1-\tilde{\mu}_{2}}F_{2}^{(0,1)}+F_{2}^{(1,0)}\right)+\mu_{1}r_{2}\hat{F}_{2}^{(1,0)}\\
&+&\left(\frac{\mu_{1}}{1-\tilde{\mu}_{2}}F_{2}^{(0,1)}+F_{2}^{(1,0)}\right)\hat{F}_{2}^{(1,0)}+\frac{\mu_{1}\hat{\mu}_{1}}{1-\tilde{\mu}_{2}}F_{2}^{(0,1)}+\mu_{1}\hat{\mu}_{1}\tilde{\theta}_{2}^{(2)}\left(1\right)F_{2}^{(0,1)}\\
&+&\mu_{1}\hat{\mu}_{1}\left(\frac{1}{1-\tilde{\mu}_{2}}\right)^{2}F_{2}^{(0,2)}+\frac{\hat{\mu}_{1}}{1-\tilde{\mu}_{2}}F_{2}^{(1,1)}.
\end{eqnarray*}
%4/4/1
\item \begin{eqnarray*}
&&\frac{\partial}{\partial w_2}\frac{\partial}{\partial z_1}\left(R_2\left(P_1\left(z_1\right)\bar{P}_2\left(z_2\right)\hat{P}_1\left(w_1\right)\hat{P}_2\left(w_2\right)\right)
F_2\left(z_1,\theta_2\left(P_1\left(z_1\right)\hat{P}_1\left(w_1\right)\hat{P}_2\left(w_2\right)\right)\right)\hat{F}_2\left(w_1,w_2\right)\right)\\
&=&\mu_{1}\hat{\mu}_{2}r_{2}+\mu_{1}\hat{\mu}_{2}R_{2}^{(2)}\left(1\right)+r_{2}\frac{\mu_{1}\hat{\mu}_{2}}{1-\tilde{\mu}_{2}}F_{2}^{(0,1)}+\mu_{1}r_{2}\hat{F}_{2}^{(0,1)}
+r_{2}\hat{\mu}_{2}\left(\frac{\mu_{1}}{1-\tilde{\mu}_{2}}F_{2}^{(0,1)}+F_{2}^{(1,0)}\right)\\
&+&\hat{F}_{2}^{(1,0)}\left(\frac{\mu_{1}}{1-\tilde{\mu}_{2}}F_{2}^{(0,1)}+F_{2}^{(1,0)}\right)+\frac{\mu_{1}\hat{\mu}_{2}}{1-\tilde{\mu}_{2}}F_{2}^{(0,1)}
+\mu_{1}\hat{\mu}_{2}\tilde{\theta}_{2}^{(2)}\left(1\right)F_{2}^{(0,1)}+\mu_{1}\hat{\mu}_{2}\left(\frac{1}{1-\tilde{\mu}_{2}}\right)^{2}F_{2}^{(0,2)}\\
&+&\frac{\hat{\mu}_{2}}{1-\tilde{\mu}_{2}}F_{2}^{(1,1)}.
\end{eqnarray*}
%___________________________________________________________________________________________
\subsubsection{Mixtas para $z_{2}$:}
%___________________________________________________________________________________________
%5
\item \begin{eqnarray*} &&\frac{\partial}{\partial
z_1}\frac{\partial}{\partial
z_2}\left(R_2\left(P_1\left(z_1\right)\bar{P}_2\left(z_2\right)\hat{P}_1\left(w_1\right)\hat{P}_2\left(w_2\right)\right)
F_2\left(z_1,\theta_2\left(P_1\left(z_1\right)\hat{P}_1\left(w_1\right)\hat{P}_2\left(w_2\right)\right)\right)\hat{F}_2\left(w_1,w_2\right)\right)\\
&=&\mu_{1}\tilde{\mu}_{2}r_{2}+\mu_{1}\tilde{\mu}_{2}R_{2}^{(2)}\left(1\right)+r_{2}\tilde{\mu}_{2}\left(\frac{\mu_{1}}{1-\tilde{\mu}_{2}}F_{2}^{(0,1)}+F_{2}^{(1,0)}\right).
\end{eqnarray*}

%6

\item \begin{eqnarray*} &&\frac{\partial}{\partial
z_2}\frac{\partial}{\partial
z_2}\left(R_2\left(P_1\left(z_1\right)\bar{P}_2\left(z_2\right)\hat{P}_1\left(w_1\right)\hat{P}_2\left(w_2\right)\right)
F_2\left(z_1,\theta_2\left(P_1\left(z_1\right)\hat{P}_1\left(w_1\right)\hat{P}_2\left(w_2\right)\right)\right)\hat{F}_2\left(w_1,w_2\right)\right)\\
&=&\tilde{\mu}_{2}^{2}R_{2}^{(2)}(1)+r_{2}\tilde{P}_{2}^{(2)}\left(1\right).
\end{eqnarray*}

%7
\item \begin{eqnarray*} &&\frac{\partial}{\partial
w_1}\frac{\partial}{\partial
z_2}\left(R_2\left(P_1\left(z_1\right)\bar{P}_2\left(z_2\right)\hat{P}_1\left(w_1\right)\hat{P}_2\left(w_2\right)\right)
F_2\left(z_1,\theta_2\left(P_1\left(z_1\right)\hat{P}_1\left(w_1\right)\hat{P}_2\left(w_2\right)\right)\right)\hat{F}_2\left(w_1,w_2\right)\right)\\
&=&\hat{\mu}_{1}\tilde{\mu}_{2}r_{2}+\hat{\mu}_{1}\tilde{\mu}_{2}R_{2}^{(2)}(1)+
r_{2}\frac{\hat{\mu}_{1}\tilde{\mu}_{2}}{1-\tilde{\mu}_{2}}F_{2}^{(0,1)}+r_{2}\tilde{\mu}_{2}\hat{F}_{2}^{(1,0)}.
\end{eqnarray*}
%8
\item \begin{eqnarray*} &&\frac{\partial}{\partial
w_2}\frac{\partial}{\partial
z_2}\left(R_2\left(P_1\left(z_1\right)\bar{P}_2\left(z_2\right)\hat{P}_1\left(w_1\right)\hat{P}_2\left(w_2\right)\right)
F_2\left(z_1,\theta_2\left(P_1\left(z_1\right)\hat{P}_1\left(w_1\right)\hat{P}_2\left(w_2\right)\right)\right)\hat{F}_2\left(w_1,w_2\right)\right)\\
&=&\hat{\mu}_{2}\tilde{\mu}_{2}r_{2}+\hat{\mu}_{2}\tilde{\mu}_{2}R_{2}^{(2)}(1)+
r_{2}\frac{\hat{\mu}_{2}\tilde{\mu}_{2}}{1-\tilde{\mu}_{2}}F_{2}^{(0,1)}+r_{2}\tilde{\mu}_{2}\hat{F}_{2}^{(0,1)}.
\end{eqnarray*}
%___________________________________________________________________________________________
\subsubsection{Mixtas para $w_{1}$:}
%___________________________________________________________________________________________

%9
\item \begin{eqnarray*} &&\frac{\partial}{\partial
z_1}\frac{\partial}{\partial
w_1}\left(R_2\left(P_1\left(z_1\right)\bar{P}_2\left(z_2\right)\hat{P}_1\left(w_1\right)\hat{P}_2\left(w_2\right)\right)
F_2\left(z_1,\theta_2\left(P_1\left(z_1\right)\hat{P}_1\left(w_1\right)\hat{P}_2\left(w_2\right)\right)\right)\hat{F}_2\left(w_1,w_2\right)\right)\\
&=&\mu_{1}\hat{\mu}_{1}r_{2}+\mu_{1}\hat{\mu}_{1}R_{2}^{(2)}\left(1\right)+\frac{\mu_{1}\hat{\mu}_{1}}{1-\tilde{\mu}_{2}}F_{2}^{(0,1)}+r_{2}\frac{\mu_{1}\hat{\mu}_{1}}{1-\tilde{\mu}_{2}}F_{2}^{(0,1)}+\mu_{1}\hat{\mu}_{1}\tilde{\theta}_{2}^{(2)}\left(1\right)F_{2}^{(0,1)}\\
&+&r_{2}\hat{\mu}_{1}\left(\frac{\mu_{1}}{1-\tilde{\mu}_{2}}F_{2}^{(0,1)}+F_{2}^{(1,0)}\right)+r_{2}\mu_{1}\hat{F}_{2}^{(1,0)}
+\left(\frac{\mu_{1}}{1-\tilde{\mu}_{2}}F_{2}^{(0,1)}+F_{2}^{(1,0)}\right)\hat{F}_{2}^{(1,0)}\\
&+&\frac{\hat{\mu}_{1}}{1-\tilde{\mu}_{2}}\left(\frac{\mu_{1}}{1-\tilde{\mu}_{2}}F_{2}^{(0,2)}+F_{2}^{(1,1)}\right).
\end{eqnarray*}
%10
\item \begin{eqnarray*} &&\frac{\partial}{\partial
z_2}\frac{\partial}{\partial
w_1}\left(R_2\left(P_1\left(z_1\right)\bar{P}_2\left(z_2\right)\hat{P}_1\left(w_1\right)\hat{P}_2\left(w_2\right)\right)
F_2\left(z_1,\theta_2\left(P_1\left(z_1\right)\hat{P}_1\left(w_1\right)\hat{P}_2\left(w_2\right)\right)\right)\hat{F}_2\left(w_1,w_2\right)\right)\\
&=&\tilde{\mu}_{2}\hat{\mu}_{1}r_{2}+\tilde{\mu}_{2}\hat{\mu}_{1}R_{2}^{(2)}\left(1\right)+r_{2}\frac{\tilde{\mu}_{2}\hat{\mu}_{1}}{1-\tilde{\mu}_{2}}F_{2}^{(0,1)}
+r_{2}\tilde{\mu}_{2}\hat{F}_{2}^{(1,0)}.
\end{eqnarray*}
%11
\item \begin{eqnarray*} &&\frac{\partial}{\partial
w_1}\frac{\partial}{\partial
w_1}\left(R_2\left(P_1\left(z_1\right)\bar{P}_2\left(z_2\right)\hat{P}_1\left(w_1\right)\hat{P}_2\left(w_2\right)\right)
F_2\left(z_1,\theta_2\left(P_1\left(z_1\right)\hat{P}_1\left(w_1\right)\hat{P}_2\left(w_2\right)\right)\right)\hat{F}_2\left(w_1,w_2\right)\right)\\
&=&\hat{\mu}_{1}^{2}R_{2}^{(2)}\left(1\right)+r_{2}\hat{P}_{1}^{(2)}\left(1\right)+2r_{2}\frac{\hat{\mu}_{1}^{2}}{1-\tilde{\mu}_{2}}F_{2}^{(0,1)}+
\hat{\mu}_{1}^{2}\tilde{\theta}_{2}^{(2)}\left(1\right)F_{2}^{(0,1)}+\frac{1}{1-\tilde{\mu}_{2}}\hat{P}_{1}^{(2)}\left(1\right)F_{2}^{(0,1)}\\
&+&\frac{\hat{\mu}_{1}^{2}}{1-\tilde{\mu}_{2}}F_{2}^{(0,2)}+2r_{2}\hat{\mu}_{1}\hat{F}_{2}^{(1,0)}+2\frac{\hat{\mu}_{1}}{1-\tilde{\mu}_{2}}F_{2}^{(0,1)}\hat{F}_{2}^{(1,0)}+\hat{F}_{2}^{(2,0)}.
\end{eqnarray*}
%12
\item \begin{eqnarray*} &&\frac{\partial}{\partial
w_2}\frac{\partial}{\partial
w_1}\left(R_2\left(P_1\left(z_1\right)\bar{P}_2\left(z_2\right)\hat{P}_1\left(w_1\right)\hat{P}_2\left(w_2\right)\right)
F_2\left(z_1,\theta_2\left(P_1\left(z_1\right)\hat{P}_1\left(w_1\right)\hat{P}_2\left(w_2\right)\right)\right)\hat{F}_2\left(w_1,w_2\right)\right)\\
&=&r_{2}\hat{\mu}_{2}\hat{\mu}_{1}+\hat{\mu}_{1}\hat{\mu}_{2}R_{2}^{(2)}(1)+\frac{\hat{\mu}_{1}\hat{\mu}_{2}}{1-\tilde{\mu}_{2}}F_{2}^{(0,1)}
+2r_{2}\frac{\hat{\mu}_{1}\hat{\mu}_{2}}{1-\tilde{\mu}_{2}}F_{2}^{(0,1)}+\hat{\mu}_{2}\hat{\mu}_{1}\tilde{\theta}_{2}^{(2)}\left(1\right)F_{2}^{(0,1)}+
r_{2}\hat{\mu}_{1}\hat{F}_{2}^{(0,1)}\\
&+&\frac{\hat{\mu}_{1}}{1-\tilde{\mu}_{2}}F_{2}^{(0,1)}\hat{F}_{2}^{(0,1)}+\hat{\mu}_{1}\hat{\mu}_{2}\left(\frac{1}{1-\tilde{\mu}_{2}}\right)^{2}F_{2}^{(0,2)}+
r_{2}\hat{\mu}_{2}\hat{F}_{2}^{(1,0)}+\frac{\hat{\mu}_{2}}{1-\tilde{\mu}_{2}}F_{2}^{(0,1)}\hat{F}_{2}^{(1,0)}+\hat{F}_{2}^{(1,1)}.
\end{eqnarray*}
%___________________________________________________________________________________________
\subsubsection{Mixtas para $w_{2}$:}
%___________________________________________________________________________________________
%13

\item \begin{eqnarray*} &&\frac{\partial}{\partial
z_1}\frac{\partial}{\partial
w_2}\left(R_2\left(P_1\left(z_1\right)\bar{P}_2\left(z_2\right)\hat{P}_1\left(w_1\right)\hat{P}_2\left(w_2\right)\right)
F_2\left(z_1,\theta_2\left(P_1\left(z_1\right)\hat{P}_1\left(w_1\right)\hat{P}_2\left(w_2\right)\right)\right)\hat{F}_2\left(w_1,w_2\right)\right)\\
&=&r_{2}\mu_{1}\hat{\mu}_{2}+\mu_{1}\hat{\mu}_{2}R_{2}^{(2)}(1)+\frac{\mu_{1}\hat{\mu}_{2}}{1-\tilde{\mu}_{2}}F_{2}^{(0,1)}+r_{2}\frac{\mu_{1}\hat{\mu}_{2}}{1-\tilde{\mu}_{2}}F_{2}^{(0,1)}+\mu_{1}\hat{\mu}_{2}\tilde{\theta}_{2}^{(2)}\left(1\right)F_{2}^{(0,1)}+r_{2}\mu_{1}\hat{F}_{2}^{(0,1)}\\
&+&r_{2}\hat{\mu}_{2}\left(\frac{\mu_{1}}{1-\tilde{\mu}_{2}}F_{2}^{(0,1)}+F_{2}^{(1,0)}\right)+\hat{F}_{2}^{(0,1)}\left(\frac{\mu_{1}}{1-\tilde{\mu}_{2}}F_{2}^{(0,1)}+F_{2}^{(1,0)}\right)+\frac{\hat{\mu}_{2}}{1-\tilde{\mu}_{2}}\left(\frac{\mu_{1}}{1-\tilde{\mu}_{2}}F_{2}^{(0,2)}+F_{2}^{(1,1)}\right).
\end{eqnarray*}
%14
\item \begin{eqnarray*} &&\frac{\partial}{\partial
z_2}\frac{\partial}{\partial
w_2}\left(R_2\left(P_1\left(z_1\right)\bar{P}_2\left(z_2\right)\hat{P}_1\left(w_1\right)\hat{P}_2\left(w_2\right)\right)
F_2\left(z_1,\theta_2\left(P_1\left(z_1\right)\hat{P}_1\left(w_1\right)\hat{P}_2\left(w_2\right)\right)\right)\hat{F}_2\left(w_1,w_2\right)\right)\\
&=&r_{2}\tilde{\mu}_{2}\hat{\mu}_{2}+\tilde{\mu}_{2}\hat{\mu}_{2}R_{2}^{(2)}(1)+r_{2}\frac{\tilde{\mu}_{2}\hat{\mu}_{2}}{1-\tilde{\mu}_{2}}F_{2}^{(0,1)}+r_{2}\tilde{\mu}_{2}\hat{F}_{2}^{(0,1)}.
\end{eqnarray*}
%15
\item \begin{eqnarray*} &&\frac{\partial}{\partial
w_1}\frac{\partial}{\partial
w_2}\left(R_2\left(P_1\left(z_1\right)\bar{P}_2\left(z_2\right)\hat{P}_1\left(w_1\right)\hat{P}_2\left(w_2\right)\right)
F_2\left(z_1,\theta_2\left(P_1\left(z_1\right)\hat{P}_1\left(w_1\right)\hat{P}_2\left(w_2\right)\right)\right)\hat{F}_2\left(w_1,w_2\right)\right)\\
&=&r_{2}\hat{\mu}_{1}\hat{\mu}_{2}+\hat{\mu}_{1}\hat{\mu}_{2}R_{2}^{(2)}\left(1\right)+\frac{\hat{\mu}_{1}\hat{\mu}_{2}}{1-\tilde{\mu}_{2}}F_{2}^{(0,1)}+2r_{2}\frac{\hat{\mu}_{1}\hat{\mu}_{2}}{1-\tilde{\mu}_{2}}F_{2}^{(0,1)}+\hat{\mu}_{1}\hat{\mu}_{2}\theta_{2}^{(2)}\left(1\right)F_{2}^{(0,1)}+r_{2}\hat{\mu}_{1}\hat{F}_{2}^{(0,1)}\\
&+&\frac{\hat{\mu}_{1}}{1-\tilde{\mu}_{2}}F_{2}^{(0,1)}\hat{F}_{2}^{(0,1)}+\hat{\mu}_{1}\hat{\mu}_{2}\left(\frac{1}{1-\tilde{\mu}_{2}}\right)^{2}F_{2}^{(0,2)}+r_{2}\hat{\mu}_{2}\hat{F}_{2}^{(0,1)}+\frac{\hat{\mu}_{2}}{1-\tilde{\mu}_{2}}F_{2}^{(0,1)}\hat{F}_{2}^{(1,0)}+\hat{F}_{2}^{(1,1)}.
\end{eqnarray*}
%16

\item \begin{eqnarray*} &&\frac{\partial}{\partial
w_2}\frac{\partial}{\partial
w_2}\left(R_2\left(P_1\left(z_1\right)\bar{P}_2\left(z_2\right)\hat{P}_1\left(w_1\right)\hat{P}_2\left(w_2\right)\right)
F_2\left(z_1,\theta_2\left(P_1\left(z_1\right)\hat{P}_1\left(w_1\right)\hat{P}_2\left(w_2\right)\right)\right)\hat{F}_2\left(w_1,w_2\right)\right)\\
&=&\hat{\mu}_{2}^{2}R_{2}^{(2)}(1)+r_{2}\hat{P}_{2}^{(2)}\left(1\right)+2r_{2}\frac{\hat{\mu}_{2}^{2}}{1-\tilde{\mu}_{2}}F_{2}^{(0,1)}+\hat{\mu}_{2}^{2}\tilde{\theta}_{2}^{(2)}\left(1\right)F_{2}^{(0,1)}+\frac{1}{1-\tilde{\mu}_{2}}\hat{P}_{2}^{(2)}\left(1\right)F_{2}^{(0,1)}\\
&+&2r_{2}\hat{\mu}_{2}\hat{F}_{2}^{(0,1)}+2\frac{\hat{\mu}_{2}}{1-\tilde{\mu}_{2}}F_{2}^{(0,1)}\hat{F}_{2}^{(0,1)}+\left(\frac{\hat{\mu}_{2}}{1-\tilde{\mu}_{2}}\right)^{2}F_{2}^{(0,2)}+\hat{F}_{2}^{(0,2)}.
\end{eqnarray*}
\end{enumerate}
%___________________________________________________________________________________________
%
\subsection{Derivadas de Segundo Orden para $F_{2}$}
%___________________________________________________________________________________________


\begin{enumerate}

%___________________________________________________________________________________________
\subsubsection{Mixtas para $z_{1}$:}
%___________________________________________________________________________________________

%1/17
\item \begin{eqnarray*} &&\frac{\partial}{\partial
z_1}\frac{\partial}{\partial
z_1}\left(R_1\left(P_1\left(z_1\right)\bar{P}_2\left(z_2\right)\hat{P}_1\left(w_1\right)\hat{P}_2\left(w_2\right)\right)
F_1\left(\theta_1\left(\tilde{P}_2\left(z_1\right)\hat{P}_1\left(w_1\right)\hat{P}_2\left(w_2\right)\right)\right)\hat{F}_1\left(w_1,w_2\right)\right)\\
&=&r_{1}P_{1}^{(2)}\left(1\right)+\mu_{1}^{2}R_{1}^{(2)}\left(1\right).
\end{eqnarray*}

%2/18
\item \begin{eqnarray*} &&\frac{\partial}{\partial
z_2}\frac{\partial}{\partial
z_1}\left(R_1\left(P_1\left(z_1\right)\bar{P}_2\left(z_2\right)\hat{P}_1\left(w_1\right)\hat{P}_2\left(w_2\right)\right)F_1\left(\theta_1\left(\tilde{P}_2\left(z_1\right)\hat{P}_1\left(w_1\right)\hat{P}_2\left(w_2\right)\right)\right)\hat{F}_1\left(w_1,w_2\right)\right)\\
&=&\mu_{1}\tilde{\mu}_{2}r_{1}+\mu_{1}\tilde{\mu}_{2}R_{1}^{(2)}(1)+
r_{1}\mu_{1}\left(\frac{\tilde{\mu}_{2}}{1-\mu_{1}}F_{1}^{(1,0)}+F_{1}^{(0,1)}\right).
\end{eqnarray*}

%3/19
\item \begin{eqnarray*} &&\frac{\partial}{\partial
w_1}\frac{\partial}{\partial
z_1}\left(R_1\left(P_1\left(z_1\right)\bar{P}_2\left(z_2\right)\hat{P}_1\left(w_1\right)\hat{P}_2\left(w_2\right)\right)F_1\left(\theta_1\left(\tilde{P}_2\left(z_1\right)\hat{P}_1\left(w_1\right)\hat{P}_2\left(w_2\right)\right)\right)\hat{F}_1\left(w_1,w_2\right)\right)\\
&=&r_{1}\mu_{1}\hat{\mu}_{1}+\mu_{1}\hat{\mu}_{1}R_{1}^{(2)}\left(1\right)+r_{1}\frac{\mu_{1}\hat{\mu}_{1}}{1-\mu_{1}}F_{1}^{(1,0)}+r_{1}\mu_{1}\hat{F}_{1}^{(1,0)}.
\end{eqnarray*}
%4/20
\item \begin{eqnarray*} &&\frac{\partial}{\partial
w_2}\frac{\partial}{\partial
z_1}\left(R_1\left(P_1\left(z_1\right)\bar{P}_2\left(z_2\right)\hat{P}_1\left(w_1\right)\hat{P}_2\left(w_2\right)\right)F_1\left(\theta_1\left(\tilde{P}_2\left(z_1\right)\hat{P}_1\left(w_1\right)\hat{P}_2\left(w_2\right)\right)\right)\hat{F}_1\left(w_1,w_2\right)\right)\\
&=&\mu_{1}\hat{\mu}_{2}r_{1}+\mu_{1}\hat{\mu}_{2}R_{1}^{(2)}\left(1\right)+r_{1}\mu_{1}\hat{F}_{1}^{(0,1)}+r_{1}\frac{\mu_{1}\hat{\mu}_{2}}{1-\mu_{1}}F_{1}^{(1,0)}.
\end{eqnarray*}
%___________________________________________________________________________________________
\subsubsection{Mixtas para $z_{2}$:}
%___________________________________________________________________________________________
%5/21
\item \begin{eqnarray*}
&&\frac{\partial}{\partial z_1}\frac{\partial}{\partial z_2}\left(R_1\left(P_1\left(z_1\right)\bar{P}_2\left(z_2\right)\hat{P}_1\left(w_1\right)\hat{P}_2\left(w_2\right)\right)F_1\left(\theta_1\left(\tilde{P}_2\left(z_1\right)\hat{P}_1\left(w_1\right)\hat{P}_2\left(w_2\right)\right)\right)\hat{F}_1\left(w_1,w_2\right)\right)\\
&=&r_{1}\mu_{1}\tilde{\mu}_{2}+\mu_{1}\tilde{\mu}_{2}R_{1}^{(2)}\left(1\right)+r_{1}\mu_{1}\left(\frac{\tilde{\mu}_{2}}{1-\mu_{1}}F_{1}^{(1,0)}+F_{1}^{(0,1)}\right).
\end{eqnarray*}

%6/22
\item \begin{eqnarray*}
&&\frac{\partial}{\partial z_2}\frac{\partial}{\partial z_2}\left(R_1\left(P_1\left(z_1\right)\bar{P}_2\left(z_2\right)\hat{P}_1\left(w_1\right)\hat{P}_2\left(w_2\right)\right)F_1\left(\theta_1\left(\tilde{P}_2\left(z_1\right)\hat{P}_1\left(w_1\right)\hat{P}_2\left(w_2\right)\right)\right)\hat{F}_1\left(w_1,w_2\right)\right)\\
&=&\tilde{\mu}_{2}^{2}R_{1}^{(2)}\left(1\right)+r_{1}\tilde{P}_{2}^{(2)}\left(1\right)+2r_{1}\tilde{\mu}_{2}\left(\frac{\tilde{\mu}_{2}}{1-\mu_{1}}F_{1}^{(1,0)}+F_{1}^{(0,1)}\right)+F_{1}^{(0,2)}+\tilde{\mu}_{2}^{2}\theta_{1}^{(2)}\left(1\right)F_{1}^{(1,0)}\\
&+&\frac{1}{1-\mu_{1}}\tilde{P}_{2}^{(2)}\left(1\right)F_{1}^{(1,0)}+\frac{\tilde{\mu}_{2}}{1-\mu_{1}}F_{1}^{(1,1)}+\frac{\tilde{\mu}_{2}}{1-\mu_{1}}\left(\frac{\tilde{\mu}_{2}}{1-\mu_{1}}F_{1}^{(2,0)}+F_{1}^{(1,1)}\right).
\end{eqnarray*}
%7/23
\item \begin{eqnarray*}
&&\frac{\partial}{\partial w_1}\frac{\partial}{\partial z_2}\left(R_1\left(P_1\left(z_1\right)\bar{P}_2\left(z_2\right)\hat{P}_1\left(w_1\right)\hat{P}_2\left(w_2\right)\right)F_1\left(\theta_1\left(\tilde{P}_2\left(z_1\right)\hat{P}_1\left(w_1\right)\hat{P}_2\left(w_2\right)\right)\right)\hat{F}_1\left(w_1,w_2\right)\right)\\
&=&\tilde{\mu}_{2}\hat{\mu}_{1}r_{1}+\tilde{\mu}_{2}\hat{\mu}_{1}R_{1}^{(2)}\left(1\right)+r_{1}\frac{\tilde{\mu}_{2}\hat{\mu}_{1}}{1-\mu_{1}}F_{1}^{(1,0)}+\hat{\mu}_{1}r_{1}\left(\frac{\tilde{\mu}_{2}}{1-\mu_{1}}F_{1}^{(1,0)}+F_{1}^{(0,1)}\right)+r_{1}\tilde{\mu}_{2}\hat{F}_{1}^{(1,0)}\\
&+&\left(\frac{\tilde{\mu}_{2}}{1-\mu_{1}}F_{1}^{(1,0)}+F_{1}^{(0,1)}\right)\hat{F}_{1}^{(1,0)}+\frac{\tilde{\mu}_{2}\hat{\mu}_{1}}{1-\mu_{1}}F_{1}^{(1,0)}+\tilde{\mu}_{2}\hat{\mu}_{1}\theta_{1}^{(2)}\left(1\right)F_{1}^{(1,0)}+\frac{\hat{\mu}_{1}}{1-\mu_{1}}F_{1}^{(1,1)}\\
&+&\left(\frac{1}{1-\mu_{1}}\right)^{2}\tilde{\mu}_{2}\hat{\mu}_{1}F_{1}^{(2,0)}.
\end{eqnarray*}
%8/24
\item \begin{eqnarray*}
&&\frac{\partial}{\partial w_2}\frac{\partial}{\partial z_2}\left(R_1\left(P_1\left(z_1\right)\bar{P}_2\left(z_2\right)\hat{P}_1\left(w_1\right)\hat{P}_2\left(w_2\right)\right)F_1\left(\theta_1\left(\tilde{P}_2\left(z_1\right)\hat{P}_1\left(w_1\right)\hat{P}_2\left(w_2\right)\right)\right)\hat{F}_1\left(w_1,w_2\right)\right)\\
&=&\hat{\mu}_{2}\tilde{\mu}_{2}r_{1}+\hat{\mu}_{2}\tilde{\mu}_{2}R_{1}^{(2)}(1)+r_{1}\tilde{\mu}_{2}\hat{F}_{1}^{(0,1)}+r_{1}\frac{\hat{\mu}_{2}\tilde{\mu}_{2}}{1-\mu_{1}}F_{1}^{(1,0)}+\hat{\mu}_{2}r_{1}\left(\frac{\tilde{\mu}_{2}}{1-\mu_{1}}F_{1}^{(1,0)}+F_{1}^{(0,1)}\right)\\
&+&\left(\frac{\tilde{\mu}_{2}}{1-\mu_{1}}F_{1}^{(1,0)}+F_{1}^{(0,1)}\right)\hat{F}_{1}^{(0,1)}+\frac{\tilde{\mu}_{2}\hat{\mu_{2}}}{1-\mu_{1}}F_{1}^{(1,0)}+\hat{\mu}_{2}\tilde{\mu}_{2}\theta_{1}^{(2)}\left(1\right)F_{1}^{(1,0)}+\frac{\hat{\mu}_{2}}{1-\mu_{1}}F_{1}^{(1,1)}\\
&+&\left(\frac{1}{1-\mu_{1}}\right)^{2}\tilde{\mu}_{2}\hat{\mu}_{2}F_{1}^{(2,0)}.
\end{eqnarray*}
%___________________________________________________________________________________________
\subsubsection{Mixtas para $w_{1}$:}
%___________________________________________________________________________________________
%9/25
\item \begin{eqnarray*} &&\frac{\partial}{\partial
z_1}\frac{\partial}{\partial
w_1}\left(R_1\left(P_1\left(z_1\right)\bar{P}_2\left(z_2\right)\hat{P}_1\left(w_1\right)\hat{P}_2\left(w_2\right)\right)F_1\left(\theta_1\left(\tilde{P}_2\left(z_1\right)\hat{P}_1\left(w_1\right)\hat{P}_2\left(w_2\right)\right)\right)\hat{F}_1\left(w_1,w_2\right)\right)\\
&=&r_{1}\mu_{1}\hat{\mu}_{1}+\mu_{1}\hat{\mu}_{1}R_{1}^{(2)}(1)+r_{1}\frac{\mu_{1}\hat{\mu}_{1}}{1-\mu_{1}}F_{1}^{(1,0)}+r_{1}\mu_{1}\hat{F}_{1}^{(1,0)}.
\end{eqnarray*}
%10/26
\item \begin{eqnarray*} &&\frac{\partial}{\partial
z_2}\frac{\partial}{\partial
w_1}\left(R_1\left(P_1\left(z_1\right)\bar{P}_2\left(z_2\right)\hat{P}_1\left(w_1\right)\hat{P}_2\left(w_2\right)\right)F_1\left(\theta_1\left(\tilde{P}_2\left(z_1\right)\hat{P}_1\left(w_1\right)\hat{P}_2\left(w_2\right)\right)\right)\hat{F}_1\left(w_1,w_2\right)\right)\\
&=&r_{1}\hat{\mu}_{1}\tilde{\mu}_{2}+\tilde{\mu}_{2}\hat{\mu}_{1}R_{1}^{(2)}\left(1\right)+
\frac{\hat{\mu}_{1}\tilde{\mu}_{2}}{1-\mu_{1}}F_{1}^{1,0)}+r_{1}\frac{\hat{\mu}_{1}\tilde{\mu}_{2}}{1-\mu_{1}}F_{1}^{(1,0)}+\hat{\mu}_{1}\tilde{\mu}_{2}\theta_{1}^{(2)}\left(1\right)F_{2}^{(1,0)}\\
&+&r_{1}\hat{\mu}_{1}\left(F_{1}^{(1,0)}+\frac{\tilde{\mu}_{2}}{1-\mu_{1}}F_{1}^{1,0)}\right)+
r_{1}\tilde{\mu}_{2}\hat{F}_{1}^{(1,0)}+\left(F_{1}^{(0,1)}+\frac{\tilde{\mu}_{2}}{1-\mu_{1}}F_{1}^{1,0)}\right)\hat{F}_{1}^{(1,0)}\\
&+&\frac{\hat{\mu}_{1}}{1-\mu_{1}}\left(F_{1}^{(1,1)}+\frac{\tilde{\mu}_{2}}{1-\mu_{1}}F_{1}^{2,0)}\right).
\end{eqnarray*}
%11/27
\item \begin{eqnarray*} &&\frac{\partial}{\partial
w_1}\frac{\partial}{\partial
w_1}\left(R_1\left(P_1\left(z_1\right)\bar{P}_2\left(z_2\right)\hat{P}_1\left(w_1\right)\hat{P}_2\left(w_2\right)\right)F_1\left(\theta_1\left(\tilde{P}_2\left(z_1\right)\hat{P}_1\left(w_1\right)\hat{P}_2\left(w_2\right)\right)\right)\hat{F}_1\left(w_1,w_2\right)\right)\\
&=&\hat{\mu}_{1}^{2}R_{1}^{(2)}\left(1\right)+r_{1}\hat{P}_{1}^{(2)}\left(1\right)+2r_{1}\frac{\hat{\mu}_{1}^{2}}{1-\mu_{1}}F_{1}^{(1,0)}+\hat{\mu}_{1}^{2}\theta_{1}^{(2)}\left(1\right)F_{1}^{(1,0)}+\frac{1}{1-\mu_{1}}\hat{P}_{1}^{(2)}\left(1\right)F_{1}^{(1,0)}\\
&+&2r_{1}\hat{\mu}_{1}\hat{F}_{1}^{(1,0)}+2\frac{\hat{\mu}_{1}}{1-\mu_{1}}F_{1}^{(1,0)}\hat{F}_{1}^{(1,0)}+\left(\frac{\hat{\mu}_{1}}{1-\mu_{1}}\right)^{2}F_{1}^{(2,0)}+\hat{F}_{1}^{(2,0)}.
\end{eqnarray*}
%12/28
\item \begin{eqnarray*} &&\frac{\partial}{\partial
w_2}\frac{\partial}{\partial
w_1}\left(R_1\left(P_1\left(z_1\right)\bar{P}_2\left(z_2\right)\hat{P}_1\left(w_1\right)\hat{P}_2\left(w_2\right)\right)F_1\left(\theta_1\left(\tilde{P}_2\left(z_1\right)\hat{P}_1\left(w_1\right)\hat{P}_2\left(w_2\right)\right)\right)\hat{F}_1\left(w_1,w_2\right)\right)\\
&=&r_{1}\hat{\mu}_{1}\hat{\mu}_{2}+\hat{\mu}_{1}\hat{\mu}_{2}R_{1}^{(2)}\left(1\right)+r_{1}\hat{\mu}_{1}\hat{F}_{1}^{(0,1)}+
\frac{\hat{\mu}_{1}\hat{\mu}_{2}}{1-\mu_{1}}F_{1}^{(1,0)}+2r_{1}\frac{\hat{\mu}_{1}\hat{\mu}_{2}}{1-\mu_{1}}F_{1}^{1,0)}+\hat{\mu}_{1}\hat{\mu}_{2}\theta_{1}^{(2)}\left(1\right)F_{1}^{(1,0)}\\
&+&\frac{\hat{\mu}_{1}}{1-\mu_{1}}F_{1}^{(1,0)}\hat{F}_{1}^{(0,1)}+
r_{1}\hat{\mu}_{2}\hat{F}_{1}^{(1,0)}+\frac{\hat{\mu}_{2}}{1-\mu_{1}}\hat{F}_{1}^{(1,0)}F_{1}^{(1,0)}+\hat{F}_{1}^{(1,1)}+\hat{\mu}_{1}\hat{\mu}_{2}\left(\frac{1}{1-\mu_{1}}\right)^{2}F_{1}^{(2,0)}.
\end{eqnarray*}
%___________________________________________________________________________________________
\subsubsection{Mixtas para $w_{2}$:}
%___________________________________________________________________________________________
%13/29
\item \begin{eqnarray*} &&\frac{\partial}{\partial
z_1}\frac{\partial}{\partial
w_2}\left(R_1\left(P_1\left(z_1\right)\bar{P}_2\left(z_2\right)\hat{P}_1\left(w_1\right)\hat{P}_2\left(w_2\right)\right)F_1\left(\theta_1\left(\tilde{P}_2\left(z_1\right)\hat{P}_1\left(w_1\right)\hat{P}_2\left(w_2\right)\right)\right)\hat{F}_1\left(w_1,w_2\right)\right)\\
&=&r_{1}\mu_{1}\hat{\mu}_{2}+\mu_{1}\hat{\mu}_{2}R_{1}^{(2)}\left(1\right)+r_{1}\mu_{1}\hat{F}_{1}^{(0,1)}+r_{1}\frac{\mu_{1}\hat{\mu}_{2}}{1-\mu_{1}}F_{1}^{(1,0)}.
\end{eqnarray*}
%14/30
\item \begin{eqnarray*} &&\frac{\partial}{\partial
z_2}\frac{\partial}{\partial
w_2}\left(R_1\left(P_1\left(z_1\right)\bar{P}_2\left(z_2\right)\hat{P}_1\left(w_1\right)\hat{P}_2\left(w_2\right)\right)F_1\left(\theta_1\left(\tilde{P}_2\left(z_1\right)\hat{P}_1\left(w_1\right)\hat{P}_2\left(w_2\right)\right)\right)\hat{F}_1\left(w_1,w_2\right)\right)\\
&=&r_{1}\hat{\mu}_{2}\tilde{\mu}_{2}+\hat{\mu}_{2}\tilde{\mu}_{2}R_{1}^{(2)}\left(1\right)+r_{1}\tilde{\mu}_{2}\hat{F}_{1}^{(0,1)}+\frac{\hat{\mu}_{2}\tilde{\mu}_{2}}{1-\mu_{1}}F_{1}^{(1,0)}+r_{1}\frac{\hat{\mu}_{2}\tilde{\mu}_{2}}{1-\mu_{1}}F_{1}^{(1,0)}\\
&+&\hat{\mu}_{2}\tilde{\mu}_{2}\theta_{1}^{(2)}\left(1\right)F_{1}^{(1,0)}+r_{1}\hat{\mu}_{2}\left(F_{1}^{(0,1)}+\frac{\tilde{\mu}_{2}}{1-\mu_{1}}F_{1}^{(1,0)}\right)+\left(F_{1}^{(0,1)}+\frac{\tilde{\mu}_{2}}{1-\mu_{1}}F_{1}^{(1,0)}\right)\hat{F}_{1}^{(0,1)}\\&+&\frac{\hat{\mu}_{2}}{1-\mu_{1}}\left(F_{1}^{(1,1)}+\frac{\tilde{\mu}_{2}}{1-\mu_{1}}F_{1}^{(2,0)}\right).
\end{eqnarray*}
%15/31
\item \begin{eqnarray*} &&\frac{\partial}{\partial
w_1}\frac{\partial}{\partial
w_2}\left(R_1\left(P_1\left(z_1\right)\bar{P}_2\left(z_2\right)\hat{P}_1\left(w_1\right)\hat{P}_2\left(w_2\right)\right)F_1\left(\theta_1\left(\tilde{P}_2\left(z_1\right)\hat{P}_1\left(w_1\right)\hat{P}_2\left(w_2\right)\right)\right)\hat{F}_1\left(w_1,w_2\right)\right)\\
&=&r_{1}\hat{\mu}_{1}\hat{\mu}_{2}+\hat{\mu}_{1}\hat{\mu}_{2}R_{1}^{(2)}\left(1\right)+r_{1}\hat{\mu}_{1}\hat{F}_{1}^{(0,1)}+
\frac{\hat{\mu}_{1}\hat{\mu}_{2}}{1-\mu_{1}}F_{1}^{(1,0)}+2r_{1}\frac{\hat{\mu}_{1}\hat{\mu}_{2}}{1-\mu_{1}}F_{1}^{(1,0)}+\hat{\mu}_{1}\hat{\mu}_{2}\theta_{1}^{(2)}\left(1\right)F_{1}^{(1,0)}\\
&+&\frac{\hat{\mu}_{1}}{1-\mu_{1}}\hat{F}_{1}^{(0,1)}F_{1}^{(1,0)}+r_{1}\hat{\mu}_{2}\hat{F}_{1}^{(1,0)}+\frac{\hat{\mu}_{2}}{1-\mu_{1}}\hat{F}_{1}^{(1,0)}F_{1}^{(1,0)}+\hat{F}_{1}^{(1,1)}+\hat{\mu}_{1}\hat{\mu}_{2}\left(\frac{1}{1-\mu_{1}}\right)^{2}F_{1}^{(2,0)}.
\end{eqnarray*}
%16/32
\item \begin{eqnarray*} &&\frac{\partial}{\partial
w_2}\frac{\partial}{\partial
w_2}\left(R_1\left(P_1\left(z_1\right)\bar{P}_2\left(z_2\right)\hat{P}_1\left(w_1\right)\hat{P}_2\left(w_2\right)\right)F_1\left(\theta_1\left(\tilde{P}_2\left(z_1\right)\hat{P}_1\left(w_1\right)\hat{P}_2\left(w_2\right)\right)\right)\hat{F}_1\left(w_1,w_2\right)\right)\\
&=&\hat{\mu}_{2}R_{1}^{(2)}\left(1\right)+r_{1}\hat{P}_{2}^{(2)}\left(1\right)+2r_{1}\hat{\mu}_{2}\hat{F}_{1}^{(0,1)}+\hat{F}_{1}^{(0,2)}+2r_{1}\frac{\hat{\mu}_{2}^{2}}{1-\mu_{1}}F_{1}^{(1,0)}+\hat{\mu}_{2}^{2}\theta_{1}^{(2)}\left(1\right)F_{1}^{(1,0)}\\
&+&\frac{1}{1-\mu_{1}}\hat{P}_{2}^{(2)}\left(1\right)F_{1}^{(1,0)} +
2\frac{\hat{\mu}_{2}}{1-\mu_{1}}F_{1}^{(1,0)}\hat{F}_{1}^{(0,1)}+\left(\frac{\hat{\mu}_{2}}{1-\mu_{1}}\right)^{2}F_{1}^{(2,0)}.
\end{eqnarray*}
\end{enumerate}

%___________________________________________________________________________________________
%
\subsection{Derivadas de Segundo Orden para $\hat{F}_{1}$}
%___________________________________________________________________________________________


\begin{enumerate}
%___________________________________________________________________________________________
\subsubsection{Mixtas para $z_{1}$:}
%___________________________________________________________________________________________
%1/33

\item \begin{eqnarray*} &&\frac{\partial}{\partial
z_1}\frac{\partial}{\partial
z_1}\left(\hat{R}_{2}\left(P_{1}\left(z_{1}\right)\tilde{P}_{2}\left(z_{2}\right)\hat{P}_{1}\left(w_{1}\right)\hat{P}_{2}\left(w_{2}\right)\right)\hat{F}_{2}\left(w_{1},\hat{\theta}_{2}\left(P_{1}\left(z_{1}\right)\tilde{P}_{2}\left(z_{2}\right)\hat{P}_{1}\left(w_{1}\right)\right)\right)F_{2}\left(z_{1},z_{2}\right)\right)\\
&=&\hat{r}_{2}P_{1}^{(2)}\left(1\right)+
\mu_{1}^{2}\hat{R}_{2}^{(2)}\left(1\right)+
2\hat{r}_{2}\frac{\mu_{1}^{2}}{1-\hat{\mu}_{2}}\hat{F}_{2}^{(0,1)}+
\frac{1}{1-\hat{\mu}_{2}}P_{1}^{(2)}\left(1\right)\hat{F}_{2}^{(0,1)}+
\mu_{1}^{2}\hat{\theta}_{2}^{(2)}\left(1\right)\hat{F}_{2}^{(0,1)}\\
&+&\left(\frac{\mu_{1}^{2}}{1-\hat{\mu}_{2}}\right)^{2}\hat{F}_{2}^{(0,2)}+
2\hat{r}_{2}\mu_{1}F_{2}^{(1,0)}+2\frac{\mu_{1}}{1-\hat{\mu}_{2}}\hat{F}_{2}^{(0,1)}F_{2}^{(1,0)}+F_{2}^{(2,0)}.
\end{eqnarray*}

%2/34
\item \begin{eqnarray*} &&\frac{\partial}{\partial
z_2}\frac{\partial}{\partial
z_1}\left(\hat{R}_{2}\left(P_{1}\left(z_{1}\right)\tilde{P}_{2}\left(z_{2}\right)\hat{P}_{1}\left(w_{1}\right)\hat{P}_{2}\left(w_{2}\right)\right)\hat{F}_{2}\left(w_{1},\hat{\theta}_{2}\left(P_{1}\left(z_{1}\right)\tilde{P}_{2}\left(z_{2}\right)\hat{P}_{1}\left(w_{1}\right)\right)\right)F_{2}\left(z_{1},z_{2}\right)\right)\\
&=&\hat{r}_{2}\mu_{1}\tilde{\mu}_{2}+\mu_{1}\tilde{\mu}_{2}\hat{R}_{2}^{(2)}\left(1\right)+\hat{r}_{2}\mu_{1}F_{2}^{(0,1)}+
\frac{\mu_{1}\tilde{\mu}_{2}}{1-\hat{\mu}_{2}}\hat{F}_{2}^{(0,1)}+2\hat{r}_{2}\frac{\mu_{1}\tilde{\mu}_{2}}{1-\hat{\mu}_{2}}\hat{F}_{2}^{(0,1)}+\mu_{1}\tilde{\mu}_{2}\hat{\theta}_{2}^{(2)}\left(1\right)\hat{F}_{2}^{(0,1)}\\
&+&\frac{\mu_{1}}{1-\hat{\mu}_{2}}F_{2}^{(0,1)}\hat{F}_{2}^{(0,1)}+\mu_{1} \tilde{\mu}_{2}\left(\frac{1}{1-\hat{\mu}_{2}}\right)^{2}\hat{F}_{2}^{(0,2)}+\hat{r}_{2}\tilde{\mu}_{2}F_{2}^{(1,0)}+\frac{\tilde{\mu}_{2}}{1-\hat{\mu}_{2}}\hat{F}_{2}^{(0,1)}F_{2}^{(1,0)}+F_{2}^{(1,1)}.
\end{eqnarray*}


%3/35

\item \begin{eqnarray*} &&\frac{\partial}{\partial
w_1}\frac{\partial}{\partial
z_1}\left(\hat{R}_{2}\left(P_{1}\left(z_{1}\right)\tilde{P}_{2}\left(z_{2}\right)\hat{P}_{1}\left(w_{1}\right)\hat{P}_{2}\left(w_{2}\right)\right)\hat{F}_{2}\left(w_{1},\hat{\theta}_{2}\left(P_{1}\left(z_{1}\right)\tilde{P}_{2}\left(z_{2}\right)\hat{P}_{1}\left(w_{1}\right)\right)\right)F_{2}\left(z_{1},z_{2}\right)\right)\\
&=&\hat{r}_{2}\mu_{1}\hat{\mu}_{1}+\mu_{1}\hat{\mu}_{1}\hat{R}_{2}^{(2)}\left(1\right)+\hat{r}_{2}\frac{\mu_{1}\hat{\mu}_{1}}{1-\hat{\mu}_{2}}\hat{F}_{2}^{(0,1)}+\hat{r}_{2}\hat{\mu}_{1}F_{2}^{(1,0)}+\hat{r}_{2}\mu_{1}\hat{F}_{2}^{(1,0)}+F_{2}^{(1,0)}\hat{F}_{2}^{(1,0)}+\frac{\mu_{1}}{1-\hat{\mu}_{2}}\hat{F}_{2}^{(1,1)}.
\end{eqnarray*}

%4/36

\item \begin{eqnarray*} &&\frac{\partial}{\partial
w_2}\frac{\partial}{\partial
z_1}\left(\hat{R}_{2}\left(P_{1}\left(z_{1}\right)\tilde{P}_{2}\left(z_{2}\right)\hat{P}_{1}\left(w_{1}\right)\hat{P}_{2}\left(w_{2}\right)\right)\hat{F}_{2}\left(w_{1},\hat{\theta}_{2}\left(P_{1}\left(z_{1}\right)\tilde{P}_{2}\left(z_{2}\right)\hat{P}_{1}\left(w_{1}\right)\right)\right)F_{2}\left(z_{1},z_{2}\right)\right)\\
&=&\hat{r}_{2}\mu_{1}\hat{\mu}_{2}+\mu_{1}\hat{\mu}_{2}\hat{R}_{2}^{(2)}\left(1\right)+\frac{\mu_{1}\hat{\mu}_{2}}{1-\hat{\mu}_{2}}\hat{F}_{2}^{(0,1)}+2\hat{r}_{2}\frac{\mu_{1}\hat{\mu}_{2}}{1-\hat{\mu}_{2}}\hat{F}_{2}^{(0,1)}+\mu_{1}\hat{\mu}_{2}\hat{\theta}_{2}^{(2)}\left(1\right)\hat{F}_{2}^{(0,1)}\\
&+&\mu_{1}\hat{\mu}_{2}\left(\frac{1}{1-\hat{\mu}_{2}}\right)^{2}\hat{F}_{2}^{(0,2)}+\hat{r}_{2}\hat{\mu}_{2}F_{2}^{(1,0)}+\frac{\hat{\mu}_{2}}{1-\hat{\mu}_{2}}\hat{F}_{2}^{(0,1)}F_{2}^{(1,0)}.
\end{eqnarray*}
%___________________________________________________________________________________________
\subsubsection{Mixtas para $z_{2}$:}
%___________________________________________________________________________________________

%5/37

\item \begin{eqnarray*} &&\frac{\partial}{\partial
z_1}\frac{\partial}{\partial
z_2}\left(\hat{R}_{2}\left(P_{1}\left(z_{1}\right)\tilde{P}_{2}\left(z_{2}\right)\hat{P}_{1}\left(w_{1}\right)\hat{P}_{2}\left(w_{2}\right)\right)\hat{F}_{2}\left(w_{1},\hat{\theta}_{2}\left(P_{1}\left(z_{1}\right)\tilde{P}_{2}\left(z_{2}\right)\hat{P}_{1}\left(w_{1}\right)\right)\right)F_{2}\left(z_{1},z_{2}\right)\right)\\
&=&\hat{r}_{2}\mu_{1}\tilde{\mu}_{2}+\mu_{1}\tilde{\mu}_{2}\hat{R}_{2}^{(2)}\left(1\right)+\mu_{1}\hat{r}_{2}F_{2}^{(0,1)}+
\frac{\mu_{1}\tilde{\mu}_{2}}{1-\hat{\mu}_{2}}\hat{F}_{2}^{(0,1)}+2\hat{r}_{2}\frac{\mu_{1}\tilde{\mu}_{2}}{1-\hat{\mu}_{2}}\hat{F}_{2}^{(0,1)}+\mu_{1}\tilde{\mu}_{2}\hat{\theta}_{2}^{(2)}\left(1\right)\hat{F}_{2}^{(0,1)}\\
&+&\frac{\mu_{1}}{1-\hat{\mu}_{2}}F_{2}^{(0,1)}\hat{F}_{2}^{(0,1)}+\mu_{1}\tilde{\mu}_{2}\left(\frac{1}{1-\hat{\mu}_{2}}\right)^{2}\hat{F}_{2}^{(0,2)}+\hat{r}_{2}\tilde{\mu}_{2}F_{2}^{(1,0)}+\frac{\tilde{\mu}_{2}}{1-\hat{\mu}_{2}}\hat{F}_{2}^{(0,1)}F_{2}^{(1,0)}+F_{2}^{(1,1)}.
\end{eqnarray*}

%6/38

\item \begin{eqnarray*} &&\frac{\partial}{\partial
z_2}\frac{\partial}{\partial
z_2}\left(\hat{R}_{2}\left(P_{1}\left(z_{1}\right)\tilde{P}_{2}\left(z_{2}\right)\hat{P}_{1}\left(w_{1}\right)\hat{P}_{2}\left(w_{2}\right)\right)\hat{F}_{2}\left(w_{1},\hat{\theta}_{2}\left(P_{1}\left(z_{1}\right)\tilde{P}_{2}\left(z_{2}\right)\hat{P}_{1}\left(w_{1}\right)\right)\right)F_{2}\left(z_{1},z_{2}\right)\right)\\
&=&\hat{r}_{2}\tilde{P}_{2}^{(2)}\left(1\right)+\tilde{\mu}_{2}^{2}\hat{R}_{2}^{(2)}\left(1\right)+2\hat{r}_{2}\tilde{\mu}_{2}F_{2}^{(0,1)}+2\hat{r}_{2}\frac{\tilde{\mu}_{2}^{2}}{1-\hat{\mu}_{2}}\hat{F}_{2}^{(0,1)}+\frac{1}{1-\hat{\mu}_{2}}\tilde{P}_{2}^{(2)}\left(1\right)\hat{F}_{2}^{(0,1)}\\
&+&\tilde{\mu}_{2}^{2}\hat{\theta}_{2}^{(2)}\left(1\right)\hat{F}_{2}^{(0,1)}+2\frac{\tilde{\mu}_{2}}{1-\hat{\mu}_{2}}F_{2}^{(0,1)}\hat{F}_{2}^{(0,1)}+F_{2}^{(0,2)}+\left(\frac{\tilde{\mu}_{2}}{1-\hat{\mu}_{2}}\right)^{2}\hat{F}_{2}^{(0,2)}.
\end{eqnarray*}

%7/39

\item \begin{eqnarray*} &&\frac{\partial}{\partial
w_1}\frac{\partial}{\partial
z_2}\left(\hat{R}_{2}\left(P_{1}\left(z_{1}\right)\tilde{P}_{2}\left(z_{2}\right)\hat{P}_{1}\left(w_{1}\right)\hat{P}_{2}\left(w_{2}\right)\right)\hat{F}_{2}\left(w_{1},\hat{\theta}_{2}\left(P_{1}\left(z_{1}\right)\tilde{P}_{2}\left(z_{2}\right)\hat{P}_{1}\left(w_{1}\right)\right)\right)F_{2}\left(z_{1},z_{2}\right)\right)\\
&=&\hat{r}_{2}\tilde{\mu}_{2}\hat{\mu}_{1}+\tilde{\mu}_{2}\hat{\mu}_{1}\hat{R}_{2}^{(2)}\left(1\right)+\hat{r}_{2}\hat{\mu}_{1}F_{2}^{(0,1)}+\hat{r}_{2}\frac{\tilde{\mu}_{2}\hat{\mu}_{1}}{1-\hat{\mu}_{2}}\hat{F}_{2}^{(0,1)}+\hat{r}_{2}\tilde{\mu}_{2}\hat{F}_{2}^{(1,0)}+F_{2}^{(0,1)}\hat{F}_{2}^{(1,0)}+\frac{\tilde{\mu}_{2}}{1-\hat{\mu}_{2}}\hat{F}_{2}^{(1,1)}.
\end{eqnarray*}
%8/40

\item \begin{eqnarray*} &&\frac{\partial}{\partial
w_2}\frac{\partial}{\partial
z_2}\left(\hat{R}_{2}\left(P_{1}\left(z_{1}\right)\tilde{P}_{2}\left(z_{2}\right)\hat{P}_{1}\left(w_{1}\right)\hat{P}_{2}\left(w_{2}\right)\right)\hat{F}_{2}\left(w_{1},\hat{\theta}_{2}\left(P_{1}\left(z_{1}\right)\tilde{P}_{2}\left(z_{2}\right)\hat{P}_{1}\left(w_{1}\right)\right)\right)F_{2}\left(z_{1},z_{2}\right)\right)\\
&=&\hat{r}_{2}\tilde{\mu}_{2}\hat{\mu}_{2}+\tilde{\mu}_{2}\hat{\mu}_{2}\hat{R}_{2}^{(2)}\left(1\right)+\hat{r}_{2}\hat{\mu}_{2}F_{2}^{(0,1)}+
\frac{\tilde{\mu}_{2}\hat{\mu}_{2}}{1-\hat{\mu}_{2}}\hat{F}_{2}^{(0,1)}+2\hat{r}_{2}\frac{\tilde{\mu}_{2}\hat{\mu}_{2}}{1-\hat{\mu}_{2}}\hat{F}_{2}^{(0,1)}+\tilde{\mu}_{2}\hat{\mu}_{2}\hat{\theta}_{2}^{(2)}\left(1\right)\hat{F}_{2}^{(0,1)}\\
&+&\frac{\hat{\mu}_{2}}{1-\hat{\mu}_{2}}F_{2}^{(0,1)}\hat{F}_{2}^{(1,0)}+\tilde{\mu}_{2}\hat{\mu}_{2}\left(\frac{1}{1-\hat{\mu}_{2}}\right)\hat{F}_{2}^{(0,2)}.
\end{eqnarray*}
%___________________________________________________________________________________________
\subsubsection{Mixtas para $w_{1}$:}
%___________________________________________________________________________________________

%9/41
\item \begin{eqnarray*} &&\frac{\partial}{\partial
z_1}\frac{\partial}{\partial
w_1}\left(\hat{R}_{2}\left(P_{1}\left(z_{1}\right)\tilde{P}_{2}\left(z_{2}\right)\hat{P}_{1}\left(w_{1}\right)\hat{P}_{2}\left(w_{2}\right)\right)\hat{F}_{2}\left(w_{1},\hat{\theta}_{2}\left(P_{1}\left(z_{1}\right)\tilde{P}_{2}\left(z_{2}\right)\hat{P}_{1}\left(w_{1}\right)\right)\right)F_{2}\left(z_{1},z_{2}\right)\right)\\
&=&\hat{r}_{2}\mu_{1}\hat{\mu}_{1}+\mu_{1}\hat{\mu}_{1}\hat{R}_{2}^{(2)}\left(1\right)+\hat{r}_{2}\frac{\mu_{1}\hat{\mu}_{1}}{1-\hat{\mu}_{2}}\hat{F}_{2}^{(0,1)}+\hat{r}_{2}\hat{\mu}_{1}F_{2}^{(1,0)}+\hat{r}_{2}\mu_{1}\hat{F}_{2}^{(1,0)}+F_{2}^{(1,0)}\hat{F}_{2}^{(1,0)}+\frac{\mu_{1}}{1-\hat{\mu}_{2}}\hat{F}_{2}^{(1,1)}.
\end{eqnarray*}


%10/42
\item \begin{eqnarray*} &&\frac{\partial}{\partial
z_2}\frac{\partial}{\partial
w_1}\left(\hat{R}_{2}\left(P_{1}\left(z_{1}\right)\tilde{P}_{2}\left(z_{2}\right)\hat{P}_{1}\left(w_{1}\right)\hat{P}_{2}\left(w_{2}\right)\right)\hat{F}_{2}\left(w_{1},\hat{\theta}_{2}\left(P_{1}\left(z_{1}\right)\tilde{P}_{2}\left(z_{2}\right)\hat{P}_{1}\left(w_{1}\right)\right)\right)F_{2}\left(z_{1},z_{2}\right)\right)\\
&=&\hat{r}_{2}\tilde{\mu}_{2}\hat{\mu}_{1}+\tilde{\mu}_{2}\hat{\mu}_{1}\hat{R}_{2}^{(2)}\left(1\right)+\hat{r}_{2}\hat{\mu}_{1}F_{2}^{(0,1)}+
\hat{r}_{2}\frac{\tilde{\mu}_{2}\hat{\mu}_{1}}{1-\hat{\mu}_{2}}\hat{F}_{2}^{(0,1)}+\hat{r}_{2}\tilde{\mu}_{2}\hat{F}_{2}^{(1,0)}+F_{2}^{(0,1)}\hat{F}_{2}^{(1,0)}+\frac{\tilde{\mu}_{2}}{1-\hat{\mu}_{2}}\hat{F}_{2}^{(1,1)}.
\end{eqnarray*}


%11/43
\item \begin{eqnarray*} &&\frac{\partial}{\partial
w_1}\frac{\partial}{\partial
w_1}\left(\hat{R}_{2}\left(P_{1}\left(z_{1}\right)\tilde{P}_{2}\left(z_{2}\right)\hat{P}_{1}\left(w_{1}\right)\hat{P}_{2}\left(w_{2}\right)\right)\hat{F}_{2}\left(w_{1},\hat{\theta}_{2}\left(P_{1}\left(z_{1}\right)\tilde{P}_{2}\left(z_{2}\right)\hat{P}_{1}\left(w_{1}\right)\right)\right)F_{2}\left(z_{1},z_{2}\right)\right)\\
&=&\hat{r}_{2}\hat{P}_{1}^{(2)}\left(1\right)+\hat{\mu}_{1}^{2}\hat{R}_{2}^{(2)}\left(1\right)+2\hat{r}_{2}\hat{\mu}_{1}\hat{F}_{2}^{(1,0)}
+\hat{F}_{2}^{(2,0)}.
\end{eqnarray*}


%12/44
\item \begin{eqnarray*} &&\frac{\partial}{\partial
w_2}\frac{\partial}{\partial
w_1}\left(\hat{R}_{2}\left(P_{1}\left(z_{1}\right)\tilde{P}_{2}\left(z_{2}\right)\hat{P}_{1}\left(w_{1}\right)\hat{P}_{2}\left(w_{2}\right)\right)\hat{F}_{2}\left(w_{1},\hat{\theta}_{2}\left(P_{1}\left(z_{1}\right)\tilde{P}_{2}\left(z_{2}\right)\hat{P}_{1}\left(w_{1}\right)\right)\right)F_{2}\left(z_{1},z_{2}\right)\right)\\
&=&\hat{r}_{2}\hat{\mu}_{1}\hat{\mu}_{2}+\hat{\mu}_{1}\hat{\mu}_{2}\hat{R}_{2}^{(2)}\left(1\right)+
\hat{r}_{2}\frac{\hat{\mu}_{2}\hat{\mu}_{1}}{1-\hat{\mu}_{2}}\hat{F}_{2}^{(0,1)}
+\hat{r}_{2}\hat{\mu}_{2}\hat{F}_{2}^{(1,0)}+\frac{\hat{\mu}_{2}}{1-\hat{\mu}_{2}}\hat{F}_{2}^{(1,1)}.
\end{eqnarray*}
%___________________________________________________________________________________________
\subsubsection{Mixtas para $w_{2}$:}
%___________________________________________________________________________________________
%13/45
\item \begin{eqnarray*} &&\frac{\partial}{\partial
z_1}\frac{\partial}{\partial
w_2}\left(\hat{R}_{2}\left(P_{1}\left(z_{1}\right)\tilde{P}_{2}\left(z_{2}\right)\hat{P}_{1}\left(w_{1}\right)\hat{P}_{2}\left(w_{2}\right)\right)\hat{F}_{2}\left(w_{1},\hat{\theta}_{2}\left(P_{1}\left(z_{1}\right)\tilde{P}_{2}\left(z_{2}\right)\hat{P}_{1}\left(w_{1}\right)\right)\right)F_{2}\left(z_{1},z_{2}\right)\right)\\
&=&\hat{r}_{2}\mu_{1}\hat{\mu}_{2}+\mu_{1}\hat{\mu}_{2}\hat{R}_{2}^{(2)}\left(1\right)+
\frac{\mu_{1}\hat{\mu}_{2}}{1-\hat{\mu}_{2}}\hat{F}_{2}^{(0,1)} +2\hat{r}_{2}\frac{\mu_{1}\hat{\mu}_{2}}{1-\hat{\mu}_{2}}\hat{F}_{2}^{(0,1)}\\
&+&\mu_{1}\hat{\mu}_{2}\hat{\theta}_{2}^{(2)}\left(1\right)\hat{F}_{2}^{(0,1)}+\mu_{1}\hat{\mu}_{2}\left(\frac{1}{1-\hat{\mu}_{2}}\right)^{2}\hat{F}_{2}^{(0,2)}+\hat{r}_{2}\hat{\mu}_{2}F_{2}^{(1,0)}+\frac{\hat{\mu}_{2}}{1-\hat{\mu}_{2}}\hat{F}_{2}^{(0,1)}F_{2}^{(1,0)}.\end{eqnarray*}


%14/46
\item \begin{eqnarray*} &&\frac{\partial}{\partial
z_2}\frac{\partial}{\partial
w_2}\left(\hat{R}_{2}\left(P_{1}\left(z_{1}\right)\tilde{P}_{2}\left(z_{2}\right)\hat{P}_{1}\left(w_{1}\right)\hat{P}_{2}\left(w_{2}\right)\right)\hat{F}_{2}\left(w_{1},\hat{\theta}_{2}\left(P_{1}\left(z_{1}\right)\tilde{P}_{2}\left(z_{2}\right)\hat{P}_{1}\left(w_{1}\right)\right)\right)F_{2}\left(z_{1},z_{2}\right)\right)\\
&=&\hat{r}_{2}\tilde{\mu}_{2}\hat{\mu}_{2}+\tilde{\mu}_{2}\hat{\mu}_{2}\hat{R}_{2}^{(2)}\left(1\right)+\hat{r}_{2}\hat{\mu}_{2}F_{2}^{(0,1)}+\frac{\tilde{\mu}_{2}\hat{\mu}_{2}}{1-\hat{\mu}_{2}}\hat{F}_{2}^{(0,1)}+
2\hat{r}_{2}\frac{\tilde{\mu}_{2}\hat{\mu}_{2}}{1-\hat{\mu}_{2}}\hat{F}_{2}^{(0,1)}+\tilde{\mu}_{2}\hat{\mu}_{2}\hat{\theta}_{2}^{(2)}\left(1\right)\hat{F}_{2}^{(0,1)}\\
&+&\frac{\hat{\mu}_{2}}{1-\hat{\mu}_{2}}\hat{F}_{2}^{(0,1)}F_{2}^{(0,1)}+\tilde{\mu}_{2}\hat{\mu}_{2}\left(\frac{1}{1-\hat{\mu}_{2}}\right)^{2}\hat{F}_{2}^{(0,2)}.
\end{eqnarray*}

%15/47

\item \begin{eqnarray*} &&\frac{\partial}{\partial
w_1}\frac{\partial}{\partial
w_2}\left(\hat{R}_{2}\left(P_{1}\left(z_{1}\right)\tilde{P}_{2}\left(z_{2}\right)\hat{P}_{1}\left(w_{1}\right)\hat{P}_{2}\left(w_{2}\right)\right)\hat{F}_{2}\left(w_{1},\hat{\theta}_{2}\left(P_{1}\left(z_{1}\right)\tilde{P}_{2}\left(z_{2}\right)\hat{P}_{1}\left(w_{1}\right)\right)\right)F_{2}\left(z_{1},z_{2}\right)\right)\\
&=&\hat{r}_{2}\hat{\mu}_{1}\hat{\mu}_{2}+\hat{\mu}_{1}\hat{\mu}_{2}\hat{R}_{2}^{(2)}\left(1\right)+
\hat{r}_{2}\frac{\hat{\mu}_{1}\hat{\mu}_{2}}{1-\hat{\mu}_{2}}\hat{F}_{2}^{(0,1)}+
\hat{r}_{2}\hat{\mu}_{2}\hat{F}_{2}^{(1,0)}+\frac{\hat{\mu}_{2}}{1-\hat{\mu}_{2}}\hat{F}_{2}^{(1,1)}.
\end{eqnarray*}

%16/48
\item \begin{eqnarray*} &&\frac{\partial}{\partial
w_2}\frac{\partial}{\partial
w_2}\left(\hat{R}_{2}\left(P_{1}\left(z_{1}\right)\tilde{P}_{2}\left(z_{2}\right)\hat{P}_{1}\left(w_{1}\right)\hat{P}_{2}\left(w_{2}\right)\right)\hat{F}_{2}\left(w_{1},\hat{\theta}_{2}\left(P_{1}\left(z_{1}\right)\tilde{P}_{2}\left(z_{2}\right)\hat{P}_{1}\left(w_{1}\right)\right)\right)F_{2}\left(z_{1},z_{2};\zeta_{2}\right)\right)\\
&=&\hat{r}_{2}P_{2}^{(2)}\left(1\right)+\hat{\mu}_{2}^{2}\hat{R}_{2}^{(2)}\left(1\right)+2\hat{r}_{2}\frac{\hat{\mu}_{2}^{2}}{1-\hat{\mu}_{2}}\hat{F}_{2}^{(0,1)}+\frac{1}{1-\hat{\mu}_{2}}\hat{P}_{2}^{(2)}\left(1\right)\hat{F}_{2}^{(0,1)}+\hat{\mu}_{2}^{2}\hat{\theta}_{2}^{(2)}\left(1\right)\hat{F}_{2}^{(0,1)}\\
&+&\left(\frac{\hat{\mu}_{2}}{1-\hat{\mu}_{2}}\right)^{2}\hat{F}_{2}^{(0,2)}.
\end{eqnarray*}


\end{enumerate}



%___________________________________________________________________________________________
%
\subsection{Derivadas de Segundo Orden para $\hat{F}_{2}$}
%___________________________________________________________________________________________
\begin{enumerate}
%___________________________________________________________________________________________
\subsubsection{Mixtas para $z_{1}$:}
%___________________________________________________________________________________________
%1/49

\item \begin{eqnarray*} &&\frac{\partial}{\partial
z_1}\frac{\partial}{\partial
z_1}\left(\hat{R}_{1}\left(P_{1}\left(z_{1}\right)\tilde{P}_{2}\left(z_{2}\right)\hat{P}_{1}\left(w_{1}\right)\hat{P}_{2}\left(w_{2}\right)\right)\hat{F}_{1}\left(\hat{\theta}_{1}\left(P_{1}\left(z_{1}\right)\tilde{P}_{2}\left(z_{2}\right)
\hat{P}_{2}\left(w_{2}\right)\right),w_{2}\right)F_{1}\left(z_{1},z_{2}\right)\right)\\
&=&\hat{r}_{1}P_{1}^{(2)}\left(1\right)+
\mu_{1}^{2}\hat{R}_{1}^{(2)}\left(1\right)+
2\hat{r}_{1}\mu_{1}F_{1}^{(1,0)}+
2\hat{r}_{1}\frac{\mu_{1}^{2}}{1-\hat{\mu}_{1}}\hat{F}_{1}^{(1,0)}+
\frac{1}{1-\hat{\mu}_{1}}P_{1}^{(2)}\left(1\right)\hat{F}_{1}^{(1,0)}+\mu_{1}^{2}\hat{\theta}_{1}^{(2)}\left(1\right)\hat{F}_{1}^{(1,0)}\\
&+&2\frac{\mu_{1}}{1-\hat{\mu}_{1}}\hat{F}_{1}^{(1,0)}F_{1}^{(1,0)}+F_{1}^{(2,0)}
+\left(\frac{\mu_{1}}{1-\hat{\mu}_{1}}\right)^{2}\hat{F}_{1}^{(2,0)}.
\end{eqnarray*}

%2/50

\item \begin{eqnarray*} &&\frac{\partial}{\partial
z_2}\frac{\partial}{\partial
z_1}\left(\hat{R}_{1}\left(P_{1}\left(z_{1}\right)\tilde{P}_{2}\left(z_{2}\right)\hat{P}_{1}\left(w_{1}\right)\hat{P}_{2}\left(w_{2}\right)\right)\hat{F}_{1}\left(\hat{\theta}_{1}\left(P_{1}\left(z_{1}\right)\tilde{P}_{2}\left(z_{2}\right)
\hat{P}_{2}\left(w_{2}\right)\right),w_{2}\right)F_{1}\left(z_{1},z_{2}\right)\right)\\
&=&\hat{r}_{1}\mu_{1}\tilde{\mu}_{2}+\mu_{1}\tilde{\mu}_{2}\hat{R}_{1}^{(2)}\left(1\right)+
\hat{r}_{1}\mu_{1}F_{1}^{(0,1)}+\tilde{\mu}_{2}\hat{r}_{1}F_{1}^{(1,0)}+
\frac{\mu_{1}\tilde{\mu}_{2}}{1-\hat{\mu}_{1}}\hat{F}_{1}^{(1,0)}+2\hat{r}_{1}\frac{\mu_{1}\tilde{\mu}_{2}}{1-\hat{\mu}_{1}}\hat{F}_{1}^{(1,0)}\\
&+&\mu_{1}\tilde{\mu}_{2}\hat{\theta}_{1}^{(2)}\left(1\right)\hat{F}_{1}^{(1,0)}+
\frac{\mu_{1}}{1-\hat{\mu}_{1}}\hat{F}_{1}^{(1,0)}F_{1}^{(0,1)}+
\frac{\tilde{\mu}_{2}}{1-\hat{\mu}_{1}}\hat{F}_{1}^{(1,0)}F_{1}^{(1,0)}+
F_{1}^{(1,1)}\\
&+&\mu_{1}\tilde{\mu}_{2}\left(\frac{1}{1-\hat{\mu}_{1}}\right)^{2}\hat{F}_{1}^{(2,0)}.
\end{eqnarray*}

%3/51

\item \begin{eqnarray*} &&\frac{\partial}{\partial
w_1}\frac{\partial}{\partial
z_1}\left(\hat{R}_{1}\left(P_{1}\left(z_{1}\right)\tilde{P}_{2}\left(z_{2}\right)\hat{P}_{1}\left(w_{1}\right)\hat{P}_{2}\left(w_{2}\right)\right)\hat{F}_{1}\left(\hat{\theta}_{1}\left(P_{1}\left(z_{1}\right)\tilde{P}_{2}\left(z_{2}\right)
\hat{P}_{2}\left(w_{2}\right)\right),w_{2}\right)F_{1}\left(z_{1},z_{2}\right)\right)\\
&=&\hat{r}_{1}\mu_{1}\hat{\mu}_{1}+\mu_{1}\hat{\mu}_{1}\hat{R}_{1}^{(2)}\left(1\right)+\hat{r}_{1}\hat{\mu}_{1}F_{1}^{(1,0)}+
\hat{r}_{1}\frac{\mu_{1}\hat{\mu}_{1}}{1-\hat{\mu}_{1}}\hat{F}_{1}^{(1,0)}.
\end{eqnarray*}

%4/52

\item \begin{eqnarray*} &&\frac{\partial}{\partial
w_2}\frac{\partial}{\partial
z_1}\left(\hat{R}_{1}\left(P_{1}\left(z_{1}\right)\tilde{P}_{2}\left(z_{2}\right)\hat{P}_{1}\left(w_{1}\right)\hat{P}_{2}\left(w_{2}\right)\right)\hat{F}_{1}\left(\hat{\theta}_{1}\left(P_{1}\left(z_{1}\right)\tilde{P}_{2}\left(z_{2}\right)
\hat{P}_{2}\left(w_{2}\right)\right),w_{2}\right)F_{1}\left(z_{1},z_{2}\right)\right)\\
&=&\hat{r}_{1}\mu_{1}\hat{\mu}_{2}+\mu_{1}\hat{\mu}_{2}\hat{R}_{1}^{(2)}\left(1\right)+\hat{r}_{1}\hat{\mu}_{2}F_{1}^{(1,0)}+\frac{\mu_{1}\hat{\mu}_{2}}{1-\hat{\mu}_{1}}\hat{F}_{1}^{(1,0)}+\hat{r}_{1}\frac{\mu_{1}\hat{\mu}_{2}}{1-\hat{\mu}_{1}}\hat{F}_{1}^{(1,0)}+\mu_{1}\hat{\mu}_{2}\hat{\theta}_{1}^{(2)}\left(1\right)\hat{F}_{1}^{(1,0)}\\
&+&\hat{r}_{1}\mu_{1}\left(\hat{F}_{1}^{(0,1)}+\frac{\hat{\mu}_{2}}{1-\hat{\mu}_{1}}\hat{F}_{1}^{(1,0)}\right)+F_{1}^{(1,0)}\left(\hat{F}_{1}^{(0,1)}+\frac{\hat{\mu}_{2}}{1-\hat{\mu}_{1}}\hat{F}_{1}^{(1,0)}\right)+\frac{\mu_{1}}{1-\hat{\mu}_{1}}\left(\hat{F}_{1}^{(1,1)}+\frac{\hat{\mu}_{2}}{1-\hat{\mu}_{1}}\hat{F}_{1}^{(2,0)}\right).
\end{eqnarray*}
%___________________________________________________________________________________________
\subsubsection{Mixtas para $z_{2}$:}
%___________________________________________________________________________________________
%5/53

\item \begin{eqnarray*} &&\frac{\partial}{\partial
z_1}\frac{\partial}{\partial
z_2}\left(\hat{R}_{1}\left(P_{1}\left(z_{1}\right)\tilde{P}_{2}\left(z_{2}\right)\hat{P}_{1}\left(w_{1}\right)\hat{P}_{2}\left(w_{2}\right)\right)\hat{F}_{1}\left(\hat{\theta}_{1}\left(P_{1}\left(z_{1}\right)\tilde{P}_{2}\left(z_{2}\right)
\hat{P}_{2}\left(w_{2}\right)\right),w_{2}\right)F_{1}\left(z_{1},z_{2}\right)\right)\\
&=&\hat{r}_{1}\mu_{1}\tilde{\mu}_{2}+\mu_{1}\tilde{\mu}_{2}\hat{R}_{1}^{(2)}\left(1\right)+\hat{r}_{1}\mu_{1}F_{1}^{(0,1)}+\hat{r}_{1}\tilde{\mu}_{2}F_{1}^{(1,0)}+\frac{\mu_{1}\tilde{\mu}_{2}}{1-\hat{\mu}_{1}}\hat{F}_{1}^{(1,0)}+2\hat{r}_{1}\frac{\mu_{1}\tilde{\mu}_{2}}{1-\hat{\mu}_{1}}\hat{F}_{1}^{(1,0)}\\
&+&\mu_{1}\tilde{\mu}_{2}\hat{\theta}_{1}^{(2)}\left(1\right)\hat{F}_{1}^{(1,0)}+\frac{\mu_{1}}{1-\hat{\mu}_{1}}\hat{F}_{1}^{(1,0)}F_{1}^{(0,1)}+\frac{\tilde{\mu}_{2}}{1-\hat{\mu}_{1}}\hat{F}_{1}^{(1,0)}F_{1}^{(1,0)}+F_{1}^{(1,1)}+\mu_{1}\tilde{\mu}_{2}\left(\frac{1}{1-\hat{\mu}_{1}}\right)^{2}\hat{F}_{1}^{(2,0)}.
\end{eqnarray*}

%6/54
\item \begin{eqnarray*} &&\frac{\partial}{\partial
z_2}\frac{\partial}{\partial
z_2}\left(\hat{R}_{1}\left(P_{1}\left(z_{1}\right)\tilde{P}_{2}\left(z_{2}\right)\hat{P}_{1}\left(w_{1}\right)\hat{P}_{2}\left(w_{2}\right)\right)\hat{F}_{1}\left(\hat{\theta}_{1}\left(P_{1}\left(z_{1}\right)\tilde{P}_{2}\left(z_{2}\right)
\hat{P}_{2}\left(w_{2}\right)\right),w_{2}\right)F_{1}\left(z_{1},z_{2}\right)\right)\\
&=&\hat{r}_{1}\tilde{P}_{2}^{(2)}\left(1\right)+\tilde{\mu}_{2}^{2}\hat{R}_{1}^{(2)}\left(1\right)+2\hat{r}_{1}\tilde{\mu}_{2}F_{1}^{(0,1)}+ F_{1}^{(0,2)}+2\hat{r}_{1}\frac{\tilde{\mu}_{2}^{2}}{1-\hat{\mu}_{1}}\hat{F}_{1}^{(1,0)}+\frac{1}{1-\hat{\mu}_{1}}\tilde{P}_{2}^{(2)}\left(1\right)\hat{F}_{1}^{(1,0)}\\
&+&\tilde{\mu}_{2}^{2}\hat{\theta}_{1}^{(2)}\left(1\right)\hat{F}_{1}^{(1,0)}+2\frac{\tilde{\mu}_{2}}{1-\hat{\mu}_{1}}F^{(0,1)}\hat{F}_{1}^{(1,0)}+\left(\frac{\tilde{\mu}_{2}}{1-\hat{\mu}_{1}}\right)^{2}\hat{F}_{1}^{(2,0)}.
\end{eqnarray*}
%7/55

\item \begin{eqnarray*} &&\frac{\partial}{\partial
w_1}\frac{\partial}{\partial
z_2}\left(\hat{R}_{1}\left(P_{1}\left(z_{1}\right)\tilde{P}_{2}\left(z_{2}\right)\hat{P}_{1}\left(w_{1}\right)\hat{P}_{2}\left(w_{2}\right)\right)\hat{F}_{1}\left(\hat{\theta}_{1}\left(P_{1}\left(z_{1}\right)\tilde{P}_{2}\left(z_{2}\right)
\hat{P}_{2}\left(w_{2}\right)\right),w_{2}\right)F_{1}\left(z_{1},z_{2}\right)\right)\\
&=&\hat{r}_{1}\hat{\mu}_{1}\tilde{\mu}_{2}+\hat{\mu}_{1}\tilde{\mu}_{2}\hat{R}_{1}^{(2)}\left(1\right)+
\hat{r}_{1}\hat{\mu}_{1}F_{1}^{(0,1)}+\hat{r}_{1}\frac{\hat{\mu}_{1}\tilde{\mu}_{2}}{1-\hat{\mu}_{1}}\hat{F}_{1}^{(1,0)}.
\end{eqnarray*}
%8/56

\item \begin{eqnarray*} &&\frac{\partial}{\partial
w_2}\frac{\partial}{\partial
z_2}\left(\hat{R}_{1}\left(P_{1}\left(z_{1}\right)\tilde{P}_{2}\left(z_{2}\right)\hat{P}_{1}\left(w_{1}\right)\hat{P}_{2}\left(w_{2}\right)\right)\hat{F}_{1}\left(\hat{\theta}_{1}\left(P_{1}\left(z_{1}\right)\tilde{P}_{2}\left(z_{2}\right)
\hat{P}_{2}\left(w_{2}\right)\right),w_{2}\right)F_{1}\left(z_{1},z_{2}\right)\right)\\
&=&\hat{r}_{1}\tilde{\mu}_{2}\hat{\mu}_{2}+\hat{\mu}_{2}\tilde{\mu}_{2}\hat{R}_{1}^{(2)}\left(1\right)+\hat{\mu}_{2}\hat{R}_{1}^{(2)}\left(1\right)F_{1}^{(0,1)}+\frac{\hat{\mu}_{2}\tilde{\mu}_{2}}{1-\hat{\mu}_{1}}\hat{F}_{1}^{(1,0)}+
\hat{r}_{1}\frac{\hat{\mu}_{2}\tilde{\mu}_{2}}{1-\hat{\mu}_{1}}\hat{F}_{1}^{(1,0)}\\
&+&\hat{\mu}_{2}\tilde{\mu}_{2}\hat{\theta}_{1}^{(2)}\left(1\right)\hat{F}_{1}^{(1,0)}+\hat{r}_{1}\tilde{\mu}_{2}\left(\hat{F}_{1}^{(0,1)}+\frac{\hat{\mu}_{2}}{1-\hat{\mu}_{1}}\hat{F}_{1}^{(1,0)}\right)+F_{1}^{(0,1)}\left(\hat{F}_{1}^{(0,1)}+\frac{\hat{\mu}_{2}}{1-\hat{\mu}_{1}}\hat{F}_{1}^{(1,0)}\right)\\
&+&\frac{\tilde{\mu}_{2}}{1-\hat{\mu}_{1}}\left(\hat{F}_{1}^{(1,1)}+\frac{\hat{\mu}_{2}}{1-\hat{\mu}_{1}}\hat{F}_{1}^{(2,0)}\right).
\end{eqnarray*}
%___________________________________________________________________________________________
\subsubsection{Mixtas para $w_{1}$:}
%___________________________________________________________________________________________
%9/57
\item \begin{eqnarray*} &&\frac{\partial}{\partial
z_1}\frac{\partial}{\partial
w_1}\left(\hat{R}_{1}\left(P_{1}\left(z_{1}\right)\tilde{P}_{2}\left(z_{2}\right)\hat{P}_{1}\left(w_{1}\right)\hat{P}_{2}\left(w_{2}\right)\right)\hat{F}_{1}\left(\hat{\theta}_{1}\left(P_{1}\left(z_{1}\right)\tilde{P}_{2}\left(z_{2}\right)
\hat{P}_{2}\left(w_{2}\right)\right),w_{2}\right)F_{1}\left(z_{1},z_{2}\right)\right)\\
&=&\hat{r}_{1}\mu_{1}\hat{\mu}_{1}+\mu_{1}\hat{\mu}_{1}\hat{R}_{1}^{(2)}\left(1\right)+\hat{r}_{1}\hat{\mu}_{1}F_{1}^{(1,0)}+\hat{r}_{1}\frac{\mu_{1}\hat{\mu}_{1}}{1-\hat{\mu}_{1}}\hat{F}_{1}^{(1,0)}.
\end{eqnarray*}
%10/58
\item \begin{eqnarray*} &&\frac{\partial}{\partial
z_2}\frac{\partial}{\partial
w_1}\left(\hat{R}_{1}\left(P_{1}\left(z_{1}\right)\tilde{P}_{2}\left(z_{2}\right)\hat{P}_{1}\left(w_{1}\right)\hat{P}_{2}\left(w_{2}\right)\right)\hat{F}_{1}\left(\hat{\theta}_{1}\left(P_{1}\left(z_{1}\right)\tilde{P}_{2}\left(z_{2}\right)
\hat{P}_{2}\left(w_{2}\right)\right),w_{2}\right)F_{1}\left(z_{1},z_{2}\right)\right)\\
&=&\hat{r}_{1}\tilde{\mu}_{2}\hat{\mu}_{1}+\tilde{\mu}_{2}\hat{\mu}_{1}\hat{R}_{1}^{(2)}\left(1\right)+\hat{r}_{1}\hat{\mu}_{1}F_{1}^{(0,1)}+\hat{r}_{1}\frac{\tilde{\mu}_{2}\hat{\mu}_{1}}{1-\hat{\mu}_{1}}\hat{F}_{1}^{(1,0)}.
\end{eqnarray*}
%11/59
\item \begin{eqnarray*} &&\frac{\partial}{\partial
w_1}\frac{\partial}{\partial
w_1}\left(\hat{R}_{1}\left(P_{1}\left(z_{1}\right)\tilde{P}_{2}\left(z_{2}\right)\hat{P}_{1}\left(w_{1}\right)\hat{P}_{2}\left(w_{2}\right)\right)\hat{F}_{1}\left(\hat{\theta}_{1}\left(P_{1}\left(z_{1}\right)\tilde{P}_{2}\left(z_{2}\right)
\hat{P}_{2}\left(w_{2}\right)\right),w_{2}\right)F_{1}\left(z_{1},z_{2}\right)\right)\\
&=&\hat{r}_{1}\hat{P}_{1}^{(2)}\left(1\right)+\hat{\mu}_{1}^{2}\hat{R}_{1}^{(2)}\left(1\right).
\end{eqnarray*}
%12/60
\item \begin{eqnarray*} &&\frac{\partial}{\partial
w_2}\frac{\partial}{\partial
w_1}\left(\hat{R}_{1}\left(P_{1}\left(z_{1}\right)\tilde{P}_{2}\left(z_{2}\right)\hat{P}_{1}\left(w_{1}\right)\hat{P}_{2}\left(w_{2}\right)\right)\hat{F}_{1}\left(\hat{\theta}_{1}\left(P_{1}\left(z_{1}\right)\tilde{P}_{2}\left(z_{2}\right)
\hat{P}_{2}\left(w_{2}\right)\right),w_{2}\right)F_{1}\left(z_{1},z_{2}\right)\right)\\
&=&\hat{r}_{1}\hat{\mu}_{2}\hat{\mu}_{1}+\hat{\mu}_{2}\hat{\mu}_{1}\hat{R}_{1}^{(2)}\left(1\right)+\hat{r}_{1}\hat{\mu}_{1}\left(\hat{F}_{1}^{(0,1)}+\frac{\hat{\mu}_{2}}{1-\hat{\mu}_{1}}\hat{F}_{1}^{(1,0)}\right).
\end{eqnarray*}
%___________________________________________________________________________________________
\subsubsection{Mixtas para $w_{1}$:}
%___________________________________________________________________________________________
%13/61



\item \begin{eqnarray*} &&\frac{\partial}{\partial
z_1}\frac{\partial}{\partial
w_2}\left(\hat{R}_{1}\left(P_{1}\left(z_{1}\right)\tilde{P}_{2}\left(z_{2}\right)\hat{P}_{1}\left(w_{1}\right)\hat{P}_{2}\left(w_{2}\right)\right)\hat{F}_{1}\left(\hat{\theta}_{1}\left(P_{1}\left(z_{1}\right)\tilde{P}_{2}\left(z_{2}\right)
\hat{P}_{2}\left(w_{2}\right)\right),w_{2}\right)F_{1}\left(z_{1},z_{2}\right)\right)\\
&=&\hat{r}_{1}\mu_{1}\hat{\mu}_{2}+\mu_{1}\hat{\mu}_{2}\hat{R}_{1}^{(2)}\left(1\right)+\hat{r}_{1}\hat{\mu}_{2}F_{1}^{(1,0)}+
\hat{r}_{1}\frac{\mu_{1}\hat{\mu}_{2}}{1-\hat{\mu}_{1}}\hat{F}_{1}^{(1,0)}+\hat{r}_{1}\mu_{1}\left(\hat{F}_{1}^{(0,1)}+\frac{\hat{\mu}_{2}}{1-\hat{\mu}_{1}}\hat{F}_{1}^{(1,0)}\right)\\
&+&F_{1}^{(1,0)}\left(\hat{F}_{1}^{(0,1)}+\frac{\hat{\mu}_{2}}{1-\hat{\mu}_{1}}\hat{F}_{1}^{(1,0)}\right)+\frac{\mu_{1}\hat{\mu}_{2}}{1-\hat{\mu}_{1}}\hat{F}_{1}^{(1,0)}+\mu_{1}\hat{\mu}_{2}\hat{\theta}_{1}^{(2)}\left(1\right)\hat{F}_{1}^{(1,0)}+\frac{\mu_{1}}{1-\hat{\mu}_{1}}\hat{F}_{1}^{(1,1)}\\
&+&\mu_{1}\hat{\mu}_{2}\left(\frac{1}{1-\hat{\mu}_{1}}\right)^{2}\hat{F}_{1}^{(2,0)}.
\end{eqnarray*}

%14/62
\item \begin{eqnarray*} &&\frac{\partial}{\partial
z_2}\frac{\partial}{\partial
w_2}\left(\hat{R}_{1}\left(P_{1}\left(z_{1}\right)\tilde{P}_{2}\left(z_{2}\right)\hat{P}_{1}\left(w_{1}\right)\hat{P}_{2}\left(w_{2}\right)\right)\hat{F}_{1}\left(\hat{\theta}_{1}\left(P_{1}\left(z_{1}\right)\tilde{P}_{2}\left(z_{2}\right)
\hat{P}_{2}\left(w_{2}\right)\right),w_{2}\right)F_{1}\left(z_{1},z_{2}\right)\right)\\
&=&\hat{r}_{1}\tilde{\mu}_{2}\hat{\mu}_{2}+\tilde{\mu}_{2}\hat{\mu}_{2}\hat{R}_{1}^{(2)}\left(1\right)+\hat{r}_{1}\hat{\mu}_{2}F_{1}^{(0,1)}+\hat{r}_{1}\frac{\tilde{\mu}_{2}\hat{\mu}_{2}}{1-\hat{\mu}_{1}}\hat{F}_{1}^{(1,0)}+\hat{r}_{1}\tilde{\mu}_{2}\left(\hat{F}_{1}^{(0,1)}+\frac{\hat{\mu}_{2}}{1-\hat{\mu}_{1}}\hat{F}_{1}^{(1,0)}\right)\\
&+&F_{1}^{(0,1)}\left(\hat{F}_{1}^{(0,1)}+\frac{\hat{\mu}_{2}}{1-\hat{\mu}_{1}}\hat{F}_{1}^{(1,0)}\right)+\frac{\tilde{\mu}_{2}\hat{\mu}_{2}}{1-\hat{\mu}_{1}}\hat{F}_{1}^{(1,0)}+\tilde{\mu}_{2}\hat{\mu}_{2}\hat{\theta}_{1}^{(2)}\left(1\right)\hat{F}_{1}^{(1,0)}+\frac{\tilde{\mu}_{2}}{1-\hat{\mu}_{1}}\hat{F}_{1}^{(1,1)}\\
&+&\tilde{\mu}_{2}\hat{\mu}_{2}\left(\frac{1}{1-\hat{\mu}_{1}}\right)^{2}\hat{F}_{1}^{(2,0)}.
\end{eqnarray*}

%15/63

\item \begin{eqnarray*} &&\frac{\partial}{\partial
w_1}\frac{\partial}{\partial
w_2}\left(\hat{R}_{1}\left(P_{1}\left(z_{1}\right)\tilde{P}_{2}\left(z_{2}\right)\hat{P}_{1}\left(w_{1}\right)\hat{P}_{2}\left(w_{2}\right)\right)\hat{F}_{1}\left(\hat{\theta}_{1}\left(P_{1}\left(z_{1}\right)\tilde{P}_{2}\left(z_{2}\right)
\hat{P}_{2}\left(w_{2}\right)\right),w_{2}\right)F_{1}\left(z_{1},z_{2}\right)\right)\\
&=&\hat{r}_{1}\hat{\mu}_{2}\hat{\mu}_{1}+\hat{\mu}_{2}\hat{\mu}_{1}\hat{R}_{1}^{(2)}\left(1\right)+\hat{r}_{1}\hat{\mu}_{1}\left(\hat{F}_{1}^{(0,1)}+\frac{\hat{\mu}_{2}}{1-\hat{\mu}_{1}}\hat{F}_{1}^{(1,0)}\right).
\end{eqnarray*}

%16/64

\item \begin{eqnarray*} &&\frac{\partial}{\partial
w_2}\frac{\partial}{\partial
w_2}\left(\hat{R}_{1}\left(P_{1}\left(z_{1}\right)\tilde{P}_{2}\left(z_{2}\right)\hat{P}_{1}\left(w_{1}\right)\hat{P}_{2}\left(w_{2}\right)\right)\hat{F}_{1}\left(\hat{\theta}_{1}\left(P_{1}\left(z_{1}\right)\tilde{P}_{2}\left(z_{2}\right)
\hat{P}_{2}\left(w_{2}\right)\right),w_{2}\right)F_{1}\left(z_{1},z_{2}\right)\right)\\
&=&\hat{r}_{1}\hat{P}_{2}^{(2)}\left(1\right)+\hat{\mu}_{2}^{2}\hat{R}_{1}^{(2)}\left(1\right)+
2\hat{r}_{1}\hat{\mu}_{2}\left(\hat{F}_{1}^{(0,1)}+\frac{\hat{\mu}_{2}}{1-\hat{\mu}_{1}}\hat{F}_{1}^{(1,0)}\right)+
\hat{F}_{1}^{(0,2)}+\frac{1}{1-\hat{\mu}_{1}}\hat{P}_{2}^{(2)}\left(1\right)\hat{F}_{1}^{(1,0)}\\
&+&\hat{\mu}_{2}^{2}\hat{\theta}_{1}^{(2)}\left(1\right)\hat{F}_{1}^{(1,0)}+\frac{\hat{\mu}_{2}}{1-\hat{\mu}_{1}}\hat{F}_{1}^{(1,1)}+\frac{\hat{\mu}_{2}}{1-\hat{\mu}_{1}}\left(\hat{F}_{1}^{(1,1)}+\frac{\hat{\mu}_{2}}{1-\hat{\mu}_{1}}\hat{F}_{1}^{(2,0)}\right).
\end{eqnarray*}
%_________________________________________________________________________________________________________
%
%_________________________________________________________________________________________________________

\end{enumerate}


%----------------------------------------------------------------------------------------
%   INTRODUCTION
%----------------------------------------------------------------------------------------

\color{SaddleBrown} % SaddleBrown color for the introduction

\section*{Introducci\'on}
Un sistema de visitas (Polling System) consiste en una cola a la cu\'al llegan los usuarios para ser atendidos por uno o varios servidores de acuerdo a una pol\'itica determinada, en la cual se puede asumir que la manera en que los usuarios llegan a la misma es conforme a un proceso Poisson con tasa de llegada $\mu$. De igual manera se puede asumir que la distribuci\'on de los servicios a cada uno de los usuarios presentes en la cola es conforme a una variable aleatoria exponencial. Esto es la base para la conformación de los Sistemas de Visitas C\'iclicas, de los cuales es posible obtener sus Funciones Generadoras de Probabilidades, primeros y segundos momentos as\'i como medidas de desempe\~no que permiten tener una mejor descripci\'on del funcionamiento del sistema en cualquier momento $t$ asumiendo estabilidad.



%----------------------------------------------------------------------------------------
%   OBJECTIVES
%----------------------------------------------------------------------------------------

\color{DarkSlateGray} % DarkSlateGray color for the rest of the content

\section*{Objetivos Principales}

\begin{itemize}
%\item Generalizar los principales resultados existentes para Sistemas de Visitas C\'iclicas para el caso en el que se tienen dos Sistemas de Visitas C\'iclicas con propiedades similares.

\item Encontrar las ecuaciones que modelan el comportamiento de una Red de Sistemas de Visitas C\'iclicas (RSVC) con propiedades similares.

\item Encontrar expresiones anal\'iticas para las longitudes de las colas al momento en que el servidor llega a una de ellas para comenzar a dar servicio, as\'i como de sus segundos momentos.

\item Determinar las principales medidas de Desempe\~no para la RSVC tales como: N\'umero de usuarios presentes en cada una de las colas del sistema cuando uno de los servidores est\'a presente atendiendo, Tiempos que transcurre entre las visitas del servidor a la misma cola.


\end{itemize}

%----------------------------------------------------------------------------------------
%   MATERIALS AND METHODS
%----------------------------------------------------------------------------------------

\section*{Descripci\'on de la Red de Sistemas de Visitas C\'iclicas}

El uso de la Funci\'on Generadora de Probabilidades (FGP's) nos permite determinar las Funciones de Distribuci\'on de Probabilidades Conjunta de manera indirecta sin necesidad de hacer uso de las propiedades de las distribuciones de probabilidad de cada uno de los procesos que intervienen en la Red de Sistemas de Visitas C\'iclicas.\\
\begin{itemize}
\item Se definen los procesos para los arribos para cada una de las colas:$X_{i}\left(t\right)$ y $\hat{X}_{i}\left(t\right)$.  Y para los usuarios que se trasladan de un sistema a otro se tiene el proceso $Y\left(t\right)$,% entonces $P_{i}\left(z_{i}\right)&=&\esp\left[z_{i}^{X_{i}\left(t\right)}\right],\check{P}_{2}\left(z_{2}\right)&=&\esp\left[z_{2}^{Y_{2}\left(t\right)}\right]$, y $\hat{P}_{i}\left(w_{i}\right)&=&\esp\left[w_{i}^{\hat{X}_{i}\left(t\right)}\right]$.
\item En lo que respecta al servidor, en t\'erminos de los tiempos de
visita a cada una de las colas, se definen las variables
aleatorias $\tau_{1},\tau_{2}$ para $Q_{1},Q_{2}$ respectivamente;
y $\zeta_{1},\zeta_{2}$ para $\hat{Q}_{1},\hat{Q}_{2}$ del sistema
2. \item A los tiempos en que el servidor termina de atender en las
colas $Q_{1},Q_{2},\hat{Q}_{1},\hat{Q}_{2}$, se les denotar\'a por
$\overline{\tau}_{1},\overline{\tau}_{2},\overline{\zeta}_{1},\overline{\zeta}_{2}$
respectivamente.
\item Los tiempos de traslado del servidor desde el
momento en que termina de atender a una cola y llega a la
siguiente para comenzar a dar servicio est\'an dados por
$\tau_{2}-\overline{\tau}_{1},\tau_{1}-\overline{\tau}_{2}$ y
$\zeta_{2}-\overline{\zeta}_{1},\zeta_{1}-\overline{\zeta}_{2}$
para el sistema 1 y el sistema 2, respectivamente.
\end{itemize}
Cada uno de estos procesos con su respectiva FGP. Adem\'as, para cada una de las colas en cada sistema, el n\'umero de usuarios al tiempo en que llega el servidor a dar servicio est\'a
dado por el n\'umero de usuarios presentes en la cola al tiempo
$t$, m\'as el n\'umero de usuarios que llegan a la cola en el intervalo de tiempo
$\left[\tau_{i},\overline{\tau}_{i}\right]$, es decir
{\small{
\begin{eqnarray*}
L_{1}\left(\overline{\tau}_{1}\right)=L_{1}\left(\tau_{1}\right)+X_{1}\left(\overline{\tau}_{1}-\tau_{1}\right),\hat{L}_{i}\left(\overline{\tau}_{i}\right)=\hat{L}_{i}\left(\tau_{i}\right)+\hat{X}_{i}\left(\overline{\tau}_{i}-\tau_{i}\right),L_{2}\left(\overline{\tau}_{1}\right)=L_{2}\left(\tau_{1}\right)+X_{2}\left(\overline{\tau}_{1}-\tau_{1}\right)+Y_{2}\left(\overline{\tau}_{1}-\tau_{1}\right),
\end{eqnarray*}}}




%\begin{center}\vspace{1cm}
%%%%\includegraphics[width=0.6\linewidth]{RedSVC2}
%\captionof{figure}{\color{Green} Red de Sistema de Visitas C\'iclicas}
%\end{center}\vspace{1cm}




Una vez definidas las Funciones Generadoras de Probabilidades Conjuntas se construyen las ecuaciones recursivas que permiten obtener la información sobre la longitud de cada una de las colas, al momento en que uno de los servidores llega a una de las colas para dar servicio, bas\'andose en la informaci\'on que se tiene sobre su llegada a la cola inmediata anterior.\\
{\footnotesize{
\begin{eqnarray*}
F_{2}\left(z_{1},z_{2},w_{1},w_{2}\right)&=&R_{1}\left(P_{1}\left(z_{1}\right)\tilde{P}_{2}\left(z_{2}\right)\prod_{i=1}^{2}
\hat{P}_{i}\left(w_{i}\right)\right)F_{1}\left(\theta_{1}\left(\tilde{P}_{2}\left(z_{2}\right)\hat{P}_{1}\left(w_{1}\right)\hat{P}_{2}\left(w_{2}\right)\right),z_{2},w_{1},w_{2}\right),\\
F_{1}\left(z_{1},z_{2},w_{1},w_{2}\right)&=&R_{2}\left(P_{1}\left(z_{1}\right)\tilde{P}_{2}\left(z_{2}\right)\prod_{i=1}^{2}
\hat{P}_{i}\left(w_{i}\right)\right)F_{2}\left(z_{1},\tilde{\theta}_{2}\left(P_{1}\left(z_{1}\right)\hat{P}_{1}\left(w_{1}\right)\hat{P}_{2}\left(w_{2}\right)\right),w_{1},w_{2}\right),\\
\hat{F}_{2}\left(z_{1},z_{2},w_{1},w_{2}\right)&=&\hat{R}_{1}\left(P_{1}\left(z_{1}\right)\tilde{P}_{2}\left(z_{2}\right)\prod_{i=1}^{2}
\hat{P}_{i}\left(w_{i}\right)\right)\hat{F}_{1}\left(z_{1},z_{2},\hat{\theta}_{1}\left(P_{1}\left(z_{1}\right)\tilde{P}_{2}\left(z_{2}\right)\hat{P}_{2}\left(w_{2}\right)\right),w_{2}\right),\\
%\end{eqnarray*}}}
%{\small{
%\begin{eqnarray*}
\hat{F}_{1}\left(z_{1},z_{2},w_{1},w_{2}\right)&=&\hat{R}_{2}\left(P_{1}\left(z_{1}\right)\tilde{P}_{2}\left(z_{2}\right)\prod_{i=1}^{2}
\hat{P}_{i}\left(w_{i}\right)\right)\hat{F}_{2}\left(z_{1},z_{2},w_{1},\hat{\theta}_{2}\left(P_{1}\left(z_{1}\right)\tilde{P}_{2}\left(z_{2}\right)\hat{P}_{1}\left(w_{1}\right)\right)\right).
\end{eqnarray*}}}


%------------------------------------------------
%\subsection*{Descripci\'on de la Red de Sistemas de Visitas C\'iclicas}
%------------------------------------------------

%----------------------------------------------------------------------------------------
%   RESULTS
%----------------------------------------------------------------------------------------
\section*{Resultado Principal}
%----------------------------------------------------------------------------------------
Sean $\mu_{1},\mu_{2},\check{\mu}_{2},\hat{\mu}_{1},\hat{\mu}_{2}$ y $\tilde{\mu}_{2}=\mu_{2}+\check{\mu}_{2}$ los valores esperados de los proceso definidos anteriormente, y sean $r_{1},r_{2}, \hat{r}_{1}$ y $\hat{r}_{2}$ los valores esperado s de los tiempos de traslado del servidor entre las colas para cada uno de los sistemas de visitas c\'iclicas. Si se definen $\mu=\mu_{1}+\tilde{\mu}_{2}$, $\hat{\mu}=\hat{\mu}_{1}+\hat{\mu}_{2}$, y $r=r_{1}+r_{2}$ y  $\hat{r}=\hat{r}_{1}+\hat{r}_{2}$, entonces se tiene el siguiente resultado.

\begin{Teo}
Supongamos que $\mu<1$, $\hat{\mu}<1$, entonces, el n\'umero de usuarios presentes en cada una de las colas que conforman la Red de Sistemas de Visitas C\'iclicas cuando uno de los servidores visita a alguna de ellas est\'a dada por la soluci\'on del Sistema de Ecuaciones Lineales presentados arriba cuyas expresiones damos a continuaci\'on:
%{\footnotesize{
\[ \begin{array}{lll}
f_{1}\left(1\right)=r\frac{\mu_{1}\left(1-\mu_{1}\right)}{1-\mu},&f_{1}\left(2\right)=r_{2}\tilde{\mu}_{2},&f_{1}\left(3\right)=\hat{\mu}_{1}\left(\frac{r_{2}\mu_{2}+1}{\mu_{2}}+r\frac{\tilde{\mu}_{2}}{1-\mu}\right),\\
f_{1}\left(4\right)=\hat{\mu}_{2}\left(\frac{r_{2}\mu_{2}+1}{\mu_{2}}+r\frac{\tilde{\mu}_{2}}{1-\mu}\right),&f_{2}\left(1\right)=r_{1}\mu_{1},&f_{2}\left(2\right)=r\frac{\tilde{\mu}_{2}\left(1-\tilde{\mu}_{2}\right)}{1-\mu},\\
f_{2}\left(3\right)=\hat{\mu}_{1}\left(\frac{r_{1}\mu_{1}+1}{\mu_{1}}+r\frac{\mu_{1}}{1-\mu}\right),&f_{2}\left(4\right)=\hat{\mu}_{2}\left(\frac{r_{1}\mu_{1}+1}{\mu_{1}}+r\frac{\mu_{1}}{1-\mu}\right),&\hat{f}_{1}\left(1\right)=\mu_{1}\left(\frac{\hat{r}_{2}\hat{\mu}_{2}+1}{\hat{\mu}_{2}}+\hat{r}\frac{\hat{\mu}_{2}}{1-\hat{\mu}}\right),\\
\hat{f}_{1}\left(2\right)=\tilde{\mu}_{2}\left(\hat{r}_{2}+\hat{r}\frac{\hat{\mu}_{2}}{1-\hat{\mu}}\right)+\frac{\mu_{2}}{\hat{\mu}_{2}},&\hat{f}_{1}\left(3\right)=\hat{r}\frac{\hat{\mu}_{1}\left(1-\hat{\mu}_{1}\right)}{1-\hat{\mu}},&\hat{f}_{1}\left(4\right)=\hat{r}_{2}\hat{\mu}_{2},\\
\hat{f}_{2}\left(1\right)=\mu_{1}\left(\frac{\hat{r}_{1}\hat{\mu}_{1}+1}{\hat{\mu}_{1}}+\hat{r}\frac{\hat{\mu}_{1}}{1-\hat{\mu}}\right),&\hat{f}_{2}\left(2\right)=\tilde{\mu}_{2}\left(\hat{r}_{1}+\hat{r}\frac{\hat{\mu}_{1}}{1-\hat{\mu}}\right)+\frac{\hat{\mu_{2}}}{\hat{\mu}_{1}},&\hat{f}_{2}\left(3\right)=\hat{r}_{1}\hat{\mu}_{1},\\
&\hat{f}_{2}\left(4\right)=\hat{r}\frac{\hat{\mu}_{2}\left(1-\hat{\mu}_{2}\right)}{1-\hat{\mu}}.&\\
\end{array}\] %}}
\end{Teo}


Las ecuaciones que determinan los segundos momentos de las longitudes de las colas de los dos sistemas se pueden ver en \href{http://sitio.expresauacm.org/s/carlosmartinez/wp-content/uploads/sites/13/2014/01/SegundosMomentos.pdf}{este sitio}

%\url{http://ubuntu_es_el_diablo.org},\href{http://www.latex-project.org/}{latex project}

%http://sitio.expresauacm.org/s/carlosmartinez/wp-content/uploads/sites/13/2014/01/SegundosMomentos.jpg
%http://sitio.expresauacm.org/s/carlosmartinez/wp-content/uploads/sites/13/2014/01/SegundosMomentos.pdf




%___________________________________________________________________________________________
%\section*{Tiempos de Ciclo e Intervisita}
%___________________________________________________________________________________________



%----------------------------------------------------------------------------------------
%\section*{Medidas de Desempe\~no de la Red de Sistemas de Visita C\'iclicas}
%----------------------------------------------------------------------------------------
%Se puede demostrar que las expresiones para los tiempos entre visitas de los servidores a las colas

%----------------------------------------------------------------------------------------
%   CONCLUSIONS
%----------------------------------------------------------------------------------------

%\color{SaddleBrown} % SaddleBrown color for the conclusions to make them stand out

\section*{Medidas de Desempe\~no}


\begin{Def}
Sea $L_{i}^{*}$el n\'umero de usuarios cuando el servidor visita la cola $Q_{i}$ para dar servicio, para $i=1,2$.
\end{Def}

Entonces
\begin{Prop} Para la cola $Q_{i}$, $i=1,2$, se tiene que el n\'umero de usuarios presentes al momento de ser visitada por el servidor est\'a dado por
\begin{eqnarray}
\esp\left[L_{i}^{*}\right]&=&f_{i}\left(i\right)\\
Var\left[L_{i}^{*}\right]&=&f_{i}\left(i,i\right)+\esp\left[L_{i}^{*}\right]-\esp\left[L_{i}^{*}\right]^{2}.
\end{eqnarray}
\end{Prop}


\begin{Def}
El tiempo de Ciclo $C_{i}$ es el periodo de tiempo que comienza
cuando la cola $i$ es visitada por primera vez en un ciclo, y
termina cuando es visitado nuevamente en el pr\'oximo ciclo, bajo condiciones de estabilidad.

\begin{eqnarray*}
C_{i}\left(z\right)=\esp\left[z^{\overline{\tau}_{i}\left(m+1\right)-\overline{\tau}_{i}\left(m\right)}\right]
\end{eqnarray*}
\end{Def}

\begin{Def}
El tiempo de intervisita $I_{i}$ es el periodo de tiempo que
comienza cuando se ha completado el servicio en un ciclo y termina
cuando es visitada nuevamente en el pr\'oximo ciclo.
\begin{eqnarray*}I_{i}\left(z\right)&=&\esp\left[z^{\tau_{i}\left(m+1\right)-\overline{\tau}_{i}\left(m\right)}\right]\end{eqnarray*}
\end{Def}

\begin{Prop}
Para los tiempos de intervisita del servidor $I_{i}$, se tiene que

\begin{eqnarray*}
\esp\left[I_{i}\right]&=&\frac{f_{i}\left(i\right)}{\mu_{i}},\\
Var\left[I_{i}\right]&=&\frac{Var\left[L_{i}^{*}\right]}{\mu_{i}^{2}}-\frac{\sigma_{i}^{2}}{\mu_{i}^{2}}f_{i}\left(i\right).
\end{eqnarray*}
\end{Prop}


\begin{Prop}
Para los tiempos que ocupa el servidor para atender a los usuarios presentes en la cola $Q_{i}$, con FGP denotada por $S_{i}$, se tiene que
\begin{eqnarray*}
\esp\left[S_{i}\right]&=&\frac{\esp\left[L_{i}^{*}\right]}{1-\mu_{i}}=\frac{f_{i}\left(i\right)}{1-\mu_{i}},\\
Var\left[S_{i}\right]&=&\frac{Var\left[L_{i}^{*}\right]}{\left(1-\mu_{i}\right)^{2}}+\frac{\sigma^{2}\esp\left[L_{i}^{*}\right]}{\left(1-\mu_{i}\right)^{3}}
\end{eqnarray*}
\end{Prop}


\begin{Prop}
Para la duraci\'on de los ciclos $C_{i}$ se tiene que
\begin{eqnarray*}
\esp\left[C_{i}\right]&=&\esp\left[I_{i}\right]\esp\left[\theta_{i}\left(z\right)\right]=\frac{\esp\left[L_{i}^{*}\right]}{\mu_{i}}\frac{1}{1-\mu_{i}}=\frac{f_{i}\left(i\right)}{\mu_{i}\left(1-\mu_{i}\right)}\\
Var\left[C_{i}\right]&=&\frac{Var\left[L_{i}^{*}\right]}{\mu_{i}^{2}\left(1-\mu_{i}\right)^{2}}.
\end{eqnarray*}

\end{Prop}


%----------------------------------------------------------------------------------------
%   REFERENCES
%----------------------------------------------------------------------------------------
%_________________________________________________________________________
%\section*{REFERENCIAS}
%_________________________________________________________________________
\color{DarkSlateBlue} % DarkSlateGray color for the rest of the content
\section*{Conjeturas}
%----------------------------------------------------------------------------------------

\begin{Def}
Dada una cola $Q_{i}$, sea $\mathcal{L}=\left\{L_{1}\left(t\right),L_{2}\left(t\right),\hat{L}_{1}\left(t\right),\hat{L}_{2}\left(t\right)\right\}$ las longitudes de todas las colas de la Red de Sistemas de Visitas C\'iclicas. Sup\'ongase que el servidor visita $Q_{i}$, si $L_{i}\left(t\right)=0$ y $\hat{L}_{i}\left(t\right)=0$ para $i=1,2$, entonces los elementos de $\mathcal{L}$ son puntos regenerativos.
\end{Def}


\begin{Def}
Un ciclo regenerativo es el intervalo de tiempo que ocurre entre dos puntos regenerativos sucesivos, $\mathcal{L}_{1},\mathcal{L}_{2}$.
\end{Def}


Def\'inanse los puntos de regenaraci\'on  en el proceso
$\left[L_{1}\left(t\right),L_{2}\left(t\right),\ldots,L_{N}\left(t\right)\right]$.
Los puntos cuando la cola $i$ es visitada y todos los
$L_{j}\left(\tau_{i}\left(m\right)\right)=0$ para $i=1,2$  son
puntos de regeneraci\'on. Se llama ciclo regenerativo al intervalo
entre dos puntos regenerativos sucesivos.

Sea $M_{i}$  el n\'umero de ciclos de visita en un ciclo
regenerativo, y sea $C_{i}^{(m)}$, para $m=1,2,\ldots,M_{i}$ la
duraci\'on del $m$-\'esimo ciclo de visita en un ciclo
regenerativo. Se define el ciclo del tiempo de visita promedio
$\esp\left[C_{i}\right]$ como
\begin{eqnarray*}
\esp\left[C_{i}\right]&=&\frac{\esp\left[\sum_{m=1}^{M_{i}}C_{i}^{(m)}\right]}{\esp\left[M_{i}\right]}
\end{eqnarray*}


En Stid72 y Heym82 se muestra que una condici\'on suficiente para
que el proceso regenerativo estacionario sea un procesoo
estacionario es que el valor esperado del tiempo del ciclo
regenerativo sea finito:

\begin{eqnarray*}
\esp\left[\sum_{m=1}^{M_{i}}C_{i}^{(m)}\right]<\infty.
\end{eqnarray*}



como cada $C_{i}^{(m)}$ contiene intervalos de r\'eplica
positivos, se tiene que $\esp\left[M_{i}\right]<\infty$, adem\'as,
como $M_{i}>0$, se tiene que la condici\'on anterior es
equivalente a tener que

\begin{eqnarray*}
\esp\left[C_{i}\right]<\infty,
\end{eqnarray*}
por lo tanto una condici\'on suficiente para la existencia del
proceso regenerativo est\'a dada por
\begin{eqnarray*}
\sum_{k=1}^{N}\mu_{k}<1.
\end{eqnarray*}



Sea la funci\'on generadora de momentos para $L_{i}$, el n\'umero
de usuarios en la cola $Q_{i}\left(z\right)$ en cualquier momento,
est\'a dada por el tiempo promedio de $z^{L_{i}\left(t\right)}$
sobre el ciclo regenerativo definido anteriormente:

\begin{eqnarray*}
Q_{i}\left(z\right)&=&\esp\left[z^{L_{i}\left(t\right)}\right]=\frac{\esp\left[\sum_{m=1}^{M_{i}}\sum_{t=\tau_{i}\left(m\right)}^{\tau_{i}\left(m+1\right)-1}z^{L_{i}\left(t\right)}\right]}{\esp\left[\sum_{m=1}^{M_{i}}\tau_{i}\left(m+1\right)-\tau_{i}\left(m\right)\right]}
\end{eqnarray*}


$M_{i}$ es un tiempo de paro en el proceso regenerativo con
$\esp\left[M_{i}\right]<\infty$, se sigue del lema de Wald que:


\begin{eqnarray*}
\esp\left[\sum_{m=1}^{M_{i}}\sum_{t=\tau_{i}\left(m\right)}^{\tau_{i}\left(m+1\right)-1}z^{L_{i}\left(t\right)}\right]&=&\esp\left[M_{i}\right]\esp\left[\sum_{t=\tau_{i}\left(m\right)}^{\tau_{i}\left(m+1\right)-1}z^{L_{i}\left(t\right)}\right]\\
\esp\left[\sum_{m=1}^{M_{i}}\tau_{i}\left(m+1\right)-\tau_{i}\left(m\right)\right]&=&\esp\left[M_{i}\right]\esp\left[\tau_{i}\left(m+1\right)-\tau_{i}\left(m\right)\right]
\end{eqnarray*}

por tanto se tiene que


\begin{eqnarray*}
Q_{i}\left(z\right)&=&\frac{\esp\left[\sum_{t=\tau_{i}\left(m\right)}^{\tau_{i}\left(m+1\right)-1}z^{L_{i}\left(t\right)}\right]}{\esp\left[\tau_{i}\left(m+1\right)-\tau_{i}\left(m\right)\right]}
\end{eqnarray*}

observar que el denominador es simplemente la duraci\'on promedio
del tiempo del ciclo.




Se puede demostrar (ver Hideaki Takagi 1986) que

\begin{eqnarray*}
\esp\left[\sum_{t=\tau_{i}\left(m\right)}^{\tau_{i}\left(m+1\right)-1}z^{L_{i}\left(t\right)}\right]=z\frac{F_{i}\left(z\right)-1}{z-P_{i}\left(z\right)}
\end{eqnarray*}

Durante el tiempo de intervisita para la cola $i$,
$L_{i}\left(t\right)$ solamente se incrementa de manera que el
incremento por intervalo de tiempo est\'a dado por la funci\'on
generadora de probabilidades de $P_{i}\left(z\right)$, por tanto
la suma sobre el tiempo de intervisita puede evaluarse como:

\begin{eqnarray*}
\esp\left[\sum_{t=\tau_{i}\left(m\right)}^{\tau_{i}\left(m+1\right)-1}z^{L_{i}\left(t\right)}\right]&=&\esp\left[\sum_{t=\tau_{i}\left(m\right)}^{\tau_{i}\left(m+1\right)-1}\left\{P_{i}\left(z\right)\right\}^{t-\overline{\tau}_{i}\left(m\right)}\right]\\
&=&\frac{1-\esp\left[\left\{P_{i}\left(z\right)\right\}^{\tau_{i}\left(m+1\right)-\overline{\tau}_{i}\left(m\right)}\right]}{1-P_{i}\left(z\right)}=\frac{1-I_{i}\left[P_{i}\left(z\right)\right]}{1-P_{i}\left(z\right)}
\end{eqnarray*}
por tanto



\begin{eqnarray*}
\esp\left[\sum_{t=\tau_{i}\left(m\right)}^{\tau_{i}\left(m+1\right)-1}z^{L_{i}\left(t\right)}\right]&=&\frac{1-F_{i}\left(z\right)}{1-P_{i}\left(z\right)}
\end{eqnarray*}


Haciendo uso de lo hasta ahora desarrollado se tiene que

\begin{eqnarray*}
Q_{i}\left(z\right)&=&\frac{1}{\esp\left[C_{i}\right]}\cdot\frac{1-F_{i}\left(z\right)}{P_{i}\left(z\right)-z}\cdot\frac{\left(1-z\right)P_{i}\left(z\right)}{1-P_{i}\left(z\right)}\\
&=&\frac{\mu_{i}\left(1-\mu_{i}\right)}{f_{i}\left(i\right)}\cdot\frac{1-F_{i}\left(z\right)}{P_{i}\left(z\right)-z}\cdot\frac{\left(1-z\right)P_{i}\left(z\right)}{1-P_{i}\left(z\right)}
\end{eqnarray*}

derivando con respecto a $z$




\begin{eqnarray*}
\frac{d Q_{i}\left(z\right)}{d z}&=&\frac{\left(1-F_{i}\left(z\right)\right)P_{i}\left(z\right)}{\esp\left[C_{i}\right]\left(1-P_{i}\left(z\right)\right)\left(P_{i}\left(z\right)-z\right)}\\
&-&\frac{\left(1-z\right)P_{i}\left(z\right)F_{i}^{'}\left(z\right)}{\esp\left[C_{i}\right]\left(1-P_{i}\left(z\right)\right)\left(P_{i}\left(z\right)-z\right)}\\
&-&\frac{\left(1-z\right)\left(1-F_{i}\left(z\right)\right)P_{i}\left(z\right)\left(P_{i}^{'}\left(z\right)-1\right)}{\esp\left[C_{i}\right]\left(1-P_{i}\left(z\right)\right)\left(P_{i}\left(z\right)-z\right)^{2}}\\
&+&\frac{\left(1-z\right)\left(1-F_{i}\left(z\right)\right)P_{i}^{'}\left(z\right)}{\esp\left[C_{i}\right]\left(1-P_{i}\left(z\right)\right)\left(P_{i}\left(z\right)-z\right)}\\
&+&\frac{\left(1-z\right)\left(1-F_{i}\left(z\right)\right)P_{i}\left(z\right)P_{i}^{'}\left(z\right)}{\esp\left[C_{i}\right]\left(1-P_{i}\left(z\right)\right)^{2}\left(P_{i}\left(z\right)-z\right)}
\end{eqnarray*}

%______________________________________________________



Calculando el l\'imite cuando $z\rightarrow1^{+}$:
\begin{eqnarray}
Q_{i}^{(1)}\left(z\right)&=&lim_{z\rightarrow1^{+}}\frac{d Q_{i}\left(z\right)}{dz}\\
&=&lim_{z\rightarrow1}\frac{\left(1-F_{i}\left(z\right)\right)P_{i}\left(z\right)}{\esp\left[C_{i}\right]\left(1-P_{i}\left(z\right)\right)\left(P_{i}\left(z\right)-z\right)}\\
&-&lim_{z\rightarrow1^{+}}\frac{\left(1-z\right)P_{i}\left(z\right)F_{i}^{'}\left(z\right)}{\esp\left[C_{i}\right]\left(1-P_{i}\left(z\right)\right)\left(P_{i}\left(z\right)-z\right)}\\
&-&lim_{z\rightarrow1^{+}}\frac{\left(1-z\right)\left(1-F_{i}\left(z\right)\right)P_{i}\left(z\right)\left(P_{i}^{'}\left(z\right)-1\right)}{\esp\left[C_{i}\right]\left(1-P_{i}\left(z\right)\right)\left(P_{i}\left(z\right)-z\right)^{2}}\\
&+&lim_{z\rightarrow1^{+}}\frac{\left(1-z\right)\left(1-F_{i}\left(z\right)\right)P_{i}^{'}\left(z\right)}{\esp\left[C_{i}\right]\left(1-P_{i}\left(z\right)\right)\left(P_{i}\left(z\right)-z\right)}\\
&+&lim_{z\rightarrow1^{+}}\frac{\left(1-z\right)\left(1-F_{i}\left(nz\right)\right)P_{i}\left(z\right)P_{i}^{'}\left(z\right)}{\esp\left[C_{i}\right]\left(1-P_{i}\left(z\right)\right)^{2}\left(P_{i}\left(z\right)-z\right)}
\end{eqnarray}

Entonces:



\begin{eqnarray*}
&&lim_{z\rightarrow1^{+}}\frac{\left(1-F_{i}\left(z\right)\right)P_{i}\left(z\right)}{\left(1-P_{i}\left(z\right)\right)\left(P_{i}\left(z\right)-z\right)}=lim_{z\rightarrow1^{+}}\frac{\frac{d}{dz}\left[\left(1-F_{i}\left(z\right)\right)P_{i}\left(z\right)\right]}{\frac{d}{dz}\left[\left(1-P_{i}\left(z\right)\right)\left(-z+P_{i}\left(z\right)\right)\right]}\\
&=&lim_{z\rightarrow1^{+}}\frac{-P_{i}\left(z\right)F_{i}^{'}\left(z\right)+\left(1-F_{i}\left(z\right)\right)P_{i}^{'}\left(z\right)}{\left(1-P_{i}\left(z\right)\right)\left(-1+P_{i}^{'}\left(z\right)\right)-\left(-z+P_{i}\left(z\right)\right)P_{i}^{'}\left(z\right)}
\end{eqnarray*}


\begin{eqnarray*}
&&lim_{z\rightarrow1^{+}}\frac{\left(1-z\right)P_{i}\left(z\right)F_{i}^{'}\left(z\right)}{\left(1-P_{i}\left(z\right)\right)\left(P_{i}\left(z\right)-z\right)}=lim_{z\rightarrow1^{+}}\frac{\frac{d}{dz}\left[\left(1-z\right)P_{i}\left(z\right)F_{i}^{'}\left(z\right)\right]}{\frac{d}{dz}\left[\left(1-P_{i}\left(z\right)\right)\left(P_{i}\left(z\right)-z\right)\right]}\\
&=&lim_{z\rightarrow1^{+}}\frac{-P_{i}\left(z\right)
F_{i}^{'}\left(z\right)+(1-z) F_{i}^{'}\left(z\right)
P_{i}^{'}\left(z\right)+(1-z)
P_{i}\left(z\right)F_{i}^{''}\left(z\right)}{\left(1-P_{i}\left(z\right)\right)\left(-1+P_{i}^{'}\left(z\right)\right)-\left(-z+P_{i}\left(z\right)\right)P_{i}^{'}\left(z\right)}
\end{eqnarray*}

\footnotesize{
\begin{eqnarray*}
&&lim_{z\rightarrow1^{+}}\frac{\left(1-z\right)\left(1-F_{i}\left(z\right)\right)P_{i}\left(z\right)\left(P_{i}^{'}\left(z\right)-1\right)}{\left(1-P_{i}\left(z\right)\right)\left(P_{i}\left(z\right)-z\right)^{2}}\\
&=&lim_{z\rightarrow1^{+}}\frac{\frac{d}{dz}\left[\left(1-z\right)\left(1-F_{i}\left(z\right)\right)P_{i}\left(z\right)\left(P_{i}^{'}\left(z\right)-1\right)\right]}{\frac{d}{dz}\left[\left(1-P_{i}\left(z\right)\right)\left(P_{i}\left(z\right)-z\right)^{2}\right]}\\
&=&lim_{z\rightarrow1^{+}}\frac{-\left(1-F_{i}\left(z\right)\right) P_{i}\left(z\right)\left(-1+P_{i}^{'}\left(z\right)\right)-(1-z) P_{i}\left(z\right)F_{i}^{'}\left(z\right)\left(-1+P_{i}^{'}\left(z\right)\right)}{2\left(1-P_{i}\left(z\right)\right)\left(-z+P_{i}\left(z\right)\right) \left(-1+P_{i}^{'}\left(z\right)\right)-\left(-z+P_{i}\left(z\right)\right)^2 P_{i}^{'}\left(z\right)}\\
&+&lim_{z\rightarrow1^{+}}\frac{+(1-z) \left(1-F_{i}\left(z\right)\right) \left(-1+P_{i}^{'}\left(z\right)\right) P_{i}^{'}\left(z\right)}{{2\left(1-P_{i}\left(z\right)\right)\left(-z+P_{i}\left(z\right)\right) \left(-1+P_{i}^{'}\left(z\right)\right)-\left(-z+P_{i}\left(z\right)\right)^2 P_{i}^{'}\left(z\right)}}\\
&+&lim_{z\rightarrow1^{+}}\frac{+(1-z)
\left(1-F_{i}\left(z\right)\right)
P_{i}\left(z\right)P_{i}^{''}\left(z\right)}{{2\left(1-P_{i}\left(z\right)\right)\left(-z+P_{i}\left(z\right)\right)
\left(-1+P_{i}^{'}\left(z\right)\right)-\left(-z+P_{i}\left(z\right)\right)^2
P_{i}^{'}\left(z\right)}}
\end{eqnarray*}}

\footnotesize{
%______________________________________________________
\begin{eqnarray*}
&&lim_{z\rightarrow1^{+}}\frac{\left(1-z\right)\left(1-F_{i}\left(z\right)\right)P_{i}^{'}\left(z\right)}{\left(1-P_{i}\left(z\right)\right)\left(P_{i}\left(z\right)-z\right)}=lim_{z\rightarrow1^{+}}\frac{\frac{d}{dz}\left[\left(1-z\right)\left(1-F_{i}\left(z\right)\right)P_{i}^{'}\left(z\right)\right]}{\frac{d}{dz}\left[\left(1-P_{i}\left(z\right)\right)\left(P_{i}\left(z\right)-z\right)\right]}\\
&=&lim_{z\rightarrow1^{+}}\frac{-\left(1-F_{i}\left(z\right)\right)
P_{i}^{'}\left(z\right)-(1-z) F_{i}^{'}\left(z\right)
P_{i}^{'}\left(z\right)+(1-z) \left(1-F_{i}\left(z\right)\right)
P_{i}^{''}\left(z\right)}{\left(1-P_{i}\left(z\right)\right)
\left(-1+P_{i}^{'}\left(z\right)\right)-\left(-z+P_{i}\left(z\right)\right)
P_{i}^{'}\left(z\right)}\frac{}{}
\end{eqnarray*}}

\footnotesize{

%______________________________________________________
\begin{eqnarray*}
&&lim_{z\rightarrow1^{+}}\frac{\left(1-z\right)\left(1-F_{i}\left(z\right)\right)P_{i}\left(z\right)P_{i}^{'}\left(z\right)}{\left(1-P_{i}\left(z\right)\right)^{2}\left(P_{i}\left(z\right)-z\right)}\\
&=&lim_{z\rightarrow1^{+}}\frac{\frac{d}{dz}\left[\left(1-z\right)\left(1-F_{i}\left(z\right)\right)P_{i}\left(z\right)P_{i}^{'}\left(z\right)\right]}{\frac{d}{dz}\left[\left(1-P_{i}\left(z\right)\right)^{2}\left(P_{i}\left(z\right)-z\right)\right]}\\
&=&lim_{z\rightarrow1^{+}}\frac{-\left(1-F_{i}\left(z\right)\right) P_{i}\left(z\right) P_{i}^{'}\left(z\right)-(1-z) P_{i}\left(z\right) F_{i}^{'}\left(z\right)P_i'[z]}{\left(1-P_{i}\left(z\right)\right)^2 \left(-1+P_{i}^{'}\left(z\right)\right)-2 \left(1-P_{i}\left(z\right)\right) \left(-z+P_{i}\left(z\right)\right) P_{i}^{'}\left(z\right)}\\
&+&lim_{z\rightarrow1^{+}}\frac{(1-z) \left(1-F_{i}\left(z\right)\right) P_{i}^{'}\left(z\right)^2+(1-z) \left(1-F_{i}\left(z\right)\right) P_{i}\left(z\right) P_{i}^{''}\left(z\right)}{\left(1-P_{i}\left(z\right)\right)^2 \left(-1+P_{i}^{'}\left(z\right)\right)-2 \left(1-P_{i}\left(z\right)\right) \left(-z+P_{i}\left(z\right)\right) P_{i}^{'}\left(z\right)}\\
\end{eqnarray*}}



%___________________________________________________________________________________________
\subsection{Longitudes de la Cola en cualquier tiempo}
%___________________________________________________________________________________________



Sea
$V_{i}\left(z\right)=\frac{1}{\esp\left[C_{i}\right]}\frac{I_{i}\left(z\right)-1}{z-P_{i}\left(z\right)}$

%{\esp\lef[I_{i}\right]}\frac{1-\mu_{i}}{z-P_{i}\left(z\right)}

\begin{eqnarray*}
\frac{\partial V_{i}\left(z\right)}{\partial
z}&=&\frac{1}{\esp\left[C_{i}\right]}\left[\frac{I_{i}{'}\left(z\right)\left(z-P_{i}\left(z\right)\right)}{z-P_{i}\left(z\right)}-\frac{\left(I_{i}\left(z\right)-1\right)\left(1-P_{i}{'}\left(z\right)\right)}{\left(z-P_{i}\left(z\right)\right)^{2}}\right]
\end{eqnarray*}


La FGP para el tiempo de espera para cualquier usuario en la cola
est\'a dada por:
\[U_{i}\left(z\right)=\frac{1}{\esp\left[C_{i}\right]}\cdot\frac{1-P_{i}\left(z\right)}{z-P_{i}\left(z\right)}\cdot\frac{I_{i}\left(z\right)-1}{1-z}\]

entonces
%\frac{I_{i}\left(z\right)-1}{1-z}
%+\frac{1-P_{i}\left(z\right)}{z-P_{i}\frac{d}{dz}\left(\frac{I_{i}\left(z\right)-1}{1-z}\right)


\footnotesize{
\begin{eqnarray*}
\frac{d}{dz}V_{i}\left(z\right)&=&\frac{1}{\esp\left[C_{i}\right]}\left\{\frac{d}{dz}\left(\frac{1-P_{i}\left(z\right)}{z-P_{i}\left(z\right)}\right)\frac{I_{i}\left(z\right)-1}{1-z}+\frac{1-P_{i}\left(z\right)}{z-P_{i}\left(z\right)}\frac{d}{dz}\left(\frac{I_{i}\left(z\right)-1}{1-z}\right)\right\}\\
&=&\frac{1}{\esp\left[C_{i}\right]}\left\{\frac{-P_{i}\left(z\right)\left(z-P_{i}\left(z\right)\right)-\left(1-P_{i}\left(z\right)\right)\left(1-P_{i}^{'}\left(z\right)\right)}{\left(z-P_{i}\left(z\right)\right)^{2}}\cdot\frac{I_{i}\left(z\right)-1}{1-z}\right\}\\
&+&\frac{1}{\esp\left[C_{i}\right]}\left\{\frac{1-P_{i}\left(z\right)}{z-P_{i}\left(z\right)}\cdot\frac{I_{i}^{'}\left(z\right)\left(1-z\right)+\left(I_{i}\left(z\right)-1\right)}{\left(1-z\right)^{2}}\right\}
\end{eqnarray*}}
\begin{eqnarray*}
\frac{\partial U_{i}\left(z\right)}{\partial z}&=&\frac{(-1+I_{i}[z]) (1-P_{i}[z])}{(1-z)^2 \esp[I_{i}] (z-P_{i}[z])}+\frac{(1-P_{i}[z]) I_{i}^{'}[z]}{(1-z) \esp[I_{i}] (z-P_{i}[z])}\\
&-&\frac{(-1+I_{i}[z]) (1-P_{i}[z])\left(1-P{'}[z]\right)}{(1-z) \esp[I_{i}] (z-P_{i}[z])^2}-\frac{(-1+I_{i}[z]) P_{i}{'}[z]}{(1-z) \esp[I_{i}](z-P_{i}[z])}
\end{eqnarray*}





%__________________________________________________________________________
%\subsection{Definiciones}
%__________________________________________________________________________


\section{Descripci\'on de una Red de Sistemas de Visitas C\'iclicas}

Consideremos una red de sistema de visitas c\'iclicas conformada por dos sistemas de visitas c\'iclicas, cada una con dos colas independientes, donde adem\'as se permite el intercambio de usuarios entre los dos sistemas en la segunda cola de cada uno de ellos.

%____________________________________________________________________
\subsection*{Supuestos sobe la Red de Sistemas de Visitas C\'iclicas}
%____________________________________________________________________

\begin{itemize}
\item Los arribos de los usuarios ocurren
conforme a un proceso Poisson con tasa de llegada $\mu_{1}$ y
$\mu_{2}$ para el sistema 1, mientras que para el sistema 2,
lo hacen conforme a un proceso Poisson con tasa
$\hat{\mu}_{1},\hat{\mu}_{2}$ respectivamente.



\item Se considerar\'an intervalos de tiempo de la forma
$\left[t,t+1\right]$. Los usuarios arriban por paquetes de manera
independiente del resto de las colas. Se define el grupo de
usuarios que llegan a cada una de las colas del sistema 1,
caracterizadas por $Q_{1}$ y $Q_{2}$ respectivamente, en el
intervalo de tiempo $\left[t,t+1\right]$ por
$X_{1}\left(t\right),X_{2}\left(t\right)$.


\item Se definen los procesos
$\hat{X}_{1}\left(t\right),\hat{X}_{2}\left(t\right)$ para las
colas del sistema 2, denotadas por $\hat{Q}_{1}$ y $\hat{Q}_{2}$
respectivamente. Donde adem\'as se supone que $\mu_{i}<1$ y $\hat{\mu}<1$ para $i=1,2$.


\item Se define el proceso
$Y_{2}\left(t\right)$ para el n\'umero de usuarios que se trasladan del sistema 2 al sistema 1, de la cola $\hat{Q}_{2}$ a la cola
$Q_{2}$, en el intervalo de tiempo $\left[t,t+1\right]$. El traslado de un sistema a otro ocurre de manera que los tiempos entre llegadas de los usuarios a la cola dos del sistema 1 provenientes del sistema 2, se distribuye de manera general con par\'ametro $\check{\mu}_{2}$, con $\check{\mu}_{2}<1$.



\item En lo que respecta al servidor, en t\'erminos de los tiempos de
visita a cada una de las colas, se definen las variables
aleatorias $\tau_{i},$ para $Q_{i}$, para $i=1,2$, respectivamente;
y $\zeta_{i}$ para $\hat{Q}_{i}$,  $i=1,2$,  del sistema
2 respectivamente. A los tiempos en que el servidor termina de atender en las colas $Q_{i},\hat{Q}_{i}$,se les denotar\'a por
$\overline{\tau}_{i},\overline{\zeta}_{i}$ para  $i=1,2$,
respectivamente.

\item Los tiempos de traslado del servidor desde el momento en que termina de atender a una cola y llega a la siguiente para comenzar a dar servicio est\'an dados por
$\tau_{i+1}-\overline{\tau}_{i}$ y
$\zeta_{i+1}-\overline{\zeta}_{i}$,  $i=1,2$, para el sistema 1 y el sistema 2, respectivamente.

\end{itemize}




%\begin{figure}[H]
%\centering
%%%\includegraphics[width=5cm]{RedSistemasVisitasCiclicas.jpg}
%%\end{figure}\label{RSVC}

El uso de la Funci\'on Generadora de Probabilidades (FGP's) nos permite determinar las Funciones de Distribuci\'on de Probabilidades Conjunta de manera indirecta sin necesidad de hacer uso de las propiedades de las distribuciones de probabilidad de cada uno de los procesos que intervienen en la Red de Sistemas de Visitas C\'iclicas.\smallskip

Cada uno de estos procesos con su respectiva FGP. Adem\'as, para cada una de las colas en cada sistema, el n\'umero de usuarios al tiempo en que llega el servidor a dar servicio est\'a
dado por el n\'umero de usuarios presentes en la cola al tiempo
$t$, m\'as el n\'umero de usuarios que llegan a la cola en el intervalo de tiempo
$\left[\tau_{i},\overline{\tau}_{i}\right]$.




Una vez definidas las Funciones Generadoras de Probabilidades Conjuntas se construyen las ecuaciones recursivas que permiten obtener la informaci\'on sobre la longitud de cada una de las colas, al momento en que uno de los servidores llega a una de las colas para dar servicio, bas\'andose en la informaci\'on que se tiene sobre su llegada a la cola inmediata anterior.\smallskip

%__________________________________________________________________________
\subsection{Funciones Generadoras de Probabilidades}
%__________________________________________________________________________


Para cada uno de los procesos de llegada a las colas $X_{i},\hat{X}_{i}$,  $i=1,2$,  y $Y_{2}$, con $\tilde{X}_{2}=X_{2}+Y_{2}$ anteriores se define su Funci\'on
Generadora de Probabilidades (FGP): $P_{i}\left(z_{i}\right)=\esp\left[z_{i}^{X_{i}\left(t\right)}\right],\hat{P}_{i}\left(w_{i}\right)=\esp\left[w_{i}^{\hat{X}_{i}\left(t\right)}\right]$, para
$i=1,2$, y $\check{P}_{2}\left(z_{2}\right)=\esp\left[z_{2}^{Y_{2}\left(t\right)}\right], \tilde{P}_{2}\left(z_{2}\right)=\esp\left[z_{2}^{\tilde{X}_{2}\left(t\right)}\right]$ , con primer momento definidos por $\mu_{i}=\esp\left[X_{i}\left(t\right)\right]=P_{i}^{(1)}\left(1\right), \hat{\mu}_{i}=\esp\left[\hat{X}_{i}\left(t\right)\right]=\hat{P}_{i}^{(1)}\left(1\right)$, para $i=1,2$, y
$\check{\mu}_{2}=\esp\left[Y_{2}\left(t\right)\right]=\check{P}_{2}^{(1)}\left(1\right),\tilde{\mu}_{2}=\esp\left[\tilde{X}_{2}\left(t\right)\right]=\tilde{P}_{2}^{(1)}\left(1\right)$.

En lo que respecta al servidor, en t\'erminos de los tiempos de
visita a cada una de las colas, se denotar\'an por
$B_{i}\left(t\right)$ a los procesos
correspondientes a las variables aleatorias $\tau_{i}$
para $Q_{i}$, respectivamente; y
$\hat{B}_{i}\left(t\right)$ con
par\'ametros $\zeta_{i}$ para $\hat{Q}_{i}$, del sistema 2 respectivamente. Y a los tiempos en que el servidor termina de
atender en las colas $Q_{i},\hat{Q}_{i}$, se les
denotar\'a por
$\overline{\tau}_{i},\overline{\zeta}_{i}$ respectivamente. Entonces, los tiempos de servicio est\'an dados por las diferencias
$\overline{\tau}_{i}-\tau_{i}$ para
$Q_{i}$, y
$\overline{\zeta}_{i}-\zeta_{i}$ para $\hat{Q}_{i}$ respectivamente, para $i=1,2$.

Sus procesos se definen por: $S_{i}\left(z_{i}\right)=\esp\left[z_{i}^{\overline{\tau}_{i}-\tau_{i}}\right]$ y $\hat{S}_{i}\left(w_{i}\right)=\esp\left[w_{i}^{\overline{\zeta}_{i}-\zeta_{i}}\right]$, con primer momento dado por: $s_{i}=\esp\left[\overline{\tau}_{i}-\tau_{i}\right]$ y $\hat{s}_{i}=\esp\left[\overline{\zeta}_{i}-\zeta_{i}\right]$, para $i=1,2$. An\'alogamente los tiempos de traslado del servidor desde el momento en que termina de atender a una cola y llega a la
siguiente para comenzar a dar servicio est\'an dados por
$\tau_{i+1}-\overline{\tau}_{i}$ y
$\zeta_{i+1}-\overline{\zeta}_{i}$ para el sistema 1 y el sistema 2, respectivamente, con $i=1,2$.

La FGP para estos tiempos de traslado est\'an dados por $R_{i}\left(z_{i}\right)=\esp\left[z_{1}^{\tau_{i+1}-\overline{\tau}_{i}}\right]$ y $\hat{R}_{i}\left(w_{i}\right)=\esp\left[w_{i}^{\zeta_{i+1}-\overline{\zeta}_{i}}\right]$ y al igual que como se hizo con anterioridad, se tienen los primeros momentos de estos procesos de traslado del servidor entre las colas de cada uno de los sistemas que conforman la red de sistemas de visitas c\'iclicas: $r_{i}=R_{i}^{(1)}\left(1\right)=\esp\left[\tau_{i+1}-\overline{\tau}_{i}\right]$ y $\hat{r}_{i}=\hat{R}_{i}^{(1)}\left(1\right)=\esp\left[\zeta_{i+1}-\overline{\zeta}_{i}\right]$ para $i=1,2$.


Se definen los procesos de conteo para el n\'umero de usuarios en
cada una de las colas al tiempo $t$,
$L_{i}\left(t\right)$, para
$H_{i}\left(t\right)$ del sistema 1,
mientras que para el segundo sistema, se tienen los procesos
$\hat{L}_{i}\left(t\right)$ para
$\hat{H}_{i}\left(t\right)$, es decir, $H_{i}\left(t\right)=\esp\left[z_{i}^{L_{i}\left(t\right)}\right]$ y $\hat{H}_{i}\left(t\right)=\esp\left[w_{i}^{\hat{L}_{i}\left(t\right)}\right]$. Con lo dichohasta ahora, se tiene que el n\'umero de usuarios
presentes en los tiempos $\overline{\tau}_{1},\overline{\tau}_{2},
\overline{\zeta}_{1},\overline{\zeta}_{2}$, es cero, es decir,
 $L_{i}\left(\overline{\tau_{i}}\right)=0,$ y
$\hat{L}_{i}\left(\overline{\zeta_{i}}\right)=0$ para i=1,2 para
cada uno de los dos sistemas.


Para cada una de las colas en cada sistema, el n\'umero de
usuarios al tiempo en que llega el servidor a dar servicio est\'a
dado por el n\'umero de usuarios presentes en la cola al tiempo
$t=\tau_{i},\zeta_{i}$, m\'as el n\'umero de usuarios que llegan a
la cola en el intervalo de tiempo
$\left[\tau_{i},\overline{\tau}_{i}\right],\left[\zeta_{i},\overline{\zeta}_{i}\right]$,
es decir $\hat{L}_{i}\left(\overline{\tau}_{j}\right)=\hat{L}_{i}\left(\tau_{j}\right)+\hat{X}_{i}\left(\overline{\tau}_{j}-\tau_{j}\right)$, para $i,j=1,2$, mientras que para el primer sistema: $L_{1}\left(\overline{\tau}_{j}\right)=L_{1}\left(\tau_{j}\right)+X_{1}\left(\overline{\tau}_{j}-\tau_{j}\right)$. En el caso espec\'ifico de $Q_{2}$, adem\'as, hay que considerar
el n\'umero de usuarios que pasan del sistema 2 al sistema 1, a
traves de $\hat{Q}_{2}$ mientras el servidor en $Q_{2}$ est\'a
ausente, es decir:

\begin{equation}\label{Eq.UsuariosTotalesZ2}
L_{2}\left(\overline{\tau}_{1}\right)=L_{2}\left(\tau_{1}\right)+X_{2}\left(\overline{\tau}_{1}-\tau_{1}\right)+Y_{2}\left(\overline{\tau}_{1}-\tau_{1}\right).
\end{equation}

%_________________________________________________________________________
\subsection{El problema de la ruina del jugador}
%_________________________________________________________________________

Supongamos que se tiene un jugador que cuenta con un capital
inicial de $\tilde{L}_{0}\geq0$ unidades, esta persona realiza una
serie de dos juegos simult\'aneos e independientes de manera
sucesiva, dichos eventos son independientes e id\'enticos entre
s\'i para cada realizaci\'on. La ganancia en el $n$-\'esimo juego es $\tilde{X}_{n}=X_{n}+Y_{n}$ unidades de las cuales se resta una cuota de 1 unidad por cada juego simult\'aneo, es decir, se restan dos unidades por cada
juego realizado. En t\'erminos de la teor\'ia de colas puede pensarse como el n\'umero de usuarios que llegan a una cola v\'ia dos procesos de arribo distintos e independientes entre s\'i. Su Funci\'on Generadora de Probabilidades (FGP) est\'a dada por $F\left(z\right)=\esp\left[z^{\tilde{L}_{0}}\right]$, adem\'as
$$\tilde{P}\left(z\right)=\esp\left[z^{\tilde{X}_{n}}\right]=\esp\left[z^{X_{n}+Y_{n}}\right]=\esp\left[z^{X_{n}}z^{Y_{n}}\right]=\esp\left[z^{X_{n}}\right]\esp\left[z^{Y_{n}}\right]=P\left(z\right)\check{P}\left(z\right),$$

con $\tilde{\mu}=\esp\left[\tilde{X}_{n}\right]=\tilde{P}\left[z\right]<1$. Sea $\tilde{L}_{n}$ el capital remanente despu\'es del $n$-\'esimo
juego. Entonces

$$\tilde{L}_{n}=\tilde{L}_{0}+\tilde{X}_{1}+\tilde{X}_{2}+\cdots+\tilde{X}_{n}-2n.$$

La ruina del jugador ocurre despu\'es del $n$-\'esimo juego, es decir, la cola se vac\'ia despu\'es del $n$-\'esimo juego,
entonces sea $T$ definida como $T=min\left\{\tilde{L}_{n}=0\right\}$. Si $\tilde{L}_{0}=0$, entonces claramente $T=0$. En este sentido $T$
puede interpretarse como la longitud del periodo de tiempo que el servidor ocupa para dar servicio en la cola, comenzando con $\tilde{L}_{0}$ grupos de usuarios presentes en la cola, quienes arribaron conforme a un proceso dado
por $\tilde{P}\left(z\right)$.\smallskip


Sea $g_{n,k}$ la probabilidad del evento de que el jugador no
caiga en ruina antes del $n$-\'esimo juego, y que adem\'as tenga
un capital de $k$ unidades antes del $n$-\'esimo juego, es decir,

Dada $n\in\left\{1,2,\ldots,\right\}$ y
$k\in\left\{0,1,2,\ldots,\right\}$
\begin{eqnarray*}
g_{n,k}:=P\left\{\tilde{L}_{j}>0, j=1,\ldots,n,
\tilde{L}_{n}=k\right\}
\end{eqnarray*}

la cual adem\'as se puede escribir como:

\begin{eqnarray*}
g_{n,k}&=&P\left\{\tilde{L}_{j}>0, j=1,\ldots,n,
\tilde{L}_{n}=k\right\}=\sum_{j=1}^{k+1}g_{n-1,j}P\left\{\tilde{X}_{n}=k-j+1\right\}\\
&=&\sum_{j=1}^{k+1}g_{n-1,j}P\left\{X_{n}+Y_{n}=k-j+1\right\}=\sum_{j=1}^{k+1}\sum_{l=1}^{j}g_{n-1,j}P\left\{X_{n}+Y_{n}=k-j+1,Y_{n}=l\right\}\\
&=&\sum_{j=1}^{k+1}\sum_{l=1}^{j}g_{n-1,j}P\left\{X_{n}+Y_{n}=k-j+1|Y_{n}=l\right\}P\left\{Y_{n}=l\right\}\\
&=&\sum_{j=1}^{k+1}\sum_{l=1}^{j}g_{n-1,j}P\left\{X_{n}=k-j-l+1\right\}P\left\{Y_{n}=l\right\}\\
\end{eqnarray*}

es decir
\begin{eqnarray}\label{Eq.Gnk.2S}
g_{n,k}=\sum_{j=1}^{k+1}\sum_{l=1}^{j}g_{n-1,j}P\left\{X_{n}=k-j-l+1\right\}P\left\{Y_{n}=l\right\}
\end{eqnarray}
adem\'as

\begin{equation}\label{Eq.L02S}
g_{0,k}=P\left\{\tilde{L}_{0}=k\right\}.
\end{equation}

Se definen las siguientes FGP:
\begin{equation}\label{Eq.3.16.a.2S}
G_{n}\left(z\right)=\sum_{k=0}^{\infty}g_{n,k}z^{k},\textrm{ para
}n=0,1,\ldots,
\end{equation}

\begin{equation}\label{Eq.3.16.b.2S}
G\left(z,w\right)=\sum_{n=0}^{\infty}G_{n}\left(z\right)w^{n}.
\end{equation}


En particular para $k=0$,
\begin{eqnarray*}
g_{n,0}=G_{n}\left(0\right)=P\left\{\tilde{L}_{j}>0,\textrm{ para
}j<n,\textrm{ y }\tilde{L}_{n}=0\right\}=P\left\{T=n\right\},
\end{eqnarray*}

adem\'as

\begin{eqnarray*}%\label{Eq.G0w.2S}
G\left(0,w\right)=\sum_{n=0}^{\infty}G_{n}\left(0\right)w^{n}=\sum_{n=0}^{\infty}P\left\{T=n\right\}w^{n}
=\esp\left[w^{T}\right]
\end{eqnarray*}
la cu\'al resulta ser la FGP del tiempo de ruina $T$.

%__________________________________________________________________________________
% INICIA LA PROPOSICIÓN
%__________________________________________________________________________________


\begin{Prop}\label{Prop.1.1.2S}
Sean $G_{n}\left(z\right)$ y $G\left(z,w\right)$ definidas como en
(\ref{Eq.3.16.a.2S}) y (\ref{Eq.3.16.b.2S}) respectivamente,
entonces
\begin{equation}\label{Eq.Pag.45}
G_{n}\left(z\right)=\frac{1}{z}\left[G_{n-1}\left(z\right)-G_{n-1}\left(0\right)\right]\tilde{P}\left(z\right).
\end{equation}

Adem\'as


\begin{equation}\label{Eq.Pag.46}
G\left(z,w\right)=\frac{zF\left(z\right)-wP\left(z\right)G\left(0,w\right)}{z-wR\left(z\right)},
\end{equation}

con un \'unico polo en el c\'irculo unitario, adem\'as, el polo es
de la forma $z=\theta\left(w\right)$ y satisface que

\begin{enumerate}
\item[i)]$\tilde{\theta}\left(1\right)=1$,

\item[ii)] $\tilde{\theta}^{(1)}\left(1\right)=\frac{1}{1-\tilde{\mu}}$,

\item[iii)]
$\tilde{\theta}^{(2)}\left(1\right)=\frac{\tilde{\mu}}{\left(1-\tilde{\mu}\right)^{2}}+\frac{\tilde{\sigma}}{\left(1-\tilde{\mu}\right)^{3}}$.
\end{enumerate}

Finalmente, adem\'as se cumple que
\begin{equation}
\esp\left[w^{T}\right]=G\left(0,w\right)=F\left[\tilde{\theta}\left(w\right)\right].
\end{equation}
\end{Prop}
%__________________________________________________________________________________
% TERMINA LA PROPOSICIÓN E INICIA LA DEMOSTRACI\'ON
%__________________________________________________________________________________


Multiplicando las ecuaciones (\ref{Eq.Gnk.2S}) y (\ref{Eq.L02S})
por el t\'ermino $z^{k}$:

\begin{eqnarray*}
g_{n,k}z^{k}&=&\sum_{j=1}^{k+1}\sum_{l=1}^{j}g_{n-1,j}P\left\{X_{n}=k-j-l+1\right\}P\left\{Y_{n}=l\right\}z^{k},\\
g_{0,k}z^{k}&=&P\left\{\tilde{L}_{0}=k\right\}z^{k},
\end{eqnarray*}

ahora sumamos sobre $k$
\begin{eqnarray*}
\sum_{k=0}^{\infty}g_{n,k}z^{k}&=&\sum_{k=0}^{\infty}\sum_{j=1}^{k+1}\sum_{l=1}^{j}g_{n-1,j}P\left\{X_{n}=k-j-l+1\right\}P\left\{Y_{n}=l\right\}z^{k}\\
&=&\sum_{k=0}^{\infty}z^{k}\sum_{j=1}^{k+1}\sum_{l=1}^{j}g_{n-1,j}P\left\{X_{n}=k-\left(j+l
-1\right)\right\}P\left\{Y_{n}=l\right\}\\
&=&\sum_{k=0}^{\infty}z^{k+\left(j+l-1\right)-\left(j+l-1\right)}\sum_{j=1}^{k+1}\sum_{l=1}^{j}g_{n-1,j}P\left\{X_{n}=k-
\left(j+l-1\right)\right\}P\left\{Y_{n}=l\right\}\\
&=&\sum_{k=0}^{\infty}\sum_{j=1}^{k+1}\sum_{l=1}^{j}g_{n-1,j}z^{j-1}P\left\{X_{n}=k-
\left(j+l-1\right)\right\}z^{k-\left(j+l-1\right)}P\left\{Y_{n}=l\right\}z^{l}\\
&=&\sum_{j=1}^{\infty}\sum_{l=1}^{j}g_{n-1,j}z^{j-1}\sum_{k=j+l-1}^{\infty}P\left\{X_{n}=k-
\left(j+l-1\right)\right\}z^{k-\left(j+l-1\right)}P\left\{Y_{n}=l\right\}z^{l}\\
&=&\sum_{j=1}^{\infty}g_{n-1,j}z^{j-1}\sum_{l=1}^{j}\sum_{k=j+l-1}^{\infty}P\left\{X_{n}=k-
\left(j+l-1\right)\right\}z^{k-\left(j+l-1\right)}P\left\{Y_{n}=l\right\}z^{l}\\
&=&\sum_{j=1}^{\infty}g_{n-1,j}z^{j-1}\sum_{k=j+l-1}^{\infty}\sum_{l=1}^{j}P\left\{X_{n}=k-
\left(j+l-1\right)\right\}z^{k-\left(j+l-1\right)}P\left\{Y_{n}=l\right\}z^{l}\\
\end{eqnarray*}


luego
\begin{eqnarray*}
&=&\sum_{j=1}^{\infty}g_{n-1,j}z^{j-1}\sum_{k=j+l-1}^{\infty}\sum_{l=1}^{j}P\left\{X_{n}=k-
\left(j+l-1\right)\right\}z^{k-\left(j+l-1\right)}\sum_{l=1}^{j}P
\left\{Y_{n}=l\right\}z^{l}\\
&=&\sum_{j=1}^{\infty}g_{n-1,j}z^{j-1}\sum_{l=1}^{\infty}P\left\{Y_{n}=l\right\}z^{l}
\sum_{k=j+l-1}^{\infty}\sum_{l=1}^{j}
P\left\{X_{n}=k-\left(j+l-1\right)\right\}z^{k-\left(j+l-1\right)}\\
&=&\frac{1}{z}\left[G_{n-1}\left(z\right)-G_{n-1}\left(0\right)\right]\tilde{P}\left(z\right)
\sum_{k=j+l-1}^{\infty}\sum_{l=1}^{j}
P\left\{X_{n}=k-\left(j+l-1\right)\right\}z^{k-\left(j+l-1\right)}\\
&=&\frac{1}{z}\left[G_{n-1}\left(z\right)-G_{n-1}\left(0\right)\right]\tilde{P}\left(z\right)P\left(z\right)=\frac{1}{z}\left[G_{n-1}\left(z\right)-G_{n-1}\left(0\right)\right]\tilde{P}\left(z\right),\\
\end{eqnarray*}

es decir la ecuaci\'on (\ref{Eq.3.16.a.2S}) se puede reescribir
como
\begin{equation}\label{Eq.3.16.a.2Sbis}
G_{n}\left(z\right)=\frac{1}{z}\left[G_{n-1}\left(z\right)-G_{n-1}\left(0\right)\right]\tilde{P}\left(z\right).
\end{equation}

Por otra parte recordemos la ecuaci\'on (\ref{Eq.3.16.a.2S})

\begin{eqnarray*}
G_{n}\left(z\right)&=&\sum_{k=0}^{\infty}g_{n,k}z^{k},\textrm{ entonces }\frac{G_{n}\left(z\right)}{z}=\sum_{k=1}^{\infty}g_{n,k}z^{k-1},\\
\end{eqnarray*}

Por lo tanto utilizando la ecuaci\'on (\ref{Eq.3.16.a.2Sbis}):

\begin{eqnarray*}
G\left(z,w\right)&=&\sum_{n=0}^{\infty}G_{n}\left(z\right)w^{n}=G_{0}\left(z\right)+
\sum_{n=1}^{\infty}G_{n}\left(z\right)w^{n}=F\left(z\right)+\sum_{n=0}^{\infty}\left[G_{n}\left(z\right)-G_{n}\left(0\right)\right]w^{n}\frac{\tilde{P}\left(z\right)}{z}\\
&=&F\left(z\right)+\frac{w}{z}\sum_{n=0}^{\infty}\left[G_{n}\left(z\right)-G_{n}\left(0\right)\right]w^{n-1}\tilde{P}\left(z\right)\\
\end{eqnarray*}

es decir
\begin{eqnarray*}
G\left(z,w\right)&=&F\left(z\right)+\frac{w}{z}\left[G\left(z,w\right)-G\left(0,w\right)\right]\tilde{P}\left(z\right),
\end{eqnarray*}


entonces

\begin{eqnarray*}
G\left(z,w\right)=F\left(z\right)+\frac{w}{z}\left[G\left(z,w\right)-G\left(0,w\right)\right]\tilde{P}\left(z\right)&=&F\left(z\right)+\frac{w}{z}\tilde{P}\left(z\right)G\left(z,w\right)-\frac{w}{z}\tilde{P}\left(z\right)G\left(0,w\right)\\
&\Leftrightarrow&\\
G\left(z,w\right)\left\{1-\frac{w}{z}\tilde{P}\left(z\right)\right\}&=&F\left(z\right)-\frac{w}{z}\tilde{P}\left(z\right)G\left(0,w\right),
\end{eqnarray*}
por lo tanto,
\begin{equation}
G\left(z,w\right)=\frac{zF\left(z\right)-w\tilde{P}\left(z\right)G\left(0,w\right)}{1-w\tilde{P}\left(z\right)}.
\end{equation}


Ahora $G\left(z,w\right)$ es anal\'itica en $|z|=1$. Sean $z,w$ tales que $|z|=1$ y $|w|\leq1$, como $\tilde{P}\left(z\right)$ es FGP
\begin{eqnarray*}
|z-\left(z-w\tilde{P}\left(z\right)\right)|<|z|\Leftrightarrow|w\tilde{P}\left(z\right)|<|z|
\end{eqnarray*}
es decir, se cumplen las condiciones del Teorema de Rouch\'e y por
tanto, $z$ y $z-w\tilde{P}\left(z\right)$ tienen el mismo n\'umero de
ceros en $|z|=1$. Sea $z=\tilde{\theta}\left(w\right)$ la soluci\'on
\'unica de $z-w\tilde{P}\left(z\right)$, es decir

\begin{equation}\label{Eq.Theta.w}
\tilde{\theta}\left(w\right)-w\tilde{P}\left(\tilde{\theta}\left(w\right)\right)=0,
\end{equation}
 con $|\tilde{\theta}\left(w\right)|<1$. Cabe hacer menci\'on que $\tilde{\theta}\left(w\right)$ es la FGP para el tiempo de ruina cuando $\tilde{L}_{0}=1$.


Considerando la ecuaci\'on (\ref{Eq.Theta.w})
\begin{eqnarray*}
0&=&\frac{\partial}{\partial w}\tilde{\theta}\left(w\right)|_{w=1}-\frac{\partial}{\partial w}\left\{w\tilde{P}\left(\tilde{\theta}\left(w\right)\right)\right\}|_{w=1}=\tilde{\theta}^{(1)}\left(w\right)|_{w=1}-\frac{\partial}{\partial w}w\left\{\tilde{P}\left(\tilde{\theta}\left(w\right)\right)\right\}|_{w=1}\\
&-&w\frac{\partial}{\partial w}\tilde{P}\left(\tilde{\theta}\left(w\right)\right)|_{w=1}=\tilde{\theta}^{(1)}\left(1\right)-\tilde{P}\left(\tilde{\theta}\left(1\right)\right)-w\left\{\frac{\partial \tilde{P}\left(\tilde{\theta}\left(w\right)\right)}{\partial \tilde{\theta}\left(w\right)}\cdot\frac{\partial\tilde{\theta}\left(w\right)}{\partial w}|_{w=1}\right\}\\
&&\tilde{\theta}^{(1)}\left(1\right)-\tilde{P}\left(\tilde{\theta}\left(1\right)
\right)-\tilde{P}^{(1)}\left(\tilde{\theta}\left(1\right)\right)\cdot\tilde{\theta}^{(1)}\left(1\right),
\end{eqnarray*}


luego
$$\tilde{P}\left(\tilde{\theta}\left(1\right)\right)=\tilde{\theta}^{(1)}\left(1\right)-\tilde{P}^{(1)}\left(\tilde{\theta}\left(1\right)\right)\cdot
\tilde{\theta}^{(1)}\left(1\right)=\tilde{\theta}^{(1)}\left(1\right)\left(1-\tilde{P}^{(1)}\left(\tilde{\theta}\left(1\right)\right)\right),$$

por tanto $$\tilde{\theta}^{(1)}\left(1\right)=\frac{\tilde{P}\left(\tilde{\theta}\left(1\right)\right)}{\left(1-\tilde{P}^{(1)}\left(\tilde{\theta}\left(1\right)\right)\right)}=\frac{1}{1-\tilde{\mu}}.$$

Ahora determinemos el segundo momento de $\tilde{\theta}\left(w\right)$,
nuevamente consideremos la ecuaci\'on (\ref{Eq.Theta.w}):

\begin{eqnarray*}
0&=&\tilde{\theta}\left(w\right)-w\tilde{P}\left(\tilde{\theta}\left(w\right)\right)\Rightarrow 0=\frac{\partial}{\partial w}\left\{\tilde{\theta}\left(w\right)-w\tilde{P}\left(\tilde{\theta}\left(w\right)\right)\right\}\Rightarrow 0=\frac{\partial}{\partial w}\left\{\frac{\partial}{\partial w}\left\{\tilde{\theta}\left(w\right)-w\tilde{P}\left(\tilde{\theta}\left(w\right)\right)\right\}\right\}\\
\end{eqnarray*}
luego
\begin{eqnarray*}
&&\frac{\partial}{\partial w}\left\{\frac{\partial}{\partial w}\tilde{\theta}\left(w\right)-\frac{\partial}{\partial w}\left[w\tilde{P}\left(\tilde{\theta}\left(w\right)\right)\right]\right\}
=\frac{\partial}{\partial w}\left\{\frac{\partial}{\partial w}\tilde{\theta}\left(w\right)-\frac{\partial}{\partial w}\left[w\tilde{P}\left(\tilde{\theta}\left(w\right)\right)\right]\right\}\\
&=&\frac{\partial}{\partial w}\left\{\frac{\partial \tilde{\theta}\left(w\right)}{\partial w}-\left[\tilde{P}\left(\tilde{\theta}\left(w\right)\right)+w\frac{\partial}{\partial w}R\left(\tilde{\theta}\left(w\right)\right)\right]\right\}=\frac{\partial}{\partial w}\left\{\frac{\partial \tilde{\theta}\left(w\right)}{\partial w}-\left[\tilde{P}\left(\tilde{\theta}\left(w\right)\right)+w\frac{\partial \tilde{P}\left(\tilde{\theta}\left(w\right)\right)}{\partial w}\frac{\partial \tilde{\theta}\left(w\right)}{\partial w}\right]\right\}\\
&=&\frac{\partial}{\partial w}\left\{\tilde{\theta}^{(1)}\left(w\right)-\tilde{P}\left(\tilde{\theta}\left(w\right)\right)-w\tilde{P}^{(1)}\left(\tilde{\theta}\left(w\right)\right)\tilde{\theta}^{(1)}\left(w\right)\right\}\\
&=&\frac{\partial}{\partial w}\tilde{\theta}^{(1)}\left(w\right)-\frac{\partial}{\partial w}\tilde{P}\left(\tilde{\theta}\left(w\right)\right)-\frac{\partial}{\partial w}\left[w\tilde{P}^{(1)}\left(\tilde{\theta}\left(w\right)\right)\tilde{\theta}^{(1)}\left(w\right)\right]\\
&=&\frac{\partial}{\partial
w}\tilde{\theta}^{(1)}\left(w\right)-\frac{\partial
\tilde{P}\left(\tilde{\theta}\left(w\right)\right)}{\partial
\tilde{\theta}\left(w\right)}\frac{\partial \tilde{\theta}\left(w\right)}{\partial
w}-\tilde{P}^{(1)}\left(\tilde{\theta}\left(w\right)\right)\tilde{\theta}^{(1)}\left(w\right)-w\frac{\partial
\tilde{P}^{(1)}\left(\tilde{\theta}\left(w\right)\right)}{\partial
w}\tilde{\theta}^{(1)}\left(w\right)-w\tilde{P}^{(1)}\left(\tilde{\theta}\left(w\right)\right)\frac{\partial
\tilde{\theta}^{(1)}\left(w\right)}{\partial w}\\
&=&\tilde{\theta}^{(2)}\left(w\right)-\tilde{P}^{(1)}\left(\tilde{\theta}\left(w\right)\right)\tilde{\theta}^{(1)}\left(w\right)
-\tilde{P}^{(1)}\left(\tilde{\theta}\left(w\right)\right)\tilde{\theta}^{(1)}\left(w\right)-w\tilde{P}^{(2)}\left(\tilde{\theta}\left(w\right)\right)\left(\tilde{\theta}^{(1)}\left(w\right)\right)^{2}-w\tilde{P}^{(1)}\left(\tilde{\theta}\left(w\right)\right)\tilde{\theta}^{(2)}\left(w\right)\\
&=&\tilde{\theta}^{(2)}\left(w\right)-2\tilde{P}^{(1)}\left(\tilde{\theta}\left(w\right)\right)\tilde{\theta}^{(1)}\left(w\right)-w\tilde{P}^{(2)}\left(\tilde{\theta}\left(w\right)\right)\left(\tilde{\theta}^{(1)}\left(w\right)\right)^{2}-w\tilde{P}^{(1)}\left(\tilde{\theta}\left(w\right)\right)\tilde{\theta}^{(2)}\left(w\right)\\
&=&\tilde{\theta}^{(2)}\left(w\right)\left[1-w\tilde{P}^{(1)}\left(\tilde{\theta}\left(w\right)\right)\right]-
\tilde{\theta}^{(1)}\left(w\right)\left[w\tilde{\theta}^{(1)}\left(w\right)\tilde{P}^{(2)}\left(\tilde{\theta}\left(w\right)\right)+2\tilde{P}^{(1)}\left(\tilde{\theta}\left(w\right)\right)\right]
\end{eqnarray*}


luego

\begin{eqnarray*}
&&\tilde{\theta}^{(2)}\left(w\right)\left[1-w\tilde{P}^{(1)}\left(\tilde{\theta}\left(w\right)\right)\right]-\tilde{\theta}^{(1)}\left(w\right)\left[w\tilde{\theta}^{(1)}\left(w\right)\tilde{P}^{(2)}\left(\tilde{\theta}\left(w\right)\right)
+2\tilde{P}^{(1)}\left(\tilde{\theta}\left(w\right)\right)\right]=0\\
\tilde{\theta}^{(2)}\left(w\right)&=&\frac{\tilde{\theta}^{(1)}\left(w\right)\left[w\tilde{\theta}^{(1)}\left(w\right)\tilde{P}^{(2)}\left(\tilde{\theta}\left(w\right)\right)+2R^{(1)}\left(\tilde{\theta}\left(w\right)\right)\right]}{1-w\tilde{P}^{(1)}\left(\tilde{\theta}\left(w\right)\right)}=\frac{\tilde{\theta}^{(1)}\left(w\right)w\tilde{\theta}^{(1)}\left(w\right)\tilde{P}^{(2)}\left(\tilde{\theta}\left(w\right)\right)}{1-w\tilde{P}^{(1)}\left(\tilde{\theta}\left(w\right)\right)}\\
&+&\frac{2\tilde{\theta}^{(1)}\left(w\right)\tilde{P}^{(1)}\left(\tilde{\theta}\left(w\right)\right)}{1-w\tilde{P}^{(1)}\left(\tilde{\theta}\left(w\right)\right)}
\end{eqnarray*}


si evaluamos la expresi\'on anterior en $w=1$:
\begin{eqnarray*}
\tilde{\theta}^{(2)}\left(1\right)&=&\frac{\left(\tilde{\theta}^{(1)}\left(1\right)\right)^{2}\tilde{P}^{(2)}\left(\tilde{\theta}\left(1\right)\right)}{1-\tilde{P}^{(1)}\left(\tilde{\theta}\left(1\right)\right)}+\frac{2\tilde{\theta}^{(1)}\left(1\right)\tilde{P}^{(1)}\left(\tilde{\theta}\left(1\right)\right)}{1-\tilde{P}^{(1)}\left(\tilde{\theta}\left(1\right)\right)}=\frac{\left(\tilde{\theta}^{(1)}\left(1\right)\right)^{2}\tilde{P}^{(2)}\left(1\right)}{1-\tilde{P}^{(1)}\left(1\right)}+\frac{2\tilde{\theta}^{(1)}\left(1\right)\tilde{P}^{(1)}\left(1\right)}{1-\tilde{P}^{(1)}\left(1\right)}\\
&=&\frac{\left(\frac{1}{1-\tilde{\mu}}\right)^{2}\tilde{P}^{(2)}\left(1\right)}{1-\tilde{\mu}}+\frac{2\left(\frac{1}{1-\tilde{\mu}}\right)\tilde{\mu}}{1-\tilde{\mu}}=\frac{\tilde{P}^{(2)}\left(1\right)}{\left(1-\tilde{\mu}\right)^{3}}+\frac{2\tilde{\mu}}{\left(1-\tilde{\mu}\right)^{2}}=\frac{\sigma^{2}-\tilde{\mu}+\tilde{\mu}^{2}}{\left(1-\tilde{\mu}\right)^{3}}+\frac{2\tilde{\mu}}{\left(1-\tilde{\mu}\right)^{2}}\\
&=&\frac{\sigma^{2}-\tilde{\mu}+\tilde{\mu}^{2}+2\tilde{\mu}\left(1-\tilde{\mu}\right)}{\left(1-\tilde{\mu}\right)^{3}}\\
\end{eqnarray*}


es decir
\begin{eqnarray*}
\tilde{\theta}^{(2)}\left(1\right)&=&\frac{\sigma^{2}}{\left(1-\tilde{\mu}\right)^{3}}+\frac{\tilde{\mu}}{\left(1-\tilde{\mu}\right)^{2}}.
\end{eqnarray*}

\begin{Coro}
El tiempo de ruina del jugador tiene primer y segundo momento
dados por

\begin{eqnarray}
\esp\left[T\right]&=&\frac{\esp\left[\tilde{L}_{0}\right]}{1-\tilde{\mu}}\\
Var\left[T\right]&=&\frac{Var\left[\tilde{L}_{0}\right]}{\left(1-\tilde{\mu}\right)^{2}}+\frac{\sigma^{2}\esp\left[\tilde{L}_{0}\right]}{\left(1-\tilde{\mu}\right)^{3}}.
\end{eqnarray}
\end{Coro}



%__________________________________________________________________________
\section{Procesos de Llegadas a las colas en la RSVC}
%__________________________________________________________________________

Se definen los procesos de llegada de los usuarios a cada una de
las colas dependiendo de la llegada del servidor pero del sistema
al cu\'al no pertenece la cola en cuesti\'on:

Para el sistema 1 y el servidor del segundo sistema

\begin{eqnarray*}
F_{i,j}\left(z_{i};\zeta_{j}\right)=\esp\left[z_{i}^{L_{i}\left(\zeta_{j}\right)}\right]=
\sum_{k=0}^{\infty}\prob\left[L_{i}\left(\zeta_{j}\right)=k\right]z_{i}^{k}\textrm{, para }i,j=1,2.
%F_{1,1}\left(z_{1};\zeta_{1}\right)&=&\esp\left[z_{1}^{L_{1}\left(\zeta_{1}\right)}\right]=
%\sum_{k=0}^{\infty}\prob\left[L_{1}\left(\zeta_{1}\right)=k\right]z_{1}^{k};\\
%F_{2,1}\left(z_{2};\zeta_{1}\right)&=&\esp\left[z_{2}^{L_{2}\left(\zeta_{1}\right)}\right]=
%\sum_{k=0}^{\infty}\prob\left[L_{2}\left(\zeta_{1}\right)=k\right]z_{2}^{k};\\
%F_{1,2}\left(z_{1};\zeta_{2}\right)&=&\esp\left[z_{1}^{L_{1}\left(\zeta_{2}\right)}\right]=
%\sum_{k=0}^{\infty}\prob\left[L_{1}\left(\zeta_{2}\right)=k\right]z_{1}^{k};\\
%F_{2,2}\left(z_{2};\zeta_{2}\right)&=&\esp\left[z_{2}^{L_{2}\left(\zeta_{2}\right)}\right]=
%\sum_{k=0}^{\infty}\prob\left[L_{2}\left(\zeta_{2}\right)=k\right]z_{2}^{k}.\\
\end{eqnarray*}

Ahora se definen para el segundo sistema y el servidor del primero


\begin{eqnarray*}
\hat{F}_{i,j}\left(w_{i};\tau_{j}\right)&=&\esp\left[w_{i}^{\hat{L}_{i}\left(\tau_{j}\right)}\right] =\sum_{k=0}^{\infty}\prob\left[\hat{L}_{i}\left(\tau_{j}\right)=k\right]w_{i}^{k}\textrm{, para }i,j=1,2.
%\hat{F}_{1,1}\left(w_{1};\tau_{1}\right)&=&\esp\left[w_{1}^{\hat{L}_{1}\left(\tau_{1}\right)}\right] =\sum_{k=0}^{\infty}\prob\left[\hat{L}_{1}\left(\tau_{1}\right)=k\right]w_{1}^{k}\\
%\hat{F}_{2,1}\left(w_{2};\tau_{1}\right)&=&\esp\left[w_{2}^{\hat{L}_{2}\left(\tau_{1}\right)}\right] =\sum_{k=0}^{\infty}\prob\left[\hat{L}_{2}\left(\tau_{1}\right)=k\right]w_{2}^{k}\\
%\hat{F}_{1,2}\left(w_{1};\tau_{2}\right)&=&\esp\left[w_{1}^{\hat{L}_{1}\left(\tau_{2}\right)}\right]
%=\sum_{k=0}^{\infty}\prob\left[\hat{L}_{1}\left(\tau_{2}\right)=k\right]w_{1}^{k}\\
%\hat{F}_{2,2}\left(w_{2};\tau_{2}\right)&=&\esp\left[w_{2}^{\hat{L}_{2}\left(\tau_{2}\right)}\right]
%=\sum_{k=0}^{\infty}\prob\left[\hat{L}_{2}\left(\tau_{2}\right)=k\right]w_{2}^{k}\\
\end{eqnarray*}


Ahora, con lo anterior definamos la FGP conjunta para el segundo sistema;% y $\tau_{1}$:


\begin{eqnarray*}
\esp\left[w_{1}^{\hat{L}_{1}\left(\tau_{j}\right)}w_{2}^{\hat{L}_{2}\left(\tau_{j}\right)}\right]
&=&\esp\left[w_{1}^{\hat{L}_{1}\left(\tau_{j}\right)}\right]
\esp\left[w_{2}^{\hat{L}_{2}\left(\tau_{j}\right)}\right]=\hat{F}_{1,j}\left(w_{1};\tau_{j}\right)\hat{F}_{2,j}\left(w_{2};\tau_{j}\right)=\hat{F}_{j}\left(w_{1},w_{2};\tau_{j}\right).\\
%\esp\left[w_{1}^{\hat{L}_{1}\left(\tau_{1}\right)}w_{2}^{\hat{L}_{2}\left(\tau_{1}\right)}\right]
%&=&\esp\left[w_{1}^{\hat{L}_{1}\left(\tau_{1}\right)}\right]
%\esp\left[w_{2}^{\hat{L}_{2}\left(\tau_{1}\right)}\right]=\hat{F}_{1,1}\left(w_{1};\tau_{1}\right)\hat{F}_{2,1}\left(w_{2};\tau_{1}\right)=\hat{F}_{1}\left(w_{1},w_{2};\tau_{1}\right)\\
%\esp\left[w_{1}^{\hat{L}_{1}\left(\tau_{2}\right)}w_{2}^{\hat{L}_{2}\left(\tau_{2}\right)}\right]
%&=&\esp\left[w_{1}^{\hat{L}_{1}\left(\tau_{2}\right)}\right]
%   \esp\left[w_{2}^{\hat{L}_{2}\left(\tau_{2}\right)}\right]=\hat{F}_{1,2}\left(w_{1};\tau_{2}\right)\hat{F}_{2,2}\left(w_{2};\tau_{2}\right)=\hat{F}_{2}\left(w_{1},w_{2};\tau_{2}\right).
\end{eqnarray*}

Con respecto al sistema 1 se tiene la FGP conjunta con respecto al servidor del otro sistema:
\begin{eqnarray*}
\esp\left[z_{1}^{L_{1}\left(\zeta_{j}\right)}z_{2}^{L_{2}\left(\zeta_{j}\right)}\right]
&=&\esp\left[z_{1}^{L_{1}\left(\zeta_{j}\right)}\right]
\esp\left[z_{2}^{L_{2}\left(\zeta_{j}\right)}\right]=F_{1,j}\left(z_{1};\zeta_{j}\right)F_{2,j}\left(z_{2};\zeta_{j}\right)=F_{j}\left(z_{1},z_{2};\zeta_{j}\right).
%\esp\left[z_{1}^{L_{1}\left(\zeta_{1}\right)}z_{2}^{L_{2}\left(\zeta_{1}\right)}\right]
%&=&\esp\left[z_{1}^{L_{1}\left(\zeta_{1}\right)}\right]
%\esp\left[z_{2}^{L_{2}\left(\zeta_{1}\right)}\right]=F_{1,1}\left(z_{1};\zeta_{1}\right)F_{2,1}\left(z_{2};\zeta_{1}\right)=F_{1}\left(z_{1},z_{2};\zeta_{1}\right)\\
%\esp\left[z_{1}^{L_{1}\left(\zeta_{2}\right)}z_{2}^{L_{2}\left(\zeta_{2}\right)}\right]
%&=&\esp\left[z_{1}^{L_{1}\left(\zeta_{2}\right)}\right]
%\esp\left[z_{2}^{L_{2}\left(\zeta_{2}\right)}\right]=F_{1,2}\left(z_{1};\zeta_{2}\right)F_{2,2}\left(z_{2};\zeta_{2}\right)=F_{2}\left(z_{1},z_{2};\zeta_{2}\right).
\end{eqnarray*}

Ahora analicemos la Red de Sistemas de Visitas C\'iclicas, entonces se define la PGF conjunta al tiempo $t$ para los tiempos de visita del servidor en cada una de las colas, para comenzar a dar servicio, definidos anteriormente al tiempo
$t=\left\{\tau_{1},\tau_{2},\zeta_{1},\zeta_{2}\right\}$:

\begin{eqnarray}\label{Eq.Conjuntas}
F_{j}\left(z_{1},z_{2},w_{1},w_{2}\right)&=&\esp\left[\prod_{i=1}^{2}z_{i}^{L_{i}\left(\tau_{j}
\right)}\prod_{i=1}^{2}w_{i}^{\hat{L}_{i}\left(\tau_{j}\right)}\right]\\
\hat{F}_{j}\left(z_{1},z_{2},w_{1},w_{2}\right)&=&\esp\left[\prod_{i=1}^{2}z_{i}^{L_{i}
\left(\zeta_{j}\right)}\prod_{i=1}^{2}w_{i}^{\hat{L}_{i}\left(\zeta_{j}\right)}\right]
\end{eqnarray}
para $j=1,2$. Entonces, con la finalidad de encontrar el n\'umero de usuarios
presentes en el sistema cuando el servidor deja de atender una de
las colas de cualquier sistema se tiene lo siguiente


\begin{eqnarray*}
&&\esp\left[z_{1}^{L_{1}\left(\overline{\tau}_{1}\right)}z_{2}^{L_{2}\left(\overline{\tau}_{1}\right)}w_{1}^{\hat{L}_{1}\left(\overline{\tau}_{1}\right)}w_{2}^{\hat{L}_{2}\left(\overline{\tau}_{1}\right)}\right]=
\esp\left[z_{2}^{L_{2}\left(\overline{\tau}_{1}\right)}w_{1}^{\hat{L}_{1}\left(\overline{\tau}_{1}
\right)}w_{2}^{\hat{L}_{2}\left(\overline{\tau}_{1}\right)}\right]\\
&=&\esp\left[z_{2}^{L_{2}\left(\tau_{1}\right)+X_{2}\left(\overline{\tau}_{1}-\tau_{1}\right)+Y_{2}\left(\overline{\tau}_{1}-\tau_{1}\right)}w_{1}^{\hat{L}_{1}\left(\tau_{1}\right)+\hat{X}_{1}\left(\overline{\tau}_{1}-\tau_{1}\right)}w_{2}^{\hat{L}_{2}\left(\tau_{1}\right)+\hat{X}_{2}\left(\overline{\tau}_{1}-\tau_{1}\right)}\right]
\end{eqnarray*}
utilizando la ecuacion dada (\ref{Eq.UsuariosTotalesZ2}), luego


\begin{eqnarray*}
&=&\esp\left[z_{2}^{L_{2}\left(\tau_{1}\right)}z_{2}^{X_{2}\left(\overline{\tau}_{1}-\tau_{1}\right)}z_{2}^{Y_{2}\left(\overline{\tau}_{1}-\tau_{1}\right)}w_{1}^{\hat{L}_{1}\left(\tau_{1}\right)}w_{1}^{\hat{X}_{1}\left(\overline{\tau}_{1}-\tau_{1}\right)}w_{2}^{\hat{L}_{2}\left(\tau_{1}\right)}w_{2}^{\hat{X}_{2}\left(\overline{\tau}_{1}-\tau_{1}\right)}\right]\\
&=&\esp\left[z_{2}^{L_{2}\left(\tau_{1}\right)}\left\{w_{1}^{\hat{L}_{1}\left(\tau_{1}\right)}w_{2}^{\hat{L}_{2}\left(\tau_{1}\right)}\right\}\left\{z_{2}^{X_{2}\left(\overline{\tau}_{1}-\tau_{1}\right)}
z_{2}^{Y_{2}\left(\overline{\tau}_{1}-\tau_{1}\right)}w_{1}^{\hat{X}_{1}\left(\overline{\tau}_{1}-\tau_{1}\right)}w_{2}^{\hat{X}_{2}\left(\overline{\tau}_{1}-\tau_{1}\right)}\right\}\right]\\
\end{eqnarray*}
Aplicando el hecho de que el n\'umero de usuarios que llegan a cada una de las colas del segundo sistema es independiente de las llegadas a las colas del primer sistema:

\begin{eqnarray*}
&=&\esp\left[z_{2}^{L_{2}\left(\tau_{1}\right)}\left\{z_{2}^{X_{2}\left(\overline{\tau}_{1}-\tau_{1}\right)}z_{2}^{Y_{2}\left(\overline{\tau}_{1}-\tau_{1}\right)}w_{1}^{\hat{X}_{1}\left(\overline{\tau}_{1}-\tau_{1}\right)}w_{2}^{\hat{X}_{2}\left(\overline{\tau}_{1}-\tau_{1}\right)}\right\}\right]\esp\left[w_{1}^{\hat{L}_{1}\left(\tau_{1}\right)}w_{2}^{\hat{L}_{2}\left(\tau_{1}\right)}\right]
\end{eqnarray*}
dado que los arribos a cada una de las colas son independientes, podemos separar la esperanza para los procesos de llegada a $Q_{1}$ y $Q_{2}$ al tiempo $\tau_{1}$, que es el tiempo en que el servidor visita a $Q_{1}$. Recordando que $\tilde{X}_{2}\left(z_{2}\right)=X_{2}\left(z_{2}\right)+Y_{2}\left(z_{2}\right)$ se tiene


\begin{eqnarray*}
&=&\esp\left[z_{2}^{L_{2}\left(\tau_{1}\right)}\left\{z_{2}^{\tilde{X}_{2}\left(\overline{\tau}_{1}-\tau_{1}\right)}w_{1}^{\hat{X}_{1}\left(\overline{\tau}_{1}-\tau_{1}\right)}w_{2}^{\hat{X}_{2}\left(\overline{\tau}_{1}-\tau_{1}\right)}\right\}\right]\esp\left[w_{1}^{\hat{L}_{1}\left(\tau_{1}\right)}w_{2}^{\hat{L}_{2}\left(\tau_{1}\right)}\right]\\
&=&\esp\left[z_{2}^{L_{2}\left(\tau_{1}\right)}\left\{\tilde{P}_{2}\left(z_{2}\right)^{\overline{\tau}_{1}-\tau_{1}}\hat{P}_{1}\left(w_{1}\right)^{\overline{\tau}_{1}-\tau_{1}}\hat{P}_{2}\left(w_{2}\right)^{\overline{\tau}_{1}-\tau_{1}}\right\}\right]\esp\left[w_{1}^{\hat{L}_{1}\left(\tau_{1}\right)}w_{2}^{\hat{L}_{2}\left(\tau_{1}\right)}\right]\\
&=&\esp\left[z_{2}^{L_{2}\left(\tau_{1}\right)}\left\{\tilde{P}_{2}\left(z_{2}\right)\hat{P}_{1}\left(w_{1}\right)\hat{P}_{2}\left(w_{2}\right)\right\}^{\overline{\tau}_{1}-\tau_{1}}\right]\esp\left[w_{1}^{\hat{L}_{1}\left(\tau_{1}\right)}w_{2}^{\hat{L}_{2}\left(\tau_{1}\right)}\right]\\
&=&\esp\left[z_{2}^{L_{2}\left(\tau_{1}\right)}\theta_{1}\left(\tilde{P}_{2}\left(z_{2}\right)\hat{P}_{1}\left(w_{1}\right)\hat{P}_{2}\left(w_{2}\right)\right)^{L_{1}\left(\tau_{1}\right)}\right]\esp\left[w_{1}^{\hat{L}_{1}\left(\tau_{1}\right)}w_{2}^{\hat{L}_{2}\left(\tau_{1}\right)}\right]\\
&=&F_{1}\left(\theta_{1}\left(\tilde{P}_{2}\left(z_{2}\right)\hat{P}_{1}\left(w_{1}\right)\hat{P}_{2}\left(w_{2}\right)\right),z{2}\right)\hat{F}_{1}\left(w_{1},w_{2};\tau_{1}\right)\\
&\equiv&
F_{1}\left(\theta_{1}\left(\tilde{P}_{2}\left(z_{2}\right)\hat{P}_{1}\left(w_{1}\right)\hat{P}_{2}\left(w_{2}\right)\right),z_{2},w_{1},w_{2}\right)
\end{eqnarray*}

Las igualdades anteriores son ciertas pues el n\'umero de usuarios
que llegan a $\hat{Q}_{2}$ durante el intervalo
$\left[\tau_{1},\overline{\tau}_{1}\right]$ a\'un no han sido
atendidos por el servidor del sistema $2$ y por tanto a\'un no
pueden pasar al sistema $1$ a traves de $Q_{2}$. Por tanto el n\'umero de
usuarios que pasan de $\hat{Q}_{2}$ a $Q_{2}$ en el intervalo de
tiempo $\left[\tau_{1},\overline{\tau}_{1}\right]$ depende de la
pol\'itica de traslado entre los dos sistemas, conforme a la
secci\'on anterior.\smallskip

Por lo tanto
\begin{eqnarray}\label{Eq.Fs}
\esp\left[z_{1}^{L_{1}\left(\overline{\tau}_{1}\right)}z_{2}^{L_{2}\left(\overline{\tau}_{1}
\right)}w_{1}^{\hat{L}_{1}\left(\overline{\tau}_{1}\right)}w_{2}^{\hat{L}_{2}\left(
\overline{\tau}_{1}\right)}\right]&=&F_{1}\left(\theta_{1}\left(\tilde{P}_{2}\left(z_{2}\right)
\hat{P}_{1}\left(w_{1}\right)\hat{P}_{2}\left(w_{2}\right)\right),z_{2},w_{1},w_{2}\right)\\
&=&F_{1}\left(\theta_{1}\left(\tilde{P}_{2}\left(z_{2}\right)\hat{P}_{1}\left(w_{1}\right)\hat{P}_{2}\left(w_{2}\right)\right),z{2}\right)\hat{F}_{1}\left(w_{1},w_{2};\tau_{1}\right)
\end{eqnarray}


Utilizando un razonamiento an\'alogo para $\overline{\tau}_{2}$:



\begin{eqnarray*}
&&\esp\left[z_{1}^{L_{1}\left(\overline{\tau}_{2}\right)}z_{2}^{L_{2}\left(\overline{\tau}_{2}\right)}w_{1}^{\hat{L}_{1}\left(\overline{\tau}_{2}\right)}w_{2}^{\hat{L}_{2}\left(\overline{\tau}_{2}\right)}\right]=
\esp\left[z_{1}^{L_{1}\left(\overline{\tau}_{2}\right)}w_{1}^{\hat{L}_{1}\left(\overline{\tau}_{2}\right)}w_{2}^{\hat{L}_{2}\left(\overline{\tau}_{2}\right)}\right]\\
&=&\esp\left[z_{1}^{L_{1}\left(\tau_{2}\right)+X_{1}\left(\overline{\tau}_{2}-\tau_{2}\right)}w_{1}^{\hat{L}_{1}\left(\tau_{2}\right)+\hat{X}_{1}\left(\overline{\tau}_{2}-\tau_{2}\right)}w_{2}^{\hat{L}_{2}\left(\tau_{2}\right)+\hat{X}_{2}\left(\overline{\tau}_{2}-\tau_{2}\right)}\right]\\
&=&\esp\left[z_{1}^{L_{1}\left(\tau_{2}\right)}z_{1}^{X_{1}\left(\overline{\tau}_{2}-\tau_{2}\right)}w_{1}^{\hat{L}_{1}\left(\tau_{2}\right)}w_{1}^{\hat{X}_{1}\left(\overline{\tau}_{2}-\tau_{2}\right)}w_{2}^{\hat{L}_{2}\left(\tau_{2}\right)}w_{2}^{\hat{X}_{2}\left(\overline{\tau}_{2}-\tau_{2}\right)}\right]\\
&=&\esp\left[z_{1}^{L_{1}\left(\tau_{2}\right)}z_{1}^{X_{1}\left(\overline{\tau}_{2}-\tau_{2}\right)}w_{1}^{\hat{X}_{1}\left(\overline{\tau}_{2}-\tau_{2}\right)}w_{2}^{\hat{X}_{2}\left(\overline{\tau}_{2}-\tau_{2}\right)}\right]\esp\left[w_{1}^{\hat{L}_{1}\left(\tau_{2}\right)}w_{2}^{\hat{L}_{2}\left(\tau_{2}\right)}\right]\\
&=&\esp\left[z_{1}^{L_{1}\left(\tau_{2}\right)}P_{1}\left(z_{1}\right)^{\overline{\tau}_{2}-\tau_{2}}\hat{P}_{1}\left(w_{1}\right)^{\overline{\tau}_{2}-\tau_{2}}\hat{P}_{2}\left(w_{2}\right)^{\overline{\tau}_{2}-\tau_{2}}\right]
\esp\left[w_{1}^{\hat{L}_{1}\left(\tau_{2}\right)}w_{2}^{\hat{L}_{2}\left(\tau_{2}\right)}\right]
\end{eqnarray*}
utlizando la proposici\'on (\ref{Prop.1.1.2S}) referente al problema de la ruina del jugador:


\begin{eqnarray*}
&=&\esp\left[z_{1}^{L_{1}\left(\tau_{2}\right)}\left\{P_{1}\left(z_{1}\right)\hat{P}_{1}\left(w_{1}\right)\hat{P}_{2}\left(w_{2}\right)\right\}^{\overline{\tau}_{2}-\tau_{2}}\right]
\esp\left[w_{1}^{\hat{L}_{1}\left(\tau_{2}\right)}w_{2}^{\hat{L}_{2}\left(\tau_{2}\right)}\right]\\
&=&\esp\left[z_{1}^{L_{1}\left(\tau_{2}\right)}\tilde{\theta}_{2}\left(P_{1}\left(z_{1}\right)\hat{P}_{1}\left(w_{1}\right)\hat{P}_{2}\left(w_{2}\right)\right)^{L_{2}\left(\tau_{2}\right)}\right]
\esp\left[w_{1}^{\hat{L}_{1}\left(\tau_{2}\right)}w_{2}^{\hat{L}_{2}\left(\tau_{2}\right)}\right]\\
&=&F_{2}\left(z_{1},\tilde{\theta}_{2}\left(P_{1}\left(z_{1}\right)\hat{P}_{1}\left(w_{1}\right)\hat{P}_{2}\left(w_{2}\right)\right)\right)
\hat{F}_{2}\left(w_{1},w_{2};\tau_{2}\right)\\
\end{eqnarray*}


entonces se define
\begin{eqnarray}
\esp\left[z_{1}^{L_{1}\left(\overline{\tau}_{2}\right)}z_{2}^{L_{2}\left(\overline{\tau}_{2}\right)}w_{1}^{\hat{L}_{1}\left(\overline{\tau}_{2}\right)}w_{2}^{\hat{L}_{2}\left(\overline{\tau}_{2}\right)}\right]=F_{2}\left(z_{1},\tilde{\theta}_{2}\left(P_{1}\left(z_{1}\right)\hat{P}_{1}\left(w_{1}\right)\hat{P}_{2}\left(w_{2}\right)\right),w_{1},w_{2}\right)\\
\equiv F_{2}\left(z_{1},\tilde{\theta}_{2}\left(P_{1}\left(z_{1}\right)\hat{P}_{1}\left(w_{1}\right)\hat{P}_{2}\left(w_{2}\right)\right)\right)
\hat{F}_{2}\left(w_{1},w_{2};\tau_{2}\right)
\end{eqnarray}

Ahora para $\overline{\zeta}_{1}:$

\begin{eqnarray*}
&&\esp\left[z_{1}^{L_{1}\left(\overline{\zeta}_{1}\right)}z_{2}^{L_{2}\left(\overline{\zeta}_{1}\right)}w_{1}^{\hat{L}_{1}\left(\overline{\zeta}_{1}\right)}w_{2}^{\hat{L}_{2}\left(\overline{\zeta}_{1}\right)}\right]=
\esp\left[z_{1}^{L_{1}\left(\overline{\zeta}_{1}\right)}z_{2}^{L_{2}\left(\overline{\zeta}_{1}\right)}w_{2}^{\hat{L}_{2}\left(\overline{\zeta}_{1}\right)}\right]\\
%&=&\esp\left[z_{1}^{L_{1}\left(\zeta_{1}\right)+X_{1}\left(\overline{\zeta}_{1}-\zeta_{1}\right)}z_{2}^{L_{2}\left(\zeta_{1}\right)+X_{2}\left(\overline{\zeta}_{1}-\zeta_{1}\right)+\hat{Y}_{2}\left(\overline{\zeta}_{1}-\zeta_{1}\right)}w_{2}^{\hat{L}_{2}\left(\zeta_{1}\right)+\hat{X}_{2}\left(\overline{\zeta}_{1}-\zeta_{1}\right)}\right]\\
&=&\esp\left[z_{1}^{L_{1}\left(\zeta_{1}\right)}z_{1}^{X_{1}\left(\overline{\zeta}_{1}-\zeta_{1}\right)}z_{2}^{L_{2}\left(\zeta_{1}\right)}z_{2}^{X_{2}\left(\overline{\zeta}_{1}-\zeta_{1}\right)}
z_{2}^{Y_{2}\left(\overline{\zeta}_{1}-\zeta_{1}\right)}w_{2}^{\hat{L}_{2}\left(\zeta_{1}\right)}w_{2}^{\hat{X}_{2}\left(\overline{\zeta}_{1}-\zeta_{1}\right)}\right]\\
&=&\esp\left[z_{1}^{L_{1}\left(\zeta_{1}\right)}z_{2}^{L_{2}\left(\zeta_{1}\right)}\right]\esp\left[z_{1}^{X_{1}\left(\overline{\zeta}_{1}-\zeta_{1}\right)}z_{2}^{\tilde{X}_{2}\left(\overline{\zeta}_{1}-\zeta_{1}\right)}w_{2}^{\hat{X}_{2}\left(\overline{\zeta}_{1}-\zeta_{1}\right)}w_{2}^{\hat{L}_{2}\left(\zeta_{1}\right)}\right]\\
&=&\esp\left[z_{1}^{L_{1}\left(\zeta_{1}\right)}z_{2}^{L_{2}\left(\zeta_{1}\right)}\right]
\esp\left[P_{1}\left(z_{1}\right)^{\overline{\zeta}_{1}-\zeta_{1}}\tilde{P}_{2}\left(z_{2}\right)^{\overline{\zeta}_{1}-\zeta_{1}}\hat{P}_{2}\left(w_{2}\right)^{\overline{\zeta}_{1}-\zeta_{1}}w_{2}^{\hat{L}_{2}\left(\zeta_{1}\right)}\right]\\
&=&\esp\left[z_{1}^{L_{1}\left(\zeta_{1}\right)}z_{2}^{L_{2}\left(\zeta_{1}\right)}\right]
\esp\left[\left\{P_{1}\left(z_{1}\right)\tilde{P}_{2}\left(z_{2}\right)\hat{P}_{2}\left(w_{2}\right)\right\}^{\overline{\zeta}_{1}-\zeta_{1}}w_{2}^{\hat{L}_{2}\left(\zeta_{1}\right)}\right]\\
&=&\esp\left[z_{1}^{L_{1}\left(\zeta_{1}\right)}z_{2}^{L_{2}\left(\zeta_{1}\right)}\right]
\esp\left[\hat{\theta}_{1}\left(P_{1}\left(z_{1}\right)\tilde{P}_{2}\left(z_{2}\right)\hat{P}_{2}\left(w_{2}\right)\right)^{\hat{L}_{1}\left(\zeta_{1}\right)}w_{2}^{\hat{L}_{2}\left(\zeta_{1}\right)}\right]\\
&=&F_{1}\left(z_{1},z_{2};\zeta_{1}\right)\hat{F}_{1}\left(\hat{\theta}_{1}\left(P_{1}\left(z_{1}\right)\tilde{P}_{2}\left(z_{2}\right)\hat{P}_{2}\left(w_{2}\right)\right),w_{2}\right)
\end{eqnarray*}


es decir,

\begin{eqnarray}
\esp\left[z_{1}^{L_{1}\left(\overline{\zeta}_{1}\right)}z_{2}^{L_{2}\left(\overline{\zeta}_{1}
\right)}w_{1}^{\hat{L}_{1}\left(\overline{\zeta}_{1}\right)}w_{2}^{\hat{L}_{2}\left(
\overline{\zeta}_{1}\right)}\right]&=&\hat{F}_{1}\left(z_{1},z_{2},\hat{\theta}_{1}\left(P_{1}\left(z_{1}\right)\tilde{P}_{2}\left(z_{2}\right)\hat{P}_{2}\left(w_{2}\right)\right),w_{2}\right)\\
&=&F_{1}\left(z_{1},z_{2};\zeta_{1}\right)\hat{F}_{1}\left(\hat{\theta}_{1}\left(P_{1}\left(z_{1}\right)\tilde{P}_{2}\left(z_{2}\right)\hat{P}_{2}\left(w_{2}\right)\right),w_{2}\right).
\end{eqnarray}


Finalmente para $\overline{\zeta}_{2}:$
\begin{eqnarray*}
&&\esp\left[z_{1}^{L_{1}\left(\overline{\zeta}_{2}\right)}z_{2}^{L_{2}\left(\overline{\zeta}_{2}\right)}w_{1}^{\hat{L}_{1}\left(\overline{\zeta}_{2}\right)}w_{2}^{\hat{L}_{2}\left(\overline{\zeta}_{2}\right)}\right]=
\esp\left[z_{1}^{L_{1}\left(\overline{\zeta}_{2}\right)}z_{2}^{L_{2}\left(\overline{\zeta}_{2}\right)}w_{1}^{\hat{L}_{1}\left(\overline{\zeta}_{2}\right)}\right]\\
%&=&\esp\left[z_{1}^{L_{1}\left(\zeta_{2}\right)+X_{1}\left(\overline{\zeta}_{2}-\zeta_{2}\right)}z_{2}^{L_{2}\left(\zeta_{2}\right)+X_{2}\left(\overline{\zeta}_{2}-\zeta_{2}\right)+\hat{Y}_{2}\left(\overline{\zeta}_{2}-\zeta_{2}\right)}w_{1}^{\hat{L}_{1}\left(\zeta_{2}\right)+\hat{X}_{1}\left(\overline{\zeta}_{2}-\zeta_{2}\right)}\right]\\
&=&\esp\left[z_{1}^{L_{1}\left(\zeta_{2}\right)}z_{1}^{X_{1}\left(\overline{\zeta}_{2}-\zeta_{2}\right)}z_{2}^{L_{2}\left(\zeta_{2}\right)}z_{2}^{X_{2}\left(\overline{\zeta}_{2}-\zeta_{2}\right)}
z_{2}^{Y_{2}\left(\overline{\zeta}_{2}-\zeta_{2}\right)}w_{1}^{\hat{L}_{1}\left(\zeta_{2}\right)}w_{1}^{\hat{X}_{1}\left(\overline{\zeta}_{2}-\zeta_{2}\right)}\right]\\
&=&\esp\left[z_{1}^{L_{1}\left(\zeta_{2}\right)}z_{2}^{L_{2}\left(\zeta_{2}\right)}\right]\esp\left[z_{1}^{X_{1}\left(\overline{\zeta}_{2}-\zeta_{2}\right)}z_{2}^{\tilde{X}_{2}\left(\overline{\zeta}_{2}-\zeta_{2}\right)}w_{1}^{\hat{X}_{1}\left(\overline{\zeta}_{2}-\zeta_{2}\right)}w_{1}^{\hat{L}_{1}\left(\zeta_{2}\right)}\right]\\
&=&\esp\left[z_{1}^{L_{1}\left(\zeta_{2}\right)}z_{2}^{L_{2}\left(\zeta_{2}\right)}\right]\esp\left[P_{1}\left(z_{1}\right)^{\overline{\zeta}_{2}-\zeta_{2}}\tilde{P}_{2}\left(z_{2}\right)^{\overline{\zeta}_{2}-\zeta_{2}}\hat{P}\left(w_{1}\right)^{\overline{\zeta}_{2}-\zeta_{2}}w_{1}^{\hat{L}_{1}\left(\zeta_{2}\right)}\right]\\
&=&\esp\left[z_{1}^{L_{1}\left(\zeta_{2}\right)}z_{2}^{L_{2}\left(\zeta_{2}\right)}\right]\esp\left[w_{1}^{\hat{L}_{1}\left(\zeta_{2}\right)}\left\{P_{1}\left(z_{1}\right)\tilde{P}_{2}\left(z_{2}\right)\hat{P}\left(w_{1}\right)\right\}^{\overline{\zeta}_{2}-\zeta_{2}}\right]\\
&=&\esp\left[z_{1}^{L_{1}\left(\zeta_{2}\right)}z_{2}^{L_{2}\left(\zeta_{2}\right)}\right]\esp\left[w_{1}^{\hat{L}_{1}\left(\zeta_{2}\right)}\hat{\theta}_{2}\left(P_{1}\left(z_{1}\right)\tilde{P}_{2}\left(z_{2}\right)\hat{P}\left(w_{1}\right)\right)^{\hat{L}_{2}\zeta_{2}}\right]\\
&=&F_{2}\left(z_{1},z_{2};\zeta_{2}\right)\hat{F}_{2}\left(w_{1},\hat{\theta}_{2}\left(P_{1}\left(z_{1}\right)\tilde{P}_{2}\left(z_{2}\right)\hat{P}_{1}\left(w_{1}\right)\right)\right]\\
%&\equiv&\hat{F}_{2}\left(z_{1},z_{2},w_{1},\hat{\theta}_{2}\left(P_{1}\left(z_{1}\right)\tilde{P}_{2}\left(z_{2}\right)\hat{P}_{1}\left(w_{1}\right)\right)\right)
\end{eqnarray*}

es decir
\begin{eqnarray}
\esp\left[z_{1}^{L_{1}\left(\overline{\zeta}_{2}\right)}z_{2}^{L_{2}\left(\overline{\zeta}_{2}\right)}w_{1}^{\hat{L}_{1}\left(\overline{\zeta}_{2}\right)}w_{2}^{\hat{L}_{2}\left(\overline{\zeta}_{2}\right)}\right]&=&\hat{F}_{2}\left(z_{1},z_{2},w_{1},\hat{\theta}_{2}\left(P_{1}\left(z_{1}\right)\tilde{P}_{2}\left(z_{2}\right)\hat{P}_{1}\left(w_{1}\right)\right)\right)\\
&=&F_{2}\left(z_{1},z_{2};\zeta_{2}\right)\hat{F}_{2}\left(w_{1},\hat{\theta}_{2}\left(P_{1}\left(z_{1}\right)\tilde{P}_{2}\left(z_{2}\right)\hat{P}_{1}\left(w_{1}
\right)\right)\right)
\end{eqnarray}
%__________________________________________________________________________
\section{Ecuaciones Recursivas para la R.S.V.C.}
%__________________________________________________________________________




Con lo desarrollado hasta ahora podemos encontrar las ecuaciones
recursivas que modelan la Red de Sistemas de Visitas C\'iclicas
(R.S.V.C):
\begin{eqnarray*}
F_{2}\left(z_{1},z_{2},w_{1},w_{2}\right)&=&R_{1}\left(z_{1},z_{2},w_{1},w_{2}\right)\esp\left[z_{1}^{L_{1}\left(
\overline{\tau}_{1}\right)}z_{2}^{L_{2}\left(\overline{\tau}_{1}\right)}w_{1}^{\hat{L}_{1}\left(\overline{\tau}_{1}\right)}
w_{2}^{\hat{L}_{2}\left(\overline{\tau}_{1}\right)}\right]\\
&=&R_{1}\left(P_{1}\left(z_{1}\right)\tilde{P}_{2}\left(z_{2}\right)\prod_{i=1}^{2}
\hat{P}_{i}\left(w_{i}\right)\right)F_{1}\left(\theta_{1}\left(\tilde{P}_{2}\left(z_{2}\right)\hat{P}_{1}\left(w_{1}\right)\hat{P}_{2}\left(w_{2}\right)\right),z{2}\right)\hat{F}_{1}\left(w_{1},w_{2};\tau_{1}\right)
\end{eqnarray*}


\begin{eqnarray*}
F_{1}\left(z_{1},z_{2},w_{1},w_{2}\right)&=&R_{2}\left(z_{1},z_{2},w_{1},w_{2}\right)\esp\left[z_{1}^{L_{1}\left(
\overline{\tau}_{2}\right)}z_{2}^{L_{2}\left(\overline{\tau}_{2}\right)}w_{1}^{\hat{L}_{1}\left(\overline{\tau}_{2}\right)}
w_{2}^{\hat{L}_{2}\left(\overline{\tau}_{1}\right)}\right]\\
&=&R_{2}\left(P_{1}\left(z_{1}\right)\tilde{P}_{2}\left(z_{2}\right)\prod_{i=1}^{2}
\hat{P}_{i}\left(w_{i}\right)\right)F_{2}\left(z_{1},\tilde{\theta}_{2}\left(P_{1}\left(z_{1}\right)\hat{P}_{1}\left(w_{1}\right)\hat{P}_{2}\left(w_{2}\right)\right)\right)
\hat{F}_{2}\left(w_{1},w_{2};\tau_{2}\right)
\end{eqnarray*}

\begin{eqnarray*}
\hat{F}_{2}\left(z_{1},z_{2},w_{1},w_{2}\right)&=&\hat{R}_{1}\left(z_{1},z_{2},w_{1},w_{2}\right)\esp\left[z_{1}^{L_{1}\left(\overline{\zeta}_{1}\right)}z_{2}^{L_{2}\left(\overline{\zeta}_{1}\right)}w_{1}^{\hat{L}_{1}\left(\overline{\zeta}_{1}\right)}w_{2}^{\hat{L}_{2}\left(\overline{\zeta}_{1}\right)}\right]\\
&=&\hat{R}_{1}\left(P_{1}\left(z_{1}\right)\tilde{P}_{2}\left(z_{2}\right)\prod_{i=1}^{2}
\hat{P}_{i}\left(w_{i}\right)\right)F_{1}\left(z_{1},z_{2};\zeta_{1}\right)\hat{F}_{1}\left(\hat{\theta}_{1}\left(P_{1}\left(z_{1}\right)\tilde{P}_{2}\left(z_{2}\right)\hat{P}_{2}\left(w_{2}\right)\right),w_{2}\right)
\end{eqnarray*}

\begin{eqnarray*}
\hat{F}_{1}\left(z_{1},z_{2},w_{1},w_{2}\right)&=&\hat{R}_{2}\left(z_{1},z_{2},
w_{1},w_{2}\right)\esp\left[z_{1}^{L_{1}\left(\overline{\zeta}_{2}\right)}z_{2}
^{L_{2}\left(\overline{\zeta}_{2}\right)}w_{1}^{\hat{L}_{1}\left(
\overline{\zeta}_{2}\right)}w_{2}^{\hat{L}_{2}\left(\overline{\zeta}_{2}\right)}
\right]\\
&=&\hat{R}_{2}\left(P_{1}\left(z_{1}\right)\tilde{P}_{2}\left(z_{2}\right)\prod_{i=1}^{2}
\hat{P}_{i}\left(w_{i}\right)\right)F_{2}\left(z_{1},z_{2};\zeta_{2}\right)\hat{F}_{2}\left(w_{1},\hat{\theta}_{2}\left(P_{1}\left(z_{1}\right)\tilde{P}_{2}\left(z_{2}\right)\hat{P}_{1}\left(w_{1}
\right)\right)\right)
\end{eqnarray*}



%_________________________________________________________________________________________________
\subsection{Tiempos de Traslado del Servidor}
%_________________________________________________________________________________________________


Para
%\begin{multicols}{2}

\begin{eqnarray}\label{Ec.R1}
R_{1}\left(\mathbf{z,w}\right)=R_{1}\left(P_{1}\left(z_{1}\right)\tilde{P}_{2}\left(z_{2}\right)\hat{P}_{1}\left(w_{1}\right)\hat{P}_{2}\left(w_{2}\right)\right)
\end{eqnarray}
%\end{multicols}

se tiene que


\begin{eqnarray*}
\begin{array}{llll}
\frac{\partial R_{1}\left(\mathbf{z,w}\right)}{\partial
z_{1}}|_{\mathbf{z,w}=1}=r_{1}\mu_{1},&
\frac{\partial R_{1}\left(\mathbf{z,w}\right)}{\partial
z_{2}}|_{\mathbf{z,w}=1}=r_{1}\tilde{\mu}_{2},&
\frac{\partial R_{1}\left(\mathbf{z,w}\right)}{\partial
w_{1}}|_{\mathbf{z,w}=1}=r_{1}\hat{\mu}_{1},&
\frac{\partial R_{1}\left(\mathbf{z,w}\right)}{\partial
w_{2}}|_{\mathbf{z,w}=1}=r_{1}\hat{\mu}_{2},
\end{array}
\end{eqnarray*}

An\'alogamente se tiene

\begin{eqnarray}
R_{2}\left(\mathbf{z,w}\right)=R_{2}\left(P_{1}\left(z_{1}\right)\tilde{P}_{2}\left(z_{2}\right)\hat{P}_{1}\left(w_{1}\right)\hat{P}_{2}\left(w_{2}\right)\right)
\end{eqnarray}


\begin{eqnarray*}
\begin{array}{llll}
\frac{\partial R_{2}\left(\mathbf{z,w}\right)}{\partial
z_{1}}|_{\mathbf{z,w}=1}=r_{2}\mu_{1},&
\frac{\partial R_{2}\left(\mathbf{z,w}\right)}{\partial
z_{2}}|_{\mathbf{z,w}=1}=r_{2}\tilde{\mu}_{2},&
\frac{\partial R_{2}\left(\mathbf{z,w}\right)}{\partial
w_{1}}|_{\mathbf{z,w}=1}=r_{2}\hat{\mu}_{1},&
\frac{\partial R_{2}\left(\mathbf{z,w}\right)}{\partial
w_{2}}|_{\mathbf{z,w}=1}=r_{2}\hat{\mu}_{2},\\
\end{array}
\end{eqnarray*}

Para el segundo sistema:

\begin{eqnarray}
\hat{R}_{1}\left(\mathbf{z,w}\right)=\hat{R}_{1}\left(P_{1}\left(z_{1}\right)\tilde{P}_{2}\left(z_{2}\right)\hat{P}_{1}\left(w_{1}\right)\hat{P}_{2}\left(w_{2}\right)\right)
\end{eqnarray}


\begin{eqnarray*}
\begin{array}{llll}
\frac{\partial \hat{R}_{1}\left(\mathbf{z,w}\right)}{\partial
z_{1}}|_{\mathbf{z,w}=1}=\hat{r}_{1}\mu_{1},&
\frac{\partial \hat{R}_{1}\left(\mathbf{z,w}\right)}{\partial
z_{2}}|_{\mathbf{z,w}=1}=\hat{r}_{1}\tilde{\mu}_{2},&
\frac{\partial \hat{R}_{1}\left(\mathbf{z,w}\right)}{\partial
w_{1}}|_{\mathbf{z,w}=1}=\hat{r}_{1}\hat{\mu}_{1},&
\frac{\partial \hat{R}_{1}\left(\mathbf{z,w}\right)}{\partial
w_{2}}|_{\mathbf{z,w}=1}=\hat{r}_{1}\hat{\mu}_{2},
\end{array}
\end{eqnarray*}

Finalmente

\begin{eqnarray}
\hat{R}_{2}\left(\mathbf{z,w}\right)=\hat{R}_{2}\left(P_{1}\left(z_{1}\right)\tilde{P}_{2}\left(z_{2}\right)\hat{P}_{1}\left(w_{1}\right)\hat{P}_{2}\left(w_{2}\right)\right)
\end{eqnarray}



\begin{eqnarray*}
\begin{array}{llll}
\frac{\partial \hat{R}_{2}\left(\mathbf{z,w}\right)}{\partial
z_{1}}|_{\mathbf{z,w}=1}=\hat{r}_{2}\mu_{1},&
\frac{\partial \hat{R}_{2}\left(\mathbf{z,w}\right)}{\partial
z_{2}}|_{\mathbf{z,w}=1}=\hat{r}_{2}\tilde{\mu}_{2},&
\frac{\partial \hat{R}_{2}\left(\mathbf{z,w}\right)}{\partial
w_{1}}|_{\mathbf{z,w}=1}=\hat{r}_{2}\hat{\mu}_{1},&
\frac{\partial \hat{R}_{2}\left(\mathbf{z,w}\right)}{\partial
w_{2}}|_{\mathbf{z,w}=1}=\hat{r}_{2}\hat{\mu}_{2}.
\end{array}
\end{eqnarray*}


%_________________________________________________________________________________________________
\subsection{Usuarios presentes en la cola}
%_________________________________________________________________________________________________

Hagamos lo correspondiente con las siguientes
expresiones obtenidas en la secci\'on anterior, recordemos que

\begin{eqnarray*}
F_{1}\left(\theta_{1}\left(\tilde{P}_{2}\left(z_{2}\right)\hat{P}_{1}\left(w_{1}\right)
\hat{P}_{2}\left(w_{2}\right)\right),z_{2},w_{1},w_{2}\right)=
F_{1}\left(\theta_{1}\left(\tilde{P}_{2}\left(z_{2}\right)\hat{P}_{1}\left(w_{1}
\right)\hat{P}_{2}\left(w_{2}\right)\right),z_{2}\right)
\hat{F}_{1}\left(w_{1},w_{2};\tau_{1}\right)\\
\end{eqnarray*}

entonces

\begin{eqnarray*}
\frac{\partial F_{1}\left(\theta_{1}\left(\tilde{P}_{2}\left(z_{2}\right)\hat{P}_{1}\left(w_{1}\right)\hat{P}_{2}\left(w_{2}\right)\right),z_{2},w_{1},w_{2}\right)}{\partial z_{1}}|_{\mathbf{z},\mathbf{w}=1}&=&0\\
\frac{\partial
F_{1}\left(\theta_{1}\left(\tilde{P}_{2}\left(z_{2}\right)\hat{P}_{1}\left(w_{1}\right)\hat{P}_{2}\left(w_{2}\right)\right),z_{2},w_{1},w_{2}\right)}{\partial
z_{2}}|_{\mathbf{z},\mathbf{w}=1}&=&\frac{\partial F_{1}}{\partial
z_{1}}\cdot\frac{\partial \theta_{1}}{\partial
\tilde{P}_{2}}\cdot\frac{\partial \tilde{P}_{2}}{\partial
z_{2}}+\frac{\partial F_{1}}{\partial z_{2}}=f_{1}\left(1\right)\left(\frac{1}{1-\mu_{1}}\right)\tilde{\mu}_{2}+f_{1}\left(2\right)\\
\frac{\partial
F_{1}\left(\theta_{1}\left(\tilde{P}_{2}\left(z_{2}\right)\hat{P}_{1}\left(w_{1}\right)\hat{P}_{2}\left(w_{2}\right)\right),z_{2},w_{1},w_{2}\right)}{\partial
w_{1}}|_{\mathbf{z},\mathbf{w}=1}&=&\frac{\partial F_{1}}{\partial
z_{1}}\cdot\frac{\partial
\theta_{1}}{\partial\hat{P}_{1}}\cdot\frac{\partial\hat{P}_{1}}{\partial
w_{1}}+\frac{\partial\hat{F}_{1}}{\partial w_{1}}=f_{1}\left(1\right)\left(\frac{1}{1-\mu_{1}}\right)\hat{\mu}_{1}+\hat{F}_{1,1}^{(1)}\left(1\right)\\
\frac{\partial
F_{1}\left(\theta_{1}\left(\tilde{P}_{2}\left(z_{2}\right)\hat{P}_{1}\left(w_{1}\right)\hat{P}_{2}\left(w_{2}\right)\right),z_{2},w_{1},w_{2}\right)}{\partial
w_{2}}|_{\mathbf{z},\mathbf{w}=1}&=&\frac{\partial F_{1}}{\partial
z_{1}}\cdot\frac{\partial\theta_{1}}{\partial\hat{P}_{2}}\cdot\frac{\partial\hat{P}_{2}}{\partial
w_{2}}+\frac{\partial \hat{F}_{1}}{\partial w_{2}}=f_{1}\left(1\right)\left(\frac{1}{1-\mu_{1}}\right)\hat{\mu}_{2}+\hat{F}_{2,1}^{(1)}\left(1\right)\\
\end{eqnarray*}

para $\tau_{2}$:

\begin{eqnarray*}
F_{2}\left(z_{1},\tilde{\theta}_{2}\left(P_{1}\left(z_{1}\right)\hat{P}_{1}\left(w_{1}\right)\hat{P}_{2}\left(w_{2}\right)\right),
w_{1},w_{2}\right)=F_{2}\left(z_{1},\tilde{\theta}_{2}\left(P_{1}\left(z_{1}\right)\hat{P}_{1}\left(w_{1}\right)
\hat{P}_{2}\left(w_{2}\right)\right)\right)\hat{F}_{2}\left(w_{1},w_{2};\tau_{2}\right)
\end{eqnarray*}
al igual que antes

\begin{eqnarray*}
\frac{\partial
F_{2}\left(z_{1},\tilde{\theta}_{2}\left(P_{1}\left(z_{1}\right)\hat{P}_{1}\left(w_{1}\right)\hat{P}_{2}\left(w_{2}\right)\right),w_{1},w_{2}\right)}{\partial
z_{1}}|_{\mathbf{z},\mathbf{w}=1}&=&\frac{\partial F_{2}}{\partial
z_{2}}\cdot\frac{\partial\tilde{\theta}_{2}}{\partial
P_{1}}\cdot\frac{\partial P_{1}}{\partial z_{2}}+\frac{\partial
F_{2}}{\partial z_{1}}=f_{2}\left(2\right)\left(\frac{1}{1-\tilde{\mu}_{2}}\right)\mu_{1}+f_{2}\left(1\right)\\
\frac{\partial F_{2}\left(z_{1},\tilde{\theta}_{2}\left(P_{1}\left(z_{1}\right)\hat{P}_{1}\left(w_{1}\right)\hat{P}_{2}\left(w_{2}\right)\right),w_{1},w_{2}\right)}{\partial z_{2}}|_{\mathbf{z},\mathbf{w}=1}&=&0\\
\frac{\partial
F_{2}\left(z_{1},\tilde{\theta}_{2}\left(P_{1}\left(z_{1}\right)\hat{P}_{1}\left(w_{1}\right)\hat{P}_{2}\left(w_{2}\right)\right),w_{1},w_{2}\right)}{\partial
w_{1}}|_{\mathbf{z},\mathbf{w}=1}&=&\frac{\partial F_{2}}{\partial
z_{2}}\cdot\frac{\partial \tilde{\theta}_{2}}{\partial
\hat{P}_{1}}\cdot\frac{\partial \hat{P}_{1}}{\partial
w_{1}}+\frac{\partial \hat{F}_{2}}{\partial w_{1}}
=f_{2}\left(2\right)\left(\frac{1}{1-\tilde{\mu}_{2}}\right)\hat{\mu}_{1}+\hat{F}_{2,1}^{(1)}\left(1\right)\\
\frac{\partial
F_{2}\left(z_{1},\tilde{\theta}_{2}\left(P_{1}\left(z_{1}\right)\hat{P}_{1}\left(w_{1}\right)\hat{P}_{2}\left(w_{2}\right)\right),w_{1},w_{2}\right)}{\partial
w_{2}}|_{\mathbf{z},\mathbf{w}=1}&=&\frac{\partial F_{2}}{\partial
z_{2}}\cdot\frac{\partial
\tilde{\theta}_{2}}{\partial\hat{P}_{2}}\cdot\frac{\partial\hat{P}_{2}}{\partial
w_{2}}+\frac{\partial\hat{F}_{2}}{\partial w_{2}}
=f_{2}\left(2\right)\left(\frac{1}{1-\tilde{\mu}_{2}}\right)\hat{\mu}_{2}+\hat{F}_{2,2}^{(1)}\left(1\right)\\
\end{eqnarray*}


Ahora para el segundo sistema

\begin{eqnarray*}\hat{F}_{1}\left(z_{1},z_{2},\hat{\theta}_{1}\left(P_{1}\left(z_{1}\right)\tilde{P}_{2}\left(z_{2}\right)\hat{P}_{2}\left(w_{2}\right)\right),
w_{2}\right)=F_{1}\left(z_{1},z_{2};\zeta_{1}\right)\hat{F}_{1}\left(\hat{\theta}_{1}\left(P_{1}\left(z_{1}\right)\tilde{P}_{2}\left(z_{2}\right)
\hat{P}_{2}\left(w_{2}\right)\right),w_{2}\right)
\end{eqnarray*}
entonces


\begin{eqnarray*}
\frac{\partial
\hat{F}_{1}\left(z_{1},z_{2},\hat{\theta}_{1}\left(P_{1}\left(z_{1}\right)\tilde{P}_{2}\left(z_{2}\right)\hat{P}_{2}\left(w_{2}\right)\right),w_{2}\right)}{\partial
z_{1}}|_{\mathbf{z},\mathbf{w}=1}&=&\frac{\partial \hat{F}_{1}
}{\partial w_{1}}\cdot\frac{\partial\hat{\theta}_{1}}{\partial
P_{1}}\cdot\frac{\partial P_{1}}{\partial z_{1}}+\frac{\partial
F_{1}}{\partial z_{1}}=\hat{f}_{1}\left(1\right)\left(\frac{1}{1-\hat{\mu}_{1}}\right)\mu_{1}+F_{1,1}^{(1)}\left(1\right)\\
\frac{\partial
\hat{F}_{1}\left(z_{1},z_{2},\hat{\theta}_{1}\left(P_{1}\left(z_{1}\right)\tilde{P}_{2}\left(z_{2}\right)\hat{P}_{2}\left(w_{2}\right)\right),w_{2}\right)}{\partial
z_{2}}|_{\mathbf{z},\mathbf{w}=1}&=&\frac{\partial
\hat{F}_{1}}{\partial
w_{1}}\cdot\frac{\partial\hat{\theta}_{1}}{\partial\tilde{P}_{2}}\cdot\frac{\partial\tilde{P}_{2}}{\partial
z_{2}}+\frac{\partial F_{1}}{\partial z_{2}}
=\hat{f}_{1}\left(1\right)\left(\frac{1}{1-\hat{\mu}_{1}}\right)\tilde{\mu}_{2}+F_{2,1}^{(1)}\left(1\right)\\
\frac{\partial \hat{F}_{1}\left(z_{1},z_{2},\hat{\theta}_{1}\left(P_{1}\left(z_{1}\right)\tilde{P}_{2}\left(z_{2}\right)\hat{P}_{2}\left(w_{2}\right)\right),w_{2}\right)}{\partial w_{1}}|_{\mathbf{z},\mathbf{w}=1}&=&0\\
\frac{\partial \hat{F}_{1}\left(z_{1},z_{2},\hat{\theta}_{1}\left(P_{1}\left(z_{1}\right)\tilde{P}_{2}\left(z_{2}\right)\hat{P}_{2}\left(w_{2}\right)\right),w_{2}\right)}{\partial w_{2}}|_{\mathbf{z},\mathbf{w}=1}&=&\frac{\partial\hat{F}_{1}}{\partial w_{1}}\cdot\frac{\partial\hat{\theta}_{1}}{\partial\hat{P}_{2}}\cdot\frac{\partial\hat{P}_{2}}{\partial w_{2}}+\frac{\partial \hat{F}_{1}}{\partial w_{2}}=\hat{f}_{1}\left(1\right)\left(\frac{1}{1-\hat{\mu}_{1}}\right)\hat{\mu}_{2}+\hat{f}_{1}\left(2\right)\\
\end{eqnarray*}



Finalmente para $\zeta_{2}$

\begin{eqnarray*}\hat{F}_{2}\left(z_{1},z_{2},w_{1},\hat{\theta}_{2}\left(P_{1}\left(z_{1}\right)\tilde{P}_{2}\left(z_{2}\right)\hat{P}_{1}\left(w_{1}\right)\right)\right)&=&F_{2}\left(z_{1},z_{2};\zeta_{2}\right)\hat{F}_{2}\left(w_{1},\hat{\theta}_{2}\left(P_{1}\left(z_{1}\right)\tilde{P}_{2}\left(z_{2}\right)\hat{P}_{1}\left(w_{1}\right)\right)\right]
\end{eqnarray*}
por tanto:

\begin{eqnarray*}
\frac{\partial
\hat{F}_{2}\left(z_{1},z_{2},w_{1},\hat{\theta}_{2}\left(P_{1}\left(z_{1}\right)\tilde{P}_{2}\left(z_{2}\right)\hat{P}_{1}\left(w_{1}\right)\right)\right)}{\partial
z_{1}}|_{\mathbf{z},\mathbf{w}=1}&=&\frac{\partial\hat{F}_{2}}{\partial
w_{2}}\cdot\frac{\partial\hat{\theta}_{2}}{\partial
P_{1}}\cdot\frac{\partial P_{1}}{\partial z_{1}}+\frac{\partial
F_{2}}{\partial z_{1}}=\hat{f}_{2}\left(1\right)\left(\frac{1}{1-\hat{\mu}_{2}}\right)\mu_{1}+F_{1,2}^{(1)}\left(1\right)   \\
\frac{\partial \hat{F}_{2}\left(z_{1},z_{2},w_{1},\hat{\theta}_{2}\left(P_{1}\left(z_{1}\right)\tilde{P}_{2}\left(z_{2}\right)\hat{P}_{1}\left(w_{1}\right)\right)\right)}{\partial z_{2}}|_{\mathbf{z},\mathbf{w}=1}&=&\frac{\partial\hat{F}_{2}}{\partial w_{2}}\cdot\frac{\partial\hat{\theta}_{2}}{\partial \tilde{P}_{2}}\cdot\frac{\partial \tilde{P}_{2}}{\partial z_{2}}+\frac{\partial F_{2}}{\partial z_{2}}=\hat{f}_{2}\left(2\right)\left(\frac{1}{1-\hat{\mu}_{2}}\right)\tilde{\mu}_{2}+F_{2,2}^{(1)}\left(1\right)\\
\frac{\partial \hat{F}_{2}\left(z_{1},z_{2},w_{1},\hat{\theta}_{2}\left(P_{1}\left(z_{1}\right)\tilde{P}_{2}\left(z_{2}\right)\hat{P}_{1}\left(w_{1}\right)\right)\right)}{\partial w_{1}}|_{\mathbf{z},\mathbf{w}=1}&=&\frac{\partial\hat{F}_{2}}{\partial w_{2}}\cdot\frac{\partial\hat{\theta}_{2}}{\partial \hat{P}_{1}}\cdot\frac{\partial \hat{P}_{1}}{\partial w_{1}}+\frac{\partial \hat{F}_{2}}{\partial w_{1}}=\hat{f}_{2}\left(2\right)\left(\frac{1}{1-\hat{\mu}_{2}}\right)\hat{\mu}_{1}+\hat{f}_{2}\left(1\right)\\
\frac{\partial \hat{F}_{2}\left(z_{1},z_{2},w_{1},\hat{\theta}_{2}\left(P_{1}\left(z_{1}\right)\tilde{P}_{2}\left(z_{2}\right)\hat{P}_{1}\left(w_{1}\right)\right)\right)}{\partial w_{2}}|_{\mathbf{z},\mathbf{w}=1}&=&0\\
\end{eqnarray*}

%_________________________________________________________________________________________________
\subsection{Ecuaciones Recursivas}
%_________________________________________________________________________________________________

Entonces, de todo lo desarrollado hasta ahora se tienen las siguientes ecuaciones:

\begin{eqnarray*}
\begin{array}{ll}
\frac{\partial F_{2}\left(\mathbf{z},\mathbf{w}\right)}{\partial z_{1}}|_{\mathbf{z},\mathbf{w}=1}=r_{1}\mu_{1},&
\frac{\partial F_{2}\left(\mathbf{z},\mathbf{w}\right)}{\partial z_{2}}|_{\mathbf{z},\mathbf{w}=1}=r_{1}\tilde{\mu}_{2}+f_{1}\left(1\right)\left(\frac{1}{1-\mu_{1}}\right)\tilde{\mu}_{2}+f_{1}\left(2\right),\\
\frac{\partial F_{2}\left(\mathbf{z},\mathbf{w}\right)}{\partial w_{1}}|_{\mathbf{z},\mathbf{w}=1}=r_{1}\hat{\mu}_{1}+f_{1}\left(1\right)\left(\frac{1}{1-\mu_{1}}\right)\hat{\mu}_{1}+\hat{F}_{1,1}^{(1)}\left(1\right),&
\frac{\partial F_{2}\left(\mathbf{z},\mathbf{w}\right)}{\partial
w_{2}}|_{\mathbf{z},\mathbf{w}=1}=r_{1}\hat{\mu}_{2}+f_{1}\left(1\right)\left(\frac{1}{1-\mu_{1}}\right)\hat{\mu}_{2}+\hat{F}_{2,1}^{(1)}\left(1\right),\\
\frac{\partial F_{1}\left(\mathbf{z},\mathbf{w}\right)}{\partial z_{1}}|_{\mathbf{z},\mathbf{w}=1}=r_{2}\mu_{1}+f_{2}\left(2\right)\left(\frac{1}{1-\tilde{\mu}_{2}}\right)\mu_{1}+f_{2}\left(1\right),&
\frac{\partial F_{1}\left(\mathbf{z},\mathbf{w}\right)}{\partial z_{2}}|_{\mathbf{z},\mathbf{w}=1}=r_{2}\tilde{\mu}_{2},\\
\frac{\partial F_{1}\left(\mathbf{z},\mathbf{w}\right)}{\partial w_{1}}|_{\mathbf{z},\mathbf{w}=1}=r_{2}\hat{\mu}_{1}+f_{2}\left(2\right)\left(\frac{1}{1-\tilde{\mu}_{2}}\right)\hat{\mu}_{1}+\hat{F}_{2,1}^{(1)}\left(1\right),&
\frac{\partial F_{1}\left(\mathbf{z},\mathbf{w}\right)}{\partial
w_{2}}|_{\mathbf{z},\mathbf{w}=1}=r_{2}\hat{\mu}_{2}+f_{2}\left(2\right)\left(\frac{1}{1-\tilde{\mu}_{2}}\right)\hat{\mu}_{2}+\hat{F}_{2,2}^{(1)}\left(1\right),\\
\frac{\partial \hat{F}_{2}\left(\mathbf{z},\mathbf{w}\right)}{\partial z_{1}}|_{\mathbf{z},\mathbf{w}=1}=\hat{r}_{1}\mu_{1}+\hat{f}_{1}\left(1\right)\left(\frac{1}{1-\hat{\mu}_{1}}\right)\mu_{1}+F_{1,1}^{(1)}\left(1\right),&
\frac{\partial \hat{F}_{2}\left(\mathbf{z},\mathbf{w}\right)}{\partial z_{2}}|_{\mathbf{z},\mathbf{w}=1}=\hat{r}_{1}\mu_{2}+\hat{f}_{1}\left(1\right)\left(\frac{1}{1-\hat{\mu}_{1}}\right)\tilde{\mu}_{2}+F_{2,1}^{(1)}\left(1\right),\\
\frac{\partial \hat{F}_{2}\left(\mathbf{z},\mathbf{w}\right)}{\partial w_{1}}|_{\mathbf{z},\mathbf{w}=1}=\hat{r}_{1}\hat{\mu}_{1},&
\frac{\partial \hat{F}_{2}\left(\mathbf{z},\mathbf{w}\right)}{\partial w_{2}}|_{\mathbf{z},\mathbf{w}=1}=\hat{r}_{1}\hat{\mu}_{2}+\hat{f}_{1}\left(1\right)\left(\frac{1}{1-\hat{\mu}_{1}}\right)\hat{\mu}_{2}+\hat{f}_{1}\left(2\right),\\
\frac{\partial \hat{F}_{1}\left(\mathbf{z},\mathbf{w}\right)}{\partial z_{1}}|_{\mathbf{z},\mathbf{w}=1}=\hat{r}_{2}\mu_{1}+\hat{f}_{2}\left(1\right)\left(\frac{1}{1-\hat{\mu}_{2}}\right)\mu_{1}+F_{1,2}^{(1)}\left(1\right),&
\frac{\partial \hat{F}_{1}\left(\mathbf{z},\mathbf{w}\right)}{\partial z_{2}}|_{\mathbf{z},\mathbf{w}=1}=\hat{r}_{2}\tilde{\mu}_{2}+\hat{f}_{2}\left(2\right)\left(\frac{1}{1-\hat{\mu}_{2}}\right)\tilde{\mu}_{2}+F_{2,2}^{(1)}\left(1\right),\\
\frac{\partial \hat{F}_{1}\left(\mathbf{z},\mathbf{w}\right)}{\partial w_{1}}|_{\mathbf{z},\mathbf{w}=1}=\hat{r}_{2}\hat{\mu}_{1}+\hat{f}_{2}\left(2\right)\left(\frac{1}{1-\hat{\mu}_{2}}\right)\hat{\mu}_{1}+\hat{f}_{2}\left(1\right),&
\frac{\partial
\hat{F}_{1}\left(\mathbf{z},\mathbf{w}\right)}{\partial
w_{2}}|_{\mathbf{z},\mathbf{w}=1}=\hat{r}_{2}\hat{\mu}_{2}
\end{array}
\end{eqnarray*}

Es decir, se tienen las siguientes ecuaciones:

\begin{eqnarray*}
\begin{array}{llll}
f_{2}\left(1\right)=r_{1}\mu_{1},&
f_{1}\left(2\right)=r_{2}\tilde{\mu}_{2},&
\hat{f}_{1}\left(4\right)=\hat{r}_{2}\hat{\mu}_{2},&
\hat{f}_{2}\left(3\right)=\hat{r}_{1}\hat{\mu}_{1}
\end{array}
\end{eqnarray*}

\begin{eqnarray*}
f_{1}\left(1\right)&=&r_{2}\mu_{1}+\mu_{1}\left(\frac{f_{2}\left(2\right)}{1-\tilde{\mu}_{2}}\right)+r_{1}\mu_{1}=\mu_{1}\left(r_{1}+r_{2}+\frac{f_{2}\left(2\right)}{1-\tilde{\mu}_{2}}\right)=\mu_{1}\left(r+\frac{f_{2}\left(2\right)}{1-\tilde{\mu}_{2}}\right),\\
f_{1}\left(3\right)&=&r_{2}\hat{\mu}_{1}+\hat{\mu}_{1}\left(\frac{f_{2}\left(2\right)}{1-\tilde{\mu}_{2}}\right)+\hat{F}^{(1)}_{1,2}\left(1\right)=\hat{\mu}_{1}\left(r_{2}+\frac{f_{2}\left(2\right)}{1-\tilde{\mu}_{2}}\right)+\frac{\hat{\mu}_{1}}{\mu_{2}},\\
f_{1}\left(4\right)&=&r_{2}\hat{\mu}_{2}+\hat{\mu}_{2}\left(\frac{f_{2}\left(2\right)}{1-\tilde{\mu}_{2}}\right)+\hat{F}_{2,2}^{(1)}\left(1\right)=\hat{\mu}_{2}\left(r_{2}+\frac{f_{2}\left(2\right)}{1-\tilde{\mu}_{2}}\right)+\frac{\hat{\mu}_{2}}{\mu_{2}},\\
f_{2}\left(2\right)&=&r_{1}\tilde{\mu}_{2}+\tilde{\mu}_{2}\left(\frac{f_{1}\left(1\right)}{1-\mu_{1}}\right)+f_{1}\left(2\right)=\left(r_{1}+\frac{f_{1}\left(1\right)}{1-\mu_{1}}\right)\tilde{\mu}_{2}+r_{2}\tilde{\mu}_{2}=\left(r_{1}+r_{2}+\frac{f_{1}\left(1\right)}{1-\mu_{1}}\right)\tilde{\mu}_{2}=\left(r+\frac{f_{1}\left(1\right)}{1-\mu_{1}}\right)\tilde{\mu}_{2},\\
f_{2}\left(3\right)&=&r_{1}\hat{\mu}_{1}+\hat{\mu}_{1}\left(\frac{f_{1}\left(1\right)}{1-\mu_{1}}\right)+\hat{F}_{1,1}^{(1)}\left(1\right)=\hat{\mu}_{1}\left(r_{1}+\frac{f_{1}\left(1\right)}{1-\mu_{1}}\right)+\frac{\hat{\mu}_{1}}{\mu_{1}},\\
f_{2}\left(4\right)&=&r_{1}\hat{\mu}_{2}+\hat{\mu}_{2}\left(\frac{f_{1}\left(1\right)}{1-\mu_{1}}\right)+\hat{F}_{2,1}^{(1)}\left(1\right)=\hat{\mu}_{2}\left(r_{1}+\frac{f_{1}\left(1\right)}{1-\mu_{1}}\right)+\frac{\hat{\mu}_{2}}{\mu_{1}},
\end{eqnarray*}


\begin{eqnarray*}
\hat{f}_{1}\left(1\right)&=&\hat{r}_{2}\mu_{1}+\mu_{1}\left(\frac{\hat{f}_{2}\left(4\right)}{1-\hat{\mu}_{2}}\right)+F_{1,2}^{(1)}\left(1\right)=\left(\hat{r}_{2}+\frac{\hat{f}_{2}\left(4\right)}{1-\hat{\mu}_{2}}\right)\mu_{1}+\frac{\mu_{1}}{\hat{\mu}_{2}},\\
\hat{f}_{1}\left(2\right)&=&\hat{r}_{2}\tilde{\mu}_{2}+\tilde{\mu}_{2}\left(\frac{\hat{f}_{2}\left(4\right)}{1-\hat{\mu}_{2}}\right)+F_{2,2}^{(1)}\left(1\right)=
\left(\hat{r}_{2}+\frac{\hat{f}_{2}\left(4\right)}{1-\hat{\mu}_{2}}\right)\tilde{\mu}_{2}+\frac{\mu_{2}}{\hat{\mu}_{2}},\\
\hat{f}_{1}\left(3\right)&=&\hat{r}_{2}\hat{\mu}_{1}+\hat{\mu}_{1}\left(\frac{\hat{f}_{2}\left(4\right)}{1-\hat{\mu}_{2}}\right)+\hat{f}_{2}\left(3\right)=\left(\hat{r}_{2}+\frac{\hat{f}_{2}\left(4\right)}{1-\hat{\mu}_{2}}\right)\hat{\mu}_{1}+\hat{r}_{1}\hat{\mu}_{1}=\left(\hat{r}_{1}+\hat{r}_{2}+\frac{\hat{f}_{2}\left(4\right)}{1-\hat{\mu}_{2}}\right)\hat{\mu}_{1}=\left(\hat{r}+\frac{\hat{f}_{2}\left(4\right)}{1-\hat{\mu}_{2}}\right)\hat{\mu}_{1},\\
\hat{f}_{2}\left(1\right)&=&\hat{r}_{1}\mu_{1}+\mu_{1}\left(\frac{\hat{f}_{1}\left(3\right)}{1-\hat{\mu}_{1}}\right)+F_{1,1}^{(1)}\left(1\right)=\left(\hat{r}_{1}+\frac{\hat{f}_{1}\left(3\right)}{1-\hat{\mu}_{1}}\right)\mu_{1}+\frac{\mu_{1}}{\hat{\mu}_{1}},\\
\hat{f}_{2}\left(2\right)&=&\hat{r}_{1}\tilde{\mu}_{2}+\tilde{\mu}_{2}\left(\frac{\hat{f}_{1}\left(3\right)}{1-\hat{\mu}_{1}}\right)+F_{2,1}^{(1)}\left(1\right)=\left(\hat{r}_{1}+\frac{\hat{f}_{1}\left(3\right)}{1-\hat{\mu}_{1}}\right)\tilde{\mu}_{2}+\frac{\mu_{2}}{\hat{\mu}_{1}},\\
\hat{f}_{2}\left(4\right)&=&\hat{r}_{1}\hat{\mu}_{2}+\hat{\mu}_{2}\left(\frac{\hat{f}_{1}\left(3\right)}{1-\hat{\mu}_{1}}\right)+\hat{f}_{1}\left(4\right)=\hat{r}_{1}\hat{\mu}_{2}+\hat{r}_{2}\hat{\mu}_{2}+\hat{\mu}_{2}\left(\frac{\hat{f}_{1}\left(3\right)}{1-\hat{\mu}_{1}}\right)=\left(\hat{r}+\frac{\hat{f}_{1}\left(3\right)}{1-\hat{\mu}_{1}}\right)\hat{\mu}_{2},\\
\end{eqnarray*}

es decir,


\begin{eqnarray*}
\begin{array}{lll}
f_{1}\left(1\right)=\mu_{1}\left(r+\frac{f_{2}\left(2\right)}{1-\tilde{\mu}_{2}}\right)&f_{1}\left(2\right)=r_{2}\tilde{\mu}_{2}&f_{1}\left(3\right)=\hat{\mu}_{1}\left(r_{2}+\frac{f_{2}\left(2\right)}{1-\tilde{\mu}_{2}}\right)+\frac{\hat{\mu}_{1}}{\mu_{2}}\\
f_{1}\left(4\right)=\hat{\mu}_{2}\left(r_{2}+\frac{f_{2}\left(2\right)}{1-\tilde{\mu}_{2}}\right)+\frac{\hat{\mu}_{2}}{\mu_{2}}&f_{2}\left(1\right)=r_{1}\mu_{1}&f_{2}\left(2\right)=\left(r+\frac{f_{1}\left(1\right)}{1-\mu_{1}}\right)\tilde{\mu}_{2}\\
f_{2}\left(3\right)=\hat{\mu}_{1}\left(r_{1}+\frac{f_{1}\left(1\right)}{1-\mu_{1}}\right)+\frac{\hat{\mu}_{1}}{\mu_{1}}&
f_{2}\left(4\right)=\hat{\mu}_{2}\left(r_{1}+\frac{f_{1}\left(1\right)}{1-\mu_{1}}\right)+\frac{\hat{\mu}_{2}}{\mu_{1}}&\hat{f}_{1}\left(1\right)=\left(\hat{r}_{2}+\frac{\hat{f}_{2}\left(4\right)}{1-\hat{\mu}_{2}}\right)\mu_{1}+\frac{\mu_{1}}{\hat{\mu}_{2}}\\
\hat{f}_{1}\left(2\right)=\left(\hat{r}_{2}+\frac{\hat{f}_{2}\left(4\right)}{1-\hat{\mu}_{2}}\right)\tilde{\mu}_{2}+\frac{\mu_{2}}{\hat{\mu}_{2}}&\hat{f}_{1}\left(3\right)=\left(\hat{r}+\frac{\hat{f}_{2}\left(4\right)}{1-\hat{\mu}_{2}}\right)\hat{\mu}_{1}&\hat{f}_{1}\left(4\right)=\hat{r}_{2}\hat{\mu}_{2}\\
\hat{f}_{2}\left(1\right)=\left(\hat{r}_{1}+\frac{\hat{f}_{1}\left(3\right)}{1-\hat{\mu}_{1}}\right)\mu_{1}+\frac{\mu_{1}}{\hat{\mu}_{1}}&\hat{f}_{2}\left(2\right)=\left(\hat{r}_{1}+\frac{\hat{f}_{1}\left(3\right)}{1-\hat{\mu}_{1}}\right)\tilde{\mu}_{2}+\frac{\mu_{2}}{\hat{\mu}_{1}}&\hat{f}_{2}\left(3\right)=\hat{r}_{1}\hat{\mu}_{1}\\
&\hat{f}_{2}\left(4\right)=\left(\hat{r}+\frac{\hat{f}_{1}\left(3\right)}{1-\hat{\mu}_{1}}\right)\hat{\mu}_{2}&
\end{array}
\end{eqnarray*}

%_______________________________________________________________________________________________
\subsection{Soluci\'on del Sistema de Ecuaciones Lineales}
%_________________________________________________________________________________________________

Se puede demostrar que la soluci\'on del sistema de
ecuaciones est\'a dado por las siguientes expresiones, donde

\begin{eqnarray*}
\mu=\mu_{1}+\tilde{\mu}_{2}\textrm{ , }\hat{\mu}=\hat{\mu}_{1}+\hat{\mu}_{2}\textrm{ , }
r=r_{1}+r_{2}\textrm{ y }\hat{r}=\hat{r}_{1}+\hat{r}_{2}
\end{eqnarray*}
entonces

\begin{eqnarray*}
\begin{array}{lll}
f_{1}\left(1\right)=r\frac{\mu_{1}\left(1-\mu_{1}\right)}{1-\mu}&
f_{1}\left(3\right)=\hat{\mu}_{1}\left(\frac{r_{2}\mu_{2}+1}{\mu_{2}}+r\frac{\tilde{\mu}_{2}}{1-\mu}\right)&
f_{1}\left(4\right)=\hat{\mu}_{2}\left(\frac{r_{2}\mu_{2}+1}{\mu_{2}}+r\frac{\tilde{\mu}_{2}}{1-\mu}\right)\\
f_{2}\left(2\right)=r\frac{\tilde{\mu}_{2}\left(1-\tilde{\mu}_{2}\right)}{1-\mu}&
f_{2}\left(3\right)=\hat{\mu}_{1}\left(\frac{r_{1}\mu_{1}+1}{\mu_{1}}+r\frac{\mu_{1}}{1-\mu}\right)&
f_{2}\left(4\right)=\hat{\mu}_{2}\left(\frac{r_{1}\mu_{1}+1}{\mu_{1}}+r\frac{\mu_{1}}{1-\mu}\right)\\
\hat{f}_{1}\left(1\right)=\mu_{1}\left(\frac{\hat{r}_{2}\hat{\mu}_{2}+1}{\hat{\mu}_{2}}+\hat{r}\frac{\hat{\mu}_{2}}{1-\hat{\mu}}\right)&
\hat{f}_{1}\left(2\right)=\tilde{\mu}_{2}\left(\hat{r}_{2}+\hat{r}\frac{\hat{\mu}_{2}}{1-\hat{\mu}}\right)+\frac{\mu_{2}}{\hat{\mu}_{2}}&
\hat{f}_{1}\left(3\right)=\hat{r}\frac{\hat{\mu}_{1}\left(1-\hat{\mu}_{1}\right)}{1-\hat{\mu}}\\
\hat{f}_{2}\left(1\right)=\mu_{1}\left(\frac{\hat{r}_{1}\hat{\mu}_{1}+1}{\hat{\mu}_{1}}+\hat{r}\frac{\hat{\mu}_{1}}{1-\hat{\mu}}\right)&
\hat{f}_{2}\left(2\right)=\tilde{\mu}_{2}\left(\hat{r}_{1}+\hat{r}\frac{\hat{\mu}_{1}}{1-\hat{\mu}}\right)+\frac{\hat{\mu_{2}}}{\hat{\mu}_{1}}&
\hat{f}_{2}\left(4\right)=\hat{r}\frac{\hat{\mu}_{2}\left(1-\hat{\mu}_{2}\right)}{1-\hat{\mu}}\\
\end{array}
\end{eqnarray*}




A saber

\begin{eqnarray*}
f_{1}\left(3\right)&=&\hat{\mu}_{1}\left(r_{2}+\frac{f_{2}\left(2\right)}{1-\tilde{\mu}_{2}}\right)+\frac{\hat{\mu}_{1}}{\mu_{2}}=\hat{\mu}_{1}\left(r_{2}+\frac{r\frac{\tilde{\mu}_{2}\left(1-\tilde{\mu}_{2}\right)}{1-\mu}}{1-\tilde{\mu}_{2}}\right)+\frac{\hat{\mu}_{1}}{\mu_{2}}=\hat{\mu}_{1}\left(r_{2}+\frac{r\tilde{\mu}_{2}}{1-\mu}\right)+\frac{\hat{\mu}_{1}}{\mu_{2}}\\
&=&\hat{\mu}_{1}\left(r_{2}+\frac{r\tilde{\mu}_{2}}{1-\mu}+\frac{1}{\mu_{2}}\right)=\hat{\mu}_{1}\left(\frac{r_{2}\mu_{2}+1}{\mu_{2}}+\frac{r\tilde{\mu}_{2}}{1-\mu}\right)
\end{eqnarray*}

\begin{eqnarray*}
f_{1}\left(4\right)&=&\hat{\mu}_{2}\left(r_{2}+\frac{f_{2}\left(2\right)}{1-\tilde{\mu}_{2}}\right)+\frac{\hat{\mu}_{2}}{\mu_{2}}=\hat{\mu}_{2}\left(r_{2}+\frac{r\frac{\tilde{\mu}_{2}\left(1-\tilde{\mu}_{2}\right)}{1-\mu}}{1-\tilde{\mu}_{2}}\right)+\frac{\hat{\mu}_{2}}{\mu_{2}}=\hat{\mu}_{2}\left(r_{2}+\frac{r\tilde{\mu}_{2}}{1-\mu}\right)+\frac{\hat{\mu}_{1}}{\mu_{2}}\\
&=&\hat{\mu}_{2}\left(r_{2}+\frac{r\tilde{\mu}_{2}}{1-\mu}+\frac{1}{\mu_{2}}\right)=\hat{\mu}_{2}\left(\frac{r_{2}\mu_{2}+1}{\mu_{2}}+\frac{r\tilde{\mu}_{2}}{1-\mu}\right)
\end{eqnarray*}

\begin{eqnarray*}
f_{2}\left(3\right)&=&\hat{\mu}_{1}\left(r_{1}+\frac{f_{1}\left(1\right)}{1-\mu_{1}}\right)+\frac{\hat{\mu}_{1}}{\mu_{1}}=\hat{\mu}_{1}\left(r_{1}+\frac{r\frac{\mu_{1}\left(1-\mu_{1}\right)}{1-\mu}}{1-\mu_{1}}\right)+\frac{\hat{\mu}_{1}}{\mu_{1}}=\hat{\mu}_{1}\left(r_{1}+\frac{r\mu_{1}}{1-\mu}\right)+\frac{\hat{\mu}_{1}}{\mu_{1}}\\
&=&\hat{\mu}_{1}\left(r_{1}+\frac{r\mu_{1}}{1-\mu}+\frac{1}{\mu_{1}}\right)=\hat{\mu}_{1}\left(\frac{r_{1}\mu_{1}+1}{\mu_{1}}+\frac{r\mu_{1}}{1-\mu}\right)
\end{eqnarray*}

\begin{eqnarray*}
f_{2}\left(4\right)&=&\hat{\mu}_{2}\left(r_{1}+\frac{f_{1}\left(1\right)}{1-\mu_{1}}\right)+\frac{\hat{\mu}_{2}}{\mu_{1}}=\hat{\mu}_{2}\left(r_{1}+\frac{r\frac{\mu_{1}\left(1-\mu_{1}\right)}{1-\mu}}{1-\mu_{1}}\right)+\frac{\hat{\mu}_{1}}{\mu_{1}}=\hat{\mu}_{2}\left(r_{1}+\frac{r\mu_{1}}{1-\mu}\right)+\frac{\hat{\mu}_{1}}{\mu_{1}}\\
&=&\hat{\mu}_{2}\left(r_{1}+\frac{r\mu_{1}}{1-\mu}+\frac{1}{\mu_{1}}\right)=\hat{\mu}_{2}\left(\frac{r_{1}\mu_{1}+1}{\mu_{1}}+\frac{r\mu_{1}}{1-\mu}\right)\end{eqnarray*}


\begin{eqnarray*}
\hat{f}_{1}\left(1\right)&=&\mu_{1}\left(\hat{r}_{2}+\frac{\hat{f}_{2}\left(4\right)}{1-\tilde{\mu}_{2}}\right)+\frac{\mu_{1}}{\hat{\mu}_{2}}=\mu_{1}\left(\hat{r}_{2}+\frac{\hat{r}\frac{\hat{\mu}_{2}\left(1-\hat{\mu}_{2}\right)}{1-\hat{\mu}}}{1-\hat{\mu}_{2}}\right)+\frac{\mu_{1}}{\hat{\mu}_{2}}=\mu_{1}\left(\hat{r}_{2}+\frac{\hat{r}\hat{\mu}_{2}}{1-\hat{\mu}}\right)+\frac{\mu_{1}}{\mu_{2}}\\
&=&\mu_{1}\left(\hat{r}_{2}+\frac{\hat{r}\mu_{2}}{1-\hat{\mu}}+\frac{1}{\hat{\mu}_{2}}\right)=\mu_{1}\left(\frac{\hat{r}_{2}\hat{\mu}_{2}+1}{\hat{\mu}_{2}}+\frac{\hat{r}\hat{\mu}_{2}}{1-\hat{\mu}}\right)
\end{eqnarray*}

\begin{eqnarray*}
\hat{f}_{1}\left(2\right)&=&\tilde{\mu}_{2}\left(\hat{r}_{2}+\frac{\hat{f}_{2}\left(4\right)}{1-\tilde{\mu}_{2}}\right)+\frac{\mu_{2}}{\hat{\mu}_{2}}=\tilde{\mu}_{2}\left(\hat{r}_{2}+\frac{\hat{r}\frac{\hat{\mu}_{2}\left(1-\hat{\mu}_{2}\right)}{1-\hat{\mu}}}{1-\hat{\mu}_{2}}\right)+\frac{\mu_{2}}{\hat{\mu}_{2}}=\tilde{\mu}_{2}\left(\hat{r}_{2}+\frac{\hat{r}\hat{\mu}_{2}}{1-\hat{\mu}}\right)+\frac{\mu_{2}}{\hat{\mu}_{2}}
\end{eqnarray*}

\begin{eqnarray*}
\hat{f}_{2}\left(1\right)&=&\mu_{1}\left(\hat{r}_{1}+\frac{\hat{f}_{1}\left(3\right)}{1-\hat{\mu}_{1}}\right)+\frac{\mu_{1}}{\hat{\mu}_{1}}=\mu_{1}\left(\hat{r}_{1}+\frac{\hat{r}\frac{\hat{\mu}_{1}\left(1-\hat{\mu}_{1}\right)}{1-\hat{\mu}}}{1-\hat{\mu}_{1}}\right)+\frac{\mu_{1}}{\hat{\mu}_{1}}=\mu_{1}\left(\hat{r}_{1}+\frac{\hat{r}\hat{\mu}_{1}}{1-\hat{\mu}}\right)+\frac{\mu_{1}}{\hat{\mu}_{1}}\\
&=&\mu_{1}\left(\hat{r}_{1}+\frac{\hat{r}\hat{\mu}_{1}}{1-\hat{\mu}}+\frac{1}{\hat{\mu}_{1}}\right)=\mu_{1}\left(\frac{\hat{r}_{1}\hat{\mu}_{1}+1}{\hat{\mu}_{1}}+\frac{\hat{r}\hat{\mu}_{1}}{1-\hat{\mu}}\right)
\end{eqnarray*}

\begin{eqnarray*}
\hat{f}_{2}\left(2\right)&=&\tilde{\mu}_{2}\left(\hat{r}_{1}+\frac{\hat{f}_{1}\left(3\right)}{1-\tilde{\mu}_{1}}\right)+\frac{\mu_{2}}{\hat{\mu}_{1}}=\tilde{\mu}_{2}\left(\hat{r}_{1}+\frac{\hat{r}\frac{\hat{\mu}_{1}
\left(1-\hat{\mu}_{1}\right)}{1-\hat{\mu}}}{1-\hat{\mu}_{1}}\right)+\frac{\mu_{2}}{\hat{\mu}_{1}}=\tilde{\mu}_{2}\left(\hat{r}_{1}+\frac{\hat{r}\hat{\mu}_{1}}{1-\hat{\mu}}\right)+\frac{\mu_{2}}{\hat{\mu}_{1}}
\end{eqnarray*}

%----------------------------------------------------------------------------------------
\section{Resultado Principal}
%----------------------------------------------------------------------------------------
Sean $\mu_{1},\mu_{2},\check{\mu}_{2},\hat{\mu}_{1},\hat{\mu}_{2}$ y $\tilde{\mu}_{2}=\mu_{2}+\check{\mu}_{2}$ los valores esperados de los proceso definidos anteriormente, y sean $r_{1},r_{2}, \hat{r}_{1}$ y $\hat{r}_{2}$ los valores esperado s de los tiempos de traslado del servidor entre las colas para cada uno de los sistemas de visitas c\'iclicas. Si se definen $\mu=\mu_{1}+\tilde{\mu}_{2}$, $\hat{\mu}=\hat{\mu}_{1}+\hat{\mu}_{2}$, y $r=r_{1}+r_{2}$ y  $\hat{r}=\hat{r}_{1}+\hat{r}_{2}$, entonces se tiene el siguiente resultado.

\begin{Teo}
Supongamos que $\mu<1$, $\hat{\mu}<1$, entonces, el n\'umero de usuarios presentes en cada una de las colas que conforman la Red de Sistemas de Visitas C\'iclicas cuando uno de los servidores visita a alguna de ellas est\'a dada por la soluci\'on del Sistema de Ecuaciones Lineales presentados arriba cuyas expresiones damos a continuaci\'on:
%{\footnotesize{
\begin{eqnarray*}
\begin{array}{lll}
f_{1}\left(1\right)=r\frac{\mu_{1}\left(1-\mu_{1}\right)}{1-\mu},&f_{1}\left(2\right)=r_{2}\tilde{\mu}_{2},&f_{1}\left(3\right)=\hat{\mu}_{1}\left(\frac{r_{2}\mu_{2}+1}{\mu_{2}}+r\frac{\tilde{\mu}_{2}}{1-\mu}\right),\\
f_{1}\left(4\right)=\hat{\mu}_{2}\left(\frac{r_{2}\mu_{2}+1}{\mu_{2}}+r\frac{\tilde{\mu}_{2}}{1-\mu}\right),&f_{2}\left(1\right)=r_{1}\mu_{1},&f_{2}\left(2\right)=r\frac{\tilde{\mu}_{2}\left(1-\tilde{\mu}_{2}\right)}{1-\mu},\\
f_{2}\left(3\right)=\hat{\mu}_{1}\left(\frac{r_{1}\mu_{1}+1}{\mu_{1}}+r\frac{\mu_{1}}{1-\mu}\right),&f_{2}\left(4\right)=\hat{\mu}_{2}\left(\frac{r_{1}\mu_{1}+1}{\mu_{1}}+r\frac{\mu_{1}}{1-\mu}\right),&\hat{f}_{1}\left(1\right)=\mu_{1}\left(\frac{\hat{r}_{2}\hat{\mu}_{2}+1}{\hat{\mu}_{2}}+\hat{r}\frac{\hat{\mu}_{2}}{1-\hat{\mu}}\right),\\
\hat{f}_{1}\left(2\right)=\tilde{\mu}_{2}\left(\hat{r}_{2}+\hat{r}\frac{\hat{\mu}_{2}}{1-\hat{\mu}}\right)+\frac{\mu_{2}}{\hat{\mu}_{2}},&\hat{f}_{1}\left(3\right)=\hat{r}\frac{\hat{\mu}_{1}\left(1-\hat{\mu}_{1}\right)}{1-\hat{\mu}},&\hat{f}_{1}\left(4\right)=\hat{r}_{2}\hat{\mu}_{2},\\
\hat{f}_{2}\left(1\right)=\mu_{1}\left(\frac{\hat{r}_{1}\hat{\mu}_{1}+1}{\hat{\mu}_{1}}+\hat{r}\frac{\hat{\mu}_{1}}{1-\hat{\mu}}\right),&\hat{f}_{2}\left(2\right)=\tilde{\mu}_{2}\left(\hat{r}_{1}+\hat{r}\frac{\hat{\mu}_{1}}{1-\hat{\mu}}\right)+\frac{\hat{\mu_{2}}}{\hat{\mu}_{1}},&\hat{f}_{2}\left(3\right)=\hat{r}_{1}\hat{\mu}_{1},\\
&\hat{f}_{2}\left(4\right)=\hat{r}\frac{\hat{\mu}_{2}\left(1-\hat{\mu}_{2}\right)}{1-\hat{\mu}}.&\\
\end{array}
\end{eqnarray*} %}}
\end{Teo}





%___________________________________________________________________________________________
%
\section{Segundos Momentos}
%___________________________________________________________________________________________
%
%___________________________________________________________________________________________
%
%\subsection{Derivadas de Segundo Orden: Tiempos de Traslado del Servidor}
%___________________________________________________________________________________________



Para poder determinar los segundos momentos para los tiempos de traslado del servidor es necesaria la siguiente proposici\'on:

\begin{Prop}\label{Prop.Segundas.Derivadas}
Sea $f\left(g\left(x\right)h\left(y\right)\right)$ funci\'on continua tal que tiene derivadas parciales mixtas de segundo orden, entonces se tiene lo siguiente:

\begin{eqnarray*}
\frac{\partial}{\partial x}f\left(g\left(x\right)h\left(y\right)\right)=\frac{\partial f\left(g\left(x\right)h\left(y\right)\right)}{\partial x}\cdot \frac{\partial g\left(x\right)}{\partial x}\cdot h\left(y\right)
\end{eqnarray*}

por tanto

\begin{eqnarray}
\frac{\partial}{\partial x}\frac{\partial}{\partial x}f\left(g\left(x\right)h\left(y\right)\right)
&=&\frac{\partial^{2}}{\partial x}f\left(g\left(x\right)h\left(y\right)\right)\cdot \left(\frac{\partial g\left(x\right)}{\partial x}\right)^{2}\cdot h^{2}\left(y\right)+\frac{\partial}{\partial x}f\left(g\left(x\right)h\left(y\right)\right)\cdot \frac{\partial g^{2}\left(x\right)}{\partial x^{2}}\cdot h\left(y\right).
\end{eqnarray}

y

\begin{eqnarray*}
\frac{\partial}{\partial y}\frac{\partial}{\partial x}f\left(g\left(x\right)h\left(y\right)\right)&=&\frac{\partial g\left(x\right)}{\partial x}\cdot \frac{\partial h\left(y\right)}{\partial y}\left\{\frac{\partial^{2}}{\partial y\partial x}f\left(g\left(x\right)h\left(y\right)\right)\cdot g\left(x\right)\cdot h\left(y\right)+\frac{\partial}{\partial x}f\left(g\left(x\right)h\left(y\right)\right)\right\}
\end{eqnarray*}
\end{Prop}
\begin{proof}
\footnotesize{
\begin{eqnarray*}
\frac{\partial}{\partial x}\frac{\partial}{\partial x}f\left(g\left(x\right)h\left(y\right)\right)&=&\frac{\partial}{\partial x}\left\{\frac{\partial f\left(g\left(x\right)h\left(y\right)\right)}{\partial x}\cdot \frac{\partial g\left(x\right)}{\partial x}\cdot h\left(y\right)\right\}\\
&=&\frac{\partial}{\partial x}\left\{\frac{\partial}{\partial x}f\left(g\left(x\right)h\left(y\right)\right)\right\}\cdot \frac{\partial g\left(x\right)}{\partial x}\cdot h\left(y\right)+\frac{\partial}{\partial x}f\left(g\left(x\right)h\left(y\right)\right)\cdot \frac{\partial g^{2}\left(x\right)}{\partial x^{2}}\cdot h\left(y\right)\\
&=&\frac{\partial^{2}}{\partial x}f\left(g\left(x\right)h\left(y\right)\right)\cdot \frac{\partial g\left(x\right)}{\partial x}\cdot h\left(y\right)\cdot \frac{\partial g\left(x\right)}{\partial x}\cdot h\left(y\right)+\frac{\partial}{\partial x}f\left(g\left(x\right)h\left(y\right)\right)\cdot \frac{\partial g^{2}\left(x\right)}{\partial x^{2}}\cdot h\left(y\right)\\
&=&\frac{\partial^{2}}{\partial x}f\left(g\left(x\right)h\left(y\right)\right)\cdot \left(\frac{\partial g\left(x\right)}{\partial x}\right)^{2}\cdot h^{2}\left(y\right)+\frac{\partial}{\partial x}f\left(g\left(x\right)h\left(y\right)\right)\cdot \frac{\partial g^{2}\left(x\right)}{\partial x^{2}}\cdot h\left(y\right).
\end{eqnarray*}}


Por otra parte:
\footnotesize{
\begin{eqnarray*}
\frac{\partial}{\partial y}\frac{\partial}{\partial x}f\left(g\left(x\right)h\left(y\right)\right)&=&\frac{\partial}{\partial y}\left\{\frac{\partial f\left(g\left(x\right)h\left(y\right)\right)}{\partial x}\cdot \frac{\partial g\left(x\right)}{\partial x}\cdot h\left(y\right)\right\}\\
&=&\frac{\partial}{\partial y}\left\{\frac{\partial}{\partial x}f\left(g\left(x\right)h\left(y\right)\right)\right\}\cdot \frac{\partial g\left(x\right)}{\partial x}\cdot h\left(y\right)+\frac{\partial}{\partial x}f\left(g\left(x\right)h\left(y\right)\right)\cdot \frac{\partial g\left(x\right)}{\partial x}\cdot \frac{\partial h\left(y\right)}{y}\\
&=&\frac{\partial^{2}}{\partial y\partial x}f\left(g\left(x\right)h\left(y\right)\right)\cdot \frac{\partial h\left(y\right)}{\partial y}\cdot g\left(x\right)\cdot \frac{\partial g\left(x\right)}{\partial x}\cdot h\left(y\right)+\frac{\partial}{\partial x}f\left(g\left(x\right)h\left(y\right)\right)\cdot \frac{\partial g\left(x\right)}{\partial x}\cdot \frac{\partial h\left(y\right)}{\partial y}\\
&=&\frac{\partial g\left(x\right)}{\partial x}\cdot \frac{\partial h\left(y\right)}{\partial y}\left\{\frac{\partial^{2}}{\partial y\partial x}f\left(g\left(x\right)h\left(y\right)\right)\cdot g\left(x\right)\cdot h\left(y\right)+\frac{\partial}{\partial x}f\left(g\left(x\right)h\left(y\right)\right)\right\}
\end{eqnarray*}}
\end{proof}

Utilizando la proposici\'on anterior (Proposici\'ion \ref{Prop.Segundas.Derivadas})se tiene el siguiente resultado que me dice como calcular los segundos momentos para los procesos de traslado del servidor:

\begin{Prop}
Sea $R_{i}$ la Funci\'on Generadora de Probabilidades para el n\'umero de arribos a cada una de las colas de la Red de Sistemas de Visitas C\'iclicas definidas como en (\ref{Ec.R1}). Entonces las derivadas parciales est\'an dadas por las siguientes expresiones:


\begin{eqnarray*}
\frac{\partial^{2} R_{i}\left(P_{1}\left(z_{1}\right)\tilde{P}_{2}\left(z_{2}\right)\hat{P}_{1}\left(w_{1}\right)\hat{P}_{2}\left(w_{2}\right)\right)}{\partial z_{i}^{2}}&=&\left(\frac{\partial P_{i}\left(z_{i}\right)}{\partial z_{i}}\right)^{2}\cdot\frac{\partial^{2} R_{i}\left(P_{1}\left(z_{1}\right)\tilde{P}_{2}\left(z_{2}\right)\hat{P}_{1}\left(w_{1}\right)\hat{P}_{2}\left(w_{2}\right)\right)}{\partial^{2} z_{i}}\\
&+&\left(\frac{\partial P_{i}\left(z_{i}\right)}{\partial z_{i}}\right)^{2}\cdot
\frac{\partial R_{i}\left(P_{1}\left(z_{1}\right)\tilde{P}_{2}\left(z_{2}\right)\hat{P}_{1}\left(w_{1}\right)\hat{P}_{2}\left(w_{2}\right)\right)}{\partial z_{i}}
\end{eqnarray*}



y adem\'as


\begin{eqnarray*}
\frac{\partial^{2} R_{i}\left(P_{1}\left(z_{1}\right)\tilde{P}_{2}\left(z_{2}\right)\hat{P}_{1}\left(w_{1}\right)\hat{P}_{2}\left(w_{2}\right)\right)}{\partial z_{2}\partial z_{1}}&=&\frac{\partial \tilde{P}_{2}\left(z_{2}\right)}{\partial z_{2}}\cdot\frac{\partial P_{1}\left(z_{1}\right)}{\partial z_{1}}\cdot\frac{\partial^{2} R_{i}\left(P_{1}\left(z_{1}\right)\tilde{P}_{2}\left(z_{2}\right)\hat{P}_{1}\left(w_{1}\right)\hat{P}_{2}\left(w_{2}\right)\right)}{\partial z_{2}\partial z_{1}}\\
&+&\frac{\partial \tilde{P}_{2}\left(z_{2}\right)}{\partial z_{2}}\cdot\frac{\partial P_{1}\left(z_{1}\right)}{\partial z_{1}}\cdot\frac{\partial R_{i}\left(P_{1}\left(z_{1}\right)\tilde{P}_{2}\left(z_{2}\right)\hat{P}_{1}\left(w_{1}\right)\hat{P}_{2}\left(w_{2}\right)\right)}{\partial z_{1}},
\end{eqnarray*}



\begin{eqnarray*}
\frac{\partial^{2} R_{i}\left(P_{1}\left(z_{1}\right)\tilde{P}_{2}\left(z_{2}\right)\hat{P}_{1}\left(w_{1}\right)\hat{P}_{2}\left(w_{2}\right)\right)}{\partial w_{i}\partial z_{1}}&=&\frac{\partial \hat{P}_{i}\left(w_{i}\right)}{\partial z_{2}}\cdot\frac{\partial P_{1}\left(z_{1}\right)}{\partial z_{1}}\cdot\frac{\partial^{2} R_{i}\left(P_{1}\left(z_{1}\right)\tilde{P}_{2}\left(z_{2}\right)\hat{P}_{1}\left(w_{1}\right)\hat{P}_{2}\left(w_{2}\right)\right)}{\partial w_{i}\partial z_{1}}\\
&+&\frac{\partial \hat{P}_{i}\left(w_{i}\right)}{\partial z_{2}}\cdot\frac{\partial P_{1}\left(z_{1}\right)}{\partial z_{1}}\cdot\frac{\partial R_{i}\left(P_{1}\left(z_{1}\right)\tilde{P}_{2}\left(z_{2}\right)\hat{P}_{1}\left(w_{1}\right)\hat{P}_{2}\left(w_{2}\right)\right)}{\partial z_{1}},
\end{eqnarray*}
finalmente

\begin{eqnarray*}
\frac{\partial^{2} R_{i}\left(P_{1}\left(z_{1}\right)\tilde{P}_{2}\left(z_{2}\right)\hat{P}_{1}\left(w_{1}\right)\hat{P}_{2}\left(w_{2}\right)\right)}{\partial w_{i}\partial z_{2}}&=&\frac{\partial \hat{P}_{i}\left(w_{i}\right)}{\partial w_{i}}\cdot\frac{\partial \tilde{P}_{2}\left(z_{2}\right)}{\partial z_{2}}\cdot\frac{\partial^{2} R_{i}\left(P_{1}\left(z_{1}\right)\tilde{P}_{2}\left(z_{2}\right)\hat{P}_{1}\left(w_{1}\right)\hat{P}_{2}\left(w_{2}\right)\right)}{\partial w_{i}\partial z_{2}}\\
&+&\frac{\partial \hat{P}_{i}\left(w_{i}\right)}{\partial w_{i}}\cdot\frac{\partial \tilde{P}_{2}\left(z_{2}\right)}{\partial z_{1}}\cdot\frac{\partial R_{i}\left(P_{1}\left(z_{1}\right)\tilde{P}_{2}\left(z_{2}\right)\hat{P}_{1}\left(w_{1}\right)\hat{P}_{2}\left(w_{2}\right)\right)}{\partial z_{2}},
\end{eqnarray*}

para $i=1,2$.
\end{Prop}

%___________________________________________________________________________________________
%
\subsection{Sistema de Ecuaciones Lineales para los Segundos Momentos}
%___________________________________________________________________________________________

En el ap\'endice (\ref{Segundos.Momentos}) se demuestra que las ecuaciones para las ecuaciones parciales mixtas est\'an dadas por:



%___________________________________________________________________________________________
%\subsubsection{Mixtas para $z_{1}$:}
%___________________________________________________________________________________________
%1
\begin{eqnarray*}
f_{1}\left(1,1\right)&=&r_{2}P_{1}^{(2)}\left(1\right)+\mu_{1}^{2}R_{2}^{(2)}\left(1\right)+2\mu_{1}r_{2}\left(\frac{\mu_{1}}{1-\tilde{\mu}_{2}}f_{2}\left(2\right)+f_{2}\left(1\right)\right)+\frac{1}{1-\tilde{\mu}_{2}}P_{1}^{(2)}f_{2}\left(2\right)+\mu_{1}^{2}\tilde{\theta}_{2}^{(2)}\left(1\right)f_{2}\left(2\right)\\
&+&\frac{\mu_{1}}{1-\tilde{\mu}_{2}}f_{2}(1,2)+\frac{\mu_{1}}{1-\tilde{\mu}_{2}}\left(\frac{\mu_{1}}{1-\tilde{\mu}_{2}}f_{2}(2,2)+f_{2}(1,2)\right)+f_{2}(1,1),\\
f_{1}\left(2,1\right)&=&\mu_{1}r_{2}\tilde{\mu}_{2}+\mu_{1}\tilde{\mu}_{2}R_{2}^{(2)}\left(1\right)+r_{2}\tilde{\mu}_{2}\left(\frac{\mu_{1}}{1-\tilde{\mu}_{2}}f_{2}(2)+f_{2}(1)\right),\\
f_{1}\left(3,1\right)&=&\mu_{1}\hat{\mu}_{1}r_{2}+\mu_{1}\hat{\mu}_{1}R_{2}^{(2)}\left(1\right)+r_{2}\frac{\mu_{1}}{1-\tilde{\mu}_{2}}f_{2}(2)+r_{2}\hat{\mu}_{1}\left(\frac{\mu_{1}}{1-\tilde{\mu}_{2}}f_{2}(2)+f_{2}(1)\right)+\mu_{1}r_{2}\hat{F}_{2,1}^{(1)}(1)+\frac{\hat{\mu}_{1}}{1-\tilde{\mu}_{2}}f_{2}(1,2)\\
&+&\left(\frac{\mu_{1}}{1-\tilde{\mu}_{2}}f_{2}(2)+f_{2}(1)\right)\hat{F}_{2,1}^{(1)}(1)+\frac{\mu_{1}\hat{\mu}_{1}}{1-\tilde{\mu}_{2}}f_{2}(2)+\mu_{1}\hat{\mu}_{1}\tilde{\theta}_{2}^{(2)}\left(1\right)f_{2}(2)+\mu_{1}\hat{\mu}_{1}\left(\frac{1}{1-\tilde{\mu}_{2}}\right)^{2}f_{2}(2,2)\\
f_{1}\left(4,1\right)&=&\mu_{1}\hat{\mu}_{2}r_{2}+\mu_{1}\hat{\mu}_{2}R_{2}^{(2)}\left(1\right)+r_{2}\frac{\mu_{1}\hat{\mu}_{2}}{1-\tilde{\mu}_{2}}f_{2}(2)+\mu_{1}r_{2}\hat{F}_{2,2}^{(1)}(1)+r_{2}\hat{\mu}_{2}\left(\frac{\mu_{1}}{1-\tilde{\mu}_{2}}f_{2}(2)+f_{2}(1)\right)+\frac{\hat{\mu}_{2}}{1-\tilde{\mu}_{2}}f_{2}^{(1,2)}\\
&+&\hat{F}_{2,1}^{(1)}(1)\left(\frac{\mu_{1}}{1-\tilde{\mu}_{2}}f_{2}(2)+f_{2}(1)\right)+\frac{\mu_{1}\hat{\mu}_{2}}{1-\tilde{\mu}_{2}}f_{2}(2)
+\mu_{1}\hat{\mu}_{2}\tilde{\theta}_{2}^{(2)}\left(1\right)f_{2}(2)+\mu_{1}\hat{\mu}_{2}\left(\frac{1}{1-\tilde{\mu}_{2}}\right)^{2}f_{2}(2,2),
\end{eqnarray*}



\begin{eqnarray*}
f_{1}\left(1,2\right)&=&\mu_{1}\tilde{\mu}_{2}r_{2}+\mu_{1}\tilde{\mu}_{2}R_{2}^{(2)}\left(1\right)+r_{2}\tilde{\mu}_{2}\left(\frac{\mu_{1}}{1-\tilde{\mu}_{2}}f_{2}(2)+f_{2}(1)\right),\\
f_{1}\left(2,2\right)&=&\tilde{\mu}_{2}^{2}R_{2}^{(2)}(1)+r_{2}\tilde{P}_{2}^{(2)}\left(1\right),\\
f_{1}\left(3,2\right)&=&\hat{\mu}_{1}\tilde{\mu}_{2}r_{2}+\hat{\mu}_{1}\tilde{\mu}_{2}R_{2}^{(2)}(1)+
r_{2}\frac{\hat{\mu}_{1}\tilde{\mu}_{2}}{1-\tilde{\mu}_{2}}f_{2}(2)+r_{2}\tilde{\mu}_{2}\hat{F}_{2,2}^{(1)}(1),\\
f_{1}\left(4,2\right)&=&\hat{\mu}_{2}\tilde{\mu}_{2}r_{2}+\hat{\mu}_{2}\tilde{\mu}_{2}R_{2}^{(2)}(1)+
r_{2}\frac{\hat{\mu}_{2}\tilde{\mu}_{2}}{1-\tilde{\mu}_{2}}f_{2}(2)+r_{2}\tilde{\mu}_{2}\hat{F}_{2,2}^{(1)}(1),
\end{eqnarray*}



\begin{eqnarray*}
f_{1}\left(1,3\right)&=&\mu_{1}\hat{\mu}_{1}r_{2}+\mu_{1}\hat{\mu}_{1}R_{2}^{(2)}\left(1\right)+\frac{\mu_{1}\hat{\mu}_{1}}{1-\tilde{\mu}_{2}}f_{2}(2)+r_{2}\frac{\mu_{1}\hat{\mu}_{1}}{1-\tilde{\mu}_{2}}f_{2}(2)+\mu_{1}\hat{\mu}_{1}\tilde{\theta}_{2}^{(2)}\left(1\right)f_{2}(2)+r_{2}\mu_{1}\hat{F}_{2,1}^{(1)}(1)\\
&+&r_{2}\hat{\mu}_{1}\left(\frac{\mu_{1}}{1-\tilde{\mu}_{2}}f_{2}(2)+f_{2}\left(1\right)\right)+\left(\frac{\mu_{1}}{1-\tilde{\mu}_{2}}f_{2}\left(1\right)+f_{2}\left(1\right)\right)\hat{F}_{2,1}^{(1)}(1)+\frac{\hat{\mu}_{1}}{1-\tilde{\mu}_{2}}\left(\frac{\mu_{1}}{1-\tilde{\mu}_{2}}f_{2}(2,2)+f_{2}^{(1,2)}\right),\\
f_{1}\left(2,3\right)&=&\tilde{\mu}_{2}\hat{\mu}_{1}r_{2}+\tilde{\mu}_{2}\hat{\mu}_{1}R_{2}^{(2)}\left(1\right)+r_{2}\frac{\tilde{\mu}_{2}\hat{\mu}_{1}}{1-\tilde{\mu}_{2}}f_{2}(2)+r_{2}\tilde{\mu}_{2}\hat{F}_{2,1}^{(1)}(1),\\
f_{1}\left(3,3\right)&=&\hat{\mu}_{1}^{2}R_{2}^{(2)}\left(1\right)+r_{2}\hat{P}_{1}^{(2)}\left(1\right)+2r_{2}\frac{\hat{\mu}_{1}^{2}}{1-\tilde{\mu}_{2}}f_{2}(2)+\hat{\mu}_{1}^{2}\tilde{\theta}_{2}^{(2)}\left(1\right)f_{2}(2)+\frac{1}{1-\tilde{\mu}_{2}}\hat{P}_{1}^{(2)}\left(1\right)f_{2}(2)\\
&+&\frac{\hat{\mu}_{1}^{2}}{1-\tilde{\mu}_{2}}f_{2}(2,2)+2r_{2}\hat{\mu}_{1}\hat{F}_{2,1}^{(1)}(1)+2\frac{\hat{\mu}_{1}}{1-\tilde{\mu}_{2}}f_{2}(2)\hat{F}_{2,1}^{(1)}(1)+\hat{f}_{2,1}^{(2)}(1),\\
f_{1}\left(4,3\right)&=&r_{2}\hat{\mu}_{2}\hat{\mu}_{1}+\hat{\mu}_{1}\hat{\mu}_{2}R_{2}^{(2)}(1)+\frac{\hat{\mu}_{1}\hat{\mu}_{2}}{1-\tilde{\mu}_{2}}f_{2}\left(2\right)+2r_{2}\frac{\hat{\mu}_{1}\hat{\mu}_{2}}{1-\tilde{\mu}_{2}}f_{2}\left(2\right)+\hat{\mu}_{2}\hat{\mu}_{1}\tilde{\theta}_{2}^{(2)}\left(1\right)f_{2}\left(2\right)+r_{2}\hat{\mu}_{1}\hat{F}_{2,2}^{(1)}(1)\\
&+&\frac{\hat{\mu}_{1}}{1-\tilde{\mu}_{2}}f_{2}\left(2\right)\hat{F}_{2,2}^{(1)}(1)+\hat{\mu}_{1}\hat{\mu}_{2}\left(\frac{1}{1-\tilde{\mu}_{2}}\right)^{2}f_{2}(2,2)+r_{2}\hat{\mu}_{2}\hat{F}_{2,1}^{(1)}(1)+\frac{\hat{\mu}_{2}}{1-\tilde{\mu}_{2}}f_{2}\left(2\right)\hat{F}_{2,1}^{(1)}(1)+\hat{f}_{2}(1,2),
\end{eqnarray*}



\begin{eqnarray*}
f_{1}\left(1,4\right)&=&r_{2}\mu_{1}\hat{\mu}_{2}+\mu_{1}\hat{\mu}_{2}R_{2}^{(2)}(1)+\frac{\mu_{1}\hat{\mu}_{2}}{1-\tilde{\mu}_{2}}f_{2}(2)+r_{2}\frac{\mu_{1}\hat{\mu}_{2}}{1-\tilde{\mu}_{2}}f_{2}(2)+\mu_{1}\hat{\mu}_{2}\tilde{\theta}_{2}^{(2)}\left(1\right)f_{2}(2)+r_{2}\mu_{1}\hat{F}_{2,2}^{(1)}(1)\\
&+&r_{2}\hat{\mu}_{2}\left(\frac{\mu_{1}}{1-\tilde{\mu}_{2}}f_{2}(2)+f_{2}(1)\right)+\hat{F}_{2,2}^{(1)}(1)\left(\frac{\mu_{1}}{1-\tilde{\mu}_{2}}f_{2}(2)+f_{2}(1)\right)+\frac{\hat{\mu}_{2}}{1-\tilde{\mu}_{2}}\left(\frac{\mu_{1}}{1-\tilde{\mu}_{2}}f_{2}(2,2)+f_{2}(1,2)\right),\\
f_{1}\left(2,4\right)
&=&r_{2}\tilde{\mu}_{2}\hat{\mu}_{2}+\tilde{\mu}_{2}\hat{\mu}_{2}R_{2}^{(2)}(1)+r_{2}\frac{\tilde{\mu}_{2}\hat{\mu}_{2}}{1-\tilde{\mu}_{2}}f_{2}(2)+r_{2}\tilde{\mu}_{2}\hat{F}_{2,2}^{(1)}(1),\\
f_{1}\left(3,4\right)&=&r_{2}\hat{\mu}_{1}\hat{\mu}_{2}+\hat{\mu}_{1}\hat{\mu}_{2}R_{2}^{(2)}\left(1\right)+\frac{\hat{\mu}_{1}\hat{\mu}_{2}}{1-\tilde{\mu}_{2}}f_{2}(2)+2r_{2}\frac{\hat{\mu}_{1}\hat{\mu}_{2}}{1-\tilde{\mu}_{2}}f_{2}(2)+\hat{\mu}_{1}\hat{\mu}_{2}\theta_{2}^{(2)}\left(1\right)f_{2}(2)+r_{2}\hat{\mu}_{1}\hat{F}_{2,2}^{(1)}(1)\\
&+&\frac{\hat{\mu}_{1}}{1-\tilde{\mu}_{2}}f_{2}(2)\hat{F}_{2,2}^{(1)}(1)+\hat{\mu}_{1}\hat{\mu}_{2}\left(\frac{1}{1-\tilde{\mu}_{2}}\right)^{2}f_{2}(2,2)+r_{2}\hat{\mu}_{2}\hat{F}_{2,2}^{(1)}(1)+\frac{\hat{\mu}_{2}}{1-\tilde{\mu}_{2}}f_{2}(2)\hat{F}_{2,1}^{(1)}(1)+\hat{f}_{2}^{(2)}(1,2),\\
f_{1}\left(4,4\right)&=&\hat{\mu}_{2}^{2}R_{2}^{(2)}(1)+r_{2}\hat{P}_{2}^{(2)}\left(1\right)+2r_{2}\frac{\hat{\mu}_{2}^{2}}{1-\tilde{\mu}_{2}}f_{2}(2)+\hat{\mu}_{2}^{2}\tilde{\theta}_{2}^{(2)}\left(1\right)f_{2}(2)+\frac{1}{1-\tilde{\mu}_{2}}\hat{P}_{2}^{(2)}\left(1\right)f_{2}(2)\\
&+&2r_{2}\hat{\mu}_{2}\hat{F}_{2,2}^{(1)}(1)+2\frac{\hat{\mu}_{2}}{1-\tilde{\mu}_{2}}f_{2}(2)\hat{F}_{2,2}^{(1)}(1)+\left(\frac{\hat{\mu}_{2}}{1-\tilde{\mu}_{2}}\right)^{2}f_{2}(2,2)+\hat{f}_{2,2}^{(2)}(1),
\end{eqnarray*}



\begin{eqnarray*}
f_{2}\left(1,1\right)&=&r_{1}P_{1}^{(2)}\left(1\right)+\mu_{1}^{2}R_{1}^{(2)}\left(1\right),\\
f_{2}\left(2,1\right)&=&\mu_{1}\tilde{\mu}_{2}r_{1}+\mu_{1}\tilde{\mu}_{2}R_{1}^{(2)}(1)+
r_{1}\mu_{1}\left(\frac{\tilde{\mu}_{2}}{1-\mu_{1}}f_{1}(1)+f_{1}(2)\right),\\
f_{2}\left(3,1\right)&=&r_{1}\mu_{1}\hat{\mu}_{1}+\mu_{1}\hat{\mu}_{1}R_{1}^{(2)}\left(1\right)+r_{1}\frac{\mu_{1}\hat{\mu}_{1}}{1-\mu_{1}}f_{1}(1)+r_{1}\mu_{1}\hat{F}_{1,1}^{(1)}(1),\\
f_{2}\left(4,1\right)&=&\mu_{1}\hat{\mu}_{2}r_{1}+\mu_{1}\hat{\mu}_{2}R_{1}^{(2)}\left(1\right)+r_{1}\mu_{1}\hat{F}_{1,2}^{(1)}(1)+r_{1}\frac{\mu_{1}\hat{\mu}_{2}}{1-\mu_{1}}f_{1}(1),
\end{eqnarray*}
\begin{eqnarray*}
f_{2}\left(1,2\right)&=&r_{1}\mu_{1}\tilde{\mu}_{2}+\mu_{1}\tilde{\mu}_{2}R_{1}^{(2)}\left(1\right)+r_{1}\mu_{1}\left(\frac{\tilde{\mu}_{2}}{1-\mu_{1}}f_{1}(1)+f_{1}(2)\right),\\
f_{2}\left(2,2\right)&=&\tilde{\mu}_{2}^{2}R_{1}^{(2)}\left(1\right)+r_{1}\tilde{P}_{2}^{(2)}\left(1\right)+2r_{1}\tilde{\mu}_{2}\left(\frac{\tilde{\mu}_{2}}{1-\mu_{1}}f_{1}(1)+f_{1}(2)\right)+f_{1}(2,2)+\tilde{\mu}_{2}^{2}\theta_{1}^{(2)}\left(1\right)f_{1}(1)\\
&+&\frac{1}{1-\mu_{1}}\tilde{P}_{2}^{(2)}\left(1\right)f_{1}(1)+\frac{\tilde{\mu}_{2}}{1-\mu_{1}}f_{1}(1,2)+\frac{\tilde{\mu}_{2}}{1-\mu_{1}}\left(\frac{\tilde{\mu}_{2}}{1-\mu_{1}}f_{1}(1,1)+f_{1}(1,2)\right),\\
f_{2}\left(3,2\right)&=&\tilde{\mu}_{2}\hat{\mu}_{1}r_{1}+\tilde{\mu}_{2}\hat{\mu}_{1}R_{1}^{(2)}\left(1\right)+r_{1}\frac{\tilde{\mu}_{2}\hat{\mu}_{1}}{1-\mu_{1}}f_{1}(1)+\hat{\mu}_{1}r_{1}\left(\frac{\tilde{\mu}_{2}}{1-\mu_{1}}f_{1}(1)+f_{1}(2)\right)+r_{1}\tilde{\mu}_{2}\hat{F}_{1,1}^{(1)}(1)\\
&+&\left(\frac{\tilde{\mu}_{2}}{1-\mu_{1}}f_{1}(1)+f_{1}(2)\right)\hat{F}_{1,1}^{(1)}(1)+\frac{\tilde{\mu}_{2}\hat{\mu}_{1}}{1-\mu_{1}}f_{1}(1)+\tilde{\mu}_{2}\hat{\mu}_{1}\theta_{1}^{(2)}\left(1\right)f_{1}(1)+\frac{\hat{\mu}_{1}}{1-\mu_{1}}f_{1}(1,2)\\
&+&\left(\frac{1}{1-\mu_{1}}\right)^{2}\tilde{\mu}_{2}\hat{\mu}_{1}f_{1}(1,1),\\
f_{2}\left(4,2\right)&=&\hat{\mu}_{2}\tilde{\mu}_{2}r_{1}+\hat{\mu}_{2}\tilde{\mu}_{2}R_{1}^{(2)}(1)+r_{1}\tilde{\mu}_{2}\hat{F}_{1,2}^{(1)}(1)+r_{1}\frac{\hat{\mu}_{2}\tilde{\mu}_{2}}{1-\mu_{1}}f_{1}(1)+\hat{\mu}_{2}r_{1}\left(\frac{\tilde{\mu}_{2}}{1-\mu_{1}}f_{1}(1)+f_{1}(2)\right)\\
&+&\left(\frac{\tilde{\mu}_{2}}{1-\mu_{1}}f_{1}(1)+f_{1}(2)\right)\hat{F}_{1,2}^{(1)}(1)+\frac{\tilde{\mu}_{2}\hat{\mu_{2}}}{1-\mu_{1}}f_{1}(1)+\hat{\mu}_{2}\tilde{\mu}_{2}\theta_{1}^{(2)}\left(1\right)f_{1}(1)+\frac{\hat{\mu}_{2}}{1-\mu_{1}}f_{1}(1,2)\\
&+&\tilde{\mu}_{2}\hat{\mu}_{2}\left(\frac{1}{1-\mu_{1}}\right)^{2}f_{1}(1,1),
\end{eqnarray*}



\begin{eqnarray*}
f_{2}\left(1,3\right)&=&r_{1}\mu_{1}\hat{\mu}_{1}+\mu_{1}\hat{\mu}_{1}R_{1}^{(2)}(1)+r_{1}\frac{\mu_{1}\hat{\mu}_{1}}{1-\mu_{1}}f_{1}(1)+r_{1}\mu_{1}\hat{F}_{1,1}^{(1)}(1),\\
 f_{2}\left(2,3\right)&=&r_{1}\hat{\mu}_{1}\tilde{\mu}_{2}+\tilde{\mu}_{2}\hat{\mu}_{1}R_{1}^{(2)}\left(1\right)+\frac{\hat{\mu}_{1}\tilde{\mu}_{2}}{1-\mu_{1}}f_{1}(1)+r_{1}\frac{\hat{\mu}_{1}\tilde{\mu}_{2}}{1-\mu_{1}}f_{1}(1)+\hat{\mu}_{1}\tilde{\mu}_{2}\theta_{1}^{(2)}\left(1\right)f_{1}(1)+r_{1}\tilde{\mu}_{2}\hat{F}_{1,1}(1)\\
&+&r_{1}\hat{\mu}_{1}\left(f_{1}(1)+\frac{\tilde{\mu}_{2}}{1-\mu_{1}}f_{1}(1)\right)+
+\left(f_{1}(2)+\frac{\tilde{\mu}_{2}}{1-\mu_{1}}f_{1}(1)\right)\hat{F}_{1,1}(1)+\frac{\hat{\mu}_{1}}{1-\mu_{1}}\left(f_{1}(1,2)+\frac{\tilde{\mu}_{2}}{1-\mu_{1}}f_{1}(1,1)\right),\\
f_{2}\left(3,3\right)&=&\hat{\mu}_{1}^{2}R_{1}^{(2)}\left(1\right)+r_{1}\hat{P}_{1}^{(2)}\left(1\right)+2r_{1}\frac{\hat{\mu}_{1}^{2}}{1-\mu_{1}}f_{1}(1)+\hat{\mu}_{1}^{2}\theta_{1}^{(2)}\left(1\right)f_{1}(1)+2r_{1}\hat{\mu}_{1}\hat{F}_{1,1}^{(1)}(1)\\
&+&\frac{1}{1-\mu_{1}}\hat{P}_{1}^{(2)}\left(1\right)f_{1}(1)+2\frac{\hat{\mu}_{1}}{1-\mu_{1}}f_{1}(1)\hat{F}_{1,1}(1)+\left(\frac{\hat{\mu}_{1}}{1-\mu_{1}}\right)^{2}f_{1}(1,1)+\hat{f}_{1,1}^{(2)}(1),\\
f_{2}\left(4,3\right)&=&r_{1}\hat{\mu}_{1}\hat{\mu}_{2}+\hat{\mu}_{1}\hat{\mu}_{2}R_{1}^{(2)}\left(1\right)+r_{1}\hat{\mu}_{1}\hat{F}_{1,2}(1)+
\frac{\hat{\mu}_{1}\hat{\mu}_{2}}{1-\mu_{1}}f_{1}(1)+2r_{1}\frac{\hat{\mu}_{1}\hat{\mu}_{2}}{1-\mu_{1}}f_{1}(1)+r_{1}\hat{\mu}_{2}\hat{F}_{1,1}(1)+\hat{f}_{1}^{(2)}(1,2)\\
&+&\hat{\mu}_{1}\hat{\mu}_{2}\theta_{1}^{(2)}\left(1\right)f_{1}(1)+\frac{\hat{\mu}_{1}}{1-\mu_{1}}f_{1}(1)\hat{F}_{1,2}(1)+\frac{\hat{\mu}_{2}}{1-\mu_{1}}\hat{F}_{1,1}(1)f_{1}(1)+\hat{\mu}_{1}\hat{\mu}_{2}\left(\frac{1}{1-\mu_{1}}\right)^{2}f_{1}(2,2),
\end{eqnarray*}



\begin{eqnarray*}
f_{2}\left(1,4\right)&=&r_{1}\mu_{1}\hat{\mu}_{2}+\mu_{1}\hat{\mu}_{2}R_{1}^{(2)}\left(1\right)+r_{1}\mu_{1}\hat{F}_{1,2}(1)+r_{1}\frac{\mu_{1}\hat{\mu}_{2}}{1-\mu_{1}}f_{1}(1),\\
f_{2}\left(2,4\right)&=&r_{1}\hat{\mu}_{2}\tilde{\mu}_{2}+\hat{\mu}_{2}\tilde{\mu}_{2}R_{1}^{(2)}\left(1\right)+r_{1}\tilde{\mu}_{2}\hat{F}_{1,2}(1)+\frac{\hat{\mu}_{2}\tilde{\mu}_{2}}{1-\mu_{1}}f_{1}(1)+r_{1}\frac{\hat{\mu}_{2}\tilde{\mu}_{2}}{1-\mu_{1}}f_{1}(1)+\hat{\mu}_{2}\tilde{\mu}_{2}\theta_{1}^{(2)}\left(1\right)f_{1}(1)\\
&+&r_{1}\hat{\mu}_{2}\left(f_{1}(2)+\frac{\tilde{\mu}_{2}}{1-\mu_{1}}f_{1}(1)\right)+\left(f_{1}(2)+\frac{\tilde{\mu}_{2}}{1-\mu_{1}}f_{1}(1)\right)\hat{F}_{1,2}(1)+\frac{\hat{\mu}_{2}}{1-\mu_{1}}\left(f_{1}(1,2)+\frac{\tilde{\mu}_{2}}{1-\mu_{1}}f_{1}(1,1)\right),\\
f_{2}\left(3,4\right)&=&r_{1}\hat{\mu}_{1}\hat{\mu}_{2}+\hat{\mu}_{1}\hat{\mu}_{2}R_{1}^{(2)}\left(1\right)+r_{1}\hat{\mu}_{1}\hat{F}_{1,2}(1)+
\frac{\hat{\mu}_{1}\hat{\mu}_{2}}{1-\mu_{1}}f_{1}(1)+2r_{1}\frac{\hat{\mu}_{1}\hat{\mu}_{2}}{1-\mu_{1}}f_{1}(1)+\hat{\mu}_{1}\hat{\mu}_{2}\theta_{1}^{(2)}\left(1\right)f_{1}(1)\\
&+&+\frac{\hat{\mu}_{1}}{1-\mu_{1}}\hat{F}_{1,2}(1)f_{1}(1)+r_{1}\hat{\mu}_{2}\hat{F}_{1,1}(1)+\frac{\hat{\mu}_{2}}{1-\mu_{1}}\hat{F}_{1,1}(1)f_{1}(1)+\hat{f}_{1}^{(2)}(1,2)+\hat{\mu}_{1}\hat{\mu}_{2}\left(\frac{1}{1-\mu_{1}}\right)^{2}f_{1}(1,1),\\
f_{2}\left(4,4\right)&=&\hat{\mu}_{2}R_{1}^{(2)}\left(1\right)+r_{1}\hat{P}_{2}^{(2)}\left(1\right)+2r_{1}\hat{\mu}_{2}\hat{F}_{1}^{(0,1)}+\hat{f}_{1,2}^{(2)}(1)+2r_{1}\frac{\hat{\mu}_{2}^{2}}{1-\mu_{1}}f_{1}(1)+\hat{\mu}_{2}^{2}\theta_{1}^{(2)}\left(1\right)f_{1}(1)\\
&+&\frac{1}{1-\mu_{1}}\hat{P}_{2}^{(2)}\left(1\right)f_{1}(1) +
2\frac{\hat{\mu}_{2}}{1-\mu_{1}}f_{1}(1)\hat{F}_{1,2}(1)+\left(\frac{\hat{\mu}_{2}}{1-\mu_{1}}\right)^{2}f_{1}(1,1),
\end{eqnarray*}



\begin{eqnarray*}
\hat{f}_{1}\left(1,1\right)&=&\hat{r}_{2}P_{1}^{(2)}\left(1\right)+
\mu_{1}^{2}\hat{R}_{2}^{(2)}\left(1\right)+
2\hat{r}_{2}\frac{\mu_{1}^{2}}{1-\hat{\mu}_{2}}\hat{f}_{2}(2)+
\frac{1}{1-\hat{\mu}_{2}}P_{1}^{(2)}\left(1\right)\hat{f}_{2}(2)+
\mu_{1}^{2}\hat{\theta}_{2}^{(2)}\left(1\right)\hat{f}_{2}(2)\\
&+&\left(\frac{\mu_{1}^{2}}{1-\hat{\mu}_{2}}\right)^{2}\hat{f}_{2}(2,2)+2\hat{r}_{2}\mu_{1}F_{2,1}(1)+2\frac{\mu_{1}}{1-\hat{\mu}_{2}}\hat{f}_{2}(2)F_{2,1}(1)+F_{2,1}^{(2)}(1),\\
\hat{f}_{1}\left(2,1\right)&=&\hat{r}_{2}\mu_{1}\tilde{\mu}_{2}+\mu_{1}\tilde{\mu}_{2}\hat{R}_{2}^{(2)}\left(1\right)+\hat{r}_{2}\mu_{1}F_{2,2}(1)+
\frac{\mu_{1}\tilde{\mu}_{2}}{1-\hat{\mu}_{2}}\hat{f}_{2}(2)+2\hat{r}_{2}\frac{\mu_{1}\tilde{\mu}_{2}}{1-\hat{\mu}_{2}}\hat{f}_{2}(2)+f_{2,1}^{(2)}(1)\\
&+&\mu_{1}\tilde{\mu}_{2}\hat{\theta}_{2}^{(2)}\left(1\right)\hat{f}_{2}(2)+\frac{\mu_{1}}{1-\hat{\mu}_{2}}F_{2,2}(1)\hat{f}_{2}(2)+\mu_{1} \tilde{\mu}_{2}\left(\frac{1}{1-\hat{\mu}_{2}}\right)^{2}\hat{f}_{2}(2,2)+\hat{r}_{2}\tilde{\mu}_{2}F_{2,1}(1)\\
&+&\frac{\tilde{\mu}_{2}}{1-\hat{\mu}_{2}}\hat{f}_{2}(2)F_{2,1}(1),\\
\hat{f}_{1}\left(3,1\right)&=&\hat{r}_{2}\mu_{1}\hat{\mu}_{1}+\mu_{1}\hat{\mu}_{1}\hat{R}_{2}^{(2)}\left(1\right)+\hat{r}_{2}\frac{\mu_{1}\hat{\mu}_{1}}{1-\hat{\mu}_{2}}\hat{f}_{2}(2)+\hat{r}_{2}\hat{\mu}_{1}F_{2,1}(1)+\hat{r}_{2}\mu_{1}\hat{f}_{2}(1)\\
&+&F_{2,1}(1)\hat{f}_{2}(1)+\frac{\mu_{1}}{1-\hat{\mu}_{2}}\hat{f}_{2}(1,2),\\
\hat{f}_{1}\left(4,1\right)&=&\hat{r}_{2}\mu_{1}\hat{\mu}_{2}+\mu_{1}\hat{\mu}_{2}\hat{R}_{2}^{(2)}\left(1\right)+\frac{\mu_{1}\hat{\mu}_{2}}{1-\hat{\mu}_{2}}\hat{f}_{2}(2)+2\hat{r}_{2}\frac{\mu_{1}\hat{\mu}_{2}}{1-\hat{\mu}_{2}}\hat{f}_{2}(2)+\mu_{1}\hat{\mu}_{2}\hat{\theta}_{2}^{(2)}\left(1\right)\hat{f}_{2}(2)\\
&+&\mu_{1}\hat{\mu}_{2}\left(\frac{1}{1-\hat{\mu}_{2}}\right)^{2}\hat{f}_{2}(2,2)+\hat{r}_{2}\hat{\mu}_{2}F_{2,1}(1)+\frac{\hat{\mu}_{2}}{1-\hat{\mu}_{2}}\hat{f}_{2}(2)F_{2,1}(1),
\end{eqnarray*}



\begin{eqnarray*}
\hat{f}_{1}\left(1,2\right)&=&\hat{r}_{2}\mu_{1}\tilde{\mu}_{2}+\mu_{1}\tilde{\mu}_{2}\hat{R}_{2}^{(2)}\left(1\right)+\mu_{1}\hat{r}_{2}F_{2,2}(1)+
\frac{\mu_{1}\tilde{\mu}_{2}}{1-\hat{\mu}_{2}}\hat{f}_{2}(2)+2\hat{r}_{2}\frac{\mu_{1}\tilde{\mu}_{2}}{1-\hat{\mu}_{2}}\hat{f}_{2}(2)\\
&+&\mu_{1}\tilde{\mu}_{2}\hat{\theta}_{2}^{(2)}\left(1\right)\hat{f}_{2}(2)+\frac{\mu_{1}}{1-\hat{\mu}_{2}}F_{2,2}(1)\hat{f}_{2}(2)+\mu_{1}\tilde{\mu}_{2}\left(\frac{1}{1-\hat{\mu}_{2}}\right)^{2}\hat{f}_{2}(2,2)\\
&+&\hat{r}_{2}\tilde{\mu}_{2}F_{2,1}(1)+\frac{\tilde{\mu}_{2}}{1-\hat{\mu}_{2}}\hat{f}_{2}(2)F_{2,1}(1)+f_{2}^{(2)}(1,2),\\
\hat{f}_{1}\left(2,2\right)&=&\hat{r}_{2}\tilde{P}_{2}^{(2)}\left(1\right)+\tilde{\mu}_{2}^{2}\hat{R}_{2}^{(2)}\left(1\right)+2\hat{r}_{2}\tilde{\mu}_{2}F_{2,2}(1)+2\hat{r}_{2}\frac{\tilde{\mu}_{2}^{2}}{1-\hat{\mu}_{2}}\hat{f}_{2}(2)+f_{2,2}^{(2)}(1)\\
&+&\frac{1}{1-\hat{\mu}_{2}}\tilde{P}_{2}^{(2)}\left(1\right)\hat{f}_{2}(2)+\tilde{\mu}_{2}^{2}\hat{\theta}_{2}^{(2)}\left(1\right)\hat{f}_{2}(2)+2\frac{\tilde{\mu}_{2}}{1-\hat{\mu}_{2}}F_{2,2}(1)\hat{f}_{2}(2)+\left(\frac{\tilde{\mu}_{2}}{1-\hat{\mu}_{2}}\right)^{2}\hat{f}_{2}(2,2),\\
\hat{f}_{1}\left(3,2\right)&=&\hat{r}_{2}\tilde{\mu}_{2}\hat{\mu}_{1}+\tilde{\mu}_{2}\hat{\mu}_{1}\hat{R}_{2}^{(2)}\left(1\right)+\hat{r}_{2}\hat{\mu}_{1}F_{2,2}(1)+\hat{r}_{2}\frac{\tilde{\mu}_{2}\hat{\mu}_{1}}{1-\hat{\mu}_{2}}\hat{f}_{2}(2)+\hat{r}_{2}\tilde{\mu}_{2}\hat{f}_{2}(1)+F_{2,2}(1)\hat{f}_{2}(1)+\frac{\tilde{\mu}_{2}}{1-\hat{\mu}_{2}}\hat{f}_{2}(1,2),\\
\hat{f}_{1}\left(4,2\right)&=&\hat{r}_{2}\tilde{\mu}_{2}\hat{\mu}_{2}+\tilde{\mu}_{2}\hat{\mu}_{2}\hat{R}_{2}^{(2)}\left(1\right)+\hat{r}_{2}\hat{\mu}_{2}F_{2,2}(1)+
\frac{\tilde{\mu}_{2}\hat{\mu}_{2}}{1-\hat{\mu}_{2}}\hat{f}_{2}(2)+2\hat{r}_{2}\frac{\tilde{\mu}_{2}\hat{\mu}_{2}}{1-\hat{\mu}_{2}}\hat{f}_{2}(2)\\
&+&\tilde{\mu}_{2}\hat{\mu}_{2}\hat{\theta}_{2}^{(2)}\left(1\right)\hat{f}_{2}(2)+\frac{\hat{\mu}_{2}}{1-\hat{\mu}_{2}}F_{2,2}(1)\hat{f}_{2}(1)+\tilde{\mu}_{2}\hat{\mu}_{2}\left(\frac{1}{1-\hat{\mu}_{2}}\right)\hat{f}_{2}(2,2),\\
\end{eqnarray*}

\begin{eqnarray*}
\hat{f}_{1}\left(1,3\right)&=&\hat{r}_{2}\mu_{1}\hat{\mu}_{1}+\mu_{1}\hat{\mu}_{1}\hat{R}_{2}^{(2)}\left(1\right)+\hat{r}_{2}\frac{\mu_{1}\hat{\mu}_{1}}{1-\hat{\mu}_{2}}\hat{f}_{2}(2)+\hat{r}_{2}\hat{\mu}_{1}F_{2,1}(1)+\hat{r}_{2}\mu_{1}\hat{f}_{2}(1)+F_{2,1}(1)\hat{f}_{2}(1)+\frac{\mu_{1}}{1-\hat{\mu}_{2}}\hat{f}_{2}(1,2),\\
\hat{f}_{1}\left(2,3\right)&=&\hat{r}_{2}\tilde{\mu}_{2}\hat{\mu}_{1}+\tilde{\mu}_{2}\hat{\mu}_{1}\hat{R}_{2}^{(2)}\left(1\right)+\hat{r}_{2}\hat{\mu}_{1}F_{2,2}(1)+\hat{r}_{2}\frac{\tilde{\mu}_{2}\hat{\mu}_{1}}{1-\hat{\mu}_{2}}\hat{f}_{2}(2)+\hat{r}_{2}\tilde{\mu}_{2}\hat{f}_{2}(1)+F_{2,2}(1)\hat{f}_{2}(1)+\frac{\tilde{\mu}_{2}}{1-\hat{\mu}_{2}}\hat{f}_{2}(1,2),\\
\hat{f}_{1}\left(3,3\right)&=&\hat{r}_{2}\hat{P}_{1}^{(2)}\left(1\right)+\hat{\mu}_{1}^{2}\hat{R}_{2}^{(2)}\left(1\right)+2\hat{r}_{2}\hat{\mu}_{1}\hat{f}_{2}(1)+\hat{f}_{2}(1,1),\\
\hat{f}_{1}\left(4,3\right)&=&\hat{r}_{2}\hat{\mu}_{1}\hat{\mu}_{2}+\hat{\mu}_{1}\hat{\mu}_{2}\hat{R}_{2}^{(2)}\left(1\right)+
\hat{r}_{2}\frac{\hat{\mu}_{2}\hat{\mu}_{1}}{1-\hat{\mu}_{2}}\hat{f}_{2}(2)+\hat{r}_{2}\hat{\mu}_{2}\hat{f}_{2}(1)+\frac{\hat{\mu}_{2}}{1-\hat{\mu}_{2}}\hat{f}_{2}(1,2),
\end{eqnarray*}



\begin{eqnarray*}
\hat{f}_{1}\left(1,4\right)&=&\hat{r}_{2}\mu_{1}\hat{\mu}_{2}+\mu_{1}\hat{\mu}_{2}\hat{R}_{2}^{(2)}\left(1\right)+
\frac{\mu_{1}\hat{\mu}_{2}}{1-\hat{\mu}_{2}}\hat{f}_{2}(2) +2\hat{r}_{2}\frac{\mu_{1}\hat{\mu}_{2}}{1-\hat{\mu}_{2}}\hat{f}_{2}(2)\\
&+&\mu_{1}\hat{\mu}_{2}\hat{\theta}_{2}^{(2)}\left(1\right)\hat{f}_{2}(2)+\mu_{1}\hat{\mu}_{2}\left(\frac{1}{1-\hat{\mu}_{2}}\right)^{2}\hat{f}_{2}(2,2)+\hat{r}_{2}\hat{\mu}_{2}F_{2,1}(1)+\frac{\hat{\mu}_{2}}{1-\hat{\mu}_{2}}\hat{f}_{2}(2)F_{2,1}(1),\\\hat{f}_{1}\left(2,4\right)&=&\hat{r}_{2}\tilde{\mu}_{2}\hat{\mu}_{2}+\tilde{\mu}_{2}\hat{\mu}_{2}\hat{R}_{2}^{(2)}\left(1\right)+\hat{r}_{2}\hat{\mu}_{2}F_{2,2}(1)+\frac{\tilde{\mu}_{2}\hat{\mu}_{2}}{1-\hat{\mu}_{2}}\hat{f}_{2}(2)+2\hat{r}_{2}\frac{\tilde{\mu}_{2}\hat{\mu}_{2}}{1-\hat{\mu}_{2}}\hat{f}_{2}(2)\\
&+&\tilde{\mu}_{2}\hat{\mu}_{2}\hat{\theta}_{2}^{(2)}\left(1\right)\hat{f}_{2}(2)+\frac{\hat{\mu}_{2}}{1-\hat{\mu}_{2}}\hat{f}_{2}(2)F_{2,2}(1)+\tilde{\mu}_{2}\hat{\mu}_{2}\left(\frac{1}{1-\hat{\mu}_{2}}\right)^{2}\hat{f}_{2}(2,2),\\
\hat{f}_{1}\left(3,4\right)&=&\hat{r}_{2}\hat{\mu}_{1}\hat{\mu}_{2}+\hat{\mu}_{1}\hat{\mu}_{2}\hat{R}_{2}^{(2)}\left(1\right)+
\hat{r}_{2}\frac{\hat{\mu}_{1}\hat{\mu}_{2}}{1-\hat{\mu}_{2}}\hat{f}_{2}(2)+
\hat{r}_{2}\hat{\mu}_{2}\hat{f}_{2}(1)+\frac{\hat{\mu}_{2}}{1-\hat{\mu}_{2}}\hat{f}_{2}(1,2),\\
\hat{f}_{1}\left(4,4\right)&=&\hat{r}_{2}P_{2}^{(2)}\left(1\right)+\hat{\mu}_{2}^{2}\hat{R}_{2}^{(2)}\left(1\right)+2\hat{r}_{2}\frac{\hat{\mu}_{2}^{2}}{1-\hat{\mu}_{2}}\hat{f}_{2}(2)+\frac{1}{1-\hat{\mu}_{2}}\hat{P}_{2}^{(2)}\left(1\right)\hat{f}_{2}(2)+\hat{\mu}_{2}^{2}\hat{\theta}_{2}^{(2)}\left(1\right)\hat{f}_{2}(2)+\left(\frac{\hat{\mu}_{2}}{1-\hat{\mu}_{2}}\right)^{2}\hat{f}_{2}(2,2),
\end{eqnarray*}



\begin{eqnarray*}
\hat{f}_{2}\left(1,1\right)&=&\hat{r}_{1}P_{1}^{(2)}\left(1\right)+
\mu_{1}^{2}\hat{R}_{1}^{(2)}\left(1\right)+2\hat{r}_{1}\mu_{1}F_{1,1}(1)+
2\hat{r}_{1}\frac{\mu_{1}^{2}}{1-\hat{\mu}_{1}}\hat{f}_{1}(1)+\frac{1}{1-\hat{\mu}_{1}}P_{1}^{(2)}\left(1\right)\hat{f}_{1}(1)\\
&+&\mu_{1}^{2}\hat{\theta}_{1}^{(2)}\left(1\right)\hat{f}_{1}(1)+2\frac{\mu_{1}}{1-\hat{\mu}_{1}}\hat{f}_{1}^(1)F_{1,1}(1)+f_{1,1}^{(2)}(1)+\left(\frac{\mu_{1}}{1-\hat{\mu}_{1}}\right)^{2}\hat{f}_{1}^{(1,1)},\\
\hat{f}_{2}\left(2,1\right)&=&\hat{r}_{1}\mu_{1}\tilde{\mu}_{2}+\mu_{1}\tilde{\mu}_{2}\hat{R}_{1}^{(2)}\left(1\right)+
\hat{r}_{1}\mu_{1}F_{1,2}(1)+\tilde{\mu}_{2}\hat{r}_{1}F_{1,1}(1)+
\frac{\mu_{1}\tilde{\mu}_{2}}{1-\hat{\mu}_{1}}\hat{f}_{1}(1)+f_{1}^{(2)}(1,2)+\mu_{1}\tilde{\mu}_{2}\hat{\theta}_{1}^{(2)}\left(1\right)\hat{f}_{1}(1)\\
&+&2\hat{r}_{1}\frac{\mu_{1}\tilde{\mu}_{2}}{1-\hat{\mu}_{1}}\hat{f}_{1}(1)+
\frac{\mu_{1}}{1-\hat{\mu}_{1}}\hat{f}_{1}(1)F_{1,2}(1)+\frac{\tilde{\mu}_{2}}{1-\hat{\mu}_{1}}\hat{f}_{1}(1)F_{1,1}(1)+\mu_{1}\tilde{\mu}_{2}\left(\frac{1}{1-\hat{\mu}_{1}}\right)^{2}\hat{f}_{1}(1,1),\\
\hat{f}_{2}\left(3,1\right)&=&\hat{r}_{1}\mu_{1}\hat{\mu}_{1}+\mu_{1}\hat{\mu}_{1}\hat{R}_{1}^{(2)}\left(1\right)+\hat{r}_{1}\hat{\mu}_{1}F_{1,1}(1)+\hat{r}_{1}\frac{\mu_{1}\hat{\mu}_{1}}{1-\hat{\mu}_{1}}\hat{F}_{1}(1),\\
\hat{f}_{2}\left(4,1\right)&=&\hat{r}_{1}\mu_{1}\hat{\mu}_{2}+\mu_{1}\hat{\mu}_{2}\hat{R}_{1}^{(2)}\left(1\right)+\hat{r}_{1}\hat{\mu}_{2}F_{1,1}(1)+\frac{\mu_{1}\hat{\mu}_{2}}{1-\hat{\mu}_{1}}\hat{f}_{1}(1)+\hat{r}_{1}\frac{\mu_{1}\hat{\mu}_{2}}{1-\hat{\mu}_{1}}\hat{f}_{1}(1)+\mu_{1}\hat{\mu}_{2}\hat{\theta}_{1}^{(2)}\left(1\right)\hat{f}_{1}(1)\\
&+&\hat{r}_{1}\mu_{1}\left(\hat{f}_{1}(2)+\frac{\hat{\mu}_{2}}{1-\hat{\mu}_{1}}\hat{f}_{1}(1)\right)+F_{1,1}(1)\left(\hat{f}_{1}(2)+\frac{\hat{\mu}_{2}}{1-\hat{\mu}_{1}}\hat{f}_{1}(1)\right)+\frac{\mu_{1}}{1-\hat{\mu}_{1}}\left(\hat{f}_{1}(1,2)+\frac{\hat{\mu}_{2}}{1-\hat{\mu}_{1}}\hat{f}_{1}(1,1)\right),
\end{eqnarray*}



\begin{eqnarray*}
\hat{f}_{2}\left(1,2\right)&=&\hat{r}_{1}\mu_{1}\tilde{\mu}_{2}+\mu_{1}\tilde{\mu}_{2}\hat{R}_{1}^{(2)}\left(1\right)+\hat{r}_{1}\mu_{1}F_{1,2}(1)+\hat{r}_{1}\tilde{\mu}_{2}F_{1,1}(1)+\frac{\mu_{1}\tilde{\mu}_{2}}{1-\hat{\mu}_{1}}\hat{f}_{1}(1)+\frac{\tilde{\mu}_{2}}{1-\hat{\mu}_{1}}\hat{f}_{1}(1)F_{1,1}(1)\\
&+&2\hat{r}_{1}\frac{\mu_{1}\tilde{\mu}_{2}}{1-\hat{\mu}_{1}}\hat{f}_{1}(1)+\mu_{1}\tilde{\mu}_{2}\hat{\theta}_{1}^{(2)}\left(1\right)\hat{f}_{1}(1)+\frac{\mu_{1}}{1-\hat{\mu}_{1}}\hat{f}_{1}(1)F_{1,2}(1)+f_{1}^{(2)}(1,2)+\mu_{1}\tilde{\mu}_{2}\left(\frac{1}{1-\hat{\mu}_{1}}\right)^{2}\hat{f}_{1}(1,1),\\
\hat{f}_{2}\left(2,2\right)&=&\hat{r}_{1}\tilde{P}_{2}^{(2)}\left(1\right)+\tilde{\mu}_{2}^{2}\hat{R}_{1}^{(2)}\left(1\right)+2\hat{r}_{1}\tilde{\mu}_{2}F_{1,2}(1)+ f_{1,2}^{(2)}(1)+2\hat{r}_{1}\frac{\tilde{\mu}_{2}^{2}}{1-\hat{\mu}_{1}}\hat{f}_{1}(1)\\
&+&\frac{1}{1-\hat{\mu}_{1}}\tilde{P}_{2}^{(2)}\left(1\right)\hat{f}_{1}(1)+\tilde{\mu}_{2}^{2}\hat{\theta}_{1}^{(2)}\left(1\right)\hat{f}_{1}(1)+2\frac{\tilde{\mu}_{2}}{1-\hat{\mu}_{1}}F_{1,2}(1)\hat{f}_{1}(1)+\left(\frac{\tilde{\mu}_{2}}{1-\hat{\mu}_{1}}\right)^{2}\hat{f}_{1}(1,1),\\
\hat{f}_{2}\left(3,2\right)&=&\hat{r}_{1}\hat{\mu}_{1}\tilde{\mu}_{2}+\hat{\mu}_{1}\tilde{\mu}_{2}\hat{R}_{1}^{(2)}\left(1\right)+
\hat{r}_{1}\hat{\mu}_{1}F_{1,2}(1)+\hat{r}_{1}\frac{\hat{\mu}_{1}\tilde{\mu}_{2}}{1-\hat{\mu}_{1}}\hat{f}_{1}(1),\\
\hat{f}_{2}\left(4,2\right)&=&\hat{r}_{1}\tilde{\mu}_{2}\hat{\mu}_{2}+\hat{\mu}_{2}\tilde{\mu}_{2}\hat{R}_{1}^{(2)}\left(1\right)+\hat{\mu}_{2}\hat{R}_{1}^{(2)}\left(1\right)F_{1,2}(1)+\frac{\hat{\mu}_{2}\tilde{\mu}_{2}}{1-\hat{\mu}_{1}}\hat{f}_{1}(1)+F_{1,2}(1)\left(\hat{f}_{1}(2)+\frac{\hat{\mu}_{2}}{1-\hat{\mu}_{1}}\hat{f}_{1}(1)\right)\\
&+&\hat{r}_{1}\frac{\hat{\mu}_{2}\tilde{\mu}_{2}}{1-\hat{\mu}_{1}}\hat{f}_{1}(1)+\hat{\mu}_{2}\tilde{\mu}_{2}\hat{\theta}_{1}^{(2)}\left(1\right)\hat{f}_{1}(1)+\hat{r}_{1}\tilde{\mu}_{2}\left(\hat{f}_{1}(2)+\frac{\hat{\mu}_{2}}{1-\hat{\mu}_{1}}\hat{f}_{1}(1)\right)+\frac{\tilde{\mu}_{2}}{1-\hat{\mu}_{1}}\left(\hat{f}_{1}(1,2)+\frac{\hat{\mu}_{2}}{1-\hat{\mu}_{1}}\hat{f}_{1}(1,1)\right),
\end{eqnarray*}



\begin{eqnarray*}
\hat{f}_{2}\left(1,3\right)&=&\hat{r}_{1}\mu_{1}\hat{\mu}_{1}+\mu_{1}\hat{\mu}_{1}\hat{R}_{1}^{(2)}\left(1\right)+\hat{r}_{1}\hat{\mu}_{1}F_{1,1}(1)+\hat{r}_{1}\frac{\mu_{1}\hat{\mu}_{1}}{1-\hat{\mu}_{1}}\hat{f}_{1}(1),\\
\hat{f}_{2}\left(2,3\right)&=&\hat{r}_{1}\tilde{\mu}_{2}\hat{\mu}_{1}+\tilde{\mu}_{2}\hat{\mu}_{1}\hat{R}_{1}^{(2)}\left(1\right)+\hat{r}_{1}\hat{\mu}_{1}F_{1,2}(1)+\hat{r}_{1}\frac{\tilde{\mu}_{2}\hat{\mu}_{1}}{1-\hat{\mu}_{1}}\hat{f}_{1}(1),\\
\hat{f}_{2}\left(3,3\right)&=&\hat{r}_{1}\hat{P}_{1}^{(2)}\left(1\right)+\hat{\mu}_{1}^{2}\hat{R}_{1}^{(2)}\left(1\right),\\
\hat{f}_{2}\left(4,3\right)&=&\hat{r}_{1}\hat{\mu}_{2}\hat{\mu}_{1}+\hat{\mu}_{2}\hat{\mu}_{1}\hat{R}_{1}^{(2)}\left(1\right)+\hat{r}_{1}\hat{\mu}_{1}\left(\hat{f}_{1}(2)+\frac{\hat{\mu}_{2}}{1-\hat{\mu}_{1}}\hat{f}_{1}(1)\right),
\end{eqnarray*}



\begin{eqnarray*}
\hat{f}_{2}\left(1,4\right)&=&\hat{r}_{1}\mu_{1}\hat{\mu}_{2}+\mu_{1}\hat{\mu}_{2}\hat{R}_{1}^{(2)}\left(1\right)+\hat{r}_{1}\hat{\mu}_{2}F_{1,1}(1)+\hat{r}_{1}\frac{\mu_{1}\hat{\mu}_{2}}{1-\hat{\mu}_{1}}\hat{f}_{1}(1)+\hat{r}_{1}\mu_{1}\left(\hat{f}_{1}(2)+\frac{\hat{\mu}_{2}}{1-\hat{\mu}_{1}}\hat{f}_{1}(1)\right)+\frac{\mu_{1}}{1-\hat{\mu}_{1}}\hat{f}_{1}(1,2)\\
&+&F_{1,1}(1)\left(\hat{f}_{1}(2)+\frac{\hat{\mu}_{2}}{1-\hat{\mu}_{1}}\hat{f}_{1}(1)\right)+\frac{\mu_{1}\hat{\mu}_{2}}{1-\hat{\mu}_{1}}\hat{f}_{1}(1)+\mu_{1}\hat{\mu}_{2}\hat{\theta}_{1}^{(2)}\left(1\right)\hat{f}_{1}(1)+\mu_{1}\hat{\mu}_{2}\left(\frac{1}{1-\hat{\mu}_{1}}\right)^{2}\hat{f}_{1}(1,1),\\
\hat{f}_{2}\left(2,4\right)&=&\hat{r}_{1}\tilde{\mu}_{2}\hat{\mu}_{2}+\tilde{\mu}_{2}\hat{\mu}_{2}\hat{R}_{1}^{(2)}\left(1\right)+\hat{r}_{1}\hat{\mu}_{2}F_{1,2}(1)+\hat{r}_{1}\frac{\tilde{\mu}_{2}\hat{\mu}_{2}}{1-\hat{\mu}_{1}}\hat{f}_{1}(1)+\tilde{\mu}_{2}\hat{\mu}_{2}\hat{\theta}_{1}^{(2)}\left(1\right)\hat{f}_{1}(1)+\frac{\tilde{\mu}_{2}}{1-\hat{\mu}_{1}}\hat{f}_{1}(1,2)\\
&+&\hat{r}_{1}\tilde{\mu}_{2}\left(\hat{f}_{1}(2)+\frac{\hat{\mu}_{2}}{1-\hat{\mu}_{1}}\hat{f}_{1}(1)\right)+F_{1,2}(1)\left(\hat{f}_{1}(2)+\frac{\hat{\mu}_{2}}{1-\hat{\mu}_{1}}\hat{F}_{1}^{(1,0)}\right)+\frac{\tilde{\mu}_{2}\hat{\mu}_{2}}{1-\hat{\mu}_{1}}\hat{f}_{1}(1)+\tilde{\mu}_{2}\hat{\mu}_{2}\left(\frac{1}{1-\hat{\mu}_{1}}\right)^{2}\hat{f}_{1}(1,1),\\
\hat{f}_{2}\left(3,4\right)&=&\hat{r}_{1}\hat{\mu}_{2}\hat{\mu}_{1}+\hat{\mu}_{2}\hat{\mu}_{1}\hat{R}_{1}^{(2)}\left(1\right)+\hat{r}_{1}\hat{\mu}_{1}\left(\hat{f}_{1}(2)+\frac{\hat{\mu}_{2}}{1-\hat{\mu}_{1}}\hat{f}_{1}(1)\right),\\
\hat{f}_{2}\left(4,4\right)&=&\hat{r}_{1}\hat{P}_{2}^{(2)}\left(1\right)+\hat{\mu}_{2}^{2}\hat{R}_{1}^{(2)}\left(1\right)+
2\hat{r}_{1}\hat{\mu}_{2}\left(\hat{f}_{1}(2)+\frac{\hat{\mu}_{2}}{1-\hat{\mu}_{1}}\hat{f}_{1}(1)\right)+\hat{f}_{1}(2,2)\\
&+&\frac{1}{1-\hat{\mu}_{1}}\hat{P}_{2}^{(2)}\left(1\right)\hat{f}_{1}(1)+\hat{\mu}_{2}^{2}\hat{\theta}_{1}^{(2)}\left(1\right)\hat{f}_{1}(1)+\frac{\hat{\mu}_{2}}{1-\hat{\mu}_{1}}\hat{f}_{1}(1,2)+\frac{\hat{\mu}_{2}}{1-\hat{\mu}_{1}}\left(\hat{f}_{1}(1,2)+\frac{\hat{\mu}_{2}}{1-\hat{\mu}_{1}}\hat{f}_{1}(1,1)\right).
\end{eqnarray*}
%_________________________________________________________________________________________________________
\section{Medidas de Desempe\~no}
%_________________________________________________________________________________________________________

\begin{Def}
Sea $L_{i}^{*}$el n\'umero de usuarios cuando el servidor visita la cola $Q_{i}$ para dar servicio, para $i=1,2$.
\end{Def}

Entonces
\begin{Prop} Para la cola $Q_{i}$, $i=1,2$, se tiene que el n\'umero de usuarios presentes al momento de ser visitada por el servidor est\'a dado por
\begin{eqnarray}
\esp\left[L_{i}^{*}\right]&=&f_{i}\left(i\right)\\
Var\left[L_{i}^{*}\right]&=&f_{i}\left(i,i\right)+\esp\left[L_{i}^{*}\right]-\esp\left[L_{i}^{*}\right]^{2}.
\end{eqnarray}
\end{Prop}


\begin{Def}
El tiempo de Ciclo $C_{i}$ es el periodo de tiempo que comienza
cuando la cola $i$ es visitada por primera vez en un ciclo, y
termina cuando es visitado nuevamente en el pr\'oximo ciclo, bajo condiciones de estabilidad.

\begin{eqnarray*}
C_{i}\left(z\right)=\esp\left[z^{\overline{\tau}_{i}\left(m+1\right)-\overline{\tau}_{i}\left(m\right)}\right]
\end{eqnarray*}
\end{Def}

\begin{Def}
El tiempo de intervisita $I_{i}$ es el periodo de tiempo que
comienza cuando se ha completado el servicio en un ciclo y termina
cuando es visitada nuevamente en el pr\'oximo ciclo.
\begin{eqnarray*}I_{i}\left(z\right)&=&\esp\left[z^{\tau_{i}\left(m+1\right)-\overline{\tau}_{i}\left(m\right)}\right]\end{eqnarray*}
\end{Def}

\begin{Prop}
Para los tiempos de intervisita del servidor $I_{i}$, se tiene que

\begin{eqnarray*}
\esp\left[I_{i}\right]&=&\frac{f_{i}\left(i\right)}{\mu_{i}},\\
Var\left[I_{i}\right]&=&\frac{Var\left[L_{i}^{*}\right]}{\mu_{i}^{2}}-\frac{\sigma_{i}^{2}}{\mu_{i}^{2}}f_{i}\left(i\right).
\end{eqnarray*}
\end{Prop}


\begin{Prop}
Para los tiempos que ocupa el servidor para atender a los usuarios presentes en la cola $Q_{i}$, con FGP denotada por $S_{i}$, se tiene que
\begin{eqnarray*}
\esp\left[S_{i}\right]&=&\frac{\esp\left[L_{i}^{*}\right]}{1-\mu_{i}}=\frac{f_{i}\left(i\right)}{1-\mu_{i}},\\
Var\left[S_{i}\right]&=&\frac{Var\left[L_{i}^{*}\right]}{\left(1-\mu_{i}\right)^{2}}+\frac{\sigma^{2}\esp\left[L_{i}^{*}\right]}{\left(1-\mu_{i}\right)^{3}}
\end{eqnarray*}
\end{Prop}


\begin{Prop}
Para la duraci\'on de los ciclos $C_{i}$ se tiene que
\begin{eqnarray*}
\esp\left[C_{i}\right]&=&\esp\left[I_{i}\right]\esp\left[\theta_{i}\left(z\right)\right]=\frac{\esp\left[L_{i}^{*}\right]}{\mu_{i}}\frac{1}{1-\mu_{i}}=\frac{f_{i}\left(i\right)}{\mu_{i}\left(1-\mu_{i}\right)}\\
Var\left[C_{i}\right]&=&\frac{Var\left[L_{i}^{*}\right]}{\mu_{i}^{2}\left(1-\mu_{i}\right)^{2}}.
\end{eqnarray*}

\end{Prop}
%___________________________________________________________________________________________

%___________________________________________________________________________________________
%
\section*{Ap\'endice A}\label{Segundos.Momentos}
%___________________________________________________________________________________________


%___________________________________________________________________________________________

%\subsubsection{Mixtas para $z_{1}$:}
%___________________________________________________________________________________________
\begin{enumerate}

%1/1/1
\item \begin{eqnarray*}
&&\frac{\partial}{\partial z_1}\frac{\partial}{\partial z_1}\left(R_2\left(P_1\left(z_1\right)\bar{P}_2\left(z_2\right)\hat{P}_1\left(w_1\right)\hat{P}_2\left(w_2\right)\right)F_2\left(z_1,\theta
_2\left(P_1\left(z_1\right)\hat{P}_1\left(w_1\right)\hat{P}_2\left(w_2\right)\right)\right)\hat{F}_2\left(w_1,w_2\right)\right)\\
&=&r_{2}P_{1}^{(2)}\left(1\right)+\mu_{1}^{2}R_{2}^{(2)}\left(1\right)+2\mu_{1}r_{2}\left(\frac{\mu_{1}}{1-\tilde{\mu}_{2}}F_{2}^{(0,1)}+F_{2}^{1,0)}\right)+\frac{1}{1-\tilde{\mu}_{2}}P_{1}^{(2)}F_{2}^{(0,1)}+\mu_{1}^{2}\tilde{\theta}_{2}^{(2)}\left(1\right)F_{2}^{(0,1)}\\
&+&\frac{\mu_{1}}{1-\tilde{\mu}_{2}}F_{2}^{(1,1)}+\frac{\mu_{1}}{1-\tilde{\mu}_{2}}\left(\frac{\mu_{1}}{1-\tilde{\mu}_{2}}F_{2}^{(0,2)}+F_{2}^{(1,1)}\right)+F_{2}^{(2,0)}.
\end{eqnarray*}

%2/2/1

\item \begin{eqnarray*}
&&\frac{\partial}{\partial z_2}\frac{\partial}{\partial z_1}\left(R_2\left(P_1\left(z_1\right)\bar{P}_2\left(z_2\right)\hat{P}_1\left(w_1\right)\hat{P}_2\left(w_2\right)\right)F_2\left(z_1,\theta
_2\left(P_1\left(z_1\right)\hat{P}_1\left(w_1\right)\hat{P}_2\left(w_2\right)\right)\right)\hat{F}_2\left(w_1,w_2\right)\right)\\
&=&\mu_{1}r_{2}\tilde{\mu}_{2}+\mu_{1}\tilde{\mu}_{2}R_{2}^{(2)}\left(1\right)+r_{2}\tilde{\mu}_{2}\left(\frac{\mu_{1}}{1-\tilde{\mu}_{2}}F_{2}^{(0,1)}+F_{2}^{(1,0)}\right).
\end{eqnarray*}
%3/3/1
\item \begin{eqnarray*}
&&\frac{\partial}{\partial w_1}\frac{\partial}{\partial z_1}\left(R_2\left(P_1\left(z_1\right)\bar{P}_2\left(z_2\right)\hat{P}_1\left(w_1\right)\hat{P}_2\left(w_2\right)\right)F_2\left(z_1,\theta
_2\left(P_1\left(z_1\right)\hat{P}_1\left(w_1\right)\hat{P}_2\left(w_2\right)\right)\right)\hat{F}_2\left(w_1,w_2\right)\right)\\
&=&\mu_{1}\hat{\mu}_{1}r_{2}+\mu_{1}\hat{\mu}_{1}R_{2}^{(2)}\left(1\right)+r_{2}\frac{\mu_{1}}{1-\tilde{\mu}_{2}}F_{2}^{(0,1)}+r_{2}\hat{\mu}_{1}\left(\frac{\mu_{1}}{1-\tilde{\mu}_{2}}F_{2}^{(0,1)}+F_{2}^{(1,0)}\right)+\mu_{1}r_{2}\hat{F}_{2}^{(1,0)}\\
&+&\left(\frac{\mu_{1}}{1-\tilde{\mu}_{2}}F_{2}^{(0,1)}+F_{2}^{(1,0)}\right)\hat{F}_{2}^{(1,0)}+\frac{\mu_{1}\hat{\mu}_{1}}{1-\tilde{\mu}_{2}}F_{2}^{(0,1)}+\mu_{1}\hat{\mu}_{1}\tilde{\theta}_{2}^{(2)}\left(1\right)F_{2}^{(0,1)}\\
&+&\mu_{1}\hat{\mu}_{1}\left(\frac{1}{1-\tilde{\mu}_{2}}\right)^{2}F_{2}^{(0,2)}+\frac{\hat{\mu}_{1}}{1-\tilde{\mu}_{2}}F_{2}^{(1,1)}.
\end{eqnarray*}
%4/4/1
\item \begin{eqnarray*}
&&\frac{\partial}{\partial w_2}\frac{\partial}{\partial z_1}\left(R_2\left(P_1\left(z_1\right)\bar{P}_2\left(z_2\right)\hat{P}_1\left(w_1\right)\hat{P}_2\left(w_2\right)\right)
F_2\left(z_1,\theta_2\left(P_1\left(z_1\right)\hat{P}_1\left(w_1\right)\hat{P}_2\left(w_2\right)\right)\right)\hat{F}_2\left(w_1,w_2\right)\right)\\
&=&\mu_{1}\hat{\mu}_{2}r_{2}+\mu_{1}\hat{\mu}_{2}R_{2}^{(2)}\left(1\right)+r_{2}\frac{\mu_{1}\hat{\mu}_{2}}{1-\tilde{\mu}_{2}}F_{2}^{(0,1)}+\mu_{1}r_{2}\hat{F}_{2}^{(0,1)}
+r_{2}\hat{\mu}_{2}\left(\frac{\mu_{1}}{1-\tilde{\mu}_{2}}F_{2}^{(0,1)}+F_{2}^{(1,0)}\right)\\
&+&\hat{F}_{2}^{(1,0)}\left(\frac{\mu_{1}}{1-\tilde{\mu}_{2}}F_{2}^{(0,1)}+F_{2}^{(1,0)}\right)+\frac{\mu_{1}\hat{\mu}_{2}}{1-\tilde{\mu}_{2}}F_{2}^{(0,1)}
+\mu_{1}\hat{\mu}_{2}\tilde{\theta}_{2}^{(2)}\left(1\right)F_{2}^{(0,1)}+\mu_{1}\hat{\mu}_{2}\left(\frac{1}{1-\tilde{\mu}_{2}}\right)^{2}F_{2}^{(0,2)}\\
&+&\frac{\hat{\mu}_{2}}{1-\tilde{\mu}_{2}}F_{2}^{(1,1)}.
\end{eqnarray*}
%___________________________________________________________________________________________
%\subsubsection{Mixtas para $z_{2}$:}
%___________________________________________________________________________________________
%5
\item \begin{eqnarray*} &&\frac{\partial}{\partial
z_1}\frac{\partial}{\partial
z_2}\left(R_2\left(P_1\left(z_1\right)\bar{P}_2\left(z_2\right)\hat{P}_1\left(w_1\right)\hat{P}_2\left(w_2\right)\right)
F_2\left(z_1,\theta_2\left(P_1\left(z_1\right)\hat{P}_1\left(w_1\right)\hat{P}_2\left(w_2\right)\right)\right)\hat{F}_2\left(w_1,w_2\right)\right)\\
&=&\mu_{1}\tilde{\mu}_{2}r_{2}+\mu_{1}\tilde{\mu}_{2}R_{2}^{(2)}\left(1\right)+r_{2}\tilde{\mu}_{2}\left(\frac{\mu_{1}}{1-\tilde{\mu}_{2}}F_{2}^{(0,1)}+F_{2}^{(1,0)}\right).
\end{eqnarray*}

%6

\item \begin{eqnarray*} &&\frac{\partial}{\partial
z_2}\frac{\partial}{\partial
z_2}\left(R_2\left(P_1\left(z_1\right)\bar{P}_2\left(z_2\right)\hat{P}_1\left(w_1\right)\hat{P}_2\left(w_2\right)\right)
F_2\left(z_1,\theta_2\left(P_1\left(z_1\right)\hat{P}_1\left(w_1\right)\hat{P}_2\left(w_2\right)\right)\right)\hat{F}_2\left(w_1,w_2\right)\right)\\
&=&\tilde{\mu}_{2}^{2}R_{2}^{(2)}(1)+r_{2}\tilde{P}_{2}^{(2)}\left(1\right).
\end{eqnarray*}

%7
\item \begin{eqnarray*} &&\frac{\partial}{\partial
w_1}\frac{\partial}{\partial
z_2}\left(R_2\left(P_1\left(z_1\right)\bar{P}_2\left(z_2\right)\hat{P}_1\left(w_1\right)\hat{P}_2\left(w_2\right)\right)
F_2\left(z_1,\theta_2\left(P_1\left(z_1\right)\hat{P}_1\left(w_1\right)\hat{P}_2\left(w_2\right)\right)\right)\hat{F}_2\left(w_1,w_2\right)\right)\\
&=&\hat{\mu}_{1}\tilde{\mu}_{2}r_{2}+\hat{\mu}_{1}\tilde{\mu}_{2}R_{2}^{(2)}(1)+
r_{2}\frac{\hat{\mu}_{1}\tilde{\mu}_{2}}{1-\tilde{\mu}_{2}}F_{2}^{(0,1)}+r_{2}\tilde{\mu}_{2}\hat{F}_{2}^{(1,0)}.
\end{eqnarray*}
%8
\item \begin{eqnarray*} &&\frac{\partial}{\partial
w_2}\frac{\partial}{\partial
z_2}\left(R_2\left(P_1\left(z_1\right)\bar{P}_2\left(z_2\right)\hat{P}_1\left(w_1\right)\hat{P}_2\left(w_2\right)\right)
F_2\left(z_1,\theta_2\left(P_1\left(z_1\right)\hat{P}_1\left(w_1\right)\hat{P}_2\left(w_2\right)\right)\right)\hat{F}_2\left(w_1,w_2\right)\right)\\
&=&\hat{\mu}_{2}\tilde{\mu}_{2}r_{2}+\hat{\mu}_{2}\tilde{\mu}_{2}R_{2}^{(2)}(1)+
r_{2}\frac{\hat{\mu}_{2}\tilde{\mu}_{2}}{1-\tilde{\mu}_{2}}F_{2}^{(0,1)}+r_{2}\tilde{\mu}_{2}\hat{F}_{2}^{(0,1)}.
\end{eqnarray*}
%___________________________________________________________________________________________
%\subsubsection{Mixtas para $w_{1}$:}
%___________________________________________________________________________________________

%9
\item \begin{eqnarray*} &&\frac{\partial}{\partial
z_1}\frac{\partial}{\partial
w_1}\left(R_2\left(P_1\left(z_1\right)\bar{P}_2\left(z_2\right)\hat{P}_1\left(w_1\right)\hat{P}_2\left(w_2\right)\right)
F_2\left(z_1,\theta_2\left(P_1\left(z_1\right)\hat{P}_1\left(w_1\right)\hat{P}_2\left(w_2\right)\right)\right)\hat{F}_2\left(w_1,w_2\right)\right)\\
&=&\mu_{1}\hat{\mu}_{1}r_{2}+\mu_{1}\hat{\mu}_{1}R_{2}^{(2)}\left(1\right)+\frac{\mu_{1}\hat{\mu}_{1}}{1-\tilde{\mu}_{2}}F_{2}^{(0,1)}+r_{2}\frac{\mu_{1}\hat{\mu}_{1}}{1-\tilde{\mu}_{2}}F_{2}^{(0,1)}+\mu_{1}\hat{\mu}_{1}\tilde{\theta}_{2}^{(2)}\left(1\right)F_{2}^{(0,1)}\\
&+&r_{2}\hat{\mu}_{1}\left(\frac{\mu_{1}}{1-\tilde{\mu}_{2}}F_{2}^{(0,1)}+F_{2}^{(1,0)}\right)+r_{2}\mu_{1}\hat{F}_{2}^{(1,0)}
+\left(\frac{\mu_{1}}{1-\tilde{\mu}_{2}}F_{2}^{(0,1)}+F_{2}^{(1,0)}\right)\hat{F}_{2}^{(1,0)}\\
&+&\frac{\hat{\mu}_{1}}{1-\tilde{\mu}_{2}}\left(\frac{\mu_{1}}{1-\tilde{\mu}_{2}}F_{2}^{(0,2)}+F_{2}^{(1,1)}\right).
\end{eqnarray*}
%10
\item \begin{eqnarray*} &&\frac{\partial}{\partial
z_2}\frac{\partial}{\partial
w_1}\left(R_2\left(P_1\left(z_1\right)\bar{P}_2\left(z_2\right)\hat{P}_1\left(w_1\right)\hat{P}_2\left(w_2\right)\right)
F_2\left(z_1,\theta_2\left(P_1\left(z_1\right)\hat{P}_1\left(w_1\right)\hat{P}_2\left(w_2\right)\right)\right)\hat{F}_2\left(w_1,w_2\right)\right)\\
&=&\tilde{\mu}_{2}\hat{\mu}_{1}r_{2}+\tilde{\mu}_{2}\hat{\mu}_{1}R_{2}^{(2)}\left(1\right)+r_{2}\frac{\tilde{\mu}_{2}\hat{\mu}_{1}}{1-\tilde{\mu}_{2}}F_{2}^{(0,1)}
+r_{2}\tilde{\mu}_{2}\hat{F}_{2}^{(1,0)}.
\end{eqnarray*}
%11
\item \begin{eqnarray*} &&\frac{\partial}{\partial
w_1}\frac{\partial}{\partial
w_1}\left(R_2\left(P_1\left(z_1\right)\bar{P}_2\left(z_2\right)\hat{P}_1\left(w_1\right)\hat{P}_2\left(w_2\right)\right)
F_2\left(z_1,\theta_2\left(P_1\left(z_1\right)\hat{P}_1\left(w_1\right)\hat{P}_2\left(w_2\right)\right)\right)\hat{F}_2\left(w_1,w_2\right)\right)\\
&=&\hat{\mu}_{1}^{2}R_{2}^{(2)}\left(1\right)+r_{2}\hat{P}_{1}^{(2)}\left(1\right)+2r_{2}\frac{\hat{\mu}_{1}^{2}}{1-\tilde{\mu}_{2}}F_{2}^{(0,1)}+
\hat{\mu}_{1}^{2}\tilde{\theta}_{2}^{(2)}\left(1\right)F_{2}^{(0,1)}+\frac{1}{1-\tilde{\mu}_{2}}\hat{P}_{1}^{(2)}\left(1\right)F_{2}^{(0,1)}\\
&+&\frac{\hat{\mu}_{1}^{2}}{1-\tilde{\mu}_{2}}F_{2}^{(0,2)}+2r_{2}\hat{\mu}_{1}\hat{F}_{2}^{(1,0)}+2\frac{\hat{\mu}_{1}}{1-\tilde{\mu}_{2}}F_{2}^{(0,1)}\hat{F}_{2}^{(1,0)}+\hat{F}_{2}^{(2,0)}.
\end{eqnarray*}
%12
\item \begin{eqnarray*} &&\frac{\partial}{\partial
w_2}\frac{\partial}{\partial
w_1}\left(R_2\left(P_1\left(z_1\right)\bar{P}_2\left(z_2\right)\hat{P}_1\left(w_1\right)\hat{P}_2\left(w_2\right)\right)
F_2\left(z_1,\theta_2\left(P_1\left(z_1\right)\hat{P}_1\left(w_1\right)\hat{P}_2\left(w_2\right)\right)\right)\hat{F}_2\left(w_1,w_2\right)\right)\\
&=&r_{2}\hat{\mu}_{2}\hat{\mu}_{1}+\hat{\mu}_{1}\hat{\mu}_{2}R_{2}^{(2)}(1)+\frac{\hat{\mu}_{1}\hat{\mu}_{2}}{1-\tilde{\mu}_{2}}F_{2}^{(0,1)}
+2r_{2}\frac{\hat{\mu}_{1}\hat{\mu}_{2}}{1-\tilde{\mu}_{2}}F_{2}^{(0,1)}+\hat{\mu}_{2}\hat{\mu}_{1}\tilde{\theta}_{2}^{(2)}\left(1\right)F_{2}^{(0,1)}+
r_{2}\hat{\mu}_{1}\hat{F}_{2}^{(0,1)}\\
&+&\frac{\hat{\mu}_{1}}{1-\tilde{\mu}_{2}}F_{2}^{(0,1)}\hat{F}_{2}^{(0,1)}+\hat{\mu}_{1}\hat{\mu}_{2}\left(\frac{1}{1-\tilde{\mu}_{2}}\right)^{2}F_{2}^{(0,2)}+
r_{2}\hat{\mu}_{2}\hat{F}_{2}^{(1,0)}+\frac{\hat{\mu}_{2}}{1-\tilde{\mu}_{2}}F_{2}^{(0,1)}\hat{F}_{2}^{(1,0)}+\hat{F}_{2}^{(1,1)}.
\end{eqnarray*}
%___________________________________________________________________________________________
%\subsubsection{Mixtas para $w_{2}$:}
%___________________________________________________________________________________________
%13

\item \begin{eqnarray*} &&\frac{\partial}{\partial
z_1}\frac{\partial}{\partial
w_2}\left(R_2\left(P_1\left(z_1\right)\bar{P}_2\left(z_2\right)\hat{P}_1\left(w_1\right)\hat{P}_2\left(w_2\right)\right)
F_2\left(z_1,\theta_2\left(P_1\left(z_1\right)\hat{P}_1\left(w_1\right)\hat{P}_2\left(w_2\right)\right)\right)\hat{F}_2\left(w_1,w_2\right)\right)\\
&=&r_{2}\mu_{1}\hat{\mu}_{2}+\mu_{1}\hat{\mu}_{2}R_{2}^{(2)}(1)+\frac{\mu_{1}\hat{\mu}_{2}}{1-\tilde{\mu}_{2}}F_{2}^{(0,1)}+r_{2}\frac{\mu_{1}\hat{\mu}_{2}}{1-\tilde{\mu}_{2}}F_{2}^{(0,1)}+\mu_{1}\hat{\mu}_{2}\tilde{\theta}_{2}^{(2)}\left(1\right)F_{2}^{(0,1)}+r_{2}\mu_{1}\hat{F}_{2}^{(0,1)}\\
&+&r_{2}\hat{\mu}_{2}\left(\frac{\mu_{1}}{1-\tilde{\mu}_{2}}F_{2}^{(0,1)}+F_{2}^{(1,0)}\right)+\hat{F}_{2}^{(0,1)}\left(\frac{\mu_{1}}{1-\tilde{\mu}_{2}}F_{2}^{(0,1)}+F_{2}^{(1,0)}\right)+\frac{\hat{\mu}_{2}}{1-\tilde{\mu}_{2}}\left(\frac{\mu_{1}}{1-\tilde{\mu}_{2}}F_{2}^{(0,2)}+F_{2}^{(1,1)}\right).
\end{eqnarray*}
%14
\item \begin{eqnarray*} &&\frac{\partial}{\partial
z_2}\frac{\partial}{\partial
w_2}\left(R_2\left(P_1\left(z_1\right)\bar{P}_2\left(z_2\right)\hat{P}_1\left(w_1\right)\hat{P}_2\left(w_2\right)\right)
F_2\left(z_1,\theta_2\left(P_1\left(z_1\right)\hat{P}_1\left(w_1\right)\hat{P}_2\left(w_2\right)\right)\right)\hat{F}_2\left(w_1,w_2\right)\right)\\
&=&r_{2}\tilde{\mu}_{2}\hat{\mu}_{2}+\tilde{\mu}_{2}\hat{\mu}_{2}R_{2}^{(2)}(1)+r_{2}\frac{\tilde{\mu}_{2}\hat{\mu}_{2}}{1-\tilde{\mu}_{2}}F_{2}^{(0,1)}+r_{2}\tilde{\mu}_{2}\hat{F}_{2}^{(0,1)}.
\end{eqnarray*}
%15
\item \begin{eqnarray*} &&\frac{\partial}{\partial
w_1}\frac{\partial}{\partial
w_2}\left(R_2\left(P_1\left(z_1\right)\bar{P}_2\left(z_2\right)\hat{P}_1\left(w_1\right)\hat{P}_2\left(w_2\right)\right)
F_2\left(z_1,\theta_2\left(P_1\left(z_1\right)\hat{P}_1\left(w_1\right)\hat{P}_2\left(w_2\right)\right)\right)\hat{F}_2\left(w_1,w_2\right)\right)\\
&=&r_{2}\hat{\mu}_{1}\hat{\mu}_{2}+\hat{\mu}_{1}\hat{\mu}_{2}R_{2}^{(2)}\left(1\right)+\frac{\hat{\mu}_{1}\hat{\mu}_{2}}{1-\tilde{\mu}_{2}}F_{2}^{(0,1)}+2r_{2}\frac{\hat{\mu}_{1}\hat{\mu}_{2}}{1-\tilde{\mu}_{2}}F_{2}^{(0,1)}+\hat{\mu}_{1}\hat{\mu}_{2}\theta_{2}^{(2)}\left(1\right)F_{2}^{(0,1)}+r_{2}\hat{\mu}_{1}\hat{F}_{2}^{(0,1)}\\
&+&\frac{\hat{\mu}_{1}}{1-\tilde{\mu}_{2}}F_{2}^{(0,1)}\hat{F}_{2}^{(0,1)}+\hat{\mu}_{1}\hat{\mu}_{2}\left(\frac{1}{1-\tilde{\mu}_{2}}\right)^{2}F_{2}^{(0,2)}+r_{2}\hat{\mu}_{2}\hat{F}_{2}^{(0,1)}+\frac{\hat{\mu}_{2}}{1-\tilde{\mu}_{2}}F_{2}^{(0,1)}\hat{F}_{2}^{(1,0)}+\hat{F}_{2}^{(1,1)}.
\end{eqnarray*}
%16

\item \begin{eqnarray*} &&\frac{\partial}{\partial
w_2}\frac{\partial}{\partial
w_2}\left(R_2\left(P_1\left(z_1\right)\bar{P}_2\left(z_2\right)\hat{P}_1\left(w_1\right)\hat{P}_2\left(w_2\right)\right)
F_2\left(z_1,\theta_2\left(P_1\left(z_1\right)\hat{P}_1\left(w_1\right)\hat{P}_2\left(w_2\right)\right)\right)\hat{F}_2\left(w_1,w_2\right)\right)\\
&=&\hat{\mu}_{2}^{2}R_{2}^{(2)}(1)+r_{2}\hat{P}_{2}^{(2)}\left(1\right)+2r_{2}\frac{\hat{\mu}_{2}^{2}}{1-\tilde{\mu}_{2}}F_{2}^{(0,1)}+\hat{\mu}_{2}^{2}\tilde{\theta}_{2}^{(2)}\left(1\right)F_{2}^{(0,1)}+\frac{1}{1-\tilde{\mu}_{2}}\hat{P}_{2}^{(2)}\left(1\right)F_{2}^{(0,1)}\\
&+&2r_{2}\hat{\mu}_{2}\hat{F}_{2}^{(0,1)}+2\frac{\hat{\mu}_{2}}{1-\tilde{\mu}_{2}}F_{2}^{(0,1)}\hat{F}_{2}^{(0,1)}+\left(\frac{\hat{\mu}_{2}}{1-\tilde{\mu}_{2}}\right)^{2}F_{2}^{(0,2)}+\hat{F}_{2}^{(0,2)}.
\end{eqnarray*}
\end{enumerate}
%___________________________________________________________________________________________
%
%\subsection{Derivadas de Segundo Orden para $F_{2}$}
%___________________________________________________________________________________________


\begin{enumerate}

%___________________________________________________________________________________________
%\subsubsection{Mixtas para $z_{1}$:}
%___________________________________________________________________________________________

%1/17
\item \begin{eqnarray*} &&\frac{\partial}{\partial
z_1}\frac{\partial}{\partial
z_1}\left(R_1\left(P_1\left(z_1\right)\bar{P}_2\left(z_2\right)\hat{P}_1\left(w_1\right)\hat{P}_2\left(w_2\right)\right)
F_1\left(\theta_1\left(\tilde{P}_2\left(z_1\right)\hat{P}_1\left(w_1\right)\hat{P}_2\left(w_2\right)\right)\right)\hat{F}_1\left(w_1,w_2\right)\right)\\
&=&r_{1}P_{1}^{(2)}\left(1\right)+\mu_{1}^{2}R_{1}^{(2)}\left(1\right).
\end{eqnarray*}

%2/18
\item \begin{eqnarray*} &&\frac{\partial}{\partial
z_2}\frac{\partial}{\partial
z_1}\left(R_1\left(P_1\left(z_1\right)\bar{P}_2\left(z_2\right)\hat{P}_1\left(w_1\right)\hat{P}_2\left(w_2\right)\right)F_1\left(\theta_1\left(\tilde{P}_2\left(z_1\right)\hat{P}_1\left(w_1\right)\hat{P}_2\left(w_2\right)\right)\right)\hat{F}_1\left(w_1,w_2\right)\right)\\
&=&\mu_{1}\tilde{\mu}_{2}r_{1}+\mu_{1}\tilde{\mu}_{2}R_{1}^{(2)}(1)+
r_{1}\mu_{1}\left(\frac{\tilde{\mu}_{2}}{1-\mu_{1}}F_{1}^{(1,0)}+F_{1}^{(0,1)}\right).
\end{eqnarray*}

%3/19
\item \begin{eqnarray*} &&\frac{\partial}{\partial
w_1}\frac{\partial}{\partial
z_1}\left(R_1\left(P_1\left(z_1\right)\bar{P}_2\left(z_2\right)\hat{P}_1\left(w_1\right)\hat{P}_2\left(w_2\right)\right)F_1\left(\theta_1\left(\tilde{P}_2\left(z_1\right)\hat{P}_1\left(w_1\right)\hat{P}_2\left(w_2\right)\right)\right)\hat{F}_1\left(w_1,w_2\right)\right)\\
&=&r_{1}\mu_{1}\hat{\mu}_{1}+\mu_{1}\hat{\mu}_{1}R_{1}^{(2)}\left(1\right)+r_{1}\frac{\mu_{1}\hat{\mu}_{1}}{1-\mu_{1}}F_{1}^{(1,0)}+r_{1}\mu_{1}\hat{F}_{1}^{(1,0)}.
\end{eqnarray*}
%4/20
\item \begin{eqnarray*} &&\frac{\partial}{\partial
w_2}\frac{\partial}{\partial
z_1}\left(R_1\left(P_1\left(z_1\right)\bar{P}_2\left(z_2\right)\hat{P}_1\left(w_1\right)\hat{P}_2\left(w_2\right)\right)F_1\left(\theta_1\left(\tilde{P}_2\left(z_1\right)\hat{P}_1\left(w_1\right)\hat{P}_2\left(w_2\right)\right)\right)\hat{F}_1\left(w_1,w_2\right)\right)\\
&=&\mu_{1}\hat{\mu}_{2}r_{1}+\mu_{1}\hat{\mu}_{2}R_{1}^{(2)}\left(1\right)+r_{1}\mu_{1}\hat{F}_{1}^{(0,1)}+r_{1}\frac{\mu_{1}\hat{\mu}_{2}}{1-\mu_{1}}F_{1}^{(1,0)}.
\end{eqnarray*}
%___________________________________________________________________________________________
%\subsubsection{Mixtas para $z_{2}$:}
%___________________________________________________________________________________________
%5/21
\item \begin{eqnarray*}
&&\frac{\partial}{\partial z_1}\frac{\partial}{\partial z_2}\left(R_1\left(P_1\left(z_1\right)\bar{P}_2\left(z_2\right)\hat{P}_1\left(w_1\right)\hat{P}_2\left(w_2\right)\right)F_1\left(\theta_1\left(\tilde{P}_2\left(z_1\right)\hat{P}_1\left(w_1\right)\hat{P}_2\left(w_2\right)\right)\right)\hat{F}_1\left(w_1,w_2\right)\right)\\
&=&r_{1}\mu_{1}\tilde{\mu}_{2}+\mu_{1}\tilde{\mu}_{2}R_{1}^{(2)}\left(1\right)+r_{1}\mu_{1}\left(\frac{\tilde{\mu}_{2}}{1-\mu_{1}}F_{1}^{(1,0)}+F_{1}^{(0,1)}\right).
\end{eqnarray*}

%6/22
\item \begin{eqnarray*}
&&\frac{\partial}{\partial z_2}\frac{\partial}{\partial z_2}\left(R_1\left(P_1\left(z_1\right)\bar{P}_2\left(z_2\right)\hat{P}_1\left(w_1\right)\hat{P}_2\left(w_2\right)\right)F_1\left(\theta_1\left(\tilde{P}_2\left(z_1\right)\hat{P}_1\left(w_1\right)\hat{P}_2\left(w_2\right)\right)\right)\hat{F}_1\left(w_1,w_2\right)\right)\\
&=&\tilde{\mu}_{2}^{2}R_{1}^{(2)}\left(1\right)+r_{1}\tilde{P}_{2}^{(2)}\left(1\right)+2r_{1}\tilde{\mu}_{2}\left(\frac{\tilde{\mu}_{2}}{1-\mu_{1}}F_{1}^{(1,0)}+F_{1}^{(0,1)}\right)+F_{1}^{(0,2)}+\tilde{\mu}_{2}^{2}\theta_{1}^{(2)}\left(1\right)F_{1}^{(1,0)}\\
&+&\frac{1}{1-\mu_{1}}\tilde{P}_{2}^{(2)}\left(1\right)F_{1}^{(1,0)}+\frac{\tilde{\mu}_{2}}{1-\mu_{1}}F_{1}^{(1,1)}+\frac{\tilde{\mu}_{2}}{1-\mu_{1}}\left(\frac{\tilde{\mu}_{2}}{1-\mu_{1}}F_{1}^{(2,0)}+F_{1}^{(1,1)}\right).
\end{eqnarray*}
%7/23
\item \begin{eqnarray*}
&&\frac{\partial}{\partial w_1}\frac{\partial}{\partial z_2}\left(R_1\left(P_1\left(z_1\right)\bar{P}_2\left(z_2\right)\hat{P}_1\left(w_1\right)\hat{P}_2\left(w_2\right)\right)F_1\left(\theta_1\left(\tilde{P}_2\left(z_1\right)\hat{P}_1\left(w_1\right)\hat{P}_2\left(w_2\right)\right)\right)\hat{F}_1\left(w_1,w_2\right)\right)\\
&=&\tilde{\mu}_{2}\hat{\mu}_{1}r_{1}+\tilde{\mu}_{2}\hat{\mu}_{1}R_{1}^{(2)}\left(1\right)+r_{1}\frac{\tilde{\mu}_{2}\hat{\mu}_{1}}{1-\mu_{1}}F_{1}^{(1,0)}+\hat{\mu}_{1}r_{1}\left(\frac{\tilde{\mu}_{2}}{1-\mu_{1}}F_{1}^{(1,0)}+F_{1}^{(0,1)}\right)+r_{1}\tilde{\mu}_{2}\hat{F}_{1}^{(1,0)}\\
&+&\left(\frac{\tilde{\mu}_{2}}{1-\mu_{1}}F_{1}^{(1,0)}+F_{1}^{(0,1)}\right)\hat{F}_{1}^{(1,0)}+\frac{\tilde{\mu}_{2}\hat{\mu}_{1}}{1-\mu_{1}}F_{1}^{(1,0)}+\tilde{\mu}_{2}\hat{\mu}_{1}\theta_{1}^{(2)}\left(1\right)F_{1}^{(1,0)}+\frac{\hat{\mu}_{1}}{1-\mu_{1}}F_{1}^{(1,1)}\\
&+&\left(\frac{1}{1-\mu_{1}}\right)^{2}\tilde{\mu}_{2}\hat{\mu}_{1}F_{1}^{(2,0)}.
\end{eqnarray*}
%8/24
\item \begin{eqnarray*}
&&\frac{\partial}{\partial w_2}\frac{\partial}{\partial z_2}\left(R_1\left(P_1\left(z_1\right)\bar{P}_2\left(z_2\right)\hat{P}_1\left(w_1\right)\hat{P}_2\left(w_2\right)\right)F_1\left(\theta_1\left(\tilde{P}_2\left(z_1\right)\hat{P}_1\left(w_1\right)\hat{P}_2\left(w_2\right)\right)\right)\hat{F}_1\left(w_1,w_2\right)\right)\\
&=&\hat{\mu}_{2}\tilde{\mu}_{2}r_{1}+\hat{\mu}_{2}\tilde{\mu}_{2}R_{1}^{(2)}(1)+r_{1}\tilde{\mu}_{2}\hat{F}_{1}^{(0,1)}+r_{1}\frac{\hat{\mu}_{2}\tilde{\mu}_{2}}{1-\mu_{1}}F_{1}^{(1,0)}+\hat{\mu}_{2}r_{1}\left(\frac{\tilde{\mu}_{2}}{1-\mu_{1}}F_{1}^{(1,0)}+F_{1}^{(0,1)}\right)\\
&+&\left(\frac{\tilde{\mu}_{2}}{1-\mu_{1}}F_{1}^{(1,0)}+F_{1}^{(0,1)}\right)\hat{F}_{1}^{(0,1)}+\frac{\tilde{\mu}_{2}\hat{\mu_{2}}}{1-\mu_{1}}F_{1}^{(1,0)}+\hat{\mu}_{2}\tilde{\mu}_{2}\theta_{1}^{(2)}\left(1\right)F_{1}^{(1,0)}+\frac{\hat{\mu}_{2}}{1-\mu_{1}}F_{1}^{(1,1)}\\
&+&\left(\frac{1}{1-\mu_{1}}\right)^{2}\tilde{\mu}_{2}\hat{\mu}_{2}F_{1}^{(2,0)}.
\end{eqnarray*}
%___________________________________________________________________________________________
%\subsubsection{Mixtas para $w_{1}$:}
%___________________________________________________________________________________________
%9/25
\item \begin{eqnarray*} &&\frac{\partial}{\partial
z_1}\frac{\partial}{\partial
w_1}\left(R_1\left(P_1\left(z_1\right)\bar{P}_2\left(z_2\right)\hat{P}_1\left(w_1\right)\hat{P}_2\left(w_2\right)\right)F_1\left(\theta_1\left(\tilde{P}_2\left(z_1\right)\hat{P}_1\left(w_1\right)\hat{P}_2\left(w_2\right)\right)\right)\hat{F}_1\left(w_1,w_2\right)\right)\\
&=&r_{1}\mu_{1}\hat{\mu}_{1}+\mu_{1}\hat{\mu}_{1}R_{1}^{(2)}(1)+r_{1}\frac{\mu_{1}\hat{\mu}_{1}}{1-\mu_{1}}F_{1}^{(1,0)}+r_{1}\mu_{1}\hat{F}_{1}^{(1,0)}.
\end{eqnarray*}
%10/26
\item \begin{eqnarray*} &&\frac{\partial}{\partial
z_2}\frac{\partial}{\partial
w_1}\left(R_1\left(P_1\left(z_1\right)\bar{P}_2\left(z_2\right)\hat{P}_1\left(w_1\right)\hat{P}_2\left(w_2\right)\right)F_1\left(\theta_1\left(\tilde{P}_2\left(z_1\right)\hat{P}_1\left(w_1\right)\hat{P}_2\left(w_2\right)\right)\right)\hat{F}_1\left(w_1,w_2\right)\right)\\
&=&r_{1}\hat{\mu}_{1}\tilde{\mu}_{2}+\tilde{\mu}_{2}\hat{\mu}_{1}R_{1}^{(2)}\left(1\right)+
\frac{\hat{\mu}_{1}\tilde{\mu}_{2}}{1-\mu_{1}}F_{1}^{1,0)}+r_{1}\frac{\hat{\mu}_{1}\tilde{\mu}_{2}}{1-\mu_{1}}F_{1}^{(1,0)}+\hat{\mu}_{1}\tilde{\mu}_{2}\theta_{1}^{(2)}\left(1\right)F_{2}^{(1,0)}\\
&+&r_{1}\hat{\mu}_{1}\left(F_{1}^{(1,0)}+\frac{\tilde{\mu}_{2}}{1-\mu_{1}}F_{1}^{1,0)}\right)+
r_{1}\tilde{\mu}_{2}\hat{F}_{1}^{(1,0)}+\left(F_{1}^{(0,1)}+\frac{\tilde{\mu}_{2}}{1-\mu_{1}}F_{1}^{1,0)}\right)\hat{F}_{1}^{(1,0)}\\
&+&\frac{\hat{\mu}_{1}}{1-\mu_{1}}\left(F_{1}^{(1,1)}+\frac{\tilde{\mu}_{2}}{1-\mu_{1}}F_{1}^{2,0)}\right).
\end{eqnarray*}
%11/27
\item \begin{eqnarray*} &&\frac{\partial}{\partial
w_1}\frac{\partial}{\partial
w_1}\left(R_1\left(P_1\left(z_1\right)\bar{P}_2\left(z_2\right)\hat{P}_1\left(w_1\right)\hat{P}_2\left(w_2\right)\right)F_1\left(\theta_1\left(\tilde{P}_2\left(z_1\right)\hat{P}_1\left(w_1\right)\hat{P}_2\left(w_2\right)\right)\right)\hat{F}_1\left(w_1,w_2\right)\right)\\
&=&\hat{\mu}_{1}^{2}R_{1}^{(2)}\left(1\right)+r_{1}\hat{P}_{1}^{(2)}\left(1\right)+2r_{1}\frac{\hat{\mu}_{1}^{2}}{1-\mu_{1}}F_{1}^{(1,0)}+\hat{\mu}_{1}^{2}\theta_{1}^{(2)}\left(1\right)F_{1}^{(1,0)}+\frac{1}{1-\mu_{1}}\hat{P}_{1}^{(2)}\left(1\right)F_{1}^{(1,0)}\\
&+&2r_{1}\hat{\mu}_{1}\hat{F}_{1}^{(1,0)}+2\frac{\hat{\mu}_{1}}{1-\mu_{1}}F_{1}^{(1,0)}\hat{F}_{1}^{(1,0)}+\left(\frac{\hat{\mu}_{1}}{1-\mu_{1}}\right)^{2}F_{1}^{(2,0)}+\hat{F}_{1}^{(2,0)}.
\end{eqnarray*}
%12/28
\item \begin{eqnarray*} &&\frac{\partial}{\partial
w_2}\frac{\partial}{\partial
w_1}\left(R_1\left(P_1\left(z_1\right)\bar{P}_2\left(z_2\right)\hat{P}_1\left(w_1\right)\hat{P}_2\left(w_2\right)\right)F_1\left(\theta_1\left(\tilde{P}_2\left(z_1\right)\hat{P}_1\left(w_1\right)\hat{P}_2\left(w_2\right)\right)\right)\hat{F}_1\left(w_1,w_2\right)\right)\\
&=&r_{1}\hat{\mu}_{1}\hat{\mu}_{2}+\hat{\mu}_{1}\hat{\mu}_{2}R_{1}^{(2)}\left(1\right)+r_{1}\hat{\mu}_{1}\hat{F}_{1}^{(0,1)}+
\frac{\hat{\mu}_{1}\hat{\mu}_{2}}{1-\mu_{1}}F_{1}^{(1,0)}+2r_{1}\frac{\hat{\mu}_{1}\hat{\mu}_{2}}{1-\mu_{1}}F_{1}^{1,0)}+\hat{\mu}_{1}\hat{\mu}_{2}\theta_{1}^{(2)}\left(1\right)F_{1}^{(1,0)}\\
&+&\frac{\hat{\mu}_{1}}{1-\mu_{1}}F_{1}^{(1,0)}\hat{F}_{1}^{(0,1)}+
r_{1}\hat{\mu}_{2}\hat{F}_{1}^{(1,0)}+\frac{\hat{\mu}_{2}}{1-\mu_{1}}\hat{F}_{1}^{(1,0)}F_{1}^{(1,0)}+\hat{F}_{1}^{(1,1)}+\hat{\mu}_{1}\hat{\mu}_{2}\left(\frac{1}{1-\mu_{1}}\right)^{2}F_{1}^{(2,0)}.
\end{eqnarray*}
%___________________________________________________________________________________________
%\subsubsection{Mixtas para $w_{2}$:}
%___________________________________________________________________________________________
%13/29
\item \begin{eqnarray*} &&\frac{\partial}{\partial
z_1}\frac{\partial}{\partial
w_2}\left(R_1\left(P_1\left(z_1\right)\bar{P}_2\left(z_2\right)\hat{P}_1\left(w_1\right)\hat{P}_2\left(w_2\right)\right)F_1\left(\theta_1\left(\tilde{P}_2\left(z_1\right)\hat{P}_1\left(w_1\right)\hat{P}_2\left(w_2\right)\right)\right)\hat{F}_1\left(w_1,w_2\right)\right)\\
&=&r_{1}\mu_{1}\hat{\mu}_{2}+\mu_{1}\hat{\mu}_{2}R_{1}^{(2)}\left(1\right)+r_{1}\mu_{1}\hat{F}_{1}^{(0,1)}+r_{1}\frac{\mu_{1}\hat{\mu}_{2}}{1-\mu_{1}}F_{1}^{(1,0)}.
\end{eqnarray*}
%14/30
\item \begin{eqnarray*} &&\frac{\partial}{\partial
z_2}\frac{\partial}{\partial
w_2}\left(R_1\left(P_1\left(z_1\right)\bar{P}_2\left(z_2\right)\hat{P}_1\left(w_1\right)\hat{P}_2\left(w_2\right)\right)F_1\left(\theta_1\left(\tilde{P}_2\left(z_1\right)\hat{P}_1\left(w_1\right)\hat{P}_2\left(w_2\right)\right)\right)\hat{F}_1\left(w_1,w_2\right)\right)\\
&=&r_{1}\hat{\mu}_{2}\tilde{\mu}_{2}+\hat{\mu}_{2}\tilde{\mu}_{2}R_{1}^{(2)}\left(1\right)+r_{1}\tilde{\mu}_{2}\hat{F}_{1}^{(0,1)}+\frac{\hat{\mu}_{2}\tilde{\mu}_{2}}{1-\mu_{1}}F_{1}^{(1,0)}+r_{1}\frac{\hat{\mu}_{2}\tilde{\mu}_{2}}{1-\mu_{1}}F_{1}^{(1,0)}\\
&+&\hat{\mu}_{2}\tilde{\mu}_{2}\theta_{1}^{(2)}\left(1\right)F_{1}^{(1,0)}+r_{1}\hat{\mu}_{2}\left(F_{1}^{(0,1)}+\frac{\tilde{\mu}_{2}}{1-\mu_{1}}F_{1}^{(1,0)}\right)+\left(F_{1}^{(0,1)}+\frac{\tilde{\mu}_{2}}{1-\mu_{1}}F_{1}^{(1,0)}\right)\hat{F}_{1}^{(0,1)}\\&+&\frac{\hat{\mu}_{2}}{1-\mu_{1}}\left(F_{1}^{(1,1)}+\frac{\tilde{\mu}_{2}}{1-\mu_{1}}F_{1}^{(2,0)}\right).
\end{eqnarray*}
%15/31
\item \begin{eqnarray*} &&\frac{\partial}{\partial
w_1}\frac{\partial}{\partial
w_2}\left(R_1\left(P_1\left(z_1\right)\bar{P}_2\left(z_2\right)\hat{P}_1\left(w_1\right)\hat{P}_2\left(w_2\right)\right)F_1\left(\theta_1\left(\tilde{P}_2\left(z_1\right)\hat{P}_1\left(w_1\right)\hat{P}_2\left(w_2\right)\right)\right)\hat{F}_1\left(w_1,w_2\right)\right)\\
&=&r_{1}\hat{\mu}_{1}\hat{\mu}_{2}+\hat{\mu}_{1}\hat{\mu}_{2}R_{1}^{(2)}\left(1\right)+r_{1}\hat{\mu}_{1}\hat{F}_{1}^{(0,1)}+
\frac{\hat{\mu}_{1}\hat{\mu}_{2}}{1-\mu_{1}}F_{1}^{(1,0)}+2r_{1}\frac{\hat{\mu}_{1}\hat{\mu}_{2}}{1-\mu_{1}}F_{1}^{(1,0)}+\hat{\mu}_{1}\hat{\mu}_{2}\theta_{1}^{(2)}\left(1\right)F_{1}^{(1,0)}\\
&+&\frac{\hat{\mu}_{1}}{1-\mu_{1}}\hat{F}_{1}^{(0,1)}F_{1}^{(1,0)}+r_{1}\hat{\mu}_{2}\hat{F}_{1}^{(1,0)}+\frac{\hat{\mu}_{2}}{1-\mu_{1}}\hat{F}_{1}^{(1,0)}F_{1}^{(1,0)}+\hat{F}_{1}^{(1,1)}+\hat{\mu}_{1}\hat{\mu}_{2}\left(\frac{1}{1-\mu_{1}}\right)^{2}F_{1}^{(2,0)}.
\end{eqnarray*}
%16/32
\item \begin{eqnarray*} &&\frac{\partial}{\partial
w_2}\frac{\partial}{\partial
w_2}\left(R_1\left(P_1\left(z_1\right)\bar{P}_2\left(z_2\right)\hat{P}_1\left(w_1\right)\hat{P}_2\left(w_2\right)\right)F_1\left(\theta_1\left(\tilde{P}_2\left(z_1\right)\hat{P}_1\left(w_1\right)\hat{P}_2\left(w_2\right)\right)\right)\hat{F}_1\left(w_1,w_2\right)\right)\\
&=&\hat{\mu}_{2}R_{1}^{(2)}\left(1\right)+r_{1}\hat{P}_{2}^{(2)}\left(1\right)+2r_{1}\hat{\mu}_{2}\hat{F}_{1}^{(0,1)}+\hat{F}_{1}^{(0,2)}+2r_{1}\frac{\hat{\mu}_{2}^{2}}{1-\mu_{1}}F_{1}^{(1,0)}+\hat{\mu}_{2}^{2}\theta_{1}^{(2)}\left(1\right)F_{1}^{(1,0)}\\
&+&\frac{1}{1-\mu_{1}}\hat{P}_{2}^{(2)}\left(1\right)F_{1}^{(1,0)} +
2\frac{\hat{\mu}_{2}}{1-\mu_{1}}F_{1}^{(1,0)}\hat{F}_{1}^{(0,1)}+\left(\frac{\hat{\mu}_{2}}{1-\mu_{1}}\right)^{2}F_{1}^{(2,0)}.
\end{eqnarray*}
\end{enumerate}

%___________________________________________________________________________________________
%
%\subsection{Derivadas de Segundo Orden para $\hat{F}_{1}$}
%___________________________________________________________________________________________


\begin{enumerate}
%___________________________________________________________________________________________
%\subsubsection{Mixtas para $z_{1}$:}
%___________________________________________________________________________________________
%1/33

\item \begin{eqnarray*} &&\frac{\partial}{\partial
z_1}\frac{\partial}{\partial
z_1}\left(\hat{R}_{2}\left(P_{1}\left(z_{1}\right)\tilde{P}_{2}\left(z_{2}\right)\hat{P}_{1}\left(w_{1}\right)\hat{P}_{2}\left(w_{2}\right)\right)\hat{F}_{2}\left(w_{1},\hat{\theta}_{2}\left(P_{1}\left(z_{1}\right)\tilde{P}_{2}\left(z_{2}\right)\hat{P}_{1}\left(w_{1}\right)\right)\right)F_{2}\left(z_{1},z_{2}\right)\right)\\
&=&\hat{r}_{2}P_{1}^{(2)}\left(1\right)+
\mu_{1}^{2}\hat{R}_{2}^{(2)}\left(1\right)+
2\hat{r}_{2}\frac{\mu_{1}^{2}}{1-\hat{\mu}_{2}}\hat{F}_{2}^{(0,1)}+
\frac{1}{1-\hat{\mu}_{2}}P_{1}^{(2)}\left(1\right)\hat{F}_{2}^{(0,1)}+
\mu_{1}^{2}\hat{\theta}_{2}^{(2)}\left(1\right)\hat{F}_{2}^{(0,1)}\\
&+&\left(\frac{\mu_{1}^{2}}{1-\hat{\mu}_{2}}\right)^{2}\hat{F}_{2}^{(0,2)}+
2\hat{r}_{2}\mu_{1}F_{2}^{(1,0)}+2\frac{\mu_{1}}{1-\hat{\mu}_{2}}\hat{F}_{2}^{(0,1)}F_{2}^{(1,0)}+F_{2}^{(2,0)}.
\end{eqnarray*}

%2/34
\item \begin{eqnarray*} &&\frac{\partial}{\partial
z_2}\frac{\partial}{\partial
z_1}\left(\hat{R}_{2}\left(P_{1}\left(z_{1}\right)\tilde{P}_{2}\left(z_{2}\right)\hat{P}_{1}\left(w_{1}\right)\hat{P}_{2}\left(w_{2}\right)\right)\hat{F}_{2}\left(w_{1},\hat{\theta}_{2}\left(P_{1}\left(z_{1}\right)\tilde{P}_{2}\left(z_{2}\right)\hat{P}_{1}\left(w_{1}\right)\right)\right)F_{2}\left(z_{1},z_{2}\right)\right)\\
&=&\hat{r}_{2}\mu_{1}\tilde{\mu}_{2}+\mu_{1}\tilde{\mu}_{2}\hat{R}_{2}^{(2)}\left(1\right)+\hat{r}_{2}\mu_{1}F_{2}^{(0,1)}+
\frac{\mu_{1}\tilde{\mu}_{2}}{1-\hat{\mu}_{2}}\hat{F}_{2}^{(0,1)}+2\hat{r}_{2}\frac{\mu_{1}\tilde{\mu}_{2}}{1-\hat{\mu}_{2}}\hat{F}_{2}^{(0,1)}+\mu_{1}\tilde{\mu}_{2}\hat{\theta}_{2}^{(2)}\left(1\right)\hat{F}_{2}^{(0,1)}\\
&+&\frac{\mu_{1}}{1-\hat{\mu}_{2}}F_{2}^{(0,1)}\hat{F}_{2}^{(0,1)}+\mu_{1} \tilde{\mu}_{2}\left(\frac{1}{1-\hat{\mu}_{2}}\right)^{2}\hat{F}_{2}^{(0,2)}+\hat{r}_{2}\tilde{\mu}_{2}F_{2}^{(1,0)}+\frac{\tilde{\mu}_{2}}{1-\hat{\mu}_{2}}\hat{F}_{2}^{(0,1)}F_{2}^{(1,0)}+F_{2}^{(1,1)}.
\end{eqnarray*}


%3/35

\item \begin{eqnarray*} &&\frac{\partial}{\partial
w_1}\frac{\partial}{\partial
z_1}\left(\hat{R}_{2}\left(P_{1}\left(z_{1}\right)\tilde{P}_{2}\left(z_{2}\right)\hat{P}_{1}\left(w_{1}\right)\hat{P}_{2}\left(w_{2}\right)\right)\hat{F}_{2}\left(w_{1},\hat{\theta}_{2}\left(P_{1}\left(z_{1}\right)\tilde{P}_{2}\left(z_{2}\right)\hat{P}_{1}\left(w_{1}\right)\right)\right)F_{2}\left(z_{1},z_{2}\right)\right)\\
&=&\hat{r}_{2}\mu_{1}\hat{\mu}_{1}+\mu_{1}\hat{\mu}_{1}\hat{R}_{2}^{(2)}\left(1\right)+\hat{r}_{2}\frac{\mu_{1}\hat{\mu}_{1}}{1-\hat{\mu}_{2}}\hat{F}_{2}^{(0,1)}+\hat{r}_{2}\hat{\mu}_{1}F_{2}^{(1,0)}+\hat{r}_{2}\mu_{1}\hat{F}_{2}^{(1,0)}+F_{2}^{(1,0)}\hat{F}_{2}^{(1,0)}+\frac{\mu_{1}}{1-\hat{\mu}_{2}}\hat{F}_{2}^{(1,1)}.
\end{eqnarray*}

%4/36

\item \begin{eqnarray*} &&\frac{\partial}{\partial
w_2}\frac{\partial}{\partial
z_1}\left(\hat{R}_{2}\left(P_{1}\left(z_{1}\right)\tilde{P}_{2}\left(z_{2}\right)\hat{P}_{1}\left(w_{1}\right)\hat{P}_{2}\left(w_{2}\right)\right)\hat{F}_{2}\left(w_{1},\hat{\theta}_{2}\left(P_{1}\left(z_{1}\right)\tilde{P}_{2}\left(z_{2}\right)\hat{P}_{1}\left(w_{1}\right)\right)\right)F_{2}\left(z_{1},z_{2}\right)\right)\\
&=&\hat{r}_{2}\mu_{1}\hat{\mu}_{2}+\mu_{1}\hat{\mu}_{2}\hat{R}_{2}^{(2)}\left(1\right)+\frac{\mu_{1}\hat{\mu}_{2}}{1-\hat{\mu}_{2}}\hat{F}_{2}^{(0,1)}+2\hat{r}_{2}\frac{\mu_{1}\hat{\mu}_{2}}{1-\hat{\mu}_{2}}\hat{F}_{2}^{(0,1)}+\mu_{1}\hat{\mu}_{2}\hat{\theta}_{2}^{(2)}\left(1\right)\hat{F}_{2}^{(0,1)}\\
&+&\mu_{1}\hat{\mu}_{2}\left(\frac{1}{1-\hat{\mu}_{2}}\right)^{2}\hat{F}_{2}^{(0,2)}+\hat{r}_{2}\hat{\mu}_{2}F_{2}^{(1,0)}+\frac{\hat{\mu}_{2}}{1-\hat{\mu}_{2}}\hat{F}_{2}^{(0,1)}F_{2}^{(1,0)}.
\end{eqnarray*}
%___________________________________________________________________________________________
%\subsubsection{Mixtas para $z_{2}$:}
%___________________________________________________________________________________________

%5/37

\item \begin{eqnarray*} &&\frac{\partial}{\partial
z_1}\frac{\partial}{\partial
z_2}\left(\hat{R}_{2}\left(P_{1}\left(z_{1}\right)\tilde{P}_{2}\left(z_{2}\right)\hat{P}_{1}\left(w_{1}\right)\hat{P}_{2}\left(w_{2}\right)\right)\hat{F}_{2}\left(w_{1},\hat{\theta}_{2}\left(P_{1}\left(z_{1}\right)\tilde{P}_{2}\left(z_{2}\right)\hat{P}_{1}\left(w_{1}\right)\right)\right)F_{2}\left(z_{1},z_{2}\right)\right)\\
&=&\hat{r}_{2}\mu_{1}\tilde{\mu}_{2}+\mu_{1}\tilde{\mu}_{2}\hat{R}_{2}^{(2)}\left(1\right)+\mu_{1}\hat{r}_{2}F_{2}^{(0,1)}+
\frac{\mu_{1}\tilde{\mu}_{2}}{1-\hat{\mu}_{2}}\hat{F}_{2}^{(0,1)}+2\hat{r}_{2}\frac{\mu_{1}\tilde{\mu}_{2}}{1-\hat{\mu}_{2}}\hat{F}_{2}^{(0,1)}+\mu_{1}\tilde{\mu}_{2}\hat{\theta}_{2}^{(2)}\left(1\right)\hat{F}_{2}^{(0,1)}\\
&+&\frac{\mu_{1}}{1-\hat{\mu}_{2}}F_{2}^{(0,1)}\hat{F}_{2}^{(0,1)}+\mu_{1}\tilde{\mu}_{2}\left(\frac{1}{1-\hat{\mu}_{2}}\right)^{2}\hat{F}_{2}^{(0,2)}+\hat{r}_{2}\tilde{\mu}_{2}F_{2}^{(1,0)}+\frac{\tilde{\mu}_{2}}{1-\hat{\mu}_{2}}\hat{F}_{2}^{(0,1)}F_{2}^{(1,0)}+F_{2}^{(1,1)}.
\end{eqnarray*}

%6/38

\item \begin{eqnarray*} &&\frac{\partial}{\partial
z_2}\frac{\partial}{\partial
z_2}\left(\hat{R}_{2}\left(P_{1}\left(z_{1}\right)\tilde{P}_{2}\left(z_{2}\right)\hat{P}_{1}\left(w_{1}\right)\hat{P}_{2}\left(w_{2}\right)\right)\hat{F}_{2}\left(w_{1},\hat{\theta}_{2}\left(P_{1}\left(z_{1}\right)\tilde{P}_{2}\left(z_{2}\right)\hat{P}_{1}\left(w_{1}\right)\right)\right)F_{2}\left(z_{1},z_{2}\right)\right)\\
&=&\hat{r}_{2}\tilde{P}_{2}^{(2)}\left(1\right)+\tilde{\mu}_{2}^{2}\hat{R}_{2}^{(2)}\left(1\right)+2\hat{r}_{2}\tilde{\mu}_{2}F_{2}^{(0,1)}+2\hat{r}_{2}\frac{\tilde{\mu}_{2}^{2}}{1-\hat{\mu}_{2}}\hat{F}_{2}^{(0,1)}+\frac{1}{1-\hat{\mu}_{2}}\tilde{P}_{2}^{(2)}\left(1\right)\hat{F}_{2}^{(0,1)}\\
&+&\tilde{\mu}_{2}^{2}\hat{\theta}_{2}^{(2)}\left(1\right)\hat{F}_{2}^{(0,1)}+2\frac{\tilde{\mu}_{2}}{1-\hat{\mu}_{2}}F_{2}^{(0,1)}\hat{F}_{2}^{(0,1)}+F_{2}^{(0,2)}+\left(\frac{\tilde{\mu}_{2}}{1-\hat{\mu}_{2}}\right)^{2}\hat{F}_{2}^{(0,2)}.
\end{eqnarray*}

%7/39

\item \begin{eqnarray*} &&\frac{\partial}{\partial
w_1}\frac{\partial}{\partial
z_2}\left(\hat{R}_{2}\left(P_{1}\left(z_{1}\right)\tilde{P}_{2}\left(z_{2}\right)\hat{P}_{1}\left(w_{1}\right)\hat{P}_{2}\left(w_{2}\right)\right)\hat{F}_{2}\left(w_{1},\hat{\theta}_{2}\left(P_{1}\left(z_{1}\right)\tilde{P}_{2}\left(z_{2}\right)\hat{P}_{1}\left(w_{1}\right)\right)\right)F_{2}\left(z_{1},z_{2}\right)\right)\\
&=&\hat{r}_{2}\tilde{\mu}_{2}\hat{\mu}_{1}+\tilde{\mu}_{2}\hat{\mu}_{1}\hat{R}_{2}^{(2)}\left(1\right)+\hat{r}_{2}\hat{\mu}_{1}F_{2}^{(0,1)}+\hat{r}_{2}\frac{\tilde{\mu}_{2}\hat{\mu}_{1}}{1-\hat{\mu}_{2}}\hat{F}_{2}^{(0,1)}+\hat{r}_{2}\tilde{\mu}_{2}\hat{F}_{2}^{(1,0)}+F_{2}^{(0,1)}\hat{F}_{2}^{(1,0)}+\frac{\tilde{\mu}_{2}}{1-\hat{\mu}_{2}}\hat{F}_{2}^{(1,1)}.
\end{eqnarray*}
%8/40

\item \begin{eqnarray*} &&\frac{\partial}{\partial
w_2}\frac{\partial}{\partial
z_2}\left(\hat{R}_{2}\left(P_{1}\left(z_{1}\right)\tilde{P}_{2}\left(z_{2}\right)\hat{P}_{1}\left(w_{1}\right)\hat{P}_{2}\left(w_{2}\right)\right)\hat{F}_{2}\left(w_{1},\hat{\theta}_{2}\left(P_{1}\left(z_{1}\right)\tilde{P}_{2}\left(z_{2}\right)\hat{P}_{1}\left(w_{1}\right)\right)\right)F_{2}\left(z_{1},z_{2}\right)\right)\\
&=&\hat{r}_{2}\tilde{\mu}_{2}\hat{\mu}_{2}+\tilde{\mu}_{2}\hat{\mu}_{2}\hat{R}_{2}^{(2)}\left(1\right)+\hat{r}_{2}\hat{\mu}_{2}F_{2}^{(0,1)}+
\frac{\tilde{\mu}_{2}\hat{\mu}_{2}}{1-\hat{\mu}_{2}}\hat{F}_{2}^{(0,1)}+2\hat{r}_{2}\frac{\tilde{\mu}_{2}\hat{\mu}_{2}}{1-\hat{\mu}_{2}}\hat{F}_{2}^{(0,1)}+\tilde{\mu}_{2}\hat{\mu}_{2}\hat{\theta}_{2}^{(2)}\left(1\right)\hat{F}_{2}^{(0,1)}\\
&+&\frac{\hat{\mu}_{2}}{1-\hat{\mu}_{2}}F_{2}^{(0,1)}\hat{F}_{2}^{(1,0)}+\tilde{\mu}_{2}\hat{\mu}_{2}\left(\frac{1}{1-\hat{\mu}_{2}}\right)\hat{F}_{2}^{(0,2)}.
\end{eqnarray*}
%___________________________________________________________________________________________
%\subsubsection{Mixtas para $w_{1}$:}
%___________________________________________________________________________________________

%9/41
\item \begin{eqnarray*} &&\frac{\partial}{\partial
z_1}\frac{\partial}{\partial
w_1}\left(\hat{R}_{2}\left(P_{1}\left(z_{1}\right)\tilde{P}_{2}\left(z_{2}\right)\hat{P}_{1}\left(w_{1}\right)\hat{P}_{2}\left(w_{2}\right)\right)\hat{F}_{2}\left(w_{1},\hat{\theta}_{2}\left(P_{1}\left(z_{1}\right)\tilde{P}_{2}\left(z_{2}\right)\hat{P}_{1}\left(w_{1}\right)\right)\right)F_{2}\left(z_{1},z_{2}\right)\right)\\
&=&\hat{r}_{2}\mu_{1}\hat{\mu}_{1}+\mu_{1}\hat{\mu}_{1}\hat{R}_{2}^{(2)}\left(1\right)+\hat{r}_{2}\frac{\mu_{1}\hat{\mu}_{1}}{1-\hat{\mu}_{2}}\hat{F}_{2}^{(0,1)}+\hat{r}_{2}\hat{\mu}_{1}F_{2}^{(1,0)}+\hat{r}_{2}\mu_{1}\hat{F}_{2}^{(1,0)}+F_{2}^{(1,0)}\hat{F}_{2}^{(1,0)}+\frac{\mu_{1}}{1-\hat{\mu}_{2}}\hat{F}_{2}^{(1,1)}.
\end{eqnarray*}


%10/42
\item \begin{eqnarray*} &&\frac{\partial}{\partial
z_2}\frac{\partial}{\partial
w_1}\left(\hat{R}_{2}\left(P_{1}\left(z_{1}\right)\tilde{P}_{2}\left(z_{2}\right)\hat{P}_{1}\left(w_{1}\right)\hat{P}_{2}\left(w_{2}\right)\right)\hat{F}_{2}\left(w_{1},\hat{\theta}_{2}\left(P_{1}\left(z_{1}\right)\tilde{P}_{2}\left(z_{2}\right)\hat{P}_{1}\left(w_{1}\right)\right)\right)F_{2}\left(z_{1},z_{2}\right)\right)\\
&=&\hat{r}_{2}\tilde{\mu}_{2}\hat{\mu}_{1}+\tilde{\mu}_{2}\hat{\mu}_{1}\hat{R}_{2}^{(2)}\left(1\right)+\hat{r}_{2}\hat{\mu}_{1}F_{2}^{(0,1)}+
\hat{r}_{2}\frac{\tilde{\mu}_{2}\hat{\mu}_{1}}{1-\hat{\mu}_{2}}\hat{F}_{2}^{(0,1)}+\hat{r}_{2}\tilde{\mu}_{2}\hat{F}_{2}^{(1,0)}+F_{2}^{(0,1)}\hat{F}_{2}^{(1,0)}+\frac{\tilde{\mu}_{2}}{1-\hat{\mu}_{2}}\hat{F}_{2}^{(1,1)}.
\end{eqnarray*}


%11/43
\item \begin{eqnarray*} &&\frac{\partial}{\partial
w_1}\frac{\partial}{\partial
w_1}\left(\hat{R}_{2}\left(P_{1}\left(z_{1}\right)\tilde{P}_{2}\left(z_{2}\right)\hat{P}_{1}\left(w_{1}\right)\hat{P}_{2}\left(w_{2}\right)\right)\hat{F}_{2}\left(w_{1},\hat{\theta}_{2}\left(P_{1}\left(z_{1}\right)\tilde{P}_{2}\left(z_{2}\right)\hat{P}_{1}\left(w_{1}\right)\right)\right)F_{2}\left(z_{1},z_{2}\right)\right)\\
&=&\hat{r}_{2}\hat{P}_{1}^{(2)}\left(1\right)+\hat{\mu}_{1}^{2}\hat{R}_{2}^{(2)}\left(1\right)+2\hat{r}_{2}\hat{\mu}_{1}\hat{F}_{2}^{(1,0)}
+\hat{F}_{2}^{(2,0)}.
\end{eqnarray*}


%12/44
\item \begin{eqnarray*} &&\frac{\partial}{\partial
w_2}\frac{\partial}{\partial
w_1}\left(\hat{R}_{2}\left(P_{1}\left(z_{1}\right)\tilde{P}_{2}\left(z_{2}\right)\hat{P}_{1}\left(w_{1}\right)\hat{P}_{2}\left(w_{2}\right)\right)\hat{F}_{2}\left(w_{1},\hat{\theta}_{2}\left(P_{1}\left(z_{1}\right)\tilde{P}_{2}\left(z_{2}\right)\hat{P}_{1}\left(w_{1}\right)\right)\right)F_{2}\left(z_{1},z_{2}\right)\right)\\
&=&\hat{r}_{2}\hat{\mu}_{1}\hat{\mu}_{2}+\hat{\mu}_{1}\hat{\mu}_{2}\hat{R}_{2}^{(2)}\left(1\right)+
\hat{r}_{2}\frac{\hat{\mu}_{2}\hat{\mu}_{1}}{1-\hat{\mu}_{2}}\hat{F}_{2}^{(0,1)}
+\hat{r}_{2}\hat{\mu}_{2}\hat{F}_{2}^{(1,0)}+\frac{\hat{\mu}_{2}}{1-\hat{\mu}_{2}}\hat{F}_{2}^{(1,1)}.
\end{eqnarray*}
%___________________________________________________________________________________________
%\subsubsection{Mixtas para $w_{2}$:}
%___________________________________________________________________________________________
%13/45
\item \begin{eqnarray*} &&\frac{\partial}{\partial
z_1}\frac{\partial}{\partial
w_2}\left(\hat{R}_{2}\left(P_{1}\left(z_{1}\right)\tilde{P}_{2}\left(z_{2}\right)\hat{P}_{1}\left(w_{1}\right)\hat{P}_{2}\left(w_{2}\right)\right)\hat{F}_{2}\left(w_{1},\hat{\theta}_{2}\left(P_{1}\left(z_{1}\right)\tilde{P}_{2}\left(z_{2}\right)\hat{P}_{1}\left(w_{1}\right)\right)\right)F_{2}\left(z_{1},z_{2}\right)\right)\\
&=&\hat{r}_{2}\mu_{1}\hat{\mu}_{2}+\mu_{1}\hat{\mu}_{2}\hat{R}_{2}^{(2)}\left(1\right)+
\frac{\mu_{1}\hat{\mu}_{2}}{1-\hat{\mu}_{2}}\hat{F}_{2}^{(0,1)} +2\hat{r}_{2}\frac{\mu_{1}\hat{\mu}_{2}}{1-\hat{\mu}_{2}}\hat{F}_{2}^{(0,1)}\\
&+&\mu_{1}\hat{\mu}_{2}\hat{\theta}_{2}^{(2)}\left(1\right)\hat{F}_{2}^{(0,1)}+\mu_{1}\hat{\mu}_{2}\left(\frac{1}{1-\hat{\mu}_{2}}\right)^{2}\hat{F}_{2}^{(0,2)}+\hat{r}_{2}\hat{\mu}_{2}F_{2}^{(1,0)}+\frac{\hat{\mu}_{2}}{1-\hat{\mu}_{2}}\hat{F}_{2}^{(0,1)}F_{2}^{(1,0)}.\end{eqnarray*}


%14/46
\item \begin{eqnarray*} &&\frac{\partial}{\partial
z_2}\frac{\partial}{\partial
w_2}\left(\hat{R}_{2}\left(P_{1}\left(z_{1}\right)\tilde{P}_{2}\left(z_{2}\right)\hat{P}_{1}\left(w_{1}\right)\hat{P}_{2}\left(w_{2}\right)\right)\hat{F}_{2}\left(w_{1},\hat{\theta}_{2}\left(P_{1}\left(z_{1}\right)\tilde{P}_{2}\left(z_{2}\right)\hat{P}_{1}\left(w_{1}\right)\right)\right)F_{2}\left(z_{1},z_{2}\right)\right)\\
&=&\hat{r}_{2}\tilde{\mu}_{2}\hat{\mu}_{2}+\tilde{\mu}_{2}\hat{\mu}_{2}\hat{R}_{2}^{(2)}\left(1\right)+\hat{r}_{2}\hat{\mu}_{2}F_{2}^{(0,1)}+\frac{\tilde{\mu}_{2}\hat{\mu}_{2}}{1-\hat{\mu}_{2}}\hat{F}_{2}^{(0,1)}+
2\hat{r}_{2}\frac{\tilde{\mu}_{2}\hat{\mu}_{2}}{1-\hat{\mu}_{2}}\hat{F}_{2}^{(0,1)}+\tilde{\mu}_{2}\hat{\mu}_{2}\hat{\theta}_{2}^{(2)}\left(1\right)\hat{F}_{2}^{(0,1)}\\
&+&\frac{\hat{\mu}_{2}}{1-\hat{\mu}_{2}}\hat{F}_{2}^{(0,1)}F_{2}^{(0,1)}+\tilde{\mu}_{2}\hat{\mu}_{2}\left(\frac{1}{1-\hat{\mu}_{2}}\right)^{2}\hat{F}_{2}^{(0,2)}.
\end{eqnarray*}

%15/47

\item \begin{eqnarray*} &&\frac{\partial}{\partial
w_1}\frac{\partial}{\partial
w_2}\left(\hat{R}_{2}\left(P_{1}\left(z_{1}\right)\tilde{P}_{2}\left(z_{2}\right)\hat{P}_{1}\left(w_{1}\right)\hat{P}_{2}\left(w_{2}\right)\right)\hat{F}_{2}\left(w_{1},\hat{\theta}_{2}\left(P_{1}\left(z_{1}\right)\tilde{P}_{2}\left(z_{2}\right)\hat{P}_{1}\left(w_{1}\right)\right)\right)F_{2}\left(z_{1},z_{2}\right)\right)\\
&=&\hat{r}_{2}\hat{\mu}_{1}\hat{\mu}_{2}+\hat{\mu}_{1}\hat{\mu}_{2}\hat{R}_{2}^{(2)}\left(1\right)+
\hat{r}_{2}\frac{\hat{\mu}_{1}\hat{\mu}_{2}}{1-\hat{\mu}_{2}}\hat{F}_{2}^{(0,1)}+
\hat{r}_{2}\hat{\mu}_{2}\hat{F}_{2}^{(1,0)}+\frac{\hat{\mu}_{2}}{1-\hat{\mu}_{2}}\hat{F}_{2}^{(1,1)}.
\end{eqnarray*}

%16/48
\item \begin{eqnarray*} &&\frac{\partial}{\partial
w_2}\frac{\partial}{\partial
w_2}\left(\hat{R}_{2}\left(P_{1}\left(z_{1}\right)\tilde{P}_{2}\left(z_{2}\right)\hat{P}_{1}\left(w_{1}\right)\hat{P}_{2}\left(w_{2}\right)\right)\hat{F}_{2}\left(w_{1},\hat{\theta}_{2}\left(P_{1}\left(z_{1}\right)\tilde{P}_{2}\left(z_{2}\right)\hat{P}_{1}\left(w_{1}\right)\right)\right)F_{2}\left(z_{1},z_{2};\zeta_{2}\right)\right)\\
&=&\hat{r}_{2}P_{2}^{(2)}\left(1\right)+\hat{\mu}_{2}^{2}\hat{R}_{2}^{(2)}\left(1\right)+2\hat{r}_{2}\frac{\hat{\mu}_{2}^{2}}{1-\hat{\mu}_{2}}\hat{F}_{2}^{(0,1)}+\frac{1}{1-\hat{\mu}_{2}}\hat{P}_{2}^{(2)}\left(1\right)\hat{F}_{2}^{(0,1)}+\hat{\mu}_{2}^{2}\hat{\theta}_{2}^{(2)}\left(1\right)\hat{F}_{2}^{(0,1)}\\
&+&\left(\frac{\hat{\mu}_{2}}{1-\hat{\mu}_{2}}\right)^{2}\hat{F}_{2}^{(0,2)}.
\end{eqnarray*}


\end{enumerate}



%___________________________________________________________________________________________
%
%\subsection{Derivadas de Segundo Orden para $\hat{F}_{2}$}
%___________________________________________________________________________________________
\begin{enumerate}
%___________________________________________________________________________________________
%\subsubsection{Mixtas para $z_{1}$:}
%___________________________________________________________________________________________
%1/49

\item \begin{eqnarray*} &&\frac{\partial}{\partial
z_1}\frac{\partial}{\partial
z_1}\left(\hat{R}_{1}\left(P_{1}\left(z_{1}\right)\tilde{P}_{2}\left(z_{2}\right)\hat{P}_{1}\left(w_{1}\right)\hat{P}_{2}\left(w_{2}\right)\right)\hat{F}_{1}\left(\hat{\theta}_{1}\left(P_{1}\left(z_{1}\right)\tilde{P}_{2}\left(z_{2}\right)
\hat{P}_{2}\left(w_{2}\right)\right),w_{2}\right)F_{1}\left(z_{1},z_{2}\right)\right)\\
&=&\hat{r}_{1}P_{1}^{(2)}\left(1\right)+
\mu_{1}^{2}\hat{R}_{1}^{(2)}\left(1\right)+
2\hat{r}_{1}\mu_{1}F_{1}^{(1,0)}+
2\hat{r}_{1}\frac{\mu_{1}^{2}}{1-\hat{\mu}_{1}}\hat{F}_{1}^{(1,0)}+
\frac{1}{1-\hat{\mu}_{1}}P_{1}^{(2)}\left(1\right)\hat{F}_{1}^{(1,0)}+\mu_{1}^{2}\hat{\theta}_{1}^{(2)}\left(1\right)\hat{F}_{1}^{(1,0)}\\
&+&2\frac{\mu_{1}}{1-\hat{\mu}_{1}}\hat{F}_{1}^{(1,0)}F_{1}^{(1,0)}+F_{1}^{(2,0)}
+\left(\frac{\mu_{1}}{1-\hat{\mu}_{1}}\right)^{2}\hat{F}_{1}^{(2,0)}.
\end{eqnarray*}

%2/50

\item \begin{eqnarray*} &&\frac{\partial}{\partial
z_2}\frac{\partial}{\partial
z_1}\left(\hat{R}_{1}\left(P_{1}\left(z_{1}\right)\tilde{P}_{2}\left(z_{2}\right)\hat{P}_{1}\left(w_{1}\right)\hat{P}_{2}\left(w_{2}\right)\right)\hat{F}_{1}\left(\hat{\theta}_{1}\left(P_{1}\left(z_{1}\right)\tilde{P}_{2}\left(z_{2}\right)
\hat{P}_{2}\left(w_{2}\right)\right),w_{2}\right)F_{1}\left(z_{1},z_{2}\right)\right)\\
&=&\hat{r}_{1}\mu_{1}\tilde{\mu}_{2}+\mu_{1}\tilde{\mu}_{2}\hat{R}_{1}^{(2)}\left(1\right)+
\hat{r}_{1}\mu_{1}F_{1}^{(0,1)}+\tilde{\mu}_{2}\hat{r}_{1}F_{1}^{(1,0)}+
\frac{\mu_{1}\tilde{\mu}_{2}}{1-\hat{\mu}_{1}}\hat{F}_{1}^{(1,0)}+2\hat{r}_{1}\frac{\mu_{1}\tilde{\mu}_{2}}{1-\hat{\mu}_{1}}\hat{F}_{1}^{(1,0)}\\
&+&\mu_{1}\tilde{\mu}_{2}\hat{\theta}_{1}^{(2)}\left(1\right)\hat{F}_{1}^{(1,0)}+
\frac{\mu_{1}}{1-\hat{\mu}_{1}}\hat{F}_{1}^{(1,0)}F_{1}^{(0,1)}+
\frac{\tilde{\mu}_{2}}{1-\hat{\mu}_{1}}\hat{F}_{1}^{(1,0)}F_{1}^{(1,0)}+
F_{1}^{(1,1)}\\
&+&\mu_{1}\tilde{\mu}_{2}\left(\frac{1}{1-\hat{\mu}_{1}}\right)^{2}\hat{F}_{1}^{(2,0)}.
\end{eqnarray*}

%3/51

\item \begin{eqnarray*} &&\frac{\partial}{\partial
w_1}\frac{\partial}{\partial
z_1}\left(\hat{R}_{1}\left(P_{1}\left(z_{1}\right)\tilde{P}_{2}\left(z_{2}\right)\hat{P}_{1}\left(w_{1}\right)\hat{P}_{2}\left(w_{2}\right)\right)\hat{F}_{1}\left(\hat{\theta}_{1}\left(P_{1}\left(z_{1}\right)\tilde{P}_{2}\left(z_{2}\right)
\hat{P}_{2}\left(w_{2}\right)\right),w_{2}\right)F_{1}\left(z_{1},z_{2}\right)\right)\\
&=&\hat{r}_{1}\mu_{1}\hat{\mu}_{1}+\mu_{1}\hat{\mu}_{1}\hat{R}_{1}^{(2)}\left(1\right)+\hat{r}_{1}\hat{\mu}_{1}F_{1}^{(1,0)}+
\hat{r}_{1}\frac{\mu_{1}\hat{\mu}_{1}}{1-\hat{\mu}_{1}}\hat{F}_{1}^{(1,0)}.
\end{eqnarray*}

%4/52

\item \begin{eqnarray*} &&\frac{\partial}{\partial
w_2}\frac{\partial}{\partial
z_1}\left(\hat{R}_{1}\left(P_{1}\left(z_{1}\right)\tilde{P}_{2}\left(z_{2}\right)\hat{P}_{1}\left(w_{1}\right)\hat{P}_{2}\left(w_{2}\right)\right)\hat{F}_{1}\left(\hat{\theta}_{1}\left(P_{1}\left(z_{1}\right)\tilde{P}_{2}\left(z_{2}\right)
\hat{P}_{2}\left(w_{2}\right)\right),w_{2}\right)F_{1}\left(z_{1},z_{2}\right)\right)\\
&=&\hat{r}_{1}\mu_{1}\hat{\mu}_{2}+\mu_{1}\hat{\mu}_{2}\hat{R}_{1}^{(2)}\left(1\right)+\hat{r}_{1}\hat{\mu}_{2}F_{1}^{(1,0)}+\frac{\mu_{1}\hat{\mu}_{2}}{1-\hat{\mu}_{1}}\hat{F}_{1}^{(1,0)}+\hat{r}_{1}\frac{\mu_{1}\hat{\mu}_{2}}{1-\hat{\mu}_{1}}\hat{F}_{1}^{(1,0)}+\mu_{1}\hat{\mu}_{2}\hat{\theta}_{1}^{(2)}\left(1\right)\hat{F}_{1}^{(1,0)}\\
&+&\hat{r}_{1}\mu_{1}\left(\hat{F}_{1}^{(0,1)}+\frac{\hat{\mu}_{2}}{1-\hat{\mu}_{1}}\hat{F}_{1}^{(1,0)}\right)+F_{1}^{(1,0)}\left(\hat{F}_{1}^{(0,1)}+\frac{\hat{\mu}_{2}}{1-\hat{\mu}_{1}}\hat{F}_{1}^{(1,0)}\right)+\frac{\mu_{1}}{1-\hat{\mu}_{1}}\left(\hat{F}_{1}^{(1,1)}+\frac{\hat{\mu}_{2}}{1-\hat{\mu}_{1}}\hat{F}_{1}^{(2,0)}\right).
\end{eqnarray*}
%___________________________________________________________________________________________
%\subsubsection{Mixtas para $z_{2}$:}
%___________________________________________________________________________________________
%5/53

\item \begin{eqnarray*} &&\frac{\partial}{\partial
z_1}\frac{\partial}{\partial
z_2}\left(\hat{R}_{1}\left(P_{1}\left(z_{1}\right)\tilde{P}_{2}\left(z_{2}\right)\hat{P}_{1}\left(w_{1}\right)\hat{P}_{2}\left(w_{2}\right)\right)\hat{F}_{1}\left(\hat{\theta}_{1}\left(P_{1}\left(z_{1}\right)\tilde{P}_{2}\left(z_{2}\right)
\hat{P}_{2}\left(w_{2}\right)\right),w_{2}\right)F_{1}\left(z_{1},z_{2}\right)\right)\\
&=&\hat{r}_{1}\mu_{1}\tilde{\mu}_{2}+\mu_{1}\tilde{\mu}_{2}\hat{R}_{1}^{(2)}\left(1\right)+\hat{r}_{1}\mu_{1}F_{1}^{(0,1)}+\hat{r}_{1}\tilde{\mu}_{2}F_{1}^{(1,0)}+\frac{\mu_{1}\tilde{\mu}_{2}}{1-\hat{\mu}_{1}}\hat{F}_{1}^{(1,0)}+2\hat{r}_{1}\frac{\mu_{1}\tilde{\mu}_{2}}{1-\hat{\mu}_{1}}\hat{F}_{1}^{(1,0)}\\
&+&\mu_{1}\tilde{\mu}_{2}\hat{\theta}_{1}^{(2)}\left(1\right)\hat{F}_{1}^{(1,0)}+\frac{\mu_{1}}{1-\hat{\mu}_{1}}\hat{F}_{1}^{(1,0)}F_{1}^{(0,1)}+\frac{\tilde{\mu}_{2}}{1-\hat{\mu}_{1}}\hat{F}_{1}^{(1,0)}F_{1}^{(1,0)}+F_{1}^{(1,1)}+\mu_{1}\tilde{\mu}_{2}\left(\frac{1}{1-\hat{\mu}_{1}}\right)^{2}\hat{F}_{1}^{(2,0)}.
\end{eqnarray*}

%6/54
\item \begin{eqnarray*} &&\frac{\partial}{\partial
z_2}\frac{\partial}{\partial
z_2}\left(\hat{R}_{1}\left(P_{1}\left(z_{1}\right)\tilde{P}_{2}\left(z_{2}\right)\hat{P}_{1}\left(w_{1}\right)\hat{P}_{2}\left(w_{2}\right)\right)\hat{F}_{1}\left(\hat{\theta}_{1}\left(P_{1}\left(z_{1}\right)\tilde{P}_{2}\left(z_{2}\right)
\hat{P}_{2}\left(w_{2}\right)\right),w_{2}\right)F_{1}\left(z_{1},z_{2}\right)\right)\\
&=&\hat{r}_{1}\tilde{P}_{2}^{(2)}\left(1\right)+\tilde{\mu}_{2}^{2}\hat{R}_{1}^{(2)}\left(1\right)+2\hat{r}_{1}\tilde{\mu}_{2}F_{1}^{(0,1)}+ F_{1}^{(0,2)}+2\hat{r}_{1}\frac{\tilde{\mu}_{2}^{2}}{1-\hat{\mu}_{1}}\hat{F}_{1}^{(1,0)}+\frac{1}{1-\hat{\mu}_{1}}\tilde{P}_{2}^{(2)}\left(1\right)\hat{F}_{1}^{(1,0)}\\
&+&\tilde{\mu}_{2}^{2}\hat{\theta}_{1}^{(2)}\left(1\right)\hat{F}_{1}^{(1,0)}+2\frac{\tilde{\mu}_{2}}{1-\hat{\mu}_{1}}F^{(0,1)}\hat{F}_{1}^{(1,0)}+\left(\frac{\tilde{\mu}_{2}}{1-\hat{\mu}_{1}}\right)^{2}\hat{F}_{1}^{(2,0)}.
\end{eqnarray*}
%7/55

\item \begin{eqnarray*} &&\frac{\partial}{\partial
w_1}\frac{\partial}{\partial
z_2}\left(\hat{R}_{1}\left(P_{1}\left(z_{1}\right)\tilde{P}_{2}\left(z_{2}\right)\hat{P}_{1}\left(w_{1}\right)\hat{P}_{2}\left(w_{2}\right)\right)\hat{F}_{1}\left(\hat{\theta}_{1}\left(P_{1}\left(z_{1}\right)\tilde{P}_{2}\left(z_{2}\right)
\hat{P}_{2}\left(w_{2}\right)\right),w_{2}\right)F_{1}\left(z_{1},z_{2}\right)\right)\\
&=&\hat{r}_{1}\hat{\mu}_{1}\tilde{\mu}_{2}+\hat{\mu}_{1}\tilde{\mu}_{2}\hat{R}_{1}^{(2)}\left(1\right)+
\hat{r}_{1}\hat{\mu}_{1}F_{1}^{(0,1)}+\hat{r}_{1}\frac{\hat{\mu}_{1}\tilde{\mu}_{2}}{1-\hat{\mu}_{1}}\hat{F}_{1}^{(1,0)}.
\end{eqnarray*}
%8/56

\item \begin{eqnarray*} &&\frac{\partial}{\partial
w_2}\frac{\partial}{\partial
z_2}\left(\hat{R}_{1}\left(P_{1}\left(z_{1}\right)\tilde{P}_{2}\left(z_{2}\right)\hat{P}_{1}\left(w_{1}\right)\hat{P}_{2}\left(w_{2}\right)\right)\hat{F}_{1}\left(\hat{\theta}_{1}\left(P_{1}\left(z_{1}\right)\tilde{P}_{2}\left(z_{2}\right)
\hat{P}_{2}\left(w_{2}\right)\right),w_{2}\right)F_{1}\left(z_{1},z_{2}\right)\right)\\
&=&\hat{r}_{1}\tilde{\mu}_{2}\hat{\mu}_{2}+\hat{\mu}_{2}\tilde{\mu}_{2}\hat{R}_{1}^{(2)}\left(1\right)+\hat{\mu}_{2}\hat{R}_{1}^{(2)}\left(1\right)F_{1}^{(0,1)}+\frac{\hat{\mu}_{2}\tilde{\mu}_{2}}{1-\hat{\mu}_{1}}\hat{F}_{1}^{(1,0)}+
\hat{r}_{1}\frac{\hat{\mu}_{2}\tilde{\mu}_{2}}{1-\hat{\mu}_{1}}\hat{F}_{1}^{(1,0)}\\
&+&\hat{\mu}_{2}\tilde{\mu}_{2}\hat{\theta}_{1}^{(2)}\left(1\right)\hat{F}_{1}^{(1,0)}+\hat{r}_{1}\tilde{\mu}_{2}\left(\hat{F}_{1}^{(0,1)}+\frac{\hat{\mu}_{2}}{1-\hat{\mu}_{1}}\hat{F}_{1}^{(1,0)}\right)+F_{1}^{(0,1)}\left(\hat{F}_{1}^{(0,1)}+\frac{\hat{\mu}_{2}}{1-\hat{\mu}_{1}}\hat{F}_{1}^{(1,0)}\right)\\
&+&\frac{\tilde{\mu}_{2}}{1-\hat{\mu}_{1}}\left(\hat{F}_{1}^{(1,1)}+\frac{\hat{\mu}_{2}}{1-\hat{\mu}_{1}}\hat{F}_{1}^{(2,0)}\right).
\end{eqnarray*}
%___________________________________________________________________________________________
%\subsubsection{Mixtas para $w_{1}$:}
%___________________________________________________________________________________________
%9/57
\item \begin{eqnarray*} &&\frac{\partial}{\partial
z_1}\frac{\partial}{\partial
w_1}\left(\hat{R}_{1}\left(P_{1}\left(z_{1}\right)\tilde{P}_{2}\left(z_{2}\right)\hat{P}_{1}\left(w_{1}\right)\hat{P}_{2}\left(w_{2}\right)\right)\hat{F}_{1}\left(\hat{\theta}_{1}\left(P_{1}\left(z_{1}\right)\tilde{P}_{2}\left(z_{2}\right)
\hat{P}_{2}\left(w_{2}\right)\right),w_{2}\right)F_{1}\left(z_{1},z_{2}\right)\right)\\
&=&\hat{r}_{1}\mu_{1}\hat{\mu}_{1}+\mu_{1}\hat{\mu}_{1}\hat{R}_{1}^{(2)}\left(1\right)+\hat{r}_{1}\hat{\mu}_{1}F_{1}^{(1,0)}+\hat{r}_{1}\frac{\mu_{1}\hat{\mu}_{1}}{1-\hat{\mu}_{1}}\hat{F}_{1}^{(1,0)}.
\end{eqnarray*}
%10/58
\item \begin{eqnarray*} &&\frac{\partial}{\partial
z_2}\frac{\partial}{\partial
w_1}\left(\hat{R}_{1}\left(P_{1}\left(z_{1}\right)\tilde{P}_{2}\left(z_{2}\right)\hat{P}_{1}\left(w_{1}\right)\hat{P}_{2}\left(w_{2}\right)\right)\hat{F}_{1}\left(\hat{\theta}_{1}\left(P_{1}\left(z_{1}\right)\tilde{P}_{2}\left(z_{2}\right)
\hat{P}_{2}\left(w_{2}\right)\right),w_{2}\right)F_{1}\left(z_{1},z_{2}\right)\right)\\
&=&\hat{r}_{1}\tilde{\mu}_{2}\hat{\mu}_{1}+\tilde{\mu}_{2}\hat{\mu}_{1}\hat{R}_{1}^{(2)}\left(1\right)+\hat{r}_{1}\hat{\mu}_{1}F_{1}^{(0,1)}+\hat{r}_{1}\frac{\tilde{\mu}_{2}\hat{\mu}_{1}}{1-\hat{\mu}_{1}}\hat{F}_{1}^{(1,0)}.
\end{eqnarray*}
%11/59
\item \begin{eqnarray*} &&\frac{\partial}{\partial
w_1}\frac{\partial}{\partial
w_1}\left(\hat{R}_{1}\left(P_{1}\left(z_{1}\right)\tilde{P}_{2}\left(z_{2}\right)\hat{P}_{1}\left(w_{1}\right)\hat{P}_{2}\left(w_{2}\right)\right)\hat{F}_{1}\left(\hat{\theta}_{1}\left(P_{1}\left(z_{1}\right)\tilde{P}_{2}\left(z_{2}\right)
\hat{P}_{2}\left(w_{2}\right)\right),w_{2}\right)F_{1}\left(z_{1},z_{2}\right)\right)\\
&=&\hat{r}_{1}\hat{P}_{1}^{(2)}\left(1\right)+\hat{\mu}_{1}^{2}\hat{R}_{1}^{(2)}\left(1\right).
\end{eqnarray*}
%12/60
\item \begin{eqnarray*} &&\frac{\partial}{\partial
w_2}\frac{\partial}{\partial
w_1}\left(\hat{R}_{1}\left(P_{1}\left(z_{1}\right)\tilde{P}_{2}\left(z_{2}\right)\hat{P}_{1}\left(w_{1}\right)\hat{P}_{2}\left(w_{2}\right)\right)\hat{F}_{1}\left(\hat{\theta}_{1}\left(P_{1}\left(z_{1}\right)\tilde{P}_{2}\left(z_{2}\right)
\hat{P}_{2}\left(w_{2}\right)\right),w_{2}\right)F_{1}\left(z_{1},z_{2}\right)\right)\\
&=&\hat{r}_{1}\hat{\mu}_{2}\hat{\mu}_{1}+\hat{\mu}_{2}\hat{\mu}_{1}\hat{R}_{1}^{(2)}\left(1\right)+\hat{r}_{1}\hat{\mu}_{1}\left(\hat{F}_{1}^{(0,1)}+\frac{\hat{\mu}_{2}}{1-\hat{\mu}_{1}}\hat{F}_{1}^{(1,0)}\right).
\end{eqnarray*}
%___________________________________________________________________________________________
%\subsubsection{Mixtas para $w_{1}$:}
%___________________________________________________________________________________________
%13/61



\item \begin{eqnarray*} &&\frac{\partial}{\partial
z_1}\frac{\partial}{\partial
w_2}\left(\hat{R}_{1}\left(P_{1}\left(z_{1}\right)\tilde{P}_{2}\left(z_{2}\right)\hat{P}_{1}\left(w_{1}\right)\hat{P}_{2}\left(w_{2}\right)\right)\hat{F}_{1}\left(\hat{\theta}_{1}\left(P_{1}\left(z_{1}\right)\tilde{P}_{2}\left(z_{2}\right)
\hat{P}_{2}\left(w_{2}\right)\right),w_{2}\right)F_{1}\left(z_{1},z_{2}\right)\right)\\
&=&\hat{r}_{1}\mu_{1}\hat{\mu}_{2}+\mu_{1}\hat{\mu}_{2}\hat{R}_{1}^{(2)}\left(1\right)+\hat{r}_{1}\hat{\mu}_{2}F_{1}^{(1,0)}+
\hat{r}_{1}\frac{\mu_{1}\hat{\mu}_{2}}{1-\hat{\mu}_{1}}\hat{F}_{1}^{(1,0)}+\hat{r}_{1}\mu_{1}\left(\hat{F}_{1}^{(0,1)}+\frac{\hat{\mu}_{2}}{1-\hat{\mu}_{1}}\hat{F}_{1}^{(1,0)}\right)\\
&+&F_{1}^{(1,0)}\left(\hat{F}_{1}^{(0,1)}+\frac{\hat{\mu}_{2}}{1-\hat{\mu}_{1}}\hat{F}_{1}^{(1,0)}\right)+\frac{\mu_{1}\hat{\mu}_{2}}{1-\hat{\mu}_{1}}\hat{F}_{1}^{(1,0)}+\mu_{1}\hat{\mu}_{2}\hat{\theta}_{1}^{(2)}\left(1\right)\hat{F}_{1}^{(1,0)}+\frac{\mu_{1}}{1-\hat{\mu}_{1}}\hat{F}_{1}^{(1,1)}\\
&+&\mu_{1}\hat{\mu}_{2}\left(\frac{1}{1-\hat{\mu}_{1}}\right)^{2}\hat{F}_{1}^{(2,0)}.
\end{eqnarray*}

%14/62
\item \begin{eqnarray*} &&\frac{\partial}{\partial
z_2}\frac{\partial}{\partial
w_2}\left(\hat{R}_{1}\left(P_{1}\left(z_{1}\right)\tilde{P}_{2}\left(z_{2}\right)\hat{P}_{1}\left(w_{1}\right)\hat{P}_{2}\left(w_{2}\right)\right)\hat{F}_{1}\left(\hat{\theta}_{1}\left(P_{1}\left(z_{1}\right)\tilde{P}_{2}\left(z_{2}\right)
\hat{P}_{2}\left(w_{2}\right)\right),w_{2}\right)F_{1}\left(z_{1},z_{2}\right)\right)\\
&=&\hat{r}_{1}\tilde{\mu}_{2}\hat{\mu}_{2}+\tilde{\mu}_{2}\hat{\mu}_{2}\hat{R}_{1}^{(2)}\left(1\right)+\hat{r}_{1}\hat{\mu}_{2}F_{1}^{(0,1)}+\hat{r}_{1}\frac{\tilde{\mu}_{2}\hat{\mu}_{2}}{1-\hat{\mu}_{1}}\hat{F}_{1}^{(1,0)}+\hat{r}_{1}\tilde{\mu}_{2}\left(\hat{F}_{1}^{(0,1)}+\frac{\hat{\mu}_{2}}{1-\hat{\mu}_{1}}\hat{F}_{1}^{(1,0)}\right)\\
&+&F_{1}^{(0,1)}\left(\hat{F}_{1}^{(0,1)}+\frac{\hat{\mu}_{2}}{1-\hat{\mu}_{1}}\hat{F}_{1}^{(1,0)}\right)+\frac{\tilde{\mu}_{2}\hat{\mu}_{2}}{1-\hat{\mu}_{1}}\hat{F}_{1}^{(1,0)}+\tilde{\mu}_{2}\hat{\mu}_{2}\hat{\theta}_{1}^{(2)}\left(1\right)\hat{F}_{1}^{(1,0)}+\frac{\tilde{\mu}_{2}}{1-\hat{\mu}_{1}}\hat{F}_{1}^{(1,1)}\\
&+&\tilde{\mu}_{2}\hat{\mu}_{2}\left(\frac{1}{1-\hat{\mu}_{1}}\right)^{2}\hat{F}_{1}^{(2,0)}.
\end{eqnarray*}

%15/63

\item \begin{eqnarray*} &&\frac{\partial}{\partial
w_1}\frac{\partial}{\partial
w_2}\left(\hat{R}_{1}\left(P_{1}\left(z_{1}\right)\tilde{P}_{2}\left(z_{2}\right)\hat{P}_{1}\left(w_{1}\right)\hat{P}_{2}\left(w_{2}\right)\right)\hat{F}_{1}\left(\hat{\theta}_{1}\left(P_{1}\left(z_{1}\right)\tilde{P}_{2}\left(z_{2}\right)
\hat{P}_{2}\left(w_{2}\right)\right),w_{2}\right)F_{1}\left(z_{1},z_{2}\right)\right)\\
&=&\hat{r}_{1}\hat{\mu}_{2}\hat{\mu}_{1}+\hat{\mu}_{2}\hat{\mu}_{1}\hat{R}_{1}^{(2)}\left(1\right)+\hat{r}_{1}\hat{\mu}_{1}\left(\hat{F}_{1}^{(0,1)}+\frac{\hat{\mu}_{2}}{1-\hat{\mu}_{1}}\hat{F}_{1}^{(1,0)}\right).
\end{eqnarray*}

%16/64

\item \begin{eqnarray*} &&\frac{\partial}{\partial
w_2}\frac{\partial}{\partial
w_2}\left(\hat{R}_{1}\left(P_{1}\left(z_{1}\right)\tilde{P}_{2}\left(z_{2}\right)\hat{P}_{1}\left(w_{1}\right)\hat{P}_{2}\left(w_{2}\right)\right)\hat{F}_{1}\left(\hat{\theta}_{1}\left(P_{1}\left(z_{1}\right)\tilde{P}_{2}\left(z_{2}\right)
\hat{P}_{2}\left(w_{2}\right)\right),w_{2}\right)F_{1}\left(z_{1},z_{2}\right)\right)\\
&=&\hat{r}_{1}\hat{P}_{2}^{(2)}\left(1\right)+\hat{\mu}_{2}^{2}\hat{R}_{1}^{(2)}\left(1\right)+
2\hat{r}_{1}\hat{\mu}_{2}\left(\hat{F}_{1}^{(0,1)}+\frac{\hat{\mu}_{2}}{1-\hat{\mu}_{1}}\hat{F}_{1}^{(1,0)}\right)+
\hat{F}_{1}^{(0,2)}+\frac{1}{1-\hat{\mu}_{1}}\hat{P}_{2}^{(2)}\left(1\right)\hat{F}_{1}^{(1,0)}\\
&+&\hat{\mu}_{2}^{2}\hat{\theta}_{1}^{(2)}\left(1\right)\hat{F}_{1}^{(1,0)}+\frac{\hat{\mu}_{2}}{1-\hat{\mu}_{1}}\hat{F}_{1}^{(1,1)}+\frac{\hat{\mu}_{2}}{1-\hat{\mu}_{1}}\left(\hat{F}_{1}^{(1,1)}+\frac{\hat{\mu}_{2}}{1-\hat{\mu}_{1}}\hat{F}_{1}^{(2,0)}\right).
\end{eqnarray*}
%_________________________________________________________________________________________________________
%
%_________________________________________________________________________________________________________

\end{enumerate}




Las ecuaciones que determinan los segundos momentos de las longitudes de las colas de los dos sistemas se pueden ver en \href{http://sitio.expresauacm.org/s/carlosmartinez/wp-content/uploads/sites/13/2014/01/SegundosMomentos.pdf}{este sitio}

%\url{http://ubuntu_es_el_diablo.org},\href{http://www.latex-project.org/}{latex project}

%http://sitio.expresauacm.org/s/carlosmartinez/wp-content/uploads/sites/13/2014/01/SegundosMomentos.jpg
%http://sitio.expresauacm.org/s/carlosmartinez/wp-content/uploads/sites/13/2014/01/SegundosMomentos.pdf




%_____________________________________________________________________________________
%Distribuci\'on del n\'umero de usuaruios que pasan del sistema 1 al sistema 2
%_____________________________________________________________________________________
\section*{Ap\'endice B}
%________________________________________________________________________________________
%
%________________________________________________________________________________________
\subsection*{Distribuci\'on para los usuarios de traslado}
%________________________________________________________________________________________
Se puede demostrar que
\begin{equation}
\frac{d^{k}}{dy}\left(\frac{\lambda +\mu}{\lambda
+\mu-y}\right)=\frac{k!}{\left(\lambda+\mu\right)^{k}}
\end{equation}



\begin{Prop}
Sea $\tau$ variable aleatoria no negativa con distribuci\'on exponencial con media $\mu$, y sea $L\left(t\right)$ proceso
Poisson con par\'ametro $\lambda$. Entonces
\begin{equation}
\prob\left\{L\left(\tau\right)=k\right\}=f_{L\left(\tau\right)}\left(k\right)=\left(\frac{\mu}{\lambda
+\mu}\right)\left(\frac{\lambda}{\lambda+\mu}\right)^{k}.
\end{equation}
Adem\'as

\begin{eqnarray}
\esp\left[L\left(\tau\right)\right]&=&\frac{\lambda}{\mu}\\
\esp\left[\left(L\left(\tau\right)\right)^{2}\right]&=&\frac{\lambda}{\mu}\left(2\frac{\lambda}{\mu}+1\right)\\
V\left[L\left(\tau\right)\right]&=&\frac{\lambda}{\mu}\left(\frac{\lambda}{\mu}+1\right).
\end{eqnarray}
\end{Prop}

\begin{Proof}
A saber, para $k$ fijo se tiene que

\begin{eqnarray*}
\prob\left\{L\left(\tau\right)=k\right\}&=&\prob\left\{L\left(\tau\right)=k,\tau\in\left(0,\infty\right)\right\}\\
&=&\int_{0}^{\infty}\prob\left\{L\left(\tau\right)=k,\tau=y\right\}f_{\tau}\left(y\right)dy=\int_{0}^{\infty}\prob\left\{L\left(y\right)=k\right\}f_{\tau}\left(y\right)dy\\
&=&\int_{0}^{\infty}\frac{e^{-\lambda
y}}{k!}\left(\lambda y\right)^{k}\left(\mu e^{-\mu
y}\right)dy=\frac{\lambda^{k}\mu}{k!}\int_{0}^{\infty}y^{k}e^{-\left(\mu+\lambda\right)y}dy\\
&=&\frac{\lambda^{k}\mu}{\left(\lambda
+\mu\right)k!}\int_{0}^{\infty}y^{k}\left(\lambda+\mu\right)e^{-\left(\lambda+\mu\right)y}dy=\frac{\lambda^{k}\mu}{\left(\lambda
+\mu\right)k!}\int_{0}^{\infty}y^{k}f_{Y}\left(y\right)dy\\
&=&\frac{\lambda^{k}\mu}{\left(\lambda
+\mu\right)k!}\esp\left[Y^{k}\right]=\frac{\lambda^{k}\mu}{\left(\lambda
+\mu\right)k!}\frac{d^{k}}{dy}\left(\frac{\lambda
+\mu}{\lambda
+\mu-y}\right)|_{y=0}\\
&=&\frac{\lambda^{k}\mu}{\left(\lambda
+\mu\right)k!}\frac{k!}{\left(\lambda+\mu\right)^{k}}=\left(\frac{\mu}{\lambda
+\mu}\right)\left(\frac{\lambda}{\lambda+\mu}\right)^{k}.\\
\end{eqnarray*}


Adem\'as
\begin{eqnarray*}
\sum_{k=0}^{\infty}\prob\left\{L\left(\tau\right)=k\right\}&=&\sum_{k=0}^{\infty}\left(\frac{\mu}{\lambda
+\mu}\right)\left(\frac{\lambda}{\lambda+\mu}\right)^{k}=\frac{\mu}{\lambda
+\mu}\sum_{k=0}^{\infty}\left(\frac{\lambda}{\lambda+\mu}\right)^{k}\\
&=&\frac{\mu}{\lambda
+\mu}\left(\frac{1}{1-\frac{\lambda}{\lambda+\mu}}\right)=\frac{\mu}{\lambda
+\mu}\left(\frac{\lambda+\mu}{\mu}\right)\\
&=&1.\\
\end{eqnarray*}

determinemos primero la esperanza de
$L\left(\tau\right)$:


\begin{eqnarray*}
\esp\left[L\left(\tau\right)\right]&=&\sum_{k=0}^{\infty}k\prob\left\{L\left(\tau\right)=k\right\}=\sum_{k=0}^{\infty}k\left(\frac{\mu}{\lambda
+\mu}\right)\left(\frac{\lambda}{\lambda+\mu}\right)^{k}\\
&=&\left(\frac{\mu}{\lambda
+\mu}\right)\sum_{k=0}^{\infty}k\left(\frac{\lambda}{\lambda+\mu}\right)^{k}=\left(\frac{\mu}{\lambda
+\mu}\right)\left(\frac{\lambda}{\lambda+\mu}\right)\sum_{k=1}^{\infty}k\left(\frac{\lambda}{\lambda+\mu}\right)^{k-1}\\
&=&\frac{\mu\lambda}{\left(\lambda
+\mu\right)^{2}}\left(\frac{1}{1-\frac{\lambda}{\lambda+\mu}}\right)^{2}=\frac{\mu\lambda}{\left(\lambda
+\mu\right)^{2}}\left(\frac{\lambda+\mu}{\mu}\right)^{2}\\
&=&\frac{\lambda}{\mu}.
\end{eqnarray*}

Ahora su segundo momento:

\begin{eqnarray*}
\esp\left[\left(L\left(\tau\right)\right)^{2}\right]&=&\sum_{k=0}^{\infty}k^{2}\prob\left\{L\left(\tau\right)=k\right\}=\sum_{k=0}^{\infty}k^{2}\left(\frac{\mu}{\lambda
+\mu}\right)\left(\frac{\lambda}{\lambda+\mu}\right)^{k}\\
&=&\left(\frac{\mu}{\lambda
+\mu}\right)\sum_{k=0}^{\infty}k^{2}\left(\frac{\lambda}{\lambda+\mu}\right)^{k}=
\frac{\mu\lambda}{\left(\lambda
+\mu\right)^{2}}\sum_{k=2}^{\infty}\left(k-1\right)^{2}\left(\frac{\lambda}{\lambda+\mu}\right)^{k-2}\\
&=&\frac{\mu\lambda}{\left(\lambda
+\mu\right)^{2}}\left(\frac{\frac{\lambda}{\lambda+\mu}+1}{\left(\frac{\lambda}{\lambda+\mu}-1\right)^{3}}\right)=\frac{\mu\lambda}{\left(\lambda
+\mu\right)^{2}}\left(-\frac{\frac{2\lambda+\mu}{\lambda+\mu}}{\left(-\frac{\mu}{\lambda+\mu}\right)^{3}}\right)\\
&=&\frac{\mu\lambda}{\left(\lambda
+\mu\right)^{2}}\left(\frac{2\lambda+\mu}{\lambda+\mu}\right)\left(\frac{\lambda+\mu}{\mu}\right)^{3}=\frac{\lambda\left(2\lambda
+\mu\right)}{\mu^{2}}\\
&=&\frac{\lambda}{\mu}\left(2\frac{\lambda}{\mu}+1\right).
\end{eqnarray*}

y por tanto

\begin{eqnarray*}
V\left[L\left(\tau\right)\right]&=&\frac{\lambda\left(2\lambda
+\mu\right)}{\mu^{2}}-\left(\frac{\lambda}{\mu}\right)^{2}=\frac{\lambda^{2}+\mu\lambda}{\mu^{2}}\\
&=&\frac{\lambda}{\mu}\left(\frac{\lambda}{\mu}+1\right).
\end{eqnarray*}
\end{Proof}

Ahora, determinemos la distribuci\'on del n\'umero de usuarios que
pasan de $\hat{Q}_{2}$ a $Q_{2}$ considerando dos pol\'iticas de
traslado en espec\'ifico:

\begin{enumerate}
\item Solamente pasa un usuario,

\item Se permite el paso de $k$ usuarios,
\end{enumerate}
una vez que son atendidos por el servidor en $\hat{Q}_{2}$.

\begin{description}


\item[Pol\'itica de un solo usuario:] Sea $R_{2}$ el n\'umero de
usuarios que llegan a $\hat{Q}_{2}$ al tiempo $t$, sea $R_{1}$ el
n\'umero de usuarios que pasan de $\hat{Q}_{2}$ a $Q_{2}$ al
tiempo $t$.
\end{description}


A saber:
\begin{eqnarray*}
\esp\left[R_{1}\right]&=&\sum_{y\geq0}\prob\left[R_{2}=y\right]\esp\left[R_{1}|R_{2}=y\right]\\
&=&\sum_{y\geq0}\prob\left[R_{2}=y\right]\sum_{x\geq0}x\prob\left[R_{1}=x|R_{2}=y\right]\\
&=&\sum_{y\geq0}\sum_{x\geq0}x\prob\left[R_{1}=x|R_{2}=y\right]\prob\left[R_{2}=y\right].\\
\end{eqnarray*}

Determinemos
\begin{equation}
\esp\left[R_{1}|R_{2}=y\right]=\sum_{x\geq0}x\prob\left[R_{1}=x|R_{2}=y\right].
\end{equation}

supongamos que $y=0$, entonces
\begin{eqnarray*}
\prob\left[R_{1}=0|R_{2}=0\right]&=&1,\\
\prob\left[R_{1}=x|R_{2}=0\right]&=&0,\textrm{ para cualquier }x\geq1,\\
\end{eqnarray*}


por tanto
\begin{eqnarray*}
\esp\left[R_{1}|R_{2}=0\right]=0.
\end{eqnarray*}

Para $y=1$,
\begin{eqnarray*}
\prob\left[R_{1}=0|R_{2}=1\right]&=&0,\\
\prob\left[R_{1}=1|R_{2}=1\right]&=&1,
\end{eqnarray*}

entonces
\begin{eqnarray*}
\esp\left[R_{1}|R_{2}=1\right]=1.
\end{eqnarray*}

Para $y>1$:
\begin{eqnarray*}
\prob\left[R_{1}=0|R_{2}\geq1\right]&=&0,\\
\prob\left[R_{1}=1|R_{2}\geq1\right]&=&1,\\
\prob\left[R_{1}>1|R_{2}\geq1\right]&=&0,
\end{eqnarray*}

entonces
\begin{eqnarray*}
\esp\left[R_{1}|R_{2}=y\right]=1,\textrm{ para cualquier }y>1.
\end{eqnarray*}
es decir
\begin{eqnarray*}
\esp\left[R_{1}|R_{2}=y\right]=1,\textrm{ para cualquier }y\geq1.
\end{eqnarray*}

Entonces
\begin{eqnarray*}
\esp\left[R_{1}\right]&=&\sum_{y\geq0}\sum_{x\geq0}x\prob\left[R_{1}=x|R_{2}=y\right]\prob\left[R_{2}=y\right]=\sum_{y\geq0}\sum_{x}\esp\left[R_{1}|R_{2}=y\right]\prob\left[R_{2}=y\right]\\
&=&\sum_{y\geq0}\prob\left[R_{2}=y\right]=\sum_{y\geq1}\frac{\left(\lambda
t\right)^{k}}{k!}e^{-\lambda t}=1.
\end{eqnarray*}

Adem\'as para $k\in Z^{+}$
\begin{eqnarray*}
f_{R_{1}}\left(k\right)&=&\prob\left[R_{1}=k\right]=\sum_{n=0}^{\infty}\prob\left[R_{1}=k|R_{2}=n\right]\prob\left[R_{2}=n\right]\\
&=&\prob\left[R_{1}=k|R_{2}=0\right]\prob\left[R_{2}=0\right]+\prob\left[R_{1}=k|R_{2}=1\right]\prob\left[R_{2}=1\right]\\
&+&\prob\left[R_{1}=k|R_{2}>1\right]\prob\left[R_{2}>1\right],
\end{eqnarray*}

donde para


\begin{description}
\item[$k=0$:]
\begin{eqnarray*}
\prob\left[R_{1}=0\right]=\prob\left[R_{1}=0|R_{2}=0\right]\prob\left[R_{2}=0\right]+\prob\left[R_{1}=0|R_{2}=1\right]\prob\left[R_{2}=1\right]\\
+\prob\left[R_{1}=0|R_{2}>1\right]\prob\left[R_{2}>1\right]=\prob\left[R_{2}=0\right].
\end{eqnarray*}
\item[$k=1$:]
\begin{eqnarray*}
\prob\left[R_{1}=1\right]=\prob\left[R_{1}=1|R_{2}=0\right]\prob\left[R_{2}=0\right]+\prob\left[R_{1}=1|R_{2}=1\right]\prob\left[R_{2}=1\right]\\
+\prob\left[R_{1}=1|R_{2}>1\right]\prob\left[R_{2}>1\right]=\sum_{n=1}^{\infty}\prob\left[R_{2}=n\right].
\end{eqnarray*}

\item[$k=2$:]
\begin{eqnarray*}
\prob\left[R_{1}=2\right]=\prob\left[R_{1}=2|R_{2}=0\right]\prob\left[R_{2}=0\right]+\prob\left[R_{1}=2|R_{2}=1\right]\prob\left[R_{2}=1\right]\\
+\prob\left[R_{1}=2|R_{2}>1\right]\prob\left[R_{2}>1\right]=0.
\end{eqnarray*}

\item[$k=j$:]
\begin{eqnarray*}
\prob\left[R_{1}=j\right]=\prob\left[R_{1}=j|R_{2}=0\right]\prob\left[R_{2}=0\right]+\prob\left[R_{1}=j|R_{2}=1\right]\prob\left[R_{2}=1\right]\\
+\prob\left[R_{1}=j|R_{2}>1\right]\prob\left[R_{2}>1\right]=0.
\end{eqnarray*}
\end{description}


Por lo tanto
\begin{eqnarray*}
f_{R_{1}}\left(0\right)&=&\prob\left[R_{2}=0\right]\\
f_{R_{1}}\left(1\right)&=&\sum_{n\geq1}^{\infty}\prob\left[R_{2}=n\right]\\
f_{R_{1}}\left(j\right)&=&0,\textrm{ para }j>1.
\end{eqnarray*}



\begin{description}
\item[Pol\'itica de $k$ usuarios:]Al igual que antes, para $y\in Z^{+}$ fijo
\begin{eqnarray*}
\esp\left[R_{1}|R_{2}=y\right]=\sum_{x}x\prob\left[R_{1}=x|R_{2}=y\right].\\
\end{eqnarray*}
\end{description}
Entonces, si tomamos diversos valore para $y$:\\

$y=0$:
\begin{eqnarray*}
\prob\left[R_{1}=0|R_{2}=0\right]&=&1,\\
\prob\left[R_{1}=x|R_{2}=0\right]&=&0,\textrm{ para cualquier }x\geq1,
\end{eqnarray*}

entonces
\begin{eqnarray*}
\esp\left[R_{1}|R_{2}=0\right]=0.
\end{eqnarray*}


Para $y=1$,
\begin{eqnarray*}
\prob\left[R_{1}=0|R_{2}=1\right]&=&0,\\
\prob\left[R_{1}=1|R_{2}=1\right]&=&1,
\end{eqnarray*}

entonces {\scriptsize{
\begin{eqnarray*}
\esp\left[R_{1}|R_{2}=1\right]=1.
\end{eqnarray*}}}


Para $y=2$,
\begin{eqnarray*}
\prob\left[R_{1}=0|R_{2}=2\right]&=&0,\\
\prob\left[R_{1}=1|R_{2}=2\right]&=&1,\\
\prob\left[R_{1}=2|R_{2}=2\right]&=&1,\\
\prob\left[R_{1}=3|R_{2}=2\right]&=&0,
\end{eqnarray*}

entonces
\begin{eqnarray*}
\esp\left[R_{1}|R_{2}=2\right]=3.
\end{eqnarray*}

Para $y=3$,
\begin{eqnarray*}
\prob\left[R_{1}=0|R_{2}=3\right]&=&0,\\
\prob\left[R_{1}=1|R_{2}=3\right]&=&1,\\
\prob\left[R_{1}=2|R_{2}=3\right]&=&1,\\
\prob\left[R_{1}=3|R_{2}=3\right]&=&1,\\
\prob\left[R_{1}=4|R_{2}=3\right]&=&0,
\end{eqnarray*}

entonces
\begin{eqnarray*}
\esp\left[R_{1}|R_{2}=3\right]=6.
\end{eqnarray*}

En general, para $k\geq0$,
\begin{eqnarray*}
\prob\left[R_{1}=0|R_{2}=k\right]&=&0,\\
\prob\left[R_{1}=j|R_{2}=k\right]&=&1,\textrm{ para }1\leq j\leq k,\\
\prob\left[R_{1}=j|R_{2}=k\right]&=&0,\textrm{ para }j> k,
\end{eqnarray*}

entonces
\begin{eqnarray*}
\esp\left[R_{1}|R_{2}=k\right]=\frac{k\left(k+1\right)}{2}.
\end{eqnarray*}



Por lo tanto


\begin{eqnarray*}
\esp\left[R_{1}\right]&=&\sum_{y}\esp\left[R_{1}|R_{2}=y\right]\prob\left[R_{2}=y\right]\\
&=&\sum_{y}\prob\left[R_{2}=y\right]\frac{y\left(y+1\right)}{2}=\sum_{y\geq1}\left(\frac{y\left(y+1\right)}{2}\right)\frac{\left(\lambda t\right)^{y}}{y!}e^{-\lambda t}\\
&=&\frac{\lambda t}{2}e^{-\lambda t}\sum_{y\geq1}\left(y+1\right)\frac{\left(\lambda t\right)^{y-1}}{\left(y-1\right)!}=\frac{\lambda t}{2}e^{-\lambda t}\left(e^{\lambda t}\left(\lambda t+2\right)\right)\\
&=&\frac{\lambda t\left(\lambda t+2\right)}{2},
\end{eqnarray*}
es decir,


\begin{equation}
\esp\left[R_{1}\right]=\frac{\lambda t\left(\lambda
t+2\right)}{2}.
\end{equation}

Adem\'as para $k\in Z^{+}$ fijo
\begin{eqnarray*}
f_{R_{1}}\left(k\right)&=&\prob\left[R_{1}=k\right]=\sum_{n=0}^{\infty}\prob\left[R_{1}=k|R_{2}=n\right]\prob\left[R_{2}=n\right]\\
&=&\prob\left[R_{1}=k|R_{2}=0\right]\prob\left[R_{2}=0\right]+\prob\left[R_{1}=k|R_{2}=1\right]\prob\left[R_{2}=1\right]\\
&+&\prob\left[R_{1}=k|R_{2}=2\right]\prob\left[R_{2}=2\right]+\cdots+\prob\left[R_{1}=k|R_{2}=j\right]\prob\left[R_{2}=j\right]+\cdots+
\end{eqnarray*}
donde para

\begin{description}
\item[$k=0$:]
\begin{eqnarray*}
\prob\left[R_{1}=0\right]=\prob\left[R_{1}=0|R_{2}=0\right]\prob\left[R_{2}=0\right]+\prob\left[R_{1}=0|R_{2}=1\right]\prob\left[R_{2}=1\right]\\
+\prob\left[R_{1}=0|R_{2}=j\right]\prob\left[R_{2}=j\right]=\prob\left[R_{2}=0\right].
\end{eqnarray*}
\item[$k=1$:]
\begin{eqnarray*}
\prob\left[R_{1}=1\right]=\prob\left[R_{1}=1|R_{2}=0\right]\prob\left[R_{2}=0\right]+\prob\left[R_{1}=1|R_{2}=1\right]\prob\left[R_{2}=1\right]\\
+\prob\left[R_{1}=1|R_{2}=1\right]\prob\left[R_{2}=1\right]+\cdots+\prob\left[R_{1}=1|R_{2}=j\right]\prob\left[R_{2}=j\right]\\
=\sum_{n=1}^{\infty}\prob\left[R_{2}=n\right].
\end{eqnarray*}

\item[$k=2$:]
\begin{eqnarray*}
\prob\left[R_{1}=2\right]=\prob\left[R_{1}=2|R_{2}=0\right]\prob\left[R_{2}=0\right]+\prob\left[R_{1}=2|R_{2}=1\right]\prob\left[R_{2}=1\right]\\
+\prob\left[R_{1}=2|R_{2}=2\right]\prob\left[R_{2}=2\right]+\cdots+\prob\left[R_{1}=2|R_{2}=j\right]\prob\left[R_{2}=j\right]\\
=\sum_{n=2}^{\infty}\prob\left[R_{2}=n\right].
\end{eqnarray*}
\end{description}

En general

\begin{eqnarray*}
\prob\left[R_{1}=k\right]=\prob\left[R_{1}=k|R_{2}=0\right]\prob\left[R_{2}=0\right]+\prob\left[R_{1}=k|R_{2}=1\right]\prob\left[R_{2}=1\right]\\
+\prob\left[R_{1}=k|R_{2}=2\right]\prob\left[R_{2}=2\right]+\cdots+\prob\left[R_{1}=k|R_{2}=k\right]\prob\left[R_{2}=k\right]\\
=\sum_{n=k}^{\infty}\prob\left[R_{2}=n\right].\\
\end{eqnarray*}



Por lo tanto

\begin{eqnarray*}
f_{R_{1}}\left(k\right)&=&\prob\left[R_{1}=k\right]=\sum_{n=k}^{\infty}\prob\left[R_{2}=n\right].
\end{eqnarray*}








\section*{Objetivos Principales}

\begin{itemize}
%\item Generalizar los principales resultados existentes para Sistemas de Visitas C\'iclicas para el caso en el que se tienen dos Sistemas de Visitas C\'iclicas con propiedades similares.

\item Encontrar las ecuaciones que modelan el comportamiento de una Red de Sistemas de Visitas C\'iclicas (RSVC) con propiedades similares.

\item Encontrar expresiones anal\'iticas para las longitudes de las colas al momento en que el servidor llega a una de ellas para comenzar a dar servicio, as\'i como de sus segundos momentos.

\item Determinar las principales medidas de Desempe\~no para la RSVC tales como: N\'umero de usuarios presentes en cada una de las colas del sistema cuando uno de los servidores est\'a presente atendiendo, Tiempos que transcurre entre las visitas del servidor a la misma cola.


\end{itemize}


%_________________________________________________________________________
%\section{Sistemas de Visitas C\'iclicas}
%_________________________________________________________________________
\numberwithin{equation}{section}%
%__________________________________________________________________________




%\section*{Introducci\'on}




%__________________________________________________________________________
%\subsection{Definiciones}
%__________________________________________________________________________


\section{Descripci\'on de una Red de Sistemas de Visitas C\'iclicas}

Consideremos una red de sistema de visitas c\'iclicas conformada por dos sistemas de visitas c\'iclicas, cada una con dos colas independientes, donde adem\'as se permite el intercambio de usuarios entre los dos sistemas en la segunda cola de cada uno de ellos.

%____________________________________________________________________
\subsection*{Supuestos sobe la Red de Sistemas de Visitas C\'iclicas}
%____________________________________________________________________

\begin{itemize}
\item Los arribos de los usuarios ocurren
conforme a un proceso Poisson con tasa de llegada $\mu_{1}$ y
$\mu_{2}$ para el sistema 1, mientras que para el sistema 2,
lo hacen conforme a un proceso Poisson con tasa
$\hat{\mu}_{1},\hat{\mu}_{2}$ respectivamente.



\item Se considerar\'an intervalos de tiempo de la forma
$\left[t,t+1\right]$. Los usuarios arriban por paquetes de manera
independiente del resto de las colas. Se define el grupo de
usuarios que llegan a cada una de las colas del sistema 1,
caracterizadas por $Q_{1}$ y $Q_{2}$ respectivamente, en el
intervalo de tiempo $\left[t,t+1\right]$ por
$X_{1}\left(t\right),X_{2}\left(t\right)$.


\item Se definen los procesos
$\hat{X}_{1}\left(t\right),\hat{X}_{2}\left(t\right)$ para las
colas del sistema 2, denotadas por $\hat{Q}_{1}$ y $\hat{Q}_{2}$
respectivamente. Donde adem\'as se supone que $\mu_{i}<1$ y $\hat{\mu}<1$ para $i=1,2$.


\item Se define el proceso
$Y_{2}\left(t\right)$ para el n\'umero de usuarios que se trasladan del sistema 2 al sistema 1, de la cola $\hat{Q}_{2}$ a la cola
$Q_{2}$, en el intervalo de tiempo $\left[t,t+1\right]$. El traslado de un sistema a otro ocurre de manera que los tiempos entre llegadas de los usuarios a la cola dos del sistema 1 provenientes del sistema 2, se distribuye de manera general con par\'ametro $\check{\mu}_{2}$, con $\check{\mu}_{2}<1$.



\item En lo que respecta al servidor, en t\'erminos de los tiempos de
visita a cada una de las colas, se definen las variables
aleatorias $\tau_{i},$ para $Q_{i}$, para $i=1,2$, respectivamente;
y $\zeta_{i}$ para $\hat{Q}_{i}$,  $i=1,2$,  del sistema
2 respectivamente. A los tiempos en que el servidor termina de atender en las colas $Q_{i},\hat{Q}_{i}$,se les denotar\'a por
$\overline{\tau}_{i},\overline{\zeta}_{i}$ para  $i=1,2$,
respectivamente.

\item Los tiempos de traslado del servidor desde el momento en que termina de atender a una cola y llega a la siguiente para comenzar a dar servicio est\'an dados por
$\tau_{i+1}-\overline{\tau}_{i}$ y
$\zeta_{i+1}-\overline{\zeta}_{i}$,  $i=1,2$, para el sistema 1 y el sistema 2, respectivamente.

\end{itemize}




%\begin{figure}[H]
%\centering
%%%\includegraphics[width=5cm]{RedSistemasVisitasCiclicas.jpg}
%%\end{figure}\label{RSVC}

El uso de la Funci\'on Generadora de Probabilidades (FGP's) nos permite determinar las Funciones de Distribuci\'on de Probabilidades Conjunta de manera indirecta sin necesidad de hacer uso de las propiedades de las distribuciones de probabilidad de cada uno de los procesos que intervienen en la Red de Sistemas de Visitas C\'iclicas.\smallskip

Cada uno de estos procesos con su respectiva FGP. Adem\'as, para cada una de las colas en cada sistema, el n\'umero de usuarios al tiempo en que llega el servidor a dar servicio est\'a
dado por el n\'umero de usuarios presentes en la cola al tiempo
$t$, m\'as el n\'umero de usuarios que llegan a la cola en el intervalo de tiempo
$\left[\tau_{i},\overline{\tau}_{i}\right]$.




Una vez definidas las Funciones Generadoras de Probabilidades Conjuntas se construyen las ecuaciones recursivas que permiten obtener la informaci\'on sobre la longitud de cada una de las colas, al momento en que uno de los servidores llega a una de las colas para dar servicio, bas\'andose en la informaci\'on que se tiene sobre su llegada a la cola inmediata anterior.\smallskip

%__________________________________________________________________________
\subsection{Funciones Generadoras de Probabilidades}
%__________________________________________________________________________


Para cada uno de los procesos de llegada a las colas $X_{i},\hat{X}_{i}$,  $i=1,2$,  y $Y_{2}$, con $\tilde{X}_{2}=X_{2}+Y_{2}$ anteriores se define su Funci\'on
Generadora de Probabilidades (FGP): $P_{i}\left(z_{i}\right)=\esp\left[z_{i}^{X_{i}\left(t\right)}\right],\hat{P}_{i}\left(w_{i}\right)=\esp\left[w_{i}^{\hat{X}_{i}\left(t\right)}\right]$, para
$i=1,2$, y $\check{P}_{2}\left(z_{2}\right)=\esp\left[z_{2}^{Y_{2}\left(t\right)}\right], \tilde{P}_{2}\left(z_{2}\right)=\esp\left[z_{2}^{\tilde{X}_{2}\left(t\right)}\right]$ , con primer momento definidos por $\mu_{i}=\esp\left[X_{i}\left(t\right)\right]=P_{i}^{(1)}\left(1\right), \hat{\mu}_{i}=\esp\left[\hat{X}_{i}\left(t\right)\right]=\hat{P}_{i}^{(1)}\left(1\right)$, para $i=1,2$, y
$\check{\mu}_{2}=\esp\left[Y_{2}\left(t\right)\right]=\check{P}_{2}^{(1)}\left(1\right),\tilde{\mu}_{2}=\esp\left[\tilde{X}_{2}\left(t\right)\right]=\tilde{P}_{2}^{(1)}\left(1\right)$.

En lo que respecta al servidor, en t\'erminos de los tiempos de
visita a cada una de las colas, se denotar\'an por
$B_{i}\left(t\right)$ a los procesos
correspondientes a las variables aleatorias $\tau_{i}$
para $Q_{i}$, respectivamente; y
$\hat{B}_{i}\left(t\right)$ con
par\'ametros $\zeta_{i}$ para $\hat{Q}_{i}$, del sistema 2 respectivamente. Y a los tiempos en que el servidor termina de
atender en las colas $Q_{i},\hat{Q}_{i}$, se les
denotar\'a por
$\overline{\tau}_{i},\overline{\zeta}_{i}$ respectivamente. Entonces, los tiempos de servicio est\'an dados por las diferencias
$\overline{\tau}_{i}-\tau_{i}$ para
$Q_{i}$, y
$\overline{\zeta}_{i}-\zeta_{i}$ para $\hat{Q}_{i}$ respectivamente, para $i=1,2$.

Sus procesos se definen por: $S_{i}\left(z_{i}\right)=\esp\left[z_{i}^{\overline{\tau}_{i}-\tau_{i}}\right]$ y $\hat{S}_{i}\left(w_{i}\right)=\esp\left[w_{i}^{\overline{\zeta}_{i}-\zeta_{i}}\right]$, con primer momento dado por: $s_{i}=\esp\left[\overline{\tau}_{i}-\tau_{i}\right]$ y $\hat{s}_{i}=\esp\left[\overline{\zeta}_{i}-\zeta_{i}\right]$, para $i=1,2$. An\'alogamente los tiempos de traslado del servidor desde el momento en que termina de atender a una cola y llega a la
siguiente para comenzar a dar servicio est\'an dados por
$\tau_{i+1}-\overline{\tau}_{i}$ y
$\zeta_{i+1}-\overline{\zeta}_{i}$ para el sistema 1 y el sistema 2, respectivamente, con $i=1,2$.

La FGP para estos tiempos de traslado est\'an dados por $R_{i}\left(z_{i}\right)=\esp\left[z_{1}^{\tau_{i+1}-\overline{\tau}_{i}}\right]$ y $\hat{R}_{i}\left(w_{i}\right)=\esp\left[w_{i}^{\zeta_{i+1}-\overline{\zeta}_{i}}\right]$ y al igual que como se hizo con anterioridad, se tienen los primeros momentos de estos procesos de traslado del servidor entre las colas de cada uno de los sistemas que conforman la red de sistemas de visitas c\'iclicas: $r_{i}=R_{i}^{(1)}\left(1\right)=\esp\left[\tau_{i+1}-\overline{\tau}_{i}\right]$ y $\hat{r}_{i}=\hat{R}_{i}^{(1)}\left(1\right)=\esp\left[\zeta_{i+1}-\overline{\zeta}_{i}\right]$ para $i=1,2$.


Se definen los procesos de conteo para el n\'umero de usuarios en
cada una de las colas al tiempo $t$,
$L_{i}\left(t\right)$, para
$H_{i}\left(t\right)$ del sistema 1,
mientras que para el segundo sistema, se tienen los procesos
$\hat{L}_{i}\left(t\right)$ para
$\hat{H}_{i}\left(t\right)$, es decir, $H_{i}\left(t\right)=\esp\left[z_{i}^{L_{i}\left(t\right)}\right]$ y $\hat{H}_{i}\left(t\right)=\esp\left[w_{i}^{\hat{L}_{i}\left(t\right)}\right]$. Con lo dichohasta ahora, se tiene que el n\'umero de usuarios
presentes en los tiempos $\overline{\tau}_{1},\overline{\tau}_{2},
\overline{\zeta}_{1},\overline{\zeta}_{2}$, es cero, es decir,
 $L_{i}\left(\overline{\tau_{i}}\right)=0,$ y
$\hat{L}_{i}\left(\overline{\zeta_{i}}\right)=0$ para i=1,2 para
cada uno de los dos sistemas.


Para cada una de las colas en cada sistema, el n\'umero de
usuarios al tiempo en que llega el servidor a dar servicio est\'a
dado por el n\'umero de usuarios presentes en la cola al tiempo
$t=\tau_{i},\zeta_{i}$, m\'as el n\'umero de usuarios que llegan a
la cola en el intervalo de tiempo
$\left[\tau_{i},\overline{\tau}_{i}\right],\left[\zeta_{i},\overline{\zeta}_{i}\right]$,
es decir $\hat{L}_{i}\left(\overline{\tau}_{j}\right)=\hat{L}_{i}\left(\tau_{j}\right)+\hat{X}_{i}\left(\overline{\tau}_{j}-\tau_{j}\right)$, para $i,j=1,2$, mientras que para el primer sistema: $L_{1}\left(\overline{\tau}_{j}\right)=L_{1}\left(\tau_{j}\right)+X_{1}\left(\overline{\tau}_{j}-\tau_{j}\right)$. En el caso espec\'ifico de $Q_{2}$, adem\'as, hay que considerar
el n\'umero de usuarios que pasan del sistema 2 al sistema 1, a
traves de $\hat{Q}_{2}$ mientras el servidor en $Q_{2}$ est\'a
ausente, es decir:

\begin{equation}\label{Eq.UsuariosTotalesZ2}
L_{2}\left(\overline{\tau}_{1}\right)=L_{2}\left(\tau_{1}\right)+X_{2}\left(\overline{\tau}_{1}-\tau_{1}\right)+Y_{2}\left(\overline{\tau}_{1}-\tau_{1}\right).
\end{equation}

%_________________________________________________________________________
\subsection{El problema de la ruina del jugador}
%_________________________________________________________________________

Supongamos que se tiene un jugador que cuenta con un capital
inicial de $\tilde{L}_{0}\geq0$ unidades, esta persona realiza una
serie de dos juegos simult\'aneos e independientes de manera
sucesiva, dichos eventos son independientes e id\'enticos entre
s\'i para cada realizaci\'on. La ganancia en el $n$-\'esimo juego es $\tilde{X}_{n}=X_{n}+Y_{n}$ unidades de las cuales se resta una cuota de 1 unidad por cada juego simult\'aneo, es decir, se restan dos unidades por cada
juego realizado. En t\'erminos de la teor\'ia de colas puede pensarse como el n\'umero de usuarios que llegan a una cola v\'ia dos procesos de arribo distintos e independientes entre s\'i. Su Funci\'on Generadora de Probabilidades (FGP) est\'a dada por $F\left(z\right)=\esp\left[z^{\tilde{L}_{0}}\right]$, adem\'as
$$\tilde{P}\left(z\right)=\esp\left[z^{\tilde{X}_{n}}\right]=\esp\left[z^{X_{n}+Y_{n}}\right]=\esp\left[z^{X_{n}}z^{Y_{n}}\right]=\esp\left[z^{X_{n}}\right]\esp\left[z^{Y_{n}}\right]=P\left(z\right)\check{P}\left(z\right),$$

con $\tilde{\mu}=\esp\left[\tilde{X}_{n}\right]=\tilde{P}\left[z\right]<1$. Sea $\tilde{L}_{n}$ el capital remanente despu\'es del $n$-\'esimo
juego. Entonces

$$\tilde{L}_{n}=\tilde{L}_{0}+\tilde{X}_{1}+\tilde{X}_{2}+\cdots+\tilde{X}_{n}-2n.$$

La ruina del jugador ocurre despu\'es del $n$-\'esimo juego, es decir, la cola se vac\'ia despu\'es del $n$-\'esimo juego,
entonces sea $T$ definida como $T=min\left\{\tilde{L}_{n}=0\right\}$. Si $\tilde{L}_{0}=0$, entonces claramente $T=0$. En este sentido $T$
puede interpretarse como la longitud del periodo de tiempo que el servidor ocupa para dar servicio en la cola, comenzando con $\tilde{L}_{0}$ grupos de usuarios presentes en la cola, quienes arribaron conforme a un proceso dado
por $\tilde{P}\left(z\right)$.\smallskip


Sea $g_{n,k}$ la probabilidad del evento de que el jugador no
caiga en ruina antes del $n$-\'esimo juego, y que adem\'as tenga
un capital de $k$ unidades antes del $n$-\'esimo juego, es decir,

Dada $n\in\left\{1,2,\ldots,\right\}$ y
$k\in\left\{0,1,2,\ldots,\right\}$
\begin{eqnarray*}
g_{n,k}:=P\left\{\tilde{L}_{j}>0, j=1,\ldots,n,
\tilde{L}_{n}=k\right\}
\end{eqnarray*}

la cual adem\'as se puede escribir como:

\begin{eqnarray*}
g_{n,k}&=&P\left\{\tilde{L}_{j}>0, j=1,\ldots,n,
\tilde{L}_{n}=k\right\}=\sum_{j=1}^{k+1}g_{n-1,j}P\left\{\tilde{X}_{n}=k-j+1\right\}\\
&=&\sum_{j=1}^{k+1}g_{n-1,j}P\left\{X_{n}+Y_{n}=k-j+1\right\}=\sum_{j=1}^{k+1}\sum_{l=1}^{j}g_{n-1,j}P\left\{X_{n}+Y_{n}=k-j+1,Y_{n}=l\right\}\\
&=&\sum_{j=1}^{k+1}\sum_{l=1}^{j}g_{n-1,j}P\left\{X_{n}+Y_{n}=k-j+1|Y_{n}=l\right\}P\left\{Y_{n}=l\right\}\\
&=&\sum_{j=1}^{k+1}\sum_{l=1}^{j}g_{n-1,j}P\left\{X_{n}=k-j-l+1\right\}P\left\{Y_{n}=l\right\}\\
\end{eqnarray*}

es decir
\begin{eqnarray}\label{Eq.Gnk.2S}
g_{n,k}=\sum_{j=1}^{k+1}\sum_{l=1}^{j}g_{n-1,j}P\left\{X_{n}=k-j-l+1\right\}P\left\{Y_{n}=l\right\}
\end{eqnarray}
adem\'as

\begin{equation}\label{Eq.L02S}
g_{0,k}=P\left\{\tilde{L}_{0}=k\right\}.
\end{equation}

Se definen las siguientes FGP:
\begin{equation}\label{Eq.3.16.a.2S}
G_{n}\left(z\right)=\sum_{k=0}^{\infty}g_{n,k}z^{k},\textrm{ para
}n=0,1,\ldots,
\end{equation}

\begin{equation}\label{Eq.3.16.b.2S}
G\left(z,w\right)=\sum_{n=0}^{\infty}G_{n}\left(z\right)w^{n}.
\end{equation}


En particular para $k=0$,
\begin{eqnarray*}
g_{n,0}=G_{n}\left(0\right)=P\left\{\tilde{L}_{j}>0,\textrm{ para
}j<n,\textrm{ y }\tilde{L}_{n}=0\right\}=P\left\{T=n\right\},
\end{eqnarray*}

adem\'as

\begin{eqnarray*}%\label{Eq.G0w.2S}
G\left(0,w\right)=\sum_{n=0}^{\infty}G_{n}\left(0\right)w^{n}=\sum_{n=0}^{\infty}P\left\{T=n\right\}w^{n}
=\esp\left[w^{T}\right]
\end{eqnarray*}
la cu\'al resulta ser la FGP del tiempo de ruina $T$.

%__________________________________________________________________________________
% INICIA LA PROPOSICIÓN
%__________________________________________________________________________________


\begin{Prop}\label{Prop.1.1.2S}
Sean $G_{n}\left(z\right)$ y $G\left(z,w\right)$ definidas como en
(\ref{Eq.3.16.a.2S}) y (\ref{Eq.3.16.b.2S}) respectivamente,
entonces
\begin{equation}\label{Eq.Pag.45}
G_{n}\left(z\right)=\frac{1}{z}\left[G_{n-1}\left(z\right)-G_{n-1}\left(0\right)\right]\tilde{P}\left(z\right).
\end{equation}

Adem\'as


\begin{equation}\label{Eq.Pag.46}
G\left(z,w\right)=\frac{zF\left(z\right)-wP\left(z\right)G\left(0,w\right)}{z-wR\left(z\right)},
\end{equation}

con un \'unico polo en el c\'irculo unitario, adem\'as, el polo es
de la forma $z=\theta\left(w\right)$ y satisface que

\begin{enumerate}
\item[i)]$\tilde{\theta}\left(1\right)=1$,

\item[ii)] $\tilde{\theta}^{(1)}\left(1\right)=\frac{1}{1-\tilde{\mu}}$,

\item[iii)]
$\tilde{\theta}^{(2)}\left(1\right)=\frac{\tilde{\mu}}{\left(1-\tilde{\mu}\right)^{2}}+\frac{\tilde{\sigma}}{\left(1-\tilde{\mu}\right)^{3}}$.
\end{enumerate}

Finalmente, adem\'as se cumple que
\begin{equation}
\esp\left[w^{T}\right]=G\left(0,w\right)=F\left[\tilde{\theta}\left(w\right)\right].
\end{equation}
\end{Prop}
%__________________________________________________________________________________
% TERMINA LA PROPOSICIÓN E INICIA LA DEMOSTRACI\'ON
%__________________________________________________________________________________


Multiplicando las ecuaciones (\ref{Eq.Gnk.2S}) y (\ref{Eq.L02S})
por el t\'ermino $z^{k}$:

\begin{eqnarray*}
g_{n,k}z^{k}&=&\sum_{j=1}^{k+1}\sum_{l=1}^{j}g_{n-1,j}P\left\{X_{n}=k-j-l+1\right\}P\left\{Y_{n}=l\right\}z^{k},\\
g_{0,k}z^{k}&=&P\left\{\tilde{L}_{0}=k\right\}z^{k},
\end{eqnarray*}

ahora sumamos sobre $k$
\begin{eqnarray*}
\sum_{k=0}^{\infty}g_{n,k}z^{k}&=&\sum_{k=0}^{\infty}\sum_{j=1}^{k+1}\sum_{l=1}^{j}g_{n-1,j}P\left\{X_{n}=k-j-l+1\right\}P\left\{Y_{n}=l\right\}z^{k}\\
&=&\sum_{k=0}^{\infty}z^{k}\sum_{j=1}^{k+1}\sum_{l=1}^{j}g_{n-1,j}P\left\{X_{n}=k-\left(j+l
-1\right)\right\}P\left\{Y_{n}=l\right\}\\
&=&\sum_{k=0}^{\infty}z^{k+\left(j+l-1\right)-\left(j+l-1\right)}\sum_{j=1}^{k+1}\sum_{l=1}^{j}g_{n-1,j}P\left\{X_{n}=k-
\left(j+l-1\right)\right\}P\left\{Y_{n}=l\right\}\\
&=&\sum_{k=0}^{\infty}\sum_{j=1}^{k+1}\sum_{l=1}^{j}g_{n-1,j}z^{j-1}P\left\{X_{n}=k-
\left(j+l-1\right)\right\}z^{k-\left(j+l-1\right)}P\left\{Y_{n}=l\right\}z^{l}\\
&=&\sum_{j=1}^{\infty}\sum_{l=1}^{j}g_{n-1,j}z^{j-1}\sum_{k=j+l-1}^{\infty}P\left\{X_{n}=k-
\left(j+l-1\right)\right\}z^{k-\left(j+l-1\right)}P\left\{Y_{n}=l\right\}z^{l}\\
&=&\sum_{j=1}^{\infty}g_{n-1,j}z^{j-1}\sum_{l=1}^{j}\sum_{k=j+l-1}^{\infty}P\left\{X_{n}=k-
\left(j+l-1\right)\right\}z^{k-\left(j+l-1\right)}P\left\{Y_{n}=l\right\}z^{l}\\
&=&\sum_{j=1}^{\infty}g_{n-1,j}z^{j-1}\sum_{k=j+l-1}^{\infty}\sum_{l=1}^{j}P\left\{X_{n}=k-
\left(j+l-1\right)\right\}z^{k-\left(j+l-1\right)}P\left\{Y_{n}=l\right\}z^{l}\\
\end{eqnarray*}


luego
\begin{eqnarray*}
&=&\sum_{j=1}^{\infty}g_{n-1,j}z^{j-1}\sum_{k=j+l-1}^{\infty}\sum_{l=1}^{j}P\left\{X_{n}=k-
\left(j+l-1\right)\right\}z^{k-\left(j+l-1\right)}\sum_{l=1}^{j}P
\left\{Y_{n}=l\right\}z^{l}\\
&=&\sum_{j=1}^{\infty}g_{n-1,j}z^{j-1}\sum_{l=1}^{\infty}P\left\{Y_{n}=l\right\}z^{l}
\sum_{k=j+l-1}^{\infty}\sum_{l=1}^{j}
P\left\{X_{n}=k-\left(j+l-1\right)\right\}z^{k-\left(j+l-1\right)}\\
&=&\frac{1}{z}\left[G_{n-1}\left(z\right)-G_{n-1}\left(0\right)\right]\tilde{P}\left(z\right)
\sum_{k=j+l-1}^{\infty}\sum_{l=1}^{j}
P\left\{X_{n}=k-\left(j+l-1\right)\right\}z^{k-\left(j+l-1\right)}\\
&=&\frac{1}{z}\left[G_{n-1}\left(z\right)-G_{n-1}\left(0\right)\right]\tilde{P}\left(z\right)P\left(z\right)=\frac{1}{z}\left[G_{n-1}\left(z\right)-G_{n-1}\left(0\right)\right]\tilde{P}\left(z\right),\\
\end{eqnarray*}

es decir la ecuaci\'on (\ref{Eq.3.16.a.2S}) se puede reescribir
como
\begin{equation}\label{Eq.3.16.a.2Sbis}
G_{n}\left(z\right)=\frac{1}{z}\left[G_{n-1}\left(z\right)-G_{n-1}\left(0\right)\right]\tilde{P}\left(z\right).
\end{equation}

Por otra parte recordemos la ecuaci\'on (\ref{Eq.3.16.a.2S})

\begin{eqnarray*}
G_{n}\left(z\right)&=&\sum_{k=0}^{\infty}g_{n,k}z^{k},\textrm{ entonces }\frac{G_{n}\left(z\right)}{z}=\sum_{k=1}^{\infty}g_{n,k}z^{k-1},\\
\end{eqnarray*}

Por lo tanto utilizando la ecuaci\'on (\ref{Eq.3.16.a.2Sbis}):

\begin{eqnarray*}
G\left(z,w\right)&=&\sum_{n=0}^{\infty}G_{n}\left(z\right)w^{n}=G_{0}\left(z\right)+
\sum_{n=1}^{\infty}G_{n}\left(z\right)w^{n}=F\left(z\right)+\sum_{n=0}^{\infty}\left[G_{n}\left(z\right)-G_{n}\left(0\right)\right]w^{n}\frac{\tilde{P}\left(z\right)}{z}\\
&=&F\left(z\right)+\frac{w}{z}\sum_{n=0}^{\infty}\left[G_{n}\left(z\right)-G_{n}\left(0\right)\right]w^{n-1}\tilde{P}\left(z\right)\\
\end{eqnarray*}

es decir
\begin{eqnarray*}
G\left(z,w\right)&=&F\left(z\right)+\frac{w}{z}\left[G\left(z,w\right)-G\left(0,w\right)\right]\tilde{P}\left(z\right),
\end{eqnarray*}


entonces

\begin{eqnarray*}
G\left(z,w\right)=F\left(z\right)+\frac{w}{z}\left[G\left(z,w\right)-G\left(0,w\right)\right]\tilde{P}\left(z\right)&=&F\left(z\right)+\frac{w}{z}\tilde{P}\left(z\right)G\left(z,w\right)-\frac{w}{z}\tilde{P}\left(z\right)G\left(0,w\right)\\
&\Leftrightarrow&\\
G\left(z,w\right)\left\{1-\frac{w}{z}\tilde{P}\left(z\right)\right\}&=&F\left(z\right)-\frac{w}{z}\tilde{P}\left(z\right)G\left(0,w\right),
\end{eqnarray*}
por lo tanto,
\begin{equation}
G\left(z,w\right)=\frac{zF\left(z\right)-w\tilde{P}\left(z\right)G\left(0,w\right)}{1-w\tilde{P}\left(z\right)}.
\end{equation}


Ahora $G\left(z,w\right)$ es anal\'itica en $|z|=1$. Sean $z,w$ tales que $|z|=1$ y $|w|\leq1$, como $\tilde{P}\left(z\right)$ es FGP
\begin{eqnarray*}
|z-\left(z-w\tilde{P}\left(z\right)\right)|<|z|\Leftrightarrow|w\tilde{P}\left(z\right)|<|z|
\end{eqnarray*}
es decir, se cumplen las condiciones del Teorema de Rouch\'e y por
tanto, $z$ y $z-w\tilde{P}\left(z\right)$ tienen el mismo n\'umero de
ceros en $|z|=1$. Sea $z=\tilde{\theta}\left(w\right)$ la soluci\'on
\'unica de $z-w\tilde{P}\left(z\right)$, es decir

\begin{equation}\label{Eq.Theta.w}
\tilde{\theta}\left(w\right)-w\tilde{P}\left(\tilde{\theta}\left(w\right)\right)=0,
\end{equation}
 con $|\tilde{\theta}\left(w\right)|<1$. Cabe hacer menci\'on que $\tilde{\theta}\left(w\right)$ es la FGP para el tiempo de ruina cuando $\tilde{L}_{0}=1$.


Considerando la ecuaci\'on (\ref{Eq.Theta.w})
\begin{eqnarray*}
0&=&\frac{\partial}{\partial w}\tilde{\theta}\left(w\right)|_{w=1}-\frac{\partial}{\partial w}\left\{w\tilde{P}\left(\tilde{\theta}\left(w\right)\right)\right\}|_{w=1}=\tilde{\theta}^{(1)}\left(w\right)|_{w=1}-\frac{\partial}{\partial w}w\left\{\tilde{P}\left(\tilde{\theta}\left(w\right)\right)\right\}|_{w=1}\\
&-&w\frac{\partial}{\partial w}\tilde{P}\left(\tilde{\theta}\left(w\right)\right)|_{w=1}=\tilde{\theta}^{(1)}\left(1\right)-\tilde{P}\left(\tilde{\theta}\left(1\right)\right)-w\left\{\frac{\partial \tilde{P}\left(\tilde{\theta}\left(w\right)\right)}{\partial \tilde{\theta}\left(w\right)}\cdot\frac{\partial\tilde{\theta}\left(w\right)}{\partial w}|_{w=1}\right\}\\
&&\tilde{\theta}^{(1)}\left(1\right)-\tilde{P}\left(\tilde{\theta}\left(1\right)
\right)-\tilde{P}^{(1)}\left(\tilde{\theta}\left(1\right)\right)\cdot\tilde{\theta}^{(1)}\left(1\right),
\end{eqnarray*}


luego
$$\tilde{P}\left(\tilde{\theta}\left(1\right)\right)=\tilde{\theta}^{(1)}\left(1\right)-\tilde{P}^{(1)}\left(\tilde{\theta}\left(1\right)\right)\cdot
\tilde{\theta}^{(1)}\left(1\right)=\tilde{\theta}^{(1)}\left(1\right)\left(1-\tilde{P}^{(1)}\left(\tilde{\theta}\left(1\right)\right)\right),$$

por tanto $$\tilde{\theta}^{(1)}\left(1\right)=\frac{\tilde{P}\left(\tilde{\theta}\left(1\right)\right)}{\left(1-\tilde{P}^{(1)}\left(\tilde{\theta}\left(1\right)\right)\right)}=\frac{1}{1-\tilde{\mu}}.$$

Ahora determinemos el segundo momento de $\tilde{\theta}\left(w\right)$,
nuevamente consideremos la ecuaci\'on (\ref{Eq.Theta.w}):

\begin{eqnarray*}
0&=&\tilde{\theta}\left(w\right)-w\tilde{P}\left(\tilde{\theta}\left(w\right)\right)\Rightarrow 0=\frac{\partial}{\partial w}\left\{\tilde{\theta}\left(w\right)-w\tilde{P}\left(\tilde{\theta}\left(w\right)\right)\right\}\Rightarrow 0=\frac{\partial}{\partial w}\left\{\frac{\partial}{\partial w}\left\{\tilde{\theta}\left(w\right)-w\tilde{P}\left(\tilde{\theta}\left(w\right)\right)\right\}\right\}\\
\end{eqnarray*}
luego
\begin{eqnarray*}
&&\frac{\partial}{\partial w}\left\{\frac{\partial}{\partial w}\tilde{\theta}\left(w\right)-\frac{\partial}{\partial w}\left[w\tilde{P}\left(\tilde{\theta}\left(w\right)\right)\right]\right\}
=\frac{\partial}{\partial w}\left\{\frac{\partial}{\partial w}\tilde{\theta}\left(w\right)-\frac{\partial}{\partial w}\left[w\tilde{P}\left(\tilde{\theta}\left(w\right)\right)\right]\right\}\\
&=&\frac{\partial}{\partial w}\left\{\frac{\partial \tilde{\theta}\left(w\right)}{\partial w}-\left[\tilde{P}\left(\tilde{\theta}\left(w\right)\right)+w\frac{\partial}{\partial w}R\left(\tilde{\theta}\left(w\right)\right)\right]\right\}=\frac{\partial}{\partial w}\left\{\frac{\partial \tilde{\theta}\left(w\right)}{\partial w}-\left[\tilde{P}\left(\tilde{\theta}\left(w\right)\right)+w\frac{\partial \tilde{P}\left(\tilde{\theta}\left(w\right)\right)}{\partial w}\frac{\partial \tilde{\theta}\left(w\right)}{\partial w}\right]\right\}\\
&=&\frac{\partial}{\partial w}\left\{\tilde{\theta}^{(1)}\left(w\right)-\tilde{P}\left(\tilde{\theta}\left(w\right)\right)-w\tilde{P}^{(1)}\left(\tilde{\theta}\left(w\right)\right)\tilde{\theta}^{(1)}\left(w\right)\right\}\\
&=&\frac{\partial}{\partial w}\tilde{\theta}^{(1)}\left(w\right)-\frac{\partial}{\partial w}\tilde{P}\left(\tilde{\theta}\left(w\right)\right)-\frac{\partial}{\partial w}\left[w\tilde{P}^{(1)}\left(\tilde{\theta}\left(w\right)\right)\tilde{\theta}^{(1)}\left(w\right)\right]\\
&=&\frac{\partial}{\partial
w}\tilde{\theta}^{(1)}\left(w\right)-\frac{\partial
\tilde{P}\left(\tilde{\theta}\left(w\right)\right)}{\partial
\tilde{\theta}\left(w\right)}\frac{\partial \tilde{\theta}\left(w\right)}{\partial
w}-\tilde{P}^{(1)}\left(\tilde{\theta}\left(w\right)\right)\tilde{\theta}^{(1)}\left(w\right)-w\frac{\partial
\tilde{P}^{(1)}\left(\tilde{\theta}\left(w\right)\right)}{\partial
w}\tilde{\theta}^{(1)}\left(w\right)-w\tilde{P}^{(1)}\left(\tilde{\theta}\left(w\right)\right)\frac{\partial
\tilde{\theta}^{(1)}\left(w\right)}{\partial w}\\
&=&\tilde{\theta}^{(2)}\left(w\right)-\tilde{P}^{(1)}\left(\tilde{\theta}\left(w\right)\right)\tilde{\theta}^{(1)}\left(w\right)
-\tilde{P}^{(1)}\left(\tilde{\theta}\left(w\right)\right)\tilde{\theta}^{(1)}\left(w\right)-w\tilde{P}^{(2)}\left(\tilde{\theta}\left(w\right)\right)\left(\tilde{\theta}^{(1)}\left(w\right)\right)^{2}-w\tilde{P}^{(1)}\left(\tilde{\theta}\left(w\right)\right)\tilde{\theta}^{(2)}\left(w\right)\\
&=&\tilde{\theta}^{(2)}\left(w\right)-2\tilde{P}^{(1)}\left(\tilde{\theta}\left(w\right)\right)\tilde{\theta}^{(1)}\left(w\right)-w\tilde{P}^{(2)}\left(\tilde{\theta}\left(w\right)\right)\left(\tilde{\theta}^{(1)}\left(w\right)\right)^{2}-w\tilde{P}^{(1)}\left(\tilde{\theta}\left(w\right)\right)\tilde{\theta}^{(2)}\left(w\right)\\
&=&\tilde{\theta}^{(2)}\left(w\right)\left[1-w\tilde{P}^{(1)}\left(\tilde{\theta}\left(w\right)\right)\right]-
\tilde{\theta}^{(1)}\left(w\right)\left[w\tilde{\theta}^{(1)}\left(w\right)\tilde{P}^{(2)}\left(\tilde{\theta}\left(w\right)\right)+2\tilde{P}^{(1)}\left(\tilde{\theta}\left(w\right)\right)\right]
\end{eqnarray*}


luego

\begin{eqnarray*}
\tilde{\theta}^{(2)}\left(w\right)\left[1-w\tilde{P}^{(1)}\left(\tilde{\theta}\left(w\right)\right)\right]&-&\tilde{\theta}^{(1)}\left(w\right)\left[w\tilde{\theta}^{(1)}\left(w\right)\tilde{P}^{(2)}\left(\tilde{\theta}\left(w\right)\right)
+2\tilde{P}^{(1)}\left(\tilde{\theta}\left(w\right)\right)\right]=0\\
\tilde{\theta}^{(2)}\left(w\right)&=&\frac{\tilde{\theta}^{(1)}\left(w\right)\left[w\tilde{\theta}^{(1)}\left(w\right)\tilde{P}^{(2)}\left(\tilde{\theta}\left(w\right)\right)+2R^{(1)}\left(\tilde{\theta}\left(w\right)\right)\right]}{1-w\tilde{P}^{(1)}\left(\tilde{\theta}\left(w\right)\right)}\\
\tilde{\theta}^{(2)}\left(w\right)&=&\frac{\tilde{\theta}^{(1)}\left(w\right)w\tilde{\theta}^{(1)}\left(w\right)\tilde{P}^{(2)}\left(\tilde{\theta}\left(w\right)\right)}{1-w\tilde{P}^{(1)}\left(\tilde{\theta}\left(w\right)\right)}+\frac{2\tilde{\theta}^{(1)}\left(w\right)\tilde{P}^{(1)}\left(\tilde{\theta}\left(w\right)\right)}{1-w\tilde{P}^{(1)}\left(\tilde{\theta}\left(w\right)\right)}
\end{eqnarray*}


si evaluamos la expresi\'on anterior en $w=1$:
\begin{eqnarray*}
\tilde{\theta}^{(2)}\left(1\right)&=&\frac{\left(\tilde{\theta}^{(1)}\left(1\right)\right)^{2}\tilde{P}^{(2)}\left(\tilde{\theta}\left(1\right)\right)}{1-\tilde{P}^{(1)}\left(\tilde{\theta}\left(1\right)\right)}+\frac{2\tilde{\theta}^{(1)}\left(1\right)\tilde{P}^{(1)}\left(\tilde{\theta}\left(1\right)\right)}{1-\tilde{P}^{(1)}\left(\tilde{\theta}\left(1\right)\right)}=\frac{\left(\tilde{\theta}^{(1)}\left(1\right)\right)^{2}\tilde{P}^{(2)}\left(1\right)}{1-\tilde{P}^{(1)}\left(1\right)}+\frac{2\tilde{\theta}^{(1)}\left(1\right)\tilde{P}^{(1)}\left(1\right)}{1-\tilde{P}^{(1)}\left(1\right)}\\
&=&\frac{\left(\frac{1}{1-\tilde{\mu}}\right)^{2}\tilde{P}^{(2)}\left(1\right)}{1-\tilde{\mu}}+\frac{2\left(\frac{1}{1-\tilde{\mu}}\right)\tilde{\mu}}{1-\tilde{\mu}}=\frac{\tilde{P}^{(2)}\left(1\right)}{\left(1-\tilde{\mu}\right)^{3}}+\frac{2\tilde{\mu}}{\left(1-\tilde{\mu}\right)^{2}}=\frac{\sigma^{2}-\tilde{\mu}+\tilde{\mu}^{2}}{\left(1-\tilde{\mu}\right)^{3}}+\frac{2\tilde{\mu}}{\left(1-\tilde{\mu}\right)^{2}}\\
&=&\frac{\sigma^{2}-\tilde{\mu}+\tilde{\mu}^{2}+2\tilde{\mu}\left(1-\tilde{\mu}\right)}{\left(1-\tilde{\mu}\right)^{3}}\\
\end{eqnarray*}


es decir
\begin{eqnarray*}
\tilde{\theta}^{(2)}\left(1\right)&=&\frac{\sigma^{2}+\tilde{\mu}-\tilde{\mu}^{2}}{\left(1-\tilde{\mu}\right)^{3}}=\frac{\sigma^{2}}{\left(1-\tilde{\mu}\right)^{3}}+\frac{\tilde{\mu}\left(1-\tilde{\mu}\right)}{\left(1-\tilde{\mu}\right)^{3}}=\frac{\sigma^{2}}{\left(1-\tilde{\mu}\right)^{3}}+\frac{\tilde{\mu}}{\left(1-\tilde{\mu}\right)^{2}}.
\end{eqnarray*}

\begin{Coro}
El tiempo de ruina del jugador tiene primer y segundo momento
dados por

\begin{eqnarray}
\esp\left[T\right]&=&\frac{\esp\left[\tilde{L}_{0}\right]}{1-\tilde{\mu}}\\
Var\left[T\right]&=&\frac{Var\left[\tilde{L}_{0}\right]}{\left(1-\tilde{\mu}\right)^{2}}+\frac{\sigma^{2}\esp\left[\tilde{L}_{0}\right]}{\left(1-\tilde{\mu}\right)^{3}}.
\end{eqnarray}
\end{Coro}



%__________________________________________________________________________
\section{Procesos de Llegadas a las colas en la RSVC}
%__________________________________________________________________________

Se definen los procesos de llegada de los usuarios a cada una de
las colas dependiendo de la llegada del servidor pero del sistema
al cu\'al no pertenece la cola en cuesti\'on:

Para el sistema 1 y el servidor del segundo sistema

\begin{eqnarray*}
F_{i,j}\left(z_{i};\zeta_{j}\right)=\esp\left[z_{i}^{L_{i}\left(\zeta_{j}\right)}\right]=
\sum_{k=0}^{\infty}\prob\left[L_{i}\left(\zeta_{j}\right)=k\right]z_{i}^{k}\textrm{, para }i,j=1,2.
%F_{1,1}\left(z_{1};\zeta_{1}\right)&=&\esp\left[z_{1}^{L_{1}\left(\zeta_{1}\right)}\right]=
%\sum_{k=0}^{\infty}\prob\left[L_{1}\left(\zeta_{1}\right)=k\right]z_{1}^{k};\\
%F_{2,1}\left(z_{2};\zeta_{1}\right)&=&\esp\left[z_{2}^{L_{2}\left(\zeta_{1}\right)}\right]=
%\sum_{k=0}^{\infty}\prob\left[L_{2}\left(\zeta_{1}\right)=k\right]z_{2}^{k};\\
%F_{1,2}\left(z_{1};\zeta_{2}\right)&=&\esp\left[z_{1}^{L_{1}\left(\zeta_{2}\right)}\right]=
%\sum_{k=0}^{\infty}\prob\left[L_{1}\left(\zeta_{2}\right)=k\right]z_{1}^{k};\\
%F_{2,2}\left(z_{2};\zeta_{2}\right)&=&\esp\left[z_{2}^{L_{2}\left(\zeta_{2}\right)}\right]=
%\sum_{k=0}^{\infty}\prob\left[L_{2}\left(\zeta_{2}\right)=k\right]z_{2}^{k}.\\
\end{eqnarray*}

Ahora se definen para el segundo sistema y el servidor del primero


\begin{eqnarray*}
\hat{F}_{i,j}\left(w_{i};\tau_{j}\right)&=&\esp\left[w_{i}^{\hat{L}_{i}\left(\tau_{j}\right)}\right] =\sum_{k=0}^{\infty}\prob\left[\hat{L}_{i}\left(\tau_{j}\right)=k\right]w_{i}^{k}\textrm{, para }i,j=1,2.
%\hat{F}_{1,1}\left(w_{1};\tau_{1}\right)&=&\esp\left[w_{1}^{\hat{L}_{1}\left(\tau_{1}\right)}\right] =\sum_{k=0}^{\infty}\prob\left[\hat{L}_{1}\left(\tau_{1}\right)=k\right]w_{1}^{k}\\
%\hat{F}_{2,1}\left(w_{2};\tau_{1}\right)&=&\esp\left[w_{2}^{\hat{L}_{2}\left(\tau_{1}\right)}\right] =\sum_{k=0}^{\infty}\prob\left[\hat{L}_{2}\left(\tau_{1}\right)=k\right]w_{2}^{k}\\
%\hat{F}_{1,2}\left(w_{1};\tau_{2}\right)&=&\esp\left[w_{1}^{\hat{L}_{1}\left(\tau_{2}\right)}\right]
%=\sum_{k=0}^{\infty}\prob\left[\hat{L}_{1}\left(\tau_{2}\right)=k\right]w_{1}^{k}\\
%\hat{F}_{2,2}\left(w_{2};\tau_{2}\right)&=&\esp\left[w_{2}^{\hat{L}_{2}\left(\tau_{2}\right)}\right]
%=\sum_{k=0}^{\infty}\prob\left[\hat{L}_{2}\left(\tau_{2}\right)=k\right]w_{2}^{k}\\
\end{eqnarray*}


Ahora, con lo anterior definamos la FGP conjunta para el segundo sistema;% y $\tau_{1}$:


\begin{eqnarray*}
\esp\left[w_{1}^{\hat{L}_{1}\left(\tau_{j}\right)}w_{2}^{\hat{L}_{2}\left(\tau_{j}\right)}\right]
&=&\esp\left[w_{1}^{\hat{L}_{1}\left(\tau_{j}\right)}\right]
\esp\left[w_{2}^{\hat{L}_{2}\left(\tau_{j}\right)}\right]=\hat{F}_{1,j}\left(w_{1};\tau_{j}\right)\hat{F}_{2,j}\left(w_{2};\tau_{j}\right)=\hat{F}_{j}\left(w_{1},w_{2};\tau_{j}\right).\\
%\esp\left[w_{1}^{\hat{L}_{1}\left(\tau_{1}\right)}w_{2}^{\hat{L}_{2}\left(\tau_{1}\right)}\right]
%&=&\esp\left[w_{1}^{\hat{L}_{1}\left(\tau_{1}\right)}\right]
%\esp\left[w_{2}^{\hat{L}_{2}\left(\tau_{1}\right)}\right]=\hat{F}_{1,1}\left(w_{1};\tau_{1}\right)\hat{F}_{2,1}\left(w_{2};\tau_{1}\right)=\hat{F}_{1}\left(w_{1},w_{2};\tau_{1}\right)\\
%\esp\left[w_{1}^{\hat{L}_{1}\left(\tau_{2}\right)}w_{2}^{\hat{L}_{2}\left(\tau_{2}\right)}\right]
%&=&\esp\left[w_{1}^{\hat{L}_{1}\left(\tau_{2}\right)}\right]
%   \esp\left[w_{2}^{\hat{L}_{2}\left(\tau_{2}\right)}\right]=\hat{F}_{1,2}\left(w_{1};\tau_{2}\right)\hat{F}_{2,2}\left(w_{2};\tau_{2}\right)=\hat{F}_{2}\left(w_{1},w_{2};\tau_{2}\right).
\end{eqnarray*}

Con respecto al sistema 1 se tiene la FGP conjunta con respecto al servidor del otro sistema:
\begin{eqnarray*}
\esp\left[z_{1}^{L_{1}\left(\zeta_{j}\right)}z_{2}^{L_{2}\left(\zeta_{j}\right)}\right]
&=&\esp\left[z_{1}^{L_{1}\left(\zeta_{j}\right)}\right]
\esp\left[z_{2}^{L_{2}\left(\zeta_{j}\right)}\right]=F_{1,j}\left(z_{1};\zeta_{j}\right)F_{2,j}\left(z_{2};\zeta_{j}\right)=F_{j}\left(z_{1},z_{2};\zeta_{j}\right).
%\esp\left[z_{1}^{L_{1}\left(\zeta_{1}\right)}z_{2}^{L_{2}\left(\zeta_{1}\right)}\right]
%&=&\esp\left[z_{1}^{L_{1}\left(\zeta_{1}\right)}\right]
%\esp\left[z_{2}^{L_{2}\left(\zeta_{1}\right)}\right]=F_{1,1}\left(z_{1};\zeta_{1}\right)F_{2,1}\left(z_{2};\zeta_{1}\right)=F_{1}\left(z_{1},z_{2};\zeta_{1}\right)\\
%\esp\left[z_{1}^{L_{1}\left(\zeta_{2}\right)}z_{2}^{L_{2}\left(\zeta_{2}\right)}\right]
%&=&\esp\left[z_{1}^{L_{1}\left(\zeta_{2}\right)}\right]
%\esp\left[z_{2}^{L_{2}\left(\zeta_{2}\right)}\right]=F_{1,2}\left(z_{1};\zeta_{2}\right)F_{2,2}\left(z_{2};\zeta_{2}\right)=F_{2}\left(z_{1},z_{2};\zeta_{2}\right).
\end{eqnarray*}

Ahora analicemos la Red de Sistemas de Visitas C\'iclicas, entonces se define la PGF conjunta al tiempo $t$ para los tiempos de visita del servidor en cada una de las colas, para comenzar a dar servicio, definidos anteriormente al tiempo
$t=\left\{\tau_{1},\tau_{2},\zeta_{1},\zeta_{2}\right\}$:

\begin{eqnarray}\label{Eq.Conjuntas}
F_{j}\left(z_{1},z_{2},w_{1},w_{2}\right)&=&\esp\left[\prod_{i=1}^{2}z_{i}^{L_{i}\left(\tau_{j}
\right)}\prod_{i=1}^{2}w_{i}^{\hat{L}_{i}\left(\tau_{j}\right)}\right]\\
\hat{F}_{j}\left(z_{1},z_{2},w_{1},w_{2}\right)&=&\esp\left[\prod_{i=1}^{2}z_{i}^{L_{i}
\left(\zeta_{j}\right)}\prod_{i=1}^{2}w_{i}^{\hat{L}_{i}\left(\zeta_{j}\right)}\right]
\end{eqnarray}
para $j=1,2$. Entonces, con la finalidad de encontrar el n\'umero de usuarios
presentes en el sistema cuando el servidor deja de atender una de
las colas de cualquier sistema se tiene lo siguiente


\begin{eqnarray*}
&&\esp\left[z_{1}^{L_{1}\left(\overline{\tau}_{1}\right)}z_{2}^{L_{2}\left(\overline{\tau}_{1}\right)}w_{1}^{\hat{L}_{1}\left(\overline{\tau}_{1}\right)}w_{2}^{\hat{L}_{2}\left(\overline{\tau}_{1}\right)}\right]=
\esp\left[z_{2}^{L_{2}\left(\overline{\tau}_{1}\right)}w_{1}^{\hat{L}_{1}\left(\overline{\tau}_{1}
\right)}w_{2}^{\hat{L}_{2}\left(\overline{\tau}_{1}\right)}\right]\\
&=&\esp\left[z_{2}^{L_{2}\left(\tau_{1}\right)+X_{2}\left(\overline{\tau}_{1}-\tau_{1}\right)+Y_{2}\left(\overline{\tau}_{1}-\tau_{1}\right)}w_{1}^{\hat{L}_{1}\left(\tau_{1}\right)+\hat{X}_{1}\left(\overline{\tau}_{1}-\tau_{1}\right)}w_{2}^{\hat{L}_{2}\left(\tau_{1}\right)+\hat{X}_{2}\left(\overline{\tau}_{1}-\tau_{1}\right)}\right]
\end{eqnarray*}
utilizando la ecuacion dada (\ref{Eq.UsuariosTotalesZ2}), luego


\begin{eqnarray*}
&=&\esp\left[z_{2}^{L_{2}\left(\tau_{1}\right)}z_{2}^{X_{2}\left(\overline{\tau}_{1}-\tau_{1}\right)}z_{2}^{Y_{2}\left(\overline{\tau}_{1}-\tau_{1}\right)}w_{1}^{\hat{L}_{1}\left(\tau_{1}\right)}w_{1}^{\hat{X}_{1}\left(\overline{\tau}_{1}-\tau_{1}\right)}w_{2}^{\hat{L}_{2}\left(\tau_{1}\right)}w_{2}^{\hat{X}_{2}\left(\overline{\tau}_{1}-\tau_{1}\right)}\right]\\
&=&\esp\left[z_{2}^{L_{2}\left(\tau_{1}\right)}\left\{w_{1}^{\hat{L}_{1}\left(\tau_{1}\right)}w_{2}^{\hat{L}_{2}\left(\tau_{1}\right)}\right\}\left\{z_{2}^{X_{2}\left(\overline{\tau}_{1}-\tau_{1}\right)}
z_{2}^{Y_{2}\left(\overline{\tau}_{1}-\tau_{1}\right)}w_{1}^{\hat{X}_{1}\left(\overline{\tau}_{1}-\tau_{1}\right)}w_{2}^{\hat{X}_{2}\left(\overline{\tau}_{1}-\tau_{1}\right)}\right\}\right]\\
\end{eqnarray*}
Aplicando el hecho de que el n\'umero de usuarios que llegan a cada una de las colas del segundo sistema es independiente de las llegadas a las colas del primer sistema:

\begin{eqnarray*}
&=&\esp\left[z_{2}^{L_{2}\left(\tau_{1}\right)}\left\{z_{2}^{X_{2}\left(\overline{\tau}_{1}-\tau_{1}\right)}z_{2}^{Y_{2}\left(\overline{\tau}_{1}-\tau_{1}\right)}w_{1}^{\hat{X}_{1}\left(\overline{\tau}_{1}-\tau_{1}\right)}w_{2}^{\hat{X}_{2}\left(\overline{\tau}_{1}-\tau_{1}\right)}\right\}\right]\esp\left[w_{1}^{\hat{L}_{1}\left(\tau_{1}\right)}w_{2}^{\hat{L}_{2}\left(\tau_{1}\right)}\right]
\end{eqnarray*}
dado que los arribos a cada una de las colas son independientes, podemos separar la esperanza para los procesos de llegada a $Q_{1}$ y $Q_{2}$ al tiempo $\tau_{1}$, que es el tiempo en que el servidor visita a $Q_{1}$. Recordando que $\tilde{X}_{2}\left(z_{2}\right)=X_{2}\left(z_{2}\right)+Y_{2}\left(z_{2}\right)$ se tiene


\begin{eqnarray*}
&=&\esp\left[z_{2}^{L_{2}\left(\tau_{1}\right)}\left\{z_{2}^{\tilde{X}_{2}\left(\overline{\tau}_{1}-\tau_{1}\right)}w_{1}^{\hat{X}_{1}\left(\overline{\tau}_{1}-\tau_{1}\right)}w_{2}^{\hat{X}_{2}\left(\overline{\tau}_{1}-\tau_{1}\right)}\right\}\right]\esp\left[w_{1}^{\hat{L}_{1}\left(\tau_{1}\right)}w_{2}^{\hat{L}_{2}\left(\tau_{1}\right)}\right]\\
&=&\esp\left[z_{2}^{L_{2}\left(\tau_{1}\right)}\left\{\tilde{P}_{2}\left(z_{2}\right)^{\overline{\tau}_{1}-\tau_{1}}\hat{P}_{1}\left(w_{1}\right)^{\overline{\tau}_{1}-\tau_{1}}\hat{P}_{2}\left(w_{2}\right)^{\overline{\tau}_{1}-\tau_{1}}\right\}\right]\esp\left[w_{1}^{\hat{L}_{1}\left(\tau_{1}\right)}w_{2}^{\hat{L}_{2}\left(\tau_{1}\right)}\right]\\
&=&\esp\left[z_{2}^{L_{2}\left(\tau_{1}\right)}\left\{\tilde{P}_{2}\left(z_{2}\right)\hat{P}_{1}\left(w_{1}\right)\hat{P}_{2}\left(w_{2}\right)\right\}^{\overline{\tau}_{1}-\tau_{1}}\right]\esp\left[w_{1}^{\hat{L}_{1}\left(\tau_{1}\right)}w_{2}^{\hat{L}_{2}\left(\tau_{1}\right)}\right]\\
&=&\esp\left[z_{2}^{L_{2}\left(\tau_{1}\right)}\theta_{1}\left(\tilde{P}_{2}\left(z_{2}\right)\hat{P}_{1}\left(w_{1}\right)\hat{P}_{2}\left(w_{2}\right)\right)^{L_{1}\left(\tau_{1}\right)}\right]\esp\left[w_{1}^{\hat{L}_{1}\left(\tau_{1}\right)}w_{2}^{\hat{L}_{2}\left(\tau_{1}\right)}\right]\\
&=&F_{1}\left(\theta_{1}\left(\tilde{P}_{2}\left(z_{2}\right)\hat{P}_{1}\left(w_{1}\right)\hat{P}_{2}\left(w_{2}\right)\right),z{2}\right)\hat{F}_{1}\left(w_{1},w_{2};\tau_{1}\right)\\
&\equiv&
F_{1}\left(\theta_{1}\left(\tilde{P}_{2}\left(z_{2}\right)\hat{P}_{1}\left(w_{1}\right)\hat{P}_{2}\left(w_{2}\right)\right),z_{2},w_{1},w_{2}\right)
\end{eqnarray*}

Las igualdades anteriores son ciertas pues el n\'umero de usuarios
que llegan a $\hat{Q}_{2}$ durante el intervalo
$\left[\tau_{1},\overline{\tau}_{1}\right]$ a\'un no han sido
atendidos por el servidor del sistema $2$ y por tanto a\'un no
pueden pasar al sistema $1$ a traves de $Q_{2}$. Por tanto el n\'umero de
usuarios que pasan de $\hat{Q}_{2}$ a $Q_{2}$ en el intervalo de
tiempo $\left[\tau_{1},\overline{\tau}_{1}\right]$ depende de la
pol\'itica de traslado entre los dos sistemas, conforme a la
secci\'on anterior.\smallskip

Por lo tanto
\begin{eqnarray}\label{Eq.Fs}
\esp\left[z_{1}^{L_{1}\left(\overline{\tau}_{1}\right)}z_{2}^{L_{2}\left(\overline{\tau}_{1}
\right)}w_{1}^{\hat{L}_{1}\left(\overline{\tau}_{1}\right)}w_{2}^{\hat{L}_{2}\left(
\overline{\tau}_{1}\right)}\right]&=&F_{1}\left(\theta_{1}\left(\tilde{P}_{2}\left(z_{2}\right)
\hat{P}_{1}\left(w_{1}\right)\hat{P}_{2}\left(w_{2}\right)\right),z_{2},w_{1},w_{2}\right)\\
&=&F_{1}\left(\theta_{1}\left(\tilde{P}_{2}\left(z_{2}\right)\hat{P}_{1}\left(w_{1}\right)\hat{P}_{2}\left(w_{2}\right)\right),z{2}\right)\hat{F}_{1}\left(w_{1},w_{2};\tau_{1}\right)
\end{eqnarray}


Utilizando un razonamiento an\'alogo para $\overline{\tau}_{2}$:



\begin{eqnarray*}
&&\esp\left[z_{1}^{L_{1}\left(\overline{\tau}_{2}\right)}z_{2}^{L_{2}\left(\overline{\tau}_{2}\right)}w_{1}^{\hat{L}_{1}\left(\overline{\tau}_{2}\right)}w_{2}^{\hat{L}_{2}\left(\overline{\tau}_{2}\right)}\right]=
\esp\left[z_{1}^{L_{1}\left(\overline{\tau}_{2}\right)}w_{1}^{\hat{L}_{1}\left(\overline{\tau}_{2}\right)}w_{2}^{\hat{L}_{2}\left(\overline{\tau}_{2}\right)}\right]\\
&=&\esp\left[z_{1}^{L_{1}\left(\tau_{2}\right)+X_{1}\left(\overline{\tau}_{2}-\tau_{2}\right)}w_{1}^{\hat{L}_{1}\left(\tau_{2}\right)+\hat{X}_{1}\left(\overline{\tau}_{2}-\tau_{2}\right)}w_{2}^{\hat{L}_{2}\left(\tau_{2}\right)+\hat{X}_{2}\left(\overline{\tau}_{2}-\tau_{2}\right)}\right]\\
&=&\esp\left[z_{1}^{L_{1}\left(\tau_{2}\right)}z_{1}^{X_{1}\left(\overline{\tau}_{2}-\tau_{2}\right)}w_{1}^{\hat{L}_{1}\left(\tau_{2}\right)}w_{1}^{\hat{X}_{1}\left(\overline{\tau}_{2}-\tau_{2}\right)}w_{2}^{\hat{L}_{2}\left(\tau_{2}\right)}w_{2}^{\hat{X}_{2}\left(\overline{\tau}_{2}-\tau_{2}\right)}\right]\\
&=&\esp\left[z_{1}^{L_{1}\left(\tau_{2}\right)}z_{1}^{X_{1}\left(\overline{\tau}_{2}-\tau_{2}\right)}w_{1}^{\hat{X}_{1}\left(\overline{\tau}_{2}-\tau_{2}\right)}w_{2}^{\hat{X}_{2}\left(\overline{\tau}_{2}-\tau_{2}\right)}\right]\esp\left[w_{1}^{\hat{L}_{1}\left(\tau_{2}\right)}w_{2}^{\hat{L}_{2}\left(\tau_{2}\right)}\right]\\
&=&\esp\left[z_{1}^{L_{1}\left(\tau_{2}\right)}P_{1}\left(z_{1}\right)^{\overline{\tau}_{2}-\tau_{2}}\hat{P}_{1}\left(w_{1}\right)^{\overline{\tau}_{2}-\tau_{2}}\hat{P}_{2}\left(w_{2}\right)^{\overline{\tau}_{2}-\tau_{2}}\right]
\esp\left[w_{1}^{\hat{L}_{1}\left(\tau_{2}\right)}w_{2}^{\hat{L}_{2}\left(\tau_{2}\right)}\right]
\end{eqnarray*}
utlizando la proposici\'on (\ref{Prop.1.1.2S}) referente al problema de la ruina del jugador:


\begin{eqnarray*}
&=&\esp\left[z_{1}^{L_{1}\left(\tau_{2}\right)}\left\{P_{1}\left(z_{1}\right)\hat{P}_{1}\left(w_{1}\right)\hat{P}_{2}\left(w_{2}\right)\right\}^{\overline{\tau}_{2}-\tau_{2}}\right]
\esp\left[w_{1}^{\hat{L}_{1}\left(\tau_{2}\right)}w_{2}^{\hat{L}_{2}\left(\tau_{2}\right)}\right]\\
&=&\esp\left[z_{1}^{L_{1}\left(\tau_{2}\right)}\tilde{\theta}_{2}\left(P_{1}\left(z_{1}\right)\hat{P}_{1}\left(w_{1}\right)\hat{P}_{2}\left(w_{2}\right)\right)^{L_{2}\left(\tau_{2}\right)}\right]
\esp\left[w_{1}^{\hat{L}_{1}\left(\tau_{2}\right)}w_{2}^{\hat{L}_{2}\left(\tau_{2}\right)}\right]\\
&=&F_{2}\left(z_{1},\tilde{\theta}_{2}\left(P_{1}\left(z_{1}\right)\hat{P}_{1}\left(w_{1}\right)\hat{P}_{2}\left(w_{2}\right)\right)\right)
\hat{F}_{2}\left(w_{1},w_{2};\tau_{2}\right)\\
\end{eqnarray*}


entonces se define
\begin{eqnarray}
\esp\left[z_{1}^{L_{1}\left(\overline{\tau}_{2}\right)}z_{2}^{L_{2}\left(\overline{\tau}_{2}\right)}w_{1}^{\hat{L}_{1}\left(\overline{\tau}_{2}\right)}w_{2}^{\hat{L}_{2}\left(\overline{\tau}_{2}\right)}\right]=F_{2}\left(z_{1},\tilde{\theta}_{2}\left(P_{1}\left(z_{1}\right)\hat{P}_{1}\left(w_{1}\right)\hat{P}_{2}\left(w_{2}\right)\right),w_{1},w_{2}\right)\\
\equiv F_{2}\left(z_{1},\tilde{\theta}_{2}\left(P_{1}\left(z_{1}\right)\hat{P}_{1}\left(w_{1}\right)\hat{P}_{2}\left(w_{2}\right)\right)\right)
\hat{F}_{2}\left(w_{1},w_{2};\tau_{2}\right)
\end{eqnarray}

Ahora para $\overline{\zeta}_{1}:$

\begin{eqnarray*}
&&\esp\left[z_{1}^{L_{1}\left(\overline{\zeta}_{1}\right)}z_{2}^{L_{2}\left(\overline{\zeta}_{1}\right)}w_{1}^{\hat{L}_{1}\left(\overline{\zeta}_{1}\right)}w_{2}^{\hat{L}_{2}\left(\overline{\zeta}_{1}\right)}\right]=
\esp\left[z_{1}^{L_{1}\left(\overline{\zeta}_{1}\right)}z_{2}^{L_{2}\left(\overline{\zeta}_{1}\right)}w_{2}^{\hat{L}_{2}\left(\overline{\zeta}_{1}\right)}\right]\\
%&=&\esp\left[z_{1}^{L_{1}\left(\zeta_{1}\right)+X_{1}\left(\overline{\zeta}_{1}-\zeta_{1}\right)}z_{2}^{L_{2}\left(\zeta_{1}\right)+X_{2}\left(\overline{\zeta}_{1}-\zeta_{1}\right)+\hat{Y}_{2}\left(\overline{\zeta}_{1}-\zeta_{1}\right)}w_{2}^{\hat{L}_{2}\left(\zeta_{1}\right)+\hat{X}_{2}\left(\overline{\zeta}_{1}-\zeta_{1}\right)}\right]\\
&=&\esp\left[z_{1}^{L_{1}\left(\zeta_{1}\right)}z_{1}^{X_{1}\left(\overline{\zeta}_{1}-\zeta_{1}\right)}z_{2}^{L_{2}\left(\zeta_{1}\right)}z_{2}^{X_{2}\left(\overline{\zeta}_{1}-\zeta_{1}\right)}
z_{2}^{Y_{2}\left(\overline{\zeta}_{1}-\zeta_{1}\right)}w_{2}^{\hat{L}_{2}\left(\zeta_{1}\right)}w_{2}^{\hat{X}_{2}\left(\overline{\zeta}_{1}-\zeta_{1}\right)}\right]\\
&=&\esp\left[z_{1}^{L_{1}\left(\zeta_{1}\right)}z_{2}^{L_{2}\left(\zeta_{1}\right)}\right]\esp\left[z_{1}^{X_{1}\left(\overline{\zeta}_{1}-\zeta_{1}\right)}z_{2}^{\tilde{X}_{2}\left(\overline{\zeta}_{1}-\zeta_{1}\right)}w_{2}^{\hat{X}_{2}\left(\overline{\zeta}_{1}-\zeta_{1}\right)}w_{2}^{\hat{L}_{2}\left(\zeta_{1}\right)}\right]\\
&=&\esp\left[z_{1}^{L_{1}\left(\zeta_{1}\right)}z_{2}^{L_{2}\left(\zeta_{1}\right)}\right]
\esp\left[P_{1}\left(z_{1}\right)^{\overline{\zeta}_{1}-\zeta_{1}}\tilde{P}_{2}\left(z_{2}\right)^{\overline{\zeta}_{1}-\zeta_{1}}\hat{P}_{2}\left(w_{2}\right)^{\overline{\zeta}_{1}-\zeta_{1}}w_{2}^{\hat{L}_{2}\left(\zeta_{1}\right)}\right]\\
&=&\esp\left[z_{1}^{L_{1}\left(\zeta_{1}\right)}z_{2}^{L_{2}\left(\zeta_{1}\right)}\right]
\esp\left[\left\{P_{1}\left(z_{1}\right)\tilde{P}_{2}\left(z_{2}\right)\hat{P}_{2}\left(w_{2}\right)\right\}^{\overline{\zeta}_{1}-\zeta_{1}}w_{2}^{\hat{L}_{2}\left(\zeta_{1}\right)}\right]\\
&=&\esp\left[z_{1}^{L_{1}\left(\zeta_{1}\right)}z_{2}^{L_{2}\left(\zeta_{1}\right)}\right]
\esp\left[\hat{\theta}_{1}\left(P_{1}\left(z_{1}\right)\tilde{P}_{2}\left(z_{2}\right)\hat{P}_{2}\left(w_{2}\right)\right)^{\hat{L}_{1}\left(\zeta_{1}\right)}w_{2}^{\hat{L}_{2}\left(\zeta_{1}\right)}\right]\\
&=&F_{1}\left(z_{1},z_{2};\zeta_{1}\right)\hat{F}_{1}\left(\hat{\theta}_{1}\left(P_{1}\left(z_{1}\right)\tilde{P}_{2}\left(z_{2}\right)\hat{P}_{2}\left(w_{2}\right)\right),w_{2}\right)
\end{eqnarray*}


es decir,

\begin{eqnarray}
\esp\left[z_{1}^{L_{1}\left(\overline{\zeta}_{1}\right)}z_{2}^{L_{2}\left(\overline{\zeta}_{1}
\right)}w_{1}^{\hat{L}_{1}\left(\overline{\zeta}_{1}\right)}w_{2}^{\hat{L}_{2}\left(
\overline{\zeta}_{1}\right)}\right]&=&\hat{F}_{1}\left(z_{1},z_{2},\hat{\theta}_{1}\left(P_{1}\left(z_{1}\right)\tilde{P}_{2}\left(z_{2}\right)\hat{P}_{2}\left(w_{2}\right)\right),w_{2}\right)\\
&=&F_{1}\left(z_{1},z_{2};\zeta_{1}\right)\hat{F}_{1}\left(\hat{\theta}_{1}\left(P_{1}\left(z_{1}\right)\tilde{P}_{2}\left(z_{2}\right)\hat{P}_{2}\left(w_{2}\right)\right),w_{2}\right).
\end{eqnarray}


Finalmente para $\overline{\zeta}_{2}:$
\begin{eqnarray*}
&&\esp\left[z_{1}^{L_{1}\left(\overline{\zeta}_{2}\right)}z_{2}^{L_{2}\left(\overline{\zeta}_{2}\right)}w_{1}^{\hat{L}_{1}\left(\overline{\zeta}_{2}\right)}w_{2}^{\hat{L}_{2}\left(\overline{\zeta}_{2}\right)}\right]=
\esp\left[z_{1}^{L_{1}\left(\overline{\zeta}_{2}\right)}z_{2}^{L_{2}\left(\overline{\zeta}_{2}\right)}w_{1}^{\hat{L}_{1}\left(\overline{\zeta}_{2}\right)}\right]\\
%&=&\esp\left[z_{1}^{L_{1}\left(\zeta_{2}\right)+X_{1}\left(\overline{\zeta}_{2}-\zeta_{2}\right)}z_{2}^{L_{2}\left(\zeta_{2}\right)+X_{2}\left(\overline{\zeta}_{2}-\zeta_{2}\right)+\hat{Y}_{2}\left(\overline{\zeta}_{2}-\zeta_{2}\right)}w_{1}^{\hat{L}_{1}\left(\zeta_{2}\right)+\hat{X}_{1}\left(\overline{\zeta}_{2}-\zeta_{2}\right)}\right]\\
&=&\esp\left[z_{1}^{L_{1}\left(\zeta_{2}\right)}z_{1}^{X_{1}\left(\overline{\zeta}_{2}-\zeta_{2}\right)}z_{2}^{L_{2}\left(\zeta_{2}\right)}z_{2}^{X_{2}\left(\overline{\zeta}_{2}-\zeta_{2}\right)}
z_{2}^{Y_{2}\left(\overline{\zeta}_{2}-\zeta_{2}\right)}w_{1}^{\hat{L}_{1}\left(\zeta_{2}\right)}w_{1}^{\hat{X}_{1}\left(\overline{\zeta}_{2}-\zeta_{2}\right)}\right]\\
&=&\esp\left[z_{1}^{L_{1}\left(\zeta_{2}\right)}z_{2}^{L_{2}\left(\zeta_{2}\right)}\right]\esp\left[z_{1}^{X_{1}\left(\overline{\zeta}_{2}-\zeta_{2}\right)}z_{2}^{\tilde{X}_{2}\left(\overline{\zeta}_{2}-\zeta_{2}\right)}w_{1}^{\hat{X}_{1}\left(\overline{\zeta}_{2}-\zeta_{2}\right)}w_{1}^{\hat{L}_{1}\left(\zeta_{2}\right)}\right]\\
&=&\esp\left[z_{1}^{L_{1}\left(\zeta_{2}\right)}z_{2}^{L_{2}\left(\zeta_{2}\right)}\right]\esp\left[P_{1}\left(z_{1}\right)^{\overline{\zeta}_{2}-\zeta_{2}}\tilde{P}_{2}\left(z_{2}\right)^{\overline{\zeta}_{2}-\zeta_{2}}\hat{P}\left(w_{1}\right)^{\overline{\zeta}_{2}-\zeta_{2}}w_{1}^{\hat{L}_{1}\left(\zeta_{2}\right)}\right]\\
&=&\esp\left[z_{1}^{L_{1}\left(\zeta_{2}\right)}z_{2}^{L_{2}\left(\zeta_{2}\right)}\right]\esp\left[w_{1}^{\hat{L}_{1}\left(\zeta_{2}\right)}\left\{P_{1}\left(z_{1}\right)\tilde{P}_{2}\left(z_{2}\right)\hat{P}\left(w_{1}\right)\right\}^{\overline{\zeta}_{2}-\zeta_{2}}\right]\\
&=&\esp\left[z_{1}^{L_{1}\left(\zeta_{2}\right)}z_{2}^{L_{2}\left(\zeta_{2}\right)}\right]\esp\left[w_{1}^{\hat{L}_{1}\left(\zeta_{2}\right)}\hat{\theta}_{2}\left(P_{1}\left(z_{1}\right)\tilde{P}_{2}\left(z_{2}\right)\hat{P}\left(w_{1}\right)\right)^{\hat{L}_{2}\zeta_{2}}\right]\\
&=&F_{2}\left(z_{1},z_{2};\zeta_{2}\right)\hat{F}_{2}\left(w_{1},\hat{\theta}_{2}\left(P_{1}\left(z_{1}\right)\tilde{P}_{2}\left(z_{2}\right)\hat{P}_{1}\left(w_{1}\right)\right)\right]\\
%&\equiv&\hat{F}_{2}\left(z_{1},z_{2},w_{1},\hat{\theta}_{2}\left(P_{1}\left(z_{1}\right)\tilde{P}_{2}\left(z_{2}\right)\hat{P}_{1}\left(w_{1}\right)\right)\right)
\end{eqnarray*}

es decir
\begin{eqnarray}
\esp\left[z_{1}^{L_{1}\left(\overline{\zeta}_{2}\right)}z_{2}^{L_{2}\left(\overline{\zeta}_{2}\right)}w_{1}^{\hat{L}_{1}\left(\overline{\zeta}_{2}\right)}w_{2}^{\hat{L}_{2}\left(\overline{\zeta}_{2}\right)}\right]=\hat{F}_{2}\left(z_{1},z_{2},w_{1},\hat{\theta}_{2}\left(P_{1}\left(z_{1}\right)\tilde{P}_{2}\left(z_{2}\right)\hat{P}_{1}\left(w_{1}\right)\right)\right)\\
=F_{2}\left(z_{1},z_{2};\zeta_{2}\right)\hat{F}_{2}\left(w_{1},\hat{\theta}_{2}\left(P_{1}\left(z_{1}\right)\tilde{P}_{2}\left(z_{2}\right)\hat{P}_{1}\left(w_{1}
\right)\right)\right)
\end{eqnarray}
%__________________________________________________________________________
\section{Ecuaciones Recursivas para la R.S.V.C.}
%__________________________________________________________________________




Con lo desarrollado hasta ahora podemos encontrar las ecuaciones
recursivas que modelan la Red de Sistemas de Visitas C\'iclicas
(R.S.V.C):
\begin{eqnarray*}
F_{2}\left(z_{1},z_{2},w_{1},w_{2}\right)&=&R_{1}\left(z_{1},z_{2},w_{1},w_{2}\right)\esp\left[z_{1}^{L_{1}\left(
\overline{\tau}_{1}\right)}z_{2}^{L_{2}\left(\overline{\tau}_{1}\right)}w_{1}^{\hat{L}_{1}\left(\overline{\tau}_{1}\right)}
w_{2}^{\hat{L}_{2}\left(\overline{\tau}_{1}\right)}\right]\\
&=&R_{1}\left(P_{1}\left(z_{1}\right)\tilde{P}_{2}\left(z_{2}\right)\prod_{i=1}^{2}
\hat{P}_{i}\left(w_{i}\right)\right)F_{1}\left(\theta_{1}\left(\tilde{P}_{2}\left(z_{2}\right)\hat{P}_{1}\left(w_{1}
\right)\hat{P}_{2}\left(w_{2}\right)\right),z_{2},w_{1},w_{2}\right)
\end{eqnarray*}


\begin{eqnarray*}
F_{1}\left(z_{1},z_{2},w_{1},w_{2}\right)&=&R_{2}\left(z_{1},z_{2},w_{1},w_{2}\right)\esp\left[z_{1}^{L_{1}\left(
\overline{\tau}_{2}\right)}z_{2}^{L_{2}\left(\overline{\tau}_{2}\right)}w_{1}^{\hat{L}_{1}\left(\overline{\tau}_{2}\right)}
w_{2}^{\hat{L}_{2}\left(\overline{\tau}_{1}\right)}\right]\\
&=&R_{2}\left(P_{1}\left(z_{1}\right)\tilde{P}_{2}\left(z_{2}\right)\prod_{i=1}^{2}
\hat{P}_{i}\left(w_{i}\right)\right)F_{2}\left(z_{1},\tilde{\theta}_{2}\left(P_{1}\left(z_{1}\right)\hat{P}_{1}\left(w_{1}\right)\hat{P}_{2}\left(w_{2}\right)\right),w_{1},w_{2}\right)\\
\end{eqnarray*}




que son equivalentes a las siguientes ecuaciones
\begin{eqnarray*}
&&\hat{F}_{2}\left(z_{1},z_{2},w_{1},w_{2}\right)=\hat{R}_{1}\left(z_{1},z_{2},w_{1},w_{2}\right)\esp\left[z_{1}^{L_{1}\left(\overline{\zeta}_{1}\right)}z_{2}^{L_{2}\left(\overline{\zeta}_{1}\right)}w_{1}^{\hat{L}_{1}\left(\overline{\zeta}_{1}\right)}w_{2}^{\hat{L}_{2}\left(\overline{\zeta}_{1}\right)}\right]\\
&&\hat{F}_{1}\left(z_{1},z_{2},w_{1},w_{2}\right)=\hat{R}_{2}\left(z_{1},z_{2},
w_{1},w_{2}\right)\esp\left[z_{1}^{L_{1}\left(\overline{\zeta}_{2}\right)}z_{2}
^{L_{2}\left(\overline{\zeta}_{2}\right)}w_{1}^{\hat{L}_{1}\left(
\overline{\zeta}_{2}\right)}w_{2}^{\hat{L}_{2}\left(\overline{\zeta}_{2}\right)}
\right]
\end{eqnarray*}


que son equivalentes a las siguientes ecuaciones
\begin{eqnarray}
\hat{F}_{2}\left(z_{1},z_{2},w_{1},w_{2}\right)&=&\hat{R}_{1}\left(P_{1}\left(z_{1}\right)\tilde{P}_{2}\left(z_{2}\right)\prod_{i=1}^{2}
\hat{P}_{i}\left(w_{i}\right)\right)\hat{F}_{1}\left(z_{1},z_{2},\hat{\theta}_{1}\left(P_{1}\left(z_{1}\right)\tilde{P}_{2}\left(z_{2}\right)\hat{P}_{2}\left(w_{2}\right)\right),w_{2}\right)\\
\hat{F}_{1}\left(z_{1},z_{2},w_{1},w_{2}\right)&=&\hat{R}_{2}\left(P_{1}\left(z_{1}\right)\tilde{P}_{2}\left(z_{2}\right)\prod_{i=1}^{2}
\hat{P}_{i}\left(w_{i}\right)\right)\hat{F}_{2}\left(z_{1},z_{2},w_{1},\hat{\theta}_{2}\left(P_{1}\left(z_{1}\right)\tilde{P}_{2}\left(z_{2}\right)
\hat{P}_{1}\left(w_{1}\right)\right)\right)
\end{eqnarray}



%_________________________________________________________________________________________________
\subsection{Tiempos de Traslado del Servidor}
%_________________________________________________________________________________________________


Para
%\begin{multicols}{2}

\begin{eqnarray}\label{Ec.R1}
R_{1}\left(\mathbf{z,w}\right)=R_{1}\left((P_{1}\left(z_{1}\right)\tilde{P}_{2}\left(z_{2}\right)\hat{P}_{1}\left(w_{1}\right)\hat{P}_{2}\left(w_{2}\right)\right)
\end{eqnarray}
%\end{multicols}

se tiene que


\begin{eqnarray*}
\begin{array}{cc}
\frac{\partial R_{1}\left(\mathbf{z,w}\right)}{\partial
z_{1}}|_{\mathbf{z,w}=1}=R_{1}^{(1)}\left(1\right)P_{1}^{(1)}\left(1\right)=r_{1}\mu_{1},&
\frac{\partial R_{1}\left(\mathbf{z,w}\right)}{\partial
z_{2}}|_{\mathbf{z,w}=1}=R_{1}^{(1)}\left(1\right)\tilde{P}_{2}^{(1)}\left(1\right)=r_{1}\tilde{\mu}_{2},\\
\frac{\partial R_{1}\left(\mathbf{z,w}\right)}{\partial
w_{1}}|_{\mathbf{z,w}=1}=R_{1}^{(1)}\left(1\right)\hat{P}_{1}^{(1)}\left(1\right)=r_{1}\hat{\mu}_{1},&
\frac{\partial R_{1}\left(\mathbf{z,w}\right)}{\partial
w_{2}}|_{\mathbf{z,w}=1}=R_{1}^{(1)}\left(1\right)\hat{P}_{2}^{(1)}\left(1\right)=r_{1}\hat{\mu}_{2},
\end{array}
\end{eqnarray*}

An\'alogamente se tiene

\begin{eqnarray}
R_{2}\left(\mathbf{z,w}\right)=R_{2}\left(P_{1}\left(z_{1}\right)\tilde{P}_{2}\left(z_{2}\right)\hat{P}_{1}\left(w_{1}\right)\hat{P}_{2}\left(w_{2}\right)\right)
\end{eqnarray}


\begin{eqnarray*}
\begin{array}{cc}
\frac{\partial R_{2}\left(\mathbf{z,w}\right)}{\partial
z_{1}}|_{\mathbf{z,w}=1}=R_{2}^{(1)}\left(1\right)P_{1}^{(1)}\left(1\right)=r_{2}\mu_{1},&
\frac{\partial R_{2}\left(\mathbf{z,w}\right)}{\partial
z_{2}}|_{\mathbf{z,w}=1}=R_{2}^{(1)}\left(1\right)\tilde{P}_{2}^{(1)}\left(1\right)=r_{2}\tilde{\mu}_{2},\\
\frac{\partial R_{2}\left(\mathbf{z,w}\right)}{\partial
w_{1}}|_{\mathbf{z,w}=1}=R_{2}^{(1)}\left(1\right)\hat{P}_{1}^{(1)}\left(1\right)=r_{2}\hat{\mu}_{1},&
\frac{\partial R_{2}\left(\mathbf{z,w}\right)}{\partial
w_{2}}|_{\mathbf{z,w}=1}=R_{2}^{(1)}\left(1\right)\hat{P}_{2}^{(1)}\left(1\right)=r_{2}\hat{\mu}_{2},\\
\end{array}
\end{eqnarray*}

Para el segundo sistema:

\begin{eqnarray}
\hat{R}_{1}\left(\mathbf{z,w}\right)=\hat{R}_{1}\left(P_{1}\left(z_{1}\right)\tilde{P}_{2}\left(z_{2}\right)\hat{P}_{1}\left(w_{1}\right)\hat{P}_{2}\left(w_{2}\right)\right)
\end{eqnarray}


\begin{eqnarray*}
\frac{\partial \hat{R}_{1}\left(\mathbf{z,w}\right)}{\partial
z_{1}}|_{\mathbf{z,w}=1}=\hat{R}_{1}^{(1)}\left(1\right)P_{1}^{(1)}\left(1\right)=\hat{r}_{1}\mu_{1},&
\frac{\partial \hat{R}_{1}\left(\mathbf{z,w}\right)}{\partial
z_{2}}|_{\mathbf{z,w}=1}=\hat{R}_{1}^{(1)}\left(1\right)\tilde{P}_{2}^{(1)}\left(1\right)=\hat{r}_{1}\tilde{\mu}_{2},\\
\frac{\partial \hat{R}_{1}\left(\mathbf{z,w}\right)}{\partial
w_{1}}|_{\mathbf{z,w}=1}=\hat{R}_{1}^{(1)}\left(1\right)\hat{P}_{1}^{(1)}\left(1\right)=\hat{r}_{1}\hat{\mu}_{1},&
\frac{\partial \hat{R}_{1}\left(\mathbf{z,w}\right)}{\partial
w_{2}}|_{\mathbf{z,w}=1}=\hat{R}_{1}^{(1)}\left(1\right)\hat{P}_{2}^{(1)}\left(1\right)=\hat{r}_{1}\hat{\mu}_{2},
\end{eqnarray*}

Finalmente

\begin{eqnarray}
\hat{R}_{2}\left(\mathbf{z,w}\right)=\hat{R}_{2}\left(P_{1}\left(z_{1}\right)\tilde{P}_{2}\left(z_{2}\right)\hat{P}_{1}\left(w_{1}\right)\hat{P}_{2}\left(w_{2}\right)\right)
\end{eqnarray}



\begin{eqnarray*}
\frac{\partial \hat{R}_{2}\left(\mathbf{z,w}\right)}{\partial
z_{1}}|_{\mathbf{z,w}=1}=\hat{R}_{2}^{(1)}\left(1\right)P_{1}^{(1)}\left(1\right)=\hat{r}_{2}\mu_{1},&
\frac{\partial \hat{R}_{2}\left(\mathbf{z,w}\right)}{\partial
z_{2}}|_{\mathbf{z,w}=1}=\hat{R}_{2}^{(1)}\left(1\right)\tilde{P}_{2}^{(1)}\left(1\right)=\hat{r}_{2}\tilde{\mu}_{2},\\
\frac{\partial \hat{R}_{2}\left(\mathbf{z,w}\right)}{\partial
w_{1}}|_{\mathbf{z,w}=1}=\hat{R}_{2}^{(1)}\left(1\right)\hat{P}_{1}^{(1)}\left(1\right)=\hat{r}_{2}\hat{\mu}_{1},&
\frac{\partial \hat{R}_{2}\left(\mathbf{z,w}\right)}{\partial
w_{2}}|_{\mathbf{z,w}=1}=\hat{R}_{2}^{(1)}\left(1\right)\hat{P}_{2}^{(1)}\left(1\right)
=\hat{r}_{2}\hat{\mu}_{2}.
\end{eqnarray*}


%_________________________________________________________________________________________________
\subsection{Usuarios presentes en la cola}
%_________________________________________________________________________________________________

Hagamos lo correspondiente con las siguientes
expresiones obtenidas en la secci\'on anterior:
Recordemos que

\begin{eqnarray*}
F_{1}\left(\theta_{1}\left(\tilde{P}_{2}\left(z_{2}\right)\hat{P}_{1}\left(w_{1}\right)
\hat{P}_{2}\left(w_{2}\right)\right),z_{2},w_{1},w_{2}\right)=
F_{1}\left(\theta_{1}\left(\tilde{P}_{2}\left(z_{2}\right)\hat{P}_{1}\left(w_{1}
\right)\hat{P}_{2}\left(w_{2}\right)\right),z_{2}\right)
\hat{F}_{1}\left(w_{1},w_{2};\tau_{1}\right)
\end{eqnarray*}

entonces

\begin{eqnarray*}
\frac{\partial F_{1}\left(\theta_{1}\left(\tilde{P}_{2}\left(z_{2}\right)\hat{P}_{1}\left(w_{1}\right)\hat{P}_{2}\left(w_{2}\right)\right),z_{2},w_{1},w_{2}\right)}{\partial z_{1}}|_{\mathbf{z},\mathbf{w}=1}&=&0\\
\frac{\partial
F_{1}\left(\theta_{1}\left(\tilde{P}_{2}\left(z_{2}\right)\hat{P}_{1}\left(w_{1}\right)\hat{P}_{2}\left(w_{2}\right)\right),z_{2},w_{1},w_{2}\right)}{\partial
z_{2}}|_{\mathbf{z},\mathbf{w}=1}&=&\frac{\partial F_{1}}{\partial
z_{1}}\cdot\frac{\partial \theta_{1}}{\partial
\tilde{P}_{2}}\cdot\frac{\partial \tilde{P}_{2}}{\partial
z_{2}}+\frac{\partial F_{1}}{\partial z_{2}}
\\
\frac{\partial
F_{1}\left(\theta_{1}\left(\tilde{P}_{2}\left(z_{2}\right)\hat{P}_{1}\left(w_{1}\right)\hat{P}_{2}\left(w_{2}\right)\right),z_{2},w_{1},w_{2}\right)}{\partial
w_{1}}|_{\mathbf{z},\mathbf{w}=1}&=&\frac{\partial F_{1}}{\partial
z_{1}}\cdot\frac{\partial
\theta_{1}}{\partial\hat{P}_{1}}\cdot\frac{\partial\hat{P}_{1}}{\partial
w_{1}}+\frac{\partial\hat{F}_{1}}{\partial w_{1}}
\\
\frac{\partial
F_{1}\left(\theta_{1}\left(\tilde{P}_{2}\left(z_{2}\right)\hat{P}_{1}\left(w_{1}\right)\hat{P}_{2}\left(w_{2}\right)\right),z_{2},w_{1},w_{2}\right)}{\partial
w_{2}}|_{\mathbf{z},\mathbf{w}=1}&=&\frac{\partial F_{1}}{\partial
z_{1}}\cdot\frac{\partial\theta_{1}}{\partial\hat{P}_{2}}\cdot\frac{\partial\hat{P}_{2}}{\partial
w_{2}}+\frac{\partial \hat{F}_{1}}{\partial w_{2}}
\\
\end{eqnarray*}

para $\tau_{2}$:

\begin{eqnarray*}
F_{2}\left(z_{1},\tilde{\theta}_{2}\left(P_{1}\left(z_{1}\right)\hat{P}_{1}\left(w_{1}\right)\hat{P}_{2}\left(w_{2}\right)\right),
w_{1},w_{2}\right)=F_{2}\left(z_{1},\tilde{\theta}_{2}\left(P_{1}\left(z_{1}\right)\hat{P}_{1}\left(w_{1}\right)
\hat{P}_{2}\left(w_{2}\right)\right)\right)\hat{F}_{2}\left(w_{1},w_{2};\tau_{2}\right)
\end{eqnarray*}
al igual que antes

\begin{eqnarray*}
\frac{\partial
F_{2}\left(z_{1},\tilde{\theta}_{2}\left(P_{1}\left(z_{1}\right)\hat{P}_{1}\left(w_{1}\right)\hat{P}_{2}\left(w_{2}\right)\right),w_{1},w_{2}\right)}{\partial
z_{1}}|_{\mathbf{z},\mathbf{w}=1}&=&\frac{\partial F_{2}}{\partial
z_{2}}\cdot\frac{\partial\tilde{\theta}_{2}}{\partial
P_{1}}\cdot\frac{\partial P_{1}}{\partial z_{2}}+\frac{\partial
F_{2}}{\partial z_{1}}
\\
\frac{\partial F_{2}\left(z_{1},\tilde{\theta}_{2}\left(P_{1}\left(z_{1}\right)\hat{P}_{1}\left(w_{1}\right)\hat{P}_{2}\left(w_{2}\right)\right),w_{1},w_{2}\right)}{\partial z_{2}}|_{\mathbf{z},\mathbf{w}=1}&=&0\\
\frac{\partial
F_{2}\left(z_{1},\tilde{\theta}_{2}\left(P_{1}\left(z_{1}\right)\hat{P}_{1}\left(w_{1}\right)\hat{P}_{2}\left(w_{2}\right)\right),w_{1},w_{2}\right)}{\partial
w_{1}}|_{\mathbf{z},\mathbf{w}=1}&=&\frac{\partial F_{2}}{\partial
z_{2}}\cdot\frac{\partial \tilde{\theta}_{2}}{\partial
\hat{P}_{1}}\cdot\frac{\partial \hat{P}_{1}}{\partial
w_{1}}+\frac{\partial \hat{F}_{2}}{\partial w_{1}}
\\
\frac{\partial
F_{2}\left(z_{1},\tilde{\theta}_{2}\left(P_{1}\left(z_{1}\right)\hat{P}_{1}\left(w_{1}\right)\hat{P}_{2}\left(w_{2}\right)\right),w_{1},w_{2}\right)}{\partial
w_{2}}|_{\mathbf{z},\mathbf{w}=1}&=&\frac{\partial F_{2}}{\partial
z_{2}}\cdot\frac{\partial
\tilde{\theta}_{2}}{\partial\hat{P}_{2}}\cdot\frac{\partial\hat{P}_{2}}{\partial
w_{2}}+\frac{\partial\hat{F}_{2}}{\partial w_{2}}
\\
\end{eqnarray*}


Ahora para el segundo sistema

\begin{eqnarray*}\hat{F}_{1}\left(z_{1},z_{2},\hat{\theta}_{1}\left(P_{1}\left(z_{1}\right)\tilde{P}_{2}\left(z_{2}\right)\hat{P}_{2}\left(w_{2}\right)\right),
w_{2}\right)=F_{1}\left(z_{1},z_{2};\zeta_{1}\right)\hat{F}_{1}\left(\hat{\theta}_{1}\left(P_{1}\left(z_{1}\right)\tilde{P}_{2}\left(z_{2}\right)
\hat{P}_{2}\left(w_{2}\right)\right),w_{2}\right)
\end{eqnarray*}
entonces


\begin{eqnarray*}
\frac{\partial
\hat{F}_{1}\left(z_{1},z_{2},\hat{\theta}_{1}\left(P_{1}\left(z_{1}\right)\tilde{P}_{2}\left(z_{2}\right)\hat{P}_{2}\left(w_{2}\right)\right),w_{2}\right)}{\partial
z_{1}}|_{\mathbf{z},\mathbf{w}=1}&=&\frac{\partial \hat{F}_{1}
}{\partial w_{1}}\cdot\frac{\partial\hat{\theta}_{1}}{\partial
P_{1}}\cdot\frac{\partial P_{1}}{\partial z_{1}}+\frac{\partial
F_{1}}{\partial z_{1}}
\\
\frac{\partial
\hat{F}_{1}\left(z_{1},z_{2},\hat{\theta}_{1}\left(P_{1}\left(z_{1}\right)\tilde{P}_{2}\left(z_{2}\right)\hat{P}_{2}\left(w_{2}\right)\right),w_{2}\right)}{\partial
z_{2}}|_{\mathbf{z},\mathbf{w}=1}&=&\frac{\partial
\hat{F}_{1}}{\partial
w_{1}}\cdot\frac{\partial\hat{\theta}_{1}}{\partial\tilde{P}_{2}}\cdot\frac{\partial\tilde{P}_{2}}{\partial
z_{2}}+\frac{\partial F_{1}}{\partial z_{2}}
\\
\frac{\partial \hat{F}_{1}\left(z_{1},z_{2},\hat{\theta}_{1}\left(P_{1}\left(z_{1}\right)\tilde{P}_{2}\left(z_{2}\right)\hat{P}_{2}\left(w_{2}\right)\right),w_{2}\right)}{\partial w_{1}}|_{\mathbf{z},\mathbf{w}=1}&=&0\\
\frac{\partial \hat{F}_{1}\left(z_{1},z_{2},\hat{\theta}_{1}\left(P_{1}\left(z_{1}\right)\tilde{P}_{2}\left(z_{2}\right)\hat{P}_{2}\left(w_{2}\right)\right),w_{2}\right)}{\partial w_{2}}|_{\mathbf{z},\mathbf{w}=1}&=&\frac{\partial\hat{F}_{1}}{\partial w_{1}}\cdot\frac{\partial\hat{\theta}_{1}}{\partial\hat{P}_{2}}\cdot\frac{\partial\hat{P}_{2}}{\partial w_{2}}+\frac{\partial \hat{F}_{1}}{\partial w_{2}}\\
\end{eqnarray*}



Finalmente para $\zeta_{2}$

\begin{eqnarray*}\hat{F}_{2}\left(z_{1},z_{2},w_{1},\hat{\theta}_{2}\left(P_{1}\left(z_{1}\right)\tilde{P}_{2}\left(z_{2}\right)\hat{P}_{1}\left(w_{1}\right)\right)\right)&=&F_{2}\left(z_{1},z_{2};\zeta_{2}\right)\hat{F}_{2}\left(w_{1},\hat{\theta}_{2}\left(P_{1}\left(z_{1}\right)\tilde{P}_{2}\left(z_{2}\right)\hat{P}_{1}\left(w_{1}\right)\right)\right]
\end{eqnarray*}
por tanto:

\begin{eqnarray*}
\frac{\partial
\hat{F}_{2}\left(z_{1},z_{2},w_{1},\hat{\theta}_{2}\left(P_{1}\left(z_{1}\right)\tilde{P}_{2}\left(z_{2}\right)\hat{P}_{1}\left(w_{1}\right)\right)\right)}{\partial
z_{1}}|_{\mathbf{z},\mathbf{w}=1}&=&\frac{\partial\hat{F}_{2}}{\partial
w_{2}}\cdot\frac{\partial\hat{\theta}_{2}}{\partial
P_{1}}\cdot\frac{\partial P_{1}}{\partial z_{1}}+\frac{\partial
F_{2}}{\partial z_{1}}
\\
\frac{\partial \hat{F}_{2}\left(z_{1},z_{2},w_{1},\hat{\theta}_{2}\left(P_{1}\left(z_{1}\right)\tilde{P}_{2}\left(z_{2}\right)\hat{P}_{1}\left(w_{1}\right)\right)\right)}{\partial z_{2}}|_{\mathbf{z},\mathbf{w}=1}&=&\frac{\partial\hat{F}_{2}}{\partial w_{2}}\cdot\frac{\partial\hat{\theta}_{2}}{\partial \tilde{P}_{2}}\cdot\frac{\partial \tilde{P}_{2}}{\partial z_{2}}+\frac{\partial F_{2}}{\partial z_{2}}\\
\frac{\partial \hat{F}_{2}\left(z_{1},z_{2},w_{1},\hat{\theta}_{2}\left(P_{1}\left(z_{1}\right)\tilde{P}_{2}\left(z_{2}\right)\hat{P}_{1}\left(w_{1}\right)\right)\right)}{\partial w_{1}}|_{\mathbf{z},\mathbf{w}=1}&=&\frac{\partial\hat{F}_{2}}{\partial w_{2}}\cdot\frac{\partial\hat{\theta}_{2}}{\partial \hat{P}_{1}}\cdot\frac{\partial \hat{P}_{1}}{\partial w_{1}}+\frac{\partial \hat{F}_{2}}{\partial w_{1}}\\
\frac{\partial \hat{F}_{2}\left(z_{1},z_{2},w_{1},\hat{\theta}_{2}\left(P_{1}\left(z_{1}\right)\tilde{P}_{2}\left(z_{2}\right)\hat{P}_{1}\left(w_{1}\right)\right)\right)}{\partial w_{2}}|_{\mathbf{z},\mathbf{w}=1}&=&0\\
\end{eqnarray*}

%_________________________________________________________________________________________________
\subsection{Ecuaciones Recursivas}
%_________________________________________________________________________________________________

Entonces, de todo lo desarrollado hasta ahora se tienen las siguientes ecuaciones:

\begin{eqnarray*}
\frac{\partial F_{2}\left(\mathbf{z},\mathbf{w}\right)}{\partial z_{1}}|_{\mathbf{z},\mathbf{w}=1}&=&r_{1}\mu_{1}\\
\frac{\partial F_{2}\left(\mathbf{z},\mathbf{w}\right)}{\partial z_{2}}|_{\mathbf{z},\mathbf{w}=1}&=&=r_{1}\tilde{\mu}_{2}+f_{1}\left(1\right)\left(\frac{1}{1-\mu_{1}}\right)\tilde{\mu}_{2}+f_{1}\left(2\right)\\
\frac{\partial F_{2}\left(\mathbf{z},\mathbf{w}\right)}{\partial w_{1}}|_{\mathbf{z},\mathbf{w}=1}&=&r_{1}\hat{\mu}_{1}+f_{1}\left(1\right)\left(\frac{1}{1-\mu_{1}}\right)\hat{\mu}_{1}+\hat{F}_{1,1}^{(1)}\left(1\right)\\
\frac{\partial F_{2}\left(\mathbf{z},\mathbf{w}\right)}{\partial
w_{2}}|_{\mathbf{z},\mathbf{w}=1}&=&r_{1}\hat{\mu}_{2}+f_{1}\left(1\right)\left(\frac{1}{1-\mu_{1}}\right)\hat{\mu}_{2}+\hat{F}_{2,1}^{(1)}\left(1\right)\\
\frac{\partial F_{1}\left(\mathbf{z},\mathbf{w}\right)}{\partial z_{1}}|_{\mathbf{z},\mathbf{w}=1}&=&r_{2}\mu_{1}+f_{2}\left(2\right)\left(\frac{1}{1-\tilde{\mu}_{2}}\right)\mu_{1}+f_{2}\left(1\right)\\
\frac{\partial F_{1}\left(\mathbf{z},\mathbf{w}\right)}{\partial z_{2}}|_{\mathbf{z},\mathbf{w}=1}&=&r_{2}\tilde{\mu}_{2}\\
\frac{\partial F_{1}\left(\mathbf{z},\mathbf{w}\right)}{\partial w_{1}}|_{\mathbf{z},\mathbf{w}=1}&=&r_{2}\hat{\mu}_{1}+f_{2}\left(2\right)\left(\frac{1}{1-\tilde{\mu}_{2}}\right)\hat{\mu}_{1}+\hat{F}_{2,1}^{(1)}\left(1\right)\\
\frac{\partial F_{1}\left(\mathbf{z},\mathbf{w}\right)}{\partial
w_{2}}|_{\mathbf{z},\mathbf{w}=1}&=&r_{2}\hat{\mu}_{2}+f_{2}\left(2\right)\left(\frac{1}{1-\tilde{\mu}_{2}}\right)\hat{\mu}_{2}+\hat{F}_{2,2}^{(1)}\left(1\right)\\
\frac{\partial \hat{F}_{2}\left(\mathbf{z},\mathbf{w}\right)}{\partial z_{1}}|_{\mathbf{z},\mathbf{w}=1}&=&\hat{r}_{1}\mu_{1}+\hat{f}_{1}\left(1\right)\left(\frac{1}{1-\hat{\mu}_{1}}\right)\mu_{1}+F_{1,1}^{(1)}\left(1\right)\\
\frac{\partial \hat{F}_{2}\left(\mathbf{z},\mathbf{w}\right)}{\partial z_{2}}|_{\mathbf{z},\mathbf{w}=1}&=&\hat{r}_{1}\mu_{2}+\hat{f}_{1}\left(1\right)\left(\frac{1}{1-\hat{\mu}_{1}}\right)\tilde{\mu}_{2}+F_{2,1}^{(1)}\left(1\right)\\
\frac{\partial \hat{F}_{2}\left(\mathbf{z},\mathbf{w}\right)}{\partial w_{1}}|_{\mathbf{z},\mathbf{w}=1}&=&\hat{r}_{1}\hat{\mu}_{1}\\
\frac{\partial \hat{F}_{2}\left(\mathbf{z},\mathbf{w}\right)}{\partial w_{2}}|_{\mathbf{z},\mathbf{w}=1}&=&\hat{r}_{1}\hat{\mu}_{2}+\hat{f}_{1}\left(1\right)\left(\frac{1}{1-\hat{\mu}_{1}}\right)\hat{\mu}_{2}+\hat{f}_{1}\left(2\right)\\
\frac{\partial \hat{F}_{1}\left(\mathbf{z},\mathbf{w}\right)}{\partial z_{1}}|_{\mathbf{z},\mathbf{w}=1}&=&\hat{r}_{2}\mu_{1}+\hat{f}_{2}\left(1\right)\left(\frac{1}{1-\hat{\mu}_{2}}\right)\mu_{1}+F_{1,2}^{(1)}\left(1\right)\\
\frac{\partial \hat{F}_{1}\left(\mathbf{z},\mathbf{w}\right)}{\partial z_{2}}|_{\mathbf{z},\mathbf{w}=1}&=&\hat{r}_{2}\tilde{\mu}_{2}+\hat{f}_{2}\left(2\right)\left(\frac{1}{1-\hat{\mu}_{2}}\right)\tilde{\mu}_{2}+F_{2,2}^{(1)}\left(1\right)\\
\frac{\partial \hat{F}_{1}\left(\mathbf{z},\mathbf{w}\right)}{\partial w_{1}}|_{\mathbf{z},\mathbf{w}=1}&=&\hat{r}_{2}\hat{\mu}_{1}+\hat{f}_{2}\left(2\right)\left(\frac{1}{1-\hat{\mu}_{2}}\right)\hat{\mu}_{1}+\hat{f}_{2}\left(1\right)\\
\frac{\partial
\hat{F}_{1}\left(\mathbf{z},\mathbf{w}\right)}{\partial
w_{2}}|_{\mathbf{z},\mathbf{w}=1}&=&\hat{r}_{2}\hat{\mu}_{2}
\end{eqnarray*}

Es decir, se tienen las siguientes ecuaciones:




\begin{eqnarray*}
f_{2}\left(1\right)&=&r_{1}\mu_{1}\\
f_{1}\left(2\right)&=&r_{2}\tilde{\mu}_{2}\\
f_{2}\left(2\right)&=&r_{1}\tilde{\mu}_{2}+\tilde{\mu}_{2}\left(\frac{f_{1}\left(1\right)}{1-\mu_{1}}\right)+f_{1}\left(2\right)=\left(r_{1}+\frac{f_{1}\left(1\right)}{1-\mu_{1}}\right)\tilde{\mu}_{2}+r_{2}\tilde{\mu}_{2}\\
&=&\left(r_{1}+r_{2}+\frac{f_{1}\left(1\right)}{1-\mu_{1}}\right)\tilde{\mu}_{2}=\left(r+\frac{f_{1}\left(1\right)}{1-\mu_{1}}\right)\tilde{\mu}_{2}\\
f_{2}\left(3\right)&=&r_{1}\hat{\mu}_{1}+\hat{\mu}_{1}\left(\frac{f_{1}\left(1\right)}{1-\mu_{1}}\right)+\hat{F}_{1,1}^{(1)}\left(1\right)=\hat{\mu}_{1}\left(r_{1}+\frac{f_{1}\left(1\right)}{1-\mu_{1}}\right)+\frac{\hat{\mu}_{1}}{\mu_{1}}\\
f_{2}\left(4\right)&=&r_{1}\hat{\mu}_{2}+\hat{\mu}_{2}\left(\frac{f_{1}\left(1\right)}{1-\mu_{1}}\right)+\hat{F}_{2,1}^{(1)}\left(1\right)=\hat{\mu}_{2}\left(r_{1}+\frac{f_{1}\left(1\right)}{1-\mu_{1}}\right)+\frac{\hat{\mu}_{2}}{\mu_{1}}\\
f_{1}\left(1\right)&=&r_{2}\mu_{1}+\mu_{1}\left(\frac{f_{2}\left(2\right)}{1-\tilde{\mu}_{2}}\right)+r_{1}\mu_{1}=\mu_{1}\left(r_{1}+r_{2}+\frac{f_{2}\left(2\right)}{1-\tilde{\mu}_{2}}\right)\\
&=&\mu_{1}\left(r+\frac{f_{2}\left(2\right)}{1-\tilde{\mu}_{2}}\right)\\
f_{1}\left(3\right)&=&r_{2}\hat{\mu}_{1}+\hat{\mu}_{1}\left(\frac{f_{2}\left(2\right)}{1-\tilde{\mu}_{2}}\right)+\hat{F}^{(1)}_{1,2}\left(1\right)=\hat{\mu}_{1}\left(r_{2}+\frac{f_{2}\left(2\right)}{1-\tilde{\mu}_{2}}\right)+\frac{\hat{\mu}_{1}}{\mu_{2}}\\
f_{1}\left(4\right)&=&r_{2}\hat{\mu}_{2}+\hat{\mu}_{2}\left(\frac{f_{2}\left(2\right)}{1-\tilde{\mu}_{2}}\right)+\hat{F}_{2,2}^{(1)}\left(1\right)=\hat{\mu}_{2}\left(r_{2}+\frac{f_{2}\left(2\right)}{1-\tilde{\mu}_{2}}\right)+\frac{\hat{\mu}_{2}}{\mu_{2}}\\
\hat{f}_{1}\left(4\right)&=&\hat{r}_{2}\hat{\mu}_{2}\\
\hat{f}_{2}\left(3\right)&=&\hat{r}_{1}\hat{\mu}_{1}\\
\hat{f}_{1}\left(1\right)&=&\hat{r}_{2}\mu_{1}+\mu_{1}\left(\frac{\hat{f}_{2}\left(4\right)}{1-\hat{\mu}_{2}}\right)+F_{1,2}^{(1)}\left(1\right)=\left(\hat{r}_{2}+\frac{\hat{f}_{2}\left(4\right)}{1-\hat{\mu}_{2}}\right)\mu_{1}+\frac{\mu_{1}}{\hat{\mu}_{2}}\\
\hat{f}_{1}\left(2\right)&=&\hat{r}_{2}\tilde{\mu}_{2}+\tilde{\mu}_{2}\left(\frac{\hat{f}_{2}\left(4\right)}{1-\hat{\mu}_{2}}\right)+F_{2,2}^{(1)}\left(1\right)=
\left(\hat{r}_{2}+\frac{\hat{f}_{2}\left(4\right)}{1-\hat{\mu}_{2}}\right)\tilde{\mu}_{2}+\frac{\mu_{2}}{\hat{\mu}_{2}}\\
\hat{f}_{1}\left(3\right)&=&\hat{r}_{2}\hat{\mu}_{1}+\hat{\mu}_{1}\left(\frac{\hat{f}_{2}\left(4\right)}{1-\hat{\mu}_{2}}\right)+\hat{f}_{2}\left(3\right)=\left(\hat{r}_{2}+\frac{\hat{f}_{2}\left(4\right)}{1-\hat{\mu}_{2}}\right)\hat{\mu}_{1}+\hat{r}_{1}\hat{\mu}_{1}\\
&=&\left(\hat{r}_{1}+\hat{r}_{2}+\frac{\hat{f}_{2}\left(4\right)}{1-\hat{\mu}_{2}}\right)\hat{\mu}_{1}=\left(\hat{r}+\frac{\hat{f}_{2}\left(4\right)}{1-\hat{\mu}_{2}}\right)\hat{\mu}_{1}\\
\hat{f}_{2}\left(1\right)&=&\hat{r}_{1}\mu_{1}+\mu_{1}\left(\frac{\hat{f}_{1}\left(3\right)}{1-\hat{\mu}_{1}}\right)+F_{1,1}^{(1)}\left(1\right)=\left(\hat{r}_{1}+\frac{\hat{f}_{1}\left(3\right)}{1-\hat{\mu}_{1}}\right)\mu_{1}+\frac{\mu_{1}}{\hat{\mu}_{1}}\\
\hat{f}_{2}\left(2\right)&=&\hat{r}_{1}\tilde{\mu}_{2}+\tilde{\mu}_{2}\left(\frac{\hat{f}_{1}\left(3\right)}{1-\hat{\mu}_{1}}\right)+F_{2,1}^{(1)}\left(1\right)=\left(\hat{r}_{1}+\frac{\hat{f}_{1}\left(3\right)}{1-\hat{\mu}_{1}}\right)\tilde{\mu}_{2}+\frac{\mu_{2}}{\hat{\mu}_{1}}\\
\hat{f}_{2}\left(4\right)&=&\hat{r}_{1}\hat{\mu}_{2}+\hat{\mu}_{2}\left(\frac{\hat{f}_{1}\left(3\right)}{1-\hat{\mu}_{1}}\right)+\hat{f}_{1}\left(4\right)=\hat{r}_{1}\hat{\mu}_{2}+\hat{r}_{2}\hat{\mu}_{2}+\hat{\mu}_{2}\left(\frac{\hat{f}_{1}\left(3\right)}{1-\hat{\mu}_{1}}\right)\\
&=&\left(\hat{r}+\frac{\hat{f}_{1}\left(3\right)}{1-\hat{\mu}_{1}}\right)\hat{\mu}_{2}\\
\end{eqnarray*}

es decir,


\begin{eqnarray*}
\begin{array}{lll}
f_{1}\left(1\right)=\mu_{1}\left(r+\frac{f_{2}\left(2\right)}{1-\tilde{\mu}_{2}}\right)&f_{1}\left(2\right)=r_{2}\tilde{\mu}_{2}&f_{1}\left(3\right)=\hat{\mu}_{1}\left(r_{2}+\frac{f_{2}\left(2\right)}{1-\tilde{\mu}_{2}}\right)+\frac{\hat{\mu}_{1}}{\mu_{2}}\\
f_{1}\left(4\right)=\hat{\mu}_{2}\left(r_{2}+\frac{f_{2}\left(2\right)}{1-\tilde{\mu}_{2}}\right)+\frac{\hat{\mu}_{2}}{\mu_{2}}&f_{2}\left(1\right)=r_{1}\mu_{1}&f_{2}\left(2\right)=\left(r+\frac{f_{1}\left(1\right)}{1-\mu_{1}}\right)\tilde{\mu}_{2}\\
f_{2}\left(3\right)=\hat{\mu}_{1}\left(r_{1}+\frac{f_{1}\left(1\right)}{1-\mu_{1}}\right)+\frac{\hat{\mu}_{1}}{\mu_{1}}&
f_{2}\left(4\right)=\hat{\mu}_{2}\left(r_{1}+\frac{f_{1}\left(1\right)}{1-\mu_{1}}\right)+\frac{\hat{\mu}_{2}}{\mu_{1}}&\hat{f}_{1}\left(1\right)=\left(\hat{r}_{2}+\frac{\hat{f}_{2}\left(4\right)}{1-\hat{\mu}_{2}}\right)\mu_{1}+\frac{\mu_{1}}{\hat{\mu}_{2}}\\
\hat{f}_{1}\left(2\right)=\left(\hat{r}_{2}+\frac{\hat{f}_{2}\left(4\right)}{1-\hat{\mu}_{2}}\right)\tilde{\mu}_{2}+\frac{\mu_{2}}{\hat{\mu}_{2}}&\hat{f}_{1}\left(3\right)=\left(\hat{r}+\frac{\hat{f}_{2}\left(4\right)}{1-\hat{\mu}_{2}}\right)\hat{\mu}_{1}&\hat{f}_{1}\left(4\right)=\hat{r}_{2}\hat{\mu}_{2}\\
\hat{f}_{2}\left(1\right)=\left(\hat{r}_{1}+\frac{\hat{f}_{1}\left(3\right)}{1-\hat{\mu}_{1}}\right)\mu_{1}+\frac{\mu_{1}}{\hat{\mu}_{1}}&\hat{f}_{2}\left(2\right)=\left(\hat{r}_{1}+\frac{\hat{f}_{1}\left(3\right)}{1-\hat{\mu}_{1}}\right)\tilde{\mu}_{2}+\frac{\mu_{2}}{\hat{\mu}_{1}}&\hat{f}_{2}\left(3\right)=\hat{r}_{1}\hat{\mu}_{1}\\
&\hat{f}_{2}\left(4\right)=\left(\hat{r}+\frac{\hat{f}_{1}\left(3\right)}{1-\hat{\mu}_{1}}\right)\hat{\mu}_{2}&
\end{array}
\end{eqnarray*}

%_______________________________________________________________________________________________
\subsection{Soluci\'on del Sistema de Ecuaciones Lineales}
%_________________________________________________________________________________________________

A saber, se puede demostrar que la soluci\'on del sistema de
ecuaciones est\'a dado por las siguientes expresiones, donde

\begin{eqnarray*}
\mu=\mu_{1}+\tilde{\mu}_{2}\textrm{ , }\hat{\mu}=\hat{\mu}_{1}+\hat{\mu}_{2}\textrm{ , }
r=r_{1}+r_{2}\textrm{ y }\hat{r}=\hat{r}_{1}+\hat{r}_{2}
\end{eqnarray*}
entonces

\begin{eqnarray*}
\begin{array}{lll}
f_{1}\left(1\right)=r\frac{\mu_{1}\left(1-\mu_{1}\right)}{1-\mu}&
f_{1}\left(3\right)=\hat{\mu}_{1}\left(\frac{r_{2}\mu_{2}+1}{\mu_{2}}+r\frac{\tilde{\mu}_{2}}{1-\mu}\right)&
f_{1}\left(4\right)=\hat{\mu}_{2}\left(\frac{r_{2}\mu_{2}+1}{\mu_{2}}+r\frac{\tilde{\mu}_{2}}{1-\mu}\right)\\
f_{2}\left(2\right)=r\frac{\tilde{\mu}_{2}\left(1-\tilde{\mu}_{2}\right)}{1-\mu}&
f_{2}\left(3\right)=\hat{\mu}_{1}\left(\frac{r_{1}\mu_{1}+1}{\mu_{1}}+r\frac{\mu_{1}}{1-\mu}\right)&
f_{2}\left(4\right)=\hat{\mu}_{2}\left(\frac{r_{1}\mu_{1}+1}{\mu_{1}}+r\frac{\mu_{1}}{1-\mu}\right)\\
\hat{f}_{1}\left(1\right)=\mu_{1}\left(\frac{\hat{r}_{2}\hat{\mu}_{2}+1}{\hat{\mu}_{2}}+\hat{r}\frac{\hat{\mu}_{2}}{1-\hat{\mu}}\right)&
\hat{f}_{1}\left(2\right)=\tilde{\mu}_{2}\left(\hat{r}_{2}+\hat{r}\frac{\hat{\mu}_{2}}{1-\hat{\mu}}\right)+\frac{\mu_{2}}{\hat{\mu}_{2}}&
\hat{f}_{1}\left(3\right)=\hat{r}\frac{\hat{\mu}_{1}\left(1-\hat{\mu}_{1}\right)}{1-\hat{\mu}}\\
\hat{f}_{2}\left(1\right)=\mu_{1}\left(\frac{\hat{r}_{1}\hat{\mu}_{1}+1}{\hat{\mu}_{1}}+\hat{r}\frac{\hat{\mu}_{1}}{1-\hat{\mu}}\right)&
\hat{f}_{2}\left(2\right)=\tilde{\mu}_{2}\left(\hat{r}_{1}+\hat{r}\frac{\hat{\mu}_{1}}{1-\hat{\mu}}\right)+\frac{\hat{\mu_{2}}}{\hat{\mu}_{1}}&
\hat{f}_{2}\left(4\right)=\hat{r}\frac{\hat{\mu}_{2}\left(1-\hat{\mu}_{2}\right)}{1-\hat{\mu}}\\
\end{array}
\end{eqnarray*}




A saber

\begin{eqnarray*}
f_{1}\left(3\right)&=&\hat{\mu}_{1}\left(r_{2}+\frac{f_{2}\left(2\right)}{1-\tilde{\mu}_{2}}\right)+\frac{\hat{\mu}_{1}}{\mu_{2}}=\hat{\mu}_{1}\left(r_{2}+\frac{r\frac{\tilde{\mu}_{2}\left(1-\tilde{\mu}_{2}\right)}{1-\mu}}{1-\tilde{\mu}_{2}}\right)+\frac{\hat{\mu}_{1}}{\mu_{2}}=\hat{\mu}_{1}\left(r_{2}+\frac{r\tilde{\mu}_{2}}{1-\mu}\right)+\frac{\hat{\mu}_{1}}{\mu_{2}}\\
&=&\hat{\mu}_{1}\left(r_{2}+\frac{r\tilde{\mu}_{2}}{1-\mu}+\frac{1}{\mu_{2}}\right)=\hat{\mu}_{1}\left(\frac{r_{2}\mu_{2}+1}{\mu_{2}}+\frac{r\tilde{\mu}_{2}}{1-\mu}\right)
\end{eqnarray*}

\begin{eqnarray*}
f_{1}\left(4\right)&=&\hat{\mu}_{2}\left(r_{2}+\frac{f_{2}\left(2\right)}{1-\tilde{\mu}_{2}}\right)+\frac{\hat{\mu}_{2}}{\mu_{2}}=\hat{\mu}_{2}\left(r_{2}+\frac{r\frac{\tilde{\mu}_{2}\left(1-\tilde{\mu}_{2}\right)}{1-\mu}}{1-\tilde{\mu}_{2}}\right)+\frac{\hat{\mu}_{2}}{\mu_{2}}=\hat{\mu}_{2}\left(r_{2}+\frac{r\tilde{\mu}_{2}}{1-\mu}\right)+\frac{\hat{\mu}_{1}}{\mu_{2}}\\
&=&\hat{\mu}_{2}\left(r_{2}+\frac{r\tilde{\mu}_{2}}{1-\mu}+\frac{1}{\mu_{2}}\right)=\hat{\mu}_{2}\left(\frac{r_{2}\mu_{2}+1}{\mu_{2}}+\frac{r\tilde{\mu}_{2}}{1-\mu}\right)
\end{eqnarray*}

\begin{eqnarray*}
f_{2}\left(3\right)&=&\hat{\mu}_{1}\left(r_{1}+\frac{f_{1}\left(1\right)}{1-\mu_{1}}\right)+\frac{\hat{\mu}_{1}}{\mu_{1}}=\hat{\mu}_{1}\left(r_{1}+\frac{r\frac{\mu_{1}\left(1-\mu_{1}\right)}{1-\mu}}{1-\mu_{1}}\right)+\frac{\hat{\mu}_{1}}{\mu_{1}}=\hat{\mu}_{1}\left(r_{1}+\frac{r\mu_{1}}{1-\mu}\right)+\frac{\hat{\mu}_{1}}{\mu_{1}}\\
&=&\hat{\mu}_{1}\left(r_{1}+\frac{r\mu_{1}}{1-\mu}+\frac{1}{\mu_{1}}\right)=\hat{\mu}_{1}\left(\frac{r_{1}\mu_{1}+1}{\mu_{1}}+\frac{r\mu_{1}}{1-\mu}\right)
\end{eqnarray*}

\begin{eqnarray*}
f_{2}\left(4\right)&=&\hat{\mu}_{2}\left(r_{1}+\frac{f_{1}\left(1\right)}{1-\mu_{1}}\right)+\frac{\hat{\mu}_{2}}{\mu_{1}}=\hat{\mu}_{2}\left(r_{1}+\frac{r\frac{\mu_{1}\left(1-\mu_{1}\right)}{1-\mu}}{1-\mu_{1}}\right)+\frac{\hat{\mu}_{1}}{\mu_{1}}=\hat{\mu}_{2}\left(r_{1}+\frac{r\mu_{1}}{1-\mu}\right)+\frac{\hat{\mu}_{1}}{\mu_{1}}\\
&=&\hat{\mu}_{2}\left(r_{1}+\frac{r\mu_{1}}{1-\mu}+\frac{1}{\mu_{1}}\right)=\hat{\mu}_{2}\left(\frac{r_{1}\mu_{1}+1}{\mu_{1}}+\frac{r\mu_{1}}{1-\mu}\right)\end{eqnarray*}


\begin{eqnarray*}
\hat{f}_{1}\left(1\right)&=&\mu_{1}\left(\hat{r}_{2}+\frac{\hat{f}_{2}\left(4\right)}{1-\tilde{\mu}_{2}}\right)+\frac{\mu_{1}}{\hat{\mu}_{2}}=\mu_{1}\left(\hat{r}_{2}+\frac{\hat{r}\frac{\hat{\mu}_{2}\left(1-\hat{\mu}_{2}\right)}{1-\hat{\mu}}}{1-\hat{\mu}_{2}}\right)+\frac{\mu_{1}}{\hat{\mu}_{2}}=\mu_{1}\left(\hat{r}_{2}+\frac{\hat{r}\hat{\mu}_{2}}{1-\hat{\mu}}\right)+\frac{\mu_{1}}{\mu_{2}}\\
&=&\mu_{1}\left(\hat{r}_{2}+\frac{\hat{r}\mu_{2}}{1-\hat{\mu}}+\frac{1}{\hat{\mu}_{2}}\right)=\mu_{1}\left(\frac{\hat{r}_{2}\hat{\mu}_{2}+1}{\hat{\mu}_{2}}+\frac{\hat{r}\hat{\mu}_{2}}{1-\hat{\mu}}\right)
\end{eqnarray*}

\begin{eqnarray*}
\hat{f}_{1}\left(2\right)&=&\tilde{\mu}_{2}\left(\hat{r}_{2}+\frac{\hat{f}_{2}\left(4\right)}{1-\tilde{\mu}_{2}}\right)+\frac{\mu_{2}}{\hat{\mu}_{2}}=\tilde{\mu}_{2}\left(\hat{r}_{2}+\frac{\hat{r}\frac{\hat{\mu}_{2}\left(1-\hat{\mu}_{2}\right)}{1-\hat{\mu}}}{1-\hat{\mu}_{2}}\right)+\frac{\mu_{2}}{\hat{\mu}_{2}}=\tilde{\mu}_{2}\left(\hat{r}_{2}+\frac{\hat{r}\hat{\mu}_{2}}{1-\hat{\mu}}\right)+\frac{\mu_{2}}{\hat{\mu}_{2}}
\end{eqnarray*}

\begin{eqnarray*}
\hat{f}_{2}\left(1\right)&=&\mu_{1}\left(\hat{r}_{1}+\frac{\hat{f}_{1}\left(3\right)}{1-\hat{\mu}_{1}}\right)+\frac{\mu_{1}}{\hat{\mu}_{1}}=\mu_{1}\left(\hat{r}_{1}+\frac{\hat{r}\frac{\hat{\mu}_{1}\left(1-\hat{\mu}_{1}\right)}{1-\hat{\mu}}}{1-\hat{\mu}_{1}}\right)+\frac{\mu_{1}}{\hat{\mu}_{1}}=\mu_{1}\left(\hat{r}_{1}+\frac{\hat{r}\hat{\mu}_{1}}{1-\hat{\mu}}\right)+\frac{\mu_{1}}{\hat{\mu}_{1}}\\
&=&\mu_{1}\left(\hat{r}_{1}+\frac{\hat{r}\hat{\mu}_{1}}{1-\hat{\mu}}+\frac{1}{\hat{\mu}_{1}}\right)=\mu_{1}\left(\frac{\hat{r}_{1}\hat{\mu}_{1}+1}{\hat{\mu}_{1}}+\frac{\hat{r}\hat{\mu}_{1}}{1-\hat{\mu}}\right)
\end{eqnarray*}

\begin{eqnarray*}
\hat{f}_{2}\left(2\right)&=&\tilde{\mu}_{2}\left(\hat{r}_{1}+\frac{\hat{f}_{1}\left(3\right)}{1-\tilde{\mu}_{1}}\right)+\frac{\mu_{2}}{\hat{\mu}_{1}}=\tilde{\mu}_{2}\left(\hat{r}_{1}+\frac{\hat{r}\frac{\hat{\mu}_{1}
\left(1-\hat{\mu}_{1}\right)}{1-\hat{\mu}}}{1-\hat{\mu}_{1}}\right)+\frac{\mu_{2}}{\hat{\mu}_{1}}=\tilde{\mu}_{2}\left(\hat{r}_{1}+\frac{\hat{r}\hat{\mu}_{1}}{1-\hat{\mu}}\right)+\frac{\mu_{2}}{\hat{\mu}_{1}}
\end{eqnarray*}

%----------------------------------------------------------------------------------------
\section{Resultado Principal}
%----------------------------------------------------------------------------------------
Sean $\mu_{1},\mu_{2},\check{\mu}_{2},\hat{\mu}_{1},\hat{\mu}_{2}$ y $\tilde{\mu}_{2}=\mu_{2}+\check{\mu}_{2}$ los valores esperados de los proceso definidos anteriormente, y sean $r_{1},r_{2}, \hat{r}_{1}$ y $\hat{r}_{2}$ los valores esperado s de los tiempos de traslado del servidor entre las colas para cada uno de los sistemas de visitas c\'iclicas. Si se definen $\mu=\mu_{1}+\tilde{\mu}_{2}$, $\hat{\mu}=\hat{\mu}_{1}+\hat{\mu}_{2}$, y $r=r_{1}+r_{2}$ y  $\hat{r}=\hat{r}_{1}+\hat{r}_{2}$, entonces se tiene el siguiente resultado.

\begin{Teo}
Supongamos que $\mu<1$, $\hat{\mu}<1$, entonces, el n\'umero de usuarios presentes en cada una de las colas que conforman la Red de Sistemas de Visitas C\'iclicas cuando uno de los servidores visita a alguna de ellas est\'a dada por la soluci\'on del Sistema de Ecuaciones Lineales presentados arriba cuyas expresiones damos a continuaci\'on:
%{\footnotesize{
\begin{eqnarray*}
\begin{array}{lll}
f_{1}\left(1\right)=r\frac{\mu_{1}\left(1-\mu_{1}\right)}{1-\mu},&f_{1}\left(2\right)=r_{2}\tilde{\mu}_{2},&f_{1}\left(3\right)=\hat{\mu}_{1}\left(\frac{r_{2}\mu_{2}+1}{\mu_{2}}+r\frac{\tilde{\mu}_{2}}{1-\mu}\right),\\
f_{1}\left(4\right)=\hat{\mu}_{2}\left(\frac{r_{2}\mu_{2}+1}{\mu_{2}}+r\frac{\tilde{\mu}_{2}}{1-\mu}\right),&f_{2}\left(1\right)=r_{1}\mu_{1},&f_{2}\left(2\right)=r\frac{\tilde{\mu}_{2}\left(1-\tilde{\mu}_{2}\right)}{1-\mu},\\
f_{2}\left(3\right)=\hat{\mu}_{1}\left(\frac{r_{1}\mu_{1}+1}{\mu_{1}}+r\frac{\mu_{1}}{1-\mu}\right),&f_{2}\left(4\right)=\hat{\mu}_{2}\left(\frac{r_{1}\mu_{1}+1}{\mu_{1}}+r\frac{\mu_{1}}{1-\mu}\right),&\hat{f}_{1}\left(1\right)=\mu_{1}\left(\frac{\hat{r}_{2}\hat{\mu}_{2}+1}{\hat{\mu}_{2}}+\hat{r}\frac{\hat{\mu}_{2}}{1-\hat{\mu}}\right),\\
\hat{f}_{1}\left(2\right)=\tilde{\mu}_{2}\left(\hat{r}_{2}+\hat{r}\frac{\hat{\mu}_{2}}{1-\hat{\mu}}\right)+\frac{\mu_{2}}{\hat{\mu}_{2}},&\hat{f}_{1}\left(3\right)=\hat{r}\frac{\hat{\mu}_{1}\left(1-\hat{\mu}_{1}\right)}{1-\hat{\mu}},&\hat{f}_{1}\left(4\right)=\hat{r}_{2}\hat{\mu}_{2},\\
\hat{f}_{2}\left(1\right)=\mu_{1}\left(\frac{\hat{r}_{1}\hat{\mu}_{1}+1}{\hat{\mu}_{1}}+\hat{r}\frac{\hat{\mu}_{1}}{1-\hat{\mu}}\right),&\hat{f}_{2}\left(2\right)=\tilde{\mu}_{2}\left(\hat{r}_{1}+\hat{r}\frac{\hat{\mu}_{1}}{1-\hat{\mu}}\right)+\frac{\hat{\mu_{2}}}{\hat{\mu}_{1}},&\hat{f}_{2}\left(3\right)=\hat{r}_{1}\hat{\mu}_{1},\\
&\hat{f}_{2}\left(4\right)=\hat{r}\frac{\hat{\mu}_{2}\left(1-\hat{\mu}_{2}\right)}{1-\hat{\mu}}.&\\
\end{array}
\end{eqnarray*} %}}
\end{Teo}





%___________________________________________________________________________________________
%
\section{Segundos Momentos}
%___________________________________________________________________________________________
%
%___________________________________________________________________________________________
%
%\subsection{Derivadas de Segundo Orden: Tiempos de Traslado del Servidor}
%___________________________________________________________________________________________



Para poder determinar los segundos momentos para los tiempos de traslado del servidor es necesaria la siguiente proposici\'on:

\begin{Prop}\label{Prop.Segundas.Derivadas}
Sea $f\left(g\left(x\right)h\left(y\right)\right)$ funci\'on continua tal que tiene derivadas parciales mixtas de segundo orden, entonces se tiene lo siguiente:

\begin{eqnarray*}
\frac{\partial}{\partial x}f\left(g\left(x\right)h\left(y\right)\right)=\frac{\partial f\left(g\left(x\right)h\left(y\right)\right)}{\partial x}\cdot \frac{\partial g\left(x\right)}{\partial x}\cdot h\left(y\right)
\end{eqnarray*}

por tanto

\begin{eqnarray}
\frac{\partial}{\partial x}\frac{\partial}{\partial x}f\left(g\left(x\right)h\left(y\right)\right)
&=&\frac{\partial^{2}}{\partial x}f\left(g\left(x\right)h\left(y\right)\right)\cdot \left(\frac{\partial g\left(x\right)}{\partial x}\right)^{2}\cdot h^{2}\left(y\right)+\frac{\partial}{\partial x}f\left(g\left(x\right)h\left(y\right)\right)\cdot \frac{\partial g^{2}\left(x\right)}{\partial x^{2}}\cdot h\left(y\right).
\end{eqnarray}

y

\begin{eqnarray*}
\frac{\partial}{\partial y}\frac{\partial}{\partial x}f\left(g\left(x\right)h\left(y\right)\right)&=&\frac{\partial g\left(x\right)}{\partial x}\cdot \frac{\partial h\left(y\right)}{\partial y}\left\{\frac{\partial^{2}}{\partial y\partial x}f\left(g\left(x\right)h\left(y\right)\right)\cdot g\left(x\right)\cdot h\left(y\right)+\frac{\partial}{\partial x}f\left(g\left(x\right)h\left(y\right)\right)\right\}
\end{eqnarray*}
\end{Prop}
\begin{proof}
\footnotesize{
\begin{eqnarray*}
\frac{\partial}{\partial x}\frac{\partial}{\partial x}f\left(g\left(x\right)h\left(y\right)\right)&=&\frac{\partial}{\partial x}\left\{\frac{\partial f\left(g\left(x\right)h\left(y\right)\right)}{\partial x}\cdot \frac{\partial g\left(x\right)}{\partial x}\cdot h\left(y\right)\right\}\\
&=&\frac{\partial}{\partial x}\left\{\frac{\partial}{\partial x}f\left(g\left(x\right)h\left(y\right)\right)\right\}\cdot \frac{\partial g\left(x\right)}{\partial x}\cdot h\left(y\right)+\frac{\partial}{\partial x}f\left(g\left(x\right)h\left(y\right)\right)\cdot \frac{\partial g^{2}\left(x\right)}{\partial x^{2}}\cdot h\left(y\right)\\
&=&\frac{\partial^{2}}{\partial x}f\left(g\left(x\right)h\left(y\right)\right)\cdot \frac{\partial g\left(x\right)}{\partial x}\cdot h\left(y\right)\cdot \frac{\partial g\left(x\right)}{\partial x}\cdot h\left(y\right)+\frac{\partial}{\partial x}f\left(g\left(x\right)h\left(y\right)\right)\cdot \frac{\partial g^{2}\left(x\right)}{\partial x^{2}}\cdot h\left(y\right)\\
&=&\frac{\partial^{2}}{\partial x}f\left(g\left(x\right)h\left(y\right)\right)\cdot \left(\frac{\partial g\left(x\right)}{\partial x}\right)^{2}\cdot h^{2}\left(y\right)+\frac{\partial}{\partial x}f\left(g\left(x\right)h\left(y\right)\right)\cdot \frac{\partial g^{2}\left(x\right)}{\partial x^{2}}\cdot h\left(y\right).
\end{eqnarray*}}


Por otra parte:
\footnotesize{
\begin{eqnarray*}
\frac{\partial}{\partial y}\frac{\partial}{\partial x}f\left(g\left(x\right)h\left(y\right)\right)&=&\frac{\partial}{\partial y}\left\{\frac{\partial f\left(g\left(x\right)h\left(y\right)\right)}{\partial x}\cdot \frac{\partial g\left(x\right)}{\partial x}\cdot h\left(y\right)\right\}\\
&=&\frac{\partial}{\partial y}\left\{\frac{\partial}{\partial x}f\left(g\left(x\right)h\left(y\right)\right)\right\}\cdot \frac{\partial g\left(x\right)}{\partial x}\cdot h\left(y\right)+\frac{\partial}{\partial x}f\left(g\left(x\right)h\left(y\right)\right)\cdot \frac{\partial g\left(x\right)}{\partial x}\cdot \frac{\partial h\left(y\right)}{y}\\
&=&\frac{\partial^{2}}{\partial y\partial x}f\left(g\left(x\right)h\left(y\right)\right)\cdot \frac{\partial h\left(y\right)}{\partial y}\cdot g\left(x\right)\cdot \frac{\partial g\left(x\right)}{\partial x}\cdot h\left(y\right)+\frac{\partial}{\partial x}f\left(g\left(x\right)h\left(y\right)\right)\cdot \frac{\partial g\left(x\right)}{\partial x}\cdot \frac{\partial h\left(y\right)}{\partial y}\\
&=&\frac{\partial g\left(x\right)}{\partial x}\cdot \frac{\partial h\left(y\right)}{\partial y}\left\{\frac{\partial^{2}}{\partial y\partial x}f\left(g\left(x\right)h\left(y\right)\right)\cdot g\left(x\right)\cdot h\left(y\right)+\frac{\partial}{\partial x}f\left(g\left(x\right)h\left(y\right)\right)\right\}
\end{eqnarray*}}
\end{proof}

Utilizando la proposici\'on anterior (Proposici\'ion \ref{Prop.Segundas.Derivadas})se tiene el siguiente resultado que me dice como calcular los segundos momentos para los procesos de traslado del servidor:

\begin{Prop}
Sea $R_{i}$ la Funci\'on Generadora de Probabilidades para el n\'umero de arribos a cada una de las colas de la Red de Sistemas de Visitas C\'iclicas definidas como en (\ref{Ec.R1}). Entonces las derivadas parciales est\'an dadas por las siguientes expresiones:


\begin{eqnarray*}
\frac{\partial^{2} R_{i}\left(P_{1}\left(z_{1}\right)\tilde{P}_{2}\left(z_{2}\right)\hat{P}_{1}\left(w_{1}\right)\hat{P}_{2}\left(w_{2}\right)\right)}{\partial z_{i}^{2}}&=&\left(\frac{\partial P_{i}\left(z_{i}\right)}{\partial z_{i}}\right)^{2}\cdot\frac{\partial^{2} R_{i}\left(P_{1}\left(z_{1}\right)\tilde{P}_{2}\left(z_{2}\right)\hat{P}_{1}\left(w_{1}\right)\hat{P}_{2}\left(w_{2}\right)\right)}{\partial^{2} z_{i}}\\
&+&\left(\frac{\partial P_{i}\left(z_{i}\right)}{\partial z_{i}}\right)^{2}\cdot
\frac{\partial R_{i}\left(P_{1}\left(z_{1}\right)\tilde{P}_{2}\left(z_{2}\right)\hat{P}_{1}\left(w_{1}\right)\hat{P}_{2}\left(w_{2}\right)\right)}{\partial z_{i}}
\end{eqnarray*}



y adem\'as


\begin{eqnarray*}
\frac{\partial^{2} R_{i}\left(P_{1}\left(z_{1}\right)\tilde{P}_{2}\left(z_{2}\right)\hat{P}_{1}\left(w_{1}\right)\hat{P}_{2}\left(w_{2}\right)\right)}{\partial z_{2}\partial z_{1}}&=&\frac{\partial \tilde{P}_{2}\left(z_{2}\right)}{\partial z_{2}}\cdot\frac{\partial P_{1}\left(z_{1}\right)}{\partial z_{1}}\cdot\frac{\partial^{2} R_{i}\left(P_{1}\left(z_{1}\right)\tilde{P}_{2}\left(z_{2}\right)\hat{P}_{1}\left(w_{1}\right)\hat{P}_{2}\left(w_{2}\right)\right)}{\partial z_{2}\partial z_{1}}\\
&+&\frac{\partial \tilde{P}_{2}\left(z_{2}\right)}{\partial z_{2}}\cdot\frac{\partial P_{1}\left(z_{1}\right)}{\partial z_{1}}\cdot\frac{\partial R_{i}\left(P_{1}\left(z_{1}\right)\tilde{P}_{2}\left(z_{2}\right)\hat{P}_{1}\left(w_{1}\right)\hat{P}_{2}\left(w_{2}\right)\right)}{\partial z_{1}},
\end{eqnarray*}



\begin{eqnarray*}
\frac{\partial^{2} R_{i}\left(P_{1}\left(z_{1}\right)\tilde{P}_{2}\left(z_{2}\right)\hat{P}_{1}\left(w_{1}\right)\hat{P}_{2}\left(w_{2}\right)\right)}{\partial w_{i}\partial z_{1}}&=&\frac{\partial \hat{P}_{i}\left(w_{i}\right)}{\partial z_{2}}\cdot\frac{\partial P_{1}\left(z_{1}\right)}{\partial z_{1}}\cdot\frac{\partial^{2} R_{i}\left(P_{1}\left(z_{1}\right)\tilde{P}_{2}\left(z_{2}\right)\hat{P}_{1}\left(w_{1}\right)\hat{P}_{2}\left(w_{2}\right)\right)}{\partial w_{i}\partial z_{1}}\\
&+&\frac{\partial \hat{P}_{i}\left(w_{i}\right)}{\partial z_{2}}\cdot\frac{\partial P_{1}\left(z_{1}\right)}{\partial z_{1}}\cdot\frac{\partial R_{i}\left(P_{1}\left(z_{1}\right)\tilde{P}_{2}\left(z_{2}\right)\hat{P}_{1}\left(w_{1}\right)\hat{P}_{2}\left(w_{2}\right)\right)}{\partial z_{1}},
\end{eqnarray*}
finalmente

\begin{eqnarray*}
\frac{\partial^{2} R_{i}\left(P_{1}\left(z_{1}\right)\tilde{P}_{2}\left(z_{2}\right)\hat{P}_{1}\left(w_{1}\right)\hat{P}_{2}\left(w_{2}\right)\right)}{\partial w_{i}\partial z_{2}}&=&\frac{\partial \hat{P}_{i}\left(w_{i}\right)}{\partial w_{i}}\cdot\frac{\partial \tilde{P}_{2}\left(z_{2}\right)}{\partial z_{2}}\cdot\frac{\partial^{2} R_{i}\left(P_{1}\left(z_{1}\right)\tilde{P}_{2}\left(z_{2}\right)\hat{P}_{1}\left(w_{1}\right)\hat{P}_{2}\left(w_{2}\right)\right)}{\partial w_{i}\partial z_{2}}\\
&+&\frac{\partial \hat{P}_{i}\left(w_{i}\right)}{\partial w_{i}}\cdot\frac{\partial \tilde{P}_{2}\left(z_{2}\right)}{\partial z_{1}}\cdot\frac{\partial R_{i}\left(P_{1}\left(z_{1}\right)\tilde{P}_{2}\left(z_{2}\right)\hat{P}_{1}\left(w_{1}\right)\hat{P}_{2}\left(w_{2}\right)\right)}{\partial z_{2}},
\end{eqnarray*}

para $i=1,2$.
\end{Prop}

%___________________________________________________________________________________________
%
\subsection{Sistema de Ecuaciones Lineales para los Segundos Momentos}
%___________________________________________________________________________________________

En el ap\'endice (\ref{Segundos.Momentos}) se demuestra que las ecuaciones para las ecuaciones parciales mixtas est\'an dadas por:



%___________________________________________________________________________________________
%\subsubsection{Mixtas para $z_{1}$:}
%___________________________________________________________________________________________
%1
\begin{eqnarray*}
f_{1}\left(1,1\right)&=&r_{2}P_{1}^{(2)}\left(1\right)+\mu_{1}^{2}R_{2}^{(2)}\left(1\right)+2\mu_{1}r_{2}\left(\frac{\mu_{1}}{1-\tilde{\mu}_{2}}f_{2}\left(2\right)+f_{2}\left(1\right)\right)+\frac{1}{1-\tilde{\mu}_{2}}P_{1}^{(2)}f_{2}\left(2\right)+\mu_{1}^{2}\tilde{\theta}_{2}^{(2)}\left(1\right)f_{2}\left(2\right)\\
&+&\frac{\mu_{1}}{1-\tilde{\mu}_{2}}f_{2}(1,2)+\frac{\mu_{1}}{1-\tilde{\mu}_{2}}\left(\frac{\mu_{1}}{1-\tilde{\mu}_{2}}f_{2}(2,2)+f_{2}(1,2)\right)+f_{2}(1,1),\\
f_{1}\left(2,1\right)&=&\mu_{1}r_{2}\tilde{\mu}_{2}+\mu_{1}\tilde{\mu}_{2}R_{2}^{(2)}\left(1\right)+r_{2}\tilde{\mu}_{2}\left(\frac{\mu_{1}}{1-\tilde{\mu}_{2}}f_{2}(2)+f_{2}(1)\right),\\
f_{1}\left(3,1\right)&=&\mu_{1}\hat{\mu}_{1}r_{2}+\mu_{1}\hat{\mu}_{1}R_{2}^{(2)}\left(1\right)+r_{2}\frac{\mu_{1}}{1-\tilde{\mu}_{2}}f_{2}(2)+r_{2}\hat{\mu}_{1}\left(\frac{\mu_{1}}{1-\tilde{\mu}_{2}}f_{2}(2)+f_{2}(1)\right)+\mu_{1}r_{2}\hat{F}_{2,1}^{(1)}(1)\\
&+&\left(\frac{\mu_{1}}{1-\tilde{\mu}_{2}}f_{2}(2)+f_{2}(1)\right)\hat{F}_{2,1}^{(1)}(1)+\frac{\mu_{1}\hat{\mu}_{1}}{1-\tilde{\mu}_{2}}f_{2}(2)+\mu_{1}\hat{\mu}_{1}\tilde{\theta}_{2}^{(2)}\left(1\right)f_{2}(2)+\mu_{1}\hat{\mu}_{1}\left(\frac{1}{1-\tilde{\mu}_{2}}\right)^{2}f_{2}(2,2)\\
&+&+\frac{\hat{\mu}_{1}}{1-\tilde{\mu}_{2}}f_{2}(1,2),\\
f_{1}\left(4,1\right)&=&\mu_{1}\hat{\mu}_{2}r_{2}+\mu_{1}\hat{\mu}_{2}R_{2}^{(2)}\left(1\right)+r_{2}\frac{\mu_{1}\hat{\mu}_{2}}{1-\tilde{\mu}_{2}}f_{2}(2)+\mu_{1}r_{2}\hat{F}_{2,2}^{(1)}(1)+r_{2}\hat{\mu}_{2}\left(\frac{\mu_{1}}{1-\tilde{\mu}_{2}}f_{2}(2)+f_{2}(1)\right)\\
&+&\hat{F}_{2,1}^{(1)}(1)\left(\frac{\mu_{1}}{1-\tilde{\mu}_{2}}f_{2}(2)+f_{2}(1)\right)+\frac{\mu_{1}\hat{\mu}_{2}}{1-\tilde{\mu}_{2}}f_{2}(2)
+\mu_{1}\hat{\mu}_{2}\tilde{\theta}_{2}^{(2)}\left(1\right)f_{2}(2)+\mu_{1}\hat{\mu}_{2}\left(\frac{1}{1-\tilde{\mu}_{2}}\right)^{2}f_{2}(2,2)\\
&+&\frac{\hat{\mu}_{2}}{1-\tilde{\mu}_{2}}f_{2}^{(1,2)},\\
\end{eqnarray*}
\begin{eqnarray*}
f_{1}\left(1,2\right)&=&\mu_{1}\tilde{\mu}_{2}r_{2}+\mu_{1}\tilde{\mu}_{2}R_{2}^{(2)}\left(1\right)+r_{2}\tilde{\mu}_{2}\left(\frac{\mu_{1}}{1-\tilde{\mu}_{2}}f_{2}(2)+f_{2}(1)\right),\\
f_{1}\left(2,2\right)&=&\tilde{\mu}_{2}^{2}R_{2}^{(2)}(1)+r_{2}\tilde{P}_{2}^{(2)}\left(1\right),\\
f_{1}\left(3,2\right)&=&\hat{\mu}_{1}\tilde{\mu}_{2}r_{2}+\hat{\mu}_{1}\tilde{\mu}_{2}R_{2}^{(2)}(1)+
r_{2}\frac{\hat{\mu}_{1}\tilde{\mu}_{2}}{1-\tilde{\mu}_{2}}f_{2}(2)+r_{2}\tilde{\mu}_{2}\hat{F}_{2,2}^{(1)}(1),\\
f_{1}\left(4,2\right)&=&\hat{\mu}_{2}\tilde{\mu}_{2}r_{2}+\hat{\mu}_{2}\tilde{\mu}_{2}R_{2}^{(2)}(1)+
r_{2}\frac{\hat{\mu}_{2}\tilde{\mu}_{2}}{1-\tilde{\mu}_{2}}f_{2}(2)+r_{2}\tilde{\mu}_{2}\hat{F}_{2,2}^{(1)}(1),\\
f_{1}\left(1,3\right)&=&\mu_{1}\hat{\mu}_{1}r_{2}+\mu_{1}\hat{\mu}_{1}R_{2}^{(2)}\left(1\right)+\frac{\mu_{1}\hat{\mu}_{1}}{1-\tilde{\mu}_{2}}f_{2}(2)+r_{2}\frac{\mu_{1}\hat{\mu}_{1}}{1-\tilde{\mu}_{2}}f_{2}(2)+\mu_{1}\hat{\mu}_{1}\tilde{\theta}_{2}^{(2)}\left(1\right)f_{2}(2)+r_{2}\mu_{1}\hat{F}_{2,1}^{(1)}(1)\\
&+&r_{2}\hat{\mu}_{1}\left(\frac{\mu_{1}}{1-\tilde{\mu}_{2}}f_{2}(2)+f_{2}\left(1\right)\right)+\left(\frac{\mu_{1}}{1-\tilde{\mu}_{2}}f_{2}\left(1\right)+f_{2}\left(1\right)\right)\hat{F}_{2,1}^{(1)}(1)\\
&+&\frac{\hat{\mu}_{1}}{1-\tilde{\mu}_{2}}\left(\frac{\mu_{1}}{1-\tilde{\mu}_{2}}f_{2}(2,2)+f_{2}^{(1,2)}\right),\\
f_{1}\left(2,3\right)&=&\tilde{\mu}_{2}\hat{\mu}_{1}r_{2}+\tilde{\mu}_{2}\hat{\mu}_{1}R_{2}^{(2)}\left(1\right)+r_{2}\frac{\tilde{\mu}_{2}\hat{\mu}_{1}}{1-\tilde{\mu}_{2}}f_{2}(2)+r_{2}\tilde{\mu}_{2}\hat{F}_{2,1}^{(1)}(1),\\
f_{1}\left(3,3\right)&=&\hat{\mu}_{1}^{2}R_{2}^{(2)}\left(1\right)+r_{2}\hat{P}_{1}^{(2)}\left(1\right)+2r_{2}\frac{\hat{\mu}_{1}^{2}}{1-\tilde{\mu}_{2}}f_{2}(2)+\hat{\mu}_{1}^{2}\tilde{\theta}_{2}^{(2)}\left(1\right)f_{2}(2)+\frac{1}{1-\tilde{\mu}_{2}}\hat{P}_{1}^{(2)}\left(1\right)f_{2}(2)\\
&+&\frac{\hat{\mu}_{1}^{2}}{1-\tilde{\mu}_{2}}f_{2}(2,2)+2r_{2}\hat{\mu}_{1}\hat{F}_{2,1}^{(1)}(1)+2\frac{\hat{\mu}_{1}}{1-\tilde{\mu}_{2}}f_{2}(2)\hat{F}_{2,1}^{(1)}(1)+\hat{f}_{2,1}^{(2)}(1),\\
f_{1}\left(4,3\right)&=&r_{2}\hat{\mu}_{2}\hat{\mu}_{1}+\hat{\mu}_{1}\hat{\mu}_{2}R_{2}^{(2)}(1)+\frac{\hat{\mu}_{1}\hat{\mu}_{2}}{1-\tilde{\mu}_{2}}f_{2}\left(2\right)+2r_{2}\frac{\hat{\mu}_{1}\hat{\mu}_{2}}{1-\tilde{\mu}_{2}}f_{2}\left(2\right)+\hat{\mu}_{2}\hat{\mu}_{1}\tilde{\theta}_{2}^{(2)}\left(1\right)f_{2}\left(2\right)+r_{2}\hat{\mu}_{1}\hat{F}_{2,2}^{(1)}(1)\\
&+&\frac{\hat{\mu}_{1}}{1-\tilde{\mu}_{2}}f_{2}\left(2\right)\hat{F}_{2,2}^{(1)}(1)+\hat{\mu}_{1}\hat{\mu}_{2}\left(\frac{1}{1-\tilde{\mu}_{2}}\right)^{2}f_{2}(2,2)+r_{2}\hat{\mu}_{2}\hat{F}_{2,1}^{(1)}(1)+\frac{\hat{\mu}_{2}}{1-\tilde{\mu}_{2}}f_{2}\left(2\right)\hat{F}_{2,1}^{(1)}(1)+\hat{f}_{2}(1,2),\\
f_{1}\left(1,4\right)&=&r_{2}\mu_{1}\hat{\mu}_{2}+\mu_{1}\hat{\mu}_{2}R_{2}^{(2)}(1)+\frac{\mu_{1}\hat{\mu}_{2}}{1-\tilde{\mu}_{2}}f_{2}(2)+r_{2}\frac{\mu_{1}\hat{\mu}_{2}}{1-\tilde{\mu}_{2}}f_{2}(2)+\mu_{1}\hat{\mu}_{2}\tilde{\theta}_{2}^{(2)}\left(1\right)f_{2}(2)+r_{2}\mu_{1}\hat{F}_{2,2}^{(1)}(1)\\
&+&r_{2}\hat{\mu}_{2}\left(\frac{\mu_{1}}{1-\tilde{\mu}_{2}}f_{2}(2)+f_{2}(1)\right)+\hat{F}_{2,2}^{(1)}(1)\left(\frac{\mu_{1}}{1-\tilde{\mu}_{2}}f_{2}(2)+f_{2}(1)\right)\\
&+&\frac{\hat{\mu}_{2}}{1-\tilde{\mu}_{2}}\left(\frac{\mu_{1}}{1-\tilde{\mu}_{2}}f_{2}(2,2)+f_{2}(1,2)\right),\\
f_{1}\left(2,4\right)
&=&r_{2}\tilde{\mu}_{2}\hat{\mu}_{2}+\tilde{\mu}_{2}\hat{\mu}_{2}R_{2}^{(2)}(1)+r_{2}\frac{\tilde{\mu}_{2}\hat{\mu}_{2}}{1-\tilde{\mu}_{2}}f_{2}(2)+r_{2}\tilde{\mu}_{2}\hat{F}_{2,2}^{(1)}(1),\\
f_{1}\left(3,4\right)&=&r_{2}\hat{\mu}_{1}\hat{\mu}_{2}+\hat{\mu}_{1}\hat{\mu}_{2}R_{2}^{(2)}\left(1\right)+\frac{\hat{\mu}_{1}\hat{\mu}_{2}}{1-\tilde{\mu}_{2}}f_{2}(2)+2r_{2}\frac{\hat{\mu}_{1}\hat{\mu}_{2}}{1-\tilde{\mu}_{2}}f_{2}(2)+\hat{\mu}_{1}\hat{\mu}_{2}\theta_{2}^{(2)}\left(1\right)f_{2}(2)+r_{2}\hat{\mu}_{1}\hat{F}_{2,2}^{(1)}(1)\\
&+&\frac{\hat{\mu}_{1}}{1-\tilde{\mu}_{2}}f_{2}(2)\hat{F}_{2,2}^{(1)}(1)+\hat{\mu}_{1}\hat{\mu}_{2}\left(\frac{1}{1-\tilde{\mu}_{2}}\right)^{2}f_{2}(2,2)+r_{2}\hat{\mu}_{2}\hat{F}_{2,2}^{(1)}(1)+\frac{\hat{\mu}_{2}}{1-\tilde{\mu}_{2}}f_{2}(2)\hat{F}_{2,1}^{(1)}(1)+\hat{f}_{2}^{(2)}(1,2),\\
f_{1}\left(4,4\right)&=&\hat{\mu}_{2}^{2}R_{2}^{(2)}(1)+r_{2}\hat{P}_{2}^{(2)}\left(1\right)+2r_{2}\frac{\hat{\mu}_{2}^{2}}{1-\tilde{\mu}_{2}}f_{2}(2)+\hat{\mu}_{2}^{2}\tilde{\theta}_{2}^{(2)}\left(1\right)f_{2}(2)+\frac{1}{1-\tilde{\mu}_{2}}\hat{P}_{2}^{(2)}\left(1\right)f_{2}(2)\\
&+&2r_{2}\hat{\mu}_{2}\hat{F}_{2,2}^{(1)}(1)+2\frac{\hat{\mu}_{2}}{1-\tilde{\mu}_{2}}f_{2}(2)\hat{F}_{2,2}^{(1)}(1)+\left(\frac{\hat{\mu}_{2}}{1-\tilde{\mu}_{2}}\right)^{2}f_{2}(2,2)+\hat{f}_{2,2}^{(2)}(1),\\
f_{2}\left(1,1\right)&=&r_{1}P_{1}^{(2)}\left(1\right)+\mu_{1}^{2}R_{1}^{(2)}\left(1\right),\\
f_{2}\left(2,1\right)&=&\mu_{1}\tilde{\mu}_{2}r_{1}+\mu_{1}\tilde{\mu}_{2}R_{1}^{(2)}(1)+
r_{1}\mu_{1}\left(\frac{\tilde{\mu}_{2}}{1-\mu_{1}}f_{1}(1)+f_{1}(2)\right),\\
f_{2}\left(3,1\right)&=&r_{1}\mu_{1}\hat{\mu}_{1}+\mu_{1}\hat{\mu}_{1}R_{1}^{(2)}\left(1\right)+r_{1}\frac{\mu_{1}\hat{\mu}_{1}}{1-\mu_{1}}f_{1}(1)+r_{1}\mu_{1}\hat{F}_{1,1}^{(1)}(1),\\
f_{2}\left(4,1\right)&=&\mu_{1}\hat{\mu}_{2}r_{1}+\mu_{1}\hat{\mu}_{2}R_{1}^{(2)}\left(1\right)+r_{1}\mu_{1}\hat{F}_{1,2}^{(1)}(1)+r_{1}\frac{\mu_{1}\hat{\mu}_{2}}{1-\mu_{1}}f_{1}(1),\\
\end{eqnarray*}
\begin{eqnarray*}
f_{2}\left(1,2\right)&=&r_{1}\mu_{1}\tilde{\mu}_{2}+\mu_{1}\tilde{\mu}_{2}R_{1}^{(2)}\left(1\right)+r_{1}\mu_{1}\left(\frac{\tilde{\mu}_{2}}{1-\mu_{1}}f_{1}(1)+f_{1}(2)\right),\\
f_{2}\left(2,2\right)&=&\tilde{\mu}_{2}^{2}R_{1}^{(2)}\left(1\right)+r_{1}\tilde{P}_{2}^{(2)}\left(1\right)+2r_{1}\tilde{\mu}_{2}\left(\frac{\tilde{\mu}_{2}}{1-\mu_{1}}f_{1}(1)+f_{1}(2)\right)+f_{1}(2,2)+\tilde{\mu}_{2}^{2}\theta_{1}^{(2)}\left(1\right)f_{1}(1)\\
&+&\frac{1}{1-\mu_{1}}\tilde{P}_{2}^{(2)}\left(1\right)f_{1}(1)+\frac{\tilde{\mu}_{2}}{1-\mu_{1}}f_{1}(1,2)+\frac{\tilde{\mu}_{2}}{1-\mu_{1}}\left(\frac{\tilde{\mu}_{2}}{1-\mu_{1}}f_{1}(1,1)+f_{1}(1,2)\right),\\
f_{2}\left(3,2\right)&=&\tilde{\mu}_{2}\hat{\mu}_{1}r_{1}+\tilde{\mu}_{2}\hat{\mu}_{1}R_{1}^{(2)}\left(1\right)+r_{1}\frac{\tilde{\mu}_{2}\hat{\mu}_{1}}{1-\mu_{1}}f_{1}(1)+\hat{\mu}_{1}r_{1}\left(\frac{\tilde{\mu}_{2}}{1-\mu_{1}}f_{1}(1)+f_{1}(2)\right)+r_{1}\tilde{\mu}_{2}\hat{F}_{1,1}^{(1)}(1)\\
&+&\left(\frac{\tilde{\mu}_{2}}{1-\mu_{1}}f_{1}(1)+f_{1}(2)\right)\hat{F}_{1,1}^{(1)}(1)+\frac{\tilde{\mu}_{2}\hat{\mu}_{1}}{1-\mu_{1}}f_{1}(1)+\tilde{\mu}_{2}\hat{\mu}_{1}\theta_{1}^{(2)}\left(1\right)f_{1}(1)+\frac{\hat{\mu}_{1}}{1-\mu_{1}}f_{1}(1,2)\\
&+&\left(\frac{1}{1-\mu_{1}}\right)^{2}\tilde{\mu}_{2}\hat{\mu}_{1}f_{1}(1,1),\\
f_{2}\left(4,2\right)&=&\hat{\mu}_{2}\tilde{\mu}_{2}r_{1}+\hat{\mu}_{2}\tilde{\mu}_{2}R_{1}^{(2)}(1)+r_{1}\tilde{\mu}_{2}\hat{F}_{1,2}^{(1)}(1)+r_{1}\frac{\hat{\mu}_{2}\tilde{\mu}_{2}}{1-\mu_{1}}f_{1}(1)+\hat{\mu}_{2}r_{1}\left(\frac{\tilde{\mu}_{2}}{1-\mu_{1}}f_{1}(1)+f_{1}(2)\right)\\
&+&\left(\frac{\tilde{\mu}_{2}}{1-\mu_{1}}f_{1}(1)+f_{1}(2)\right)\hat{F}_{1,2}^{(1)}(1)+\frac{\tilde{\mu}_{2}\hat{\mu_{2}}}{1-\mu_{1}}f_{1}(1)+\hat{\mu}_{2}\tilde{\mu}_{2}\theta_{1}^{(2)}\left(1\right)f_{1}(1)+\frac{\hat{\mu}_{2}}{1-\mu_{1}}f_{1}(1,2)\\
&+&\tilde{\mu}_{2}\hat{\mu}_{2}\left(\frac{1}{1-\mu_{1}}\right)^{2}f_{1}(1,1),\\
f_{2}\left(1,3\right)&=&r_{1}\mu_{1}\hat{\mu}_{1}+\mu_{1}\hat{\mu}_{1}R_{1}^{(2)}(1)+r_{1}\frac{\mu_{1}\hat{\mu}_{1}}{1-\mu_{1}}f_{1}(1)+r_{1}\mu_{1}\hat{F}_{1,1}^{(1)}(1),\\
 f_{2}\left(2,3\right)&=&r_{1}\hat{\mu}_{1}\tilde{\mu}_{2}+\tilde{\mu}_{2}\hat{\mu}_{1}R_{1}^{(2)}\left(1\right)+\frac{\hat{\mu}_{1}\tilde{\mu}_{2}}{1-\mu_{1}}f_{1}(1)+r_{1}\frac{\hat{\mu}_{1}\tilde{\mu}_{2}}{1-\mu_{1}}f_{1}(1)+\hat{\mu}_{1}\tilde{\mu}_{2}\theta_{1}^{(2)}\left(1\right)f_{1}(1)+r_{1}\tilde{\mu}_{2}\hat{F}_{1,1}(1)\\
&+&r_{1}\hat{\mu}_{1}\left(f_{1}(1)+\frac{\tilde{\mu}_{2}}{1-\mu_{1}}f_{1}(1)\right)+
+\left(f_{1}(2)+\frac{\tilde{\mu}_{2}}{1-\mu_{1}}f_{1}(1)\right)\hat{F}_{1,1}(1)\\
&+&\frac{\hat{\mu}_{1}}{1-\mu_{1}}\left(f_{1}(1,2)+\frac{\tilde{\mu}_{2}}{1-\mu_{1}}f_{1}(1,1)\right),\\
f_{2}\left(3,3\right)&=&\hat{\mu}_{1}^{2}R_{1}^{(2)}\left(1\right)+r_{1}\hat{P}_{1}^{(2)}\left(1\right)+2r_{1}\frac{\hat{\mu}_{1}^{2}}{1-\mu_{1}}f_{1}(1)+\hat{\mu}_{1}^{2}\theta_{1}^{(2)}\left(1\right)f_{1}(1)+2r_{1}\hat{\mu}_{1}\hat{F}_{1,1}^{(1)}(1)\\
&+&\frac{1}{1-\mu_{1}}\hat{P}_{1}^{(2)}\left(1\right)f_{1}(1)+2\frac{\hat{\mu}_{1}}{1-\mu_{1}}f_{1}(1)\hat{F}_{1,1}(1)+\left(\frac{\hat{\mu}_{1}}{1-\mu_{1}}\right)^{2}f_{1}(1,1)+\hat{f}_{1,1}^{(2)}(1),\\
f_{2}\left(4,3\right)&=&r_{1}\hat{\mu}_{1}\hat{\mu}_{2}+\hat{\mu}_{1}\hat{\mu}_{2}R_{1}^{(2)}\left(1\right)+r_{1}\hat{\mu}_{1}\hat{F}_{1,2}(1)+
\frac{\hat{\mu}_{1}\hat{\mu}_{2}}{1-\mu_{1}}f_{1}(1)+2r_{1}\frac{\hat{\mu}_{1}\hat{\mu}_{2}}{1-\mu_{1}}f_{1}(1)+r_{1}\hat{\mu}_{2}\hat{F}_{1,1}(1)\\
&+&\hat{\mu}_{1}\hat{\mu}_{2}\theta_{1}^{(2)}\left(1\right)f_{1}(1)+\frac{\hat{\mu}_{1}}{1-\mu_{1}}f_{1}(1)\hat{F}_{1,2}(1)+\frac{\hat{\mu}_{2}}{1-\mu_{1}}\hat{F}_{1,1}(1)f_{1}(1)\\
&+&\hat{f}_{1}^{(2)}(1,2)+\hat{\mu}_{1}\hat{\mu}_{2}\left(\frac{1}{1-\mu_{1}}\right)^{2}f_{1}(2,2),\\
f_{2}\left(1,4\right)&=&r_{1}\mu_{1}\hat{\mu}_{2}+\mu_{1}\hat{\mu}_{2}R_{1}^{(2)}\left(1\right)+r_{1}\mu_{1}\hat{F}_{1,2}(1)+r_{1}\frac{\mu_{1}\hat{\mu}_{2}}{1-\mu_{1}}f_{1}(1),\\
f_{2}\left(2,4\right)&=&r_{1}\hat{\mu}_{2}\tilde{\mu}_{2}+\hat{\mu}_{2}\tilde{\mu}_{2}R_{1}^{(2)}\left(1\right)+r_{1}\tilde{\mu}_{2}\hat{F}_{1,2}(1)+\frac{\hat{\mu}_{2}\tilde{\mu}_{2}}{1-\mu_{1}}f_{1}(1)+r_{1}\frac{\hat{\mu}_{2}\tilde{\mu}_{2}}{1-\mu_{1}}f_{1}(1)+\hat{\mu}_{2}\tilde{\mu}_{2}\theta_{1}^{(2)}\left(1\right)f_{1}(1)\\
&+&r_{1}\hat{\mu}_{2}\left(f_{1}(2)+\frac{\tilde{\mu}_{2}}{1-\mu_{1}}f_{1}(1)\right)+\left(f_{1}(2)+\frac{\tilde{\mu}_{2}}{1-\mu_{1}}f_{1}(1)\right)\hat{F}_{1,2}(1)\\&+&\frac{\hat{\mu}_{2}}{1-\mu_{1}}\left(f_{1}(1,2)+\frac{\tilde{\mu}_{2}}{1-\mu_{1}}f_{1}(1,1)\right),\\
\end{eqnarray*}
\begin{eqnarray*}
f_{2}\left(3,4\right)&=&r_{1}\hat{\mu}_{1}\hat{\mu}_{2}+\hat{\mu}_{1}\hat{\mu}_{2}R_{1}^{(2)}\left(1\right)+r_{1}\hat{\mu}_{1}\hat{F}_{1,2}(1)+
\frac{\hat{\mu}_{1}\hat{\mu}_{2}}{1-\mu_{1}}f_{1}(1)+2r_{1}\frac{\hat{\mu}_{1}\hat{\mu}_{2}}{1-\mu_{1}}f_{1}(1)+\hat{\mu}_{1}\hat{\mu}_{2}\theta_{1}^{(2)}\left(1\right)f_{1}(1)\\
&+&+\frac{\hat{\mu}_{1}}{1-\mu_{1}}\hat{F}_{1,2}(1)f_{1}(1)+r_{1}\hat{\mu}_{2}\hat{F}_{1,1}(1)+\frac{\hat{\mu}_{2}}{1-\mu_{1}}\hat{F}_{1,1}(1)f_{1}(1)+\hat{f}_{1}^{(2)}(1,2)+\hat{\mu}_{1}\hat{\mu}_{2}\left(\frac{1}{1-\mu_{1}}\right)^{2}f_{1}(1,1),\\
f_{2}\left(4,4\right)&=&\hat{\mu}_{2}R_{1}^{(2)}\left(1\right)+r_{1}\hat{P}_{2}^{(2)}\left(1\right)+2r_{1}\hat{\mu}_{2}\hat{F}_{1}^{(0,1)}+\hat{f}_{1,2}^{(2)}(1)+2r_{1}\frac{\hat{\mu}_{2}^{2}}{1-\mu_{1}}f_{1}(1)+\hat{\mu}_{2}^{2}\theta_{1}^{(2)}\left(1\right)f_{1}(1)\\
&+&\frac{1}{1-\mu_{1}}\hat{P}_{2}^{(2)}\left(1\right)f_{1}(1) +
2\frac{\hat{\mu}_{2}}{1-\mu_{1}}f_{1}(1)\hat{F}_{1,2}(1)+\left(\frac{\hat{\mu}_{2}}{1-\mu_{1}}\right)^{2}f_{1}(1,1),\\
\hat{f}_{1}\left(1,1\right)&=&\hat{r}_{2}P_{1}^{(2)}\left(1\right)+
\mu_{1}^{2}\hat{R}_{2}^{(2)}\left(1\right)+
2\hat{r}_{2}\frac{\mu_{1}^{2}}{1-\hat{\mu}_{2}}\hat{f}_{2}(2)+
\frac{1}{1-\hat{\mu}_{2}}P_{1}^{(2)}\left(1\right)\hat{f}_{2}(2)+
\mu_{1}^{2}\hat{\theta}_{2}^{(2)}\left(1\right)\hat{f}_{2}(2)\\
&+&\left(\frac{\mu_{1}^{2}}{1-\hat{\mu}_{2}}\right)^{2}\hat{f}_{2}(2,2)+2\hat{r}_{2}\mu_{1}F_{2,1}(1)+2\frac{\mu_{1}}{1-\hat{\mu}_{2}}\hat{f}_{2}(2)F_{2,1}(1)+F_{2,1}^{(2)}(1),\\
\hat{f}_{1}\left(2,1\right)&=&\hat{r}_{2}\mu_{1}\tilde{\mu}_{2}+\mu_{1}\tilde{\mu}_{2}\hat{R}_{2}^{(2)}\left(1\right)+\hat{r}_{2}\mu_{1}F_{2,2}(1)+
\frac{\mu_{1}\tilde{\mu}_{2}}{1-\hat{\mu}_{2}}\hat{f}_{2}(2)+2\hat{r}_{2}\frac{\mu_{1}\tilde{\mu}_{2}}{1-\hat{\mu}_{2}}\hat{f}_{2}(2)\\
&+&\mu_{1}\tilde{\mu}_{2}\hat{\theta}_{2}^{(2)}\left(1\right)\hat{f}_{2}(2)+\frac{\mu_{1}}{1-\hat{\mu}_{2}}F_{2,2}(1)\hat{f}_{2}(2)+\mu_{1} \tilde{\mu}_{2}\left(\frac{1}{1-\hat{\mu}_{2}}\right)^{2}\hat{f}_{2}(2,2)+\hat{r}_{2}\tilde{\mu}_{2}F_{2,1}(1)\\
&+&\frac{\tilde{\mu}_{2}}{1-\hat{\mu}_{2}}\hat{f}_{2}(2)F_{2,1}(1)+f_{2,1}^{(2)}(1),\\
\hat{f}_{1}\left(3,1\right)&=&\hat{r}_{2}\mu_{1}\hat{\mu}_{1}+\mu_{1}\hat{\mu}_{1}\hat{R}_{2}^{(2)}\left(1\right)+\hat{r}_{2}\frac{\mu_{1}\hat{\mu}_{1}}{1-\hat{\mu}_{2}}\hat{f}_{2}(2)+\hat{r}_{2}\hat{\mu}_{1}F_{2,1}(1)+\hat{r}_{2}\mu_{1}\hat{f}_{2}(1)\\
&+&F_{2,1}(1)\hat{f}_{2}(1)+\frac{\mu_{1}}{1-\hat{\mu}_{2}}\hat{f}_{2}(1,2),\\
\hat{f}_{1}\left(4,1\right)&=&\hat{r}_{2}\mu_{1}\hat{\mu}_{2}+\mu_{1}\hat{\mu}_{2}\hat{R}_{2}^{(2)}\left(1\right)+\frac{\mu_{1}\hat{\mu}_{2}}{1-\hat{\mu}_{2}}\hat{f}_{2}(2)+2\hat{r}_{2}\frac{\mu_{1}\hat{\mu}_{2}}{1-\hat{\mu}_{2}}\hat{f}_{2}(2)+\mu_{1}\hat{\mu}_{2}\hat{\theta}_{2}^{(2)}\left(1\right)\hat{f}_{2}(2)\\
&+&\mu_{1}\hat{\mu}_{2}\left(\frac{1}{1-\hat{\mu}_{2}}\right)^{2}\hat{f}_{2}(2,2)+\hat{r}_{2}\hat{\mu}_{2}F_{2,1}(1)+\frac{\hat{\mu}_{2}}{1-\hat{\mu}_{2}}\hat{f}_{2}(2)F_{2,1}(1),\\
\hat{f}_{1}\left(1,2\right)&=&\hat{r}_{2}\mu_{1}\tilde{\mu}_{2}+\mu_{1}\tilde{\mu}_{2}\hat{R}_{2}^{(2)}\left(1\right)+\mu_{1}\hat{r}_{2}F_{2,2}(1)+
\frac{\mu_{1}\tilde{\mu}_{2}}{1-\hat{\mu}_{2}}\hat{f}_{2}(2)+2\hat{r}_{2}\frac{\mu_{1}\tilde{\mu}_{2}}{1-\hat{\mu}_{2}}\hat{f}_{2}(2)\\
&+&\mu_{1}\tilde{\mu}_{2}\hat{\theta}_{2}^{(2)}\left(1\right)\hat{f}_{2}(2)+\frac{\mu_{1}}{1-\hat{\mu}_{2}}F_{2,2}(1)\hat{f}_{2}(2)+\mu_{1}\tilde{\mu}_{2}\left(\frac{1}{1-\hat{\mu}_{2}}\right)^{2}\hat{f}_{2}(2,2)\\
&+&\hat{r}_{2}\tilde{\mu}_{2}F_{2,1}(1)+\frac{\tilde{\mu}_{2}}{1-\hat{\mu}_{2}}\hat{f}_{2}(2)F_{2,1}(1)+f_{2}^{(2)}(1,2),\\
\hat{f}_{1}\left(2,2\right)&=&\hat{r}_{2}\tilde{P}_{2}^{(2)}\left(1\right)+\tilde{\mu}_{2}^{2}\hat{R}_{2}^{(2)}\left(1\right)+2\hat{r}_{2}\tilde{\mu}_{2}F_{2,2}(1)+2\hat{r}_{2}\frac{\tilde{\mu}_{2}^{2}}{1-\hat{\mu}_{2}}\hat{f}_{2}(2)+f_{2,2}^{(2)}(1)\\
&+&\frac{1}{1-\hat{\mu}_{2}}\tilde{P}_{2}^{(2)}\left(1\right)\hat{f}_{2}(2)+\tilde{\mu}_{2}^{2}\hat{\theta}_{2}^{(2)}\left(1\right)\hat{f}_{2}(2)+2\frac{\tilde{\mu}_{2}}{1-\hat{\mu}_{2}}F_{2,2}(1)\hat{f}_{2}(2)+\left(\frac{\tilde{\mu}_{2}}{1-\hat{\mu}_{2}}\right)^{2}\hat{f}_{2}(2,2),\\
\hat{f}_{1}\left(3,2\right)&=&\hat{r}_{2}\tilde{\mu}_{2}\hat{\mu}_{1}+\tilde{\mu}_{2}\hat{\mu}_{1}\hat{R}_{2}^{(2)}\left(1\right)+\hat{r}_{2}\hat{\mu}_{1}F_{2,2}(1)+\hat{r}_{2}\frac{\tilde{\mu}_{2}\hat{\mu}_{1}}{1-\hat{\mu}_{2}}\hat{f}_{2}(2)+\hat{r}_{2}\tilde{\mu}_{2}\hat{f}_{2}(1)+F_{2,2}(1)\hat{f}_{2}(1)\\
&+&\frac{\tilde{\mu}_{2}}{1-\hat{\mu}_{2}}\hat{f}_{2}(1,2),\\
\hat{f}_{1}\left(4,2\right)&=&\hat{r}_{2}\tilde{\mu}_{2}\hat{\mu}_{2}+\tilde{\mu}_{2}\hat{\mu}_{2}\hat{R}_{2}^{(2)}\left(1\right)+\hat{r}_{2}\hat{\mu}_{2}F_{2,2}(1)+
\frac{\tilde{\mu}_{2}\hat{\mu}_{2}}{1-\hat{\mu}_{2}}\hat{f}_{2}(2)+2\hat{r}_{2}\frac{\tilde{\mu}_{2}\hat{\mu}_{2}}{1-\hat{\mu}_{2}}\hat{f}_{2}(2)\\
&+&\tilde{\mu}_{2}\hat{\mu}_{2}\hat{\theta}_{2}^{(2)}\left(1\right)\hat{f}_{2}(2)+\frac{\hat{\mu}_{2}}{1-\hat{\mu}_{2}}F_{2,2}(1)\hat{f}_{2}(1)+\tilde{\mu}_{2}\hat{\mu}_{2}\left(\frac{1}{1-\hat{\mu}_{2}}\right)\hat{f}_{2}(2,2),\\
\end{eqnarray*}
\begin{eqnarray*}
\hat{f}_{1}\left(1,3\right)&=&\hat{r}_{2}\mu_{1}\hat{\mu}_{1}+\mu_{1}\hat{\mu}_{1}\hat{R}_{2}^{(2)}\left(1\right)+\hat{r}_{2}\frac{\mu_{1}\hat{\mu}_{1}}{1-\hat{\mu}_{2}}\hat{f}_{2}(2)+\hat{r}_{2}\hat{\mu}_{1}F_{2,1}(1)+\hat{r}_{2}\mu_{1}\hat{f}_{2}(1)\\
&+&F_{2,1}(1)\hat{f}_{2}(1)+\frac{\mu_{1}}{1-\hat{\mu}_{2}}\hat{f}_{2}(1,2),\\
\hat{f}_{1}\left(2,3\right)&=&\hat{r}_{2}\tilde{\mu}_{2}\hat{\mu}_{1}+\tilde{\mu}_{2}\hat{\mu}_{1}\hat{R}_{2}^{(2)}\left(1\right)+\hat{r}_{2}\hat{\mu}_{1}F_{2,2}(1)+\hat{r}_{2}\frac{\tilde{\mu}_{2}\hat{\mu}_{1}}{1-\hat{\mu}_{2}}\hat{f}_{2}(2)+\hat{r}_{2}\tilde{\mu}_{2}\hat{f}_{2}(1)\\
&+&F_{2,2}(1)\hat{f}_{2}(1)+\frac{\tilde{\mu}_{2}}{1-\hat{\mu}_{2}}\hat{f}_{2}(1,2),\\
\hat{f}_{1}\left(3,3\right)&=&\hat{r}_{2}\hat{P}_{1}^{(2)}\left(1\right)+\hat{\mu}_{1}^{2}\hat{R}_{2}^{(2)}\left(1\right)+2\hat{r}_{2}\hat{\mu}_{1}\hat{f}_{2}(1)+\hat{f}_{2}(1,1),\\
\hat{f}_{1}\left(4,3\right)&=&\hat{r}_{2}\hat{\mu}_{1}\hat{\mu}_{2}+\hat{\mu}_{1}\hat{\mu}_{2}\hat{R}_{2}^{(2)}\left(1\right)+
\hat{r}_{2}\frac{\hat{\mu}_{2}\hat{\mu}_{1}}{1-\hat{\mu}_{2}}\hat{f}_{2}(2)+\hat{r}_{2}\hat{\mu}_{2}\hat{f}_{2}(1)+\frac{\hat{\mu}_{2}}{1-\hat{\mu}_{2}}\hat{f}_{2}(1,2),\\
\hat{f}_{1}\left(1,4\right)&=&\hat{r}_{2}\mu_{1}\hat{\mu}_{2}+\mu_{1}\hat{\mu}_{2}\hat{R}_{2}^{(2)}\left(1\right)+
\frac{\mu_{1}\hat{\mu}_{2}}{1-\hat{\mu}_{2}}\hat{f}_{2}(2) +2\hat{r}_{2}\frac{\mu_{1}\hat{\mu}_{2}}{1-\hat{\mu}_{2}}\hat{f}_{2}(2)\\
&+&\mu_{1}\hat{\mu}_{2}\hat{\theta}_{2}^{(2)}\left(1\right)\hat{f}_{2}(2)+\mu_{1}\hat{\mu}_{2}\left(\frac{1}{1-\hat{\mu}_{2}}\right)^{2}\hat{f}_{2}(2,2)+\hat{r}_{2}\hat{\mu}_{2}F_{2,1}(1)+\frac{\hat{\mu}_{2}}{1-\hat{\mu}_{2}}\hat{f}_{2}(2)F_{2,1}(1),\\\hat{f}_{1}\left(2,4\right)&=&\hat{r}_{2}\tilde{\mu}_{2}\hat{\mu}_{2}+\tilde{\mu}_{2}\hat{\mu}_{2}\hat{R}_{2}^{(2)}\left(1\right)+\hat{r}_{2}\hat{\mu}_{2}F_{2,2}(1)+\frac{\tilde{\mu}_{2}\hat{\mu}_{2}}{1-\hat{\mu}_{2}}\hat{f}_{2}(2)+2\hat{r}_{2}\frac{\tilde{\mu}_{2}\hat{\mu}_{2}}{1-\hat{\mu}_{2}}\hat{f}_{2}(2)\\
&+&\tilde{\mu}_{2}\hat{\mu}_{2}\hat{\theta}_{2}^{(2)}\left(1\right)\hat{f}_{2}(2)+\frac{\hat{\mu}_{2}}{1-\hat{\mu}_{2}}\hat{f}_{2}(2)F_{2,2}(1)+\tilde{\mu}_{2}\hat{\mu}_{2}\left(\frac{1}{1-\hat{\mu}_{2}}\right)^{2}\hat{f}_{2}(2,2),\\
\hat{f}_{1}\left(3,4\right)&=&\hat{r}_{2}\hat{\mu}_{1}\hat{\mu}_{2}+\hat{\mu}_{1}\hat{\mu}_{2}\hat{R}_{2}^{(2)}\left(1\right)+
\hat{r}_{2}\frac{\hat{\mu}_{1}\hat{\mu}_{2}}{1-\hat{\mu}_{2}}\hat{f}_{2}(2)+
\hat{r}_{2}\hat{\mu}_{2}\hat{f}_{2}(1)+\frac{\hat{\mu}_{2}}{1-\hat{\mu}_{2}}\hat{f}_{2}(1,2),\\
\hat{f}_{1}\left(4,4\right)&=&\hat{r}_{2}P_{2}^{(2)}\left(1\right)+\hat{\mu}_{2}^{2}\hat{R}_{2}^{(2)}\left(1\right)+2\hat{r}_{2}\frac{\hat{\mu}_{2}^{2}}{1-\hat{\mu}_{2}}\hat{f}_{2}(2)+\frac{1}{1-\hat{\mu}_{2}}\hat{P}_{2}^{(2)}\left(1\right)\hat{f}_{2}(2)\\
&+&\hat{\mu}_{2}^{2}\hat{\theta}_{2}^{(2)}\left(1\right)\hat{f}_{2}(2)+\left(\frac{\hat{\mu}_{2}}{1-\hat{\mu}_{2}}\right)^{2}\hat{f}_{2}(2,2),\\
\hat{f}_{2}\left(,1\right)&=&\hat{r}_{1}P_{1}^{(2)}\left(1\right)+
\mu_{1}^{2}\hat{R}_{1}^{(2)}\left(1\right)+2\hat{r}_{1}\mu_{1}F_{1,1}(1)+
2\hat{r}_{1}\frac{\mu_{1}^{2}}{1-\hat{\mu}_{1}}\hat{f}_{1}(1)+\frac{1}{1-\hat{\mu}_{1}}P_{1}^{(2)}\left(1\right)\hat{f}_{1}(1)\\
&+&\mu_{1}^{2}\hat{\theta}_{1}^{(2)}\left(1\right)\hat{f}_{1}(1)+2\frac{\mu_{1}}{1-\hat{\mu}_{1}}\hat{f}_{1}^(1)F_{1,1}(1)+f_{1,1}^{(2)}(1)+\left(\frac{\mu_{1}}{1-\hat{\mu}_{1}}\right)^{2}\hat{f}_{1}^{(1,1)},\\
\hat{f}_{2}\left(2,1\right)&=&\hat{r}_{1}\mu_{1}\tilde{\mu}_{2}+\mu_{1}\tilde{\mu}_{2}\hat{R}_{1}^{(2)}\left(1\right)+
\hat{r}_{1}\mu_{1}F_{1,2}(1)+\tilde{\mu}_{2}\hat{r}_{1}F_{1,1}(1)+
\frac{\mu_{1}\tilde{\mu}_{2}}{1-\hat{\mu}_{1}}\hat{f}_{1}(1)\\
&+&2\hat{r}_{1}\frac{\mu_{1}\tilde{\mu}_{2}}{1-\hat{\mu}_{1}}\hat{f}_{1}(1)+\mu_{1}\tilde{\mu}_{2}\hat{\theta}_{1}^{(2)}\left(1\right)\hat{f}_{1}(1)+
\frac{\mu_{1}}{1-\hat{\mu}_{1}}\hat{f}_{1}(1)F_{1,2}(1)+\frac{\tilde{\mu}_{2}}{1-\hat{\mu}_{1}}\hat{f}_{1}(1)F_{1,1}(1)\\
&+&f_{1}^{(2)}(1,2)+\mu_{1}\tilde{\mu}_{2}\left(\frac{1}{1-\hat{\mu}_{1}}\right)^{2}\hat{f}_{1}(1,1),\\
\hat{f}_{2}\left(3,1\right)&=&\hat{r}_{1}\mu_{1}\hat{\mu}_{1}+\mu_{1}\hat{\mu}_{1}\hat{R}_{1}^{(2)}\left(1\right)+\hat{r}_{1}\hat{\mu}_{1}F_{1,1}(1)+\hat{r}_{1}\frac{\mu_{1}\hat{\mu}_{1}}{1-\hat{\mu}_{1}}\hat{F}_{1}(1),\\
\hat{f}_{2}\left(4,1\right)&=&\hat{r}_{1}\mu_{1}\hat{\mu}_{2}+\mu_{1}\hat{\mu}_{2}\hat{R}_{1}^{(2)}\left(1\right)+\hat{r}_{1}\hat{\mu}_{2}F_{1,1}(1)+\frac{\mu_{1}\hat{\mu}_{2}}{1-\hat{\mu}_{1}}\hat{f}_{1}(1)+\hat{r}_{1}\frac{\mu_{1}\hat{\mu}_{2}}{1-\hat{\mu}_{1}}\hat{f}_{1}(1)\\
&+&\mu_{1}\hat{\mu}_{2}\hat{\theta}_{1}^{(2)}\left(1\right)\hat{f}_{1}(1)+\hat{r}_{1}\mu_{1}\left(\hat{f}_{1}(2)+\frac{\hat{\mu}_{2}}{1-\hat{\mu}_{1}}\hat{f}_{1}(1)\right)+F_{1,1}(1)\left(\hat{f}_{1}(2)+\frac{\hat{\mu}_{2}}{1-\hat{\mu}_{1}}\hat{f}_{1}(1)\right)\\
&+&\frac{\mu_{1}}{1-\hat{\mu}_{1}}\left(\hat{f}_{1}(1,2)+\frac{\hat{\mu}_{2}}{1-\hat{\mu}_{1}}\hat{f}_{1}(1,1)\right),\\
\hat{f}_{2}\left(1,2\right)&=&\hat{r}_{1}\mu_{1}\tilde{\mu}_{2}+\mu_{1}\tilde{\mu}_{2}\hat{R}_{1}^{(2)}\left(1\right)+\hat{r}_{1}\mu_{1}F_{1,2}(1)+\hat{r}_{1}\tilde{\mu}_{2}F_{1,1}(1)+\frac{\mu_{1}\tilde{\mu}_{2}}{1-\hat{\mu}_{1}}\hat{f}_{1}(1)\\
&+&2\hat{r}_{1}\frac{\mu_{1}\tilde{\mu}_{2}}{1-\hat{\mu}_{1}}\hat{f}_{1}(1)+\mu_{1}\tilde{\mu}_{2}\hat{\theta}_{1}^{(2)}\left(1\right)\hat{f}_{1}(1)+\frac{\mu_{1}}{1-\hat{\mu}_{1}}\hat{f}_{1}(1)F_{1,2}(1)\\
&+&\frac{\tilde{\mu}_{2}}{1-\hat{\mu}_{1}}\hat{f}_{1}(1)F_{1,1}(1)+f_{1}^{(2)}(1,2)+\mu_{1}\tilde{\mu}_{2}\left(\frac{1}{1-\hat{\mu}_{1}}\right)^{2}\hat{f}_{1}(1,1),\\
\end{eqnarray*}
\begin{eqnarray*}
\hat{f}_{2}\left(2,2\right)&=&\hat{r}_{1}\tilde{P}_{2}^{(2)}\left(1\right)+\tilde{\mu}_{2}^{2}\hat{R}_{1}^{(2)}\left(1\right)+2\hat{r}_{1}\tilde{\mu}_{2}F_{1,2}(1)+ f_{1,2}^{(2)}(1)+2\hat{r}_{1}\frac{\tilde{\mu}_{2}^{2}}{1-\hat{\mu}_{1}}\hat{f}_{1}(1)\\
&+&\frac{1}{1-\hat{\mu}_{1}}\tilde{P}_{2}^{(2)}\left(1\right)\hat{f}_{1}(1)+\tilde{\mu}_{2}^{2}\hat{\theta}_{1}^{(2)}\left(1\right)\hat{f}_{1}(1)+2\frac{\tilde{\mu}_{2}}{1-\hat{\mu}_{1}}F_{1,2}(1)\hat{f}_{1}(1)+\left(\frac{\tilde{\mu}_{2}}{1-\hat{\mu}_{1}}\right)^{2}\hat{f}_{1}(1,1),\\
\hat{f}_{2}\left(3,2\right)&=&\hat{r}_{1}\hat{\mu}_{1}\tilde{\mu}_{2}+\hat{\mu}_{1}\tilde{\mu}_{2}\hat{R}_{1}^{(2)}\left(1\right)+
\hat{r}_{1}\hat{\mu}_{1}F_{1,2}(1)+\hat{r}_{1}\frac{\hat{\mu}_{1}\tilde{\mu}_{2}}{1-\hat{\mu}_{1}}\hat{f}_{1}(1),\\
\hat{f}_{2}\left(4,2\right)&=&\hat{r}_{1}\tilde{\mu}_{2}\hat{\mu}_{2}+\hat{\mu}_{2}\tilde{\mu}_{2}\hat{R}_{1}^{(2)}\left(1\right)+\hat{\mu}_{2}\hat{R}_{1}^{(2)}\left(1\right)F_{1,2}(1)+\frac{\hat{\mu}_{2}\tilde{\mu}_{2}}{1-\hat{\mu}_{1}}\hat{f}_{1}(1)\\
&+&\hat{r}_{1}\frac{\hat{\mu}_{2}\tilde{\mu}_{2}}{1-\hat{\mu}_{1}}\hat{f}_{1}(1)+\hat{\mu}_{2}\tilde{\mu}_{2}\hat{\theta}_{1}^{(2)}\left(1\right)\hat{f}_{1}(1)+\hat{r}_{1}\tilde{\mu}_{2}\left(\hat{f}_{1}(2)+\frac{\hat{\mu}_{2}}{1-\hat{\mu}_{1}}\hat{f}_{1}(1)\right)\\
&+&F_{1,2}(1)\left(\hat{f}_{1}(2)+\frac{\hat{\mu}_{2}}{1-\hat{\mu}_{1}}\hat{f}_{1}(1)\right)+\frac{\tilde{\mu}_{2}}{1-\hat{\mu}_{1}}\left(\hat{f}_{1}(1,2)+\frac{\hat{\mu}_{2}}{1-\hat{\mu}_{1}}\hat{f}_{1}(1,1)\right),\\
\hat{f}_{2}\left(1,3\right)&=&\hat{r}_{1}\mu_{1}\hat{\mu}_{1}+\mu_{1}\hat{\mu}_{1}\hat{R}_{1}^{(2)}\left(1\right)+\hat{r}_{1}\hat{\mu}_{1}F_{1,1}(1)+\hat{r}_{1}\frac{\mu_{1}\hat{\mu}_{1}}{1-\hat{\mu}_{1}}\hat{f}_{1}(1),\\
\hat{f}_{2}\left(2,3\right)&=&\hat{r}_{1}\tilde{\mu}_{2}\hat{\mu}_{1}+\tilde{\mu}_{2}\hat{\mu}_{1}\hat{R}_{1}^{(2)}\left(1\right)+\hat{r}_{1}\hat{\mu}_{1}F_{1,2}(1)+\hat{r}_{1}\frac{\tilde{\mu}_{2}\hat{\mu}_{1}}{1-\hat{\mu}_{1}}\hat{f}_{1}(1),\\
\hat{f}_{2}\left(3,3\right)&=&\hat{r}_{1}\hat{P}_{1}^{(2)}\left(1\right)+\hat{\mu}_{1}^{2}\hat{R}_{1}^{(2)}\left(1\right),\\
\hat{f}_{2}\left(4,3\right)&=&\hat{r}_{1}\hat{\mu}_{2}\hat{\mu}_{1}+\hat{\mu}_{2}\hat{\mu}_{1}\hat{R}_{1}^{(2)}\left(1\right)+\hat{r}_{1}\hat{\mu}_{1}\left(\hat{f}_{1}(2)+\frac{\hat{\mu}_{2}}{1-\hat{\mu}_{1}}\hat{f}_{1}(1)\right),\\
\hat{f}_{2}\left(1,4\right)&=&\hat{r}_{1}\mu_{1}\hat{\mu}_{2}+\mu_{1}\hat{\mu}_{2}\hat{R}_{1}^{(2)}\left(1\right)+\hat{r}_{1}\hat{\mu}_{2}F_{1,1}(1)+\hat{r}_{1}\frac{\mu_{1}\hat{\mu}_{2}}{1-\hat{\mu}_{1}}\hat{f}_{1}(1)+\hat{r}_{1}\mu_{1}\left(\hat{f}_{1}(2)+\frac{\hat{\mu}_{2}}{1-\hat{\mu}_{1}}\hat{f}_{1}(1)\right)\\
&+&F_{1,1}(1)\left(\hat{f}_{1}(2)+\frac{\hat{\mu}_{2}}{1-\hat{\mu}_{1}}\hat{f}_{1}(1)\right)+\frac{\mu_{1}\hat{\mu}_{2}}{1-\hat{\mu}_{1}}\hat{f}_{1}(1)+\mu_{1}\hat{\mu}_{2}\hat{\theta}_{1}^{(2)}\left(1\right)\hat{f}_{1}(1)\\
&+&\frac{\mu_{1}}{1-\hat{\mu}_{1}}\hat{f}_{1}(1,2)+\mu_{1}\hat{\mu}_{2}\left(\frac{1}{1-\hat{\mu}_{1}}\right)^{2}\hat{f}_{1}(1,1),\\
\hat{f}_{2}\left(2,4\right)&=&\hat{r}_{1}\tilde{\mu}_{2}\hat{\mu}_{2}+\tilde{\mu}_{2}\hat{\mu}_{2}\hat{R}_{1}^{(2)}\left(1\right)+\hat{r}_{1}\hat{\mu}_{2}F_{1,2}(1)+\hat{r}_{1}\frac{\tilde{\mu}_{2}\hat{\mu}_{2}}{1-\hat{\mu}_{1}}\hat{f}_{1}(1)\\
&+&\hat{r}_{1}\tilde{\mu}_{2}\left(\hat{f}_{1}(2)+\frac{\hat{\mu}_{2}}{1-\hat{\mu}_{1}}\hat{f}_{1}(1)\right)+F_{1,2}(1)\left(\hat{f}_{1}(2)+\frac{\hat{\mu}_{2}}{1-\hat{\mu}_{1}}\hat{F}_{1}^{(1,0)}\right)+\frac{\tilde{\mu}_{2}\hat{\mu}_{2}}{1-\hat{\mu}_{1}}\hat{f}_{1}(1)\\
&+&\tilde{\mu}_{2}\hat{\mu}_{2}\hat{\theta}_{1}^{(2)}\left(1\right)\hat{f}_{1}(1)+\frac{\tilde{\mu}_{2}}{1-\hat{\mu}_{1}}\hat{f}_{1}(1,2)+\tilde{\mu}_{2}\hat{\mu}_{2}\left(\frac{1}{1-\hat{\mu}_{1}}\right)^{2}\hat{f}_{1}(1,1),\\
\hat{f}_{2}\left(3,4\right)&=&\hat{r}_{1}\hat{\mu}_{2}\hat{\mu}_{1}+\hat{\mu}_{2}\hat{\mu}_{1}\hat{R}_{1}^{(2)}\left(1\right)+\hat{r}_{1}\hat{\mu}_{1}\left(\hat{f}_{1}(2)+\frac{\hat{\mu}_{2}}{1-\hat{\mu}_{1}}\hat{f}_{1}(1)\right),\\
\hat{f}_{2}\left(4,4\right)&=&\hat{r}_{1}\hat{P}_{2}^{(2)}\left(1\right)+\hat{\mu}_{2}^{2}\hat{R}_{1}^{(2)}\left(1\right)+
2\hat{r}_{1}\hat{\mu}_{2}\left(\hat{f}_{1}(2)+\frac{\hat{\mu}_{2}}{1-\hat{\mu}_{1}}\hat{f}_{1}(1)\right)+\hat{f}_{1}(2,2)\\
&+&\frac{1}{1-\hat{\mu}_{1}}\hat{P}_{2}^{(2)}\left(1\right)\hat{f}_{1}(1)+\hat{\mu}_{2}^{2}\hat{\theta}_{1}^{(2)}\left(1\right)\hat{f}_{1}(1)+\frac{\hat{\mu}_{2}}{1-\hat{\mu}_{1}}\hat{f}_{1}(1,2)\\
&+&\frac{\hat{\mu}_{2}}{1-\hat{\mu}_{1}}\left(\hat{f}_{1}(1,2)+\frac{\hat{\mu}_{2}}{1-\hat{\mu}_{1}}\hat{f}_{1}(1,1)\right).
\end{eqnarray*}
%_________________________________________________________________________________________________________
\section{Medidas de Desempe\~no}
%_________________________________________________________________________________________________________

\begin{Def}
Sea $L_{i}^{*}$el n\'umero de usuarios cuando el servidor visita la cola $Q_{i}$ para dar servicio, para $i=1,2$.
\end{Def}

Entonces
\begin{Prop} Para la cola $Q_{i}$, $i=1,2$, se tiene que el n\'umero de usuarios presentes al momento de ser visitada por el servidor est\'a dado por
\begin{eqnarray}
\esp\left[L_{i}^{*}\right]&=&f_{i}\left(i\right)\\
Var\left[L_{i}^{*}\right]&=&f_{i}\left(i,i\right)+\esp\left[L_{i}^{*}\right]-\esp\left[L_{i}^{*}\right]^{2}.
\end{eqnarray}
\end{Prop}


\begin{Def}
El tiempo de Ciclo $C_{i}$ es el periodo de tiempo que comienza
cuando la cola $i$ es visitada por primera vez en un ciclo, y
termina cuando es visitado nuevamente en el pr\'oximo ciclo, bajo condiciones de estabilidad.

\begin{eqnarray*}
C_{i}\left(z\right)=\esp\left[z^{\overline{\tau}_{i}\left(m+1\right)-\overline{\tau}_{i}\left(m\right)}\right]
\end{eqnarray*}
\end{Def}

\begin{Def}
El tiempo de intervisita $I_{i}$ es el periodo de tiempo que
comienza cuando se ha completado el servicio en un ciclo y termina
cuando es visitada nuevamente en el pr\'oximo ciclo.
\begin{eqnarray*}I_{i}\left(z\right)&=&\esp\left[z^{\tau_{i}\left(m+1\right)-\overline{\tau}_{i}\left(m\right)}\right]\end{eqnarray*}
\end{Def}

\begin{Prop}
Para los tiempos de intervisita del servidor $I_{i}$, se tiene que

\begin{eqnarray*}
\esp\left[I_{i}\right]&=&\frac{f_{i}\left(i\right)}{\mu_{i}},\\
Var\left[I_{i}\right]&=&\frac{Var\left[L_{i}^{*}\right]}{\mu_{i}^{2}}-\frac{\sigma_{i}^{2}}{\mu_{i}^{2}}f_{i}\left(i\right).
\end{eqnarray*}
\end{Prop}


\begin{Prop}
Para los tiempos que ocupa el servidor para atender a los usuarios presentes en la cola $Q_{i}$, con FGP denotada por $S_{i}$, se tiene que
\begin{eqnarray*}
\esp\left[S_{i}\right]&=&\frac{\esp\left[L_{i}^{*}\right]}{1-\mu_{i}}=\frac{f_{i}\left(i\right)}{1-\mu_{i}},\\
Var\left[S_{i}\right]&=&\frac{Var\left[L_{i}^{*}\right]}{\left(1-\mu_{i}\right)^{2}}+\frac{\sigma^{2}\esp\left[L_{i}^{*}\right]}{\left(1-\mu_{i}\right)^{3}}
\end{eqnarray*}
\end{Prop}


\begin{Prop}
Para la duraci\'on de los ciclos $C_{i}$ se tiene que
\begin{eqnarray*}
\esp\left[C_{i}\right]&=&\esp\left[I_{i}\right]\esp\left[\theta_{i}\left(z\right)\right]=\frac{\esp\left[L_{i}^{*}\right]}{\mu_{i}}\frac{1}{1-\mu_{i}}=\frac{f_{i}\left(i\right)}{\mu_{i}\left(1-\mu_{i}\right)}\\
Var\left[C_{i}\right]&=&\frac{Var\left[L_{i}^{*}\right]}{\mu_{i}^{2}\left(1-\mu_{i}\right)^{2}}.
\end{eqnarray*}

\end{Prop}

%___________________________________________________________________________________________
%
\section*{Ap\'endice A}\label{Segundos.Momentos}
%___________________________________________________________________________________________


%___________________________________________________________________________________________

%\subsubsection{Mixtas para $z_{1}$:}
%___________________________________________________________________________________________
\begin{enumerate}

%1/1/1
\item \begin{eqnarray*}
&&\frac{\partial}{\partial z_1}\frac{\partial}{\partial z_1}\left(R_2\left(P_1\left(z_1\right)\bar{P}_2\left(z_2\right)\hat{P}_1\left(w_1\right)\hat{P}_2\left(w_2\right)\right)F_2\left(z_1,\theta
_2\left(P_1\left(z_1\right)\hat{P}_1\left(w_1\right)\hat{P}_2\left(w_2\right)\right)\right)\hat{F}_2\left(w_1,w_2\right)\right)\\
&=&r_{2}P_{1}^{(2)}\left(1\right)+\mu_{1}^{2}R_{2}^{(2)}\left(1\right)+2\mu_{1}r_{2}\left(\frac{\mu_{1}}{1-\tilde{\mu}_{2}}F_{2}^{(0,1)}+F_{2}^{1,0)}\right)+\frac{1}{1-\tilde{\mu}_{2}}P_{1}^{(2)}F_{2}^{(0,1)}+\mu_{1}^{2}\tilde{\theta}_{2}^{(2)}\left(1\right)F_{2}^{(0,1)}\\
&+&\frac{\mu_{1}}{1-\tilde{\mu}_{2}}F_{2}^{(1,1)}+\frac{\mu_{1}}{1-\tilde{\mu}_{2}}\left(\frac{\mu_{1}}{1-\tilde{\mu}_{2}}F_{2}^{(0,2)}+F_{2}^{(1,1)}\right)+F_{2}^{(2,0)}.
\end{eqnarray*}

%2/2/1

\item \begin{eqnarray*}
&&\frac{\partial}{\partial z_2}\frac{\partial}{\partial z_1}\left(R_2\left(P_1\left(z_1\right)\bar{P}_2\left(z_2\right)\hat{P}_1\left(w_1\right)\hat{P}_2\left(w_2\right)\right)F_2\left(z_1,\theta
_2\left(P_1\left(z_1\right)\hat{P}_1\left(w_1\right)\hat{P}_2\left(w_2\right)\right)\right)\hat{F}_2\left(w_1,w_2\right)\right)\\
&=&\mu_{1}r_{2}\tilde{\mu}_{2}+\mu_{1}\tilde{\mu}_{2}R_{2}^{(2)}\left(1\right)+r_{2}\tilde{\mu}_{2}\left(\frac{\mu_{1}}{1-\tilde{\mu}_{2}}F_{2}^{(0,1)}+F_{2}^{(1,0)}\right).
\end{eqnarray*}
%3/3/1
\item \begin{eqnarray*}
&&\frac{\partial}{\partial w_1}\frac{\partial}{\partial z_1}\left(R_2\left(P_1\left(z_1\right)\bar{P}_2\left(z_2\right)\hat{P}_1\left(w_1\right)\hat{P}_2\left(w_2\right)\right)F_2\left(z_1,\theta
_2\left(P_1\left(z_1\right)\hat{P}_1\left(w_1\right)\hat{P}_2\left(w_2\right)\right)\right)\hat{F}_2\left(w_1,w_2\right)\right)\\
&=&\mu_{1}\hat{\mu}_{1}r_{2}+\mu_{1}\hat{\mu}_{1}R_{2}^{(2)}\left(1\right)+r_{2}\frac{\mu_{1}}{1-\tilde{\mu}_{2}}F_{2}^{(0,1)}+r_{2}\hat{\mu}_{1}\left(\frac{\mu_{1}}{1-\tilde{\mu}_{2}}F_{2}^{(0,1)}+F_{2}^{(1,0)}\right)+\mu_{1}r_{2}\hat{F}_{2}^{(1,0)}\\
&+&\left(\frac{\mu_{1}}{1-\tilde{\mu}_{2}}F_{2}^{(0,1)}+F_{2}^{(1,0)}\right)\hat{F}_{2}^{(1,0)}+\frac{\mu_{1}\hat{\mu}_{1}}{1-\tilde{\mu}_{2}}F_{2}^{(0,1)}+\mu_{1}\hat{\mu}_{1}\tilde{\theta}_{2}^{(2)}\left(1\right)F_{2}^{(0,1)}\\
&+&\mu_{1}\hat{\mu}_{1}\left(\frac{1}{1-\tilde{\mu}_{2}}\right)^{2}F_{2}^{(0,2)}+\frac{\hat{\mu}_{1}}{1-\tilde{\mu}_{2}}F_{2}^{(1,1)}.
\end{eqnarray*}
%4/4/1
\item \begin{eqnarray*}
&&\frac{\partial}{\partial w_2}\frac{\partial}{\partial z_1}\left(R_2\left(P_1\left(z_1\right)\bar{P}_2\left(z_2\right)\hat{P}_1\left(w_1\right)\hat{P}_2\left(w_2\right)\right)
F_2\left(z_1,\theta_2\left(P_1\left(z_1\right)\hat{P}_1\left(w_1\right)\hat{P}_2\left(w_2\right)\right)\right)\hat{F}_2\left(w_1,w_2\right)\right)\\
&=&\mu_{1}\hat{\mu}_{2}r_{2}+\mu_{1}\hat{\mu}_{2}R_{2}^{(2)}\left(1\right)+r_{2}\frac{\mu_{1}\hat{\mu}_{2}}{1-\tilde{\mu}_{2}}F_{2}^{(0,1)}+\mu_{1}r_{2}\hat{F}_{2}^{(0,1)}
+r_{2}\hat{\mu}_{2}\left(\frac{\mu_{1}}{1-\tilde{\mu}_{2}}F_{2}^{(0,1)}+F_{2}^{(1,0)}\right)\\
&+&\hat{F}_{2}^{(1,0)}\left(\frac{\mu_{1}}{1-\tilde{\mu}_{2}}F_{2}^{(0,1)}+F_{2}^{(1,0)}\right)+\frac{\mu_{1}\hat{\mu}_{2}}{1-\tilde{\mu}_{2}}F_{2}^{(0,1)}
+\mu_{1}\hat{\mu}_{2}\tilde{\theta}_{2}^{(2)}\left(1\right)F_{2}^{(0,1)}+\mu_{1}\hat{\mu}_{2}\left(\frac{1}{1-\tilde{\mu}_{2}}\right)^{2}F_{2}^{(0,2)}\\
&+&\frac{\hat{\mu}_{2}}{1-\tilde{\mu}_{2}}F_{2}^{(1,1)}.
\end{eqnarray*}
%___________________________________________________________________________________________
%\subsubsection{Mixtas para $z_{2}$:}
%___________________________________________________________________________________________
%5
\item \begin{eqnarray*} &&\frac{\partial}{\partial
z_1}\frac{\partial}{\partial
z_2}\left(R_2\left(P_1\left(z_1\right)\bar{P}_2\left(z_2\right)\hat{P}_1\left(w_1\right)\hat{P}_2\left(w_2\right)\right)
F_2\left(z_1,\theta_2\left(P_1\left(z_1\right)\hat{P}_1\left(w_1\right)\hat{P}_2\left(w_2\right)\right)\right)\hat{F}_2\left(w_1,w_2\right)\right)\\
&=&\mu_{1}\tilde{\mu}_{2}r_{2}+\mu_{1}\tilde{\mu}_{2}R_{2}^{(2)}\left(1\right)+r_{2}\tilde{\mu}_{2}\left(\frac{\mu_{1}}{1-\tilde{\mu}_{2}}F_{2}^{(0,1)}+F_{2}^{(1,0)}\right).
\end{eqnarray*}

%6

\item \begin{eqnarray*} &&\frac{\partial}{\partial
z_2}\frac{\partial}{\partial
z_2}\left(R_2\left(P_1\left(z_1\right)\bar{P}_2\left(z_2\right)\hat{P}_1\left(w_1\right)\hat{P}_2\left(w_2\right)\right)
F_2\left(z_1,\theta_2\left(P_1\left(z_1\right)\hat{P}_1\left(w_1\right)\hat{P}_2\left(w_2\right)\right)\right)\hat{F}_2\left(w_1,w_2\right)\right)\\
&=&\tilde{\mu}_{2}^{2}R_{2}^{(2)}(1)+r_{2}\tilde{P}_{2}^{(2)}\left(1\right).
\end{eqnarray*}

%7
\item \begin{eqnarray*} &&\frac{\partial}{\partial
w_1}\frac{\partial}{\partial
z_2}\left(R_2\left(P_1\left(z_1\right)\bar{P}_2\left(z_2\right)\hat{P}_1\left(w_1\right)\hat{P}_2\left(w_2\right)\right)
F_2\left(z_1,\theta_2\left(P_1\left(z_1\right)\hat{P}_1\left(w_1\right)\hat{P}_2\left(w_2\right)\right)\right)\hat{F}_2\left(w_1,w_2\right)\right)\\
&=&\hat{\mu}_{1}\tilde{\mu}_{2}r_{2}+\hat{\mu}_{1}\tilde{\mu}_{2}R_{2}^{(2)}(1)+
r_{2}\frac{\hat{\mu}_{1}\tilde{\mu}_{2}}{1-\tilde{\mu}_{2}}F_{2}^{(0,1)}+r_{2}\tilde{\mu}_{2}\hat{F}_{2}^{(1,0)}.
\end{eqnarray*}
%8
\item \begin{eqnarray*} &&\frac{\partial}{\partial
w_2}\frac{\partial}{\partial
z_2}\left(R_2\left(P_1\left(z_1\right)\bar{P}_2\left(z_2\right)\hat{P}_1\left(w_1\right)\hat{P}_2\left(w_2\right)\right)
F_2\left(z_1,\theta_2\left(P_1\left(z_1\right)\hat{P}_1\left(w_1\right)\hat{P}_2\left(w_2\right)\right)\right)\hat{F}_2\left(w_1,w_2\right)\right)\\
&=&\hat{\mu}_{2}\tilde{\mu}_{2}r_{2}+\hat{\mu}_{2}\tilde{\mu}_{2}R_{2}^{(2)}(1)+
r_{2}\frac{\hat{\mu}_{2}\tilde{\mu}_{2}}{1-\tilde{\mu}_{2}}F_{2}^{(0,1)}+r_{2}\tilde{\mu}_{2}\hat{F}_{2}^{(0,1)}.
\end{eqnarray*}
%___________________________________________________________________________________________
%\subsubsection{Mixtas para $w_{1}$:}
%___________________________________________________________________________________________

%9
\item \begin{eqnarray*} &&\frac{\partial}{\partial
z_1}\frac{\partial}{\partial
w_1}\left(R_2\left(P_1\left(z_1\right)\bar{P}_2\left(z_2\right)\hat{P}_1\left(w_1\right)\hat{P}_2\left(w_2\right)\right)
F_2\left(z_1,\theta_2\left(P_1\left(z_1\right)\hat{P}_1\left(w_1\right)\hat{P}_2\left(w_2\right)\right)\right)\hat{F}_2\left(w_1,w_2\right)\right)\\
&=&\mu_{1}\hat{\mu}_{1}r_{2}+\mu_{1}\hat{\mu}_{1}R_{2}^{(2)}\left(1\right)+\frac{\mu_{1}\hat{\mu}_{1}}{1-\tilde{\mu}_{2}}F_{2}^{(0,1)}+r_{2}\frac{\mu_{1}\hat{\mu}_{1}}{1-\tilde{\mu}_{2}}F_{2}^{(0,1)}+\mu_{1}\hat{\mu}_{1}\tilde{\theta}_{2}^{(2)}\left(1\right)F_{2}^{(0,1)}\\
&+&r_{2}\hat{\mu}_{1}\left(\frac{\mu_{1}}{1-\tilde{\mu}_{2}}F_{2}^{(0,1)}+F_{2}^{(1,0)}\right)+r_{2}\mu_{1}\hat{F}_{2}^{(1,0)}
+\left(\frac{\mu_{1}}{1-\tilde{\mu}_{2}}F_{2}^{(0,1)}+F_{2}^{(1,0)}\right)\hat{F}_{2}^{(1,0)}\\
&+&\frac{\hat{\mu}_{1}}{1-\tilde{\mu}_{2}}\left(\frac{\mu_{1}}{1-\tilde{\mu}_{2}}F_{2}^{(0,2)}+F_{2}^{(1,1)}\right).
\end{eqnarray*}
%10
\item \begin{eqnarray*} &&\frac{\partial}{\partial
z_2}\frac{\partial}{\partial
w_1}\left(R_2\left(P_1\left(z_1\right)\bar{P}_2\left(z_2\right)\hat{P}_1\left(w_1\right)\hat{P}_2\left(w_2\right)\right)
F_2\left(z_1,\theta_2\left(P_1\left(z_1\right)\hat{P}_1\left(w_1\right)\hat{P}_2\left(w_2\right)\right)\right)\hat{F}_2\left(w_1,w_2\right)\right)\\
&=&\tilde{\mu}_{2}\hat{\mu}_{1}r_{2}+\tilde{\mu}_{2}\hat{\mu}_{1}R_{2}^{(2)}\left(1\right)+r_{2}\frac{\tilde{\mu}_{2}\hat{\mu}_{1}}{1-\tilde{\mu}_{2}}F_{2}^{(0,1)}
+r_{2}\tilde{\mu}_{2}\hat{F}_{2}^{(1,0)}.
\end{eqnarray*}
%11
\item \begin{eqnarray*} &&\frac{\partial}{\partial
w_1}\frac{\partial}{\partial
w_1}\left(R_2\left(P_1\left(z_1\right)\bar{P}_2\left(z_2\right)\hat{P}_1\left(w_1\right)\hat{P}_2\left(w_2\right)\right)
F_2\left(z_1,\theta_2\left(P_1\left(z_1\right)\hat{P}_1\left(w_1\right)\hat{P}_2\left(w_2\right)\right)\right)\hat{F}_2\left(w_1,w_2\right)\right)\\
&=&\hat{\mu}_{1}^{2}R_{2}^{(2)}\left(1\right)+r_{2}\hat{P}_{1}^{(2)}\left(1\right)+2r_{2}\frac{\hat{\mu}_{1}^{2}}{1-\tilde{\mu}_{2}}F_{2}^{(0,1)}+
\hat{\mu}_{1}^{2}\tilde{\theta}_{2}^{(2)}\left(1\right)F_{2}^{(0,1)}+\frac{1}{1-\tilde{\mu}_{2}}\hat{P}_{1}^{(2)}\left(1\right)F_{2}^{(0,1)}\\
&+&\frac{\hat{\mu}_{1}^{2}}{1-\tilde{\mu}_{2}}F_{2}^{(0,2)}+2r_{2}\hat{\mu}_{1}\hat{F}_{2}^{(1,0)}+2\frac{\hat{\mu}_{1}}{1-\tilde{\mu}_{2}}F_{2}^{(0,1)}\hat{F}_{2}^{(1,0)}+\hat{F}_{2}^{(2,0)}.
\end{eqnarray*}
%12
\item \begin{eqnarray*} &&\frac{\partial}{\partial
w_2}\frac{\partial}{\partial
w_1}\left(R_2\left(P_1\left(z_1\right)\bar{P}_2\left(z_2\right)\hat{P}_1\left(w_1\right)\hat{P}_2\left(w_2\right)\right)
F_2\left(z_1,\theta_2\left(P_1\left(z_1\right)\hat{P}_1\left(w_1\right)\hat{P}_2\left(w_2\right)\right)\right)\hat{F}_2\left(w_1,w_2\right)\right)\\
&=&r_{2}\hat{\mu}_{2}\hat{\mu}_{1}+\hat{\mu}_{1}\hat{\mu}_{2}R_{2}^{(2)}(1)+\frac{\hat{\mu}_{1}\hat{\mu}_{2}}{1-\tilde{\mu}_{2}}F_{2}^{(0,1)}
+2r_{2}\frac{\hat{\mu}_{1}\hat{\mu}_{2}}{1-\tilde{\mu}_{2}}F_{2}^{(0,1)}+\hat{\mu}_{2}\hat{\mu}_{1}\tilde{\theta}_{2}^{(2)}\left(1\right)F_{2}^{(0,1)}+
r_{2}\hat{\mu}_{1}\hat{F}_{2}^{(0,1)}\\
&+&\frac{\hat{\mu}_{1}}{1-\tilde{\mu}_{2}}F_{2}^{(0,1)}\hat{F}_{2}^{(0,1)}+\hat{\mu}_{1}\hat{\mu}_{2}\left(\frac{1}{1-\tilde{\mu}_{2}}\right)^{2}F_{2}^{(0,2)}+
r_{2}\hat{\mu}_{2}\hat{F}_{2}^{(1,0)}+\frac{\hat{\mu}_{2}}{1-\tilde{\mu}_{2}}F_{2}^{(0,1)}\hat{F}_{2}^{(1,0)}+\hat{F}_{2}^{(1,1)}.
\end{eqnarray*}
%___________________________________________________________________________________________
%\subsubsection{Mixtas para $w_{2}$:}
%___________________________________________________________________________________________
%13

\item \begin{eqnarray*} &&\frac{\partial}{\partial
z_1}\frac{\partial}{\partial
w_2}\left(R_2\left(P_1\left(z_1\right)\bar{P}_2\left(z_2\right)\hat{P}_1\left(w_1\right)\hat{P}_2\left(w_2\right)\right)
F_2\left(z_1,\theta_2\left(P_1\left(z_1\right)\hat{P}_1\left(w_1\right)\hat{P}_2\left(w_2\right)\right)\right)\hat{F}_2\left(w_1,w_2\right)\right)\\
&=&r_{2}\mu_{1}\hat{\mu}_{2}+\mu_{1}\hat{\mu}_{2}R_{2}^{(2)}(1)+\frac{\mu_{1}\hat{\mu}_{2}}{1-\tilde{\mu}_{2}}F_{2}^{(0,1)}+r_{2}\frac{\mu_{1}\hat{\mu}_{2}}{1-\tilde{\mu}_{2}}F_{2}^{(0,1)}+\mu_{1}\hat{\mu}_{2}\tilde{\theta}_{2}^{(2)}\left(1\right)F_{2}^{(0,1)}+r_{2}\mu_{1}\hat{F}_{2}^{(0,1)}\\
&+&r_{2}\hat{\mu}_{2}\left(\frac{\mu_{1}}{1-\tilde{\mu}_{2}}F_{2}^{(0,1)}+F_{2}^{(1,0)}\right)+\hat{F}_{2}^{(0,1)}\left(\frac{\mu_{1}}{1-\tilde{\mu}_{2}}F_{2}^{(0,1)}+F_{2}^{(1,0)}\right)+\frac{\hat{\mu}_{2}}{1-\tilde{\mu}_{2}}\left(\frac{\mu_{1}}{1-\tilde{\mu}_{2}}F_{2}^{(0,2)}+F_{2}^{(1,1)}\right).
\end{eqnarray*}
%14
\item \begin{eqnarray*} &&\frac{\partial}{\partial
z_2}\frac{\partial}{\partial
w_2}\left(R_2\left(P_1\left(z_1\right)\bar{P}_2\left(z_2\right)\hat{P}_1\left(w_1\right)\hat{P}_2\left(w_2\right)\right)
F_2\left(z_1,\theta_2\left(P_1\left(z_1\right)\hat{P}_1\left(w_1\right)\hat{P}_2\left(w_2\right)\right)\right)\hat{F}_2\left(w_1,w_2\right)\right)\\
&=&r_{2}\tilde{\mu}_{2}\hat{\mu}_{2}+\tilde{\mu}_{2}\hat{\mu}_{2}R_{2}^{(2)}(1)+r_{2}\frac{\tilde{\mu}_{2}\hat{\mu}_{2}}{1-\tilde{\mu}_{2}}F_{2}^{(0,1)}+r_{2}\tilde{\mu}_{2}\hat{F}_{2}^{(0,1)}.
\end{eqnarray*}
%15
\item \begin{eqnarray*} &&\frac{\partial}{\partial
w_1}\frac{\partial}{\partial
w_2}\left(R_2\left(P_1\left(z_1\right)\bar{P}_2\left(z_2\right)\hat{P}_1\left(w_1\right)\hat{P}_2\left(w_2\right)\right)
F_2\left(z_1,\theta_2\left(P_1\left(z_1\right)\hat{P}_1\left(w_1\right)\hat{P}_2\left(w_2\right)\right)\right)\hat{F}_2\left(w_1,w_2\right)\right)\\
&=&r_{2}\hat{\mu}_{1}\hat{\mu}_{2}+\hat{\mu}_{1}\hat{\mu}_{2}R_{2}^{(2)}\left(1\right)+\frac{\hat{\mu}_{1}\hat{\mu}_{2}}{1-\tilde{\mu}_{2}}F_{2}^{(0,1)}+2r_{2}\frac{\hat{\mu}_{1}\hat{\mu}_{2}}{1-\tilde{\mu}_{2}}F_{2}^{(0,1)}+\hat{\mu}_{1}\hat{\mu}_{2}\theta_{2}^{(2)}\left(1\right)F_{2}^{(0,1)}+r_{2}\hat{\mu}_{1}\hat{F}_{2}^{(0,1)}\\
&+&\frac{\hat{\mu}_{1}}{1-\tilde{\mu}_{2}}F_{2}^{(0,1)}\hat{F}_{2}^{(0,1)}+\hat{\mu}_{1}\hat{\mu}_{2}\left(\frac{1}{1-\tilde{\mu}_{2}}\right)^{2}F_{2}^{(0,2)}+r_{2}\hat{\mu}_{2}\hat{F}_{2}^{(0,1)}+\frac{\hat{\mu}_{2}}{1-\tilde{\mu}_{2}}F_{2}^{(0,1)}\hat{F}_{2}^{(1,0)}+\hat{F}_{2}^{(1,1)}.
\end{eqnarray*}
%16

\item \begin{eqnarray*} &&\frac{\partial}{\partial
w_2}\frac{\partial}{\partial
w_2}\left(R_2\left(P_1\left(z_1\right)\bar{P}_2\left(z_2\right)\hat{P}_1\left(w_1\right)\hat{P}_2\left(w_2\right)\right)
F_2\left(z_1,\theta_2\left(P_1\left(z_1\right)\hat{P}_1\left(w_1\right)\hat{P}_2\left(w_2\right)\right)\right)\hat{F}_2\left(w_1,w_2\right)\right)\\
&=&\hat{\mu}_{2}^{2}R_{2}^{(2)}(1)+r_{2}\hat{P}_{2}^{(2)}\left(1\right)+2r_{2}\frac{\hat{\mu}_{2}^{2}}{1-\tilde{\mu}_{2}}F_{2}^{(0,1)}+\hat{\mu}_{2}^{2}\tilde{\theta}_{2}^{(2)}\left(1\right)F_{2}^{(0,1)}+\frac{1}{1-\tilde{\mu}_{2}}\hat{P}_{2}^{(2)}\left(1\right)F_{2}^{(0,1)}\\
&+&2r_{2}\hat{\mu}_{2}\hat{F}_{2}^{(0,1)}+2\frac{\hat{\mu}_{2}}{1-\tilde{\mu}_{2}}F_{2}^{(0,1)}\hat{F}_{2}^{(0,1)}+\left(\frac{\hat{\mu}_{2}}{1-\tilde{\mu}_{2}}\right)^{2}F_{2}^{(0,2)}+\hat{F}_{2}^{(0,2)}.
\end{eqnarray*}
\end{enumerate}
%___________________________________________________________________________________________
%
%\subsection{Derivadas de Segundo Orden para $F_{2}$}
%___________________________________________________________________________________________


\begin{enumerate}

%___________________________________________________________________________________________
%\subsubsection{Mixtas para $z_{1}$:}
%___________________________________________________________________________________________

%1/17
\item \begin{eqnarray*} &&\frac{\partial}{\partial
z_1}\frac{\partial}{\partial
z_1}\left(R_1\left(P_1\left(z_1\right)\bar{P}_2\left(z_2\right)\hat{P}_1\left(w_1\right)\hat{P}_2\left(w_2\right)\right)
F_1\left(\theta_1\left(\tilde{P}_2\left(z_1\right)\hat{P}_1\left(w_1\right)\hat{P}_2\left(w_2\right)\right)\right)\hat{F}_1\left(w_1,w_2\right)\right)\\
&=&r_{1}P_{1}^{(2)}\left(1\right)+\mu_{1}^{2}R_{1}^{(2)}\left(1\right).
\end{eqnarray*}

%2/18
\item \begin{eqnarray*} &&\frac{\partial}{\partial
z_2}\frac{\partial}{\partial
z_1}\left(R_1\left(P_1\left(z_1\right)\bar{P}_2\left(z_2\right)\hat{P}_1\left(w_1\right)\hat{P}_2\left(w_2\right)\right)F_1\left(\theta_1\left(\tilde{P}_2\left(z_1\right)\hat{P}_1\left(w_1\right)\hat{P}_2\left(w_2\right)\right)\right)\hat{F}_1\left(w_1,w_2\right)\right)\\
&=&\mu_{1}\tilde{\mu}_{2}r_{1}+\mu_{1}\tilde{\mu}_{2}R_{1}^{(2)}(1)+
r_{1}\mu_{1}\left(\frac{\tilde{\mu}_{2}}{1-\mu_{1}}F_{1}^{(1,0)}+F_{1}^{(0,1)}\right).
\end{eqnarray*}

%3/19
\item \begin{eqnarray*} &&\frac{\partial}{\partial
w_1}\frac{\partial}{\partial
z_1}\left(R_1\left(P_1\left(z_1\right)\bar{P}_2\left(z_2\right)\hat{P}_1\left(w_1\right)\hat{P}_2\left(w_2\right)\right)F_1\left(\theta_1\left(\tilde{P}_2\left(z_1\right)\hat{P}_1\left(w_1\right)\hat{P}_2\left(w_2\right)\right)\right)\hat{F}_1\left(w_1,w_2\right)\right)\\
&=&r_{1}\mu_{1}\hat{\mu}_{1}+\mu_{1}\hat{\mu}_{1}R_{1}^{(2)}\left(1\right)+r_{1}\frac{\mu_{1}\hat{\mu}_{1}}{1-\mu_{1}}F_{1}^{(1,0)}+r_{1}\mu_{1}\hat{F}_{1}^{(1,0)}.
\end{eqnarray*}
%4/20
\item \begin{eqnarray*} &&\frac{\partial}{\partial
w_2}\frac{\partial}{\partial
z_1}\left(R_1\left(P_1\left(z_1\right)\bar{P}_2\left(z_2\right)\hat{P}_1\left(w_1\right)\hat{P}_2\left(w_2\right)\right)F_1\left(\theta_1\left(\tilde{P}_2\left(z_1\right)\hat{P}_1\left(w_1\right)\hat{P}_2\left(w_2\right)\right)\right)\hat{F}_1\left(w_1,w_2\right)\right)\\
&=&\mu_{1}\hat{\mu}_{2}r_{1}+\mu_{1}\hat{\mu}_{2}R_{1}^{(2)}\left(1\right)+r_{1}\mu_{1}\hat{F}_{1}^{(0,1)}+r_{1}\frac{\mu_{1}\hat{\mu}_{2}}{1-\mu_{1}}F_{1}^{(1,0)}.
\end{eqnarray*}
%___________________________________________________________________________________________
%\subsubsection{Mixtas para $z_{2}$:}
%___________________________________________________________________________________________
%5/21
\item \begin{eqnarray*}
&&\frac{\partial}{\partial z_1}\frac{\partial}{\partial z_2}\left(R_1\left(P_1\left(z_1\right)\bar{P}_2\left(z_2\right)\hat{P}_1\left(w_1\right)\hat{P}_2\left(w_2\right)\right)F_1\left(\theta_1\left(\tilde{P}_2\left(z_1\right)\hat{P}_1\left(w_1\right)\hat{P}_2\left(w_2\right)\right)\right)\hat{F}_1\left(w_1,w_2\right)\right)\\
&=&r_{1}\mu_{1}\tilde{\mu}_{2}+\mu_{1}\tilde{\mu}_{2}R_{1}^{(2)}\left(1\right)+r_{1}\mu_{1}\left(\frac{\tilde{\mu}_{2}}{1-\mu_{1}}F_{1}^{(1,0)}+F_{1}^{(0,1)}\right).
\end{eqnarray*}

%6/22
\item \begin{eqnarray*}
&&\frac{\partial}{\partial z_2}\frac{\partial}{\partial z_2}\left(R_1\left(P_1\left(z_1\right)\bar{P}_2\left(z_2\right)\hat{P}_1\left(w_1\right)\hat{P}_2\left(w_2\right)\right)F_1\left(\theta_1\left(\tilde{P}_2\left(z_1\right)\hat{P}_1\left(w_1\right)\hat{P}_2\left(w_2\right)\right)\right)\hat{F}_1\left(w_1,w_2\right)\right)\\
&=&\tilde{\mu}_{2}^{2}R_{1}^{(2)}\left(1\right)+r_{1}\tilde{P}_{2}^{(2)}\left(1\right)+2r_{1}\tilde{\mu}_{2}\left(\frac{\tilde{\mu}_{2}}{1-\mu_{1}}F_{1}^{(1,0)}+F_{1}^{(0,1)}\right)+F_{1}^{(0,2)}+\tilde{\mu}_{2}^{2}\theta_{1}^{(2)}\left(1\right)F_{1}^{(1,0)}\\
&+&\frac{1}{1-\mu_{1}}\tilde{P}_{2}^{(2)}\left(1\right)F_{1}^{(1,0)}+\frac{\tilde{\mu}_{2}}{1-\mu_{1}}F_{1}^{(1,1)}+\frac{\tilde{\mu}_{2}}{1-\mu_{1}}\left(\frac{\tilde{\mu}_{2}}{1-\mu_{1}}F_{1}^{(2,0)}+F_{1}^{(1,1)}\right).
\end{eqnarray*}
%7/23
\item \begin{eqnarray*}
&&\frac{\partial}{\partial w_1}\frac{\partial}{\partial z_2}\left(R_1\left(P_1\left(z_1\right)\bar{P}_2\left(z_2\right)\hat{P}_1\left(w_1\right)\hat{P}_2\left(w_2\right)\right)F_1\left(\theta_1\left(\tilde{P}_2\left(z_1\right)\hat{P}_1\left(w_1\right)\hat{P}_2\left(w_2\right)\right)\right)\hat{F}_1\left(w_1,w_2\right)\right)\\
&=&\tilde{\mu}_{2}\hat{\mu}_{1}r_{1}+\tilde{\mu}_{2}\hat{\mu}_{1}R_{1}^{(2)}\left(1\right)+r_{1}\frac{\tilde{\mu}_{2}\hat{\mu}_{1}}{1-\mu_{1}}F_{1}^{(1,0)}+\hat{\mu}_{1}r_{1}\left(\frac{\tilde{\mu}_{2}}{1-\mu_{1}}F_{1}^{(1,0)}+F_{1}^{(0,1)}\right)+r_{1}\tilde{\mu}_{2}\hat{F}_{1}^{(1,0)}\\
&+&\left(\frac{\tilde{\mu}_{2}}{1-\mu_{1}}F_{1}^{(1,0)}+F_{1}^{(0,1)}\right)\hat{F}_{1}^{(1,0)}+\frac{\tilde{\mu}_{2}\hat{\mu}_{1}}{1-\mu_{1}}F_{1}^{(1,0)}+\tilde{\mu}_{2}\hat{\mu}_{1}\theta_{1}^{(2)}\left(1\right)F_{1}^{(1,0)}+\frac{\hat{\mu}_{1}}{1-\mu_{1}}F_{1}^{(1,1)}\\
&+&\left(\frac{1}{1-\mu_{1}}\right)^{2}\tilde{\mu}_{2}\hat{\mu}_{1}F_{1}^{(2,0)}.
\end{eqnarray*}
%8/24
\item \begin{eqnarray*}
&&\frac{\partial}{\partial w_2}\frac{\partial}{\partial z_2}\left(R_1\left(P_1\left(z_1\right)\bar{P}_2\left(z_2\right)\hat{P}_1\left(w_1\right)\hat{P}_2\left(w_2\right)\right)F_1\left(\theta_1\left(\tilde{P}_2\left(z_1\right)\hat{P}_1\left(w_1\right)\hat{P}_2\left(w_2\right)\right)\right)\hat{F}_1\left(w_1,w_2\right)\right)\\
&=&\hat{\mu}_{2}\tilde{\mu}_{2}r_{1}+\hat{\mu}_{2}\tilde{\mu}_{2}R_{1}^{(2)}(1)+r_{1}\tilde{\mu}_{2}\hat{F}_{1}^{(0,1)}+r_{1}\frac{\hat{\mu}_{2}\tilde{\mu}_{2}}{1-\mu_{1}}F_{1}^{(1,0)}+\hat{\mu}_{2}r_{1}\left(\frac{\tilde{\mu}_{2}}{1-\mu_{1}}F_{1}^{(1,0)}+F_{1}^{(0,1)}\right)\\
&+&\left(\frac{\tilde{\mu}_{2}}{1-\mu_{1}}F_{1}^{(1,0)}+F_{1}^{(0,1)}\right)\hat{F}_{1}^{(0,1)}+\frac{\tilde{\mu}_{2}\hat{\mu_{2}}}{1-\mu_{1}}F_{1}^{(1,0)}+\hat{\mu}_{2}\tilde{\mu}_{2}\theta_{1}^{(2)}\left(1\right)F_{1}^{(1,0)}+\frac{\hat{\mu}_{2}}{1-\mu_{1}}F_{1}^{(1,1)}\\
&+&\left(\frac{1}{1-\mu_{1}}\right)^{2}\tilde{\mu}_{2}\hat{\mu}_{2}F_{1}^{(2,0)}.
\end{eqnarray*}
%___________________________________________________________________________________________
%\subsubsection{Mixtas para $w_{1}$:}
%___________________________________________________________________________________________
%9/25
\item \begin{eqnarray*} &&\frac{\partial}{\partial
z_1}\frac{\partial}{\partial
w_1}\left(R_1\left(P_1\left(z_1\right)\bar{P}_2\left(z_2\right)\hat{P}_1\left(w_1\right)\hat{P}_2\left(w_2\right)\right)F_1\left(\theta_1\left(\tilde{P}_2\left(z_1\right)\hat{P}_1\left(w_1\right)\hat{P}_2\left(w_2\right)\right)\right)\hat{F}_1\left(w_1,w_2\right)\right)\\
&=&r_{1}\mu_{1}\hat{\mu}_{1}+\mu_{1}\hat{\mu}_{1}R_{1}^{(2)}(1)+r_{1}\frac{\mu_{1}\hat{\mu}_{1}}{1-\mu_{1}}F_{1}^{(1,0)}+r_{1}\mu_{1}\hat{F}_{1}^{(1,0)}.
\end{eqnarray*}
%10/26
\item \begin{eqnarray*} &&\frac{\partial}{\partial
z_2}\frac{\partial}{\partial
w_1}\left(R_1\left(P_1\left(z_1\right)\bar{P}_2\left(z_2\right)\hat{P}_1\left(w_1\right)\hat{P}_2\left(w_2\right)\right)F_1\left(\theta_1\left(\tilde{P}_2\left(z_1\right)\hat{P}_1\left(w_1\right)\hat{P}_2\left(w_2\right)\right)\right)\hat{F}_1\left(w_1,w_2\right)\right)\\
&=&r_{1}\hat{\mu}_{1}\tilde{\mu}_{2}+\tilde{\mu}_{2}\hat{\mu}_{1}R_{1}^{(2)}\left(1\right)+
\frac{\hat{\mu}_{1}\tilde{\mu}_{2}}{1-\mu_{1}}F_{1}^{1,0)}+r_{1}\frac{\hat{\mu}_{1}\tilde{\mu}_{2}}{1-\mu_{1}}F_{1}^{(1,0)}+\hat{\mu}_{1}\tilde{\mu}_{2}\theta_{1}^{(2)}\left(1\right)F_{2}^{(1,0)}\\
&+&r_{1}\hat{\mu}_{1}\left(F_{1}^{(1,0)}+\frac{\tilde{\mu}_{2}}{1-\mu_{1}}F_{1}^{1,0)}\right)+
r_{1}\tilde{\mu}_{2}\hat{F}_{1}^{(1,0)}+\left(F_{1}^{(0,1)}+\frac{\tilde{\mu}_{2}}{1-\mu_{1}}F_{1}^{1,0)}\right)\hat{F}_{1}^{(1,0)}\\
&+&\frac{\hat{\mu}_{1}}{1-\mu_{1}}\left(F_{1}^{(1,1)}+\frac{\tilde{\mu}_{2}}{1-\mu_{1}}F_{1}^{2,0)}\right).
\end{eqnarray*}
%11/27
\item \begin{eqnarray*} &&\frac{\partial}{\partial
w_1}\frac{\partial}{\partial
w_1}\left(R_1\left(P_1\left(z_1\right)\bar{P}_2\left(z_2\right)\hat{P}_1\left(w_1\right)\hat{P}_2\left(w_2\right)\right)F_1\left(\theta_1\left(\tilde{P}_2\left(z_1\right)\hat{P}_1\left(w_1\right)\hat{P}_2\left(w_2\right)\right)\right)\hat{F}_1\left(w_1,w_2\right)\right)\\
&=&\hat{\mu}_{1}^{2}R_{1}^{(2)}\left(1\right)+r_{1}\hat{P}_{1}^{(2)}\left(1\right)+2r_{1}\frac{\hat{\mu}_{1}^{2}}{1-\mu_{1}}F_{1}^{(1,0)}+\hat{\mu}_{1}^{2}\theta_{1}^{(2)}\left(1\right)F_{1}^{(1,0)}+\frac{1}{1-\mu_{1}}\hat{P}_{1}^{(2)}\left(1\right)F_{1}^{(1,0)}\\
&+&2r_{1}\hat{\mu}_{1}\hat{F}_{1}^{(1,0)}+2\frac{\hat{\mu}_{1}}{1-\mu_{1}}F_{1}^{(1,0)}\hat{F}_{1}^{(1,0)}+\left(\frac{\hat{\mu}_{1}}{1-\mu_{1}}\right)^{2}F_{1}^{(2,0)}+\hat{F}_{1}^{(2,0)}.
\end{eqnarray*}
%12/28
\item \begin{eqnarray*} &&\frac{\partial}{\partial
w_2}\frac{\partial}{\partial
w_1}\left(R_1\left(P_1\left(z_1\right)\bar{P}_2\left(z_2\right)\hat{P}_1\left(w_1\right)\hat{P}_2\left(w_2\right)\right)F_1\left(\theta_1\left(\tilde{P}_2\left(z_1\right)\hat{P}_1\left(w_1\right)\hat{P}_2\left(w_2\right)\right)\right)\hat{F}_1\left(w_1,w_2\right)\right)\\
&=&r_{1}\hat{\mu}_{1}\hat{\mu}_{2}+\hat{\mu}_{1}\hat{\mu}_{2}R_{1}^{(2)}\left(1\right)+r_{1}\hat{\mu}_{1}\hat{F}_{1}^{(0,1)}+
\frac{\hat{\mu}_{1}\hat{\mu}_{2}}{1-\mu_{1}}F_{1}^{(1,0)}+2r_{1}\frac{\hat{\mu}_{1}\hat{\mu}_{2}}{1-\mu_{1}}F_{1}^{1,0)}+\hat{\mu}_{1}\hat{\mu}_{2}\theta_{1}^{(2)}\left(1\right)F_{1}^{(1,0)}\\
&+&\frac{\hat{\mu}_{1}}{1-\mu_{1}}F_{1}^{(1,0)}\hat{F}_{1}^{(0,1)}+
r_{1}\hat{\mu}_{2}\hat{F}_{1}^{(1,0)}+\frac{\hat{\mu}_{2}}{1-\mu_{1}}\hat{F}_{1}^{(1,0)}F_{1}^{(1,0)}+\hat{F}_{1}^{(1,1)}+\hat{\mu}_{1}\hat{\mu}_{2}\left(\frac{1}{1-\mu_{1}}\right)^{2}F_{1}^{(2,0)}.
\end{eqnarray*}
%___________________________________________________________________________________________
%\subsubsection{Mixtas para $w_{2}$:}
%___________________________________________________________________________________________
%13/29
\item \begin{eqnarray*} &&\frac{\partial}{\partial
z_1}\frac{\partial}{\partial
w_2}\left(R_1\left(P_1\left(z_1\right)\bar{P}_2\left(z_2\right)\hat{P}_1\left(w_1\right)\hat{P}_2\left(w_2\right)\right)F_1\left(\theta_1\left(\tilde{P}_2\left(z_1\right)\hat{P}_1\left(w_1\right)\hat{P}_2\left(w_2\right)\right)\right)\hat{F}_1\left(w_1,w_2\right)\right)\\
&=&r_{1}\mu_{1}\hat{\mu}_{2}+\mu_{1}\hat{\mu}_{2}R_{1}^{(2)}\left(1\right)+r_{1}\mu_{1}\hat{F}_{1}^{(0,1)}+r_{1}\frac{\mu_{1}\hat{\mu}_{2}}{1-\mu_{1}}F_{1}^{(1,0)}.
\end{eqnarray*}
%14/30
\item \begin{eqnarray*} &&\frac{\partial}{\partial
z_2}\frac{\partial}{\partial
w_2}\left(R_1\left(P_1\left(z_1\right)\bar{P}_2\left(z_2\right)\hat{P}_1\left(w_1\right)\hat{P}_2\left(w_2\right)\right)F_1\left(\theta_1\left(\tilde{P}_2\left(z_1\right)\hat{P}_1\left(w_1\right)\hat{P}_2\left(w_2\right)\right)\right)\hat{F}_1\left(w_1,w_2\right)\right)\\
&=&r_{1}\hat{\mu}_{2}\tilde{\mu}_{2}+\hat{\mu}_{2}\tilde{\mu}_{2}R_{1}^{(2)}\left(1\right)+r_{1}\tilde{\mu}_{2}\hat{F}_{1}^{(0,1)}+\frac{\hat{\mu}_{2}\tilde{\mu}_{2}}{1-\mu_{1}}F_{1}^{(1,0)}+r_{1}\frac{\hat{\mu}_{2}\tilde{\mu}_{2}}{1-\mu_{1}}F_{1}^{(1,0)}\\
&+&\hat{\mu}_{2}\tilde{\mu}_{2}\theta_{1}^{(2)}\left(1\right)F_{1}^{(1,0)}+r_{1}\hat{\mu}_{2}\left(F_{1}^{(0,1)}+\frac{\tilde{\mu}_{2}}{1-\mu_{1}}F_{1}^{(1,0)}\right)+\left(F_{1}^{(0,1)}+\frac{\tilde{\mu}_{2}}{1-\mu_{1}}F_{1}^{(1,0)}\right)\hat{F}_{1}^{(0,1)}\\&+&\frac{\hat{\mu}_{2}}{1-\mu_{1}}\left(F_{1}^{(1,1)}+\frac{\tilde{\mu}_{2}}{1-\mu_{1}}F_{1}^{(2,0)}\right).
\end{eqnarray*}
%15/31
\item \begin{eqnarray*} &&\frac{\partial}{\partial
w_1}\frac{\partial}{\partial
w_2}\left(R_1\left(P_1\left(z_1\right)\bar{P}_2\left(z_2\right)\hat{P}_1\left(w_1\right)\hat{P}_2\left(w_2\right)\right)F_1\left(\theta_1\left(\tilde{P}_2\left(z_1\right)\hat{P}_1\left(w_1\right)\hat{P}_2\left(w_2\right)\right)\right)\hat{F}_1\left(w_1,w_2\right)\right)\\
&=&r_{1}\hat{\mu}_{1}\hat{\mu}_{2}+\hat{\mu}_{1}\hat{\mu}_{2}R_{1}^{(2)}\left(1\right)+r_{1}\hat{\mu}_{1}\hat{F}_{1}^{(0,1)}+
\frac{\hat{\mu}_{1}\hat{\mu}_{2}}{1-\mu_{1}}F_{1}^{(1,0)}+2r_{1}\frac{\hat{\mu}_{1}\hat{\mu}_{2}}{1-\mu_{1}}F_{1}^{(1,0)}+\hat{\mu}_{1}\hat{\mu}_{2}\theta_{1}^{(2)}\left(1\right)F_{1}^{(1,0)}\\
&+&\frac{\hat{\mu}_{1}}{1-\mu_{1}}\hat{F}_{1}^{(0,1)}F_{1}^{(1,0)}+r_{1}\hat{\mu}_{2}\hat{F}_{1}^{(1,0)}+\frac{\hat{\mu}_{2}}{1-\mu_{1}}\hat{F}_{1}^{(1,0)}F_{1}^{(1,0)}+\hat{F}_{1}^{(1,1)}+\hat{\mu}_{1}\hat{\mu}_{2}\left(\frac{1}{1-\mu_{1}}\right)^{2}F_{1}^{(2,0)}.
\end{eqnarray*}
%16/32
\item \begin{eqnarray*} &&\frac{\partial}{\partial
w_2}\frac{\partial}{\partial
w_2}\left(R_1\left(P_1\left(z_1\right)\bar{P}_2\left(z_2\right)\hat{P}_1\left(w_1\right)\hat{P}_2\left(w_2\right)\right)F_1\left(\theta_1\left(\tilde{P}_2\left(z_1\right)\hat{P}_1\left(w_1\right)\hat{P}_2\left(w_2\right)\right)\right)\hat{F}_1\left(w_1,w_2\right)\right)\\
&=&\hat{\mu}_{2}R_{1}^{(2)}\left(1\right)+r_{1}\hat{P}_{2}^{(2)}\left(1\right)+2r_{1}\hat{\mu}_{2}\hat{F}_{1}^{(0,1)}+\hat{F}_{1}^{(0,2)}+2r_{1}\frac{\hat{\mu}_{2}^{2}}{1-\mu_{1}}F_{1}^{(1,0)}+\hat{\mu}_{2}^{2}\theta_{1}^{(2)}\left(1\right)F_{1}^{(1,0)}\\
&+&\frac{1}{1-\mu_{1}}\hat{P}_{2}^{(2)}\left(1\right)F_{1}^{(1,0)} +
2\frac{\hat{\mu}_{2}}{1-\mu_{1}}F_{1}^{(1,0)}\hat{F}_{1}^{(0,1)}+\left(\frac{\hat{\mu}_{2}}{1-\mu_{1}}\right)^{2}F_{1}^{(2,0)}.
\end{eqnarray*}
\end{enumerate}

%___________________________________________________________________________________________
%
%\subsection{Derivadas de Segundo Orden para $\hat{F}_{1}$}
%___________________________________________________________________________________________


\begin{enumerate}
%___________________________________________________________________________________________
%\subsubsection{Mixtas para $z_{1}$:}
%___________________________________________________________________________________________
%1/33

\item \begin{eqnarray*} &&\frac{\partial}{\partial
z_1}\frac{\partial}{\partial
z_1}\left(\hat{R}_{2}\left(P_{1}\left(z_{1}\right)\tilde{P}_{2}\left(z_{2}\right)\hat{P}_{1}\left(w_{1}\right)\hat{P}_{2}\left(w_{2}\right)\right)\hat{F}_{2}\left(w_{1},\hat{\theta}_{2}\left(P_{1}\left(z_{1}\right)\tilde{P}_{2}\left(z_{2}\right)\hat{P}_{1}\left(w_{1}\right)\right)\right)F_{2}\left(z_{1},z_{2}\right)\right)\\
&=&\hat{r}_{2}P_{1}^{(2)}\left(1\right)+
\mu_{1}^{2}\hat{R}_{2}^{(2)}\left(1\right)+
2\hat{r}_{2}\frac{\mu_{1}^{2}}{1-\hat{\mu}_{2}}\hat{F}_{2}^{(0,1)}+
\frac{1}{1-\hat{\mu}_{2}}P_{1}^{(2)}\left(1\right)\hat{F}_{2}^{(0,1)}+
\mu_{1}^{2}\hat{\theta}_{2}^{(2)}\left(1\right)\hat{F}_{2}^{(0,1)}\\
&+&\left(\frac{\mu_{1}^{2}}{1-\hat{\mu}_{2}}\right)^{2}\hat{F}_{2}^{(0,2)}+
2\hat{r}_{2}\mu_{1}F_{2}^{(1,0)}+2\frac{\mu_{1}}{1-\hat{\mu}_{2}}\hat{F}_{2}^{(0,1)}F_{2}^{(1,0)}+F_{2}^{(2,0)}.
\end{eqnarray*}

%2/34
\item \begin{eqnarray*} &&\frac{\partial}{\partial
z_2}\frac{\partial}{\partial
z_1}\left(\hat{R}_{2}\left(P_{1}\left(z_{1}\right)\tilde{P}_{2}\left(z_{2}\right)\hat{P}_{1}\left(w_{1}\right)\hat{P}_{2}\left(w_{2}\right)\right)\hat{F}_{2}\left(w_{1},\hat{\theta}_{2}\left(P_{1}\left(z_{1}\right)\tilde{P}_{2}\left(z_{2}\right)\hat{P}_{1}\left(w_{1}\right)\right)\right)F_{2}\left(z_{1},z_{2}\right)\right)\\
&=&\hat{r}_{2}\mu_{1}\tilde{\mu}_{2}+\mu_{1}\tilde{\mu}_{2}\hat{R}_{2}^{(2)}\left(1\right)+\hat{r}_{2}\mu_{1}F_{2}^{(0,1)}+
\frac{\mu_{1}\tilde{\mu}_{2}}{1-\hat{\mu}_{2}}\hat{F}_{2}^{(0,1)}+2\hat{r}_{2}\frac{\mu_{1}\tilde{\mu}_{2}}{1-\hat{\mu}_{2}}\hat{F}_{2}^{(0,1)}+\mu_{1}\tilde{\mu}_{2}\hat{\theta}_{2}^{(2)}\left(1\right)\hat{F}_{2}^{(0,1)}\\
&+&\frac{\mu_{1}}{1-\hat{\mu}_{2}}F_{2}^{(0,1)}\hat{F}_{2}^{(0,1)}+\mu_{1} \tilde{\mu}_{2}\left(\frac{1}{1-\hat{\mu}_{2}}\right)^{2}\hat{F}_{2}^{(0,2)}+\hat{r}_{2}\tilde{\mu}_{2}F_{2}^{(1,0)}+\frac{\tilde{\mu}_{2}}{1-\hat{\mu}_{2}}\hat{F}_{2}^{(0,1)}F_{2}^{(1,0)}+F_{2}^{(1,1)}.
\end{eqnarray*}


%3/35

\item \begin{eqnarray*} &&\frac{\partial}{\partial
w_1}\frac{\partial}{\partial
z_1}\left(\hat{R}_{2}\left(P_{1}\left(z_{1}\right)\tilde{P}_{2}\left(z_{2}\right)\hat{P}_{1}\left(w_{1}\right)\hat{P}_{2}\left(w_{2}\right)\right)\hat{F}_{2}\left(w_{1},\hat{\theta}_{2}\left(P_{1}\left(z_{1}\right)\tilde{P}_{2}\left(z_{2}\right)\hat{P}_{1}\left(w_{1}\right)\right)\right)F_{2}\left(z_{1},z_{2}\right)\right)\\
&=&\hat{r}_{2}\mu_{1}\hat{\mu}_{1}+\mu_{1}\hat{\mu}_{1}\hat{R}_{2}^{(2)}\left(1\right)+\hat{r}_{2}\frac{\mu_{1}\hat{\mu}_{1}}{1-\hat{\mu}_{2}}\hat{F}_{2}^{(0,1)}+\hat{r}_{2}\hat{\mu}_{1}F_{2}^{(1,0)}+\hat{r}_{2}\mu_{1}\hat{F}_{2}^{(1,0)}+F_{2}^{(1,0)}\hat{F}_{2}^{(1,0)}+\frac{\mu_{1}}{1-\hat{\mu}_{2}}\hat{F}_{2}^{(1,1)}.
\end{eqnarray*}

%4/36

\item \begin{eqnarray*} &&\frac{\partial}{\partial
w_2}\frac{\partial}{\partial
z_1}\left(\hat{R}_{2}\left(P_{1}\left(z_{1}\right)\tilde{P}_{2}\left(z_{2}\right)\hat{P}_{1}\left(w_{1}\right)\hat{P}_{2}\left(w_{2}\right)\right)\hat{F}_{2}\left(w_{1},\hat{\theta}_{2}\left(P_{1}\left(z_{1}\right)\tilde{P}_{2}\left(z_{2}\right)\hat{P}_{1}\left(w_{1}\right)\right)\right)F_{2}\left(z_{1},z_{2}\right)\right)\\
&=&\hat{r}_{2}\mu_{1}\hat{\mu}_{2}+\mu_{1}\hat{\mu}_{2}\hat{R}_{2}^{(2)}\left(1\right)+\frac{\mu_{1}\hat{\mu}_{2}}{1-\hat{\mu}_{2}}\hat{F}_{2}^{(0,1)}+2\hat{r}_{2}\frac{\mu_{1}\hat{\mu}_{2}}{1-\hat{\mu}_{2}}\hat{F}_{2}^{(0,1)}+\mu_{1}\hat{\mu}_{2}\hat{\theta}_{2}^{(2)}\left(1\right)\hat{F}_{2}^{(0,1)}\\
&+&\mu_{1}\hat{\mu}_{2}\left(\frac{1}{1-\hat{\mu}_{2}}\right)^{2}\hat{F}_{2}^{(0,2)}+\hat{r}_{2}\hat{\mu}_{2}F_{2}^{(1,0)}+\frac{\hat{\mu}_{2}}{1-\hat{\mu}_{2}}\hat{F}_{2}^{(0,1)}F_{2}^{(1,0)}.
\end{eqnarray*}
%___________________________________________________________________________________________
%\subsubsection{Mixtas para $z_{2}$:}
%___________________________________________________________________________________________

%5/37

\item \begin{eqnarray*} &&\frac{\partial}{\partial
z_1}\frac{\partial}{\partial
z_2}\left(\hat{R}_{2}\left(P_{1}\left(z_{1}\right)\tilde{P}_{2}\left(z_{2}\right)\hat{P}_{1}\left(w_{1}\right)\hat{P}_{2}\left(w_{2}\right)\right)\hat{F}_{2}\left(w_{1},\hat{\theta}_{2}\left(P_{1}\left(z_{1}\right)\tilde{P}_{2}\left(z_{2}\right)\hat{P}_{1}\left(w_{1}\right)\right)\right)F_{2}\left(z_{1},z_{2}\right)\right)\\
&=&\hat{r}_{2}\mu_{1}\tilde{\mu}_{2}+\mu_{1}\tilde{\mu}_{2}\hat{R}_{2}^{(2)}\left(1\right)+\mu_{1}\hat{r}_{2}F_{2}^{(0,1)}+
\frac{\mu_{1}\tilde{\mu}_{2}}{1-\hat{\mu}_{2}}\hat{F}_{2}^{(0,1)}+2\hat{r}_{2}\frac{\mu_{1}\tilde{\mu}_{2}}{1-\hat{\mu}_{2}}\hat{F}_{2}^{(0,1)}+\mu_{1}\tilde{\mu}_{2}\hat{\theta}_{2}^{(2)}\left(1\right)\hat{F}_{2}^{(0,1)}\\
&+&\frac{\mu_{1}}{1-\hat{\mu}_{2}}F_{2}^{(0,1)}\hat{F}_{2}^{(0,1)}+\mu_{1}\tilde{\mu}_{2}\left(\frac{1}{1-\hat{\mu}_{2}}\right)^{2}\hat{F}_{2}^{(0,2)}+\hat{r}_{2}\tilde{\mu}_{2}F_{2}^{(1,0)}+\frac{\tilde{\mu}_{2}}{1-\hat{\mu}_{2}}\hat{F}_{2}^{(0,1)}F_{2}^{(1,0)}+F_{2}^{(1,1)}.
\end{eqnarray*}

%6/38

\item \begin{eqnarray*} &&\frac{\partial}{\partial
z_2}\frac{\partial}{\partial
z_2}\left(\hat{R}_{2}\left(P_{1}\left(z_{1}\right)\tilde{P}_{2}\left(z_{2}\right)\hat{P}_{1}\left(w_{1}\right)\hat{P}_{2}\left(w_{2}\right)\right)\hat{F}_{2}\left(w_{1},\hat{\theta}_{2}\left(P_{1}\left(z_{1}\right)\tilde{P}_{2}\left(z_{2}\right)\hat{P}_{1}\left(w_{1}\right)\right)\right)F_{2}\left(z_{1},z_{2}\right)\right)\\
&=&\hat{r}_{2}\tilde{P}_{2}^{(2)}\left(1\right)+\tilde{\mu}_{2}^{2}\hat{R}_{2}^{(2)}\left(1\right)+2\hat{r}_{2}\tilde{\mu}_{2}F_{2}^{(0,1)}+2\hat{r}_{2}\frac{\tilde{\mu}_{2}^{2}}{1-\hat{\mu}_{2}}\hat{F}_{2}^{(0,1)}+\frac{1}{1-\hat{\mu}_{2}}\tilde{P}_{2}^{(2)}\left(1\right)\hat{F}_{2}^{(0,1)}\\
&+&\tilde{\mu}_{2}^{2}\hat{\theta}_{2}^{(2)}\left(1\right)\hat{F}_{2}^{(0,1)}+2\frac{\tilde{\mu}_{2}}{1-\hat{\mu}_{2}}F_{2}^{(0,1)}\hat{F}_{2}^{(0,1)}+F_{2}^{(0,2)}+\left(\frac{\tilde{\mu}_{2}}{1-\hat{\mu}_{2}}\right)^{2}\hat{F}_{2}^{(0,2)}.
\end{eqnarray*}

%7/39

\item \begin{eqnarray*} &&\frac{\partial}{\partial
w_1}\frac{\partial}{\partial
z_2}\left(\hat{R}_{2}\left(P_{1}\left(z_{1}\right)\tilde{P}_{2}\left(z_{2}\right)\hat{P}_{1}\left(w_{1}\right)\hat{P}_{2}\left(w_{2}\right)\right)\hat{F}_{2}\left(w_{1},\hat{\theta}_{2}\left(P_{1}\left(z_{1}\right)\tilde{P}_{2}\left(z_{2}\right)\hat{P}_{1}\left(w_{1}\right)\right)\right)F_{2}\left(z_{1},z_{2}\right)\right)\\
&=&\hat{r}_{2}\tilde{\mu}_{2}\hat{\mu}_{1}+\tilde{\mu}_{2}\hat{\mu}_{1}\hat{R}_{2}^{(2)}\left(1\right)+\hat{r}_{2}\hat{\mu}_{1}F_{2}^{(0,1)}+\hat{r}_{2}\frac{\tilde{\mu}_{2}\hat{\mu}_{1}}{1-\hat{\mu}_{2}}\hat{F}_{2}^{(0,1)}+\hat{r}_{2}\tilde{\mu}_{2}\hat{F}_{2}^{(1,0)}+F_{2}^{(0,1)}\hat{F}_{2}^{(1,0)}+\frac{\tilde{\mu}_{2}}{1-\hat{\mu}_{2}}\hat{F}_{2}^{(1,1)}.
\end{eqnarray*}
%8/40

\item \begin{eqnarray*} &&\frac{\partial}{\partial
w_2}\frac{\partial}{\partial
z_2}\left(\hat{R}_{2}\left(P_{1}\left(z_{1}\right)\tilde{P}_{2}\left(z_{2}\right)\hat{P}_{1}\left(w_{1}\right)\hat{P}_{2}\left(w_{2}\right)\right)\hat{F}_{2}\left(w_{1},\hat{\theta}_{2}\left(P_{1}\left(z_{1}\right)\tilde{P}_{2}\left(z_{2}\right)\hat{P}_{1}\left(w_{1}\right)\right)\right)F_{2}\left(z_{1},z_{2}\right)\right)\\
&=&\hat{r}_{2}\tilde{\mu}_{2}\hat{\mu}_{2}+\tilde{\mu}_{2}\hat{\mu}_{2}\hat{R}_{2}^{(2)}\left(1\right)+\hat{r}_{2}\hat{\mu}_{2}F_{2}^{(0,1)}+
\frac{\tilde{\mu}_{2}\hat{\mu}_{2}}{1-\hat{\mu}_{2}}\hat{F}_{2}^{(0,1)}+2\hat{r}_{2}\frac{\tilde{\mu}_{2}\hat{\mu}_{2}}{1-\hat{\mu}_{2}}\hat{F}_{2}^{(0,1)}+\tilde{\mu}_{2}\hat{\mu}_{2}\hat{\theta}_{2}^{(2)}\left(1\right)\hat{F}_{2}^{(0,1)}\\
&+&\frac{\hat{\mu}_{2}}{1-\hat{\mu}_{2}}F_{2}^{(0,1)}\hat{F}_{2}^{(1,0)}+\tilde{\mu}_{2}\hat{\mu}_{2}\left(\frac{1}{1-\hat{\mu}_{2}}\right)\hat{F}_{2}^{(0,2)}.
\end{eqnarray*}
%___________________________________________________________________________________________
%\subsubsection{Mixtas para $w_{1}$:}
%___________________________________________________________________________________________

%9/41
\item \begin{eqnarray*} &&\frac{\partial}{\partial
z_1}\frac{\partial}{\partial
w_1}\left(\hat{R}_{2}\left(P_{1}\left(z_{1}\right)\tilde{P}_{2}\left(z_{2}\right)\hat{P}_{1}\left(w_{1}\right)\hat{P}_{2}\left(w_{2}\right)\right)\hat{F}_{2}\left(w_{1},\hat{\theta}_{2}\left(P_{1}\left(z_{1}\right)\tilde{P}_{2}\left(z_{2}\right)\hat{P}_{1}\left(w_{1}\right)\right)\right)F_{2}\left(z_{1},z_{2}\right)\right)\\
&=&\hat{r}_{2}\mu_{1}\hat{\mu}_{1}+\mu_{1}\hat{\mu}_{1}\hat{R}_{2}^{(2)}\left(1\right)+\hat{r}_{2}\frac{\mu_{1}\hat{\mu}_{1}}{1-\hat{\mu}_{2}}\hat{F}_{2}^{(0,1)}+\hat{r}_{2}\hat{\mu}_{1}F_{2}^{(1,0)}+\hat{r}_{2}\mu_{1}\hat{F}_{2}^{(1,0)}+F_{2}^{(1,0)}\hat{F}_{2}^{(1,0)}+\frac{\mu_{1}}{1-\hat{\mu}_{2}}\hat{F}_{2}^{(1,1)}.
\end{eqnarray*}


%10/42
\item \begin{eqnarray*} &&\frac{\partial}{\partial
z_2}\frac{\partial}{\partial
w_1}\left(\hat{R}_{2}\left(P_{1}\left(z_{1}\right)\tilde{P}_{2}\left(z_{2}\right)\hat{P}_{1}\left(w_{1}\right)\hat{P}_{2}\left(w_{2}\right)\right)\hat{F}_{2}\left(w_{1},\hat{\theta}_{2}\left(P_{1}\left(z_{1}\right)\tilde{P}_{2}\left(z_{2}\right)\hat{P}_{1}\left(w_{1}\right)\right)\right)F_{2}\left(z_{1},z_{2}\right)\right)\\
&=&\hat{r}_{2}\tilde{\mu}_{2}\hat{\mu}_{1}+\tilde{\mu}_{2}\hat{\mu}_{1}\hat{R}_{2}^{(2)}\left(1\right)+\hat{r}_{2}\hat{\mu}_{1}F_{2}^{(0,1)}+
\hat{r}_{2}\frac{\tilde{\mu}_{2}\hat{\mu}_{1}}{1-\hat{\mu}_{2}}\hat{F}_{2}^{(0,1)}+\hat{r}_{2}\tilde{\mu}_{2}\hat{F}_{2}^{(1,0)}+F_{2}^{(0,1)}\hat{F}_{2}^{(1,0)}+\frac{\tilde{\mu}_{2}}{1-\hat{\mu}_{2}}\hat{F}_{2}^{(1,1)}.
\end{eqnarray*}


%11/43
\item \begin{eqnarray*} &&\frac{\partial}{\partial
w_1}\frac{\partial}{\partial
w_1}\left(\hat{R}_{2}\left(P_{1}\left(z_{1}\right)\tilde{P}_{2}\left(z_{2}\right)\hat{P}_{1}\left(w_{1}\right)\hat{P}_{2}\left(w_{2}\right)\right)\hat{F}_{2}\left(w_{1},\hat{\theta}_{2}\left(P_{1}\left(z_{1}\right)\tilde{P}_{2}\left(z_{2}\right)\hat{P}_{1}\left(w_{1}\right)\right)\right)F_{2}\left(z_{1},z_{2}\right)\right)\\
&=&\hat{r}_{2}\hat{P}_{1}^{(2)}\left(1\right)+\hat{\mu}_{1}^{2}\hat{R}_{2}^{(2)}\left(1\right)+2\hat{r}_{2}\hat{\mu}_{1}\hat{F}_{2}^{(1,0)}
+\hat{F}_{2}^{(2,0)}.
\end{eqnarray*}


%12/44
\item \begin{eqnarray*} &&\frac{\partial}{\partial
w_2}\frac{\partial}{\partial
w_1}\left(\hat{R}_{2}\left(P_{1}\left(z_{1}\right)\tilde{P}_{2}\left(z_{2}\right)\hat{P}_{1}\left(w_{1}\right)\hat{P}_{2}\left(w_{2}\right)\right)\hat{F}_{2}\left(w_{1},\hat{\theta}_{2}\left(P_{1}\left(z_{1}\right)\tilde{P}_{2}\left(z_{2}\right)\hat{P}_{1}\left(w_{1}\right)\right)\right)F_{2}\left(z_{1},z_{2}\right)\right)\\
&=&\hat{r}_{2}\hat{\mu}_{1}\hat{\mu}_{2}+\hat{\mu}_{1}\hat{\mu}_{2}\hat{R}_{2}^{(2)}\left(1\right)+
\hat{r}_{2}\frac{\hat{\mu}_{2}\hat{\mu}_{1}}{1-\hat{\mu}_{2}}\hat{F}_{2}^{(0,1)}
+\hat{r}_{2}\hat{\mu}_{2}\hat{F}_{2}^{(1,0)}+\frac{\hat{\mu}_{2}}{1-\hat{\mu}_{2}}\hat{F}_{2}^{(1,1)}.
\end{eqnarray*}
%___________________________________________________________________________________________
%\subsubsection{Mixtas para $w_{2}$:}
%___________________________________________________________________________________________
%13/45
\item \begin{eqnarray*} &&\frac{\partial}{\partial
z_1}\frac{\partial}{\partial
w_2}\left(\hat{R}_{2}\left(P_{1}\left(z_{1}\right)\tilde{P}_{2}\left(z_{2}\right)\hat{P}_{1}\left(w_{1}\right)\hat{P}_{2}\left(w_{2}\right)\right)\hat{F}_{2}\left(w_{1},\hat{\theta}_{2}\left(P_{1}\left(z_{1}\right)\tilde{P}_{2}\left(z_{2}\right)\hat{P}_{1}\left(w_{1}\right)\right)\right)F_{2}\left(z_{1},z_{2}\right)\right)\\
&=&\hat{r}_{2}\mu_{1}\hat{\mu}_{2}+\mu_{1}\hat{\mu}_{2}\hat{R}_{2}^{(2)}\left(1\right)+
\frac{\mu_{1}\hat{\mu}_{2}}{1-\hat{\mu}_{2}}\hat{F}_{2}^{(0,1)} +2\hat{r}_{2}\frac{\mu_{1}\hat{\mu}_{2}}{1-\hat{\mu}_{2}}\hat{F}_{2}^{(0,1)}\\
&+&\mu_{1}\hat{\mu}_{2}\hat{\theta}_{2}^{(2)}\left(1\right)\hat{F}_{2}^{(0,1)}+\mu_{1}\hat{\mu}_{2}\left(\frac{1}{1-\hat{\mu}_{2}}\right)^{2}\hat{F}_{2}^{(0,2)}+\hat{r}_{2}\hat{\mu}_{2}F_{2}^{(1,0)}+\frac{\hat{\mu}_{2}}{1-\hat{\mu}_{2}}\hat{F}_{2}^{(0,1)}F_{2}^{(1,0)}.\end{eqnarray*}


%14/46
\item \begin{eqnarray*} &&\frac{\partial}{\partial
z_2}\frac{\partial}{\partial
w_2}\left(\hat{R}_{2}\left(P_{1}\left(z_{1}\right)\tilde{P}_{2}\left(z_{2}\right)\hat{P}_{1}\left(w_{1}\right)\hat{P}_{2}\left(w_{2}\right)\right)\hat{F}_{2}\left(w_{1},\hat{\theta}_{2}\left(P_{1}\left(z_{1}\right)\tilde{P}_{2}\left(z_{2}\right)\hat{P}_{1}\left(w_{1}\right)\right)\right)F_{2}\left(z_{1},z_{2}\right)\right)\\
&=&\hat{r}_{2}\tilde{\mu}_{2}\hat{\mu}_{2}+\tilde{\mu}_{2}\hat{\mu}_{2}\hat{R}_{2}^{(2)}\left(1\right)+\hat{r}_{2}\hat{\mu}_{2}F_{2}^{(0,1)}+\frac{\tilde{\mu}_{2}\hat{\mu}_{2}}{1-\hat{\mu}_{2}}\hat{F}_{2}^{(0,1)}+
2\hat{r}_{2}\frac{\tilde{\mu}_{2}\hat{\mu}_{2}}{1-\hat{\mu}_{2}}\hat{F}_{2}^{(0,1)}+\tilde{\mu}_{2}\hat{\mu}_{2}\hat{\theta}_{2}^{(2)}\left(1\right)\hat{F}_{2}^{(0,1)}\\
&+&\frac{\hat{\mu}_{2}}{1-\hat{\mu}_{2}}\hat{F}_{2}^{(0,1)}F_{2}^{(0,1)}+\tilde{\mu}_{2}\hat{\mu}_{2}\left(\frac{1}{1-\hat{\mu}_{2}}\right)^{2}\hat{F}_{2}^{(0,2)}.
\end{eqnarray*}

%15/47

\item \begin{eqnarray*} &&\frac{\partial}{\partial
w_1}\frac{\partial}{\partial
w_2}\left(\hat{R}_{2}\left(P_{1}\left(z_{1}\right)\tilde{P}_{2}\left(z_{2}\right)\hat{P}_{1}\left(w_{1}\right)\hat{P}_{2}\left(w_{2}\right)\right)\hat{F}_{2}\left(w_{1},\hat{\theta}_{2}\left(P_{1}\left(z_{1}\right)\tilde{P}_{2}\left(z_{2}\right)\hat{P}_{1}\left(w_{1}\right)\right)\right)F_{2}\left(z_{1},z_{2}\right)\right)\\
&=&\hat{r}_{2}\hat{\mu}_{1}\hat{\mu}_{2}+\hat{\mu}_{1}\hat{\mu}_{2}\hat{R}_{2}^{(2)}\left(1\right)+
\hat{r}_{2}\frac{\hat{\mu}_{1}\hat{\mu}_{2}}{1-\hat{\mu}_{2}}\hat{F}_{2}^{(0,1)}+
\hat{r}_{2}\hat{\mu}_{2}\hat{F}_{2}^{(1,0)}+\frac{\hat{\mu}_{2}}{1-\hat{\mu}_{2}}\hat{F}_{2}^{(1,1)}.
\end{eqnarray*}

%16/48
\item \begin{eqnarray*} &&\frac{\partial}{\partial
w_2}\frac{\partial}{\partial
w_2}\left(\hat{R}_{2}\left(P_{1}\left(z_{1}\right)\tilde{P}_{2}\left(z_{2}\right)\hat{P}_{1}\left(w_{1}\right)\hat{P}_{2}\left(w_{2}\right)\right)\hat{F}_{2}\left(w_{1},\hat{\theta}_{2}\left(P_{1}\left(z_{1}\right)\tilde{P}_{2}\left(z_{2}\right)\hat{P}_{1}\left(w_{1}\right)\right)\right)F_{2}\left(z_{1},z_{2};\zeta_{2}\right)\right)\\
&=&\hat{r}_{2}P_{2}^{(2)}\left(1\right)+\hat{\mu}_{2}^{2}\hat{R}_{2}^{(2)}\left(1\right)+2\hat{r}_{2}\frac{\hat{\mu}_{2}^{2}}{1-\hat{\mu}_{2}}\hat{F}_{2}^{(0,1)}+\frac{1}{1-\hat{\mu}_{2}}\hat{P}_{2}^{(2)}\left(1\right)\hat{F}_{2}^{(0,1)}+\hat{\mu}_{2}^{2}\hat{\theta}_{2}^{(2)}\left(1\right)\hat{F}_{2}^{(0,1)}\\
&+&\left(\frac{\hat{\mu}_{2}}{1-\hat{\mu}_{2}}\right)^{2}\hat{F}_{2}^{(0,2)}.
\end{eqnarray*}


\end{enumerate}



%___________________________________________________________________________________________
%
%\subsection{Derivadas de Segundo Orden para $\hat{F}_{2}$}
%___________________________________________________________________________________________
\begin{enumerate}
%___________________________________________________________________________________________
%\subsubsection{Mixtas para $z_{1}$:}
%___________________________________________________________________________________________
%1/49

\item \begin{eqnarray*} &&\frac{\partial}{\partial
z_1}\frac{\partial}{\partial
z_1}\left(\hat{R}_{1}\left(P_{1}\left(z_{1}\right)\tilde{P}_{2}\left(z_{2}\right)\hat{P}_{1}\left(w_{1}\right)\hat{P}_{2}\left(w_{2}\right)\right)\hat{F}_{1}\left(\hat{\theta}_{1}\left(P_{1}\left(z_{1}\right)\tilde{P}_{2}\left(z_{2}\right)
\hat{P}_{2}\left(w_{2}\right)\right),w_{2}\right)F_{1}\left(z_{1},z_{2}\right)\right)\\
&=&\hat{r}_{1}P_{1}^{(2)}\left(1\right)+
\mu_{1}^{2}\hat{R}_{1}^{(2)}\left(1\right)+
2\hat{r}_{1}\mu_{1}F_{1}^{(1,0)}+
2\hat{r}_{1}\frac{\mu_{1}^{2}}{1-\hat{\mu}_{1}}\hat{F}_{1}^{(1,0)}+
\frac{1}{1-\hat{\mu}_{1}}P_{1}^{(2)}\left(1\right)\hat{F}_{1}^{(1,0)}+\mu_{1}^{2}\hat{\theta}_{1}^{(2)}\left(1\right)\hat{F}_{1}^{(1,0)}\\
&+&2\frac{\mu_{1}}{1-\hat{\mu}_{1}}\hat{F}_{1}^{(1,0)}F_{1}^{(1,0)}+F_{1}^{(2,0)}
+\left(\frac{\mu_{1}}{1-\hat{\mu}_{1}}\right)^{2}\hat{F}_{1}^{(2,0)}.
\end{eqnarray*}

%2/50

\item \begin{eqnarray*} &&\frac{\partial}{\partial
z_2}\frac{\partial}{\partial
z_1}\left(\hat{R}_{1}\left(P_{1}\left(z_{1}\right)\tilde{P}_{2}\left(z_{2}\right)\hat{P}_{1}\left(w_{1}\right)\hat{P}_{2}\left(w_{2}\right)\right)\hat{F}_{1}\left(\hat{\theta}_{1}\left(P_{1}\left(z_{1}\right)\tilde{P}_{2}\left(z_{2}\right)
\hat{P}_{2}\left(w_{2}\right)\right),w_{2}\right)F_{1}\left(z_{1},z_{2}\right)\right)\\
&=&\hat{r}_{1}\mu_{1}\tilde{\mu}_{2}+\mu_{1}\tilde{\mu}_{2}\hat{R}_{1}^{(2)}\left(1\right)+
\hat{r}_{1}\mu_{1}F_{1}^{(0,1)}+\tilde{\mu}_{2}\hat{r}_{1}F_{1}^{(1,0)}+
\frac{\mu_{1}\tilde{\mu}_{2}}{1-\hat{\mu}_{1}}\hat{F}_{1}^{(1,0)}+2\hat{r}_{1}\frac{\mu_{1}\tilde{\mu}_{2}}{1-\hat{\mu}_{1}}\hat{F}_{1}^{(1,0)}\\
&+&\mu_{1}\tilde{\mu}_{2}\hat{\theta}_{1}^{(2)}\left(1\right)\hat{F}_{1}^{(1,0)}+
\frac{\mu_{1}}{1-\hat{\mu}_{1}}\hat{F}_{1}^{(1,0)}F_{1}^{(0,1)}+
\frac{\tilde{\mu}_{2}}{1-\hat{\mu}_{1}}\hat{F}_{1}^{(1,0)}F_{1}^{(1,0)}+
F_{1}^{(1,1)}\\
&+&\mu_{1}\tilde{\mu}_{2}\left(\frac{1}{1-\hat{\mu}_{1}}\right)^{2}\hat{F}_{1}^{(2,0)}.
\end{eqnarray*}

%3/51

\item \begin{eqnarray*} &&\frac{\partial}{\partial
w_1}\frac{\partial}{\partial
z_1}\left(\hat{R}_{1}\left(P_{1}\left(z_{1}\right)\tilde{P}_{2}\left(z_{2}\right)\hat{P}_{1}\left(w_{1}\right)\hat{P}_{2}\left(w_{2}\right)\right)\hat{F}_{1}\left(\hat{\theta}_{1}\left(P_{1}\left(z_{1}\right)\tilde{P}_{2}\left(z_{2}\right)
\hat{P}_{2}\left(w_{2}\right)\right),w_{2}\right)F_{1}\left(z_{1},z_{2}\right)\right)\\
&=&\hat{r}_{1}\mu_{1}\hat{\mu}_{1}+\mu_{1}\hat{\mu}_{1}\hat{R}_{1}^{(2)}\left(1\right)+\hat{r}_{1}\hat{\mu}_{1}F_{1}^{(1,0)}+
\hat{r}_{1}\frac{\mu_{1}\hat{\mu}_{1}}{1-\hat{\mu}_{1}}\hat{F}_{1}^{(1,0)}.
\end{eqnarray*}

%4/52

\item \begin{eqnarray*} &&\frac{\partial}{\partial
w_2}\frac{\partial}{\partial
z_1}\left(\hat{R}_{1}\left(P_{1}\left(z_{1}\right)\tilde{P}_{2}\left(z_{2}\right)\hat{P}_{1}\left(w_{1}\right)\hat{P}_{2}\left(w_{2}\right)\right)\hat{F}_{1}\left(\hat{\theta}_{1}\left(P_{1}\left(z_{1}\right)\tilde{P}_{2}\left(z_{2}\right)
\hat{P}_{2}\left(w_{2}\right)\right),w_{2}\right)F_{1}\left(z_{1},z_{2}\right)\right)\\
&=&\hat{r}_{1}\mu_{1}\hat{\mu}_{2}+\mu_{1}\hat{\mu}_{2}\hat{R}_{1}^{(2)}\left(1\right)+\hat{r}_{1}\hat{\mu}_{2}F_{1}^{(1,0)}+\frac{\mu_{1}\hat{\mu}_{2}}{1-\hat{\mu}_{1}}\hat{F}_{1}^{(1,0)}+\hat{r}_{1}\frac{\mu_{1}\hat{\mu}_{2}}{1-\hat{\mu}_{1}}\hat{F}_{1}^{(1,0)}+\mu_{1}\hat{\mu}_{2}\hat{\theta}_{1}^{(2)}\left(1\right)\hat{F}_{1}^{(1,0)}\\
&+&\hat{r}_{1}\mu_{1}\left(\hat{F}_{1}^{(0,1)}+\frac{\hat{\mu}_{2}}{1-\hat{\mu}_{1}}\hat{F}_{1}^{(1,0)}\right)+F_{1}^{(1,0)}\left(\hat{F}_{1}^{(0,1)}+\frac{\hat{\mu}_{2}}{1-\hat{\mu}_{1}}\hat{F}_{1}^{(1,0)}\right)+\frac{\mu_{1}}{1-\hat{\mu}_{1}}\left(\hat{F}_{1}^{(1,1)}+\frac{\hat{\mu}_{2}}{1-\hat{\mu}_{1}}\hat{F}_{1}^{(2,0)}\right).
\end{eqnarray*}
%___________________________________________________________________________________________
%\subsubsection{Mixtas para $z_{2}$:}
%___________________________________________________________________________________________
%5/53

\item \begin{eqnarray*} &&\frac{\partial}{\partial
z_1}\frac{\partial}{\partial
z_2}\left(\hat{R}_{1}\left(P_{1}\left(z_{1}\right)\tilde{P}_{2}\left(z_{2}\right)\hat{P}_{1}\left(w_{1}\right)\hat{P}_{2}\left(w_{2}\right)\right)\hat{F}_{1}\left(\hat{\theta}_{1}\left(P_{1}\left(z_{1}\right)\tilde{P}_{2}\left(z_{2}\right)
\hat{P}_{2}\left(w_{2}\right)\right),w_{2}\right)F_{1}\left(z_{1},z_{2}\right)\right)\\
&=&\hat{r}_{1}\mu_{1}\tilde{\mu}_{2}+\mu_{1}\tilde{\mu}_{2}\hat{R}_{1}^{(2)}\left(1\right)+\hat{r}_{1}\mu_{1}F_{1}^{(0,1)}+\hat{r}_{1}\tilde{\mu}_{2}F_{1}^{(1,0)}+\frac{\mu_{1}\tilde{\mu}_{2}}{1-\hat{\mu}_{1}}\hat{F}_{1}^{(1,0)}+2\hat{r}_{1}\frac{\mu_{1}\tilde{\mu}_{2}}{1-\hat{\mu}_{1}}\hat{F}_{1}^{(1,0)}\\
&+&\mu_{1}\tilde{\mu}_{2}\hat{\theta}_{1}^{(2)}\left(1\right)\hat{F}_{1}^{(1,0)}+\frac{\mu_{1}}{1-\hat{\mu}_{1}}\hat{F}_{1}^{(1,0)}F_{1}^{(0,1)}+\frac{\tilde{\mu}_{2}}{1-\hat{\mu}_{1}}\hat{F}_{1}^{(1,0)}F_{1}^{(1,0)}+F_{1}^{(1,1)}+\mu_{1}\tilde{\mu}_{2}\left(\frac{1}{1-\hat{\mu}_{1}}\right)^{2}\hat{F}_{1}^{(2,0)}.
\end{eqnarray*}

%6/54
\item \begin{eqnarray*} &&\frac{\partial}{\partial
z_2}\frac{\partial}{\partial
z_2}\left(\hat{R}_{1}\left(P_{1}\left(z_{1}\right)\tilde{P}_{2}\left(z_{2}\right)\hat{P}_{1}\left(w_{1}\right)\hat{P}_{2}\left(w_{2}\right)\right)\hat{F}_{1}\left(\hat{\theta}_{1}\left(P_{1}\left(z_{1}\right)\tilde{P}_{2}\left(z_{2}\right)
\hat{P}_{2}\left(w_{2}\right)\right),w_{2}\right)F_{1}\left(z_{1},z_{2}\right)\right)\\
&=&\hat{r}_{1}\tilde{P}_{2}^{(2)}\left(1\right)+\tilde{\mu}_{2}^{2}\hat{R}_{1}^{(2)}\left(1\right)+2\hat{r}_{1}\tilde{\mu}_{2}F_{1}^{(0,1)}+ F_{1}^{(0,2)}+2\hat{r}_{1}\frac{\tilde{\mu}_{2}^{2}}{1-\hat{\mu}_{1}}\hat{F}_{1}^{(1,0)}+\frac{1}{1-\hat{\mu}_{1}}\tilde{P}_{2}^{(2)}\left(1\right)\hat{F}_{1}^{(1,0)}\\
&+&\tilde{\mu}_{2}^{2}\hat{\theta}_{1}^{(2)}\left(1\right)\hat{F}_{1}^{(1,0)}+2\frac{\tilde{\mu}_{2}}{1-\hat{\mu}_{1}}F^{(0,1)}\hat{F}_{1}^{(1,0)}+\left(\frac{\tilde{\mu}_{2}}{1-\hat{\mu}_{1}}\right)^{2}\hat{F}_{1}^{(2,0)}.
\end{eqnarray*}
%7/55

\item \begin{eqnarray*} &&\frac{\partial}{\partial
w_1}\frac{\partial}{\partial
z_2}\left(\hat{R}_{1}\left(P_{1}\left(z_{1}\right)\tilde{P}_{2}\left(z_{2}\right)\hat{P}_{1}\left(w_{1}\right)\hat{P}_{2}\left(w_{2}\right)\right)\hat{F}_{1}\left(\hat{\theta}_{1}\left(P_{1}\left(z_{1}\right)\tilde{P}_{2}\left(z_{2}\right)
\hat{P}_{2}\left(w_{2}\right)\right),w_{2}\right)F_{1}\left(z_{1},z_{2}\right)\right)\\
&=&\hat{r}_{1}\hat{\mu}_{1}\tilde{\mu}_{2}+\hat{\mu}_{1}\tilde{\mu}_{2}\hat{R}_{1}^{(2)}\left(1\right)+
\hat{r}_{1}\hat{\mu}_{1}F_{1}^{(0,1)}+\hat{r}_{1}\frac{\hat{\mu}_{1}\tilde{\mu}_{2}}{1-\hat{\mu}_{1}}\hat{F}_{1}^{(1,0)}.
\end{eqnarray*}
%8/56

\item \begin{eqnarray*} &&\frac{\partial}{\partial
w_2}\frac{\partial}{\partial
z_2}\left(\hat{R}_{1}\left(P_{1}\left(z_{1}\right)\tilde{P}_{2}\left(z_{2}\right)\hat{P}_{1}\left(w_{1}\right)\hat{P}_{2}\left(w_{2}\right)\right)\hat{F}_{1}\left(\hat{\theta}_{1}\left(P_{1}\left(z_{1}\right)\tilde{P}_{2}\left(z_{2}\right)
\hat{P}_{2}\left(w_{2}\right)\right),w_{2}\right)F_{1}\left(z_{1},z_{2}\right)\right)\\
&=&\hat{r}_{1}\tilde{\mu}_{2}\hat{\mu}_{2}+\hat{\mu}_{2}\tilde{\mu}_{2}\hat{R}_{1}^{(2)}\left(1\right)+\hat{\mu}_{2}\hat{R}_{1}^{(2)}\left(1\right)F_{1}^{(0,1)}+\frac{\hat{\mu}_{2}\tilde{\mu}_{2}}{1-\hat{\mu}_{1}}\hat{F}_{1}^{(1,0)}+
\hat{r}_{1}\frac{\hat{\mu}_{2}\tilde{\mu}_{2}}{1-\hat{\mu}_{1}}\hat{F}_{1}^{(1,0)}\\
&+&\hat{\mu}_{2}\tilde{\mu}_{2}\hat{\theta}_{1}^{(2)}\left(1\right)\hat{F}_{1}^{(1,0)}+\hat{r}_{1}\tilde{\mu}_{2}\left(\hat{F}_{1}^{(0,1)}+\frac{\hat{\mu}_{2}}{1-\hat{\mu}_{1}}\hat{F}_{1}^{(1,0)}\right)+F_{1}^{(0,1)}\left(\hat{F}_{1}^{(0,1)}+\frac{\hat{\mu}_{2}}{1-\hat{\mu}_{1}}\hat{F}_{1}^{(1,0)}\right)\\
&+&\frac{\tilde{\mu}_{2}}{1-\hat{\mu}_{1}}\left(\hat{F}_{1}^{(1,1)}+\frac{\hat{\mu}_{2}}{1-\hat{\mu}_{1}}\hat{F}_{1}^{(2,0)}\right).
\end{eqnarray*}
%___________________________________________________________________________________________
%\subsubsection{Mixtas para $w_{1}$:}
%___________________________________________________________________________________________
%9/57
\item \begin{eqnarray*} &&\frac{\partial}{\partial
z_1}\frac{\partial}{\partial
w_1}\left(\hat{R}_{1}\left(P_{1}\left(z_{1}\right)\tilde{P}_{2}\left(z_{2}\right)\hat{P}_{1}\left(w_{1}\right)\hat{P}_{2}\left(w_{2}\right)\right)\hat{F}_{1}\left(\hat{\theta}_{1}\left(P_{1}\left(z_{1}\right)\tilde{P}_{2}\left(z_{2}\right)
\hat{P}_{2}\left(w_{2}\right)\right),w_{2}\right)F_{1}\left(z_{1},z_{2}\right)\right)\\
&=&\hat{r}_{1}\mu_{1}\hat{\mu}_{1}+\mu_{1}\hat{\mu}_{1}\hat{R}_{1}^{(2)}\left(1\right)+\hat{r}_{1}\hat{\mu}_{1}F_{1}^{(1,0)}+\hat{r}_{1}\frac{\mu_{1}\hat{\mu}_{1}}{1-\hat{\mu}_{1}}\hat{F}_{1}^{(1,0)}.
\end{eqnarray*}
%10/58
\item \begin{eqnarray*} &&\frac{\partial}{\partial
z_2}\frac{\partial}{\partial
w_1}\left(\hat{R}_{1}\left(P_{1}\left(z_{1}\right)\tilde{P}_{2}\left(z_{2}\right)\hat{P}_{1}\left(w_{1}\right)\hat{P}_{2}\left(w_{2}\right)\right)\hat{F}_{1}\left(\hat{\theta}_{1}\left(P_{1}\left(z_{1}\right)\tilde{P}_{2}\left(z_{2}\right)
\hat{P}_{2}\left(w_{2}\right)\right),w_{2}\right)F_{1}\left(z_{1},z_{2}\right)\right)\\
&=&\hat{r}_{1}\tilde{\mu}_{2}\hat{\mu}_{1}+\tilde{\mu}_{2}\hat{\mu}_{1}\hat{R}_{1}^{(2)}\left(1\right)+\hat{r}_{1}\hat{\mu}_{1}F_{1}^{(0,1)}+\hat{r}_{1}\frac{\tilde{\mu}_{2}\hat{\mu}_{1}}{1-\hat{\mu}_{1}}\hat{F}_{1}^{(1,0)}.
\end{eqnarray*}
%11/59
\item \begin{eqnarray*} &&\frac{\partial}{\partial
w_1}\frac{\partial}{\partial
w_1}\left(\hat{R}_{1}\left(P_{1}\left(z_{1}\right)\tilde{P}_{2}\left(z_{2}\right)\hat{P}_{1}\left(w_{1}\right)\hat{P}_{2}\left(w_{2}\right)\right)\hat{F}_{1}\left(\hat{\theta}_{1}\left(P_{1}\left(z_{1}\right)\tilde{P}_{2}\left(z_{2}\right)
\hat{P}_{2}\left(w_{2}\right)\right),w_{2}\right)F_{1}\left(z_{1},z_{2}\right)\right)\\
&=&\hat{r}_{1}\hat{P}_{1}^{(2)}\left(1\right)+\hat{\mu}_{1}^{2}\hat{R}_{1}^{(2)}\left(1\right).
\end{eqnarray*}
%12/60
\item \begin{eqnarray*} &&\frac{\partial}{\partial
w_2}\frac{\partial}{\partial
w_1}\left(\hat{R}_{1}\left(P_{1}\left(z_{1}\right)\tilde{P}_{2}\left(z_{2}\right)\hat{P}_{1}\left(w_{1}\right)\hat{P}_{2}\left(w_{2}\right)\right)\hat{F}_{1}\left(\hat{\theta}_{1}\left(P_{1}\left(z_{1}\right)\tilde{P}_{2}\left(z_{2}\right)
\hat{P}_{2}\left(w_{2}\right)\right),w_{2}\right)F_{1}\left(z_{1},z_{2}\right)\right)\\
&=&\hat{r}_{1}\hat{\mu}_{2}\hat{\mu}_{1}+\hat{\mu}_{2}\hat{\mu}_{1}\hat{R}_{1}^{(2)}\left(1\right)+\hat{r}_{1}\hat{\mu}_{1}\left(\hat{F}_{1}^{(0,1)}+\frac{\hat{\mu}_{2}}{1-\hat{\mu}_{1}}\hat{F}_{1}^{(1,0)}\right).
\end{eqnarray*}
%___________________________________________________________________________________________
%\subsubsection{Mixtas para $w_{1}$:}
%___________________________________________________________________________________________
%13/61



\item \begin{eqnarray*} &&\frac{\partial}{\partial
z_1}\frac{\partial}{\partial
w_2}\left(\hat{R}_{1}\left(P_{1}\left(z_{1}\right)\tilde{P}_{2}\left(z_{2}\right)\hat{P}_{1}\left(w_{1}\right)\hat{P}_{2}\left(w_{2}\right)\right)\hat{F}_{1}\left(\hat{\theta}_{1}\left(P_{1}\left(z_{1}\right)\tilde{P}_{2}\left(z_{2}\right)
\hat{P}_{2}\left(w_{2}\right)\right),w_{2}\right)F_{1}\left(z_{1},z_{2}\right)\right)\\
&=&\hat{r}_{1}\mu_{1}\hat{\mu}_{2}+\mu_{1}\hat{\mu}_{2}\hat{R}_{1}^{(2)}\left(1\right)+\hat{r}_{1}\hat{\mu}_{2}F_{1}^{(1,0)}+
\hat{r}_{1}\frac{\mu_{1}\hat{\mu}_{2}}{1-\hat{\mu}_{1}}\hat{F}_{1}^{(1,0)}+\hat{r}_{1}\mu_{1}\left(\hat{F}_{1}^{(0,1)}+\frac{\hat{\mu}_{2}}{1-\hat{\mu}_{1}}\hat{F}_{1}^{(1,0)}\right)\\
&+&F_{1}^{(1,0)}\left(\hat{F}_{1}^{(0,1)}+\frac{\hat{\mu}_{2}}{1-\hat{\mu}_{1}}\hat{F}_{1}^{(1,0)}\right)+\frac{\mu_{1}\hat{\mu}_{2}}{1-\hat{\mu}_{1}}\hat{F}_{1}^{(1,0)}+\mu_{1}\hat{\mu}_{2}\hat{\theta}_{1}^{(2)}\left(1\right)\hat{F}_{1}^{(1,0)}+\frac{\mu_{1}}{1-\hat{\mu}_{1}}\hat{F}_{1}^{(1,1)}\\
&+&\mu_{1}\hat{\mu}_{2}\left(\frac{1}{1-\hat{\mu}_{1}}\right)^{2}\hat{F}_{1}^{(2,0)}.
\end{eqnarray*}

%14/62
\item \begin{eqnarray*} &&\frac{\partial}{\partial
z_2}\frac{\partial}{\partial
w_2}\left(\hat{R}_{1}\left(P_{1}\left(z_{1}\right)\tilde{P}_{2}\left(z_{2}\right)\hat{P}_{1}\left(w_{1}\right)\hat{P}_{2}\left(w_{2}\right)\right)\hat{F}_{1}\left(\hat{\theta}_{1}\left(P_{1}\left(z_{1}\right)\tilde{P}_{2}\left(z_{2}\right)
\hat{P}_{2}\left(w_{2}\right)\right),w_{2}\right)F_{1}\left(z_{1},z_{2}\right)\right)\\
&=&\hat{r}_{1}\tilde{\mu}_{2}\hat{\mu}_{2}+\tilde{\mu}_{2}\hat{\mu}_{2}\hat{R}_{1}^{(2)}\left(1\right)+\hat{r}_{1}\hat{\mu}_{2}F_{1}^{(0,1)}+\hat{r}_{1}\frac{\tilde{\mu}_{2}\hat{\mu}_{2}}{1-\hat{\mu}_{1}}\hat{F}_{1}^{(1,0)}+\hat{r}_{1}\tilde{\mu}_{2}\left(\hat{F}_{1}^{(0,1)}+\frac{\hat{\mu}_{2}}{1-\hat{\mu}_{1}}\hat{F}_{1}^{(1,0)}\right)\\
&+&F_{1}^{(0,1)}\left(\hat{F}_{1}^{(0,1)}+\frac{\hat{\mu}_{2}}{1-\hat{\mu}_{1}}\hat{F}_{1}^{(1,0)}\right)+\frac{\tilde{\mu}_{2}\hat{\mu}_{2}}{1-\hat{\mu}_{1}}\hat{F}_{1}^{(1,0)}+\tilde{\mu}_{2}\hat{\mu}_{2}\hat{\theta}_{1}^{(2)}\left(1\right)\hat{F}_{1}^{(1,0)}+\frac{\tilde{\mu}_{2}}{1-\hat{\mu}_{1}}\hat{F}_{1}^{(1,1)}\\
&+&\tilde{\mu}_{2}\hat{\mu}_{2}\left(\frac{1}{1-\hat{\mu}_{1}}\right)^{2}\hat{F}_{1}^{(2,0)}.
\end{eqnarray*}

%15/63

\item \begin{eqnarray*} &&\frac{\partial}{\partial
w_1}\frac{\partial}{\partial
w_2}\left(\hat{R}_{1}\left(P_{1}\left(z_{1}\right)\tilde{P}_{2}\left(z_{2}\right)\hat{P}_{1}\left(w_{1}\right)\hat{P}_{2}\left(w_{2}\right)\right)\hat{F}_{1}\left(\hat{\theta}_{1}\left(P_{1}\left(z_{1}\right)\tilde{P}_{2}\left(z_{2}\right)
\hat{P}_{2}\left(w_{2}\right)\right),w_{2}\right)F_{1}\left(z_{1},z_{2}\right)\right)\\
&=&\hat{r}_{1}\hat{\mu}_{2}\hat{\mu}_{1}+\hat{\mu}_{2}\hat{\mu}_{1}\hat{R}_{1}^{(2)}\left(1\right)+\hat{r}_{1}\hat{\mu}_{1}\left(\hat{F}_{1}^{(0,1)}+\frac{\hat{\mu}_{2}}{1-\hat{\mu}_{1}}\hat{F}_{1}^{(1,0)}\right).
\end{eqnarray*}

%16/64

\item \begin{eqnarray*} &&\frac{\partial}{\partial
w_2}\frac{\partial}{\partial
w_2}\left(\hat{R}_{1}\left(P_{1}\left(z_{1}\right)\tilde{P}_{2}\left(z_{2}\right)\hat{P}_{1}\left(w_{1}\right)\hat{P}_{2}\left(w_{2}\right)\right)\hat{F}_{1}\left(\hat{\theta}_{1}\left(P_{1}\left(z_{1}\right)\tilde{P}_{2}\left(z_{2}\right)
\hat{P}_{2}\left(w_{2}\right)\right),w_{2}\right)F_{1}\left(z_{1},z_{2}\right)\right)\\
&=&\hat{r}_{1}\hat{P}_{2}^{(2)}\left(1\right)+\hat{\mu}_{2}^{2}\hat{R}_{1}^{(2)}\left(1\right)+
2\hat{r}_{1}\hat{\mu}_{2}\left(\hat{F}_{1}^{(0,1)}+\frac{\hat{\mu}_{2}}{1-\hat{\mu}_{1}}\hat{F}_{1}^{(1,0)}\right)+
\hat{F}_{1}^{(0,2)}+\frac{1}{1-\hat{\mu}_{1}}\hat{P}_{2}^{(2)}\left(1\right)\hat{F}_{1}^{(1,0)}\\
&+&\hat{\mu}_{2}^{2}\hat{\theta}_{1}^{(2)}\left(1\right)\hat{F}_{1}^{(1,0)}+\frac{\hat{\mu}_{2}}{1-\hat{\mu}_{1}}\hat{F}_{1}^{(1,1)}+\frac{\hat{\mu}_{2}}{1-\hat{\mu}_{1}}\left(\hat{F}_{1}^{(1,1)}+\frac{\hat{\mu}_{2}}{1-\hat{\mu}_{1}}\hat{F}_{1}^{(2,0)}\right).
\end{eqnarray*}
%_________________________________________________________________________________________________________
%
%_________________________________________________________________________________________________________

\end{enumerate}




Las ecuaciones que determinan los segundos momentos de las longitudes de las colas de los dos sistemas se pueden ver en \href{http://sitio.expresauacm.org/s/carlosmartinez/wp-content/uploads/sites/13/2014/01/SegundosMomentos.pdf}{este sitio}

%\url{http://ubuntu_es_el_diablo.org},\href{http://www.latex-project.org/}{latex project}

%http://sitio.expresauacm.org/s/carlosmartinez/wp-content/uploads/sites/13/2014/01/SegundosMomentos.jpg
%http://sitio.expresauacm.org/s/carlosmartinez/wp-content/uploads/sites/13/2014/01/SegundosMomentos.pdf




%_____________________________________________________________________________________
%Distribuci\'on del n\'umero de usuaruios que pasan del sistema 1 al sistema 2
%_____________________________________________________________________________________
\section*{Ap\'endice B}
%________________________________________________________________________________________
%
%________________________________________________________________________________________
\subsection*{Distribuci\'on para los usuarios de traslado}
%________________________________________________________________________________________
Se puede demostrar que
\begin{equation}
\frac{d^{k}}{dy}\left(\frac{\lambda +\mu}{\lambda
+\mu-y}\right)=\frac{k!}{\left(\lambda+\mu\right)^{k}}
\end{equation}



\begin{Prop}
Sea $\tau$ variable aleatoria no negativa con distribuci\'on exponencial con media $\mu$, y sea $L\left(t\right)$ proceso
Poisson con par\'ametro $\lambda$. Entonces
\begin{equation}
\prob\left\{L\left(\tau\right)=k\right\}=f_{L\left(\tau\right)}\left(k\right)=\left(\frac{\mu}{\lambda
+\mu}\right)\left(\frac{\lambda}{\lambda+\mu}\right)^{k}.
\end{equation}
Adem\'as

\begin{eqnarray}
\esp\left[L\left(\tau\right)\right]&=&\frac{\lambda}{\mu}\\
\esp\left[\left(L\left(\tau\right)\right)^{2}\right]&=&\frac{\lambda}{\mu}\left(2\frac{\lambda}{\mu}+1\right)\\
V\left[L\left(\tau\right)\right]&=&\frac{\lambda}{\mu}\left(\frac{\lambda}{\mu}+1\right).
\end{eqnarray}
\end{Prop}

\begin{Proof}
A saber, para $k$ fijo se tiene que

\begin{eqnarray*}
\prob\left\{L\left(\tau\right)=k\right\}&=&\prob\left\{L\left(\tau\right)=k,\tau\in\left(0,\infty\right)\right\}\\
&=&\int_{0}^{\infty}\prob\left\{L\left(\tau\right)=k,\tau=y\right\}f_{\tau}\left(y\right)dy=\int_{0}^{\infty}\prob\left\{L\left(y\right)=k\right\}f_{\tau}\left(y\right)dy\\
&=&\int_{0}^{\infty}\frac{e^{-\lambda
y}}{k!}\left(\lambda y\right)^{k}\left(\mu e^{-\mu
y}\right)dy=\frac{\lambda^{k}\mu}{k!}\int_{0}^{\infty}y^{k}e^{-\left(\mu+\lambda\right)y}dy\\
&=&\frac{\lambda^{k}\mu}{\left(\lambda
+\mu\right)k!}\int_{0}^{\infty}y^{k}\left(\lambda+\mu\right)e^{-\left(\lambda+\mu\right)y}dy=\frac{\lambda^{k}\mu}{\left(\lambda
+\mu\right)k!}\int_{0}^{\infty}y^{k}f_{Y}\left(y\right)dy\\
&=&\frac{\lambda^{k}\mu}{\left(\lambda
+\mu\right)k!}\esp\left[Y^{k}\right]=\frac{\lambda^{k}\mu}{\left(\lambda
+\mu\right)k!}\frac{d^{k}}{dy}\left(\frac{\lambda
+\mu}{\lambda
+\mu-y}\right)|_{y=0}\\
&=&\frac{\lambda^{k}\mu}{\left(\lambda
+\mu\right)k!}\frac{k!}{\left(\lambda+\mu\right)^{k}}=\left(\frac{\mu}{\lambda
+\mu}\right)\left(\frac{\lambda}{\lambda+\mu}\right)^{k}.\\
\end{eqnarray*}


Adem\'as
\begin{eqnarray*}
\sum_{k=0}^{\infty}\prob\left\{L\left(\tau\right)=k\right\}&=&\sum_{k=0}^{\infty}\left(\frac{\mu}{\lambda
+\mu}\right)\left(\frac{\lambda}{\lambda+\mu}\right)^{k}=\frac{\mu}{\lambda
+\mu}\sum_{k=0}^{\infty}\left(\frac{\lambda}{\lambda+\mu}\right)^{k}\\
&=&\frac{\mu}{\lambda
+\mu}\left(\frac{1}{1-\frac{\lambda}{\lambda+\mu}}\right)=\frac{\mu}{\lambda
+\mu}\left(\frac{\lambda+\mu}{\mu}\right)\\
&=&1.\\
\end{eqnarray*}

determinemos primero la esperanza de
$L\left(\tau\right)$:


\begin{eqnarray*}
\esp\left[L\left(\tau\right)\right]&=&\sum_{k=0}^{\infty}k\prob\left\{L\left(\tau\right)=k\right\}=\sum_{k=0}^{\infty}k\left(\frac{\mu}{\lambda
+\mu}\right)\left(\frac{\lambda}{\lambda+\mu}\right)^{k}\\
&=&\left(\frac{\mu}{\lambda
+\mu}\right)\sum_{k=0}^{\infty}k\left(\frac{\lambda}{\lambda+\mu}\right)^{k}=\left(\frac{\mu}{\lambda
+\mu}\right)\left(\frac{\lambda}{\lambda+\mu}\right)\sum_{k=1}^{\infty}k\left(\frac{\lambda}{\lambda+\mu}\right)^{k-1}\\
&=&\frac{\mu\lambda}{\left(\lambda
+\mu\right)^{2}}\left(\frac{1}{1-\frac{\lambda}{\lambda+\mu}}\right)^{2}=\frac{\mu\lambda}{\left(\lambda
+\mu\right)^{2}}\left(\frac{\lambda+\mu}{\mu}\right)^{2}\\
&=&\frac{\lambda}{\mu}.
\end{eqnarray*}

Ahora su segundo momento:

\begin{eqnarray*}
\esp\left[\left(L\left(\tau\right)\right)^{2}\right]&=&\sum_{k=0}^{\infty}k^{2}\prob\left\{L\left(\tau\right)=k\right\}=\sum_{k=0}^{\infty}k^{2}\left(\frac{\mu}{\lambda
+\mu}\right)\left(\frac{\lambda}{\lambda+\mu}\right)^{k}\\
&=&\left(\frac{\mu}{\lambda
+\mu}\right)\sum_{k=0}^{\infty}k^{2}\left(\frac{\lambda}{\lambda+\mu}\right)^{k}=
\frac{\mu\lambda}{\left(\lambda
+\mu\right)^{2}}\sum_{k=2}^{\infty}\left(k-1\right)^{2}\left(\frac{\lambda}{\lambda+\mu}\right)^{k-2}\\
&=&\frac{\mu\lambda}{\left(\lambda
+\mu\right)^{2}}\left(\frac{\frac{\lambda}{\lambda+\mu}+1}{\left(\frac{\lambda}{\lambda+\mu}-1\right)^{3}}\right)=\frac{\mu\lambda}{\left(\lambda
+\mu\right)^{2}}\left(-\frac{\frac{2\lambda+\mu}{\lambda+\mu}}{\left(-\frac{\mu}{\lambda+\mu}\right)^{3}}\right)\\
&=&\frac{\mu\lambda}{\left(\lambda
+\mu\right)^{2}}\left(\frac{2\lambda+\mu}{\lambda+\mu}\right)\left(\frac{\lambda+\mu}{\mu}\right)^{3}=\frac{\lambda\left(2\lambda
+\mu\right)}{\mu^{2}}\\
&=&\frac{\lambda}{\mu}\left(2\frac{\lambda}{\mu}+1\right).
\end{eqnarray*}

y por tanto

\begin{eqnarray*}
V\left[L\left(\tau\right)\right]&=&\frac{\lambda\left(2\lambda
+\mu\right)}{\mu^{2}}-\left(\frac{\lambda}{\mu}\right)^{2}=\frac{\lambda^{2}+\mu\lambda}{\mu^{2}}\\
&=&\frac{\lambda}{\mu}\left(\frac{\lambda}{\mu}+1\right).
\end{eqnarray*}
\end{Proof}

Ahora, determinemos la distribuci\'on del n\'umero de usuarios que
pasan de $\hat{Q}_{2}$ a $Q_{2}$ considerando dos pol\'iticas de
traslado en espec\'ifico:

\begin{enumerate}
\item Solamente pasa un usuario,

\item Se permite el paso de $k$ usuarios,
\end{enumerate}
una vez que son atendidos por el servidor en $\hat{Q}_{2}$.

\begin{description}


\item[Pol\'itica de un solo usuario:] Sea $R_{2}$ el n\'umero de
usuarios que llegan a $\hat{Q}_{2}$ al tiempo $t$, sea $R_{1}$ el
n\'umero de usuarios que pasan de $\hat{Q}_{2}$ a $Q_{2}$ al
tiempo $t$.
\end{description}


A saber:
\begin{eqnarray*}
\esp\left[R_{1}\right]&=&\sum_{y\geq0}\prob\left[R_{2}=y\right]\esp\left[R_{1}|R_{2}=y\right]\\
&=&\sum_{y\geq0}\prob\left[R_{2}=y\right]\sum_{x\geq0}x\prob\left[R_{1}=x|R_{2}=y\right]\\
&=&\sum_{y\geq0}\sum_{x\geq0}x\prob\left[R_{1}=x|R_{2}=y\right]\prob\left[R_{2}=y\right].\\
\end{eqnarray*}

Determinemos
\begin{equation}
\esp\left[R_{1}|R_{2}=y\right]=\sum_{x\geq0}x\prob\left[R_{1}=x|R_{2}=y\right].
\end{equation}

supongamos que $y=0$, entonces
\begin{eqnarray*}
\prob\left[R_{1}=0|R_{2}=0\right]&=&1,\\
\prob\left[R_{1}=x|R_{2}=0\right]&=&0,\textrm{ para cualquier }x\geq1,\\
\end{eqnarray*}


por tanto
\begin{eqnarray*}
\esp\left[R_{1}|R_{2}=0\right]=0.
\end{eqnarray*}

Para $y=1$,
\begin{eqnarray*}
\prob\left[R_{1}=0|R_{2}=1\right]&=&0,\\
\prob\left[R_{1}=1|R_{2}=1\right]&=&1,
\end{eqnarray*}

entonces
\begin{eqnarray*}
\esp\left[R_{1}|R_{2}=1\right]=1.
\end{eqnarray*}

Para $y>1$:
\begin{eqnarray*}
\prob\left[R_{1}=0|R_{2}\geq1\right]&=&0,\\
\prob\left[R_{1}=1|R_{2}\geq1\right]&=&1,\\
\prob\left[R_{1}>1|R_{2}\geq1\right]&=&0,
\end{eqnarray*}

entonces
\begin{eqnarray*}
\esp\left[R_{1}|R_{2}=y\right]=1,\textrm{ para cualquier }y>1.
\end{eqnarray*}
es decir
\begin{eqnarray*}
\esp\left[R_{1}|R_{2}=y\right]=1,\textrm{ para cualquier }y\geq1.
\end{eqnarray*}

Entonces
\begin{eqnarray*}
\esp\left[R_{1}\right]&=&\sum_{y\geq0}\sum_{x\geq0}x\prob\left[R_{1}=x|R_{2}=y\right]\prob\left[R_{2}=y\right]=\sum_{y\geq0}\sum_{x}\esp\left[R_{1}|R_{2}=y\right]\prob\left[R_{2}=y\right]\\
&=&\sum_{y\geq0}\prob\left[R_{2}=y\right]=\sum_{y\geq1}\frac{\left(\lambda
t\right)^{k}}{k!}e^{-\lambda t}=1.
\end{eqnarray*}

Adem\'as para $k\in Z^{+}$
\begin{eqnarray*}
f_{R_{1}}\left(k\right)&=&\prob\left[R_{1}=k\right]=\sum_{n=0}^{\infty}\prob\left[R_{1}=k|R_{2}=n\right]\prob\left[R_{2}=n\right]\\
&=&\prob\left[R_{1}=k|R_{2}=0\right]\prob\left[R_{2}=0\right]+\prob\left[R_{1}=k|R_{2}=1\right]\prob\left[R_{2}=1\right]\\
&+&\prob\left[R_{1}=k|R_{2}>1\right]\prob\left[R_{2}>1\right],
\end{eqnarray*}

donde para


\begin{description}
\item[$k=0$:]
\begin{eqnarray*}
\prob\left[R_{1}=0\right]=\prob\left[R_{1}=0|R_{2}=0\right]\prob\left[R_{2}=0\right]+\prob\left[R_{1}=0|R_{2}=1\right]\prob\left[R_{2}=1\right]\\
+\prob\left[R_{1}=0|R_{2}>1\right]\prob\left[R_{2}>1\right]=\prob\left[R_{2}=0\right].
\end{eqnarray*}
\item[$k=1$:]
\begin{eqnarray*}
\prob\left[R_{1}=1\right]=\prob\left[R_{1}=1|R_{2}=0\right]\prob\left[R_{2}=0\right]+\prob\left[R_{1}=1|R_{2}=1\right]\prob\left[R_{2}=1\right]\\
+\prob\left[R_{1}=1|R_{2}>1\right]\prob\left[R_{2}>1\right]=\sum_{n=1}^{\infty}\prob\left[R_{2}=n\right].
\end{eqnarray*}

\item[$k=2$:]
\begin{eqnarray*}
\prob\left[R_{1}=2\right]=\prob\left[R_{1}=2|R_{2}=0\right]\prob\left[R_{2}=0\right]+\prob\left[R_{1}=2|R_{2}=1\right]\prob\left[R_{2}=1\right]\\
+\prob\left[R_{1}=2|R_{2}>1\right]\prob\left[R_{2}>1\right]=0.
\end{eqnarray*}

\item[$k=j$:]
\begin{eqnarray*}
\prob\left[R_{1}=j\right]=\prob\left[R_{1}=j|R_{2}=0\right]\prob\left[R_{2}=0\right]+\prob\left[R_{1}=j|R_{2}=1\right]\prob\left[R_{2}=1\right]\\
+\prob\left[R_{1}=j|R_{2}>1\right]\prob\left[R_{2}>1\right]=0.
\end{eqnarray*}
\end{description}


Por lo tanto
\begin{eqnarray*}
f_{R_{1}}\left(0\right)&=&\prob\left[R_{2}=0\right]\\
f_{R_{1}}\left(1\right)&=&\sum_{n\geq1}^{\infty}\prob\left[R_{2}=n\right]\\
f_{R_{1}}\left(j\right)&=&0,\textrm{ para }j>1.
\end{eqnarray*}



\begin{description}
\item[Pol\'itica de $k$ usuarios:]Al igual que antes, para $y\in Z^{+}$ fijo
\begin{eqnarray*}
\esp\left[R_{1}|R_{2}=y\right]=\sum_{x}x\prob\left[R_{1}=x|R_{2}=y\right].\\
\end{eqnarray*}
\end{description}
Entonces, si tomamos diversos valore para $y$:\\

$y=0$:
\begin{eqnarray*}
\prob\left[R_{1}=0|R_{2}=0\right]&=&1,\\
\prob\left[R_{1}=x|R_{2}=0\right]&=&0,\textrm{ para cualquier }x\geq1,
\end{eqnarray*}

entonces
\begin{eqnarray*}
\esp\left[R_{1}|R_{2}=0\right]=0.
\end{eqnarray*}


Para $y=1$,
\begin{eqnarray*}
\prob\left[R_{1}=0|R_{2}=1\right]&=&0,\\
\prob\left[R_{1}=1|R_{2}=1\right]&=&1,
\end{eqnarray*}

entonces {\scriptsize{
\begin{eqnarray*}
\esp\left[R_{1}|R_{2}=1\right]=1.
\end{eqnarray*}}}


Para $y=2$,
\begin{eqnarray*}
\prob\left[R_{1}=0|R_{2}=2\right]&=&0,\\
\prob\left[R_{1}=1|R_{2}=2\right]&=&1,\\
\prob\left[R_{1}=2|R_{2}=2\right]&=&1,\\
\prob\left[R_{1}=3|R_{2}=2\right]&=&0,
\end{eqnarray*}

entonces
\begin{eqnarray*}
\esp\left[R_{1}|R_{2}=2\right]=3.
\end{eqnarray*}

Para $y=3$,
\begin{eqnarray*}
\prob\left[R_{1}=0|R_{2}=3\right]&=&0,\\
\prob\left[R_{1}=1|R_{2}=3\right]&=&1,\\
\prob\left[R_{1}=2|R_{2}=3\right]&=&1,\\
\prob\left[R_{1}=3|R_{2}=3\right]&=&1,\\
\prob\left[R_{1}=4|R_{2}=3\right]&=&0,
\end{eqnarray*}

entonces
\begin{eqnarray*}
\esp\left[R_{1}|R_{2}=3\right]=6.
\end{eqnarray*}

En general, para $k\geq0$,
\begin{eqnarray*}
\prob\left[R_{1}=0|R_{2}=k\right]&=&0,\\
\prob\left[R_{1}=j|R_{2}=k\right]&=&1,\textrm{ para }1\leq j\leq k,\\
\prob\left[R_{1}=j|R_{2}=k\right]&=&0,\textrm{ para }j> k,
\end{eqnarray*}

entonces
\begin{eqnarray*}
\esp\left[R_{1}|R_{2}=k\right]=\frac{k\left(k+1\right)}{2}.
\end{eqnarray*}



Por lo tanto


\begin{eqnarray*}
\esp\left[R_{1}\right]&=&\sum_{y}\esp\left[R_{1}|R_{2}=y\right]\prob\left[R_{2}=y\right]\\
&=&\sum_{y}\prob\left[R_{2}=y\right]\frac{y\left(y+1\right)}{2}=\sum_{y\geq1}\left(\frac{y\left(y+1\right)}{2}\right)\frac{\left(\lambda t\right)^{y}}{y!}e^{-\lambda t}\\
&=&\frac{\lambda t}{2}e^{-\lambda t}\sum_{y\geq1}\left(y+1\right)\frac{\left(\lambda t\right)^{y-1}}{\left(y-1\right)!}=\frac{\lambda t}{2}e^{-\lambda t}\left(e^{\lambda t}\left(\lambda t+2\right)\right)\\
&=&\frac{\lambda t\left(\lambda t+2\right)}{2},
\end{eqnarray*}
es decir,


\begin{equation}
\esp\left[R_{1}\right]=\frac{\lambda t\left(\lambda
t+2\right)}{2}.
\end{equation}

Adem\'as para $k\in Z^{+}$ fijo
\begin{eqnarray*}
f_{R_{1}}\left(k\right)&=&\prob\left[R_{1}=k\right]=\sum_{n=0}^{\infty}\prob\left[R_{1}=k|R_{2}=n\right]\prob\left[R_{2}=n\right]\\
&=&\prob\left[R_{1}=k|R_{2}=0\right]\prob\left[R_{2}=0\right]+\prob\left[R_{1}=k|R_{2}=1\right]\prob\left[R_{2}=1\right]\\
&+&\prob\left[R_{1}=k|R_{2}=2\right]\prob\left[R_{2}=2\right]+\cdots+\prob\left[R_{1}=k|R_{2}=j\right]\prob\left[R_{2}=j\right]+\cdots+
\end{eqnarray*}
donde para

\begin{description}
\item[$k=0$:]
\begin{eqnarray*}
\prob\left[R_{1}=0\right]=\prob\left[R_{1}=0|R_{2}=0\right]\prob\left[R_{2}=0\right]+\prob\left[R_{1}=0|R_{2}=1\right]\prob\left[R_{2}=1\right]\\
+\prob\left[R_{1}=0|R_{2}=j\right]\prob\left[R_{2}=j\right]=\prob\left[R_{2}=0\right].
\end{eqnarray*}
\item[$k=1$:]
\begin{eqnarray*}
\prob\left[R_{1}=1\right]=\prob\left[R_{1}=1|R_{2}=0\right]\prob\left[R_{2}=0\right]+\prob\left[R_{1}=1|R_{2}=1\right]\prob\left[R_{2}=1\right]\\
+\prob\left[R_{1}=1|R_{2}=1\right]\prob\left[R_{2}=1\right]+\cdots+\prob\left[R_{1}=1|R_{2}=j\right]\prob\left[R_{2}=j\right]\\
=\sum_{n=1}^{\infty}\prob\left[R_{2}=n\right].
\end{eqnarray*}

\item[$k=2$:]
\begin{eqnarray*}
\prob\left[R_{1}=2\right]=\prob\left[R_{1}=2|R_{2}=0\right]\prob\left[R_{2}=0\right]+\prob\left[R_{1}=2|R_{2}=1\right]\prob\left[R_{2}=1\right]\\
+\prob\left[R_{1}=2|R_{2}=2\right]\prob\left[R_{2}=2\right]+\cdots+\prob\left[R_{1}=2|R_{2}=j\right]\prob\left[R_{2}=j\right]\\
=\sum_{n=2}^{\infty}\prob\left[R_{2}=n\right].
\end{eqnarray*}
\end{description}

En general

\begin{eqnarray*}
\prob\left[R_{1}=k\right]=\prob\left[R_{1}=k|R_{2}=0\right]\prob\left[R_{2}=0\right]+\prob\left[R_{1}=k|R_{2}=1\right]\prob\left[R_{2}=1\right]\\
+\prob\left[R_{1}=k|R_{2}=2\right]\prob\left[R_{2}=2\right]+\cdots+\prob\left[R_{1}=k|R_{2}=k\right]\prob\left[R_{2}=k\right]\\
=\sum_{n=k}^{\infty}\prob\left[R_{2}=n\right].\\
\end{eqnarray*}



Por lo tanto

\begin{eqnarray*}
f_{R_{1}}\left(k\right)&=&\prob\left[R_{1}=k\right]=\sum_{n=k}^{\infty}\prob\left[R_{2}=n\right].
\end{eqnarray*}







\section*{Objetivos Principales}

\begin{itemize}
%\item Generalizar los principales resultados existentes para Sistemas de Visitas C\'iclicas para el caso en el que se tienen dos Sistemas de Visitas C\'iclicas con propiedades similares.

\item Encontrar las ecuaciones que modelan el comportamiento de una Red de Sistemas de Visitas C\'iclicas (RSVC) con propiedades similares.

\item Encontrar expresiones anal\'iticas para las longitudes de las colas al momento en que el servidor llega a una de ellas para comenzar a dar servicio, as\'i como de sus segundos momentos.

\item Determinar las principales medidas de Desempe\~no para la RSVC tales como: N\'umero de usuarios presentes en cada una de las colas del sistema cuando uno de los servidores est\'a presente atendiendo, Tiempos que transcurre entre las visitas del servidor a la misma cola.


\end{itemize}


%_________________________________________________________________________
%\section{Sistemas de Visitas C\'iclicas}
%_________________________________________________________________________
\numberwithin{equation}{section}%
%__________________________________________________________________________




%\section*{Introducci\'on}




%__________________________________________________________________________
%\subsection{Definiciones}
%__________________________________________________________________________


\section{Descripci\'on de una Red de Sistemas de Visitas C\'iclicas}

Consideremos una red de sistema de visitas c\'iclicas conformada por dos sistemas de visitas c\'iclicas, cada una con dos colas independientes, donde adem\'as se permite el intercambio de usuarios entre los dos sistemas en la segunda cola de cada uno de ellos.

%____________________________________________________________________
\subsection*{Supuestos sobe la Red de Sistemas de Visitas C\'iclicas}
%____________________________________________________________________

\begin{itemize}
\item Los arribos de los usuarios ocurren
conforme a un proceso Poisson con tasa de llegada $\mu_{1}$ y
$\mu_{2}$ para el sistema 1, mientras que para el sistema 2,
lo hacen conforme a un proceso Poisson con tasa
$\hat{\mu}_{1},\hat{\mu}_{2}$ respectivamente.



\item Se considerar\'an intervalos de tiempo de la forma
$\left[t,t+1\right]$. Los usuarios arriban por paquetes de manera
independiente del resto de las colas. Se define el grupo de
usuarios que llegan a cada una de las colas del sistema 1,
caracterizadas por $Q_{1}$ y $Q_{2}$ respectivamente, en el
intervalo de tiempo $\left[t,t+1\right]$ por
$X_{1}\left(t\right),X_{2}\left(t\right)$.


\item Se definen los procesos
$\hat{X}_{1}\left(t\right),\hat{X}_{2}\left(t\right)$ para las
colas del sistema 2, denotadas por $\hat{Q}_{1}$ y $\hat{Q}_{2}$
respectivamente. Donde adem\'as se supone que $\mu_{i}<1$ y $\hat{\mu}<1$ para $i=1,2$.


\item Se define el proceso
$Y_{2}\left(t\right)$ para el n\'umero de usuarios que se trasladan del sistema 2 al sistema 1, de la cola $\hat{Q}_{2}$ a la cola
$Q_{2}$, en el intervalo de tiempo $\left[t,t+1\right]$. El traslado de un sistema a otro ocurre de manera que los tiempos entre llegadas de los usuarios a la cola dos del sistema 1 provenientes del sistema 2, se distribuye de manera general con par\'ametro $\check{\mu}_{2}$, con $\check{\mu}_{2}<1$.



\item En lo que respecta al servidor, en t\'erminos de los tiempos de
visita a cada una de las colas, se definen las variables
aleatorias $\tau_{i},$ para $Q_{i}$, para $i=1,2$, respectivamente;
y $\zeta_{i}$ para $\hat{Q}_{i}$,  $i=1,2$,  del sistema
2 respectivamente. A los tiempos en que el servidor termina de atender en las colas $Q_{i},\hat{Q}_{i}$,se les denotar\'a por
$\overline{\tau}_{i},\overline{\zeta}_{i}$ para  $i=1,2$,
respectivamente.

\item Los tiempos de traslado del servidor desde el momento en que termina de atender a una cola y llega a la siguiente para comenzar a dar servicio est\'an dados por
$\tau_{i+1}-\overline{\tau}_{i}$ y
$\zeta_{i+1}-\overline{\zeta}_{i}$,  $i=1,2$, para el sistema 1 y el sistema 2, respectivamente.

\end{itemize}




%\begin{figure}[H]
%\centering
%%%\includegraphics[width=5cm]{RedSistemasVisitasCiclicas.jpg}
%%\end{figure}\label{RSVC}

El uso de la Funci\'on Generadora de Probabilidades (FGP's) nos permite determinar las Funciones de Distribuci\'on de Probabilidades Conjunta de manera indirecta sin necesidad de hacer uso de las propiedades de las distribuciones de probabilidad de cada uno de los procesos que intervienen en la Red de Sistemas de Visitas C\'iclicas.\smallskip

Cada uno de estos procesos con su respectiva FGP. Adem\'as, para cada una de las colas en cada sistema, el n\'umero de usuarios al tiempo en que llega el servidor a dar servicio est\'a
dado por el n\'umero de usuarios presentes en la cola al tiempo
$t$, m\'as el n\'umero de usuarios que llegan a la cola en el intervalo de tiempo
$\left[\tau_{i},\overline{\tau}_{i}\right]$.




Una vez definidas las Funciones Generadoras de Probabilidades Conjuntas se construyen las ecuaciones recursivas que permiten obtener la informaci\'on sobre la longitud de cada una de las colas, al momento en que uno de los servidores llega a una de las colas para dar servicio, bas\'andose en la informaci\'on que se tiene sobre su llegada a la cola inmediata anterior.\smallskip

%__________________________________________________________________________
\subsection{Funciones Generadoras de Probabilidades}
%__________________________________________________________________________


Para cada uno de los procesos de llegada a las colas $X_{i},\hat{X}_{i}$,  $i=1,2$,  y $Y_{2}$, con $\tilde{X}_{2}=X_{2}+Y_{2}$ anteriores se define su Funci\'on
Generadora de Probabilidades (FGP): $P_{i}\left(z_{i}\right)=\esp\left[z_{i}^{X_{i}\left(t\right)}\right],\hat{P}_{i}\left(w_{i}\right)=\esp\left[w_{i}^{\hat{X}_{i}\left(t\right)}\right]$, para
$i=1,2$, y $\check{P}_{2}\left(z_{2}\right)=\esp\left[z_{2}^{Y_{2}\left(t\right)}\right], \tilde{P}_{2}\left(z_{2}\right)=\esp\left[z_{2}^{\tilde{X}_{2}\left(t\right)}\right]$ , con primer momento definidos por $\mu_{i}=\esp\left[X_{i}\left(t\right)\right]=P_{i}^{(1)}\left(1\right), \hat{\mu}_{i}=\esp\left[\hat{X}_{i}\left(t\right)\right]=\hat{P}_{i}^{(1)}\left(1\right)$, para $i=1,2$, y
$\check{\mu}_{2}=\esp\left[Y_{2}\left(t\right)\right]=\check{P}_{2}^{(1)}\left(1\right),\tilde{\mu}_{2}=\esp\left[\tilde{X}_{2}\left(t\right)\right]=\tilde{P}_{2}^{(1)}\left(1\right)$.

En lo que respecta al servidor, en t\'erminos de los tiempos de
visita a cada una de las colas, se denotar\'an por
$B_{i}\left(t\right)$ a los procesos
correspondientes a las variables aleatorias $\tau_{i}$
para $Q_{i}$, respectivamente; y
$\hat{B}_{i}\left(t\right)$ con
par\'ametros $\zeta_{i}$ para $\hat{Q}_{i}$, del sistema 2 respectivamente. Y a los tiempos en que el servidor termina de
atender en las colas $Q_{i},\hat{Q}_{i}$, se les
denotar\'a por
$\overline{\tau}_{i},\overline{\zeta}_{i}$ respectivamente. Entonces, los tiempos de servicio est\'an dados por las diferencias
$\overline{\tau}_{i}-\tau_{i}$ para
$Q_{i}$, y
$\overline{\zeta}_{i}-\zeta_{i}$ para $\hat{Q}_{i}$ respectivamente, para $i=1,2$.

Sus procesos se definen por: $S_{i}\left(z_{i}\right)=\esp\left[z_{i}^{\overline{\tau}_{i}-\tau_{i}}\right]$ y $\hat{S}_{i}\left(w_{i}\right)=\esp\left[w_{i}^{\overline{\zeta}_{i}-\zeta_{i}}\right]$, con primer momento dado por: $s_{i}=\esp\left[\overline{\tau}_{i}-\tau_{i}\right]$ y $\hat{s}_{i}=\esp\left[\overline{\zeta}_{i}-\zeta_{i}\right]$, para $i=1,2$. An\'alogamente los tiempos de traslado del servidor desde el momento en que termina de atender a una cola y llega a la
siguiente para comenzar a dar servicio est\'an dados por
$\tau_{i+1}-\overline{\tau}_{i}$ y
$\zeta_{i+1}-\overline{\zeta}_{i}$ para el sistema 1 y el sistema 2, respectivamente, con $i=1,2$.

La FGP para estos tiempos de traslado est\'an dados por $R_{i}\left(z_{i}\right)=\esp\left[z_{1}^{\tau_{i+1}-\overline{\tau}_{i}}\right]$ y $\hat{R}_{i}\left(w_{i}\right)=\esp\left[w_{i}^{\zeta_{i+1}-\overline{\zeta}_{i}}\right]$ y al igual que como se hizo con anterioridad, se tienen los primeros momentos de estos procesos de traslado del servidor entre las colas de cada uno de los sistemas que conforman la red de sistemas de visitas c\'iclicas: $r_{i}=R_{i}^{(1)}\left(1\right)=\esp\left[\tau_{i+1}-\overline{\tau}_{i}\right]$ y $\hat{r}_{i}=\hat{R}_{i}^{(1)}\left(1\right)=\esp\left[\zeta_{i+1}-\overline{\zeta}_{i}\right]$ para $i=1,2$.


Se definen los procesos de conteo para el n\'umero de usuarios en
cada una de las colas al tiempo $t$,
$L_{i}\left(t\right)$, para
$H_{i}\left(t\right)$ del sistema 1,
mientras que para el segundo sistema, se tienen los procesos
$\hat{L}_{i}\left(t\right)$ para
$\hat{H}_{i}\left(t\right)$, es decir, $H_{i}\left(t\right)=\esp\left[z_{i}^{L_{i}\left(t\right)}\right]$ y $\hat{H}_{i}\left(t\right)=\esp\left[w_{i}^{\hat{L}_{i}\left(t\right)}\right]$. Con lo dichohasta ahora, se tiene que el n\'umero de usuarios
presentes en los tiempos $\overline{\tau}_{1},\overline{\tau}_{2},
\overline{\zeta}_{1},\overline{\zeta}_{2}$, es cero, es decir,
 $L_{i}\left(\overline{\tau_{i}}\right)=0,$ y
$\hat{L}_{i}\left(\overline{\zeta_{i}}\right)=0$ para i=1,2 para
cada uno de los dos sistemas.


Para cada una de las colas en cada sistema, el n\'umero de
usuarios al tiempo en que llega el servidor a dar servicio est\'a
dado por el n\'umero de usuarios presentes en la cola al tiempo
$t=\tau_{i},\zeta_{i}$, m\'as el n\'umero de usuarios que llegan a
la cola en el intervalo de tiempo
$\left[\tau_{i},\overline{\tau}_{i}\right],\left[\zeta_{i},\overline{\zeta}_{i}\right]$,
es decir $\hat{L}_{i}\left(\overline{\tau}_{j}\right)=\hat{L}_{i}\left(\tau_{j}\right)+\hat{X}_{i}\left(\overline{\tau}_{j}-\tau_{j}\right)$, para $i,j=1,2$, mientras que para el primer sistema: $L_{1}\left(\overline{\tau}_{j}\right)=L_{1}\left(\tau_{j}\right)+X_{1}\left(\overline{\tau}_{j}-\tau_{j}\right)$. En el caso espec\'ifico de $Q_{2}$, adem\'as, hay que considerar
el n\'umero de usuarios que pasan del sistema 2 al sistema 1, a
traves de $\hat{Q}_{2}$ mientras el servidor en $Q_{2}$ est\'a
ausente, es decir:

\begin{equation}\label{Eq.UsuariosTotalesZ2}
L_{2}\left(\overline{\tau}_{1}\right)=L_{2}\left(\tau_{1}\right)+X_{2}\left(\overline{\tau}_{1}-\tau_{1}\right)+Y_{2}\left(\overline{\tau}_{1}-\tau_{1}\right).
\end{equation}

%_________________________________________________________________________
\subsection{El problema de la ruina del jugador}
%_________________________________________________________________________

Supongamos que se tiene un jugador que cuenta con un capital
inicial de $\tilde{L}_{0}\geq0$ unidades, esta persona realiza una
serie de dos juegos simult\'aneos e independientes de manera
sucesiva, dichos eventos son independientes e id\'enticos entre
s\'i para cada realizaci\'on. La ganancia en el $n$-\'esimo juego es $\tilde{X}_{n}=X_{n}+Y_{n}$ unidades de las cuales se resta una cuota de 1 unidad por cada juego simult\'aneo, es decir, se restan dos unidades por cada
juego realizado. En t\'erminos de la teor\'ia de colas puede pensarse como el n\'umero de usuarios que llegan a una cola v\'ia dos procesos de arribo distintos e independientes entre s\'i. Su Funci\'on Generadora de Probabilidades (FGP) est\'a dada por $F\left(z\right)=\esp\left[z^{\tilde{L}_{0}}\right]$, adem\'as
$$\tilde{P}\left(z\right)=\esp\left[z^{\tilde{X}_{n}}\right]=\esp\left[z^{X_{n}+Y_{n}}\right]=\esp\left[z^{X_{n}}z^{Y_{n}}\right]=\esp\left[z^{X_{n}}\right]\esp\left[z^{Y_{n}}\right]=P\left(z\right)\check{P}\left(z\right),$$

con $\tilde{\mu}=\esp\left[\tilde{X}_{n}\right]=\tilde{P}\left[z\right]<1$. Sea $\tilde{L}_{n}$ el capital remanente despu\'es del $n$-\'esimo
juego. Entonces

$$\tilde{L}_{n}=\tilde{L}_{0}+\tilde{X}_{1}+\tilde{X}_{2}+\cdots+\tilde{X}_{n}-2n.$$

La ruina del jugador ocurre despu\'es del $n$-\'esimo juego, es decir, la cola se vac\'ia despu\'es del $n$-\'esimo juego,
entonces sea $T$ definida como $T=min\left\{\tilde{L}_{n}=0\right\}$. Si $\tilde{L}_{0}=0$, entonces claramente $T=0$. En este sentido $T$
puede interpretarse como la longitud del periodo de tiempo que el servidor ocupa para dar servicio en la cola, comenzando con $\tilde{L}_{0}$ grupos de usuarios presentes en la cola, quienes arribaron conforme a un proceso dado
por $\tilde{P}\left(z\right)$.\smallskip


Sea $g_{n,k}$ la probabilidad del evento de que el jugador no
caiga en ruina antes del $n$-\'esimo juego, y que adem\'as tenga
un capital de $k$ unidades antes del $n$-\'esimo juego, es decir,

Dada $n\in\left\{1,2,\ldots,\right\}$ y
$k\in\left\{0,1,2,\ldots,\right\}$
\begin{eqnarray*}
g_{n,k}:=P\left\{\tilde{L}_{j}>0, j=1,\ldots,n,
\tilde{L}_{n}=k\right\}
\end{eqnarray*}

la cual adem\'as se puede escribir como:

\begin{eqnarray*}
g_{n,k}&=&P\left\{\tilde{L}_{j}>0, j=1,\ldots,n,
\tilde{L}_{n}=k\right\}=\sum_{j=1}^{k+1}g_{n-1,j}P\left\{\tilde{X}_{n}=k-j+1\right\}\\
&=&\sum_{j=1}^{k+1}g_{n-1,j}P\left\{X_{n}+Y_{n}=k-j+1\right\}=\sum_{j=1}^{k+1}\sum_{l=1}^{j}g_{n-1,j}P\left\{X_{n}+Y_{n}=k-j+1,Y_{n}=l\right\}\\
&=&\sum_{j=1}^{k+1}\sum_{l=1}^{j}g_{n-1,j}P\left\{X_{n}+Y_{n}=k-j+1|Y_{n}=l\right\}P\left\{Y_{n}=l\right\}\\
&=&\sum_{j=1}^{k+1}\sum_{l=1}^{j}g_{n-1,j}P\left\{X_{n}=k-j-l+1\right\}P\left\{Y_{n}=l\right\}\\
\end{eqnarray*}

es decir
\begin{eqnarray}\label{Eq.Gnk.2S}
g_{n,k}=\sum_{j=1}^{k+1}\sum_{l=1}^{j}g_{n-1,j}P\left\{X_{n}=k-j-l+1\right\}P\left\{Y_{n}=l\right\}
\end{eqnarray}
adem\'as

\begin{equation}\label{Eq.L02S}
g_{0,k}=P\left\{\tilde{L}_{0}=k\right\}.
\end{equation}

Se definen las siguientes FGP:
\begin{equation}\label{Eq.3.16.a.2S}
G_{n}\left(z\right)=\sum_{k=0}^{\infty}g_{n,k}z^{k},\textrm{ para
}n=0,1,\ldots,
\end{equation}

\begin{equation}\label{Eq.3.16.b.2S}
G\left(z,w\right)=\sum_{n=0}^{\infty}G_{n}\left(z\right)w^{n}.
\end{equation}


En particular para $k=0$,
\begin{eqnarray*}
g_{n,0}=G_{n}\left(0\right)=P\left\{\tilde{L}_{j}>0,\textrm{ para
}j<n,\textrm{ y }\tilde{L}_{n}=0\right\}=P\left\{T=n\right\},
\end{eqnarray*}

adem\'as

\begin{eqnarray*}%\label{Eq.G0w.2S}
G\left(0,w\right)=\sum_{n=0}^{\infty}G_{n}\left(0\right)w^{n}=\sum_{n=0}^{\infty}P\left\{T=n\right\}w^{n}
=\esp\left[w^{T}\right]
\end{eqnarray*}
la cu\'al resulta ser la FGP del tiempo de ruina $T$.

%__________________________________________________________________________________
% INICIA LA PROPOSICIÓN
%__________________________________________________________________________________


\begin{Prop}\label{Prop.1.1.2S}
Sean $G_{n}\left(z\right)$ y $G\left(z,w\right)$ definidas como en
(\ref{Eq.3.16.a.2S}) y (\ref{Eq.3.16.b.2S}) respectivamente,
entonces
\begin{equation}\label{Eq.Pag.45}
G_{n}\left(z\right)=\frac{1}{z}\left[G_{n-1}\left(z\right)-G_{n-1}\left(0\right)\right]\tilde{P}\left(z\right).
\end{equation}

Adem\'as


\begin{equation}\label{Eq.Pag.46}
G\left(z,w\right)=\frac{zF\left(z\right)-wP\left(z\right)G\left(0,w\right)}{z-wR\left(z\right)},
\end{equation}

con un \'unico polo en el c\'irculo unitario, adem\'as, el polo es
de la forma $z=\theta\left(w\right)$ y satisface que

\begin{enumerate}
\item[i)]$\tilde{\theta}\left(1\right)=1$,

\item[ii)] $\tilde{\theta}^{(1)}\left(1\right)=\frac{1}{1-\tilde{\mu}}$,

\item[iii)]
$\tilde{\theta}^{(2)}\left(1\right)=\frac{\tilde{\mu}}{\left(1-\tilde{\mu}\right)^{2}}+\frac{\tilde{\sigma}}{\left(1-\tilde{\mu}\right)^{3}}$.
\end{enumerate}

Finalmente, adem\'as se cumple que
\begin{equation}
\esp\left[w^{T}\right]=G\left(0,w\right)=F\left[\tilde{\theta}\left(w\right)\right].
\end{equation}
\end{Prop}
%__________________________________________________________________________________
% TERMINA LA PROPOSICIÓN E INICIA LA DEMOSTRACI\'ON
%__________________________________________________________________________________


Multiplicando las ecuaciones (\ref{Eq.Gnk.2S}) y (\ref{Eq.L02S})
por el t\'ermino $z^{k}$:

\begin{eqnarray*}
g_{n,k}z^{k}&=&\sum_{j=1}^{k+1}\sum_{l=1}^{j}g_{n-1,j}P\left\{X_{n}=k-j-l+1\right\}P\left\{Y_{n}=l\right\}z^{k},\\
g_{0,k}z^{k}&=&P\left\{\tilde{L}_{0}=k\right\}z^{k},
\end{eqnarray*}

ahora sumamos sobre $k$
\begin{eqnarray*}
\sum_{k=0}^{\infty}g_{n,k}z^{k}&=&\sum_{k=0}^{\infty}\sum_{j=1}^{k+1}\sum_{l=1}^{j}g_{n-1,j}P\left\{X_{n}=k-j-l+1\right\}P\left\{Y_{n}=l\right\}z^{k}\\
&=&\sum_{k=0}^{\infty}z^{k}\sum_{j=1}^{k+1}\sum_{l=1}^{j}g_{n-1,j}P\left\{X_{n}=k-\left(j+l
-1\right)\right\}P\left\{Y_{n}=l\right\}\\
&=&\sum_{k=0}^{\infty}z^{k+\left(j+l-1\right)-\left(j+l-1\right)}\sum_{j=1}^{k+1}\sum_{l=1}^{j}g_{n-1,j}P\left\{X_{n}=k-
\left(j+l-1\right)\right\}P\left\{Y_{n}=l\right\}\\
&=&\sum_{k=0}^{\infty}\sum_{j=1}^{k+1}\sum_{l=1}^{j}g_{n-1,j}z^{j-1}P\left\{X_{n}=k-
\left(j+l-1\right)\right\}z^{k-\left(j+l-1\right)}P\left\{Y_{n}=l\right\}z^{l}\\
&=&\sum_{j=1}^{\infty}\sum_{l=1}^{j}g_{n-1,j}z^{j-1}\sum_{k=j+l-1}^{\infty}P\left\{X_{n}=k-
\left(j+l-1\right)\right\}z^{k-\left(j+l-1\right)}P\left\{Y_{n}=l\right\}z^{l}\\
&=&\sum_{j=1}^{\infty}g_{n-1,j}z^{j-1}\sum_{l=1}^{j}\sum_{k=j+l-1}^{\infty}P\left\{X_{n}=k-
\left(j+l-1\right)\right\}z^{k-\left(j+l-1\right)}P\left\{Y_{n}=l\right\}z^{l}\\
&=&\sum_{j=1}^{\infty}g_{n-1,j}z^{j-1}\sum_{k=j+l-1}^{\infty}\sum_{l=1}^{j}P\left\{X_{n}=k-
\left(j+l-1\right)\right\}z^{k-\left(j+l-1\right)}P\left\{Y_{n}=l\right\}z^{l}\\
\end{eqnarray*}


luego
\begin{eqnarray*}
&=&\sum_{j=1}^{\infty}g_{n-1,j}z^{j-1}\sum_{k=j+l-1}^{\infty}\sum_{l=1}^{j}P\left\{X_{n}=k-
\left(j+l-1\right)\right\}z^{k-\left(j+l-1\right)}\sum_{l=1}^{j}P
\left\{Y_{n}=l\right\}z^{l}\\
&=&\sum_{j=1}^{\infty}g_{n-1,j}z^{j-1}\sum_{l=1}^{\infty}P\left\{Y_{n}=l\right\}z^{l}
\sum_{k=j+l-1}^{\infty}\sum_{l=1}^{j}
P\left\{X_{n}=k-\left(j+l-1\right)\right\}z^{k-\left(j+l-1\right)}\\
&=&\frac{1}{z}\left[G_{n-1}\left(z\right)-G_{n-1}\left(0\right)\right]\tilde{P}\left(z\right)
\sum_{k=j+l-1}^{\infty}\sum_{l=1}^{j}
P\left\{X_{n}=k-\left(j+l-1\right)\right\}z^{k-\left(j+l-1\right)}\\
&=&\frac{1}{z}\left[G_{n-1}\left(z\right)-G_{n-1}\left(0\right)\right]\tilde{P}\left(z\right)P\left(z\right)=\frac{1}{z}\left[G_{n-1}\left(z\right)-G_{n-1}\left(0\right)\right]\tilde{P}\left(z\right),\\
\end{eqnarray*}

es decir la ecuaci\'on (\ref{Eq.3.16.a.2S}) se puede reescribir
como
\begin{equation}\label{Eq.3.16.a.2Sbis}
G_{n}\left(z\right)=\frac{1}{z}\left[G_{n-1}\left(z\right)-G_{n-1}\left(0\right)\right]\tilde{P}\left(z\right).
\end{equation}

Por otra parte recordemos la ecuaci\'on (\ref{Eq.3.16.a.2S})

\begin{eqnarray*}
G_{n}\left(z\right)&=&\sum_{k=0}^{\infty}g_{n,k}z^{k},\textrm{ entonces }\frac{G_{n}\left(z\right)}{z}=\sum_{k=1}^{\infty}g_{n,k}z^{k-1},\\
\end{eqnarray*}

Por lo tanto utilizando la ecuaci\'on (\ref{Eq.3.16.a.2Sbis}):

\begin{eqnarray*}
G\left(z,w\right)&=&\sum_{n=0}^{\infty}G_{n}\left(z\right)w^{n}=G_{0}\left(z\right)+
\sum_{n=1}^{\infty}G_{n}\left(z\right)w^{n}=F\left(z\right)+\sum_{n=0}^{\infty}\left[G_{n}\left(z\right)-G_{n}\left(0\right)\right]w^{n}\frac{\tilde{P}\left(z\right)}{z}\\
&=&F\left(z\right)+\frac{w}{z}\sum_{n=0}^{\infty}\left[G_{n}\left(z\right)-G_{n}\left(0\right)\right]w^{n-1}\tilde{P}\left(z\right)\\
\end{eqnarray*}

es decir
\begin{eqnarray*}
G\left(z,w\right)&=&F\left(z\right)+\frac{w}{z}\left[G\left(z,w\right)-G\left(0,w\right)\right]\tilde{P}\left(z\right),
\end{eqnarray*}


entonces

\begin{eqnarray*}
G\left(z,w\right)=F\left(z\right)+\frac{w}{z}\left[G\left(z,w\right)-G\left(0,w\right)\right]\tilde{P}\left(z\right)&=&F\left(z\right)+\frac{w}{z}\tilde{P}\left(z\right)G\left(z,w\right)-\frac{w}{z}\tilde{P}\left(z\right)G\left(0,w\right)\\
&\Leftrightarrow&\\
G\left(z,w\right)\left\{1-\frac{w}{z}\tilde{P}\left(z\right)\right\}&=&F\left(z\right)-\frac{w}{z}\tilde{P}\left(z\right)G\left(0,w\right),
\end{eqnarray*}
por lo tanto,
\begin{equation}
G\left(z,w\right)=\frac{zF\left(z\right)-w\tilde{P}\left(z\right)G\left(0,w\right)}{1-w\tilde{P}\left(z\right)}.
\end{equation}


Ahora $G\left(z,w\right)$ es anal\'itica en $|z|=1$. Sean $z,w$ tales que $|z|=1$ y $|w|\leq1$, como $\tilde{P}\left(z\right)$ es FGP
\begin{eqnarray*}
|z-\left(z-w\tilde{P}\left(z\right)\right)|<|z|\Leftrightarrow|w\tilde{P}\left(z\right)|<|z|
\end{eqnarray*}
es decir, se cumplen las condiciones del Teorema de Rouch\'e y por
tanto, $z$ y $z-w\tilde{P}\left(z\right)$ tienen el mismo n\'umero de
ceros en $|z|=1$. Sea $z=\tilde{\theta}\left(w\right)$ la soluci\'on
\'unica de $z-w\tilde{P}\left(z\right)$, es decir

\begin{equation}\label{Eq.Theta.w}
\tilde{\theta}\left(w\right)-w\tilde{P}\left(\tilde{\theta}\left(w\right)\right)=0,
\end{equation}
 con $|\tilde{\theta}\left(w\right)|<1$. Cabe hacer menci\'on que $\tilde{\theta}\left(w\right)$ es la FGP para el tiempo de ruina cuando $\tilde{L}_{0}=1$.


Considerando la ecuaci\'on (\ref{Eq.Theta.w})
\begin{eqnarray*}
0&=&\frac{\partial}{\partial w}\tilde{\theta}\left(w\right)|_{w=1}-\frac{\partial}{\partial w}\left\{w\tilde{P}\left(\tilde{\theta}\left(w\right)\right)\right\}|_{w=1}=\tilde{\theta}^{(1)}\left(w\right)|_{w=1}-\frac{\partial}{\partial w}w\left\{\tilde{P}\left(\tilde{\theta}\left(w\right)\right)\right\}|_{w=1}\\
&-&w\frac{\partial}{\partial w}\tilde{P}\left(\tilde{\theta}\left(w\right)\right)|_{w=1}=\tilde{\theta}^{(1)}\left(1\right)-\tilde{P}\left(\tilde{\theta}\left(1\right)\right)-w\left\{\frac{\partial \tilde{P}\left(\tilde{\theta}\left(w\right)\right)}{\partial \tilde{\theta}\left(w\right)}\cdot\frac{\partial\tilde{\theta}\left(w\right)}{\partial w}|_{w=1}\right\}\\
&&\tilde{\theta}^{(1)}\left(1\right)-\tilde{P}\left(\tilde{\theta}\left(1\right)
\right)-\tilde{P}^{(1)}\left(\tilde{\theta}\left(1\right)\right)\cdot\tilde{\theta}^{(1)}\left(1\right),
\end{eqnarray*}


luego
$$\tilde{P}\left(\tilde{\theta}\left(1\right)\right)=\tilde{\theta}^{(1)}\left(1\right)-\tilde{P}^{(1)}\left(\tilde{\theta}\left(1\right)\right)\cdot
\tilde{\theta}^{(1)}\left(1\right)=\tilde{\theta}^{(1)}\left(1\right)\left(1-\tilde{P}^{(1)}\left(\tilde{\theta}\left(1\right)\right)\right),$$

por tanto $$\tilde{\theta}^{(1)}\left(1\right)=\frac{\tilde{P}\left(\tilde{\theta}\left(1\right)\right)}{\left(1-\tilde{P}^{(1)}\left(\tilde{\theta}\left(1\right)\right)\right)}=\frac{1}{1-\tilde{\mu}}.$$

Ahora determinemos el segundo momento de $\tilde{\theta}\left(w\right)$,
nuevamente consideremos la ecuaci\'on (\ref{Eq.Theta.w}):

\begin{eqnarray*}
0&=&\tilde{\theta}\left(w\right)-w\tilde{P}\left(\tilde{\theta}\left(w\right)\right)\Rightarrow 0=\frac{\partial}{\partial w}\left\{\tilde{\theta}\left(w\right)-w\tilde{P}\left(\tilde{\theta}\left(w\right)\right)\right\}\Rightarrow 0=\frac{\partial}{\partial w}\left\{\frac{\partial}{\partial w}\left\{\tilde{\theta}\left(w\right)-w\tilde{P}\left(\tilde{\theta}\left(w\right)\right)\right\}\right\}\\
\end{eqnarray*}
luego
\begin{eqnarray*}
&&\frac{\partial}{\partial w}\left\{\frac{\partial}{\partial w}\tilde{\theta}\left(w\right)-\frac{\partial}{\partial w}\left[w\tilde{P}\left(\tilde{\theta}\left(w\right)\right)\right]\right\}
=\frac{\partial}{\partial w}\left\{\frac{\partial}{\partial w}\tilde{\theta}\left(w\right)-\frac{\partial}{\partial w}\left[w\tilde{P}\left(\tilde{\theta}\left(w\right)\right)\right]\right\}\\
&=&\frac{\partial}{\partial w}\left\{\frac{\partial \tilde{\theta}\left(w\right)}{\partial w}-\left[\tilde{P}\left(\tilde{\theta}\left(w\right)\right)+w\frac{\partial}{\partial w}R\left(\tilde{\theta}\left(w\right)\right)\right]\right\}=\frac{\partial}{\partial w}\left\{\frac{\partial \tilde{\theta}\left(w\right)}{\partial w}-\left[\tilde{P}\left(\tilde{\theta}\left(w\right)\right)+w\frac{\partial \tilde{P}\left(\tilde{\theta}\left(w\right)\right)}{\partial w}\frac{\partial \tilde{\theta}\left(w\right)}{\partial w}\right]\right\}\\
&=&\frac{\partial}{\partial w}\left\{\tilde{\theta}^{(1)}\left(w\right)-\tilde{P}\left(\tilde{\theta}\left(w\right)\right)-w\tilde{P}^{(1)}\left(\tilde{\theta}\left(w\right)\right)\tilde{\theta}^{(1)}\left(w\right)\right\}\\
&=&\frac{\partial}{\partial w}\tilde{\theta}^{(1)}\left(w\right)-\frac{\partial}{\partial w}\tilde{P}\left(\tilde{\theta}\left(w\right)\right)-\frac{\partial}{\partial w}\left[w\tilde{P}^{(1)}\left(\tilde{\theta}\left(w\right)\right)\tilde{\theta}^{(1)}\left(w\right)\right]\\
&=&\frac{\partial}{\partial
w}\tilde{\theta}^{(1)}\left(w\right)-\frac{\partial
\tilde{P}\left(\tilde{\theta}\left(w\right)\right)}{\partial
\tilde{\theta}\left(w\right)}\frac{\partial \tilde{\theta}\left(w\right)}{\partial
w}-\tilde{P}^{(1)}\left(\tilde{\theta}\left(w\right)\right)\tilde{\theta}^{(1)}\left(w\right)-w\frac{\partial
\tilde{P}^{(1)}\left(\tilde{\theta}\left(w\right)\right)}{\partial
w}\tilde{\theta}^{(1)}\left(w\right)-w\tilde{P}^{(1)}\left(\tilde{\theta}\left(w\right)\right)\frac{\partial
\tilde{\theta}^{(1)}\left(w\right)}{\partial w}\\
&=&\tilde{\theta}^{(2)}\left(w\right)-\tilde{P}^{(1)}\left(\tilde{\theta}\left(w\right)\right)\tilde{\theta}^{(1)}\left(w\right)
-\tilde{P}^{(1)}\left(\tilde{\theta}\left(w\right)\right)\tilde{\theta}^{(1)}\left(w\right)-w\tilde{P}^{(2)}\left(\tilde{\theta}\left(w\right)\right)\left(\tilde{\theta}^{(1)}\left(w\right)\right)^{2}-w\tilde{P}^{(1)}\left(\tilde{\theta}\left(w\right)\right)\tilde{\theta}^{(2)}\left(w\right)\\
&=&\tilde{\theta}^{(2)}\left(w\right)-2\tilde{P}^{(1)}\left(\tilde{\theta}\left(w\right)\right)\tilde{\theta}^{(1)}\left(w\right)-w\tilde{P}^{(2)}\left(\tilde{\theta}\left(w\right)\right)\left(\tilde{\theta}^{(1)}\left(w\right)\right)^{2}-w\tilde{P}^{(1)}\left(\tilde{\theta}\left(w\right)\right)\tilde{\theta}^{(2)}\left(w\right)\\
&=&\tilde{\theta}^{(2)}\left(w\right)\left[1-w\tilde{P}^{(1)}\left(\tilde{\theta}\left(w\right)\right)\right]-
\tilde{\theta}^{(1)}\left(w\right)\left[w\tilde{\theta}^{(1)}\left(w\right)\tilde{P}^{(2)}\left(\tilde{\theta}\left(w\right)\right)+2\tilde{P}^{(1)}\left(\tilde{\theta}\left(w\right)\right)\right]
\end{eqnarray*}


luego

\begin{eqnarray*}
\tilde{\theta}^{(2)}\left(w\right)\left[1-w\tilde{P}^{(1)}\left(\tilde{\theta}\left(w\right)\right)\right]&-&\tilde{\theta}^{(1)}\left(w\right)\left[w\tilde{\theta}^{(1)}\left(w\right)\tilde{P}^{(2)}\left(\tilde{\theta}\left(w\right)\right)
+2\tilde{P}^{(1)}\left(\tilde{\theta}\left(w\right)\right)\right]=0\\
\tilde{\theta}^{(2)}\left(w\right)&=&\frac{\tilde{\theta}^{(1)}\left(w\right)\left[w\tilde{\theta}^{(1)}\left(w\right)\tilde{P}^{(2)}\left(\tilde{\theta}\left(w\right)\right)+2R^{(1)}\left(\tilde{\theta}\left(w\right)\right)\right]}{1-w\tilde{P}^{(1)}\left(\tilde{\theta}\left(w\right)\right)}\\
\tilde{\theta}^{(2)}\left(w\right)&=&\frac{\tilde{\theta}^{(1)}\left(w\right)w\tilde{\theta}^{(1)}\left(w\right)\tilde{P}^{(2)}\left(\tilde{\theta}\left(w\right)\right)}{1-w\tilde{P}^{(1)}\left(\tilde{\theta}\left(w\right)\right)}+\frac{2\tilde{\theta}^{(1)}\left(w\right)\tilde{P}^{(1)}\left(\tilde{\theta}\left(w\right)\right)}{1-w\tilde{P}^{(1)}\left(\tilde{\theta}\left(w\right)\right)}
\end{eqnarray*}


si evaluamos la expresi\'on anterior en $w=1$:
\begin{eqnarray*}
\tilde{\theta}^{(2)}\left(1\right)&=&\frac{\left(\tilde{\theta}^{(1)}\left(1\right)\right)^{2}\tilde{P}^{(2)}\left(\tilde{\theta}\left(1\right)\right)}{1-\tilde{P}^{(1)}\left(\tilde{\theta}\left(1\right)\right)}+\frac{2\tilde{\theta}^{(1)}\left(1\right)\tilde{P}^{(1)}\left(\tilde{\theta}\left(1\right)\right)}{1-\tilde{P}^{(1)}\left(\tilde{\theta}\left(1\right)\right)}=\frac{\left(\tilde{\theta}^{(1)}\left(1\right)\right)^{2}\tilde{P}^{(2)}\left(1\right)}{1-\tilde{P}^{(1)}\left(1\right)}+\frac{2\tilde{\theta}^{(1)}\left(1\right)\tilde{P}^{(1)}\left(1\right)}{1-\tilde{P}^{(1)}\left(1\right)}\\
&=&\frac{\left(\frac{1}{1-\tilde{\mu}}\right)^{2}\tilde{P}^{(2)}\left(1\right)}{1-\tilde{\mu}}+\frac{2\left(\frac{1}{1-\tilde{\mu}}\right)\tilde{\mu}}{1-\tilde{\mu}}=\frac{\tilde{P}^{(2)}\left(1\right)}{\left(1-\tilde{\mu}\right)^{3}}+\frac{2\tilde{\mu}}{\left(1-\tilde{\mu}\right)^{2}}=\frac{\sigma^{2}-\tilde{\mu}+\tilde{\mu}^{2}}{\left(1-\tilde{\mu}\right)^{3}}+\frac{2\tilde{\mu}}{\left(1-\tilde{\mu}\right)^{2}}\\
&=&\frac{\sigma^{2}-\tilde{\mu}+\tilde{\mu}^{2}+2\tilde{\mu}\left(1-\tilde{\mu}\right)}{\left(1-\tilde{\mu}\right)^{3}}\\
\end{eqnarray*}


es decir
\begin{eqnarray*}
\tilde{\theta}^{(2)}\left(1\right)&=&\frac{\sigma^{2}+\tilde{\mu}-\tilde{\mu}^{2}}{\left(1-\tilde{\mu}\right)^{3}}=\frac{\sigma^{2}}{\left(1-\tilde{\mu}\right)^{3}}+\frac{\tilde{\mu}\left(1-\tilde{\mu}\right)}{\left(1-\tilde{\mu}\right)^{3}}=\frac{\sigma^{2}}{\left(1-\tilde{\mu}\right)^{3}}+\frac{\tilde{\mu}}{\left(1-\tilde{\mu}\right)^{2}}.
\end{eqnarray*}

\begin{Coro}
El tiempo de ruina del jugador tiene primer y segundo momento
dados por

\begin{eqnarray}
\esp\left[T\right]&=&\frac{\esp\left[\tilde{L}_{0}\right]}{1-\tilde{\mu}}\\
Var\left[T\right]&=&\frac{Var\left[\tilde{L}_{0}\right]}{\left(1-\tilde{\mu}\right)^{2}}+\frac{\sigma^{2}\esp\left[\tilde{L}_{0}\right]}{\left(1-\tilde{\mu}\right)^{3}}.
\end{eqnarray}
\end{Coro}



%__________________________________________________________________________
\section{Procesos de Llegadas a las colas en la RSVC}
%__________________________________________________________________________

Se definen los procesos de llegada de los usuarios a cada una de
las colas dependiendo de la llegada del servidor pero del sistema
al cu\'al no pertenece la cola en cuesti\'on:

Para el sistema 1 y el servidor del segundo sistema

\begin{eqnarray*}
F_{i,j}\left(z_{i};\zeta_{j}\right)=\esp\left[z_{i}^{L_{i}\left(\zeta_{j}\right)}\right]=
\sum_{k=0}^{\infty}\prob\left[L_{i}\left(\zeta_{j}\right)=k\right]z_{i}^{k}\textrm{, para }i,j=1,2.
%F_{1,1}\left(z_{1};\zeta_{1}\right)&=&\esp\left[z_{1}^{L_{1}\left(\zeta_{1}\right)}\right]=
%\sum_{k=0}^{\infty}\prob\left[L_{1}\left(\zeta_{1}\right)=k\right]z_{1}^{k};\\
%F_{2,1}\left(z_{2};\zeta_{1}\right)&=&\esp\left[z_{2}^{L_{2}\left(\zeta_{1}\right)}\right]=
%\sum_{k=0}^{\infty}\prob\left[L_{2}\left(\zeta_{1}\right)=k\right]z_{2}^{k};\\
%F_{1,2}\left(z_{1};\zeta_{2}\right)&=&\esp\left[z_{1}^{L_{1}\left(\zeta_{2}\right)}\right]=
%\sum_{k=0}^{\infty}\prob\left[L_{1}\left(\zeta_{2}\right)=k\right]z_{1}^{k};\\
%F_{2,2}\left(z_{2};\zeta_{2}\right)&=&\esp\left[z_{2}^{L_{2}\left(\zeta_{2}\right)}\right]=
%\sum_{k=0}^{\infty}\prob\left[L_{2}\left(\zeta_{2}\right)=k\right]z_{2}^{k}.\\
\end{eqnarray*}

Ahora se definen para el segundo sistema y el servidor del primero


\begin{eqnarray*}
\hat{F}_{i,j}\left(w_{i};\tau_{j}\right)&=&\esp\left[w_{i}^{\hat{L}_{i}\left(\tau_{j}\right)}\right] =\sum_{k=0}^{\infty}\prob\left[\hat{L}_{i}\left(\tau_{j}\right)=k\right]w_{i}^{k}\textrm{, para }i,j=1,2.
%\hat{F}_{1,1}\left(w_{1};\tau_{1}\right)&=&\esp\left[w_{1}^{\hat{L}_{1}\left(\tau_{1}\right)}\right] =\sum_{k=0}^{\infty}\prob\left[\hat{L}_{1}\left(\tau_{1}\right)=k\right]w_{1}^{k}\\
%\hat{F}_{2,1}\left(w_{2};\tau_{1}\right)&=&\esp\left[w_{2}^{\hat{L}_{2}\left(\tau_{1}\right)}\right] =\sum_{k=0}^{\infty}\prob\left[\hat{L}_{2}\left(\tau_{1}\right)=k\right]w_{2}^{k}\\
%\hat{F}_{1,2}\left(w_{1};\tau_{2}\right)&=&\esp\left[w_{1}^{\hat{L}_{1}\left(\tau_{2}\right)}\right]
%=\sum_{k=0}^{\infty}\prob\left[\hat{L}_{1}\left(\tau_{2}\right)=k\right]w_{1}^{k}\\
%\hat{F}_{2,2}\left(w_{2};\tau_{2}\right)&=&\esp\left[w_{2}^{\hat{L}_{2}\left(\tau_{2}\right)}\right]
%=\sum_{k=0}^{\infty}\prob\left[\hat{L}_{2}\left(\tau_{2}\right)=k\right]w_{2}^{k}\\
\end{eqnarray*}


Ahora, con lo anterior definamos la FGP conjunta para el segundo sistema;% y $\tau_{1}$:


\begin{eqnarray*}
\esp\left[w_{1}^{\hat{L}_{1}\left(\tau_{j}\right)}w_{2}^{\hat{L}_{2}\left(\tau_{j}\right)}\right]
&=&\esp\left[w_{1}^{\hat{L}_{1}\left(\tau_{j}\right)}\right]
\esp\left[w_{2}^{\hat{L}_{2}\left(\tau_{j}\right)}\right]=\hat{F}_{1,j}\left(w_{1};\tau_{j}\right)\hat{F}_{2,j}\left(w_{2};\tau_{j}\right)=\hat{F}_{j}\left(w_{1},w_{2};\tau_{j}\right).\\
%\esp\left[w_{1}^{\hat{L}_{1}\left(\tau_{1}\right)}w_{2}^{\hat{L}_{2}\left(\tau_{1}\right)}\right]
%&=&\esp\left[w_{1}^{\hat{L}_{1}\left(\tau_{1}\right)}\right]
%\esp\left[w_{2}^{\hat{L}_{2}\left(\tau_{1}\right)}\right]=\hat{F}_{1,1}\left(w_{1};\tau_{1}\right)\hat{F}_{2,1}\left(w_{2};\tau_{1}\right)=\hat{F}_{1}\left(w_{1},w_{2};\tau_{1}\right)\\
%\esp\left[w_{1}^{\hat{L}_{1}\left(\tau_{2}\right)}w_{2}^{\hat{L}_{2}\left(\tau_{2}\right)}\right]
%&=&\esp\left[w_{1}^{\hat{L}_{1}\left(\tau_{2}\right)}\right]
%   \esp\left[w_{2}^{\hat{L}_{2}\left(\tau_{2}\right)}\right]=\hat{F}_{1,2}\left(w_{1};\tau_{2}\right)\hat{F}_{2,2}\left(w_{2};\tau_{2}\right)=\hat{F}_{2}\left(w_{1},w_{2};\tau_{2}\right).
\end{eqnarray*}

Con respecto al sistema 1 se tiene la FGP conjunta con respecto al servidor del otro sistema:
\begin{eqnarray*}
\esp\left[z_{1}^{L_{1}\left(\zeta_{j}\right)}z_{2}^{L_{2}\left(\zeta_{j}\right)}\right]
&=&\esp\left[z_{1}^{L_{1}\left(\zeta_{j}\right)}\right]
\esp\left[z_{2}^{L_{2}\left(\zeta_{j}\right)}\right]=F_{1,j}\left(z_{1};\zeta_{j}\right)F_{2,j}\left(z_{2};\zeta_{j}\right)=F_{j}\left(z_{1},z_{2};\zeta_{j}\right).
%\esp\left[z_{1}^{L_{1}\left(\zeta_{1}\right)}z_{2}^{L_{2}\left(\zeta_{1}\right)}\right]
%&=&\esp\left[z_{1}^{L_{1}\left(\zeta_{1}\right)}\right]
%\esp\left[z_{2}^{L_{2}\left(\zeta_{1}\right)}\right]=F_{1,1}\left(z_{1};\zeta_{1}\right)F_{2,1}\left(z_{2};\zeta_{1}\right)=F_{1}\left(z_{1},z_{2};\zeta_{1}\right)\\
%\esp\left[z_{1}^{L_{1}\left(\zeta_{2}\right)}z_{2}^{L_{2}\left(\zeta_{2}\right)}\right]
%&=&\esp\left[z_{1}^{L_{1}\left(\zeta_{2}\right)}\right]
%\esp\left[z_{2}^{L_{2}\left(\zeta_{2}\right)}\right]=F_{1,2}\left(z_{1};\zeta_{2}\right)F_{2,2}\left(z_{2};\zeta_{2}\right)=F_{2}\left(z_{1},z_{2};\zeta_{2}\right).
\end{eqnarray*}

Ahora analicemos la Red de Sistemas de Visitas C\'iclicas, entonces se define la PGF conjunta al tiempo $t$ para los tiempos de visita del servidor en cada una de las colas, para comenzar a dar servicio, definidos anteriormente al tiempo
$t=\left\{\tau_{1},\tau_{2},\zeta_{1},\zeta_{2}\right\}$:

\begin{eqnarray}\label{Eq.Conjuntas}
F_{j}\left(z_{1},z_{2},w_{1},w_{2}\right)&=&\esp\left[\prod_{i=1}^{2}z_{i}^{L_{i}\left(\tau_{j}
\right)}\prod_{i=1}^{2}w_{i}^{\hat{L}_{i}\left(\tau_{j}\right)}\right]\\
\hat{F}_{j}\left(z_{1},z_{2},w_{1},w_{2}\right)&=&\esp\left[\prod_{i=1}^{2}z_{i}^{L_{i}
\left(\zeta_{j}\right)}\prod_{i=1}^{2}w_{i}^{\hat{L}_{i}\left(\zeta_{j}\right)}\right]
\end{eqnarray}
para $j=1,2$. Entonces, con la finalidad de encontrar el n\'umero de usuarios
presentes en el sistema cuando el servidor deja de atender una de
las colas de cualquier sistema se tiene lo siguiente


\begin{eqnarray*}
&&\esp\left[z_{1}^{L_{1}\left(\overline{\tau}_{1}\right)}z_{2}^{L_{2}\left(\overline{\tau}_{1}\right)}w_{1}^{\hat{L}_{1}\left(\overline{\tau}_{1}\right)}w_{2}^{\hat{L}_{2}\left(\overline{\tau}_{1}\right)}\right]=
\esp\left[z_{2}^{L_{2}\left(\overline{\tau}_{1}\right)}w_{1}^{\hat{L}_{1}\left(\overline{\tau}_{1}
\right)}w_{2}^{\hat{L}_{2}\left(\overline{\tau}_{1}\right)}\right]\\
&=&\esp\left[z_{2}^{L_{2}\left(\tau_{1}\right)+X_{2}\left(\overline{\tau}_{1}-\tau_{1}\right)+Y_{2}\left(\overline{\tau}_{1}-\tau_{1}\right)}w_{1}^{\hat{L}_{1}\left(\tau_{1}\right)+\hat{X}_{1}\left(\overline{\tau}_{1}-\tau_{1}\right)}w_{2}^{\hat{L}_{2}\left(\tau_{1}\right)+\hat{X}_{2}\left(\overline{\tau}_{1}-\tau_{1}\right)}\right]
\end{eqnarray*}
utilizando la ecuacion dada (\ref{Eq.UsuariosTotalesZ2}), luego


\begin{eqnarray*}
&=&\esp\left[z_{2}^{L_{2}\left(\tau_{1}\right)}z_{2}^{X_{2}\left(\overline{\tau}_{1}-\tau_{1}\right)}z_{2}^{Y_{2}\left(\overline{\tau}_{1}-\tau_{1}\right)}w_{1}^{\hat{L}_{1}\left(\tau_{1}\right)}w_{1}^{\hat{X}_{1}\left(\overline{\tau}_{1}-\tau_{1}\right)}w_{2}^{\hat{L}_{2}\left(\tau_{1}\right)}w_{2}^{\hat{X}_{2}\left(\overline{\tau}_{1}-\tau_{1}\right)}\right]\\
&=&\esp\left[z_{2}^{L_{2}\left(\tau_{1}\right)}\left\{w_{1}^{\hat{L}_{1}\left(\tau_{1}\right)}w_{2}^{\hat{L}_{2}\left(\tau_{1}\right)}\right\}\left\{z_{2}^{X_{2}\left(\overline{\tau}_{1}-\tau_{1}\right)}
z_{2}^{Y_{2}\left(\overline{\tau}_{1}-\tau_{1}\right)}w_{1}^{\hat{X}_{1}\left(\overline{\tau}_{1}-\tau_{1}\right)}w_{2}^{\hat{X}_{2}\left(\overline{\tau}_{1}-\tau_{1}\right)}\right\}\right]\\
\end{eqnarray*}
Aplicando el hecho de que el n\'umero de usuarios que llegan a cada una de las colas del segundo sistema es independiente de las llegadas a las colas del primer sistema:

\begin{eqnarray*}
&=&\esp\left[z_{2}^{L_{2}\left(\tau_{1}\right)}\left\{z_{2}^{X_{2}\left(\overline{\tau}_{1}-\tau_{1}\right)}z_{2}^{Y_{2}\left(\overline{\tau}_{1}-\tau_{1}\right)}w_{1}^{\hat{X}_{1}\left(\overline{\tau}_{1}-\tau_{1}\right)}w_{2}^{\hat{X}_{2}\left(\overline{\tau}_{1}-\tau_{1}\right)}\right\}\right]\esp\left[w_{1}^{\hat{L}_{1}\left(\tau_{1}\right)}w_{2}^{\hat{L}_{2}\left(\tau_{1}\right)}\right]
\end{eqnarray*}
dado que los arribos a cada una de las colas son independientes, podemos separar la esperanza para los procesos de llegada a $Q_{1}$ y $Q_{2}$ al tiempo $\tau_{1}$, que es el tiempo en que el servidor visita a $Q_{1}$. Recordando que $\tilde{X}_{2}\left(z_{2}\right)=X_{2}\left(z_{2}\right)+Y_{2}\left(z_{2}\right)$ se tiene


\begin{eqnarray*}
&=&\esp\left[z_{2}^{L_{2}\left(\tau_{1}\right)}\left\{z_{2}^{\tilde{X}_{2}\left(\overline{\tau}_{1}-\tau_{1}\right)}w_{1}^{\hat{X}_{1}\left(\overline{\tau}_{1}-\tau_{1}\right)}w_{2}^{\hat{X}_{2}\left(\overline{\tau}_{1}-\tau_{1}\right)}\right\}\right]\esp\left[w_{1}^{\hat{L}_{1}\left(\tau_{1}\right)}w_{2}^{\hat{L}_{2}\left(\tau_{1}\right)}\right]\\
&=&\esp\left[z_{2}^{L_{2}\left(\tau_{1}\right)}\left\{\tilde{P}_{2}\left(z_{2}\right)^{\overline{\tau}_{1}-\tau_{1}}\hat{P}_{1}\left(w_{1}\right)^{\overline{\tau}_{1}-\tau_{1}}\hat{P}_{2}\left(w_{2}\right)^{\overline{\tau}_{1}-\tau_{1}}\right\}\right]\esp\left[w_{1}^{\hat{L}_{1}\left(\tau_{1}\right)}w_{2}^{\hat{L}_{2}\left(\tau_{1}\right)}\right]\\
&=&\esp\left[z_{2}^{L_{2}\left(\tau_{1}\right)}\left\{\tilde{P}_{2}\left(z_{2}\right)\hat{P}_{1}\left(w_{1}\right)\hat{P}_{2}\left(w_{2}\right)\right\}^{\overline{\tau}_{1}-\tau_{1}}\right]\esp\left[w_{1}^{\hat{L}_{1}\left(\tau_{1}\right)}w_{2}^{\hat{L}_{2}\left(\tau_{1}\right)}\right]\\
&=&\esp\left[z_{2}^{L_{2}\left(\tau_{1}\right)}\theta_{1}\left(\tilde{P}_{2}\left(z_{2}\right)\hat{P}_{1}\left(w_{1}\right)\hat{P}_{2}\left(w_{2}\right)\right)^{L_{1}\left(\tau_{1}\right)}\right]\esp\left[w_{1}^{\hat{L}_{1}\left(\tau_{1}\right)}w_{2}^{\hat{L}_{2}\left(\tau_{1}\right)}\right]\\
&=&F_{1}\left(\theta_{1}\left(\tilde{P}_{2}\left(z_{2}\right)\hat{P}_{1}\left(w_{1}\right)\hat{P}_{2}\left(w_{2}\right)\right),z{2}\right)\hat{F}_{1}\left(w_{1},w_{2};\tau_{1}\right)\\
&\equiv&
F_{1}\left(\theta_{1}\left(\tilde{P}_{2}\left(z_{2}\right)\hat{P}_{1}\left(w_{1}\right)\hat{P}_{2}\left(w_{2}\right)\right),z_{2},w_{1},w_{2}\right)
\end{eqnarray*}

Las igualdades anteriores son ciertas pues el n\'umero de usuarios
que llegan a $\hat{Q}_{2}$ durante el intervalo
$\left[\tau_{1},\overline{\tau}_{1}\right]$ a\'un no han sido
atendidos por el servidor del sistema $2$ y por tanto a\'un no
pueden pasar al sistema $1$ a traves de $Q_{2}$. Por tanto el n\'umero de
usuarios que pasan de $\hat{Q}_{2}$ a $Q_{2}$ en el intervalo de
tiempo $\left[\tau_{1},\overline{\tau}_{1}\right]$ depende de la
pol\'itica de traslado entre los dos sistemas, conforme a la
secci\'on anterior.\smallskip

Por lo tanto
\begin{eqnarray}\label{Eq.Fs}
\esp\left[z_{1}^{L_{1}\left(\overline{\tau}_{1}\right)}z_{2}^{L_{2}\left(\overline{\tau}_{1}
\right)}w_{1}^{\hat{L}_{1}\left(\overline{\tau}_{1}\right)}w_{2}^{\hat{L}_{2}\left(
\overline{\tau}_{1}\right)}\right]&=&F_{1}\left(\theta_{1}\left(\tilde{P}_{2}\left(z_{2}\right)
\hat{P}_{1}\left(w_{1}\right)\hat{P}_{2}\left(w_{2}\right)\right),z_{2},w_{1},w_{2}\right)\\
&=&F_{1}\left(\theta_{1}\left(\tilde{P}_{2}\left(z_{2}\right)\hat{P}_{1}\left(w_{1}\right)\hat{P}_{2}\left(w_{2}\right)\right),z{2}\right)\hat{F}_{1}\left(w_{1},w_{2};\tau_{1}\right)
\end{eqnarray}


Utilizando un razonamiento an\'alogo para $\overline{\tau}_{2}$:



\begin{eqnarray*}
&&\esp\left[z_{1}^{L_{1}\left(\overline{\tau}_{2}\right)}z_{2}^{L_{2}\left(\overline{\tau}_{2}\right)}w_{1}^{\hat{L}_{1}\left(\overline{\tau}_{2}\right)}w_{2}^{\hat{L}_{2}\left(\overline{\tau}_{2}\right)}\right]=
\esp\left[z_{1}^{L_{1}\left(\overline{\tau}_{2}\right)}w_{1}^{\hat{L}_{1}\left(\overline{\tau}_{2}\right)}w_{2}^{\hat{L}_{2}\left(\overline{\tau}_{2}\right)}\right]\\
&=&\esp\left[z_{1}^{L_{1}\left(\tau_{2}\right)+X_{1}\left(\overline{\tau}_{2}-\tau_{2}\right)}w_{1}^{\hat{L}_{1}\left(\tau_{2}\right)+\hat{X}_{1}\left(\overline{\tau}_{2}-\tau_{2}\right)}w_{2}^{\hat{L}_{2}\left(\tau_{2}\right)+\hat{X}_{2}\left(\overline{\tau}_{2}-\tau_{2}\right)}\right]\\
&=&\esp\left[z_{1}^{L_{1}\left(\tau_{2}\right)}z_{1}^{X_{1}\left(\overline{\tau}_{2}-\tau_{2}\right)}w_{1}^{\hat{L}_{1}\left(\tau_{2}\right)}w_{1}^{\hat{X}_{1}\left(\overline{\tau}_{2}-\tau_{2}\right)}w_{2}^{\hat{L}_{2}\left(\tau_{2}\right)}w_{2}^{\hat{X}_{2}\left(\overline{\tau}_{2}-\tau_{2}\right)}\right]\\
&=&\esp\left[z_{1}^{L_{1}\left(\tau_{2}\right)}z_{1}^{X_{1}\left(\overline{\tau}_{2}-\tau_{2}\right)}w_{1}^{\hat{X}_{1}\left(\overline{\tau}_{2}-\tau_{2}\right)}w_{2}^{\hat{X}_{2}\left(\overline{\tau}_{2}-\tau_{2}\right)}\right]\esp\left[w_{1}^{\hat{L}_{1}\left(\tau_{2}\right)}w_{2}^{\hat{L}_{2}\left(\tau_{2}\right)}\right]\\
&=&\esp\left[z_{1}^{L_{1}\left(\tau_{2}\right)}P_{1}\left(z_{1}\right)^{\overline{\tau}_{2}-\tau_{2}}\hat{P}_{1}\left(w_{1}\right)^{\overline{\tau}_{2}-\tau_{2}}\hat{P}_{2}\left(w_{2}\right)^{\overline{\tau}_{2}-\tau_{2}}\right]
\esp\left[w_{1}^{\hat{L}_{1}\left(\tau_{2}\right)}w_{2}^{\hat{L}_{2}\left(\tau_{2}\right)}\right]
\end{eqnarray*}
utlizando la proposici\'on relacionada con la ruina del jugador


\begin{eqnarray*}
&=&\esp\left[z_{1}^{L_{1}\left(\tau_{2}\right)}\left\{P_{1}\left(z_{1}\right)\hat{P}_{1}\left(w_{1}\right)\hat{P}_{2}\left(w_{2}\right)\right\}^{\overline{\tau}_{2}-\tau_{2}}\right]
\esp\left[w_{1}^{\hat{L}_{1}\left(\tau_{2}\right)}w_{2}^{\hat{L}_{2}\left(\tau_{2}\right)}\right]\\
&=&\esp\left[z_{1}^{L_{1}\left(\tau_{2}\right)}\tilde{\theta}_{2}\left(P_{1}\left(z_{1}\right)\hat{P}_{1}\left(w_{1}\right)\hat{P}_{2}\left(w_{2}\right)\right)^{L_{2}\left(\tau_{2}\right)}\right]
\esp\left[w_{1}^{\hat{L}_{1}\left(\tau_{2}\right)}w_{2}^{\hat{L}_{2}\left(\tau_{2}\right)}\right]\\
&=&F_{2}\left(z_{1},\tilde{\theta}_{2}\left(P_{1}\left(z_{1}\right)\hat{P}_{1}\left(w_{1}\right)\hat{P}_{2}\left(w_{2}\right)\right)\right)
\hat{F}_{2}\left(w_{1},w_{2};\tau_{2}\right)\\
\end{eqnarray*}


entonces se define
\begin{eqnarray}
\esp\left[z_{1}^{L_{1}\left(\overline{\tau}_{2}\right)}z_{2}^{L_{2}\left(\overline{\tau}_{2}\right)}w_{1}^{\hat{L}_{1}\left(\overline{\tau}_{2}\right)}w_{2}^{\hat{L}_{2}\left(\overline{\tau}_{2}\right)}\right]=F_{2}\left(z_{1},\tilde{\theta}_{2}\left(P_{1}\left(z_{1}\right)\hat{P}_{1}\left(w_{1}\right)\hat{P}_{2}\left(w_{2}\right)\right),w_{1},w_{2}\right)\\
\equiv F_{2}\left(z_{1},\tilde{\theta}_{2}\left(P_{1}\left(z_{1}\right)\hat{P}_{1}\left(w_{1}\right)\hat{P}_{2}\left(w_{2}\right)\right)\right)
\hat{F}_{2}\left(w_{1},w_{2};\tau_{2}\right)
\end{eqnarray}
Ahora para $\overline{\zeta}_{1}:$
\begin{eqnarray*}
&&\esp\left[z_{1}^{L_{1}\left(\overline{\zeta}_{1}\right)}z_{2}^{L_{2}\left(\overline{\zeta}_{1}\right)}w_{1}^{\hat{L}_{1}\left(\overline{\zeta}_{1}\right)}w_{2}^{\hat{L}_{2}\left(\overline{\zeta}_{1}\right)}\right]=
\esp\left[z_{1}^{L_{1}\left(\overline{\zeta}_{1}\right)}z_{2}^{L_{2}\left(\overline{\zeta}_{1}\right)}w_{2}^{\hat{L}_{2}\left(\overline{\zeta}_{1}\right)}\right]\\
%&=&\esp\left[z_{1}^{L_{1}\left(\zeta_{1}\right)+X_{1}\left(\overline{\zeta}_{1}-\zeta_{1}\right)}z_{2}^{L_{2}\left(\zeta_{1}\right)+X_{2}\left(\overline{\zeta}_{1}-\zeta_{1}\right)+\hat{Y}_{2}\left(\overline{\zeta}_{1}-\zeta_{1}\right)}w_{2}^{\hat{L}_{2}\left(\zeta_{1}\right)+\hat{X}_{2}\left(\overline{\zeta}_{1}-\zeta_{1}\right)}\right]\\
&=&\esp\left[z_{1}^{L_{1}\left(\zeta_{1}\right)}z_{1}^{X_{1}\left(\overline{\zeta}_{1}-\zeta_{1}\right)}z_{2}^{L_{2}\left(\zeta_{1}\right)}z_{2}^{X_{2}\left(\overline{\zeta}_{1}-\zeta_{1}\right)}
z_{2}^{Y_{2}\left(\overline{\zeta}_{1}-\zeta_{1}\right)}w_{2}^{\hat{L}_{2}\left(\zeta_{1}\right)}w_{2}^{\hat{X}_{2}\left(\overline{\zeta}_{1}-\zeta_{1}\right)}\right]\\
&=&\esp\left[z_{1}^{L_{1}\left(\zeta_{1}\right)}z_{2}^{L_{2}\left(\zeta_{1}\right)}\right]\esp\left[z_{1}^{X_{1}\left(\overline{\zeta}_{1}-\zeta_{1}\right)}z_{2}^{\tilde{X}_{2}\left(\overline{\zeta}_{1}-\zeta_{1}\right)}w_{2}^{\hat{X}_{2}\left(\overline{\zeta}_{1}-\zeta_{1}\right)}w_{2}^{\hat{L}_{2}\left(\zeta_{1}\right)}\right]\\
&=&\esp\left[z_{1}^{L_{1}\left(\zeta_{1}\right)}z_{2}^{L_{2}\left(\zeta_{1}\right)}\right]
\esp\left[P_{1}\left(z_{1}\right)^{\overline{\zeta}_{1}-\zeta_{1}}\tilde{P}_{2}\left(z_{2}\right)^{\overline{\zeta}_{1}-\zeta_{1}}\hat{P}_{2}\left(w_{2}\right)^{\overline{\zeta}_{1}-\zeta_{1}}w_{2}^{\hat{L}_{2}\left(\zeta_{1}\right)}\right]\\
&=&\esp\left[z_{1}^{L_{1}\left(\zeta_{1}\right)}z_{2}^{L_{2}\left(\zeta_{1}\right)}\right]
\esp\left[\left\{P_{1}\left(z_{1}\right)\tilde{P}_{2}\left(z_{2}\right)\hat{P}_{2}\left(w_{2}\right)\right\}^{\overline{\zeta}_{1}-\zeta_{1}}w_{2}^{\hat{L}_{2}\left(\zeta_{1}\right)}\right]\\
&=&\esp\left[z_{1}^{L_{1}\left(\zeta_{1}\right)}z_{2}^{L_{2}\left(\zeta_{1}\right)}\right]
\esp\left[\hat{\theta}_{1}\left(P_{1}\left(z_{1}\right)\tilde{P}_{2}\left(z_{2}\right)\hat{P}_{2}\left(w_{2}\right)\right)^{\hat{L}_{1}\left(\zeta_{1}\right)}w_{2}^{\hat{L}_{2}\left(\zeta_{1}\right)}\right]\\
&=&F_{1}\left(z_{1},z_{2};\zeta_{1}\right)\hat{F}_{1}\left(\hat{\theta}_{1}\left(P_{1}\left(z_{1}\right)\tilde{P}_{2}\left(z_{2}\right)\hat{P}_{2}\left(w_{2}\right)\right),w_{2}\right)
\end{eqnarray*}


es decir
\begin{eqnarray}
\esp\left[z_{1}^{L_{1}\left(\overline{\zeta}_{1}\right)}z_{2}^{L_{2}\left(\overline{\zeta}_{1}
\right)}w_{1}^{\hat{L}_{1}\left(\overline{\zeta}_{1}\right)}w_{2}^{\hat{L}_{2}\left(
\overline{\zeta}_{1}\right)}\right]&=&\hat{F}_{1}\left(z_{1},z_{2},\hat{\theta}_{1}\left(P_{1}\left(z_{1}\right)\tilde{P}_{2}\left(z_{2}\right)\hat{P}_{2}\left(w_{2}\right)\right),w_{2}\right)\\
&=&F_{1}\left(z_{1},z_{2};\zeta_{1}\right)\hat{F}_{1}\left(\hat{\theta}_{1}\left(P_{1}\left(z_{1}\right)\tilde{P}_{2}\left(z_{2}\right)\hat{P}_{2}\left(w_{2}\right)\right),w_{2}\right).
\end{eqnarray}


Finalmente para $\overline{\zeta}_{2}:$
\begin{eqnarray*}
&&\esp\left[z_{1}^{L_{1}\left(\overline{\zeta}_{2}\right)}z_{2}^{L_{2}\left(\overline{\zeta}_{2}\right)}w_{1}^{\hat{L}_{1}\left(\overline{\zeta}_{2}\right)}w_{2}^{\hat{L}_{2}\left(\overline{\zeta}_{2}\right)}\right]=
\esp\left[z_{1}^{L_{1}\left(\overline{\zeta}_{2}\right)}z_{2}^{L_{2}\left(\overline{\zeta}_{2}\right)}w_{1}^{\hat{L}_{1}\left(\overline{\zeta}_{2}\right)}\right]\\
%&=&\esp\left[z_{1}^{L_{1}\left(\zeta_{2}\right)+X_{1}\left(\overline{\zeta}_{2}-\zeta_{2}\right)}z_{2}^{L_{2}\left(\zeta_{2}\right)+X_{2}\left(\overline{\zeta}_{2}-\zeta_{2}\right)+\hat{Y}_{2}\left(\overline{\zeta}_{2}-\zeta_{2}\right)}w_{1}^{\hat{L}_{1}\left(\zeta_{2}\right)+\hat{X}_{1}\left(\overline{\zeta}_{2}-\zeta_{2}\right)}\right]\\
&=&\esp\left[z_{1}^{L_{1}\left(\zeta_{2}\right)}z_{1}^{X_{1}\left(\overline{\zeta}_{2}-\zeta_{2}\right)}z_{2}^{L_{2}\left(\zeta_{2}\right)}z_{2}^{X_{2}\left(\overline{\zeta}_{2}-\zeta_{2}\right)}
z_{2}^{Y_{2}\left(\overline{\zeta}_{2}-\zeta_{2}\right)}w_{1}^{\hat{L}_{1}\left(\zeta_{2}\right)}w_{1}^{\hat{X}_{1}\left(\overline{\zeta}_{2}-\zeta_{2}\right)}\right]\\
&=&\esp\left[z_{1}^{L_{1}\left(\zeta_{2}\right)}z_{2}^{L_{2}\left(\zeta_{2}\right)}\right]\esp\left[z_{1}^{X_{1}\left(\overline{\zeta}_{2}-\zeta_{2}\right)}z_{2}^{\tilde{X}_{2}\left(\overline{\zeta}_{2}-\zeta_{2}\right)}w_{1}^{\hat{X}_{1}\left(\overline{\zeta}_{2}-\zeta_{2}\right)}w_{1}^{\hat{L}_{1}\left(\zeta_{2}\right)}\right]\\
&=&\esp\left[z_{1}^{L_{1}\left(\zeta_{2}\right)}z_{2}^{L_{2}\left(\zeta_{2}\right)}\right]\esp\left[P_{1}\left(z_{1}\right)^{\overline{\zeta}_{2}-\zeta_{2}}\tilde{P}_{2}\left(z_{2}\right)^{\overline{\zeta}_{2}-\zeta_{2}}\hat{P}\left(w_{1}\right)^{\overline{\zeta}_{2}-\zeta_{2}}w_{1}^{\hat{L}_{1}\left(\zeta_{2}\right)}\right]\\
&=&\esp\left[z_{1}^{L_{1}\left(\zeta_{2}\right)}z_{2}^{L_{2}\left(\zeta_{2}\right)}\right]\esp\left[w_{1}^{\hat{L}_{1}\left(\zeta_{2}\right)}\left\{P_{1}\left(z_{1}\right)\tilde{P}_{2}\left(z_{2}\right)\hat{P}\left(w_{1}\right)\right\}^{\overline{\zeta}_{2}-\zeta_{2}}\right]\\
&=&\esp\left[z_{1}^{L_{1}\left(\zeta_{2}\right)}z_{2}^{L_{2}\left(\zeta_{2}\right)}\right]\esp\left[w_{1}^{\hat{L}_{1}\left(\zeta_{2}\right)}\hat{\theta}_{2}\left(P_{1}\left(z_{1}\right)\tilde{P}_{2}\left(z_{2}\right)\hat{P}\left(w_{1}\right)\right)^{\hat{L}_{2}\zeta_{2}}\right]\\
&=&F_{2}\left(z_{1},z_{2};\zeta_{2}\right)\hat{F}_{2}\left(w_{1},\hat{\theta}_{2}\left(P_{1}\left(z_{1}\right)\tilde{P}_{2}\left(z_{2}\right)\hat{P}_{1}\left(w_{1}\right)\right)\right]\\
%&\equiv&\hat{F}_{2}\left(z_{1},z_{2},w_{1},\hat{\theta}_{2}\left(P_{1}\left(z_{1}\right)\tilde{P}_{2}\left(z_{2}\right)\hat{P}_{1}\left(w_{1}\right)\right)\right)
\end{eqnarray*}

es decir
\begin{eqnarray}
\esp\left[z_{1}^{L_{1}\left(\overline{\zeta}_{2}\right)}z_{2}^{L_{2}\left(\overline{\zeta}_{2}\right)}w_{1}^{\hat{L}_{1}\left(\overline{\zeta}_{2}\right)}w_{2}^{\hat{L}_{2}\left(\overline{\zeta}_{2}\right)}\right]=\hat{F}_{2}\left(z_{1},z_{2},w_{1},\hat{\theta}_{2}\left(P_{1}\left(z_{1}\right)\tilde{P}_{2}\left(z_{2}\right)\hat{P}_{1}\left(w_{1}\right)\right)\right)\\
=F_{2}\left(z_{1},z_{2};\zeta_{2}\right)\hat{F}_{2}\left(w_{1},\hat{\theta}_{2}\left(P_{1}\left(z_{1}\right)\tilde{P}_{2}\left(z_{2}\right)\hat{P}_{1}\left(w_{1}
\right)\right)\right)
\end{eqnarray}
%__________________________________________________________________________
\section{Ecuaciones Recursivas para la R.S.V.C.}
%__________________________________________________________________________




Con lo desarrollado hasta ahora podemos encontrar las ecuaciones
recursivas que modelan la Red de Sistemas de Visitas C\'iclicas
(R.S.V.C):
\begin{eqnarray*}
&&F_{2}\left(z_{1},z_{2},w_{1},w_{2}\right)=R_{1}\left(z_{1},z_{2},w_{1},w_{2}\right)\esp\left[z_{1}^{L_{1}\left(\overline{\tau}_{1}\right)}z_{2}^{L_{2}\left(\overline{\tau}_{1}\right)}w_{1}^{\hat{L}_{1}\left(\overline{\tau}_{1}\right)}w_{2}^{\hat{L}_{2}\left(\overline{\tau}_{1}\right)}\right]\\
&&F_{1}\left(z_{1},z_{2},w_{1},w_{2}\right)=R_{2}\left(z_{1},z_{2},w_{1},w_{2}\right)\esp\left[z_{1}^{L_{1}\left(\overline{\tau}_{2}\right)}z_{2}^{L_{2}\left(\overline{\tau}_{2}\right)}w_{1}^{\hat{L}_{1}\left(\overline{\tau}_{2}\right)}w_{2}^{\hat{L}_{2}\left(\overline{\tau}_{1}\right)}\right]\\
&&\hat{F}_{2}\left(z_{1},z_{2},w_{1},w_{2}\right)=\hat{R}_{1}\left(z_{1},z_{2},w_{1},w_{2}\right)\esp\left[z_{1}^{L_{1}\left(\overline{\zeta}_{1}\right)}z_{2}^{L_{2}\left(\overline{\zeta}_{1}\right)}w_{1}^{\hat{L}_{1}\left(\overline{\zeta}_{1}\right)}w_{2}^{\hat{L}_{2}\left(\overline{\zeta}_{1}\right)}\right]\\
&&\hat{F}_{1}\left(z_{1},z_{2},w_{1},w_{2}\right)=\hat{R}_{2}\left(z_{1},z_{2},
w_{1},w_{2}\right)\esp\left[z_{1}^{L_{1}\left(\overline{\zeta}_{2}\right)}z_{2}
^{L_{2}\left(\overline{\zeta}_{2}\right)}w_{1}^{\hat{L}_{1}\left(
\overline{\zeta}_{2}\right)}w_{2}^{\hat{L}_{2}\left(\overline{\zeta}_{2}\right)}
\right]
\end{eqnarray*}

%&=&R_{1}\left(P_{1}\left(z_{1}\right)\tilde{P}_{2}\left(z_{2}\right)\hat{P}_{1}\left(w_{1}\right)\hat{P}_{2}\left(w_{2}\right)\right)
%F_{1}\left(\theta\left(\tilde{P}_{2}\left(z_{2}\right)\hat{P}_{1}\left(w_{1}\right)\hat{P}_{2}\left(w_{2}\right)\right),z_{2},w_{1},w_{2}\right)\\
%&=&R_{2}\left(P_{1}\left(z_{1}\right)\tilde{P}_{2}\left(z_{2}\right)\hat{P}_{1}\left(w_{1}\right)\hat{P}_{2}\left(w_{2}\right)\right)F_{2}\left(z_{1},\tilde{\theta}_{2}\left(P_{1}\left(z_{1}\right)\hat{P}_{1}\left(w_{1}\right)\hat{P}_{2}\left(w_{2}\right)\right),w_{1},w_{2}\right)\\
%&=&\hat{R}_{1}\left(P_{1}\left(z_{1}\right)\tilde{P}_{2}\left(z_{2}\right)\hat{P}_{1}\left(w_{1}\right)\hat{P}_{2}\left(w_{2}\right)\right)\hat{F}_{1}\left(z_{1},z_{2},\hat{\theta}_{1}\left(P_{1}\left(z_{1}\right)\tilde{P}_{2}\left(z_{2}\right)\hat{P}_{2}\left(w_{2}\right)\right),w_{2}\right)
%&=&\hat{R}_{2}\left(P_{1}\left(z_{1}\right)\tilde{P}_{2}\left(z_{2}\right)\hat{P}_{1}\left(w_{1}\right)\hat{P}_{2}\left(w_{2}\right)\right)\hat{F}_{2}\left(z_{1},z_{2},w_{1},\hat{\theta}_{2}\left(P_{1}\left(z_{1}\right)\tilde{P}_{2}\left(z_{2}\right)\hat{P}_{1}\left(w_{1}\right)\right)\right)


que son equivalentes a las siguientes ecuaciones
\begin{eqnarray}
F_{2}\left(z_{1},z_{2},w_{1},w_{2}\right)&=&R_{1}\left(P_{1}\left(z_{1}\right)\tilde{P}_{2}\left(z_{2}\right)\prod_{i=1}^{2}
\hat{P}_{i}\left(w_{i}\right)\right)F_{1}\left(\theta_{1}\left(\tilde{P}_{2}\left(z_{2}\right)\hat{P}_{1}\left(w_{1}\right)\hat{P}_{2}\left(w_{2}\right)\right),z_{2},w_{1},w_{2}\right)\\
F_{1}\left(z_{1},z_{2},w_{1},w_{2}\right)&=&R_{2}\left(P_{1}\left(z_{1}\right)\tilde{P}_{2}\left(z_{2}\right)\prod_{i=1}^{2}
\hat{P}_{i}\left(w_{i}\right)\right)F_{2}\left(z_{1},\tilde{\theta}_{2}\left(P_{1}\left(z_{1}\right)\hat{P}_{1}\left(w_{1}\right)\hat{P}_{2}\left(w_{2}\right)\right),w_{1},w_{2}\right)\\
\hat{F}_{2}\left(z_{1},z_{2},w_{1},w_{2}\right)&=&\hat{R}_{1}\left(P_{1}\left(z_{1}\right)\tilde{P}_{2}\left(z_{2}\right)\prod_{i=1}^{2}
\hat{P}_{i}\left(w_{i}\right)\right)\hat{F}_{1}\left(z_{1},z_{2},\hat{\theta}_{1}\left(P_{1}\left(z_{1}\right)\tilde{P}_{2}\left(z_{2}\right)\hat{P}_{2}\left(w_{2}\right)\right),w_{2}\right)\\
\hat{F}_{1}\left(z_{1},z_{2},w_{1},w_{2}\right)&=&\hat{R}_{2}\left(P_{1}\left(z_{1}\right)\tilde{P}_{2}\left(z_{2}\right)\prod_{i=1}^{2}
\hat{P}_{i}\left(w_{i}\right)\right)\hat{F}_{2}\left(z_{1},z_{2},w_{1},\hat{\theta}_{2}\left(P_{1}\left(z_{1}\right)\tilde{P}_{2}\left(z_{2}\right)
\hat{P}_{1}\left(w_{1}\right)\right)\right)
\end{eqnarray}



%_________________________________________________________________________________________________
\subsection{Tiempos de Traslado del Servidor}
%_________________________________________________________________________________________________


Para
%\begin{multicols}{2}

\begin{eqnarray}\label{Ec.R1}
R_{1}\left(\mathbf{z,w}\right)=R_{1}\left((P_{1}\left(z_{1}\right)\tilde{P}_{2}\left(z_{2}\right)\hat{P}_{1}\left(w_{1}\right)\hat{P}_{2}\left(w_{2}\right)\right)
\end{eqnarray}
%\end{multicols}

se tiene que


\begin{eqnarray*}
\begin{array}{cc}
\frac{\partial R_{1}\left(\mathbf{z,w}\right)}{\partial
z_{1}}|_{\mathbf{z,w}=1}=R_{1}^{(1)}\left(1\right)P_{1}^{(1)}\left(1\right)=r_{1}\mu_{1},&
\frac{\partial R_{1}\left(\mathbf{z,w}\right)}{\partial
z_{2}}|_{\mathbf{z,w}=1}=R_{1}^{(1)}\left(1\right)\tilde{P}_{2}^{(1)}\left(1\right)=r_{1}\tilde{\mu}_{2},\\
\frac{\partial R_{1}\left(\mathbf{z,w}\right)}{\partial
w_{1}}|_{\mathbf{z,w}=1}=R_{1}^{(1)}\left(1\right)\hat{P}_{1}^{(1)}\left(1\right)=r_{1}\hat{\mu}_{1},&
\frac{\partial R_{1}\left(\mathbf{z,w}\right)}{\partial
w_{2}}|_{\mathbf{z,w}=1}=R_{1}^{(1)}\left(1\right)\hat{P}_{2}^{(1)}\left(1\right)=r_{1}\hat{\mu}_{2},
\end{array}
\end{eqnarray*}

An\'alogamente se tiene

\begin{eqnarray}
R_{2}\left(\mathbf{z,w}\right)=R_{2}\left(P_{1}\left(z_{1}\right)\tilde{P}_{2}\left(z_{2}\right)\hat{P}_{1}\left(w_{1}\right)\hat{P}_{2}\left(w_{2}\right)\right)
\end{eqnarray}


\begin{eqnarray*}
\begin{array}{cc}
\frac{\partial R_{2}\left(\mathbf{z,w}\right)}{\partial
z_{1}}|_{\mathbf{z,w}=1}=R_{2}^{(1)}\left(1\right)P_{1}^{(1)}\left(1\right)=r_{2}\mu_{1},&
\frac{\partial R_{2}\left(\mathbf{z,w}\right)}{\partial
z_{2}}|_{\mathbf{z,w}=1}=R_{2}^{(1)}\left(1\right)\tilde{P}_{2}^{(1)}\left(1\right)=r_{2}\tilde{\mu}_{2},\\
\frac{\partial R_{2}\left(\mathbf{z,w}\right)}{\partial
w_{1}}|_{\mathbf{z,w}=1}=R_{2}^{(1)}\left(1\right)\hat{P}_{1}^{(1)}\left(1\right)=r_{2}\hat{\mu}_{1},&
\frac{\partial R_{2}\left(\mathbf{z,w}\right)}{\partial
w_{2}}|_{\mathbf{z,w}=1}=R_{2}^{(1)}\left(1\right)\hat{P}_{2}^{(1)}\left(1\right)=r_{2}\hat{\mu}_{2},\\
\end{array}
\end{eqnarray*}

Para el segundo sistema:

\begin{eqnarray}
\hat{R}_{1}\left(\mathbf{z,w}\right)=\hat{R}_{1}\left(P_{1}\left(z_{1}\right)\tilde{P}_{2}\left(z_{2}\right)\hat{P}_{1}\left(w_{1}\right)\hat{P}_{2}\left(w_{2}\right)\right)
\end{eqnarray}


\begin{eqnarray*}
\frac{\partial \hat{R}_{1}\left(\mathbf{z,w}\right)}{\partial
z_{1}}|_{\mathbf{z,w}=1}=\hat{R}_{1}^{(1)}\left(1\right)P_{1}^{(1)}\left(1\right)=\hat{r}_{1}\mu_{1},&
\frac{\partial \hat{R}_{1}\left(\mathbf{z,w}\right)}{\partial
z_{2}}|_{\mathbf{z,w}=1}=\hat{R}_{1}^{(1)}\left(1\right)\tilde{P}_{2}^{(1)}\left(1\right)=\hat{r}_{1}\tilde{\mu}_{2},\\
\frac{\partial \hat{R}_{1}\left(\mathbf{z,w}\right)}{\partial
w_{1}}|_{\mathbf{z,w}=1}=\hat{R}_{1}^{(1)}\left(1\right)\hat{P}_{1}^{(1)}\left(1\right)=\hat{r}_{1}\hat{\mu}_{1},&
\frac{\partial \hat{R}_{1}\left(\mathbf{z,w}\right)}{\partial
w_{2}}|_{\mathbf{z,w}=1}=\hat{R}_{1}^{(1)}\left(1\right)\hat{P}_{2}^{(1)}\left(1\right)=\hat{r}_{1}\hat{\mu}_{2},
\end{eqnarray*}

Finalmente

\begin{eqnarray}
\hat{R}_{2}\left(\mathbf{z,w}\right)=\hat{R}_{2}\left(P_{1}\left(z_{1}\right)\tilde{P}_{2}\left(z_{2}\right)\hat{P}_{1}\left(w_{1}\right)\hat{P}_{2}\left(w_{2}\right)\right)
\end{eqnarray}



\begin{eqnarray*}
\frac{\partial \hat{R}_{2}\left(\mathbf{z,w}\right)}{\partial
z_{1}}|_{\mathbf{z,w}=1}=\hat{R}_{2}^{(1)}\left(1\right)P_{1}^{(1)}\left(1\right)=\hat{r}_{2}\mu_{1},&
\frac{\partial \hat{R}_{2}\left(\mathbf{z,w}\right)}{\partial
z_{2}}|_{\mathbf{z,w}=1}=\hat{R}_{2}^{(1)}\left(1\right)\tilde{P}_{2}^{(1)}\left(1\right)=\hat{r}_{2}\tilde{\mu}_{2},\\
\frac{\partial \hat{R}_{2}\left(\mathbf{z,w}\right)}{\partial
w_{1}}|_{\mathbf{z,w}=1}=\hat{R}_{2}^{(1)}\left(1\right)\hat{P}_{1}^{(1)}\left(1\right)=\hat{r}_{2}\hat{\mu}_{1},&
\frac{\partial \hat{R}_{2}\left(\mathbf{z,w}\right)}{\partial
w_{2}}|_{\mathbf{z,w}=1}=\hat{R}_{2}^{(1)}\left(1\right)\hat{P}_{2}^{(1)}\left(1\right)
=\hat{r}_{2}\hat{\mu}_{2}.
\end{eqnarray*}


%_________________________________________________________________________________________________
\subsection{Usuarios presentes en la cola}
%_________________________________________________________________________________________________

Hagamos lo correspondiente con las siguientes
expresiones obtenidas en la secci\'on anterior:
Recordemos que

\begin{eqnarray*}
F_{1}\left(\theta_{1}\left(\tilde{P}_{2}\left(z_{2}\right)\hat{P}_{1}\left(w_{1}\right)
\hat{P}_{2}\left(w_{2}\right)\right),z_{2},w_{1},w_{2}\right)=
F_{1}\left(\theta_{1}\left(\tilde{P}_{2}\left(z_{2}\right)\hat{P}_{1}\left(w_{1}
\right)\hat{P}_{2}\left(w_{2}\right)\right),z_{2}\right)
\hat{F}_{1}\left(w_{1},w_{2};\tau_{1}\right)
\end{eqnarray*}

entonces

\begin{eqnarray*}
\frac{\partial F_{1}\left(\theta_{1}\left(\tilde{P}_{2}\left(z_{2}\right)\hat{P}_{1}\left(w_{1}\right)\hat{P}_{2}\left(w_{2}\right)\right),z_{2},w_{1},w_{2}\right)}{\partial z_{1}}|_{\mathbf{z},\mathbf{w}=1}&=&0\\
\frac{\partial
F_{1}\left(\theta_{1}\left(\tilde{P}_{2}\left(z_{2}\right)\hat{P}_{1}\left(w_{1}\right)\hat{P}_{2}\left(w_{2}\right)\right),z_{2},w_{1},w_{2}\right)}{\partial
z_{2}}|_{\mathbf{z},\mathbf{w}=1}&=&\frac{\partial F_{1}}{\partial
z_{1}}\cdot\frac{\partial \theta_{1}}{\partial
\tilde{P}_{2}}\cdot\frac{\partial \tilde{P}_{2}}{\partial
z_{2}}+\frac{\partial F_{1}}{\partial z_{2}}
\\
\frac{\partial
F_{1}\left(\theta_{1}\left(\tilde{P}_{2}\left(z_{2}\right)\hat{P}_{1}\left(w_{1}\right)\hat{P}_{2}\left(w_{2}\right)\right),z_{2},w_{1},w_{2}\right)}{\partial
w_{1}}|_{\mathbf{z},\mathbf{w}=1}&=&\frac{\partial F_{1}}{\partial
z_{1}}\cdot\frac{\partial
\theta_{1}}{\partial\hat{P}_{1}}\cdot\frac{\partial\hat{P}_{1}}{\partial
w_{1}}+\frac{\partial\hat{F}_{1}}{\partial w_{1}}
\\
\frac{\partial
F_{1}\left(\theta_{1}\left(\tilde{P}_{2}\left(z_{2}\right)\hat{P}_{1}\left(w_{1}\right)\hat{P}_{2}\left(w_{2}\right)\right),z_{2},w_{1},w_{2}\right)}{\partial
w_{2}}|_{\mathbf{z},\mathbf{w}=1}&=&\frac{\partial F_{1}}{\partial
z_{1}}\cdot\frac{\partial\theta_{1}}{\partial\hat{P}_{2}}\cdot\frac{\partial\hat{P}_{2}}{\partial
w_{2}}+\frac{\partial \hat{F}_{1}}{\partial w_{2}}
\\
\end{eqnarray*}

para $\tau_{2}$:

\begin{eqnarray*}
F_{2}\left(z_{1},\tilde{\theta}_{2}\left(P_{1}\left(z_{1}\right)\hat{P}_{1}\left(w_{1}\right)\hat{P}_{2}\left(w_{2}\right)\right),
w_{1},w_{2}\right)=F_{2}\left(z_{1},\tilde{\theta}_{2}\left(P_{1}\left(z_{1}\right)\hat{P}_{1}\left(w_{1}\right)
\hat{P}_{2}\left(w_{2}\right)\right)\right)\hat{F}_{2}\left(w_{1},w_{2};\tau_{2}\right)
\end{eqnarray*}
al igual que antes

\begin{eqnarray*}
\frac{\partial
F_{2}\left(z_{1},\tilde{\theta}_{2}\left(P_{1}\left(z_{1}\right)\hat{P}_{1}\left(w_{1}\right)\hat{P}_{2}\left(w_{2}\right)\right),w_{1},w_{2}\right)}{\partial
z_{1}}|_{\mathbf{z},\mathbf{w}=1}&=&\frac{\partial F_{2}}{\partial
z_{2}}\cdot\frac{\partial\tilde{\theta}_{2}}{\partial
P_{1}}\cdot\frac{\partial P_{1}}{\partial z_{2}}+\frac{\partial
F_{2}}{\partial z_{1}}
\\
\frac{\partial F_{2}\left(z_{1},\tilde{\theta}_{2}\left(P_{1}\left(z_{1}\right)\hat{P}_{1}\left(w_{1}\right)\hat{P}_{2}\left(w_{2}\right)\right),w_{1},w_{2}\right)}{\partial z_{2}}|_{\mathbf{z},\mathbf{w}=1}&=&0\\
\frac{\partial
F_{2}\left(z_{1},\tilde{\theta}_{2}\left(P_{1}\left(z_{1}\right)\hat{P}_{1}\left(w_{1}\right)\hat{P}_{2}\left(w_{2}\right)\right),w_{1},w_{2}\right)}{\partial
w_{1}}|_{\mathbf{z},\mathbf{w}=1}&=&\frac{\partial F_{2}}{\partial
z_{2}}\cdot\frac{\partial \tilde{\theta}_{2}}{\partial
\hat{P}_{1}}\cdot\frac{\partial \hat{P}_{1}}{\partial
w_{1}}+\frac{\partial \hat{F}_{2}}{\partial w_{1}}
\\
\frac{\partial
F_{2}\left(z_{1},\tilde{\theta}_{2}\left(P_{1}\left(z_{1}\right)\hat{P}_{1}\left(w_{1}\right)\hat{P}_{2}\left(w_{2}\right)\right),w_{1},w_{2}\right)}{\partial
w_{2}}|_{\mathbf{z},\mathbf{w}=1}&=&\frac{\partial F_{2}}{\partial
z_{2}}\cdot\frac{\partial
\tilde{\theta}_{2}}{\partial\hat{P}_{2}}\cdot\frac{\partial\hat{P}_{2}}{\partial
w_{2}}+\frac{\partial\hat{F}_{2}}{\partial w_{2}}
\\
\end{eqnarray*}


Ahora para el segundo sistema

\begin{eqnarray*}\hat{F}_{1}\left(z_{1},z_{2},\hat{\theta}_{1}\left(P_{1}\left(z_{1}\right)\tilde{P}_{2}\left(z_{2}\right)\hat{P}_{2}\left(w_{2}\right)\right),
w_{2}\right)=F_{1}\left(z_{1},z_{2};\zeta_{1}\right)\hat{F}_{1}\left(\hat{\theta}_{1}\left(P_{1}\left(z_{1}\right)\tilde{P}_{2}\left(z_{2}\right)
\hat{P}_{2}\left(w_{2}\right)\right),w_{2}\right)
\end{eqnarray*}
entonces


\begin{eqnarray*}
\frac{\partial
\hat{F}_{1}\left(z_{1},z_{2},\hat{\theta}_{1}\left(P_{1}\left(z_{1}\right)\tilde{P}_{2}\left(z_{2}\right)\hat{P}_{2}\left(w_{2}\right)\right),w_{2}\right)}{\partial
z_{1}}|_{\mathbf{z},\mathbf{w}=1}&=&\frac{\partial \hat{F}_{1}
}{\partial w_{1}}\cdot\frac{\partial\hat{\theta}_{1}}{\partial
P_{1}}\cdot\frac{\partial P_{1}}{\partial z_{1}}+\frac{\partial
F_{1}}{\partial z_{1}}
\\
\frac{\partial
\hat{F}_{1}\left(z_{1},z_{2},\hat{\theta}_{1}\left(P_{1}\left(z_{1}\right)\tilde{P}_{2}\left(z_{2}\right)\hat{P}_{2}\left(w_{2}\right)\right),w_{2}\right)}{\partial
z_{2}}|_{\mathbf{z},\mathbf{w}=1}&=&\frac{\partial
\hat{F}_{1}}{\partial
w_{1}}\cdot\frac{\partial\hat{\theta}_{1}}{\partial\tilde{P}_{2}}\cdot\frac{\partial\tilde{P}_{2}}{\partial
z_{2}}+\frac{\partial F_{1}}{\partial z_{2}}
\\
\frac{\partial \hat{F}_{1}\left(z_{1},z_{2},\hat{\theta}_{1}\left(P_{1}\left(z_{1}\right)\tilde{P}_{2}\left(z_{2}\right)\hat{P}_{2}\left(w_{2}\right)\right),w_{2}\right)}{\partial w_{1}}|_{\mathbf{z},\mathbf{w}=1}&=&0\\
\frac{\partial \hat{F}_{1}\left(z_{1},z_{2},\hat{\theta}_{1}\left(P_{1}\left(z_{1}\right)\tilde{P}_{2}\left(z_{2}\right)\hat{P}_{2}\left(w_{2}\right)\right),w_{2}\right)}{\partial w_{2}}|_{\mathbf{z},\mathbf{w}=1}&=&\frac{\partial\hat{F}_{1}}{\partial w_{1}}\cdot\frac{\partial\hat{\theta}_{1}}{\partial\hat{P}_{2}}\cdot\frac{\partial\hat{P}_{2}}{\partial w_{2}}+\frac{\partial \hat{F}_{1}}{\partial w_{2}}\\
\end{eqnarray*}



Finalmente para $\zeta_{2}$

\begin{eqnarray*}\hat{F}_{2}\left(z_{1},z_{2},w_{1},\hat{\theta}_{2}\left(P_{1}\left(z_{1}\right)\tilde{P}_{2}\left(z_{2}\right)\hat{P}_{1}\left(w_{1}\right)\right)\right)&=&F_{2}\left(z_{1},z_{2};\zeta_{2}\right)\hat{F}_{2}\left(w_{1},\hat{\theta}_{2}\left(P_{1}\left(z_{1}\right)\tilde{P}_{2}\left(z_{2}\right)\hat{P}_{1}\left(w_{1}\right)\right)\right]
\end{eqnarray*}
por tanto:

\begin{eqnarray*}
\frac{\partial
\hat{F}_{2}\left(z_{1},z_{2},w_{1},\hat{\theta}_{2}\left(P_{1}\left(z_{1}\right)\tilde{P}_{2}\left(z_{2}\right)\hat{P}_{1}\left(w_{1}\right)\right)\right)}{\partial
z_{1}}|_{\mathbf{z},\mathbf{w}=1}&=&\frac{\partial\hat{F}_{2}}{\partial
w_{2}}\cdot\frac{\partial\hat{\theta}_{2}}{\partial
P_{1}}\cdot\frac{\partial P_{1}}{\partial z_{1}}+\frac{\partial
F_{2}}{\partial z_{1}}
\\
\frac{\partial \hat{F}_{2}\left(z_{1},z_{2},w_{1},\hat{\theta}_{2}\left(P_{1}\left(z_{1}\right)\tilde{P}_{2}\left(z_{2}\right)\hat{P}_{1}\left(w_{1}\right)\right)\right)}{\partial z_{2}}|_{\mathbf{z},\mathbf{w}=1}&=&\frac{\partial\hat{F}_{2}}{\partial w_{2}}\cdot\frac{\partial\hat{\theta}_{2}}{\partial \tilde{P}_{2}}\cdot\frac{\partial \tilde{P}_{2}}{\partial z_{2}}+\frac{\partial F_{2}}{\partial z_{2}}\\
\frac{\partial \hat{F}_{2}\left(z_{1},z_{2},w_{1},\hat{\theta}_{2}\left(P_{1}\left(z_{1}\right)\tilde{P}_{2}\left(z_{2}\right)\hat{P}_{1}\left(w_{1}\right)\right)\right)}{\partial w_{1}}|_{\mathbf{z},\mathbf{w}=1}&=&\frac{\partial\hat{F}_{2}}{\partial w_{2}}\cdot\frac{\partial\hat{\theta}_{2}}{\partial \hat{P}_{1}}\cdot\frac{\partial \hat{P}_{1}}{\partial w_{1}}+\frac{\partial \hat{F}_{2}}{\partial w_{1}}\\
\frac{\partial \hat{F}_{2}\left(z_{1},z_{2},w_{1},\hat{\theta}_{2}\left(P_{1}\left(z_{1}\right)\tilde{P}_{2}\left(z_{2}\right)\hat{P}_{1}\left(w_{1}\right)\right)\right)}{\partial w_{2}}|_{\mathbf{z},\mathbf{w}=1}&=&0\\
\end{eqnarray*}

%_________________________________________________________________________________________________
\subsection{Ecuaciones Recursivas}
%_________________________________________________________________________________________________

Entonces, de todo lo desarrollado hasta ahora se tienen las siguientes ecuaciones:

\begin{eqnarray*}
\frac{\partial F_{2}\left(\mathbf{z},\mathbf{w}\right)}{\partial z_{1}}|_{\mathbf{z},\mathbf{w}=1}&=&r_{1}\mu_{1}\\
\frac{\partial F_{2}\left(\mathbf{z},\mathbf{w}\right)}{\partial z_{2}}|_{\mathbf{z},\mathbf{w}=1}&=&=r_{1}\tilde{\mu}_{2}+f_{1}\left(1\right)\left(\frac{1}{1-\mu_{1}}\right)\tilde{\mu}_{2}+f_{1}\left(2\right)\\
\frac{\partial F_{2}\left(\mathbf{z},\mathbf{w}\right)}{\partial w_{1}}|_{\mathbf{z},\mathbf{w}=1}&=&r_{1}\hat{\mu}_{1}+f_{1}\left(1\right)\left(\frac{1}{1-\mu_{1}}\right)\hat{\mu}_{1}+\hat{F}_{1,1}^{(1)}\left(1\right)\\
\frac{\partial F_{2}\left(\mathbf{z},\mathbf{w}\right)}{\partial
w_{2}}|_{\mathbf{z},\mathbf{w}=1}&=&r_{1}\hat{\mu}_{2}+f_{1}\left(1\right)\left(\frac{1}{1-\mu_{1}}\right)\hat{\mu}_{2}+\hat{F}_{2,1}^{(1)}\left(1\right)\\
\frac{\partial F_{1}\left(\mathbf{z},\mathbf{w}\right)}{\partial z_{1}}|_{\mathbf{z},\mathbf{w}=1}&=&r_{2}\mu_{1}+f_{2}\left(2\right)\left(\frac{1}{1-\tilde{\mu}_{2}}\right)\mu_{1}+f_{2}\left(1\right)\\
\frac{\partial F_{1}\left(\mathbf{z},\mathbf{w}\right)}{\partial z_{2}}|_{\mathbf{z},\mathbf{w}=1}&=&r_{2}\tilde{\mu}_{2}\\
\frac{\partial F_{1}\left(\mathbf{z},\mathbf{w}\right)}{\partial w_{1}}|_{\mathbf{z},\mathbf{w}=1}&=&r_{2}\hat{\mu}_{1}+f_{2}\left(2\right)\left(\frac{1}{1-\tilde{\mu}_{2}}\right)\hat{\mu}_{1}+\hat{F}_{2,1}^{(1)}\left(1\right)\\
\frac{\partial F_{1}\left(\mathbf{z},\mathbf{w}\right)}{\partial
w_{2}}|_{\mathbf{z},\mathbf{w}=1}&=&r_{2}\hat{\mu}_{2}+f_{2}\left(2\right)\left(\frac{1}{1-\tilde{\mu}_{2}}\right)\hat{\mu}_{2}+\hat{F}_{2,2}^{(1)}\left(1\right)\\
\frac{\partial \hat{F}_{2}\left(\mathbf{z},\mathbf{w}\right)}{\partial z_{1}}|_{\mathbf{z},\mathbf{w}=1}&=&\hat{r}_{1}\mu_{1}+\hat{f}_{1}\left(1\right)\left(\frac{1}{1-\hat{\mu}_{1}}\right)\mu_{1}+F_{1,1}^{(1)}\left(1\right)\\
\frac{\partial \hat{F}_{2}\left(\mathbf{z},\mathbf{w}\right)}{\partial z_{2}}|_{\mathbf{z},\mathbf{w}=1}&=&\hat{r}_{1}\mu_{2}+\hat{f}_{1}\left(1\right)\left(\frac{1}{1-\hat{\mu}_{1}}\right)\tilde{\mu}_{2}+F_{2,1}^{(1)}\left(1\right)\\
\frac{\partial \hat{F}_{2}\left(\mathbf{z},\mathbf{w}\right)}{\partial w_{1}}|_{\mathbf{z},\mathbf{w}=1}&=&\hat{r}_{1}\hat{\mu}_{1}\\
\frac{\partial \hat{F}_{2}\left(\mathbf{z},\mathbf{w}\right)}{\partial w_{2}}|_{\mathbf{z},\mathbf{w}=1}&=&\hat{r}_{1}\hat{\mu}_{2}+\hat{f}_{1}\left(1\right)\left(\frac{1}{1-\hat{\mu}_{1}}\right)\hat{\mu}_{2}+\hat{f}_{1}\left(2\right)\\
\frac{\partial \hat{F}_{1}\left(\mathbf{z},\mathbf{w}\right)}{\partial z_{1}}|_{\mathbf{z},\mathbf{w}=1}&=&\hat{r}_{2}\mu_{1}+\hat{f}_{2}\left(1\right)\left(\frac{1}{1-\hat{\mu}_{2}}\right)\mu_{1}+F_{1,2}^{(1)}\left(1\right)\\
\frac{\partial \hat{F}_{1}\left(\mathbf{z},\mathbf{w}\right)}{\partial z_{2}}|_{\mathbf{z},\mathbf{w}=1}&=&\hat{r}_{2}\tilde{\mu}_{2}+\hat{f}_{2}\left(2\right)\left(\frac{1}{1-\hat{\mu}_{2}}\right)\tilde{\mu}_{2}+F_{2,2}^{(1)}\left(1\right)\\
\frac{\partial \hat{F}_{1}\left(\mathbf{z},\mathbf{w}\right)}{\partial w_{1}}|_{\mathbf{z},\mathbf{w}=1}&=&\hat{r}_{2}\hat{\mu}_{1}+\hat{f}_{2}\left(2\right)\left(\frac{1}{1-\hat{\mu}_{2}}\right)\hat{\mu}_{1}+\hat{f}_{2}\left(1\right)\\
\frac{\partial
\hat{F}_{1}\left(\mathbf{z},\mathbf{w}\right)}{\partial
w_{2}}|_{\mathbf{z},\mathbf{w}=1}&=&\hat{r}_{2}\hat{\mu}_{2}
\end{eqnarray*}

Es decir, se tienen las siguientes ecuaciones:




\begin{eqnarray*}
f_{2}\left(1\right)&=&r_{1}\mu_{1}\\
f_{1}\left(2\right)&=&r_{2}\tilde{\mu}_{2}\\
f_{2}\left(2\right)&=&r_{1}\tilde{\mu}_{2}+\tilde{\mu}_{2}\left(\frac{f_{1}\left(1\right)}{1-\mu_{1}}\right)+f_{1}\left(2\right)=\left(r_{1}+\frac{f_{1}\left(1\right)}{1-\mu_{1}}\right)\tilde{\mu}_{2}+r_{2}\tilde{\mu}_{2}\\
&=&\left(r_{1}+r_{2}+\frac{f_{1}\left(1\right)}{1-\mu_{1}}\right)\tilde{\mu}_{2}=\left(r+\frac{f_{1}\left(1\right)}{1-\mu_{1}}\right)\tilde{\mu}_{2}\\
f_{2}\left(3\right)&=&r_{1}\hat{\mu}_{1}+\hat{\mu}_{1}\left(\frac{f_{1}\left(1\right)}{1-\mu_{1}}\right)+\hat{F}_{1,1}^{(1)}\left(1\right)=\hat{\mu}_{1}\left(r_{1}+\frac{f_{1}\left(1\right)}{1-\mu_{1}}\right)+\frac{\hat{\mu}_{1}}{\mu_{1}}\\
f_{2}\left(4\right)&=&r_{1}\hat{\mu}_{2}+\hat{\mu}_{2}\left(\frac{f_{1}\left(1\right)}{1-\mu_{1}}\right)+\hat{F}_{2,1}^{(1)}\left(1\right)=\hat{\mu}_{2}\left(r_{1}+\frac{f_{1}\left(1\right)}{1-\mu_{1}}\right)+\frac{\hat{\mu}_{2}}{\mu_{1}}\\
f_{1}\left(1\right)&=&r_{2}\mu_{1}+\mu_{1}\left(\frac{f_{2}\left(2\right)}{1-\tilde{\mu}_{2}}\right)+r_{1}\mu_{1}=\mu_{1}\left(r_{1}+r_{2}+\frac{f_{2}\left(2\right)}{1-\tilde{\mu}_{2}}\right)\\
&=&\mu_{1}\left(r+\frac{f_{2}\left(2\right)}{1-\tilde{\mu}_{2}}\right)\\
f_{1}\left(3\right)&=&r_{2}\hat{\mu}_{1}+\hat{\mu}_{1}\left(\frac{f_{2}\left(2\right)}{1-\tilde{\mu}_{2}}\right)+\hat{F}^{(1)}_{1,2}\left(1\right)=\hat{\mu}_{1}\left(r_{2}+\frac{f_{2}\left(2\right)}{1-\tilde{\mu}_{2}}\right)+\frac{\hat{\mu}_{1}}{\mu_{2}}\\
f_{1}\left(4\right)&=&r_{2}\hat{\mu}_{2}+\hat{\mu}_{2}\left(\frac{f_{2}\left(2\right)}{1-\tilde{\mu}_{2}}\right)+\hat{F}_{2,2}^{(1)}\left(1\right)=\hat{\mu}_{2}\left(r_{2}+\frac{f_{2}\left(2\right)}{1-\tilde{\mu}_{2}}\right)+\frac{\hat{\mu}_{2}}{\mu_{2}}\\
\hat{f}_{1}\left(4\right)&=&\hat{r}_{2}\hat{\mu}_{2}\\
\hat{f}_{2}\left(3\right)&=&\hat{r}_{1}\hat{\mu}_{1}\\
\hat{f}_{1}\left(1\right)&=&\hat{r}_{2}\mu_{1}+\mu_{1}\left(\frac{\hat{f}_{2}\left(4\right)}{1-\hat{\mu}_{2}}\right)+F_{1,2}^{(1)}\left(1\right)=\left(\hat{r}_{2}+\frac{\hat{f}_{2}\left(4\right)}{1-\hat{\mu}_{2}}\right)\mu_{1}+\frac{\mu_{1}}{\hat{\mu}_{2}}\\
\hat{f}_{1}\left(2\right)&=&\hat{r}_{2}\tilde{\mu}_{2}+\tilde{\mu}_{2}\left(\frac{\hat{f}_{2}\left(4\right)}{1-\hat{\mu}_{2}}\right)+F_{2,2}^{(1)}\left(1\right)=
\left(\hat{r}_{2}+\frac{\hat{f}_{2}\left(4\right)}{1-\hat{\mu}_{2}}\right)\tilde{\mu}_{2}+\frac{\mu_{2}}{\hat{\mu}_{2}}\\
\hat{f}_{1}\left(3\right)&=&\hat{r}_{2}\hat{\mu}_{1}+\hat{\mu}_{1}\left(\frac{\hat{f}_{2}\left(4\right)}{1-\hat{\mu}_{2}}\right)+\hat{f}_{2}\left(3\right)=\left(\hat{r}_{2}+\frac{\hat{f}_{2}\left(4\right)}{1-\hat{\mu}_{2}}\right)\hat{\mu}_{1}+\hat{r}_{1}\hat{\mu}_{1}\\
&=&\left(\hat{r}_{1}+\hat{r}_{2}+\frac{\hat{f}_{2}\left(4\right)}{1-\hat{\mu}_{2}}\right)\hat{\mu}_{1}=\left(\hat{r}+\frac{\hat{f}_{2}\left(4\right)}{1-\hat{\mu}_{2}}\right)\hat{\mu}_{1}\\
\hat{f}_{2}\left(1\right)&=&\hat{r}_{1}\mu_{1}+\mu_{1}\left(\frac{\hat{f}_{1}\left(3\right)}{1-\hat{\mu}_{1}}\right)+F_{1,1}^{(1)}\left(1\right)=\left(\hat{r}_{1}+\frac{\hat{f}_{1}\left(3\right)}{1-\hat{\mu}_{1}}\right)\mu_{1}+\frac{\mu_{1}}{\hat{\mu}_{1}}\\
\hat{f}_{2}\left(2\right)&=&\hat{r}_{1}\tilde{\mu}_{2}+\tilde{\mu}_{2}\left(\frac{\hat{f}_{1}\left(3\right)}{1-\hat{\mu}_{1}}\right)+F_{2,1}^{(1)}\left(1\right)=\left(\hat{r}_{1}+\frac{\hat{f}_{1}\left(3\right)}{1-\hat{\mu}_{1}}\right)\tilde{\mu}_{2}+\frac{\mu_{2}}{\hat{\mu}_{1}}\\
\hat{f}_{2}\left(4\right)&=&\hat{r}_{1}\hat{\mu}_{2}+\hat{\mu}_{2}\left(\frac{\hat{f}_{1}\left(3\right)}{1-\hat{\mu}_{1}}\right)+\hat{f}_{1}\left(4\right)=\hat{r}_{1}\hat{\mu}_{2}+\hat{r}_{2}\hat{\mu}_{2}+\hat{\mu}_{2}\left(\frac{\hat{f}_{1}\left(3\right)}{1-\hat{\mu}_{1}}\right)\\
&=&\left(\hat{r}+\frac{\hat{f}_{1}\left(3\right)}{1-\hat{\mu}_{1}}\right)\hat{\mu}_{2}\\
\end{eqnarray*}

es decir,


\begin{eqnarray*}
\begin{array}{lll}
f_{1}\left(1\right)=\mu_{1}\left(r+\frac{f_{2}\left(2\right)}{1-\tilde{\mu}_{2}}\right)&f_{1}\left(2\right)=r_{2}\tilde{\mu}_{2}&f_{1}\left(3\right)=\hat{\mu}_{1}\left(r_{2}+\frac{f_{2}\left(2\right)}{1-\tilde{\mu}_{2}}\right)+\frac{\hat{\mu}_{1}}{\mu_{2}}\\
f_{1}\left(4\right)=\hat{\mu}_{2}\left(r_{2}+\frac{f_{2}\left(2\right)}{1-\tilde{\mu}_{2}}\right)+\frac{\hat{\mu}_{2}}{\mu_{2}}&f_{2}\left(1\right)=r_{1}\mu_{1}&f_{2}\left(2\right)=\left(r+\frac{f_{1}\left(1\right)}{1-\mu_{1}}\right)\tilde{\mu}_{2}\\
f_{2}\left(3\right)=\hat{\mu}_{1}\left(r_{1}+\frac{f_{1}\left(1\right)}{1-\mu_{1}}\right)+\frac{\hat{\mu}_{1}}{\mu_{1}}&
f_{2}\left(4\right)=\hat{\mu}_{2}\left(r_{1}+\frac{f_{1}\left(1\right)}{1-\mu_{1}}\right)+\frac{\hat{\mu}_{2}}{\mu_{1}}&\hat{f}_{1}\left(1\right)=\left(\hat{r}_{2}+\frac{\hat{f}_{2}\left(4\right)}{1-\hat{\mu}_{2}}\right)\mu_{1}+\frac{\mu_{1}}{\hat{\mu}_{2}}\\
\hat{f}_{1}\left(2\right)=\left(\hat{r}_{2}+\frac{\hat{f}_{2}\left(4\right)}{1-\hat{\mu}_{2}}\right)\tilde{\mu}_{2}+\frac{\mu_{2}}{\hat{\mu}_{2}}&\hat{f}_{1}\left(3\right)=\left(\hat{r}+\frac{\hat{f}_{2}\left(4\right)}{1-\hat{\mu}_{2}}\right)\hat{\mu}_{1}&\hat{f}_{1}\left(4\right)=\hat{r}_{2}\hat{\mu}_{2}\\
\hat{f}_{2}\left(1\right)=\left(\hat{r}_{1}+\frac{\hat{f}_{1}\left(3\right)}{1-\hat{\mu}_{1}}\right)\mu_{1}+\frac{\mu_{1}}{\hat{\mu}_{1}}&\hat{f}_{2}\left(2\right)=\left(\hat{r}_{1}+\frac{\hat{f}_{1}\left(3\right)}{1-\hat{\mu}_{1}}\right)\tilde{\mu}_{2}+\frac{\mu_{2}}{\hat{\mu}_{1}}&\hat{f}_{2}\left(3\right)=\hat{r}_{1}\hat{\mu}_{1}\\
&\hat{f}_{2}\left(4\right)=\left(\hat{r}+\frac{\hat{f}_{1}\left(3\right)}{1-\hat{\mu}_{1}}\right)\hat{\mu}_{2}&
\end{array}
\end{eqnarray*}

%_______________________________________________________________________________________________
\subsection{Soluci\'on del Sistema de Ecuaciones Lineales}
%_________________________________________________________________________________________________

A saber, se puede demostrar que la soluci\'on del sistema de
ecuaciones est\'a dado por las siguientes expresiones, donde

\begin{eqnarray*}
\mu=\mu_{1}+\tilde{\mu}_{2}\textrm{ , }\hat{\mu}=\hat{\mu}_{1}+\hat{\mu}_{2}\textrm{ , }
r=r_{1}+r_{2}\textrm{ y }\hat{r}=\hat{r}_{1}+\hat{r}_{2}
\end{eqnarray*}
entonces

\begin{eqnarray*}
\begin{array}{lll}
f_{1}\left(1\right)=r\frac{\mu_{1}\left(1-\mu_{1}\right)}{1-\mu}&
f_{1}\left(3\right)=\hat{\mu}_{1}\left(\frac{r_{2}\mu_{2}+1}{\mu_{2}}+r\frac{\tilde{\mu}_{2}}{1-\mu}\right)&
f_{1}\left(4\right)=\hat{\mu}_{2}\left(\frac{r_{2}\mu_{2}+1}{\mu_{2}}+r\frac{\tilde{\mu}_{2}}{1-\mu}\right)\\
f_{2}\left(2\right)=r\frac{\tilde{\mu}_{2}\left(1-\tilde{\mu}_{2}\right)}{1-\mu}&
f_{2}\left(3\right)=\hat{\mu}_{1}\left(\frac{r_{1}\mu_{1}+1}{\mu_{1}}+r\frac{\mu_{1}}{1-\mu}\right)&
f_{2}\left(4\right)=\hat{\mu}_{2}\left(\frac{r_{1}\mu_{1}+1}{\mu_{1}}+r\frac{\mu_{1}}{1-\mu}\right)\\
\hat{f}_{1}\left(1\right)=\mu_{1}\left(\frac{\hat{r}_{2}\hat{\mu}_{2}+1}{\hat{\mu}_{2}}+\hat{r}\frac{\hat{\mu}_{2}}{1-\hat{\mu}}\right)&
\hat{f}_{1}\left(2\right)=\tilde{\mu}_{2}\left(\hat{r}_{2}+\hat{r}\frac{\hat{\mu}_{2}}{1-\hat{\mu}}\right)+\frac{\mu_{2}}{\hat{\mu}_{2}}&
\hat{f}_{1}\left(3\right)=\hat{r}\frac{\hat{\mu}_{1}\left(1-\hat{\mu}_{1}\right)}{1-\hat{\mu}}\\
\hat{f}_{2}\left(1\right)=\mu_{1}\left(\frac{\hat{r}_{1}\hat{\mu}_{1}+1}{\hat{\mu}_{1}}+\hat{r}\frac{\hat{\mu}_{1}}{1-\hat{\mu}}\right)&
\hat{f}_{2}\left(2\right)=\tilde{\mu}_{2}\left(\hat{r}_{1}+\hat{r}\frac{\hat{\mu}_{1}}{1-\hat{\mu}}\right)+\frac{\hat{\mu_{2}}}{\hat{\mu}_{1}}&
\hat{f}_{2}\left(4\right)=\hat{r}\frac{\hat{\mu}_{2}\left(1-\hat{\mu}_{2}\right)}{1-\hat{\mu}}\\
\end{array}
\end{eqnarray*}




A saber

\begin{eqnarray*}
f_{1}\left(3\right)&=&\hat{\mu}_{1}\left(r_{2}+\frac{f_{2}\left(2\right)}{1-\tilde{\mu}_{2}}\right)+\frac{\hat{\mu}_{1}}{\mu_{2}}=\hat{\mu}_{1}\left(r_{2}+\frac{r\frac{\tilde{\mu}_{2}\left(1-\tilde{\mu}_{2}\right)}{1-\mu}}{1-\tilde{\mu}_{2}}\right)+\frac{\hat{\mu}_{1}}{\mu_{2}}=\hat{\mu}_{1}\left(r_{2}+\frac{r\tilde{\mu}_{2}}{1-\mu}\right)+\frac{\hat{\mu}_{1}}{\mu_{2}}\\
&=&\hat{\mu}_{1}\left(r_{2}+\frac{r\tilde{\mu}_{2}}{1-\mu}+\frac{1}{\mu_{2}}\right)=\hat{\mu}_{1}\left(\frac{r_{2}\mu_{2}+1}{\mu_{2}}+\frac{r\tilde{\mu}_{2}}{1-\mu}\right)
\end{eqnarray*}

\begin{eqnarray*}
f_{1}\left(4\right)&=&\hat{\mu}_{2}\left(r_{2}+\frac{f_{2}\left(2\right)}{1-\tilde{\mu}_{2}}\right)+\frac{\hat{\mu}_{2}}{\mu_{2}}=\hat{\mu}_{2}\left(r_{2}+\frac{r\frac{\tilde{\mu}_{2}\left(1-\tilde{\mu}_{2}\right)}{1-\mu}}{1-\tilde{\mu}_{2}}\right)+\frac{\hat{\mu}_{2}}{\mu_{2}}=\hat{\mu}_{2}\left(r_{2}+\frac{r\tilde{\mu}_{2}}{1-\mu}\right)+\frac{\hat{\mu}_{1}}{\mu_{2}}\\
&=&\hat{\mu}_{2}\left(r_{2}+\frac{r\tilde{\mu}_{2}}{1-\mu}+\frac{1}{\mu_{2}}\right)=\hat{\mu}_{2}\left(\frac{r_{2}\mu_{2}+1}{\mu_{2}}+\frac{r\tilde{\mu}_{2}}{1-\mu}\right)
\end{eqnarray*}

\begin{eqnarray*}
f_{2}\left(3\right)&=&\hat{\mu}_{1}\left(r_{1}+\frac{f_{1}\left(1\right)}{1-\mu_{1}}\right)+\frac{\hat{\mu}_{1}}{\mu_{1}}=\hat{\mu}_{1}\left(r_{1}+\frac{r\frac{\mu_{1}\left(1-\mu_{1}\right)}{1-\mu}}{1-\mu_{1}}\right)+\frac{\hat{\mu}_{1}}{\mu_{1}}=\hat{\mu}_{1}\left(r_{1}+\frac{r\mu_{1}}{1-\mu}\right)+\frac{\hat{\mu}_{1}}{\mu_{1}}\\
&=&\hat{\mu}_{1}\left(r_{1}+\frac{r\mu_{1}}{1-\mu}+\frac{1}{\mu_{1}}\right)=\hat{\mu}_{1}\left(\frac{r_{1}\mu_{1}+1}{\mu_{1}}+\frac{r\mu_{1}}{1-\mu}\right)
\end{eqnarray*}

\begin{eqnarray*}
f_{2}\left(4\right)&=&\hat{\mu}_{2}\left(r_{1}+\frac{f_{1}\left(1\right)}{1-\mu_{1}}\right)+\frac{\hat{\mu}_{2}}{\mu_{1}}=\hat{\mu}_{2}\left(r_{1}+\frac{r\frac{\mu_{1}\left(1-\mu_{1}\right)}{1-\mu}}{1-\mu_{1}}\right)+\frac{\hat{\mu}_{1}}{\mu_{1}}=\hat{\mu}_{2}\left(r_{1}+\frac{r\mu_{1}}{1-\mu}\right)+\frac{\hat{\mu}_{1}}{\mu_{1}}\\
&=&\hat{\mu}_{2}\left(r_{1}+\frac{r\mu_{1}}{1-\mu}+\frac{1}{\mu_{1}}\right)=\hat{\mu}_{2}\left(\frac{r_{1}\mu_{1}+1}{\mu_{1}}+\frac{r\mu_{1}}{1-\mu}\right)\end{eqnarray*}


\begin{eqnarray*}
\hat{f}_{1}\left(1\right)&=&\mu_{1}\left(\hat{r}_{2}+\frac{\hat{f}_{2}\left(4\right)}{1-\tilde{\mu}_{2}}\right)+\frac{\mu_{1}}{\hat{\mu}_{2}}=\mu_{1}\left(\hat{r}_{2}+\frac{\hat{r}\frac{\hat{\mu}_{2}\left(1-\hat{\mu}_{2}\right)}{1-\hat{\mu}}}{1-\hat{\mu}_{2}}\right)+\frac{\mu_{1}}{\hat{\mu}_{2}}=\mu_{1}\left(\hat{r}_{2}+\frac{\hat{r}\hat{\mu}_{2}}{1-\hat{\mu}}\right)+\frac{\mu_{1}}{\mu_{2}}\\
&=&\mu_{1}\left(\hat{r}_{2}+\frac{\hat{r}\mu_{2}}{1-\hat{\mu}}+\frac{1}{\hat{\mu}_{2}}\right)=\mu_{1}\left(\frac{\hat{r}_{2}\hat{\mu}_{2}+1}{\hat{\mu}_{2}}+\frac{\hat{r}\hat{\mu}_{2}}{1-\hat{\mu}}\right)
\end{eqnarray*}

\begin{eqnarray*}
\hat{f}_{1}\left(2\right)&=&\tilde{\mu}_{2}\left(\hat{r}_{2}+\frac{\hat{f}_{2}\left(4\right)}{1-\tilde{\mu}_{2}}\right)+\frac{\mu_{2}}{\hat{\mu}_{2}}=\tilde{\mu}_{2}\left(\hat{r}_{2}+\frac{\hat{r}\frac{\hat{\mu}_{2}\left(1-\hat{\mu}_{2}\right)}{1-\hat{\mu}}}{1-\hat{\mu}_{2}}\right)+\frac{\mu_{2}}{\hat{\mu}_{2}}=\tilde{\mu}_{2}\left(\hat{r}_{2}+\frac{\hat{r}\hat{\mu}_{2}}{1-\hat{\mu}}\right)+\frac{\mu_{2}}{\hat{\mu}_{2}}
\end{eqnarray*}

\begin{eqnarray*}
\hat{f}_{2}\left(1\right)&=&\mu_{1}\left(\hat{r}_{1}+\frac{\hat{f}_{1}\left(3\right)}{1-\hat{\mu}_{1}}\right)+\frac{\mu_{1}}{\hat{\mu}_{1}}=\mu_{1}\left(\hat{r}_{1}+\frac{\hat{r}\frac{\hat{\mu}_{1}\left(1-\hat{\mu}_{1}\right)}{1-\hat{\mu}}}{1-\hat{\mu}_{1}}\right)+\frac{\mu_{1}}{\hat{\mu}_{1}}=\mu_{1}\left(\hat{r}_{1}+\frac{\hat{r}\hat{\mu}_{1}}{1-\hat{\mu}}\right)+\frac{\mu_{1}}{\hat{\mu}_{1}}\\
&=&\mu_{1}\left(\hat{r}_{1}+\frac{\hat{r}\hat{\mu}_{1}}{1-\hat{\mu}}+\frac{1}{\hat{\mu}_{1}}\right)=\mu_{1}\left(\frac{\hat{r}_{1}\hat{\mu}_{1}+1}{\hat{\mu}_{1}}+\frac{\hat{r}\hat{\mu}_{1}}{1-\hat{\mu}}\right)
\end{eqnarray*}

\begin{eqnarray*}
\hat{f}_{2}\left(2\right)&=&\tilde{\mu}_{2}\left(\hat{r}_{1}+\frac{\hat{f}_{1}\left(3\right)}{1-\tilde{\mu}_{1}}\right)+\frac{\mu_{2}}{\hat{\mu}_{1}}=\tilde{\mu}_{2}\left(\hat{r}_{1}+\frac{\hat{r}\frac{\hat{\mu}_{1}
\left(1-\hat{\mu}_{1}\right)}{1-\hat{\mu}}}{1-\hat{\mu}_{1}}\right)+\frac{\mu_{2}}{\hat{\mu}_{1}}=\tilde{\mu}_{2}\left(\hat{r}_{1}+\frac{\hat{r}\hat{\mu}_{1}}{1-\hat{\mu}}\right)+\frac{\mu_{2}}{\hat{\mu}_{1}}
\end{eqnarray*}

%----------------------------------------------------------------------------------------
\section{Resultado Principal}
%----------------------------------------------------------------------------------------
Sean $\mu_{1},\mu_{2},\check{\mu}_{2},\hat{\mu}_{1},\hat{\mu}_{2}$ y $\tilde{\mu}_{2}=\mu_{2}+\check{\mu}_{2}$ los valores esperados de los proceso definidos anteriormente, y sean $r_{1},r_{2}, \hat{r}_{1}$ y $\hat{r}_{2}$ los valores esperado s de los tiempos de traslado del servidor entre las colas para cada uno de los sistemas de visitas c\'iclicas. Si se definen $\mu=\mu_{1}+\tilde{\mu}_{2}$, $\hat{\mu}=\hat{\mu}_{1}+\hat{\mu}_{2}$, y $r=r_{1}+r_{2}$ y  $\hat{r}=\hat{r}_{1}+\hat{r}_{2}$, entonces se tiene el siguiente resultado.

\begin{Teo}
Supongamos que $\mu<1$, $\hat{\mu}<1$, entonces, el n\'umero de usuarios presentes en cada una de las colas que conforman la Red de Sistemas de Visitas C\'iclicas cuando uno de los servidores visita a alguna de ellas est\'a dada por la soluci\'on del Sistema de Ecuaciones Lineales presentados arriba cuyas expresiones damos a continuaci\'on:
%{\footnotesize{
\begin{eqnarray*}
\begin{array}{lll}
f_{1}\left(1\right)=r\frac{\mu_{1}\left(1-\mu_{1}\right)}{1-\mu},&f_{1}\left(2\right)=r_{2}\tilde{\mu}_{2},&f_{1}\left(3\right)=\hat{\mu}_{1}\left(\frac{r_{2}\mu_{2}+1}{\mu_{2}}+r\frac{\tilde{\mu}_{2}}{1-\mu}\right),\\
f_{1}\left(4\right)=\hat{\mu}_{2}\left(\frac{r_{2}\mu_{2}+1}{\mu_{2}}+r\frac{\tilde{\mu}_{2}}{1-\mu}\right),&f_{2}\left(1\right)=r_{1}\mu_{1},&f_{2}\left(2\right)=r\frac{\tilde{\mu}_{2}\left(1-\tilde{\mu}_{2}\right)}{1-\mu},\\
f_{2}\left(3\right)=\hat{\mu}_{1}\left(\frac{r_{1}\mu_{1}+1}{\mu_{1}}+r\frac{\mu_{1}}{1-\mu}\right),&f_{2}\left(4\right)=\hat{\mu}_{2}\left(\frac{r_{1}\mu_{1}+1}{\mu_{1}}+r\frac{\mu_{1}}{1-\mu}\right),&\hat{f}_{1}\left(1\right)=\mu_{1}\left(\frac{\hat{r}_{2}\hat{\mu}_{2}+1}{\hat{\mu}_{2}}+\hat{r}\frac{\hat{\mu}_{2}}{1-\hat{\mu}}\right),\\
\hat{f}_{1}\left(2\right)=\tilde{\mu}_{2}\left(\hat{r}_{2}+\hat{r}\frac{\hat{\mu}_{2}}{1-\hat{\mu}}\right)+\frac{\mu_{2}}{\hat{\mu}_{2}},&\hat{f}_{1}\left(3\right)=\hat{r}\frac{\hat{\mu}_{1}\left(1-\hat{\mu}_{1}\right)}{1-\hat{\mu}},&\hat{f}_{1}\left(4\right)=\hat{r}_{2}\hat{\mu}_{2},\\
\hat{f}_{2}\left(1\right)=\mu_{1}\left(\frac{\hat{r}_{1}\hat{\mu}_{1}+1}{\hat{\mu}_{1}}+\hat{r}\frac{\hat{\mu}_{1}}{1-\hat{\mu}}\right),&\hat{f}_{2}\left(2\right)=\tilde{\mu}_{2}\left(\hat{r}_{1}+\hat{r}\frac{\hat{\mu}_{1}}{1-\hat{\mu}}\right)+\frac{\hat{\mu_{2}}}{\hat{\mu}_{1}},&\hat{f}_{2}\left(3\right)=\hat{r}_{1}\hat{\mu}_{1},\\
&\hat{f}_{2}\left(4\right)=\hat{r}\frac{\hat{\mu}_{2}\left(1-\hat{\mu}_{2}\right)}{1-\hat{\mu}}.&\\
\end{array}
\end{eqnarray*} %}}
\end{Teo}





%___________________________________________________________________________________________
%
\section{Segundos Momentos}
%___________________________________________________________________________________________
%
%___________________________________________________________________________________________
%
%\subsection{Derivadas de Segundo Orden: Tiempos de Traslado del Servidor}
%___________________________________________________________________________________________



Para poder determinar los segundos momentos para los tiempos de traslado del servidor es necesaria la siguiente proposici\'on:

\begin{Prop}\label{Prop.Segundas.Derivadas}
Sea $f\left(g\left(x\right)h\left(y\right)\right)$ funci\'on continua tal que tiene derivadas parciales mixtas de segundo orden, entonces se tiene lo siguiente:

\begin{eqnarray*}
\frac{\partial}{\partial x}f\left(g\left(x\right)h\left(y\right)\right)=\frac{\partial f\left(g\left(x\right)h\left(y\right)\right)}{\partial x}\cdot \frac{\partial g\left(x\right)}{\partial x}\cdot h\left(y\right)
\end{eqnarray*}

por tanto

\begin{eqnarray}
\frac{\partial}{\partial x}\frac{\partial}{\partial x}f\left(g\left(x\right)h\left(y\right)\right)
&=&\frac{\partial^{2}}{\partial x}f\left(g\left(x\right)h\left(y\right)\right)\cdot \left(\frac{\partial g\left(x\right)}{\partial x}\right)^{2}\cdot h^{2}\left(y\right)+\frac{\partial}{\partial x}f\left(g\left(x\right)h\left(y\right)\right)\cdot \frac{\partial g^{2}\left(x\right)}{\partial x^{2}}\cdot h\left(y\right).
\end{eqnarray}

y

\begin{eqnarray*}
\frac{\partial}{\partial y}\frac{\partial}{\partial x}f\left(g\left(x\right)h\left(y\right)\right)&=&\frac{\partial g\left(x\right)}{\partial x}\cdot \frac{\partial h\left(y\right)}{\partial y}\left\{\frac{\partial^{2}}{\partial y\partial x}f\left(g\left(x\right)h\left(y\right)\right)\cdot g\left(x\right)\cdot h\left(y\right)+\frac{\partial}{\partial x}f\left(g\left(x\right)h\left(y\right)\right)\right\}
\end{eqnarray*}
\end{Prop}
\begin{proof}
\footnotesize{
\begin{eqnarray*}
\frac{\partial}{\partial x}\frac{\partial}{\partial x}f\left(g\left(x\right)h\left(y\right)\right)&=&\frac{\partial}{\partial x}\left\{\frac{\partial f\left(g\left(x\right)h\left(y\right)\right)}{\partial x}\cdot \frac{\partial g\left(x\right)}{\partial x}\cdot h\left(y\right)\right\}\\
&=&\frac{\partial}{\partial x}\left\{\frac{\partial}{\partial x}f\left(g\left(x\right)h\left(y\right)\right)\right\}\cdot \frac{\partial g\left(x\right)}{\partial x}\cdot h\left(y\right)+\frac{\partial}{\partial x}f\left(g\left(x\right)h\left(y\right)\right)\cdot \frac{\partial g^{2}\left(x\right)}{\partial x^{2}}\cdot h\left(y\right)\\
&=&\frac{\partial^{2}}{\partial x}f\left(g\left(x\right)h\left(y\right)\right)\cdot \frac{\partial g\left(x\right)}{\partial x}\cdot h\left(y\right)\cdot \frac{\partial g\left(x\right)}{\partial x}\cdot h\left(y\right)+\frac{\partial}{\partial x}f\left(g\left(x\right)h\left(y\right)\right)\cdot \frac{\partial g^{2}\left(x\right)}{\partial x^{2}}\cdot h\left(y\right)\\
&=&\frac{\partial^{2}}{\partial x}f\left(g\left(x\right)h\left(y\right)\right)\cdot \left(\frac{\partial g\left(x\right)}{\partial x}\right)^{2}\cdot h^{2}\left(y\right)+\frac{\partial}{\partial x}f\left(g\left(x\right)h\left(y\right)\right)\cdot \frac{\partial g^{2}\left(x\right)}{\partial x^{2}}\cdot h\left(y\right).
\end{eqnarray*}}


Por otra parte:
\footnotesize{
\begin{eqnarray*}
\frac{\partial}{\partial y}\frac{\partial}{\partial x}f\left(g\left(x\right)h\left(y\right)\right)&=&\frac{\partial}{\partial y}\left\{\frac{\partial f\left(g\left(x\right)h\left(y\right)\right)}{\partial x}\cdot \frac{\partial g\left(x\right)}{\partial x}\cdot h\left(y\right)\right\}\\
&=&\frac{\partial}{\partial y}\left\{\frac{\partial}{\partial x}f\left(g\left(x\right)h\left(y\right)\right)\right\}\cdot \frac{\partial g\left(x\right)}{\partial x}\cdot h\left(y\right)+\frac{\partial}{\partial x}f\left(g\left(x\right)h\left(y\right)\right)\cdot \frac{\partial g\left(x\right)}{\partial x}\cdot \frac{\partial h\left(y\right)}{y}\\
&=&\frac{\partial^{2}}{\partial y\partial x}f\left(g\left(x\right)h\left(y\right)\right)\cdot \frac{\partial h\left(y\right)}{\partial y}\cdot g\left(x\right)\cdot \frac{\partial g\left(x\right)}{\partial x}\cdot h\left(y\right)+\frac{\partial}{\partial x}f\left(g\left(x\right)h\left(y\right)\right)\cdot \frac{\partial g\left(x\right)}{\partial x}\cdot \frac{\partial h\left(y\right)}{\partial y}\\
&=&\frac{\partial g\left(x\right)}{\partial x}\cdot \frac{\partial h\left(y\right)}{\partial y}\left\{\frac{\partial^{2}}{\partial y\partial x}f\left(g\left(x\right)h\left(y\right)\right)\cdot g\left(x\right)\cdot h\left(y\right)+\frac{\partial}{\partial x}f\left(g\left(x\right)h\left(y\right)\right)\right\}
\end{eqnarray*}}
\end{proof}

Utilizando la proposici\'on anterior (Proposici\'ion \ref{Prop.Segundas.Derivadas})se tiene el siguiente resultado que me dice como calcular los segundos momentos para los procesos de traslado del servidor:

\begin{Prop}
Sea $R_{i}$ la Funci\'on Generadora de Probabilidades para el n\'umero de arribos a cada una de las colas de la Red de Sistemas de Visitas C\'iclicas definidas como en (\ref{Ec.R1}). Entonces las derivadas parciales est\'an dadas por las siguientes expresiones:


\begin{eqnarray*}
\frac{\partial^{2} R_{i}\left(P_{1}\left(z_{1}\right)\tilde{P}_{2}\left(z_{2}\right)\hat{P}_{1}\left(w_{1}\right)\hat{P}_{2}\left(w_{2}\right)\right)}{\partial z_{i}^{2}}&=&\left(\frac{\partial P_{i}\left(z_{i}\right)}{\partial z_{i}}\right)^{2}\cdot\frac{\partial^{2} R_{i}\left(P_{1}\left(z_{1}\right)\tilde{P}_{2}\left(z_{2}\right)\hat{P}_{1}\left(w_{1}\right)\hat{P}_{2}\left(w_{2}\right)\right)}{\partial^{2} z_{i}}\\
&+&\left(\frac{\partial P_{i}\left(z_{i}\right)}{\partial z_{i}}\right)^{2}\cdot
\frac{\partial R_{i}\left(P_{1}\left(z_{1}\right)\tilde{P}_{2}\left(z_{2}\right)\hat{P}_{1}\left(w_{1}\right)\hat{P}_{2}\left(w_{2}\right)\right)}{\partial z_{i}}
\end{eqnarray*}



y adem\'as


\begin{eqnarray*}
\frac{\partial^{2} R_{i}\left(P_{1}\left(z_{1}\right)\tilde{P}_{2}\left(z_{2}\right)\hat{P}_{1}\left(w_{1}\right)\hat{P}_{2}\left(w_{2}\right)\right)}{\partial z_{2}\partial z_{1}}&=&\frac{\partial \tilde{P}_{2}\left(z_{2}\right)}{\partial z_{2}}\cdot\frac{\partial P_{1}\left(z_{1}\right)}{\partial z_{1}}\cdot\frac{\partial^{2} R_{i}\left(P_{1}\left(z_{1}\right)\tilde{P}_{2}\left(z_{2}\right)\hat{P}_{1}\left(w_{1}\right)\hat{P}_{2}\left(w_{2}\right)\right)}{\partial z_{2}\partial z_{1}}\\
&+&\frac{\partial \tilde{P}_{2}\left(z_{2}\right)}{\partial z_{2}}\cdot\frac{\partial P_{1}\left(z_{1}\right)}{\partial z_{1}}\cdot\frac{\partial R_{i}\left(P_{1}\left(z_{1}\right)\tilde{P}_{2}\left(z_{2}\right)\hat{P}_{1}\left(w_{1}\right)\hat{P}_{2}\left(w_{2}\right)\right)}{\partial z_{1}},
\end{eqnarray*}



\begin{eqnarray*}
\frac{\partial^{2} R_{i}\left(P_{1}\left(z_{1}\right)\tilde{P}_{2}\left(z_{2}\right)\hat{P}_{1}\left(w_{1}\right)\hat{P}_{2}\left(w_{2}\right)\right)}{\partial w_{i}\partial z_{1}}&=&\frac{\partial \hat{P}_{i}\left(w_{i}\right)}{\partial z_{2}}\cdot\frac{\partial P_{1}\left(z_{1}\right)}{\partial z_{1}}\cdot\frac{\partial^{2} R_{i}\left(P_{1}\left(z_{1}\right)\tilde{P}_{2}\left(z_{2}\right)\hat{P}_{1}\left(w_{1}\right)\hat{P}_{2}\left(w_{2}\right)\right)}{\partial w_{i}\partial z_{1}}\\
&+&\frac{\partial \hat{P}_{i}\left(w_{i}\right)}{\partial z_{2}}\cdot\frac{\partial P_{1}\left(z_{1}\right)}{\partial z_{1}}\cdot\frac{\partial R_{i}\left(P_{1}\left(z_{1}\right)\tilde{P}_{2}\left(z_{2}\right)\hat{P}_{1}\left(w_{1}\right)\hat{P}_{2}\left(w_{2}\right)\right)}{\partial z_{1}},
\end{eqnarray*}
finalmente

\begin{eqnarray*}
\frac{\partial^{2} R_{i}\left(P_{1}\left(z_{1}\right)\tilde{P}_{2}\left(z_{2}\right)\hat{P}_{1}\left(w_{1}\right)\hat{P}_{2}\left(w_{2}\right)\right)}{\partial w_{i}\partial z_{2}}&=&\frac{\partial \hat{P}_{i}\left(w_{i}\right)}{\partial w_{i}}\cdot\frac{\partial \tilde{P}_{2}\left(z_{2}\right)}{\partial z_{2}}\cdot\frac{\partial^{2} R_{i}\left(P_{1}\left(z_{1}\right)\tilde{P}_{2}\left(z_{2}\right)\hat{P}_{1}\left(w_{1}\right)\hat{P}_{2}\left(w_{2}\right)\right)}{\partial w_{i}\partial z_{2}}\\
&+&\frac{\partial \hat{P}_{i}\left(w_{i}\right)}{\partial w_{i}}\cdot\frac{\partial \tilde{P}_{2}\left(z_{2}\right)}{\partial z_{1}}\cdot\frac{\partial R_{i}\left(P_{1}\left(z_{1}\right)\tilde{P}_{2}\left(z_{2}\right)\hat{P}_{1}\left(w_{1}\right)\hat{P}_{2}\left(w_{2}\right)\right)}{\partial z_{2}},
\end{eqnarray*}

para $i=1,2$.
\end{Prop}

%___________________________________________________________________________________________
%
\subsection{Sistema de Ecuaciones Lineales para los Segundos Momentos}
%___________________________________________________________________________________________

En el ap\'endice (\ref{Segundos.Momentos}) se demuestra que las ecuaciones para las ecuaciones parciales mixtas est\'an dadas por:



%___________________________________________________________________________________________
%\subsubsection{Mixtas para $z_{1}$:}
%___________________________________________________________________________________________
%1
\begin{eqnarray*}
f_{1}\left(1,1\right)&=&r_{2}P_{1}^{(2)}\left(1\right)+\mu_{1}^{2}R_{2}^{(2)}\left(1\right)+2\mu_{1}r_{2}\left(\frac{\mu_{1}}{1-\tilde{\mu}_{2}}f_{2}\left(2\right)+f_{2}\left(1\right)\right)+\frac{1}{1-\tilde{\mu}_{2}}P_{1}^{(2)}f_{2}\left(2\right)+\mu_{1}^{2}\tilde{\theta}_{2}^{(2)}\left(1\right)f_{2}\left(2\right)\\
&+&\frac{\mu_{1}}{1-\tilde{\mu}_{2}}f_{2}(1,2)+\frac{\mu_{1}}{1-\tilde{\mu}_{2}}\left(\frac{\mu_{1}}{1-\tilde{\mu}_{2}}f_{2}(2,2)+f_{2}(1,2)\right)+f_{2}(1,1),\\
f_{1}\left(2,1\right)&=&\mu_{1}r_{2}\tilde{\mu}_{2}+\mu_{1}\tilde{\mu}_{2}R_{2}^{(2)}\left(1\right)+r_{2}\tilde{\mu}_{2}\left(\frac{\mu_{1}}{1-\tilde{\mu}_{2}}f_{2}(2)+f_{2}(1)\right),\\
f_{1}\left(3,1\right)&=&\mu_{1}\hat{\mu}_{1}r_{2}+\mu_{1}\hat{\mu}_{1}R_{2}^{(2)}\left(1\right)+r_{2}\frac{\mu_{1}}{1-\tilde{\mu}_{2}}f_{2}(2)+r_{2}\hat{\mu}_{1}\left(\frac{\mu_{1}}{1-\tilde{\mu}_{2}}f_{2}(2)+f_{2}(1)\right)+\mu_{1}r_{2}\hat{F}_{2,1}^{(1)}(1)\\
&+&\left(\frac{\mu_{1}}{1-\tilde{\mu}_{2}}f_{2}(2)+f_{2}(1)\right)\hat{F}_{2,1}^{(1)}(1)+\frac{\mu_{1}\hat{\mu}_{1}}{1-\tilde{\mu}_{2}}f_{2}(2)+\mu_{1}\hat{\mu}_{1}\tilde{\theta}_{2}^{(2)}\left(1\right)f_{2}(2)+\mu_{1}\hat{\mu}_{1}\left(\frac{1}{1-\tilde{\mu}_{2}}\right)^{2}f_{2}(2,2)\\
&+&+\frac{\hat{\mu}_{1}}{1-\tilde{\mu}_{2}}f_{2}(1,2),\\
f_{1}\left(4,1\right)&=&\mu_{1}\hat{\mu}_{2}r_{2}+\mu_{1}\hat{\mu}_{2}R_{2}^{(2)}\left(1\right)+r_{2}\frac{\mu_{1}\hat{\mu}_{2}}{1-\tilde{\mu}_{2}}f_{2}(2)+\mu_{1}r_{2}\hat{F}_{2,2}^{(1)}(1)+r_{2}\hat{\mu}_{2}\left(\frac{\mu_{1}}{1-\tilde{\mu}_{2}}f_{2}(2)+f_{2}(1)\right)\\
&+&\hat{F}_{2,1}^{(1)}(1)\left(\frac{\mu_{1}}{1-\tilde{\mu}_{2}}f_{2}(2)+f_{2}(1)\right)+\frac{\mu_{1}\hat{\mu}_{2}}{1-\tilde{\mu}_{2}}f_{2}(2)
+\mu_{1}\hat{\mu}_{2}\tilde{\theta}_{2}^{(2)}\left(1\right)f_{2}(2)+\mu_{1}\hat{\mu}_{2}\left(\frac{1}{1-\tilde{\mu}_{2}}\right)^{2}f_{2}(2,2)\\
&+&\frac{\hat{\mu}_{2}}{1-\tilde{\mu}_{2}}f_{2}^{(1,2)},\\
\end{eqnarray*}
\begin{eqnarray*}
f_{1}\left(1,2\right)&=&\mu_{1}\tilde{\mu}_{2}r_{2}+\mu_{1}\tilde{\mu}_{2}R_{2}^{(2)}\left(1\right)+r_{2}\tilde{\mu}_{2}\left(\frac{\mu_{1}}{1-\tilde{\mu}_{2}}f_{2}(2)+f_{2}(1)\right),\\
f_{1}\left(2,2\right)&=&\tilde{\mu}_{2}^{2}R_{2}^{(2)}(1)+r_{2}\tilde{P}_{2}^{(2)}\left(1\right),\\
f_{1}\left(3,2\right)&=&\hat{\mu}_{1}\tilde{\mu}_{2}r_{2}+\hat{\mu}_{1}\tilde{\mu}_{2}R_{2}^{(2)}(1)+
r_{2}\frac{\hat{\mu}_{1}\tilde{\mu}_{2}}{1-\tilde{\mu}_{2}}f_{2}(2)+r_{2}\tilde{\mu}_{2}\hat{F}_{2,2}^{(1)}(1),\\
f_{1}\left(4,2\right)&=&\hat{\mu}_{2}\tilde{\mu}_{2}r_{2}+\hat{\mu}_{2}\tilde{\mu}_{2}R_{2}^{(2)}(1)+
r_{2}\frac{\hat{\mu}_{2}\tilde{\mu}_{2}}{1-\tilde{\mu}_{2}}f_{2}(2)+r_{2}\tilde{\mu}_{2}\hat{F}_{2,2}^{(1)}(1),\\
f_{1}\left(1,3\right)&=&\mu_{1}\hat{\mu}_{1}r_{2}+\mu_{1}\hat{\mu}_{1}R_{2}^{(2)}\left(1\right)+\frac{\mu_{1}\hat{\mu}_{1}}{1-\tilde{\mu}_{2}}f_{2}(2)+r_{2}\frac{\mu_{1}\hat{\mu}_{1}}{1-\tilde{\mu}_{2}}f_{2}(2)+\mu_{1}\hat{\mu}_{1}\tilde{\theta}_{2}^{(2)}\left(1\right)f_{2}(2)+r_{2}\mu_{1}\hat{F}_{2,1}^{(1)}(1)\\
&+&r_{2}\hat{\mu}_{1}\left(\frac{\mu_{1}}{1-\tilde{\mu}_{2}}f_{2}(2)+f_{2}\left(1\right)\right)+\left(\frac{\mu_{1}}{1-\tilde{\mu}_{2}}f_{2}\left(1\right)+f_{2}\left(1\right)\right)\hat{F}_{2,1}^{(1)}(1)\\
&+&\frac{\hat{\mu}_{1}}{1-\tilde{\mu}_{2}}\left(\frac{\mu_{1}}{1-\tilde{\mu}_{2}}f_{2}(2,2)+f_{2}^{(1,2)}\right),\\
f_{1}\left(2,3\right)&=&\tilde{\mu}_{2}\hat{\mu}_{1}r_{2}+\tilde{\mu}_{2}\hat{\mu}_{1}R_{2}^{(2)}\left(1\right)+r_{2}\frac{\tilde{\mu}_{2}\hat{\mu}_{1}}{1-\tilde{\mu}_{2}}f_{2}(2)+r_{2}\tilde{\mu}_{2}\hat{F}_{2,1}^{(1)}(1),\\
f_{1}\left(3,3\right)&=&\hat{\mu}_{1}^{2}R_{2}^{(2)}\left(1\right)+r_{2}\hat{P}_{1}^{(2)}\left(1\right)+2r_{2}\frac{\hat{\mu}_{1}^{2}}{1-\tilde{\mu}_{2}}f_{2}(2)+\hat{\mu}_{1}^{2}\tilde{\theta}_{2}^{(2)}\left(1\right)f_{2}(2)+\frac{1}{1-\tilde{\mu}_{2}}\hat{P}_{1}^{(2)}\left(1\right)f_{2}(2)\\
&+&\frac{\hat{\mu}_{1}^{2}}{1-\tilde{\mu}_{2}}f_{2}(2,2)+2r_{2}\hat{\mu}_{1}\hat{F}_{2,1}^{(1)}(1)+2\frac{\hat{\mu}_{1}}{1-\tilde{\mu}_{2}}f_{2}(2)\hat{F}_{2,1}^{(1)}(1)+\hat{f}_{2,1}^{(2)}(1),\\
f_{1}\left(4,3\right)&=&r_{2}\hat{\mu}_{2}\hat{\mu}_{1}+\hat{\mu}_{1}\hat{\mu}_{2}R_{2}^{(2)}(1)+\frac{\hat{\mu}_{1}\hat{\mu}_{2}}{1-\tilde{\mu}_{2}}f_{2}\left(2\right)+2r_{2}\frac{\hat{\mu}_{1}\hat{\mu}_{2}}{1-\tilde{\mu}_{2}}f_{2}\left(2\right)+\hat{\mu}_{2}\hat{\mu}_{1}\tilde{\theta}_{2}^{(2)}\left(1\right)f_{2}\left(2\right)+r_{2}\hat{\mu}_{1}\hat{F}_{2,2}^{(1)}(1)\\
&+&\frac{\hat{\mu}_{1}}{1-\tilde{\mu}_{2}}f_{2}\left(2\right)\hat{F}_{2,2}^{(1)}(1)+\hat{\mu}_{1}\hat{\mu}_{2}\left(\frac{1}{1-\tilde{\mu}_{2}}\right)^{2}f_{2}(2,2)+r_{2}\hat{\mu}_{2}\hat{F}_{2,1}^{(1)}(1)+\frac{\hat{\mu}_{2}}{1-\tilde{\mu}_{2}}f_{2}\left(2\right)\hat{F}_{2,1}^{(1)}(1)+\hat{f}_{2}(1,2),\\
f_{1}\left(1,4\right)&=&r_{2}\mu_{1}\hat{\mu}_{2}+\mu_{1}\hat{\mu}_{2}R_{2}^{(2)}(1)+\frac{\mu_{1}\hat{\mu}_{2}}{1-\tilde{\mu}_{2}}f_{2}(2)+r_{2}\frac{\mu_{1}\hat{\mu}_{2}}{1-\tilde{\mu}_{2}}f_{2}(2)+\mu_{1}\hat{\mu}_{2}\tilde{\theta}_{2}^{(2)}\left(1\right)f_{2}(2)+r_{2}\mu_{1}\hat{F}_{2,2}^{(1)}(1)\\
&+&r_{2}\hat{\mu}_{2}\left(\frac{\mu_{1}}{1-\tilde{\mu}_{2}}f_{2}(2)+f_{2}(1)\right)+\hat{F}_{2,2}^{(1)}(1)\left(\frac{\mu_{1}}{1-\tilde{\mu}_{2}}f_{2}(2)+f_{2}(1)\right)\\
&+&\frac{\hat{\mu}_{2}}{1-\tilde{\mu}_{2}}\left(\frac{\mu_{1}}{1-\tilde{\mu}_{2}}f_{2}(2,2)+f_{2}(1,2)\right),\\
f_{1}\left(2,4\right)
&=&r_{2}\tilde{\mu}_{2}\hat{\mu}_{2}+\tilde{\mu}_{2}\hat{\mu}_{2}R_{2}^{(2)}(1)+r_{2}\frac{\tilde{\mu}_{2}\hat{\mu}_{2}}{1-\tilde{\mu}_{2}}f_{2}(2)+r_{2}\tilde{\mu}_{2}\hat{F}_{2,2}^{(1)}(1),\\
f_{1}\left(3,4\right)&=&r_{2}\hat{\mu}_{1}\hat{\mu}_{2}+\hat{\mu}_{1}\hat{\mu}_{2}R_{2}^{(2)}\left(1\right)+\frac{\hat{\mu}_{1}\hat{\mu}_{2}}{1-\tilde{\mu}_{2}}f_{2}(2)+2r_{2}\frac{\hat{\mu}_{1}\hat{\mu}_{2}}{1-\tilde{\mu}_{2}}f_{2}(2)+\hat{\mu}_{1}\hat{\mu}_{2}\theta_{2}^{(2)}\left(1\right)f_{2}(2)+r_{2}\hat{\mu}_{1}\hat{F}_{2,2}^{(1)}(1)\\
&+&\frac{\hat{\mu}_{1}}{1-\tilde{\mu}_{2}}f_{2}(2)\hat{F}_{2,2}^{(1)}(1)+\hat{\mu}_{1}\hat{\mu}_{2}\left(\frac{1}{1-\tilde{\mu}_{2}}\right)^{2}f_{2}(2,2)+r_{2}\hat{\mu}_{2}\hat{F}_{2,2}^{(1)}(1)+\frac{\hat{\mu}_{2}}{1-\tilde{\mu}_{2}}f_{2}(2)\hat{F}_{2,1}^{(1)}(1)+\hat{f}_{2}^{(2)}(1,2),\\
f_{1}\left(4,4\right)&=&\hat{\mu}_{2}^{2}R_{2}^{(2)}(1)+r_{2}\hat{P}_{2}^{(2)}\left(1\right)+2r_{2}\frac{\hat{\mu}_{2}^{2}}{1-\tilde{\mu}_{2}}f_{2}(2)+\hat{\mu}_{2}^{2}\tilde{\theta}_{2}^{(2)}\left(1\right)f_{2}(2)+\frac{1}{1-\tilde{\mu}_{2}}\hat{P}_{2}^{(2)}\left(1\right)f_{2}(2)\\
&+&2r_{2}\hat{\mu}_{2}\hat{F}_{2,2}^{(1)}(1)+2\frac{\hat{\mu}_{2}}{1-\tilde{\mu}_{2}}f_{2}(2)\hat{F}_{2,2}^{(1)}(1)+\left(\frac{\hat{\mu}_{2}}{1-\tilde{\mu}_{2}}\right)^{2}f_{2}(2,2)+\hat{f}_{2,2}^{(2)}(1),\\
f_{2}\left(1,1\right)&=&r_{1}P_{1}^{(2)}\left(1\right)+\mu_{1}^{2}R_{1}^{(2)}\left(1\right),\\
f_{2}\left(2,1\right)&=&\mu_{1}\tilde{\mu}_{2}r_{1}+\mu_{1}\tilde{\mu}_{2}R_{1}^{(2)}(1)+
r_{1}\mu_{1}\left(\frac{\tilde{\mu}_{2}}{1-\mu_{1}}f_{1}(1)+f_{1}(2)\right),\\
f_{2}\left(3,1\right)&=&r_{1}\mu_{1}\hat{\mu}_{1}+\mu_{1}\hat{\mu}_{1}R_{1}^{(2)}\left(1\right)+r_{1}\frac{\mu_{1}\hat{\mu}_{1}}{1-\mu_{1}}f_{1}(1)+r_{1}\mu_{1}\hat{F}_{1,1}^{(1)}(1),\\
f_{2}\left(4,1\right)&=&\mu_{1}\hat{\mu}_{2}r_{1}+\mu_{1}\hat{\mu}_{2}R_{1}^{(2)}\left(1\right)+r_{1}\mu_{1}\hat{F}_{1,2}^{(1)}(1)+r_{1}\frac{\mu_{1}\hat{\mu}_{2}}{1-\mu_{1}}f_{1}(1),\\
\end{eqnarray*}
\begin{eqnarray*}
f_{2}\left(1,2\right)&=&r_{1}\mu_{1}\tilde{\mu}_{2}+\mu_{1}\tilde{\mu}_{2}R_{1}^{(2)}\left(1\right)+r_{1}\mu_{1}\left(\frac{\tilde{\mu}_{2}}{1-\mu_{1}}f_{1}(1)+f_{1}(2)\right),\\
f_{2}\left(2,2\right)&=&\tilde{\mu}_{2}^{2}R_{1}^{(2)}\left(1\right)+r_{1}\tilde{P}_{2}^{(2)}\left(1\right)+2r_{1}\tilde{\mu}_{2}\left(\frac{\tilde{\mu}_{2}}{1-\mu_{1}}f_{1}(1)+f_{1}(2)\right)+f_{1}(2,2)+\tilde{\mu}_{2}^{2}\theta_{1}^{(2)}\left(1\right)f_{1}(1)\\
&+&\frac{1}{1-\mu_{1}}\tilde{P}_{2}^{(2)}\left(1\right)f_{1}(1)+\frac{\tilde{\mu}_{2}}{1-\mu_{1}}f_{1}(1,2)+\frac{\tilde{\mu}_{2}}{1-\mu_{1}}\left(\frac{\tilde{\mu}_{2}}{1-\mu_{1}}f_{1}(1,1)+f_{1}(1,2)\right),\\
f_{2}\left(3,2\right)&=&\tilde{\mu}_{2}\hat{\mu}_{1}r_{1}+\tilde{\mu}_{2}\hat{\mu}_{1}R_{1}^{(2)}\left(1\right)+r_{1}\frac{\tilde{\mu}_{2}\hat{\mu}_{1}}{1-\mu_{1}}f_{1}(1)+\hat{\mu}_{1}r_{1}\left(\frac{\tilde{\mu}_{2}}{1-\mu_{1}}f_{1}(1)+f_{1}(2)\right)+r_{1}\tilde{\mu}_{2}\hat{F}_{1,1}^{(1)}(1)\\
&+&\left(\frac{\tilde{\mu}_{2}}{1-\mu_{1}}f_{1}(1)+f_{1}(2)\right)\hat{F}_{1,1}^{(1)}(1)+\frac{\tilde{\mu}_{2}\hat{\mu}_{1}}{1-\mu_{1}}f_{1}(1)+\tilde{\mu}_{2}\hat{\mu}_{1}\theta_{1}^{(2)}\left(1\right)f_{1}(1)+\frac{\hat{\mu}_{1}}{1-\mu_{1}}f_{1}(1,2)\\
&+&\left(\frac{1}{1-\mu_{1}}\right)^{2}\tilde{\mu}_{2}\hat{\mu}_{1}f_{1}(1,1),\\
f_{2}\left(4,2\right)&=&\hat{\mu}_{2}\tilde{\mu}_{2}r_{1}+\hat{\mu}_{2}\tilde{\mu}_{2}R_{1}^{(2)}(1)+r_{1}\tilde{\mu}_{2}\hat{F}_{1,2}^{(1)}(1)+r_{1}\frac{\hat{\mu}_{2}\tilde{\mu}_{2}}{1-\mu_{1}}f_{1}(1)+\hat{\mu}_{2}r_{1}\left(\frac{\tilde{\mu}_{2}}{1-\mu_{1}}f_{1}(1)+f_{1}(2)\right)\\
&+&\left(\frac{\tilde{\mu}_{2}}{1-\mu_{1}}f_{1}(1)+f_{1}(2)\right)\hat{F}_{1,2}^{(1)}(1)+\frac{\tilde{\mu}_{2}\hat{\mu_{2}}}{1-\mu_{1}}f_{1}(1)+\hat{\mu}_{2}\tilde{\mu}_{2}\theta_{1}^{(2)}\left(1\right)f_{1}(1)+\frac{\hat{\mu}_{2}}{1-\mu_{1}}f_{1}(1,2)\\
&+&\tilde{\mu}_{2}\hat{\mu}_{2}\left(\frac{1}{1-\mu_{1}}\right)^{2}f_{1}(1,1),\\
f_{2}\left(1,3\right)&=&r_{1}\mu_{1}\hat{\mu}_{1}+\mu_{1}\hat{\mu}_{1}R_{1}^{(2)}(1)+r_{1}\frac{\mu_{1}\hat{\mu}_{1}}{1-\mu_{1}}f_{1}(1)+r_{1}\mu_{1}\hat{F}_{1,1}^{(1)}(1),\\
 f_{2}\left(2,3\right)&=&r_{1}\hat{\mu}_{1}\tilde{\mu}_{2}+\tilde{\mu}_{2}\hat{\mu}_{1}R_{1}^{(2)}\left(1\right)+\frac{\hat{\mu}_{1}\tilde{\mu}_{2}}{1-\mu_{1}}f_{1}(1)+r_{1}\frac{\hat{\mu}_{1}\tilde{\mu}_{2}}{1-\mu_{1}}f_{1}(1)+\hat{\mu}_{1}\tilde{\mu}_{2}\theta_{1}^{(2)}\left(1\right)f_{1}(1)+r_{1}\tilde{\mu}_{2}\hat{F}_{1,1}(1)\\
&+&r_{1}\hat{\mu}_{1}\left(f_{1}(1)+\frac{\tilde{\mu}_{2}}{1-\mu_{1}}f_{1}(1)\right)+
+\left(f_{1}(2)+\frac{\tilde{\mu}_{2}}{1-\mu_{1}}f_{1}(1)\right)\hat{F}_{1,1}(1)\\
&+&\frac{\hat{\mu}_{1}}{1-\mu_{1}}\left(f_{1}(1,2)+\frac{\tilde{\mu}_{2}}{1-\mu_{1}}f_{1}(1,1)\right),\\
f_{2}\left(3,3\right)&=&\hat{\mu}_{1}^{2}R_{1}^{(2)}\left(1\right)+r_{1}\hat{P}_{1}^{(2)}\left(1\right)+2r_{1}\frac{\hat{\mu}_{1}^{2}}{1-\mu_{1}}f_{1}(1)+\hat{\mu}_{1}^{2}\theta_{1}^{(2)}\left(1\right)f_{1}(1)+2r_{1}\hat{\mu}_{1}\hat{F}_{1,1}^{(1)}(1)\\
&+&\frac{1}{1-\mu_{1}}\hat{P}_{1}^{(2)}\left(1\right)f_{1}(1)+2\frac{\hat{\mu}_{1}}{1-\mu_{1}}f_{1}(1)\hat{F}_{1,1}(1)+\left(\frac{\hat{\mu}_{1}}{1-\mu_{1}}\right)^{2}f_{1}(1,1)+\hat{f}_{1,1}^{(2)}(1),\\
f_{2}\left(4,3\right)&=&r_{1}\hat{\mu}_{1}\hat{\mu}_{2}+\hat{\mu}_{1}\hat{\mu}_{2}R_{1}^{(2)}\left(1\right)+r_{1}\hat{\mu}_{1}\hat{F}_{1,2}(1)+
\frac{\hat{\mu}_{1}\hat{\mu}_{2}}{1-\mu_{1}}f_{1}(1)+2r_{1}\frac{\hat{\mu}_{1}\hat{\mu}_{2}}{1-\mu_{1}}f_{1}(1)+r_{1}\hat{\mu}_{2}\hat{F}_{1,1}(1)\\
&+&\hat{\mu}_{1}\hat{\mu}_{2}\theta_{1}^{(2)}\left(1\right)f_{1}(1)+\frac{\hat{\mu}_{1}}{1-\mu_{1}}f_{1}(1)\hat{F}_{1,2}(1)+\frac{\hat{\mu}_{2}}{1-\mu_{1}}\hat{F}_{1,1}(1)f_{1}(1)\\
&+&\hat{f}_{1}^{(2)}(1,2)+\hat{\mu}_{1}\hat{\mu}_{2}\left(\frac{1}{1-\mu_{1}}\right)^{2}f_{1}(2,2),\\
f_{2}\left(1,4\right)&=&r_{1}\mu_{1}\hat{\mu}_{2}+\mu_{1}\hat{\mu}_{2}R_{1}^{(2)}\left(1\right)+r_{1}\mu_{1}\hat{F}_{1,2}(1)+r_{1}\frac{\mu_{1}\hat{\mu}_{2}}{1-\mu_{1}}f_{1}(1),\\
f_{2}\left(2,4\right)&=&r_{1}\hat{\mu}_{2}\tilde{\mu}_{2}+\hat{\mu}_{2}\tilde{\mu}_{2}R_{1}^{(2)}\left(1\right)+r_{1}\tilde{\mu}_{2}\hat{F}_{1,2}(1)+\frac{\hat{\mu}_{2}\tilde{\mu}_{2}}{1-\mu_{1}}f_{1}(1)+r_{1}\frac{\hat{\mu}_{2}\tilde{\mu}_{2}}{1-\mu_{1}}f_{1}(1)+\hat{\mu}_{2}\tilde{\mu}_{2}\theta_{1}^{(2)}\left(1\right)f_{1}(1)\\
&+&r_{1}\hat{\mu}_{2}\left(f_{1}(2)+\frac{\tilde{\mu}_{2}}{1-\mu_{1}}f_{1}(1)\right)+\left(f_{1}(2)+\frac{\tilde{\mu}_{2}}{1-\mu_{1}}f_{1}(1)\right)\hat{F}_{1,2}(1)\\&+&\frac{\hat{\mu}_{2}}{1-\mu_{1}}\left(f_{1}(1,2)+\frac{\tilde{\mu}_{2}}{1-\mu_{1}}f_{1}(1,1)\right),\\
\end{eqnarray*}
\begin{eqnarray*}
f_{2}\left(3,4\right)&=&r_{1}\hat{\mu}_{1}\hat{\mu}_{2}+\hat{\mu}_{1}\hat{\mu}_{2}R_{1}^{(2)}\left(1\right)+r_{1}\hat{\mu}_{1}\hat{F}_{1,2}(1)+
\frac{\hat{\mu}_{1}\hat{\mu}_{2}}{1-\mu_{1}}f_{1}(1)+2r_{1}\frac{\hat{\mu}_{1}\hat{\mu}_{2}}{1-\mu_{1}}f_{1}(1)+\hat{\mu}_{1}\hat{\mu}_{2}\theta_{1}^{(2)}\left(1\right)f_{1}(1)\\
&+&+\frac{\hat{\mu}_{1}}{1-\mu_{1}}\hat{F}_{1,2}(1)f_{1}(1)+r_{1}\hat{\mu}_{2}\hat{F}_{1,1}(1)+\frac{\hat{\mu}_{2}}{1-\mu_{1}}\hat{F}_{1,1}(1)f_{1}(1)+\hat{f}_{1}^{(2)}(1,2)+\hat{\mu}_{1}\hat{\mu}_{2}\left(\frac{1}{1-\mu_{1}}\right)^{2}f_{1}(1,1),\\
f_{2}\left(4,4\right)&=&\hat{\mu}_{2}R_{1}^{(2)}\left(1\right)+r_{1}\hat{P}_{2}^{(2)}\left(1\right)+2r_{1}\hat{\mu}_{2}\hat{F}_{1}^{(0,1)}+\hat{f}_{1,2}^{(2)}(1)+2r_{1}\frac{\hat{\mu}_{2}^{2}}{1-\mu_{1}}f_{1}(1)+\hat{\mu}_{2}^{2}\theta_{1}^{(2)}\left(1\right)f_{1}(1)\\
&+&\frac{1}{1-\mu_{1}}\hat{P}_{2}^{(2)}\left(1\right)f_{1}(1) +
2\frac{\hat{\mu}_{2}}{1-\mu_{1}}f_{1}(1)\hat{F}_{1,2}(1)+\left(\frac{\hat{\mu}_{2}}{1-\mu_{1}}\right)^{2}f_{1}(1,1),\\
\hat{f}_{1}\left(1,1\right)&=&\hat{r}_{2}P_{1}^{(2)}\left(1\right)+
\mu_{1}^{2}\hat{R}_{2}^{(2)}\left(1\right)+
2\hat{r}_{2}\frac{\mu_{1}^{2}}{1-\hat{\mu}_{2}}\hat{f}_{2}(2)+
\frac{1}{1-\hat{\mu}_{2}}P_{1}^{(2)}\left(1\right)\hat{f}_{2}(2)+
\mu_{1}^{2}\hat{\theta}_{2}^{(2)}\left(1\right)\hat{f}_{2}(2)\\
&+&\left(\frac{\mu_{1}^{2}}{1-\hat{\mu}_{2}}\right)^{2}\hat{f}_{2}(2,2)+2\hat{r}_{2}\mu_{1}F_{2,1}(1)+2\frac{\mu_{1}}{1-\hat{\mu}_{2}}\hat{f}_{2}(2)F_{2,1}(1)+F_{2,1}^{(2)}(1),\\
\hat{f}_{1}\left(2,1\right)&=&\hat{r}_{2}\mu_{1}\tilde{\mu}_{2}+\mu_{1}\tilde{\mu}_{2}\hat{R}_{2}^{(2)}\left(1\right)+\hat{r}_{2}\mu_{1}F_{2,2}(1)+
\frac{\mu_{1}\tilde{\mu}_{2}}{1-\hat{\mu}_{2}}\hat{f}_{2}(2)+2\hat{r}_{2}\frac{\mu_{1}\tilde{\mu}_{2}}{1-\hat{\mu}_{2}}\hat{f}_{2}(2)\\
&+&\mu_{1}\tilde{\mu}_{2}\hat{\theta}_{2}^{(2)}\left(1\right)\hat{f}_{2}(2)+\frac{\mu_{1}}{1-\hat{\mu}_{2}}F_{2,2}(1)\hat{f}_{2}(2)+\mu_{1} \tilde{\mu}_{2}\left(\frac{1}{1-\hat{\mu}_{2}}\right)^{2}\hat{f}_{2}(2,2)+\hat{r}_{2}\tilde{\mu}_{2}F_{2,1}(1)\\
&+&\frac{\tilde{\mu}_{2}}{1-\hat{\mu}_{2}}\hat{f}_{2}(2)F_{2,1}(1)+f_{2,1}^{(2)}(1),\\
\hat{f}_{1}\left(3,1\right)&=&\hat{r}_{2}\mu_{1}\hat{\mu}_{1}+\mu_{1}\hat{\mu}_{1}\hat{R}_{2}^{(2)}\left(1\right)+\hat{r}_{2}\frac{\mu_{1}\hat{\mu}_{1}}{1-\hat{\mu}_{2}}\hat{f}_{2}(2)+\hat{r}_{2}\hat{\mu}_{1}F_{2,1}(1)+\hat{r}_{2}\mu_{1}\hat{f}_{2}(1)\\
&+&F_{2,1}(1)\hat{f}_{2}(1)+\frac{\mu_{1}}{1-\hat{\mu}_{2}}\hat{f}_{2}(1,2),\\
\hat{f}_{1}\left(4,1\right)&=&\hat{r}_{2}\mu_{1}\hat{\mu}_{2}+\mu_{1}\hat{\mu}_{2}\hat{R}_{2}^{(2)}\left(1\right)+\frac{\mu_{1}\hat{\mu}_{2}}{1-\hat{\mu}_{2}}\hat{f}_{2}(2)+2\hat{r}_{2}\frac{\mu_{1}\hat{\mu}_{2}}{1-\hat{\mu}_{2}}\hat{f}_{2}(2)+\mu_{1}\hat{\mu}_{2}\hat{\theta}_{2}^{(2)}\left(1\right)\hat{f}_{2}(2)\\
&+&\mu_{1}\hat{\mu}_{2}\left(\frac{1}{1-\hat{\mu}_{2}}\right)^{2}\hat{f}_{2}(2,2)+\hat{r}_{2}\hat{\mu}_{2}F_{2,1}(1)+\frac{\hat{\mu}_{2}}{1-\hat{\mu}_{2}}\hat{f}_{2}(2)F_{2,1}(1),\\
\hat{f}_{1}\left(1,2\right)&=&\hat{r}_{2}\mu_{1}\tilde{\mu}_{2}+\mu_{1}\tilde{\mu}_{2}\hat{R}_{2}^{(2)}\left(1\right)+\mu_{1}\hat{r}_{2}F_{2,2}(1)+
\frac{\mu_{1}\tilde{\mu}_{2}}{1-\hat{\mu}_{2}}\hat{f}_{2}(2)+2\hat{r}_{2}\frac{\mu_{1}\tilde{\mu}_{2}}{1-\hat{\mu}_{2}}\hat{f}_{2}(2)\\
&+&\mu_{1}\tilde{\mu}_{2}\hat{\theta}_{2}^{(2)}\left(1\right)\hat{f}_{2}(2)+\frac{\mu_{1}}{1-\hat{\mu}_{2}}F_{2,2}(1)\hat{f}_{2}(2)+\mu_{1}\tilde{\mu}_{2}\left(\frac{1}{1-\hat{\mu}_{2}}\right)^{2}\hat{f}_{2}(2,2)\\
&+&\hat{r}_{2}\tilde{\mu}_{2}F_{2,1}(1)+\frac{\tilde{\mu}_{2}}{1-\hat{\mu}_{2}}\hat{f}_{2}(2)F_{2,1}(1)+f_{2}^{(2)}(1,2),\\
\hat{f}_{1}\left(2,2\right)&=&\hat{r}_{2}\tilde{P}_{2}^{(2)}\left(1\right)+\tilde{\mu}_{2}^{2}\hat{R}_{2}^{(2)}\left(1\right)+2\hat{r}_{2}\tilde{\mu}_{2}F_{2,2}(1)+2\hat{r}_{2}\frac{\tilde{\mu}_{2}^{2}}{1-\hat{\mu}_{2}}\hat{f}_{2}(2)+f_{2,2}^{(2)}(1)\\
&+&\frac{1}{1-\hat{\mu}_{2}}\tilde{P}_{2}^{(2)}\left(1\right)\hat{f}_{2}(2)+\tilde{\mu}_{2}^{2}\hat{\theta}_{2}^{(2)}\left(1\right)\hat{f}_{2}(2)+2\frac{\tilde{\mu}_{2}}{1-\hat{\mu}_{2}}F_{2,2}(1)\hat{f}_{2}(2)+\left(\frac{\tilde{\mu}_{2}}{1-\hat{\mu}_{2}}\right)^{2}\hat{f}_{2}(2,2),\\
\hat{f}_{1}\left(3,2\right)&=&\hat{r}_{2}\tilde{\mu}_{2}\hat{\mu}_{1}+\tilde{\mu}_{2}\hat{\mu}_{1}\hat{R}_{2}^{(2)}\left(1\right)+\hat{r}_{2}\hat{\mu}_{1}F_{2,2}(1)+\hat{r}_{2}\frac{\tilde{\mu}_{2}\hat{\mu}_{1}}{1-\hat{\mu}_{2}}\hat{f}_{2}(2)+\hat{r}_{2}\tilde{\mu}_{2}\hat{f}_{2}(1)+F_{2,2}(1)\hat{f}_{2}(1)\\
&+&\frac{\tilde{\mu}_{2}}{1-\hat{\mu}_{2}}\hat{f}_{2}(1,2),\\
\hat{f}_{1}\left(4,2\right)&=&\hat{r}_{2}\tilde{\mu}_{2}\hat{\mu}_{2}+\tilde{\mu}_{2}\hat{\mu}_{2}\hat{R}_{2}^{(2)}\left(1\right)+\hat{r}_{2}\hat{\mu}_{2}F_{2,2}(1)+
\frac{\tilde{\mu}_{2}\hat{\mu}_{2}}{1-\hat{\mu}_{2}}\hat{f}_{2}(2)+2\hat{r}_{2}\frac{\tilde{\mu}_{2}\hat{\mu}_{2}}{1-\hat{\mu}_{2}}\hat{f}_{2}(2)\\
&+&\tilde{\mu}_{2}\hat{\mu}_{2}\hat{\theta}_{2}^{(2)}\left(1\right)\hat{f}_{2}(2)+\frac{\hat{\mu}_{2}}{1-\hat{\mu}_{2}}F_{2,2}(1)\hat{f}_{2}(1)+\tilde{\mu}_{2}\hat{\mu}_{2}\left(\frac{1}{1-\hat{\mu}_{2}}\right)\hat{f}_{2}(2,2),\\
\end{eqnarray*}
\begin{eqnarray*}
\hat{f}_{1}\left(1,3\right)&=&\hat{r}_{2}\mu_{1}\hat{\mu}_{1}+\mu_{1}\hat{\mu}_{1}\hat{R}_{2}^{(2)}\left(1\right)+\hat{r}_{2}\frac{\mu_{1}\hat{\mu}_{1}}{1-\hat{\mu}_{2}}\hat{f}_{2}(2)+\hat{r}_{2}\hat{\mu}_{1}F_{2,1}(1)+\hat{r}_{2}\mu_{1}\hat{f}_{2}(1)\\
&+&F_{2,1}(1)\hat{f}_{2}(1)+\frac{\mu_{1}}{1-\hat{\mu}_{2}}\hat{f}_{2}(1,2),\\
\hat{f}_{1}\left(2,3\right)&=&\hat{r}_{2}\tilde{\mu}_{2}\hat{\mu}_{1}+\tilde{\mu}_{2}\hat{\mu}_{1}\hat{R}_{2}^{(2)}\left(1\right)+\hat{r}_{2}\hat{\mu}_{1}F_{2,2}(1)+\hat{r}_{2}\frac{\tilde{\mu}_{2}\hat{\mu}_{1}}{1-\hat{\mu}_{2}}\hat{f}_{2}(2)+\hat{r}_{2}\tilde{\mu}_{2}\hat{f}_{2}(1)\\
&+&F_{2,2}(1)\hat{f}_{2}(1)+\frac{\tilde{\mu}_{2}}{1-\hat{\mu}_{2}}\hat{f}_{2}(1,2),\\
\hat{f}_{1}\left(3,3\right)&=&\hat{r}_{2}\hat{P}_{1}^{(2)}\left(1\right)+\hat{\mu}_{1}^{2}\hat{R}_{2}^{(2)}\left(1\right)+2\hat{r}_{2}\hat{\mu}_{1}\hat{f}_{2}(1)+\hat{f}_{2}(1,1),\\
\hat{f}_{1}\left(4,3\right)&=&\hat{r}_{2}\hat{\mu}_{1}\hat{\mu}_{2}+\hat{\mu}_{1}\hat{\mu}_{2}\hat{R}_{2}^{(2)}\left(1\right)+
\hat{r}_{2}\frac{\hat{\mu}_{2}\hat{\mu}_{1}}{1-\hat{\mu}_{2}}\hat{f}_{2}(2)+\hat{r}_{2}\hat{\mu}_{2}\hat{f}_{2}(1)+\frac{\hat{\mu}_{2}}{1-\hat{\mu}_{2}}\hat{f}_{2}(1,2),\\
\hat{f}_{1}\left(1,4\right)&=&\hat{r}_{2}\mu_{1}\hat{\mu}_{2}+\mu_{1}\hat{\mu}_{2}\hat{R}_{2}^{(2)}\left(1\right)+
\frac{\mu_{1}\hat{\mu}_{2}}{1-\hat{\mu}_{2}}\hat{f}_{2}(2) +2\hat{r}_{2}\frac{\mu_{1}\hat{\mu}_{2}}{1-\hat{\mu}_{2}}\hat{f}_{2}(2)\\
&+&\mu_{1}\hat{\mu}_{2}\hat{\theta}_{2}^{(2)}\left(1\right)\hat{f}_{2}(2)+\mu_{1}\hat{\mu}_{2}\left(\frac{1}{1-\hat{\mu}_{2}}\right)^{2}\hat{f}_{2}(2,2)+\hat{r}_{2}\hat{\mu}_{2}F_{2,1}(1)+\frac{\hat{\mu}_{2}}{1-\hat{\mu}_{2}}\hat{f}_{2}(2)F_{2,1}(1),\\\hat{f}_{1}\left(2,4\right)&=&\hat{r}_{2}\tilde{\mu}_{2}\hat{\mu}_{2}+\tilde{\mu}_{2}\hat{\mu}_{2}\hat{R}_{2}^{(2)}\left(1\right)+\hat{r}_{2}\hat{\mu}_{2}F_{2,2}(1)+\frac{\tilde{\mu}_{2}\hat{\mu}_{2}}{1-\hat{\mu}_{2}}\hat{f}_{2}(2)+2\hat{r}_{2}\frac{\tilde{\mu}_{2}\hat{\mu}_{2}}{1-\hat{\mu}_{2}}\hat{f}_{2}(2)\\
&+&\tilde{\mu}_{2}\hat{\mu}_{2}\hat{\theta}_{2}^{(2)}\left(1\right)\hat{f}_{2}(2)+\frac{\hat{\mu}_{2}}{1-\hat{\mu}_{2}}\hat{f}_{2}(2)F_{2,2}(1)+\tilde{\mu}_{2}\hat{\mu}_{2}\left(\frac{1}{1-\hat{\mu}_{2}}\right)^{2}\hat{f}_{2}(2,2),\\
\hat{f}_{1}\left(3,4\right)&=&\hat{r}_{2}\hat{\mu}_{1}\hat{\mu}_{2}+\hat{\mu}_{1}\hat{\mu}_{2}\hat{R}_{2}^{(2)}\left(1\right)+
\hat{r}_{2}\frac{\hat{\mu}_{1}\hat{\mu}_{2}}{1-\hat{\mu}_{2}}\hat{f}_{2}(2)+
\hat{r}_{2}\hat{\mu}_{2}\hat{f}_{2}(1)+\frac{\hat{\mu}_{2}}{1-\hat{\mu}_{2}}\hat{f}_{2}(1,2),\\
\hat{f}_{1}\left(4,4\right)&=&\hat{r}_{2}P_{2}^{(2)}\left(1\right)+\hat{\mu}_{2}^{2}\hat{R}_{2}^{(2)}\left(1\right)+2\hat{r}_{2}\frac{\hat{\mu}_{2}^{2}}{1-\hat{\mu}_{2}}\hat{f}_{2}(2)+\frac{1}{1-\hat{\mu}_{2}}\hat{P}_{2}^{(2)}\left(1\right)\hat{f}_{2}(2)\\
&+&\hat{\mu}_{2}^{2}\hat{\theta}_{2}^{(2)}\left(1\right)\hat{f}_{2}(2)+\left(\frac{\hat{\mu}_{2}}{1-\hat{\mu}_{2}}\right)^{2}\hat{f}_{2}(2,2),\\
\hat{f}_{2}\left(,1\right)&=&\hat{r}_{1}P_{1}^{(2)}\left(1\right)+
\mu_{1}^{2}\hat{R}_{1}^{(2)}\left(1\right)+2\hat{r}_{1}\mu_{1}F_{1,1}(1)+
2\hat{r}_{1}\frac{\mu_{1}^{2}}{1-\hat{\mu}_{1}}\hat{f}_{1}(1)+\frac{1}{1-\hat{\mu}_{1}}P_{1}^{(2)}\left(1\right)\hat{f}_{1}(1)\\
&+&\mu_{1}^{2}\hat{\theta}_{1}^{(2)}\left(1\right)\hat{f}_{1}(1)+2\frac{\mu_{1}}{1-\hat{\mu}_{1}}\hat{f}_{1}^(1)F_{1,1}(1)+f_{1,1}^{(2)}(1)+\left(\frac{\mu_{1}}{1-\hat{\mu}_{1}}\right)^{2}\hat{f}_{1}^{(1,1)},\\
\hat{f}_{2}\left(2,1\right)&=&\hat{r}_{1}\mu_{1}\tilde{\mu}_{2}+\mu_{1}\tilde{\mu}_{2}\hat{R}_{1}^{(2)}\left(1\right)+
\hat{r}_{1}\mu_{1}F_{1,2}(1)+\tilde{\mu}_{2}\hat{r}_{1}F_{1,1}(1)+
\frac{\mu_{1}\tilde{\mu}_{2}}{1-\hat{\mu}_{1}}\hat{f}_{1}(1)\\
&+&2\hat{r}_{1}\frac{\mu_{1}\tilde{\mu}_{2}}{1-\hat{\mu}_{1}}\hat{f}_{1}(1)+\mu_{1}\tilde{\mu}_{2}\hat{\theta}_{1}^{(2)}\left(1\right)\hat{f}_{1}(1)+
\frac{\mu_{1}}{1-\hat{\mu}_{1}}\hat{f}_{1}(1)F_{1,2}(1)+\frac{\tilde{\mu}_{2}}{1-\hat{\mu}_{1}}\hat{f}_{1}(1)F_{1,1}(1)\\
&+&f_{1}^{(2)}(1,2)+\mu_{1}\tilde{\mu}_{2}\left(\frac{1}{1-\hat{\mu}_{1}}\right)^{2}\hat{f}_{1}(1,1),\\
\hat{f}_{2}\left(3,1\right)&=&\hat{r}_{1}\mu_{1}\hat{\mu}_{1}+\mu_{1}\hat{\mu}_{1}\hat{R}_{1}^{(2)}\left(1\right)+\hat{r}_{1}\hat{\mu}_{1}F_{1,1}(1)+\hat{r}_{1}\frac{\mu_{1}\hat{\mu}_{1}}{1-\hat{\mu}_{1}}\hat{F}_{1}(1),\\
\hat{f}_{2}\left(4,1\right)&=&\hat{r}_{1}\mu_{1}\hat{\mu}_{2}+\mu_{1}\hat{\mu}_{2}\hat{R}_{1}^{(2)}\left(1\right)+\hat{r}_{1}\hat{\mu}_{2}F_{1,1}(1)+\frac{\mu_{1}\hat{\mu}_{2}}{1-\hat{\mu}_{1}}\hat{f}_{1}(1)+\hat{r}_{1}\frac{\mu_{1}\hat{\mu}_{2}}{1-\hat{\mu}_{1}}\hat{f}_{1}(1)\\
&+&\mu_{1}\hat{\mu}_{2}\hat{\theta}_{1}^{(2)}\left(1\right)\hat{f}_{1}(1)+\hat{r}_{1}\mu_{1}\left(\hat{f}_{1}(2)+\frac{\hat{\mu}_{2}}{1-\hat{\mu}_{1}}\hat{f}_{1}(1)\right)+F_{1,1}(1)\left(\hat{f}_{1}(2)+\frac{\hat{\mu}_{2}}{1-\hat{\mu}_{1}}\hat{f}_{1}(1)\right)\\
&+&\frac{\mu_{1}}{1-\hat{\mu}_{1}}\left(\hat{f}_{1}(1,2)+\frac{\hat{\mu}_{2}}{1-\hat{\mu}_{1}}\hat{f}_{1}(1,1)\right),\\
\hat{f}_{2}\left(1,2\right)&=&\hat{r}_{1}\mu_{1}\tilde{\mu}_{2}+\mu_{1}\tilde{\mu}_{2}\hat{R}_{1}^{(2)}\left(1\right)+\hat{r}_{1}\mu_{1}F_{1,2}(1)+\hat{r}_{1}\tilde{\mu}_{2}F_{1,1}(1)+\frac{\mu_{1}\tilde{\mu}_{2}}{1-\hat{\mu}_{1}}\hat{f}_{1}(1)\\
&+&2\hat{r}_{1}\frac{\mu_{1}\tilde{\mu}_{2}}{1-\hat{\mu}_{1}}\hat{f}_{1}(1)+\mu_{1}\tilde{\mu}_{2}\hat{\theta}_{1}^{(2)}\left(1\right)\hat{f}_{1}(1)+\frac{\mu_{1}}{1-\hat{\mu}_{1}}\hat{f}_{1}(1)F_{1,2}(1)\\
&+&\frac{\tilde{\mu}_{2}}{1-\hat{\mu}_{1}}\hat{f}_{1}(1)F_{1,1}(1)+f_{1}^{(2)}(1,2)+\mu_{1}\tilde{\mu}_{2}\left(\frac{1}{1-\hat{\mu}_{1}}\right)^{2}\hat{f}_{1}(1,1),\\
\end{eqnarray*}
\begin{eqnarray*}
\hat{f}_{2}\left(2,2\right)&=&\hat{r}_{1}\tilde{P}_{2}^{(2)}\left(1\right)+\tilde{\mu}_{2}^{2}\hat{R}_{1}^{(2)}\left(1\right)+2\hat{r}_{1}\tilde{\mu}_{2}F_{1,2}(1)+ f_{1,2}^{(2)}(1)+2\hat{r}_{1}\frac{\tilde{\mu}_{2}^{2}}{1-\hat{\mu}_{1}}\hat{f}_{1}(1)\\
&+&\frac{1}{1-\hat{\mu}_{1}}\tilde{P}_{2}^{(2)}\left(1\right)\hat{f}_{1}(1)+\tilde{\mu}_{2}^{2}\hat{\theta}_{1}^{(2)}\left(1\right)\hat{f}_{1}(1)+2\frac{\tilde{\mu}_{2}}{1-\hat{\mu}_{1}}F_{1,2}(1)\hat{f}_{1}(1)+\left(\frac{\tilde{\mu}_{2}}{1-\hat{\mu}_{1}}\right)^{2}\hat{f}_{1}(1,1),\\
\hat{f}_{2}\left(3,2\right)&=&\hat{r}_{1}\hat{\mu}_{1}\tilde{\mu}_{2}+\hat{\mu}_{1}\tilde{\mu}_{2}\hat{R}_{1}^{(2)}\left(1\right)+
\hat{r}_{1}\hat{\mu}_{1}F_{1,2}(1)+\hat{r}_{1}\frac{\hat{\mu}_{1}\tilde{\mu}_{2}}{1-\hat{\mu}_{1}}\hat{f}_{1}(1),\\
\hat{f}_{2}\left(4,2\right)&=&\hat{r}_{1}\tilde{\mu}_{2}\hat{\mu}_{2}+\hat{\mu}_{2}\tilde{\mu}_{2}\hat{R}_{1}^{(2)}\left(1\right)+\hat{\mu}_{2}\hat{R}_{1}^{(2)}\left(1\right)F_{1,2}(1)+\frac{\hat{\mu}_{2}\tilde{\mu}_{2}}{1-\hat{\mu}_{1}}\hat{f}_{1}(1)\\
&+&\hat{r}_{1}\frac{\hat{\mu}_{2}\tilde{\mu}_{2}}{1-\hat{\mu}_{1}}\hat{f}_{1}(1)+\hat{\mu}_{2}\tilde{\mu}_{2}\hat{\theta}_{1}^{(2)}\left(1\right)\hat{f}_{1}(1)+\hat{r}_{1}\tilde{\mu}_{2}\left(\hat{f}_{1}(2)+\frac{\hat{\mu}_{2}}{1-\hat{\mu}_{1}}\hat{f}_{1}(1)\right)\\
&+&F_{1,2}(1)\left(\hat{f}_{1}(2)+\frac{\hat{\mu}_{2}}{1-\hat{\mu}_{1}}\hat{f}_{1}(1)\right)+\frac{\tilde{\mu}_{2}}{1-\hat{\mu}_{1}}\left(\hat{f}_{1}(1,2)+\frac{\hat{\mu}_{2}}{1-\hat{\mu}_{1}}\hat{f}_{1}(1,1)\right),\\
\hat{f}_{2}\left(1,3\right)&=&\hat{r}_{1}\mu_{1}\hat{\mu}_{1}+\mu_{1}\hat{\mu}_{1}\hat{R}_{1}^{(2)}\left(1\right)+\hat{r}_{1}\hat{\mu}_{1}F_{1,1}(1)+\hat{r}_{1}\frac{\mu_{1}\hat{\mu}_{1}}{1-\hat{\mu}_{1}}\hat{f}_{1}(1),\\
\hat{f}_{2}\left(2,3\right)&=&\hat{r}_{1}\tilde{\mu}_{2}\hat{\mu}_{1}+\tilde{\mu}_{2}\hat{\mu}_{1}\hat{R}_{1}^{(2)}\left(1\right)+\hat{r}_{1}\hat{\mu}_{1}F_{1,2}(1)+\hat{r}_{1}\frac{\tilde{\mu}_{2}\hat{\mu}_{1}}{1-\hat{\mu}_{1}}\hat{f}_{1}(1),\\
\hat{f}_{2}\left(3,3\right)&=&\hat{r}_{1}\hat{P}_{1}^{(2)}\left(1\right)+\hat{\mu}_{1}^{2}\hat{R}_{1}^{(2)}\left(1\right),\\
\hat{f}_{2}\left(4,3\right)&=&\hat{r}_{1}\hat{\mu}_{2}\hat{\mu}_{1}+\hat{\mu}_{2}\hat{\mu}_{1}\hat{R}_{1}^{(2)}\left(1\right)+\hat{r}_{1}\hat{\mu}_{1}\left(\hat{f}_{1}(2)+\frac{\hat{\mu}_{2}}{1-\hat{\mu}_{1}}\hat{f}_{1}(1)\right),\\
\hat{f}_{2}\left(1,4\right)&=&\hat{r}_{1}\mu_{1}\hat{\mu}_{2}+\mu_{1}\hat{\mu}_{2}\hat{R}_{1}^{(2)}\left(1\right)+\hat{r}_{1}\hat{\mu}_{2}F_{1,1}(1)+\hat{r}_{1}\frac{\mu_{1}\hat{\mu}_{2}}{1-\hat{\mu}_{1}}\hat{f}_{1}(1)+\hat{r}_{1}\mu_{1}\left(\hat{f}_{1}(2)+\frac{\hat{\mu}_{2}}{1-\hat{\mu}_{1}}\hat{f}_{1}(1)\right)\\
&+&F_{1,1}(1)\left(\hat{f}_{1}(2)+\frac{\hat{\mu}_{2}}{1-\hat{\mu}_{1}}\hat{f}_{1}(1)\right)+\frac{\mu_{1}\hat{\mu}_{2}}{1-\hat{\mu}_{1}}\hat{f}_{1}(1)+\mu_{1}\hat{\mu}_{2}\hat{\theta}_{1}^{(2)}\left(1\right)\hat{f}_{1}(1)\\
&+&\frac{\mu_{1}}{1-\hat{\mu}_{1}}\hat{f}_{1}(1,2)+\mu_{1}\hat{\mu}_{2}\left(\frac{1}{1-\hat{\mu}_{1}}\right)^{2}\hat{f}_{1}(1,1),\\
\hat{f}_{2}\left(2,4\right)&=&\hat{r}_{1}\tilde{\mu}_{2}\hat{\mu}_{2}+\tilde{\mu}_{2}\hat{\mu}_{2}\hat{R}_{1}^{(2)}\left(1\right)+\hat{r}_{1}\hat{\mu}_{2}F_{1,2}(1)+\hat{r}_{1}\frac{\tilde{\mu}_{2}\hat{\mu}_{2}}{1-\hat{\mu}_{1}}\hat{f}_{1}(1)\\
&+&\hat{r}_{1}\tilde{\mu}_{2}\left(\hat{f}_{1}(2)+\frac{\hat{\mu}_{2}}{1-\hat{\mu}_{1}}\hat{f}_{1}(1)\right)+F_{1,2}(1)\left(\hat{f}_{1}(2)+\frac{\hat{\mu}_{2}}{1-\hat{\mu}_{1}}\hat{F}_{1}^{(1,0)}\right)+\frac{\tilde{\mu}_{2}\hat{\mu}_{2}}{1-\hat{\mu}_{1}}\hat{f}_{1}(1)\\
&+&\tilde{\mu}_{2}\hat{\mu}_{2}\hat{\theta}_{1}^{(2)}\left(1\right)\hat{f}_{1}(1)+\frac{\tilde{\mu}_{2}}{1-\hat{\mu}_{1}}\hat{f}_{1}(1,2)+\tilde{\mu}_{2}\hat{\mu}_{2}\left(\frac{1}{1-\hat{\mu}_{1}}\right)^{2}\hat{f}_{1}(1,1),\\
\hat{f}_{2}\left(3,4\right)&=&\hat{r}_{1}\hat{\mu}_{2}\hat{\mu}_{1}+\hat{\mu}_{2}\hat{\mu}_{1}\hat{R}_{1}^{(2)}\left(1\right)+\hat{r}_{1}\hat{\mu}_{1}\left(\hat{f}_{1}(2)+\frac{\hat{\mu}_{2}}{1-\hat{\mu}_{1}}\hat{f}_{1}(1)\right),\\
\hat{f}_{2}\left(4,4\right)&=&\hat{r}_{1}\hat{P}_{2}^{(2)}\left(1\right)+\hat{\mu}_{2}^{2}\hat{R}_{1}^{(2)}\left(1\right)+
2\hat{r}_{1}\hat{\mu}_{2}\left(\hat{f}_{1}(2)+\frac{\hat{\mu}_{2}}{1-\hat{\mu}_{1}}\hat{f}_{1}(1)\right)+\hat{f}_{1}(2,2)\\
&+&\frac{1}{1-\hat{\mu}_{1}}\hat{P}_{2}^{(2)}\left(1\right)\hat{f}_{1}(1)+\hat{\mu}_{2}^{2}\hat{\theta}_{1}^{(2)}\left(1\right)\hat{f}_{1}(1)+\frac{\hat{\mu}_{2}}{1-\hat{\mu}_{1}}\hat{f}_{1}(1,2)\\
&+&\frac{\hat{\mu}_{2}}{1-\hat{\mu}_{1}}\left(\hat{f}_{1}(1,2)+\frac{\hat{\mu}_{2}}{1-\hat{\mu}_{1}}\hat{f}_{1}(1,1)\right).
\end{eqnarray*}
%_________________________________________________________________________________________________________
\section{Medidas de Desempe\~no}
%_________________________________________________________________________________________________________

\begin{Def}
Sea $L_{i}^{*}$el n\'umero de usuarios cuando el servidor visita la cola $Q_{i}$ para dar servicio, para $i=1,2$.
\end{Def}

Entonces
\begin{Prop} Para la cola $Q_{i}$, $i=1,2$, se tiene que el n\'umero de usuarios presentes al momento de ser visitada por el servidor est\'a dado por
\begin{eqnarray}
\esp\left[L_{i}^{*}\right]&=&f_{i}\left(i\right)\\
Var\left[L_{i}^{*}\right]&=&f_{i}\left(i,i\right)+\esp\left[L_{i}^{*}\right]-\esp\left[L_{i}^{*}\right]^{2}.
\end{eqnarray}
\end{Prop}


\begin{Def}
El tiempo de Ciclo $C_{i}$ es el periodo de tiempo que comienza
cuando la cola $i$ es visitada por primera vez en un ciclo, y
termina cuando es visitado nuevamente en el pr\'oximo ciclo, bajo condiciones de estabilidad.

\begin{eqnarray*}
C_{i}\left(z\right)=\esp\left[z^{\overline{\tau}_{i}\left(m+1\right)-\overline{\tau}_{i}\left(m\right)}\right]
\end{eqnarray*}
\end{Def}

\begin{Def}
El tiempo de intervisita $I_{i}$ es el periodo de tiempo que
comienza cuando se ha completado el servicio en un ciclo y termina
cuando es visitada nuevamente en el pr\'oximo ciclo.
\begin{eqnarray*}I_{i}\left(z\right)&=&\esp\left[z^{\tau_{i}\left(m+1\right)-\overline{\tau}_{i}\left(m\right)}\right]\end{eqnarray*}
\end{Def}

\begin{Prop}
Para los tiempos de intervisita del servidor $I_{i}$, se tiene que

\begin{eqnarray*}
\esp\left[I_{i}\right]&=&\frac{f_{i}\left(i\right)}{\mu_{i}},\\
Var\left[I_{i}\right]&=&\frac{Var\left[L_{i}^{*}\right]}{\mu_{i}^{2}}-\frac{\sigma_{i}^{2}}{\mu_{i}^{2}}f_{i}\left(i\right).
\end{eqnarray*}
\end{Prop}


\begin{Prop}
Para los tiempos que ocupa el servidor para atender a los usuarios presentes en la cola $Q_{i}$, con FGP denotada por $S_{i}$, se tiene que
\begin{eqnarray*}
\esp\left[S_{i}\right]&=&\frac{\esp\left[L_{i}^{*}\right]}{1-\mu_{i}}=\frac{f_{i}\left(i\right)}{1-\mu_{i}},\\
Var\left[S_{i}\right]&=&\frac{Var\left[L_{i}^{*}\right]}{\left(1-\mu_{i}\right)^{2}}+\frac{\sigma^{2}\esp\left[L_{i}^{*}\right]}{\left(1-\mu_{i}\right)^{3}}
\end{eqnarray*}
\end{Prop}


\begin{Prop}
Para la duraci\'on de los ciclos $C_{i}$ se tiene que
\begin{eqnarray*}
\esp\left[C_{i}\right]&=&\esp\left[I_{i}\right]\esp\left[\theta_{i}\left(z\right)\right]=\frac{\esp\left[L_{i}^{*}\right]}{\mu_{i}}\frac{1}{1-\mu_{i}}=\frac{f_{i}\left(i\right)}{\mu_{i}\left(1-\mu_{i}\right)}\\
Var\left[C_{i}\right]&=&\frac{Var\left[L_{i}^{*}\right]}{\mu_{i}^{2}\left(1-\mu_{i}\right)^{2}}.
\end{eqnarray*}

\end{Prop}

%___________________________________________________________________________________________
%
\section*{Ap\'endice A}\label{Segundos.Momentos}
%___________________________________________________________________________________________


%___________________________________________________________________________________________

%\subsubsection{Mixtas para $z_{1}$:}
%___________________________________________________________________________________________
\begin{enumerate}

%1/1/1
\item \begin{eqnarray*}
&&\frac{\partial}{\partial z_1}\frac{\partial}{\partial z_1}\left(R_2\left(P_1\left(z_1\right)\bar{P}_2\left(z_2\right)\hat{P}_1\left(w_1\right)\hat{P}_2\left(w_2\right)\right)F_2\left(z_1,\theta
_2\left(P_1\left(z_1\right)\hat{P}_1\left(w_1\right)\hat{P}_2\left(w_2\right)\right)\right)\hat{F}_2\left(w_1,w_2\right)\right)\\
&=&r_{2}P_{1}^{(2)}\left(1\right)+\mu_{1}^{2}R_{2}^{(2)}\left(1\right)+2\mu_{1}r_{2}\left(\frac{\mu_{1}}{1-\tilde{\mu}_{2}}F_{2}^{(0,1)}+F_{2}^{1,0)}\right)+\frac{1}{1-\tilde{\mu}_{2}}P_{1}^{(2)}F_{2}^{(0,1)}+\mu_{1}^{2}\tilde{\theta}_{2}^{(2)}\left(1\right)F_{2}^{(0,1)}\\
&+&\frac{\mu_{1}}{1-\tilde{\mu}_{2}}F_{2}^{(1,1)}+\frac{\mu_{1}}{1-\tilde{\mu}_{2}}\left(\frac{\mu_{1}}{1-\tilde{\mu}_{2}}F_{2}^{(0,2)}+F_{2}^{(1,1)}\right)+F_{2}^{(2,0)}.
\end{eqnarray*}

%2/2/1

\item \begin{eqnarray*}
&&\frac{\partial}{\partial z_2}\frac{\partial}{\partial z_1}\left(R_2\left(P_1\left(z_1\right)\bar{P}_2\left(z_2\right)\hat{P}_1\left(w_1\right)\hat{P}_2\left(w_2\right)\right)F_2\left(z_1,\theta
_2\left(P_1\left(z_1\right)\hat{P}_1\left(w_1\right)\hat{P}_2\left(w_2\right)\right)\right)\hat{F}_2\left(w_1,w_2\right)\right)\\
&=&\mu_{1}r_{2}\tilde{\mu}_{2}+\mu_{1}\tilde{\mu}_{2}R_{2}^{(2)}\left(1\right)+r_{2}\tilde{\mu}_{2}\left(\frac{\mu_{1}}{1-\tilde{\mu}_{2}}F_{2}^{(0,1)}+F_{2}^{(1,0)}\right).
\end{eqnarray*}
%3/3/1
\item \begin{eqnarray*}
&&\frac{\partial}{\partial w_1}\frac{\partial}{\partial z_1}\left(R_2\left(P_1\left(z_1\right)\bar{P}_2\left(z_2\right)\hat{P}_1\left(w_1\right)\hat{P}_2\left(w_2\right)\right)F_2\left(z_1,\theta
_2\left(P_1\left(z_1\right)\hat{P}_1\left(w_1\right)\hat{P}_2\left(w_2\right)\right)\right)\hat{F}_2\left(w_1,w_2\right)\right)\\
&=&\mu_{1}\hat{\mu}_{1}r_{2}+\mu_{1}\hat{\mu}_{1}R_{2}^{(2)}\left(1\right)+r_{2}\frac{\mu_{1}}{1-\tilde{\mu}_{2}}F_{2}^{(0,1)}+r_{2}\hat{\mu}_{1}\left(\frac{\mu_{1}}{1-\tilde{\mu}_{2}}F_{2}^{(0,1)}+F_{2}^{(1,0)}\right)+\mu_{1}r_{2}\hat{F}_{2}^{(1,0)}\\
&+&\left(\frac{\mu_{1}}{1-\tilde{\mu}_{2}}F_{2}^{(0,1)}+F_{2}^{(1,0)}\right)\hat{F}_{2}^{(1,0)}+\frac{\mu_{1}\hat{\mu}_{1}}{1-\tilde{\mu}_{2}}F_{2}^{(0,1)}+\mu_{1}\hat{\mu}_{1}\tilde{\theta}_{2}^{(2)}\left(1\right)F_{2}^{(0,1)}\\
&+&\mu_{1}\hat{\mu}_{1}\left(\frac{1}{1-\tilde{\mu}_{2}}\right)^{2}F_{2}^{(0,2)}+\frac{\hat{\mu}_{1}}{1-\tilde{\mu}_{2}}F_{2}^{(1,1)}.
\end{eqnarray*}
%4/4/1
\item \begin{eqnarray*}
&&\frac{\partial}{\partial w_2}\frac{\partial}{\partial z_1}\left(R_2\left(P_1\left(z_1\right)\bar{P}_2\left(z_2\right)\hat{P}_1\left(w_1\right)\hat{P}_2\left(w_2\right)\right)
F_2\left(z_1,\theta_2\left(P_1\left(z_1\right)\hat{P}_1\left(w_1\right)\hat{P}_2\left(w_2\right)\right)\right)\hat{F}_2\left(w_1,w_2\right)\right)\\
&=&\mu_{1}\hat{\mu}_{2}r_{2}+\mu_{1}\hat{\mu}_{2}R_{2}^{(2)}\left(1\right)+r_{2}\frac{\mu_{1}\hat{\mu}_{2}}{1-\tilde{\mu}_{2}}F_{2}^{(0,1)}+\mu_{1}r_{2}\hat{F}_{2}^{(0,1)}
+r_{2}\hat{\mu}_{2}\left(\frac{\mu_{1}}{1-\tilde{\mu}_{2}}F_{2}^{(0,1)}+F_{2}^{(1,0)}\right)\\
&+&\hat{F}_{2}^{(1,0)}\left(\frac{\mu_{1}}{1-\tilde{\mu}_{2}}F_{2}^{(0,1)}+F_{2}^{(1,0)}\right)+\frac{\mu_{1}\hat{\mu}_{2}}{1-\tilde{\mu}_{2}}F_{2}^{(0,1)}
+\mu_{1}\hat{\mu}_{2}\tilde{\theta}_{2}^{(2)}\left(1\right)F_{2}^{(0,1)}+\mu_{1}\hat{\mu}_{2}\left(\frac{1}{1-\tilde{\mu}_{2}}\right)^{2}F_{2}^{(0,2)}\\
&+&\frac{\hat{\mu}_{2}}{1-\tilde{\mu}_{2}}F_{2}^{(1,1)}.
\end{eqnarray*}
%___________________________________________________________________________________________
%\subsubsection{Mixtas para $z_{2}$:}
%___________________________________________________________________________________________
%5
\item \begin{eqnarray*} &&\frac{\partial}{\partial
z_1}\frac{\partial}{\partial
z_2}\left(R_2\left(P_1\left(z_1\right)\bar{P}_2\left(z_2\right)\hat{P}_1\left(w_1\right)\hat{P}_2\left(w_2\right)\right)
F_2\left(z_1,\theta_2\left(P_1\left(z_1\right)\hat{P}_1\left(w_1\right)\hat{P}_2\left(w_2\right)\right)\right)\hat{F}_2\left(w_1,w_2\right)\right)\\
&=&\mu_{1}\tilde{\mu}_{2}r_{2}+\mu_{1}\tilde{\mu}_{2}R_{2}^{(2)}\left(1\right)+r_{2}\tilde{\mu}_{2}\left(\frac{\mu_{1}}{1-\tilde{\mu}_{2}}F_{2}^{(0,1)}+F_{2}^{(1,0)}\right).
\end{eqnarray*}

%6

\item \begin{eqnarray*} &&\frac{\partial}{\partial
z_2}\frac{\partial}{\partial
z_2}\left(R_2\left(P_1\left(z_1\right)\bar{P}_2\left(z_2\right)\hat{P}_1\left(w_1\right)\hat{P}_2\left(w_2\right)\right)
F_2\left(z_1,\theta_2\left(P_1\left(z_1\right)\hat{P}_1\left(w_1\right)\hat{P}_2\left(w_2\right)\right)\right)\hat{F}_2\left(w_1,w_2\right)\right)\\
&=&\tilde{\mu}_{2}^{2}R_{2}^{(2)}(1)+r_{2}\tilde{P}_{2}^{(2)}\left(1\right).
\end{eqnarray*}

%7
\item \begin{eqnarray*} &&\frac{\partial}{\partial
w_1}\frac{\partial}{\partial
z_2}\left(R_2\left(P_1\left(z_1\right)\bar{P}_2\left(z_2\right)\hat{P}_1\left(w_1\right)\hat{P}_2\left(w_2\right)\right)
F_2\left(z_1,\theta_2\left(P_1\left(z_1\right)\hat{P}_1\left(w_1\right)\hat{P}_2\left(w_2\right)\right)\right)\hat{F}_2\left(w_1,w_2\right)\right)\\
&=&\hat{\mu}_{1}\tilde{\mu}_{2}r_{2}+\hat{\mu}_{1}\tilde{\mu}_{2}R_{2}^{(2)}(1)+
r_{2}\frac{\hat{\mu}_{1}\tilde{\mu}_{2}}{1-\tilde{\mu}_{2}}F_{2}^{(0,1)}+r_{2}\tilde{\mu}_{2}\hat{F}_{2}^{(1,0)}.
\end{eqnarray*}
%8
\item \begin{eqnarray*} &&\frac{\partial}{\partial
w_2}\frac{\partial}{\partial
z_2}\left(R_2\left(P_1\left(z_1\right)\bar{P}_2\left(z_2\right)\hat{P}_1\left(w_1\right)\hat{P}_2\left(w_2\right)\right)
F_2\left(z_1,\theta_2\left(P_1\left(z_1\right)\hat{P}_1\left(w_1\right)\hat{P}_2\left(w_2\right)\right)\right)\hat{F}_2\left(w_1,w_2\right)\right)\\
&=&\hat{\mu}_{2}\tilde{\mu}_{2}r_{2}+\hat{\mu}_{2}\tilde{\mu}_{2}R_{2}^{(2)}(1)+
r_{2}\frac{\hat{\mu}_{2}\tilde{\mu}_{2}}{1-\tilde{\mu}_{2}}F_{2}^{(0,1)}+r_{2}\tilde{\mu}_{2}\hat{F}_{2}^{(0,1)}.
\end{eqnarray*}
%___________________________________________________________________________________________
%\subsubsection{Mixtas para $w_{1}$:}
%___________________________________________________________________________________________

%9
\item \begin{eqnarray*} &&\frac{\partial}{\partial
z_1}\frac{\partial}{\partial
w_1}\left(R_2\left(P_1\left(z_1\right)\bar{P}_2\left(z_2\right)\hat{P}_1\left(w_1\right)\hat{P}_2\left(w_2\right)\right)
F_2\left(z_1,\theta_2\left(P_1\left(z_1\right)\hat{P}_1\left(w_1\right)\hat{P}_2\left(w_2\right)\right)\right)\hat{F}_2\left(w_1,w_2\right)\right)\\
&=&\mu_{1}\hat{\mu}_{1}r_{2}+\mu_{1}\hat{\mu}_{1}R_{2}^{(2)}\left(1\right)+\frac{\mu_{1}\hat{\mu}_{1}}{1-\tilde{\mu}_{2}}F_{2}^{(0,1)}+r_{2}\frac{\mu_{1}\hat{\mu}_{1}}{1-\tilde{\mu}_{2}}F_{2}^{(0,1)}+\mu_{1}\hat{\mu}_{1}\tilde{\theta}_{2}^{(2)}\left(1\right)F_{2}^{(0,1)}\\
&+&r_{2}\hat{\mu}_{1}\left(\frac{\mu_{1}}{1-\tilde{\mu}_{2}}F_{2}^{(0,1)}+F_{2}^{(1,0)}\right)+r_{2}\mu_{1}\hat{F}_{2}^{(1,0)}
+\left(\frac{\mu_{1}}{1-\tilde{\mu}_{2}}F_{2}^{(0,1)}+F_{2}^{(1,0)}\right)\hat{F}_{2}^{(1,0)}\\
&+&\frac{\hat{\mu}_{1}}{1-\tilde{\mu}_{2}}\left(\frac{\mu_{1}}{1-\tilde{\mu}_{2}}F_{2}^{(0,2)}+F_{2}^{(1,1)}\right).
\end{eqnarray*}
%10
\item \begin{eqnarray*} &&\frac{\partial}{\partial
z_2}\frac{\partial}{\partial
w_1}\left(R_2\left(P_1\left(z_1\right)\bar{P}_2\left(z_2\right)\hat{P}_1\left(w_1\right)\hat{P}_2\left(w_2\right)\right)
F_2\left(z_1,\theta_2\left(P_1\left(z_1\right)\hat{P}_1\left(w_1\right)\hat{P}_2\left(w_2\right)\right)\right)\hat{F}_2\left(w_1,w_2\right)\right)\\
&=&\tilde{\mu}_{2}\hat{\mu}_{1}r_{2}+\tilde{\mu}_{2}\hat{\mu}_{1}R_{2}^{(2)}\left(1\right)+r_{2}\frac{\tilde{\mu}_{2}\hat{\mu}_{1}}{1-\tilde{\mu}_{2}}F_{2}^{(0,1)}
+r_{2}\tilde{\mu}_{2}\hat{F}_{2}^{(1,0)}.
\end{eqnarray*}
%11
\item \begin{eqnarray*} &&\frac{\partial}{\partial
w_1}\frac{\partial}{\partial
w_1}\left(R_2\left(P_1\left(z_1\right)\bar{P}_2\left(z_2\right)\hat{P}_1\left(w_1\right)\hat{P}_2\left(w_2\right)\right)
F_2\left(z_1,\theta_2\left(P_1\left(z_1\right)\hat{P}_1\left(w_1\right)\hat{P}_2\left(w_2\right)\right)\right)\hat{F}_2\left(w_1,w_2\right)\right)\\
&=&\hat{\mu}_{1}^{2}R_{2}^{(2)}\left(1\right)+r_{2}\hat{P}_{1}^{(2)}\left(1\right)+2r_{2}\frac{\hat{\mu}_{1}^{2}}{1-\tilde{\mu}_{2}}F_{2}^{(0,1)}+
\hat{\mu}_{1}^{2}\tilde{\theta}_{2}^{(2)}\left(1\right)F_{2}^{(0,1)}+\frac{1}{1-\tilde{\mu}_{2}}\hat{P}_{1}^{(2)}\left(1\right)F_{2}^{(0,1)}\\
&+&\frac{\hat{\mu}_{1}^{2}}{1-\tilde{\mu}_{2}}F_{2}^{(0,2)}+2r_{2}\hat{\mu}_{1}\hat{F}_{2}^{(1,0)}+2\frac{\hat{\mu}_{1}}{1-\tilde{\mu}_{2}}F_{2}^{(0,1)}\hat{F}_{2}^{(1,0)}+\hat{F}_{2}^{(2,0)}.
\end{eqnarray*}
%12
\item \begin{eqnarray*} &&\frac{\partial}{\partial
w_2}\frac{\partial}{\partial
w_1}\left(R_2\left(P_1\left(z_1\right)\bar{P}_2\left(z_2\right)\hat{P}_1\left(w_1\right)\hat{P}_2\left(w_2\right)\right)
F_2\left(z_1,\theta_2\left(P_1\left(z_1\right)\hat{P}_1\left(w_1\right)\hat{P}_2\left(w_2\right)\right)\right)\hat{F}_2\left(w_1,w_2\right)\right)\\
&=&r_{2}\hat{\mu}_{2}\hat{\mu}_{1}+\hat{\mu}_{1}\hat{\mu}_{2}R_{2}^{(2)}(1)+\frac{\hat{\mu}_{1}\hat{\mu}_{2}}{1-\tilde{\mu}_{2}}F_{2}^{(0,1)}
+2r_{2}\frac{\hat{\mu}_{1}\hat{\mu}_{2}}{1-\tilde{\mu}_{2}}F_{2}^{(0,1)}+\hat{\mu}_{2}\hat{\mu}_{1}\tilde{\theta}_{2}^{(2)}\left(1\right)F_{2}^{(0,1)}+
r_{2}\hat{\mu}_{1}\hat{F}_{2}^{(0,1)}\\
&+&\frac{\hat{\mu}_{1}}{1-\tilde{\mu}_{2}}F_{2}^{(0,1)}\hat{F}_{2}^{(0,1)}+\hat{\mu}_{1}\hat{\mu}_{2}\left(\frac{1}{1-\tilde{\mu}_{2}}\right)^{2}F_{2}^{(0,2)}+
r_{2}\hat{\mu}_{2}\hat{F}_{2}^{(1,0)}+\frac{\hat{\mu}_{2}}{1-\tilde{\mu}_{2}}F_{2}^{(0,1)}\hat{F}_{2}^{(1,0)}+\hat{F}_{2}^{(1,1)}.
\end{eqnarray*}
%___________________________________________________________________________________________
%\subsubsection{Mixtas para $w_{2}$:}
%___________________________________________________________________________________________
%13

\item \begin{eqnarray*} &&\frac{\partial}{\partial
z_1}\frac{\partial}{\partial
w_2}\left(R_2\left(P_1\left(z_1\right)\bar{P}_2\left(z_2\right)\hat{P}_1\left(w_1\right)\hat{P}_2\left(w_2\right)\right)
F_2\left(z_1,\theta_2\left(P_1\left(z_1\right)\hat{P}_1\left(w_1\right)\hat{P}_2\left(w_2\right)\right)\right)\hat{F}_2\left(w_1,w_2\right)\right)\\
&=&r_{2}\mu_{1}\hat{\mu}_{2}+\mu_{1}\hat{\mu}_{2}R_{2}^{(2)}(1)+\frac{\mu_{1}\hat{\mu}_{2}}{1-\tilde{\mu}_{2}}F_{2}^{(0,1)}+r_{2}\frac{\mu_{1}\hat{\mu}_{2}}{1-\tilde{\mu}_{2}}F_{2}^{(0,1)}+\mu_{1}\hat{\mu}_{2}\tilde{\theta}_{2}^{(2)}\left(1\right)F_{2}^{(0,1)}+r_{2}\mu_{1}\hat{F}_{2}^{(0,1)}\\
&+&r_{2}\hat{\mu}_{2}\left(\frac{\mu_{1}}{1-\tilde{\mu}_{2}}F_{2}^{(0,1)}+F_{2}^{(1,0)}\right)+\hat{F}_{2}^{(0,1)}\left(\frac{\mu_{1}}{1-\tilde{\mu}_{2}}F_{2}^{(0,1)}+F_{2}^{(1,0)}\right)+\frac{\hat{\mu}_{2}}{1-\tilde{\mu}_{2}}\left(\frac{\mu_{1}}{1-\tilde{\mu}_{2}}F_{2}^{(0,2)}+F_{2}^{(1,1)}\right).
\end{eqnarray*}
%14
\item \begin{eqnarray*} &&\frac{\partial}{\partial
z_2}\frac{\partial}{\partial
w_2}\left(R_2\left(P_1\left(z_1\right)\bar{P}_2\left(z_2\right)\hat{P}_1\left(w_1\right)\hat{P}_2\left(w_2\right)\right)
F_2\left(z_1,\theta_2\left(P_1\left(z_1\right)\hat{P}_1\left(w_1\right)\hat{P}_2\left(w_2\right)\right)\right)\hat{F}_2\left(w_1,w_2\right)\right)\\
&=&r_{2}\tilde{\mu}_{2}\hat{\mu}_{2}+\tilde{\mu}_{2}\hat{\mu}_{2}R_{2}^{(2)}(1)+r_{2}\frac{\tilde{\mu}_{2}\hat{\mu}_{2}}{1-\tilde{\mu}_{2}}F_{2}^{(0,1)}+r_{2}\tilde{\mu}_{2}\hat{F}_{2}^{(0,1)}.
\end{eqnarray*}
%15
\item \begin{eqnarray*} &&\frac{\partial}{\partial
w_1}\frac{\partial}{\partial
w_2}\left(R_2\left(P_1\left(z_1\right)\bar{P}_2\left(z_2\right)\hat{P}_1\left(w_1\right)\hat{P}_2\left(w_2\right)\right)
F_2\left(z_1,\theta_2\left(P_1\left(z_1\right)\hat{P}_1\left(w_1\right)\hat{P}_2\left(w_2\right)\right)\right)\hat{F}_2\left(w_1,w_2\right)\right)\\
&=&r_{2}\hat{\mu}_{1}\hat{\mu}_{2}+\hat{\mu}_{1}\hat{\mu}_{2}R_{2}^{(2)}\left(1\right)+\frac{\hat{\mu}_{1}\hat{\mu}_{2}}{1-\tilde{\mu}_{2}}F_{2}^{(0,1)}+2r_{2}\frac{\hat{\mu}_{1}\hat{\mu}_{2}}{1-\tilde{\mu}_{2}}F_{2}^{(0,1)}+\hat{\mu}_{1}\hat{\mu}_{2}\theta_{2}^{(2)}\left(1\right)F_{2}^{(0,1)}+r_{2}\hat{\mu}_{1}\hat{F}_{2}^{(0,1)}\\
&+&\frac{\hat{\mu}_{1}}{1-\tilde{\mu}_{2}}F_{2}^{(0,1)}\hat{F}_{2}^{(0,1)}+\hat{\mu}_{1}\hat{\mu}_{2}\left(\frac{1}{1-\tilde{\mu}_{2}}\right)^{2}F_{2}^{(0,2)}+r_{2}\hat{\mu}_{2}\hat{F}_{2}^{(0,1)}+\frac{\hat{\mu}_{2}}{1-\tilde{\mu}_{2}}F_{2}^{(0,1)}\hat{F}_{2}^{(1,0)}+\hat{F}_{2}^{(1,1)}.
\end{eqnarray*}
%16

\item \begin{eqnarray*} &&\frac{\partial}{\partial
w_2}\frac{\partial}{\partial
w_2}\left(R_2\left(P_1\left(z_1\right)\bar{P}_2\left(z_2\right)\hat{P}_1\left(w_1\right)\hat{P}_2\left(w_2\right)\right)
F_2\left(z_1,\theta_2\left(P_1\left(z_1\right)\hat{P}_1\left(w_1\right)\hat{P}_2\left(w_2\right)\right)\right)\hat{F}_2\left(w_1,w_2\right)\right)\\
&=&\hat{\mu}_{2}^{2}R_{2}^{(2)}(1)+r_{2}\hat{P}_{2}^{(2)}\left(1\right)+2r_{2}\frac{\hat{\mu}_{2}^{2}}{1-\tilde{\mu}_{2}}F_{2}^{(0,1)}+\hat{\mu}_{2}^{2}\tilde{\theta}_{2}^{(2)}\left(1\right)F_{2}^{(0,1)}+\frac{1}{1-\tilde{\mu}_{2}}\hat{P}_{2}^{(2)}\left(1\right)F_{2}^{(0,1)}\\
&+&2r_{2}\hat{\mu}_{2}\hat{F}_{2}^{(0,1)}+2\frac{\hat{\mu}_{2}}{1-\tilde{\mu}_{2}}F_{2}^{(0,1)}\hat{F}_{2}^{(0,1)}+\left(\frac{\hat{\mu}_{2}}{1-\tilde{\mu}_{2}}\right)^{2}F_{2}^{(0,2)}+\hat{F}_{2}^{(0,2)}.
\end{eqnarray*}
\end{enumerate}
%___________________________________________________________________________________________
%
%\subsection{Derivadas de Segundo Orden para $F_{2}$}
%___________________________________________________________________________________________


\begin{enumerate}

%___________________________________________________________________________________________
%\subsubsection{Mixtas para $z_{1}$:}
%___________________________________________________________________________________________

%1/17
\item \begin{eqnarray*} &&\frac{\partial}{\partial
z_1}\frac{\partial}{\partial
z_1}\left(R_1\left(P_1\left(z_1\right)\bar{P}_2\left(z_2\right)\hat{P}_1\left(w_1\right)\hat{P}_2\left(w_2\right)\right)
F_1\left(\theta_1\left(\tilde{P}_2\left(z_1\right)\hat{P}_1\left(w_1\right)\hat{P}_2\left(w_2\right)\right)\right)\hat{F}_1\left(w_1,w_2\right)\right)\\
&=&r_{1}P_{1}^{(2)}\left(1\right)+\mu_{1}^{2}R_{1}^{(2)}\left(1\right).
\end{eqnarray*}

%2/18
\item \begin{eqnarray*} &&\frac{\partial}{\partial
z_2}\frac{\partial}{\partial
z_1}\left(R_1\left(P_1\left(z_1\right)\bar{P}_2\left(z_2\right)\hat{P}_1\left(w_1\right)\hat{P}_2\left(w_2\right)\right)F_1\left(\theta_1\left(\tilde{P}_2\left(z_1\right)\hat{P}_1\left(w_1\right)\hat{P}_2\left(w_2\right)\right)\right)\hat{F}_1\left(w_1,w_2\right)\right)\\
&=&\mu_{1}\tilde{\mu}_{2}r_{1}+\mu_{1}\tilde{\mu}_{2}R_{1}^{(2)}(1)+
r_{1}\mu_{1}\left(\frac{\tilde{\mu}_{2}}{1-\mu_{1}}F_{1}^{(1,0)}+F_{1}^{(0,1)}\right).
\end{eqnarray*}

%3/19
\item \begin{eqnarray*} &&\frac{\partial}{\partial
w_1}\frac{\partial}{\partial
z_1}\left(R_1\left(P_1\left(z_1\right)\bar{P}_2\left(z_2\right)\hat{P}_1\left(w_1\right)\hat{P}_2\left(w_2\right)\right)F_1\left(\theta_1\left(\tilde{P}_2\left(z_1\right)\hat{P}_1\left(w_1\right)\hat{P}_2\left(w_2\right)\right)\right)\hat{F}_1\left(w_1,w_2\right)\right)\\
&=&r_{1}\mu_{1}\hat{\mu}_{1}+\mu_{1}\hat{\mu}_{1}R_{1}^{(2)}\left(1\right)+r_{1}\frac{\mu_{1}\hat{\mu}_{1}}{1-\mu_{1}}F_{1}^{(1,0)}+r_{1}\mu_{1}\hat{F}_{1}^{(1,0)}.
\end{eqnarray*}
%4/20
\item \begin{eqnarray*} &&\frac{\partial}{\partial
w_2}\frac{\partial}{\partial
z_1}\left(R_1\left(P_1\left(z_1\right)\bar{P}_2\left(z_2\right)\hat{P}_1\left(w_1\right)\hat{P}_2\left(w_2\right)\right)F_1\left(\theta_1\left(\tilde{P}_2\left(z_1\right)\hat{P}_1\left(w_1\right)\hat{P}_2\left(w_2\right)\right)\right)\hat{F}_1\left(w_1,w_2\right)\right)\\
&=&\mu_{1}\hat{\mu}_{2}r_{1}+\mu_{1}\hat{\mu}_{2}R_{1}^{(2)}\left(1\right)+r_{1}\mu_{1}\hat{F}_{1}^{(0,1)}+r_{1}\frac{\mu_{1}\hat{\mu}_{2}}{1-\mu_{1}}F_{1}^{(1,0)}.
\end{eqnarray*}
%___________________________________________________________________________________________
%\subsubsection{Mixtas para $z_{2}$:}
%___________________________________________________________________________________________
%5/21
\item \begin{eqnarray*}
&&\frac{\partial}{\partial z_1}\frac{\partial}{\partial z_2}\left(R_1\left(P_1\left(z_1\right)\bar{P}_2\left(z_2\right)\hat{P}_1\left(w_1\right)\hat{P}_2\left(w_2\right)\right)F_1\left(\theta_1\left(\tilde{P}_2\left(z_1\right)\hat{P}_1\left(w_1\right)\hat{P}_2\left(w_2\right)\right)\right)\hat{F}_1\left(w_1,w_2\right)\right)\\
&=&r_{1}\mu_{1}\tilde{\mu}_{2}+\mu_{1}\tilde{\mu}_{2}R_{1}^{(2)}\left(1\right)+r_{1}\mu_{1}\left(\frac{\tilde{\mu}_{2}}{1-\mu_{1}}F_{1}^{(1,0)}+F_{1}^{(0,1)}\right).
\end{eqnarray*}

%6/22
\item \begin{eqnarray*}
&&\frac{\partial}{\partial z_2}\frac{\partial}{\partial z_2}\left(R_1\left(P_1\left(z_1\right)\bar{P}_2\left(z_2\right)\hat{P}_1\left(w_1\right)\hat{P}_2\left(w_2\right)\right)F_1\left(\theta_1\left(\tilde{P}_2\left(z_1\right)\hat{P}_1\left(w_1\right)\hat{P}_2\left(w_2\right)\right)\right)\hat{F}_1\left(w_1,w_2\right)\right)\\
&=&\tilde{\mu}_{2}^{2}R_{1}^{(2)}\left(1\right)+r_{1}\tilde{P}_{2}^{(2)}\left(1\right)+2r_{1}\tilde{\mu}_{2}\left(\frac{\tilde{\mu}_{2}}{1-\mu_{1}}F_{1}^{(1,0)}+F_{1}^{(0,1)}\right)+F_{1}^{(0,2)}+\tilde{\mu}_{2}^{2}\theta_{1}^{(2)}\left(1\right)F_{1}^{(1,0)}\\
&+&\frac{1}{1-\mu_{1}}\tilde{P}_{2}^{(2)}\left(1\right)F_{1}^{(1,0)}+\frac{\tilde{\mu}_{2}}{1-\mu_{1}}F_{1}^{(1,1)}+\frac{\tilde{\mu}_{2}}{1-\mu_{1}}\left(\frac{\tilde{\mu}_{2}}{1-\mu_{1}}F_{1}^{(2,0)}+F_{1}^{(1,1)}\right).
\end{eqnarray*}
%7/23
\item \begin{eqnarray*}
&&\frac{\partial}{\partial w_1}\frac{\partial}{\partial z_2}\left(R_1\left(P_1\left(z_1\right)\bar{P}_2\left(z_2\right)\hat{P}_1\left(w_1\right)\hat{P}_2\left(w_2\right)\right)F_1\left(\theta_1\left(\tilde{P}_2\left(z_1\right)\hat{P}_1\left(w_1\right)\hat{P}_2\left(w_2\right)\right)\right)\hat{F}_1\left(w_1,w_2\right)\right)\\
&=&\tilde{\mu}_{2}\hat{\mu}_{1}r_{1}+\tilde{\mu}_{2}\hat{\mu}_{1}R_{1}^{(2)}\left(1\right)+r_{1}\frac{\tilde{\mu}_{2}\hat{\mu}_{1}}{1-\mu_{1}}F_{1}^{(1,0)}+\hat{\mu}_{1}r_{1}\left(\frac{\tilde{\mu}_{2}}{1-\mu_{1}}F_{1}^{(1,0)}+F_{1}^{(0,1)}\right)+r_{1}\tilde{\mu}_{2}\hat{F}_{1}^{(1,0)}\\
&+&\left(\frac{\tilde{\mu}_{2}}{1-\mu_{1}}F_{1}^{(1,0)}+F_{1}^{(0,1)}\right)\hat{F}_{1}^{(1,0)}+\frac{\tilde{\mu}_{2}\hat{\mu}_{1}}{1-\mu_{1}}F_{1}^{(1,0)}+\tilde{\mu}_{2}\hat{\mu}_{1}\theta_{1}^{(2)}\left(1\right)F_{1}^{(1,0)}+\frac{\hat{\mu}_{1}}{1-\mu_{1}}F_{1}^{(1,1)}\\
&+&\left(\frac{1}{1-\mu_{1}}\right)^{2}\tilde{\mu}_{2}\hat{\mu}_{1}F_{1}^{(2,0)}.
\end{eqnarray*}
%8/24
\item \begin{eqnarray*}
&&\frac{\partial}{\partial w_2}\frac{\partial}{\partial z_2}\left(R_1\left(P_1\left(z_1\right)\bar{P}_2\left(z_2\right)\hat{P}_1\left(w_1\right)\hat{P}_2\left(w_2\right)\right)F_1\left(\theta_1\left(\tilde{P}_2\left(z_1\right)\hat{P}_1\left(w_1\right)\hat{P}_2\left(w_2\right)\right)\right)\hat{F}_1\left(w_1,w_2\right)\right)\\
&=&\hat{\mu}_{2}\tilde{\mu}_{2}r_{1}+\hat{\mu}_{2}\tilde{\mu}_{2}R_{1}^{(2)}(1)+r_{1}\tilde{\mu}_{2}\hat{F}_{1}^{(0,1)}+r_{1}\frac{\hat{\mu}_{2}\tilde{\mu}_{2}}{1-\mu_{1}}F_{1}^{(1,0)}+\hat{\mu}_{2}r_{1}\left(\frac{\tilde{\mu}_{2}}{1-\mu_{1}}F_{1}^{(1,0)}+F_{1}^{(0,1)}\right)\\
&+&\left(\frac{\tilde{\mu}_{2}}{1-\mu_{1}}F_{1}^{(1,0)}+F_{1}^{(0,1)}\right)\hat{F}_{1}^{(0,1)}+\frac{\tilde{\mu}_{2}\hat{\mu_{2}}}{1-\mu_{1}}F_{1}^{(1,0)}+\hat{\mu}_{2}\tilde{\mu}_{2}\theta_{1}^{(2)}\left(1\right)F_{1}^{(1,0)}+\frac{\hat{\mu}_{2}}{1-\mu_{1}}F_{1}^{(1,1)}\\
&+&\left(\frac{1}{1-\mu_{1}}\right)^{2}\tilde{\mu}_{2}\hat{\mu}_{2}F_{1}^{(2,0)}.
\end{eqnarray*}
%___________________________________________________________________________________________
%\subsubsection{Mixtas para $w_{1}$:}
%___________________________________________________________________________________________
%9/25
\item \begin{eqnarray*} &&\frac{\partial}{\partial
z_1}\frac{\partial}{\partial
w_1}\left(R_1\left(P_1\left(z_1\right)\bar{P}_2\left(z_2\right)\hat{P}_1\left(w_1\right)\hat{P}_2\left(w_2\right)\right)F_1\left(\theta_1\left(\tilde{P}_2\left(z_1\right)\hat{P}_1\left(w_1\right)\hat{P}_2\left(w_2\right)\right)\right)\hat{F}_1\left(w_1,w_2\right)\right)\\
&=&r_{1}\mu_{1}\hat{\mu}_{1}+\mu_{1}\hat{\mu}_{1}R_{1}^{(2)}(1)+r_{1}\frac{\mu_{1}\hat{\mu}_{1}}{1-\mu_{1}}F_{1}^{(1,0)}+r_{1}\mu_{1}\hat{F}_{1}^{(1,0)}.
\end{eqnarray*}
%10/26
\item \begin{eqnarray*} &&\frac{\partial}{\partial
z_2}\frac{\partial}{\partial
w_1}\left(R_1\left(P_1\left(z_1\right)\bar{P}_2\left(z_2\right)\hat{P}_1\left(w_1\right)\hat{P}_2\left(w_2\right)\right)F_1\left(\theta_1\left(\tilde{P}_2\left(z_1\right)\hat{P}_1\left(w_1\right)\hat{P}_2\left(w_2\right)\right)\right)\hat{F}_1\left(w_1,w_2\right)\right)\\
&=&r_{1}\hat{\mu}_{1}\tilde{\mu}_{2}+\tilde{\mu}_{2}\hat{\mu}_{1}R_{1}^{(2)}\left(1\right)+
\frac{\hat{\mu}_{1}\tilde{\mu}_{2}}{1-\mu_{1}}F_{1}^{1,0)}+r_{1}\frac{\hat{\mu}_{1}\tilde{\mu}_{2}}{1-\mu_{1}}F_{1}^{(1,0)}+\hat{\mu}_{1}\tilde{\mu}_{2}\theta_{1}^{(2)}\left(1\right)F_{2}^{(1,0)}\\
&+&r_{1}\hat{\mu}_{1}\left(F_{1}^{(1,0)}+\frac{\tilde{\mu}_{2}}{1-\mu_{1}}F_{1}^{1,0)}\right)+
r_{1}\tilde{\mu}_{2}\hat{F}_{1}^{(1,0)}+\left(F_{1}^{(0,1)}+\frac{\tilde{\mu}_{2}}{1-\mu_{1}}F_{1}^{1,0)}\right)\hat{F}_{1}^{(1,0)}\\
&+&\frac{\hat{\mu}_{1}}{1-\mu_{1}}\left(F_{1}^{(1,1)}+\frac{\tilde{\mu}_{2}}{1-\mu_{1}}F_{1}^{2,0)}\right).
\end{eqnarray*}
%11/27
\item \begin{eqnarray*} &&\frac{\partial}{\partial
w_1}\frac{\partial}{\partial
w_1}\left(R_1\left(P_1\left(z_1\right)\bar{P}_2\left(z_2\right)\hat{P}_1\left(w_1\right)\hat{P}_2\left(w_2\right)\right)F_1\left(\theta_1\left(\tilde{P}_2\left(z_1\right)\hat{P}_1\left(w_1\right)\hat{P}_2\left(w_2\right)\right)\right)\hat{F}_1\left(w_1,w_2\right)\right)\\
&=&\hat{\mu}_{1}^{2}R_{1}^{(2)}\left(1\right)+r_{1}\hat{P}_{1}^{(2)}\left(1\right)+2r_{1}\frac{\hat{\mu}_{1}^{2}}{1-\mu_{1}}F_{1}^{(1,0)}+\hat{\mu}_{1}^{2}\theta_{1}^{(2)}\left(1\right)F_{1}^{(1,0)}+\frac{1}{1-\mu_{1}}\hat{P}_{1}^{(2)}\left(1\right)F_{1}^{(1,0)}\\
&+&2r_{1}\hat{\mu}_{1}\hat{F}_{1}^{(1,0)}+2\frac{\hat{\mu}_{1}}{1-\mu_{1}}F_{1}^{(1,0)}\hat{F}_{1}^{(1,0)}+\left(\frac{\hat{\mu}_{1}}{1-\mu_{1}}\right)^{2}F_{1}^{(2,0)}+\hat{F}_{1}^{(2,0)}.
\end{eqnarray*}
%12/28
\item \begin{eqnarray*} &&\frac{\partial}{\partial
w_2}\frac{\partial}{\partial
w_1}\left(R_1\left(P_1\left(z_1\right)\bar{P}_2\left(z_2\right)\hat{P}_1\left(w_1\right)\hat{P}_2\left(w_2\right)\right)F_1\left(\theta_1\left(\tilde{P}_2\left(z_1\right)\hat{P}_1\left(w_1\right)\hat{P}_2\left(w_2\right)\right)\right)\hat{F}_1\left(w_1,w_2\right)\right)\\
&=&r_{1}\hat{\mu}_{1}\hat{\mu}_{2}+\hat{\mu}_{1}\hat{\mu}_{2}R_{1}^{(2)}\left(1\right)+r_{1}\hat{\mu}_{1}\hat{F}_{1}^{(0,1)}+
\frac{\hat{\mu}_{1}\hat{\mu}_{2}}{1-\mu_{1}}F_{1}^{(1,0)}+2r_{1}\frac{\hat{\mu}_{1}\hat{\mu}_{2}}{1-\mu_{1}}F_{1}^{1,0)}+\hat{\mu}_{1}\hat{\mu}_{2}\theta_{1}^{(2)}\left(1\right)F_{1}^{(1,0)}\\
&+&\frac{\hat{\mu}_{1}}{1-\mu_{1}}F_{1}^{(1,0)}\hat{F}_{1}^{(0,1)}+
r_{1}\hat{\mu}_{2}\hat{F}_{1}^{(1,0)}+\frac{\hat{\mu}_{2}}{1-\mu_{1}}\hat{F}_{1}^{(1,0)}F_{1}^{(1,0)}+\hat{F}_{1}^{(1,1)}+\hat{\mu}_{1}\hat{\mu}_{2}\left(\frac{1}{1-\mu_{1}}\right)^{2}F_{1}^{(2,0)}.
\end{eqnarray*}
%___________________________________________________________________________________________
%\subsubsection{Mixtas para $w_{2}$:}
%___________________________________________________________________________________________
%13/29
\item \begin{eqnarray*} &&\frac{\partial}{\partial
z_1}\frac{\partial}{\partial
w_2}\left(R_1\left(P_1\left(z_1\right)\bar{P}_2\left(z_2\right)\hat{P}_1\left(w_1\right)\hat{P}_2\left(w_2\right)\right)F_1\left(\theta_1\left(\tilde{P}_2\left(z_1\right)\hat{P}_1\left(w_1\right)\hat{P}_2\left(w_2\right)\right)\right)\hat{F}_1\left(w_1,w_2\right)\right)\\
&=&r_{1}\mu_{1}\hat{\mu}_{2}+\mu_{1}\hat{\mu}_{2}R_{1}^{(2)}\left(1\right)+r_{1}\mu_{1}\hat{F}_{1}^{(0,1)}+r_{1}\frac{\mu_{1}\hat{\mu}_{2}}{1-\mu_{1}}F_{1}^{(1,0)}.
\end{eqnarray*}
%14/30
\item \begin{eqnarray*} &&\frac{\partial}{\partial
z_2}\frac{\partial}{\partial
w_2}\left(R_1\left(P_1\left(z_1\right)\bar{P}_2\left(z_2\right)\hat{P}_1\left(w_1\right)\hat{P}_2\left(w_2\right)\right)F_1\left(\theta_1\left(\tilde{P}_2\left(z_1\right)\hat{P}_1\left(w_1\right)\hat{P}_2\left(w_2\right)\right)\right)\hat{F}_1\left(w_1,w_2\right)\right)\\
&=&r_{1}\hat{\mu}_{2}\tilde{\mu}_{2}+\hat{\mu}_{2}\tilde{\mu}_{2}R_{1}^{(2)}\left(1\right)+r_{1}\tilde{\mu}_{2}\hat{F}_{1}^{(0,1)}+\frac{\hat{\mu}_{2}\tilde{\mu}_{2}}{1-\mu_{1}}F_{1}^{(1,0)}+r_{1}\frac{\hat{\mu}_{2}\tilde{\mu}_{2}}{1-\mu_{1}}F_{1}^{(1,0)}\\
&+&\hat{\mu}_{2}\tilde{\mu}_{2}\theta_{1}^{(2)}\left(1\right)F_{1}^{(1,0)}+r_{1}\hat{\mu}_{2}\left(F_{1}^{(0,1)}+\frac{\tilde{\mu}_{2}}{1-\mu_{1}}F_{1}^{(1,0)}\right)+\left(F_{1}^{(0,1)}+\frac{\tilde{\mu}_{2}}{1-\mu_{1}}F_{1}^{(1,0)}\right)\hat{F}_{1}^{(0,1)}\\&+&\frac{\hat{\mu}_{2}}{1-\mu_{1}}\left(F_{1}^{(1,1)}+\frac{\tilde{\mu}_{2}}{1-\mu_{1}}F_{1}^{(2,0)}\right).
\end{eqnarray*}
%15/31
\item \begin{eqnarray*} &&\frac{\partial}{\partial
w_1}\frac{\partial}{\partial
w_2}\left(R_1\left(P_1\left(z_1\right)\bar{P}_2\left(z_2\right)\hat{P}_1\left(w_1\right)\hat{P}_2\left(w_2\right)\right)F_1\left(\theta_1\left(\tilde{P}_2\left(z_1\right)\hat{P}_1\left(w_1\right)\hat{P}_2\left(w_2\right)\right)\right)\hat{F}_1\left(w_1,w_2\right)\right)\\
&=&r_{1}\hat{\mu}_{1}\hat{\mu}_{2}+\hat{\mu}_{1}\hat{\mu}_{2}R_{1}^{(2)}\left(1\right)+r_{1}\hat{\mu}_{1}\hat{F}_{1}^{(0,1)}+
\frac{\hat{\mu}_{1}\hat{\mu}_{2}}{1-\mu_{1}}F_{1}^{(1,0)}+2r_{1}\frac{\hat{\mu}_{1}\hat{\mu}_{2}}{1-\mu_{1}}F_{1}^{(1,0)}+\hat{\mu}_{1}\hat{\mu}_{2}\theta_{1}^{(2)}\left(1\right)F_{1}^{(1,0)}\\
&+&\frac{\hat{\mu}_{1}}{1-\mu_{1}}\hat{F}_{1}^{(0,1)}F_{1}^{(1,0)}+r_{1}\hat{\mu}_{2}\hat{F}_{1}^{(1,0)}+\frac{\hat{\mu}_{2}}{1-\mu_{1}}\hat{F}_{1}^{(1,0)}F_{1}^{(1,0)}+\hat{F}_{1}^{(1,1)}+\hat{\mu}_{1}\hat{\mu}_{2}\left(\frac{1}{1-\mu_{1}}\right)^{2}F_{1}^{(2,0)}.
\end{eqnarray*}
%16/32
\item \begin{eqnarray*} &&\frac{\partial}{\partial
w_2}\frac{\partial}{\partial
w_2}\left(R_1\left(P_1\left(z_1\right)\bar{P}_2\left(z_2\right)\hat{P}_1\left(w_1\right)\hat{P}_2\left(w_2\right)\right)F_1\left(\theta_1\left(\tilde{P}_2\left(z_1\right)\hat{P}_1\left(w_1\right)\hat{P}_2\left(w_2\right)\right)\right)\hat{F}_1\left(w_1,w_2\right)\right)\\
&=&\hat{\mu}_{2}R_{1}^{(2)}\left(1\right)+r_{1}\hat{P}_{2}^{(2)}\left(1\right)+2r_{1}\hat{\mu}_{2}\hat{F}_{1}^{(0,1)}+\hat{F}_{1}^{(0,2)}+2r_{1}\frac{\hat{\mu}_{2}^{2}}{1-\mu_{1}}F_{1}^{(1,0)}+\hat{\mu}_{2}^{2}\theta_{1}^{(2)}\left(1\right)F_{1}^{(1,0)}\\
&+&\frac{1}{1-\mu_{1}}\hat{P}_{2}^{(2)}\left(1\right)F_{1}^{(1,0)} +
2\frac{\hat{\mu}_{2}}{1-\mu_{1}}F_{1}^{(1,0)}\hat{F}_{1}^{(0,1)}+\left(\frac{\hat{\mu}_{2}}{1-\mu_{1}}\right)^{2}F_{1}^{(2,0)}.
\end{eqnarray*}
\end{enumerate}

%___________________________________________________________________________________________
%
%\subsection{Derivadas de Segundo Orden para $\hat{F}_{1}$}
%___________________________________________________________________________________________


\begin{enumerate}
%___________________________________________________________________________________________
%\subsubsection{Mixtas para $z_{1}$:}
%___________________________________________________________________________________________
%1/33

\item \begin{eqnarray*} &&\frac{\partial}{\partial
z_1}\frac{\partial}{\partial
z_1}\left(\hat{R}_{2}\left(P_{1}\left(z_{1}\right)\tilde{P}_{2}\left(z_{2}\right)\hat{P}_{1}\left(w_{1}\right)\hat{P}_{2}\left(w_{2}\right)\right)\hat{F}_{2}\left(w_{1},\hat{\theta}_{2}\left(P_{1}\left(z_{1}\right)\tilde{P}_{2}\left(z_{2}\right)\hat{P}_{1}\left(w_{1}\right)\right)\right)F_{2}\left(z_{1},z_{2}\right)\right)\\
&=&\hat{r}_{2}P_{1}^{(2)}\left(1\right)+
\mu_{1}^{2}\hat{R}_{2}^{(2)}\left(1\right)+
2\hat{r}_{2}\frac{\mu_{1}^{2}}{1-\hat{\mu}_{2}}\hat{F}_{2}^{(0,1)}+
\frac{1}{1-\hat{\mu}_{2}}P_{1}^{(2)}\left(1\right)\hat{F}_{2}^{(0,1)}+
\mu_{1}^{2}\hat{\theta}_{2}^{(2)}\left(1\right)\hat{F}_{2}^{(0,1)}\\
&+&\left(\frac{\mu_{1}^{2}}{1-\hat{\mu}_{2}}\right)^{2}\hat{F}_{2}^{(0,2)}+
2\hat{r}_{2}\mu_{1}F_{2}^{(1,0)}+2\frac{\mu_{1}}{1-\hat{\mu}_{2}}\hat{F}_{2}^{(0,1)}F_{2}^{(1,0)}+F_{2}^{(2,0)}.
\end{eqnarray*}

%2/34
\item \begin{eqnarray*} &&\frac{\partial}{\partial
z_2}\frac{\partial}{\partial
z_1}\left(\hat{R}_{2}\left(P_{1}\left(z_{1}\right)\tilde{P}_{2}\left(z_{2}\right)\hat{P}_{1}\left(w_{1}\right)\hat{P}_{2}\left(w_{2}\right)\right)\hat{F}_{2}\left(w_{1},\hat{\theta}_{2}\left(P_{1}\left(z_{1}\right)\tilde{P}_{2}\left(z_{2}\right)\hat{P}_{1}\left(w_{1}\right)\right)\right)F_{2}\left(z_{1},z_{2}\right)\right)\\
&=&\hat{r}_{2}\mu_{1}\tilde{\mu}_{2}+\mu_{1}\tilde{\mu}_{2}\hat{R}_{2}^{(2)}\left(1\right)+\hat{r}_{2}\mu_{1}F_{2}^{(0,1)}+
\frac{\mu_{1}\tilde{\mu}_{2}}{1-\hat{\mu}_{2}}\hat{F}_{2}^{(0,1)}+2\hat{r}_{2}\frac{\mu_{1}\tilde{\mu}_{2}}{1-\hat{\mu}_{2}}\hat{F}_{2}^{(0,1)}+\mu_{1}\tilde{\mu}_{2}\hat{\theta}_{2}^{(2)}\left(1\right)\hat{F}_{2}^{(0,1)}\\
&+&\frac{\mu_{1}}{1-\hat{\mu}_{2}}F_{2}^{(0,1)}\hat{F}_{2}^{(0,1)}+\mu_{1} \tilde{\mu}_{2}\left(\frac{1}{1-\hat{\mu}_{2}}\right)^{2}\hat{F}_{2}^{(0,2)}+\hat{r}_{2}\tilde{\mu}_{2}F_{2}^{(1,0)}+\frac{\tilde{\mu}_{2}}{1-\hat{\mu}_{2}}\hat{F}_{2}^{(0,1)}F_{2}^{(1,0)}+F_{2}^{(1,1)}.
\end{eqnarray*}


%3/35

\item \begin{eqnarray*} &&\frac{\partial}{\partial
w_1}\frac{\partial}{\partial
z_1}\left(\hat{R}_{2}\left(P_{1}\left(z_{1}\right)\tilde{P}_{2}\left(z_{2}\right)\hat{P}_{1}\left(w_{1}\right)\hat{P}_{2}\left(w_{2}\right)\right)\hat{F}_{2}\left(w_{1},\hat{\theta}_{2}\left(P_{1}\left(z_{1}\right)\tilde{P}_{2}\left(z_{2}\right)\hat{P}_{1}\left(w_{1}\right)\right)\right)F_{2}\left(z_{1},z_{2}\right)\right)\\
&=&\hat{r}_{2}\mu_{1}\hat{\mu}_{1}+\mu_{1}\hat{\mu}_{1}\hat{R}_{2}^{(2)}\left(1\right)+\hat{r}_{2}\frac{\mu_{1}\hat{\mu}_{1}}{1-\hat{\mu}_{2}}\hat{F}_{2}^{(0,1)}+\hat{r}_{2}\hat{\mu}_{1}F_{2}^{(1,0)}+\hat{r}_{2}\mu_{1}\hat{F}_{2}^{(1,0)}+F_{2}^{(1,0)}\hat{F}_{2}^{(1,0)}+\frac{\mu_{1}}{1-\hat{\mu}_{2}}\hat{F}_{2}^{(1,1)}.
\end{eqnarray*}

%4/36

\item \begin{eqnarray*} &&\frac{\partial}{\partial
w_2}\frac{\partial}{\partial
z_1}\left(\hat{R}_{2}\left(P_{1}\left(z_{1}\right)\tilde{P}_{2}\left(z_{2}\right)\hat{P}_{1}\left(w_{1}\right)\hat{P}_{2}\left(w_{2}\right)\right)\hat{F}_{2}\left(w_{1},\hat{\theta}_{2}\left(P_{1}\left(z_{1}\right)\tilde{P}_{2}\left(z_{2}\right)\hat{P}_{1}\left(w_{1}\right)\right)\right)F_{2}\left(z_{1},z_{2}\right)\right)\\
&=&\hat{r}_{2}\mu_{1}\hat{\mu}_{2}+\mu_{1}\hat{\mu}_{2}\hat{R}_{2}^{(2)}\left(1\right)+\frac{\mu_{1}\hat{\mu}_{2}}{1-\hat{\mu}_{2}}\hat{F}_{2}^{(0,1)}+2\hat{r}_{2}\frac{\mu_{1}\hat{\mu}_{2}}{1-\hat{\mu}_{2}}\hat{F}_{2}^{(0,1)}+\mu_{1}\hat{\mu}_{2}\hat{\theta}_{2}^{(2)}\left(1\right)\hat{F}_{2}^{(0,1)}\\
&+&\mu_{1}\hat{\mu}_{2}\left(\frac{1}{1-\hat{\mu}_{2}}\right)^{2}\hat{F}_{2}^{(0,2)}+\hat{r}_{2}\hat{\mu}_{2}F_{2}^{(1,0)}+\frac{\hat{\mu}_{2}}{1-\hat{\mu}_{2}}\hat{F}_{2}^{(0,1)}F_{2}^{(1,0)}.
\end{eqnarray*}
%___________________________________________________________________________________________
%\subsubsection{Mixtas para $z_{2}$:}
%___________________________________________________________________________________________

%5/37

\item \begin{eqnarray*} &&\frac{\partial}{\partial
z_1}\frac{\partial}{\partial
z_2}\left(\hat{R}_{2}\left(P_{1}\left(z_{1}\right)\tilde{P}_{2}\left(z_{2}\right)\hat{P}_{1}\left(w_{1}\right)\hat{P}_{2}\left(w_{2}\right)\right)\hat{F}_{2}\left(w_{1},\hat{\theta}_{2}\left(P_{1}\left(z_{1}\right)\tilde{P}_{2}\left(z_{2}\right)\hat{P}_{1}\left(w_{1}\right)\right)\right)F_{2}\left(z_{1},z_{2}\right)\right)\\
&=&\hat{r}_{2}\mu_{1}\tilde{\mu}_{2}+\mu_{1}\tilde{\mu}_{2}\hat{R}_{2}^{(2)}\left(1\right)+\mu_{1}\hat{r}_{2}F_{2}^{(0,1)}+
\frac{\mu_{1}\tilde{\mu}_{2}}{1-\hat{\mu}_{2}}\hat{F}_{2}^{(0,1)}+2\hat{r}_{2}\frac{\mu_{1}\tilde{\mu}_{2}}{1-\hat{\mu}_{2}}\hat{F}_{2}^{(0,1)}+\mu_{1}\tilde{\mu}_{2}\hat{\theta}_{2}^{(2)}\left(1\right)\hat{F}_{2}^{(0,1)}\\
&+&\frac{\mu_{1}}{1-\hat{\mu}_{2}}F_{2}^{(0,1)}\hat{F}_{2}^{(0,1)}+\mu_{1}\tilde{\mu}_{2}\left(\frac{1}{1-\hat{\mu}_{2}}\right)^{2}\hat{F}_{2}^{(0,2)}+\hat{r}_{2}\tilde{\mu}_{2}F_{2}^{(1,0)}+\frac{\tilde{\mu}_{2}}{1-\hat{\mu}_{2}}\hat{F}_{2}^{(0,1)}F_{2}^{(1,0)}+F_{2}^{(1,1)}.
\end{eqnarray*}

%6/38

\item \begin{eqnarray*} &&\frac{\partial}{\partial
z_2}\frac{\partial}{\partial
z_2}\left(\hat{R}_{2}\left(P_{1}\left(z_{1}\right)\tilde{P}_{2}\left(z_{2}\right)\hat{P}_{1}\left(w_{1}\right)\hat{P}_{2}\left(w_{2}\right)\right)\hat{F}_{2}\left(w_{1},\hat{\theta}_{2}\left(P_{1}\left(z_{1}\right)\tilde{P}_{2}\left(z_{2}\right)\hat{P}_{1}\left(w_{1}\right)\right)\right)F_{2}\left(z_{1},z_{2}\right)\right)\\
&=&\hat{r}_{2}\tilde{P}_{2}^{(2)}\left(1\right)+\tilde{\mu}_{2}^{2}\hat{R}_{2}^{(2)}\left(1\right)+2\hat{r}_{2}\tilde{\mu}_{2}F_{2}^{(0,1)}+2\hat{r}_{2}\frac{\tilde{\mu}_{2}^{2}}{1-\hat{\mu}_{2}}\hat{F}_{2}^{(0,1)}+\frac{1}{1-\hat{\mu}_{2}}\tilde{P}_{2}^{(2)}\left(1\right)\hat{F}_{2}^{(0,1)}\\
&+&\tilde{\mu}_{2}^{2}\hat{\theta}_{2}^{(2)}\left(1\right)\hat{F}_{2}^{(0,1)}+2\frac{\tilde{\mu}_{2}}{1-\hat{\mu}_{2}}F_{2}^{(0,1)}\hat{F}_{2}^{(0,1)}+F_{2}^{(0,2)}+\left(\frac{\tilde{\mu}_{2}}{1-\hat{\mu}_{2}}\right)^{2}\hat{F}_{2}^{(0,2)}.
\end{eqnarray*}

%7/39

\item \begin{eqnarray*} &&\frac{\partial}{\partial
w_1}\frac{\partial}{\partial
z_2}\left(\hat{R}_{2}\left(P_{1}\left(z_{1}\right)\tilde{P}_{2}\left(z_{2}\right)\hat{P}_{1}\left(w_{1}\right)\hat{P}_{2}\left(w_{2}\right)\right)\hat{F}_{2}\left(w_{1},\hat{\theta}_{2}\left(P_{1}\left(z_{1}\right)\tilde{P}_{2}\left(z_{2}\right)\hat{P}_{1}\left(w_{1}\right)\right)\right)F_{2}\left(z_{1},z_{2}\right)\right)\\
&=&\hat{r}_{2}\tilde{\mu}_{2}\hat{\mu}_{1}+\tilde{\mu}_{2}\hat{\mu}_{1}\hat{R}_{2}^{(2)}\left(1\right)+\hat{r}_{2}\hat{\mu}_{1}F_{2}^{(0,1)}+\hat{r}_{2}\frac{\tilde{\mu}_{2}\hat{\mu}_{1}}{1-\hat{\mu}_{2}}\hat{F}_{2}^{(0,1)}+\hat{r}_{2}\tilde{\mu}_{2}\hat{F}_{2}^{(1,0)}+F_{2}^{(0,1)}\hat{F}_{2}^{(1,0)}+\frac{\tilde{\mu}_{2}}{1-\hat{\mu}_{2}}\hat{F}_{2}^{(1,1)}.
\end{eqnarray*}
%8/40

\item \begin{eqnarray*} &&\frac{\partial}{\partial
w_2}\frac{\partial}{\partial
z_2}\left(\hat{R}_{2}\left(P_{1}\left(z_{1}\right)\tilde{P}_{2}\left(z_{2}\right)\hat{P}_{1}\left(w_{1}\right)\hat{P}_{2}\left(w_{2}\right)\right)\hat{F}_{2}\left(w_{1},\hat{\theta}_{2}\left(P_{1}\left(z_{1}\right)\tilde{P}_{2}\left(z_{2}\right)\hat{P}_{1}\left(w_{1}\right)\right)\right)F_{2}\left(z_{1},z_{2}\right)\right)\\
&=&\hat{r}_{2}\tilde{\mu}_{2}\hat{\mu}_{2}+\tilde{\mu}_{2}\hat{\mu}_{2}\hat{R}_{2}^{(2)}\left(1\right)+\hat{r}_{2}\hat{\mu}_{2}F_{2}^{(0,1)}+
\frac{\tilde{\mu}_{2}\hat{\mu}_{2}}{1-\hat{\mu}_{2}}\hat{F}_{2}^{(0,1)}+2\hat{r}_{2}\frac{\tilde{\mu}_{2}\hat{\mu}_{2}}{1-\hat{\mu}_{2}}\hat{F}_{2}^{(0,1)}+\tilde{\mu}_{2}\hat{\mu}_{2}\hat{\theta}_{2}^{(2)}\left(1\right)\hat{F}_{2}^{(0,1)}\\
&+&\frac{\hat{\mu}_{2}}{1-\hat{\mu}_{2}}F_{2}^{(0,1)}\hat{F}_{2}^{(1,0)}+\tilde{\mu}_{2}\hat{\mu}_{2}\left(\frac{1}{1-\hat{\mu}_{2}}\right)\hat{F}_{2}^{(0,2)}.
\end{eqnarray*}
%___________________________________________________________________________________________
%\subsubsection{Mixtas para $w_{1}$:}
%___________________________________________________________________________________________

%9/41
\item \begin{eqnarray*} &&\frac{\partial}{\partial
z_1}\frac{\partial}{\partial
w_1}\left(\hat{R}_{2}\left(P_{1}\left(z_{1}\right)\tilde{P}_{2}\left(z_{2}\right)\hat{P}_{1}\left(w_{1}\right)\hat{P}_{2}\left(w_{2}\right)\right)\hat{F}_{2}\left(w_{1},\hat{\theta}_{2}\left(P_{1}\left(z_{1}\right)\tilde{P}_{2}\left(z_{2}\right)\hat{P}_{1}\left(w_{1}\right)\right)\right)F_{2}\left(z_{1},z_{2}\right)\right)\\
&=&\hat{r}_{2}\mu_{1}\hat{\mu}_{1}+\mu_{1}\hat{\mu}_{1}\hat{R}_{2}^{(2)}\left(1\right)+\hat{r}_{2}\frac{\mu_{1}\hat{\mu}_{1}}{1-\hat{\mu}_{2}}\hat{F}_{2}^{(0,1)}+\hat{r}_{2}\hat{\mu}_{1}F_{2}^{(1,0)}+\hat{r}_{2}\mu_{1}\hat{F}_{2}^{(1,0)}+F_{2}^{(1,0)}\hat{F}_{2}^{(1,0)}+\frac{\mu_{1}}{1-\hat{\mu}_{2}}\hat{F}_{2}^{(1,1)}.
\end{eqnarray*}


%10/42
\item \begin{eqnarray*} &&\frac{\partial}{\partial
z_2}\frac{\partial}{\partial
w_1}\left(\hat{R}_{2}\left(P_{1}\left(z_{1}\right)\tilde{P}_{2}\left(z_{2}\right)\hat{P}_{1}\left(w_{1}\right)\hat{P}_{2}\left(w_{2}\right)\right)\hat{F}_{2}\left(w_{1},\hat{\theta}_{2}\left(P_{1}\left(z_{1}\right)\tilde{P}_{2}\left(z_{2}\right)\hat{P}_{1}\left(w_{1}\right)\right)\right)F_{2}\left(z_{1},z_{2}\right)\right)\\
&=&\hat{r}_{2}\tilde{\mu}_{2}\hat{\mu}_{1}+\tilde{\mu}_{2}\hat{\mu}_{1}\hat{R}_{2}^{(2)}\left(1\right)+\hat{r}_{2}\hat{\mu}_{1}F_{2}^{(0,1)}+
\hat{r}_{2}\frac{\tilde{\mu}_{2}\hat{\mu}_{1}}{1-\hat{\mu}_{2}}\hat{F}_{2}^{(0,1)}+\hat{r}_{2}\tilde{\mu}_{2}\hat{F}_{2}^{(1,0)}+F_{2}^{(0,1)}\hat{F}_{2}^{(1,0)}+\frac{\tilde{\mu}_{2}}{1-\hat{\mu}_{2}}\hat{F}_{2}^{(1,1)}.
\end{eqnarray*}


%11/43
\item \begin{eqnarray*} &&\frac{\partial}{\partial
w_1}\frac{\partial}{\partial
w_1}\left(\hat{R}_{2}\left(P_{1}\left(z_{1}\right)\tilde{P}_{2}\left(z_{2}\right)\hat{P}_{1}\left(w_{1}\right)\hat{P}_{2}\left(w_{2}\right)\right)\hat{F}_{2}\left(w_{1},\hat{\theta}_{2}\left(P_{1}\left(z_{1}\right)\tilde{P}_{2}\left(z_{2}\right)\hat{P}_{1}\left(w_{1}\right)\right)\right)F_{2}\left(z_{1},z_{2}\right)\right)\\
&=&\hat{r}_{2}\hat{P}_{1}^{(2)}\left(1\right)+\hat{\mu}_{1}^{2}\hat{R}_{2}^{(2)}\left(1\right)+2\hat{r}_{2}\hat{\mu}_{1}\hat{F}_{2}^{(1,0)}
+\hat{F}_{2}^{(2,0)}.
\end{eqnarray*}


%12/44
\item \begin{eqnarray*} &&\frac{\partial}{\partial
w_2}\frac{\partial}{\partial
w_1}\left(\hat{R}_{2}\left(P_{1}\left(z_{1}\right)\tilde{P}_{2}\left(z_{2}\right)\hat{P}_{1}\left(w_{1}\right)\hat{P}_{2}\left(w_{2}\right)\right)\hat{F}_{2}\left(w_{1},\hat{\theta}_{2}\left(P_{1}\left(z_{1}\right)\tilde{P}_{2}\left(z_{2}\right)\hat{P}_{1}\left(w_{1}\right)\right)\right)F_{2}\left(z_{1},z_{2}\right)\right)\\
&=&\hat{r}_{2}\hat{\mu}_{1}\hat{\mu}_{2}+\hat{\mu}_{1}\hat{\mu}_{2}\hat{R}_{2}^{(2)}\left(1\right)+
\hat{r}_{2}\frac{\hat{\mu}_{2}\hat{\mu}_{1}}{1-\hat{\mu}_{2}}\hat{F}_{2}^{(0,1)}
+\hat{r}_{2}\hat{\mu}_{2}\hat{F}_{2}^{(1,0)}+\frac{\hat{\mu}_{2}}{1-\hat{\mu}_{2}}\hat{F}_{2}^{(1,1)}.
\end{eqnarray*}
%___________________________________________________________________________________________
%\subsubsection{Mixtas para $w_{2}$:}
%___________________________________________________________________________________________
%13/45
\item \begin{eqnarray*} &&\frac{\partial}{\partial
z_1}\frac{\partial}{\partial
w_2}\left(\hat{R}_{2}\left(P_{1}\left(z_{1}\right)\tilde{P}_{2}\left(z_{2}\right)\hat{P}_{1}\left(w_{1}\right)\hat{P}_{2}\left(w_{2}\right)\right)\hat{F}_{2}\left(w_{1},\hat{\theta}_{2}\left(P_{1}\left(z_{1}\right)\tilde{P}_{2}\left(z_{2}\right)\hat{P}_{1}\left(w_{1}\right)\right)\right)F_{2}\left(z_{1},z_{2}\right)\right)\\
&=&\hat{r}_{2}\mu_{1}\hat{\mu}_{2}+\mu_{1}\hat{\mu}_{2}\hat{R}_{2}^{(2)}\left(1\right)+
\frac{\mu_{1}\hat{\mu}_{2}}{1-\hat{\mu}_{2}}\hat{F}_{2}^{(0,1)} +2\hat{r}_{2}\frac{\mu_{1}\hat{\mu}_{2}}{1-\hat{\mu}_{2}}\hat{F}_{2}^{(0,1)}\\
&+&\mu_{1}\hat{\mu}_{2}\hat{\theta}_{2}^{(2)}\left(1\right)\hat{F}_{2}^{(0,1)}+\mu_{1}\hat{\mu}_{2}\left(\frac{1}{1-\hat{\mu}_{2}}\right)^{2}\hat{F}_{2}^{(0,2)}+\hat{r}_{2}\hat{\mu}_{2}F_{2}^{(1,0)}+\frac{\hat{\mu}_{2}}{1-\hat{\mu}_{2}}\hat{F}_{2}^{(0,1)}F_{2}^{(1,0)}.\end{eqnarray*}


%14/46
\item \begin{eqnarray*} &&\frac{\partial}{\partial
z_2}\frac{\partial}{\partial
w_2}\left(\hat{R}_{2}\left(P_{1}\left(z_{1}\right)\tilde{P}_{2}\left(z_{2}\right)\hat{P}_{1}\left(w_{1}\right)\hat{P}_{2}\left(w_{2}\right)\right)\hat{F}_{2}\left(w_{1},\hat{\theta}_{2}\left(P_{1}\left(z_{1}\right)\tilde{P}_{2}\left(z_{2}\right)\hat{P}_{1}\left(w_{1}\right)\right)\right)F_{2}\left(z_{1},z_{2}\right)\right)\\
&=&\hat{r}_{2}\tilde{\mu}_{2}\hat{\mu}_{2}+\tilde{\mu}_{2}\hat{\mu}_{2}\hat{R}_{2}^{(2)}\left(1\right)+\hat{r}_{2}\hat{\mu}_{2}F_{2}^{(0,1)}+\frac{\tilde{\mu}_{2}\hat{\mu}_{2}}{1-\hat{\mu}_{2}}\hat{F}_{2}^{(0,1)}+
2\hat{r}_{2}\frac{\tilde{\mu}_{2}\hat{\mu}_{2}}{1-\hat{\mu}_{2}}\hat{F}_{2}^{(0,1)}+\tilde{\mu}_{2}\hat{\mu}_{2}\hat{\theta}_{2}^{(2)}\left(1\right)\hat{F}_{2}^{(0,1)}\\
&+&\frac{\hat{\mu}_{2}}{1-\hat{\mu}_{2}}\hat{F}_{2}^{(0,1)}F_{2}^{(0,1)}+\tilde{\mu}_{2}\hat{\mu}_{2}\left(\frac{1}{1-\hat{\mu}_{2}}\right)^{2}\hat{F}_{2}^{(0,2)}.
\end{eqnarray*}

%15/47

\item \begin{eqnarray*} &&\frac{\partial}{\partial
w_1}\frac{\partial}{\partial
w_2}\left(\hat{R}_{2}\left(P_{1}\left(z_{1}\right)\tilde{P}_{2}\left(z_{2}\right)\hat{P}_{1}\left(w_{1}\right)\hat{P}_{2}\left(w_{2}\right)\right)\hat{F}_{2}\left(w_{1},\hat{\theta}_{2}\left(P_{1}\left(z_{1}\right)\tilde{P}_{2}\left(z_{2}\right)\hat{P}_{1}\left(w_{1}\right)\right)\right)F_{2}\left(z_{1},z_{2}\right)\right)\\
&=&\hat{r}_{2}\hat{\mu}_{1}\hat{\mu}_{2}+\hat{\mu}_{1}\hat{\mu}_{2}\hat{R}_{2}^{(2)}\left(1\right)+
\hat{r}_{2}\frac{\hat{\mu}_{1}\hat{\mu}_{2}}{1-\hat{\mu}_{2}}\hat{F}_{2}^{(0,1)}+
\hat{r}_{2}\hat{\mu}_{2}\hat{F}_{2}^{(1,0)}+\frac{\hat{\mu}_{2}}{1-\hat{\mu}_{2}}\hat{F}_{2}^{(1,1)}.
\end{eqnarray*}

%16/48
\item \begin{eqnarray*} &&\frac{\partial}{\partial
w_2}\frac{\partial}{\partial
w_2}\left(\hat{R}_{2}\left(P_{1}\left(z_{1}\right)\tilde{P}_{2}\left(z_{2}\right)\hat{P}_{1}\left(w_{1}\right)\hat{P}_{2}\left(w_{2}\right)\right)\hat{F}_{2}\left(w_{1},\hat{\theta}_{2}\left(P_{1}\left(z_{1}\right)\tilde{P}_{2}\left(z_{2}\right)\hat{P}_{1}\left(w_{1}\right)\right)\right)F_{2}\left(z_{1},z_{2};\zeta_{2}\right)\right)\\
&=&\hat{r}_{2}P_{2}^{(2)}\left(1\right)+\hat{\mu}_{2}^{2}\hat{R}_{2}^{(2)}\left(1\right)+2\hat{r}_{2}\frac{\hat{\mu}_{2}^{2}}{1-\hat{\mu}_{2}}\hat{F}_{2}^{(0,1)}+\frac{1}{1-\hat{\mu}_{2}}\hat{P}_{2}^{(2)}\left(1\right)\hat{F}_{2}^{(0,1)}+\hat{\mu}_{2}^{2}\hat{\theta}_{2}^{(2)}\left(1\right)\hat{F}_{2}^{(0,1)}\\
&+&\left(\frac{\hat{\mu}_{2}}{1-\hat{\mu}_{2}}\right)^{2}\hat{F}_{2}^{(0,2)}.
\end{eqnarray*}


\end{enumerate}



%___________________________________________________________________________________________
%
%\subsection{Derivadas de Segundo Orden para $\hat{F}_{2}$}
%___________________________________________________________________________________________
\begin{enumerate}
%___________________________________________________________________________________________
%\subsubsection{Mixtas para $z_{1}$:}
%___________________________________________________________________________________________
%1/49

\item \begin{eqnarray*} &&\frac{\partial}{\partial
z_1}\frac{\partial}{\partial
z_1}\left(\hat{R}_{1}\left(P_{1}\left(z_{1}\right)\tilde{P}_{2}\left(z_{2}\right)\hat{P}_{1}\left(w_{1}\right)\hat{P}_{2}\left(w_{2}\right)\right)\hat{F}_{1}\left(\hat{\theta}_{1}\left(P_{1}\left(z_{1}\right)\tilde{P}_{2}\left(z_{2}\right)
\hat{P}_{2}\left(w_{2}\right)\right),w_{2}\right)F_{1}\left(z_{1},z_{2}\right)\right)\\
&=&\hat{r}_{1}P_{1}^{(2)}\left(1\right)+
\mu_{1}^{2}\hat{R}_{1}^{(2)}\left(1\right)+
2\hat{r}_{1}\mu_{1}F_{1}^{(1,0)}+
2\hat{r}_{1}\frac{\mu_{1}^{2}}{1-\hat{\mu}_{1}}\hat{F}_{1}^{(1,0)}+
\frac{1}{1-\hat{\mu}_{1}}P_{1}^{(2)}\left(1\right)\hat{F}_{1}^{(1,0)}+\mu_{1}^{2}\hat{\theta}_{1}^{(2)}\left(1\right)\hat{F}_{1}^{(1,0)}\\
&+&2\frac{\mu_{1}}{1-\hat{\mu}_{1}}\hat{F}_{1}^{(1,0)}F_{1}^{(1,0)}+F_{1}^{(2,0)}
+\left(\frac{\mu_{1}}{1-\hat{\mu}_{1}}\right)^{2}\hat{F}_{1}^{(2,0)}.
\end{eqnarray*}

%2/50

\item \begin{eqnarray*} &&\frac{\partial}{\partial
z_2}\frac{\partial}{\partial
z_1}\left(\hat{R}_{1}\left(P_{1}\left(z_{1}\right)\tilde{P}_{2}\left(z_{2}\right)\hat{P}_{1}\left(w_{1}\right)\hat{P}_{2}\left(w_{2}\right)\right)\hat{F}_{1}\left(\hat{\theta}_{1}\left(P_{1}\left(z_{1}\right)\tilde{P}_{2}\left(z_{2}\right)
\hat{P}_{2}\left(w_{2}\right)\right),w_{2}\right)F_{1}\left(z_{1},z_{2}\right)\right)\\
&=&\hat{r}_{1}\mu_{1}\tilde{\mu}_{2}+\mu_{1}\tilde{\mu}_{2}\hat{R}_{1}^{(2)}\left(1\right)+
\hat{r}_{1}\mu_{1}F_{1}^{(0,1)}+\tilde{\mu}_{2}\hat{r}_{1}F_{1}^{(1,0)}+
\frac{\mu_{1}\tilde{\mu}_{2}}{1-\hat{\mu}_{1}}\hat{F}_{1}^{(1,0)}+2\hat{r}_{1}\frac{\mu_{1}\tilde{\mu}_{2}}{1-\hat{\mu}_{1}}\hat{F}_{1}^{(1,0)}\\
&+&\mu_{1}\tilde{\mu}_{2}\hat{\theta}_{1}^{(2)}\left(1\right)\hat{F}_{1}^{(1,0)}+
\frac{\mu_{1}}{1-\hat{\mu}_{1}}\hat{F}_{1}^{(1,0)}F_{1}^{(0,1)}+
\frac{\tilde{\mu}_{2}}{1-\hat{\mu}_{1}}\hat{F}_{1}^{(1,0)}F_{1}^{(1,0)}+
F_{1}^{(1,1)}\\
&+&\mu_{1}\tilde{\mu}_{2}\left(\frac{1}{1-\hat{\mu}_{1}}\right)^{2}\hat{F}_{1}^{(2,0)}.
\end{eqnarray*}

%3/51

\item \begin{eqnarray*} &&\frac{\partial}{\partial
w_1}\frac{\partial}{\partial
z_1}\left(\hat{R}_{1}\left(P_{1}\left(z_{1}\right)\tilde{P}_{2}\left(z_{2}\right)\hat{P}_{1}\left(w_{1}\right)\hat{P}_{2}\left(w_{2}\right)\right)\hat{F}_{1}\left(\hat{\theta}_{1}\left(P_{1}\left(z_{1}\right)\tilde{P}_{2}\left(z_{2}\right)
\hat{P}_{2}\left(w_{2}\right)\right),w_{2}\right)F_{1}\left(z_{1},z_{2}\right)\right)\\
&=&\hat{r}_{1}\mu_{1}\hat{\mu}_{1}+\mu_{1}\hat{\mu}_{1}\hat{R}_{1}^{(2)}\left(1\right)+\hat{r}_{1}\hat{\mu}_{1}F_{1}^{(1,0)}+
\hat{r}_{1}\frac{\mu_{1}\hat{\mu}_{1}}{1-\hat{\mu}_{1}}\hat{F}_{1}^{(1,0)}.
\end{eqnarray*}

%4/52

\item \begin{eqnarray*} &&\frac{\partial}{\partial
w_2}\frac{\partial}{\partial
z_1}\left(\hat{R}_{1}\left(P_{1}\left(z_{1}\right)\tilde{P}_{2}\left(z_{2}\right)\hat{P}_{1}\left(w_{1}\right)\hat{P}_{2}\left(w_{2}\right)\right)\hat{F}_{1}\left(\hat{\theta}_{1}\left(P_{1}\left(z_{1}\right)\tilde{P}_{2}\left(z_{2}\right)
\hat{P}_{2}\left(w_{2}\right)\right),w_{2}\right)F_{1}\left(z_{1},z_{2}\right)\right)\\
&=&\hat{r}_{1}\mu_{1}\hat{\mu}_{2}+\mu_{1}\hat{\mu}_{2}\hat{R}_{1}^{(2)}\left(1\right)+\hat{r}_{1}\hat{\mu}_{2}F_{1}^{(1,0)}+\frac{\mu_{1}\hat{\mu}_{2}}{1-\hat{\mu}_{1}}\hat{F}_{1}^{(1,0)}+\hat{r}_{1}\frac{\mu_{1}\hat{\mu}_{2}}{1-\hat{\mu}_{1}}\hat{F}_{1}^{(1,0)}+\mu_{1}\hat{\mu}_{2}\hat{\theta}_{1}^{(2)}\left(1\right)\hat{F}_{1}^{(1,0)}\\
&+&\hat{r}_{1}\mu_{1}\left(\hat{F}_{1}^{(0,1)}+\frac{\hat{\mu}_{2}}{1-\hat{\mu}_{1}}\hat{F}_{1}^{(1,0)}\right)+F_{1}^{(1,0)}\left(\hat{F}_{1}^{(0,1)}+\frac{\hat{\mu}_{2}}{1-\hat{\mu}_{1}}\hat{F}_{1}^{(1,0)}\right)+\frac{\mu_{1}}{1-\hat{\mu}_{1}}\left(\hat{F}_{1}^{(1,1)}+\frac{\hat{\mu}_{2}}{1-\hat{\mu}_{1}}\hat{F}_{1}^{(2,0)}\right).
\end{eqnarray*}
%___________________________________________________________________________________________
%\subsubsection{Mixtas para $z_{2}$:}
%___________________________________________________________________________________________
%5/53

\item \begin{eqnarray*} &&\frac{\partial}{\partial
z_1}\frac{\partial}{\partial
z_2}\left(\hat{R}_{1}\left(P_{1}\left(z_{1}\right)\tilde{P}_{2}\left(z_{2}\right)\hat{P}_{1}\left(w_{1}\right)\hat{P}_{2}\left(w_{2}\right)\right)\hat{F}_{1}\left(\hat{\theta}_{1}\left(P_{1}\left(z_{1}\right)\tilde{P}_{2}\left(z_{2}\right)
\hat{P}_{2}\left(w_{2}\right)\right),w_{2}\right)F_{1}\left(z_{1},z_{2}\right)\right)\\
&=&\hat{r}_{1}\mu_{1}\tilde{\mu}_{2}+\mu_{1}\tilde{\mu}_{2}\hat{R}_{1}^{(2)}\left(1\right)+\hat{r}_{1}\mu_{1}F_{1}^{(0,1)}+\hat{r}_{1}\tilde{\mu}_{2}F_{1}^{(1,0)}+\frac{\mu_{1}\tilde{\mu}_{2}}{1-\hat{\mu}_{1}}\hat{F}_{1}^{(1,0)}+2\hat{r}_{1}\frac{\mu_{1}\tilde{\mu}_{2}}{1-\hat{\mu}_{1}}\hat{F}_{1}^{(1,0)}\\
&+&\mu_{1}\tilde{\mu}_{2}\hat{\theta}_{1}^{(2)}\left(1\right)\hat{F}_{1}^{(1,0)}+\frac{\mu_{1}}{1-\hat{\mu}_{1}}\hat{F}_{1}^{(1,0)}F_{1}^{(0,1)}+\frac{\tilde{\mu}_{2}}{1-\hat{\mu}_{1}}\hat{F}_{1}^{(1,0)}F_{1}^{(1,0)}+F_{1}^{(1,1)}+\mu_{1}\tilde{\mu}_{2}\left(\frac{1}{1-\hat{\mu}_{1}}\right)^{2}\hat{F}_{1}^{(2,0)}.
\end{eqnarray*}

%6/54
\item \begin{eqnarray*} &&\frac{\partial}{\partial
z_2}\frac{\partial}{\partial
z_2}\left(\hat{R}_{1}\left(P_{1}\left(z_{1}\right)\tilde{P}_{2}\left(z_{2}\right)\hat{P}_{1}\left(w_{1}\right)\hat{P}_{2}\left(w_{2}\right)\right)\hat{F}_{1}\left(\hat{\theta}_{1}\left(P_{1}\left(z_{1}\right)\tilde{P}_{2}\left(z_{2}\right)
\hat{P}_{2}\left(w_{2}\right)\right),w_{2}\right)F_{1}\left(z_{1},z_{2}\right)\right)\\
&=&\hat{r}_{1}\tilde{P}_{2}^{(2)}\left(1\right)+\tilde{\mu}_{2}^{2}\hat{R}_{1}^{(2)}\left(1\right)+2\hat{r}_{1}\tilde{\mu}_{2}F_{1}^{(0,1)}+ F_{1}^{(0,2)}+2\hat{r}_{1}\frac{\tilde{\mu}_{2}^{2}}{1-\hat{\mu}_{1}}\hat{F}_{1}^{(1,0)}+\frac{1}{1-\hat{\mu}_{1}}\tilde{P}_{2}^{(2)}\left(1\right)\hat{F}_{1}^{(1,0)}\\
&+&\tilde{\mu}_{2}^{2}\hat{\theta}_{1}^{(2)}\left(1\right)\hat{F}_{1}^{(1,0)}+2\frac{\tilde{\mu}_{2}}{1-\hat{\mu}_{1}}F^{(0,1)}\hat{F}_{1}^{(1,0)}+\left(\frac{\tilde{\mu}_{2}}{1-\hat{\mu}_{1}}\right)^{2}\hat{F}_{1}^{(2,0)}.
\end{eqnarray*}
%7/55

\item \begin{eqnarray*} &&\frac{\partial}{\partial
w_1}\frac{\partial}{\partial
z_2}\left(\hat{R}_{1}\left(P_{1}\left(z_{1}\right)\tilde{P}_{2}\left(z_{2}\right)\hat{P}_{1}\left(w_{1}\right)\hat{P}_{2}\left(w_{2}\right)\right)\hat{F}_{1}\left(\hat{\theta}_{1}\left(P_{1}\left(z_{1}\right)\tilde{P}_{2}\left(z_{2}\right)
\hat{P}_{2}\left(w_{2}\right)\right),w_{2}\right)F_{1}\left(z_{1},z_{2}\right)\right)\\
&=&\hat{r}_{1}\hat{\mu}_{1}\tilde{\mu}_{2}+\hat{\mu}_{1}\tilde{\mu}_{2}\hat{R}_{1}^{(2)}\left(1\right)+
\hat{r}_{1}\hat{\mu}_{1}F_{1}^{(0,1)}+\hat{r}_{1}\frac{\hat{\mu}_{1}\tilde{\mu}_{2}}{1-\hat{\mu}_{1}}\hat{F}_{1}^{(1,0)}.
\end{eqnarray*}
%8/56

\item \begin{eqnarray*} &&\frac{\partial}{\partial
w_2}\frac{\partial}{\partial
z_2}\left(\hat{R}_{1}\left(P_{1}\left(z_{1}\right)\tilde{P}_{2}\left(z_{2}\right)\hat{P}_{1}\left(w_{1}\right)\hat{P}_{2}\left(w_{2}\right)\right)\hat{F}_{1}\left(\hat{\theta}_{1}\left(P_{1}\left(z_{1}\right)\tilde{P}_{2}\left(z_{2}\right)
\hat{P}_{2}\left(w_{2}\right)\right),w_{2}\right)F_{1}\left(z_{1},z_{2}\right)\right)\\
&=&\hat{r}_{1}\tilde{\mu}_{2}\hat{\mu}_{2}+\hat{\mu}_{2}\tilde{\mu}_{2}\hat{R}_{1}^{(2)}\left(1\right)+\hat{\mu}_{2}\hat{R}_{1}^{(2)}\left(1\right)F_{1}^{(0,1)}+\frac{\hat{\mu}_{2}\tilde{\mu}_{2}}{1-\hat{\mu}_{1}}\hat{F}_{1}^{(1,0)}+
\hat{r}_{1}\frac{\hat{\mu}_{2}\tilde{\mu}_{2}}{1-\hat{\mu}_{1}}\hat{F}_{1}^{(1,0)}\\
&+&\hat{\mu}_{2}\tilde{\mu}_{2}\hat{\theta}_{1}^{(2)}\left(1\right)\hat{F}_{1}^{(1,0)}+\hat{r}_{1}\tilde{\mu}_{2}\left(\hat{F}_{1}^{(0,1)}+\frac{\hat{\mu}_{2}}{1-\hat{\mu}_{1}}\hat{F}_{1}^{(1,0)}\right)+F_{1}^{(0,1)}\left(\hat{F}_{1}^{(0,1)}+\frac{\hat{\mu}_{2}}{1-\hat{\mu}_{1}}\hat{F}_{1}^{(1,0)}\right)\\
&+&\frac{\tilde{\mu}_{2}}{1-\hat{\mu}_{1}}\left(\hat{F}_{1}^{(1,1)}+\frac{\hat{\mu}_{2}}{1-\hat{\mu}_{1}}\hat{F}_{1}^{(2,0)}\right).
\end{eqnarray*}
%___________________________________________________________________________________________
%\subsubsection{Mixtas para $w_{1}$:}
%___________________________________________________________________________________________
%9/57
\item \begin{eqnarray*} &&\frac{\partial}{\partial
z_1}\frac{\partial}{\partial
w_1}\left(\hat{R}_{1}\left(P_{1}\left(z_{1}\right)\tilde{P}_{2}\left(z_{2}\right)\hat{P}_{1}\left(w_{1}\right)\hat{P}_{2}\left(w_{2}\right)\right)\hat{F}_{1}\left(\hat{\theta}_{1}\left(P_{1}\left(z_{1}\right)\tilde{P}_{2}\left(z_{2}\right)
\hat{P}_{2}\left(w_{2}\right)\right),w_{2}\right)F_{1}\left(z_{1},z_{2}\right)\right)\\
&=&\hat{r}_{1}\mu_{1}\hat{\mu}_{1}+\mu_{1}\hat{\mu}_{1}\hat{R}_{1}^{(2)}\left(1\right)+\hat{r}_{1}\hat{\mu}_{1}F_{1}^{(1,0)}+\hat{r}_{1}\frac{\mu_{1}\hat{\mu}_{1}}{1-\hat{\mu}_{1}}\hat{F}_{1}^{(1,0)}.
\end{eqnarray*}
%10/58
\item \begin{eqnarray*} &&\frac{\partial}{\partial
z_2}\frac{\partial}{\partial
w_1}\left(\hat{R}_{1}\left(P_{1}\left(z_{1}\right)\tilde{P}_{2}\left(z_{2}\right)\hat{P}_{1}\left(w_{1}\right)\hat{P}_{2}\left(w_{2}\right)\right)\hat{F}_{1}\left(\hat{\theta}_{1}\left(P_{1}\left(z_{1}\right)\tilde{P}_{2}\left(z_{2}\right)
\hat{P}_{2}\left(w_{2}\right)\right),w_{2}\right)F_{1}\left(z_{1},z_{2}\right)\right)\\
&=&\hat{r}_{1}\tilde{\mu}_{2}\hat{\mu}_{1}+\tilde{\mu}_{2}\hat{\mu}_{1}\hat{R}_{1}^{(2)}\left(1\right)+\hat{r}_{1}\hat{\mu}_{1}F_{1}^{(0,1)}+\hat{r}_{1}\frac{\tilde{\mu}_{2}\hat{\mu}_{1}}{1-\hat{\mu}_{1}}\hat{F}_{1}^{(1,0)}.
\end{eqnarray*}
%11/59
\item \begin{eqnarray*} &&\frac{\partial}{\partial
w_1}\frac{\partial}{\partial
w_1}\left(\hat{R}_{1}\left(P_{1}\left(z_{1}\right)\tilde{P}_{2}\left(z_{2}\right)\hat{P}_{1}\left(w_{1}\right)\hat{P}_{2}\left(w_{2}\right)\right)\hat{F}_{1}\left(\hat{\theta}_{1}\left(P_{1}\left(z_{1}\right)\tilde{P}_{2}\left(z_{2}\right)
\hat{P}_{2}\left(w_{2}\right)\right),w_{2}\right)F_{1}\left(z_{1},z_{2}\right)\right)\\
&=&\hat{r}_{1}\hat{P}_{1}^{(2)}\left(1\right)+\hat{\mu}_{1}^{2}\hat{R}_{1}^{(2)}\left(1\right).
\end{eqnarray*}
%12/60
\item \begin{eqnarray*} &&\frac{\partial}{\partial
w_2}\frac{\partial}{\partial
w_1}\left(\hat{R}_{1}\left(P_{1}\left(z_{1}\right)\tilde{P}_{2}\left(z_{2}\right)\hat{P}_{1}\left(w_{1}\right)\hat{P}_{2}\left(w_{2}\right)\right)\hat{F}_{1}\left(\hat{\theta}_{1}\left(P_{1}\left(z_{1}\right)\tilde{P}_{2}\left(z_{2}\right)
\hat{P}_{2}\left(w_{2}\right)\right),w_{2}\right)F_{1}\left(z_{1},z_{2}\right)\right)\\
&=&\hat{r}_{1}\hat{\mu}_{2}\hat{\mu}_{1}+\hat{\mu}_{2}\hat{\mu}_{1}\hat{R}_{1}^{(2)}\left(1\right)+\hat{r}_{1}\hat{\mu}_{1}\left(\hat{F}_{1}^{(0,1)}+\frac{\hat{\mu}_{2}}{1-\hat{\mu}_{1}}\hat{F}_{1}^{(1,0)}\right).
\end{eqnarray*}
%___________________________________________________________________________________________
%\subsubsection{Mixtas para $w_{1}$:}
%___________________________________________________________________________________________
%13/61



\item \begin{eqnarray*} &&\frac{\partial}{\partial
z_1}\frac{\partial}{\partial
w_2}\left(\hat{R}_{1}\left(P_{1}\left(z_{1}\right)\tilde{P}_{2}\left(z_{2}\right)\hat{P}_{1}\left(w_{1}\right)\hat{P}_{2}\left(w_{2}\right)\right)\hat{F}_{1}\left(\hat{\theta}_{1}\left(P_{1}\left(z_{1}\right)\tilde{P}_{2}\left(z_{2}\right)
\hat{P}_{2}\left(w_{2}\right)\right),w_{2}\right)F_{1}\left(z_{1},z_{2}\right)\right)\\
&=&\hat{r}_{1}\mu_{1}\hat{\mu}_{2}+\mu_{1}\hat{\mu}_{2}\hat{R}_{1}^{(2)}\left(1\right)+\hat{r}_{1}\hat{\mu}_{2}F_{1}^{(1,0)}+
\hat{r}_{1}\frac{\mu_{1}\hat{\mu}_{2}}{1-\hat{\mu}_{1}}\hat{F}_{1}^{(1,0)}+\hat{r}_{1}\mu_{1}\left(\hat{F}_{1}^{(0,1)}+\frac{\hat{\mu}_{2}}{1-\hat{\mu}_{1}}\hat{F}_{1}^{(1,0)}\right)\\
&+&F_{1}^{(1,0)}\left(\hat{F}_{1}^{(0,1)}+\frac{\hat{\mu}_{2}}{1-\hat{\mu}_{1}}\hat{F}_{1}^{(1,0)}\right)+\frac{\mu_{1}\hat{\mu}_{2}}{1-\hat{\mu}_{1}}\hat{F}_{1}^{(1,0)}+\mu_{1}\hat{\mu}_{2}\hat{\theta}_{1}^{(2)}\left(1\right)\hat{F}_{1}^{(1,0)}+\frac{\mu_{1}}{1-\hat{\mu}_{1}}\hat{F}_{1}^{(1,1)}\\
&+&\mu_{1}\hat{\mu}_{2}\left(\frac{1}{1-\hat{\mu}_{1}}\right)^{2}\hat{F}_{1}^{(2,0)}.
\end{eqnarray*}

%14/62
\item \begin{eqnarray*} &&\frac{\partial}{\partial
z_2}\frac{\partial}{\partial
w_2}\left(\hat{R}_{1}\left(P_{1}\left(z_{1}\right)\tilde{P}_{2}\left(z_{2}\right)\hat{P}_{1}\left(w_{1}\right)\hat{P}_{2}\left(w_{2}\right)\right)\hat{F}_{1}\left(\hat{\theta}_{1}\left(P_{1}\left(z_{1}\right)\tilde{P}_{2}\left(z_{2}\right)
\hat{P}_{2}\left(w_{2}\right)\right),w_{2}\right)F_{1}\left(z_{1},z_{2}\right)\right)\\
&=&\hat{r}_{1}\tilde{\mu}_{2}\hat{\mu}_{2}+\tilde{\mu}_{2}\hat{\mu}_{2}\hat{R}_{1}^{(2)}\left(1\right)+\hat{r}_{1}\hat{\mu}_{2}F_{1}^{(0,1)}+\hat{r}_{1}\frac{\tilde{\mu}_{2}\hat{\mu}_{2}}{1-\hat{\mu}_{1}}\hat{F}_{1}^{(1,0)}+\hat{r}_{1}\tilde{\mu}_{2}\left(\hat{F}_{1}^{(0,1)}+\frac{\hat{\mu}_{2}}{1-\hat{\mu}_{1}}\hat{F}_{1}^{(1,0)}\right)\\
&+&F_{1}^{(0,1)}\left(\hat{F}_{1}^{(0,1)}+\frac{\hat{\mu}_{2}}{1-\hat{\mu}_{1}}\hat{F}_{1}^{(1,0)}\right)+\frac{\tilde{\mu}_{2}\hat{\mu}_{2}}{1-\hat{\mu}_{1}}\hat{F}_{1}^{(1,0)}+\tilde{\mu}_{2}\hat{\mu}_{2}\hat{\theta}_{1}^{(2)}\left(1\right)\hat{F}_{1}^{(1,0)}+\frac{\tilde{\mu}_{2}}{1-\hat{\mu}_{1}}\hat{F}_{1}^{(1,1)}\\
&+&\tilde{\mu}_{2}\hat{\mu}_{2}\left(\frac{1}{1-\hat{\mu}_{1}}\right)^{2}\hat{F}_{1}^{(2,0)}.
\end{eqnarray*}

%15/63

\item \begin{eqnarray*} &&\frac{\partial}{\partial
w_1}\frac{\partial}{\partial
w_2}\left(\hat{R}_{1}\left(P_{1}\left(z_{1}\right)\tilde{P}_{2}\left(z_{2}\right)\hat{P}_{1}\left(w_{1}\right)\hat{P}_{2}\left(w_{2}\right)\right)\hat{F}_{1}\left(\hat{\theta}_{1}\left(P_{1}\left(z_{1}\right)\tilde{P}_{2}\left(z_{2}\right)
\hat{P}_{2}\left(w_{2}\right)\right),w_{2}\right)F_{1}\left(z_{1},z_{2}\right)\right)\\
&=&\hat{r}_{1}\hat{\mu}_{2}\hat{\mu}_{1}+\hat{\mu}_{2}\hat{\mu}_{1}\hat{R}_{1}^{(2)}\left(1\right)+\hat{r}_{1}\hat{\mu}_{1}\left(\hat{F}_{1}^{(0,1)}+\frac{\hat{\mu}_{2}}{1-\hat{\mu}_{1}}\hat{F}_{1}^{(1,0)}\right).
\end{eqnarray*}

%16/64

\item \begin{eqnarray*} &&\frac{\partial}{\partial
w_2}\frac{\partial}{\partial
w_2}\left(\hat{R}_{1}\left(P_{1}\left(z_{1}\right)\tilde{P}_{2}\left(z_{2}\right)\hat{P}_{1}\left(w_{1}\right)\hat{P}_{2}\left(w_{2}\right)\right)\hat{F}_{1}\left(\hat{\theta}_{1}\left(P_{1}\left(z_{1}\right)\tilde{P}_{2}\left(z_{2}\right)
\hat{P}_{2}\left(w_{2}\right)\right),w_{2}\right)F_{1}\left(z_{1},z_{2}\right)\right)\\
&=&\hat{r}_{1}\hat{P}_{2}^{(2)}\left(1\right)+\hat{\mu}_{2}^{2}\hat{R}_{1}^{(2)}\left(1\right)+
2\hat{r}_{1}\hat{\mu}_{2}\left(\hat{F}_{1}^{(0,1)}+\frac{\hat{\mu}_{2}}{1-\hat{\mu}_{1}}\hat{F}_{1}^{(1,0)}\right)+
\hat{F}_{1}^{(0,2)}+\frac{1}{1-\hat{\mu}_{1}}\hat{P}_{2}^{(2)}\left(1\right)\hat{F}_{1}^{(1,0)}\\
&+&\hat{\mu}_{2}^{2}\hat{\theta}_{1}^{(2)}\left(1\right)\hat{F}_{1}^{(1,0)}+\frac{\hat{\mu}_{2}}{1-\hat{\mu}_{1}}\hat{F}_{1}^{(1,1)}+\frac{\hat{\mu}_{2}}{1-\hat{\mu}_{1}}\left(\hat{F}_{1}^{(1,1)}+\frac{\hat{\mu}_{2}}{1-\hat{\mu}_{1}}\hat{F}_{1}^{(2,0)}\right).
\end{eqnarray*}
%_________________________________________________________________________________________________________
%
%_________________________________________________________________________________________________________

\end{enumerate}




Las ecuaciones que determinan los segundos momentos de las longitudes de las colas de los dos sistemas se pueden ver en \href{http://sitio.expresauacm.org/s/carlosmartinez/wp-content/uploads/sites/13/2014/01/SegundosMomentos.pdf}{este sitio}

%\url{http://ubuntu_es_el_diablo.org},\href{http://www.latex-project.org/}{latex project}

%http://sitio.expresauacm.org/s/carlosmartinez/wp-content/uploads/sites/13/2014/01/SegundosMomentos.jpg
%http://sitio.expresauacm.org/s/carlosmartinez/wp-content/uploads/sites/13/2014/01/SegundosMomentos.pdf




%_____________________________________________________________________________________
%Distribuci\'on del n\'umero de usuaruios que pasan del sistema 1 al sistema 2
%_____________________________________________________________________________________
\section*{Ap\'endice B}
%________________________________________________________________________________________
%
%________________________________________________________________________________________
\subsection*{Distribuci\'on para los usuarios de traslado}
%________________________________________________________________________________________
Se puede demostrar que
\begin{equation}
\frac{d^{k}}{dy}\left(\frac{\lambda +\mu}{\lambda
+\mu-y}\right)=\frac{k!}{\left(\lambda+\mu\right)^{k}}
\end{equation}



\begin{Prop}
Sea $\tau$ variable aleatoria no negativa con distribuci\'on exponencial con media $\mu$, y sea $L\left(t\right)$ proceso
Poisson con par\'ametro $\lambda$. Entonces
\begin{equation}
\prob\left\{L\left(\tau\right)=k\right\}=f_{L\left(\tau\right)}\left(k\right)=\left(\frac{\mu}{\lambda
+\mu}\right)\left(\frac{\lambda}{\lambda+\mu}\right)^{k}.
\end{equation}
Adem\'as

\begin{eqnarray}
\esp\left[L\left(\tau\right)\right]&=&\frac{\lambda}{\mu}\\
\esp\left[\left(L\left(\tau\right)\right)^{2}\right]&=&\frac{\lambda}{\mu}\left(2\frac{\lambda}{\mu}+1\right)\\
V\left[L\left(\tau\right)\right]&=&\frac{\lambda}{\mu}\left(\frac{\lambda}{\mu}+1\right).
\end{eqnarray}
\end{Prop}

\begin{Proof}
A saber, para $k$ fijo se tiene que

\begin{eqnarray*}
\prob\left\{L\left(\tau\right)=k\right\}&=&\prob\left\{L\left(\tau\right)=k,\tau\in\left(0,\infty\right)\right\}\\
&=&\int_{0}^{\infty}\prob\left\{L\left(\tau\right)=k,\tau=y\right\}f_{\tau}\left(y\right)dy=\int_{0}^{\infty}\prob\left\{L\left(y\right)=k\right\}f_{\tau}\left(y\right)dy\\
&=&\int_{0}^{\infty}\frac{e^{-\lambda
y}}{k!}\left(\lambda y\right)^{k}\left(\mu e^{-\mu
y}\right)dy=\frac{\lambda^{k}\mu}{k!}\int_{0}^{\infty}y^{k}e^{-\left(\mu+\lambda\right)y}dy\\
&=&\frac{\lambda^{k}\mu}{\left(\lambda
+\mu\right)k!}\int_{0}^{\infty}y^{k}\left(\lambda+\mu\right)e^{-\left(\lambda+\mu\right)y}dy=\frac{\lambda^{k}\mu}{\left(\lambda
+\mu\right)k!}\int_{0}^{\infty}y^{k}f_{Y}\left(y\right)dy\\
&=&\frac{\lambda^{k}\mu}{\left(\lambda
+\mu\right)k!}\esp\left[Y^{k}\right]=\frac{\lambda^{k}\mu}{\left(\lambda
+\mu\right)k!}\frac{d^{k}}{dy}\left(\frac{\lambda
+\mu}{\lambda
+\mu-y}\right)|_{y=0}\\
&=&\frac{\lambda^{k}\mu}{\left(\lambda
+\mu\right)k!}\frac{k!}{\left(\lambda+\mu\right)^{k}}=\left(\frac{\mu}{\lambda
+\mu}\right)\left(\frac{\lambda}{\lambda+\mu}\right)^{k}.\\
\end{eqnarray*}


Adem\'as
\begin{eqnarray*}
\sum_{k=0}^{\infty}\prob\left\{L\left(\tau\right)=k\right\}&=&\sum_{k=0}^{\infty}\left(\frac{\mu}{\lambda
+\mu}\right)\left(\frac{\lambda}{\lambda+\mu}\right)^{k}=\frac{\mu}{\lambda
+\mu}\sum_{k=0}^{\infty}\left(\frac{\lambda}{\lambda+\mu}\right)^{k}\\
&=&\frac{\mu}{\lambda
+\mu}\left(\frac{1}{1-\frac{\lambda}{\lambda+\mu}}\right)=\frac{\mu}{\lambda
+\mu}\left(\frac{\lambda+\mu}{\mu}\right)\\
&=&1.\\
\end{eqnarray*}

determinemos primero la esperanza de
$L\left(\tau\right)$:


\begin{eqnarray*}
\esp\left[L\left(\tau\right)\right]&=&\sum_{k=0}^{\infty}k\prob\left\{L\left(\tau\right)=k\right\}=\sum_{k=0}^{\infty}k\left(\frac{\mu}{\lambda
+\mu}\right)\left(\frac{\lambda}{\lambda+\mu}\right)^{k}\\
&=&\left(\frac{\mu}{\lambda
+\mu}\right)\sum_{k=0}^{\infty}k\left(\frac{\lambda}{\lambda+\mu}\right)^{k}=\left(\frac{\mu}{\lambda
+\mu}\right)\left(\frac{\lambda}{\lambda+\mu}\right)\sum_{k=1}^{\infty}k\left(\frac{\lambda}{\lambda+\mu}\right)^{k-1}\\
&=&\frac{\mu\lambda}{\left(\lambda
+\mu\right)^{2}}\left(\frac{1}{1-\frac{\lambda}{\lambda+\mu}}\right)^{2}=\frac{\mu\lambda}{\left(\lambda
+\mu\right)^{2}}\left(\frac{\lambda+\mu}{\mu}\right)^{2}\\
&=&\frac{\lambda}{\mu}.
\end{eqnarray*}

Ahora su segundo momento:

\begin{eqnarray*}
\esp\left[\left(L\left(\tau\right)\right)^{2}\right]&=&\sum_{k=0}^{\infty}k^{2}\prob\left\{L\left(\tau\right)=k\right\}=\sum_{k=0}^{\infty}k^{2}\left(\frac{\mu}{\lambda
+\mu}\right)\left(\frac{\lambda}{\lambda+\mu}\right)^{k}\\
&=&\left(\frac{\mu}{\lambda
+\mu}\right)\sum_{k=0}^{\infty}k^{2}\left(\frac{\lambda}{\lambda+\mu}\right)^{k}=
\frac{\mu\lambda}{\left(\lambda
+\mu\right)^{2}}\sum_{k=2}^{\infty}\left(k-1\right)^{2}\left(\frac{\lambda}{\lambda+\mu}\right)^{k-2}\\
&=&\frac{\mu\lambda}{\left(\lambda
+\mu\right)^{2}}\left(\frac{\frac{\lambda}{\lambda+\mu}+1}{\left(\frac{\lambda}{\lambda+\mu}-1\right)^{3}}\right)=\frac{\mu\lambda}{\left(\lambda
+\mu\right)^{2}}\left(-\frac{\frac{2\lambda+\mu}{\lambda+\mu}}{\left(-\frac{\mu}{\lambda+\mu}\right)^{3}}\right)\\
&=&\frac{\mu\lambda}{\left(\lambda
+\mu\right)^{2}}\left(\frac{2\lambda+\mu}{\lambda+\mu}\right)\left(\frac{\lambda+\mu}{\mu}\right)^{3}=\frac{\lambda\left(2\lambda
+\mu\right)}{\mu^{2}}\\
&=&\frac{\lambda}{\mu}\left(2\frac{\lambda}{\mu}+1\right).
\end{eqnarray*}

y por tanto

\begin{eqnarray*}
V\left[L\left(\tau\right)\right]&=&\frac{\lambda\left(2\lambda
+\mu\right)}{\mu^{2}}-\left(\frac{\lambda}{\mu}\right)^{2}=\frac{\lambda^{2}+\mu\lambda}{\mu^{2}}\\
&=&\frac{\lambda}{\mu}\left(\frac{\lambda}{\mu}+1\right).
\end{eqnarray*}
\end{Proof}

Ahora, determinemos la distribuci\'on del n\'umero de usuarios que
pasan de $\hat{Q}_{2}$ a $Q_{2}$ considerando dos pol\'iticas de
traslado en espec\'ifico:

\begin{enumerate}
\item Solamente pasa un usuario,

\item Se permite el paso de $k$ usuarios,
\end{enumerate}
una vez que son atendidos por el servidor en $\hat{Q}_{2}$.

\begin{description}


\item[Pol\'itica de un solo usuario:] Sea $R_{2}$ el n\'umero de
usuarios que llegan a $\hat{Q}_{2}$ al tiempo $t$, sea $R_{1}$ el
n\'umero de usuarios que pasan de $\hat{Q}_{2}$ a $Q_{2}$ al
tiempo $t$.
\end{description}


A saber:
\begin{eqnarray*}
\esp\left[R_{1}\right]&=&\sum_{y\geq0}\prob\left[R_{2}=y\right]\esp\left[R_{1}|R_{2}=y\right]\\
&=&\sum_{y\geq0}\prob\left[R_{2}=y\right]\sum_{x\geq0}x\prob\left[R_{1}=x|R_{2}=y\right]\\
&=&\sum_{y\geq0}\sum_{x\geq0}x\prob\left[R_{1}=x|R_{2}=y\right]\prob\left[R_{2}=y\right].\\
\end{eqnarray*}

Determinemos
\begin{equation}
\esp\left[R_{1}|R_{2}=y\right]=\sum_{x\geq0}x\prob\left[R_{1}=x|R_{2}=y\right].
\end{equation}

supongamos que $y=0$, entonces
\begin{eqnarray*}
\prob\left[R_{1}=0|R_{2}=0\right]&=&1,\\
\prob\left[R_{1}=x|R_{2}=0\right]&=&0,\textrm{ para cualquier }x\geq1,\\
\end{eqnarray*}


por tanto
\begin{eqnarray*}
\esp\left[R_{1}|R_{2}=0\right]=0.
\end{eqnarray*}

Para $y=1$,
\begin{eqnarray*}
\prob\left[R_{1}=0|R_{2}=1\right]&=&0,\\
\prob\left[R_{1}=1|R_{2}=1\right]&=&1,
\end{eqnarray*}

entonces
\begin{eqnarray*}
\esp\left[R_{1}|R_{2}=1\right]=1.
\end{eqnarray*}

Para $y>1$:
\begin{eqnarray*}
\prob\left[R_{1}=0|R_{2}\geq1\right]&=&0,\\
\prob\left[R_{1}=1|R_{2}\geq1\right]&=&1,\\
\prob\left[R_{1}>1|R_{2}\geq1\right]&=&0,
\end{eqnarray*}

entonces
\begin{eqnarray*}
\esp\left[R_{1}|R_{2}=y\right]=1,\textrm{ para cualquier }y>1.
\end{eqnarray*}
es decir
\begin{eqnarray*}
\esp\left[R_{1}|R_{2}=y\right]=1,\textrm{ para cualquier }y\geq1.
\end{eqnarray*}

Entonces
\begin{eqnarray*}
\esp\left[R_{1}\right]&=&\sum_{y\geq0}\sum_{x\geq0}x\prob\left[R_{1}=x|R_{2}=y\right]\prob\left[R_{2}=y\right]=\sum_{y\geq0}\sum_{x}\esp\left[R_{1}|R_{2}=y\right]\prob\left[R_{2}=y\right]\\
&=&\sum_{y\geq0}\prob\left[R_{2}=y\right]=\sum_{y\geq1}\frac{\left(\lambda
t\right)^{k}}{k!}e^{-\lambda t}=1.
\end{eqnarray*}

Adem\'as para $k\in Z^{+}$
\begin{eqnarray*}
f_{R_{1}}\left(k\right)&=&\prob\left[R_{1}=k\right]=\sum_{n=0}^{\infty}\prob\left[R_{1}=k|R_{2}=n\right]\prob\left[R_{2}=n\right]\\
&=&\prob\left[R_{1}=k|R_{2}=0\right]\prob\left[R_{2}=0\right]+\prob\left[R_{1}=k|R_{2}=1\right]\prob\left[R_{2}=1\right]\\
&+&\prob\left[R_{1}=k|R_{2}>1\right]\prob\left[R_{2}>1\right],
\end{eqnarray*}

donde para


\begin{description}
\item[$k=0$:]
\begin{eqnarray*}
\prob\left[R_{1}=0\right]=\prob\left[R_{1}=0|R_{2}=0\right]\prob\left[R_{2}=0\right]+\prob\left[R_{1}=0|R_{2}=1\right]\prob\left[R_{2}=1\right]\\
+\prob\left[R_{1}=0|R_{2}>1\right]\prob\left[R_{2}>1\right]=\prob\left[R_{2}=0\right].
\end{eqnarray*}
\item[$k=1$:]
\begin{eqnarray*}
\prob\left[R_{1}=1\right]=\prob\left[R_{1}=1|R_{2}=0\right]\prob\left[R_{2}=0\right]+\prob\left[R_{1}=1|R_{2}=1\right]\prob\left[R_{2}=1\right]\\
+\prob\left[R_{1}=1|R_{2}>1\right]\prob\left[R_{2}>1\right]=\sum_{n=1}^{\infty}\prob\left[R_{2}=n\right].
\end{eqnarray*}

\item[$k=2$:]
\begin{eqnarray*}
\prob\left[R_{1}=2\right]=\prob\left[R_{1}=2|R_{2}=0\right]\prob\left[R_{2}=0\right]+\prob\left[R_{1}=2|R_{2}=1\right]\prob\left[R_{2}=1\right]\\
+\prob\left[R_{1}=2|R_{2}>1\right]\prob\left[R_{2}>1\right]=0.
\end{eqnarray*}

\item[$k=j$:]
\begin{eqnarray*}
\prob\left[R_{1}=j\right]=\prob\left[R_{1}=j|R_{2}=0\right]\prob\left[R_{2}=0\right]+\prob\left[R_{1}=j|R_{2}=1\right]\prob\left[R_{2}=1\right]\\
+\prob\left[R_{1}=j|R_{2}>1\right]\prob\left[R_{2}>1\right]=0.
\end{eqnarray*}
\end{description}


Por lo tanto
\begin{eqnarray*}
f_{R_{1}}\left(0\right)&=&\prob\left[R_{2}=0\right]\\
f_{R_{1}}\left(1\right)&=&\sum_{n\geq1}^{\infty}\prob\left[R_{2}=n\right]\\
f_{R_{1}}\left(j\right)&=&0,\textrm{ para }j>1.
\end{eqnarray*}



\begin{description}
\item[Pol\'itica de $k$ usuarios:]Al igual que antes, para $y\in Z^{+}$ fijo
\begin{eqnarray*}
\esp\left[R_{1}|R_{2}=y\right]=\sum_{x}x\prob\left[R_{1}=x|R_{2}=y\right].\\
\end{eqnarray*}
\end{description}
Entonces, si tomamos diversos valore para $y$:\\

$y=0$:
\begin{eqnarray*}
\prob\left[R_{1}=0|R_{2}=0\right]&=&1,\\
\prob\left[R_{1}=x|R_{2}=0\right]&=&0,\textrm{ para cualquier }x\geq1,
\end{eqnarray*}

entonces
\begin{eqnarray*}
\esp\left[R_{1}|R_{2}=0\right]=0.
\end{eqnarray*}


Para $y=1$,
\begin{eqnarray*}
\prob\left[R_{1}=0|R_{2}=1\right]&=&0,\\
\prob\left[R_{1}=1|R_{2}=1\right]&=&1,
\end{eqnarray*}

entonces {\scriptsize{
\begin{eqnarray*}
\esp\left[R_{1}|R_{2}=1\right]=1.
\end{eqnarray*}}}


Para $y=2$,
\begin{eqnarray*}
\prob\left[R_{1}=0|R_{2}=2\right]&=&0,\\
\prob\left[R_{1}=1|R_{2}=2\right]&=&1,\\
\prob\left[R_{1}=2|R_{2}=2\right]&=&1,\\
\prob\left[R_{1}=3|R_{2}=2\right]&=&0,
\end{eqnarray*}

entonces
\begin{eqnarray*}
\esp\left[R_{1}|R_{2}=2\right]=3.
\end{eqnarray*}

Para $y=3$,
\begin{eqnarray*}
\prob\left[R_{1}=0|R_{2}=3\right]&=&0,\\
\prob\left[R_{1}=1|R_{2}=3\right]&=&1,\\
\prob\left[R_{1}=2|R_{2}=3\right]&=&1,\\
\prob\left[R_{1}=3|R_{2}=3\right]&=&1,\\
\prob\left[R_{1}=4|R_{2}=3\right]&=&0,
\end{eqnarray*}

entonces
\begin{eqnarray*}
\esp\left[R_{1}|R_{2}=3\right]=6.
\end{eqnarray*}

En general, para $k\geq0$,
\begin{eqnarray*}
\prob\left[R_{1}=0|R_{2}=k\right]&=&0,\\
\prob\left[R_{1}=j|R_{2}=k\right]&=&1,\textrm{ para }1\leq j\leq k,\\
\prob\left[R_{1}=j|R_{2}=k\right]&=&0,\textrm{ para }j> k,
\end{eqnarray*}

entonces
\begin{eqnarray*}
\esp\left[R_{1}|R_{2}=k\right]=\frac{k\left(k+1\right)}{2}.
\end{eqnarray*}



Por lo tanto


\begin{eqnarray*}
\esp\left[R_{1}\right]&=&\sum_{y}\esp\left[R_{1}|R_{2}=y\right]\prob\left[R_{2}=y\right]\\
&=&\sum_{y}\prob\left[R_{2}=y\right]\frac{y\left(y+1\right)}{2}=\sum_{y\geq1}\left(\frac{y\left(y+1\right)}{2}\right)\frac{\left(\lambda t\right)^{y}}{y!}e^{-\lambda t}\\
&=&\frac{\lambda t}{2}e^{-\lambda t}\sum_{y\geq1}\left(y+1\right)\frac{\left(\lambda t\right)^{y-1}}{\left(y-1\right)!}=\frac{\lambda t}{2}e^{-\lambda t}\left(e^{\lambda t}\left(\lambda t+2\right)\right)\\
&=&\frac{\lambda t\left(\lambda t+2\right)}{2},
\end{eqnarray*}
es decir,


\begin{equation}
\esp\left[R_{1}\right]=\frac{\lambda t\left(\lambda
t+2\right)}{2}.
\end{equation}

Adem\'as para $k\in Z^{+}$ fijo
\begin{eqnarray*}
f_{R_{1}}\left(k\right)&=&\prob\left[R_{1}=k\right]=\sum_{n=0}^{\infty}\prob\left[R_{1}=k|R_{2}=n\right]\prob\left[R_{2}=n\right]\\
&=&\prob\left[R_{1}=k|R_{2}=0\right]\prob\left[R_{2}=0\right]+\prob\left[R_{1}=k|R_{2}=1\right]\prob\left[R_{2}=1\right]\\
&+&\prob\left[R_{1}=k|R_{2}=2\right]\prob\left[R_{2}=2\right]+\cdots+\prob\left[R_{1}=k|R_{2}=j\right]\prob\left[R_{2}=j\right]+\cdots+
\end{eqnarray*}
donde para

\begin{description}
\item[$k=0$:]
\begin{eqnarray*}
\prob\left[R_{1}=0\right]=\prob\left[R_{1}=0|R_{2}=0\right]\prob\left[R_{2}=0\right]+\prob\left[R_{1}=0|R_{2}=1\right]\prob\left[R_{2}=1\right]\\
+\prob\left[R_{1}=0|R_{2}=j\right]\prob\left[R_{2}=j\right]=\prob\left[R_{2}=0\right].
\end{eqnarray*}
\item[$k=1$:]
\begin{eqnarray*}
\prob\left[R_{1}=1\right]=\prob\left[R_{1}=1|R_{2}=0\right]\prob\left[R_{2}=0\right]+\prob\left[R_{1}=1|R_{2}=1\right]\prob\left[R_{2}=1\right]\\
+\prob\left[R_{1}=1|R_{2}=1\right]\prob\left[R_{2}=1\right]+\cdots+\prob\left[R_{1}=1|R_{2}=j\right]\prob\left[R_{2}=j\right]\\
=\sum_{n=1}^{\infty}\prob\left[R_{2}=n\right].
\end{eqnarray*}

\item[$k=2$:]
\begin{eqnarray*}
\prob\left[R_{1}=2\right]=\prob\left[R_{1}=2|R_{2}=0\right]\prob\left[R_{2}=0\right]+\prob\left[R_{1}=2|R_{2}=1\right]\prob\left[R_{2}=1\right]\\
+\prob\left[R_{1}=2|R_{2}=2\right]\prob\left[R_{2}=2\right]+\cdots+\prob\left[R_{1}=2|R_{2}=j\right]\prob\left[R_{2}=j\right]\\
=\sum_{n=2}^{\infty}\prob\left[R_{2}=n\right].
\end{eqnarray*}
\end{description}

En general

\begin{eqnarray*}
\prob\left[R_{1}=k\right]=\prob\left[R_{1}=k|R_{2}=0\right]\prob\left[R_{2}=0\right]+\prob\left[R_{1}=k|R_{2}=1\right]\prob\left[R_{2}=1\right]\\
+\prob\left[R_{1}=k|R_{2}=2\right]\prob\left[R_{2}=2\right]+\cdots+\prob\left[R_{1}=k|R_{2}=k\right]\prob\left[R_{2}=k\right]\\
=\sum_{n=k}^{\infty}\prob\left[R_{2}=n\right].\\
\end{eqnarray*}



Por lo tanto

\begin{eqnarray*}
f_{R_{1}}\left(k\right)&=&\prob\left[R_{1}=k\right]=\sum_{n=k}^{\infty}\prob\left[R_{2}=n\right].
\end{eqnarray*}






\section*{Objetivos Principales}

\begin{itemize}
%\item Generalizar los principales resultados existentes para Sistemas de Visitas C\'iclicas para el caso en el que se tienen dos Sistemas de Visitas C\'iclicas con propiedades similares.

\item Encontrar las ecuaciones que modelan el comportamiento de una Red de Sistemas de Visitas C\'iclicas (RSVC) con propiedades similares.

\item Encontrar expresiones anal\'iticas para las longitudes de las colas al momento en que el servidor llega a una de ellas para comenzar a dar servicio, as\'i como de sus segundos momentos.

\item Determinar las principales medidas de Desempe\~no para la RSVC tales como: N\'umero de usuarios presentes en cada una de las colas del sistema cuando uno de los servidores est\'a presente atendiendo, Tiempos que transcurre entre las visitas del servidor a la misma cola.


\end{itemize}


%_________________________________________________________________________
%\section{Sistemas de Visitas C\'iclicas}
%_________________________________________________________________________
\numberwithin{equation}{section}%
%__________________________________________________________________________




%\section*{Introducci\'on}




%__________________________________________________________________________
%\subsection{Definiciones}
%__________________________________________________________________________


\section{Descripci\'on de una Red de Sistemas de Visitas C\'iclicas}



Consideremos una red de sistema de visitas c\'iclicas conformada por dos sistemas de visitas c\'iclicas, cada una con dos colas independientes, donde adem\'as se permite el intercambio de usuarios entre los dos sistemas en la segunda cola de cada uno de ellos.\smallskip

Sup\'ongase adem\'as que los arribos de los usuarios ocurren
conforme a un proceso Poisson con tasa de llegada $\mu_{1}$ y
$\mu_{2}$ para el sistema 1, mientras que para el sistema 2,
lo hacen conforme a un proceso Poisson con tasa
$\hat{\mu}_{1},\hat{\mu}_{2}$ respectivamente.\smallskip

El traslado de un sistema a otro ocurre de manera que los tiempos
entre llegadas de los usuarios a la cola dos del sistema 1
provenientes del sistema 2, se distribuye de manera exponencial
con par\'ametro $\check{\mu}_{2}$.\smallskip

Se considerar\'an intervalos de tiempo de la forma
$\left[t,t+1\right]$. Los usuarios arriban por paquetes de manera
independiente del resto de las colas. Se define el grupo de
usuarios que llegan a cada una de las colas del sistema 1,
caracterizadas por $Q_{1}$ y $Q_{2}$ respectivamente, en el
intervalo de tiempo $\left[t,t+1\right]$ por
$X_{1}\left(t\right),X_{2}\left(t\right)$. De igual manera se
definen los procesos
$\hat{X}_{1}\left(t\right),\hat{X}_{2}\left(t\right)$ para las
colas del sistema 2, denotadas por $\hat{Q}_{1}$ y $\hat{Q}_{2}$
respectivamente.\smallskip

Para el n\'umero de usuarios que se trasladan del sistema 2 al
sistema 1, de la cola $\hat{Q}_{2}$ a la cola
$Q_{2}$, en el intervalo de tiempo
$\left[t,t+1\right]$, se define el proceso
$Y_{2}\left(t\right)$.\smallskip

El uso de la Funci\'on Generadora de Probabilidades (FGP's) nos permite determinar las Funciones de Distribuci\'on de Probabilidades Conjunta de manera indirecta sin necesidad de hacer uso de las propiedades de las distribuciones de probabilidad de cada uno de los procesos que intervienen en la Red de Sistemas de Visitas C\'iclicas.\smallskip

En lo que respecta al servidor, en t\'erminos de los tiempos de
visita a cada una de las colas, se definen las variables
aleatorias $\tau_{1},\tau_{2}$ para $Q_{1},Q_{2}$ respectivamente;
y $\zeta_{1},\zeta_{2}$ para $\hat{Q}_{1},\hat{Q}_{2}$ del sistema
2. A los tiempos en que el servidor termina de atender en las
colas $Q_{1},Q_{2},\hat{Q}_{1},\hat{Q}_{2}$, se les denotar\'a por
$\overline{\tau}_{1},\overline{\tau}_{2},\overline{\zeta}_{1},\overline{\zeta}_{2}$
respectivamente.\smallskip

Los tiempos de traslado del servidor desde el momento en que termina de atender a una cola y llega a la siguiente para comenzar a dar servicio est\'an dados por
$\tau_{2}-\overline{\tau}_{1},\tau_{1}-\overline{\tau}_{2}$ y
$\zeta_{2}-\overline{\zeta}_{1},\zeta_{1}-\overline{\zeta}_{2}$
para el sistema 1 y el sistema 2, respectivamente.\smallskip

Cada uno de estos procesos con su respectiva FGP. Adem\'as, para cada una de las colas en cada sistema, el n\'umero de usuarios al tiempo en que llega el servidor a dar servicio est\'a
dado por el n\'umero de usuarios presentes en la cola al tiempo
$t$, m\'as el n\'umero de usuarios que llegan a la cola en el intervalo de tiempo
$\left[\tau_{i},\overline{\tau}_{i}\right]$.

%es decir
%{\small{
%\begin{eqnarray*}
%L_{1}\left(\overline{\tau}_{1}\right)=L_{1}\left(\tau_{1}\right)+X_{1}\left(\overline{\tau}_{1}-\tau_{1}\right),\hat{L}_{i}\left(\overline{\tau}_{i}\right)=\hat{L}_{i}\left(\tau_{i}\right)+\hat{X}_{i}\left(\overline{\tau}_{i}-\tau_{i}\right),L_{2}\left(\overline{\tau}_{1}\right)=L_{2}\left(\tau_{1}\right)+X_{2}\left(\overline{\tau}_{1}-\tau_{1}\right)+Y_{2}\left(\overline{\tau}_{1}-\tau_{1}\right),
%\end{eqnarray*}}}




%\begin{center}\vspace{1cm}
%%%%\includegraphics[width=0.6\linewidth]{RedSVC2}
%\captionof{figure}{\color{Green} Red de Sistema de Visitas C\'iclicas}
%\end{center}\vspace{1cm}




Una vez definidas las Funciones Generadoras de Probabilidades Conjuntas se construyen las ecuaciones recursivas que permiten obtener la informaci\'on sobre la longitud de cada una de las colas, al momento en que uno de los servidores llega a una de las colas para dar servicio, bas\'andose en la informaci\'on que se tiene sobre su llegada a la cola inmediata anterior.\smallskip
%{\footnotesize{
%\begin{eqnarray*}
%F_{2}\left(z_{1},z_{2},w_{1},w_{2}\right)&=&R_{1}\left(P_{1}\left(z_{1}\right)\tilde{P}_{2}\left(z_{2}\right)\prod_{i=1}^{2}
%\hat{P}_{i}\left(w_{i}\right)\right)F_{1}\left(\theta_{1}\left(\tilde{P}_{2}\left(z_{2}\right)\hat{P}_{1}\left(w_{1}\right)\hat{P}_{2}\left(w_{2}\right)\right),z_{2},w_{1},w_{2}\right),\\
%F_{1}\left(z_{1},z_{2},w_{1},w_{2}\right)&=&R_{2}\left(P_{1}\left(z_{1}\right)\tilde{P}_{2}\left(z_{2}\right)\prod_{i=1}^{2}
%\hat{P}_{i}\left(w_{i}\right)\right)F_{2}\left(z_{1},\tilde{\theta}_{2}\left(P_{1}\left(z_{1}\right)\hat{P}_{1}\left(w_{1}\right)\hat{P}_{2}\left(w_{2}\right)\right),w_{1},w_{2}\right),\\
%\hat{F}_{2}\left(z_{1},z_{2},w_{1},w_{2}\right)&=&\hat{R}_{1}\left(P_{1}\left(z_{1}\right)\tilde{P}_{2}\left(z_{2}\right)\prod_{i=1}^{2}
%\hat{P}_{i}\left(w_{i}\right)\right)\hat{F}_{1}\left(z_{1},z_{2},\hat{\theta}_{1}\left(P_{1}\left(z_{1}\right)\tilde{P}_{2}\left(z_{2}\right)\hat{P}_{2}\left(w_{2}\right)\right),w_{2}\right),\\
%\end{eqnarray*}}}
%{\small{
%\begin{eqnarray*}
%\hat{F}_{1}\left(z_{1},z_{2},w_{1},w_{2}\right)&=&\hat{R}_{2}\left(P_{1}\left(z_{1}\right)\tilde{P}_{2}\left(z_{2}\right)\prod_{i=1}^{2}
%\hat{P}_{i}\left(w_{i}\right)\right)\hat{F}_{2}\left(z_{1},z_{2},w_{1},\hat{\theta}_{2}\left(P_{1}\left(z_{1}\right)\tilde{P}_{2}\left(z_{2}\right)\hat{P}_{1}\left(w_{1}\right)\right)\right).
%\end{eqnarray*}}}

%__________________________________________________________________________
\subsection{Funciones Generadoras de Probabilidades}
%__________________________________________________________________________


Para cada uno de los procesos de llegada a las colas $X_{1},X_{2},\hat{X}_{1},\hat{X}_{2}$ y $Y_{2}$, con $\tilde{X}_{2}=X_{2}+Y_{2}$ anteriores se define su Funci\'on
Generadora de Probabilidades (FGP):
%\begin{multicols}{3}
\begin{eqnarray*}
\begin{array}{ccc}
P_{1}\left(z_{1}\right)=\esp\left[z_{1}^{X_{1}\left(t\right)}\right],&P_{2}\left(z_{2}\right)=\esp\left[z_{2}^{X_{2}\left(t\right)}\right],&\check{P}_{2}\left(z_{2}\right)=\esp\left[z_{2}^{Y_{2}\left(t\right)}\right],\\
\hat{P}_{1}\left(w_{1}\right)=\esp\left[w_{1}^{\hat{X}_{1}\left(t\right)}\right],&\hat{P}_{2}\left(w_{2}\right)=\esp\left[w_{2}^{\hat{X}_{2}\left(t\right)}\right],&\tilde{P}_{2}\left(z_{2}\right)=\esp\left[z_{2}^{\tilde{X}_{2}\left(t\right)}\right].
\end{array}
\end{eqnarray*}

Con primer momento definidos por

\begin{eqnarray*}
\begin{array}{cc}
\mu_{1}=\esp\left[X_{1}\left(t\right)\right]=P_{1}^{(1)}\left(1\right),&\mu_{2}=\esp\left[X_{2}\left(t\right)\right]=P_{2}^{(1)}\left(1\right),\\
\check{\mu}_{2}=\esp\left[Y_{2}\left(t\right)\right]=\check{P}_{2}^{(1)}\left(1\right),&
\hat{\mu}_{1}=\esp\left[\hat{X}_{1}\left(t\right)\right]=\hat{P}_{1}^{(1)}\left(1\right),\\
\hat{\mu}_{2}=\esp\left[\hat{X}_{2}\left(t\right)\right]=\hat{P}_{2}^{(1)}\left(1\right),&\tilde{\mu}_{2}=\esp\left[\tilde{X}_{2}\left(t\right)\right]=\tilde{P}_{2}^{(1)}\left(1\right).
\end{array}
\end{eqnarray*}

En lo que respecta al servidor, en t\'erminos de los tiempos de
visita a cada una de las colas, se denotar\'an por
$B_{1}\left(t\right),B_{2}\left(t\right)$ los procesos
correspondientes a las variables aleatorias $\tau_{1},\tau_{2}$
para $Q_{1},Q_{2}$ respectivamente; y
$\hat{B}_{1}\left(t\right),\hat{B}_{2}\left(t\right)$ con
par\'ametros $\zeta_{1},\zeta_{2}$ para $\hat{Q}_{1},\hat{Q}_{2}$
del sistema 2. Y a los tiempos en que el servidor termina de
atender en las colas $Q_{1},Q_{2},\hat{Q}_{1},\hat{Q}_{2}$, se les
denotar\'a por
$\overline{\tau}_{1},\overline{\tau}_{2},\overline{\zeta}_{1},\overline{\zeta}_{2}$
respectivamente. Entonces, los tiempos de servicio est\'an dados
por las diferencias
$\overline{\tau}_{1}-\tau_{1},\overline{\tau}_{2}-\tau_{2}$ para
$Q_{1},Q_{2}$, y
$\overline{\zeta}_{1}-\zeta_{1},\overline{\zeta}_{2}-\zeta_{2}$
para $\hat{Q}_{1},\hat{Q}_{2}$ respectivamente.

Sus procesos se definen por:


\begin{eqnarray*}
\begin{array}{cc}
S_{1}\left(z_{1}\right)=\esp\left[z_{1}^{\overline{\tau}_{1}-\tau_{1}}\right],&S_{2}\left(z_{2}\right)=\esp\left[z_{1}^{\overline{\tau}_{2}-\tau_{2}}\right],\\
\hat{S}_{1}\left(w_{1}\right)=\esp\left[w_{1}^{\overline{\zeta}_{1}-\zeta_{1}}\right],&\hat{S}_{2}\left(w_{2}\right)=\esp\left[w_{2}^{\overline{\zeta}_{2}-\zeta_{2}}\right],
\end{array}
\end{eqnarray*}

con primer momento dado por:


\begin{eqnarray*}
\begin{array}{cccc}
s_{1}=\esp\left[\overline{\tau}_{1}-\tau_{1}\right],&s_{2}=\esp\left[\overline{\tau}_{2}-\tau_{2}\right],&
\hat{s}_{1}=\esp\left[\overline{\zeta}_{1}-\zeta_{1}\right],&
\hat{s}_{2}=\esp\left[\overline{\zeta}_{2}-\zeta_{2}\right].
\end{array}
\end{eqnarray*}

An\'alogamente los tiempos de traslado del servidor desde el
momento en que termina de atender a una cola y llega a la
siguiente para comenzar a dar servicio est\'an dados por
$\tau_{2}-\overline{\tau}_{1},\tau_{1}-\overline{\tau}_{2}$ y
$\zeta_{2}-\overline{\zeta}_{1},\zeta_{1}-\overline{\zeta}_{2}$
para el sistema 1 y el sistema 2, respectivamente.

La FGP para estos tiempos de traslado est\'an dados por

\begin{eqnarray*}
\begin{array}{cc}
R_{1}\left(z_{1}\right)=\esp\left[z_{1}^{\tau_{2}-\overline{\tau}_{1}}\right],&R_{2}\left(z_{2}\right)=\esp\left[z_{2}^{\tau_{1}-\overline{\tau}_{2}}\right],\\
\hat{R}_{1}\left(w_{1}\right)=\esp\left[w_{1}^{\zeta_{2}-\overline{\zeta}_{1}}\right],&\hat{R}_{2}\left(w_{2}\right)=\esp\left[w_{2}^{\zeta_{1}-\overline{\zeta}_{2}}\right],
\end{array}
\end{eqnarray*}
y al igual que como se hizo con anterioridad

\begin{eqnarray*}
\begin{array}{cc}
r_{1}=R_{1}^{(1)}\left(1\right)=\esp\left[\tau_{2}-\overline{\tau}_{1}\right],&r_{2}=R_{2}^{(1)}\left(1\right)=\esp\left[\tau_{1}-\overline{\tau}_{2}\right],\\
\hat{r}_{1}=\hat{R}_{1}^{(1)}\left(1\right)=\esp\left[\zeta_{2}-\overline{\zeta}_{1}\right],&
\hat{r}_{2}=\hat{R}_{2}^{(1)}\left(1\right)=\esp\left[\zeta_{1}-\overline{\zeta}_{2}\right].
\end{array}
\end{eqnarray*}

Se definen los procesos de conteo para el n\'umero de usuarios en
cada una de las colas al tiempo $t$,
$L_{1}\left(t\right),L_{2}\left(t\right)$, para
$H_{1}\left(t\right),H_{2}\left(t\right)$ del sistema 1,
respectivamente. Y para el segundo sistema, se tienen los procesos
$\hat{L}_{1}\left(t\right),\hat{L}_{2}\left(t\right)$ para
$\hat{H}_{1}\left(t\right),\hat{H}_{2}\left(t\right)$,
respectivamente, es decir,


\begin{eqnarray*}
\begin{array}{cccc}
H_{1}\left(t\right)=\esp\left[z_{1}^{L_{1}\left(t\right)}\right],&
H_{2}\left(t\right)=\esp\left[z_{2}^{L_{2}\left(t\right)}\right],&
\hat{H}_{1}\left(t\right)=\esp\left[w_{1}^{\hat{L}_{1}\left(t\right)}\right],&\hat{H}_{2}\left(t\right)=\esp\left[w_{2}^{\hat{L}_{2}\left(t\right)}\right].
\end{array}
\end{eqnarray*}
Por lo dicho anteriormente se tiene que el n\'umero de usuarios
presentes en los tiempos $\overline{\tau}_{1},\overline{\tau}_{2},
\overline{\zeta}_{1},\overline{\zeta}_{2}$, es cero, es decir,
 $L_{i}\left(\overline{\tau_{i}}\right)=0,$ y
$\hat{L}_{i}\left(\overline{\zeta_{i}}\right)=0$ para i=1,2 para
cada uno de los dos sistemas.


Para cada una de las colas en cada sistema, el n\'umero de
usuarios al tiempo en que llega el servidor a dar servicio est\'a
dado por el n\'umero de usuarios presentes en la cola al tiempo
$t=\tau_{i},\zeta_{i}$, m\'as el n\'umero de usuarios que llegan a
la cola en el intervalo de tiempo
$\left[\tau_{i},\overline{\tau}_{i}\right],\left[\zeta_{i},\overline{\zeta}_{i}\right]$,
es decir

\begin{eqnarray*}\label{Eq.TiemposLlegada}
\begin{array}{cc}
L_{1}\left(\overline{\tau}_{1}\right)=L_{1}\left(\tau_{1}\right)+X_{1}\left(\overline{\tau}_{1}-\tau_{1}\right),&\hat{L}_{1}\left(\overline{\tau}_{1}\right)=\hat{L}_{1}\left(\tau_{1}\right)+\hat{X}_{1}\left(\overline{\tau}_{1}-\tau_{1}\right),\\
\hat{L}_{2}\left(\overline{\tau}_{1}\right)=\hat{L}_{2}\left(\tau_{1}\right)+\hat{X}_{2}\left(\overline{\tau}_{1}-\tau_{1}\right).&
\end{array}
\end{eqnarray*}

En el caso espec\'ifico de $Q_{2}$, adem\'as, hay que considerar
el n\'umero de usuarios que pasan del sistema 2 al sistema 1, a
traves de $\hat{Q}_{2}$ mientras el servidor en $Q_{2}$ est\'a
ausente, es decir:

\begin{equation}\label{Eq.UsuariosTotalesZ2}
L_{2}\left(\overline{\tau}_{1}\right)=L_{2}\left(\tau_{1}\right)+X_{2}\left(\overline{\tau}_{1}-\tau_{1}\right)+Y_{2}\left(\overline{\tau}_{1}-\tau_{1}\right).
\end{equation}

%_________________________________________________________________________
\subsection{El problema de la ruina del jugador}
%_________________________________________________________________________

Supongamos que se tiene un jugador que cuenta con un capital
inicial de $\tilde{L}_{0}\geq0$ unidades, esta persona realiza una
serie de dos juegos simult\'aneos e independientes de manera
sucesiva, dichos eventos son independientes e id\'enticos entre
s\'i para cada realizaci\'on.\smallskip

La ganancia en el $n$-\'esimo juego es
\begin{eqnarray*}\label{Eq.Cero}
\tilde{X}_{n}=X_{n}+Y_{n}
\end{eqnarray*}

unidades de las cuales se resta una cuota de 1 unidad por cada
juego simult\'aneo, es decir, se restan dos unidades por cada
juego realizado.\smallskip

En t\'erminos de la teor\'ia de colas puede pensarse como el n\'umero de usuarios que llegan a una cola v\'ia dos procesos de arribo distintos e independientes entre s\'i.

Su Funci\'on Generadora de Probabilidades (FGP) est\'a dada por

\begin{eqnarray*}
F\left(z\right)=\esp\left[z^{\tilde{L}_{0}}\right]
\end{eqnarray*}

\begin{eqnarray*}
\tilde{P}\left(z\right)=\esp\left[z^{\tilde{X}_{n}}\right]=\esp\left[z^{X_{n}+Y_{n}}\right]=\esp\left[z^{X_{n}}z^{Y_{n}}\right]=\esp\left[z^{X_{n}}\right]\esp\left[z^{Y_{n}}\right]=P\left(z\right)\check{P}\left(z\right),
\end{eqnarray*}
entonces
\begin{eqnarray*}
\tilde{\mu}&=&\esp\left[\tilde{X}_{n}\right]=\tilde{P}\left[z\right]<1.\\
\end{eqnarray*}

Sea $\tilde{L}_{n}$ el capital remanente despu\'es del $n$-\'esimo
juego. Entonces

\begin{eqnarray*}
\tilde{L}_{n}&=&\tilde{L}_{0}+\tilde{X}_{1}+\tilde{X}_{2}+\cdots+\tilde{X}_{n}-2n.
\end{eqnarray*}

La ruina del jugador ocurre despu\'es del $n$-\'esimo juego, es decir, la cola se vac\'ia despu\'es del $n$-\'esimo juego,
entonces sea $T$ definida como

\begin{eqnarray*}
T&=&min\left\{\tilde{L}_{n}=0\right\}
\end{eqnarray*}

Si $\tilde{L}_{0}=0$, entonces claramente $T=0$. En este sentido $T$
puede interpretarse como la longitud del periodo de tiempo que el servidor ocupa para dar servicio en la cola, comenzando con $\tilde{L}_{0}$ grupos de usuarios
presentes en la cola, quienes arribaron conforme a un proceso dado
por $\tilde{P}\left(z\right)$.\smallskip


Sea $g_{n,k}$ la probabilidad del evento de que el jugador no
caiga en ruina antes del $n$-\'esimo juego, y que adem\'as tenga
un capital de $k$ unidades antes del $n$-\'esimo juego, es decir,

Dada $n\in\left\{1,2,\ldots,\right\}$ y
$k\in\left\{0,1,2,\ldots,\right\}$
\begin{eqnarray*}
g_{n,k}:=P\left\{\tilde{L}_{j}>0, j=1,\ldots,n,
\tilde{L}_{n}=k\right\}
\end{eqnarray*}

la cual adem\'as se puede escribir como:

\begin{eqnarray*}
g_{n,k}&=&P\left\{\tilde{L}_{j}>0, j=1,\ldots,n,
\tilde{L}_{n}=k\right\}=\sum_{j=1}^{k+1}g_{n-1,j}P\left\{\tilde{X}_{n}=k-j+1\right\}\\
&=&\sum_{j=1}^{k+1}g_{n-1,j}P\left\{X_{n}+Y_{n}=k-j+1\right\}=\sum_{j=1}^{k+1}\sum_{l=1}^{j}g_{n-1,j}P\left\{X_{n}+Y_{n}=k-j+1,Y_{n}=l\right\}\\
&=&\sum_{j=1}^{k+1}\sum_{l=1}^{j}g_{n-1,j}P\left\{X_{n}+Y_{n}=k-j+1|Y_{n}=l\right\}P\left\{Y_{n}=l\right\}\\
&=&\sum_{j=1}^{k+1}\sum_{l=1}^{j}g_{n-1,j}P\left\{X_{n}=k-j-l+1\right\}P\left\{Y_{n}=l\right\}\\
\end{eqnarray*}

es decir
\begin{eqnarray}\label{Eq.Gnk.2S}
g_{n,k}=\sum_{j=1}^{k+1}\sum_{l=1}^{j}g_{n-1,j}P\left\{X_{n}=k-j-l+1\right\}P\left\{Y_{n}=l\right\}
\end{eqnarray}
adem\'as

\begin{equation}\label{Eq.L02S}
g_{0,k}=P\left\{\tilde{L}_{0}=k\right\}.
\end{equation}

Se definen las siguientes FGP:
\begin{equation}\label{Eq.3.16.a.2S}
G_{n}\left(z\right)=\sum_{k=0}^{\infty}g_{n,k}z^{k},\textrm{ para
}n=0,1,\ldots,
\end{equation}

\begin{equation}\label{Eq.3.16.b.2S}
G\left(z,w\right)=\sum_{n=0}^{\infty}G_{n}\left(z\right)w^{n}.
\end{equation}


En particular para $k=0$,
\begin{eqnarray*}
g_{n,0}=G_{n}\left(0\right)=P\left\{\tilde{L}_{j}>0,\textrm{ para
}j<n,\textrm{ y }\tilde{L}_{n}=0\right\}=P\left\{T=n\right\},
\end{eqnarray*}

adem\'as

\begin{eqnarray*}%\label{Eq.G0w.2S}
G\left(0,w\right)=\sum_{n=0}^{\infty}G_{n}\left(0\right)w^{n}=\sum_{n=0}^{\infty}P\left\{T=n\right\}w^{n}
=\esp\left[w^{T}\right]
\end{eqnarray*}
la cu\'al resulta ser la FGP del tiempo de ruina $T$.

%__________________________________________________________________________________
% INICIA LA PROPOSICIÓN
%__________________________________________________________________________________


\begin{Prop}\label{Prop.1.1.2S}
Sean $G_{n}\left(z\right)$ y $G\left(z,w\right)$ definidas como en
(\ref{Eq.3.16.a.2S}) y (\ref{Eq.3.16.b.2S}) respectivamente,
entonces
\begin{equation}\label{Eq.Pag.45}
G_{n}\left(z\right)=\frac{1}{z}\left[G_{n-1}\left(z\right)-G_{n-1}\left(0\right)\right]\tilde{P}\left(z\right).
\end{equation}

Adem\'as


\begin{equation}\label{Eq.Pag.46}
G\left(z,w\right)=\frac{zF\left(z\right)-wP\left(z\right)G\left(0,w\right)}{z-wR\left(z\right)},
\end{equation}

con un \'unico polo en el c\'irculo unitario, adem\'as, el polo es
de la forma $z=\theta\left(w\right)$ y satisface que

\begin{enumerate}
\item[i)]$\tilde{\theta}\left(1\right)=1$,

\item[ii)] $\tilde{\theta}^{(1)}\left(1\right)=\frac{1}{1-\tilde{\mu}}$,

\item[iii)]
$\tilde{\theta}^{(2)}\left(1\right)=\frac{\tilde{\mu}}{\left(1-\tilde{\mu}\right)^{2}}+\frac{\tilde{\sigma}}{\left(1-\tilde{\mu}\right)^{3}}$.
\end{enumerate}

Finalmente, adem\'as se cumple que
\begin{equation}
\esp\left[w^{T}\right]=G\left(0,w\right)=F\left[\tilde{\theta}\left(w\right)\right].
\end{equation}
\end{Prop}
%__________________________________________________________________________________
% TERMINA LA PROPOSICIÓN E INICIA LA DEMOSTRACI\'ON
%__________________________________________________________________________________


Multiplicando las ecuaciones (\ref{Eq.Gnk.2S}) y (\ref{Eq.L02S})
por el t\'ermino $z^{k}$:

\begin{eqnarray*}
g_{n,k}z^{k}&=&\sum_{j=1}^{k+1}\sum_{l=1}^{j}g_{n-1,j}P\left\{X_{n}=k-j-l+1\right\}P\left\{Y_{n}=l\right\}z^{k},\\
g_{0,k}z^{k}&=&P\left\{\tilde{L}_{0}=k\right\}z^{k},
\end{eqnarray*}

ahora sumamos sobre $k$
\begin{eqnarray*}
\sum_{k=0}^{\infty}g_{n,k}z^{k}&=&\sum_{k=0}^{\infty}\sum_{j=1}^{k+1}\sum_{l=1}^{j}g_{n-1,j}P\left\{X_{n}=k-j-l+1\right\}P\left\{Y_{n}=l\right\}z^{k}\\
&=&\sum_{k=0}^{\infty}z^{k}\sum_{j=1}^{k+1}\sum_{l=1}^{j}g_{n-1,j}P\left\{X_{n}=k-\left(j+l
-1\right)\right\}P\left\{Y_{n}=l\right\}\\
&=&\sum_{k=0}^{\infty}z^{k+\left(j+l-1\right)-\left(j+l-1\right)}\sum_{j=1}^{k+1}\sum_{l=1}^{j}g_{n-1,j}P\left\{X_{n}=k-
\left(j+l-1\right)\right\}P\left\{Y_{n}=l\right\}\\
&=&\sum_{k=0}^{\infty}\sum_{j=1}^{k+1}\sum_{l=1}^{j}g_{n-1,j}z^{j-1}P\left\{X_{n}=k-
\left(j+l-1\right)\right\}z^{k-\left(j+l-1\right)}P\left\{Y_{n}=l\right\}z^{l}\\
&=&\sum_{j=1}^{\infty}\sum_{l=1}^{j}g_{n-1,j}z^{j-1}\sum_{k=j+l-1}^{\infty}P\left\{X_{n}=k-
\left(j+l-1\right)\right\}z^{k-\left(j+l-1\right)}P\left\{Y_{n}=l\right\}z^{l}\\
&=&\sum_{j=1}^{\infty}g_{n-1,j}z^{j-1}\sum_{l=1}^{j}\sum_{k=j+l-1}^{\infty}P\left\{X_{n}=k-
\left(j+l-1\right)\right\}z^{k-\left(j+l-1\right)}P\left\{Y_{n}=l\right\}z^{l}\\
&=&\sum_{j=1}^{\infty}g_{n-1,j}z^{j-1}\sum_{k=j+l-1}^{\infty}\sum_{l=1}^{j}P\left\{X_{n}=k-
\left(j+l-1\right)\right\}z^{k-\left(j+l-1\right)}P\left\{Y_{n}=l\right\}z^{l}\\
\end{eqnarray*}


luego
\begin{eqnarray*}
&=&\sum_{j=1}^{\infty}g_{n-1,j}z^{j-1}\sum_{k=j+l-1}^{\infty}\sum_{l=1}^{j}P\left\{X_{n}=k-
\left(j+l-1\right)\right\}z^{k-\left(j+l-1\right)}\sum_{l=1}^{j}P
\left\{Y_{n}=l\right\}z^{l}\\
&=&\sum_{j=1}^{\infty}g_{n-1,j}z^{j-1}\sum_{l=1}^{\infty}P\left\{Y_{n}=l\right\}z^{l}
\sum_{k=j+l-1}^{\infty}\sum_{l=1}^{j}
P\left\{X_{n}=k-\left(j+l-1\right)\right\}z^{k-\left(j+l-1\right)}\\
&=&\frac{1}{z}\left[G_{n-1}\left(z\right)-G_{n-1}\left(0\right)\right]\tilde{P}\left(z\right)
\sum_{k=j+l-1}^{\infty}\sum_{l=1}^{j}
P\left\{X_{n}=k-\left(j+l-1\right)\right\}z^{k-\left(j+l-1\right)}\\
&=&\frac{1}{z}\left[G_{n-1}\left(z\right)-G_{n-1}\left(0\right)\right]\tilde{P}\left(z\right)P\left(z\right)=\frac{1}{z}\left[G_{n-1}\left(z\right)-G_{n-1}\left(0\right)\right]\tilde{P}\left(z\right),\\
\end{eqnarray*}

es decir la ecuaci\'on (\ref{Eq.3.16.a.2S}) se puede reescribir
como
\begin{equation}\label{Eq.3.16.a.2Sbis}
G_{n}\left(z\right)=\frac{1}{z}\left[G_{n-1}\left(z\right)-G_{n-1}\left(0\right)\right]\tilde{P}\left(z\right).
\end{equation}

Por otra parte recordemos la ecuaci\'on (\ref{Eq.3.16.a.2S})

\begin{eqnarray*}
G_{n}\left(z\right)&=&\sum_{k=0}^{\infty}g_{n,k}z^{k},\textrm{ entonces }\frac{G_{n}\left(z\right)}{z}=\sum_{k=1}^{\infty}g_{n,k}z^{k-1},\\
\end{eqnarray*}

Por lo tanto utilizando la ecuaci\'on (\ref{Eq.3.16.a.2Sbis}):

\begin{eqnarray*}
G\left(z,w\right)&=&\sum_{n=0}^{\infty}G_{n}\left(z\right)w^{n}=G_{0}\left(z\right)+
\sum_{n=1}^{\infty}G_{n}\left(z\right)w^{n}=F\left(z\right)+\sum_{n=0}^{\infty}\left[G_{n}\left(z\right)-G_{n}\left(0\right)\right]w^{n}\frac{\tilde{P}\left(z\right)}{z}\\
&=&F\left(z\right)+\frac{w}{z}\sum_{n=0}^{\infty}\left[G_{n}\left(z\right)-G_{n}\left(0\right)\right]w^{n-1}\tilde{P}\left(z\right)\\
\end{eqnarray*}

es decir
\begin{eqnarray*}
G\left(z,w\right)&=&F\left(z\right)+\frac{w}{z}\left[G\left(z,w\right)-G\left(0,w\right)\right]\tilde{P}\left(z\right),
\end{eqnarray*}


entonces

\begin{eqnarray*}
G\left(z,w\right)=F\left(z\right)+\frac{w}{z}\left[G\left(z,w\right)-G\left(0,w\right)\right]\tilde{P}\left(z\right)&=&F\left(z\right)+\frac{w}{z}\tilde{P}\left(z\right)G\left(z,w\right)-\frac{w}{z}\tilde{P}\left(z\right)G\left(0,w\right)\\
&\Leftrightarrow&\\
G\left(z,w\right)\left\{1-\frac{w}{z}\tilde{P}\left(z\right)\right\}&=&F\left(z\right)-\frac{w}{z}\tilde{P}\left(z\right)G\left(0,w\right),
\end{eqnarray*}
por lo tanto,
\begin{equation}
G\left(z,w\right)=\frac{zF\left(z\right)-w\tilde{P}\left(z\right)G\left(0,w\right)}{1-w\tilde{P}\left(z\right)}.
\end{equation}


Ahora $G\left(z,w\right)$ es anal\'itica en $|z|=1$. Sean $z,w$ tales que $|z|=1$ y $|w|\leq1$, como $\tilde{P}\left(z\right)$ es FGP
\begin{eqnarray*}
|z-\left(z-w\tilde{P}\left(z\right)\right)|<|z|\Leftrightarrow|w\tilde{P}\left(z\right)|<|z|
\end{eqnarray*}
es decir, se cumplen las condiciones del Teorema de Rouch\'e y por
tanto, $z$ y $z-w\tilde{P}\left(z\right)$ tienen el mismo n\'umero de
ceros en $|z|=1$. Sea $z=\tilde{\theta}\left(w\right)$ la soluci\'on
\'unica de $z-w\tilde{P}\left(z\right)$, es decir

\begin{equation}\label{Eq.Theta.w}
\tilde{\theta}\left(w\right)-w\tilde{P}\left(\tilde{\theta}\left(w\right)\right)=0,
\end{equation}
 con $|\tilde{\theta}\left(w\right)|<1$. Cabe hacer menci\'on que $\tilde{\theta}\left(w\right)$ es la FGP para el tiempo de ruina cuando $\tilde{L}_{0}=1$.


Considerando la ecuaci\'on (\ref{Eq.Theta.w})
\begin{eqnarray*}
&&\frac{\partial}{\partial w}\tilde{\theta}\left(w\right)|_{w=1}-\frac{\partial}{\partial w}\left\{w\tilde{P}\left(\tilde{\theta}\left(w\right)\right)\right\}|_{w=1}=0\\
&&\tilde{\theta}^{(1)}\left(w\right)|_{w=1}-\frac{\partial}{\partial w}w\left\{\tilde{P}\left(\tilde{\theta}\left(w\right)\right)\right\}|_{w=1}-w\frac{\partial}{\partial w}\tilde{P}\left(\tilde{\theta}\left(w\right)\right)|_{w=1}=0\\
&&\tilde{\theta}^{(1)}\left(1\right)-\tilde{P}\left(\tilde{\theta}\left(1\right)\right)-w\left\{\frac{\partial \tilde{P}\left(\tilde{\theta}\left(w\right)\right)}{\partial \tilde{\theta}\left(w\right)}\cdot\frac{\partial\tilde{\theta}\left(w\right)}{\partial w}|_{w=1}\right\}=0\\
&&\tilde{\theta}^{(1)}\left(1\right)-\tilde{P}\left(\tilde{\theta}\left(1\right)
\right)-\tilde{P}^{(1)}\left(\tilde{\theta}\left(1\right)\right)\cdot\tilde{\theta}^{(1)}\left(1\right)=0
\end{eqnarray*}


luego
\begin{eqnarray*}
&&\tilde{\theta}^{(1)}\left(1\right)-\tilde{P}^{(1)}\left(\tilde{\theta}\left(1\right)\right)\cdot
\tilde{\theta}^{(1)}\left(1\right)=\tilde{P}\left(\tilde{\theta}\left(1\right)\right)\\
&&\tilde{\theta}^{(1)}\left(1\right)\left(1-\tilde{P}^{(1)}\left(\tilde{\theta}\left(1\right)\right)\right)
=\tilde{P}\left(\tilde{\theta}\left(1\right)\right)\\
&&\tilde{\theta}^{(1)}\left(1\right)=\frac{\tilde{P}\left(\tilde{\theta}\left(1\right)\right)}{\left(1-\tilde{P}^{(1)}\left(\tilde{\theta}\left(1\right)\right)\right)}=\frac{1}{1-\tilde{\mu}}.
\end{eqnarray*}

Ahora determinemos el segundo momento de $\tilde{\theta}\left(w\right)$,
nuevamente consideremos la ecuaci\'on (\ref{Eq.Theta.w}):

\begin{eqnarray*}
&&\tilde{\theta}\left(w\right)-w\tilde{P}\left(\tilde{\theta}\left(w\right)\right)=0\\
&&\frac{\partial}{\partial w}\left\{\tilde{\theta}\left(w\right)-w\tilde{P}\left(\tilde{\theta}\left(w\right)\right)\right\}=0\\
&&\frac{\partial}{\partial w}\left\{\frac{\partial}{\partial w}\left\{\tilde{\theta}\left(w\right)-w\tilde{P}\left(\tilde{\theta}\left(w\right)\right)\right\}\right\}=0\\
\end{eqnarray*}
luego
\begin{eqnarray*}
&&\frac{\partial}{\partial w}\left\{\frac{\partial}{\partial w}\tilde{\theta}\left(w\right)-\frac{\partial}{\partial w}\left[w\tilde{P}\left(\tilde{\theta}\left(w\right)\right)\right]\right\}
=\frac{\partial}{\partial w}\left\{\frac{\partial}{\partial w}\tilde{\theta}\left(w\right)-\frac{\partial}{\partial w}\left[w\tilde{P}\left(\tilde{\theta}\left(w\right)\right)\right]\right\}\\
&=&\frac{\partial}{\partial w}\left\{\frac{\partial \tilde{\theta}\left(w\right)}{\partial w}-\left[\tilde{P}\left(\tilde{\theta}\left(w\right)\right)+w\frac{\partial}{\partial w}R\left(\tilde{\theta}\left(w\right)\right)\right]\right\}\\
&=&\frac{\partial}{\partial w}\left\{\frac{\partial \tilde{\theta}\left(w\right)}{\partial w}-\left[\tilde{P}\left(\tilde{\theta}\left(w\right)\right)+w\frac{\partial \tilde{P}\left(\tilde{\theta}\left(w\right)\right)}{\partial w}\frac{\partial \tilde{\theta}\left(w\right)}{\partial w}\right]\right\}\\
&=&\frac{\partial}{\partial w}\left\{\tilde{\theta}^{(1)}\left(w\right)-\tilde{P}\left(\tilde{\theta}\left(w\right)\right)-w\tilde{P}^{(1)}\left(\tilde{\theta}\left(w\right)\right)\tilde{\theta}^{(1)}\left(w\right)\right\}\\
&=&\frac{\partial}{\partial w}\tilde{\theta}^{(1)}\left(w\right)-\frac{\partial}{\partial w}\tilde{P}\left(\tilde{\theta}\left(w\right)\right)-\frac{\partial}{\partial w}\left[w\tilde{P}^{(1)}\left(\tilde{\theta}\left(w\right)\right)\tilde{\theta}^{(1)}\left(w\right)\right]\\
\end{eqnarray*}
\begin{eqnarray*}
&=&\frac{\partial}{\partial
w}\tilde{\theta}^{(1)}\left(w\right)-\frac{\partial
\tilde{P}\left(\tilde{\theta}\left(w\right)\right)}{\partial
\tilde{\theta}\left(w\right)}\frac{\partial \tilde{\theta}\left(w\right)}{\partial
w}-\tilde{P}^{(1)}\left(\tilde{\theta}\left(w\right)\right)\tilde{\theta}^{(1)}\left(w\right)\\
&-&w\frac{\partial
\tilde{P}^{(1)}\left(\tilde{\theta}\left(w\right)\right)}{\partial
w}\tilde{\theta}^{(1)}\left(w\right)-w\tilde{P}^{(1)}\left(\tilde{\theta}\left(w\right)\right)\frac{\partial
\tilde{\theta}^{(1)}\left(w\right)}{\partial w}\\
&=&\tilde{\theta}^{(2)}\left(w\right)-\tilde{P}^{(1)}\left(\tilde{\theta}\left(w\right)\right)\tilde{\theta}^{(1)}\left(w\right)
-\tilde{P}^{(1)}\left(\tilde{\theta}\left(w\right)\right)\tilde{\theta}^{(1)}\left(w\right)\\
&-&w\tilde{P}^{(2)}\left(\tilde{\theta}\left(w\right)\right)\left(\tilde{\theta}^{(1)}\left(w\right)\right)^{2}-w\tilde{P}^{(1)}\left(\tilde{\theta}\left(w\right)\right)\tilde{\theta}^{(2)}\left(w\right)\\
&=&\tilde{\theta}^{(2)}\left(w\right)-2\tilde{P}^{(1)}\left(\tilde{\theta}\left(w\right)\right)\tilde{\theta}^{(1)}\left(w\right)\\
&-&w\tilde{P}^{(2)}\left(\tilde{\theta}\left(w\right)\right)\left(\tilde{\theta}^{(1)}\left(w\right)\right)^{2}-w\tilde{P}^{(1)}\left(\tilde{\theta}\left(w\right)\right)\tilde{\theta}^{(2)}\left(w\right)\\
&=&\tilde{\theta}^{(2)}\left(w\right)\left[1-w\tilde{P}^{(1)}\left(\tilde{\theta}\left(w\right)\right)\right]-
\tilde{\theta}^{(1)}\left(w\right)\left[w\tilde{\theta}^{(1)}\left(w\right)\tilde{P}^{(2)}\left(\tilde{\theta}\left(w\right)\right)+2\tilde{P}^{(1)}\left(\tilde{\theta}\left(w\right)\right)\right]
\end{eqnarray*}


luego

\begin{eqnarray*}
\tilde{\theta}^{(2)}\left(w\right)\left[1-w\tilde{P}^{(1)}\left(\tilde{\theta}\left(w\right)\right)\right]&-&\tilde{\theta}^{(1)}\left(w\right)\left[w\tilde{\theta}^{(1)}\left(w\right)\tilde{P}^{(2)}\left(\tilde{\theta}\left(w\right)\right)
+2\tilde{P}^{(1)}\left(\tilde{\theta}\left(w\right)\right)\right]=0\\
\tilde{\theta}^{(2)}\left(w\right)&=&\frac{\tilde{\theta}^{(1)}\left(w\right)\left[w\tilde{\theta}^{(1)}\left(w\right)\tilde{P}^{(2)}\left(\tilde{\theta}\left(w\right)\right)+2R^{(1)}\left(\tilde{\theta}\left(w\right)\right)\right]}{1-w\tilde{P}^{(1)}\left(\tilde{\theta}\left(w\right)\right)}\\
\tilde{\theta}^{(2)}\left(w\right)&=&\frac{\tilde{\theta}^{(1)}\left(w\right)w\tilde{\theta}^{(1)}\left(w\right)\tilde{P}^{(2)}\left(\tilde{\theta}\left(w\right)\right)}{1-w\tilde{P}^{(1)}\left(\tilde{\theta}\left(w\right)\right)}+\frac{2\tilde{\theta}^{(1)}\left(w\right)\tilde{P}^{(1)}\left(\tilde{\theta}\left(w\right)\right)}{1-w\tilde{P}^{(1)}\left(\tilde{\theta}\left(w\right)\right)}
\end{eqnarray*}


si evaluamos la expresi\'on anterior en $w=1$:
\begin{eqnarray*}
\tilde{\theta}^{(2)}\left(1\right)&=&\frac{\left(\tilde{\theta}^{(1)}\left(1\right)\right)^{2}\tilde{P}^{(2)}\left(\tilde{\theta}\left(1\right)\right)}{1-\tilde{P}^{(1)}\left(\tilde{\theta}\left(1\right)\right)}+\frac{2\tilde{\theta}^{(1)}\left(1\right)\tilde{P}^{(1)}\left(\tilde{\theta}\left(1\right)\right)}{1-\tilde{P}^{(1)}\left(\tilde{\theta}\left(1\right)\right)}=\frac{\left(\tilde{\theta}^{(1)}\left(1\right)\right)^{2}\tilde{P}^{(2)}\left(1\right)}{1-\tilde{P}^{(1)}\left(1\right)}+\frac{2\tilde{\theta}^{(1)}\left(1\right)\tilde{P}^{(1)}\left(1\right)}{1-\tilde{P}^{(1)}\left(1\right)}\\
&=&\frac{\left(\frac{1}{1-\tilde{\mu}}\right)^{2}\tilde{P}^{(2)}\left(1\right)}{1-\tilde{\mu}}+\frac{2\left(\frac{1}{1-\tilde{\mu}}\right)\tilde{\mu}}{1-\tilde{\mu}}=\frac{\tilde{P}^{(2)}\left(1\right)}{\left(1-\tilde{\mu}\right)^{3}}+\frac{2\tilde{\mu}}{\left(1-\tilde{\mu}\right)^{2}}=\frac{\sigma^{2}-\tilde{\mu}+\tilde{\mu}^{2}}{\left(1-\tilde{\mu}\right)^{3}}+\frac{2\tilde{\mu}}{\left(1-\tilde{\mu}\right)^{2}}\\
&=&\frac{\sigma^{2}-\tilde{\mu}+\tilde{\mu}^{2}+2\tilde{\mu}\left(1-\tilde{\mu}\right)}{\left(1-\tilde{\mu}\right)^{3}}\\
\end{eqnarray*}


es decir
\begin{eqnarray*}
\tilde{\theta}^{(2)}\left(1\right)&=&\frac{\sigma^{2}+\tilde{\mu}-\tilde{\mu}^{2}}{\left(1-\tilde{\mu}\right)^{3}}=\frac{\sigma^{2}}{\left(1-\tilde{\mu}\right)^{3}}+\frac{\tilde{\mu}\left(1-\tilde{\mu}\right)}{\left(1-\tilde{\mu}\right)^{3}}=\frac{\sigma^{2}}{\left(1-\tilde{\mu}\right)^{3}}+\frac{\tilde{\mu}}{\left(1-\tilde{\mu}\right)^{2}}.
\end{eqnarray*}

\begin{Coro}
El tiempo de ruina del jugador tiene primer y segundo momento
dados por

\begin{eqnarray}
\esp\left[T\right]&=&\frac{\esp\left[\tilde{L}_{0}\right]}{1-\tilde{\mu}}\\
Var\left[T\right]&=&\frac{Var\left[\tilde{L}_{0}\right]}{\left(1-\tilde{\mu}\right)^{2}}+\frac{\sigma^{2}\esp\left[\tilde{L}_{0}\right]}{\left(1-\tilde{\mu}\right)^{3}}.
\end{eqnarray}
\end{Coro}



%__________________________________________________________________________
\section{Procesos de Llegadas a las colas en la RSVC}
%__________________________________________________________________________

Se definen los procesos de llegada de los usuarios a cada una de
las colas dependiendo de la llegada del servidor pero del sistema
al cu\'al no pertenece la cola en cuesti\'on:

Para el sistema 1 y el servidor del segundo sistema

\begin{eqnarray*}
F_{i,j}\left(z_{i};\zeta_{j}\right)=\esp\left[z_{i}^{L_{i}\left(\zeta_{j}\right)}\right]=
\sum_{k=0}^{\infty}\prob\left[L_{i}\left(\zeta_{j}\right)=k\right]z_{i}^{k}\textrm{, para }i,j=1,2.
%F_{1,1}\left(z_{1};\zeta_{1}\right)&=&\esp\left[z_{1}^{L_{1}\left(\zeta_{1}\right)}\right]=
%\sum_{k=0}^{\infty}\prob\left[L_{1}\left(\zeta_{1}\right)=k\right]z_{1}^{k};\\
%F_{2,1}\left(z_{2};\zeta_{1}\right)&=&\esp\left[z_{2}^{L_{2}\left(\zeta_{1}\right)}\right]=
%\sum_{k=0}^{\infty}\prob\left[L_{2}\left(\zeta_{1}\right)=k\right]z_{2}^{k};\\
%F_{1,2}\left(z_{1};\zeta_{2}\right)&=&\esp\left[z_{1}^{L_{1}\left(\zeta_{2}\right)}\right]=
%\sum_{k=0}^{\infty}\prob\left[L_{1}\left(\zeta_{2}\right)=k\right]z_{1}^{k};\\
%F_{2,2}\left(z_{2};\zeta_{2}\right)&=&\esp\left[z_{2}^{L_{2}\left(\zeta_{2}\right)}\right]=
%\sum_{k=0}^{\infty}\prob\left[L_{2}\left(\zeta_{2}\right)=k\right]z_{2}^{k}.\\
\end{eqnarray*}

Ahora se definen para el segundo sistema y el servidor del primero


\begin{eqnarray*}
\hat{F}_{i,j}\left(w_{i};\tau_{j}\right)&=&\esp\left[w_{i}^{\hat{L}_{i}\left(\tau_{j}\right)}\right] =\sum_{k=0}^{\infty}\prob\left[\hat{L}_{i}\left(\tau_{j}\right)=k\right]w_{i}^{k}\textrm{, para }i,j=1,2.
%\hat{F}_{1,1}\left(w_{1};\tau_{1}\right)&=&\esp\left[w_{1}^{\hat{L}_{1}\left(\tau_{1}\right)}\right] =\sum_{k=0}^{\infty}\prob\left[\hat{L}_{1}\left(\tau_{1}\right)=k\right]w_{1}^{k}\\
%\hat{F}_{2,1}\left(w_{2};\tau_{1}\right)&=&\esp\left[w_{2}^{\hat{L}_{2}\left(\tau_{1}\right)}\right] =\sum_{k=0}^{\infty}\prob\left[\hat{L}_{2}\left(\tau_{1}\right)=k\right]w_{2}^{k}\\
%\hat{F}_{1,2}\left(w_{1};\tau_{2}\right)&=&\esp\left[w_{1}^{\hat{L}_{1}\left(\tau_{2}\right)}\right]
%=\sum_{k=0}^{\infty}\prob\left[\hat{L}_{1}\left(\tau_{2}\right)=k\right]w_{1}^{k}\\
%\hat{F}_{2,2}\left(w_{2};\tau_{2}\right)&=&\esp\left[w_{2}^{\hat{L}_{2}\left(\tau_{2}\right)}\right]
%=\sum_{k=0}^{\infty}\prob\left[\hat{L}_{2}\left(\tau_{2}\right)=k\right]w_{2}^{k}\\
\end{eqnarray*}


Ahora, con lo anterior definamos la FGP conjunta para el segundo sistema;% y $\tau_{1}$:


\begin{eqnarray*}
\esp\left[w_{1}^{\hat{L}_{1}\left(\tau_{j}\right)}w_{2}^{\hat{L}_{2}\left(\tau_{j}\right)}\right]
&=&\esp\left[w_{1}^{\hat{L}_{1}\left(\tau_{j}\right)}\right]
\esp\left[w_{2}^{\hat{L}_{2}\left(\tau_{j}\right)}\right]=\hat{F}_{1,j}\left(w_{1};\tau_{j}\right)\hat{F}_{2,j}\left(w_{2};\tau_{j}\right)=\hat{F}_{j}\left(w_{1},w_{2};\tau_{j}\right).\\
%\esp\left[w_{1}^{\hat{L}_{1}\left(\tau_{1}\right)}w_{2}^{\hat{L}_{2}\left(\tau_{1}\right)}\right]
%&=&\esp\left[w_{1}^{\hat{L}_{1}\left(\tau_{1}\right)}\right]
%\esp\left[w_{2}^{\hat{L}_{2}\left(\tau_{1}\right)}\right]=\hat{F}_{1,1}\left(w_{1};\tau_{1}\right)\hat{F}_{2,1}\left(w_{2};\tau_{1}\right)=\hat{F}_{1}\left(w_{1},w_{2};\tau_{1}\right)\\
%\esp\left[w_{1}^{\hat{L}_{1}\left(\tau_{2}\right)}w_{2}^{\hat{L}_{2}\left(\tau_{2}\right)}\right]
%&=&\esp\left[w_{1}^{\hat{L}_{1}\left(\tau_{2}\right)}\right]
%   \esp\left[w_{2}^{\hat{L}_{2}\left(\tau_{2}\right)}\right]=\hat{F}_{1,2}\left(w_{1};\tau_{2}\right)\hat{F}_{2,2}\left(w_{2};\tau_{2}\right)=\hat{F}_{2}\left(w_{1},w_{2};\tau_{2}\right).
\end{eqnarray*}

Con respecto al sistema 1 se tiene la FGP conjunta con respecto al servidor del otro sistema:
\begin{eqnarray*}
\esp\left[z_{1}^{L_{1}\left(\zeta_{j}\right)}z_{2}^{L_{2}\left(\zeta_{j}\right)}\right]
&=&\esp\left[z_{1}^{L_{1}\left(\zeta_{j}\right)}\right]
\esp\left[z_{2}^{L_{2}\left(\zeta_{j}\right)}\right]=F_{1,j}\left(z_{1};\zeta_{j}\right)F_{2,j}\left(z_{2};\zeta_{j}\right)=F_{j}\left(z_{1},z_{2};\zeta_{j}\right).
%\esp\left[z_{1}^{L_{1}\left(\zeta_{1}\right)}z_{2}^{L_{2}\left(\zeta_{1}\right)}\right]
%&=&\esp\left[z_{1}^{L_{1}\left(\zeta_{1}\right)}\right]
%\esp\left[z_{2}^{L_{2}\left(\zeta_{1}\right)}\right]=F_{1,1}\left(z_{1};\zeta_{1}\right)F_{2,1}\left(z_{2};\zeta_{1}\right)=F_{1}\left(z_{1},z_{2};\zeta_{1}\right)\\
%\esp\left[z_{1}^{L_{1}\left(\zeta_{2}\right)}z_{2}^{L_{2}\left(\zeta_{2}\right)}\right]
%&=&\esp\left[z_{1}^{L_{1}\left(\zeta_{2}\right)}\right]
%\esp\left[z_{2}^{L_{2}\left(\zeta_{2}\right)}\right]=F_{1,2}\left(z_{1};\zeta_{2}\right)F_{2,2}\left(z_{2};\zeta_{2}\right)=F_{2}\left(z_{1},z_{2};\zeta_{2}\right).
\end{eqnarray*}

Ahora analicemos la Red de Sistemas de Visitas C\'iclicas, entonces se define la PGF conjunta al tiempo $t$ para los tiempos de visita del servidor en cada una de las colas, para comenzar a dar servicio, definidos anteriormente al tiempo
$t=\left\{\tau_{1},\tau_{2},\zeta_{1},\zeta_{2}\right\}$:

\begin{eqnarray}\label{Eq.Conjuntas}
F_{j}\left(z_{1},z_{2},w_{1},w_{2}\right)&=&\esp\left[\prod_{i=1}^{2}z_{i}^{L_{i}\left(\tau_{j}
\right)}\prod_{i=1}^{2}w_{i}^{\hat{L}_{i}\left(\tau_{j}\right)}\right]\\
\hat{F}_{j}\left(z_{1},z_{2},w_{1},w_{2}\right)&=&\esp\left[\prod_{i=1}^{2}z_{i}^{L_{i}
\left(\zeta_{j}\right)}\prod_{i=1}^{2}w_{i}^{\hat{L}_{i}\left(\zeta_{j}\right)}\right]
\end{eqnarray}
para $j=1,2$. Entonces, con la finalidad de encontrar el n\'umero de usuarios
presentes en el sistema cuando el servidor deja de atender una de
las colas de cualquier sistema se tiene lo siguiente


\begin{eqnarray*}
&&\esp\left[z_{1}^{L_{1}\left(\overline{\tau}_{1}\right)}z_{2}^{L_{2}\left(\overline{\tau}_{1}\right)}w_{1}^{\hat{L}_{1}\left(\overline{\tau}_{1}\right)}w_{2}^{\hat{L}_{2}\left(\overline{\tau}_{1}\right)}\right]=
\esp\left[z_{2}^{L_{2}\left(\overline{\tau}_{1}\right)}w_{1}^{\hat{L}_{1}\left(\overline{\tau}_{1}
\right)}w_{2}^{\hat{L}_{2}\left(\overline{\tau}_{1}\right)}\right]\\
&=&\esp\left[z_{2}^{L_{2}\left(\tau_{1}\right)+X_{2}\left(\overline{\tau}_{1}-\tau_{1}\right)+Y_{2}\left(\overline{\tau}_{1}-\tau_{1}\right)}w_{1}^{\hat{L}_{1}\left(\tau_{1}\right)+\hat{X}_{1}\left(\overline{\tau}_{1}-\tau_{1}\right)}w_{2}^{\hat{L}_{2}\left(\tau_{1}\right)+\hat{X}_{2}\left(\overline{\tau}_{1}-\tau_{1}\right)}\right]
\end{eqnarray*}
utilizando la ecuacion dada (\ref{Eq.TiemposLlegada}), luego


\begin{eqnarray*}
&=&\esp\left[z_{2}^{L_{2}\left(\tau_{1}\right)}z_{2}^{X_{2}\left(\overline{\tau}_{1}-\tau_{1}\right)}z_{2}^{Y_{2}\left(\overline{\tau}_{1}-\tau_{1}\right)}w_{1}^{\hat{L}_{1}\left(\tau_{1}\right)}w_{1}^{\hat{X}_{1}\left(\overline{\tau}_{1}-\tau_{1}\right)}w_{2}^{\hat{L}_{2}\left(\tau_{1}\right)}w_{2}^{\hat{X}_{2}\left(\overline{\tau}_{1}-\tau_{1}\right)}\right]\\
&=&\esp\left[z_{2}^{L_{2}\left(\tau_{1}\right)}\left\{w_{1}^{\hat{L}_{1}\left(\tau_{1}\right)}w_{2}^{\hat{L}_{2}\left(\tau_{1}\right)}\right\}\left\{z_{2}^{X_{2}\left(\overline{\tau}_{1}-\tau_{1}\right)}
z_{2}^{Y_{2}\left(\overline{\tau}_{1}-\tau_{1}\right)}w_{1}^{\hat{X}_{1}\left(\overline{\tau}_{1}-\tau_{1}\right)}w_{2}^{\hat{X}_{2}\left(\overline{\tau}_{1}-\tau_{1}\right)}\right\}\right]\\
\end{eqnarray*}
Aplicando la ecuaci\'on (\ref{Eq.Cero})

\begin{eqnarray*}
&=&\esp\left[z_{2}^{L_{2}\left(\tau_{1}\right)}\left\{z_{2}^{X_{2}\left(\overline{\tau}_{1}-\tau_{1}\right)}z_{2}^{Y_{2}\left(\overline{\tau}_{1}-\tau_{1}\right)}w_{1}^{\hat{X}_{1}\left(\overline{\tau}_{1}-\tau_{1}\right)}w_{2}^{\hat{X}_{2}\left(\overline{\tau}_{1}-\tau_{1}\right)}\right\}\right]\esp\left[w_{1}^{\hat{L}_{1}\left(\tau_{1}\right)}w_{2}^{\hat{L}_{2}\left(\tau_{1}\right)}\right]
\end{eqnarray*}
dado que los arribos a cada una de las colas son independientes, podemos separar la esperanza para los procesos de llegada a $Q_{1}$ y $Q_{2}$ en $\tau_{1}$

Recordando que $\tilde{X}_{2}\left(z_{2}\right)=X_{2}\left(z_{2}\right)+Y_{2}\left(z_{2}\right)$ se tiene


\begin{eqnarray*}
&=&\esp\left[z_{2}^{L_{2}\left(\tau_{1}\right)}\left\{z_{2}^{\tilde{X}_{2}\left(\overline{\tau}_{1}-\tau_{1}\right)}w_{1}^{\hat{X}_{1}\left(\overline{\tau}_{1}-\tau_{1}\right)}w_{2}^{\hat{X}_{2}\left(\overline{\tau}_{1}-\tau_{1}\right)}\right\}\right]\esp\left[w_{1}^{\hat{L}_{1}\left(\tau_{1}\right)}w_{2}^{\hat{L}_{2}\left(\tau_{1}\right)}\right]\\
&=&\esp\left[z_{2}^{L_{2}\left(\tau_{1}\right)}\left\{\tilde{P}_{2}\left(z_{2}\right)^{\overline{\tau}_{1}-\tau_{1}}\hat{P}_{1}\left(w_{1}\right)^{\overline{\tau}_{1}-\tau_{1}}\hat{P}_{2}\left(w_{2}\right)^{\overline{\tau}_{1}-\tau_{1}}\right\}\right]\esp\left[w_{1}^{\hat{L}_{1}\left(\tau_{1}\right)}w_{2}^{\hat{L}_{2}\left(\tau_{1}\right)}\right]\\
&=&\esp\left[z_{2}^{L_{2}\left(\tau_{1}\right)}\left\{\tilde{P}_{2}\left(z_{2}\right)\hat{P}_{1}\left(w_{1}\right)\hat{P}_{2}\left(w_{2}\right)\right\}^{\overline{\tau}_{1}-\tau_{1}}\right]\esp\left[w_{1}^{\hat{L}_{1}\left(\tau_{1}\right)}w_{2}^{\hat{L}_{2}\left(\tau_{1}\right)}\right]\\
\end{eqnarray*}

Entonces


\begin{eqnarray*}
&=&\esp\left[z_{2}^{L_{2}\left(\tau_{1}\right)}\theta_{1}\left(\tilde{P}_{2}\left(z_{2}\right)\hat{P}_{1}\left(w_{1}\right)\hat{P}_{2}\left(w_{2}\right)\right)^{L_{1}\left(\tau_{1}\right)}\right]\esp\left[w_{1}^{\hat{L}_{1}\left(\tau_{1}\right)}w_{2}^{\hat{L}_{2}\left(\tau_{1}\right)}\right]\\
&=&F_{1}\left(\theta_{1}\left(\tilde{P}_{2}\left(z_{2}\right)\hat{P}_{1}\left(w_{1}\right)\hat{P}_{2}\left(w_{2}\right)\right),z{2}\right)\hat{F}_{1}\left(w_{1},w_{2};\tau_{1}\right)\\
&\equiv&
F_{1}\left(\theta_{1}\left(\tilde{P}_{2}\left(z_{2}\right)\hat{P}_{1}\left(w_{1}\right)\hat{P}_{2}\left(w_{2}\right)\right),z_{2},w_{1},w_{2}\right)
\end{eqnarray*}

Las igualdades anteriores son ciertas pues el n\'umero de usuarios
que llegan a $\hat{Q}_{2}$ durante el intervalo
$\left[\tau_{1},\overline{\tau}_{1}\right]$ a\'un no han sido
atendidos por el servidor del sistema $2$ y por tanto a\'un no
pueden pasar al sistema $1$ por $Q_{2}$. Por tanto el n\'umero de
usuarios que pasan de $\hat{Q}_{2}$ a $Q_{2}$ en el intervalo de
tiempo $\left[\tau_{1},\overline{\tau}_{1}\right]$ depende de la
pol\'itica de traslado entre los dos sistemas, conforme a la
secci\'on anterior.\smallskip

Por lo tanto
\begin{eqnarray}\label{Eq.Fs}
\esp\left[z_{1}^{L_{1}\left(\overline{\tau}_{1}\right)}z_{2}^{L_{2}\left(\overline{\tau}_{1}
\right)}w_{1}^{\hat{L}_{1}\left(\overline{\tau}_{1}\right)}w_{2}^{\hat{L}_{2}\left(
\overline{\tau}_{1}\right)}\right]&=&F_{1}\left(\theta_{1}\left(\tilde{P}_{2}\left(z_{2}\right)
\hat{P}_{1}\left(w_{1}\right)\hat{P}_{2}\left(w_{2}\right)\right),z_{2},w_{1},w_{2}\right)\\
&=&F_{1}\left(\theta_{1}\left(\tilde{P}_{2}\left(z_{2}\right)\hat{P}_{1}\left(w_{1}\right)\hat{P}_{2}\left(w_{2}\right)\right),z{2}\right)\hat{F}_{1}\left(w_{1},w_{2};\tau_{1}\right)
\end{eqnarray}


Utilizando un razonamiento an\'alogo para $\overline{\tau}_{2}$:



\begin{eqnarray*}
&&\esp\left[z_{1}^{L_{1}\left(\overline{\tau}_{2}\right)}z_{2}^{L_{2}\left(\overline{\tau}_{2}\right)}w_{1}^{\hat{L}_{1}\left(\overline{\tau}_{2}\right)}w_{2}^{\hat{L}_{2}\left(\overline{\tau}_{2}\right)}\right]=
\esp\left[z_{1}^{L_{1}\left(\overline{\tau}_{2}\right)}w_{1}^{\hat{L}_{1}\left(\overline{\tau}_{2}\right)}w_{2}^{\hat{L}_{2}\left(\overline{\tau}_{2}\right)}\right]\\
&=&\esp\left[z_{1}^{L_{1}\left(\tau_{2}\right)+X_{1}\left(\overline{\tau}_{2}-\tau_{2}\right)}w_{1}^{\hat{L}_{1}\left(\tau_{2}\right)+\hat{X}_{1}\left(\overline{\tau}_{2}-\tau_{2}\right)}w_{2}^{\hat{L}_{2}\left(\tau_{2}\right)+\hat{X}_{2}\left(\overline{\tau}_{2}-\tau_{2}\right)}\right]\\
&=&\esp\left[z_{1}^{L_{1}\left(\tau_{2}\right)}z_{1}^{X_{1}\left(\overline{\tau}_{2}-\tau_{2}\right)}w_{1}^{\hat{L}_{1}\left(\tau_{2}\right)}w_{1}^{\hat{X}_{1}\left(\overline{\tau}_{2}-\tau_{2}\right)}w_{2}^{\hat{L}_{2}\left(\tau_{2}\right)}w_{2}^{\hat{X}_{2}\left(\overline{\tau}_{2}-\tau_{2}\right)}\right]\\
&=&\esp\left[z_{1}^{L_{1}\left(\tau_{2}\right)}z_{1}^{X_{1}\left(\overline{\tau}_{2}-\tau_{2}\right)}w_{1}^{\hat{X}_{1}\left(\overline{\tau}_{2}-\tau_{2}\right)}w_{2}^{\hat{X}_{2}\left(\overline{\tau}_{2}-\tau_{2}\right)}\right]\esp\left[w_{1}^{\hat{L}_{1}\left(\tau_{2}\right)}w_{2}^{\hat{L}_{2}\left(\tau_{2}\right)}\right]\\
&=&\esp\left[z_{1}^{L_{1}\left(\tau_{2}\right)}P_{1}\left(z_{1}\right)^{\overline{\tau}_{2}-\tau_{2}}\hat{P}_{1}\left(w_{1}\right)^{\overline{\tau}_{2}-\tau_{2}}\hat{P}_{2}\left(w_{2}\right)^{\overline{\tau}_{2}-\tau_{2}}\right]
\esp\left[w_{1}^{\hat{L}_{1}\left(\tau_{2}\right)}w_{2}^{\hat{L}_{2}\left(\tau_{2}\right)}\right]
\end{eqnarray*}
utlizando la proposici\'on relacionada con la ruina del jugador


\begin{eqnarray*}
&=&\esp\left[z_{1}^{L_{1}\left(\tau_{2}\right)}\left\{P_{1}\left(z_{1}\right)\hat{P}_{1}\left(w_{1}\right)\hat{P}_{2}\left(w_{2}\right)\right\}^{\overline{\tau}_{2}-\tau_{2}}\right]
\esp\left[w_{1}^{\hat{L}_{1}\left(\tau_{2}\right)}w_{2}^{\hat{L}_{2}\left(\tau_{2}\right)}\right]\\
&=&\esp\left[z_{1}^{L_{1}\left(\tau_{2}\right)}\tilde{\theta}_{2}\left(P_{1}\left(z_{1}\right)\hat{P}_{1}\left(w_{1}\right)\hat{P}_{2}\left(w_{2}\right)\right)^{L_{2}\left(\tau_{2}\right)}\right]
\esp\left[w_{1}^{\hat{L}_{1}\left(\tau_{2}\right)}w_{2}^{\hat{L}_{2}\left(\tau_{2}\right)}\right]\\
&=&F_{2}\left(z_{1},\tilde{\theta}_{2}\left(P_{1}\left(z_{1}\right)\hat{P}_{1}\left(w_{1}\right)\hat{P}_{2}\left(w_{2}\right)\right)\right)
\hat{F}_{2}\left(w_{1},w_{2};\tau_{2}\right)\\
\end{eqnarray*}


entonces se define
\begin{eqnarray}
\esp\left[z_{1}^{L_{1}\left(\overline{\tau}_{2}\right)}z_{2}^{L_{2}\left(\overline{\tau}_{2}\right)}w_{1}^{\hat{L}_{1}\left(\overline{\tau}_{2}\right)}w_{2}^{\hat{L}_{2}\left(\overline{\tau}_{2}\right)}\right]=F_{2}\left(z_{1},\tilde{\theta}_{2}\left(P_{1}\left(z_{1}\right)\hat{P}_{1}\left(w_{1}\right)\hat{P}_{2}\left(w_{2}\right)\right),w_{1},w_{2}\right)\\
\equiv F_{2}\left(z_{1},\tilde{\theta}_{2}\left(P_{1}\left(z_{1}\right)\hat{P}_{1}\left(w_{1}\right)\hat{P}_{2}\left(w_{2}\right)\right)\right)
\hat{F}_{2}\left(w_{1},w_{2};\tau_{2}\right)
\end{eqnarray}
Ahora para $\overline{\zeta}_{1}:$
\begin{eqnarray*}
&&\esp\left[z_{1}^{L_{1}\left(\overline{\zeta}_{1}\right)}z_{2}^{L_{2}\left(\overline{\zeta}_{1}\right)}w_{1}^{\hat{L}_{1}\left(\overline{\zeta}_{1}\right)}w_{2}^{\hat{L}_{2}\left(\overline{\zeta}_{1}\right)}\right]=
\esp\left[z_{1}^{L_{1}\left(\overline{\zeta}_{1}\right)}z_{2}^{L_{2}\left(\overline{\zeta}_{1}\right)}w_{2}^{\hat{L}_{2}\left(\overline{\zeta}_{1}\right)}\right]\\
%&=&\esp\left[z_{1}^{L_{1}\left(\zeta_{1}\right)+X_{1}\left(\overline{\zeta}_{1}-\zeta_{1}\right)}z_{2}^{L_{2}\left(\zeta_{1}\right)+X_{2}\left(\overline{\zeta}_{1}-\zeta_{1}\right)+\hat{Y}_{2}\left(\overline{\zeta}_{1}-\zeta_{1}\right)}w_{2}^{\hat{L}_{2}\left(\zeta_{1}\right)+\hat{X}_{2}\left(\overline{\zeta}_{1}-\zeta_{1}\right)}\right]\\
&=&\esp\left[z_{1}^{L_{1}\left(\zeta_{1}\right)}z_{1}^{X_{1}\left(\overline{\zeta}_{1}-\zeta_{1}\right)}z_{2}^{L_{2}\left(\zeta_{1}\right)}z_{2}^{X_{2}\left(\overline{\zeta}_{1}-\zeta_{1}\right)}
z_{2}^{Y_{2}\left(\overline{\zeta}_{1}-\zeta_{1}\right)}w_{2}^{\hat{L}_{2}\left(\zeta_{1}\right)}w_{2}^{\hat{X}_{2}\left(\overline{\zeta}_{1}-\zeta_{1}\right)}\right]\\
&=&\esp\left[z_{1}^{L_{1}\left(\zeta_{1}\right)}z_{2}^{L_{2}\left(\zeta_{1}\right)}\right]\esp\left[z_{1}^{X_{1}\left(\overline{\zeta}_{1}-\zeta_{1}\right)}z_{2}^{\tilde{X}_{2}\left(\overline{\zeta}_{1}-\zeta_{1}\right)}w_{2}^{\hat{X}_{2}\left(\overline{\zeta}_{1}-\zeta_{1}\right)}w_{2}^{\hat{L}_{2}\left(\zeta_{1}\right)}\right]\\
&=&\esp\left[z_{1}^{L_{1}\left(\zeta_{1}\right)}z_{2}^{L_{2}\left(\zeta_{1}\right)}\right]
\esp\left[P_{1}\left(z_{1}\right)^{\overline{\zeta}_{1}-\zeta_{1}}\tilde{P}_{2}\left(z_{2}\right)^{\overline{\zeta}_{1}-\zeta_{1}}\hat{P}_{2}\left(w_{2}\right)^{\overline{\zeta}_{1}-\zeta_{1}}w_{2}^{\hat{L}_{2}\left(\zeta_{1}\right)}\right]\\
&=&\esp\left[z_{1}^{L_{1}\left(\zeta_{1}\right)}z_{2}^{L_{2}\left(\zeta_{1}\right)}\right]
\esp\left[\left\{P_{1}\left(z_{1}\right)\tilde{P}_{2}\left(z_{2}\right)\hat{P}_{2}\left(w_{2}\right)\right\}^{\overline{\zeta}_{1}-\zeta_{1}}w_{2}^{\hat{L}_{2}\left(\zeta_{1}\right)}\right]\\
&=&\esp\left[z_{1}^{L_{1}\left(\zeta_{1}\right)}z_{2}^{L_{2}\left(\zeta_{1}\right)}\right]
\esp\left[\hat{\theta}_{1}\left(P_{1}\left(z_{1}\right)\tilde{P}_{2}\left(z_{2}\right)\hat{P}_{2}\left(w_{2}\right)\right)^{\hat{L}_{1}\left(\zeta_{1}\right)}w_{2}^{\hat{L}_{2}\left(\zeta_{1}\right)}\right]\\
&=&F_{1}\left(z_{1},z_{2};\zeta_{1}\right)\hat{F}_{1}\left(\hat{\theta}_{1}\left(P_{1}\left(z_{1}\right)\tilde{P}_{2}\left(z_{2}\right)\hat{P}_{2}\left(w_{2}\right)\right),w_{2}\right)
\end{eqnarray*}


es decir
\begin{eqnarray}
\esp\left[z_{1}^{L_{1}\left(\overline{\zeta}_{1}\right)}z_{2}^{L_{2}\left(\overline{\zeta}_{1}
\right)}w_{1}^{\hat{L}_{1}\left(\overline{\zeta}_{1}\right)}w_{2}^{\hat{L}_{2}\left(
\overline{\zeta}_{1}\right)}\right]&=&\hat{F}_{1}\left(z_{1},z_{2},\hat{\theta}_{1}\left(P_{1}\left(z_{1}\right)\tilde{P}_{2}\left(z_{2}\right)\hat{P}_{2}\left(w_{2}\right)\right),w_{2}\right)\\
&=&F_{1}\left(z_{1},z_{2};\zeta_{1}\right)\hat{F}_{1}\left(\hat{\theta}_{1}\left(P_{1}\left(z_{1}\right)\tilde{P}_{2}\left(z_{2}\right)\hat{P}_{2}\left(w_{2}\right)\right),w_{2}\right).
\end{eqnarray}


Finalmente para $\overline{\zeta}_{2}:$
\begin{eqnarray*}
&&\esp\left[z_{1}^{L_{1}\left(\overline{\zeta}_{2}\right)}z_{2}^{L_{2}\left(\overline{\zeta}_{2}\right)}w_{1}^{\hat{L}_{1}\left(\overline{\zeta}_{2}\right)}w_{2}^{\hat{L}_{2}\left(\overline{\zeta}_{2}\right)}\right]=
\esp\left[z_{1}^{L_{1}\left(\overline{\zeta}_{2}\right)}z_{2}^{L_{2}\left(\overline{\zeta}_{2}\right)}w_{1}^{\hat{L}_{1}\left(\overline{\zeta}_{2}\right)}\right]\\
%&=&\esp\left[z_{1}^{L_{1}\left(\zeta_{2}\right)+X_{1}\left(\overline{\zeta}_{2}-\zeta_{2}\right)}z_{2}^{L_{2}\left(\zeta_{2}\right)+X_{2}\left(\overline{\zeta}_{2}-\zeta_{2}\right)+\hat{Y}_{2}\left(\overline{\zeta}_{2}-\zeta_{2}\right)}w_{1}^{\hat{L}_{1}\left(\zeta_{2}\right)+\hat{X}_{1}\left(\overline{\zeta}_{2}-\zeta_{2}\right)}\right]\\
&=&\esp\left[z_{1}^{L_{1}\left(\zeta_{2}\right)}z_{1}^{X_{1}\left(\overline{\zeta}_{2}-\zeta_{2}\right)}z_{2}^{L_{2}\left(\zeta_{2}\right)}z_{2}^{X_{2}\left(\overline{\zeta}_{2}-\zeta_{2}\right)}
z_{2}^{Y_{2}\left(\overline{\zeta}_{2}-\zeta_{2}\right)}w_{1}^{\hat{L}_{1}\left(\zeta_{2}\right)}w_{1}^{\hat{X}_{1}\left(\overline{\zeta}_{2}-\zeta_{2}\right)}\right]\\
&=&\esp\left[z_{1}^{L_{1}\left(\zeta_{2}\right)}z_{2}^{L_{2}\left(\zeta_{2}\right)}\right]\esp\left[z_{1}^{X_{1}\left(\overline{\zeta}_{2}-\zeta_{2}\right)}z_{2}^{\tilde{X}_{2}\left(\overline{\zeta}_{2}-\zeta_{2}\right)}w_{1}^{\hat{X}_{1}\left(\overline{\zeta}_{2}-\zeta_{2}\right)}w_{1}^{\hat{L}_{1}\left(\zeta_{2}\right)}\right]\\
&=&\esp\left[z_{1}^{L_{1}\left(\zeta_{2}\right)}z_{2}^{L_{2}\left(\zeta_{2}\right)}\right]\esp\left[P_{1}\left(z_{1}\right)^{\overline{\zeta}_{2}-\zeta_{2}}\tilde{P}_{2}\left(z_{2}\right)^{\overline{\zeta}_{2}-\zeta_{2}}\hat{P}\left(w_{1}\right)^{\overline{\zeta}_{2}-\zeta_{2}}w_{1}^{\hat{L}_{1}\left(\zeta_{2}\right)}\right]\\
&=&\esp\left[z_{1}^{L_{1}\left(\zeta_{2}\right)}z_{2}^{L_{2}\left(\zeta_{2}\right)}\right]\esp\left[w_{1}^{\hat{L}_{1}\left(\zeta_{2}\right)}\left\{P_{1}\left(z_{1}\right)\tilde{P}_{2}\left(z_{2}\right)\hat{P}\left(w_{1}\right)\right\}^{\overline{\zeta}_{2}-\zeta_{2}}\right]\\
&=&\esp\left[z_{1}^{L_{1}\left(\zeta_{2}\right)}z_{2}^{L_{2}\left(\zeta_{2}\right)}\right]\esp\left[w_{1}^{\hat{L}_{1}\left(\zeta_{2}\right)}\hat{\theta}_{2}\left(P_{1}\left(z_{1}\right)\tilde{P}_{2}\left(z_{2}\right)\hat{P}\left(w_{1}\right)\right)^{\hat{L}_{2}\zeta_{2}}\right]\\
&=&F_{2}\left(z_{1},z_{2};\zeta_{2}\right)\hat{F}_{2}\left(w_{1},\hat{\theta}_{2}\left(P_{1}\left(z_{1}\right)\tilde{P}_{2}\left(z_{2}\right)\hat{P}_{1}\left(w_{1}\right)\right)\right]\\
%&\equiv&\hat{F}_{2}\left(z_{1},z_{2},w_{1},\hat{\theta}_{2}\left(P_{1}\left(z_{1}\right)\tilde{P}_{2}\left(z_{2}\right)\hat{P}_{1}\left(w_{1}\right)\right)\right)
\end{eqnarray*}

es decir
\begin{eqnarray}
\esp\left[z_{1}^{L_{1}\left(\overline{\zeta}_{2}\right)}z_{2}^{L_{2}\left(\overline{\zeta}_{2}\right)}w_{1}^{\hat{L}_{1}\left(\overline{\zeta}_{2}\right)}w_{2}^{\hat{L}_{2}\left(\overline{\zeta}_{2}\right)}\right]=\hat{F}_{2}\left(z_{1},z_{2},w_{1},\hat{\theta}_{2}\left(P_{1}\left(z_{1}\right)\tilde{P}_{2}\left(z_{2}\right)\hat{P}_{1}\left(w_{1}\right)\right)\right)\\
=F_{2}\left(z_{1},z_{2};\zeta_{2}\right)\hat{F}_{2}\left(w_{1},\hat{\theta}_{2}\left(P_{1}\left(z_{1}\right)\tilde{P}_{2}\left(z_{2}\right)\hat{P}_{1}\left(w_{1}
\right)\right)\right)
\end{eqnarray}
%__________________________________________________________________________
\section{Ecuaciones Recursivas para la R.S.V.C.}
%__________________________________________________________________________




Con lo desarrollado hasta ahora podemos encontrar las ecuaciones
recursivas que modelan la Red de Sistemas de Visitas C\'iclicas
(R.S.V.C):
\begin{eqnarray*}
&&F_{2}\left(z_{1},z_{2},w_{1},w_{2}\right)=R_{1}\left(z_{1},z_{2},w_{1},w_{2}\right)\esp\left[z_{1}^{L_{1}\left(\overline{\tau}_{1}\right)}z_{2}^{L_{2}\left(\overline{\tau}_{1}\right)}w_{1}^{\hat{L}_{1}\left(\overline{\tau}_{1}\right)}w_{2}^{\hat{L}_{2}\left(\overline{\tau}_{1}\right)}\right]\\
&&F_{1}\left(z_{1},z_{2},w_{1},w_{2}\right)=R_{2}\left(z_{1},z_{2},w_{1},w_{2}\right)\esp\left[z_{1}^{L_{1}\left(\overline{\tau}_{2}\right)}z_{2}^{L_{2}\left(\overline{\tau}_{2}\right)}w_{1}^{\hat{L}_{1}\left(\overline{\tau}_{2}\right)}w_{2}^{\hat{L}_{2}\left(\overline{\tau}_{1}\right)}\right]\\
&&\hat{F}_{2}\left(z_{1},z_{2},w_{1},w_{2}\right)=\hat{R}_{1}\left(z_{1},z_{2},w_{1},w_{2}\right)\esp\left[z_{1}^{L_{1}\left(\overline{\zeta}_{1}\right)}z_{2}^{L_{2}\left(\overline{\zeta}_{1}\right)}w_{1}^{\hat{L}_{1}\left(\overline{\zeta}_{1}\right)}w_{2}^{\hat{L}_{2}\left(\overline{\zeta}_{1}\right)}\right]\\
&&\hat{F}_{1}\left(z_{1},z_{2},w_{1},w_{2}\right)=\hat{R}_{2}\left(z_{1},z_{2},
w_{1},w_{2}\right)\esp\left[z_{1}^{L_{1}\left(\overline{\zeta}_{2}\right)}z_{2}
^{L_{2}\left(\overline{\zeta}_{2}\right)}w_{1}^{\hat{L}_{1}\left(
\overline{\zeta}_{2}\right)}w_{2}^{\hat{L}_{2}\left(\overline{\zeta}_{2}\right)}
\right]
\end{eqnarray*}

%&=&R_{1}\left(P_{1}\left(z_{1}\right)\tilde{P}_{2}\left(z_{2}\right)\hat{P}_{1}\left(w_{1}\right)\hat{P}_{2}\left(w_{2}\right)\right)
%F_{1}\left(\theta\left(\tilde{P}_{2}\left(z_{2}\right)\hat{P}_{1}\left(w_{1}\right)\hat{P}_{2}\left(w_{2}\right)\right),z_{2},w_{1},w_{2}\right)\\
%&=&R_{2}\left(P_{1}\left(z_{1}\right)\tilde{P}_{2}\left(z_{2}\right)\hat{P}_{1}\left(w_{1}\right)\hat{P}_{2}\left(w_{2}\right)\right)F_{2}\left(z_{1},\tilde{\theta}_{2}\left(P_{1}\left(z_{1}\right)\hat{P}_{1}\left(w_{1}\right)\hat{P}_{2}\left(w_{2}\right)\right),w_{1},w_{2}\right)\\
%&=&\hat{R}_{1}\left(P_{1}\left(z_{1}\right)\tilde{P}_{2}\left(z_{2}\right)\hat{P}_{1}\left(w_{1}\right)\hat{P}_{2}\left(w_{2}\right)\right)\hat{F}_{1}\left(z_{1},z_{2},\hat{\theta}_{1}\left(P_{1}\left(z_{1}\right)\tilde{P}_{2}\left(z_{2}\right)\hat{P}_{2}\left(w_{2}\right)\right),w_{2}\right)
%&=&\hat{R}_{2}\left(P_{1}\left(z_{1}\right)\tilde{P}_{2}\left(z_{2}\right)\hat{P}_{1}\left(w_{1}\right)\hat{P}_{2}\left(w_{2}\right)\right)\hat{F}_{2}\left(z_{1},z_{2},w_{1},\hat{\theta}_{2}\left(P_{1}\left(z_{1}\right)\tilde{P}_{2}\left(z_{2}\right)\hat{P}_{1}\left(w_{1}\right)\right)\right)


que son equivalentes a las siguientes ecuaciones
\begin{eqnarray}
F_{2}\left(z_{1},z_{2},w_{1},w_{2}\right)&=&R_{1}\left(P_{1}\left(z_{1}\right)\tilde{P}_{2}\left(z_{2}\right)\prod_{i=1}^{2}
\hat{P}_{i}\left(w_{i}\right)\right)F_{1}\left(\theta_{1}\left(\tilde{P}_{2}\left(z_{2}\right)\hat{P}_{1}\left(w_{1}\right)\hat{P}_{2}\left(w_{2}\right)\right),z_{2},w_{1},w_{2}\right)\\
F_{1}\left(z_{1},z_{2},w_{1},w_{2}\right)&=&R_{2}\left(P_{1}\left(z_{1}\right)\tilde{P}_{2}\left(z_{2}\right)\prod_{i=1}^{2}
\hat{P}_{i}\left(w_{i}\right)\right)F_{2}\left(z_{1},\tilde{\theta}_{2}\left(P_{1}\left(z_{1}\right)\hat{P}_{1}\left(w_{1}\right)\hat{P}_{2}\left(w_{2}\right)\right),w_{1},w_{2}\right)\\
\hat{F}_{2}\left(z_{1},z_{2},w_{1},w_{2}\right)&=&\hat{R}_{1}\left(P_{1}\left(z_{1}\right)\tilde{P}_{2}\left(z_{2}\right)\prod_{i=1}^{2}
\hat{P}_{i}\left(w_{i}\right)\right)\hat{F}_{1}\left(z_{1},z_{2},\hat{\theta}_{1}\left(P_{1}\left(z_{1}\right)\tilde{P}_{2}\left(z_{2}\right)\hat{P}_{2}\left(w_{2}\right)\right),w_{2}\right)\\
\hat{F}_{1}\left(z_{1},z_{2},w_{1},w_{2}\right)&=&\hat{R}_{2}\left(P_{1}\left(z_{1}\right)\tilde{P}_{2}\left(z_{2}\right)\prod_{i=1}^{2}
\hat{P}_{i}\left(w_{i}\right)\right)\hat{F}_{2}\left(z_{1},z_{2},w_{1},\hat{\theta}_{2}\left(P_{1}\left(z_{1}\right)\tilde{P}_{2}\left(z_{2}\right)
\hat{P}_{1}\left(w_{1}\right)\right)\right)
\end{eqnarray}



%_________________________________________________________________________________________________
\subsection{Tiempos de Traslado del Servidor}
%_________________________________________________________________________________________________


Para
%\begin{multicols}{2}

\begin{eqnarray}\label{Ec.R1}
R_{1}\left(\mathbf{z,w}\right)=R_{1}\left((P_{1}\left(z_{1}\right)\tilde{P}_{2}\left(z_{2}\right)\hat{P}_{1}\left(w_{1}\right)\hat{P}_{2}\left(w_{2}\right)\right)
\end{eqnarray}
%\end{multicols}

se tiene que


\begin{eqnarray*}
\begin{array}{cc}
\frac{\partial R_{1}\left(\mathbf{z,w}\right)}{\partial
z_{1}}|_{\mathbf{z,w}=1}=R_{1}^{(1)}\left(1\right)P_{1}^{(1)}\left(1\right)=r_{1}\mu_{1},&
\frac{\partial R_{1}\left(\mathbf{z,w}\right)}{\partial
z_{2}}|_{\mathbf{z,w}=1}=R_{1}^{(1)}\left(1\right)\tilde{P}_{2}^{(1)}\left(1\right)=r_{1}\tilde{\mu}_{2},\\
\frac{\partial R_{1}\left(\mathbf{z,w}\right)}{\partial
w_{1}}|_{\mathbf{z,w}=1}=R_{1}^{(1)}\left(1\right)\hat{P}_{1}^{(1)}\left(1\right)=r_{1}\hat{\mu}_{1},&
\frac{\partial R_{1}\left(\mathbf{z,w}\right)}{\partial
w_{2}}|_{\mathbf{z,w}=1}=R_{1}^{(1)}\left(1\right)\hat{P}_{2}^{(1)}\left(1\right)=r_{1}\hat{\mu}_{2},
\end{array}
\end{eqnarray*}

An\'alogamente se tiene

\begin{eqnarray}
R_{2}\left(\mathbf{z,w}\right)=R_{2}\left(P_{1}\left(z_{1}\right)\tilde{P}_{2}\left(z_{2}\right)\hat{P}_{1}\left(w_{1}\right)\hat{P}_{2}\left(w_{2}\right)\right)
\end{eqnarray}


\begin{eqnarray*}
\begin{array}{cc}
\frac{\partial R_{2}\left(\mathbf{z,w}\right)}{\partial
z_{1}}|_{\mathbf{z,w}=1}=R_{2}^{(1)}\left(1\right)P_{1}^{(1)}\left(1\right)=r_{2}\mu_{1},&
\frac{\partial R_{2}\left(\mathbf{z,w}\right)}{\partial
z_{2}}|_{\mathbf{z,w}=1}=R_{2}^{(1)}\left(1\right)\tilde{P}_{2}^{(1)}\left(1\right)=r_{2}\tilde{\mu}_{2},\\
\frac{\partial R_{2}\left(\mathbf{z,w}\right)}{\partial
w_{1}}|_{\mathbf{z,w}=1}=R_{2}^{(1)}\left(1\right)\hat{P}_{1}^{(1)}\left(1\right)=r_{2}\hat{\mu}_{1},&
\frac{\partial R_{2}\left(\mathbf{z,w}\right)}{\partial
w_{2}}|_{\mathbf{z,w}=1}=R_{2}^{(1)}\left(1\right)\hat{P}_{2}^{(1)}\left(1\right)=r_{2}\hat{\mu}_{2},\\
\end{array}
\end{eqnarray*}

Para el segundo sistema:

\begin{eqnarray}
\hat{R}_{1}\left(\mathbf{z,w}\right)=\hat{R}_{1}\left(P_{1}\left(z_{1}\right)\tilde{P}_{2}\left(z_{2}\right)\hat{P}_{1}\left(w_{1}\right)\hat{P}_{2}\left(w_{2}\right)\right)
\end{eqnarray}


\begin{eqnarray*}
\frac{\partial \hat{R}_{1}\left(\mathbf{z,w}\right)}{\partial
z_{1}}|_{\mathbf{z,w}=1}=\hat{R}_{1}^{(1)}\left(1\right)P_{1}^{(1)}\left(1\right)=\hat{r}_{1}\mu_{1},&
\frac{\partial \hat{R}_{1}\left(\mathbf{z,w}\right)}{\partial
z_{2}}|_{\mathbf{z,w}=1}=\hat{R}_{1}^{(1)}\left(1\right)\tilde{P}_{2}^{(1)}\left(1\right)=\hat{r}_{1}\tilde{\mu}_{2},\\
\frac{\partial \hat{R}_{1}\left(\mathbf{z,w}\right)}{\partial
w_{1}}|_{\mathbf{z,w}=1}=\hat{R}_{1}^{(1)}\left(1\right)\hat{P}_{1}^{(1)}\left(1\right)=\hat{r}_{1}\hat{\mu}_{1},&
\frac{\partial \hat{R}_{1}\left(\mathbf{z,w}\right)}{\partial
w_{2}}|_{\mathbf{z,w}=1}=\hat{R}_{1}^{(1)}\left(1\right)\hat{P}_{2}^{(1)}\left(1\right)=\hat{r}_{1}\hat{\mu}_{2},
\end{eqnarray*}

Finalmente

\begin{eqnarray}
\hat{R}_{2}\left(\mathbf{z,w}\right)=\hat{R}_{2}\left(P_{1}\left(z_{1}\right)\tilde{P}_{2}\left(z_{2}\right)\hat{P}_{1}\left(w_{1}\right)\hat{P}_{2}\left(w_{2}\right)\right)
\end{eqnarray}



\begin{eqnarray*}
\frac{\partial \hat{R}_{2}\left(\mathbf{z,w}\right)}{\partial
z_{1}}|_{\mathbf{z,w}=1}=\hat{R}_{2}^{(1)}\left(1\right)P_{1}^{(1)}\left(1\right)=\hat{r}_{2}\mu_{1},&
\frac{\partial \hat{R}_{2}\left(\mathbf{z,w}\right)}{\partial
z_{2}}|_{\mathbf{z,w}=1}=\hat{R}_{2}^{(1)}\left(1\right)\tilde{P}_{2}^{(1)}\left(1\right)=\hat{r}_{2}\tilde{\mu}_{2},\\
\frac{\partial \hat{R}_{2}\left(\mathbf{z,w}\right)}{\partial
w_{1}}|_{\mathbf{z,w}=1}=\hat{R}_{2}^{(1)}\left(1\right)\hat{P}_{1}^{(1)}\left(1\right)=\hat{r}_{2}\hat{\mu}_{1},&
\frac{\partial \hat{R}_{2}\left(\mathbf{z,w}\right)}{\partial
w_{2}}|_{\mathbf{z,w}=1}=\hat{R}_{2}^{(1)}\left(1\right)\hat{P}_{2}^{(1)}\left(1\right)
=\hat{r}_{2}\hat{\mu}_{2}.
\end{eqnarray*}


%_________________________________________________________________________________________________
\subsection{Usuarios presentes en la cola}
%_________________________________________________________________________________________________

Hagamos lo correspondiente con las siguientes
expresiones obtenidas en la secci\'on anterior:
Recordemos que

\begin{eqnarray*}
F_{1}\left(\theta_{1}\left(\tilde{P}_{2}\left(z_{2}\right)\hat{P}_{1}\left(w_{1}\right)
\hat{P}_{2}\left(w_{2}\right)\right),z_{2},w_{1},w_{2}\right)=
F_{1}\left(\theta_{1}\left(\tilde{P}_{2}\left(z_{2}\right)\hat{P}_{1}\left(w_{1}
\right)\hat{P}_{2}\left(w_{2}\right)\right),z_{2}\right)
\hat{F}_{1}\left(w_{1},w_{2};\tau_{1}\right)
\end{eqnarray*}

entonces

\begin{eqnarray*}
\frac{\partial F_{1}\left(\theta_{1}\left(\tilde{P}_{2}\left(z_{2}\right)\hat{P}_{1}\left(w_{1}\right)\hat{P}_{2}\left(w_{2}\right)\right),z_{2},w_{1},w_{2}\right)}{\partial z_{1}}|_{\mathbf{z},\mathbf{w}=1}&=&0\\
\frac{\partial
F_{1}\left(\theta_{1}\left(\tilde{P}_{2}\left(z_{2}\right)\hat{P}_{1}\left(w_{1}\right)\hat{P}_{2}\left(w_{2}\right)\right),z_{2},w_{1},w_{2}\right)}{\partial
z_{2}}|_{\mathbf{z},\mathbf{w}=1}&=&\frac{\partial F_{1}}{\partial
z_{1}}\cdot\frac{\partial \theta_{1}}{\partial
\tilde{P}_{2}}\cdot\frac{\partial \tilde{P}_{2}}{\partial
z_{2}}+\frac{\partial F_{1}}{\partial z_{2}}
\\
\frac{\partial
F_{1}\left(\theta_{1}\left(\tilde{P}_{2}\left(z_{2}\right)\hat{P}_{1}\left(w_{1}\right)\hat{P}_{2}\left(w_{2}\right)\right),z_{2},w_{1},w_{2}\right)}{\partial
w_{1}}|_{\mathbf{z},\mathbf{w}=1}&=&\frac{\partial F_{1}}{\partial
z_{1}}\cdot\frac{\partial
\theta_{1}}{\partial\hat{P}_{1}}\cdot\frac{\partial\hat{P}_{1}}{\partial
w_{1}}+\frac{\partial\hat{F}_{1}}{\partial w_{1}}
\\
\frac{\partial
F_{1}\left(\theta_{1}\left(\tilde{P}_{2}\left(z_{2}\right)\hat{P}_{1}\left(w_{1}\right)\hat{P}_{2}\left(w_{2}\right)\right),z_{2},w_{1},w_{2}\right)}{\partial
w_{2}}|_{\mathbf{z},\mathbf{w}=1}&=&\frac{\partial F_{1}}{\partial
z_{1}}\cdot\frac{\partial\theta_{1}}{\partial\hat{P}_{2}}\cdot\frac{\partial\hat{P}_{2}}{\partial
w_{2}}+\frac{\partial \hat{F}_{1}}{\partial w_{2}}
\\
\end{eqnarray*}

para $\tau_{2}$:

\begin{eqnarray*}
F_{2}\left(z_{1},\tilde{\theta}_{2}\left(P_{1}\left(z_{1}\right)\hat{P}_{1}\left(w_{1}\right)\hat{P}_{2}\left(w_{2}\right)\right),
w_{1},w_{2}\right)=F_{2}\left(z_{1},\tilde{\theta}_{2}\left(P_{1}\left(z_{1}\right)\hat{P}_{1}\left(w_{1}\right)
\hat{P}_{2}\left(w_{2}\right)\right)\right)\hat{F}_{2}\left(w_{1},w_{2};\tau_{2}\right)
\end{eqnarray*}
al igual que antes

\begin{eqnarray*}
\frac{\partial
F_{2}\left(z_{1},\tilde{\theta}_{2}\left(P_{1}\left(z_{1}\right)\hat{P}_{1}\left(w_{1}\right)\hat{P}_{2}\left(w_{2}\right)\right),w_{1},w_{2}\right)}{\partial
z_{1}}|_{\mathbf{z},\mathbf{w}=1}&=&\frac{\partial F_{2}}{\partial
z_{2}}\cdot\frac{\partial\tilde{\theta}_{2}}{\partial
P_{1}}\cdot\frac{\partial P_{1}}{\partial z_{2}}+\frac{\partial
F_{2}}{\partial z_{1}}
\\
\frac{\partial F_{2}\left(z_{1},\tilde{\theta}_{2}\left(P_{1}\left(z_{1}\right)\hat{P}_{1}\left(w_{1}\right)\hat{P}_{2}\left(w_{2}\right)\right),w_{1},w_{2}\right)}{\partial z_{2}}|_{\mathbf{z},\mathbf{w}=1}&=&0\\
\frac{\partial
F_{2}\left(z_{1},\tilde{\theta}_{2}\left(P_{1}\left(z_{1}\right)\hat{P}_{1}\left(w_{1}\right)\hat{P}_{2}\left(w_{2}\right)\right),w_{1},w_{2}\right)}{\partial
w_{1}}|_{\mathbf{z},\mathbf{w}=1}&=&\frac{\partial F_{2}}{\partial
z_{2}}\cdot\frac{\partial \tilde{\theta}_{2}}{\partial
\hat{P}_{1}}\cdot\frac{\partial \hat{P}_{1}}{\partial
w_{1}}+\frac{\partial \hat{F}_{2}}{\partial w_{1}}
\\
\frac{\partial
F_{2}\left(z_{1},\tilde{\theta}_{2}\left(P_{1}\left(z_{1}\right)\hat{P}_{1}\left(w_{1}\right)\hat{P}_{2}\left(w_{2}\right)\right),w_{1},w_{2}\right)}{\partial
w_{2}}|_{\mathbf{z},\mathbf{w}=1}&=&\frac{\partial F_{2}}{\partial
z_{2}}\cdot\frac{\partial
\tilde{\theta}_{2}}{\partial\hat{P}_{2}}\cdot\frac{\partial\hat{P}_{2}}{\partial
w_{2}}+\frac{\partial\hat{F}_{2}}{\partial w_{2}}
\\
\end{eqnarray*}


Ahora para el segundo sistema

\begin{eqnarray*}\hat{F}_{1}\left(z_{1},z_{2},\hat{\theta}_{1}\left(P_{1}\left(z_{1}\right)\tilde{P}_{2}\left(z_{2}\right)\hat{P}_{2}\left(w_{2}\right)\right),
w_{2}\right)=F_{1}\left(z_{1},z_{2};\zeta_{1}\right)\hat{F}_{1}\left(\hat{\theta}_{1}\left(P_{1}\left(z_{1}\right)\tilde{P}_{2}\left(z_{2}\right)
\hat{P}_{2}\left(w_{2}\right)\right),w_{2}\right)
\end{eqnarray*}
entonces


\begin{eqnarray*}
\frac{\partial
\hat{F}_{1}\left(z_{1},z_{2},\hat{\theta}_{1}\left(P_{1}\left(z_{1}\right)\tilde{P}_{2}\left(z_{2}\right)\hat{P}_{2}\left(w_{2}\right)\right),w_{2}\right)}{\partial
z_{1}}|_{\mathbf{z},\mathbf{w}=1}&=&\frac{\partial \hat{F}_{1}
}{\partial w_{1}}\cdot\frac{\partial\hat{\theta}_{1}}{\partial
P_{1}}\cdot\frac{\partial P_{1}}{\partial z_{1}}+\frac{\partial
F_{1}}{\partial z_{1}}
\\
\frac{\partial
\hat{F}_{1}\left(z_{1},z_{2},\hat{\theta}_{1}\left(P_{1}\left(z_{1}\right)\tilde{P}_{2}\left(z_{2}\right)\hat{P}_{2}\left(w_{2}\right)\right),w_{2}\right)}{\partial
z_{2}}|_{\mathbf{z},\mathbf{w}=1}&=&\frac{\partial
\hat{F}_{1}}{\partial
w_{1}}\cdot\frac{\partial\hat{\theta}_{1}}{\partial\tilde{P}_{2}}\cdot\frac{\partial\tilde{P}_{2}}{\partial
z_{2}}+\frac{\partial F_{1}}{\partial z_{2}}
\\
\frac{\partial \hat{F}_{1}\left(z_{1},z_{2},\hat{\theta}_{1}\left(P_{1}\left(z_{1}\right)\tilde{P}_{2}\left(z_{2}\right)\hat{P}_{2}\left(w_{2}\right)\right),w_{2}\right)}{\partial w_{1}}|_{\mathbf{z},\mathbf{w}=1}&=&0\\
\frac{\partial \hat{F}_{1}\left(z_{1},z_{2},\hat{\theta}_{1}\left(P_{1}\left(z_{1}\right)\tilde{P}_{2}\left(z_{2}\right)\hat{P}_{2}\left(w_{2}\right)\right),w_{2}\right)}{\partial w_{2}}|_{\mathbf{z},\mathbf{w}=1}&=&\frac{\partial\hat{F}_{1}}{\partial w_{1}}\cdot\frac{\partial\hat{\theta}_{1}}{\partial\hat{P}_{2}}\cdot\frac{\partial\hat{P}_{2}}{\partial w_{2}}+\frac{\partial \hat{F}_{1}}{\partial w_{2}}\\
\end{eqnarray*}



Finalmente para $\zeta_{2}$

\begin{eqnarray*}\hat{F}_{2}\left(z_{1},z_{2},w_{1},\hat{\theta}_{2}\left(P_{1}\left(z_{1}\right)\tilde{P}_{2}\left(z_{2}\right)\hat{P}_{1}\left(w_{1}\right)\right)\right)&=&F_{2}\left(z_{1},z_{2};\zeta_{2}\right)\hat{F}_{2}\left(w_{1},\hat{\theta}_{2}\left(P_{1}\left(z_{1}\right)\tilde{P}_{2}\left(z_{2}\right)\hat{P}_{1}\left(w_{1}\right)\right)\right]
\end{eqnarray*}
por tanto:

\begin{eqnarray*}
\frac{\partial
\hat{F}_{2}\left(z_{1},z_{2},w_{1},\hat{\theta}_{2}\left(P_{1}\left(z_{1}\right)\tilde{P}_{2}\left(z_{2}\right)\hat{P}_{1}\left(w_{1}\right)\right)\right)}{\partial
z_{1}}|_{\mathbf{z},\mathbf{w}=1}&=&\frac{\partial\hat{F}_{2}}{\partial
w_{2}}\cdot\frac{\partial\hat{\theta}_{2}}{\partial
P_{1}}\cdot\frac{\partial P_{1}}{\partial z_{1}}+\frac{\partial
F_{2}}{\partial z_{1}}
\\
\frac{\partial \hat{F}_{2}\left(z_{1},z_{2},w_{1},\hat{\theta}_{2}\left(P_{1}\left(z_{1}\right)\tilde{P}_{2}\left(z_{2}\right)\hat{P}_{1}\left(w_{1}\right)\right)\right)}{\partial z_{2}}|_{\mathbf{z},\mathbf{w}=1}&=&\frac{\partial\hat{F}_{2}}{\partial w_{2}}\cdot\frac{\partial\hat{\theta}_{2}}{\partial \tilde{P}_{2}}\cdot\frac{\partial \tilde{P}_{2}}{\partial z_{2}}+\frac{\partial F_{2}}{\partial z_{2}}\\
\frac{\partial \hat{F}_{2}\left(z_{1},z_{2},w_{1},\hat{\theta}_{2}\left(P_{1}\left(z_{1}\right)\tilde{P}_{2}\left(z_{2}\right)\hat{P}_{1}\left(w_{1}\right)\right)\right)}{\partial w_{1}}|_{\mathbf{z},\mathbf{w}=1}&=&\frac{\partial\hat{F}_{2}}{\partial w_{2}}\cdot\frac{\partial\hat{\theta}_{2}}{\partial \hat{P}_{1}}\cdot\frac{\partial \hat{P}_{1}}{\partial w_{1}}+\frac{\partial \hat{F}_{2}}{\partial w_{1}}\\
\frac{\partial \hat{F}_{2}\left(z_{1},z_{2},w_{1},\hat{\theta}_{2}\left(P_{1}\left(z_{1}\right)\tilde{P}_{2}\left(z_{2}\right)\hat{P}_{1}\left(w_{1}\right)\right)\right)}{\partial w_{2}}|_{\mathbf{z},\mathbf{w}=1}&=&0\\
\end{eqnarray*}

%_________________________________________________________________________________________________
\subsection{Ecuaciones Recursivas}
%_________________________________________________________________________________________________

Entonces, de todo lo desarrollado hasta ahora se tienen las siguientes ecuaciones:

\begin{eqnarray*}
\frac{\partial F_{2}\left(\mathbf{z},\mathbf{w}\right)}{\partial z_{1}}|_{\mathbf{z},\mathbf{w}=1}&=&r_{1}\mu_{1}\\
\frac{\partial F_{2}\left(\mathbf{z},\mathbf{w}\right)}{\partial z_{2}}|_{\mathbf{z},\mathbf{w}=1}&=&=r_{1}\tilde{\mu}_{2}+f_{1}\left(1\right)\left(\frac{1}{1-\mu_{1}}\right)\tilde{\mu}_{2}+f_{1}\left(2\right)\\
\frac{\partial F_{2}\left(\mathbf{z},\mathbf{w}\right)}{\partial w_{1}}|_{\mathbf{z},\mathbf{w}=1}&=&r_{1}\hat{\mu}_{1}+f_{1}\left(1\right)\left(\frac{1}{1-\mu_{1}}\right)\hat{\mu}_{1}+\hat{F}_{1,1}^{(1)}\left(1\right)\\
\frac{\partial F_{2}\left(\mathbf{z},\mathbf{w}\right)}{\partial
w_{2}}|_{\mathbf{z},\mathbf{w}=1}&=&r_{1}\hat{\mu}_{2}+f_{1}\left(1\right)\left(\frac{1}{1-\mu_{1}}\right)\hat{\mu}_{2}+\hat{F}_{2,1}^{(1)}\left(1\right)\\
\frac{\partial F_{1}\left(\mathbf{z},\mathbf{w}\right)}{\partial z_{1}}|_{\mathbf{z},\mathbf{w}=1}&=&r_{2}\mu_{1}+f_{2}\left(2\right)\left(\frac{1}{1-\tilde{\mu}_{2}}\right)\mu_{1}+f_{2}\left(1\right)\\
\frac{\partial F_{1}\left(\mathbf{z},\mathbf{w}\right)}{\partial z_{2}}|_{\mathbf{z},\mathbf{w}=1}&=&r_{2}\tilde{\mu}_{2}\\
\frac{\partial F_{1}\left(\mathbf{z},\mathbf{w}\right)}{\partial w_{1}}|_{\mathbf{z},\mathbf{w}=1}&=&r_{2}\hat{\mu}_{1}+f_{2}\left(2\right)\left(\frac{1}{1-\tilde{\mu}_{2}}\right)\hat{\mu}_{1}+\hat{F}_{2,1}^{(1)}\left(1\right)\\
\frac{\partial F_{1}\left(\mathbf{z},\mathbf{w}\right)}{\partial
w_{2}}|_{\mathbf{z},\mathbf{w}=1}&=&r_{2}\hat{\mu}_{2}+f_{2}\left(2\right)\left(\frac{1}{1-\tilde{\mu}_{2}}\right)\hat{\mu}_{2}+\hat{F}_{2,2}^{(1)}\left(1\right)\\
\frac{\partial \hat{F}_{2}\left(\mathbf{z},\mathbf{w}\right)}{\partial z_{1}}|_{\mathbf{z},\mathbf{w}=1}&=&\hat{r}_{1}\mu_{1}+\hat{f}_{1}\left(1\right)\left(\frac{1}{1-\hat{\mu}_{1}}\right)\mu_{1}+F_{1,1}^{(1)}\left(1\right)\\
\frac{\partial \hat{F}_{2}\left(\mathbf{z},\mathbf{w}\right)}{\partial z_{2}}|_{\mathbf{z},\mathbf{w}=1}&=&\hat{r}_{1}\mu_{2}+\hat{f}_{1}\left(1\right)\left(\frac{1}{1-\hat{\mu}_{1}}\right)\tilde{\mu}_{2}+F_{2,1}^{(1)}\left(1\right)\\
\frac{\partial \hat{F}_{2}\left(\mathbf{z},\mathbf{w}\right)}{\partial w_{1}}|_{\mathbf{z},\mathbf{w}=1}&=&\hat{r}_{1}\hat{\mu}_{1}\\
\frac{\partial \hat{F}_{2}\left(\mathbf{z},\mathbf{w}\right)}{\partial w_{2}}|_{\mathbf{z},\mathbf{w}=1}&=&\hat{r}_{1}\hat{\mu}_{2}+\hat{f}_{1}\left(1\right)\left(\frac{1}{1-\hat{\mu}_{1}}\right)\hat{\mu}_{2}+\hat{f}_{1}\left(2\right)\\
\frac{\partial \hat{F}_{1}\left(\mathbf{z},\mathbf{w}\right)}{\partial z_{1}}|_{\mathbf{z},\mathbf{w}=1}&=&\hat{r}_{2}\mu_{1}+\hat{f}_{2}\left(1\right)\left(\frac{1}{1-\hat{\mu}_{2}}\right)\mu_{1}+F_{1,2}^{(1)}\left(1\right)\\
\frac{\partial \hat{F}_{1}\left(\mathbf{z},\mathbf{w}\right)}{\partial z_{2}}|_{\mathbf{z},\mathbf{w}=1}&=&\hat{r}_{2}\tilde{\mu}_{2}+\hat{f}_{2}\left(2\right)\left(\frac{1}{1-\hat{\mu}_{2}}\right)\tilde{\mu}_{2}+F_{2,2}^{(1)}\left(1\right)\\
\frac{\partial \hat{F}_{1}\left(\mathbf{z},\mathbf{w}\right)}{\partial w_{1}}|_{\mathbf{z},\mathbf{w}=1}&=&\hat{r}_{2}\hat{\mu}_{1}+\hat{f}_{2}\left(2\right)\left(\frac{1}{1-\hat{\mu}_{2}}\right)\hat{\mu}_{1}+\hat{f}_{2}\left(1\right)\\
\frac{\partial
\hat{F}_{1}\left(\mathbf{z},\mathbf{w}\right)}{\partial
w_{2}}|_{\mathbf{z},\mathbf{w}=1}&=&\hat{r}_{2}\hat{\mu}_{2}
\end{eqnarray*}

Es decir, se tienen las siguientes ecuaciones:




\begin{eqnarray*}
f_{2}\left(1\right)&=&r_{1}\mu_{1}\\
f_{1}\left(2\right)&=&r_{2}\tilde{\mu}_{2}\\
f_{2}\left(2\right)&=&r_{1}\tilde{\mu}_{2}+\tilde{\mu}_{2}\left(\frac{f_{1}\left(1\right)}{1-\mu_{1}}\right)+f_{1}\left(2\right)=\left(r_{1}+\frac{f_{1}\left(1\right)}{1-\mu_{1}}\right)\tilde{\mu}_{2}+r_{2}\tilde{\mu}_{2}\\
&=&\left(r_{1}+r_{2}+\frac{f_{1}\left(1\right)}{1-\mu_{1}}\right)\tilde{\mu}_{2}=\left(r+\frac{f_{1}\left(1\right)}{1-\mu_{1}}\right)\tilde{\mu}_{2}\\
f_{2}\left(3\right)&=&r_{1}\hat{\mu}_{1}+\hat{\mu}_{1}\left(\frac{f_{1}\left(1\right)}{1-\mu_{1}}\right)+\hat{F}_{1,1}^{(1)}\left(1\right)=\hat{\mu}_{1}\left(r_{1}+\frac{f_{1}\left(1\right)}{1-\mu_{1}}\right)+\frac{\hat{\mu}_{1}}{\mu_{1}}\\
f_{2}\left(4\right)&=&r_{1}\hat{\mu}_{2}+\hat{\mu}_{2}\left(\frac{f_{1}\left(1\right)}{1-\mu_{1}}\right)+\hat{F}_{2,1}^{(1)}\left(1\right)=\hat{\mu}_{2}\left(r_{1}+\frac{f_{1}\left(1\right)}{1-\mu_{1}}\right)+\frac{\hat{\mu}_{2}}{\mu_{1}}\\
f_{1}\left(1\right)&=&r_{2}\mu_{1}+\mu_{1}\left(\frac{f_{2}\left(2\right)}{1-\tilde{\mu}_{2}}\right)+r_{1}\mu_{1}=\mu_{1}\left(r_{1}+r_{2}+\frac{f_{2}\left(2\right)}{1-\tilde{\mu}_{2}}\right)\\
&=&\mu_{1}\left(r+\frac{f_{2}\left(2\right)}{1-\tilde{\mu}_{2}}\right)\\
f_{1}\left(3\right)&=&r_{2}\hat{\mu}_{1}+\hat{\mu}_{1}\left(\frac{f_{2}\left(2\right)}{1-\tilde{\mu}_{2}}\right)+\hat{F}^{(1)}_{1,2}\left(1\right)=\hat{\mu}_{1}\left(r_{2}+\frac{f_{2}\left(2\right)}{1-\tilde{\mu}_{2}}\right)+\frac{\hat{\mu}_{1}}{\mu_{2}}\\
f_{1}\left(4\right)&=&r_{2}\hat{\mu}_{2}+\hat{\mu}_{2}\left(\frac{f_{2}\left(2\right)}{1-\tilde{\mu}_{2}}\right)+\hat{F}_{2,2}^{(1)}\left(1\right)=\hat{\mu}_{2}\left(r_{2}+\frac{f_{2}\left(2\right)}{1-\tilde{\mu}_{2}}\right)+\frac{\hat{\mu}_{2}}{\mu_{2}}\\
\hat{f}_{1}\left(4\right)&=&\hat{r}_{2}\hat{\mu}_{2}\\
\hat{f}_{2}\left(3\right)&=&\hat{r}_{1}\hat{\mu}_{1}\\
\hat{f}_{1}\left(1\right)&=&\hat{r}_{2}\mu_{1}+\mu_{1}\left(\frac{\hat{f}_{2}\left(4\right)}{1-\hat{\mu}_{2}}\right)+F_{1,2}^{(1)}\left(1\right)=\left(\hat{r}_{2}+\frac{\hat{f}_{2}\left(4\right)}{1-\hat{\mu}_{2}}\right)\mu_{1}+\frac{\mu_{1}}{\hat{\mu}_{2}}\\
\hat{f}_{1}\left(2\right)&=&\hat{r}_{2}\tilde{\mu}_{2}+\tilde{\mu}_{2}\left(\frac{\hat{f}_{2}\left(4\right)}{1-\hat{\mu}_{2}}\right)+F_{2,2}^{(1)}\left(1\right)=
\left(\hat{r}_{2}+\frac{\hat{f}_{2}\left(4\right)}{1-\hat{\mu}_{2}}\right)\tilde{\mu}_{2}+\frac{\mu_{2}}{\hat{\mu}_{2}}\\
\hat{f}_{1}\left(3\right)&=&\hat{r}_{2}\hat{\mu}_{1}+\hat{\mu}_{1}\left(\frac{\hat{f}_{2}\left(4\right)}{1-\hat{\mu}_{2}}\right)+\hat{f}_{2}\left(3\right)=\left(\hat{r}_{2}+\frac{\hat{f}_{2}\left(4\right)}{1-\hat{\mu}_{2}}\right)\hat{\mu}_{1}+\hat{r}_{1}\hat{\mu}_{1}\\
&=&\left(\hat{r}_{1}+\hat{r}_{2}+\frac{\hat{f}_{2}\left(4\right)}{1-\hat{\mu}_{2}}\right)\hat{\mu}_{1}=\left(\hat{r}+\frac{\hat{f}_{2}\left(4\right)}{1-\hat{\mu}_{2}}\right)\hat{\mu}_{1}\\
\hat{f}_{2}\left(1\right)&=&\hat{r}_{1}\mu_{1}+\mu_{1}\left(\frac{\hat{f}_{1}\left(3\right)}{1-\hat{\mu}_{1}}\right)+F_{1,1}^{(1)}\left(1\right)=\left(\hat{r}_{1}+\frac{\hat{f}_{1}\left(3\right)}{1-\hat{\mu}_{1}}\right)\mu_{1}+\frac{\mu_{1}}{\hat{\mu}_{1}}\\
\hat{f}_{2}\left(2\right)&=&\hat{r}_{1}\tilde{\mu}_{2}+\tilde{\mu}_{2}\left(\frac{\hat{f}_{1}\left(3\right)}{1-\hat{\mu}_{1}}\right)+F_{2,1}^{(1)}\left(1\right)=\left(\hat{r}_{1}+\frac{\hat{f}_{1}\left(3\right)}{1-\hat{\mu}_{1}}\right)\tilde{\mu}_{2}+\frac{\mu_{2}}{\hat{\mu}_{1}}\\
\hat{f}_{2}\left(4\right)&=&\hat{r}_{1}\hat{\mu}_{2}+\hat{\mu}_{2}\left(\frac{\hat{f}_{1}\left(3\right)}{1-\hat{\mu}_{1}}\right)+\hat{f}_{1}\left(4\right)=\hat{r}_{1}\hat{\mu}_{2}+\hat{r}_{2}\hat{\mu}_{2}+\hat{\mu}_{2}\left(\frac{\hat{f}_{1}\left(3\right)}{1-\hat{\mu}_{1}}\right)\\
&=&\left(\hat{r}+\frac{\hat{f}_{1}\left(3\right)}{1-\hat{\mu}_{1}}\right)\hat{\mu}_{2}\\
\end{eqnarray*}

es decir,


\begin{eqnarray*}
\begin{array}{lll}
f_{1}\left(1\right)=\mu_{1}\left(r+\frac{f_{2}\left(2\right)}{1-\tilde{\mu}_{2}}\right)&f_{1}\left(2\right)=r_{2}\tilde{\mu}_{2}&f_{1}\left(3\right)=\hat{\mu}_{1}\left(r_{2}+\frac{f_{2}\left(2\right)}{1-\tilde{\mu}_{2}}\right)+\frac{\hat{\mu}_{1}}{\mu_{2}}\\
f_{1}\left(4\right)=\hat{\mu}_{2}\left(r_{2}+\frac{f_{2}\left(2\right)}{1-\tilde{\mu}_{2}}\right)+\frac{\hat{\mu}_{2}}{\mu_{2}}&f_{2}\left(1\right)=r_{1}\mu_{1}&f_{2}\left(2\right)=\left(r+\frac{f_{1}\left(1\right)}{1-\mu_{1}}\right)\tilde{\mu}_{2}\\
f_{2}\left(3\right)=\hat{\mu}_{1}\left(r_{1}+\frac{f_{1}\left(1\right)}{1-\mu_{1}}\right)+\frac{\hat{\mu}_{1}}{\mu_{1}}&
f_{2}\left(4\right)=\hat{\mu}_{2}\left(r_{1}+\frac{f_{1}\left(1\right)}{1-\mu_{1}}\right)+\frac{\hat{\mu}_{2}}{\mu_{1}}&\hat{f}_{1}\left(1\right)=\left(\hat{r}_{2}+\frac{\hat{f}_{2}\left(4\right)}{1-\hat{\mu}_{2}}\right)\mu_{1}+\frac{\mu_{1}}{\hat{\mu}_{2}}\\
\hat{f}_{1}\left(2\right)=\left(\hat{r}_{2}+\frac{\hat{f}_{2}\left(4\right)}{1-\hat{\mu}_{2}}\right)\tilde{\mu}_{2}+\frac{\mu_{2}}{\hat{\mu}_{2}}&\hat{f}_{1}\left(3\right)=\left(\hat{r}+\frac{\hat{f}_{2}\left(4\right)}{1-\hat{\mu}_{2}}\right)\hat{\mu}_{1}&\hat{f}_{1}\left(4\right)=\hat{r}_{2}\hat{\mu}_{2}\\
\hat{f}_{2}\left(1\right)=\left(\hat{r}_{1}+\frac{\hat{f}_{1}\left(3\right)}{1-\hat{\mu}_{1}}\right)\mu_{1}+\frac{\mu_{1}}{\hat{\mu}_{1}}&\hat{f}_{2}\left(2\right)=\left(\hat{r}_{1}+\frac{\hat{f}_{1}\left(3\right)}{1-\hat{\mu}_{1}}\right)\tilde{\mu}_{2}+\frac{\mu_{2}}{\hat{\mu}_{1}}&\hat{f}_{2}\left(3\right)=\hat{r}_{1}\hat{\mu}_{1}\\
&\hat{f}_{2}\left(4\right)=\left(\hat{r}+\frac{\hat{f}_{1}\left(3\right)}{1-\hat{\mu}_{1}}\right)\hat{\mu}_{2}&
\end{array}
\end{eqnarray*}

%_______________________________________________________________________________________________
\subsection{Soluci\'on del Sistema de Ecuaciones Lineales}
%_________________________________________________________________________________________________

A saber, se puede demostrar que la soluci\'on del sistema de
ecuaciones est\'a dado por las siguientes expresiones, donde

\begin{eqnarray*}
\mu=\mu_{1}+\tilde{\mu}_{2}\textrm{ , }\hat{\mu}=\hat{\mu}_{1}+\hat{\mu}_{2}\textrm{ , }
r=r_{1}+r_{2}\textrm{ y }\hat{r}=\hat{r}_{1}+\hat{r}_{2}
\end{eqnarray*}
entonces

\begin{eqnarray*}
\begin{array}{lll}
f_{1}\left(1\right)=r\frac{\mu_{1}\left(1-\mu_{1}\right)}{1-\mu}&
f_{1}\left(3\right)=\hat{\mu}_{1}\left(\frac{r_{2}\mu_{2}+1}{\mu_{2}}+r\frac{\tilde{\mu}_{2}}{1-\mu}\right)&
f_{1}\left(4\right)=\hat{\mu}_{2}\left(\frac{r_{2}\mu_{2}+1}{\mu_{2}}+r\frac{\tilde{\mu}_{2}}{1-\mu}\right)\\
f_{2}\left(2\right)=r\frac{\tilde{\mu}_{2}\left(1-\tilde{\mu}_{2}\right)}{1-\mu}&
f_{2}\left(3\right)=\hat{\mu}_{1}\left(\frac{r_{1}\mu_{1}+1}{\mu_{1}}+r\frac{\mu_{1}}{1-\mu}\right)&
f_{2}\left(4\right)=\hat{\mu}_{2}\left(\frac{r_{1}\mu_{1}+1}{\mu_{1}}+r\frac{\mu_{1}}{1-\mu}\right)\\
\hat{f}_{1}\left(1\right)=\mu_{1}\left(\frac{\hat{r}_{2}\hat{\mu}_{2}+1}{\hat{\mu}_{2}}+\hat{r}\frac{\hat{\mu}_{2}}{1-\hat{\mu}}\right)&
\hat{f}_{1}\left(2\right)=\tilde{\mu}_{2}\left(\hat{r}_{2}+\hat{r}\frac{\hat{\mu}_{2}}{1-\hat{\mu}}\right)+\frac{\mu_{2}}{\hat{\mu}_{2}}&
\hat{f}_{1}\left(3\right)=\hat{r}\frac{\hat{\mu}_{1}\left(1-\hat{\mu}_{1}\right)}{1-\hat{\mu}}\\
\hat{f}_{2}\left(1\right)=\mu_{1}\left(\frac{\hat{r}_{1}\hat{\mu}_{1}+1}{\hat{\mu}_{1}}+\hat{r}\frac{\hat{\mu}_{1}}{1-\hat{\mu}}\right)&
\hat{f}_{2}\left(2\right)=\tilde{\mu}_{2}\left(\hat{r}_{1}+\hat{r}\frac{\hat{\mu}_{1}}{1-\hat{\mu}}\right)+\frac{\hat{\mu_{2}}}{\hat{\mu}_{1}}&
\hat{f}_{2}\left(4\right)=\hat{r}\frac{\hat{\mu}_{2}\left(1-\hat{\mu}_{2}\right)}{1-\hat{\mu}}\\
\end{array}
\end{eqnarray*}




A saber

\begin{eqnarray*}
f_{1}\left(3\right)&=&\hat{\mu}_{1}\left(r_{2}+\frac{f_{2}\left(2\right)}{1-\tilde{\mu}_{2}}\right)+\frac{\hat{\mu}_{1}}{\mu_{2}}=\hat{\mu}_{1}\left(r_{2}+\frac{r\frac{\tilde{\mu}_{2}\left(1-\tilde{\mu}_{2}\right)}{1-\mu}}{1-\tilde{\mu}_{2}}\right)+\frac{\hat{\mu}_{1}}{\mu_{2}}=\hat{\mu}_{1}\left(r_{2}+\frac{r\tilde{\mu}_{2}}{1-\mu}\right)+\frac{\hat{\mu}_{1}}{\mu_{2}}\\
&=&\hat{\mu}_{1}\left(r_{2}+\frac{r\tilde{\mu}_{2}}{1-\mu}+\frac{1}{\mu_{2}}\right)=\hat{\mu}_{1}\left(\frac{r_{2}\mu_{2}+1}{\mu_{2}}+\frac{r\tilde{\mu}_{2}}{1-\mu}\right)
\end{eqnarray*}

\begin{eqnarray*}
f_{1}\left(4\right)&=&\hat{\mu}_{2}\left(r_{2}+\frac{f_{2}\left(2\right)}{1-\tilde{\mu}_{2}}\right)+\frac{\hat{\mu}_{2}}{\mu_{2}}=\hat{\mu}_{2}\left(r_{2}+\frac{r\frac{\tilde{\mu}_{2}\left(1-\tilde{\mu}_{2}\right)}{1-\mu}}{1-\tilde{\mu}_{2}}\right)+\frac{\hat{\mu}_{2}}{\mu_{2}}=\hat{\mu}_{2}\left(r_{2}+\frac{r\tilde{\mu}_{2}}{1-\mu}\right)+\frac{\hat{\mu}_{1}}{\mu_{2}}\\
&=&\hat{\mu}_{2}\left(r_{2}+\frac{r\tilde{\mu}_{2}}{1-\mu}+\frac{1}{\mu_{2}}\right)=\hat{\mu}_{2}\left(\frac{r_{2}\mu_{2}+1}{\mu_{2}}+\frac{r\tilde{\mu}_{2}}{1-\mu}\right)
\end{eqnarray*}

\begin{eqnarray*}
f_{2}\left(3\right)&=&\hat{\mu}_{1}\left(r_{1}+\frac{f_{1}\left(1\right)}{1-\mu_{1}}\right)+\frac{\hat{\mu}_{1}}{\mu_{1}}=\hat{\mu}_{1}\left(r_{1}+\frac{r\frac{\mu_{1}\left(1-\mu_{1}\right)}{1-\mu}}{1-\mu_{1}}\right)+\frac{\hat{\mu}_{1}}{\mu_{1}}=\hat{\mu}_{1}\left(r_{1}+\frac{r\mu_{1}}{1-\mu}\right)+\frac{\hat{\mu}_{1}}{\mu_{1}}\\
&=&\hat{\mu}_{1}\left(r_{1}+\frac{r\mu_{1}}{1-\mu}+\frac{1}{\mu_{1}}\right)=\hat{\mu}_{1}\left(\frac{r_{1}\mu_{1}+1}{\mu_{1}}+\frac{r\mu_{1}}{1-\mu}\right)
\end{eqnarray*}

\begin{eqnarray*}
f_{2}\left(4\right)&=&\hat{\mu}_{2}\left(r_{1}+\frac{f_{1}\left(1\right)}{1-\mu_{1}}\right)+\frac{\hat{\mu}_{2}}{\mu_{1}}=\hat{\mu}_{2}\left(r_{1}+\frac{r\frac{\mu_{1}\left(1-\mu_{1}\right)}{1-\mu}}{1-\mu_{1}}\right)+\frac{\hat{\mu}_{1}}{\mu_{1}}=\hat{\mu}_{2}\left(r_{1}+\frac{r\mu_{1}}{1-\mu}\right)+\frac{\hat{\mu}_{1}}{\mu_{1}}\\
&=&\hat{\mu}_{2}\left(r_{1}+\frac{r\mu_{1}}{1-\mu}+\frac{1}{\mu_{1}}\right)=\hat{\mu}_{2}\left(\frac{r_{1}\mu_{1}+1}{\mu_{1}}+\frac{r\mu_{1}}{1-\mu}\right)\end{eqnarray*}


\begin{eqnarray*}
\hat{f}_{1}\left(1\right)&=&\mu_{1}\left(\hat{r}_{2}+\frac{\hat{f}_{2}\left(4\right)}{1-\tilde{\mu}_{2}}\right)+\frac{\mu_{1}}{\hat{\mu}_{2}}=\mu_{1}\left(\hat{r}_{2}+\frac{\hat{r}\frac{\hat{\mu}_{2}\left(1-\hat{\mu}_{2}\right)}{1-\hat{\mu}}}{1-\hat{\mu}_{2}}\right)+\frac{\mu_{1}}{\hat{\mu}_{2}}=\mu_{1}\left(\hat{r}_{2}+\frac{\hat{r}\hat{\mu}_{2}}{1-\hat{\mu}}\right)+\frac{\mu_{1}}{\mu_{2}}\\
&=&\mu_{1}\left(\hat{r}_{2}+\frac{\hat{r}\mu_{2}}{1-\hat{\mu}}+\frac{1}{\hat{\mu}_{2}}\right)=\mu_{1}\left(\frac{\hat{r}_{2}\hat{\mu}_{2}+1}{\hat{\mu}_{2}}+\frac{\hat{r}\hat{\mu}_{2}}{1-\hat{\mu}}\right)
\end{eqnarray*}

\begin{eqnarray*}
\hat{f}_{1}\left(2\right)&=&\tilde{\mu}_{2}\left(\hat{r}_{2}+\frac{\hat{f}_{2}\left(4\right)}{1-\tilde{\mu}_{2}}\right)+\frac{\mu_{2}}{\hat{\mu}_{2}}=\tilde{\mu}_{2}\left(\hat{r}_{2}+\frac{\hat{r}\frac{\hat{\mu}_{2}\left(1-\hat{\mu}_{2}\right)}{1-\hat{\mu}}}{1-\hat{\mu}_{2}}\right)+\frac{\mu_{2}}{\hat{\mu}_{2}}=\tilde{\mu}_{2}\left(\hat{r}_{2}+\frac{\hat{r}\hat{\mu}_{2}}{1-\hat{\mu}}\right)+\frac{\mu_{2}}{\hat{\mu}_{2}}
\end{eqnarray*}

\begin{eqnarray*}
\hat{f}_{2}\left(1\right)&=&\mu_{1}\left(\hat{r}_{1}+\frac{\hat{f}_{1}\left(3\right)}{1-\hat{\mu}_{1}}\right)+\frac{\mu_{1}}{\hat{\mu}_{1}}=\mu_{1}\left(\hat{r}_{1}+\frac{\hat{r}\frac{\hat{\mu}_{1}\left(1-\hat{\mu}_{1}\right)}{1-\hat{\mu}}}{1-\hat{\mu}_{1}}\right)+\frac{\mu_{1}}{\hat{\mu}_{1}}=\mu_{1}\left(\hat{r}_{1}+\frac{\hat{r}\hat{\mu}_{1}}{1-\hat{\mu}}\right)+\frac{\mu_{1}}{\hat{\mu}_{1}}\\
&=&\mu_{1}\left(\hat{r}_{1}+\frac{\hat{r}\hat{\mu}_{1}}{1-\hat{\mu}}+\frac{1}{\hat{\mu}_{1}}\right)=\mu_{1}\left(\frac{\hat{r}_{1}\hat{\mu}_{1}+1}{\hat{\mu}_{1}}+\frac{\hat{r}\hat{\mu}_{1}}{1-\hat{\mu}}\right)
\end{eqnarray*}

\begin{eqnarray*}
\hat{f}_{2}\left(2\right)&=&\tilde{\mu}_{2}\left(\hat{r}_{1}+\frac{\hat{f}_{1}\left(3\right)}{1-\tilde{\mu}_{1}}\right)+\frac{\mu_{2}}{\hat{\mu}_{1}}=\tilde{\mu}_{2}\left(\hat{r}_{1}+\frac{\hat{r}\frac{\hat{\mu}_{1}
\left(1-\hat{\mu}_{1}\right)}{1-\hat{\mu}}}{1-\hat{\mu}_{1}}\right)+\frac{\mu_{2}}{\hat{\mu}_{1}}=\tilde{\mu}_{2}\left(\hat{r}_{1}+\frac{\hat{r}\hat{\mu}_{1}}{1-\hat{\mu}}\right)+\frac{\mu_{2}}{\hat{\mu}_{1}}
\end{eqnarray*}

%----------------------------------------------------------------------------------------
\section{Resultado Principal}
%----------------------------------------------------------------------------------------
Sean $\mu_{1},\mu_{2},\check{\mu}_{2},\hat{\mu}_{1},\hat{\mu}_{2}$ y $\tilde{\mu}_{2}=\mu_{2}+\check{\mu}_{2}$ los valores esperados de los proceso definidos anteriormente, y sean $r_{1},r_{2}, \hat{r}_{1}$ y $\hat{r}_{2}$ los valores esperado s de los tiempos de traslado del servidor entre las colas para cada uno de los sistemas de visitas c\'iclicas. Si se definen $\mu=\mu_{1}+\tilde{\mu}_{2}$, $\hat{\mu}=\hat{\mu}_{1}+\hat{\mu}_{2}$, y $r=r_{1}+r_{2}$ y  $\hat{r}=\hat{r}_{1}+\hat{r}_{2}$, entonces se tiene el siguiente resultado.

\begin{Teo}
Supongamos que $\mu<1$, $\hat{\mu}<1$, entonces, el n\'umero de usuarios presentes en cada una de las colas que conforman la Red de Sistemas de Visitas C\'iclicas cuando uno de los servidores visita a alguna de ellas est\'a dada por la soluci\'on del Sistema de Ecuaciones Lineales presentados arriba cuyas expresiones damos a continuaci\'on:
%{\footnotesize{
\begin{eqnarray*}
\begin{array}{lll}
f_{1}\left(1\right)=r\frac{\mu_{1}\left(1-\mu_{1}\right)}{1-\mu},&f_{1}\left(2\right)=r_{2}\tilde{\mu}_{2},&f_{1}\left(3\right)=\hat{\mu}_{1}\left(\frac{r_{2}\mu_{2}+1}{\mu_{2}}+r\frac{\tilde{\mu}_{2}}{1-\mu}\right),\\
f_{1}\left(4\right)=\hat{\mu}_{2}\left(\frac{r_{2}\mu_{2}+1}{\mu_{2}}+r\frac{\tilde{\mu}_{2}}{1-\mu}\right),&f_{2}\left(1\right)=r_{1}\mu_{1},&f_{2}\left(2\right)=r\frac{\tilde{\mu}_{2}\left(1-\tilde{\mu}_{2}\right)}{1-\mu},\\
f_{2}\left(3\right)=\hat{\mu}_{1}\left(\frac{r_{1}\mu_{1}+1}{\mu_{1}}+r\frac{\mu_{1}}{1-\mu}\right),&f_{2}\left(4\right)=\hat{\mu}_{2}\left(\frac{r_{1}\mu_{1}+1}{\mu_{1}}+r\frac{\mu_{1}}{1-\mu}\right),&\hat{f}_{1}\left(1\right)=\mu_{1}\left(\frac{\hat{r}_{2}\hat{\mu}_{2}+1}{\hat{\mu}_{2}}+\hat{r}\frac{\hat{\mu}_{2}}{1-\hat{\mu}}\right),\\
\hat{f}_{1}\left(2\right)=\tilde{\mu}_{2}\left(\hat{r}_{2}+\hat{r}\frac{\hat{\mu}_{2}}{1-\hat{\mu}}\right)+\frac{\mu_{2}}{\hat{\mu}_{2}},&\hat{f}_{1}\left(3\right)=\hat{r}\frac{\hat{\mu}_{1}\left(1-\hat{\mu}_{1}\right)}{1-\hat{\mu}},&\hat{f}_{1}\left(4\right)=\hat{r}_{2}\hat{\mu}_{2},\\
\hat{f}_{2}\left(1\right)=\mu_{1}\left(\frac{\hat{r}_{1}\hat{\mu}_{1}+1}{\hat{\mu}_{1}}+\hat{r}\frac{\hat{\mu}_{1}}{1-\hat{\mu}}\right),&\hat{f}_{2}\left(2\right)=\tilde{\mu}_{2}\left(\hat{r}_{1}+\hat{r}\frac{\hat{\mu}_{1}}{1-\hat{\mu}}\right)+\frac{\hat{\mu_{2}}}{\hat{\mu}_{1}},&\hat{f}_{2}\left(3\right)=\hat{r}_{1}\hat{\mu}_{1},\\
&\hat{f}_{2}\left(4\right)=\hat{r}\frac{\hat{\mu}_{2}\left(1-\hat{\mu}_{2}\right)}{1-\hat{\mu}}.&\\
\end{array}
\end{eqnarray*} %}}
\end{Teo}





%___________________________________________________________________________________________
%
\section{Segundos Momentos}
%___________________________________________________________________________________________
%
%___________________________________________________________________________________________
%
%\subsection{Derivadas de Segundo Orden: Tiempos de Traslado del Servidor}
%___________________________________________________________________________________________



Para poder determinar los segundos momentos para los tiempos de traslado del servidor es necesaria la siguiente proposici\'on:

\begin{Prop}\label{Prop.Segundas.Derivadas}
Sea $f\left(g\left(x\right)h\left(y\right)\right)$ funci\'on continua tal que tiene derivadas parciales mixtas de segundo orden, entonces se tiene lo siguiente:

\begin{eqnarray*}
\frac{\partial}{\partial x}f\left(g\left(x\right)h\left(y\right)\right)=\frac{\partial f\left(g\left(x\right)h\left(y\right)\right)}{\partial x}\cdot \frac{\partial g\left(x\right)}{\partial x}\cdot h\left(y\right)
\end{eqnarray*}

por tanto

\begin{eqnarray}
\frac{\partial}{\partial x}\frac{\partial}{\partial x}f\left(g\left(x\right)h\left(y\right)\right)
&=&\frac{\partial^{2}}{\partial x}f\left(g\left(x\right)h\left(y\right)\right)\cdot \left(\frac{\partial g\left(x\right)}{\partial x}\right)^{2}\cdot h^{2}\left(y\right)+\frac{\partial}{\partial x}f\left(g\left(x\right)h\left(y\right)\right)\cdot \frac{\partial g^{2}\left(x\right)}{\partial x^{2}}\cdot h\left(y\right).
\end{eqnarray}

y

\begin{eqnarray*}
\frac{\partial}{\partial y}\frac{\partial}{\partial x}f\left(g\left(x\right)h\left(y\right)\right)&=&\frac{\partial g\left(x\right)}{\partial x}\cdot \frac{\partial h\left(y\right)}{\partial y}\left\{\frac{\partial^{2}}{\partial y\partial x}f\left(g\left(x\right)h\left(y\right)\right)\cdot g\left(x\right)\cdot h\left(y\right)+\frac{\partial}{\partial x}f\left(g\left(x\right)h\left(y\right)\right)\right\}
\end{eqnarray*}
\end{Prop}
\begin{proof}
\footnotesize{
\begin{eqnarray*}
\frac{\partial}{\partial x}\frac{\partial}{\partial x}f\left(g\left(x\right)h\left(y\right)\right)&=&\frac{\partial}{\partial x}\left\{\frac{\partial f\left(g\left(x\right)h\left(y\right)\right)}{\partial x}\cdot \frac{\partial g\left(x\right)}{\partial x}\cdot h\left(y\right)\right\}\\
&=&\frac{\partial}{\partial x}\left\{\frac{\partial}{\partial x}f\left(g\left(x\right)h\left(y\right)\right)\right\}\cdot \frac{\partial g\left(x\right)}{\partial x}\cdot h\left(y\right)+\frac{\partial}{\partial x}f\left(g\left(x\right)h\left(y\right)\right)\cdot \frac{\partial g^{2}\left(x\right)}{\partial x^{2}}\cdot h\left(y\right)\\
&=&\frac{\partial^{2}}{\partial x}f\left(g\left(x\right)h\left(y\right)\right)\cdot \frac{\partial g\left(x\right)}{\partial x}\cdot h\left(y\right)\cdot \frac{\partial g\left(x\right)}{\partial x}\cdot h\left(y\right)+\frac{\partial}{\partial x}f\left(g\left(x\right)h\left(y\right)\right)\cdot \frac{\partial g^{2}\left(x\right)}{\partial x^{2}}\cdot h\left(y\right)\\
&=&\frac{\partial^{2}}{\partial x}f\left(g\left(x\right)h\left(y\right)\right)\cdot \left(\frac{\partial g\left(x\right)}{\partial x}\right)^{2}\cdot h^{2}\left(y\right)+\frac{\partial}{\partial x}f\left(g\left(x\right)h\left(y\right)\right)\cdot \frac{\partial g^{2}\left(x\right)}{\partial x^{2}}\cdot h\left(y\right).
\end{eqnarray*}}


Por otra parte:
\footnotesize{
\begin{eqnarray*}
\frac{\partial}{\partial y}\frac{\partial}{\partial x}f\left(g\left(x\right)h\left(y\right)\right)&=&\frac{\partial}{\partial y}\left\{\frac{\partial f\left(g\left(x\right)h\left(y\right)\right)}{\partial x}\cdot \frac{\partial g\left(x\right)}{\partial x}\cdot h\left(y\right)\right\}\\
&=&\frac{\partial}{\partial y}\left\{\frac{\partial}{\partial x}f\left(g\left(x\right)h\left(y\right)\right)\right\}\cdot \frac{\partial g\left(x\right)}{\partial x}\cdot h\left(y\right)+\frac{\partial}{\partial x}f\left(g\left(x\right)h\left(y\right)\right)\cdot \frac{\partial g\left(x\right)}{\partial x}\cdot \frac{\partial h\left(y\right)}{y}\\
&=&\frac{\partial^{2}}{\partial y\partial x}f\left(g\left(x\right)h\left(y\right)\right)\cdot \frac{\partial h\left(y\right)}{\partial y}\cdot g\left(x\right)\cdot \frac{\partial g\left(x\right)}{\partial x}\cdot h\left(y\right)+\frac{\partial}{\partial x}f\left(g\left(x\right)h\left(y\right)\right)\cdot \frac{\partial g\left(x\right)}{\partial x}\cdot \frac{\partial h\left(y\right)}{\partial y}\\
&=&\frac{\partial g\left(x\right)}{\partial x}\cdot \frac{\partial h\left(y\right)}{\partial y}\left\{\frac{\partial^{2}}{\partial y\partial x}f\left(g\left(x\right)h\left(y\right)\right)\cdot g\left(x\right)\cdot h\left(y\right)+\frac{\partial}{\partial x}f\left(g\left(x\right)h\left(y\right)\right)\right\}
\end{eqnarray*}}
\end{proof}

Utilizando la proposici\'on anterior (Proposici\'ion \ref{Prop.Segundas.Derivadas})se tiene el siguiente resultado que me dice como calcular los segundos momentos para los procesos de traslado del servidor:

\begin{Prop}
Sea $R_{i}$ la Funci\'on Generadora de Probabilidades para el n\'umero de arribos a cada una de las colas de la Red de Sistemas de Visitas C\'iclicas definidas como en (\ref{Ec.R1}). Entonces las derivadas parciales est\'an dadas por las siguientes expresiones:


\begin{eqnarray*}
\frac{\partial^{2} R_{i}\left(P_{1}\left(z_{1}\right)\tilde{P}_{2}\left(z_{2}\right)\hat{P}_{1}\left(w_{1}\right)\hat{P}_{2}\left(w_{2}\right)\right)}{\partial z_{i}^{2}}&=&\left(\frac{\partial P_{i}\left(z_{i}\right)}{\partial z_{i}}\right)^{2}\cdot\frac{\partial^{2} R_{i}\left(P_{1}\left(z_{1}\right)\tilde{P}_{2}\left(z_{2}\right)\hat{P}_{1}\left(w_{1}\right)\hat{P}_{2}\left(w_{2}\right)\right)}{\partial^{2} z_{i}}\\
&+&\left(\frac{\partial P_{i}\left(z_{i}\right)}{\partial z_{i}}\right)^{2}\cdot
\frac{\partial R_{i}\left(P_{1}\left(z_{1}\right)\tilde{P}_{2}\left(z_{2}\right)\hat{P}_{1}\left(w_{1}\right)\hat{P}_{2}\left(w_{2}\right)\right)}{\partial z_{i}}
\end{eqnarray*}



y adem\'as


\begin{eqnarray*}
\frac{\partial^{2} R_{i}\left(P_{1}\left(z_{1}\right)\tilde{P}_{2}\left(z_{2}\right)\hat{P}_{1}\left(w_{1}\right)\hat{P}_{2}\left(w_{2}\right)\right)}{\partial z_{2}\partial z_{1}}&=&\frac{\partial \tilde{P}_{2}\left(z_{2}\right)}{\partial z_{2}}\cdot\frac{\partial P_{1}\left(z_{1}\right)}{\partial z_{1}}\cdot\frac{\partial^{2} R_{i}\left(P_{1}\left(z_{1}\right)\tilde{P}_{2}\left(z_{2}\right)\hat{P}_{1}\left(w_{1}\right)\hat{P}_{2}\left(w_{2}\right)\right)}{\partial z_{2}\partial z_{1}}\\
&+&\frac{\partial \tilde{P}_{2}\left(z_{2}\right)}{\partial z_{2}}\cdot\frac{\partial P_{1}\left(z_{1}\right)}{\partial z_{1}}\cdot\frac{\partial R_{i}\left(P_{1}\left(z_{1}\right)\tilde{P}_{2}\left(z_{2}\right)\hat{P}_{1}\left(w_{1}\right)\hat{P}_{2}\left(w_{2}\right)\right)}{\partial z_{1}},
\end{eqnarray*}



\begin{eqnarray*}
\frac{\partial^{2} R_{i}\left(P_{1}\left(z_{1}\right)\tilde{P}_{2}\left(z_{2}\right)\hat{P}_{1}\left(w_{1}\right)\hat{P}_{2}\left(w_{2}\right)\right)}{\partial w_{i}\partial z_{1}}&=&\frac{\partial \hat{P}_{i}\left(w_{i}\right)}{\partial z_{2}}\cdot\frac{\partial P_{1}\left(z_{1}\right)}{\partial z_{1}}\cdot\frac{\partial^{2} R_{i}\left(P_{1}\left(z_{1}\right)\tilde{P}_{2}\left(z_{2}\right)\hat{P}_{1}\left(w_{1}\right)\hat{P}_{2}\left(w_{2}\right)\right)}{\partial w_{i}\partial z_{1}}\\
&+&\frac{\partial \hat{P}_{i}\left(w_{i}\right)}{\partial z_{2}}\cdot\frac{\partial P_{1}\left(z_{1}\right)}{\partial z_{1}}\cdot\frac{\partial R_{i}\left(P_{1}\left(z_{1}\right)\tilde{P}_{2}\left(z_{2}\right)\hat{P}_{1}\left(w_{1}\right)\hat{P}_{2}\left(w_{2}\right)\right)}{\partial z_{1}},
\end{eqnarray*}
finalmente

\begin{eqnarray*}
\frac{\partial^{2} R_{i}\left(P_{1}\left(z_{1}\right)\tilde{P}_{2}\left(z_{2}\right)\hat{P}_{1}\left(w_{1}\right)\hat{P}_{2}\left(w_{2}\right)\right)}{\partial w_{i}\partial z_{2}}&=&\frac{\partial \hat{P}_{i}\left(w_{i}\right)}{\partial w_{i}}\cdot\frac{\partial \tilde{P}_{2}\left(z_{2}\right)}{\partial z_{2}}\cdot\frac{\partial^{2} R_{i}\left(P_{1}\left(z_{1}\right)\tilde{P}_{2}\left(z_{2}\right)\hat{P}_{1}\left(w_{1}\right)\hat{P}_{2}\left(w_{2}\right)\right)}{\partial w_{i}\partial z_{2}}\\
&+&\frac{\partial \hat{P}_{i}\left(w_{i}\right)}{\partial w_{i}}\cdot\frac{\partial \tilde{P}_{2}\left(z_{2}\right)}{\partial z_{1}}\cdot\frac{\partial R_{i}\left(P_{1}\left(z_{1}\right)\tilde{P}_{2}\left(z_{2}\right)\hat{P}_{1}\left(w_{1}\right)\hat{P}_{2}\left(w_{2}\right)\right)}{\partial z_{2}},
\end{eqnarray*}

para $i=1,2$.
\end{Prop}

%___________________________________________________________________________________________
%
\subsection{Sistema de Ecuaciones Lineales para los Segundos Momentos}
%___________________________________________________________________________________________

En el ap\'endice (\ref{Segundos.Momentos}) se demuestra que las ecuaciones para las ecuaciones parciales mixtas est\'an dadas por:



%___________________________________________________________________________________________
%\subsubsection{Mixtas para $z_{1}$:}
%___________________________________________________________________________________________
%1
\begin{eqnarray*}
f_{1}\left(1,1\right)&=&r_{2}P_{1}^{(2)}\left(1\right)+\mu_{1}^{2}R_{2}^{(2)}\left(1\right)+2\mu_{1}r_{2}\left(\frac{\mu_{1}}{1-\tilde{\mu}_{2}}f_{2}\left(2\right)+f_{2}\left(1\right)\right)+\frac{1}{1-\tilde{\mu}_{2}}P_{1}^{(2)}f_{2}\left(2\right)+\mu_{1}^{2}\tilde{\theta}_{2}^{(2)}\left(1\right)f_{2}\left(2\right)\\
&+&\frac{\mu_{1}}{1-\tilde{\mu}_{2}}f_{2}(1,2)+\frac{\mu_{1}}{1-\tilde{\mu}_{2}}\left(\frac{\mu_{1}}{1-\tilde{\mu}_{2}}f_{2}(2,2)+f_{2}(1,2)\right)+f_{2}(1,1),\\
f_{1}\left(2,1\right)&=&\mu_{1}r_{2}\tilde{\mu}_{2}+\mu_{1}\tilde{\mu}_{2}R_{2}^{(2)}\left(1\right)+r_{2}\tilde{\mu}_{2}\left(\frac{\mu_{1}}{1-\tilde{\mu}_{2}}f_{2}(2)+f_{2}(1)\right),\\
f_{1}\left(3,1\right)&=&\mu_{1}\hat{\mu}_{1}r_{2}+\mu_{1}\hat{\mu}_{1}R_{2}^{(2)}\left(1\right)+r_{2}\frac{\mu_{1}}{1-\tilde{\mu}_{2}}f_{2}(2)+r_{2}\hat{\mu}_{1}\left(\frac{\mu_{1}}{1-\tilde{\mu}_{2}}f_{2}(2)+f_{2}(1)\right)+\mu_{1}r_{2}\hat{F}_{2,1}^{(1)}(1)\\
&+&\left(\frac{\mu_{1}}{1-\tilde{\mu}_{2}}f_{2}(2)+f_{2}(1)\right)\hat{F}_{2,1}^{(1)}(1)+\frac{\mu_{1}\hat{\mu}_{1}}{1-\tilde{\mu}_{2}}f_{2}(2)+\mu_{1}\hat{\mu}_{1}\tilde{\theta}_{2}^{(2)}\left(1\right)f_{2}(2)+\mu_{1}\hat{\mu}_{1}\left(\frac{1}{1-\tilde{\mu}_{2}}\right)^{2}f_{2}(2,2)\\
&+&+\frac{\hat{\mu}_{1}}{1-\tilde{\mu}_{2}}f_{2}(1,2),\\
f_{1}\left(4,1\right)&=&\mu_{1}\hat{\mu}_{2}r_{2}+\mu_{1}\hat{\mu}_{2}R_{2}^{(2)}\left(1\right)+r_{2}\frac{\mu_{1}\hat{\mu}_{2}}{1-\tilde{\mu}_{2}}f_{2}(2)+\mu_{1}r_{2}\hat{F}_{2,2}^{(1)}(1)+r_{2}\hat{\mu}_{2}\left(\frac{\mu_{1}}{1-\tilde{\mu}_{2}}f_{2}(2)+f_{2}(1)\right)\\
&+&\hat{F}_{2,1}^{(1)}(1)\left(\frac{\mu_{1}}{1-\tilde{\mu}_{2}}f_{2}(2)+f_{2}(1)\right)+\frac{\mu_{1}\hat{\mu}_{2}}{1-\tilde{\mu}_{2}}f_{2}(2)
+\mu_{1}\hat{\mu}_{2}\tilde{\theta}_{2}^{(2)}\left(1\right)f_{2}(2)+\mu_{1}\hat{\mu}_{2}\left(\frac{1}{1-\tilde{\mu}_{2}}\right)^{2}f_{2}(2,2)\\
&+&\frac{\hat{\mu}_{2}}{1-\tilde{\mu}_{2}}f_{2}^{(1,2)},\\
\end{eqnarray*}
\begin{eqnarray*}
f_{1}\left(1,2\right)&=&\mu_{1}\tilde{\mu}_{2}r_{2}+\mu_{1}\tilde{\mu}_{2}R_{2}^{(2)}\left(1\right)+r_{2}\tilde{\mu}_{2}\left(\frac{\mu_{1}}{1-\tilde{\mu}_{2}}f_{2}(2)+f_{2}(1)\right),\\
f_{1}\left(2,2\right)&=&\tilde{\mu}_{2}^{2}R_{2}^{(2)}(1)+r_{2}\tilde{P}_{2}^{(2)}\left(1\right),\\
f_{1}\left(3,2\right)&=&\hat{\mu}_{1}\tilde{\mu}_{2}r_{2}+\hat{\mu}_{1}\tilde{\mu}_{2}R_{2}^{(2)}(1)+
r_{2}\frac{\hat{\mu}_{1}\tilde{\mu}_{2}}{1-\tilde{\mu}_{2}}f_{2}(2)+r_{2}\tilde{\mu}_{2}\hat{F}_{2,2}^{(1)}(1),\\
f_{1}\left(4,2\right)&=&\hat{\mu}_{2}\tilde{\mu}_{2}r_{2}+\hat{\mu}_{2}\tilde{\mu}_{2}R_{2}^{(2)}(1)+
r_{2}\frac{\hat{\mu}_{2}\tilde{\mu}_{2}}{1-\tilde{\mu}_{2}}f_{2}(2)+r_{2}\tilde{\mu}_{2}\hat{F}_{2,2}^{(1)}(1),\\
f_{1}\left(1,3\right)&=&\mu_{1}\hat{\mu}_{1}r_{2}+\mu_{1}\hat{\mu}_{1}R_{2}^{(2)}\left(1\right)+\frac{\mu_{1}\hat{\mu}_{1}}{1-\tilde{\mu}_{2}}f_{2}(2)+r_{2}\frac{\mu_{1}\hat{\mu}_{1}}{1-\tilde{\mu}_{2}}f_{2}(2)+\mu_{1}\hat{\mu}_{1}\tilde{\theta}_{2}^{(2)}\left(1\right)f_{2}(2)+r_{2}\mu_{1}\hat{F}_{2,1}^{(1)}(1)\\
&+&r_{2}\hat{\mu}_{1}\left(\frac{\mu_{1}}{1-\tilde{\mu}_{2}}f_{2}(2)+f_{2}\left(1\right)\right)+\left(\frac{\mu_{1}}{1-\tilde{\mu}_{2}}f_{2}\left(1\right)+f_{2}\left(1\right)\right)\hat{F}_{2,1}^{(1)}(1)\\
&+&\frac{\hat{\mu}_{1}}{1-\tilde{\mu}_{2}}\left(\frac{\mu_{1}}{1-\tilde{\mu}_{2}}f_{2}(2,2)+f_{2}^{(1,2)}\right),\\
f_{1}\left(2,3\right)&=&\tilde{\mu}_{2}\hat{\mu}_{1}r_{2}+\tilde{\mu}_{2}\hat{\mu}_{1}R_{2}^{(2)}\left(1\right)+r_{2}\frac{\tilde{\mu}_{2}\hat{\mu}_{1}}{1-\tilde{\mu}_{2}}f_{2}(2)+r_{2}\tilde{\mu}_{2}\hat{F}_{2,1}^{(1)}(1),\\
f_{1}\left(3,3\right)&=&\hat{\mu}_{1}^{2}R_{2}^{(2)}\left(1\right)+r_{2}\hat{P}_{1}^{(2)}\left(1\right)+2r_{2}\frac{\hat{\mu}_{1}^{2}}{1-\tilde{\mu}_{2}}f_{2}(2)+\hat{\mu}_{1}^{2}\tilde{\theta}_{2}^{(2)}\left(1\right)f_{2}(2)+\frac{1}{1-\tilde{\mu}_{2}}\hat{P}_{1}^{(2)}\left(1\right)f_{2}(2)\\
&+&\frac{\hat{\mu}_{1}^{2}}{1-\tilde{\mu}_{2}}f_{2}(2,2)+2r_{2}\hat{\mu}_{1}\hat{F}_{2,1}^{(1)}(1)+2\frac{\hat{\mu}_{1}}{1-\tilde{\mu}_{2}}f_{2}(2)\hat{F}_{2,1}^{(1)}(1)+\hat{f}_{2,1}^{(2)}(1),\\
f_{1}\left(4,3\right)&=&r_{2}\hat{\mu}_{2}\hat{\mu}_{1}+\hat{\mu}_{1}\hat{\mu}_{2}R_{2}^{(2)}(1)+\frac{\hat{\mu}_{1}\hat{\mu}_{2}}{1-\tilde{\mu}_{2}}f_{2}\left(2\right)+2r_{2}\frac{\hat{\mu}_{1}\hat{\mu}_{2}}{1-\tilde{\mu}_{2}}f_{2}\left(2\right)+\hat{\mu}_{2}\hat{\mu}_{1}\tilde{\theta}_{2}^{(2)}\left(1\right)f_{2}\left(2\right)+r_{2}\hat{\mu}_{1}\hat{F}_{2,2}^{(1)}(1)\\
&+&\frac{\hat{\mu}_{1}}{1-\tilde{\mu}_{2}}f_{2}\left(2\right)\hat{F}_{2,2}^{(1)}(1)+\hat{\mu}_{1}\hat{\mu}_{2}\left(\frac{1}{1-\tilde{\mu}_{2}}\right)^{2}f_{2}(2,2)+r_{2}\hat{\mu}_{2}\hat{F}_{2,1}^{(1)}(1)+\frac{\hat{\mu}_{2}}{1-\tilde{\mu}_{2}}f_{2}\left(2\right)\hat{F}_{2,1}^{(1)}(1)+\hat{f}_{2}(1,2),\\
f_{1}\left(1,4\right)&=&r_{2}\mu_{1}\hat{\mu}_{2}+\mu_{1}\hat{\mu}_{2}R_{2}^{(2)}(1)+\frac{\mu_{1}\hat{\mu}_{2}}{1-\tilde{\mu}_{2}}f_{2}(2)+r_{2}\frac{\mu_{1}\hat{\mu}_{2}}{1-\tilde{\mu}_{2}}f_{2}(2)+\mu_{1}\hat{\mu}_{2}\tilde{\theta}_{2}^{(2)}\left(1\right)f_{2}(2)+r_{2}\mu_{1}\hat{F}_{2,2}^{(1)}(1)\\
&+&r_{2}\hat{\mu}_{2}\left(\frac{\mu_{1}}{1-\tilde{\mu}_{2}}f_{2}(2)+f_{2}(1)\right)+\hat{F}_{2,2}^{(1)}(1)\left(\frac{\mu_{1}}{1-\tilde{\mu}_{2}}f_{2}(2)+f_{2}(1)\right)\\
&+&\frac{\hat{\mu}_{2}}{1-\tilde{\mu}_{2}}\left(\frac{\mu_{1}}{1-\tilde{\mu}_{2}}f_{2}(2,2)+f_{2}(1,2)\right),\\
f_{1}\left(2,4\right)
&=&r_{2}\tilde{\mu}_{2}\hat{\mu}_{2}+\tilde{\mu}_{2}\hat{\mu}_{2}R_{2}^{(2)}(1)+r_{2}\frac{\tilde{\mu}_{2}\hat{\mu}_{2}}{1-\tilde{\mu}_{2}}f_{2}(2)+r_{2}\tilde{\mu}_{2}\hat{F}_{2,2}^{(1)}(1),\\
f_{1}\left(3,4\right)&=&r_{2}\hat{\mu}_{1}\hat{\mu}_{2}+\hat{\mu}_{1}\hat{\mu}_{2}R_{2}^{(2)}\left(1\right)+\frac{\hat{\mu}_{1}\hat{\mu}_{2}}{1-\tilde{\mu}_{2}}f_{2}(2)+2r_{2}\frac{\hat{\mu}_{1}\hat{\mu}_{2}}{1-\tilde{\mu}_{2}}f_{2}(2)+\hat{\mu}_{1}\hat{\mu}_{2}\theta_{2}^{(2)}\left(1\right)f_{2}(2)+r_{2}\hat{\mu}_{1}\hat{F}_{2,2}^{(1)}(1)\\
&+&\frac{\hat{\mu}_{1}}{1-\tilde{\mu}_{2}}f_{2}(2)\hat{F}_{2,2}^{(1)}(1)+\hat{\mu}_{1}\hat{\mu}_{2}\left(\frac{1}{1-\tilde{\mu}_{2}}\right)^{2}f_{2}(2,2)+r_{2}\hat{\mu}_{2}\hat{F}_{2,2}^{(1)}(1)+\frac{\hat{\mu}_{2}}{1-\tilde{\mu}_{2}}f_{2}(2)\hat{F}_{2,1}^{(1)}(1)+\hat{f}_{2}^{(2)}(1,2),\\
f_{1}\left(4,4\right)&=&\hat{\mu}_{2}^{2}R_{2}^{(2)}(1)+r_{2}\hat{P}_{2}^{(2)}\left(1\right)+2r_{2}\frac{\hat{\mu}_{2}^{2}}{1-\tilde{\mu}_{2}}f_{2}(2)+\hat{\mu}_{2}^{2}\tilde{\theta}_{2}^{(2)}\left(1\right)f_{2}(2)+\frac{1}{1-\tilde{\mu}_{2}}\hat{P}_{2}^{(2)}\left(1\right)f_{2}(2)\\
&+&2r_{2}\hat{\mu}_{2}\hat{F}_{2,2}^{(1)}(1)+2\frac{\hat{\mu}_{2}}{1-\tilde{\mu}_{2}}f_{2}(2)\hat{F}_{2,2}^{(1)}(1)+\left(\frac{\hat{\mu}_{2}}{1-\tilde{\mu}_{2}}\right)^{2}f_{2}(2,2)+\hat{f}_{2,2}^{(2)}(1),\\
f_{2}\left(1,1\right)&=&r_{1}P_{1}^{(2)}\left(1\right)+\mu_{1}^{2}R_{1}^{(2)}\left(1\right),\\
f_{2}\left(2,1\right)&=&\mu_{1}\tilde{\mu}_{2}r_{1}+\mu_{1}\tilde{\mu}_{2}R_{1}^{(2)}(1)+
r_{1}\mu_{1}\left(\frac{\tilde{\mu}_{2}}{1-\mu_{1}}f_{1}(1)+f_{1}(2)\right),\\
f_{2}\left(3,1\right)&=&r_{1}\mu_{1}\hat{\mu}_{1}+\mu_{1}\hat{\mu}_{1}R_{1}^{(2)}\left(1\right)+r_{1}\frac{\mu_{1}\hat{\mu}_{1}}{1-\mu_{1}}f_{1}(1)+r_{1}\mu_{1}\hat{F}_{1,1}^{(1)}(1),\\
f_{2}\left(4,1\right)&=&\mu_{1}\hat{\mu}_{2}r_{1}+\mu_{1}\hat{\mu}_{2}R_{1}^{(2)}\left(1\right)+r_{1}\mu_{1}\hat{F}_{1,2}^{(1)}(1)+r_{1}\frac{\mu_{1}\hat{\mu}_{2}}{1-\mu_{1}}f_{1}(1),\\
\end{eqnarray*}
\begin{eqnarray*}
f_{2}\left(1,2\right)&=&r_{1}\mu_{1}\tilde{\mu}_{2}+\mu_{1}\tilde{\mu}_{2}R_{1}^{(2)}\left(1\right)+r_{1}\mu_{1}\left(\frac{\tilde{\mu}_{2}}{1-\mu_{1}}f_{1}(1)+f_{1}(2)\right),\\
f_{2}\left(2,2\right)&=&\tilde{\mu}_{2}^{2}R_{1}^{(2)}\left(1\right)+r_{1}\tilde{P}_{2}^{(2)}\left(1\right)+2r_{1}\tilde{\mu}_{2}\left(\frac{\tilde{\mu}_{2}}{1-\mu_{1}}f_{1}(1)+f_{1}(2)\right)+f_{1}(2,2)+\tilde{\mu}_{2}^{2}\theta_{1}^{(2)}\left(1\right)f_{1}(1)\\
&+&\frac{1}{1-\mu_{1}}\tilde{P}_{2}^{(2)}\left(1\right)f_{1}(1)+\frac{\tilde{\mu}_{2}}{1-\mu_{1}}f_{1}(1,2)+\frac{\tilde{\mu}_{2}}{1-\mu_{1}}\left(\frac{\tilde{\mu}_{2}}{1-\mu_{1}}f_{1}(1,1)+f_{1}(1,2)\right),\\
f_{2}\left(3,2\right)&=&\tilde{\mu}_{2}\hat{\mu}_{1}r_{1}+\tilde{\mu}_{2}\hat{\mu}_{1}R_{1}^{(2)}\left(1\right)+r_{1}\frac{\tilde{\mu}_{2}\hat{\mu}_{1}}{1-\mu_{1}}f_{1}(1)+\hat{\mu}_{1}r_{1}\left(\frac{\tilde{\mu}_{2}}{1-\mu_{1}}f_{1}(1)+f_{1}(2)\right)+r_{1}\tilde{\mu}_{2}\hat{F}_{1,1}^{(1)}(1)\\
&+&\left(\frac{\tilde{\mu}_{2}}{1-\mu_{1}}f_{1}(1)+f_{1}(2)\right)\hat{F}_{1,1}^{(1)}(1)+\frac{\tilde{\mu}_{2}\hat{\mu}_{1}}{1-\mu_{1}}f_{1}(1)+\tilde{\mu}_{2}\hat{\mu}_{1}\theta_{1}^{(2)}\left(1\right)f_{1}(1)+\frac{\hat{\mu}_{1}}{1-\mu_{1}}f_{1}(1,2)\\
&+&\left(\frac{1}{1-\mu_{1}}\right)^{2}\tilde{\mu}_{2}\hat{\mu}_{1}f_{1}(1,1),\\
f_{2}\left(4,2\right)&=&\hat{\mu}_{2}\tilde{\mu}_{2}r_{1}+\hat{\mu}_{2}\tilde{\mu}_{2}R_{1}^{(2)}(1)+r_{1}\tilde{\mu}_{2}\hat{F}_{1,2}^{(1)}(1)+r_{1}\frac{\hat{\mu}_{2}\tilde{\mu}_{2}}{1-\mu_{1}}f_{1}(1)+\hat{\mu}_{2}r_{1}\left(\frac{\tilde{\mu}_{2}}{1-\mu_{1}}f_{1}(1)+f_{1}(2)\right)\\
&+&\left(\frac{\tilde{\mu}_{2}}{1-\mu_{1}}f_{1}(1)+f_{1}(2)\right)\hat{F}_{1,2}^{(1)}(1)+\frac{\tilde{\mu}_{2}\hat{\mu_{2}}}{1-\mu_{1}}f_{1}(1)+\hat{\mu}_{2}\tilde{\mu}_{2}\theta_{1}^{(2)}\left(1\right)f_{1}(1)+\frac{\hat{\mu}_{2}}{1-\mu_{1}}f_{1}(1,2)\\
&+&\tilde{\mu}_{2}\hat{\mu}_{2}\left(\frac{1}{1-\mu_{1}}\right)^{2}f_{1}(1,1),\\
f_{2}\left(1,3\right)&=&r_{1}\mu_{1}\hat{\mu}_{1}+\mu_{1}\hat{\mu}_{1}R_{1}^{(2)}(1)+r_{1}\frac{\mu_{1}\hat{\mu}_{1}}{1-\mu_{1}}f_{1}(1)+r_{1}\mu_{1}\hat{F}_{1,1}^{(1)}(1),\\
 f_{2}\left(2,3\right)&=&r_{1}\hat{\mu}_{1}\tilde{\mu}_{2}+\tilde{\mu}_{2}\hat{\mu}_{1}R_{1}^{(2)}\left(1\right)+\frac{\hat{\mu}_{1}\tilde{\mu}_{2}}{1-\mu_{1}}f_{1}(1)+r_{1}\frac{\hat{\mu}_{1}\tilde{\mu}_{2}}{1-\mu_{1}}f_{1}(1)+\hat{\mu}_{1}\tilde{\mu}_{2}\theta_{1}^{(2)}\left(1\right)f_{1}(1)+r_{1}\tilde{\mu}_{2}\hat{F}_{1,1}(1)\\
&+&r_{1}\hat{\mu}_{1}\left(f_{1}(1)+\frac{\tilde{\mu}_{2}}{1-\mu_{1}}f_{1}(1)\right)+
+\left(f_{1}(2)+\frac{\tilde{\mu}_{2}}{1-\mu_{1}}f_{1}(1)\right)\hat{F}_{1,1}(1)\\
&+&\frac{\hat{\mu}_{1}}{1-\mu_{1}}\left(f_{1}(1,2)+\frac{\tilde{\mu}_{2}}{1-\mu_{1}}f_{1}(1,1)\right),\\
f_{2}\left(3,3\right)&=&\hat{\mu}_{1}^{2}R_{1}^{(2)}\left(1\right)+r_{1}\hat{P}_{1}^{(2)}\left(1\right)+2r_{1}\frac{\hat{\mu}_{1}^{2}}{1-\mu_{1}}f_{1}(1)+\hat{\mu}_{1}^{2}\theta_{1}^{(2)}\left(1\right)f_{1}(1)+2r_{1}\hat{\mu}_{1}\hat{F}_{1,1}^{(1)}(1)\\
&+&\frac{1}{1-\mu_{1}}\hat{P}_{1}^{(2)}\left(1\right)f_{1}(1)+2\frac{\hat{\mu}_{1}}{1-\mu_{1}}f_{1}(1)\hat{F}_{1,1}(1)+\left(\frac{\hat{\mu}_{1}}{1-\mu_{1}}\right)^{2}f_{1}(1,1)+\hat{f}_{1,1}^{(2)}(1),\\
f_{2}\left(4,3\right)&=&r_{1}\hat{\mu}_{1}\hat{\mu}_{2}+\hat{\mu}_{1}\hat{\mu}_{2}R_{1}^{(2)}\left(1\right)+r_{1}\hat{\mu}_{1}\hat{F}_{1,2}(1)+
\frac{\hat{\mu}_{1}\hat{\mu}_{2}}{1-\mu_{1}}f_{1}(1)+2r_{1}\frac{\hat{\mu}_{1}\hat{\mu}_{2}}{1-\mu_{1}}f_{1}(1)+r_{1}\hat{\mu}_{2}\hat{F}_{1,1}(1)\\
&+&\hat{\mu}_{1}\hat{\mu}_{2}\theta_{1}^{(2)}\left(1\right)f_{1}(1)+\frac{\hat{\mu}_{1}}{1-\mu_{1}}f_{1}(1)\hat{F}_{1,2}(1)+\frac{\hat{\mu}_{2}}{1-\mu_{1}}\hat{F}_{1,1}(1)f_{1}(1)\\
&+&\hat{f}_{1}^{(2)}(1,2)+\hat{\mu}_{1}\hat{\mu}_{2}\left(\frac{1}{1-\mu_{1}}\right)^{2}f_{1}(2,2),\\
f_{2}\left(1,4\right)&=&r_{1}\mu_{1}\hat{\mu}_{2}+\mu_{1}\hat{\mu}_{2}R_{1}^{(2)}\left(1\right)+r_{1}\mu_{1}\hat{F}_{1,2}(1)+r_{1}\frac{\mu_{1}\hat{\mu}_{2}}{1-\mu_{1}}f_{1}(1),\\
f_{2}\left(2,4\right)&=&r_{1}\hat{\mu}_{2}\tilde{\mu}_{2}+\hat{\mu}_{2}\tilde{\mu}_{2}R_{1}^{(2)}\left(1\right)+r_{1}\tilde{\mu}_{2}\hat{F}_{1,2}(1)+\frac{\hat{\mu}_{2}\tilde{\mu}_{2}}{1-\mu_{1}}f_{1}(1)+r_{1}\frac{\hat{\mu}_{2}\tilde{\mu}_{2}}{1-\mu_{1}}f_{1}(1)+\hat{\mu}_{2}\tilde{\mu}_{2}\theta_{1}^{(2)}\left(1\right)f_{1}(1)\\
&+&r_{1}\hat{\mu}_{2}\left(f_{1}(2)+\frac{\tilde{\mu}_{2}}{1-\mu_{1}}f_{1}(1)\right)+\left(f_{1}(2)+\frac{\tilde{\mu}_{2}}{1-\mu_{1}}f_{1}(1)\right)\hat{F}_{1,2}(1)\\&+&\frac{\hat{\mu}_{2}}{1-\mu_{1}}\left(f_{1}(1,2)+\frac{\tilde{\mu}_{2}}{1-\mu_{1}}f_{1}(1,1)\right),\\
\end{eqnarray*}
\begin{eqnarray*}
f_{2}\left(3,4\right)&=&r_{1}\hat{\mu}_{1}\hat{\mu}_{2}+\hat{\mu}_{1}\hat{\mu}_{2}R_{1}^{(2)}\left(1\right)+r_{1}\hat{\mu}_{1}\hat{F}_{1,2}(1)+
\frac{\hat{\mu}_{1}\hat{\mu}_{2}}{1-\mu_{1}}f_{1}(1)+2r_{1}\frac{\hat{\mu}_{1}\hat{\mu}_{2}}{1-\mu_{1}}f_{1}(1)+\hat{\mu}_{1}\hat{\mu}_{2}\theta_{1}^{(2)}\left(1\right)f_{1}(1)\\
&+&+\frac{\hat{\mu}_{1}}{1-\mu_{1}}\hat{F}_{1,2}(1)f_{1}(1)+r_{1}\hat{\mu}_{2}\hat{F}_{1,1}(1)+\frac{\hat{\mu}_{2}}{1-\mu_{1}}\hat{F}_{1,1}(1)f_{1}(1)+\hat{f}_{1}^{(2)}(1,2)+\hat{\mu}_{1}\hat{\mu}_{2}\left(\frac{1}{1-\mu_{1}}\right)^{2}f_{1}(1,1),\\
f_{2}\left(4,4\right)&=&\hat{\mu}_{2}R_{1}^{(2)}\left(1\right)+r_{1}\hat{P}_{2}^{(2)}\left(1\right)+2r_{1}\hat{\mu}_{2}\hat{F}_{1}^{(0,1)}+\hat{f}_{1,2}^{(2)}(1)+2r_{1}\frac{\hat{\mu}_{2}^{2}}{1-\mu_{1}}f_{1}(1)+\hat{\mu}_{2}^{2}\theta_{1}^{(2)}\left(1\right)f_{1}(1)\\
&+&\frac{1}{1-\mu_{1}}\hat{P}_{2}^{(2)}\left(1\right)f_{1}(1) +
2\frac{\hat{\mu}_{2}}{1-\mu_{1}}f_{1}(1)\hat{F}_{1,2}(1)+\left(\frac{\hat{\mu}_{2}}{1-\mu_{1}}\right)^{2}f_{1}(1,1),\\
\hat{f}_{1}\left(1,1\right)&=&\hat{r}_{2}P_{1}^{(2)}\left(1\right)+
\mu_{1}^{2}\hat{R}_{2}^{(2)}\left(1\right)+
2\hat{r}_{2}\frac{\mu_{1}^{2}}{1-\hat{\mu}_{2}}\hat{f}_{2}(2)+
\frac{1}{1-\hat{\mu}_{2}}P_{1}^{(2)}\left(1\right)\hat{f}_{2}(2)+
\mu_{1}^{2}\hat{\theta}_{2}^{(2)}\left(1\right)\hat{f}_{2}(2)\\
&+&\left(\frac{\mu_{1}^{2}}{1-\hat{\mu}_{2}}\right)^{2}\hat{f}_{2}(2,2)+2\hat{r}_{2}\mu_{1}F_{2,1}(1)+2\frac{\mu_{1}}{1-\hat{\mu}_{2}}\hat{f}_{2}(2)F_{2,1}(1)+F_{2,1}^{(2)}(1),\\
\hat{f}_{1}\left(2,1\right)&=&\hat{r}_{2}\mu_{1}\tilde{\mu}_{2}+\mu_{1}\tilde{\mu}_{2}\hat{R}_{2}^{(2)}\left(1\right)+\hat{r}_{2}\mu_{1}F_{2,2}(1)+
\frac{\mu_{1}\tilde{\mu}_{2}}{1-\hat{\mu}_{2}}\hat{f}_{2}(2)+2\hat{r}_{2}\frac{\mu_{1}\tilde{\mu}_{2}}{1-\hat{\mu}_{2}}\hat{f}_{2}(2)\\
&+&\mu_{1}\tilde{\mu}_{2}\hat{\theta}_{2}^{(2)}\left(1\right)\hat{f}_{2}(2)+\frac{\mu_{1}}{1-\hat{\mu}_{2}}F_{2,2}(1)\hat{f}_{2}(2)+\mu_{1} \tilde{\mu}_{2}\left(\frac{1}{1-\hat{\mu}_{2}}\right)^{2}\hat{f}_{2}(2,2)+\hat{r}_{2}\tilde{\mu}_{2}F_{2,1}(1)\\
&+&\frac{\tilde{\mu}_{2}}{1-\hat{\mu}_{2}}\hat{f}_{2}(2)F_{2,1}(1)+f_{2,1}^{(2)}(1),\\
\hat{f}_{1}\left(3,1\right)&=&\hat{r}_{2}\mu_{1}\hat{\mu}_{1}+\mu_{1}\hat{\mu}_{1}\hat{R}_{2}^{(2)}\left(1\right)+\hat{r}_{2}\frac{\mu_{1}\hat{\mu}_{1}}{1-\hat{\mu}_{2}}\hat{f}_{2}(2)+\hat{r}_{2}\hat{\mu}_{1}F_{2,1}(1)+\hat{r}_{2}\mu_{1}\hat{f}_{2}(1)\\
&+&F_{2,1}(1)\hat{f}_{2}(1)+\frac{\mu_{1}}{1-\hat{\mu}_{2}}\hat{f}_{2}(1,2),\\
\hat{f}_{1}\left(4,1\right)&=&\hat{r}_{2}\mu_{1}\hat{\mu}_{2}+\mu_{1}\hat{\mu}_{2}\hat{R}_{2}^{(2)}\left(1\right)+\frac{\mu_{1}\hat{\mu}_{2}}{1-\hat{\mu}_{2}}\hat{f}_{2}(2)+2\hat{r}_{2}\frac{\mu_{1}\hat{\mu}_{2}}{1-\hat{\mu}_{2}}\hat{f}_{2}(2)+\mu_{1}\hat{\mu}_{2}\hat{\theta}_{2}^{(2)}\left(1\right)\hat{f}_{2}(2)\\
&+&\mu_{1}\hat{\mu}_{2}\left(\frac{1}{1-\hat{\mu}_{2}}\right)^{2}\hat{f}_{2}(2,2)+\hat{r}_{2}\hat{\mu}_{2}F_{2,1}(1)+\frac{\hat{\mu}_{2}}{1-\hat{\mu}_{2}}\hat{f}_{2}(2)F_{2,1}(1),\\
\hat{f}_{1}\left(1,2\right)&=&\hat{r}_{2}\mu_{1}\tilde{\mu}_{2}+\mu_{1}\tilde{\mu}_{2}\hat{R}_{2}^{(2)}\left(1\right)+\mu_{1}\hat{r}_{2}F_{2,2}(1)+
\frac{\mu_{1}\tilde{\mu}_{2}}{1-\hat{\mu}_{2}}\hat{f}_{2}(2)+2\hat{r}_{2}\frac{\mu_{1}\tilde{\mu}_{2}}{1-\hat{\mu}_{2}}\hat{f}_{2}(2)\\
&+&\mu_{1}\tilde{\mu}_{2}\hat{\theta}_{2}^{(2)}\left(1\right)\hat{f}_{2}(2)+\frac{\mu_{1}}{1-\hat{\mu}_{2}}F_{2,2}(1)\hat{f}_{2}(2)+\mu_{1}\tilde{\mu}_{2}\left(\frac{1}{1-\hat{\mu}_{2}}\right)^{2}\hat{f}_{2}(2,2)\\
&+&\hat{r}_{2}\tilde{\mu}_{2}F_{2,1}(1)+\frac{\tilde{\mu}_{2}}{1-\hat{\mu}_{2}}\hat{f}_{2}(2)F_{2,1}(1)+f_{2}^{(2)}(1,2),\\
\hat{f}_{1}\left(2,2\right)&=&\hat{r}_{2}\tilde{P}_{2}^{(2)}\left(1\right)+\tilde{\mu}_{2}^{2}\hat{R}_{2}^{(2)}\left(1\right)+2\hat{r}_{2}\tilde{\mu}_{2}F_{2,2}(1)+2\hat{r}_{2}\frac{\tilde{\mu}_{2}^{2}}{1-\hat{\mu}_{2}}\hat{f}_{2}(2)+f_{2,2}^{(2)}(1)\\
&+&\frac{1}{1-\hat{\mu}_{2}}\tilde{P}_{2}^{(2)}\left(1\right)\hat{f}_{2}(2)+\tilde{\mu}_{2}^{2}\hat{\theta}_{2}^{(2)}\left(1\right)\hat{f}_{2}(2)+2\frac{\tilde{\mu}_{2}}{1-\hat{\mu}_{2}}F_{2,2}(1)\hat{f}_{2}(2)+\left(\frac{\tilde{\mu}_{2}}{1-\hat{\mu}_{2}}\right)^{2}\hat{f}_{2}(2,2),\\
\hat{f}_{1}\left(3,2\right)&=&\hat{r}_{2}\tilde{\mu}_{2}\hat{\mu}_{1}+\tilde{\mu}_{2}\hat{\mu}_{1}\hat{R}_{2}^{(2)}\left(1\right)+\hat{r}_{2}\hat{\mu}_{1}F_{2,2}(1)+\hat{r}_{2}\frac{\tilde{\mu}_{2}\hat{\mu}_{1}}{1-\hat{\mu}_{2}}\hat{f}_{2}(2)+\hat{r}_{2}\tilde{\mu}_{2}\hat{f}_{2}(1)+F_{2,2}(1)\hat{f}_{2}(1)\\
&+&\frac{\tilde{\mu}_{2}}{1-\hat{\mu}_{2}}\hat{f}_{2}(1,2),\\
\hat{f}_{1}\left(4,2\right)&=&\hat{r}_{2}\tilde{\mu}_{2}\hat{\mu}_{2}+\tilde{\mu}_{2}\hat{\mu}_{2}\hat{R}_{2}^{(2)}\left(1\right)+\hat{r}_{2}\hat{\mu}_{2}F_{2,2}(1)+
\frac{\tilde{\mu}_{2}\hat{\mu}_{2}}{1-\hat{\mu}_{2}}\hat{f}_{2}(2)+2\hat{r}_{2}\frac{\tilde{\mu}_{2}\hat{\mu}_{2}}{1-\hat{\mu}_{2}}\hat{f}_{2}(2)\\
&+&\tilde{\mu}_{2}\hat{\mu}_{2}\hat{\theta}_{2}^{(2)}\left(1\right)\hat{f}_{2}(2)+\frac{\hat{\mu}_{2}}{1-\hat{\mu}_{2}}F_{2,2}(1)\hat{f}_{2}(1)+\tilde{\mu}_{2}\hat{\mu}_{2}\left(\frac{1}{1-\hat{\mu}_{2}}\right)\hat{f}_{2}(2,2),\\
\end{eqnarray*}
\begin{eqnarray*}
\hat{f}_{1}\left(1,3\right)&=&\hat{r}_{2}\mu_{1}\hat{\mu}_{1}+\mu_{1}\hat{\mu}_{1}\hat{R}_{2}^{(2)}\left(1\right)+\hat{r}_{2}\frac{\mu_{1}\hat{\mu}_{1}}{1-\hat{\mu}_{2}}\hat{f}_{2}(2)+\hat{r}_{2}\hat{\mu}_{1}F_{2,1}(1)+\hat{r}_{2}\mu_{1}\hat{f}_{2}(1)\\
&+&F_{2,1}(1)\hat{f}_{2}(1)+\frac{\mu_{1}}{1-\hat{\mu}_{2}}\hat{f}_{2}(1,2),\\
\hat{f}_{1}\left(2,3\right)&=&\hat{r}_{2}\tilde{\mu}_{2}\hat{\mu}_{1}+\tilde{\mu}_{2}\hat{\mu}_{1}\hat{R}_{2}^{(2)}\left(1\right)+\hat{r}_{2}\hat{\mu}_{1}F_{2,2}(1)+\hat{r}_{2}\frac{\tilde{\mu}_{2}\hat{\mu}_{1}}{1-\hat{\mu}_{2}}\hat{f}_{2}(2)+\hat{r}_{2}\tilde{\mu}_{2}\hat{f}_{2}(1)\\
&+&F_{2,2}(1)\hat{f}_{2}(1)+\frac{\tilde{\mu}_{2}}{1-\hat{\mu}_{2}}\hat{f}_{2}(1,2),\\
\hat{f}_{1}\left(3,3\right)&=&\hat{r}_{2}\hat{P}_{1}^{(2)}\left(1\right)+\hat{\mu}_{1}^{2}\hat{R}_{2}^{(2)}\left(1\right)+2\hat{r}_{2}\hat{\mu}_{1}\hat{f}_{2}(1)+\hat{f}_{2}(1,1),\\
\hat{f}_{1}\left(4,3\right)&=&\hat{r}_{2}\hat{\mu}_{1}\hat{\mu}_{2}+\hat{\mu}_{1}\hat{\mu}_{2}\hat{R}_{2}^{(2)}\left(1\right)+
\hat{r}_{2}\frac{\hat{\mu}_{2}\hat{\mu}_{1}}{1-\hat{\mu}_{2}}\hat{f}_{2}(2)+\hat{r}_{2}\hat{\mu}_{2}\hat{f}_{2}(1)+\frac{\hat{\mu}_{2}}{1-\hat{\mu}_{2}}\hat{f}_{2}(1,2),\\
\hat{f}_{1}\left(1,4\right)&=&\hat{r}_{2}\mu_{1}\hat{\mu}_{2}+\mu_{1}\hat{\mu}_{2}\hat{R}_{2}^{(2)}\left(1\right)+
\frac{\mu_{1}\hat{\mu}_{2}}{1-\hat{\mu}_{2}}\hat{f}_{2}(2) +2\hat{r}_{2}\frac{\mu_{1}\hat{\mu}_{2}}{1-\hat{\mu}_{2}}\hat{f}_{2}(2)\\
&+&\mu_{1}\hat{\mu}_{2}\hat{\theta}_{2}^{(2)}\left(1\right)\hat{f}_{2}(2)+\mu_{1}\hat{\mu}_{2}\left(\frac{1}{1-\hat{\mu}_{2}}\right)^{2}\hat{f}_{2}(2,2)+\hat{r}_{2}\hat{\mu}_{2}F_{2,1}(1)+\frac{\hat{\mu}_{2}}{1-\hat{\mu}_{2}}\hat{f}_{2}(2)F_{2,1}(1),\\\hat{f}_{1}\left(2,4\right)&=&\hat{r}_{2}\tilde{\mu}_{2}\hat{\mu}_{2}+\tilde{\mu}_{2}\hat{\mu}_{2}\hat{R}_{2}^{(2)}\left(1\right)+\hat{r}_{2}\hat{\mu}_{2}F_{2,2}(1)+\frac{\tilde{\mu}_{2}\hat{\mu}_{2}}{1-\hat{\mu}_{2}}\hat{f}_{2}(2)+2\hat{r}_{2}\frac{\tilde{\mu}_{2}\hat{\mu}_{2}}{1-\hat{\mu}_{2}}\hat{f}_{2}(2)\\
&+&\tilde{\mu}_{2}\hat{\mu}_{2}\hat{\theta}_{2}^{(2)}\left(1\right)\hat{f}_{2}(2)+\frac{\hat{\mu}_{2}}{1-\hat{\mu}_{2}}\hat{f}_{2}(2)F_{2,2}(1)+\tilde{\mu}_{2}\hat{\mu}_{2}\left(\frac{1}{1-\hat{\mu}_{2}}\right)^{2}\hat{f}_{2}(2,2),\\
\hat{f}_{1}\left(3,4\right)&=&\hat{r}_{2}\hat{\mu}_{1}\hat{\mu}_{2}+\hat{\mu}_{1}\hat{\mu}_{2}\hat{R}_{2}^{(2)}\left(1\right)+
\hat{r}_{2}\frac{\hat{\mu}_{1}\hat{\mu}_{2}}{1-\hat{\mu}_{2}}\hat{f}_{2}(2)+
\hat{r}_{2}\hat{\mu}_{2}\hat{f}_{2}(1)+\frac{\hat{\mu}_{2}}{1-\hat{\mu}_{2}}\hat{f}_{2}(1,2),\\
\hat{f}_{1}\left(4,4\right)&=&\hat{r}_{2}P_{2}^{(2)}\left(1\right)+\hat{\mu}_{2}^{2}\hat{R}_{2}^{(2)}\left(1\right)+2\hat{r}_{2}\frac{\hat{\mu}_{2}^{2}}{1-\hat{\mu}_{2}}\hat{f}_{2}(2)+\frac{1}{1-\hat{\mu}_{2}}\hat{P}_{2}^{(2)}\left(1\right)\hat{f}_{2}(2)\\
&+&\hat{\mu}_{2}^{2}\hat{\theta}_{2}^{(2)}\left(1\right)\hat{f}_{2}(2)+\left(\frac{\hat{\mu}_{2}}{1-\hat{\mu}_{2}}\right)^{2}\hat{f}_{2}(2,2),\\
\hat{f}_{2}\left(,1\right)&=&\hat{r}_{1}P_{1}^{(2)}\left(1\right)+
\mu_{1}^{2}\hat{R}_{1}^{(2)}\left(1\right)+2\hat{r}_{1}\mu_{1}F_{1,1}(1)+
2\hat{r}_{1}\frac{\mu_{1}^{2}}{1-\hat{\mu}_{1}}\hat{f}_{1}(1)+\frac{1}{1-\hat{\mu}_{1}}P_{1}^{(2)}\left(1\right)\hat{f}_{1}(1)\\
&+&\mu_{1}^{2}\hat{\theta}_{1}^{(2)}\left(1\right)\hat{f}_{1}(1)+2\frac{\mu_{1}}{1-\hat{\mu}_{1}}\hat{f}_{1}^(1)F_{1,1}(1)+f_{1,1}^{(2)}(1)+\left(\frac{\mu_{1}}{1-\hat{\mu}_{1}}\right)^{2}\hat{f}_{1}^{(1,1)},\\
\hat{f}_{2}\left(2,1\right)&=&\hat{r}_{1}\mu_{1}\tilde{\mu}_{2}+\mu_{1}\tilde{\mu}_{2}\hat{R}_{1}^{(2)}\left(1\right)+
\hat{r}_{1}\mu_{1}F_{1,2}(1)+\tilde{\mu}_{2}\hat{r}_{1}F_{1,1}(1)+
\frac{\mu_{1}\tilde{\mu}_{2}}{1-\hat{\mu}_{1}}\hat{f}_{1}(1)\\
&+&2\hat{r}_{1}\frac{\mu_{1}\tilde{\mu}_{2}}{1-\hat{\mu}_{1}}\hat{f}_{1}(1)+\mu_{1}\tilde{\mu}_{2}\hat{\theta}_{1}^{(2)}\left(1\right)\hat{f}_{1}(1)+
\frac{\mu_{1}}{1-\hat{\mu}_{1}}\hat{f}_{1}(1)F_{1,2}(1)+\frac{\tilde{\mu}_{2}}{1-\hat{\mu}_{1}}\hat{f}_{1}(1)F_{1,1}(1)\\
&+&f_{1}^{(2)}(1,2)+\mu_{1}\tilde{\mu}_{2}\left(\frac{1}{1-\hat{\mu}_{1}}\right)^{2}\hat{f}_{1}(1,1),\\
\hat{f}_{2}\left(3,1\right)&=&\hat{r}_{1}\mu_{1}\hat{\mu}_{1}+\mu_{1}\hat{\mu}_{1}\hat{R}_{1}^{(2)}\left(1\right)+\hat{r}_{1}\hat{\mu}_{1}F_{1,1}(1)+\hat{r}_{1}\frac{\mu_{1}\hat{\mu}_{1}}{1-\hat{\mu}_{1}}\hat{F}_{1}(1),\\
\hat{f}_{2}\left(4,1\right)&=&\hat{r}_{1}\mu_{1}\hat{\mu}_{2}+\mu_{1}\hat{\mu}_{2}\hat{R}_{1}^{(2)}\left(1\right)+\hat{r}_{1}\hat{\mu}_{2}F_{1,1}(1)+\frac{\mu_{1}\hat{\mu}_{2}}{1-\hat{\mu}_{1}}\hat{f}_{1}(1)+\hat{r}_{1}\frac{\mu_{1}\hat{\mu}_{2}}{1-\hat{\mu}_{1}}\hat{f}_{1}(1)\\
&+&\mu_{1}\hat{\mu}_{2}\hat{\theta}_{1}^{(2)}\left(1\right)\hat{f}_{1}(1)+\hat{r}_{1}\mu_{1}\left(\hat{f}_{1}(2)+\frac{\hat{\mu}_{2}}{1-\hat{\mu}_{1}}\hat{f}_{1}(1)\right)+F_{1,1}(1)\left(\hat{f}_{1}(2)+\frac{\hat{\mu}_{2}}{1-\hat{\mu}_{1}}\hat{f}_{1}(1)\right)\\
&+&\frac{\mu_{1}}{1-\hat{\mu}_{1}}\left(\hat{f}_{1}(1,2)+\frac{\hat{\mu}_{2}}{1-\hat{\mu}_{1}}\hat{f}_{1}(1,1)\right),\\
\hat{f}_{2}\left(1,2\right)&=&\hat{r}_{1}\mu_{1}\tilde{\mu}_{2}+\mu_{1}\tilde{\mu}_{2}\hat{R}_{1}^{(2)}\left(1\right)+\hat{r}_{1}\mu_{1}F_{1,2}(1)+\hat{r}_{1}\tilde{\mu}_{2}F_{1,1}(1)+\frac{\mu_{1}\tilde{\mu}_{2}}{1-\hat{\mu}_{1}}\hat{f}_{1}(1)\\
&+&2\hat{r}_{1}\frac{\mu_{1}\tilde{\mu}_{2}}{1-\hat{\mu}_{1}}\hat{f}_{1}(1)+\mu_{1}\tilde{\mu}_{2}\hat{\theta}_{1}^{(2)}\left(1\right)\hat{f}_{1}(1)+\frac{\mu_{1}}{1-\hat{\mu}_{1}}\hat{f}_{1}(1)F_{1,2}(1)\\
&+&\frac{\tilde{\mu}_{2}}{1-\hat{\mu}_{1}}\hat{f}_{1}(1)F_{1,1}(1)+f_{1}^{(2)}(1,2)+\mu_{1}\tilde{\mu}_{2}\left(\frac{1}{1-\hat{\mu}_{1}}\right)^{2}\hat{f}_{1}(1,1),\\
\end{eqnarray*}
\begin{eqnarray*}
\hat{f}_{2}\left(2,2\right)&=&\hat{r}_{1}\tilde{P}_{2}^{(2)}\left(1\right)+\tilde{\mu}_{2}^{2}\hat{R}_{1}^{(2)}\left(1\right)+2\hat{r}_{1}\tilde{\mu}_{2}F_{1,2}(1)+ f_{1,2}^{(2)}(1)+2\hat{r}_{1}\frac{\tilde{\mu}_{2}^{2}}{1-\hat{\mu}_{1}}\hat{f}_{1}(1)\\
&+&\frac{1}{1-\hat{\mu}_{1}}\tilde{P}_{2}^{(2)}\left(1\right)\hat{f}_{1}(1)+\tilde{\mu}_{2}^{2}\hat{\theta}_{1}^{(2)}\left(1\right)\hat{f}_{1}(1)+2\frac{\tilde{\mu}_{2}}{1-\hat{\mu}_{1}}F_{1,2}(1)\hat{f}_{1}(1)+\left(\frac{\tilde{\mu}_{2}}{1-\hat{\mu}_{1}}\right)^{2}\hat{f}_{1}(1,1),\\
\hat{f}_{2}\left(3,2\right)&=&\hat{r}_{1}\hat{\mu}_{1}\tilde{\mu}_{2}+\hat{\mu}_{1}\tilde{\mu}_{2}\hat{R}_{1}^{(2)}\left(1\right)+
\hat{r}_{1}\hat{\mu}_{1}F_{1,2}(1)+\hat{r}_{1}\frac{\hat{\mu}_{1}\tilde{\mu}_{2}}{1-\hat{\mu}_{1}}\hat{f}_{1}(1),\\
\hat{f}_{2}\left(4,2\right)&=&\hat{r}_{1}\tilde{\mu}_{2}\hat{\mu}_{2}+\hat{\mu}_{2}\tilde{\mu}_{2}\hat{R}_{1}^{(2)}\left(1\right)+\hat{\mu}_{2}\hat{R}_{1}^{(2)}\left(1\right)F_{1,2}(1)+\frac{\hat{\mu}_{2}\tilde{\mu}_{2}}{1-\hat{\mu}_{1}}\hat{f}_{1}(1)\\
&+&\hat{r}_{1}\frac{\hat{\mu}_{2}\tilde{\mu}_{2}}{1-\hat{\mu}_{1}}\hat{f}_{1}(1)+\hat{\mu}_{2}\tilde{\mu}_{2}\hat{\theta}_{1}^{(2)}\left(1\right)\hat{f}_{1}(1)+\hat{r}_{1}\tilde{\mu}_{2}\left(\hat{f}_{1}(2)+\frac{\hat{\mu}_{2}}{1-\hat{\mu}_{1}}\hat{f}_{1}(1)\right)\\
&+&F_{1,2}(1)\left(\hat{f}_{1}(2)+\frac{\hat{\mu}_{2}}{1-\hat{\mu}_{1}}\hat{f}_{1}(1)\right)+\frac{\tilde{\mu}_{2}}{1-\hat{\mu}_{1}}\left(\hat{f}_{1}(1,2)+\frac{\hat{\mu}_{2}}{1-\hat{\mu}_{1}}\hat{f}_{1}(1,1)\right),\\
\hat{f}_{2}\left(1,3\right)&=&\hat{r}_{1}\mu_{1}\hat{\mu}_{1}+\mu_{1}\hat{\mu}_{1}\hat{R}_{1}^{(2)}\left(1\right)+\hat{r}_{1}\hat{\mu}_{1}F_{1,1}(1)+\hat{r}_{1}\frac{\mu_{1}\hat{\mu}_{1}}{1-\hat{\mu}_{1}}\hat{f}_{1}(1),\\
\hat{f}_{2}\left(2,3\right)&=&\hat{r}_{1}\tilde{\mu}_{2}\hat{\mu}_{1}+\tilde{\mu}_{2}\hat{\mu}_{1}\hat{R}_{1}^{(2)}\left(1\right)+\hat{r}_{1}\hat{\mu}_{1}F_{1,2}(1)+\hat{r}_{1}\frac{\tilde{\mu}_{2}\hat{\mu}_{1}}{1-\hat{\mu}_{1}}\hat{f}_{1}(1),\\
\hat{f}_{2}\left(3,3\right)&=&\hat{r}_{1}\hat{P}_{1}^{(2)}\left(1\right)+\hat{\mu}_{1}^{2}\hat{R}_{1}^{(2)}\left(1\right),\\
\hat{f}_{2}\left(4,3\right)&=&\hat{r}_{1}\hat{\mu}_{2}\hat{\mu}_{1}+\hat{\mu}_{2}\hat{\mu}_{1}\hat{R}_{1}^{(2)}\left(1\right)+\hat{r}_{1}\hat{\mu}_{1}\left(\hat{f}_{1}(2)+\frac{\hat{\mu}_{2}}{1-\hat{\mu}_{1}}\hat{f}_{1}(1)\right),\\
\hat{f}_{2}\left(1,4\right)&=&\hat{r}_{1}\mu_{1}\hat{\mu}_{2}+\mu_{1}\hat{\mu}_{2}\hat{R}_{1}^{(2)}\left(1\right)+\hat{r}_{1}\hat{\mu}_{2}F_{1,1}(1)+\hat{r}_{1}\frac{\mu_{1}\hat{\mu}_{2}}{1-\hat{\mu}_{1}}\hat{f}_{1}(1)+\hat{r}_{1}\mu_{1}\left(\hat{f}_{1}(2)+\frac{\hat{\mu}_{2}}{1-\hat{\mu}_{1}}\hat{f}_{1}(1)\right)\\
&+&F_{1,1}(1)\left(\hat{f}_{1}(2)+\frac{\hat{\mu}_{2}}{1-\hat{\mu}_{1}}\hat{f}_{1}(1)\right)+\frac{\mu_{1}\hat{\mu}_{2}}{1-\hat{\mu}_{1}}\hat{f}_{1}(1)+\mu_{1}\hat{\mu}_{2}\hat{\theta}_{1}^{(2)}\left(1\right)\hat{f}_{1}(1)\\
&+&\frac{\mu_{1}}{1-\hat{\mu}_{1}}\hat{f}_{1}(1,2)+\mu_{1}\hat{\mu}_{2}\left(\frac{1}{1-\hat{\mu}_{1}}\right)^{2}\hat{f}_{1}(1,1),\\
\hat{f}_{2}\left(2,4\right)&=&\hat{r}_{1}\tilde{\mu}_{2}\hat{\mu}_{2}+\tilde{\mu}_{2}\hat{\mu}_{2}\hat{R}_{1}^{(2)}\left(1\right)+\hat{r}_{1}\hat{\mu}_{2}F_{1,2}(1)+\hat{r}_{1}\frac{\tilde{\mu}_{2}\hat{\mu}_{2}}{1-\hat{\mu}_{1}}\hat{f}_{1}(1)\\
&+&\hat{r}_{1}\tilde{\mu}_{2}\left(\hat{f}_{1}(2)+\frac{\hat{\mu}_{2}}{1-\hat{\mu}_{1}}\hat{f}_{1}(1)\right)+F_{1,2}(1)\left(\hat{f}_{1}(2)+\frac{\hat{\mu}_{2}}{1-\hat{\mu}_{1}}\hat{F}_{1}^{(1,0)}\right)+\frac{\tilde{\mu}_{2}\hat{\mu}_{2}}{1-\hat{\mu}_{1}}\hat{f}_{1}(1)\\
&+&\tilde{\mu}_{2}\hat{\mu}_{2}\hat{\theta}_{1}^{(2)}\left(1\right)\hat{f}_{1}(1)+\frac{\tilde{\mu}_{2}}{1-\hat{\mu}_{1}}\hat{f}_{1}(1,2)+\tilde{\mu}_{2}\hat{\mu}_{2}\left(\frac{1}{1-\hat{\mu}_{1}}\right)^{2}\hat{f}_{1}(1,1),\\
\hat{f}_{2}\left(3,4\right)&=&\hat{r}_{1}\hat{\mu}_{2}\hat{\mu}_{1}+\hat{\mu}_{2}\hat{\mu}_{1}\hat{R}_{1}^{(2)}\left(1\right)+\hat{r}_{1}\hat{\mu}_{1}\left(\hat{f}_{1}(2)+\frac{\hat{\mu}_{2}}{1-\hat{\mu}_{1}}\hat{f}_{1}(1)\right),\\
\hat{f}_{2}\left(4,4\right)&=&\hat{r}_{1}\hat{P}_{2}^{(2)}\left(1\right)+\hat{\mu}_{2}^{2}\hat{R}_{1}^{(2)}\left(1\right)+
2\hat{r}_{1}\hat{\mu}_{2}\left(\hat{f}_{1}(2)+\frac{\hat{\mu}_{2}}{1-\hat{\mu}_{1}}\hat{f}_{1}(1)\right)+\hat{f}_{1}(2,2)\\
&+&\frac{1}{1-\hat{\mu}_{1}}\hat{P}_{2}^{(2)}\left(1\right)\hat{f}_{1}(1)+\hat{\mu}_{2}^{2}\hat{\theta}_{1}^{(2)}\left(1\right)\hat{f}_{1}(1)+\frac{\hat{\mu}_{2}}{1-\hat{\mu}_{1}}\hat{f}_{1}(1,2)\\
&+&\frac{\hat{\mu}_{2}}{1-\hat{\mu}_{1}}\left(\hat{f}_{1}(1,2)+\frac{\hat{\mu}_{2}}{1-\hat{\mu}_{1}}\hat{f}_{1}(1,1)\right).
\end{eqnarray*}
%_________________________________________________________________________________________________________
\section{Medidas de Desempe\~no}
%_________________________________________________________________________________________________________

\begin{Def}
Sea $L_{i}^{*}$el n\'umero de usuarios cuando el servidor visita la cola $Q_{i}$ para dar servicio, para $i=1,2$.
\end{Def}

Entonces
\begin{Prop} Para la cola $Q_{i}$, $i=1,2$, se tiene que el n\'umero de usuarios presentes al momento de ser visitada por el servidor est\'a dado por
\begin{eqnarray}
\esp\left[L_{i}^{*}\right]&=&f_{i}\left(i\right)\\
Var\left[L_{i}^{*}\right]&=&f_{i}\left(i,i\right)+\esp\left[L_{i}^{*}\right]-\esp\left[L_{i}^{*}\right]^{2}.
\end{eqnarray}
\end{Prop}


\begin{Def}
El tiempo de Ciclo $C_{i}$ es el periodo de tiempo que comienza
cuando la cola $i$ es visitada por primera vez en un ciclo, y
termina cuando es visitado nuevamente en el pr\'oximo ciclo, bajo condiciones de estabilidad.

\begin{eqnarray*}
C_{i}\left(z\right)=\esp\left[z^{\overline{\tau}_{i}\left(m+1\right)-\overline{\tau}_{i}\left(m\right)}\right]
\end{eqnarray*}
\end{Def}

\begin{Def}
El tiempo de intervisita $I_{i}$ es el periodo de tiempo que
comienza cuando se ha completado el servicio en un ciclo y termina
cuando es visitada nuevamente en el pr\'oximo ciclo.
\begin{eqnarray*}I_{i}\left(z\right)&=&\esp\left[z^{\tau_{i}\left(m+1\right)-\overline{\tau}_{i}\left(m\right)}\right]\end{eqnarray*}
\end{Def}

\begin{Prop}
Para los tiempos de intervisita del servidor $I_{i}$, se tiene que

\begin{eqnarray*}
\esp\left[I_{i}\right]&=&\frac{f_{i}\left(i\right)}{\mu_{i}},\\
Var\left[I_{i}\right]&=&\frac{Var\left[L_{i}^{*}\right]}{\mu_{i}^{2}}-\frac{\sigma_{i}^{2}}{\mu_{i}^{2}}f_{i}\left(i\right).
\end{eqnarray*}
\end{Prop}


\begin{Prop}
Para los tiempos que ocupa el servidor para atender a los usuarios presentes en la cola $Q_{i}$, con FGP denotada por $S_{i}$, se tiene que
\begin{eqnarray*}
\esp\left[S_{i}\right]&=&\frac{\esp\left[L_{i}^{*}\right]}{1-\mu_{i}}=\frac{f_{i}\left(i\right)}{1-\mu_{i}},\\
Var\left[S_{i}\right]&=&\frac{Var\left[L_{i}^{*}\right]}{\left(1-\mu_{i}\right)^{2}}+\frac{\sigma^{2}\esp\left[L_{i}^{*}\right]}{\left(1-\mu_{i}\right)^{3}}
\end{eqnarray*}
\end{Prop}


\begin{Prop}
Para la duraci\'on de los ciclos $C_{i}$ se tiene que
\begin{eqnarray*}
\esp\left[C_{i}\right]&=&\esp\left[I_{i}\right]\esp\left[\theta_{i}\left(z\right)\right]=\frac{\esp\left[L_{i}^{*}\right]}{\mu_{i}}\frac{1}{1-\mu_{i}}=\frac{f_{i}\left(i\right)}{\mu_{i}\left(1-\mu_{i}\right)}\\
Var\left[C_{i}\right]&=&\frac{Var\left[L_{i}^{*}\right]}{\mu_{i}^{2}\left(1-\mu_{i}\right)^{2}}.
\end{eqnarray*}

\end{Prop}

%___________________________________________________________________________________________
%
\section*{Ap\'endice A}\label{Segundos.Momentos}
%___________________________________________________________________________________________


%___________________________________________________________________________________________

%\subsubsection{Mixtas para $z_{1}$:}
%___________________________________________________________________________________________
\begin{enumerate}

%1/1/1
\item \begin{eqnarray*}
&&\frac{\partial}{\partial z_1}\frac{\partial}{\partial z_1}\left(R_2\left(P_1\left(z_1\right)\bar{P}_2\left(z_2\right)\hat{P}_1\left(w_1\right)\hat{P}_2\left(w_2\right)\right)F_2\left(z_1,\theta
_2\left(P_1\left(z_1\right)\hat{P}_1\left(w_1\right)\hat{P}_2\left(w_2\right)\right)\right)\hat{F}_2\left(w_1,w_2\right)\right)\\
&=&r_{2}P_{1}^{(2)}\left(1\right)+\mu_{1}^{2}R_{2}^{(2)}\left(1\right)+2\mu_{1}r_{2}\left(\frac{\mu_{1}}{1-\tilde{\mu}_{2}}F_{2}^{(0,1)}+F_{2}^{1,0)}\right)+\frac{1}{1-\tilde{\mu}_{2}}P_{1}^{(2)}F_{2}^{(0,1)}+\mu_{1}^{2}\tilde{\theta}_{2}^{(2)}\left(1\right)F_{2}^{(0,1)}\\
&+&\frac{\mu_{1}}{1-\tilde{\mu}_{2}}F_{2}^{(1,1)}+\frac{\mu_{1}}{1-\tilde{\mu}_{2}}\left(\frac{\mu_{1}}{1-\tilde{\mu}_{2}}F_{2}^{(0,2)}+F_{2}^{(1,1)}\right)+F_{2}^{(2,0)}.
\end{eqnarray*}

%2/2/1

\item \begin{eqnarray*}
&&\frac{\partial}{\partial z_2}\frac{\partial}{\partial z_1}\left(R_2\left(P_1\left(z_1\right)\bar{P}_2\left(z_2\right)\hat{P}_1\left(w_1\right)\hat{P}_2\left(w_2\right)\right)F_2\left(z_1,\theta
_2\left(P_1\left(z_1\right)\hat{P}_1\left(w_1\right)\hat{P}_2\left(w_2\right)\right)\right)\hat{F}_2\left(w_1,w_2\right)\right)\\
&=&\mu_{1}r_{2}\tilde{\mu}_{2}+\mu_{1}\tilde{\mu}_{2}R_{2}^{(2)}\left(1\right)+r_{2}\tilde{\mu}_{2}\left(\frac{\mu_{1}}{1-\tilde{\mu}_{2}}F_{2}^{(0,1)}+F_{2}^{(1,0)}\right).
\end{eqnarray*}
%3/3/1
\item \begin{eqnarray*}
&&\frac{\partial}{\partial w_1}\frac{\partial}{\partial z_1}\left(R_2\left(P_1\left(z_1\right)\bar{P}_2\left(z_2\right)\hat{P}_1\left(w_1\right)\hat{P}_2\left(w_2\right)\right)F_2\left(z_1,\theta
_2\left(P_1\left(z_1\right)\hat{P}_1\left(w_1\right)\hat{P}_2\left(w_2\right)\right)\right)\hat{F}_2\left(w_1,w_2\right)\right)\\
&=&\mu_{1}\hat{\mu}_{1}r_{2}+\mu_{1}\hat{\mu}_{1}R_{2}^{(2)}\left(1\right)+r_{2}\frac{\mu_{1}}{1-\tilde{\mu}_{2}}F_{2}^{(0,1)}+r_{2}\hat{\mu}_{1}\left(\frac{\mu_{1}}{1-\tilde{\mu}_{2}}F_{2}^{(0,1)}+F_{2}^{(1,0)}\right)+\mu_{1}r_{2}\hat{F}_{2}^{(1,0)}\\
&+&\left(\frac{\mu_{1}}{1-\tilde{\mu}_{2}}F_{2}^{(0,1)}+F_{2}^{(1,0)}\right)\hat{F}_{2}^{(1,0)}+\frac{\mu_{1}\hat{\mu}_{1}}{1-\tilde{\mu}_{2}}F_{2}^{(0,1)}+\mu_{1}\hat{\mu}_{1}\tilde{\theta}_{2}^{(2)}\left(1\right)F_{2}^{(0,1)}\\
&+&\mu_{1}\hat{\mu}_{1}\left(\frac{1}{1-\tilde{\mu}_{2}}\right)^{2}F_{2}^{(0,2)}+\frac{\hat{\mu}_{1}}{1-\tilde{\mu}_{2}}F_{2}^{(1,1)}.
\end{eqnarray*}
%4/4/1
\item \begin{eqnarray*}
&&\frac{\partial}{\partial w_2}\frac{\partial}{\partial z_1}\left(R_2\left(P_1\left(z_1\right)\bar{P}_2\left(z_2\right)\hat{P}_1\left(w_1\right)\hat{P}_2\left(w_2\right)\right)
F_2\left(z_1,\theta_2\left(P_1\left(z_1\right)\hat{P}_1\left(w_1\right)\hat{P}_2\left(w_2\right)\right)\right)\hat{F}_2\left(w_1,w_2\right)\right)\\
&=&\mu_{1}\hat{\mu}_{2}r_{2}+\mu_{1}\hat{\mu}_{2}R_{2}^{(2)}\left(1\right)+r_{2}\frac{\mu_{1}\hat{\mu}_{2}}{1-\tilde{\mu}_{2}}F_{2}^{(0,1)}+\mu_{1}r_{2}\hat{F}_{2}^{(0,1)}
+r_{2}\hat{\mu}_{2}\left(\frac{\mu_{1}}{1-\tilde{\mu}_{2}}F_{2}^{(0,1)}+F_{2}^{(1,0)}\right)\\
&+&\hat{F}_{2}^{(1,0)}\left(\frac{\mu_{1}}{1-\tilde{\mu}_{2}}F_{2}^{(0,1)}+F_{2}^{(1,0)}\right)+\frac{\mu_{1}\hat{\mu}_{2}}{1-\tilde{\mu}_{2}}F_{2}^{(0,1)}
+\mu_{1}\hat{\mu}_{2}\tilde{\theta}_{2}^{(2)}\left(1\right)F_{2}^{(0,1)}+\mu_{1}\hat{\mu}_{2}\left(\frac{1}{1-\tilde{\mu}_{2}}\right)^{2}F_{2}^{(0,2)}\\
&+&\frac{\hat{\mu}_{2}}{1-\tilde{\mu}_{2}}F_{2}^{(1,1)}.
\end{eqnarray*}
%___________________________________________________________________________________________
%\subsubsection{Mixtas para $z_{2}$:}
%___________________________________________________________________________________________
%5
\item \begin{eqnarray*} &&\frac{\partial}{\partial
z_1}\frac{\partial}{\partial
z_2}\left(R_2\left(P_1\left(z_1\right)\bar{P}_2\left(z_2\right)\hat{P}_1\left(w_1\right)\hat{P}_2\left(w_2\right)\right)
F_2\left(z_1,\theta_2\left(P_1\left(z_1\right)\hat{P}_1\left(w_1\right)\hat{P}_2\left(w_2\right)\right)\right)\hat{F}_2\left(w_1,w_2\right)\right)\\
&=&\mu_{1}\tilde{\mu}_{2}r_{2}+\mu_{1}\tilde{\mu}_{2}R_{2}^{(2)}\left(1\right)+r_{2}\tilde{\mu}_{2}\left(\frac{\mu_{1}}{1-\tilde{\mu}_{2}}F_{2}^{(0,1)}+F_{2}^{(1,0)}\right).
\end{eqnarray*}

%6

\item \begin{eqnarray*} &&\frac{\partial}{\partial
z_2}\frac{\partial}{\partial
z_2}\left(R_2\left(P_1\left(z_1\right)\bar{P}_2\left(z_2\right)\hat{P}_1\left(w_1\right)\hat{P}_2\left(w_2\right)\right)
F_2\left(z_1,\theta_2\left(P_1\left(z_1\right)\hat{P}_1\left(w_1\right)\hat{P}_2\left(w_2\right)\right)\right)\hat{F}_2\left(w_1,w_2\right)\right)\\
&=&\tilde{\mu}_{2}^{2}R_{2}^{(2)}(1)+r_{2}\tilde{P}_{2}^{(2)}\left(1\right).
\end{eqnarray*}

%7
\item \begin{eqnarray*} &&\frac{\partial}{\partial
w_1}\frac{\partial}{\partial
z_2}\left(R_2\left(P_1\left(z_1\right)\bar{P}_2\left(z_2\right)\hat{P}_1\left(w_1\right)\hat{P}_2\left(w_2\right)\right)
F_2\left(z_1,\theta_2\left(P_1\left(z_1\right)\hat{P}_1\left(w_1\right)\hat{P}_2\left(w_2\right)\right)\right)\hat{F}_2\left(w_1,w_2\right)\right)\\
&=&\hat{\mu}_{1}\tilde{\mu}_{2}r_{2}+\hat{\mu}_{1}\tilde{\mu}_{2}R_{2}^{(2)}(1)+
r_{2}\frac{\hat{\mu}_{1}\tilde{\mu}_{2}}{1-\tilde{\mu}_{2}}F_{2}^{(0,1)}+r_{2}\tilde{\mu}_{2}\hat{F}_{2}^{(1,0)}.
\end{eqnarray*}
%8
\item \begin{eqnarray*} &&\frac{\partial}{\partial
w_2}\frac{\partial}{\partial
z_2}\left(R_2\left(P_1\left(z_1\right)\bar{P}_2\left(z_2\right)\hat{P}_1\left(w_1\right)\hat{P}_2\left(w_2\right)\right)
F_2\left(z_1,\theta_2\left(P_1\left(z_1\right)\hat{P}_1\left(w_1\right)\hat{P}_2\left(w_2\right)\right)\right)\hat{F}_2\left(w_1,w_2\right)\right)\\
&=&\hat{\mu}_{2}\tilde{\mu}_{2}r_{2}+\hat{\mu}_{2}\tilde{\mu}_{2}R_{2}^{(2)}(1)+
r_{2}\frac{\hat{\mu}_{2}\tilde{\mu}_{2}}{1-\tilde{\mu}_{2}}F_{2}^{(0,1)}+r_{2}\tilde{\mu}_{2}\hat{F}_{2}^{(0,1)}.
\end{eqnarray*}
%___________________________________________________________________________________________
%\subsubsection{Mixtas para $w_{1}$:}
%___________________________________________________________________________________________

%9
\item \begin{eqnarray*} &&\frac{\partial}{\partial
z_1}\frac{\partial}{\partial
w_1}\left(R_2\left(P_1\left(z_1\right)\bar{P}_2\left(z_2\right)\hat{P}_1\left(w_1\right)\hat{P}_2\left(w_2\right)\right)
F_2\left(z_1,\theta_2\left(P_1\left(z_1\right)\hat{P}_1\left(w_1\right)\hat{P}_2\left(w_2\right)\right)\right)\hat{F}_2\left(w_1,w_2\right)\right)\\
&=&\mu_{1}\hat{\mu}_{1}r_{2}+\mu_{1}\hat{\mu}_{1}R_{2}^{(2)}\left(1\right)+\frac{\mu_{1}\hat{\mu}_{1}}{1-\tilde{\mu}_{2}}F_{2}^{(0,1)}+r_{2}\frac{\mu_{1}\hat{\mu}_{1}}{1-\tilde{\mu}_{2}}F_{2}^{(0,1)}+\mu_{1}\hat{\mu}_{1}\tilde{\theta}_{2}^{(2)}\left(1\right)F_{2}^{(0,1)}\\
&+&r_{2}\hat{\mu}_{1}\left(\frac{\mu_{1}}{1-\tilde{\mu}_{2}}F_{2}^{(0,1)}+F_{2}^{(1,0)}\right)+r_{2}\mu_{1}\hat{F}_{2}^{(1,0)}
+\left(\frac{\mu_{1}}{1-\tilde{\mu}_{2}}F_{2}^{(0,1)}+F_{2}^{(1,0)}\right)\hat{F}_{2}^{(1,0)}\\
&+&\frac{\hat{\mu}_{1}}{1-\tilde{\mu}_{2}}\left(\frac{\mu_{1}}{1-\tilde{\mu}_{2}}F_{2}^{(0,2)}+F_{2}^{(1,1)}\right).
\end{eqnarray*}
%10
\item \begin{eqnarray*} &&\frac{\partial}{\partial
z_2}\frac{\partial}{\partial
w_1}\left(R_2\left(P_1\left(z_1\right)\bar{P}_2\left(z_2\right)\hat{P}_1\left(w_1\right)\hat{P}_2\left(w_2\right)\right)
F_2\left(z_1,\theta_2\left(P_1\left(z_1\right)\hat{P}_1\left(w_1\right)\hat{P}_2\left(w_2\right)\right)\right)\hat{F}_2\left(w_1,w_2\right)\right)\\
&=&\tilde{\mu}_{2}\hat{\mu}_{1}r_{2}+\tilde{\mu}_{2}\hat{\mu}_{1}R_{2}^{(2)}\left(1\right)+r_{2}\frac{\tilde{\mu}_{2}\hat{\mu}_{1}}{1-\tilde{\mu}_{2}}F_{2}^{(0,1)}
+r_{2}\tilde{\mu}_{2}\hat{F}_{2}^{(1,0)}.
\end{eqnarray*}
%11
\item \begin{eqnarray*} &&\frac{\partial}{\partial
w_1}\frac{\partial}{\partial
w_1}\left(R_2\left(P_1\left(z_1\right)\bar{P}_2\left(z_2\right)\hat{P}_1\left(w_1\right)\hat{P}_2\left(w_2\right)\right)
F_2\left(z_1,\theta_2\left(P_1\left(z_1\right)\hat{P}_1\left(w_1\right)\hat{P}_2\left(w_2\right)\right)\right)\hat{F}_2\left(w_1,w_2\right)\right)\\
&=&\hat{\mu}_{1}^{2}R_{2}^{(2)}\left(1\right)+r_{2}\hat{P}_{1}^{(2)}\left(1\right)+2r_{2}\frac{\hat{\mu}_{1}^{2}}{1-\tilde{\mu}_{2}}F_{2}^{(0,1)}+
\hat{\mu}_{1}^{2}\tilde{\theta}_{2}^{(2)}\left(1\right)F_{2}^{(0,1)}+\frac{1}{1-\tilde{\mu}_{2}}\hat{P}_{1}^{(2)}\left(1\right)F_{2}^{(0,1)}\\
&+&\frac{\hat{\mu}_{1}^{2}}{1-\tilde{\mu}_{2}}F_{2}^{(0,2)}+2r_{2}\hat{\mu}_{1}\hat{F}_{2}^{(1,0)}+2\frac{\hat{\mu}_{1}}{1-\tilde{\mu}_{2}}F_{2}^{(0,1)}\hat{F}_{2}^{(1,0)}+\hat{F}_{2}^{(2,0)}.
\end{eqnarray*}
%12
\item \begin{eqnarray*} &&\frac{\partial}{\partial
w_2}\frac{\partial}{\partial
w_1}\left(R_2\left(P_1\left(z_1\right)\bar{P}_2\left(z_2\right)\hat{P}_1\left(w_1\right)\hat{P}_2\left(w_2\right)\right)
F_2\left(z_1,\theta_2\left(P_1\left(z_1\right)\hat{P}_1\left(w_1\right)\hat{P}_2\left(w_2\right)\right)\right)\hat{F}_2\left(w_1,w_2\right)\right)\\
&=&r_{2}\hat{\mu}_{2}\hat{\mu}_{1}+\hat{\mu}_{1}\hat{\mu}_{2}R_{2}^{(2)}(1)+\frac{\hat{\mu}_{1}\hat{\mu}_{2}}{1-\tilde{\mu}_{2}}F_{2}^{(0,1)}
+2r_{2}\frac{\hat{\mu}_{1}\hat{\mu}_{2}}{1-\tilde{\mu}_{2}}F_{2}^{(0,1)}+\hat{\mu}_{2}\hat{\mu}_{1}\tilde{\theta}_{2}^{(2)}\left(1\right)F_{2}^{(0,1)}+
r_{2}\hat{\mu}_{1}\hat{F}_{2}^{(0,1)}\\
&+&\frac{\hat{\mu}_{1}}{1-\tilde{\mu}_{2}}F_{2}^{(0,1)}\hat{F}_{2}^{(0,1)}+\hat{\mu}_{1}\hat{\mu}_{2}\left(\frac{1}{1-\tilde{\mu}_{2}}\right)^{2}F_{2}^{(0,2)}+
r_{2}\hat{\mu}_{2}\hat{F}_{2}^{(1,0)}+\frac{\hat{\mu}_{2}}{1-\tilde{\mu}_{2}}F_{2}^{(0,1)}\hat{F}_{2}^{(1,0)}+\hat{F}_{2}^{(1,1)}.
\end{eqnarray*}
%___________________________________________________________________________________________
%\subsubsection{Mixtas para $w_{2}$:}
%___________________________________________________________________________________________
%13

\item \begin{eqnarray*} &&\frac{\partial}{\partial
z_1}\frac{\partial}{\partial
w_2}\left(R_2\left(P_1\left(z_1\right)\bar{P}_2\left(z_2\right)\hat{P}_1\left(w_1\right)\hat{P}_2\left(w_2\right)\right)
F_2\left(z_1,\theta_2\left(P_1\left(z_1\right)\hat{P}_1\left(w_1\right)\hat{P}_2\left(w_2\right)\right)\right)\hat{F}_2\left(w_1,w_2\right)\right)\\
&=&r_{2}\mu_{1}\hat{\mu}_{2}+\mu_{1}\hat{\mu}_{2}R_{2}^{(2)}(1)+\frac{\mu_{1}\hat{\mu}_{2}}{1-\tilde{\mu}_{2}}F_{2}^{(0,1)}+r_{2}\frac{\mu_{1}\hat{\mu}_{2}}{1-\tilde{\mu}_{2}}F_{2}^{(0,1)}+\mu_{1}\hat{\mu}_{2}\tilde{\theta}_{2}^{(2)}\left(1\right)F_{2}^{(0,1)}+r_{2}\mu_{1}\hat{F}_{2}^{(0,1)}\\
&+&r_{2}\hat{\mu}_{2}\left(\frac{\mu_{1}}{1-\tilde{\mu}_{2}}F_{2}^{(0,1)}+F_{2}^{(1,0)}\right)+\hat{F}_{2}^{(0,1)}\left(\frac{\mu_{1}}{1-\tilde{\mu}_{2}}F_{2}^{(0,1)}+F_{2}^{(1,0)}\right)+\frac{\hat{\mu}_{2}}{1-\tilde{\mu}_{2}}\left(\frac{\mu_{1}}{1-\tilde{\mu}_{2}}F_{2}^{(0,2)}+F_{2}^{(1,1)}\right).
\end{eqnarray*}
%14
\item \begin{eqnarray*} &&\frac{\partial}{\partial
z_2}\frac{\partial}{\partial
w_2}\left(R_2\left(P_1\left(z_1\right)\bar{P}_2\left(z_2\right)\hat{P}_1\left(w_1\right)\hat{P}_2\left(w_2\right)\right)
F_2\left(z_1,\theta_2\left(P_1\left(z_1\right)\hat{P}_1\left(w_1\right)\hat{P}_2\left(w_2\right)\right)\right)\hat{F}_2\left(w_1,w_2\right)\right)\\
&=&r_{2}\tilde{\mu}_{2}\hat{\mu}_{2}+\tilde{\mu}_{2}\hat{\mu}_{2}R_{2}^{(2)}(1)+r_{2}\frac{\tilde{\mu}_{2}\hat{\mu}_{2}}{1-\tilde{\mu}_{2}}F_{2}^{(0,1)}+r_{2}\tilde{\mu}_{2}\hat{F}_{2}^{(0,1)}.
\end{eqnarray*}
%15
\item \begin{eqnarray*} &&\frac{\partial}{\partial
w_1}\frac{\partial}{\partial
w_2}\left(R_2\left(P_1\left(z_1\right)\bar{P}_2\left(z_2\right)\hat{P}_1\left(w_1\right)\hat{P}_2\left(w_2\right)\right)
F_2\left(z_1,\theta_2\left(P_1\left(z_1\right)\hat{P}_1\left(w_1\right)\hat{P}_2\left(w_2\right)\right)\right)\hat{F}_2\left(w_1,w_2\right)\right)\\
&=&r_{2}\hat{\mu}_{1}\hat{\mu}_{2}+\hat{\mu}_{1}\hat{\mu}_{2}R_{2}^{(2)}\left(1\right)+\frac{\hat{\mu}_{1}\hat{\mu}_{2}}{1-\tilde{\mu}_{2}}F_{2}^{(0,1)}+2r_{2}\frac{\hat{\mu}_{1}\hat{\mu}_{2}}{1-\tilde{\mu}_{2}}F_{2}^{(0,1)}+\hat{\mu}_{1}\hat{\mu}_{2}\theta_{2}^{(2)}\left(1\right)F_{2}^{(0,1)}+r_{2}\hat{\mu}_{1}\hat{F}_{2}^{(0,1)}\\
&+&\frac{\hat{\mu}_{1}}{1-\tilde{\mu}_{2}}F_{2}^{(0,1)}\hat{F}_{2}^{(0,1)}+\hat{\mu}_{1}\hat{\mu}_{2}\left(\frac{1}{1-\tilde{\mu}_{2}}\right)^{2}F_{2}^{(0,2)}+r_{2}\hat{\mu}_{2}\hat{F}_{2}^{(0,1)}+\frac{\hat{\mu}_{2}}{1-\tilde{\mu}_{2}}F_{2}^{(0,1)}\hat{F}_{2}^{(1,0)}+\hat{F}_{2}^{(1,1)}.
\end{eqnarray*}
%16

\item \begin{eqnarray*} &&\frac{\partial}{\partial
w_2}\frac{\partial}{\partial
w_2}\left(R_2\left(P_1\left(z_1\right)\bar{P}_2\left(z_2\right)\hat{P}_1\left(w_1\right)\hat{P}_2\left(w_2\right)\right)
F_2\left(z_1,\theta_2\left(P_1\left(z_1\right)\hat{P}_1\left(w_1\right)\hat{P}_2\left(w_2\right)\right)\right)\hat{F}_2\left(w_1,w_2\right)\right)\\
&=&\hat{\mu}_{2}^{2}R_{2}^{(2)}(1)+r_{2}\hat{P}_{2}^{(2)}\left(1\right)+2r_{2}\frac{\hat{\mu}_{2}^{2}}{1-\tilde{\mu}_{2}}F_{2}^{(0,1)}+\hat{\mu}_{2}^{2}\tilde{\theta}_{2}^{(2)}\left(1\right)F_{2}^{(0,1)}+\frac{1}{1-\tilde{\mu}_{2}}\hat{P}_{2}^{(2)}\left(1\right)F_{2}^{(0,1)}\\
&+&2r_{2}\hat{\mu}_{2}\hat{F}_{2}^{(0,1)}+2\frac{\hat{\mu}_{2}}{1-\tilde{\mu}_{2}}F_{2}^{(0,1)}\hat{F}_{2}^{(0,1)}+\left(\frac{\hat{\mu}_{2}}{1-\tilde{\mu}_{2}}\right)^{2}F_{2}^{(0,2)}+\hat{F}_{2}^{(0,2)}.
\end{eqnarray*}
\end{enumerate}
%___________________________________________________________________________________________
%
%\subsection{Derivadas de Segundo Orden para $F_{2}$}
%___________________________________________________________________________________________


\begin{enumerate}

%___________________________________________________________________________________________
%\subsubsection{Mixtas para $z_{1}$:}
%___________________________________________________________________________________________

%1/17
\item \begin{eqnarray*} &&\frac{\partial}{\partial
z_1}\frac{\partial}{\partial
z_1}\left(R_1\left(P_1\left(z_1\right)\bar{P}_2\left(z_2\right)\hat{P}_1\left(w_1\right)\hat{P}_2\left(w_2\right)\right)
F_1\left(\theta_1\left(\tilde{P}_2\left(z_1\right)\hat{P}_1\left(w_1\right)\hat{P}_2\left(w_2\right)\right)\right)\hat{F}_1\left(w_1,w_2\right)\right)\\
&=&r_{1}P_{1}^{(2)}\left(1\right)+\mu_{1}^{2}R_{1}^{(2)}\left(1\right).
\end{eqnarray*}

%2/18
\item \begin{eqnarray*} &&\frac{\partial}{\partial
z_2}\frac{\partial}{\partial
z_1}\left(R_1\left(P_1\left(z_1\right)\bar{P}_2\left(z_2\right)\hat{P}_1\left(w_1\right)\hat{P}_2\left(w_2\right)\right)F_1\left(\theta_1\left(\tilde{P}_2\left(z_1\right)\hat{P}_1\left(w_1\right)\hat{P}_2\left(w_2\right)\right)\right)\hat{F}_1\left(w_1,w_2\right)\right)\\
&=&\mu_{1}\tilde{\mu}_{2}r_{1}+\mu_{1}\tilde{\mu}_{2}R_{1}^{(2)}(1)+
r_{1}\mu_{1}\left(\frac{\tilde{\mu}_{2}}{1-\mu_{1}}F_{1}^{(1,0)}+F_{1}^{(0,1)}\right).
\end{eqnarray*}

%3/19
\item \begin{eqnarray*} &&\frac{\partial}{\partial
w_1}\frac{\partial}{\partial
z_1}\left(R_1\left(P_1\left(z_1\right)\bar{P}_2\left(z_2\right)\hat{P}_1\left(w_1\right)\hat{P}_2\left(w_2\right)\right)F_1\left(\theta_1\left(\tilde{P}_2\left(z_1\right)\hat{P}_1\left(w_1\right)\hat{P}_2\left(w_2\right)\right)\right)\hat{F}_1\left(w_1,w_2\right)\right)\\
&=&r_{1}\mu_{1}\hat{\mu}_{1}+\mu_{1}\hat{\mu}_{1}R_{1}^{(2)}\left(1\right)+r_{1}\frac{\mu_{1}\hat{\mu}_{1}}{1-\mu_{1}}F_{1}^{(1,0)}+r_{1}\mu_{1}\hat{F}_{1}^{(1,0)}.
\end{eqnarray*}
%4/20
\item \begin{eqnarray*} &&\frac{\partial}{\partial
w_2}\frac{\partial}{\partial
z_1}\left(R_1\left(P_1\left(z_1\right)\bar{P}_2\left(z_2\right)\hat{P}_1\left(w_1\right)\hat{P}_2\left(w_2\right)\right)F_1\left(\theta_1\left(\tilde{P}_2\left(z_1\right)\hat{P}_1\left(w_1\right)\hat{P}_2\left(w_2\right)\right)\right)\hat{F}_1\left(w_1,w_2\right)\right)\\
&=&\mu_{1}\hat{\mu}_{2}r_{1}+\mu_{1}\hat{\mu}_{2}R_{1}^{(2)}\left(1\right)+r_{1}\mu_{1}\hat{F}_{1}^{(0,1)}+r_{1}\frac{\mu_{1}\hat{\mu}_{2}}{1-\mu_{1}}F_{1}^{(1,0)}.
\end{eqnarray*}
%___________________________________________________________________________________________
%\subsubsection{Mixtas para $z_{2}$:}
%___________________________________________________________________________________________
%5/21
\item \begin{eqnarray*}
&&\frac{\partial}{\partial z_1}\frac{\partial}{\partial z_2}\left(R_1\left(P_1\left(z_1\right)\bar{P}_2\left(z_2\right)\hat{P}_1\left(w_1\right)\hat{P}_2\left(w_2\right)\right)F_1\left(\theta_1\left(\tilde{P}_2\left(z_1\right)\hat{P}_1\left(w_1\right)\hat{P}_2\left(w_2\right)\right)\right)\hat{F}_1\left(w_1,w_2\right)\right)\\
&=&r_{1}\mu_{1}\tilde{\mu}_{2}+\mu_{1}\tilde{\mu}_{2}R_{1}^{(2)}\left(1\right)+r_{1}\mu_{1}\left(\frac{\tilde{\mu}_{2}}{1-\mu_{1}}F_{1}^{(1,0)}+F_{1}^{(0,1)}\right).
\end{eqnarray*}

%6/22
\item \begin{eqnarray*}
&&\frac{\partial}{\partial z_2}\frac{\partial}{\partial z_2}\left(R_1\left(P_1\left(z_1\right)\bar{P}_2\left(z_2\right)\hat{P}_1\left(w_1\right)\hat{P}_2\left(w_2\right)\right)F_1\left(\theta_1\left(\tilde{P}_2\left(z_1\right)\hat{P}_1\left(w_1\right)\hat{P}_2\left(w_2\right)\right)\right)\hat{F}_1\left(w_1,w_2\right)\right)\\
&=&\tilde{\mu}_{2}^{2}R_{1}^{(2)}\left(1\right)+r_{1}\tilde{P}_{2}^{(2)}\left(1\right)+2r_{1}\tilde{\mu}_{2}\left(\frac{\tilde{\mu}_{2}}{1-\mu_{1}}F_{1}^{(1,0)}+F_{1}^{(0,1)}\right)+F_{1}^{(0,2)}+\tilde{\mu}_{2}^{2}\theta_{1}^{(2)}\left(1\right)F_{1}^{(1,0)}\\
&+&\frac{1}{1-\mu_{1}}\tilde{P}_{2}^{(2)}\left(1\right)F_{1}^{(1,0)}+\frac{\tilde{\mu}_{2}}{1-\mu_{1}}F_{1}^{(1,1)}+\frac{\tilde{\mu}_{2}}{1-\mu_{1}}\left(\frac{\tilde{\mu}_{2}}{1-\mu_{1}}F_{1}^{(2,0)}+F_{1}^{(1,1)}\right).
\end{eqnarray*}
%7/23
\item \begin{eqnarray*}
&&\frac{\partial}{\partial w_1}\frac{\partial}{\partial z_2}\left(R_1\left(P_1\left(z_1\right)\bar{P}_2\left(z_2\right)\hat{P}_1\left(w_1\right)\hat{P}_2\left(w_2\right)\right)F_1\left(\theta_1\left(\tilde{P}_2\left(z_1\right)\hat{P}_1\left(w_1\right)\hat{P}_2\left(w_2\right)\right)\right)\hat{F}_1\left(w_1,w_2\right)\right)\\
&=&\tilde{\mu}_{2}\hat{\mu}_{1}r_{1}+\tilde{\mu}_{2}\hat{\mu}_{1}R_{1}^{(2)}\left(1\right)+r_{1}\frac{\tilde{\mu}_{2}\hat{\mu}_{1}}{1-\mu_{1}}F_{1}^{(1,0)}+\hat{\mu}_{1}r_{1}\left(\frac{\tilde{\mu}_{2}}{1-\mu_{1}}F_{1}^{(1,0)}+F_{1}^{(0,1)}\right)+r_{1}\tilde{\mu}_{2}\hat{F}_{1}^{(1,0)}\\
&+&\left(\frac{\tilde{\mu}_{2}}{1-\mu_{1}}F_{1}^{(1,0)}+F_{1}^{(0,1)}\right)\hat{F}_{1}^{(1,0)}+\frac{\tilde{\mu}_{2}\hat{\mu}_{1}}{1-\mu_{1}}F_{1}^{(1,0)}+\tilde{\mu}_{2}\hat{\mu}_{1}\theta_{1}^{(2)}\left(1\right)F_{1}^{(1,0)}+\frac{\hat{\mu}_{1}}{1-\mu_{1}}F_{1}^{(1,1)}\\
&+&\left(\frac{1}{1-\mu_{1}}\right)^{2}\tilde{\mu}_{2}\hat{\mu}_{1}F_{1}^{(2,0)}.
\end{eqnarray*}
%8/24
\item \begin{eqnarray*}
&&\frac{\partial}{\partial w_2}\frac{\partial}{\partial z_2}\left(R_1\left(P_1\left(z_1\right)\bar{P}_2\left(z_2\right)\hat{P}_1\left(w_1\right)\hat{P}_2\left(w_2\right)\right)F_1\left(\theta_1\left(\tilde{P}_2\left(z_1\right)\hat{P}_1\left(w_1\right)\hat{P}_2\left(w_2\right)\right)\right)\hat{F}_1\left(w_1,w_2\right)\right)\\
&=&\hat{\mu}_{2}\tilde{\mu}_{2}r_{1}+\hat{\mu}_{2}\tilde{\mu}_{2}R_{1}^{(2)}(1)+r_{1}\tilde{\mu}_{2}\hat{F}_{1}^{(0,1)}+r_{1}\frac{\hat{\mu}_{2}\tilde{\mu}_{2}}{1-\mu_{1}}F_{1}^{(1,0)}+\hat{\mu}_{2}r_{1}\left(\frac{\tilde{\mu}_{2}}{1-\mu_{1}}F_{1}^{(1,0)}+F_{1}^{(0,1)}\right)\\
&+&\left(\frac{\tilde{\mu}_{2}}{1-\mu_{1}}F_{1}^{(1,0)}+F_{1}^{(0,1)}\right)\hat{F}_{1}^{(0,1)}+\frac{\tilde{\mu}_{2}\hat{\mu_{2}}}{1-\mu_{1}}F_{1}^{(1,0)}+\hat{\mu}_{2}\tilde{\mu}_{2}\theta_{1}^{(2)}\left(1\right)F_{1}^{(1,0)}+\frac{\hat{\mu}_{2}}{1-\mu_{1}}F_{1}^{(1,1)}\\
&+&\left(\frac{1}{1-\mu_{1}}\right)^{2}\tilde{\mu}_{2}\hat{\mu}_{2}F_{1}^{(2,0)}.
\end{eqnarray*}
%___________________________________________________________________________________________
%\subsubsection{Mixtas para $w_{1}$:}
%___________________________________________________________________________________________
%9/25
\item \begin{eqnarray*} &&\frac{\partial}{\partial
z_1}\frac{\partial}{\partial
w_1}\left(R_1\left(P_1\left(z_1\right)\bar{P}_2\left(z_2\right)\hat{P}_1\left(w_1\right)\hat{P}_2\left(w_2\right)\right)F_1\left(\theta_1\left(\tilde{P}_2\left(z_1\right)\hat{P}_1\left(w_1\right)\hat{P}_2\left(w_2\right)\right)\right)\hat{F}_1\left(w_1,w_2\right)\right)\\
&=&r_{1}\mu_{1}\hat{\mu}_{1}+\mu_{1}\hat{\mu}_{1}R_{1}^{(2)}(1)+r_{1}\frac{\mu_{1}\hat{\mu}_{1}}{1-\mu_{1}}F_{1}^{(1,0)}+r_{1}\mu_{1}\hat{F}_{1}^{(1,0)}.
\end{eqnarray*}
%10/26
\item \begin{eqnarray*} &&\frac{\partial}{\partial
z_2}\frac{\partial}{\partial
w_1}\left(R_1\left(P_1\left(z_1\right)\bar{P}_2\left(z_2\right)\hat{P}_1\left(w_1\right)\hat{P}_2\left(w_2\right)\right)F_1\left(\theta_1\left(\tilde{P}_2\left(z_1\right)\hat{P}_1\left(w_1\right)\hat{P}_2\left(w_2\right)\right)\right)\hat{F}_1\left(w_1,w_2\right)\right)\\
&=&r_{1}\hat{\mu}_{1}\tilde{\mu}_{2}+\tilde{\mu}_{2}\hat{\mu}_{1}R_{1}^{(2)}\left(1\right)+
\frac{\hat{\mu}_{1}\tilde{\mu}_{2}}{1-\mu_{1}}F_{1}^{1,0)}+r_{1}\frac{\hat{\mu}_{1}\tilde{\mu}_{2}}{1-\mu_{1}}F_{1}^{(1,0)}+\hat{\mu}_{1}\tilde{\mu}_{2}\theta_{1}^{(2)}\left(1\right)F_{2}^{(1,0)}\\
&+&r_{1}\hat{\mu}_{1}\left(F_{1}^{(1,0)}+\frac{\tilde{\mu}_{2}}{1-\mu_{1}}F_{1}^{1,0)}\right)+
r_{1}\tilde{\mu}_{2}\hat{F}_{1}^{(1,0)}+\left(F_{1}^{(0,1)}+\frac{\tilde{\mu}_{2}}{1-\mu_{1}}F_{1}^{1,0)}\right)\hat{F}_{1}^{(1,0)}\\
&+&\frac{\hat{\mu}_{1}}{1-\mu_{1}}\left(F_{1}^{(1,1)}+\frac{\tilde{\mu}_{2}}{1-\mu_{1}}F_{1}^{2,0)}\right).
\end{eqnarray*}
%11/27
\item \begin{eqnarray*} &&\frac{\partial}{\partial
w_1}\frac{\partial}{\partial
w_1}\left(R_1\left(P_1\left(z_1\right)\bar{P}_2\left(z_2\right)\hat{P}_1\left(w_1\right)\hat{P}_2\left(w_2\right)\right)F_1\left(\theta_1\left(\tilde{P}_2\left(z_1\right)\hat{P}_1\left(w_1\right)\hat{P}_2\left(w_2\right)\right)\right)\hat{F}_1\left(w_1,w_2\right)\right)\\
&=&\hat{\mu}_{1}^{2}R_{1}^{(2)}\left(1\right)+r_{1}\hat{P}_{1}^{(2)}\left(1\right)+2r_{1}\frac{\hat{\mu}_{1}^{2}}{1-\mu_{1}}F_{1}^{(1,0)}+\hat{\mu}_{1}^{2}\theta_{1}^{(2)}\left(1\right)F_{1}^{(1,0)}+\frac{1}{1-\mu_{1}}\hat{P}_{1}^{(2)}\left(1\right)F_{1}^{(1,0)}\\
&+&2r_{1}\hat{\mu}_{1}\hat{F}_{1}^{(1,0)}+2\frac{\hat{\mu}_{1}}{1-\mu_{1}}F_{1}^{(1,0)}\hat{F}_{1}^{(1,0)}+\left(\frac{\hat{\mu}_{1}}{1-\mu_{1}}\right)^{2}F_{1}^{(2,0)}+\hat{F}_{1}^{(2,0)}.
\end{eqnarray*}
%12/28
\item \begin{eqnarray*} &&\frac{\partial}{\partial
w_2}\frac{\partial}{\partial
w_1}\left(R_1\left(P_1\left(z_1\right)\bar{P}_2\left(z_2\right)\hat{P}_1\left(w_1\right)\hat{P}_2\left(w_2\right)\right)F_1\left(\theta_1\left(\tilde{P}_2\left(z_1\right)\hat{P}_1\left(w_1\right)\hat{P}_2\left(w_2\right)\right)\right)\hat{F}_1\left(w_1,w_2\right)\right)\\
&=&r_{1}\hat{\mu}_{1}\hat{\mu}_{2}+\hat{\mu}_{1}\hat{\mu}_{2}R_{1}^{(2)}\left(1\right)+r_{1}\hat{\mu}_{1}\hat{F}_{1}^{(0,1)}+
\frac{\hat{\mu}_{1}\hat{\mu}_{2}}{1-\mu_{1}}F_{1}^{(1,0)}+2r_{1}\frac{\hat{\mu}_{1}\hat{\mu}_{2}}{1-\mu_{1}}F_{1}^{1,0)}+\hat{\mu}_{1}\hat{\mu}_{2}\theta_{1}^{(2)}\left(1\right)F_{1}^{(1,0)}\\
&+&\frac{\hat{\mu}_{1}}{1-\mu_{1}}F_{1}^{(1,0)}\hat{F}_{1}^{(0,1)}+
r_{1}\hat{\mu}_{2}\hat{F}_{1}^{(1,0)}+\frac{\hat{\mu}_{2}}{1-\mu_{1}}\hat{F}_{1}^{(1,0)}F_{1}^{(1,0)}+\hat{F}_{1}^{(1,1)}+\hat{\mu}_{1}\hat{\mu}_{2}\left(\frac{1}{1-\mu_{1}}\right)^{2}F_{1}^{(2,0)}.
\end{eqnarray*}
%___________________________________________________________________________________________
%\subsubsection{Mixtas para $w_{2}$:}
%___________________________________________________________________________________________
%13/29
\item \begin{eqnarray*} &&\frac{\partial}{\partial
z_1}\frac{\partial}{\partial
w_2}\left(R_1\left(P_1\left(z_1\right)\bar{P}_2\left(z_2\right)\hat{P}_1\left(w_1\right)\hat{P}_2\left(w_2\right)\right)F_1\left(\theta_1\left(\tilde{P}_2\left(z_1\right)\hat{P}_1\left(w_1\right)\hat{P}_2\left(w_2\right)\right)\right)\hat{F}_1\left(w_1,w_2\right)\right)\\
&=&r_{1}\mu_{1}\hat{\mu}_{2}+\mu_{1}\hat{\mu}_{2}R_{1}^{(2)}\left(1\right)+r_{1}\mu_{1}\hat{F}_{1}^{(0,1)}+r_{1}\frac{\mu_{1}\hat{\mu}_{2}}{1-\mu_{1}}F_{1}^{(1,0)}.
\end{eqnarray*}
%14/30
\item \begin{eqnarray*} &&\frac{\partial}{\partial
z_2}\frac{\partial}{\partial
w_2}\left(R_1\left(P_1\left(z_1\right)\bar{P}_2\left(z_2\right)\hat{P}_1\left(w_1\right)\hat{P}_2\left(w_2\right)\right)F_1\left(\theta_1\left(\tilde{P}_2\left(z_1\right)\hat{P}_1\left(w_1\right)\hat{P}_2\left(w_2\right)\right)\right)\hat{F}_1\left(w_1,w_2\right)\right)\\
&=&r_{1}\hat{\mu}_{2}\tilde{\mu}_{2}+\hat{\mu}_{2}\tilde{\mu}_{2}R_{1}^{(2)}\left(1\right)+r_{1}\tilde{\mu}_{2}\hat{F}_{1}^{(0,1)}+\frac{\hat{\mu}_{2}\tilde{\mu}_{2}}{1-\mu_{1}}F_{1}^{(1,0)}+r_{1}\frac{\hat{\mu}_{2}\tilde{\mu}_{2}}{1-\mu_{1}}F_{1}^{(1,0)}\\
&+&\hat{\mu}_{2}\tilde{\mu}_{2}\theta_{1}^{(2)}\left(1\right)F_{1}^{(1,0)}+r_{1}\hat{\mu}_{2}\left(F_{1}^{(0,1)}+\frac{\tilde{\mu}_{2}}{1-\mu_{1}}F_{1}^{(1,0)}\right)+\left(F_{1}^{(0,1)}+\frac{\tilde{\mu}_{2}}{1-\mu_{1}}F_{1}^{(1,0)}\right)\hat{F}_{1}^{(0,1)}\\&+&\frac{\hat{\mu}_{2}}{1-\mu_{1}}\left(F_{1}^{(1,1)}+\frac{\tilde{\mu}_{2}}{1-\mu_{1}}F_{1}^{(2,0)}\right).
\end{eqnarray*}
%15/31
\item \begin{eqnarray*} &&\frac{\partial}{\partial
w_1}\frac{\partial}{\partial
w_2}\left(R_1\left(P_1\left(z_1\right)\bar{P}_2\left(z_2\right)\hat{P}_1\left(w_1\right)\hat{P}_2\left(w_2\right)\right)F_1\left(\theta_1\left(\tilde{P}_2\left(z_1\right)\hat{P}_1\left(w_1\right)\hat{P}_2\left(w_2\right)\right)\right)\hat{F}_1\left(w_1,w_2\right)\right)\\
&=&r_{1}\hat{\mu}_{1}\hat{\mu}_{2}+\hat{\mu}_{1}\hat{\mu}_{2}R_{1}^{(2)}\left(1\right)+r_{1}\hat{\mu}_{1}\hat{F}_{1}^{(0,1)}+
\frac{\hat{\mu}_{1}\hat{\mu}_{2}}{1-\mu_{1}}F_{1}^{(1,0)}+2r_{1}\frac{\hat{\mu}_{1}\hat{\mu}_{2}}{1-\mu_{1}}F_{1}^{(1,0)}+\hat{\mu}_{1}\hat{\mu}_{2}\theta_{1}^{(2)}\left(1\right)F_{1}^{(1,0)}\\
&+&\frac{\hat{\mu}_{1}}{1-\mu_{1}}\hat{F}_{1}^{(0,1)}F_{1}^{(1,0)}+r_{1}\hat{\mu}_{2}\hat{F}_{1}^{(1,0)}+\frac{\hat{\mu}_{2}}{1-\mu_{1}}\hat{F}_{1}^{(1,0)}F_{1}^{(1,0)}+\hat{F}_{1}^{(1,1)}+\hat{\mu}_{1}\hat{\mu}_{2}\left(\frac{1}{1-\mu_{1}}\right)^{2}F_{1}^{(2,0)}.
\end{eqnarray*}
%16/32
\item \begin{eqnarray*} &&\frac{\partial}{\partial
w_2}\frac{\partial}{\partial
w_2}\left(R_1\left(P_1\left(z_1\right)\bar{P}_2\left(z_2\right)\hat{P}_1\left(w_1\right)\hat{P}_2\left(w_2\right)\right)F_1\left(\theta_1\left(\tilde{P}_2\left(z_1\right)\hat{P}_1\left(w_1\right)\hat{P}_2\left(w_2\right)\right)\right)\hat{F}_1\left(w_1,w_2\right)\right)\\
&=&\hat{\mu}_{2}R_{1}^{(2)}\left(1\right)+r_{1}\hat{P}_{2}^{(2)}\left(1\right)+2r_{1}\hat{\mu}_{2}\hat{F}_{1}^{(0,1)}+\hat{F}_{1}^{(0,2)}+2r_{1}\frac{\hat{\mu}_{2}^{2}}{1-\mu_{1}}F_{1}^{(1,0)}+\hat{\mu}_{2}^{2}\theta_{1}^{(2)}\left(1\right)F_{1}^{(1,0)}\\
&+&\frac{1}{1-\mu_{1}}\hat{P}_{2}^{(2)}\left(1\right)F_{1}^{(1,0)} +
2\frac{\hat{\mu}_{2}}{1-\mu_{1}}F_{1}^{(1,0)}\hat{F}_{1}^{(0,1)}+\left(\frac{\hat{\mu}_{2}}{1-\mu_{1}}\right)^{2}F_{1}^{(2,0)}.
\end{eqnarray*}
\end{enumerate}

%___________________________________________________________________________________________
%
%\subsection{Derivadas de Segundo Orden para $\hat{F}_{1}$}
%___________________________________________________________________________________________


\begin{enumerate}
%___________________________________________________________________________________________
%\subsubsection{Mixtas para $z_{1}$:}
%___________________________________________________________________________________________
%1/33

\item \begin{eqnarray*} &&\frac{\partial}{\partial
z_1}\frac{\partial}{\partial
z_1}\left(\hat{R}_{2}\left(P_{1}\left(z_{1}\right)\tilde{P}_{2}\left(z_{2}\right)\hat{P}_{1}\left(w_{1}\right)\hat{P}_{2}\left(w_{2}\right)\right)\hat{F}_{2}\left(w_{1},\hat{\theta}_{2}\left(P_{1}\left(z_{1}\right)\tilde{P}_{2}\left(z_{2}\right)\hat{P}_{1}\left(w_{1}\right)\right)\right)F_{2}\left(z_{1},z_{2}\right)\right)\\
&=&\hat{r}_{2}P_{1}^{(2)}\left(1\right)+
\mu_{1}^{2}\hat{R}_{2}^{(2)}\left(1\right)+
2\hat{r}_{2}\frac{\mu_{1}^{2}}{1-\hat{\mu}_{2}}\hat{F}_{2}^{(0,1)}+
\frac{1}{1-\hat{\mu}_{2}}P_{1}^{(2)}\left(1\right)\hat{F}_{2}^{(0,1)}+
\mu_{1}^{2}\hat{\theta}_{2}^{(2)}\left(1\right)\hat{F}_{2}^{(0,1)}\\
&+&\left(\frac{\mu_{1}^{2}}{1-\hat{\mu}_{2}}\right)^{2}\hat{F}_{2}^{(0,2)}+
2\hat{r}_{2}\mu_{1}F_{2}^{(1,0)}+2\frac{\mu_{1}}{1-\hat{\mu}_{2}}\hat{F}_{2}^{(0,1)}F_{2}^{(1,0)}+F_{2}^{(2,0)}.
\end{eqnarray*}

%2/34
\item \begin{eqnarray*} &&\frac{\partial}{\partial
z_2}\frac{\partial}{\partial
z_1}\left(\hat{R}_{2}\left(P_{1}\left(z_{1}\right)\tilde{P}_{2}\left(z_{2}\right)\hat{P}_{1}\left(w_{1}\right)\hat{P}_{2}\left(w_{2}\right)\right)\hat{F}_{2}\left(w_{1},\hat{\theta}_{2}\left(P_{1}\left(z_{1}\right)\tilde{P}_{2}\left(z_{2}\right)\hat{P}_{1}\left(w_{1}\right)\right)\right)F_{2}\left(z_{1},z_{2}\right)\right)\\
&=&\hat{r}_{2}\mu_{1}\tilde{\mu}_{2}+\mu_{1}\tilde{\mu}_{2}\hat{R}_{2}^{(2)}\left(1\right)+\hat{r}_{2}\mu_{1}F_{2}^{(0,1)}+
\frac{\mu_{1}\tilde{\mu}_{2}}{1-\hat{\mu}_{2}}\hat{F}_{2}^{(0,1)}+2\hat{r}_{2}\frac{\mu_{1}\tilde{\mu}_{2}}{1-\hat{\mu}_{2}}\hat{F}_{2}^{(0,1)}+\mu_{1}\tilde{\mu}_{2}\hat{\theta}_{2}^{(2)}\left(1\right)\hat{F}_{2}^{(0,1)}\\
&+&\frac{\mu_{1}}{1-\hat{\mu}_{2}}F_{2}^{(0,1)}\hat{F}_{2}^{(0,1)}+\mu_{1} \tilde{\mu}_{2}\left(\frac{1}{1-\hat{\mu}_{2}}\right)^{2}\hat{F}_{2}^{(0,2)}+\hat{r}_{2}\tilde{\mu}_{2}F_{2}^{(1,0)}+\frac{\tilde{\mu}_{2}}{1-\hat{\mu}_{2}}\hat{F}_{2}^{(0,1)}F_{2}^{(1,0)}+F_{2}^{(1,1)}.
\end{eqnarray*}


%3/35

\item \begin{eqnarray*} &&\frac{\partial}{\partial
w_1}\frac{\partial}{\partial
z_1}\left(\hat{R}_{2}\left(P_{1}\left(z_{1}\right)\tilde{P}_{2}\left(z_{2}\right)\hat{P}_{1}\left(w_{1}\right)\hat{P}_{2}\left(w_{2}\right)\right)\hat{F}_{2}\left(w_{1},\hat{\theta}_{2}\left(P_{1}\left(z_{1}\right)\tilde{P}_{2}\left(z_{2}\right)\hat{P}_{1}\left(w_{1}\right)\right)\right)F_{2}\left(z_{1},z_{2}\right)\right)\\
&=&\hat{r}_{2}\mu_{1}\hat{\mu}_{1}+\mu_{1}\hat{\mu}_{1}\hat{R}_{2}^{(2)}\left(1\right)+\hat{r}_{2}\frac{\mu_{1}\hat{\mu}_{1}}{1-\hat{\mu}_{2}}\hat{F}_{2}^{(0,1)}+\hat{r}_{2}\hat{\mu}_{1}F_{2}^{(1,0)}+\hat{r}_{2}\mu_{1}\hat{F}_{2}^{(1,0)}+F_{2}^{(1,0)}\hat{F}_{2}^{(1,0)}+\frac{\mu_{1}}{1-\hat{\mu}_{2}}\hat{F}_{2}^{(1,1)}.
\end{eqnarray*}

%4/36

\item \begin{eqnarray*} &&\frac{\partial}{\partial
w_2}\frac{\partial}{\partial
z_1}\left(\hat{R}_{2}\left(P_{1}\left(z_{1}\right)\tilde{P}_{2}\left(z_{2}\right)\hat{P}_{1}\left(w_{1}\right)\hat{P}_{2}\left(w_{2}\right)\right)\hat{F}_{2}\left(w_{1},\hat{\theta}_{2}\left(P_{1}\left(z_{1}\right)\tilde{P}_{2}\left(z_{2}\right)\hat{P}_{1}\left(w_{1}\right)\right)\right)F_{2}\left(z_{1},z_{2}\right)\right)\\
&=&\hat{r}_{2}\mu_{1}\hat{\mu}_{2}+\mu_{1}\hat{\mu}_{2}\hat{R}_{2}^{(2)}\left(1\right)+\frac{\mu_{1}\hat{\mu}_{2}}{1-\hat{\mu}_{2}}\hat{F}_{2}^{(0,1)}+2\hat{r}_{2}\frac{\mu_{1}\hat{\mu}_{2}}{1-\hat{\mu}_{2}}\hat{F}_{2}^{(0,1)}+\mu_{1}\hat{\mu}_{2}\hat{\theta}_{2}^{(2)}\left(1\right)\hat{F}_{2}^{(0,1)}\\
&+&\mu_{1}\hat{\mu}_{2}\left(\frac{1}{1-\hat{\mu}_{2}}\right)^{2}\hat{F}_{2}^{(0,2)}+\hat{r}_{2}\hat{\mu}_{2}F_{2}^{(1,0)}+\frac{\hat{\mu}_{2}}{1-\hat{\mu}_{2}}\hat{F}_{2}^{(0,1)}F_{2}^{(1,0)}.
\end{eqnarray*}
%___________________________________________________________________________________________
%\subsubsection{Mixtas para $z_{2}$:}
%___________________________________________________________________________________________

%5/37

\item \begin{eqnarray*} &&\frac{\partial}{\partial
z_1}\frac{\partial}{\partial
z_2}\left(\hat{R}_{2}\left(P_{1}\left(z_{1}\right)\tilde{P}_{2}\left(z_{2}\right)\hat{P}_{1}\left(w_{1}\right)\hat{P}_{2}\left(w_{2}\right)\right)\hat{F}_{2}\left(w_{1},\hat{\theta}_{2}\left(P_{1}\left(z_{1}\right)\tilde{P}_{2}\left(z_{2}\right)\hat{P}_{1}\left(w_{1}\right)\right)\right)F_{2}\left(z_{1},z_{2}\right)\right)\\
&=&\hat{r}_{2}\mu_{1}\tilde{\mu}_{2}+\mu_{1}\tilde{\mu}_{2}\hat{R}_{2}^{(2)}\left(1\right)+\mu_{1}\hat{r}_{2}F_{2}^{(0,1)}+
\frac{\mu_{1}\tilde{\mu}_{2}}{1-\hat{\mu}_{2}}\hat{F}_{2}^{(0,1)}+2\hat{r}_{2}\frac{\mu_{1}\tilde{\mu}_{2}}{1-\hat{\mu}_{2}}\hat{F}_{2}^{(0,1)}+\mu_{1}\tilde{\mu}_{2}\hat{\theta}_{2}^{(2)}\left(1\right)\hat{F}_{2}^{(0,1)}\\
&+&\frac{\mu_{1}}{1-\hat{\mu}_{2}}F_{2}^{(0,1)}\hat{F}_{2}^{(0,1)}+\mu_{1}\tilde{\mu}_{2}\left(\frac{1}{1-\hat{\mu}_{2}}\right)^{2}\hat{F}_{2}^{(0,2)}+\hat{r}_{2}\tilde{\mu}_{2}F_{2}^{(1,0)}+\frac{\tilde{\mu}_{2}}{1-\hat{\mu}_{2}}\hat{F}_{2}^{(0,1)}F_{2}^{(1,0)}+F_{2}^{(1,1)}.
\end{eqnarray*}

%6/38

\item \begin{eqnarray*} &&\frac{\partial}{\partial
z_2}\frac{\partial}{\partial
z_2}\left(\hat{R}_{2}\left(P_{1}\left(z_{1}\right)\tilde{P}_{2}\left(z_{2}\right)\hat{P}_{1}\left(w_{1}\right)\hat{P}_{2}\left(w_{2}\right)\right)\hat{F}_{2}\left(w_{1},\hat{\theta}_{2}\left(P_{1}\left(z_{1}\right)\tilde{P}_{2}\left(z_{2}\right)\hat{P}_{1}\left(w_{1}\right)\right)\right)F_{2}\left(z_{1},z_{2}\right)\right)\\
&=&\hat{r}_{2}\tilde{P}_{2}^{(2)}\left(1\right)+\tilde{\mu}_{2}^{2}\hat{R}_{2}^{(2)}\left(1\right)+2\hat{r}_{2}\tilde{\mu}_{2}F_{2}^{(0,1)}+2\hat{r}_{2}\frac{\tilde{\mu}_{2}^{2}}{1-\hat{\mu}_{2}}\hat{F}_{2}^{(0,1)}+\frac{1}{1-\hat{\mu}_{2}}\tilde{P}_{2}^{(2)}\left(1\right)\hat{F}_{2}^{(0,1)}\\
&+&\tilde{\mu}_{2}^{2}\hat{\theta}_{2}^{(2)}\left(1\right)\hat{F}_{2}^{(0,1)}+2\frac{\tilde{\mu}_{2}}{1-\hat{\mu}_{2}}F_{2}^{(0,1)}\hat{F}_{2}^{(0,1)}+F_{2}^{(0,2)}+\left(\frac{\tilde{\mu}_{2}}{1-\hat{\mu}_{2}}\right)^{2}\hat{F}_{2}^{(0,2)}.
\end{eqnarray*}

%7/39

\item \begin{eqnarray*} &&\frac{\partial}{\partial
w_1}\frac{\partial}{\partial
z_2}\left(\hat{R}_{2}\left(P_{1}\left(z_{1}\right)\tilde{P}_{2}\left(z_{2}\right)\hat{P}_{1}\left(w_{1}\right)\hat{P}_{2}\left(w_{2}\right)\right)\hat{F}_{2}\left(w_{1},\hat{\theta}_{2}\left(P_{1}\left(z_{1}\right)\tilde{P}_{2}\left(z_{2}\right)\hat{P}_{1}\left(w_{1}\right)\right)\right)F_{2}\left(z_{1},z_{2}\right)\right)\\
&=&\hat{r}_{2}\tilde{\mu}_{2}\hat{\mu}_{1}+\tilde{\mu}_{2}\hat{\mu}_{1}\hat{R}_{2}^{(2)}\left(1\right)+\hat{r}_{2}\hat{\mu}_{1}F_{2}^{(0,1)}+\hat{r}_{2}\frac{\tilde{\mu}_{2}\hat{\mu}_{1}}{1-\hat{\mu}_{2}}\hat{F}_{2}^{(0,1)}+\hat{r}_{2}\tilde{\mu}_{2}\hat{F}_{2}^{(1,0)}+F_{2}^{(0,1)}\hat{F}_{2}^{(1,0)}+\frac{\tilde{\mu}_{2}}{1-\hat{\mu}_{2}}\hat{F}_{2}^{(1,1)}.
\end{eqnarray*}
%8/40

\item \begin{eqnarray*} &&\frac{\partial}{\partial
w_2}\frac{\partial}{\partial
z_2}\left(\hat{R}_{2}\left(P_{1}\left(z_{1}\right)\tilde{P}_{2}\left(z_{2}\right)\hat{P}_{1}\left(w_{1}\right)\hat{P}_{2}\left(w_{2}\right)\right)\hat{F}_{2}\left(w_{1},\hat{\theta}_{2}\left(P_{1}\left(z_{1}\right)\tilde{P}_{2}\left(z_{2}\right)\hat{P}_{1}\left(w_{1}\right)\right)\right)F_{2}\left(z_{1},z_{2}\right)\right)\\
&=&\hat{r}_{2}\tilde{\mu}_{2}\hat{\mu}_{2}+\tilde{\mu}_{2}\hat{\mu}_{2}\hat{R}_{2}^{(2)}\left(1\right)+\hat{r}_{2}\hat{\mu}_{2}F_{2}^{(0,1)}+
\frac{\tilde{\mu}_{2}\hat{\mu}_{2}}{1-\hat{\mu}_{2}}\hat{F}_{2}^{(0,1)}+2\hat{r}_{2}\frac{\tilde{\mu}_{2}\hat{\mu}_{2}}{1-\hat{\mu}_{2}}\hat{F}_{2}^{(0,1)}+\tilde{\mu}_{2}\hat{\mu}_{2}\hat{\theta}_{2}^{(2)}\left(1\right)\hat{F}_{2}^{(0,1)}\\
&+&\frac{\hat{\mu}_{2}}{1-\hat{\mu}_{2}}F_{2}^{(0,1)}\hat{F}_{2}^{(1,0)}+\tilde{\mu}_{2}\hat{\mu}_{2}\left(\frac{1}{1-\hat{\mu}_{2}}\right)\hat{F}_{2}^{(0,2)}.
\end{eqnarray*}
%___________________________________________________________________________________________
%\subsubsection{Mixtas para $w_{1}$:}
%___________________________________________________________________________________________

%9/41
\item \begin{eqnarray*} &&\frac{\partial}{\partial
z_1}\frac{\partial}{\partial
w_1}\left(\hat{R}_{2}\left(P_{1}\left(z_{1}\right)\tilde{P}_{2}\left(z_{2}\right)\hat{P}_{1}\left(w_{1}\right)\hat{P}_{2}\left(w_{2}\right)\right)\hat{F}_{2}\left(w_{1},\hat{\theta}_{2}\left(P_{1}\left(z_{1}\right)\tilde{P}_{2}\left(z_{2}\right)\hat{P}_{1}\left(w_{1}\right)\right)\right)F_{2}\left(z_{1},z_{2}\right)\right)\\
&=&\hat{r}_{2}\mu_{1}\hat{\mu}_{1}+\mu_{1}\hat{\mu}_{1}\hat{R}_{2}^{(2)}\left(1\right)+\hat{r}_{2}\frac{\mu_{1}\hat{\mu}_{1}}{1-\hat{\mu}_{2}}\hat{F}_{2}^{(0,1)}+\hat{r}_{2}\hat{\mu}_{1}F_{2}^{(1,0)}+\hat{r}_{2}\mu_{1}\hat{F}_{2}^{(1,0)}+F_{2}^{(1,0)}\hat{F}_{2}^{(1,0)}+\frac{\mu_{1}}{1-\hat{\mu}_{2}}\hat{F}_{2}^{(1,1)}.
\end{eqnarray*}


%10/42
\item \begin{eqnarray*} &&\frac{\partial}{\partial
z_2}\frac{\partial}{\partial
w_1}\left(\hat{R}_{2}\left(P_{1}\left(z_{1}\right)\tilde{P}_{2}\left(z_{2}\right)\hat{P}_{1}\left(w_{1}\right)\hat{P}_{2}\left(w_{2}\right)\right)\hat{F}_{2}\left(w_{1},\hat{\theta}_{2}\left(P_{1}\left(z_{1}\right)\tilde{P}_{2}\left(z_{2}\right)\hat{P}_{1}\left(w_{1}\right)\right)\right)F_{2}\left(z_{1},z_{2}\right)\right)\\
&=&\hat{r}_{2}\tilde{\mu}_{2}\hat{\mu}_{1}+\tilde{\mu}_{2}\hat{\mu}_{1}\hat{R}_{2}^{(2)}\left(1\right)+\hat{r}_{2}\hat{\mu}_{1}F_{2}^{(0,1)}+
\hat{r}_{2}\frac{\tilde{\mu}_{2}\hat{\mu}_{1}}{1-\hat{\mu}_{2}}\hat{F}_{2}^{(0,1)}+\hat{r}_{2}\tilde{\mu}_{2}\hat{F}_{2}^{(1,0)}+F_{2}^{(0,1)}\hat{F}_{2}^{(1,0)}+\frac{\tilde{\mu}_{2}}{1-\hat{\mu}_{2}}\hat{F}_{2}^{(1,1)}.
\end{eqnarray*}


%11/43
\item \begin{eqnarray*} &&\frac{\partial}{\partial
w_1}\frac{\partial}{\partial
w_1}\left(\hat{R}_{2}\left(P_{1}\left(z_{1}\right)\tilde{P}_{2}\left(z_{2}\right)\hat{P}_{1}\left(w_{1}\right)\hat{P}_{2}\left(w_{2}\right)\right)\hat{F}_{2}\left(w_{1},\hat{\theta}_{2}\left(P_{1}\left(z_{1}\right)\tilde{P}_{2}\left(z_{2}\right)\hat{P}_{1}\left(w_{1}\right)\right)\right)F_{2}\left(z_{1},z_{2}\right)\right)\\
&=&\hat{r}_{2}\hat{P}_{1}^{(2)}\left(1\right)+\hat{\mu}_{1}^{2}\hat{R}_{2}^{(2)}\left(1\right)+2\hat{r}_{2}\hat{\mu}_{1}\hat{F}_{2}^{(1,0)}
+\hat{F}_{2}^{(2,0)}.
\end{eqnarray*}


%12/44
\item \begin{eqnarray*} &&\frac{\partial}{\partial
w_2}\frac{\partial}{\partial
w_1}\left(\hat{R}_{2}\left(P_{1}\left(z_{1}\right)\tilde{P}_{2}\left(z_{2}\right)\hat{P}_{1}\left(w_{1}\right)\hat{P}_{2}\left(w_{2}\right)\right)\hat{F}_{2}\left(w_{1},\hat{\theta}_{2}\left(P_{1}\left(z_{1}\right)\tilde{P}_{2}\left(z_{2}\right)\hat{P}_{1}\left(w_{1}\right)\right)\right)F_{2}\left(z_{1},z_{2}\right)\right)\\
&=&\hat{r}_{2}\hat{\mu}_{1}\hat{\mu}_{2}+\hat{\mu}_{1}\hat{\mu}_{2}\hat{R}_{2}^{(2)}\left(1\right)+
\hat{r}_{2}\frac{\hat{\mu}_{2}\hat{\mu}_{1}}{1-\hat{\mu}_{2}}\hat{F}_{2}^{(0,1)}
+\hat{r}_{2}\hat{\mu}_{2}\hat{F}_{2}^{(1,0)}+\frac{\hat{\mu}_{2}}{1-\hat{\mu}_{2}}\hat{F}_{2}^{(1,1)}.
\end{eqnarray*}
%___________________________________________________________________________________________
%\subsubsection{Mixtas para $w_{2}$:}
%___________________________________________________________________________________________
%13/45
\item \begin{eqnarray*} &&\frac{\partial}{\partial
z_1}\frac{\partial}{\partial
w_2}\left(\hat{R}_{2}\left(P_{1}\left(z_{1}\right)\tilde{P}_{2}\left(z_{2}\right)\hat{P}_{1}\left(w_{1}\right)\hat{P}_{2}\left(w_{2}\right)\right)\hat{F}_{2}\left(w_{1},\hat{\theta}_{2}\left(P_{1}\left(z_{1}\right)\tilde{P}_{2}\left(z_{2}\right)\hat{P}_{1}\left(w_{1}\right)\right)\right)F_{2}\left(z_{1},z_{2}\right)\right)\\
&=&\hat{r}_{2}\mu_{1}\hat{\mu}_{2}+\mu_{1}\hat{\mu}_{2}\hat{R}_{2}^{(2)}\left(1\right)+
\frac{\mu_{1}\hat{\mu}_{2}}{1-\hat{\mu}_{2}}\hat{F}_{2}^{(0,1)} +2\hat{r}_{2}\frac{\mu_{1}\hat{\mu}_{2}}{1-\hat{\mu}_{2}}\hat{F}_{2}^{(0,1)}\\
&+&\mu_{1}\hat{\mu}_{2}\hat{\theta}_{2}^{(2)}\left(1\right)\hat{F}_{2}^{(0,1)}+\mu_{1}\hat{\mu}_{2}\left(\frac{1}{1-\hat{\mu}_{2}}\right)^{2}\hat{F}_{2}^{(0,2)}+\hat{r}_{2}\hat{\mu}_{2}F_{2}^{(1,0)}+\frac{\hat{\mu}_{2}}{1-\hat{\mu}_{2}}\hat{F}_{2}^{(0,1)}F_{2}^{(1,0)}.\end{eqnarray*}


%14/46
\item \begin{eqnarray*} &&\frac{\partial}{\partial
z_2}\frac{\partial}{\partial
w_2}\left(\hat{R}_{2}\left(P_{1}\left(z_{1}\right)\tilde{P}_{2}\left(z_{2}\right)\hat{P}_{1}\left(w_{1}\right)\hat{P}_{2}\left(w_{2}\right)\right)\hat{F}_{2}\left(w_{1},\hat{\theta}_{2}\left(P_{1}\left(z_{1}\right)\tilde{P}_{2}\left(z_{2}\right)\hat{P}_{1}\left(w_{1}\right)\right)\right)F_{2}\left(z_{1},z_{2}\right)\right)\\
&=&\hat{r}_{2}\tilde{\mu}_{2}\hat{\mu}_{2}+\tilde{\mu}_{2}\hat{\mu}_{2}\hat{R}_{2}^{(2)}\left(1\right)+\hat{r}_{2}\hat{\mu}_{2}F_{2}^{(0,1)}+\frac{\tilde{\mu}_{2}\hat{\mu}_{2}}{1-\hat{\mu}_{2}}\hat{F}_{2}^{(0,1)}+
2\hat{r}_{2}\frac{\tilde{\mu}_{2}\hat{\mu}_{2}}{1-\hat{\mu}_{2}}\hat{F}_{2}^{(0,1)}+\tilde{\mu}_{2}\hat{\mu}_{2}\hat{\theta}_{2}^{(2)}\left(1\right)\hat{F}_{2}^{(0,1)}\\
&+&\frac{\hat{\mu}_{2}}{1-\hat{\mu}_{2}}\hat{F}_{2}^{(0,1)}F_{2}^{(0,1)}+\tilde{\mu}_{2}\hat{\mu}_{2}\left(\frac{1}{1-\hat{\mu}_{2}}\right)^{2}\hat{F}_{2}^{(0,2)}.
\end{eqnarray*}

%15/47

\item \begin{eqnarray*} &&\frac{\partial}{\partial
w_1}\frac{\partial}{\partial
w_2}\left(\hat{R}_{2}\left(P_{1}\left(z_{1}\right)\tilde{P}_{2}\left(z_{2}\right)\hat{P}_{1}\left(w_{1}\right)\hat{P}_{2}\left(w_{2}\right)\right)\hat{F}_{2}\left(w_{1},\hat{\theta}_{2}\left(P_{1}\left(z_{1}\right)\tilde{P}_{2}\left(z_{2}\right)\hat{P}_{1}\left(w_{1}\right)\right)\right)F_{2}\left(z_{1},z_{2}\right)\right)\\
&=&\hat{r}_{2}\hat{\mu}_{1}\hat{\mu}_{2}+\hat{\mu}_{1}\hat{\mu}_{2}\hat{R}_{2}^{(2)}\left(1\right)+
\hat{r}_{2}\frac{\hat{\mu}_{1}\hat{\mu}_{2}}{1-\hat{\mu}_{2}}\hat{F}_{2}^{(0,1)}+
\hat{r}_{2}\hat{\mu}_{2}\hat{F}_{2}^{(1,0)}+\frac{\hat{\mu}_{2}}{1-\hat{\mu}_{2}}\hat{F}_{2}^{(1,1)}.
\end{eqnarray*}

%16/48
\item \begin{eqnarray*} &&\frac{\partial}{\partial
w_2}\frac{\partial}{\partial
w_2}\left(\hat{R}_{2}\left(P_{1}\left(z_{1}\right)\tilde{P}_{2}\left(z_{2}\right)\hat{P}_{1}\left(w_{1}\right)\hat{P}_{2}\left(w_{2}\right)\right)\hat{F}_{2}\left(w_{1},\hat{\theta}_{2}\left(P_{1}\left(z_{1}\right)\tilde{P}_{2}\left(z_{2}\right)\hat{P}_{1}\left(w_{1}\right)\right)\right)F_{2}\left(z_{1},z_{2};\zeta_{2}\right)\right)\\
&=&\hat{r}_{2}P_{2}^{(2)}\left(1\right)+\hat{\mu}_{2}^{2}\hat{R}_{2}^{(2)}\left(1\right)+2\hat{r}_{2}\frac{\hat{\mu}_{2}^{2}}{1-\hat{\mu}_{2}}\hat{F}_{2}^{(0,1)}+\frac{1}{1-\hat{\mu}_{2}}\hat{P}_{2}^{(2)}\left(1\right)\hat{F}_{2}^{(0,1)}+\hat{\mu}_{2}^{2}\hat{\theta}_{2}^{(2)}\left(1\right)\hat{F}_{2}^{(0,1)}\\
&+&\left(\frac{\hat{\mu}_{2}}{1-\hat{\mu}_{2}}\right)^{2}\hat{F}_{2}^{(0,2)}.
\end{eqnarray*}


\end{enumerate}



%___________________________________________________________________________________________
%
%\subsection{Derivadas de Segundo Orden para $\hat{F}_{2}$}
%___________________________________________________________________________________________
\begin{enumerate}
%___________________________________________________________________________________________
%\subsubsection{Mixtas para $z_{1}$:}
%___________________________________________________________________________________________
%1/49

\item \begin{eqnarray*} &&\frac{\partial}{\partial
z_1}\frac{\partial}{\partial
z_1}\left(\hat{R}_{1}\left(P_{1}\left(z_{1}\right)\tilde{P}_{2}\left(z_{2}\right)\hat{P}_{1}\left(w_{1}\right)\hat{P}_{2}\left(w_{2}\right)\right)\hat{F}_{1}\left(\hat{\theta}_{1}\left(P_{1}\left(z_{1}\right)\tilde{P}_{2}\left(z_{2}\right)
\hat{P}_{2}\left(w_{2}\right)\right),w_{2}\right)F_{1}\left(z_{1},z_{2}\right)\right)\\
&=&\hat{r}_{1}P_{1}^{(2)}\left(1\right)+
\mu_{1}^{2}\hat{R}_{1}^{(2)}\left(1\right)+
2\hat{r}_{1}\mu_{1}F_{1}^{(1,0)}+
2\hat{r}_{1}\frac{\mu_{1}^{2}}{1-\hat{\mu}_{1}}\hat{F}_{1}^{(1,0)}+
\frac{1}{1-\hat{\mu}_{1}}P_{1}^{(2)}\left(1\right)\hat{F}_{1}^{(1,0)}+\mu_{1}^{2}\hat{\theta}_{1}^{(2)}\left(1\right)\hat{F}_{1}^{(1,0)}\\
&+&2\frac{\mu_{1}}{1-\hat{\mu}_{1}}\hat{F}_{1}^{(1,0)}F_{1}^{(1,0)}+F_{1}^{(2,0)}
+\left(\frac{\mu_{1}}{1-\hat{\mu}_{1}}\right)^{2}\hat{F}_{1}^{(2,0)}.
\end{eqnarray*}

%2/50

\item \begin{eqnarray*} &&\frac{\partial}{\partial
z_2}\frac{\partial}{\partial
z_1}\left(\hat{R}_{1}\left(P_{1}\left(z_{1}\right)\tilde{P}_{2}\left(z_{2}\right)\hat{P}_{1}\left(w_{1}\right)\hat{P}_{2}\left(w_{2}\right)\right)\hat{F}_{1}\left(\hat{\theta}_{1}\left(P_{1}\left(z_{1}\right)\tilde{P}_{2}\left(z_{2}\right)
\hat{P}_{2}\left(w_{2}\right)\right),w_{2}\right)F_{1}\left(z_{1},z_{2}\right)\right)\\
&=&\hat{r}_{1}\mu_{1}\tilde{\mu}_{2}+\mu_{1}\tilde{\mu}_{2}\hat{R}_{1}^{(2)}\left(1\right)+
\hat{r}_{1}\mu_{1}F_{1}^{(0,1)}+\tilde{\mu}_{2}\hat{r}_{1}F_{1}^{(1,0)}+
\frac{\mu_{1}\tilde{\mu}_{2}}{1-\hat{\mu}_{1}}\hat{F}_{1}^{(1,0)}+2\hat{r}_{1}\frac{\mu_{1}\tilde{\mu}_{2}}{1-\hat{\mu}_{1}}\hat{F}_{1}^{(1,0)}\\
&+&\mu_{1}\tilde{\mu}_{2}\hat{\theta}_{1}^{(2)}\left(1\right)\hat{F}_{1}^{(1,0)}+
\frac{\mu_{1}}{1-\hat{\mu}_{1}}\hat{F}_{1}^{(1,0)}F_{1}^{(0,1)}+
\frac{\tilde{\mu}_{2}}{1-\hat{\mu}_{1}}\hat{F}_{1}^{(1,0)}F_{1}^{(1,0)}+
F_{1}^{(1,1)}\\
&+&\mu_{1}\tilde{\mu}_{2}\left(\frac{1}{1-\hat{\mu}_{1}}\right)^{2}\hat{F}_{1}^{(2,0)}.
\end{eqnarray*}

%3/51

\item \begin{eqnarray*} &&\frac{\partial}{\partial
w_1}\frac{\partial}{\partial
z_1}\left(\hat{R}_{1}\left(P_{1}\left(z_{1}\right)\tilde{P}_{2}\left(z_{2}\right)\hat{P}_{1}\left(w_{1}\right)\hat{P}_{2}\left(w_{2}\right)\right)\hat{F}_{1}\left(\hat{\theta}_{1}\left(P_{1}\left(z_{1}\right)\tilde{P}_{2}\left(z_{2}\right)
\hat{P}_{2}\left(w_{2}\right)\right),w_{2}\right)F_{1}\left(z_{1},z_{2}\right)\right)\\
&=&\hat{r}_{1}\mu_{1}\hat{\mu}_{1}+\mu_{1}\hat{\mu}_{1}\hat{R}_{1}^{(2)}\left(1\right)+\hat{r}_{1}\hat{\mu}_{1}F_{1}^{(1,0)}+
\hat{r}_{1}\frac{\mu_{1}\hat{\mu}_{1}}{1-\hat{\mu}_{1}}\hat{F}_{1}^{(1,0)}.
\end{eqnarray*}

%4/52

\item \begin{eqnarray*} &&\frac{\partial}{\partial
w_2}\frac{\partial}{\partial
z_1}\left(\hat{R}_{1}\left(P_{1}\left(z_{1}\right)\tilde{P}_{2}\left(z_{2}\right)\hat{P}_{1}\left(w_{1}\right)\hat{P}_{2}\left(w_{2}\right)\right)\hat{F}_{1}\left(\hat{\theta}_{1}\left(P_{1}\left(z_{1}\right)\tilde{P}_{2}\left(z_{2}\right)
\hat{P}_{2}\left(w_{2}\right)\right),w_{2}\right)F_{1}\left(z_{1},z_{2}\right)\right)\\
&=&\hat{r}_{1}\mu_{1}\hat{\mu}_{2}+\mu_{1}\hat{\mu}_{2}\hat{R}_{1}^{(2)}\left(1\right)+\hat{r}_{1}\hat{\mu}_{2}F_{1}^{(1,0)}+\frac{\mu_{1}\hat{\mu}_{2}}{1-\hat{\mu}_{1}}\hat{F}_{1}^{(1,0)}+\hat{r}_{1}\frac{\mu_{1}\hat{\mu}_{2}}{1-\hat{\mu}_{1}}\hat{F}_{1}^{(1,0)}+\mu_{1}\hat{\mu}_{2}\hat{\theta}_{1}^{(2)}\left(1\right)\hat{F}_{1}^{(1,0)}\\
&+&\hat{r}_{1}\mu_{1}\left(\hat{F}_{1}^{(0,1)}+\frac{\hat{\mu}_{2}}{1-\hat{\mu}_{1}}\hat{F}_{1}^{(1,0)}\right)+F_{1}^{(1,0)}\left(\hat{F}_{1}^{(0,1)}+\frac{\hat{\mu}_{2}}{1-\hat{\mu}_{1}}\hat{F}_{1}^{(1,0)}\right)+\frac{\mu_{1}}{1-\hat{\mu}_{1}}\left(\hat{F}_{1}^{(1,1)}+\frac{\hat{\mu}_{2}}{1-\hat{\mu}_{1}}\hat{F}_{1}^{(2,0)}\right).
\end{eqnarray*}
%___________________________________________________________________________________________
%\subsubsection{Mixtas para $z_{2}$:}
%___________________________________________________________________________________________
%5/53

\item \begin{eqnarray*} &&\frac{\partial}{\partial
z_1}\frac{\partial}{\partial
z_2}\left(\hat{R}_{1}\left(P_{1}\left(z_{1}\right)\tilde{P}_{2}\left(z_{2}\right)\hat{P}_{1}\left(w_{1}\right)\hat{P}_{2}\left(w_{2}\right)\right)\hat{F}_{1}\left(\hat{\theta}_{1}\left(P_{1}\left(z_{1}\right)\tilde{P}_{2}\left(z_{2}\right)
\hat{P}_{2}\left(w_{2}\right)\right),w_{2}\right)F_{1}\left(z_{1},z_{2}\right)\right)\\
&=&\hat{r}_{1}\mu_{1}\tilde{\mu}_{2}+\mu_{1}\tilde{\mu}_{2}\hat{R}_{1}^{(2)}\left(1\right)+\hat{r}_{1}\mu_{1}F_{1}^{(0,1)}+\hat{r}_{1}\tilde{\mu}_{2}F_{1}^{(1,0)}+\frac{\mu_{1}\tilde{\mu}_{2}}{1-\hat{\mu}_{1}}\hat{F}_{1}^{(1,0)}+2\hat{r}_{1}\frac{\mu_{1}\tilde{\mu}_{2}}{1-\hat{\mu}_{1}}\hat{F}_{1}^{(1,0)}\\
&+&\mu_{1}\tilde{\mu}_{2}\hat{\theta}_{1}^{(2)}\left(1\right)\hat{F}_{1}^{(1,0)}+\frac{\mu_{1}}{1-\hat{\mu}_{1}}\hat{F}_{1}^{(1,0)}F_{1}^{(0,1)}+\frac{\tilde{\mu}_{2}}{1-\hat{\mu}_{1}}\hat{F}_{1}^{(1,0)}F_{1}^{(1,0)}+F_{1}^{(1,1)}+\mu_{1}\tilde{\mu}_{2}\left(\frac{1}{1-\hat{\mu}_{1}}\right)^{2}\hat{F}_{1}^{(2,0)}.
\end{eqnarray*}

%6/54
\item \begin{eqnarray*} &&\frac{\partial}{\partial
z_2}\frac{\partial}{\partial
z_2}\left(\hat{R}_{1}\left(P_{1}\left(z_{1}\right)\tilde{P}_{2}\left(z_{2}\right)\hat{P}_{1}\left(w_{1}\right)\hat{P}_{2}\left(w_{2}\right)\right)\hat{F}_{1}\left(\hat{\theta}_{1}\left(P_{1}\left(z_{1}\right)\tilde{P}_{2}\left(z_{2}\right)
\hat{P}_{2}\left(w_{2}\right)\right),w_{2}\right)F_{1}\left(z_{1},z_{2}\right)\right)\\
&=&\hat{r}_{1}\tilde{P}_{2}^{(2)}\left(1\right)+\tilde{\mu}_{2}^{2}\hat{R}_{1}^{(2)}\left(1\right)+2\hat{r}_{1}\tilde{\mu}_{2}F_{1}^{(0,1)}+ F_{1}^{(0,2)}+2\hat{r}_{1}\frac{\tilde{\mu}_{2}^{2}}{1-\hat{\mu}_{1}}\hat{F}_{1}^{(1,0)}+\frac{1}{1-\hat{\mu}_{1}}\tilde{P}_{2}^{(2)}\left(1\right)\hat{F}_{1}^{(1,0)}\\
&+&\tilde{\mu}_{2}^{2}\hat{\theta}_{1}^{(2)}\left(1\right)\hat{F}_{1}^{(1,0)}+2\frac{\tilde{\mu}_{2}}{1-\hat{\mu}_{1}}F^{(0,1)}\hat{F}_{1}^{(1,0)}+\left(\frac{\tilde{\mu}_{2}}{1-\hat{\mu}_{1}}\right)^{2}\hat{F}_{1}^{(2,0)}.
\end{eqnarray*}
%7/55

\item \begin{eqnarray*} &&\frac{\partial}{\partial
w_1}\frac{\partial}{\partial
z_2}\left(\hat{R}_{1}\left(P_{1}\left(z_{1}\right)\tilde{P}_{2}\left(z_{2}\right)\hat{P}_{1}\left(w_{1}\right)\hat{P}_{2}\left(w_{2}\right)\right)\hat{F}_{1}\left(\hat{\theta}_{1}\left(P_{1}\left(z_{1}\right)\tilde{P}_{2}\left(z_{2}\right)
\hat{P}_{2}\left(w_{2}\right)\right),w_{2}\right)F_{1}\left(z_{1},z_{2}\right)\right)\\
&=&\hat{r}_{1}\hat{\mu}_{1}\tilde{\mu}_{2}+\hat{\mu}_{1}\tilde{\mu}_{2}\hat{R}_{1}^{(2)}\left(1\right)+
\hat{r}_{1}\hat{\mu}_{1}F_{1}^{(0,1)}+\hat{r}_{1}\frac{\hat{\mu}_{1}\tilde{\mu}_{2}}{1-\hat{\mu}_{1}}\hat{F}_{1}^{(1,0)}.
\end{eqnarray*}
%8/56

\item \begin{eqnarray*} &&\frac{\partial}{\partial
w_2}\frac{\partial}{\partial
z_2}\left(\hat{R}_{1}\left(P_{1}\left(z_{1}\right)\tilde{P}_{2}\left(z_{2}\right)\hat{P}_{1}\left(w_{1}\right)\hat{P}_{2}\left(w_{2}\right)\right)\hat{F}_{1}\left(\hat{\theta}_{1}\left(P_{1}\left(z_{1}\right)\tilde{P}_{2}\left(z_{2}\right)
\hat{P}_{2}\left(w_{2}\right)\right),w_{2}\right)F_{1}\left(z_{1},z_{2}\right)\right)\\
&=&\hat{r}_{1}\tilde{\mu}_{2}\hat{\mu}_{2}+\hat{\mu}_{2}\tilde{\mu}_{2}\hat{R}_{1}^{(2)}\left(1\right)+\hat{\mu}_{2}\hat{R}_{1}^{(2)}\left(1\right)F_{1}^{(0,1)}+\frac{\hat{\mu}_{2}\tilde{\mu}_{2}}{1-\hat{\mu}_{1}}\hat{F}_{1}^{(1,0)}+
\hat{r}_{1}\frac{\hat{\mu}_{2}\tilde{\mu}_{2}}{1-\hat{\mu}_{1}}\hat{F}_{1}^{(1,0)}\\
&+&\hat{\mu}_{2}\tilde{\mu}_{2}\hat{\theta}_{1}^{(2)}\left(1\right)\hat{F}_{1}^{(1,0)}+\hat{r}_{1}\tilde{\mu}_{2}\left(\hat{F}_{1}^{(0,1)}+\frac{\hat{\mu}_{2}}{1-\hat{\mu}_{1}}\hat{F}_{1}^{(1,0)}\right)+F_{1}^{(0,1)}\left(\hat{F}_{1}^{(0,1)}+\frac{\hat{\mu}_{2}}{1-\hat{\mu}_{1}}\hat{F}_{1}^{(1,0)}\right)\\
&+&\frac{\tilde{\mu}_{2}}{1-\hat{\mu}_{1}}\left(\hat{F}_{1}^{(1,1)}+\frac{\hat{\mu}_{2}}{1-\hat{\mu}_{1}}\hat{F}_{1}^{(2,0)}\right).
\end{eqnarray*}
%___________________________________________________________________________________________
%\subsubsection{Mixtas para $w_{1}$:}
%___________________________________________________________________________________________
%9/57
\item \begin{eqnarray*} &&\frac{\partial}{\partial
z_1}\frac{\partial}{\partial
w_1}\left(\hat{R}_{1}\left(P_{1}\left(z_{1}\right)\tilde{P}_{2}\left(z_{2}\right)\hat{P}_{1}\left(w_{1}\right)\hat{P}_{2}\left(w_{2}\right)\right)\hat{F}_{1}\left(\hat{\theta}_{1}\left(P_{1}\left(z_{1}\right)\tilde{P}_{2}\left(z_{2}\right)
\hat{P}_{2}\left(w_{2}\right)\right),w_{2}\right)F_{1}\left(z_{1},z_{2}\right)\right)\\
&=&\hat{r}_{1}\mu_{1}\hat{\mu}_{1}+\mu_{1}\hat{\mu}_{1}\hat{R}_{1}^{(2)}\left(1\right)+\hat{r}_{1}\hat{\mu}_{1}F_{1}^{(1,0)}+\hat{r}_{1}\frac{\mu_{1}\hat{\mu}_{1}}{1-\hat{\mu}_{1}}\hat{F}_{1}^{(1,0)}.
\end{eqnarray*}
%10/58
\item \begin{eqnarray*} &&\frac{\partial}{\partial
z_2}\frac{\partial}{\partial
w_1}\left(\hat{R}_{1}\left(P_{1}\left(z_{1}\right)\tilde{P}_{2}\left(z_{2}\right)\hat{P}_{1}\left(w_{1}\right)\hat{P}_{2}\left(w_{2}\right)\right)\hat{F}_{1}\left(\hat{\theta}_{1}\left(P_{1}\left(z_{1}\right)\tilde{P}_{2}\left(z_{2}\right)
\hat{P}_{2}\left(w_{2}\right)\right),w_{2}\right)F_{1}\left(z_{1},z_{2}\right)\right)\\
&=&\hat{r}_{1}\tilde{\mu}_{2}\hat{\mu}_{1}+\tilde{\mu}_{2}\hat{\mu}_{1}\hat{R}_{1}^{(2)}\left(1\right)+\hat{r}_{1}\hat{\mu}_{1}F_{1}^{(0,1)}+\hat{r}_{1}\frac{\tilde{\mu}_{2}\hat{\mu}_{1}}{1-\hat{\mu}_{1}}\hat{F}_{1}^{(1,0)}.
\end{eqnarray*}
%11/59
\item \begin{eqnarray*} &&\frac{\partial}{\partial
w_1}\frac{\partial}{\partial
w_1}\left(\hat{R}_{1}\left(P_{1}\left(z_{1}\right)\tilde{P}_{2}\left(z_{2}\right)\hat{P}_{1}\left(w_{1}\right)\hat{P}_{2}\left(w_{2}\right)\right)\hat{F}_{1}\left(\hat{\theta}_{1}\left(P_{1}\left(z_{1}\right)\tilde{P}_{2}\left(z_{2}\right)
\hat{P}_{2}\left(w_{2}\right)\right),w_{2}\right)F_{1}\left(z_{1},z_{2}\right)\right)\\
&=&\hat{r}_{1}\hat{P}_{1}^{(2)}\left(1\right)+\hat{\mu}_{1}^{2}\hat{R}_{1}^{(2)}\left(1\right).
\end{eqnarray*}
%12/60
\item \begin{eqnarray*} &&\frac{\partial}{\partial
w_2}\frac{\partial}{\partial
w_1}\left(\hat{R}_{1}\left(P_{1}\left(z_{1}\right)\tilde{P}_{2}\left(z_{2}\right)\hat{P}_{1}\left(w_{1}\right)\hat{P}_{2}\left(w_{2}\right)\right)\hat{F}_{1}\left(\hat{\theta}_{1}\left(P_{1}\left(z_{1}\right)\tilde{P}_{2}\left(z_{2}\right)
\hat{P}_{2}\left(w_{2}\right)\right),w_{2}\right)F_{1}\left(z_{1},z_{2}\right)\right)\\
&=&\hat{r}_{1}\hat{\mu}_{2}\hat{\mu}_{1}+\hat{\mu}_{2}\hat{\mu}_{1}\hat{R}_{1}^{(2)}\left(1\right)+\hat{r}_{1}\hat{\mu}_{1}\left(\hat{F}_{1}^{(0,1)}+\frac{\hat{\mu}_{2}}{1-\hat{\mu}_{1}}\hat{F}_{1}^{(1,0)}\right).
\end{eqnarray*}
%___________________________________________________________________________________________
%\subsubsection{Mixtas para $w_{1}$:}
%___________________________________________________________________________________________
%13/61



\item \begin{eqnarray*} &&\frac{\partial}{\partial
z_1}\frac{\partial}{\partial
w_2}\left(\hat{R}_{1}\left(P_{1}\left(z_{1}\right)\tilde{P}_{2}\left(z_{2}\right)\hat{P}_{1}\left(w_{1}\right)\hat{P}_{2}\left(w_{2}\right)\right)\hat{F}_{1}\left(\hat{\theta}_{1}\left(P_{1}\left(z_{1}\right)\tilde{P}_{2}\left(z_{2}\right)
\hat{P}_{2}\left(w_{2}\right)\right),w_{2}\right)F_{1}\left(z_{1},z_{2}\right)\right)\\
&=&\hat{r}_{1}\mu_{1}\hat{\mu}_{2}+\mu_{1}\hat{\mu}_{2}\hat{R}_{1}^{(2)}\left(1\right)+\hat{r}_{1}\hat{\mu}_{2}F_{1}^{(1,0)}+
\hat{r}_{1}\frac{\mu_{1}\hat{\mu}_{2}}{1-\hat{\mu}_{1}}\hat{F}_{1}^{(1,0)}+\hat{r}_{1}\mu_{1}\left(\hat{F}_{1}^{(0,1)}+\frac{\hat{\mu}_{2}}{1-\hat{\mu}_{1}}\hat{F}_{1}^{(1,0)}\right)\\
&+&F_{1}^{(1,0)}\left(\hat{F}_{1}^{(0,1)}+\frac{\hat{\mu}_{2}}{1-\hat{\mu}_{1}}\hat{F}_{1}^{(1,0)}\right)+\frac{\mu_{1}\hat{\mu}_{2}}{1-\hat{\mu}_{1}}\hat{F}_{1}^{(1,0)}+\mu_{1}\hat{\mu}_{2}\hat{\theta}_{1}^{(2)}\left(1\right)\hat{F}_{1}^{(1,0)}+\frac{\mu_{1}}{1-\hat{\mu}_{1}}\hat{F}_{1}^{(1,1)}\\
&+&\mu_{1}\hat{\mu}_{2}\left(\frac{1}{1-\hat{\mu}_{1}}\right)^{2}\hat{F}_{1}^{(2,0)}.
\end{eqnarray*}

%14/62
\item \begin{eqnarray*} &&\frac{\partial}{\partial
z_2}\frac{\partial}{\partial
w_2}\left(\hat{R}_{1}\left(P_{1}\left(z_{1}\right)\tilde{P}_{2}\left(z_{2}\right)\hat{P}_{1}\left(w_{1}\right)\hat{P}_{2}\left(w_{2}\right)\right)\hat{F}_{1}\left(\hat{\theta}_{1}\left(P_{1}\left(z_{1}\right)\tilde{P}_{2}\left(z_{2}\right)
\hat{P}_{2}\left(w_{2}\right)\right),w_{2}\right)F_{1}\left(z_{1},z_{2}\right)\right)\\
&=&\hat{r}_{1}\tilde{\mu}_{2}\hat{\mu}_{2}+\tilde{\mu}_{2}\hat{\mu}_{2}\hat{R}_{1}^{(2)}\left(1\right)+\hat{r}_{1}\hat{\mu}_{2}F_{1}^{(0,1)}+\hat{r}_{1}\frac{\tilde{\mu}_{2}\hat{\mu}_{2}}{1-\hat{\mu}_{1}}\hat{F}_{1}^{(1,0)}+\hat{r}_{1}\tilde{\mu}_{2}\left(\hat{F}_{1}^{(0,1)}+\frac{\hat{\mu}_{2}}{1-\hat{\mu}_{1}}\hat{F}_{1}^{(1,0)}\right)\\
&+&F_{1}^{(0,1)}\left(\hat{F}_{1}^{(0,1)}+\frac{\hat{\mu}_{2}}{1-\hat{\mu}_{1}}\hat{F}_{1}^{(1,0)}\right)+\frac{\tilde{\mu}_{2}\hat{\mu}_{2}}{1-\hat{\mu}_{1}}\hat{F}_{1}^{(1,0)}+\tilde{\mu}_{2}\hat{\mu}_{2}\hat{\theta}_{1}^{(2)}\left(1\right)\hat{F}_{1}^{(1,0)}+\frac{\tilde{\mu}_{2}}{1-\hat{\mu}_{1}}\hat{F}_{1}^{(1,1)}\\
&+&\tilde{\mu}_{2}\hat{\mu}_{2}\left(\frac{1}{1-\hat{\mu}_{1}}\right)^{2}\hat{F}_{1}^{(2,0)}.
\end{eqnarray*}

%15/63

\item \begin{eqnarray*} &&\frac{\partial}{\partial
w_1}\frac{\partial}{\partial
w_2}\left(\hat{R}_{1}\left(P_{1}\left(z_{1}\right)\tilde{P}_{2}\left(z_{2}\right)\hat{P}_{1}\left(w_{1}\right)\hat{P}_{2}\left(w_{2}\right)\right)\hat{F}_{1}\left(\hat{\theta}_{1}\left(P_{1}\left(z_{1}\right)\tilde{P}_{2}\left(z_{2}\right)
\hat{P}_{2}\left(w_{2}\right)\right),w_{2}\right)F_{1}\left(z_{1},z_{2}\right)\right)\\
&=&\hat{r}_{1}\hat{\mu}_{2}\hat{\mu}_{1}+\hat{\mu}_{2}\hat{\mu}_{1}\hat{R}_{1}^{(2)}\left(1\right)+\hat{r}_{1}\hat{\mu}_{1}\left(\hat{F}_{1}^{(0,1)}+\frac{\hat{\mu}_{2}}{1-\hat{\mu}_{1}}\hat{F}_{1}^{(1,0)}\right).
\end{eqnarray*}

%16/64

\item \begin{eqnarray*} &&\frac{\partial}{\partial
w_2}\frac{\partial}{\partial
w_2}\left(\hat{R}_{1}\left(P_{1}\left(z_{1}\right)\tilde{P}_{2}\left(z_{2}\right)\hat{P}_{1}\left(w_{1}\right)\hat{P}_{2}\left(w_{2}\right)\right)\hat{F}_{1}\left(\hat{\theta}_{1}\left(P_{1}\left(z_{1}\right)\tilde{P}_{2}\left(z_{2}\right)
\hat{P}_{2}\left(w_{2}\right)\right),w_{2}\right)F_{1}\left(z_{1},z_{2}\right)\right)\\
&=&\hat{r}_{1}\hat{P}_{2}^{(2)}\left(1\right)+\hat{\mu}_{2}^{2}\hat{R}_{1}^{(2)}\left(1\right)+
2\hat{r}_{1}\hat{\mu}_{2}\left(\hat{F}_{1}^{(0,1)}+\frac{\hat{\mu}_{2}}{1-\hat{\mu}_{1}}\hat{F}_{1}^{(1,0)}\right)+
\hat{F}_{1}^{(0,2)}+\frac{1}{1-\hat{\mu}_{1}}\hat{P}_{2}^{(2)}\left(1\right)\hat{F}_{1}^{(1,0)}\\
&+&\hat{\mu}_{2}^{2}\hat{\theta}_{1}^{(2)}\left(1\right)\hat{F}_{1}^{(1,0)}+\frac{\hat{\mu}_{2}}{1-\hat{\mu}_{1}}\hat{F}_{1}^{(1,1)}+\frac{\hat{\mu}_{2}}{1-\hat{\mu}_{1}}\left(\hat{F}_{1}^{(1,1)}+\frac{\hat{\mu}_{2}}{1-\hat{\mu}_{1}}\hat{F}_{1}^{(2,0)}\right).
\end{eqnarray*}
%_________________________________________________________________________________________________________
%
%_________________________________________________________________________________________________________

\end{enumerate}




Las ecuaciones que determinan los segundos momentos de las longitudes de las colas de los dos sistemas se pueden ver en \href{http://sitio.expresauacm.org/s/carlosmartinez/wp-content/uploads/sites/13/2014/01/SegundosMomentos.pdf}{este sitio}

%\url{http://ubuntu_es_el_diablo.org},\href{http://www.latex-project.org/}{latex project}

%http://sitio.expresauacm.org/s/carlosmartinez/wp-content/uploads/sites/13/2014/01/SegundosMomentos.jpg
%http://sitio.expresauacm.org/s/carlosmartinez/wp-content/uploads/sites/13/2014/01/SegundosMomentos.pdf




%_____________________________________________________________________________________
%Distribuci\'on del n\'umero de usuaruios que pasan del sistema 1 al sistema 2
%_____________________________________________________________________________________
\section*{Ap\'endice B}
%________________________________________________________________________________________
%
%________________________________________________________________________________________
\subsection*{Distribuci\'on para los usuarios de traslado}
%________________________________________________________________________________________
Se puede demostrar que
\begin{equation}
\frac{d^{k}}{dy}\left(\frac{\lambda +\mu}{\lambda
+\mu-y}\right)=\frac{k!}{\left(\lambda+\mu\right)^{k}}
\end{equation}



\begin{Prop}
Sea $\tau$ variable aleatoria no negativa con distribuci\'on exponencial con media $\mu$, y sea $L\left(t\right)$ proceso
Poisson con par\'ametro $\lambda$. Entonces
\begin{equation}
\prob\left\{L\left(\tau\right)=k\right\}=f_{L\left(\tau\right)}\left(k\right)=\left(\frac{\mu}{\lambda
+\mu}\right)\left(\frac{\lambda}{\lambda+\mu}\right)^{k}.
\end{equation}
Adem\'as

\begin{eqnarray}
\esp\left[L\left(\tau\right)\right]&=&\frac{\lambda}{\mu}\\
\esp\left[\left(L\left(\tau\right)\right)^{2}\right]&=&\frac{\lambda}{\mu}\left(2\frac{\lambda}{\mu}+1\right)\\
V\left[L\left(\tau\right)\right]&=&\frac{\lambda}{\mu}\left(\frac{\lambda}{\mu}+1\right).
\end{eqnarray}
\end{Prop}

\begin{Proof}
A saber, para $k$ fijo se tiene que

\begin{eqnarray*}
\prob\left\{L\left(\tau\right)=k\right\}&=&\prob\left\{L\left(\tau\right)=k,\tau\in\left(0,\infty\right)\right\}\\
&=&\int_{0}^{\infty}\prob\left\{L\left(\tau\right)=k,\tau=y\right\}f_{\tau}\left(y\right)dy=\int_{0}^{\infty}\prob\left\{L\left(y\right)=k\right\}f_{\tau}\left(y\right)dy\\
&=&\int_{0}^{\infty}\frac{e^{-\lambda
y}}{k!}\left(\lambda y\right)^{k}\left(\mu e^{-\mu
y}\right)dy=\frac{\lambda^{k}\mu}{k!}\int_{0}^{\infty}y^{k}e^{-\left(\mu+\lambda\right)y}dy\\
&=&\frac{\lambda^{k}\mu}{\left(\lambda
+\mu\right)k!}\int_{0}^{\infty}y^{k}\left(\lambda+\mu\right)e^{-\left(\lambda+\mu\right)y}dy=\frac{\lambda^{k}\mu}{\left(\lambda
+\mu\right)k!}\int_{0}^{\infty}y^{k}f_{Y}\left(y\right)dy\\
&=&\frac{\lambda^{k}\mu}{\left(\lambda
+\mu\right)k!}\esp\left[Y^{k}\right]=\frac{\lambda^{k}\mu}{\left(\lambda
+\mu\right)k!}\frac{d^{k}}{dy}\left(\frac{\lambda
+\mu}{\lambda
+\mu-y}\right)|_{y=0}\\
&=&\frac{\lambda^{k}\mu}{\left(\lambda
+\mu\right)k!}\frac{k!}{\left(\lambda+\mu\right)^{k}}=\left(\frac{\mu}{\lambda
+\mu}\right)\left(\frac{\lambda}{\lambda+\mu}\right)^{k}.\\
\end{eqnarray*}


Adem\'as
\begin{eqnarray*}
\sum_{k=0}^{\infty}\prob\left\{L\left(\tau\right)=k\right\}&=&\sum_{k=0}^{\infty}\left(\frac{\mu}{\lambda
+\mu}\right)\left(\frac{\lambda}{\lambda+\mu}\right)^{k}=\frac{\mu}{\lambda
+\mu}\sum_{k=0}^{\infty}\left(\frac{\lambda}{\lambda+\mu}\right)^{k}\\
&=&\frac{\mu}{\lambda
+\mu}\left(\frac{1}{1-\frac{\lambda}{\lambda+\mu}}\right)=\frac{\mu}{\lambda
+\mu}\left(\frac{\lambda+\mu}{\mu}\right)\\
&=&1.\\
\end{eqnarray*}

determinemos primero la esperanza de
$L\left(\tau\right)$:


\begin{eqnarray*}
\esp\left[L\left(\tau\right)\right]&=&\sum_{k=0}^{\infty}k\prob\left\{L\left(\tau\right)=k\right\}=\sum_{k=0}^{\infty}k\left(\frac{\mu}{\lambda
+\mu}\right)\left(\frac{\lambda}{\lambda+\mu}\right)^{k}\\
&=&\left(\frac{\mu}{\lambda
+\mu}\right)\sum_{k=0}^{\infty}k\left(\frac{\lambda}{\lambda+\mu}\right)^{k}=\left(\frac{\mu}{\lambda
+\mu}\right)\left(\frac{\lambda}{\lambda+\mu}\right)\sum_{k=1}^{\infty}k\left(\frac{\lambda}{\lambda+\mu}\right)^{k-1}\\
&=&\frac{\mu\lambda}{\left(\lambda
+\mu\right)^{2}}\left(\frac{1}{1-\frac{\lambda}{\lambda+\mu}}\right)^{2}=\frac{\mu\lambda}{\left(\lambda
+\mu\right)^{2}}\left(\frac{\lambda+\mu}{\mu}\right)^{2}\\
&=&\frac{\lambda}{\mu}.
\end{eqnarray*}

Ahora su segundo momento:

\begin{eqnarray*}
\esp\left[\left(L\left(\tau\right)\right)^{2}\right]&=&\sum_{k=0}^{\infty}k^{2}\prob\left\{L\left(\tau\right)=k\right\}=\sum_{k=0}^{\infty}k^{2}\left(\frac{\mu}{\lambda
+\mu}\right)\left(\frac{\lambda}{\lambda+\mu}\right)^{k}\\
&=&\left(\frac{\mu}{\lambda
+\mu}\right)\sum_{k=0}^{\infty}k^{2}\left(\frac{\lambda}{\lambda+\mu}\right)^{k}=
\frac{\mu\lambda}{\left(\lambda
+\mu\right)^{2}}\sum_{k=2}^{\infty}\left(k-1\right)^{2}\left(\frac{\lambda}{\lambda+\mu}\right)^{k-2}\\
&=&\frac{\mu\lambda}{\left(\lambda
+\mu\right)^{2}}\left(\frac{\frac{\lambda}{\lambda+\mu}+1}{\left(\frac{\lambda}{\lambda+\mu}-1\right)^{3}}\right)=\frac{\mu\lambda}{\left(\lambda
+\mu\right)^{2}}\left(-\frac{\frac{2\lambda+\mu}{\lambda+\mu}}{\left(-\frac{\mu}{\lambda+\mu}\right)^{3}}\right)\\
&=&\frac{\mu\lambda}{\left(\lambda
+\mu\right)^{2}}\left(\frac{2\lambda+\mu}{\lambda+\mu}\right)\left(\frac{\lambda+\mu}{\mu}\right)^{3}=\frac{\lambda\left(2\lambda
+\mu\right)}{\mu^{2}}\\
&=&\frac{\lambda}{\mu}\left(2\frac{\lambda}{\mu}+1\right).
\end{eqnarray*}

y por tanto

\begin{eqnarray*}
V\left[L\left(\tau\right)\right]&=&\frac{\lambda\left(2\lambda
+\mu\right)}{\mu^{2}}-\left(\frac{\lambda}{\mu}\right)^{2}=\frac{\lambda^{2}+\mu\lambda}{\mu^{2}}\\
&=&\frac{\lambda}{\mu}\left(\frac{\lambda}{\mu}+1\right).
\end{eqnarray*}
\end{Proof}

Ahora, determinemos la distribuci\'on del n\'umero de usuarios que
pasan de $\hat{Q}_{2}$ a $Q_{2}$ considerando dos pol\'iticas de
traslado en espec\'ifico:

\begin{enumerate}
\item Solamente pasa un usuario,

\item Se permite el paso de $k$ usuarios,
\end{enumerate}
una vez que son atendidos por el servidor en $\hat{Q}_{2}$.

\begin{description}


\item[Pol\'itica de un solo usuario:] Sea $R_{2}$ el n\'umero de
usuarios que llegan a $\hat{Q}_{2}$ al tiempo $t$, sea $R_{1}$ el
n\'umero de usuarios que pasan de $\hat{Q}_{2}$ a $Q_{2}$ al
tiempo $t$.
\end{description}


A saber:
\begin{eqnarray*}
\esp\left[R_{1}\right]&=&\sum_{y\geq0}\prob\left[R_{2}=y\right]\esp\left[R_{1}|R_{2}=y\right]\\
&=&\sum_{y\geq0}\prob\left[R_{2}=y\right]\sum_{x\geq0}x\prob\left[R_{1}=x|R_{2}=y\right]\\
&=&\sum_{y\geq0}\sum_{x\geq0}x\prob\left[R_{1}=x|R_{2}=y\right]\prob\left[R_{2}=y\right].\\
\end{eqnarray*}

Determinemos
\begin{equation}
\esp\left[R_{1}|R_{2}=y\right]=\sum_{x\geq0}x\prob\left[R_{1}=x|R_{2}=y\right].
\end{equation}

supongamos que $y=0$, entonces
\begin{eqnarray*}
\prob\left[R_{1}=0|R_{2}=0\right]&=&1,\\
\prob\left[R_{1}=x|R_{2}=0\right]&=&0,\textrm{ para cualquier }x\geq1,\\
\end{eqnarray*}


por tanto
\begin{eqnarray*}
\esp\left[R_{1}|R_{2}=0\right]=0.
\end{eqnarray*}

Para $y=1$,
\begin{eqnarray*}
\prob\left[R_{1}=0|R_{2}=1\right]&=&0,\\
\prob\left[R_{1}=1|R_{2}=1\right]&=&1,
\end{eqnarray*}

entonces
\begin{eqnarray*}
\esp\left[R_{1}|R_{2}=1\right]=1.
\end{eqnarray*}

Para $y>1$:
\begin{eqnarray*}
\prob\left[R_{1}=0|R_{2}\geq1\right]&=&0,\\
\prob\left[R_{1}=1|R_{2}\geq1\right]&=&1,\\
\prob\left[R_{1}>1|R_{2}\geq1\right]&=&0,
\end{eqnarray*}

entonces
\begin{eqnarray*}
\esp\left[R_{1}|R_{2}=y\right]=1,\textrm{ para cualquier }y>1.
\end{eqnarray*}
es decir
\begin{eqnarray*}
\esp\left[R_{1}|R_{2}=y\right]=1,\textrm{ para cualquier }y\geq1.
\end{eqnarray*}

Entonces
\begin{eqnarray*}
\esp\left[R_{1}\right]&=&\sum_{y\geq0}\sum_{x\geq0}x\prob\left[R_{1}=x|R_{2}=y\right]\prob\left[R_{2}=y\right]=\sum_{y\geq0}\sum_{x}\esp\left[R_{1}|R_{2}=y\right]\prob\left[R_{2}=y\right]\\
&=&\sum_{y\geq0}\prob\left[R_{2}=y\right]=\sum_{y\geq1}\frac{\left(\lambda
t\right)^{k}}{k!}e^{-\lambda t}=1.
\end{eqnarray*}

Adem\'as para $k\in Z^{+}$
\begin{eqnarray*}
f_{R_{1}}\left(k\right)&=&\prob\left[R_{1}=k\right]=\sum_{n=0}^{\infty}\prob\left[R_{1}=k|R_{2}=n\right]\prob\left[R_{2}=n\right]\\
&=&\prob\left[R_{1}=k|R_{2}=0\right]\prob\left[R_{2}=0\right]+\prob\left[R_{1}=k|R_{2}=1\right]\prob\left[R_{2}=1\right]\\
&+&\prob\left[R_{1}=k|R_{2}>1\right]\prob\left[R_{2}>1\right],
\end{eqnarray*}

donde para


\begin{description}
\item[$k=0$:]
\begin{eqnarray*}
\prob\left[R_{1}=0\right]=\prob\left[R_{1}=0|R_{2}=0\right]\prob\left[R_{2}=0\right]+\prob\left[R_{1}=0|R_{2}=1\right]\prob\left[R_{2}=1\right]\\
+\prob\left[R_{1}=0|R_{2}>1\right]\prob\left[R_{2}>1\right]=\prob\left[R_{2}=0\right].
\end{eqnarray*}
\item[$k=1$:]
\begin{eqnarray*}
\prob\left[R_{1}=1\right]=\prob\left[R_{1}=1|R_{2}=0\right]\prob\left[R_{2}=0\right]+\prob\left[R_{1}=1|R_{2}=1\right]\prob\left[R_{2}=1\right]\\
+\prob\left[R_{1}=1|R_{2}>1\right]\prob\left[R_{2}>1\right]=\sum_{n=1}^{\infty}\prob\left[R_{2}=n\right].
\end{eqnarray*}

\item[$k=2$:]
\begin{eqnarray*}
\prob\left[R_{1}=2\right]=\prob\left[R_{1}=2|R_{2}=0\right]\prob\left[R_{2}=0\right]+\prob\left[R_{1}=2|R_{2}=1\right]\prob\left[R_{2}=1\right]\\
+\prob\left[R_{1}=2|R_{2}>1\right]\prob\left[R_{2}>1\right]=0.
\end{eqnarray*}

\item[$k=j$:]
\begin{eqnarray*}
\prob\left[R_{1}=j\right]=\prob\left[R_{1}=j|R_{2}=0\right]\prob\left[R_{2}=0\right]+\prob\left[R_{1}=j|R_{2}=1\right]\prob\left[R_{2}=1\right]\\
+\prob\left[R_{1}=j|R_{2}>1\right]\prob\left[R_{2}>1\right]=0.
\end{eqnarray*}
\end{description}


Por lo tanto
\begin{eqnarray*}
f_{R_{1}}\left(0\right)&=&\prob\left[R_{2}=0\right]\\
f_{R_{1}}\left(1\right)&=&\sum_{n\geq1}^{\infty}\prob\left[R_{2}=n\right]\\
f_{R_{1}}\left(j\right)&=&0,\textrm{ para }j>1.
\end{eqnarray*}



\begin{description}
\item[Pol\'itica de $k$ usuarios:]Al igual que antes, para $y\in Z^{+}$ fijo
\begin{eqnarray*}
\esp\left[R_{1}|R_{2}=y\right]=\sum_{x}x\prob\left[R_{1}=x|R_{2}=y\right].\\
\end{eqnarray*}
\end{description}
Entonces, si tomamos diversos valore para $y$:\\

$y=0$:
\begin{eqnarray*}
\prob\left[R_{1}=0|R_{2}=0\right]&=&1,\\
\prob\left[R_{1}=x|R_{2}=0\right]&=&0,\textrm{ para cualquier }x\geq1,
\end{eqnarray*}

entonces
\begin{eqnarray*}
\esp\left[R_{1}|R_{2}=0\right]=0.
\end{eqnarray*}


Para $y=1$,
\begin{eqnarray*}
\prob\left[R_{1}=0|R_{2}=1\right]&=&0,\\
\prob\left[R_{1}=1|R_{2}=1\right]&=&1,
\end{eqnarray*}

entonces {\scriptsize{
\begin{eqnarray*}
\esp\left[R_{1}|R_{2}=1\right]=1.
\end{eqnarray*}}}


Para $y=2$,
\begin{eqnarray*}
\prob\left[R_{1}=0|R_{2}=2\right]&=&0,\\
\prob\left[R_{1}=1|R_{2}=2\right]&=&1,\\
\prob\left[R_{1}=2|R_{2}=2\right]&=&1,\\
\prob\left[R_{1}=3|R_{2}=2\right]&=&0,
\end{eqnarray*}

entonces
\begin{eqnarray*}
\esp\left[R_{1}|R_{2}=2\right]=3.
\end{eqnarray*}

Para $y=3$,
\begin{eqnarray*}
\prob\left[R_{1}=0|R_{2}=3\right]&=&0,\\
\prob\left[R_{1}=1|R_{2}=3\right]&=&1,\\
\prob\left[R_{1}=2|R_{2}=3\right]&=&1,\\
\prob\left[R_{1}=3|R_{2}=3\right]&=&1,\\
\prob\left[R_{1}=4|R_{2}=3\right]&=&0,
\end{eqnarray*}

entonces
\begin{eqnarray*}
\esp\left[R_{1}|R_{2}=3\right]=6.
\end{eqnarray*}

En general, para $k\geq0$,
\begin{eqnarray*}
\prob\left[R_{1}=0|R_{2}=k\right]&=&0,\\
\prob\left[R_{1}=j|R_{2}=k\right]&=&1,\textrm{ para }1\leq j\leq k,\\
\prob\left[R_{1}=j|R_{2}=k\right]&=&0,\textrm{ para }j> k,
\end{eqnarray*}

entonces
\begin{eqnarray*}
\esp\left[R_{1}|R_{2}=k\right]=\frac{k\left(k+1\right)}{2}.
\end{eqnarray*}



Por lo tanto


\begin{eqnarray*}
\esp\left[R_{1}\right]&=&\sum_{y}\esp\left[R_{1}|R_{2}=y\right]\prob\left[R_{2}=y\right]\\
&=&\sum_{y}\prob\left[R_{2}=y\right]\frac{y\left(y+1\right)}{2}=\sum_{y\geq1}\left(\frac{y\left(y+1\right)}{2}\right)\frac{\left(\lambda t\right)^{y}}{y!}e^{-\lambda t}\\
&=&\frac{\lambda t}{2}e^{-\lambda t}\sum_{y\geq1}\left(y+1\right)\frac{\left(\lambda t\right)^{y-1}}{\left(y-1\right)!}=\frac{\lambda t}{2}e^{-\lambda t}\left(e^{\lambda t}\left(\lambda t+2\right)\right)\\
&=&\frac{\lambda t\left(\lambda t+2\right)}{2},
\end{eqnarray*}
es decir,


\begin{equation}
\esp\left[R_{1}\right]=\frac{\lambda t\left(\lambda
t+2\right)}{2}.
\end{equation}

Adem\'as para $k\in Z^{+}$ fijo
\begin{eqnarray*}
f_{R_{1}}\left(k\right)&=&\prob\left[R_{1}=k\right]=\sum_{n=0}^{\infty}\prob\left[R_{1}=k|R_{2}=n\right]\prob\left[R_{2}=n\right]\\
&=&\prob\left[R_{1}=k|R_{2}=0\right]\prob\left[R_{2}=0\right]+\prob\left[R_{1}=k|R_{2}=1\right]\prob\left[R_{2}=1\right]\\
&+&\prob\left[R_{1}=k|R_{2}=2\right]\prob\left[R_{2}=2\right]+\cdots+\prob\left[R_{1}=k|R_{2}=j\right]\prob\left[R_{2}=j\right]+\cdots+
\end{eqnarray*}
donde para

\begin{description}
\item[$k=0$:]
\begin{eqnarray*}
\prob\left[R_{1}=0\right]=\prob\left[R_{1}=0|R_{2}=0\right]\prob\left[R_{2}=0\right]+\prob\left[R_{1}=0|R_{2}=1\right]\prob\left[R_{2}=1\right]\\
+\prob\left[R_{1}=0|R_{2}=j\right]\prob\left[R_{2}=j\right]=\prob\left[R_{2}=0\right].
\end{eqnarray*}
\item[$k=1$:]
\begin{eqnarray*}
\prob\left[R_{1}=1\right]=\prob\left[R_{1}=1|R_{2}=0\right]\prob\left[R_{2}=0\right]+\prob\left[R_{1}=1|R_{2}=1\right]\prob\left[R_{2}=1\right]\\
+\prob\left[R_{1}=1|R_{2}=1\right]\prob\left[R_{2}=1\right]+\cdots+\prob\left[R_{1}=1|R_{2}=j\right]\prob\left[R_{2}=j\right]\\
=\sum_{n=1}^{\infty}\prob\left[R_{2}=n\right].
\end{eqnarray*}

\item[$k=2$:]
\begin{eqnarray*}
\prob\left[R_{1}=2\right]=\prob\left[R_{1}=2|R_{2}=0\right]\prob\left[R_{2}=0\right]+\prob\left[R_{1}=2|R_{2}=1\right]\prob\left[R_{2}=1\right]\\
+\prob\left[R_{1}=2|R_{2}=2\right]\prob\left[R_{2}=2\right]+\cdots+\prob\left[R_{1}=2|R_{2}=j\right]\prob\left[R_{2}=j\right]\\
=\sum_{n=2}^{\infty}\prob\left[R_{2}=n\right].
\end{eqnarray*}
\end{description}

En general

\begin{eqnarray*}
\prob\left[R_{1}=k\right]=\prob\left[R_{1}=k|R_{2}=0\right]\prob\left[R_{2}=0\right]+\prob\left[R_{1}=k|R_{2}=1\right]\prob\left[R_{2}=1\right]\\
+\prob\left[R_{1}=k|R_{2}=2\right]\prob\left[R_{2}=2\right]+\cdots+\prob\left[R_{1}=k|R_{2}=k\right]\prob\left[R_{2}=k\right]\\
=\sum_{n=k}^{\infty}\prob\left[R_{2}=n\right].\\
\end{eqnarray*}



Por lo tanto

\begin{eqnarray*}
f_{R_{1}}\left(k\right)&=&\prob\left[R_{1}=k\right]=\sum_{n=k}^{\infty}\prob\left[R_{2}=n\right].
\end{eqnarray*}








\section*{Objetivos Principales}

\begin{itemize}
%\item Generalizar los principales resultados existentes para Sistemas de Visitas C\'iclicas para el caso en el que se tienen dos Sistemas de Visitas C\'iclicas con propiedades similares.

\item Encontrar las ecuaciones que modelan el comportamiento de una Red de Sistemas de Visitas C\'iclicas (RSVC) con propiedades similares.

\item Encontrar expresiones anal\'iticas para las longitudes de las colas al momento en que el servidor llega a una de ellas para comenzar a dar servicio, as\'i como de sus segundos momentos.

\item Determinar las principales medidas de Desempe\~no para la RSVC tales como: N\'umero de usuarios presentes en cada una de las colas del sistema cuando uno de los servidores est\'a presente atendiendo, Tiempos que transcurre entre las visitas del servidor a la misma cola.


\end{itemize}


%_________________________________________________________________________
%\section{Sistemas de Visitas C\'iclicas}
%_________________________________________________________________________
\numberwithin{equation}{section}%
%__________________________________________________________________________




%\section*{Introducci\'on}




%__________________________________________________________________________
%\subsection{Definiciones}
%__________________________________________________________________________


\section{Descripci\'on de una Red de Sistemas de Visitas C\'iclicas}



Consideremos una red de sistema de visitas c\'iclicas conformada por dos sistemas de visitas c\'iclicas, cada una con dos colas independientes, donde adem\'as se permite el intercambio de usuarios entre los dos sistemas en la segunda cola de cada uno de ellos.\smallskip

Sup\'ongase adem\'as que los arribos de los usuarios ocurren
conforme a un proceso Poisson con tasa de llegada $\mu_{1}$ y
$\mu_{2}$ para el sistema 1, mientras que para el sistema 2,
lo hacen conforme a un proceso Poisson con tasa
$\hat{\mu}_{1},\hat{\mu}_{2}$ respectivamente.\smallskip

El traslado de un sistema a otro ocurre de manera que los tiempos
entre llegadas de los usuarios a la cola dos del sistema 1
provenientes del sistema 2, se distribuye de manera exponencial
con par\'ametro $\check{\mu}_{2}$.\smallskip

Se considerar\'an intervalos de tiempo de la forma
$\left[t,t+1\right]$. Los usuarios arriban por paquetes de manera
independiente del resto de las colas. Se define el grupo de
usuarios que llegan a cada una de las colas del sistema 1,
caracterizadas por $Q_{1}$ y $Q_{2}$ respectivamente, en el
intervalo de tiempo $\left[t,t+1\right]$ por
$X_{1}\left(t\right),X_{2}\left(t\right)$. De igual manera se
definen los procesos
$\hat{X}_{1}\left(t\right),\hat{X}_{2}\left(t\right)$ para las
colas del sistema 2, denotadas por $\hat{Q}_{1}$ y $\hat{Q}_{2}$
respectivamente.\smallskip

Para el n\'umero de usuarios que se trasladan del sistema 2 al
sistema 1, de la cola $\hat{Q}_{2}$ a la cola
$Q_{2}$, en el intervalo de tiempo
$\left[t,t+1\right]$, se define el proceso
$Y_{2}\left(t\right)$.\smallskip

El uso de la Funci\'on Generadora de Probabilidades (FGP's) nos permite determinar las Funciones de Distribuci\'on de Probabilidades Conjunta de manera indirecta sin necesidad de hacer uso de las propiedades de las distribuciones de probabilidad de cada uno de los procesos que intervienen en la Red de Sistemas de Visitas C\'iclicas.\smallskip

En lo que respecta al servidor, en t\'erminos de los tiempos de
visita a cada una de las colas, se definen las variables
aleatorias $\tau_{1},\tau_{2}$ para $Q_{1},Q_{2}$ respectivamente;
y $\zeta_{1},\zeta_{2}$ para $\hat{Q}_{1},\hat{Q}_{2}$ del sistema
2. A los tiempos en que el servidor termina de atender en las
colas $Q_{1},Q_{2},\hat{Q}_{1},\hat{Q}_{2}$, se les denotar\'a por
$\overline{\tau}_{1},\overline{\tau}_{2},\overline{\zeta}_{1},\overline{\zeta}_{2}$
respectivamente.\smallskip

Los tiempos de traslado del servidor desde el momento en que termina de atender a una cola y llega a la siguiente para comenzar a dar servicio est\'an dados por
$\tau_{2}-\overline{\tau}_{1},\tau_{1}-\overline{\tau}_{2}$ y
$\zeta_{2}-\overline{\zeta}_{1},\zeta_{1}-\overline{\zeta}_{2}$
para el sistema 1 y el sistema 2, respectivamente.\smallskip

Cada uno de estos procesos con su respectiva FGP. Adem\'as, para cada una de las colas en cada sistema, el n\'umero de usuarios al tiempo en que llega el servidor a dar servicio est\'a
dado por el n\'umero de usuarios presentes en la cola al tiempo
$t$, m\'as el n\'umero de usuarios que llegan a la cola en el intervalo de tiempo
$\left[\tau_{i},\overline{\tau}_{i}\right]$.

%es decir
%{\small{
%\begin{eqnarray*}
%L_{1}\left(\overline{\tau}_{1}\right)=L_{1}\left(\tau_{1}\right)+X_{1}\left(\overline{\tau}_{1}-\tau_{1}\right),\hat{L}_{i}\left(\overline{\tau}_{i}\right)=\hat{L}_{i}\left(\tau_{i}\right)+\hat{X}_{i}\left(\overline{\tau}_{i}-\tau_{i}\right),L_{2}\left(\overline{\tau}_{1}\right)=L_{2}\left(\tau_{1}\right)+X_{2}\left(\overline{\tau}_{1}-\tau_{1}\right)+Y_{2}\left(\overline{\tau}_{1}-\tau_{1}\right),
%\end{eqnarray*}}}




%\begin{center}\vspace{1cm}
%%%%\includegraphics[width=0.6\linewidth]{RedSVC2}
%\captionof{figure}{\color{Green} Red de Sistema de Visitas C\'iclicas}
%\end{center}\vspace{1cm}




Una vez definidas las Funciones Generadoras de Probabilidades Conjuntas se construyen las ecuaciones recursivas que permiten obtener la informaci\'on sobre la longitud de cada una de las colas, al momento en que uno de los servidores llega a una de las colas para dar servicio, bas\'andose en la informaci\'on que se tiene sobre su llegada a la cola inmediata anterior.\smallskip
%{\footnotesize{
%\begin{eqnarray*}
%F_{2}\left(z_{1},z_{2},w_{1},w_{2}\right)&=&R_{1}\left(P_{1}\left(z_{1}\right)\tilde{P}_{2}\left(z_{2}\right)\prod_{i=1}^{2}
%\hat{P}_{i}\left(w_{i}\right)\right)F_{1}\left(\theta_{1}\left(\tilde{P}_{2}\left(z_{2}\right)\hat{P}_{1}\left(w_{1}\right)\hat{P}_{2}\left(w_{2}\right)\right),z_{2},w_{1},w_{2}\right),\\
%F_{1}\left(z_{1},z_{2},w_{1},w_{2}\right)&=&R_{2}\left(P_{1}\left(z_{1}\right)\tilde{P}_{2}\left(z_{2}\right)\prod_{i=1}^{2}
%\hat{P}_{i}\left(w_{i}\right)\right)F_{2}\left(z_{1},\tilde{\theta}_{2}\left(P_{1}\left(z_{1}\right)\hat{P}_{1}\left(w_{1}\right)\hat{P}_{2}\left(w_{2}\right)\right),w_{1},w_{2}\right),\\
%\hat{F}_{2}\left(z_{1},z_{2},w_{1},w_{2}\right)&=&\hat{R}_{1}\left(P_{1}\left(z_{1}\right)\tilde{P}_{2}\left(z_{2}\right)\prod_{i=1}^{2}
%\hat{P}_{i}\left(w_{i}\right)\right)\hat{F}_{1}\left(z_{1},z_{2},\hat{\theta}_{1}\left(P_{1}\left(z_{1}\right)\tilde{P}_{2}\left(z_{2}\right)\hat{P}_{2}\left(w_{2}\right)\right),w_{2}\right),\\
%\end{eqnarray*}}}
%{\small{
%\begin{eqnarray*}
%\hat{F}_{1}\left(z_{1},z_{2},w_{1},w_{2}\right)&=&\hat{R}_{2}\left(P_{1}\left(z_{1}\right)\tilde{P}_{2}\left(z_{2}\right)\prod_{i=1}^{2}
%\hat{P}_{i}\left(w_{i}\right)\right)\hat{F}_{2}\left(z_{1},z_{2},w_{1},\hat{\theta}_{2}\left(P_{1}\left(z_{1}\right)\tilde{P}_{2}\left(z_{2}\right)\hat{P}_{1}\left(w_{1}\right)\right)\right).
%\end{eqnarray*}}}

%__________________________________________________________________________
\subsection{Funciones Generadoras de Probabilidades}
%__________________________________________________________________________


Para cada uno de los procesos de llegada a las colas $X_{1},X_{2},\hat{X}_{1},\hat{X}_{2}$ y $Y_{2}$, con $\tilde{X}_{2}=X_{2}+Y_{2}$ anteriores se define su Funci\'on
Generadora de Probabilidades (FGP):
%\begin{multicols}{3}
\begin{eqnarray*}
\begin{array}{ccc}
P_{1}\left(z_{1}\right)=\esp\left[z_{1}^{X_{1}\left(t\right)}\right],&P_{2}\left(z_{2}\right)=\esp\left[z_{2}^{X_{2}\left(t\right)}\right],&\check{P}_{2}\left(z_{2}\right)=\esp\left[z_{2}^{Y_{2}\left(t\right)}\right],\\
\hat{P}_{1}\left(w_{1}\right)=\esp\left[w_{1}^{\hat{X}_{1}\left(t\right)}\right],&\hat{P}_{2}\left(w_{2}\right)=\esp\left[w_{2}^{\hat{X}_{2}\left(t\right)}\right],&\tilde{P}_{2}\left(z_{2}\right)=\esp\left[z_{2}^{\tilde{X}_{2}\left(t\right)}\right].
\end{array}
\end{eqnarray*}

Con primer momento definidos por

\begin{eqnarray*}
\begin{array}{cc}
\mu_{1}=\esp\left[X_{1}\left(t\right)\right]=P_{1}^{(1)}\left(1\right),&\mu_{2}=\esp\left[X_{2}\left(t\right)\right]=P_{2}^{(1)}\left(1\right),\\
\check{\mu}_{2}=\esp\left[Y_{2}\left(t\right)\right]=\check{P}_{2}^{(1)}\left(1\right),&
\hat{\mu}_{1}=\esp\left[\hat{X}_{1}\left(t\right)\right]=\hat{P}_{1}^{(1)}\left(1\right),\\
\hat{\mu}_{2}=\esp\left[\hat{X}_{2}\left(t\right)\right]=\hat{P}_{2}^{(1)}\left(1\right),&\tilde{\mu}_{2}=\esp\left[\tilde{X}_{2}\left(t\right)\right]=\tilde{P}_{2}^{(1)}\left(1\right).
\end{array}
\end{eqnarray*}

En lo que respecta al servidor, en t\'erminos de los tiempos de
visita a cada una de las colas, se denotar\'an por
$B_{1}\left(t\right),B_{2}\left(t\right)$ los procesos
correspondientes a las variables aleatorias $\tau_{1},\tau_{2}$
para $Q_{1},Q_{2}$ respectivamente; y
$\hat{B}_{1}\left(t\right),\hat{B}_{2}\left(t\right)$ con
par\'ametros $\zeta_{1},\zeta_{2}$ para $\hat{Q}_{1},\hat{Q}_{2}$
del sistema 2. Y a los tiempos en que el servidor termina de
atender en las colas $Q_{1},Q_{2},\hat{Q}_{1},\hat{Q}_{2}$, se les
denotar\'a por
$\overline{\tau}_{1},\overline{\tau}_{2},\overline{\zeta}_{1},\overline{\zeta}_{2}$
respectivamente. Entonces, los tiempos de servicio est\'an dados
por las diferencias
$\overline{\tau}_{1}-\tau_{1},\overline{\tau}_{2}-\tau_{2}$ para
$Q_{1},Q_{2}$, y
$\overline{\zeta}_{1}-\zeta_{1},\overline{\zeta}_{2}-\zeta_{2}$
para $\hat{Q}_{1},\hat{Q}_{2}$ respectivamente.

Sus procesos se definen por:


\begin{eqnarray*}
\begin{array}{cc}
S_{1}\left(z_{1}\right)=\esp\left[z_{1}^{\overline{\tau}_{1}-\tau_{1}}\right],&S_{2}\left(z_{2}\right)=\esp\left[z_{1}^{\overline{\tau}_{2}-\tau_{2}}\right],\\
\hat{S}_{1}\left(w_{1}\right)=\esp\left[w_{1}^{\overline{\zeta}_{1}-\zeta_{1}}\right],&\hat{S}_{2}\left(w_{2}\right)=\esp\left[w_{2}^{\overline{\zeta}_{2}-\zeta_{2}}\right],
\end{array}
\end{eqnarray*}

con primer momento dado por:


\begin{eqnarray*}
\begin{array}{cccc}
s_{1}=\esp\left[\overline{\tau}_{1}-\tau_{1}\right],&s_{2}=\esp\left[\overline{\tau}_{2}-\tau_{2}\right],&
\hat{s}_{1}=\esp\left[\overline{\zeta}_{1}-\zeta_{1}\right],&
\hat{s}_{2}=\esp\left[\overline{\zeta}_{2}-\zeta_{2}\right].
\end{array}
\end{eqnarray*}

An\'alogamente los tiempos de traslado del servidor desde el
momento en que termina de atender a una cola y llega a la
siguiente para comenzar a dar servicio est\'an dados por
$\tau_{2}-\overline{\tau}_{1},\tau_{1}-\overline{\tau}_{2}$ y
$\zeta_{2}-\overline{\zeta}_{1},\zeta_{1}-\overline{\zeta}_{2}$
para el sistema 1 y el sistema 2, respectivamente.

La FGP para estos tiempos de traslado est\'an dados por

\begin{eqnarray*}
\begin{array}{cc}
R_{1}\left(z_{1}\right)=\esp\left[z_{1}^{\tau_{2}-\overline{\tau}_{1}}\right],&R_{2}\left(z_{2}\right)=\esp\left[z_{2}^{\tau_{1}-\overline{\tau}_{2}}\right],\\
\hat{R}_{1}\left(w_{1}\right)=\esp\left[w_{1}^{\zeta_{2}-\overline{\zeta}_{1}}\right],&\hat{R}_{2}\left(w_{2}\right)=\esp\left[w_{2}^{\zeta_{1}-\overline{\zeta}_{2}}\right],
\end{array}
\end{eqnarray*}
y al igual que como se hizo con anterioridad

\begin{eqnarray*}
\begin{array}{cc}
r_{1}=R_{1}^{(1)}\left(1\right)=\esp\left[\tau_{2}-\overline{\tau}_{1}\right],&r_{2}=R_{2}^{(1)}\left(1\right)=\esp\left[\tau_{1}-\overline{\tau}_{2}\right],\\
\hat{r}_{1}=\hat{R}_{1}^{(1)}\left(1\right)=\esp\left[\zeta_{2}-\overline{\zeta}_{1}\right],&
\hat{r}_{2}=\hat{R}_{2}^{(1)}\left(1\right)=\esp\left[\zeta_{1}-\overline{\zeta}_{2}\right].
\end{array}
\end{eqnarray*}

Se definen los procesos de conteo para el n\'umero de usuarios en
cada una de las colas al tiempo $t$,
$L_{1}\left(t\right),L_{2}\left(t\right)$, para
$H_{1}\left(t\right),H_{2}\left(t\right)$ del sistema 1,
respectivamente. Y para el segundo sistema, se tienen los procesos
$\hat{L}_{1}\left(t\right),\hat{L}_{2}\left(t\right)$ para
$\hat{H}_{1}\left(t\right),\hat{H}_{2}\left(t\right)$,
respectivamente, es decir,


\begin{eqnarray*}
\begin{array}{cccc}
H_{1}\left(t\right)=\esp\left[z_{1}^{L_{1}\left(t\right)}\right],&
H_{2}\left(t\right)=\esp\left[z_{2}^{L_{2}\left(t\right)}\right],&
\hat{H}_{1}\left(t\right)=\esp\left[w_{1}^{\hat{L}_{1}\left(t\right)}\right],&\hat{H}_{2}\left(t\right)=\esp\left[w_{2}^{\hat{L}_{2}\left(t\right)}\right].
\end{array}
\end{eqnarray*}
Por lo dicho anteriormente se tiene que el n\'umero de usuarios
presentes en los tiempos $\overline{\tau}_{1},\overline{\tau}_{2},
\overline{\zeta}_{1},\overline{\zeta}_{2}$, es cero, es decir,
 $L_{i}\left(\overline{\tau_{i}}\right)=0,$ y
$\hat{L}_{i}\left(\overline{\zeta_{i}}\right)=0$ para i=1,2 para
cada uno de los dos sistemas.


Para cada una de las colas en cada sistema, el n\'umero de
usuarios al tiempo en que llega el servidor a dar servicio est\'a
dado por el n\'umero de usuarios presentes en la cola al tiempo
$t=\tau_{i},\zeta_{i}$, m\'as el n\'umero de usuarios que llegan a
la cola en el intervalo de tiempo
$\left[\tau_{i},\overline{\tau}_{i}\right],\left[\zeta_{i},\overline{\zeta}_{i}\right]$,
es decir

\begin{eqnarray*}\label{Eq.TiemposLlegada}
\begin{array}{cc}
L_{1}\left(\overline{\tau}_{1}\right)=L_{1}\left(\tau_{1}\right)+X_{1}\left(\overline{\tau}_{1}-\tau_{1}\right),&\hat{L}_{1}\left(\overline{\tau}_{1}\right)=\hat{L}_{1}\left(\tau_{1}\right)+\hat{X}_{1}\left(\overline{\tau}_{1}-\tau_{1}\right),\\
\hat{L}_{2}\left(\overline{\tau}_{1}\right)=\hat{L}_{2}\left(\tau_{1}\right)+\hat{X}_{2}\left(\overline{\tau}_{1}-\tau_{1}\right).&
\end{array}
\end{eqnarray*}

En el caso espec\'ifico de $Q_{2}$, adem\'as, hay que considerar
el n\'umero de usuarios que pasan del sistema 2 al sistema 1, a
traves de $\hat{Q}_{2}$ mientras el servidor en $Q_{2}$ est\'a
ausente, es decir:

\begin{equation}\label{Eq.UsuariosTotalesZ2}
L_{2}\left(\overline{\tau}_{1}\right)=L_{2}\left(\tau_{1}\right)+X_{2}\left(\overline{\tau}_{1}-\tau_{1}\right)+Y_{2}\left(\overline{\tau}_{1}-\tau_{1}\right).
\end{equation}

%_________________________________________________________________________
\subsection{El problema de la ruina del jugador}
%_________________________________________________________________________

Supongamos que se tiene un jugador que cuenta con un capital
inicial de $\tilde{L}_{0}\geq0$ unidades, esta persona realiza una
serie de dos juegos simult\'aneos e independientes de manera
sucesiva, dichos eventos son independientes e id\'enticos entre
s\'i para cada realizaci\'on.\smallskip

La ganancia en el $n$-\'esimo juego es
\begin{eqnarray*}\label{Eq.Cero}
\tilde{X}_{n}=X_{n}+Y_{n}
\end{eqnarray*}

unidades de las cuales se resta una cuota de 1 unidad por cada
juego simult\'aneo, es decir, se restan dos unidades por cada
juego realizado.\smallskip

En t\'erminos de la teor\'ia de colas puede pensarse como el n\'umero de usuarios que llegan a una cola v\'ia dos procesos de arribo distintos e independientes entre s\'i.

Su Funci\'on Generadora de Probabilidades (FGP) est\'a dada por

\begin{eqnarray*}
F\left(z\right)=\esp\left[z^{\tilde{L}_{0}}\right]
\end{eqnarray*}

\begin{eqnarray*}
\tilde{P}\left(z\right)=\esp\left[z^{\tilde{X}_{n}}\right]=\esp\left[z^{X_{n}+Y_{n}}\right]=\esp\left[z^{X_{n}}z^{Y_{n}}\right]=\esp\left[z^{X_{n}}\right]\esp\left[z^{Y_{n}}\right]=P\left(z\right)\check{P}\left(z\right),
\end{eqnarray*}
entonces
\begin{eqnarray*}
\tilde{\mu}&=&\esp\left[\tilde{X}_{n}\right]=\tilde{P}\left[z\right]<1.\\
\end{eqnarray*}

Sea $\tilde{L}_{n}$ el capital remanente despu\'es del $n$-\'esimo
juego. Entonces

\begin{eqnarray*}
\tilde{L}_{n}&=&\tilde{L}_{0}+\tilde{X}_{1}+\tilde{X}_{2}+\cdots+\tilde{X}_{n}-2n.
\end{eqnarray*}

La ruina del jugador ocurre despu\'es del $n$-\'esimo juego, es decir, la cola se vac\'ia despu\'es del $n$-\'esimo juego,
entonces sea $T$ definida como

\begin{eqnarray*}
T&=&min\left\{\tilde{L}_{n}=0\right\}
\end{eqnarray*}

Si $\tilde{L}_{0}=0$, entonces claramente $T=0$. En este sentido $T$
puede interpretarse como la longitud del periodo de tiempo que el servidor ocupa para dar servicio en la cola, comenzando con $\tilde{L}_{0}$ grupos de usuarios
presentes en la cola, quienes arribaron conforme a un proceso dado
por $\tilde{P}\left(z\right)$.\smallskip


Sea $g_{n,k}$ la probabilidad del evento de que el jugador no
caiga en ruina antes del $n$-\'esimo juego, y que adem\'as tenga
un capital de $k$ unidades antes del $n$-\'esimo juego, es decir,

Dada $n\in\left\{1,2,\ldots,\right\}$ y
$k\in\left\{0,1,2,\ldots,\right\}$
\begin{eqnarray*}
g_{n,k}:=P\left\{\tilde{L}_{j}>0, j=1,\ldots,n,
\tilde{L}_{n}=k\right\}
\end{eqnarray*}

la cual adem\'as se puede escribir como:

\begin{eqnarray*}
g_{n,k}&=&P\left\{\tilde{L}_{j}>0, j=1,\ldots,n,
\tilde{L}_{n}=k\right\}=\sum_{j=1}^{k+1}g_{n-1,j}P\left\{\tilde{X}_{n}=k-j+1\right\}\\
&=&\sum_{j=1}^{k+1}g_{n-1,j}P\left\{X_{n}+Y_{n}=k-j+1\right\}=\sum_{j=1}^{k+1}\sum_{l=1}^{j}g_{n-1,j}P\left\{X_{n}+Y_{n}=k-j+1,Y_{n}=l\right\}\\
&=&\sum_{j=1}^{k+1}\sum_{l=1}^{j}g_{n-1,j}P\left\{X_{n}+Y_{n}=k-j+1|Y_{n}=l\right\}P\left\{Y_{n}=l\right\}\\
&=&\sum_{j=1}^{k+1}\sum_{l=1}^{j}g_{n-1,j}P\left\{X_{n}=k-j-l+1\right\}P\left\{Y_{n}=l\right\}\\
\end{eqnarray*}

es decir
\begin{eqnarray}\label{Eq.Gnk.2S}
g_{n,k}=\sum_{j=1}^{k+1}\sum_{l=1}^{j}g_{n-1,j}P\left\{X_{n}=k-j-l+1\right\}P\left\{Y_{n}=l\right\}
\end{eqnarray}
adem\'as

\begin{equation}\label{Eq.L02S}
g_{0,k}=P\left\{\tilde{L}_{0}=k\right\}.
\end{equation}

Se definen las siguientes FGP:
\begin{equation}\label{Eq.3.16.a.2S}
G_{n}\left(z\right)=\sum_{k=0}^{\infty}g_{n,k}z^{k},\textrm{ para
}n=0,1,\ldots,
\end{equation}

\begin{equation}\label{Eq.3.16.b.2S}
G\left(z,w\right)=\sum_{n=0}^{\infty}G_{n}\left(z\right)w^{n}.
\end{equation}


En particular para $k=0$,
\begin{eqnarray*}
g_{n,0}=G_{n}\left(0\right)=P\left\{\tilde{L}_{j}>0,\textrm{ para
}j<n,\textrm{ y }\tilde{L}_{n}=0\right\}=P\left\{T=n\right\},
\end{eqnarray*}

adem\'as

\begin{eqnarray*}%\label{Eq.G0w.2S}
G\left(0,w\right)=\sum_{n=0}^{\infty}G_{n}\left(0\right)w^{n}=\sum_{n=0}^{\infty}P\left\{T=n\right\}w^{n}
=\esp\left[w^{T}\right]
\end{eqnarray*}
la cu\'al resulta ser la FGP del tiempo de ruina $T$.

%__________________________________________________________________________________
% INICIA LA PROPOSICIÓN
%__________________________________________________________________________________


\begin{Prop}\label{Prop.1.1.2S}
Sean $G_{n}\left(z\right)$ y $G\left(z,w\right)$ definidas como en
(\ref{Eq.3.16.a.2S}) y (\ref{Eq.3.16.b.2S}) respectivamente,
entonces
\begin{equation}\label{Eq.Pag.45}
G_{n}\left(z\right)=\frac{1}{z}\left[G_{n-1}\left(z\right)-G_{n-1}\left(0\right)\right]\tilde{P}\left(z\right).
\end{equation}

Adem\'as


\begin{equation}\label{Eq.Pag.46}
G\left(z,w\right)=\frac{zF\left(z\right)-wP\left(z\right)G\left(0,w\right)}{z-wR\left(z\right)},
\end{equation}

con un \'unico polo en el c\'irculo unitario, adem\'as, el polo es
de la forma $z=\theta\left(w\right)$ y satisface que

\begin{enumerate}
\item[i)]$\tilde{\theta}\left(1\right)=1$,

\item[ii)] $\tilde{\theta}^{(1)}\left(1\right)=\frac{1}{1-\tilde{\mu}}$,

\item[iii)]
$\tilde{\theta}^{(2)}\left(1\right)=\frac{\tilde{\mu}}{\left(1-\tilde{\mu}\right)^{2}}+\frac{\tilde{\sigma}}{\left(1-\tilde{\mu}\right)^{3}}$.
\end{enumerate}

Finalmente, adem\'as se cumple que
\begin{equation}
\esp\left[w^{T}\right]=G\left(0,w\right)=F\left[\tilde{\theta}\left(w\right)\right].
\end{equation}
\end{Prop}
%__________________________________________________________________________________
% TERMINA LA PROPOSICIÓN E INICIA LA DEMOSTRACI\'ON
%__________________________________________________________________________________


Multiplicando las ecuaciones (\ref{Eq.Gnk.2S}) y (\ref{Eq.L02S})
por el t\'ermino $z^{k}$:

\begin{eqnarray*}
g_{n,k}z^{k}&=&\sum_{j=1}^{k+1}\sum_{l=1}^{j}g_{n-1,j}P\left\{X_{n}=k-j-l+1\right\}P\left\{Y_{n}=l\right\}z^{k},\\
g_{0,k}z^{k}&=&P\left\{\tilde{L}_{0}=k\right\}z^{k},
\end{eqnarray*}

ahora sumamos sobre $k$
\begin{eqnarray*}
\sum_{k=0}^{\infty}g_{n,k}z^{k}&=&\sum_{k=0}^{\infty}\sum_{j=1}^{k+1}\sum_{l=1}^{j}g_{n-1,j}P\left\{X_{n}=k-j-l+1\right\}P\left\{Y_{n}=l\right\}z^{k}\\
&=&\sum_{k=0}^{\infty}z^{k}\sum_{j=1}^{k+1}\sum_{l=1}^{j}g_{n-1,j}P\left\{X_{n}=k-\left(j+l
-1\right)\right\}P\left\{Y_{n}=l\right\}\\
&=&\sum_{k=0}^{\infty}z^{k+\left(j+l-1\right)-\left(j+l-1\right)}\sum_{j=1}^{k+1}\sum_{l=1}^{j}g_{n-1,j}P\left\{X_{n}=k-
\left(j+l-1\right)\right\}P\left\{Y_{n}=l\right\}\\
&=&\sum_{k=0}^{\infty}\sum_{j=1}^{k+1}\sum_{l=1}^{j}g_{n-1,j}z^{j-1}P\left\{X_{n}=k-
\left(j+l-1\right)\right\}z^{k-\left(j+l-1\right)}P\left\{Y_{n}=l\right\}z^{l}\\
&=&\sum_{j=1}^{\infty}\sum_{l=1}^{j}g_{n-1,j}z^{j-1}\sum_{k=j+l-1}^{\infty}P\left\{X_{n}=k-
\left(j+l-1\right)\right\}z^{k-\left(j+l-1\right)}P\left\{Y_{n}=l\right\}z^{l}\\
&=&\sum_{j=1}^{\infty}g_{n-1,j}z^{j-1}\sum_{l=1}^{j}\sum_{k=j+l-1}^{\infty}P\left\{X_{n}=k-
\left(j+l-1\right)\right\}z^{k-\left(j+l-1\right)}P\left\{Y_{n}=l\right\}z^{l}\\
&=&\sum_{j=1}^{\infty}g_{n-1,j}z^{j-1}\sum_{k=j+l-1}^{\infty}\sum_{l=1}^{j}P\left\{X_{n}=k-
\left(j+l-1\right)\right\}z^{k-\left(j+l-1\right)}P\left\{Y_{n}=l\right\}z^{l}\\
\end{eqnarray*}


luego
\begin{eqnarray*}
&=&\sum_{j=1}^{\infty}g_{n-1,j}z^{j-1}\sum_{k=j+l-1}^{\infty}\sum_{l=1}^{j}P\left\{X_{n}=k-
\left(j+l-1\right)\right\}z^{k-\left(j+l-1\right)}\sum_{l=1}^{j}P
\left\{Y_{n}=l\right\}z^{l}\\
&=&\sum_{j=1}^{\infty}g_{n-1,j}z^{j-1}\sum_{l=1}^{\infty}P\left\{Y_{n}=l\right\}z^{l}
\sum_{k=j+l-1}^{\infty}\sum_{l=1}^{j}
P\left\{X_{n}=k-\left(j+l-1\right)\right\}z^{k-\left(j+l-1\right)}\\
&=&\frac{1}{z}\left[G_{n-1}\left(z\right)-G_{n-1}\left(0\right)\right]\tilde{P}\left(z\right)
\sum_{k=j+l-1}^{\infty}\sum_{l=1}^{j}
P\left\{X_{n}=k-\left(j+l-1\right)\right\}z^{k-\left(j+l-1\right)}\\
&=&\frac{1}{z}\left[G_{n-1}\left(z\right)-G_{n-1}\left(0\right)\right]\tilde{P}\left(z\right)P\left(z\right)=\frac{1}{z}\left[G_{n-1}\left(z\right)-G_{n-1}\left(0\right)\right]\tilde{P}\left(z\right),\\
\end{eqnarray*}

es decir la ecuaci\'on (\ref{Eq.3.16.a.2S}) se puede reescribir
como
\begin{equation}\label{Eq.3.16.a.2Sbis}
G_{n}\left(z\right)=\frac{1}{z}\left[G_{n-1}\left(z\right)-G_{n-1}\left(0\right)\right]\tilde{P}\left(z\right).
\end{equation}

Por otra parte recordemos la ecuaci\'on (\ref{Eq.3.16.a.2S})

\begin{eqnarray*}
G_{n}\left(z\right)&=&\sum_{k=0}^{\infty}g_{n,k}z^{k},\textrm{ entonces }\frac{G_{n}\left(z\right)}{z}=\sum_{k=1}^{\infty}g_{n,k}z^{k-1},\\
\end{eqnarray*}

Por lo tanto utilizando la ecuaci\'on (\ref{Eq.3.16.a.2Sbis}):

\begin{eqnarray*}
G\left(z,w\right)&=&\sum_{n=0}^{\infty}G_{n}\left(z\right)w^{n}=G_{0}\left(z\right)+
\sum_{n=1}^{\infty}G_{n}\left(z\right)w^{n}=F\left(z\right)+\sum_{n=0}^{\infty}\left[G_{n}\left(z\right)-G_{n}\left(0\right)\right]w^{n}\frac{\tilde{P}\left(z\right)}{z}\\
&=&F\left(z\right)+\frac{w}{z}\sum_{n=0}^{\infty}\left[G_{n}\left(z\right)-G_{n}\left(0\right)\right]w^{n-1}\tilde{P}\left(z\right)\\
\end{eqnarray*}

es decir
\begin{eqnarray*}
G\left(z,w\right)&=&F\left(z\right)+\frac{w}{z}\left[G\left(z,w\right)-G\left(0,w\right)\right]\tilde{P}\left(z\right),
\end{eqnarray*}


entonces

\begin{eqnarray*}
G\left(z,w\right)=F\left(z\right)+\frac{w}{z}\left[G\left(z,w\right)-G\left(0,w\right)\right]\tilde{P}\left(z\right)&=&F\left(z\right)+\frac{w}{z}\tilde{P}\left(z\right)G\left(z,w\right)-\frac{w}{z}\tilde{P}\left(z\right)G\left(0,w\right)\\
&\Leftrightarrow&\\
G\left(z,w\right)\left\{1-\frac{w}{z}\tilde{P}\left(z\right)\right\}&=&F\left(z\right)-\frac{w}{z}\tilde{P}\left(z\right)G\left(0,w\right),
\end{eqnarray*}
por lo tanto,
\begin{equation}
G\left(z,w\right)=\frac{zF\left(z\right)-w\tilde{P}\left(z\right)G\left(0,w\right)}{1-w\tilde{P}\left(z\right)}.
\end{equation}


Ahora $G\left(z,w\right)$ es anal\'itica en $|z|=1$. Sean $z,w$ tales que $|z|=1$ y $|w|\leq1$, como $\tilde{P}\left(z\right)$ es FGP
\begin{eqnarray*}
|z-\left(z-w\tilde{P}\left(z\right)\right)|<|z|\Leftrightarrow|w\tilde{P}\left(z\right)|<|z|
\end{eqnarray*}
es decir, se cumplen las condiciones del Teorema de Rouch\'e y por
tanto, $z$ y $z-w\tilde{P}\left(z\right)$ tienen el mismo n\'umero de
ceros en $|z|=1$. Sea $z=\tilde{\theta}\left(w\right)$ la soluci\'on
\'unica de $z-w\tilde{P}\left(z\right)$, es decir

\begin{equation}\label{Eq.Theta.w}
\tilde{\theta}\left(w\right)-w\tilde{P}\left(\tilde{\theta}\left(w\right)\right)=0,
\end{equation}
 con $|\tilde{\theta}\left(w\right)|<1$. Cabe hacer menci\'on que $\tilde{\theta}\left(w\right)$ es la FGP para el tiempo de ruina cuando $\tilde{L}_{0}=1$.


Considerando la ecuaci\'on (\ref{Eq.Theta.w})
\begin{eqnarray*}
&&\frac{\partial}{\partial w}\tilde{\theta}\left(w\right)|_{w=1}-\frac{\partial}{\partial w}\left\{w\tilde{P}\left(\tilde{\theta}\left(w\right)\right)\right\}|_{w=1}=0\\
&&\tilde{\theta}^{(1)}\left(w\right)|_{w=1}-\frac{\partial}{\partial w}w\left\{\tilde{P}\left(\tilde{\theta}\left(w\right)\right)\right\}|_{w=1}-w\frac{\partial}{\partial w}\tilde{P}\left(\tilde{\theta}\left(w\right)\right)|_{w=1}=0\\
&&\tilde{\theta}^{(1)}\left(1\right)-\tilde{P}\left(\tilde{\theta}\left(1\right)\right)-w\left\{\frac{\partial \tilde{P}\left(\tilde{\theta}\left(w\right)\right)}{\partial \tilde{\theta}\left(w\right)}\cdot\frac{\partial\tilde{\theta}\left(w\right)}{\partial w}|_{w=1}\right\}=0\\
&&\tilde{\theta}^{(1)}\left(1\right)-\tilde{P}\left(\tilde{\theta}\left(1\right)
\right)-\tilde{P}^{(1)}\left(\tilde{\theta}\left(1\right)\right)\cdot\tilde{\theta}^{(1)}\left(1\right)=0
\end{eqnarray*}


luego
\begin{eqnarray*}
&&\tilde{\theta}^{(1)}\left(1\right)-\tilde{P}^{(1)}\left(\tilde{\theta}\left(1\right)\right)\cdot
\tilde{\theta}^{(1)}\left(1\right)=\tilde{P}\left(\tilde{\theta}\left(1\right)\right)\\
&&\tilde{\theta}^{(1)}\left(1\right)\left(1-\tilde{P}^{(1)}\left(\tilde{\theta}\left(1\right)\right)\right)
=\tilde{P}\left(\tilde{\theta}\left(1\right)\right)\\
&&\tilde{\theta}^{(1)}\left(1\right)=\frac{\tilde{P}\left(\tilde{\theta}\left(1\right)\right)}{\left(1-\tilde{P}^{(1)}\left(\tilde{\theta}\left(1\right)\right)\right)}=\frac{1}{1-\tilde{\mu}}.
\end{eqnarray*}

Ahora determinemos el segundo momento de $\tilde{\theta}\left(w\right)$,
nuevamente consideremos la ecuaci\'on (\ref{Eq.Theta.w}):

\begin{eqnarray*}
&&\tilde{\theta}\left(w\right)-w\tilde{P}\left(\tilde{\theta}\left(w\right)\right)=0\\
&&\frac{\partial}{\partial w}\left\{\tilde{\theta}\left(w\right)-w\tilde{P}\left(\tilde{\theta}\left(w\right)\right)\right\}=0\\
&&\frac{\partial}{\partial w}\left\{\frac{\partial}{\partial w}\left\{\tilde{\theta}\left(w\right)-w\tilde{P}\left(\tilde{\theta}\left(w\right)\right)\right\}\right\}=0\\
\end{eqnarray*}
luego
\begin{eqnarray*}
&&\frac{\partial}{\partial w}\left\{\frac{\partial}{\partial w}\tilde{\theta}\left(w\right)-\frac{\partial}{\partial w}\left[w\tilde{P}\left(\tilde{\theta}\left(w\right)\right)\right]\right\}
=\frac{\partial}{\partial w}\left\{\frac{\partial}{\partial w}\tilde{\theta}\left(w\right)-\frac{\partial}{\partial w}\left[w\tilde{P}\left(\tilde{\theta}\left(w\right)\right)\right]\right\}\\
&=&\frac{\partial}{\partial w}\left\{\frac{\partial \tilde{\theta}\left(w\right)}{\partial w}-\left[\tilde{P}\left(\tilde{\theta}\left(w\right)\right)+w\frac{\partial}{\partial w}R\left(\tilde{\theta}\left(w\right)\right)\right]\right\}\\
&=&\frac{\partial}{\partial w}\left\{\frac{\partial \tilde{\theta}\left(w\right)}{\partial w}-\left[\tilde{P}\left(\tilde{\theta}\left(w\right)\right)+w\frac{\partial \tilde{P}\left(\tilde{\theta}\left(w\right)\right)}{\partial w}\frac{\partial \tilde{\theta}\left(w\right)}{\partial w}\right]\right\}\\
&=&\frac{\partial}{\partial w}\left\{\tilde{\theta}^{(1)}\left(w\right)-\tilde{P}\left(\tilde{\theta}\left(w\right)\right)-w\tilde{P}^{(1)}\left(\tilde{\theta}\left(w\right)\right)\tilde{\theta}^{(1)}\left(w\right)\right\}\\
&=&\frac{\partial}{\partial w}\tilde{\theta}^{(1)}\left(w\right)-\frac{\partial}{\partial w}\tilde{P}\left(\tilde{\theta}\left(w\right)\right)-\frac{\partial}{\partial w}\left[w\tilde{P}^{(1)}\left(\tilde{\theta}\left(w\right)\right)\tilde{\theta}^{(1)}\left(w\right)\right]\\
\end{eqnarray*}
\begin{eqnarray*}
&=&\frac{\partial}{\partial
w}\tilde{\theta}^{(1)}\left(w\right)-\frac{\partial
\tilde{P}\left(\tilde{\theta}\left(w\right)\right)}{\partial
\tilde{\theta}\left(w\right)}\frac{\partial \tilde{\theta}\left(w\right)}{\partial
w}-\tilde{P}^{(1)}\left(\tilde{\theta}\left(w\right)\right)\tilde{\theta}^{(1)}\left(w\right)\\
&-&w\frac{\partial
\tilde{P}^{(1)}\left(\tilde{\theta}\left(w\right)\right)}{\partial
w}\tilde{\theta}^{(1)}\left(w\right)-w\tilde{P}^{(1)}\left(\tilde{\theta}\left(w\right)\right)\frac{\partial
\tilde{\theta}^{(1)}\left(w\right)}{\partial w}\\
&=&\tilde{\theta}^{(2)}\left(w\right)-\tilde{P}^{(1)}\left(\tilde{\theta}\left(w\right)\right)\tilde{\theta}^{(1)}\left(w\right)
-\tilde{P}^{(1)}\left(\tilde{\theta}\left(w\right)\right)\tilde{\theta}^{(1)}\left(w\right)\\
&-&w\tilde{P}^{(2)}\left(\tilde{\theta}\left(w\right)\right)\left(\tilde{\theta}^{(1)}\left(w\right)\right)^{2}-w\tilde{P}^{(1)}\left(\tilde{\theta}\left(w\right)\right)\tilde{\theta}^{(2)}\left(w\right)\\
&=&\tilde{\theta}^{(2)}\left(w\right)-2\tilde{P}^{(1)}\left(\tilde{\theta}\left(w\right)\right)\tilde{\theta}^{(1)}\left(w\right)\\
&-&w\tilde{P}^{(2)}\left(\tilde{\theta}\left(w\right)\right)\left(\tilde{\theta}^{(1)}\left(w\right)\right)^{2}-w\tilde{P}^{(1)}\left(\tilde{\theta}\left(w\right)\right)\tilde{\theta}^{(2)}\left(w\right)\\
&=&\tilde{\theta}^{(2)}\left(w\right)\left[1-w\tilde{P}^{(1)}\left(\tilde{\theta}\left(w\right)\right)\right]-
\tilde{\theta}^{(1)}\left(w\right)\left[w\tilde{\theta}^{(1)}\left(w\right)\tilde{P}^{(2)}\left(\tilde{\theta}\left(w\right)\right)+2\tilde{P}^{(1)}\left(\tilde{\theta}\left(w\right)\right)\right]
\end{eqnarray*}


luego

\begin{eqnarray*}
\tilde{\theta}^{(2)}\left(w\right)\left[1-w\tilde{P}^{(1)}\left(\tilde{\theta}\left(w\right)\right)\right]&-&\tilde{\theta}^{(1)}\left(w\right)\left[w\tilde{\theta}^{(1)}\left(w\right)\tilde{P}^{(2)}\left(\tilde{\theta}\left(w\right)\right)
+2\tilde{P}^{(1)}\left(\tilde{\theta}\left(w\right)\right)\right]=0\\
\tilde{\theta}^{(2)}\left(w\right)&=&\frac{\tilde{\theta}^{(1)}\left(w\right)\left[w\tilde{\theta}^{(1)}\left(w\right)\tilde{P}^{(2)}\left(\tilde{\theta}\left(w\right)\right)+2R^{(1)}\left(\tilde{\theta}\left(w\right)\right)\right]}{1-w\tilde{P}^{(1)}\left(\tilde{\theta}\left(w\right)\right)}\\
\tilde{\theta}^{(2)}\left(w\right)&=&\frac{\tilde{\theta}^{(1)}\left(w\right)w\tilde{\theta}^{(1)}\left(w\right)\tilde{P}^{(2)}\left(\tilde{\theta}\left(w\right)\right)}{1-w\tilde{P}^{(1)}\left(\tilde{\theta}\left(w\right)\right)}+\frac{2\tilde{\theta}^{(1)}\left(w\right)\tilde{P}^{(1)}\left(\tilde{\theta}\left(w\right)\right)}{1-w\tilde{P}^{(1)}\left(\tilde{\theta}\left(w\right)\right)}
\end{eqnarray*}


si evaluamos la expresi\'on anterior en $w=1$:
\begin{eqnarray*}
\tilde{\theta}^{(2)}\left(1\right)&=&\frac{\left(\tilde{\theta}^{(1)}\left(1\right)\right)^{2}\tilde{P}^{(2)}\left(\tilde{\theta}\left(1\right)\right)}{1-\tilde{P}^{(1)}\left(\tilde{\theta}\left(1\right)\right)}+\frac{2\tilde{\theta}^{(1)}\left(1\right)\tilde{P}^{(1)}\left(\tilde{\theta}\left(1\right)\right)}{1-\tilde{P}^{(1)}\left(\tilde{\theta}\left(1\right)\right)}=\frac{\left(\tilde{\theta}^{(1)}\left(1\right)\right)^{2}\tilde{P}^{(2)}\left(1\right)}{1-\tilde{P}^{(1)}\left(1\right)}+\frac{2\tilde{\theta}^{(1)}\left(1\right)\tilde{P}^{(1)}\left(1\right)}{1-\tilde{P}^{(1)}\left(1\right)}\\
&=&\frac{\left(\frac{1}{1-\tilde{\mu}}\right)^{2}\tilde{P}^{(2)}\left(1\right)}{1-\tilde{\mu}}+\frac{2\left(\frac{1}{1-\tilde{\mu}}\right)\tilde{\mu}}{1-\tilde{\mu}}=\frac{\tilde{P}^{(2)}\left(1\right)}{\left(1-\tilde{\mu}\right)^{3}}+\frac{2\tilde{\mu}}{\left(1-\tilde{\mu}\right)^{2}}=\frac{\sigma^{2}-\tilde{\mu}+\tilde{\mu}^{2}}{\left(1-\tilde{\mu}\right)^{3}}+\frac{2\tilde{\mu}}{\left(1-\tilde{\mu}\right)^{2}}\\
&=&\frac{\sigma^{2}-\tilde{\mu}+\tilde{\mu}^{2}+2\tilde{\mu}\left(1-\tilde{\mu}\right)}{\left(1-\tilde{\mu}\right)^{3}}\\
\end{eqnarray*}


es decir
\begin{eqnarray*}
\tilde{\theta}^{(2)}\left(1\right)&=&\frac{\sigma^{2}+\tilde{\mu}-\tilde{\mu}^{2}}{\left(1-\tilde{\mu}\right)^{3}}=\frac{\sigma^{2}}{\left(1-\tilde{\mu}\right)^{3}}+\frac{\tilde{\mu}\left(1-\tilde{\mu}\right)}{\left(1-\tilde{\mu}\right)^{3}}=\frac{\sigma^{2}}{\left(1-\tilde{\mu}\right)^{3}}+\frac{\tilde{\mu}}{\left(1-\tilde{\mu}\right)^{2}}.
\end{eqnarray*}

\begin{Coro}
El tiempo de ruina del jugador tiene primer y segundo momento
dados por

\begin{eqnarray}
\esp\left[T\right]&=&\frac{\esp\left[\tilde{L}_{0}\right]}{1-\tilde{\mu}}\\
Var\left[T\right]&=&\frac{Var\left[\tilde{L}_{0}\right]}{\left(1-\tilde{\mu}\right)^{2}}+\frac{\sigma^{2}\esp\left[\tilde{L}_{0}\right]}{\left(1-\tilde{\mu}\right)^{3}}.
\end{eqnarray}
\end{Coro}



%__________________________________________________________________________
\section{Procesos de Llegadas a las colas en la RSVC}
%__________________________________________________________________________

Se definen los procesos de llegada de los usuarios a cada una de
las colas dependiendo de la llegada del servidor pero del sistema
al cu\'al no pertenece la cola en cuesti\'on:

Para el sistema 1 y el servidor del segundo sistema

\begin{eqnarray*}
F_{i,j}\left(z_{i};\zeta_{j}\right)=\esp\left[z_{i}^{L_{i}\left(\zeta_{j}\right)}\right]=
\sum_{k=0}^{\infty}\prob\left[L_{i}\left(\zeta_{j}\right)=k\right]z_{i}^{k}\textrm{, para }i,j=1,2.
%F_{1,1}\left(z_{1};\zeta_{1}\right)&=&\esp\left[z_{1}^{L_{1}\left(\zeta_{1}\right)}\right]=
%\sum_{k=0}^{\infty}\prob\left[L_{1}\left(\zeta_{1}\right)=k\right]z_{1}^{k};\\
%F_{2,1}\left(z_{2};\zeta_{1}\right)&=&\esp\left[z_{2}^{L_{2}\left(\zeta_{1}\right)}\right]=
%\sum_{k=0}^{\infty}\prob\left[L_{2}\left(\zeta_{1}\right)=k\right]z_{2}^{k};\\
%F_{1,2}\left(z_{1};\zeta_{2}\right)&=&\esp\left[z_{1}^{L_{1}\left(\zeta_{2}\right)}\right]=
%\sum_{k=0}^{\infty}\prob\left[L_{1}\left(\zeta_{2}\right)=k\right]z_{1}^{k};\\
%F_{2,2}\left(z_{2};\zeta_{2}\right)&=&\esp\left[z_{2}^{L_{2}\left(\zeta_{2}\right)}\right]=
%\sum_{k=0}^{\infty}\prob\left[L_{2}\left(\zeta_{2}\right)=k\right]z_{2}^{k}.\\
\end{eqnarray*}

Ahora se definen para el segundo sistema y el servidor del primero


\begin{eqnarray*}
\hat{F}_{i,j}\left(w_{i};\tau_{j}\right)&=&\esp\left[w_{i}^{\hat{L}_{i}\left(\tau_{j}\right)}\right] =\sum_{k=0}^{\infty}\prob\left[\hat{L}_{i}\left(\tau_{j}\right)=k\right]w_{i}^{k}\textrm{, para }i,j=1,2.
%\hat{F}_{1,1}\left(w_{1};\tau_{1}\right)&=&\esp\left[w_{1}^{\hat{L}_{1}\left(\tau_{1}\right)}\right] =\sum_{k=0}^{\infty}\prob\left[\hat{L}_{1}\left(\tau_{1}\right)=k\right]w_{1}^{k}\\
%\hat{F}_{2,1}\left(w_{2};\tau_{1}\right)&=&\esp\left[w_{2}^{\hat{L}_{2}\left(\tau_{1}\right)}\right] =\sum_{k=0}^{\infty}\prob\left[\hat{L}_{2}\left(\tau_{1}\right)=k\right]w_{2}^{k}\\
%\hat{F}_{1,2}\left(w_{1};\tau_{2}\right)&=&\esp\left[w_{1}^{\hat{L}_{1}\left(\tau_{2}\right)}\right]
%=\sum_{k=0}^{\infty}\prob\left[\hat{L}_{1}\left(\tau_{2}\right)=k\right]w_{1}^{k}\\
%\hat{F}_{2,2}\left(w_{2};\tau_{2}\right)&=&\esp\left[w_{2}^{\hat{L}_{2}\left(\tau_{2}\right)}\right]
%=\sum_{k=0}^{\infty}\prob\left[\hat{L}_{2}\left(\tau_{2}\right)=k\right]w_{2}^{k}\\
\end{eqnarray*}


Ahora, con lo anterior definamos la FGP conjunta para el segundo sistema;% y $\tau_{1}$:


\begin{eqnarray*}
\esp\left[w_{1}^{\hat{L}_{1}\left(\tau_{j}\right)}w_{2}^{\hat{L}_{2}\left(\tau_{j}\right)}\right]
&=&\esp\left[w_{1}^{\hat{L}_{1}\left(\tau_{j}\right)}\right]
\esp\left[w_{2}^{\hat{L}_{2}\left(\tau_{j}\right)}\right]=\hat{F}_{1,j}\left(w_{1};\tau_{j}\right)\hat{F}_{2,j}\left(w_{2};\tau_{j}\right)=\hat{F}_{j}\left(w_{1},w_{2};\tau_{j}\right).\\
%\esp\left[w_{1}^{\hat{L}_{1}\left(\tau_{1}\right)}w_{2}^{\hat{L}_{2}\left(\tau_{1}\right)}\right]
%&=&\esp\left[w_{1}^{\hat{L}_{1}\left(\tau_{1}\right)}\right]
%\esp\left[w_{2}^{\hat{L}_{2}\left(\tau_{1}\right)}\right]=\hat{F}_{1,1}\left(w_{1};\tau_{1}\right)\hat{F}_{2,1}\left(w_{2};\tau_{1}\right)=\hat{F}_{1}\left(w_{1},w_{2};\tau_{1}\right)\\
%\esp\left[w_{1}^{\hat{L}_{1}\left(\tau_{2}\right)}w_{2}^{\hat{L}_{2}\left(\tau_{2}\right)}\right]
%&=&\esp\left[w_{1}^{\hat{L}_{1}\left(\tau_{2}\right)}\right]
%   \esp\left[w_{2}^{\hat{L}_{2}\left(\tau_{2}\right)}\right]=\hat{F}_{1,2}\left(w_{1};\tau_{2}\right)\hat{F}_{2,2}\left(w_{2};\tau_{2}\right)=\hat{F}_{2}\left(w_{1},w_{2};\tau_{2}\right).
\end{eqnarray*}

Con respecto al sistema 1 se tiene la FGP conjunta con respecto al servidor del otro sistema:
\begin{eqnarray*}
\esp\left[z_{1}^{L_{1}\left(\zeta_{j}\right)}z_{2}^{L_{2}\left(\zeta_{j}\right)}\right]
&=&\esp\left[z_{1}^{L_{1}\left(\zeta_{j}\right)}\right]
\esp\left[z_{2}^{L_{2}\left(\zeta_{j}\right)}\right]=F_{1,j}\left(z_{1};\zeta_{j}\right)F_{2,j}\left(z_{2};\zeta_{j}\right)=F_{j}\left(z_{1},z_{2};\zeta_{j}\right).
%\esp\left[z_{1}^{L_{1}\left(\zeta_{1}\right)}z_{2}^{L_{2}\left(\zeta_{1}\right)}\right]
%&=&\esp\left[z_{1}^{L_{1}\left(\zeta_{1}\right)}\right]
%\esp\left[z_{2}^{L_{2}\left(\zeta_{1}\right)}\right]=F_{1,1}\left(z_{1};\zeta_{1}\right)F_{2,1}\left(z_{2};\zeta_{1}\right)=F_{1}\left(z_{1},z_{2};\zeta_{1}\right)\\
%\esp\left[z_{1}^{L_{1}\left(\zeta_{2}\right)}z_{2}^{L_{2}\left(\zeta_{2}\right)}\right]
%&=&\esp\left[z_{1}^{L_{1}\left(\zeta_{2}\right)}\right]
%\esp\left[z_{2}^{L_{2}\left(\zeta_{2}\right)}\right]=F_{1,2}\left(z_{1};\zeta_{2}\right)F_{2,2}\left(z_{2};\zeta_{2}\right)=F_{2}\left(z_{1},z_{2};\zeta_{2}\right).
\end{eqnarray*}

Ahora analicemos la Red de Sistemas de Visitas C\'iclicas, entonces se define la PGF conjunta al tiempo $t$ para los tiempos de visita del servidor en cada una de las colas, para comenzar a dar servicio, definidos anteriormente al tiempo
$t=\left\{\tau_{1},\tau_{2},\zeta_{1},\zeta_{2}\right\}$:

\begin{eqnarray}\label{Eq.Conjuntas}
F_{j}\left(z_{1},z_{2},w_{1},w_{2}\right)&=&\esp\left[\prod_{i=1}^{2}z_{i}^{L_{i}\left(\tau_{j}
\right)}\prod_{i=1}^{2}w_{i}^{\hat{L}_{i}\left(\tau_{j}\right)}\right]\\
\hat{F}_{j}\left(z_{1},z_{2},w_{1},w_{2}\right)&=&\esp\left[\prod_{i=1}^{2}z_{i}^{L_{i}
\left(\zeta_{j}\right)}\prod_{i=1}^{2}w_{i}^{\hat{L}_{i}\left(\zeta_{j}\right)}\right]
\end{eqnarray}
para $j=1,2$. Entonces, con la finalidad de encontrar el n\'umero de usuarios
presentes en el sistema cuando el servidor deja de atender una de
las colas de cualquier sistema se tiene lo siguiente


\begin{eqnarray*}
&&\esp\left[z_{1}^{L_{1}\left(\overline{\tau}_{1}\right)}z_{2}^{L_{2}\left(\overline{\tau}_{1}\right)}w_{1}^{\hat{L}_{1}\left(\overline{\tau}_{1}\right)}w_{2}^{\hat{L}_{2}\left(\overline{\tau}_{1}\right)}\right]=
\esp\left[z_{2}^{L_{2}\left(\overline{\tau}_{1}\right)}w_{1}^{\hat{L}_{1}\left(\overline{\tau}_{1}
\right)}w_{2}^{\hat{L}_{2}\left(\overline{\tau}_{1}\right)}\right]\\
&=&\esp\left[z_{2}^{L_{2}\left(\tau_{1}\right)+X_{2}\left(\overline{\tau}_{1}-\tau_{1}\right)+Y_{2}\left(\overline{\tau}_{1}-\tau_{1}\right)}w_{1}^{\hat{L}_{1}\left(\tau_{1}\right)+\hat{X}_{1}\left(\overline{\tau}_{1}-\tau_{1}\right)}w_{2}^{\hat{L}_{2}\left(\tau_{1}\right)+\hat{X}_{2}\left(\overline{\tau}_{1}-\tau_{1}\right)}\right]
\end{eqnarray*}
utilizando la ecuacion dada (\ref{Eq.TiemposLlegada}), luego


\begin{eqnarray*}
&=&\esp\left[z_{2}^{L_{2}\left(\tau_{1}\right)}z_{2}^{X_{2}\left(\overline{\tau}_{1}-\tau_{1}\right)}z_{2}^{Y_{2}\left(\overline{\tau}_{1}-\tau_{1}\right)}w_{1}^{\hat{L}_{1}\left(\tau_{1}\right)}w_{1}^{\hat{X}_{1}\left(\overline{\tau}_{1}-\tau_{1}\right)}w_{2}^{\hat{L}_{2}\left(\tau_{1}\right)}w_{2}^{\hat{X}_{2}\left(\overline{\tau}_{1}-\tau_{1}\right)}\right]\\
&=&\esp\left[z_{2}^{L_{2}\left(\tau_{1}\right)}\left\{w_{1}^{\hat{L}_{1}\left(\tau_{1}\right)}w_{2}^{\hat{L}_{2}\left(\tau_{1}\right)}\right\}\left\{z_{2}^{X_{2}\left(\overline{\tau}_{1}-\tau_{1}\right)}
z_{2}^{Y_{2}\left(\overline{\tau}_{1}-\tau_{1}\right)}w_{1}^{\hat{X}_{1}\left(\overline{\tau}_{1}-\tau_{1}\right)}w_{2}^{\hat{X}_{2}\left(\overline{\tau}_{1}-\tau_{1}\right)}\right\}\right]\\
\end{eqnarray*}
Aplicando la ecuaci\'on (\ref{Eq.Cero})

\begin{eqnarray*}
&=&\esp\left[z_{2}^{L_{2}\left(\tau_{1}\right)}\left\{z_{2}^{X_{2}\left(\overline{\tau}_{1}-\tau_{1}\right)}z_{2}^{Y_{2}\left(\overline{\tau}_{1}-\tau_{1}\right)}w_{1}^{\hat{X}_{1}\left(\overline{\tau}_{1}-\tau_{1}\right)}w_{2}^{\hat{X}_{2}\left(\overline{\tau}_{1}-\tau_{1}\right)}\right\}\right]\esp\left[w_{1}^{\hat{L}_{1}\left(\tau_{1}\right)}w_{2}^{\hat{L}_{2}\left(\tau_{1}\right)}\right]
\end{eqnarray*}
dado que los arribos a cada una de las colas son independientes, podemos separar la esperanza para los procesos de llegada a $Q_{1}$ y $Q_{2}$ en $\tau_{1}$

Recordando que $\tilde{X}_{2}\left(z_{2}\right)=X_{2}\left(z_{2}\right)+Y_{2}\left(z_{2}\right)$ se tiene


\begin{eqnarray*}
&=&\esp\left[z_{2}^{L_{2}\left(\tau_{1}\right)}\left\{z_{2}^{\tilde{X}_{2}\left(\overline{\tau}_{1}-\tau_{1}\right)}w_{1}^{\hat{X}_{1}\left(\overline{\tau}_{1}-\tau_{1}\right)}w_{2}^{\hat{X}_{2}\left(\overline{\tau}_{1}-\tau_{1}\right)}\right\}\right]\esp\left[w_{1}^{\hat{L}_{1}\left(\tau_{1}\right)}w_{2}^{\hat{L}_{2}\left(\tau_{1}\right)}\right]\\
&=&\esp\left[z_{2}^{L_{2}\left(\tau_{1}\right)}\left\{\tilde{P}_{2}\left(z_{2}\right)^{\overline{\tau}_{1}-\tau_{1}}\hat{P}_{1}\left(w_{1}\right)^{\overline{\tau}_{1}-\tau_{1}}\hat{P}_{2}\left(w_{2}\right)^{\overline{\tau}_{1}-\tau_{1}}\right\}\right]\esp\left[w_{1}^{\hat{L}_{1}\left(\tau_{1}\right)}w_{2}^{\hat{L}_{2}\left(\tau_{1}\right)}\right]\\
&=&\esp\left[z_{2}^{L_{2}\left(\tau_{1}\right)}\left\{\tilde{P}_{2}\left(z_{2}\right)\hat{P}_{1}\left(w_{1}\right)\hat{P}_{2}\left(w_{2}\right)\right\}^{\overline{\tau}_{1}-\tau_{1}}\right]\esp\left[w_{1}^{\hat{L}_{1}\left(\tau_{1}\right)}w_{2}^{\hat{L}_{2}\left(\tau_{1}\right)}\right]\\
\end{eqnarray*}

Entonces


\begin{eqnarray*}
&=&\esp\left[z_{2}^{L_{2}\left(\tau_{1}\right)}\theta_{1}\left(\tilde{P}_{2}\left(z_{2}\right)\hat{P}_{1}\left(w_{1}\right)\hat{P}_{2}\left(w_{2}\right)\right)^{L_{1}\left(\tau_{1}\right)}\right]\esp\left[w_{1}^{\hat{L}_{1}\left(\tau_{1}\right)}w_{2}^{\hat{L}_{2}\left(\tau_{1}\right)}\right]\\
&=&F_{1}\left(\theta_{1}\left(\tilde{P}_{2}\left(z_{2}\right)\hat{P}_{1}\left(w_{1}\right)\hat{P}_{2}\left(w_{2}\right)\right),z{2}\right)\hat{F}_{1}\left(w_{1},w_{2};\tau_{1}\right)\\
&\equiv&
F_{1}\left(\theta_{1}\left(\tilde{P}_{2}\left(z_{2}\right)\hat{P}_{1}\left(w_{1}\right)\hat{P}_{2}\left(w_{2}\right)\right),z_{2},w_{1},w_{2}\right)
\end{eqnarray*}

Las igualdades anteriores son ciertas pues el n\'umero de usuarios
que llegan a $\hat{Q}_{2}$ durante el intervalo
$\left[\tau_{1},\overline{\tau}_{1}\right]$ a\'un no han sido
atendidos por el servidor del sistema $2$ y por tanto a\'un no
pueden pasar al sistema $1$ por $Q_{2}$. Por tanto el n\'umero de
usuarios que pasan de $\hat{Q}_{2}$ a $Q_{2}$ en el intervalo de
tiempo $\left[\tau_{1},\overline{\tau}_{1}\right]$ depende de la
pol\'itica de traslado entre los dos sistemas, conforme a la
secci\'on anterior.\smallskip

Por lo tanto
\begin{eqnarray}\label{Eq.Fs}
\esp\left[z_{1}^{L_{1}\left(\overline{\tau}_{1}\right)}z_{2}^{L_{2}\left(\overline{\tau}_{1}
\right)}w_{1}^{\hat{L}_{1}\left(\overline{\tau}_{1}\right)}w_{2}^{\hat{L}_{2}\left(
\overline{\tau}_{1}\right)}\right]&=&F_{1}\left(\theta_{1}\left(\tilde{P}_{2}\left(z_{2}\right)
\hat{P}_{1}\left(w_{1}\right)\hat{P}_{2}\left(w_{2}\right)\right),z_{2},w_{1},w_{2}\right)\\
&=&F_{1}\left(\theta_{1}\left(\tilde{P}_{2}\left(z_{2}\right)\hat{P}_{1}\left(w_{1}\right)\hat{P}_{2}\left(w_{2}\right)\right),z{2}\right)\hat{F}_{1}\left(w_{1},w_{2};\tau_{1}\right)
\end{eqnarray}


Utilizando un razonamiento an\'alogo para $\overline{\tau}_{2}$:



\begin{eqnarray*}
&&\esp\left[z_{1}^{L_{1}\left(\overline{\tau}_{2}\right)}z_{2}^{L_{2}\left(\overline{\tau}_{2}\right)}w_{1}^{\hat{L}_{1}\left(\overline{\tau}_{2}\right)}w_{2}^{\hat{L}_{2}\left(\overline{\tau}_{2}\right)}\right]=
\esp\left[z_{1}^{L_{1}\left(\overline{\tau}_{2}\right)}w_{1}^{\hat{L}_{1}\left(\overline{\tau}_{2}\right)}w_{2}^{\hat{L}_{2}\left(\overline{\tau}_{2}\right)}\right]\\
&=&\esp\left[z_{1}^{L_{1}\left(\tau_{2}\right)+X_{1}\left(\overline{\tau}_{2}-\tau_{2}\right)}w_{1}^{\hat{L}_{1}\left(\tau_{2}\right)+\hat{X}_{1}\left(\overline{\tau}_{2}-\tau_{2}\right)}w_{2}^{\hat{L}_{2}\left(\tau_{2}\right)+\hat{X}_{2}\left(\overline{\tau}_{2}-\tau_{2}\right)}\right]\\
&=&\esp\left[z_{1}^{L_{1}\left(\tau_{2}\right)}z_{1}^{X_{1}\left(\overline{\tau}_{2}-\tau_{2}\right)}w_{1}^{\hat{L}_{1}\left(\tau_{2}\right)}w_{1}^{\hat{X}_{1}\left(\overline{\tau}_{2}-\tau_{2}\right)}w_{2}^{\hat{L}_{2}\left(\tau_{2}\right)}w_{2}^{\hat{X}_{2}\left(\overline{\tau}_{2}-\tau_{2}\right)}\right]\\
&=&\esp\left[z_{1}^{L_{1}\left(\tau_{2}\right)}z_{1}^{X_{1}\left(\overline{\tau}_{2}-\tau_{2}\right)}w_{1}^{\hat{X}_{1}\left(\overline{\tau}_{2}-\tau_{2}\right)}w_{2}^{\hat{X}_{2}\left(\overline{\tau}_{2}-\tau_{2}\right)}\right]\esp\left[w_{1}^{\hat{L}_{1}\left(\tau_{2}\right)}w_{2}^{\hat{L}_{2}\left(\tau_{2}\right)}\right]\\
&=&\esp\left[z_{1}^{L_{1}\left(\tau_{2}\right)}P_{1}\left(z_{1}\right)^{\overline{\tau}_{2}-\tau_{2}}\hat{P}_{1}\left(w_{1}\right)^{\overline{\tau}_{2}-\tau_{2}}\hat{P}_{2}\left(w_{2}\right)^{\overline{\tau}_{2}-\tau_{2}}\right]
\esp\left[w_{1}^{\hat{L}_{1}\left(\tau_{2}\right)}w_{2}^{\hat{L}_{2}\left(\tau_{2}\right)}\right]
\end{eqnarray*}
utlizando la proposici\'on relacionada con la ruina del jugador


\begin{eqnarray*}
&=&\esp\left[z_{1}^{L_{1}\left(\tau_{2}\right)}\left\{P_{1}\left(z_{1}\right)\hat{P}_{1}\left(w_{1}\right)\hat{P}_{2}\left(w_{2}\right)\right\}^{\overline{\tau}_{2}-\tau_{2}}\right]
\esp\left[w_{1}^{\hat{L}_{1}\left(\tau_{2}\right)}w_{2}^{\hat{L}_{2}\left(\tau_{2}\right)}\right]\\
&=&\esp\left[z_{1}^{L_{1}\left(\tau_{2}\right)}\tilde{\theta}_{2}\left(P_{1}\left(z_{1}\right)\hat{P}_{1}\left(w_{1}\right)\hat{P}_{2}\left(w_{2}\right)\right)^{L_{2}\left(\tau_{2}\right)}\right]
\esp\left[w_{1}^{\hat{L}_{1}\left(\tau_{2}\right)}w_{2}^{\hat{L}_{2}\left(\tau_{2}\right)}\right]\\
&=&F_{2}\left(z_{1},\tilde{\theta}_{2}\left(P_{1}\left(z_{1}\right)\hat{P}_{1}\left(w_{1}\right)\hat{P}_{2}\left(w_{2}\right)\right)\right)
\hat{F}_{2}\left(w_{1},w_{2};\tau_{2}\right)\\
\end{eqnarray*}


entonces se define
\begin{eqnarray}
\esp\left[z_{1}^{L_{1}\left(\overline{\tau}_{2}\right)}z_{2}^{L_{2}\left(\overline{\tau}_{2}\right)}w_{1}^{\hat{L}_{1}\left(\overline{\tau}_{2}\right)}w_{2}^{\hat{L}_{2}\left(\overline{\tau}_{2}\right)}\right]=F_{2}\left(z_{1},\tilde{\theta}_{2}\left(P_{1}\left(z_{1}\right)\hat{P}_{1}\left(w_{1}\right)\hat{P}_{2}\left(w_{2}\right)\right),w_{1},w_{2}\right)\\
\equiv F_{2}\left(z_{1},\tilde{\theta}_{2}\left(P_{1}\left(z_{1}\right)\hat{P}_{1}\left(w_{1}\right)\hat{P}_{2}\left(w_{2}\right)\right)\right)
\hat{F}_{2}\left(w_{1},w_{2};\tau_{2}\right)
\end{eqnarray}
Ahora para $\overline{\zeta}_{1}:$
\begin{eqnarray*}
&&\esp\left[z_{1}^{L_{1}\left(\overline{\zeta}_{1}\right)}z_{2}^{L_{2}\left(\overline{\zeta}_{1}\right)}w_{1}^{\hat{L}_{1}\left(\overline{\zeta}_{1}\right)}w_{2}^{\hat{L}_{2}\left(\overline{\zeta}_{1}\right)}\right]=
\esp\left[z_{1}^{L_{1}\left(\overline{\zeta}_{1}\right)}z_{2}^{L_{2}\left(\overline{\zeta}_{1}\right)}w_{2}^{\hat{L}_{2}\left(\overline{\zeta}_{1}\right)}\right]\\
%&=&\esp\left[z_{1}^{L_{1}\left(\zeta_{1}\right)+X_{1}\left(\overline{\zeta}_{1}-\zeta_{1}\right)}z_{2}^{L_{2}\left(\zeta_{1}\right)+X_{2}\left(\overline{\zeta}_{1}-\zeta_{1}\right)+\hat{Y}_{2}\left(\overline{\zeta}_{1}-\zeta_{1}\right)}w_{2}^{\hat{L}_{2}\left(\zeta_{1}\right)+\hat{X}_{2}\left(\overline{\zeta}_{1}-\zeta_{1}\right)}\right]\\
&=&\esp\left[z_{1}^{L_{1}\left(\zeta_{1}\right)}z_{1}^{X_{1}\left(\overline{\zeta}_{1}-\zeta_{1}\right)}z_{2}^{L_{2}\left(\zeta_{1}\right)}z_{2}^{X_{2}\left(\overline{\zeta}_{1}-\zeta_{1}\right)}
z_{2}^{Y_{2}\left(\overline{\zeta}_{1}-\zeta_{1}\right)}w_{2}^{\hat{L}_{2}\left(\zeta_{1}\right)}w_{2}^{\hat{X}_{2}\left(\overline{\zeta}_{1}-\zeta_{1}\right)}\right]\\
&=&\esp\left[z_{1}^{L_{1}\left(\zeta_{1}\right)}z_{2}^{L_{2}\left(\zeta_{1}\right)}\right]\esp\left[z_{1}^{X_{1}\left(\overline{\zeta}_{1}-\zeta_{1}\right)}z_{2}^{\tilde{X}_{2}\left(\overline{\zeta}_{1}-\zeta_{1}\right)}w_{2}^{\hat{X}_{2}\left(\overline{\zeta}_{1}-\zeta_{1}\right)}w_{2}^{\hat{L}_{2}\left(\zeta_{1}\right)}\right]\\
&=&\esp\left[z_{1}^{L_{1}\left(\zeta_{1}\right)}z_{2}^{L_{2}\left(\zeta_{1}\right)}\right]
\esp\left[P_{1}\left(z_{1}\right)^{\overline{\zeta}_{1}-\zeta_{1}}\tilde{P}_{2}\left(z_{2}\right)^{\overline{\zeta}_{1}-\zeta_{1}}\hat{P}_{2}\left(w_{2}\right)^{\overline{\zeta}_{1}-\zeta_{1}}w_{2}^{\hat{L}_{2}\left(\zeta_{1}\right)}\right]\\
&=&\esp\left[z_{1}^{L_{1}\left(\zeta_{1}\right)}z_{2}^{L_{2}\left(\zeta_{1}\right)}\right]
\esp\left[\left\{P_{1}\left(z_{1}\right)\tilde{P}_{2}\left(z_{2}\right)\hat{P}_{2}\left(w_{2}\right)\right\}^{\overline{\zeta}_{1}-\zeta_{1}}w_{2}^{\hat{L}_{2}\left(\zeta_{1}\right)}\right]\\
&=&\esp\left[z_{1}^{L_{1}\left(\zeta_{1}\right)}z_{2}^{L_{2}\left(\zeta_{1}\right)}\right]
\esp\left[\hat{\theta}_{1}\left(P_{1}\left(z_{1}\right)\tilde{P}_{2}\left(z_{2}\right)\hat{P}_{2}\left(w_{2}\right)\right)^{\hat{L}_{1}\left(\zeta_{1}\right)}w_{2}^{\hat{L}_{2}\left(\zeta_{1}\right)}\right]\\
&=&F_{1}\left(z_{1},z_{2};\zeta_{1}\right)\hat{F}_{1}\left(\hat{\theta}_{1}\left(P_{1}\left(z_{1}\right)\tilde{P}_{2}\left(z_{2}\right)\hat{P}_{2}\left(w_{2}\right)\right),w_{2}\right)
\end{eqnarray*}


es decir
\begin{eqnarray}
\esp\left[z_{1}^{L_{1}\left(\overline{\zeta}_{1}\right)}z_{2}^{L_{2}\left(\overline{\zeta}_{1}
\right)}w_{1}^{\hat{L}_{1}\left(\overline{\zeta}_{1}\right)}w_{2}^{\hat{L}_{2}\left(
\overline{\zeta}_{1}\right)}\right]&=&\hat{F}_{1}\left(z_{1},z_{2},\hat{\theta}_{1}\left(P_{1}\left(z_{1}\right)\tilde{P}_{2}\left(z_{2}\right)\hat{P}_{2}\left(w_{2}\right)\right),w_{2}\right)\\
&=&F_{1}\left(z_{1},z_{2};\zeta_{1}\right)\hat{F}_{1}\left(\hat{\theta}_{1}\left(P_{1}\left(z_{1}\right)\tilde{P}_{2}\left(z_{2}\right)\hat{P}_{2}\left(w_{2}\right)\right),w_{2}\right).
\end{eqnarray}


Finalmente para $\overline{\zeta}_{2}:$
\begin{eqnarray*}
&&\esp\left[z_{1}^{L_{1}\left(\overline{\zeta}_{2}\right)}z_{2}^{L_{2}\left(\overline{\zeta}_{2}\right)}w_{1}^{\hat{L}_{1}\left(\overline{\zeta}_{2}\right)}w_{2}^{\hat{L}_{2}\left(\overline{\zeta}_{2}\right)}\right]=
\esp\left[z_{1}^{L_{1}\left(\overline{\zeta}_{2}\right)}z_{2}^{L_{2}\left(\overline{\zeta}_{2}\right)}w_{1}^{\hat{L}_{1}\left(\overline{\zeta}_{2}\right)}\right]\\
%&=&\esp\left[z_{1}^{L_{1}\left(\zeta_{2}\right)+X_{1}\left(\overline{\zeta}_{2}-\zeta_{2}\right)}z_{2}^{L_{2}\left(\zeta_{2}\right)+X_{2}\left(\overline{\zeta}_{2}-\zeta_{2}\right)+\hat{Y}_{2}\left(\overline{\zeta}_{2}-\zeta_{2}\right)}w_{1}^{\hat{L}_{1}\left(\zeta_{2}\right)+\hat{X}_{1}\left(\overline{\zeta}_{2}-\zeta_{2}\right)}\right]\\
&=&\esp\left[z_{1}^{L_{1}\left(\zeta_{2}\right)}z_{1}^{X_{1}\left(\overline{\zeta}_{2}-\zeta_{2}\right)}z_{2}^{L_{2}\left(\zeta_{2}\right)}z_{2}^{X_{2}\left(\overline{\zeta}_{2}-\zeta_{2}\right)}
z_{2}^{Y_{2}\left(\overline{\zeta}_{2}-\zeta_{2}\right)}w_{1}^{\hat{L}_{1}\left(\zeta_{2}\right)}w_{1}^{\hat{X}_{1}\left(\overline{\zeta}_{2}-\zeta_{2}\right)}\right]\\
&=&\esp\left[z_{1}^{L_{1}\left(\zeta_{2}\right)}z_{2}^{L_{2}\left(\zeta_{2}\right)}\right]\esp\left[z_{1}^{X_{1}\left(\overline{\zeta}_{2}-\zeta_{2}\right)}z_{2}^{\tilde{X}_{2}\left(\overline{\zeta}_{2}-\zeta_{2}\right)}w_{1}^{\hat{X}_{1}\left(\overline{\zeta}_{2}-\zeta_{2}\right)}w_{1}^{\hat{L}_{1}\left(\zeta_{2}\right)}\right]\\
&=&\esp\left[z_{1}^{L_{1}\left(\zeta_{2}\right)}z_{2}^{L_{2}\left(\zeta_{2}\right)}\right]\esp\left[P_{1}\left(z_{1}\right)^{\overline{\zeta}_{2}-\zeta_{2}}\tilde{P}_{2}\left(z_{2}\right)^{\overline{\zeta}_{2}-\zeta_{2}}\hat{P}\left(w_{1}\right)^{\overline{\zeta}_{2}-\zeta_{2}}w_{1}^{\hat{L}_{1}\left(\zeta_{2}\right)}\right]\\
&=&\esp\left[z_{1}^{L_{1}\left(\zeta_{2}\right)}z_{2}^{L_{2}\left(\zeta_{2}\right)}\right]\esp\left[w_{1}^{\hat{L}_{1}\left(\zeta_{2}\right)}\left\{P_{1}\left(z_{1}\right)\tilde{P}_{2}\left(z_{2}\right)\hat{P}\left(w_{1}\right)\right\}^{\overline{\zeta}_{2}-\zeta_{2}}\right]\\
&=&\esp\left[z_{1}^{L_{1}\left(\zeta_{2}\right)}z_{2}^{L_{2}\left(\zeta_{2}\right)}\right]\esp\left[w_{1}^{\hat{L}_{1}\left(\zeta_{2}\right)}\hat{\theta}_{2}\left(P_{1}\left(z_{1}\right)\tilde{P}_{2}\left(z_{2}\right)\hat{P}\left(w_{1}\right)\right)^{\hat{L}_{2}\zeta_{2}}\right]\\
&=&F_{2}\left(z_{1},z_{2};\zeta_{2}\right)\hat{F}_{2}\left(w_{1},\hat{\theta}_{2}\left(P_{1}\left(z_{1}\right)\tilde{P}_{2}\left(z_{2}\right)\hat{P}_{1}\left(w_{1}\right)\right)\right]\\
%&\equiv&\hat{F}_{2}\left(z_{1},z_{2},w_{1},\hat{\theta}_{2}\left(P_{1}\left(z_{1}\right)\tilde{P}_{2}\left(z_{2}\right)\hat{P}_{1}\left(w_{1}\right)\right)\right)
\end{eqnarray*}

es decir
\begin{eqnarray}
\esp\left[z_{1}^{L_{1}\left(\overline{\zeta}_{2}\right)}z_{2}^{L_{2}\left(\overline{\zeta}_{2}\right)}w_{1}^{\hat{L}_{1}\left(\overline{\zeta}_{2}\right)}w_{2}^{\hat{L}_{2}\left(\overline{\zeta}_{2}\right)}\right]=\hat{F}_{2}\left(z_{1},z_{2},w_{1},\hat{\theta}_{2}\left(P_{1}\left(z_{1}\right)\tilde{P}_{2}\left(z_{2}\right)\hat{P}_{1}\left(w_{1}\right)\right)\right)\\
=F_{2}\left(z_{1},z_{2};\zeta_{2}\right)\hat{F}_{2}\left(w_{1},\hat{\theta}_{2}\left(P_{1}\left(z_{1}\right)\tilde{P}_{2}\left(z_{2}\right)\hat{P}_{1}\left(w_{1}
\right)\right)\right)
\end{eqnarray}
%__________________________________________________________________________
\section{Ecuaciones Recursivas para la R.S.V.C.}
%__________________________________________________________________________




Con lo desarrollado hasta ahora podemos encontrar las ecuaciones
recursivas que modelan la Red de Sistemas de Visitas C\'iclicas
(R.S.V.C):
\begin{eqnarray*}
&&F_{2}\left(z_{1},z_{2},w_{1},w_{2}\right)=R_{1}\left(z_{1},z_{2},w_{1},w_{2}\right)\esp\left[z_{1}^{L_{1}\left(\overline{\tau}_{1}\right)}z_{2}^{L_{2}\left(\overline{\tau}_{1}\right)}w_{1}^{\hat{L}_{1}\left(\overline{\tau}_{1}\right)}w_{2}^{\hat{L}_{2}\left(\overline{\tau}_{1}\right)}\right]\\
&&F_{1}\left(z_{1},z_{2},w_{1},w_{2}\right)=R_{2}\left(z_{1},z_{2},w_{1},w_{2}\right)\esp\left[z_{1}^{L_{1}\left(\overline{\tau}_{2}\right)}z_{2}^{L_{2}\left(\overline{\tau}_{2}\right)}w_{1}^{\hat{L}_{1}\left(\overline{\tau}_{2}\right)}w_{2}^{\hat{L}_{2}\left(\overline{\tau}_{1}\right)}\right]\\
&&\hat{F}_{2}\left(z_{1},z_{2},w_{1},w_{2}\right)=\hat{R}_{1}\left(z_{1},z_{2},w_{1},w_{2}\right)\esp\left[z_{1}^{L_{1}\left(\overline{\zeta}_{1}\right)}z_{2}^{L_{2}\left(\overline{\zeta}_{1}\right)}w_{1}^{\hat{L}_{1}\left(\overline{\zeta}_{1}\right)}w_{2}^{\hat{L}_{2}\left(\overline{\zeta}_{1}\right)}\right]\\
&&\hat{F}_{1}\left(z_{1},z_{2},w_{1},w_{2}\right)=\hat{R}_{2}\left(z_{1},z_{2},
w_{1},w_{2}\right)\esp\left[z_{1}^{L_{1}\left(\overline{\zeta}_{2}\right)}z_{2}
^{L_{2}\left(\overline{\zeta}_{2}\right)}w_{1}^{\hat{L}_{1}\left(
\overline{\zeta}_{2}\right)}w_{2}^{\hat{L}_{2}\left(\overline{\zeta}_{2}\right)}
\right]
\end{eqnarray*}

%&=&R_{1}\left(P_{1}\left(z_{1}\right)\tilde{P}_{2}\left(z_{2}\right)\hat{P}_{1}\left(w_{1}\right)\hat{P}_{2}\left(w_{2}\right)\right)
%F_{1}\left(\theta\left(\tilde{P}_{2}\left(z_{2}\right)\hat{P}_{1}\left(w_{1}\right)\hat{P}_{2}\left(w_{2}\right)\right),z_{2},w_{1},w_{2}\right)\\
%&=&R_{2}\left(P_{1}\left(z_{1}\right)\tilde{P}_{2}\left(z_{2}\right)\hat{P}_{1}\left(w_{1}\right)\hat{P}_{2}\left(w_{2}\right)\right)F_{2}\left(z_{1},\tilde{\theta}_{2}\left(P_{1}\left(z_{1}\right)\hat{P}_{1}\left(w_{1}\right)\hat{P}_{2}\left(w_{2}\right)\right),w_{1},w_{2}\right)\\
%&=&\hat{R}_{1}\left(P_{1}\left(z_{1}\right)\tilde{P}_{2}\left(z_{2}\right)\hat{P}_{1}\left(w_{1}\right)\hat{P}_{2}\left(w_{2}\right)\right)\hat{F}_{1}\left(z_{1},z_{2},\hat{\theta}_{1}\left(P_{1}\left(z_{1}\right)\tilde{P}_{2}\left(z_{2}\right)\hat{P}_{2}\left(w_{2}\right)\right),w_{2}\right)
%&=&\hat{R}_{2}\left(P_{1}\left(z_{1}\right)\tilde{P}_{2}\left(z_{2}\right)\hat{P}_{1}\left(w_{1}\right)\hat{P}_{2}\left(w_{2}\right)\right)\hat{F}_{2}\left(z_{1},z_{2},w_{1},\hat{\theta}_{2}\left(P_{1}\left(z_{1}\right)\tilde{P}_{2}\left(z_{2}\right)\hat{P}_{1}\left(w_{1}\right)\right)\right)


que son equivalentes a las siguientes ecuaciones
\begin{eqnarray}
F_{2}\left(z_{1},z_{2},w_{1},w_{2}\right)&=&R_{1}\left(P_{1}\left(z_{1}\right)\tilde{P}_{2}\left(z_{2}\right)\prod_{i=1}^{2}
\hat{P}_{i}\left(w_{i}\right)\right)F_{1}\left(\theta_{1}\left(\tilde{P}_{2}\left(z_{2}\right)\hat{P}_{1}\left(w_{1}\right)\hat{P}_{2}\left(w_{2}\right)\right),z_{2},w_{1},w_{2}\right)\\
F_{1}\left(z_{1},z_{2},w_{1},w_{2}\right)&=&R_{2}\left(P_{1}\left(z_{1}\right)\tilde{P}_{2}\left(z_{2}\right)\prod_{i=1}^{2}
\hat{P}_{i}\left(w_{i}\right)\right)F_{2}\left(z_{1},\tilde{\theta}_{2}\left(P_{1}\left(z_{1}\right)\hat{P}_{1}\left(w_{1}\right)\hat{P}_{2}\left(w_{2}\right)\right),w_{1},w_{2}\right)\\
\hat{F}_{2}\left(z_{1},z_{2},w_{1},w_{2}\right)&=&\hat{R}_{1}\left(P_{1}\left(z_{1}\right)\tilde{P}_{2}\left(z_{2}\right)\prod_{i=1}^{2}
\hat{P}_{i}\left(w_{i}\right)\right)\hat{F}_{1}\left(z_{1},z_{2},\hat{\theta}_{1}\left(P_{1}\left(z_{1}\right)\tilde{P}_{2}\left(z_{2}\right)\hat{P}_{2}\left(w_{2}\right)\right),w_{2}\right)\\
\hat{F}_{1}\left(z_{1},z_{2},w_{1},w_{2}\right)&=&\hat{R}_{2}\left(P_{1}\left(z_{1}\right)\tilde{P}_{2}\left(z_{2}\right)\prod_{i=1}^{2}
\hat{P}_{i}\left(w_{i}\right)\right)\hat{F}_{2}\left(z_{1},z_{2},w_{1},\hat{\theta}_{2}\left(P_{1}\left(z_{1}\right)\tilde{P}_{2}\left(z_{2}\right)
\hat{P}_{1}\left(w_{1}\right)\right)\right)
\end{eqnarray}



%_________________________________________________________________________________________________
\subsection{Tiempos de Traslado del Servidor}
%_________________________________________________________________________________________________


Para
%\begin{multicols}{2}

\begin{eqnarray}\label{Ec.R1}
R_{1}\left(\mathbf{z,w}\right)=R_{1}\left((P_{1}\left(z_{1}\right)\tilde{P}_{2}\left(z_{2}\right)\hat{P}_{1}\left(w_{1}\right)\hat{P}_{2}\left(w_{2}\right)\right)
\end{eqnarray}
%\end{multicols}

se tiene que


\begin{eqnarray*}
\begin{array}{cc}
\frac{\partial R_{1}\left(\mathbf{z,w}\right)}{\partial
z_{1}}|_{\mathbf{z,w}=1}=R_{1}^{(1)}\left(1\right)P_{1}^{(1)}\left(1\right)=r_{1}\mu_{1},&
\frac{\partial R_{1}\left(\mathbf{z,w}\right)}{\partial
z_{2}}|_{\mathbf{z,w}=1}=R_{1}^{(1)}\left(1\right)\tilde{P}_{2}^{(1)}\left(1\right)=r_{1}\tilde{\mu}_{2},\\
\frac{\partial R_{1}\left(\mathbf{z,w}\right)}{\partial
w_{1}}|_{\mathbf{z,w}=1}=R_{1}^{(1)}\left(1\right)\hat{P}_{1}^{(1)}\left(1\right)=r_{1}\hat{\mu}_{1},&
\frac{\partial R_{1}\left(\mathbf{z,w}\right)}{\partial
w_{2}}|_{\mathbf{z,w}=1}=R_{1}^{(1)}\left(1\right)\hat{P}_{2}^{(1)}\left(1\right)=r_{1}\hat{\mu}_{2},
\end{array}
\end{eqnarray*}

An\'alogamente se tiene

\begin{eqnarray}
R_{2}\left(\mathbf{z,w}\right)=R_{2}\left(P_{1}\left(z_{1}\right)\tilde{P}_{2}\left(z_{2}\right)\hat{P}_{1}\left(w_{1}\right)\hat{P}_{2}\left(w_{2}\right)\right)
\end{eqnarray}


\begin{eqnarray*}
\begin{array}{cc}
\frac{\partial R_{2}\left(\mathbf{z,w}\right)}{\partial
z_{1}}|_{\mathbf{z,w}=1}=R_{2}^{(1)}\left(1\right)P_{1}^{(1)}\left(1\right)=r_{2}\mu_{1},&
\frac{\partial R_{2}\left(\mathbf{z,w}\right)}{\partial
z_{2}}|_{\mathbf{z,w}=1}=R_{2}^{(1)}\left(1\right)\tilde{P}_{2}^{(1)}\left(1\right)=r_{2}\tilde{\mu}_{2},\\
\frac{\partial R_{2}\left(\mathbf{z,w}\right)}{\partial
w_{1}}|_{\mathbf{z,w}=1}=R_{2}^{(1)}\left(1\right)\hat{P}_{1}^{(1)}\left(1\right)=r_{2}\hat{\mu}_{1},&
\frac{\partial R_{2}\left(\mathbf{z,w}\right)}{\partial
w_{2}}|_{\mathbf{z,w}=1}=R_{2}^{(1)}\left(1\right)\hat{P}_{2}^{(1)}\left(1\right)=r_{2}\hat{\mu}_{2},\\
\end{array}
\end{eqnarray*}

Para el segundo sistema:

\begin{eqnarray}
\hat{R}_{1}\left(\mathbf{z,w}\right)=\hat{R}_{1}\left(P_{1}\left(z_{1}\right)\tilde{P}_{2}\left(z_{2}\right)\hat{P}_{1}\left(w_{1}\right)\hat{P}_{2}\left(w_{2}\right)\right)
\end{eqnarray}


\begin{eqnarray*}
\frac{\partial \hat{R}_{1}\left(\mathbf{z,w}\right)}{\partial
z_{1}}|_{\mathbf{z,w}=1}=\hat{R}_{1}^{(1)}\left(1\right)P_{1}^{(1)}\left(1\right)=\hat{r}_{1}\mu_{1},&
\frac{\partial \hat{R}_{1}\left(\mathbf{z,w}\right)}{\partial
z_{2}}|_{\mathbf{z,w}=1}=\hat{R}_{1}^{(1)}\left(1\right)\tilde{P}_{2}^{(1)}\left(1\right)=\hat{r}_{1}\tilde{\mu}_{2},\\
\frac{\partial \hat{R}_{1}\left(\mathbf{z,w}\right)}{\partial
w_{1}}|_{\mathbf{z,w}=1}=\hat{R}_{1}^{(1)}\left(1\right)\hat{P}_{1}^{(1)}\left(1\right)=\hat{r}_{1}\hat{\mu}_{1},&
\frac{\partial \hat{R}_{1}\left(\mathbf{z,w}\right)}{\partial
w_{2}}|_{\mathbf{z,w}=1}=\hat{R}_{1}^{(1)}\left(1\right)\hat{P}_{2}^{(1)}\left(1\right)=\hat{r}_{1}\hat{\mu}_{2},
\end{eqnarray*}

Finalmente

\begin{eqnarray}
\hat{R}_{2}\left(\mathbf{z,w}\right)=\hat{R}_{2}\left(P_{1}\left(z_{1}\right)\tilde{P}_{2}\left(z_{2}\right)\hat{P}_{1}\left(w_{1}\right)\hat{P}_{2}\left(w_{2}\right)\right)
\end{eqnarray}



\begin{eqnarray*}
\frac{\partial \hat{R}_{2}\left(\mathbf{z,w}\right)}{\partial
z_{1}}|_{\mathbf{z,w}=1}=\hat{R}_{2}^{(1)}\left(1\right)P_{1}^{(1)}\left(1\right)=\hat{r}_{2}\mu_{1},&
\frac{\partial \hat{R}_{2}\left(\mathbf{z,w}\right)}{\partial
z_{2}}|_{\mathbf{z,w}=1}=\hat{R}_{2}^{(1)}\left(1\right)\tilde{P}_{2}^{(1)}\left(1\right)=\hat{r}_{2}\tilde{\mu}_{2},\\
\frac{\partial \hat{R}_{2}\left(\mathbf{z,w}\right)}{\partial
w_{1}}|_{\mathbf{z,w}=1}=\hat{R}_{2}^{(1)}\left(1\right)\hat{P}_{1}^{(1)}\left(1\right)=\hat{r}_{2}\hat{\mu}_{1},&
\frac{\partial \hat{R}_{2}\left(\mathbf{z,w}\right)}{\partial
w_{2}}|_{\mathbf{z,w}=1}=\hat{R}_{2}^{(1)}\left(1\right)\hat{P}_{2}^{(1)}\left(1\right)
=\hat{r}_{2}\hat{\mu}_{2}.
\end{eqnarray*}


%_________________________________________________________________________________________________
\subsection{Usuarios presentes en la cola}
%_________________________________________________________________________________________________

Hagamos lo correspondiente con las siguientes
expresiones obtenidas en la secci\'on anterior:
Recordemos que

\begin{eqnarray*}
F_{1}\left(\theta_{1}\left(\tilde{P}_{2}\left(z_{2}\right)\hat{P}_{1}\left(w_{1}\right)
\hat{P}_{2}\left(w_{2}\right)\right),z_{2},w_{1},w_{2}\right)=
F_{1}\left(\theta_{1}\left(\tilde{P}_{2}\left(z_{2}\right)\hat{P}_{1}\left(w_{1}
\right)\hat{P}_{2}\left(w_{2}\right)\right),z_{2}\right)
\hat{F}_{1}\left(w_{1},w_{2};\tau_{1}\right)
\end{eqnarray*}

entonces

\begin{eqnarray*}
\frac{\partial F_{1}\left(\theta_{1}\left(\tilde{P}_{2}\left(z_{2}\right)\hat{P}_{1}\left(w_{1}\right)\hat{P}_{2}\left(w_{2}\right)\right),z_{2},w_{1},w_{2}\right)}{\partial z_{1}}|_{\mathbf{z},\mathbf{w}=1}&=&0\\
\frac{\partial
F_{1}\left(\theta_{1}\left(\tilde{P}_{2}\left(z_{2}\right)\hat{P}_{1}\left(w_{1}\right)\hat{P}_{2}\left(w_{2}\right)\right),z_{2},w_{1},w_{2}\right)}{\partial
z_{2}}|_{\mathbf{z},\mathbf{w}=1}&=&\frac{\partial F_{1}}{\partial
z_{1}}\cdot\frac{\partial \theta_{1}}{\partial
\tilde{P}_{2}}\cdot\frac{\partial \tilde{P}_{2}}{\partial
z_{2}}+\frac{\partial F_{1}}{\partial z_{2}}
\\
\frac{\partial
F_{1}\left(\theta_{1}\left(\tilde{P}_{2}\left(z_{2}\right)\hat{P}_{1}\left(w_{1}\right)\hat{P}_{2}\left(w_{2}\right)\right),z_{2},w_{1},w_{2}\right)}{\partial
w_{1}}|_{\mathbf{z},\mathbf{w}=1}&=&\frac{\partial F_{1}}{\partial
z_{1}}\cdot\frac{\partial
\theta_{1}}{\partial\hat{P}_{1}}\cdot\frac{\partial\hat{P}_{1}}{\partial
w_{1}}+\frac{\partial\hat{F}_{1}}{\partial w_{1}}
\\
\frac{\partial
F_{1}\left(\theta_{1}\left(\tilde{P}_{2}\left(z_{2}\right)\hat{P}_{1}\left(w_{1}\right)\hat{P}_{2}\left(w_{2}\right)\right),z_{2},w_{1},w_{2}\right)}{\partial
w_{2}}|_{\mathbf{z},\mathbf{w}=1}&=&\frac{\partial F_{1}}{\partial
z_{1}}\cdot\frac{\partial\theta_{1}}{\partial\hat{P}_{2}}\cdot\frac{\partial\hat{P}_{2}}{\partial
w_{2}}+\frac{\partial \hat{F}_{1}}{\partial w_{2}}
\\
\end{eqnarray*}

para $\tau_{2}$:

\begin{eqnarray*}
F_{2}\left(z_{1},\tilde{\theta}_{2}\left(P_{1}\left(z_{1}\right)\hat{P}_{1}\left(w_{1}\right)\hat{P}_{2}\left(w_{2}\right)\right),
w_{1},w_{2}\right)=F_{2}\left(z_{1},\tilde{\theta}_{2}\left(P_{1}\left(z_{1}\right)\hat{P}_{1}\left(w_{1}\right)
\hat{P}_{2}\left(w_{2}\right)\right)\right)\hat{F}_{2}\left(w_{1},w_{2};\tau_{2}\right)
\end{eqnarray*}
al igual que antes

\begin{eqnarray*}
\frac{\partial
F_{2}\left(z_{1},\tilde{\theta}_{2}\left(P_{1}\left(z_{1}\right)\hat{P}_{1}\left(w_{1}\right)\hat{P}_{2}\left(w_{2}\right)\right),w_{1},w_{2}\right)}{\partial
z_{1}}|_{\mathbf{z},\mathbf{w}=1}&=&\frac{\partial F_{2}}{\partial
z_{2}}\cdot\frac{\partial\tilde{\theta}_{2}}{\partial
P_{1}}\cdot\frac{\partial P_{1}}{\partial z_{2}}+\frac{\partial
F_{2}}{\partial z_{1}}
\\
\frac{\partial F_{2}\left(z_{1},\tilde{\theta}_{2}\left(P_{1}\left(z_{1}\right)\hat{P}_{1}\left(w_{1}\right)\hat{P}_{2}\left(w_{2}\right)\right),w_{1},w_{2}\right)}{\partial z_{2}}|_{\mathbf{z},\mathbf{w}=1}&=&0\\
\frac{\partial
F_{2}\left(z_{1},\tilde{\theta}_{2}\left(P_{1}\left(z_{1}\right)\hat{P}_{1}\left(w_{1}\right)\hat{P}_{2}\left(w_{2}\right)\right),w_{1},w_{2}\right)}{\partial
w_{1}}|_{\mathbf{z},\mathbf{w}=1}&=&\frac{\partial F_{2}}{\partial
z_{2}}\cdot\frac{\partial \tilde{\theta}_{2}}{\partial
\hat{P}_{1}}\cdot\frac{\partial \hat{P}_{1}}{\partial
w_{1}}+\frac{\partial \hat{F}_{2}}{\partial w_{1}}
\\
\frac{\partial
F_{2}\left(z_{1},\tilde{\theta}_{2}\left(P_{1}\left(z_{1}\right)\hat{P}_{1}\left(w_{1}\right)\hat{P}_{2}\left(w_{2}\right)\right),w_{1},w_{2}\right)}{\partial
w_{2}}|_{\mathbf{z},\mathbf{w}=1}&=&\frac{\partial F_{2}}{\partial
z_{2}}\cdot\frac{\partial
\tilde{\theta}_{2}}{\partial\hat{P}_{2}}\cdot\frac{\partial\hat{P}_{2}}{\partial
w_{2}}+\frac{\partial\hat{F}_{2}}{\partial w_{2}}
\\
\end{eqnarray*}


Ahora para el segundo sistema

\begin{eqnarray*}\hat{F}_{1}\left(z_{1},z_{2},\hat{\theta}_{1}\left(P_{1}\left(z_{1}\right)\tilde{P}_{2}\left(z_{2}\right)\hat{P}_{2}\left(w_{2}\right)\right),
w_{2}\right)=F_{1}\left(z_{1},z_{2};\zeta_{1}\right)\hat{F}_{1}\left(\hat{\theta}_{1}\left(P_{1}\left(z_{1}\right)\tilde{P}_{2}\left(z_{2}\right)
\hat{P}_{2}\left(w_{2}\right)\right),w_{2}\right)
\end{eqnarray*}
entonces


\begin{eqnarray*}
\frac{\partial
\hat{F}_{1}\left(z_{1},z_{2},\hat{\theta}_{1}\left(P_{1}\left(z_{1}\right)\tilde{P}_{2}\left(z_{2}\right)\hat{P}_{2}\left(w_{2}\right)\right),w_{2}\right)}{\partial
z_{1}}|_{\mathbf{z},\mathbf{w}=1}&=&\frac{\partial \hat{F}_{1}
}{\partial w_{1}}\cdot\frac{\partial\hat{\theta}_{1}}{\partial
P_{1}}\cdot\frac{\partial P_{1}}{\partial z_{1}}+\frac{\partial
F_{1}}{\partial z_{1}}
\\
\frac{\partial
\hat{F}_{1}\left(z_{1},z_{2},\hat{\theta}_{1}\left(P_{1}\left(z_{1}\right)\tilde{P}_{2}\left(z_{2}\right)\hat{P}_{2}\left(w_{2}\right)\right),w_{2}\right)}{\partial
z_{2}}|_{\mathbf{z},\mathbf{w}=1}&=&\frac{\partial
\hat{F}_{1}}{\partial
w_{1}}\cdot\frac{\partial\hat{\theta}_{1}}{\partial\tilde{P}_{2}}\cdot\frac{\partial\tilde{P}_{2}}{\partial
z_{2}}+\frac{\partial F_{1}}{\partial z_{2}}
\\
\frac{\partial \hat{F}_{1}\left(z_{1},z_{2},\hat{\theta}_{1}\left(P_{1}\left(z_{1}\right)\tilde{P}_{2}\left(z_{2}\right)\hat{P}_{2}\left(w_{2}\right)\right),w_{2}\right)}{\partial w_{1}}|_{\mathbf{z},\mathbf{w}=1}&=&0\\
\frac{\partial \hat{F}_{1}\left(z_{1},z_{2},\hat{\theta}_{1}\left(P_{1}\left(z_{1}\right)\tilde{P}_{2}\left(z_{2}\right)\hat{P}_{2}\left(w_{2}\right)\right),w_{2}\right)}{\partial w_{2}}|_{\mathbf{z},\mathbf{w}=1}&=&\frac{\partial\hat{F}_{1}}{\partial w_{1}}\cdot\frac{\partial\hat{\theta}_{1}}{\partial\hat{P}_{2}}\cdot\frac{\partial\hat{P}_{2}}{\partial w_{2}}+\frac{\partial \hat{F}_{1}}{\partial w_{2}}\\
\end{eqnarray*}



Finalmente para $\zeta_{2}$

\begin{eqnarray*}\hat{F}_{2}\left(z_{1},z_{2},w_{1},\hat{\theta}_{2}\left(P_{1}\left(z_{1}\right)\tilde{P}_{2}\left(z_{2}\right)\hat{P}_{1}\left(w_{1}\right)\right)\right)&=&F_{2}\left(z_{1},z_{2};\zeta_{2}\right)\hat{F}_{2}\left(w_{1},\hat{\theta}_{2}\left(P_{1}\left(z_{1}\right)\tilde{P}_{2}\left(z_{2}\right)\hat{P}_{1}\left(w_{1}\right)\right)\right]
\end{eqnarray*}
por tanto:

\begin{eqnarray*}
\frac{\partial
\hat{F}_{2}\left(z_{1},z_{2},w_{1},\hat{\theta}_{2}\left(P_{1}\left(z_{1}\right)\tilde{P}_{2}\left(z_{2}\right)\hat{P}_{1}\left(w_{1}\right)\right)\right)}{\partial
z_{1}}|_{\mathbf{z},\mathbf{w}=1}&=&\frac{\partial\hat{F}_{2}}{\partial
w_{2}}\cdot\frac{\partial\hat{\theta}_{2}}{\partial
P_{1}}\cdot\frac{\partial P_{1}}{\partial z_{1}}+\frac{\partial
F_{2}}{\partial z_{1}}
\\
\frac{\partial \hat{F}_{2}\left(z_{1},z_{2},w_{1},\hat{\theta}_{2}\left(P_{1}\left(z_{1}\right)\tilde{P}_{2}\left(z_{2}\right)\hat{P}_{1}\left(w_{1}\right)\right)\right)}{\partial z_{2}}|_{\mathbf{z},\mathbf{w}=1}&=&\frac{\partial\hat{F}_{2}}{\partial w_{2}}\cdot\frac{\partial\hat{\theta}_{2}}{\partial \tilde{P}_{2}}\cdot\frac{\partial \tilde{P}_{2}}{\partial z_{2}}+\frac{\partial F_{2}}{\partial z_{2}}\\
\frac{\partial \hat{F}_{2}\left(z_{1},z_{2},w_{1},\hat{\theta}_{2}\left(P_{1}\left(z_{1}\right)\tilde{P}_{2}\left(z_{2}\right)\hat{P}_{1}\left(w_{1}\right)\right)\right)}{\partial w_{1}}|_{\mathbf{z},\mathbf{w}=1}&=&\frac{\partial\hat{F}_{2}}{\partial w_{2}}\cdot\frac{\partial\hat{\theta}_{2}}{\partial \hat{P}_{1}}\cdot\frac{\partial \hat{P}_{1}}{\partial w_{1}}+\frac{\partial \hat{F}_{2}}{\partial w_{1}}\\
\frac{\partial \hat{F}_{2}\left(z_{1},z_{2},w_{1},\hat{\theta}_{2}\left(P_{1}\left(z_{1}\right)\tilde{P}_{2}\left(z_{2}\right)\hat{P}_{1}\left(w_{1}\right)\right)\right)}{\partial w_{2}}|_{\mathbf{z},\mathbf{w}=1}&=&0\\
\end{eqnarray*}

%_________________________________________________________________________________________________
\subsection{Ecuaciones Recursivas}
%_________________________________________________________________________________________________

Entonces, de todo lo desarrollado hasta ahora se tienen las siguientes ecuaciones:

\begin{eqnarray*}
\frac{\partial F_{2}\left(\mathbf{z},\mathbf{w}\right)}{\partial z_{1}}|_{\mathbf{z},\mathbf{w}=1}&=&r_{1}\mu_{1}\\
\frac{\partial F_{2}\left(\mathbf{z},\mathbf{w}\right)}{\partial z_{2}}|_{\mathbf{z},\mathbf{w}=1}&=&=r_{1}\tilde{\mu}_{2}+f_{1}\left(1\right)\left(\frac{1}{1-\mu_{1}}\right)\tilde{\mu}_{2}+f_{1}\left(2\right)\\
\frac{\partial F_{2}\left(\mathbf{z},\mathbf{w}\right)}{\partial w_{1}}|_{\mathbf{z},\mathbf{w}=1}&=&r_{1}\hat{\mu}_{1}+f_{1}\left(1\right)\left(\frac{1}{1-\mu_{1}}\right)\hat{\mu}_{1}+\hat{F}_{1,1}^{(1)}\left(1\right)\\
\frac{\partial F_{2}\left(\mathbf{z},\mathbf{w}\right)}{\partial
w_{2}}|_{\mathbf{z},\mathbf{w}=1}&=&r_{1}\hat{\mu}_{2}+f_{1}\left(1\right)\left(\frac{1}{1-\mu_{1}}\right)\hat{\mu}_{2}+\hat{F}_{2,1}^{(1)}\left(1\right)\\
\frac{\partial F_{1}\left(\mathbf{z},\mathbf{w}\right)}{\partial z_{1}}|_{\mathbf{z},\mathbf{w}=1}&=&r_{2}\mu_{1}+f_{2}\left(2\right)\left(\frac{1}{1-\tilde{\mu}_{2}}\right)\mu_{1}+f_{2}\left(1\right)\\
\frac{\partial F_{1}\left(\mathbf{z},\mathbf{w}\right)}{\partial z_{2}}|_{\mathbf{z},\mathbf{w}=1}&=&r_{2}\tilde{\mu}_{2}\\
\frac{\partial F_{1}\left(\mathbf{z},\mathbf{w}\right)}{\partial w_{1}}|_{\mathbf{z},\mathbf{w}=1}&=&r_{2}\hat{\mu}_{1}+f_{2}\left(2\right)\left(\frac{1}{1-\tilde{\mu}_{2}}\right)\hat{\mu}_{1}+\hat{F}_{2,1}^{(1)}\left(1\right)\\
\frac{\partial F_{1}\left(\mathbf{z},\mathbf{w}\right)}{\partial
w_{2}}|_{\mathbf{z},\mathbf{w}=1}&=&r_{2}\hat{\mu}_{2}+f_{2}\left(2\right)\left(\frac{1}{1-\tilde{\mu}_{2}}\right)\hat{\mu}_{2}+\hat{F}_{2,2}^{(1)}\left(1\right)\\
\frac{\partial \hat{F}_{2}\left(\mathbf{z},\mathbf{w}\right)}{\partial z_{1}}|_{\mathbf{z},\mathbf{w}=1}&=&\hat{r}_{1}\mu_{1}+\hat{f}_{1}\left(1\right)\left(\frac{1}{1-\hat{\mu}_{1}}\right)\mu_{1}+F_{1,1}^{(1)}\left(1\right)\\
\frac{\partial \hat{F}_{2}\left(\mathbf{z},\mathbf{w}\right)}{\partial z_{2}}|_{\mathbf{z},\mathbf{w}=1}&=&\hat{r}_{1}\mu_{2}+\hat{f}_{1}\left(1\right)\left(\frac{1}{1-\hat{\mu}_{1}}\right)\tilde{\mu}_{2}+F_{2,1}^{(1)}\left(1\right)\\
\frac{\partial \hat{F}_{2}\left(\mathbf{z},\mathbf{w}\right)}{\partial w_{1}}|_{\mathbf{z},\mathbf{w}=1}&=&\hat{r}_{1}\hat{\mu}_{1}\\
\frac{\partial \hat{F}_{2}\left(\mathbf{z},\mathbf{w}\right)}{\partial w_{2}}|_{\mathbf{z},\mathbf{w}=1}&=&\hat{r}_{1}\hat{\mu}_{2}+\hat{f}_{1}\left(1\right)\left(\frac{1}{1-\hat{\mu}_{1}}\right)\hat{\mu}_{2}+\hat{f}_{1}\left(2\right)\\
\frac{\partial \hat{F}_{1}\left(\mathbf{z},\mathbf{w}\right)}{\partial z_{1}}|_{\mathbf{z},\mathbf{w}=1}&=&\hat{r}_{2}\mu_{1}+\hat{f}_{2}\left(1\right)\left(\frac{1}{1-\hat{\mu}_{2}}\right)\mu_{1}+F_{1,2}^{(1)}\left(1\right)\\
\frac{\partial \hat{F}_{1}\left(\mathbf{z},\mathbf{w}\right)}{\partial z_{2}}|_{\mathbf{z},\mathbf{w}=1}&=&\hat{r}_{2}\tilde{\mu}_{2}+\hat{f}_{2}\left(2\right)\left(\frac{1}{1-\hat{\mu}_{2}}\right)\tilde{\mu}_{2}+F_{2,2}^{(1)}\left(1\right)\\
\frac{\partial \hat{F}_{1}\left(\mathbf{z},\mathbf{w}\right)}{\partial w_{1}}|_{\mathbf{z},\mathbf{w}=1}&=&\hat{r}_{2}\hat{\mu}_{1}+\hat{f}_{2}\left(2\right)\left(\frac{1}{1-\hat{\mu}_{2}}\right)\hat{\mu}_{1}+\hat{f}_{2}\left(1\right)\\
\frac{\partial
\hat{F}_{1}\left(\mathbf{z},\mathbf{w}\right)}{\partial
w_{2}}|_{\mathbf{z},\mathbf{w}=1}&=&\hat{r}_{2}\hat{\mu}_{2}
\end{eqnarray*}

Es decir, se tienen las siguientes ecuaciones:




\begin{eqnarray*}
f_{2}\left(1\right)&=&r_{1}\mu_{1}\\
f_{1}\left(2\right)&=&r_{2}\tilde{\mu}_{2}\\
f_{2}\left(2\right)&=&r_{1}\tilde{\mu}_{2}+\tilde{\mu}_{2}\left(\frac{f_{1}\left(1\right)}{1-\mu_{1}}\right)+f_{1}\left(2\right)=\left(r_{1}+\frac{f_{1}\left(1\right)}{1-\mu_{1}}\right)\tilde{\mu}_{2}+r_{2}\tilde{\mu}_{2}\\
&=&\left(r_{1}+r_{2}+\frac{f_{1}\left(1\right)}{1-\mu_{1}}\right)\tilde{\mu}_{2}=\left(r+\frac{f_{1}\left(1\right)}{1-\mu_{1}}\right)\tilde{\mu}_{2}\\
f_{2}\left(3\right)&=&r_{1}\hat{\mu}_{1}+\hat{\mu}_{1}\left(\frac{f_{1}\left(1\right)}{1-\mu_{1}}\right)+\hat{F}_{1,1}^{(1)}\left(1\right)=\hat{\mu}_{1}\left(r_{1}+\frac{f_{1}\left(1\right)}{1-\mu_{1}}\right)+\frac{\hat{\mu}_{1}}{\mu_{1}}\\
f_{2}\left(4\right)&=&r_{1}\hat{\mu}_{2}+\hat{\mu}_{2}\left(\frac{f_{1}\left(1\right)}{1-\mu_{1}}\right)+\hat{F}_{2,1}^{(1)}\left(1\right)=\hat{\mu}_{2}\left(r_{1}+\frac{f_{1}\left(1\right)}{1-\mu_{1}}\right)+\frac{\hat{\mu}_{2}}{\mu_{1}}\\
f_{1}\left(1\right)&=&r_{2}\mu_{1}+\mu_{1}\left(\frac{f_{2}\left(2\right)}{1-\tilde{\mu}_{2}}\right)+r_{1}\mu_{1}=\mu_{1}\left(r_{1}+r_{2}+\frac{f_{2}\left(2\right)}{1-\tilde{\mu}_{2}}\right)\\
&=&\mu_{1}\left(r+\frac{f_{2}\left(2\right)}{1-\tilde{\mu}_{2}}\right)\\
f_{1}\left(3\right)&=&r_{2}\hat{\mu}_{1}+\hat{\mu}_{1}\left(\frac{f_{2}\left(2\right)}{1-\tilde{\mu}_{2}}\right)+\hat{F}^{(1)}_{1,2}\left(1\right)=\hat{\mu}_{1}\left(r_{2}+\frac{f_{2}\left(2\right)}{1-\tilde{\mu}_{2}}\right)+\frac{\hat{\mu}_{1}}{\mu_{2}}\\
f_{1}\left(4\right)&=&r_{2}\hat{\mu}_{2}+\hat{\mu}_{2}\left(\frac{f_{2}\left(2\right)}{1-\tilde{\mu}_{2}}\right)+\hat{F}_{2,2}^{(1)}\left(1\right)=\hat{\mu}_{2}\left(r_{2}+\frac{f_{2}\left(2\right)}{1-\tilde{\mu}_{2}}\right)+\frac{\hat{\mu}_{2}}{\mu_{2}}\\
\hat{f}_{1}\left(4\right)&=&\hat{r}_{2}\hat{\mu}_{2}\\
\hat{f}_{2}\left(3\right)&=&\hat{r}_{1}\hat{\mu}_{1}\\
\hat{f}_{1}\left(1\right)&=&\hat{r}_{2}\mu_{1}+\mu_{1}\left(\frac{\hat{f}_{2}\left(4\right)}{1-\hat{\mu}_{2}}\right)+F_{1,2}^{(1)}\left(1\right)=\left(\hat{r}_{2}+\frac{\hat{f}_{2}\left(4\right)}{1-\hat{\mu}_{2}}\right)\mu_{1}+\frac{\mu_{1}}{\hat{\mu}_{2}}\\
\hat{f}_{1}\left(2\right)&=&\hat{r}_{2}\tilde{\mu}_{2}+\tilde{\mu}_{2}\left(\frac{\hat{f}_{2}\left(4\right)}{1-\hat{\mu}_{2}}\right)+F_{2,2}^{(1)}\left(1\right)=
\left(\hat{r}_{2}+\frac{\hat{f}_{2}\left(4\right)}{1-\hat{\mu}_{2}}\right)\tilde{\mu}_{2}+\frac{\mu_{2}}{\hat{\mu}_{2}}\\
\hat{f}_{1}\left(3\right)&=&\hat{r}_{2}\hat{\mu}_{1}+\hat{\mu}_{1}\left(\frac{\hat{f}_{2}\left(4\right)}{1-\hat{\mu}_{2}}\right)+\hat{f}_{2}\left(3\right)=\left(\hat{r}_{2}+\frac{\hat{f}_{2}\left(4\right)}{1-\hat{\mu}_{2}}\right)\hat{\mu}_{1}+\hat{r}_{1}\hat{\mu}_{1}\\
&=&\left(\hat{r}_{1}+\hat{r}_{2}+\frac{\hat{f}_{2}\left(4\right)}{1-\hat{\mu}_{2}}\right)\hat{\mu}_{1}=\left(\hat{r}+\frac{\hat{f}_{2}\left(4\right)}{1-\hat{\mu}_{2}}\right)\hat{\mu}_{1}\\
\hat{f}_{2}\left(1\right)&=&\hat{r}_{1}\mu_{1}+\mu_{1}\left(\frac{\hat{f}_{1}\left(3\right)}{1-\hat{\mu}_{1}}\right)+F_{1,1}^{(1)}\left(1\right)=\left(\hat{r}_{1}+\frac{\hat{f}_{1}\left(3\right)}{1-\hat{\mu}_{1}}\right)\mu_{1}+\frac{\mu_{1}}{\hat{\mu}_{1}}\\
\hat{f}_{2}\left(2\right)&=&\hat{r}_{1}\tilde{\mu}_{2}+\tilde{\mu}_{2}\left(\frac{\hat{f}_{1}\left(3\right)}{1-\hat{\mu}_{1}}\right)+F_{2,1}^{(1)}\left(1\right)=\left(\hat{r}_{1}+\frac{\hat{f}_{1}\left(3\right)}{1-\hat{\mu}_{1}}\right)\tilde{\mu}_{2}+\frac{\mu_{2}}{\hat{\mu}_{1}}\\
\hat{f}_{2}\left(4\right)&=&\hat{r}_{1}\hat{\mu}_{2}+\hat{\mu}_{2}\left(\frac{\hat{f}_{1}\left(3\right)}{1-\hat{\mu}_{1}}\right)+\hat{f}_{1}\left(4\right)=\hat{r}_{1}\hat{\mu}_{2}+\hat{r}_{2}\hat{\mu}_{2}+\hat{\mu}_{2}\left(\frac{\hat{f}_{1}\left(3\right)}{1-\hat{\mu}_{1}}\right)\\
&=&\left(\hat{r}+\frac{\hat{f}_{1}\left(3\right)}{1-\hat{\mu}_{1}}\right)\hat{\mu}_{2}\\
\end{eqnarray*}

es decir,


\begin{eqnarray*}
\begin{array}{lll}
f_{1}\left(1\right)=\mu_{1}\left(r+\frac{f_{2}\left(2\right)}{1-\tilde{\mu}_{2}}\right)&f_{1}\left(2\right)=r_{2}\tilde{\mu}_{2}&f_{1}\left(3\right)=\hat{\mu}_{1}\left(r_{2}+\frac{f_{2}\left(2\right)}{1-\tilde{\mu}_{2}}\right)+\frac{\hat{\mu}_{1}}{\mu_{2}}\\
f_{1}\left(4\right)=\hat{\mu}_{2}\left(r_{2}+\frac{f_{2}\left(2\right)}{1-\tilde{\mu}_{2}}\right)+\frac{\hat{\mu}_{2}}{\mu_{2}}&f_{2}\left(1\right)=r_{1}\mu_{1}&f_{2}\left(2\right)=\left(r+\frac{f_{1}\left(1\right)}{1-\mu_{1}}\right)\tilde{\mu}_{2}\\
f_{2}\left(3\right)=\hat{\mu}_{1}\left(r_{1}+\frac{f_{1}\left(1\right)}{1-\mu_{1}}\right)+\frac{\hat{\mu}_{1}}{\mu_{1}}&
f_{2}\left(4\right)=\hat{\mu}_{2}\left(r_{1}+\frac{f_{1}\left(1\right)}{1-\mu_{1}}\right)+\frac{\hat{\mu}_{2}}{\mu_{1}}&\hat{f}_{1}\left(1\right)=\left(\hat{r}_{2}+\frac{\hat{f}_{2}\left(4\right)}{1-\hat{\mu}_{2}}\right)\mu_{1}+\frac{\mu_{1}}{\hat{\mu}_{2}}\\
\hat{f}_{1}\left(2\right)=\left(\hat{r}_{2}+\frac{\hat{f}_{2}\left(4\right)}{1-\hat{\mu}_{2}}\right)\tilde{\mu}_{2}+\frac{\mu_{2}}{\hat{\mu}_{2}}&\hat{f}_{1}\left(3\right)=\left(\hat{r}+\frac{\hat{f}_{2}\left(4\right)}{1-\hat{\mu}_{2}}\right)\hat{\mu}_{1}&\hat{f}_{1}\left(4\right)=\hat{r}_{2}\hat{\mu}_{2}\\
\hat{f}_{2}\left(1\right)=\left(\hat{r}_{1}+\frac{\hat{f}_{1}\left(3\right)}{1-\hat{\mu}_{1}}\right)\mu_{1}+\frac{\mu_{1}}{\hat{\mu}_{1}}&\hat{f}_{2}\left(2\right)=\left(\hat{r}_{1}+\frac{\hat{f}_{1}\left(3\right)}{1-\hat{\mu}_{1}}\right)\tilde{\mu}_{2}+\frac{\mu_{2}}{\hat{\mu}_{1}}&\hat{f}_{2}\left(3\right)=\hat{r}_{1}\hat{\mu}_{1}\\
&\hat{f}_{2}\left(4\right)=\left(\hat{r}+\frac{\hat{f}_{1}\left(3\right)}{1-\hat{\mu}_{1}}\right)\hat{\mu}_{2}&
\end{array}
\end{eqnarray*}

%_______________________________________________________________________________________________
\subsection{Soluci\'on del Sistema de Ecuaciones Lineales}
%_________________________________________________________________________________________________

A saber, se puede demostrar que la soluci\'on del sistema de
ecuaciones est\'a dado por las siguientes expresiones, donde

\begin{eqnarray*}
\mu=\mu_{1}+\tilde{\mu}_{2}\textrm{ , }\hat{\mu}=\hat{\mu}_{1}+\hat{\mu}_{2}\textrm{ , }
r=r_{1}+r_{2}\textrm{ y }\hat{r}=\hat{r}_{1}+\hat{r}_{2}
\end{eqnarray*}
entonces

\begin{eqnarray*}
\begin{array}{lll}
f_{1}\left(1\right)=r\frac{\mu_{1}\left(1-\mu_{1}\right)}{1-\mu}&
f_{1}\left(3\right)=\hat{\mu}_{1}\left(\frac{r_{2}\mu_{2}+1}{\mu_{2}}+r\frac{\tilde{\mu}_{2}}{1-\mu}\right)&
f_{1}\left(4\right)=\hat{\mu}_{2}\left(\frac{r_{2}\mu_{2}+1}{\mu_{2}}+r\frac{\tilde{\mu}_{2}}{1-\mu}\right)\\
f_{2}\left(2\right)=r\frac{\tilde{\mu}_{2}\left(1-\tilde{\mu}_{2}\right)}{1-\mu}&
f_{2}\left(3\right)=\hat{\mu}_{1}\left(\frac{r_{1}\mu_{1}+1}{\mu_{1}}+r\frac{\mu_{1}}{1-\mu}\right)&
f_{2}\left(4\right)=\hat{\mu}_{2}\left(\frac{r_{1}\mu_{1}+1}{\mu_{1}}+r\frac{\mu_{1}}{1-\mu}\right)\\
\hat{f}_{1}\left(1\right)=\mu_{1}\left(\frac{\hat{r}_{2}\hat{\mu}_{2}+1}{\hat{\mu}_{2}}+\hat{r}\frac{\hat{\mu}_{2}}{1-\hat{\mu}}\right)&
\hat{f}_{1}\left(2\right)=\tilde{\mu}_{2}\left(\hat{r}_{2}+\hat{r}\frac{\hat{\mu}_{2}}{1-\hat{\mu}}\right)+\frac{\mu_{2}}{\hat{\mu}_{2}}&
\hat{f}_{1}\left(3\right)=\hat{r}\frac{\hat{\mu}_{1}\left(1-\hat{\mu}_{1}\right)}{1-\hat{\mu}}\\
\hat{f}_{2}\left(1\right)=\mu_{1}\left(\frac{\hat{r}_{1}\hat{\mu}_{1}+1}{\hat{\mu}_{1}}+\hat{r}\frac{\hat{\mu}_{1}}{1-\hat{\mu}}\right)&
\hat{f}_{2}\left(2\right)=\tilde{\mu}_{2}\left(\hat{r}_{1}+\hat{r}\frac{\hat{\mu}_{1}}{1-\hat{\mu}}\right)+\frac{\hat{\mu_{2}}}{\hat{\mu}_{1}}&
\hat{f}_{2}\left(4\right)=\hat{r}\frac{\hat{\mu}_{2}\left(1-\hat{\mu}_{2}\right)}{1-\hat{\mu}}\\
\end{array}
\end{eqnarray*}




A saber

\begin{eqnarray*}
f_{1}\left(3\right)&=&\hat{\mu}_{1}\left(r_{2}+\frac{f_{2}\left(2\right)}{1-\tilde{\mu}_{2}}\right)+\frac{\hat{\mu}_{1}}{\mu_{2}}=\hat{\mu}_{1}\left(r_{2}+\frac{r\frac{\tilde{\mu}_{2}\left(1-\tilde{\mu}_{2}\right)}{1-\mu}}{1-\tilde{\mu}_{2}}\right)+\frac{\hat{\mu}_{1}}{\mu_{2}}=\hat{\mu}_{1}\left(r_{2}+\frac{r\tilde{\mu}_{2}}{1-\mu}\right)+\frac{\hat{\mu}_{1}}{\mu_{2}}\\
&=&\hat{\mu}_{1}\left(r_{2}+\frac{r\tilde{\mu}_{2}}{1-\mu}+\frac{1}{\mu_{2}}\right)=\hat{\mu}_{1}\left(\frac{r_{2}\mu_{2}+1}{\mu_{2}}+\frac{r\tilde{\mu}_{2}}{1-\mu}\right)
\end{eqnarray*}

\begin{eqnarray*}
f_{1}\left(4\right)&=&\hat{\mu}_{2}\left(r_{2}+\frac{f_{2}\left(2\right)}{1-\tilde{\mu}_{2}}\right)+\frac{\hat{\mu}_{2}}{\mu_{2}}=\hat{\mu}_{2}\left(r_{2}+\frac{r\frac{\tilde{\mu}_{2}\left(1-\tilde{\mu}_{2}\right)}{1-\mu}}{1-\tilde{\mu}_{2}}\right)+\frac{\hat{\mu}_{2}}{\mu_{2}}=\hat{\mu}_{2}\left(r_{2}+\frac{r\tilde{\mu}_{2}}{1-\mu}\right)+\frac{\hat{\mu}_{1}}{\mu_{2}}\\
&=&\hat{\mu}_{2}\left(r_{2}+\frac{r\tilde{\mu}_{2}}{1-\mu}+\frac{1}{\mu_{2}}\right)=\hat{\mu}_{2}\left(\frac{r_{2}\mu_{2}+1}{\mu_{2}}+\frac{r\tilde{\mu}_{2}}{1-\mu}\right)
\end{eqnarray*}

\begin{eqnarray*}
f_{2}\left(3\right)&=&\hat{\mu}_{1}\left(r_{1}+\frac{f_{1}\left(1\right)}{1-\mu_{1}}\right)+\frac{\hat{\mu}_{1}}{\mu_{1}}=\hat{\mu}_{1}\left(r_{1}+\frac{r\frac{\mu_{1}\left(1-\mu_{1}\right)}{1-\mu}}{1-\mu_{1}}\right)+\frac{\hat{\mu}_{1}}{\mu_{1}}=\hat{\mu}_{1}\left(r_{1}+\frac{r\mu_{1}}{1-\mu}\right)+\frac{\hat{\mu}_{1}}{\mu_{1}}\\
&=&\hat{\mu}_{1}\left(r_{1}+\frac{r\mu_{1}}{1-\mu}+\frac{1}{\mu_{1}}\right)=\hat{\mu}_{1}\left(\frac{r_{1}\mu_{1}+1}{\mu_{1}}+\frac{r\mu_{1}}{1-\mu}\right)
\end{eqnarray*}

\begin{eqnarray*}
f_{2}\left(4\right)&=&\hat{\mu}_{2}\left(r_{1}+\frac{f_{1}\left(1\right)}{1-\mu_{1}}\right)+\frac{\hat{\mu}_{2}}{\mu_{1}}=\hat{\mu}_{2}\left(r_{1}+\frac{r\frac{\mu_{1}\left(1-\mu_{1}\right)}{1-\mu}}{1-\mu_{1}}\right)+\frac{\hat{\mu}_{1}}{\mu_{1}}=\hat{\mu}_{2}\left(r_{1}+\frac{r\mu_{1}}{1-\mu}\right)+\frac{\hat{\mu}_{1}}{\mu_{1}}\\
&=&\hat{\mu}_{2}\left(r_{1}+\frac{r\mu_{1}}{1-\mu}+\frac{1}{\mu_{1}}\right)=\hat{\mu}_{2}\left(\frac{r_{1}\mu_{1}+1}{\mu_{1}}+\frac{r\mu_{1}}{1-\mu}\right)\end{eqnarray*}


\begin{eqnarray*}
\hat{f}_{1}\left(1\right)&=&\mu_{1}\left(\hat{r}_{2}+\frac{\hat{f}_{2}\left(4\right)}{1-\tilde{\mu}_{2}}\right)+\frac{\mu_{1}}{\hat{\mu}_{2}}=\mu_{1}\left(\hat{r}_{2}+\frac{\hat{r}\frac{\hat{\mu}_{2}\left(1-\hat{\mu}_{2}\right)}{1-\hat{\mu}}}{1-\hat{\mu}_{2}}\right)+\frac{\mu_{1}}{\hat{\mu}_{2}}=\mu_{1}\left(\hat{r}_{2}+\frac{\hat{r}\hat{\mu}_{2}}{1-\hat{\mu}}\right)+\frac{\mu_{1}}{\mu_{2}}\\
&=&\mu_{1}\left(\hat{r}_{2}+\frac{\hat{r}\mu_{2}}{1-\hat{\mu}}+\frac{1}{\hat{\mu}_{2}}\right)=\mu_{1}\left(\frac{\hat{r}_{2}\hat{\mu}_{2}+1}{\hat{\mu}_{2}}+\frac{\hat{r}\hat{\mu}_{2}}{1-\hat{\mu}}\right)
\end{eqnarray*}

\begin{eqnarray*}
\hat{f}_{1}\left(2\right)&=&\tilde{\mu}_{2}\left(\hat{r}_{2}+\frac{\hat{f}_{2}\left(4\right)}{1-\tilde{\mu}_{2}}\right)+\frac{\mu_{2}}{\hat{\mu}_{2}}=\tilde{\mu}_{2}\left(\hat{r}_{2}+\frac{\hat{r}\frac{\hat{\mu}_{2}\left(1-\hat{\mu}_{2}\right)}{1-\hat{\mu}}}{1-\hat{\mu}_{2}}\right)+\frac{\mu_{2}}{\hat{\mu}_{2}}=\tilde{\mu}_{2}\left(\hat{r}_{2}+\frac{\hat{r}\hat{\mu}_{2}}{1-\hat{\mu}}\right)+\frac{\mu_{2}}{\hat{\mu}_{2}}
\end{eqnarray*}

\begin{eqnarray*}
\hat{f}_{2}\left(1\right)&=&\mu_{1}\left(\hat{r}_{1}+\frac{\hat{f}_{1}\left(3\right)}{1-\hat{\mu}_{1}}\right)+\frac{\mu_{1}}{\hat{\mu}_{1}}=\mu_{1}\left(\hat{r}_{1}+\frac{\hat{r}\frac{\hat{\mu}_{1}\left(1-\hat{\mu}_{1}\right)}{1-\hat{\mu}}}{1-\hat{\mu}_{1}}\right)+\frac{\mu_{1}}{\hat{\mu}_{1}}=\mu_{1}\left(\hat{r}_{1}+\frac{\hat{r}\hat{\mu}_{1}}{1-\hat{\mu}}\right)+\frac{\mu_{1}}{\hat{\mu}_{1}}\\
&=&\mu_{1}\left(\hat{r}_{1}+\frac{\hat{r}\hat{\mu}_{1}}{1-\hat{\mu}}+\frac{1}{\hat{\mu}_{1}}\right)=\mu_{1}\left(\frac{\hat{r}_{1}\hat{\mu}_{1}+1}{\hat{\mu}_{1}}+\frac{\hat{r}\hat{\mu}_{1}}{1-\hat{\mu}}\right)
\end{eqnarray*}

\begin{eqnarray*}
\hat{f}_{2}\left(2\right)&=&\tilde{\mu}_{2}\left(\hat{r}_{1}+\frac{\hat{f}_{1}\left(3\right)}{1-\tilde{\mu}_{1}}\right)+\frac{\mu_{2}}{\hat{\mu}_{1}}=\tilde{\mu}_{2}\left(\hat{r}_{1}+\frac{\hat{r}\frac{\hat{\mu}_{1}
\left(1-\hat{\mu}_{1}\right)}{1-\hat{\mu}}}{1-\hat{\mu}_{1}}\right)+\frac{\mu_{2}}{\hat{\mu}_{1}}=\tilde{\mu}_{2}\left(\hat{r}_{1}+\frac{\hat{r}\hat{\mu}_{1}}{1-\hat{\mu}}\right)+\frac{\mu_{2}}{\hat{\mu}_{1}}
\end{eqnarray*}

%----------------------------------------------------------------------------------------
\section{Resultado Principal}
%----------------------------------------------------------------------------------------
Sean $\mu_{1},\mu_{2},\check{\mu}_{2},\hat{\mu}_{1},\hat{\mu}_{2}$ y $\tilde{\mu}_{2}=\mu_{2}+\check{\mu}_{2}$ los valores esperados de los proceso definidos anteriormente, y sean $r_{1},r_{2}, \hat{r}_{1}$ y $\hat{r}_{2}$ los valores esperado s de los tiempos de traslado del servidor entre las colas para cada uno de los sistemas de visitas c\'iclicas. Si se definen $\mu=\mu_{1}+\tilde{\mu}_{2}$, $\hat{\mu}=\hat{\mu}_{1}+\hat{\mu}_{2}$, y $r=r_{1}+r_{2}$ y  $\hat{r}=\hat{r}_{1}+\hat{r}_{2}$, entonces se tiene el siguiente resultado.

\begin{Teo}
Supongamos que $\mu<1$, $\hat{\mu}<1$, entonces, el n\'umero de usuarios presentes en cada una de las colas que conforman la Red de Sistemas de Visitas C\'iclicas cuando uno de los servidores visita a alguna de ellas est\'a dada por la soluci\'on del Sistema de Ecuaciones Lineales presentados arriba cuyas expresiones damos a continuaci\'on:
%{\footnotesize{
\begin{eqnarray*}
\begin{array}{lll}
f_{1}\left(1\right)=r\frac{\mu_{1}\left(1-\mu_{1}\right)}{1-\mu},&f_{1}\left(2\right)=r_{2}\tilde{\mu}_{2},&f_{1}\left(3\right)=\hat{\mu}_{1}\left(\frac{r_{2}\mu_{2}+1}{\mu_{2}}+r\frac{\tilde{\mu}_{2}}{1-\mu}\right),\\
f_{1}\left(4\right)=\hat{\mu}_{2}\left(\frac{r_{2}\mu_{2}+1}{\mu_{2}}+r\frac{\tilde{\mu}_{2}}{1-\mu}\right),&f_{2}\left(1\right)=r_{1}\mu_{1},&f_{2}\left(2\right)=r\frac{\tilde{\mu}_{2}\left(1-\tilde{\mu}_{2}\right)}{1-\mu},\\
f_{2}\left(3\right)=\hat{\mu}_{1}\left(\frac{r_{1}\mu_{1}+1}{\mu_{1}}+r\frac{\mu_{1}}{1-\mu}\right),&f_{2}\left(4\right)=\hat{\mu}_{2}\left(\frac{r_{1}\mu_{1}+1}{\mu_{1}}+r\frac{\mu_{1}}{1-\mu}\right),&\hat{f}_{1}\left(1\right)=\mu_{1}\left(\frac{\hat{r}_{2}\hat{\mu}_{2}+1}{\hat{\mu}_{2}}+\hat{r}\frac{\hat{\mu}_{2}}{1-\hat{\mu}}\right),\\
\hat{f}_{1}\left(2\right)=\tilde{\mu}_{2}\left(\hat{r}_{2}+\hat{r}\frac{\hat{\mu}_{2}}{1-\hat{\mu}}\right)+\frac{\mu_{2}}{\hat{\mu}_{2}},&\hat{f}_{1}\left(3\right)=\hat{r}\frac{\hat{\mu}_{1}\left(1-\hat{\mu}_{1}\right)}{1-\hat{\mu}},&\hat{f}_{1}\left(4\right)=\hat{r}_{2}\hat{\mu}_{2},\\
\hat{f}_{2}\left(1\right)=\mu_{1}\left(\frac{\hat{r}_{1}\hat{\mu}_{1}+1}{\hat{\mu}_{1}}+\hat{r}\frac{\hat{\mu}_{1}}{1-\hat{\mu}}\right),&\hat{f}_{2}\left(2\right)=\tilde{\mu}_{2}\left(\hat{r}_{1}+\hat{r}\frac{\hat{\mu}_{1}}{1-\hat{\mu}}\right)+\frac{\hat{\mu_{2}}}{\hat{\mu}_{1}},&\hat{f}_{2}\left(3\right)=\hat{r}_{1}\hat{\mu}_{1},\\
&\hat{f}_{2}\left(4\right)=\hat{r}\frac{\hat{\mu}_{2}\left(1-\hat{\mu}_{2}\right)}{1-\hat{\mu}}.&\\
\end{array}
\end{eqnarray*} %}}
\end{Teo}





%___________________________________________________________________________________________
%
\section{Segundos Momentos}
%___________________________________________________________________________________________
%
%___________________________________________________________________________________________
%
%\subsection{Derivadas de Segundo Orden: Tiempos de Traslado del Servidor}
%___________________________________________________________________________________________



Para poder determinar los segundos momentos para los tiempos de traslado del servidor es necesaria la siguiente proposici\'on:

\begin{Prop}\label{Prop.Segundas.Derivadas}
Sea $f\left(g\left(x\right)h\left(y\right)\right)$ funci\'on continua tal que tiene derivadas parciales mixtas de segundo orden, entonces se tiene lo siguiente:

\begin{eqnarray*}
\frac{\partial}{\partial x}f\left(g\left(x\right)h\left(y\right)\right)=\frac{\partial f\left(g\left(x\right)h\left(y\right)\right)}{\partial x}\cdot \frac{\partial g\left(x\right)}{\partial x}\cdot h\left(y\right)
\end{eqnarray*}

por tanto

\begin{eqnarray}
\frac{\partial}{\partial x}\frac{\partial}{\partial x}f\left(g\left(x\right)h\left(y\right)\right)
&=&\frac{\partial^{2}}{\partial x}f\left(g\left(x\right)h\left(y\right)\right)\cdot \left(\frac{\partial g\left(x\right)}{\partial x}\right)^{2}\cdot h^{2}\left(y\right)+\frac{\partial}{\partial x}f\left(g\left(x\right)h\left(y\right)\right)\cdot \frac{\partial g^{2}\left(x\right)}{\partial x^{2}}\cdot h\left(y\right).
\end{eqnarray}

y

\begin{eqnarray*}
\frac{\partial}{\partial y}\frac{\partial}{\partial x}f\left(g\left(x\right)h\left(y\right)\right)&=&\frac{\partial g\left(x\right)}{\partial x}\cdot \frac{\partial h\left(y\right)}{\partial y}\left\{\frac{\partial^{2}}{\partial y\partial x}f\left(g\left(x\right)h\left(y\right)\right)\cdot g\left(x\right)\cdot h\left(y\right)+\frac{\partial}{\partial x}f\left(g\left(x\right)h\left(y\right)\right)\right\}
\end{eqnarray*}
\end{Prop}
\begin{proof}
\footnotesize{
\begin{eqnarray*}
\frac{\partial}{\partial x}\frac{\partial}{\partial x}f\left(g\left(x\right)h\left(y\right)\right)&=&\frac{\partial}{\partial x}\left\{\frac{\partial f\left(g\left(x\right)h\left(y\right)\right)}{\partial x}\cdot \frac{\partial g\left(x\right)}{\partial x}\cdot h\left(y\right)\right\}\\
&=&\frac{\partial}{\partial x}\left\{\frac{\partial}{\partial x}f\left(g\left(x\right)h\left(y\right)\right)\right\}\cdot \frac{\partial g\left(x\right)}{\partial x}\cdot h\left(y\right)+\frac{\partial}{\partial x}f\left(g\left(x\right)h\left(y\right)\right)\cdot \frac{\partial g^{2}\left(x\right)}{\partial x^{2}}\cdot h\left(y\right)\\
&=&\frac{\partial^{2}}{\partial x}f\left(g\left(x\right)h\left(y\right)\right)\cdot \frac{\partial g\left(x\right)}{\partial x}\cdot h\left(y\right)\cdot \frac{\partial g\left(x\right)}{\partial x}\cdot h\left(y\right)+\frac{\partial}{\partial x}f\left(g\left(x\right)h\left(y\right)\right)\cdot \frac{\partial g^{2}\left(x\right)}{\partial x^{2}}\cdot h\left(y\right)\\
&=&\frac{\partial^{2}}{\partial x}f\left(g\left(x\right)h\left(y\right)\right)\cdot \left(\frac{\partial g\left(x\right)}{\partial x}\right)^{2}\cdot h^{2}\left(y\right)+\frac{\partial}{\partial x}f\left(g\left(x\right)h\left(y\right)\right)\cdot \frac{\partial g^{2}\left(x\right)}{\partial x^{2}}\cdot h\left(y\right).
\end{eqnarray*}}


Por otra parte:
\footnotesize{
\begin{eqnarray*}
\frac{\partial}{\partial y}\frac{\partial}{\partial x}f\left(g\left(x\right)h\left(y\right)\right)&=&\frac{\partial}{\partial y}\left\{\frac{\partial f\left(g\left(x\right)h\left(y\right)\right)}{\partial x}\cdot \frac{\partial g\left(x\right)}{\partial x}\cdot h\left(y\right)\right\}\\
&=&\frac{\partial}{\partial y}\left\{\frac{\partial}{\partial x}f\left(g\left(x\right)h\left(y\right)\right)\right\}\cdot \frac{\partial g\left(x\right)}{\partial x}\cdot h\left(y\right)+\frac{\partial}{\partial x}f\left(g\left(x\right)h\left(y\right)\right)\cdot \frac{\partial g\left(x\right)}{\partial x}\cdot \frac{\partial h\left(y\right)}{y}\\
&=&\frac{\partial^{2}}{\partial y\partial x}f\left(g\left(x\right)h\left(y\right)\right)\cdot \frac{\partial h\left(y\right)}{\partial y}\cdot g\left(x\right)\cdot \frac{\partial g\left(x\right)}{\partial x}\cdot h\left(y\right)+\frac{\partial}{\partial x}f\left(g\left(x\right)h\left(y\right)\right)\cdot \frac{\partial g\left(x\right)}{\partial x}\cdot \frac{\partial h\left(y\right)}{\partial y}\\
&=&\frac{\partial g\left(x\right)}{\partial x}\cdot \frac{\partial h\left(y\right)}{\partial y}\left\{\frac{\partial^{2}}{\partial y\partial x}f\left(g\left(x\right)h\left(y\right)\right)\cdot g\left(x\right)\cdot h\left(y\right)+\frac{\partial}{\partial x}f\left(g\left(x\right)h\left(y\right)\right)\right\}
\end{eqnarray*}}
\end{proof}

Utilizando la proposici\'on anterior (Proposici\'ion \ref{Prop.Segundas.Derivadas})se tiene el siguiente resultado que me dice como calcular los segundos momentos para los procesos de traslado del servidor:

\begin{Prop}
Sea $R_{i}$ la Funci\'on Generadora de Probabilidades para el n\'umero de arribos a cada una de las colas de la Red de Sistemas de Visitas C\'iclicas definidas como en (\ref{Ec.R1}). Entonces las derivadas parciales est\'an dadas por las siguientes expresiones:


\begin{eqnarray*}
\frac{\partial^{2} R_{i}\left(P_{1}\left(z_{1}\right)\tilde{P}_{2}\left(z_{2}\right)\hat{P}_{1}\left(w_{1}\right)\hat{P}_{2}\left(w_{2}\right)\right)}{\partial z_{i}^{2}}&=&\left(\frac{\partial P_{i}\left(z_{i}\right)}{\partial z_{i}}\right)^{2}\cdot\frac{\partial^{2} R_{i}\left(P_{1}\left(z_{1}\right)\tilde{P}_{2}\left(z_{2}\right)\hat{P}_{1}\left(w_{1}\right)\hat{P}_{2}\left(w_{2}\right)\right)}{\partial^{2} z_{i}}\\
&+&\left(\frac{\partial P_{i}\left(z_{i}\right)}{\partial z_{i}}\right)^{2}\cdot
\frac{\partial R_{i}\left(P_{1}\left(z_{1}\right)\tilde{P}_{2}\left(z_{2}\right)\hat{P}_{1}\left(w_{1}\right)\hat{P}_{2}\left(w_{2}\right)\right)}{\partial z_{i}}
\end{eqnarray*}



y adem\'as


\begin{eqnarray*}
\frac{\partial^{2} R_{i}\left(P_{1}\left(z_{1}\right)\tilde{P}_{2}\left(z_{2}\right)\hat{P}_{1}\left(w_{1}\right)\hat{P}_{2}\left(w_{2}\right)\right)}{\partial z_{2}\partial z_{1}}&=&\frac{\partial \tilde{P}_{2}\left(z_{2}\right)}{\partial z_{2}}\cdot\frac{\partial P_{1}\left(z_{1}\right)}{\partial z_{1}}\cdot\frac{\partial^{2} R_{i}\left(P_{1}\left(z_{1}\right)\tilde{P}_{2}\left(z_{2}\right)\hat{P}_{1}\left(w_{1}\right)\hat{P}_{2}\left(w_{2}\right)\right)}{\partial z_{2}\partial z_{1}}\\
&+&\frac{\partial \tilde{P}_{2}\left(z_{2}\right)}{\partial z_{2}}\cdot\frac{\partial P_{1}\left(z_{1}\right)}{\partial z_{1}}\cdot\frac{\partial R_{i}\left(P_{1}\left(z_{1}\right)\tilde{P}_{2}\left(z_{2}\right)\hat{P}_{1}\left(w_{1}\right)\hat{P}_{2}\left(w_{2}\right)\right)}{\partial z_{1}},
\end{eqnarray*}



\begin{eqnarray*}
\frac{\partial^{2} R_{i}\left(P_{1}\left(z_{1}\right)\tilde{P}_{2}\left(z_{2}\right)\hat{P}_{1}\left(w_{1}\right)\hat{P}_{2}\left(w_{2}\right)\right)}{\partial w_{i}\partial z_{1}}&=&\frac{\partial \hat{P}_{i}\left(w_{i}\right)}{\partial z_{2}}\cdot\frac{\partial P_{1}\left(z_{1}\right)}{\partial z_{1}}\cdot\frac{\partial^{2} R_{i}\left(P_{1}\left(z_{1}\right)\tilde{P}_{2}\left(z_{2}\right)\hat{P}_{1}\left(w_{1}\right)\hat{P}_{2}\left(w_{2}\right)\right)}{\partial w_{i}\partial z_{1}}\\
&+&\frac{\partial \hat{P}_{i}\left(w_{i}\right)}{\partial z_{2}}\cdot\frac{\partial P_{1}\left(z_{1}\right)}{\partial z_{1}}\cdot\frac{\partial R_{i}\left(P_{1}\left(z_{1}\right)\tilde{P}_{2}\left(z_{2}\right)\hat{P}_{1}\left(w_{1}\right)\hat{P}_{2}\left(w_{2}\right)\right)}{\partial z_{1}},
\end{eqnarray*}
finalmente

\begin{eqnarray*}
\frac{\partial^{2} R_{i}\left(P_{1}\left(z_{1}\right)\tilde{P}_{2}\left(z_{2}\right)\hat{P}_{1}\left(w_{1}\right)\hat{P}_{2}\left(w_{2}\right)\right)}{\partial w_{i}\partial z_{2}}&=&\frac{\partial \hat{P}_{i}\left(w_{i}\right)}{\partial w_{i}}\cdot\frac{\partial \tilde{P}_{2}\left(z_{2}\right)}{\partial z_{2}}\cdot\frac{\partial^{2} R_{i}\left(P_{1}\left(z_{1}\right)\tilde{P}_{2}\left(z_{2}\right)\hat{P}_{1}\left(w_{1}\right)\hat{P}_{2}\left(w_{2}\right)\right)}{\partial w_{i}\partial z_{2}}\\
&+&\frac{\partial \hat{P}_{i}\left(w_{i}\right)}{\partial w_{i}}\cdot\frac{\partial \tilde{P}_{2}\left(z_{2}\right)}{\partial z_{1}}\cdot\frac{\partial R_{i}\left(P_{1}\left(z_{1}\right)\tilde{P}_{2}\left(z_{2}\right)\hat{P}_{1}\left(w_{1}\right)\hat{P}_{2}\left(w_{2}\right)\right)}{\partial z_{2}},
\end{eqnarray*}

para $i=1,2$.
\end{Prop}

%___________________________________________________________________________________________
%
\subsection{Sistema de Ecuaciones Lineales para los Segundos Momentos}
%___________________________________________________________________________________________

En el ap\'endice (\ref{Segundos.Momentos}) se demuestra que las ecuaciones para las ecuaciones parciales mixtas est\'an dadas por:



%___________________________________________________________________________________________
%\subsubsection{Mixtas para $z_{1}$:}
%___________________________________________________________________________________________
%1
\begin{eqnarray*}
f_{1}\left(1,1\right)&=&r_{2}P_{1}^{(2)}\left(1\right)+\mu_{1}^{2}R_{2}^{(2)}\left(1\right)+2\mu_{1}r_{2}\left(\frac{\mu_{1}}{1-\tilde{\mu}_{2}}f_{2}\left(2\right)+f_{2}\left(1\right)\right)+\frac{1}{1-\tilde{\mu}_{2}}P_{1}^{(2)}f_{2}\left(2\right)+\mu_{1}^{2}\tilde{\theta}_{2}^{(2)}\left(1\right)f_{2}\left(2\right)\\
&+&\frac{\mu_{1}}{1-\tilde{\mu}_{2}}f_{2}(1,2)+\frac{\mu_{1}}{1-\tilde{\mu}_{2}}\left(\frac{\mu_{1}}{1-\tilde{\mu}_{2}}f_{2}(2,2)+f_{2}(1,2)\right)+f_{2}(1,1),\\
f_{1}\left(2,1\right)&=&\mu_{1}r_{2}\tilde{\mu}_{2}+\mu_{1}\tilde{\mu}_{2}R_{2}^{(2)}\left(1\right)+r_{2}\tilde{\mu}_{2}\left(\frac{\mu_{1}}{1-\tilde{\mu}_{2}}f_{2}(2)+f_{2}(1)\right),\\
f_{1}\left(3,1\right)&=&\mu_{1}\hat{\mu}_{1}r_{2}+\mu_{1}\hat{\mu}_{1}R_{2}^{(2)}\left(1\right)+r_{2}\frac{\mu_{1}}{1-\tilde{\mu}_{2}}f_{2}(2)+r_{2}\hat{\mu}_{1}\left(\frac{\mu_{1}}{1-\tilde{\mu}_{2}}f_{2}(2)+f_{2}(1)\right)+\mu_{1}r_{2}\hat{F}_{2,1}^{(1)}(1)\\
&+&\left(\frac{\mu_{1}}{1-\tilde{\mu}_{2}}f_{2}(2)+f_{2}(1)\right)\hat{F}_{2,1}^{(1)}(1)+\frac{\mu_{1}\hat{\mu}_{1}}{1-\tilde{\mu}_{2}}f_{2}(2)+\mu_{1}\hat{\mu}_{1}\tilde{\theta}_{2}^{(2)}\left(1\right)f_{2}(2)+\mu_{1}\hat{\mu}_{1}\left(\frac{1}{1-\tilde{\mu}_{2}}\right)^{2}f_{2}(2,2)\\
&+&+\frac{\hat{\mu}_{1}}{1-\tilde{\mu}_{2}}f_{2}(1,2),\\
f_{1}\left(4,1\right)&=&\mu_{1}\hat{\mu}_{2}r_{2}+\mu_{1}\hat{\mu}_{2}R_{2}^{(2)}\left(1\right)+r_{2}\frac{\mu_{1}\hat{\mu}_{2}}{1-\tilde{\mu}_{2}}f_{2}(2)+\mu_{1}r_{2}\hat{F}_{2,2}^{(1)}(1)+r_{2}\hat{\mu}_{2}\left(\frac{\mu_{1}}{1-\tilde{\mu}_{2}}f_{2}(2)+f_{2}(1)\right)\\
&+&\hat{F}_{2,1}^{(1)}(1)\left(\frac{\mu_{1}}{1-\tilde{\mu}_{2}}f_{2}(2)+f_{2}(1)\right)+\frac{\mu_{1}\hat{\mu}_{2}}{1-\tilde{\mu}_{2}}f_{2}(2)
+\mu_{1}\hat{\mu}_{2}\tilde{\theta}_{2}^{(2)}\left(1\right)f_{2}(2)+\mu_{1}\hat{\mu}_{2}\left(\frac{1}{1-\tilde{\mu}_{2}}\right)^{2}f_{2}(2,2)\\
&+&\frac{\hat{\mu}_{2}}{1-\tilde{\mu}_{2}}f_{2}^{(1,2)},\\
\end{eqnarray*}
\begin{eqnarray*}
f_{1}\left(1,2\right)&=&\mu_{1}\tilde{\mu}_{2}r_{2}+\mu_{1}\tilde{\mu}_{2}R_{2}^{(2)}\left(1\right)+r_{2}\tilde{\mu}_{2}\left(\frac{\mu_{1}}{1-\tilde{\mu}_{2}}f_{2}(2)+f_{2}(1)\right),\\
f_{1}\left(2,2\right)&=&\tilde{\mu}_{2}^{2}R_{2}^{(2)}(1)+r_{2}\tilde{P}_{2}^{(2)}\left(1\right),\\
f_{1}\left(3,2\right)&=&\hat{\mu}_{1}\tilde{\mu}_{2}r_{2}+\hat{\mu}_{1}\tilde{\mu}_{2}R_{2}^{(2)}(1)+
r_{2}\frac{\hat{\mu}_{1}\tilde{\mu}_{2}}{1-\tilde{\mu}_{2}}f_{2}(2)+r_{2}\tilde{\mu}_{2}\hat{F}_{2,2}^{(1)}(1),\\
f_{1}\left(4,2\right)&=&\hat{\mu}_{2}\tilde{\mu}_{2}r_{2}+\hat{\mu}_{2}\tilde{\mu}_{2}R_{2}^{(2)}(1)+
r_{2}\frac{\hat{\mu}_{2}\tilde{\mu}_{2}}{1-\tilde{\mu}_{2}}f_{2}(2)+r_{2}\tilde{\mu}_{2}\hat{F}_{2,2}^{(1)}(1),\\
f_{1}\left(1,3\right)&=&\mu_{1}\hat{\mu}_{1}r_{2}+\mu_{1}\hat{\mu}_{1}R_{2}^{(2)}\left(1\right)+\frac{\mu_{1}\hat{\mu}_{1}}{1-\tilde{\mu}_{2}}f_{2}(2)+r_{2}\frac{\mu_{1}\hat{\mu}_{1}}{1-\tilde{\mu}_{2}}f_{2}(2)+\mu_{1}\hat{\mu}_{1}\tilde{\theta}_{2}^{(2)}\left(1\right)f_{2}(2)+r_{2}\mu_{1}\hat{F}_{2,1}^{(1)}(1)\\
&+&r_{2}\hat{\mu}_{1}\left(\frac{\mu_{1}}{1-\tilde{\mu}_{2}}f_{2}(2)+f_{2}\left(1\right)\right)+\left(\frac{\mu_{1}}{1-\tilde{\mu}_{2}}f_{2}\left(1\right)+f_{2}\left(1\right)\right)\hat{F}_{2,1}^{(1)}(1)\\
&+&\frac{\hat{\mu}_{1}}{1-\tilde{\mu}_{2}}\left(\frac{\mu_{1}}{1-\tilde{\mu}_{2}}f_{2}(2,2)+f_{2}^{(1,2)}\right),\\
f_{1}\left(2,3\right)&=&\tilde{\mu}_{2}\hat{\mu}_{1}r_{2}+\tilde{\mu}_{2}\hat{\mu}_{1}R_{2}^{(2)}\left(1\right)+r_{2}\frac{\tilde{\mu}_{2}\hat{\mu}_{1}}{1-\tilde{\mu}_{2}}f_{2}(2)+r_{2}\tilde{\mu}_{2}\hat{F}_{2,1}^{(1)}(1),\\
f_{1}\left(3,3\right)&=&\hat{\mu}_{1}^{2}R_{2}^{(2)}\left(1\right)+r_{2}\hat{P}_{1}^{(2)}\left(1\right)+2r_{2}\frac{\hat{\mu}_{1}^{2}}{1-\tilde{\mu}_{2}}f_{2}(2)+\hat{\mu}_{1}^{2}\tilde{\theta}_{2}^{(2)}\left(1\right)f_{2}(2)+\frac{1}{1-\tilde{\mu}_{2}}\hat{P}_{1}^{(2)}\left(1\right)f_{2}(2)\\
&+&\frac{\hat{\mu}_{1}^{2}}{1-\tilde{\mu}_{2}}f_{2}(2,2)+2r_{2}\hat{\mu}_{1}\hat{F}_{2,1}^{(1)}(1)+2\frac{\hat{\mu}_{1}}{1-\tilde{\mu}_{2}}f_{2}(2)\hat{F}_{2,1}^{(1)}(1)+\hat{f}_{2,1}^{(2)}(1),\\
f_{1}\left(4,3\right)&=&r_{2}\hat{\mu}_{2}\hat{\mu}_{1}+\hat{\mu}_{1}\hat{\mu}_{2}R_{2}^{(2)}(1)+\frac{\hat{\mu}_{1}\hat{\mu}_{2}}{1-\tilde{\mu}_{2}}f_{2}\left(2\right)+2r_{2}\frac{\hat{\mu}_{1}\hat{\mu}_{2}}{1-\tilde{\mu}_{2}}f_{2}\left(2\right)+\hat{\mu}_{2}\hat{\mu}_{1}\tilde{\theta}_{2}^{(2)}\left(1\right)f_{2}\left(2\right)+r_{2}\hat{\mu}_{1}\hat{F}_{2,2}^{(1)}(1)\\
&+&\frac{\hat{\mu}_{1}}{1-\tilde{\mu}_{2}}f_{2}\left(2\right)\hat{F}_{2,2}^{(1)}(1)+\hat{\mu}_{1}\hat{\mu}_{2}\left(\frac{1}{1-\tilde{\mu}_{2}}\right)^{2}f_{2}(2,2)+r_{2}\hat{\mu}_{2}\hat{F}_{2,1}^{(1)}(1)+\frac{\hat{\mu}_{2}}{1-\tilde{\mu}_{2}}f_{2}\left(2\right)\hat{F}_{2,1}^{(1)}(1)+\hat{f}_{2}(1,2),\\
f_{1}\left(1,4\right)&=&r_{2}\mu_{1}\hat{\mu}_{2}+\mu_{1}\hat{\mu}_{2}R_{2}^{(2)}(1)+\frac{\mu_{1}\hat{\mu}_{2}}{1-\tilde{\mu}_{2}}f_{2}(2)+r_{2}\frac{\mu_{1}\hat{\mu}_{2}}{1-\tilde{\mu}_{2}}f_{2}(2)+\mu_{1}\hat{\mu}_{2}\tilde{\theta}_{2}^{(2)}\left(1\right)f_{2}(2)+r_{2}\mu_{1}\hat{F}_{2,2}^{(1)}(1)\\
&+&r_{2}\hat{\mu}_{2}\left(\frac{\mu_{1}}{1-\tilde{\mu}_{2}}f_{2}(2)+f_{2}(1)\right)+\hat{F}_{2,2}^{(1)}(1)\left(\frac{\mu_{1}}{1-\tilde{\mu}_{2}}f_{2}(2)+f_{2}(1)\right)\\
&+&\frac{\hat{\mu}_{2}}{1-\tilde{\mu}_{2}}\left(\frac{\mu_{1}}{1-\tilde{\mu}_{2}}f_{2}(2,2)+f_{2}(1,2)\right),\\
f_{1}\left(2,4\right)
&=&r_{2}\tilde{\mu}_{2}\hat{\mu}_{2}+\tilde{\mu}_{2}\hat{\mu}_{2}R_{2}^{(2)}(1)+r_{2}\frac{\tilde{\mu}_{2}\hat{\mu}_{2}}{1-\tilde{\mu}_{2}}f_{2}(2)+r_{2}\tilde{\mu}_{2}\hat{F}_{2,2}^{(1)}(1),\\
f_{1}\left(3,4\right)&=&r_{2}\hat{\mu}_{1}\hat{\mu}_{2}+\hat{\mu}_{1}\hat{\mu}_{2}R_{2}^{(2)}\left(1\right)+\frac{\hat{\mu}_{1}\hat{\mu}_{2}}{1-\tilde{\mu}_{2}}f_{2}(2)+2r_{2}\frac{\hat{\mu}_{1}\hat{\mu}_{2}}{1-\tilde{\mu}_{2}}f_{2}(2)+\hat{\mu}_{1}\hat{\mu}_{2}\theta_{2}^{(2)}\left(1\right)f_{2}(2)+r_{2}\hat{\mu}_{1}\hat{F}_{2,2}^{(1)}(1)\\
&+&\frac{\hat{\mu}_{1}}{1-\tilde{\mu}_{2}}f_{2}(2)\hat{F}_{2,2}^{(1)}(1)+\hat{\mu}_{1}\hat{\mu}_{2}\left(\frac{1}{1-\tilde{\mu}_{2}}\right)^{2}f_{2}(2,2)+r_{2}\hat{\mu}_{2}\hat{F}_{2,2}^{(1)}(1)+\frac{\hat{\mu}_{2}}{1-\tilde{\mu}_{2}}f_{2}(2)\hat{F}_{2,1}^{(1)}(1)+\hat{f}_{2}^{(2)}(1,2),\\
f_{1}\left(4,4\right)&=&\hat{\mu}_{2}^{2}R_{2}^{(2)}(1)+r_{2}\hat{P}_{2}^{(2)}\left(1\right)+2r_{2}\frac{\hat{\mu}_{2}^{2}}{1-\tilde{\mu}_{2}}f_{2}(2)+\hat{\mu}_{2}^{2}\tilde{\theta}_{2}^{(2)}\left(1\right)f_{2}(2)+\frac{1}{1-\tilde{\mu}_{2}}\hat{P}_{2}^{(2)}\left(1\right)f_{2}(2)\\
&+&2r_{2}\hat{\mu}_{2}\hat{F}_{2,2}^{(1)}(1)+2\frac{\hat{\mu}_{2}}{1-\tilde{\mu}_{2}}f_{2}(2)\hat{F}_{2,2}^{(1)}(1)+\left(\frac{\hat{\mu}_{2}}{1-\tilde{\mu}_{2}}\right)^{2}f_{2}(2,2)+\hat{f}_{2,2}^{(2)}(1),\\
f_{2}\left(1,1\right)&=&r_{1}P_{1}^{(2)}\left(1\right)+\mu_{1}^{2}R_{1}^{(2)}\left(1\right),\\
f_{2}\left(2,1\right)&=&\mu_{1}\tilde{\mu}_{2}r_{1}+\mu_{1}\tilde{\mu}_{2}R_{1}^{(2)}(1)+
r_{1}\mu_{1}\left(\frac{\tilde{\mu}_{2}}{1-\mu_{1}}f_{1}(1)+f_{1}(2)\right),\\
f_{2}\left(3,1\right)&=&r_{1}\mu_{1}\hat{\mu}_{1}+\mu_{1}\hat{\mu}_{1}R_{1}^{(2)}\left(1\right)+r_{1}\frac{\mu_{1}\hat{\mu}_{1}}{1-\mu_{1}}f_{1}(1)+r_{1}\mu_{1}\hat{F}_{1,1}^{(1)}(1),\\
f_{2}\left(4,1\right)&=&\mu_{1}\hat{\mu}_{2}r_{1}+\mu_{1}\hat{\mu}_{2}R_{1}^{(2)}\left(1\right)+r_{1}\mu_{1}\hat{F}_{1,2}^{(1)}(1)+r_{1}\frac{\mu_{1}\hat{\mu}_{2}}{1-\mu_{1}}f_{1}(1),\\
\end{eqnarray*}
\begin{eqnarray*}
f_{2}\left(1,2\right)&=&r_{1}\mu_{1}\tilde{\mu}_{2}+\mu_{1}\tilde{\mu}_{2}R_{1}^{(2)}\left(1\right)+r_{1}\mu_{1}\left(\frac{\tilde{\mu}_{2}}{1-\mu_{1}}f_{1}(1)+f_{1}(2)\right),\\
f_{2}\left(2,2\right)&=&\tilde{\mu}_{2}^{2}R_{1}^{(2)}\left(1\right)+r_{1}\tilde{P}_{2}^{(2)}\left(1\right)+2r_{1}\tilde{\mu}_{2}\left(\frac{\tilde{\mu}_{2}}{1-\mu_{1}}f_{1}(1)+f_{1}(2)\right)+f_{1}(2,2)+\tilde{\mu}_{2}^{2}\theta_{1}^{(2)}\left(1\right)f_{1}(1)\\
&+&\frac{1}{1-\mu_{1}}\tilde{P}_{2}^{(2)}\left(1\right)f_{1}(1)+\frac{\tilde{\mu}_{2}}{1-\mu_{1}}f_{1}(1,2)+\frac{\tilde{\mu}_{2}}{1-\mu_{1}}\left(\frac{\tilde{\mu}_{2}}{1-\mu_{1}}f_{1}(1,1)+f_{1}(1,2)\right),\\
f_{2}\left(3,2\right)&=&\tilde{\mu}_{2}\hat{\mu}_{1}r_{1}+\tilde{\mu}_{2}\hat{\mu}_{1}R_{1}^{(2)}\left(1\right)+r_{1}\frac{\tilde{\mu}_{2}\hat{\mu}_{1}}{1-\mu_{1}}f_{1}(1)+\hat{\mu}_{1}r_{1}\left(\frac{\tilde{\mu}_{2}}{1-\mu_{1}}f_{1}(1)+f_{1}(2)\right)+r_{1}\tilde{\mu}_{2}\hat{F}_{1,1}^{(1)}(1)\\
&+&\left(\frac{\tilde{\mu}_{2}}{1-\mu_{1}}f_{1}(1)+f_{1}(2)\right)\hat{F}_{1,1}^{(1)}(1)+\frac{\tilde{\mu}_{2}\hat{\mu}_{1}}{1-\mu_{1}}f_{1}(1)+\tilde{\mu}_{2}\hat{\mu}_{1}\theta_{1}^{(2)}\left(1\right)f_{1}(1)+\frac{\hat{\mu}_{1}}{1-\mu_{1}}f_{1}(1,2)\\
&+&\left(\frac{1}{1-\mu_{1}}\right)^{2}\tilde{\mu}_{2}\hat{\mu}_{1}f_{1}(1,1),\\
f_{2}\left(4,2\right)&=&\hat{\mu}_{2}\tilde{\mu}_{2}r_{1}+\hat{\mu}_{2}\tilde{\mu}_{2}R_{1}^{(2)}(1)+r_{1}\tilde{\mu}_{2}\hat{F}_{1,2}^{(1)}(1)+r_{1}\frac{\hat{\mu}_{2}\tilde{\mu}_{2}}{1-\mu_{1}}f_{1}(1)+\hat{\mu}_{2}r_{1}\left(\frac{\tilde{\mu}_{2}}{1-\mu_{1}}f_{1}(1)+f_{1}(2)\right)\\
&+&\left(\frac{\tilde{\mu}_{2}}{1-\mu_{1}}f_{1}(1)+f_{1}(2)\right)\hat{F}_{1,2}^{(1)}(1)+\frac{\tilde{\mu}_{2}\hat{\mu_{2}}}{1-\mu_{1}}f_{1}(1)+\hat{\mu}_{2}\tilde{\mu}_{2}\theta_{1}^{(2)}\left(1\right)f_{1}(1)+\frac{\hat{\mu}_{2}}{1-\mu_{1}}f_{1}(1,2)\\
&+&\tilde{\mu}_{2}\hat{\mu}_{2}\left(\frac{1}{1-\mu_{1}}\right)^{2}f_{1}(1,1),\\
f_{2}\left(1,3\right)&=&r_{1}\mu_{1}\hat{\mu}_{1}+\mu_{1}\hat{\mu}_{1}R_{1}^{(2)}(1)+r_{1}\frac{\mu_{1}\hat{\mu}_{1}}{1-\mu_{1}}f_{1}(1)+r_{1}\mu_{1}\hat{F}_{1,1}^{(1)}(1),\\
 f_{2}\left(2,3\right)&=&r_{1}\hat{\mu}_{1}\tilde{\mu}_{2}+\tilde{\mu}_{2}\hat{\mu}_{1}R_{1}^{(2)}\left(1\right)+\frac{\hat{\mu}_{1}\tilde{\mu}_{2}}{1-\mu_{1}}f_{1}(1)+r_{1}\frac{\hat{\mu}_{1}\tilde{\mu}_{2}}{1-\mu_{1}}f_{1}(1)+\hat{\mu}_{1}\tilde{\mu}_{2}\theta_{1}^{(2)}\left(1\right)f_{1}(1)+r_{1}\tilde{\mu}_{2}\hat{F}_{1,1}(1)\\
&+&r_{1}\hat{\mu}_{1}\left(f_{1}(1)+\frac{\tilde{\mu}_{2}}{1-\mu_{1}}f_{1}(1)\right)+
+\left(f_{1}(2)+\frac{\tilde{\mu}_{2}}{1-\mu_{1}}f_{1}(1)\right)\hat{F}_{1,1}(1)\\
&+&\frac{\hat{\mu}_{1}}{1-\mu_{1}}\left(f_{1}(1,2)+\frac{\tilde{\mu}_{2}}{1-\mu_{1}}f_{1}(1,1)\right),\\
f_{2}\left(3,3\right)&=&\hat{\mu}_{1}^{2}R_{1}^{(2)}\left(1\right)+r_{1}\hat{P}_{1}^{(2)}\left(1\right)+2r_{1}\frac{\hat{\mu}_{1}^{2}}{1-\mu_{1}}f_{1}(1)+\hat{\mu}_{1}^{2}\theta_{1}^{(2)}\left(1\right)f_{1}(1)+2r_{1}\hat{\mu}_{1}\hat{F}_{1,1}^{(1)}(1)\\
&+&\frac{1}{1-\mu_{1}}\hat{P}_{1}^{(2)}\left(1\right)f_{1}(1)+2\frac{\hat{\mu}_{1}}{1-\mu_{1}}f_{1}(1)\hat{F}_{1,1}(1)+\left(\frac{\hat{\mu}_{1}}{1-\mu_{1}}\right)^{2}f_{1}(1,1)+\hat{f}_{1,1}^{(2)}(1),\\
f_{2}\left(4,3\right)&=&r_{1}\hat{\mu}_{1}\hat{\mu}_{2}+\hat{\mu}_{1}\hat{\mu}_{2}R_{1}^{(2)}\left(1\right)+r_{1}\hat{\mu}_{1}\hat{F}_{1,2}(1)+
\frac{\hat{\mu}_{1}\hat{\mu}_{2}}{1-\mu_{1}}f_{1}(1)+2r_{1}\frac{\hat{\mu}_{1}\hat{\mu}_{2}}{1-\mu_{1}}f_{1}(1)+r_{1}\hat{\mu}_{2}\hat{F}_{1,1}(1)\\
&+&\hat{\mu}_{1}\hat{\mu}_{2}\theta_{1}^{(2)}\left(1\right)f_{1}(1)+\frac{\hat{\mu}_{1}}{1-\mu_{1}}f_{1}(1)\hat{F}_{1,2}(1)+\frac{\hat{\mu}_{2}}{1-\mu_{1}}\hat{F}_{1,1}(1)f_{1}(1)\\
&+&\hat{f}_{1}^{(2)}(1,2)+\hat{\mu}_{1}\hat{\mu}_{2}\left(\frac{1}{1-\mu_{1}}\right)^{2}f_{1}(2,2),\\
f_{2}\left(1,4\right)&=&r_{1}\mu_{1}\hat{\mu}_{2}+\mu_{1}\hat{\mu}_{2}R_{1}^{(2)}\left(1\right)+r_{1}\mu_{1}\hat{F}_{1,2}(1)+r_{1}\frac{\mu_{1}\hat{\mu}_{2}}{1-\mu_{1}}f_{1}(1),\\
f_{2}\left(2,4\right)&=&r_{1}\hat{\mu}_{2}\tilde{\mu}_{2}+\hat{\mu}_{2}\tilde{\mu}_{2}R_{1}^{(2)}\left(1\right)+r_{1}\tilde{\mu}_{2}\hat{F}_{1,2}(1)+\frac{\hat{\mu}_{2}\tilde{\mu}_{2}}{1-\mu_{1}}f_{1}(1)+r_{1}\frac{\hat{\mu}_{2}\tilde{\mu}_{2}}{1-\mu_{1}}f_{1}(1)+\hat{\mu}_{2}\tilde{\mu}_{2}\theta_{1}^{(2)}\left(1\right)f_{1}(1)\\
&+&r_{1}\hat{\mu}_{2}\left(f_{1}(2)+\frac{\tilde{\mu}_{2}}{1-\mu_{1}}f_{1}(1)\right)+\left(f_{1}(2)+\frac{\tilde{\mu}_{2}}{1-\mu_{1}}f_{1}(1)\right)\hat{F}_{1,2}(1)\\&+&\frac{\hat{\mu}_{2}}{1-\mu_{1}}\left(f_{1}(1,2)+\frac{\tilde{\mu}_{2}}{1-\mu_{1}}f_{1}(1,1)\right),\\
\end{eqnarray*}
\begin{eqnarray*}
f_{2}\left(3,4\right)&=&r_{1}\hat{\mu}_{1}\hat{\mu}_{2}+\hat{\mu}_{1}\hat{\mu}_{2}R_{1}^{(2)}\left(1\right)+r_{1}\hat{\mu}_{1}\hat{F}_{1,2}(1)+
\frac{\hat{\mu}_{1}\hat{\mu}_{2}}{1-\mu_{1}}f_{1}(1)+2r_{1}\frac{\hat{\mu}_{1}\hat{\mu}_{2}}{1-\mu_{1}}f_{1}(1)+\hat{\mu}_{1}\hat{\mu}_{2}\theta_{1}^{(2)}\left(1\right)f_{1}(1)\\
&+&+\frac{\hat{\mu}_{1}}{1-\mu_{1}}\hat{F}_{1,2}(1)f_{1}(1)+r_{1}\hat{\mu}_{2}\hat{F}_{1,1}(1)+\frac{\hat{\mu}_{2}}{1-\mu_{1}}\hat{F}_{1,1}(1)f_{1}(1)+\hat{f}_{1}^{(2)}(1,2)+\hat{\mu}_{1}\hat{\mu}_{2}\left(\frac{1}{1-\mu_{1}}\right)^{2}f_{1}(1,1),\\
f_{2}\left(4,4\right)&=&\hat{\mu}_{2}R_{1}^{(2)}\left(1\right)+r_{1}\hat{P}_{2}^{(2)}\left(1\right)+2r_{1}\hat{\mu}_{2}\hat{F}_{1}^{(0,1)}+\hat{f}_{1,2}^{(2)}(1)+2r_{1}\frac{\hat{\mu}_{2}^{2}}{1-\mu_{1}}f_{1}(1)+\hat{\mu}_{2}^{2}\theta_{1}^{(2)}\left(1\right)f_{1}(1)\\
&+&\frac{1}{1-\mu_{1}}\hat{P}_{2}^{(2)}\left(1\right)f_{1}(1) +
2\frac{\hat{\mu}_{2}}{1-\mu_{1}}f_{1}(1)\hat{F}_{1,2}(1)+\left(\frac{\hat{\mu}_{2}}{1-\mu_{1}}\right)^{2}f_{1}(1,1),\\
\hat{f}_{1}\left(1,1\right)&=&\hat{r}_{2}P_{1}^{(2)}\left(1\right)+
\mu_{1}^{2}\hat{R}_{2}^{(2)}\left(1\right)+
2\hat{r}_{2}\frac{\mu_{1}^{2}}{1-\hat{\mu}_{2}}\hat{f}_{2}(2)+
\frac{1}{1-\hat{\mu}_{2}}P_{1}^{(2)}\left(1\right)\hat{f}_{2}(2)+
\mu_{1}^{2}\hat{\theta}_{2}^{(2)}\left(1\right)\hat{f}_{2}(2)\\
&+&\left(\frac{\mu_{1}^{2}}{1-\hat{\mu}_{2}}\right)^{2}\hat{f}_{2}(2,2)+2\hat{r}_{2}\mu_{1}F_{2,1}(1)+2\frac{\mu_{1}}{1-\hat{\mu}_{2}}\hat{f}_{2}(2)F_{2,1}(1)+F_{2,1}^{(2)}(1),\\
\hat{f}_{1}\left(2,1\right)&=&\hat{r}_{2}\mu_{1}\tilde{\mu}_{2}+\mu_{1}\tilde{\mu}_{2}\hat{R}_{2}^{(2)}\left(1\right)+\hat{r}_{2}\mu_{1}F_{2,2}(1)+
\frac{\mu_{1}\tilde{\mu}_{2}}{1-\hat{\mu}_{2}}\hat{f}_{2}(2)+2\hat{r}_{2}\frac{\mu_{1}\tilde{\mu}_{2}}{1-\hat{\mu}_{2}}\hat{f}_{2}(2)\\
&+&\mu_{1}\tilde{\mu}_{2}\hat{\theta}_{2}^{(2)}\left(1\right)\hat{f}_{2}(2)+\frac{\mu_{1}}{1-\hat{\mu}_{2}}F_{2,2}(1)\hat{f}_{2}(2)+\mu_{1} \tilde{\mu}_{2}\left(\frac{1}{1-\hat{\mu}_{2}}\right)^{2}\hat{f}_{2}(2,2)+\hat{r}_{2}\tilde{\mu}_{2}F_{2,1}(1)\\
&+&\frac{\tilde{\mu}_{2}}{1-\hat{\mu}_{2}}\hat{f}_{2}(2)F_{2,1}(1)+f_{2,1}^{(2)}(1),\\
\hat{f}_{1}\left(3,1\right)&=&\hat{r}_{2}\mu_{1}\hat{\mu}_{1}+\mu_{1}\hat{\mu}_{1}\hat{R}_{2}^{(2)}\left(1\right)+\hat{r}_{2}\frac{\mu_{1}\hat{\mu}_{1}}{1-\hat{\mu}_{2}}\hat{f}_{2}(2)+\hat{r}_{2}\hat{\mu}_{1}F_{2,1}(1)+\hat{r}_{2}\mu_{1}\hat{f}_{2}(1)\\
&+&F_{2,1}(1)\hat{f}_{2}(1)+\frac{\mu_{1}}{1-\hat{\mu}_{2}}\hat{f}_{2}(1,2),\\
\hat{f}_{1}\left(4,1\right)&=&\hat{r}_{2}\mu_{1}\hat{\mu}_{2}+\mu_{1}\hat{\mu}_{2}\hat{R}_{2}^{(2)}\left(1\right)+\frac{\mu_{1}\hat{\mu}_{2}}{1-\hat{\mu}_{2}}\hat{f}_{2}(2)+2\hat{r}_{2}\frac{\mu_{1}\hat{\mu}_{2}}{1-\hat{\mu}_{2}}\hat{f}_{2}(2)+\mu_{1}\hat{\mu}_{2}\hat{\theta}_{2}^{(2)}\left(1\right)\hat{f}_{2}(2)\\
&+&\mu_{1}\hat{\mu}_{2}\left(\frac{1}{1-\hat{\mu}_{2}}\right)^{2}\hat{f}_{2}(2,2)+\hat{r}_{2}\hat{\mu}_{2}F_{2,1}(1)+\frac{\hat{\mu}_{2}}{1-\hat{\mu}_{2}}\hat{f}_{2}(2)F_{2,1}(1),\\
\hat{f}_{1}\left(1,2\right)&=&\hat{r}_{2}\mu_{1}\tilde{\mu}_{2}+\mu_{1}\tilde{\mu}_{2}\hat{R}_{2}^{(2)}\left(1\right)+\mu_{1}\hat{r}_{2}F_{2,2}(1)+
\frac{\mu_{1}\tilde{\mu}_{2}}{1-\hat{\mu}_{2}}\hat{f}_{2}(2)+2\hat{r}_{2}\frac{\mu_{1}\tilde{\mu}_{2}}{1-\hat{\mu}_{2}}\hat{f}_{2}(2)\\
&+&\mu_{1}\tilde{\mu}_{2}\hat{\theta}_{2}^{(2)}\left(1\right)\hat{f}_{2}(2)+\frac{\mu_{1}}{1-\hat{\mu}_{2}}F_{2,2}(1)\hat{f}_{2}(2)+\mu_{1}\tilde{\mu}_{2}\left(\frac{1}{1-\hat{\mu}_{2}}\right)^{2}\hat{f}_{2}(2,2)\\
&+&\hat{r}_{2}\tilde{\mu}_{2}F_{2,1}(1)+\frac{\tilde{\mu}_{2}}{1-\hat{\mu}_{2}}\hat{f}_{2}(2)F_{2,1}(1)+f_{2}^{(2)}(1,2),\\
\hat{f}_{1}\left(2,2\right)&=&\hat{r}_{2}\tilde{P}_{2}^{(2)}\left(1\right)+\tilde{\mu}_{2}^{2}\hat{R}_{2}^{(2)}\left(1\right)+2\hat{r}_{2}\tilde{\mu}_{2}F_{2,2}(1)+2\hat{r}_{2}\frac{\tilde{\mu}_{2}^{2}}{1-\hat{\mu}_{2}}\hat{f}_{2}(2)+f_{2,2}^{(2)}(1)\\
&+&\frac{1}{1-\hat{\mu}_{2}}\tilde{P}_{2}^{(2)}\left(1\right)\hat{f}_{2}(2)+\tilde{\mu}_{2}^{2}\hat{\theta}_{2}^{(2)}\left(1\right)\hat{f}_{2}(2)+2\frac{\tilde{\mu}_{2}}{1-\hat{\mu}_{2}}F_{2,2}(1)\hat{f}_{2}(2)+\left(\frac{\tilde{\mu}_{2}}{1-\hat{\mu}_{2}}\right)^{2}\hat{f}_{2}(2,2),\\
\hat{f}_{1}\left(3,2\right)&=&\hat{r}_{2}\tilde{\mu}_{2}\hat{\mu}_{1}+\tilde{\mu}_{2}\hat{\mu}_{1}\hat{R}_{2}^{(2)}\left(1\right)+\hat{r}_{2}\hat{\mu}_{1}F_{2,2}(1)+\hat{r}_{2}\frac{\tilde{\mu}_{2}\hat{\mu}_{1}}{1-\hat{\mu}_{2}}\hat{f}_{2}(2)+\hat{r}_{2}\tilde{\mu}_{2}\hat{f}_{2}(1)+F_{2,2}(1)\hat{f}_{2}(1)\\
&+&\frac{\tilde{\mu}_{2}}{1-\hat{\mu}_{2}}\hat{f}_{2}(1,2),\\
\hat{f}_{1}\left(4,2\right)&=&\hat{r}_{2}\tilde{\mu}_{2}\hat{\mu}_{2}+\tilde{\mu}_{2}\hat{\mu}_{2}\hat{R}_{2}^{(2)}\left(1\right)+\hat{r}_{2}\hat{\mu}_{2}F_{2,2}(1)+
\frac{\tilde{\mu}_{2}\hat{\mu}_{2}}{1-\hat{\mu}_{2}}\hat{f}_{2}(2)+2\hat{r}_{2}\frac{\tilde{\mu}_{2}\hat{\mu}_{2}}{1-\hat{\mu}_{2}}\hat{f}_{2}(2)\\
&+&\tilde{\mu}_{2}\hat{\mu}_{2}\hat{\theta}_{2}^{(2)}\left(1\right)\hat{f}_{2}(2)+\frac{\hat{\mu}_{2}}{1-\hat{\mu}_{2}}F_{2,2}(1)\hat{f}_{2}(1)+\tilde{\mu}_{2}\hat{\mu}_{2}\left(\frac{1}{1-\hat{\mu}_{2}}\right)\hat{f}_{2}(2,2),\\
\end{eqnarray*}
\begin{eqnarray*}
\hat{f}_{1}\left(1,3\right)&=&\hat{r}_{2}\mu_{1}\hat{\mu}_{1}+\mu_{1}\hat{\mu}_{1}\hat{R}_{2}^{(2)}\left(1\right)+\hat{r}_{2}\frac{\mu_{1}\hat{\mu}_{1}}{1-\hat{\mu}_{2}}\hat{f}_{2}(2)+\hat{r}_{2}\hat{\mu}_{1}F_{2,1}(1)+\hat{r}_{2}\mu_{1}\hat{f}_{2}(1)\\
&+&F_{2,1}(1)\hat{f}_{2}(1)+\frac{\mu_{1}}{1-\hat{\mu}_{2}}\hat{f}_{2}(1,2),\\
\hat{f}_{1}\left(2,3\right)&=&\hat{r}_{2}\tilde{\mu}_{2}\hat{\mu}_{1}+\tilde{\mu}_{2}\hat{\mu}_{1}\hat{R}_{2}^{(2)}\left(1\right)+\hat{r}_{2}\hat{\mu}_{1}F_{2,2}(1)+\hat{r}_{2}\frac{\tilde{\mu}_{2}\hat{\mu}_{1}}{1-\hat{\mu}_{2}}\hat{f}_{2}(2)+\hat{r}_{2}\tilde{\mu}_{2}\hat{f}_{2}(1)\\
&+&F_{2,2}(1)\hat{f}_{2}(1)+\frac{\tilde{\mu}_{2}}{1-\hat{\mu}_{2}}\hat{f}_{2}(1,2),\\
\hat{f}_{1}\left(3,3\right)&=&\hat{r}_{2}\hat{P}_{1}^{(2)}\left(1\right)+\hat{\mu}_{1}^{2}\hat{R}_{2}^{(2)}\left(1\right)+2\hat{r}_{2}\hat{\mu}_{1}\hat{f}_{2}(1)+\hat{f}_{2}(1,1),\\
\hat{f}_{1}\left(4,3\right)&=&\hat{r}_{2}\hat{\mu}_{1}\hat{\mu}_{2}+\hat{\mu}_{1}\hat{\mu}_{2}\hat{R}_{2}^{(2)}\left(1\right)+
\hat{r}_{2}\frac{\hat{\mu}_{2}\hat{\mu}_{1}}{1-\hat{\mu}_{2}}\hat{f}_{2}(2)+\hat{r}_{2}\hat{\mu}_{2}\hat{f}_{2}(1)+\frac{\hat{\mu}_{2}}{1-\hat{\mu}_{2}}\hat{f}_{2}(1,2),\\
\hat{f}_{1}\left(1,4\right)&=&\hat{r}_{2}\mu_{1}\hat{\mu}_{2}+\mu_{1}\hat{\mu}_{2}\hat{R}_{2}^{(2)}\left(1\right)+
\frac{\mu_{1}\hat{\mu}_{2}}{1-\hat{\mu}_{2}}\hat{f}_{2}(2) +2\hat{r}_{2}\frac{\mu_{1}\hat{\mu}_{2}}{1-\hat{\mu}_{2}}\hat{f}_{2}(2)\\
&+&\mu_{1}\hat{\mu}_{2}\hat{\theta}_{2}^{(2)}\left(1\right)\hat{f}_{2}(2)+\mu_{1}\hat{\mu}_{2}\left(\frac{1}{1-\hat{\mu}_{2}}\right)^{2}\hat{f}_{2}(2,2)+\hat{r}_{2}\hat{\mu}_{2}F_{2,1}(1)+\frac{\hat{\mu}_{2}}{1-\hat{\mu}_{2}}\hat{f}_{2}(2)F_{2,1}(1),\\\hat{f}_{1}\left(2,4\right)&=&\hat{r}_{2}\tilde{\mu}_{2}\hat{\mu}_{2}+\tilde{\mu}_{2}\hat{\mu}_{2}\hat{R}_{2}^{(2)}\left(1\right)+\hat{r}_{2}\hat{\mu}_{2}F_{2,2}(1)+\frac{\tilde{\mu}_{2}\hat{\mu}_{2}}{1-\hat{\mu}_{2}}\hat{f}_{2}(2)+2\hat{r}_{2}\frac{\tilde{\mu}_{2}\hat{\mu}_{2}}{1-\hat{\mu}_{2}}\hat{f}_{2}(2)\\
&+&\tilde{\mu}_{2}\hat{\mu}_{2}\hat{\theta}_{2}^{(2)}\left(1\right)\hat{f}_{2}(2)+\frac{\hat{\mu}_{2}}{1-\hat{\mu}_{2}}\hat{f}_{2}(2)F_{2,2}(1)+\tilde{\mu}_{2}\hat{\mu}_{2}\left(\frac{1}{1-\hat{\mu}_{2}}\right)^{2}\hat{f}_{2}(2,2),\\
\hat{f}_{1}\left(3,4\right)&=&\hat{r}_{2}\hat{\mu}_{1}\hat{\mu}_{2}+\hat{\mu}_{1}\hat{\mu}_{2}\hat{R}_{2}^{(2)}\left(1\right)+
\hat{r}_{2}\frac{\hat{\mu}_{1}\hat{\mu}_{2}}{1-\hat{\mu}_{2}}\hat{f}_{2}(2)+
\hat{r}_{2}\hat{\mu}_{2}\hat{f}_{2}(1)+\frac{\hat{\mu}_{2}}{1-\hat{\mu}_{2}}\hat{f}_{2}(1,2),\\
\hat{f}_{1}\left(4,4\right)&=&\hat{r}_{2}P_{2}^{(2)}\left(1\right)+\hat{\mu}_{2}^{2}\hat{R}_{2}^{(2)}\left(1\right)+2\hat{r}_{2}\frac{\hat{\mu}_{2}^{2}}{1-\hat{\mu}_{2}}\hat{f}_{2}(2)+\frac{1}{1-\hat{\mu}_{2}}\hat{P}_{2}^{(2)}\left(1\right)\hat{f}_{2}(2)\\
&+&\hat{\mu}_{2}^{2}\hat{\theta}_{2}^{(2)}\left(1\right)\hat{f}_{2}(2)+\left(\frac{\hat{\mu}_{2}}{1-\hat{\mu}_{2}}\right)^{2}\hat{f}_{2}(2,2),\\
\hat{f}_{2}\left(,1\right)&=&\hat{r}_{1}P_{1}^{(2)}\left(1\right)+
\mu_{1}^{2}\hat{R}_{1}^{(2)}\left(1\right)+2\hat{r}_{1}\mu_{1}F_{1,1}(1)+
2\hat{r}_{1}\frac{\mu_{1}^{2}}{1-\hat{\mu}_{1}}\hat{f}_{1}(1)+\frac{1}{1-\hat{\mu}_{1}}P_{1}^{(2)}\left(1\right)\hat{f}_{1}(1)\\
&+&\mu_{1}^{2}\hat{\theta}_{1}^{(2)}\left(1\right)\hat{f}_{1}(1)+2\frac{\mu_{1}}{1-\hat{\mu}_{1}}\hat{f}_{1}^(1)F_{1,1}(1)+f_{1,1}^{(2)}(1)+\left(\frac{\mu_{1}}{1-\hat{\mu}_{1}}\right)^{2}\hat{f}_{1}^{(1,1)},\\
\hat{f}_{2}\left(2,1\right)&=&\hat{r}_{1}\mu_{1}\tilde{\mu}_{2}+\mu_{1}\tilde{\mu}_{2}\hat{R}_{1}^{(2)}\left(1\right)+
\hat{r}_{1}\mu_{1}F_{1,2}(1)+\tilde{\mu}_{2}\hat{r}_{1}F_{1,1}(1)+
\frac{\mu_{1}\tilde{\mu}_{2}}{1-\hat{\mu}_{1}}\hat{f}_{1}(1)\\
&+&2\hat{r}_{1}\frac{\mu_{1}\tilde{\mu}_{2}}{1-\hat{\mu}_{1}}\hat{f}_{1}(1)+\mu_{1}\tilde{\mu}_{2}\hat{\theta}_{1}^{(2)}\left(1\right)\hat{f}_{1}(1)+
\frac{\mu_{1}}{1-\hat{\mu}_{1}}\hat{f}_{1}(1)F_{1,2}(1)+\frac{\tilde{\mu}_{2}}{1-\hat{\mu}_{1}}\hat{f}_{1}(1)F_{1,1}(1)\\
&+&f_{1}^{(2)}(1,2)+\mu_{1}\tilde{\mu}_{2}\left(\frac{1}{1-\hat{\mu}_{1}}\right)^{2}\hat{f}_{1}(1,1),\\
\hat{f}_{2}\left(3,1\right)&=&\hat{r}_{1}\mu_{1}\hat{\mu}_{1}+\mu_{1}\hat{\mu}_{1}\hat{R}_{1}^{(2)}\left(1\right)+\hat{r}_{1}\hat{\mu}_{1}F_{1,1}(1)+\hat{r}_{1}\frac{\mu_{1}\hat{\mu}_{1}}{1-\hat{\mu}_{1}}\hat{F}_{1}(1),\\
\hat{f}_{2}\left(4,1\right)&=&\hat{r}_{1}\mu_{1}\hat{\mu}_{2}+\mu_{1}\hat{\mu}_{2}\hat{R}_{1}^{(2)}\left(1\right)+\hat{r}_{1}\hat{\mu}_{2}F_{1,1}(1)+\frac{\mu_{1}\hat{\mu}_{2}}{1-\hat{\mu}_{1}}\hat{f}_{1}(1)+\hat{r}_{1}\frac{\mu_{1}\hat{\mu}_{2}}{1-\hat{\mu}_{1}}\hat{f}_{1}(1)\\
&+&\mu_{1}\hat{\mu}_{2}\hat{\theta}_{1}^{(2)}\left(1\right)\hat{f}_{1}(1)+\hat{r}_{1}\mu_{1}\left(\hat{f}_{1}(2)+\frac{\hat{\mu}_{2}}{1-\hat{\mu}_{1}}\hat{f}_{1}(1)\right)+F_{1,1}(1)\left(\hat{f}_{1}(2)+\frac{\hat{\mu}_{2}}{1-\hat{\mu}_{1}}\hat{f}_{1}(1)\right)\\
&+&\frac{\mu_{1}}{1-\hat{\mu}_{1}}\left(\hat{f}_{1}(1,2)+\frac{\hat{\mu}_{2}}{1-\hat{\mu}_{1}}\hat{f}_{1}(1,1)\right),\\
\hat{f}_{2}\left(1,2\right)&=&\hat{r}_{1}\mu_{1}\tilde{\mu}_{2}+\mu_{1}\tilde{\mu}_{2}\hat{R}_{1}^{(2)}\left(1\right)+\hat{r}_{1}\mu_{1}F_{1,2}(1)+\hat{r}_{1}\tilde{\mu}_{2}F_{1,1}(1)+\frac{\mu_{1}\tilde{\mu}_{2}}{1-\hat{\mu}_{1}}\hat{f}_{1}(1)\\
&+&2\hat{r}_{1}\frac{\mu_{1}\tilde{\mu}_{2}}{1-\hat{\mu}_{1}}\hat{f}_{1}(1)+\mu_{1}\tilde{\mu}_{2}\hat{\theta}_{1}^{(2)}\left(1\right)\hat{f}_{1}(1)+\frac{\mu_{1}}{1-\hat{\mu}_{1}}\hat{f}_{1}(1)F_{1,2}(1)\\
&+&\frac{\tilde{\mu}_{2}}{1-\hat{\mu}_{1}}\hat{f}_{1}(1)F_{1,1}(1)+f_{1}^{(2)}(1,2)+\mu_{1}\tilde{\mu}_{2}\left(\frac{1}{1-\hat{\mu}_{1}}\right)^{2}\hat{f}_{1}(1,1),\\
\end{eqnarray*}
\begin{eqnarray*}
\hat{f}_{2}\left(2,2\right)&=&\hat{r}_{1}\tilde{P}_{2}^{(2)}\left(1\right)+\tilde{\mu}_{2}^{2}\hat{R}_{1}^{(2)}\left(1\right)+2\hat{r}_{1}\tilde{\mu}_{2}F_{1,2}(1)+ f_{1,2}^{(2)}(1)+2\hat{r}_{1}\frac{\tilde{\mu}_{2}^{2}}{1-\hat{\mu}_{1}}\hat{f}_{1}(1)\\
&+&\frac{1}{1-\hat{\mu}_{1}}\tilde{P}_{2}^{(2)}\left(1\right)\hat{f}_{1}(1)+\tilde{\mu}_{2}^{2}\hat{\theta}_{1}^{(2)}\left(1\right)\hat{f}_{1}(1)+2\frac{\tilde{\mu}_{2}}{1-\hat{\mu}_{1}}F_{1,2}(1)\hat{f}_{1}(1)+\left(\frac{\tilde{\mu}_{2}}{1-\hat{\mu}_{1}}\right)^{2}\hat{f}_{1}(1,1),\\
\hat{f}_{2}\left(3,2\right)&=&\hat{r}_{1}\hat{\mu}_{1}\tilde{\mu}_{2}+\hat{\mu}_{1}\tilde{\mu}_{2}\hat{R}_{1}^{(2)}\left(1\right)+
\hat{r}_{1}\hat{\mu}_{1}F_{1,2}(1)+\hat{r}_{1}\frac{\hat{\mu}_{1}\tilde{\mu}_{2}}{1-\hat{\mu}_{1}}\hat{f}_{1}(1),\\
\hat{f}_{2}\left(4,2\right)&=&\hat{r}_{1}\tilde{\mu}_{2}\hat{\mu}_{2}+\hat{\mu}_{2}\tilde{\mu}_{2}\hat{R}_{1}^{(2)}\left(1\right)+\hat{\mu}_{2}\hat{R}_{1}^{(2)}\left(1\right)F_{1,2}(1)+\frac{\hat{\mu}_{2}\tilde{\mu}_{2}}{1-\hat{\mu}_{1}}\hat{f}_{1}(1)\\
&+&\hat{r}_{1}\frac{\hat{\mu}_{2}\tilde{\mu}_{2}}{1-\hat{\mu}_{1}}\hat{f}_{1}(1)+\hat{\mu}_{2}\tilde{\mu}_{2}\hat{\theta}_{1}^{(2)}\left(1\right)\hat{f}_{1}(1)+\hat{r}_{1}\tilde{\mu}_{2}\left(\hat{f}_{1}(2)+\frac{\hat{\mu}_{2}}{1-\hat{\mu}_{1}}\hat{f}_{1}(1)\right)\\
&+&F_{1,2}(1)\left(\hat{f}_{1}(2)+\frac{\hat{\mu}_{2}}{1-\hat{\mu}_{1}}\hat{f}_{1}(1)\right)+\frac{\tilde{\mu}_{2}}{1-\hat{\mu}_{1}}\left(\hat{f}_{1}(1,2)+\frac{\hat{\mu}_{2}}{1-\hat{\mu}_{1}}\hat{f}_{1}(1,1)\right),\\
\hat{f}_{2}\left(1,3\right)&=&\hat{r}_{1}\mu_{1}\hat{\mu}_{1}+\mu_{1}\hat{\mu}_{1}\hat{R}_{1}^{(2)}\left(1\right)+\hat{r}_{1}\hat{\mu}_{1}F_{1,1}(1)+\hat{r}_{1}\frac{\mu_{1}\hat{\mu}_{1}}{1-\hat{\mu}_{1}}\hat{f}_{1}(1),\\
\hat{f}_{2}\left(2,3\right)&=&\hat{r}_{1}\tilde{\mu}_{2}\hat{\mu}_{1}+\tilde{\mu}_{2}\hat{\mu}_{1}\hat{R}_{1}^{(2)}\left(1\right)+\hat{r}_{1}\hat{\mu}_{1}F_{1,2}(1)+\hat{r}_{1}\frac{\tilde{\mu}_{2}\hat{\mu}_{1}}{1-\hat{\mu}_{1}}\hat{f}_{1}(1),\\
\hat{f}_{2}\left(3,3\right)&=&\hat{r}_{1}\hat{P}_{1}^{(2)}\left(1\right)+\hat{\mu}_{1}^{2}\hat{R}_{1}^{(2)}\left(1\right),\\
\hat{f}_{2}\left(4,3\right)&=&\hat{r}_{1}\hat{\mu}_{2}\hat{\mu}_{1}+\hat{\mu}_{2}\hat{\mu}_{1}\hat{R}_{1}^{(2)}\left(1\right)+\hat{r}_{1}\hat{\mu}_{1}\left(\hat{f}_{1}(2)+\frac{\hat{\mu}_{2}}{1-\hat{\mu}_{1}}\hat{f}_{1}(1)\right),\\
\hat{f}_{2}\left(1,4\right)&=&\hat{r}_{1}\mu_{1}\hat{\mu}_{2}+\mu_{1}\hat{\mu}_{2}\hat{R}_{1}^{(2)}\left(1\right)+\hat{r}_{1}\hat{\mu}_{2}F_{1,1}(1)+\hat{r}_{1}\frac{\mu_{1}\hat{\mu}_{2}}{1-\hat{\mu}_{1}}\hat{f}_{1}(1)+\hat{r}_{1}\mu_{1}\left(\hat{f}_{1}(2)+\frac{\hat{\mu}_{2}}{1-\hat{\mu}_{1}}\hat{f}_{1}(1)\right)\\
&+&F_{1,1}(1)\left(\hat{f}_{1}(2)+\frac{\hat{\mu}_{2}}{1-\hat{\mu}_{1}}\hat{f}_{1}(1)\right)+\frac{\mu_{1}\hat{\mu}_{2}}{1-\hat{\mu}_{1}}\hat{f}_{1}(1)+\mu_{1}\hat{\mu}_{2}\hat{\theta}_{1}^{(2)}\left(1\right)\hat{f}_{1}(1)\\
&+&\frac{\mu_{1}}{1-\hat{\mu}_{1}}\hat{f}_{1}(1,2)+\mu_{1}\hat{\mu}_{2}\left(\frac{1}{1-\hat{\mu}_{1}}\right)^{2}\hat{f}_{1}(1,1),\\
\hat{f}_{2}\left(2,4\right)&=&\hat{r}_{1}\tilde{\mu}_{2}\hat{\mu}_{2}+\tilde{\mu}_{2}\hat{\mu}_{2}\hat{R}_{1}^{(2)}\left(1\right)+\hat{r}_{1}\hat{\mu}_{2}F_{1,2}(1)+\hat{r}_{1}\frac{\tilde{\mu}_{2}\hat{\mu}_{2}}{1-\hat{\mu}_{1}}\hat{f}_{1}(1)\\
&+&\hat{r}_{1}\tilde{\mu}_{2}\left(\hat{f}_{1}(2)+\frac{\hat{\mu}_{2}}{1-\hat{\mu}_{1}}\hat{f}_{1}(1)\right)+F_{1,2}(1)\left(\hat{f}_{1}(2)+\frac{\hat{\mu}_{2}}{1-\hat{\mu}_{1}}\hat{F}_{1}^{(1,0)}\right)+\frac{\tilde{\mu}_{2}\hat{\mu}_{2}}{1-\hat{\mu}_{1}}\hat{f}_{1}(1)\\
&+&\tilde{\mu}_{2}\hat{\mu}_{2}\hat{\theta}_{1}^{(2)}\left(1\right)\hat{f}_{1}(1)+\frac{\tilde{\mu}_{2}}{1-\hat{\mu}_{1}}\hat{f}_{1}(1,2)+\tilde{\mu}_{2}\hat{\mu}_{2}\left(\frac{1}{1-\hat{\mu}_{1}}\right)^{2}\hat{f}_{1}(1,1),\\
\hat{f}_{2}\left(3,4\right)&=&\hat{r}_{1}\hat{\mu}_{2}\hat{\mu}_{1}+\hat{\mu}_{2}\hat{\mu}_{1}\hat{R}_{1}^{(2)}\left(1\right)+\hat{r}_{1}\hat{\mu}_{1}\left(\hat{f}_{1}(2)+\frac{\hat{\mu}_{2}}{1-\hat{\mu}_{1}}\hat{f}_{1}(1)\right),\\
\hat{f}_{2}\left(4,4\right)&=&\hat{r}_{1}\hat{P}_{2}^{(2)}\left(1\right)+\hat{\mu}_{2}^{2}\hat{R}_{1}^{(2)}\left(1\right)+
2\hat{r}_{1}\hat{\mu}_{2}\left(\hat{f}_{1}(2)+\frac{\hat{\mu}_{2}}{1-\hat{\mu}_{1}}\hat{f}_{1}(1)\right)+\hat{f}_{1}(2,2)\\
&+&\frac{1}{1-\hat{\mu}_{1}}\hat{P}_{2}^{(2)}\left(1\right)\hat{f}_{1}(1)+\hat{\mu}_{2}^{2}\hat{\theta}_{1}^{(2)}\left(1\right)\hat{f}_{1}(1)+\frac{\hat{\mu}_{2}}{1-\hat{\mu}_{1}}\hat{f}_{1}(1,2)\\
&+&\frac{\hat{\mu}_{2}}{1-\hat{\mu}_{1}}\left(\hat{f}_{1}(1,2)+\frac{\hat{\mu}_{2}}{1-\hat{\mu}_{1}}\hat{f}_{1}(1,1)\right).
\end{eqnarray*}
%_________________________________________________________________________________________________________
\section{Medidas de Desempe\~no}
%_________________________________________________________________________________________________________

\begin{Def}
Sea $L_{i}^{*}$el n\'umero de usuarios cuando el servidor visita la cola $Q_{i}$ para dar servicio, para $i=1,2$.
\end{Def}

Entonces
\begin{Prop} Para la cola $Q_{i}$, $i=1,2$, se tiene que el n\'umero de usuarios presentes al momento de ser visitada por el servidor est\'a dado por
\begin{eqnarray}
\esp\left[L_{i}^{*}\right]&=&f_{i}\left(i\right)\\
Var\left[L_{i}^{*}\right]&=&f_{i}\left(i,i\right)+\esp\left[L_{i}^{*}\right]-\esp\left[L_{i}^{*}\right]^{2}.
\end{eqnarray}
\end{Prop}


\begin{Def}
El tiempo de Ciclo $C_{i}$ es el periodo de tiempo que comienza
cuando la cola $i$ es visitada por primera vez en un ciclo, y
termina cuando es visitado nuevamente en el pr\'oximo ciclo, bajo condiciones de estabilidad.

\begin{eqnarray*}
C_{i}\left(z\right)=\esp\left[z^{\overline{\tau}_{i}\left(m+1\right)-\overline{\tau}_{i}\left(m\right)}\right]
\end{eqnarray*}
\end{Def}

\begin{Def}
El tiempo de intervisita $I_{i}$ es el periodo de tiempo que
comienza cuando se ha completado el servicio en un ciclo y termina
cuando es visitada nuevamente en el pr\'oximo ciclo.
\begin{eqnarray*}I_{i}\left(z\right)&=&\esp\left[z^{\tau_{i}\left(m+1\right)-\overline{\tau}_{i}\left(m\right)}\right]\end{eqnarray*}
\end{Def}

\begin{Prop}
Para los tiempos de intervisita del servidor $I_{i}$, se tiene que

\begin{eqnarray*}
\esp\left[I_{i}\right]&=&\frac{f_{i}\left(i\right)}{\mu_{i}},\\
Var\left[I_{i}\right]&=&\frac{Var\left[L_{i}^{*}\right]}{\mu_{i}^{2}}-\frac{\sigma_{i}^{2}}{\mu_{i}^{2}}f_{i}\left(i\right).
\end{eqnarray*}
\end{Prop}


\begin{Prop}
Para los tiempos que ocupa el servidor para atender a los usuarios presentes en la cola $Q_{i}$, con FGP denotada por $S_{i}$, se tiene que
\begin{eqnarray*}
\esp\left[S_{i}\right]&=&\frac{\esp\left[L_{i}^{*}\right]}{1-\mu_{i}}=\frac{f_{i}\left(i\right)}{1-\mu_{i}},\\
Var\left[S_{i}\right]&=&\frac{Var\left[L_{i}^{*}\right]}{\left(1-\mu_{i}\right)^{2}}+\frac{\sigma^{2}\esp\left[L_{i}^{*}\right]}{\left(1-\mu_{i}\right)^{3}}
\end{eqnarray*}
\end{Prop}


\begin{Prop}
Para la duraci\'on de los ciclos $C_{i}$ se tiene que
\begin{eqnarray*}
\esp\left[C_{i}\right]&=&\esp\left[I_{i}\right]\esp\left[\theta_{i}\left(z\right)\right]=\frac{\esp\left[L_{i}^{*}\right]}{\mu_{i}}\frac{1}{1-\mu_{i}}=\frac{f_{i}\left(i\right)}{\mu_{i}\left(1-\mu_{i}\right)}\\
Var\left[C_{i}\right]&=&\frac{Var\left[L_{i}^{*}\right]}{\mu_{i}^{2}\left(1-\mu_{i}\right)^{2}}.
\end{eqnarray*}

\end{Prop}

%___________________________________________________________________________________________
%
\section*{Ap\'endice A}\label{Segundos.Momentos}
%___________________________________________________________________________________________


%___________________________________________________________________________________________

%\subsubsection{Mixtas para $z_{1}$:}
%___________________________________________________________________________________________
\begin{enumerate}

%1/1/1
\item \begin{eqnarray*}
&&\frac{\partial}{\partial z_1}\frac{\partial}{\partial z_1}\left(R_2\left(P_1\left(z_1\right)\bar{P}_2\left(z_2\right)\hat{P}_1\left(w_1\right)\hat{P}_2\left(w_2\right)\right)F_2\left(z_1,\theta
_2\left(P_1\left(z_1\right)\hat{P}_1\left(w_1\right)\hat{P}_2\left(w_2\right)\right)\right)\hat{F}_2\left(w_1,w_2\right)\right)\\
&=&r_{2}P_{1}^{(2)}\left(1\right)+\mu_{1}^{2}R_{2}^{(2)}\left(1\right)+2\mu_{1}r_{2}\left(\frac{\mu_{1}}{1-\tilde{\mu}_{2}}F_{2}^{(0,1)}+F_{2}^{1,0)}\right)+\frac{1}{1-\tilde{\mu}_{2}}P_{1}^{(2)}F_{2}^{(0,1)}+\mu_{1}^{2}\tilde{\theta}_{2}^{(2)}\left(1\right)F_{2}^{(0,1)}\\
&+&\frac{\mu_{1}}{1-\tilde{\mu}_{2}}F_{2}^{(1,1)}+\frac{\mu_{1}}{1-\tilde{\mu}_{2}}\left(\frac{\mu_{1}}{1-\tilde{\mu}_{2}}F_{2}^{(0,2)}+F_{2}^{(1,1)}\right)+F_{2}^{(2,0)}.
\end{eqnarray*}

%2/2/1

\item \begin{eqnarray*}
&&\frac{\partial}{\partial z_2}\frac{\partial}{\partial z_1}\left(R_2\left(P_1\left(z_1\right)\bar{P}_2\left(z_2\right)\hat{P}_1\left(w_1\right)\hat{P}_2\left(w_2\right)\right)F_2\left(z_1,\theta
_2\left(P_1\left(z_1\right)\hat{P}_1\left(w_1\right)\hat{P}_2\left(w_2\right)\right)\right)\hat{F}_2\left(w_1,w_2\right)\right)\\
&=&\mu_{1}r_{2}\tilde{\mu}_{2}+\mu_{1}\tilde{\mu}_{2}R_{2}^{(2)}\left(1\right)+r_{2}\tilde{\mu}_{2}\left(\frac{\mu_{1}}{1-\tilde{\mu}_{2}}F_{2}^{(0,1)}+F_{2}^{(1,0)}\right).
\end{eqnarray*}
%3/3/1
\item \begin{eqnarray*}
&&\frac{\partial}{\partial w_1}\frac{\partial}{\partial z_1}\left(R_2\left(P_1\left(z_1\right)\bar{P}_2\left(z_2\right)\hat{P}_1\left(w_1\right)\hat{P}_2\left(w_2\right)\right)F_2\left(z_1,\theta
_2\left(P_1\left(z_1\right)\hat{P}_1\left(w_1\right)\hat{P}_2\left(w_2\right)\right)\right)\hat{F}_2\left(w_1,w_2\right)\right)\\
&=&\mu_{1}\hat{\mu}_{1}r_{2}+\mu_{1}\hat{\mu}_{1}R_{2}^{(2)}\left(1\right)+r_{2}\frac{\mu_{1}}{1-\tilde{\mu}_{2}}F_{2}^{(0,1)}+r_{2}\hat{\mu}_{1}\left(\frac{\mu_{1}}{1-\tilde{\mu}_{2}}F_{2}^{(0,1)}+F_{2}^{(1,0)}\right)+\mu_{1}r_{2}\hat{F}_{2}^{(1,0)}\\
&+&\left(\frac{\mu_{1}}{1-\tilde{\mu}_{2}}F_{2}^{(0,1)}+F_{2}^{(1,0)}\right)\hat{F}_{2}^{(1,0)}+\frac{\mu_{1}\hat{\mu}_{1}}{1-\tilde{\mu}_{2}}F_{2}^{(0,1)}+\mu_{1}\hat{\mu}_{1}\tilde{\theta}_{2}^{(2)}\left(1\right)F_{2}^{(0,1)}\\
&+&\mu_{1}\hat{\mu}_{1}\left(\frac{1}{1-\tilde{\mu}_{2}}\right)^{2}F_{2}^{(0,2)}+\frac{\hat{\mu}_{1}}{1-\tilde{\mu}_{2}}F_{2}^{(1,1)}.
\end{eqnarray*}
%4/4/1
\item \begin{eqnarray*}
&&\frac{\partial}{\partial w_2}\frac{\partial}{\partial z_1}\left(R_2\left(P_1\left(z_1\right)\bar{P}_2\left(z_2\right)\hat{P}_1\left(w_1\right)\hat{P}_2\left(w_2\right)\right)
F_2\left(z_1,\theta_2\left(P_1\left(z_1\right)\hat{P}_1\left(w_1\right)\hat{P}_2\left(w_2\right)\right)\right)\hat{F}_2\left(w_1,w_2\right)\right)\\
&=&\mu_{1}\hat{\mu}_{2}r_{2}+\mu_{1}\hat{\mu}_{2}R_{2}^{(2)}\left(1\right)+r_{2}\frac{\mu_{1}\hat{\mu}_{2}}{1-\tilde{\mu}_{2}}F_{2}^{(0,1)}+\mu_{1}r_{2}\hat{F}_{2}^{(0,1)}
+r_{2}\hat{\mu}_{2}\left(\frac{\mu_{1}}{1-\tilde{\mu}_{2}}F_{2}^{(0,1)}+F_{2}^{(1,0)}\right)\\
&+&\hat{F}_{2}^{(1,0)}\left(\frac{\mu_{1}}{1-\tilde{\mu}_{2}}F_{2}^{(0,1)}+F_{2}^{(1,0)}\right)+\frac{\mu_{1}\hat{\mu}_{2}}{1-\tilde{\mu}_{2}}F_{2}^{(0,1)}
+\mu_{1}\hat{\mu}_{2}\tilde{\theta}_{2}^{(2)}\left(1\right)F_{2}^{(0,1)}+\mu_{1}\hat{\mu}_{2}\left(\frac{1}{1-\tilde{\mu}_{2}}\right)^{2}F_{2}^{(0,2)}\\
&+&\frac{\hat{\mu}_{2}}{1-\tilde{\mu}_{2}}F_{2}^{(1,1)}.
\end{eqnarray*}
%___________________________________________________________________________________________
%\subsubsection{Mixtas para $z_{2}$:}
%___________________________________________________________________________________________
%5
\item \begin{eqnarray*} &&\frac{\partial}{\partial
z_1}\frac{\partial}{\partial
z_2}\left(R_2\left(P_1\left(z_1\right)\bar{P}_2\left(z_2\right)\hat{P}_1\left(w_1\right)\hat{P}_2\left(w_2\right)\right)
F_2\left(z_1,\theta_2\left(P_1\left(z_1\right)\hat{P}_1\left(w_1\right)\hat{P}_2\left(w_2\right)\right)\right)\hat{F}_2\left(w_1,w_2\right)\right)\\
&=&\mu_{1}\tilde{\mu}_{2}r_{2}+\mu_{1}\tilde{\mu}_{2}R_{2}^{(2)}\left(1\right)+r_{2}\tilde{\mu}_{2}\left(\frac{\mu_{1}}{1-\tilde{\mu}_{2}}F_{2}^{(0,1)}+F_{2}^{(1,0)}\right).
\end{eqnarray*}

%6

\item \begin{eqnarray*} &&\frac{\partial}{\partial
z_2}\frac{\partial}{\partial
z_2}\left(R_2\left(P_1\left(z_1\right)\bar{P}_2\left(z_2\right)\hat{P}_1\left(w_1\right)\hat{P}_2\left(w_2\right)\right)
F_2\left(z_1,\theta_2\left(P_1\left(z_1\right)\hat{P}_1\left(w_1\right)\hat{P}_2\left(w_2\right)\right)\right)\hat{F}_2\left(w_1,w_2\right)\right)\\
&=&\tilde{\mu}_{2}^{2}R_{2}^{(2)}(1)+r_{2}\tilde{P}_{2}^{(2)}\left(1\right).
\end{eqnarray*}

%7
\item \begin{eqnarray*} &&\frac{\partial}{\partial
w_1}\frac{\partial}{\partial
z_2}\left(R_2\left(P_1\left(z_1\right)\bar{P}_2\left(z_2\right)\hat{P}_1\left(w_1\right)\hat{P}_2\left(w_2\right)\right)
F_2\left(z_1,\theta_2\left(P_1\left(z_1\right)\hat{P}_1\left(w_1\right)\hat{P}_2\left(w_2\right)\right)\right)\hat{F}_2\left(w_1,w_2\right)\right)\\
&=&\hat{\mu}_{1}\tilde{\mu}_{2}r_{2}+\hat{\mu}_{1}\tilde{\mu}_{2}R_{2}^{(2)}(1)+
r_{2}\frac{\hat{\mu}_{1}\tilde{\mu}_{2}}{1-\tilde{\mu}_{2}}F_{2}^{(0,1)}+r_{2}\tilde{\mu}_{2}\hat{F}_{2}^{(1,0)}.
\end{eqnarray*}
%8
\item \begin{eqnarray*} &&\frac{\partial}{\partial
w_2}\frac{\partial}{\partial
z_2}\left(R_2\left(P_1\left(z_1\right)\bar{P}_2\left(z_2\right)\hat{P}_1\left(w_1\right)\hat{P}_2\left(w_2\right)\right)
F_2\left(z_1,\theta_2\left(P_1\left(z_1\right)\hat{P}_1\left(w_1\right)\hat{P}_2\left(w_2\right)\right)\right)\hat{F}_2\left(w_1,w_2\right)\right)\\
&=&\hat{\mu}_{2}\tilde{\mu}_{2}r_{2}+\hat{\mu}_{2}\tilde{\mu}_{2}R_{2}^{(2)}(1)+
r_{2}\frac{\hat{\mu}_{2}\tilde{\mu}_{2}}{1-\tilde{\mu}_{2}}F_{2}^{(0,1)}+r_{2}\tilde{\mu}_{2}\hat{F}_{2}^{(0,1)}.
\end{eqnarray*}
%___________________________________________________________________________________________
%\subsubsection{Mixtas para $w_{1}$:}
%___________________________________________________________________________________________

%9
\item \begin{eqnarray*} &&\frac{\partial}{\partial
z_1}\frac{\partial}{\partial
w_1}\left(R_2\left(P_1\left(z_1\right)\bar{P}_2\left(z_2\right)\hat{P}_1\left(w_1\right)\hat{P}_2\left(w_2\right)\right)
F_2\left(z_1,\theta_2\left(P_1\left(z_1\right)\hat{P}_1\left(w_1\right)\hat{P}_2\left(w_2\right)\right)\right)\hat{F}_2\left(w_1,w_2\right)\right)\\
&=&\mu_{1}\hat{\mu}_{1}r_{2}+\mu_{1}\hat{\mu}_{1}R_{2}^{(2)}\left(1\right)+\frac{\mu_{1}\hat{\mu}_{1}}{1-\tilde{\mu}_{2}}F_{2}^{(0,1)}+r_{2}\frac{\mu_{1}\hat{\mu}_{1}}{1-\tilde{\mu}_{2}}F_{2}^{(0,1)}+\mu_{1}\hat{\mu}_{1}\tilde{\theta}_{2}^{(2)}\left(1\right)F_{2}^{(0,1)}\\
&+&r_{2}\hat{\mu}_{1}\left(\frac{\mu_{1}}{1-\tilde{\mu}_{2}}F_{2}^{(0,1)}+F_{2}^{(1,0)}\right)+r_{2}\mu_{1}\hat{F}_{2}^{(1,0)}
+\left(\frac{\mu_{1}}{1-\tilde{\mu}_{2}}F_{2}^{(0,1)}+F_{2}^{(1,0)}\right)\hat{F}_{2}^{(1,0)}\\
&+&\frac{\hat{\mu}_{1}}{1-\tilde{\mu}_{2}}\left(\frac{\mu_{1}}{1-\tilde{\mu}_{2}}F_{2}^{(0,2)}+F_{2}^{(1,1)}\right).
\end{eqnarray*}
%10
\item \begin{eqnarray*} &&\frac{\partial}{\partial
z_2}\frac{\partial}{\partial
w_1}\left(R_2\left(P_1\left(z_1\right)\bar{P}_2\left(z_2\right)\hat{P}_1\left(w_1\right)\hat{P}_2\left(w_2\right)\right)
F_2\left(z_1,\theta_2\left(P_1\left(z_1\right)\hat{P}_1\left(w_1\right)\hat{P}_2\left(w_2\right)\right)\right)\hat{F}_2\left(w_1,w_2\right)\right)\\
&=&\tilde{\mu}_{2}\hat{\mu}_{1}r_{2}+\tilde{\mu}_{2}\hat{\mu}_{1}R_{2}^{(2)}\left(1\right)+r_{2}\frac{\tilde{\mu}_{2}\hat{\mu}_{1}}{1-\tilde{\mu}_{2}}F_{2}^{(0,1)}
+r_{2}\tilde{\mu}_{2}\hat{F}_{2}^{(1,0)}.
\end{eqnarray*}
%11
\item \begin{eqnarray*} &&\frac{\partial}{\partial
w_1}\frac{\partial}{\partial
w_1}\left(R_2\left(P_1\left(z_1\right)\bar{P}_2\left(z_2\right)\hat{P}_1\left(w_1\right)\hat{P}_2\left(w_2\right)\right)
F_2\left(z_1,\theta_2\left(P_1\left(z_1\right)\hat{P}_1\left(w_1\right)\hat{P}_2\left(w_2\right)\right)\right)\hat{F}_2\left(w_1,w_2\right)\right)\\
&=&\hat{\mu}_{1}^{2}R_{2}^{(2)}\left(1\right)+r_{2}\hat{P}_{1}^{(2)}\left(1\right)+2r_{2}\frac{\hat{\mu}_{1}^{2}}{1-\tilde{\mu}_{2}}F_{2}^{(0,1)}+
\hat{\mu}_{1}^{2}\tilde{\theta}_{2}^{(2)}\left(1\right)F_{2}^{(0,1)}+\frac{1}{1-\tilde{\mu}_{2}}\hat{P}_{1}^{(2)}\left(1\right)F_{2}^{(0,1)}\\
&+&\frac{\hat{\mu}_{1}^{2}}{1-\tilde{\mu}_{2}}F_{2}^{(0,2)}+2r_{2}\hat{\mu}_{1}\hat{F}_{2}^{(1,0)}+2\frac{\hat{\mu}_{1}}{1-\tilde{\mu}_{2}}F_{2}^{(0,1)}\hat{F}_{2}^{(1,0)}+\hat{F}_{2}^{(2,0)}.
\end{eqnarray*}
%12
\item \begin{eqnarray*} &&\frac{\partial}{\partial
w_2}\frac{\partial}{\partial
w_1}\left(R_2\left(P_1\left(z_1\right)\bar{P}_2\left(z_2\right)\hat{P}_1\left(w_1\right)\hat{P}_2\left(w_2\right)\right)
F_2\left(z_1,\theta_2\left(P_1\left(z_1\right)\hat{P}_1\left(w_1\right)\hat{P}_2\left(w_2\right)\right)\right)\hat{F}_2\left(w_1,w_2\right)\right)\\
&=&r_{2}\hat{\mu}_{2}\hat{\mu}_{1}+\hat{\mu}_{1}\hat{\mu}_{2}R_{2}^{(2)}(1)+\frac{\hat{\mu}_{1}\hat{\mu}_{2}}{1-\tilde{\mu}_{2}}F_{2}^{(0,1)}
+2r_{2}\frac{\hat{\mu}_{1}\hat{\mu}_{2}}{1-\tilde{\mu}_{2}}F_{2}^{(0,1)}+\hat{\mu}_{2}\hat{\mu}_{1}\tilde{\theta}_{2}^{(2)}\left(1\right)F_{2}^{(0,1)}+
r_{2}\hat{\mu}_{1}\hat{F}_{2}^{(0,1)}\\
&+&\frac{\hat{\mu}_{1}}{1-\tilde{\mu}_{2}}F_{2}^{(0,1)}\hat{F}_{2}^{(0,1)}+\hat{\mu}_{1}\hat{\mu}_{2}\left(\frac{1}{1-\tilde{\mu}_{2}}\right)^{2}F_{2}^{(0,2)}+
r_{2}\hat{\mu}_{2}\hat{F}_{2}^{(1,0)}+\frac{\hat{\mu}_{2}}{1-\tilde{\mu}_{2}}F_{2}^{(0,1)}\hat{F}_{2}^{(1,0)}+\hat{F}_{2}^{(1,1)}.
\end{eqnarray*}
%___________________________________________________________________________________________
%\subsubsection{Mixtas para $w_{2}$:}
%___________________________________________________________________________________________
%13

\item \begin{eqnarray*} &&\frac{\partial}{\partial
z_1}\frac{\partial}{\partial
w_2}\left(R_2\left(P_1\left(z_1\right)\bar{P}_2\left(z_2\right)\hat{P}_1\left(w_1\right)\hat{P}_2\left(w_2\right)\right)
F_2\left(z_1,\theta_2\left(P_1\left(z_1\right)\hat{P}_1\left(w_1\right)\hat{P}_2\left(w_2\right)\right)\right)\hat{F}_2\left(w_1,w_2\right)\right)\\
&=&r_{2}\mu_{1}\hat{\mu}_{2}+\mu_{1}\hat{\mu}_{2}R_{2}^{(2)}(1)+\frac{\mu_{1}\hat{\mu}_{2}}{1-\tilde{\mu}_{2}}F_{2}^{(0,1)}+r_{2}\frac{\mu_{1}\hat{\mu}_{2}}{1-\tilde{\mu}_{2}}F_{2}^{(0,1)}+\mu_{1}\hat{\mu}_{2}\tilde{\theta}_{2}^{(2)}\left(1\right)F_{2}^{(0,1)}+r_{2}\mu_{1}\hat{F}_{2}^{(0,1)}\\
&+&r_{2}\hat{\mu}_{2}\left(\frac{\mu_{1}}{1-\tilde{\mu}_{2}}F_{2}^{(0,1)}+F_{2}^{(1,0)}\right)+\hat{F}_{2}^{(0,1)}\left(\frac{\mu_{1}}{1-\tilde{\mu}_{2}}F_{2}^{(0,1)}+F_{2}^{(1,0)}\right)+\frac{\hat{\mu}_{2}}{1-\tilde{\mu}_{2}}\left(\frac{\mu_{1}}{1-\tilde{\mu}_{2}}F_{2}^{(0,2)}+F_{2}^{(1,1)}\right).
\end{eqnarray*}
%14
\item \begin{eqnarray*} &&\frac{\partial}{\partial
z_2}\frac{\partial}{\partial
w_2}\left(R_2\left(P_1\left(z_1\right)\bar{P}_2\left(z_2\right)\hat{P}_1\left(w_1\right)\hat{P}_2\left(w_2\right)\right)
F_2\left(z_1,\theta_2\left(P_1\left(z_1\right)\hat{P}_1\left(w_1\right)\hat{P}_2\left(w_2\right)\right)\right)\hat{F}_2\left(w_1,w_2\right)\right)\\
&=&r_{2}\tilde{\mu}_{2}\hat{\mu}_{2}+\tilde{\mu}_{2}\hat{\mu}_{2}R_{2}^{(2)}(1)+r_{2}\frac{\tilde{\mu}_{2}\hat{\mu}_{2}}{1-\tilde{\mu}_{2}}F_{2}^{(0,1)}+r_{2}\tilde{\mu}_{2}\hat{F}_{2}^{(0,1)}.
\end{eqnarray*}
%15
\item \begin{eqnarray*} &&\frac{\partial}{\partial
w_1}\frac{\partial}{\partial
w_2}\left(R_2\left(P_1\left(z_1\right)\bar{P}_2\left(z_2\right)\hat{P}_1\left(w_1\right)\hat{P}_2\left(w_2\right)\right)
F_2\left(z_1,\theta_2\left(P_1\left(z_1\right)\hat{P}_1\left(w_1\right)\hat{P}_2\left(w_2\right)\right)\right)\hat{F}_2\left(w_1,w_2\right)\right)\\
&=&r_{2}\hat{\mu}_{1}\hat{\mu}_{2}+\hat{\mu}_{1}\hat{\mu}_{2}R_{2}^{(2)}\left(1\right)+\frac{\hat{\mu}_{1}\hat{\mu}_{2}}{1-\tilde{\mu}_{2}}F_{2}^{(0,1)}+2r_{2}\frac{\hat{\mu}_{1}\hat{\mu}_{2}}{1-\tilde{\mu}_{2}}F_{2}^{(0,1)}+\hat{\mu}_{1}\hat{\mu}_{2}\theta_{2}^{(2)}\left(1\right)F_{2}^{(0,1)}+r_{2}\hat{\mu}_{1}\hat{F}_{2}^{(0,1)}\\
&+&\frac{\hat{\mu}_{1}}{1-\tilde{\mu}_{2}}F_{2}^{(0,1)}\hat{F}_{2}^{(0,1)}+\hat{\mu}_{1}\hat{\mu}_{2}\left(\frac{1}{1-\tilde{\mu}_{2}}\right)^{2}F_{2}^{(0,2)}+r_{2}\hat{\mu}_{2}\hat{F}_{2}^{(0,1)}+\frac{\hat{\mu}_{2}}{1-\tilde{\mu}_{2}}F_{2}^{(0,1)}\hat{F}_{2}^{(1,0)}+\hat{F}_{2}^{(1,1)}.
\end{eqnarray*}
%16

\item \begin{eqnarray*} &&\frac{\partial}{\partial
w_2}\frac{\partial}{\partial
w_2}\left(R_2\left(P_1\left(z_1\right)\bar{P}_2\left(z_2\right)\hat{P}_1\left(w_1\right)\hat{P}_2\left(w_2\right)\right)
F_2\left(z_1,\theta_2\left(P_1\left(z_1\right)\hat{P}_1\left(w_1\right)\hat{P}_2\left(w_2\right)\right)\right)\hat{F}_2\left(w_1,w_2\right)\right)\\
&=&\hat{\mu}_{2}^{2}R_{2}^{(2)}(1)+r_{2}\hat{P}_{2}^{(2)}\left(1\right)+2r_{2}\frac{\hat{\mu}_{2}^{2}}{1-\tilde{\mu}_{2}}F_{2}^{(0,1)}+\hat{\mu}_{2}^{2}\tilde{\theta}_{2}^{(2)}\left(1\right)F_{2}^{(0,1)}+\frac{1}{1-\tilde{\mu}_{2}}\hat{P}_{2}^{(2)}\left(1\right)F_{2}^{(0,1)}\\
&+&2r_{2}\hat{\mu}_{2}\hat{F}_{2}^{(0,1)}+2\frac{\hat{\mu}_{2}}{1-\tilde{\mu}_{2}}F_{2}^{(0,1)}\hat{F}_{2}^{(0,1)}+\left(\frac{\hat{\mu}_{2}}{1-\tilde{\mu}_{2}}\right)^{2}F_{2}^{(0,2)}+\hat{F}_{2}^{(0,2)}.
\end{eqnarray*}
\end{enumerate}
%___________________________________________________________________________________________
%
%\subsection{Derivadas de Segundo Orden para $F_{2}$}
%___________________________________________________________________________________________


\begin{enumerate}

%___________________________________________________________________________________________
%\subsubsection{Mixtas para $z_{1}$:}
%___________________________________________________________________________________________

%1/17
\item \begin{eqnarray*} &&\frac{\partial}{\partial
z_1}\frac{\partial}{\partial
z_1}\left(R_1\left(P_1\left(z_1\right)\bar{P}_2\left(z_2\right)\hat{P}_1\left(w_1\right)\hat{P}_2\left(w_2\right)\right)
F_1\left(\theta_1\left(\tilde{P}_2\left(z_1\right)\hat{P}_1\left(w_1\right)\hat{P}_2\left(w_2\right)\right)\right)\hat{F}_1\left(w_1,w_2\right)\right)\\
&=&r_{1}P_{1}^{(2)}\left(1\right)+\mu_{1}^{2}R_{1}^{(2)}\left(1\right).
\end{eqnarray*}

%2/18
\item \begin{eqnarray*} &&\frac{\partial}{\partial
z_2}\frac{\partial}{\partial
z_1}\left(R_1\left(P_1\left(z_1\right)\bar{P}_2\left(z_2\right)\hat{P}_1\left(w_1\right)\hat{P}_2\left(w_2\right)\right)F_1\left(\theta_1\left(\tilde{P}_2\left(z_1\right)\hat{P}_1\left(w_1\right)\hat{P}_2\left(w_2\right)\right)\right)\hat{F}_1\left(w_1,w_2\right)\right)\\
&=&\mu_{1}\tilde{\mu}_{2}r_{1}+\mu_{1}\tilde{\mu}_{2}R_{1}^{(2)}(1)+
r_{1}\mu_{1}\left(\frac{\tilde{\mu}_{2}}{1-\mu_{1}}F_{1}^{(1,0)}+F_{1}^{(0,1)}\right).
\end{eqnarray*}

%3/19
\item \begin{eqnarray*} &&\frac{\partial}{\partial
w_1}\frac{\partial}{\partial
z_1}\left(R_1\left(P_1\left(z_1\right)\bar{P}_2\left(z_2\right)\hat{P}_1\left(w_1\right)\hat{P}_2\left(w_2\right)\right)F_1\left(\theta_1\left(\tilde{P}_2\left(z_1\right)\hat{P}_1\left(w_1\right)\hat{P}_2\left(w_2\right)\right)\right)\hat{F}_1\left(w_1,w_2\right)\right)\\
&=&r_{1}\mu_{1}\hat{\mu}_{1}+\mu_{1}\hat{\mu}_{1}R_{1}^{(2)}\left(1\right)+r_{1}\frac{\mu_{1}\hat{\mu}_{1}}{1-\mu_{1}}F_{1}^{(1,0)}+r_{1}\mu_{1}\hat{F}_{1}^{(1,0)}.
\end{eqnarray*}
%4/20
\item \begin{eqnarray*} &&\frac{\partial}{\partial
w_2}\frac{\partial}{\partial
z_1}\left(R_1\left(P_1\left(z_1\right)\bar{P}_2\left(z_2\right)\hat{P}_1\left(w_1\right)\hat{P}_2\left(w_2\right)\right)F_1\left(\theta_1\left(\tilde{P}_2\left(z_1\right)\hat{P}_1\left(w_1\right)\hat{P}_2\left(w_2\right)\right)\right)\hat{F}_1\left(w_1,w_2\right)\right)\\
&=&\mu_{1}\hat{\mu}_{2}r_{1}+\mu_{1}\hat{\mu}_{2}R_{1}^{(2)}\left(1\right)+r_{1}\mu_{1}\hat{F}_{1}^{(0,1)}+r_{1}\frac{\mu_{1}\hat{\mu}_{2}}{1-\mu_{1}}F_{1}^{(1,0)}.
\end{eqnarray*}
%___________________________________________________________________________________________
%\subsubsection{Mixtas para $z_{2}$:}
%___________________________________________________________________________________________
%5/21
\item \begin{eqnarray*}
&&\frac{\partial}{\partial z_1}\frac{\partial}{\partial z_2}\left(R_1\left(P_1\left(z_1\right)\bar{P}_2\left(z_2\right)\hat{P}_1\left(w_1\right)\hat{P}_2\left(w_2\right)\right)F_1\left(\theta_1\left(\tilde{P}_2\left(z_1\right)\hat{P}_1\left(w_1\right)\hat{P}_2\left(w_2\right)\right)\right)\hat{F}_1\left(w_1,w_2\right)\right)\\
&=&r_{1}\mu_{1}\tilde{\mu}_{2}+\mu_{1}\tilde{\mu}_{2}R_{1}^{(2)}\left(1\right)+r_{1}\mu_{1}\left(\frac{\tilde{\mu}_{2}}{1-\mu_{1}}F_{1}^{(1,0)}+F_{1}^{(0,1)}\right).
\end{eqnarray*}

%6/22
\item \begin{eqnarray*}
&&\frac{\partial}{\partial z_2}\frac{\partial}{\partial z_2}\left(R_1\left(P_1\left(z_1\right)\bar{P}_2\left(z_2\right)\hat{P}_1\left(w_1\right)\hat{P}_2\left(w_2\right)\right)F_1\left(\theta_1\left(\tilde{P}_2\left(z_1\right)\hat{P}_1\left(w_1\right)\hat{P}_2\left(w_2\right)\right)\right)\hat{F}_1\left(w_1,w_2\right)\right)\\
&=&\tilde{\mu}_{2}^{2}R_{1}^{(2)}\left(1\right)+r_{1}\tilde{P}_{2}^{(2)}\left(1\right)+2r_{1}\tilde{\mu}_{2}\left(\frac{\tilde{\mu}_{2}}{1-\mu_{1}}F_{1}^{(1,0)}+F_{1}^{(0,1)}\right)+F_{1}^{(0,2)}+\tilde{\mu}_{2}^{2}\theta_{1}^{(2)}\left(1\right)F_{1}^{(1,0)}\\
&+&\frac{1}{1-\mu_{1}}\tilde{P}_{2}^{(2)}\left(1\right)F_{1}^{(1,0)}+\frac{\tilde{\mu}_{2}}{1-\mu_{1}}F_{1}^{(1,1)}+\frac{\tilde{\mu}_{2}}{1-\mu_{1}}\left(\frac{\tilde{\mu}_{2}}{1-\mu_{1}}F_{1}^{(2,0)}+F_{1}^{(1,1)}\right).
\end{eqnarray*}
%7/23
\item \begin{eqnarray*}
&&\frac{\partial}{\partial w_1}\frac{\partial}{\partial z_2}\left(R_1\left(P_1\left(z_1\right)\bar{P}_2\left(z_2\right)\hat{P}_1\left(w_1\right)\hat{P}_2\left(w_2\right)\right)F_1\left(\theta_1\left(\tilde{P}_2\left(z_1\right)\hat{P}_1\left(w_1\right)\hat{P}_2\left(w_2\right)\right)\right)\hat{F}_1\left(w_1,w_2\right)\right)\\
&=&\tilde{\mu}_{2}\hat{\mu}_{1}r_{1}+\tilde{\mu}_{2}\hat{\mu}_{1}R_{1}^{(2)}\left(1\right)+r_{1}\frac{\tilde{\mu}_{2}\hat{\mu}_{1}}{1-\mu_{1}}F_{1}^{(1,0)}+\hat{\mu}_{1}r_{1}\left(\frac{\tilde{\mu}_{2}}{1-\mu_{1}}F_{1}^{(1,0)}+F_{1}^{(0,1)}\right)+r_{1}\tilde{\mu}_{2}\hat{F}_{1}^{(1,0)}\\
&+&\left(\frac{\tilde{\mu}_{2}}{1-\mu_{1}}F_{1}^{(1,0)}+F_{1}^{(0,1)}\right)\hat{F}_{1}^{(1,0)}+\frac{\tilde{\mu}_{2}\hat{\mu}_{1}}{1-\mu_{1}}F_{1}^{(1,0)}+\tilde{\mu}_{2}\hat{\mu}_{1}\theta_{1}^{(2)}\left(1\right)F_{1}^{(1,0)}+\frac{\hat{\mu}_{1}}{1-\mu_{1}}F_{1}^{(1,1)}\\
&+&\left(\frac{1}{1-\mu_{1}}\right)^{2}\tilde{\mu}_{2}\hat{\mu}_{1}F_{1}^{(2,0)}.
\end{eqnarray*}
%8/24
\item \begin{eqnarray*}
&&\frac{\partial}{\partial w_2}\frac{\partial}{\partial z_2}\left(R_1\left(P_1\left(z_1\right)\bar{P}_2\left(z_2\right)\hat{P}_1\left(w_1\right)\hat{P}_2\left(w_2\right)\right)F_1\left(\theta_1\left(\tilde{P}_2\left(z_1\right)\hat{P}_1\left(w_1\right)\hat{P}_2\left(w_2\right)\right)\right)\hat{F}_1\left(w_1,w_2\right)\right)\\
&=&\hat{\mu}_{2}\tilde{\mu}_{2}r_{1}+\hat{\mu}_{2}\tilde{\mu}_{2}R_{1}^{(2)}(1)+r_{1}\tilde{\mu}_{2}\hat{F}_{1}^{(0,1)}+r_{1}\frac{\hat{\mu}_{2}\tilde{\mu}_{2}}{1-\mu_{1}}F_{1}^{(1,0)}+\hat{\mu}_{2}r_{1}\left(\frac{\tilde{\mu}_{2}}{1-\mu_{1}}F_{1}^{(1,0)}+F_{1}^{(0,1)}\right)\\
&+&\left(\frac{\tilde{\mu}_{2}}{1-\mu_{1}}F_{1}^{(1,0)}+F_{1}^{(0,1)}\right)\hat{F}_{1}^{(0,1)}+\frac{\tilde{\mu}_{2}\hat{\mu_{2}}}{1-\mu_{1}}F_{1}^{(1,0)}+\hat{\mu}_{2}\tilde{\mu}_{2}\theta_{1}^{(2)}\left(1\right)F_{1}^{(1,0)}+\frac{\hat{\mu}_{2}}{1-\mu_{1}}F_{1}^{(1,1)}\\
&+&\left(\frac{1}{1-\mu_{1}}\right)^{2}\tilde{\mu}_{2}\hat{\mu}_{2}F_{1}^{(2,0)}.
\end{eqnarray*}
%___________________________________________________________________________________________
%\subsubsection{Mixtas para $w_{1}$:}
%___________________________________________________________________________________________
%9/25
\item \begin{eqnarray*} &&\frac{\partial}{\partial
z_1}\frac{\partial}{\partial
w_1}\left(R_1\left(P_1\left(z_1\right)\bar{P}_2\left(z_2\right)\hat{P}_1\left(w_1\right)\hat{P}_2\left(w_2\right)\right)F_1\left(\theta_1\left(\tilde{P}_2\left(z_1\right)\hat{P}_1\left(w_1\right)\hat{P}_2\left(w_2\right)\right)\right)\hat{F}_1\left(w_1,w_2\right)\right)\\
&=&r_{1}\mu_{1}\hat{\mu}_{1}+\mu_{1}\hat{\mu}_{1}R_{1}^{(2)}(1)+r_{1}\frac{\mu_{1}\hat{\mu}_{1}}{1-\mu_{1}}F_{1}^{(1,0)}+r_{1}\mu_{1}\hat{F}_{1}^{(1,0)}.
\end{eqnarray*}
%10/26
\item \begin{eqnarray*} &&\frac{\partial}{\partial
z_2}\frac{\partial}{\partial
w_1}\left(R_1\left(P_1\left(z_1\right)\bar{P}_2\left(z_2\right)\hat{P}_1\left(w_1\right)\hat{P}_2\left(w_2\right)\right)F_1\left(\theta_1\left(\tilde{P}_2\left(z_1\right)\hat{P}_1\left(w_1\right)\hat{P}_2\left(w_2\right)\right)\right)\hat{F}_1\left(w_1,w_2\right)\right)\\
&=&r_{1}\hat{\mu}_{1}\tilde{\mu}_{2}+\tilde{\mu}_{2}\hat{\mu}_{1}R_{1}^{(2)}\left(1\right)+
\frac{\hat{\mu}_{1}\tilde{\mu}_{2}}{1-\mu_{1}}F_{1}^{1,0)}+r_{1}\frac{\hat{\mu}_{1}\tilde{\mu}_{2}}{1-\mu_{1}}F_{1}^{(1,0)}+\hat{\mu}_{1}\tilde{\mu}_{2}\theta_{1}^{(2)}\left(1\right)F_{2}^{(1,0)}\\
&+&r_{1}\hat{\mu}_{1}\left(F_{1}^{(1,0)}+\frac{\tilde{\mu}_{2}}{1-\mu_{1}}F_{1}^{1,0)}\right)+
r_{1}\tilde{\mu}_{2}\hat{F}_{1}^{(1,0)}+\left(F_{1}^{(0,1)}+\frac{\tilde{\mu}_{2}}{1-\mu_{1}}F_{1}^{1,0)}\right)\hat{F}_{1}^{(1,0)}\\
&+&\frac{\hat{\mu}_{1}}{1-\mu_{1}}\left(F_{1}^{(1,1)}+\frac{\tilde{\mu}_{2}}{1-\mu_{1}}F_{1}^{2,0)}\right).
\end{eqnarray*}
%11/27
\item \begin{eqnarray*} &&\frac{\partial}{\partial
w_1}\frac{\partial}{\partial
w_1}\left(R_1\left(P_1\left(z_1\right)\bar{P}_2\left(z_2\right)\hat{P}_1\left(w_1\right)\hat{P}_2\left(w_2\right)\right)F_1\left(\theta_1\left(\tilde{P}_2\left(z_1\right)\hat{P}_1\left(w_1\right)\hat{P}_2\left(w_2\right)\right)\right)\hat{F}_1\left(w_1,w_2\right)\right)\\
&=&\hat{\mu}_{1}^{2}R_{1}^{(2)}\left(1\right)+r_{1}\hat{P}_{1}^{(2)}\left(1\right)+2r_{1}\frac{\hat{\mu}_{1}^{2}}{1-\mu_{1}}F_{1}^{(1,0)}+\hat{\mu}_{1}^{2}\theta_{1}^{(2)}\left(1\right)F_{1}^{(1,0)}+\frac{1}{1-\mu_{1}}\hat{P}_{1}^{(2)}\left(1\right)F_{1}^{(1,0)}\\
&+&2r_{1}\hat{\mu}_{1}\hat{F}_{1}^{(1,0)}+2\frac{\hat{\mu}_{1}}{1-\mu_{1}}F_{1}^{(1,0)}\hat{F}_{1}^{(1,0)}+\left(\frac{\hat{\mu}_{1}}{1-\mu_{1}}\right)^{2}F_{1}^{(2,0)}+\hat{F}_{1}^{(2,0)}.
\end{eqnarray*}
%12/28
\item \begin{eqnarray*} &&\frac{\partial}{\partial
w_2}\frac{\partial}{\partial
w_1}\left(R_1\left(P_1\left(z_1\right)\bar{P}_2\left(z_2\right)\hat{P}_1\left(w_1\right)\hat{P}_2\left(w_2\right)\right)F_1\left(\theta_1\left(\tilde{P}_2\left(z_1\right)\hat{P}_1\left(w_1\right)\hat{P}_2\left(w_2\right)\right)\right)\hat{F}_1\left(w_1,w_2\right)\right)\\
&=&r_{1}\hat{\mu}_{1}\hat{\mu}_{2}+\hat{\mu}_{1}\hat{\mu}_{2}R_{1}^{(2)}\left(1\right)+r_{1}\hat{\mu}_{1}\hat{F}_{1}^{(0,1)}+
\frac{\hat{\mu}_{1}\hat{\mu}_{2}}{1-\mu_{1}}F_{1}^{(1,0)}+2r_{1}\frac{\hat{\mu}_{1}\hat{\mu}_{2}}{1-\mu_{1}}F_{1}^{1,0)}+\hat{\mu}_{1}\hat{\mu}_{2}\theta_{1}^{(2)}\left(1\right)F_{1}^{(1,0)}\\
&+&\frac{\hat{\mu}_{1}}{1-\mu_{1}}F_{1}^{(1,0)}\hat{F}_{1}^{(0,1)}+
r_{1}\hat{\mu}_{2}\hat{F}_{1}^{(1,0)}+\frac{\hat{\mu}_{2}}{1-\mu_{1}}\hat{F}_{1}^{(1,0)}F_{1}^{(1,0)}+\hat{F}_{1}^{(1,1)}+\hat{\mu}_{1}\hat{\mu}_{2}\left(\frac{1}{1-\mu_{1}}\right)^{2}F_{1}^{(2,0)}.
\end{eqnarray*}
%___________________________________________________________________________________________
%\subsubsection{Mixtas para $w_{2}$:}
%___________________________________________________________________________________________
%13/29
\item \begin{eqnarray*} &&\frac{\partial}{\partial
z_1}\frac{\partial}{\partial
w_2}\left(R_1\left(P_1\left(z_1\right)\bar{P}_2\left(z_2\right)\hat{P}_1\left(w_1\right)\hat{P}_2\left(w_2\right)\right)F_1\left(\theta_1\left(\tilde{P}_2\left(z_1\right)\hat{P}_1\left(w_1\right)\hat{P}_2\left(w_2\right)\right)\right)\hat{F}_1\left(w_1,w_2\right)\right)\\
&=&r_{1}\mu_{1}\hat{\mu}_{2}+\mu_{1}\hat{\mu}_{2}R_{1}^{(2)}\left(1\right)+r_{1}\mu_{1}\hat{F}_{1}^{(0,1)}+r_{1}\frac{\mu_{1}\hat{\mu}_{2}}{1-\mu_{1}}F_{1}^{(1,0)}.
\end{eqnarray*}
%14/30
\item \begin{eqnarray*} &&\frac{\partial}{\partial
z_2}\frac{\partial}{\partial
w_2}\left(R_1\left(P_1\left(z_1\right)\bar{P}_2\left(z_2\right)\hat{P}_1\left(w_1\right)\hat{P}_2\left(w_2\right)\right)F_1\left(\theta_1\left(\tilde{P}_2\left(z_1\right)\hat{P}_1\left(w_1\right)\hat{P}_2\left(w_2\right)\right)\right)\hat{F}_1\left(w_1,w_2\right)\right)\\
&=&r_{1}\hat{\mu}_{2}\tilde{\mu}_{2}+\hat{\mu}_{2}\tilde{\mu}_{2}R_{1}^{(2)}\left(1\right)+r_{1}\tilde{\mu}_{2}\hat{F}_{1}^{(0,1)}+\frac{\hat{\mu}_{2}\tilde{\mu}_{2}}{1-\mu_{1}}F_{1}^{(1,0)}+r_{1}\frac{\hat{\mu}_{2}\tilde{\mu}_{2}}{1-\mu_{1}}F_{1}^{(1,0)}\\
&+&\hat{\mu}_{2}\tilde{\mu}_{2}\theta_{1}^{(2)}\left(1\right)F_{1}^{(1,0)}+r_{1}\hat{\mu}_{2}\left(F_{1}^{(0,1)}+\frac{\tilde{\mu}_{2}}{1-\mu_{1}}F_{1}^{(1,0)}\right)+\left(F_{1}^{(0,1)}+\frac{\tilde{\mu}_{2}}{1-\mu_{1}}F_{1}^{(1,0)}\right)\hat{F}_{1}^{(0,1)}\\&+&\frac{\hat{\mu}_{2}}{1-\mu_{1}}\left(F_{1}^{(1,1)}+\frac{\tilde{\mu}_{2}}{1-\mu_{1}}F_{1}^{(2,0)}\right).
\end{eqnarray*}
%15/31
\item \begin{eqnarray*} &&\frac{\partial}{\partial
w_1}\frac{\partial}{\partial
w_2}\left(R_1\left(P_1\left(z_1\right)\bar{P}_2\left(z_2\right)\hat{P}_1\left(w_1\right)\hat{P}_2\left(w_2\right)\right)F_1\left(\theta_1\left(\tilde{P}_2\left(z_1\right)\hat{P}_1\left(w_1\right)\hat{P}_2\left(w_2\right)\right)\right)\hat{F}_1\left(w_1,w_2\right)\right)\\
&=&r_{1}\hat{\mu}_{1}\hat{\mu}_{2}+\hat{\mu}_{1}\hat{\mu}_{2}R_{1}^{(2)}\left(1\right)+r_{1}\hat{\mu}_{1}\hat{F}_{1}^{(0,1)}+
\frac{\hat{\mu}_{1}\hat{\mu}_{2}}{1-\mu_{1}}F_{1}^{(1,0)}+2r_{1}\frac{\hat{\mu}_{1}\hat{\mu}_{2}}{1-\mu_{1}}F_{1}^{(1,0)}+\hat{\mu}_{1}\hat{\mu}_{2}\theta_{1}^{(2)}\left(1\right)F_{1}^{(1,0)}\\
&+&\frac{\hat{\mu}_{1}}{1-\mu_{1}}\hat{F}_{1}^{(0,1)}F_{1}^{(1,0)}+r_{1}\hat{\mu}_{2}\hat{F}_{1}^{(1,0)}+\frac{\hat{\mu}_{2}}{1-\mu_{1}}\hat{F}_{1}^{(1,0)}F_{1}^{(1,0)}+\hat{F}_{1}^{(1,1)}+\hat{\mu}_{1}\hat{\mu}_{2}\left(\frac{1}{1-\mu_{1}}\right)^{2}F_{1}^{(2,0)}.
\end{eqnarray*}
%16/32
\item \begin{eqnarray*} &&\frac{\partial}{\partial
w_2}\frac{\partial}{\partial
w_2}\left(R_1\left(P_1\left(z_1\right)\bar{P}_2\left(z_2\right)\hat{P}_1\left(w_1\right)\hat{P}_2\left(w_2\right)\right)F_1\left(\theta_1\left(\tilde{P}_2\left(z_1\right)\hat{P}_1\left(w_1\right)\hat{P}_2\left(w_2\right)\right)\right)\hat{F}_1\left(w_1,w_2\right)\right)\\
&=&\hat{\mu}_{2}R_{1}^{(2)}\left(1\right)+r_{1}\hat{P}_{2}^{(2)}\left(1\right)+2r_{1}\hat{\mu}_{2}\hat{F}_{1}^{(0,1)}+\hat{F}_{1}^{(0,2)}+2r_{1}\frac{\hat{\mu}_{2}^{2}}{1-\mu_{1}}F_{1}^{(1,0)}+\hat{\mu}_{2}^{2}\theta_{1}^{(2)}\left(1\right)F_{1}^{(1,0)}\\
&+&\frac{1}{1-\mu_{1}}\hat{P}_{2}^{(2)}\left(1\right)F_{1}^{(1,0)} +
2\frac{\hat{\mu}_{2}}{1-\mu_{1}}F_{1}^{(1,0)}\hat{F}_{1}^{(0,1)}+\left(\frac{\hat{\mu}_{2}}{1-\mu_{1}}\right)^{2}F_{1}^{(2,0)}.
\end{eqnarray*}
\end{enumerate}

%___________________________________________________________________________________________
%
%\subsection{Derivadas de Segundo Orden para $\hat{F}_{1}$}
%___________________________________________________________________________________________


\begin{enumerate}
%___________________________________________________________________________________________
%\subsubsection{Mixtas para $z_{1}$:}
%___________________________________________________________________________________________
%1/33

\item \begin{eqnarray*} &&\frac{\partial}{\partial
z_1}\frac{\partial}{\partial
z_1}\left(\hat{R}_{2}\left(P_{1}\left(z_{1}\right)\tilde{P}_{2}\left(z_{2}\right)\hat{P}_{1}\left(w_{1}\right)\hat{P}_{2}\left(w_{2}\right)\right)\hat{F}_{2}\left(w_{1},\hat{\theta}_{2}\left(P_{1}\left(z_{1}\right)\tilde{P}_{2}\left(z_{2}\right)\hat{P}_{1}\left(w_{1}\right)\right)\right)F_{2}\left(z_{1},z_{2}\right)\right)\\
&=&\hat{r}_{2}P_{1}^{(2)}\left(1\right)+
\mu_{1}^{2}\hat{R}_{2}^{(2)}\left(1\right)+
2\hat{r}_{2}\frac{\mu_{1}^{2}}{1-\hat{\mu}_{2}}\hat{F}_{2}^{(0,1)}+
\frac{1}{1-\hat{\mu}_{2}}P_{1}^{(2)}\left(1\right)\hat{F}_{2}^{(0,1)}+
\mu_{1}^{2}\hat{\theta}_{2}^{(2)}\left(1\right)\hat{F}_{2}^{(0,1)}\\
&+&\left(\frac{\mu_{1}^{2}}{1-\hat{\mu}_{2}}\right)^{2}\hat{F}_{2}^{(0,2)}+
2\hat{r}_{2}\mu_{1}F_{2}^{(1,0)}+2\frac{\mu_{1}}{1-\hat{\mu}_{2}}\hat{F}_{2}^{(0,1)}F_{2}^{(1,0)}+F_{2}^{(2,0)}.
\end{eqnarray*}

%2/34
\item \begin{eqnarray*} &&\frac{\partial}{\partial
z_2}\frac{\partial}{\partial
z_1}\left(\hat{R}_{2}\left(P_{1}\left(z_{1}\right)\tilde{P}_{2}\left(z_{2}\right)\hat{P}_{1}\left(w_{1}\right)\hat{P}_{2}\left(w_{2}\right)\right)\hat{F}_{2}\left(w_{1},\hat{\theta}_{2}\left(P_{1}\left(z_{1}\right)\tilde{P}_{2}\left(z_{2}\right)\hat{P}_{1}\left(w_{1}\right)\right)\right)F_{2}\left(z_{1},z_{2}\right)\right)\\
&=&\hat{r}_{2}\mu_{1}\tilde{\mu}_{2}+\mu_{1}\tilde{\mu}_{2}\hat{R}_{2}^{(2)}\left(1\right)+\hat{r}_{2}\mu_{1}F_{2}^{(0,1)}+
\frac{\mu_{1}\tilde{\mu}_{2}}{1-\hat{\mu}_{2}}\hat{F}_{2}^{(0,1)}+2\hat{r}_{2}\frac{\mu_{1}\tilde{\mu}_{2}}{1-\hat{\mu}_{2}}\hat{F}_{2}^{(0,1)}+\mu_{1}\tilde{\mu}_{2}\hat{\theta}_{2}^{(2)}\left(1\right)\hat{F}_{2}^{(0,1)}\\
&+&\frac{\mu_{1}}{1-\hat{\mu}_{2}}F_{2}^{(0,1)}\hat{F}_{2}^{(0,1)}+\mu_{1} \tilde{\mu}_{2}\left(\frac{1}{1-\hat{\mu}_{2}}\right)^{2}\hat{F}_{2}^{(0,2)}+\hat{r}_{2}\tilde{\mu}_{2}F_{2}^{(1,0)}+\frac{\tilde{\mu}_{2}}{1-\hat{\mu}_{2}}\hat{F}_{2}^{(0,1)}F_{2}^{(1,0)}+F_{2}^{(1,1)}.
\end{eqnarray*}


%3/35

\item \begin{eqnarray*} &&\frac{\partial}{\partial
w_1}\frac{\partial}{\partial
z_1}\left(\hat{R}_{2}\left(P_{1}\left(z_{1}\right)\tilde{P}_{2}\left(z_{2}\right)\hat{P}_{1}\left(w_{1}\right)\hat{P}_{2}\left(w_{2}\right)\right)\hat{F}_{2}\left(w_{1},\hat{\theta}_{2}\left(P_{1}\left(z_{1}\right)\tilde{P}_{2}\left(z_{2}\right)\hat{P}_{1}\left(w_{1}\right)\right)\right)F_{2}\left(z_{1},z_{2}\right)\right)\\
&=&\hat{r}_{2}\mu_{1}\hat{\mu}_{1}+\mu_{1}\hat{\mu}_{1}\hat{R}_{2}^{(2)}\left(1\right)+\hat{r}_{2}\frac{\mu_{1}\hat{\mu}_{1}}{1-\hat{\mu}_{2}}\hat{F}_{2}^{(0,1)}+\hat{r}_{2}\hat{\mu}_{1}F_{2}^{(1,0)}+\hat{r}_{2}\mu_{1}\hat{F}_{2}^{(1,0)}+F_{2}^{(1,0)}\hat{F}_{2}^{(1,0)}+\frac{\mu_{1}}{1-\hat{\mu}_{2}}\hat{F}_{2}^{(1,1)}.
\end{eqnarray*}

%4/36

\item \begin{eqnarray*} &&\frac{\partial}{\partial
w_2}\frac{\partial}{\partial
z_1}\left(\hat{R}_{2}\left(P_{1}\left(z_{1}\right)\tilde{P}_{2}\left(z_{2}\right)\hat{P}_{1}\left(w_{1}\right)\hat{P}_{2}\left(w_{2}\right)\right)\hat{F}_{2}\left(w_{1},\hat{\theta}_{2}\left(P_{1}\left(z_{1}\right)\tilde{P}_{2}\left(z_{2}\right)\hat{P}_{1}\left(w_{1}\right)\right)\right)F_{2}\left(z_{1},z_{2}\right)\right)\\
&=&\hat{r}_{2}\mu_{1}\hat{\mu}_{2}+\mu_{1}\hat{\mu}_{2}\hat{R}_{2}^{(2)}\left(1\right)+\frac{\mu_{1}\hat{\mu}_{2}}{1-\hat{\mu}_{2}}\hat{F}_{2}^{(0,1)}+2\hat{r}_{2}\frac{\mu_{1}\hat{\mu}_{2}}{1-\hat{\mu}_{2}}\hat{F}_{2}^{(0,1)}+\mu_{1}\hat{\mu}_{2}\hat{\theta}_{2}^{(2)}\left(1\right)\hat{F}_{2}^{(0,1)}\\
&+&\mu_{1}\hat{\mu}_{2}\left(\frac{1}{1-\hat{\mu}_{2}}\right)^{2}\hat{F}_{2}^{(0,2)}+\hat{r}_{2}\hat{\mu}_{2}F_{2}^{(1,0)}+\frac{\hat{\mu}_{2}}{1-\hat{\mu}_{2}}\hat{F}_{2}^{(0,1)}F_{2}^{(1,0)}.
\end{eqnarray*}
%___________________________________________________________________________________________
%\subsubsection{Mixtas para $z_{2}$:}
%___________________________________________________________________________________________

%5/37

\item \begin{eqnarray*} &&\frac{\partial}{\partial
z_1}\frac{\partial}{\partial
z_2}\left(\hat{R}_{2}\left(P_{1}\left(z_{1}\right)\tilde{P}_{2}\left(z_{2}\right)\hat{P}_{1}\left(w_{1}\right)\hat{P}_{2}\left(w_{2}\right)\right)\hat{F}_{2}\left(w_{1},\hat{\theta}_{2}\left(P_{1}\left(z_{1}\right)\tilde{P}_{2}\left(z_{2}\right)\hat{P}_{1}\left(w_{1}\right)\right)\right)F_{2}\left(z_{1},z_{2}\right)\right)\\
&=&\hat{r}_{2}\mu_{1}\tilde{\mu}_{2}+\mu_{1}\tilde{\mu}_{2}\hat{R}_{2}^{(2)}\left(1\right)+\mu_{1}\hat{r}_{2}F_{2}^{(0,1)}+
\frac{\mu_{1}\tilde{\mu}_{2}}{1-\hat{\mu}_{2}}\hat{F}_{2}^{(0,1)}+2\hat{r}_{2}\frac{\mu_{1}\tilde{\mu}_{2}}{1-\hat{\mu}_{2}}\hat{F}_{2}^{(0,1)}+\mu_{1}\tilde{\mu}_{2}\hat{\theta}_{2}^{(2)}\left(1\right)\hat{F}_{2}^{(0,1)}\\
&+&\frac{\mu_{1}}{1-\hat{\mu}_{2}}F_{2}^{(0,1)}\hat{F}_{2}^{(0,1)}+\mu_{1}\tilde{\mu}_{2}\left(\frac{1}{1-\hat{\mu}_{2}}\right)^{2}\hat{F}_{2}^{(0,2)}+\hat{r}_{2}\tilde{\mu}_{2}F_{2}^{(1,0)}+\frac{\tilde{\mu}_{2}}{1-\hat{\mu}_{2}}\hat{F}_{2}^{(0,1)}F_{2}^{(1,0)}+F_{2}^{(1,1)}.
\end{eqnarray*}

%6/38

\item \begin{eqnarray*} &&\frac{\partial}{\partial
z_2}\frac{\partial}{\partial
z_2}\left(\hat{R}_{2}\left(P_{1}\left(z_{1}\right)\tilde{P}_{2}\left(z_{2}\right)\hat{P}_{1}\left(w_{1}\right)\hat{P}_{2}\left(w_{2}\right)\right)\hat{F}_{2}\left(w_{1},\hat{\theta}_{2}\left(P_{1}\left(z_{1}\right)\tilde{P}_{2}\left(z_{2}\right)\hat{P}_{1}\left(w_{1}\right)\right)\right)F_{2}\left(z_{1},z_{2}\right)\right)\\
&=&\hat{r}_{2}\tilde{P}_{2}^{(2)}\left(1\right)+\tilde{\mu}_{2}^{2}\hat{R}_{2}^{(2)}\left(1\right)+2\hat{r}_{2}\tilde{\mu}_{2}F_{2}^{(0,1)}+2\hat{r}_{2}\frac{\tilde{\mu}_{2}^{2}}{1-\hat{\mu}_{2}}\hat{F}_{2}^{(0,1)}+\frac{1}{1-\hat{\mu}_{2}}\tilde{P}_{2}^{(2)}\left(1\right)\hat{F}_{2}^{(0,1)}\\
&+&\tilde{\mu}_{2}^{2}\hat{\theta}_{2}^{(2)}\left(1\right)\hat{F}_{2}^{(0,1)}+2\frac{\tilde{\mu}_{2}}{1-\hat{\mu}_{2}}F_{2}^{(0,1)}\hat{F}_{2}^{(0,1)}+F_{2}^{(0,2)}+\left(\frac{\tilde{\mu}_{2}}{1-\hat{\mu}_{2}}\right)^{2}\hat{F}_{2}^{(0,2)}.
\end{eqnarray*}

%7/39

\item \begin{eqnarray*} &&\frac{\partial}{\partial
w_1}\frac{\partial}{\partial
z_2}\left(\hat{R}_{2}\left(P_{1}\left(z_{1}\right)\tilde{P}_{2}\left(z_{2}\right)\hat{P}_{1}\left(w_{1}\right)\hat{P}_{2}\left(w_{2}\right)\right)\hat{F}_{2}\left(w_{1},\hat{\theta}_{2}\left(P_{1}\left(z_{1}\right)\tilde{P}_{2}\left(z_{2}\right)\hat{P}_{1}\left(w_{1}\right)\right)\right)F_{2}\left(z_{1},z_{2}\right)\right)\\
&=&\hat{r}_{2}\tilde{\mu}_{2}\hat{\mu}_{1}+\tilde{\mu}_{2}\hat{\mu}_{1}\hat{R}_{2}^{(2)}\left(1\right)+\hat{r}_{2}\hat{\mu}_{1}F_{2}^{(0,1)}+\hat{r}_{2}\frac{\tilde{\mu}_{2}\hat{\mu}_{1}}{1-\hat{\mu}_{2}}\hat{F}_{2}^{(0,1)}+\hat{r}_{2}\tilde{\mu}_{2}\hat{F}_{2}^{(1,0)}+F_{2}^{(0,1)}\hat{F}_{2}^{(1,0)}+\frac{\tilde{\mu}_{2}}{1-\hat{\mu}_{2}}\hat{F}_{2}^{(1,1)}.
\end{eqnarray*}
%8/40

\item \begin{eqnarray*} &&\frac{\partial}{\partial
w_2}\frac{\partial}{\partial
z_2}\left(\hat{R}_{2}\left(P_{1}\left(z_{1}\right)\tilde{P}_{2}\left(z_{2}\right)\hat{P}_{1}\left(w_{1}\right)\hat{P}_{2}\left(w_{2}\right)\right)\hat{F}_{2}\left(w_{1},\hat{\theta}_{2}\left(P_{1}\left(z_{1}\right)\tilde{P}_{2}\left(z_{2}\right)\hat{P}_{1}\left(w_{1}\right)\right)\right)F_{2}\left(z_{1},z_{2}\right)\right)\\
&=&\hat{r}_{2}\tilde{\mu}_{2}\hat{\mu}_{2}+\tilde{\mu}_{2}\hat{\mu}_{2}\hat{R}_{2}^{(2)}\left(1\right)+\hat{r}_{2}\hat{\mu}_{2}F_{2}^{(0,1)}+
\frac{\tilde{\mu}_{2}\hat{\mu}_{2}}{1-\hat{\mu}_{2}}\hat{F}_{2}^{(0,1)}+2\hat{r}_{2}\frac{\tilde{\mu}_{2}\hat{\mu}_{2}}{1-\hat{\mu}_{2}}\hat{F}_{2}^{(0,1)}+\tilde{\mu}_{2}\hat{\mu}_{2}\hat{\theta}_{2}^{(2)}\left(1\right)\hat{F}_{2}^{(0,1)}\\
&+&\frac{\hat{\mu}_{2}}{1-\hat{\mu}_{2}}F_{2}^{(0,1)}\hat{F}_{2}^{(1,0)}+\tilde{\mu}_{2}\hat{\mu}_{2}\left(\frac{1}{1-\hat{\mu}_{2}}\right)\hat{F}_{2}^{(0,2)}.
\end{eqnarray*}
%___________________________________________________________________________________________
%\subsubsection{Mixtas para $w_{1}$:}
%___________________________________________________________________________________________

%9/41
\item \begin{eqnarray*} &&\frac{\partial}{\partial
z_1}\frac{\partial}{\partial
w_1}\left(\hat{R}_{2}\left(P_{1}\left(z_{1}\right)\tilde{P}_{2}\left(z_{2}\right)\hat{P}_{1}\left(w_{1}\right)\hat{P}_{2}\left(w_{2}\right)\right)\hat{F}_{2}\left(w_{1},\hat{\theta}_{2}\left(P_{1}\left(z_{1}\right)\tilde{P}_{2}\left(z_{2}\right)\hat{P}_{1}\left(w_{1}\right)\right)\right)F_{2}\left(z_{1},z_{2}\right)\right)\\
&=&\hat{r}_{2}\mu_{1}\hat{\mu}_{1}+\mu_{1}\hat{\mu}_{1}\hat{R}_{2}^{(2)}\left(1\right)+\hat{r}_{2}\frac{\mu_{1}\hat{\mu}_{1}}{1-\hat{\mu}_{2}}\hat{F}_{2}^{(0,1)}+\hat{r}_{2}\hat{\mu}_{1}F_{2}^{(1,0)}+\hat{r}_{2}\mu_{1}\hat{F}_{2}^{(1,0)}+F_{2}^{(1,0)}\hat{F}_{2}^{(1,0)}+\frac{\mu_{1}}{1-\hat{\mu}_{2}}\hat{F}_{2}^{(1,1)}.
\end{eqnarray*}


%10/42
\item \begin{eqnarray*} &&\frac{\partial}{\partial
z_2}\frac{\partial}{\partial
w_1}\left(\hat{R}_{2}\left(P_{1}\left(z_{1}\right)\tilde{P}_{2}\left(z_{2}\right)\hat{P}_{1}\left(w_{1}\right)\hat{P}_{2}\left(w_{2}\right)\right)\hat{F}_{2}\left(w_{1},\hat{\theta}_{2}\left(P_{1}\left(z_{1}\right)\tilde{P}_{2}\left(z_{2}\right)\hat{P}_{1}\left(w_{1}\right)\right)\right)F_{2}\left(z_{1},z_{2}\right)\right)\\
&=&\hat{r}_{2}\tilde{\mu}_{2}\hat{\mu}_{1}+\tilde{\mu}_{2}\hat{\mu}_{1}\hat{R}_{2}^{(2)}\left(1\right)+\hat{r}_{2}\hat{\mu}_{1}F_{2}^{(0,1)}+
\hat{r}_{2}\frac{\tilde{\mu}_{2}\hat{\mu}_{1}}{1-\hat{\mu}_{2}}\hat{F}_{2}^{(0,1)}+\hat{r}_{2}\tilde{\mu}_{2}\hat{F}_{2}^{(1,0)}+F_{2}^{(0,1)}\hat{F}_{2}^{(1,0)}+\frac{\tilde{\mu}_{2}}{1-\hat{\mu}_{2}}\hat{F}_{2}^{(1,1)}.
\end{eqnarray*}


%11/43
\item \begin{eqnarray*} &&\frac{\partial}{\partial
w_1}\frac{\partial}{\partial
w_1}\left(\hat{R}_{2}\left(P_{1}\left(z_{1}\right)\tilde{P}_{2}\left(z_{2}\right)\hat{P}_{1}\left(w_{1}\right)\hat{P}_{2}\left(w_{2}\right)\right)\hat{F}_{2}\left(w_{1},\hat{\theta}_{2}\left(P_{1}\left(z_{1}\right)\tilde{P}_{2}\left(z_{2}\right)\hat{P}_{1}\left(w_{1}\right)\right)\right)F_{2}\left(z_{1},z_{2}\right)\right)\\
&=&\hat{r}_{2}\hat{P}_{1}^{(2)}\left(1\right)+\hat{\mu}_{1}^{2}\hat{R}_{2}^{(2)}\left(1\right)+2\hat{r}_{2}\hat{\mu}_{1}\hat{F}_{2}^{(1,0)}
+\hat{F}_{2}^{(2,0)}.
\end{eqnarray*}


%12/44
\item \begin{eqnarray*} &&\frac{\partial}{\partial
w_2}\frac{\partial}{\partial
w_1}\left(\hat{R}_{2}\left(P_{1}\left(z_{1}\right)\tilde{P}_{2}\left(z_{2}\right)\hat{P}_{1}\left(w_{1}\right)\hat{P}_{2}\left(w_{2}\right)\right)\hat{F}_{2}\left(w_{1},\hat{\theta}_{2}\left(P_{1}\left(z_{1}\right)\tilde{P}_{2}\left(z_{2}\right)\hat{P}_{1}\left(w_{1}\right)\right)\right)F_{2}\left(z_{1},z_{2}\right)\right)\\
&=&\hat{r}_{2}\hat{\mu}_{1}\hat{\mu}_{2}+\hat{\mu}_{1}\hat{\mu}_{2}\hat{R}_{2}^{(2)}\left(1\right)+
\hat{r}_{2}\frac{\hat{\mu}_{2}\hat{\mu}_{1}}{1-\hat{\mu}_{2}}\hat{F}_{2}^{(0,1)}
+\hat{r}_{2}\hat{\mu}_{2}\hat{F}_{2}^{(1,0)}+\frac{\hat{\mu}_{2}}{1-\hat{\mu}_{2}}\hat{F}_{2}^{(1,1)}.
\end{eqnarray*}
%___________________________________________________________________________________________
%\subsubsection{Mixtas para $w_{2}$:}
%___________________________________________________________________________________________
%13/45
\item \begin{eqnarray*} &&\frac{\partial}{\partial
z_1}\frac{\partial}{\partial
w_2}\left(\hat{R}_{2}\left(P_{1}\left(z_{1}\right)\tilde{P}_{2}\left(z_{2}\right)\hat{P}_{1}\left(w_{1}\right)\hat{P}_{2}\left(w_{2}\right)\right)\hat{F}_{2}\left(w_{1},\hat{\theta}_{2}\left(P_{1}\left(z_{1}\right)\tilde{P}_{2}\left(z_{2}\right)\hat{P}_{1}\left(w_{1}\right)\right)\right)F_{2}\left(z_{1},z_{2}\right)\right)\\
&=&\hat{r}_{2}\mu_{1}\hat{\mu}_{2}+\mu_{1}\hat{\mu}_{2}\hat{R}_{2}^{(2)}\left(1\right)+
\frac{\mu_{1}\hat{\mu}_{2}}{1-\hat{\mu}_{2}}\hat{F}_{2}^{(0,1)} +2\hat{r}_{2}\frac{\mu_{1}\hat{\mu}_{2}}{1-\hat{\mu}_{2}}\hat{F}_{2}^{(0,1)}\\
&+&\mu_{1}\hat{\mu}_{2}\hat{\theta}_{2}^{(2)}\left(1\right)\hat{F}_{2}^{(0,1)}+\mu_{1}\hat{\mu}_{2}\left(\frac{1}{1-\hat{\mu}_{2}}\right)^{2}\hat{F}_{2}^{(0,2)}+\hat{r}_{2}\hat{\mu}_{2}F_{2}^{(1,0)}+\frac{\hat{\mu}_{2}}{1-\hat{\mu}_{2}}\hat{F}_{2}^{(0,1)}F_{2}^{(1,0)}.\end{eqnarray*}


%14/46
\item \begin{eqnarray*} &&\frac{\partial}{\partial
z_2}\frac{\partial}{\partial
w_2}\left(\hat{R}_{2}\left(P_{1}\left(z_{1}\right)\tilde{P}_{2}\left(z_{2}\right)\hat{P}_{1}\left(w_{1}\right)\hat{P}_{2}\left(w_{2}\right)\right)\hat{F}_{2}\left(w_{1},\hat{\theta}_{2}\left(P_{1}\left(z_{1}\right)\tilde{P}_{2}\left(z_{2}\right)\hat{P}_{1}\left(w_{1}\right)\right)\right)F_{2}\left(z_{1},z_{2}\right)\right)\\
&=&\hat{r}_{2}\tilde{\mu}_{2}\hat{\mu}_{2}+\tilde{\mu}_{2}\hat{\mu}_{2}\hat{R}_{2}^{(2)}\left(1\right)+\hat{r}_{2}\hat{\mu}_{2}F_{2}^{(0,1)}+\frac{\tilde{\mu}_{2}\hat{\mu}_{2}}{1-\hat{\mu}_{2}}\hat{F}_{2}^{(0,1)}+
2\hat{r}_{2}\frac{\tilde{\mu}_{2}\hat{\mu}_{2}}{1-\hat{\mu}_{2}}\hat{F}_{2}^{(0,1)}+\tilde{\mu}_{2}\hat{\mu}_{2}\hat{\theta}_{2}^{(2)}\left(1\right)\hat{F}_{2}^{(0,1)}\\
&+&\frac{\hat{\mu}_{2}}{1-\hat{\mu}_{2}}\hat{F}_{2}^{(0,1)}F_{2}^{(0,1)}+\tilde{\mu}_{2}\hat{\mu}_{2}\left(\frac{1}{1-\hat{\mu}_{2}}\right)^{2}\hat{F}_{2}^{(0,2)}.
\end{eqnarray*}

%15/47

\item \begin{eqnarray*} &&\frac{\partial}{\partial
w_1}\frac{\partial}{\partial
w_2}\left(\hat{R}_{2}\left(P_{1}\left(z_{1}\right)\tilde{P}_{2}\left(z_{2}\right)\hat{P}_{1}\left(w_{1}\right)\hat{P}_{2}\left(w_{2}\right)\right)\hat{F}_{2}\left(w_{1},\hat{\theta}_{2}\left(P_{1}\left(z_{1}\right)\tilde{P}_{2}\left(z_{2}\right)\hat{P}_{1}\left(w_{1}\right)\right)\right)F_{2}\left(z_{1},z_{2}\right)\right)\\
&=&\hat{r}_{2}\hat{\mu}_{1}\hat{\mu}_{2}+\hat{\mu}_{1}\hat{\mu}_{2}\hat{R}_{2}^{(2)}\left(1\right)+
\hat{r}_{2}\frac{\hat{\mu}_{1}\hat{\mu}_{2}}{1-\hat{\mu}_{2}}\hat{F}_{2}^{(0,1)}+
\hat{r}_{2}\hat{\mu}_{2}\hat{F}_{2}^{(1,0)}+\frac{\hat{\mu}_{2}}{1-\hat{\mu}_{2}}\hat{F}_{2}^{(1,1)}.
\end{eqnarray*}

%16/48
\item \begin{eqnarray*} &&\frac{\partial}{\partial
w_2}\frac{\partial}{\partial
w_2}\left(\hat{R}_{2}\left(P_{1}\left(z_{1}\right)\tilde{P}_{2}\left(z_{2}\right)\hat{P}_{1}\left(w_{1}\right)\hat{P}_{2}\left(w_{2}\right)\right)\hat{F}_{2}\left(w_{1},\hat{\theta}_{2}\left(P_{1}\left(z_{1}\right)\tilde{P}_{2}\left(z_{2}\right)\hat{P}_{1}\left(w_{1}\right)\right)\right)F_{2}\left(z_{1},z_{2};\zeta_{2}\right)\right)\\
&=&\hat{r}_{2}P_{2}^{(2)}\left(1\right)+\hat{\mu}_{2}^{2}\hat{R}_{2}^{(2)}\left(1\right)+2\hat{r}_{2}\frac{\hat{\mu}_{2}^{2}}{1-\hat{\mu}_{2}}\hat{F}_{2}^{(0,1)}+\frac{1}{1-\hat{\mu}_{2}}\hat{P}_{2}^{(2)}\left(1\right)\hat{F}_{2}^{(0,1)}+\hat{\mu}_{2}^{2}\hat{\theta}_{2}^{(2)}\left(1\right)\hat{F}_{2}^{(0,1)}\\
&+&\left(\frac{\hat{\mu}_{2}}{1-\hat{\mu}_{2}}\right)^{2}\hat{F}_{2}^{(0,2)}.
\end{eqnarray*}


\end{enumerate}



%___________________________________________________________________________________________
%
%\subsection{Derivadas de Segundo Orden para $\hat{F}_{2}$}
%___________________________________________________________________________________________
\begin{enumerate}
%___________________________________________________________________________________________
%\subsubsection{Mixtas para $z_{1}$:}
%___________________________________________________________________________________________
%1/49

\item \begin{eqnarray*} &&\frac{\partial}{\partial
z_1}\frac{\partial}{\partial
z_1}\left(\hat{R}_{1}\left(P_{1}\left(z_{1}\right)\tilde{P}_{2}\left(z_{2}\right)\hat{P}_{1}\left(w_{1}\right)\hat{P}_{2}\left(w_{2}\right)\right)\hat{F}_{1}\left(\hat{\theta}_{1}\left(P_{1}\left(z_{1}\right)\tilde{P}_{2}\left(z_{2}\right)
\hat{P}_{2}\left(w_{2}\right)\right),w_{2}\right)F_{1}\left(z_{1},z_{2}\right)\right)\\
&=&\hat{r}_{1}P_{1}^{(2)}\left(1\right)+
\mu_{1}^{2}\hat{R}_{1}^{(2)}\left(1\right)+
2\hat{r}_{1}\mu_{1}F_{1}^{(1,0)}+
2\hat{r}_{1}\frac{\mu_{1}^{2}}{1-\hat{\mu}_{1}}\hat{F}_{1}^{(1,0)}+
\frac{1}{1-\hat{\mu}_{1}}P_{1}^{(2)}\left(1\right)\hat{F}_{1}^{(1,0)}+\mu_{1}^{2}\hat{\theta}_{1}^{(2)}\left(1\right)\hat{F}_{1}^{(1,0)}\\
&+&2\frac{\mu_{1}}{1-\hat{\mu}_{1}}\hat{F}_{1}^{(1,0)}F_{1}^{(1,0)}+F_{1}^{(2,0)}
+\left(\frac{\mu_{1}}{1-\hat{\mu}_{1}}\right)^{2}\hat{F}_{1}^{(2,0)}.
\end{eqnarray*}

%2/50

\item \begin{eqnarray*} &&\frac{\partial}{\partial
z_2}\frac{\partial}{\partial
z_1}\left(\hat{R}_{1}\left(P_{1}\left(z_{1}\right)\tilde{P}_{2}\left(z_{2}\right)\hat{P}_{1}\left(w_{1}\right)\hat{P}_{2}\left(w_{2}\right)\right)\hat{F}_{1}\left(\hat{\theta}_{1}\left(P_{1}\left(z_{1}\right)\tilde{P}_{2}\left(z_{2}\right)
\hat{P}_{2}\left(w_{2}\right)\right),w_{2}\right)F_{1}\left(z_{1},z_{2}\right)\right)\\
&=&\hat{r}_{1}\mu_{1}\tilde{\mu}_{2}+\mu_{1}\tilde{\mu}_{2}\hat{R}_{1}^{(2)}\left(1\right)+
\hat{r}_{1}\mu_{1}F_{1}^{(0,1)}+\tilde{\mu}_{2}\hat{r}_{1}F_{1}^{(1,0)}+
\frac{\mu_{1}\tilde{\mu}_{2}}{1-\hat{\mu}_{1}}\hat{F}_{1}^{(1,0)}+2\hat{r}_{1}\frac{\mu_{1}\tilde{\mu}_{2}}{1-\hat{\mu}_{1}}\hat{F}_{1}^{(1,0)}\\
&+&\mu_{1}\tilde{\mu}_{2}\hat{\theta}_{1}^{(2)}\left(1\right)\hat{F}_{1}^{(1,0)}+
\frac{\mu_{1}}{1-\hat{\mu}_{1}}\hat{F}_{1}^{(1,0)}F_{1}^{(0,1)}+
\frac{\tilde{\mu}_{2}}{1-\hat{\mu}_{1}}\hat{F}_{1}^{(1,0)}F_{1}^{(1,0)}+
F_{1}^{(1,1)}\\
&+&\mu_{1}\tilde{\mu}_{2}\left(\frac{1}{1-\hat{\mu}_{1}}\right)^{2}\hat{F}_{1}^{(2,0)}.
\end{eqnarray*}

%3/51

\item \begin{eqnarray*} &&\frac{\partial}{\partial
w_1}\frac{\partial}{\partial
z_1}\left(\hat{R}_{1}\left(P_{1}\left(z_{1}\right)\tilde{P}_{2}\left(z_{2}\right)\hat{P}_{1}\left(w_{1}\right)\hat{P}_{2}\left(w_{2}\right)\right)\hat{F}_{1}\left(\hat{\theta}_{1}\left(P_{1}\left(z_{1}\right)\tilde{P}_{2}\left(z_{2}\right)
\hat{P}_{2}\left(w_{2}\right)\right),w_{2}\right)F_{1}\left(z_{1},z_{2}\right)\right)\\
&=&\hat{r}_{1}\mu_{1}\hat{\mu}_{1}+\mu_{1}\hat{\mu}_{1}\hat{R}_{1}^{(2)}\left(1\right)+\hat{r}_{1}\hat{\mu}_{1}F_{1}^{(1,0)}+
\hat{r}_{1}\frac{\mu_{1}\hat{\mu}_{1}}{1-\hat{\mu}_{1}}\hat{F}_{1}^{(1,0)}.
\end{eqnarray*}

%4/52

\item \begin{eqnarray*} &&\frac{\partial}{\partial
w_2}\frac{\partial}{\partial
z_1}\left(\hat{R}_{1}\left(P_{1}\left(z_{1}\right)\tilde{P}_{2}\left(z_{2}\right)\hat{P}_{1}\left(w_{1}\right)\hat{P}_{2}\left(w_{2}\right)\right)\hat{F}_{1}\left(\hat{\theta}_{1}\left(P_{1}\left(z_{1}\right)\tilde{P}_{2}\left(z_{2}\right)
\hat{P}_{2}\left(w_{2}\right)\right),w_{2}\right)F_{1}\left(z_{1},z_{2}\right)\right)\\
&=&\hat{r}_{1}\mu_{1}\hat{\mu}_{2}+\mu_{1}\hat{\mu}_{2}\hat{R}_{1}^{(2)}\left(1\right)+\hat{r}_{1}\hat{\mu}_{2}F_{1}^{(1,0)}+\frac{\mu_{1}\hat{\mu}_{2}}{1-\hat{\mu}_{1}}\hat{F}_{1}^{(1,0)}+\hat{r}_{1}\frac{\mu_{1}\hat{\mu}_{2}}{1-\hat{\mu}_{1}}\hat{F}_{1}^{(1,0)}+\mu_{1}\hat{\mu}_{2}\hat{\theta}_{1}^{(2)}\left(1\right)\hat{F}_{1}^{(1,0)}\\
&+&\hat{r}_{1}\mu_{1}\left(\hat{F}_{1}^{(0,1)}+\frac{\hat{\mu}_{2}}{1-\hat{\mu}_{1}}\hat{F}_{1}^{(1,0)}\right)+F_{1}^{(1,0)}\left(\hat{F}_{1}^{(0,1)}+\frac{\hat{\mu}_{2}}{1-\hat{\mu}_{1}}\hat{F}_{1}^{(1,0)}\right)+\frac{\mu_{1}}{1-\hat{\mu}_{1}}\left(\hat{F}_{1}^{(1,1)}+\frac{\hat{\mu}_{2}}{1-\hat{\mu}_{1}}\hat{F}_{1}^{(2,0)}\right).
\end{eqnarray*}
%___________________________________________________________________________________________
%\subsubsection{Mixtas para $z_{2}$:}
%___________________________________________________________________________________________
%5/53

\item \begin{eqnarray*} &&\frac{\partial}{\partial
z_1}\frac{\partial}{\partial
z_2}\left(\hat{R}_{1}\left(P_{1}\left(z_{1}\right)\tilde{P}_{2}\left(z_{2}\right)\hat{P}_{1}\left(w_{1}\right)\hat{P}_{2}\left(w_{2}\right)\right)\hat{F}_{1}\left(\hat{\theta}_{1}\left(P_{1}\left(z_{1}\right)\tilde{P}_{2}\left(z_{2}\right)
\hat{P}_{2}\left(w_{2}\right)\right),w_{2}\right)F_{1}\left(z_{1},z_{2}\right)\right)\\
&=&\hat{r}_{1}\mu_{1}\tilde{\mu}_{2}+\mu_{1}\tilde{\mu}_{2}\hat{R}_{1}^{(2)}\left(1\right)+\hat{r}_{1}\mu_{1}F_{1}^{(0,1)}+\hat{r}_{1}\tilde{\mu}_{2}F_{1}^{(1,0)}+\frac{\mu_{1}\tilde{\mu}_{2}}{1-\hat{\mu}_{1}}\hat{F}_{1}^{(1,0)}+2\hat{r}_{1}\frac{\mu_{1}\tilde{\mu}_{2}}{1-\hat{\mu}_{1}}\hat{F}_{1}^{(1,0)}\\
&+&\mu_{1}\tilde{\mu}_{2}\hat{\theta}_{1}^{(2)}\left(1\right)\hat{F}_{1}^{(1,0)}+\frac{\mu_{1}}{1-\hat{\mu}_{1}}\hat{F}_{1}^{(1,0)}F_{1}^{(0,1)}+\frac{\tilde{\mu}_{2}}{1-\hat{\mu}_{1}}\hat{F}_{1}^{(1,0)}F_{1}^{(1,0)}+F_{1}^{(1,1)}+\mu_{1}\tilde{\mu}_{2}\left(\frac{1}{1-\hat{\mu}_{1}}\right)^{2}\hat{F}_{1}^{(2,0)}.
\end{eqnarray*}

%6/54
\item \begin{eqnarray*} &&\frac{\partial}{\partial
z_2}\frac{\partial}{\partial
z_2}\left(\hat{R}_{1}\left(P_{1}\left(z_{1}\right)\tilde{P}_{2}\left(z_{2}\right)\hat{P}_{1}\left(w_{1}\right)\hat{P}_{2}\left(w_{2}\right)\right)\hat{F}_{1}\left(\hat{\theta}_{1}\left(P_{1}\left(z_{1}\right)\tilde{P}_{2}\left(z_{2}\right)
\hat{P}_{2}\left(w_{2}\right)\right),w_{2}\right)F_{1}\left(z_{1},z_{2}\right)\right)\\
&=&\hat{r}_{1}\tilde{P}_{2}^{(2)}\left(1\right)+\tilde{\mu}_{2}^{2}\hat{R}_{1}^{(2)}\left(1\right)+2\hat{r}_{1}\tilde{\mu}_{2}F_{1}^{(0,1)}+ F_{1}^{(0,2)}+2\hat{r}_{1}\frac{\tilde{\mu}_{2}^{2}}{1-\hat{\mu}_{1}}\hat{F}_{1}^{(1,0)}+\frac{1}{1-\hat{\mu}_{1}}\tilde{P}_{2}^{(2)}\left(1\right)\hat{F}_{1}^{(1,0)}\\
&+&\tilde{\mu}_{2}^{2}\hat{\theta}_{1}^{(2)}\left(1\right)\hat{F}_{1}^{(1,0)}+2\frac{\tilde{\mu}_{2}}{1-\hat{\mu}_{1}}F^{(0,1)}\hat{F}_{1}^{(1,0)}+\left(\frac{\tilde{\mu}_{2}}{1-\hat{\mu}_{1}}\right)^{2}\hat{F}_{1}^{(2,0)}.
\end{eqnarray*}
%7/55

\item \begin{eqnarray*} &&\frac{\partial}{\partial
w_1}\frac{\partial}{\partial
z_2}\left(\hat{R}_{1}\left(P_{1}\left(z_{1}\right)\tilde{P}_{2}\left(z_{2}\right)\hat{P}_{1}\left(w_{1}\right)\hat{P}_{2}\left(w_{2}\right)\right)\hat{F}_{1}\left(\hat{\theta}_{1}\left(P_{1}\left(z_{1}\right)\tilde{P}_{2}\left(z_{2}\right)
\hat{P}_{2}\left(w_{2}\right)\right),w_{2}\right)F_{1}\left(z_{1},z_{2}\right)\right)\\
&=&\hat{r}_{1}\hat{\mu}_{1}\tilde{\mu}_{2}+\hat{\mu}_{1}\tilde{\mu}_{2}\hat{R}_{1}^{(2)}\left(1\right)+
\hat{r}_{1}\hat{\mu}_{1}F_{1}^{(0,1)}+\hat{r}_{1}\frac{\hat{\mu}_{1}\tilde{\mu}_{2}}{1-\hat{\mu}_{1}}\hat{F}_{1}^{(1,0)}.
\end{eqnarray*}
%8/56

\item \begin{eqnarray*} &&\frac{\partial}{\partial
w_2}\frac{\partial}{\partial
z_2}\left(\hat{R}_{1}\left(P_{1}\left(z_{1}\right)\tilde{P}_{2}\left(z_{2}\right)\hat{P}_{1}\left(w_{1}\right)\hat{P}_{2}\left(w_{2}\right)\right)\hat{F}_{1}\left(\hat{\theta}_{1}\left(P_{1}\left(z_{1}\right)\tilde{P}_{2}\left(z_{2}\right)
\hat{P}_{2}\left(w_{2}\right)\right),w_{2}\right)F_{1}\left(z_{1},z_{2}\right)\right)\\
&=&\hat{r}_{1}\tilde{\mu}_{2}\hat{\mu}_{2}+\hat{\mu}_{2}\tilde{\mu}_{2}\hat{R}_{1}^{(2)}\left(1\right)+\hat{\mu}_{2}\hat{R}_{1}^{(2)}\left(1\right)F_{1}^{(0,1)}+\frac{\hat{\mu}_{2}\tilde{\mu}_{2}}{1-\hat{\mu}_{1}}\hat{F}_{1}^{(1,0)}+
\hat{r}_{1}\frac{\hat{\mu}_{2}\tilde{\mu}_{2}}{1-\hat{\mu}_{1}}\hat{F}_{1}^{(1,0)}\\
&+&\hat{\mu}_{2}\tilde{\mu}_{2}\hat{\theta}_{1}^{(2)}\left(1\right)\hat{F}_{1}^{(1,0)}+\hat{r}_{1}\tilde{\mu}_{2}\left(\hat{F}_{1}^{(0,1)}+\frac{\hat{\mu}_{2}}{1-\hat{\mu}_{1}}\hat{F}_{1}^{(1,0)}\right)+F_{1}^{(0,1)}\left(\hat{F}_{1}^{(0,1)}+\frac{\hat{\mu}_{2}}{1-\hat{\mu}_{1}}\hat{F}_{1}^{(1,0)}\right)\\
&+&\frac{\tilde{\mu}_{2}}{1-\hat{\mu}_{1}}\left(\hat{F}_{1}^{(1,1)}+\frac{\hat{\mu}_{2}}{1-\hat{\mu}_{1}}\hat{F}_{1}^{(2,0)}\right).
\end{eqnarray*}
%___________________________________________________________________________________________
%\subsubsection{Mixtas para $w_{1}$:}
%___________________________________________________________________________________________
%9/57
\item \begin{eqnarray*} &&\frac{\partial}{\partial
z_1}\frac{\partial}{\partial
w_1}\left(\hat{R}_{1}\left(P_{1}\left(z_{1}\right)\tilde{P}_{2}\left(z_{2}\right)\hat{P}_{1}\left(w_{1}\right)\hat{P}_{2}\left(w_{2}\right)\right)\hat{F}_{1}\left(\hat{\theta}_{1}\left(P_{1}\left(z_{1}\right)\tilde{P}_{2}\left(z_{2}\right)
\hat{P}_{2}\left(w_{2}\right)\right),w_{2}\right)F_{1}\left(z_{1},z_{2}\right)\right)\\
&=&\hat{r}_{1}\mu_{1}\hat{\mu}_{1}+\mu_{1}\hat{\mu}_{1}\hat{R}_{1}^{(2)}\left(1\right)+\hat{r}_{1}\hat{\mu}_{1}F_{1}^{(1,0)}+\hat{r}_{1}\frac{\mu_{1}\hat{\mu}_{1}}{1-\hat{\mu}_{1}}\hat{F}_{1}^{(1,0)}.
\end{eqnarray*}
%10/58
\item \begin{eqnarray*} &&\frac{\partial}{\partial
z_2}\frac{\partial}{\partial
w_1}\left(\hat{R}_{1}\left(P_{1}\left(z_{1}\right)\tilde{P}_{2}\left(z_{2}\right)\hat{P}_{1}\left(w_{1}\right)\hat{P}_{2}\left(w_{2}\right)\right)\hat{F}_{1}\left(\hat{\theta}_{1}\left(P_{1}\left(z_{1}\right)\tilde{P}_{2}\left(z_{2}\right)
\hat{P}_{2}\left(w_{2}\right)\right),w_{2}\right)F_{1}\left(z_{1},z_{2}\right)\right)\\
&=&\hat{r}_{1}\tilde{\mu}_{2}\hat{\mu}_{1}+\tilde{\mu}_{2}\hat{\mu}_{1}\hat{R}_{1}^{(2)}\left(1\right)+\hat{r}_{1}\hat{\mu}_{1}F_{1}^{(0,1)}+\hat{r}_{1}\frac{\tilde{\mu}_{2}\hat{\mu}_{1}}{1-\hat{\mu}_{1}}\hat{F}_{1}^{(1,0)}.
\end{eqnarray*}
%11/59
\item \begin{eqnarray*} &&\frac{\partial}{\partial
w_1}\frac{\partial}{\partial
w_1}\left(\hat{R}_{1}\left(P_{1}\left(z_{1}\right)\tilde{P}_{2}\left(z_{2}\right)\hat{P}_{1}\left(w_{1}\right)\hat{P}_{2}\left(w_{2}\right)\right)\hat{F}_{1}\left(\hat{\theta}_{1}\left(P_{1}\left(z_{1}\right)\tilde{P}_{2}\left(z_{2}\right)
\hat{P}_{2}\left(w_{2}\right)\right),w_{2}\right)F_{1}\left(z_{1},z_{2}\right)\right)\\
&=&\hat{r}_{1}\hat{P}_{1}^{(2)}\left(1\right)+\hat{\mu}_{1}^{2}\hat{R}_{1}^{(2)}\left(1\right).
\end{eqnarray*}
%12/60
\item \begin{eqnarray*} &&\frac{\partial}{\partial
w_2}\frac{\partial}{\partial
w_1}\left(\hat{R}_{1}\left(P_{1}\left(z_{1}\right)\tilde{P}_{2}\left(z_{2}\right)\hat{P}_{1}\left(w_{1}\right)\hat{P}_{2}\left(w_{2}\right)\right)\hat{F}_{1}\left(\hat{\theta}_{1}\left(P_{1}\left(z_{1}\right)\tilde{P}_{2}\left(z_{2}\right)
\hat{P}_{2}\left(w_{2}\right)\right),w_{2}\right)F_{1}\left(z_{1},z_{2}\right)\right)\\
&=&\hat{r}_{1}\hat{\mu}_{2}\hat{\mu}_{1}+\hat{\mu}_{2}\hat{\mu}_{1}\hat{R}_{1}^{(2)}\left(1\right)+\hat{r}_{1}\hat{\mu}_{1}\left(\hat{F}_{1}^{(0,1)}+\frac{\hat{\mu}_{2}}{1-\hat{\mu}_{1}}\hat{F}_{1}^{(1,0)}\right).
\end{eqnarray*}
%___________________________________________________________________________________________
%\subsubsection{Mixtas para $w_{1}$:}
%___________________________________________________________________________________________
%13/61



\item \begin{eqnarray*} &&\frac{\partial}{\partial
z_1}\frac{\partial}{\partial
w_2}\left(\hat{R}_{1}\left(P_{1}\left(z_{1}\right)\tilde{P}_{2}\left(z_{2}\right)\hat{P}_{1}\left(w_{1}\right)\hat{P}_{2}\left(w_{2}\right)\right)\hat{F}_{1}\left(\hat{\theta}_{1}\left(P_{1}\left(z_{1}\right)\tilde{P}_{2}\left(z_{2}\right)
\hat{P}_{2}\left(w_{2}\right)\right),w_{2}\right)F_{1}\left(z_{1},z_{2}\right)\right)\\
&=&\hat{r}_{1}\mu_{1}\hat{\mu}_{2}+\mu_{1}\hat{\mu}_{2}\hat{R}_{1}^{(2)}\left(1\right)+\hat{r}_{1}\hat{\mu}_{2}F_{1}^{(1,0)}+
\hat{r}_{1}\frac{\mu_{1}\hat{\mu}_{2}}{1-\hat{\mu}_{1}}\hat{F}_{1}^{(1,0)}+\hat{r}_{1}\mu_{1}\left(\hat{F}_{1}^{(0,1)}+\frac{\hat{\mu}_{2}}{1-\hat{\mu}_{1}}\hat{F}_{1}^{(1,0)}\right)\\
&+&F_{1}^{(1,0)}\left(\hat{F}_{1}^{(0,1)}+\frac{\hat{\mu}_{2}}{1-\hat{\mu}_{1}}\hat{F}_{1}^{(1,0)}\right)+\frac{\mu_{1}\hat{\mu}_{2}}{1-\hat{\mu}_{1}}\hat{F}_{1}^{(1,0)}+\mu_{1}\hat{\mu}_{2}\hat{\theta}_{1}^{(2)}\left(1\right)\hat{F}_{1}^{(1,0)}+\frac{\mu_{1}}{1-\hat{\mu}_{1}}\hat{F}_{1}^{(1,1)}\\
&+&\mu_{1}\hat{\mu}_{2}\left(\frac{1}{1-\hat{\mu}_{1}}\right)^{2}\hat{F}_{1}^{(2,0)}.
\end{eqnarray*}

%14/62
\item \begin{eqnarray*} &&\frac{\partial}{\partial
z_2}\frac{\partial}{\partial
w_2}\left(\hat{R}_{1}\left(P_{1}\left(z_{1}\right)\tilde{P}_{2}\left(z_{2}\right)\hat{P}_{1}\left(w_{1}\right)\hat{P}_{2}\left(w_{2}\right)\right)\hat{F}_{1}\left(\hat{\theta}_{1}\left(P_{1}\left(z_{1}\right)\tilde{P}_{2}\left(z_{2}\right)
\hat{P}_{2}\left(w_{2}\right)\right),w_{2}\right)F_{1}\left(z_{1},z_{2}\right)\right)\\
&=&\hat{r}_{1}\tilde{\mu}_{2}\hat{\mu}_{2}+\tilde{\mu}_{2}\hat{\mu}_{2}\hat{R}_{1}^{(2)}\left(1\right)+\hat{r}_{1}\hat{\mu}_{2}F_{1}^{(0,1)}+\hat{r}_{1}\frac{\tilde{\mu}_{2}\hat{\mu}_{2}}{1-\hat{\mu}_{1}}\hat{F}_{1}^{(1,0)}+\hat{r}_{1}\tilde{\mu}_{2}\left(\hat{F}_{1}^{(0,1)}+\frac{\hat{\mu}_{2}}{1-\hat{\mu}_{1}}\hat{F}_{1}^{(1,0)}\right)\\
&+&F_{1}^{(0,1)}\left(\hat{F}_{1}^{(0,1)}+\frac{\hat{\mu}_{2}}{1-\hat{\mu}_{1}}\hat{F}_{1}^{(1,0)}\right)+\frac{\tilde{\mu}_{2}\hat{\mu}_{2}}{1-\hat{\mu}_{1}}\hat{F}_{1}^{(1,0)}+\tilde{\mu}_{2}\hat{\mu}_{2}\hat{\theta}_{1}^{(2)}\left(1\right)\hat{F}_{1}^{(1,0)}+\frac{\tilde{\mu}_{2}}{1-\hat{\mu}_{1}}\hat{F}_{1}^{(1,1)}\\
&+&\tilde{\mu}_{2}\hat{\mu}_{2}\left(\frac{1}{1-\hat{\mu}_{1}}\right)^{2}\hat{F}_{1}^{(2,0)}.
\end{eqnarray*}

%15/63

\item \begin{eqnarray*} &&\frac{\partial}{\partial
w_1}\frac{\partial}{\partial
w_2}\left(\hat{R}_{1}\left(P_{1}\left(z_{1}\right)\tilde{P}_{2}\left(z_{2}\right)\hat{P}_{1}\left(w_{1}\right)\hat{P}_{2}\left(w_{2}\right)\right)\hat{F}_{1}\left(\hat{\theta}_{1}\left(P_{1}\left(z_{1}\right)\tilde{P}_{2}\left(z_{2}\right)
\hat{P}_{2}\left(w_{2}\right)\right),w_{2}\right)F_{1}\left(z_{1},z_{2}\right)\right)\\
&=&\hat{r}_{1}\hat{\mu}_{2}\hat{\mu}_{1}+\hat{\mu}_{2}\hat{\mu}_{1}\hat{R}_{1}^{(2)}\left(1\right)+\hat{r}_{1}\hat{\mu}_{1}\left(\hat{F}_{1}^{(0,1)}+\frac{\hat{\mu}_{2}}{1-\hat{\mu}_{1}}\hat{F}_{1}^{(1,0)}\right).
\end{eqnarray*}

%16/64

\item \begin{eqnarray*} &&\frac{\partial}{\partial
w_2}\frac{\partial}{\partial
w_2}\left(\hat{R}_{1}\left(P_{1}\left(z_{1}\right)\tilde{P}_{2}\left(z_{2}\right)\hat{P}_{1}\left(w_{1}\right)\hat{P}_{2}\left(w_{2}\right)\right)\hat{F}_{1}\left(\hat{\theta}_{1}\left(P_{1}\left(z_{1}\right)\tilde{P}_{2}\left(z_{2}\right)
\hat{P}_{2}\left(w_{2}\right)\right),w_{2}\right)F_{1}\left(z_{1},z_{2}\right)\right)\\
&=&\hat{r}_{1}\hat{P}_{2}^{(2)}\left(1\right)+\hat{\mu}_{2}^{2}\hat{R}_{1}^{(2)}\left(1\right)+
2\hat{r}_{1}\hat{\mu}_{2}\left(\hat{F}_{1}^{(0,1)}+\frac{\hat{\mu}_{2}}{1-\hat{\mu}_{1}}\hat{F}_{1}^{(1,0)}\right)+
\hat{F}_{1}^{(0,2)}+\frac{1}{1-\hat{\mu}_{1}}\hat{P}_{2}^{(2)}\left(1\right)\hat{F}_{1}^{(1,0)}\\
&+&\hat{\mu}_{2}^{2}\hat{\theta}_{1}^{(2)}\left(1\right)\hat{F}_{1}^{(1,0)}+\frac{\hat{\mu}_{2}}{1-\hat{\mu}_{1}}\hat{F}_{1}^{(1,1)}+\frac{\hat{\mu}_{2}}{1-\hat{\mu}_{1}}\left(\hat{F}_{1}^{(1,1)}+\frac{\hat{\mu}_{2}}{1-\hat{\mu}_{1}}\hat{F}_{1}^{(2,0)}\right).
\end{eqnarray*}
%_________________________________________________________________________________________________________
%
%_________________________________________________________________________________________________________

\end{enumerate}




Las ecuaciones que determinan los segundos momentos de las longitudes de las colas de los dos sistemas se pueden ver en \href{http://sitio.expresauacm.org/s/carlosmartinez/wp-content/uploads/sites/13/2014/01/SegundosMomentos.pdf}{este sitio}

%\url{http://ubuntu_es_el_diablo.org},\href{http://www.latex-project.org/}{latex project}

%http://sitio.expresauacm.org/s/carlosmartinez/wp-content/uploads/sites/13/2014/01/SegundosMomentos.jpg
%http://sitio.expresauacm.org/s/carlosmartinez/wp-content/uploads/sites/13/2014/01/SegundosMomentos.pdf




%_____________________________________________________________________________________
%Distribuci\'on del n\'umero de usuaruios que pasan del sistema 1 al sistema 2
%_____________________________________________________________________________________
\section*{Ap\'endice B}
%________________________________________________________________________________________
%
%________________________________________________________________________________________
\subsection{Distribuci\'on para los usuarios de traslado}
%________________________________________________________________________________________

Ahora, determinemos la distribuci\'on del n\'umero de usuarios que
pasan de $\hat{Q}_{2}$ a $Q_{2}$ considerando dos pol\'iticas de
traslado en espec\'ifico:

\begin{enumerate}
\item Solamente pasa un usuario,

\item Se permite el paso de $k$ usuarios,
\end{enumerate}
una vez que son atendidos por el servidor en $\hat{Q}_{2}$.

\begin{description}


\item[Pol\'itica de un solo usuario:] Sea $R_{2}$ el n\'umero de
usuarios que llegan a $\hat{Q}_{2}$ al tiempo $t$, sea $R_{1}$ el
n\'umero de usuarios que pasan de $\hat{Q}_{2}$ a $Q_{2}$ al
tiempo $t$.
\end{description}


A saber:
\begin{eqnarray*}
\esp\left[R_{1}\right]&=&\sum_{y\geq0}\prob\left[R_{2}=y\right]\esp\left[R_{1}|R_{2}=y\right]\\
&=&\sum_{y\geq0}\prob\left[R_{2}=y\right]\sum_{x\geq0}x\prob\left[R_{1}=x|R_{2}=y\right]\\
&=&\sum_{y\geq0}\sum_{x\geq0}x\prob\left[R_{1}=x|R_{2}=y\right]\prob\left[R_{2}=y\right].\\
\end{eqnarray*}

Determinemos
\begin{equation}
\esp\left[R_{1}|R_{2}=y\right]=\sum_{x\geq0}x\prob\left[R_{1}=x|R_{2}=y\right].
\end{equation}

supongamos que $y=0$, entonces
\begin{eqnarray*}
\prob\left[R_{1}=0|R_{2}=0\right]&=&1,\\
\prob\left[R_{1}=x|R_{2}=0\right]&=&0,\textrm{ para cualquier }x\geq1,\\
\end{eqnarray*}


por tanto
\begin{eqnarray*}
\esp\left[R_{1}|R_{2}=0\right]=0.
\end{eqnarray*}

Para $y=1$,
\begin{eqnarray*}
\prob\left[R_{1}=0|R_{2}=1\right]&=&0,\\
\prob\left[R_{1}=1|R_{2}=1\right]&=&1,
\end{eqnarray*}

entonces
\begin{eqnarray*}
\esp\left[R_{1}|R_{2}=1\right]=1.
\end{eqnarray*}

Para $y>1$:
\begin{eqnarray*}
\prob\left[R_{1}=0|R_{2}\geq1\right]&=&0,\\
\prob\left[R_{1}=1|R_{2}\geq1\right]&=&1,\\
\prob\left[R_{1}>1|R_{2}\geq1\right]&=&0,
\end{eqnarray*}

entonces
\begin{eqnarray*}
\esp\left[R_{1}|R_{2}=y\right]=1,\textrm{ para cualquier }y>1.
\end{eqnarray*}
es decir
\begin{eqnarray*}
\esp\left[R_{1}|R_{2}=y\right]=1,\textrm{ para cualquier }y\geq1.
\end{eqnarray*}

Entonces
\begin{eqnarray*}
\esp\left[R_{1}\right]&=&\sum_{y\geq0}\sum_{x\geq0}x\prob\left[R_{1}=x|R_{2}=y\right]\prob\left[R_{2}=y\right]=\sum_{y\geq0}\sum_{x}\esp\left[R_{1}|R_{2}=y\right]\prob\left[R_{2}=y\right]\\
&=&\sum_{y\geq0}\prob\left[R_{2}=y\right]=\sum_{y\geq1}\frac{\left(\lambda
t\right)^{k}}{k!}e^{-\lambda t}=1.
\end{eqnarray*}

Adem\'as para $k\in Z^{+}$
\begin{eqnarray*}
f_{R_{1}}\left(k\right)&=&\prob\left[R_{1}=k\right]=\sum_{n=0}^{\infty}\prob\left[R_{1}=k|R_{2}=n\right]\prob\left[R_{2}=n\right]\\
&=&\prob\left[R_{1}=k|R_{2}=0\right]\prob\left[R_{2}=0\right]+\prob\left[R_{1}=k|R_{2}=1\right]\prob\left[R_{2}=1\right]\\
&+&\prob\left[R_{1}=k|R_{2}>1\right]\prob\left[R_{2}>1\right],
\end{eqnarray*}

donde para


\begin{description}
\item[$k=0$:]
\begin{eqnarray*}
\prob\left[R_{1}=0\right]=\prob\left[R_{1}=0|R_{2}=0\right]\prob\left[R_{2}=0\right]+\prob\left[R_{1}=0|R_{2}=1\right]\prob\left[R_{2}=1\right]\\
+\prob\left[R_{1}=0|R_{2}>1\right]\prob\left[R_{2}>1\right]=\prob\left[R_{2}=0\right].
\end{eqnarray*}
\item[$k=1$:]
\begin{eqnarray*}
\prob\left[R_{1}=1\right]=\prob\left[R_{1}=1|R_{2}=0\right]\prob\left[R_{2}=0\right]+\prob\left[R_{1}=1|R_{2}=1\right]\prob\left[R_{2}=1\right]\\
+\prob\left[R_{1}=1|R_{2}>1\right]\prob\left[R_{2}>1\right]=\sum_{n=1}^{\infty}\prob\left[R_{2}=n\right].
\end{eqnarray*}

\item[$k=2$:]
\begin{eqnarray*}
\prob\left[R_{1}=2\right]=\prob\left[R_{1}=2|R_{2}=0\right]\prob\left[R_{2}=0\right]+\prob\left[R_{1}=2|R_{2}=1\right]\prob\left[R_{2}=1\right]\\
+\prob\left[R_{1}=2|R_{2}>1\right]\prob\left[R_{2}>1\right]=0.
\end{eqnarray*}

\item[$k=j$:]
\begin{eqnarray*}
\prob\left[R_{1}=j\right]=\prob\left[R_{1}=j|R_{2}=0\right]\prob\left[R_{2}=0\right]+\prob\left[R_{1}=j|R_{2}=1\right]\prob\left[R_{2}=1\right]\\
+\prob\left[R_{1}=j|R_{2}>1\right]\prob\left[R_{2}>1\right]=0.
\end{eqnarray*}
\end{description}


Por lo tanto
\begin{eqnarray*}
f_{R_{1}}\left(0\right)&=&\prob\left[R_{2}=0\right]\\
f_{R_{1}}\left(1\right)&=&\sum_{n\geq1}^{\infty}\prob\left[R_{2}=n\right]\\
f_{R_{1}}\left(j\right)&=&0,\textrm{ para }j>1.
\end{eqnarray*}



\begin{description}
\item[Pol\'itica de $k$ usuarios:]Al igual que antes, para $y\in Z^{+}$ fijo
\begin{eqnarray*}
\esp\left[R_{1}|R_{2}=y\right]=\sum_{x}x\prob\left[R_{1}=x|R_{2}=y\right].\\
\end{eqnarray*}
\end{description}
Entonces, si tomamos diversos valore para $y$:\\

$y=0$:
\begin{eqnarray*}
\prob\left[R_{1}=0|R_{2}=0\right]&=&1,\\
\prob\left[R_{1}=x|R_{2}=0\right]&=&0,\textrm{ para cualquier }x\geq1,
\end{eqnarray*}

entonces
\begin{eqnarray*}
\esp\left[R_{1}|R_{2}=0\right]=0.
\end{eqnarray*}


Para $y=1$,
\begin{eqnarray*}
\prob\left[R_{1}=0|R_{2}=1\right]&=&0,\\
\prob\left[R_{1}=1|R_{2}=1\right]&=&1,
\end{eqnarray*}

entonces {\scriptsize{
\begin{eqnarray*}
\esp\left[R_{1}|R_{2}=1\right]=1.
\end{eqnarray*}}}


Para $y=2$,
\begin{eqnarray*}
\prob\left[R_{1}=0|R_{2}=2\right]&=&0,\\
\prob\left[R_{1}=1|R_{2}=2\right]&=&1,\\
\prob\left[R_{1}=2|R_{2}=2\right]&=&1,\\
\prob\left[R_{1}=3|R_{2}=2\right]&=&0,
\end{eqnarray*}

entonces
\begin{eqnarray*}
\esp\left[R_{1}|R_{2}=2\right]=3.
\end{eqnarray*}

Para $y=3$,
\begin{eqnarray*}
\prob\left[R_{1}=0|R_{2}=3\right]&=&0,\\
\prob\left[R_{1}=1|R_{2}=3\right]&=&1,\\
\prob\left[R_{1}=2|R_{2}=3\right]&=&1,\\
\prob\left[R_{1}=3|R_{2}=3\right]&=&1,\\
\prob\left[R_{1}=4|R_{2}=3\right]&=&0,
\end{eqnarray*}

entonces
\begin{eqnarray*}
\esp\left[R_{1}|R_{2}=3\right]=6.
\end{eqnarray*}

En general, para $k\geq0$,
\begin{eqnarray*}
\prob\left[R_{1}=0|R_{2}=k\right]&=&0,\\
\prob\left[R_{1}=j|R_{2}=k\right]&=&1,\textrm{ para }1\leq j\leq k,\\
\prob\left[R_{1}=j|R_{2}=k\right]&=&0,\textrm{ para }j> k,
\end{eqnarray*}

entonces
\begin{eqnarray*}
\esp\left[R_{1}|R_{2}=k\right]=\frac{k\left(k+1\right)}{2}.
\end{eqnarray*}



Por lo tanto


\begin{eqnarray*}
\esp\left[R_{1}\right]&=&\sum_{y}\esp\left[R_{1}|R_{2}=y\right]\prob\left[R_{2}=y\right]\\
&=&\sum_{y}\prob\left[R_{2}=y\right]\frac{y\left(y+1\right)}{2}=\sum_{y\geq1}\left(\frac{y\left(y+1\right)}{2}\right)\frac{\left(\lambda t\right)^{y}}{y!}e^{-\lambda t}\\
&=&\frac{\lambda t}{2}e^{-\lambda t}\sum_{y\geq1}\left(y+1\right)\frac{\left(\lambda t\right)^{y-1}}{\left(y-1\right)!}=\frac{\lambda t}{2}e^{-\lambda t}\left(e^{\lambda t}\left(\lambda t+2\right)\right)\\
&=&\frac{\lambda t\left(\lambda t+2\right)}{2},
\end{eqnarray*}
es decir,


\begin{equation}
\esp\left[R_{1}\right]=\frac{\lambda t\left(\lambda
t+2\right)}{2}.
\end{equation}

Adem\'as para $k\in Z^{+}$ fijo
\begin{eqnarray*}
f_{R_{1}}\left(k\right)&=&\prob\left[R_{1}=k\right]=\sum_{n=0}^{\infty}\prob\left[R_{1}=k|R_{2}=n\right]\prob\left[R_{2}=n\right]\\
&=&\prob\left[R_{1}=k|R_{2}=0\right]\prob\left[R_{2}=0\right]+\prob\left[R_{1}=k|R_{2}=1\right]\prob\left[R_{2}=1\right]\\
&+&\prob\left[R_{1}=k|R_{2}=2\right]\prob\left[R_{2}=2\right]+\cdots+\prob\left[R_{1}=k|R_{2}=j\right]\prob\left[R_{2}=j\right]+\cdots+
\end{eqnarray*}
donde para

\begin{description}
\item[$k=0$:]
\begin{eqnarray*}
\prob\left[R_{1}=0\right]=\prob\left[R_{1}=0|R_{2}=0\right]\prob\left[R_{2}=0\right]+\prob\left[R_{1}=0|R_{2}=1\right]\prob\left[R_{2}=1\right]\\
+\prob\left[R_{1}=0|R_{2}=j\right]\prob\left[R_{2}=j\right]=\prob\left[R_{2}=0\right].
\end{eqnarray*}
\item[$k=1$:]
\begin{eqnarray*}
\prob\left[R_{1}=1\right]=\prob\left[R_{1}=1|R_{2}=0\right]\prob\left[R_{2}=0\right]+\prob\left[R_{1}=1|R_{2}=1\right]\prob\left[R_{2}=1\right]\\
+\prob\left[R_{1}=1|R_{2}=1\right]\prob\left[R_{2}=1\right]+\cdots+\prob\left[R_{1}=1|R_{2}=j\right]\prob\left[R_{2}=j\right]\\
=\sum_{n=1}^{\infty}\prob\left[R_{2}=n\right].
\end{eqnarray*}

\item[$k=2$:]
\begin{eqnarray*}
\prob\left[R_{1}=2\right]=\prob\left[R_{1}=2|R_{2}=0\right]\prob\left[R_{2}=0\right]+\prob\left[R_{1}=2|R_{2}=1\right]\prob\left[R_{2}=1\right]\\
+\prob\left[R_{1}=2|R_{2}=2\right]\prob\left[R_{2}=2\right]+\cdots+\prob\left[R_{1}=2|R_{2}=j\right]\prob\left[R_{2}=j\right]\\
=\sum_{n=2}^{\infty}\prob\left[R_{2}=n\right].
\end{eqnarray*}
\end{description}

En general

\begin{eqnarray*}
\prob\left[R_{1}=k\right]=\prob\left[R_{1}=k|R_{2}=0\right]\prob\left[R_{2}=0\right]+\prob\left[R_{1}=k|R_{2}=1\right]\prob\left[R_{2}=1\right]\\
+\prob\left[R_{1}=k|R_{2}=2\right]\prob\left[R_{2}=2\right]+\cdots+\prob\left[R_{1}=k|R_{2}=k\right]\prob\left[R_{2}=k\right]\\
=\sum_{n=k}^{\infty}\prob\left[R_{2}=n\right].\\
\end{eqnarray*}



Por lo tanto

\begin{eqnarray*}
f_{R_{1}}\left(k\right)&=&\prob\left[R_{1}=k\right]=\sum_{n=k}^{\infty}\prob\left[R_{2}=n\right].
\end{eqnarray*}






\section*{Objetivos Principales}

\begin{itemize}
%\item Generalizar los principales resultados existentes para Sistemas de Visitas C\'iclicas para el caso en el que se tienen dos Sistemas de Visitas C\'iclicas con propiedades similares.

\item Encontrar las ecuaciones que modelan el comportamiento de una Red de Sistemas de Visitas C\'iclicas (RSVC) con propiedades similares.

\item Encontrar expresiones anal\'iticas para las longitudes de las colas al momento en que el servidor llega a una de ellas para comenzar a dar servicio, as\'i como de sus segundos momentos.

\item Determinar las principales medidas de Desempe\~no para la RSVC tales como: N\'umero de usuarios presentes en cada una de las colas del sistema cuando uno de los servidores est\'a presente atendiendo, Tiempos que transcurre entre las visitas del servidor a la misma cola.


\end{itemize}


%_________________________________________________________________________
%\section{Sistemas de Visitas C\'iclicas}
%_________________________________________________________________________
\numberwithin{equation}{section}%
%__________________________________________________________________________




%\section*{Introducci\'on}




%__________________________________________________________________________
%\subsection{Definiciones}
%__________________________________________________________________________


\section{Descripci\'on de una Red de Sistemas de Visitas C\'iclicas}



Consideremos una red de sistema de visitas c\'iclicas conformada por dos sistemas de visitas c\'iclicas, cada una con dos colas independientes, donde adem\'as se permite el intercambio de usuarios entre los dos sistemas en la segunda cola de cada uno de ellos.\smallskip

Sup\'ongase adem\'as que los arribos de los usuarios ocurren
conforme a un proceso Poisson con tasa de llegada $\mu_{1}$ y
$\mu_{2}$ para el sistema 1, mientras que para el sistema 2,
lo hacen conforme a un proceso Poisson con tasa
$\hat{\mu}_{1},\hat{\mu}_{2}$ respectivamente.\smallskip

El traslado de un sistema a otro ocurre de manera que los tiempos
entre llegadas de los usuarios a la cola dos del sistema 1
provenientes del sistema 2, se distribuye de manera exponencial
con par\'ametro $\check{\mu}_{2}$.\smallskip

Se considerar\'an intervalos de tiempo de la forma
$\left[t,t+1\right]$. Los usuarios arriban por paquetes de manera
independiente del resto de las colas. Se define el grupo de
usuarios que llegan a cada una de las colas del sistema 1,
caracterizadas por $Q_{1}$ y $Q_{2}$ respectivamente, en el
intervalo de tiempo $\left[t,t+1\right]$ por
$X_{1}\left(t\right),X_{2}\left(t\right)$. De igual manera se
definen los procesos
$\hat{X}_{1}\left(t\right),\hat{X}_{2}\left(t\right)$ para las
colas del sistema 2, denotadas por $\hat{Q}_{1}$ y $\hat{Q}_{2}$
respectivamente.\smallskip

Para el n\'umero de usuarios que se trasladan del sistema 2 al
sistema 1, de la cola $\hat{Q}_{2}$ a la cola
$Q_{2}$, en el intervalo de tiempo
$\left[t,t+1\right]$, se define el proceso
$Y_{2}\left(t\right)$.\smallskip

El uso de la Funci\'on Generadora de Probabilidades (FGP's) nos permite determinar las Funciones de Distribuci\'on de Probabilidades Conjunta de manera indirecta sin necesidad de hacer uso de las propiedades de las distribuciones de probabilidad de cada uno de los procesos que intervienen en la Red de Sistemas de Visitas C\'iclicas.\smallskip

En lo que respecta al servidor, en t\'erminos de los tiempos de
visita a cada una de las colas, se definen las variables
aleatorias $\tau_{1},\tau_{2}$ para $Q_{1},Q_{2}$ respectivamente;
y $\zeta_{1},\zeta_{2}$ para $\hat{Q}_{1},\hat{Q}_{2}$ del sistema
2. A los tiempos en que el servidor termina de atender en las
colas $Q_{1},Q_{2},\hat{Q}_{1},\hat{Q}_{2}$, se les denotar\'a por
$\overline{\tau}_{1},\overline{\tau}_{2},\overline{\zeta}_{1},\overline{\zeta}_{2}$
respectivamente.\smallskip

Los tiempos de traslado del servidor desde el momento en que termina de atender a una cola y llega a la siguiente para comenzar a dar servicio est\'an dados por
$\tau_{2}-\overline{\tau}_{1},\tau_{1}-\overline{\tau}_{2}$ y
$\zeta_{2}-\overline{\zeta}_{1},\zeta_{1}-\overline{\zeta}_{2}$
para el sistema 1 y el sistema 2, respectivamente.\smallskip

Cada uno de estos procesos con su respectiva FGP. Adem\'as, para cada una de las colas en cada sistema, el n\'umero de usuarios al tiempo en que llega el servidor a dar servicio est\'a
dado por el n\'umero de usuarios presentes en la cola al tiempo
$t$, m\'as el n\'umero de usuarios que llegan a la cola en el intervalo de tiempo
$\left[\tau_{i},\overline{\tau}_{i}\right]$.

%es decir
%{\small{
%\begin{eqnarray*}
%L_{1}\left(\overline{\tau}_{1}\right)=L_{1}\left(\tau_{1}\right)+X_{1}\left(\overline{\tau}_{1}-\tau_{1}\right),\hat{L}_{i}\left(\overline{\tau}_{i}\right)=\hat{L}_{i}\left(\tau_{i}\right)+\hat{X}_{i}\left(\overline{\tau}_{i}-\tau_{i}\right),L_{2}\left(\overline{\tau}_{1}\right)=L_{2}\left(\tau_{1}\right)+X_{2}\left(\overline{\tau}_{1}-\tau_{1}\right)+Y_{2}\left(\overline{\tau}_{1}-\tau_{1}\right),
%\end{eqnarray*}}}




%\begin{center}\vspace{1cm}
%%%%\includegraphics[width=0.6\linewidth]{RedSVC2}
%\captionof{figure}{\color{Green} Red de Sistema de Visitas C\'iclicas}
%\end{center}\vspace{1cm}




Una vez definidas las Funciones Generadoras de Probabilidades Conjuntas se construyen las ecuaciones recursivas que permiten obtener la informaci\'on sobre la longitud de cada una de las colas, al momento en que uno de los servidores llega a una de las colas para dar servicio, bas\'andose en la informaci\'on que se tiene sobre su llegada a la cola inmediata anterior.\smallskip
%{\footnotesize{
%\begin{eqnarray*}
%F_{2}\left(z_{1},z_{2},w_{1},w_{2}\right)&=&R_{1}\left(P_{1}\left(z_{1}\right)\tilde{P}_{2}\left(z_{2}\right)\prod_{i=1}^{2}
%\hat{P}_{i}\left(w_{i}\right)\right)F_{1}\left(\theta_{1}\left(\tilde{P}_{2}\left(z_{2}\right)\hat{P}_{1}\left(w_{1}\right)\hat{P}_{2}\left(w_{2}\right)\right),z_{2},w_{1},w_{2}\right),\\
%F_{1}\left(z_{1},z_{2},w_{1},w_{2}\right)&=&R_{2}\left(P_{1}\left(z_{1}\right)\tilde{P}_{2}\left(z_{2}\right)\prod_{i=1}^{2}
%\hat{P}_{i}\left(w_{i}\right)\right)F_{2}\left(z_{1},\tilde{\theta}_{2}\left(P_{1}\left(z_{1}\right)\hat{P}_{1}\left(w_{1}\right)\hat{P}_{2}\left(w_{2}\right)\right),w_{1},w_{2}\right),\\
%\hat{F}_{2}\left(z_{1},z_{2},w_{1},w_{2}\right)&=&\hat{R}_{1}\left(P_{1}\left(z_{1}\right)\tilde{P}_{2}\left(z_{2}\right)\prod_{i=1}^{2}
%\hat{P}_{i}\left(w_{i}\right)\right)\hat{F}_{1}\left(z_{1},z_{2},\hat{\theta}_{1}\left(P_{1}\left(z_{1}\right)\tilde{P}_{2}\left(z_{2}\right)\hat{P}_{2}\left(w_{2}\right)\right),w_{2}\right),\\
%\end{eqnarray*}}}
%{\small{
%\begin{eqnarray*}
%\hat{F}_{1}\left(z_{1},z_{2},w_{1},w_{2}\right)&=&\hat{R}_{2}\left(P_{1}\left(z_{1}\right)\tilde{P}_{2}\left(z_{2}\right)\prod_{i=1}^{2}
%\hat{P}_{i}\left(w_{i}\right)\right)\hat{F}_{2}\left(z_{1},z_{2},w_{1},\hat{\theta}_{2}\left(P_{1}\left(z_{1}\right)\tilde{P}_{2}\left(z_{2}\right)\hat{P}_{1}\left(w_{1}\right)\right)\right).
%\end{eqnarray*}}}

%__________________________________________________________________________
\subsection{Funciones Generadoras de Probabilidades}
%__________________________________________________________________________


Para cada uno de los procesos de llegada a las colas $X_{1},X_{2},\hat{X}_{1},\hat{X}_{2}$ y $Y_{2}$, con $\tilde{X}_{2}=X_{2}+Y_{2}$ anteriores se define su Funci\'on
Generadora de Probabilidades (FGP):
%\begin{multicols}{3}
\begin{eqnarray*}
\begin{array}{ccc}
P_{1}\left(z_{1}\right)=\esp\left[z_{1}^{X_{1}\left(t\right)}\right],&P_{2}\left(z_{2}\right)=\esp\left[z_{2}^{X_{2}\left(t\right)}\right],&\check{P}_{2}\left(z_{2}\right)=\esp\left[z_{2}^{Y_{2}\left(t\right)}\right],\\
\hat{P}_{1}\left(w_{1}\right)=\esp\left[w_{1}^{\hat{X}_{1}\left(t\right)}\right],&\hat{P}_{2}\left(w_{2}\right)=\esp\left[w_{2}^{\hat{X}_{2}\left(t\right)}\right],&\tilde{P}_{2}\left(z_{2}\right)=\esp\left[z_{2}^{\tilde{X}_{2}\left(t\right)}\right].
\end{array}
\end{eqnarray*}

Con primer momento definidos por

\begin{eqnarray*}
\begin{array}{cc}
\mu_{1}=\esp\left[X_{1}\left(t\right)\right]=P_{1}^{(1)}\left(1\right),&\mu_{2}=\esp\left[X_{2}\left(t\right)\right]=P_{2}^{(1)}\left(1\right),\\
\check{\mu}_{2}=\esp\left[Y_{2}\left(t\right)\right]=\check{P}_{2}^{(1)}\left(1\right),&
\hat{\mu}_{1}=\esp\left[\hat{X}_{1}\left(t\right)\right]=\hat{P}_{1}^{(1)}\left(1\right),\\
\hat{\mu}_{2}=\esp\left[\hat{X}_{2}\left(t\right)\right]=\hat{P}_{2}^{(1)}\left(1\right),&\tilde{\mu}_{2}=\esp\left[\tilde{X}_{2}\left(t\right)\right]=\tilde{P}_{2}^{(1)}\left(1\right).
\end{array}
\end{eqnarray*}

En lo que respecta al servidor, en t\'erminos de los tiempos de
visita a cada una de las colas, se denotar\'an por
$B_{1}\left(t\right),B_{2}\left(t\right)$ los procesos
correspondientes a las variables aleatorias $\tau_{1},\tau_{2}$
para $Q_{1},Q_{2}$ respectivamente; y
$\hat{B}_{1}\left(t\right),\hat{B}_{2}\left(t\right)$ con
par\'ametros $\zeta_{1},\zeta_{2}$ para $\hat{Q}_{1},\hat{Q}_{2}$
del sistema 2. Y a los tiempos en que el servidor termina de
atender en las colas $Q_{1},Q_{2},\hat{Q}_{1},\hat{Q}_{2}$, se les
denotar\'a por
$\overline{\tau}_{1},\overline{\tau}_{2},\overline{\zeta}_{1},\overline{\zeta}_{2}$
respectivamente. Entonces, los tiempos de servicio est\'an dados
por las diferencias
$\overline{\tau}_{1}-\tau_{1},\overline{\tau}_{2}-\tau_{2}$ para
$Q_{1},Q_{2}$, y
$\overline{\zeta}_{1}-\zeta_{1},\overline{\zeta}_{2}-\zeta_{2}$
para $\hat{Q}_{1},\hat{Q}_{2}$ respectivamente.

Sus procesos se definen por:


\begin{eqnarray*}
\begin{array}{cc}
S_{1}\left(z_{1}\right)=\esp\left[z_{1}^{\overline{\tau}_{1}-\tau_{1}}\right],&S_{2}\left(z_{2}\right)=\esp\left[z_{1}^{\overline{\tau}_{2}-\tau_{2}}\right],\\
\hat{S}_{1}\left(w_{1}\right)=\esp\left[w_{1}^{\overline{\zeta}_{1}-\zeta_{1}}\right],&\hat{S}_{2}\left(w_{2}\right)=\esp\left[w_{2}^{\overline{\zeta}_{2}-\zeta_{2}}\right],
\end{array}
\end{eqnarray*}

con primer momento dado por:


\begin{eqnarray*}
\begin{array}{cccc}
s_{1}=\esp\left[\overline{\tau}_{1}-\tau_{1}\right],&s_{2}=\esp\left[\overline{\tau}_{2}-\tau_{2}\right],&
\hat{s}_{1}=\esp\left[\overline{\zeta}_{1}-\zeta_{1}\right],&
\hat{s}_{2}=\esp\left[\overline{\zeta}_{2}-\zeta_{2}\right].
\end{array}
\end{eqnarray*}

An\'alogamente los tiempos de traslado del servidor desde el
momento en que termina de atender a una cola y llega a la
siguiente para comenzar a dar servicio est\'an dados por
$\tau_{2}-\overline{\tau}_{1},\tau_{1}-\overline{\tau}_{2}$ y
$\zeta_{2}-\overline{\zeta}_{1},\zeta_{1}-\overline{\zeta}_{2}$
para el sistema 1 y el sistema 2, respectivamente.

La FGP para estos tiempos de traslado est\'an dados por

\begin{eqnarray*}
\begin{array}{cc}
R_{1}\left(z_{1}\right)=\esp\left[z_{1}^{\tau_{2}-\overline{\tau}_{1}}\right],&R_{2}\left(z_{2}\right)=\esp\left[z_{2}^{\tau_{1}-\overline{\tau}_{2}}\right],\\
\hat{R}_{1}\left(w_{1}\right)=\esp\left[w_{1}^{\zeta_{2}-\overline{\zeta}_{1}}\right],&\hat{R}_{2}\left(w_{2}\right)=\esp\left[w_{2}^{\zeta_{1}-\overline{\zeta}_{2}}\right],
\end{array}
\end{eqnarray*}
y al igual que como se hizo con anterioridad

\begin{eqnarray*}
\begin{array}{cc}
r_{1}=R_{1}^{(1)}\left(1\right)=\esp\left[\tau_{2}-\overline{\tau}_{1}\right],&r_{2}=R_{2}^{(1)}\left(1\right)=\esp\left[\tau_{1}-\overline{\tau}_{2}\right],\\
\hat{r}_{1}=\hat{R}_{1}^{(1)}\left(1\right)=\esp\left[\zeta_{2}-\overline{\zeta}_{1}\right],&
\hat{r}_{2}=\hat{R}_{2}^{(1)}\left(1\right)=\esp\left[\zeta_{1}-\overline{\zeta}_{2}\right].
\end{array}
\end{eqnarray*}

Se definen los procesos de conteo para el n\'umero de usuarios en
cada una de las colas al tiempo $t$,
$L_{1}\left(t\right),L_{2}\left(t\right)$, para
$H_{1}\left(t\right),H_{2}\left(t\right)$ del sistema 1,
respectivamente. Y para el segundo sistema, se tienen los procesos
$\hat{L}_{1}\left(t\right),\hat{L}_{2}\left(t\right)$ para
$\hat{H}_{1}\left(t\right),\hat{H}_{2}\left(t\right)$,
respectivamente, es decir,


\begin{eqnarray*}
\begin{array}{cccc}
H_{1}\left(t\right)=\esp\left[z_{1}^{L_{1}\left(t\right)}\right],&
H_{2}\left(t\right)=\esp\left[z_{2}^{L_{2}\left(t\right)}\right],&
\hat{H}_{1}\left(t\right)=\esp\left[w_{1}^{\hat{L}_{1}\left(t\right)}\right],&\hat{H}_{2}\left(t\right)=\esp\left[w_{2}^{\hat{L}_{2}\left(t\right)}\right].
\end{array}
\end{eqnarray*}
Por lo dicho anteriormente se tiene que el n\'umero de usuarios
presentes en los tiempos $\overline{\tau}_{1},\overline{\tau}_{2},
\overline{\zeta}_{1},\overline{\zeta}_{2}$, es cero, es decir,
 $L_{i}\left(\overline{\tau_{i}}\right)=0,$ y
$\hat{L}_{i}\left(\overline{\zeta_{i}}\right)=0$ para i=1,2 para
cada uno de los dos sistemas.


Para cada una de las colas en cada sistema, el n\'umero de
usuarios al tiempo en que llega el servidor a dar servicio est\'a
dado por el n\'umero de usuarios presentes en la cola al tiempo
$t=\tau_{i},\zeta_{i}$, m\'as el n\'umero de usuarios que llegan a
la cola en el intervalo de tiempo
$\left[\tau_{i},\overline{\tau}_{i}\right],\left[\zeta_{i},\overline{\zeta}_{i}\right]$,
es decir

\begin{eqnarray*}\label{Eq.TiemposLlegada}
\begin{array}{cc}
L_{1}\left(\overline{\tau}_{1}\right)=L_{1}\left(\tau_{1}\right)+X_{1}\left(\overline{\tau}_{1}-\tau_{1}\right),&\hat{L}_{1}\left(\overline{\tau}_{1}\right)=\hat{L}_{1}\left(\tau_{1}\right)+\hat{X}_{1}\left(\overline{\tau}_{1}-\tau_{1}\right),\\
\hat{L}_{2}\left(\overline{\tau}_{1}\right)=\hat{L}_{2}\left(\tau_{1}\right)+\hat{X}_{2}\left(\overline{\tau}_{1}-\tau_{1}\right).&
\end{array}
\end{eqnarray*}

En el caso espec\'ifico de $Q_{2}$, adem\'as, hay que considerar
el n\'umero de usuarios que pasan del sistema 2 al sistema 1, a
traves de $\hat{Q}_{2}$ mientras el servidor en $Q_{2}$ est\'a
ausente, es decir:

\begin{equation}\label{Eq.UsuariosTotalesZ2}
L_{2}\left(\overline{\tau}_{1}\right)=L_{2}\left(\tau_{1}\right)+X_{2}\left(\overline{\tau}_{1}-\tau_{1}\right)+Y_{2}\left(\overline{\tau}_{1}-\tau_{1}\right).
\end{equation}

%_________________________________________________________________________
\subsection{La ruina del jugador}
%_________________________________________________________________________

Supongamos que se tiene un jugador que cuenta con un capital
inicial de $\tilde{L}_{0}\geq0$ unidades, esta persona realiza una
serie de dos juegos simult\'aneos e independientes de manera
sucesiva, dichos eventos son independientes e id\'enticos entre
s\'i para cada realizaci\'on.\smallskip

La ganancia en el $n$-\'esimo juego es
\begin{eqnarray*}\label{Eq.Cero}
\tilde{X}_{n}=X_{n}+Y_{n}
\end{eqnarray*}

unidades de las cuales se resta una cuota de 1 unidad por cada
juego simult\'aneo, es decir, se restan dos unidades por cada
juego realizado.\smallskip

En t\'erminos de la teor\'ia de colas puede pensarse como el n\'umero de usuarios que llegan a una cola v\'ia dos procesos de arribo distintos e independientes entre s\'i.

Su Funci\'on Generadora de Probabilidades (FGP) est\'a dada por

\begin{eqnarray*}
F\left(z\right)=\esp\left[z^{\tilde{L}_{0}}\right]
\end{eqnarray*}

\begin{eqnarray*}
\tilde{P}\left(z\right)=\esp\left[z^{\tilde{X}_{n}}\right]=\esp\left[z^{X_{n}+Y_{n}}\right]=\esp\left[z^{X_{n}}z^{Y_{n}}\right]=\esp\left[z^{X_{n}}\right]\esp\left[z^{Y_{n}}\right]=P\left(z\right)\check{P}\left(z\right),
\end{eqnarray*}
entonces
\begin{eqnarray*}
\tilde{\mu}&=&\esp\left[\tilde{X}_{n}\right]=\tilde{P}\left[z\right]<1.\\
\end{eqnarray*}

Sea $\tilde{L}_{n}$ el capital remanente despu\'es del $n$-\'esimo
juego. Entonces

\begin{eqnarray*}
\tilde{L}_{n}&=&\tilde{L}_{0}+\tilde{X}_{1}+\tilde{X}_{2}+\cdots+\tilde{X}_{n}-2n.
\end{eqnarray*}

La ruina del jugador ocurre despu\'es del $n$-\'esimo juego, es decir, la cola se vac\'ia despu\'es del $n$-\'esimo juego,
entonces sea $T$ definida como

\begin{eqnarray*}
T&=&min\left\{\tilde{L}_{n}=0\right\}
\end{eqnarray*}

Si $\tilde{L}_{0}=0$, entonces claramente $T=0$. En este sentido $T$
puede interpretarse como la longitud del periodo de tiempo que el servidor ocupa para dar servicio en la cola, comenzando con $\tilde{L}_{0}$ grupos de usuarios
presentes en la cola, quienes arribaron conforme a un proceso dado
por $\tilde{P}\left(z\right)$.\smallskip


Sea $g_{n,k}$ la probabilidad del evento de que el jugador no
caiga en ruina antes del $n$-\'esimo juego, y que adem\'as tenga
un capital de $k$ unidades antes del $n$-\'esimo juego, es decir,

Dada $n\in\left\{1,2,\ldots,\right\}$ y
$k\in\left\{0,1,2,\ldots,\right\}$
\begin{eqnarray*}
g_{n,k}:=P\left\{\tilde{L}_{j}>0, j=1,\ldots,n,
\tilde{L}_{n}=k\right\}
\end{eqnarray*}

la cual adem\'as se puede escribir como:

\begin{eqnarray*}
g_{n,k}&=&P\left\{\tilde{L}_{j}>0, j=1,\ldots,n,
\tilde{L}_{n}=k\right\}=\sum_{j=1}^{k+1}g_{n-1,j}P\left\{\tilde{X}_{n}=k-j+1\right\}\\
&=&\sum_{j=1}^{k+1}g_{n-1,j}P\left\{X_{n}+Y_{n}=k-j+1\right\}=\sum_{j=1}^{k+1}\sum_{l=1}^{j}g_{n-1,j}P\left\{X_{n}+Y_{n}=k-j+1,Y_{n}=l\right\}\\
&=&\sum_{j=1}^{k+1}\sum_{l=1}^{j}g_{n-1,j}P\left\{X_{n}+Y_{n}=k-j+1|Y_{n}=l\right\}P\left\{Y_{n}=l\right\}\\
&=&\sum_{j=1}^{k+1}\sum_{l=1}^{j}g_{n-1,j}P\left\{X_{n}=k-j-l+1\right\}P\left\{Y_{n}=l\right\}\\
\end{eqnarray*}

es decir
\begin{eqnarray}\label{Eq.Gnk.2S}
g_{n,k}=\sum_{j=1}^{k+1}\sum_{l=1}^{j}g_{n-1,j}P\left\{X_{n}=k-j-l+1\right\}P\left\{Y_{n}=l\right\}
\end{eqnarray}
adem\'as

\begin{equation}\label{Eq.L02S}
g_{0,k}=P\left\{\tilde{L}_{0}=k\right\}.
\end{equation}

Se definen las siguientes FGP:
\begin{equation}\label{Eq.3.16.a.2S}
G_{n}\left(z\right)=\sum_{k=0}^{\infty}g_{n,k}z^{k},\textrm{ para
}n=0,1,\ldots,
\end{equation}

\begin{equation}\label{Eq.3.16.b.2S}
G\left(z,w\right)=\sum_{n=0}^{\infty}G_{n}\left(z\right)w^{n}.
\end{equation}


En particular para $k=0$,
\begin{eqnarray*}
g_{n,0}=G_{n}\left(0\right)=P\left\{\tilde{L}_{j}>0,\textrm{ para
}j<n,\textrm{ y }\tilde{L}_{n}=0\right\}=P\left\{T=n\right\},
\end{eqnarray*}

adem\'as

\begin{eqnarray*}%\label{Eq.G0w.2S}
G\left(0,w\right)=\sum_{n=0}^{\infty}G_{n}\left(0\right)w^{n}=\sum_{n=0}^{\infty}P\left\{T=n\right\}w^{n}
=\esp\left[w^{T}\right]
\end{eqnarray*}
la cu\'al resulta ser la FGP del tiempo de ruina $T$.

\begin{Prop}\label{Prop.1.1.2S}
Sean $G_{n}\left(z\right)$ y $G\left(z,w\right)$ definidas como en
(\ref{Eq.3.16.a.2S}) y (\ref{Eq.3.16.b.2S}) respectivamente,
entonces
\begin{equation}\label{Eq.Pag.45}
G_{n}\left(z\right)=\frac{1}{z}\left[G_{n-1}\left(z\right)-G_{n-1}\left(0\right)\right]\tilde{P}\left(z\right).
\end{equation}

Adem\'as


\begin{equation}\label{Eq.Pag.46}
G\left(z,w\right)=\frac{zF\left(z\right)-wP\left(z\right)G\left(0,w\right)}{z-wR\left(z\right)},
\end{equation}

con un \'unico polo en el c\'irculo unitario, adem\'as, el polo es
de la forma $z=\theta\left(w\right)$ y satisface que

\begin{enumerate}
\item[i)]$\tilde{\theta}\left(1\right)=1$,

\item[ii)] $\tilde{\theta}^{(1)}\left(1\right)=\frac{1}{1-\tilde{\mu}}$,

\item[iii)]
$\tilde{\theta}^{(2)}\left(1\right)=\frac{\tilde{\mu}}{\left(1-\tilde{\mu}\right)^{2}}+\frac{\tilde{\sigma}}{\left(1-\tilde{\mu}\right)^{3}}$.
\end{enumerate}

Finalmente, adem\'as se cumple que
\begin{equation}
\esp\left[w^{T}\right]=G\left(0,w\right)=F\left[\tilde{\theta}\left(w\right)\right].
\end{equation}
\end{Prop}

Multiplicando las ecuaciones (\ref{Eq.Gnk.2S}) y (\ref{Eq.L02S})
por el t\'ermino $z^{k}$:

\begin{eqnarray*}
g_{n,k}z^{k}&=&\sum_{j=1}^{k+1}\sum_{l=1}^{j}g_{n-1,j}P\left\{X_{n}=k-j-l+1\right\}P\left\{Y_{n}=l\right\}z^{k},\\
g_{0,k}z^{k}&=&P\left\{\tilde{L}_{0}=k\right\}z^{k},
\end{eqnarray*}

ahora sumamos sobre $k$
\begin{eqnarray*}
\sum_{k=0}^{\infty}g_{n,k}z^{k}&=&\sum_{k=0}^{\infty}\sum_{j=1}^{k+1}\sum_{l=1}^{j}g_{n-1,j}P\left\{X_{n}=k-j-l+1\right\}P\left\{Y_{n}=l\right\}z^{k}\\
&=&\sum_{k=0}^{\infty}z^{k}\sum_{j=1}^{k+1}\sum_{l=1}^{j}g_{n-1,j}P\left\{X_{n}=k-\left(j+l
-1\right)\right\}P\left\{Y_{n}=l\right\}\\
&=&\sum_{k=0}^{\infty}z^{k+\left(j+l-1\right)-\left(j+l-1\right)}\sum_{j=1}^{k+1}\sum_{l=1}^{j}g_{n-1,j}P\left\{X_{n}=k-
\left(j+l-1\right)\right\}P\left\{Y_{n}=l\right\}\\
&=&\sum_{k=0}^{\infty}\sum_{j=1}^{k+1}\sum_{l=1}^{j}g_{n-1,j}z^{j-1}P\left\{X_{n}=k-
\left(j+l-1\right)\right\}z^{k-\left(j+l-1\right)}P\left\{Y_{n}=l\right\}z^{l}\\
\end{eqnarray*}


luego
\begin{eqnarray*}
&=&\sum_{j=1}^{\infty}\sum_{l=1}^{j}g_{n-1,j}z^{j-1}\sum_{k=j+l-1}^{\infty}P\left\{X_{n}=k-
\left(j+l-1\right)\right\}z^{k-\left(j+l-1\right)}P\left\{Y_{n}=l\right\}z^{l}\\
&=&\sum_{j=1}^{\infty}g_{n-1,j}z^{j-1}\sum_{l=1}^{j}\sum_{k=j+l-1}^{\infty}P\left\{X_{n}=k-
\left(j+l-1\right)\right\}z^{k-\left(j+l-1\right)}P\left\{Y_{n}=l\right\}z^{l}\\
&=&\sum_{j=1}^{\infty}g_{n-1,j}z^{j-1}\sum_{k=j+l-1}^{\infty}\sum_{l=1}^{j}P\left\{X_{n}=k-
\left(j+l-1\right)\right\}z^{k-\left(j+l-1\right)}P\left\{Y_{n}=l\right\}z^{l}\\
&=&\sum_{j=1}^{\infty}g_{n-1,j}z^{j-1}\sum_{k=j+l-1}^{\infty}\sum_{l=1}^{j}P\left\{X_{n}=k-
\left(j+l-1\right)\right\}z^{k-\left(j+l-1\right)}\sum_{l=1}^{j}P
\left\{Y_{n}=l\right\}z^{l}\\
&=&\sum_{j=1}^{\infty}g_{n-1,j}z^{j-1}\sum_{l=1}^{\infty}P\left\{Y_{n}=l\right\}z^{l}
\sum_{k=j+l-1}^{\infty}\sum_{l=1}^{j}
P\left\{X_{n}=k-\left(j+l-1\right)\right\}z^{k-\left(j+l-1\right)}\\
&=&\frac{1}{z}\left[G_{n-1}\left(z\right)-G_{n-1}\left(0\right)\right]\tilde{P}\left(z\right)
\sum_{k=j+l-1}^{\infty}\sum_{l=1}^{j}
P\left\{X_{n}=k-\left(j+l-1\right)\right\}z^{k-\left(j+l-1\right)}\\
&=&\frac{1}{z}\left[G_{n-1}\left(z\right)-G_{n-1}\left(0\right)\right]\tilde{P}\left(z\right)P\left(z\right)=\frac{1}{z}\left[G_{n-1}\left(z\right)-G_{n-1}\left(0\right)\right]\tilde{P}\left(z\right),\\
\end{eqnarray*}

es decir la ecuaci\'on (\ref{Eq.3.16.a.2S}) se puede reescribir
como
\begin{equation}\label{Eq.3.16.a.2Sbis}
G_{n}\left(z\right)=\frac{1}{z}\left[G_{n-1}\left(z\right)-G_{n-1}\left(0\right)\right]\tilde{P}\left(z\right).
\end{equation}

Por otra parte recordemos la ecuaci\'on (\ref{Eq.3.16.a.2S})

\begin{eqnarray*}
G_{n}\left(z\right)&=&\sum_{k=0}^{\infty}g_{n,k}z^{k},\textrm{ entonces }\frac{G_{n}\left(z\right)}{z}=\sum_{k=1}^{\infty}g_{n,k}z^{k-1},\\
\end{eqnarray*}

Por lo tanto utilizando la ecuaci\'on (\ref{Eq.3.16.a.2Sbis}):

\begin{eqnarray*}
G\left(z,w\right)&=&\sum_{n=0}^{\infty}G_{n}\left(z\right)w^{n}=G_{0}\left(z\right)+
\sum_{n=1}^{\infty}G_{n}\left(z\right)w^{n}\\
&=&F\left(z\right)+\sum_{n=0}^{\infty}\left[G_{n}\left(z\right)-G_{n}\left(0\right)\right]w^{n}\frac{\tilde{P}\left(z\right)}{z}\\
&=&F\left(z\right)+\frac{w}{z}\sum_{n=0}^{\infty}\left[G_{n}\left(z\right)-G_{n}\left(0\right)\right]w^{n-1}\tilde{P}\left(z\right)\\
\end{eqnarray*}

es decir
\begin{eqnarray*}
G\left(z,w\right)&=&F\left(z\right)+\frac{w}{z}\left[G\left(z,w\right)-G\left(0,w\right)\right]\tilde{P}\left(z\right),
\end{eqnarray*}


entonces

\begin{eqnarray*}
G\left(z,w\right)&=&F\left(z\right)+\frac{w}{z}\left[G\left(z,w\right)-G\left(0,w\right)\right]\tilde{P}\left(z\right)\\
&=&F\left(z\right)+\frac{w}{z}\tilde{P}\left(z\right)G\left(z,w\right)-\frac{w}{z}\tilde{P}\left(z\right)G\left(0,w\right)\\
&\Leftrightarrow&\\
G\left(z,w\right)\left\{1-\frac{w}{z}\tilde{P}\left(z\right)\right\}&=&F\left(z\right)-\frac{w}{z}\tilde{P}\left(z\right)G\left(0,w\right),
\end{eqnarray*}
por lo tanto,
\begin{equation}
G\left(z,w\right)=\frac{zF\left(z\right)-w\tilde{P}\left(z\right)G\left(0,w\right)}{1-w\tilde{P}\left(z\right)}.
\end{equation}


Ahora $G\left(z,w\right)$ es anal\'itica en $|z|=1$.

Sean $z,w$ tales que $|z|=1$ y $|w|\leq1$, como $\tilde{P}\left(z\right)$
es FGP
\begin{eqnarray*}
|z-\left(z-w\tilde{P}\left(z\right)\right)|<|z|\Leftrightarrow|w\tilde{P}\left(z\right)|<|z|
\end{eqnarray*}
es decir, se cumplen las condiciones del Teorema de Rouch\'e y por
tanto, $z$ y $z-w\tilde{P}\left(z\right)$ tienen el mismo n\'umero de
ceros en $|z|=1$. Sea $z=\tilde{\theta}\left(w\right)$ la soluci\'on
\'unica de $z-w\tilde{P}\left(z\right)$, es decir

\begin{equation}\label{Eq.Theta.w}
\tilde{\theta}\left(w\right)-w\tilde{P}\left(\tilde{\theta}\left(w\right)\right)=0,
\end{equation}
 con $|\tilde{\theta}\left(w\right)|<1$. Cabe hacer menci\'on que $\tilde{\theta}\left(w\right)$ es la FGP para el tiempo de ruina cuando $\tilde{L}_{0}=1$.


Considerando la ecuaci\'on (\ref{Eq.Theta.w})
\begin{eqnarray*}
&&\frac{\partial}{\partial w}\tilde{\theta}\left(w\right)|_{w=1}-\frac{\partial}{\partial w}\left\{w\tilde{P}\left(\tilde{\theta}\left(w\right)\right)\right\}|_{w=1}=0\\
&&\tilde{\theta}^{(1)}\left(w\right)|_{w=1}-\frac{\partial}{\partial w}w\left\{\tilde{P}\left(\tilde{\theta}\left(w\right)\right)\right\}|_{w=1}-w\frac{\partial}{\partial w}\tilde{P}\left(\tilde{\theta}\left(w\right)\right)|_{w=1}=0\\
&&\tilde{\theta}^{(1)}\left(1\right)-\tilde{P}\left(\tilde{\theta}\left(1\right)\right)-w\left\{\frac{\partial \tilde{P}\left(\tilde{\theta}\left(w\right)\right)}{\partial \tilde{\theta}\left(w\right)}\cdot\frac{\partial\tilde{\theta}\left(w\right)}{\partial w}|_{w=1}\right\}=0\\
&&\tilde{\theta}^{(1)}\left(1\right)-\tilde{P}\left(\tilde{\theta}\left(1\right)
\right)-\tilde{P}^{(1)}\left(\tilde{\theta}\left(1\right)\right)\cdot\tilde{\theta}^{(1)}\left(1\right)=0
\end{eqnarray*}


luego
\begin{eqnarray*}
&&\tilde{\theta}^{(1)}\left(1\right)-\tilde{P}^{(1)}\left(\tilde{\theta}\left(1\right)\right)\cdot
\tilde{\theta}^{(1)}\left(1\right)=\tilde{P}\left(\tilde{\theta}\left(1\right)\right)\\
&&\tilde{\theta}^{(1)}\left(1\right)\left(1-\tilde{P}^{(1)}\left(\tilde{\theta}\left(1\right)\right)\right)
=\tilde{P}\left(\tilde{\theta}\left(1\right)\right)\\
&&\tilde{\theta}^{(1)}\left(1\right)=\frac{\tilde{P}\left(\tilde{\theta}\left(1\right)\right)}{\left(1-\tilde{P}^{(1)}\left(\tilde{\theta}\left(1\right)\right)\right)}=\frac{1}{1-\tilde{\mu}}.
\end{eqnarray*}

Ahora determinemos el segundo momento de $\tilde{\theta}\left(w\right)$,
nuevamente consideremos la ecuaci\'on (\ref{Eq.Theta.w}):


\begin{eqnarray*}
\tilde{\theta}\left(w\right)-w\tilde{P}\left(\tilde{\theta}\left(w\right)\right)&=&0\\
\frac{\partial}{\partial w}\left\{\tilde{\theta}\left(w\right)-w\tilde{P}\left(\tilde{\theta}\left(w\right)\right)\right\}&=&0\\
\frac{\partial}{\partial w}\left\{\frac{\partial}{\partial w}\left\{\tilde{\theta}\left(w\right)-w\tilde{P}\left(\tilde{\theta}\left(w\right)\right)\right\}\right\}&=&0\\
\end{eqnarray*}
\begin{eqnarray*}
&&\frac{\partial}{\partial w}\left\{\frac{\partial}{\partial w}\tilde{\theta}\left(w\right)-\frac{\partial}{\partial w}\left[w\tilde{P}\left(\tilde{\theta}\left(w\right)\right)\right]\right\}
=\frac{\partial}{\partial w}\left\{\frac{\partial}{\partial w}\tilde{\theta}\left(w\right)-\frac{\partial}{\partial w}\left[w\tilde{P}\left(\tilde{\theta}\left(w\right)\right)\right]\right\}\\
&=&\frac{\partial}{\partial w}\left\{\frac{\partial \tilde{\theta}\left(w\right)}{\partial w}-\left[\tilde{P}\left(\tilde{\theta}\left(w\right)\right)+w\frac{\partial}{\partial w}R\left(\tilde{\theta}\left(w\right)\right)\right]\right\}\\
&=&\frac{\partial}{\partial w}\left\{\frac{\partial \tilde{\theta}\left(w\right)}{\partial w}-\left[\tilde{P}\left(\tilde{\theta}\left(w\right)\right)+w\frac{\partial \tilde{P}\left(\tilde{\theta}\left(w\right)\right)}{\partial w}\frac{\partial \tilde{\theta}\left(w\right)}{\partial w}\right]\right\}\\
&=&\frac{\partial}{\partial w}\left\{\tilde{\theta}^{(1)}\left(w\right)-\tilde{P}\left(\tilde{\theta}\left(w\right)\right)-w\tilde{P}^{(1)}\left(\tilde{\theta}\left(w\right)\right)\tilde{\theta}^{(1)}\left(w\right)\right\}\\
&=&\frac{\partial}{\partial w}\tilde{\theta}^{(1)}\left(w\right)-\frac{\partial}{\partial w}\tilde{P}\left(\tilde{\theta}\left(w\right)\right)-\frac{\partial}{\partial w}\left[w\tilde{P}^{(1)}\left(\tilde{\theta}\left(w\right)\right)\tilde{\theta}^{(1)}\left(w\right)\right]\\
\end{eqnarray*}
\begin{eqnarray*}
&=&\frac{\partial}{\partial
w}\tilde{\theta}^{(1)}\left(w\right)-\frac{\partial
\tilde{P}\left(\tilde{\theta}\left(w\right)\right)}{\partial
\tilde{\theta}\left(w\right)}\frac{\partial \tilde{\theta}\left(w\right)}{\partial
w}-\tilde{P}^{(1)}\left(\tilde{\theta}\left(w\right)\right)\tilde{\theta}^{(1)}\left(w\right)\\
&-&w\frac{\partial
\tilde{P}^{(1)}\left(\tilde{\theta}\left(w\right)\right)}{\partial
w}\tilde{\theta}^{(1)}\left(w\right)-w\tilde{P}^{(1)}\left(\tilde{\theta}\left(w\right)\right)\frac{\partial
\tilde{\theta}^{(1)}\left(w\right)}{\partial w}\\
&=&\tilde{\theta}^{(2)}\left(w\right)-\tilde{P}^{(1)}\left(\tilde{\theta}\left(w\right)\right)\tilde{\theta}^{(1)}\left(w\right)
-\tilde{P}^{(1)}\left(\tilde{\theta}\left(w\right)\right)\tilde{\theta}^{(1)}\left(w\right)\\
&-&w\tilde{P}^{(2)}\left(\tilde{\theta}\left(w\right)\right)\left(\tilde{\theta}^{(1)}\left(w\right)\right)^{2}-w\tilde{P}^{(1)}\left(\tilde{\theta}\left(w\right)\right)\tilde{\theta}^{(2)}\left(w\right)\\
&=&\tilde{\theta}^{(2)}\left(w\right)-2\tilde{P}^{(1)}\left(\tilde{\theta}\left(w\right)\right)\tilde{\theta}^{(1)}\left(w\right)\\
&-&w\tilde{P}^{(2)}\left(\tilde{\theta}\left(w\right)\right)\left(\tilde{\theta}^{(1)}\left(w\right)\right)^{2}-w\tilde{P}^{(1)}\left(\tilde{\theta}\left(w\right)\right)\tilde{\theta}^{(2)}\left(w\right)\\
&=&\tilde{\theta}^{(2)}\left(w\right)\left[1-w\tilde{P}^{(1)}\left(\tilde{\theta}\left(w\right)\right)\right]-
\tilde{\theta}^{(1)}\left(w\right)\left[w\tilde{\theta}^{(1)}\left(w\right)\tilde{P}^{(2)}\left(\tilde{\theta}\left(w\right)\right)+2\tilde{P}^{(1)}\left(\tilde{\theta}\left(w\right)\right)\right]
\end{eqnarray*}
luego



\begin{eqnarray*}
\tilde{\theta}^{(2)}\left(w\right)\left[1-w\tilde{P}^{(1)}\left(\tilde{\theta}\left(w\right)\right)\right]&-&\tilde{\theta}^{(1)}\left(w\right)\left[w\tilde{\theta}^{(1)}\left(w\right)\tilde{P}^{(2)}\left(\tilde{\theta}\left(w\right)\right)
+2\tilde{P}^{(1)}\left(\tilde{\theta}\left(w\right)\right)\right]=0\\
\tilde{\theta}^{(2)}\left(w\right)&=&\frac{\tilde{\theta}^{(1)}\left(w\right)\left[w\tilde{\theta}^{(1)}\left(w\right)\tilde{P}^{(2)}\left(\tilde{\theta}\left(w\right)\right)+2R^{(1)}\left(\tilde{\theta}\left(w\right)\right)\right]}{1-w\tilde{P}^{(1)}\left(\tilde{\theta}\left(w\right)\right)}\\
\tilde{\theta}^{(2)}\left(w\right)&=&\frac{\tilde{\theta}^{(1)}\left(w\right)w\tilde{\theta}^{(1)}\left(w\right)\tilde{P}^{(2)}\left(\tilde{\theta}\left(w\right)\right)}{1-w\tilde{P}^{(1)}\left(\tilde{\theta}\left(w\right)\right)}+\frac{2\tilde{\theta}^{(1)}\left(w\right)\tilde{P}^{(1)}\left(\tilde{\theta}\left(w\right)\right)}{1-w\tilde{P}^{(1)}\left(\tilde{\theta}\left(w\right)\right)}
\end{eqnarray*}


si evaluamos la expresi\'on anterior en $w=1$:
\begin{eqnarray*}
\tilde{\theta}^{(2)}\left(1\right)&=&\frac{\left(\tilde{\theta}^{(1)}\left(1\right)\right)^{2}\tilde{P}^{(2)}\left(\tilde{\theta}\left(1\right)\right)}{1-\tilde{P}^{(1)}\left(\tilde{\theta}\left(1\right)\right)}+\frac{2\tilde{\theta}^{(1)}\left(1\right)\tilde{P}^{(1)}\left(\tilde{\theta}\left(1\right)\right)}{1-\tilde{P}^{(1)}\left(\tilde{\theta}\left(1\right)\right)}\\
&=&\frac{\left(\tilde{\theta}^{(1)}\left(1\right)\right)^{2}\tilde{P}^{(2)}\left(1\right)}{1-\tilde{P}^{(1)}\left(1\right)}+\frac{2\tilde{\theta}^{(1)}\left(1\right)\tilde{P}^{(1)}\left(1\right)}{1-\tilde{P}^{(1)}\left(1\right)}\\
&=&\frac{\left(\frac{1}{1-\tilde{\mu}}\right)^{2}\tilde{P}^{(2)}\left(1\right)}{1-\tilde{\mu}}+\frac{2\left(\frac{1}{1-\tilde{\mu}}\right)\tilde{\mu}}{1-\tilde{\mu}}=\frac{\tilde{P}^{(2)}\left(1\right)}{\left(1-\tilde{\mu}\right)^{3}}+\frac{2\tilde{\mu}}{\left(1-\tilde{\mu}\right)^{2}}\\
\end{eqnarray*}

luego

\begin{eqnarray*}
&=&\frac{\sigma^{2}-\tilde{\mu}+\tilde{\mu}^{2}}{\left(1-\tilde{\mu}\right)^{3}}+\frac{2\tilde{\mu}}{\left(1-\tilde{\mu}\right)^{2}}=\frac{\sigma^{2}-\tilde{\mu}+\tilde{\mu}^{2}+2\tilde{\mu}\left(1-\tilde{\mu}\right)}{\left(1-\tilde{\mu}\right)^{3}}\\
\end{eqnarray*}


es decir
\begin{eqnarray*}
\tilde{\theta}^{(2)}\left(1\right)&=&\frac{\sigma^{2}+\tilde{\mu}-\tilde{\mu}^{2}}{\left(1-\tilde{\mu}\right)^{3}}=\frac{\sigma^{2}}{\left(1-\tilde{\mu}\right)^{3}}+\frac{\tilde{\mu}\left(1-\tilde{\mu}\right)}{\left(1-\tilde{\mu}\right)^{3}}\\
&=&\frac{\sigma^{2}}{\left(1-\tilde{\mu}\right)^{3}}+\frac{\tilde{\mu}}{\left(1-\tilde{\mu}\right)^{2}}.
\end{eqnarray*}

\begin{Coro}
El tiempo de ruina del jugador tiene primer y segundo momento
dados por

\begin{eqnarray}
\esp\left[T\right]&=&\frac{\esp\left[\tilde{L}_{0}\right]}{1-\tilde{\mu}}\\
Var\left[T\right]&=&\frac{Var\left[\tilde{L}_{0}\right]}{\left(1-\tilde{\mu}\right)^{2}}+\frac{\sigma^{2}\esp\left[\tilde{L}_{0}\right]}{\left(1-\tilde{\mu}\right)^{3}}.
\end{eqnarray}
\end{Coro}





Ahora, determinemos la distribuci\'on del n\'umero de usuarios que
pasan de $\hat{Q}_{2}$ a $Q_{2}$ considerando dos pol\'iticas de
traslado en espec\'ifico:

\begin{enumerate}
\item Solamente pasa un usuario,

\item Se permite el paso de $k$ usuarios,
\end{enumerate}
una vez que son atendidos por el servidor en $\hat{Q}_{2}$.

\begin{description}


\item[Pol\'itica de un solo usuario:] Sea $R_{2}$ el n\'umero de
usuarios que llegan a $\hat{Q}_{2}$ al tiempo $t$, sea $R_{1}$ el
n\'umero de usuarios que pasan de $\hat{Q}_{2}$ a $Q_{2}$ al
tiempo $t$.
\end{description}


A saber:
\begin{eqnarray*}
\esp\left[R_{1}\right]&=&\sum_{y\geq0}\prob\left[R_{2}=y\right]\esp\left[R_{1}|R_{2}=y\right]\\
&=&\sum_{y\geq0}\prob\left[R_{2}=y\right]\sum_{x\geq0}x\prob\left[R_{1}=x|R_{2}=y\right]\\
&=&\sum_{y\geq0}\sum_{x\geq0}x\prob\left[R_{1}=x|R_{2}=y\right]\prob\left[R_{2}=y\right].\\
\end{eqnarray*}

Determinemos
\begin{equation}
\esp\left[R_{1}|R_{2}=y\right]=\sum_{x\geq0}x\prob\left[R_{1}=x|R_{2}=y\right].
\end{equation}

supongamos que $y=0$, entonces
\begin{eqnarray*}
\prob\left[R_{1}=0|R_{2}=0\right]&=&1,\\
\prob\left[R_{1}=x|R_{2}=0\right]&=&0,\textrm{ para cualquier }x\geq1,\\
\end{eqnarray*}


por tanto
\begin{eqnarray*}
\esp\left[R_{1}|R_{2}=0\right]=0.
\end{eqnarray*}

Para $y=1$,
\begin{eqnarray*}
\prob\left[R_{1}=0|R_{2}=1\right]&=&0,\\
\prob\left[R_{1}=1|R_{2}=1\right]&=&1,
\end{eqnarray*}

entonces
\begin{eqnarray*}
\esp\left[R_{1}|R_{2}=1\right]=1.
\end{eqnarray*}

Para $y>1$:
\begin{eqnarray*}
\prob\left[R_{1}=0|R_{2}\geq1\right]&=&0,\\
\prob\left[R_{1}=1|R_{2}\geq1\right]&=&1,\\
\prob\left[R_{1}>1|R_{2}\geq1\right]&=&0,
\end{eqnarray*}

entonces
\begin{eqnarray*}
\esp\left[R_{1}|R_{2}=y\right]=1,\textrm{ para cualquier }y>1.
\end{eqnarray*}
es decir
\begin{eqnarray*}
\esp\left[R_{1}|R_{2}=y\right]=1,\textrm{ para cualquier }y\geq1.
\end{eqnarray*}

Entonces
\begin{eqnarray*}
\esp\left[R_{1}\right]&=&\sum_{y\geq0}\sum_{x\geq0}x\prob\left[R_{1}=x|R_{2}=y\right]\prob\left[R_{2}=y\right]=\sum_{y\geq0}\sum_{x}\esp\left[R_{1}|R_{2}=y\right]\prob\left[R_{2}=y\right]\\
&=&\sum_{y\geq0}\prob\left[R_{2}=y\right]=\sum_{y\geq1}\frac{\left(\lambda
t\right)^{k}}{k!}e^{-\lambda t}=1.
\end{eqnarray*}

Adem\'as para $k\in Z^{+}$
\begin{eqnarray*}
f_{R_{1}}\left(k\right)&=&\prob\left[R_{1}=k\right]=\sum_{n=0}^{\infty}\prob\left[R_{1}=k|R_{2}=n\right]\prob\left[R_{2}=n\right]\\
&=&\prob\left[R_{1}=k|R_{2}=0\right]\prob\left[R_{2}=0\right]+\prob\left[R_{1}=k|R_{2}=1\right]\prob\left[R_{2}=1\right]\\
&+&\prob\left[R_{1}=k|R_{2}>1\right]\prob\left[R_{2}>1\right],
\end{eqnarray*}

donde para


\begin{description}
\item[$k=0$:]
\begin{eqnarray*}
\prob\left[R_{1}=0\right]=\prob\left[R_{1}=0|R_{2}=0\right]\prob\left[R_{2}=0\right]+\prob\left[R_{1}=0|R_{2}=1\right]\prob\left[R_{2}=1\right]\\
+\prob\left[R_{1}=0|R_{2}>1\right]\prob\left[R_{2}>1\right]=\prob\left[R_{2}=0\right].
\end{eqnarray*}
\item[$k=1$:]
\begin{eqnarray*}
\prob\left[R_{1}=1\right]=\prob\left[R_{1}=1|R_{2}=0\right]\prob\left[R_{2}=0\right]+\prob\left[R_{1}=1|R_{2}=1\right]\prob\left[R_{2}=1\right]\\
+\prob\left[R_{1}=1|R_{2}>1\right]\prob\left[R_{2}>1\right]=\sum_{n=1}^{\infty}\prob\left[R_{2}=n\right].
\end{eqnarray*}

\item[$k=2$:]
\begin{eqnarray*}
\prob\left[R_{1}=2\right]=\prob\left[R_{1}=2|R_{2}=0\right]\prob\left[R_{2}=0\right]+\prob\left[R_{1}=2|R_{2}=1\right]\prob\left[R_{2}=1\right]\\
+\prob\left[R_{1}=2|R_{2}>1\right]\prob\left[R_{2}>1\right]=0.
\end{eqnarray*}

\item[$k=j$:]
\begin{eqnarray*}
\prob\left[R_{1}=j\right]=\prob\left[R_{1}=j|R_{2}=0\right]\prob\left[R_{2}=0\right]+\prob\left[R_{1}=j|R_{2}=1\right]\prob\left[R_{2}=1\right]\\
+\prob\left[R_{1}=j|R_{2}>1\right]\prob\left[R_{2}>1\right]=0.
\end{eqnarray*}
\end{description}


Por lo tanto
\begin{eqnarray*}
f_{R_{1}}\left(0\right)&=&\prob\left[R_{2}=0\right]\\
f_{R_{1}}\left(1\right)&=&\sum_{n\geq1}^{\infty}\prob\left[R_{2}=n\right]\\
f_{R_{1}}\left(j\right)&=&0,\textrm{ para }j>1.
\end{eqnarray*}



\begin{description}
\item[Pol\'itica de $k$ usuarios:]Al igual que antes, para $y\in Z^{+}$ fijo
\begin{eqnarray*}
\esp\left[R_{1}|R_{2}=y\right]=\sum_{x}x\prob\left[R_{1}=x|R_{2}=y\right].\\
\end{eqnarray*}
\end{description}
Entonces, si tomamos diversos valore para $y$:\\

$y=0$:
\begin{eqnarray*}
\prob\left[R_{1}=0|R_{2}=0\right]&=&1,\\
\prob\left[R_{1}=x|R_{2}=0\right]&=&0,\textrm{ para cualquier }x\geq1,
\end{eqnarray*}

entonces
\begin{eqnarray*}
\esp\left[R_{1}|R_{2}=0\right]=0.
\end{eqnarray*}


Para $y=1$,
\begin{eqnarray*}
\prob\left[R_{1}=0|R_{2}=1\right]&=&0,\\
\prob\left[R_{1}=1|R_{2}=1\right]&=&1,
\end{eqnarray*}

entonces {\scriptsize{
\begin{eqnarray*}
\esp\left[R_{1}|R_{2}=1\right]=1.
\end{eqnarray*}}}


Para $y=2$,
\begin{eqnarray*}
\prob\left[R_{1}=0|R_{2}=2\right]&=&0,\\
\prob\left[R_{1}=1|R_{2}=2\right]&=&1,\\
\prob\left[R_{1}=2|R_{2}=2\right]&=&1,\\
\prob\left[R_{1}=3|R_{2}=2\right]&=&0,
\end{eqnarray*}

entonces
\begin{eqnarray*}
\esp\left[R_{1}|R_{2}=2\right]=3.
\end{eqnarray*}

Para $y=3$,
\begin{eqnarray*}
\prob\left[R_{1}=0|R_{2}=3\right]&=&0,\\
\prob\left[R_{1}=1|R_{2}=3\right]&=&1,\\
\prob\left[R_{1}=2|R_{2}=3\right]&=&1,\\
\prob\left[R_{1}=3|R_{2}=3\right]&=&1,\\
\prob\left[R_{1}=4|R_{2}=3\right]&=&0,
\end{eqnarray*}

entonces
\begin{eqnarray*}
\esp\left[R_{1}|R_{2}=3\right]=6.
\end{eqnarray*}

En general, para $k\geq0$,
\begin{eqnarray*}
\prob\left[R_{1}=0|R_{2}=k\right]&=&0,\\
\prob\left[R_{1}=j|R_{2}=k\right]&=&1,\textrm{ para }1\leq j\leq k,\\
\prob\left[R_{1}=j|R_{2}=k\right]&=&0,\textrm{ para }j> k,
\end{eqnarray*}

entonces
\begin{eqnarray*}
\esp\left[R_{1}|R_{2}=k\right]=\frac{k\left(k+1\right)}{2}.
\end{eqnarray*}



Por lo tanto


\begin{eqnarray*}
\esp\left[R_{1}\right]&=&\sum_{y}\esp\left[R_{1}|R_{2}=y\right]\prob\left[R_{2}=y\right]\\
&=&\sum_{y}\prob\left[R_{2}=y\right]\frac{y\left(y+1\right)}{2}=\sum_{y\geq1}\left(\frac{y\left(y+1\right)}{2}\right)\frac{\left(\lambda t\right)^{y}}{y!}e^{-\lambda t}\\
&=&\frac{\lambda t}{2}e^{-\lambda t}\sum_{y\geq1}\left(y+1\right)\frac{\left(\lambda t\right)^{y-1}}{\left(y-1\right)!}=\frac{\lambda t}{2}e^{-\lambda t}\left(e^{\lambda t}\left(\lambda t+2\right)\right)\\
&=&\frac{\lambda t\left(\lambda t+2\right)}{2},
\end{eqnarray*}
es decir,


\begin{equation}
\esp\left[R_{1}\right]=\frac{\lambda t\left(\lambda
t+2\right)}{2}.
\end{equation}

Adem\'as para $k\in Z^{+}$ fijo
\begin{eqnarray*}
f_{R_{1}}\left(k\right)&=&\prob\left[R_{1}=k\right]=\sum_{n=0}^{\infty}\prob\left[R_{1}=k|R_{2}=n\right]\prob\left[R_{2}=n\right]\\
&=&\prob\left[R_{1}=k|R_{2}=0\right]\prob\left[R_{2}=0\right]+\prob\left[R_{1}=k|R_{2}=1\right]\prob\left[R_{2}=1\right]\\
&+&\prob\left[R_{1}=k|R_{2}=2\right]\prob\left[R_{2}=2\right]+\cdots+\prob\left[R_{1}=k|R_{2}=j\right]\prob\left[R_{2}=j\right]+\cdots+
\end{eqnarray*}
donde para

\begin{description}
\item[$k=0$:]
\begin{eqnarray*}
\prob\left[R_{1}=0\right]=\prob\left[R_{1}=0|R_{2}=0\right]\prob\left[R_{2}=0\right]+\prob\left[R_{1}=0|R_{2}=1\right]\prob\left[R_{2}=1\right]\\
+\prob\left[R_{1}=0|R_{2}=j\right]\prob\left[R_{2}=j\right]=\prob\left[R_{2}=0\right].
\end{eqnarray*}
\item[$k=1$:]
\begin{eqnarray*}
\prob\left[R_{1}=1\right]=\prob\left[R_{1}=1|R_{2}=0\right]\prob\left[R_{2}=0\right]+\prob\left[R_{1}=1|R_{2}=1\right]\prob\left[R_{2}=1\right]\\
+\prob\left[R_{1}=1|R_{2}=1\right]\prob\left[R_{2}=1\right]+\cdots+\prob\left[R_{1}=1|R_{2}=j\right]\prob\left[R_{2}=j\right]\\
=\sum_{n=1}^{\infty}\prob\left[R_{2}=n\right].
\end{eqnarray*}

\item[$k=2$:]
\begin{eqnarray*}
\prob\left[R_{1}=2\right]=\prob\left[R_{1}=2|R_{2}=0\right]\prob\left[R_{2}=0\right]+\prob\left[R_{1}=2|R_{2}=1\right]\prob\left[R_{2}=1\right]\\
+\prob\left[R_{1}=2|R_{2}=2\right]\prob\left[R_{2}=2\right]+\cdots+\prob\left[R_{1}=2|R_{2}=j\right]\prob\left[R_{2}=j\right]\\
=\sum_{n=2}^{\infty}\prob\left[R_{2}=n\right].
\end{eqnarray*}
\end{description}

En general

\begin{eqnarray*}
\prob\left[R_{1}=k\right]=\prob\left[R_{1}=k|R_{2}=0\right]\prob\left[R_{2}=0\right]+\prob\left[R_{1}=k|R_{2}=1\right]\prob\left[R_{2}=1\right]\\
+\prob\left[R_{1}=k|R_{2}=2\right]\prob\left[R_{2}=2\right]+\cdots+\prob\left[R_{1}=k|R_{2}=k\right]\prob\left[R_{2}=k\right]\\
=\sum_{n=k}^{\infty}\prob\left[R_{2}=n\right].\\
\end{eqnarray*}



Por lo tanto

\begin{eqnarray*}
f_{R_{1}}\left(k\right)&=&\prob\left[R_{1}=k\right]=\sum_{n=k}^{\infty}\prob\left[R_{2}=n\right].
\end{eqnarray*}

%__________________________________________________________________________
\section{Descripci\'on de una Red de S.V.C.}
%__________________________________________________________________________

Se definen los procesos de llegada de los usuarios a cada una de
las colas dependiendo de la llegada del servidor pero del sistema
al cu\'al no pertenece la cola en cuesti\'on:

Para el sistema 1 y el servidor del segundo sistema

\begin{eqnarray*}
F_{1,1}\left(z_{1};\zeta_{1}\right)&=&\esp\left[z_{1}^{L_{1}\left(\zeta_{1}\right)}\right]=
\sum_{k=0}^{\infty}\prob\left[L_{1}\left(\zeta_{1}\right)=k\right]z_{1}^{k}\\
F_{2,1}\left(z_{2};\zeta_{1}\right)&=&\esp\left[z_{2}^{L_{2}\left(\zeta_{1}\right)}\right]=
\sum_{k=0}^{\infty}\prob\left[L_{2}\left(\zeta_{1}\right)=k\right]z_{2}^{k}\\
F_{1,2}\left(z_{1};\zeta_{2}\right)&=&\esp\left[z_{1}^{L_{1}\left(\zeta_{2}\right)}\right]=
\sum_{k=0}^{\infty}\prob\left[L_{1}\left(\zeta_{2}\right)=k\right]z_{1}^{k}\\
F_{2,2}\left(z_{2};\zeta_{2}\right)&=&\esp\left[z_{2}^{L_{2}\left(\zeta_{2}\right)}\right]=
\sum_{k=0}^{\infty}\prob\left[L_{2}\left(\zeta_{2}\right)=k\right]z_{2}^{k}\\
\end{eqnarray*}

Ahora se definen para el segundo sistema y el servidor del primero


\begin{eqnarray*}
\hat{F}_{1,1}\left(w_{1};\tau_{1}\right)&=&\esp\left[w_{1}^{\hat{L}_{1}\left(\tau_{1}\right)}\right] =\sum_{k=0}^{\infty}\prob\left[\hat{L}_{1}\left(\tau_{1}\right)=k\right]w_{1}^{k}\\
\hat{F}_{2,1}\left(w_{2};\tau_{1}\right)&=&\esp\left[w_{2}^{\hat{L}_{2}\left(\tau_{1}\right)}\right] =\sum_{k=0}^{\infty}\prob\left[\hat{L}_{2}\left(\tau_{1}\right)=k\right]w_{2}^{k}\\
\hat{F}_{1,2}\left(w_{1};\tau_{2}\right)&=&\esp\left[w_{1}^{\hat{L}_{1}\left(\tau_{2}\right)}\right]
=\sum_{k=0}^{\infty}\prob\left[\hat{L}_{1}\left(\tau_{2}\right)=k\right]w_{1}^{k}\\
\hat{F}_{2,2}\left(w_{2};\tau_{2}\right)&=&\esp\left[w_{2}^{\hat{L}_{2}\left(\tau_{2}\right)}\right]
=\sum_{k=0}^{\infty}\prob\left[\hat{L}_{2}\left(\tau_{2}\right)=k\right]w_{2}^{k}\\
\end{eqnarray*}


Ahora, con lo anterior definamos la FGP conjunta para el segundo sistema y $\tau_{1}$:


\begin{eqnarray*}
\esp\left[w_{1}^{\hat{L}_{1}\left(\tau_{1}\right)}w_{2}^{\hat{L}_{2}\left(\tau_{1}\right)}\right]
&=&\esp\left[w_{1}^{\hat{L}_{1}\left(\tau_{1}\right)}\right]
\esp\left[w_{2}^{\hat{L}_{2}\left(\tau_{1}\right)}\right]=\hat{F}_{1,1}\left(w_{1};\tau_{1}\right)\hat{F}_{2,1}\left(w_{2};\tau_{1}\right)\\
&=&\hat{F}_{1}\left(w_{1},w_{2};\tau_{1}\right).
\end{eqnarray*}
hagamos lo mismo para $\tau_{2}$


\begin{eqnarray*}
\esp\left[w_{1}^{\hat{L}_{1}\left(\tau_{2}\right)}w_{2}^{\hat{L}_{2}\left(\tau_{2}\right)}\right]
&=&\esp\left[w_{1}^{\hat{L}_{1}\left(\tau_{2}\right)}\right]
\esp\left[w_{2}^{\hat{L}_{2}\left(\tau_{2}\right)}\right]=\hat{F}_{1,2}\left(w_{1};\tau_{2}\right)\hat{F}_{2,2}\left(w_{2};\tau_{2}\right)\\
&=&\hat{F}_{2}\left(w_{1},w_{2};\tau_{2}\right).
\end{eqnarray*}

Con respecto al sistema 1 se tiene la FGP conjunta con respecto a $\zeta_{1}$:
\begin{eqnarray*}
\esp\left[z_{1}^{L_{1}\left(\zeta_{1}\right)}z_{2}^{L_{2}\left(\zeta_{1}\right)}\right]
&=&\esp\left[z_{1}^{L_{1}\left(\zeta_{1}\right)}\right]
\esp\left[z_{2}^{L_{2}\left(\zeta_{1}\right)}\right]=F_{1,1}\left(z_{1};\zeta_{1}\right)F_{2,1}\left(z_{2};\zeta_{1}\right)\\
&=&F_{1}\left(z_{1},z_{2};\zeta_{1}\right).
\end{eqnarray*}

Finalmente
\begin{eqnarray*}
\esp\left[z_{1}^{L_{1}\left(\zeta_{2}\right)}z_{2}^{L_{2}\left(\zeta_{2}\right)}\right]
&=&\esp\left[z_{1}^{L_{1}\left(\zeta_{2}\right)}\right]
\esp\left[z_{2}^{L_{2}\left(\zeta_{2}\right)}\right]=F_{1,2}\left(z_{1};\zeta_{2}\right)F_{2,2}\left(z_{2};\zeta_{2}\right)\\
&=&F_{2}\left(z_{1},z_{2};\zeta_{2}\right).
\end{eqnarray*}

Ahora analicemos la Red de Sistemas de Visitas C\'iclicas, entonces se define la PGF conjunta al tiempo $t$ para los tiempos de visita del servidor en cada una de las colas, para comenzar a dar servicio, definidos anteriormente al tiempo
$t=\left\{\tau_{1},\tau_{2},\zeta_{1},\zeta_{2}\right\}$:

\begin{eqnarray}\label{Eq.Conjuntas}
F_{1}\left(z_{1},z_{2},w_{1},w_{2}\right)&=&\esp\left[z_{1}^{L_{1}\left(\tau_{1}\right)}z_{2}^{L_{2}\left(\tau_{1}\right)}w_{1}^{\hat{L}_{1}\left(\tau_{1}\right)}w_{2}^{\hat{L}_{2}\left(\tau_{1}\right)}\right]\\
F_{2}\left(z_{1},z_{2},w_{1},w_{2}\right)&=&\esp\left[z_{1}^{L_{1}\left(\tau_{2}\right)}z_{2}^{L_{2}\left(\tau_{2}\right)}w_{1}^{\hat{L}_{1}\left(\tau_{2}\right)}w_{2}^{\hat{L}_{2}\left(\tau_{2}\right)}\right]\\
\hat{F}_{1}\left(z_{1},z_{2},w_{1},w_{2}\right)&=&\esp\left[z_{1}^{L_{1}\left(\zeta_{1}\right)}z_{2}^{L_{2}\left(\zeta_{1}\right)}w_{1}^{\hat{L}_{1}\left(\zeta_{1}\right)}w_{2}^{\hat{L}_{2}\left(\zeta_{1}\right)}\right]\\
\hat{F}_{2}\left(z_{1},z_{2},w_{1},w_{2}\right)&=&\esp\left[z_{1}^{L_{1}\left(\zeta_{2}\right)}z_{2}^{L_{2}\left(\zeta_{2}\right)}w_{1}^{\hat{L}_{1}\left(\zeta_{2}\right)}w_{2}^{\hat{L}_{2}\left(\zeta_{2}\right)}\right]
\end{eqnarray}

Entonces, con la finalidad de encontrar el n\'umero de usuarios
presentes en el sistema cuando el servidor deja de atender una de
las colas de cualquier sistema se tiene lo siguiente


\begin{eqnarray*}
&&\esp\left[z_{1}^{L_{1}\left(\overline{\tau}_{1}\right)}z_{2}^{L_{2}\left(\overline{\tau}_{1}\right)}w_{1}^{\hat{L}_{1}\left(\overline{\tau}_{1}\right)}w_{2}^{\hat{L}_{2}\left(\overline{\tau}_{1}\right)}\right]=
\esp\left[z_{2}^{L_{2}\left(\overline{\tau}_{1}\right)}w_{1}^{\hat{L}_{1}\left(\overline{\tau}_{1}\right)}w_{2}^{\hat{L}_{2}\left(\overline{\tau}_{1}\right)}\right]\\
&=&\esp\left[z_{2}^{L_{2}\left(\tau_{1}\right)+X_{2}\left(\overline{\tau}_{1}-\tau_{1}\right)+Y_{2}\left(\overline{\tau}_{1}-\tau_{1}\right)}w_{1}^{\hat{L}_{1}\left(\tau_{1}\right)+\hat{X}_{1}\left(\overline{\tau}_{1}-\tau_{1}\right)}w_{2}^{\hat{L}_{2}\left(\tau_{1}\right)+\hat{X}_{2}\left(\overline{\tau}_{1}-\tau_{1}\right)}\right]
\end{eqnarray*}
utilizando la ecuacion dada (\ref{Eq.TiemposLlegada}), luego


\begin{eqnarray*}
&=&\esp\left[z_{2}^{L_{2}\left(\tau_{1}\right)}z_{2}^{X_{2}\left(\overline{\tau}_{1}-\tau_{1}\right)}z_{2}^{Y_{2}\left(\overline{\tau}_{1}-\tau_{1}\right)}w_{1}^{\hat{L}_{1}\left(\tau_{1}\right)}w_{1}^{\hat{X}_{1}\left(\overline{\tau}_{1}-\tau_{1}\right)}w_{2}^{\hat{L}_{2}\left(\tau_{1}\right)}w_{2}^{\hat{X}_{2}\left(\overline{\tau}_{1}-\tau_{1}\right)}\right]\\
&=&\esp\left[z_{2}^{L_{2}\left(\tau_{1}\right)}\left\{w_{1}^{\hat{L}_{1}\left(\tau_{1}\right)}w_{2}^{\hat{L}_{2}\left(\tau_{1}\right)}\right\}\left\{z_{2}^{X_{2}\left(\overline{\tau}_{1}-\tau_{1}\right)}
z_{2}^{Y_{2}\left(\overline{\tau}_{1}-\tau_{1}\right)}w_{1}^{\hat{X}_{1}\left(\overline{\tau}_{1}-\tau_{1}\right)}w_{2}^{\hat{X}_{2}\left(\overline{\tau}_{1}-\tau_{1}\right)}\right\}\right]\\
\end{eqnarray*}
Aplicando la ecuaci\'on (\ref{Eq.Cero})

\begin{eqnarray*}
&=&\esp\left[z_{2}^{L_{2}\left(\tau_{1}\right)}\left\{z_{2}^{X_{2}\left(\overline{\tau}_{1}-\tau_{1}\right)}z_{2}^{Y_{2}\left(\overline{\tau}_{1}-\tau_{1}\right)}w_{1}^{\hat{X}_{1}\left(\overline{\tau}_{1}-\tau_{1}\right)}w_{2}^{\hat{X}_{2}\left(\overline{\tau}_{1}-\tau_{1}\right)}\right\}\right]\esp\left[w_{1}^{\hat{L}_{1}\left(\tau_{1}\right)}w_{2}^{\hat{L}_{2}\left(\tau_{1}\right)}\right]
\end{eqnarray*}
dado que los arribos a cada una de las colas son independientes, podemos separar la esperanza para los procesos de llegada a $Q_{1}$ y $Q_{2}$ en $\tau_{1}$

Recordando que $\tilde{X}_{2}\left(z_{2}\right)=X_{2}\left(z_{2}\right)+Y_{2}\left(z_{2}\right)$ se tiene


\begin{eqnarray*}
&=&\esp\left[z_{2}^{L_{2}\left(\tau_{1}\right)}\left\{z_{2}^{\tilde{X}_{2}\left(\overline{\tau}_{1}-\tau_{1}\right)}w_{1}^{\hat{X}_{1}\left(\overline{\tau}_{1}-\tau_{1}\right)}w_{2}^{\hat{X}_{2}\left(\overline{\tau}_{1}-\tau_{1}\right)}\right\}\right]\esp\left[w_{1}^{\hat{L}_{1}\left(\tau_{1}\right)}w_{2}^{\hat{L}_{2}\left(\tau_{1}\right)}\right]\\
&=&\esp\left[z_{2}^{L_{2}\left(\tau_{1}\right)}\left\{\tilde{P}_{2}\left(z_{2}\right)^{\overline{\tau}_{1}-\tau_{1}}\hat{P}_{1}\left(w_{1}\right)^{\overline{\tau}_{1}-\tau_{1}}\hat{P}_{2}\left(w_{2}\right)^{\overline{\tau}_{1}-\tau_{1}}\right\}\right]\esp\left[w_{1}^{\hat{L}_{1}\left(\tau_{1}\right)}w_{2}^{\hat{L}_{2}\left(\tau_{1}\right)}\right]\\
&=&\esp\left[z_{2}^{L_{2}\left(\tau_{1}\right)}\left\{\tilde{P}_{2}\left(z_{2}\right)\hat{P}_{1}\left(w_{1}\right)\hat{P}_{2}\left(w_{2}\right)\right\}^{\overline{\tau}_{1}-\tau_{1}}\right]\esp\left[w_{1}^{\hat{L}_{1}\left(\tau_{1}\right)}w_{2}^{\hat{L}_{2}\left(\tau_{1}\right)}\right]\\
\end{eqnarray*}

Entonces


\begin{eqnarray*}
&=&\esp\left[z_{2}^{L_{2}\left(\tau_{1}\right)}\theta_{1}\left(\tilde{P}_{2}\left(z_{2}\right)\hat{P}_{1}\left(w_{1}\right)\hat{P}_{2}\left(w_{2}\right)\right)^{L_{1}\left(\tau_{1}\right)}\right]\esp\left[w_{1}^{\hat{L}_{1}\left(\tau_{1}\right)}w_{2}^{\hat{L}_{2}\left(\tau_{1}\right)}\right]\\
&=&F_{1}\left(\theta_{1}\left(\tilde{P}_{2}\left(z_{2}\right)\hat{P}_{1}\left(w_{1}\right)\hat{P}_{2}\left(w_{2}\right)\right),z{2}\right)\hat{F}_{1}\left(w_{1},w_{2};\tau_{1}\right)\\
&\equiv&
F_{1}\left(\theta_{1}\left(\tilde{P}_{2}\left(z_{2}\right)\hat{P}_{1}\left(w_{1}\right)\hat{P}_{2}\left(w_{2}\right)\right),z_{2},w_{1},w_{2}\right)
\end{eqnarray*}

Las igualdades anteriores son ciertas pues el n\'umero de usuarios
que llegan a $\hat{Q}_{2}$ durante el intervalo
$\left[\tau_{1},\overline{\tau}_{1}\right]$ a\'un no han sido
atendidos por el servidor del sistema $2$ y por tanto a\'un no
pueden pasar al sistema $1$ por $Q_{2}$. Por tanto el n\'umero de
usuarios que pasan de $\hat{Q}_{2}$ a $Q_{2}$ en el intervalo de
tiempo $\left[\tau_{1},\overline{\tau}_{1}\right]$ depende de la
pol\'itica de traslado entre los dos sistemas, conforme a la
secci\'on anterior.\smallskip

Por lo tanto
\begin{equation}\label{Eq.Fs}
\esp\left[z_{1}^{L_{1}\left(\overline{\tau}_{1}\right)}z_{2}^{L_{2}\left(\overline{\tau}_{1}\right)}w_{1}^{\hat{L}_{1}\left(\overline{\tau}_{1}\right)}w_{2}^{\hat{L}_{2}\left(\overline{\tau}_{1}\right)}\right]=F_{1}\left(\theta_{1}\left(\tilde{P}_{2}\left(z_{2}\right)\hat{P}_{1}\left(w_{1}\right)\hat{P}_{2}\left(w_{2}\right)\right),z_{2},w_{1},w_{2}\right)
\end{equation}


Utilizando un razonamiento an\'alogo para $\overline{\tau}_{2}$:



\begin{eqnarray*}
&&\esp\left[z_{1}^{L_{1}\left(\overline{\tau}_{2}\right)}z_{2}^{L_{2}\left(\overline{\tau}_{2}\right)}w_{1}^{\hat{L}_{1}\left(\overline{\tau}_{2}\right)}w_{2}^{\hat{L}_{2}\left(\overline{\tau}_{2}\right)}\right]=
\esp\left[z_{1}^{L_{1}\left(\overline{\tau}_{2}\right)}w_{1}^{\hat{L}_{1}\left(\overline{\tau}_{2}\right)}w_{2}^{\hat{L}_{2}\left(\overline{\tau}_{2}\right)}\right]\\
&=&\esp\left[z_{1}^{L_{1}\left(\tau_{2}\right)+X_{1}\left(\overline{\tau}_{2}-\tau_{2}\right)}w_{1}^{\hat{L}_{1}\left(\tau_{2}\right)+\hat{X}_{1}\left(\overline{\tau}_{2}-\tau_{2}\right)}w_{2}^{\hat{L}_{2}\left(\tau_{2}\right)+\hat{X}_{2}\left(\overline{\tau}_{2}-\tau_{2}\right)}\right]\\
&=&\esp\left[z_{1}^{L_{1}\left(\tau_{2}\right)}z_{1}^{X_{1}\left(\overline{\tau}_{2}-\tau_{2}\right)}w_{1}^{\hat{L}_{1}\left(\tau_{2}\right)}w_{1}^{\hat{X}_{1}\left(\overline{\tau}_{2}-\tau_{2}\right)}w_{2}^{\hat{L}_{2}\left(\tau_{2}\right)}w_{2}^{\hat{X}_{2}\left(\overline{\tau}_{2}-\tau_{2}\right)}\right]\\
&=&\esp\left[z_{1}^{L_{1}\left(\tau_{2}\right)}z_{1}^{X_{1}\left(\overline{\tau}_{2}-\tau_{2}\right)}w_{1}^{\hat{X}_{1}\left(\overline{\tau}_{2}-\tau_{2}\right)}w_{2}^{\hat{X}_{2}\left(\overline{\tau}_{2}-\tau_{2}\right)}\right]\esp\left[w_{1}^{\hat{L}_{1}\left(\tau_{2}\right)}w_{2}^{\hat{L}_{2}\left(\tau_{2}\right)}\right]\\
&=&\esp\left[z_{1}^{L_{1}\left(\tau_{2}\right)}P_{1}\left(z_{1}\right)^{\overline{\tau}_{2}-\tau_{2}}\hat{P}_{1}\left(w_{1}\right)^{\overline{\tau}_{2}-\tau_{2}}\hat{P}_{2}\left(w_{2}\right)^{\overline{\tau}_{2}-\tau_{2}}\right]
\esp\left[w_{1}^{\hat{L}_{1}\left(\tau_{2}\right)}w_{2}^{\hat{L}_{2}\left(\tau_{2}\right)}\right]
\end{eqnarray*}
utlizando la proposici\'on relacionada con la ruina del jugador


\begin{eqnarray*}
&=&\esp\left[z_{1}^{L_{1}\left(\tau_{2}\right)}\left\{P_{1}\left(z_{1}\right)\hat{P}_{1}\left(w_{1}\right)\hat{P}_{2}\left(w_{2}\right)\right\}^{\overline{\tau}_{2}-\tau_{2}}\right]
\esp\left[w_{1}^{\hat{L}_{1}\left(\tau_{2}\right)}w_{2}^{\hat{L}_{2}\left(\tau_{2}\right)}\right]\\
&=&\esp\left[z_{1}^{L_{1}\left(\tau_{2}\right)}\tilde{\theta}_{2}\left(P_{1}\left(z_{1}\right)\hat{P}_{1}\left(w_{1}\right)\hat{P}_{2}\left(w_{2}\right)\right)^{L_{2}\left(\tau_{2}\right)}\right]
\esp\left[w_{1}^{\hat{L}_{1}\left(\tau_{2}\right)}w_{2}^{\hat{L}_{2}\left(\tau_{2}\right)}\right]\\
&=&F_{2}\left(z_{1},\tilde{\theta}_{2}\left(P_{1}\left(z_{1}\right)\hat{P}_{1}\left(w_{1}\right)\hat{P}_{2}\left(w_{2}\right)\right)\right)
\hat{F}_{2}\left(w_{1},w_{2};\tau_{2}\right)\\
\end{eqnarray*}


entonces se define
\begin{eqnarray}
\esp\left[z_{1}^{L_{1}\left(\overline{\tau}_{2}\right)}z_{2}^{L_{2}\left(\overline{\tau}_{2}\right)}w_{1}^{\hat{L}_{1}\left(\overline{\tau}_{2}\right)}w_{2}^{\hat{L}_{2}\left(\overline{\tau}_{2}\right)}\right]=F_{2}\left(z_{1},\tilde{\theta}_{2}\left(P_{1}\left(z_{1}\right)\hat{P}_{1}\left(w_{1}\right)\hat{P}_{2}\left(w_{2}\right)\right),w_{1},w_{2}\right)\\
\equiv F_{2}\left(z_{1},\tilde{\theta}_{2}\left(P_{1}\left(z_{1}\right)\hat{P}_{1}\left(w_{1}\right)\hat{P}_{2}\left(w_{2}\right)\right)\right)
\hat{F}_{2}\left(w_{1},w_{2};\tau_{2}\right)
\end{eqnarray}
Ahora para $\overline{\zeta}_{1}:$
\begin{eqnarray*}
&&\esp\left[z_{1}^{L_{1}\left(\overline{\zeta}_{1}\right)}z_{2}^{L_{2}\left(\overline{\zeta}_{1}\right)}w_{1}^{\hat{L}_{1}\left(\overline{\zeta}_{1}\right)}w_{2}^{\hat{L}_{2}\left(\overline{\zeta}_{1}\right)}\right]=
\esp\left[z_{1}^{L_{1}\left(\overline{\zeta}_{1}\right)}z_{2}^{L_{2}\left(\overline{\zeta}_{1}\right)}w_{2}^{\hat{L}_{2}\left(\overline{\zeta}_{1}\right)}\right]\\
%&=&\esp\left[z_{1}^{L_{1}\left(\zeta_{1}\right)+X_{1}\left(\overline{\zeta}_{1}-\zeta_{1}\right)}z_{2}^{L_{2}\left(\zeta_{1}\right)+X_{2}\left(\overline{\zeta}_{1}-\zeta_{1}\right)+\hat{Y}_{2}\left(\overline{\zeta}_{1}-\zeta_{1}\right)}w_{2}^{\hat{L}_{2}\left(\zeta_{1}\right)+\hat{X}_{2}\left(\overline{\zeta}_{1}-\zeta_{1}\right)}\right]\\
&=&\esp\left[z_{1}^{L_{1}\left(\zeta_{1}\right)}z_{1}^{X_{1}\left(\overline{\zeta}_{1}-\zeta_{1}\right)}z_{2}^{L_{2}\left(\zeta_{1}\right)}z_{2}^{X_{2}\left(\overline{\zeta}_{1}-\zeta_{1}\right)}
z_{2}^{Y_{2}\left(\overline{\zeta}_{1}-\zeta_{1}\right)}w_{2}^{\hat{L}_{2}\left(\zeta_{1}\right)}w_{2}^{\hat{X}_{2}\left(\overline{\zeta}_{1}-\zeta_{1}\right)}\right]\\
&=&\esp\left[z_{1}^{L_{1}\left(\zeta_{1}\right)}z_{2}^{L_{2}\left(\zeta_{1}\right)}\right]\esp\left[z_{1}^{X_{1}\left(\overline{\zeta}_{1}-\zeta_{1}\right)}z_{2}^{\tilde{X}_{2}\left(\overline{\zeta}_{1}-\zeta_{1}\right)}w_{2}^{\hat{X}_{2}\left(\overline{\zeta}_{1}-\zeta_{1}\right)}w_{2}^{\hat{L}_{2}\left(\zeta_{1}\right)}\right]\\
&=&\esp\left[z_{1}^{L_{1}\left(\zeta_{1}\right)}z_{2}^{L_{2}\left(\zeta_{1}\right)}\right]
\esp\left[P_{1}\left(z_{1}\right)^{\overline{\zeta}_{1}-\zeta_{1}}\tilde{P}_{2}\left(z_{2}\right)^{\overline{\zeta}_{1}-\zeta_{1}}\hat{P}_{2}\left(w_{2}\right)^{\overline{\zeta}_{1}-\zeta_{1}}w_{2}^{\hat{L}_{2}\left(\zeta_{1}\right)}\right]\\
&=&\esp\left[z_{1}^{L_{1}\left(\zeta_{1}\right)}z_{2}^{L_{2}\left(\zeta_{1}\right)}\right]
\esp\left[\left\{P_{1}\left(z_{1}\right)\tilde{P}_{2}\left(z_{2}\right)\hat{P}_{2}\left(w_{2}\right)\right\}^{\overline{\zeta}_{1}-\zeta_{1}}w_{2}^{\hat{L}_{2}\left(\zeta_{1}\right)}\right]\\
&=&\esp\left[z_{1}^{L_{1}\left(\zeta_{1}\right)}z_{2}^{L_{2}\left(\zeta_{1}\right)}\right]
\esp\left[\hat{\theta}_{1}\left(P_{1}\left(z_{1}\right)\tilde{P}_{2}\left(z_{2}\right)\hat{P}_{2}\left(w_{2}\right)\right)^{\hat{L}_{1}\left(\zeta_{1}\right)}w_{2}^{\hat{L}_{2}\left(\zeta_{1}\right)}\right]\\
&=&F_{1}\left(z_{1},z_{2};\zeta_{1}\right)\hat{F}_{1}\left(\hat{\theta}_{1}\left(P_{1}\left(z_{1}\right)\tilde{P}_{2}\left(z_{2}\right)\hat{P}_{2}\left(w_{2}\right)\right),w_{2}\right)
\end{eqnarray*}


es decir
\begin{eqnarray}
\esp\left[z_{1}^{L_{1}\left(\overline{\zeta}_{1}\right)}z_{2}^{L_{2}\left(\overline{\zeta}_{1}\right)}w_{1}^{\hat{L}_{1}\left(\overline{\zeta}_{1}\right)}w_{2}^{\hat{L}_{2}\left(\overline{\zeta}_{1}\right)}\right]=\hat{F}_{1}\left(z_{1},z_{2},\hat{\theta}_{1}\left(P_{1}\left(z_{1}\right)\tilde{P}_{2}\left(z_{2}\right)\hat{P}_{2}\left(w_{2}\right)\right),w_{2}\right)\\
&=&F_{1}\left(z_{1},z_{2};\zeta_{1}\right)\hat{F}_{1}\left(\hat{\theta}_{1}\left(P_{1}\left(z_{1}\right)\tilde{P}_{2}\left(z_{2}\right)\hat{P}_{2}\left(w_{2}\right)\right),w_{2}\right).
\end{eqnarray}


Finalmente para $\overline{\zeta}_{2}:$
\begin{eqnarray*}
&&\esp\left[z_{1}^{L_{1}\left(\overline{\zeta}_{2}\right)}z_{2}^{L_{2}\left(\overline{\zeta}_{2}\right)}w_{1}^{\hat{L}_{1}\left(\overline{\zeta}_{2}\right)}w_{2}^{\hat{L}_{2}\left(\overline{\zeta}_{2}\right)}\right]=
\esp\left[z_{1}^{L_{1}\left(\overline{\zeta}_{2}\right)}z_{2}^{L_{2}\left(\overline{\zeta}_{2}\right)}w_{1}^{\hat{L}_{1}\left(\overline{\zeta}_{2}\right)}\right]\\
%&=&\esp\left[z_{1}^{L_{1}\left(\zeta_{2}\right)+X_{1}\left(\overline{\zeta}_{2}-\zeta_{2}\right)}z_{2}^{L_{2}\left(\zeta_{2}\right)+X_{2}\left(\overline{\zeta}_{2}-\zeta_{2}\right)+\hat{Y}_{2}\left(\overline{\zeta}_{2}-\zeta_{2}\right)}w_{1}^{\hat{L}_{1}\left(\zeta_{2}\right)+\hat{X}_{1}\left(\overline{\zeta}_{2}-\zeta_{2}\right)}\right]\\
&=&\esp\left[z_{1}^{L_{1}\left(\zeta_{2}\right)}z_{1}^{X_{1}\left(\overline{\zeta}_{2}-\zeta_{2}\right)}z_{2}^{L_{2}\left(\zeta_{2}\right)}z_{2}^{X_{2}\left(\overline{\zeta}_{2}-\zeta_{2}\right)}
z_{2}^{Y_{2}\left(\overline{\zeta}_{2}-\zeta_{2}\right)}w_{1}^{\hat{L}_{1}\left(\zeta_{2}\right)}w_{1}^{\hat{X}_{1}\left(\overline{\zeta}_{2}-\zeta_{2}\right)}\right]\\
&=&\esp\left[z_{1}^{L_{1}\left(\zeta_{2}\right)}z_{2}^{L_{2}\left(\zeta_{2}\right)}\right]\esp\left[z_{1}^{X_{1}\left(\overline{\zeta}_{2}-\zeta_{2}\right)}z_{2}^{\tilde{X}_{2}\left(\overline{\zeta}_{2}-\zeta_{2}\right)}w_{1}^{\hat{X}_{1}\left(\overline{\zeta}_{2}-\zeta_{2}\right)}w_{1}^{\hat{L}_{1}\left(\zeta_{2}\right)}\right]\\
&=&\esp\left[z_{1}^{L_{1}\left(\zeta_{2}\right)}z_{2}^{L_{2}\left(\zeta_{2}\right)}\right]\esp\left[P_{1}\left(z_{1}\right)^{\overline{\zeta}_{2}-\zeta_{2}}\tilde{P}_{2}\left(z_{2}\right)^{\overline{\zeta}_{2}-\zeta_{2}}\hat{P}\left(w_{1}\right)^{\overline{\zeta}_{2}-\zeta_{2}}w_{1}^{\hat{L}_{1}\left(\zeta_{2}\right)}\right]\\
&=&\esp\left[z_{1}^{L_{1}\left(\zeta_{2}\right)}z_{2}^{L_{2}\left(\zeta_{2}\right)}\right]\esp\left[w_{1}^{\hat{L}_{1}\left(\zeta_{2}\right)}\left\{P_{1}\left(z_{1}\right)\tilde{P}_{2}\left(z_{2}\right)\hat{P}\left(w_{1}\right)\right\}^{\overline{\zeta}_{2}-\zeta_{2}}\right]\\
&=&\esp\left[z_{1}^{L_{1}\left(\zeta_{2}\right)}z_{2}^{L_{2}\left(\zeta_{2}\right)}\right]\esp\left[w_{1}^{\hat{L}_{1}\left(\zeta_{2}\right)}\hat{\theta}_{2}\left(P_{1}\left(z_{1}\right)\tilde{P}_{2}\left(z_{2}\right)\hat{P}\left(w_{1}\right)\right)^{\hat{L}_{2}\zeta_{2}}\right]\\
&=&F_{2}\left(z_{1},z_{2};\zeta_{2}\right)\hat{F}_{2}\left(w_{1},\hat{\theta}_{2}\left(P_{1}\left(z_{1}\right)\tilde{P}_{2}\left(z_{2}\right)\hat{P}_{1}\left(w_{1}\right)\right)\right]\\
%&\equiv&\hat{F}_{2}\left(z_{1},z_{2},w_{1},\hat{\theta}_{2}\left(P_{1}\left(z_{1}\right)\tilde{P}_{2}\left(z_{2}\right)\hat{P}_{1}\left(w_{1}\right)\right)\right)
\end{eqnarray*}


%__________________________________________________________________________
\section{Ecuaciones Recursivas para la R.S.V.C.}
%__________________________________________________________________________


es decir
\begin{eqnarray}
\esp\left[z_{1}^{L_{1}\left(\overline{\zeta}_{2}\right)}z_{2}^{L_{2}\left(\overline{\zeta}_{2}\right)}w_{1}^{\hat{L}_{1}\left(\overline{\zeta}_{2}\right)}w_{2}^{\hat{L}_{2}\left(\overline{\zeta}_{2}\right)}\right]=\hat{F}_{2}\left(z_{1},z_{2},w_{1},\hat{\theta}_{2}\left(P_{1}\left(z_{1}\right)\tilde{P}_{2}\left(z_{2}\right)\hat{P}_{1}\left(w_{1}\right)\right)\right)\\
=F_{2}\left(z_{1},z_{2};\zeta_{2}\right)\hat{F}_{2}\left(w_{1},\hat{\theta}_{2}\left(P_{1}\left(z_{1}\right)\tilde{P}_{2}\left(z_{2}\right)\hat{P}_{1}\left(w_{1}\right)\right)\right]\\
\end{eqnarray}

Con lo desarrollado hasta ahora podemos encontrar las ecuaciones
recursivas que modelan la Red de Sistemas de Visitas C\'iclicas
(R.S.V.C):
\begin{eqnarray*}
&&F_{2}\left(z_{1},z_{2},w_{1},w_{2}\right)=R_{1}\left(z_{1},z_{2},w_{1},w_{2}\right)\esp\left[z_{1}^{L_{1}\left(\overline{\tau}_{1}\right)}z_{2}^{L_{2}\left(\overline{\tau}_{1}\right)}w_{1}^{\hat{L}_{1}\left(\overline{\tau}_{1}\right)}w_{2}^{\hat{L}_{2}\left(\overline{\tau}_{1}\right)}\right]\\
%&=&R_{1}\left(P_{1}\left(z_{1}\right)\tilde{P}_{2}\left(z_{2}\right)\hat{P}_{1}\left(w_{1}\right)\hat{P}_{2}\left(w_{2}\right)\right)
%F_{1}\left(\theta\left(\tilde{P}_{2}\left(z_{2}\right)\hat{P}_{1}\left(w_{1}\right)\hat{P}_{2}\left(w_{2}\right)\right),z_{2},w_{1},w_{2}\right)\\
&&F_{1}\left(z_{1},z_{2},w_{1},w_{2}\right)=R_{2}\left(z_{1},z_{2},w_{1},w_{2}\right)\esp\left[z_{1}^{L_{1}\left(\overline{\tau}_{2}\right)}z_{2}^{L_{2}\left(\overline{\tau}_{2}\right)}w_{1}^{\hat{L}_{1}\left(\overline{\tau}_{2}\right)}w_{2}^{\hat{L}_{2}\left(\overline{\tau}_{1}\right)}\right]\\
%&=&R_{2}\left(P_{1}\left(z_{1}\right)\tilde{P}_{2}\left(z_{2}\right)\hat{P}_{1}\left(w_{1}\right)\hat{P}_{2}\left(w_{2}\right)\right)F_{2}\left(z_{1},\tilde{\theta}_{2}\left(P_{1}\left(z_{1}\right)\hat{P}_{1}\left(w_{1}\right)\hat{P}_{2}\left(w_{2}\right)\right),w_{1},w_{2}\right)\\
&&\hat{F}_{2}\left(z_{1},z_{2},w_{1},w_{2}\right)=\hat{R}_{1}\left(z_{1},z_{2},w_{1},w_{2}\right)\esp\left[z_{1}^{L_{1}\left(\overline{\zeta}_{1}\right)}z_{2}^{L_{2}\left(\overline{\zeta}_{1}\right)}w_{1}^{\hat{L}_{1}\left(\overline{\zeta}_{1}\right)}w_{2}^{\hat{L}_{2}\left(\overline{\zeta}_{1}\right)}\right]\\
%&=&\hat{R}_{1}\left(P_{1}\left(z_{1}\right)\tilde{P}_{2}\left(z_{2}\right)\hat{P}_{1}\left(w_{1}\right)\hat{P}_{2}\left(w_{2}\right)\right)\hat{F}_{1}\left(z_{1},z_{2},\hat{\theta}_{1}\left(P_{1}\left(z_{1}\right)\tilde{P}_{2}\left(z_{2}\right)\hat{P}_{2}\left(w_{2}\right)\right),w_{2}\right)
\end{eqnarray*}


y finalmente
\begin{eqnarray*}
&&\hat{F}_{1}\left(z_{1},z_{2},w_{1},w_{2}\right)=\hat{R}_{2}\left(z_{1},z_{2},w_{1},w_{2}\right)\esp\left[z_{1}^{L_{1}\left(\overline{\zeta}_{2}\right)}z_{2}^{L_{2}\left(\overline{\zeta}_{2}\right)}w_{1}^{\hat{L}_{1}\left(\overline{\zeta}_{2}\right)}w_{2}^{\hat{L}_{2}\left(\overline{\zeta}_{2}\right)}\right]\\
%&=&\hat{R}_{2}\left(P_{1}\left(z_{1}\right)\tilde{P}_{2}\left(z_{2}\right)\hat{P}_{1}\left(w_{1}\right)\hat{P}_{2}\left(w_{2}\right)\right)\hat{F}_{2}\left(z_{1},z_{2},w_{1},\hat{\theta}_{2}\left(P_{1}\left(z_{1}\right)\tilde{P}_{2}\left(z_{2}\right)\hat{P}_{1}\left(w_{1}\right)\right)\right)
\end{eqnarray*}

que son equivalentes a las siguientes ecuaciones
\begin{eqnarray*}
F_{2}\left(z_{1},z_{2},w_{1},w_{2}\right)&=&R_{1}\left(P_{1}\left(z_{1}\right)\tilde{P}_{2}\left(z_{2}\right)\prod_{i=1}^{2}
\hat{P}_{i}\left(w_{i}\right)\right)\\
&&F_{1}\left(\theta_{1}\left(\tilde{P}_{2}\left(z_{2}\right)\hat{P}_{1}\left(w_{1}\right)\hat{P}_{2}\left(w_{2}\right)\right),z_{2},w_{1},w_{2}\right)\\
\end{eqnarray*}


\begin{eqnarray*}
F_{1}\left(z_{1},z_{2},w_{1},w_{2}\right)&=&R_{2}\left(P_{1}\left(z_{1}\right)\tilde{P}_{2}\left(z_{2}\right)\prod_{i=1}^{2}
\hat{P}_{i}\left(w_{i}\right)\right)\\
&&F_{2}\left(z_{1},\tilde{\theta}_{2}\left(P_{1}\left(z_{1}\right)\hat{P}_{1}\left(w_{1}\right)\hat{P}_{2}\left(w_{2}\right)\right),w_{1},w_{2}\right)\\
\end{eqnarray*}

%_________________________________________________________________________________________________
\subsection{Tiempos de Traslado del Servidor}
%_________________________________________________________________________________________________



\begin{eqnarray*}
\hat{F}_{2}\left(z_{1},z_{2},w_{1},w_{2}\right)&=&\hat{R}_{1}\left(P_{1}\left(z_{1}\right)\tilde{P}_{2}\left(z_{2}\right)\prod_{i=1}^{2}
\hat{P}_{i}\left(w_{i}\right)\right)\\
&&\hat{F}_{1}\left(z_{1},z_{2},\hat{\theta}_{1}\left(P_{1}\left(z_{1}\right)\tilde{P}_{2}\left(z_{2}\right)\hat{P}_{2}\left(w_{2}\right)\right),w_{2}\right)\\
\end{eqnarray*}

\begin{eqnarray*}
\hat{F}_{1}\left(z_{1},z_{2},w_{1},w_{2}\right)&=&\hat{R}_{2}\left(P_{1}\left(z_{1}\right)\tilde{P}_{2}\left(z_{2}\right)\prod_{i=1}^{2}
\hat{P}_{i}\left(w_{i}\right)\right)\\
&&\hat{F}_{2}\left(z_{1},z_{2},w_{1},\hat{\theta}_{2}\left(P_{1}\left(z_{1}\right)\tilde{P}_{2}\left(z_{2}\right)\hat{P}_{1}\left(w_{1}\right)\right)\right)
\end{eqnarray*}


Para
%\begin{multicols}{2}

\begin{eqnarray}\label{Ec.R1}
R_{1}\left(\mathbf{z,w}\right)=R_{1}\left(P_{1}\left(z_{1}\right)\tilde{P}_{2}\left(z_{2}\right)\hat{P}_{1}\left(w_{1}\right)\hat{P}_{2}\left(w_{2}\right)\right)
\end{eqnarray}
%\end{multicols}

se tiene que


\begin{eqnarray*}
\frac{\partial R_{1}\left(\mathbf{z,w}\right)}{\partial
z_{1}}|_{\mathbf{z,w}=1}&=&R_{1}^{(1)}\left(1\right)P_{1}^{(1)}\left(1\right)=r_{1}\mu_{1},\\
\frac{\partial R_{1}\left(\mathbf{z,w}\right)}{\partial
z_{2}}|_{\mathbf{z,w}=1}&=&R_{1}^{(1)}\left(1\right)\tilde{P}_{2}^{(1)}\left(1\right)=r_{1}\tilde{\mu}_{2},\\
\frac{\partial R_{1}\left(\mathbf{z,w}\right)}{\partial
w_{1}}|_{\mathbf{z,w}=1}&=&R_{1}^{(1)}\left(1\right)\hat{P}_{1}^{(1)}\left(1\right)=r_{1}\hat{\mu}_{1},\\
\frac{\partial R_{1}\left(\mathbf{z,w}\right)}{\partial
w_{2}}|_{\mathbf{z,w}=1}&=&R_{1}^{(1)}\left(1\right)\hat{P}_{2}^{(1)}\left(1\right)=r_{1}\hat{\mu}_{2},
\end{eqnarray*}

An\'alogamente se tiene

\begin{eqnarray}
R_{2}\left(\mathbf{z,w}\right)=R_{2}\left(P_{1}\left(z_{1}\right)\tilde{P}_{2}\left(z_{2}\right)\hat{P}_{1}\left(w_{1}\right)\hat{P}_{2}\left(w_{2}\right)\right)
\end{eqnarray}


\begin{eqnarray*}
\frac{\partial R_{2}\left(\mathbf{z,w}\right)}{\partial
z_{1}}|_{\mathbf{z,w}=1}&=&R_{2}^{(1)}\left(1\right)P_{1}^{(1)}\left(1\right)=r_{2}\mu_{1},\\
\frac{\partial R_{2}\left(\mathbf{z,w}\right)}{\partial
z_{2}}|_{\mathbf{z,w}=1}&=&R_{2}^{(1)}\left(1\right)\tilde{P}_{2}^{(1)}\left(1\right)=r_{2}\tilde{\mu}_{2},\\
\frac{\partial R_{2}\left(\mathbf{z,w}\right)}{\partial
w_{1}}|_{\mathbf{z,w}=1}&=&R_{2}^{(1)}\left(1\right)\hat{P}_{1}^{(1)}\left(1\right)=r_{2}\hat{\mu}_{1},\\
\frac{\partial R_{2}\left(\mathbf{z,w}\right)}{\partial
w_{2}}|_{\mathbf{z,w}=1}&=&R_{2}^{(1)}\left(1\right)\hat{P}_{2}^{(1)}\left(1\right)=r_{2}\hat{\mu}_{2},\\
\end{eqnarray*}

Para el segundo sistema:

\begin{eqnarray}
\hat{R}_{1}\left(\mathbf{z,w}\right)=\hat{R}_{1}\left(P_{1}\left(z_{1}\right)\tilde{P}_{2}\left(z_{2}\right)\hat{P}_{1}\left(w_{1}\right)\hat{P}_{2}\left(w_{2}\right)\right)
\end{eqnarray}


\begin{eqnarray*}
\frac{\partial \hat{R}_{1}\left(\mathbf{z,w}\right)}{\partial
z_{1}}|_{\mathbf{z,w}=1}&=&\hat{R}_{1}^{(1)}\left(1\right)P_{1}^{(1)}\left(1\right)=\hat{r}_{1}\mu_{1},\\
\frac{\partial \hat{R}_{1}\left(\mathbf{z,w}\right)}{\partial
z_{2}}|_{\mathbf{z,w}=1}&=&\hat{R}_{1}^{(1)}\left(1\right)\tilde{P}_{2}^{(1)}\left(1\right)=\hat{r}_{1}\tilde{\mu}_{2},\\
\frac{\partial \hat{R}_{1}\left(\mathbf{z,w}\right)}{\partial
w_{1}}|_{\mathbf{z,w}=1}&=&\hat{R}_{1}^{(1)}\left(1\right)\hat{P}_{1}^{(1)}\left(1\right)=\hat{r}_{1}\hat{\mu}_{1},\\
\frac{\partial \hat{R}_{1}\left(\mathbf{z,w}\right)}{\partial
w_{2}}|_{\mathbf{z,w}=1}&=&\hat{R}_{1}^{(1)}\left(1\right)\hat{P}_{2}^{(1)}\left(1\right)=\hat{r}_{1}\hat{\mu}_{2},
\end{eqnarray*}

Finalmente

\begin{eqnarray}
\hat{R}_{2}\left(\mathbf{z,w}\right)=\hat{R}_{2}\left(P_{1}\left(z_{1}\right)\tilde{P}_{2}\left(z_{2}\right)\hat{P}_{1}\left(w_{1}\right)\hat{P}_{2}\left(w_{2}\right)\right)
\end{eqnarray}



\begin{eqnarray*}
\frac{\partial \hat{R}_{2}\left(\mathbf{z,w}\right)}{\partial
z_{1}}|_{\mathbf{z,w}=1}&=&\hat{R}_{2}^{(1)}\left(1\right)P_{1}^{(1)}\left(1\right)=\hat{r}_{2}\mu_{1},\\
\frac{\partial \hat{R}_{2}\left(\mathbf{z,w}\right)}{\partial
z_{2}}|_{\mathbf{z,w}=1}&=&\hat{R}_{2}^{(1)}\left(1\right)\tilde{P}_{2}^{(1)}\left(1\right)=\hat{r}_{2}\tilde{\mu}_{2},\\
\frac{\partial \hat{R}_{2}\left(\mathbf{z,w}\right)}{\partial
w_{1}}|_{\mathbf{z,w}=1}&=&\hat{R}_{2}^{(1)}\left(1\right)\hat{P}_{1}^{(1)}\left(1\right)=\hat{r}_{2}\hat{\mu}_{1},\\
\frac{\partial \hat{R}_{2}\left(\mathbf{z,w}\right)}{\partial
w_{2}}|_{\mathbf{z,w}=1}&=&\hat{R}_{2}^{(1)}\left(1\right)\hat{P}_{2}^{(1)}\left(1\right)
=\hat{r}_{2}\hat{\mu}_{2}.
\end{eqnarray*}


%_________________________________________________________________________________________________
\subsection{Usuarios presentes en la cola}
%_________________________________________________________________________________________________

Hagamos lo correspondiente con las siguientes
expresiones obtenidas en la secci\'on anterior:
Recordemos que

\begin{eqnarray*}
F_{1}\left(\theta_{1}\left(\tilde{P}_{2}\left(z_{2}\right)\hat{P}_{1}\left(w_{1}\right)
\hat{P}_{2}\left(w_{2}\right)\right),z_{2},w_{1},w_{2}\right)&=&
F_{1}\left(\theta_{1}\left(\tilde{P}_{2}\left(z_{2}\right)\hat{P}_{1}\left(w_{1}\right)\hat{P}_{2}\left(w_{2}\right)\right),z{2}\right)\\
&&\hat{F}_{1}\left(w_{1},w_{2};\tau_{1}\right)
\end{eqnarray*}

entonces

\begin{eqnarray*}
\frac{\partial F_{1}\left(\theta_{1}\left(\tilde{P}_{2}\left(z_{2}\right)\hat{P}_{1}\left(w_{1}\right)\hat{P}_{2}\left(w_{2}\right)\right),z_{2},w_{1},w_{2}\right)}{\partial z_{1}}|_{\mathbf{z},\mathbf{w}=1}&=&0\\
\frac{\partial
F_{1}\left(\theta_{1}\left(\tilde{P}_{2}\left(z_{2}\right)\hat{P}_{1}\left(w_{1}\right)\hat{P}_{2}\left(w_{2}\right)\right),z_{2},w_{1},w_{2}\right)}{\partial
z_{2}}|_{\mathbf{z},\mathbf{w}=1}&=&\frac{\partial F_{1}}{\partial
z_{1}}\cdot\frac{\partial \theta_{1}}{\partial
\tilde{P}_{2}}\cdot\frac{\partial \tilde{P}_{2}}{\partial
z_{2}}+\frac{\partial F_{1}}{\partial z_{2}}
\\
\frac{\partial
F_{1}\left(\theta_{1}\left(\tilde{P}_{2}\left(z_{2}\right)\hat{P}_{1}\left(w_{1}\right)\hat{P}_{2}\left(w_{2}\right)\right),z_{2},w_{1},w_{2}\right)}{\partial
w_{1}}|_{\mathbf{z},\mathbf{w}=1}&=&\frac{\partial F_{1}}{\partial
z_{1}}\cdot\frac{\partial
\theta_{1}}{\partial\hat{P}_{1}}\cdot\frac{\partial\hat{P}_{1}}{\partial
w_{1}}+\frac{\partial\hat{F}_{1}}{\partial w_{1}}
\\
\frac{\partial
F_{1}\left(\theta_{1}\left(\tilde{P}_{2}\left(z_{2}\right)\hat{P}_{1}\left(w_{1}\right)\hat{P}_{2}\left(w_{2}\right)\right),z_{2},w_{1},w_{2}\right)}{\partial
w_{2}}|_{\mathbf{z},\mathbf{w}=1}&=&\frac{\partial F_{1}}{\partial
z_{1}}\cdot\frac{\partial\theta_{1}}{\partial\hat{P}_{2}}\cdot\frac{\partial\hat{P}_{2}}{\partial
w_{2}}+\frac{\partial \hat{F}_{1}}{\partial w_{2}}
\\
\end{eqnarray*}

para $\tau_{2}$:

\begin{eqnarray*}
F_{2}\left(z_{1},\tilde{\theta}_{2}\left(P_{1}\left(z_{1}\right)\hat{P}_{1}\left(w_{1}\right)\hat{P}_{2}\left(w_{2}\right)\right),
w_{1},w_{2}\right)&=&F_{2}\left(z_{1},\tilde{\theta}_{2}\left(P_{1}\left(z_{1}\right)\hat{P}_{1}\left(w_{1}\right)\hat{P}_{2}\left(w_{2}\right)\right)\right)\\
&&\hat{F}_{2}\left(w_{1},w_{2};\tau_{2}\right)
\end{eqnarray*}
al igual que antes

\begin{eqnarray*}
\frac{\partial
F_{2}\left(z_{1},\tilde{\theta}_{2}\left(P_{1}\left(z_{1}\right)\hat{P}_{1}\left(w_{1}\right)\hat{P}_{2}\left(w_{2}\right)\right),w_{1},w_{2}\right)}{\partial
z_{1}}|_{\mathbf{z},\mathbf{w}=1}&=&\frac{\partial F_{2}}{\partial
z_{2}}\cdot\frac{\partial\tilde{\theta}_{2}}{\partial
P_{1}}\cdot\frac{\partial P_{1}}{\partial z_{2}}+\frac{\partial
F_{2}}{\partial z_{1}}
\\
\frac{\partial F_{2}\left(z_{1},\tilde{\theta}_{2}\left(P_{1}\left(z_{1}\right)\hat{P}_{1}\left(w_{1}\right)\hat{P}_{2}\left(w_{2}\right)\right),w_{1},w_{2}\right)}{\partial z_{2}}|_{\mathbf{z},\mathbf{w}=1}&=&0\\
\frac{\partial
F_{2}\left(z_{1},\tilde{\theta}_{2}\left(P_{1}\left(z_{1}\right)\hat{P}_{1}\left(w_{1}\right)\hat{P}_{2}\left(w_{2}\right)\right),w_{1},w_{2}\right)}{\partial
w_{1}}|_{\mathbf{z},\mathbf{w}=1}&=&\frac{\partial F_{2}}{\partial
z_{2}}\cdot\frac{\partial \tilde{\theta}_{2}}{\partial
\hat{P}_{1}}\cdot\frac{\partial \hat{P}_{1}}{\partial
w_{1}}+\frac{\partial \hat{F}_{2}}{\partial w_{1}}
\\
\frac{\partial
F_{2}\left(z_{1},\tilde{\theta}_{2}\left(P_{1}\left(z_{1}\right)\hat{P}_{1}\left(w_{1}\right)\hat{P}_{2}\left(w_{2}\right)\right),w_{1},w_{2}\right)}{\partial
w_{2}}|_{\mathbf{z},\mathbf{w}=1}&=&\frac{\partial F_{2}}{\partial
z_{2}}\cdot\frac{\partial
\tilde{\theta}_{2}}{\partial\hat{P}_{2}}\cdot\frac{\partial\hat{P}_{2}}{\partial
w_{2}}+\frac{\partial\hat{F}_{2}}{\partial w_{2}}
\\
\end{eqnarray*}


Ahora para el segundo sistema

\begin{eqnarray*}\hat{F}_{1}\left(z_{1},z_{2},\hat{\theta}_{1}\left(P_{1}\left(z_{1}\right)\tilde{P}_{2}\left(z_{2}\right)\hat{P}_{2}\left(w_{2}\right)\right),
w_{2}\right)&=&F_{1}\left(z_{1},z_{2};\zeta_{1}\right)\\
&&\hat{F}_{1}\left(\hat{\theta}_{1}\left(P_{1}\left(z_{1}\right)\tilde{P}_{2}\left(z_{2}\right)
\hat{P}_{2}\left(w_{2}\right)\right),w_{2}\right)
\end{eqnarray*}
entonces


\begin{eqnarray*}
\frac{\partial
\hat{F}_{1}\left(z_{1},z_{2},\hat{\theta}_{1}\left(P_{1}\left(z_{1}\right)\tilde{P}_{2}\left(z_{2}\right)\hat{P}_{2}\left(w_{2}\right)\right),w_{2}\right)}{\partial
z_{1}}|_{\mathbf{z},\mathbf{w}=1}&=&\frac{\partial \hat{F}_{1}
}{\partial w_{1}}\cdot\frac{\partial\hat{\theta}_{1}}{\partial
P_{1}}\cdot\frac{\partial P_{1}}{\partial z_{1}}+\frac{\partial
F_{1}}{\partial z_{1}}
\\
\frac{\partial
\hat{F}_{1}\left(z_{1},z_{2},\hat{\theta}_{1}\left(P_{1}\left(z_{1}\right)\tilde{P}_{2}\left(z_{2}\right)\hat{P}_{2}\left(w_{2}\right)\right),w_{2}\right)}{\partial
z_{2}}|_{\mathbf{z},\mathbf{w}=1}&=&\frac{\partial
\hat{F}_{1}}{\partial
w_{1}}\cdot\frac{\partial\hat{\theta}_{1}}{\partial\tilde{P}_{2}}\cdot\frac{\partial\tilde{P}_{2}}{\partial
z_{2}}+\frac{\partial F_{1}}{\partial z_{2}}
\\
\frac{\partial \hat{F}_{1}\left(z_{1},z_{2},\hat{\theta}_{1}\left(P_{1}\left(z_{1}\right)\tilde{P}_{2}\left(z_{2}\right)\hat{P}_{2}\left(w_{2}\right)\right),w_{2}\right)}{\partial w_{1}}|_{\mathbf{z},\mathbf{w}=1}&=&0\\
\frac{\partial \hat{F}_{1}\left(z_{1},z_{2},\hat{\theta}_{1}\left(P_{1}\left(z_{1}\right)\tilde{P}_{2}\left(z_{2}\right)\hat{P}_{2}\left(w_{2}\right)\right),w_{2}\right)}{\partial w_{2}}|_{\mathbf{z},\mathbf{w}=1}&=&\frac{\partial\hat{F}_{1}}{\partial w_{1}}\cdot\frac{\partial\hat{\theta}_{1}}{\partial\hat{P}_{2}}\cdot\frac{\partial\hat{P}_{2}}{\partial w_{2}}+\frac{\partial \hat{F}_{1}}{\partial w_{2}}\\
\end{eqnarray*}



Finalmente para $\zeta_{2}$

\begin{eqnarray*}\hat{F}_{2}\left(z_{1},z_{2},w_{1},\hat{\theta}_{2}\left(P_{1}\left(z_{1}\right)\tilde{P}_{2}\left(z_{2}\right)\hat{P}_{1}\left(w_{1}\right)\right)\right)&=&F_{2}\left(z_{1},z_{2};\zeta_{2}\right)\\
&&\hat{F}_{2}\left(w_{1},\hat{\theta}_{2}\left(P_{1}\left(z_{1}\right)\tilde{P}_{2}\left(z_{2}\right)\hat{P}_{1}\left(w_{1}\right)\right)\right]
\end{eqnarray*}
por tanto:

\begin{eqnarray*}
\frac{\partial
\hat{F}_{2}\left(z_{1},z_{2},w_{1},\hat{\theta}_{2}\left(P_{1}\left(z_{1}\right)\tilde{P}_{2}\left(z_{2}\right)\hat{P}_{1}\left(w_{1}\right)\right)\right)}{\partial
z_{1}}|_{\mathbf{z},\mathbf{w}=1}&=&\frac{\partial\hat{F}_{2}}{\partial
w_{2}}\cdot\frac{\partial\hat{\theta}_{2}}{\partial
P_{1}}\cdot\frac{\partial P_{1}}{\partial z_{1}}+\frac{\partial
F_{2}}{\partial z_{1}}
\\
\frac{\partial \hat{F}_{2}\left(z_{1},z_{2},w_{1},\hat{\theta}_{2}\left(P_{1}\left(z_{1}\right)\tilde{P}_{2}\left(z_{2}\right)\hat{P}_{1}\left(w_{1}\right)\right)\right)}{\partial z_{2}}|_{\mathbf{z},\mathbf{w}=1}&=&\frac{\partial\hat{F}_{2}}{\partial w_{2}}\cdot\frac{\partial\hat{\theta}_{2}}{\partial \tilde{P}_{2}}\cdot\frac{\partial \tilde{P}_{2}}{\partial z_{2}}+\frac{\partial F_{2}}{\partial z_{2}}\\
\frac{\partial \hat{F}_{2}\left(z_{1},z_{2},w_{1},\hat{\theta}_{2}\left(P_{1}\left(z_{1}\right)\tilde{P}_{2}\left(z_{2}\right)\hat{P}_{1}\left(w_{1}\right)\right)\right)}{\partial w_{1}}|_{\mathbf{z},\mathbf{w}=1}&=&\frac{\partial\hat{F}_{2}}{\partial w_{2}}\cdot\frac{\partial\hat{\theta}_{2}}{\partial \hat{P}_{1}}\cdot\frac{\partial \hat{P}_{1}}{\partial w_{1}}+\frac{\partial \hat{F}_{2}}{\partial w_{1}}\\
\frac{\partial \hat{F}_{2}\left(z_{1},z_{2},w_{1},\hat{\theta}_{2}\left(P_{1}\left(z_{1}\right)\tilde{P}_{2}\left(z_{2}\right)\hat{P}_{1}\left(w_{1}\right)\right)\right)}{\partial w_{2}}|_{\mathbf{z},\mathbf{w}=1}&=&0\\
\end{eqnarray*}

%_________________________________________________________________________________________________
\subsection{Ecuaciones Recursivas}
%_________________________________________________________________________________________________

Entonces, de todo lo desarrollado hasta ahora se tienen las siguientes ecuaciones:

\begin{eqnarray*}
\frac{\partial F_{2}\left(\mathbf{z},\mathbf{w}\right)}{\partial z_{1}}|_{\mathbf{z},\mathbf{w}=1}&=&\frac{\partial R_{1}}{\partial z_{1}}+\frac{\partial F_{1}}{\partial z_{1}}=r_{1}\mu_{1}\\
\frac{\partial F_{2}\left(\mathbf{z},\mathbf{w}\right)}{\partial z_{2}}|_{\mathbf{z},\mathbf{w}=1}&=&\frac{\partial R_{1}}{\partial z_{2}}+\frac{\partial F_{1}}{\partial z_{2}}=r_{1}\tilde{\mu}_{2}+f_{1}\left(1\right)\left(\frac{1}{1-\mu_{1}}\right)\tilde{\mu}_{2}+f_{1}\left(2\right)\\
\frac{\partial F_{2}\left(\mathbf{z},\mathbf{w}\right)}{\partial w_{1}}|_{\mathbf{z},\mathbf{w}=1}&=&\frac{\partial R_{1}}{\partial w_{1}}+\frac{\partial F_{1}}{\partial w_{1}}=r_{1}\hat{\mu}_{1}+f_{1}\left(1\right)\left(\frac{1}{1-\mu_{1}}\right)\hat{\mu}_{1}+\hat{F}_{1,1}^{(1)}\left(1\right)\\
\frac{\partial F_{2}\left(\mathbf{z},\mathbf{w}\right)}{\partial
w_{2}}|_{\mathbf{z},\mathbf{w}=1}&=&\frac{\partial R_{1}}{\partial
w_{2}}+\frac{\partial F_{1}}{\partial
w_{2}}=r_{1}\hat{\mu}_{2}+f_{1}\left(1\right)\left(\frac{1}{1-\mu_{1}}\right)\hat{\mu}_{2}+\hat{F}_{2,1}^{(1)}\left(1\right)
\end{eqnarray*}



\begin{eqnarray*}
\frac{\partial F_{1}\left(\mathbf{z},\mathbf{w}\right)}{\partial z_{1}}|_{\mathbf{z},\mathbf{w}=1}&=&\frac{\partial R_{2}}{\partial z_{1}}+\frac{\partial F_{2}}{\partial z_{1}}=r_{2}\mu_{1}+f_{2}\left(2\right)\left(\frac{1}{1-\tilde{\mu}_{2}}\right)\mu_{1}+f_{2}\left(1\right)\\
\frac{\partial F_{1}\left(\mathbf{z},\mathbf{w}\right)}{\partial z_{2}}|_{\mathbf{z},\mathbf{w}=1}&=&\frac{\partial R_{2}}{\partial z_{2}}+\frac{\partial F_{2}}{\partial z_{2}}=r_{2}\tilde{\mu}_{2}\\
\frac{\partial F_{1}\left(\mathbf{z},\mathbf{w}\right)}{\partial w_{1}}|_{\mathbf{z},\mathbf{w}=1}&=&\frac{\partial R_{2}}{\partial w_{1}}+\frac{\partial F_{2}}{\partial w_{1}}=r_{2}\hat{\mu}_{1}+f_{2}\left(2\right)\left(\frac{1}{1-\tilde{\mu}_{2}}\right)\hat{\mu}_{1}+\hat{F}_{2,1}^{(1)}\left(1\right)\\
\frac{\partial F_{1}\left(\mathbf{z},\mathbf{w}\right)}{\partial
w_{2}}|_{\mathbf{z},\mathbf{w}=1}&=&\frac{\partial R_{2}}{\partial
w_{2}}+\frac{\partial F_{2}}{\partial
w_{2}}=r_{2}\hat{\mu}_{2}+f_{2}\left(2\right)\left(\frac{1}{1-\tilde{\mu}_{2}}\right)\hat{\mu}_{2}+\hat{F}_{2,2}^{(1)}\left(1\right)
\end{eqnarray*}




\begin{eqnarray*}
\frac{\partial \hat{F}_{2}\left(\mathbf{z},\mathbf{w}\right)}{\partial z_{1}}|_{\mathbf{z},\mathbf{w}=1}&=&\frac{\partial \hat{R}_{1}}{\partial z_{1}}+\frac{\partial \hat{F}_{1}}{\partial z_{1}}=\hat{r}_{1}\mu_{1}+\hat{f}_{1}\left(1\right)\left(\frac{1}{1-\hat{\mu}_{1}}\right)\mu_{1}+F_{1,1}^{(1)}\left(1\right)\\
\frac{\partial \hat{F}_{2}\left(\mathbf{z},\mathbf{w}\right)}{\partial z_{2}}|_{\mathbf{z},\mathbf{w}=1}&=&\frac{\partial \hat{R}_{1}}{\partial z_{2}}+\frac{\partial \hat{F}_{1}}{\partial z_{2}}=\hat{r}_{1}\mu_{2}+\hat{f}_{1}\left(1\right)\left(\frac{1}{1-\hat{\mu}_{1}}\right)\tilde{\mu}_{2}+F_{2,1}^{(1)}\left(1\right)\\
\frac{\partial \hat{F}_{2}\left(\mathbf{z},\mathbf{w}\right)}{\partial w_{1}}|_{\mathbf{z},\mathbf{w}=1}&=&\frac{\partial \hat{R}_{1}}{\partial w_{1}}+\frac{\partial \hat{F}_{1}}{\partial w_{1}}=\hat{r}_{1}\hat{\mu}_{1}\\
\frac{\partial \hat{F}_{2}\left(\mathbf{z},\mathbf{w}\right)}{\partial w_{2}}|_{\mathbf{z},\mathbf{w}=1}&=&\frac{\partial \hat{R}_{1}}{\partial w_{2}}+\frac{\partial \hat{F}_{1}}{\partial w_{2}}=\hat{r}_{1}\hat{\mu}_{2}+\hat{f}_{1}\left(1\right)\left(\frac{1}{1-\hat{\mu}_{1}}\right)\hat{\mu}_{2}+\hat{f}_{1}\left(2\right)
\end{eqnarray*}



\begin{eqnarray*}
\frac{\partial \hat{F}_{1}\left(\mathbf{z},\mathbf{w}\right)}{\partial z_{1}}|_{\mathbf{z},\mathbf{w}=1}&=&\frac{\partial \hat{R}_{2}}{\partial z_{1}}+\frac{\partial \hat{F}_{2}}{\partial z_{1}}=\hat{r}_{2}\mu_{1}+\hat{f}_{2}\left(1\right)\left(\frac{1}{1-\hat{\mu}_{2}}\right)\mu_{1}+F_{1,2}^{(1)}\left(1\right)\\
\frac{\partial \hat{F}_{1}\left(\mathbf{z},\mathbf{w}\right)}{\partial z_{2}}|_{\mathbf{z},\mathbf{w}=1}&=&\frac{\partial \hat{R}_{2}}{\partial z_{2}}+\frac{\partial \hat{F}_{2}}{\partial z_{2}}=\hat{r}_{2}\tilde{\mu}_{2}+\hat{f}_{2}\left(2\right)\left(\frac{1}{1-\hat{\mu}_{2}}\right)\tilde{\mu}_{2}+F_{2,2}^{(1)}\left(1\right)\\
\frac{\partial \hat{F}_{1}\left(\mathbf{z},\mathbf{w}\right)}{\partial w_{1}}|_{\mathbf{z},\mathbf{w}=1}&=&\frac{\partial \hat{R}_{2}}{\partial w_{1}}+\frac{\partial \hat{F}_{2}}{\partial w_{1}}=\hat{r}_{2}\hat{\mu}_{1}+\hat{f}_{2}\left(2\right)\left(\frac{1}{1-\hat{\mu}_{2}}\right)\hat{\mu}_{1}+\hat{f}_{2}\left(1\right)\\
\frac{\partial
\hat{F}_{1}\left(\mathbf{z},\mathbf{w}\right)}{\partial
w_{2}}|_{\mathbf{z},\mathbf{w}=1}&=&\frac{\partial
\hat{R}_{2}}{\partial w_{2}}+\frac{\partial \hat{F}_{2}}{\partial
w_{2}}=\hat{r}_{2}\hat{\mu}_{2}
\end{eqnarray*}

Es decir, se tienen las siguientes ecuaciones:




\begin{eqnarray*}
f_{2}\left(1\right)&=&r_{1}\mu_{1}\\
f_{1}\left(2\right)&=&r_{2}\tilde{\mu}_{2}\\
f_{2}\left(2\right)&=&r_{1}\tilde{\mu}_{2}+\tilde{\mu}_{2}\left(\frac{f_{1}\left(1\right)}{1-\mu_{1}}\right)+f_{1}\left(2\right)=\left(r_{1}+\frac{f_{1}\left(1\right)}{1-\mu_{1}}\right)\tilde{\mu}_{2}+r_{2}\tilde{\mu}_{2}\\
&=&\left(r_{1}+r_{2}+\frac{f_{1}\left(1\right)}{1-\mu_{1}}\right)\tilde{\mu}_{2}=\left(r+\frac{f_{1}\left(1\right)}{1-\mu_{1}}\right)\tilde{\mu}_{2}\\
f_{2}\left(3\right)&=&r_{1}\hat{\mu}_{1}+\hat{\mu}_{1}\left(\frac{f_{1}\left(1\right)}{1-\mu_{1}}\right)+\hat{F}_{1,1}^{(1)}\left(1\right)=\hat{\mu}_{1}\left(r_{1}+\frac{f_{1}\left(1\right)}{1-\mu_{1}}\right)+\frac{\hat{\mu}_{1}}{\mu_{1}}\\
f_{2}\left(4\right)&=&r_{1}\hat{\mu}_{2}+\hat{\mu}_{2}\left(\frac{f_{1}\left(1\right)}{1-\mu_{1}}\right)+\hat{F}_{2,1}^{(1)}\left(1\right)=\hat{\mu}_{2}\left(r_{1}+\frac{f_{1}\left(1\right)}{1-\mu_{1}}\right)+\frac{\hat{\mu}_{2}}{\mu_{1}}\\
\end{eqnarray*}


\begin{eqnarray*}
f_{1}\left(1\right)&=&r_{2}\mu_{1}+\mu_{1}\left(\frac{f_{2}\left(2\right)}{1-\tilde{\mu}_{2}}\right)+r_{1}\mu_{1}=\mu_{1}\left(r_{1}+r_{2}+\frac{f_{2}\left(2\right)}{1-\tilde{\mu}_{2}}\right)\\
&=&\mu_{1}\left(r+\frac{f_{2}\left(2\right)}{1-\tilde{\mu}_{2}}\right)\\
f_{1}\left(3\right)&=&r_{2}\hat{\mu}_{1}+\hat{\mu}_{1}\left(\frac{f_{2}\left(2\right)}{1-\tilde{\mu}_{2}}\right)+\hat{F}^{(1)}_{1,2}\left(1\right)=\hat{\mu}_{1}\left(r_{2}+\frac{f_{2}\left(2\right)}{1-\tilde{\mu}_{2}}\right)+\frac{\hat{\mu}_{1}}{\mu_{2}}\\
f_{1}\left(4\right)&=&r_{2}\hat{\mu}_{2}+\hat{\mu}_{2}\left(\frac{f_{2}\left(2\right)}{1-\tilde{\mu}_{2}}\right)+\hat{F}_{2,2}^{(1)}\left(1\right)=\hat{\mu}_{2}\left(r_{2}+\frac{f_{2}\left(2\right)}{1-\tilde{\mu}_{2}}\right)+\frac{\hat{\mu}_{2}}{\mu_{2}}\\
\hat{f}_{1}\left(4\right)&=&\hat{r}_{2}\hat{\mu}_{2}\\
\hat{f}_{2}\left(3\right)&=&\hat{r}_{1}\hat{\mu}_{1}\\
\hat{f}_{1}\left(1\right)&=&\hat{r}_{2}\mu_{1}+\mu_{1}\left(\frac{\hat{f}_{2}\left(4\right)}{1-\hat{\mu}_{2}}\right)+F_{1,2}^{(1)}\left(1\right)=\left(\hat{r}_{2}+\frac{\hat{f}_{2}\left(4\right)}{1-\hat{\mu}_{2}}\right)\mu_{1}+\frac{\mu_{1}}{\hat{\mu}_{2}}
\end{eqnarray*}

\begin{eqnarray*}
\hat{f}_{1}\left(2\right)&=&\hat{r}_{2}\tilde{\mu}_{2}+\tilde{\mu}_{2}\left(\frac{\hat{f}_{2}\left(4\right)}{1-\hat{\mu}_{2}}\right)+F_{2,2}^{(1)}\left(1\right)=
\left(\hat{r}_{2}+\frac{\hat{f}_{2}\left(4\right)}{1-\hat{\mu}_{2}}\right)\tilde{\mu}_{2}+\frac{\mu_{2}}{\hat{\mu}_{2}}\\
\hat{f}_{1}\left(3\right)&=&\hat{r}_{2}\hat{\mu}_{1}+\hat{\mu}_{1}\left(\frac{\hat{f}_{2}\left(4\right)}{1-\hat{\mu}_{2}}\right)+\hat{f}_{2}\left(3\right)=\left(\hat{r}_{2}+\frac{\hat{f}_{2}\left(4\right)}{1-\hat{\mu}_{2}}\right)\hat{\mu}_{1}+\hat{r}_{1}\hat{\mu}_{1}\\
&=&\left(\hat{r}_{1}+\hat{r}_{2}+\frac{\hat{f}_{2}\left(4\right)}{1-\hat{\mu}_{2}}\right)\hat{\mu}_{1}=\left(\hat{r}+\frac{\hat{f}_{2}\left(4\right)}{1-\hat{\mu}_{2}}\right)\hat{\mu}_{1}\\
\hat{f}_{2}\left(1\right)&=&\hat{r}_{1}\mu_{1}+\mu_{1}\left(\frac{\hat{f}_{1}\left(3\right)}{1-\hat{\mu}_{1}}\right)+F_{1,1}^{(1)}\left(1\right)=\left(\hat{r}_{1}+\frac{\hat{f}_{1}\left(3\right)}{1-\hat{\mu}_{1}}\right)\mu_{1}+\frac{\mu_{1}}{\hat{\mu}_{1}}\\
\hat{f}_{2}\left(2\right)&=&\hat{r}_{1}\tilde{\mu}_{2}+\tilde{\mu}_{2}\left(\frac{\hat{f}_{1}\left(3\right)}{1-\hat{\mu}_{1}}\right)+F_{2,1}^{(1)}\left(1\right)=\left(\hat{r}_{1}+\frac{\hat{f}_{1}\left(3\right)}{1-\hat{\mu}_{1}}\right)\tilde{\mu}_{2}+\frac{\mu_{2}}{\hat{\mu}_{1}}\\
\hat{f}_{2}\left(4\right)&=&\hat{r}_{1}\hat{\mu}_{2}+\hat{\mu}_{2}\left(\frac{\hat{f}_{1}\left(3\right)}{1-\hat{\mu}_{1}}\right)+\hat{f}_{1}\left(4\right)=\hat{r}_{1}\hat{\mu}_{2}+\hat{r}_{2}\hat{\mu}_{2}+\hat{\mu}_{2}\left(\frac{\hat{f}_{1}\left(3\right)}{1-\hat{\mu}_{1}}\right)\\
&=&\left(\hat{r}+\frac{\hat{f}_{1}\left(3\right)}{1-\hat{\mu}_{1}}\right)\hat{\mu}_{2}
\end{eqnarray*}


%_______________________________________________________________________________________________
\subsection{Soluci\'on del Sistema de Ecuaciones Lineales}
%_________________________________________________________________________________________________

A saber, se puede demostrar que la soluci\'on del sistema de
ecuaciones est\'a dado por las siguientes expresiones, donde

\begin{eqnarray*}
\mu=\mu_{1}+\tilde{\mu}_{2}\textrm{ , }\hat{\mu}=\hat{\mu}_{1}+\hat{\mu}_{2}\textrm{ , }
r=r_{1}+r_{2}\textrm{ y }\hat{r}=\hat{r}_{1}+\hat{r}_{2}
\end{eqnarray*}
entonces

\begin{eqnarray*}
f_{1}\left(1\right)&=&r\frac{\mu_{1}\left(1-\mu_{1}\right)}{1-\mu}\\
f_{2}\left(2\right)&=&r\frac{\tilde{\mu}_{2}\left(1-\tilde{\mu}_{2}\right)}{1-\mu}
\end{eqnarray*}

\begin{eqnarray*}
f_{1}\left(3\right)&=&\hat{\mu}_{1}\left(\frac{r_{2}\mu_{2}+1}{\mu_{2}}+r\frac{\tilde{\mu}_{2}}{1-\mu}\right)\\
f_{1}\left(4\right)&=&\hat{\mu}_{2}\left(\frac{r_{2}\mu_{2}+1}{\mu_{2}}+r\frac{\tilde{\mu}_{2}}{1-\mu}\right)\\
\end{eqnarray*}



\begin{eqnarray*}
f_{2}\left(3\right)&=&\hat{\mu}_{1}\left(\frac{r_{1}\mu_{1}+1}{\mu_{1}}+r\frac{\mu_{1}}{1-\mu}\right)\\
f_{2}\left(4\right)&=&\hat{\mu}_{2}\left(\frac{r_{1}\mu_{1}+1}{\mu_{1}}+r\frac{\mu_{1}}{1-\mu}\right)\\
\end{eqnarray*}
\begin{eqnarray*}
\hat{f}_{2}\left(4\right)&=&\hat{r}\frac{\hat{\mu}_{2}\left(1-\hat{\mu}_{2}\right)}{1-\hat{\mu}}\\
\hat{f}_{1}\left(3\right)&=&\hat{r}\frac{\hat{\mu}_{1}\left(1-\hat{\mu}_{1}\right)}{1-\hat{\mu}}
\end{eqnarray*}

\begin{eqnarray*}
\hat{f}_{1}\left(1\right)&=&\mu_{1}\left(\frac{\hat{r}_{2}\hat{\mu}_{2}+1}{\hat{\mu}_{2}}+\hat{r}\frac{\hat{\mu}_{2}}{1-\hat{\mu}}\right)\\
\hat{f}_{1}\left(2\right)&=&\tilde{\mu}_{2}\left(\hat{r}_{2}+\hat{r}\frac{\hat{\mu}_{2}}{1-\hat{\mu}}\right)+\frac{\mu_{2}}{\hat{\mu}_{2}}\\\\
\hat{f}_{2}\left(1\right)&=&\mu_{1}\left(\frac{\hat{r}_{1}\hat{\mu}_{1}+1}{\hat{\mu}_{1}}+\hat{r}\frac{\hat{\mu}_{1}}{1-\hat{\mu}}\right)\\
\hat{f}_{2}\left(2\right)&=&\tilde{\mu}_{2}\left(\hat{r}_{1}+\hat{r}\frac{\hat{\mu}_{1}}{1-\hat{\mu}}\right)+\frac{\hat{\mu_{2}}}{\hat{\mu}_{1}}\\
\end{eqnarray*}

A saber

\begin{eqnarray*}
f_{1}\left(3\right)&=&\hat{\mu}_{1}\left(r_{2}+\frac{f_{2}\left(2\right)}{1-\tilde{\mu}_{2}}\right)+\frac{\hat{\mu}_{1}}{\mu_{2}}=\hat{\mu}_{1}\left(r_{2}+\frac{r\frac{\tilde{\mu}_{2}\left(1-\tilde{\mu}_{2}\right)}{1-\mu}}{1-\tilde{\mu}_{2}}\right)+\frac{\hat{\mu}_{1}}{\mu_{2}}\\
&=&\hat{\mu}_{1}\left(r_{2}+\frac{r\tilde{\mu}_{2}}{1-\mu}\right)+\frac{\hat{\mu}_{1}}{\mu_{2}}=
\hat{\mu}_{1}\left(r_{2}+\frac{r\tilde{\mu}_{2}}{1-\mu}+\frac{1}{\mu_{2}}\right)\\
&=&\hat{\mu}_{1}\left(\frac{r_{2}\mu_{2}+1}{\mu_{2}}+\frac{r\tilde{\mu}_{2}}{1-\mu}\right)
\end{eqnarray*}

\begin{eqnarray*}
f_{1}\left(4\right)&=&\hat{\mu}_{2}\left(r_{2}+\frac{f_{2}\left(2\right)}{1-\tilde{\mu}_{2}}\right)+\frac{\hat{\mu}_{2}}{\mu_{2}}=\hat{\mu}_{2}\left(r_{2}+\frac{r\frac{\tilde{\mu}_{2}\left(1-\tilde{\mu}_{2}\right)}{1-\mu}}{1-\tilde{\mu}_{2}}\right)+\frac{\hat{\mu}_{2}}{\mu_{2}}\\
&=&\hat{\mu}_{2}\left(r_{2}+\frac{r\tilde{\mu}_{2}}{1-\mu}\right)+\frac{\hat{\mu}_{1}}{\mu_{2}}=
\hat{\mu}_{2}\left(r_{2}+\frac{r\tilde{\mu}_{2}}{1-\mu}+\frac{1}{\mu_{2}}\right)\\
&=&\hat{\mu}_{2}\left(\frac{r_{2}\mu_{2}+1}{\mu_{2}}+\frac{r\tilde{\mu}_{2}}{1-\mu}\right)
\end{eqnarray*}

\begin{eqnarray*}
f_{2}\left(3\right)&=&\hat{\mu}_{1}\left(r_{1}+\frac{f_{1}\left(1\right)}{1-\mu_{1}}\right)+\frac{\hat{\mu}_{1}}{\mu_{1}}=\hat{\mu}_{1}\left(r_{1}+\frac{r\frac{\mu_{1}\left(1-\mu_{1}\right)}{1-\mu}}{1-\mu_{1}}\right)+\frac{\hat{\mu}_{1}}{\mu_{1}}\\
&=&\hat{\mu}_{1}\left(r_{1}+\frac{r\mu_{1}}{1-\mu}\right)+\frac{\hat{\mu}_{1}}{\mu_{1}}=
\hat{\mu}_{1}\left(r_{1}+\frac{r\mu_{1}}{1-\mu}+\frac{1}{\mu_{1}}\right)\\
&=&\hat{\mu}_{1}\left(\frac{r_{1}\mu_{1}+1}{\mu_{1}}+\frac{r\mu_{1}}{1-\mu}\right)
\end{eqnarray*}

\begin{eqnarray*}
f_{2}\left(4\right)&=&\hat{\mu}_{2}\left(r_{1}+\frac{f_{1}\left(1\right)}{1-\mu_{1}}\right)+\frac{\hat{\mu}_{2}}{\mu_{1}}=\hat{\mu}_{2}\left(r_{1}+\frac{r\frac{\mu_{1}\left(1-\mu_{1}\right)}{1-\mu}}{1-\mu_{1}}\right)+\frac{\hat{\mu}_{1}}{\mu_{1}}\\
&=&\hat{\mu}_{2}\left(r_{1}+\frac{r\mu_{1}}{1-\mu}\right)+\frac{\hat{\mu}_{1}}{\mu_{1}}=
\hat{\mu}_{2}\left(r_{1}+\frac{r\mu_{1}}{1-\mu}+\frac{1}{\mu_{1}}\right)\\
&=&\hat{\mu}_{2}\left(\frac{r_{1}\mu_{1}+1}{\mu_{1}}+\frac{r\mu_{1}}{1-\mu}\right)\end{eqnarray*}

A saber

\begin{eqnarray*}
\hat{f}_{1}\left(1\right)&=&\mu_{1}\left(\hat{r}_{2}+\frac{\hat{f}_{2}\left(4\right)}{1-\tilde{\mu}_{2}}\right)+\frac{\mu_{1}}{\hat{\mu}_{2}}=\mu_{1}\left(\hat{r}_{2}+\frac{\hat{r}\frac{\hat{\mu}_{2}\left(1-\hat{\mu}_{2}\right)}{1-\hat{\mu}}}{1-\hat{\mu}_{2}}\right)+\frac{\mu_{1}}{\hat{\mu}_{2}}\\
&=&\mu_{1}\left(\hat{r}_{2}+\frac{\hat{r}\hat{\mu}_{2}}{1-\hat{\mu}}\right)+\frac{\mu_{1}}{\mu_{2}}
=\mu_{1}\left(\hat{r}_{2}+\frac{\hat{r}\mu_{2}}{1-\hat{\mu}}+\frac{1}{\hat{\mu}_{2}}\right)\\
&=&\mu_{1}\left(\frac{\hat{r}_{2}\hat{\mu}_{2}+1}{\hat{\mu}_{2}}+\frac{\hat{r}\hat{\mu}_{2}}{1-\hat{\mu}}\right)
\end{eqnarray*}

\begin{eqnarray*}
\hat{f}_{1}\left(2\right)&=&\tilde{\mu}_{2}\left(\hat{r}_{2}+\frac{\hat{f}_{2}\left(4\right)}{1-\tilde{\mu}_{2}}\right)+\frac{\mu_{2}}{\hat{\mu}_{2}}=\tilde{\mu}_{2}\left(\hat{r}_{2}+\frac{\hat{r}\frac{\hat{\mu}_{2}\left(1-\hat{\mu}_{2}\right)}{1-\hat{\mu}}}{1-\hat{\mu}_{2}}\right)+\frac{\mu_{2}}{\hat{\mu}_{2}}\\
&=&\tilde{\mu}_{2}\left(\hat{r}_{2}+\frac{\hat{r}\hat{\mu}_{2}}{1-\hat{\mu}}\right)+\frac{\mu_{2}}{\hat{\mu}_{2}}
\end{eqnarray*}

\begin{eqnarray*}
\hat{f}_{2}\left(1\right)&=&\mu_{1}\left(\hat{r}_{1}+\frac{\hat{f}_{1}\left(3\right)}{1-\hat{\mu}_{1}}\right)+\frac{\mu_{1}}{\hat{\mu}_{1}}=\mu_{1}\left(\hat{r}_{1}+\frac{\hat{r}\frac{\hat{\mu}_{1}\left(1-\hat{\mu}_{1}\right)}{1-\hat{\mu}}}{1-\hat{\mu}_{1}}\right)+\frac{\mu_{1}}{\hat{\mu}_{1}}\\
&=&\mu_{1}\left(\hat{r}_{1}+\frac{\hat{r}\hat{\mu}_{1}}{1-\hat{\mu}}\right)+\frac{\mu_{1}}{\hat{\mu}_{1}}
=\mu_{1}\left(\hat{r}_{1}+\frac{\hat{r}\hat{\mu}_{1}}{1-\hat{\mu}}+\frac{1}{\hat{\mu}_{1}}\right)\\
&=&\mu_{1}\left(\frac{\hat{r}_{1}\hat{\mu}_{1}+1}{\hat{\mu}_{1}}+\frac{\hat{r}\hat{\mu}_{1}}{1-\hat{\mu}}\right)
\end{eqnarray*}

\begin{eqnarray*}
\hat{f}_{2}\left(2\right)&=&\tilde{\mu}_{2}\left(\hat{r}_{1}+\frac{\hat{f}_{1}\left(3\right)}{1-\tilde{\mu}_{1}}\right)+\frac{\mu_{2}}{\hat{\mu}_{1}}=\tilde{\mu}_{2}\left(\hat{r}_{1}+\frac{\hat{r}\frac{\hat{\mu}_{1}
\left(1-\hat{\mu}_{1}\right)}{1-\hat{\mu}}}{1-\hat{\mu}_{1}}\right)+\frac{\mu_{2}}{\hat{\mu}_{1}}\\
&=&\tilde{\mu}_{2}\left(\hat{r}_{1}+\frac{\hat{r}\hat{\mu}_{1}}{1-\hat{\mu}}\right)+\frac{\mu_{2}}{\hat{\mu}_{1}}
\end{eqnarray*}
%___________________________________________________________________________________________
%
\section{Segundos Momentos}
%___________________________________________________________________________________________
%
%___________________________________________________________________________________________
%
%\subsection{Derivadas de Segundo Orden: Tiempos de Traslado del Servidor}
%___________________________________________________________________________________________



Para poder determinar los segundos momentos para los tiempos de traslado del servidor es necesario enunciar y demostrar la siguiente proposici\'on:

\begin{Prop}\label{Prop.Segundas.Derivadas}
Sea $f\left(g\left(x\right)h\left(y\right)\right)$ funci\'on continua tal que tiene derivadas parciales mixtas de segundo orden, entonces se tiene lo siguiente:

\begin{eqnarray*}
\frac{\partial}{\partial x}f\left(g\left(x\right)h\left(y\right)\right)=\frac{\partial f\left(g\left(x\right)h\left(y\right)\right)}{\partial x}\cdot \frac{\partial g\left(x\right)}{\partial x}\cdot h\left(y\right)
\end{eqnarray*}

por tanto

\begin{eqnarray}
\frac{\partial}{\partial x}\frac{\partial}{\partial x}f\left(g\left(x\right)h\left(y\right)\right)
&=&\frac{\partial^{2}}{\partial x}f\left(g\left(x\right)h\left(y\right)\right)\cdot \left(\frac{\partial g\left(x\right)}{\partial x}\right)^{2}\cdot h^{2}\left(y\right)+\frac{\partial}{\partial x}f\left(g\left(x\right)h\left(y\right)\right)\cdot \frac{\partial g^{2}\left(x\right)}{\partial x^{2}}\cdot h\left(y\right).
\end{eqnarray}

y

\begin{eqnarray*}
\frac{\partial}{\partial y}\frac{\partial}{\partial x}f\left(g\left(x\right)h\left(y\right)\right)&=&\frac{\partial g\left(x\right)}{\partial x}\cdot \frac{\partial h\left(y\right)}{\partial y}\left\{\frac{\partial^{2}}{\partial y\partial x}f\left(g\left(x\right)h\left(y\right)\right)\cdot g\left(x\right)\cdot h\left(y\right)+\frac{\partial}{\partial x}f\left(g\left(x\right)h\left(y\right)\right)\right\}
\end{eqnarray*}
\end{Prop}
\begin{proof}
\footnotesize{
\begin{eqnarray*}
\frac{\partial}{\partial x}\frac{\partial}{\partial x}f\left(g\left(x\right)h\left(y\right)\right)&=&\frac{\partial}{\partial x}\left\{\frac{\partial f\left(g\left(x\right)h\left(y\right)\right)}{\partial x}\cdot \frac{\partial g\left(x\right)}{\partial x}\cdot h\left(y\right)\right\}\\
&=&\frac{\partial}{\partial x}\left\{\frac{\partial}{\partial x}f\left(g\left(x\right)h\left(y\right)\right)\right\}\cdot \frac{\partial g\left(x\right)}{\partial x}\cdot h\left(y\right)+\frac{\partial}{\partial x}f\left(g\left(x\right)h\left(y\right)\right)\cdot \frac{\partial g^{2}\left(x\right)}{\partial x^{2}}\cdot h\left(y\right)\\
&=&\frac{\partial^{2}}{\partial x}f\left(g\left(x\right)h\left(y\right)\right)\cdot \frac{\partial g\left(x\right)}{\partial x}\cdot h\left(y\right)\cdot \frac{\partial g\left(x\right)}{\partial x}\cdot h\left(y\right)+\frac{\partial}{\partial x}f\left(g\left(x\right)h\left(y\right)\right)\cdot \frac{\partial g^{2}\left(x\right)}{\partial x^{2}}\cdot h\left(y\right)\\
&=&\frac{\partial^{2}}{\partial x}f\left(g\left(x\right)h\left(y\right)\right)\cdot \left(\frac{\partial g\left(x\right)}{\partial x}\right)^{2}\cdot h^{2}\left(y\right)+\frac{\partial}{\partial x}f\left(g\left(x\right)h\left(y\right)\right)\cdot \frac{\partial g^{2}\left(x\right)}{\partial x^{2}}\cdot h\left(y\right).
\end{eqnarray*}}


Por otra parte:
\footnotesize{
\begin{eqnarray*}
\frac{\partial}{\partial y}\frac{\partial}{\partial x}f\left(g\left(x\right)h\left(y\right)\right)&=&\frac{\partial}{\partial y}\left\{\frac{\partial f\left(g\left(x\right)h\left(y\right)\right)}{\partial x}\cdot \frac{\partial g\left(x\right)}{\partial x}\cdot h\left(y\right)\right\}\\
&=&\frac{\partial}{\partial y}\left\{\frac{\partial}{\partial x}f\left(g\left(x\right)h\left(y\right)\right)\right\}\cdot \frac{\partial g\left(x\right)}{\partial x}\cdot h\left(y\right)+\frac{\partial}{\partial x}f\left(g\left(x\right)h\left(y\right)\right)\cdot \frac{\partial g\left(x\right)}{\partial x}\cdot \frac{\partial h\left(y\right)}{y}\\
&=&\frac{\partial^{2}}{\partial y\partial x}f\left(g\left(x\right)h\left(y\right)\right)\cdot \frac{\partial h\left(y\right)}{\partial y}\cdot g\left(x\right)\cdot \frac{\partial g\left(x\right)}{\partial x}\cdot h\left(y\right)+\frac{\partial}{\partial x}f\left(g\left(x\right)h\left(y\right)\right)\cdot \frac{\partial g\left(x\right)}{\partial x}\cdot \frac{\partial h\left(y\right)}{\partial y}\\
&=&\frac{\partial g\left(x\right)}{\partial x}\cdot \frac{\partial h\left(y\right)}{\partial y}\left\{\frac{\partial^{2}}{\partial y\partial x}f\left(g\left(x\right)h\left(y\right)\right)\cdot g\left(x\right)\cdot h\left(y\right)+\frac{\partial}{\partial x}f\left(g\left(x\right)h\left(y\right)\right)\right\}
\end{eqnarray*}}
\end{proof}

Para la siguiente proposici\'on es necesario utilizar  el resultado (\ref{Prop.Segundas.Derivadas})

\begin{Prop}
Sea $R_{i}$ la Funci\'on Generadora de Probabilidades para el n\'umero de arribos a cada una de las colas de la Red de Sistemas de Visitas C\'iclicas definidas como en (\ref{Ec.R1}). Entonces las derivadas parciales est\'an dadas por las siguientes expresiones:


\begin{eqnarray*}
\frac{\partial^{2} R_{i}\left(P_{1}\left(z_{1}\right)\tilde{P}_{2}\left(z_{2}\right)\hat{P}_{1}\left(w_{1}\right)\hat{P}_{2}\left(w_{2}\right)\right)}{\partial z_{i}^{2}}&=&\left(\frac{\partial P_{i}\left(z_{i}\right)}{\partial z_{i}}\right)^{2}\cdot\frac{\partial^{2} R_{i}\left(P_{1}\left(z_{1}\right)\tilde{P}_{2}\left(z_{2}\right)\hat{P}_{1}\left(w_{1}\right)\hat{P}_{2}\left(w_{2}\right)\right)}{\partial^{2} z_{i}}\\
&+&\left(\frac{\partial P_{i}\left(z_{i}\right)}{\partial z_{i}}\right)^{2}\cdot
\frac{\partial R_{i}\left(P_{1}\left(z_{1}\right)\tilde{P}_{2}\left(z_{2}\right)\hat{P}_{1}\left(w_{1}\right)\hat{P}_{2}\left(w_{2}\right)\right)}{\partial z_{i}}
\end{eqnarray*}



y adem\'as


\begin{eqnarray*}
\frac{\partial^{2} R_{i}\left(P_{1}\left(z_{1}\right)\tilde{P}_{2}\left(z_{2}\right)\hat{P}_{1}\left(w_{1}\right)\hat{P}_{2}\left(w_{2}\right)\right)}{\partial z_{2}\partial z_{1}}&=&\frac{\partial \tilde{P}_{2}\left(z_{2}\right)}{\partial z_{2}}\cdot\frac{\partial P_{1}\left(z_{1}\right)}{\partial z_{1}}\cdot\frac{\partial^{2} R_{i}\left(P_{1}\left(z_{1}\right)\tilde{P}_{2}\left(z_{2}\right)\hat{P}_{1}\left(w_{1}\right)\hat{P}_{2}\left(w_{2}\right)\right)}{\partial z_{2}\partial z_{1}}\\
&+&\frac{\partial \tilde{P}_{2}\left(z_{2}\right)}{\partial z_{2}}\cdot\frac{\partial P_{1}\left(z_{1}\right)}{\partial z_{1}}\cdot\frac{\partial R_{i}\left(P_{1}\left(z_{1}\right)\tilde{P}_{2}\left(z_{2}\right)\hat{P}_{1}\left(w_{1}\right)\hat{P}_{2}\left(w_{2}\right)\right)}{\partial z_{1}},
\end{eqnarray*}



\begin{eqnarray*}
\frac{\partial^{2} R_{i}\left(P_{1}\left(z_{1}\right)\tilde{P}_{2}\left(z_{2}\right)\hat{P}_{1}\left(w_{1}\right)\hat{P}_{2}\left(w_{2}\right)\right)}{\partial w_{i}\partial z_{1}}&=&\frac{\partial \hat{P}_{i}\left(w_{i}\right)}{\partial z_{2}}\cdot\frac{\partial P_{1}\left(z_{1}\right)}{\partial z_{1}}\cdot\frac{\partial^{2} R_{i}\left(P_{1}\left(z_{1}\right)\tilde{P}_{2}\left(z_{2}\right)\hat{P}_{1}\left(w_{1}\right)\hat{P}_{2}\left(w_{2}\right)\right)}{\partial w_{i}\partial z_{1}}\\
&+&\frac{\partial \hat{P}_{i}\left(w_{i}\right)}{\partial z_{2}}\cdot\frac{\partial P_{1}\left(z_{1}\right)}{\partial z_{1}}\cdot\frac{\partial R_{i}\left(P_{1}\left(z_{1}\right)\tilde{P}_{2}\left(z_{2}\right)\hat{P}_{1}\left(w_{1}\right)\hat{P}_{2}\left(w_{2}\right)\right)}{\partial z_{1}},
\end{eqnarray*}
finalmente

\begin{eqnarray*}
\frac{\partial^{2} R_{i}\left(P_{1}\left(z_{1}\right)\tilde{P}_{2}\left(z_{2}\right)\hat{P}_{1}\left(w_{1}\right)\hat{P}_{2}\left(w_{2}\right)\right)}{\partial w_{i}\partial z_{2}}&=&\frac{\partial \hat{P}_{i}\left(w_{i}\right)}{\partial w_{i}}\cdot\frac{\partial \tilde{P}_{2}\left(z_{2}\right)}{\partial z_{2}}\cdot\frac{\partial^{2} R_{i}\left(P_{1}\left(z_{1}\right)\tilde{P}_{2}\left(z_{2}\right)\hat{P}_{1}\left(w_{1}\right)\hat{P}_{2}\left(w_{2}\right)\right)}{\partial w_{i}\partial z_{2}}\\
&+&\frac{\partial \hat{P}_{i}\left(w_{i}\right)}{\partial w_{i}}\cdot\frac{\partial \tilde{P}_{2}\left(z_{2}\right)}{\partial z_{1}}\cdot\frac{\partial R_{i}\left(P_{1}\left(z_{1}\right)\tilde{P}_{2}\left(z_{2}\right)\hat{P}_{1}\left(w_{1}\right)\hat{P}_{2}\left(w_{2}\right)\right)}{\partial z_{2}},
\end{eqnarray*}

para $i=1,2$.
\end{Prop}

%___________________________________________________________________________________________
%
\subsection{Sistema de Ecuaciones Lineales para los Segundos Momentos}
%___________________________________________________________________________________________

En el ap\'endice A se demuestra que las ecuaciones para las ecuaciones parciales mixtas est\'an dadas por:


\begin{enumerate}
%___________________________________________________________________________________________
%\subsubsection{Mixtas para $z_{1}$:}
%___________________________________________________________________________________________
%1
\item \begin{eqnarray*}
f_{1}\left(1,1\right)&=&r_{2}P_{1}^{(2)}\left(1\right)+\mu_{1}^{2}R_{2}^{(2)}\left(1\right)+2\mu_{1}r_{2}\left(\frac{\mu_{1}}{1-\tilde{\mu}_{2}}f_{2}\left(2\right)+f_{2}\left(1\right)\right)+\frac{1}{1-\tilde{\mu}_{2}}P_{1}^{(2)}f_{2}\left(2\right)\\
&+&\mu_{1}^{2}\tilde{\theta}_{2}^{(2)}\left(1\right)f_{2}\left(2\right)+\frac{\mu_{1}}{1-\tilde{\mu}_{2}}f_{2}(1,2)+\frac{\mu_{1}}{1-\tilde{\mu}_{2}}\left(\frac{\mu_{1}}{1-\tilde{\mu}_{2}}f_{2}(2,2)+f_{2}(1,2)\right)+f_{2}(1,1).
\end{eqnarray*}

%2

\item \begin{eqnarray*}
f_{1}\left(2,1\right)&=&\mu_{1}r_{2}\tilde{\mu}_{2}+\mu_{1}\tilde{\mu}_{2}R_{2}^{(2)}\left(1\right)+r_{2}\tilde{\mu}_{2}\left(\frac{\mu_{1}}{1-\tilde{\mu}_{2}}f_{2}(2)+f_{2}(1)\right).
\end{eqnarray*}

%3

\item \begin{eqnarray*}
f_{1}\left(3,1\right)&=&\mu_{1}\hat{\mu}_{1}r_{2}+\mu_{1}\hat{\mu}_{1}R_{2}^{(2)}\left(1\right)+r_{2}\frac{\mu_{1}}{1-\tilde{\mu}_{2}}f_{2}(2)+r_{2}\hat{\mu}_{1}\left(\frac{\mu_{1}}{1-\tilde{\mu}_{2}}f_{2}(2)+f_{2}(1)\right)+\mu_{1}r_{2}\hat{F}_{2,1}^{(1)}(1)\\
&+&\left(\frac{\mu_{1}}{1-\tilde{\mu}_{2}}f_{2}(2)+f_{2}(1)\right)\hat{F}_{2,1}^{(1)}(1)+\frac{\mu_{1}\hat{\mu}_{1}}{1-\tilde{\mu}_{2}}f_{2}(2)+\mu_{1}\hat{\mu}_{1}\tilde{\theta}_{2}^{(2)}\left(1\right)f_{2}(2)\\
&+&\mu_{1}\hat{\mu}_{1}\left(\frac{1}{1-\tilde{\mu}_{2}}\right)^{2}f_{2}(2,2)+\frac{\hat{\mu}_{1}}{1-\tilde{\mu}_{2}}f_{2}(1,2).
\end{eqnarray*}

%4

\item \begin{eqnarray*}
f_{1}\left(4,1\right)&=&\mu_{1}\hat{\mu}_{2}r_{2}+\mu_{1}\hat{\mu}_{2}R_{2}^{(2)}\left(1\right)+r_{2}\frac{\mu_{1}\hat{\mu}_{2}}{1-\tilde{\mu}_{2}}f_{2}(2)+\mu_{1}r_{2}\hat{F}_{2,2}^{(1)}(1)+r_{2}\hat{\mu}_{2}\left(\frac{\mu_{1}}{1-\tilde{\mu}_{2}}f_{2}(2)+f_{2}(1)\right)\\
&+&\hat{F}_{2,1}^{(1)}(1)\left(\frac{\mu_{1}}{1-\tilde{\mu}_{2}}f_{2}(2)+f_{2}(1)\right)+\frac{\mu_{1}\hat{\mu}_{2}}{1-\tilde{\mu}_{2}}f_{2}(2)
+\mu_{1}\hat{\mu}_{2}\tilde{\theta}_{2}^{(2)}\left(1\right)f_{2}(2)\\
&+&\mu_{1}\hat{\mu}_{2}\left(\frac{1}{1-\tilde{\mu}_{2}}\right)^{2}f_{2}(2,2)+\frac{\hat{\mu}_{2}}{1-\tilde{\mu}_{2}}f_{2}^{(1,2)}.
\end{eqnarray*}
%___________________________________________________________________________________________
%\subsubsection{Mixtas para $z_{2}$:}
%___________________________________________________________________________________________
%5
\item \begin{eqnarray*}
f_{1}\left(1,2\right)&=&\mu_{1}\tilde{\mu}_{2}r_{2}+\mu_{1}\tilde{\mu}_{2}R_{2}^{(2)}\left(1\right)+r_{2}\tilde{\mu}_{2}\left(\frac{\mu_{1}}{1-\tilde{\mu}_{2}}f_{2}(2)+f_{2}(1)\right).
\end{eqnarray*}

%6

\item \begin{eqnarray*}
f_{1}\left(2,2\right)&=&\tilde{\mu}_{2}^{2}R_{2}^{(2)}(1)+r_{2}\tilde{P}_{2}^{(2)}\left(1\right).
\end{eqnarray*}

%7
\item \begin{eqnarray*}
f_{1}\left(3,2\right)&=&\hat{\mu}_{1}\tilde{\mu}_{2}r_{2}+\hat{\mu}_{1}\tilde{\mu}_{2}R_{2}^{(2)}(1)+
r_{2}\frac{\hat{\mu}_{1}\tilde{\mu}_{2}}{1-\tilde{\mu}_{2}}f_{2}(2)+r_{2}\tilde{\mu}_{2}\hat{F}_{2,2}^{(1)}(1).
\end{eqnarray*}
%8
\item \begin{eqnarray*} f_{1}\left(4,2\right)&=&\hat{\mu}_{2}\tilde{\mu}_{2}r_{2}+\hat{\mu}_{2}\tilde{\mu}_{2}R_{2}^{(2)}(1)+
r_{2}\frac{\hat{\mu}_{2}\tilde{\mu}_{2}}{1-\tilde{\mu}_{2}}f_{2}(2)+r_{2}\tilde{\mu}_{2}\hat{F}_{2,2}^{(1)}(1).
\end{eqnarray*}
%___________________________________________________________________________________________
%\subsubsection{Mixtas para $w_{1}$:}
%___________________________________________________________________________________________

%9
\item \begin{eqnarray*} f_{1}\left(1,3\right)&=&\mu_{1}\hat{\mu}_{1}r_{2}+\mu_{1}\hat{\mu}_{1}R_{2}^{(2)}\left(1\right)+\frac{\mu_{1}\hat{\mu}_{1}}{1-\tilde{\mu}_{2}}f_{2}(2)+r_{2}\frac{\mu_{1}\hat{\mu}_{1}}{1-\tilde{\mu}_{2}}f_{2}(2)+\mu_{1}\hat{\mu}_{1}\tilde{\theta}_{2}^{(2)}\left(1\right)f_{2}(2)\\
&+&r_{2}\hat{\mu}_{1}\left(\frac{\mu_{1}}{1-\tilde{\mu}_{2}}f_{2}(2)+f_{2}\left(1\right)\right)+r_{2}\mu_{1}\hat{F}_{2,1}^{(1)}(1)+\left(\frac{\mu_{1}}{1-\tilde{\mu}_{2}}f_{2}\left(1\right)+f_{2}\left(1\right)\right)\hat{F}_{2,1}^{(1)}(1)\\
&+&\frac{\hat{\mu}_{1}}{1-\tilde{\mu}_{2}}\left(\frac{\mu_{1}}{1-\tilde{\mu}_{2}}f_{2}(2,2)+f_{2}^{(1,2)}\right).
\end{eqnarray*}

%10

\item \begin{eqnarray*} f_{1}\left(2,3\right)&=&\tilde{\mu}_{2}\hat{\mu}_{1}r_{2}+\tilde{\mu}_{2}\hat{\mu}_{1}R_{2}^{(2)}\left(1\right)+r_{2}\frac{\tilde{\mu}_{2}\hat{\mu}_{1}}{1-\tilde{\mu}_{2}}f_{2}(2)+r_{2}\tilde{\mu}_{2}\hat{F}_{2,1}^{(1)}(1).
\end{eqnarray*}

%11

\item \begin{eqnarray*} f_{1}\left(3,3\right)&=&\hat{\mu}_{1}^{2}R_{2}^{(2)}\left(1\right)+r_{2}\hat{P}_{1}^{(2)}\left(1\right)+2r_{2}\frac{\hat{\mu}_{1}^{2}}{1-\tilde{\mu}_{2}}f_{2}(2)+\hat{\mu}_{1}^{2}\tilde{\theta}_{2}^{(2)}\left(1\right)f_{2}(2)+\frac{1}{1-\tilde{\mu}_{2}}\hat{P}_{1}^{(2)}\left(1\right)f_{2}(2)\\
&+&\frac{\hat{\mu}_{1}^{2}}{1-\tilde{\mu}_{2}}f_{2}(2,2)+2r_{2}\hat{\mu}_{1}\hat{F}_{2,1}^{(1)}(1)+2\frac{\hat{\mu}_{1}}{1-\tilde{\mu}_{2}}f_{2}(2)\hat{F}_{2,1}^{(1)}(1)+\hat{f}_{2,1}^{(2)}(1).
\end{eqnarray*}

%12

\item \begin{eqnarray*}
f_{1}\left(4,3\right)&=&r_{2}\hat{\mu}_{2}\hat{\mu}_{1}+\hat{\mu}_{1}\hat{\mu}_{2}R_{2}^{(2)}(1)+\frac{\hat{\mu}_{1}\hat{\mu}_{2}}{1-\tilde{\mu}_{2}}f_{2}\left(2\right)+2r_{2}\frac{\hat{\mu}_{1}\hat{\mu}_{2}}{1-\tilde{\mu}_{2}}f_{2}\left(2\right)+\hat{\mu}_{2}\hat{\mu}_{1}\tilde{\theta}_{2}^{(2)}\left(1\right)f_{2}\left(2\right)\\
&+&r_{2}\hat{\mu}_{1}\hat{F}_{2,2}^{(1)}(1)+\frac{\hat{\mu}_{1}}{1-\tilde{\mu}_{2}}f_{2}\left(2\right)\hat{F}_{2,2}^{(1)}(1)+\hat{\mu}_{1}\hat{\mu}_{2}\left(\frac{1}{1-\tilde{\mu}_{2}}\right)^{2}f_{2}(2,2)+r_{2}\hat{\mu}_{2}\hat{F}_{2,1}^{(1)}(1)\\
&+&\frac{\hat{\mu}_{2}}{1-\tilde{\mu}_{2}}f_{2}\left(2\right)\hat{F}_{2,1}^{(1)}(1)+\hat{f}_{2}(1,2).
\end{eqnarray*}
%___________________________________________________________________________________________
%\subsubsection{Mixtas para $w_{2}$:}
%___________________________________________________________________________________________
%13

\item \begin{eqnarray*}
f_{1}\left(1,4\right)&=&r_{2}\mu_{1}\hat{\mu}_{2}+\mu_{1}\hat{\mu}_{2}R_{2}^{(2)}(1)+\frac{\mu_{1}\hat{\mu}_{2}}{1-\tilde{\mu}_{2}}f_{2}(2)+r_{2}\frac{\mu_{1}\hat{\mu}_{2}}{1-\tilde{\mu}_{2}}f_{2}(2)+\mu_{1}\hat{\mu}_{2}\tilde{\theta}_{2}^{(2)}\left(1\right)f_{2}(2)\\
&+&r_{2}\mu_{1}\hat{F}_{2,2}^{(1)}(1)+r_{2}\hat{\mu}_{2}\left(\frac{\mu_{1}}{1-\tilde{\mu}_{2}}f_{2}(2)+f_{2}(1)\right)+\hat{F}_{2,2}^{(1)}(1)\left(\frac{\mu_{1}}{1-\tilde{\mu}_{2}}f_{2}(2)+f_{2}(1)\right)\\
&+&\frac{\hat{\mu}_{2}}{1-\tilde{\mu}_{2}}\left(\frac{\mu_{1}}{1-\tilde{\mu}_{2}}f_{2}(2,2)+f_{2}(1,2)\right).
\end{eqnarray*}

%14

\item \begin{eqnarray*} f_{1}\left(2,4\right)
&=&r_{2}\tilde{\mu}_{2}\hat{\mu}_{2}+\tilde{\mu}_{2}\hat{\mu}_{2}R_{2}^{(2)}(1)+r_{2}\frac{\tilde{\mu}_{2}\hat{\mu}_{2}}{1-\tilde{\mu}_{2}}f_{2}(2)+r_{2}\tilde{\mu}_{2}\hat{F}_{2,2}^{(1)}(1).
\end{eqnarray*}


%15
\item \begin{eqnarray*} f_{1}\left(3,4\right)&=&r_{2}\hat{\mu}_{1}\hat{\mu}_{2}+\hat{\mu}_{1}\hat{\mu}_{2}R_{2}^{(2)}\left(1\right)+\frac{\hat{\mu}_{1}\hat{\mu}_{2}}{1-\tilde{\mu}_{2}}f_{2}(2)+2r_{2}\frac{\hat{\mu}_{1}\hat{\mu}_{2}}{1-\tilde{\mu}_{2}}f_{2}(2)+\hat{\mu}_{1}\hat{\mu}_{2}\theta_{2}^{(2)}\left(1\right)f_{2}(2)\\
&+&r_{2}\hat{\mu}_{1}\hat{F}_{2,2}^{(1)}(1)+\frac{\hat{\mu}_{1}}{1-\tilde{\mu}_{2}}f_{2}(2)\hat{F}_{2,2}^{(1)}(1)+\hat{\mu}_{1}\hat{\mu}_{2}\left(\frac{1}{1-\tilde{\mu}_{2}}\right)^{2}f_{2}(2,2)+r_{2}\hat{\mu}_{2}\hat{F}_{2,2}^{(1)}(1)\\
&+&\frac{\hat{\mu}_{2}}{1-\tilde{\mu}_{2}}f_{2}(2)\hat{F}_{2,1}^{(1)}(1)+\hat{f}_{2}^{(2)}(1,2).
\end{eqnarray*}

%16

\item \begin{eqnarray*} f_{1}\left(4,4\right)&=&\hat{\mu}_{2}^{2}R_{2}^{(2)}(1)+r_{2}\hat{P}_{2}^{(2)}\left(1\right)+2r_{2}\frac{\hat{\mu}_{2}^{2}}{1-\tilde{\mu}_{2}}f_{2}(2)+\hat{\mu}_{2}^{2}\tilde{\theta}_{2}^{(2)}\left(1\right)f_{2}(2)+\frac{1}{1-\tilde{\mu}_{2}}\hat{P}_{2}^{(2)}\left(1\right)f_{2}(2)\\
&+&2r_{2}\hat{\mu}_{2}\hat{F}_{2,2}^{(1)}(1)+2\frac{\hat{\mu}_{2}}{1-\tilde{\mu}_{2}}f_{2}(2)\hat{F}_{2,2}^{(1)}(1)+\left(\frac{\hat{\mu}_{2}}{1-\tilde{\mu}_{2}}\right)^{2}f_{2}(2,2)+\hat{f}_{2,2}^{(2)}(1).
\end{eqnarray*}
%\end{enumerate}
%___________________________________________________________________________________________
%
%\subsection{Derivadas de Segundo Orden para $F_{2}$}
%___________________________________________________________________________________________


%\begin{enumerate}

%___________________________________________________________________________________________
%\subsubsection{Mixtas para $z_{1}$:}
%___________________________________________________________________________________________

%17

\item \begin{eqnarray*} f_{2}\left(1,1\right)&=&r_{1}P_{1}^{(2)}\left(1\right)+\mu_{1}^{2}R_{1}^{(2)}\left(1\right).
\end{eqnarray*}

%18

\item \begin{eqnarray*} f_{2}\left(2,1\right)&=&\mu_{1}\tilde{\mu}_{2}r_{1}+\mu_{1}\tilde{\mu}_{2}R_{1}^{(2)}(1)+
r_{1}\mu_{1}\left(\frac{\tilde{\mu}_{2}}{1-\mu_{1}}f_{1}(1)+f_{1}(2)\right).
\end{eqnarray*}

%19

\item \begin{eqnarray*} f_{2}\left(3,1\right)&=&r_{1}\mu_{1}\hat{\mu}_{1}+\mu_{1}\hat{\mu}_{1}R_{1}^{(2)}\left(1\right)+r_{1}\frac{\mu_{1}\hat{\mu}_{1}}{1-\mu_{1}}f_{1}(1)+r_{1}\mu_{1}\hat{F}_{1,1}^{(1)}(1).
\end{eqnarray*}

%20

\item \begin{eqnarray*}
f_{2}\left(4,1\right)&=&\mu_{1}\hat{\mu}_{2}r_{1}+\mu_{1}\hat{\mu}_{2}R_{1}^{(2)}\left(1\right)+r_{1}\mu_{1}\hat{F}_{1,2}^{(1)}(1)+r_{1}\frac{\mu_{1}\hat{\mu}_{2}}{1-\mu_{1}}f_{1}(1).
\end{eqnarray*}
%___________________________________________________________________________________________
%\subsubsection{Mixtas para $z_{2}$:}
%___________________________________________________________________________________________
%21
\item \begin{eqnarray*}
f_{2}\left(1,2\right)&=&r_{1}\mu_{1}\tilde{\mu}_{2}+\mu_{1}\tilde{\mu}_{2}R_{1}^{(2)}\left(1\right)+r_{1}\mu_{1}\left(\frac{\tilde{\mu}_{2}}{1-\mu_{1}}f_{1}(1)+f_{1}(2)\right).
\end{eqnarray*}

%22

\item \begin{eqnarray*}
f_{2}\left(2,2\right)&=&\tilde{\mu}_{2}^{2}R_{1}^{(2)}\left(1\right)+r_{1}\tilde{P}_{2}^{(2)}\left(1\right)+2r_{1}\tilde{\mu}_{2}\left(\frac{\tilde{\mu}_{2}}{1-\mu_{1}}f_{1}(1)+f_{1}(2)\right)+f_{1}(2,2)\\
&+&\tilde{\mu}_{2}^{2}\theta_{1}^{(2)}\left(1\right)f_{1}(1)+\frac{1}{1-\mu_{1}}\tilde{P}_{2}^{(2)}\left(1\right)f_{1}(1)+\frac{\tilde{\mu}_{2}}{1-\mu_{1}}f_{1}(1,2)\\
&+&\frac{\tilde{\mu}_{2}}{1-\mu_{1}}\left(\frac{\tilde{\mu}_{2}}{1-\mu_{1}}f_{1}(1,1)+f_{1}(1,2)\right).
\end{eqnarray*}

%23

\item \begin{eqnarray*}
f_{2}\left(3,2\right)&=&\tilde{\mu}_{2}\hat{\mu}_{1}r_{1}+\tilde{\mu}_{2}\hat{\mu}_{1}R_{1}^{(2)}\left(1\right)+r_{1}\frac{\tilde{\mu}_{2}\hat{\mu}_{1}}{1-\mu_{1}}f_{1}(1)+\hat{\mu}_{1}r_{1}\left(\frac{\tilde{\mu}_{2}}{1-\mu_{1}}f_{1}(1)+f_{1}(2)\right)+r_{1}\tilde{\mu}_{2}\hat{F}_{1,1}^{(1)}(1)\\
&+&\left(\frac{\tilde{\mu}_{2}}{1-\mu_{1}}f_{1}(1)+f_{1}(2)\right)\hat{F}_{1,1}^{(1)}(1)+\frac{\tilde{\mu}_{2}\hat{\mu}_{1}}{1-\mu_{1}}f_{1}(1)+\tilde{\mu}_{2}\hat{\mu}_{1}\theta_{1}^{(2)}\left(1\right)f_{1}(1)+\frac{\hat{\mu}_{1}}{1-\mu_{1}}f_{1}(1,2)\\
&+&\left(\frac{1}{1-\mu_{1}}\right)^{2}\tilde{\mu}_{2}\hat{\mu}_{1}f_{1}(1,1).
\end{eqnarray*}

%24


\item \begin{eqnarray*}
f_{2}\left(4,2\right)&=&\hat{\mu}_{2}\tilde{\mu}_{2}r_{1}+\hat{\mu}_{2}\tilde{\mu}_{2}R_{1}^{(2)}(1)+r_{1}\tilde{\mu}_{2}\hat{F}_{1,2}^{(1)}(1)+r_{1}\frac{\hat{\mu}_{2}\tilde{\mu}_{2}}{1-\mu_{1}}f_{1}(1)+\hat{\mu}_{2}r_{1}\left(\frac{\tilde{\mu}_{2}}{1-\mu_{1}}f_{1}(1)+f_{1}(2)\right)\\
&+&\left(\frac{\tilde{\mu}_{2}}{1-\mu_{1}}f_{1}(1)+f_{1}(2)\right)\hat{F}_{1,2}^{(1)}(1)+\frac{\tilde{\mu}_{2}\hat{\mu_{2}}}{1-\mu_{1}}f_{1}(1)+\hat{\mu}_{2}\tilde{\mu}_{2}\theta_{1}^{(2)}\left(1\right)f_{1}(1)+\frac{\hat{\mu}_{2}}{1-\mu_{1}}f_{1}(1,2)\\
&+&\tilde{\mu}_{2}\hat{\mu}_{2}\left(\frac{1}{1-\mu_{1}}\right)^{2}f_{1}(1,1).
\end{eqnarray*}
%___________________________________________________________________________________________
%\subsubsection{Mixtas para $w_{1}$:}
%___________________________________________________________________________________________

%25

\item \begin{eqnarray*} f_{2}\left(1,3\right)&=&r_{1}\mu_{1}\hat{\mu}_{1}+\mu_{1}\hat{\mu}_{1}R_{1}^{(2)}(1)+r_{1}\frac{\mu_{1}\hat{\mu}_{1}}{1-\mu_{1}}f_{1}(1)+r_{1}\mu_{1}\hat{F}_{1,1}^{(1)}(1).
\end{eqnarray*}

%26

\item \begin{eqnarray*} f_{2}\left(2,3\right)&=&r_{1}\hat{\mu}_{1}\tilde{\mu}_{2}+\tilde{\mu}_{2}\hat{\mu}_{1}R_{1}^{(2)}\left(1\right)+\frac{\hat{\mu}_{1}\tilde{\mu}_{2}}{1-\mu_{1}}f_{1}(1)+r_{1}\frac{\hat{\mu}_{1}\tilde{\mu}_{2}}{1-\mu_{1}}f_{1}(1)+\hat{\mu}_{1}\tilde{\mu}_{2}\theta_{1}^{(2)}\left(1\right)f_{1}(1)\\
&+&r_{1}\hat{\mu}_{1}\left(f_{1}(1)+\frac{\tilde{\mu}_{2}}{1-\mu_{1}}f_{1}(1)\right)+
r_{1}\tilde{\mu}_{2}\hat{F}_{1,1}(1)+\left(f_{1}(2)+\frac{\tilde{\mu}_{2}}{1-\mu_{1}}f_{1}(1)\right)\hat{F}_{1,1}(1)\\
&+&\frac{\hat{\mu}_{1}}{1-\mu_{1}}\left(f_{1}(1,2)+\frac{\tilde{\mu}_{2}}{1-\mu_{1}}f_{1}(1,1)\right).
\end{eqnarray*}

%27

\item \begin{eqnarray*} f_{2}\left(3,3\right)&=&\hat{\mu}_{1}^{2}R_{1}^{(2)}\left(1\right)+r_{1}\hat{P}_{1}^{(2)}\left(1\right)+2r_{1}\frac{\hat{\mu}_{1}^{2}}{1-\mu_{1}}f_{1}(1)+\hat{\mu}_{1}^{2}\theta_{1}^{(2)}\left(1\right)f_{1}(1)\\
&+&\frac{1}{1-\mu_{1}}\hat{P}_{1}^{(2)}\left(1\right)f_{1}(1)+2r_{1}\hat{\mu}_{1}\hat{F}_{1,1}^{(1)}(1)+2\frac{\hat{\mu}_{1}}{1-\mu_{1}}f_{1}(1)\hat{F}_{1,1}(1)\\
&+&\left(\frac{\hat{\mu}_{1}}{1-\mu_{1}}\right)^{2}f_{1}(1,1)+\hat{f}_{1,1}^{(2)}(1).
\end{eqnarray*}

%28

\item \begin{eqnarray*}
f_{2}\left(4,3\right)&=&r_{1}\hat{\mu}_{1}\hat{\mu}_{2}+\hat{\mu}_{1}\hat{\mu}_{2}R_{1}^{(2)}\left(1\right)+r_{1}\hat{\mu}_{1}\hat{F}_{1,2}(1)+
\frac{\hat{\mu}_{1}\hat{\mu}_{2}}{1-\mu_{1}}f_{1}(1)+2r_{1}\frac{\hat{\mu}_{1}\hat{\mu}_{2}}{1-\mu_{1}}f_{1}(1)\\
&+&\hat{\mu}_{1}\hat{\mu}_{2}\theta_{1}^{(2)}\left(1\right)f_{1}(1)+\frac{\hat{\mu}_{1}}{1-\mu_{1}}f_{1}(1)\hat{F}_{1,2}(1)+r_{1}\hat{\mu}_{2}\hat{F}_{1,1}(1)+\frac{\hat{\mu}_{2}}{1-\mu_{1}}\hat{F}_{1,1}(1)f_{1}(1)\\
&+&\hat{f}_{1}^{(2)}(1,2)+\hat{\mu}_{1}\hat{\mu}_{2}\left(\frac{1}{1-\mu_{1}}\right)^{2}f_{1}(2,2).
\end{eqnarray*}
%___________________________________________________________________________________________
%\subsubsection{Mixtas para $w_{2}$:}
%___________________________________________________________________________________________

%29

\item \begin{eqnarray*} f_{2}\left(1,4\right)&=&r_{1}\mu_{1}\hat{\mu}_{2}+\mu_{1}\hat{\mu}_{2}R_{1}^{(2)}\left(1\right)+r_{1}\mu_{1}\hat{F}_{1,2}(1)+r_{1}\frac{\mu_{1}\hat{\mu}_{2}}{1-\mu_{1}}f_{1}(1).
\end{eqnarray*}


%30

\item \begin{eqnarray*} f_{2}\left(2,4\right)&=&r_{1}\hat{\mu}_{2}\tilde{\mu}_{2}+\hat{\mu}_{2}\tilde{\mu}_{2}R_{1}^{(2)}\left(1\right)+r_{1}\tilde{\mu}_{2}\hat{F}_{1,2}(1)+\frac{\hat{\mu}_{2}\tilde{\mu}_{2}}{1-\mu_{1}}f_{1}(1)+r_{1}\frac{\hat{\mu}_{2}\tilde{\mu}_{2}}{1-\mu_{1}}f_{1}(1)\\
&+&\hat{\mu}_{2}\tilde{\mu}_{2}\theta_{1}^{(2)}\left(1\right)f_{1}(1)+r_{1}\hat{\mu}_{2}\left(f_{1}(2)+\frac{\tilde{\mu}_{2}}{1-\mu_{1}}f_{1}(1)\right)+\left(f_{1}(2)+\frac{\tilde{\mu}_{2}}{1-\mu_{1}}f_{1}(1)\right)\hat{F}_{1,2}(1)\\&+&\frac{\hat{\mu}_{2}}{1-\mu_{1}}\left(f_{1}(1,2)+\frac{\tilde{\mu}_{2}}{1-\mu_{1}}f_{1}(1,1)\right).
\end{eqnarray*}

%31

\item \begin{eqnarray*}
f_{2}\left(3,4\right)&=&r_{1}\hat{\mu}_{1}\hat{\mu}_{2}+\hat{\mu}_{1}\hat{\mu}_{2}R_{1}^{(2)}\left(1\right)+r_{1}\hat{\mu}_{1}\hat{F}_{1,2}(1)+
\frac{\hat{\mu}_{1}\hat{\mu}_{2}}{1-\mu_{1}}f_{1}(1)+2r_{1}\frac{\hat{\mu}_{1}\hat{\mu}_{2}}{1-\mu_{1}}f_{1}(1)\\
&+&\hat{\mu}_{1}\hat{\mu}_{2}\theta_{1}^{(2)}\left(1\right)f_{1}(1)+\frac{\hat{\mu}_{1}}{1-\mu_{1}}\hat{F}_{1,2}(1)f_{1}(1)+r_{1}\hat{\mu}_{2}\hat{F}_{1,1}(1)+\frac{\hat{\mu}_{2}}{1-\mu_{1}}\hat{F}_{1,1}(1)f_{1}(1)\\
&+&\hat{f}_{1}^{(2)}(1,2)+\hat{\mu}_{1}\hat{\mu}_{2}\left(\frac{1}{1-\mu_{1}}\right)^{2}f_{1}(1,1).
\end{eqnarray*}

%32

\item \begin{eqnarray*} f_{2}\left(4,4\right)&=&\hat{\mu}_{2}R_{1}^{(2)}\left(1\right)+r_{1}\hat{P}_{2}^{(2)}\left(1\right)+2r_{1}\hat{\mu}_{2}\hat{F}_{1}^{(0,1)}+\hat{f}_{1,2}^{(2)}(1)+2r_{1}\frac{\hat{\mu}_{2}^{2}}{1-\mu_{1}}f_{1}(1)+\hat{\mu}_{2}^{2}\theta_{1}^{(2)}\left(1\right)f_{1}(1)\\
&+&\frac{1}{1-\mu_{1}}\hat{P}_{2}^{(2)}\left(1\right)f_{1}(1) +
2\frac{\hat{\mu}_{2}}{1-\mu_{1}}f_{1}(1)\hat{F}_{1,2}(1)+\left(\frac{\hat{\mu}_{2}}{1-\mu_{1}}\right)^{2}f_{1}(1,1).
\end{eqnarray*}
%\end{enumerate}

%___________________________________________________________________________________________
%
%\subsection{Derivadas de Segundo Orden para $\hat{F}_{1}$}
%___________________________________________________________________________________________


%\begin{enumerate}
%___________________________________________________________________________________________
%\subsubsection{Mixtas para $z_{1}$:}
%___________________________________________________________________________________________
%33

\item \begin{eqnarray*} \hat{f}_{1}\left(1,1\right)&=&\hat{r}_{2}P_{1}^{(2)}\left(1\right)+
\mu_{1}^{2}\hat{R}_{2}^{(2)}\left(1\right)+
2\hat{r}_{2}\frac{\mu_{1}^{2}}{1-\hat{\mu}_{2}}\hat{f}_{2}(2)+
\frac{1}{1-\hat{\mu}_{2}}P_{1}^{(2)}\left(1\right)\hat{f}_{2}(2)+
\mu_{1}^{2}\hat{\theta}_{2}^{(2)}\left(1\right)\hat{f}_{2}(2)\\
&+&\left(\frac{\mu_{1}^{2}}{1-\hat{\mu}_{2}}\right)^{2}\hat{f}_{2}(2,2)+2\hat{r}_{2}\mu_{1}F_{2,1}(1)+2\frac{\mu_{1}}{1-\hat{\mu}_{2}}\hat{f}_{2}(2)F_{2,1}(1)+F_{2,1}^{(2)}(1).
\end{eqnarray*}

%34

\item \begin{eqnarray*} \hat{f}_{1}\left(2,1\right)&=&\hat{r}_{2}\mu_{1}\tilde{\mu}_{2}+\mu_{1}\tilde{\mu}_{2}\hat{R}_{2}^{(2)}\left(1\right)+\hat{r}_{2}\mu_{1}F_{2,2}(1)+
\frac{\mu_{1}\tilde{\mu}_{2}}{1-\hat{\mu}_{2}}\hat{f}_{2}(2)+2\hat{r}_{2}\frac{\mu_{1}\tilde{\mu}_{2}}{1-\hat{\mu}_{2}}\hat{f}_{2}(2)\\
&+&\mu_{1}\tilde{\mu}_{2}\hat{\theta}_{2}^{(2)}\left(1\right)\hat{f}_{2}(2)+\frac{\mu_{1}}{1-\hat{\mu}_{2}}F_{2,2}(1)\hat{f}_{2}(2)+\mu_{1} \tilde{\mu}_{2}\left(\frac{1}{1-\hat{\mu}_{2}}\right)^{2}\hat{f}_{2}(2,2)+\hat{r}_{2}\tilde{\mu}_{2}F_{2,1}(1)\\
&+&\frac{\tilde{\mu}_{2}}{1-\hat{\mu}_{2}}\hat{f}_{2}(2)F_{2,1}(1)+f_{2,1}^{(2)}(1).
\end{eqnarray*}


%35

\item \begin{eqnarray*} \hat{f}_{1}\left(3,1\right)&=&\hat{r}_{2}\mu_{1}\hat{\mu}_{1}+\mu_{1}\hat{\mu}_{1}\hat{R}_{2}^{(2)}\left(1\right)+\hat{r}_{2}\frac{\mu_{1}\hat{\mu}_{1}}{1-\hat{\mu}_{2}}\hat{f}_{2}(2)+\hat{r}_{2}\hat{\mu}_{1}F_{2,1}(1)+\hat{r}_{2}\mu_{1}\hat{f}_{2}(1)\\
&+&F_{2,1}(1)\hat{f}_{2}(1)+\frac{\mu_{1}}{1-\hat{\mu}_{2}}\hat{f}_{2}(1,2).
\end{eqnarray*}

%36

\item \begin{eqnarray*} \hat{f}_{1}\left(4,1\right)&=&\hat{r}_{2}\mu_{1}\hat{\mu}_{2}+\mu_{1}\hat{\mu}_{2}\hat{R}_{2}^{(2)}\left(1\right)+\frac{\mu_{1}\hat{\mu}_{2}}{1-\hat{\mu}_{2}}\hat{f}_{2}(2)+2\hat{r}_{2}\frac{\mu_{1}\hat{\mu}_{2}}{1-\hat{\mu}_{2}}\hat{f}_{2}(2)+\mu_{1}\hat{\mu}_{2}\hat{\theta}_{2}^{(2)}\left(1\right)\hat{f}_{2}(2)\\
&+&\mu_{1}\hat{\mu}_{2}\left(\frac{1}{1-\hat{\mu}_{2}}\right)^{2}\hat{f}_{2}(2,2)+\hat{r}_{2}\hat{\mu}_{2}F_{2,1}(1)+\frac{\hat{\mu}_{2}}{1-\hat{\mu}_{2}}\hat{f}_{2}(2)F_{2,1}(1).
\end{eqnarray*}
%___________________________________________________________________________________________
%\subsubsection{Mixtas para $z_{2}$:}
%___________________________________________________________________________________________

%37

\item \begin{eqnarray*} \hat{f}_{1}\left(1,2\right)&=&\hat{r}_{2}\mu_{1}\tilde{\mu}_{2}+\mu_{1}\tilde{\mu}_{2}\hat{R}_{2}^{(2)}\left(1\right)+\mu_{1}\hat{r}_{2}F_{2,2}(1)+
\frac{\mu_{1}\tilde{\mu}_{2}}{1-\hat{\mu}_{2}}\hat{f}_{2}(2)+2\hat{r}_{2}\frac{\mu_{1}\tilde{\mu}_{2}}{1-\hat{\mu}_{2}}\hat{f}_{2}(2)\\
&+&\mu_{1}\tilde{\mu}_{2}\hat{\theta}_{2}^{(2)}\left(1\right)\hat{f}_{2}(2)+\frac{\mu_{1}}{1-\hat{\mu}_{2}}F_{2,2}(1)\hat{f}_{2}(2)+\mu_{1}\tilde{\mu}_{2}\left(\frac{1}{1-\hat{\mu}_{2}}\right)^{2}\hat{f}_{2}(2,2)\\
&+&\hat{r}_{2}\tilde{\mu}_{2}F_{2,1}(1)+\frac{\tilde{\mu}_{2}}{1-\hat{\mu}_{2}}\hat{f}_{2}(2)F_{2,1}(1)+f_{2}^{(2)}(1,2).
\end{eqnarray*}

%38

\item \begin{eqnarray*}\hat{f}_{1}\left(2,2\right)&=&\hat{r}_{2}\tilde{P}_{2}^{(2)}\left(1\right)+\tilde{\mu}_{2}^{2}\hat{R}_{2}^{(2)}\left(1\right)+2\hat{r}_{2}\tilde{\mu}_{2}F_{2,2}(1)+2\hat{r}_{2}\frac{\tilde{\mu}_{2}^{2}}{1-\hat{\mu}_{2}}\hat{f}_{2}(2)\\
&+&\frac{1}{1-\hat{\mu}_{2}}\tilde{P}_{2}^{(2)}\left(1\right)\hat{f}_{2}(2)+\tilde{\mu}_{2}^{2}\hat{\theta}_{2}^{(2)}\left(1\right)\hat{f}_{2}(2)+2\frac{\tilde{\mu}_{2}}{1-\hat{\mu}_{2}}F_{2,2}(1)\hat{f}_{2}(2)\\
&+&f_{2,2}^{(2)}(1)+\left(\frac{\tilde{\mu}_{2}}{1-\hat{\mu}_{2}}\right)^{2}\hat{f}_{2}(2,2).
\end{eqnarray*}

%39

\item \begin{eqnarray*} \hat{f}_{1}\left(3,2\right)&=&\hat{r}_{2}\tilde{\mu}_{2}\hat{\mu}_{1}+\tilde{\mu}_{2}\hat{\mu}_{1}\hat{R}_{2}^{(2)}\left(1\right)+\hat{r}_{2}\hat{\mu}_{1}F_{2,2}(1)+\hat{r}_{2}\frac{\tilde{\mu}_{2}\hat{\mu}_{1}}{1-\hat{\mu}_{2}}\hat{f}_{2}(2)+\hat{r}_{2}\tilde{\mu}_{2}\hat{f}_{2}(1)+F_{2,2}(1)\hat{f}_{2}(1)\\
&+&\frac{\tilde{\mu}_{2}}{1-\hat{\mu}_{2}}\hat{f}_{2}(1,2).
\end{eqnarray*}

%40

\item \begin{eqnarray*} \hat{f}_{1}\left(4,2\right)&=&\hat{r}_{2}\tilde{\mu}_{2}\hat{\mu}_{2}+\tilde{\mu}_{2}\hat{\mu}_{2}\hat{R}_{2}^{(2)}\left(1\right)+\hat{r}_{2}\hat{\mu}_{2}F_{2,2}(1)+
\frac{\tilde{\mu}_{2}\hat{\mu}_{2}}{1-\hat{\mu}_{2}}\hat{f}_{2}(2)+2\hat{r}_{2}\frac{\tilde{\mu}_{2}\hat{\mu}_{2}}{1-\hat{\mu}_{2}}\hat{f}_{2}(2)\\
&+&\tilde{\mu}_{2}\hat{\mu}_{2}\hat{\theta}_{2}^{(2)}\left(1\right)\hat{f}_{2}(2)+\frac{\hat{\mu}_{2}}{1-\hat{\mu}_{2}}F_{2,2}(1)\hat{f}_{2}(1)+\tilde{\mu}_{2}\hat{\mu}_{2}\left(\frac{1}{1-\hat{\mu}_{2}}\right)\hat{f}_{2}(2,2).
\end{eqnarray*}
%___________________________________________________________________________________________
%\subsubsection{Mixtas para $w_{1}$:}
%___________________________________________________________________________________________

%41


\item \begin{eqnarray*} \hat{f}_{1}\left(1,3\right)&=&\hat{r}_{2}\mu_{1}\hat{\mu}_{1}+\mu_{1}\hat{\mu}_{1}\hat{R}_{2}^{(2)}\left(1\right)+\hat{r}_{2}\frac{\mu_{1}\hat{\mu}_{1}}{1-\hat{\mu}_{2}}\hat{f}_{2}(2)+\hat{r}_{2}\hat{\mu}_{1}F_{2,1}(1)+\hat{r}_{2}\mu_{1}\hat{f}_{2}(1)\\
&+&F_{2,1}(1)\hat{f}_{2}(1)+\frac{\mu_{1}}{1-\hat{\mu}_{2}}\hat{f}_{2}(1,2).
\end{eqnarray*}


%42

\item \begin{eqnarray*} \hat{f}_{1}\left(2,3\right)&=&\hat{r}_{2}\tilde{\mu}_{2}\hat{\mu}_{1}+\tilde{\mu}_{2}\hat{\mu}_{1}\hat{R}_{2}^{(2)}\left(1\right)+\hat{r}_{2}\hat{\mu}_{1}F_{2,2}(1)+\hat{r}_{2}\frac{\tilde{\mu}_{2}\hat{\mu}_{1}}{1-\hat{\mu}_{2}}\hat{f}_{2}(2)+\hat{r}_{2}\tilde{\mu}_{2}\hat{f}_{2}(1)\\
&+&F_{2,2}(1)\hat{f}_{2}(1)+\frac{\tilde{\mu}_{2}}{1-\hat{\mu}_{2}}\hat{f}_{2}(1,2).
\end{eqnarray*}


%43

\item \begin{eqnarray*} \hat{f}_{1}\left(3,3\right)&=&\hat{r}_{2}\hat{P}_{1}^{(2)}\left(1\right)+\hat{\mu}_{1}^{2}\hat{R}_{2}^{(2)}\left(1\right)+2\hat{r}_{2}\hat{\mu}_{1}\hat{f}_{2}(1)+\hat{f}_{2}(1,1).
\end{eqnarray*}


%44

\item \begin{eqnarray*} \hat{f}_{1}\left(4,3\right)&=&\hat{r}_{2}\hat{\mu}_{1}\hat{\mu}_{2}+\hat{\mu}_{1}\hat{\mu}_{2}\hat{R}_{2}^{(2)}\left(1\right)+
\hat{r}_{2}\frac{\hat{\mu}_{2}\hat{\mu}_{1}}{1-\hat{\mu}_{2}}\hat{f}_{2}(2)+\hat{r}_{2}\hat{\mu}_{2}\hat{f}_{2}(1)+\frac{\hat{\mu}_{2}}{1-\hat{\mu}_{2}}\hat{f}_{2}(1,2).
\end{eqnarray*}
%___________________________________________________________________________________________
%\subsubsection{Mixtas para $w_{2}$:}
%___________________________________________________________________________________________


%45


\item \begin{eqnarray*} \hat{f}_{1}\left(1,4\right)&=&\hat{r}_{2}\mu_{1}\hat{\mu}_{2}+\mu_{1}\hat{\mu}_{2}\hat{R}_{2}^{(2)}\left(1\right)+
\frac{\mu_{1}\hat{\mu}_{2}}{1-\hat{\mu}_{2}}\hat{f}_{2}(2) +2\hat{r}_{2}\frac{\mu_{1}\hat{\mu}_{2}}{1-\hat{\mu}_{2}}\hat{f}_{2}(2)\\
&+&\mu_{1}\hat{\mu}_{2}\hat{\theta}_{2}^{(2)}\left(1\right)\hat{f}_{2}(2)+\mu_{1}\hat{\mu}_{2}\left(\frac{1}{1-\hat{\mu}_{2}}\right)^{2}\hat{f}_{2}(2,2)+\hat{r}_{2}\hat{\mu}_{2}F_{2,1}(1)+\frac{\hat{\mu}_{2}}{1-\hat{\mu}_{2}}\hat{f}_{2}(2)F_{2,1}(1).\end{eqnarray*}


%46
\item \begin{eqnarray*} \hat{f}_{1}\left(2,4\right)&=&\hat{r}_{2}\tilde{\mu}_{2}\hat{\mu}_{2}+\tilde{\mu}_{2}\hat{\mu}_{2}\hat{R}_{2}^{(2)}\left(1\right)+\hat{r}_{2}\hat{\mu}_{2}F_{2,2}(1)+\frac{\tilde{\mu}_{2}\hat{\mu}_{2}}{1-\hat{\mu}_{2}}\hat{f}_{2}(2)+2\hat{r}_{2}\frac{\tilde{\mu}_{2}\hat{\mu}_{2}}{1-\hat{\mu}_{2}}\hat{f}_{2}(2)\\
&+&\tilde{\mu}_{2}\hat{\mu}_{2}\hat{\theta}_{2}^{(2)}\left(1\right)\hat{f}_{2}(2)+\frac{\hat{\mu}_{2}}{1-\hat{\mu}_{2}}\hat{f}_{2}(2)F_{2,2}(1)+\tilde{\mu}_{2}\hat{\mu}_{2}\left(\frac{1}{1-\hat{\mu}_{2}}\right)^{2}\hat{f}_{2}(2,2).
\end{eqnarray*}

%47

\item \begin{eqnarray*} \hat{f}_{1}\left(3,4\right)&=&\hat{r}_{2}\hat{\mu}_{1}\hat{\mu}_{2}+\hat{\mu}_{1}\hat{\mu}_{2}\hat{R}_{2}^{(2)}\left(1\right)+
\hat{r}_{2}\frac{\hat{\mu}_{1}\hat{\mu}_{2}}{1-\hat{\mu}_{2}}\hat{f}_{2}(2)+
\hat{r}_{2}\hat{\mu}_{2}\hat{f}_{2}(1)+\frac{\hat{\mu}_{2}}{1-\hat{\mu}_{2}}\hat{f}_{2}(1,2).
\end{eqnarray*}

%48

\item \begin{eqnarray*} \hat{f}_{1}\left(4,4\right)&=&\hat{r}_{2}P_{2}^{(2)}\left(1\right)+\hat{\mu}_{2}^{2}\hat{R}_{2}^{(2)}\left(1\right)+2\hat{r}_{2}\frac{\hat{\mu}_{2}^{2}}{1-\hat{\mu}_{2}}\hat{f}_{2}(2)+\frac{1}{1-\hat{\mu}_{2}}\hat{P}_{2}^{(2)}\left(1\right)\hat{f}_{2}(2)\\
&+&\hat{\mu}_{2}^{2}\hat{\theta}_{2}^{(2)}\left(1\right)\hat{f}_{2}(2)+\left(\frac{\hat{\mu}_{2}}{1-\hat{\mu}_{2}}\right)^{2}\hat{f}_{2}(2,2).
\end{eqnarray*}


%\end{enumerate}



%___________________________________________________________________________________________
%
%\subsection{Derivadas de Segundo Orden para $\hat{F}_{2}$}
%___________________________________________________________________________________________
%\begin{enumerate}
%___________________________________________________________________________________________
%\subsubsection{Mixtas para $z_{1}$:}
%___________________________________________________________________________________________
%49

\item \begin{eqnarray*} \hat{f}_{2}\left(,1\right)&=&\hat{r}_{1}P_{1}^{(2)}\left(1\right)+
\mu_{1}^{2}\hat{R}_{1}^{(2)}\left(1\right)+2\hat{r}_{1}\mu_{1}F_{1,1}(1)+
2\hat{r}_{1}\frac{\mu_{1}^{2}}{1-\hat{\mu}_{1}}\hat{f}_{1}(1)+\frac{1}{1-\hat{\mu}_{1}}P_{1}^{(2)}\left(1\right)\hat{f}_{1}(1)\\
&+&\mu_{1}^{2}\hat{\theta}_{1}^{(2)}\left(1\right)\hat{f}_{1}(1)+2\frac{\mu_{1}}{1-\hat{\mu}_{1}}\hat{f}_{1}^(1)F_{1,1}(1)+f_{1,1}^{(2)}(1)+\left(\frac{\mu_{1}}{1-\hat{\mu}_{1}}\right)^{2}\hat{f}_{1}^{(1,1)}.
\end{eqnarray*}

%50

\item \begin{eqnarray*} \hat{f}_{2}\left(2,1\right)&=&\hat{r}_{1}\mu_{1}\tilde{\mu}_{2}+\mu_{1}\tilde{\mu}_{2}\hat{R}_{1}^{(2)}\left(1\right)+
\hat{r}_{1}\mu_{1}F_{1,2}(1)+\tilde{\mu}_{2}\hat{r}_{1}F_{1,1}(1)+
\frac{\mu_{1}\tilde{\mu}_{2}}{1-\hat{\mu}_{1}}\hat{f}_{1}(1)\\
&+&2\hat{r}_{1}\frac{\mu_{1}\tilde{\mu}_{2}}{1-\hat{\mu}_{1}}\hat{f}_{1}(1)+\mu_{1}\tilde{\mu}_{2}\hat{\theta}_{1}^{(2)}\left(1\right)\hat{f}_{1}(1)+
\frac{\mu_{1}}{1-\hat{\mu}_{1}}\hat{f}_{1}(1)F_{1,2}(1)+\frac{\tilde{\mu}_{2}}{1-\hat{\mu}_{1}}\hat{f}_{1}(1)F_{1,1}(1)\\
&+&f_{1}^{(2)}(1,2)+\mu_{1}\tilde{\mu}_{2}\left(\frac{1}{1-\hat{\mu}_{1}}\right)^{2}\hat{f}_{1}(1,1).
\end{eqnarray*}

%51

\item \begin{eqnarray*} \hat{f}_{2}\left(3,1\right)&=&\hat{r}_{1}\mu_{1}\hat{\mu}_{1}+\mu_{1}\hat{\mu}_{1}\hat{R}_{1}^{(2)}\left(1\right)+\hat{r}_{1}\hat{\mu}_{1}F_{1,1}(1)+\hat{r}_{1}\frac{\mu_{1}\hat{\mu}_{1}}{1-\hat{\mu}_{1}}\hat{F}_{1}(1).
\end{eqnarray*}

%52

\item \begin{eqnarray*} \hat{f}_{2}\left(4,1\right)&=&\hat{r}_{1}\mu_{1}\hat{\mu}_{2}+\mu_{1}\hat{\mu}_{2}\hat{R}_{1}^{(2)}\left(1\right)+\hat{r}_{1}\hat{\mu}_{2}F_{1,1}(1)+\frac{\mu_{1}\hat{\mu}_{2}}{1-\hat{\mu}_{1}}\hat{f}_{1}(1)+\hat{r}_{1}\frac{\mu_{1}\hat{\mu}_{2}}{1-\hat{\mu}_{1}}\hat{f}_{1}(1)\\
&+&\mu_{1}\hat{\mu}_{2}\hat{\theta}_{1}^{(2)}\left(1\right)\hat{f}_{1}(1)+\hat{r}_{1}\mu_{1}\left(\hat{f}_{1}(2)+\frac{\hat{\mu}_{2}}{1-\hat{\mu}_{1}}\hat{f}_{1}(1)\right)+F_{1,1}(1)\left(\hat{f}_{1}(2)+\frac{\hat{\mu}_{2}}{1-\hat{\mu}_{1}}\hat{f}_{1}(1)\right)\\
&+&\frac{\mu_{1}}{1-\hat{\mu}_{1}}\left(\hat{f}_{1}(1,2)+\frac{\hat{\mu}_{2}}{1-\hat{\mu}_{1}}\hat{f}_{1}(1,1)\right).
\end{eqnarray*}
%___________________________________________________________________________________________
%\subsubsection{Mixtas para $z_{2}$:}
%___________________________________________________________________________________________
%53

\item \begin{eqnarray*} \hat{f}_{2}\left(1,2\right)&=&\hat{r}_{1}\mu_{1}\tilde{\mu}_{2}+\mu_{1}\tilde{\mu}_{2}\hat{R}_{1}^{(2)}\left(1\right)+\hat{r}_{1}\mu_{1}F_{1,2}(1)+\hat{r}_{1}\tilde{\mu}_{2}F_{1,1}(1)+\frac{\mu_{1}\tilde{\mu}_{2}}{1-\hat{\mu}_{1}}\hat{f}_{1}(1)\\
&+&2\hat{r}_{1}\frac{\mu_{1}\tilde{\mu}_{2}}{1-\hat{\mu}_{1}}\hat{f}_{1}(1)+\mu_{1}\tilde{\mu}_{2}\hat{\theta}_{1}^{(2)}\left(1\right)\hat{f}_{1}(1)+\frac{\mu_{1}}{1-\hat{\mu}_{1}}\hat{f}_{1}(1)F_{1,2}(1)\\
&+&\frac{\tilde{\mu}_{2}}{1-\hat{\mu}_{1}}\hat{f}_{1}(1)F_{1,1}(1)+f_{1}^{(2)}(1,2)+\mu_{1}\tilde{\mu}_{2}\left(\frac{1}{1-\hat{\mu}_{1}}\right)^{2}\hat{f}_{1}(1,1).
\end{eqnarray*}

%54

\item \begin{eqnarray*} \hat{f}_{2}\left(2,2\right)&=&\hat{r}_{1}\tilde{P}_{2}^{(2)}\left(1\right)+\tilde{\mu}_{2}^{2}\hat{R}_{1}^{(2)}\left(1\right)+2\hat{r}_{1}\tilde{\mu}_{2}F_{1,2}(1)+ f_{1,2}^{(2)}(1)+2\hat{r}_{1}\frac{\tilde{\mu}_{2}^{2}}{1-\hat{\mu}_{1}}\hat{f}_{1}(1)\\
&+&\frac{1}{1-\hat{\mu}_{1}}\tilde{P}_{2}^{(2)}\left(1\right)\hat{f}_{1}(1)+\tilde{\mu}_{2}^{2}\hat{\theta}_{1}^{(2)}\left(1\right)\hat{f}_{1}(1)+2\frac{\tilde{\mu}_{2}}{1-\hat{\mu}_{1}}F_{1,2}(1)\hat{f}_{1}(1)+\left(\frac{\tilde{\mu}_{2}}{1-\hat{\mu}_{1}}\right)^{2}\hat{f}_{1}(1,1).
\end{eqnarray*}

%55

\item \begin{eqnarray*} \hat{f}_{2}\left(3,2\right)&=&\hat{r}_{1}\hat{\mu}_{1}\tilde{\mu}_{2}+\hat{\mu}_{1}\tilde{\mu}_{2}\hat{R}_{1}^{(2)}\left(1\right)+
\hat{r}_{1}\hat{\mu}_{1}F_{1,2}(1)+\hat{r}_{1}\frac{\hat{\mu}_{1}\tilde{\mu}_{2}}{1-\hat{\mu}_{1}}\hat{f}_{1}(1).
\end{eqnarray*}

%56

\item \begin{eqnarray*} \hat{f}_{2}\left(4,2\right)&=&\hat{r}_{1}\tilde{\mu}_{2}\hat{\mu}_{2}+\hat{\mu}_{2}\tilde{\mu}_{2}\hat{R}_{1}^{(2)}\left(1\right)+\hat{\mu}_{2}\hat{R}_{1}^{(2)}\left(1\right)F_{1,2}(1)+\frac{\hat{\mu}_{2}\tilde{\mu}_{2}}{1-\hat{\mu}_{1}}\hat{f}_{1}(1)\\
&+&\hat{r}_{1}\frac{\hat{\mu}_{2}\tilde{\mu}_{2}}{1-\hat{\mu}_{1}}\hat{f}_{1}(1)+\hat{\mu}_{2}\tilde{\mu}_{2}\hat{\theta}_{1}^{(2)}\left(1\right)\hat{f}_{1}(1)+\hat{r}_{1}\tilde{\mu}_{2}\left(\hat{f}_{1}(2)+\frac{\hat{\mu}_{2}}{1-\hat{\mu}_{1}}\hat{f}_{1}(1)\right)\\
&+&F_{1,2}(1)\left(\hat{f}_{1}(2)+\frac{\hat{\mu}_{2}}{1-\hat{\mu}_{1}}\hat{f}_{1}(1)\right)+\frac{\tilde{\mu}_{2}}{1-\hat{\mu}_{1}}\left(\hat{f}_{1}(1,2)+\frac{\hat{\mu}_{2}}{1-\hat{\mu}_{1}}\hat{f}_{1}(1,1)\right).
\end{eqnarray*}
%___________________________________________________________________________________________
%\subsubsection{Mixtas para $w_{1}$:}
%___________________________________________________________________________________________

%57


\item \begin{eqnarray*} \hat{f}_{2}\left(1,3\right)&=&\hat{r}_{1}\mu_{1}\hat{\mu}_{1}+\mu_{1}\hat{\mu}_{1}\hat{R}_{1}^{(2)}\left(1\right)+\hat{r}_{1}\hat{\mu}_{1}F_{1,1}(1)+\hat{r}_{1}\frac{\mu_{1}\hat{\mu}_{1}}{1-\hat{\mu}_{1}}\hat{f}_{1}(1).
\end{eqnarray*}

%58

\item \begin{eqnarray*} \hat{f}_{2}\left(2,3\right)&=&\hat{r}_{1}\tilde{\mu}_{2}\hat{\mu}_{1}+\tilde{\mu}_{2}\hat{\mu}_{1}\hat{R}_{1}^{(2)}\left(1\right)+\hat{r}_{1}\hat{\mu}_{1}F_{1,2}(1)+\hat{r}_{1}\frac{\tilde{\mu}_{2}\hat{\mu}_{1}}{1-\hat{\mu}_{1}}\hat{f}_{1}(1).
\end{eqnarray*}

%59

\item \begin{eqnarray*} \hat{f}_{2}\left(3,3\right)&=&\hat{r}_{1}\hat{P}_{1}^{(2)}\left(1\right)+\hat{\mu}_{1}^{2}\hat{R}_{1}^{(2)}\left(1\right).
\end{eqnarray*}

%60

\item \begin{eqnarray*} \hat{f}_{2}\left(4,3\right)&=&\hat{r}_{1}\hat{\mu}_{2}\hat{\mu}_{1}+\hat{\mu}_{2}\hat{\mu}_{1}\hat{R}_{1}^{(2)}\left(1\right)+\hat{r}_{1}\hat{\mu}_{1}\left(\hat{f}_{1}(2)+\frac{\hat{\mu}_{2}}{1-\hat{\mu}_{1}}\hat{f}_{1}(1)\right).
\end{eqnarray*}
%___________________________________________________________________________________________
%\subsubsection{Mixtas para $w_{1}$:}
%___________________________________________________________________________________________
%61

\item \begin{eqnarray*} \hat{f}_{2}\left(1,4\right)&=&\hat{r}_{1}\mu_{1}\hat{\mu}_{2}+\mu_{1}\hat{\mu}_{2}\hat{R}_{1}^{(2)}\left(1\right)+\hat{r}_{1}\hat{\mu}_{2}F_{1,1}(1)+\hat{r}_{1}\frac{\mu_{1}\hat{\mu}_{2}}{1-\hat{\mu}_{1}}\hat{f}_{1}(1)+\hat{r}_{1}\mu_{1}\left(\hat{f}_{1}(2)+\frac{\hat{\mu}_{2}}{1-\hat{\mu}_{1}}\hat{f}_{1}(1)\right)\\
&+&F_{1,1}(1)\left(\hat{f}_{1}(2)+\frac{\hat{\mu}_{2}}{1-\hat{\mu}_{1}}\hat{f}_{1}(1)\right)+\frac{\mu_{1}\hat{\mu}_{2}}{1-\hat{\mu}_{1}}\hat{f}_{1}(1)+\mu_{1}\hat{\mu}_{2}\hat{\theta}_{1}^{(2)}\left(1\right)\hat{f}_{1}(1)\\
&+&\frac{\mu_{1}}{1-\hat{\mu}_{1}}\hat{f}_{1}(1,2)+\mu_{1}\hat{\mu}_{2}\left(\frac{1}{1-\hat{\mu}_{1}}\right)^{2}\hat{f}_{1}(1,1).
\end{eqnarray*}

%62

\item \begin{eqnarray*} \hat{f}_{2}\left(2,4\right)&=&\hat{r}_{1}\tilde{\mu}_{2}\hat{\mu}_{2}+\tilde{\mu}_{2}\hat{\mu}_{2}\hat{R}_{1}^{(2)}\left(1\right)+\hat{r}_{1}\hat{\mu}_{2}F_{1,2}(1)+\hat{r}_{1}\frac{\tilde{\mu}_{2}\hat{\mu}_{2}}{1-\hat{\mu}_{1}}\hat{f}_{1}(1)\\
&+&\hat{r}_{1}\tilde{\mu}_{2}\left(\hat{f}_{1}(2)+\frac{\hat{\mu}_{2}}{1-\hat{\mu}_{1}}\hat{f}_{1}(1)\right)+F_{1,2}(1)\left(\hat{f}_{1}(2)+\frac{\hat{\mu}_{2}}{1-\hat{\mu}_{1}}\hat{F}_{1}^{(1,0)}\right)+\frac{\tilde{\mu}_{2}\hat{\mu}_{2}}{1-\hat{\mu}_{1}}\hat{f}_{1}(1)\\
&+&\tilde{\mu}_{2}\hat{\mu}_{2}\hat{\theta}_{1}^{(2)}\left(1\right)\hat{f}_{1}(1)+\frac{\tilde{\mu}_{2}}{1-\hat{\mu}_{1}}\hat{f}_{1}(1,2)+\tilde{\mu}_{2}\hat{\mu}_{2}\left(\frac{1}{1-\hat{\mu}_{1}}\right)^{2}\hat{f}_{1}(1,1).
\end{eqnarray*}

%63

\item \begin{eqnarray*} \hat{f}_{2}\left(3,4\right)&=&\hat{r}_{1}\hat{\mu}_{2}\hat{\mu}_{1}+\hat{\mu}_{2}\hat{\mu}_{1}\hat{R}_{1}^{(2)}\left(1\right)+\hat{r}_{1}\hat{\mu}_{1}\left(\hat{f}_{1}(2)+\frac{\hat{\mu}_{2}}{1-\hat{\mu}_{1}}\hat{f}_{1}(1)\right).
\end{eqnarray*}

%64

\item \begin{eqnarray*} \hat{f}_{2}\left(4,4\right)&=&\hat{r}_{1}\hat{P}_{2}^{(2)}\left(1\right)+\hat{\mu}_{2}^{2}\hat{R}_{1}^{(2)}\left(1\right)+
2\hat{r}_{1}\hat{\mu}_{2}\left(\hat{f}_{1}(2)+\frac{\hat{\mu}_{2}}{1-\hat{\mu}_{1}}\hat{f}_{1}(1)\right)+\hat{f}_{1}(2,2)\\
&+&\frac{1}{1-\hat{\mu}_{1}}\hat{P}_{2}^{(2)}\left(1\right)\hat{f}_{1}(1)+\hat{\mu}_{2}^{2}\hat{\theta}_{1}^{(2)}\left(1\right)\hat{f}_{1}(1)+\frac{\hat{\mu}_{2}}{1-\hat{\mu}_{1}}\hat{f}_{1}(1,2)\\
&+&\frac{\hat{\mu}_{2}}{1-\hat{\mu}_{1}}\left(\hat{f}_{1}(1,2)+\frac{\hat{\mu}_{2}}{1-\hat{\mu}_{1}}\hat{f}_{1}(1,1)\right).
\end{eqnarray*}
%_________________________________________________________________________________________________________
%
%_________________________________________________________________________________________________________

\end{enumerate}
%___________________________________________________________________________________________
\section{Tiempos de Ciclo e Intervisita}
%___________________________________________________________________________________________


\begin{Def}
Sea $L_{i}^{*}$el n\'umero de usuarios en la cola $Q_{i}$ cuando es visitada por el servidor para dar servicio, entonces

\begin{eqnarray}
\esp\left[L_{i}^{*}\right]&=&f_{i}\left(i\right)\\
Var\left[L_{i}^{*}\right]&=&f_{i}\left(i,i\right)+\esp\left[L_{i}^{*}\right]-\esp\left[L_{i}^{*}\right]^{2}.
\end{eqnarray}

\end{Def}

\begin{Def}
El tiempo de Ciclo $C_{i}$ es e periodo de tiempo que comienza cuando la cola $i$ es visitada por primera vez en un ciclo, y termina cuando es visitado nuevamente en el pr\'oximo ciclo. La duraci\'on del mismo est\'a dada por $\tau_{i}\left(m+1\right)-\tau_{i}\left(m\right)$, o equivalentemente $\overline{\tau}_{i}\left(m+1\right)-\overline{\tau}_{i}\left(m\right)$ bajo condiciones de estabilidad.
\end{Def}

\begin{Def}
El tiempo de intervisita $I_{i}$ es el periodo de tiempo que comienza cuando se ha completado el servicio en un ciclo y termina cuando es visitada nuevamente en el pr\'oximo ciclo. Su  duraci\'on del mismo est\'a dada por $\tau_{i}\left(m+1\right)-\overline{\tau}_{i}\left(m\right)$.
\end{Def}


Recordemos las siguientes expresiones:

\begin{eqnarray*}
S_{i}\left(z\right)&=&\esp\left[z^{\overline{\tau}_{i}\left(m\right)-\tau_{i}\left(m\right)}\right]=F_{i}\left(\theta\left(z\right)\right),\\
F\left(z\right)&=&\esp\left[z^{L_{0}}\right],\\
P\left(z\right)&=&\esp\left[z^{X_{n}}\right],\\
F_{i}\left(z\right)&=&\esp\left[z^{L_{i}\left(\tau_{i}\left(m\right)\right)}\right],
\theta_{i}\left(z\right)-zP_{i}
\end{eqnarray*}

entonces

\begin{eqnarray*}
\esp\left[S_{i}\right]&=&\frac{\esp\left[L_{i}^{*}\right]}{1-\mu_{i}}=\frac{f_{i}\left(i\right)}{1-\mu_{i}},\\
Var\left[S_{i}\right]&=&\frac{Var\left[L_{i}^{*}\right]}{\left(1-\mu_{i}\right)^{2}}+\frac{\sigma^{2}\esp\left[L_{i}^{*}\right]}{\left(1-\mu_{i}\right)^{3}}
\end{eqnarray*}

donde recordemos que

\begin{eqnarray*}
Var\left[L_{i}^{*}\right]&=&f_{i}\left(i,i\right)+f_{i}\left(i\right)-f_{i}\left(i\right)^{2}.
\end{eqnarray*}

La duraci\'on del tiempo de intervisita es $\tau_{i}\left(m+1\right)-\overline{\tau}\left(m\right)$. Dado que el n\'umero de usuarios presentes en $Q_{i}$ al tiempo $t=\tau_{i}\left(m+1\right)$ es igual al n\'umero de arribos durante el intervalo de tiempo $\left[\overline{\tau}\left(m\right),\tau_{i}\left(m+1\right)\right]$ se tiene que


\begin{eqnarray*}
\esp\left[z_{i}^{L_{i}\left(\tau_{i}\left(m+1\right)\right)}\right]=\esp\left[\left\{P_{i}\left(z_{i}\right)\right\}^{\tau_{i}\left(m+1\right)-\overline{\tau}\left(m\right)}\right]
\end{eqnarray*}

entonces, si \begin{eqnarray*}I_{i}\left(z\right)&=&\esp\left[z^{\tau_{i}\left(m+1\right)-\overline{\tau}\left(m\right)}\right]\end{eqnarray*} se tienen que

\begin{eqnarray*}
F_{i}\left(z\right)=I_{i}\left[P_{i}\left(z\right)\right]
\end{eqnarray*}
para $i=1,2$, por tanto



\begin{eqnarray*}
\esp\left[L_{i}^{*}\right]&=&\mu_{i}\esp\left[I_{i}\right]\\
Var\left[L_{i}^{*}\right]&=&\mu_{i}^{2}Var\left[I_{i}\right]+\sigma^{2}\esp\left[I_{i}\right]
\end{eqnarray*}
para $i=1,2$, por tanto


\begin{eqnarray*}
\esp\left[I_{i}\right]&=&\frac{f_{i}\left(i\right)}{\mu_{i}},
\end{eqnarray*}
adem\'as

\begin{eqnarray*}
Var\left[I_{i}\right]&=&\frac{Var\left[L_{i}^{*}\right]}{\mu_{i}^{2}}-\frac{\sigma_{i}^{2}}{\mu_{i}^{2}}f_{i}\left(i\right).
\end{eqnarray*}


Si  $C_{i}\left(z\right)=\esp\left[z^{\overline{\tau}\left(m+1\right)-\overline{\tau}_{i}\left(m\right)}\right]$el tiempo de duraci\'on del ciclo, entonces, por lo hasta ahora establecido, se tiene que

\begin{eqnarray*}
C_{i}\left(z\right)=I_{i}\left[\theta_{i}\left(z\right)\right],
\end{eqnarray*}
entonces

\begin{eqnarray*}
\esp\left[C_{i}\right]&=&\esp\left[I_{i}\right]\esp\left[\theta_{i}\left(z\right)\right]=\frac{\esp\left[L_{i}^{*}\right]}{\mu_{i}}\frac{1}{1-\mu_{i}}=\frac{f_{i}\left(i\right)}{\mu_{i}\left(1-\mu_{i}\right)}\\
Var\left[C_{i}\right]&=&\frac{Var\left[L_{i}^{*}\right]}{\mu_{i}^{2}\left(1-\mu_{i}\right)^{2}}.
\end{eqnarray*}

Por tanto se tienen las siguientes igualdades


\begin{eqnarray*}
\esp\left[S_{i}\right]&=&\mu_{i}\esp\left[C_{i}\right],\\
\esp\left[I_{i}\right]&=&\left(1-\mu_{i}\right)\esp\left[C_{i}\right]\\
\end{eqnarray*}

Def\'inanse los puntos de regenaraci\'on  en el proceso $\left[L_{1}\left(t\right),L_{2}\left(t\right),\ldots,L_{N}\left(t\right)\right]$. Los puntos cuando la cola $i$ es visitada y todos los $L_{j}\left(\tau_{i}\left(m\right)\right)=0$ para $i=1,2$  son puntos de regeneraci\'on. Se llama ciclo regenerativo al intervalo entre dos puntos regenerativos sucesivos.

Sea $M_{i}$  el n\'umero de ciclos de visita en un ciclo regenerativo, y sea $C_{i}^{(m)}$, para $m=1,2,\ldots,M_{i}$ la duraci\'on del $m$-\'esimo ciclo de visita en un ciclo regenerativo. Se define el ciclo del tiempo de visita promedio $\esp\left[C_{i}\right]$ como

\begin{eqnarray*}
\esp\left[C_{i}\right]&=&\frac{\esp\left[\sum_{m=1}^{M_{i}}C_{i}^{(m)}\right]}{\esp\left[M_{i}\right]}
\end{eqnarray*}


En Stid72 y Heym82 se muestra que una condici\'on suficiente para que el proceso regenerativo
estacionario sea un procesoo estacionario es que el valor esperado del tiempo del ciclo regenerativo sea finito:

\begin{eqnarray*}
\esp\left[\sum_{m=1}^{M_{i}}C_{i}^{(m)}\right]<\infty.
\end{eqnarray*}

como cada $C_{i}^{(m)}$ contiene intervalos de r\'eplica positivos, se tiene que $\esp\left[M_{i}\right]<\infty$, adem\'as, como $M_{i}>0$, se tiene que la condici\'on anterior es equivalente a tener que

\begin{eqnarray*}
\esp\left[C_{i}\right]<\infty,
\end{eqnarray*}
por lo tanto una condici\'on suficiente para la existencia del proceso regenerativo est\'a dada por

\begin{eqnarray*}
\sum_{k=1}^{N}\mu_{k}<1.
\end{eqnarray*}

Sea la funci\'on generadora de momentos para $L_{i}$, el n\'umero de usuarios en la cola $Q_{i}\left(z\right)$ en cualquier momento, est\'a dada por el tiempo promedio de $z^{L_{i}\left(t\right)}$ sobre el ciclo regenerativo definido anteriormente:

\begin{eqnarray*}
Q_{i}\left(z\right)&=&\esp\left[z^{L_{i}\left(t\right)}\right]=\frac{\esp\left[\sum_{m=1}^{M_{i}}\sum_{t=\tau_{i}\left(m\right)}^{\tau_{i}\left(m+1\right)-1}z^{L_{i}\left(t\right)}\right]}{\esp\left[\sum_{m=1}^{M_{i}}\tau_{i}\left(m+1\right)-\tau_{i}\left(m\right)\right]}
\end{eqnarray*}

$M_{i}$ es un tiempo de paro en el proceso regenerativo con $\esp\left[M_{i}\right]<\infty$, se sigue del lema de Wald que:


\begin{eqnarray*}
\esp\left[\sum_{m=1}^{M_{i}}\sum_{t=\tau_{i}\left(m\right)}^{\tau_{i}\left(m+1\right)-1}z^{L_{i}\left(t\right)}\right]&=&\esp\left[M_{i}\right]\esp\left[\sum_{t=\tau_{i}\left(m\right)}^{\tau_{i}\left(m+1\right)-1}z^{L_{i}\left(t\right)}\right]\\
\esp\left[\sum_{m=1}^{M_{i}}\tau_{i}\left(m+1\right)-\tau_{i}\left(m\right)\right]&=&\esp\left[M_{i}\right]\esp\left[\tau_{i}\left(m+1\right)-\tau_{i}\left(m\right)\right]
\end{eqnarray*}

por tanto se tiene que


\begin{eqnarray*}
Q_{i}\left(z\right)&=&\frac{\esp\left[\sum_{t=\tau_{i}\left(m\right)}^{\tau_{i}\left(m+1\right)-1}z^{L_{i}\left(t\right)}\right]}{\esp\left[\tau_{i}\left(m+1\right)-\tau_{i}\left(m\right)\right]}
\end{eqnarray*}

observar que el denominador es simplemente la duraci\'on promedio del tiempo del ciclo.


Se puede demostrar (ver Hideaki Takagi 1986) que

\begin{eqnarray*}
\esp\left[\sum_{t=\tau_{i}\left(m\right)}^{\tau_{i}\left(m+1\right)-1}z^{L_{i}\left(t\right)}\right]=z\frac{F_{i}\left(z\right)-1}{z-P_{i}\left(z\right)}
\end{eqnarray*}

Durante el tiempo de intervisita para la cola $i$, $L_{i}\left(t\right)$ solamente se incrementa de manera que el incremento por intervalo de tiempo est\'a dado por la funci\'on generadora de probabilidades de $P_{i}\left(z\right)$, por tanto la suma sobre el tiempo de intervisita puede evaluarse como:

\begin{eqnarray*}
\esp\left[\sum_{t=\tau_{i}\left(m\right)}^{\tau_{i}\left(m+1\right)-1}z^{L_{i}\left(t\right)}\right]&=&\esp\left[\sum_{t=\tau_{i}\left(m\right)}^{\tau_{i}\left(m+1\right)-1}\left\{P_{i}\left(z\right)\right\}^{t-\overline{\tau}_{i}\left(m\right)}\right]=\frac{1-\esp\left[\left\{P_{i}\left(z\right)\right\}^{\tau_{i}\left(m+1\right)-\overline{\tau}_{i}\left(m\right)}\right]}{1-P_{i}\left(z\right)}\\
&=&\frac{1-I_{i}\left[P_{i}\left(z\right)\right]}{1-P_{i}\left(z\right)}
\end{eqnarray*}
por tanto

\begin{eqnarray*}
\esp\left[\sum_{t=\tau_{i}\left(m\right)}^{\tau_{i}\left(m+1\right)-1}z^{L_{i}\left(t\right)}\right]&=&\frac{1-F_{i}\left(z\right)}{1-P_{i}\left(z\right)}
\end{eqnarray*}

Haciendo uso de lo hasta ahora desarrollado se tiene que

\begin{eqnarray*}
Q_{i}\left(z\right)&=&\frac{1}{\esp\left[C_{i}\right]}\cdot\frac{1-F_{i}\left(z\right)}{P_{i}\left(z\right)-z}\cdot\frac{\left(1-z\right)P_{i}\left(z\right)}{1-P_{i}\left(z\right)}\\
&=&\frac{\mu_{i}\left(1-\mu_{i}\right)}{f_{i}\left(i\right)}\cdot\frac{1-F_{i}\left(z\right)}{P_{i}\left(z\right)-z}\cdot\frac{\left(1-z\right)P_{i}\left(z\right)}{1-P_{i}\left(z\right)}
\end{eqnarray*}

derivando con respecto a $z$



\begin{eqnarray*}
\frac{d Q_{i}\left(z\right)}{d z}&=&\frac{\left(1-F_{i}\left(z\right)\right)P_{i}\left(z\right)}{\esp\left[C_{i}\right]\left(1-P_{i}\left(z\right)\right)\left(P_{i}\left(z\right)-z\right)}\\
&-&\frac{\left(1-z\right)P_{i}\left(z\right)F_{i}^{'}\left(z\right)}{\esp\left[C_{i}\right]\left(1-P_{i}\left(z\right)\right)\left(P_{i}\left(z\right)-z\right)}\\
&-&\frac{\left(1-z\right)\left(1-F_{i}\left(z\right)\right)P_{i}\left(z\right)\left(P_{i}^{'}\left(z\right)-1\right)}{\esp\left[C_{i}\right]\left(1-P_{i}\left(z\right)\right)\left(P_{i}\left(z\right)-z\right)^{2}}\\
&+&\frac{\left(1-z\right)\left(1-F_{i}\left(z\right)\right)P_{i}^{'}\left(z\right)}{\esp\left[C_{i}\right]\left(1-P_{i}\left(z\right)\right)\left(P_{i}\left(z\right)-z\right)}\\
&+&\frac{\left(1-z\right)\left(1-F_{i}\left(z\right)\right)P_{i}\left(z\right)P_{i}^{'}\left(z\right)}{\esp\left[C_{i}\right]\left(1-P_{i}\left(z\right)\right)^{2}\left(P_{i}\left(z\right)-z\right)}
\end{eqnarray*}

Calculando el l\'imite cuando $z\rightarrow1^{+}$:
\begin{eqnarray}
Q_{i}^{(1)}\left(z\right)=\lim_{z\rightarrow1^{+}}\frac{d Q_{i}\left(z\right)}{dz}&=&\lim_{z\rightarrow1}\frac{\left(1-F_{i}\left(z\right)\right)P_{i}\left(z\right)}{\esp\left[C_{i}\right]\left(1-P_{i}\left(z\right)\right)\left(P_{i}\left(z\right)-z\right)}\\
&-&\lim_{z\rightarrow1^{+}}\frac{\left(1-z\right)P_{i}\left(z\right)F_{i}^{'}\left(z\right)}{\esp\left[C_{i}\right]\left(1-P_{i}\left(z\right)\right)\left(P_{i}\left(z\right)-z\right)}\\
&-&\lim_{z\rightarrow1^{+}}\frac{\left(1-z\right)\left(1-F_{i}\left(z\right)\right)P_{i}\left(z\right)\left(P_{i}^{'}\left(z\right)-1\right)}{\esp\left[C_{i}\right]\left(1-P_{i}\left(z\right)\right)\left(P_{i}\left(z\right)-z\right)^{2}}\\
&+&\lim_{z\rightarrow1^{+}}\frac{\left(1-z\right)\left(1-F_{i}\left(z\right)\right)P_{i}^{'}\left(z\right)}{\esp\left[C_{i}\right]\left(1-P_{i}\left(z\right)\right)\left(P_{i}\left(z\right)-z\right)}\\
&+&\lim_{z\rightarrow1^{+}}\frac{\left(1-z\right)\left(1-F_{i}\left(z\right)\right)P_{i}\left(z\right)P_{i}^{'}\left(z\right)}{\esp\left[C_{i}\right]\left(1-P_{i}\left(z\right)\right)^{2}\left(P_{i}\left(z\right)-z\right)}
\end{eqnarray}

Entonces:
%______________________________________________________

\begin{eqnarray*}
\lim_{z\rightarrow1^{+}}\frac{\left(1-F_{i}\left(z\right)\right)P_{i}\left(z\right)}{\left(1-P_{i}\left(z\right)\right)\left(P_{i}\left(z\right)-z\right)}&=&\lim_{z\rightarrow1^{+}}\frac{\frac{d}{dz}\left[\left(1-F_{i}\left(z\right)\right)P_{i}\left(z\right)\right]}{\frac{d}{dz}\left[\left(1-P_{i}\left(z\right)\right)\left(-z+P_{i}\left(z\right)\right)\right]}\\
&=&\lim_{z\rightarrow1^{+}}\frac{-P_{i}\left(z\right)F_{i}^{'}\left(z\right)+\left(1-F_{i}\left(z\right)\right)P_{i}^{'}\left(z\right)}{\left(1-P_{i}\left(z\right)\right)\left(-1+P_{i}^{'}\left(z\right)\right)-\left(-z+P_{i}\left(z\right)\right)P_{i}^{'}\left(z\right)}
\end{eqnarray*}


%______________________________________________________


\begin{eqnarray*}
\lim_{z\rightarrow1^{+}}\frac{\left(1-z\right)P_{i}\left(z\right)F_{i}^{'}\left(z\right)}{\left(1-P_{i}\left(z\right)\right)\left(P_{i}\left(z\right)-z\right)}&=&\lim_{z\rightarrow1^{+}}\frac{\frac{d}{dz}\left[\left(1-z\right)P_{i}\left(z\right)F_{i}^{'}\left(z\right)\right]}{\frac{d}{dz}\left[\left(1-P_{i}\left(z\right)\right)\left(P_{i}\left(z\right)-z\right)\right]}\\
&=&\lim_{z\rightarrow1^{+}}\frac{-P_{i}\left(z\right) F_{i}^{'}\left(z\right)+(1-z) F_{i}^{'}\left(z\right) P_{i}^{'}\left(z\right)+(1-z) P_{i}\left(z\right)F_{i}^{''}\left(z\right)}{\left(1-P_{i}\left(z\right)\right)\left(-1+P_{i}^{'}\left(z\right)\right)-\left(-z+P_{i}\left(z\right)\right)P_{i}^{'}\left(z\right)}
\end{eqnarray*}


%______________________________________________________

\begin{eqnarray*}
&&\lim_{z\rightarrow1^{+}}\frac{\left(1-z\right)\left(1-F_{i}\left(z\right)\right)P_{i}\left(z\right)\left(P_{i}^{'}\left(z\right)-1\right)}{\left(1-P_{i}\left(z\right)\right)\left(P_{i}\left(z\right)-z\right)^{2}}=\lim_{z\rightarrow1^{+}}\frac{\frac{d}{dz}\left[\left(1-z\right)\left(1-F_{i}\left(z\right)\right)P_{i}\left(z\right)\left(P_{i}^{'}\left(z\right)-1\right)\right]}{\frac{d}{dz}\left[\left(1-P_{i}\left(z\right)\right)\left(P_{i}\left(z\right)-z\right)^{2}\right]}\\
&=&\lim_{z\rightarrow1^{+}}\frac{-\left(1-F_{i}\left(z\right)\right) P_{i}\left(z\right)\left(-1+P_{i}^{'}\left(z\right)\right)-(1-z) P_{i}\left(z\right)F_{i}^{'}\left(z\right)\left(-1+P_{i}^{'}\left(z\right)\right)}{2\left(1-P_{i}\left(z\right)\right)\left(-z+P_{i}\left(z\right)\right) \left(-1+P_{i}^{'}\left(z\right)\right)-\left(-z+P_{i}\left(z\right)\right)^2 P_{i}^{'}\left(z\right)}\\
&+&\lim_{z\rightarrow1^{+}}\frac{+(1-z) \left(1-F_{i}\left(z\right)\right) \left(-1+P_{i}^{'}\left(z\right)\right) P_{i}^{'}\left(z\right)}{{2\left(1-P_{i}\left(z\right)\right)\left(-z+P_{i}\left(z\right)\right) \left(-1+P_{i}^{'}\left(z\right)\right)-\left(-z+P_{i}\left(z\right)\right)^2 P_{i}^{'}\left(z\right)}}\\
&+&\lim_{z\rightarrow1^{+}}\frac{+(1-z) \left(1-F_{i}\left(z\right)\right) P_{i}\left(z\right)P_{i}^{''}\left(z\right)}{{2\left(1-P_{i}\left(z\right)\right)\left(-z+P_{i}\left(z\right)\right) \left(-1+P_{i}^{'}\left(z\right)\right)-\left(-z+P_{i}\left(z\right)\right)^2 P_{i}^{'}\left(z\right)}}
\end{eqnarray*}











%______________________________________________________
\begin{eqnarray*}
&&\lim_{z\rightarrow1^{+}}\frac{\left(1-z\right)\left(1-F_{i}\left(z\right)\right)P_{i}^{'}\left(z\right)}{\left(1-P_{i}\left(z\right)\right)\left(P_{i}\left(z\right)-z\right)}=\lim_{z\rightarrow1^{+}}\frac{\frac{d}{dz}\left[\left(1-z\right)\left(1-F_{i}\left(z\right)\right)P_{i}^{'}\left(z\right)\right]}{\frac{d}{dz}\left[\left(1-P_{i}\left(z\right)\right)\left(P_{i}\left(z\right)-z\right)\right]}\\
&=&\lim_{z\rightarrow1^{+}}\frac{-\left(1-F_{i}\left(z\right)\right) P_{i}^{'}\left(z\right)-(1-z) F_{i}^{'}\left(z\right) P_{i}^{'}\left(z\right)+(1-z) \left(1-F_{i}\left(z\right)\right) P_{i}^{''}\left(z\right)}{\left(1-P_{i}\left(z\right)\right) \left(-1+P_{i}^{'}\left(z\right)\right)-\left(-z+P_{i}\left(z\right)\right) P_{i}^{'}\left(z\right)}\frac{}{}
\end{eqnarray*}

%______________________________________________________
\begin{eqnarray*}
&&\lim_{z\rightarrow1^{+}}\frac{\left(1-z\right)\left(1-F_{i}\left(z\right)\right)P_{i}\left(z\right)P_{i}^{'}\left(z\right)}{\left(1-P_{i}\left(z\right)\right)^{2}\left(P_{i}\left(z\right)-z\right)}=\lim_{z\rightarrow1^{+}}\frac{\frac{d}{dz}\left[\left(1-z\right)\left(1-F_{i}\left(z\right)\right)P_{i}\left(z\right)P_{i}^{'}\left(z\right)\right]}{\frac{d}{dz}\left[\left(1-P_{i}\left(z\right)\right)^{2}\left(P_{i}\left(z\right)-z\right)\right]}\\
&=&\lim_{z\rightarrow1^{+}}\frac{-\left(1-F_{i}\left(z\right)\right) P_{i}\left(z\right) P_{i}^{'}\left(z\right)-(1-z) P_{i}\left(z\right) F_{i}^{'}\left(z\right)P_i'[z]}{\left(1-P_{i}\left(z\right)\right)^2 \left(-1+P_{i}^{'}\left(z\right)\right)-2 \left(1-P_{i}\left(z\right)\right) \left(-z+P_{i}\left(z\right)\right) P_{i}^{'}\left(z\right)}\\
&+&\lim_{z\rightarrow1^{+}}\frac{(1-z) \left(1-F_{i}\left(z\right)\right) P_{i}^{'}\left(z\right)^2+(1-z) \left(1-F_{i}\left(z\right)\right) P_{i}\left(z\right) P_{i}^{''}\left(z\right)}{\left(1-P_{i}\left(z\right)\right)^2 \left(-1+P_{i}^{'}\left(z\right)\right)-2 \left(1-P_{i}\left(z\right)\right) \left(-z+P_{i}\left(z\right)\right) P_{i}^{'}\left(z\right)}\\
\end{eqnarray*}

%___________________________________________________________________________________________
\subsection{Longitudes de la Cola en cualquier tiempo}
%___________________________________________________________________________________________

Sea
$V_{i}\left(z\right)=\frac{1}{\esp\left[C_{i}\right]}\frac{I_{i}\left(z\right)-1}{z-P_{i}\left(z\right)}$

%{\esp\lef[I_{i}\right]}\frac{1-\mu_{i}}{z-P_{i}\left(z\right)}

\begin{eqnarray*}
\frac{\partial V_{i}\left(z\right)}{\partial z}&=&\frac{1}{\esp\left[C_{i}\right]}\left[\frac{I_{i}{'}\left(z\right)\left(z-P_{i}\left(z\right)\right)}{z-P_{i}\left(z\right)}-\frac{\left(I_{i}\left(z\right)-1\right)\left(1-P_{i}{'}\left(z\right)\right)}{\left(z-P_{i}\left(z\right)\right)^{2}}\right]
\end{eqnarray*}


La FGP para el tiempo de espera para cualquier usuario en la cola est\'a dada por:
\[U_{i}\left(z\right)=\frac{1}{\esp\left[C_{i}\right]}\cdot\frac{1-P_{i}\left(z\right)}{z-P_{i}\left(z\right)}\cdot\frac{I_{i}\left(z\right)-1}{1-z}\]

entonces


\begin{eqnarray*}
\frac{d}{dz}V_{i}\left(z\right)&=&\frac{1}{\esp\left[C_{i}\right]}\left\{\frac{d}{dz}\left(\frac{1-P_{i}\left(z\right)}{z-P_{i}\left(z\right)}\right)\frac{I_{i}\left(z\right)-1}{1-z}+\frac{1-P_{i}\left(z\right)}{z-P_{i}\left(z\right)}\frac{d}{dz}\left(\frac{I_{i}\left(z\right)-1}{1-z}\right)\right\}\\
&=&\frac{1}{\esp\left[C_{i}\right]}\left\{\frac{-P_{i}\left(z\right)\left(z-P_{i}\left(z\right)\right)-\left(1-P_{i}\left(z\right)\right)\left(1-P_{i}^{'}\left(z\right)\right)}{\left(z-P_{i}\left(z\right)\right)^{2}}\cdot\frac{I_{i}\left(z\right)-1}{1-z}\right\}\\
&+&\frac{1}{\esp\left[C_{i}\right]}\left\{\frac{1-P_{i}\left(z\right)}{z-P_{i}\left(z\right)}\cdot\frac{I_{i}^{'}\left(z\right)\left(1-z\right)+\left(I_{i}\left(z\right)-1\right)}{\left(1-z\right)^{2}}\right\}
\end{eqnarray*}
%\frac{I_{i}\left(z\right)-1}{1-z}
%+\frac{1-P_{i}\left(z\right)}{z-P_{i}\frac{d}{dz}\left(\frac{I_{i}\left(z\right)-1}{1-z}\right)


\begin{eqnarray*}
\frac{\partial U_{i}\left(z\right)}{\partial z}&=&\frac{(-1+I_{i}[z]) (1-P_{i}[z])}{(1-z)^2 \esp[I_{i}] (z-P_{i}[z])}+\frac{(1-P_{i}[z]) I_{i}^{'}[z]}{(1-z) \esp[I_{i}] (z-P_{i}[z])}-\frac{(-1+I_{i}[z]) (1-P_{i}[z])\left(1-P{'}[z]\right)}{(1-z) \esp[I_{i}] (z-P_{i}[z])^2}\\
&-&\frac{(-1+I_{i}[z]) P_{i}{'}[z]}{(1-z) \esp[I_{i}](z-P_{i}[z])}
\end{eqnarray*}
%______________________________________________________________________
\section{Procesos de Renovaci\'on y Regenerativos}
%______________________________________________________________________

\begin{Def}\label{Def.Tn}
Sean $0\leq T_{1}\leq T_{2}\leq \ldots$ son tiempos aleatorios infinitos en los cuales ocurren ciertos eventos. El n\'umero de tiempos $T_{n}$ en el intervalo $\left[0,t\right)$ es

\begin{eqnarray}
N\left(t\right)=\sum_{n=1}^{\infty}\indora\left(T_{n}\leq t\right),
\end{eqnarray}
para $t\geq0$.
\end{Def}

Si se consideran los puntos $T_{n}$ como elementos de $\rea_{+}$, y $N\left(t\right)$ es el n\'umero de puntos en $\rea$. El proceso denotado por $\left\{N\left(t\right):t\geq0\right\}$, denotado por $N\left(t\right)$, es un proceso puntual en $\rea_{+}$. Los $T_{n}$ son los tiempos de ocurrencia, el proceso puntual $N\left(t\right)$ es simple si su n\'umero de ocurrencias son distintas: $0<T_{1}<T_{2}<\ldots$ casi seguramente.

\begin{Def}
Un proceso puntual $N\left(t\right)$ es un proceso de renovaci\'on si los tiempos de interocurrencia $\xi_{n}=T_{n}-T_{n-1}$, para $n\geq1$, son independientes e identicamente distribuidos con distribuci\'on $F$, donde $F\left(0\right)=0$ y $T_{0}=0$. Los $T_{n}$ son llamados tiempos de renovaci\'on, referente a la independencia o renovaci\'on de la informaci\'on estoc\'astica en estos tiempos. Los $\xi_{n}$ son los tiempos de inter-renovaci\'on, y $N\left(t\right)$ es el n\'umero de renovaciones en el intervalo $\left[0,t\right)$
\end{Def}


\begin{Note}
Para definir un proceso de renovaci\'on para cualquier contexto, solamente hay que especificar una distribuci\'on $F$, con $F\left(0\right)=0$, para los tiempos de inter-renovaci\'on. La funci\'on $F$ en turno degune las otra variables aleatorias. De manera formal, existe un espacio de probabilidad y una sucesi\'on de variables aleatorias $\xi_{1},\xi_{2},\ldots$ definidas en este con distribuci\'on $F$. Entonces las otras cantidades son $T_{n}=\sum_{k=1}^{n}\xi_{k}$ y $N\left(t\right)=\sum_{n=1}^{\infty}\indora\left(T_{n}\leq t\right)$, donde $T_{n}\rightarrow\infty$ casi seguramente por la Ley Fuerte de los Grandes Números.
\end{Note}



%___________________________________________________________________________________________
%
\subsection{Propiedades de los Procesos de Renovaci\'on}
%___________________________________________________________________________________________
%

Los tiempos $T_{n}$ est\'an relacionados con los conteos de $N\left(t\right)$ por

\begin{eqnarray*}
\left\{N\left(t\right)\geq n\right\}&=&\left\{T_{n}\leq t\right\}\\
T_{N\left(t\right)}\leq &t&<T_{N\left(t\right)+1},
\end{eqnarray*}

adem\'as $N\left(T_{n}\right)=n$, y

\begin{eqnarray*}
N\left(t\right)=\max\left\{n:T_{n}\leq t\right\}=\min\left\{n:T_{n+1}>t\right\}
\end{eqnarray*}

Por propiedades de la convoluci\'on se sabe que

\begin{eqnarray*}
P\left\{T_{n}\leq t\right\}=F^{n\star}\left(t\right)
\end{eqnarray*}
que es la $n$-\'esima convoluci\'on de $F$. Entonces

\begin{eqnarray*}
\left\{N\left(t\right)\geq n\right\}&=&\left\{T_{n}\leq t\right\}\\
P\left\{N\left(t\right)\leq n\right\}&=&1-F^{\left(n+1\right)\star}\left(t\right)
\end{eqnarray*}

Adem\'as usando el hecho de que $\esp\left[N\left(t\right)\right]=\sum_{n=1}^{\infty}P\left\{N\left(t\right)\geq n\right\}$
se tiene que

\begin{eqnarray*}
\esp\left[N\left(t\right)\right]=\sum_{n=1}^{\infty}F^{n\star}\left(t\right)
\end{eqnarray*}

\begin{Prop}
Para cada $t\geq0$, la funci\'on generadora de momentos $\esp\left[e^{\alpha N\left(t\right)}\right]$ existe para alguna $\alpha$ en una vecindad del 0, y de aqu\'i que $\esp\left[N\left(t\right)^{m}\right]<\infty$, para $m\geq1$.
\end{Prop}


\begin{Note}
Si el primer tiempo de renovaci\'on $\xi_{1}$ no tiene la misma distribuci\'on que el resto de las $\xi_{n}$, para $n\geq2$, a $N\left(t\right)$ se le llama Proceso de Renovaci\'on retardado, donde si $\xi$ tiene distribuci\'on $G$, entonces el tiempo $T_{n}$ de la $n$-\'esima renovaci\'on tiene distribuci\'on $G\star F^{\left(n-1\right)\star}\left(t\right)$
\end{Note}


\begin{Teo}
Para una constante $\mu\leq\infty$ ( o variable aleatoria), las siguientes expresiones son equivalentes:

\begin{eqnarray}
lim_{n\rightarrow\infty}n^{-1}T_{n}&=&\mu,\textrm{ c.s.}\\
lim_{t\rightarrow\infty}t^{-1}N\left(t\right)&=&1/\mu,\textrm{ c.s.}
\end{eqnarray}
\end{Teo}


Es decir, $T_{n}$ satisface la Ley Fuerte de los Grandes N\'umeros s\'i y s\'olo s\'i $N\left/t\right)$ la cumple.


\begin{Coro}[Ley Fuerte de los Grandes N\'umeros para Procesos de Renovaci\'on]
Si $N\left(t\right)$ es un proceso de renovaci\'on cuyos tiempos de inter-renovaci\'on tienen media $\mu\leq\infty$, entonces
\begin{eqnarray}
t^{-1}N\left(t\right)\rightarrow 1/\mu,\textrm{ c.s. cuando }t\rightarrow\infty.
\end{eqnarray}

\end{Coro}


Considerar el proceso estoc\'astico de valores reales $\left\{Z\left(t\right):t\geq0\right\}$ en el mismo espacio de probabilidad que $N\left(t\right)$

\begin{Def}
Para el proceso $\left\{Z\left(t\right):t\geq0\right\}$ se define la fluctuaci\'on m\'axima de $Z\left(t\right)$ en el intervalo $\left(T_{n-1},T_{n}\right]$:
\begin{eqnarray*}
M_{n}=\sup_{T_{n-1}<t\leq T_{n}}|Z\left(t\right)-Z\left(T_{n-1}\right)|
\end{eqnarray*}
\end{Def}

\begin{Teo}
Sup\'ongase que $n^{-1}T_{n}\rightarrow\mu$ c.s. cuando $n\rightarrow\infty$, donde $\mu\leq\infty$ es una constante o variable aleatoria. Sea $a$ una constante o variable aleatoria que puede ser infinita cuando $\mu$ es finita, y considere las expresiones l\'imite:
\begin{eqnarray}
lim_{n\rightarrow\infty}n^{-1}Z\left(T_{n}\right)&=&a,\textrm{ c.s.}\\
lim_{t\rightarrow\infty}t^{-1}Z\left(t\right)&=&a/\mu,\textrm{ c.s.}
\end{eqnarray}
La segunda expresi\'on implica la primera. Conversamente, la primera implica la segunda si el proceso $Z\left(t\right)$ es creciente, o si $lim_{n\rightarrow\infty}n^{-1}M_{n}=0$ c.s.
\end{Teo}

\begin{Coro}
Si $N\left(t\right)$ es un proceso de renovaci\'on, y $\left(Z\left(T_{n}\right)-Z\left(T_{n-1}\right),M_{n}\right)$, para $n\geq1$, son variables aleatorias independientes e id\'enticamente distribuidas con media finita, entonces,
\begin{eqnarray}
lim_{t\rightarrow\infty}t^{-1}Z\left(t\right)\rightarrow\frac{\esp\left[Z\left(T_{1}\right)-Z\left(T_{0}\right)\right]}{\esp\left[T_{1}\right]},\textrm{ c.s. cuando  }t\rightarrow\infty.
\end{eqnarray}
\end{Coro}

%___________________________________________________________________________________________
%
\subsection{Funci\'on de Renovaci\'on}
%___________________________________________________________________________________________
%


Sup\'ongase que $N\left(t\right)$ es un proceso de renovaci\'on con distribuci\'on $F$ con media finita $\mu$.

\begin{Def}
La funci\'on de renovaci\'on asociada con la distribuci\'on $F$, del proceso $N\left(t\right)$, es
\begin{eqnarray*}
U\left(t\right)=\sum_{n=1}^{\infty}F^{n\star}\left(t\right),\textrm{   }t\geq0,
\end{eqnarray*}
donde $F^{0\star}\left(t\right)=\indora\left(t\geq0\right)$.
\end{Def}


\begin{Prop}
Sup\'ongase que la distribuci\'on de inter-renovaci\'on $F$ tiene densidad $f$. Entonces $U\left(t\right)$ tambi\'en tiene densidad, para $t>0$, y es $U^{'}\left(t\right)=\sum_{n=0}^{\infty}f^{n\star}\left(t\right)$. Adem\'as
\begin{eqnarray*}
\prob\left\{N\left(t\right)>N\left(t-\right)\right\}=0\textrm{,   }t\geq0.
\end{eqnarray*}
\end{Prop}

\begin{Def}
La Transformada de Laplace-Stieljes de $F$ est\'a dada por

\begin{eqnarray*}
\hat{F}\left(\alpha\right)=\int_{\rea_{+}}e^{-\alpha t}dF\left(t\right)\textrm{,  }\alpha\geq0.
\end{eqnarray*}
\end{Def}

Entonces

\begin{eqnarray*}
\hat{U}\left(\alpha\right)=\sum_{n=0}^{\infty}\hat{F^{n\star}}\left(\alpha\right)=\sum_{n=0}^{\infty}\hat{F}\left(\alpha\right)^{n}=\frac{1}{1-\hat{F}\left(\alpha\right)}.
\end{eqnarray*}


\begin{Prop}
La Transformada de Laplace $\hat{U}\left(\alpha\right)$ y $\hat{F}\left(\alpha\right)$ determina una a la otra de manera \'unica por la relaci\'on $\hat{U}\left(\alpha\right)=\frac{1}{1-\hat{F}\left(\alpha\right)}$.
\end{Prop}


\begin{Note}
Un proceso de renovaci\'on $N\left(t\right)$ cuyos tiempos de inter-renovaci\'on tienen media finita, es un proceso Poisson con tasa $\lambda$ si y s\'olo s\'i $\esp\left[U\left(t\right)\right]=\lambda t$, para $t\geq0$.
\end{Note}


\begin{Teo}
Sea $N\left(t\right)$ un proceso puntual simple con puntos de localizaci\'on $T_{n}$ tal que $\eta\left(t\right)=\esp\left[N\left(\right)\right]$ es finita para cada $t$. Entonces para cualquier funci\'on $f:\rea_{+}\rightarrow\rea$,
\begin{eqnarray*}
\esp\left[\sum_{n=1}^{N\left(\right)}f\left(T_{n}\right)\right]=\int_{\left(0,t\right]}f\left(s\right)d\eta\left(s\right)\textrm{,  }t\geq0,
\end{eqnarray*}
suponiendo que la integral exista. Adem\'as si $X_{1},X_{2},\ldots$ son variables aleatorias definidas en el mismo espacio de probabilidad que el proceso $N\left(t\right)$ tal que $\esp\left[X_{n}|T_{n}=s\right]=f\left(s\right)$, independiente de $n$. Entonces
\begin{eqnarray*}
\esp\left[\sum_{n=1}^{N\left(t\right)}X_{n}\right]=\int_{\left(0,t\right]}f\left(s\right)d\eta\left(s\right)\textrm{,  }t\geq0,
\end{eqnarray*}
suponiendo que la integral exista.
\end{Teo}

\begin{Coro}[Identidad de Wald para Renovaciones]
Para el proceso de renovaci\'on $N\left(t\right)$,
\begin{eqnarray*}
\esp\left[T_{N\left(t\right)+1}\right]=\mu\esp\left[N\left(t\right)+1\right]\textrm{,  }t\geq0,
\end{eqnarray*}
\end{Coro}

%___________________________________________________________________________________________
%
%\subsection{Funci\'on de Renovaci\'on}
%___________________________________________________________________________________________
%


\begin{Def}
Sea $h\left(t\right)$ funci\'on de valores reales en $\rea$ acotada en intervalos finitos e igual a cero para $t<0$ La ecuaci\'on de renovaci\'on para $h\left(t\right)$ y la distribuci\'on $F$ es

\begin{eqnarray}\label{Ec.Renovacion}
H\left(t\right)=h\left(t\right)+\int_{\left[0,t\right]}H\left(t-s\right)dF\left(s\right)\textrm{,    }t\geq0,
\end{eqnarray}
donde $H\left(t\right)$ es una funci\'on de valores reales. Esto es $H=h+F\star H$. Decimos que $H\left(t\right)$ es soluci\'on de esta ecuaci\'on si satisface la ecuaci\'on, y es acotada en intervalos finitos e iguales a cero para $t<0$.
\end{Def}

\begin{Prop}
La funci\'on $U\star h\left(t\right)$ es la \'unica soluci\'on de la ecuaci\'on de renovaci\'on (\ref{Ec.Renovacion}).
\end{Prop}

\begin{Teo}[Teorema Renovaci\'on Elemental]
\begin{eqnarray*}
t^{-1}U\left(t\right)\rightarrow 1/\mu\textrm{,    cuando }t\rightarrow\infty.
\end{eqnarray*}
\end{Teo}
%___________________________________________________________________________________________
%
\subsection{Teorema Principal de Renovaci\'on}
%___________________________________________________________________________________________
%

\begin{Note} Una funci\'on $h:\rea_{+}\rightarrow\rea$ es Directamente Riemann Integrable en los siguientes casos:
\begin{itemize}
\item[a)] $h\left(t\right)\geq0$ es decreciente y Riemann Integrable.
\item[b)] $h$ es continua excepto posiblemente en un conjunto de Lebesgue de medida 0, y $|h\left(t\right)|\leq b\left(t\right)$, donde $b$ es DRI.
\end{itemize}
\end{Note}

\begin{Teo}[Teorema Principal de Renovaci\'on]
Si $F$ es no aritm\'etica y $h\left(t\right)$ es Directamente Riemann Integrable (DRI), entonces

\begin{eqnarray*}
lim_{t\rightarrow\infty}U\star h=\frac{1}{\mu}\int_{\rea_{+}}h\left(s\right)ds.
\end{eqnarray*}
\end{Teo}

\begin{Prop}
Cualquier funci\'on $H\left(t\right)$ acotada en intervalos finitos y que es 0 para $t<0$ puede expresarse como
\begin{eqnarray*}
H\left(t\right)=U\star h\left(t\right)\textrm{,  donde }h\left(t\right)=H\left(t\right)-F\star H\left(t\right)
\end{eqnarray*}
\end{Prop}

\begin{Def}
Un proceso estoc\'astico $X\left(t\right)$ es crudamente regenerativo en un tiempo aleatorio positivo $T$ si
\begin{eqnarray*}
\esp\left[X\left(T+t\right)|T\right]=\esp\left[X\left(t\right)\right]\textrm{, para }t\geq0,\end{eqnarray*}
y con las esperanzas anteriores finitas.
\end{Def}

\begin{Prop}
Sup\'ongase que $X\left(t\right)$ es un proceso crudamente regenerativo en $T$, que tiene distribuci\'on $F$. Si $\esp\left[X\left(t\right)\right]$ es acotado en intervalos finitos, entonces
\begin{eqnarray*}
\esp\left[X\left(t\right)\right]=U\star h\left(t\right)\textrm{,  donde }h\left(t\right)=\esp\left[X\left(t\right)\indora\left(T>t\right)\right].
\end{eqnarray*}
\end{Prop}

\begin{Teo}[Regeneraci\'on Cruda]
Sup\'ongase que $X\left(t\right)$ es un proceso con valores positivo crudamente regenerativo en $T$, y def\'inase $M=\sup\left\{|X\left(t\right)|:t\leq T\right\}$. Si $T$ es no aritm\'etico y $M$ y $MT$ tienen media finita, entonces
\begin{eqnarray*}
lim_{t\rightarrow\infty}\esp\left[X\left(t\right)\right]=\frac{1}{\mu}\int_{\rea_{+}}h\left(s\right)ds,
\end{eqnarray*}
donde $h\left(t\right)=\esp\left[X\left(t\right)\indora\left(T>t\right)\right]$.
\end{Teo}
%________________________________________________________________________
\subsection{Procesos Regenerativos}
%________________________________________________________________________

Para $\left\{X\left(t\right):t\geq0\right\}$ Proceso Estoc\'astico a tiempo continuo con estado de espacios $S$, que es un espacio m\'etrico, con trayectorias continuas por la derecha y con l\'imites por la izquierda c.s. Sea $N\left(t\right)$ un proceso de renovaci\'on en $\rea_{+}$ definido en el mismo espacio de probabilidad que $X\left(t\right)$, con tiempos de renovaci\'on $T$ y tiempos de inter-renovaci\'on $\xi_{n}=T_{n}-T_{n-1}$, con misma distribuci\'on $F$ de media finita $\mu$.



\begin{Def}
Para el proceso $\left\{\left(N\left(t\right),X\left(t\right)\right):t\geq0\right\}$, sus trayectoria muestrales en el intervalo de tiempo $\left[T_{n-1},T_{n}\right)$ est\'an descritas por
\begin{eqnarray*}
\zeta_{n}=\left(\xi_{n},\left\{X\left(T_{n-1}+t\right):0\leq t<\xi_{n}\right\}\right)
\end{eqnarray*}
Este $\zeta_{n}$ es el $n$-\'esimo segmento del proceso. El proceso es regenerativo sobre los tiempos $T_{n}$ si sus segmentos $\zeta_{n}$ son independientes e id\'enticamennte distribuidos.
\end{Def}


\begin{Obs}
Si $\tilde{X}\left(t\right)$ con espacio de estados $\tilde{S}$ es regenerativo sobre $T_{n}$, entonces $X\left(t\right)=f\left(\tilde{X}\left(t\right)\right)$ tambi\'en es regenerativo sobre $T_{n}$, para cualquier funci\'on $f:\tilde{S}\rightarrow S$.
\end{Obs}

\begin{Obs}
Los procesos regenerativos son crudamente regenerativos, pero no al rev\'es.
\end{Obs}

\begin{Def}[Definici\'on Cl\'asica]
Un proceso estoc\'astico $X=\left\{X\left(t\right):t\geq0\right\}$ es llamado regenerativo is existe una variable aleatoria $R_{1}>0$ tal que
\begin{itemize}
\item[i)] $\left\{X\left(t+R_{1}\right):t\geq0\right\}$ es independiente de $\left\{\left\{X\left(t\right):t<R_{1}\right\},\right\}$
\item[ii)] $\left\{X\left(t+R_{1}\right):t\geq0\right\}$ es estoc\'asticamente equivalente a $\left\{X\left(t\right):t>0\right\}$
\end{itemize}

Llamamos a $R_{1}$ tiempo de regeneraci\'on, y decimos que $X$ se regenera en este punto.
\end{Def}

$\left\{X\left(t+R_{1}\right)\right\}$ es regenerativo con tiempo de regeneraci\'on $R_{2}$, independiente de $R_{1}$ pero con la misma distribuci\'on que $R_{1}$. Procediendo de esta manera se obtiene una secuencia de variables aleatorias independientes e id\'enticamente distribuidas $\left\{R_{n}\right\}$ llamados longitudes de ciclo. Si definimos a $Z_{k}\equiv R_{1}+R_{2}+\cdots+R_{k}$, se tiene un proceso de renovaci\'on llamado proceso de renovaci\'on encajado para $X$.

\begin{Note}
Un proceso regenerativo con media de la longitud de ciclo finita es llamado positivo recurrente.
\end{Note}


\begin{Def}
Para $x$ fijo y para cada $t\geq0$, sea $I_{x}\left(t\right)=1$ si $X\left(t\right)\leq x$,  $I_{x}\left(t\right)=0$ en caso contrario, y def\'inanse los tiempos promedio
\begin{eqnarray*}
\overline{X}&=&lim_{t\rightarrow\infty}\frac{1}{t}\int_{0}^{\infty}X\left(u\right)du\\
\prob\left(X_{\infty}\leq x\right)&=&lim_{t\rightarrow\infty}\frac{1}{t}\int_{0}^{\infty}I_{x}\left(u\right)du,
\end{eqnarray*}
cuando estos l\'imites existan.
\end{Def}

Como consecuencia del teorema de Renovaci\'on-Recompensa, se tiene que el primer l\'imite  existe y es igual a la constante
\begin{eqnarray*}
\overline{X}&=&\frac{\esp\left[\int_{0}^{R_{1}}X\left(t\right)dt\right]}{\esp\left[R_{1}\right]},
\end{eqnarray*}
suponiendo que ambas esperanzas son finitas.

\begin{Note}
\begin{itemize}
\item[a)] Si el proceso regenerativo $X$ es positivo recurrente y tiene trayectorias muestrales no negativas, entonces la ecuaci\'on anterior es v\'alida.
\item[b)] Si $X$ es positivo recurrente regenerativo, podemos construir una \'unica versi\'on estacionaria de este proceso, $X_{e}=\left\{X_{e}\left(t\right)\right\}$, donde $X_{e}$ es un proceso estoc\'astico regenerativo y estrictamente estacionario, con distribuci\'on marginal distribuida como $X_{\infty}$
\end{itemize}
\end{Note}

%__________________________________________________________________________________________
\subsection{Procesos Regenerativos Estacionarios}
%__________________________________________________________________________________________


Un proceso estoc\'astico a tiempo continuo $\left\{V\left(t\right),t\geq0\right\}$ es un proceso regenerativo si existe una sucesi\'on de variables aleatorias independientes e id\'enticamente distribuidas $\left\{X_{1},X_{2},\ldots\right\}$, sucesi\'on de renovaci\'on, tal que para cualquier conjunto de Borel $A$,

\begin{eqnarray*}
\prob\left\{V\left(t\right)\in A|X_{1}+X_{2}+\cdots+X_{R\left(t\right)}=s,\left\{V\left(\tau\right),\tau<s\right\}\right\}=\prob\left\{V\left(t-s\right)\in A|X_{1}>t-s\right\},
\end{eqnarray*}
para todo $0\leq s\leq t$, donde $R\left(t\right)=\max\left\{X_{1}+X_{2}+\cdots+X_{j}\leq t\right\}=$n\'umero de renovaciones que ocurren en $\left[0,t\right]$.

Sea $X=X_{1}$ y sea $F$ la funci\'on de distrbuci\'on de $X$


\begin{Def}
Se define el proceso estacionario, $\left\{V^{*}\left(t\right),t\geq0\right\}$, para $\left\{V\left(t\right),t\geq0\right\}$ por

\begin{eqnarray*}
\prob\left\{V\left(t\right)\in A\right\}=\frac{1}{\esp\left[X\right]}\int_{0}^{\infty}\prob\left\{V\left(t+x\right)\in A|X>x\right\}\left(1-F\left(x\right)\right)dx,
\end{eqnarray*}
para todo $t\geq0$ y todo conjunto de Borel $A$.
\end{Def}

\begin{Def}
Una modificaci\'on medible de un proceso $\left\{V\left(t\right),t\geq0\right\}$, es una versi\'on de este, $\left\{V\left(t,w\right)\right\}$ conjuntamente medible para $t\geq0$ y para $w\in S$, $S$ espacio de estados para $\left\{V\left(t\right),t\geq0\right\}$.
\end{Def}

\begin{Teo}
Sea $\left\{V\left(t\right),t\geq\right\}$ un proceso regenerativo no negativo con modificaci\'on medible. Sea $\esp\left[X\right]<\infty$. Entonces el proceso estacionario dado por la ecuaci\'on anterior est\'a bien definido y tiene funci\'on de distribuci\'on independiente de $t$, adem\'as
\begin{itemize}
\item[i)] \begin{eqnarray*}
\esp\left[V^{*}\left(0\right)\right]&=&\frac{\esp\left[\int_{0}^{X}V\left(s\right)ds\right]}{\esp\left[X\right]}\end{eqnarray*}
\item[ii)] Si $\esp\left[V^{*}\left(0\right)\right]<\infty$, equivalentemente, si $\esp\left[\int_{0}^{X}V\left(s\right)ds\right]<\infty$,entonces
\begin{eqnarray*}
\frac{\int_{0}^{t}V\left(s\right)ds}{t}\rightarrow\frac{\esp\left[\int_{0}^{X}V\left(s\right)ds\right]}{\esp\left[X\right]}
\end{eqnarray*}
con probabilidad 1 y en media, cuando $t\rightarrow\infty$.
\end{itemize}
\end{Teo}

%_______________________________________________________________________________________________________
\section{Tiempo de Ciclo Promedio}
%_______________________________________________________________________________________________________

Consideremos una cola de la red de sistemas de visitas c\'iclicas fija, $Q_{l}$.


Conforme a la definici\'on dada al principio del cap\'itulo, definici\'on (\ref{Def.Tn}), sean $T_{1},T_{2},\ldots$ los puntos donde las longitudes de las colas de la red de sistemas de visitas c\'iclicas son cero simult\'aneamente, cuando la cola $Q_{l}$ es visitada por el servidor para dar servicio, es decir, $L_{1}\left(T_{i}\right)=0,L_{2}\left(T_{i}\right)=0,\hat{L}_{1}\left(T_{i}\right)=0$ y $\hat{L}_{2}\left(T_{i}\right)=0$, a estos puntos se les denominar\'a puntos regenerativos. Entonces,

\begin{Def}
Al intervalo de tiempo entre dos puntos regenerativos se le llamar\'a ciclo regenerativo.
\end{Def}

\begin{Def}
Para $T_{i}$ se define, $M_{i}$, el n\'umero de ciclos de visita a la cola $Q_{l}$, durante el ciclo regenerativo, es decir, $M_{i}$ es un proceso de renovaci\'on.
\end{Def}

\begin{Def}
Para cada uno de los $M_{i}$'s, se definen a su vez la duraci\'on de cada uno de estos ciclos de visita en el ciclo regenerativo, $C_{i}^{(m)}$, para $m=1,2,\ldots,M_{i}$, que a su vez, tambi\'en es n proceso de renovaci\'on.
\end{Def}

En nuestra notaci\'on $V\left(t\right)\equiv C_{i}$ y $X_{i}=C_{i}^{(m)}$ para nuestra segunda definici\'on, mientras que para la primera la notaci\'on es: $X\left(t\right)\equiv C_{i}$ y $R_{i}\equiv C_{i}^{(m)}$.


%___________________________________________________________________________________________
%
\section{Expresion de las Parciales mixtas para $F_{1}$ y $F_{2}$}
%___________________________________________________________________________________________
\begin{enumerate}

%1

\item \begin{eqnarray*}
\frac{\partial}{\partial z_{1}}\frac{\partial}{\partial z_{1}}F_{1}\left(\theta_{1}\left(\tilde{P}_{2}\left(z_{2}\right)\hat{P}_{1}\left(w_{1}\right)
\hat{P}_{2}\left(w_{2}\right),z_{2}\right)\right)|_{\mathbf{z,w}=1}&=&0\\
\end{eqnarray*}

%2

\item
\begin{eqnarray*}
\frac{\partial}{\partial z_{2}}\frac{\partial}{\partial z_{1}}F_{1}\left(\theta_{1}\left(\tilde{P}_{2}\left(z_{2}\right)\hat{P}_{1}\left(w_{1}\right)
\hat{P}_{2}\left(w_{2}\right),z_{2}\right)\right)|_{\mathbf{z,w}=1}&=&0\\
\end{eqnarray*}

%3

\item
\begin{eqnarray*}
\frac{\partial}{\partial w_{1}}\frac{\partial}{\partial z_{1}}F_{1}\left(\theta_{1}\left(\tilde{P}_{2}\left(z_{2}\right)\hat{P}_{1}\left(w_{1}\right)
\hat{P}_{2}\left(w_{2}\right),z_{2}\right)\right)|_{\mathbf{z,w}=1}&=&0\\
\end{eqnarray*}

%4

\item
\begin{eqnarray*}
\frac{\partial}{\partial w_{2}}\frac{\partial}{\partial z_{1}}F_{1}\left(\theta_{1}\left(\tilde{P}_{2}\left(z_{2}\right)\hat{P}_{1}\left(w_{1}\right)
\hat{P}_{2}\left(w_{2}\right),z_{2}\right)\right)|_{\mathbf{z,w}=1}&=&0
\end{eqnarray*}

%5

\item
\begin{eqnarray*}
\frac{\partial}{\partial z_{1}}\frac{\partial}{\partial z_{2}}F_{1}\left(\theta_{1}\left(\tilde{P}_{2}\left(z_{2}\right)\hat{P}_{1}\left(w_{1}\right)
\hat{P}_{2}\left(w_{2}\right),z_{2}\right)\right)|_{\mathbf{z,w}=1}&=&0
\end{eqnarray*}

%6

\item
\begin{eqnarray*}
&&\frac{\partial}{\partial z_{2}}\frac{\partial}{\partial z_{2}}F_{1}\left(\theta_{1}\left(\tilde{P}_{2}\left(z_{2}\right)\hat{P}_{1}\left(w_{1}\right)
\hat{P}_{2}\left(w_{2}\right)\right),z_{2}\right)|_{\mathbf{z,w}=1}=f_{1}\left(2,2\right)+\frac{1}{1-\mu_{1}}\tilde{P}_{2}^{(2)}\left(1\right)f_{1}\left(1\right)\\
&+&\tilde{\mu}_{2}^{2}\theta_{1}^{(2)}\left(1\right)f_{1}\left(1\right)+2\frac{\tilde{\mu}_{2}}{1-\mu_{1}}f_{1}\left(1,2\right)+\left(\frac{\tilde{\mu}_{2}}{1-\mu_{1}}\right)^{2}f_{1}\left(1,1\right)
\end{eqnarray*}

%7

\item
\begin{eqnarray*}
&&\frac{\partial}{\partial w_{1}}\frac{\partial}{\partial z_{2}}F_{1}\left(\theta_{1}\left(\tilde{P}_{2}\left(z_{2}\right)\hat{P}_{1}\left(w_{1}\right)
\hat{P}_{2}\left(w_{2}\right),z_{2}\right)\right)|_{\mathbf{z,w}=1}=\frac{\tilde{\mu}_{2}\hat{\mu}_{1}}{1-\mu_{1}}f_{1}\left(1\right)\\
&+&\tilde{\mu}_{2}\hat{\mu}_{1}\theta_{1}^{(2)}\left(1\right)f_{1}\left(1\right)+\frac{\hat{\mu}_{1}}{1-\mu_{1}}f_{1}\left(1,2\right)+\tilde{\mu}_{2}\hat{\mu}_{1}\left(\frac{1}{1-\mu_{1}}\right)^{2}f_{1}\left(1,1\right)
\end{eqnarray*}

%8

\item \begin{eqnarray*}
&&\frac{\partial}{\partial w_{2}}\frac{\partial}{\partial z_{2}}F_{1}\left(\theta_{1}\left(\tilde{P}_{2}\left(z_{2}\right)\hat{P}_{1}\left(w_{1}\right)
\hat{P}_{2}\left(w_{2}\right),z_{2}\right)\right)|_{\mathbf{z,w}=1}=\frac{\tilde{\mu}_{2}\hat{\mu}_{2}}{1-\mu_{1}}f_{1}\left(1\right)\\
&+&\tilde{\mu}_{2}\hat{\mu}_{2}\theta_{1}^{(2)}\left(1\right)f_{1}\left(1\right)+\frac{\hat{\mu}_{2}}{1-\mu_{1}}f_{1}\left(1,2\right)+\tilde{\mu}_{2}\hat{\mu}_{2}\left(\frac{1}{1-\mu_{1}}\right)^{2}f_{1}\left(1,1\right)
\end{eqnarray*}

%9

\item \begin{eqnarray*}
\frac{\partial}{\partial z_{1}}\frac{\partial}{\partial w_{1}}F_{1}\left(\theta_{1}\left(\tilde{P}_{2}\left(z_{2}\right)\hat{P}_{1}\left(w_{1}\right)
\hat{P}_{2}\left(w_{2}\right),z_{2}\right)\right)|_{\mathbf{z,w}=1}&=&0
\end{eqnarray*}

%10

\item \begin{eqnarray*}
&&\frac{\partial}{\partial z_{2}}\frac{\partial}{\partial w_{1}}F_{1}\left(\theta_{1}\left(\tilde{P}_{2}\left(z_{2}\right)\hat{P}_{1}\left(w_{1}\right)
\hat{P}_{2}\left(w_{2}\right),z_{2}\right)\right)|_{\mathbf{z,w}=1}=\frac{\tilde{\mu}_{2}\hat{\mu}_{1}}{1-\mu_{1}}f_{1}\left(2\right)\\
&+&\tilde{\mu}_{2}\hat{\mu}_{1}\theta_{1}^{(2)}\left(1\right)f_{1}\left(2\right)+\frac{\hat{\mu}_{1}}{1-\mu_{1}}f_{1}\left(2,1\right)+\tilde{\mu}_{2}\hat{\mu}_{1}\left(\frac{1}{1-\mu_{1}}\right)^{2}f_{1}\left(1,1\right)
\end{eqnarray*}

%11

\item
\begin{eqnarray*}
&&\frac{\partial}{\partial w_{1}}\frac{\partial}{\partial w_{1}}F_{1}\left(\theta_{1}\left(\tilde{P}_{2}\left(z_{2}\right)\hat{P}_{1}\left(w_{1}\right)
\hat{P}_{2}\left(w_{2}\right),z_{2}\right)\right)|_{\mathbf{z,w}=1}=\frac{1}{1-\mu_{1}} \hat{P}_{1}^{(2)}\left(1\right)f_{1}\left(1\right)\\
&+&\hat{\mu}_{1}\theta_{1}^{(2)}\left(1\right)f_{1}\left(1\right)+\left(\frac{\hat{\mu}_{1}}{1-\mu_{1}}\right)^{2}f_{1}\left(1,1\right)
\end{eqnarray*}

%12

\item
\begin{eqnarray*}
&&\frac{\partial}{\partial w_{2}}\frac{\partial}{\partial w_{1}}F_{1}\left(\theta_{1}\left(\tilde{P}_{2}\left(z_{2}\right)\hat{P}_{1}\left(w_{1}\right)
\hat{P}_{2}\left(w_{2}\right),z_{2}\right)\right)|_{\mathbf{z,w}=1}=\hat{\mu}_{1}\hat{\mu}_{2}f_{1}\left(1\right)\\
&+&\frac{\hat{\mu}_{1}\hat{\mu}_{2}}{1-\mu_{1}}f_{1}\left(1\right)+\hat{\mu}_{1}\hat{\mu}_{2}\theta_{1}^{(2)}\left(1\right)f_{1}\left(1\right)+\hat{\mu}_{1}\hat{\mu}_{2}\left(\frac{1}{1-\mu_{1}}\right)^{2}f_{1}\left(1,1\right)
\end{eqnarray*}

%13

\item \begin{eqnarray*}
\frac{\partial}{\partial z_{1}}\frac{\partial}{\partial w_{2}}F_{1}\left(\theta_{1}\left(\tilde{P}_{2}\left(z_{2}\right)\hat{P}_{1}\left(w_{1}\right)
\hat{P}_{2}\left(w_{2}\right),z_{2}\right)\right)|_{\mathbf{z,w}=1}&=&0
\end{eqnarray*}

%14

\item \begin{eqnarray*}
&&\frac{\partial}{\partial z_{2}}\frac{\partial}{\partial w_{2}}F_{1}\left(\theta_{1}\left(\tilde{P}_{2}\left(z_{2}\right)\hat{P}_{1}\left(w_{1}\right)
\hat{P}_{2}\left(w_{2}\right),z_{2}\right)\right)|_{\mathbf{z,w}=1}=\frac{\tilde{\mu}_{2}\hat{\mu}_{2}}{1-\mu_{1}}f_{1}\left(1\right)\\
&+&\tilde{\mu}_{2}\hat{\mu}_{2}\theta_{1}^{(2)}\left(1\right)f_{1}\left(1\right)+\frac{\hat{\mu}_{2}}{1-\mu_{1}}f_{1}\left(2,1\right)+\tilde{\mu}_{2}\hat{\mu}_{2}\left(\frac{1}{1-\mu_{1}}\right)^{2}f_{1}\left(2,2\right)
\end{eqnarray*}

%15

\item \begin{eqnarray*}
&&\frac{\partial}{\partial w_{1}}\frac{\partial}{\partial w_{2}}F_{1}\left(\theta_{1}\left(\tilde{P}_{2}\left(z_{2}\right)\hat{P}_{1}\left(w_{1}\right)
\hat{P}_{2}\left(w_{2}\right),z_{2}\right)\right)|_{\mathbf{z,w}=1}=\frac{\hat{\mu}_{1}\hat{\mu}_{2}}{1-\mu_{1}}f_{1}\left(1\right)\\
&+&\hat{\mu}_{1}\hat{\mu}_{2}\theta_{1}^{(2)}\left(1\right)f_{1}\left(1\right)+\hat{\mu}_{1}\hat{\mu}_{2}\left(\frac{1}{1-\mu_{1}}\right)^{2}f_{1}\left(1,1\right)
\end{eqnarray*}

%16

\item
\begin{eqnarray*}
&&\frac{\partial}{\partial w_{2}}\frac{\partial}{\partial w_{2}}F_{1}\left(\theta_{1}\left(\tilde{P}_{2}\left(z_{2}\right)\hat{P}_{1}\left(w_{1}\right)
\hat{P}_{2}\left(w_{2}\right),z_{2}\right)\right)|_{\mathbf{z,w}=1}=\frac{1}{1-\mu_{1}}\hat{P}_{2}^{(2)}\left(w_{2}\right)f_{1}\left(1\right)\\
&+&\hat{\mu}_{2}^{2}\theta_{1}^{(2)}\left(1\right)f_{1}\left(1\right)+\left(\hat{\mu}_{2}\frac{1}{1-\mu_{1}}\right)^{2}f_{1}\left(1,1\right)
\end{eqnarray*}

%17

\item
\begin{eqnarray*}
&&\frac{\partial}{\partial z_{1}}\frac{\partial}{\partial z_{1}}F_{2}\left(z_{1},\tilde{\theta}_{2}\left(P_{1}\left(z_{1}\right)\hat{P}_{1}\left(w_{1}\right)
\hat{P}_{2}\left(w_{2}\right)\right)\right)|_{\mathbf{z,w}=1}=\frac{1}{1-\tilde{\mu}_{2}}P_{1}^{(2)}\left(1\right)
f_{2}\left(2\right)+f_{2}\left(1,1\right)\\
&+&\mu_{1}^{2}\tilde{\theta}_{2}^{(2)}\left(1\right)f_{2}\left(2\right)+\mu_{1}\frac{1}{1-\tilde{\mu}_{2}}f_{2}\left(1,2\right)+\left(\mu_{1}\frac{1}{1-\tilde{\mu}_{2}}\right)^{2}f_{2}\left(2,2\right)+\frac{\mu_{1}}{1-\tilde{\mu}_{2}}f_{2}\left(1,2\right)\\
\end{eqnarray*}

%18

\item \begin{eqnarray*}
\frac{\partial}{\partial z_{2}}\frac{\partial}{\partial z_{1}}F_{2}\left(z_{1},\tilde{\theta}_{2}\left(P_{1}\left(z_{1}\right)\hat{P}_{1}\left(w_{1}\right)
\hat{P}_{2}\left(w_{2}\right)\right)\right)|_{\mathbf{z,w}=1}&=&0
\end{eqnarray*}

%19

\item \begin{eqnarray*}
&&\frac{\partial}{\partial w_{1}}\frac{\partial}{\partial z_{1}}F_{2}\left(z_{1},\tilde{\theta}_{2}\left(P_{1}\left(z_{1}\right)\hat{P}_{1}\left(w_{1}\right)
\hat{P}_{2}\left(w_{2}\right)\right)\right)|_{\mathbf{z,w}=1}=\frac{\mu_{1}\hat{\mu}_{1}}{1-\tilde{\mu}_{2}}f_{2}\left(2\right)\\
&+&\mu_{1}\hat{\mu}_{1}\tilde{\theta}_{2}^{(2)}\left(1\right)f_{2}\left(2\right)+\mu_{1}\hat{\mu}_{1}\left(\frac{1}{1-\tilde{\mu}_{2}}\right)^{2}f_{2}\left(2,2\right)+\frac{\hat{\mu}_{1}}{1-\tilde{\mu}_{2}}f_{2}\left(1,2\right)\end{eqnarray*}

%20

\item \begin{eqnarray*}
&&\frac{\partial}{\partial w_{2}}\frac{\partial}{\partial z_{1}}F_{2}\left(z_{1},\tilde{\theta}_{2}\left(P_{1}\left(z_{1}\right)\hat{P}_{1}\left(w_{1}\right)
\hat{P}_{2}\left(w_{2}\right)\right)\right)|_{\mathbf{z,w}=1}=\frac{\mu_{1}\hat{\mu}_{2}}{1-\tilde{\mu}_{2}}f_{2}\left(2\right)\\
&+&\mu_{1}\hat{\mu}_{2}\tilde{\theta}_{2}^{(2)}\left(1\right)f_{2}\left(2\right)+\mu_{1}\hat{\mu}_{2}
\left(\frac{1}{1-\tilde{\mu}_{2}}\right)^{2}f_{2}\left(2,2\right)+\frac{\hat{\mu}_{2}}{1-\tilde{\mu}_{2}}f_{2}\left(1,2\right)\end{eqnarray*}
%___________________________________________________________________________________________


%\newpage

%___________________________________________________________________________________________
%
%\section{Parciales mixtas de $F_{2}$ para $z_{2}$}
%___________________________________________________________________________________________
%___________________________________________________________________________________________
\item
\begin{eqnarray*}
\frac{\partial}{\partial z_{1}}\frac{\partial}{\partial z_{2}}F_{2}\left(z_{1},\tilde{\theta}_{2}\left(P_{1}\left(z_{1}\right)\hat{P}_{1}\left(w_{1}\right)
\hat{P}_{2}\left(w_{2}\right)\right)\right)|_{\mathbf{z,w}=1}&=&0;\\
\end{eqnarray*}
\item
\begin{eqnarray*}
\frac{\partial}{\partial z_{2}}\frac{\partial}{\partial z_{2}}F_{2}\left(z_{1},\tilde{\theta}_{2}\left(P_{1}\left(z_{1}\right)\hat{P}_{1}\left(w_{1}\right)
\hat{P}_{2}\left(w_{2}\right)\right)\right)|_{\mathbf{z,w}=1}&=&0\\
\end{eqnarray*}
\item
\begin{eqnarray*}\frac{\partial}{\partial w_{1}}\frac{\partial}{\partial z_{2}}F_{2}\left(z_{1},\tilde{\theta}_{2}\left(P_{1}\left(z_{1}\right)\hat{P}_{1}\left(w_{1}\right)
\hat{P}_{2}\left(w_{2}\right)\right)\right)|_{\mathbf{z,w}=1}&=&0\\
\end{eqnarray*}
\item
\begin{eqnarray*}\frac{\partial}{\partial w_{2}}\frac{\partial}{\partial z_{2}}F_{2}\left(z_{1},\tilde{\theta}_{2}\left(P_{1}\left(z_{1}\right)\hat{P}_{1}\left(w_{1}\right)
\hat{P}_{2}\left(w_{2}\right)\right)\right)|_{\mathbf{z,w}=1}&=&0
\end{eqnarray*}
%___________________________________________________________________________________________

%\newpage

%___________________________________________________________________________________________
%
%\section{Parciales mixtas de $F_{2}$ para $w_{1}$}
%___________________________________________________________________________________________
\item
\begin{eqnarray*}
\frac{\partial}{\partial z_{1}}\frac{\partial}{\partial w_{1}}F_{2}\left(z_{1},\tilde{\theta}_{2}\left(P_{1}\left(z_{1}\right)\hat{P}_{1}\left(w_{1}\right)
\hat{P}_{2}\left(w_{2}\right)\right)\right)|_{\mathbf{z,w}=1}&=&\frac{1}{1-\tilde{\mu}_{2}}P_{1}^{(2)}\left(1\right)\frac{\partial}{\partial
z_{2}}F_{2}\left(1,1\right)+\mu_{1}^{2}\tilde{\theta}_{2}^{(2)}\left(1\right)\frac{\partial}{\partial
z_{2}}F_{2}\left(1,1\right)\\
&+&\mu_{1}\frac{1}{1-\tilde{\mu}_{2}}f_{2}\left(1,2\right)+\left(\mu_{1}\frac{1}{1-\tilde{\mu}_{2}}\right)^{2}f_{2}\left(2,2\right)\\
&+&\mu_{1}\frac{1}{1-\tilde{\mu}_{2}}f_{2}\left(1,2\right)+f_{2}\left(1,1\right)
\end{eqnarray*}
%___________________________________________________________________________________________
%___________________________________________________________________________________________
\item \begin{eqnarray*}
\frac{\partial}{\partial z_{2}}\frac{\partial}{\partial w_{1}}F_{2}\left(z_{1},\tilde{\theta}_{2}\left(P_{1}\left(z_{1}\right)\hat{P}_{1}\left(w_{1}\right)
\hat{P}_{2}\left(w_{2}\right)\right)\right)|_{\mathbf{z,w}=1}&=&0
\end{eqnarray*}
%___________________________________________________________________________________________
\item
\begin{eqnarray*}
\frac{\partial}{\partial w_{1}}\frac{\partial}{\partial w_{1}}F_{2}\left(z_{1},\tilde{\theta}_{2}\left(P_{1}\left(z_{1}\right)\hat{P}_{1}\left(w_{1}\right)
\hat{P}_{2}\left(w_{2}\right)\right)\right)|_{\mathbf{z,w}=1}&=&\mu_{1}\hat{\mu}_{1}\frac{1}{1-\tilde{\mu}_{2}}\frac{\partial}{\partial
z_{2}}F_{2}\left(1,1\right)+\mu_{1}\hat{\mu}_{1}\left(\frac{1}{1-\tilde{\mu}_{2}}\right)^{2}\frac{\partial}{\partial
z_{2}}F_{2}\left(1,1\right)\\
&+&\mu_{1}\hat{\mu}_{1}
\left(\frac{1}{1-\tilde{\mu}_{2}}\right)^{2}\frac{\partial}{\partial
z_{2}}F_{2}\left(1,1\right)+\hat{\mu}_{1}\frac{1}{1-\tilde{\mu}_{2}}f_{2}\left(1,2\right)\end{eqnarray*}
\item
\begin{eqnarray*}
\frac{\partial}{\partial w_{2}}\frac{\partial}{\partial w_{1}}F_{2}\left(z_{1},\tilde{\theta}_{2}\left(P_{1}\left(z_{1}\right)\hat{P}_{1}\left(w_{1}\right)
\hat{P}_{2}\left(w_{2}\right)\right)\right)|_{\mathbf{z,w}=1}&=&\hat{\mu}_{1}\hat{\mu}_{2}\frac{1}{1-\tilde{\mu}_{2}}\frac{\partial}{\partial
z_{2}}F_{2}\left(1,1\right)+\hat{\mu}_{1}\hat{\mu}_{2}\tilde{\theta}_{2}^{(2)}\left(1\right)\frac{\partial}{\partial
z_{2}}F_{2}\left(1,1\right)\\
&+&\hat{\mu}_{1}\hat{\mu}_{2}\left(\frac{1}{1-\tilde{\mu}_{2}}\right)^{2}f_{2}\left(2,2\right)\end{eqnarray*}
%___________________________________________________________________________________________

%\newpage

%___________________________________________________________________________________________
%
%\section{Parciales mixtas de $F_{2}$ para $w_{2}$}
%___________________________________________________________________________________________
%___________________________________________________________________________________________
\item \begin{eqnarray*}
\frac{\partial}{\partial z_{1}}\frac{\partial}{\partial w_{2}}F_{2}\left(z_{1},\tilde{\theta}_{2}\left(P_{1}\left(z_{1}\right)\hat{P}_{1}\left(w_{1}\right)
\hat{P}_{2}\left(w_{2}\right)\right)\right)|_{\mathbf{z,w}=1}&=&\mu_{1}\hat{\mu}_{2}\frac{1}{1-\tilde{\mu}_{2}}\frac{\partial}{\partial
z_{1}}F_{2}\left(1\right)+\mu_{1}\hat{\mu}_{2}\tilde{\theta}_{2}^{(2)}\left(1\right)\frac{\partial}{\partial
z_{2}}F_{2}\left(1,1\right)\\
&+&\hat{\mu}_{2}\mu_{1}\left(\frac{1}{1-\tilde{\mu}_{2}}\right)^{2}f_{2}\left(2,2\right)+\hat{\mu}_{2}\frac{1}{1-\tilde{\mu}_{2}}f_{2}\left(1,2\right)\end{eqnarray*}
\item
\begin{eqnarray*}
\frac{\partial}{\partial z_{2}}\frac{\partial}{\partial w_{2}}F_{2}\left(z_{1},\tilde{\theta}_{2}\left(P_{1}\left(z_{1}\right)\hat{P}_{1}\left(w_{1}\right)
\hat{P}_{2}\left(w_{2}\right)\right)\right)|_{\mathbf{z,w}=1}&=&0
\end{eqnarray*}
\item
\begin{eqnarray*}
\frac{\partial}{\partial w_{1}}\frac{\partial}{\partial w_{2}}F_{2}\left(z_{1},\tilde{\theta}_{2}\left(P_{1}\left(z_{1}\right)\hat{P}_{1}\left(w_{1}\right)
\hat{P}_{2}\left(w_{2}\right)\right)\right)|_{\mathbf{z,w}=1}&=&\hat{\mu}_{1}\hat{\mu}_{2}\frac{1}{1-\tilde{\mu}_{2}}\frac{\partial}{\partial
z_{2}}F_{2}\left(1,1\right)+\hat{\mu}_{1}\hat{\mu}_{2}\tilde{\theta}_{2}^{(2)}\left(1\right)\frac{\partial}{\partial
z_{2}}F_{2}\left(1,1\right)\\
&+&\hat{\mu}_{1}\hat{\mu}_{2}\left(\frac{1}{1-\tilde{\mu}_{2}}\right)^{2}f_{2}\left(2,2\right)\end{eqnarray*}
\item
\begin{eqnarray*}
\frac{\partial}{\partial w_{2}}\frac{\partial}{\partial w_{2}}F_{2}\left(z_{1},\tilde{\theta}_{2}\left(P_{1}\left(z_{1}\right)\hat{P}_{1}\left(w_{1}\right)
\hat{P}_{2}\left(w_{2}\right)\right)\right)|_{\mathbf{z,w}=1}&=&\hat{P}_{2}^{(2)}\left(1\right)\frac{1}{1-\tilde{\mu}_{2}}\frac{\partial}{\partial
z_{2}}F_{2}\left(1,1\right)+\hat{\mu}_{2}^{2}\tilde{\theta}_{2}^{(2)}\left(1\right)\frac{\partial}{\partial
z_{2}}F_{2}\left(1,1\right)\\
&+&\left(\hat{\mu}_{2}\frac{1}{1-\tilde{\mu}_{2}}\right)^{2}f_{2}\left(2,2\right)
\end{eqnarray*}
%___________________________________________________________________________________________




%\newpage
%___________________________________________________________________________________________
%
%\section{Parciales mixtas de $\hat{F}_{1}$ para $z_{1}$}
%___________________________________________________________________________________________
\item \begin{eqnarray*}
\frac{\partial}{\partial z_{1}}\frac{\partial}{\partial z_{1}}\hat{F}_{1}\left(\hat{\theta}_{1}\left(P_{1}\left(z_{1}\right)\tilde{P}_{2}\left(z_{2}\right)
\hat{P}_{2}\left(w_{2}\right)\right),w_{2}\right)|_{\mathbf{z,w}=1}&=&\frac{1}{1-\hat{\mu}_{1}}P_{1}^{(2)}\frac{\partial}{\partial w_{1}}\hat{F}_{1}\left(1,1\right)+\mu_{1}^2\hat{\theta}_{1}^{(2)}\left(1\right)\frac{\partial}{\partial w_{1}}\hat{F}_{1}\left(1,1\right)\\
&+&\mu_{1}^2\left(\frac{1}{1- \hat{\mu}_{1}}\right)^2\hat{f}_{1}\left(1,1\right)
\end{eqnarray*}
%___________________________________________________________________________________________

%___________________________________________________________________________________________
\item
\begin{eqnarray*}
\frac{\partial}{\partial z_{2}}\frac{\partial}{\partial z_{1}}\hat{F}_{1}\left(\hat{\theta}_{1}\left(P_{1}\left(z_{1}\right)\tilde{P}_{2}\left(z_{2}\right)
\hat{P}_{2}\left(w_{2}\right)\right),w_{2}\right)|_{\mathbf{z,w}=1}&=&\mu_{1}\frac{1}{1-\hat{\mu}_{1}}\tilde{\mu}_{2}\frac{\partial}{\partial w_{1}}\hat{F}_{1}\left(1,1\right)\\
&+&\mu_{1}\tilde{\mu}_{2}\hat{\theta
}_{1}^{(2)}\left(1\right)\frac{\partial}{\partial w_{1}}\hat{F}_{1}\left(1,1\right)\\
&+&\mu_{1}\left(\frac{1}{1-\hat{\mu}_{1}}\right)^2\tilde{\mu}_{2}\hat{f}_{1}\left(1,1\right)
\end{eqnarray*}
%___________________________________________________________________________________________

%___________________________________________________________________________________________
\item \begin{eqnarray*}
\frac{\partial}{\partial w_{1}}\frac{\partial}{\partial z_{1}}\hat{F}_{1}\left(\hat{\theta}_{1}\left(P_{1}\left(z_{1}\right)\tilde{P}_{2}\left(z_{2}\right)
\hat{P}_{2}\left(w_{2}\right)\right),w_{2}\right)|_{\mathbf{z,w}=1}&=&0
\end{eqnarray*}
%___________________________________________________________________________________________

%___________________________________________________________________________________________
\item
\begin{eqnarray*}
\frac{\partial}{\partial w_{2}}\frac{\partial}{\partial z_{1}}\hat{F}_{1}\left(\hat{\theta}_{1}\left(P_{1}\left(z_{1}\right)\tilde{P}_{2}\left(z_{2}\right)
\hat{P}_{2}\left(w_{2}\right)\right),w_{2}\right)|_{\mathbf{z,w}=1}&=&\mu_{1}
\hat{\mu}_{2}\frac{1}{1-\hat{\mu
}_{1}}\frac{\partial}{\partial w_{1}}\hat{F}_{1}\left(1,1\right)+\mu_{1}\hat{\mu}_{2} \hat{\theta
}_{1}^{(2)}\left(1\right)\frac{\partial}{\partial w_{1}}\hat{F}_{1}\left(1,1\right)\\
&+&\mu_{1}\frac{1}{1-\hat{\mu}_{1}}f_{1}\left(1,2\right)+\mu_{1}\hat{\mu}_{2}\left(\frac{1}{1-\hat{\mu}_{1}}\right)^{2}\hat{f}_{1}\left(1,1\right)
\end{eqnarray*}
%___________________________________________________________________________________________


%___________________________________________________________________________________________
%
%\section{Parciales mixtas de $\hat{F}_{1}$ para $z_{2}$}
%___________________________________________________________________________________________
\item
\begin{eqnarray*}
\frac{\partial}{\partial z_{1}}\frac{\partial}{\partial z_{2}}\hat{F}_{1}\left(\hat{\theta}_{1}\left(P_{1}\left(z_{1}\right)\tilde{P}_{2}\left(z_{2}\right)
\hat{P}_{2}\left(w_{2}\right)\right),w_{2}\right)|_{\mathbf{z,w}=1}&=&\mu_{1}\tilde{\mu}_{2}\frac{1}{1-\hat{\mu}_{1}}\frac{\partial}{\partial w_{1}}
\hat{F}_{1}\left(1,1\right)+\mu_{1}\tilde{\mu}_{2}\hat{\theta
}_{1}^{(2)}\left(1\right)\frac{\partial}{\partial w_{1}}\hat{F}_{1}\left(1,1\right)\\
&+&\mu_{1}\tilde{\mu}_{2}\left(\frac{1}{1-\hat{\mu}_{1}}\right)^{2}\hat{f}_{1}\left(1,1\right)
\end{eqnarray*}
%___________________________________________________________________________________________

%___________________________________________________________________________________________
\item
\begin{eqnarray*}
\frac{\partial}{\partial z_{2}}\frac{\partial}{\partial z_{2}}\hat{F}_{1}\left(\hat{\theta}_{1}\left(P_{1}\left(z_{1}\right)\tilde{P}_{2}\left(z_{2}\right)
\hat{P}_{2}\left(w_{2}\right)\right),w_{2}\right)|_{\mathbf{z,w}=1}&=&\tilde{\mu}_{2}^{2}\hat{\theta
}_{1}^{(2)}\left(1\right)\frac{\partial}{\partial w_{1}}\hat{F}_{1}\left(1,1\right)+\frac{1}{1-\hat{\mu}_{1}}\tilde{P}_{2}^{(2)}\frac{\partial}{\partial w_{1}}\hat{F}_{1}\left(1,1\right)\\
&+&\tilde{\mu}_{2}^{2}\left(\frac{1}{1-\hat{\mu}_{1}}\right)^{2}\hat{f}_{1}\left(1,1\right)
\end{eqnarray*}
%___________________________________________________________________________________________

%___________________________________________________________________________________________
\item \begin{eqnarray*}
\frac{\partial}{\partial w_{1}}\frac{\partial}{\partial z_{2}}\hat{F}_{1}\left(\hat{\theta}_{1}\left(P_{1}\left(z_{1}\right)\tilde{P}_{2}\left(z_{2}\right)
\hat{P}_{2}\left(w_{2}\right)\right),w_{2}\right)|_{\mathbf{z,w}=1}&=&0
\end{eqnarray*}
%___________________________________________________________________________________________
%___________________________________________________________________________________________
\item
\begin{eqnarray*}
\frac{\partial}{\partial w_{2}}\frac{\partial}{\partial z_{2}}\hat{F}_{1}\left(\hat{\theta}_{1}\left(P_{1}\left(z_{1}\right)\tilde{P}_{2}\left(z_{2}\right)
\hat{P}_{2}\left(w_{2}\right)\right),w_{2}\right)|_{\mathbf{z,w}=1}&=&\hat{\mu}_{2}\tilde{\mu}_{2}\frac{1}{1-\hat{\mu}_{1}}
\frac{\partial}{\partial w_{1}}\hat{F}_{1}\left(1,1\right)+\hat{\mu}_{2}\tilde{\mu}_{2}\hat{\theta
}_{1}^{(2)}\left(1\right)\frac{\partial}{\partial w_{1}}\hat{F}_{1}\left(1,1\right)\\
&+&\frac{1}{1-\hat{\mu
}_{1}}\tilde{\mu}_{2}\hat{f}_{1}\left(1,2\right)+\tilde{\mu}_{2}\hat{\mu}_{2}\left(\frac{1}{1-\hat{\mu}_{1}}\right)^{2}\hat{f}_{1}\left(1,1\right)
\end{eqnarray*}
%___________________________________________________________________________________________

%\newpage

%___________________________________________________________________________________________
%
%\section{Parciales mixtas de $\hat{F}_{1}$ para $w_{1}$}
%___________________________________________________________________________________________
%___________________________________________________________________________________________
\item \begin{eqnarray*}
\frac{\partial}{\partial z_{1}}\frac{\partial}{\partial w_{1}}\hat{F}_{1}\left(\hat{\theta}_{1}\left(P_{1}\left(z_{1}\right)\tilde{P}_{2}\left(z_{2}\right)
\hat{P}_{2}\left(w_{2}\right)\right),w_{2}\right)|_{\mathbf{z,w}=1}&=&0
\end{eqnarray*}
%___________________________________________________________________________________________

%___________________________________________________________________________________________
\item
\begin{eqnarray*}
\frac{\partial}{\partial z_{2}}\frac{\partial}{\partial w_{1}}\hat{F}_{1}\left(\hat{\theta}_{1}\left(P_{1}\left(z_{1}\right)\tilde{P}_{2}\left(z_{2}\right)
\hat{P}_{2}\left(w_{2}\right)\right),w_{2}\right)|_{\mathbf{z,w}=1}&=&0
\end{eqnarray*}
%___________________________________________________________________________________________

%___________________________________________________________________________________________
\item
\begin{eqnarray*}
\frac{\partial}{\partial w_{1}}\frac{\partial}{\partial w_{1}}\hat{F}_{1}\left(\hat{\theta}_{1}\left(P_{1}\left(z_{1}\right)\tilde{P}_{2}\left(z_{2}\right)
\hat{P}_{2}\left(w_{2}\right)\right),w_{2}\right)|_{\mathbf{z,w}=1}&=&0
\end{eqnarray*}
%___________________________________________________________________________________________

%___________________________________________________________________________________________
\item
\begin{eqnarray*}
\frac{\partial}{\partial w_{2}}\frac{\partial}{\partial w_{1}}\hat{F}_{1}\left(\hat{\theta}_{1}\left(P_{1}\left(z_{1}\right)\tilde{P}_{2}\left(z_{2}\right)
\hat{P}_{2}\left(w_{2}\right)\right),w_{2}\right)|_{\mathbf{z,w}=1}&=&0
\end{eqnarray*}
%___________________________________________________________________________________________


%\newpage
%___________________________________________________________________________________________
%
%\section{Parciales mixtas de $\hat{F}_{1}$ para $w_{2}$}
%___________________________________________________________________________________________
%___________________________________________________________________________________________
\item \begin{eqnarray*}
\frac{\partial}{\partial z_{1}}\frac{\partial}{\partial w_{2}}\hat{F}_{1}\left(\hat{\theta}_{1}\left(P_{1}\left(z_{1}\right)\tilde{P}_{2}\left(z_{2}\right)
\hat{P}_{2}\left(w_{2}\right)\right),w_{2}\right)|_{\mathbf{z,w}=1}&=&\mu_{1}\hat{\mu}_{2}\frac{1}{1-\hat{\mu}_{1}}\frac{\partial}{\partial w_{1}}\hat{F}_{1}\left(1,1\right)+\mu_{1}\hat{\mu}_{2}\hat{\theta
}_{1}^{(2)}\frac{\partial}{\partial w_{1}}\hat{F}_{1}\left(1,1\right)\\
&+&\mu_{1}\frac{1}{1-\hat{\mu}_{1}}\hat{f}_{1}\left(1,2\right)+\mu_{1}\hat{\mu}_{2}\left(\frac{1}{1-\hat{\mu}_{1}}\right)^{2}\hat{f}_1\left(1,1\right)
\end{eqnarray*}
%___________________________________________________________________________________________

%___________________________________________________________________________________________
\begin{eqnarray*}
&&\frac{\partial}{\partial z_{2}}\frac{\partial}{\partial w_{2}}\hat{F}_{1}\left(\hat{\theta}_{1}\left(P_{1}\left(z_{1}\right)\tilde{P}_{2}\left(z_{2}\right)
\hat{P}_{2}\left(w_{2}\right)\right),w_{2}\right)|_{\mathbf{z,w}=1}\\
&=&P_1\left(z_1\right) \hat{P}_2'\left(w_2\right)
\hat{\theta }_1'\left(P_1\left(z_1\right)
\hat{P}_2\left(w_2\right) \tilde{P}_2\left(z_2\right)\right)
\tilde{P}_2'\left(z_2\right)\hat{F}_1^{(1,0)}\left(\hat{\theta }_1\left(P_1\left(z_1\right)
\hat{P}_2\left(w_2\right)
\tilde{P}_2\left(z_2\right)\right),w_2\right)\\
&+&P_1\left(z_1\right)^2
\hat{P}_2\left(w_2\right)\tilde{P}_2\left(z_2\right) \hat{P}_2'\left(w_2\right)
\tilde{P}_2'\left(z_2\right) \hat{\theta
}_1''\left(P_1\left(z_1\right) \hat{P}_2\left(w_2\right)
\tilde{P}_2\left(z_2\right)\right)\hat{F}_1^{(1,0)}\left(\hat{\theta }_1\left(P_1\left(z_1\right) \hat{P}_2\left(w_2\right) \tilde{P}_2\left(z_2\right)\right),w_2\right)\\
&+&P_1\left(z_1\right) \hat{P}_2\left(w_2\right) \hat{\theta
}_1'\left(P_1\left(z_1\right) \hat{P}_2\left(w_2\right)
\tilde{P}_2\left(z_2\right)\right)
\tilde{P}_2'\left(z_2\right)\hat{F}_1^{(1,1)}\left(\hat{\theta }_1\left(P_1\left(z_1\right) \hat{P}_2\left(w_2\right) \tilde{P}_2\left(z_2\right)\right),w_2\right)\\
&+&P_1\left(z_1\right)^2 \hat{P}_2\left(w_2\right)
\tilde{P}_2\left(z_2\right) \hat{P}_2'\left(w_2\right) \hat{\theta
}_1'\left(P_1\left(z_1\right)
\hat{P}_2\left(w_2\right) \tilde{P}_2\left(z_2\right)\right)^2\tilde{P}_2'\left(z_2\right) \hat{F}_1^{(2,0)}\left(\hat{\theta
}_1\left(P_1\left(z_1\right) \hat{P}_2\left(w_2\right)
\tilde{P}_2\left(z_2\right)\right),w_2\right)
\end{eqnarray*}
%___________________________________________________________________________________________

%___________________________________________________________________________________________
\begin{eqnarray*}
\frac{\partial}{\partial w_{1}}\frac{\partial}{\partial w_{2}}\hat{F}_{1}\left(\hat{\theta}_{1}\left(P_{1}\left(z_{1}\right)\tilde{P}_{2}\left(z_{2}\right)
\hat{P}_{2}\left(w_{2}\right)\right),w_{2}\right)|_{\mathbf{z,w}=1}&=&0
\end{eqnarray*}
%___________________________________________________________________________________________

%___________________________________________________________________________________________
\begin{eqnarray*}
&&\frac{\partial}{\partial w_{2}}\frac{\partial}{\partial w_{2}}\hat{F}_{1}\left(\hat{\theta}_{1}\left(P_{1}\left(z_{1}\right)\tilde{P}_{2}\left(z_{2}\right)
\hat{P}_{2}\left(w_{2}\right)\right),w_{2}\right)|_{\mathbf{z,w}=1}\\
&=&\hat{F}_1^{(0,2)}\left(\hat{\theta }_1\left(P_1\left(z_1\right) \hat{P}_2\left(w_2\right) \tilde{P}_2\left(z_2\right)\right),w_2\right)\\
&+&P_1\left(z_1\right) \tilde{P}_2\left(z_2\right) \hat{\theta
}_1'\left(P_1\left(z_1\right) \hat{P}_2\left(w_2\right)
\tilde{P}_2\left(z_2\right)\right)\hat{P}_2''\left(w_2\right) \hat{F}_1^{(1,0)}\left(\hat{\theta }_1\left(P_1\left(z_1\right) \hat{P}_2\left(w_2\right) \tilde{P}_2\left(z_2\right)\right),w_2\right)\\
&+&P_1\left(z_1\right)^2 \tilde{P}_2\left(z_2\right)^2
\hat{P}_2'\left(w_2\right)^2 \hat{\theta
}_1''\left(P_1\left(z_1\right) \hat{P}_2\left(w_2\right)
\tilde{P}_2\left(z_2\right)\right)\hat{F}_1^{(1,0)}\left(\hat{\theta }_1\left(P_1\left(z_1\right) \hat{P}_2\left(w_2\right) \tilde{P}_2\left(z_2\right)\right),w_2\right)\\
&+&P_1\left(z_1\right) \tilde{P}_2\left(z_2\right)
\hat{P}_2'\left(w_2\right) \hat{\theta
}_1'\left(P_1\left(z_1\right) \hat{P}_2\left(w_2\right)
\tilde{P}_2\left(z_2\right)\right)\\
&+&P_1\left(z_1\right) \tilde{P}_2\left(z_2\right)
\hat{P}_2'\left(w_2\right) \hat{\theta
}_1'\left(P_1\left(z_1\right) \hat{P}_2\left(w_2\right)
\tilde{P}_2\left(z_2\right)\right)\hat{F}_1^{(1,1)}\left(\hat{\theta }_1\left(P_1\left(z_1\right) \hat{P}_2\left(w_2\right) \tilde{P}_2\left(z_2\right)\right),w_2\right)\\
&+&P_1\left(z_1\right) \tilde{P}_2\left(z_2\right)
\hat{P}_2'\left(w_2\right) \hat{\theta
}_1'\left(P_1\left(z_1\right) \hat{P}_2\left(w_2\right)
\tilde{P}_2\left(z_2\right)\right)
P_1\left(z_1\right) \tilde{P}_2\left(z_2\right)
\hat{P}_2'\left(w_2\right) \hat{\theta
}_1'\left(P_1\left(z_1\right) \hat{P}_2\left(w_2\right)
\tilde{P}_2\left(z_2\right)\right)
\\
&&\left.\hat{F}_1^{(2,0)}\left(\hat{\theta
}_1\left(P_1\left(z_1\right) \hat{P}_2\left(w_2\right)
\tilde{P}_2\left(z_2\right)\right),w_2\right)\right)
\end{eqnarray*}
%___________________________________________________________________________________________


%___________________________________________________________________________________________
%
%\section{Parciales mixtas de $\hat{F}_{2}$ para $z_{1}$}
%___________________________________________________________________________________________
%___________________________________________________________________________________________
\begin{eqnarray*}
&&\frac{\partial}{\partial z_{1}}\frac{\partial}{\partial z_{1}}\hat{F}_{2}\left(w_{1},\hat{\theta}_{2}\left(P_{1}\left(z_{1}\right)\tilde{P}_{2}\left(z_{2}\right)
\hat{P}_{1}\left(w_{1}\right)\right)\right)|_{\mathbf{z,w}=1}\\
&=&P_1\left(w_1\right) \tilde{P}_2\left(z_2\right)
\hat{\theta }_2'\left(P_1\left(w_1\right) P_1\left(z_1\right)
\tilde{P}_2\left(z_2\right)\right)P_1''\left(z_1\right) \hat{F}_2^{(0,1)}\left(w_1,\hat{\theta }_2\left(P_1\left(w_1\right) P_1\left(z_1\right) \tilde{P}_2\left(z_2\right)\right)\right)\\
&+&P_1\left(w_1\right)^2 \tilde{P}_2\left(z_2\right)^2
P_1'\left(z_1\right)^2 \hat{\theta }_2''\left(P_1\left(w_1\right)
P_1\left(z_1\right) \tilde{P}_2\left(z_2\right)\right)\hat{F}_2^{(0,1)}\left(w_1,\hat{\theta }_2\left(P_1\left(w_1\right) P_1\left(z_1\right) \tilde{P}_2\left(z_2\right)\right)\right)\\
&+&P_1\left(w_1\right)^2 \tilde{P}_2\left(z_2\right)^2
P_1'\left(z_1\right)^2 \hat{\theta }_2'\left(P_1\left(w_1\right)
P_1\left(z_1\right) \tilde{P}_2\left(z_2\right)\right)^2\hat{F}_2^{(0,2)}\left(w_1,\hat{\theta
}_2\left(P_1\left(w_1\right) P_1\left(z_1\right)
\tilde{P}_2\left(z_2\right)\right)\right)
\end{eqnarray*}
%___________________________________________________________________________________________


%___________________________________________________________________________________________
\begin{eqnarray*}
&&\frac{\partial}{\partial z_{2}}\frac{\partial}{\partial z_{1}}\hat{F}_{2}\left(w_{1},\hat{\theta}_{2}\left(P_{1}\left(z_{1}\right)\tilde{P}_{2}\left(z_{2}\right)
\hat{P}_{1}\left(w_{1}\right)\right)\right)|_{\mathbf{z,w}=1}\\
&=&P_1\left(w_1\right) P_1'\left(z_1\right) \hat{\theta
}_2'\left(P_1\left(w_1\right) P_1\left(z_1\right)
\tilde{P}_2\left(z_2\right)\right)
\tilde{P}_2'\left(z_2\right)\hat{F}_2^{(0,1)}\left(w_1,\hat{\theta
}_2\left(P_1\left(w_1\right) P_1\left(z_1\right)
\tilde{P}_2\left(z_2\right)\right)\right)\\
&+&P_1\left(w_1\right)^2 P_1\left(z_1\right)\tilde{P}_2\left(z_2\right) P_1'\left(z_1\right)\tilde{P}_2'\left(z_2\right) \hat{\theta
}_2''\left(P_1\left(w_1\right) P_1\left(z_1\right)
\tilde{P}_2\left(z_2\right)\right)\hat{F}_2^{(0,1)}\left(w_1,\hat{\theta }_2\left(P_1\left(w_1\right) P_1\left(z_1\right) \tilde{P}_2\left(z_2\right)\right)\right)\\
&+&P_1\left(w_1\right)^2 P_1\left(z_1\right)
\tilde{P}_2\left(z_2\right) P_1'\left(z_1\right) \hat{\theta
}_2'\left(P_1\left(w_1\right) P_1\left(z_1\right)
\tilde{P}_2\left(z_2\right)\right)^2 \tilde{P}_2'\left(z_2\right)
\hat{F}_2^{(0,2)}\left(w_1,\hat{\theta
}_2\left(P_1\left(w_1\right) P_1\left(z_1\right)
\tilde{P}_2\left(z_2\right)\right)\right)
\end{eqnarray*}
%___________________________________________________________________________________________

%___________________________________________________________________________________________
\begin{eqnarray*}
&&\frac{\partial}{\partial w_{1}}\frac{\partial}{\partial z_{1}}\hat{F}_{2}\left(w_{1},\hat{\theta}_{2}\left(P_{1}\left(z_{1}\right)\tilde{P}_{2}\left(z_{2}\right)
\hat{P}_{1}\left(w_{1}\right)\right)\right)|_{\mathbf{z,w}=1}\\
&=&\tilde{P}_2\left(z_2\right) P_1'\left(w_1\right)
P_1'\left(z_1\right) \hat{\theta }_2'\left(P_1\left(w_1\right)
P_1\left(z_1\right) \tilde{P}_2\left(z_2\right)\right)\hat{F}_2^{(0,1)}\left(w_1,\hat{\theta
}_2\left(P_1\left(w_1\right) P_1\left(z_1\right)
\tilde{P}_2\left(z_2\right)\right)\right)\\
&+&P_1\left(w_1\right)P_1\left(z_1\right)\tilde{P}_2\left(z_2\right)^2 P_1'\left(w_1\right)P_1'\left(z_1\right) \hat{\theta }_2''\left(P_1\left(w_1\right)P_1\left(z_1\right) \tilde{P}_2\left(z_2\right)\right)\hat{F}_2^{(0,1)}\left(w_1,\hat{\theta }_2\left(P_1\left(w_1\right) P_1\left(z_1\right) \tilde{P}_2\left(z_2\right)\right)\right)\\
&+&P_1\left(w_1\right) \tilde{P}_2\left(z_2\right)
P_1'\left(z_1\right) \hat{\theta }_2'\left(P_1\left(w_1\right)
P_1\left(z_1\right) \tilde{P}_2\left(z_2\right)\right)P_1\left(z_1\right) \tilde{P}_2\left(z_2\right)
P_1'\left(w_1\right) \hat{\theta }_2'\left(P_1\left(w_1\right)
P_1\left(z_1\right) \tilde{P}_2\left(z_2\right)\right)\\
&&\hat{F}_2^{(0,2)}\left(w_1,\hat{\theta }_2\left(P_1\left(w_1\right) P_1\left(z_1\right) \tilde{P}_2\left(z_2\right)\right)\right)\\
&+&P_1\left(w_1\right) \tilde{P}_2\left(z_2\right)
P_1'\left(z_1\right) \hat{\theta }_2'\left(P_1\left(w_1\right)
P_1\left(z_1\right) \tilde{P}_2\left(z_2\right)\right)\hat{F}_2^{(1,1)}\left(w_1,\hat{\theta
}_2\left(P_1\left(w_1\right) P_1\left(z_1\right)
\tilde{P}_2\left(z_2\right)\right)\right)
\end{eqnarray*}
%___________________________________________________________________________________________


%___________________________________________________________________________________________
\begin{eqnarray*}
\frac{\partial}{\partial w_{2}}\frac{\partial}{\partial z_{1}}\hat{F}_{2}\left(w_{1},\hat{\theta}_{2}\left(P_{1}\left(z_{1}\right)\tilde{P}_{2}\left(z_{2}\right)
\hat{P}_{1}\left(w_{1}\right)\right)\right)|_{\mathbf{z,w}=1}&=&0
\end{eqnarray*}
%___________________________________________________________________________________________

%___________________________________________________________________________________________
%
%\section{Parciales mixtas de $\hat{F}_{2}$ para $z_{2}$}
%___________________________________________________________________________________________
%___________________________________________________________________________________________
\begin{eqnarray*}
&&\frac{\partial}{\partial z_{1}}\frac{\partial}{\partial z_{2}}\hat{F}_{2}\left(w_{1},\hat{\theta}_{2}\left(P_{1}\left(z_{1}\right)\tilde{P}_{2}\left(z_{2}\right)
\hat{P}_{1}\left(w_{1}\right)\right)\right)|_{\mathbf{z,w}=1}\\
&=&P_1\left(w_1\right) P_1'\left(z_1\right) \hat{\theta
}_2'\left(P_1\left(w_1\right) P_1\left(z_1\right)
\tilde{P}_2\left(z_2\right)\right)
\tilde{P}_2'\left(z_2\right)\hat{F}_2^{(0,1)}\left(w_1,\hat{\theta
}_2\left(P_1\left(w_1\right) P_1\left(z_1\right)
\tilde{P}_2\left(z_2\right)\right)\right)\\
&+&P_1\left(w_1\right)^2
P_1\left(z_1\right)\tilde{P}_2\left(z_2\right) P_1'\left(z_1\right)
\tilde{P}_2'\left(z_2\right) \hat{\theta
}_2''\left(P_1\left(w_1\right) P_1\left(z_1\right)
\tilde{P}_2\left(z_2\right)\right)\hat{F}_2^{(0,1)}\left(w_1,\hat{\theta }_2\left(P_1\left(w_1\right) P_1\left(z_1\right) \tilde{P}_2\left(z_2\right)\right)\right)\\
&+&P_1\left(w_1\right)^2 P_1\left(z_1\right)
\tilde{P}_2\left(z_2\right) P_1'\left(z_1\right) \hat{\theta
}_2'\left(P_1\left(w_1\right) P_1\left(z_1\right)
\tilde{P}_2\left(z_2\right)\right)^2\tilde{P}_2'\left(z_2\right)
\hat{F}_2^{(0,2)}\left(w_1,\hat{\theta
}_2\left(P_1\left(w_1\right) P_1\left(z_1\right)
\tilde{P}_2\left(z_2\right)\right)\right)
\end{eqnarray*}
%___________________________________________________________________________________________

%___________________________________________________________________________________________
\begin{eqnarray*}
&&\frac{\partial}{\partial z_{2}}\frac{\partial}{\partial z_{2}}\hat{F}_{2}\left(w_{1},\hat{\theta}_{2}\left(P_{1}\left(z_{1}\right)\tilde{P}_{2}\left(z_{2}\right)
\hat{P}_{1}\left(w_{1}\right)\right)\right)|_{\mathbf{z,w}=1}\\
&=&P_1\left(w_1\right)^2 P_1\left(z_1\right)^2
\tilde{P}_2'\left(z_2\right)^2 \hat{\theta
}_2''\left(P_1\left(w_1\right) P_1\left(z_1\right)
\tilde{P}_2\left(z_2\right)\right)\hat{F}_2^{(0,1)}\left(w_1,\hat{\theta }_2\left(P_1\left(w_1\right) P_1\left(z_1\right) \tilde{P}_2\left(z_2\right)\right)\right)\\
&+&P_1\left(w_1\right) P_1\left(z_1\right) \hat{\theta
}_2'\left(P_1\left(w_1\right) P_1\left(z_1\right)
\tilde{P}_2\left(z_2\right)\right) \tilde{P}_2''\left(z_2\right)\hat{F}_2^{(0,1)}\left(w_1,\hat{\theta }_2\left(P_1\left(w_1\right) P_1\left(z_1\right) \tilde{P}_2\left(z_2\right)\right)\right)\\
&+&P_1\left(w_1\right)^2 P_1\left(z_1\right)^2 \hat{\theta }_2'\left(P_1\left(w_1\right) P_1\left(z_1\right) \tilde{P}_2\left(z_2\right)\right)^2\tilde{P}_2'\left(z_2\right)^2
\hat{F}_2^{(0,2)}\left(w_1,\hat{\theta
}_2\left(P_1\left(w_1\right) P_1\left(z_1\right)
\tilde{P}_2\left(z_2\right)\right)\right)
\end{eqnarray*}
%___________________________________________________________________________________________

%___________________________________________________________________________________________
\begin{eqnarray*}
&&\frac{\partial}{\partial w_{1}}\frac{\partial}{\partial z_{2}}\hat{F}_{2}\left(w_{1},\hat{\theta}_{2}\left(P_{1}\left(z_{1}\right)\tilde{P}_{2}\left(z_{2}\right)
\hat{P}_{1}\left(w_{1}\right)\right)\right)|_{\mathbf{z,w}=1}\\
&=&P_1\left(z_1\right) P_1'\left(w_1\right) \hat{\theta
}_2'\left(P_1\left(w_1\right) P_1\left(z_1\right)
\tilde{P}_2\left(z_2\right)\right)
\tilde{P}_2'\left(z_2\right)\hat{F}_2^{(0,1)}\left(w_1,\hat{\theta
}_2\left(P_1\left(w_1\right) P_1\left(z_1\right)
\tilde{P}_2\left(z_2\right)\right)\right)\\
&+&P_1\left(w_1\right)P_1\left(z_1\right)^2\tilde{P}_2\left(z_2\right) P_1'\left(w_1\right)\tilde{P}_2'\left(z_2\right) \hat{\theta
}_2''\left(P_1\left(w_1\right) P_1\left(z_1\right)
\tilde{P}_2\left(z_2\right)\right)\hat{F}_2^{(0,1)}\left(w_1,\hat{\theta }_2\left(P_1\left(w_1\right) P_1\left(z_1\right) \tilde{P}_2\left(z_2\right)\right)\right)\\
&+&P_1\left(w_1\right) P_1\left(z_1\right) \hat{\theta
}_2'\left(P_1\left(w_1\right) P_1\left(z_1\right)
\tilde{P}_2\left(z_2\right)\right) \tilde{P}_2'\left(z_2\right)P_1\left(z_1\right) \tilde{P}_2\left(z_2\right)
P_1'\left(w_1\right) \hat{\theta }_2'\left(P_1\left(w_1\right)
P_1\left(z_1\right) \tilde{P}_2\left(z_2\right)\right)\\
&&\hat{F}_2^{(0,2)}\left(w_1,\hat{\theta }_2\left(P_1\left(w_1\right) P_1\left(z_1\right) \tilde{P}_2\left(z_2\right)\right)\right)\\
&+&P_1\left(w_1\right) P_1\left(z_1\right) \hat{\theta
}_2'\left(P_1\left(w_1\right) P_1\left(z_1\right)
\tilde{P}_2\left(z_2\right)\right) \tilde{P}_2'\left(z_2\right)
\hat{F}_2^{(1,1)}\left(w_1,\hat{\theta
}_2\left(P_1\left(w_1\right) P_1\left(z_1\right)
\tilde{P}_2\left(z_2\right)\right)\right)
\end{eqnarray*}
%___________________________________________________________________________________________

%___________________________________________________________________________________________
\begin{eqnarray*}
\frac{\partial}{\partial w_{2}}\frac{\partial}{\partial z_{2}}\hat{F}_{2}\left(w_{1},\hat{\theta}_{2}\left(P_{1}\left(z_{1}\right)\tilde{P}_{2}\left(z_{2}\right)
\hat{P}_{1}\left(w_{1}\right)\right)\right)|_{\mathbf{z,w}=1}&=&0
\end{eqnarray*}
%___________________________________________________________________________________________


%___________________________________________________________________________________________
%
%\section{Parciales mixtas de $\hat{F}_{2}$ para $w_{1}$}
%___________________________________________________________________________________________
%___________________________________________________________________________________________
\begin{eqnarray*}
&&\frac{\partial}{\partial z_{1}}\frac{\partial}{\partial w_{1}}\hat{F}_{2}\left(w_{1},\hat{\theta}_{2}\left(P_{1}\left(z_{1}\right)\tilde{P}_{2}\left(z_{2}\right)
\hat{P}_{1}\left(w_{1}\right)\right)\right)|_{\mathbf{z,w}=1}\\
&=&\tilde{P}_2\left(z_2\right) P_1'\left(w_1\right)
P_1'\left(z_1\right) \hat{\theta }_2'\left(P_1\left(w_1\right)
P_1\left(z_1\right) \tilde{P}_2\left(z_2\right)\right)\hat{F}_2^{(0,1)}\left(w_1,\hat{\theta
}_2\left(P_1\left(w_1\right) P_1\left(z_1\right)
\tilde{P}_2\left(z_2\right)\right)\right)\\
&+&P_1\left(w_1\right)P_1\left(z_1\right)
\tilde{P}_2\left(z_2\right)^2 P_1'\left(w_1\right)
P_1'\left(z_1\right) \hat{\theta }_2''\left(P_1\left(w_1\right)
P_1\left(z_1\right) \tilde{P}_2\left(z_2\right)\right)\hat{F}_2^{(0,1)}\left(w_1,\hat{\theta
}_2\left(P_1\left(w_1\right) P_1\left(z_1\right)
\tilde{P}_2\left(z_2\right)\right)\right)\\
&+&P_1\left(w_1\right)P_1\left(z_1\right)
\tilde{P}_2\left(z_2\right)^2 P_1'\left(w_1\right)
P_1'\left(z_1\right) \hat{\theta }_2'\left(P_1\left(w_1\right)
P_1\left(z_1\right) \tilde{P}_2\left(z_2\right)\right)^2\hat{F}_2^{(0,2)}\left(w_1,\hat{\theta }_2\left(P_1\left(w_1\right) P_1\left(z_1\right) \tilde{P}_2\left(z_2\right)\right)\right)\\
&+&P_1\left(w_1\right) \tilde{P}_2\left(z_2\right)
P_1'\left(z_1\right) \hat{\theta }_2'\left(P_1\left(w_1\right)
P_1\left(z_1\right) \tilde{P}_2\left(z_2\right)\right)\hat{F}_2^{(1,1)}\left(w_1,\hat{\theta
}_2\left(P_1\left(w_1\right) P_1\left(z_1\right)
\tilde{P}_2\left(z_2\right)\right)\right)
\end{eqnarray*}
%___________________________________________________________________________________________

%___________________________________________________________________________________________
\begin{eqnarray*}
&&\frac{\partial}{\partial z_{2}}\frac{\partial}{\partial w_{1}}\hat{F}_{2}\left(w_{1},\hat{\theta}_{2}\left(P_{1}\left(z_{1}\right)\tilde{P}_{2}\left(z_{2}\right)
\hat{P}_{1}\left(w_{1}\right)\right)\right)|_{\mathbf{z,w}=1}\\
&=&P_1\left(z_1\right) P_1'\left(w_1\right) \hat{\theta
}_2'\left(P_1\left(w_1\right) P_1\left(z_1\right)
\tilde{P}_2\left(z_2\right)\right)
\tilde{P}_2'\left(z_2\right)\hat{F}_2^{(0,1)}\left(w_1,\hat{\theta
}_2\left(P_1\left(w_1\right) P_1\left(z_1\right)
\tilde{P}_2\left(z_2\right)\right)\right)\\
&+&P_1\left(w_1\right)P_1\left(z_1\right)^2
\tilde{P}_2\left(z_2\right) P_1'\left(w_1\right)
\tilde{P}_2'\left(z_2\right) \hat{\theta
}_2''\left(P_1\left(w_1\right) P_1\left(z_1\right)
\tilde{P}_2\left(z_2\right)\right)\hat{F}_2^{(0,1)}\left(w_1,\hat{\theta }_2\left(P_1\left(w_1\right) P_1\left(z_1\right) \tilde{P}_2\left(z_2\right)\right)\right)\\
&+&P_1\left(w_1\right) P_1\left(z_1\right)^2
\tilde{P}_2\left(z_2\right) P_1'\left(w_1\right) \hat{\theta
}_2'\left(P_1\left(w_1\right) P_1\left(z_1\right)
\tilde{P}_2\left(z_2\right)\right)^2 \tilde{P}_2'\left(z_2\right) \hat{F}_2^{(0,2)}\left(w_1,\hat{\theta }_2\left(P_1\left(w_1\right) P_1\left(z_1\right) \tilde{P}_2\left(z_2\right)\right)\right)\\
&+&P_1\left(w_1\right) P_1\left(z_1\right) \hat{\theta
}_2'\left(P_1\left(w_1\right) P_1\left(z_1\right)
\tilde{P}_2\left(z_2\right)\right) \tilde{P}_2'\left(z_2\right)\hat{F}_2^{(1,1)}\left(w_1,\hat{\theta
}_2\left(P_1\left(w_1\right) P_1\left(z_1\right)
\tilde{P}_2\left(z_2\right)\right)\right)
\end{eqnarray*}
%___________________________________________________________________________________________

\begin{eqnarray*}
&&\frac{\partial}{\partial w_{1}}\frac{\partial}{\partial w_{1}}\hat{F}_{2}\left(w_{1},\hat{\theta}_{2}\left(P_{1}\left(z_{1}\right)\tilde{P}_{2}\left(z_{2}\right)
\hat{P}_{1}\left(w_{1}\right)\right)\right)|_{\mathbf{z,w}=1}\\
&=&P_1\left(z_1\right) \tilde{P}_2\left(z_2\right)
\hat{\theta }_2'\left(P_1\left(w_1\right) P_1\left(z_1\right)
\tilde{P}_2\left(z_2\right)\right)P_1''\left(w_1\right) \hat{F}_2^{(0,1)}\left(w_1,\hat{\theta }_2\left(P_1\left(w_1\right) P_1\left(z_1\right) \tilde{P}_2\left(z_2\right)\right)\right)\\
&+&P_1\left(z_1\right)^2 \tilde{P}_2\left(z_2\right)^2
P_1'\left(w_1\right)^2 \hat{\theta }_2''\left(P_1\left(w_1\right)
P_1\left(z_1\right) \tilde{P}_2\left(z_2\right)\right)\hat{F}_2^{(0,1)}\left(w_1,\hat{\theta }_2\left(P_1\left(w_1\right) P_1\left(z_1\right) \tilde{P}_2\left(z_2\right)\right)\right)\\
&+&P_1\left(z_1\right) \tilde{P}_2\left(z_2\right)
P_1'\left(w_1\right) \hat{\theta }_2'\left(P_1\left(w_1\right)
P_1\left(z_1\right) \tilde{P}_2\left(z_2\right)\right)\hat{F}_2^{(1,1)}\left(w_1,\hat{\theta }_2\left(P_1\left(w_1\right) P_1\left(z_1\right) \tilde{P}_2\left(z_2\right)\right)\right)\\
&+&P_1\left(z_1\right) \tilde{P}_2\left(z_2\right)
P_1'\left(w_1\right) \hat{\theta }_2'\left(P_1\left(w_1\right)
P_1\left(z_1\right) \tilde{P}_2\left(z_2\right)\right)P_1\left(z_1\right) \tilde{P}_2\left(z_2\right)
P_1'\left(w_1\right) \hat{\theta }_2'\left(P_1\left(w_1\right)
P_1\left(z_1\right) \tilde{P}_2\left(z_2\right)\right)\\
&&\hat{F}_2^{(0,2)}\left(w_1,\hat{\theta }_2\left(P_1\left(w_1\right) P_1\left(z_1\right) \tilde{P}_2\left(z_2\right)\right)\right)\\
&+&P_1\left(z_1\right) \tilde{P}_2\left(z_2\right)
P_1'\left(w_1\right) \hat{\theta }_2'\left(P_1\left(w_1\right)
P_1\left(z_1\right) \tilde{P}_2\left(z_2\right)\right)\hat{F}_2^{(1,1)}\left(w_1,\hat{\theta }_2\left(P_1\left(w_1\right) P_1\left(z_1\right) \tilde{P}_2\left(z_2\right)\right)\right)\\
&+&\hat{F}_2^{(2,0)}\left(w_1,\hat{\theta
}_2\left(P_1\left(w_1\right) P_1\left(z_1\right)
\tilde{P}_2\left(z_2\right)\right)\right)
\end{eqnarray*}



\begin{eqnarray*}
\frac{\partial}{\partial w_{2}}\frac{\partial}{\partial w_{1}}\hat{F}_{2}\left(w_{1},\hat{\theta}_{2}\left(P_{1}\left(z_{1}\right)\tilde{P}_{2}\left(z_{2}\right)
\hat{P}_{1}\left(w_{1}\right)\right)\right)|_{\mathbf{z,w}=1}&=&0
\end{eqnarray*}

%___________________________________________________________________________________________
%
%\section{Parciales mixtas de $\hat{F}_{2}$ para $w_{2}$}
%___________________________________________________________________________________________
\begin{eqnarray*}
\frac{\partial}{\partial z_{1}}\frac{\partial}{\partial w_{2}}\hat{F}_{2}\left(w_{1},\hat{\theta}_{2}\left(P_{1}\left(z_{1}\right)\tilde{P}_{2}\left(z_{2}\right)
\hat{P}_{1}\left(w_{1}\right)\right)\right)|_{\mathbf{z,w}=1}&=&0
\end{eqnarray*}

%___________________________________________________________________________________________
\begin{eqnarray*}
\frac{\partial}{\partial z_{2}}\frac{\partial}{\partial w_{2}}\hat{F}_{2}\left(w_{1},\hat{\theta}_{2}\left(P_{1}\left(z_{1}\right)\tilde{P}_{2}\left(z_{2}\right)
\hat{P}_{1}\left(w_{1}\right)\right)\right)|_{\mathbf{z,w}=1}&=&0
\end{eqnarray*}

%___________________________________________________________________________________________

%___________________________________________________________________________________________
\begin{eqnarray*}
\frac{\partial}{\partial w_{1}}\frac{\partial}{\partial w_{2}}\hat{F}_{2}\left(w_{1},\hat{\theta}_{2}\left(P_{1}\left(z_{1}\right)\tilde{P}_{2}\left(z_{2}\right)
\hat{P}_{1}\left(w_{1}\right)\right)\right)|_{\mathbf{z,w}=1}&=&0
\end{eqnarray*}

%___________________________________________________________________________________________

%___________________________________________________________________________________________
\begin{eqnarray*}
\frac{\partial}{\partial w_{2}}\frac{\partial}{\partial w_{2}}\hat{F}_{2}\left(w_{1},\hat{\theta}_{2}\left(P_{1}\left(z_{1}\right)\tilde{P}_{2}\left(z_{2}\right)
\hat{P}_{1}\left(w_{1}\right)\right)\right)|_{\mathbf{z,w}=1}&=&0
\end{eqnarray*}
\end{enumerate}




%___________________________________________________________________________________________
%
%\subsection{Derivadas de Segundo Orden para $F_{1}$}
%___________________________________________________________________________________________

%\subsubsection{Mixtas para $z_{1}$:}
%___________________________________________________________________________________________
\begin{enumerate}

%1/1/1
\item \begin{eqnarray*}
&&\frac{\partial}{\partial z_1}\frac{\partial}{\partial z_1}\left(R_2\left(P_1\left(z_1\right)\bar{P}_2\left(z_2\right)\hat{P}_1\left(w_1\right)\hat{P}_2\left(w_2\right)\right)F_2\left(z_1,\theta
_2\left(P_1\left(z_1\right)\hat{P}_1\left(w_1\right)\hat{P}_2\left(w_2\right)\right)\right)\hat{F}_2\left(w_1,w_2\right)\right)\\
&=&r_{2}P_{1}^{(2)}\left(1\right)+\mu_{1}^{2}R_{2}^{(2)}\left(1\right)+2\mu_{1}r_{2}\left(\frac{\mu_{1}}{1-\tilde{\mu}_{2}}F_{2}^{(0,1)}+F_{2}^{1,0)}\right)+\frac{1}{1-\tilde{\mu}_{2}}P_{1}^{(2)}F_{2}^{(0,1)}+\mu_{1}^{2}\tilde{\theta}_{2}^{(2)}\left(1\right)F_{2}^{(0,1)}\\
&+&\frac{\mu_{1}}{1-\tilde{\mu}_{2}}F_{2}^{(1,1)}+\frac{\mu_{1}}{1-\tilde{\mu}_{2}}\left(\frac{\mu_{1}}{1-\tilde{\mu}_{2}}F_{2}^{(0,2)}+F_{2}^{(1,1)}\right)+F_{2}^{(2,0)}.
\end{eqnarray*}

%2/2/1

\item \begin{eqnarray*}
&&\frac{\partial}{\partial z_2}\frac{\partial}{\partial z_1}\left(R_2\left(P_1\left(z_1\right)\bar{P}_2\left(z_2\right)\hat{P}_1\left(w_1\right)\hat{P}_2\left(w_2\right)\right)F_2\left(z_1,\theta
_2\left(P_1\left(z_1\right)\hat{P}_1\left(w_1\right)\hat{P}_2\left(w_2\right)\right)\right)\hat{F}_2\left(w_1,w_2\right)\right)\\
&=&\mu_{1}r_{2}\tilde{\mu}_{2}+\mu_{1}\tilde{\mu}_{2}R_{2}^{(2)}\left(1\right)+r_{2}\tilde{\mu}_{2}\left(\frac{\mu_{1}}{1-\tilde{\mu}_{2}}F_{2}^{(0,1)}+F_{2}^{(1,0)}\right).
\end{eqnarray*}
%3/3/1
\item \begin{eqnarray*}
&&\frac{\partial}{\partial w_1}\frac{\partial}{\partial z_1}\left(R_2\left(P_1\left(z_1\right)\bar{P}_2\left(z_2\right)\hat{P}_1\left(w_1\right)\hat{P}_2\left(w_2\right)\right)F_2\left(z_1,\theta
_2\left(P_1\left(z_1\right)\hat{P}_1\left(w_1\right)\hat{P}_2\left(w_2\right)\right)\right)\hat{F}_2\left(w_1,w_2\right)\right)\\
&=&\mu_{1}\hat{\mu}_{1}r_{2}+\mu_{1}\hat{\mu}_{1}R_{2}^{(2)}\left(1\right)+r_{2}\frac{\mu_{1}}{1-\tilde{\mu}_{2}}F_{2}^{(0,1)}+r_{2}\hat{\mu}_{1}\left(\frac{\mu_{1}}{1-\tilde{\mu}_{2}}F_{2}^{(0,1)}+F_{2}^{(1,0)}\right)+\mu_{1}r_{2}\hat{F}_{2}^{(1,0)}\\
&+&\left(\frac{\mu_{1}}{1-\tilde{\mu}_{2}}F_{2}^{(0,1)}+F_{2}^{(1,0)}\right)\hat{F}_{2}^{(1,0)}+\frac{\mu_{1}\hat{\mu}_{1}}{1-\tilde{\mu}_{2}}F_{2}^{(0,1)}+\mu_{1}\hat{\mu}_{1}\tilde{\theta}_{2}^{(2)}\left(1\right)F_{2}^{(0,1)}\\
&+&\mu_{1}\hat{\mu}_{1}\left(\frac{1}{1-\tilde{\mu}_{2}}\right)^{2}F_{2}^{(0,2)}+\frac{\hat{\mu}_{1}}{1-\tilde{\mu}_{2}}F_{2}^{(1,1)}.
\end{eqnarray*}
%4/4/1
\item \begin{eqnarray*}
&&\frac{\partial}{\partial w_2}\frac{\partial}{\partial z_1}\left(R_2\left(P_1\left(z_1\right)\bar{P}_2\left(z_2\right)\hat{P}_1\left(w_1\right)\hat{P}_2\left(w_2\right)\right)
F_2\left(z_1,\theta_2\left(P_1\left(z_1\right)\hat{P}_1\left(w_1\right)\hat{P}_2\left(w_2\right)\right)\right)\hat{F}_2\left(w_1,w_2\right)\right)\\
&=&\mu_{1}\hat{\mu}_{2}r_{2}+\mu_{1}\hat{\mu}_{2}R_{2}^{(2)}\left(1\right)+r_{2}\frac{\mu_{1}\hat{\mu}_{2}}{1-\tilde{\mu}_{2}}F_{2}^{(0,1)}+\mu_{1}r_{2}\hat{F}_{2}^{(0,1)}
+r_{2}\hat{\mu}_{2}\left(\frac{\mu_{1}}{1-\tilde{\mu}_{2}}F_{2}^{(0,1)}+F_{2}^{(1,0)}\right)\\
&+&\hat{F}_{2}^{(1,0)}\left(\frac{\mu_{1}}{1-\tilde{\mu}_{2}}F_{2}^{(0,1)}+F_{2}^{(1,0)}\right)+\frac{\mu_{1}\hat{\mu}_{2}}{1-\tilde{\mu}_{2}}F_{2}^{(0,1)}
+\mu_{1}\hat{\mu}_{2}\tilde{\theta}_{2}^{(2)}\left(1\right)F_{2}^{(0,1)}+\mu_{1}\hat{\mu}_{2}\left(\frac{1}{1-\tilde{\mu}_{2}}\right)^{2}F_{2}^{(0,2)}\\
&+&\frac{\hat{\mu}_{2}}{1-\tilde{\mu}_{2}}F_{2}^{(1,1)}.
\end{eqnarray*}
%___________________________________________________________________________________________
%\subsubsection{Mixtas para $z_{2}$:}
%___________________________________________________________________________________________
%5
\item \begin{eqnarray*} &&\frac{\partial}{\partial
z_1}\frac{\partial}{\partial
z_2}\left(R_2\left(P_1\left(z_1\right)\bar{P}_2\left(z_2\right)\hat{P}_1\left(w_1\right)\hat{P}_2\left(w_2\right)\right)
F_2\left(z_1,\theta_2\left(P_1\left(z_1\right)\hat{P}_1\left(w_1\right)\hat{P}_2\left(w_2\right)\right)\right)\hat{F}_2\left(w_1,w_2\right)\right)\\
&=&\mu_{1}\tilde{\mu}_{2}r_{2}+\mu_{1}\tilde{\mu}_{2}R_{2}^{(2)}\left(1\right)+r_{2}\tilde{\mu}_{2}\left(\frac{\mu_{1}}{1-\tilde{\mu}_{2}}F_{2}^{(0,1)}+F_{2}^{(1,0)}\right).
\end{eqnarray*}

%6

\item \begin{eqnarray*} &&\frac{\partial}{\partial
z_2}\frac{\partial}{\partial
z_2}\left(R_2\left(P_1\left(z_1\right)\bar{P}_2\left(z_2\right)\hat{P}_1\left(w_1\right)\hat{P}_2\left(w_2\right)\right)
F_2\left(z_1,\theta_2\left(P_1\left(z_1\right)\hat{P}_1\left(w_1\right)\hat{P}_2\left(w_2\right)\right)\right)\hat{F}_2\left(w_1,w_2\right)\right)\\
&=&\tilde{\mu}_{2}^{2}R_{2}^{(2)}(1)+r_{2}\tilde{P}_{2}^{(2)}\left(1\right).
\end{eqnarray*}

%7
\item \begin{eqnarray*} &&\frac{\partial}{\partial
w_1}\frac{\partial}{\partial
z_2}\left(R_2\left(P_1\left(z_1\right)\bar{P}_2\left(z_2\right)\hat{P}_1\left(w_1\right)\hat{P}_2\left(w_2\right)\right)
F_2\left(z_1,\theta_2\left(P_1\left(z_1\right)\hat{P}_1\left(w_1\right)\hat{P}_2\left(w_2\right)\right)\right)\hat{F}_2\left(w_1,w_2\right)\right)\\
&=&\hat{\mu}_{1}\tilde{\mu}_{2}r_{2}+\hat{\mu}_{1}\tilde{\mu}_{2}R_{2}^{(2)}(1)+
r_{2}\frac{\hat{\mu}_{1}\tilde{\mu}_{2}}{1-\tilde{\mu}_{2}}F_{2}^{(0,1)}+r_{2}\tilde{\mu}_{2}\hat{F}_{2}^{(1,0)}.
\end{eqnarray*}
%8
\item \begin{eqnarray*} &&\frac{\partial}{\partial
w_2}\frac{\partial}{\partial
z_2}\left(R_2\left(P_1\left(z_1\right)\bar{P}_2\left(z_2\right)\hat{P}_1\left(w_1\right)\hat{P}_2\left(w_2\right)\right)
F_2\left(z_1,\theta_2\left(P_1\left(z_1\right)\hat{P}_1\left(w_1\right)\hat{P}_2\left(w_2\right)\right)\right)\hat{F}_2\left(w_1,w_2\right)\right)\\
&=&\hat{\mu}_{2}\tilde{\mu}_{2}r_{2}+\hat{\mu}_{2}\tilde{\mu}_{2}R_{2}^{(2)}(1)+
r_{2}\frac{\hat{\mu}_{2}\tilde{\mu}_{2}}{1-\tilde{\mu}_{2}}F_{2}^{(0,1)}+r_{2}\tilde{\mu}_{2}\hat{F}_{2}^{(0,1)}.
\end{eqnarray*}
%___________________________________________________________________________________________
%\subsubsection{Mixtas para $w_{1}$:}
%___________________________________________________________________________________________

%9
\item \begin{eqnarray*} &&\frac{\partial}{\partial
z_1}\frac{\partial}{\partial
w_1}\left(R_2\left(P_1\left(z_1\right)\bar{P}_2\left(z_2\right)\hat{P}_1\left(w_1\right)\hat{P}_2\left(w_2\right)\right)
F_2\left(z_1,\theta_2\left(P_1\left(z_1\right)\hat{P}_1\left(w_1\right)\hat{P}_2\left(w_2\right)\right)\right)\hat{F}_2\left(w_1,w_2\right)\right)\\
&=&\mu_{1}\hat{\mu}_{1}r_{2}+\mu_{1}\hat{\mu}_{1}R_{2}^{(2)}\left(1\right)+\frac{\mu_{1}\hat{\mu}_{1}}{1-\tilde{\mu}_{2}}F_{2}^{(0,1)}+r_{2}\frac{\mu_{1}\hat{\mu}_{1}}{1-\tilde{\mu}_{2}}F_{2}^{(0,1)}+\mu_{1}\hat{\mu}_{1}\tilde{\theta}_{2}^{(2)}\left(1\right)F_{2}^{(0,1)}\\
&+&r_{2}\hat{\mu}_{1}\left(\frac{\mu_{1}}{1-\tilde{\mu}_{2}}F_{2}^{(0,1)}+F_{2}^{(1,0)}\right)+r_{2}\mu_{1}\hat{F}_{2}^{(1,0)}
+\left(\frac{\mu_{1}}{1-\tilde{\mu}_{2}}F_{2}^{(0,1)}+F_{2}^{(1,0)}\right)\hat{F}_{2}^{(1,0)}\\
&+&\frac{\hat{\mu}_{1}}{1-\tilde{\mu}_{2}}\left(\frac{\mu_{1}}{1-\tilde{\mu}_{2}}F_{2}^{(0,2)}+F_{2}^{(1,1)}\right).
\end{eqnarray*}
%10
\item \begin{eqnarray*} &&\frac{\partial}{\partial
z_2}\frac{\partial}{\partial
w_1}\left(R_2\left(P_1\left(z_1\right)\bar{P}_2\left(z_2\right)\hat{P}_1\left(w_1\right)\hat{P}_2\left(w_2\right)\right)
F_2\left(z_1,\theta_2\left(P_1\left(z_1\right)\hat{P}_1\left(w_1\right)\hat{P}_2\left(w_2\right)\right)\right)\hat{F}_2\left(w_1,w_2\right)\right)\\
&=&\tilde{\mu}_{2}\hat{\mu}_{1}r_{2}+\tilde{\mu}_{2}\hat{\mu}_{1}R_{2}^{(2)}\left(1\right)+r_{2}\frac{\tilde{\mu}_{2}\hat{\mu}_{1}}{1-\tilde{\mu}_{2}}F_{2}^{(0,1)}
+r_{2}\tilde{\mu}_{2}\hat{F}_{2}^{(1,0)}.
\end{eqnarray*}
%11
\item \begin{eqnarray*} &&\frac{\partial}{\partial
w_1}\frac{\partial}{\partial
w_1}\left(R_2\left(P_1\left(z_1\right)\bar{P}_2\left(z_2\right)\hat{P}_1\left(w_1\right)\hat{P}_2\left(w_2\right)\right)
F_2\left(z_1,\theta_2\left(P_1\left(z_1\right)\hat{P}_1\left(w_1\right)\hat{P}_2\left(w_2\right)\right)\right)\hat{F}_2\left(w_1,w_2\right)\right)\\
&=&\hat{\mu}_{1}^{2}R_{2}^{(2)}\left(1\right)+r_{2}\hat{P}_{1}^{(2)}\left(1\right)+2r_{2}\frac{\hat{\mu}_{1}^{2}}{1-\tilde{\mu}_{2}}F_{2}^{(0,1)}+
\hat{\mu}_{1}^{2}\tilde{\theta}_{2}^{(2)}\left(1\right)F_{2}^{(0,1)}+\frac{1}{1-\tilde{\mu}_{2}}\hat{P}_{1}^{(2)}\left(1\right)F_{2}^{(0,1)}\\
&+&\frac{\hat{\mu}_{1}^{2}}{1-\tilde{\mu}_{2}}F_{2}^{(0,2)}+2r_{2}\hat{\mu}_{1}\hat{F}_{2}^{(1,0)}+2\frac{\hat{\mu}_{1}}{1-\tilde{\mu}_{2}}F_{2}^{(0,1)}\hat{F}_{2}^{(1,0)}+\hat{F}_{2}^{(2,0)}.
\end{eqnarray*}
%12
\item \begin{eqnarray*} &&\frac{\partial}{\partial
w_2}\frac{\partial}{\partial
w_1}\left(R_2\left(P_1\left(z_1\right)\bar{P}_2\left(z_2\right)\hat{P}_1\left(w_1\right)\hat{P}_2\left(w_2\right)\right)
F_2\left(z_1,\theta_2\left(P_1\left(z_1\right)\hat{P}_1\left(w_1\right)\hat{P}_2\left(w_2\right)\right)\right)\hat{F}_2\left(w_1,w_2\right)\right)\\
&=&r_{2}\hat{\mu}_{2}\hat{\mu}_{1}+\hat{\mu}_{1}\hat{\mu}_{2}R_{2}^{(2)}(1)+\frac{\hat{\mu}_{1}\hat{\mu}_{2}}{1-\tilde{\mu}_{2}}F_{2}^{(0,1)}
+2r_{2}\frac{\hat{\mu}_{1}\hat{\mu}_{2}}{1-\tilde{\mu}_{2}}F_{2}^{(0,1)}+\hat{\mu}_{2}\hat{\mu}_{1}\tilde{\theta}_{2}^{(2)}\left(1\right)F_{2}^{(0,1)}+
r_{2}\hat{\mu}_{1}\hat{F}_{2}^{(0,1)}\\
&+&\frac{\hat{\mu}_{1}}{1-\tilde{\mu}_{2}}F_{2}^{(0,1)}\hat{F}_{2}^{(0,1)}+\hat{\mu}_{1}\hat{\mu}_{2}\left(\frac{1}{1-\tilde{\mu}_{2}}\right)^{2}F_{2}^{(0,2)}+
r_{2}\hat{\mu}_{2}\hat{F}_{2}^{(1,0)}+\frac{\hat{\mu}_{2}}{1-\tilde{\mu}_{2}}F_{2}^{(0,1)}\hat{F}_{2}^{(1,0)}+\hat{F}_{2}^{(1,1)}.
\end{eqnarray*}
%___________________________________________________________________________________________
%\subsubsection{Mixtas para $w_{2}$:}
%___________________________________________________________________________________________
%13

\item \begin{eqnarray*} &&\frac{\partial}{\partial
z_1}\frac{\partial}{\partial
w_2}\left(R_2\left(P_1\left(z_1\right)\bar{P}_2\left(z_2\right)\hat{P}_1\left(w_1\right)\hat{P}_2\left(w_2\right)\right)
F_2\left(z_1,\theta_2\left(P_1\left(z_1\right)\hat{P}_1\left(w_1\right)\hat{P}_2\left(w_2\right)\right)\right)\hat{F}_2\left(w_1,w_2\right)\right)\\
&=&r_{2}\mu_{1}\hat{\mu}_{2}+\mu_{1}\hat{\mu}_{2}R_{2}^{(2)}(1)+\frac{\mu_{1}\hat{\mu}_{2}}{1-\tilde{\mu}_{2}}F_{2}^{(0,1)}+r_{2}\frac{\mu_{1}\hat{\mu}_{2}}{1-\tilde{\mu}_{2}}F_{2}^{(0,1)}+\mu_{1}\hat{\mu}_{2}\tilde{\theta}_{2}^{(2)}\left(1\right)F_{2}^{(0,1)}+r_{2}\mu_{1}\hat{F}_{2}^{(0,1)}\\
&+&r_{2}\hat{\mu}_{2}\left(\frac{\mu_{1}}{1-\tilde{\mu}_{2}}F_{2}^{(0,1)}+F_{2}^{(1,0)}\right)+\hat{F}_{2}^{(0,1)}\left(\frac{\mu_{1}}{1-\tilde{\mu}_{2}}F_{2}^{(0,1)}+F_{2}^{(1,0)}\right)+\frac{\hat{\mu}_{2}}{1-\tilde{\mu}_{2}}\left(\frac{\mu_{1}}{1-\tilde{\mu}_{2}}F_{2}^{(0,2)}+F_{2}^{(1,1)}\right).
\end{eqnarray*}
%14
\item \begin{eqnarray*} &&\frac{\partial}{\partial
z_2}\frac{\partial}{\partial
w_2}\left(R_2\left(P_1\left(z_1\right)\bar{P}_2\left(z_2\right)\hat{P}_1\left(w_1\right)\hat{P}_2\left(w_2\right)\right)
F_2\left(z_1,\theta_2\left(P_1\left(z_1\right)\hat{P}_1\left(w_1\right)\hat{P}_2\left(w_2\right)\right)\right)\hat{F}_2\left(w_1,w_2\right)\right)\\
&=&r_{2}\tilde{\mu}_{2}\hat{\mu}_{2}+\tilde{\mu}_{2}\hat{\mu}_{2}R_{2}^{(2)}(1)+r_{2}\frac{\tilde{\mu}_{2}\hat{\mu}_{2}}{1-\tilde{\mu}_{2}}F_{2}^{(0,1)}+r_{2}\tilde{\mu}_{2}\hat{F}_{2}^{(0,1)}.
\end{eqnarray*}
%15
\item \begin{eqnarray*} &&\frac{\partial}{\partial
w_1}\frac{\partial}{\partial
w_2}\left(R_2\left(P_1\left(z_1\right)\bar{P}_2\left(z_2\right)\hat{P}_1\left(w_1\right)\hat{P}_2\left(w_2\right)\right)
F_2\left(z_1,\theta_2\left(P_1\left(z_1\right)\hat{P}_1\left(w_1\right)\hat{P}_2\left(w_2\right)\right)\right)\hat{F}_2\left(w_1,w_2\right)\right)\\
&=&r_{2}\hat{\mu}_{1}\hat{\mu}_{2}+\hat{\mu}_{1}\hat{\mu}_{2}R_{2}^{(2)}\left(1\right)+\frac{\hat{\mu}_{1}\hat{\mu}_{2}}{1-\tilde{\mu}_{2}}F_{2}^{(0,1)}+2r_{2}\frac{\hat{\mu}_{1}\hat{\mu}_{2}}{1-\tilde{\mu}_{2}}F_{2}^{(0,1)}+\hat{\mu}_{1}\hat{\mu}_{2}\theta_{2}^{(2)}\left(1\right)F_{2}^{(0,1)}+r_{2}\hat{\mu}_{1}\hat{F}_{2}^{(0,1)}\\
&+&\frac{\hat{\mu}_{1}}{1-\tilde{\mu}_{2}}F_{2}^{(0,1)}\hat{F}_{2}^{(0,1)}+\hat{\mu}_{1}\hat{\mu}_{2}\left(\frac{1}{1-\tilde{\mu}_{2}}\right)^{2}F_{2}^{(0,2)}+r_{2}\hat{\mu}_{2}\hat{F}_{2}^{(0,1)}+\frac{\hat{\mu}_{2}}{1-\tilde{\mu}_{2}}F_{2}^{(0,1)}\hat{F}_{2}^{(1,0)}+\hat{F}_{2}^{(1,1)}.
\end{eqnarray*}
%16

\item \begin{eqnarray*} &&\frac{\partial}{\partial
w_2}\frac{\partial}{\partial
w_2}\left(R_2\left(P_1\left(z_1\right)\bar{P}_2\left(z_2\right)\hat{P}_1\left(w_1\right)\hat{P}_2\left(w_2\right)\right)
F_2\left(z_1,\theta_2\left(P_1\left(z_1\right)\hat{P}_1\left(w_1\right)\hat{P}_2\left(w_2\right)\right)\right)\hat{F}_2\left(w_1,w_2\right)\right)\\
&=&\hat{\mu}_{2}^{2}R_{2}^{(2)}(1)+r_{2}\hat{P}_{2}^{(2)}\left(1\right)+2r_{2}\frac{\hat{\mu}_{2}^{2}}{1-\tilde{\mu}_{2}}F_{2}^{(0,1)}+\hat{\mu}_{2}^{2}\tilde{\theta}_{2}^{(2)}\left(1\right)F_{2}^{(0,1)}+\frac{1}{1-\tilde{\mu}_{2}}\hat{P}_{2}^{(2)}\left(1\right)F_{2}^{(0,1)}\\
&+&2r_{2}\hat{\mu}_{2}\hat{F}_{2}^{(0,1)}+2\frac{\hat{\mu}_{2}}{1-\tilde{\mu}_{2}}F_{2}^{(0,1)}\hat{F}_{2}^{(0,1)}+\left(\frac{\hat{\mu}_{2}}{1-\tilde{\mu}_{2}}\right)^{2}F_{2}^{(0,2)}+\hat{F}_{2}^{(0,2)}.
\end{eqnarray*}
\end{enumerate}
%___________________________________________________________________________________________
%
%\subsection{Derivadas de Segundo Orden para $F_{2}$}
%___________________________________________________________________________________________


\begin{enumerate}

%___________________________________________________________________________________________
%\subsubsection{Mixtas para $z_{1}$:}
%___________________________________________________________________________________________

%1/17
\item \begin{eqnarray*} &&\frac{\partial}{\partial
z_1}\frac{\partial}{\partial
z_1}\left(R_1\left(P_1\left(z_1\right)\bar{P}_2\left(z_2\right)\hat{P}_1\left(w_1\right)\hat{P}_2\left(w_2\right)\right)
F_1\left(\theta_1\left(\tilde{P}_2\left(z_1\right)\hat{P}_1\left(w_1\right)\hat{P}_2\left(w_2\right)\right)\right)\hat{F}_1\left(w_1,w_2\right)\right)\\
&=&r_{1}P_{1}^{(2)}\left(1\right)+\mu_{1}^{2}R_{1}^{(2)}\left(1\right).
\end{eqnarray*}

%2/18
\item \begin{eqnarray*} &&\frac{\partial}{\partial
z_2}\frac{\partial}{\partial
z_1}\left(R_1\left(P_1\left(z_1\right)\bar{P}_2\left(z_2\right)\hat{P}_1\left(w_1\right)\hat{P}_2\left(w_2\right)\right)F_1\left(\theta_1\left(\tilde{P}_2\left(z_1\right)\hat{P}_1\left(w_1\right)\hat{P}_2\left(w_2\right)\right)\right)\hat{F}_1\left(w_1,w_2\right)\right)\\
&=&\mu_{1}\tilde{\mu}_{2}r_{1}+\mu_{1}\tilde{\mu}_{2}R_{1}^{(2)}(1)+
r_{1}\mu_{1}\left(\frac{\tilde{\mu}_{2}}{1-\mu_{1}}F_{1}^{(1,0)}+F_{1}^{(0,1)}\right).
\end{eqnarray*}

%3/19
\item \begin{eqnarray*} &&\frac{\partial}{\partial
w_1}\frac{\partial}{\partial
z_1}\left(R_1\left(P_1\left(z_1\right)\bar{P}_2\left(z_2\right)\hat{P}_1\left(w_1\right)\hat{P}_2\left(w_2\right)\right)F_1\left(\theta_1\left(\tilde{P}_2\left(z_1\right)\hat{P}_1\left(w_1\right)\hat{P}_2\left(w_2\right)\right)\right)\hat{F}_1\left(w_1,w_2\right)\right)\\
&=&r_{1}\mu_{1}\hat{\mu}_{1}+\mu_{1}\hat{\mu}_{1}R_{1}^{(2)}\left(1\right)+r_{1}\frac{\mu_{1}\hat{\mu}_{1}}{1-\mu_{1}}F_{1}^{(1,0)}+r_{1}\mu_{1}\hat{F}_{1}^{(1,0)}.
\end{eqnarray*}
%4/20
\item \begin{eqnarray*} &&\frac{\partial}{\partial
w_2}\frac{\partial}{\partial
z_1}\left(R_1\left(P_1\left(z_1\right)\bar{P}_2\left(z_2\right)\hat{P}_1\left(w_1\right)\hat{P}_2\left(w_2\right)\right)F_1\left(\theta_1\left(\tilde{P}_2\left(z_1\right)\hat{P}_1\left(w_1\right)\hat{P}_2\left(w_2\right)\right)\right)\hat{F}_1\left(w_1,w_2\right)\right)\\
&=&\mu_{1}\hat{\mu}_{2}r_{1}+\mu_{1}\hat{\mu}_{2}R_{1}^{(2)}\left(1\right)+r_{1}\mu_{1}\hat{F}_{1}^{(0,1)}+r_{1}\frac{\mu_{1}\hat{\mu}_{2}}{1-\mu_{1}}F_{1}^{(1,0)}.
\end{eqnarray*}
%___________________________________________________________________________________________
%\subsubsection{Mixtas para $z_{2}$:}
%___________________________________________________________________________________________
%5/21
\item \begin{eqnarray*}
&&\frac{\partial}{\partial z_1}\frac{\partial}{\partial z_2}\left(R_1\left(P_1\left(z_1\right)\bar{P}_2\left(z_2\right)\hat{P}_1\left(w_1\right)\hat{P}_2\left(w_2\right)\right)F_1\left(\theta_1\left(\tilde{P}_2\left(z_1\right)\hat{P}_1\left(w_1\right)\hat{P}_2\left(w_2\right)\right)\right)\hat{F}_1\left(w_1,w_2\right)\right)\\
&=&r_{1}\mu_{1}\tilde{\mu}_{2}+\mu_{1}\tilde{\mu}_{2}R_{1}^{(2)}\left(1\right)+r_{1}\mu_{1}\left(\frac{\tilde{\mu}_{2}}{1-\mu_{1}}F_{1}^{(1,0)}+F_{1}^{(0,1)}\right).
\end{eqnarray*}

%6/22
\item \begin{eqnarray*}
&&\frac{\partial}{\partial z_2}\frac{\partial}{\partial z_2}\left(R_1\left(P_1\left(z_1\right)\bar{P}_2\left(z_2\right)\hat{P}_1\left(w_1\right)\hat{P}_2\left(w_2\right)\right)F_1\left(\theta_1\left(\tilde{P}_2\left(z_1\right)\hat{P}_1\left(w_1\right)\hat{P}_2\left(w_2\right)\right)\right)\hat{F}_1\left(w_1,w_2\right)\right)\\
&=&\tilde{\mu}_{2}^{2}R_{1}^{(2)}\left(1\right)+r_{1}\tilde{P}_{2}^{(2)}\left(1\right)+2r_{1}\tilde{\mu}_{2}\left(\frac{\tilde{\mu}_{2}}{1-\mu_{1}}F_{1}^{(1,0)}+F_{1}^{(0,1)}\right)+F_{1}^{(0,2)}+\tilde{\mu}_{2}^{2}\theta_{1}^{(2)}\left(1\right)F_{1}^{(1,0)}\\
&+&\frac{1}{1-\mu_{1}}\tilde{P}_{2}^{(2)}\left(1\right)F_{1}^{(1,0)}+\frac{\tilde{\mu}_{2}}{1-\mu_{1}}F_{1}^{(1,1)}+\frac{\tilde{\mu}_{2}}{1-\mu_{1}}\left(\frac{\tilde{\mu}_{2}}{1-\mu_{1}}F_{1}^{(2,0)}+F_{1}^{(1,1)}\right).
\end{eqnarray*}
%7/23
\item \begin{eqnarray*}
&&\frac{\partial}{\partial w_1}\frac{\partial}{\partial z_2}\left(R_1\left(P_1\left(z_1\right)\bar{P}_2\left(z_2\right)\hat{P}_1\left(w_1\right)\hat{P}_2\left(w_2\right)\right)F_1\left(\theta_1\left(\tilde{P}_2\left(z_1\right)\hat{P}_1\left(w_1\right)\hat{P}_2\left(w_2\right)\right)\right)\hat{F}_1\left(w_1,w_2\right)\right)\\
&=&\tilde{\mu}_{2}\hat{\mu}_{1}r_{1}+\tilde{\mu}_{2}\hat{\mu}_{1}R_{1}^{(2)}\left(1\right)+r_{1}\frac{\tilde{\mu}_{2}\hat{\mu}_{1}}{1-\mu_{1}}F_{1}^{(1,0)}+\hat{\mu}_{1}r_{1}\left(\frac{\tilde{\mu}_{2}}{1-\mu_{1}}F_{1}^{(1,0)}+F_{1}^{(0,1)}\right)+r_{1}\tilde{\mu}_{2}\hat{F}_{1}^{(1,0)}\\
&+&\left(\frac{\tilde{\mu}_{2}}{1-\mu_{1}}F_{1}^{(1,0)}+F_{1}^{(0,1)}\right)\hat{F}_{1}^{(1,0)}+\frac{\tilde{\mu}_{2}\hat{\mu}_{1}}{1-\mu_{1}}F_{1}^{(1,0)}+\tilde{\mu}_{2}\hat{\mu}_{1}\theta_{1}^{(2)}\left(1\right)F_{1}^{(1,0)}+\frac{\hat{\mu}_{1}}{1-\mu_{1}}F_{1}^{(1,1)}\\
&+&\left(\frac{1}{1-\mu_{1}}\right)^{2}\tilde{\mu}_{2}\hat{\mu}_{1}F_{1}^{(2,0)}.
\end{eqnarray*}
%8/24
\item \begin{eqnarray*}
&&\frac{\partial}{\partial w_2}\frac{\partial}{\partial z_2}\left(R_1\left(P_1\left(z_1\right)\bar{P}_2\left(z_2\right)\hat{P}_1\left(w_1\right)\hat{P}_2\left(w_2\right)\right)F_1\left(\theta_1\left(\tilde{P}_2\left(z_1\right)\hat{P}_1\left(w_1\right)\hat{P}_2\left(w_2\right)\right)\right)\hat{F}_1\left(w_1,w_2\right)\right)\\
&=&\hat{\mu}_{2}\tilde{\mu}_{2}r_{1}+\hat{\mu}_{2}\tilde{\mu}_{2}R_{1}^{(2)}(1)+r_{1}\tilde{\mu}_{2}\hat{F}_{1}^{(0,1)}+r_{1}\frac{\hat{\mu}_{2}\tilde{\mu}_{2}}{1-\mu_{1}}F_{1}^{(1,0)}+\hat{\mu}_{2}r_{1}\left(\frac{\tilde{\mu}_{2}}{1-\mu_{1}}F_{1}^{(1,0)}+F_{1}^{(0,1)}\right)\\
&+&\left(\frac{\tilde{\mu}_{2}}{1-\mu_{1}}F_{1}^{(1,0)}+F_{1}^{(0,1)}\right)\hat{F}_{1}^{(0,1)}+\frac{\tilde{\mu}_{2}\hat{\mu_{2}}}{1-\mu_{1}}F_{1}^{(1,0)}+\hat{\mu}_{2}\tilde{\mu}_{2}\theta_{1}^{(2)}\left(1\right)F_{1}^{(1,0)}+\frac{\hat{\mu}_{2}}{1-\mu_{1}}F_{1}^{(1,1)}\\
&+&\left(\frac{1}{1-\mu_{1}}\right)^{2}\tilde{\mu}_{2}\hat{\mu}_{2}F_{1}^{(2,0)}.
\end{eqnarray*}
%___________________________________________________________________________________________
%\subsubsection{Mixtas para $w_{1}$:}
%___________________________________________________________________________________________
%9/25
\item \begin{eqnarray*} &&\frac{\partial}{\partial
z_1}\frac{\partial}{\partial
w_1}\left(R_1\left(P_1\left(z_1\right)\bar{P}_2\left(z_2\right)\hat{P}_1\left(w_1\right)\hat{P}_2\left(w_2\right)\right)F_1\left(\theta_1\left(\tilde{P}_2\left(z_1\right)\hat{P}_1\left(w_1\right)\hat{P}_2\left(w_2\right)\right)\right)\hat{F}_1\left(w_1,w_2\right)\right)\\
&=&r_{1}\mu_{1}\hat{\mu}_{1}+\mu_{1}\hat{\mu}_{1}R_{1}^{(2)}(1)+r_{1}\frac{\mu_{1}\hat{\mu}_{1}}{1-\mu_{1}}F_{1}^{(1,0)}+r_{1}\mu_{1}\hat{F}_{1}^{(1,0)}.
\end{eqnarray*}
%10/26
\item \begin{eqnarray*} &&\frac{\partial}{\partial
z_2}\frac{\partial}{\partial
w_1}\left(R_1\left(P_1\left(z_1\right)\bar{P}_2\left(z_2\right)\hat{P}_1\left(w_1\right)\hat{P}_2\left(w_2\right)\right)F_1\left(\theta_1\left(\tilde{P}_2\left(z_1\right)\hat{P}_1\left(w_1\right)\hat{P}_2\left(w_2\right)\right)\right)\hat{F}_1\left(w_1,w_2\right)\right)\\
&=&r_{1}\hat{\mu}_{1}\tilde{\mu}_{2}+\tilde{\mu}_{2}\hat{\mu}_{1}R_{1}^{(2)}\left(1\right)+
\frac{\hat{\mu}_{1}\tilde{\mu}_{2}}{1-\mu_{1}}F_{1}^{1,0)}+r_{1}\frac{\hat{\mu}_{1}\tilde{\mu}_{2}}{1-\mu_{1}}F_{1}^{(1,0)}+\hat{\mu}_{1}\tilde{\mu}_{2}\theta_{1}^{(2)}\left(1\right)F_{2}^{(1,0)}\\
&+&r_{1}\hat{\mu}_{1}\left(F_{1}^{(1,0)}+\frac{\tilde{\mu}_{2}}{1-\mu_{1}}F_{1}^{1,0)}\right)+
r_{1}\tilde{\mu}_{2}\hat{F}_{1}^{(1,0)}+\left(F_{1}^{(0,1)}+\frac{\tilde{\mu}_{2}}{1-\mu_{1}}F_{1}^{1,0)}\right)\hat{F}_{1}^{(1,0)}\\
&+&\frac{\hat{\mu}_{1}}{1-\mu_{1}}\left(F_{1}^{(1,1)}+\frac{\tilde{\mu}_{2}}{1-\mu_{1}}F_{1}^{2,0)}\right).
\end{eqnarray*}
%11/27
\item \begin{eqnarray*} &&\frac{\partial}{\partial
w_1}\frac{\partial}{\partial
w_1}\left(R_1\left(P_1\left(z_1\right)\bar{P}_2\left(z_2\right)\hat{P}_1\left(w_1\right)\hat{P}_2\left(w_2\right)\right)F_1\left(\theta_1\left(\tilde{P}_2\left(z_1\right)\hat{P}_1\left(w_1\right)\hat{P}_2\left(w_2\right)\right)\right)\hat{F}_1\left(w_1,w_2\right)\right)\\
&=&\hat{\mu}_{1}^{2}R_{1}^{(2)}\left(1\right)+r_{1}\hat{P}_{1}^{(2)}\left(1\right)+2r_{1}\frac{\hat{\mu}_{1}^{2}}{1-\mu_{1}}F_{1}^{(1,0)}+\hat{\mu}_{1}^{2}\theta_{1}^{(2)}\left(1\right)F_{1}^{(1,0)}+\frac{1}{1-\mu_{1}}\hat{P}_{1}^{(2)}\left(1\right)F_{1}^{(1,0)}\\
&+&2r_{1}\hat{\mu}_{1}\hat{F}_{1}^{(1,0)}+2\frac{\hat{\mu}_{1}}{1-\mu_{1}}F_{1}^{(1,0)}\hat{F}_{1}^{(1,0)}+\left(\frac{\hat{\mu}_{1}}{1-\mu_{1}}\right)^{2}F_{1}^{(2,0)}+\hat{F}_{1}^{(2,0)}.
\end{eqnarray*}
%12/28
\item \begin{eqnarray*} &&\frac{\partial}{\partial
w_2}\frac{\partial}{\partial
w_1}\left(R_1\left(P_1\left(z_1\right)\bar{P}_2\left(z_2\right)\hat{P}_1\left(w_1\right)\hat{P}_2\left(w_2\right)\right)F_1\left(\theta_1\left(\tilde{P}_2\left(z_1\right)\hat{P}_1\left(w_1\right)\hat{P}_2\left(w_2\right)\right)\right)\hat{F}_1\left(w_1,w_2\right)\right)\\
&=&r_{1}\hat{\mu}_{1}\hat{\mu}_{2}+\hat{\mu}_{1}\hat{\mu}_{2}R_{1}^{(2)}\left(1\right)+r_{1}\hat{\mu}_{1}\hat{F}_{1}^{(0,1)}+
\frac{\hat{\mu}_{1}\hat{\mu}_{2}}{1-\mu_{1}}F_{1}^{(1,0)}+2r_{1}\frac{\hat{\mu}_{1}\hat{\mu}_{2}}{1-\mu_{1}}F_{1}^{1,0)}+\hat{\mu}_{1}\hat{\mu}_{2}\theta_{1}^{(2)}\left(1\right)F_{1}^{(1,0)}\\
&+&\frac{\hat{\mu}_{1}}{1-\mu_{1}}F_{1}^{(1,0)}\hat{F}_{1}^{(0,1)}+
r_{1}\hat{\mu}_{2}\hat{F}_{1}^{(1,0)}+\frac{\hat{\mu}_{2}}{1-\mu_{1}}\hat{F}_{1}^{(1,0)}F_{1}^{(1,0)}+\hat{F}_{1}^{(1,1)}+\hat{\mu}_{1}\hat{\mu}_{2}\left(\frac{1}{1-\mu_{1}}\right)^{2}F_{1}^{(2,0)}.
\end{eqnarray*}
%___________________________________________________________________________________________
%\subsubsection{Mixtas para $w_{2}$:}
%___________________________________________________________________________________________
%13/29
\item \begin{eqnarray*} &&\frac{\partial}{\partial
z_1}\frac{\partial}{\partial
w_2}\left(R_1\left(P_1\left(z_1\right)\bar{P}_2\left(z_2\right)\hat{P}_1\left(w_1\right)\hat{P}_2\left(w_2\right)\right)F_1\left(\theta_1\left(\tilde{P}_2\left(z_1\right)\hat{P}_1\left(w_1\right)\hat{P}_2\left(w_2\right)\right)\right)\hat{F}_1\left(w_1,w_2\right)\right)\\
&=&r_{1}\mu_{1}\hat{\mu}_{2}+\mu_{1}\hat{\mu}_{2}R_{1}^{(2)}\left(1\right)+r_{1}\mu_{1}\hat{F}_{1}^{(0,1)}+r_{1}\frac{\mu_{1}\hat{\mu}_{2}}{1-\mu_{1}}F_{1}^{(1,0)}.
\end{eqnarray*}
%14/30
\item \begin{eqnarray*} &&\frac{\partial}{\partial
z_2}\frac{\partial}{\partial
w_2}\left(R_1\left(P_1\left(z_1\right)\bar{P}_2\left(z_2\right)\hat{P}_1\left(w_1\right)\hat{P}_2\left(w_2\right)\right)F_1\left(\theta_1\left(\tilde{P}_2\left(z_1\right)\hat{P}_1\left(w_1\right)\hat{P}_2\left(w_2\right)\right)\right)\hat{F}_1\left(w_1,w_2\right)\right)\\
&=&r_{1}\hat{\mu}_{2}\tilde{\mu}_{2}+\hat{\mu}_{2}\tilde{\mu}_{2}R_{1}^{(2)}\left(1\right)+r_{1}\tilde{\mu}_{2}\hat{F}_{1}^{(0,1)}+\frac{\hat{\mu}_{2}\tilde{\mu}_{2}}{1-\mu_{1}}F_{1}^{(1,0)}+r_{1}\frac{\hat{\mu}_{2}\tilde{\mu}_{2}}{1-\mu_{1}}F_{1}^{(1,0)}\\
&+&\hat{\mu}_{2}\tilde{\mu}_{2}\theta_{1}^{(2)}\left(1\right)F_{1}^{(1,0)}+r_{1}\hat{\mu}_{2}\left(F_{1}^{(0,1)}+\frac{\tilde{\mu}_{2}}{1-\mu_{1}}F_{1}^{(1,0)}\right)+\left(F_{1}^{(0,1)}+\frac{\tilde{\mu}_{2}}{1-\mu_{1}}F_{1}^{(1,0)}\right)\hat{F}_{1}^{(0,1)}\\&+&\frac{\hat{\mu}_{2}}{1-\mu_{1}}\left(F_{1}^{(1,1)}+\frac{\tilde{\mu}_{2}}{1-\mu_{1}}F_{1}^{(2,0)}\right).
\end{eqnarray*}
%15/31
\item \begin{eqnarray*} &&\frac{\partial}{\partial
w_1}\frac{\partial}{\partial
w_2}\left(R_1\left(P_1\left(z_1\right)\bar{P}_2\left(z_2\right)\hat{P}_1\left(w_1\right)\hat{P}_2\left(w_2\right)\right)F_1\left(\theta_1\left(\tilde{P}_2\left(z_1\right)\hat{P}_1\left(w_1\right)\hat{P}_2\left(w_2\right)\right)\right)\hat{F}_1\left(w_1,w_2\right)\right)\\
&=&r_{1}\hat{\mu}_{1}\hat{\mu}_{2}+\hat{\mu}_{1}\hat{\mu}_{2}R_{1}^{(2)}\left(1\right)+r_{1}\hat{\mu}_{1}\hat{F}_{1}^{(0,1)}+
\frac{\hat{\mu}_{1}\hat{\mu}_{2}}{1-\mu_{1}}F_{1}^{(1,0)}+2r_{1}\frac{\hat{\mu}_{1}\hat{\mu}_{2}}{1-\mu_{1}}F_{1}^{(1,0)}+\hat{\mu}_{1}\hat{\mu}_{2}\theta_{1}^{(2)}\left(1\right)F_{1}^{(1,0)}\\
&+&\frac{\hat{\mu}_{1}}{1-\mu_{1}}\hat{F}_{1}^{(0,1)}F_{1}^{(1,0)}+r_{1}\hat{\mu}_{2}\hat{F}_{1}^{(1,0)}+\frac{\hat{\mu}_{2}}{1-\mu_{1}}\hat{F}_{1}^{(1,0)}F_{1}^{(1,0)}+\hat{F}_{1}^{(1,1)}+\hat{\mu}_{1}\hat{\mu}_{2}\left(\frac{1}{1-\mu_{1}}\right)^{2}F_{1}^{(2,0)}.
\end{eqnarray*}
%16/32
\item \begin{eqnarray*} &&\frac{\partial}{\partial
w_2}\frac{\partial}{\partial
w_2}\left(R_1\left(P_1\left(z_1\right)\bar{P}_2\left(z_2\right)\hat{P}_1\left(w_1\right)\hat{P}_2\left(w_2\right)\right)F_1\left(\theta_1\left(\tilde{P}_2\left(z_1\right)\hat{P}_1\left(w_1\right)\hat{P}_2\left(w_2\right)\right)\right)\hat{F}_1\left(w_1,w_2\right)\right)\\
&=&\hat{\mu}_{2}R_{1}^{(2)}\left(1\right)+r_{1}\hat{P}_{2}^{(2)}\left(1\right)+2r_{1}\hat{\mu}_{2}\hat{F}_{1}^{(0,1)}+\hat{F}_{1}^{(0,2)}+2r_{1}\frac{\hat{\mu}_{2}^{2}}{1-\mu_{1}}F_{1}^{(1,0)}+\hat{\mu}_{2}^{2}\theta_{1}^{(2)}\left(1\right)F_{1}^{(1,0)}\\
&+&\frac{1}{1-\mu_{1}}\hat{P}_{2}^{(2)}\left(1\right)F_{1}^{(1,0)} +
2\frac{\hat{\mu}_{2}}{1-\mu_{1}}F_{1}^{(1,0)}\hat{F}_{1}^{(0,1)}+\left(\frac{\hat{\mu}_{2}}{1-\mu_{1}}\right)^{2}F_{1}^{(2,0)}.
\end{eqnarray*}
\end{enumerate}

%___________________________________________________________________________________________
%
%\subsection{Derivadas de Segundo Orden para $\hat{F}_{1}$}
%___________________________________________________________________________________________


\begin{enumerate}
%___________________________________________________________________________________________
%\subsubsection{Mixtas para $z_{1}$:}
%___________________________________________________________________________________________
%1/33

\item \begin{eqnarray*} &&\frac{\partial}{\partial
z_1}\frac{\partial}{\partial
z_1}\left(\hat{R}_{2}\left(P_{1}\left(z_{1}\right)\tilde{P}_{2}\left(z_{2}\right)\hat{P}_{1}\left(w_{1}\right)\hat{P}_{2}\left(w_{2}\right)\right)\hat{F}_{2}\left(w_{1},\hat{\theta}_{2}\left(P_{1}\left(z_{1}\right)\tilde{P}_{2}\left(z_{2}\right)\hat{P}_{1}\left(w_{1}\right)\right)\right)F_{2}\left(z_{1},z_{2}\right)\right)\\
&=&\hat{r}_{2}P_{1}^{(2)}\left(1\right)+
\mu_{1}^{2}\hat{R}_{2}^{(2)}\left(1\right)+
2\hat{r}_{2}\frac{\mu_{1}^{2}}{1-\hat{\mu}_{2}}\hat{F}_{2}^{(0,1)}+
\frac{1}{1-\hat{\mu}_{2}}P_{1}^{(2)}\left(1\right)\hat{F}_{2}^{(0,1)}+
\mu_{1}^{2}\hat{\theta}_{2}^{(2)}\left(1\right)\hat{F}_{2}^{(0,1)}\\
&+&\left(\frac{\mu_{1}^{2}}{1-\hat{\mu}_{2}}\right)^{2}\hat{F}_{2}^{(0,2)}+
2\hat{r}_{2}\mu_{1}F_{2}^{(1,0)}+2\frac{\mu_{1}}{1-\hat{\mu}_{2}}\hat{F}_{2}^{(0,1)}F_{2}^{(1,0)}+F_{2}^{(2,0)}.
\end{eqnarray*}

%2/34
\item \begin{eqnarray*} &&\frac{\partial}{\partial
z_2}\frac{\partial}{\partial
z_1}\left(\hat{R}_{2}\left(P_{1}\left(z_{1}\right)\tilde{P}_{2}\left(z_{2}\right)\hat{P}_{1}\left(w_{1}\right)\hat{P}_{2}\left(w_{2}\right)\right)\hat{F}_{2}\left(w_{1},\hat{\theta}_{2}\left(P_{1}\left(z_{1}\right)\tilde{P}_{2}\left(z_{2}\right)\hat{P}_{1}\left(w_{1}\right)\right)\right)F_{2}\left(z_{1},z_{2}\right)\right)\\
&=&\hat{r}_{2}\mu_{1}\tilde{\mu}_{2}+\mu_{1}\tilde{\mu}_{2}\hat{R}_{2}^{(2)}\left(1\right)+\hat{r}_{2}\mu_{1}F_{2}^{(0,1)}+
\frac{\mu_{1}\tilde{\mu}_{2}}{1-\hat{\mu}_{2}}\hat{F}_{2}^{(0,1)}+2\hat{r}_{2}\frac{\mu_{1}\tilde{\mu}_{2}}{1-\hat{\mu}_{2}}\hat{F}_{2}^{(0,1)}+\mu_{1}\tilde{\mu}_{2}\hat{\theta}_{2}^{(2)}\left(1\right)\hat{F}_{2}^{(0,1)}\\
&+&\frac{\mu_{1}}{1-\hat{\mu}_{2}}F_{2}^{(0,1)}\hat{F}_{2}^{(0,1)}+\mu_{1} \tilde{\mu}_{2}\left(\frac{1}{1-\hat{\mu}_{2}}\right)^{2}\hat{F}_{2}^{(0,2)}+\hat{r}_{2}\tilde{\mu}_{2}F_{2}^{(1,0)}+\frac{\tilde{\mu}_{2}}{1-\hat{\mu}_{2}}\hat{F}_{2}^{(0,1)}F_{2}^{(1,0)}+F_{2}^{(1,1)}.
\end{eqnarray*}


%3/35

\item \begin{eqnarray*} &&\frac{\partial}{\partial
w_1}\frac{\partial}{\partial
z_1}\left(\hat{R}_{2}\left(P_{1}\left(z_{1}\right)\tilde{P}_{2}\left(z_{2}\right)\hat{P}_{1}\left(w_{1}\right)\hat{P}_{2}\left(w_{2}\right)\right)\hat{F}_{2}\left(w_{1},\hat{\theta}_{2}\left(P_{1}\left(z_{1}\right)\tilde{P}_{2}\left(z_{2}\right)\hat{P}_{1}\left(w_{1}\right)\right)\right)F_{2}\left(z_{1},z_{2}\right)\right)\\
&=&\hat{r}_{2}\mu_{1}\hat{\mu}_{1}+\mu_{1}\hat{\mu}_{1}\hat{R}_{2}^{(2)}\left(1\right)+\hat{r}_{2}\frac{\mu_{1}\hat{\mu}_{1}}{1-\hat{\mu}_{2}}\hat{F}_{2}^{(0,1)}+\hat{r}_{2}\hat{\mu}_{1}F_{2}^{(1,0)}+\hat{r}_{2}\mu_{1}\hat{F}_{2}^{(1,0)}+F_{2}^{(1,0)}\hat{F}_{2}^{(1,0)}+\frac{\mu_{1}}{1-\hat{\mu}_{2}}\hat{F}_{2}^{(1,1)}.
\end{eqnarray*}

%4/36

\item \begin{eqnarray*} &&\frac{\partial}{\partial
w_2}\frac{\partial}{\partial
z_1}\left(\hat{R}_{2}\left(P_{1}\left(z_{1}\right)\tilde{P}_{2}\left(z_{2}\right)\hat{P}_{1}\left(w_{1}\right)\hat{P}_{2}\left(w_{2}\right)\right)\hat{F}_{2}\left(w_{1},\hat{\theta}_{2}\left(P_{1}\left(z_{1}\right)\tilde{P}_{2}\left(z_{2}\right)\hat{P}_{1}\left(w_{1}\right)\right)\right)F_{2}\left(z_{1},z_{2}\right)\right)\\
&=&\hat{r}_{2}\mu_{1}\hat{\mu}_{2}+\mu_{1}\hat{\mu}_{2}\hat{R}_{2}^{(2)}\left(1\right)+\frac{\mu_{1}\hat{\mu}_{2}}{1-\hat{\mu}_{2}}\hat{F}_{2}^{(0,1)}+2\hat{r}_{2}\frac{\mu_{1}\hat{\mu}_{2}}{1-\hat{\mu}_{2}}\hat{F}_{2}^{(0,1)}+\mu_{1}\hat{\mu}_{2}\hat{\theta}_{2}^{(2)}\left(1\right)\hat{F}_{2}^{(0,1)}\\
&+&\mu_{1}\hat{\mu}_{2}\left(\frac{1}{1-\hat{\mu}_{2}}\right)^{2}\hat{F}_{2}^{(0,2)}+\hat{r}_{2}\hat{\mu}_{2}F_{2}^{(1,0)}+\frac{\hat{\mu}_{2}}{1-\hat{\mu}_{2}}\hat{F}_{2}^{(0,1)}F_{2}^{(1,0)}.
\end{eqnarray*}
%___________________________________________________________________________________________
%\subsubsection{Mixtas para $z_{2}$:}
%___________________________________________________________________________________________

%5/37

\item \begin{eqnarray*} &&\frac{\partial}{\partial
z_1}\frac{\partial}{\partial
z_2}\left(\hat{R}_{2}\left(P_{1}\left(z_{1}\right)\tilde{P}_{2}\left(z_{2}\right)\hat{P}_{1}\left(w_{1}\right)\hat{P}_{2}\left(w_{2}\right)\right)\hat{F}_{2}\left(w_{1},\hat{\theta}_{2}\left(P_{1}\left(z_{1}\right)\tilde{P}_{2}\left(z_{2}\right)\hat{P}_{1}\left(w_{1}\right)\right)\right)F_{2}\left(z_{1},z_{2}\right)\right)\\
&=&\hat{r}_{2}\mu_{1}\tilde{\mu}_{2}+\mu_{1}\tilde{\mu}_{2}\hat{R}_{2}^{(2)}\left(1\right)+\mu_{1}\hat{r}_{2}F_{2}^{(0,1)}+
\frac{\mu_{1}\tilde{\mu}_{2}}{1-\hat{\mu}_{2}}\hat{F}_{2}^{(0,1)}+2\hat{r}_{2}\frac{\mu_{1}\tilde{\mu}_{2}}{1-\hat{\mu}_{2}}\hat{F}_{2}^{(0,1)}+\mu_{1}\tilde{\mu}_{2}\hat{\theta}_{2}^{(2)}\left(1\right)\hat{F}_{2}^{(0,1)}\\
&+&\frac{\mu_{1}}{1-\hat{\mu}_{2}}F_{2}^{(0,1)}\hat{F}_{2}^{(0,1)}+\mu_{1}\tilde{\mu}_{2}\left(\frac{1}{1-\hat{\mu}_{2}}\right)^{2}\hat{F}_{2}^{(0,2)}+\hat{r}_{2}\tilde{\mu}_{2}F_{2}^{(1,0)}+\frac{\tilde{\mu}_{2}}{1-\hat{\mu}_{2}}\hat{F}_{2}^{(0,1)}F_{2}^{(1,0)}+F_{2}^{(1,1)}.
\end{eqnarray*}

%6/38

\item \begin{eqnarray*} &&\frac{\partial}{\partial
z_2}\frac{\partial}{\partial
z_2}\left(\hat{R}_{2}\left(P_{1}\left(z_{1}\right)\tilde{P}_{2}\left(z_{2}\right)\hat{P}_{1}\left(w_{1}\right)\hat{P}_{2}\left(w_{2}\right)\right)\hat{F}_{2}\left(w_{1},\hat{\theta}_{2}\left(P_{1}\left(z_{1}\right)\tilde{P}_{2}\left(z_{2}\right)\hat{P}_{1}\left(w_{1}\right)\right)\right)F_{2}\left(z_{1},z_{2}\right)\right)\\
&=&\hat{r}_{2}\tilde{P}_{2}^{(2)}\left(1\right)+\tilde{\mu}_{2}^{2}\hat{R}_{2}^{(2)}\left(1\right)+2\hat{r}_{2}\tilde{\mu}_{2}F_{2}^{(0,1)}+2\hat{r}_{2}\frac{\tilde{\mu}_{2}^{2}}{1-\hat{\mu}_{2}}\hat{F}_{2}^{(0,1)}+\frac{1}{1-\hat{\mu}_{2}}\tilde{P}_{2}^{(2)}\left(1\right)\hat{F}_{2}^{(0,1)}\\
&+&\tilde{\mu}_{2}^{2}\hat{\theta}_{2}^{(2)}\left(1\right)\hat{F}_{2}^{(0,1)}+2\frac{\tilde{\mu}_{2}}{1-\hat{\mu}_{2}}F_{2}^{(0,1)}\hat{F}_{2}^{(0,1)}+F_{2}^{(0,2)}+\left(\frac{\tilde{\mu}_{2}}{1-\hat{\mu}_{2}}\right)^{2}\hat{F}_{2}^{(0,2)}.
\end{eqnarray*}

%7/39

\item \begin{eqnarray*} &&\frac{\partial}{\partial
w_1}\frac{\partial}{\partial
z_2}\left(\hat{R}_{2}\left(P_{1}\left(z_{1}\right)\tilde{P}_{2}\left(z_{2}\right)\hat{P}_{1}\left(w_{1}\right)\hat{P}_{2}\left(w_{2}\right)\right)\hat{F}_{2}\left(w_{1},\hat{\theta}_{2}\left(P_{1}\left(z_{1}\right)\tilde{P}_{2}\left(z_{2}\right)\hat{P}_{1}\left(w_{1}\right)\right)\right)F_{2}\left(z_{1},z_{2}\right)\right)\\
&=&\hat{r}_{2}\tilde{\mu}_{2}\hat{\mu}_{1}+\tilde{\mu}_{2}\hat{\mu}_{1}\hat{R}_{2}^{(2)}\left(1\right)+\hat{r}_{2}\hat{\mu}_{1}F_{2}^{(0,1)}+\hat{r}_{2}\frac{\tilde{\mu}_{2}\hat{\mu}_{1}}{1-\hat{\mu}_{2}}\hat{F}_{2}^{(0,1)}+\hat{r}_{2}\tilde{\mu}_{2}\hat{F}_{2}^{(1,0)}+F_{2}^{(0,1)}\hat{F}_{2}^{(1,0)}+\frac{\tilde{\mu}_{2}}{1-\hat{\mu}_{2}}\hat{F}_{2}^{(1,1)}.
\end{eqnarray*}
%8/40

\item \begin{eqnarray*} &&\frac{\partial}{\partial
w_2}\frac{\partial}{\partial
z_2}\left(\hat{R}_{2}\left(P_{1}\left(z_{1}\right)\tilde{P}_{2}\left(z_{2}\right)\hat{P}_{1}\left(w_{1}\right)\hat{P}_{2}\left(w_{2}\right)\right)\hat{F}_{2}\left(w_{1},\hat{\theta}_{2}\left(P_{1}\left(z_{1}\right)\tilde{P}_{2}\left(z_{2}\right)\hat{P}_{1}\left(w_{1}\right)\right)\right)F_{2}\left(z_{1},z_{2}\right)\right)\\
&=&\hat{r}_{2}\tilde{\mu}_{2}\hat{\mu}_{2}+\tilde{\mu}_{2}\hat{\mu}_{2}\hat{R}_{2}^{(2)}\left(1\right)+\hat{r}_{2}\hat{\mu}_{2}F_{2}^{(0,1)}+
\frac{\tilde{\mu}_{2}\hat{\mu}_{2}}{1-\hat{\mu}_{2}}\hat{F}_{2}^{(0,1)}+2\hat{r}_{2}\frac{\tilde{\mu}_{2}\hat{\mu}_{2}}{1-\hat{\mu}_{2}}\hat{F}_{2}^{(0,1)}+\tilde{\mu}_{2}\hat{\mu}_{2}\hat{\theta}_{2}^{(2)}\left(1\right)\hat{F}_{2}^{(0,1)}\\
&+&\frac{\hat{\mu}_{2}}{1-\hat{\mu}_{2}}F_{2}^{(0,1)}\hat{F}_{2}^{(1,0)}+\tilde{\mu}_{2}\hat{\mu}_{2}\left(\frac{1}{1-\hat{\mu}_{2}}\right)\hat{F}_{2}^{(0,2)}.
\end{eqnarray*}
%___________________________________________________________________________________________
%\subsubsection{Mixtas para $w_{1}$:}
%___________________________________________________________________________________________

%9/41
\item \begin{eqnarray*} &&\frac{\partial}{\partial
z_1}\frac{\partial}{\partial
w_1}\left(\hat{R}_{2}\left(P_{1}\left(z_{1}\right)\tilde{P}_{2}\left(z_{2}\right)\hat{P}_{1}\left(w_{1}\right)\hat{P}_{2}\left(w_{2}\right)\right)\hat{F}_{2}\left(w_{1},\hat{\theta}_{2}\left(P_{1}\left(z_{1}\right)\tilde{P}_{2}\left(z_{2}\right)\hat{P}_{1}\left(w_{1}\right)\right)\right)F_{2}\left(z_{1},z_{2}\right)\right)\\
&=&\hat{r}_{2}\mu_{1}\hat{\mu}_{1}+\mu_{1}\hat{\mu}_{1}\hat{R}_{2}^{(2)}\left(1\right)+\hat{r}_{2}\frac{\mu_{1}\hat{\mu}_{1}}{1-\hat{\mu}_{2}}\hat{F}_{2}^{(0,1)}+\hat{r}_{2}\hat{\mu}_{1}F_{2}^{(1,0)}+\hat{r}_{2}\mu_{1}\hat{F}_{2}^{(1,0)}+F_{2}^{(1,0)}\hat{F}_{2}^{(1,0)}+\frac{\mu_{1}}{1-\hat{\mu}_{2}}\hat{F}_{2}^{(1,1)}.
\end{eqnarray*}


%10/42
\item \begin{eqnarray*} &&\frac{\partial}{\partial
z_2}\frac{\partial}{\partial
w_1}\left(\hat{R}_{2}\left(P_{1}\left(z_{1}\right)\tilde{P}_{2}\left(z_{2}\right)\hat{P}_{1}\left(w_{1}\right)\hat{P}_{2}\left(w_{2}\right)\right)\hat{F}_{2}\left(w_{1},\hat{\theta}_{2}\left(P_{1}\left(z_{1}\right)\tilde{P}_{2}\left(z_{2}\right)\hat{P}_{1}\left(w_{1}\right)\right)\right)F_{2}\left(z_{1},z_{2}\right)\right)\\
&=&\hat{r}_{2}\tilde{\mu}_{2}\hat{\mu}_{1}+\tilde{\mu}_{2}\hat{\mu}_{1}\hat{R}_{2}^{(2)}\left(1\right)+\hat{r}_{2}\hat{\mu}_{1}F_{2}^{(0,1)}+
\hat{r}_{2}\frac{\tilde{\mu}_{2}\hat{\mu}_{1}}{1-\hat{\mu}_{2}}\hat{F}_{2}^{(0,1)}+\hat{r}_{2}\tilde{\mu}_{2}\hat{F}_{2}^{(1,0)}+F_{2}^{(0,1)}\hat{F}_{2}^{(1,0)}+\frac{\tilde{\mu}_{2}}{1-\hat{\mu}_{2}}\hat{F}_{2}^{(1,1)}.
\end{eqnarray*}


%11/43
\item \begin{eqnarray*} &&\frac{\partial}{\partial
w_1}\frac{\partial}{\partial
w_1}\left(\hat{R}_{2}\left(P_{1}\left(z_{1}\right)\tilde{P}_{2}\left(z_{2}\right)\hat{P}_{1}\left(w_{1}\right)\hat{P}_{2}\left(w_{2}\right)\right)\hat{F}_{2}\left(w_{1},\hat{\theta}_{2}\left(P_{1}\left(z_{1}\right)\tilde{P}_{2}\left(z_{2}\right)\hat{P}_{1}\left(w_{1}\right)\right)\right)F_{2}\left(z_{1},z_{2}\right)\right)\\
&=&\hat{r}_{2}\hat{P}_{1}^{(2)}\left(1\right)+\hat{\mu}_{1}^{2}\hat{R}_{2}^{(2)}\left(1\right)+2\hat{r}_{2}\hat{\mu}_{1}\hat{F}_{2}^{(1,0)}
+\hat{F}_{2}^{(2,0)}.
\end{eqnarray*}


%12/44
\item \begin{eqnarray*} &&\frac{\partial}{\partial
w_2}\frac{\partial}{\partial
w_1}\left(\hat{R}_{2}\left(P_{1}\left(z_{1}\right)\tilde{P}_{2}\left(z_{2}\right)\hat{P}_{1}\left(w_{1}\right)\hat{P}_{2}\left(w_{2}\right)\right)\hat{F}_{2}\left(w_{1},\hat{\theta}_{2}\left(P_{1}\left(z_{1}\right)\tilde{P}_{2}\left(z_{2}\right)\hat{P}_{1}\left(w_{1}\right)\right)\right)F_{2}\left(z_{1},z_{2}\right)\right)\\
&=&\hat{r}_{2}\hat{\mu}_{1}\hat{\mu}_{2}+\hat{\mu}_{1}\hat{\mu}_{2}\hat{R}_{2}^{(2)}\left(1\right)+
\hat{r}_{2}\frac{\hat{\mu}_{2}\hat{\mu}_{1}}{1-\hat{\mu}_{2}}\hat{F}_{2}^{(0,1)}
+\hat{r}_{2}\hat{\mu}_{2}\hat{F}_{2}^{(1,0)}+\frac{\hat{\mu}_{2}}{1-\hat{\mu}_{2}}\hat{F}_{2}^{(1,1)}.
\end{eqnarray*}
%___________________________________________________________________________________________
%\subsubsection{Mixtas para $w_{2}$:}
%___________________________________________________________________________________________
%13/45
\item \begin{eqnarray*} &&\frac{\partial}{\partial
z_1}\frac{\partial}{\partial
w_2}\left(\hat{R}_{2}\left(P_{1}\left(z_{1}\right)\tilde{P}_{2}\left(z_{2}\right)\hat{P}_{1}\left(w_{1}\right)\hat{P}_{2}\left(w_{2}\right)\right)\hat{F}_{2}\left(w_{1},\hat{\theta}_{2}\left(P_{1}\left(z_{1}\right)\tilde{P}_{2}\left(z_{2}\right)\hat{P}_{1}\left(w_{1}\right)\right)\right)F_{2}\left(z_{1},z_{2}\right)\right)\\
&=&\hat{r}_{2}\mu_{1}\hat{\mu}_{2}+\mu_{1}\hat{\mu}_{2}\hat{R}_{2}^{(2)}\left(1\right)+
\frac{\mu_{1}\hat{\mu}_{2}}{1-\hat{\mu}_{2}}\hat{F}_{2}^{(0,1)} +2\hat{r}_{2}\frac{\mu_{1}\hat{\mu}_{2}}{1-\hat{\mu}_{2}}\hat{F}_{2}^{(0,1)}\\
&+&\mu_{1}\hat{\mu}_{2}\hat{\theta}_{2}^{(2)}\left(1\right)\hat{F}_{2}^{(0,1)}+\mu_{1}\hat{\mu}_{2}\left(\frac{1}{1-\hat{\mu}_{2}}\right)^{2}\hat{F}_{2}^{(0,2)}+\hat{r}_{2}\hat{\mu}_{2}F_{2}^{(1,0)}+\frac{\hat{\mu}_{2}}{1-\hat{\mu}_{2}}\hat{F}_{2}^{(0,1)}F_{2}^{(1,0)}.\end{eqnarray*}


%14/46
\item \begin{eqnarray*} &&\frac{\partial}{\partial
z_2}\frac{\partial}{\partial
w_2}\left(\hat{R}_{2}\left(P_{1}\left(z_{1}\right)\tilde{P}_{2}\left(z_{2}\right)\hat{P}_{1}\left(w_{1}\right)\hat{P}_{2}\left(w_{2}\right)\right)\hat{F}_{2}\left(w_{1},\hat{\theta}_{2}\left(P_{1}\left(z_{1}\right)\tilde{P}_{2}\left(z_{2}\right)\hat{P}_{1}\left(w_{1}\right)\right)\right)F_{2}\left(z_{1},z_{2}\right)\right)\\
&=&\hat{r}_{2}\tilde{\mu}_{2}\hat{\mu}_{2}+\tilde{\mu}_{2}\hat{\mu}_{2}\hat{R}_{2}^{(2)}\left(1\right)+\hat{r}_{2}\hat{\mu}_{2}F_{2}^{(0,1)}+\frac{\tilde{\mu}_{2}\hat{\mu}_{2}}{1-\hat{\mu}_{2}}\hat{F}_{2}^{(0,1)}+
2\hat{r}_{2}\frac{\tilde{\mu}_{2}\hat{\mu}_{2}}{1-\hat{\mu}_{2}}\hat{F}_{2}^{(0,1)}+\tilde{\mu}_{2}\hat{\mu}_{2}\hat{\theta}_{2}^{(2)}\left(1\right)\hat{F}_{2}^{(0,1)}\\
&+&\frac{\hat{\mu}_{2}}{1-\hat{\mu}_{2}}\hat{F}_{2}^{(0,1)}F_{2}^{(0,1)}+\tilde{\mu}_{2}\hat{\mu}_{2}\left(\frac{1}{1-\hat{\mu}_{2}}\right)^{2}\hat{F}_{2}^{(0,2)}.
\end{eqnarray*}

%15/47

\item \begin{eqnarray*} &&\frac{\partial}{\partial
w_1}\frac{\partial}{\partial
w_2}\left(\hat{R}_{2}\left(P_{1}\left(z_{1}\right)\tilde{P}_{2}\left(z_{2}\right)\hat{P}_{1}\left(w_{1}\right)\hat{P}_{2}\left(w_{2}\right)\right)\hat{F}_{2}\left(w_{1},\hat{\theta}_{2}\left(P_{1}\left(z_{1}\right)\tilde{P}_{2}\left(z_{2}\right)\hat{P}_{1}\left(w_{1}\right)\right)\right)F_{2}\left(z_{1},z_{2}\right)\right)\\
&=&\hat{r}_{2}\hat{\mu}_{1}\hat{\mu}_{2}+\hat{\mu}_{1}\hat{\mu}_{2}\hat{R}_{2}^{(2)}\left(1\right)+
\hat{r}_{2}\frac{\hat{\mu}_{1}\hat{\mu}_{2}}{1-\hat{\mu}_{2}}\hat{F}_{2}^{(0,1)}+
\hat{r}_{2}\hat{\mu}_{2}\hat{F}_{2}^{(1,0)}+\frac{\hat{\mu}_{2}}{1-\hat{\mu}_{2}}\hat{F}_{2}^{(1,1)}.
\end{eqnarray*}

%16/48
\item \begin{eqnarray*} &&\frac{\partial}{\partial
w_2}\frac{\partial}{\partial
w_2}\left(\hat{R}_{2}\left(P_{1}\left(z_{1}\right)\tilde{P}_{2}\left(z_{2}\right)\hat{P}_{1}\left(w_{1}\right)\hat{P}_{2}\left(w_{2}\right)\right)\hat{F}_{2}\left(w_{1},\hat{\theta}_{2}\left(P_{1}\left(z_{1}\right)\tilde{P}_{2}\left(z_{2}\right)\hat{P}_{1}\left(w_{1}\right)\right)\right)F_{2}\left(z_{1},z_{2};\zeta_{2}\right)\right)\\
&=&\hat{r}_{2}P_{2}^{(2)}\left(1\right)+\hat{\mu}_{2}^{2}\hat{R}_{2}^{(2)}\left(1\right)+2\hat{r}_{2}\frac{\hat{\mu}_{2}^{2}}{1-\hat{\mu}_{2}}\hat{F}_{2}^{(0,1)}+\frac{1}{1-\hat{\mu}_{2}}\hat{P}_{2}^{(2)}\left(1\right)\hat{F}_{2}^{(0,1)}+\hat{\mu}_{2}^{2}\hat{\theta}_{2}^{(2)}\left(1\right)\hat{F}_{2}^{(0,1)}\\
&+&\left(\frac{\hat{\mu}_{2}}{1-\hat{\mu}_{2}}\right)^{2}\hat{F}_{2}^{(0,2)}.
\end{eqnarray*}


\end{enumerate}



%___________________________________________________________________________________________
%
%\subsection{Derivadas de Segundo Orden para $\hat{F}_{2}$}
%___________________________________________________________________________________________
\begin{enumerate}
%___________________________________________________________________________________________
%\subsubsection{Mixtas para $z_{1}$:}
%___________________________________________________________________________________________
%1/49

\item \begin{eqnarray*} &&\frac{\partial}{\partial
z_1}\frac{\partial}{\partial
z_1}\left(\hat{R}_{1}\left(P_{1}\left(z_{1}\right)\tilde{P}_{2}\left(z_{2}\right)\hat{P}_{1}\left(w_{1}\right)\hat{P}_{2}\left(w_{2}\right)\right)\hat{F}_{1}\left(\hat{\theta}_{1}\left(P_{1}\left(z_{1}\right)\tilde{P}_{2}\left(z_{2}\right)
\hat{P}_{2}\left(w_{2}\right)\right),w_{2}\right)F_{1}\left(z_{1},z_{2}\right)\right)\\
&=&\hat{r}_{1}P_{1}^{(2)}\left(1\right)+
\mu_{1}^{2}\hat{R}_{1}^{(2)}\left(1\right)+
2\hat{r}_{1}\mu_{1}F_{1}^{(1,0)}+
2\hat{r}_{1}\frac{\mu_{1}^{2}}{1-\hat{\mu}_{1}}\hat{F}_{1}^{(1,0)}+
\frac{1}{1-\hat{\mu}_{1}}P_{1}^{(2)}\left(1\right)\hat{F}_{1}^{(1,0)}+\mu_{1}^{2}\hat{\theta}_{1}^{(2)}\left(1\right)\hat{F}_{1}^{(1,0)}\\
&+&2\frac{\mu_{1}}{1-\hat{\mu}_{1}}\hat{F}_{1}^{(1,0)}F_{1}^{(1,0)}+F_{1}^{(2,0)}
+\left(\frac{\mu_{1}}{1-\hat{\mu}_{1}}\right)^{2}\hat{F}_{1}^{(2,0)}.
\end{eqnarray*}

%2/50

\item \begin{eqnarray*} &&\frac{\partial}{\partial
z_2}\frac{\partial}{\partial
z_1}\left(\hat{R}_{1}\left(P_{1}\left(z_{1}\right)\tilde{P}_{2}\left(z_{2}\right)\hat{P}_{1}\left(w_{1}\right)\hat{P}_{2}\left(w_{2}\right)\right)\hat{F}_{1}\left(\hat{\theta}_{1}\left(P_{1}\left(z_{1}\right)\tilde{P}_{2}\left(z_{2}\right)
\hat{P}_{2}\left(w_{2}\right)\right),w_{2}\right)F_{1}\left(z_{1},z_{2}\right)\right)\\
&=&\hat{r}_{1}\mu_{1}\tilde{\mu}_{2}+\mu_{1}\tilde{\mu}_{2}\hat{R}_{1}^{(2)}\left(1\right)+
\hat{r}_{1}\mu_{1}F_{1}^{(0,1)}+\tilde{\mu}_{2}\hat{r}_{1}F_{1}^{(1,0)}+
\frac{\mu_{1}\tilde{\mu}_{2}}{1-\hat{\mu}_{1}}\hat{F}_{1}^{(1,0)}+2\hat{r}_{1}\frac{\mu_{1}\tilde{\mu}_{2}}{1-\hat{\mu}_{1}}\hat{F}_{1}^{(1,0)}\\
&+&\mu_{1}\tilde{\mu}_{2}\hat{\theta}_{1}^{(2)}\left(1\right)\hat{F}_{1}^{(1,0)}+
\frac{\mu_{1}}{1-\hat{\mu}_{1}}\hat{F}_{1}^{(1,0)}F_{1}^{(0,1)}+
\frac{\tilde{\mu}_{2}}{1-\hat{\mu}_{1}}\hat{F}_{1}^{(1,0)}F_{1}^{(1,0)}+
F_{1}^{(1,1)}\\
&+&\mu_{1}\tilde{\mu}_{2}\left(\frac{1}{1-\hat{\mu}_{1}}\right)^{2}\hat{F}_{1}^{(2,0)}.
\end{eqnarray*}

%3/51

\item \begin{eqnarray*} &&\frac{\partial}{\partial
w_1}\frac{\partial}{\partial
z_1}\left(\hat{R}_{1}\left(P_{1}\left(z_{1}\right)\tilde{P}_{2}\left(z_{2}\right)\hat{P}_{1}\left(w_{1}\right)\hat{P}_{2}\left(w_{2}\right)\right)\hat{F}_{1}\left(\hat{\theta}_{1}\left(P_{1}\left(z_{1}\right)\tilde{P}_{2}\left(z_{2}\right)
\hat{P}_{2}\left(w_{2}\right)\right),w_{2}\right)F_{1}\left(z_{1},z_{2}\right)\right)\\
&=&\hat{r}_{1}\mu_{1}\hat{\mu}_{1}+\mu_{1}\hat{\mu}_{1}\hat{R}_{1}^{(2)}\left(1\right)+\hat{r}_{1}\hat{\mu}_{1}F_{1}^{(1,0)}+
\hat{r}_{1}\frac{\mu_{1}\hat{\mu}_{1}}{1-\hat{\mu}_{1}}\hat{F}_{1}^{(1,0)}.
\end{eqnarray*}

%4/52

\item \begin{eqnarray*} &&\frac{\partial}{\partial
w_2}\frac{\partial}{\partial
z_1}\left(\hat{R}_{1}\left(P_{1}\left(z_{1}\right)\tilde{P}_{2}\left(z_{2}\right)\hat{P}_{1}\left(w_{1}\right)\hat{P}_{2}\left(w_{2}\right)\right)\hat{F}_{1}\left(\hat{\theta}_{1}\left(P_{1}\left(z_{1}\right)\tilde{P}_{2}\left(z_{2}\right)
\hat{P}_{2}\left(w_{2}\right)\right),w_{2}\right)F_{1}\left(z_{1},z_{2}\right)\right)\\
&=&\hat{r}_{1}\mu_{1}\hat{\mu}_{2}+\mu_{1}\hat{\mu}_{2}\hat{R}_{1}^{(2)}\left(1\right)+\hat{r}_{1}\hat{\mu}_{2}F_{1}^{(1,0)}+\frac{\mu_{1}\hat{\mu}_{2}}{1-\hat{\mu}_{1}}\hat{F}_{1}^{(1,0)}+\hat{r}_{1}\frac{\mu_{1}\hat{\mu}_{2}}{1-\hat{\mu}_{1}}\hat{F}_{1}^{(1,0)}+\mu_{1}\hat{\mu}_{2}\hat{\theta}_{1}^{(2)}\left(1\right)\hat{F}_{1}^{(1,0)}\\
&+&\hat{r}_{1}\mu_{1}\left(\hat{F}_{1}^{(0,1)}+\frac{\hat{\mu}_{2}}{1-\hat{\mu}_{1}}\hat{F}_{1}^{(1,0)}\right)+F_{1}^{(1,0)}\left(\hat{F}_{1}^{(0,1)}+\frac{\hat{\mu}_{2}}{1-\hat{\mu}_{1}}\hat{F}_{1}^{(1,0)}\right)+\frac{\mu_{1}}{1-\hat{\mu}_{1}}\left(\hat{F}_{1}^{(1,1)}+\frac{\hat{\mu}_{2}}{1-\hat{\mu}_{1}}\hat{F}_{1}^{(2,0)}\right).
\end{eqnarray*}
%___________________________________________________________________________________________
%\subsubsection{Mixtas para $z_{2}$:}
%___________________________________________________________________________________________
%5/53

\item \begin{eqnarray*} &&\frac{\partial}{\partial
z_1}\frac{\partial}{\partial
z_2}\left(\hat{R}_{1}\left(P_{1}\left(z_{1}\right)\tilde{P}_{2}\left(z_{2}\right)\hat{P}_{1}\left(w_{1}\right)\hat{P}_{2}\left(w_{2}\right)\right)\hat{F}_{1}\left(\hat{\theta}_{1}\left(P_{1}\left(z_{1}\right)\tilde{P}_{2}\left(z_{2}\right)
\hat{P}_{2}\left(w_{2}\right)\right),w_{2}\right)F_{1}\left(z_{1},z_{2}\right)\right)\\
&=&\hat{r}_{1}\mu_{1}\tilde{\mu}_{2}+\mu_{1}\tilde{\mu}_{2}\hat{R}_{1}^{(2)}\left(1\right)+\hat{r}_{1}\mu_{1}F_{1}^{(0,1)}+\hat{r}_{1}\tilde{\mu}_{2}F_{1}^{(1,0)}+\frac{\mu_{1}\tilde{\mu}_{2}}{1-\hat{\mu}_{1}}\hat{F}_{1}^{(1,0)}+2\hat{r}_{1}\frac{\mu_{1}\tilde{\mu}_{2}}{1-\hat{\mu}_{1}}\hat{F}_{1}^{(1,0)}\\
&+&\mu_{1}\tilde{\mu}_{2}\hat{\theta}_{1}^{(2)}\left(1\right)\hat{F}_{1}^{(1,0)}+\frac{\mu_{1}}{1-\hat{\mu}_{1}}\hat{F}_{1}^{(1,0)}F_{1}^{(0,1)}+\frac{\tilde{\mu}_{2}}{1-\hat{\mu}_{1}}\hat{F}_{1}^{(1,0)}F_{1}^{(1,0)}+F_{1}^{(1,1)}+\mu_{1}\tilde{\mu}_{2}\left(\frac{1}{1-\hat{\mu}_{1}}\right)^{2}\hat{F}_{1}^{(2,0)}.
\end{eqnarray*}

%6/54
\item \begin{eqnarray*} &&\frac{\partial}{\partial
z_2}\frac{\partial}{\partial
z_2}\left(\hat{R}_{1}\left(P_{1}\left(z_{1}\right)\tilde{P}_{2}\left(z_{2}\right)\hat{P}_{1}\left(w_{1}\right)\hat{P}_{2}\left(w_{2}\right)\right)\hat{F}_{1}\left(\hat{\theta}_{1}\left(P_{1}\left(z_{1}\right)\tilde{P}_{2}\left(z_{2}\right)
\hat{P}_{2}\left(w_{2}\right)\right),w_{2}\right)F_{1}\left(z_{1},z_{2}\right)\right)\\
&=&\hat{r}_{1}\tilde{P}_{2}^{(2)}\left(1\right)+\tilde{\mu}_{2}^{2}\hat{R}_{1}^{(2)}\left(1\right)+2\hat{r}_{1}\tilde{\mu}_{2}F_{1}^{(0,1)}+ F_{1}^{(0,2)}+2\hat{r}_{1}\frac{\tilde{\mu}_{2}^{2}}{1-\hat{\mu}_{1}}\hat{F}_{1}^{(1,0)}+\frac{1}{1-\hat{\mu}_{1}}\tilde{P}_{2}^{(2)}\left(1\right)\hat{F}_{1}^{(1,0)}\\
&+&\tilde{\mu}_{2}^{2}\hat{\theta}_{1}^{(2)}\left(1\right)\hat{F}_{1}^{(1,0)}+2\frac{\tilde{\mu}_{2}}{1-\hat{\mu}_{1}}F^{(0,1)}\hat{F}_{1}^{(1,0)}+\left(\frac{\tilde{\mu}_{2}}{1-\hat{\mu}_{1}}\right)^{2}\hat{F}_{1}^{(2,0)}.
\end{eqnarray*}
%7/55

\item \begin{eqnarray*} &&\frac{\partial}{\partial
w_1}\frac{\partial}{\partial
z_2}\left(\hat{R}_{1}\left(P_{1}\left(z_{1}\right)\tilde{P}_{2}\left(z_{2}\right)\hat{P}_{1}\left(w_{1}\right)\hat{P}_{2}\left(w_{2}\right)\right)\hat{F}_{1}\left(\hat{\theta}_{1}\left(P_{1}\left(z_{1}\right)\tilde{P}_{2}\left(z_{2}\right)
\hat{P}_{2}\left(w_{2}\right)\right),w_{2}\right)F_{1}\left(z_{1},z_{2}\right)\right)\\
&=&\hat{r}_{1}\hat{\mu}_{1}\tilde{\mu}_{2}+\hat{\mu}_{1}\tilde{\mu}_{2}\hat{R}_{1}^{(2)}\left(1\right)+
\hat{r}_{1}\hat{\mu}_{1}F_{1}^{(0,1)}+\hat{r}_{1}\frac{\hat{\mu}_{1}\tilde{\mu}_{2}}{1-\hat{\mu}_{1}}\hat{F}_{1}^{(1,0)}.
\end{eqnarray*}
%8/56

\item \begin{eqnarray*} &&\frac{\partial}{\partial
w_2}\frac{\partial}{\partial
z_2}\left(\hat{R}_{1}\left(P_{1}\left(z_{1}\right)\tilde{P}_{2}\left(z_{2}\right)\hat{P}_{1}\left(w_{1}\right)\hat{P}_{2}\left(w_{2}\right)\right)\hat{F}_{1}\left(\hat{\theta}_{1}\left(P_{1}\left(z_{1}\right)\tilde{P}_{2}\left(z_{2}\right)
\hat{P}_{2}\left(w_{2}\right)\right),w_{2}\right)F_{1}\left(z_{1},z_{2}\right)\right)\\
&=&\hat{r}_{1}\tilde{\mu}_{2}\hat{\mu}_{2}+\hat{\mu}_{2}\tilde{\mu}_{2}\hat{R}_{1}^{(2)}\left(1\right)+\hat{\mu}_{2}\hat{R}_{1}^{(2)}\left(1\right)F_{1}^{(0,1)}+\frac{\hat{\mu}_{2}\tilde{\mu}_{2}}{1-\hat{\mu}_{1}}\hat{F}_{1}^{(1,0)}+
\hat{r}_{1}\frac{\hat{\mu}_{2}\tilde{\mu}_{2}}{1-\hat{\mu}_{1}}\hat{F}_{1}^{(1,0)}\\
&+&\hat{\mu}_{2}\tilde{\mu}_{2}\hat{\theta}_{1}^{(2)}\left(1\right)\hat{F}_{1}^{(1,0)}+\hat{r}_{1}\tilde{\mu}_{2}\left(\hat{F}_{1}^{(0,1)}+\frac{\hat{\mu}_{2}}{1-\hat{\mu}_{1}}\hat{F}_{1}^{(1,0)}\right)+F_{1}^{(0,1)}\left(\hat{F}_{1}^{(0,1)}+\frac{\hat{\mu}_{2}}{1-\hat{\mu}_{1}}\hat{F}_{1}^{(1,0)}\right)\\
&+&\frac{\tilde{\mu}_{2}}{1-\hat{\mu}_{1}}\left(\hat{F}_{1}^{(1,1)}+\frac{\hat{\mu}_{2}}{1-\hat{\mu}_{1}}\hat{F}_{1}^{(2,0)}\right).
\end{eqnarray*}
%___________________________________________________________________________________________
%\subsubsection{Mixtas para $w_{1}$:}
%___________________________________________________________________________________________
%9/57
\item \begin{eqnarray*} &&\frac{\partial}{\partial
z_1}\frac{\partial}{\partial
w_1}\left(\hat{R}_{1}\left(P_{1}\left(z_{1}\right)\tilde{P}_{2}\left(z_{2}\right)\hat{P}_{1}\left(w_{1}\right)\hat{P}_{2}\left(w_{2}\right)\right)\hat{F}_{1}\left(\hat{\theta}_{1}\left(P_{1}\left(z_{1}\right)\tilde{P}_{2}\left(z_{2}\right)
\hat{P}_{2}\left(w_{2}\right)\right),w_{2}\right)F_{1}\left(z_{1},z_{2}\right)\right)\\
&=&\hat{r}_{1}\mu_{1}\hat{\mu}_{1}+\mu_{1}\hat{\mu}_{1}\hat{R}_{1}^{(2)}\left(1\right)+\hat{r}_{1}\hat{\mu}_{1}F_{1}^{(1,0)}+\hat{r}_{1}\frac{\mu_{1}\hat{\mu}_{1}}{1-\hat{\mu}_{1}}\hat{F}_{1}^{(1,0)}.
\end{eqnarray*}
%10/58
\item \begin{eqnarray*} &&\frac{\partial}{\partial
z_2}\frac{\partial}{\partial
w_1}\left(\hat{R}_{1}\left(P_{1}\left(z_{1}\right)\tilde{P}_{2}\left(z_{2}\right)\hat{P}_{1}\left(w_{1}\right)\hat{P}_{2}\left(w_{2}\right)\right)\hat{F}_{1}\left(\hat{\theta}_{1}\left(P_{1}\left(z_{1}\right)\tilde{P}_{2}\left(z_{2}\right)
\hat{P}_{2}\left(w_{2}\right)\right),w_{2}\right)F_{1}\left(z_{1},z_{2}\right)\right)\\
&=&\hat{r}_{1}\tilde{\mu}_{2}\hat{\mu}_{1}+\tilde{\mu}_{2}\hat{\mu}_{1}\hat{R}_{1}^{(2)}\left(1\right)+\hat{r}_{1}\hat{\mu}_{1}F_{1}^{(0,1)}+\hat{r}_{1}\frac{\tilde{\mu}_{2}\hat{\mu}_{1}}{1-\hat{\mu}_{1}}\hat{F}_{1}^{(1,0)}.
\end{eqnarray*}
%11/59
\item \begin{eqnarray*} &&\frac{\partial}{\partial
w_1}\frac{\partial}{\partial
w_1}\left(\hat{R}_{1}\left(P_{1}\left(z_{1}\right)\tilde{P}_{2}\left(z_{2}\right)\hat{P}_{1}\left(w_{1}\right)\hat{P}_{2}\left(w_{2}\right)\right)\hat{F}_{1}\left(\hat{\theta}_{1}\left(P_{1}\left(z_{1}\right)\tilde{P}_{2}\left(z_{2}\right)
\hat{P}_{2}\left(w_{2}\right)\right),w_{2}\right)F_{1}\left(z_{1},z_{2}\right)\right)\\
&=&\hat{r}_{1}\hat{P}_{1}^{(2)}\left(1\right)+\hat{\mu}_{1}^{2}\hat{R}_{1}^{(2)}\left(1\right).
\end{eqnarray*}
%12/60
\item \begin{eqnarray*} &&\frac{\partial}{\partial
w_2}\frac{\partial}{\partial
w_1}\left(\hat{R}_{1}\left(P_{1}\left(z_{1}\right)\tilde{P}_{2}\left(z_{2}\right)\hat{P}_{1}\left(w_{1}\right)\hat{P}_{2}\left(w_{2}\right)\right)\hat{F}_{1}\left(\hat{\theta}_{1}\left(P_{1}\left(z_{1}\right)\tilde{P}_{2}\left(z_{2}\right)
\hat{P}_{2}\left(w_{2}\right)\right),w_{2}\right)F_{1}\left(z_{1},z_{2}\right)\right)\\
&=&\hat{r}_{1}\hat{\mu}_{2}\hat{\mu}_{1}+\hat{\mu}_{2}\hat{\mu}_{1}\hat{R}_{1}^{(2)}\left(1\right)+\hat{r}_{1}\hat{\mu}_{1}\left(\hat{F}_{1}^{(0,1)}+\frac{\hat{\mu}_{2}}{1-\hat{\mu}_{1}}\hat{F}_{1}^{(1,0)}\right).
\end{eqnarray*}
%___________________________________________________________________________________________
%\subsubsection{Mixtas para $w_{1}$:}
%___________________________________________________________________________________________
%13/61



\item \begin{eqnarray*} &&\frac{\partial}{\partial
z_1}\frac{\partial}{\partial
w_2}\left(\hat{R}_{1}\left(P_{1}\left(z_{1}\right)\tilde{P}_{2}\left(z_{2}\right)\hat{P}_{1}\left(w_{1}\right)\hat{P}_{2}\left(w_{2}\right)\right)\hat{F}_{1}\left(\hat{\theta}_{1}\left(P_{1}\left(z_{1}\right)\tilde{P}_{2}\left(z_{2}\right)
\hat{P}_{2}\left(w_{2}\right)\right),w_{2}\right)F_{1}\left(z_{1},z_{2}\right)\right)\\
&=&\hat{r}_{1}\mu_{1}\hat{\mu}_{2}+\mu_{1}\hat{\mu}_{2}\hat{R}_{1}^{(2)}\left(1\right)+\hat{r}_{1}\hat{\mu}_{2}F_{1}^{(1,0)}+
\hat{r}_{1}\frac{\mu_{1}\hat{\mu}_{2}}{1-\hat{\mu}_{1}}\hat{F}_{1}^{(1,0)}+\hat{r}_{1}\mu_{1}\left(\hat{F}_{1}^{(0,1)}+\frac{\hat{\mu}_{2}}{1-\hat{\mu}_{1}}\hat{F}_{1}^{(1,0)}\right)\\
&+&F_{1}^{(1,0)}\left(\hat{F}_{1}^{(0,1)}+\frac{\hat{\mu}_{2}}{1-\hat{\mu}_{1}}\hat{F}_{1}^{(1,0)}\right)+\frac{\mu_{1}\hat{\mu}_{2}}{1-\hat{\mu}_{1}}\hat{F}_{1}^{(1,0)}+\mu_{1}\hat{\mu}_{2}\hat{\theta}_{1}^{(2)}\left(1\right)\hat{F}_{1}^{(1,0)}+\frac{\mu_{1}}{1-\hat{\mu}_{1}}\hat{F}_{1}^{(1,1)}\\
&+&\mu_{1}\hat{\mu}_{2}\left(\frac{1}{1-\hat{\mu}_{1}}\right)^{2}\hat{F}_{1}^{(2,0)}.
\end{eqnarray*}

%14/62
\item \begin{eqnarray*} &&\frac{\partial}{\partial
z_2}\frac{\partial}{\partial
w_2}\left(\hat{R}_{1}\left(P_{1}\left(z_{1}\right)\tilde{P}_{2}\left(z_{2}\right)\hat{P}_{1}\left(w_{1}\right)\hat{P}_{2}\left(w_{2}\right)\right)\hat{F}_{1}\left(\hat{\theta}_{1}\left(P_{1}\left(z_{1}\right)\tilde{P}_{2}\left(z_{2}\right)
\hat{P}_{2}\left(w_{2}\right)\right),w_{2}\right)F_{1}\left(z_{1},z_{2}\right)\right)\\
&=&\hat{r}_{1}\tilde{\mu}_{2}\hat{\mu}_{2}+\tilde{\mu}_{2}\hat{\mu}_{2}\hat{R}_{1}^{(2)}\left(1\right)+\hat{r}_{1}\hat{\mu}_{2}F_{1}^{(0,1)}+\hat{r}_{1}\frac{\tilde{\mu}_{2}\hat{\mu}_{2}}{1-\hat{\mu}_{1}}\hat{F}_{1}^{(1,0)}+\hat{r}_{1}\tilde{\mu}_{2}\left(\hat{F}_{1}^{(0,1)}+\frac{\hat{\mu}_{2}}{1-\hat{\mu}_{1}}\hat{F}_{1}^{(1,0)}\right)\\
&+&F_{1}^{(0,1)}\left(\hat{F}_{1}^{(0,1)}+\frac{\hat{\mu}_{2}}{1-\hat{\mu}_{1}}\hat{F}_{1}^{(1,0)}\right)+\frac{\tilde{\mu}_{2}\hat{\mu}_{2}}{1-\hat{\mu}_{1}}\hat{F}_{1}^{(1,0)}+\tilde{\mu}_{2}\hat{\mu}_{2}\hat{\theta}_{1}^{(2)}\left(1\right)\hat{F}_{1}^{(1,0)}+\frac{\tilde{\mu}_{2}}{1-\hat{\mu}_{1}}\hat{F}_{1}^{(1,1)}\\
&+&\tilde{\mu}_{2}\hat{\mu}_{2}\left(\frac{1}{1-\hat{\mu}_{1}}\right)^{2}\hat{F}_{1}^{(2,0)}.
\end{eqnarray*}

%15/63

\item \begin{eqnarray*} &&\frac{\partial}{\partial
w_1}\frac{\partial}{\partial
w_2}\left(\hat{R}_{1}\left(P_{1}\left(z_{1}\right)\tilde{P}_{2}\left(z_{2}\right)\hat{P}_{1}\left(w_{1}\right)\hat{P}_{2}\left(w_{2}\right)\right)\hat{F}_{1}\left(\hat{\theta}_{1}\left(P_{1}\left(z_{1}\right)\tilde{P}_{2}\left(z_{2}\right)
\hat{P}_{2}\left(w_{2}\right)\right),w_{2}\right)F_{1}\left(z_{1},z_{2}\right)\right)\\
&=&\hat{r}_{1}\hat{\mu}_{2}\hat{\mu}_{1}+\hat{\mu}_{2}\hat{\mu}_{1}\hat{R}_{1}^{(2)}\left(1\right)+\hat{r}_{1}\hat{\mu}_{1}\left(\hat{F}_{1}^{(0,1)}+\frac{\hat{\mu}_{2}}{1-\hat{\mu}_{1}}\hat{F}_{1}^{(1,0)}\right).
\end{eqnarray*}

%16/64

\item \begin{eqnarray*} &&\frac{\partial}{\partial
w_2}\frac{\partial}{\partial
w_2}\left(\hat{R}_{1}\left(P_{1}\left(z_{1}\right)\tilde{P}_{2}\left(z_{2}\right)\hat{P}_{1}\left(w_{1}\right)\hat{P}_{2}\left(w_{2}\right)\right)\hat{F}_{1}\left(\hat{\theta}_{1}\left(P_{1}\left(z_{1}\right)\tilde{P}_{2}\left(z_{2}\right)
\hat{P}_{2}\left(w_{2}\right)\right),w_{2}\right)F_{1}\left(z_{1},z_{2}\right)\right)\\
&=&\hat{r}_{1}\hat{P}_{2}^{(2)}\left(1\right)+\hat{\mu}_{2}^{2}\hat{R}_{1}^{(2)}\left(1\right)+
2\hat{r}_{1}\hat{\mu}_{2}\left(\hat{F}_{1}^{(0,1)}+\frac{\hat{\mu}_{2}}{1-\hat{\mu}_{1}}\hat{F}_{1}^{(1,0)}\right)+
\hat{F}_{1}^{(0,2)}+\frac{1}{1-\hat{\mu}_{1}}\hat{P}_{2}^{(2)}\left(1\right)\hat{F}_{1}^{(1,0)}\\
&+&\hat{\mu}_{2}^{2}\hat{\theta}_{1}^{(2)}\left(1\right)\hat{F}_{1}^{(1,0)}+\frac{\hat{\mu}_{2}}{1-\hat{\mu}_{1}}\hat{F}_{1}^{(1,1)}+\frac{\hat{\mu}_{2}}{1-\hat{\mu}_{1}}\left(\hat{F}_{1}^{(1,1)}+\frac{\hat{\mu}_{2}}{1-\hat{\mu}_{1}}\hat{F}_{1}^{(2,0)}\right).
\end{eqnarray*}
%_________________________________________________________________________________________________________
%
%_________________________________________________________________________________________________________

\end{enumerate}



%----------------------------------------------------------------------------------------
%   INTRODUCTION
%----------------------------------------------------------------------------------------



%----------------------------------------------------------------------------------------
%   OBJECTIVES
%----------------------------------------------------------------------------------------





%----------------------------------------------------------------------------------------
%   MATERIALS AND METHODS
%----------------------------------------------------------------------------------------


%------------------------------------------------
%\subsection*{Descripci\'on de la Red de Sistemas de Visitas C\'iclicas}
%------------------------------------------------

%----------------------------------------------------------------------------------------
%   RESULTS
%----------------------------------------------------------------------------------------
\section*{Resultado Principal}
%----------------------------------------------------------------------------------------
Sean $\mu_{1},\mu_{2},\check{\mu}_{2},\hat{\mu}_{1},\hat{\mu}_{2}$ y $\tilde{\mu}_{2}=\mu_{2}+\check{\mu}_{2}$ los valores esperados de los proceso definidos anteriormente, y sean $r_{1},r_{2}, \hat{r}_{1}$ y $\hat{r}_{2}$ los valores esperado s de los tiempos de traslado del servidor entre las colas para cada uno de los sistemas de visitas c\'iclicas. Si se definen $\mu=\mu_{1}+\tilde{\mu}_{2}$, $\hat{\mu}=\hat{\mu}_{1}+\hat{\mu}_{2}$, y $r=r_{1}+r_{2}$ y  $\hat{r}=\hat{r}_{1}+\hat{r}_{2}$, entonces se tiene el siguiente resultado.

\begin{Teo}
Supongamos que $\mu<1$, $\hat{\mu}<1$, entonces, el n\'umero de usuarios presentes en cada una de las colas que conforman la Red de Sistemas de Visitas C\'iclicas cuando uno de los servidores visita a alguna de ellas est\'a dada por la soluci\'on del Sistema de Ecuaciones Lineales presentados arriba cuyas expresiones damos a continuaci\'on:
%{\footnotesize{
\[ \begin{array}{lll}
f_{1}\left(1\right)=r\frac{\mu_{1}\left(1-\mu_{1}\right)}{1-\mu},&f_{1}\left(2\right)=r_{2}\tilde{\mu}_{2},&f_{1}\left(3\right)=\hat{\mu}_{1}\left(\frac{r_{2}\mu_{2}+1}{\mu_{2}}+r\frac{\tilde{\mu}_{2}}{1-\mu}\right),\\
f_{1}\left(4\right)=\hat{\mu}_{2}\left(\frac{r_{2}\mu_{2}+1}{\mu_{2}}+r\frac{\tilde{\mu}_{2}}{1-\mu}\right),&f_{2}\left(1\right)=r_{1}\mu_{1},&f_{2}\left(2\right)=r\frac{\tilde{\mu}_{2}\left(1-\tilde{\mu}_{2}\right)}{1-\mu},\\
f_{2}\left(3\right)=\hat{\mu}_{1}\left(\frac{r_{1}\mu_{1}+1}{\mu_{1}}+r\frac{\mu_{1}}{1-\mu}\right),&f_{2}\left(4\right)=\hat{\mu}_{2}\left(\frac{r_{1}\mu_{1}+1}{\mu_{1}}+r\frac{\mu_{1}}{1-\mu}\right),&\hat{f}_{1}\left(1\right)=\mu_{1}\left(\frac{\hat{r}_{2}\hat{\mu}_{2}+1}{\hat{\mu}_{2}}+\hat{r}\frac{\hat{\mu}_{2}}{1-\hat{\mu}}\right),\\
\hat{f}_{1}\left(2\right)=\tilde{\mu}_{2}\left(\hat{r}_{2}+\hat{r}\frac{\hat{\mu}_{2}}{1-\hat{\mu}}\right)+\frac{\mu_{2}}{\hat{\mu}_{2}},&\hat{f}_{1}\left(3\right)=\hat{r}\frac{\hat{\mu}_{1}\left(1-\hat{\mu}_{1}\right)}{1-\hat{\mu}},&\hat{f}_{1}\left(4\right)=\hat{r}_{2}\hat{\mu}_{2},\\
\hat{f}_{2}\left(1\right)=\mu_{1}\left(\frac{\hat{r}_{1}\hat{\mu}_{1}+1}{\hat{\mu}_{1}}+\hat{r}\frac{\hat{\mu}_{1}}{1-\hat{\mu}}\right),&\hat{f}_{2}\left(2\right)=\tilde{\mu}_{2}\left(\hat{r}_{1}+\hat{r}\frac{\hat{\mu}_{1}}{1-\hat{\mu}}\right)+\frac{\hat{\mu_{2}}}{\hat{\mu}_{1}},&\hat{f}_{2}\left(3\right)=\hat{r}_{1}\hat{\mu}_{1},\\
&\hat{f}_{2}\left(4\right)=\hat{r}\frac{\hat{\mu}_{2}\left(1-\hat{\mu}_{2}\right)}{1-\hat{\mu}}.&\\
\end{array}\] %}}
\end{Teo}


Las ecuaciones que determinan los segundos momentos de las longitudes de las colas de los dos sistemas se pueden ver en \href{http://sitio.expresauacm.org/s/carlosmartinez/wp-content/uploads/sites/13/2014/01/SegundosMomentos.pdf}{este sitio}

%\url{http://ubuntu_es_el_diablo.org},\href{http://www.latex-project.org/}{latex project}

%http://sitio.expresauacm.org/s/carlosmartinez/wp-content/uploads/sites/13/2014/01/SegundosMomentos.jpg
%http://sitio.expresauacm.org/s/carlosmartinez/wp-content/uploads/sites/13/2014/01/SegundosMomentos.pdf




%___________________________________________________________________________________________
%\section*{Tiempos de Ciclo e Intervisita}
%___________________________________________________________________________________________



%----------------------------------------------------------------------------------------
%\section*{Medidas de Desempe\~no de la Red de Sistemas de Visita C\'iclicas}
%----------------------------------------------------------------------------------------
%Se puede demostrar que las expresiones para los tiempos entre visitas de los servidores a las colas

%----------------------------------------------------------------------------------------
%   CONCLUSIONS
%----------------------------------------------------------------------------------------

%\color{SaddleBrown} % SaddleBrown color for the conclusions to make them stand out

\section*{Medidas de Desempe\~no}


\begin{Def}
Sea $L_{i}^{*}$el n\'umero de usuarios cuando el servidor visita la cola $Q_{i}$ para dar servicio, para $i=1,2$.
\end{Def}

Entonces
\begin{Prop} Para la cola $Q_{i}$, $i=1,2$, se tiene que el n\'umero de usuarios presentes al momento de ser visitada por el servidor est\'a dado por
\begin{eqnarray}
\esp\left[L_{i}^{*}\right]&=&f_{i}\left(i\right)\\
Var\left[L_{i}^{*}\right]&=&f_{i}\left(i,i\right)+\esp\left[L_{i}^{*}\right]-\esp\left[L_{i}^{*}\right]^{2}.
\end{eqnarray}
\end{Prop}


\begin{Def}
El tiempo de Ciclo $C_{i}$ es el periodo de tiempo que comienza
cuando la cola $i$ es visitada por primera vez en un ciclo, y
termina cuando es visitado nuevamente en el pr\'oximo ciclo, bajo condiciones de estabilidad.

\begin{eqnarray*}
C_{i}\left(z\right)=\esp\left[z^{\overline{\tau}_{i}\left(m+1\right)-\overline{\tau}_{i}\left(m\right)}\right]
\end{eqnarray*}
\end{Def}

\begin{Def}
El tiempo de intervisita $I_{i}$ es el periodo de tiempo que
comienza cuando se ha completado el servicio en un ciclo y termina
cuando es visitada nuevamente en el pr\'oximo ciclo.
\begin{eqnarray*}I_{i}\left(z\right)&=&\esp\left[z^{\tau_{i}\left(m+1\right)-\overline{\tau}_{i}\left(m\right)}\right]\end{eqnarray*}
\end{Def}

\begin{Prop}
Para los tiempos de intervisita del servidor $I_{i}$, se tiene que

\begin{eqnarray*}
\esp\left[I_{i}\right]&=&\frac{f_{i}\left(i\right)}{\mu_{i}},\\
Var\left[I_{i}\right]&=&\frac{Var\left[L_{i}^{*}\right]}{\mu_{i}^{2}}-\frac{\sigma_{i}^{2}}{\mu_{i}^{2}}f_{i}\left(i\right).
\end{eqnarray*}
\end{Prop}


\begin{Prop}
Para los tiempos que ocupa el servidor para atender a los usuarios presentes en la cola $Q_{i}$, con FGP denotada por $S_{i}$, se tiene que
\begin{eqnarray*}
\esp\left[S_{i}\right]&=&\frac{\esp\left[L_{i}^{*}\right]}{1-\mu_{i}}=\frac{f_{i}\left(i\right)}{1-\mu_{i}},\\
Var\left[S_{i}\right]&=&\frac{Var\left[L_{i}^{*}\right]}{\left(1-\mu_{i}\right)^{2}}+\frac{\sigma^{2}\esp\left[L_{i}^{*}\right]}{\left(1-\mu_{i}\right)^{3}}
\end{eqnarray*}
\end{Prop}


\begin{Prop}
Para la duraci\'on de los ciclos $C_{i}$ se tiene que
\begin{eqnarray*}
\esp\left[C_{i}\right]&=&\esp\left[I_{i}\right]\esp\left[\theta_{i}\left(z\right)\right]=\frac{\esp\left[L_{i}^{*}\right]}{\mu_{i}}\frac{1}{1-\mu_{i}}=\frac{f_{i}\left(i\right)}{\mu_{i}\left(1-\mu_{i}\right)}\\
Var\left[C_{i}\right]&=&\frac{Var\left[L_{i}^{*}\right]}{\mu_{i}^{2}\left(1-\mu_{i}\right)^{2}}.
\end{eqnarray*}

\end{Prop}


%----------------------------------------------------------------------------------------
%   REFERENCES
%----------------------------------------------------------------------------------------
%_________________________________________________________________________
%\section*{REFERENCIAS}
%_________________________________________________________________________

\section*{Conjeturas}
%----------------------------------------------------------------------------------------

\begin{Def}
Dada una cola $Q_{i}$, sea $\mathcal{L}=\left\{L_{1}\left(t\right),L_{2}\left(t\right),\hat{L}_{1}\left(t\right),\hat{L}_{2}\left(t\right)\right\}$ las longitudes de todas las colas de la Red de Sistemas de Visitas C\'iclicas. Sup\'ongase que el servidor visita $Q_{i}$, si $L_{i}\left(t\right)=0$ y $\hat{L}_{i}\left(t\right)=0$ para $i=1,2$, entonces los elementos de $\mathcal{L}$ son puntos regenerativos.
\end{Def}


\begin{Def}
Un ciclo regenerativo es el intervalo de tiempo que ocurre entre dos puntos regenerativos sucesivos, $\mathcal{L}_{1},\mathcal{L}_{2}$.
\end{Def}


Def\'inanse los puntos de regenaraci\'on  en el proceso
$\left[L_{1}\left(t\right),L_{2}\left(t\right),\ldots,L_{N}\left(t\right)\right]$.
Los puntos cuando la cola $i$ es visitada y todos los
$L_{j}\left(\tau_{i}\left(m\right)\right)=0$ para $i=1,2$  son
puntos de regeneraci\'on. Se llama ciclo regenerativo al intervalo
entre dos puntos regenerativos sucesivos.

Sea $M_{i}$  el n\'umero de ciclos de visita en un ciclo
regenerativo, y sea $C_{i}^{(m)}$, para $m=1,2,\ldots,M_{i}$ la
duraci\'on del $m$-\'esimo ciclo de visita en un ciclo
regenerativo. Se define el ciclo del tiempo de visita promedio
$\esp\left[C_{i}\right]$ como
\begin{eqnarray*}
\esp\left[C_{i}\right]&=&\frac{\esp\left[\sum_{m=1}^{M_{i}}C_{i}^{(m)}\right]}{\esp\left[M_{i}\right]}
\end{eqnarray*}


En Stid72 y Heym82 se muestra que una condici\'on suficiente para
que el proceso regenerativo estacionario sea un procesoo
estacionario es que el valor esperado del tiempo del ciclo
regenerativo sea finito:

\begin{eqnarray*}
\esp\left[\sum_{m=1}^{M_{i}}C_{i}^{(m)}\right]<\infty.
\end{eqnarray*}



como cada $C_{i}^{(m)}$ contiene intervalos de r\'eplica
positivos, se tiene que $\esp\left[M_{i}\right]<\infty$, adem\'as,
como $M_{i}>0$, se tiene que la condici\'on anterior es
equivalente a tener que

\begin{eqnarray*}
\esp\left[C_{i}\right]<\infty,
\end{eqnarray*}
por lo tanto una condici\'on suficiente para la existencia del
proceso regenerativo est\'a dada por
\begin{eqnarray*}
\sum_{k=1}^{N}\mu_{k}<1.
\end{eqnarray*}



Sea la funci\'on generadora de momentos para $L_{i}$, el n\'umero
de usuarios en la cola $Q_{i}\left(z\right)$ en cualquier momento,
est\'a dada por el tiempo promedio de $z^{L_{i}\left(t\right)}$
sobre el ciclo regenerativo definido anteriormente:

\begin{eqnarray*}
Q_{i}\left(z\right)&=&\esp\left[z^{L_{i}\left(t\right)}\right]=\frac{\esp\left[\sum_{m=1}^{M_{i}}\sum_{t=\tau_{i}\left(m\right)}^{\tau_{i}\left(m+1\right)-1}z^{L_{i}\left(t\right)}\right]}{\esp\left[\sum_{m=1}^{M_{i}}\tau_{i}\left(m+1\right)-\tau_{i}\left(m\right)\right]}
\end{eqnarray*}


$M_{i}$ es un tiempo de paro en el proceso regenerativo con
$\esp\left[M_{i}\right]<\infty$, se sigue del lema de Wald que:


\begin{eqnarray*}
\esp\left[\sum_{m=1}^{M_{i}}\sum_{t=\tau_{i}\left(m\right)}^{\tau_{i}\left(m+1\right)-1}z^{L_{i}\left(t\right)}\right]&=&\esp\left[M_{i}\right]\esp\left[\sum_{t=\tau_{i}\left(m\right)}^{\tau_{i}\left(m+1\right)-1}z^{L_{i}\left(t\right)}\right]\\
\esp\left[\sum_{m=1}^{M_{i}}\tau_{i}\left(m+1\right)-\tau_{i}\left(m\right)\right]&=&\esp\left[M_{i}\right]\esp\left[\tau_{i}\left(m+1\right)-\tau_{i}\left(m\right)\right]
\end{eqnarray*}

por tanto se tiene que


\begin{eqnarray*}
Q_{i}\left(z\right)&=&\frac{\esp\left[\sum_{t=\tau_{i}\left(m\right)}^{\tau_{i}\left(m+1\right)-1}z^{L_{i}\left(t\right)}\right]}{\esp\left[\tau_{i}\left(m+1\right)-\tau_{i}\left(m\right)\right]}
\end{eqnarray*}

observar que el denominador es simplemente la duraci\'on promedio
del tiempo del ciclo.




Se puede demostrar (ver Hideaki Takagi 1986) que

\begin{eqnarray*}
\esp\left[\sum_{t=\tau_{i}\left(m\right)}^{\tau_{i}\left(m+1\right)-1}z^{L_{i}\left(t\right)}\right]=z\frac{F_{i}\left(z\right)-1}{z-P_{i}\left(z\right)}
\end{eqnarray*}

Durante el tiempo de intervisita para la cola $i$,
$L_{i}\left(t\right)$ solamente se incrementa de manera que el
incremento por intervalo de tiempo est\'a dado por la funci\'on
generadora de probabilidades de $P_{i}\left(z\right)$, por tanto
la suma sobre el tiempo de intervisita puede evaluarse como:

\begin{eqnarray*}
\esp\left[\sum_{t=\tau_{i}\left(m\right)}^{\tau_{i}\left(m+1\right)-1}z^{L_{i}\left(t\right)}\right]&=&\esp\left[\sum_{t=\tau_{i}\left(m\right)}^{\tau_{i}\left(m+1\right)-1}\left\{P_{i}\left(z\right)\right\}^{t-\overline{\tau}_{i}\left(m\right)}\right]\\
&=&\frac{1-\esp\left[\left\{P_{i}\left(z\right)\right\}^{\tau_{i}\left(m+1\right)-\overline{\tau}_{i}\left(m\right)}\right]}{1-P_{i}\left(z\right)}=\frac{1-I_{i}\left[P_{i}\left(z\right)\right]}{1-P_{i}\left(z\right)}
\end{eqnarray*}
por tanto



\begin{eqnarray*}
\esp\left[\sum_{t=\tau_{i}\left(m\right)}^{\tau_{i}\left(m+1\right)-1}z^{L_{i}\left(t\right)}\right]&=&\frac{1-F_{i}\left(z\right)}{1-P_{i}\left(z\right)}
\end{eqnarray*}


Haciendo uso de lo hasta ahora desarrollado se tiene que

\begin{eqnarray*}
Q_{i}\left(z\right)&=&\frac{1}{\esp\left[C_{i}\right]}\cdot\frac{1-F_{i}\left(z\right)}{P_{i}\left(z\right)-z}\cdot\frac{\left(1-z\right)P_{i}\left(z\right)}{1-P_{i}\left(z\right)}\\
&=&\frac{\mu_{i}\left(1-\mu_{i}\right)}{f_{i}\left(i\right)}\cdot\frac{1-F_{i}\left(z\right)}{P_{i}\left(z\right)-z}\cdot\frac{\left(1-z\right)P_{i}\left(z\right)}{1-P_{i}\left(z\right)}
\end{eqnarray*}

derivando con respecto a $z$




\begin{eqnarray*}
\frac{d Q_{i}\left(z\right)}{d z}&=&\frac{\left(1-F_{i}\left(z\right)\right)P_{i}\left(z\right)}{\esp\left[C_{i}\right]\left(1-P_{i}\left(z\right)\right)\left(P_{i}\left(z\right)-z\right)}\\
&-&\frac{\left(1-z\right)P_{i}\left(z\right)F_{i}^{'}\left(z\right)}{\esp\left[C_{i}\right]\left(1-P_{i}\left(z\right)\right)\left(P_{i}\left(z\right)-z\right)}\\
&-&\frac{\left(1-z\right)\left(1-F_{i}\left(z\right)\right)P_{i}\left(z\right)\left(P_{i}^{'}\left(z\right)-1\right)}{\esp\left[C_{i}\right]\left(1-P_{i}\left(z\right)\right)\left(P_{i}\left(z\right)-z\right)^{2}}\\
&+&\frac{\left(1-z\right)\left(1-F_{i}\left(z\right)\right)P_{i}^{'}\left(z\right)}{\esp\left[C_{i}\right]\left(1-P_{i}\left(z\right)\right)\left(P_{i}\left(z\right)-z\right)}\\
&+&\frac{\left(1-z\right)\left(1-F_{i}\left(z\right)\right)P_{i}\left(z\right)P_{i}^{'}\left(z\right)}{\esp\left[C_{i}\right]\left(1-P_{i}\left(z\right)\right)^{2}\left(P_{i}\left(z\right)-z\right)}
\end{eqnarray*}

%______________________________________________________



Calculando el l\'imite cuando $z\rightarrow1^{+}$:
\begin{eqnarray}
Q_{i}^{(1)}\left(z\right)&=&lim_{z\rightarrow1^{+}}\frac{d Q_{i}\left(z\right)}{dz}\\
&=&lim_{z\rightarrow1}\frac{\left(1-F_{i}\left(z\right)\right)P_{i}\left(z\right)}{\esp\left[C_{i}\right]\left(1-P_{i}\left(z\right)\right)\left(P_{i}\left(z\right)-z\right)}\\
&-&lim_{z\rightarrow1^{+}}\frac{\left(1-z\right)P_{i}\left(z\right)F_{i}^{'}\left(z\right)}{\esp\left[C_{i}\right]\left(1-P_{i}\left(z\right)\right)\left(P_{i}\left(z\right)-z\right)}\\
&-&lim_{z\rightarrow1^{+}}\frac{\left(1-z\right)\left(1-F_{i}\left(z\right)\right)P_{i}\left(z\right)\left(P_{i}^{'}\left(z\right)-1\right)}{\esp\left[C_{i}\right]\left(1-P_{i}\left(z\right)\right)\left(P_{i}\left(z\right)-z\right)^{2}}\\
&+&lim_{z\rightarrow1^{+}}\frac{\left(1-z\right)\left(1-F_{i}\left(z\right)\right)P_{i}^{'}\left(z\right)}{\esp\left[C_{i}\right]\left(1-P_{i}\left(z\right)\right)\left(P_{i}\left(z\right)-z\right)}\\
&+&lim_{z\rightarrow1^{+}}\frac{\left(1-z\right)\left(1-F_{i}\left(nz\right)\right)P_{i}\left(z\right)P_{i}^{'}\left(z\right)}{\esp\left[C_{i}\right]\left(1-P_{i}\left(z\right)\right)^{2}\left(P_{i}\left(z\right)-z\right)}
\end{eqnarray}

Entonces:



\begin{eqnarray*}
&&lim_{z\rightarrow1^{+}}\frac{\left(1-F_{i}\left(z\right)\right)P_{i}\left(z\right)}{\left(1-P_{i}\left(z\right)\right)\left(P_{i}\left(z\right)-z\right)}=lim_{z\rightarrow1^{+}}\frac{\frac{d}{dz}\left[\left(1-F_{i}\left(z\right)\right)P_{i}\left(z\right)\right]}{\frac{d}{dz}\left[\left(1-P_{i}\left(z\right)\right)\left(-z+P_{i}\left(z\right)\right)\right]}\\
&=&lim_{z\rightarrow1^{+}}\frac{-P_{i}\left(z\right)F_{i}^{'}\left(z\right)+\left(1-F_{i}\left(z\right)\right)P_{i}^{'}\left(z\right)}{\left(1-P_{i}\left(z\right)\right)\left(-1+P_{i}^{'}\left(z\right)\right)-\left(-z+P_{i}\left(z\right)\right)P_{i}^{'}\left(z\right)}
\end{eqnarray*}


\begin{eqnarray*}
&&lim_{z\rightarrow1^{+}}\frac{\left(1-z\right)P_{i}\left(z\right)F_{i}^{'}\left(z\right)}{\left(1-P_{i}\left(z\right)\right)\left(P_{i}\left(z\right)-z\right)}=lim_{z\rightarrow1^{+}}\frac{\frac{d}{dz}\left[\left(1-z\right)P_{i}\left(z\right)F_{i}^{'}\left(z\right)\right]}{\frac{d}{dz}\left[\left(1-P_{i}\left(z\right)\right)\left(P_{i}\left(z\right)-z\right)\right]}\\
&=&lim_{z\rightarrow1^{+}}\frac{-P_{i}\left(z\right)
F_{i}^{'}\left(z\right)+(1-z) F_{i}^{'}\left(z\right)
P_{i}^{'}\left(z\right)+(1-z)
P_{i}\left(z\right)F_{i}^{''}\left(z\right)}{\left(1-P_{i}\left(z\right)\right)\left(-1+P_{i}^{'}\left(z\right)\right)-\left(-z+P_{i}\left(z\right)\right)P_{i}^{'}\left(z\right)}
\end{eqnarray*}

\footnotesize{
\begin{eqnarray*}
&&lim_{z\rightarrow1^{+}}\frac{\left(1-z\right)\left(1-F_{i}\left(z\right)\right)P_{i}\left(z\right)\left(P_{i}^{'}\left(z\right)-1\right)}{\left(1-P_{i}\left(z\right)\right)\left(P_{i}\left(z\right)-z\right)^{2}}\\
&=&lim_{z\rightarrow1^{+}}\frac{\frac{d}{dz}\left[\left(1-z\right)\left(1-F_{i}\left(z\right)\right)P_{i}\left(z\right)\left(P_{i}^{'}\left(z\right)-1\right)\right]}{\frac{d}{dz}\left[\left(1-P_{i}\left(z\right)\right)\left(P_{i}\left(z\right)-z\right)^{2}\right]}\\
&=&lim_{z\rightarrow1^{+}}\frac{-\left(1-F_{i}\left(z\right)\right) P_{i}\left(z\right)\left(-1+P_{i}^{'}\left(z\right)\right)-(1-z) P_{i}\left(z\right)F_{i}^{'}\left(z\right)\left(-1+P_{i}^{'}\left(z\right)\right)}{2\left(1-P_{i}\left(z\right)\right)\left(-z+P_{i}\left(z\right)\right) \left(-1+P_{i}^{'}\left(z\right)\right)-\left(-z+P_{i}\left(z\right)\right)^2 P_{i}^{'}\left(z\right)}\\
&+&lim_{z\rightarrow1^{+}}\frac{+(1-z) \left(1-F_{i}\left(z\right)\right) \left(-1+P_{i}^{'}\left(z\right)\right) P_{i}^{'}\left(z\right)}{{2\left(1-P_{i}\left(z\right)\right)\left(-z+P_{i}\left(z\right)\right) \left(-1+P_{i}^{'}\left(z\right)\right)-\left(-z+P_{i}\left(z\right)\right)^2 P_{i}^{'}\left(z\right)}}\\
&+&lim_{z\rightarrow1^{+}}\frac{+(1-z)
\left(1-F_{i}\left(z\right)\right)
P_{i}\left(z\right)P_{i}^{''}\left(z\right)}{{2\left(1-P_{i}\left(z\right)\right)\left(-z+P_{i}\left(z\right)\right)
\left(-1+P_{i}^{'}\left(z\right)\right)-\left(-z+P_{i}\left(z\right)\right)^2
P_{i}^{'}\left(z\right)}}
\end{eqnarray*}}

\footnotesize{
%______________________________________________________
\begin{eqnarray*}
&&lim_{z\rightarrow1^{+}}\frac{\left(1-z\right)\left(1-F_{i}\left(z\right)\right)P_{i}^{'}\left(z\right)}{\left(1-P_{i}\left(z\right)\right)\left(P_{i}\left(z\right)-z\right)}=lim_{z\rightarrow1^{+}}\frac{\frac{d}{dz}\left[\left(1-z\right)\left(1-F_{i}\left(z\right)\right)P_{i}^{'}\left(z\right)\right]}{\frac{d}{dz}\left[\left(1-P_{i}\left(z\right)\right)\left(P_{i}\left(z\right)-z\right)\right]}\\
&=&lim_{z\rightarrow1^{+}}\frac{-\left(1-F_{i}\left(z\right)\right)
P_{i}^{'}\left(z\right)-(1-z) F_{i}^{'}\left(z\right)
P_{i}^{'}\left(z\right)+(1-z) \left(1-F_{i}\left(z\right)\right)
P_{i}^{''}\left(z\right)}{\left(1-P_{i}\left(z\right)\right)
\left(-1+P_{i}^{'}\left(z\right)\right)-\left(-z+P_{i}\left(z\right)\right)
P_{i}^{'}\left(z\right)}\frac{}{}
\end{eqnarray*}}

\footnotesize{

%______________________________________________________
\begin{eqnarray*}
&&lim_{z\rightarrow1^{+}}\frac{\left(1-z\right)\left(1-F_{i}\left(z\right)\right)P_{i}\left(z\right)P_{i}^{'}\left(z\right)}{\left(1-P_{i}\left(z\right)\right)^{2}\left(P_{i}\left(z\right)-z\right)}\\
&=&lim_{z\rightarrow1^{+}}\frac{\frac{d}{dz}\left[\left(1-z\right)\left(1-F_{i}\left(z\right)\right)P_{i}\left(z\right)P_{i}^{'}\left(z\right)\right]}{\frac{d}{dz}\left[\left(1-P_{i}\left(z\right)\right)^{2}\left(P_{i}\left(z\right)-z\right)\right]}\\
&=&lim_{z\rightarrow1^{+}}\frac{-\left(1-F_{i}\left(z\right)\right) P_{i}\left(z\right) P_{i}^{'}\left(z\right)-(1-z) P_{i}\left(z\right) F_{i}^{'}\left(z\right)P_i'[z]}{\left(1-P_{i}\left(z\right)\right)^2 \left(-1+P_{i}^{'}\left(z\right)\right)-2 \left(1-P_{i}\left(z\right)\right) \left(-z+P_{i}\left(z\right)\right) P_{i}^{'}\left(z\right)}\\
&+&lim_{z\rightarrow1^{+}}\frac{(1-z) \left(1-F_{i}\left(z\right)\right) P_{i}^{'}\left(z\right)^2+(1-z) \left(1-F_{i}\left(z\right)\right) P_{i}\left(z\right) P_{i}^{''}\left(z\right)}{\left(1-P_{i}\left(z\right)\right)^2 \left(-1+P_{i}^{'}\left(z\right)\right)-2 \left(1-P_{i}\left(z\right)\right) \left(-z+P_{i}\left(z\right)\right) P_{i}^{'}\left(z\right)}\\
\end{eqnarray*}}



%___________________________________________________________________________________________
\subsection*{Longitudes de la Cola en cualquier tiempo}
%___________________________________________________________________________________________



Sea
$V_{i}\left(z\right)=\frac{1}{\esp\left[C_{i}\right]}\frac{I_{i}\left(z\right)-1}{z-P_{i}\left(z\right)}$

%{\esp\lef[I_{i}\right]}\frac{1-\mu_{i}}{z-P_{i}\left(z\right)}

\begin{eqnarray*}
\frac{\partial V_{i}\left(z\right)}{\partial
z}&=&\frac{1}{\esp\left[C_{i}\right]}\left[\frac{I_{i}{'}\left(z\right)\left(z-P_{i}\left(z\right)\right)}{z-P_{i}\left(z\right)}-\frac{\left(I_{i}\left(z\right)-1\right)\left(1-P_{i}{'}\left(z\right)\right)}{\left(z-P_{i}\left(z\right)\right)^{2}}\right]
\end{eqnarray*}


La FGP para el tiempo de espera para cualquier usuario en la cola
est\'a dada por:
\[U_{i}\left(z\right)=\frac{1}{\esp\left[C_{i}\right]}\cdot\frac{1-P_{i}\left(z\right)}{z-P_{i}\left(z\right)}\cdot\frac{I_{i}\left(z\right)-1}{1-z}\]

entonces
%\frac{I_{i}\left(z\right)-1}{1-z}
%+\frac{1-P_{i}\left(z\right)}{z-P_{i}\frac{d}{dz}\left(\frac{I_{i}\left(z\right)-1}{1-z}\right)


\footnotesize{
\begin{eqnarray*}
\frac{d}{dz}V_{i}\left(z\right)&=&\frac{1}{\esp\left[C_{i}\right]}\left\{\frac{d}{dz}\left(\frac{1-P_{i}\left(z\right)}{z-P_{i}\left(z\right)}\right)\frac{I_{i}\left(z\right)-1}{1-z}+\frac{1-P_{i}\left(z\right)}{z-P_{i}\left(z\right)}\frac{d}{dz}\left(\frac{I_{i}\left(z\right)-1}{1-z}\right)\right\}\\
&=&\frac{1}{\esp\left[C_{i}\right]}\left\{\frac{-P_{i}\left(z\right)\left(z-P_{i}\left(z\right)\right)-\left(1-P_{i}\left(z\right)\right)\left(1-P_{i}^{'}\left(z\right)\right)}{\left(z-P_{i}\left(z\right)\right)^{2}}\cdot\frac{I_{i}\left(z\right)-1}{1-z}\right\}\\
&+&\frac{1}{\esp\left[C_{i}\right]}\left\{\frac{1-P_{i}\left(z\right)}{z-P_{i}\left(z\right)}\cdot\frac{I_{i}^{'}\left(z\right)\left(1-z\right)+\left(I_{i}\left(z\right)-1\right)}{\left(1-z\right)^{2}}\right\}
\end{eqnarray*}}
\begin{eqnarray*}
\frac{\partial U_{i}\left(z\right)}{\partial z}&=&\frac{(-1+I_{i}[z]) (1-P_{i}[z])}{(1-z)^2 \esp[I_{i}] (z-P_{i}[z])}+\frac{(1-P_{i}[z]) I_{i}^{'}[z]}{(1-z) \esp[I_{i}] (z-P_{i}[z])}\\
&-&\frac{(-1+I_{i}[z]) (1-P_{i}[z])\left(1-P{'}[z]\right)}{(1-z) \esp[I_{i}] (z-P_{i}[z])^2}-\frac{(-1+I_{i}[z]) P_{i}{'}[z]}{(1-z) \esp[I_{i}](z-P_{i}[z])}
\end{eqnarray*}







%_________________________________________________________________________
\section{Sistemas de Visitas C\'iclicas}
%_________________________________________________________________________
\numberwithin{equation}{section}%
%__________________________________________________________________________
\section{Definiciones}
%__________________________________________________________________________

Se considerar\'an intervalos de tiempo de la forma
$\left[t,t+1\right]$. Los usuarios arriban por paquetes de manera
independiente del resto de las colas. Se define el grupo de
usuarios que llegan a cada una de las colas del sistema 1,
caracterizadas por $Q_{1}$ y $Q_{2}$ respectivamente, en el
intervalo de tiempo $\left[t,t+1\right]$ por
$X_{1}\left(t\right),X_{2}\left(t\right)$.



Para cada uno de los procesos anteriores se define su Funci\'on
Generadora de Probabilidades (PGF):

\begin{eqnarray*}
\begin{array}{cc}
P_{1}\left(z_{1}\right)=\esp\left[z_{1}^{X_{1}\left(t\right)}\right], & P_{2}\left(z_{2}\right)=\esp\left[z_{2}^{X_{2}\left(t\right)}\right].\\
\end{array}
\end{eqnarray*}

Con primer momento definidos por

\begin{eqnarray*}
%\begin{array}{cc}
\mu_{1}&=&\esp\left[X_{1}\left(t\right)\right]=P_{1}^{(1)}\left(1\right),\\
\mu_{2}&=&\esp\left[X_{2}\left(t\right)\right]=P_{2}^{(1)}\left(1\right).\\
%\end{array}
\end{eqnarray*}


En lo que respecta al servidor, en t\'erminos de los tiempos de
visita a cada una de las colas, se denotar\'an por
$\tau_{1},\tau_{2}$ para $Q_{1},Q_{2}$ respectivamente; y a los
tiempos en que el servidor termina de atender en las colas
$Q_{1},Q_{2}$, se les denotar\'a por
$\overline{\tau}_{1},\overline{\tau}_{2}$ respectivamente.
Entonces, los tiempos de servicio est\'an dados por las
diferencias
$\overline{\tau}_{1}-\tau_{1},\overline{\tau}_{2}-\tau_{2}$ para
$Q_{1},Q_{2}$. An\'alogamente los tiempos de traslado del servidor
desde el momento en que termina de atender a una cola y llega a la
siguiente para comenzar a dar servicio est\'an dados por
$\tau_{2}-\overline{\tau}_{1},\tau_{1}-\overline{\tau}_{2}$.


La FGP para estos tiempos de traslado est\'an dados por

\begin{eqnarray*}
\begin{array}{cc}
R_{1}\left(z_{1}\right)=\esp\left[z_{1}^{\tau_{2}-\overline{\tau}_{1}}\right],
&
R_{2}\left(z_{2}\right)=\esp\left[z_{2}^{\tau_{1}-\overline{\tau}_{2}}\right],
\end{array}
\end{eqnarray*}

y al igual que como se hizo con anterioridad

\begin{eqnarray*}
\begin{array}{cc}
r_{1}=R_{1}^{(1)}\left(1\right)=\esp\left[\tau_{2}-\overline{\tau}_{1}\right],
&
r_{2}=R_{2}^{(1)}\left(1\right)=\esp\left[\tau_{1}-\overline{\tau}_{2}\right],\\
\end{array}
\end{eqnarray*}


Sean $\alpha_{1},\alpha_{2}$ el n\'umero de usuarios que arriban
en grupo a la cola $Q_{1}$ y $Q_{2}$ respectivamente. Sus PGF's
est\'an definidas como

\begin{eqnarray*}
\begin{array}{cc}
A_{1}\left(z\right)=\esp\left[z^{\alpha_{1}\left(t\right)}\right],&
A_{2}\left(z\right)=\esp\left[z^{\alpha_{2}\left(t\right)}\right].\\
\end{array}
\end{eqnarray*}

Su primer momento est\'a dado por

\begin{eqnarray*}
\begin{array}{cc}
\lambda_{1}=\esp\left[\alpha_{1}\left(t\right)\right]=A_{1}^{(1)}\left(1\right),&
\lambda_{2}=\esp\left[\alpha_{2}\left(t\right)\right]=A_{2}^{(1)}\left(1\right).\\
\end{array}
\end{eqnarray*}


Sean $\beta_{1},\beta_{2}$ el n\'umero de usuarios que arriban en
el grupo $\alpha_{1},\alpha_{2}$ a la cola $Q_{1}$ y $Q_{2}$,
respectivamente, de igual manera se definen sus PGF's

\begin{eqnarray*}
\begin{array}{cc}
B_{1}\left(z\right)=\esp\left[z^{\beta_{1}\left(t\right)}\right],&
B_{2}\left(z\right)=\esp\left[z^{\beta_{2}\left(t\right)}\right],\\
\end{array}
\end{eqnarray*}

con

\begin{eqnarray*}
\begin{array}{cc}
b_{1}=\esp\left[\beta_{1}\left(t\right)\right]=B_{1}^{(1)}\left(1\right),&
b_{2}=\esp\left[\beta_{2}\left(t\right)\right]=B_{2}^{(1)}\left(1\right).\\
\end{array}
\end{eqnarray*}

La distribuci\'on para el n\'umero de grupos que arriban al
sistema en cada una de las colas se definen por:

\begin{eqnarray*}
\begin{array}{cc}
P_{1}\left(z_{1}\right)=A_{1}\left[B_{1}\left(z_{1}\right)\right]=\esp\left[B_{1}\left(z_{1}\right)^{\alpha_{1}\left(t\right)}\right],&
P_{2}\left(z_{1}\right)=A_{1}\left[B_{1}\left(z_{1}\right)\right]=\esp\left[B_{1}\left(z_{1}\right)^{\alpha_{1}\left(t\right)}\right],\\
\end{array}
\end{eqnarray*}

entonces

\begin{eqnarray*}
P_{1}^{(1)}\left(1\right)&=&\esp\left[\alpha_{1}\left(t\right)B_{1}^{(1)}\left(1\right)\right]=B_{1}^{(1)}\left(1\right)\esp\left[\alpha_{1}\left(t\right)\right]=\lambda_{1}b_{1}\\
P_{2}^{(1)}\left(1\right)&=&\esp\left[\alpha_{2}\left(t\right)B_{2}^{(1)}\left(1\right)\right]=B_{2}^{(1)}\left(1\right)\esp\left[\alpha_{2}\left(t\right)\right]=\lambda_{2}b_{2}.\\
\end{eqnarray*}


%_________________________________________________________________________
\section{La ruina del jugador}
%_________________________________________________________________________

Supongamos que se tiene un jugador que cuenta con un capital
inicial de $L_{0}\geq0$ unidades, esta persona realiza una seria
de juegos de manera sucesiva, dichos eventos son independientes e
id\'enticos.

La ganancia en el $n$-\'esimo juego es $X_{n}$ unidades de las
cuales se resta una cuota de 1 unidad por cada juego. Su PGF
est\'a dada por

\begin{eqnarray*}
F\left(z\right)&=&\esp\left[z^{L_{0}}\right]\\
P\left(z\right)&=&\esp\left[z^{X_{n}}\right]\\
\mu&=&\esp\left[X_{n}\right]<1.\\
\end{eqnarray*}

Sea $L_{n}$ el capital remanente despu\'es del $n$-\'esimo juego.
Entonces

\begin{eqnarray*}
L_{n}&=&L_{0}+X_{1}+X_{2}+\cdots+X_{n}-n.
\end{eqnarray*}

La ruina del jugador ocurre despu\'es del $n$-\'esimo juego:

\begin{eqnarray*}
T&=&min\left\{L_{n}=0\right\}
\end{eqnarray*}

Si $L_{0}=0$ entonces claramente $T=0$. En este sentido $T$ es la
longitud del periodo de ocuapci\'on del servidor comenzando con
$L_{0}$ grupos de usuarios que llegan a la cola conforme a un
proceso dado por $P\left(z\right)$.

%\begin{Def}
Sea $g_{n,k}$ la probabilidad del evento de que el jugador no
caiga en la ruina antes del $n$-\'esimo juego, y que tenga un
capital de $k$ unidades antes del $n$-\'esimo juego, es decir

Dada $n\in\left\{1,2,\ldots,\right\}$ y
$k\in\left\{0,1,2,\ldots,\right\}$
\begin{equation}
g_{n,k}=P\left\{L_{j}>0, j=1,\ldots,n, L_{n}=k\right\}
\end{equation}

la cual adem\'as se puede escribir como

\begin{eqnarray}
g_{n,k}&=&P\left\{L_{j}>0, j=1,\ldots,n,
L_{n}=k\right\}=\sum_{j=1}^{k+1}g_{n-1,j}P\left\{X_{n}=k-j+1\right\}
\end{eqnarray}

y
\begin{equation}
g_{0,k}=P\left\{L_{0}=k\right\}
\end{equation}

Se definen las siguientes PGF

\begin{eqnarray*}\label{Eq.3.16.a}
G_{n}\left(z\right)&=&\sum_{k=0}^{\infty}g_{n,k}z^{k},\textrm{
para }n=0,1,\ldots,
\end{eqnarray*}

\begin{equation}\label{Eq.3.16.b}
G\left(z,w\right)=\sum_{n=0}^{\infty}G_{n}\left(z\right)w^{n}
\end{equation}


En particular para $k=0$

\begin{eqnarray*}
g_{n,0}=G_{n}\left(0\right)=P\left\{L_{j}>0,\textrm{ para
}j<n,\textrm{ y }L_{n}=0\right\}=P\left\{T=n\right\},
\end{eqnarray*}

adem\'as

\begin{eqnarray*}
G\left(0,w\right)&=&\sum_{n=0}^{\infty}G_{n}\left(0\right)w^{n}=\sum_{n=0}^{\infty}P\left\{T=n\right\}w^{n}=\esp\left[w^{T}\right]
\end{eqnarray*}
la cu\'al por lo anterior es la PGF del tiempo de ruina $T$.


\begin{Prop}
Sean $G_{n}\left(z\right)$ y $G\left(z,w\right)$ definidas como en
\ref{Eq.3.16.a} y \ref{Eq.3.16.b} respectivamente, entonces

\begin{equation}
G_{n}\left(z\right)=\frac{1}{z}\left[G_{n-1}\left(z\right)-G_{n-1}\left(0\right)\right]P\left(z\right).
\end{equation}

Adem\'as

\begin{equation}
G\left(z,w\right)=\frac{zF\left(z\right)-wP\left(z\right)G\left(0,w\right)}{z-wP\left(z\right)},
\end{equation}

con un \'unico polo en el c\'irculo unitario, adem\'as, el polo es
de la forma $z=\theta\left(w\right)$ y satisface que

\begin{enumerate}
\item[i)]$\theta\left(1\right)=1$,

\item[ii)] $\theta^{(1)}\left(1\right)=\frac{1}{1-\mu}$,

%\item[iii)] $\theta^{(2)}\left(1\right)=\frac{1}{1-\mu}$
\end{enumerate}

Finalmente, adem\'as se cumple que

\begin{equation}
\esp\left[w^{T}\right]=G\left(0,w\right)=F\left[\theta\left(w\right)\right].
\end{equation}

\end{Prop}


\begin{Coro}
El tiempo de ruina del jugador tiene primer y segundo momento
dados por
\begin{eqnarray}
\esp\left[T\right]&=&\frac{\esp\left[L_{0}\right]}{1-\mu}\\
Var\left[T\right]&=&\frac{Var\left[L_{0}\right]}{\left(1-\mu\right)^{2}}+\frac{\sigma^{2}\esp\left[L_{0}\right]}{\left(1-\mu\right)^{3}}.
\end{eqnarray}
\end{Coro}

%\end{Def}

%________________________________________________________
\section{Funciones Generadoras de Probabilidad Conjunta}
%________________________________________________________


De lo desarrollado hasta ahora se tiene lo siguiente

\begin{eqnarray*}
&&\esp\left[z_{1}^{L_{1}\left(\overline{\tau}_{1}\right)}z_{2}^{L_{2}\left(\overline{\tau}_{1}\right)}\right]=\esp\left[z_{2}^{L_{2}\left(\overline{\tau}_{1}\right)}\right]=\esp\left[z_{2}^{L_{2}\left(\tau_{1}\right)+X_{2}\left(\overline{\tau}_{1}-\tau_{1}\right)}\right]\\
&=&\esp\left[\left\{z_{2}^{L_{2}\left(\tau_{1}\right)}\right\}\left\{z_{2}^{X_{2}\left(\overline{\tau}_{1}-\tau_{1}\right)}\right\}\right]=\esp\left[\left\{z_{2}^{L_{2}\left(\tau_{1}\right)}\right\}\left\{P_{2}\left(z_{2}\right)\right\}^{\overline{\tau}_{1}-\tau_{1}}\right]\\
&=&\esp\left[\left\{z_{2}^{L_{2}\left(\tau_{1}\right)}\right\}\left\{\theta_{1}\left(P_{2}\left(z_{2}\right)\right)\right\}^{L_{1}\left(\tau_{1}\right)}\right]=F_{1}\left(\theta_{1}\left(P_{2}\left(z_{2}\right)\right),z_{2}\right)
\end{eqnarray*}

es decir %{{\tiny
\begin{equation}\label{Eq.base.F1}
\esp\left[z_{1}^{L_{1}\left(\overline{\tau}_{1}\right)}z_{2}^{L_{2}\left(\overline{\tau}_{1}\right)}\right]=F_{1}\left(\theta_{1}\left(P_{2}\left(z_{2}\right)\right),z_{2}\right).
\end{equation}

Procediendo de manera an\'aloga para $\overline{\tau}_{2}$:

\begin{eqnarray*}
\esp\left[z_{1}^{L_{1}\left(\overline{\tau}_{2}\right)}z_{2}^{L_{2}\left(\overline{\tau}_{2}\right)}\right]&=&\esp\left[z_{1}^{L_{1}\left(\overline{\tau}_{2}\right)}\right]=\esp\left[z_{1}^{L_{1}\left(\tau_{2}\right)+X_{1}\left(\overline{\tau}_{2}-\tau_{2}\right)}\right]=\esp\left[\left\{z_{1}^{L_{1}\left(\tau_{2}\right)}\right\}\left\{z_{1}^{X_{1}\left(\overline{\tau}_{2}-\tau_{2}\right)}\right\}\right]\\
&=&\esp\left[\left\{z_{1}^{L_{1}\left(\tau_{2}\right)}\right\}\left\{P_{1}\left(z_{1}\right)\right\}^{\overline{\tau}_{2}-\tau_{2}}\right]=\esp\left[\left\{z_{1}^{L_{1}\left(\tau_{2}\right)}\right\}\left\{\theta_{2}\left(P_{1}\left(z_{1}\right)\right)\right\}^{L_{2}\left(\tau_{2}\right)}\right]\\
&=&F_{2}\left(z_{1},\theta_{2}\left(P_{1}\left(z_{1}\right)\right)\right)
\end{eqnarray*}%}}


\begin{equation}\label{Eq.PGF.Conjunta.Tau2}
\esp\left[z_{1}^{L_{1}\left(\overline{\tau}_{2}\right)}z_{2}^{L_{2}\left(\overline{\tau}_{2}\right)}\right]=F_{2}\left(z_{1},\theta_{2}\left(P_{1}\left(z_{1}\right)\right)\right)
\end{equation}%}

Ahora, para el intervalo de tiempo
$\left[\overline{\tau}_{1},\tau_{2}\right]$ y
$\left[\overline{\tau}_{2},\tau_{1}\right]$, los arribos de los
usuarios modifican el n\'umero de usuarios que llegan a las colas,
es decir, los procesos
$L_{1}\left(t\right)$
y $L_{2}\left(t\right)$. La PGF para el n\'umero de arribos
a todas las estaciones durante el intervalo
$\left[\overline{\tau}_{1},\tau_{2}\right]$  cuya distribuci\'on
est\'a especificada por la distribuci\'on compuesta
$R_{1}\left(\mathbf{z}\right),R_{2}\left(\mathbf{z}\right)$:

\begin{eqnarray*}
R_{1}\left(\mathbf{z}\right)=R_{1}\left(\prod_{i=1}^{2}P\left(z_{i}\right)\right)=\esp\left[\left\{\prod_{i=1}^{2}P\left(z_{i}\right)\right\}^{\tau_{2}-\overline{\tau}_{1}}\right]\\
R_{2}\left(\mathbf{z}\right)=R_{2}\left(\prod_{i=1}^{2}P\left(z_{i}\right)\right)=\esp\left[\left\{\prod_{i=1}^{2}P\left(z_{i}\right)\right\}^{\tau_{1}-\overline{\tau}_{2}}\right]\\
\end{eqnarray*}


Dado que los eventos en
$\left[\tau_{1},\overline{\tau}_{1}\right]$ y
$\left[\overline{\tau}_{1},\tau_{2}\right]$ son independientes, la
PGF conjunta para el n\'umero de usuarios en el sistema al tiempo
$t=\tau_{2}$ la PGF conjunta para el n\'umero de usuarios en el sistema est\'an dadas por

{\footnotesize{
\begin{eqnarray*}
F_{1}\left(\mathbf{z}\right)&=&R_{2}\left(\prod_{i=1}^{2}P\left(z_{i}\right)\right)F_{2}\left(z_{1},\theta_{2}\left(P_{1}\left(z_{1}\right)\right)\right)\\
F_{2}\left(\mathbf{z}\right)&=&R_{1}\left(\prod_{i=1}^{2}P\left(z_{i}\right)\right)F_{1}\left(\theta_{1}\left(P_{2}\left(z_{2}\right)\right),z_{2}\right)\\
\end{eqnarray*}}}


Entonces debemos de determinar las siguientes expresiones:


\begin{eqnarray*}
\begin{array}{cc}
f_{1}\left(1\right)=\frac{\partial F_{1}\left(\mathbf{z}\right)}{\partial z_{1}}|_{\mathbf{z}=1}, & f_{1}\left(2\right)=\frac{\partial F_{1}\left(\mathbf{z}\right)}{\partial z_{2}}|_{\mathbf{z}=1},\\
f_{2}\left(1\right)=\frac{\partial F_{2}\left(\mathbf{z}\right)}{\partial z_{1}}|_{\mathbf{z}=1}, & f_{2}\left(2\right)=\frac{\partial F_{2}\left(\mathbf{z}\right)}{\partial z_{2}}|_{\mathbf{z}=1},\\
\end{array}
\end{eqnarray*}


\begin{eqnarray*}
\frac{\partial R_{1}\left(\mathbf{z}\right)}{\partial
z_{1}}|_{\mathbf{z}=1}&=&R_{1}^{(1)}\left(1\right)P_{1}^{(1)}\left(1\right)\\
\frac{\partial R_{1}\left(\mathbf{z}\right)}{\partial
z_{2}}|_{\mathbf{z}=1}&=&R_{1}^{(1)}\left(1\right)P_{2}^{(1)}\left(1\right)\\
\frac{\partial R_{2}\left(\mathbf{z}\right)}{\partial
z_{1}}|_{\mathbf{z}=1}&=&R_{2}^{(1)}\left(1\right)P_{1}^{(1)}\left(1\right)\\
\frac{\partial R_{2}\left(\mathbf{z}\right)}{\partial
z_{2}}|_{\mathbf{z}=1}&=&R_{2}^{(1)}\left(1\right)P_{2}^{(1)}\left(1\right)\\
\end{eqnarray*}



\begin{eqnarray*}
\frac{\partial}{\partial
z_{1}}F_{1}\left(\theta_{1}\left(P_{2}\left(z_{2}\right)\right),z_{2}\right)&=&0\\
\frac{\partial}{\partial
z_{2}}F_{1}\left(\theta_{1}\left(P_{2}\left(z_{2}\right)\right),z_{2}\right)&=&\frac{\partial
F_{1}}{\partial z_{2}}+\frac{\partial F_{1}}{\partial
z_{1}}\theta_{1}^{(1)}P_{2}^{(1)}\left(1\right)\\
\frac{\partial}{\partial
z_{1}}F_{2}\left(z_{1},\theta_{2}\left(P_{1}\left(z_{1}\right)\right)\right)&=&\frac{\partial
F_{2}}{\partial z_{1}}+\frac{\partial F_{2}}{\partial
z_{2}}\theta_{2}^{(1)}P_{1}^{(1)}\left(1\right)\\
\frac{\partial}{\partial
z_{2}}F_{2}\left(z_{1},\theta_{2}\left(P_{1}\left(z_{1}\right)\right)\right)&=&0\\
\end{eqnarray*}


Por lo tanto de las dos secciones anteriores se tiene que:


\begin{eqnarray*}
\frac{\partial F_{1}}{\partial z_{1}}&=&\frac{\partial
R_{2}}{\partial z_{1}}|_{\mathbf{z}=1}+\frac{\partial F_{2}}{\partial z_{1}}|_{\mathbf{z}=1}=R_{2}^{(1)}\left(1\right)P_{1}^{(1)}\left(1\right)+f_{2}\left(1\right)+f_{2}\left(2\right)\theta_{2}^{(1)}\left(1\right)P_{1}^{(1)}\left(1\right)\\
\frac{\partial F_{1}}{\partial z_{2}}&=&\frac{\partial
R_{2}}{\partial z_{2}}|_{\mathbf{z}=1}+\frac{\partial F_{2}}{\partial z_{2}}|_{\mathbf{z}=1}=R_{2}^{(1)}\left(1\right)P_{2}^{(1)}\left(1\right)\\
\frac{\partial F_{2}}{\partial z_{1}}&=&\frac{\partial
R_{1}}{\partial z_{1}}|_{\mathbf{z}=1}+\frac{\partial F_{1}}{\partial z_{1}}|_{\mathbf{z}=1}=R_{1}^{(1)}\left(1\right)P_{1}^{(1)}\left(1\right)\\
\frac{\partial F_{2}}{\partial z_{2}}&=&\frac{\partial
R_{1}}{\partial z_{2}}|_{\mathbf{z}=1}+\frac{\partial F_{1}}{\partial z_{2}}|_{\mathbf{z}=1}
=R_{1}^{(1)}\left(1\right)P_{2}^{(1)}\left(1\right)+f_{1}\left(1\right)\theta_{1}^{(1)}\left(1\right)P_{2}^{(1)}\left(1\right)\\
\end{eqnarray*}


El cual se puede escribir en forma equivalente:
\begin{eqnarray*}
f_{1}\left(1\right)&=&r_{2}\mu_{1}+f_{2}\left(1\right)+f_{2}\left(2\right)\frac{\mu_{1}}{1-\mu_{2}}\\
f_{1}\left(2\right)&=&r_{2}\mu_{2}\\
f_{2}\left(1\right)&=&r_{1}\mu_{1}\\
f_{2}\left(2\right)&=&r_{1}\mu_{2}+f_{1}\left(2\right)+f_{1}\left(1\right)\frac{\mu_{2}}{1-\mu_{1}}\\
\end{eqnarray*}

De donde:
\begin{eqnarray*}
f_{1}\left(1\right)&=&\mu_{1}\left[r_{2}+\frac{f_{2}\left(2\right)}{1-\mu_{2}}\right]+f_{2}\left(1\right)\\
f_{2}\left(2\right)&=&\mu_{2}\left[r_{1}+\frac{f_{1}\left(1\right)}{1-\mu_{1}}\right]+f_{1}\left(2\right)\\
\end{eqnarray*}

Resolviendo para $f_{1}\left(1\right)$:
\begin{eqnarray*}
f_{1}\left(1\right)&=&r_{2}\mu_{1}+f_{2}\left(1\right)+f_{2}\left(2\right)\frac{\mu_{1}}{1-\mu_{2}}=r_{2}\mu_{1}+r_{1}\mu_{1}+f_{2}\left(2\right)\frac{\mu_{1}}{1-\mu_{2}}\\
&=&\mu_{1}\left(r_{2}+r_{1}\right)+f_{2}\left(2\right)\frac{\mu_{1}}{1-\mu_{2}}=\mu_{1}\left(r+\frac{f_{2}\left(2\right)}{1-\mu_{2}}\right),\\
\end{eqnarray*}

entonces

\begin{eqnarray*}
f_{2}\left(2\right)&=&\mu_{2}\left(r_{1}+\frac{f_{1}\left(1\right)}{1-\mu_{1}}\right)+f_{1}\left(2\right)=\mu_{2}\left(r_{1}+\frac{f_{1}\left(1\right)}{1-\mu_{1}}\right)+r_{2}\mu_{2}\\
&=&\mu_{2}\left[r_{1}+r_{2}+\frac{f_{1}\left(1\right)}{1-\mu_{1}}\right]=\mu_{2}\left[r+\frac{f_{1}\left(1\right)}{1-\mu_{1}}\right]\\
&=&\mu_{2}r+\mu_{1}\left(r+\frac{f_{2}\left(2\right)}{1-\mu_{2}}\right)\frac{\mu_{2}}{1-\mu_{1}}\\
&=&\mu_{2}r+\mu_{2}\frac{r\mu_{1}}{1-\mu_{1}}+f_{2}\left(2\right)\frac{\mu_{1}\mu_{2}}{\left(1-\mu_{1}\right)\left(1-\mu_{2}\right)}\\
&=&\mu_{2}\left(r+\frac{r\mu_{1}}{1-\mu_{1}}\right)+f_{2}\left(2\right)\frac{\mu_{1}\mu_{2}}{\left(1-\mu_{1}\right)\left(1-\mu_{2}\right)}\\
&=&\mu_{2}\left(\frac{r}{1-\mu_{1}}\right)+f_{2}\left(2\right)\frac{\mu_{1}\mu_{2}}{\left(1-\mu_{1}\right)\left(1-\mu_{2}\right)}\\
\end{eqnarray*}
entonces
\begin{eqnarray*}
f_{2}\left(2\right)-f_{2}\left(2\right)\frac{\mu_{1}\mu_{2}}{\left(1-\mu_{1}\right)\left(1-\mu_{2}\right)}&=&\mu_{2}\left(\frac{r}{1-\mu_{1}}\right)\\
f_{2}\left(2\right)\left(1-\frac{\mu_{1}\mu_{2}}{\left(1-\mu_{1}\right)\left(1-\mu_{2}\right)}\right)&=&\mu_{2}\left(\frac{r}{1-\mu_{1}}\right)\\
f_{2}\left(2\right)\left(\frac{1-\mu_{1}-\mu_{2}+\mu_{1}\mu_{2}-\mu_{1}\mu_{2}}{\left(1-\mu_{1}\right)\left(1-\mu_{2}\right)}\right)&=&\mu_{2}\left(\frac{r}{1-\mu_{1}}\right)\\
f_{2}\left(2\right)\left(\frac{1-\mu}{\left(1-\mu_{1}\right)\left(1-\mu_{2}\right)}\right)&=&\mu_{2}\left(\frac{r}{1-\mu_{1}}\right)\\
\end{eqnarray*}
por tanto
\begin{eqnarray*}
f_{2}\left(2\right)&=&\frac{r\frac{\mu_{2}}{1-\mu_{1}}}{\frac{1-\mu}{\left(1-\mu_{1}\right)\left(1-\mu_{2}\right)}}=\frac{r\mu_{2}\left(1-\mu_{1}\right)\left(1-\mu_{2}\right)}{\left(1-\mu_{1}\right)\left(1-\mu\right)}\\
&=&\frac{\mu_{2}\left(1-\mu_{2}\right)}{1-\mu}r=r\mu_{2}\frac{1-\mu_{2}}{1-\mu}.
\end{eqnarray*}
es decir

\begin{eqnarray}
f_{2}\left(2\right)&=&r\mu_{2}\frac{1-\mu_{2}}{1-\mu}.
\end{eqnarray}

Entonces

\begin{eqnarray*}
f_{1}\left(1\right)&=&\mu_{1}r+f_{2}\left(2\right)\frac{\mu_{1}}{1-\mu_{2}}=\mu_{1}r+\left(\frac{\mu_{2}\left(1-\mu_{2}\right)}{1-\mu}r\right)\frac{\mu_{1}}{1-\mu_{2}}\\
&=&\mu_{1}r+\mu_{1}r\left(\frac{\mu_{2}}{1-\mu}\right)=\mu_{1}r\left[1+\frac{\mu_{2}}{1-\mu}\right]\\
&=&r\mu_{1}\frac{1-\mu_{1}}{1-\mu}\\
\end{eqnarray*}
%__________________________________________________________________________
\section{Definiciones}
%__________________________________________________________________________

\begin{itemize}
\item Consideremos una red de sistema de visitas c\'iclicas conformada
por dos sistemas de visitas c\'iclicas, cada una con dos colas
independientes, donde adem\'as se permite el intercambio de
usuarios entre los dos sistemas en la segunda cola de cada uno de
ellos.\smallskip

\item Sup\'ongase adem\'as que los arribos de los usuarios ocurren
conforme a un proceso Poisson con tasa de llegada $\mu_{1}$ y
$\mu_{2}$ para el sistema 1, mientras que para el sistema 2,
lo hacen conforme a un proceso Poisson con tasa
$\hat{\mu}_{1},\hat{\mu}_{2}$ respectivamente.\smallskip
\end{itemize}


El traslado de un sistema a otro ocurre de manera que los tiempos
entre llegadas de los usuarios a la cola dos del sistema 1
provenientes del sistema 2, se distribuye de manera exponencial
con par\'ametro $\check{\mu}_{2}$.\smallskip

Se considerar\'an intervalos de tiempo de la forma
$\left[t,t+1\right]$. Los usuarios arriban por paquetes de manera
independiente del resto de las colas. Se define el grupo de
usuarios que llegan a cada una de las colas del sistema 1,
caracterizadas por $Q_{1}$ y $Q_{2}$ respectivamente, en el
intervalo de tiempo $\left[t,t+1\right]$ por
$X_{1}\left(t\right),X_{2}\left(t\right)$. De igual manera se
definen los procesos
$\hat{X}_{1}\left(t\right),\hat{X}_{2}\left(t\right)$ para las
colas del sistema 2, denotadas por $\hat{Q}_{1}$ y $\hat{Q}_{2}$
respectivamente.\smallskip

Para el n\'umero de usuarios que se trasladan del sistema 2 al
sistema 1, de la cola $\hat{Q}_{2}$ a la cola
$Q_{2}$, en el intervalo de tiempo
$\left[t,t+1\right]$, se define el proceso
$Y_{2}\left(t\right)$.


%_________________________________________________________________________
\section{La ruina del jugador}
%_________________________________________________________________________

Supongamos que se tiene un jugador que cuenta con un capital
inicial de $\tilde{L}_{0}\geq0$ unidades, esta persona realiza una
serie de dos juegos simult\'aneos e independientes de manera
sucesiva, dichos eventos son independientes e id\'enticos entre
s\'i para cada realizaci\'on.\smallskip

La ganancia en el $n$-\'esimo juego es
\begin{equation}\label{Eq.Cero}
\tilde{X}_{n}=X_{n}+Y_{n}
\end{equation}

unidades de las cuales se resta una cuota de 1 unidad por cada
juego simult\'aneo, es decir, se restan dos unidades por cada
juego realizado.\smallskip

En t\'erminos de la teor\'ia de colas puede pensarse como el n\'umero de usuarios que llegan a una cola v\'ia dos procesos de arribo distintos e independientes entre s\'i.

Su Funci\'on Generadora de Probabilidades (FGP) est\'a dada por

\begin{eqnarray*}
F\left(z\right)&=&\esp\left[z^{\tilde{L}_{0}}\right]\\
\tilde{P}\left(z\right)&=&\esp\left[z^{\tilde{X}_{n}}\right]=\esp\left[z^{X_{n}+Y_{n}}\right]\\
&=&\esp\left[z^{X_{n}}z^{Y_{n}}\right]=\esp\left[z^{X_{n}}\right]\esp\left[z^{Y_{n}}\right]\\
&=&P\left(z\right)\check{P}\left(z\right)
\end{eqnarray*}
entonces
\begin{eqnarray*}
\tilde{\mu}&=&\esp\left[\tilde{X}_{n}\right]=\tilde{P}\left[z\right]<1.\\
\end{eqnarray*}

Sea $\tilde{L}_{n}$ el capital remanente despu\'es del $n$-\'esimo
juego. Entonces

\begin{eqnarray*}
\tilde{L}_{n}&=&\tilde{L}_{0}+\tilde{X}_{1}+\tilde{X}_{2}+\cdots+\tilde{X}_{n}-2n.
\end{eqnarray*}

La ruina del jugador ocurre despu\'es del $n$-\'esimo juego, es decir, la cola se vac\'ia despu\'es del $n$-\'esimo juego,
entonces sea $T$ definida como

\begin{eqnarray*}
T&=&min\left\{\tilde{L}_{n}=0\right\}
\end{eqnarray*}

Si $\tilde{L}_{0}=0$, entonces claramente $T=0$. En este sentido $T$
puede interpretarse como la longitud del periodo de tiempo que el servidor ocupa para dar servicio en la cola, comenzando con $\tilde{L}_{0}$ grupos de usuarios
presentes en la cola, quienes arribaron conforme a un proceso dado
por $\tilde{P}\left(z\right)$.\smallskip


Sea $g_{n,k}$ la probabilidad del evento de que el jugador no
caiga en ruina antes del $n$-\'esimo juego, y que adem\'as tenga
un capital de $k$ unidades antes del $n$-\'esimo juego, es decir,

Dada $n\in\left\{1,2,\ldots,\right\}$ y
$k\in\left\{0,1,2,\ldots,\right\}$
\begin{eqnarray*}
g_{n,k}:=P\left\{\tilde{L}_{j}>0, j=1,\ldots,n,
\tilde{L}_{n}=k\right\}
\end{eqnarray*}

la cual adem\'as se puede escribir como:

{\scriptsize{
\begin{eqnarray*}
g_{n,k}&=&P\left\{\tilde{L}_{j}>0, j=1,\ldots,n,
\tilde{L}_{n}=k\right\}=\sum_{j=1}^{k+1}g_{n-1,j}P\left\{\tilde{X}_{n}=k-j+1\right\}\\
&=&\sum_{j=1}^{k+1}g_{n-1,j}P\left\{X_{n}+Y_{n}=k-j+1\right\}\\
&=&\sum_{j=1}^{k+1}\sum_{l=1}^{j}g_{n-1,j}P\left\{X_{n}+Y_{n}=k-j+1,Y_{n}=l\right\}\\
\end{eqnarray*}}}

{\scriptsize{
\begin{eqnarray*}
&=&\sum_{j=1}^{k+1}\sum_{l=1}^{j}g_{n-1,j}P\left\{X_{n}+Y_{n}=k-j+1|Y_{n}=l\right\}P\left\{Y_{n}=l\right\}\\
&=&\sum_{j=1}^{k+1}\sum_{l=1}^{j}g_{n-1,j}P\left\{X_{n}=k-j-l+1\right\}P\left\{Y_{n}=l\right\}\\
\end{eqnarray*}}}

es decir {\scriptsize{
\begin{eqnarray}\label{Eq.Gnk.2S}
g_{n,k}=\sum_{j=1}^{k+1}\sum_{l=1}^{j}g_{n-1,j}P\left\{X_{n}=k-j-l+1\right\}P\left\{Y_{n}=l\right\}
\end{eqnarray}}}
adem\'as
{\scriptsize{
\begin{equation}\label{Eq.L02S}
g_{0,k}=P\left\{\tilde{L}_{0}=k\right\}.
\end{equation}}}



Se definen las siguientes FGP: {\scriptsize{
\begin{equation}\label{Eq.3.16.a.2S}
G_{n}\left(z\right)=\sum_{k=0}^{\infty}g_{n,k}z^{k},\textrm{ para
}n=0,1,\ldots,
\end{equation}}}
{\scriptsize{
\begin{equation}\label{Eq.3.16.b.2S}
G\left(z,w\right)=\sum_{n=0}^{\infty}G_{n}\left(z\right)w^{n}.
\end{equation}}}


En particular para $k=0$, {\scriptsize{
\begin{eqnarray*}
g_{n,0}=G_{n}\left(0\right)=P\left\{\tilde{L}_{j}>0,\textrm{ para
}j<n,\textrm{ y }\tilde{L}_{n}=0\right\}=P\left\{T=n\right\},
\end{eqnarray*}}}

adem\'as
{\scriptsize{
\begin{eqnarray*}%\label{Eq.G0w.2S}
G\left(0,w\right)=\sum_{n=0}^{\infty}G_{n}\left(0\right)w^{n}=\sum_{n=0}^{\infty}P\left\{T=n\right\}w^{n}
=\esp\left[w^{T}\right]
\end{eqnarray*}}}
la cu\'al resulta ser la FGP del tiempo de ruina $T$.

\begin{Prop}\label{Prop.1.1.2S}
Sean $G_{n}\left(z\right)$ y $G\left(z,w\right)$ definidas como en
(\ref{Eq.3.16.a.2S}) y (\ref{Eq.3.16.b.2S}) respectivamente,
entonces {\footnotesize{
\begin{equation}\label{Eq.Pag.45}
G_{n}\left(z\right)=\frac{1}{z}\left[G_{n-1}\left(z\right)-G_{n-1}\left(0\right)\right]\tilde{P}\left(z\right).
\end{equation}}}

Adem\'as

{\footnotesize{
\begin{equation}\label{Eq.Pag.46}
G\left(z,w\right)=\frac{zF\left(z\right)-wP\left(z\right)G\left(0,w\right)}{z-wR\left(z\right)},
\end{equation}}}

con un \'unico polo en el c\'irculo unitario, adem\'as, el polo es
de la forma $z=\theta\left(w\right)$ y satisface que
{\footnotesize{
\begin{enumerate}
\item[i)]$\tilde{\theta}\left(1\right)=1$,

\item[ii)] $\tilde{\theta}^{(1)}\left(1\right)=\frac{1}{1-\tilde{\mu}}$,

\item[iii)]
$\tilde{\theta}^{(2)}\left(1\right)=\frac{\tilde{\mu}}{\left(1-\tilde{\mu}\right)^{2}}+\frac{\tilde{\sigma}}{\left(1-\tilde{\mu}\right)^{3}}$.
\end{enumerate}}}

Finalmente, adem\'as se cumple que {\footnotesize{
\begin{equation}
\esp\left[w^{T}\right]=G\left(0,w\right)=F\left[\tilde{\theta}\left(w\right)\right].
\end{equation}}}
\end{Prop}

Multiplicando las ecuaciones (\ref{Eq.Gnk.2S}) y (\ref{Eq.L02S})
por el t\'ermino $z^{k}$:

\begin{eqnarray*}
g_{n,k}z^{k}&=&\sum_{j=1}^{k+1}\sum_{l=1}^{j}g_{n-1,j}P\left\{X_{n}=k-j-l+1\right\}P\left\{Y_{n}=l\right\}z^{k},\\
g_{0,k}z^{k}&=&P\left\{\tilde{L}_{0}=k\right\}z^{k},
\end{eqnarray*}

ahora sumamos sobre $k$
\begin{eqnarray*}
\sum_{k=0}^{\infty}g_{n,k}z^{k}&=&\sum_{k=0}^{\infty}\sum_{j=1}^{k+1}\sum_{l=1}^{j}g_{n-1,j}P\left\{X_{n}=k-j-l+1\right\}P\left\{Y_{n}=l\right\}z^{k}\\
&=&\sum_{k=0}^{\infty}z^{k}\sum_{j=1}^{k+1}\sum_{l=1}^{j}g_{n-1,j}P\left\{X_{n}=k-\left(j+l
-1\right)\right\}P\left\{Y_{n}=l\right\}\\
&=&\sum_{k=0}^{\infty}z^{k+\left(j+l-1\right)-\left(j+l-1\right)}\sum_{j=1}^{k+1}\sum_{l=1}^{j}g_{n-1,j}P\left\{X_{n}=k-
\left(j+l-1\right)\right\}P\left\{Y_{n}=l\right\}\\
&=&\sum_{k=0}^{\infty}\sum_{j=1}^{k+1}\sum_{l=1}^{j}g_{n-1,j}z^{j-1}P\left\{X_{n}=k-
\left(j+l-1\right)\right\}z^{k-\left(j+l-1\right)}P\left\{Y_{n}=l\right\}z^{l}\\
\end{eqnarray*}


luego
\begin{eqnarray*}
&=&\sum_{j=1}^{\infty}\sum_{l=1}^{j}g_{n-1,j}z^{j-1}\sum_{k=j+l-1}^{\infty}P\left\{X_{n}=k-
\left(j+l-1\right)\right\}z^{k-\left(j+l-1\right)}P\left\{Y_{n}=l\right\}z^{l}\\
&=&\sum_{j=1}^{\infty}g_{n-1,j}z^{j-1}\sum_{l=1}^{j}\sum_{k=j+l-1}^{\infty}P\left\{X_{n}=k-
\left(j+l-1\right)\right\}z^{k-\left(j+l-1\right)}P\left\{Y_{n}=l\right\}z^{l}\\
&=&\sum_{j=1}^{\infty}g_{n-1,j}z^{j-1}\sum_{k=j+l-1}^{\infty}\sum_{l=1}^{j}P\left\{X_{n}=k-
\left(j+l-1\right)\right\}z^{k-\left(j+l-1\right)}P\left\{Y_{n}=l\right\}z^{l}\\
&=&\sum_{j=1}^{\infty}g_{n-1,j}z^{j-1}\sum_{k=j+l-1}^{\infty}\sum_{l=1}^{j}P\left\{X_{n}=k-
\left(j+l-1\right)\right\}z^{k-\left(j+l-1\right)}\sum_{l=1}^{j}P
\left\{Y_{n}=l\right\}z^{l}\\
&=&\sum_{j=1}^{\infty}g_{n-1,j}z^{j-1}\sum_{l=1}^{\infty}P\left\{Y_{n}=l\right\}z^{l}
\sum_{k=j+l-1}^{\infty}\sum_{l=1}^{j}
P\left\{X_{n}=k-\left(j+l-1\right)\right\}z^{k-\left(j+l-1\right)}\\
&=&\frac{1}{z}\left[G_{n-1}\left(z\right)-G_{n-1}\left(0\right)\right]\tilde{P}\left(z\right)
\sum_{k=j+l-1}^{\infty}\sum_{l=1}^{j}
P\left\{X_{n}=k-\left(j+l-1\right)\right\}z^{k-\left(j+l-1\right)}\\
&=&\frac{1}{z}\left[G_{n-1}\left(z\right)-G_{n-1}\left(0\right)\right]\tilde{P}\left(z\right)P\left(z\right)=\frac{1}{z}\left[G_{n-1}\left(z\right)-G_{n-1}\left(0\right)\right]\tilde{P}\left(z\right),\\
\end{eqnarray*}

es decir la ecuaci\'on (\ref{Eq.3.16.a.2S}) se puede reescribir
como
\begin{equation}\label{Eq.3.16.a.2Sbis}
G_{n}\left(z\right)=\frac{1}{z}\left[G_{n-1}\left(z\right)-G_{n-1}\left(0\right)\right]\tilde{P}\left(z\right).
\end{equation}

Por otra parte recordemos la ecuaci\'on (\ref{Eq.3.16.a.2S})

\begin{eqnarray*}
G_{n}\left(z\right)&=&\sum_{k=0}^{\infty}g_{n,k}z^{k},\textrm{ entonces }\frac{G_{n}\left(z\right)}{z}=\sum_{k=1}^{\infty}g_{n,k}z^{k-1},\\
\end{eqnarray*}

Por lo tanto utilizando la ecuaci\'on (\ref{Eq.3.16.a.2Sbis}):

\begin{eqnarray*}
G\left(z,w\right)&=&\sum_{n=0}^{\infty}G_{n}\left(z\right)w^{n}=G_{0}\left(z\right)+
\sum_{n=1}^{\infty}G_{n}\left(z\right)w^{n}\\
&=&F\left(z\right)+\sum_{n=0}^{\infty}\left[G_{n}\left(z\right)-G_{n}\left(0\right)\right]w^{n}\frac{\tilde{P}\left(z\right)}{z}\\
&=&F\left(z\right)+\frac{w}{z}\sum_{n=0}^{\infty}\left[G_{n}\left(z\right)-G_{n}\left(0\right)\right]w^{n-1}\tilde{P}\left(z\right)\\
\end{eqnarray*}

es decir
\begin{eqnarray*}
G\left(z,w\right)&=&F\left(z\right)+\frac{w}{z}\left[G\left(z,w\right)-G\left(0,w\right)\right]\tilde{P}\left(z\right),
\end{eqnarray*}


entonces

\begin{eqnarray*}
G\left(z,w\right)&=&F\left(z\right)+\frac{w}{z}\left[G\left(z,w\right)-G\left(0,w\right)\right]\tilde{P}\left(z\right)\\
&=&F\left(z\right)+\frac{w}{z}\tilde{P}\left(z\right)G\left(z,w\right)-\frac{w}{z}\tilde{P}\left(z\right)G\left(0,w\right)\\
&\Leftrightarrow&\\
G\left(z,w\right)\left\{1-\frac{w}{z}\tilde{P}\left(z\right)\right\}&=&F\left(z\right)-\frac{w}{z}\tilde{P}\left(z\right)G\left(0,w\right),
\end{eqnarray*}
por lo tanto,
\begin{equation}
G\left(z,w\right)=\frac{zF\left(z\right)-w\tilde{P}\left(z\right)G\left(0,w\right)}{1-w\tilde{P}\left(z\right)}.
\end{equation}


Ahora $G\left(z,w\right)$ es anal\'itica en $|z|=1$.

Sean $z,w$ tales que $|z|=1$ y $|w|\leq1$, como $\tilde{P}\left(z\right)$
es FGP
\begin{eqnarray*}
|z-\left(z-w\tilde{P}\left(z\right)\right)|<|z|\Leftrightarrow|w\tilde{P}\left(z\right)|<|z|
\end{eqnarray*}
es decir, se cumplen las condiciones del Teorema de Rouch\'e y por
tanto, $z$ y $z-w\tilde{P}\left(z\right)$ tienen el mismo n\'umero de
ceros en $|z|=1$. Sea $z=\tilde{\theta}\left(w\right)$ la soluci\'on
\'unica de $z-w\tilde{P}\left(z\right)$, es decir

\begin{equation}\label{Eq.Theta.w}
\tilde{\theta}\left(w\right)-w\tilde{P}\left(\tilde{\theta}\left(w\right)\right)=0,
\end{equation}
 con $|\tilde{\theta}\left(w\right)|<1$. Cabe hacer menci\'on que $\tilde{\theta}\left(w\right)$ es la FGP para el tiempo de ruina cuando $\tilde{L}_{0}=1$.


Considerando la ecuaci\'on (\ref{Eq.Theta.w})
\begin{eqnarray*}
&&\frac{\partial}{\partial w}\tilde{\theta}\left(w\right)|_{w=1}-\frac{\partial}{\partial w}\left\{w\tilde{P}\left(\tilde{\theta}\left(w\right)\right)\right\}|_{w=1}=0\\
&&\tilde{\theta}^{(1)}\left(w\right)|_{w=1}-\frac{\partial}{\partial w}w\left\{\tilde{P}\left(\tilde{\theta}\left(w\right)\right)\right\}|_{w=1}-w\frac{\partial}{\partial w}\tilde{P}\left(\tilde{\theta}\left(w\right)\right)|_{w=1}=0\\
&&\tilde{\theta}^{(1)}\left(1\right)-\tilde{P}\left(\tilde{\theta}\left(1\right)\right)-w\left\{\frac{\partial \tilde{P}\left(\tilde{\theta}\left(w\right)\right)}{\partial \tilde{\theta}\left(w\right)}\cdot\frac{\partial\tilde{\theta}\left(w\right)}{\partial w}|_{w=1}\right\}=0\\
&&\tilde{\theta}^{(1)}\left(1\right)-\tilde{P}\left(\tilde{\theta}\left(1\right)
\right)-\tilde{P}^{(1)}\left(\tilde{\theta}\left(1\right)\right)\cdot\tilde{\theta}^{(1)}\left(1\right)=0
\end{eqnarray*}


luego
\begin{eqnarray*}
&&\tilde{\theta}^{(1)}\left(1\right)-\tilde{P}^{(1)}\left(\tilde{\theta}\left(1\right)\right)\cdot
\tilde{\theta}^{(1)}\left(1\right)=\tilde{P}\left(\tilde{\theta}\left(1\right)\right)\\
&&\tilde{\theta}^{(1)}\left(1\right)\left(1-\tilde{P}^{(1)}\left(\tilde{\theta}\left(1\right)\right)\right)
=\tilde{P}\left(\tilde{\theta}\left(1\right)\right)\\
&&\tilde{\theta}^{(1)}\left(1\right)=\frac{\tilde{P}\left(\tilde{\theta}\left(1\right)\right)}{\left(1-\tilde{P}^{(1)}\left(\tilde{\theta}\left(1\right)\right)\right)}=\frac{1}{1-\tilde{\mu}}.
\end{eqnarray*}

Ahora determinemos el segundo momento de $\tilde{\theta}\left(w\right)$,
nuevamente consideremos la ecuaci\'on (\ref{Eq.Theta.w}):


\begin{eqnarray*}
\tilde{\theta}\left(w\right)-w\tilde{P}\left(\tilde{\theta}\left(w\right)\right)&=&0\\
\frac{\partial}{\partial w}\left\{\tilde{\theta}\left(w\right)-w\tilde{P}\left(\tilde{\theta}\left(w\right)\right)\right\}&=&0\\
\frac{\partial}{\partial w}\left\{\frac{\partial}{\partial w}\left\{\tilde{\theta}\left(w\right)-w\tilde{P}\left(\tilde{\theta}\left(w\right)\right)\right\}\right\}&=&0\\
\end{eqnarray*}
\begin{eqnarray*}
&&\frac{\partial}{\partial w}\left\{\frac{\partial}{\partial w}\tilde{\theta}\left(w\right)-\frac{\partial}{\partial w}\left[w\tilde{P}\left(\tilde{\theta}\left(w\right)\right)\right]\right\}
=\frac{\partial}{\partial w}\left\{\frac{\partial}{\partial w}\tilde{\theta}\left(w\right)-\frac{\partial}{\partial w}\left[w\tilde{P}\left(\tilde{\theta}\left(w\right)\right)\right]\right\}\\
&=&\frac{\partial}{\partial w}\left\{\frac{\partial \tilde{\theta}\left(w\right)}{\partial w}-\left[\tilde{P}\left(\tilde{\theta}\left(w\right)\right)+w\frac{\partial}{\partial w}R\left(\tilde{\theta}\left(w\right)\right)\right]\right\}\\
&=&\frac{\partial}{\partial w}\left\{\frac{\partial \tilde{\theta}\left(w\right)}{\partial w}-\left[\tilde{P}\left(\tilde{\theta}\left(w\right)\right)+w\frac{\partial \tilde{P}\left(\tilde{\theta}\left(w\right)\right)}{\partial w}\frac{\partial \tilde{\theta}\left(w\right)}{\partial w}\right]\right\}\\
&=&\frac{\partial}{\partial w}\left\{\tilde{\theta}^{(1)}\left(w\right)-\tilde{P}\left(\tilde{\theta}\left(w\right)\right)-w\tilde{P}^{(1)}\left(\tilde{\theta}\left(w\right)\right)\tilde{\theta}^{(1)}\left(w\right)\right\}\\
&=&\frac{\partial}{\partial w}\tilde{\theta}^{(1)}\left(w\right)-\frac{\partial}{\partial w}\tilde{P}\left(\tilde{\theta}\left(w\right)\right)-\frac{\partial}{\partial w}\left[w\tilde{P}^{(1)}\left(\tilde{\theta}\left(w\right)\right)\tilde{\theta}^{(1)}\left(w\right)\right]\\
\end{eqnarray*}
\begin{eqnarray*}
&=&\frac{\partial}{\partial
w}\tilde{\theta}^{(1)}\left(w\right)-\frac{\partial
\tilde{P}\left(\tilde{\theta}\left(w\right)\right)}{\partial
\tilde{\theta}\left(w\right)}\frac{\partial \tilde{\theta}\left(w\right)}{\partial
w}-\tilde{P}^{(1)}\left(\tilde{\theta}\left(w\right)\right)\tilde{\theta}^{(1)}\left(w\right)\\
&-&w\frac{\partial
\tilde{P}^{(1)}\left(\tilde{\theta}\left(w\right)\right)}{\partial
w}\tilde{\theta}^{(1)}\left(w\right)-w\tilde{P}^{(1)}\left(\tilde{\theta}\left(w\right)\right)\frac{\partial
\tilde{\theta}^{(1)}\left(w\right)}{\partial w}\\
&=&\tilde{\theta}^{(2)}\left(w\right)-\tilde{P}^{(1)}\left(\tilde{\theta}\left(w\right)\right)\tilde{\theta}^{(1)}\left(w\right)
-\tilde{P}^{(1)}\left(\tilde{\theta}\left(w\right)\right)\tilde{\theta}^{(1)}\left(w\right)\\
&-&w\tilde{P}^{(2)}\left(\tilde{\theta}\left(w\right)\right)\left(\tilde{\theta}^{(1)}\left(w\right)\right)^{2}-w\tilde{P}^{(1)}\left(\tilde{\theta}\left(w\right)\right)\tilde{\theta}^{(2)}\left(w\right)\\
&=&\tilde{\theta}^{(2)}\left(w\right)-2\tilde{P}^{(1)}\left(\tilde{\theta}\left(w\right)\right)\tilde{\theta}^{(1)}\left(w\right)\\
&-&w\tilde{P}^{(2)}\left(\tilde{\theta}\left(w\right)\right)\left(\tilde{\theta}^{(1)}\left(w\right)\right)^{2}-w\tilde{P}^{(1)}\left(\tilde{\theta}\left(w\right)\right)\tilde{\theta}^{(2)}\left(w\right)\\
&=&\tilde{\theta}^{(2)}\left(w\right)\left[1-w\tilde{P}^{(1)}\left(\tilde{\theta}\left(w\right)\right)\right]-
\tilde{\theta}^{(1)}\left(w\right)\left[w\tilde{\theta}^{(1)}\left(w\right)\tilde{P}^{(2)}\left(\tilde{\theta}\left(w\right)\right)+2\tilde{P}^{(1)}\left(\tilde{\theta}\left(w\right)\right)\right]
\end{eqnarray*}
luego



\begin{eqnarray*}
\tilde{\theta}^{(2)}\left(w\right)\left[1-w\tilde{P}^{(1)}\left(\tilde{\theta}\left(w\right)\right)\right]&-&\tilde{\theta}^{(1)}\left(w\right)\left[w\tilde{\theta}^{(1)}\left(w\right)\tilde{P}^{(2)}\left(\tilde{\theta}\left(w\right)\right)
+2\tilde{P}^{(1)}\left(\tilde{\theta}\left(w\right)\right)\right]=0\\
\tilde{\theta}^{(2)}\left(w\right)&=&\frac{\tilde{\theta}^{(1)}\left(w\right)\left[w\tilde{\theta}^{(1)}\left(w\right)\tilde{P}^{(2)}\left(\tilde{\theta}\left(w\right)\right)+2R^{(1)}\left(\tilde{\theta}\left(w\right)\right)\right]}{1-w\tilde{P}^{(1)}\left(\tilde{\theta}\left(w\right)\right)}\\
\tilde{\theta}^{(2)}\left(w\right)&=&\frac{\tilde{\theta}^{(1)}\left(w\right)w\tilde{\theta}^{(1)}\left(w\right)\tilde{P}^{(2)}\left(\tilde{\theta}\left(w\right)\right)}{1-w\tilde{P}^{(1)}\left(\tilde{\theta}\left(w\right)\right)}+\frac{2\tilde{\theta}^{(1)}\left(w\right)\tilde{P}^{(1)}\left(\tilde{\theta}\left(w\right)\right)}{1-w\tilde{P}^{(1)}\left(\tilde{\theta}\left(w\right)\right)}
\end{eqnarray*}


si evaluamos la expresi\'on anterior en $w=1$:
\begin{eqnarray*}
\tilde{\theta}^{(2)}\left(1\right)&=&\frac{\left(\tilde{\theta}^{(1)}\left(1\right)\right)^{2}\tilde{P}^{(2)}\left(\tilde{\theta}\left(1\right)\right)}{1-\tilde{P}^{(1)}\left(\tilde{\theta}\left(1\right)\right)}+\frac{2\tilde{\theta}^{(1)}\left(1\right)\tilde{P}^{(1)}\left(\tilde{\theta}\left(1\right)\right)}{1-\tilde{P}^{(1)}\left(\tilde{\theta}\left(1\right)\right)}\\
&=&\frac{\left(\tilde{\theta}^{(1)}\left(1\right)\right)^{2}\tilde{P}^{(2)}\left(1\right)}{1-\tilde{P}^{(1)}\left(1\right)}+\frac{2\tilde{\theta}^{(1)}\left(1\right)\tilde{P}^{(1)}\left(1\right)}{1-\tilde{P}^{(1)}\left(1\right)}\\
&=&\frac{\left(\frac{1}{1-\tilde{\mu}}\right)^{2}\tilde{P}^{(2)}\left(1\right)}{1-\tilde{\mu}}+\frac{2\left(\frac{1}{1-\tilde{\mu}}\right)\tilde{\mu}}{1-\tilde{\mu}}=\frac{\tilde{P}^{(2)}\left(1\right)}{\left(1-\tilde{\mu}\right)^{3}}+\frac{2\tilde{\mu}}{\left(1-\tilde{\mu}\right)^{2}}\\
\end{eqnarray*}

luego

\begin{eqnarray*}
&=&\frac{\sigma^{2}-\tilde{\mu}+\tilde{\mu}^{2}}{\left(1-\tilde{\mu}\right)^{3}}+\frac{2\tilde{\mu}}{\left(1-\tilde{\mu}\right)^{2}}=\frac{\sigma^{2}-\tilde{\mu}+\tilde{\mu}^{2}+2\tilde{\mu}\left(1-\tilde{\mu}\right)}{\left(1-\tilde{\mu}\right)^{3}}\\
\end{eqnarray*}


es decir
\begin{eqnarray*}
\tilde{\theta}^{(2)}\left(1\right)&=&\frac{\sigma^{2}+\tilde{\mu}-\tilde{\mu}^{2}}{\left(1-\tilde{\mu}\right)^{3}}=\frac{\sigma^{2}}{\left(1-\tilde{\mu}\right)^{3}}+\frac{\tilde{\mu}\left(1-\tilde{\mu}\right)}{\left(1-\tilde{\mu}\right)^{3}}\\
&=&\frac{\sigma^{2}}{\left(1-\tilde{\mu}\right)^{3}}+\frac{\tilde{\mu}}{\left(1-\tilde{\mu}\right)^{2}}.
\end{eqnarray*}

\begin{Coro}
El tiempo de ruina del jugador tiene primer y segundo momento
dados por

\begin{eqnarray}
\esp\left[T\right]&=&\frac{\esp\left[\tilde{L}_{0}\right]}{1-\tilde{\mu}}\\
Var\left[T\right]&=&\frac{Var\left[\tilde{L}_{0}\right]}{\left(1-\tilde{\mu}\right)^{2}}+\frac{\sigma^{2}\esp\left[\tilde{L}_{0}\right]}{\left(1-\tilde{\mu}\right)^{3}}.
\end{eqnarray}
\end{Coro}



%__________________________________________________________________________
\section{Funciones Generadoras de Probabilidades}
%__________________________________________________________________________


Para cada uno de los procesos de llegada a las colas $X_{1},X_{2},\hat{X}_{1}.\hat{X}_{2}$ y $Y_{2}$ anteriores se define su Funci\'on
Generadora de Probabilidades (PGF):
\begin{eqnarray*}
P_{1}\left(z_{1}\right)&=&\esp\left[z_{1}^{X_{1}\left(t\right)}\right],\\
P_{2}\left(z_{2}\right)&=&\esp\left[z_{2}^{X_{2}\left(t\right)}\right],\\
\check{P}_{2}\left(z_{2}\right)&=&\esp\left[z_{2}^{Y_{2}\left(t\right)}\right],\\
\hat{P}_{1}\left(w_{1}\right)&=&\esp\left[w_{1}^{\hat{X}_{1}\left(t\right)}\right],\\
\hat{P}_{2}\left(w_{2}\right)&=&\esp\left[w_{2}^{\hat{X}_{2}\left(t\right)}\right].
\end{eqnarray*}
entonces

\begin{eqnarray*}
\tilde{P}_{2}\left(z_{2}\right)&=&\esp\left[z_{2}^{\tilde{X}_{2}\left(t\right)}\right]
\end{eqnarray*}

Con primer momento definidos por



\begin{eqnarray*}
\mu_{1}&=&\esp\left[X_{1}\left(t\right)\right]=P_{1}^{(1)}\left(1\right),\\
\mu_{2}&=&\esp\left[X_{2}\left(t\right)\right]=P_{2}^{(1)}\left(1\right),\\
\check{\mu}_{2}&=&\esp\left[Y_{2}\left(t\right)\right]=\check{P}_{2}^{(1)}\left(1\right),\\
\hat{\mu}_{1}&=&\esp\left[\hat{X}_{1}\left(t\right)\right]=\hat{P}_{1}^{(1)}\left(1\right),\\
\hat{\mu}_{2}&=&\esp\left[\hat{X}_{2}\left(t\right)\right]=\hat{P}_{2}^{(1)}\left(1\right),\\
\tilde{\mu}_{2}&=&\esp\left[\tilde{X}_{2}\left(t\right)\right]=\tilde{P}_{2}^{(1)}\left(1\right).
\end{eqnarray*}

En lo que respecta al servidor, en t\'erminos de los tiempos de
visita a cada una de las colas, se denotar\'an por
$B_{1}\left(t\right),B_{2}\left(t\right)$ los procesos
correspondientes a las variables aleatorias $\tau_{1},\tau_{2}$
para $Q_{1},Q_{2}$ respectivamente; y
$\hat{B}_{1}\left(t\right),\hat{B}_{2}\left(t\right)$ con
par\'ametros $\zeta_{1},\zeta_{2}$ para $\hat{Q}_{1},\hat{Q}_{2}$
del sistema 2. Y a los tiempos en que el servidor termina de
atender en las colas $Q_{1},Q_{2},\hat{Q}_{1},\hat{Q}_{2}$, se les
denotar\'a por
$\overline{\tau}_{1},\overline{\tau}_{2},\overline{\zeta}_{1},\overline{\zeta}_{2}$
respectivamente. Entonces, los tiempos de servicio est\'an dados
por las diferencias
$\overline{\tau}_{1}-\tau_{1},\overline{\tau}_{2}-\tau_{2}$ para
$Q_{1},Q_{2}$, y
$\overline{\zeta}_{1}-\zeta_{1},\overline{\zeta}_{2}-\zeta_{2}$
para $\hat{Q}_{1},\hat{Q}_{2}$ respectivamente.

Sus procesos se definen por:


\begin{eqnarray*}
S_{1}\left(z_{1}\right)&=&\esp\left[z_{1}^{\overline{\tau}_{1}-\tau_{1}}\right],\\
S_{2}\left(z_{2}\right)&=&\esp\left[z_{1}^{\overline{\tau}_{2}-\tau_{2}}\right],\\
\hat{S}_{1}\left(w_{1}\right)&=&\esp\left[w_{1}^{\overline{\zeta}_{1}-\zeta_{1}}\right],\\
\hat{S}_{2}\left(w_{2}\right)&=&\esp\left[w_{2}^{\overline{\zeta}_{2}-\zeta_{2}}\right],
\end{eqnarray*}

con primer momento dado por:


\begin{eqnarray*}
s_{1}&=&\esp\left[\overline{\tau}_{1}-\tau_{1}\right],\\
s_{2}&=&\esp\left[\overline{\tau}_{2}-\tau_{2}\right],\\
\hat{s}_{1}&=&\esp\left[\overline{\zeta}_{1}-\zeta_{1}\right],\\
\hat{s}_{2}&=&\esp\left[\overline{\zeta}_{2}-\zeta_{2}\right],
\end{eqnarray*}

An\'alogamente los tiempos de traslado del servidor desde el
momento en que termina de atender a una cola y llega a la
siguiente para comenzar a dar servicio est\'an dados por
$\tau_{2}-\overline{\tau}_{1},\tau_{1}-\overline{\tau}_{2}$ y
$\zeta_{2}-\overline{\zeta}_{1},\zeta_{1}-\overline{\zeta}_{2}$
para el sistema 1 y el sistema 2, respectivamente.

La FGP para estos tiempos de traslado est\'an dados por

\begin{eqnarray*}
%\begin{array}{cc}
R_{1}\left(z_{1}\right)&=&\esp\left[z_{1}^{\tau_{2}-\overline{\tau}_{1}}\right],\\
R_{2}\left(z_{2}\right)&=&\esp\left[z_{2}^{\tau_{1}-\overline{\tau}_{2}}\right],\\
\hat{R}_{1}\left(w_{1}\right)&=&\esp\left[w_{1}^{\zeta_{2}-\overline{\zeta}_{1}}\right],\\
\hat{R}_{2}\left(w_{2}\right)&=&\esp\left[w_{2}^{\zeta_{1}-\overline{\zeta}_{2}}\right],
%\end{array}
\end{eqnarray*}
y al igual que como se hizo con anterioridad

\begin{eqnarray*}
r_{1}&=&R_{1}^{(1)}\left(1\right)=\esp\left[\tau_{2}-\overline{\tau}_{1}\right],\\
r_{2}&=&R_{2}^{(1)}\left(1\right)=\esp\left[\tau_{1}-\overline{\tau}_{2}\right],\\
\hat{r}_{1}&=&\hat{R}_{1}^{(1)}\left(1\right)=\esp\left[\zeta_{2}-\overline{\zeta}_{1}\right],\\
\hat{r}_{2}&=&\hat{R}_{2}^{(1)}\left(1\right)=\esp\left[\zeta_{1}-\overline{\zeta}_{2}\right].
\end{eqnarray*}

Se definen los procesos de conteo para el n\'umero de usuarios en
cada una de las colas al tiempo $t$,
$L_{1}\left(t\right),L_{2}\left(t\right)$, para
$H_{1}\left(t\right),H_{2}\left(t\right)$ del sistema 1,
respectivamente. Y para el segundo sistema, se tienen los procesos
$\hat{L}_{1}\left(t\right),\hat{L}_{2}\left(t\right)$ para
$\hat{H}_{1}\left(t\right),\hat{H}_{2}\left(t\right)$,
respectivamente, es decir,


\begin{eqnarray*}
H_{1}\left(t\right)&=&\esp\left[z_{1}^{L_{1}\left(t\right)}\right],\\
H_{2}\left(t\right)&=&\esp\left[z_{2}^{L_{2}\left(t\right)}\right],\\
\hat{H}_{1}\left(t\right)&=&\esp\left[w_{1}^{\hat{L}_{1}\left(t\right)}\right],\\
\hat{H}_{2}\left(t\right)&=&\esp\left[w_{2}^{\hat{L}_{2}\left(t\right)}\right].
\end{eqnarray*}
Por lo dicho anteriormente se tiene que el n\'umero de usuarios
presentes en los tiempos $\overline{\tau}_{1},\overline{\tau}_{2},
\overline{\zeta}_{1},\overline{\zeta}_{2}$, es cero, es decir,
 $L_{i}\left(\overline{\tau_{i}}\right)=0,$ y
$\hat{L}_{i}\left(\overline{\zeta_{i}}\right)=0$ para i=1,2 para
cada uno de los dos sistemas.


Para cada una de las colas en cada sistema, el n\'umero de
usuarios al tiempo en que llega el servidor a dar servicio est\'a
dado por el n\'umero de usuarios presentes en la cola al tiempo
$t=\tau_{i},\zeta_{i}$, m\'as el n\'umero de usuarios que llegan a
la cola en el intervalo de tiempo
$\left[\tau_{i},\overline{\tau}_{i}\right],\left[\zeta_{i},\overline{\zeta}_{i}\right]$,
es decir

\begin{eqnarray}\label{Eq.TiemposLlegada}
L_{1}\left(\overline{\tau}_{1}\right)&=&L_{1}\left(\tau_{1}\right)+X_{1}\left(\overline{\tau}_{1}-\tau_{1}\right),\\
\hat{L}_{1}\left(\overline{\tau}_{1}\right)&=&\hat{L}_{1}\left(\tau_{1}\right)+\hat{X}_{1}\left(\overline{\tau}_{1}-\tau_{1}\right),\\
\hat{L}_{2}\left(\overline{\tau}_{1}\right)&=&\hat{L}_{2}\left(\tau_{1}\right)+\hat{X}_{2}\left(\overline{\tau}_{1}-\tau_{1}\right).
\end{eqnarray}

En el caso espec\'ifico de $Q_{2}$, adem\'as, hay que considerar
el n\'umero de usuarios que pasan del sistema 2 al sistema 1, a
traves de $\hat{Q}_{2}$ mientras el servidor en $Q_{2}$ est\'a
ausente, es decir:

\begin{equation}\label{Eq.UsuariosTotalesZ2}
L_{2}\left(\overline{\tau}_{1}\right)=L_{2}\left(\tau_{1}\right)+X_{2}\left(\overline{\tau}_{1}-\tau_{1}\right)+Y_{2}\left(\overline{\tau}_{1}-\tau_{1}\right).
\end{equation}


Ahora, determinemos la distribuci\'on del n\'umero de usuarios que
pasan de $\hat{Q}_{2}$ a $Q_{2}$ considerando dos pol\'iticas de
traslado en espec\'ifico:

\begin{enumerate}
\item Solamente pasa un usuario,

\item Se permite el paso de $k$ usuarios,
\end{enumerate}
una vez que son atendidos por el servidor en $\hat{Q}_{2}$.

\begin{description}


\item[Pol\'itica de un solo usuario:] Sea $R_{2}$ el n\'umero de
usuarios que llegan a $\hat{Q}_{2}$ al tiempo $t$, sea $R_{1}$ el
n\'umero de usuarios que pasan de $\hat{Q}_{2}$ a $Q_{2}$ al
tiempo $t$.
\end{description}


A saber:
\begin{eqnarray*}
\esp\left[R_{1}\right]&=&\sum_{y\geq0}\prob\left[R_{2}=y\right]\esp\left[R_{1}|R_{2}=y\right]\\
&=&\sum_{y\geq0}\prob\left[R_{2}=y\right]\sum_{x\geq0}x\prob\left[R_{1}=x|R_{2}=y\right]\\
&=&\sum_{y\geq0}\sum_{x\geq0}x\prob\left[R_{1}=x|R_{2}=y\right]\prob\left[R_{2}=y\right].\\
\end{eqnarray*}

Determinemos
\begin{equation}
\esp\left[R_{1}|R_{2}=y\right]=\sum_{x\geq0}x\prob\left[R_{1}=x|R_{2}=y\right].
\end{equation}

supongamos que $y=0$, entonces
\begin{eqnarray*}
\prob\left[R_{1}=0|R_{2}=0\right]&=&1,\\
\prob\left[R_{1}=x|R_{2}=0\right]&=&0,\textrm{ para cualquier }x\geq1,\\
\end{eqnarray*}


por tanto
\begin{eqnarray*}
\esp\left[R_{1}|R_{2}=0\right]=0.
\end{eqnarray*}

Para $y=1$,
\begin{eqnarray*}
\prob\left[R_{1}=0|R_{2}=1\right]&=&0,\\
\prob\left[R_{1}=1|R_{2}=1\right]&=&1,
\end{eqnarray*}

entonces
\begin{eqnarray*}
\esp\left[R_{1}|R_{2}=1\right]=1.
\end{eqnarray*}

Para $y>1$:
\begin{eqnarray*}
\prob\left[R_{1}=0|R_{2}\geq1\right]&=&0,\\
\prob\left[R_{1}=1|R_{2}\geq1\right]&=&1,\\
\prob\left[R_{1}>1|R_{2}\geq1\right]&=&0,
\end{eqnarray*}

entonces
\begin{eqnarray*}
\esp\left[R_{1}|R_{2}=y\right]=1,\textrm{ para cualquier }y>1.
\end{eqnarray*}
es decir
\begin{eqnarray*}
\esp\left[R_{1}|R_{2}=y\right]=1,\textrm{ para cualquier }y\geq1.
\end{eqnarray*}

Entonces
\begin{eqnarray*}
\esp\left[R_{1}\right]&=&\sum_{y\geq0}\sum_{x\geq0}x\prob\left[R_{1}=x|R_{2}=y\right]\prob\left[R_{2}=y\right]=\sum_{y\geq0}\sum_{x}\esp\left[R_{1}|R_{2}=y\right]\prob\left[R_{2}=y\right]\\
&=&\sum_{y\geq0}\prob\left[R_{2}=y\right]=\sum_{y\geq1}\frac{\left(\lambda
t\right)^{k}}{k!}e^{-\lambda t}=1.
\end{eqnarray*}

Adem\'as para $k\in Z^{+}$
\begin{eqnarray*}
f_{R_{1}}\left(k\right)&=&\prob\left[R_{1}=k\right]=\sum_{n=0}^{\infty}\prob\left[R_{1}=k|R_{2}=n\right]\prob\left[R_{2}=n\right]\\
&=&\prob\left[R_{1}=k|R_{2}=0\right]\prob\left[R_{2}=0\right]+\prob\left[R_{1}=k|R_{2}=1\right]\prob\left[R_{2}=1\right]\\
&+&\prob\left[R_{1}=k|R_{2}>1\right]\prob\left[R_{2}>1\right],
\end{eqnarray*}

donde para


\begin{description}
\item[$k=0$:]
\begin{eqnarray*}
\prob\left[R_{1}=0\right]=\prob\left[R_{1}=0|R_{2}=0\right]\prob\left[R_{2}=0\right]+\prob\left[R_{1}=0|R_{2}=1\right]\prob\left[R_{2}=1\right]\\
+\prob\left[R_{1}=0|R_{2}>1\right]\prob\left[R_{2}>1\right]=\prob\left[R_{2}=0\right].
\end{eqnarray*}
\item[$k=1$:]
\begin{eqnarray*}
\prob\left[R_{1}=1\right]=\prob\left[R_{1}=1|R_{2}=0\right]\prob\left[R_{2}=0\right]+\prob\left[R_{1}=1|R_{2}=1\right]\prob\left[R_{2}=1\right]\\
+\prob\left[R_{1}=1|R_{2}>1\right]\prob\left[R_{2}>1\right]=\sum_{n=1}^{\infty}\prob\left[R_{2}=n\right].
\end{eqnarray*}

\item[$k=2$:]
\begin{eqnarray*}
\prob\left[R_{1}=2\right]=\prob\left[R_{1}=2|R_{2}=0\right]\prob\left[R_{2}=0\right]+\prob\left[R_{1}=2|R_{2}=1\right]\prob\left[R_{2}=1\right]\\
+\prob\left[R_{1}=2|R_{2}>1\right]\prob\left[R_{2}>1\right]=0.
\end{eqnarray*}

\item[$k=j$:]
\begin{eqnarray*}
\prob\left[R_{1}=j\right]=\prob\left[R_{1}=j|R_{2}=0\right]\prob\left[R_{2}=0\right]+\prob\left[R_{1}=j|R_{2}=1\right]\prob\left[R_{2}=1\right]\\
+\prob\left[R_{1}=j|R_{2}>1\right]\prob\left[R_{2}>1\right]=0.
\end{eqnarray*}
\end{description}


Por lo tanto
\begin{eqnarray*}
f_{R_{1}}\left(0\right)&=&\prob\left[R_{2}=0\right]\\
f_{R_{1}}\left(1\right)&=&\sum_{n\geq1}^{\infty}\prob\left[R_{2}=n\right]\\
f_{R_{1}}\left(j\right)&=&0,\textrm{ para }j>1.
\end{eqnarray*}



\begin{description}
\item[Pol\'itica de $k$ usuarios:]Al igual que antes, para $y\in Z^{+}$ fijo
\begin{eqnarray*}
\esp\left[R_{1}|R_{2}=y\right]=\sum_{x}x\prob\left[R_{1}=x|R_{2}=y\right].\\
\end{eqnarray*}
\end{description}
Entonces, si tomamos diversos valore para $y$:\\

$y=0$:
\begin{eqnarray*}
\prob\left[R_{1}=0|R_{2}=0\right]&=&1,\\
\prob\left[R_{1}=x|R_{2}=0\right]&=&0,\textrm{ para cualquier }x\geq1,
\end{eqnarray*}

entonces
\begin{eqnarray*}
\esp\left[R_{1}|R_{2}=0\right]=0.
\end{eqnarray*}


Para $y=1$,
\begin{eqnarray*}
\prob\left[R_{1}=0|R_{2}=1\right]&=&0,\\
\prob\left[R_{1}=1|R_{2}=1\right]&=&1,
\end{eqnarray*}

entonces {\scriptsize{
\begin{eqnarray*}
\esp\left[R_{1}|R_{2}=1\right]=1.
\end{eqnarray*}}}


Para $y=2$,
\begin{eqnarray*}
\prob\left[R_{1}=0|R_{2}=2\right]&=&0,\\
\prob\left[R_{1}=1|R_{2}=2\right]&=&1,\\
\prob\left[R_{1}=2|R_{2}=2\right]&=&1,\\
\prob\left[R_{1}=3|R_{2}=2\right]&=&0,
\end{eqnarray*}

entonces
\begin{eqnarray*}
\esp\left[R_{1}|R_{2}=2\right]=3.
\end{eqnarray*}

Para $y=3$,
\begin{eqnarray*}
\prob\left[R_{1}=0|R_{2}=3\right]&=&0,\\
\prob\left[R_{1}=1|R_{2}=3\right]&=&1,\\
\prob\left[R_{1}=2|R_{2}=3\right]&=&1,\\
\prob\left[R_{1}=3|R_{2}=3\right]&=&1,\\
\prob\left[R_{1}=4|R_{2}=3\right]&=&0,
\end{eqnarray*}

entonces
\begin{eqnarray*}
\esp\left[R_{1}|R_{2}=3\right]=6.
\end{eqnarray*}

En general, para $k\geq0$,
\begin{eqnarray*}
\prob\left[R_{1}=0|R_{2}=k\right]&=&0,\\
\prob\left[R_{1}=j|R_{2}=k\right]&=&1,\textrm{ para }1\leq j\leq k,\\
\prob\left[R_{1}=j|R_{2}=k\right]&=&0,\textrm{ para }j> k,
\end{eqnarray*}

entonces
\begin{eqnarray*}
\esp\left[R_{1}|R_{2}=k\right]=\frac{k\left(k+1\right)}{2}.
\end{eqnarray*}



Por lo tanto


\begin{eqnarray*}
\esp\left[R_{1}\right]&=&\sum_{y}\esp\left[R_{1}|R_{2}=y\right]\prob\left[R_{2}=y\right]\\
&=&\sum_{y}\prob\left[R_{2}=y\right]\frac{y\left(y+1\right)}{2}=\sum_{y\geq1}\left(\frac{y\left(y+1\right)}{2}\right)\frac{\left(\lambda t\right)^{y}}{y!}e^{-\lambda t}\\
&=&\frac{\lambda t}{2}e^{-\lambda t}\sum_{y\geq1}\left(y+1\right)\frac{\left(\lambda t\right)^{y-1}}{\left(y-1\right)!}=\frac{\lambda t}{2}e^{-\lambda t}\left(e^{\lambda t}\left(\lambda t+2\right)\right)\\
&=&\frac{\lambda t\left(\lambda t+2\right)}{2},
\end{eqnarray*}
es decir,


\begin{equation}
\esp\left[R_{1}\right]=\frac{\lambda t\left(\lambda
t+2\right)}{2}.
\end{equation}

Adem\'as para $k\in Z^{+}$ fijo
\begin{eqnarray*}
f_{R_{1}}\left(k\right)&=&\prob\left[R_{1}=k\right]=\sum_{n=0}^{\infty}\prob\left[R_{1}=k|R_{2}=n\right]\prob\left[R_{2}=n\right]\\
&=&\prob\left[R_{1}=k|R_{2}=0\right]\prob\left[R_{2}=0\right]+\prob\left[R_{1}=k|R_{2}=1\right]\prob\left[R_{2}=1\right]\\
&+&\prob\left[R_{1}=k|R_{2}=2\right]\prob\left[R_{2}=2\right]+\cdots+\prob\left[R_{1}=k|R_{2}=j\right]\prob\left[R_{2}=j\right]+\cdots+
\end{eqnarray*}
donde para

\begin{description}
\item[$k=0$:]
\begin{eqnarray*}
\prob\left[R_{1}=0\right]=\prob\left[R_{1}=0|R_{2}=0\right]\prob\left[R_{2}=0\right]+\prob\left[R_{1}=0|R_{2}=1\right]\prob\left[R_{2}=1\right]\\
+\prob\left[R_{1}=0|R_{2}=j\right]\prob\left[R_{2}=j\right]=\prob\left[R_{2}=0\right].
\end{eqnarray*}
\item[$k=1$:]
\begin{eqnarray*}
\prob\left[R_{1}=1\right]=\prob\left[R_{1}=1|R_{2}=0\right]\prob\left[R_{2}=0\right]+\prob\left[R_{1}=1|R_{2}=1\right]\prob\left[R_{2}=1\right]\\
+\prob\left[R_{1}=1|R_{2}=1\right]\prob\left[R_{2}=1\right]+\cdots+\prob\left[R_{1}=1|R_{2}=j\right]\prob\left[R_{2}=j\right]\\
=\sum_{n=1}^{\infty}\prob\left[R_{2}=n\right].
\end{eqnarray*}

\item[$k=2$:]
\begin{eqnarray*}
\prob\left[R_{1}=2\right]=\prob\left[R_{1}=2|R_{2}=0\right]\prob\left[R_{2}=0\right]+\prob\left[R_{1}=2|R_{2}=1\right]\prob\left[R_{2}=1\right]\\
+\prob\left[R_{1}=2|R_{2}=2\right]\prob\left[R_{2}=2\right]+\cdots+\prob\left[R_{1}=2|R_{2}=j\right]\prob\left[R_{2}=j\right]\\
=\sum_{n=2}^{\infty}\prob\left[R_{2}=n\right].
\end{eqnarray*}
\end{description}

En general

\begin{eqnarray*}
\prob\left[R_{1}=k\right]=\prob\left[R_{1}=k|R_{2}=0\right]\prob\left[R_{2}=0\right]+\prob\left[R_{1}=k|R_{2}=1\right]\prob\left[R_{2}=1\right]\\
+\prob\left[R_{1}=k|R_{2}=2\right]\prob\left[R_{2}=2\right]+\cdots+\prob\left[R_{1}=k|R_{2}=k\right]\prob\left[R_{2}=k\right]\\
=\sum_{n=k}^{\infty}\prob\left[R_{2}=n\right].\\
\end{eqnarray*}



Por lo tanto

\begin{eqnarray*}
f_{R_{1}}\left(k\right)&=&\prob\left[R_{1}=k\right]=\sum_{n=k}^{\infty}\prob\left[R_{2}=n\right].
\end{eqnarray*}

%__________________________________________________________________________
\section{Descripci\'on de una Red de S.V.C.}
%__________________________________________________________________________

Se definen los procesos de llegada de los usuarios a cada una de
las colas dependiendo de la llegada del servidor pero del sistema
al cu\'al no pertenece la cola en cuesti\'on:

Para el sistema 1 y el servidor del segundo sistema

\begin{eqnarray*}
F_{1,1}\left(z_{1};\zeta_{1}\right)&=&\esp\left[z_{1}^{L_{1}\left(\zeta_{1}\right)}\right]=
\sum_{k=0}^{\infty}\prob\left[L_{1}\left(\zeta_{1}\right)=k\right]z_{1}^{k}\\
F_{2,1}\left(z_{2};\zeta_{1}\right)&=&\esp\left[z_{2}^{L_{2}\left(\zeta_{1}\right)}\right]=
\sum_{k=0}^{\infty}\prob\left[L_{2}\left(\zeta_{1}\right)=k\right]z_{2}^{k}\\
F_{1,2}\left(z_{1};\zeta_{2}\right)&=&\esp\left[z_{1}^{L_{1}\left(\zeta_{2}\right)}\right]=
\sum_{k=0}^{\infty}\prob\left[L_{1}\left(\zeta_{2}\right)=k\right]z_{1}^{k}\\
F_{2,2}\left(z_{2};\zeta_{2}\right)&=&\esp\left[z_{2}^{L_{2}\left(\zeta_{2}\right)}\right]=
\sum_{k=0}^{\infty}\prob\left[L_{2}\left(\zeta_{2}\right)=k\right]z_{2}^{k}\\
\end{eqnarray*}

Ahora se definen para el segundo sistema y el servidor del primero


\begin{eqnarray*}
\hat{F}_{1,1}\left(w_{1};\tau_{1}\right)&=&\esp\left[w_{1}^{\hat{L}_{1}\left(\tau_{1}\right)}\right] =\sum_{k=0}^{\infty}\prob\left[\hat{L}_{1}\left(\tau_{1}\right)=k\right]w_{1}^{k}\\
\hat{F}_{2,1}\left(w_{2};\tau_{1}\right)&=&\esp\left[w_{2}^{\hat{L}_{2}\left(\tau_{1}\right)}\right] =\sum_{k=0}^{\infty}\prob\left[\hat{L}_{2}\left(\tau_{1}\right)=k\right]w_{2}^{k}\\
\hat{F}_{1,2}\left(w_{1};\tau_{2}\right)&=&\esp\left[w_{1}^{\hat{L}_{1}\left(\tau_{2}\right)}\right]
=\sum_{k=0}^{\infty}\prob\left[\hat{L}_{1}\left(\tau_{2}\right)=k\right]w_{1}^{k}\\
\hat{F}_{2,2}\left(w_{2};\tau_{2}\right)&=&\esp\left[w_{2}^{\hat{L}_{2}\left(\tau_{2}\right)}\right]
=\sum_{k=0}^{\infty}\prob\left[\hat{L}_{2}\left(\tau_{2}\right)=k\right]w_{2}^{k}\\
\end{eqnarray*}


Ahora, con lo anterior definamos la FGP conjunta para el segundo sistema y $\tau_{1}$:


\begin{eqnarray*}
\esp\left[w_{1}^{\hat{L}_{1}\left(\tau_{1}\right)}w_{2}^{\hat{L}_{2}\left(\tau_{1}\right)}\right]
&=&\esp\left[w_{1}^{\hat{L}_{1}\left(\tau_{1}\right)}\right]
\esp\left[w_{2}^{\hat{L}_{2}\left(\tau_{1}\right)}\right]=\hat{F}_{1,1}\left(w_{1};\tau_{1}\right)\hat{F}_{2,1}\left(w_{2};\tau_{1}\right)\\
&=&\hat{F}_{1}\left(w_{1},w_{2};\tau_{1}\right).
\end{eqnarray*}
hagamos lo mismo para $\tau_{2}$


\begin{eqnarray*}
\esp\left[w_{1}^{\hat{L}_{1}\left(\tau_{2}\right)}w_{2}^{\hat{L}_{2}\left(\tau_{2}\right)}\right]
&=&\esp\left[w_{1}^{\hat{L}_{1}\left(\tau_{2}\right)}\right]
\esp\left[w_{2}^{\hat{L}_{2}\left(\tau_{2}\right)}\right]=\hat{F}_{1,2}\left(w_{1};\tau_{2}\right)\hat{F}_{2,2}\left(w_{2};\tau_{2}\right)\\
&=&\hat{F}_{2}\left(w_{1},w_{2};\tau_{2}\right).
\end{eqnarray*}

Con respecto al sistema 1 se tiene la FGP conjunta con respecto a $\zeta_{1}$:
\begin{eqnarray*}
\esp\left[z_{1}^{L_{1}\left(\zeta_{1}\right)}z_{2}^{L_{2}\left(\zeta_{1}\right)}\right]
&=&\esp\left[z_{1}^{L_{1}\left(\zeta_{1}\right)}\right]
\esp\left[z_{2}^{L_{2}\left(\zeta_{1}\right)}\right]=F_{1,1}\left(z_{1};\zeta_{1}\right)F_{2,1}\left(z_{2};\zeta_{1}\right)\\
&=&F_{1}\left(z_{1},z_{2};\zeta_{1}\right).
\end{eqnarray*}

Finalmente
\begin{eqnarray*}
\esp\left[z_{1}^{L_{1}\left(\zeta_{2}\right)}z_{2}^{L_{2}\left(\zeta_{2}\right)}\right]
&=&\esp\left[z_{1}^{L_{1}\left(\zeta_{2}\right)}\right]
\esp\left[z_{2}^{L_{2}\left(\zeta_{2}\right)}\right]=F_{1,2}\left(z_{1};\zeta_{2}\right)F_{2,2}\left(z_{2};\zeta_{2}\right)\\
&=&F_{2}\left(z_{1},z_{2};\zeta_{2}\right).
\end{eqnarray*}

Ahora analicemos la Red de Sistemas de Visitas C\'iclicas, entonces se define la PGF conjunta al tiempo $t$ para los tiempos de visita del servidor en cada una de las colas, para comenzar a dar servicio, definidos anteriormente al tiempo
$t=\left\{\tau_{1},\tau_{2},\zeta_{1},\zeta_{2}\right\}$:

\begin{eqnarray}\label{Eq.Conjuntas}
F_{1}\left(z_{1},z_{2},w_{1},w_{2}\right)&=&\esp\left[z_{1}^{L_{1}\left(\tau_{1}\right)}z_{2}^{L_{2}\left(\tau_{1}\right)}w_{1}^{\hat{L}_{1}\left(\tau_{1}\right)}w_{2}^{\hat{L}_{2}\left(\tau_{1}\right)}\right]\\
F_{2}\left(z_{1},z_{2},w_{1},w_{2}\right)&=&\esp\left[z_{1}^{L_{1}\left(\tau_{2}\right)}z_{2}^{L_{2}\left(\tau_{2}\right)}w_{1}^{\hat{L}_{1}\left(\tau_{2}\right)}w_{2}^{\hat{L}_{2}\left(\tau_{2}\right)}\right]\\
\hat{F}_{1}\left(z_{1},z_{2},w_{1},w_{2}\right)&=&\esp\left[z_{1}^{L_{1}\left(\zeta_{1}\right)}z_{2}^{L_{2}\left(\zeta_{1}\right)}w_{1}^{\hat{L}_{1}\left(\zeta_{1}\right)}w_{2}^{\hat{L}_{2}\left(\zeta_{1}\right)}\right]\\
\hat{F}_{2}\left(z_{1},z_{2},w_{1},w_{2}\right)&=&\esp\left[z_{1}^{L_{1}\left(\zeta_{2}\right)}z_{2}^{L_{2}\left(\zeta_{2}\right)}w_{1}^{\hat{L}_{1}\left(\zeta_{2}\right)}w_{2}^{\hat{L}_{2}\left(\zeta_{2}\right)}\right]
\end{eqnarray}

Entonces, con la finalidad de encontrar el n\'umero de usuarios
presentes en el sistema cuando el servidor deja de atender una de
las colas de cualquier sistema se tiene lo siguiente


\begin{eqnarray*}
&&\esp\left[z_{1}^{L_{1}\left(\overline{\tau}_{1}\right)}z_{2}^{L_{2}\left(\overline{\tau}_{1}\right)}w_{1}^{\hat{L}_{1}\left(\overline{\tau}_{1}\right)}w_{2}^{\hat{L}_{2}\left(\overline{\tau}_{1}\right)}\right]=
\esp\left[z_{2}^{L_{2}\left(\overline{\tau}_{1}\right)}w_{1}^{\hat{L}_{1}\left(\overline{\tau}_{1}\right)}w_{2}^{\hat{L}_{2}\left(\overline{\tau}_{1}\right)}\right]\\
&=&\esp\left[z_{2}^{L_{2}\left(\tau_{1}\right)+X_{2}\left(\overline{\tau}_{1}-\tau_{1}\right)+Y_{2}\left(\overline{\tau}_{1}-\tau_{1}\right)}w_{1}^{\hat{L}_{1}\left(\tau_{1}\right)+\hat{X}_{1}\left(\overline{\tau}_{1}-\tau_{1}\right)}w_{2}^{\hat{L}_{2}\left(\tau_{1}\right)+\hat{X}_{2}\left(\overline{\tau}_{1}-\tau_{1}\right)}\right]
\end{eqnarray*}
utilizando la ecuacion dada (\ref{Eq.TiemposLlegada}), luego


\begin{eqnarray*}
&=&\esp\left[z_{2}^{L_{2}\left(\tau_{1}\right)}z_{2}^{X_{2}\left(\overline{\tau}_{1}-\tau_{1}\right)}z_{2}^{Y_{2}\left(\overline{\tau}_{1}-\tau_{1}\right)}w_{1}^{\hat{L}_{1}\left(\tau_{1}\right)}w_{1}^{\hat{X}_{1}\left(\overline{\tau}_{1}-\tau_{1}\right)}w_{2}^{\hat{L}_{2}\left(\tau_{1}\right)}w_{2}^{\hat{X}_{2}\left(\overline{\tau}_{1}-\tau_{1}\right)}\right]\\
&=&\esp\left[z_{2}^{L_{2}\left(\tau_{1}\right)}\left\{w_{1}^{\hat{L}_{1}\left(\tau_{1}\right)}w_{2}^{\hat{L}_{2}\left(\tau_{1}\right)}\right\}\left\{z_{2}^{X_{2}\left(\overline{\tau}_{1}-\tau_{1}\right)}
z_{2}^{Y_{2}\left(\overline{\tau}_{1}-\tau_{1}\right)}w_{1}^{\hat{X}_{1}\left(\overline{\tau}_{1}-\tau_{1}\right)}w_{2}^{\hat{X}_{2}\left(\overline{\tau}_{1}-\tau_{1}\right)}\right\}\right]\\
\end{eqnarray*}
Aplicando la ecuaci\'on (\ref{Eq.Cero})

\begin{eqnarray*}
&=&\esp\left[z_{2}^{L_{2}\left(\tau_{1}\right)}\left\{z_{2}^{X_{2}\left(\overline{\tau}_{1}-\tau_{1}\right)}z_{2}^{Y_{2}\left(\overline{\tau}_{1}-\tau_{1}\right)}w_{1}^{\hat{X}_{1}\left(\overline{\tau}_{1}-\tau_{1}\right)}w_{2}^{\hat{X}_{2}\left(\overline{\tau}_{1}-\tau_{1}\right)}\right\}\right]\esp\left[w_{1}^{\hat{L}_{1}\left(\tau_{1}\right)}w_{2}^{\hat{L}_{2}\left(\tau_{1}\right)}\right]
\end{eqnarray*}
dado que los arribos a cada una de las colas son independientes, podemos separar la esperanza para los procesos de llegada a $Q_{1}$ y $Q_{2}$ en $\tau_{1}$

Recordando que $\tilde{X}_{2}\left(z_{2}\right)=X_{2}\left(z_{2}\right)+Y_{2}\left(z_{2}\right)$ se tiene


\begin{eqnarray*}
&=&\esp\left[z_{2}^{L_{2}\left(\tau_{1}\right)}\left\{z_{2}^{\tilde{X}_{2}\left(\overline{\tau}_{1}-\tau_{1}\right)}w_{1}^{\hat{X}_{1}\left(\overline{\tau}_{1}-\tau_{1}\right)}w_{2}^{\hat{X}_{2}\left(\overline{\tau}_{1}-\tau_{1}\right)}\right\}\right]\esp\left[w_{1}^{\hat{L}_{1}\left(\tau_{1}\right)}w_{2}^{\hat{L}_{2}\left(\tau_{1}\right)}\right]\\
&=&\esp\left[z_{2}^{L_{2}\left(\tau_{1}\right)}\left\{\tilde{P}_{2}\left(z_{2}\right)^{\overline{\tau}_{1}-\tau_{1}}\hat{P}_{1}\left(w_{1}\right)^{\overline{\tau}_{1}-\tau_{1}}\hat{P}_{2}\left(w_{2}\right)^{\overline{\tau}_{1}-\tau_{1}}\right\}\right]\esp\left[w_{1}^{\hat{L}_{1}\left(\tau_{1}\right)}w_{2}^{\hat{L}_{2}\left(\tau_{1}\right)}\right]\\
&=&\esp\left[z_{2}^{L_{2}\left(\tau_{1}\right)}\left\{\tilde{P}_{2}\left(z_{2}\right)\hat{P}_{1}\left(w_{1}\right)\hat{P}_{2}\left(w_{2}\right)\right\}^{\overline{\tau}_{1}-\tau_{1}}\right]\esp\left[w_{1}^{\hat{L}_{1}\left(\tau_{1}\right)}w_{2}^{\hat{L}_{2}\left(\tau_{1}\right)}\right]\\
\end{eqnarray*}

Entonces


\begin{eqnarray*}
&=&\esp\left[z_{2}^{L_{2}\left(\tau_{1}\right)}\theta_{1}\left(\tilde{P}_{2}\left(z_{2}\right)\hat{P}_{1}\left(w_{1}\right)\hat{P}_{2}\left(w_{2}\right)\right)^{L_{1}\left(\tau_{1}\right)}\right]\esp\left[w_{1}^{\hat{L}_{1}\left(\tau_{1}\right)}w_{2}^{\hat{L}_{2}\left(\tau_{1}\right)}\right]\\
&=&F_{1}\left(\theta_{1}\left(\tilde{P}_{2}\left(z_{2}\right)\hat{P}_{1}\left(w_{1}\right)\hat{P}_{2}\left(w_{2}\right)\right),z{2}\right)\hat{F}_{1}\left(w_{1},w_{2};\tau_{1}\right)\\
&\equiv&
F_{1}\left(\theta_{1}\left(\tilde{P}_{2}\left(z_{2}\right)\hat{P}_{1}\left(w_{1}\right)\hat{P}_{2}\left(w_{2}\right)\right),z_{2},w_{1},w_{2}\right)
\end{eqnarray*}

Las igualdades anteriores son ciertas pues el n\'umero de usuarios
que llegan a $\hat{Q}_{2}$ durante el intervalo
$\left[\tau_{1},\overline{\tau}_{1}\right]$ a\'un no han sido
atendidos por el servidor del sistema $2$ y por tanto a\'un no
pueden pasar al sistema $1$ por $Q_{2}$. Por tanto el n\'umero de
usuarios que pasan de $\hat{Q}_{2}$ a $Q_{2}$ en el intervalo de
tiempo $\left[\tau_{1},\overline{\tau}_{1}\right]$ depende de la
pol\'itica de traslado entre los dos sistemas, conforme a la
secci\'on anterior.\smallskip

Por lo tanto
\begin{equation}\label{Eq.Fs}
\esp\left[z_{1}^{L_{1}\left(\overline{\tau}_{1}\right)}z_{2}^{L_{2}\left(\overline{\tau}_{1}\right)}w_{1}^{\hat{L}_{1}\left(\overline{\tau}_{1}\right)}w_{2}^{\hat{L}_{2}\left(\overline{\tau}_{1}\right)}\right]=F_{1}\left(\theta_{1}\left(\tilde{P}_{2}\left(z_{2}\right)\hat{P}_{1}\left(w_{1}\right)\hat{P}_{2}\left(w_{2}\right)\right),z_{2},w_{1},w_{2}\right)
\end{equation}


Utilizando un razonamiento an\'alogo para $\overline{\tau}_{2}$:



\begin{eqnarray*}
&&\esp\left[z_{1}^{L_{1}\left(\overline{\tau}_{2}\right)}z_{2}^{L_{2}\left(\overline{\tau}_{2}\right)}w_{1}^{\hat{L}_{1}\left(\overline{\tau}_{2}\right)}w_{2}^{\hat{L}_{2}\left(\overline{\tau}_{2}\right)}\right]=
\esp\left[z_{1}^{L_{1}\left(\overline{\tau}_{2}\right)}w_{1}^{\hat{L}_{1}\left(\overline{\tau}_{2}\right)}w_{2}^{\hat{L}_{2}\left(\overline{\tau}_{2}\right)}\right]\\
&=&\esp\left[z_{1}^{L_{1}\left(\tau_{2}\right)+X_{1}\left(\overline{\tau}_{2}-\tau_{2}\right)}w_{1}^{\hat{L}_{1}\left(\tau_{2}\right)+\hat{X}_{1}\left(\overline{\tau}_{2}-\tau_{2}\right)}w_{2}^{\hat{L}_{2}\left(\tau_{2}\right)+\hat{X}_{2}\left(\overline{\tau}_{2}-\tau_{2}\right)}\right]\\
&=&\esp\left[z_{1}^{L_{1}\left(\tau_{2}\right)}z_{1}^{X_{1}\left(\overline{\tau}_{2}-\tau_{2}\right)}w_{1}^{\hat{L}_{1}\left(\tau_{2}\right)}w_{1}^{\hat{X}_{1}\left(\overline{\tau}_{2}-\tau_{2}\right)}w_{2}^{\hat{L}_{2}\left(\tau_{2}\right)}w_{2}^{\hat{X}_{2}\left(\overline{\tau}_{2}-\tau_{2}\right)}\right]\\
&=&\esp\left[z_{1}^{L_{1}\left(\tau_{2}\right)}z_{1}^{X_{1}\left(\overline{\tau}_{2}-\tau_{2}\right)}w_{1}^{\hat{X}_{1}\left(\overline{\tau}_{2}-\tau_{2}\right)}w_{2}^{\hat{X}_{2}\left(\overline{\tau}_{2}-\tau_{2}\right)}\right]\esp\left[w_{1}^{\hat{L}_{1}\left(\tau_{2}\right)}w_{2}^{\hat{L}_{2}\left(\tau_{2}\right)}\right]\\
&=&\esp\left[z_{1}^{L_{1}\left(\tau_{2}\right)}P_{1}\left(z_{1}\right)^{\overline{\tau}_{2}-\tau_{2}}\hat{P}_{1}\left(w_{1}\right)^{\overline{\tau}_{2}-\tau_{2}}\hat{P}_{2}\left(w_{2}\right)^{\overline{\tau}_{2}-\tau_{2}}\right]
\esp\left[w_{1}^{\hat{L}_{1}\left(\tau_{2}\right)}w_{2}^{\hat{L}_{2}\left(\tau_{2}\right)}\right]
\end{eqnarray*}
utlizando la proposici\'on relacionada con la ruina del jugador


\begin{eqnarray*}
&=&\esp\left[z_{1}^{L_{1}\left(\tau_{2}\right)}\left\{P_{1}\left(z_{1}\right)\hat{P}_{1}\left(w_{1}\right)\hat{P}_{2}\left(w_{2}\right)\right\}^{\overline{\tau}_{2}-\tau_{2}}\right]
\esp\left[w_{1}^{\hat{L}_{1}\left(\tau_{2}\right)}w_{2}^{\hat{L}_{2}\left(\tau_{2}\right)}\right]\\
&=&\esp\left[z_{1}^{L_{1}\left(\tau_{2}\right)}\tilde{\theta}_{2}\left(P_{1}\left(z_{1}\right)\hat{P}_{1}\left(w_{1}\right)\hat{P}_{2}\left(w_{2}\right)\right)^{L_{2}\left(\tau_{2}\right)}\right]
\esp\left[w_{1}^{\hat{L}_{1}\left(\tau_{2}\right)}w_{2}^{\hat{L}_{2}\left(\tau_{2}\right)}\right]\\
&=&F_{2}\left(z_{1},\tilde{\theta}_{2}\left(P_{1}\left(z_{1}\right)\hat{P}_{1}\left(w_{1}\right)\hat{P}_{2}\left(w_{2}\right)\right)\right)
\hat{F}_{2}\left(w_{1},w_{2};\tau_{2}\right)\\
\end{eqnarray*}


entonces se define
\begin{eqnarray}
\esp\left[z_{1}^{L_{1}\left(\overline{\tau}_{2}\right)}z_{2}^{L_{2}\left(\overline{\tau}_{2}\right)}w_{1}^{\hat{L}_{1}\left(\overline{\tau}_{2}\right)}w_{2}^{\hat{L}_{2}\left(\overline{\tau}_{2}\right)}\right]=F_{2}\left(z_{1},\tilde{\theta}_{2}\left(P_{1}\left(z_{1}\right)\hat{P}_{1}\left(w_{1}\right)\hat{P}_{2}\left(w_{2}\right)\right),w_{1},w_{2}\right)\\
\equiv F_{2}\left(z_{1},\tilde{\theta}_{2}\left(P_{1}\left(z_{1}\right)\hat{P}_{1}\left(w_{1}\right)\hat{P}_{2}\left(w_{2}\right)\right)\right)
\hat{F}_{2}\left(w_{1},w_{2};\tau_{2}\right)
\end{eqnarray}
Ahora para $\overline{\zeta}_{1}:$
\begin{eqnarray*}
&&\esp\left[z_{1}^{L_{1}\left(\overline{\zeta}_{1}\right)}z_{2}^{L_{2}\left(\overline{\zeta}_{1}\right)}w_{1}^{\hat{L}_{1}\left(\overline{\zeta}_{1}\right)}w_{2}^{\hat{L}_{2}\left(\overline{\zeta}_{1}\right)}\right]=
\esp\left[z_{1}^{L_{1}\left(\overline{\zeta}_{1}\right)}z_{2}^{L_{2}\left(\overline{\zeta}_{1}\right)}w_{2}^{\hat{L}_{2}\left(\overline{\zeta}_{1}\right)}\right]\\
%&=&\esp\left[z_{1}^{L_{1}\left(\zeta_{1}\right)+X_{1}\left(\overline{\zeta}_{1}-\zeta_{1}\right)}z_{2}^{L_{2}\left(\zeta_{1}\right)+X_{2}\left(\overline{\zeta}_{1}-\zeta_{1}\right)+\hat{Y}_{2}\left(\overline{\zeta}_{1}-\zeta_{1}\right)}w_{2}^{\hat{L}_{2}\left(\zeta_{1}\right)+\hat{X}_{2}\left(\overline{\zeta}_{1}-\zeta_{1}\right)}\right]\\
&=&\esp\left[z_{1}^{L_{1}\left(\zeta_{1}\right)}z_{1}^{X_{1}\left(\overline{\zeta}_{1}-\zeta_{1}\right)}z_{2}^{L_{2}\left(\zeta_{1}\right)}z_{2}^{X_{2}\left(\overline{\zeta}_{1}-\zeta_{1}\right)}
z_{2}^{Y_{2}\left(\overline{\zeta}_{1}-\zeta_{1}\right)}w_{2}^{\hat{L}_{2}\left(\zeta_{1}\right)}w_{2}^{\hat{X}_{2}\left(\overline{\zeta}_{1}-\zeta_{1}\right)}\right]\\
&=&\esp\left[z_{1}^{L_{1}\left(\zeta_{1}\right)}z_{2}^{L_{2}\left(\zeta_{1}\right)}\right]\esp\left[z_{1}^{X_{1}\left(\overline{\zeta}_{1}-\zeta_{1}\right)}z_{2}^{\tilde{X}_{2}\left(\overline{\zeta}_{1}-\zeta_{1}\right)}w_{2}^{\hat{X}_{2}\left(\overline{\zeta}_{1}-\zeta_{1}\right)}w_{2}^{\hat{L}_{2}\left(\zeta_{1}\right)}\right]\\
&=&\esp\left[z_{1}^{L_{1}\left(\zeta_{1}\right)}z_{2}^{L_{2}\left(\zeta_{1}\right)}\right]
\esp\left[P_{1}\left(z_{1}\right)^{\overline{\zeta}_{1}-\zeta_{1}}\tilde{P}_{2}\left(z_{2}\right)^{\overline{\zeta}_{1}-\zeta_{1}}\hat{P}_{2}\left(w_{2}\right)^{\overline{\zeta}_{1}-\zeta_{1}}w_{2}^{\hat{L}_{2}\left(\zeta_{1}\right)}\right]\\
&=&\esp\left[z_{1}^{L_{1}\left(\zeta_{1}\right)}z_{2}^{L_{2}\left(\zeta_{1}\right)}\right]
\esp\left[\left\{P_{1}\left(z_{1}\right)\tilde{P}_{2}\left(z_{2}\right)\hat{P}_{2}\left(w_{2}\right)\right\}^{\overline{\zeta}_{1}-\zeta_{1}}w_{2}^{\hat{L}_{2}\left(\zeta_{1}\right)}\right]\\
&=&\esp\left[z_{1}^{L_{1}\left(\zeta_{1}\right)}z_{2}^{L_{2}\left(\zeta_{1}\right)}\right]
\esp\left[\hat{\theta}_{1}\left(P_{1}\left(z_{1}\right)\tilde{P}_{2}\left(z_{2}\right)\hat{P}_{2}\left(w_{2}\right)\right)^{\hat{L}_{1}\left(\zeta_{1}\right)}w_{2}^{\hat{L}_{2}\left(\zeta_{1}\right)}\right]\\
&=&F_{1}\left(z_{1},z_{2};\zeta_{1}\right)\hat{F}_{1}\left(\hat{\theta}_{1}\left(P_{1}\left(z_{1}\right)\tilde{P}_{2}\left(z_{2}\right)\hat{P}_{2}\left(w_{2}\right)\right),w_{2}\right)
\end{eqnarray*}


es decir
\begin{eqnarray}
\esp\left[z_{1}^{L_{1}\left(\overline{\zeta}_{1}\right)}z_{2}^{L_{2}\left(\overline{\zeta}_{1}\right)}w_{1}^{\hat{L}_{1}\left(\overline{\zeta}_{1}\right)}w_{2}^{\hat{L}_{2}\left(\overline{\zeta}_{1}\right)}\right]=\hat{F}_{1}\left(z_{1},z_{2},\hat{\theta}_{1}\left(P_{1}\left(z_{1}\right)\tilde{P}_{2}\left(z_{2}\right)\hat{P}_{2}\left(w_{2}\right)\right),w_{2}\right)\\
&=&F_{1}\left(z_{1},z_{2};\zeta_{1}\right)\hat{F}_{1}\left(\hat{\theta}_{1}\left(P_{1}\left(z_{1}\right)\tilde{P}_{2}\left(z_{2}\right)\hat{P}_{2}\left(w_{2}\right)\right),w_{2}\right).
\end{eqnarray}


Finalmente para $\overline{\zeta}_{2}:$
\begin{eqnarray*}
&&\esp\left[z_{1}^{L_{1}\left(\overline{\zeta}_{2}\right)}z_{2}^{L_{2}\left(\overline{\zeta}_{2}\right)}w_{1}^{\hat{L}_{1}\left(\overline{\zeta}_{2}\right)}w_{2}^{\hat{L}_{2}\left(\overline{\zeta}_{2}\right)}\right]=
\esp\left[z_{1}^{L_{1}\left(\overline{\zeta}_{2}\right)}z_{2}^{L_{2}\left(\overline{\zeta}_{2}\right)}w_{1}^{\hat{L}_{1}\left(\overline{\zeta}_{2}\right)}\right]\\
%&=&\esp\left[z_{1}^{L_{1}\left(\zeta_{2}\right)+X_{1}\left(\overline{\zeta}_{2}-\zeta_{2}\right)}z_{2}^{L_{2}\left(\zeta_{2}\right)+X_{2}\left(\overline{\zeta}_{2}-\zeta_{2}\right)+\hat{Y}_{2}\left(\overline{\zeta}_{2}-\zeta_{2}\right)}w_{1}^{\hat{L}_{1}\left(\zeta_{2}\right)+\hat{X}_{1}\left(\overline{\zeta}_{2}-\zeta_{2}\right)}\right]\\
&=&\esp\left[z_{1}^{L_{1}\left(\zeta_{2}\right)}z_{1}^{X_{1}\left(\overline{\zeta}_{2}-\zeta_{2}\right)}z_{2}^{L_{2}\left(\zeta_{2}\right)}z_{2}^{X_{2}\left(\overline{\zeta}_{2}-\zeta_{2}\right)}
z_{2}^{Y_{2}\left(\overline{\zeta}_{2}-\zeta_{2}\right)}w_{1}^{\hat{L}_{1}\left(\zeta_{2}\right)}w_{1}^{\hat{X}_{1}\left(\overline{\zeta}_{2}-\zeta_{2}\right)}\right]\\
&=&\esp\left[z_{1}^{L_{1}\left(\zeta_{2}\right)}z_{2}^{L_{2}\left(\zeta_{2}\right)}\right]\esp\left[z_{1}^{X_{1}\left(\overline{\zeta}_{2}-\zeta_{2}\right)}z_{2}^{\tilde{X}_{2}\left(\overline{\zeta}_{2}-\zeta_{2}\right)}w_{1}^{\hat{X}_{1}\left(\overline{\zeta}_{2}-\zeta_{2}\right)}w_{1}^{\hat{L}_{1}\left(\zeta_{2}\right)}\right]\\
&=&\esp\left[z_{1}^{L_{1}\left(\zeta_{2}\right)}z_{2}^{L_{2}\left(\zeta_{2}\right)}\right]\esp\left[P_{1}\left(z_{1}\right)^{\overline{\zeta}_{2}-\zeta_{2}}\tilde{P}_{2}\left(z_{2}\right)^{\overline{\zeta}_{2}-\zeta_{2}}\hat{P}\left(w_{1}\right)^{\overline{\zeta}_{2}-\zeta_{2}}w_{1}^{\hat{L}_{1}\left(\zeta_{2}\right)}\right]\\
&=&\esp\left[z_{1}^{L_{1}\left(\zeta_{2}\right)}z_{2}^{L_{2}\left(\zeta_{2}\right)}\right]\esp\left[w_{1}^{\hat{L}_{1}\left(\zeta_{2}\right)}\left\{P_{1}\left(z_{1}\right)\tilde{P}_{2}\left(z_{2}\right)\hat{P}\left(w_{1}\right)\right\}^{\overline{\zeta}_{2}-\zeta_{2}}\right]\\
&=&\esp\left[z_{1}^{L_{1}\left(\zeta_{2}\right)}z_{2}^{L_{2}\left(\zeta_{2}\right)}\right]\esp\left[w_{1}^{\hat{L}_{1}\left(\zeta_{2}\right)}\hat{\theta}_{2}\left(P_{1}\left(z_{1}\right)\tilde{P}_{2}\left(z_{2}\right)\hat{P}\left(w_{1}\right)\right)^{\hat{L}_{2}\zeta_{2}}\right]\\
&=&F_{2}\left(z_{1},z_{2};\zeta_{2}\right)\hat{F}_{2}\left(w_{1},\hat{\theta}_{2}\left(P_{1}\left(z_{1}\right)\tilde{P}_{2}\left(z_{2}\right)\hat{P}_{1}\left(w_{1}\right)\right)\right]\\
%&\equiv&\hat{F}_{2}\left(z_{1},z_{2},w_{1},\hat{\theta}_{2}\left(P_{1}\left(z_{1}\right)\tilde{P}_{2}\left(z_{2}\right)\hat{P}_{1}\left(w_{1}\right)\right)\right)
\end{eqnarray*}


%__________________________________________________________________________
\section{Ecuaciones Recursivas para la R.S.V.C.}
%__________________________________________________________________________


es decir
\begin{eqnarray}
\esp\left[z_{1}^{L_{1}\left(\overline{\zeta}_{2}\right)}z_{2}^{L_{2}\left(\overline{\zeta}_{2}\right)}w_{1}^{\hat{L}_{1}\left(\overline{\zeta}_{2}\right)}w_{2}^{\hat{L}_{2}\left(\overline{\zeta}_{2}\right)}\right]=\hat{F}_{2}\left(z_{1},z_{2},w_{1},\hat{\theta}_{2}\left(P_{1}\left(z_{1}\right)\tilde{P}_{2}\left(z_{2}\right)\hat{P}_{1}\left(w_{1}\right)\right)\right)\\
=F_{2}\left(z_{1},z_{2};\zeta_{2}\right)\hat{F}_{2}\left(w_{1},\hat{\theta}_{2}\left(P_{1}\left(z_{1}\right)\tilde{P}_{2}\left(z_{2}\right)\hat{P}_{1}\left(w_{1}\right)\right)\right]\\
\end{eqnarray}

Con lo desarrollado hasta ahora podemos encontrar las ecuaciones
recursivas que modelan la Red de Sistemas de Visitas C\'iclicas
(R.S.V.C):
\begin{eqnarray*}
&&F_{2}\left(z_{1},z_{2},w_{1},w_{2}\right)=R_{1}\left(z_{1},z_{2},w_{1},w_{2}\right)\esp\left[z_{1}^{L_{1}\left(\overline{\tau}_{1}\right)}z_{2}^{L_{2}\left(\overline{\tau}_{1}\right)}w_{1}^{\hat{L}_{1}\left(\overline{\tau}_{1}\right)}w_{2}^{\hat{L}_{2}\left(\overline{\tau}_{1}\right)}\right]\\
%&=&R_{1}\left(P_{1}\left(z_{1}\right)\tilde{P}_{2}\left(z_{2}\right)\hat{P}_{1}\left(w_{1}\right)\hat{P}_{2}\left(w_{2}\right)\right)
%F_{1}\left(\theta\left(\tilde{P}_{2}\left(z_{2}\right)\hat{P}_{1}\left(w_{1}\right)\hat{P}_{2}\left(w_{2}\right)\right),z_{2},w_{1},w_{2}\right)\\
&&F_{1}\left(z_{1},z_{2},w_{1},w_{2}\right)=R_{2}\left(z_{1},z_{2},w_{1},w_{2}\right)\esp\left[z_{1}^{L_{1}\left(\overline{\tau}_{2}\right)}z_{2}^{L_{2}\left(\overline{\tau}_{2}\right)}w_{1}^{\hat{L}_{1}\left(\overline{\tau}_{2}\right)}w_{2}^{\hat{L}_{2}\left(\overline{\tau}_{1}\right)}\right]\\
%&=&R_{2}\left(P_{1}\left(z_{1}\right)\tilde{P}_{2}\left(z_{2}\right)\hat{P}_{1}\left(w_{1}\right)\hat{P}_{2}\left(w_{2}\right)\right)F_{2}\left(z_{1},\tilde{\theta}_{2}\left(P_{1}\left(z_{1}\right)\hat{P}_{1}\left(w_{1}\right)\hat{P}_{2}\left(w_{2}\right)\right),w_{1},w_{2}\right)\\
&&\hat{F}_{2}\left(z_{1},z_{2},w_{1},w_{2}\right)=\hat{R}_{1}\left(z_{1},z_{2},w_{1},w_{2}\right)\esp\left[z_{1}^{L_{1}\left(\overline{\zeta}_{1}\right)}z_{2}^{L_{2}\left(\overline{\zeta}_{1}\right)}w_{1}^{\hat{L}_{1}\left(\overline{\zeta}_{1}\right)}w_{2}^{\hat{L}_{2}\left(\overline{\zeta}_{1}\right)}\right]\\
%&=&\hat{R}_{1}\left(P_{1}\left(z_{1}\right)\tilde{P}_{2}\left(z_{2}\right)\hat{P}_{1}\left(w_{1}\right)\hat{P}_{2}\left(w_{2}\right)\right)\hat{F}_{1}\left(z_{1},z_{2},\hat{\theta}_{1}\left(P_{1}\left(z_{1}\right)\tilde{P}_{2}\left(z_{2}\right)\hat{P}_{2}\left(w_{2}\right)\right),w_{2}\right)
\end{eqnarray*}


y finalmente
\begin{eqnarray*}
&&\hat{F}_{1}\left(z_{1},z_{2},w_{1},w_{2}\right)=\hat{R}_{2}\left(z_{1},z_{2},w_{1},w_{2}\right)\esp\left[z_{1}^{L_{1}\left(\overline{\zeta}_{2}\right)}z_{2}^{L_{2}\left(\overline{\zeta}_{2}\right)}w_{1}^{\hat{L}_{1}\left(\overline{\zeta}_{2}\right)}w_{2}^{\hat{L}_{2}\left(\overline{\zeta}_{2}\right)}\right]\\
%&=&\hat{R}_{2}\left(P_{1}\left(z_{1}\right)\tilde{P}_{2}\left(z_{2}\right)\hat{P}_{1}\left(w_{1}\right)\hat{P}_{2}\left(w_{2}\right)\right)\hat{F}_{2}\left(z_{1},z_{2},w_{1},\hat{\theta}_{2}\left(P_{1}\left(z_{1}\right)\tilde{P}_{2}\left(z_{2}\right)\hat{P}_{1}\left(w_{1}\right)\right)\right)
\end{eqnarray*}

que son equivalentes a las siguientes ecuaciones
\begin{eqnarray*}
F_{2}\left(z_{1},z_{2},w_{1},w_{2}\right)&=&R_{1}\left(P_{1}\left(z_{1}\right)\tilde{P}_{2}\left(z_{2}\right)\prod_{i=1}^{2}
\hat{P}_{i}\left(w_{i}\right)\right)\\
&&F_{1}\left(\theta_{1}\left(\tilde{P}_{2}\left(z_{2}\right)\hat{P}_{1}\left(w_{1}\right)\hat{P}_{2}\left(w_{2}\right)\right),z_{2},w_{1},w_{2}\right)\\
\end{eqnarray*}


\begin{eqnarray*}
F_{1}\left(z_{1},z_{2},w_{1},w_{2}\right)&=&R_{2}\left(P_{1}\left(z_{1}\right)\tilde{P}_{2}\left(z_{2}\right)\prod_{i=1}^{2}
\hat{P}_{i}\left(w_{i}\right)\right)\\
&&F_{2}\left(z_{1},\tilde{\theta}_{2}\left(P_{1}\left(z_{1}\right)\hat{P}_{1}\left(w_{1}\right)\hat{P}_{2}\left(w_{2}\right)\right),w_{1},w_{2}\right)\\
\end{eqnarray*}

%_________________________________________________________________________________________________
\subsection{Tiempos de Traslado del Servidor}
%_________________________________________________________________________________________________



\begin{eqnarray*}
\hat{F}_{2}\left(z_{1},z_{2},w_{1},w_{2}\right)&=&\hat{R}_{1}\left(P_{1}\left(z_{1}\right)\tilde{P}_{2}\left(z_{2}\right)\prod_{i=1}^{2}
\hat{P}_{i}\left(w_{i}\right)\right)\\
&&\hat{F}_{1}\left(z_{1},z_{2},\hat{\theta}_{1}\left(P_{1}\left(z_{1}\right)\tilde{P}_{2}\left(z_{2}\right)\hat{P}_{2}\left(w_{2}\right)\right),w_{2}\right)\\
\end{eqnarray*}

\begin{eqnarray*}
\hat{F}_{1}\left(z_{1},z_{2},w_{1},w_{2}\right)&=&\hat{R}_{2}\left(P_{1}\left(z_{1}\right)\tilde{P}_{2}\left(z_{2}\right)\prod_{i=1}^{2}
\hat{P}_{i}\left(w_{i}\right)\right)\\
&&\hat{F}_{2}\left(z_{1},z_{2},w_{1},\hat{\theta}_{2}\left(P_{1}\left(z_{1}\right)\tilde{P}_{2}\left(z_{2}\right)\hat{P}_{1}\left(w_{1}\right)\right)\right)
\end{eqnarray*}


Para
%\begin{multicols}{2}

\begin{eqnarray}\label{Ec.R1}
R_{1}\left(\mathbf{z,w}\right)=R_{1}\left(P_{1}\left(z_{1}\right)\tilde{P}_{2}\left(z_{2}\right)\hat{P}_{1}\left(w_{1}\right)\hat{P}_{2}\left(w_{2}\right)\right)
\end{eqnarray}
%\end{multicols}

se tiene que


\begin{eqnarray*}
\frac{\partial R_{1}\left(\mathbf{z,w}\right)}{\partial
z_{1}}|_{\mathbf{z,w}=1}&=&R_{1}^{(1)}\left(1\right)P_{1}^{(1)}\left(1\right)=r_{1}\mu_{1},\\
\frac{\partial R_{1}\left(\mathbf{z,w}\right)}{\partial
z_{2}}|_{\mathbf{z,w}=1}&=&R_{1}^{(1)}\left(1\right)\tilde{P}_{2}^{(1)}\left(1\right)=r_{1}\tilde{\mu}_{2},\\
\frac{\partial R_{1}\left(\mathbf{z,w}\right)}{\partial
w_{1}}|_{\mathbf{z,w}=1}&=&R_{1}^{(1)}\left(1\right)\hat{P}_{1}^{(1)}\left(1\right)=r_{1}\hat{\mu}_{1},\\
\frac{\partial R_{1}\left(\mathbf{z,w}\right)}{\partial
w_{2}}|_{\mathbf{z,w}=1}&=&R_{1}^{(1)}\left(1\right)\hat{P}_{2}^{(1)}\left(1\right)=r_{1}\hat{\mu}_{2},
\end{eqnarray*}

An\'alogamente se tiene

\begin{eqnarray}
R_{2}\left(\mathbf{z,w}\right)=R_{2}\left(P_{1}\left(z_{1}\right)\tilde{P}_{2}\left(z_{2}\right)\hat{P}_{1}\left(w_{1}\right)\hat{P}_{2}\left(w_{2}\right)\right)
\end{eqnarray}


\begin{eqnarray*}
\frac{\partial R_{2}\left(\mathbf{z,w}\right)}{\partial
z_{1}}|_{\mathbf{z,w}=1}&=&R_{2}^{(1)}\left(1\right)P_{1}^{(1)}\left(1\right)=r_{2}\mu_{1},\\
\frac{\partial R_{2}\left(\mathbf{z,w}\right)}{\partial
z_{2}}|_{\mathbf{z,w}=1}&=&R_{2}^{(1)}\left(1\right)\tilde{P}_{2}^{(1)}\left(1\right)=r_{2}\tilde{\mu}_{2},\\
\frac{\partial R_{2}\left(\mathbf{z,w}\right)}{\partial
w_{1}}|_{\mathbf{z,w}=1}&=&R_{2}^{(1)}\left(1\right)\hat{P}_{1}^{(1)}\left(1\right)=r_{2}\hat{\mu}_{1},\\
\frac{\partial R_{2}\left(\mathbf{z,w}\right)}{\partial
w_{2}}|_{\mathbf{z,w}=1}&=&R_{2}^{(1)}\left(1\right)\hat{P}_{2}^{(1)}\left(1\right)=r_{2}\hat{\mu}_{2},\\
\end{eqnarray*}

Para el segundo sistema:

\begin{eqnarray}
\hat{R}_{1}\left(\mathbf{z,w}\right)=\hat{R}_{1}\left(P_{1}\left(z_{1}\right)\tilde{P}_{2}\left(z_{2}\right)\hat{P}_{1}\left(w_{1}\right)\hat{P}_{2}\left(w_{2}\right)\right)
\end{eqnarray}


\begin{eqnarray*}
\frac{\partial \hat{R}_{1}\left(\mathbf{z,w}\right)}{\partial
z_{1}}|_{\mathbf{z,w}=1}&=&\hat{R}_{1}^{(1)}\left(1\right)P_{1}^{(1)}\left(1\right)=\hat{r}_{1}\mu_{1},\\
\frac{\partial \hat{R}_{1}\left(\mathbf{z,w}\right)}{\partial
z_{2}}|_{\mathbf{z,w}=1}&=&\hat{R}_{1}^{(1)}\left(1\right)\tilde{P}_{2}^{(1)}\left(1\right)=\hat{r}_{1}\tilde{\mu}_{2},\\
\frac{\partial \hat{R}_{1}\left(\mathbf{z,w}\right)}{\partial
w_{1}}|_{\mathbf{z,w}=1}&=&\hat{R}_{1}^{(1)}\left(1\right)\hat{P}_{1}^{(1)}\left(1\right)=\hat{r}_{1}\hat{\mu}_{1},\\
\frac{\partial \hat{R}_{1}\left(\mathbf{z,w}\right)}{\partial
w_{2}}|_{\mathbf{z,w}=1}&=&\hat{R}_{1}^{(1)}\left(1\right)\hat{P}_{2}^{(1)}\left(1\right)=\hat{r}_{1}\hat{\mu}_{2},
\end{eqnarray*}

Finalmente

\begin{eqnarray}
\hat{R}_{2}\left(\mathbf{z,w}\right)=\hat{R}_{2}\left(P_{1}\left(z_{1}\right)\tilde{P}_{2}\left(z_{2}\right)\hat{P}_{1}\left(w_{1}\right)\hat{P}_{2}\left(w_{2}\right)\right)
\end{eqnarray}



\begin{eqnarray*}
\frac{\partial \hat{R}_{2}\left(\mathbf{z,w}\right)}{\partial
z_{1}}|_{\mathbf{z,w}=1}&=&\hat{R}_{2}^{(1)}\left(1\right)P_{1}^{(1)}\left(1\right)=\hat{r}_{2}\mu_{1},\\
\frac{\partial \hat{R}_{2}\left(\mathbf{z,w}\right)}{\partial
z_{2}}|_{\mathbf{z,w}=1}&=&\hat{R}_{2}^{(1)}\left(1\right)\tilde{P}_{2}^{(1)}\left(1\right)=\hat{r}_{2}\tilde{\mu}_{2},\\
\frac{\partial \hat{R}_{2}\left(\mathbf{z,w}\right)}{\partial
w_{1}}|_{\mathbf{z,w}=1}&=&\hat{R}_{2}^{(1)}\left(1\right)\hat{P}_{1}^{(1)}\left(1\right)=\hat{r}_{2}\hat{\mu}_{1},\\
\frac{\partial \hat{R}_{2}\left(\mathbf{z,w}\right)}{\partial
w_{2}}|_{\mathbf{z,w}=1}&=&\hat{R}_{2}^{(1)}\left(1\right)\hat{P}_{2}^{(1)}\left(1\right)
=\hat{r}_{2}\hat{\mu}_{2}.
\end{eqnarray*}


%_________________________________________________________________________________________________
\subsection{Usuarios presentes en la cola}
%_________________________________________________________________________________________________

Hagamos lo correspondiente con las siguientes
expresiones obtenidas en la secci\'on anterior:
Recordemos que

\begin{eqnarray*}
F_{1}\left(\theta_{1}\left(\tilde{P}_{2}\left(z_{2}\right)\hat{P}_{1}\left(w_{1}\right)
\hat{P}_{2}\left(w_{2}\right)\right),z_{2},w_{1},w_{2}\right)&=&
F_{1}\left(\theta_{1}\left(\tilde{P}_{2}\left(z_{2}\right)\hat{P}_{1}\left(w_{1}\right)\hat{P}_{2}\left(w_{2}\right)\right),z{2}\right)\\
&&\hat{F}_{1}\left(w_{1},w_{2};\tau_{1}\right)
\end{eqnarray*}

entonces

\begin{eqnarray*}
\frac{\partial F_{1}\left(\theta_{1}\left(\tilde{P}_{2}\left(z_{2}\right)\hat{P}_{1}\left(w_{1}\right)\hat{P}_{2}\left(w_{2}\right)\right),z_{2},w_{1},w_{2}\right)}{\partial z_{1}}|_{\mathbf{z},\mathbf{w}=1}&=&0\\
\frac{\partial
F_{1}\left(\theta_{1}\left(\tilde{P}_{2}\left(z_{2}\right)\hat{P}_{1}\left(w_{1}\right)\hat{P}_{2}\left(w_{2}\right)\right),z_{2},w_{1},w_{2}\right)}{\partial
z_{2}}|_{\mathbf{z},\mathbf{w}=1}&=&\frac{\partial F_{1}}{\partial
z_{1}}\cdot\frac{\partial \theta_{1}}{\partial
\tilde{P}_{2}}\cdot\frac{\partial \tilde{P}_{2}}{\partial
z_{2}}+\frac{\partial F_{1}}{\partial z_{2}}
\\
\frac{\partial
F_{1}\left(\theta_{1}\left(\tilde{P}_{2}\left(z_{2}\right)\hat{P}_{1}\left(w_{1}\right)\hat{P}_{2}\left(w_{2}\right)\right),z_{2},w_{1},w_{2}\right)}{\partial
w_{1}}|_{\mathbf{z},\mathbf{w}=1}&=&\frac{\partial F_{1}}{\partial
z_{1}}\cdot\frac{\partial
\theta_{1}}{\partial\hat{P}_{1}}\cdot\frac{\partial\hat{P}_{1}}{\partial
w_{1}}+\frac{\partial\hat{F}_{1}}{\partial w_{1}}
\\
\frac{\partial
F_{1}\left(\theta_{1}\left(\tilde{P}_{2}\left(z_{2}\right)\hat{P}_{1}\left(w_{1}\right)\hat{P}_{2}\left(w_{2}\right)\right),z_{2},w_{1},w_{2}\right)}{\partial
w_{2}}|_{\mathbf{z},\mathbf{w}=1}&=&\frac{\partial F_{1}}{\partial
z_{1}}\cdot\frac{\partial\theta_{1}}{\partial\hat{P}_{2}}\cdot\frac{\partial\hat{P}_{2}}{\partial
w_{2}}+\frac{\partial \hat{F}_{1}}{\partial w_{2}}
\\
\end{eqnarray*}

para $\tau_{2}$:

\begin{eqnarray*}
F_{2}\left(z_{1},\tilde{\theta}_{2}\left(P_{1}\left(z_{1}\right)\hat{P}_{1}\left(w_{1}\right)\hat{P}_{2}\left(w_{2}\right)\right),
w_{1},w_{2}\right)&=&F_{2}\left(z_{1},\tilde{\theta}_{2}\left(P_{1}\left(z_{1}\right)\hat{P}_{1}\left(w_{1}\right)\hat{P}_{2}\left(w_{2}\right)\right)\right)\\
&&\hat{F}_{2}\left(w_{1},w_{2};\tau_{2}\right)
\end{eqnarray*}
al igual que antes

\begin{eqnarray*}
\frac{\partial
F_{2}\left(z_{1},\tilde{\theta}_{2}\left(P_{1}\left(z_{1}\right)\hat{P}_{1}\left(w_{1}\right)\hat{P}_{2}\left(w_{2}\right)\right),w_{1},w_{2}\right)}{\partial
z_{1}}|_{\mathbf{z},\mathbf{w}=1}&=&\frac{\partial F_{2}}{\partial
z_{2}}\cdot\frac{\partial\tilde{\theta}_{2}}{\partial
P_{1}}\cdot\frac{\partial P_{1}}{\partial z_{2}}+\frac{\partial
F_{2}}{\partial z_{1}}
\\
\frac{\partial F_{2}\left(z_{1},\tilde{\theta}_{2}\left(P_{1}\left(z_{1}\right)\hat{P}_{1}\left(w_{1}\right)\hat{P}_{2}\left(w_{2}\right)\right),w_{1},w_{2}\right)}{\partial z_{2}}|_{\mathbf{z},\mathbf{w}=1}&=&0\\
\frac{\partial
F_{2}\left(z_{1},\tilde{\theta}_{2}\left(P_{1}\left(z_{1}\right)\hat{P}_{1}\left(w_{1}\right)\hat{P}_{2}\left(w_{2}\right)\right),w_{1},w_{2}\right)}{\partial
w_{1}}|_{\mathbf{z},\mathbf{w}=1}&=&\frac{\partial F_{2}}{\partial
z_{2}}\cdot\frac{\partial \tilde{\theta}_{2}}{\partial
\hat{P}_{1}}\cdot\frac{\partial \hat{P}_{1}}{\partial
w_{1}}+\frac{\partial \hat{F}_{2}}{\partial w_{1}}
\\
\frac{\partial
F_{2}\left(z_{1},\tilde{\theta}_{2}\left(P_{1}\left(z_{1}\right)\hat{P}_{1}\left(w_{1}\right)\hat{P}_{2}\left(w_{2}\right)\right),w_{1},w_{2}\right)}{\partial
w_{2}}|_{\mathbf{z},\mathbf{w}=1}&=&\frac{\partial F_{2}}{\partial
z_{2}}\cdot\frac{\partial
\tilde{\theta}_{2}}{\partial\hat{P}_{2}}\cdot\frac{\partial\hat{P}_{2}}{\partial
w_{2}}+\frac{\partial\hat{F}_{2}}{\partial w_{2}}
\\
\end{eqnarray*}


Ahora para el segundo sistema

\begin{eqnarray*}\hat{F}_{1}\left(z_{1},z_{2},\hat{\theta}_{1}\left(P_{1}\left(z_{1}\right)\tilde{P}_{2}\left(z_{2}\right)\hat{P}_{2}\left(w_{2}\right)\right),
w_{2}\right)&=&F_{1}\left(z_{1},z_{2};\zeta_{1}\right)\\
&&\hat{F}_{1}\left(\hat{\theta}_{1}\left(P_{1}\left(z_{1}\right)\tilde{P}_{2}\left(z_{2}\right)
\hat{P}_{2}\left(w_{2}\right)\right),w_{2}\right)
\end{eqnarray*}
entonces


\begin{eqnarray*}
\frac{\partial
\hat{F}_{1}\left(z_{1},z_{2},\hat{\theta}_{1}\left(P_{1}\left(z_{1}\right)\tilde{P}_{2}\left(z_{2}\right)\hat{P}_{2}\left(w_{2}\right)\right),w_{2}\right)}{\partial
z_{1}}|_{\mathbf{z},\mathbf{w}=1}&=&\frac{\partial \hat{F}_{1}
}{\partial w_{1}}\cdot\frac{\partial\hat{\theta}_{1}}{\partial
P_{1}}\cdot\frac{\partial P_{1}}{\partial z_{1}}+\frac{\partial
F_{1}}{\partial z_{1}}
\\
\frac{\partial
\hat{F}_{1}\left(z_{1},z_{2},\hat{\theta}_{1}\left(P_{1}\left(z_{1}\right)\tilde{P}_{2}\left(z_{2}\right)\hat{P}_{2}\left(w_{2}\right)\right),w_{2}\right)}{\partial
z_{2}}|_{\mathbf{z},\mathbf{w}=1}&=&\frac{\partial
\hat{F}_{1}}{\partial
w_{1}}\cdot\frac{\partial\hat{\theta}_{1}}{\partial\tilde{P}_{2}}\cdot\frac{\partial\tilde{P}_{2}}{\partial
z_{2}}+\frac{\partial F_{1}}{\partial z_{2}}
\\
\frac{\partial \hat{F}_{1}\left(z_{1},z_{2},\hat{\theta}_{1}\left(P_{1}\left(z_{1}\right)\tilde{P}_{2}\left(z_{2}\right)\hat{P}_{2}\left(w_{2}\right)\right),w_{2}\right)}{\partial w_{1}}|_{\mathbf{z},\mathbf{w}=1}&=&0\\
\frac{\partial \hat{F}_{1}\left(z_{1},z_{2},\hat{\theta}_{1}\left(P_{1}\left(z_{1}\right)\tilde{P}_{2}\left(z_{2}\right)\hat{P}_{2}\left(w_{2}\right)\right),w_{2}\right)}{\partial w_{2}}|_{\mathbf{z},\mathbf{w}=1}&=&\frac{\partial\hat{F}_{1}}{\partial w_{1}}\cdot\frac{\partial\hat{\theta}_{1}}{\partial\hat{P}_{2}}\cdot\frac{\partial\hat{P}_{2}}{\partial w_{2}}+\frac{\partial \hat{F}_{1}}{\partial w_{2}}\\
\end{eqnarray*}



Finalmente para $\zeta_{2}$

\begin{eqnarray*}\hat{F}_{2}\left(z_{1},z_{2},w_{1},\hat{\theta}_{2}\left(P_{1}\left(z_{1}\right)\tilde{P}_{2}\left(z_{2}\right)\hat{P}_{1}\left(w_{1}\right)\right)\right)&=&F_{2}\left(z_{1},z_{2};\zeta_{2}\right)\\
&&\hat{F}_{2}\left(w_{1},\hat{\theta}_{2}\left(P_{1}\left(z_{1}\right)\tilde{P}_{2}\left(z_{2}\right)\hat{P}_{1}\left(w_{1}\right)\right)\right]
\end{eqnarray*}
por tanto:

\begin{eqnarray*}
\frac{\partial
\hat{F}_{2}\left(z_{1},z_{2},w_{1},\hat{\theta}_{2}\left(P_{1}\left(z_{1}\right)\tilde{P}_{2}\left(z_{2}\right)\hat{P}_{1}\left(w_{1}\right)\right)\right)}{\partial
z_{1}}|_{\mathbf{z},\mathbf{w}=1}&=&\frac{\partial\hat{F}_{2}}{\partial
w_{2}}\cdot\frac{\partial\hat{\theta}_{2}}{\partial
P_{1}}\cdot\frac{\partial P_{1}}{\partial z_{1}}+\frac{\partial
F_{2}}{\partial z_{1}}
\\
\frac{\partial \hat{F}_{2}\left(z_{1},z_{2},w_{1},\hat{\theta}_{2}\left(P_{1}\left(z_{1}\right)\tilde{P}_{2}\left(z_{2}\right)\hat{P}_{1}\left(w_{1}\right)\right)\right)}{\partial z_{2}}|_{\mathbf{z},\mathbf{w}=1}&=&\frac{\partial\hat{F}_{2}}{\partial w_{2}}\cdot\frac{\partial\hat{\theta}_{2}}{\partial \tilde{P}_{2}}\cdot\frac{\partial \tilde{P}_{2}}{\partial z_{2}}+\frac{\partial F_{2}}{\partial z_{2}}\\
\frac{\partial \hat{F}_{2}\left(z_{1},z_{2},w_{1},\hat{\theta}_{2}\left(P_{1}\left(z_{1}\right)\tilde{P}_{2}\left(z_{2}\right)\hat{P}_{1}\left(w_{1}\right)\right)\right)}{\partial w_{1}}|_{\mathbf{z},\mathbf{w}=1}&=&\frac{\partial\hat{F}_{2}}{\partial w_{2}}\cdot\frac{\partial\hat{\theta}_{2}}{\partial \hat{P}_{1}}\cdot\frac{\partial \hat{P}_{1}}{\partial w_{1}}+\frac{\partial \hat{F}_{2}}{\partial w_{1}}\\
\frac{\partial \hat{F}_{2}\left(z_{1},z_{2},w_{1},\hat{\theta}_{2}\left(P_{1}\left(z_{1}\right)\tilde{P}_{2}\left(z_{2}\right)\hat{P}_{1}\left(w_{1}\right)\right)\right)}{\partial w_{2}}|_{\mathbf{z},\mathbf{w}=1}&=&0\\
\end{eqnarray*}

%_________________________________________________________________________________________________
\subsection{Ecuaciones Recursivas}
%_________________________________________________________________________________________________

Entonces, de todo lo desarrollado hasta ahora se tienen las siguientes ecuaciones:

\begin{eqnarray*}
\frac{\partial F_{2}\left(\mathbf{z},\mathbf{w}\right)}{\partial z_{1}}|_{\mathbf{z},\mathbf{w}=1}&=&\frac{\partial R_{1}}{\partial z_{1}}+\frac{\partial F_{1}}{\partial z_{1}}=r_{1}\mu_{1}\\
\frac{\partial F_{2}\left(\mathbf{z},\mathbf{w}\right)}{\partial z_{2}}|_{\mathbf{z},\mathbf{w}=1}&=&\frac{\partial R_{1}}{\partial z_{2}}+\frac{\partial F_{1}}{\partial z_{2}}=r_{1}\tilde{\mu}_{2}+f_{1}\left(1\right)\left(\frac{1}{1-\mu_{1}}\right)\tilde{\mu}_{2}+f_{1}\left(2\right)\\
\frac{\partial F_{2}\left(\mathbf{z},\mathbf{w}\right)}{\partial w_{1}}|_{\mathbf{z},\mathbf{w}=1}&=&\frac{\partial R_{1}}{\partial w_{1}}+\frac{\partial F_{1}}{\partial w_{1}}=r_{1}\hat{\mu}_{1}+f_{1}\left(1\right)\left(\frac{1}{1-\mu_{1}}\right)\hat{\mu}_{1}+\hat{F}_{1,1}^{(1)}\left(1\right)\\
\frac{\partial F_{2}\left(\mathbf{z},\mathbf{w}\right)}{\partial
w_{2}}|_{\mathbf{z},\mathbf{w}=1}&=&\frac{\partial R_{1}}{\partial
w_{2}}+\frac{\partial F_{1}}{\partial
w_{2}}=r_{1}\hat{\mu}_{2}+f_{1}\left(1\right)\left(\frac{1}{1-\mu_{1}}\right)\hat{\mu}_{2}+\hat{F}_{2,1}^{(1)}\left(1\right)
\end{eqnarray*}



\begin{eqnarray*}
\frac{\partial F_{1}\left(\mathbf{z},\mathbf{w}\right)}{\partial z_{1}}|_{\mathbf{z},\mathbf{w}=1}&=&\frac{\partial R_{2}}{\partial z_{1}}+\frac{\partial F_{2}}{\partial z_{1}}=r_{2}\mu_{1}+f_{2}\left(2\right)\left(\frac{1}{1-\tilde{\mu}_{2}}\right)\mu_{1}+f_{2}\left(1\right)\\
\frac{\partial F_{1}\left(\mathbf{z},\mathbf{w}\right)}{\partial z_{2}}|_{\mathbf{z},\mathbf{w}=1}&=&\frac{\partial R_{2}}{\partial z_{2}}+\frac{\partial F_{2}}{\partial z_{2}}=r_{2}\tilde{\mu}_{2}\\
\frac{\partial F_{1}\left(\mathbf{z},\mathbf{w}\right)}{\partial w_{1}}|_{\mathbf{z},\mathbf{w}=1}&=&\frac{\partial R_{2}}{\partial w_{1}}+\frac{\partial F_{2}}{\partial w_{1}}=r_{2}\hat{\mu}_{1}+f_{2}\left(2\right)\left(\frac{1}{1-\tilde{\mu}_{2}}\right)\hat{\mu}_{1}+\hat{F}_{2,1}^{(1)}\left(1\right)\\
\frac{\partial F_{1}\left(\mathbf{z},\mathbf{w}\right)}{\partial
w_{2}}|_{\mathbf{z},\mathbf{w}=1}&=&\frac{\partial R_{2}}{\partial
w_{2}}+\frac{\partial F_{2}}{\partial
w_{2}}=r_{2}\hat{\mu}_{2}+f_{2}\left(2\right)\left(\frac{1}{1-\tilde{\mu}_{2}}\right)\hat{\mu}_{2}+\hat{F}_{2,2}^{(1)}\left(1\right)
\end{eqnarray*}




\begin{eqnarray*}
\frac{\partial \hat{F}_{2}\left(\mathbf{z},\mathbf{w}\right)}{\partial z_{1}}|_{\mathbf{z},\mathbf{w}=1}&=&\frac{\partial \hat{R}_{1}}{\partial z_{1}}+\frac{\partial \hat{F}_{1}}{\partial z_{1}}=\hat{r}_{1}\mu_{1}+\hat{f}_{1}\left(1\right)\left(\frac{1}{1-\hat{\mu}_{1}}\right)\mu_{1}+F_{1,1}^{(1)}\left(1\right)\\
\frac{\partial \hat{F}_{2}\left(\mathbf{z},\mathbf{w}\right)}{\partial z_{2}}|_{\mathbf{z},\mathbf{w}=1}&=&\frac{\partial \hat{R}_{1}}{\partial z_{2}}+\frac{\partial \hat{F}_{1}}{\partial z_{2}}=\hat{r}_{1}\mu_{2}+\hat{f}_{1}\left(1\right)\left(\frac{1}{1-\hat{\mu}_{1}}\right)\tilde{\mu}_{2}+F_{2,1}^{(1)}\left(1\right)\\
\frac{\partial \hat{F}_{2}\left(\mathbf{z},\mathbf{w}\right)}{\partial w_{1}}|_{\mathbf{z},\mathbf{w}=1}&=&\frac{\partial \hat{R}_{1}}{\partial w_{1}}+\frac{\partial \hat{F}_{1}}{\partial w_{1}}=\hat{r}_{1}\hat{\mu}_{1}\\
\frac{\partial \hat{F}_{2}\left(\mathbf{z},\mathbf{w}\right)}{\partial w_{2}}|_{\mathbf{z},\mathbf{w}=1}&=&\frac{\partial \hat{R}_{1}}{\partial w_{2}}+\frac{\partial \hat{F}_{1}}{\partial w_{2}}=\hat{r}_{1}\hat{\mu}_{2}+\hat{f}_{1}\left(1\right)\left(\frac{1}{1-\hat{\mu}_{1}}\right)\hat{\mu}_{2}+\hat{f}_{1}\left(2\right)
\end{eqnarray*}



\begin{eqnarray*}
\frac{\partial \hat{F}_{1}\left(\mathbf{z},\mathbf{w}\right)}{\partial z_{1}}|_{\mathbf{z},\mathbf{w}=1}&=&\frac{\partial \hat{R}_{2}}{\partial z_{1}}+\frac{\partial \hat{F}_{2}}{\partial z_{1}}=\hat{r}_{2}\mu_{1}+\hat{f}_{2}\left(1\right)\left(\frac{1}{1-\hat{\mu}_{2}}\right)\mu_{1}+F_{1,2}^{(1)}\left(1\right)\\
\frac{\partial \hat{F}_{1}\left(\mathbf{z},\mathbf{w}\right)}{\partial z_{2}}|_{\mathbf{z},\mathbf{w}=1}&=&\frac{\partial \hat{R}_{2}}{\partial z_{2}}+\frac{\partial \hat{F}_{2}}{\partial z_{2}}=\hat{r}_{2}\tilde{\mu}_{2}+\hat{f}_{2}\left(2\right)\left(\frac{1}{1-\hat{\mu}_{2}}\right)\tilde{\mu}_{2}+F_{2,2}^{(1)}\left(1\right)\\
\frac{\partial \hat{F}_{1}\left(\mathbf{z},\mathbf{w}\right)}{\partial w_{1}}|_{\mathbf{z},\mathbf{w}=1}&=&\frac{\partial \hat{R}_{2}}{\partial w_{1}}+\frac{\partial \hat{F}_{2}}{\partial w_{1}}=\hat{r}_{2}\hat{\mu}_{1}+\hat{f}_{2}\left(2\right)\left(\frac{1}{1-\hat{\mu}_{2}}\right)\hat{\mu}_{1}+\hat{f}_{2}\left(1\right)\\
\frac{\partial
\hat{F}_{1}\left(\mathbf{z},\mathbf{w}\right)}{\partial
w_{2}}|_{\mathbf{z},\mathbf{w}=1}&=&\frac{\partial
\hat{R}_{2}}{\partial w_{2}}+\frac{\partial \hat{F}_{2}}{\partial
w_{2}}=\hat{r}_{2}\hat{\mu}_{2}
\end{eqnarray*}

Es decir, se tienen las siguientes ecuaciones:




\begin{eqnarray*}
f_{2}\left(1\right)&=&r_{1}\mu_{1}\\
f_{1}\left(2\right)&=&r_{2}\tilde{\mu}_{2}\\
f_{2}\left(2\right)&=&r_{1}\tilde{\mu}_{2}+\tilde{\mu}_{2}\left(\frac{f_{1}\left(1\right)}{1-\mu_{1}}\right)+f_{1}\left(2\right)=\left(r_{1}+\frac{f_{1}\left(1\right)}{1-\mu_{1}}\right)\tilde{\mu}_{2}+r_{2}\tilde{\mu}_{2}\\
&=&\left(r_{1}+r_{2}+\frac{f_{1}\left(1\right)}{1-\mu_{1}}\right)\tilde{\mu}_{2}=\left(r+\frac{f_{1}\left(1\right)}{1-\mu_{1}}\right)\tilde{\mu}_{2}\\
f_{2}\left(3\right)&=&r_{1}\hat{\mu}_{1}+\hat{\mu}_{1}\left(\frac{f_{1}\left(1\right)}{1-\mu_{1}}\right)+\hat{F}_{1,1}^{(1)}\left(1\right)=\hat{\mu}_{1}\left(r_{1}+\frac{f_{1}\left(1\right)}{1-\mu_{1}}\right)+\frac{\hat{\mu}_{1}}{\mu_{1}}\\
f_{2}\left(4\right)&=&r_{1}\hat{\mu}_{2}+\hat{\mu}_{2}\left(\frac{f_{1}\left(1\right)}{1-\mu_{1}}\right)+\hat{F}_{2,1}^{(1)}\left(1\right)=\hat{\mu}_{2}\left(r_{1}+\frac{f_{1}\left(1\right)}{1-\mu_{1}}\right)+\frac{\hat{\mu}_{2}}{\mu_{1}}\\
\end{eqnarray*}


\begin{eqnarray*}
f_{1}\left(1\right)&=&r_{2}\mu_{1}+\mu_{1}\left(\frac{f_{2}\left(2\right)}{1-\tilde{\mu}_{2}}\right)+r_{1}\mu_{1}=\mu_{1}\left(r_{1}+r_{2}+\frac{f_{2}\left(2\right)}{1-\tilde{\mu}_{2}}\right)\\
&=&\mu_{1}\left(r+\frac{f_{2}\left(2\right)}{1-\tilde{\mu}_{2}}\right)\\
f_{1}\left(3\right)&=&r_{2}\hat{\mu}_{1}+\hat{\mu}_{1}\left(\frac{f_{2}\left(2\right)}{1-\tilde{\mu}_{2}}\right)+\hat{F}^{(1)}_{1,2}\left(1\right)=\hat{\mu}_{1}\left(r_{2}+\frac{f_{2}\left(2\right)}{1-\tilde{\mu}_{2}}\right)+\frac{\hat{\mu}_{1}}{\mu_{2}}\\
f_{1}\left(4\right)&=&r_{2}\hat{\mu}_{2}+\hat{\mu}_{2}\left(\frac{f_{2}\left(2\right)}{1-\tilde{\mu}_{2}}\right)+\hat{F}_{2,2}^{(1)}\left(1\right)=\hat{\mu}_{2}\left(r_{2}+\frac{f_{2}\left(2\right)}{1-\tilde{\mu}_{2}}\right)+\frac{\hat{\mu}_{2}}{\mu_{2}}\\
\hat{f}_{1}\left(4\right)&=&\hat{r}_{2}\hat{\mu}_{2}\\
\hat{f}_{2}\left(3\right)&=&\hat{r}_{1}\hat{\mu}_{1}\\
\hat{f}_{1}\left(1\right)&=&\hat{r}_{2}\mu_{1}+\mu_{1}\left(\frac{\hat{f}_{2}\left(4\right)}{1-\hat{\mu}_{2}}\right)+F_{1,2}^{(1)}\left(1\right)=\left(\hat{r}_{2}+\frac{\hat{f}_{2}\left(4\right)}{1-\hat{\mu}_{2}}\right)\mu_{1}+\frac{\mu_{1}}{\hat{\mu}_{2}}
\end{eqnarray*}

\begin{eqnarray*}
\hat{f}_{1}\left(2\right)&=&\hat{r}_{2}\tilde{\mu}_{2}+\tilde{\mu}_{2}\left(\frac{\hat{f}_{2}\left(4\right)}{1-\hat{\mu}_{2}}\right)+F_{2,2}^{(1)}\left(1\right)=
\left(\hat{r}_{2}+\frac{\hat{f}_{2}\left(4\right)}{1-\hat{\mu}_{2}}\right)\tilde{\mu}_{2}+\frac{\mu_{2}}{\hat{\mu}_{2}}\\
\hat{f}_{1}\left(3\right)&=&\hat{r}_{2}\hat{\mu}_{1}+\hat{\mu}_{1}\left(\frac{\hat{f}_{2}\left(4\right)}{1-\hat{\mu}_{2}}\right)+\hat{f}_{2}\left(3\right)=\left(\hat{r}_{2}+\frac{\hat{f}_{2}\left(4\right)}{1-\hat{\mu}_{2}}\right)\hat{\mu}_{1}+\hat{r}_{1}\hat{\mu}_{1}\\
&=&\left(\hat{r}_{1}+\hat{r}_{2}+\frac{\hat{f}_{2}\left(4\right)}{1-\hat{\mu}_{2}}\right)\hat{\mu}_{1}=\left(\hat{r}+\frac{\hat{f}_{2}\left(4\right)}{1-\hat{\mu}_{2}}\right)\hat{\mu}_{1}\\
\hat{f}_{2}\left(1\right)&=&\hat{r}_{1}\mu_{1}+\mu_{1}\left(\frac{\hat{f}_{1}\left(3\right)}{1-\hat{\mu}_{1}}\right)+F_{1,1}^{(1)}\left(1\right)=\left(\hat{r}_{1}+\frac{\hat{f}_{1}\left(3\right)}{1-\hat{\mu}_{1}}\right)\mu_{1}+\frac{\mu_{1}}{\hat{\mu}_{1}}\\
\hat{f}_{2}\left(2\right)&=&\hat{r}_{1}\tilde{\mu}_{2}+\tilde{\mu}_{2}\left(\frac{\hat{f}_{1}\left(3\right)}{1-\hat{\mu}_{1}}\right)+F_{2,1}^{(1)}\left(1\right)=\left(\hat{r}_{1}+\frac{\hat{f}_{1}\left(3\right)}{1-\hat{\mu}_{1}}\right)\tilde{\mu}_{2}+\frac{\mu_{2}}{\hat{\mu}_{1}}\\
\hat{f}_{2}\left(4\right)&=&\hat{r}_{1}\hat{\mu}_{2}+\hat{\mu}_{2}\left(\frac{\hat{f}_{1}\left(3\right)}{1-\hat{\mu}_{1}}\right)+\hat{f}_{1}\left(4\right)=\hat{r}_{1}\hat{\mu}_{2}+\hat{r}_{2}\hat{\mu}_{2}+\hat{\mu}_{2}\left(\frac{\hat{f}_{1}\left(3\right)}{1-\hat{\mu}_{1}}\right)\\
&=&\left(\hat{r}+\frac{\hat{f}_{1}\left(3\right)}{1-\hat{\mu}_{1}}\right)\hat{\mu}_{2}
\end{eqnarray*}


%_______________________________________________________________________________________________
\subsection{Soluci\'on del Sistema de Ecuaciones Lineales}
%_________________________________________________________________________________________________

A saber, se puede demostrar que la soluci\'on del sistema de
ecuaciones est\'a dado por las siguientes expresiones, donde

\begin{eqnarray*}
\mu=\mu_{1}+\tilde{\mu}_{2}\textrm{ , }\hat{\mu}=\hat{\mu}_{1}+\hat{\mu}_{2}\textrm{ , }
r=r_{1}+r_{2}\textrm{ y }\hat{r}=\hat{r}_{1}+\hat{r}_{2}
\end{eqnarray*}
entonces

\begin{eqnarray*}
f_{1}\left(1\right)&=&r\frac{\mu_{1}\left(1-\mu_{1}\right)}{1-\mu}\\
f_{2}\left(2\right)&=&r\frac{\tilde{\mu}_{2}\left(1-\tilde{\mu}_{2}\right)}{1-\mu}
\end{eqnarray*}

\begin{eqnarray*}
f_{1}\left(3\right)&=&\hat{\mu}_{1}\left(\frac{r_{2}\mu_{2}+1}{\mu_{2}}+r\frac{\tilde{\mu}_{2}}{1-\mu}\right)\\
f_{1}\left(4\right)&=&\hat{\mu}_{2}\left(\frac{r_{2}\mu_{2}+1}{\mu_{2}}+r\frac{\tilde{\mu}_{2}}{1-\mu}\right)\\
\end{eqnarray*}



\begin{eqnarray*}
f_{2}\left(3\right)&=&\hat{\mu}_{1}\left(\frac{r_{1}\mu_{1}+1}{\mu_{1}}+r\frac{\mu_{1}}{1-\mu}\right)\\
f_{2}\left(4\right)&=&\hat{\mu}_{2}\left(\frac{r_{1}\mu_{1}+1}{\mu_{1}}+r\frac{\mu_{1}}{1-\mu}\right)\\
\end{eqnarray*}
\begin{eqnarray*}
\hat{f}_{2}\left(4\right)&=&\hat{r}\frac{\hat{\mu}_{2}\left(1-\hat{\mu}_{2}\right)}{1-\hat{\mu}}\\
\hat{f}_{1}\left(3\right)&=&\hat{r}\frac{\hat{\mu}_{1}\left(1-\hat{\mu}_{1}\right)}{1-\hat{\mu}}
\end{eqnarray*}

\begin{eqnarray*}
\hat{f}_{1}\left(1\right)&=&\mu_{1}\left(\frac{\hat{r}_{2}\hat{\mu}_{2}+1}{\hat{\mu}_{2}}+\hat{r}\frac{\hat{\mu}_{2}}{1-\hat{\mu}}\right)\\
\hat{f}_{1}\left(2\right)&=&\tilde{\mu}_{2}\left(\hat{r}_{2}+\hat{r}\frac{\hat{\mu}_{2}}{1-\hat{\mu}}\right)+\frac{\mu_{2}}{\hat{\mu}_{2}}\\\\
\hat{f}_{2}\left(1\right)&=&\mu_{1}\left(\frac{\hat{r}_{1}\hat{\mu}_{1}+1}{\hat{\mu}_{1}}+\hat{r}\frac{\hat{\mu}_{1}}{1-\hat{\mu}}\right)\\
\hat{f}_{2}\left(2\right)&=&\tilde{\mu}_{2}\left(\hat{r}_{1}+\hat{r}\frac{\hat{\mu}_{1}}{1-\hat{\mu}}\right)+\frac{\hat{\mu_{2}}}{\hat{\mu}_{1}}\\
\end{eqnarray*}

A saber

\begin{eqnarray*}
f_{1}\left(3\right)&=&\hat{\mu}_{1}\left(r_{2}+\frac{f_{2}\left(2\right)}{1-\tilde{\mu}_{2}}\right)+\frac{\hat{\mu}_{1}}{\mu_{2}}=\hat{\mu}_{1}\left(r_{2}+\frac{r\frac{\tilde{\mu}_{2}\left(1-\tilde{\mu}_{2}\right)}{1-\mu}}{1-\tilde{\mu}_{2}}\right)+\frac{\hat{\mu}_{1}}{\mu_{2}}\\
&=&\hat{\mu}_{1}\left(r_{2}+\frac{r\tilde{\mu}_{2}}{1-\mu}\right)+\frac{\hat{\mu}_{1}}{\mu_{2}}=
\hat{\mu}_{1}\left(r_{2}+\frac{r\tilde{\mu}_{2}}{1-\mu}+\frac{1}{\mu_{2}}\right)\\
&=&\hat{\mu}_{1}\left(\frac{r_{2}\mu_{2}+1}{\mu_{2}}+\frac{r\tilde{\mu}_{2}}{1-\mu}\right)
\end{eqnarray*}

\begin{eqnarray*}
f_{1}\left(4\right)&=&\hat{\mu}_{2}\left(r_{2}+\frac{f_{2}\left(2\right)}{1-\tilde{\mu}_{2}}\right)+\frac{\hat{\mu}_{2}}{\mu_{2}}=\hat{\mu}_{2}\left(r_{2}+\frac{r\frac{\tilde{\mu}_{2}\left(1-\tilde{\mu}_{2}\right)}{1-\mu}}{1-\tilde{\mu}_{2}}\right)+\frac{\hat{\mu}_{2}}{\mu_{2}}\\
&=&\hat{\mu}_{2}\left(r_{2}+\frac{r\tilde{\mu}_{2}}{1-\mu}\right)+\frac{\hat{\mu}_{1}}{\mu_{2}}=
\hat{\mu}_{2}\left(r_{2}+\frac{r\tilde{\mu}_{2}}{1-\mu}+\frac{1}{\mu_{2}}\right)\\
&=&\hat{\mu}_{2}\left(\frac{r_{2}\mu_{2}+1}{\mu_{2}}+\frac{r\tilde{\mu}_{2}}{1-\mu}\right)
\end{eqnarray*}

\begin{eqnarray*}
f_{2}\left(3\right)&=&\hat{\mu}_{1}\left(r_{1}+\frac{f_{1}\left(1\right)}{1-\mu_{1}}\right)+\frac{\hat{\mu}_{1}}{\mu_{1}}=\hat{\mu}_{1}\left(r_{1}+\frac{r\frac{\mu_{1}\left(1-\mu_{1}\right)}{1-\mu}}{1-\mu_{1}}\right)+\frac{\hat{\mu}_{1}}{\mu_{1}}\\
&=&\hat{\mu}_{1}\left(r_{1}+\frac{r\mu_{1}}{1-\mu}\right)+\frac{\hat{\mu}_{1}}{\mu_{1}}=
\hat{\mu}_{1}\left(r_{1}+\frac{r\mu_{1}}{1-\mu}+\frac{1}{\mu_{1}}\right)\\
&=&\hat{\mu}_{1}\left(\frac{r_{1}\mu_{1}+1}{\mu_{1}}+\frac{r\mu_{1}}{1-\mu}\right)
\end{eqnarray*}

\begin{eqnarray*}
f_{2}\left(4\right)&=&\hat{\mu}_{2}\left(r_{1}+\frac{f_{1}\left(1\right)}{1-\mu_{1}}\right)+\frac{\hat{\mu}_{2}}{\mu_{1}}=\hat{\mu}_{2}\left(r_{1}+\frac{r\frac{\mu_{1}\left(1-\mu_{1}\right)}{1-\mu}}{1-\mu_{1}}\right)+\frac{\hat{\mu}_{1}}{\mu_{1}}\\
&=&\hat{\mu}_{2}\left(r_{1}+\frac{r\mu_{1}}{1-\mu}\right)+\frac{\hat{\mu}_{1}}{\mu_{1}}=
\hat{\mu}_{2}\left(r_{1}+\frac{r\mu_{1}}{1-\mu}+\frac{1}{\mu_{1}}\right)\\
&=&\hat{\mu}_{2}\left(\frac{r_{1}\mu_{1}+1}{\mu_{1}}+\frac{r\mu_{1}}{1-\mu}\right)\end{eqnarray*}

A saber

\begin{eqnarray*}
\hat{f}_{1}\left(1\right)&=&\mu_{1}\left(\hat{r}_{2}+\frac{\hat{f}_{2}\left(4\right)}{1-\tilde{\mu}_{2}}\right)+\frac{\mu_{1}}{\hat{\mu}_{2}}=\mu_{1}\left(\hat{r}_{2}+\frac{\hat{r}\frac{\hat{\mu}_{2}\left(1-\hat{\mu}_{2}\right)}{1-\hat{\mu}}}{1-\hat{\mu}_{2}}\right)+\frac{\mu_{1}}{\hat{\mu}_{2}}\\
&=&\mu_{1}\left(\hat{r}_{2}+\frac{\hat{r}\hat{\mu}_{2}}{1-\hat{\mu}}\right)+\frac{\mu_{1}}{\mu_{2}}
=\mu_{1}\left(\hat{r}_{2}+\frac{\hat{r}\mu_{2}}{1-\hat{\mu}}+\frac{1}{\hat{\mu}_{2}}\right)\\
&=&\mu_{1}\left(\frac{\hat{r}_{2}\hat{\mu}_{2}+1}{\hat{\mu}_{2}}+\frac{\hat{r}\hat{\mu}_{2}}{1-\hat{\mu}}\right)
\end{eqnarray*}

\begin{eqnarray*}
\hat{f}_{1}\left(2\right)&=&\tilde{\mu}_{2}\left(\hat{r}_{2}+\frac{\hat{f}_{2}\left(4\right)}{1-\tilde{\mu}_{2}}\right)+\frac{\mu_{2}}{\hat{\mu}_{2}}=\tilde{\mu}_{2}\left(\hat{r}_{2}+\frac{\hat{r}\frac{\hat{\mu}_{2}\left(1-\hat{\mu}_{2}\right)}{1-\hat{\mu}}}{1-\hat{\mu}_{2}}\right)+\frac{\mu_{2}}{\hat{\mu}_{2}}\\
&=&\tilde{\mu}_{2}\left(\hat{r}_{2}+\frac{\hat{r}\hat{\mu}_{2}}{1-\hat{\mu}}\right)+\frac{\mu_{2}}{\hat{\mu}_{2}}
\end{eqnarray*}

\begin{eqnarray*}
\hat{f}_{2}\left(1\right)&=&\mu_{1}\left(\hat{r}_{1}+\frac{\hat{f}_{1}\left(3\right)}{1-\hat{\mu}_{1}}\right)+\frac{\mu_{1}}{\hat{\mu}_{1}}=\mu_{1}\left(\hat{r}_{1}+\frac{\hat{r}\frac{\hat{\mu}_{1}\left(1-\hat{\mu}_{1}\right)}{1-\hat{\mu}}}{1-\hat{\mu}_{1}}\right)+\frac{\mu_{1}}{\hat{\mu}_{1}}\\
&=&\mu_{1}\left(\hat{r}_{1}+\frac{\hat{r}\hat{\mu}_{1}}{1-\hat{\mu}}\right)+\frac{\mu_{1}}{\hat{\mu}_{1}}
=\mu_{1}\left(\hat{r}_{1}+\frac{\hat{r}\hat{\mu}_{1}}{1-\hat{\mu}}+\frac{1}{\hat{\mu}_{1}}\right)\\
&=&\mu_{1}\left(\frac{\hat{r}_{1}\hat{\mu}_{1}+1}{\hat{\mu}_{1}}+\frac{\hat{r}\hat{\mu}_{1}}{1-\hat{\mu}}\right)
\end{eqnarray*}

\begin{eqnarray*}
\hat{f}_{2}\left(2\right)&=&\tilde{\mu}_{2}\left(\hat{r}_{1}+\frac{\hat{f}_{1}\left(3\right)}{1-\tilde{\mu}_{1}}\right)+\frac{\mu_{2}}{\hat{\mu}_{1}}=\tilde{\mu}_{2}\left(\hat{r}_{1}+\frac{\hat{r}\frac{\hat{\mu}_{1}
\left(1-\hat{\mu}_{1}\right)}{1-\hat{\mu}}}{1-\hat{\mu}_{1}}\right)+\frac{\mu_{2}}{\hat{\mu}_{1}}\\
&=&\tilde{\mu}_{2}\left(\hat{r}_{1}+\frac{\hat{r}\hat{\mu}_{1}}{1-\hat{\mu}}\right)+\frac{\mu_{2}}{\hat{\mu}_{1}}
\end{eqnarray*}
%___________________________________________________________________________________________
%
\section{Segundos Momentos}
%___________________________________________________________________________________________
%
%___________________________________________________________________________________________
%
%\subsection{Derivadas de Segundo Orden: Tiempos de Traslado del Servidor}
%___________________________________________________________________________________________



Para poder determinar los segundos momentos para los tiempos de traslado del servidor es necesario enunciar y demostrar la siguiente proposici\'on:

\begin{Prop}\label{Prop.Segundas.Derivadas}
Sea $f\left(g\left(x\right)h\left(y\right)\right)$ funci\'on continua tal que tiene derivadas parciales mixtas de segundo orden, entonces se tiene lo siguiente:

\begin{eqnarray*}
\frac{\partial}{\partial x}f\left(g\left(x\right)h\left(y\right)\right)=\frac{\partial f\left(g\left(x\right)h\left(y\right)\right)}{\partial x}\cdot \frac{\partial g\left(x\right)}{\partial x}\cdot h\left(y\right)
\end{eqnarray*}

por tanto

\begin{eqnarray}
\frac{\partial}{\partial x}\frac{\partial}{\partial x}f\left(g\left(x\right)h\left(y\right)\right)
&=&\frac{\partial^{2}}{\partial x}f\left(g\left(x\right)h\left(y\right)\right)\cdot \left(\frac{\partial g\left(x\right)}{\partial x}\right)^{2}\cdot h^{2}\left(y\right)+\frac{\partial}{\partial x}f\left(g\left(x\right)h\left(y\right)\right)\cdot \frac{\partial g^{2}\left(x\right)}{\partial x^{2}}\cdot h\left(y\right).
\end{eqnarray}

y

\begin{eqnarray*}
\frac{\partial}{\partial y}\frac{\partial}{\partial x}f\left(g\left(x\right)h\left(y\right)\right)&=&\frac{\partial g\left(x\right)}{\partial x}\cdot \frac{\partial h\left(y\right)}{\partial y}\left\{\frac{\partial^{2}}{\partial y\partial x}f\left(g\left(x\right)h\left(y\right)\right)\cdot g\left(x\right)\cdot h\left(y\right)+\frac{\partial}{\partial x}f\left(g\left(x\right)h\left(y\right)\right)\right\}
\end{eqnarray*}
\end{Prop}
\begin{proof}
\footnotesize{
\begin{eqnarray*}
\frac{\partial}{\partial x}\frac{\partial}{\partial x}f\left(g\left(x\right)h\left(y\right)\right)&=&\frac{\partial}{\partial x}\left\{\frac{\partial f\left(g\left(x\right)h\left(y\right)\right)}{\partial x}\cdot \frac{\partial g\left(x\right)}{\partial x}\cdot h\left(y\right)\right\}\\
&=&\frac{\partial}{\partial x}\left\{\frac{\partial}{\partial x}f\left(g\left(x\right)h\left(y\right)\right)\right\}\cdot \frac{\partial g\left(x\right)}{\partial x}\cdot h\left(y\right)+\frac{\partial}{\partial x}f\left(g\left(x\right)h\left(y\right)\right)\cdot \frac{\partial g^{2}\left(x\right)}{\partial x^{2}}\cdot h\left(y\right)\\
&=&\frac{\partial^{2}}{\partial x}f\left(g\left(x\right)h\left(y\right)\right)\cdot \frac{\partial g\left(x\right)}{\partial x}\cdot h\left(y\right)\cdot \frac{\partial g\left(x\right)}{\partial x}\cdot h\left(y\right)+\frac{\partial}{\partial x}f\left(g\left(x\right)h\left(y\right)\right)\cdot \frac{\partial g^{2}\left(x\right)}{\partial x^{2}}\cdot h\left(y\right)\\
&=&\frac{\partial^{2}}{\partial x}f\left(g\left(x\right)h\left(y\right)\right)\cdot \left(\frac{\partial g\left(x\right)}{\partial x}\right)^{2}\cdot h^{2}\left(y\right)+\frac{\partial}{\partial x}f\left(g\left(x\right)h\left(y\right)\right)\cdot \frac{\partial g^{2}\left(x\right)}{\partial x^{2}}\cdot h\left(y\right).
\end{eqnarray*}}


Por otra parte:
\footnotesize{
\begin{eqnarray*}
\frac{\partial}{\partial y}\frac{\partial}{\partial x}f\left(g\left(x\right)h\left(y\right)\right)&=&\frac{\partial}{\partial y}\left\{\frac{\partial f\left(g\left(x\right)h\left(y\right)\right)}{\partial x}\cdot \frac{\partial g\left(x\right)}{\partial x}\cdot h\left(y\right)\right\}\\
&=&\frac{\partial}{\partial y}\left\{\frac{\partial}{\partial x}f\left(g\left(x\right)h\left(y\right)\right)\right\}\cdot \frac{\partial g\left(x\right)}{\partial x}\cdot h\left(y\right)+\frac{\partial}{\partial x}f\left(g\left(x\right)h\left(y\right)\right)\cdot \frac{\partial g\left(x\right)}{\partial x}\cdot \frac{\partial h\left(y\right)}{y}\\
&=&\frac{\partial^{2}}{\partial y\partial x}f\left(g\left(x\right)h\left(y\right)\right)\cdot \frac{\partial h\left(y\right)}{\partial y}\cdot g\left(x\right)\cdot \frac{\partial g\left(x\right)}{\partial x}\cdot h\left(y\right)+\frac{\partial}{\partial x}f\left(g\left(x\right)h\left(y\right)\right)\cdot \frac{\partial g\left(x\right)}{\partial x}\cdot \frac{\partial h\left(y\right)}{\partial y}\\
&=&\frac{\partial g\left(x\right)}{\partial x}\cdot \frac{\partial h\left(y\right)}{\partial y}\left\{\frac{\partial^{2}}{\partial y\partial x}f\left(g\left(x\right)h\left(y\right)\right)\cdot g\left(x\right)\cdot h\left(y\right)+\frac{\partial}{\partial x}f\left(g\left(x\right)h\left(y\right)\right)\right\}
\end{eqnarray*}}
\end{proof}

Para la siguiente proposici\'on es necesario utilizar  el resultado (\ref{Prop.Segundas.Derivadas})

\begin{Prop}
Sea $R_{i}$ la Funci\'on Generadora de Probabilidades para el n\'umero de arribos a cada una de las colas de la Red de Sistemas de Visitas C\'iclicas definidas como en (\ref{Ec.R1}). Entonces las derivadas parciales est\'an dadas por las siguientes expresiones:


\begin{eqnarray*}
\frac{\partial^{2} R_{i}\left(P_{1}\left(z_{1}\right)\tilde{P}_{2}\left(z_{2}\right)\hat{P}_{1}\left(w_{1}\right)\hat{P}_{2}\left(w_{2}\right)\right)}{\partial z_{i}^{2}}&=&\left(\frac{\partial P_{i}\left(z_{i}\right)}{\partial z_{i}}\right)^{2}\cdot\frac{\partial^{2} R_{i}\left(P_{1}\left(z_{1}\right)\tilde{P}_{2}\left(z_{2}\right)\hat{P}_{1}\left(w_{1}\right)\hat{P}_{2}\left(w_{2}\right)\right)}{\partial^{2} z_{i}}\\
&+&\left(\frac{\partial P_{i}\left(z_{i}\right)}{\partial z_{i}}\right)^{2}\cdot
\frac{\partial R_{i}\left(P_{1}\left(z_{1}\right)\tilde{P}_{2}\left(z_{2}\right)\hat{P}_{1}\left(w_{1}\right)\hat{P}_{2}\left(w_{2}\right)\right)}{\partial z_{i}}
\end{eqnarray*}



y adem\'as


\begin{eqnarray*}
\frac{\partial^{2} R_{i}\left(P_{1}\left(z_{1}\right)\tilde{P}_{2}\left(z_{2}\right)\hat{P}_{1}\left(w_{1}\right)\hat{P}_{2}\left(w_{2}\right)\right)}{\partial z_{2}\partial z_{1}}&=&\frac{\partial \tilde{P}_{2}\left(z_{2}\right)}{\partial z_{2}}\cdot\frac{\partial P_{1}\left(z_{1}\right)}{\partial z_{1}}\cdot\frac{\partial^{2} R_{i}\left(P_{1}\left(z_{1}\right)\tilde{P}_{2}\left(z_{2}\right)\hat{P}_{1}\left(w_{1}\right)\hat{P}_{2}\left(w_{2}\right)\right)}{\partial z_{2}\partial z_{1}}\\
&+&\frac{\partial \tilde{P}_{2}\left(z_{2}\right)}{\partial z_{2}}\cdot\frac{\partial P_{1}\left(z_{1}\right)}{\partial z_{1}}\cdot\frac{\partial R_{i}\left(P_{1}\left(z_{1}\right)\tilde{P}_{2}\left(z_{2}\right)\hat{P}_{1}\left(w_{1}\right)\hat{P}_{2}\left(w_{2}\right)\right)}{\partial z_{1}},
\end{eqnarray*}



\begin{eqnarray*}
\frac{\partial^{2} R_{i}\left(P_{1}\left(z_{1}\right)\tilde{P}_{2}\left(z_{2}\right)\hat{P}_{1}\left(w_{1}\right)\hat{P}_{2}\left(w_{2}\right)\right)}{\partial w_{i}\partial z_{1}}&=&\frac{\partial \hat{P}_{i}\left(w_{i}\right)}{\partial z_{2}}\cdot\frac{\partial P_{1}\left(z_{1}\right)}{\partial z_{1}}\cdot\frac{\partial^{2} R_{i}\left(P_{1}\left(z_{1}\right)\tilde{P}_{2}\left(z_{2}\right)\hat{P}_{1}\left(w_{1}\right)\hat{P}_{2}\left(w_{2}\right)\right)}{\partial w_{i}\partial z_{1}}\\
&+&\frac{\partial \hat{P}_{i}\left(w_{i}\right)}{\partial z_{2}}\cdot\frac{\partial P_{1}\left(z_{1}\right)}{\partial z_{1}}\cdot\frac{\partial R_{i}\left(P_{1}\left(z_{1}\right)\tilde{P}_{2}\left(z_{2}\right)\hat{P}_{1}\left(w_{1}\right)\hat{P}_{2}\left(w_{2}\right)\right)}{\partial z_{1}},
\end{eqnarray*}
finalmente

\begin{eqnarray*}
\frac{\partial^{2} R_{i}\left(P_{1}\left(z_{1}\right)\tilde{P}_{2}\left(z_{2}\right)\hat{P}_{1}\left(w_{1}\right)\hat{P}_{2}\left(w_{2}\right)\right)}{\partial w_{i}\partial z_{2}}&=&\frac{\partial \hat{P}_{i}\left(w_{i}\right)}{\partial w_{i}}\cdot\frac{\partial \tilde{P}_{2}\left(z_{2}\right)}{\partial z_{2}}\cdot\frac{\partial^{2} R_{i}\left(P_{1}\left(z_{1}\right)\tilde{P}_{2}\left(z_{2}\right)\hat{P}_{1}\left(w_{1}\right)\hat{P}_{2}\left(w_{2}\right)\right)}{\partial w_{i}\partial z_{2}}\\
&+&\frac{\partial \hat{P}_{i}\left(w_{i}\right)}{\partial w_{i}}\cdot\frac{\partial \tilde{P}_{2}\left(z_{2}\right)}{\partial z_{1}}\cdot\frac{\partial R_{i}\left(P_{1}\left(z_{1}\right)\tilde{P}_{2}\left(z_{2}\right)\hat{P}_{1}\left(w_{1}\right)\hat{P}_{2}\left(w_{2}\right)\right)}{\partial z_{2}},
\end{eqnarray*}

para $i=1,2$.
\end{Prop}

%___________________________________________________________________________________________
%
\section{Sistema de Ecuaciones Lineales para los Segundos Momentos}
%___________________________________________________________________________________________

En el ap\'endice A se demuestra que las ecuaciones para las ecuaciones parciales mixtas est\'an dadas por:


\begin{enumerate}
%___________________________________________________________________________________________
%\subsubsection{Mixtas para $z_{1}$:}
%___________________________________________________________________________________________
%1
\item \begin{eqnarray*}
f_{1}\left(1,1\right)&=&r_{2}P_{1}^{(2)}\left(1\right)+\mu_{1}^{2}R_{2}^{(2)}\left(1\right)+2\mu_{1}r_{2}\left(\frac{\mu_{1}}{1-\tilde{\mu}_{2}}f_{2}\left(2\right)+f_{2}\left(1\right)\right)+\frac{1}{1-\tilde{\mu}_{2}}P_{1}^{(2)}f_{2}\left(2\right)\\
&+&\mu_{1}^{2}\tilde{\theta}_{2}^{(2)}\left(1\right)f_{2}\left(2\right)+\frac{\mu_{1}}{1-\tilde{\mu}_{2}}f_{2}(1,2)+\frac{\mu_{1}}{1-\tilde{\mu}_{2}}\left(\frac{\mu_{1}}{1-\tilde{\mu}_{2}}f_{2}(2,2)+f_{2}(1,2)\right)+f_{2}(1,1).
\end{eqnarray*}

%2

\item \begin{eqnarray*}
f_{1}\left(2,1\right)&=&\mu_{1}r_{2}\tilde{\mu}_{2}+\mu_{1}\tilde{\mu}_{2}R_{2}^{(2)}\left(1\right)+r_{2}\tilde{\mu}_{2}\left(\frac{\mu_{1}}{1-\tilde{\mu}_{2}}f_{2}(2)+f_{2}(1)\right).
\end{eqnarray*}

%3

\item \begin{eqnarray*}
f_{1}\left(3,1\right)&=&\mu_{1}\hat{\mu}_{1}r_{2}+\mu_{1}\hat{\mu}_{1}R_{2}^{(2)}\left(1\right)+r_{2}\frac{\mu_{1}}{1-\tilde{\mu}_{2}}f_{2}(2)+r_{2}\hat{\mu}_{1}\left(\frac{\mu_{1}}{1-\tilde{\mu}_{2}}f_{2}(2)+f_{2}(1)\right)+\mu_{1}r_{2}\hat{F}_{2,1}^{(1)}(1)\\
&+&\left(\frac{\mu_{1}}{1-\tilde{\mu}_{2}}f_{2}(2)+f_{2}(1)\right)\hat{F}_{2,1}^{(1)}(1)+\frac{\mu_{1}\hat{\mu}_{1}}{1-\tilde{\mu}_{2}}f_{2}(2)+\mu_{1}\hat{\mu}_{1}\tilde{\theta}_{2}^{(2)}\left(1\right)f_{2}(2)\\
&+&\mu_{1}\hat{\mu}_{1}\left(\frac{1}{1-\tilde{\mu}_{2}}\right)^{2}f_{2}(2,2)+\frac{\hat{\mu}_{1}}{1-\tilde{\mu}_{2}}f_{2}(1,2).
\end{eqnarray*}

%4

\item \begin{eqnarray*}
f_{1}\left(4,1\right)&=&\mu_{1}\hat{\mu}_{2}r_{2}+\mu_{1}\hat{\mu}_{2}R_{2}^{(2)}\left(1\right)+r_{2}\frac{\mu_{1}\hat{\mu}_{2}}{1-\tilde{\mu}_{2}}f_{2}(2)+\mu_{1}r_{2}\hat{F}_{2,2}^{(1)}(1)+r_{2}\hat{\mu}_{2}\left(\frac{\mu_{1}}{1-\tilde{\mu}_{2}}f_{2}(2)+f_{2}(1)\right)\\
&+&\hat{F}_{2,1}^{(1)}(1)\left(\frac{\mu_{1}}{1-\tilde{\mu}_{2}}f_{2}(2)+f_{2}(1)\right)+\frac{\mu_{1}\hat{\mu}_{2}}{1-\tilde{\mu}_{2}}f_{2}(2)
+\mu_{1}\hat{\mu}_{2}\tilde{\theta}_{2}^{(2)}\left(1\right)f_{2}(2)\\
&+&\mu_{1}\hat{\mu}_{2}\left(\frac{1}{1-\tilde{\mu}_{2}}\right)^{2}f_{2}(2,2)+\frac{\hat{\mu}_{2}}{1-\tilde{\mu}_{2}}f_{2}^{(1,2)}.
\end{eqnarray*}
%___________________________________________________________________________________________
%\subsubsection{Mixtas para $z_{2}$:}
%___________________________________________________________________________________________
%5
\item \begin{eqnarray*}
f_{1}\left(1,2\right)&=&\mu_{1}\tilde{\mu}_{2}r_{2}+\mu_{1}\tilde{\mu}_{2}R_{2}^{(2)}\left(1\right)+r_{2}\tilde{\mu}_{2}\left(\frac{\mu_{1}}{1-\tilde{\mu}_{2}}f_{2}(2)+f_{2}(1)\right).
\end{eqnarray*}

%6

\item \begin{eqnarray*}
f_{1}\left(2,2\right)&=&\tilde{\mu}_{2}^{2}R_{2}^{(2)}(1)+r_{2}\tilde{P}_{2}^{(2)}\left(1\right).
\end{eqnarray*}

%7
\item \begin{eqnarray*}
f_{1}\left(3,2\right)&=&\hat{\mu}_{1}\tilde{\mu}_{2}r_{2}+\hat{\mu}_{1}\tilde{\mu}_{2}R_{2}^{(2)}(1)+
r_{2}\frac{\hat{\mu}_{1}\tilde{\mu}_{2}}{1-\tilde{\mu}_{2}}f_{2}(2)+r_{2}\tilde{\mu}_{2}\hat{F}_{2,2}^{(1)}(1).
\end{eqnarray*}
%8
\item \begin{eqnarray*} f_{1}\left(4,2\right)&=&\hat{\mu}_{2}\tilde{\mu}_{2}r_{2}+\hat{\mu}_{2}\tilde{\mu}_{2}R_{2}^{(2)}(1)+
r_{2}\frac{\hat{\mu}_{2}\tilde{\mu}_{2}}{1-\tilde{\mu}_{2}}f_{2}(2)+r_{2}\tilde{\mu}_{2}\hat{F}_{2,2}^{(1)}(1).
\end{eqnarray*}
%___________________________________________________________________________________________
%\subsubsection{Mixtas para $w_{1}$:}
%___________________________________________________________________________________________

%9
\item \begin{eqnarray*} f_{1}\left(1,3\right)&=&\mu_{1}\hat{\mu}_{1}r_{2}+\mu_{1}\hat{\mu}_{1}R_{2}^{(2)}\left(1\right)+\frac{\mu_{1}\hat{\mu}_{1}}{1-\tilde{\mu}_{2}}f_{2}(2)+r_{2}\frac{\mu_{1}\hat{\mu}_{1}}{1-\tilde{\mu}_{2}}f_{2}(2)+\mu_{1}\hat{\mu}_{1}\tilde{\theta}_{2}^{(2)}\left(1\right)f_{2}(2)\\
&+&r_{2}\hat{\mu}_{1}\left(\frac{\mu_{1}}{1-\tilde{\mu}_{2}}f_{2}(2)+f_{2}\left(1\right)\right)+r_{2}\mu_{1}\hat{F}_{2,1}^{(1)}(1)+\left(\frac{\mu_{1}}{1-\tilde{\mu}_{2}}f_{2}\left(1\right)+f_{2}\left(1\right)\right)\hat{F}_{2,1}^{(1)}(1)\\
&+&\frac{\hat{\mu}_{1}}{1-\tilde{\mu}_{2}}\left(\frac{\mu_{1}}{1-\tilde{\mu}_{2}}f_{2}(2,2)+f_{2}^{(1,2)}\right).
\end{eqnarray*}

%10

\item \begin{eqnarray*} f_{1}\left(2,3\right)&=&\tilde{\mu}_{2}\hat{\mu}_{1}r_{2}+\tilde{\mu}_{2}\hat{\mu}_{1}R_{2}^{(2)}\left(1\right)+r_{2}\frac{\tilde{\mu}_{2}\hat{\mu}_{1}}{1-\tilde{\mu}_{2}}f_{2}(2)+r_{2}\tilde{\mu}_{2}\hat{F}_{2,1}^{(1)}(1).
\end{eqnarray*}

%11

\item \begin{eqnarray*} f_{1}\left(3,3\right)&=&\hat{\mu}_{1}^{2}R_{2}^{(2)}\left(1\right)+r_{2}\hat{P}_{1}^{(2)}\left(1\right)+2r_{2}\frac{\hat{\mu}_{1}^{2}}{1-\tilde{\mu}_{2}}f_{2}(2)+\hat{\mu}_{1}^{2}\tilde{\theta}_{2}^{(2)}\left(1\right)f_{2}(2)+\frac{1}{1-\tilde{\mu}_{2}}\hat{P}_{1}^{(2)}\left(1\right)f_{2}(2)\\
&+&\frac{\hat{\mu}_{1}^{2}}{1-\tilde{\mu}_{2}}f_{2}(2,2)+2r_{2}\hat{\mu}_{1}\hat{F}_{2,1}^{(1)}(1)+2\frac{\hat{\mu}_{1}}{1-\tilde{\mu}_{2}}f_{2}(2)\hat{F}_{2,1}^{(1)}(1)+\hat{f}_{2,1}^{(2)}(1).
\end{eqnarray*}

%12

\item \begin{eqnarray*}
f_{1}\left(4,3\right)&=&r_{2}\hat{\mu}_{2}\hat{\mu}_{1}+\hat{\mu}_{1}\hat{\mu}_{2}R_{2}^{(2)}(1)+\frac{\hat{\mu}_{1}\hat{\mu}_{2}}{1-\tilde{\mu}_{2}}f_{2}\left(2\right)+2r_{2}\frac{\hat{\mu}_{1}\hat{\mu}_{2}}{1-\tilde{\mu}_{2}}f_{2}\left(2\right)+\hat{\mu}_{2}\hat{\mu}_{1}\tilde{\theta}_{2}^{(2)}\left(1\right)f_{2}\left(2\right)\\
&+&r_{2}\hat{\mu}_{1}\hat{F}_{2,2}^{(1)}(1)+\frac{\hat{\mu}_{1}}{1-\tilde{\mu}_{2}}f_{2}\left(2\right)\hat{F}_{2,2}^{(1)}(1)+\hat{\mu}_{1}\hat{\mu}_{2}\left(\frac{1}{1-\tilde{\mu}_{2}}\right)^{2}f_{2}(2,2)+r_{2}\hat{\mu}_{2}\hat{F}_{2,1}^{(1)}(1)\\
&+&\frac{\hat{\mu}_{2}}{1-\tilde{\mu}_{2}}f_{2}\left(2\right)\hat{F}_{2,1}^{(1)}(1)+\hat{f}_{2}(1,2).
\end{eqnarray*}
%___________________________________________________________________________________________
%\subsubsection{Mixtas para $w_{2}$:}
%___________________________________________________________________________________________
%13

\item \begin{eqnarray*}
f_{1}\left(1,4\right)&=&r_{2}\mu_{1}\hat{\mu}_{2}+\mu_{1}\hat{\mu}_{2}R_{2}^{(2)}(1)+\frac{\mu_{1}\hat{\mu}_{2}}{1-\tilde{\mu}_{2}}f_{2}(2)+r_{2}\frac{\mu_{1}\hat{\mu}_{2}}{1-\tilde{\mu}_{2}}f_{2}(2)+\mu_{1}\hat{\mu}_{2}\tilde{\theta}_{2}^{(2)}\left(1\right)f_{2}(2)\\
&+&r_{2}\mu_{1}\hat{F}_{2,2}^{(1)}(1)+r_{2}\hat{\mu}_{2}\left(\frac{\mu_{1}}{1-\tilde{\mu}_{2}}f_{2}(2)+f_{2}(1)\right)+\hat{F}_{2,2}^{(1)}(1)\left(\frac{\mu_{1}}{1-\tilde{\mu}_{2}}f_{2}(2)+f_{2}(1)\right)\\
&+&\frac{\hat{\mu}_{2}}{1-\tilde{\mu}_{2}}\left(\frac{\mu_{1}}{1-\tilde{\mu}_{2}}f_{2}(2,2)+f_{2}(1,2)\right).
\end{eqnarray*}

%14

\item \begin{eqnarray*} f_{1}\left(2,4\right)
&=&r_{2}\tilde{\mu}_{2}\hat{\mu}_{2}+\tilde{\mu}_{2}\hat{\mu}_{2}R_{2}^{(2)}(1)+r_{2}\frac{\tilde{\mu}_{2}\hat{\mu}_{2}}{1-\tilde{\mu}_{2}}f_{2}(2)+r_{2}\tilde{\mu}_{2}\hat{F}_{2,2}^{(1)}(1).
\end{eqnarray*}


%15
\item \begin{eqnarray*} f_{1}\left(3,4\right)&=&r_{2}\hat{\mu}_{1}\hat{\mu}_{2}+\hat{\mu}_{1}\hat{\mu}_{2}R_{2}^{(2)}\left(1\right)+\frac{\hat{\mu}_{1}\hat{\mu}_{2}}{1-\tilde{\mu}_{2}}f_{2}(2)+2r_{2}\frac{\hat{\mu}_{1}\hat{\mu}_{2}}{1-\tilde{\mu}_{2}}f_{2}(2)+\hat{\mu}_{1}\hat{\mu}_{2}\theta_{2}^{(2)}\left(1\right)f_{2}(2)\\
&+&r_{2}\hat{\mu}_{1}\hat{F}_{2,2}^{(1)}(1)+\frac{\hat{\mu}_{1}}{1-\tilde{\mu}_{2}}f_{2}(2)\hat{F}_{2,2}^{(1)}(1)+\hat{\mu}_{1}\hat{\mu}_{2}\left(\frac{1}{1-\tilde{\mu}_{2}}\right)^{2}f_{2}(2,2)+r_{2}\hat{\mu}_{2}\hat{F}_{2,2}^{(1)}(1)\\
&+&\frac{\hat{\mu}_{2}}{1-\tilde{\mu}_{2}}f_{2}(2)\hat{F}_{2,1}^{(1)}(1)+\hat{f}_{2}^{(2)}(1,2).
\end{eqnarray*}

%16

\item \begin{eqnarray*} f_{1}\left(4,4\right)&=&\hat{\mu}_{2}^{2}R_{2}^{(2)}(1)+r_{2}\hat{P}_{2}^{(2)}\left(1\right)+2r_{2}\frac{\hat{\mu}_{2}^{2}}{1-\tilde{\mu}_{2}}f_{2}(2)+\hat{\mu}_{2}^{2}\tilde{\theta}_{2}^{(2)}\left(1\right)f_{2}(2)+\frac{1}{1-\tilde{\mu}_{2}}\hat{P}_{2}^{(2)}\left(1\right)f_{2}(2)\\
&+&2r_{2}\hat{\mu}_{2}\hat{F}_{2,2}^{(1)}(1)+2\frac{\hat{\mu}_{2}}{1-\tilde{\mu}_{2}}f_{2}(2)\hat{F}_{2,2}^{(1)}(1)+\left(\frac{\hat{\mu}_{2}}{1-\tilde{\mu}_{2}}\right)^{2}f_{2}(2,2)+\hat{f}_{2,2}^{(2)}(1).
\end{eqnarray*}
%\end{enumerate}
%___________________________________________________________________________________________
%
%\subsection{Derivadas de Segundo Orden para $F_{2}$}
%___________________________________________________________________________________________


%\begin{enumerate}

%___________________________________________________________________________________________
%\subsubsection{Mixtas para $z_{1}$:}
%___________________________________________________________________________________________

%17

\item \begin{eqnarray*} f_{2}\left(1,1\right)&=&r_{1}P_{1}^{(2)}\left(1\right)+\mu_{1}^{2}R_{1}^{(2)}\left(1\right).
\end{eqnarray*}

%18

\item \begin{eqnarray*} f_{2}\left(2,1\right)&=&\mu_{1}\tilde{\mu}_{2}r_{1}+\mu_{1}\tilde{\mu}_{2}R_{1}^{(2)}(1)+
r_{1}\mu_{1}\left(\frac{\tilde{\mu}_{2}}{1-\mu_{1}}f_{1}(1)+f_{1}(2)\right).
\end{eqnarray*}

%19

\item \begin{eqnarray*} f_{2}\left(3,1\right)&=&r_{1}\mu_{1}\hat{\mu}_{1}+\mu_{1}\hat{\mu}_{1}R_{1}^{(2)}\left(1\right)+r_{1}\frac{\mu_{1}\hat{\mu}_{1}}{1-\mu_{1}}f_{1}(1)+r_{1}\mu_{1}\hat{F}_{1,1}^{(1)}(1).
\end{eqnarray*}

%20

\item \begin{eqnarray*}
f_{2}\left(4,1\right)&=&\mu_{1}\hat{\mu}_{2}r_{1}+\mu_{1}\hat{\mu}_{2}R_{1}^{(2)}\left(1\right)+r_{1}\mu_{1}\hat{F}_{1,2}^{(1)}(1)+r_{1}\frac{\mu_{1}\hat{\mu}_{2}}{1-\mu_{1}}f_{1}(1).
\end{eqnarray*}
%___________________________________________________________________________________________
%\subsubsection{Mixtas para $z_{2}$:}
%___________________________________________________________________________________________
%21
\item \begin{eqnarray*}
f_{2}\left(1,2\right)&=&r_{1}\mu_{1}\tilde{\mu}_{2}+\mu_{1}\tilde{\mu}_{2}R_{1}^{(2)}\left(1\right)+r_{1}\mu_{1}\left(\frac{\tilde{\mu}_{2}}{1-\mu_{1}}f_{1}(1)+f_{1}(2)\right).
\end{eqnarray*}

%22

\item \begin{eqnarray*}
f_{2}\left(2,2\right)&=&\tilde{\mu}_{2}^{2}R_{1}^{(2)}\left(1\right)+r_{1}\tilde{P}_{2}^{(2)}\left(1\right)+2r_{1}\tilde{\mu}_{2}\left(\frac{\tilde{\mu}_{2}}{1-\mu_{1}}f_{1}(1)+f_{1}(2)\right)+f_{1}(2,2)\\
&+&\tilde{\mu}_{2}^{2}\theta_{1}^{(2)}\left(1\right)f_{1}(1)+\frac{1}{1-\mu_{1}}\tilde{P}_{2}^{(2)}\left(1\right)f_{1}(1)+\frac{\tilde{\mu}_{2}}{1-\mu_{1}}f_{1}(1,2)\\
&+&\frac{\tilde{\mu}_{2}}{1-\mu_{1}}\left(\frac{\tilde{\mu}_{2}}{1-\mu_{1}}f_{1}(1,1)+f_{1}(1,2)\right).
\end{eqnarray*}

%23

\item \begin{eqnarray*}
f_{2}\left(3,2\right)&=&\tilde{\mu}_{2}\hat{\mu}_{1}r_{1}+\tilde{\mu}_{2}\hat{\mu}_{1}R_{1}^{(2)}\left(1\right)+r_{1}\frac{\tilde{\mu}_{2}\hat{\mu}_{1}}{1-\mu_{1}}f_{1}(1)+\hat{\mu}_{1}r_{1}\left(\frac{\tilde{\mu}_{2}}{1-\mu_{1}}f_{1}(1)+f_{1}(2)\right)+r_{1}\tilde{\mu}_{2}\hat{F}_{1,1}^{(1)}(1)\\
&+&\left(\frac{\tilde{\mu}_{2}}{1-\mu_{1}}f_{1}(1)+f_{1}(2)\right)\hat{F}_{1,1}^{(1)}(1)+\frac{\tilde{\mu}_{2}\hat{\mu}_{1}}{1-\mu_{1}}f_{1}(1)+\tilde{\mu}_{2}\hat{\mu}_{1}\theta_{1}^{(2)}\left(1\right)f_{1}(1)+\frac{\hat{\mu}_{1}}{1-\mu_{1}}f_{1}(1,2)\\
&+&\left(\frac{1}{1-\mu_{1}}\right)^{2}\tilde{\mu}_{2}\hat{\mu}_{1}f_{1}(1,1).
\end{eqnarray*}

%24


\item \begin{eqnarray*}
f_{2}\left(4,2\right)&=&\hat{\mu}_{2}\tilde{\mu}_{2}r_{1}+\hat{\mu}_{2}\tilde{\mu}_{2}R_{1}^{(2)}(1)+r_{1}\tilde{\mu}_{2}\hat{F}_{1,2}^{(1)}(1)+r_{1}\frac{\hat{\mu}_{2}\tilde{\mu}_{2}}{1-\mu_{1}}f_{1}(1)+\hat{\mu}_{2}r_{1}\left(\frac{\tilde{\mu}_{2}}{1-\mu_{1}}f_{1}(1)+f_{1}(2)\right)\\
&+&\left(\frac{\tilde{\mu}_{2}}{1-\mu_{1}}f_{1}(1)+f_{1}(2)\right)\hat{F}_{1,2}^{(1)}(1)+\frac{\tilde{\mu}_{2}\hat{\mu_{2}}}{1-\mu_{1}}f_{1}(1)+\hat{\mu}_{2}\tilde{\mu}_{2}\theta_{1}^{(2)}\left(1\right)f_{1}(1)+\frac{\hat{\mu}_{2}}{1-\mu_{1}}f_{1}(1,2)\\
&+&\tilde{\mu}_{2}\hat{\mu}_{2}\left(\frac{1}{1-\mu_{1}}\right)^{2}f_{1}(1,1).
\end{eqnarray*}
%___________________________________________________________________________________________
%\subsubsection{Mixtas para $w_{1}$:}
%___________________________________________________________________________________________

%25

\item \begin{eqnarray*} f_{2}\left(1,3\right)&=&r_{1}\mu_{1}\hat{\mu}_{1}+\mu_{1}\hat{\mu}_{1}R_{1}^{(2)}(1)+r_{1}\frac{\mu_{1}\hat{\mu}_{1}}{1-\mu_{1}}f_{1}(1)+r_{1}\mu_{1}\hat{F}_{1,1}^{(1)}(1).
\end{eqnarray*}

%26

\item \begin{eqnarray*} f_{2}\left(2,3\right)&=&r_{1}\hat{\mu}_{1}\tilde{\mu}_{2}+\tilde{\mu}_{2}\hat{\mu}_{1}R_{1}^{(2)}\left(1\right)+\frac{\hat{\mu}_{1}\tilde{\mu}_{2}}{1-\mu_{1}}f_{1}(1)+r_{1}\frac{\hat{\mu}_{1}\tilde{\mu}_{2}}{1-\mu_{1}}f_{1}(1)+\hat{\mu}_{1}\tilde{\mu}_{2}\theta_{1}^{(2)}\left(1\right)f_{1}(1)\\
&+&r_{1}\hat{\mu}_{1}\left(f_{1}(1)+\frac{\tilde{\mu}_{2}}{1-\mu_{1}}f_{1}(1)\right)+
r_{1}\tilde{\mu}_{2}\hat{F}_{1,1}(1)+\left(f_{1}(2)+\frac{\tilde{\mu}_{2}}{1-\mu_{1}}f_{1}(1)\right)\hat{F}_{1,1}(1)\\
&+&\frac{\hat{\mu}_{1}}{1-\mu_{1}}\left(f_{1}(1,2)+\frac{\tilde{\mu}_{2}}{1-\mu_{1}}f_{1}(1,1)\right).
\end{eqnarray*}

%27

\item \begin{eqnarray*} f_{2}\left(3,3\right)&=&\hat{\mu}_{1}^{2}R_{1}^{(2)}\left(1\right)+r_{1}\hat{P}_{1}^{(2)}\left(1\right)+2r_{1}\frac{\hat{\mu}_{1}^{2}}{1-\mu_{1}}f_{1}(1)+\hat{\mu}_{1}^{2}\theta_{1}^{(2)}\left(1\right)f_{1}(1)\\
&+&\frac{1}{1-\mu_{1}}\hat{P}_{1}^{(2)}\left(1\right)f_{1}(1)+2r_{1}\hat{\mu}_{1}\hat{F}_{1,1}^{(1)}(1)+2\frac{\hat{\mu}_{1}}{1-\mu_{1}}f_{1}(1)\hat{F}_{1,1}(1)\\
&+&\left(\frac{\hat{\mu}_{1}}{1-\mu_{1}}\right)^{2}f_{1}(1,1)+\hat{f}_{1,1}^{(2)}(1).
\end{eqnarray*}

%28

\item \begin{eqnarray*}
f_{2}\left(4,3\right)&=&r_{1}\hat{\mu}_{1}\hat{\mu}_{2}+\hat{\mu}_{1}\hat{\mu}_{2}R_{1}^{(2)}\left(1\right)+r_{1}\hat{\mu}_{1}\hat{F}_{1,2}(1)+
\frac{\hat{\mu}_{1}\hat{\mu}_{2}}{1-\mu_{1}}f_{1}(1)+2r_{1}\frac{\hat{\mu}_{1}\hat{\mu}_{2}}{1-\mu_{1}}f_{1}(1)\\
&+&\hat{\mu}_{1}\hat{\mu}_{2}\theta_{1}^{(2)}\left(1\right)f_{1}(1)+\frac{\hat{\mu}_{1}}{1-\mu_{1}}f_{1}(1)\hat{F}_{1,2}(1)+r_{1}\hat{\mu}_{2}\hat{F}_{1,1}(1)+\frac{\hat{\mu}_{2}}{1-\mu_{1}}\hat{F}_{1,1}(1)f_{1}(1)\\
&+&\hat{f}_{1}^{(2)}(1,2)+\hat{\mu}_{1}\hat{\mu}_{2}\left(\frac{1}{1-\mu_{1}}\right)^{2}f_{1}(2,2).
\end{eqnarray*}
%___________________________________________________________________________________________
%\subsubsection{Mixtas para $w_{2}$:}
%___________________________________________________________________________________________

%29

\item \begin{eqnarray*} f_{2}\left(1,4\right)&=&r_{1}\mu_{1}\hat{\mu}_{2}+\mu_{1}\hat{\mu}_{2}R_{1}^{(2)}\left(1\right)+r_{1}\mu_{1}\hat{F}_{1,2}(1)+r_{1}\frac{\mu_{1}\hat{\mu}_{2}}{1-\mu_{1}}f_{1}(1).
\end{eqnarray*}


%30

\item \begin{eqnarray*} f_{2}\left(2,4\right)&=&r_{1}\hat{\mu}_{2}\tilde{\mu}_{2}+\hat{\mu}_{2}\tilde{\mu}_{2}R_{1}^{(2)}\left(1\right)+r_{1}\tilde{\mu}_{2}\hat{F}_{1,2}(1)+\frac{\hat{\mu}_{2}\tilde{\mu}_{2}}{1-\mu_{1}}f_{1}(1)+r_{1}\frac{\hat{\mu}_{2}\tilde{\mu}_{2}}{1-\mu_{1}}f_{1}(1)\\
&+&\hat{\mu}_{2}\tilde{\mu}_{2}\theta_{1}^{(2)}\left(1\right)f_{1}(1)+r_{1}\hat{\mu}_{2}\left(f_{1}(2)+\frac{\tilde{\mu}_{2}}{1-\mu_{1}}f_{1}(1)\right)+\left(f_{1}(2)+\frac{\tilde{\mu}_{2}}{1-\mu_{1}}f_{1}(1)\right)\hat{F}_{1,2}(1)\\&+&\frac{\hat{\mu}_{2}}{1-\mu_{1}}\left(f_{1}(1,2)+\frac{\tilde{\mu}_{2}}{1-\mu_{1}}f_{1}(1,1)\right).
\end{eqnarray*}

%31

\item \begin{eqnarray*}
f_{2}\left(3,4\right)&=&r_{1}\hat{\mu}_{1}\hat{\mu}_{2}+\hat{\mu}_{1}\hat{\mu}_{2}R_{1}^{(2)}\left(1\right)+r_{1}\hat{\mu}_{1}\hat{F}_{1,2}(1)+
\frac{\hat{\mu}_{1}\hat{\mu}_{2}}{1-\mu_{1}}f_{1}(1)+2r_{1}\frac{\hat{\mu}_{1}\hat{\mu}_{2}}{1-\mu_{1}}f_{1}(1)\\
&+&\hat{\mu}_{1}\hat{\mu}_{2}\theta_{1}^{(2)}\left(1\right)f_{1}(1)+\frac{\hat{\mu}_{1}}{1-\mu_{1}}\hat{F}_{1,2}(1)f_{1}(1)+r_{1}\hat{\mu}_{2}\hat{F}_{1,1}(1)+\frac{\hat{\mu}_{2}}{1-\mu_{1}}\hat{F}_{1,1}(1)f_{1}(1)\\
&+&\hat{f}_{1}^{(2)}(1,2)+\hat{\mu}_{1}\hat{\mu}_{2}\left(\frac{1}{1-\mu_{1}}\right)^{2}f_{1}(1,1).
\end{eqnarray*}

%32

\item \begin{eqnarray*} f_{2}\left(4,4\right)&=&\hat{\mu}_{2}R_{1}^{(2)}\left(1\right)+r_{1}\hat{P}_{2}^{(2)}\left(1\right)+2r_{1}\hat{\mu}_{2}\hat{F}_{1}^{(0,1)}+\hat{f}_{1,2}^{(2)}(1)+2r_{1}\frac{\hat{\mu}_{2}^{2}}{1-\mu_{1}}f_{1}(1)+\hat{\mu}_{2}^{2}\theta_{1}^{(2)}\left(1\right)f_{1}(1)\\
&+&\frac{1}{1-\mu_{1}}\hat{P}_{2}^{(2)}\left(1\right)f_{1}(1) +
2\frac{\hat{\mu}_{2}}{1-\mu_{1}}f_{1}(1)\hat{F}_{1,2}(1)+\left(\frac{\hat{\mu}_{2}}{1-\mu_{1}}\right)^{2}f_{1}(1,1).
\end{eqnarray*}
%\end{enumerate}

%___________________________________________________________________________________________
%
%\subsection{Derivadas de Segundo Orden para $\hat{F}_{1}$}
%___________________________________________________________________________________________


%\begin{enumerate}
%___________________________________________________________________________________________
%\subsubsection{Mixtas para $z_{1}$:}
%___________________________________________________________________________________________
%33

\item \begin{eqnarray*} \hat{f}_{1}\left(1,1\right)&=&\hat{r}_{2}P_{1}^{(2)}\left(1\right)+
\mu_{1}^{2}\hat{R}_{2}^{(2)}\left(1\right)+
2\hat{r}_{2}\frac{\mu_{1}^{2}}{1-\hat{\mu}_{2}}\hat{f}_{2}(2)+
\frac{1}{1-\hat{\mu}_{2}}P_{1}^{(2)}\left(1\right)\hat{f}_{2}(2)+
\mu_{1}^{2}\hat{\theta}_{2}^{(2)}\left(1\right)\hat{f}_{2}(2)\\
&+&\left(\frac{\mu_{1}^{2}}{1-\hat{\mu}_{2}}\right)^{2}\hat{f}_{2}(2,2)+2\hat{r}_{2}\mu_{1}F_{2,1}(1)+2\frac{\mu_{1}}{1-\hat{\mu}_{2}}\hat{f}_{2}(2)F_{2,1}(1)+F_{2,1}^{(2)}(1).
\end{eqnarray*}

%34

\item \begin{eqnarray*} \hat{f}_{1}\left(2,1\right)&=&\hat{r}_{2}\mu_{1}\tilde{\mu}_{2}+\mu_{1}\tilde{\mu}_{2}\hat{R}_{2}^{(2)}\left(1\right)+\hat{r}_{2}\mu_{1}F_{2,2}(1)+
\frac{\mu_{1}\tilde{\mu}_{2}}{1-\hat{\mu}_{2}}\hat{f}_{2}(2)+2\hat{r}_{2}\frac{\mu_{1}\tilde{\mu}_{2}}{1-\hat{\mu}_{2}}\hat{f}_{2}(2)\\
&+&\mu_{1}\tilde{\mu}_{2}\hat{\theta}_{2}^{(2)}\left(1\right)\hat{f}_{2}(2)+\frac{\mu_{1}}{1-\hat{\mu}_{2}}F_{2,2}(1)\hat{f}_{2}(2)+\mu_{1} \tilde{\mu}_{2}\left(\frac{1}{1-\hat{\mu}_{2}}\right)^{2}\hat{f}_{2}(2,2)+\hat{r}_{2}\tilde{\mu}_{2}F_{2,1}(1)\\
&+&\frac{\tilde{\mu}_{2}}{1-\hat{\mu}_{2}}\hat{f}_{2}(2)F_{2,1}(1)+f_{2,1}^{(2)}(1).
\end{eqnarray*}


%35

\item \begin{eqnarray*} \hat{f}_{1}\left(3,1\right)&=&\hat{r}_{2}\mu_{1}\hat{\mu}_{1}+\mu_{1}\hat{\mu}_{1}\hat{R}_{2}^{(2)}\left(1\right)+\hat{r}_{2}\frac{\mu_{1}\hat{\mu}_{1}}{1-\hat{\mu}_{2}}\hat{f}_{2}(2)+\hat{r}_{2}\hat{\mu}_{1}F_{2,1}(1)+\hat{r}_{2}\mu_{1}\hat{f}_{2}(1)\\
&+&F_{2,1}(1)\hat{f}_{2}(1)+\frac{\mu_{1}}{1-\hat{\mu}_{2}}\hat{f}_{2}(1,2).
\end{eqnarray*}

%36

\item \begin{eqnarray*} \hat{f}_{1}\left(4,1\right)&=&\hat{r}_{2}\mu_{1}\hat{\mu}_{2}+\mu_{1}\hat{\mu}_{2}\hat{R}_{2}^{(2)}\left(1\right)+\frac{\mu_{1}\hat{\mu}_{2}}{1-\hat{\mu}_{2}}\hat{f}_{2}(2)+2\hat{r}_{2}\frac{\mu_{1}\hat{\mu}_{2}}{1-\hat{\mu}_{2}}\hat{f}_{2}(2)+\mu_{1}\hat{\mu}_{2}\hat{\theta}_{2}^{(2)}\left(1\right)\hat{f}_{2}(2)\\
&+&\mu_{1}\hat{\mu}_{2}\left(\frac{1}{1-\hat{\mu}_{2}}\right)^{2}\hat{f}_{2}(2,2)+\hat{r}_{2}\hat{\mu}_{2}F_{2,1}(1)+\frac{\hat{\mu}_{2}}{1-\hat{\mu}_{2}}\hat{f}_{2}(2)F_{2,1}(1).
\end{eqnarray*}
%___________________________________________________________________________________________
%\subsubsection{Mixtas para $z_{2}$:}
%___________________________________________________________________________________________

%37

\item \begin{eqnarray*} \hat{f}_{1}\left(1,2\right)&=&\hat{r}_{2}\mu_{1}\tilde{\mu}_{2}+\mu_{1}\tilde{\mu}_{2}\hat{R}_{2}^{(2)}\left(1\right)+\mu_{1}\hat{r}_{2}F_{2,2}(1)+
\frac{\mu_{1}\tilde{\mu}_{2}}{1-\hat{\mu}_{2}}\hat{f}_{2}(2)+2\hat{r}_{2}\frac{\mu_{1}\tilde{\mu}_{2}}{1-\hat{\mu}_{2}}\hat{f}_{2}(2)\\
&+&\mu_{1}\tilde{\mu}_{2}\hat{\theta}_{2}^{(2)}\left(1\right)\hat{f}_{2}(2)+\frac{\mu_{1}}{1-\hat{\mu}_{2}}F_{2,2}(1)\hat{f}_{2}(2)+\mu_{1}\tilde{\mu}_{2}\left(\frac{1}{1-\hat{\mu}_{2}}\right)^{2}\hat{f}_{2}(2,2)\\
&+&\hat{r}_{2}\tilde{\mu}_{2}F_{2,1}(1)+\frac{\tilde{\mu}_{2}}{1-\hat{\mu}_{2}}\hat{f}_{2}(2)F_{2,1}(1)+f_{2}^{(2)}(1,2).
\end{eqnarray*}

%38

\item \begin{eqnarray*}\hat{f}_{1}\left(2,2\right)&=&\hat{r}_{2}\tilde{P}_{2}^{(2)}\left(1\right)+\tilde{\mu}_{2}^{2}\hat{R}_{2}^{(2)}\left(1\right)+2\hat{r}_{2}\tilde{\mu}_{2}F_{2,2}(1)+2\hat{r}_{2}\frac{\tilde{\mu}_{2}^{2}}{1-\hat{\mu}_{2}}\hat{f}_{2}(2)\\
&+&\frac{1}{1-\hat{\mu}_{2}}\tilde{P}_{2}^{(2)}\left(1\right)\hat{f}_{2}(2)+\tilde{\mu}_{2}^{2}\hat{\theta}_{2}^{(2)}\left(1\right)\hat{f}_{2}(2)+2\frac{\tilde{\mu}_{2}}{1-\hat{\mu}_{2}}F_{2,2}(1)\hat{f}_{2}(2)\\
&+&f_{2,2}^{(2)}(1)+\left(\frac{\tilde{\mu}_{2}}{1-\hat{\mu}_{2}}\right)^{2}\hat{f}_{2}(2,2).
\end{eqnarray*}

%39

\item \begin{eqnarray*} \hat{f}_{1}\left(3,2\right)&=&\hat{r}_{2}\tilde{\mu}_{2}\hat{\mu}_{1}+\tilde{\mu}_{2}\hat{\mu}_{1}\hat{R}_{2}^{(2)}\left(1\right)+\hat{r}_{2}\hat{\mu}_{1}F_{2,2}(1)+\hat{r}_{2}\frac{\tilde{\mu}_{2}\hat{\mu}_{1}}{1-\hat{\mu}_{2}}\hat{f}_{2}(2)+\hat{r}_{2}\tilde{\mu}_{2}\hat{f}_{2}(1)+F_{2,2}(1)\hat{f}_{2}(1)\\
&+&\frac{\tilde{\mu}_{2}}{1-\hat{\mu}_{2}}\hat{f}_{2}(1,2).
\end{eqnarray*}

%40

\item \begin{eqnarray*} \hat{f}_{1}\left(4,2\right)&=&\hat{r}_{2}\tilde{\mu}_{2}\hat{\mu}_{2}+\tilde{\mu}_{2}\hat{\mu}_{2}\hat{R}_{2}^{(2)}\left(1\right)+\hat{r}_{2}\hat{\mu}_{2}F_{2,2}(1)+
\frac{\tilde{\mu}_{2}\hat{\mu}_{2}}{1-\hat{\mu}_{2}}\hat{f}_{2}(2)+2\hat{r}_{2}\frac{\tilde{\mu}_{2}\hat{\mu}_{2}}{1-\hat{\mu}_{2}}\hat{f}_{2}(2)\\
&+&\tilde{\mu}_{2}\hat{\mu}_{2}\hat{\theta}_{2}^{(2)}\left(1\right)\hat{f}_{2}(2)+\frac{\hat{\mu}_{2}}{1-\hat{\mu}_{2}}F_{2,2}(1)\hat{f}_{2}(1)+\tilde{\mu}_{2}\hat{\mu}_{2}\left(\frac{1}{1-\hat{\mu}_{2}}\right)\hat{f}_{2}(2,2).
\end{eqnarray*}
%___________________________________________________________________________________________
%\subsubsection{Mixtas para $w_{1}$:}
%___________________________________________________________________________________________

%41


\item \begin{eqnarray*} \hat{f}_{1}\left(1,3\right)&=&\hat{r}_{2}\mu_{1}\hat{\mu}_{1}+\mu_{1}\hat{\mu}_{1}\hat{R}_{2}^{(2)}\left(1\right)+\hat{r}_{2}\frac{\mu_{1}\hat{\mu}_{1}}{1-\hat{\mu}_{2}}\hat{f}_{2}(2)+\hat{r}_{2}\hat{\mu}_{1}F_{2,1}(1)+\hat{r}_{2}\mu_{1}\hat{f}_{2}(1)\\
&+&F_{2,1}(1)\hat{f}_{2}(1)+\frac{\mu_{1}}{1-\hat{\mu}_{2}}\hat{f}_{2}(1,2).
\end{eqnarray*}


%42

\item \begin{eqnarray*} \hat{f}_{1}\left(2,3\right)&=&\hat{r}_{2}\tilde{\mu}_{2}\hat{\mu}_{1}+\tilde{\mu}_{2}\hat{\mu}_{1}\hat{R}_{2}^{(2)}\left(1\right)+\hat{r}_{2}\hat{\mu}_{1}F_{2,2}(1)+\hat{r}_{2}\frac{\tilde{\mu}_{2}\hat{\mu}_{1}}{1-\hat{\mu}_{2}}\hat{f}_{2}(2)+\hat{r}_{2}\tilde{\mu}_{2}\hat{f}_{2}(1)\\
&+&F_{2,2}(1)\hat{f}_{2}(1)+\frac{\tilde{\mu}_{2}}{1-\hat{\mu}_{2}}\hat{f}_{2}(1,2).
\end{eqnarray*}


%43

\item \begin{eqnarray*} \hat{f}_{1}\left(3,3\right)&=&\hat{r}_{2}\hat{P}_{1}^{(2)}\left(1\right)+\hat{\mu}_{1}^{2}\hat{R}_{2}^{(2)}\left(1\right)+2\hat{r}_{2}\hat{\mu}_{1}\hat{f}_{2}(1)+\hat{f}_{2}(1,1).
\end{eqnarray*}


%44

\item \begin{eqnarray*} \hat{f}_{1}\left(4,3\right)&=&\hat{r}_{2}\hat{\mu}_{1}\hat{\mu}_{2}+\hat{\mu}_{1}\hat{\mu}_{2}\hat{R}_{2}^{(2)}\left(1\right)+
\hat{r}_{2}\frac{\hat{\mu}_{2}\hat{\mu}_{1}}{1-\hat{\mu}_{2}}\hat{f}_{2}(2)+\hat{r}_{2}\hat{\mu}_{2}\hat{f}_{2}(1)+\frac{\hat{\mu}_{2}}{1-\hat{\mu}_{2}}\hat{f}_{2}(1,2).
\end{eqnarray*}
%___________________________________________________________________________________________
%\subsubsection{Mixtas para $w_{2}$:}
%___________________________________________________________________________________________


%45


\item \begin{eqnarray*} \hat{f}_{1}\left(1,4\right)&=&\hat{r}_{2}\mu_{1}\hat{\mu}_{2}+\mu_{1}\hat{\mu}_{2}\hat{R}_{2}^{(2)}\left(1\right)+
\frac{\mu_{1}\hat{\mu}_{2}}{1-\hat{\mu}_{2}}\hat{f}_{2}(2) +2\hat{r}_{2}\frac{\mu_{1}\hat{\mu}_{2}}{1-\hat{\mu}_{2}}\hat{f}_{2}(2)\\
&+&\mu_{1}\hat{\mu}_{2}\hat{\theta}_{2}^{(2)}\left(1\right)\hat{f}_{2}(2)+\mu_{1}\hat{\mu}_{2}\left(\frac{1}{1-\hat{\mu}_{2}}\right)^{2}\hat{f}_{2}(2,2)+\hat{r}_{2}\hat{\mu}_{2}F_{2,1}(1)+\frac{\hat{\mu}_{2}}{1-\hat{\mu}_{2}}\hat{f}_{2}(2)F_{2,1}(1).\end{eqnarray*}


%46
\item \begin{eqnarray*} \hat{f}_{1}\left(2,4\right)&=&\hat{r}_{2}\tilde{\mu}_{2}\hat{\mu}_{2}+\tilde{\mu}_{2}\hat{\mu}_{2}\hat{R}_{2}^{(2)}\left(1\right)+\hat{r}_{2}\hat{\mu}_{2}F_{2,2}(1)+\frac{\tilde{\mu}_{2}\hat{\mu}_{2}}{1-\hat{\mu}_{2}}\hat{f}_{2}(2)+2\hat{r}_{2}\frac{\tilde{\mu}_{2}\hat{\mu}_{2}}{1-\hat{\mu}_{2}}\hat{f}_{2}(2)\\
&+&\tilde{\mu}_{2}\hat{\mu}_{2}\hat{\theta}_{2}^{(2)}\left(1\right)\hat{f}_{2}(2)+\frac{\hat{\mu}_{2}}{1-\hat{\mu}_{2}}\hat{f}_{2}(2)F_{2,2}(1)+\tilde{\mu}_{2}\hat{\mu}_{2}\left(\frac{1}{1-\hat{\mu}_{2}}\right)^{2}\hat{f}_{2}(2,2).
\end{eqnarray*}

%47

\item \begin{eqnarray*} \hat{f}_{1}\left(3,4\right)&=&\hat{r}_{2}\hat{\mu}_{1}\hat{\mu}_{2}+\hat{\mu}_{1}\hat{\mu}_{2}\hat{R}_{2}^{(2)}\left(1\right)+
\hat{r}_{2}\frac{\hat{\mu}_{1}\hat{\mu}_{2}}{1-\hat{\mu}_{2}}\hat{f}_{2}(2)+
\hat{r}_{2}\hat{\mu}_{2}\hat{f}_{2}(1)+\frac{\hat{\mu}_{2}}{1-\hat{\mu}_{2}}\hat{f}_{2}(1,2).
\end{eqnarray*}

%48

\item \begin{eqnarray*} \hat{f}_{1}\left(4,4\right)&=&\hat{r}_{2}P_{2}^{(2)}\left(1\right)+\hat{\mu}_{2}^{2}\hat{R}_{2}^{(2)}\left(1\right)+2\hat{r}_{2}\frac{\hat{\mu}_{2}^{2}}{1-\hat{\mu}_{2}}\hat{f}_{2}(2)+\frac{1}{1-\hat{\mu}_{2}}\hat{P}_{2}^{(2)}\left(1\right)\hat{f}_{2}(2)\\
&+&\hat{\mu}_{2}^{2}\hat{\theta}_{2}^{(2)}\left(1\right)\hat{f}_{2}(2)+\left(\frac{\hat{\mu}_{2}}{1-\hat{\mu}_{2}}\right)^{2}\hat{f}_{2}(2,2).
\end{eqnarray*}


%\end{enumerate}



%___________________________________________________________________________________________
%
%\subsection{Derivadas de Segundo Orden para $\hat{F}_{2}$}
%___________________________________________________________________________________________
%\begin{enumerate}
%___________________________________________________________________________________________
%\subsubsection{Mixtas para $z_{1}$:}
%___________________________________________________________________________________________
%49

\item \begin{eqnarray*} \hat{f}_{2}\left(,1\right)&=&\hat{r}_{1}P_{1}^{(2)}\left(1\right)+
\mu_{1}^{2}\hat{R}_{1}^{(2)}\left(1\right)+2\hat{r}_{1}\mu_{1}F_{1,1}(1)+
2\hat{r}_{1}\frac{\mu_{1}^{2}}{1-\hat{\mu}_{1}}\hat{f}_{1}(1)+\frac{1}{1-\hat{\mu}_{1}}P_{1}^{(2)}\left(1\right)\hat{f}_{1}(1)\\
&+&\mu_{1}^{2}\hat{\theta}_{1}^{(2)}\left(1\right)\hat{f}_{1}(1)+2\frac{\mu_{1}}{1-\hat{\mu}_{1}}\hat{f}_{1}^(1)F_{1,1}(1)+f_{1,1}^{(2)}(1)+\left(\frac{\mu_{1}}{1-\hat{\mu}_{1}}\right)^{2}\hat{f}_{1}^{(1,1)}.
\end{eqnarray*}

%50

\item \begin{eqnarray*} \hat{f}_{2}\left(2,1\right)&=&\hat{r}_{1}\mu_{1}\tilde{\mu}_{2}+\mu_{1}\tilde{\mu}_{2}\hat{R}_{1}^{(2)}\left(1\right)+
\hat{r}_{1}\mu_{1}F_{1,2}(1)+\tilde{\mu}_{2}\hat{r}_{1}F_{1,1}(1)+
\frac{\mu_{1}\tilde{\mu}_{2}}{1-\hat{\mu}_{1}}\hat{f}_{1}(1)\\
&+&2\hat{r}_{1}\frac{\mu_{1}\tilde{\mu}_{2}}{1-\hat{\mu}_{1}}\hat{f}_{1}(1)+\mu_{1}\tilde{\mu}_{2}\hat{\theta}_{1}^{(2)}\left(1\right)\hat{f}_{1}(1)+
\frac{\mu_{1}}{1-\hat{\mu}_{1}}\hat{f}_{1}(1)F_{1,2}(1)+\frac{\tilde{\mu}_{2}}{1-\hat{\mu}_{1}}\hat{f}_{1}(1)F_{1,1}(1)\\
&+&f_{1}^{(2)}(1,2)+\mu_{1}\tilde{\mu}_{2}\left(\frac{1}{1-\hat{\mu}_{1}}\right)^{2}\hat{f}_{1}(1,1).
\end{eqnarray*}

%51

\item \begin{eqnarray*} \hat{f}_{2}\left(3,1\right)&=&\hat{r}_{1}\mu_{1}\hat{\mu}_{1}+\mu_{1}\hat{\mu}_{1}\hat{R}_{1}^{(2)}\left(1\right)+\hat{r}_{1}\hat{\mu}_{1}F_{1,1}(1)+\hat{r}_{1}\frac{\mu_{1}\hat{\mu}_{1}}{1-\hat{\mu}_{1}}\hat{F}_{1}(1).
\end{eqnarray*}

%52

\item \begin{eqnarray*} \hat{f}_{2}\left(4,1\right)&=&\hat{r}_{1}\mu_{1}\hat{\mu}_{2}+\mu_{1}\hat{\mu}_{2}\hat{R}_{1}^{(2)}\left(1\right)+\hat{r}_{1}\hat{\mu}_{2}F_{1,1}(1)+\frac{\mu_{1}\hat{\mu}_{2}}{1-\hat{\mu}_{1}}\hat{f}_{1}(1)+\hat{r}_{1}\frac{\mu_{1}\hat{\mu}_{2}}{1-\hat{\mu}_{1}}\hat{f}_{1}(1)\\
&+&\mu_{1}\hat{\mu}_{2}\hat{\theta}_{1}^{(2)}\left(1\right)\hat{f}_{1}(1)+\hat{r}_{1}\mu_{1}\left(\hat{f}_{1}(2)+\frac{\hat{\mu}_{2}}{1-\hat{\mu}_{1}}\hat{f}_{1}(1)\right)+F_{1,1}(1)\left(\hat{f}_{1}(2)+\frac{\hat{\mu}_{2}}{1-\hat{\mu}_{1}}\hat{f}_{1}(1)\right)\\
&+&\frac{\mu_{1}}{1-\hat{\mu}_{1}}\left(\hat{f}_{1}(1,2)+\frac{\hat{\mu}_{2}}{1-\hat{\mu}_{1}}\hat{f}_{1}(1,1)\right).
\end{eqnarray*}
%___________________________________________________________________________________________
%\subsubsection{Mixtas para $z_{2}$:}
%___________________________________________________________________________________________
%53

\item \begin{eqnarray*} \hat{f}_{2}\left(1,2\right)&=&\hat{r}_{1}\mu_{1}\tilde{\mu}_{2}+\mu_{1}\tilde{\mu}_{2}\hat{R}_{1}^{(2)}\left(1\right)+\hat{r}_{1}\mu_{1}F_{1,2}(1)+\hat{r}_{1}\tilde{\mu}_{2}F_{1,1}(1)+\frac{\mu_{1}\tilde{\mu}_{2}}{1-\hat{\mu}_{1}}\hat{f}_{1}(1)\\
&+&2\hat{r}_{1}\frac{\mu_{1}\tilde{\mu}_{2}}{1-\hat{\mu}_{1}}\hat{f}_{1}(1)+\mu_{1}\tilde{\mu}_{2}\hat{\theta}_{1}^{(2)}\left(1\right)\hat{f}_{1}(1)+\frac{\mu_{1}}{1-\hat{\mu}_{1}}\hat{f}_{1}(1)F_{1,2}(1)\\
&+&\frac{\tilde{\mu}_{2}}{1-\hat{\mu}_{1}}\hat{f}_{1}(1)F_{1,1}(1)+f_{1}^{(2)}(1,2)+\mu_{1}\tilde{\mu}_{2}\left(\frac{1}{1-\hat{\mu}_{1}}\right)^{2}\hat{f}_{1}(1,1).
\end{eqnarray*}

%54

\item \begin{eqnarray*} \hat{f}_{2}\left(2,2\right)&=&\hat{r}_{1}\tilde{P}_{2}^{(2)}\left(1\right)+\tilde{\mu}_{2}^{2}\hat{R}_{1}^{(2)}\left(1\right)+2\hat{r}_{1}\tilde{\mu}_{2}F_{1,2}(1)+ f_{1,2}^{(2)}(1)+2\hat{r}_{1}\frac{\tilde{\mu}_{2}^{2}}{1-\hat{\mu}_{1}}\hat{f}_{1}(1)\\
&+&\frac{1}{1-\hat{\mu}_{1}}\tilde{P}_{2}^{(2)}\left(1\right)\hat{f}_{1}(1)+\tilde{\mu}_{2}^{2}\hat{\theta}_{1}^{(2)}\left(1\right)\hat{f}_{1}(1)+2\frac{\tilde{\mu}_{2}}{1-\hat{\mu}_{1}}F_{1,2}(1)\hat{f}_{1}(1)+\left(\frac{\tilde{\mu}_{2}}{1-\hat{\mu}_{1}}\right)^{2}\hat{f}_{1}(1,1).
\end{eqnarray*}

%55

\item \begin{eqnarray*} \hat{f}_{2}\left(3,2\right)&=&\hat{r}_{1}\hat{\mu}_{1}\tilde{\mu}_{2}+\hat{\mu}_{1}\tilde{\mu}_{2}\hat{R}_{1}^{(2)}\left(1\right)+
\hat{r}_{1}\hat{\mu}_{1}F_{1,2}(1)+\hat{r}_{1}\frac{\hat{\mu}_{1}\tilde{\mu}_{2}}{1-\hat{\mu}_{1}}\hat{f}_{1}(1).
\end{eqnarray*}

%56

\item \begin{eqnarray*} \hat{f}_{2}\left(4,2\right)&=&\hat{r}_{1}\tilde{\mu}_{2}\hat{\mu}_{2}+\hat{\mu}_{2}\tilde{\mu}_{2}\hat{R}_{1}^{(2)}\left(1\right)+\hat{\mu}_{2}\hat{R}_{1}^{(2)}\left(1\right)F_{1,2}(1)+\frac{\hat{\mu}_{2}\tilde{\mu}_{2}}{1-\hat{\mu}_{1}}\hat{f}_{1}(1)\\
&+&\hat{r}_{1}\frac{\hat{\mu}_{2}\tilde{\mu}_{2}}{1-\hat{\mu}_{1}}\hat{f}_{1}(1)+\hat{\mu}_{2}\tilde{\mu}_{2}\hat{\theta}_{1}^{(2)}\left(1\right)\hat{f}_{1}(1)+\hat{r}_{1}\tilde{\mu}_{2}\left(\hat{f}_{1}(2)+\frac{\hat{\mu}_{2}}{1-\hat{\mu}_{1}}\hat{f}_{1}(1)\right)\\
&+&F_{1,2}(1)\left(\hat{f}_{1}(2)+\frac{\hat{\mu}_{2}}{1-\hat{\mu}_{1}}\hat{f}_{1}(1)\right)+\frac{\tilde{\mu}_{2}}{1-\hat{\mu}_{1}}\left(\hat{f}_{1}(1,2)+\frac{\hat{\mu}_{2}}{1-\hat{\mu}_{1}}\hat{f}_{1}(1,1)\right).
\end{eqnarray*}
%___________________________________________________________________________________________
%\subsubsection{Mixtas para $w_{1}$:}
%___________________________________________________________________________________________

%57


\item \begin{eqnarray*} \hat{f}_{2}\left(1,3\right)&=&\hat{r}_{1}\mu_{1}\hat{\mu}_{1}+\mu_{1}\hat{\mu}_{1}\hat{R}_{1}^{(2)}\left(1\right)+\hat{r}_{1}\hat{\mu}_{1}F_{1,1}(1)+\hat{r}_{1}\frac{\mu_{1}\hat{\mu}_{1}}{1-\hat{\mu}_{1}}\hat{f}_{1}(1).
\end{eqnarray*}

%58

\item \begin{eqnarray*} \hat{f}_{2}\left(2,3\right)&=&\hat{r}_{1}\tilde{\mu}_{2}\hat{\mu}_{1}+\tilde{\mu}_{2}\hat{\mu}_{1}\hat{R}_{1}^{(2)}\left(1\right)+\hat{r}_{1}\hat{\mu}_{1}F_{1,2}(1)+\hat{r}_{1}\frac{\tilde{\mu}_{2}\hat{\mu}_{1}}{1-\hat{\mu}_{1}}\hat{f}_{1}(1).
\end{eqnarray*}

%59

\item \begin{eqnarray*} \hat{f}_{2}\left(3,3\right)&=&\hat{r}_{1}\hat{P}_{1}^{(2)}\left(1\right)+\hat{\mu}_{1}^{2}\hat{R}_{1}^{(2)}\left(1\right).
\end{eqnarray*}

%60

\item \begin{eqnarray*} \hat{f}_{2}\left(4,3\right)&=&\hat{r}_{1}\hat{\mu}_{2}\hat{\mu}_{1}+\hat{\mu}_{2}\hat{\mu}_{1}\hat{R}_{1}^{(2)}\left(1\right)+\hat{r}_{1}\hat{\mu}_{1}\left(\hat{f}_{1}(2)+\frac{\hat{\mu}_{2}}{1-\hat{\mu}_{1}}\hat{f}_{1}(1)\right).
\end{eqnarray*}
%___________________________________________________________________________________________
%\subsubsection{Mixtas para $w_{1}$:}
%___________________________________________________________________________________________
%61

\item \begin{eqnarray*} \hat{f}_{2}\left(1,4\right)&=&\hat{r}_{1}\mu_{1}\hat{\mu}_{2}+\mu_{1}\hat{\mu}_{2}\hat{R}_{1}^{(2)}\left(1\right)+\hat{r}_{1}\hat{\mu}_{2}F_{1,1}(1)+\hat{r}_{1}\frac{\mu_{1}\hat{\mu}_{2}}{1-\hat{\mu}_{1}}\hat{f}_{1}(1)+\hat{r}_{1}\mu_{1}\left(\hat{f}_{1}(2)+\frac{\hat{\mu}_{2}}{1-\hat{\mu}_{1}}\hat{f}_{1}(1)\right)\\
&+&F_{1,1}(1)\left(\hat{f}_{1}(2)+\frac{\hat{\mu}_{2}}{1-\hat{\mu}_{1}}\hat{f}_{1}(1)\right)+\frac{\mu_{1}\hat{\mu}_{2}}{1-\hat{\mu}_{1}}\hat{f}_{1}(1)+\mu_{1}\hat{\mu}_{2}\hat{\theta}_{1}^{(2)}\left(1\right)\hat{f}_{1}(1)\\
&+&\frac{\mu_{1}}{1-\hat{\mu}_{1}}\hat{f}_{1}(1,2)+\mu_{1}\hat{\mu}_{2}\left(\frac{1}{1-\hat{\mu}_{1}}\right)^{2}\hat{f}_{1}(1,1).
\end{eqnarray*}

%62

\item \begin{eqnarray*} \hat{f}_{2}\left(2,4\right)&=&\hat{r}_{1}\tilde{\mu}_{2}\hat{\mu}_{2}+\tilde{\mu}_{2}\hat{\mu}_{2}\hat{R}_{1}^{(2)}\left(1\right)+\hat{r}_{1}\hat{\mu}_{2}F_{1,2}(1)+\hat{r}_{1}\frac{\tilde{\mu}_{2}\hat{\mu}_{2}}{1-\hat{\mu}_{1}}\hat{f}_{1}(1)\\
&+&\hat{r}_{1}\tilde{\mu}_{2}\left(\hat{f}_{1}(2)+\frac{\hat{\mu}_{2}}{1-\hat{\mu}_{1}}\hat{f}_{1}(1)\right)+F_{1,2}(1)\left(\hat{f}_{1}(2)+\frac{\hat{\mu}_{2}}{1-\hat{\mu}_{1}}\hat{F}_{1}^{(1,0)}\right)+\frac{\tilde{\mu}_{2}\hat{\mu}_{2}}{1-\hat{\mu}_{1}}\hat{f}_{1}(1)\\
&+&\tilde{\mu}_{2}\hat{\mu}_{2}\hat{\theta}_{1}^{(2)}\left(1\right)\hat{f}_{1}(1)+\frac{\tilde{\mu}_{2}}{1-\hat{\mu}_{1}}\hat{f}_{1}(1,2)+\tilde{\mu}_{2}\hat{\mu}_{2}\left(\frac{1}{1-\hat{\mu}_{1}}\right)^{2}\hat{f}_{1}(1,1).
\end{eqnarray*}

%63

\item \begin{eqnarray*} \hat{f}_{2}\left(3,4\right)&=&\hat{r}_{1}\hat{\mu}_{2}\hat{\mu}_{1}+\hat{\mu}_{2}\hat{\mu}_{1}\hat{R}_{1}^{(2)}\left(1\right)+\hat{r}_{1}\hat{\mu}_{1}\left(\hat{f}_{1}(2)+\frac{\hat{\mu}_{2}}{1-\hat{\mu}_{1}}\hat{f}_{1}(1)\right).
\end{eqnarray*}

%64

\item \begin{eqnarray*} \hat{f}_{2}\left(4,4\right)&=&\hat{r}_{1}\hat{P}_{2}^{(2)}\left(1\right)+\hat{\mu}_{2}^{2}\hat{R}_{1}^{(2)}\left(1\right)+
2\hat{r}_{1}\hat{\mu}_{2}\left(\hat{f}_{1}(2)+\frac{\hat{\mu}_{2}}{1-\hat{\mu}_{1}}\hat{f}_{1}(1)\right)+\hat{f}_{1}(2,2)\\
&+&\frac{1}{1-\hat{\mu}_{1}}\hat{P}_{2}^{(2)}\left(1\right)\hat{f}_{1}(1)+\hat{\mu}_{2}^{2}\hat{\theta}_{1}^{(2)}\left(1\right)\hat{f}_{1}(1)+\frac{\hat{\mu}_{2}}{1-\hat{\mu}_{1}}\hat{f}_{1}(1,2)\\
&+&\frac{\hat{\mu}_{2}}{1-\hat{\mu}_{1}}\left(\hat{f}_{1}(1,2)+\frac{\hat{\mu}_{2}}{1-\hat{\mu}_{1}}\hat{f}_{1}(1,1)\right).
\end{eqnarray*}
%_________________________________________________________________________________________________________
%
%_________________________________________________________________________________________________________

\end{enumerate}
%___________________________________________________________________________________________
\section{Tiempos de Ciclo e Intervisita}
%___________________________________________________________________________________________


\begin{Def}
Sea $L_{i}^{*}$el n\'umero de usuarios en la cola $Q_{i}$ cuando es visitada por el servidor para dar servicio, entonces

\begin{eqnarray}
\esp\left[L_{i}^{*}\right]&=&f_{i}\left(i\right)\\
Var\left[L_{i}^{*}\right]&=&f_{i}\left(i,i\right)+\esp\left[L_{i}^{*}\right]-\esp\left[L_{i}^{*}\right]^{2}.
\end{eqnarray}

\end{Def}

\begin{Def}
El tiempo de Ciclo $C_{i}$ es e periodo de tiempo que comienza cuando la cola $i$ es visitada por primera vez en un ciclo, y termina cuando es visitado nuevamente en el pr\'oximo ciclo. La duraci\'on del mismo est\'a dada por $\tau_{i}\left(m+1\right)-\tau_{i}\left(m\right)$, o equivalentemente $\overline{\tau}_{i}\left(m+1\right)-\overline{\tau}_{i}\left(m\right)$ bajo condiciones de estabilidad.
\end{Def}

\begin{Def}
El tiempo de intervisita $I_{i}$ es el periodo de tiempo que comienza cuando se ha completado el servicio en un ciclo y termina cuando es visitada nuevamente en el pr\'oximo ciclo. Su  duraci\'on del mismo est\'a dada por $\tau_{i}\left(m+1\right)-\overline{\tau}_{i}\left(m\right)$.
\end{Def}


Recordemos las siguientes expresiones:

\begin{eqnarray*}
S_{i}\left(z\right)&=&\esp\left[z^{\overline{\tau}_{i}\left(m\right)-\tau_{i}\left(m\right)}\right]=F_{i}\left(\theta\left(z\right)\right),\\
F\left(z\right)&=&\esp\left[z^{L_{0}}\right],\\
P\left(z\right)&=&\esp\left[z^{X_{n}}\right],\\
F_{i}\left(z\right)&=&\esp\left[z^{L_{i}\left(\tau_{i}\left(m\right)\right)}\right],
\theta_{i}\left(z\right)-zP_{i}
\end{eqnarray*}

entonces

\begin{eqnarray*}
\esp\left[S_{i}\right]&=&\frac{\esp\left[L_{i}^{*}\right]}{1-\mu_{i}}=\frac{f_{i}\left(i\right)}{1-\mu_{i}},\\
Var\left[S_{i}\right]&=&\frac{Var\left[L_{i}^{*}\right]}{\left(1-\mu_{i}\right)^{2}}+\frac{\sigma^{2}\esp\left[L_{i}^{*}\right]}{\left(1-\mu_{i}\right)^{3}}
\end{eqnarray*}

donde recordemos que

\begin{eqnarray*}
Var\left[L_{i}^{*}\right]&=&f_{i}\left(i,i\right)+f_{i}\left(i\right)-f_{i}\left(i\right)^{2}.
\end{eqnarray*}

La duraci\'on del tiempo de intervisita es $\tau_{i}\left(m+1\right)-\overline{\tau}\left(m\right)$. Dado que el n\'umero de usuarios presentes en $Q_{i}$ al tiempo $t=\tau_{i}\left(m+1\right)$ es igual al n\'umero de arribos durante el intervalo de tiempo $\left[\overline{\tau}\left(m\right),\tau_{i}\left(m+1\right)\right]$ se tiene que


\begin{eqnarray*}
\esp\left[z_{i}^{L_{i}\left(\tau_{i}\left(m+1\right)\right)}\right]=\esp\left[\left\{P_{i}\left(z_{i}\right)\right\}^{\tau_{i}\left(m+1\right)-\overline{\tau}\left(m\right)}\right]
\end{eqnarray*}

entonces, si \begin{eqnarray*}I_{i}\left(z\right)&=&\esp\left[z^{\tau_{i}\left(m+1\right)-\overline{\tau}\left(m\right)}\right]\end{eqnarray*} se tienen que

\begin{eqnarray*}
F_{i}\left(z\right)=I_{i}\left[P_{i}\left(z\right)\right]
\end{eqnarray*}
para $i=1,2$, por tanto



\begin{eqnarray*}
\esp\left[L_{i}^{*}\right]&=&\mu_{i}\esp\left[I_{i}\right]\\
Var\left[L_{i}^{*}\right]&=&\mu_{i}^{2}Var\left[I_{i}\right]+\sigma^{2}\esp\left[I_{i}\right]
\end{eqnarray*}
para $i=1,2$, por tanto


\begin{eqnarray*}
\esp\left[I_{i}\right]&=&\frac{f_{i}\left(i\right)}{\mu_{i}},
\end{eqnarray*}
adem\'as

\begin{eqnarray*}
Var\left[I_{i}\right]&=&\frac{Var\left[L_{i}^{*}\right]}{\mu_{i}^{2}}-\frac{\sigma_{i}^{2}}{\mu_{i}^{2}}f_{i}\left(i\right).
\end{eqnarray*}


Si  $C_{i}\left(z\right)=\esp\left[z^{\overline{\tau}\left(m+1\right)-\overline{\tau}_{i}\left(m\right)}\right]$el tiempo de duraci\'on del ciclo, entonces, por lo hasta ahora establecido, se tiene que

\begin{eqnarray*}
C_{i}\left(z\right)=I_{i}\left[\theta_{i}\left(z\right)\right],
\end{eqnarray*}
entonces

\begin{eqnarray*}
\esp\left[C_{i}\right]&=&\esp\left[I_{i}\right]\esp\left[\theta_{i}\left(z\right)\right]=\frac{\esp\left[L_{i}^{*}\right]}{\mu_{i}}\frac{1}{1-\mu_{i}}=\frac{f_{i}\left(i\right)}{\mu_{i}\left(1-\mu_{i}\right)}\\
Var\left[C_{i}\right]&=&\frac{Var\left[L_{i}^{*}\right]}{\mu_{i}^{2}\left(1-\mu_{i}\right)^{2}}.
\end{eqnarray*}

Por tanto se tienen las siguientes igualdades


\begin{eqnarray*}
\esp\left[S_{i}\right]&=&\mu_{i}\esp\left[C_{i}\right],\\
\esp\left[I_{i}\right]&=&\left(1-\mu_{i}\right)\esp\left[C_{i}\right]\\
\end{eqnarray*}

Def\'inanse los puntos de regenaraci\'on  en el proceso $\left[L_{1}\left(t\right),L_{2}\left(t\right),\ldots,L_{N}\left(t\right)\right]$. Los puntos cuando la cola $i$ es visitada y todos los $L_{j}\left(\tau_{i}\left(m\right)\right)=0$ para $i=1,2$  son puntos de regeneraci\'on. Se llama ciclo regenerativo al intervalo entre dos puntos regenerativos sucesivos.

Sea $M_{i}$  el n\'umero de ciclos de visita en un ciclo regenerativo, y sea $C_{i}^{(m)}$, para $m=1,2,\ldots,M_{i}$ la duraci\'on del $m$-\'esimo ciclo de visita en un ciclo regenerativo. Se define el ciclo del tiempo de visita promedio $\esp\left[C_{i}\right]$ como

\begin{eqnarray*}
\esp\left[C_{i}\right]&=&\frac{\esp\left[\sum_{m=1}^{M_{i}}C_{i}^{(m)}\right]}{\esp\left[M_{i}\right]}
\end{eqnarray*}


En Stid72 y Heym82 se muestra que una condici\'on suficiente para que el proceso regenerativo
estacionario sea un procesoo estacionario es que el valor esperado del tiempo del ciclo regenerativo sea finito:

\begin{eqnarray*}
\esp\left[\sum_{m=1}^{M_{i}}C_{i}^{(m)}\right]<\infty.
\end{eqnarray*}

como cada $C_{i}^{(m)}$ contiene intervalos de r\'eplica positivos, se tiene que $\esp\left[M_{i}\right]<\infty$, adem\'as, como $M_{i}>0$, se tiene que la condici\'on anterior es equivalente a tener que

\begin{eqnarray*}
\esp\left[C_{i}\right]<\infty,
\end{eqnarray*}
por lo tanto una condici\'on suficiente para la existencia del proceso regenerativo est\'a dada por

\begin{eqnarray*}
\sum_{k=1}^{N}\mu_{k}<1.
\end{eqnarray*}

Sea la funci\'on generadora de momentos para $L_{i}$, el n\'umero de usuarios en la cola $Q_{i}\left(z\right)$ en cualquier momento, est\'a dada por el tiempo promedio de $z^{L_{i}\left(t\right)}$ sobre el ciclo regenerativo definido anteriormente:

\begin{eqnarray*}
Q_{i}\left(z\right)&=&\esp\left[z^{L_{i}\left(t\right)}\right]=\frac{\esp\left[\sum_{m=1}^{M_{i}}\sum_{t=\tau_{i}\left(m\right)}^{\tau_{i}\left(m+1\right)-1}z^{L_{i}\left(t\right)}\right]}{\esp\left[\sum_{m=1}^{M_{i}}\tau_{i}\left(m+1\right)-\tau_{i}\left(m\right)\right]}
\end{eqnarray*}

$M_{i}$ es un tiempo de paro en el proceso regenerativo con $\esp\left[M_{i}\right]<\infty$, se sigue del lema de Wald que:


\begin{eqnarray*}
\esp\left[\sum_{m=1}^{M_{i}}\sum_{t=\tau_{i}\left(m\right)}^{\tau_{i}\left(m+1\right)-1}z^{L_{i}\left(t\right)}\right]&=&\esp\left[M_{i}\right]\esp\left[\sum_{t=\tau_{i}\left(m\right)}^{\tau_{i}\left(m+1\right)-1}z^{L_{i}\left(t\right)}\right]\\
\esp\left[\sum_{m=1}^{M_{i}}\tau_{i}\left(m+1\right)-\tau_{i}\left(m\right)\right]&=&\esp\left[M_{i}\right]\esp\left[\tau_{i}\left(m+1\right)-\tau_{i}\left(m\right)\right]
\end{eqnarray*}

por tanto se tiene que


\begin{eqnarray*}
Q_{i}\left(z\right)&=&\frac{\esp\left[\sum_{t=\tau_{i}\left(m\right)}^{\tau_{i}\left(m+1\right)-1}z^{L_{i}\left(t\right)}\right]}{\esp\left[\tau_{i}\left(m+1\right)-\tau_{i}\left(m\right)\right]}
\end{eqnarray*}

observar que el denominador es simplemente la duraci\'on promedio del tiempo del ciclo.


Se puede demostrar (ver Hideaki Takagi 1986) que

\begin{eqnarray*}
\esp\left[\sum_{t=\tau_{i}\left(m\right)}^{\tau_{i}\left(m+1\right)-1}z^{L_{i}\left(t\right)}\right]=z\frac{F_{i}\left(z\right)-1}{z-P_{i}\left(z\right)}
\end{eqnarray*}

Durante el tiempo de intervisita para la cola $i$, $L_{i}\left(t\right)$ solamente se incrementa de manera que el incremento por intervalo de tiempo est\'a dado por la funci\'on generadora de probabilidades de $P_{i}\left(z\right)$, por tanto la suma sobre el tiempo de intervisita puede evaluarse como:

\begin{eqnarray*}
\esp\left[\sum_{t=\tau_{i}\left(m\right)}^{\tau_{i}\left(m+1\right)-1}z^{L_{i}\left(t\right)}\right]&=&\esp\left[\sum_{t=\tau_{i}\left(m\right)}^{\tau_{i}\left(m+1\right)-1}\left\{P_{i}\left(z\right)\right\}^{t-\overline{\tau}_{i}\left(m\right)}\right]=\frac{1-\esp\left[\left\{P_{i}\left(z\right)\right\}^{\tau_{i}\left(m+1\right)-\overline{\tau}_{i}\left(m\right)}\right]}{1-P_{i}\left(z\right)}\\
&=&\frac{1-I_{i}\left[P_{i}\left(z\right)\right]}{1-P_{i}\left(z\right)}
\end{eqnarray*}
por tanto

\begin{eqnarray*}
\esp\left[\sum_{t=\tau_{i}\left(m\right)}^{\tau_{i}\left(m+1\right)-1}z^{L_{i}\left(t\right)}\right]&=&\frac{1-F_{i}\left(z\right)}{1-P_{i}\left(z\right)}
\end{eqnarray*}

Haciendo uso de lo hasta ahora desarrollado se tiene que

\begin{eqnarray*}
Q_{i}\left(z\right)&=&\frac{1}{\esp\left[C_{i}\right]}\cdot\frac{1-F_{i}\left(z\right)}{P_{i}\left(z\right)-z}\cdot\frac{\left(1-z\right)P_{i}\left(z\right)}{1-P_{i}\left(z\right)}\\
&=&\frac{\mu_{i}\left(1-\mu_{i}\right)}{f_{i}\left(i\right)}\cdot\frac{1-F_{i}\left(z\right)}{P_{i}\left(z\right)-z}\cdot\frac{\left(1-z\right)P_{i}\left(z\right)}{1-P_{i}\left(z\right)}
\end{eqnarray*}

derivando con respecto a $z$



\begin{eqnarray*}
\frac{d Q_{i}\left(z\right)}{d z}&=&\frac{\left(1-F_{i}\left(z\right)\right)P_{i}\left(z\right)}{\esp\left[C_{i}\right]\left(1-P_{i}\left(z\right)\right)\left(P_{i}\left(z\right)-z\right)}\\
&-&\frac{\left(1-z\right)P_{i}\left(z\right)F_{i}^{'}\left(z\right)}{\esp\left[C_{i}\right]\left(1-P_{i}\left(z\right)\right)\left(P_{i}\left(z\right)-z\right)}\\
&-&\frac{\left(1-z\right)\left(1-F_{i}\left(z\right)\right)P_{i}\left(z\right)\left(P_{i}^{'}\left(z\right)-1\right)}{\esp\left[C_{i}\right]\left(1-P_{i}\left(z\right)\right)\left(P_{i}\left(z\right)-z\right)^{2}}\\
&+&\frac{\left(1-z\right)\left(1-F_{i}\left(z\right)\right)P_{i}^{'}\left(z\right)}{\esp\left[C_{i}\right]\left(1-P_{i}\left(z\right)\right)\left(P_{i}\left(z\right)-z\right)}\\
&+&\frac{\left(1-z\right)\left(1-F_{i}\left(z\right)\right)P_{i}\left(z\right)P_{i}^{'}\left(z\right)}{\esp\left[C_{i}\right]\left(1-P_{i}\left(z\right)\right)^{2}\left(P_{i}\left(z\right)-z\right)}
\end{eqnarray*}

Calculando el l\'imite cuando $z\rightarrow1^{+}$:
\begin{eqnarray}
Q_{i}^{(1)}\left(z\right)=\lim_{z\rightarrow1^{+}}\frac{d Q_{i}\left(z\right)}{dz}&=&\lim_{z\rightarrow1}\frac{\left(1-F_{i}\left(z\right)\right)P_{i}\left(z\right)}{\esp\left[C_{i}\right]\left(1-P_{i}\left(z\right)\right)\left(P_{i}\left(z\right)-z\right)}\\
&-&\lim_{z\rightarrow1^{+}}\frac{\left(1-z\right)P_{i}\left(z\right)F_{i}^{'}\left(z\right)}{\esp\left[C_{i}\right]\left(1-P_{i}\left(z\right)\right)\left(P_{i}\left(z\right)-z\right)}\\
&-&\lim_{z\rightarrow1^{+}}\frac{\left(1-z\right)\left(1-F_{i}\left(z\right)\right)P_{i}\left(z\right)\left(P_{i}^{'}\left(z\right)-1\right)}{\esp\left[C_{i}\right]\left(1-P_{i}\left(z\right)\right)\left(P_{i}\left(z\right)-z\right)^{2}}\\
&+&\lim_{z\rightarrow1^{+}}\frac{\left(1-z\right)\left(1-F_{i}\left(z\right)\right)P_{i}^{'}\left(z\right)}{\esp\left[C_{i}\right]\left(1-P_{i}\left(z\right)\right)\left(P_{i}\left(z\right)-z\right)}\\
&+&\lim_{z\rightarrow1^{+}}\frac{\left(1-z\right)\left(1-F_{i}\left(z\right)\right)P_{i}\left(z\right)P_{i}^{'}\left(z\right)}{\esp\left[C_{i}\right]\left(1-P_{i}\left(z\right)\right)^{2}\left(P_{i}\left(z\right)-z\right)}
\end{eqnarray}

Entonces:
%______________________________________________________

\begin{eqnarray*}
\lim_{z\rightarrow1^{+}}\frac{\left(1-F_{i}\left(z\right)\right)P_{i}\left(z\right)}{\left(1-P_{i}\left(z\right)\right)\left(P_{i}\left(z\right)-z\right)}&=&\lim_{z\rightarrow1^{+}}\frac{\frac{d}{dz}\left[\left(1-F_{i}\left(z\right)\right)P_{i}\left(z\right)\right]}{\frac{d}{dz}\left[\left(1-P_{i}\left(z\right)\right)\left(-z+P_{i}\left(z\right)\right)\right]}\\
&=&\lim_{z\rightarrow1^{+}}\frac{-P_{i}\left(z\right)F_{i}^{'}\left(z\right)+\left(1-F_{i}\left(z\right)\right)P_{i}^{'}\left(z\right)}{\left(1-P_{i}\left(z\right)\right)\left(-1+P_{i}^{'}\left(z\right)\right)-\left(-z+P_{i}\left(z\right)\right)P_{i}^{'}\left(z\right)}
\end{eqnarray*}


%______________________________________________________


\begin{eqnarray*}
\lim_{z\rightarrow1^{+}}\frac{\left(1-z\right)P_{i}\left(z\right)F_{i}^{'}\left(z\right)}{\left(1-P_{i}\left(z\right)\right)\left(P_{i}\left(z\right)-z\right)}&=&\lim_{z\rightarrow1^{+}}\frac{\frac{d}{dz}\left[\left(1-z\right)P_{i}\left(z\right)F_{i}^{'}\left(z\right)\right]}{\frac{d}{dz}\left[\left(1-P_{i}\left(z\right)\right)\left(P_{i}\left(z\right)-z\right)\right]}\\
&=&\lim_{z\rightarrow1^{+}}\frac{-P_{i}\left(z\right) F_{i}^{'}\left(z\right)+(1-z) F_{i}^{'}\left(z\right) P_{i}^{'}\left(z\right)+(1-z) P_{i}\left(z\right)F_{i}^{''}\left(z\right)}{\left(1-P_{i}\left(z\right)\right)\left(-1+P_{i}^{'}\left(z\right)\right)-\left(-z+P_{i}\left(z\right)\right)P_{i}^{'}\left(z\right)}
\end{eqnarray*}


%______________________________________________________

\begin{eqnarray*}
&&\lim_{z\rightarrow1^{+}}\frac{\left(1-z\right)\left(1-F_{i}\left(z\right)\right)P_{i}\left(z\right)\left(P_{i}^{'}\left(z\right)-1\right)}{\left(1-P_{i}\left(z\right)\right)\left(P_{i}\left(z\right)-z\right)^{2}}=\lim_{z\rightarrow1^{+}}\frac{\frac{d}{dz}\left[\left(1-z\right)\left(1-F_{i}\left(z\right)\right)P_{i}\left(z\right)\left(P_{i}^{'}\left(z\right)-1\right)\right]}{\frac{d}{dz}\left[\left(1-P_{i}\left(z\right)\right)\left(P_{i}\left(z\right)-z\right)^{2}\right]}\\
&=&\lim_{z\rightarrow1^{+}}\frac{-\left(1-F_{i}\left(z\right)\right) P_{i}\left(z\right)\left(-1+P_{i}^{'}\left(z\right)\right)-(1-z) P_{i}\left(z\right)F_{i}^{'}\left(z\right)\left(-1+P_{i}^{'}\left(z\right)\right)}{2\left(1-P_{i}\left(z\right)\right)\left(-z+P_{i}\left(z\right)\right) \left(-1+P_{i}^{'}\left(z\right)\right)-\left(-z+P_{i}\left(z\right)\right)^2 P_{i}^{'}\left(z\right)}\\
&+&\lim_{z\rightarrow1^{+}}\frac{+(1-z) \left(1-F_{i}\left(z\right)\right) \left(-1+P_{i}^{'}\left(z\right)\right) P_{i}^{'}\left(z\right)}{{2\left(1-P_{i}\left(z\right)\right)\left(-z+P_{i}\left(z\right)\right) \left(-1+P_{i}^{'}\left(z\right)\right)-\left(-z+P_{i}\left(z\right)\right)^2 P_{i}^{'}\left(z\right)}}\\
&+&\lim_{z\rightarrow1^{+}}\frac{+(1-z) \left(1-F_{i}\left(z\right)\right) P_{i}\left(z\right)P_{i}^{''}\left(z\right)}{{2\left(1-P_{i}\left(z\right)\right)\left(-z+P_{i}\left(z\right)\right) \left(-1+P_{i}^{'}\left(z\right)\right)-\left(-z+P_{i}\left(z\right)\right)^2 P_{i}^{'}\left(z\right)}}
\end{eqnarray*}











%______________________________________________________
\begin{eqnarray*}
&&\lim_{z\rightarrow1^{+}}\frac{\left(1-z\right)\left(1-F_{i}\left(z\right)\right)P_{i}^{'}\left(z\right)}{\left(1-P_{i}\left(z\right)\right)\left(P_{i}\left(z\right)-z\right)}=\lim_{z\rightarrow1^{+}}\frac{\frac{d}{dz}\left[\left(1-z\right)\left(1-F_{i}\left(z\right)\right)P_{i}^{'}\left(z\right)\right]}{\frac{d}{dz}\left[\left(1-P_{i}\left(z\right)\right)\left(P_{i}\left(z\right)-z\right)\right]}\\
&=&\lim_{z\rightarrow1^{+}}\frac{-\left(1-F_{i}\left(z\right)\right) P_{i}^{'}\left(z\right)-(1-z) F_{i}^{'}\left(z\right) P_{i}^{'}\left(z\right)+(1-z) \left(1-F_{i}\left(z\right)\right) P_{i}^{''}\left(z\right)}{\left(1-P_{i}\left(z\right)\right) \left(-1+P_{i}^{'}\left(z\right)\right)-\left(-z+P_{i}\left(z\right)\right) P_{i}^{'}\left(z\right)}\frac{}{}
\end{eqnarray*}

%______________________________________________________
\begin{eqnarray*}
&&\lim_{z\rightarrow1^{+}}\frac{\left(1-z\right)\left(1-F_{i}\left(z\right)\right)P_{i}\left(z\right)P_{i}^{'}\left(z\right)}{\left(1-P_{i}\left(z\right)\right)^{2}\left(P_{i}\left(z\right)-z\right)}=\lim_{z\rightarrow1^{+}}\frac{\frac{d}{dz}\left[\left(1-z\right)\left(1-F_{i}\left(z\right)\right)P_{i}\left(z\right)P_{i}^{'}\left(z\right)\right]}{\frac{d}{dz}\left[\left(1-P_{i}\left(z\right)\right)^{2}\left(P_{i}\left(z\right)-z\right)\right]}\\
&=&\lim_{z\rightarrow1^{+}}\frac{-\left(1-F_{i}\left(z\right)\right) P_{i}\left(z\right) P_{i}^{'}\left(z\right)-(1-z) P_{i}\left(z\right) F_{i}^{'}\left(z\right)P_i'[z]}{\left(1-P_{i}\left(z\right)\right)^2 \left(-1+P_{i}^{'}\left(z\right)\right)-2 \left(1-P_{i}\left(z\right)\right) \left(-z+P_{i}\left(z\right)\right) P_{i}^{'}\left(z\right)}\\
&+&\lim_{z\rightarrow1^{+}}\frac{(1-z) \left(1-F_{i}\left(z\right)\right) P_{i}^{'}\left(z\right)^2+(1-z) \left(1-F_{i}\left(z\right)\right) P_{i}\left(z\right) P_{i}^{''}\left(z\right)}{\left(1-P_{i}\left(z\right)\right)^2 \left(-1+P_{i}^{'}\left(z\right)\right)-2 \left(1-P_{i}\left(z\right)\right) \left(-z+P_{i}\left(z\right)\right) P_{i}^{'}\left(z\right)}\\
\end{eqnarray*}

%___________________________________________________________________________________________
\subsection{Longitudes de la Cola en cualquier tiempo}
%___________________________________________________________________________________________

Sea
$V_{i}\left(z\right)=\frac{1}{\esp\left[C_{i}\right]}\frac{I_{i}\left(z\right)-1}{z-P_{i}\left(z\right)}$

%{\esp\lef[I_{i}\right]}\frac{1-\mu_{i}}{z-P_{i}\left(z\right)}

\begin{eqnarray*}
\frac{\partial V_{i}\left(z\right)}{\partial z}&=&\frac{1}{\esp\left[C_{i}\right]}\left[\frac{I_{i}{'}\left(z\right)\left(z-P_{i}\left(z\right)\right)}{z-P_{i}\left(z\right)}-\frac{\left(I_{i}\left(z\right)-1\right)\left(1-P_{i}{'}\left(z\right)\right)}{\left(z-P_{i}\left(z\right)\right)^{2}}\right]
\end{eqnarray*}


La FGP para el tiempo de espera para cualquier usuario en la cola est\'a dada por:
\[U_{i}\left(z\right)=\frac{1}{\esp\left[C_{i}\right]}\cdot\frac{1-P_{i}\left(z\right)}{z-P_{i}\left(z\right)}\cdot\frac{I_{i}\left(z\right)-1}{1-z}\]

entonces


\begin{eqnarray*}
\frac{d}{dz}V_{i}\left(z\right)&=&\frac{1}{\esp\left[C_{i}\right]}\left\{\frac{d}{dz}\left(\frac{1-P_{i}\left(z\right)}{z-P_{i}\left(z\right)}\right)\frac{I_{i}\left(z\right)-1}{1-z}+\frac{1-P_{i}\left(z\right)}{z-P_{i}\left(z\right)}\frac{d}{dz}\left(\frac{I_{i}\left(z\right)-1}{1-z}\right)\right\}\\
&=&\frac{1}{\esp\left[C_{i}\right]}\left\{\frac{-P_{i}\left(z\right)\left(z-P_{i}\left(z\right)\right)-\left(1-P_{i}\left(z\right)\right)\left(1-P_{i}^{'}\left(z\right)\right)}{\left(z-P_{i}\left(z\right)\right)^{2}}\cdot\frac{I_{i}\left(z\right)-1}{1-z}\right\}\\
&+&\frac{1}{\esp\left[C_{i}\right]}\left\{\frac{1-P_{i}\left(z\right)}{z-P_{i}\left(z\right)}\cdot\frac{I_{i}^{'}\left(z\right)\left(1-z\right)+\left(I_{i}\left(z\right)-1\right)}{\left(1-z\right)^{2}}\right\}
\end{eqnarray*}
%\frac{I_{i}\left(z\right)-1}{1-z}
%+\frac{1-P_{i}\left(z\right)}{z-P_{i}\frac{d}{dz}\left(\frac{I_{i}\left(z\right)-1}{1-z}\right)


\begin{eqnarray*}
\frac{\partial U_{i}\left(z\right)}{\partial z}&=&\frac{(-1+I_{i}[z]) (1-P_{i}[z])}{(1-z)^2 \esp[I_{i}] (z-P_{i}[z])}+\frac{(1-P_{i}[z]) I_{i}^{'}[z]}{(1-z) \esp[I_{i}] (z-P_{i}[z])}-\frac{(-1+I_{i}[z]) (1-P_{i}[z])\left(1-P{'}[z]\right)}{(1-z) \esp[I_{i}] (z-P_{i}[z])^2}\\
&-&\frac{(-1+I_{i}[z]) P_{i}{'}[z]}{(1-z) \esp[I_{i}](z-P_{i}[z])}
\end{eqnarray*}
%______________________________________________________________________
\section{Procesos de Renovaci\'on}
%______________________________________________________________________

\begin{Def}\label{Def.Tn}
Sean $0\leq T_{1}\leq T_{2}\leq \ldots$ son tiempos aleatorios infinitos en los cuales ocurren ciertos eventos. El n\'umero de tiempos $T_{n}$ en el intervalo $\left[0,t\right)$ es

\begin{eqnarray}
N\left(t\right)=\sum_{n=1}^{\infty}\indora\left(T_{n}\leq t\right),
\end{eqnarray}
para $t\geq0$.
\end{Def}

Si se consideran los puntos $T_{n}$ como elementos de $\rea_{+}$, y $N\left(t\right)$ es el n\'umero de puntos en $\rea$. El proceso denotado por $\left\{N\left(t\right):t\geq0\right\}$, denotado por $N\left(t\right)$, es un proceso puntual en $\rea_{+}$. Los $T_{n}$ son los tiempos de ocurrencia, el proceso puntual $N\left(t\right)$ es simple si su n\'umero de ocurrencias son distintas: $0<T_{1}<T_{2}<\ldots$ casi seguramente.

\begin{Def}
Un proceso puntual $N\left(t\right)$ es un proceso de renovaci\'on si los tiempos de interocurrencia $\xi_{n}=T_{n}-T_{n-1}$, para $n\geq1$, son independientes e identicamente distribuidos con distribuci\'on $F$, donde $F\left(0\right)=0$ y $T_{0}=0$. Los $T_{n}$ son llamados tiempos de renovaci\'on, referente a la independencia o renovaci\'on de la informaci\'on estoc\'astica en estos tiempos. Los $\xi_{n}$ son los tiempos de inter-renovaci\'on, y $N\left(t\right)$ es el n\'umero de renovaciones en el intervalo $\left[0,t\right)$
\end{Def}


\begin{Note}
Para definir un proceso de renovaci\'on para cualquier contexto, solamente hay que especificar una distribuci\'on $F$, con $F\left(0\right)=0$, para los tiempos de inter-renovaci\'on. La funci\'on $F$ en turno degune las otra variables aleatorias. De manera formal, existe un espacio de probabilidad y una sucesi\'on de variables aleatorias $\xi_{1},\xi_{2},\ldots$ definidas en este con distribuci\'on $F$. Entonces las otras cantidades son $T_{n}=\sum_{k=1}^{n}\xi_{k}$ y $N\left(t\right)=\sum_{n=1}^{\infty}\indora\left(T_{n}\leq t\right)$, donde $T_{n}\rightarrow\infty$ casi seguramente por la Ley Fuerte de los Grandes Números.
\end{Note}



%___________________________________________________________________________________________
%
\subsection{Propiedades de los Procesos de Renovaci\'on}
%___________________________________________________________________________________________
%

Los tiempos $T_{n}$ est\'an relacionados con los conteos de $N\left(t\right)$ por

\begin{eqnarray*}
\left\{N\left(t\right)\geq n\right\}&=&\left\{T_{n}\leq t\right\}\\
T_{N\left(t\right)}\leq &t&<T_{N\left(t\right)+1},
\end{eqnarray*}

adem\'as $N\left(T_{n}\right)=n$, y

\begin{eqnarray*}
N\left(t\right)=\max\left\{n:T_{n}\leq t\right\}=\min\left\{n:T_{n+1}>t\right\}
\end{eqnarray*}

Por propiedades de la convoluci\'on se sabe que

\begin{eqnarray*}
P\left\{T_{n}\leq t\right\}=F^{n\star}\left(t\right)
\end{eqnarray*}
que es la $n$-\'esima convoluci\'on de $F$. Entonces

\begin{eqnarray*}
\left\{N\left(t\right)\geq n\right\}&=&\left\{T_{n}\leq t\right\}\\
P\left\{N\left(t\right)\leq n\right\}&=&1-F^{\left(n+1\right)\star}\left(t\right)
\end{eqnarray*}

Adem\'as usando el hecho de que $\esp\left[N\left(t\right)\right]=\sum_{n=1}^{\infty}P\left\{N\left(t\right)\geq n\right\}$
se tiene que

\begin{eqnarray*}
\esp\left[N\left(t\right)\right]=\sum_{n=1}^{\infty}F^{n\star}\left(t\right)
\end{eqnarray*}

\begin{Prop}
Para cada $t\geq0$, la funci\'on generadora de momentos $\esp\left[e^{\alpha N\left(t\right)}\right]$ existe para alguna $\alpha$ en una vecindad del 0, y de aqu\'i que $\esp\left[N\left(t\right)^{m}\right]<\infty$, para $m\geq1$.
\end{Prop}


\begin{Note}
Si el primer tiempo de renovaci\'on $\xi_{1}$ no tiene la misma distribuci\'on que el resto de las $\xi_{n}$, para $n\geq2$, a $N\left(t\right)$ se le llama Proceso de Renovaci\'on retardado, donde si $\xi$ tiene distribuci\'on $G$, entonces el tiempo $T_{n}$ de la $n$-\'esima renovaci\'on tiene distribuci\'on $G\star F^{\left(n-1\right)\star}\left(t\right)$
\end{Note}


\begin{Teo}
Para una constante $\mu\leq\infty$ ( o variable aleatoria), las siguientes expresiones son equivalentes:

\begin{eqnarray}
lim_{n\rightarrow\infty}n^{-1}T_{n}&=&\mu,\textrm{ c.s.}\\
lim_{t\rightarrow\infty}t^{-1}N\left(t\right)&=&1/\mu,\textrm{ c.s.}
\end{eqnarray}
\end{Teo}


Es decir, $T_{n}$ satisface la Ley Fuerte de los Grandes N\'umeros s\'i y s\'olo s\'i $N\left/t\right)$ la cumple.


\begin{Coro}[Ley Fuerte de los Grandes N\'umeros para Procesos de Renovaci\'on]
Si $N\left(t\right)$ es un proceso de renovaci\'on cuyos tiempos de inter-renovaci\'on tienen media $\mu\leq\infty$, entonces
\begin{eqnarray}
t^{-1}N\left(t\right)\rightarrow 1/\mu,\textrm{ c.s. cuando }t\rightarrow\infty.
\end{eqnarray}

\end{Coro}


Considerar el proceso estoc\'astico de valores reales $\left\{Z\left(t\right):t\geq0\right\}$ en el mismo espacio de probabilidad que $N\left(t\right)$

\begin{Def}
Para el proceso $\left\{Z\left(t\right):t\geq0\right\}$ se define la fluctuaci\'on m\'axima de $Z\left(t\right)$ en el intervalo $\left(T_{n-1},T_{n}\right]$:
\begin{eqnarray*}
M_{n}=\sup_{T_{n-1}<t\leq T_{n}}|Z\left(t\right)-Z\left(T_{n-1}\right)|
\end{eqnarray*}
\end{Def}

\begin{Teo}
Sup\'ongase que $n^{-1}T_{n}\rightarrow\mu$ c.s. cuando $n\rightarrow\infty$, donde $\mu\leq\infty$ es una constante o variable aleatoria. Sea $a$ una constante o variable aleatoria que puede ser infinita cuando $\mu$ es finita, y considere las expresiones l\'imite:
\begin{eqnarray}
lim_{n\rightarrow\infty}n^{-1}Z\left(T_{n}\right)&=&a,\textrm{ c.s.}\\
lim_{t\rightarrow\infty}t^{-1}Z\left(t\right)&=&a/\mu,\textrm{ c.s.}
\end{eqnarray}
La segunda expresi\'on implica la primera. Conversamente, la primera implica la segunda si el proceso $Z\left(t\right)$ es creciente, o si $lim_{n\rightarrow\infty}n^{-1}M_{n}=0$ c.s.
\end{Teo}

\begin{Coro}
Si $N\left(t\right)$ es un proceso de renovaci\'on, y $\left(Z\left(T_{n}\right)-Z\left(T_{n-1}\right),M_{n}\right)$, para $n\geq1$, son variables aleatorias independientes e id\'enticamente distribuidas con media finita, entonces,
\begin{eqnarray}
lim_{t\rightarrow\infty}t^{-1}Z\left(t\right)\rightarrow\frac{\esp\left[Z\left(T_{1}\right)-Z\left(T_{0}\right)\right]}{\esp\left[T_{1}\right]},\textrm{ c.s. cuando  }t\rightarrow\infty.
\end{eqnarray}
\end{Coro}

%___________________________________________________________________________________________
%
\subsection{Funci\'on de Renovaci\'on}
%___________________________________________________________________________________________
%


Sup\'ongase que $N\left(t\right)$ es un proceso de renovaci\'on con distribuci\'on $F$ con media finita $\mu$.

\begin{Def}
La funci\'on de renovaci\'on asociada con la distribuci\'on $F$, del proceso $N\left(t\right)$, es
\begin{eqnarray*}
U\left(t\right)=\sum_{n=1}^{\infty}F^{n\star}\left(t\right),\textrm{   }t\geq0,
\end{eqnarray*}
donde $F^{0\star}\left(t\right)=\indora\left(t\geq0\right)$.
\end{Def}


\begin{Prop}
Sup\'ongase que la distribuci\'on de inter-renovaci\'on $F$ tiene densidad $f$. Entonces $U\left(t\right)$ tambi\'en tiene densidad, para $t>0$, y es $U^{'}\left(t\right)=\sum_{n=0}^{\infty}f^{n\star}\left(t\right)$. Adem\'as
\begin{eqnarray*}
\prob\left\{N\left(t\right)>N\left(t-\right)\right\}=0\textrm{,   }t\geq0.
\end{eqnarray*}
\end{Prop}

\begin{Def}
La Transformada de Laplace-Stieljes de $F$ est\'a dada por

\begin{eqnarray*}
\hat{F}\left(\alpha\right)=\int_{\rea_{+}}e^{-\alpha t}dF\left(t\right)\textrm{,  }\alpha\geq0.
\end{eqnarray*}
\end{Def}

Entonces

\begin{eqnarray*}
\hat{U}\left(\alpha\right)=\sum_{n=0}^{\infty}\hat{F^{n\star}}\left(\alpha\right)=\sum_{n=0}^{\infty}\hat{F}\left(\alpha\right)^{n}=\frac{1}{1-\hat{F}\left(\alpha\right)}.
\end{eqnarray*}


\begin{Prop}
La Transformada de Laplace $\hat{U}\left(\alpha\right)$ y $\hat{F}\left(\alpha\right)$ determina una a la otra de manera \'unica por la relaci\'on $\hat{U}\left(\alpha\right)=\frac{1}{1-\hat{F}\left(\alpha\right)}$.
\end{Prop}


\begin{Note}
Un proceso de renovaci\'on $N\left(t\right)$ cuyos tiempos de inter-renovaci\'on tienen media finita, es un proceso Poisson con tasa $\lambda$ si y s\'olo s\'i $\esp\left[U\left(t\right)\right]=\lambda t$, para $t\geq0$.
\end{Note}


\begin{Teo}
Sea $N\left(t\right)$ un proceso puntual simple con puntos de localizaci\'on $T_{n}$ tal que $\eta\left(t\right)=\esp\left[N\left(\right)\right]$ es finita para cada $t$. Entonces para cualquier funci\'on $f:\rea_{+}\rightarrow\rea$,
\begin{eqnarray*}
\esp\left[\sum_{n=1}^{N\left(\right)}f\left(T_{n}\right)\right]=\int_{\left(0,t\right]}f\left(s\right)d\eta\left(s\right)\textrm{,  }t\geq0,
\end{eqnarray*}
suponiendo que la integral exista. Adem\'as si $X_{1},X_{2},\ldots$ son variables aleatorias definidas en el mismo espacio de probabilidad que el proceso $N\left(t\right)$ tal que $\esp\left[X_{n}|T_{n}=s\right]=f\left(s\right)$, independiente de $n$. Entonces
\begin{eqnarray*}
\esp\left[\sum_{n=1}^{N\left(t\right)}X_{n}\right]=\int_{\left(0,t\right]}f\left(s\right)d\eta\left(s\right)\textrm{,  }t\geq0,
\end{eqnarray*}
suponiendo que la integral exista.
\end{Teo}

\begin{Coro}[Identidad de Wald para Renovaciones]
Para el proceso de renovaci\'on $N\left(t\right)$,
\begin{eqnarray*}
\esp\left[T_{N\left(t\right)+1}\right]=\mu\esp\left[N\left(t\right)+1\right]\textrm{,  }t\geq0,
\end{eqnarray*}
\end{Coro}

%___________________________________________________________________________________________
%
\subsection{Funci\'on de Renovaci\'on}
%___________________________________________________________________________________________
%


\begin{Def}
Sea $h\left(t\right)$ funci\'on de valores reales en $\rea$ acotada en intervalos finitos e igual a cero para $t<0$ La ecuaci\'on de renovaci\'on para $h\left(t\right)$ y la distribuci\'on $F$ es

\begin{eqnarray}\label{Ec.Renovacion}
H\left(t\right)=h\left(t\right)+\int_{\left[0,t\right]}H\left(t-s\right)dF\left(s\right)\textrm{,    }t\geq0,
\end{eqnarray}
donde $H\left(t\right)$ es una funci\'on de valores reales. Esto es $H=h+F\star H$. Decimos que $H\left(t\right)$ es soluci\'on de esta ecuaci\'on si satisface la ecuaci\'on, y es acotada en intervalos finitos e iguales a cero para $t<0$.
\end{Def}

\begin{Prop}
La funci\'on $U\star h\left(t\right)$ es la \'unica soluci\'on de la ecuaci\'on de renovaci\'on (\ref{Ec.Renovacion}).
\end{Prop}

\begin{Teo}[Teorema Renovaci\'on Elemental]
\begin{eqnarray*}
t^{-1}U\left(t\right)\rightarrow 1/\mu\textrm{,    cuando }t\rightarrow\infty.
\end{eqnarray*}
\end{Teo}
%___________________________________________________________________________________________
%
\subsection{Teorema Principal de Renovaci\'on}
%___________________________________________________________________________________________
%

\begin{Note} Una funci\'on $h:\rea_{+}\rightarrow\rea$ es Directamente Riemann Integrable en los siguientes casos:
\begin{itemize}
\item[a)] $h\left(t\right)\geq0$ es decreciente y Riemann Integrable.
\item[b)] $h$ es continua excepto posiblemente en un conjunto de Lebesgue de medida 0, y $|h\left(t\right)|\leq b\left(t\right)$, donde $b$ es DRI.
\end{itemize}
\end{Note}

\begin{Teo}[Teorema Principal de Renovaci\'on]
Si $F$ es no aritm\'etica y $h\left(t\right)$ es Directamente Riemann Integrable (DRI), entonces

\begin{eqnarray*}
lim_{t\rightarrow\infty}U\star h=\frac{1}{\mu}\int_{\rea_{+}}h\left(s\right)ds.
\end{eqnarray*}
\end{Teo}

\begin{Prop}
Cualquier funci\'on $H\left(t\right)$ acotada en intervalos finitos y que es 0 para $t<0$ puede expresarse como
\begin{eqnarray*}
H\left(t\right)=U\star h\left(t\right)\textrm{,  donde }h\left(t\right)=H\left(t\right)-F\star H\left(t\right)
\end{eqnarray*}
\end{Prop}

\begin{Def}
Un proceso estoc\'astico $X\left(t\right)$ es crudamente regenerativo en un tiempo aleatorio positivo $T$ si
\begin{eqnarray*}
\esp\left[X\left(T+t\right)|T\right]=\esp\left[X\left(t\right)\right]\textrm{, para }t\geq0,\end{eqnarray*}
y con las esperanzas anteriores finitas.
\end{Def}

\begin{Prop}
Sup\'ongase que $X\left(t\right)$ es un proceso crudamente regenerativo en $T$, que tiene distribuci\'on $F$. Si $\esp\left[X\left(t\right)\right]$ es acotado en intervalos finitos, entonces
\begin{eqnarray*}
\esp\left[X\left(t\right)\right]=U\star h\left(t\right)\textrm{,  donde }h\left(t\right)=\esp\left[X\left(t\right)\indora\left(T>t\right)\right].
\end{eqnarray*}
\end{Prop}

\begin{Teo}[Regeneraci\'on Cruda]
Sup\'ongase que $X\left(t\right)$ es un proceso con valores positivo crudamente regenerativo en $T$, y def\'inase $M=\sup\left\{|X\left(t\right)|:t\leq T\right\}$. Si $T$ es no aritm\'etico y $M$ y $MT$ tienen media finita, entonces
\begin{eqnarray*}
lim_{t\rightarrow\infty}\esp\left[X\left(t\right)\right]=\frac{1}{\mu}\int_{\rea_{+}}h\left(s\right)ds,
\end{eqnarray*}
donde $h\left(t\right)=\esp\left[X\left(t\right)\indora\left(T>t\right)\right]$.
\end{Teo}
%________________________________________________________________________
\section{Procesos Regenerativos}
%________________________________________________________________________

Para $\left\{X\left(t\right):t\geq0\right\}$ Proceso Estoc\'astico a tiempo continuo con estado de espacios $S$, que es un espacio m\'etrico, con trayectorias continuas por la derecha y con l\'imites por la izquierda c.s. Sea $N\left(t\right)$ un proceso de renovaci\'on en $\rea_{+}$ definido en el mismo espacio de probabilidad que $X\left(t\right)$, con tiempos de renovaci\'on $T$ y tiempos de inter-renovaci\'on $\xi_{n}=T_{n}-T_{n-1}$, con misma distribuci\'on $F$ de media finita $\mu$.



\begin{Def}
Para el proceso $\left\{\left(N\left(t\right),X\left(t\right)\right):t\geq0\right\}$, sus trayectoria muestrales en el intervalo de tiempo $\left[T_{n-1},T_{n}\right)$ est\'an descritas por
\begin{eqnarray*}
\zeta_{n}=\left(\xi_{n},\left\{X\left(T_{n-1}+t\right):0\leq t<\xi_{n}\right\}\right)
\end{eqnarray*}
Este $\zeta_{n}$ es el $n$-\'esimo segmento del proceso. El proceso es regenerativo sobre los tiempos $T_{n}$ si sus segmentos $\zeta_{n}$ son independientes e id\'enticamennte distribuidos.
\end{Def}


\begin{Obs}
Si $\tilde{X}\left(t\right)$ con espacio de estados $\tilde{S}$ es regenerativo sobre $T_{n}$, entonces $X\left(t\right)=f\left(\tilde{X}\left(t\right)\right)$ tambi\'en es regenerativo sobre $T_{n}$, para cualquier funci\'on $f:\tilde{S}\rightarrow S$.
\end{Obs}

\begin{Obs}
Los procesos regenerativos son crudamente regenerativos, pero no al rev\'es.
\end{Obs}

\begin{Def}[Definici\'on Cl\'asica]
Un proceso estoc\'astico $X=\left\{X\left(t\right):t\geq0\right\}$ es llamado regenerativo is existe una variable aleatoria $R_{1}>0$ tal que
\begin{itemize}
\item[i)] $\left\{X\left(t+R_{1}\right):t\geq0\right\}$ es independiente de $\left\{\left\{X\left(t\right):t<R_{1}\right\},\right\}$
\item[ii)] $\left\{X\left(t+R_{1}\right):t\geq0\right\}$ es estoc\'asticamente equivalente a $\left\{X\left(t\right):t>0\right\}$
\end{itemize}

Llamamos a $R_{1}$ tiempo de regeneraci\'on, y decimos que $X$ se regenera en este punto.
\end{Def}

$\left\{X\left(t+R_{1}\right)\right\}$ es regenerativo con tiempo de regeneraci\'on $R_{2}$, independiente de $R_{1}$ pero con la misma distribuci\'on que $R_{1}$. Procediendo de esta manera se obtiene una secuencia de variables aleatorias independientes e id\'enticamente distribuidas $\left\{R_{n}\right\}$ llamados longitudes de ciclo. Si definimos a $Z_{k}\equiv R_{1}+R_{2}+\cdots+R_{k}$, se tiene un proceso de renovaci\'on llamado proceso de renovaci\'on encajado para $X$.

\begin{Note}
Un proceso regenerativo con media de la longitud de ciclo finita es llamado positivo recurrente.
\end{Note}


\begin{Def}
Para $x$ fijo y para cada $t\geq0$, sea $I_{x}\left(t\right)=1$ si $X\left(t\right)\leq x$,  $I_{x}\left(t\right)=0$ en caso contrario, y def\'inanse los tiempos promedio
\begin{eqnarray*}
\overline{X}&=&lim_{t\rightarrow\infty}\frac{1}{t}\int_{0}^{\infty}X\left(u\right)du\\
\prob\left(X_{\infty}\leq x\right)&=&lim_{t\rightarrow\infty}\frac{1}{t}\int_{0}^{\infty}I_{x}\left(u\right)du,
\end{eqnarray*}
cuando estos l\'imites existan.
\end{Def}

Como consecuencia del teorema de Renovaci\'on-Recompensa, se tiene que el primer l\'imite  existe y es igual a la constante
\begin{eqnarray*}
\overline{X}&=&\frac{\esp\left[\int_{0}^{R_{1}}X\left(t\right)dt\right]}{\esp\left[R_{1}\right]},
\end{eqnarray*}
suponiendo que ambas esperanzas son finitas.

\begin{Note}
\begin{itemize}
\item[a)] Si el proceso regenerativo $X$ es positivo recurrente y tiene trayectorias muestrales no negativas, entonces la ecuaci\'on anterior es v\'alida.
\item[b)] Si $X$ es positivo recurrente regenerativo, podemos construir una \'unica versi\'on estacionaria de este proceso, $X_{e}=\left\{X_{e}\left(t\right)\right\}$, donde $X_{e}$ es un proceso estoc\'astico regenerativo y estrictamente estacionario, con distribuci\'on marginal distribuida como $X_{\infty}$
\end{itemize}
\end{Note}

%__________________________________________________________________________________________
\subsection{Procesos Regenerativos Estacionarios}
%__________________________________________________________________________________________


Un proceso estoc\'astico a tiempo continuo $\left\{V\left(t\right),t\geq0\right\}$ es un proceso regenerativo si existe una sucesi\'on de variables aleatorias independientes e id\'enticamente distribuidas $\left\{X_{1},X_{2},\ldots\right\}$, sucesi\'on de renovaci\'on, tal que para cualquier conjunto de Borel $A$,

\begin{eqnarray*}
\prob\left\{V\left(t\right)\in A|X_{1}+X_{2}+\cdots+X_{R\left(t\right)}=s,\left\{V\left(\tau\right),\tau<s\right\}\right\}=\prob\left\{V\left(t-s\right)\in A|X_{1}>t-s\right\},
\end{eqnarray*}
para todo $0\leq s\leq t$, donde $R\left(t\right)=\max\left\{X_{1}+X_{2}+\cdots+X_{j}\leq t\right\}=$n\'umero de renovaciones que ocurren en $\left[0,t\right]$.

Sea $X=X_{1}$ y sea $F$ la funci\'on de distrbuci\'on de $X$


\begin{Def}
Se define el proceso estacionario, $\left\{V^{*}\left(t\right),t\geq0\right\}$, para $\left\{V\left(t\right),t\geq0\right\}$ por

\begin{eqnarray*}
\prob\left\{V\left(t\right)\in A\right\}=\frac{1}{\esp\left[X\right]}\int_{0}^{\infty}\prob\left\{V\left(t+x\right)\in A|X>x\right\}\left(1-F\left(x\right)\right)dx,
\end{eqnarray*}
para todo $t\geq0$ y todo conjunto de Borel $A$.
\end{Def}

\begin{Def}
Una modificaci\'on medible de un proceso $\left\{V\left(t\right),t\geq0\right\}$, es una versi\'on de este, $\left\{V\left(t,w\right)\right\}$ conjuntamente medible para $t\geq0$ y para $w\in S$, $S$ espacio de estados para $\left\{V\left(t\right),t\geq0\right\}$.
\end{Def}

\begin{Teo}
Sea $\left\{V\left(t\right),t\geq\right\}$ un proceso regenerativo no negativo con modificaci\'on medible. Sea $\esp\left[X\right]<\infty$. Entonces el proceso estacionario dado por la ecuaci\'on anterior est\'a bien definido y tiene funci\'on de distribuci\'on independiente de $t$, adem\'as
\begin{itemize}
\item[i)] \begin{eqnarray*}
\esp\left[V^{*}\left(0\right)\right]&=&\frac{\esp\left[\int_{0}^{X}V\left(s\right)ds\right]}{\esp\left[X\right]}\end{eqnarray*}
\item[ii)] Si $\esp\left[V^{*}\left(0\right)\right]<\infty$, equivalentemente, si $\esp\left[\int_{0}^{X}V\left(s\right)ds\right]<\infty$,entonces
\begin{eqnarray*}
\frac{\int_{0}^{t}V\left(s\right)ds}{t}\rightarrow\frac{\esp\left[\int_{0}^{X}V\left(s\right)ds\right]}{\esp\left[X\right]}
\end{eqnarray*}
con probabilidad 1 y en media, cuando $t\rightarrow\infty$.
\end{itemize}
\end{Teo}

%_______________________________________________________________________________________________________
\section{Tiempo de Ciclo Promedio}
%_______________________________________________________________________________________________________

Consideremos una cola de la red de sistemas de visitas c\'iclicas fija, $Q_{l}$.


Conforme a la definici\'on dada al principio del cap\'itulo, definici\'on (\ref{Def.Tn}), sean $T_{1},T_{2},\ldots$ los puntos donde las longitudes de las colas de la red de sistemas de visitas c\'iclicas son cero simult\'aneamente, cuando la cola $Q_{l}$ es visitada por el servidor para dar servicio, es decir, $L_{1}\left(T_{i}\right)=0,L_{2}\left(T_{i}\right)=0,\hat{L}_{1}\left(T_{i}\right)=0$ y $\hat{L}_{2}\left(T_{i}\right)=0$, a estos puntos se les denominar\'a puntos regenerativos. Entonces,

\begin{Def}
Al intervalo de tiempo entre dos puntos regenerativos se le llamar\'a ciclo regenerativo.
\end{Def}

\begin{Def}
Para $T_{i}$ se define, $M_{i}$, el n\'umero de ciclos de visita a la cola $Q_{l}$, durante el ciclo regenerativo, es decir, $M_{i}$ es un proceso de renovaci\'on.
\end{Def}

\begin{Def}
Para cada uno de los $M_{i}$'s, se definen a su vez la duraci\'on de cada uno de estos ciclos de visita en el ciclo regenerativo, $C_{i}^{(m)}$, para $m=1,2,\ldots,M_{i}$, que a su vez, tambi\'en es n proceso de renovaci\'on.
\end{Def}

En nuestra notaci\'on $V\left(t\right)\equiv C_{i}$ y $X_{i}=C_{i}^{(m)}$ para nuestra segunda definici\'on, mientras que para la primera la notaci\'on es: $X\left(t\right)\equiv C_{i}$ y $R_{i}\equiv C_{i}^{(m)}$.


%___________________________________________________________________________________________
%
\subsubsection{Expresion de las Parciales mixtas para $F_{1}$ y $F_{2}$}
%___________________________________________________________________________________________
\begin{enumerate}

%1

\item \begin{eqnarray*}
\frac{\partial}{\partial z_{1}}\frac{\partial}{\partial z_{1}}F_{1}\left(\theta_{1}\left(\tilde{P}_{2}\left(z_{2}\right)\hat{P}_{1}\left(w_{1}\right)
\hat{P}_{2}\left(w_{2}\right),z_{2}\right)\right)|_{\mathbf{z,w}=1}&=&0\\
\end{eqnarray*}

%2

\item
\begin{eqnarray*}
\frac{\partial}{\partial z_{2}}\frac{\partial}{\partial z_{1}}F_{1}\left(\theta_{1}\left(\tilde{P}_{2}\left(z_{2}\right)\hat{P}_{1}\left(w_{1}\right)
\hat{P}_{2}\left(w_{2}\right),z_{2}\right)\right)|_{\mathbf{z,w}=1}&=&0\\
\end{eqnarray*}

%3

\item
\begin{eqnarray*}
\frac{\partial}{\partial w_{1}}\frac{\partial}{\partial z_{1}}F_{1}\left(\theta_{1}\left(\tilde{P}_{2}\left(z_{2}\right)\hat{P}_{1}\left(w_{1}\right)
\hat{P}_{2}\left(w_{2}\right),z_{2}\right)\right)|_{\mathbf{z,w}=1}&=&0\\
\end{eqnarray*}

%4

\item
\begin{eqnarray*}
\frac{\partial}{\partial w_{2}}\frac{\partial}{\partial z_{1}}F_{1}\left(\theta_{1}\left(\tilde{P}_{2}\left(z_{2}\right)\hat{P}_{1}\left(w_{1}\right)
\hat{P}_{2}\left(w_{2}\right),z_{2}\right)\right)|_{\mathbf{z,w}=1}&=&0
\end{eqnarray*}

%5

\item
\begin{eqnarray*}
\frac{\partial}{\partial z_{1}}\frac{\partial}{\partial z_{2}}F_{1}\left(\theta_{1}\left(\tilde{P}_{2}\left(z_{2}\right)\hat{P}_{1}\left(w_{1}\right)
\hat{P}_{2}\left(w_{2}\right),z_{2}\right)\right)|_{\mathbf{z,w}=1}&=&0
\end{eqnarray*}

%6

\item
\begin{eqnarray*}
&&\frac{\partial}{\partial z_{2}}\frac{\partial}{\partial z_{2}}F_{1}\left(\theta_{1}\left(\tilde{P}_{2}\left(z_{2}\right)\hat{P}_{1}\left(w_{1}\right)
\hat{P}_{2}\left(w_{2}\right)\right),z_{2}\right)|_{\mathbf{z,w}=1}=f_{1}\left(2,2\right)+\frac{1}{1-\mu_{1}}\tilde{P}_{2}^{(2)}\left(1\right)f_{1}\left(1\right)\\
&+&\tilde{\mu}_{2}^{2}\theta_{1}^{(2)}\left(1\right)f_{1}\left(1\right)+2\frac{\tilde{\mu}_{2}}{1-\mu_{1}}f_{1}\left(1,2\right)+\left(\frac{\tilde{\mu}_{2}}{1-\mu_{1}}\right)^{2}f_{1}\left(1,1\right)
\end{eqnarray*}

%7

\item
\begin{eqnarray*}
&&\frac{\partial}{\partial w_{1}}\frac{\partial}{\partial z_{2}}F_{1}\left(\theta_{1}\left(\tilde{P}_{2}\left(z_{2}\right)\hat{P}_{1}\left(w_{1}\right)
\hat{P}_{2}\left(w_{2}\right),z_{2}\right)\right)|_{\mathbf{z,w}=1}=\frac{\tilde{\mu}_{2}\hat{\mu}_{1}}{1-\mu_{1}}f_{1}\left(1\right)\\
&+&\tilde{\mu}_{2}\hat{\mu}_{1}\theta_{1}^{(2)}\left(1\right)f_{1}\left(1\right)+\frac{\hat{\mu}_{1}}{1-\mu_{1}}f_{1}\left(1,2\right)+\tilde{\mu}_{2}\hat{\mu}_{1}\left(\frac{1}{1-\mu_{1}}\right)^{2}f_{1}\left(1,1\right)
\end{eqnarray*}

%8

\item \begin{eqnarray*}
&&\frac{\partial}{\partial w_{2}}\frac{\partial}{\partial z_{2}}F_{1}\left(\theta_{1}\left(\tilde{P}_{2}\left(z_{2}\right)\hat{P}_{1}\left(w_{1}\right)
\hat{P}_{2}\left(w_{2}\right),z_{2}\right)\right)|_{\mathbf{z,w}=1}=\frac{\tilde{\mu}_{2}\hat{\mu}_{2}}{1-\mu_{1}}f_{1}\left(1\right)\\
&+&\tilde{\mu}_{2}\hat{\mu}_{2}\theta_{1}^{(2)}\left(1\right)f_{1}\left(1\right)+\frac{\hat{\mu}_{2}}{1-\mu_{1}}f_{1}\left(1,2\right)+\tilde{\mu}_{2}\hat{\mu}_{2}\left(\frac{1}{1-\mu_{1}}\right)^{2}f_{1}\left(1,1\right)
\end{eqnarray*}

%9

\item \begin{eqnarray*}
\frac{\partial}{\partial z_{1}}\frac{\partial}{\partial w_{1}}F_{1}\left(\theta_{1}\left(\tilde{P}_{2}\left(z_{2}\right)\hat{P}_{1}\left(w_{1}\right)
\hat{P}_{2}\left(w_{2}\right),z_{2}\right)\right)|_{\mathbf{z,w}=1}&=&0
\end{eqnarray*}

%10

\item \begin{eqnarray*}
&&\frac{\partial}{\partial z_{2}}\frac{\partial}{\partial w_{1}}F_{1}\left(\theta_{1}\left(\tilde{P}_{2}\left(z_{2}\right)\hat{P}_{1}\left(w_{1}\right)
\hat{P}_{2}\left(w_{2}\right),z_{2}\right)\right)|_{\mathbf{z,w}=1}=\frac{\tilde{\mu}_{2}\hat{\mu}_{1}}{1-\mu_{1}}f_{1}\left(2\right)\\
&+&\tilde{\mu}_{2}\hat{\mu}_{1}\theta_{1}^{(2)}\left(1\right)f_{1}\left(2\right)+\frac{\hat{\mu}_{1}}{1-\mu_{1}}f_{1}\left(2,1\right)+\tilde{\mu}_{2}\hat{\mu}_{1}\left(\frac{1}{1-\mu_{1}}\right)^{2}f_{1}\left(1,1\right)
\end{eqnarray*}

%11

\item
\begin{eqnarray*}
&&\frac{\partial}{\partial w_{1}}\frac{\partial}{\partial w_{1}}F_{1}\left(\theta_{1}\left(\tilde{P}_{2}\left(z_{2}\right)\hat{P}_{1}\left(w_{1}\right)
\hat{P}_{2}\left(w_{2}\right),z_{2}\right)\right)|_{\mathbf{z,w}=1}=\frac{1}{1-\mu_{1}} \hat{P}_{1}^{(2)}\left(1\right)f_{1}\left(1\right)\\
&+&\hat{\mu}_{1}\theta_{1}^{(2)}\left(1\right)f_{1}\left(1\right)+\left(\frac{\hat{\mu}_{1}}{1-\mu_{1}}\right)^{2}f_{1}\left(1,1\right)
\end{eqnarray*}

%12

\item
\begin{eqnarray*}
&&\frac{\partial}{\partial w_{2}}\frac{\partial}{\partial w_{1}}F_{1}\left(\theta_{1}\left(\tilde{P}_{2}\left(z_{2}\right)\hat{P}_{1}\left(w_{1}\right)
\hat{P}_{2}\left(w_{2}\right),z_{2}\right)\right)|_{\mathbf{z,w}=1}=\hat{\mu}_{1}\hat{\mu}_{2}f_{1}\left(1\right)\\
&+&\frac{\hat{\mu}_{1}\hat{\mu}_{2}}{1-\mu_{1}}f_{1}\left(1\right)+\hat{\mu}_{1}\hat{\mu}_{2}\theta_{1}^{(2)}\left(1\right)f_{1}\left(1\right)+\hat{\mu}_{1}\hat{\mu}_{2}\left(\frac{1}{1-\mu_{1}}\right)^{2}f_{1}\left(1,1\right)
\end{eqnarray*}

%13

\item \begin{eqnarray*}
\frac{\partial}{\partial z_{1}}\frac{\partial}{\partial w_{2}}F_{1}\left(\theta_{1}\left(\tilde{P}_{2}\left(z_{2}\right)\hat{P}_{1}\left(w_{1}\right)
\hat{P}_{2}\left(w_{2}\right),z_{2}\right)\right)|_{\mathbf{z,w}=1}&=&0
\end{eqnarray*}

%14

\item \begin{eqnarray*}
&&\frac{\partial}{\partial z_{2}}\frac{\partial}{\partial w_{2}}F_{1}\left(\theta_{1}\left(\tilde{P}_{2}\left(z_{2}\right)\hat{P}_{1}\left(w_{1}\right)
\hat{P}_{2}\left(w_{2}\right),z_{2}\right)\right)|_{\mathbf{z,w}=1}=\frac{\tilde{\mu}_{2}\hat{\mu}_{2}}{1-\mu_{1}}f_{1}\left(1\right)\\
&+&\tilde{\mu}_{2}\hat{\mu}_{2}\theta_{1}^{(2)}\left(1\right)f_{1}\left(1\right)+\frac{\hat{\mu}_{2}}{1-\mu_{1}}f_{1}\left(2,1\right)+\tilde{\mu}_{2}\hat{\mu}_{2}\left(\frac{1}{1-\mu_{1}}\right)^{2}f_{1}\left(2,2\right)
\end{eqnarray*}

%15

\item \begin{eqnarray*}
&&\frac{\partial}{\partial w_{1}}\frac{\partial}{\partial w_{2}}F_{1}\left(\theta_{1}\left(\tilde{P}_{2}\left(z_{2}\right)\hat{P}_{1}\left(w_{1}\right)
\hat{P}_{2}\left(w_{2}\right),z_{2}\right)\right)|_{\mathbf{z,w}=1}=\frac{\hat{\mu}_{1}\hat{\mu}_{2}}{1-\mu_{1}}f_{1}\left(1\right)\\
&+&\hat{\mu}_{1}\hat{\mu}_{2}\theta_{1}^{(2)}\left(1\right)f_{1}\left(1\right)+\hat{\mu}_{1}\hat{\mu}_{2}\left(\frac{1}{1-\mu_{1}}\right)^{2}f_{1}\left(1,1\right)
\end{eqnarray*}

%16

\item
\begin{eqnarray*}
&&\frac{\partial}{\partial w_{2}}\frac{\partial}{\partial w_{2}}F_{1}\left(\theta_{1}\left(\tilde{P}_{2}\left(z_{2}\right)\hat{P}_{1}\left(w_{1}\right)
\hat{P}_{2}\left(w_{2}\right),z_{2}\right)\right)|_{\mathbf{z,w}=1}=\frac{1}{1-\mu_{1}}\hat{P}_{2}^{(2)}\left(w_{2}\right)f_{1}\left(1\right)\\
&+&\hat{\mu}_{2}^{2}\theta_{1}^{(2)}\left(1\right)f_{1}\left(1\right)+\left(\hat{\mu}_{2}\frac{1}{1-\mu_{1}}\right)^{2}f_{1}\left(1,1\right)
\end{eqnarray*}

%17

\item
\begin{eqnarray*}
&&\frac{\partial}{\partial z_{1}}\frac{\partial}{\partial z_{1}}F_{2}\left(z_{1},\tilde{\theta}_{2}\left(P_{1}\left(z_{1}\right)\hat{P}_{1}\left(w_{1}\right)
\hat{P}_{2}\left(w_{2}\right)\right)\right)|_{\mathbf{z,w}=1}=\frac{1}{1-\tilde{\mu}_{2}}P_{1}^{(2)}\left(1\right)
f_{2}\left(2\right)+f_{2}\left(1,1\right)\\
&+&\mu_{1}^{2}\tilde{\theta}_{2}^{(2)}\left(1\right)f_{2}\left(2\right)+\mu_{1}\frac{1}{1-\tilde{\mu}_{2}}f_{2}\left(1,2\right)+\left(\mu_{1}\frac{1}{1-\tilde{\mu}_{2}}\right)^{2}f_{2}\left(2,2\right)+\frac{\mu_{1}}{1-\tilde{\mu}_{2}}f_{2}\left(1,2\right)\\
\end{eqnarray*}

%18

\item \begin{eqnarray*}
\frac{\partial}{\partial z_{2}}\frac{\partial}{\partial z_{1}}F_{2}\left(z_{1},\tilde{\theta}_{2}\left(P_{1}\left(z_{1}\right)\hat{P}_{1}\left(w_{1}\right)
\hat{P}_{2}\left(w_{2}\right)\right)\right)|_{\mathbf{z,w}=1}&=&0
\end{eqnarray*}

%19

\item \begin{eqnarray*}
&&\frac{\partial}{\partial w_{1}}\frac{\partial}{\partial z_{1}}F_{2}\left(z_{1},\tilde{\theta}_{2}\left(P_{1}\left(z_{1}\right)\hat{P}_{1}\left(w_{1}\right)
\hat{P}_{2}\left(w_{2}\right)\right)\right)|_{\mathbf{z,w}=1}=\frac{\mu_{1}\hat{\mu}_{1}}{1-\tilde{\mu}_{2}}f_{2}\left(2\right)\\
&+&\mu_{1}\hat{\mu}_{1}\tilde{\theta}_{2}^{(2)}\left(1\right)f_{2}\left(2\right)+\mu_{1}\hat{\mu}_{1}\left(\frac{1}{1-\tilde{\mu}_{2}}\right)^{2}f_{2}\left(2,2\right)+\frac{\hat{\mu}_{1}}{1-\tilde{\mu}_{2}}f_{2}\left(1,2\right)\end{eqnarray*}

%20

\item \begin{eqnarray*}
&&\frac{\partial}{\partial w_{2}}\frac{\partial}{\partial z_{1}}F_{2}\left(z_{1},\tilde{\theta}_{2}\left(P_{1}\left(z_{1}\right)\hat{P}_{1}\left(w_{1}\right)
\hat{P}_{2}\left(w_{2}\right)\right)\right)|_{\mathbf{z,w}=1}=\frac{\mu_{1}\hat{\mu}_{2}}{1-\tilde{\mu}_{2}}f_{2}\left(2\right)\\
&+&\mu_{1}\hat{\mu}_{2}\tilde{\theta}_{2}^{(2)}\left(1\right)f_{2}\left(2\right)+\mu_{1}\hat{\mu}_{2}
\left(\frac{1}{1-\tilde{\mu}_{2}}\right)^{2}f_{2}\left(2,2\right)+\frac{\hat{\mu}_{2}}{1-\tilde{\mu}_{2}}f_{2}\left(1,2\right)\end{eqnarray*}
%___________________________________________________________________________________________


%\newpage

%___________________________________________________________________________________________
%
%\section{Parciales mixtas de $F_{2}$ para $z_{2}$}
%___________________________________________________________________________________________
%___________________________________________________________________________________________
\item
\begin{eqnarray*}
\frac{\partial}{\partial z_{1}}\frac{\partial}{\partial z_{2}}F_{2}\left(z_{1},\tilde{\theta}_{2}\left(P_{1}\left(z_{1}\right)\hat{P}_{1}\left(w_{1}\right)
\hat{P}_{2}\left(w_{2}\right)\right)\right)|_{\mathbf{z,w}=1}&=&0;\\
\end{eqnarray*}
\item
\begin{eqnarray*}
\frac{\partial}{\partial z_{2}}\frac{\partial}{\partial z_{2}}F_{2}\left(z_{1},\tilde{\theta}_{2}\left(P_{1}\left(z_{1}\right)\hat{P}_{1}\left(w_{1}\right)
\hat{P}_{2}\left(w_{2}\right)\right)\right)|_{\mathbf{z,w}=1}&=&0\\
\end{eqnarray*}
\item
\begin{eqnarray*}\frac{\partial}{\partial w_{1}}\frac{\partial}{\partial z_{2}}F_{2}\left(z_{1},\tilde{\theta}_{2}\left(P_{1}\left(z_{1}\right)\hat{P}_{1}\left(w_{1}\right)
\hat{P}_{2}\left(w_{2}\right)\right)\right)|_{\mathbf{z,w}=1}&=&0\\
\end{eqnarray*}
\item
\begin{eqnarray*}\frac{\partial}{\partial w_{2}}\frac{\partial}{\partial z_{2}}F_{2}\left(z_{1},\tilde{\theta}_{2}\left(P_{1}\left(z_{1}\right)\hat{P}_{1}\left(w_{1}\right)
\hat{P}_{2}\left(w_{2}\right)\right)\right)|_{\mathbf{z,w}=1}&=&0
\end{eqnarray*}
%___________________________________________________________________________________________

%\newpage

%___________________________________________________________________________________________
%
%\section{Parciales mixtas de $F_{2}$ para $w_{1}$}
%___________________________________________________________________________________________
\item
\begin{eqnarray*}
\frac{\partial}{\partial z_{1}}\frac{\partial}{\partial w_{1}}F_{2}\left(z_{1},\tilde{\theta}_{2}\left(P_{1}\left(z_{1}\right)\hat{P}_{1}\left(w_{1}\right)
\hat{P}_{2}\left(w_{2}\right)\right)\right)|_{\mathbf{z,w}=1}&=&\frac{1}{1-\tilde{\mu}_{2}}P_{1}^{(2)}\left(1\right)\frac{\partial}{\partial
z_{2}}F_{2}\left(1,1\right)+\mu_{1}^{2}\tilde{\theta}_{2}^{(2)}\left(1\right)\frac{\partial}{\partial
z_{2}}F_{2}\left(1,1\right)\\
&+&\mu_{1}\frac{1}{1-\tilde{\mu}_{2}}f_{2}\left(1,2\right)+\left(\mu_{1}\frac{1}{1-\tilde{\mu}_{2}}\right)^{2}f_{2}\left(2,2\right)\\
&+&\mu_{1}\frac{1}{1-\tilde{\mu}_{2}}f_{2}\left(1,2\right)+f_{2}\left(1,1\right)
\end{eqnarray*}
%___________________________________________________________________________________________
%___________________________________________________________________________________________
\item \begin{eqnarray*}
\frac{\partial}{\partial z_{2}}\frac{\partial}{\partial w_{1}}F_{2}\left(z_{1},\tilde{\theta}_{2}\left(P_{1}\left(z_{1}\right)\hat{P}_{1}\left(w_{1}\right)
\hat{P}_{2}\left(w_{2}\right)\right)\right)|_{\mathbf{z,w}=1}&=&0
\end{eqnarray*}
%___________________________________________________________________________________________
\item
\begin{eqnarray*}
\frac{\partial}{\partial w_{1}}\frac{\partial}{\partial w_{1}}F_{2}\left(z_{1},\tilde{\theta}_{2}\left(P_{1}\left(z_{1}\right)\hat{P}_{1}\left(w_{1}\right)
\hat{P}_{2}\left(w_{2}\right)\right)\right)|_{\mathbf{z,w}=1}&=&\mu_{1}\hat{\mu}_{1}\frac{1}{1-\tilde{\mu}_{2}}\frac{\partial}{\partial
z_{2}}F_{2}\left(1,1\right)+\mu_{1}\hat{\mu}_{1}\left(\frac{1}{1-\tilde{\mu}_{2}}\right)^{2}\frac{\partial}{\partial
z_{2}}F_{2}\left(1,1\right)\\
&+&\mu_{1}\hat{\mu}_{1}
\left(\frac{1}{1-\tilde{\mu}_{2}}\right)^{2}\frac{\partial}{\partial
z_{2}}F_{2}\left(1,1\right)+\hat{\mu}_{1}\frac{1}{1-\tilde{\mu}_{2}}f_{2}\left(1,2\right)\end{eqnarray*}
\item
\begin{eqnarray*}
\frac{\partial}{\partial w_{2}}\frac{\partial}{\partial w_{1}}F_{2}\left(z_{1},\tilde{\theta}_{2}\left(P_{1}\left(z_{1}\right)\hat{P}_{1}\left(w_{1}\right)
\hat{P}_{2}\left(w_{2}\right)\right)\right)|_{\mathbf{z,w}=1}&=&\hat{\mu}_{1}\hat{\mu}_{2}\frac{1}{1-\tilde{\mu}_{2}}\frac{\partial}{\partial
z_{2}}F_{2}\left(1,1\right)+\hat{\mu}_{1}\hat{\mu}_{2}\tilde{\theta}_{2}^{(2)}\left(1\right)\frac{\partial}{\partial
z_{2}}F_{2}\left(1,1\right)\\
&+&\hat{\mu}_{1}\hat{\mu}_{2}\left(\frac{1}{1-\tilde{\mu}_{2}}\right)^{2}f_{2}\left(2,2\right)\end{eqnarray*}
%___________________________________________________________________________________________

%\newpage

%___________________________________________________________________________________________
%
%\section{Parciales mixtas de $F_{2}$ para $w_{2}$}
%___________________________________________________________________________________________
%___________________________________________________________________________________________
\item \begin{eqnarray*}
\frac{\partial}{\partial z_{1}}\frac{\partial}{\partial w_{2}}F_{2}\left(z_{1},\tilde{\theta}_{2}\left(P_{1}\left(z_{1}\right)\hat{P}_{1}\left(w_{1}\right)
\hat{P}_{2}\left(w_{2}\right)\right)\right)|_{\mathbf{z,w}=1}&=&\mu_{1}\hat{\mu}_{2}\frac{1}{1-\tilde{\mu}_{2}}\frac{\partial}{\partial
z_{1}}F_{2}\left(1\right)+\mu_{1}\hat{\mu}_{2}\tilde{\theta}_{2}^{(2)}\left(1\right)\frac{\partial}{\partial
z_{2}}F_{2}\left(1,1\right)\\
&+&\hat{\mu}_{2}\mu_{1}\left(\frac{1}{1-\tilde{\mu}_{2}}\right)^{2}f_{2}\left(2,2\right)+\hat{\mu}_{2}\frac{1}{1-\tilde{\mu}_{2}}f_{2}\left(1,2\right)\end{eqnarray*}
\item
\begin{eqnarray*}
\frac{\partial}{\partial z_{2}}\frac{\partial}{\partial w_{2}}F_{2}\left(z_{1},\tilde{\theta}_{2}\left(P_{1}\left(z_{1}\right)\hat{P}_{1}\left(w_{1}\right)
\hat{P}_{2}\left(w_{2}\right)\right)\right)|_{\mathbf{z,w}=1}&=&0
\end{eqnarray*}
\item
\begin{eqnarray*}
\frac{\partial}{\partial w_{1}}\frac{\partial}{\partial w_{2}}F_{2}\left(z_{1},\tilde{\theta}_{2}\left(P_{1}\left(z_{1}\right)\hat{P}_{1}\left(w_{1}\right)
\hat{P}_{2}\left(w_{2}\right)\right)\right)|_{\mathbf{z,w}=1}&=&\hat{\mu}_{1}\hat{\mu}_{2}\frac{1}{1-\tilde{\mu}_{2}}\frac{\partial}{\partial
z_{2}}F_{2}\left(1,1\right)+\hat{\mu}_{1}\hat{\mu}_{2}\tilde{\theta}_{2}^{(2)}\left(1\right)\frac{\partial}{\partial
z_{2}}F_{2}\left(1,1\right)\\
&+&\hat{\mu}_{1}\hat{\mu}_{2}\left(\frac{1}{1-\tilde{\mu}_{2}}\right)^{2}f_{2}\left(2,2\right)\end{eqnarray*}
\item
\begin{eqnarray*}
\frac{\partial}{\partial w_{2}}\frac{\partial}{\partial w_{2}}F_{2}\left(z_{1},\tilde{\theta}_{2}\left(P_{1}\left(z_{1}\right)\hat{P}_{1}\left(w_{1}\right)
\hat{P}_{2}\left(w_{2}\right)\right)\right)|_{\mathbf{z,w}=1}&=&\hat{P}_{2}^{(2)}\left(1\right)\frac{1}{1-\tilde{\mu}_{2}}\frac{\partial}{\partial
z_{2}}F_{2}\left(1,1\right)+\hat{\mu}_{2}^{2}\tilde{\theta}_{2}^{(2)}\left(1\right)\frac{\partial}{\partial
z_{2}}F_{2}\left(1,1\right)\\
&+&\left(\hat{\mu}_{2}\frac{1}{1-\tilde{\mu}_{2}}\right)^{2}f_{2}\left(2,2\right)
\end{eqnarray*}
%___________________________________________________________________________________________




%\newpage
%___________________________________________________________________________________________
%
%\section{Parciales mixtas de $\hat{F}_{1}$ para $z_{1}$}
%___________________________________________________________________________________________
\item \begin{eqnarray*}
\frac{\partial}{\partial z_{1}}\frac{\partial}{\partial z_{1}}\hat{F}_{1}\left(\hat{\theta}_{1}\left(P_{1}\left(z_{1}\right)\tilde{P}_{2}\left(z_{2}\right)
\hat{P}_{2}\left(w_{2}\right)\right),w_{2}\right)|_{\mathbf{z,w}=1}&=&\frac{1}{1-\hat{\mu}_{1}}P_{1}^{(2)}\frac{\partial}{\partial w_{1}}\hat{F}_{1}\left(1,1\right)+\mu_{1}^2\hat{\theta}_{1}^{(2)}\left(1\right)\frac{\partial}{\partial w_{1}}\hat{F}_{1}\left(1,1\right)\\
&+&\mu_{1}^2\left(\frac{1}{1- \hat{\mu}_{1}}\right)^2\hat{f}_{1}\left(1,1\right)
\end{eqnarray*}
%___________________________________________________________________________________________

%___________________________________________________________________________________________
\item
\begin{eqnarray*}
\frac{\partial}{\partial z_{2}}\frac{\partial}{\partial z_{1}}\hat{F}_{1}\left(\hat{\theta}_{1}\left(P_{1}\left(z_{1}\right)\tilde{P}_{2}\left(z_{2}\right)
\hat{P}_{2}\left(w_{2}\right)\right),w_{2}\right)|_{\mathbf{z,w}=1}&=&\mu_{1}\frac{1}{1-\hat{\mu}_{1}}\tilde{\mu}_{2}\frac{\partial}{\partial w_{1}}\hat{F}_{1}\left(1,1\right)\\
&+&\mu_{1}\tilde{\mu}_{2}\hat{\theta
}_{1}^{(2)}\left(1\right)\frac{\partial}{\partial w_{1}}\hat{F}_{1}\left(1,1\right)\\
&+&\mu_{1}\left(\frac{1}{1-\hat{\mu}_{1}}\right)^2\tilde{\mu}_{2}\hat{f}_{1}\left(1,1\right)
\end{eqnarray*}
%___________________________________________________________________________________________

%___________________________________________________________________________________________
\item \begin{eqnarray*}
\frac{\partial}{\partial w_{1}}\frac{\partial}{\partial z_{1}}\hat{F}_{1}\left(\hat{\theta}_{1}\left(P_{1}\left(z_{1}\right)\tilde{P}_{2}\left(z_{2}\right)
\hat{P}_{2}\left(w_{2}\right)\right),w_{2}\right)|_{\mathbf{z,w}=1}&=&0
\end{eqnarray*}
%___________________________________________________________________________________________

%___________________________________________________________________________________________
\item
\begin{eqnarray*}
\frac{\partial}{\partial w_{2}}\frac{\partial}{\partial z_{1}}\hat{F}_{1}\left(\hat{\theta}_{1}\left(P_{1}\left(z_{1}\right)\tilde{P}_{2}\left(z_{2}\right)
\hat{P}_{2}\left(w_{2}\right)\right),w_{2}\right)|_{\mathbf{z,w}=1}&=&\mu_{1}
\hat{\mu}_{2}\frac{1}{1-\hat{\mu
}_{1}}\frac{\partial}{\partial w_{1}}\hat{F}_{1}\left(1,1\right)+\mu_{1}\hat{\mu}_{2} \hat{\theta
}_{1}^{(2)}\left(1\right)\frac{\partial}{\partial w_{1}}\hat{F}_{1}\left(1,1\right)\\
&+&\mu_{1}\frac{1}{1-\hat{\mu}_{1}}f_{1}\left(1,2\right)+\mu_{1}\hat{\mu}_{2}\left(\frac{1}{1-\hat{\mu}_{1}}\right)^{2}\hat{f}_{1}\left(1,1\right)
\end{eqnarray*}
%___________________________________________________________________________________________


%___________________________________________________________________________________________
%
%\section{Parciales mixtas de $\hat{F}_{1}$ para $z_{2}$}
%___________________________________________________________________________________________
\item
\begin{eqnarray*}
\frac{\partial}{\partial z_{1}}\frac{\partial}{\partial z_{2}}\hat{F}_{1}\left(\hat{\theta}_{1}\left(P_{1}\left(z_{1}\right)\tilde{P}_{2}\left(z_{2}\right)
\hat{P}_{2}\left(w_{2}\right)\right),w_{2}\right)|_{\mathbf{z,w}=1}&=&\mu_{1}\tilde{\mu}_{2}\frac{1}{1-\hat{\mu}_{1}}\frac{\partial}{\partial w_{1}}
\hat{F}_{1}\left(1,1\right)+\mu_{1}\tilde{\mu}_{2}\hat{\theta
}_{1}^{(2)}\left(1\right)\frac{\partial}{\partial w_{1}}\hat{F}_{1}\left(1,1\right)\\
&+&\mu_{1}\tilde{\mu}_{2}\left(\frac{1}{1-\hat{\mu}_{1}}\right)^{2}\hat{f}_{1}\left(1,1\right)
\end{eqnarray*}
%___________________________________________________________________________________________

%___________________________________________________________________________________________
\item
\begin{eqnarray*}
\frac{\partial}{\partial z_{2}}\frac{\partial}{\partial z_{2}}\hat{F}_{1}\left(\hat{\theta}_{1}\left(P_{1}\left(z_{1}\right)\tilde{P}_{2}\left(z_{2}\right)
\hat{P}_{2}\left(w_{2}\right)\right),w_{2}\right)|_{\mathbf{z,w}=1}&=&\tilde{\mu}_{2}^{2}\hat{\theta
}_{1}^{(2)}\left(1\right)\frac{\partial}{\partial w_{1}}\hat{F}_{1}\left(1,1\right)+\frac{1}{1-\hat{\mu}_{1}}\tilde{P}_{2}^{(2)}\frac{\partial}{\partial w_{1}}\hat{F}_{1}\left(1,1\right)\\
&+&\tilde{\mu}_{2}^{2}\left(\frac{1}{1-\hat{\mu}_{1}}\right)^{2}\hat{f}_{1}\left(1,1\right)
\end{eqnarray*}
%___________________________________________________________________________________________

%___________________________________________________________________________________________
\item \begin{eqnarray*}
\frac{\partial}{\partial w_{1}}\frac{\partial}{\partial z_{2}}\hat{F}_{1}\left(\hat{\theta}_{1}\left(P_{1}\left(z_{1}\right)\tilde{P}_{2}\left(z_{2}\right)
\hat{P}_{2}\left(w_{2}\right)\right),w_{2}\right)|_{\mathbf{z,w}=1}&=&0
\end{eqnarray*}
%___________________________________________________________________________________________
%___________________________________________________________________________________________
\item
\begin{eqnarray*}
\frac{\partial}{\partial w_{2}}\frac{\partial}{\partial z_{2}}\hat{F}_{1}\left(\hat{\theta}_{1}\left(P_{1}\left(z_{1}\right)\tilde{P}_{2}\left(z_{2}\right)
\hat{P}_{2}\left(w_{2}\right)\right),w_{2}\right)|_{\mathbf{z,w}=1}&=&\hat{\mu}_{2}\tilde{\mu}_{2}\frac{1}{1-\hat{\mu}_{1}}
\frac{\partial}{\partial w_{1}}\hat{F}_{1}\left(1,1\right)+\hat{\mu}_{2}\tilde{\mu}_{2}\hat{\theta
}_{1}^{(2)}\left(1\right)\frac{\partial}{\partial w_{1}}\hat{F}_{1}\left(1,1\right)\\
&+&\frac{1}{1-\hat{\mu
}_{1}}\tilde{\mu}_{2}\hat{f}_{1}\left(1,2\right)+\tilde{\mu}_{2}\hat{\mu}_{2}\left(\frac{1}{1-\hat{\mu}_{1}}\right)^{2}\hat{f}_{1}\left(1,1\right)
\end{eqnarray*}
%___________________________________________________________________________________________

%\newpage

%___________________________________________________________________________________________
%
%\section{Parciales mixtas de $\hat{F}_{1}$ para $w_{1}$}
%___________________________________________________________________________________________
%___________________________________________________________________________________________
\item \begin{eqnarray*}
\frac{\partial}{\partial z_{1}}\frac{\partial}{\partial w_{1}}\hat{F}_{1}\left(\hat{\theta}_{1}\left(P_{1}\left(z_{1}\right)\tilde{P}_{2}\left(z_{2}\right)
\hat{P}_{2}\left(w_{2}\right)\right),w_{2}\right)|_{\mathbf{z,w}=1}&=&0
\end{eqnarray*}
%___________________________________________________________________________________________

%___________________________________________________________________________________________
\item
\begin{eqnarray*}
\frac{\partial}{\partial z_{2}}\frac{\partial}{\partial w_{1}}\hat{F}_{1}\left(\hat{\theta}_{1}\left(P_{1}\left(z_{1}\right)\tilde{P}_{2}\left(z_{2}\right)
\hat{P}_{2}\left(w_{2}\right)\right),w_{2}\right)|_{\mathbf{z,w}=1}&=&0
\end{eqnarray*}
%___________________________________________________________________________________________

%___________________________________________________________________________________________
\item
\begin{eqnarray*}
\frac{\partial}{\partial w_{1}}\frac{\partial}{\partial w_{1}}\hat{F}_{1}\left(\hat{\theta}_{1}\left(P_{1}\left(z_{1}\right)\tilde{P}_{2}\left(z_{2}\right)
\hat{P}_{2}\left(w_{2}\right)\right),w_{2}\right)|_{\mathbf{z,w}=1}&=&0
\end{eqnarray*}
%___________________________________________________________________________________________

%___________________________________________________________________________________________
\item
\begin{eqnarray*}
\frac{\partial}{\partial w_{2}}\frac{\partial}{\partial w_{1}}\hat{F}_{1}\left(\hat{\theta}_{1}\left(P_{1}\left(z_{1}\right)\tilde{P}_{2}\left(z_{2}\right)
\hat{P}_{2}\left(w_{2}\right)\right),w_{2}\right)|_{\mathbf{z,w}=1}&=&0
\end{eqnarray*}
%___________________________________________________________________________________________


%\newpage
%___________________________________________________________________________________________
%
%\section{Parciales mixtas de $\hat{F}_{1}$ para $w_{2}$}
%___________________________________________________________________________________________
%___________________________________________________________________________________________
\item \begin{eqnarray*}
\frac{\partial}{\partial z_{1}}\frac{\partial}{\partial w_{2}}\hat{F}_{1}\left(\hat{\theta}_{1}\left(P_{1}\left(z_{1}\right)\tilde{P}_{2}\left(z_{2}\right)
\hat{P}_{2}\left(w_{2}\right)\right),w_{2}\right)|_{\mathbf{z,w}=1}&=&\mu_{1}\hat{\mu}_{2}\frac{1}{1-\hat{\mu}_{1}}\frac{\partial}{\partial w_{1}}\hat{F}_{1}\left(1,1\right)+\mu_{1}\hat{\mu}_{2}\hat{\theta
}_{1}^{(2)}\frac{\partial}{\partial w_{1}}\hat{F}_{1}\left(1,1\right)\\
&+&\mu_{1}\frac{1}{1-\hat{\mu}_{1}}\hat{f}_{1}\left(1,2\right)+\mu_{1}\hat{\mu}_{2}\left(\frac{1}{1-\hat{\mu}_{1}}\right)^{2}\hat{f}_1\left(1,1\right)
\end{eqnarray*}
%___________________________________________________________________________________________

%___________________________________________________________________________________________
\begin{eqnarray*}
&&\frac{\partial}{\partial z_{2}}\frac{\partial}{\partial w_{2}}\hat{F}_{1}\left(\hat{\theta}_{1}\left(P_{1}\left(z_{1}\right)\tilde{P}_{2}\left(z_{2}\right)
\hat{P}_{2}\left(w_{2}\right)\right),w_{2}\right)|_{\mathbf{z,w}=1}\\
&=&P_1\left(z_1\right) \hat{P}_2'\left(w_2\right)
\hat{\theta }_1'\left(P_1\left(z_1\right)
\hat{P}_2\left(w_2\right) \tilde{P}_2\left(z_2\right)\right)
\tilde{P}_2'\left(z_2\right)\hat{F}_1^{(1,0)}\left(\hat{\theta }_1\left(P_1\left(z_1\right)
\hat{P}_2\left(w_2\right)
\tilde{P}_2\left(z_2\right)\right),w_2\right)\\
&+&P_1\left(z_1\right)^2
\hat{P}_2\left(w_2\right)\tilde{P}_2\left(z_2\right) \hat{P}_2'\left(w_2\right)
\tilde{P}_2'\left(z_2\right) \hat{\theta
}_1''\left(P_1\left(z_1\right) \hat{P}_2\left(w_2\right)
\tilde{P}_2\left(z_2\right)\right)\hat{F}_1^{(1,0)}\left(\hat{\theta }_1\left(P_1\left(z_1\right) \hat{P}_2\left(w_2\right) \tilde{P}_2\left(z_2\right)\right),w_2\right)\\
&+&P_1\left(z_1\right) \hat{P}_2\left(w_2\right) \hat{\theta
}_1'\left(P_1\left(z_1\right) \hat{P}_2\left(w_2\right)
\tilde{P}_2\left(z_2\right)\right)
\tilde{P}_2'\left(z_2\right)\hat{F}_1^{(1,1)}\left(\hat{\theta }_1\left(P_1\left(z_1\right) \hat{P}_2\left(w_2\right) \tilde{P}_2\left(z_2\right)\right),w_2\right)\\
&+&P_1\left(z_1\right)^2 \hat{P}_2\left(w_2\right)
\tilde{P}_2\left(z_2\right) \hat{P}_2'\left(w_2\right) \hat{\theta
}_1'\left(P_1\left(z_1\right)
\hat{P}_2\left(w_2\right) \tilde{P}_2\left(z_2\right)\right)^2\tilde{P}_2'\left(z_2\right) \hat{F}_1^{(2,0)}\left(\hat{\theta
}_1\left(P_1\left(z_1\right) \hat{P}_2\left(w_2\right)
\tilde{P}_2\left(z_2\right)\right),w_2\right)
\end{eqnarray*}
%___________________________________________________________________________________________

%___________________________________________________________________________________________
\begin{eqnarray*}
\frac{\partial}{\partial w_{1}}\frac{\partial}{\partial w_{2}}\hat{F}_{1}\left(\hat{\theta}_{1}\left(P_{1}\left(z_{1}\right)\tilde{P}_{2}\left(z_{2}\right)
\hat{P}_{2}\left(w_{2}\right)\right),w_{2}\right)|_{\mathbf{z,w}=1}&=&0
\end{eqnarray*}
%___________________________________________________________________________________________

%___________________________________________________________________________________________
\begin{eqnarray*}
&&\frac{\partial}{\partial w_{2}}\frac{\partial}{\partial w_{2}}\hat{F}_{1}\left(\hat{\theta}_{1}\left(P_{1}\left(z_{1}\right)\tilde{P}_{2}\left(z_{2}\right)
\hat{P}_{2}\left(w_{2}\right)\right),w_{2}\right)|_{\mathbf{z,w}=1}\\
&=&\hat{F}_1^{(0,2)}\left(\hat{\theta }_1\left(P_1\left(z_1\right) \hat{P}_2\left(w_2\right) \tilde{P}_2\left(z_2\right)\right),w_2\right)\\
&+&P_1\left(z_1\right) \tilde{P}_2\left(z_2\right) \hat{\theta
}_1'\left(P_1\left(z_1\right) \hat{P}_2\left(w_2\right)
\tilde{P}_2\left(z_2\right)\right)\hat{P}_2''\left(w_2\right) \hat{F}_1^{(1,0)}\left(\hat{\theta }_1\left(P_1\left(z_1\right) \hat{P}_2\left(w_2\right) \tilde{P}_2\left(z_2\right)\right),w_2\right)\\
&+&P_1\left(z_1\right)^2 \tilde{P}_2\left(z_2\right)^2
\hat{P}_2'\left(w_2\right)^2 \hat{\theta
}_1''\left(P_1\left(z_1\right) \hat{P}_2\left(w_2\right)
\tilde{P}_2\left(z_2\right)\right)\hat{F}_1^{(1,0)}\left(\hat{\theta }_1\left(P_1\left(z_1\right) \hat{P}_2\left(w_2\right) \tilde{P}_2\left(z_2\right)\right),w_2\right)\\
&+&P_1\left(z_1\right) \tilde{P}_2\left(z_2\right)
\hat{P}_2'\left(w_2\right) \hat{\theta
}_1'\left(P_1\left(z_1\right) \hat{P}_2\left(w_2\right)
\tilde{P}_2\left(z_2\right)\right)\\
&+&P_1\left(z_1\right) \tilde{P}_2\left(z_2\right)
\hat{P}_2'\left(w_2\right) \hat{\theta
}_1'\left(P_1\left(z_1\right) \hat{P}_2\left(w_2\right)
\tilde{P}_2\left(z_2\right)\right)\hat{F}_1^{(1,1)}\left(\hat{\theta }_1\left(P_1\left(z_1\right) \hat{P}_2\left(w_2\right) \tilde{P}_2\left(z_2\right)\right),w_2\right)\\
&+&P_1\left(z_1\right) \tilde{P}_2\left(z_2\right)
\hat{P}_2'\left(w_2\right) \hat{\theta
}_1'\left(P_1\left(z_1\right) \hat{P}_2\left(w_2\right)
\tilde{P}_2\left(z_2\right)\right)
P_1\left(z_1\right) \tilde{P}_2\left(z_2\right)
\hat{P}_2'\left(w_2\right) \hat{\theta
}_1'\left(P_1\left(z_1\right) \hat{P}_2\left(w_2\right)
\tilde{P}_2\left(z_2\right)\right)
\\
&&\left.\hat{F}_1^{(2,0)}\left(\hat{\theta
}_1\left(P_1\left(z_1\right) \hat{P}_2\left(w_2\right)
\tilde{P}_2\left(z_2\right)\right),w_2\right)\right)
\end{eqnarray*}
%___________________________________________________________________________________________


%___________________________________________________________________________________________
%
%\section{Parciales mixtas de $\hat{F}_{2}$ para $z_{1}$}
%___________________________________________________________________________________________
%___________________________________________________________________________________________
\begin{eqnarray*}
&&\frac{\partial}{\partial z_{1}}\frac{\partial}{\partial z_{1}}\hat{F}_{2}\left(w_{1},\hat{\theta}_{2}\left(P_{1}\left(z_{1}\right)\tilde{P}_{2}\left(z_{2}\right)
\hat{P}_{1}\left(w_{1}\right)\right)\right)|_{\mathbf{z,w}=1}\\
&=&P_1\left(w_1\right) \tilde{P}_2\left(z_2\right)
\hat{\theta }_2'\left(P_1\left(w_1\right) P_1\left(z_1\right)
\tilde{P}_2\left(z_2\right)\right)P_1''\left(z_1\right) \hat{F}_2^{(0,1)}\left(w_1,\hat{\theta }_2\left(P_1\left(w_1\right) P_1\left(z_1\right) \tilde{P}_2\left(z_2\right)\right)\right)\\
&+&P_1\left(w_1\right)^2 \tilde{P}_2\left(z_2\right)^2
P_1'\left(z_1\right)^2 \hat{\theta }_2''\left(P_1\left(w_1\right)
P_1\left(z_1\right) \tilde{P}_2\left(z_2\right)\right)\hat{F}_2^{(0,1)}\left(w_1,\hat{\theta }_2\left(P_1\left(w_1\right) P_1\left(z_1\right) \tilde{P}_2\left(z_2\right)\right)\right)\\
&+&P_1\left(w_1\right)^2 \tilde{P}_2\left(z_2\right)^2
P_1'\left(z_1\right)^2 \hat{\theta }_2'\left(P_1\left(w_1\right)
P_1\left(z_1\right) \tilde{P}_2\left(z_2\right)\right)^2\hat{F}_2^{(0,2)}\left(w_1,\hat{\theta
}_2\left(P_1\left(w_1\right) P_1\left(z_1\right)
\tilde{P}_2\left(z_2\right)\right)\right)
\end{eqnarray*}
%___________________________________________________________________________________________


%___________________________________________________________________________________________
\begin{eqnarray*}
&&\frac{\partial}{\partial z_{2}}\frac{\partial}{\partial z_{1}}\hat{F}_{2}\left(w_{1},\hat{\theta}_{2}\left(P_{1}\left(z_{1}\right)\tilde{P}_{2}\left(z_{2}\right)
\hat{P}_{1}\left(w_{1}\right)\right)\right)|_{\mathbf{z,w}=1}\\
&=&P_1\left(w_1\right) P_1'\left(z_1\right) \hat{\theta
}_2'\left(P_1\left(w_1\right) P_1\left(z_1\right)
\tilde{P}_2\left(z_2\right)\right)
\tilde{P}_2'\left(z_2\right)\hat{F}_2^{(0,1)}\left(w_1,\hat{\theta
}_2\left(P_1\left(w_1\right) P_1\left(z_1\right)
\tilde{P}_2\left(z_2\right)\right)\right)\\
&+&P_1\left(w_1\right)^2 P_1\left(z_1\right)\tilde{P}_2\left(z_2\right) P_1'\left(z_1\right)\tilde{P}_2'\left(z_2\right) \hat{\theta
}_2''\left(P_1\left(w_1\right) P_1\left(z_1\right)
\tilde{P}_2\left(z_2\right)\right)\hat{F}_2^{(0,1)}\left(w_1,\hat{\theta }_2\left(P_1\left(w_1\right) P_1\left(z_1\right) \tilde{P}_2\left(z_2\right)\right)\right)\\
&+&P_1\left(w_1\right)^2 P_1\left(z_1\right)
\tilde{P}_2\left(z_2\right) P_1'\left(z_1\right) \hat{\theta
}_2'\left(P_1\left(w_1\right) P_1\left(z_1\right)
\tilde{P}_2\left(z_2\right)\right)^2 \tilde{P}_2'\left(z_2\right)
\hat{F}_2^{(0,2)}\left(w_1,\hat{\theta
}_2\left(P_1\left(w_1\right) P_1\left(z_1\right)
\tilde{P}_2\left(z_2\right)\right)\right)
\end{eqnarray*}
%___________________________________________________________________________________________

%___________________________________________________________________________________________
\begin{eqnarray*}
&&\frac{\partial}{\partial w_{1}}\frac{\partial}{\partial z_{1}}\hat{F}_{2}\left(w_{1},\hat{\theta}_{2}\left(P_{1}\left(z_{1}\right)\tilde{P}_{2}\left(z_{2}\right)
\hat{P}_{1}\left(w_{1}\right)\right)\right)|_{\mathbf{z,w}=1}\\
&=&\tilde{P}_2\left(z_2\right) P_1'\left(w_1\right)
P_1'\left(z_1\right) \hat{\theta }_2'\left(P_1\left(w_1\right)
P_1\left(z_1\right) \tilde{P}_2\left(z_2\right)\right)\hat{F}_2^{(0,1)}\left(w_1,\hat{\theta
}_2\left(P_1\left(w_1\right) P_1\left(z_1\right)
\tilde{P}_2\left(z_2\right)\right)\right)\\
&+&P_1\left(w_1\right)P_1\left(z_1\right)\tilde{P}_2\left(z_2\right)^2 P_1'\left(w_1\right)P_1'\left(z_1\right) \hat{\theta }_2''\left(P_1\left(w_1\right)P_1\left(z_1\right) \tilde{P}_2\left(z_2\right)\right)\hat{F}_2^{(0,1)}\left(w_1,\hat{\theta }_2\left(P_1\left(w_1\right) P_1\left(z_1\right) \tilde{P}_2\left(z_2\right)\right)\right)\\
&+&P_1\left(w_1\right) \tilde{P}_2\left(z_2\right)
P_1'\left(z_1\right) \hat{\theta }_2'\left(P_1\left(w_1\right)
P_1\left(z_1\right) \tilde{P}_2\left(z_2\right)\right)P_1\left(z_1\right) \tilde{P}_2\left(z_2\right)
P_1'\left(w_1\right) \hat{\theta }_2'\left(P_1\left(w_1\right)
P_1\left(z_1\right) \tilde{P}_2\left(z_2\right)\right)\\
&&\hat{F}_2^{(0,2)}\left(w_1,\hat{\theta }_2\left(P_1\left(w_1\right) P_1\left(z_1\right) \tilde{P}_2\left(z_2\right)\right)\right)\\
&+&P_1\left(w_1\right) \tilde{P}_2\left(z_2\right)
P_1'\left(z_1\right) \hat{\theta }_2'\left(P_1\left(w_1\right)
P_1\left(z_1\right) \tilde{P}_2\left(z_2\right)\right)\hat{F}_2^{(1,1)}\left(w_1,\hat{\theta
}_2\left(P_1\left(w_1\right) P_1\left(z_1\right)
\tilde{P}_2\left(z_2\right)\right)\right)
\end{eqnarray*}
%___________________________________________________________________________________________


%___________________________________________________________________________________________
\begin{eqnarray*}
\frac{\partial}{\partial w_{2}}\frac{\partial}{\partial z_{1}}\hat{F}_{2}\left(w_{1},\hat{\theta}_{2}\left(P_{1}\left(z_{1}\right)\tilde{P}_{2}\left(z_{2}\right)
\hat{P}_{1}\left(w_{1}\right)\right)\right)|_{\mathbf{z,w}=1}&=&0
\end{eqnarray*}
%___________________________________________________________________________________________

%___________________________________________________________________________________________
%
%\section{Parciales mixtas de $\hat{F}_{2}$ para $z_{2}$}
%___________________________________________________________________________________________
%___________________________________________________________________________________________
\begin{eqnarray*}
&&\frac{\partial}{\partial z_{1}}\frac{\partial}{\partial z_{2}}\hat{F}_{2}\left(w_{1},\hat{\theta}_{2}\left(P_{1}\left(z_{1}\right)\tilde{P}_{2}\left(z_{2}\right)
\hat{P}_{1}\left(w_{1}\right)\right)\right)|_{\mathbf{z,w}=1}\\
&=&P_1\left(w_1\right) P_1'\left(z_1\right) \hat{\theta
}_2'\left(P_1\left(w_1\right) P_1\left(z_1\right)
\tilde{P}_2\left(z_2\right)\right)
\tilde{P}_2'\left(z_2\right)\hat{F}_2^{(0,1)}\left(w_1,\hat{\theta
}_2\left(P_1\left(w_1\right) P_1\left(z_1\right)
\tilde{P}_2\left(z_2\right)\right)\right)\\
&+&P_1\left(w_1\right)^2
P_1\left(z_1\right)\tilde{P}_2\left(z_2\right) P_1'\left(z_1\right)
\tilde{P}_2'\left(z_2\right) \hat{\theta
}_2''\left(P_1\left(w_1\right) P_1\left(z_1\right)
\tilde{P}_2\left(z_2\right)\right)\hat{F}_2^{(0,1)}\left(w_1,\hat{\theta }_2\left(P_1\left(w_1\right) P_1\left(z_1\right) \tilde{P}_2\left(z_2\right)\right)\right)\\
&+&P_1\left(w_1\right)^2 P_1\left(z_1\right)
\tilde{P}_2\left(z_2\right) P_1'\left(z_1\right) \hat{\theta
}_2'\left(P_1\left(w_1\right) P_1\left(z_1\right)
\tilde{P}_2\left(z_2\right)\right)^2\tilde{P}_2'\left(z_2\right)
\hat{F}_2^{(0,2)}\left(w_1,\hat{\theta
}_2\left(P_1\left(w_1\right) P_1\left(z_1\right)
\tilde{P}_2\left(z_2\right)\right)\right)
\end{eqnarray*}
%___________________________________________________________________________________________

%___________________________________________________________________________________________
\begin{eqnarray*}
&&\frac{\partial}{\partial z_{2}}\frac{\partial}{\partial z_{2}}\hat{F}_{2}\left(w_{1},\hat{\theta}_{2}\left(P_{1}\left(z_{1}\right)\tilde{P}_{2}\left(z_{2}\right)
\hat{P}_{1}\left(w_{1}\right)\right)\right)|_{\mathbf{z,w}=1}\\
&=&P_1\left(w_1\right)^2 P_1\left(z_1\right)^2
\tilde{P}_2'\left(z_2\right)^2 \hat{\theta
}_2''\left(P_1\left(w_1\right) P_1\left(z_1\right)
\tilde{P}_2\left(z_2\right)\right)\hat{F}_2^{(0,1)}\left(w_1,\hat{\theta }_2\left(P_1\left(w_1\right) P_1\left(z_1\right) \tilde{P}_2\left(z_2\right)\right)\right)\\
&+&P_1\left(w_1\right) P_1\left(z_1\right) \hat{\theta
}_2'\left(P_1\left(w_1\right) P_1\left(z_1\right)
\tilde{P}_2\left(z_2\right)\right) \tilde{P}_2''\left(z_2\right)\hat{F}_2^{(0,1)}\left(w_1,\hat{\theta }_2\left(P_1\left(w_1\right) P_1\left(z_1\right) \tilde{P}_2\left(z_2\right)\right)\right)\\
&+&P_1\left(w_1\right)^2 P_1\left(z_1\right)^2 \hat{\theta }_2'\left(P_1\left(w_1\right) P_1\left(z_1\right) \tilde{P}_2\left(z_2\right)\right)^2\tilde{P}_2'\left(z_2\right)^2
\hat{F}_2^{(0,2)}\left(w_1,\hat{\theta
}_2\left(P_1\left(w_1\right) P_1\left(z_1\right)
\tilde{P}_2\left(z_2\right)\right)\right)
\end{eqnarray*}
%___________________________________________________________________________________________

%___________________________________________________________________________________________
\begin{eqnarray*}
&&\frac{\partial}{\partial w_{1}}\frac{\partial}{\partial z_{2}}\hat{F}_{2}\left(w_{1},\hat{\theta}_{2}\left(P_{1}\left(z_{1}\right)\tilde{P}_{2}\left(z_{2}\right)
\hat{P}_{1}\left(w_{1}\right)\right)\right)|_{\mathbf{z,w}=1}\\
&=&P_1\left(z_1\right) P_1'\left(w_1\right) \hat{\theta
}_2'\left(P_1\left(w_1\right) P_1\left(z_1\right)
\tilde{P}_2\left(z_2\right)\right)
\tilde{P}_2'\left(z_2\right)\hat{F}_2^{(0,1)}\left(w_1,\hat{\theta
}_2\left(P_1\left(w_1\right) P_1\left(z_1\right)
\tilde{P}_2\left(z_2\right)\right)\right)\\
&+&P_1\left(w_1\right)P_1\left(z_1\right)^2\tilde{P}_2\left(z_2\right) P_1'\left(w_1\right)\tilde{P}_2'\left(z_2\right) \hat{\theta
}_2''\left(P_1\left(w_1\right) P_1\left(z_1\right)
\tilde{P}_2\left(z_2\right)\right)\hat{F}_2^{(0,1)}\left(w_1,\hat{\theta }_2\left(P_1\left(w_1\right) P_1\left(z_1\right) \tilde{P}_2\left(z_2\right)\right)\right)\\
&+&P_1\left(w_1\right) P_1\left(z_1\right) \hat{\theta
}_2'\left(P_1\left(w_1\right) P_1\left(z_1\right)
\tilde{P}_2\left(z_2\right)\right) \tilde{P}_2'\left(z_2\right)P_1\left(z_1\right) \tilde{P}_2\left(z_2\right)
P_1'\left(w_1\right) \hat{\theta }_2'\left(P_1\left(w_1\right)
P_1\left(z_1\right) \tilde{P}_2\left(z_2\right)\right)\\
&&\hat{F}_2^{(0,2)}\left(w_1,\hat{\theta }_2\left(P_1\left(w_1\right) P_1\left(z_1\right) \tilde{P}_2\left(z_2\right)\right)\right)\\
&+&P_1\left(w_1\right) P_1\left(z_1\right) \hat{\theta
}_2'\left(P_1\left(w_1\right) P_1\left(z_1\right)
\tilde{P}_2\left(z_2\right)\right) \tilde{P}_2'\left(z_2\right)
\hat{F}_2^{(1,1)}\left(w_1,\hat{\theta
}_2\left(P_1\left(w_1\right) P_1\left(z_1\right)
\tilde{P}_2\left(z_2\right)\right)\right)
\end{eqnarray*}
%___________________________________________________________________________________________

%___________________________________________________________________________________________
\begin{eqnarray*}
\frac{\partial}{\partial w_{2}}\frac{\partial}{\partial z_{2}}\hat{F}_{2}\left(w_{1},\hat{\theta}_{2}\left(P_{1}\left(z_{1}\right)\tilde{P}_{2}\left(z_{2}\right)
\hat{P}_{1}\left(w_{1}\right)\right)\right)|_{\mathbf{z,w}=1}&=&0
\end{eqnarray*}
%___________________________________________________________________________________________


%___________________________________________________________________________________________
%
%\section{Parciales mixtas de $\hat{F}_{2}$ para $w_{1}$}
%___________________________________________________________________________________________
%___________________________________________________________________________________________
\begin{eqnarray*}
&&\frac{\partial}{\partial z_{1}}\frac{\partial}{\partial w_{1}}\hat{F}_{2}\left(w_{1},\hat{\theta}_{2}\left(P_{1}\left(z_{1}\right)\tilde{P}_{2}\left(z_{2}\right)
\hat{P}_{1}\left(w_{1}\right)\right)\right)|_{\mathbf{z,w}=1}\\
&=&\tilde{P}_2\left(z_2\right) P_1'\left(w_1\right)
P_1'\left(z_1\right) \hat{\theta }_2'\left(P_1\left(w_1\right)
P_1\left(z_1\right) \tilde{P}_2\left(z_2\right)\right)\hat{F}_2^{(0,1)}\left(w_1,\hat{\theta
}_2\left(P_1\left(w_1\right) P_1\left(z_1\right)
\tilde{P}_2\left(z_2\right)\right)\right)\\
&+&P_1\left(w_1\right)P_1\left(z_1\right)
\tilde{P}_2\left(z_2\right)^2 P_1'\left(w_1\right)
P_1'\left(z_1\right) \hat{\theta }_2''\left(P_1\left(w_1\right)
P_1\left(z_1\right) \tilde{P}_2\left(z_2\right)\right)\hat{F}_2^{(0,1)}\left(w_1,\hat{\theta
}_2\left(P_1\left(w_1\right) P_1\left(z_1\right)
\tilde{P}_2\left(z_2\right)\right)\right)\\
&+&P_1\left(w_1\right)P_1\left(z_1\right)
\tilde{P}_2\left(z_2\right)^2 P_1'\left(w_1\right)
P_1'\left(z_1\right) \hat{\theta }_2'\left(P_1\left(w_1\right)
P_1\left(z_1\right) \tilde{P}_2\left(z_2\right)\right)^2\hat{F}_2^{(0,2)}\left(w_1,\hat{\theta }_2\left(P_1\left(w_1\right) P_1\left(z_1\right) \tilde{P}_2\left(z_2\right)\right)\right)\\
&+&P_1\left(w_1\right) \tilde{P}_2\left(z_2\right)
P_1'\left(z_1\right) \hat{\theta }_2'\left(P_1\left(w_1\right)
P_1\left(z_1\right) \tilde{P}_2\left(z_2\right)\right)\hat{F}_2^{(1,1)}\left(w_1,\hat{\theta
}_2\left(P_1\left(w_1\right) P_1\left(z_1\right)
\tilde{P}_2\left(z_2\right)\right)\right)
\end{eqnarray*}
%___________________________________________________________________________________________

%___________________________________________________________________________________________
\begin{eqnarray*}
&&\frac{\partial}{\partial z_{2}}\frac{\partial}{\partial w_{1}}\hat{F}_{2}\left(w_{1},\hat{\theta}_{2}\left(P_{1}\left(z_{1}\right)\tilde{P}_{2}\left(z_{2}\right)
\hat{P}_{1}\left(w_{1}\right)\right)\right)|_{\mathbf{z,w}=1}\\
&=&P_1\left(z_1\right) P_1'\left(w_1\right) \hat{\theta
}_2'\left(P_1\left(w_1\right) P_1\left(z_1\right)
\tilde{P}_2\left(z_2\right)\right)
\tilde{P}_2'\left(z_2\right)\hat{F}_2^{(0,1)}\left(w_1,\hat{\theta
}_2\left(P_1\left(w_1\right) P_1\left(z_1\right)
\tilde{P}_2\left(z_2\right)\right)\right)\\
&+&P_1\left(w_1\right)P_1\left(z_1\right)^2
\tilde{P}_2\left(z_2\right) P_1'\left(w_1\right)
\tilde{P}_2'\left(z_2\right) \hat{\theta
}_2''\left(P_1\left(w_1\right) P_1\left(z_1\right)
\tilde{P}_2\left(z_2\right)\right)\hat{F}_2^{(0,1)}\left(w_1,\hat{\theta }_2\left(P_1\left(w_1\right) P_1\left(z_1\right) \tilde{P}_2\left(z_2\right)\right)\right)\\
&+&P_1\left(w_1\right) P_1\left(z_1\right)^2
\tilde{P}_2\left(z_2\right) P_1'\left(w_1\right) \hat{\theta
}_2'\left(P_1\left(w_1\right) P_1\left(z_1\right)
\tilde{P}_2\left(z_2\right)\right)^2 \tilde{P}_2'\left(z_2\right) \hat{F}_2^{(0,2)}\left(w_1,\hat{\theta }_2\left(P_1\left(w_1\right) P_1\left(z_1\right) \tilde{P}_2\left(z_2\right)\right)\right)\\
&+&P_1\left(w_1\right) P_1\left(z_1\right) \hat{\theta
}_2'\left(P_1\left(w_1\right) P_1\left(z_1\right)
\tilde{P}_2\left(z_2\right)\right) \tilde{P}_2'\left(z_2\right)\hat{F}_2^{(1,1)}\left(w_1,\hat{\theta
}_2\left(P_1\left(w_1\right) P_1\left(z_1\right)
\tilde{P}_2\left(z_2\right)\right)\right)
\end{eqnarray*}
%___________________________________________________________________________________________

\begin{eqnarray*}
&&\frac{\partial}{\partial w_{1}}\frac{\partial}{\partial w_{1}}\hat{F}_{2}\left(w_{1},\hat{\theta}_{2}\left(P_{1}\left(z_{1}\right)\tilde{P}_{2}\left(z_{2}\right)
\hat{P}_{1}\left(w_{1}\right)\right)\right)|_{\mathbf{z,w}=1}\\
&=&P_1\left(z_1\right) \tilde{P}_2\left(z_2\right)
\hat{\theta }_2'\left(P_1\left(w_1\right) P_1\left(z_1\right)
\tilde{P}_2\left(z_2\right)\right)P_1''\left(w_1\right) \hat{F}_2^{(0,1)}\left(w_1,\hat{\theta }_2\left(P_1\left(w_1\right) P_1\left(z_1\right) \tilde{P}_2\left(z_2\right)\right)\right)\\
&+&P_1\left(z_1\right)^2 \tilde{P}_2\left(z_2\right)^2
P_1'\left(w_1\right)^2 \hat{\theta }_2''\left(P_1\left(w_1\right)
P_1\left(z_1\right) \tilde{P}_2\left(z_2\right)\right)\hat{F}_2^{(0,1)}\left(w_1,\hat{\theta }_2\left(P_1\left(w_1\right) P_1\left(z_1\right) \tilde{P}_2\left(z_2\right)\right)\right)\\
&+&P_1\left(z_1\right) \tilde{P}_2\left(z_2\right)
P_1'\left(w_1\right) \hat{\theta }_2'\left(P_1\left(w_1\right)
P_1\left(z_1\right) \tilde{P}_2\left(z_2\right)\right)\hat{F}_2^{(1,1)}\left(w_1,\hat{\theta }_2\left(P_1\left(w_1\right) P_1\left(z_1\right) \tilde{P}_2\left(z_2\right)\right)\right)\\
&+&P_1\left(z_1\right) \tilde{P}_2\left(z_2\right)
P_1'\left(w_1\right) \hat{\theta }_2'\left(P_1\left(w_1\right)
P_1\left(z_1\right) \tilde{P}_2\left(z_2\right)\right)P_1\left(z_1\right) \tilde{P}_2\left(z_2\right)
P_1'\left(w_1\right) \hat{\theta }_2'\left(P_1\left(w_1\right)
P_1\left(z_1\right) \tilde{P}_2\left(z_2\right)\right)\\
&&\hat{F}_2^{(0,2)}\left(w_1,\hat{\theta }_2\left(P_1\left(w_1\right) P_1\left(z_1\right) \tilde{P}_2\left(z_2\right)\right)\right)\\
&+&P_1\left(z_1\right) \tilde{P}_2\left(z_2\right)
P_1'\left(w_1\right) \hat{\theta }_2'\left(P_1\left(w_1\right)
P_1\left(z_1\right) \tilde{P}_2\left(z_2\right)\right)\hat{F}_2^{(1,1)}\left(w_1,\hat{\theta }_2\left(P_1\left(w_1\right) P_1\left(z_1\right) \tilde{P}_2\left(z_2\right)\right)\right)\\
&+&\hat{F}_2^{(2,0)}\left(w_1,\hat{\theta
}_2\left(P_1\left(w_1\right) P_1\left(z_1\right)
\tilde{P}_2\left(z_2\right)\right)\right)
\end{eqnarray*}



\begin{eqnarray*}
\frac{\partial}{\partial w_{2}}\frac{\partial}{\partial w_{1}}\hat{F}_{2}\left(w_{1},\hat{\theta}_{2}\left(P_{1}\left(z_{1}\right)\tilde{P}_{2}\left(z_{2}\right)
\hat{P}_{1}\left(w_{1}\right)\right)\right)|_{\mathbf{z,w}=1}&=&0
\end{eqnarray*}

%___________________________________________________________________________________________
%
%\section{Parciales mixtas de $\hat{F}_{2}$ para $w_{2}$}
%___________________________________________________________________________________________
\begin{eqnarray*}
\frac{\partial}{\partial z_{1}}\frac{\partial}{\partial w_{2}}\hat{F}_{2}\left(w_{1},\hat{\theta}_{2}\left(P_{1}\left(z_{1}\right)\tilde{P}_{2}\left(z_{2}\right)
\hat{P}_{1}\left(w_{1}\right)\right)\right)|_{\mathbf{z,w}=1}&=&0
\end{eqnarray*}

%___________________________________________________________________________________________
\begin{eqnarray*}
\frac{\partial}{\partial z_{2}}\frac{\partial}{\partial w_{2}}\hat{F}_{2}\left(w_{1},\hat{\theta}_{2}\left(P_{1}\left(z_{1}\right)\tilde{P}_{2}\left(z_{2}\right)
\hat{P}_{1}\left(w_{1}\right)\right)\right)|_{\mathbf{z,w}=1}&=&0
\end{eqnarray*}

%___________________________________________________________________________________________

%___________________________________________________________________________________________
\begin{eqnarray*}
\frac{\partial}{\partial w_{1}}\frac{\partial}{\partial w_{2}}\hat{F}_{2}\left(w_{1},\hat{\theta}_{2}\left(P_{1}\left(z_{1}\right)\tilde{P}_{2}\left(z_{2}\right)
\hat{P}_{1}\left(w_{1}\right)\right)\right)|_{\mathbf{z,w}=1}&=&0
\end{eqnarray*}

%___________________________________________________________________________________________

%___________________________________________________________________________________________
\begin{eqnarray*}
\frac{\partial}{\partial w_{2}}\frac{\partial}{\partial w_{2}}\hat{F}_{2}\left(w_{1},\hat{\theta}_{2}\left(P_{1}\left(z_{1}\right)\tilde{P}_{2}\left(z_{2}\right)
\hat{P}_{1}\left(w_{1}\right)\right)\right)|_{\mathbf{z,w}=1}&=&0
\end{eqnarray*}
\end{enumerate}




%___________________________________________________________________________________________
%
\subsection{Derivadas de Segundo Orden para $F_{1}$}
%___________________________________________________________________________________________

\subsubsection{Mixtas para $z_{1}$:}
%___________________________________________________________________________________________
\begin{enumerate}

%1/1/1
\item \begin{eqnarray*}
&&\frac{\partial}{\partial z_1}\frac{\partial}{\partial z_1}\left(R_2\left(P_1\left(z_1\right)\bar{P}_2\left(z_2\right)\hat{P}_1\left(w_1\right)\hat{P}_2\left(w_2\right)\right)F_2\left(z_1,\theta
_2\left(P_1\left(z_1\right)\hat{P}_1\left(w_1\right)\hat{P}_2\left(w_2\right)\right)\right)\hat{F}_2\left(w_1,w_2\right)\right)\\
&=&r_{2}P_{1}^{(2)}\left(1\right)+\mu_{1}^{2}R_{2}^{(2)}\left(1\right)+2\mu_{1}r_{2}\left(\frac{\mu_{1}}{1-\tilde{\mu}_{2}}F_{2}^{(0,1)}+F_{2}^{1,0)}\right)+\frac{1}{1-\tilde{\mu}_{2}}P_{1}^{(2)}F_{2}^{(0,1)}+\mu_{1}^{2}\tilde{\theta}_{2}^{(2)}\left(1\right)F_{2}^{(0,1)}\\
&+&\frac{\mu_{1}}{1-\tilde{\mu}_{2}}F_{2}^{(1,1)}+\frac{\mu_{1}}{1-\tilde{\mu}_{2}}\left(\frac{\mu_{1}}{1-\tilde{\mu}_{2}}F_{2}^{(0,2)}+F_{2}^{(1,1)}\right)+F_{2}^{(2,0)}.
\end{eqnarray*}

%2/2/1

\item \begin{eqnarray*}
&&\frac{\partial}{\partial z_2}\frac{\partial}{\partial z_1}\left(R_2\left(P_1\left(z_1\right)\bar{P}_2\left(z_2\right)\hat{P}_1\left(w_1\right)\hat{P}_2\left(w_2\right)\right)F_2\left(z_1,\theta
_2\left(P_1\left(z_1\right)\hat{P}_1\left(w_1\right)\hat{P}_2\left(w_2\right)\right)\right)\hat{F}_2\left(w_1,w_2\right)\right)\\
&=&\mu_{1}r_{2}\tilde{\mu}_{2}+\mu_{1}\tilde{\mu}_{2}R_{2}^{(2)}\left(1\right)+r_{2}\tilde{\mu}_{2}\left(\frac{\mu_{1}}{1-\tilde{\mu}_{2}}F_{2}^{(0,1)}+F_{2}^{(1,0)}\right).
\end{eqnarray*}
%3/3/1
\item \begin{eqnarray*}
&&\frac{\partial}{\partial w_1}\frac{\partial}{\partial z_1}\left(R_2\left(P_1\left(z_1\right)\bar{P}_2\left(z_2\right)\hat{P}_1\left(w_1\right)\hat{P}_2\left(w_2\right)\right)F_2\left(z_1,\theta
_2\left(P_1\left(z_1\right)\hat{P}_1\left(w_1\right)\hat{P}_2\left(w_2\right)\right)\right)\hat{F}_2\left(w_1,w_2\right)\right)\\
&=&\mu_{1}\hat{\mu}_{1}r_{2}+\mu_{1}\hat{\mu}_{1}R_{2}^{(2)}\left(1\right)+r_{2}\frac{\mu_{1}}{1-\tilde{\mu}_{2}}F_{2}^{(0,1)}+r_{2}\hat{\mu}_{1}\left(\frac{\mu_{1}}{1-\tilde{\mu}_{2}}F_{2}^{(0,1)}+F_{2}^{(1,0)}\right)+\mu_{1}r_{2}\hat{F}_{2}^{(1,0)}\\
&+&\left(\frac{\mu_{1}}{1-\tilde{\mu}_{2}}F_{2}^{(0,1)}+F_{2}^{(1,0)}\right)\hat{F}_{2}^{(1,0)}+\frac{\mu_{1}\hat{\mu}_{1}}{1-\tilde{\mu}_{2}}F_{2}^{(0,1)}+\mu_{1}\hat{\mu}_{1}\tilde{\theta}_{2}^{(2)}\left(1\right)F_{2}^{(0,1)}\\
&+&\mu_{1}\hat{\mu}_{1}\left(\frac{1}{1-\tilde{\mu}_{2}}\right)^{2}F_{2}^{(0,2)}+\frac{\hat{\mu}_{1}}{1-\tilde{\mu}_{2}}F_{2}^{(1,1)}.
\end{eqnarray*}
%4/4/1
\item \begin{eqnarray*}
&&\frac{\partial}{\partial w_2}\frac{\partial}{\partial z_1}\left(R_2\left(P_1\left(z_1\right)\bar{P}_2\left(z_2\right)\hat{P}_1\left(w_1\right)\hat{P}_2\left(w_2\right)\right)
F_2\left(z_1,\theta_2\left(P_1\left(z_1\right)\hat{P}_1\left(w_1\right)\hat{P}_2\left(w_2\right)\right)\right)\hat{F}_2\left(w_1,w_2\right)\right)\\
&=&\mu_{1}\hat{\mu}_{2}r_{2}+\mu_{1}\hat{\mu}_{2}R_{2}^{(2)}\left(1\right)+r_{2}\frac{\mu_{1}\hat{\mu}_{2}}{1-\tilde{\mu}_{2}}F_{2}^{(0,1)}+\mu_{1}r_{2}\hat{F}_{2}^{(0,1)}
+r_{2}\hat{\mu}_{2}\left(\frac{\mu_{1}}{1-\tilde{\mu}_{2}}F_{2}^{(0,1)}+F_{2}^{(1,0)}\right)\\
&+&\hat{F}_{2}^{(1,0)}\left(\frac{\mu_{1}}{1-\tilde{\mu}_{2}}F_{2}^{(0,1)}+F_{2}^{(1,0)}\right)+\frac{\mu_{1}\hat{\mu}_{2}}{1-\tilde{\mu}_{2}}F_{2}^{(0,1)}
+\mu_{1}\hat{\mu}_{2}\tilde{\theta}_{2}^{(2)}\left(1\right)F_{2}^{(0,1)}+\mu_{1}\hat{\mu}_{2}\left(\frac{1}{1-\tilde{\mu}_{2}}\right)^{2}F_{2}^{(0,2)}\\
&+&\frac{\hat{\mu}_{2}}{1-\tilde{\mu}_{2}}F_{2}^{(1,1)}.
\end{eqnarray*}
%___________________________________________________________________________________________
\subsubsection{Mixtas para $z_{2}$:}
%___________________________________________________________________________________________
%5
\item \begin{eqnarray*} &&\frac{\partial}{\partial
z_1}\frac{\partial}{\partial
z_2}\left(R_2\left(P_1\left(z_1\right)\bar{P}_2\left(z_2\right)\hat{P}_1\left(w_1\right)\hat{P}_2\left(w_2\right)\right)
F_2\left(z_1,\theta_2\left(P_1\left(z_1\right)\hat{P}_1\left(w_1\right)\hat{P}_2\left(w_2\right)\right)\right)\hat{F}_2\left(w_1,w_2\right)\right)\\
&=&\mu_{1}\tilde{\mu}_{2}r_{2}+\mu_{1}\tilde{\mu}_{2}R_{2}^{(2)}\left(1\right)+r_{2}\tilde{\mu}_{2}\left(\frac{\mu_{1}}{1-\tilde{\mu}_{2}}F_{2}^{(0,1)}+F_{2}^{(1,0)}\right).
\end{eqnarray*}

%6

\item \begin{eqnarray*} &&\frac{\partial}{\partial
z_2}\frac{\partial}{\partial
z_2}\left(R_2\left(P_1\left(z_1\right)\bar{P}_2\left(z_2\right)\hat{P}_1\left(w_1\right)\hat{P}_2\left(w_2\right)\right)
F_2\left(z_1,\theta_2\left(P_1\left(z_1\right)\hat{P}_1\left(w_1\right)\hat{P}_2\left(w_2\right)\right)\right)\hat{F}_2\left(w_1,w_2\right)\right)\\
&=&\tilde{\mu}_{2}^{2}R_{2}^{(2)}(1)+r_{2}\tilde{P}_{2}^{(2)}\left(1\right).
\end{eqnarray*}

%7
\item \begin{eqnarray*} &&\frac{\partial}{\partial
w_1}\frac{\partial}{\partial
z_2}\left(R_2\left(P_1\left(z_1\right)\bar{P}_2\left(z_2\right)\hat{P}_1\left(w_1\right)\hat{P}_2\left(w_2\right)\right)
F_2\left(z_1,\theta_2\left(P_1\left(z_1\right)\hat{P}_1\left(w_1\right)\hat{P}_2\left(w_2\right)\right)\right)\hat{F}_2\left(w_1,w_2\right)\right)\\
&=&\hat{\mu}_{1}\tilde{\mu}_{2}r_{2}+\hat{\mu}_{1}\tilde{\mu}_{2}R_{2}^{(2)}(1)+
r_{2}\frac{\hat{\mu}_{1}\tilde{\mu}_{2}}{1-\tilde{\mu}_{2}}F_{2}^{(0,1)}+r_{2}\tilde{\mu}_{2}\hat{F}_{2}^{(1,0)}.
\end{eqnarray*}
%8
\item \begin{eqnarray*} &&\frac{\partial}{\partial
w_2}\frac{\partial}{\partial
z_2}\left(R_2\left(P_1\left(z_1\right)\bar{P}_2\left(z_2\right)\hat{P}_1\left(w_1\right)\hat{P}_2\left(w_2\right)\right)
F_2\left(z_1,\theta_2\left(P_1\left(z_1\right)\hat{P}_1\left(w_1\right)\hat{P}_2\left(w_2\right)\right)\right)\hat{F}_2\left(w_1,w_2\right)\right)\\
&=&\hat{\mu}_{2}\tilde{\mu}_{2}r_{2}+\hat{\mu}_{2}\tilde{\mu}_{2}R_{2}^{(2)}(1)+
r_{2}\frac{\hat{\mu}_{2}\tilde{\mu}_{2}}{1-\tilde{\mu}_{2}}F_{2}^{(0,1)}+r_{2}\tilde{\mu}_{2}\hat{F}_{2}^{(0,1)}.
\end{eqnarray*}
%___________________________________________________________________________________________
\subsubsection{Mixtas para $w_{1}$:}
%___________________________________________________________________________________________

%9
\item \begin{eqnarray*} &&\frac{\partial}{\partial
z_1}\frac{\partial}{\partial
w_1}\left(R_2\left(P_1\left(z_1\right)\bar{P}_2\left(z_2\right)\hat{P}_1\left(w_1\right)\hat{P}_2\left(w_2\right)\right)
F_2\left(z_1,\theta_2\left(P_1\left(z_1\right)\hat{P}_1\left(w_1\right)\hat{P}_2\left(w_2\right)\right)\right)\hat{F}_2\left(w_1,w_2\right)\right)\\
&=&\mu_{1}\hat{\mu}_{1}r_{2}+\mu_{1}\hat{\mu}_{1}R_{2}^{(2)}\left(1\right)+\frac{\mu_{1}\hat{\mu}_{1}}{1-\tilde{\mu}_{2}}F_{2}^{(0,1)}+r_{2}\frac{\mu_{1}\hat{\mu}_{1}}{1-\tilde{\mu}_{2}}F_{2}^{(0,1)}+\mu_{1}\hat{\mu}_{1}\tilde{\theta}_{2}^{(2)}\left(1\right)F_{2}^{(0,1)}\\
&+&r_{2}\hat{\mu}_{1}\left(\frac{\mu_{1}}{1-\tilde{\mu}_{2}}F_{2}^{(0,1)}+F_{2}^{(1,0)}\right)+r_{2}\mu_{1}\hat{F}_{2}^{(1,0)}
+\left(\frac{\mu_{1}}{1-\tilde{\mu}_{2}}F_{2}^{(0,1)}+F_{2}^{(1,0)}\right)\hat{F}_{2}^{(1,0)}\\
&+&\frac{\hat{\mu}_{1}}{1-\tilde{\mu}_{2}}\left(\frac{\mu_{1}}{1-\tilde{\mu}_{2}}F_{2}^{(0,2)}+F_{2}^{(1,1)}\right).
\end{eqnarray*}
%10
\item \begin{eqnarray*} &&\frac{\partial}{\partial
z_2}\frac{\partial}{\partial
w_1}\left(R_2\left(P_1\left(z_1\right)\bar{P}_2\left(z_2\right)\hat{P}_1\left(w_1\right)\hat{P}_2\left(w_2\right)\right)
F_2\left(z_1,\theta_2\left(P_1\left(z_1\right)\hat{P}_1\left(w_1\right)\hat{P}_2\left(w_2\right)\right)\right)\hat{F}_2\left(w_1,w_2\right)\right)\\
&=&\tilde{\mu}_{2}\hat{\mu}_{1}r_{2}+\tilde{\mu}_{2}\hat{\mu}_{1}R_{2}^{(2)}\left(1\right)+r_{2}\frac{\tilde{\mu}_{2}\hat{\mu}_{1}}{1-\tilde{\mu}_{2}}F_{2}^{(0,1)}
+r_{2}\tilde{\mu}_{2}\hat{F}_{2}^{(1,0)}.
\end{eqnarray*}
%11
\item \begin{eqnarray*} &&\frac{\partial}{\partial
w_1}\frac{\partial}{\partial
w_1}\left(R_2\left(P_1\left(z_1\right)\bar{P}_2\left(z_2\right)\hat{P}_1\left(w_1\right)\hat{P}_2\left(w_2\right)\right)
F_2\left(z_1,\theta_2\left(P_1\left(z_1\right)\hat{P}_1\left(w_1\right)\hat{P}_2\left(w_2\right)\right)\right)\hat{F}_2\left(w_1,w_2\right)\right)\\
&=&\hat{\mu}_{1}^{2}R_{2}^{(2)}\left(1\right)+r_{2}\hat{P}_{1}^{(2)}\left(1\right)+2r_{2}\frac{\hat{\mu}_{1}^{2}}{1-\tilde{\mu}_{2}}F_{2}^{(0,1)}+
\hat{\mu}_{1}^{2}\tilde{\theta}_{2}^{(2)}\left(1\right)F_{2}^{(0,1)}+\frac{1}{1-\tilde{\mu}_{2}}\hat{P}_{1}^{(2)}\left(1\right)F_{2}^{(0,1)}\\
&+&\frac{\hat{\mu}_{1}^{2}}{1-\tilde{\mu}_{2}}F_{2}^{(0,2)}+2r_{2}\hat{\mu}_{1}\hat{F}_{2}^{(1,0)}+2\frac{\hat{\mu}_{1}}{1-\tilde{\mu}_{2}}F_{2}^{(0,1)}\hat{F}_{2}^{(1,0)}+\hat{F}_{2}^{(2,0)}.
\end{eqnarray*}
%12
\item \begin{eqnarray*} &&\frac{\partial}{\partial
w_2}\frac{\partial}{\partial
w_1}\left(R_2\left(P_1\left(z_1\right)\bar{P}_2\left(z_2\right)\hat{P}_1\left(w_1\right)\hat{P}_2\left(w_2\right)\right)
F_2\left(z_1,\theta_2\left(P_1\left(z_1\right)\hat{P}_1\left(w_1\right)\hat{P}_2\left(w_2\right)\right)\right)\hat{F}_2\left(w_1,w_2\right)\right)\\
&=&r_{2}\hat{\mu}_{2}\hat{\mu}_{1}+\hat{\mu}_{1}\hat{\mu}_{2}R_{2}^{(2)}(1)+\frac{\hat{\mu}_{1}\hat{\mu}_{2}}{1-\tilde{\mu}_{2}}F_{2}^{(0,1)}
+2r_{2}\frac{\hat{\mu}_{1}\hat{\mu}_{2}}{1-\tilde{\mu}_{2}}F_{2}^{(0,1)}+\hat{\mu}_{2}\hat{\mu}_{1}\tilde{\theta}_{2}^{(2)}\left(1\right)F_{2}^{(0,1)}+
r_{2}\hat{\mu}_{1}\hat{F}_{2}^{(0,1)}\\
&+&\frac{\hat{\mu}_{1}}{1-\tilde{\mu}_{2}}F_{2}^{(0,1)}\hat{F}_{2}^{(0,1)}+\hat{\mu}_{1}\hat{\mu}_{2}\left(\frac{1}{1-\tilde{\mu}_{2}}\right)^{2}F_{2}^{(0,2)}+
r_{2}\hat{\mu}_{2}\hat{F}_{2}^{(1,0)}+\frac{\hat{\mu}_{2}}{1-\tilde{\mu}_{2}}F_{2}^{(0,1)}\hat{F}_{2}^{(1,0)}+\hat{F}_{2}^{(1,1)}.
\end{eqnarray*}
%___________________________________________________________________________________________
\subsubsection{Mixtas para $w_{2}$:}
%___________________________________________________________________________________________
%13

\item \begin{eqnarray*} &&\frac{\partial}{\partial
z_1}\frac{\partial}{\partial
w_2}\left(R_2\left(P_1\left(z_1\right)\bar{P}_2\left(z_2\right)\hat{P}_1\left(w_1\right)\hat{P}_2\left(w_2\right)\right)
F_2\left(z_1,\theta_2\left(P_1\left(z_1\right)\hat{P}_1\left(w_1\right)\hat{P}_2\left(w_2\right)\right)\right)\hat{F}_2\left(w_1,w_2\right)\right)\\
&=&r_{2}\mu_{1}\hat{\mu}_{2}+\mu_{1}\hat{\mu}_{2}R_{2}^{(2)}(1)+\frac{\mu_{1}\hat{\mu}_{2}}{1-\tilde{\mu}_{2}}F_{2}^{(0,1)}+r_{2}\frac{\mu_{1}\hat{\mu}_{2}}{1-\tilde{\mu}_{2}}F_{2}^{(0,1)}+\mu_{1}\hat{\mu}_{2}\tilde{\theta}_{2}^{(2)}\left(1\right)F_{2}^{(0,1)}+r_{2}\mu_{1}\hat{F}_{2}^{(0,1)}\\
&+&r_{2}\hat{\mu}_{2}\left(\frac{\mu_{1}}{1-\tilde{\mu}_{2}}F_{2}^{(0,1)}+F_{2}^{(1,0)}\right)+\hat{F}_{2}^{(0,1)}\left(\frac{\mu_{1}}{1-\tilde{\mu}_{2}}F_{2}^{(0,1)}+F_{2}^{(1,0)}\right)+\frac{\hat{\mu}_{2}}{1-\tilde{\mu}_{2}}\left(\frac{\mu_{1}}{1-\tilde{\mu}_{2}}F_{2}^{(0,2)}+F_{2}^{(1,1)}\right).
\end{eqnarray*}
%14
\item \begin{eqnarray*} &&\frac{\partial}{\partial
z_2}\frac{\partial}{\partial
w_2}\left(R_2\left(P_1\left(z_1\right)\bar{P}_2\left(z_2\right)\hat{P}_1\left(w_1\right)\hat{P}_2\left(w_2\right)\right)
F_2\left(z_1,\theta_2\left(P_1\left(z_1\right)\hat{P}_1\left(w_1\right)\hat{P}_2\left(w_2\right)\right)\right)\hat{F}_2\left(w_1,w_2\right)\right)\\
&=&r_{2}\tilde{\mu}_{2}\hat{\mu}_{2}+\tilde{\mu}_{2}\hat{\mu}_{2}R_{2}^{(2)}(1)+r_{2}\frac{\tilde{\mu}_{2}\hat{\mu}_{2}}{1-\tilde{\mu}_{2}}F_{2}^{(0,1)}+r_{2}\tilde{\mu}_{2}\hat{F}_{2}^{(0,1)}.
\end{eqnarray*}
%15
\item \begin{eqnarray*} &&\frac{\partial}{\partial
w_1}\frac{\partial}{\partial
w_2}\left(R_2\left(P_1\left(z_1\right)\bar{P}_2\left(z_2\right)\hat{P}_1\left(w_1\right)\hat{P}_2\left(w_2\right)\right)
F_2\left(z_1,\theta_2\left(P_1\left(z_1\right)\hat{P}_1\left(w_1\right)\hat{P}_2\left(w_2\right)\right)\right)\hat{F}_2\left(w_1,w_2\right)\right)\\
&=&r_{2}\hat{\mu}_{1}\hat{\mu}_{2}+\hat{\mu}_{1}\hat{\mu}_{2}R_{2}^{(2)}\left(1\right)+\frac{\hat{\mu}_{1}\hat{\mu}_{2}}{1-\tilde{\mu}_{2}}F_{2}^{(0,1)}+2r_{2}\frac{\hat{\mu}_{1}\hat{\mu}_{2}}{1-\tilde{\mu}_{2}}F_{2}^{(0,1)}+\hat{\mu}_{1}\hat{\mu}_{2}\theta_{2}^{(2)}\left(1\right)F_{2}^{(0,1)}+r_{2}\hat{\mu}_{1}\hat{F}_{2}^{(0,1)}\\
&+&\frac{\hat{\mu}_{1}}{1-\tilde{\mu}_{2}}F_{2}^{(0,1)}\hat{F}_{2}^{(0,1)}+\hat{\mu}_{1}\hat{\mu}_{2}\left(\frac{1}{1-\tilde{\mu}_{2}}\right)^{2}F_{2}^{(0,2)}+r_{2}\hat{\mu}_{2}\hat{F}_{2}^{(0,1)}+\frac{\hat{\mu}_{2}}{1-\tilde{\mu}_{2}}F_{2}^{(0,1)}\hat{F}_{2}^{(1,0)}+\hat{F}_{2}^{(1,1)}.
\end{eqnarray*}
%16

\item \begin{eqnarray*} &&\frac{\partial}{\partial
w_2}\frac{\partial}{\partial
w_2}\left(R_2\left(P_1\left(z_1\right)\bar{P}_2\left(z_2\right)\hat{P}_1\left(w_1\right)\hat{P}_2\left(w_2\right)\right)
F_2\left(z_1,\theta_2\left(P_1\left(z_1\right)\hat{P}_1\left(w_1\right)\hat{P}_2\left(w_2\right)\right)\right)\hat{F}_2\left(w_1,w_2\right)\right)\\
&=&\hat{\mu}_{2}^{2}R_{2}^{(2)}(1)+r_{2}\hat{P}_{2}^{(2)}\left(1\right)+2r_{2}\frac{\hat{\mu}_{2}^{2}}{1-\tilde{\mu}_{2}}F_{2}^{(0,1)}+\hat{\mu}_{2}^{2}\tilde{\theta}_{2}^{(2)}\left(1\right)F_{2}^{(0,1)}+\frac{1}{1-\tilde{\mu}_{2}}\hat{P}_{2}^{(2)}\left(1\right)F_{2}^{(0,1)}\\
&+&2r_{2}\hat{\mu}_{2}\hat{F}_{2}^{(0,1)}+2\frac{\hat{\mu}_{2}}{1-\tilde{\mu}_{2}}F_{2}^{(0,1)}\hat{F}_{2}^{(0,1)}+\left(\frac{\hat{\mu}_{2}}{1-\tilde{\mu}_{2}}\right)^{2}F_{2}^{(0,2)}+\hat{F}_{2}^{(0,2)}.
\end{eqnarray*}
\end{enumerate}
%___________________________________________________________________________________________
%
\subsection{Derivadas de Segundo Orden para $F_{2}$}
%___________________________________________________________________________________________


\begin{enumerate}

%___________________________________________________________________________________________
\subsubsection{Mixtas para $z_{1}$:}
%___________________________________________________________________________________________

%1/17
\item \begin{eqnarray*} &&\frac{\partial}{\partial
z_1}\frac{\partial}{\partial
z_1}\left(R_1\left(P_1\left(z_1\right)\bar{P}_2\left(z_2\right)\hat{P}_1\left(w_1\right)\hat{P}_2\left(w_2\right)\right)
F_1\left(\theta_1\left(\tilde{P}_2\left(z_1\right)\hat{P}_1\left(w_1\right)\hat{P}_2\left(w_2\right)\right)\right)\hat{F}_1\left(w_1,w_2\right)\right)\\
&=&r_{1}P_{1}^{(2)}\left(1\right)+\mu_{1}^{2}R_{1}^{(2)}\left(1\right).
\end{eqnarray*}

%2/18
\item \begin{eqnarray*} &&\frac{\partial}{\partial
z_2}\frac{\partial}{\partial
z_1}\left(R_1\left(P_1\left(z_1\right)\bar{P}_2\left(z_2\right)\hat{P}_1\left(w_1\right)\hat{P}_2\left(w_2\right)\right)F_1\left(\theta_1\left(\tilde{P}_2\left(z_1\right)\hat{P}_1\left(w_1\right)\hat{P}_2\left(w_2\right)\right)\right)\hat{F}_1\left(w_1,w_2\right)\right)\\
&=&\mu_{1}\tilde{\mu}_{2}r_{1}+\mu_{1}\tilde{\mu}_{2}R_{1}^{(2)}(1)+
r_{1}\mu_{1}\left(\frac{\tilde{\mu}_{2}}{1-\mu_{1}}F_{1}^{(1,0)}+F_{1}^{(0,1)}\right).
\end{eqnarray*}

%3/19
\item \begin{eqnarray*} &&\frac{\partial}{\partial
w_1}\frac{\partial}{\partial
z_1}\left(R_1\left(P_1\left(z_1\right)\bar{P}_2\left(z_2\right)\hat{P}_1\left(w_1\right)\hat{P}_2\left(w_2\right)\right)F_1\left(\theta_1\left(\tilde{P}_2\left(z_1\right)\hat{P}_1\left(w_1\right)\hat{P}_2\left(w_2\right)\right)\right)\hat{F}_1\left(w_1,w_2\right)\right)\\
&=&r_{1}\mu_{1}\hat{\mu}_{1}+\mu_{1}\hat{\mu}_{1}R_{1}^{(2)}\left(1\right)+r_{1}\frac{\mu_{1}\hat{\mu}_{1}}{1-\mu_{1}}F_{1}^{(1,0)}+r_{1}\mu_{1}\hat{F}_{1}^{(1,0)}.
\end{eqnarray*}
%4/20
\item \begin{eqnarray*} &&\frac{\partial}{\partial
w_2}\frac{\partial}{\partial
z_1}\left(R_1\left(P_1\left(z_1\right)\bar{P}_2\left(z_2\right)\hat{P}_1\left(w_1\right)\hat{P}_2\left(w_2\right)\right)F_1\left(\theta_1\left(\tilde{P}_2\left(z_1\right)\hat{P}_1\left(w_1\right)\hat{P}_2\left(w_2\right)\right)\right)\hat{F}_1\left(w_1,w_2\right)\right)\\
&=&\mu_{1}\hat{\mu}_{2}r_{1}+\mu_{1}\hat{\mu}_{2}R_{1}^{(2)}\left(1\right)+r_{1}\mu_{1}\hat{F}_{1}^{(0,1)}+r_{1}\frac{\mu_{1}\hat{\mu}_{2}}{1-\mu_{1}}F_{1}^{(1,0)}.
\end{eqnarray*}
%___________________________________________________________________________________________
\subsubsection{Mixtas para $z_{2}$:}
%___________________________________________________________________________________________
%5/21
\item \begin{eqnarray*}
&&\frac{\partial}{\partial z_1}\frac{\partial}{\partial z_2}\left(R_1\left(P_1\left(z_1\right)\bar{P}_2\left(z_2\right)\hat{P}_1\left(w_1\right)\hat{P}_2\left(w_2\right)\right)F_1\left(\theta_1\left(\tilde{P}_2\left(z_1\right)\hat{P}_1\left(w_1\right)\hat{P}_2\left(w_2\right)\right)\right)\hat{F}_1\left(w_1,w_2\right)\right)\\
&=&r_{1}\mu_{1}\tilde{\mu}_{2}+\mu_{1}\tilde{\mu}_{2}R_{1}^{(2)}\left(1\right)+r_{1}\mu_{1}\left(\frac{\tilde{\mu}_{2}}{1-\mu_{1}}F_{1}^{(1,0)}+F_{1}^{(0,1)}\right).
\end{eqnarray*}

%6/22
\item \begin{eqnarray*}
&&\frac{\partial}{\partial z_2}\frac{\partial}{\partial z_2}\left(R_1\left(P_1\left(z_1\right)\bar{P}_2\left(z_2\right)\hat{P}_1\left(w_1\right)\hat{P}_2\left(w_2\right)\right)F_1\left(\theta_1\left(\tilde{P}_2\left(z_1\right)\hat{P}_1\left(w_1\right)\hat{P}_2\left(w_2\right)\right)\right)\hat{F}_1\left(w_1,w_2\right)\right)\\
&=&\tilde{\mu}_{2}^{2}R_{1}^{(2)}\left(1\right)+r_{1}\tilde{P}_{2}^{(2)}\left(1\right)+2r_{1}\tilde{\mu}_{2}\left(\frac{\tilde{\mu}_{2}}{1-\mu_{1}}F_{1}^{(1,0)}+F_{1}^{(0,1)}\right)+F_{1}^{(0,2)}+\tilde{\mu}_{2}^{2}\theta_{1}^{(2)}\left(1\right)F_{1}^{(1,0)}\\
&+&\frac{1}{1-\mu_{1}}\tilde{P}_{2}^{(2)}\left(1\right)F_{1}^{(1,0)}+\frac{\tilde{\mu}_{2}}{1-\mu_{1}}F_{1}^{(1,1)}+\frac{\tilde{\mu}_{2}}{1-\mu_{1}}\left(\frac{\tilde{\mu}_{2}}{1-\mu_{1}}F_{1}^{(2,0)}+F_{1}^{(1,1)}\right).
\end{eqnarray*}
%7/23
\item \begin{eqnarray*}
&&\frac{\partial}{\partial w_1}\frac{\partial}{\partial z_2}\left(R_1\left(P_1\left(z_1\right)\bar{P}_2\left(z_2\right)\hat{P}_1\left(w_1\right)\hat{P}_2\left(w_2\right)\right)F_1\left(\theta_1\left(\tilde{P}_2\left(z_1\right)\hat{P}_1\left(w_1\right)\hat{P}_2\left(w_2\right)\right)\right)\hat{F}_1\left(w_1,w_2\right)\right)\\
&=&\tilde{\mu}_{2}\hat{\mu}_{1}r_{1}+\tilde{\mu}_{2}\hat{\mu}_{1}R_{1}^{(2)}\left(1\right)+r_{1}\frac{\tilde{\mu}_{2}\hat{\mu}_{1}}{1-\mu_{1}}F_{1}^{(1,0)}+\hat{\mu}_{1}r_{1}\left(\frac{\tilde{\mu}_{2}}{1-\mu_{1}}F_{1}^{(1,0)}+F_{1}^{(0,1)}\right)+r_{1}\tilde{\mu}_{2}\hat{F}_{1}^{(1,0)}\\
&+&\left(\frac{\tilde{\mu}_{2}}{1-\mu_{1}}F_{1}^{(1,0)}+F_{1}^{(0,1)}\right)\hat{F}_{1}^{(1,0)}+\frac{\tilde{\mu}_{2}\hat{\mu}_{1}}{1-\mu_{1}}F_{1}^{(1,0)}+\tilde{\mu}_{2}\hat{\mu}_{1}\theta_{1}^{(2)}\left(1\right)F_{1}^{(1,0)}+\frac{\hat{\mu}_{1}}{1-\mu_{1}}F_{1}^{(1,1)}\\
&+&\left(\frac{1}{1-\mu_{1}}\right)^{2}\tilde{\mu}_{2}\hat{\mu}_{1}F_{1}^{(2,0)}.
\end{eqnarray*}
%8/24
\item \begin{eqnarray*}
&&\frac{\partial}{\partial w_2}\frac{\partial}{\partial z_2}\left(R_1\left(P_1\left(z_1\right)\bar{P}_2\left(z_2\right)\hat{P}_1\left(w_1\right)\hat{P}_2\left(w_2\right)\right)F_1\left(\theta_1\left(\tilde{P}_2\left(z_1\right)\hat{P}_1\left(w_1\right)\hat{P}_2\left(w_2\right)\right)\right)\hat{F}_1\left(w_1,w_2\right)\right)\\
&=&\hat{\mu}_{2}\tilde{\mu}_{2}r_{1}+\hat{\mu}_{2}\tilde{\mu}_{2}R_{1}^{(2)}(1)+r_{1}\tilde{\mu}_{2}\hat{F}_{1}^{(0,1)}+r_{1}\frac{\hat{\mu}_{2}\tilde{\mu}_{2}}{1-\mu_{1}}F_{1}^{(1,0)}+\hat{\mu}_{2}r_{1}\left(\frac{\tilde{\mu}_{2}}{1-\mu_{1}}F_{1}^{(1,0)}+F_{1}^{(0,1)}\right)\\
&+&\left(\frac{\tilde{\mu}_{2}}{1-\mu_{1}}F_{1}^{(1,0)}+F_{1}^{(0,1)}\right)\hat{F}_{1}^{(0,1)}+\frac{\tilde{\mu}_{2}\hat{\mu_{2}}}{1-\mu_{1}}F_{1}^{(1,0)}+\hat{\mu}_{2}\tilde{\mu}_{2}\theta_{1}^{(2)}\left(1\right)F_{1}^{(1,0)}+\frac{\hat{\mu}_{2}}{1-\mu_{1}}F_{1}^{(1,1)}\\
&+&\left(\frac{1}{1-\mu_{1}}\right)^{2}\tilde{\mu}_{2}\hat{\mu}_{2}F_{1}^{(2,0)}.
\end{eqnarray*}
%___________________________________________________________________________________________
\subsubsection{Mixtas para $w_{1}$:}
%___________________________________________________________________________________________
%9/25
\item \begin{eqnarray*} &&\frac{\partial}{\partial
z_1}\frac{\partial}{\partial
w_1}\left(R_1\left(P_1\left(z_1\right)\bar{P}_2\left(z_2\right)\hat{P}_1\left(w_1\right)\hat{P}_2\left(w_2\right)\right)F_1\left(\theta_1\left(\tilde{P}_2\left(z_1\right)\hat{P}_1\left(w_1\right)\hat{P}_2\left(w_2\right)\right)\right)\hat{F}_1\left(w_1,w_2\right)\right)\\
&=&r_{1}\mu_{1}\hat{\mu}_{1}+\mu_{1}\hat{\mu}_{1}R_{1}^{(2)}(1)+r_{1}\frac{\mu_{1}\hat{\mu}_{1}}{1-\mu_{1}}F_{1}^{(1,0)}+r_{1}\mu_{1}\hat{F}_{1}^{(1,0)}.
\end{eqnarray*}
%10/26
\item \begin{eqnarray*} &&\frac{\partial}{\partial
z_2}\frac{\partial}{\partial
w_1}\left(R_1\left(P_1\left(z_1\right)\bar{P}_2\left(z_2\right)\hat{P}_1\left(w_1\right)\hat{P}_2\left(w_2\right)\right)F_1\left(\theta_1\left(\tilde{P}_2\left(z_1\right)\hat{P}_1\left(w_1\right)\hat{P}_2\left(w_2\right)\right)\right)\hat{F}_1\left(w_1,w_2\right)\right)\\
&=&r_{1}\hat{\mu}_{1}\tilde{\mu}_{2}+\tilde{\mu}_{2}\hat{\mu}_{1}R_{1}^{(2)}\left(1\right)+
\frac{\hat{\mu}_{1}\tilde{\mu}_{2}}{1-\mu_{1}}F_{1}^{1,0)}+r_{1}\frac{\hat{\mu}_{1}\tilde{\mu}_{2}}{1-\mu_{1}}F_{1}^{(1,0)}+\hat{\mu}_{1}\tilde{\mu}_{2}\theta_{1}^{(2)}\left(1\right)F_{2}^{(1,0)}\\
&+&r_{1}\hat{\mu}_{1}\left(F_{1}^{(1,0)}+\frac{\tilde{\mu}_{2}}{1-\mu_{1}}F_{1}^{1,0)}\right)+
r_{1}\tilde{\mu}_{2}\hat{F}_{1}^{(1,0)}+\left(F_{1}^{(0,1)}+\frac{\tilde{\mu}_{2}}{1-\mu_{1}}F_{1}^{1,0)}\right)\hat{F}_{1}^{(1,0)}\\
&+&\frac{\hat{\mu}_{1}}{1-\mu_{1}}\left(F_{1}^{(1,1)}+\frac{\tilde{\mu}_{2}}{1-\mu_{1}}F_{1}^{2,0)}\right).
\end{eqnarray*}
%11/27
\item \begin{eqnarray*} &&\frac{\partial}{\partial
w_1}\frac{\partial}{\partial
w_1}\left(R_1\left(P_1\left(z_1\right)\bar{P}_2\left(z_2\right)\hat{P}_1\left(w_1\right)\hat{P}_2\left(w_2\right)\right)F_1\left(\theta_1\left(\tilde{P}_2\left(z_1\right)\hat{P}_1\left(w_1\right)\hat{P}_2\left(w_2\right)\right)\right)\hat{F}_1\left(w_1,w_2\right)\right)\\
&=&\hat{\mu}_{1}^{2}R_{1}^{(2)}\left(1\right)+r_{1}\hat{P}_{1}^{(2)}\left(1\right)+2r_{1}\frac{\hat{\mu}_{1}^{2}}{1-\mu_{1}}F_{1}^{(1,0)}+\hat{\mu}_{1}^{2}\theta_{1}^{(2)}\left(1\right)F_{1}^{(1,0)}+\frac{1}{1-\mu_{1}}\hat{P}_{1}^{(2)}\left(1\right)F_{1}^{(1,0)}\\
&+&2r_{1}\hat{\mu}_{1}\hat{F}_{1}^{(1,0)}+2\frac{\hat{\mu}_{1}}{1-\mu_{1}}F_{1}^{(1,0)}\hat{F}_{1}^{(1,0)}+\left(\frac{\hat{\mu}_{1}}{1-\mu_{1}}\right)^{2}F_{1}^{(2,0)}+\hat{F}_{1}^{(2,0)}.
\end{eqnarray*}
%12/28
\item \begin{eqnarray*} &&\frac{\partial}{\partial
w_2}\frac{\partial}{\partial
w_1}\left(R_1\left(P_1\left(z_1\right)\bar{P}_2\left(z_2\right)\hat{P}_1\left(w_1\right)\hat{P}_2\left(w_2\right)\right)F_1\left(\theta_1\left(\tilde{P}_2\left(z_1\right)\hat{P}_1\left(w_1\right)\hat{P}_2\left(w_2\right)\right)\right)\hat{F}_1\left(w_1,w_2\right)\right)\\
&=&r_{1}\hat{\mu}_{1}\hat{\mu}_{2}+\hat{\mu}_{1}\hat{\mu}_{2}R_{1}^{(2)}\left(1\right)+r_{1}\hat{\mu}_{1}\hat{F}_{1}^{(0,1)}+
\frac{\hat{\mu}_{1}\hat{\mu}_{2}}{1-\mu_{1}}F_{1}^{(1,0)}+2r_{1}\frac{\hat{\mu}_{1}\hat{\mu}_{2}}{1-\mu_{1}}F_{1}^{1,0)}+\hat{\mu}_{1}\hat{\mu}_{2}\theta_{1}^{(2)}\left(1\right)F_{1}^{(1,0)}\\
&+&\frac{\hat{\mu}_{1}}{1-\mu_{1}}F_{1}^{(1,0)}\hat{F}_{1}^{(0,1)}+
r_{1}\hat{\mu}_{2}\hat{F}_{1}^{(1,0)}+\frac{\hat{\mu}_{2}}{1-\mu_{1}}\hat{F}_{1}^{(1,0)}F_{1}^{(1,0)}+\hat{F}_{1}^{(1,1)}+\hat{\mu}_{1}\hat{\mu}_{2}\left(\frac{1}{1-\mu_{1}}\right)^{2}F_{1}^{(2,0)}.
\end{eqnarray*}
%___________________________________________________________________________________________
\subsubsection{Mixtas para $w_{2}$:}
%___________________________________________________________________________________________
%13/29
\item \begin{eqnarray*} &&\frac{\partial}{\partial
z_1}\frac{\partial}{\partial
w_2}\left(R_1\left(P_1\left(z_1\right)\bar{P}_2\left(z_2\right)\hat{P}_1\left(w_1\right)\hat{P}_2\left(w_2\right)\right)F_1\left(\theta_1\left(\tilde{P}_2\left(z_1\right)\hat{P}_1\left(w_1\right)\hat{P}_2\left(w_2\right)\right)\right)\hat{F}_1\left(w_1,w_2\right)\right)\\
&=&r_{1}\mu_{1}\hat{\mu}_{2}+\mu_{1}\hat{\mu}_{2}R_{1}^{(2)}\left(1\right)+r_{1}\mu_{1}\hat{F}_{1}^{(0,1)}+r_{1}\frac{\mu_{1}\hat{\mu}_{2}}{1-\mu_{1}}F_{1}^{(1,0)}.
\end{eqnarray*}
%14/30
\item \begin{eqnarray*} &&\frac{\partial}{\partial
z_2}\frac{\partial}{\partial
w_2}\left(R_1\left(P_1\left(z_1\right)\bar{P}_2\left(z_2\right)\hat{P}_1\left(w_1\right)\hat{P}_2\left(w_2\right)\right)F_1\left(\theta_1\left(\tilde{P}_2\left(z_1\right)\hat{P}_1\left(w_1\right)\hat{P}_2\left(w_2\right)\right)\right)\hat{F}_1\left(w_1,w_2\right)\right)\\
&=&r_{1}\hat{\mu}_{2}\tilde{\mu}_{2}+\hat{\mu}_{2}\tilde{\mu}_{2}R_{1}^{(2)}\left(1\right)+r_{1}\tilde{\mu}_{2}\hat{F}_{1}^{(0,1)}+\frac{\hat{\mu}_{2}\tilde{\mu}_{2}}{1-\mu_{1}}F_{1}^{(1,0)}+r_{1}\frac{\hat{\mu}_{2}\tilde{\mu}_{2}}{1-\mu_{1}}F_{1}^{(1,0)}\\
&+&\hat{\mu}_{2}\tilde{\mu}_{2}\theta_{1}^{(2)}\left(1\right)F_{1}^{(1,0)}+r_{1}\hat{\mu}_{2}\left(F_{1}^{(0,1)}+\frac{\tilde{\mu}_{2}}{1-\mu_{1}}F_{1}^{(1,0)}\right)+\left(F_{1}^{(0,1)}+\frac{\tilde{\mu}_{2}}{1-\mu_{1}}F_{1}^{(1,0)}\right)\hat{F}_{1}^{(0,1)}\\&+&\frac{\hat{\mu}_{2}}{1-\mu_{1}}\left(F_{1}^{(1,1)}+\frac{\tilde{\mu}_{2}}{1-\mu_{1}}F_{1}^{(2,0)}\right).
\end{eqnarray*}
%15/31
\item \begin{eqnarray*} &&\frac{\partial}{\partial
w_1}\frac{\partial}{\partial
w_2}\left(R_1\left(P_1\left(z_1\right)\bar{P}_2\left(z_2\right)\hat{P}_1\left(w_1\right)\hat{P}_2\left(w_2\right)\right)F_1\left(\theta_1\left(\tilde{P}_2\left(z_1\right)\hat{P}_1\left(w_1\right)\hat{P}_2\left(w_2\right)\right)\right)\hat{F}_1\left(w_1,w_2\right)\right)\\
&=&r_{1}\hat{\mu}_{1}\hat{\mu}_{2}+\hat{\mu}_{1}\hat{\mu}_{2}R_{1}^{(2)}\left(1\right)+r_{1}\hat{\mu}_{1}\hat{F}_{1}^{(0,1)}+
\frac{\hat{\mu}_{1}\hat{\mu}_{2}}{1-\mu_{1}}F_{1}^{(1,0)}+2r_{1}\frac{\hat{\mu}_{1}\hat{\mu}_{2}}{1-\mu_{1}}F_{1}^{(1,0)}+\hat{\mu}_{1}\hat{\mu}_{2}\theta_{1}^{(2)}\left(1\right)F_{1}^{(1,0)}\\
&+&\frac{\hat{\mu}_{1}}{1-\mu_{1}}\hat{F}_{1}^{(0,1)}F_{1}^{(1,0)}+r_{1}\hat{\mu}_{2}\hat{F}_{1}^{(1,0)}+\frac{\hat{\mu}_{2}}{1-\mu_{1}}\hat{F}_{1}^{(1,0)}F_{1}^{(1,0)}+\hat{F}_{1}^{(1,1)}+\hat{\mu}_{1}\hat{\mu}_{2}\left(\frac{1}{1-\mu_{1}}\right)^{2}F_{1}^{(2,0)}.
\end{eqnarray*}
%16/32
\item \begin{eqnarray*} &&\frac{\partial}{\partial
w_2}\frac{\partial}{\partial
w_2}\left(R_1\left(P_1\left(z_1\right)\bar{P}_2\left(z_2\right)\hat{P}_1\left(w_1\right)\hat{P}_2\left(w_2\right)\right)F_1\left(\theta_1\left(\tilde{P}_2\left(z_1\right)\hat{P}_1\left(w_1\right)\hat{P}_2\left(w_2\right)\right)\right)\hat{F}_1\left(w_1,w_2\right)\right)\\
&=&\hat{\mu}_{2}R_{1}^{(2)}\left(1\right)+r_{1}\hat{P}_{2}^{(2)}\left(1\right)+2r_{1}\hat{\mu}_{2}\hat{F}_{1}^{(0,1)}+\hat{F}_{1}^{(0,2)}+2r_{1}\frac{\hat{\mu}_{2}^{2}}{1-\mu_{1}}F_{1}^{(1,0)}+\hat{\mu}_{2}^{2}\theta_{1}^{(2)}\left(1\right)F_{1}^{(1,0)}\\
&+&\frac{1}{1-\mu_{1}}\hat{P}_{2}^{(2)}\left(1\right)F_{1}^{(1,0)} +
2\frac{\hat{\mu}_{2}}{1-\mu_{1}}F_{1}^{(1,0)}\hat{F}_{1}^{(0,1)}+\left(\frac{\hat{\mu}_{2}}{1-\mu_{1}}\right)^{2}F_{1}^{(2,0)}.
\end{eqnarray*}
\end{enumerate}

%___________________________________________________________________________________________
%
\subsection{Derivadas de Segundo Orden para $\hat{F}_{1}$}
%___________________________________________________________________________________________


\begin{enumerate}
%___________________________________________________________________________________________
\subsubsection{Mixtas para $z_{1}$:}
%___________________________________________________________________________________________
%1/33

\item \begin{eqnarray*} &&\frac{\partial}{\partial
z_1}\frac{\partial}{\partial
z_1}\left(\hat{R}_{2}\left(P_{1}\left(z_{1}\right)\tilde{P}_{2}\left(z_{2}\right)\hat{P}_{1}\left(w_{1}\right)\hat{P}_{2}\left(w_{2}\right)\right)\hat{F}_{2}\left(w_{1},\hat{\theta}_{2}\left(P_{1}\left(z_{1}\right)\tilde{P}_{2}\left(z_{2}\right)\hat{P}_{1}\left(w_{1}\right)\right)\right)F_{2}\left(z_{1},z_{2}\right)\right)\\
&=&\hat{r}_{2}P_{1}^{(2)}\left(1\right)+
\mu_{1}^{2}\hat{R}_{2}^{(2)}\left(1\right)+
2\hat{r}_{2}\frac{\mu_{1}^{2}}{1-\hat{\mu}_{2}}\hat{F}_{2}^{(0,1)}+
\frac{1}{1-\hat{\mu}_{2}}P_{1}^{(2)}\left(1\right)\hat{F}_{2}^{(0,1)}+
\mu_{1}^{2}\hat{\theta}_{2}^{(2)}\left(1\right)\hat{F}_{2}^{(0,1)}\\
&+&\left(\frac{\mu_{1}^{2}}{1-\hat{\mu}_{2}}\right)^{2}\hat{F}_{2}^{(0,2)}+
2\hat{r}_{2}\mu_{1}F_{2}^{(1,0)}+2\frac{\mu_{1}}{1-\hat{\mu}_{2}}\hat{F}_{2}^{(0,1)}F_{2}^{(1,0)}+F_{2}^{(2,0)}.
\end{eqnarray*}

%2/34
\item \begin{eqnarray*} &&\frac{\partial}{\partial
z_2}\frac{\partial}{\partial
z_1}\left(\hat{R}_{2}\left(P_{1}\left(z_{1}\right)\tilde{P}_{2}\left(z_{2}\right)\hat{P}_{1}\left(w_{1}\right)\hat{P}_{2}\left(w_{2}\right)\right)\hat{F}_{2}\left(w_{1},\hat{\theta}_{2}\left(P_{1}\left(z_{1}\right)\tilde{P}_{2}\left(z_{2}\right)\hat{P}_{1}\left(w_{1}\right)\right)\right)F_{2}\left(z_{1},z_{2}\right)\right)\\
&=&\hat{r}_{2}\mu_{1}\tilde{\mu}_{2}+\mu_{1}\tilde{\mu}_{2}\hat{R}_{2}^{(2)}\left(1\right)+\hat{r}_{2}\mu_{1}F_{2}^{(0,1)}+
\frac{\mu_{1}\tilde{\mu}_{2}}{1-\hat{\mu}_{2}}\hat{F}_{2}^{(0,1)}+2\hat{r}_{2}\frac{\mu_{1}\tilde{\mu}_{2}}{1-\hat{\mu}_{2}}\hat{F}_{2}^{(0,1)}+\mu_{1}\tilde{\mu}_{2}\hat{\theta}_{2}^{(2)}\left(1\right)\hat{F}_{2}^{(0,1)}\\
&+&\frac{\mu_{1}}{1-\hat{\mu}_{2}}F_{2}^{(0,1)}\hat{F}_{2}^{(0,1)}+\mu_{1} \tilde{\mu}_{2}\left(\frac{1}{1-\hat{\mu}_{2}}\right)^{2}\hat{F}_{2}^{(0,2)}+\hat{r}_{2}\tilde{\mu}_{2}F_{2}^{(1,0)}+\frac{\tilde{\mu}_{2}}{1-\hat{\mu}_{2}}\hat{F}_{2}^{(0,1)}F_{2}^{(1,0)}+F_{2}^{(1,1)}.
\end{eqnarray*}


%3/35

\item \begin{eqnarray*} &&\frac{\partial}{\partial
w_1}\frac{\partial}{\partial
z_1}\left(\hat{R}_{2}\left(P_{1}\left(z_{1}\right)\tilde{P}_{2}\left(z_{2}\right)\hat{P}_{1}\left(w_{1}\right)\hat{P}_{2}\left(w_{2}\right)\right)\hat{F}_{2}\left(w_{1},\hat{\theta}_{2}\left(P_{1}\left(z_{1}\right)\tilde{P}_{2}\left(z_{2}\right)\hat{P}_{1}\left(w_{1}\right)\right)\right)F_{2}\left(z_{1},z_{2}\right)\right)\\
&=&\hat{r}_{2}\mu_{1}\hat{\mu}_{1}+\mu_{1}\hat{\mu}_{1}\hat{R}_{2}^{(2)}\left(1\right)+\hat{r}_{2}\frac{\mu_{1}\hat{\mu}_{1}}{1-\hat{\mu}_{2}}\hat{F}_{2}^{(0,1)}+\hat{r}_{2}\hat{\mu}_{1}F_{2}^{(1,0)}+\hat{r}_{2}\mu_{1}\hat{F}_{2}^{(1,0)}+F_{2}^{(1,0)}\hat{F}_{2}^{(1,0)}+\frac{\mu_{1}}{1-\hat{\mu}_{2}}\hat{F}_{2}^{(1,1)}.
\end{eqnarray*}

%4/36

\item \begin{eqnarray*} &&\frac{\partial}{\partial
w_2}\frac{\partial}{\partial
z_1}\left(\hat{R}_{2}\left(P_{1}\left(z_{1}\right)\tilde{P}_{2}\left(z_{2}\right)\hat{P}_{1}\left(w_{1}\right)\hat{P}_{2}\left(w_{2}\right)\right)\hat{F}_{2}\left(w_{1},\hat{\theta}_{2}\left(P_{1}\left(z_{1}\right)\tilde{P}_{2}\left(z_{2}\right)\hat{P}_{1}\left(w_{1}\right)\right)\right)F_{2}\left(z_{1},z_{2}\right)\right)\\
&=&\hat{r}_{2}\mu_{1}\hat{\mu}_{2}+\mu_{1}\hat{\mu}_{2}\hat{R}_{2}^{(2)}\left(1\right)+\frac{\mu_{1}\hat{\mu}_{2}}{1-\hat{\mu}_{2}}\hat{F}_{2}^{(0,1)}+2\hat{r}_{2}\frac{\mu_{1}\hat{\mu}_{2}}{1-\hat{\mu}_{2}}\hat{F}_{2}^{(0,1)}+\mu_{1}\hat{\mu}_{2}\hat{\theta}_{2}^{(2)}\left(1\right)\hat{F}_{2}^{(0,1)}\\
&+&\mu_{1}\hat{\mu}_{2}\left(\frac{1}{1-\hat{\mu}_{2}}\right)^{2}\hat{F}_{2}^{(0,2)}+\hat{r}_{2}\hat{\mu}_{2}F_{2}^{(1,0)}+\frac{\hat{\mu}_{2}}{1-\hat{\mu}_{2}}\hat{F}_{2}^{(0,1)}F_{2}^{(1,0)}.
\end{eqnarray*}
%___________________________________________________________________________________________
\subsubsection{Mixtas para $z_{2}$:}
%___________________________________________________________________________________________

%5/37

\item \begin{eqnarray*} &&\frac{\partial}{\partial
z_1}\frac{\partial}{\partial
z_2}\left(\hat{R}_{2}\left(P_{1}\left(z_{1}\right)\tilde{P}_{2}\left(z_{2}\right)\hat{P}_{1}\left(w_{1}\right)\hat{P}_{2}\left(w_{2}\right)\right)\hat{F}_{2}\left(w_{1},\hat{\theta}_{2}\left(P_{1}\left(z_{1}\right)\tilde{P}_{2}\left(z_{2}\right)\hat{P}_{1}\left(w_{1}\right)\right)\right)F_{2}\left(z_{1},z_{2}\right)\right)\\
&=&\hat{r}_{2}\mu_{1}\tilde{\mu}_{2}+\mu_{1}\tilde{\mu}_{2}\hat{R}_{2}^{(2)}\left(1\right)+\mu_{1}\hat{r}_{2}F_{2}^{(0,1)}+
\frac{\mu_{1}\tilde{\mu}_{2}}{1-\hat{\mu}_{2}}\hat{F}_{2}^{(0,1)}+2\hat{r}_{2}\frac{\mu_{1}\tilde{\mu}_{2}}{1-\hat{\mu}_{2}}\hat{F}_{2}^{(0,1)}+\mu_{1}\tilde{\mu}_{2}\hat{\theta}_{2}^{(2)}\left(1\right)\hat{F}_{2}^{(0,1)}\\
&+&\frac{\mu_{1}}{1-\hat{\mu}_{2}}F_{2}^{(0,1)}\hat{F}_{2}^{(0,1)}+\mu_{1}\tilde{\mu}_{2}\left(\frac{1}{1-\hat{\mu}_{2}}\right)^{2}\hat{F}_{2}^{(0,2)}+\hat{r}_{2}\tilde{\mu}_{2}F_{2}^{(1,0)}+\frac{\tilde{\mu}_{2}}{1-\hat{\mu}_{2}}\hat{F}_{2}^{(0,1)}F_{2}^{(1,0)}+F_{2}^{(1,1)}.
\end{eqnarray*}

%6/38

\item \begin{eqnarray*} &&\frac{\partial}{\partial
z_2}\frac{\partial}{\partial
z_2}\left(\hat{R}_{2}\left(P_{1}\left(z_{1}\right)\tilde{P}_{2}\left(z_{2}\right)\hat{P}_{1}\left(w_{1}\right)\hat{P}_{2}\left(w_{2}\right)\right)\hat{F}_{2}\left(w_{1},\hat{\theta}_{2}\left(P_{1}\left(z_{1}\right)\tilde{P}_{2}\left(z_{2}\right)\hat{P}_{1}\left(w_{1}\right)\right)\right)F_{2}\left(z_{1},z_{2}\right)\right)\\
&=&\hat{r}_{2}\tilde{P}_{2}^{(2)}\left(1\right)+\tilde{\mu}_{2}^{2}\hat{R}_{2}^{(2)}\left(1\right)+2\hat{r}_{2}\tilde{\mu}_{2}F_{2}^{(0,1)}+2\hat{r}_{2}\frac{\tilde{\mu}_{2}^{2}}{1-\hat{\mu}_{2}}\hat{F}_{2}^{(0,1)}+\frac{1}{1-\hat{\mu}_{2}}\tilde{P}_{2}^{(2)}\left(1\right)\hat{F}_{2}^{(0,1)}\\
&+&\tilde{\mu}_{2}^{2}\hat{\theta}_{2}^{(2)}\left(1\right)\hat{F}_{2}^{(0,1)}+2\frac{\tilde{\mu}_{2}}{1-\hat{\mu}_{2}}F_{2}^{(0,1)}\hat{F}_{2}^{(0,1)}+F_{2}^{(0,2)}+\left(\frac{\tilde{\mu}_{2}}{1-\hat{\mu}_{2}}\right)^{2}\hat{F}_{2}^{(0,2)}.
\end{eqnarray*}

%7/39

\item \begin{eqnarray*} &&\frac{\partial}{\partial
w_1}\frac{\partial}{\partial
z_2}\left(\hat{R}_{2}\left(P_{1}\left(z_{1}\right)\tilde{P}_{2}\left(z_{2}\right)\hat{P}_{1}\left(w_{1}\right)\hat{P}_{2}\left(w_{2}\right)\right)\hat{F}_{2}\left(w_{1},\hat{\theta}_{2}\left(P_{1}\left(z_{1}\right)\tilde{P}_{2}\left(z_{2}\right)\hat{P}_{1}\left(w_{1}\right)\right)\right)F_{2}\left(z_{1},z_{2}\right)\right)\\
&=&\hat{r}_{2}\tilde{\mu}_{2}\hat{\mu}_{1}+\tilde{\mu}_{2}\hat{\mu}_{1}\hat{R}_{2}^{(2)}\left(1\right)+\hat{r}_{2}\hat{\mu}_{1}F_{2}^{(0,1)}+\hat{r}_{2}\frac{\tilde{\mu}_{2}\hat{\mu}_{1}}{1-\hat{\mu}_{2}}\hat{F}_{2}^{(0,1)}+\hat{r}_{2}\tilde{\mu}_{2}\hat{F}_{2}^{(1,0)}+F_{2}^{(0,1)}\hat{F}_{2}^{(1,0)}+\frac{\tilde{\mu}_{2}}{1-\hat{\mu}_{2}}\hat{F}_{2}^{(1,1)}.
\end{eqnarray*}
%8/40

\item \begin{eqnarray*} &&\frac{\partial}{\partial
w_2}\frac{\partial}{\partial
z_2}\left(\hat{R}_{2}\left(P_{1}\left(z_{1}\right)\tilde{P}_{2}\left(z_{2}\right)\hat{P}_{1}\left(w_{1}\right)\hat{P}_{2}\left(w_{2}\right)\right)\hat{F}_{2}\left(w_{1},\hat{\theta}_{2}\left(P_{1}\left(z_{1}\right)\tilde{P}_{2}\left(z_{2}\right)\hat{P}_{1}\left(w_{1}\right)\right)\right)F_{2}\left(z_{1},z_{2}\right)\right)\\
&=&\hat{r}_{2}\tilde{\mu}_{2}\hat{\mu}_{2}+\tilde{\mu}_{2}\hat{\mu}_{2}\hat{R}_{2}^{(2)}\left(1\right)+\hat{r}_{2}\hat{\mu}_{2}F_{2}^{(0,1)}+
\frac{\tilde{\mu}_{2}\hat{\mu}_{2}}{1-\hat{\mu}_{2}}\hat{F}_{2}^{(0,1)}+2\hat{r}_{2}\frac{\tilde{\mu}_{2}\hat{\mu}_{2}}{1-\hat{\mu}_{2}}\hat{F}_{2}^{(0,1)}+\tilde{\mu}_{2}\hat{\mu}_{2}\hat{\theta}_{2}^{(2)}\left(1\right)\hat{F}_{2}^{(0,1)}\\
&+&\frac{\hat{\mu}_{2}}{1-\hat{\mu}_{2}}F_{2}^{(0,1)}\hat{F}_{2}^{(1,0)}+\tilde{\mu}_{2}\hat{\mu}_{2}\left(\frac{1}{1-\hat{\mu}_{2}}\right)\hat{F}_{2}^{(0,2)}.
\end{eqnarray*}
%___________________________________________________________________________________________
\subsubsection{Mixtas para $w_{1}$:}
%___________________________________________________________________________________________

%9/41
\item \begin{eqnarray*} &&\frac{\partial}{\partial
z_1}\frac{\partial}{\partial
w_1}\left(\hat{R}_{2}\left(P_{1}\left(z_{1}\right)\tilde{P}_{2}\left(z_{2}\right)\hat{P}_{1}\left(w_{1}\right)\hat{P}_{2}\left(w_{2}\right)\right)\hat{F}_{2}\left(w_{1},\hat{\theta}_{2}\left(P_{1}\left(z_{1}\right)\tilde{P}_{2}\left(z_{2}\right)\hat{P}_{1}\left(w_{1}\right)\right)\right)F_{2}\left(z_{1},z_{2}\right)\right)\\
&=&\hat{r}_{2}\mu_{1}\hat{\mu}_{1}+\mu_{1}\hat{\mu}_{1}\hat{R}_{2}^{(2)}\left(1\right)+\hat{r}_{2}\frac{\mu_{1}\hat{\mu}_{1}}{1-\hat{\mu}_{2}}\hat{F}_{2}^{(0,1)}+\hat{r}_{2}\hat{\mu}_{1}F_{2}^{(1,0)}+\hat{r}_{2}\mu_{1}\hat{F}_{2}^{(1,0)}+F_{2}^{(1,0)}\hat{F}_{2}^{(1,0)}+\frac{\mu_{1}}{1-\hat{\mu}_{2}}\hat{F}_{2}^{(1,1)}.
\end{eqnarray*}


%10/42
\item \begin{eqnarray*} &&\frac{\partial}{\partial
z_2}\frac{\partial}{\partial
w_1}\left(\hat{R}_{2}\left(P_{1}\left(z_{1}\right)\tilde{P}_{2}\left(z_{2}\right)\hat{P}_{1}\left(w_{1}\right)\hat{P}_{2}\left(w_{2}\right)\right)\hat{F}_{2}\left(w_{1},\hat{\theta}_{2}\left(P_{1}\left(z_{1}\right)\tilde{P}_{2}\left(z_{2}\right)\hat{P}_{1}\left(w_{1}\right)\right)\right)F_{2}\left(z_{1},z_{2}\right)\right)\\
&=&\hat{r}_{2}\tilde{\mu}_{2}\hat{\mu}_{1}+\tilde{\mu}_{2}\hat{\mu}_{1}\hat{R}_{2}^{(2)}\left(1\right)+\hat{r}_{2}\hat{\mu}_{1}F_{2}^{(0,1)}+
\hat{r}_{2}\frac{\tilde{\mu}_{2}\hat{\mu}_{1}}{1-\hat{\mu}_{2}}\hat{F}_{2}^{(0,1)}+\hat{r}_{2}\tilde{\mu}_{2}\hat{F}_{2}^{(1,0)}+F_{2}^{(0,1)}\hat{F}_{2}^{(1,0)}+\frac{\tilde{\mu}_{2}}{1-\hat{\mu}_{2}}\hat{F}_{2}^{(1,1)}.
\end{eqnarray*}


%11/43
\item \begin{eqnarray*} &&\frac{\partial}{\partial
w_1}\frac{\partial}{\partial
w_1}\left(\hat{R}_{2}\left(P_{1}\left(z_{1}\right)\tilde{P}_{2}\left(z_{2}\right)\hat{P}_{1}\left(w_{1}\right)\hat{P}_{2}\left(w_{2}\right)\right)\hat{F}_{2}\left(w_{1},\hat{\theta}_{2}\left(P_{1}\left(z_{1}\right)\tilde{P}_{2}\left(z_{2}\right)\hat{P}_{1}\left(w_{1}\right)\right)\right)F_{2}\left(z_{1},z_{2}\right)\right)\\
&=&\hat{r}_{2}\hat{P}_{1}^{(2)}\left(1\right)+\hat{\mu}_{1}^{2}\hat{R}_{2}^{(2)}\left(1\right)+2\hat{r}_{2}\hat{\mu}_{1}\hat{F}_{2}^{(1,0)}
+\hat{F}_{2}^{(2,0)}.
\end{eqnarray*}


%12/44
\item \begin{eqnarray*} &&\frac{\partial}{\partial
w_2}\frac{\partial}{\partial
w_1}\left(\hat{R}_{2}\left(P_{1}\left(z_{1}\right)\tilde{P}_{2}\left(z_{2}\right)\hat{P}_{1}\left(w_{1}\right)\hat{P}_{2}\left(w_{2}\right)\right)\hat{F}_{2}\left(w_{1},\hat{\theta}_{2}\left(P_{1}\left(z_{1}\right)\tilde{P}_{2}\left(z_{2}\right)\hat{P}_{1}\left(w_{1}\right)\right)\right)F_{2}\left(z_{1},z_{2}\right)\right)\\
&=&\hat{r}_{2}\hat{\mu}_{1}\hat{\mu}_{2}+\hat{\mu}_{1}\hat{\mu}_{2}\hat{R}_{2}^{(2)}\left(1\right)+
\hat{r}_{2}\frac{\hat{\mu}_{2}\hat{\mu}_{1}}{1-\hat{\mu}_{2}}\hat{F}_{2}^{(0,1)}
+\hat{r}_{2}\hat{\mu}_{2}\hat{F}_{2}^{(1,0)}+\frac{\hat{\mu}_{2}}{1-\hat{\mu}_{2}}\hat{F}_{2}^{(1,1)}.
\end{eqnarray*}
%___________________________________________________________________________________________
\subsubsection{Mixtas para $w_{2}$:}
%___________________________________________________________________________________________
%13/45
\item \begin{eqnarray*} &&\frac{\partial}{\partial
z_1}\frac{\partial}{\partial
w_2}\left(\hat{R}_{2}\left(P_{1}\left(z_{1}\right)\tilde{P}_{2}\left(z_{2}\right)\hat{P}_{1}\left(w_{1}\right)\hat{P}_{2}\left(w_{2}\right)\right)\hat{F}_{2}\left(w_{1},\hat{\theta}_{2}\left(P_{1}\left(z_{1}\right)\tilde{P}_{2}\left(z_{2}\right)\hat{P}_{1}\left(w_{1}\right)\right)\right)F_{2}\left(z_{1},z_{2}\right)\right)\\
&=&\hat{r}_{2}\mu_{1}\hat{\mu}_{2}+\mu_{1}\hat{\mu}_{2}\hat{R}_{2}^{(2)}\left(1\right)+
\frac{\mu_{1}\hat{\mu}_{2}}{1-\hat{\mu}_{2}}\hat{F}_{2}^{(0,1)} +2\hat{r}_{2}\frac{\mu_{1}\hat{\mu}_{2}}{1-\hat{\mu}_{2}}\hat{F}_{2}^{(0,1)}\\
&+&\mu_{1}\hat{\mu}_{2}\hat{\theta}_{2}^{(2)}\left(1\right)\hat{F}_{2}^{(0,1)}+\mu_{1}\hat{\mu}_{2}\left(\frac{1}{1-\hat{\mu}_{2}}\right)^{2}\hat{F}_{2}^{(0,2)}+\hat{r}_{2}\hat{\mu}_{2}F_{2}^{(1,0)}+\frac{\hat{\mu}_{2}}{1-\hat{\mu}_{2}}\hat{F}_{2}^{(0,1)}F_{2}^{(1,0)}.\end{eqnarray*}


%14/46
\item \begin{eqnarray*} &&\frac{\partial}{\partial
z_2}\frac{\partial}{\partial
w_2}\left(\hat{R}_{2}\left(P_{1}\left(z_{1}\right)\tilde{P}_{2}\left(z_{2}\right)\hat{P}_{1}\left(w_{1}\right)\hat{P}_{2}\left(w_{2}\right)\right)\hat{F}_{2}\left(w_{1},\hat{\theta}_{2}\left(P_{1}\left(z_{1}\right)\tilde{P}_{2}\left(z_{2}\right)\hat{P}_{1}\left(w_{1}\right)\right)\right)F_{2}\left(z_{1},z_{2}\right)\right)\\
&=&\hat{r}_{2}\tilde{\mu}_{2}\hat{\mu}_{2}+\tilde{\mu}_{2}\hat{\mu}_{2}\hat{R}_{2}^{(2)}\left(1\right)+\hat{r}_{2}\hat{\mu}_{2}F_{2}^{(0,1)}+\frac{\tilde{\mu}_{2}\hat{\mu}_{2}}{1-\hat{\mu}_{2}}\hat{F}_{2}^{(0,1)}+
2\hat{r}_{2}\frac{\tilde{\mu}_{2}\hat{\mu}_{2}}{1-\hat{\mu}_{2}}\hat{F}_{2}^{(0,1)}+\tilde{\mu}_{2}\hat{\mu}_{2}\hat{\theta}_{2}^{(2)}\left(1\right)\hat{F}_{2}^{(0,1)}\\
&+&\frac{\hat{\mu}_{2}}{1-\hat{\mu}_{2}}\hat{F}_{2}^{(0,1)}F_{2}^{(0,1)}+\tilde{\mu}_{2}\hat{\mu}_{2}\left(\frac{1}{1-\hat{\mu}_{2}}\right)^{2}\hat{F}_{2}^{(0,2)}.
\end{eqnarray*}

%15/47

\item \begin{eqnarray*} &&\frac{\partial}{\partial
w_1}\frac{\partial}{\partial
w_2}\left(\hat{R}_{2}\left(P_{1}\left(z_{1}\right)\tilde{P}_{2}\left(z_{2}\right)\hat{P}_{1}\left(w_{1}\right)\hat{P}_{2}\left(w_{2}\right)\right)\hat{F}_{2}\left(w_{1},\hat{\theta}_{2}\left(P_{1}\left(z_{1}\right)\tilde{P}_{2}\left(z_{2}\right)\hat{P}_{1}\left(w_{1}\right)\right)\right)F_{2}\left(z_{1},z_{2}\right)\right)\\
&=&\hat{r}_{2}\hat{\mu}_{1}\hat{\mu}_{2}+\hat{\mu}_{1}\hat{\mu}_{2}\hat{R}_{2}^{(2)}\left(1\right)+
\hat{r}_{2}\frac{\hat{\mu}_{1}\hat{\mu}_{2}}{1-\hat{\mu}_{2}}\hat{F}_{2}^{(0,1)}+
\hat{r}_{2}\hat{\mu}_{2}\hat{F}_{2}^{(1,0)}+\frac{\hat{\mu}_{2}}{1-\hat{\mu}_{2}}\hat{F}_{2}^{(1,1)}.
\end{eqnarray*}

%16/48
\item \begin{eqnarray*} &&\frac{\partial}{\partial
w_2}\frac{\partial}{\partial
w_2}\left(\hat{R}_{2}\left(P_{1}\left(z_{1}\right)\tilde{P}_{2}\left(z_{2}\right)\hat{P}_{1}\left(w_{1}\right)\hat{P}_{2}\left(w_{2}\right)\right)\hat{F}_{2}\left(w_{1},\hat{\theta}_{2}\left(P_{1}\left(z_{1}\right)\tilde{P}_{2}\left(z_{2}\right)\hat{P}_{1}\left(w_{1}\right)\right)\right)F_{2}\left(z_{1},z_{2};\zeta_{2}\right)\right)\\
&=&\hat{r}_{2}P_{2}^{(2)}\left(1\right)+\hat{\mu}_{2}^{2}\hat{R}_{2}^{(2)}\left(1\right)+2\hat{r}_{2}\frac{\hat{\mu}_{2}^{2}}{1-\hat{\mu}_{2}}\hat{F}_{2}^{(0,1)}+\frac{1}{1-\hat{\mu}_{2}}\hat{P}_{2}^{(2)}\left(1\right)\hat{F}_{2}^{(0,1)}+\hat{\mu}_{2}^{2}\hat{\theta}_{2}^{(2)}\left(1\right)\hat{F}_{2}^{(0,1)}\\
&+&\left(\frac{\hat{\mu}_{2}}{1-\hat{\mu}_{2}}\right)^{2}\hat{F}_{2}^{(0,2)}.
\end{eqnarray*}


\end{enumerate}



%___________________________________________________________________________________________
%
\subsection{Derivadas de Segundo Orden para $\hat{F}_{2}$}
%___________________________________________________________________________________________
\begin{enumerate}
%___________________________________________________________________________________________
\subsubsection{Mixtas para $z_{1}$:}
%___________________________________________________________________________________________
%1/49

\item \begin{eqnarray*} &&\frac{\partial}{\partial
z_1}\frac{\partial}{\partial
z_1}\left(\hat{R}_{1}\left(P_{1}\left(z_{1}\right)\tilde{P}_{2}\left(z_{2}\right)\hat{P}_{1}\left(w_{1}\right)\hat{P}_{2}\left(w_{2}\right)\right)\hat{F}_{1}\left(\hat{\theta}_{1}\left(P_{1}\left(z_{1}\right)\tilde{P}_{2}\left(z_{2}\right)
\hat{P}_{2}\left(w_{2}\right)\right),w_{2}\right)F_{1}\left(z_{1},z_{2}\right)\right)\\
&=&\hat{r}_{1}P_{1}^{(2)}\left(1\right)+
\mu_{1}^{2}\hat{R}_{1}^{(2)}\left(1\right)+
2\hat{r}_{1}\mu_{1}F_{1}^{(1,0)}+
2\hat{r}_{1}\frac{\mu_{1}^{2}}{1-\hat{\mu}_{1}}\hat{F}_{1}^{(1,0)}+
\frac{1}{1-\hat{\mu}_{1}}P_{1}^{(2)}\left(1\right)\hat{F}_{1}^{(1,0)}+\mu_{1}^{2}\hat{\theta}_{1}^{(2)}\left(1\right)\hat{F}_{1}^{(1,0)}\\
&+&2\frac{\mu_{1}}{1-\hat{\mu}_{1}}\hat{F}_{1}^{(1,0)}F_{1}^{(1,0)}+F_{1}^{(2,0)}
+\left(\frac{\mu_{1}}{1-\hat{\mu}_{1}}\right)^{2}\hat{F}_{1}^{(2,0)}.
\end{eqnarray*}

%2/50

\item \begin{eqnarray*} &&\frac{\partial}{\partial
z_2}\frac{\partial}{\partial
z_1}\left(\hat{R}_{1}\left(P_{1}\left(z_{1}\right)\tilde{P}_{2}\left(z_{2}\right)\hat{P}_{1}\left(w_{1}\right)\hat{P}_{2}\left(w_{2}\right)\right)\hat{F}_{1}\left(\hat{\theta}_{1}\left(P_{1}\left(z_{1}\right)\tilde{P}_{2}\left(z_{2}\right)
\hat{P}_{2}\left(w_{2}\right)\right),w_{2}\right)F_{1}\left(z_{1},z_{2}\right)\right)\\
&=&\hat{r}_{1}\mu_{1}\tilde{\mu}_{2}+\mu_{1}\tilde{\mu}_{2}\hat{R}_{1}^{(2)}\left(1\right)+
\hat{r}_{1}\mu_{1}F_{1}^{(0,1)}+\tilde{\mu}_{2}\hat{r}_{1}F_{1}^{(1,0)}+
\frac{\mu_{1}\tilde{\mu}_{2}}{1-\hat{\mu}_{1}}\hat{F}_{1}^{(1,0)}+2\hat{r}_{1}\frac{\mu_{1}\tilde{\mu}_{2}}{1-\hat{\mu}_{1}}\hat{F}_{1}^{(1,0)}\\
&+&\mu_{1}\tilde{\mu}_{2}\hat{\theta}_{1}^{(2)}\left(1\right)\hat{F}_{1}^{(1,0)}+
\frac{\mu_{1}}{1-\hat{\mu}_{1}}\hat{F}_{1}^{(1,0)}F_{1}^{(0,1)}+
\frac{\tilde{\mu}_{2}}{1-\hat{\mu}_{1}}\hat{F}_{1}^{(1,0)}F_{1}^{(1,0)}+
F_{1}^{(1,1)}\\
&+&\mu_{1}\tilde{\mu}_{2}\left(\frac{1}{1-\hat{\mu}_{1}}\right)^{2}\hat{F}_{1}^{(2,0)}.
\end{eqnarray*}

%3/51

\item \begin{eqnarray*} &&\frac{\partial}{\partial
w_1}\frac{\partial}{\partial
z_1}\left(\hat{R}_{1}\left(P_{1}\left(z_{1}\right)\tilde{P}_{2}\left(z_{2}\right)\hat{P}_{1}\left(w_{1}\right)\hat{P}_{2}\left(w_{2}\right)\right)\hat{F}_{1}\left(\hat{\theta}_{1}\left(P_{1}\left(z_{1}\right)\tilde{P}_{2}\left(z_{2}\right)
\hat{P}_{2}\left(w_{2}\right)\right),w_{2}\right)F_{1}\left(z_{1},z_{2}\right)\right)\\
&=&\hat{r}_{1}\mu_{1}\hat{\mu}_{1}+\mu_{1}\hat{\mu}_{1}\hat{R}_{1}^{(2)}\left(1\right)+\hat{r}_{1}\hat{\mu}_{1}F_{1}^{(1,0)}+
\hat{r}_{1}\frac{\mu_{1}\hat{\mu}_{1}}{1-\hat{\mu}_{1}}\hat{F}_{1}^{(1,0)}.
\end{eqnarray*}

%4/52

\item \begin{eqnarray*} &&\frac{\partial}{\partial
w_2}\frac{\partial}{\partial
z_1}\left(\hat{R}_{1}\left(P_{1}\left(z_{1}\right)\tilde{P}_{2}\left(z_{2}\right)\hat{P}_{1}\left(w_{1}\right)\hat{P}_{2}\left(w_{2}\right)\right)\hat{F}_{1}\left(\hat{\theta}_{1}\left(P_{1}\left(z_{1}\right)\tilde{P}_{2}\left(z_{2}\right)
\hat{P}_{2}\left(w_{2}\right)\right),w_{2}\right)F_{1}\left(z_{1},z_{2}\right)\right)\\
&=&\hat{r}_{1}\mu_{1}\hat{\mu}_{2}+\mu_{1}\hat{\mu}_{2}\hat{R}_{1}^{(2)}\left(1\right)+\hat{r}_{1}\hat{\mu}_{2}F_{1}^{(1,0)}+\frac{\mu_{1}\hat{\mu}_{2}}{1-\hat{\mu}_{1}}\hat{F}_{1}^{(1,0)}+\hat{r}_{1}\frac{\mu_{1}\hat{\mu}_{2}}{1-\hat{\mu}_{1}}\hat{F}_{1}^{(1,0)}+\mu_{1}\hat{\mu}_{2}\hat{\theta}_{1}^{(2)}\left(1\right)\hat{F}_{1}^{(1,0)}\\
&+&\hat{r}_{1}\mu_{1}\left(\hat{F}_{1}^{(0,1)}+\frac{\hat{\mu}_{2}}{1-\hat{\mu}_{1}}\hat{F}_{1}^{(1,0)}\right)+F_{1}^{(1,0)}\left(\hat{F}_{1}^{(0,1)}+\frac{\hat{\mu}_{2}}{1-\hat{\mu}_{1}}\hat{F}_{1}^{(1,0)}\right)+\frac{\mu_{1}}{1-\hat{\mu}_{1}}\left(\hat{F}_{1}^{(1,1)}+\frac{\hat{\mu}_{2}}{1-\hat{\mu}_{1}}\hat{F}_{1}^{(2,0)}\right).
\end{eqnarray*}
%___________________________________________________________________________________________
\subsubsection{Mixtas para $z_{2}$:}
%___________________________________________________________________________________________
%5/53

\item \begin{eqnarray*} &&\frac{\partial}{\partial
z_1}\frac{\partial}{\partial
z_2}\left(\hat{R}_{1}\left(P_{1}\left(z_{1}\right)\tilde{P}_{2}\left(z_{2}\right)\hat{P}_{1}\left(w_{1}\right)\hat{P}_{2}\left(w_{2}\right)\right)\hat{F}_{1}\left(\hat{\theta}_{1}\left(P_{1}\left(z_{1}\right)\tilde{P}_{2}\left(z_{2}\right)
\hat{P}_{2}\left(w_{2}\right)\right),w_{2}\right)F_{1}\left(z_{1},z_{2}\right)\right)\\
&=&\hat{r}_{1}\mu_{1}\tilde{\mu}_{2}+\mu_{1}\tilde{\mu}_{2}\hat{R}_{1}^{(2)}\left(1\right)+\hat{r}_{1}\mu_{1}F_{1}^{(0,1)}+\hat{r}_{1}\tilde{\mu}_{2}F_{1}^{(1,0)}+\frac{\mu_{1}\tilde{\mu}_{2}}{1-\hat{\mu}_{1}}\hat{F}_{1}^{(1,0)}+2\hat{r}_{1}\frac{\mu_{1}\tilde{\mu}_{2}}{1-\hat{\mu}_{1}}\hat{F}_{1}^{(1,0)}\\
&+&\mu_{1}\tilde{\mu}_{2}\hat{\theta}_{1}^{(2)}\left(1\right)\hat{F}_{1}^{(1,0)}+\frac{\mu_{1}}{1-\hat{\mu}_{1}}\hat{F}_{1}^{(1,0)}F_{1}^{(0,1)}+\frac{\tilde{\mu}_{2}}{1-\hat{\mu}_{1}}\hat{F}_{1}^{(1,0)}F_{1}^{(1,0)}+F_{1}^{(1,1)}+\mu_{1}\tilde{\mu}_{2}\left(\frac{1}{1-\hat{\mu}_{1}}\right)^{2}\hat{F}_{1}^{(2,0)}.
\end{eqnarray*}

%6/54
\item \begin{eqnarray*} &&\frac{\partial}{\partial
z_2}\frac{\partial}{\partial
z_2}\left(\hat{R}_{1}\left(P_{1}\left(z_{1}\right)\tilde{P}_{2}\left(z_{2}\right)\hat{P}_{1}\left(w_{1}\right)\hat{P}_{2}\left(w_{2}\right)\right)\hat{F}_{1}\left(\hat{\theta}_{1}\left(P_{1}\left(z_{1}\right)\tilde{P}_{2}\left(z_{2}\right)
\hat{P}_{2}\left(w_{2}\right)\right),w_{2}\right)F_{1}\left(z_{1},z_{2}\right)\right)\\
&=&\hat{r}_{1}\tilde{P}_{2}^{(2)}\left(1\right)+\tilde{\mu}_{2}^{2}\hat{R}_{1}^{(2)}\left(1\right)+2\hat{r}_{1}\tilde{\mu}_{2}F_{1}^{(0,1)}+ F_{1}^{(0,2)}+2\hat{r}_{1}\frac{\tilde{\mu}_{2}^{2}}{1-\hat{\mu}_{1}}\hat{F}_{1}^{(1,0)}+\frac{1}{1-\hat{\mu}_{1}}\tilde{P}_{2}^{(2)}\left(1\right)\hat{F}_{1}^{(1,0)}\\
&+&\tilde{\mu}_{2}^{2}\hat{\theta}_{1}^{(2)}\left(1\right)\hat{F}_{1}^{(1,0)}+2\frac{\tilde{\mu}_{2}}{1-\hat{\mu}_{1}}F^{(0,1)}\hat{F}_{1}^{(1,0)}+\left(\frac{\tilde{\mu}_{2}}{1-\hat{\mu}_{1}}\right)^{2}\hat{F}_{1}^{(2,0)}.
\end{eqnarray*}
%7/55

\item \begin{eqnarray*} &&\frac{\partial}{\partial
w_1}\frac{\partial}{\partial
z_2}\left(\hat{R}_{1}\left(P_{1}\left(z_{1}\right)\tilde{P}_{2}\left(z_{2}\right)\hat{P}_{1}\left(w_{1}\right)\hat{P}_{2}\left(w_{2}\right)\right)\hat{F}_{1}\left(\hat{\theta}_{1}\left(P_{1}\left(z_{1}\right)\tilde{P}_{2}\left(z_{2}\right)
\hat{P}_{2}\left(w_{2}\right)\right),w_{2}\right)F_{1}\left(z_{1},z_{2}\right)\right)\\
&=&\hat{r}_{1}\hat{\mu}_{1}\tilde{\mu}_{2}+\hat{\mu}_{1}\tilde{\mu}_{2}\hat{R}_{1}^{(2)}\left(1\right)+
\hat{r}_{1}\hat{\mu}_{1}F_{1}^{(0,1)}+\hat{r}_{1}\frac{\hat{\mu}_{1}\tilde{\mu}_{2}}{1-\hat{\mu}_{1}}\hat{F}_{1}^{(1,0)}.
\end{eqnarray*}
%8/56

\item \begin{eqnarray*} &&\frac{\partial}{\partial
w_2}\frac{\partial}{\partial
z_2}\left(\hat{R}_{1}\left(P_{1}\left(z_{1}\right)\tilde{P}_{2}\left(z_{2}\right)\hat{P}_{1}\left(w_{1}\right)\hat{P}_{2}\left(w_{2}\right)\right)\hat{F}_{1}\left(\hat{\theta}_{1}\left(P_{1}\left(z_{1}\right)\tilde{P}_{2}\left(z_{2}\right)
\hat{P}_{2}\left(w_{2}\right)\right),w_{2}\right)F_{1}\left(z_{1},z_{2}\right)\right)\\
&=&\hat{r}_{1}\tilde{\mu}_{2}\hat{\mu}_{2}+\hat{\mu}_{2}\tilde{\mu}_{2}\hat{R}_{1}^{(2)}\left(1\right)+\hat{\mu}_{2}\hat{R}_{1}^{(2)}\left(1\right)F_{1}^{(0,1)}+\frac{\hat{\mu}_{2}\tilde{\mu}_{2}}{1-\hat{\mu}_{1}}\hat{F}_{1}^{(1,0)}+
\hat{r}_{1}\frac{\hat{\mu}_{2}\tilde{\mu}_{2}}{1-\hat{\mu}_{1}}\hat{F}_{1}^{(1,0)}\\
&+&\hat{\mu}_{2}\tilde{\mu}_{2}\hat{\theta}_{1}^{(2)}\left(1\right)\hat{F}_{1}^{(1,0)}+\hat{r}_{1}\tilde{\mu}_{2}\left(\hat{F}_{1}^{(0,1)}+\frac{\hat{\mu}_{2}}{1-\hat{\mu}_{1}}\hat{F}_{1}^{(1,0)}\right)+F_{1}^{(0,1)}\left(\hat{F}_{1}^{(0,1)}+\frac{\hat{\mu}_{2}}{1-\hat{\mu}_{1}}\hat{F}_{1}^{(1,0)}\right)\\
&+&\frac{\tilde{\mu}_{2}}{1-\hat{\mu}_{1}}\left(\hat{F}_{1}^{(1,1)}+\frac{\hat{\mu}_{2}}{1-\hat{\mu}_{1}}\hat{F}_{1}^{(2,0)}\right).
\end{eqnarray*}
%___________________________________________________________________________________________
\subsubsection{Mixtas para $w_{1}$:}
%___________________________________________________________________________________________
%9/57
\item \begin{eqnarray*} &&\frac{\partial}{\partial
z_1}\frac{\partial}{\partial
w_1}\left(\hat{R}_{1}\left(P_{1}\left(z_{1}\right)\tilde{P}_{2}\left(z_{2}\right)\hat{P}_{1}\left(w_{1}\right)\hat{P}_{2}\left(w_{2}\right)\right)\hat{F}_{1}\left(\hat{\theta}_{1}\left(P_{1}\left(z_{1}\right)\tilde{P}_{2}\left(z_{2}\right)
\hat{P}_{2}\left(w_{2}\right)\right),w_{2}\right)F_{1}\left(z_{1},z_{2}\right)\right)\\
&=&\hat{r}_{1}\mu_{1}\hat{\mu}_{1}+\mu_{1}\hat{\mu}_{1}\hat{R}_{1}^{(2)}\left(1\right)+\hat{r}_{1}\hat{\mu}_{1}F_{1}^{(1,0)}+\hat{r}_{1}\frac{\mu_{1}\hat{\mu}_{1}}{1-\hat{\mu}_{1}}\hat{F}_{1}^{(1,0)}.
\end{eqnarray*}
%10/58
\item \begin{eqnarray*} &&\frac{\partial}{\partial
z_2}\frac{\partial}{\partial
w_1}\left(\hat{R}_{1}\left(P_{1}\left(z_{1}\right)\tilde{P}_{2}\left(z_{2}\right)\hat{P}_{1}\left(w_{1}\right)\hat{P}_{2}\left(w_{2}\right)\right)\hat{F}_{1}\left(\hat{\theta}_{1}\left(P_{1}\left(z_{1}\right)\tilde{P}_{2}\left(z_{2}\right)
\hat{P}_{2}\left(w_{2}\right)\right),w_{2}\right)F_{1}\left(z_{1},z_{2}\right)\right)\\
&=&\hat{r}_{1}\tilde{\mu}_{2}\hat{\mu}_{1}+\tilde{\mu}_{2}\hat{\mu}_{1}\hat{R}_{1}^{(2)}\left(1\right)+\hat{r}_{1}\hat{\mu}_{1}F_{1}^{(0,1)}+\hat{r}_{1}\frac{\tilde{\mu}_{2}\hat{\mu}_{1}}{1-\hat{\mu}_{1}}\hat{F}_{1}^{(1,0)}.
\end{eqnarray*}
%11/59
\item \begin{eqnarray*} &&\frac{\partial}{\partial
w_1}\frac{\partial}{\partial
w_1}\left(\hat{R}_{1}\left(P_{1}\left(z_{1}\right)\tilde{P}_{2}\left(z_{2}\right)\hat{P}_{1}\left(w_{1}\right)\hat{P}_{2}\left(w_{2}\right)\right)\hat{F}_{1}\left(\hat{\theta}_{1}\left(P_{1}\left(z_{1}\right)\tilde{P}_{2}\left(z_{2}\right)
\hat{P}_{2}\left(w_{2}\right)\right),w_{2}\right)F_{1}\left(z_{1},z_{2}\right)\right)\\
&=&\hat{r}_{1}\hat{P}_{1}^{(2)}\left(1\right)+\hat{\mu}_{1}^{2}\hat{R}_{1}^{(2)}\left(1\right).
\end{eqnarray*}
%12/60
\item \begin{eqnarray*} &&\frac{\partial}{\partial
w_2}\frac{\partial}{\partial
w_1}\left(\hat{R}_{1}\left(P_{1}\left(z_{1}\right)\tilde{P}_{2}\left(z_{2}\right)\hat{P}_{1}\left(w_{1}\right)\hat{P}_{2}\left(w_{2}\right)\right)\hat{F}_{1}\left(\hat{\theta}_{1}\left(P_{1}\left(z_{1}\right)\tilde{P}_{2}\left(z_{2}\right)
\hat{P}_{2}\left(w_{2}\right)\right),w_{2}\right)F_{1}\left(z_{1},z_{2}\right)\right)\\
&=&\hat{r}_{1}\hat{\mu}_{2}\hat{\mu}_{1}+\hat{\mu}_{2}\hat{\mu}_{1}\hat{R}_{1}^{(2)}\left(1\right)+\hat{r}_{1}\hat{\mu}_{1}\left(\hat{F}_{1}^{(0,1)}+\frac{\hat{\mu}_{2}}{1-\hat{\mu}_{1}}\hat{F}_{1}^{(1,0)}\right).
\end{eqnarray*}
%___________________________________________________________________________________________
\subsubsection{Mixtas para $w_{1}$:}
%___________________________________________________________________________________________
%13/61



\item \begin{eqnarray*} &&\frac{\partial}{\partial
z_1}\frac{\partial}{\partial
w_2}\left(\hat{R}_{1}\left(P_{1}\left(z_{1}\right)\tilde{P}_{2}\left(z_{2}\right)\hat{P}_{1}\left(w_{1}\right)\hat{P}_{2}\left(w_{2}\right)\right)\hat{F}_{1}\left(\hat{\theta}_{1}\left(P_{1}\left(z_{1}\right)\tilde{P}_{2}\left(z_{2}\right)
\hat{P}_{2}\left(w_{2}\right)\right),w_{2}\right)F_{1}\left(z_{1},z_{2}\right)\right)\\
&=&\hat{r}_{1}\mu_{1}\hat{\mu}_{2}+\mu_{1}\hat{\mu}_{2}\hat{R}_{1}^{(2)}\left(1\right)+\hat{r}_{1}\hat{\mu}_{2}F_{1}^{(1,0)}+
\hat{r}_{1}\frac{\mu_{1}\hat{\mu}_{2}}{1-\hat{\mu}_{1}}\hat{F}_{1}^{(1,0)}+\hat{r}_{1}\mu_{1}\left(\hat{F}_{1}^{(0,1)}+\frac{\hat{\mu}_{2}}{1-\hat{\mu}_{1}}\hat{F}_{1}^{(1,0)}\right)\\
&+&F_{1}^{(1,0)}\left(\hat{F}_{1}^{(0,1)}+\frac{\hat{\mu}_{2}}{1-\hat{\mu}_{1}}\hat{F}_{1}^{(1,0)}\right)+\frac{\mu_{1}\hat{\mu}_{2}}{1-\hat{\mu}_{1}}\hat{F}_{1}^{(1,0)}+\mu_{1}\hat{\mu}_{2}\hat{\theta}_{1}^{(2)}\left(1\right)\hat{F}_{1}^{(1,0)}+\frac{\mu_{1}}{1-\hat{\mu}_{1}}\hat{F}_{1}^{(1,1)}\\
&+&\mu_{1}\hat{\mu}_{2}\left(\frac{1}{1-\hat{\mu}_{1}}\right)^{2}\hat{F}_{1}^{(2,0)}.
\end{eqnarray*}

%14/62
\item \begin{eqnarray*} &&\frac{\partial}{\partial
z_2}\frac{\partial}{\partial
w_2}\left(\hat{R}_{1}\left(P_{1}\left(z_{1}\right)\tilde{P}_{2}\left(z_{2}\right)\hat{P}_{1}\left(w_{1}\right)\hat{P}_{2}\left(w_{2}\right)\right)\hat{F}_{1}\left(\hat{\theta}_{1}\left(P_{1}\left(z_{1}\right)\tilde{P}_{2}\left(z_{2}\right)
\hat{P}_{2}\left(w_{2}\right)\right),w_{2}\right)F_{1}\left(z_{1},z_{2}\right)\right)\\
&=&\hat{r}_{1}\tilde{\mu}_{2}\hat{\mu}_{2}+\tilde{\mu}_{2}\hat{\mu}_{2}\hat{R}_{1}^{(2)}\left(1\right)+\hat{r}_{1}\hat{\mu}_{2}F_{1}^{(0,1)}+\hat{r}_{1}\frac{\tilde{\mu}_{2}\hat{\mu}_{2}}{1-\hat{\mu}_{1}}\hat{F}_{1}^{(1,0)}+\hat{r}_{1}\tilde{\mu}_{2}\left(\hat{F}_{1}^{(0,1)}+\frac{\hat{\mu}_{2}}{1-\hat{\mu}_{1}}\hat{F}_{1}^{(1,0)}\right)\\
&+&F_{1}^{(0,1)}\left(\hat{F}_{1}^{(0,1)}+\frac{\hat{\mu}_{2}}{1-\hat{\mu}_{1}}\hat{F}_{1}^{(1,0)}\right)+\frac{\tilde{\mu}_{2}\hat{\mu}_{2}}{1-\hat{\mu}_{1}}\hat{F}_{1}^{(1,0)}+\tilde{\mu}_{2}\hat{\mu}_{2}\hat{\theta}_{1}^{(2)}\left(1\right)\hat{F}_{1}^{(1,0)}+\frac{\tilde{\mu}_{2}}{1-\hat{\mu}_{1}}\hat{F}_{1}^{(1,1)}\\
&+&\tilde{\mu}_{2}\hat{\mu}_{2}\left(\frac{1}{1-\hat{\mu}_{1}}\right)^{2}\hat{F}_{1}^{(2,0)}.
\end{eqnarray*}

%15/63

\item \begin{eqnarray*} &&\frac{\partial}{\partial
w_1}\frac{\partial}{\partial
w_2}\left(\hat{R}_{1}\left(P_{1}\left(z_{1}\right)\tilde{P}_{2}\left(z_{2}\right)\hat{P}_{1}\left(w_{1}\right)\hat{P}_{2}\left(w_{2}\right)\right)\hat{F}_{1}\left(\hat{\theta}_{1}\left(P_{1}\left(z_{1}\right)\tilde{P}_{2}\left(z_{2}\right)
\hat{P}_{2}\left(w_{2}\right)\right),w_{2}\right)F_{1}\left(z_{1},z_{2}\right)\right)\\
&=&\hat{r}_{1}\hat{\mu}_{2}\hat{\mu}_{1}+\hat{\mu}_{2}\hat{\mu}_{1}\hat{R}_{1}^{(2)}\left(1\right)+\hat{r}_{1}\hat{\mu}_{1}\left(\hat{F}_{1}^{(0,1)}+\frac{\hat{\mu}_{2}}{1-\hat{\mu}_{1}}\hat{F}_{1}^{(1,0)}\right).
\end{eqnarray*}

%16/64

\item \begin{eqnarray*} &&\frac{\partial}{\partial
w_2}\frac{\partial}{\partial
w_2}\left(\hat{R}_{1}\left(P_{1}\left(z_{1}\right)\tilde{P}_{2}\left(z_{2}\right)\hat{P}_{1}\left(w_{1}\right)\hat{P}_{2}\left(w_{2}\right)\right)\hat{F}_{1}\left(\hat{\theta}_{1}\left(P_{1}\left(z_{1}\right)\tilde{P}_{2}\left(z_{2}\right)
\hat{P}_{2}\left(w_{2}\right)\right),w_{2}\right)F_{1}\left(z_{1},z_{2}\right)\right)\\
&=&\hat{r}_{1}\hat{P}_{2}^{(2)}\left(1\right)+\hat{\mu}_{2}^{2}\hat{R}_{1}^{(2)}\left(1\right)+
2\hat{r}_{1}\hat{\mu}_{2}\left(\hat{F}_{1}^{(0,1)}+\frac{\hat{\mu}_{2}}{1-\hat{\mu}_{1}}\hat{F}_{1}^{(1,0)}\right)+
\hat{F}_{1}^{(0,2)}+\frac{1}{1-\hat{\mu}_{1}}\hat{P}_{2}^{(2)}\left(1\right)\hat{F}_{1}^{(1,0)}\\
&+&\hat{\mu}_{2}^{2}\hat{\theta}_{1}^{(2)}\left(1\right)\hat{F}_{1}^{(1,0)}+\frac{\hat{\mu}_{2}}{1-\hat{\mu}_{1}}\hat{F}_{1}^{(1,1)}+\frac{\hat{\mu}_{2}}{1-\hat{\mu}_{1}}\left(\hat{F}_{1}^{(1,1)}+\frac{\hat{\mu}_{2}}{1-\hat{\mu}_{1}}\hat{F}_{1}^{(2,0)}\right).
\end{eqnarray*}
%_________________________________________________________________________________________________________
%
%_________________________________________________________________________________________________________

\end{enumerate}


%----------------------------------------------------------------------------------------
%   INTRODUCTION
%----------------------------------------------------------------------------------------
\section*{Introducci\'on}
Un sistema de visitas (Polling System) consiste en una cola a la cu\'al llegan los usuarios para ser atendidos por uno o varios servidores de acuerdo a una pol\'itica determinada, en la cual se puede asumir que la manera en que los usuarios llegan a la misma es conforme a un proceso Poisson con tasa de llegada $\mu$. De igual manera se puede asumir que la distribuci\'on de los servicios a cada uno de los usuarios presentes en la cola es conforme a una variable aleatoria exponencial. Esto es la base para la conformaci\'on de los Sistemas de Visitas C\'iclicas, de los cuales es posible obtener sus Funciones Generadoras de Probabilidades, primeros y segundos momentos as\'i como medidas de desempe\~no que permiten tener una mejor descripci\'on del funcionamiento del sistema en cualquier momento $t$ asumiendo estabilidad.



%----------------------------------------------------------------------------------------
%   OBJECTIVES
%----------------------------------------------------------------------------------------



\section*{Objetivos Principales}

\begin{itemize}
%\item Generalizar los principales resultados existentes para Sistemas de Visitas C\'iclicas para el caso en el que se tienen dos Sistemas de Visitas C\'iclicas con propiedades similares.

\item Encontrar las ecuaciones que modelan el comportamiento de una Red de Sistemas de Visitas C\'iclicas (RSVC) con propiedades similares.

\item Encontrar expresiones anal\'iticas para las longitudes de las colas al momento en que el servidor llega a una de ellas para comenzar a dar servicio, as\'i como de sus segundos momentos.

\item Determinar las principales medidas de Desempe\~no para la RSVC tales como: N\'umero de usuarios presentes en cada una de las colas del sistema cuando uno de los servidores est\'a presente atendiendo, Tiempos que transcurre entre las visitas del servidor a la misma cola.


\end{itemize}

%----------------------------------------------------------------------------------------
%   MATERIALS AND METHODS
%----------------------------------------------------------------------------------------

\section*{Descripci\'on de la Red de Sistemas de Visitas C\'iclicas}

El uso de la Funci\'on Generadora de Probabilidades (FGP's) nos permite determinar las Funciones de Distribuci\'on de Probabilidades Conjunta de manera indirecta sin necesidad de hacer uso de las propiedades de las distribuciones de probabilidad de cada uno de los procesos que intervienen en la Red de Sistemas de Visitas C\'iclicas.\\
\begin{itemize}
\item Se definen los procesos para los arribos para cada una de las colas:$X_{i}\left(t\right)$ y $\hat{X}_{i}\left(t\right)$.  Y para los usuarios que se trasladan de un sistema a otro se tiene el proceso $Y\left(t\right)$,% entonces $P_{i}\left(z_{i}\right)&=&\esp\left[z_{i}^{X_{i}\left(t\right)}\right],\check{P}_{2}\left(z_{2}\right)&=&\esp\left[z_{2}^{Y_{2}\left(t\right)}\right]$, y $\hat{P}_{i}\left(w_{i}\right)&=&\esp\left[w_{i}^{\hat{X}_{i}\left(t\right)}\right]$.
\item En lo que respecta al servidor, en t\'erminos de los tiempos de
visita a cada una de las colas, se definen las variables
aleatorias $\tau_{1},\tau_{2}$ para $Q_{1},Q_{2}$ respectivamente;
y $\zeta_{1},\zeta_{2}$ para $\hat{Q}_{1},\hat{Q}_{2}$ del sistema
2. \item A los tiempos en que el servidor termina de atender en las
colas $Q_{1},Q_{2},\hat{Q}_{1},\hat{Q}_{2}$, se les denotar\'a por
$\overline{\tau}_{1},\overline{\tau}_{2},\overline{\zeta}_{1},\overline{\zeta}_{2}$
respectivamente.
\item Los tiempos de traslado del servidor desde el
momento en que termina de atender a una cola y llega a la
siguiente para comenzar a dar servicio est\'an dados por
$\tau_{2}-\overline{\tau}_{1},\tau_{1}-\overline{\tau}_{2}$ y
$\zeta_{2}-\overline{\zeta}_{1},\zeta_{1}-\overline{\zeta}_{2}$
para el sistema 1 y el sistema 2, respectivamente.
\end{itemize}
Cada uno de estos procesos con su respectiva FGP. Adem\'as, para cada una de las colas en cada sistema, el n\'umero de usuarios al tiempo en que llega el servidor a dar servicio est\'a
dado por el n\'umero de usuarios presentes en la cola al tiempo
$t$, m\'as el n\'umero de usuarios que llegan a la cola en el intervalo de tiempo
$\left[\tau_{i},\overline{\tau}_{i}\right]$, es decir
{\small{
\begin{eqnarray*}
L_{1}\left(\overline{\tau}_{1}\right)=L_{1}\left(\tau_{1}\right)+X_{1}\left(\overline{\tau}_{1}-\tau_{1}\right),\hat{L}_{i}\left(\overline{\tau}_{i}\right)=\hat{L}_{i}\left(\tau_{i}\right)+\hat{X}_{i}\left(\overline{\tau}_{i}-\tau_{i}\right),L_{2}\left(\overline{\tau}_{1}\right)=L_{2}\left(\tau_{1}\right)+X_{2}\left(\overline{\tau}_{1}-\tau_{1}\right)+Y_{2}\left(\overline{\tau}_{1}-\tau_{1}\right),
\end{eqnarray*}}}




%\begin{center}\vspace{1cm}
%%%%\includegraphics[width=0.6\linewidth]{RedSVC2}
%\captionof{figure}{\color{Green} Red de Sistema de Visitas C\'iclicas}
%\end{center}\vspace{1cm}




Una vez definidas las Funciones Generadoras de Probabilidades Conjuntas se construyen las ecuaciones recursivas que permiten obtener la informaci\'on sobre la longitud de cada una de las colas, al momento en que uno de los servidores llega a una de las colas para dar servicio, bas\'andose en la informaci\'on que se tiene sobre su llegada a la cola inmediata anterior.\\
{\footnotesize{
\begin{eqnarray*}
F_{2}\left(z_{1},z_{2},w_{1},w_{2}\right)&=&R_{1}\left(P_{1}\left(z_{1}\right)\tilde{P}_{2}\left(z_{2}\right)\prod_{i=1}^{2}
\hat{P}_{i}\left(w_{i}\right)\right)F_{1}\left(\theta_{1}\left(\tilde{P}_{2}\left(z_{2}\right)\hat{P}_{1}\left(w_{1}\right)\hat{P}_{2}\left(w_{2}\right)\right),z_{2},w_{1},w_{2}\right),\\
F_{1}\left(z_{1},z_{2},w_{1},w_{2}\right)&=&R_{2}\left(P_{1}\left(z_{1}\right)\tilde{P}_{2}\left(z_{2}\right)\prod_{i=1}^{2}
\hat{P}_{i}\left(w_{i}\right)\right)F_{2}\left(z_{1},\tilde{\theta}_{2}\left(P_{1}\left(z_{1}\right)\hat{P}_{1}\left(w_{1}\right)\hat{P}_{2}\left(w_{2}\right)\right),w_{1},w_{2}\right),\\
\hat{F}_{2}\left(z_{1},z_{2},w_{1},w_{2}\right)&=&\hat{R}_{1}\left(P_{1}\left(z_{1}\right)\tilde{P}_{2}\left(z_{2}\right)\prod_{i=1}^{2}
\hat{P}_{i}\left(w_{i}\right)\right)\hat{F}_{1}\left(z_{1},z_{2},\hat{\theta}_{1}\left(P_{1}\left(z_{1}\right)\tilde{P}_{2}\left(z_{2}\right)\hat{P}_{2}\left(w_{2}\right)\right),w_{2}\right),\\
%\end{eqnarray*}}}
%{\small{
%\begin{eqnarray*}
\hat{F}_{1}\left(z_{1},z_{2},w_{1},w_{2}\right)&=&\hat{R}_{2}\left(P_{1}\left(z_{1}\right)\tilde{P}_{2}\left(z_{2}\right)\prod_{i=1}^{2}
\hat{P}_{i}\left(w_{i}\right)\right)\hat{F}_{2}\left(z_{1},z_{2},w_{1},\hat{\theta}_{2}\left(P_{1}\left(z_{1}\right)\tilde{P}_{2}\left(z_{2}\right)\hat{P}_{1}\left(w_{1}\right)\right)\right).
\end{eqnarray*}}}


%------------------------------------------------
%\subsection*{Descripci\'on de la Red de Sistemas de Visitas C\'iclicas}
%------------------------------------------------

%----------------------------------------------------------------------------------------
%   RESULTS
%----------------------------------------------------------------------------------------
\section*{Resultado Principal}
%----------------------------------------------------------------------------------------
Sean $\mu_{1},\mu_{2},\check{\mu}_{2},\hat{\mu}_{1},\hat{\mu}_{2}$ y $\tilde{\mu}_{2}=\mu_{2}+\check{\mu}_{2}$ los valores esperados de los proceso definidos anteriormente, y sean $r_{1},r_{2}, \hat{r}_{1}$ y $\hat{r}_{2}$ los valores esperado s de los tiempos de traslado del servidor entre las colas para cada uno de los sistemas de visitas c\'iclicas. Si se definen $\mu=\mu_{1}+\tilde{\mu}_{2}$, $\hat{\mu}=\hat{\mu}_{1}+\hat{\mu}_{2}$, y $r=r_{1}+r_{2}$ y  $\hat{r}=\hat{r}_{1}+\hat{r}_{2}$, entonces se tiene el siguiente resultado.

\begin{Teo}
Supongamos que $\mu<1$, $\hat{\mu}<1$, entonces, el n\'umero de usuarios presentes en cada una de las colas que conforman la Red de Sistemas de Visitas C\'iclicas cuando uno de los servidores visita a alguna de ellas est\'a dada por la soluci\'on del Sistema de Ecuaciones Lineales presentados arriba cuyas expresiones damos a continuaci\'on:
%{\footnotesize{
\[ \begin{array}{lll}
f_{1}\left(1\right)=r\frac{\mu_{1}\left(1-\mu_{1}\right)}{1-\mu},&f_{1}\left(2\right)=r_{2}\tilde{\mu}_{2},&f_{1}\left(3\right)=\hat{\mu}_{1}\left(\frac{r_{2}\mu_{2}+1}{\mu_{2}}+r\frac{\tilde{\mu}_{2}}{1-\mu}\right),\\
f_{1}\left(4\right)=\hat{\mu}_{2}\left(\frac{r_{2}\mu_{2}+1}{\mu_{2}}+r\frac{\tilde{\mu}_{2}}{1-\mu}\right),&f_{2}\left(1\right)=r_{1}\mu_{1},&f_{2}\left(2\right)=r\frac{\tilde{\mu}_{2}\left(1-\tilde{\mu}_{2}\right)}{1-\mu},\\
f_{2}\left(3\right)=\hat{\mu}_{1}\left(\frac{r_{1}\mu_{1}+1}{\mu_{1}}+r\frac{\mu_{1}}{1-\mu}\right),&f_{2}\left(4\right)=\hat{\mu}_{2}\left(\frac{r_{1}\mu_{1}+1}{\mu_{1}}+r\frac{\mu_{1}}{1-\mu}\right),&\hat{f}_{1}\left(1\right)=\mu_{1}\left(\frac{\hat{r}_{2}\hat{\mu}_{2}+1}{\hat{\mu}_{2}}+\hat{r}\frac{\hat{\mu}_{2}}{1-\hat{\mu}}\right),\\
\hat{f}_{1}\left(2\right)=\tilde{\mu}_{2}\left(\hat{r}_{2}+\hat{r}\frac{\hat{\mu}_{2}}{1-\hat{\mu}}\right)+\frac{\mu_{2}}{\hat{\mu}_{2}},&\hat{f}_{1}\left(3\right)=\hat{r}\frac{\hat{\mu}_{1}\left(1-\hat{\mu}_{1}\right)}{1-\hat{\mu}},&\hat{f}_{1}\left(4\right)=\hat{r}_{2}\hat{\mu}_{2},\\
\hat{f}_{2}\left(1\right)=\mu_{1}\left(\frac{\hat{r}_{1}\hat{\mu}_{1}+1}{\hat{\mu}_{1}}+\hat{r}\frac{\hat{\mu}_{1}}{1-\hat{\mu}}\right),&\hat{f}_{2}\left(2\right)=\tilde{\mu}_{2}\left(\hat{r}_{1}+\hat{r}\frac{\hat{\mu}_{1}}{1-\hat{\mu}}\right)+\frac{\hat{\mu_{2}}}{\hat{\mu}_{1}},&\hat{f}_{2}\left(3\right)=\hat{r}_{1}\hat{\mu}_{1},\\
&\hat{f}_{2}\left(4\right)=\hat{r}\frac{\hat{\mu}_{2}\left(1-\hat{\mu}_{2}\right)}{1-\hat{\mu}}.&\\
\end{array}\] %}}
\end{Teo}


Las ecuaciones que determinan los segundos momentos de las longitudes de las colas de los dos sistemas se pueden ver en \href{http://sitio.expresauacm.org/s/carlosmartinez/wp-content/uploads/sites/13/2014/01/SegundosMomentos.pdf}{este sitio}

%\url{http://ubuntu_es_el_diablo.org},\href{http://www.latex-project.org/}{latex project}

%http://sitio.expresauacm.org/s/carlosmartinez/wp-content/uploads/sites/13/2014/01/SegundosMomentos.jpg
%http://sitio.expresauacm.org/s/carlosmartinez/wp-content/uploads/sites/13/2014/01/SegundosMomentos.pdf




%___________________________________________________________________________________________
%\section*{Tiempos de Ciclo e Intervisita}
%___________________________________________________________________________________________



%----------------------------------------------------------------------------------------
%\section*{Medidas de Desempe\~no de la Red de Sistemas de Visita C\'iclicas}
%----------------------------------------------------------------------------------------
%Se puede demostrar que las expresiones para los tiempos entre visitas de los servidores a las colas

%----------------------------------------------------------------------------------------
%   CONCLUSIONS
%----------------------------------------------------------------------------------------

%\color{SaddleBrown} % SaddleBrown color for the conclusions to make them stand out

\section*{Medidas de Desempe\~no}


\begin{Def}
Sea $L_{i}^{*}$el n\'umero de usuarios cuando el servidor visita la cola $Q_{i}$ para dar servicio, para $i=1,2$.
\end{Def}

Entonces
\begin{Prop} Para la cola $Q_{i}$, $i=1,2$, se tiene que el n\'umero de usuarios presentes al momento de ser visitada por el servidor est\'a dado por
\begin{eqnarray}
\esp\left[L_{i}^{*}\right]&=&f_{i}\left(i\right)\\
Var\left[L_{i}^{*}\right]&=&f_{i}\left(i,i\right)+\esp\left[L_{i}^{*}\right]-\esp\left[L_{i}^{*}\right]^{2}.
\end{eqnarray}
\end{Prop}


\begin{Def}
El tiempo de Ciclo $C_{i}$ es el periodo de tiempo que comienza
cuando la cola $i$ es visitada por primera vez en un ciclo, y
termina cuando es visitado nuevamente en el pr\'oximo ciclo, bajo condiciones de estabilidad.

\begin{eqnarray*}
C_{i}\left(z\right)=\esp\left[z^{\overline{\tau}_{i}\left(m+1\right)-\overline{\tau}_{i}\left(m\right)}\right]
\end{eqnarray*}
\end{Def}

\begin{Def}
El tiempo de intervisita $I_{i}$ es el periodo de tiempo que
comienza cuando se ha completado el servicio en un ciclo y termina
cuando es visitada nuevamente en el pr\'oximo ciclo.
\begin{eqnarray*}I_{i}\left(z\right)&=&\esp\left[z^{\tau_{i}\left(m+1\right)-\overline{\tau}_{i}\left(m\right)}\right]\end{eqnarray*}
\end{Def}

\begin{Prop}
Para los tiempos de intervisita del servidor $I_{i}$, se tiene que

\begin{eqnarray*}
\esp\left[I_{i}\right]&=&\frac{f_{i}\left(i\right)}{\mu_{i}},\\
Var\left[I_{i}\right]&=&\frac{Var\left[L_{i}^{*}\right]}{\mu_{i}^{2}}-\frac{\sigma_{i}^{2}}{\mu_{i}^{2}}f_{i}\left(i\right).
\end{eqnarray*}
\end{Prop}


\begin{Prop}
Para los tiempos que ocupa el servidor para atender a los usuarios presentes en la cola $Q_{i}$, con FGP denotada por $S_{i}$, se tiene que
\begin{eqnarray*}
\esp\left[S_{i}\right]&=&\frac{\esp\left[L_{i}^{*}\right]}{1-\mu_{i}}=\frac{f_{i}\left(i\right)}{1-\mu_{i}},\\
Var\left[S_{i}\right]&=&\frac{Var\left[L_{i}^{*}\right]}{\left(1-\mu_{i}\right)^{2}}+\frac{\sigma^{2}\esp\left[L_{i}^{*}\right]}{\left(1-\mu_{i}\right)^{3}}
\end{eqnarray*}
\end{Prop}


\begin{Prop}
Para la duraci\'on de los ciclos $C_{i}$ se tiene que
\begin{eqnarray*}
\esp\left[C_{i}\right]&=&\esp\left[I_{i}\right]\esp\left[\theta_{i}\left(z\right)\right]=\frac{\esp\left[L_{i}^{*}\right]}{\mu_{i}}\frac{1}{1-\mu_{i}}=\frac{f_{i}\left(i\right)}{\mu_{i}\left(1-\mu_{i}\right)}\\
Var\left[C_{i}\right]&=&\frac{Var\left[L_{i}^{*}\right]}{\mu_{i}^{2}\left(1-\mu_{i}\right)^{2}}.
\end{eqnarray*}

\end{Prop}


%----------------------------------------------------------------------------------------
%   REFERENCES
%----------------------------------------------------------------------------------------
%_________________________________________________________________________
%\section*{REFERENCIAS}
%_________________________________________________________________________

\section*{Conjeturas}
%----------------------------------------------------------------------------------------

\begin{Def}
Dada una cola $Q_{i}$, sea $\mathcal{L}=\left\{L_{1}\left(t\right),L_{2}\left(t\right),\hat{L}_{1}\left(t\right),\hat{L}_{2}\left(t\right)\right\}$ las longitudes de todas las colas de la Red de Sistemas de Visitas C\'iclicas. Sup\'ongase que el servidor visita $Q_{i}$, si $L_{i}\left(t\right)=0$ y $\hat{L}_{i}\left(t\right)=0$ para $i=1,2$, entonces los elementos de $\mathcal{L}$ son puntos regenerativos.
\end{Def}


\begin{Def}
Un ciclo regenerativo es el intervalo de tiempo que ocurre entre dos puntos regenerativos sucesivos, $\mathcal{L}_{1},\mathcal{L}_{2}$.
\end{Def}


Def\'inanse los puntos de regenaraci\'on  en el proceso
$\left[L_{1}\left(t\right),L_{2}\left(t\right),\ldots,L_{N}\left(t\right)\right]$.
Los puntos cuando la cola $i$ es visitada y todos los
$L_{j}\left(\tau_{i}\left(m\right)\right)=0$ para $i=1,2$  son
puntos de regeneraci\'on. Se llama ciclo regenerativo al intervalo
entre dos puntos regenerativos sucesivos.

Sea $M_{i}$  el n\'umero de ciclos de visita en un ciclo
regenerativo, y sea $C_{i}^{(m)}$, para $m=1,2,\ldots,M_{i}$ la
duraci\'on del $m$-\'esimo ciclo de visita en un ciclo
regenerativo. Se define el ciclo del tiempo de visita promedio
$\esp\left[C_{i}\right]$ como
\begin{eqnarray*}
\esp\left[C_{i}\right]&=&\frac{\esp\left[\sum_{m=1}^{M_{i}}C_{i}^{(m)}\right]}{\esp\left[M_{i}\right]}
\end{eqnarray*}


En Stid72 y Heym82 se muestra que una condici\'on suficiente para
que el proceso regenerativo estacionario sea un procesoo
estacionario es que el valor esperado del tiempo del ciclo
regenerativo sea finito:

\begin{eqnarray*}
\esp\left[\sum_{m=1}^{M_{i}}C_{i}^{(m)}\right]<\infty.
\end{eqnarray*}



como cada $C_{i}^{(m)}$ contiene intervalos de r\'eplica
positivos, se tiene que $\esp\left[M_{i}\right]<\infty$, adem\'as,
como $M_{i}>0$, se tiene que la condici\'on anterior es
equivalente a tener que

\begin{eqnarray*}
\esp\left[C_{i}\right]<\infty,
\end{eqnarray*}
por lo tanto una condici\'on suficiente para la existencia del
proceso regenerativo est\'a dada por
\begin{eqnarray*}
\sum_{k=1}^{N}\mu_{k}<1.
\end{eqnarray*}



Sea la funci\'on generadora de momentos para $L_{i}$, el n\'umero
de usuarios en la cola $Q_{i}\left(z\right)$ en cualquier momento,
est\'a dada por el tiempo promedio de $z^{L_{i}\left(t\right)}$
sobre el ciclo regenerativo definido anteriormente:

\begin{eqnarray*}
Q_{i}\left(z\right)&=&\esp\left[z^{L_{i}\left(t\right)}\right]=\frac{\esp\left[\sum_{m=1}^{M_{i}}\sum_{t=\tau_{i}\left(m\right)}^{\tau_{i}\left(m+1\right)-1}z^{L_{i}\left(t\right)}\right]}{\esp\left[\sum_{m=1}^{M_{i}}\tau_{i}\left(m+1\right)-\tau_{i}\left(m\right)\right]}
\end{eqnarray*}


$M_{i}$ es un tiempo de paro en el proceso regenerativo con
$\esp\left[M_{i}\right]<\infty$, se sigue del lema de Wald que:


\begin{eqnarray*}
\esp\left[\sum_{m=1}^{M_{i}}\sum_{t=\tau_{i}\left(m\right)}^{\tau_{i}\left(m+1\right)-1}z^{L_{i}\left(t\right)}\right]&=&\esp\left[M_{i}\right]\esp\left[\sum_{t=\tau_{i}\left(m\right)}^{\tau_{i}\left(m+1\right)-1}z^{L_{i}\left(t\right)}\right]\\
\esp\left[\sum_{m=1}^{M_{i}}\tau_{i}\left(m+1\right)-\tau_{i}\left(m\right)\right]&=&\esp\left[M_{i}\right]\esp\left[\tau_{i}\left(m+1\right)-\tau_{i}\left(m\right)\right]
\end{eqnarray*}

por tanto se tiene que


\begin{eqnarray*}
Q_{i}\left(z\right)&=&\frac{\esp\left[\sum_{t=\tau_{i}\left(m\right)}^{\tau_{i}\left(m+1\right)-1}z^{L_{i}\left(t\right)}\right]}{\esp\left[\tau_{i}\left(m+1\right)-\tau_{i}\left(m\right)\right]}
\end{eqnarray*}

observar que el denominador es simplemente la duraci\'on promedio
del tiempo del ciclo.




Se puede demostrar (ver Hideaki Takagi 1986) que

\begin{eqnarray*}
\esp\left[\sum_{t=\tau_{i}\left(m\right)}^{\tau_{i}\left(m+1\right)-1}z^{L_{i}\left(t\right)}\right]=z\frac{F_{i}\left(z\right)-1}{z-P_{i}\left(z\right)}
\end{eqnarray*}

Durante el tiempo de intervisita para la cola $i$,
$L_{i}\left(t\right)$ solamente se incrementa de manera que el
incremento por intervalo de tiempo est\'a dado por la funci\'on
generadora de probabilidades de $P_{i}\left(z\right)$, por tanto
la suma sobre el tiempo de intervisita puede evaluarse como:

\begin{eqnarray*}
\esp\left[\sum_{t=\tau_{i}\left(m\right)}^{\tau_{i}\left(m+1\right)-1}z^{L_{i}\left(t\right)}\right]&=&\esp\left[\sum_{t=\tau_{i}\left(m\right)}^{\tau_{i}\left(m+1\right)-1}\left\{P_{i}\left(z\right)\right\}^{t-\overline{\tau}_{i}\left(m\right)}\right]\\
&=&\frac{1-\esp\left[\left\{P_{i}\left(z\right)\right\}^{\tau_{i}\left(m+1\right)-\overline{\tau}_{i}\left(m\right)}\right]}{1-P_{i}\left(z\right)}=\frac{1-I_{i}\left[P_{i}\left(z\right)\right]}{1-P_{i}\left(z\right)}
\end{eqnarray*}
por tanto



\begin{eqnarray*}
\esp\left[\sum_{t=\tau_{i}\left(m\right)}^{\tau_{i}\left(m+1\right)-1}z^{L_{i}\left(t\right)}\right]&=&\frac{1-F_{i}\left(z\right)}{1-P_{i}\left(z\right)}
\end{eqnarray*}


Haciendo uso de lo hasta ahora desarrollado se tiene que

\begin{eqnarray*}
Q_{i}\left(z\right)&=&\frac{1}{\esp\left[C_{i}\right]}\cdot\frac{1-F_{i}\left(z\right)}{P_{i}\left(z\right)-z}\cdot\frac{\left(1-z\right)P_{i}\left(z\right)}{1-P_{i}\left(z\right)}\\
&=&\frac{\mu_{i}\left(1-\mu_{i}\right)}{f_{i}\left(i\right)}\cdot\frac{1-F_{i}\left(z\right)}{P_{i}\left(z\right)-z}\cdot\frac{\left(1-z\right)P_{i}\left(z\right)}{1-P_{i}\left(z\right)}
\end{eqnarray*}

derivando con respecto a $z$




\begin{eqnarray*}
\frac{d Q_{i}\left(z\right)}{d z}&=&\frac{\left(1-F_{i}\left(z\right)\right)P_{i}\left(z\right)}{\esp\left[C_{i}\right]\left(1-P_{i}\left(z\right)\right)\left(P_{i}\left(z\right)-z\right)}\\
&-&\frac{\left(1-z\right)P_{i}\left(z\right)F_{i}^{'}\left(z\right)}{\esp\left[C_{i}\right]\left(1-P_{i}\left(z\right)\right)\left(P_{i}\left(z\right)-z\right)}\\
&-&\frac{\left(1-z\right)\left(1-F_{i}\left(z\right)\right)P_{i}\left(z\right)\left(P_{i}^{'}\left(z\right)-1\right)}{\esp\left[C_{i}\right]\left(1-P_{i}\left(z\right)\right)\left(P_{i}\left(z\right)-z\right)^{2}}\\
&+&\frac{\left(1-z\right)\left(1-F_{i}\left(z\right)\right)P_{i}^{'}\left(z\right)}{\esp\left[C_{i}\right]\left(1-P_{i}\left(z\right)\right)\left(P_{i}\left(z\right)-z\right)}\\
&+&\frac{\left(1-z\right)\left(1-F_{i}\left(z\right)\right)P_{i}\left(z\right)P_{i}^{'}\left(z\right)}{\esp\left[C_{i}\right]\left(1-P_{i}\left(z\right)\right)^{2}\left(P_{i}\left(z\right)-z\right)}
\end{eqnarray*}

%______________________________________________________



Calculando el l\'imite cuando $z\rightarrow1^{+}$:
\begin{eqnarray}
Q_{i}^{(1)}\left(z\right)&=&lim_{z\rightarrow1^{+}}\frac{d Q_{i}\left(z\right)}{dz}\\
&=&lim_{z\rightarrow1}\frac{\left(1-F_{i}\left(z\right)\right)P_{i}\left(z\right)}{\esp\left[C_{i}\right]\left(1-P_{i}\left(z\right)\right)\left(P_{i}\left(z\right)-z\right)}\\
&-&lim_{z\rightarrow1^{+}}\frac{\left(1-z\right)P_{i}\left(z\right)F_{i}^{'}\left(z\right)}{\esp\left[C_{i}\right]\left(1-P_{i}\left(z\right)\right)\left(P_{i}\left(z\right)-z\right)}\\
&-&lim_{z\rightarrow1^{+}}\frac{\left(1-z\right)\left(1-F_{i}\left(z\right)\right)P_{i}\left(z\right)\left(P_{i}^{'}\left(z\right)-1\right)}{\esp\left[C_{i}\right]\left(1-P_{i}\left(z\right)\right)\left(P_{i}\left(z\right)-z\right)^{2}}\\
&+&lim_{z\rightarrow1^{+}}\frac{\left(1-z\right)\left(1-F_{i}\left(z\right)\right)P_{i}^{'}\left(z\right)}{\esp\left[C_{i}\right]\left(1-P_{i}\left(z\right)\right)\left(P_{i}\left(z\right)-z\right)}\\
&+&lim_{z\rightarrow1^{+}}\frac{\left(1-z\right)\left(1-F_{i}\left(nz\right)\right)P_{i}\left(z\right)P_{i}^{'}\left(z\right)}{\esp\left[C_{i}\right]\left(1-P_{i}\left(z\right)\right)^{2}\left(P_{i}\left(z\right)-z\right)}
\end{eqnarray}

Entonces:



\begin{eqnarray*}
&&lim_{z\rightarrow1^{+}}\frac{\left(1-F_{i}\left(z\right)\right)P_{i}\left(z\right)}{\left(1-P_{i}\left(z\right)\right)\left(P_{i}\left(z\right)-z\right)}=lim_{z\rightarrow1^{+}}\frac{\frac{d}{dz}\left[\left(1-F_{i}\left(z\right)\right)P_{i}\left(z\right)\right]}{\frac{d}{dz}\left[\left(1-P_{i}\left(z\right)\right)\left(-z+P_{i}\left(z\right)\right)\right]}\\
&=&lim_{z\rightarrow1^{+}}\frac{-P_{i}\left(z\right)F_{i}^{'}\left(z\right)+\left(1-F_{i}\left(z\right)\right)P_{i}^{'}\left(z\right)}{\left(1-P_{i}\left(z\right)\right)\left(-1+P_{i}^{'}\left(z\right)\right)-\left(-z+P_{i}\left(z\right)\right)P_{i}^{'}\left(z\right)}
\end{eqnarray*}


\begin{eqnarray*}
&&lim_{z\rightarrow1^{+}}\frac{\left(1-z\right)P_{i}\left(z\right)F_{i}^{'}\left(z\right)}{\left(1-P_{i}\left(z\right)\right)\left(P_{i}\left(z\right)-z\right)}=lim_{z\rightarrow1^{+}}\frac{\frac{d}{dz}\left[\left(1-z\right)P_{i}\left(z\right)F_{i}^{'}\left(z\right)\right]}{\frac{d}{dz}\left[\left(1-P_{i}\left(z\right)\right)\left(P_{i}\left(z\right)-z\right)\right]}\\
&=&lim_{z\rightarrow1^{+}}\frac{-P_{i}\left(z\right)
F_{i}^{'}\left(z\right)+(1-z) F_{i}^{'}\left(z\right)
P_{i}^{'}\left(z\right)+(1-z)
P_{i}\left(z\right)F_{i}^{''}\left(z\right)}{\left(1-P_{i}\left(z\right)\right)\left(-1+P_{i}^{'}\left(z\right)\right)-\left(-z+P_{i}\left(z\right)\right)P_{i}^{'}\left(z\right)}
\end{eqnarray*}

\footnotesize{
\begin{eqnarray*}
&&lim_{z\rightarrow1^{+}}\frac{\left(1-z\right)\left(1-F_{i}\left(z\right)\right)P_{i}\left(z\right)\left(P_{i}^{'}\left(z\right)-1\right)}{\left(1-P_{i}\left(z\right)\right)\left(P_{i}\left(z\right)-z\right)^{2}}\\
&=&lim_{z\rightarrow1^{+}}\frac{\frac{d}{dz}\left[\left(1-z\right)\left(1-F_{i}\left(z\right)\right)P_{i}\left(z\right)\left(P_{i}^{'}\left(z\right)-1\right)\right]}{\frac{d}{dz}\left[\left(1-P_{i}\left(z\right)\right)\left(P_{i}\left(z\right)-z\right)^{2}\right]}\\
&=&lim_{z\rightarrow1^{+}}\frac{-\left(1-F_{i}\left(z\right)\right) P_{i}\left(z\right)\left(-1+P_{i}^{'}\left(z\right)\right)-(1-z) P_{i}\left(z\right)F_{i}^{'}\left(z\right)\left(-1+P_{i}^{'}\left(z\right)\right)}{2\left(1-P_{i}\left(z\right)\right)\left(-z+P_{i}\left(z\right)\right) \left(-1+P_{i}^{'}\left(z\right)\right)-\left(-z+P_{i}\left(z\right)\right)^2 P_{i}^{'}\left(z\right)}\\
&+&lim_{z\rightarrow1^{+}}\frac{+(1-z) \left(1-F_{i}\left(z\right)\right) \left(-1+P_{i}^{'}\left(z\right)\right) P_{i}^{'}\left(z\right)}{{2\left(1-P_{i}\left(z\right)\right)\left(-z+P_{i}\left(z\right)\right) \left(-1+P_{i}^{'}\left(z\right)\right)-\left(-z+P_{i}\left(z\right)\right)^2 P_{i}^{'}\left(z\right)}}\\
&+&lim_{z\rightarrow1^{+}}\frac{+(1-z)
\left(1-F_{i}\left(z\right)\right)
P_{i}\left(z\right)P_{i}^{''}\left(z\right)}{{2\left(1-P_{i}\left(z\right)\right)\left(-z+P_{i}\left(z\right)\right)
\left(-1+P_{i}^{'}\left(z\right)\right)-\left(-z+P_{i}\left(z\right)\right)^2
P_{i}^{'}\left(z\right)}}
\end{eqnarray*}}

\footnotesize{
%______________________________________________________
\begin{eqnarray*}
&&lim_{z\rightarrow1^{+}}\frac{\left(1-z\right)\left(1-F_{i}\left(z\right)\right)P_{i}^{'}\left(z\right)}{\left(1-P_{i}\left(z\right)\right)\left(P_{i}\left(z\right)-z\right)}=lim_{z\rightarrow1^{+}}\frac{\frac{d}{dz}\left[\left(1-z\right)\left(1-F_{i}\left(z\right)\right)P_{i}^{'}\left(z\right)\right]}{\frac{d}{dz}\left[\left(1-P_{i}\left(z\right)\right)\left(P_{i}\left(z\right)-z\right)\right]}\\
&=&lim_{z\rightarrow1^{+}}\frac{-\left(1-F_{i}\left(z\right)\right)
P_{i}^{'}\left(z\right)-(1-z) F_{i}^{'}\left(z\right)
P_{i}^{'}\left(z\right)+(1-z) \left(1-F_{i}\left(z\right)\right)
P_{i}^{''}\left(z\right)}{\left(1-P_{i}\left(z\right)\right)
\left(-1+P_{i}^{'}\left(z\right)\right)-\left(-z+P_{i}\left(z\right)\right)
P_{i}^{'}\left(z\right)}\frac{}{}
\end{eqnarray*}}

\footnotesize{

%______________________________________________________
\begin{eqnarray*}
&&lim_{z\rightarrow1^{+}}\frac{\left(1-z\right)\left(1-F_{i}\left(z\right)\right)P_{i}\left(z\right)P_{i}^{'}\left(z\right)}{\left(1-P_{i}\left(z\right)\right)^{2}\left(P_{i}\left(z\right)-z\right)}\\
&=&lim_{z\rightarrow1^{+}}\frac{\frac{d}{dz}\left[\left(1-z\right)\left(1-F_{i}\left(z\right)\right)P_{i}\left(z\right)P_{i}^{'}\left(z\right)\right]}{\frac{d}{dz}\left[\left(1-P_{i}\left(z\right)\right)^{2}\left(P_{i}\left(z\right)-z\right)\right]}\\
&=&lim_{z\rightarrow1^{+}}\frac{-\left(1-F_{i}\left(z\right)\right) P_{i}\left(z\right) P_{i}^{'}\left(z\right)-(1-z) P_{i}\left(z\right) F_{i}^{'}\left(z\right)P_i'[z]}{\left(1-P_{i}\left(z\right)\right)^2 \left(-1+P_{i}^{'}\left(z\right)\right)-2 \left(1-P_{i}\left(z\right)\right) \left(-z+P_{i}\left(z\right)\right) P_{i}^{'}\left(\emph{z\right)}\\
&+&lim_{z\rightarrow1^{+}}\frac{(1-z) \left(1-F_{i}\left(z\right)\right) P_{i}^{'}\left(z\right)^2+(1-z) \left(1-F_{i}\left(z\right)\right) P_{i}\left(z\right) P_{i}^{''}\left(z\right)}{\left(1-P_{i}\left(z\right)\right)^2 \left(-1+P_{i}^{'}\left(z\right)\right)-2 \left(1-P_{i}\left(z\right)\right) \left(-z+P_{i}\left(z\right)\right) P_{i}^{'}\left(z\right)}\\
\end{eqnarray*}}



%___________________________________________________________________________________________
\subsection{Longitudes de la Cola en cualquier tiempo}
%___________________________________________________________________________________________



Sea
$V_{i}\left(z\right)=\frac{1}{\esp\left[C_{i}\right]}\frac{I_{i}\left(z\right)-1}{z-P_{i}\left(z\right)}$

%{\esp\lef[I_{i}\right]}\frac{1-\mu_{i}}{z-P_{i}\left(z\right)}

\begin{eqnarray*}
\frac{\partial V_{i}\left(z\right)}{\partial
z}&=&\frac{1}{\esp\left[C_{i}\right]}\left[\frac{I_{i}{'}\left(z\right)\left(z-P_{i}\left(z\right)\right)}{z-P_{i}\left(z\right)}-\frac{\left(I_{i}\left(z\right)-1\right)\left(1-P_{i}{'}\left(z\right)\right)}{\left(z-P_{i}\left(z\right)\right)^{2}}\right]
\end{eqnarray*}


La FGP para el tiempo de espera para cualquier usuario en la cola
est\'a dada por:
\[U_{i}\left(z\right)=\frac{1}{\esp\left[C_{i}\right]}\cdot\frac{1-P_{i}\left(z\right)}{z-P_{i}\left(z\right)}\cdot\frac{I_{i}\left(z\right)-1}{1-z}\]

entonces
%\frac{I_{i}\left(z\right)-1}{1-z}
%+\frac{1-P_{i}\left(z\right)}{z-P_{i}\frac{d}{dz}\left(\frac{I_{i}\left(z\right)-1}{1-z}\right)


\footnotesize{
\begin{eqnarray*}
\frac{d}{dz}V_{i}\left(z\right)&=&\frac{1}{\esp\left[C_{i}\right]}\left\{\frac{d}{dz}\left(\frac{1-P_{i}\left(z\right)}{z-P_{i}\left(z\right)}\right)\frac{I_{i}\left(z\right)-1}{1-z}+\frac{1-P_{i}\left(z\right)}{z-P_{i}\left(z\right)}\frac{d}{dz}\left(\frac{I_{i}\left(z\right)-1}{1-z}\right)\right\}\\
&=&\frac{1}{\esp\left[C_{i}\right]}\left\{\frac{-P_{i}\left(z\right)\left(z-P_{i}\left(z\right)\right)-\left(1-P_{i}\left(z\right)\right)\left(1-P_{i}^{'}\left(z\right)\right)}{\left(z-P_{i}\left(z\right)\right)^{2}}\cdot\frac{I_{i}\left(z\right)-1}{1-z}\right\}\\
&+&\frac{1}{\esp\left[C_{i}\right]}\left\{\frac{1-P_{i}\left(z\right)}{z-P_{i}\left(z\right)}\cdot\frac{I_{i}^{'}\left(z\right)\left(1-z\right)+\left(I_{i}\left(z\right)-1\right)}{\left(1-z\right)^{2}}\right\}
\end{eqnarray*}}
\begin{eqnarray*}
\frac{\partial U_{i}\left(z\right)}{\partial z}&=&\frac{(-1+I_{i}[z]) (1-P_{i}[z])}{(1-z)^2 \esp[I_{i}] (z-P_{i}[z])}+\frac{(1-P_{i}[z]) I_{i}^{'}[z]}{(1-z) \esp[I_{i}] (z-P_{i}[z])}\\
&-&\frac{(-1+I_{i}[z]) (1-P_{i}[z])\left(1-P{'}[z]\right)}{(1-z) \esp[I_{i}] (z-P_{i}[z])^2}-\frac{(-1+I_{i}[z]) P_{i}{'}[z]}{(1-z) \esp[I_{i}](z-P_{i}[z])}
\end{eqnarray*}



\chapter{Revi\'on de Procesos Regenerativos}
\documentclass{article}
\usepackage[utf8]{inputenc}
\usepackage[spanish,english]{babel}
\usepackage{amsmath,amssymb,amsthm,amsfonts}
\usepackage{geometry}
\usepackage{hyperref}
\usepackage{fancyhdr}
\usepackage{titlesec}
\usepackage{listings}
\usepackage{graphicx,graphics}
\usepackage{multicol}
\usepackage{multirow}
\usepackage{color}
\usepackage{float} 
\usepackage{subfig}
\usepackage[figuresright]{rotating}
\usepackage{enumerate}
\usepackage{anysize} 
\usepackage{url}

\title{Procesos Regenerativos: Revisi\'on}
\author{Carlos E. Martínez-Rodríguez}
\date{Julio 2024}

\geometry{
  a4paper,
  left=25mm,
  right=25mm,
  top=30mm,
  bottom=30mm,
}

% Configuración de encabezados y pies de página
\pagestyle{fancy}
\fancyhf{}
\fancyhead[L]{\leftmark}
\fancyfoot[C]{\thepage}
\fancyfoot[R]{\rightmark}
\fancyfoot[L]{Carlos E. Martínez-Rodríguez}

% Definiciones de nuevos entornos
\newtheorem{Algthm}{Algoritmo}
\newtheorem{Def}{Definición}
\newtheorem{Ejem}{Ejemplo}
\newtheorem{Teo}{Teorema}
\newtheorem{Dem}{Demostración}
\newtheorem{Note}{Nota}
\newtheorem{Sol}{Solución}
\newtheorem{Prop}{Proposición}
\newtheorem{Cor}{Corolario}
\newtheorem{Col}{Corolario}
\newtheorem{Coro}{Corolario}
\newtheorem{Lemma}{Lema}
\newtheorem{Lem}{Lema}
\newtheorem{Lema}{Lema}
\newtheorem{Sup}{Supuestos}
\newtheorem{Assumption}{Supuestos}
\newtheorem{Remark}{Observación}
\newtheorem{Condition}{Condición}
\newtheorem{Theorem}{Teorema}
\newtheorem{Corollary}{Corolario}
\newtheorem{Ejemplo}{Ejemplo}
\newtheorem{Example}{Ejemplo}
\newtheorem{Obs}{Observación}

% Nuevos comandos
\def\RR{\mathbb{R}}
\def\ZZ{\mathbb{Z}}
\newcommand{\nat}{\mathbb{N}}
\newcommand{\ent}{\mathbb{Z}}
\newcommand{\rea}{\mathbb{R}}
\newcommand{\Eb}{\mathbf{E}}
\newcommand{\esp}{\mathbb{E}}
\newcommand{\prob}{\mathbb{P}}
\newcommand{\indora}{\mbox{$1$\hspace{-0.8ex}$1$}}
\newcommand{\ER}{\left(E,\mathcal{E}\right)}
\newcommand{\KM}{\left(P_{s,t}\right)}
\newcommand{\Xt}{\left(X_{t}\right)_{t\in I}}
\newcommand{\PE}{\left(X_{t}\right)_{t\in I}}
\newcommand{\SG}{\left(P_{t}\right)}
\newcommand{\CM}{\mathbf{P}^{x}}
\newcommand\mypar{\par\vspace{\baselineskip}}

\begin{document}

\maketitle

\tableofcontents
%<>===<>==<>===<>==<>===<>==<>===<>==<>===<>==<>===<>==<>===<>==<>===<>==<>===
%________________________________________________________________________
\section{Procesos Regenerativos: Thorisson}
%________________________________________________________________________
%________________________________________________________________________
\subsection{Tiempos de Regeneraci\'on para Redes de Sistemas de Visitas C\'iclicas}
%________________________________________________________________________
\begin{Teo}
Dada una Red de Sistemas de Visitas C\'iclicas (RSVC), conformada por dos Sistemas de Visitas C\'iclicas (SVC), donde cada uno de ellos consta de dos colas tipo $M/M/1$. Los dos sistemas est\'an comunicados entre s\'i por medio de la transferencia de usuarios entre las colas $Q_{1}\leftrightarrow Q_{3}$ y $Q_{2}\leftrightarrow Q_{4}$.

\end{Teo}

\begin{proof}

Para cada cola $Q_{j}$, $j=1,\ldots,4$, se tienen los siguientes procesos $L_{j}\left(t\right)$ el n\'umero de usuarios presentes en la cola al tiempo $t$, $A_{j}\left(t\right)$ el residual del tiempo de arribo del siguiente usuario. $B_{j}\left(t\right)$ el residual del tiempo de servicio del usuario que est\'a siendo atendido. $C_{j}\left(t\right)$ el residual del tiempo de traslado del servidor entre una cola y otra, en caso de que se encuentre dando servicio se considera $C_{j}\left(t\right)=0$, para $j=1,\ldots,4$. Con base en lo anterior se tienen los procesos
\begin{eqnarray}\label{Procesos.RSVC}
L\left(t\right)=\left(L_{j}\left(t\right)\right)_{j=1}^{4},
A\left(t\right)=\left(A_{j}\left(t\right)\right)_{j=1}^{4}, B\left(t\right)=\left(B_{j}\left(t\right)\right)_{j=1}^{4}
\textrm{ y } C\left(t\right)=\left(C_{j}\left(t\right)\right)_{j=1}^{4}.
\end{eqnarray}
Por lo tanto se tiene el proceso estoc\'astico
\begin{eqnarray}\label{Proceso.Estocastico.Z}
\mathbb{Z}=\left(L\left(t\right),A\left(t\right),
B\left(t\right),C\left(t\right)\right)
\end{eqnarray}
Para los procesos residuales de los tiempos de traslado, servicio y de arribos, su espacio de estados es un subconjunto de $\rea_{+}=\left[0,\infty\right)$, es decir, $E\subset\left[0,\infty\right)$ y por tanto $\mathcal{E}\subset\mathcal{B}\left[0,\infty\right)$, luego el espacio $\left(E,\mathcal{E}\right)$ es un espacio polaco.
Para cada proceso de residuales se tienen los siguientes espacios producto: Para $A\left(t\right)=\left(A_{j}\left(t\right)\right)_{j=1}^{4}$ se tiene el espacio producto $\left(E_{2},\mathcal{E}_{2}\right)=\otimes_{j=1}^{4}\left(E_{j},\mathcal{E}_{j}\right)$, para $B\left(t\right)=\left(B_{j}\left(t\right)\right)_{j=1}^{4}$ se tiene el espacio producto $\left(E_{3},\mathcal{E}_{3}\right)=\otimes_{j=1}^{4}\left(E_{j},\mathcal{E}_{j}\right)$,
para $C\left(t\right)=\left(C_{j}\left(t\right)\right)_{j=1}^{4}$ se tiene el espacio producto $\left(E_{4},\mathcal{E}_{4}\right)=\otimes_{j=1}^{4}\left(E_{j},\mathcal{E}_{j}\right)$.

En lo que respecta al proceso $L\left(t\right)=\left(L_{j}\left(t\right)\right)_{j=1}^{4}$
 el proceso de estados $E_{j}\subset\mathbb{N}$ y $\mathcal{E}_{j}\subset\sigma\left(E\right)$, por lo tanto el espacio producto $\left(E_{1},\mathcal{E}_{1}\right)=\otimes_{j=1}^{4}\left(E_{j},\mathcal{E}_{j}\right)$ que adem\'as tambi\'en resulta ser polaco. Entonces con los espacios productos $\left(E_{i},\mathcal{E}_{i}\right)_{i=1}^{4}$, se define el espacio producto $\left(E,\mathcal{E}\right)=\otimes_{i=1}^{4} \left(E_{i},\mathcal{E}_{i}\right)$ que nuevamente resulta ser un espacio polaco. De acuerdo con Thorisson existe un espacio de probabilidad $\left(\Omega,\mathcal{F},\prob\right)$ en el que est\'a definido el proceso estoc\'astico definido en  (\ref{Proceso.Estocastico.Z}) que toma valores en $\left(E,\mathcal{E}\right)$.
  
Con la finalidad de analizar las propiedades del proceso $\mathbb{Z}$ consideremos el conjunto de \'indices $\mathbb{I}=\left[0,\infty\right)$, entonces tenemos el elemento aleatorio $\mathbb{Z}=\left(Z\right)_{s\in\mathbb{I}}$ que est\'a definido en el espacio de probabilidad $\left(\Omega,\mathcal{F},\prob\right)$ y con valores en $\left(E,\mathcal{E}\right)$. El proceso $Z$ as\'i definido es un PEOSCT conforme a la definici\'on dada en (\ref{PEOSCT}). Ahora consideremos al espacio de trayectorias de $Z$ conforme a la definici\'on (\ref{Conjunto.Trayectorias}); por construcci\'on el espacio de trayectorias $H:=D_{E}\left[0,\infty\right)$ que por la nota (\ref{Conjunto.Trayectorias}) resulta ser que el Proceso es Canonicamente Conjuntamente Medible (CCM) y por la nota (\ref{Nota.ISI.sii.CCM}) adem\'as es Internamente Shift Invariante (ISI), es decir, resulta ser un proceso estoc\'astico one-side a tiempo continuo shift medible, y por lo tanto satisface la primera parte de las hip\'otesis del Teorema (\ref{Tma.Existencia.Tiempos.Regeneracion}). 

Conforme a la construcci\'on dada en la secci\'on 1, se tiene que los dos tiempos $S_{0}=0$ y $S_{1}=T^{*}$ satisfacen la segunda parte de las hip\'otesis del Teorema (\ref{Tma.Existencia.Tiempos.Regeneracion}) y por tanto se puede asegurar que existe un espacio de probabilidad $\left(\Omega,\mathcal{F},\prob\right)$ en el cu\'al existe una sucesi\'on de tiempos aleatorios en los cuales el proceso se regenera, es decir, se garantiza que existe una sucesi\'on de tiempos de regeneraci\'on $T_{0}, T_{1},\ldots$ en los cuales el proceso $L\left(T_{k}\right)=\left(0,0,0,0\right)$.

Adem\'as por el Corolario (\ref{Tma.Estacionariedad}) se garantiza que existe una versi\'on estacionaria del proceso $\left(Z,S\right)$.
\end{proof}

\newpage

%_________________________________________________________________________
\subsection{Introduction to Stochastic Processes}
%_________________________________________________________________________

\begin{Def}
Un elemento aleatorio con valores en un espacio medible $\left(E,\mathcal{E}\right)$, es un mapeo definido en un espacio de probabilidad $\left(\Omega,\mathcal{F},\prob\right)$ a $\left(E,\mathcal{E}\right)$, es decir,
para $A\in \mathcal{E}$,  se tiene que $\left\{Y\in A\right\}\in\mathcal{F}$, donde $\left\{Y\in A\right\}:=\left\{w\in\Omega:Y\left(w\right)\in A\right\}=:Y^{-1}A$.
\end{Def}

\begin{Note}
Tambi\'en se dice que $Y$ est\'a soportado por el espacio de probabilidad $\left(\Omega,\mathcal{F},\prob\right)$ y que $Y$ es un mapeo medible de $\Omega$ en $E$, es decir, es \textbf{$\mathcal{F}/\mathcal{E}$ medible}.
\end{Note}

\begin{Def}
Para cada $i\in \mathbb{I}$, sea $P_{i}$ una medida de probabilidad en un espacio medible $\left(E_{i},\mathcal{E}_{i}\right)$. Se define el espacio producto
$\otimes_{i\in\mathbb{I}}\left(E_{i},\mathcal{E}_{i}\right):=\left(\prod_{i\in\mathbb{I}}E_{i},\otimes_{i\in\mathbb{I}}\mathcal{E}_{i}\right)$, donde $\prod_{i\in\mathbb{I}}E_{i}$ es el producto cartesiano de los $E_{i}$'s, y $\otimes_{i\in\mathbb{I}}\mathcal{E}_{i}$ es la \textbf{$\sigma$-\'algebra producto}, es decir, es la $\sigma$-\'algebra m\'as peque\~na en $\prod_{i\in\mathbb{I}}E_{i}$ que hace al $i$-\'esimo mapeo proyecci\'on en $E_{i}$ medible para toda $i\in\mathbb{I}$, es la $\sigma$-\'algebra inducida por los mapeos proyecci\'on, es decir
$$\otimes_{i\in\mathbb{I}}\mathcal{E}_{i}:=\sigma\left\{\left\{y:y_{i}\in A\right\}:i\in\mathbb{I}\textrm{ y }A\in\mathcal{E}_{i}\right\}.$$
\end{Def}

\begin{Def}
Un espacio de probabilidad $\left(\tilde{\Omega},\tilde{\mathcal{F}},\tilde{\prob}\right)$ es una \textbf{extensi\'on de otro espacio de probabilidad $\left(\Omega,\mathcal{F},\prob\right)$} si $\left(\tilde{\Omega},\tilde{\mathcal{F}},\tilde{\prob}\right)$ soporta un elemento aleatorio $\xi\in\left(\Omega,\mathcal{F}\right)$ que tienen a $\prob$ como distribuci\'on.
\end{Def}

\begin{Teo}
Sea $\mathbb{I}$ un conjunto de \'indices arbitrario. Para cada $i\in\mathbb{I}$ sea $P_{i}$ una medida de probabilidad en un espacio medible $\left(E_{i},\mathcal{E}_{i}\right)$. Entonces existe una \'unica medida de probabilidad $\otimes_{i\in\mathbb{I}}P_{i}$ en $\otimes_{i\in\mathbb{I}}\left(E_{i},\mathcal{E}_{i}\right)$ tal que 

\begin{eqnarray*}
\otimes_{i\in\mathbb{I}}P_{i}\left(y\in\prod_{i\in\mathbb{I}}E_{i}:y_{i}\in A_{i_{1}},\ldots,y_{n}\in A_{i_{n}}\right)=P_{i_{1}}\left(A_{i_{n}}\right)\cdots P_{i_{n}}\left(A_{i_{n}}\right)
\end{eqnarray*}
para todos los enteros $n>0$, toda $i_{1},\ldots,i_{n}\in\mathbb{I}$ y todo $A_{i_{1}}\in\mathcal{E}_{i_{1}},\ldots,A_{i_{n}}\in\mathcal{E}_{i_{n}}$
\end{Teo}

La medida $\otimes_{i\in\mathbb{I}}P_{i}$ es llamada la \textbf{medida producto} y $\otimes_{i\in\mathbb{I}}\left(E_{i},\mathcal{E}_{i},P_{i}\right):=\left(\prod_{i\in\mathbb{I}},E_{i},\otimes_{i\in\mathbb{I}}\mathcal{E}_{i},\otimes_{i\in\mathbb{I}}P_{i}\right)$, es llamado \textbf{espacio de probabilidad producto}.


\begin{Def}
Un espacio medible $\left(E,\mathcal{E}\right)$ es \textbf{\textit{Polaco}} si existe una m\'etrica en $E$ tal que $E$ es completo, es decir cada sucesi\'on de Cauchy converge a un l\'imite en $E$, y \textit{separable}, $E$ tienen un subconjunto denso numerable, y tal que $\mathcal{E}$ es generado por conjuntos abiertos.
\end{Def}


\begin{Def}
Dos espacios medibles $\left(E,\mathcal{E}\right)$ y $\left(G,\mathcal{G}\right)$ son Borel equivalentes (\textit{isomorfos}) si existe una biyecci\'on $f:E\rightarrow G$ tal que $f$ es $\mathcal{E}/\mathcal{G}$ medible y su inversa $f^{-1}$ es $\mathcal{G}/\mathcal{E}$ medible. La biyecci\'on es una equivalencia de Borel.
\end{Def}

\begin{Def}
Un espacio medible  $\left(E,\mathcal{E}\right)$ es un \textbf{espacio est\'andar} si es Borel equivalente a $\left(G,\mathcal{G}\right)$, donde $G$ es un subconjunto de Borel de $\left[0,1\right]$ y $\mathcal{G}$ son los subconjuntos de Borel de $G$.
\end{Def}

\begin{Note}
Cualquier espacio polaco es un espacio est\'andar.
\end{Note}


\begin{Def}
Un proceso estoc\'astico con conjunto de \'indices $\mathbb{I}$ y espacio de estados $\left(E,\mathcal{E}\right)$ es una familia $Z=\left(\mathbb{Z}_{s}\right)_{s\in\mathbb{I}}$ donde $\mathbb{Z}_{s}$ son elementos aleatorios definidos en un espacio de probabilidad com\'un $\left(\Omega,\mathcal{F},\prob\right)$ y todos toman valores en $\left(E,\mathcal{E}\right)$.
\end{Def}

\begin{Def}\label{PEOSCT}
Un proceso estoc\'astico \textit{one-sided contiuous time} (\textbf{PEOSCT}) es un proceso estoc\'astico con conjunto de \'indices $\mathbb{I}=\left[0,\infty\right)$.
\end{Def}


El espacio $\left(E^{\mathbb{I}},\mathcal{E}^{\mathbb{I}}\right)$ denota el espacio producto $\left(E^{\mathbb{I}},\mathcal{E}^{\mathbb{I}}\right):=\otimes_{s\in\mathbb{I}}\left(E,\mathcal{E}\right)$. Vamos a considerar $\mathbb{Z}$ como un mapeo aleatorio, es decir, como un elemento aleatorio en $\left(E^{\mathbb{I}},\mathcal{E}^{\mathbb{I}}\right)$ definido por $Z\left(w\right)=\left(Z_{s}\left(w\right)\right)_{s\in\mathbb{I}}$ y $w\in\Omega$.

\begin{Note}
La distribuci\'on de un proceso estoc\'astico $Z$ es la distribuci\'on de $Z$ como un elemento aleatorio en $\left(E^{\mathbb{I}},\mathcal{E}^{\mathbb{I}}\right)$. La distribuci\'on de $Z$ esta determinada de manera \'unica por las distribuciones finito dimensionales.
\end{Note}

\begin{Note}
En particular cuando $Z$ toma valores reales, es decir, $\left(E,\mathcal{E}\right)=\left(\mathbb{R},\mathcal{B}\right)$ las distribuciones finito dimensionales est\'an determinadas por las funciones de distribuci\'on finito dimensionales

\begin{eqnarray}
\prob\left(Z_{t_{1}}\leq x_{1},\ldots,Z_{t_{n}}\leq x_{n}\right),x_{1},\ldots,x_{n}\in\mathbb{R},t_{1},\ldots,t_{n}\in\mathbb{I},n\geq1.
\end{eqnarray}
\end{Note}

\begin{Note}
Para espacios polacos $\left(E,\mathcal{E}\right)$ el \textbf{Teorema de Consistencia de Kolmogorov} asegura que dada una colecci\'on de distribuciones finito dimensionales consistentes, siempre existe un proceso estoc\'astico que posee tales distribuciones finito dimensionales.
\end{Note}


\begin{Def}\label{Conjunto.Trayectorias}
Las trayectorias de $Z$ son las realizaciones $Z\left(w\right)$ para $w\in\Omega$ del mapeo aleatorio $Z$.
\end{Def}

\begin{Note}
Algunas restricciones se imponen sobre las trayectorias, por ejemplo que sean continuas por la derecha, o continuas por la derecha con l\'imites por la izquierda, o de manera m\'as general, se pedir\'a que caigan en alg\'un subconjunto $H$ de $E^{\mathbb{I}}$. En este caso es natural considerar a $Z$ como un elemento aleatorio que no est\'a en $\left(E^{\mathbb{I}},\mathcal{E}^{\mathbb{I}}\right)$ sino en $\left(H,\mathcal{H}\right)$, donde $\mathcal{H}$ es la $\sigma$-\'algebra generada por los mapeos proyecci\'on que toman a $z\in H$ en $z_{t}\in E$ para $t\in\mathbb{I}$. A $\mathcal{H}$ se le conoce como la traza de $H$ en $E^{\mathbb{I}}$, es decir,
\begin{eqnarray}
\mathcal{H}:=E^{\mathbb{I}}\cap H&:=&\left\{A\cap H:A\in E^{\mathbb{I}}\right\}.\\
Z_{t}:\left(\Omega.\mathcal{F}\right)&\rightarrow&\left(H,\mathcal{H}\right)
\end{eqnarray}
\end{Note}


\begin{Note}
$Z$ tiene \textbf{trayectorias con valores en $H$} y cada $Z_{t}$ es un mapeo medible de $\left(\Omega,\mathcal{F}\right)$ a $\left(H,\mathcal{H}\right)$. Cuando se considera un espacio de trayectorias en particular $H$, al espacio $\left(H,\mathcal{H}\right)$ se le llama \textbf{el espacio de trayectorias de $Z$}.
\end{Note}

\begin{Note}
La distribuci\'on del proceso estoc\'astico $Z$ con espacio de trayectorias $\left(H,\mathcal{H}\right)$ es la distribuci\'on de $Z$ como  un elemento aleatorio en $\left(H,\mathcal{H}\right)$. La distribuci\'on, nuevemente, est\'a determinada de manera \'unica por las distribuciones finito dimensionales.
\end{Note}


\begin{Def}
Sea $Z$ un PEOSCT (ver definici\'on \ref{PEOSCT}) con espacio de estados $\left(E,\mathcal{E}\right)$ y sea $T$ un tiempo aleatorio en $\left[0,\infty\right)$. Por $Z_{T}$ se entiende el mapeo con valores en $E$ definido en $\Omega$ por:
\begin{eqnarray*}
Z_{T}\left(w\right)&:=&Z_{T\left(w\right)}\left(w\right), w\in\Omega.\\
Z_{t}:\left(\Omega,\mathcal{F}\right)&\rightarrow&\left(E,\mathcal{E}\right).
\end{eqnarray*}
\end{Def}

\begin{Def}
Un PEOSCT $Z$ es conjuntamente medible (\textbf{CM}), es decir un \textbf{PEOSCTCM}, si el mapeo que toma $\left(w,t\right)\in\Omega\times\left[0,\infty\right)$ a $Z_{t}\left(w\right)\in E$ es $\mathcal{F}\otimes\mathcal{B}\left[0,\infty\right)/\mathcal{E}$ medible.
\begin{eqnarray*}
\left(\Omega,\left[0,\infty\right)\right)&\rightarrow&\left(E,\mathcal{E}\right)\\
\left(w,t\right)&\rightarrow& Z_{t}\left(w\right).
\end{eqnarray*}
\end{Def}

\begin{Note}
Un PEOSCT-CM implica que el proceso es medible, dado que $Z_{T}$ es una composici\'on  de dos mapeos continuos: el primero que toma $w$ en $\left(w,T\left(w\right)\right)$ es $\mathcal{F}/\mathcal{F}\otimes\mathcal{B}\left[0,\infty\right)$ medible, mientras que el segundo toma $\left(w,T\left(w\right)\right)$ en $Z_{T\left(w\right)}\left(w\right)$ es $\mathcal{F}\otimes\mathcal{B}\left[0,\infty\right)/\mathcal{E}$ medible.
\end{Note}


\begin{Def}
Un PEOSCT con espacio de estados $\left(H,\mathcal{H}\right)$ es can\'onicamente conjuntamente medible (\textbf{CCM}) si el mapeo $\left(z,t\right)\in H\times\left[0,\infty\right)$ en $Z_{t}\in E$ es $\mathcal{H}\otimes\mathcal{B}\left[0,\infty\right)/\mathcal{E}$ medible.
\begin{eqnarray*}
\left(H\times\left[0,\infty\right),\mathcal{H}\times\mathcal{B}\left[0,\infty\right)\right)&\rightarrow& \left(E,\mathcal{E}\right)\\
\left(z,t\right)&\rightarrow& Z_{t}
\end{eqnarray*}
\end{Def}

\begin{Note}
Un PEOSCTCCM implica que el proceso es CM, dado que un PEOSCTCCM $Z$ es un mapeo de $\Omega\times\left[0,\infty\right)$ a $E$, es la composici\'on de dos mapeos medibles: el primero, toma $\left(w,t\right)$ en $\left(Z\left(w\right),t\right)$ es $\mathcal{F}\otimes\mathcal{B}\left[0,\infty\right)/\mathcal{H}\otimes\mathcal{B}\left[0,\infty\right)$ medible, y el segundo que toma $\left(Z\left(w\right),t\right)$  en $Z_{t}\left(w\right)$ es $\mathcal{H}\otimes\mathcal{B}\left[0,\infty\right)/\mathcal{E}$ medible. Por tanto CCM es una condici\'on m\'as fuerte que CM.
\begin{eqnarray*}
\left(\Omega\times\left[0,\infty\right),\mathcal{F}\times\mathcal{B}\left[0,\infty\right)\right)
&\rightarrow& 
\left(H\times\left[0,\infty\right),\mathcal{H}\times\mathcal{B}\left[0,\infty\right)\right)
\rightarrow\left(E,\mathcal{E}\right)\\
\left(w,t\right)&\rightarrow& 
\left(Z\left(w\right),t\right])\rightarrow Z_{t}\left(w\right)
\end{eqnarray*}

\end{Note}

\begin{Def}
Un conjunto de trayectorias $H$ de un PEOSCT $Z$, es internamente shift-invariante (\textbf{ISI}) si 
\begin{eqnarray*}
\left\{\left(z_{t+s}\right)_{s\in\left[0,\infty\right)}:z\in H\right\}=H\textrm{, }t\in\left[0,\infty\right).
\end{eqnarray*}
\end{Def}


\begin{Def}
Dado un PEOSCTISI, se define el mapeo-shift $\theta_{t}$, $t\in\left[0,\infty\right)$, de $H$ a $H$ por 
\begin{eqnarray*}
\theta_{t}z=\left(z_{t+s}\right)_{s\in\left[0,\infty\right)}\textrm{, }z\in H.
\end{eqnarray*}
\end{Def}

\begin{Def}
Se dice que un proceso $Z$ es shift-medible (\textbf{SM}) si $Z$ tiene un conjunto de trayectorias $H$ que es ISI y adem\'as el mapeo que toma $\left(z,t\right)\in H\times\left[0,\infty\right)$ en $\theta_{t}z\in H$ es $\mathcal{H}\otimes\mathcal{B}\left[0,\infty\right)/\mathcal{H}$ medible.
\begin{eqnarray*}
\left(H\times\left[0,\infty\right),\mathcal{H}\times\mathcal{B}\left[0,\infty\right)\right)
&\rightarrow& 
\left(H,\mathcal{H}\right)\\
\left(z,t\right)&\rightarrow& 
\theta_{t}\left(z\right)
\end{eqnarray*}

\end{Def}

\begin{Note}\label{Nota.ISI.sii.CCM}
Un proceso estoc\'astico (PEOSCT) con conjunto de trayectorias $H$ ISI es shift-medible si y s\'olo si es PEOSCTCCM.
\end{Note}

\begin{Note}\label{Nota.ISI.CCM}
\begin{itemize}
\item Por la nota (\ref{Nota.ISI.sii.CCM}) dado el espacio polaco $\left(E,\mathcal{E}\right)$ si se tiene el  conjunto de trayectorias $D_{E}\left[0,\infty\right)$, que es ISI, entonces cumple con ser CCM.

\item Si $G$ es abierto, podemos cubrirlo por bolas abiertas cuya cerradura este contenida en $G$, y como $G$ es segundo numerable como subespacio de $E$, lo podemos cubrir por una cantidad numerable de bolas abiertas.

\end{itemize}
\end{Note}


\begin{Note}
Los procesos estoc\'asticos $Z$ a tiempo discreto con espacio de estados polaco, tambi\'en tiene un espacio de trayectorias polaco y por tanto tiene distribuciones condicionales regulares.
\end{Note}

\begin{Teo}
El producto numerable de espacios polacos es polaco.
\end{Teo}

%__________________________________________________________
\subsection{One Sided Process}
%___________________________________________________________

%\begin{Def}
Sea $\left(\Omega,\mathcal{F},\prob\right)$ espacio de probabilidad que soporta al proceso $Z=\left(Z_{s}\right)_{s\in\left[0,\infty\right)}$ y $S=\left(S_{k}\right)_{0}^{\infty}$ donde $Z$ es un PEOSCTM con espacio de estados $\left(E,\mathcal{E}\right)$  y espacio de trayectorias $\left(H,\mathcal{H}\right)$  y adem\'as $S$ es una sucesi\'on de tiempos aleatorios one-sided que satisfacen la condici\'on $0\leq S_{0}<S_{1}<\cdots\rightarrow\infty$. Considerando $S$ como un mapeo medible de $\left(\Omega,\mathcal{F}\right)$ al espacio sucesi\'on $\left(L,\mathcal{L}\right)$, $S:\left(\Omega,\mathcal{F}\right)\rightarrow\left(L,\mathcal{L}\right)$, donde 
\begin{eqnarray*}
L=\left\{\left(s_{k}\right)_{0}^{\infty}\in\left[0,\infty\right)^{\left\{0,1,\ldots\right\}}:s_{0}<s_{1}<\cdots\rightarrow\infty\right\},
\end{eqnarray*}
donde $\mathcal{L}$ son los subconjuntos de Borel de $L$, es decir, $\mathcal{L}=L\cap\mathcal{B}^{\left\{0,1,\ldots\right\}}$.

As\'i el par $\left(Z,S\right)$ es un mapeo medible de  $\left(\Omega,\mathcal{F}\right)$ en $\left(H\times L,\mathcal{H}\otimes\mathcal{L}\right)$. El par $\mathcal{H}\otimes\mathcal{L}^{+}$ denotar\'a la clase de todas las funciones medibles de $\left(H\times L,\mathcal{H}\otimes\mathcal{L}\right)$ en $\left(\left[0,\infty\right),\mathcal{B}\left[0,\infty\right)\right)$.
%\end{Def}

\begin{eqnarray*}
\left(Z,S\right):\left(\Omega,\mathcal{F}\right)&\rightarrow& \left(H\times L,\mathcal{H}\times\mathcal{L}\right)\\
\mathcal{H}\times\mathcal{L}^{*}:\left(H\times L,\mathcal{H}\times\mathcal{L}\right)
&\rightarrow& 
\left(\left[0,\infty\right),\mathcal{B}\left[0,\infty\right)\right).
\end{eqnarray*}



%_________________________________________________________
\subsection{Regeneration: Shift-Measurability}
%__________________________________________________________

\begin{Def}
Sea $\theta_{t}$ el mapeo-shift conjunto de $H\times L$ en $H\times L$ dado por
\begin{eqnarray*}
\theta_{t}\left(z,\left(s_{k}\right)_{0}^{\infty}\right)=\theta_{t}\left(z,\left(s_{n_{t-}+k}-t\right)_{0}^{\infty}\right)
\end{eqnarray*}
donde 
$n_{t-}=inf\left\{n\geq1:s_{n}\geq t\right\}$.
\end{Def}


\begin{Note}
Con la finalidad de poder realizar los shift's sin complicaciones de medibilidad, se supondr\'a que $Z$ es shit-medible, es decir, el conjunto de trayectorias $H$ es invariante bajo shifts del tiempo y el mapeo que toma $\left(z,t\right)\in H\times\left[0,\infty\right)$ en $z_{t}\in E$ es $\mathcal{H}\otimes\mathcal{B}\left[0,\infty\right)/\mathcal{E}$ medible.
\end{Note}




%_________________________________________________________
\subsection{Cycle-Stationarity}
%_________________________________________________________
%\textit{\textbf{Faltan definiciones}}
\begin{Def}
Los tiempos aleatorios $S_{n}$ dividen $Z$ en 

\begin{itemize}
\item[a)] un retraso $D=\left(Z_{s}\right)_{s\in\left[0,\infty\right)}$,
\item[b)] una sucesi\'on de ciclos $C_{n}=\left(Z_{S_{n-1}+s}\right)_{
s\in\left[0,X_{n}\right)}$, $n\geq1$,
\item[c)] las longitudes de los ciclos $X_{n}=S_{n}-S_{n-1}$, $n\neq1$.
\end{itemize}
\end{Def}

\begin{Note}
\begin{itemize}
\item[a)] El retraso $D$ y los ciclos $C_{n}$ son procesos estoc\'asticos que se desvanecen en los tiempos aleatorios $S_{0}$ y $X_{n}$ respectivamente.
\item[b)] Las longitudes de los ciclos $X_{1},X_{2},\ldots$ y el retraso de la longitud (\textit{delay-length}) $S_{0}$ son obtenidos por el mismo mapeo medible de sus respectivos ciclos $C_{1},C_{2},\ldots$ y el retraso $D$. 
\item[c)] El par $\left(Z,S\right)$ es un mapeo medible del retraso y de los ciclos y viceversa.
\end{itemize}
\end{Note}

\begin{Def}
$\left(Z,S\right)$ es \textit{zero-delayed} si $S_{0}\equiv0$. Se define el par \textit{zero-delayed} por
\begin{eqnarray*}
\left(Z^{0},S^{0}\right):=\theta_{S_{0}}\left(Z,S\right)
\end{eqnarray*}
Entonces $S_{0}^{0}\equiv0$ y $S_{0}^{0}\equiv X_{1}^{0}$, mientras que para $n\geq1$ se tiene que $X_{n}^{0}\equiv X_{n}$ y $C_{n}^{0}\equiv C_{n}$.
\end{Def}

\begin{Def}
Se le llama al par $\left(Z,S\right)$ \textbf{ciclo-stacionario} si los ciclos forman una sucesi\'on estacionaria, es decir, con $=^{D}$ denota iguales en distribuci\'on:
\begin{eqnarray*}
\left(C_{n+1},C_{n+2},\ldots\right)=^{D}\left(C_{1},C_{2},\ldots\right),\geq0
\end{eqnarray*}
Ciclo-estacionareidad es equivalente a 
\begin{eqnarray*}
\theta_{S_{n}}\left(Z,S\right)=^{D}
\left(Z^{0},S^{0}\right),\geq0,
\end{eqnarray*}
donde $\left(C_{n+1},C_{n+2},\ldots\right)$ y $\theta_{S_{n}}\left(Z,S\right)$ son mapeos medibles de cada uno y que no dependen de $n$.
\end{Def}


\begin{Def}
Un par $\left(Z^{*},S^{*}\right)$ es \textbf{estacionario} si $\theta\left(Z^{*},S^{*}\right)=^{D}
\left(Z^{*},S^{*}\right)$, para $t\geq0$.
\end{Def}


\begin{Teo}\label{Teorema.2.1}
Supongase que $\left(Z,S\right)$ es cycle-stationary con $\esp\left[X_{1}\right]<\infty$. Sea $U$ distribuida uniformemente en $\left[0,1\right)$ e independiente de $\left(Z^{0},S^{0}\right)$ y sea $\prob^{*}$ la medida de probabilidad en $\left(\Omega,\prob\right)$ definida por $$d\prob^{*}=\frac{X_{1}}{\esp\left[X_{1}\right]}d\prob$$. Sea $\left(Z^{*},S^{*}\right)$ con distribuci\'on $\prob^{*}\left(\theta_{UX_{1}}\left(Z^{0},S^{0}\right)\in\cdot\right)$. Entonces $\left(Z^{*},S^{*}\right)$ es estacionario,
\begin{eqnarray*}
\esp\left[f\left(Z^{*},S^{*}\right)\right]=\esp\left[\int_{0}^{X_{1}}f\left(\theta_{s}\left(Z^{0},S^{0}\right)\right)ds\right]/\esp\left[X_{1}\right]
\end{eqnarray*}
$f\in\mathcal{H}\otimes\mathcal{L}^{+}$, and $S_{0}^{*}$ es continuo con funci\'on distribuci\'on $G_{\infty}$ definida por $$G_{\infty}\left(x\right):=\frac{\esp\left[X_{1}\right]\wedge x}{\esp\left[X_{1}\right]}$$ para $x\geq0$ y densidad $\prob\left[X_{1}>x\right]/\esp\left[X_{1}\right]$, con $x\geq0$.

\end{Teo}

%___________________________________________________________
\subsection{Classical Regeneration}
%___________________________________________________________

\begin{Def}
Dado un proceso \textbf{PEOSSM} (Proceso Estoc\'astico One Side Shift Medible) $Z$, se dice \textbf{regenerativo cl\'asico} con tiempos de regeneraci\'on $S$ si 

\begin{eqnarray*}
\theta_{S_{n}}\left(Z,S\right)=\left(Z^{0},S^{0}\right),n\geq0
\end{eqnarray*}
y adem\'as $\theta_{S_{n}}\left(Z,S\right)$ es independiente de $\left(\left(Z_{s}\right)s\in\left[0,S_{n}\right),S_{0},\ldots,S_{n}\right)$
Si lo anterior se cumple, al par $\left(Z,S\right)$ se le llama regenerativo cl\'asico.
\end{Def}

\begin{Note}
Si el par $\left(Z,S\right)$ es regenerativo cl\'asico, entonces las longitudes de los ciclos $X_{1},X_{2},\ldots,$ son i.i.d. e independientes de la longitud del retraso $S_{0}$, es decir, $S$ es un \textbf{proceso de renovaci\'on}. Las longitudes de los ciclos tambi\'en son llamados tiempos de inter-regeneraci\'on y tiempos de ocurrencia.

\end{Note}

%___________________________________________________________
\subsection{Stationary Version}
%___________________________________________________________



\begin{Teo}\label{Teo.3.1}
Sup\'ongase que el par $\left(Z,S\right)$ es regenerativo cl\'asico con $\esp\left[X_{1}\right]<\infty$. Entonces $\left(Z^{*},S^{*}\right)$ en el teorema \ref{Teorema.2.1} es una versi\'on estacionaria de $\left(Z,S\right)$.
\end{Teo}

%___________________________________________________________
\subsection{Spread Out}
%___________________________________________________________


\begin{Def}
Una variable aleatoria $X_{1}$ es \textbf{spread out} si existe una $n\geq1$ y una  funci\'on $f\in\mathcal{B}^{+}$ tal que $\int_{\rea}f\left(x\right)dx>0$ con $X_{2},X_{3},\ldots,X_{n}$ copias i.i.d  de $X_{1}$, $$\prob\left(X_{1}+\cdots+X_{n}\in B\right)\geq\int_{B}f\left(x\right)dx$$ para $B\in\mathcal{B}$.
\end{Def}

%___________________________________________________________
\subsection{Wide Sense Regeneration}
%___________________________________________________________


\begin{Def}
Dado un proceso estoc\'astico $Z$ se le llama \textit{wide-sense regenerative} (\textbf{WSR}) con tiempos de regeneraci\'on $S$ si $\theta_{S_{n}}\left(Z,S\right)=\left(Z^{0},S^{0}\right)$ para $n\geq0$ en distribuci\'on y $\theta_{S_{n}}\left(Z,S\right)$ es independiente de $\left(S_{0},S_{1},\ldots,S_{n}\right)$ para $n\geq0$.
Se dice que el par $\left(Z,S\right)$ es WSR si lo anterior se cumple.
\end{Def}


\begin{Note}
\begin{itemize}
\item El proceso de trayectorias $\left(\theta_{s}Z\right)_{s\in\left[0,\infty\right)}$ es WSR con tiempos de regeneraci\'on $S$ pero no es regenerativo cl\'asico.

\item Si $Z$ es cualquier proceso estacionario y $S$ es un proceso de renovaci\'on que es independiente de $Z$, entonces $\left(Z,S\right)$ es WSR pero en general no es regenerativo cl\'asico

\end{itemize}

\end{Note}


\begin{Note}
Para cualquier proceso estoc\'astico $Z$, el proceso de trayectorias $\left(\theta_{s}Z\right)_{s\in\left[0,\infty\right)}$ es siempre un proceso de Markov.
\end{Note}


\begin{Teo}\label{Teo.4.1}
Supongase que el par $\left(Z,S\right)$ es WSR con $\esp\left[X_{1}\right]<\infty$. Entonces $\left(Z^{*},S^{*}\right)$ en el teorema (\ref{Teorema.2.1}) es una versi\'on estacionaria de 
$\left(Z,S\right)$.
\end{Teo}


%___________________________________________________________
\subsection{Existence of Regeneration Times}
%___________________________________________________________


\begin{Teo}\label{Tma.Existencia.Tiempos.Regeneracion}
Sea $Z$ un Proceso Estoc\'astico un lado shift-medible \textit{one-sided shift-measurable stochastic process}, (PEOSCTSM),
y $S_{0}$ y $S_{1}$ tiempos aleatorios tales que $0\leq S_{0}<S_{1}$ y
\begin{equation}
\theta_{S_{1}}Z=\theta_{S_{0}}Z\textrm{ en distribuci\'on}.
\end{equation}

Entonces el espacio de probabilidad subyacente $\left(\Omega,\mathcal{F},\prob\right)$ puede extenderse para soportar una sucesi\'on de tiempos aleatorios $S$ tales que

\begin{eqnarray}
\theta_{S_{n}}\left(Z,S\right)=\left(Z^{0},S^{0}\right),n\geq0,\textrm{ en distribuci\'on},\\
\left(Z,S_{0},S_{1}\right)\textrm{ depende de }\left(X_{2},X_{3},\ldots\right)\textrm{ solamente a traves de }\theta_{S_{1}}Z.
\end{eqnarray}
\end{Teo}

\begin{Coro}\label{Tma.Estacionariedad}
Bajo las condiciones del Teorema anterior (\ref{Tma.Existencia.Tiempos.Regeneracion}), el par $\left(Z,S\right)$ es regenerativo cl\'asico. Si adem\'as se tiene que $\esp\left[X_{1}\right]<\infty$ por el Teorema (\ref{Teo.3.1}) existe un par $\left(Z^{*},S^{*}\right)$ que es una vesi\'on estacionaria de $\left(Z,S\right)$.
\end{Coro}

\newpage

%________________________________________________________________________
\section{Procesos Regenerativos}
%________________________________________________________________________
%______________________________________________________________________
%\subsection*{Procesos Regenerativos}
%________________________________________________________________________



\begin{Note}
Si $\tilde{X}\left(t\right)$ con espacio de estados $\tilde{S}$ es regenerativo sobre $T_{n}$, entonces $X\left(t\right)=f\left(\tilde{X}\left(t\right)\right)$ tambi\'en es regenerativo sobre $T_{n}$, para cualquier funci\'on $f:\tilde{S}\rightarrow S$.
\end{Note}

\begin{Note}
Los procesos regenerativos son crudamente regenerativos, pero no al rev\'es.
\end{Note}
%\subsection*{Procesos Regenerativos: Sigman\cite{Sigman1}}
\begin{Def}[Definici\'on Cl\'asica]
Un proceso estoc\'astico $X=\left\{X\left(t\right):t\geq0\right\}$ es llamado regenerativo is existe una variable aleatoria $R_{1}>0$ tal que
\begin{itemize}
\item[i)] $\left\{X\left(t+R_{1}\right):t\geq0\right\}$ es independiente de $\left\{\left\{X\left(t\right):t<R_{1}\right\},\right\}$
\item[ii)] $\left\{X\left(t+R_{1}\right):t\geq0\right\}$ es estoc\'asticamente equivalente a $\left\{X\left(t\right):t>0\right\}$
\end{itemize}

Llamamos a $R_{1}$ tiempo de regeneraci\'on, y decimos que $X$ se regenera en este punto.
\end{Def}

$\left\{X\left(t+R_{1}\right)\right\}$ es regenerativo con tiempo de regeneraci\'on $R_{2}$, independiente de $R_{1}$ pero con la misma distribuci\'on que $R_{1}$. Procediendo de esta manera se obtiene una secuencia de variables aleatorias independientes e id\'enticamente distribuidas $\left\{R_{n}\right\}$ llamados longitudes de ciclo. Si definimos a $Z_{k}\equiv R_{1}+R_{2}+\cdots+R_{k}$, se tiene un proceso de renovaci\'on llamado proceso de renovaci\'on encajado para $X$.




\begin{Def}
Para $x$ fijo y para cada $t\geq0$, sea $I_{x}\left(t\right)=1$ si $X\left(t\right)\leq x$,  $I_{x}\left(t\right)=0$ en caso contrario, y def\'inanse los tiempos promedio
\begin{eqnarray*}
\overline{X}&=&lim_{t\rightarrow\infty}\frac{1}{t}\int_{0}^{\infty}X\left(u\right)du\\
\prob\left(X_{\infty}\leq x\right)&=&lim_{t\rightarrow\infty}\frac{1}{t}\int_{0}^{\infty}I_{x}\left(u\right)du,
\end{eqnarray*}
cuando estos l\'imites existan.
\end{Def}

Como consecuencia del teorema de Renovaci\'on-Recompensa, se tiene que el primer l\'imite  existe y es igual a la constante
\begin{eqnarray*}
\overline{X}&=&\frac{\esp\left[\int_{0}^{R_{1}}X\left(t\right)dt\right]}{\esp\left[R_{1}\right]},
\end{eqnarray*}
suponiendo que ambas esperanzas son finitas.

\begin{Note}
\begin{itemize}
\item[a)] Si el proceso regenerativo $X$ es positivo recurrente y tiene trayectorias muestrales no negativas, entonces la ecuaci\'on anterior es v\'alida.
\item[b)] Si $X$ es positivo recurrente regenerativo, podemos construir una \'unica versi\'on estacionaria de este proceso, $X_{e}=\left\{X_{e}\left(t\right)\right\}$, donde $X_{e}$ es un proceso estoc\'astico regenerativo y estrictamente estacionario, con distribuci\'on marginal distribuida como $X_{\infty}$
\end{itemize}
\end{Note}

Para $\left\{X\left(t\right):t\geq0\right\}$ Proceso Estoc\'astico a tiempo continuo con estado de espacios $S$, que es un espacio m\'etrico, con trayectorias continuas por la derecha y con l\'imites por la izquierda c.s. Sea $N\left(t\right)$ un proceso de renovaci\'on en $\rea_{+}$ definido en el mismo espacio de probabilidad que $X\left(t\right)$, con tiempos de renovaci\'on $T$ y tiempos de inter-renovaci\'on $\xi_{n}=T_{n}-T_{n-1}$, con misma distribuci\'on $F$ de media finita $\mu$.


\begin{Def}
Para el proceso $\left\{\left(N\left(t\right),X\left(t\right)\right):t\geq0\right\}$, sus trayectoria muestrales en el intervalo de tiempo $\left[T_{n-1},T_{n}\right)$ est\'an descritas por
\begin{eqnarray*}
\zeta_{n}=\left(\xi_{n},\left\{X\left(T_{n-1}+t\right):0\leq t<\xi_{n}\right\}\right)
\end{eqnarray*}
Este $\zeta_{n}$ es el $n$-\'esimo segmento del proceso. El proceso es regenerativo sobre los tiempos $T_{n}$ si sus segmentos $\zeta_{n}$ son independientes e id\'enticamennte distribuidos.
\end{Def}


\begin{Note}
Si $\tilde{X}\left(t\right)$ con espacio de estados $\tilde{S}$ es regenerativo sobre $T_{n}$, entonces $X\left(t\right)=f\left(\tilde{X}\left(t\right)\right)$ tambi\'en es regenerativo sobre $T_{n}$, para cualquier funci\'on $f:\tilde{S}\rightarrow S$.
\end{Note}

\begin{Note}
Los procesos regenerativos son crudamente regenerativos, pero no al rev\'es.
\end{Note}

\begin{Def}[Definici\'on Cl\'asica]
Un proceso estoc\'astico $X=\left\{X\left(t\right):t\geq0\right\}$ es llamado regenerativo is existe una variable aleatoria $R_{1}>0$ tal que
\begin{itemize}
\item[i)] $\left\{X\left(t+R_{1}\right):t\geq0\right\}$ es independiente de $\left\{\left\{X\left(t\right):t<R_{1}\right\},\right\}$
\item[ii)] $\left\{X\left(t+R_{1}\right):t\geq0\right\}$ es estoc\'asticamente equivalente a $\left\{X\left(t\right):t>0\right\}$
\end{itemize}

Llamamos a $R_{1}$ tiempo de regeneraci\'on, y decimos que $X$ se regenera en este punto.
\end{Def}

$\left\{X\left(t+R_{1}\right)\right\}$ es regenerativo con tiempo de regeneraci\'on $R_{2}$, independiente de $R_{1}$ pero con la misma distribuci\'on que $R_{1}$. Procediendo de esta manera se obtiene una secuencia de variables aleatorias independientes e id\'enticamente distribuidas $\left\{R_{n}\right\}$ llamados longitudes de ciclo. Si definimos a $Z_{k}\equiv R_{1}+R_{2}+\cdots+R_{k}$, se tiene un proceso de renovaci\'on llamado proceso de renovaci\'on encajado para $X$.

\begin{Note}
Un proceso regenerativo con media de la longitud de ciclo finita es llamado positivo recurrente.
\end{Note}


%_________________________________________________________________________
%
%\section{Appendix F: Output Process and Regenerative Processes}
%_________________________________________________________________________
%
En Sigman, Thorison y Wolff \cite{Sigman2} prueban que para la existencia de un una sucesi\'on infinita no decreciente de tiempos de regeneraci\'on $\tau_{1}\leq\tau_{2}\leq\cdots$ en los cuales el proceso se regenera, basta un tiempo de regeneraci\'on $R_{1}$, donde $R_{j}=\tau_{j}-\tau_{j-1}$. Para tal efecto se requiere la existencia de un espacio de probabilidad $\left(\Omega,\mathcal{F},\prob\right)$, y proceso estoc\'astico $\textit{X}=\left\{X\left(t\right):t\geq0\right\}$ con espacio de estados $\left(S,\mathcal{R}\right)$, con $\mathcal{R}$ $\sigma$-\'algebra.

\begin{Prop}
Si existe una variable aleatoria no negativa $R_{1}$ tal que $\theta_{R1}X=_{D}X$, entonces $\left(\Omega,\mathcal{F},\prob\right)$ puede extenderse para soportar una sucesi\'on estacionaria de variables aleatorias $R=\left\{R_{k}:k\geq1\right\}$, tal que para $k\geq1$,
\begin{eqnarray*}
\theta_{k}\left(X,R\right)=_{D}\left(X,R\right).
\end{eqnarray*}

Adem\'as, para $k\geq1$, $\theta_{k}R$ es condicionalmente independiente de $\left(X,R_{1},\ldots,R_{k}\right)$, dado $\theta_{\tau k}X$.

\end{Prop}


\begin{itemize}
\item Doob en 1953 demostr\'o que el estado estacionario de un proceso de partida en un sistema de espera $M/G/\infty$, es Poisson con la misma tasa que el proceso de arribos.

\item Burke en 1968, fue el primero en demostrar que el estado estacionario de un proceso de salida de una cola $M/M/s$ es un proceso Poisson.

\item Disney en 1973 obtuvo el siguiente resultado:

\begin{Teo}
Para el sistema de espera $M/G/1/L$ con disciplina FIFO, el proceso $\textbf{I}$ es un proceso de renovaci\'on si y s\'olo si el proceso denominado longitud de la cola es estacionario y se cumple cualquiera de los siguientes casos:

\begin{itemize}
\item[a)] Los tiempos de servicio son identicamente cero;
\item[b)] $L=0$, para cualquier proceso de servicio $S$;
\item[c)] $L=1$ y $G=D$;
\item[d)] $L=\infty$ y $G=M$.
\end{itemize}
En estos casos, respectivamente, las distribuciones de interpartida $P\left\{T_{n+1}-T_{n}\leq t\right\}$ son


\begin{itemize}
\item[a)] $1-e^{-\lambda t}$, $t\geq0$;
\item[b)] $1-e^{-\lambda t}*F\left(t\right)$, $t\geq0$;
\item[c)] $1-e^{-\lambda t}*\indora_{d}\left(t\right)$, $t\geq0$;
\item[d)] $1-e^{-\lambda t}*F\left(t\right)$, $t\geq0$.
\end{itemize}
\end{Teo}


\item Finch (1959) mostr\'o que para los sistemas $M/G/1/L$, con $1\leq L\leq \infty$ con distribuciones de servicio dos veces diferenciable, solamente el sistema $M/M/1/\infty$ tiene proceso de salida de renovaci\'on estacionario.

\item King (1971) demostro que un sistema de colas estacionario $M/G/1/1$ tiene sus tiempos de interpartida sucesivas $D_{n}$ y $D_{n+1}$ son independientes, si y s\'olo si, $G=D$, en cuyo caso le proceso de salida es de renovaci\'on.

\item Disney (1973) demostr\'o que el \'unico sistema estacionario $M/G/1/L$, que tiene proceso de salida de renovaci\'on  son los sistemas $M/M/1$ y $M/D/1/1$.



\item El siguiente resultado es de Disney y Koning (1985)
\begin{Teo}
En un sistema de espera $M/G/s$, el estado estacionario del proceso de salida es un proceso Poisson para cualquier distribuci\'on de los tiempos de servicio si el sistema tiene cualquiera de las siguientes cuatro propiedades.

\begin{itemize}
\item[a)] $s=\infty$
\item[b)] La disciplina de servicio es de procesador compartido.
\item[c)] La disciplina de servicio es LCFS y preemptive resume, esto se cumple para $L<\infty$
\item[d)] $G=M$.
\end{itemize}

\end{Teo}

\item El siguiente resultado es de Alamatsaz (1983)

\begin{Teo}
En cualquier sistema de colas $GI/G/1/L$ con $1\leq L<\infty$ y distribuci\'on de interarribos $A$ y distribuci\'on de los tiempos de servicio $B$, tal que $A\left(0\right)=0$, $A\left(t\right)\left(1-B\left(t\right)\right)>0$ para alguna $t>0$ y $B\left(t\right)$ para toda $t>0$, es imposible que el proceso de salida estacionario sea de renovaci\'on.
\end{Teo}

\end{itemize}

%________________________________________________________________________
%\subsection*{Procesos Regenerativos}
%________________________________________________________________________



\begin{Note}
Si $\tilde{X}\left(t\right)$ con espacio de estados $\tilde{S}$ es regenerativo sobre $T_{n}$, entonces $X\left(t\right)=f\left(\tilde{X}\left(t\right)\right)$ tambi\'en es regenerativo sobre $T_{n}$, para cualquier funci\'on $f:\tilde{S}\rightarrow S$.
\end{Note}

\begin{Note}
Los procesos regenerativos son crudamente regenerativos, pero no al rev\'es.
\end{Note}
%\subsection*{Procesos Regenerativos: Sigman\cite{Sigman1}}
\begin{Def}[Definici\'on Cl\'asica]
Un proceso estoc\'astico $X=\left\{X\left(t\right):t\geq0\right\}$ es llamado regenerativo is existe una variable aleatoria $R_{1}>0$ tal que
\begin{itemize}
\item[i)] $\left\{X\left(t+R_{1}\right):t\geq0\right\}$ es independiente de $\left\{\left\{X\left(t\right):t<R_{1}\right\},\right\}$
\item[ii)] $\left\{X\left(t+R_{1}\right):t\geq0\right\}$ es estoc\'asticamente equivalente a $\left\{X\left(t\right):t>0\right\}$
\end{itemize}

Llamamos a $R_{1}$ tiempo de regeneraci\'on, y decimos que $X$ se regenera en este punto.
\end{Def}

$\left\{X\left(t+R_{1}\right)\right\}$ es regenerativo con tiempo de regeneraci\'on $R_{2}$, independiente de $R_{1}$ pero con la misma distribuci\'on que $R_{1}$. Procediendo de esta manera se obtiene una secuencia de variables aleatorias independientes e id\'enticamente distribuidas $\left\{R_{n}\right\}$ llamados longitudes de ciclo. Si definimos a $Z_{k}\equiv R_{1}+R_{2}+\cdots+R_{k}$, se tiene un proceso de renovaci\'on llamado proceso de renovaci\'on encajado para $X$.




\begin{Def}
Para $x$ fijo y para cada $t\geq0$, sea $I_{x}\left(t\right)=1$ si $X\left(t\right)\leq x$,  $I_{x}\left(t\right)=0$ en caso contrario, y def\'inanse los tiempos promedio
\begin{eqnarray*}
\overline{X}&=&lim_{t\rightarrow\infty}\frac{1}{t}\int_{0}^{\infty}X\left(u\right)du\\
\prob\left(X_{\infty}\leq x\right)&=&lim_{t\rightarrow\infty}\frac{1}{t}\int_{0}^{\infty}I_{x}\left(u\right)du,
\end{eqnarray*}
cuando estos l\'imites existan.
\end{Def}

Como consecuencia del teorema de Renovaci\'on-Recompensa, se tiene que el primer l\'imite  existe y es igual a la constante
\begin{eqnarray*}
\overline{X}&=&\frac{\esp\left[\int_{0}^{R_{1}}X\left(t\right)dt\right]}{\esp\left[R_{1}\right]},
\end{eqnarray*}
suponiendo que ambas esperanzas son finitas.

\begin{Note}
\begin{itemize}
\item[a)] Si el proceso regenerativo $X$ es positivo recurrente y tiene trayectorias muestrales no negativas, entonces la ecuaci\'on anterior es v\'alida.
\item[b)] Si $X$ es positivo recurrente regenerativo, podemos construir una \'unica versi\'on estacionaria de este proceso, $X_{e}=\left\{X_{e}\left(t\right)\right\}$, donde $X_{e}$ es un proceso estoc\'astico regenerativo y estrictamente estacionario, con distribuci\'on marginal distribuida como $X_{\infty}$
\end{itemize}
\end{Note}

Para $\left\{X\left(t\right):t\geq0\right\}$ Proceso Estoc\'astico a tiempo continuo con estado de espacios $S$, que es un espacio m\'etrico, con trayectorias continuas por la derecha y con l\'imites por la izquierda c.s. Sea $N\left(t\right)$ un proceso de renovaci\'on en $\rea_{+}$ definido en el mismo espacio de probabilidad que $X\left(t\right)$, con tiempos de renovaci\'on $T$ y tiempos de inter-renovaci\'on $\xi_{n}=T_{n}-T_{n-1}$, con misma distribuci\'on $F$ de media finita $\mu$.


\begin{Def}
Para el proceso $\left\{\left(N\left(t\right),X\left(t\right)\right):t\geq0\right\}$, sus trayectoria muestrales en el intervalo de tiempo $\left[T_{n-1},T_{n}\right)$ est\'an descritas por
\begin{eqnarray*}
\zeta_{n}=\left(\xi_{n},\left\{X\left(T_{n-1}+t\right):0\leq t<\xi_{n}\right\}\right)
\end{eqnarray*}
Este $\zeta_{n}$ es el $n$-\'esimo segmento del proceso. El proceso es regenerativo sobre los tiempos $T_{n}$ si sus segmentos $\zeta_{n}$ son independientes e id\'enticamennte distribuidos.
\end{Def}


\begin{Note}
Si $\tilde{X}\left(t\right)$ con espacio de estados $\tilde{S}$ es regenerativo sobre $T_{n}$, entonces $X\left(t\right)=f\left(\tilde{X}\left(t\right)\right)$ tambi\'en es regenerativo sobre $T_{n}$, para cualquier funci\'on $f:\tilde{S}\rightarrow S$.
\end{Note}

\begin{Note}
Los procesos regenerativos son crudamente regenerativos, pero no al rev\'es.
\end{Note}

\begin{Def}[Definici\'on Cl\'asica]
Un proceso estoc\'astico $X=\left\{X\left(t\right):t\geq0\right\}$ es llamado regenerativo is existe una variable aleatoria $R_{1}>0$ tal que
\begin{itemize}
\item[i)] $\left\{X\left(t+R_{1}\right):t\geq0\right\}$ es independiente de $\left\{\left\{X\left(t\right):t<R_{1}\right\},\right\}$
\item[ii)] $\left\{X\left(t+R_{1}\right):t\geq0\right\}$ es estoc\'asticamente equivalente a $\left\{X\left(t\right):t>0\right\}$
\end{itemize}

Llamamos a $R_{1}$ tiempo de regeneraci\'on, y decimos que $X$ se regenera en este punto.
\end{Def}

$\left\{X\left(t+R_{1}\right)\right\}$ es regenerativo con tiempo de regeneraci\'on $R_{2}$, independiente de $R_{1}$ pero con la misma distribuci\'on que $R_{1}$. Procediendo de esta manera se obtiene una secuencia de variables aleatorias independientes e id\'enticamente distribuidas $\left\{R_{n}\right\}$ llamados longitudes de ciclo. Si definimos a $Z_{k}\equiv R_{1}+R_{2}+\cdots+R_{k}$, se tiene un proceso de renovaci\'on llamado proceso de renovaci\'on encajado para $X$.

\begin{Note}
Un proceso regenerativo con media de la longitud de ciclo finita es llamado positivo recurrente.
\end{Note}


\begin{Def}
Para $x$ fijo y para cada $t\geq0$, sea $I_{x}\left(t\right)=1$ si $X\left(t\right)\leq x$,  $I_{x}\left(t\right)=0$ en caso contrario, y def\'inanse los tiempos promedio
\begin{eqnarray*}
\overline{X}&=&lim_{t\rightarrow\infty}\frac{1}{t}\int_{0}^{\infty}X\left(u\right)du\\
\prob\left(X_{\infty}\leq x\right)&=&lim_{t\rightarrow\infty}\frac{1}{t}\int_{0}^{\infty}I_{x}\left(u\right)du,
\end{eqnarray*}
cuando estos l\'imites existan.
\end{Def}

Como consecuencia del teorema de Renovaci\'on-Recompensa, se tiene que el primer l\'imite  existe y es igual a la constante
\begin{eqnarray*}
\overline{X}&=&\frac{\esp\left[\int_{0}^{R_{1}}X\left(t\right)dt\right]}{\esp\left[R_{1}\right]},
\end{eqnarray*}
suponiendo que ambas esperanzas son finitas.

\begin{Note}
\begin{itemize}
\item[a)] Si el proceso regenerativo $X$ es positivo recurrente y tiene trayectorias muestrales no negativas, entonces la ecuaci\'on anterior es v\'alida.
\item[b)] Si $X$ es positivo recurrente regenerativo, podemos construir una \'unica versi\'on estacionaria de este proceso, $X_{e}=\left\{X_{e}\left(t\right)\right\}$, donde $X_{e}$ es un proceso estoc\'astico regenerativo y estrictamente estacionario, con distribuci\'on marginal distribuida como $X_{\infty}$
\end{itemize}
\end{Note}

%________________________________________________________________________
%\subsection*{Procesos Regenerativos}
%________________________________________________________________________



\begin{Note}
Si $\tilde{X}\left(t\right)$ con espacio de estados $\tilde{S}$ es regenerativo sobre $T_{n}$, entonces $X\left(t\right)=f\left(\tilde{X}\left(t\right)\right)$ tambi\'en es regenerativo sobre $T_{n}$, para cualquier funci\'on $f:\tilde{S}\rightarrow S$.
\end{Note}

\begin{Note}
Los procesos regenerativos son crudamente regenerativos, pero no al rev\'es.
\end{Note}
%\subsection*{Procesos Regenerativos: Sigman\cite{Sigman1}}
\begin{Def}[Definici\'on Cl\'asica]
Un proceso estoc\'astico $X=\left\{X\left(t\right):t\geq0\right\}$ es llamado regenerativo is existe una variable aleatoria $R_{1}>0$ tal que
\begin{itemize}
\item[i)] $\left\{X\left(t+R_{1}\right):t\geq0\right\}$ es independiente de $\left\{\left\{X\left(t\right):t<R_{1}\right\},\right\}$
\item[ii)] $\left\{X\left(t+R_{1}\right):t\geq0\right\}$ es estoc\'asticamente equivalente a $\left\{X\left(t\right):t>0\right\}$
\end{itemize}

Llamamos a $R_{1}$ tiempo de regeneraci\'on, y decimos que $X$ se regenera en este punto.
\end{Def}

$\left\{X\left(t+R_{1}\right)\right\}$ es regenerativo con tiempo de regeneraci\'on $R_{2}$, independiente de $R_{1}$ pero con la misma distribuci\'on que $R_{1}$. Procediendo de esta manera se obtiene una secuencia de variables aleatorias independientes e id\'enticamente distribuidas $\left\{R_{n}\right\}$ llamados longitudes de ciclo. Si definimos a $Z_{k}\equiv R_{1}+R_{2}+\cdots+R_{k}$, se tiene un proceso de renovaci\'on llamado proceso de renovaci\'on encajado para $X$.




\begin{Def}
Para $x$ fijo y para cada $t\geq0$, sea $I_{x}\left(t\right)=1$ si $X\left(t\right)\leq x$,  $I_{x}\left(t\right)=0$ en caso contrario, y def\'inanse los tiempos promedio
\begin{eqnarray*}
\overline{X}&=&lim_{t\rightarrow\infty}\frac{1}{t}\int_{0}^{\infty}X\left(u\right)du\\
\prob\left(X_{\infty}\leq x\right)&=&lim_{t\rightarrow\infty}\frac{1}{t}\int_{0}^{\infty}I_{x}\left(u\right)du,
\end{eqnarray*}
cuando estos l\'imites existan.
\end{Def}

Como consecuencia del teorema de Renovaci\'on-Recompensa, se tiene que el primer l\'imite  existe y es igual a la constante
\begin{eqnarray*}
\overline{X}&=&\frac{\esp\left[\int_{0}^{R_{1}}X\left(t\right)dt\right]}{\esp\left[R_{1}\right]},
\end{eqnarray*}
suponiendo que ambas esperanzas son finitas.

\begin{Note}
\begin{itemize}
\item[a)] Si el proceso regenerativo $X$ es positivo recurrente y tiene trayectorias muestrales no negativas, entonces la ecuaci\'on anterior es v\'alida.
\item[b)] Si $X$ es positivo recurrente regenerativo, podemos construir una \'unica versi\'on estacionaria de este proceso, $X_{e}=\left\{X_{e}\left(t\right)\right\}$, donde $X_{e}$ es un proceso estoc\'astico regenerativo y estrictamente estacionario, con distribuci\'on marginal distribuida como $X_{\infty}$
\end{itemize}
\end{Note}

%________________________________________________________________________
%\subsection*{Procesos Regenerativos}
%________________________________________________________________________

Para $\left\{X\left(t\right):t\geq0\right\}$ Proceso Estoc\'astico a tiempo continuo con estado de espacios $S$, que es un espacio m\'etrico, con trayectorias continuas por la derecha y con l\'imites por la izquierda c.s. Sea $N\left(t\right)$ un proceso de renovaci\'on en $\rea_{+}$ definido en el mismo espacio de probabilidad que $X\left(t\right)$, con tiempos de renovaci\'on $T$ y tiempos de inter-renovaci\'on $\xi_{n}=T_{n}-T_{n-1}$, con misma distribuci\'on $F$ de media finita $\mu$.



\begin{Def}
Para el proceso $\left\{\left(N\left(t\right),X\left(t\right)\right):t\geq0\right\}$, sus trayectoria muestrales en el intervalo de tiempo $\left[T_{n-1},T_{n}\right)$ est\'an descritas por
\begin{eqnarray*}
\zeta_{n}=\left(\xi_{n},\left\{X\left(T_{n-1}+t\right):0\leq t<\xi_{n}\right\}\right)
\end{eqnarray*}
Este $\zeta_{n}$ es el $n$-\'esimo segmento del proceso. El proceso es regenerativo sobre los tiempos $T_{n}$ si sus segmentos $\zeta_{n}$ son independientes e id\'enticamennte distribuidos.
\end{Def}


\begin{Note}
Si $\tilde{X}\left(t\right)$ con espacio de estados $\tilde{S}$ es regenerativo sobre $T_{n}$, entonces $X\left(t\right)=f\left(\tilde{X}\left(t\right)\right)$ tambi\'en es regenerativo sobre $T_{n}$, para cualquier funci\'on $f:\tilde{S}\rightarrow S$.
\end{Note}

\begin{Note}
Los procesos regenerativos son crudamente regenerativos, pero no al rev\'es.
\end{Note}

\begin{Def}[Definici\'on Cl\'asica]
Un proceso estoc\'astico $X=\left\{X\left(t\right):t\geq0\right\}$ es llamado regenerativo is existe una variable aleatoria $R_{1}>0$ tal que
\begin{itemize}
\item[i)] $\left\{X\left(t+R_{1}\right):t\geq0\right\}$ es independiente de $\left\{\left\{X\left(t\right):t<R_{1}\right\},\right\}$
\item[ii)] $\left\{X\left(t+R_{1}\right):t\geq0\right\}$ es estoc\'asticamente equivalente a $\left\{X\left(t\right):t>0\right\}$
\end{itemize}

Llamamos a $R_{1}$ tiempo de regeneraci\'on, y decimos que $X$ se regenera en este punto.
\end{Def}

$\left\{X\left(t+R_{1}\right)\right\}$ es regenerativo con tiempo de regeneraci\'on $R_{2}$, independiente de $R_{1}$ pero con la misma distribuci\'on que $R_{1}$. Procediendo de esta manera se obtiene una secuencia de variables aleatorias independientes e id\'enticamente distribuidas $\left\{R_{n}\right\}$ llamados longitudes de ciclo. Si definimos a $Z_{k}\equiv R_{1}+R_{2}+\cdots+R_{k}$, se tiene un proceso de renovaci\'on llamado proceso de renovaci\'on encajado para $X$.

\begin{Note}
Un proceso regenerativo con media de la longitud de ciclo finita es llamado positivo recurrente.
\end{Note}


\begin{Def}
Para $x$ fijo y para cada $t\geq0$, sea $I_{x}\left(t\right)=1$ si $X\left(t\right)\leq x$,  $I_{x}\left(t\right)=0$ en caso contrario, y def\'inanse los tiempos promedio
\begin{eqnarray*}
\overline{X}&=&lim_{t\rightarrow\infty}\frac{1}{t}\int_{0}^{\infty}X\left(u\right)du\\
\prob\left(X_{\infty}\leq x\right)&=&lim_{t\rightarrow\infty}\frac{1}{t}\int_{0}^{\infty}I_{x}\left(u\right)du,
\end{eqnarray*}
cuando estos l\'imites existan.
\end{Def}

Como consecuencia del teorema de Renovaci\'on-Recompensa, se tiene que el primer l\'imite  existe y es igual a la constante
\begin{eqnarray*}
\overline{X}&=&\frac{\esp\left[\int_{0}^{R_{1}}X\left(t\right)dt\right]}{\esp\left[R_{1}\right]},
\end{eqnarray*}
suponiendo que ambas esperanzas son finitas.

\begin{Note}
\begin{itemize}
\item[a)] Si el proceso regenerativo $X$ es positivo recurrente y tiene trayectorias muestrales no negativas, entonces la ecuaci\'on anterior es v\'alida.
\item[b)] Si $X$ es positivo recurrente regenerativo, podemos construir una \'unica versi\'on estacionaria de este proceso, $X_{e}=\left\{X_{e}\left(t\right)\right\}$, donde $X_{e}$ es un proceso estoc\'astico regenerativo y estrictamente estacionario, con distribuci\'on marginal distribuida como $X_{\infty}$
\end{itemize}
\end{Note}

%________________________________________________________________________
\section{Procesos Regenerativos Sigman, Thorisson y Wolff \cite{Sigman1}}
%________________________________________________________________________


\begin{Def}[Definici\'on Cl\'asica]
Un proceso estoc\'astico $X=\left\{X\left(t\right):t\geq0\right\}$ es llamado regenerativo is existe una variable aleatoria $R_{1}>0$ tal que
\begin{itemize}
\item[i)] $\left\{X\left(t+R_{1}\right):t\geq0\right\}$ es independiente de $\left\{\left\{X\left(t\right):t<R_{1}\right\},\right\}$
\item[ii)] $\left\{X\left(t+R_{1}\right):t\geq0\right\}$ es estoc\'asticamente equivalente a $\left\{X\left(t\right):t>0\right\}$
\end{itemize}

Llamamos a $R_{1}$ tiempo de regeneraci\'on, y decimos que $X$ se regenera en este punto.
\end{Def}

$\left\{X\left(t+R_{1}\right)\right\}$ es regenerativo con tiempo de regeneraci\'on $R_{2}$, independiente de $R_{1}$ pero con la misma distribuci\'on que $R_{1}$. Procediendo de esta manera se obtiene una secuencia de variables aleatorias independientes e id\'enticamente distribuidas $\left\{R_{n}\right\}$ llamados longitudes de ciclo. Si definimos a $Z_{k}\equiv R_{1}+R_{2}+\cdots+R_{k}$, se tiene un proceso de renovaci\'on llamado proceso de renovaci\'on encajado para $X$.


\begin{Note}
La existencia de un primer tiempo de regeneraci\'on, $R_{1}$, implica la existencia de una sucesi\'on completa de estos tiempos $R_{1},R_{2}\ldots,$ que satisfacen la propiedad deseada \cite{Sigman2}.
\end{Note}


\begin{Note} Para la cola $GI/GI/1$ los usuarios arriban con tiempos $t_{n}$ y son atendidos con tiempos de servicio $S_{n}$, los tiempos de arribo forman un proceso de renovaci\'on  con tiempos entre arribos independientes e identicamente distribuidos (\texttt{i.i.d.})$T_{n}=t_{n}-t_{n-1}$, adem\'as los tiempos de servicio son \texttt{i.i.d.} e independientes de los procesos de arribo. Por \textit{estable} se entiende que $\esp S_{n}<\esp T_{n}<\infty$.
\end{Note}
 


\begin{Def}
Para $x$ fijo y para cada $t\geq0$, sea $I_{x}\left(t\right)=1$ si $X\left(t\right)\leq x$,  $I_{x}\left(t\right)=0$ en caso contrario, y def\'inanse los tiempos promedio
\begin{eqnarray*}
\overline{X}&=&lim_{t\rightarrow\infty}\frac{1}{t}\int_{0}^{\infty}X\left(u\right)du\\
\prob\left(X_{\infty}\leq x\right)&=&lim_{t\rightarrow\infty}\frac{1}{t}\int_{0}^{\infty}I_{x}\left(u\right)du,
\end{eqnarray*}
cuando estos l\'imites existan.
\end{Def}

Como consecuencia del teorema de Renovaci\'on-Recompensa, se tiene que el primer l\'imite  existe y es igual a la constante
\begin{eqnarray*}
\overline{X}&=&\frac{\esp\left[\int_{0}^{R_{1}}X\left(t\right)dt\right]}{\esp\left[R_{1}\right]},
\end{eqnarray*}
suponiendo que ambas esperanzas son finitas.
 
\begin{Note}
Funciones de procesos regenerativos son regenerativas, es decir, si $X\left(t\right)$ es regenerativo y se define el proceso $Y\left(t\right)$ por $Y\left(t\right)=f\left(X\left(t\right)\right)$ para alguna funci\'on Borel medible $f\left(\cdot\right)$. Adem\'as $Y$ es regenerativo con los mismos tiempos de renovaci\'on que $X$. 

En general, los tiempos de renovaci\'on, $Z_{k}$ de un proceso regenerativo no requieren ser tiempos de paro con respecto a la evoluci\'on de $X\left(t\right)$.
\end{Note} 

\begin{Note}
Una funci\'on de un proceso de Markov, usualmente no ser\'a un proceso de Markov, sin embargo ser\'a regenerativo si el proceso de Markov lo es.
\end{Note}

 
\begin{Note}
Un proceso regenerativo con media de la longitud de ciclo finita es llamado positivo recurrente.
\end{Note}


\begin{Note}
\begin{itemize}
\item[a)] Si el proceso regenerativo $X$ es positivo recurrente y tiene trayectorias muestrales no negativas, entonces la ecuaci\'on anterior es v\'alida.
\item[b)] Si $X$ es positivo recurrente regenerativo, podemos construir una \'unica versi\'on estacionaria de este proceso, $X_{e}=\left\{X_{e}\left(t\right)\right\}$, donde $X_{e}$ es un proceso estoc\'astico regenerativo y estrictamente estacionario, con distribuci\'on marginal distribuida como $X_{\infty}$
\end{itemize}
\end{Note}


%________________________________________________________________________
%\subsection*{Procesos Regenerativos Sigman, Thorisson y Wolff \cite{Sigman1}}
%________________________________________________________________________


\begin{Def}[Definici\'on Cl\'asica]
Un proceso estoc\'astico $X=\left\{X\left(t\right):t\geq0\right\}$ es llamado regenerativo is existe una variable aleatoria $R_{1}>0$ tal que
\begin{itemize}
\item[i)] $\left\{X\left(t+R_{1}\right):t\geq0\right\}$ es independiente de $\left\{\left\{X\left(t\right):t<R_{1}\right\},\right\}$
\item[ii)] $\left\{X\left(t+R_{1}\right):t\geq0\right\}$ es estoc\'asticamente equivalente a $\left\{X\left(t\right):t>0\right\}$
\end{itemize}

Llamamos a $R_{1}$ tiempo de regeneraci\'on, y decimos que $X$ se regenera en este punto.
\end{Def}

$\left\{X\left(t+R_{1}\right)\right\}$ es regenerativo con tiempo de regeneraci\'on $R_{2}$, independiente de $R_{1}$ pero con la misma distribuci\'on que $R_{1}$. Procediendo de esta manera se obtiene una secuencia de variables aleatorias independientes e id\'enticamente distribuidas $\left\{R_{n}\right\}$ llamados longitudes de ciclo. Si definimos a $Z_{k}\equiv R_{1}+R_{2}+\cdots+R_{k}$, se tiene un proceso de renovaci\'on llamado proceso de renovaci\'on encajado para $X$.


\begin{Note}
La existencia de un primer tiempo de regeneraci\'on, $R_{1}$, implica la existencia de una sucesi\'on completa de estos tiempos $R_{1},R_{2}\ldots,$ que satisfacen la propiedad deseada \cite{Sigman2}.
\end{Note}


\begin{Note} Para la cola $GI/GI/1$ los usuarios arriban con tiempos $t_{n}$ y son atendidos con tiempos de servicio $S_{n}$, los tiempos de arribo forman un proceso de renovaci\'on  con tiempos entre arribos independientes e identicamente distribuidos (\texttt{i.i.d.})$T_{n}=t_{n}-t_{n-1}$, adem\'as los tiempos de servicio son \texttt{i.i.d.} e independientes de los procesos de arribo. Por \textit{estable} se entiende que $\esp S_{n}<\esp T_{n}<\infty$.
\end{Note}
 


\begin{Def}
Para $x$ fijo y para cada $t\geq0$, sea $I_{x}\left(t\right)=1$ si $X\left(t\right)\leq x$,  $I_{x}\left(t\right)=0$ en caso contrario, y def\'inanse los tiempos promedio
\begin{eqnarray*}
\overline{X}&=&lim_{t\rightarrow\infty}\frac{1}{t}\int_{0}^{\infty}X\left(u\right)du\\
\prob\left(X_{\infty}\leq x\right)&=&lim_{t\rightarrow\infty}\frac{1}{t}\int_{0}^{\infty}I_{x}\left(u\right)du,
\end{eqnarray*}
cuando estos l\'imites existan.
\end{Def}

Como consecuencia del teorema de Renovaci\'on-Recompensa, se tiene que el primer l\'imite  existe y es igual a la constante
\begin{eqnarray*}
\overline{X}&=&\frac{\esp\left[\int_{0}^{R_{1}}X\left(t\right)dt\right]}{\esp\left[R_{1}\right]},
\end{eqnarray*}
suponiendo que ambas esperanzas son finitas.
 
\begin{Note}
Funciones de procesos regenerativos son regenerativas, es decir, si $X\left(t\right)$ es regenerativo y se define el proceso $Y\left(t\right)$ por $Y\left(t\right)=f\left(X\left(t\right)\right)$ para alguna funci\'on Borel medible $f\left(\cdot\right)$. Adem\'as $Y$ es regenerativo con los mismos tiempos de renovaci\'on que $X$. 

En general, los tiempos de renovaci\'on, $Z_{k}$ de un proceso regenerativo no requieren ser tiempos de paro con respecto a la evoluci\'on de $X\left(t\right)$.
\end{Note} 

\begin{Note}
Una funci\'on de un proceso de Markov, usualmente no ser\'a un proceso de Markov, sin embargo ser\'a regenerativo si el proceso de Markov lo es.
\end{Note}

 
\begin{Note}
Un proceso regenerativo con media de la longitud de ciclo finita es llamado positivo recurrente.
\end{Note}


\begin{Note}
\begin{itemize}
\item[a)] Si el proceso regenerativo $X$ es positivo recurrente y tiene trayectorias muestrales no negativas, entonces la ecuaci\'on anterior es v\'alida.
\item[b)] Si $X$ es positivo recurrente regenerativo, podemos construir una \'unica versi\'on estacionaria de este proceso, $X_{e}=\left\{X_{e}\left(t\right)\right\}$, donde $X_{e}$ es un proceso estoc\'astico regenerativo y estrictamente estacionario, con distribuci\'on marginal distribuida como $X_{\infty}$
\end{itemize}
\end{Note}


%________________________________________________________________________
%\subsection*{Procesos Regenerativos Sigman, Thorisson y Wolff \cite{Sigman1}}
%________________________________________________________________________


\begin{Def}[Definici\'on Cl\'asica]
Un proceso estoc\'astico $X=\left\{X\left(t\right):t\geq0\right\}$ es llamado regenerativo is existe una variable aleatoria $R_{1}>0$ tal que
\begin{itemize}
\item[i)] $\left\{X\left(t+R_{1}\right):t\geq0\right\}$ es independiente de $\left\{\left\{X\left(t\right):t<R_{1}\right\},\right\}$
\item[ii)] $\left\{X\left(t+R_{1}\right):t\geq0\right\}$ es estoc\'asticamente equivalente a $\left\{X\left(t\right):t>0\right\}$
\end{itemize}

Llamamos a $R_{1}$ tiempo de regeneraci\'on, y decimos que $X$ se regenera en este punto.
\end{Def}

$\left\{X\left(t+R_{1}\right)\right\}$ es regenerativo con tiempo de regeneraci\'on $R_{2}$, independiente de $R_{1}$ pero con la misma distribuci\'on que $R_{1}$. Procediendo de esta manera se obtiene una secuencia de variables aleatorias independientes e id\'enticamente distribuidas $\left\{R_{n}\right\}$ llamados longitudes de ciclo. Si definimos a $Z_{k}\equiv R_{1}+R_{2}+\cdots+R_{k}$, se tiene un proceso de renovaci\'on llamado proceso de renovaci\'on encajado para $X$.


\begin{Note}
La existencia de un primer tiempo de regeneraci\'on, $R_{1}$, implica la existencia de una sucesi\'on completa de estos tiempos $R_{1},R_{2}\ldots,$ que satisfacen la propiedad deseada \cite{Sigman2}.
\end{Note}


\begin{Note} Para la cola $GI/GI/1$ los usuarios arriban con tiempos $t_{n}$ y son atendidos con tiempos de servicio $S_{n}$, los tiempos de arribo forman un proceso de renovaci\'on  con tiempos entre arribos independientes e identicamente distribuidos (\texttt{i.i.d.})$T_{n}=t_{n}-t_{n-1}$, adem\'as los tiempos de servicio son \texttt{i.i.d.} e independientes de los procesos de arribo. Por \textit{estable} se entiende que $\esp S_{n}<\esp T_{n}<\infty$.
\end{Note}
 


\begin{Def}
Para $x$ fijo y para cada $t\geq0$, sea $I_{x}\left(t\right)=1$ si $X\left(t\right)\leq x$,  $I_{x}\left(t\right)=0$ en caso contrario, y def\'inanse los tiempos promedio
\begin{eqnarray*}
\overline{X}&=&lim_{t\rightarrow\infty}\frac{1}{t}\int_{0}^{\infty}X\left(u\right)du\\
\prob\left(X_{\infty}\leq x\right)&=&lim_{t\rightarrow\infty}\frac{1}{t}\int_{0}^{\infty}I_{x}\left(u\right)du,
\end{eqnarray*}
cuando estos l\'imites existan.
\end{Def}

Como consecuencia del teorema de Renovaci\'on-Recompensa, se tiene que el primer l\'imite  existe y es igual a la constante
\begin{eqnarray*}
\overline{X}&=&\frac{\esp\left[\int_{0}^{R_{1}}X\left(t\right)dt\right]}{\esp\left[R_{1}\right]},
\end{eqnarray*}
suponiendo que ambas esperanzas son finitas.
 
\begin{Note}
Funciones de procesos regenerativos son regenerativas, es decir, si $X\left(t\right)$ es regenerativo y se define el proceso $Y\left(t\right)$ por $Y\left(t\right)=f\left(X\left(t\right)\right)$ para alguna funci\'on Borel medible $f\left(\cdot\right)$. Adem\'as $Y$ es regenerativo con los mismos tiempos de renovaci\'on que $X$. 

En general, los tiempos de renovaci\'on, $Z_{k}$ de un proceso regenerativo no requieren ser tiempos de paro con respecto a la evoluci\'on de $X\left(t\right)$.
\end{Note} 

\begin{Note}
Una funci\'on de un proceso de Markov, usualmente no ser\'a un proceso de Markov, sin embargo ser\'a regenerativo si el proceso de Markov lo es.
\end{Note}

 
\begin{Note}
Un proceso regenerativo con media de la longitud de ciclo finita es llamado positivo recurrente.
\end{Note}


\begin{Note}
\begin{itemize}
\item[a)] Si el proceso regenerativo $X$ es positivo recurrente y tiene trayectorias muestrales no negativas, entonces la ecuaci\'on anterior es v\'alida.
\item[b)] Si $X$ es positivo recurrente regenerativo, podemos construir una \'unica versi\'on estacionaria de este proceso, $X_{e}=\left\{X_{e}\left(t\right)\right\}$, donde $X_{e}$ es un proceso estoc\'astico regenerativo y estrictamente estacionario, con distribuci\'on marginal distribuida como $X_{\infty}$
\end{itemize}
\end{Note}


%__________________________________________________________________________________________
\section{Procesos Regenerativos Estacionarios - Stidham \cite{Stidham}}
%__________________________________________________________________________________________


Un proceso estoc\'astico a tiempo continuo $\left\{V\left(t\right),t\geq0\right\}$ es un proceso regenerativo si existe una sucesi\'on de variables aleatorias independientes e id\'enticamente distribuidas $\left\{X_{1},X_{2},\ldots\right\}$, sucesi\'on de renovaci\'on, tal que para cualquier conjunto de Borel $A$, 

\begin{eqnarray*}
\prob\left\{V\left(t\right)\in A|X_{1}+X_{2}+\cdots+X_{R\left(t\right)}=s,\left\{V\left(\tau\right),\tau<s\right\}\right\}=\prob\left\{V\left(t-s\right)\in A|X_{1}>t-s\right\},
\end{eqnarray*}
para todo $0\leq s\leq t$, donde $R\left(t\right)=\max\left\{X_{1}+X_{2}+\cdots+X_{j}\leq t\right\}=$n\'umero de renovaciones ({\emph{puntos de regeneraci\'on}}) que ocurren en $\left[0,t\right]$. El intervalo $\left[0,X_{1}\right)$ es llamado {\emph{primer ciclo de regeneraci\'on}} de $\left\{V\left(t \right),t\geq0\right\}$, $\left[X_{1},X_{1}+X_{2}\right)$ el {\emph{segundo ciclo de regeneraci\'on}}, y as\'i sucesivamente.

Sea $X=X_{1}$ y sea $F$ la funci\'on de distrbuci\'on de $X$


\begin{Def}
Se define el proceso estacionario, $\left\{V^{*}\left(t\right),t\geq0\right\}$, para $\left\{V\left(t\right),t\geq0\right\}$ por

\begin{eqnarray*}
\prob\left\{V\left(t\right)\in A\right\}=\frac{1}{\esp\left[X\right]}\int_{0}^{\infty}\prob\left\{V\left(t+x\right)\in A|X>x\right\}\left(1-F\left(x\right)\right)dx,
\end{eqnarray*} 
para todo $t\geq0$ y todo conjunto de Borel $A$.
\end{Def}

\begin{Def}
Una distribuci\'on se dice que es {\emph{aritm\'etica}} si todos sus puntos de incremento son m\'ultiplos de la forma $0,\lambda, 2\lambda,\ldots$ para alguna $\lambda>0$ entera.
\end{Def}


\begin{Def}
Una modificaci\'on medible de un proceso $\left\{V\left(t\right),t\geq0\right\}$, es una versi\'on de este, $\left\{V\left(t,w\right)\right\}$ conjuntamente medible para $t\geq0$ y para $w\in S$, $S$ espacio de estados para $\left\{V\left(t\right),t\geq0\right\}$.
\end{Def}

\begin{Teo}
Sea $\left\{V\left(t\right),t\geq\right\}$ un proceso regenerativo no negativo con modificaci\'on medible. Sea $\esp\left[X\right]<\infty$. Entonces el proceso estacionario dado por la ecuaci\'on anterior est\'a bien definido y tiene funci\'on de distribuci\'on independiente de $t$, adem\'as
\begin{itemize}
\item[i)] \begin{eqnarray*}
\esp\left[V^{*}\left(0\right)\right]&=&\frac{\esp\left[\int_{0}^{X}V\left(s\right)ds\right]}{\esp\left[X\right]}\end{eqnarray*}
\item[ii)] Si $\esp\left[V^{*}\left(0\right)\right]<\infty$, equivalentemente, si $\esp\left[\int_{0}^{X}V\left(s\right)ds\right]<\infty$,entonces
\begin{eqnarray*}
\frac{\int_{0}^{t}V\left(s\right)ds}{t}\rightarrow\frac{\esp\left[\int_{0}^{X}V\left(s\right)ds\right]}{\esp\left[X\right]}
\end{eqnarray*}
con probabilidad 1 y en media, cuando $t\rightarrow\infty$.
\end{itemize}
\end{Teo}

\begin{Coro}
Sea $\left\{V\left(t\right),t\geq0\right\}$ un proceso regenerativo no negativo, con modificaci\'on medible. Si $\esp <\infty$, $F$ es no-aritm\'etica, y para todo $x\geq0$, $P\left\{V\left(t\right)\leq x,C>x\right\}$ es de variaci\'on acotada como funci\'on de $t$ en cada intervalo finito $\left[0,\tau\right]$, entonces $V\left(t\right)$ converge en distribuci\'on  cuando $t\rightarrow\infty$ y $$\esp V=\frac{\esp \int_{0}^{X}V\left(s\right)ds}{\esp X}$$
Donde $V$ tiene la distribuci\'on l\'imite de $V\left(t\right)$ cuando $t\rightarrow\infty$.

\end{Coro}

Para el caso discreto se tienen resultados similares.


%__________________________________________________________________________________________
%\subsection*{Procesos Regenerativos Estacionarios - Stidham \cite{Stidham}}
%__________________________________________________________________________________________


Un proceso estoc\'astico a tiempo continuo $\left\{V\left(t\right),t\geq0\right\}$ es un proceso regenerativo si existe una sucesi\'on de variables aleatorias independientes e id\'enticamente distribuidas $\left\{X_{1},X_{2},\ldots\right\}$, sucesi\'on de renovaci\'on, tal que para cualquier conjunto de Borel $A$, 

\begin{eqnarray*}
\prob\left\{V\left(t\right)\in A|X_{1}+X_{2}+\cdots+X_{R\left(t\right)}=s,\left\{V\left(\tau\right),\tau<s\right\}\right\}=\prob\left\{V\left(t-s\right)\in A|X_{1}>t-s\right\},
\end{eqnarray*}
para todo $0\leq s\leq t$, donde $R\left(t\right)=\max\left\{X_{1}+X_{2}+\cdots+X_{j}\leq t\right\}=$n\'umero de renovaciones ({\emph{puntos de regeneraci\'on}}) que ocurren en $\left[0,t\right]$. El intervalo $\left[0,X_{1}\right)$ es llamado {\emph{primer ciclo de regeneraci\'on}} de $\left\{V\left(t \right),t\geq0\right\}$, $\left[X_{1},X_{1}+X_{2}\right)$ el {\emph{segundo ciclo de regeneraci\'on}}, y as\'i sucesivamente.

Sea $X=X_{1}$ y sea $F$ la funci\'on de distrbuci\'on de $X$


\begin{Def}
Se define el proceso estacionario, $\left\{V^{*}\left(t\right),t\geq0\right\}$, para $\left\{V\left(t\right),t\geq0\right\}$ por

\begin{eqnarray*}
\prob\left\{V\left(t\right)\in A\right\}=\frac{1}{\esp\left[X\right]}\int_{0}^{\infty}\prob\left\{V\left(t+x\right)\in A|X>x\right\}\left(1-F\left(x\right)\right)dx,
\end{eqnarray*} 
para todo $t\geq0$ y todo conjunto de Borel $A$.
\end{Def}

\begin{Def}
Una distribuci\'on se dice que es {\emph{aritm\'etica}} si todos sus puntos de incremento son m\'ultiplos de la forma $0,\lambda, 2\lambda,\ldots$ para alguna $\lambda>0$ entera.
\end{Def}


\begin{Def}
Una modificaci\'on medible de un proceso $\left\{V\left(t\right),t\geq0\right\}$, es una versi\'on de este, $\left\{V\left(t,w\right)\right\}$ conjuntamente medible para $t\geq0$ y para $w\in S$, $S$ espacio de estados para $\left\{V\left(t\right),t\geq0\right\}$.
\end{Def}

\begin{Teo}
Sea $\left\{V\left(t\right),t\geq\right\}$ un proceso regenerativo no negativo con modificaci\'on medible. Sea $\esp\left[X\right]<\infty$. Entonces el proceso estacionario dado por la ecuaci\'on anterior est\'a bien definido y tiene funci\'on de distribuci\'on independiente de $t$, adem\'as
\begin{itemize}
\item[i)] \begin{eqnarray*}
\esp\left[V^{*}\left(0\right)\right]&=&\frac{\esp\left[\int_{0}^{X}V\left(s\right)ds\right]}{\esp\left[X\right]}\end{eqnarray*}
\item[ii)] Si $\esp\left[V^{*}\left(0\right)\right]<\infty$, equivalentemente, si $\esp\left[\int_{0}^{X}V\left(s\right)ds\right]<\infty$,entonces
\begin{eqnarray*}
\frac{\int_{0}^{t}V\left(s\right)ds}{t}\rightarrow\frac{\esp\left[\int_{0}^{X}V\left(s\right)ds\right]}{\esp\left[X\right]}
\end{eqnarray*}
con probabilidad 1 y en media, cuando $t\rightarrow\infty$.
\end{itemize}
\end{Teo}


%__________________________________________________________________________________________
%\subsection*{Procesos Regenerativos Estacionarios - Stidham \cite{Stidham}}
%__________________________________________________________________________________________


Un proceso estoc\'astico a tiempo continuo $\left\{V\left(t\right),t\geq0\right\}$ es un proceso regenerativo si existe una sucesi\'on de variables aleatorias independientes e id\'enticamente distribuidas $\left\{X_{1},X_{2},\ldots\right\}$, sucesi\'on de renovaci\'on, tal que para cualquier conjunto de Borel $A$, 

\begin{eqnarray*}
\prob\left\{V\left(t\right)\in A|X_{1}+X_{2}+\cdots+X_{R\left(t\right)}=s,\left\{V\left(\tau\right),\tau<s\right\}\right\}=\prob\left\{V\left(t-s\right)\in A|X_{1}>t-s\right\},
\end{eqnarray*}
para todo $0\leq s\leq t$, donde $R\left(t\right)=\max\left\{X_{1}+X_{2}+\cdots+X_{j}\leq t\right\}=$n\'umero de renovaciones ({\emph{puntos de regeneraci\'on}}) que ocurren en $\left[0,t\right]$. El intervalo $\left[0,X_{1}\right)$ es llamado {\emph{primer ciclo de regeneraci\'on}} de $\left\{V\left(t \right),t\geq0\right\}$, $\left[X_{1},X_{1}+X_{2}\right)$ el {\emph{segundo ciclo de regeneraci\'on}}, y as\'i sucesivamente.

Sea $X=X_{1}$ y sea $F$ la funci\'on de distrbuci\'on de $X$


\begin{Def}
Se define el proceso estacionario, $\left\{V^{*}\left(t\right),t\geq0\right\}$, para $\left\{V\left(t\right),t\geq0\right\}$ por

\begin{eqnarray*}
\prob\left\{V\left(t\right)\in A\right\}=\frac{1}{\esp\left[X\right]}\int_{0}^{\infty}\prob\left\{V\left(t+x\right)\in A|X>x\right\}\left(1-F\left(x\right)\right)dx,
\end{eqnarray*} 
para todo $t\geq0$ y todo conjunto de Borel $A$.
\end{Def}

\begin{Def}
Una distribuci\'on se dice que es {\emph{aritm\'etica}} si todos sus puntos de incremento son m\'ultiplos de la forma $0,\lambda, 2\lambda,\ldots$ para alguna $\lambda>0$ entera.
\end{Def}


\begin{Def}
Una modificaci\'on medible de un proceso $\left\{V\left(t\right),t\geq0\right\}$, es una versi\'on de este, $\left\{V\left(t,w\right)\right\}$ conjuntamente medible para $t\geq0$ y para $w\in S$, $S$ espacio de estados para $\left\{V\left(t\right),t\geq0\right\}$.
\end{Def}

\begin{Teo}
Sea $\left\{V\left(t\right),t\geq\right\}$ un proceso regenerativo no negativo con modificaci\'on medible. Sea $\esp\left[X\right]<\infty$. Entonces el proceso estacionario dado por la ecuaci\'on anterior est\'a bien definido y tiene funci\'on de distribuci\'on independiente de $t$, adem\'as
\begin{itemize}
\item[i)] \begin{eqnarray*}
\esp\left[V^{*}\left(0\right)\right]&=&\frac{\esp\left[\int_{0}^{X}V\left(s\right)ds\right]}{\esp\left[X\right]}\end{eqnarray*}
\item[ii)] Si $\esp\left[V^{*}\left(0\right)\right]<\infty$, equivalentemente, si $\esp\left[\int_{0}^{X}V\left(s\right)ds\right]<\infty$,entonces
\begin{eqnarray*}
\frac{\int_{0}^{t}V\left(s\right)ds}{t}\rightarrow\frac{\esp\left[\int_{0}^{X}V\left(s\right)ds\right]}{\esp\left[X\right]}
\end{eqnarray*}
con probabilidad 1 y en media, cuando $t\rightarrow\infty$.
\end{itemize}
\end{Teo}

Sea la funci\'on generadora de momentos para $L_{i}$, el n\'umero de usuarios en la cola $Q_{i}\left(z\right)$ en cualquier momento, est\'a dada por el tiempo promedio de $z^{L_{i}\left(t\right)}$ sobre el ciclo regenerativo definido anteriormente. Entonces 



Es decir, es posible determinar las longitudes de las colas a cualquier tiempo $t$. Entonces, determinando el primer momento es posible ver que


\begin{Def}
El tiempo de Ciclo $C_{i}$ es el periodo de tiempo que comienza cuando la cola $i$ es visitada por primera vez en un ciclo, y termina cuando es visitado nuevamente en el pr\'oximo ciclo. La duraci\'on del mismo est\'a dada por $\tau_{i}\left(m+1\right)-\tau_{i}\left(m\right)$, o equivalentemente $\overline{\tau}_{i}\left(m+1\right)-\overline{\tau}_{i}\left(m\right)$ bajo condiciones de estabilidad.
\end{Def}


\begin{Def}
El tiempo de intervisita $I_{i}$ es el periodo de tiempo que comienza cuando se ha completado el servicio en un ciclo y termina cuando es visitada nuevamente en el pr\'oximo ciclo. Su  duraci\'on del mismo est\'a dada por $\tau_{i}\left(m+1\right)-\overline{\tau}_{i}\left(m\right)$.
\end{Def}

La duraci\'on del tiempo de intervisita es $\tau_{i}\left(m+1\right)-\overline{\tau}\left(m\right)$. Dado que el n\'umero de usuarios presentes en $Q_{i}$ al tiempo $t=\tau_{i}\left(m+1\right)$ es igual al n\'umero de arribos durante el intervalo de tiempo $\left[\overline{\tau}\left(m\right),\tau_{i}\left(m+1\right)\right]$ se tiene que


\begin{eqnarray*}
\esp\left[z_{i}^{L_{i}\left(\tau_{i}\left(m+1\right)\right)}\right]=\esp\left[\left\{P_{i}\left(z_{i}\right)\right\}^{\tau_{i}\left(m+1\right)-\overline{\tau}\left(m\right)}\right]
\end{eqnarray*}

entonces, si $I_{i}\left(z\right)=\esp\left[z^{\tau_{i}\left(m+1\right)-\overline{\tau}\left(m\right)}\right]$
se tiene que $F_{i}\left(z\right)=I_{i}\left[P_{i}\left(z\right)\right]$
para $i=1,2$.

Conforme a la definici\'on dada al principio del cap\'itulo, definici\'on (\ref{Def.Tn}), sean $T_{1},T_{2},\ldots$ los puntos donde las longitudes de las colas de la red de sistemas de visitas c\'iclicas son cero simult\'aneamente, cuando la cola $Q_{j}$ es visitada por el servidor para dar servicio, es decir, $L_{1}\left(T_{i}\right)=0,L_{2}\left(T_{i}\right)=0,\hat{L}_{1}\left(T_{i}\right)=0$ y $\hat{L}_{2}\left(T_{i}\right)=0$, a estos puntos se les denominar\'a puntos regenerativos. Entonces, 

\begin{Def}
Al intervalo de tiempo entre dos puntos regenerativos se le llamar\'a ciclo regenerativo.
\end{Def}

\begin{Def}
Para $T_{i}$ se define, $M_{i}$, el n\'umero de ciclos de visita a la cola $Q_{l}$, durante el ciclo regenerativo, es decir, $M_{i}$ es un proceso de renovaci\'on.
\end{Def}

\begin{Def}
Para cada uno de los $M_{i}$'s, se definen a su vez la duraci\'on de cada uno de estos ciclos de visita en el ciclo regenerativo, $C_{i}^{(m)}$, para $m=1,2,\ldots,M_{i}$, que a su vez, tambi\'en es n proceso de renovaci\'on.
\end{Def}

\footnote{In Stidham and  Heyman \cite{Stidham} shows that is sufficient for the regenerative process to be stationary that the mean regenerative cycle time is finite: $\esp\left[\sum_{m=1}^{M_{i}}C_{i}^{(m)}\right]<\infty$, 


 como cada $C_{i}^{(m)}$ contiene intervalos de r\'eplica positivos, se tiene que $\esp\left[M_{i}\right]<\infty$, adem\'as, como $M_{i}>0$, se tiene que la condici\'on anterior es equivalente a tener que $\esp\left[C_{i}\right]<\infty$,
por lo tanto una condici\'on suficiente para la existencia del proceso regenerativo est\'a dada por $\sum_{k=1}^{N}\mu_{k}<1.$}

%__________________________________________________________________________________________
%\subsection*{Procesos Regenerativos Estacionarios - Stidham \cite{Stidham}}
%__________________________________________________________________________________________


Un proceso estoc\'astico a tiempo continuo $\left\{V\left(t\right),t\geq0\right\}$ es un proceso regenerativo si existe una sucesi\'on de variables aleatorias independientes e id\'enticamente distribuidas $\left\{X_{1},X_{2},\ldots\right\}$, sucesi\'on de renovaci\'on, tal que para cualquier conjunto de Borel $A$, 

\begin{eqnarray*}
\prob\left\{V\left(t\right)\in A|X_{1}+X_{2}+\cdots+X_{R\left(t\right)}=s,\left\{V\left(\tau\right),\tau<s\right\}\right\}=\prob\left\{V\left(t-s\right)\in A|X_{1}>t-s\right\},
\end{eqnarray*}
para todo $0\leq s\leq t$, donde $R\left(t\right)=\max\left\{X_{1}+X_{2}+\cdots+X_{j}\leq t\right\}=$n\'umero de renovaciones ({\emph{puntos de regeneraci\'on}}) que ocurren en $\left[0,t\right]$. El intervalo $\left[0,X_{1}\right)$ es llamado {\emph{primer ciclo de regeneraci\'on}} de $\left\{V\left(t \right),t\geq0\right\}$, $\left[X_{1},X_{1}+X_{2}\right)$ el {\emph{segundo ciclo de regeneraci\'on}}, y as\'i sucesivamente.

Sea $X=X_{1}$ y sea $F$ la funci\'on de distrbuci\'on de $X$


\begin{Def}
Se define el proceso estacionario, $\left\{V^{*}\left(t\right),t\geq0\right\}$, para $\left\{V\left(t\right),t\geq0\right\}$ por

\begin{eqnarray*}
\prob\left\{V\left(t\right)\in A\right\}=\frac{1}{\esp\left[X\right]}\int_{0}^{\infty}\prob\left\{V\left(t+x\right)\in A|X>x\right\}\left(1-F\left(x\right)\right)dx,
\end{eqnarray*} 
para todo $t\geq0$ y todo conjunto de Borel $A$.
\end{Def}

\begin{Def}
Una distribuci\'on se dice que es {\emph{aritm\'etica}} si todos sus puntos de incremento son m\'ultiplos de la forma $0,\lambda, 2\lambda,\ldots$ para alguna $\lambda>0$ entera.
\end{Def}


\begin{Def}
Una modificaci\'on medible de un proceso $\left\{V\left(t\right),t\geq0\right\}$, es una versi\'on de este, $\left\{V\left(t,w\right)\right\}$ conjuntamente medible para $t\geq0$ y para $w\in S$, $S$ espacio de estados para $\left\{V\left(t\right),t\geq0\right\}$.
\end{Def}

\begin{Teo}
Sea $\left\{V\left(t\right),t\geq\right\}$ un proceso regenerativo no negativo con modificaci\'on medible. Sea $\esp\left[X\right]<\infty$. Entonces el proceso estacionario dado por la ecuaci\'on anterior est\'a bien definido y tiene funci\'on de distribuci\'on independiente de $t$, adem\'as
\begin{itemize}
\item[i)] \begin{eqnarray*}
\esp\left[V^{*}\left(0\right)\right]&=&\frac{\esp\left[\int_{0}^{X}V\left(s\right)ds\right]}{\esp\left[X\right]}\end{eqnarray*}
\item[ii)] Si $\esp\left[V^{*}\left(0\right)\right]<\infty$, equivalentemente, si $\esp\left[\int_{0}^{X}V\left(s\right)ds\right]<\infty$,entonces
\begin{eqnarray*}
\frac{\int_{0}^{t}V\left(s\right)ds}{t}\rightarrow\frac{\esp\left[\int_{0}^{X}V\left(s\right)ds\right]}{\esp\left[X\right]}
\end{eqnarray*}
con probabilidad 1 y en media, cuando $t\rightarrow\infty$.
\end{itemize}
\end{Teo}

\begin{Coro}
Sea $\left\{V\left(t\right),t\geq0\right\}$ un proceso regenerativo no negativo, con modificaci\'on medible. Si $\esp <\infty$, $F$ es no-aritm\'etica, y para todo $x\geq0$, $P\left\{V\left(t\right)\leq x,C>x\right\}$ es de variaci\'on acotada como funci\'on de $t$ en cada intervalo finito $\left[0,\tau\right]$, entonces $V\left(t\right)$ converge en distribuci\'on  cuando $t\rightarrow\infty$ y $$\esp V=\frac{\esp \int_{0}^{X}V\left(s\right)ds}{\esp X}$$
Donde $V$ tiene la distribuci\'on l\'imite de $V\left(t\right)$ cuando $t\rightarrow\infty$.

\end{Coro}

Para el caso discreto se tienen resultados similares.



%__________________________________________________________________________________________
%\subsection*{Procesos Regenerativos Estacionarios - Stidham \cite{Stidham}}
%__________________________________________________________________________________________


Un proceso estoc\'astico a tiempo continuo $\left\{V\left(t\right),t\geq0\right\}$ es un proceso regenerativo si existe una sucesi\'on de variables aleatorias independientes e id\'enticamente distribuidas $\left\{X_{1},X_{2},\ldots\right\}$, sucesi\'on de renovaci\'on, tal que para cualquier conjunto de Borel $A$, 

\begin{eqnarray*}
\prob\left\{V\left(t\right)\in A|X_{1}+X_{2}+\cdots+X_{R\left(t\right)}=s,\left\{V\left(\tau\right),\tau<s\right\}\right\}=\prob\left\{V\left(t-s\right)\in A|X_{1}>t-s\right\},
\end{eqnarray*}
para todo $0\leq s\leq t$, donde $R\left(t\right)=\max\left\{X_{1}+X_{2}+\cdots+X_{j}\leq t\right\}=$n\'umero de renovaciones ({\emph{puntos de regeneraci\'on}}) que ocurren en $\left[0,t\right]$. El intervalo $\left[0,X_{1}\right)$ es llamado {\emph{primer ciclo de regeneraci\'on}} de $\left\{V\left(t \right),t\geq0\right\}$, $\left[X_{1},X_{1}+X_{2}\right)$ el {\emph{segundo ciclo de regeneraci\'on}}, y as\'i sucesivamente.

Sea $X=X_{1}$ y sea $F$ la funci\'on de distrbuci\'on de $X$


\begin{Def}
Se define el proceso estacionario, $\left\{V^{*}\left(t\right),t\geq0\right\}$, para $\left\{V\left(t\right),t\geq0\right\}$ por

\begin{eqnarray*}
\prob\left\{V\left(t\right)\in A\right\}=\frac{1}{\esp\left[X\right]}\int_{0}^{\infty}\prob\left\{V\left(t+x\right)\in A|X>x\right\}\left(1-F\left(x\right)\right)dx,
\end{eqnarray*} 
para todo $t\geq0$ y todo conjunto de Borel $A$.
\end{Def}

\begin{Def}
Una distribuci\'on se dice que es {\emph{aritm\'etica}} si todos sus puntos de incremento son m\'ultiplos de la forma $0,\lambda, 2\lambda,\ldots$ para alguna $\lambda>0$ entera.
\end{Def}


\begin{Def}
Una modificaci\'on medible de un proceso $\left\{V\left(t\right),t\geq0\right\}$, es una versi\'on de este, $\left\{V\left(t,w\right)\right\}$ conjuntamente medible para $t\geq0$ y para $w\in S$, $S$ espacio de estados para $\left\{V\left(t\right),t\geq0\right\}$.
\end{Def}

\begin{Teo}
Sea $\left\{V\left(t\right),t\geq\right\}$ un proceso regenerativo no negativo con modificaci\'on medible. Sea $\esp\left[X\right]<\infty$. Entonces el proceso estacionario dado por la ecuaci\'on anterior est\'a bien definido y tiene funci\'on de distribuci\'on independiente de $t$, adem\'as
\begin{itemize}
\item[i)] \begin{eqnarray*}
\esp\left[V^{*}\left(0\right)\right]&=&\frac{\esp\left[\int_{0}^{X}V\left(s\right)ds\right]}{\esp\left[X\right]}\end{eqnarray*}
\item[ii)] Si $\esp\left[V^{*}\left(0\right)\right]<\infty$, equivalentemente, si $\esp\left[\int_{0}^{X}V\left(s\right)ds\right]<\infty$,entonces
\begin{eqnarray*}
\frac{\int_{0}^{t}V\left(s\right)ds}{t}\rightarrow\frac{\esp\left[\int_{0}^{X}V\left(s\right)ds\right]}{\esp\left[X\right]}
\end{eqnarray*}
con probabilidad 1 y en media, cuando $t\rightarrow\infty$.
\end{itemize}
\end{Teo}

%__________________________________________________________________________________________
%\subsection*{Procesos Regenerativos Estacionarios - Stidham \cite{Stidham}}
%__________________________________________________________________________________________


Un proceso estoc\'astico a tiempo continuo $\left\{V\left(t\right),t\geq0\right\}$ es un proceso regenerativo si existe una sucesi\'on de variables aleatorias independientes e id\'enticamente distribuidas $\left\{X_{1},X_{2},\ldots\right\}$, sucesi\'on de renovaci\'on, tal que para cualquier conjunto de Borel $A$, 

\begin{eqnarray*}
\prob\left\{V\left(t\right)\in A|X_{1}+X_{2}+\cdots+X_{R\left(t\right)}=s,\left\{V\left(\tau\right),\tau<s\right\}\right\}=\prob\left\{V\left(t-s\right)\in A|X_{1}>t-s\right\},
\end{eqnarray*}
para todo $0\leq s\leq t$, donde $R\left(t\right)=\max\left\{X_{1}+X_{2}+\cdots+X_{j}\leq t\right\}=$n\'umero de renovaciones ({\emph{puntos de regeneraci\'on}}) que ocurren en $\left[0,t\right]$. El intervalo $\left[0,X_{1}\right)$ es llamado {\emph{primer ciclo de regeneraci\'on}} de $\left\{V\left(t \right),t\geq0\right\}$, $\left[X_{1},X_{1}+X_{2}\right)$ el {\emph{segundo ciclo de regeneraci\'on}}, y as\'i sucesivamente.

Sea $X=X_{1}$ y sea $F$ la funci\'on de distrbuci\'on de $X$


\begin{Def}
Se define el proceso estacionario, $\left\{V^{*}\left(t\right),t\geq0\right\}$, para $\left\{V\left(t\right),t\geq0\right\}$ por

\begin{eqnarray*}
\prob\left\{V\left(t\right)\in A\right\}=\frac{1}{\esp\left[X\right]}\int_{0}^{\infty}\prob\left\{V\left(t+x\right)\in A|X>x\right\}\left(1-F\left(x\right)\right)dx,
\end{eqnarray*} 
para todo $t\geq0$ y todo conjunto de Borel $A$.
\end{Def}

\begin{Def}
Una distribuci\'on se dice que es {\emph{aritm\'etica}} si todos sus puntos de incremento son m\'ultiplos de la forma $0,\lambda, 2\lambda,\ldots$ para alguna $\lambda>0$ entera.
\end{Def}


\begin{Def}
Una modificaci\'on medible de un proceso $\left\{V\left(t\right),t\geq0\right\}$, es una versi\'on de este, $\left\{V\left(t,w\right)\right\}$ conjuntamente medible para $t\geq0$ y para $w\in S$, $S$ espacio de estados para $\left\{V\left(t\right),t\geq0\right\}$.
\end{Def}

\begin{Teo}
Sea $\left\{V\left(t\right),t\geq\right\}$ un proceso regenerativo no negativo con modificaci\'on medible. Sea $\esp\left[X\right]<\infty$. Entonces el proceso estacionario dado por la ecuaci\'on anterior est\'a bien definido y tiene funci\'on de distribuci\'on independiente de $t$, adem\'as
\begin{itemize}
\item[i)] \begin{eqnarray*}
\esp\left[V^{*}\left(0\right)\right]&=&\frac{\esp\left[\int_{0}^{X}V\left(s\right)ds\right]}{\esp\left[X\right]}\end{eqnarray*}
\item[ii)] Si $\esp\left[V^{*}\left(0\right)\right]<\infty$, equivalentemente, si $\esp\left[\int_{0}^{X}V\left(s\right)ds\right]<\infty$,entonces
\begin{eqnarray*}
\frac{\int_{0}^{t}V\left(s\right)ds}{t}\rightarrow\frac{\esp\left[\int_{0}^{X}V\left(s\right)ds\right]}{\esp\left[X\right]}
\end{eqnarray*}
con probabilidad 1 y en media, cuando $t\rightarrow\infty$.
\end{itemize}
\end{Teo}

Para $\left\{X\left(t\right):t\geq0\right\}$ Proceso Estoc\'astico a tiempo continuo con estado de espacios $S$, que es un espacio m\'etrico, con trayectorias continuas por la derecha y con l\'imites por la izquierda c.s. Sea $N\left(t\right)$ un proceso de renovaci\'on en $\rea_{+}$ definido en el mismo espacio de probabilidad que $X\left(t\right)$, con tiempos de renovaci\'on $T$ y tiempos de inter-renovaci\'on $\xi_{n}=T_{n}-T_{n-1}$, con misma distribuci\'on $F$ de media finita $\mu$.

%__________________________________________________________________________________________
%\subsection*{Procesos Regenerativos Estacionarios - Stidham \cite{Stidham}}
%__________________________________________________________________________________________


Un proceso estoc\'astico a tiempo continuo $\left\{V\left(t\right),t\geq0\right\}$ es un proceso regenerativo si existe una sucesi\'on de variables aleatorias independientes e id\'enticamente distribuidas $\left\{X_{1},X_{2},\ldots\right\}$, sucesi\'on de renovaci\'on, tal que para cualquier conjunto de Borel $A$, 

\begin{eqnarray*}
\prob\left\{V\left(t\right)\in A|X_{1}+X_{2}+\cdots+X_{R\left(t\right)}=s,\left\{V\left(\tau\right),\tau<s\right\}\right\}=\prob\left\{V\left(t-s\right)\in A|X_{1}>t-s\right\},
\end{eqnarray*}
para todo $0\leq s\leq t$, donde $R\left(t\right)=\max\left\{X_{1}+X_{2}+\cdots+X_{j}\leq t\right\}=$n\'umero de renovaciones ({\emph{puntos de regeneraci\'on}}) que ocurren en $\left[0,t\right]$. El intervalo $\left[0,X_{1}\right)$ es llamado {\emph{primer ciclo de regeneraci\'on}} de $\left\{V\left(t \right),t\geq0\right\}$, $\left[X_{1},X_{1}+X_{2}\right)$ el {\emph{segundo ciclo de regeneraci\'on}}, y as\'i sucesivamente.

Sea $X=X_{1}$ y sea $F$ la funci\'on de distrbuci\'on de $X$


\begin{Def}
Se define el proceso estacionario, $\left\{V^{*}\left(t\right),t\geq0\right\}$, para $\left\{V\left(t\right),t\geq0\right\}$ por

\begin{eqnarray*}
\prob\left\{V\left(t\right)\in A\right\}=\frac{1}{\esp\left[X\right]}\int_{0}^{\infty}\prob\left\{V\left(t+x\right)\in A|X>x\right\}\left(1-F\left(x\right)\right)dx,
\end{eqnarray*} 
para todo $t\geq0$ y todo conjunto de Borel $A$.
\end{Def}

\begin{Def}
Una distribuci\'on se dice que es {\emph{aritm\'etica}} si todos sus puntos de incremento son m\'ultiplos de la forma $0,\lambda, 2\lambda,\ldots$ para alguna $\lambda>0$ entera.
\end{Def}


\begin{Def}
Una modificaci\'on medible de un proceso $\left\{V\left(t\right),t\geq0\right\}$, es una versi\'on de este, $\left\{V\left(t,w\right)\right\}$ conjuntamente medible para $t\geq0$ y para $w\in S$, $S$ espacio de estados para $\left\{V\left(t\right),t\geq0\right\}$.
\end{Def}

\begin{Teo}
Sea $\left\{V\left(t\right),t\geq\right\}$ un proceso regenerativo no negativo con modificaci\'on medible. Sea $\esp\left[X\right]<\infty$. Entonces el proceso estacionario dado por la ecuaci\'on anterior est\'a bien definido y tiene funci\'on de distribuci\'on independiente de $t$, adem\'as
\begin{itemize}
\item[i)] \begin{eqnarray*}
\esp\left[V^{*}\left(0\right)\right]&=&\frac{\esp\left[\int_{0}^{X}V\left(s\right)ds\right]}{\esp\left[X\right]}\end{eqnarray*}
\item[ii)] Si $\esp\left[V^{*}\left(0\right)\right]<\infty$, equivalentemente, si $\esp\left[\int_{0}^{X}V\left(s\right)ds\right]<\infty$,entonces
\begin{eqnarray*}
\frac{\int_{0}^{t}V\left(s\right)ds}{t}\rightarrow\frac{\esp\left[\int_{0}^{X}V\left(s\right)ds\right]}{\esp\left[X\right]}
\end{eqnarray*}
con probabilidad 1 y en media, cuando $t\rightarrow\infty$.
\end{itemize}
\end{Teo}

\begin{Coro}
Sea $\left\{V\left(t\right),t\geq0\right\}$ un proceso regenerativo no negativo, con modificaci\'on medible. Si $\esp <\infty$, $F$ es no-aritm\'etica, y para todo $x\geq0$, $P\left\{V\left(t\right)\leq x,C>x\right\}$ es de variaci\'on acotada como funci\'on de $t$ en cada intervalo finito $\left[0,\tau\right]$, entonces $V\left(t\right)$ converge en distribuci\'on  cuando $t\rightarrow\infty$ y $$\esp V=\frac{\esp \int_{0}^{X}V\left(s\right)ds}{\esp X}$$
Donde $V$ tiene la distribuci\'on l\'imite de $V\left(t\right)$ cuando $t\rightarrow\infty$.

\end{Coro}

Para el caso discreto se tienen resultados similares.


%___________________________________________________________________________________________
%
\section{Teorema Principal de Renovaci\'on}
%___________________________________________________________________________________________
%

\begin{Note} Una funci\'on $h:\rea_{+}\rightarrow\rea$ es Directamente Riemann Integrable en los siguientes casos:
\begin{itemize}
\item[a)] $h\left(t\right)\geq0$ es decreciente y Riemann Integrable.
\item[b)] $h$ es continua excepto posiblemente en un conjunto de Lebesgue de medida 0, y $|h\left(t\right)|\leq b\left(t\right)$, donde $b$ es DRI.
\end{itemize}
\end{Note}

\begin{Teo}[Teorema Principal de Renovaci\'on]
Si $F$ es no aritm\'etica y $h\left(t\right)$ es Directamente Riemann Integrable (DRI), entonces

\begin{eqnarray*}
lim_{t\rightarrow\infty}U\star h=\frac{1}{\mu}\int_{\rea_{+}}h\left(s\right)ds.
\end{eqnarray*}
\end{Teo}

\begin{Prop}
Cualquier funci\'on $H\left(t\right)$ acotada en intervalos finitos y que es 0 para $t<0$ puede expresarse como
\begin{eqnarray*}
H\left(t\right)=U\star h\left(t\right)\textrm{,  donde }h\left(t\right)=H\left(t\right)-F\star H\left(t\right)
\end{eqnarray*}
\end{Prop}

\begin{Def}
Un proceso estoc\'astico $X\left(t\right)$ es crudamente regenerativo en un tiempo aleatorio positivo $T$ si
\begin{eqnarray*}
\esp\left[X\left(T+t\right)|T\right]=\esp\left[X\left(t\right)\right]\textrm{, para }t\geq0,\end{eqnarray*}
y con las esperanzas anteriores finitas.
\end{Def}

\begin{Prop}
Sup\'ongase que $X\left(t\right)$ es un proceso crudamente regenerativo en $T$, que tiene distribuci\'on $F$. Si $\esp\left[X\left(t\right)\right]$ es acotado en intervalos finitos, entonces
\begin{eqnarray*}
\esp\left[X\left(t\right)\right]=U\star h\left(t\right)\textrm{,  donde }h\left(t\right)=\esp\left[X\left(t\right)\indora\left(T>t\right)\right].
\end{eqnarray*}
\end{Prop}

\begin{Teo}[Regeneraci\'on Cruda]
Sup\'ongase que $X\left(t\right)$ es un proceso con valores positivo crudamente regenerativo en $T$, y def\'inase $M=\sup\left\{|X\left(t\right)|:t\leq T\right\}$. Si $T$ es no aritm\'etico y $M$ y $MT$ tienen media finita, entonces
\begin{eqnarray*}
lim_{t\rightarrow\infty}\esp\left[X\left(t\right)\right]=\frac{1}{\mu}\int_{\rea_{+}}h\left(s\right)ds,
\end{eqnarray*}
donde $h\left(t\right)=\esp\left[X\left(t\right)\indora\left(T>t\right)\right]$.
\end{Teo}

%___________________________________________________________________________________________
%
%\subsection*{Teorema Principal de Renovaci\'on}
%___________________________________________________________________________________________
%

\begin{Note} Una funci\'on $h:\rea_{+}\rightarrow\rea$ es Directamente Riemann Integrable en los siguientes casos:
\begin{itemize}
\item[a)] $h\left(t\right)\geq0$ es decreciente y Riemann Integrable.
\item[b)] $h$ es continua excepto posiblemente en un conjunto de Lebesgue de medida 0, y $|h\left(t\right)|\leq b\left(t\right)$, donde $b$ es DRI.
\end{itemize}
\end{Note}

\begin{Teo}[Teorema Principal de Renovaci\'on]
Si $F$ es no aritm\'etica y $h\left(t\right)$ es Directamente Riemann Integrable (DRI), entonces

\begin{eqnarray*}
lim_{t\rightarrow\infty}U\star h=\frac{1}{\mu}\int_{\rea_{+}}h\left(s\right)ds.
\end{eqnarray*}
\end{Teo}

\begin{Prop}
Cualquier funci\'on $H\left(t\right)$ acotada en intervalos finitos y que es 0 para $t<0$ puede expresarse como
\begin{eqnarray*}
H\left(t\right)=U\star h\left(t\right)\textrm{,  donde }h\left(t\right)=H\left(t\right)-F\star H\left(t\right)
\end{eqnarray*}
\end{Prop}

\begin{Def}
Un proceso estoc\'astico $X\left(t\right)$ es crudamente regenerativo en un tiempo aleatorio positivo $T$ si
\begin{eqnarray*}
\esp\left[X\left(T+t\right)|T\right]=\esp\left[X\left(t\right)\right]\textrm{, para }t\geq0,\end{eqnarray*}
y con las esperanzas anteriores finitas.
\end{Def}

\begin{Prop}
Sup\'ongase que $X\left(t\right)$ es un proceso crudamente regenerativo en $T$, que tiene distribuci\'on $F$. Si $\esp\left[X\left(t\right)\right]$ es acotado en intervalos finitos, entonces
\begin{eqnarray*}
\esp\left[X\left(t\right)\right]=U\star h\left(t\right)\textrm{,  donde }h\left(t\right)=\esp\left[X\left(t\right)\indora\left(T>t\right)\right].
\end{eqnarray*}
\end{Prop}

\begin{Teo}[Regeneraci\'on Cruda]
Sup\'ongase que $X\left(t\right)$ es un proceso con valores positivo crudamente regenerativo en $T$, y def\'inase $M=\sup\left\{|X\left(t\right)|:t\leq T\right\}$. Si $T$ es no aritm\'etico y $M$ y $MT$ tienen media finita, entonces
\begin{eqnarray*}
lim_{t\rightarrow\infty}\esp\left[X\left(t\right)\right]=\frac{1}{\mu}\int_{\rea_{+}}h\left(s\right)ds,
\end{eqnarray*}
donde $h\left(t\right)=\esp\left[X\left(t\right)\indora\left(T>t\right)\right]$.
\end{Teo}


%___________________________________________________________________________________________
%
%\subsection*{Teorema Principal de Renovaci\'on}
%___________________________________________________________________________________________
%

\begin{Note} Una funci\'on $h:\rea_{+}\rightarrow\rea$ es Directamente Riemann Integrable en los siguientes casos:
\begin{itemize}
\item[a)] $h\left(t\right)\geq0$ es decreciente y Riemann Integrable.
\item[b)] $h$ es continua excepto posiblemente en un conjunto de Lebesgue de medida 0, y $|h\left(t\right)|\leq b\left(t\right)$, donde $b$ es DRI.
\end{itemize}
\end{Note}

\begin{Teo}[Teorema Principal de Renovaci\'on]
Si $F$ es no aritm\'etica y $h\left(t\right)$ es Directamente Riemann Integrable (DRI), entonces

\begin{eqnarray*}
lim_{t\rightarrow\infty}U\star h=\frac{1}{\mu}\int_{\rea_{+}}h\left(s\right)ds.
\end{eqnarray*}
\end{Teo}

\begin{Prop}
Cualquier funci\'on $H\left(t\right)$ acotada en intervalos finitos y que es 0 para $t<0$ puede expresarse como
\begin{eqnarray*}
H\left(t\right)=U\star h\left(t\right)\textrm{,  donde }h\left(t\right)=H\left(t\right)-F\star H\left(t\right)
\end{eqnarray*}
\end{Prop}

\begin{Def}
Un proceso estoc\'astico $X\left(t\right)$ es crudamente regenerativo en un tiempo aleatorio positivo $T$ si
\begin{eqnarray*}
\esp\left[X\left(T+t\right)|T\right]=\esp\left[X\left(t\right)\right]\textrm{, para }t\geq0,\end{eqnarray*}
y con las esperanzas anteriores finitas.
\end{Def}

\begin{Prop}
Sup\'ongase que $X\left(t\right)$ es un proceso crudamente regenerativo en $T$, que tiene distribuci\'on $F$. Si $\esp\left[X\left(t\right)\right]$ es acotado en intervalos finitos, entonces
\begin{eqnarray*}
\esp\left[X\left(t\right)\right]=U\star h\left(t\right)\textrm{,  donde }h\left(t\right)=\esp\left[X\left(t\right)\indora\left(T>t\right)\right].
\end{eqnarray*}
\end{Prop}

\begin{Teo}[Regeneraci\'on Cruda]
Sup\'ongase que $X\left(t\right)$ es un proceso con valores positivo crudamente regenerativo en $T$, y def\'inase $M=\sup\left\{|X\left(t\right)|:t\leq T\right\}$. Si $T$ es no aritm\'etico y $M$ y $MT$ tienen media finita, entonces
\begin{eqnarray*}
lim_{t\rightarrow\infty}\esp\left[X\left(t\right)\right]=\frac{1}{\mu}\int_{\rea_{+}}h\left(s\right)ds,
\end{eqnarray*}
donde $h\left(t\right)=\esp\left[X\left(t\right)\indora\left(T>t\right)\right]$.
\end{Teo}

%___________________________________________________________________________________________
%
\section{Propiedades de los Procesos de Renovaci\'on}
%___________________________________________________________________________________________
%

Los tiempos $T_{n}$ est\'an relacionados con los conteos de $N\left(t\right)$ por

\begin{eqnarray*}
\left\{N\left(t\right)\geq n\right\}&=&\left\{T_{n}\leq t\right\}\\
T_{N\left(t\right)}\leq &t&<T_{N\left(t\right)+1},
\end{eqnarray*}

adem\'as $N\left(T_{n}\right)=n$, y 

\begin{eqnarray*}
N\left(t\right)=\max\left\{n:T_{n}\leq t\right\}=\min\left\{n:T_{n+1}>t\right\}
\end{eqnarray*}

Por propiedades de la convoluci\'on se sabe que

\begin{eqnarray*}
P\left\{T_{n}\leq t\right\}=F^{n\star}\left(t\right)
\end{eqnarray*}
que es la $n$-\'esima convoluci\'on de $F$. Entonces 

\begin{eqnarray*}
\left\{N\left(t\right)\geq n\right\}&=&\left\{T_{n}\leq t\right\}\\
P\left\{N\left(t\right)\leq n\right\}&=&1-F^{\left(n+1\right)\star}\left(t\right)
\end{eqnarray*}

Adem\'as usando el hecho de que $\esp\left[N\left(t\right)\right]=\sum_{n=1}^{\infty}P\left\{N\left(t\right)\geq n\right\}$
se tiene que

\begin{eqnarray*}
\esp\left[N\left(t\right)\right]=\sum_{n=1}^{\infty}F^{n\star}\left(t\right)
\end{eqnarray*}

\begin{Prop}
Para cada $t\geq0$, la funci\'on generadora de momentos $\esp\left[e^{\alpha N\left(t\right)}\right]$ existe para alguna $\alpha$ en una vecindad del 0, y de aqu\'i que $\esp\left[N\left(t\right)^{m}\right]<\infty$, para $m\geq1$.
\end{Prop}


\begin{Note}
Si el primer tiempo de renovaci\'on $\xi_{1}$ no tiene la misma distribuci\'on que el resto de las $\xi_{n}$, para $n\geq2$, a $N\left(t\right)$ se le llama Proceso de Renovaci\'on retardado, donde si $\xi$ tiene distribuci\'on $G$, entonces el tiempo $T_{n}$ de la $n$-\'esima renovaci\'on tiene distribuci\'on $G\star F^{\left(n-1\right)\star}\left(t\right)$
\end{Note}


\begin{Teo}
Para una constante $\mu\leq\infty$ ( o variable aleatoria), las siguientes expresiones son equivalentes:

\begin{eqnarray}
lim_{n\rightarrow\infty}n^{-1}T_{n}&=&\mu,\textrm{ c.s.}\\
lim_{t\rightarrow\infty}t^{-1}N\left(t\right)&=&1/\mu,\textrm{ c.s.}
\end{eqnarray}
\end{Teo}


Es decir, $T_{n}$ satisface la Ley Fuerte de los Grandes N\'umeros s\'i y s\'olo s\'i $N\left/t\right)$ la cumple.


\begin{Coro}[Ley Fuerte de los Grandes N\'umeros para Procesos de Renovaci\'on]
Si $N\left(t\right)$ es un proceso de renovaci\'on cuyos tiempos de inter-renovaci\'on tienen media $\mu\leq\infty$, entonces
\begin{eqnarray}
t^{-1}N\left(t\right)\rightarrow 1/\mu,\textrm{ c.s. cuando }t\rightarrow\infty.
\end{eqnarray}

\end{Coro}


Considerar el proceso estoc\'astico de valores reales $\left\{Z\left(t\right):t\geq0\right\}$ en el mismo espacio de probabilidad que $N\left(t\right)$

\begin{Def}
Para el proceso $\left\{Z\left(t\right):t\geq0\right\}$ se define la fluctuaci\'on m\'axima de $Z\left(t\right)$ en el intervalo $\left(T_{n-1},T_{n}\right]$:
\begin{eqnarray*}
M_{n}=\sup_{T_{n-1}<t\leq T_{n}}|Z\left(t\right)-Z\left(T_{n-1}\right)|
\end{eqnarray*}
\end{Def}

\begin{Teo}
Sup\'ongase que $n^{-1}T_{n}\rightarrow\mu$ c.s. cuando $n\rightarrow\infty$, donde $\mu\leq\infty$ es una constante o variable aleatoria. Sea $a$ una constante o variable aleatoria que puede ser infinita cuando $\mu$ es finita, y considere las expresiones l\'imite:
\begin{eqnarray}
lim_{n\rightarrow\infty}n^{-1}Z\left(T_{n}\right)&=&a,\textrm{ c.s.}\\
lim_{t\rightarrow\infty}t^{-1}Z\left(t\right)&=&a/\mu,\textrm{ c.s.}
\end{eqnarray}
La segunda expresi\'on implica la primera. Conversamente, la primera implica la segunda si el proceso $Z\left(t\right)$ es creciente, o si $lim_{n\rightarrow\infty}n^{-1}M_{n}=0$ c.s.
\end{Teo}

\begin{Coro}
Si $N\left(t\right)$ es un proceso de renovaci\'on, y $\left(Z\left(T_{n}\right)-Z\left(T_{n-1}\right),M_{n}\right)$, para $n\geq1$, son variables aleatorias independientes e id\'enticamente distribuidas con media finita, entonces,
\begin{eqnarray}
lim_{t\rightarrow\infty}t^{-1}Z\left(t\right)\rightarrow\frac{\esp\left[Z\left(T_{1}\right)-Z\left(T_{0}\right)\right]}{\esp\left[T_{1}\right]},\textrm{ c.s. cuando  }t\rightarrow\infty.
\end{eqnarray}
\end{Coro}


%___________________________________________________________________________________________
%
%\subsection*{Propiedades de los Procesos de Renovaci\'on}
%___________________________________________________________________________________________
%

Los tiempos $T_{n}$ est\'an relacionados con los conteos de $N\left(t\right)$ por

\begin{eqnarray*}
\left\{N\left(t\right)\geq n\right\}&=&\left\{T_{n}\leq t\right\}\\
T_{N\left(t\right)}\leq &t&<T_{N\left(t\right)+1},
\end{eqnarray*}

adem\'as $N\left(T_{n}\right)=n$, y 

\begin{eqnarray*}
N\left(t\right)=\max\left\{n:T_{n}\leq t\right\}=\min\left\{n:T_{n+1}>t\right\}
\end{eqnarray*}

Por propiedades de la convoluci\'on se sabe que

\begin{eqnarray*}
P\left\{T_{n}\leq t\right\}=F^{n\star}\left(t\right)
\end{eqnarray*}
que es la $n$-\'esima convoluci\'on de $F$. Entonces 

\begin{eqnarray*}
\left\{N\left(t\right)\geq n\right\}&=&\left\{T_{n}\leq t\right\}\\
P\left\{N\left(t\right)\leq n\right\}&=&1-F^{\left(n+1\right)\star}\left(t\right)
\end{eqnarray*}

Adem\'as usando el hecho de que $\esp\left[N\left(t\right)\right]=\sum_{n=1}^{\infty}P\left\{N\left(t\right)\geq n\right\}$
se tiene que

\begin{eqnarray*}
\esp\left[N\left(t\right)\right]=\sum_{n=1}^{\infty}F^{n\star}\left(t\right)
\end{eqnarray*}

\begin{Prop}
Para cada $t\geq0$, la funci\'on generadora de momentos $\esp\left[e^{\alpha N\left(t\right)}\right]$ existe para alguna $\alpha$ en una vecindad del 0, y de aqu\'i que $\esp\left[N\left(t\right)^{m}\right]<\infty$, para $m\geq1$.
\end{Prop}


\begin{Note}
Si el primer tiempo de renovaci\'on $\xi_{1}$ no tiene la misma distribuci\'on que el resto de las $\xi_{n}$, para $n\geq2$, a $N\left(t\right)$ se le llama Proceso de Renovaci\'on retardado, donde si $\xi$ tiene distribuci\'on $G$, entonces el tiempo $T_{n}$ de la $n$-\'esima renovaci\'on tiene distribuci\'on $G\star F^{\left(n-1\right)\star}\left(t\right)$
\end{Note}


\begin{Teo}
Para una constante $\mu\leq\infty$ ( o variable aleatoria), las siguientes expresiones son equivalentes:

\begin{eqnarray}
lim_{n\rightarrow\infty}n^{-1}T_{n}&=&\mu,\textrm{ c.s.}\\
lim_{t\rightarrow\infty}t^{-1}N\left(t\right)&=&1/\mu,\textrm{ c.s.}
\end{eqnarray}
\end{Teo}


Es decir, $T_{n}$ satisface la Ley Fuerte de los Grandes N\'umeros s\'i y s\'olo s\'i $N\left/t\right)$ la cumple.


\begin{Coro}[Ley Fuerte de los Grandes N\'umeros para Procesos de Renovaci\'on]
Si $N\left(t\right)$ es un proceso de renovaci\'on cuyos tiempos de inter-renovaci\'on tienen media $\mu\leq\infty$, entonces
\begin{eqnarray}
t^{-1}N\left(t\right)\rightarrow 1/\mu,\textrm{ c.s. cuando }t\rightarrow\infty.
\end{eqnarray}

\end{Coro}


Considerar el proceso estoc\'astico de valores reales $\left\{Z\left(t\right):t\geq0\right\}$ en el mismo espacio de probabilidad que $N\left(t\right)$

\begin{Def}
Para el proceso $\left\{Z\left(t\right):t\geq0\right\}$ se define la fluctuaci\'on m\'axima de $Z\left(t\right)$ en el intervalo $\left(T_{n-1},T_{n}\right]$:
\begin{eqnarray*}
M_{n}=\sup_{T_{n-1}<t\leq T_{n}}|Z\left(t\right)-Z\left(T_{n-1}\right)|
\end{eqnarray*}
\end{Def}

\begin{Teo}
Sup\'ongase que $n^{-1}T_{n}\rightarrow\mu$ c.s. cuando $n\rightarrow\infty$, donde $\mu\leq\infty$ es una constante o variable aleatoria. Sea $a$ una constante o variable aleatoria que puede ser infinita cuando $\mu$ es finita, y considere las expresiones l\'imite:
\begin{eqnarray}
lim_{n\rightarrow\infty}n^{-1}Z\left(T_{n}\right)&=&a,\textrm{ c.s.}\\
lim_{t\rightarrow\infty}t^{-1}Z\left(t\right)&=&a/\mu,\textrm{ c.s.}
\end{eqnarray}
La segunda expresi\'on implica la primera. Conversamente, la primera implica la segunda si el proceso $Z\left(t\right)$ es creciente, o si $lim_{n\rightarrow\infty}n^{-1}M_{n}=0$ c.s.
\end{Teo}

\begin{Coro}
Si $N\left(t\right)$ es un proceso de renovaci\'on, y $\left(Z\left(T_{n}\right)-Z\left(T_{n-1}\right),M_{n}\right)$, para $n\geq1$, son variables aleatorias independientes e id\'enticamente distribuidas con media finita, entonces,
\begin{eqnarray}
lim_{t\rightarrow\infty}t^{-1}Z\left(t\right)\rightarrow\frac{\esp\left[Z\left(T_{1}\right)-Z\left(T_{0}\right)\right]}{\esp\left[T_{1}\right]},\textrm{ c.s. cuando  }t\rightarrow\infty.
\end{eqnarray}
\end{Coro}



%___________________________________________________________________________________________
%
%\subsection*{Propiedades de los Procesos de Renovaci\'on}
%___________________________________________________________________________________________
%

Los tiempos $T_{n}$ est\'an relacionados con los conteos de $N\left(t\right)$ por

\begin{eqnarray*}
\left\{N\left(t\right)\geq n\right\}&=&\left\{T_{n}\leq t\right\}\\
T_{N\left(t\right)}\leq &t&<T_{N\left(t\right)+1},
\end{eqnarray*}

adem\'as $N\left(T_{n}\right)=n$, y 

\begin{eqnarray*}
N\left(t\right)=\max\left\{n:T_{n}\leq t\right\}=\min\left\{n:T_{n+1}>t\right\}
\end{eqnarray*}

Por propiedades de la convoluci\'on se sabe que

\begin{eqnarray*}
P\left\{T_{n}\leq t\right\}=F^{n\star}\left(t\right)
\end{eqnarray*}
que es la $n$-\'esima convoluci\'on de $F$. Entonces 

\begin{eqnarray*}
\left\{N\left(t\right)\geq n\right\}&=&\left\{T_{n}\leq t\right\}\\
P\left\{N\left(t\right)\leq n\right\}&=&1-F^{\left(n+1\right)\star}\left(t\right)
\end{eqnarray*}

Adem\'as usando el hecho de que $\esp\left[N\left(t\right)\right]=\sum_{n=1}^{\infty}P\left\{N\left(t\right)\geq n\right\}$
se tiene que

\begin{eqnarray*}
\esp\left[N\left(t\right)\right]=\sum_{n=1}^{\infty}F^{n\star}\left(t\right)
\end{eqnarray*}

\begin{Prop}
Para cada $t\geq0$, la funci\'on generadora de momentos $\esp\left[e^{\alpha N\left(t\right)}\right]$ existe para alguna $\alpha$ en una vecindad del 0, y de aqu\'i que $\esp\left[N\left(t\right)^{m}\right]<\infty$, para $m\geq1$.
\end{Prop}


\begin{Note}
Si el primer tiempo de renovaci\'on $\xi_{1}$ no tiene la misma distribuci\'on que el resto de las $\xi_{n}$, para $n\geq2$, a $N\left(t\right)$ se le llama Proceso de Renovaci\'on retardado, donde si $\xi$ tiene distribuci\'on $G$, entonces el tiempo $T_{n}$ de la $n$-\'esima renovaci\'on tiene distribuci\'on $G\star F^{\left(n-1\right)\star}\left(t\right)$
\end{Note}


\begin{Teo}
Para una constante $\mu\leq\infty$ ( o variable aleatoria), las siguientes expresiones son equivalentes:

\begin{eqnarray}
lim_{n\rightarrow\infty}n^{-1}T_{n}&=&\mu,\textrm{ c.s.}\\
lim_{t\rightarrow\infty}t^{-1}N\left(t\right)&=&1/\mu,\textrm{ c.s.}
\end{eqnarray}
\end{Teo}


Es decir, $T_{n}$ satisface la Ley Fuerte de los Grandes N\'umeros s\'i y s\'olo s\'i $N\left/t\right)$ la cumple.


\begin{Coro}[Ley Fuerte de los Grandes N\'umeros para Procesos de Renovaci\'on]
Si $N\left(t\right)$ es un proceso de renovaci\'on cuyos tiempos de inter-renovaci\'on tienen media $\mu\leq\infty$, entonces
\begin{eqnarray}
t^{-1}N\left(t\right)\rightarrow 1/\mu,\textrm{ c.s. cuando }t\rightarrow\infty.
\end{eqnarray}

\end{Coro}


Considerar el proceso estoc\'astico de valores reales $\left\{Z\left(t\right):t\geq0\right\}$ en el mismo espacio de probabilidad que $N\left(t\right)$

\begin{Def}
Para el proceso $\left\{Z\left(t\right):t\geq0\right\}$ se define la fluctuaci\'on m\'axima de $Z\left(t\right)$ en el intervalo $\left(T_{n-1},T_{n}\right]$:
\begin{eqnarray*}
M_{n}=\sup_{T_{n-1}<t\leq T_{n}}|Z\left(t\right)-Z\left(T_{n-1}\right)|
\end{eqnarray*}
\end{Def}

\begin{Teo}
Sup\'ongase que $n^{-1}T_{n}\rightarrow\mu$ c.s. cuando $n\rightarrow\infty$, donde $\mu\leq\infty$ es una constante o variable aleatoria. Sea $a$ una constante o variable aleatoria que puede ser infinita cuando $\mu$ es finita, y considere las expresiones l\'imite:
\begin{eqnarray}
lim_{n\rightarrow\infty}n^{-1}Z\left(T_{n}\right)&=&a,\textrm{ c.s.}\\
lim_{t\rightarrow\infty}t^{-1}Z\left(t\right)&=&a/\mu,\textrm{ c.s.}
\end{eqnarray}
La segunda expresi\'on implica la primera. Conversamente, la primera implica la segunda si el proceso $Z\left(t\right)$ es creciente, o si $lim_{n\rightarrow\infty}n^{-1}M_{n}=0$ c.s.
\end{Teo}

\begin{Coro}
Si $N\left(t\right)$ es un proceso de renovaci\'on, y $\left(Z\left(T_{n}\right)-Z\left(T_{n-1}\right),M_{n}\right)$, para $n\geq1$, son variables aleatorias independientes e id\'enticamente distribuidas con media finita, entonces,
\begin{eqnarray}
lim_{t\rightarrow\infty}t^{-1}Z\left(t\right)\rightarrow\frac{\esp\left[Z\left(T_{1}\right)-Z\left(T_{0}\right)\right]}{\esp\left[T_{1}\right]},\textrm{ c.s. cuando  }t\rightarrow\infty.
\end{eqnarray}
\end{Coro}


%___________________________________________________________________________________________
%
%\subsection*{Propiedades de los Procesos de Renovaci\'on}
%___________________________________________________________________________________________
%

Los tiempos $T_{n}$ est\'an relacionados con los conteos de $N\left(t\right)$ por

\begin{eqnarray*}
\left\{N\left(t\right)\geq n\right\}&=&\left\{T_{n}\leq t\right\}\\
T_{N\left(t\right)}\leq &t&<T_{N\left(t\right)+1},
\end{eqnarray*}

adem\'as $N\left(T_{n}\right)=n$, y 

\begin{eqnarray*}
N\left(t\right)=\max\left\{n:T_{n}\leq t\right\}=\min\left\{n:T_{n+1}>t\right\}
\end{eqnarray*}

Por propiedades de la convoluci\'on se sabe que

\begin{eqnarray*}
P\left\{T_{n}\leq t\right\}=F^{n\star}\left(t\right)
\end{eqnarray*}
que es la $n$-\'esima convoluci\'on de $F$. Entonces 

\begin{eqnarray*}
\left\{N\left(t\right)\geq n\right\}&=&\left\{T_{n}\leq t\right\}\\
P\left\{N\left(t\right)\leq n\right\}&=&1-F^{\left(n+1\right)\star}\left(t\right)
\end{eqnarray*}

Adem\'as usando el hecho de que $\esp\left[N\left(t\right)\right]=\sum_{n=1}^{\infty}P\left\{N\left(t\right)\geq n\right\}$
se tiene que

\begin{eqnarray*}
\esp\left[N\left(t\right)\right]=\sum_{n=1}^{\infty}F^{n\star}\left(t\right)
\end{eqnarray*}

\begin{Prop}
Para cada $t\geq0$, la funci\'on generadora de momentos $\esp\left[e^{\alpha N\left(t\right)}\right]$ existe para alguna $\alpha$ en una vecindad del 0, y de aqu\'i que $\esp\left[N\left(t\right)^{m}\right]<\infty$, para $m\geq1$.
\end{Prop}


\begin{Note}
Si el primer tiempo de renovaci\'on $\xi_{1}$ no tiene la misma distribuci\'on que el resto de las $\xi_{n}$, para $n\geq2$, a $N\left(t\right)$ se le llama Proceso de Renovaci\'on retardado, donde si $\xi$ tiene distribuci\'on $G$, entonces el tiempo $T_{n}$ de la $n$-\'esima renovaci\'on tiene distribuci\'on $G\star F^{\left(n-1\right)\star}\left(t\right)$
\end{Note}


\begin{Teo}
Para una constante $\mu\leq\infty$ ( o variable aleatoria), las siguientes expresiones son equivalentes:

\begin{eqnarray}
lim_{n\rightarrow\infty}n^{-1}T_{n}&=&\mu,\textrm{ c.s.}\\
lim_{t\rightarrow\infty}t^{-1}N\left(t\right)&=&1/\mu,\textrm{ c.s.}
\end{eqnarray}
\end{Teo}


Es decir, $T_{n}$ satisface la Ley Fuerte de los Grandes N\'umeros s\'i y s\'olo s\'i $N\left/t\right)$ la cumple.


\begin{Coro}[Ley Fuerte de los Grandes N\'umeros para Procesos de Renovaci\'on]
Si $N\left(t\right)$ es un proceso de renovaci\'on cuyos tiempos de inter-renovaci\'on tienen media $\mu\leq\infty$, entonces
\begin{eqnarray}
t^{-1}N\left(t\right)\rightarrow 1/\mu,\textrm{ c.s. cuando }t\rightarrow\infty.
\end{eqnarray}

\end{Coro}


Considerar el proceso estoc\'astico de valores reales $\left\{Z\left(t\right):t\geq0\right\}$ en el mismo espacio de probabilidad que $N\left(t\right)$

\begin{Def}
Para el proceso $\left\{Z\left(t\right):t\geq0\right\}$ se define la fluctuaci\'on m\'axima de $Z\left(t\right)$ en el intervalo $\left(T_{n-1},T_{n}\right]$:
\begin{eqnarray*}
M_{n}=\sup_{T_{n-1}<t\leq T_{n}}|Z\left(t\right)-Z\left(T_{n-1}\right)|
\end{eqnarray*}
\end{Def}

\begin{Teo}
Sup\'ongase que $n^{-1}T_{n}\rightarrow\mu$ c.s. cuando $n\rightarrow\infty$, donde $\mu\leq\infty$ es una constante o variable aleatoria. Sea $a$ una constante o variable aleatoria que puede ser infinita cuando $\mu$ es finita, y considere las expresiones l\'imite:
\begin{eqnarray}
lim_{n\rightarrow\infty}n^{-1}Z\left(T_{n}\right)&=&a,\textrm{ c.s.}\\
lim_{t\rightarrow\infty}t^{-1}Z\left(t\right)&=&a/\mu,\textrm{ c.s.}
\end{eqnarray}
La segunda expresi\'on implica la primera. Conversamente, la primera implica la segunda si el proceso $Z\left(t\right)$ es creciente, o si $lim_{n\rightarrow\infty}n^{-1}M_{n}=0$ c.s.
\end{Teo}

\begin{Coro}
Si $N\left(t\right)$ es un proceso de renovaci\'on, y $\left(Z\left(T_{n}\right)-Z\left(T_{n-1}\right),M_{n}\right)$, para $n\geq1$, son variables aleatorias independientes e id\'enticamente distribuidas con media finita, entonces,
\begin{eqnarray}
lim_{t\rightarrow\infty}t^{-1}Z\left(t\right)\rightarrow\frac{\esp\left[Z\left(T_{1}\right)-Z\left(T_{0}\right)\right]}{\esp\left[T_{1}\right]},\textrm{ c.s. cuando  }t\rightarrow\infty.
\end{eqnarray}
\end{Coro}

%___________________________________________________________________________________________
%
%\subsection*{Propiedades de los Procesos de Renovaci\'on}
%___________________________________________________________________________________________
%

Los tiempos $T_{n}$ est\'an relacionados con los conteos de $N\left(t\right)$ por

\begin{eqnarray*}
\left\{N\left(t\right)\geq n\right\}&=&\left\{T_{n}\leq t\right\}\\
T_{N\left(t\right)}\leq &t&<T_{N\left(t\right)+1},
\end{eqnarray*}

adem\'as $N\left(T_{n}\right)=n$, y 

\begin{eqnarray*}
N\left(t\right)=\max\left\{n:T_{n}\leq t\right\}=\min\left\{n:T_{n+1}>t\right\}
\end{eqnarray*}

Por propiedades de la convoluci\'on se sabe que

\begin{eqnarray*}
P\left\{T_{n}\leq t\right\}=F^{n\star}\left(t\right)
\end{eqnarray*}
que es la $n$-\'esima convoluci\'on de $F$. Entonces 

\begin{eqnarray*}
\left\{N\left(t\right)\geq n\right\}&=&\left\{T_{n}\leq t\right\}\\
P\left\{N\left(t\right)\leq n\right\}&=&1-F^{\left(n+1\right)\star}\left(t\right)
\end{eqnarray*}

Adem\'as usando el hecho de que $\esp\left[N\left(t\right)\right]=\sum_{n=1}^{\infty}P\left\{N\left(t\right)\geq n\right\}$
se tiene que

\begin{eqnarray*}
\esp\left[N\left(t\right)\right]=\sum_{n=1}^{\infty}F^{n\star}\left(t\right)
\end{eqnarray*}

\begin{Prop}
Para cada $t\geq0$, la funci\'on generadora de momentos $\esp\left[e^{\alpha N\left(t\right)}\right]$ existe para alguna $\alpha$ en una vecindad del 0, y de aqu\'i que $\esp\left[N\left(t\right)^{m}\right]<\infty$, para $m\geq1$.
\end{Prop}


\begin{Note}
Si el primer tiempo de renovaci\'on $\xi_{1}$ no tiene la misma distribuci\'on que el resto de las $\xi_{n}$, para $n\geq2$, a $N\left(t\right)$ se le llama Proceso de Renovaci\'on retardado, donde si $\xi$ tiene distribuci\'on $G$, entonces el tiempo $T_{n}$ de la $n$-\'esima renovaci\'on tiene distribuci\'on $G\star F^{\left(n-1\right)\star}\left(t\right)$
\end{Note}


\begin{Teo}
Para una constante $\mu\leq\infty$ ( o variable aleatoria), las siguientes expresiones son equivalentes:

\begin{eqnarray}
lim_{n\rightarrow\infty}n^{-1}T_{n}&=&\mu,\textrm{ c.s.}\\
lim_{t\rightarrow\infty}t^{-1}N\left(t\right)&=&1/\mu,\textrm{ c.s.}
\end{eqnarray}
\end{Teo}


Es decir, $T_{n}$ satisface la Ley Fuerte de los Grandes N\'umeros s\'i y s\'olo s\'i $N\left/t\right)$ la cumple.


\begin{Coro}[Ley Fuerte de los Grandes N\'umeros para Procesos de Renovaci\'on]
Si $N\left(t\right)$ es un proceso de renovaci\'on cuyos tiempos de inter-renovaci\'on tienen media $\mu\leq\infty$, entonces
\begin{eqnarray}
t^{-1}N\left(t\right)\rightarrow 1/\mu,\textrm{ c.s. cuando }t\rightarrow\infty.
\end{eqnarray}

\end{Coro}


Considerar el proceso estoc\'astico de valores reales $\left\{Z\left(t\right):t\geq0\right\}$ en el mismo espacio de probabilidad que $N\left(t\right)$

\begin{Def}
Para el proceso $\left\{Z\left(t\right):t\geq0\right\}$ se define la fluctuaci\'on m\'axima de $Z\left(t\right)$ en el intervalo $\left(T_{n-1},T_{n}\right]$:
\begin{eqnarray*}
M_{n}=\sup_{T_{n-1}<t\leq T_{n}}|Z\left(t\right)-Z\left(T_{n-1}\right)|
\end{eqnarray*}
\end{Def}

\begin{Teo}
Sup\'ongase que $n^{-1}T_{n}\rightarrow\mu$ c.s. cuando $n\rightarrow\infty$, donde $\mu\leq\infty$ es una constante o variable aleatoria. Sea $a$ una constante o variable aleatoria que puede ser infinita cuando $\mu$ es finita, y considere las expresiones l\'imite:
\begin{eqnarray}
lim_{n\rightarrow\infty}n^{-1}Z\left(T_{n}\right)&=&a,\textrm{ c.s.}\\
lim_{t\rightarrow\infty}t^{-1}Z\left(t\right)&=&a/\mu,\textrm{ c.s.}
\end{eqnarray}
La segunda expresi\'on implica la primera. Conversamente, la primera implica la segunda si el proceso $Z\left(t\right)$ es creciente, o si $lim_{n\rightarrow\infty}n^{-1}M_{n}=0$ c.s.
\end{Teo}

\begin{Coro}
Si $N\left(t\right)$ es un proceso de renovaci\'on, y $\left(Z\left(T_{n}\right)-Z\left(T_{n-1}\right),M_{n}\right)$, para $n\geq1$, son variables aleatorias independientes e id\'enticamente distribuidas con media finita, entonces,
\begin{eqnarray}
lim_{t\rightarrow\infty}t^{-1}Z\left(t\right)\rightarrow\frac{\esp\left[Z\left(T_{1}\right)-Z\left(T_{0}\right)\right]}{\esp\left[T_{1}\right]},\textrm{ c.s. cuando  }t\rightarrow\infty.
\end{eqnarray}
\end{Coro}
%___________________________________________________________________________________________
%
%\subsection*{Propiedades de los Procesos de Renovaci\'on}
%___________________________________________________________________________________________
%

Los tiempos $T_{n}$ est\'an relacionados con los conteos de $N\left(t\right)$ por

\begin{eqnarray*}
\left\{N\left(t\right)\geq n\right\}&=&\left\{T_{n}\leq t\right\}\\
T_{N\left(t\right)}\leq &t&<T_{N\left(t\right)+1},
\end{eqnarray*}

adem\'as $N\left(T_{n}\right)=n$, y 

\begin{eqnarray*}
N\left(t\right)=\max\left\{n:T_{n}\leq t\right\}=\min\left\{n:T_{n+1}>t\right\}
\end{eqnarray*}

Por propiedades de la convoluci\'on se sabe que

\begin{eqnarray*}
P\left\{T_{n}\leq t\right\}=F^{n\star}\left(t\right)
\end{eqnarray*}
que es la $n$-\'esima convoluci\'on de $F$. Entonces 

\begin{eqnarray*}
\left\{N\left(t\right)\geq n\right\}&=&\left\{T_{n}\leq t\right\}\\
P\left\{N\left(t\right)\leq n\right\}&=&1-F^{\left(n+1\right)\star}\left(t\right)
\end{eqnarray*}

Adem\'as usando el hecho de que $\esp\left[N\left(t\right)\right]=\sum_{n=1}^{\infty}P\left\{N\left(t\right)\geq n\right\}$
se tiene que

\begin{eqnarray*}
\esp\left[N\left(t\right)\right]=\sum_{n=1}^{\infty}F^{n\star}\left(t\right)
\end{eqnarray*}

\begin{Prop}
Para cada $t\geq0$, la funci\'on generadora de momentos $\esp\left[e^{\alpha N\left(t\right)}\right]$ existe para alguna $\alpha$ en una vecindad del 0, y de aqu\'i que $\esp\left[N\left(t\right)^{m}\right]<\infty$, para $m\geq1$.
\end{Prop}


\begin{Note}
Si el primer tiempo de renovaci\'on $\xi_{1}$ no tiene la misma distribuci\'on que el resto de las $\xi_{n}$, para $n\geq2$, a $N\left(t\right)$ se le llama Proceso de Renovaci\'on retardado, donde si $\xi$ tiene distribuci\'on $G$, entonces el tiempo $T_{n}$ de la $n$-\'esima renovaci\'on tiene distribuci\'on $G\star F^{\left(n-1\right)\star}\left(t\right)$
\end{Note}


\begin{Teo}
Para una constante $\mu\leq\infty$ ( o variable aleatoria), las siguientes expresiones son equivalentes:

\begin{eqnarray}
lim_{n\rightarrow\infty}n^{-1}T_{n}&=&\mu,\textrm{ c.s.}\\
lim_{t\rightarrow\infty}t^{-1}N\left(t\right)&=&1/\mu,\textrm{ c.s.}
\end{eqnarray}
\end{Teo}


Es decir, $T_{n}$ satisface la Ley Fuerte de los Grandes N\'umeros s\'i y s\'olo s\'i $N\left/t\right)$ la cumple.


\begin{Coro}[Ley Fuerte de los Grandes N\'umeros para Procesos de Renovaci\'on]
Si $N\left(t\right)$ es un proceso de renovaci\'on cuyos tiempos de inter-renovaci\'on tienen media $\mu\leq\infty$, entonces
\begin{eqnarray}
t^{-1}N\left(t\right)\rightarrow 1/\mu,\textrm{ c.s. cuando }t\rightarrow\infty.
\end{eqnarray}

\end{Coro}


Considerar el proceso estoc\'astico de valores reales $\left\{Z\left(t\right):t\geq0\right\}$ en el mismo espacio de probabilidad que $N\left(t\right)$

\begin{Def}
Para el proceso $\left\{Z\left(t\right):t\geq0\right\}$ se define la fluctuaci\'on m\'axima de $Z\left(t\right)$ en el intervalo $\left(T_{n-1},T_{n}\right]$:
\begin{eqnarray*}
M_{n}=\sup_{T_{n-1}<t\leq T_{n}}|Z\left(t\right)-Z\left(T_{n-1}\right)|
\end{eqnarray*}
\end{Def}

\begin{Teo}
Sup\'ongase que $n^{-1}T_{n}\rightarrow\mu$ c.s. cuando $n\rightarrow\infty$, donde $\mu\leq\infty$ es una constante o variable aleatoria. Sea $a$ una constante o variable aleatoria que puede ser infinita cuando $\mu$ es finita, y considere las expresiones l\'imite:
\begin{eqnarray}
lim_{n\rightarrow\infty}n^{-1}Z\left(T_{n}\right)&=&a,\textrm{ c.s.}\\
lim_{t\rightarrow\infty}t^{-1}Z\left(t\right)&=&a/\mu,\textrm{ c.s.}
\end{eqnarray}
La segunda expresi\'on implica la primera. Conversamente, la primera implica la segunda si el proceso $Z\left(t\right)$ es creciente, o si $lim_{n\rightarrow\infty}n^{-1}M_{n}=0$ c.s.
\end{Teo}

\begin{Coro}
Si $N\left(t\right)$ es un proceso de renovaci\'on, y $\left(Z\left(T_{n}\right)-Z\left(T_{n-1}\right),M_{n}\right)$, para $n\geq1$, son variables aleatorias independientes e id\'enticamente distribuidas con media finita, entonces,
\begin{eqnarray}
lim_{t\rightarrow\infty}t^{-1}Z\left(t\right)\rightarrow\frac{\esp\left[Z\left(T_{1}\right)-Z\left(T_{0}\right)\right]}{\esp\left[T_{1}\right]},\textrm{ c.s. cuando  }t\rightarrow\infty.
\end{eqnarray}
\end{Coro}


%___________________________________________________________________________________________
%
%\subsection*{Propiedades de los Procesos de Renovaci\'on}
%___________________________________________________________________________________________
%

Los tiempos $T_{n}$ est\'an relacionados con los conteos de $N\left(t\right)$ por

\begin{eqnarray*}
\left\{N\left(t\right)\geq n\right\}&=&\left\{T_{n}\leq t\right\}\\
T_{N\left(t\right)}\leq &t&<T_{N\left(t\right)+1},
\end{eqnarray*}

adem\'as $N\left(T_{n}\right)=n$, y 

\begin{eqnarray*}
N\left(t\right)=\max\left\{n:T_{n}\leq t\right\}=\min\left\{n:T_{n+1}>t\right\}
\end{eqnarray*}

Por propiedades de la convoluci\'on se sabe que

\begin{eqnarray*}
P\left\{T_{n}\leq t\right\}=F^{n\star}\left(t\right)
\end{eqnarray*}
que es la $n$-\'esima convoluci\'on de $F$. Entonces 

\begin{eqnarray*}
\left\{N\left(t\right)\geq n\right\}&=&\left\{T_{n}\leq t\right\}\\
P\left\{N\left(t\right)\leq n\right\}&=&1-F^{\left(n+1\right)\star}\left(t\right)
\end{eqnarray*}

Adem\'as usando el hecho de que $\esp\left[N\left(t\right)\right]=\sum_{n=1}^{\infty}P\left\{N\left(t\right)\geq n\right\}$
se tiene que

\begin{eqnarray*}
\esp\left[N\left(t\right)\right]=\sum_{n=1}^{\infty}F^{n\star}\left(t\right)
\end{eqnarray*}

\begin{Prop}
Para cada $t\geq0$, la funci\'on generadora de momentos $\esp\left[e^{\alpha N\left(t\right)}\right]$ existe para alguna $\alpha$ en una vecindad del 0, y de aqu\'i que $\esp\left[N\left(t\right)^{m}\right]<\infty$, para $m\geq1$.
\end{Prop}


\begin{Note}
Si el primer tiempo de renovaci\'on $\xi_{1}$ no tiene la misma distribuci\'on que el resto de las $\xi_{n}$, para $n\geq2$, a $N\left(t\right)$ se le llama Proceso de Renovaci\'on retardado, donde si $\xi$ tiene distribuci\'on $G$, entonces el tiempo $T_{n}$ de la $n$-\'esima renovaci\'on tiene distribuci\'on $G\star F^{\left(n-1\right)\star}\left(t\right)$
\end{Note}


\begin{Teo}
Para una constante $\mu\leq\infty$ ( o variable aleatoria), las siguientes expresiones son equivalentes:

\begin{eqnarray}
lim_{n\rightarrow\infty}n^{-1}T_{n}&=&\mu,\textrm{ c.s.}\\
lim_{t\rightarrow\infty}t^{-1}N\left(t\right)&=&1/\mu,\textrm{ c.s.}
\end{eqnarray}
\end{Teo}


Es decir, $T_{n}$ satisface la Ley Fuerte de los Grandes N\'umeros s\'i y s\'olo s\'i $N\left/t\right)$ la cumple.


\begin{Coro}[Ley Fuerte de los Grandes N\'umeros para Procesos de Renovaci\'on]
Si $N\left(t\right)$ es un proceso de renovaci\'on cuyos tiempos de inter-renovaci\'on tienen media $\mu\leq\infty$, entonces
\begin{eqnarray}
t^{-1}N\left(t\right)\rightarrow 1/\mu,\textrm{ c.s. cuando }t\rightarrow\infty.
\end{eqnarray}

\end{Coro}


Considerar el proceso estoc\'astico de valores reales $\left\{Z\left(t\right):t\geq0\right\}$ en el mismo espacio de probabilidad que $N\left(t\right)$

\begin{Def}
Para el proceso $\left\{Z\left(t\right):t\geq0\right\}$ se define la fluctuaci\'on m\'axima de $Z\left(t\right)$ en el intervalo $\left(T_{n-1},T_{n}\right]$:
\begin{eqnarray*}
M_{n}=\sup_{T_{n-1}<t\leq T_{n}}|Z\left(t\right)-Z\left(T_{n-1}\right)|
\end{eqnarray*}
\end{Def}

\begin{Teo}
Sup\'ongase que $n^{-1}T_{n}\rightarrow\mu$ c.s. cuando $n\rightarrow\infty$, donde $\mu\leq\infty$ es una constante o variable aleatoria. Sea $a$ una constante o variable aleatoria que puede ser infinita cuando $\mu$ es finita, y considere las expresiones l\'imite:
\begin{eqnarray}
lim_{n\rightarrow\infty}n^{-1}Z\left(T_{n}\right)&=&a,\textrm{ c.s.}\\
lim_{t\rightarrow\infty}t^{-1}Z\left(t\right)&=&a/\mu,\textrm{ c.s.}
\end{eqnarray}
La segunda expresi\'on implica la primera. Conversamente, la primera implica la segunda si el proceso $Z\left(t\right)$ es creciente, o si $lim_{n\rightarrow\infty}n^{-1}M_{n}=0$ c.s.
\end{Teo}

\begin{Coro}
Si $N\left(t\right)$ es un proceso de renovaci\'on, y $\left(Z\left(T_{n}\right)-Z\left(T_{n-1}\right),M_{n}\right)$, para $n\geq1$, son variables aleatorias independientes e id\'enticamente distribuidas con media finita, entonces,
\begin{eqnarray}
lim_{t\rightarrow\infty}t^{-1}Z\left(t\right)\rightarrow\frac{\esp\left[Z\left(T_{1}\right)-Z\left(T_{0}\right)\right]}{\esp\left[T_{1}\right]},\textrm{ c.s. cuando  }t\rightarrow\infty.
\end{eqnarray}
\end{Coro}



%___________________________________________________________________________________________
%
%\subsection*{Propiedades de los Procesos de Renovaci\'on}
%___________________________________________________________________________________________
%

Los tiempos $T_{n}$ est\'an relacionados con los conteos de $N\left(t\right)$ por

\begin{eqnarray*}
\left\{N\left(t\right)\geq n\right\}&=&\left\{T_{n}\leq t\right\}\\
T_{N\left(t\right)}\leq &t&<T_{N\left(t\right)+1},
\end{eqnarray*}

adem\'as $N\left(T_{n}\right)=n$, y 

\begin{eqnarray*}
N\left(t\right)=\max\left\{n:T_{n}\leq t\right\}=\min\left\{n:T_{n+1}>t\right\}
\end{eqnarray*}

Por propiedades de la convoluci\'on se sabe que

\begin{eqnarray*}
P\left\{T_{n}\leq t\right\}=F^{n\star}\left(t\right)
\end{eqnarray*}
que es la $n$-\'esima convoluci\'on de $F$. Entonces 

\begin{eqnarray*}
\left\{N\left(t\right)\geq n\right\}&=&\left\{T_{n}\leq t\right\}\\
P\left\{N\left(t\right)\leq n\right\}&=&1-F^{\left(n+1\right)\star}\left(t\right)
\end{eqnarray*}

Adem\'as usando el hecho de que $\esp\left[N\left(t\right)\right]=\sum_{n=1}^{\infty}P\left\{N\left(t\right)\geq n\right\}$
se tiene que

\begin{eqnarray*}
\esp\left[N\left(t\right)\right]=\sum_{n=1}^{\infty}F^{n\star}\left(t\right)
\end{eqnarray*}

\begin{Prop}
Para cada $t\geq0$, la funci\'on generadora de momentos $\esp\left[e^{\alpha N\left(t\right)}\right]$ existe para alguna $\alpha$ en una vecindad del 0, y de aqu\'i que $\esp\left[N\left(t\right)^{m}\right]<\infty$, para $m\geq1$.
\end{Prop}


\begin{Note}
Si el primer tiempo de renovaci\'on $\xi_{1}$ no tiene la misma distribuci\'on que el resto de las $\xi_{n}$, para $n\geq2$, a $N\left(t\right)$ se le llama Proceso de Renovaci\'on retardado, donde si $\xi$ tiene distribuci\'on $G$, entonces el tiempo $T_{n}$ de la $n$-\'esima renovaci\'on tiene distribuci\'on $G\star F^{\left(n-1\right)\star}\left(t\right)$
\end{Note}


\begin{Teo}
Para una constante $\mu\leq\infty$ ( o variable aleatoria), las siguientes expresiones son equivalentes:

\begin{eqnarray}
lim_{n\rightarrow\infty}n^{-1}T_{n}&=&\mu,\textrm{ c.s.}\\
lim_{t\rightarrow\infty}t^{-1}N\left(t\right)&=&1/\mu,\textrm{ c.s.}
\end{eqnarray}
\end{Teo}


Es decir, $T_{n}$ satisface la Ley Fuerte de los Grandes N\'umeros s\'i y s\'olo s\'i $N\left/t\right)$ la cumple.


\begin{Coro}[Ley Fuerte de los Grandes N\'umeros para Procesos de Renovaci\'on]
Si $N\left(t\right)$ es un proceso de renovaci\'on cuyos tiempos de inter-renovaci\'on tienen media $\mu\leq\infty$, entonces
\begin{eqnarray}
t^{-1}N\left(t\right)\rightarrow 1/\mu,\textrm{ c.s. cuando }t\rightarrow\infty.
\end{eqnarray}

\end{Coro}


Considerar el proceso estoc\'astico de valores reales $\left\{Z\left(t\right):t\geq0\right\}$ en el mismo espacio de probabilidad que $N\left(t\right)$

\begin{Def}
Para el proceso $\left\{Z\left(t\right):t\geq0\right\}$ se define la fluctuaci\'on m\'axima de $Z\left(t\right)$ en el intervalo $\left(T_{n-1},T_{n}\right]$:
\begin{eqnarray*}
M_{n}=\sup_{T_{n-1}<t\leq T_{n}}|Z\left(t\right)-Z\left(T_{n-1}\right)|
\end{eqnarray*}
\end{Def}

\begin{Teo}
Sup\'ongase que $n^{-1}T_{n}\rightarrow\mu$ c.s. cuando $n\rightarrow\infty$, donde $\mu\leq\infty$ es una constante o variable aleatoria. Sea $a$ una constante o variable aleatoria que puede ser infinita cuando $\mu$ es finita, y considere las expresiones l\'imite:
\begin{eqnarray}
lim_{n\rightarrow\infty}n^{-1}Z\left(T_{n}\right)&=&a,\textrm{ c.s.}\\
lim_{t\rightarrow\infty}t^{-1}Z\left(t\right)&=&a/\mu,\textrm{ c.s.}
\end{eqnarray}
La segunda expresi\'on implica la primera. Conversamente, la primera implica la segunda si el proceso $Z\left(t\right)$ es creciente, o si $lim_{n\rightarrow\infty}n^{-1}M_{n}=0$ c.s.
\end{Teo}

\begin{Coro}
Si $N\left(t\right)$ es un proceso de renovaci\'on, y $\left(Z\left(T_{n}\right)-Z\left(T_{n-1}\right),M_{n}\right)$, para $n\geq1$, son variables aleatorias independientes e id\'enticamente distribuidas con media finita, entonces,
\begin{eqnarray}
lim_{t\rightarrow\infty}t^{-1}Z\left(t\right)\rightarrow\frac{\esp\left[Z\left(T_{1}\right)-Z\left(T_{0}\right)\right]}{\esp\left[T_{1}\right]},\textrm{ c.s. cuando  }t\rightarrow\infty.
\end{eqnarray}
\end{Coro}


%___________________________________________________________________________________________
%
%\subsection*{Propiedades de los Procesos de Renovaci\'on}
%___________________________________________________________________________________________
%

Los tiempos $T_{n}$ est\'an relacionados con los conteos de $N\left(t\right)$ por

\begin{eqnarray*}
\left\{N\left(t\right)\geq n\right\}&=&\left\{T_{n}\leq t\right\}\\
T_{N\left(t\right)}\leq &t&<T_{N\left(t\right)+1},
\end{eqnarray*}

adem\'as $N\left(T_{n}\right)=n$, y 

\begin{eqnarray*}
N\left(t\right)=\max\left\{n:T_{n}\leq t\right\}=\min\left\{n:T_{n+1}>t\right\}
\end{eqnarray*}

Por propiedades de la convoluci\'on se sabe que

\begin{eqnarray*}
P\left\{T_{n}\leq t\right\}=F^{n\star}\left(t\right)
\end{eqnarray*}
que es la $n$-\'esima convoluci\'on de $F$. Entonces 

\begin{eqnarray*}
\left\{N\left(t\right)\geq n\right\}&=&\left\{T_{n}\leq t\right\}\\
P\left\{N\left(t\right)\leq n\right\}&=&1-F^{\left(n+1\right)\star}\left(t\right)
\end{eqnarray*}

Adem\'as usando el hecho de que $\esp\left[N\left(t\right)\right]=\sum_{n=1}^{\infty}P\left\{N\left(t\right)\geq n\right\}$
se tiene que

\begin{eqnarray*}
\esp\left[N\left(t\right)\right]=\sum_{n=1}^{\infty}F^{n\star}\left(t\right)
\end{eqnarray*}

\begin{Prop}
Para cada $t\geq0$, la funci\'on generadora de momentos $\esp\left[e^{\alpha N\left(t\right)}\right]$ existe para alguna $\alpha$ en una vecindad del 0, y de aqu\'i que $\esp\left[N\left(t\right)^{m}\right]<\infty$, para $m\geq1$.
\end{Prop}


\begin{Note}
Si el primer tiempo de renovaci\'on $\xi_{1}$ no tiene la misma distribuci\'on que el resto de las $\xi_{n}$, para $n\geq2$, a $N\left(t\right)$ se le llama Proceso de Renovaci\'on retardado, donde si $\xi$ tiene distribuci\'on $G$, entonces el tiempo $T_{n}$ de la $n$-\'esima renovaci\'on tiene distribuci\'on $G\star F^{\left(n-1\right)\star}\left(t\right)$
\end{Note}


\begin{Teo}
Para una constante $\mu\leq\infty$ ( o variable aleatoria), las siguientes expresiones son equivalentes:

\begin{eqnarray}
lim_{n\rightarrow\infty}n^{-1}T_{n}&=&\mu,\textrm{ c.s.}\\
lim_{t\rightarrow\infty}t^{-1}N\left(t\right)&=&1/\mu,\textrm{ c.s.}
\end{eqnarray}
\end{Teo}


Es decir, $T_{n}$ satisface la Ley Fuerte de los Grandes N\'umeros s\'i y s\'olo s\'i $N\left/t\right)$ la cumple.


\begin{Coro}[Ley Fuerte de los Grandes N\'umeros para Procesos de Renovaci\'on]
Si $N\left(t\right)$ es un proceso de renovaci\'on cuyos tiempos de inter-renovaci\'on tienen media $\mu\leq\infty$, entonces
\begin{eqnarray}
t^{-1}N\left(t\right)\rightarrow 1/\mu,\textrm{ c.s. cuando }t\rightarrow\infty.
\end{eqnarray}

\end{Coro}


Considerar el proceso estoc\'astico de valores reales $\left\{Z\left(t\right):t\geq0\right\}$ en el mismo espacio de probabilidad que $N\left(t\right)$

\begin{Def}
Para el proceso $\left\{Z\left(t\right):t\geq0\right\}$ se define la fluctuaci\'on m\'axima de $Z\left(t\right)$ en el intervalo $\left(T_{n-1},T_{n}\right]$:
\begin{eqnarray*}
M_{n}=\sup_{T_{n-1}<t\leq T_{n}}|Z\left(t\right)-Z\left(T_{n-1}\right)|
\end{eqnarray*}
\end{Def}

\begin{Teo}
Sup\'ongase que $n^{-1}T_{n}\rightarrow\mu$ c.s. cuando $n\rightarrow\infty$, donde $\mu\leq\infty$ es una constante o variable aleatoria. Sea $a$ una constante o variable aleatoria que puede ser infinita cuando $\mu$ es finita, y considere las expresiones l\'imite:
\begin{eqnarray}
lim_{n\rightarrow\infty}n^{-1}Z\left(T_{n}\right)&=&a,\textrm{ c.s.}\\
lim_{t\rightarrow\infty}t^{-1}Z\left(t\right)&=&a/\mu,\textrm{ c.s.}
\end{eqnarray}
La segunda expresi\'on implica la primera. Conversamente, la primera implica la segunda si el proceso $Z\left(t\right)$ es creciente, o si $lim_{n\rightarrow\infty}n^{-1}M_{n}=0$ c.s.
\end{Teo}

\begin{Coro}
Si $N\left(t\right)$ es un proceso de renovaci\'on, y $\left(Z\left(T_{n}\right)-Z\left(T_{n-1}\right),M_{n}\right)$, para $n\geq1$, son variables aleatorias independientes e id\'enticamente distribuidas con media finita, entonces,
\begin{eqnarray}
lim_{t\rightarrow\infty}t^{-1}Z\left(t\right)\rightarrow\frac{\esp\left[Z\left(T_{1}\right)-Z\left(T_{0}\right)\right]}{\esp\left[T_{1}\right]},\textrm{ c.s. cuando  }t\rightarrow\infty.
\end{eqnarray}
\end{Coro}

%___________________________________________________________________________________________
%
%\subsection*{Propiedades de los Procesos de Renovaci\'on}
%___________________________________________________________________________________________
%

Los tiempos $T_{n}$ est\'an relacionados con los conteos de $N\left(t\right)$ por

\begin{eqnarray*}
\left\{N\left(t\right)\geq n\right\}&=&\left\{T_{n}\leq t\right\}\\
T_{N\left(t\right)}\leq &t&<T_{N\left(t\right)+1},
\end{eqnarray*}

adem\'as $N\left(T_{n}\right)=n$, y 

\begin{eqnarray*}
N\left(t\right)=\max\left\{n:T_{n}\leq t\right\}=\min\left\{n:T_{n+1}>t\right\}
\end{eqnarray*}

Por propiedades de la convoluci\'on se sabe que

\begin{eqnarray*}
P\left\{T_{n}\leq t\right\}=F^{n\star}\left(t\right)
\end{eqnarray*}
que es la $n$-\'esima convoluci\'on de $F$. Entonces 

\begin{eqnarray*}
\left\{N\left(t\right)\geq n\right\}&=&\left\{T_{n}\leq t\right\}\\
P\left\{N\left(t\right)\leq n\right\}&=&1-F^{\left(n+1\right)\star}\left(t\right)
\end{eqnarray*}

Adem\'as usando el hecho de que $\esp\left[N\left(t\right)\right]=\sum_{n=1}^{\infty}P\left\{N\left(t\right)\geq n\right\}$
se tiene que

\begin{eqnarray*}
\esp\left[N\left(t\right)\right]=\sum_{n=1}^{\infty}F^{n\star}\left(t\right)
\end{eqnarray*}

\begin{Prop}
Para cada $t\geq0$, la funci\'on generadora de momentos $\esp\left[e^{\alpha N\left(t\right)}\right]$ existe para alguna $\alpha$ en una vecindad del 0, y de aqu\'i que $\esp\left[N\left(t\right)^{m}\right]<\infty$, para $m\geq1$.
\end{Prop}


\begin{Note}
Si el primer tiempo de renovaci\'on $\xi_{1}$ no tiene la misma distribuci\'on que el resto de las $\xi_{n}$, para $n\geq2$, a $N\left(t\right)$ se le llama Proceso de Renovaci\'on retardado, donde si $\xi$ tiene distribuci\'on $G$, entonces el tiempo $T_{n}$ de la $n$-\'esima renovaci\'on tiene distribuci\'on $G\star F^{\left(n-1\right)\star}\left(t\right)$
\end{Note}


\begin{Teo}
Para una constante $\mu\leq\infty$ ( o variable aleatoria), las siguientes expresiones son equivalentes:

\begin{eqnarray}
lim_{n\rightarrow\infty}n^{-1}T_{n}&=&\mu,\textrm{ c.s.}\\
lim_{t\rightarrow\infty}t^{-1}N\left(t\right)&=&1/\mu,\textrm{ c.s.}
\end{eqnarray}
\end{Teo}


Es decir, $T_{n}$ satisface la Ley Fuerte de los Grandes N\'umeros s\'i y s\'olo s\'i $N\left/t\right)$ la cumple.


\begin{Coro}[Ley Fuerte de los Grandes N\'umeros para Procesos de Renovaci\'on]
Si $N\left(t\right)$ es un proceso de renovaci\'on cuyos tiempos de inter-renovaci\'on tienen media $\mu\leq\infty$, entonces
\begin{eqnarray}
t^{-1}N\left(t\right)\rightarrow 1/\mu,\textrm{ c.s. cuando }t\rightarrow\infty.
\end{eqnarray}

\end{Coro}


Considerar el proceso estoc\'astico de valores reales $\left\{Z\left(t\right):t\geq0\right\}$ en el mismo espacio de probabilidad que $N\left(t\right)$

\begin{Def}
Para el proceso $\left\{Z\left(t\right):t\geq0\right\}$ se define la fluctuaci\'on m\'axima de $Z\left(t\right)$ en el intervalo $\left(T_{n-1},T_{n}\right]$:
\begin{eqnarray*}
M_{n}=\sup_{T_{n-1}<t\leq T_{n}}|Z\left(t\right)-Z\left(T_{n-1}\right)|
\end{eqnarray*}
\end{Def}

\begin{Teo}
Sup\'ongase que $n^{-1}T_{n}\rightarrow\mu$ c.s. cuando $n\rightarrow\infty$, donde $\mu\leq\infty$ es una constante o variable aleatoria. Sea $a$ una constante o variable aleatoria que puede ser infinita cuando $\mu$ es finita, y considere las expresiones l\'imite:
\begin{eqnarray}
lim_{n\rightarrow\infty}n^{-1}Z\left(T_{n}\right)&=&a,\textrm{ c.s.}\\
lim_{t\rightarrow\infty}t^{-1}Z\left(t\right)&=&a/\mu,\textrm{ c.s.}
\end{eqnarray}
La segunda expresi\'on implica la primera. Conversamente, la primera implica la segunda si el proceso $Z\left(t\right)$ es creciente, o si $lim_{n\rightarrow\infty}n^{-1}M_{n}=0$ c.s.
\end{Teo}

\begin{Coro}
Si $N\left(t\right)$ es un proceso de renovaci\'on, y $\left(Z\left(T_{n}\right)-Z\left(T_{n-1}\right),M_{n}\right)$, para $n\geq1$, son variables aleatorias independientes e id\'enticamente distribuidas con media finita, entonces,
\begin{eqnarray}
lim_{t\rightarrow\infty}t^{-1}Z\left(t\right)\rightarrow\frac{\esp\left[Z\left(T_{1}\right)-Z\left(T_{0}\right)\right]}{\esp\left[T_{1}\right]},\textrm{ c.s. cuando  }t\rightarrow\infty.
\end{eqnarray}
\end{Coro}
%___________________________________________________________________________________________
%
%\subsection*{Propiedades de los Procesos de Renovaci\'on}
%___________________________________________________________________________________________
%

Los tiempos $T_{n}$ est\'an relacionados con los conteos de $N\left(t\right)$ por

\begin{eqnarray*}
\left\{N\left(t\right)\geq n\right\}&=&\left\{T_{n}\leq t\right\}\\
T_{N\left(t\right)}\leq &t&<T_{N\left(t\right)+1},
\end{eqnarray*}

adem\'as $N\left(T_{n}\right)=n$, y 

\begin{eqnarray*}
N\left(t\right)=\max\left\{n:T_{n}\leq t\right\}=\min\left\{n:T_{n+1}>t\right\}
\end{eqnarray*}

Por propiedades de la convoluci\'on se sabe que

\begin{eqnarray*}
P\left\{T_{n}\leq t\right\}=F^{n\star}\left(t\right)
\end{eqnarray*}
que es la $n$-\'esima convoluci\'on de $F$. Entonces 

\begin{eqnarray*}
\left\{N\left(t\right)\geq n\right\}&=&\left\{T_{n}\leq t\right\}\\
P\left\{N\left(t\right)\leq n\right\}&=&1-F^{\left(n+1\right)\star}\left(t\right)
\end{eqnarray*}

Adem\'as usando el hecho de que $\esp\left[N\left(t\right)\right]=\sum_{n=1}^{\infty}P\left\{N\left(t\right)\geq n\right\}$
se tiene que

\begin{eqnarray*}
\esp\left[N\left(t\right)\right]=\sum_{n=1}^{\infty}F^{n\star}\left(t\right)
\end{eqnarray*}

\begin{Prop}
Para cada $t\geq0$, la funci\'on generadora de momentos $\esp\left[e^{\alpha N\left(t\right)}\right]$ existe para alguna $\alpha$ en una vecindad del 0, y de aqu\'i que $\esp\left[N\left(t\right)^{m}\right]<\infty$, para $m\geq1$.
\end{Prop}


\begin{Note}
Si el primer tiempo de renovaci\'on $\xi_{1}$ no tiene la misma distribuci\'on que el resto de las $\xi_{n}$, para $n\geq2$, a $N\left(t\right)$ se le llama Proceso de Renovaci\'on retardado, donde si $\xi$ tiene distribuci\'on $G$, entonces el tiempo $T_{n}$ de la $n$-\'esima renovaci\'on tiene distribuci\'on $G\star F^{\left(n-1\right)\star}\left(t\right)$
\end{Note}


\begin{Teo}
Para una constante $\mu\leq\infty$ ( o variable aleatoria), las siguientes expresiones son equivalentes:

\begin{eqnarray}
lim_{n\rightarrow\infty}n^{-1}T_{n}&=&\mu,\textrm{ c.s.}\\
lim_{t\rightarrow\infty}t^{-1}N\left(t\right)&=&1/\mu,\textrm{ c.s.}
\end{eqnarray}
\end{Teo}


Es decir, $T_{n}$ satisface la Ley Fuerte de los Grandes N\'umeros s\'i y s\'olo s\'i $N\left/t\right)$ la cumple.


\begin{Coro}[Ley Fuerte de los Grandes N\'umeros para Procesos de Renovaci\'on]
Si $N\left(t\right)$ es un proceso de renovaci\'on cuyos tiempos de inter-renovaci\'on tienen media $\mu\leq\infty$, entonces
\begin{eqnarray}
t^{-1}N\left(t\right)\rightarrow 1/\mu,\textrm{ c.s. cuando }t\rightarrow\infty.
\end{eqnarray}

\end{Coro}


Considerar el proceso estoc\'astico de valores reales $\left\{Z\left(t\right):t\geq0\right\}$ en el mismo espacio de probabilidad que $N\left(t\right)$

\begin{Def}
Para el proceso $\left\{Z\left(t\right):t\geq0\right\}$ se define la fluctuaci\'on m\'axima de $Z\left(t\right)$ en el intervalo $\left(T_{n-1},T_{n}\right]$:
\begin{eqnarray*}
M_{n}=\sup_{T_{n-1}<t\leq T_{n}}|Z\left(t\right)-Z\left(T_{n-1}\right)|
\end{eqnarray*}
\end{Def}

\begin{Teo}
Sup\'ongase que $n^{-1}T_{n}\rightarrow\mu$ c.s. cuando $n\rightarrow\infty$, donde $\mu\leq\infty$ es una constante o variable aleatoria. Sea $a$ una constante o variable aleatoria que puede ser infinita cuando $\mu$ es finita, y considere las expresiones l\'imite:
\begin{eqnarray}
lim_{n\rightarrow\infty}n^{-1}Z\left(T_{n}\right)&=&a,\textrm{ c.s.}\\
lim_{t\rightarrow\infty}t^{-1}Z\left(t\right)&=&a/\mu,\textrm{ c.s.}
\end{eqnarray}
La segunda expresi\'on implica la primera. Conversamente, la primera implica la segunda si el proceso $Z\left(t\right)$ es creciente, o si $lim_{n\rightarrow\infty}n^{-1}M_{n}=0$ c.s.
\end{Teo}

\begin{Coro}
Si $N\left(t\right)$ es un proceso de renovaci\'on, y $\left(Z\left(T_{n}\right)-Z\left(T_{n-1}\right),M_{n}\right)$, para $n\geq1$, son variables aleatorias independientes e id\'enticamente distribuidas con media finita, entonces,
\begin{eqnarray}
lim_{t\rightarrow\infty}t^{-1}Z\left(t\right)\rightarrow\frac{\esp\left[Z\left(T_{1}\right)-Z\left(T_{0}\right)\right]}{\esp\left[T_{1}\right]},\textrm{ c.s. cuando  }t\rightarrow\infty.
\end{eqnarray}
\end{Coro}


%___________________________________________________________________________________________
%
\section{Funci\'on de Renovaci\'on}
%___________________________________________________________________________________________
%


\begin{Def}
Sea $h\left(t\right)$ funci\'on de valores reales en $\rea$ acotada en intervalos finitos e igual a cero para $t<0$ La ecuaci\'on de renovaci\'on para $h\left(t\right)$ y la distribuci\'on $F$ es

\begin{eqnarray}\label{Ec.Renovacion}
H\left(t\right)=h\left(t\right)+\int_{\left[0,t\right]}H\left(t-s\right)dF\left(s\right)\textrm{,    }t\geq0,
\end{eqnarray}
donde $H\left(t\right)$ es una funci\'on de valores reales. Esto es $H=h+F\star H$. Decimos que $H\left(t\right)$ es soluci\'on de esta ecuaci\'on si satisface la ecuaci\'on, y es acotada en intervalos finitos e iguales a cero para $t<0$.
\end{Def}

\begin{Prop}
La funci\'on $U\star h\left(t\right)$ es la \'unica soluci\'on de la ecuaci\'on de renovaci\'on (\ref{Ec.Renovacion}).
\end{Prop}

\begin{Teo}[Teorema Renovaci\'on Elemental]
\begin{eqnarray*}
t^{-1}U\left(t\right)\rightarrow 1/\mu\textrm{,    cuando }t\rightarrow\infty.
\end{eqnarray*}
\end{Teo}

%___________________________________________________________________________________________
%
%\subsection*{Funci\'on de Renovaci\'on}
%___________________________________________________________________________________________
%


Sup\'ongase que $N\left(t\right)$ es un proceso de renovaci\'on con distribuci\'on $F$ con media finita $\mu$.

\begin{Def}
La funci\'on de renovaci\'on asociada con la distribuci\'on $F$, del proceso $N\left(t\right)$, es
\begin{eqnarray*}
U\left(t\right)=\sum_{n=1}^{\infty}F^{n\star}\left(t\right),\textrm{   }t\geq0,
\end{eqnarray*}
donde $F^{0\star}\left(t\right)=\indora\left(t\geq0\right)$.
\end{Def}


\begin{Prop}
Sup\'ongase que la distribuci\'on de inter-renovaci\'on $F$ tiene densidad $f$. Entonces $U\left(t\right)$ tambi\'en tiene densidad, para $t>0$, y es $U^{'}\left(t\right)=\sum_{n=0}^{\infty}f^{n\star}\left(t\right)$. Adem\'as
\begin{eqnarray*}
\prob\left\{N\left(t\right)>N\left(t-\right)\right\}=0\textrm{,   }t\geq0.
\end{eqnarray*}
\end{Prop}

\begin{Def}
La Transformada de Laplace-Stieljes de $F$ est\'a dada por

\begin{eqnarray*}
\hat{F}\left(\alpha\right)=\int_{\rea_{+}}e^{-\alpha t}dF\left(t\right)\textrm{,  }\alpha\geq0.
\end{eqnarray*}
\end{Def}

Entonces

\begin{eqnarray*}
\hat{U}\left(\alpha\right)=\sum_{n=0}^{\infty}\hat{F^{n\star}}\left(\alpha\right)=\sum_{n=0}^{\infty}\hat{F}\left(\alpha\right)^{n}=\frac{1}{1-\hat{F}\left(\alpha\right)}.
\end{eqnarray*}


\begin{Prop}
La Transformada de Laplace $\hat{U}\left(\alpha\right)$ y $\hat{F}\left(\alpha\right)$ determina una a la otra de manera \'unica por la relaci\'on $\hat{U}\left(\alpha\right)=\frac{1}{1-\hat{F}\left(\alpha\right)}$.
\end{Prop}


\begin{Note}
Un proceso de renovaci\'on $N\left(t\right)$ cuyos tiempos de inter-renovaci\'on tienen media finita, es un proceso Poisson con tasa $\lambda$ si y s\'olo s\'i $\esp\left[U\left(t\right)\right]=\lambda t$, para $t\geq0$.
\end{Note}


\begin{Teo}
Sea $N\left(t\right)$ un proceso puntual simple con puntos de localizaci\'on $T_{n}$ tal que $\eta\left(t\right)=\esp\left[N\left(\right)\right]$ es finita para cada $t$. Entonces para cualquier funci\'on $f:\rea_{+}\rightarrow\rea$,
\begin{eqnarray*}
\esp\left[\sum_{n=1}^{N\left(\right)}f\left(T_{n}\right)\right]=\int_{\left(0,t\right]}f\left(s\right)d\eta\left(s\right)\textrm{,  }t\geq0,
\end{eqnarray*}
suponiendo que la integral exista. Adem\'as si $X_{1},X_{2},\ldots$ son variables aleatorias definidas en el mismo espacio de probabilidad que el proceso $N\left(t\right)$ tal que $\esp\left[X_{n}|T_{n}=s\right]=f\left(s\right)$, independiente de $n$. Entonces
\begin{eqnarray*}
\esp\left[\sum_{n=1}^{N\left(t\right)}X_{n}\right]=\int_{\left(0,t\right]}f\left(s\right)d\eta\left(s\right)\textrm{,  }t\geq0,
\end{eqnarray*} 
suponiendo que la integral exista. 
\end{Teo}

\begin{Coro}[Identidad de Wald para Renovaciones]
Para el proceso de renovaci\'on $N\left(t\right)$,
\begin{eqnarray*}
\esp\left[T_{N\left(t\right)+1}\right]=\mu\esp\left[N\left(t\right)+1\right]\textrm{,  }t\geq0,
\end{eqnarray*}  
\end{Coro}



%___________________________________________________________________________________________
%
%\subsection*{Funci\'on de Renovaci\'on}
%___________________________________________________________________________________________
%


\begin{Def}
Sea $h\left(t\right)$ funci\'on de valores reales en $\rea$ acotada en intervalos finitos e igual a cero para $t<0$ La ecuaci\'on de renovaci\'on para $h\left(t\right)$ y la distribuci\'on $F$ es

\begin{eqnarray}%\label{Ec.Renovacion}
H\left(t\right)=h\left(t\right)+\int_{\left[0,t\right]}H\left(t-s\right)dF\left(s\right)\textrm{,    }t\geq0,
\end{eqnarray}
donde $H\left(t\right)$ es una funci\'on de valores reales. Esto es $H=h+F\star H$. Decimos que $H\left(t\right)$ es soluci\'on de esta ecuaci\'on si satisface la ecuaci\'on, y es acotada en intervalos finitos e iguales a cero para $t<0$.
\end{Def}

\begin{Prop}
La funci\'on $U\star h\left(t\right)$ es la \'unica soluci\'on de la ecuaci\'on de renovaci\'on (\ref{Ec.Renovacion}).
\end{Prop}

\begin{Teo}[Teorema Renovaci\'on Elemental]
\begin{eqnarray*}
t^{-1}U\left(t\right)\rightarrow 1/\mu\textrm{,    cuando }t\rightarrow\infty.
\end{eqnarray*}
\end{Teo}

%___________________________________________________________________________________________
%
%\subsection*{Funci\'on de Renovaci\'on}
%___________________________________________________________________________________________
%


Sup\'ongase que $N\left(t\right)$ es un proceso de renovaci\'on con distribuci\'on $F$ con media finita $\mu$.

\begin{Def}
La funci\'on de renovaci\'on asociada con la distribuci\'on $F$, del proceso $N\left(t\right)$, es
\begin{eqnarray*}
U\left(t\right)=\sum_{n=1}^{\infty}F^{n\star}\left(t\right),\textrm{   }t\geq0,
\end{eqnarray*}
donde $F^{0\star}\left(t\right)=\indora\left(t\geq0\right)$.
\end{Def}


\begin{Prop}
Sup\'ongase que la distribuci\'on de inter-renovaci\'on $F$ tiene densidad $f$. Entonces $U\left(t\right)$ tambi\'en tiene densidad, para $t>0$, y es $U^{'}\left(t\right)=\sum_{n=0}^{\infty}f^{n\star}\left(t\right)$. Adem\'as
\begin{eqnarray*}
\prob\left\{N\left(t\right)>N\left(t-\right)\right\}=0\textrm{,   }t\geq0.
\end{eqnarray*}
\end{Prop}

\begin{Def}
La Transformada de Laplace-Stieljes de $F$ est\'a dada por

\begin{eqnarray*}
\hat{F}\left(\alpha\right)=\int_{\rea_{+}}e^{-\alpha t}dF\left(t\right)\textrm{,  }\alpha\geq0.
\end{eqnarray*}
\end{Def}

Entonces

\begin{eqnarray*}
\hat{U}\left(\alpha\right)=\sum_{n=0}^{\infty}\hat{F^{n\star}}\left(\alpha\right)=\sum_{n=0}^{\infty}\hat{F}\left(\alpha\right)^{n}=\frac{1}{1-\hat{F}\left(\alpha\right)}.
\end{eqnarray*}


\begin{Prop}
La Transformada de Laplace $\hat{U}\left(\alpha\right)$ y $\hat{F}\left(\alpha\right)$ determina una a la otra de manera \'unica por la relaci\'on $\hat{U}\left(\alpha\right)=\frac{1}{1-\hat{F}\left(\alpha\right)}$.
\end{Prop}


\begin{Note}
Un proceso de renovaci\'on $N\left(t\right)$ cuyos tiempos de inter-renovaci\'on tienen media finita, es un proceso Poisson con tasa $\lambda$ si y s\'olo s\'i $\esp\left[U\left(t\right)\right]=\lambda t$, para $t\geq0$.
\end{Note}


\begin{Teo}
Sea $N\left(t\right)$ un proceso puntual simple con puntos de localizaci\'on $T_{n}$ tal que $\eta\left(t\right)=\esp\left[N\left(\right)\right]$ es finita para cada $t$. Entonces para cualquier funci\'on $f:\rea_{+}\rightarrow\rea$,
\begin{eqnarray*}
\esp\left[\sum_{n=1}^{N\left(\right)}f\left(T_{n}\right)\right]=\int_{\left(0,t\right]}f\left(s\right)d\eta\left(s\right)\textrm{,  }t\geq0,
\end{eqnarray*}
suponiendo que la integral exista. Adem\'as si $X_{1},X_{2},\ldots$ son variables aleatorias definidas en el mismo espacio de probabilidad que el proceso $N\left(t\right)$ tal que $\esp\left[X_{n}|T_{n}=s\right]=f\left(s\right)$, independiente de $n$. Entonces
\begin{eqnarray*}
\esp\left[\sum_{n=1}^{N\left(t\right)}X_{n}\right]=\int_{\left(0,t\right]}f\left(s\right)d\eta\left(s\right)\textrm{,  }t\geq0,
\end{eqnarray*} 
suponiendo que la integral exista. 
\end{Teo}

\begin{Coro}[Identidad de Wald para Renovaciones]
Para el proceso de renovaci\'on $N\left(t\right)$,
\begin{eqnarray*}
\esp\left[T_{N\left(t\right)+1}\right]=\mu\esp\left[N\left(t\right)+1\right]\textrm{,  }t\geq0,
\end{eqnarray*}  
\end{Coro}


%___________________________________________________________________________________________
%
%\subsection*{Funci\'on de Renovaci\'on}
%___________________________________________________________________________________________
%


\begin{Def}
Sea $h\left(t\right)$ funci\'on de valores reales en $\rea$ acotada en intervalos finitos e igual a cero para $t<0$ La ecuaci\'on de renovaci\'on para $h\left(t\right)$ y la distribuci\'on $F$ es

\begin{eqnarray}\label{Ec.Renovacion}
H\left(t\right)=h\left(t\right)+\int_{\left[0,t\right]}H\left(t-s\right)dF\left(s\right)\textrm{,    }t\geq0,
\end{eqnarray}
donde $H\left(t\right)$ es una funci\'on de valores reales. Esto es $H=h+F\star H$. Decimos que $H\left(t\right)$ es soluci\'on de esta ecuaci\'on si satisface la ecuaci\'on, y es acotada en intervalos finitos e iguales a cero para $t<0$.
\end{Def}

\begin{Prop}
La funci\'on $U\star h\left(t\right)$ es la \'unica soluci\'on de la ecuaci\'on de renovaci\'on (\ref{Ec.Renovacion}).
\end{Prop}

\begin{Teo}[Teorema Renovaci\'on Elemental]
\begin{eqnarray*}
t^{-1}U\left(t\right)\rightarrow 1/\mu\textrm{,    cuando }t\rightarrow\infty.
\end{eqnarray*}
\end{Teo}

%___________________________________________________________________________________________
%
%\subsection*{Funci\'on de Renovaci\'on}
%___________________________________________________________________________________________
%


Sup\'ongase que $N\left(t\right)$ es un proceso de renovaci\'on con distribuci\'on $F$ con media finita $\mu$.

\begin{Def}
La funci\'on de renovaci\'on asociada con la distribuci\'on $F$, del proceso $N\left(t\right)$, es
\begin{eqnarray*}
U\left(t\right)=\sum_{n=1}^{\infty}F^{n\star}\left(t\right),\textrm{   }t\geq0,
\end{eqnarray*}
donde $F^{0\star}\left(t\right)=\indora\left(t\geq0\right)$.
\end{Def}


\begin{Prop}
Sup\'ongase que la distribuci\'on de inter-renovaci\'on $F$ tiene densidad $f$. Entonces $U\left(t\right)$ tambi\'en tiene densidad, para $t>0$, y es $U^{'}\left(t\right)=\sum_{n=0}^{\infty}f^{n\star}\left(t\right)$. Adem\'as
\begin{eqnarray*}
\prob\left\{N\left(t\right)>N\left(t-\right)\right\}=0\textrm{,   }t\geq0.
\end{eqnarray*}
\end{Prop}

\begin{Def}
La Transformada de Laplace-Stieljes de $F$ est\'a dada por

\begin{eqnarray*}
\hat{F}\left(\alpha\right)=\int_{\rea_{+}}e^{-\alpha t}dF\left(t\right)\textrm{,  }\alpha\geq0.
\end{eqnarray*}
\end{Def}

Entonces

\begin{eqnarray*}
\hat{U}\left(\alpha\right)=\sum_{n=0}^{\infty}\hat{F^{n\star}}\left(\alpha\right)=\sum_{n=0}^{\infty}\hat{F}\left(\alpha\right)^{n}=\frac{1}{1-\hat{F}\left(\alpha\right)}.
\end{eqnarray*}


\begin{Prop}
La Transformada de Laplace $\hat{U}\left(\alpha\right)$ y $\hat{F}\left(\alpha\right)$ determina una a la otra de manera \'unica por la relaci\'on $\hat{U}\left(\alpha\right)=\frac{1}{1-\hat{F}\left(\alpha\right)}$.
\end{Prop}


\begin{Note}
Un proceso de renovaci\'on $N\left(t\right)$ cuyos tiempos de inter-renovaci\'on tienen media finita, es un proceso Poisson con tasa $\lambda$ si y s\'olo s\'i $\esp\left[U\left(t\right)\right]=\lambda t$, para $t\geq0$.
\end{Note}


\begin{Teo}
Sea $N\left(t\right)$ un proceso puntual simple con puntos de localizaci\'on $T_{n}$ tal que $\eta\left(t\right)=\esp\left[N\left(\right)\right]$ es finita para cada $t$. Entonces para cualquier funci\'on $f:\rea_{+}\rightarrow\rea$,
\begin{eqnarray*}
\esp\left[\sum_{n=1}^{N\left(\right)}f\left(T_{n}\right)\right]=\int_{\left(0,t\right]}f\left(s\right)d\eta\left(s\right)\textrm{,  }t\geq0,
\end{eqnarray*}
suponiendo que la integral exista. Adem\'as si $X_{1},X_{2},\ldots$ son variables aleatorias definidas en el mismo espacio de probabilidad que el proceso $N\left(t\right)$ tal que $\esp\left[X_{n}|T_{n}=s\right]=f\left(s\right)$, independiente de $n$. Entonces
\begin{eqnarray*}
\esp\left[\sum_{n=1}^{N\left(t\right)}X_{n}\right]=\int_{\left(0,t\right]}f\left(s\right)d\eta\left(s\right)\textrm{,  }t\geq0,
\end{eqnarray*} 
suponiendo que la integral exista. 
\end{Teo}

\begin{Coro}[Identidad de Wald para Renovaciones]
Para el proceso de renovaci\'on $N\left(t\right)$,
\begin{eqnarray*}
\esp\left[T_{N\left(t\right)+1}\right]=\mu\esp\left[N\left(t\right)+1\right]\textrm{,  }t\geq0,
\end{eqnarray*}  
\end{Coro}

%______________________________________________________________________
\section{Procesos de Renovaci\'on}
%______________________________________________________________________

\begin{Def}\label{Def.Tn}
Sean $0\leq T_{1}\leq T_{2}\leq \ldots$ son tiempos aleatorios infinitos en los cuales ocurren ciertos eventos. El n\'umero de tiempos $T_{n}$ en el intervalo $\left[0,t\right)$ es

\begin{eqnarray}
N\left(t\right)=\sum_{n=1}^{\infty}\indora\left(T_{n}\leq t\right),
\end{eqnarray}
para $t\geq0$.
\end{Def}

Si se consideran los puntos $T_{n}$ como elementos de $\rea_{+}$, y $N\left(t\right)$ es el n\'umero de puntos en $\rea$. El proceso denotado por $\left\{N\left(t\right):t\geq0\right\}$, denotado por $N\left(t\right)$, es un proceso puntual en $\rea_{+}$. Los $T_{n}$ son los tiempos de ocurrencia, el proceso puntual $N\left(t\right)$ es simple si su n\'umero de ocurrencias son distintas: $0<T_{1}<T_{2}<\ldots$ casi seguramente.

\begin{Def}
Un proceso puntual $N\left(t\right)$ es un proceso de renovaci\'on si los tiempos de interocurrencia $\xi_{n}=T_{n}-T_{n-1}$, para $n\geq1$, son independientes e identicamente distribuidos con distribuci\'on $F$, donde $F\left(0\right)=0$ y $T_{0}=0$. Los $T_{n}$ son llamados tiempos de renovaci\'on, referente a la independencia o renovaci\'on de la informaci\'on estoc\'astica en estos tiempos. Los $\xi_{n}$ son los tiempos de inter-renovaci\'on, y $N\left(t\right)$ es el n\'umero de renovaciones en el intervalo $\left[0,t\right)$
\end{Def}


\begin{Note}
Para definir un proceso de renovaci\'on para cualquier contexto, solamente hay que especificar una distribuci\'on $F$, con $F\left(0\right)=0$, para los tiempos de inter-renovaci\'on. La funci\'on $F$ en turno degune las otra variables aleatorias. De manera formal, existe un espacio de probabilidad y una sucesi\'on de variables aleatorias $\xi_{1},\xi_{2},\ldots$ definidas en este con distribuci\'on $F$. Entonces las otras cantidades son $T_{n}=\sum_{k=1}^{n}\xi_{k}$ y $N\left(t\right)=\sum_{n=1}^{\infty}\indora\left(T_{n}\leq t\right)$, donde $T_{n}\rightarrow\infty$ casi seguramente por la Ley Fuerte de los Grandes Números.
\end{Note}


\begin{Ejem}[\textbf{Proceso Poisson}]

Suponga que se tienen tiempos de inter-renovaci\'on \textit{i.i.d.} del proceso de renovaci\'on $N\left(t\right)$ tienen distribuci\'on exponencial $F\left(t\right)=q-e^{-\lambda t}$ con tasa $\lambda$. Entonces $N\left(t\right)$ es un proceso Poisson con tasa $\lambda$.

\end{Ejem}


\begin{Note}
Si el primer tiempo de renovaci\'on $\xi_{1}$ no tiene la misma distribuci\'on que el resto de las $\xi_{n}$, para $n\geq2$, a $N\left(t\right)$ se le llama Proceso de Renovaci\'on retardado, donde si $\xi$ tiene distribuci\'on $G$, entonces el tiempo $T_{n}$ de la $n$-\'esima renovaci\'on tiene distribuci\'on $G\star F^{\left(n-1\right)\star}\left(t\right)$
\end{Note}

\begin{Note} Una funci\'on $h:\rea_{+}\rightarrow\rea$ es Directamente Riemann Integrable en los siguientes casos:
\begin{itemize}
\item[a)] $h\left(t\right)\geq0$ es decreciente y Riemann Integrable.
\item[b)] $h$ es continua excepto posiblemente en un conjunto de Lebesgue de medida 0, y $|h\left(t\right)|\leq b\left(t\right)$, donde $b$ es DRI.
\end{itemize}
\end{Note}

\begin{Teo}[Teorema Principal de Renovaci\'on]
Si $F$ es no aritm\'etica y $h\left(t\right)$ es Directamente Riemann Integrable (DRI), entonces

\begin{eqnarray*}
lim_{t\rightarrow\infty}U\star h=\frac{1}{\mu}\int_{\rea_{+}}h\left(s\right)ds.
\end{eqnarray*}
\end{Teo}

\begin{Prop}
Cualquier funci\'on $H\left(t\right)$ acotada en intervalos finitos y que es 0 para $t<0$ puede expresarse como
\begin{eqnarray*}
H\left(t\right)=U\star h\left(t\right)\textrm{,  donde }h\left(t\right)=H\left(t\right)-F\star H\left(t\right)
\end{eqnarray*}
\end{Prop}

\begin{Def}
Un proceso estoc\'astico $X\left(t\right)$ es crudamente regenerativo en un tiempo aleatorio positivo $T$ si
\begin{eqnarray*}
\esp\left[X\left(T+t\right)|T\right]=\esp\left[X\left(t\right)\right]\textrm{, para }t\geq0,\end{eqnarray*}
y con las esperanzas anteriores finitas.
\end{Def}

\begin{Prop}
Sup\'ongase que $X\left(t\right)$ es un proceso crudamente regenerativo en $T$, que tiene distribuci\'on $F$. Si $\esp\left[X\left(t\right)\right]$ es acotado en intervalos finitos, entonces
\begin{eqnarray*}
\esp\left[X\left(t\right)\right]=U\star h\left(t\right)\textrm{,  donde }h\left(t\right)=\esp\left[X\left(t\right)\indora\left(T>t\right)\right].
\end{eqnarray*}
\end{Prop}

\begin{Teo}[Regeneraci\'on Cruda]
Sup\'ongase que $X\left(t\right)$ es un proceso con valores positivo crudamente regenerativo en $T$, y def\'inase $M=\sup\left\{|X\left(t\right)|:t\leq T\right\}$. Si $T$ es no aritm\'etico y $M$ y $MT$ tienen media finita, entonces
\begin{eqnarray*}
lim_{t\rightarrow\infty}\esp\left[X\left(t\right)\right]=\frac{1}{\mu}\int_{\rea_{+}}h\left(s\right)ds,
\end{eqnarray*}
donde $h\left(t\right)=\esp\left[X\left(t\right)\indora\left(T>t\right)\right]$.
\end{Teo}


\begin{Note} Una funci\'on $h:\rea_{+}\rightarrow\rea$ es Directamente Riemann Integrable en los siguientes casos:
\begin{itemize}
\item[a)] $h\left(t\right)\geq0$ es decreciente y Riemann Integrable.
\item[b)] $h$ es continua excepto posiblemente en un conjunto de Lebesgue de medida 0, y $|h\left(t\right)|\leq b\left(t\right)$, donde $b$ es DRI.
\end{itemize}
\end{Note}

\begin{Teo}[Teorema Principal de Renovaci\'on]
Si $F$ es no aritm\'etica y $h\left(t\right)$ es Directamente Riemann Integrable (DRI), entonces

\begin{eqnarray*}
lim_{t\rightarrow\infty}U\star h=\frac{1}{\mu}\int_{\rea_{+}}h\left(s\right)ds.
\end{eqnarray*}
\end{Teo}

\begin{Prop}
Cualquier funci\'on $H\left(t\right)$ acotada en intervalos finitos y que es 0 para $t<0$ puede expresarse como
\begin{eqnarray*}
H\left(t\right)=U\star h\left(t\right)\textrm{,  donde }h\left(t\right)=H\left(t\right)-F\star H\left(t\right)
\end{eqnarray*}
\end{Prop}

\begin{Def}
Un proceso estoc\'astico $X\left(t\right)$ es crudamente regenerativo en un tiempo aleatorio positivo $T$ si
\begin{eqnarray*}
\esp\left[X\left(T+t\right)|T\right]=\esp\left[X\left(t\right)\right]\textrm{, para }t\geq0,\end{eqnarray*}
y con las esperanzas anteriores finitas.
\end{Def}

\begin{Prop}
Sup\'ongase que $X\left(t\right)$ es un proceso crudamente regenerativo en $T$, que tiene distribuci\'on $F$. Si $\esp\left[X\left(t\right)\right]$ es acotado en intervalos finitos, entonces
\begin{eqnarray*}
\esp\left[X\left(t\right)\right]=U\star h\left(t\right)\textrm{,  donde }h\left(t\right)=\esp\left[X\left(t\right)\indora\left(T>t\right)\right].
\end{eqnarray*}
\end{Prop}

\begin{Teo}[Regeneraci\'on Cruda]
Sup\'ongase que $X\left(t\right)$ es un proceso con valores positivo crudamente regenerativo en $T$, y def\'inase $M=\sup\left\{|X\left(t\right)|:t\leq T\right\}$. Si $T$ es no aritm\'etico y $M$ y $MT$ tienen media finita, entonces
\begin{eqnarray*}
lim_{t\rightarrow\infty}\esp\left[X\left(t\right)\right]=\frac{1}{\mu}\int_{\rea_{+}}h\left(s\right)ds,
\end{eqnarray*}
donde $h\left(t\right)=\esp\left[X\left(t\right)\indora\left(T>t\right)\right]$.
\end{Teo}

\begin{Def}
Para el proceso $\left\{\left(N\left(t\right),X\left(t\right)\right):t\geq0\right\}$, sus trayectoria muestrales en el intervalo de tiempo $\left[T_{n-1},T_{n}\right)$ est\'an descritas por
\begin{eqnarray*}
\zeta_{n}=\left(\xi_{n},\left\{X\left(T_{n-1}+t\right):0\leq t<\xi_{n}\right\}\right)
\end{eqnarray*}
Este $\zeta_{n}$ es el $n$-\'esimo segmento del proceso. El proceso es regenerativo sobre los tiempos $T_{n}$ si sus segmentos $\zeta_{n}$ son independientes e id\'enticamennte distribuidos.
\end{Def}


\begin{Note}
Si $\tilde{X}\left(t\right)$ con espacio de estados $\tilde{S}$ es regenerativo sobre $T_{n}$, entonces $X\left(t\right)=f\left(\tilde{X}\left(t\right)\right)$ tambi\'en es regenerativo sobre $T_{n}$, para cualquier funci\'on $f:\tilde{S}\rightarrow S$.
\end{Note}

\begin{Note}
Los procesos regenerativos son crudamente regenerativos, pero no al rev\'es.
\end{Note}


\begin{Note}
Un proceso estoc\'astico a tiempo continuo o discreto es regenerativo si existe un proceso de renovaci\'on  tal que los segmentos del proceso entre tiempos de renovaci\'on sucesivos son i.i.d., es decir, para $\left\{X\left(t\right):t\geq0\right\}$ proceso estoc\'astico a tiempo continuo con espacio de estados $S$, espacio m\'etrico.
\end{Note}

Para $\left\{X\left(t\right):t\geq0\right\}$ Proceso Estoc\'astico a tiempo continuo con estado de espacios $S$, que es un espacio m\'etrico, con trayectorias continuas por la derecha y con l\'imites por la izquierda c.s. Sea $N\left(t\right)$ un proceso de renovaci\'on en $\rea_{+}$ definido en el mismo espacio de probabilidad que $X\left(t\right)$, con tiempos de renovaci\'on $T$ y tiempos de inter-renovaci\'on $\xi_{n}=T_{n}-T_{n-1}$, con misma distribuci\'on $F$ de media finita $\mu$.



\begin{Def}
Para el proceso $\left\{\left(N\left(t\right),X\left(t\right)\right):t\geq0\right\}$, sus trayectoria muestrales en el intervalo de tiempo $\left[T_{n-1},T_{n}\right)$ est\'an descritas por
\begin{eqnarray*}
\zeta_{n}=\left(\xi_{n},\left\{X\left(T_{n-1}+t\right):0\leq t<\xi_{n}\right\}\right)
\end{eqnarray*}
Este $\zeta_{n}$ es el $n$-\'esimo segmento del proceso. El proceso es regenerativo sobre los tiempos $T_{n}$ si sus segmentos $\zeta_{n}$ son independientes e id\'enticamennte distribuidos.
\end{Def}

\begin{Note}
Un proceso regenerativo con media de la longitud de ciclo finita es llamado positivo recurrente.
\end{Note}

\begin{Teo}[Procesos Regenerativos]
Suponga que el proceso
\end{Teo}


\begin{Def}[Renewal Process Trinity]
Para un proceso de renovaci\'on $N\left(t\right)$, los siguientes procesos proveen de informaci\'on sobre los tiempos de renovaci\'on.
\begin{itemize}
\item $A\left(t\right)=t-T_{N\left(t\right)}$, el tiempo de recurrencia hacia atr\'as al tiempo $t$, que es el tiempo desde la \'ultima renovaci\'on para $t$.

\item $B\left(t\right)=T_{N\left(t\right)+1}-t$, el tiempo de recurrencia hacia adelante al tiempo $t$, residual del tiempo de renovaci\'on, que es el tiempo para la pr\'oxima renovaci\'on despu\'es de $t$.

\item $L\left(t\right)=\xi_{N\left(t\right)+1}=A\left(t\right)+B\left(t\right)$, la longitud del intervalo de renovaci\'on que contiene a $t$.
\end{itemize}
\end{Def}

\begin{Note}
El proceso tridimensional $\left(A\left(t\right),B\left(t\right),L\left(t\right)\right)$ es regenerativo sobre $T_{n}$, y por ende cada proceso lo es. Cada proceso $A\left(t\right)$ y $B\left(t\right)$ son procesos de MArkov a tiempo continuo con trayectorias continuas por partes en el espacio de estados $\rea_{+}$. Una expresi\'on conveniente para su distribuci\'on conjunta es, para $0\leq x<t,y\geq0$
\begin{equation}\label{NoRenovacion}
P\left\{A\left(t\right)>x,B\left(t\right)>y\right\}=
P\left\{N\left(t+y\right)-N\left((t-x)\right)=0\right\}
\end{equation}
\end{Note}

\begin{Ejem}[Tiempos de recurrencia Poisson]
Si $N\left(t\right)$ es un proceso Poisson con tasa $\lambda$, entonces de la expresi\'on (\ref{NoRenovacion}) se tiene que

\begin{eqnarray*}
\begin{array}{lc}
P\left\{A\left(t\right)>x,B\left(t\right)>y\right\}=e^{-\lambda\left(x+y\right)},&0\leq x<t,y\geq0,
\end{array}
\end{eqnarray*}
que es la probabilidad Poisson de no renovaciones en un intervalo de longitud $x+y$.

\end{Ejem}

\begin{Note}
Una cadena de Markov erg\'odica tiene la propiedad de ser estacionaria si la distribución de su estado al tiempo $0$ es su distribuci\'on estacionaria.
\end{Note}


\begin{Def}
Un proceso estoc\'astico a tiempo continuo $\left\{X\left(t\right):t\geq0\right\}$ en un espacio general es estacionario si sus distribuciones finito dimensionales son invariantes bajo cualquier  traslado: para cada $0\leq s_{1}<s_{2}<\cdots<s_{k}$ y $t\geq0$,
\begin{eqnarray*}
\left(X\left(s_{1}+t\right),\ldots,X\left(s_{k}+t\right)\right)=_{d}\left(X\left(s_{1}\right),\ldots,X\left(s_{k}\right)\right).
\end{eqnarray*}
\end{Def}

\begin{Note}
Un proceso de Markov es estacionario si $X\left(t\right)=_{d}X\left(0\right)$, $t\geq0$.
\end{Note}

Considerese el proceso $N\left(t\right)=\sum_{n}\indora\left(\tau_{n}\leq t\right)$ en $\rea_{+}$, con puntos $0<\tau_{1}<\tau_{2}<\cdots$.

\begin{Prop}
Si $N$ es un proceso puntual estacionario y $\esp\left[N\left(1\right)\right]<\infty$, entonces $\esp\left[N\left(t\right)\right]=t\esp\left[N\left(1\right)\right]$, $t\geq0$

\end{Prop}

\begin{Teo}
Los siguientes enunciados son equivalentes
\begin{itemize}
\item[i)] El proceso retardado de renovaci\'on $N$ es estacionario.

\item[ii)] EL proceso de tiempos de recurrencia hacia adelante $B\left(t\right)$ es estacionario.


\item[iii)] $\esp\left[N\left(t\right)\right]=t/\mu$,


\item[iv)] $G\left(t\right)=F_{e}\left(t\right)=\frac{1}{\mu}\int_{0}^{t}\left[1-F\left(s\right)\right]ds$
\end{itemize}
Cuando estos enunciados son ciertos, $P\left\{B\left(t\right)\leq x\right\}=F_{e}\left(x\right)$, para $t,x\geq0$.

\end{Teo}

\begin{Note}
Una consecuencia del teorema anterior es que el Proceso Poisson es el \'unico proceso sin retardo que es estacionario.
\end{Note}

\begin{Coro}
El proceso de renovaci\'on $N\left(t\right)$ sin retardo, y cuyos tiempos de inter renonaci\'on tienen media finita, es estacionario si y s\'olo si es un proceso Poisson.

\end{Coro}

%______________________________________________________________________
%\subsection*{Procesos de Renovaci\'on}
%______________________________________________________________________

\begin{Def}\label{Def.Tn}
Sean $0\leq T_{1}\leq T_{2}\leq \ldots$ son tiempos aleatorios infinitos en los cuales ocurren ciertos eventos. El n\'umero de tiempos $T_{n}$ en el intervalo $\left[0,t\right)$ es

\begin{eqnarray}
N\left(t\right)=\sum_{n=1}^{\infty}\indora\left(T_{n}\leq t\right),
\end{eqnarray}
para $t\geq0$.
\end{Def}

Si se consideran los puntos $T_{n}$ como elementos de $\rea_{+}$, y $N\left(t\right)$ es el n\'umero de puntos en $\rea$. El proceso denotado por $\left\{N\left(t\right):t\geq0\right\}$, denotado por $N\left(t\right)$, es un proceso puntual en $\rea_{+}$. Los $T_{n}$ son los tiempos de ocurrencia, el proceso puntual $N\left(t\right)$ es simple si su n\'umero de ocurrencias son distintas: $0<T_{1}<T_{2}<\ldots$ casi seguramente.

\begin{Def}
Un proceso puntual $N\left(t\right)$ es un proceso de renovaci\'on si los tiempos de interocurrencia $\xi_{n}=T_{n}-T_{n-1}$, para $n\geq1$, son independientes e identicamente distribuidos con distribuci\'on $F$, donde $F\left(0\right)=0$ y $T_{0}=0$. Los $T_{n}$ son llamados tiempos de renovaci\'on, referente a la independencia o renovaci\'on de la informaci\'on estoc\'astica en estos tiempos. Los $\xi_{n}$ son los tiempos de inter-renovaci\'on, y $N\left(t\right)$ es el n\'umero de renovaciones en el intervalo $\left[0,t\right)$
\end{Def}


\begin{Note}
Para definir un proceso de renovaci\'on para cualquier contexto, solamente hay que especificar una distribuci\'on $F$, con $F\left(0\right)=0$, para los tiempos de inter-renovaci\'on. La funci\'on $F$ en turno degune las otra variables aleatorias. De manera formal, existe un espacio de probabilidad y una sucesi\'on de variables aleatorias $\xi_{1},\xi_{2},\ldots$ definidas en este con distribuci\'on $F$. Entonces las otras cantidades son $T_{n}=\sum_{k=1}^{n}\xi_{k}$ y $N\left(t\right)=\sum_{n=1}^{\infty}\indora\left(T_{n}\leq t\right)$, donde $T_{n}\rightarrow\infty$ casi seguramente por la Ley Fuerte de los Grandes Números.
\end{Note}





%______________________________________________________________________
%\subsection*{Procesos de Renovaci\'on}
%______________________________________________________________________

\begin{Def}%\label{Def.Tn}
Sean $0\leq T_{1}\leq T_{2}\leq \ldots$ son tiempos aleatorios infinitos en los cuales ocurren ciertos eventos. El n\'umero de tiempos $T_{n}$ en el intervalo $\left[0,t\right)$ es

\begin{eqnarray}
N\left(t\right)=\sum_{n=1}^{\infty}\indora\left(T_{n}\leq t\right),
\end{eqnarray}
para $t\geq0$.
\end{Def}

Si se consideran los puntos $T_{n}$ como elementos de $\rea_{+}$, y $N\left(t\right)$ es el n\'umero de puntos en $\rea$. El proceso denotado por $\left\{N\left(t\right):t\geq0\right\}$, denotado por $N\left(t\right)$, es un proceso puntual en $\rea_{+}$. Los $T_{n}$ son los tiempos de ocurrencia, el proceso puntual $N\left(t\right)$ es simple si su n\'umero de ocurrencias son distintas: $0<T_{1}<T_{2}<\ldots$ casi seguramente.

\begin{Def}
Un proceso puntual $N\left(t\right)$ es un proceso de renovaci\'on si los tiempos de interocurrencia $\xi_{n}=T_{n}-T_{n-1}$, para $n\geq1$, son independientes e identicamente distribuidos con distribuci\'on $F$, donde $F\left(0\right)=0$ y $T_{0}=0$. Los $T_{n}$ son llamados tiempos de renovaci\'on, referente a la independencia o renovaci\'on de la informaci\'on estoc\'astica en estos tiempos. Los $\xi_{n}$ son los tiempos de inter-renovaci\'on, y $N\left(t\right)$ es el n\'umero de renovaciones en el intervalo $\left[0,t\right)$
\end{Def}


\begin{Note}
Para definir un proceso de renovaci\'on para cualquier contexto, solamente hay que especificar una distribuci\'on $F$, con $F\left(0\right)=0$, para los tiempos de inter-renovaci\'on. La funci\'on $F$ en turno degune las otra variables aleatorias. De manera formal, existe un espacio de probabilidad y una sucesi\'on de variables aleatorias $\xi_{1},\xi_{2},\ldots$ definidas en este con distribuci\'on $F$. Entonces las otras cantidades son $T_{n}=\sum_{k=1}^{n}\xi_{k}$ y $N\left(t\right)=\sum_{n=1}^{\infty}\indora\left(T_{n}\leq t\right)$, donde $T_{n}\rightarrow\infty$ casi seguramente por la Ley Fuerte de los Grandes Números.
\end{Note}


%______________________________________________________________________
%\subsection*{Procesos de Renovaci\'on}
%______________________________________________________________________

\begin{Def}%\label{Def.Tn}
Sean $0\leq T_{1}\leq T_{2}\leq \ldots$ son tiempos aleatorios infinitos en los cuales ocurren ciertos eventos. El n\'umero de tiempos $T_{n}$ en el intervalo $\left[0,t\right)$ es

\begin{eqnarray}
N\left(t\right)=\sum_{n=1}^{\infty}\indora\left(T_{n}\leq t\right),
\end{eqnarray}
para $t\geq0$.
\end{Def}

Si se consideran los puntos $T_{n}$ como elementos de $\rea_{+}$, y $N\left(t\right)$ es el n\'umero de puntos en $\rea$. El proceso denotado por $\left\{N\left(t\right):t\geq0\right\}$, denotado por $N\left(t\right)$, es un proceso puntual en $\rea_{+}$. Los $T_{n}$ son los tiempos de ocurrencia, el proceso puntual $N\left(t\right)$ es simple si su n\'umero de ocurrencias son distintas: $0<T_{1}<T_{2}<\ldots$ casi seguramente.

\begin{Def}
Un proceso puntual $N\left(t\right)$ es un proceso de renovaci\'on si los tiempos de interocurrencia $\xi_{n}=T_{n}-T_{n-1}$, para $n\geq1$, son independientes e identicamente distribuidos con distribuci\'on $F$, donde $F\left(0\right)=0$ y $T_{0}=0$. Los $T_{n}$ son llamados tiempos de renovaci\'on, referente a la independencia o renovaci\'on de la informaci\'on estoc\'astica en estos tiempos. Los $\xi_{n}$ son los tiempos de inter-renovaci\'on, y $N\left(t\right)$ es el n\'umero de renovaciones en el intervalo $\left[0,t\right)$
\end{Def}


\begin{Note}
Para definir un proceso de renovaci\'on para cualquier contexto, solamente hay que especificar una distribuci\'on $F$, con $F\left(0\right)=0$, para los tiempos de inter-renovaci\'on. La funci\'on $F$ en turno degune las otra variables aleatorias. De manera formal, existe un espacio de probabilidad y una sucesi\'on de variables aleatorias $\xi_{1},\xi_{2},\ldots$ definidas en este con distribuci\'on $F$. Entonces las otras cantidades son $T_{n}=\sum_{k=1}^{n}\xi_{k}$ y $N\left(t\right)=\sum_{n=1}^{\infty}\indora\left(T_{n}\leq t\right)$, donde $T_{n}\rightarrow\infty$ casi seguramente por la Ley Fuerte de los Grandes Números.
\end{Note}


%______________________________________________________________________
%\subsection*{Procesos de Renovaci\'on}
%______________________________________________________________________

\begin{Def}\label{Def.Tn}
Sean $0\leq T_{1}\leq T_{2}\leq \ldots$ son tiempos aleatorios infinitos en los cuales ocurren ciertos eventos. El n\'umero de tiempos $T_{n}$ en el intervalo $\left[0,t\right)$ es

\begin{eqnarray}
N\left(t\right)=\sum_{n=1}^{\infty}\indora\left(T_{n}\leq t\right),
\end{eqnarray}
para $t\geq0$.
\end{Def}

Si se consideran los puntos $T_{n}$ como elementos de $\rea_{+}$, y $N\left(t\right)$ es el n\'umero de puntos en $\rea$. El proceso denotado por $\left\{N\left(t\right):t\geq0\right\}$, denotado por $N\left(t\right)$, es un proceso puntual en $\rea_{+}$. Los $T_{n}$ son los tiempos de ocurrencia, el proceso puntual $N\left(t\right)$ es simple si su n\'umero de ocurrencias son distintas: $0<T_{1}<T_{2}<\ldots$ casi seguramente.

\begin{Def}
Un proceso puntual $N\left(t\right)$ es un proceso de renovaci\'on si los tiempos de interocurrencia $\xi_{n}=T_{n}-T_{n-1}$, para $n\geq1$, son independientes e identicamente distribuidos con distribuci\'on $F$, donde $F\left(0\right)=0$ y $T_{0}=0$. Los $T_{n}$ son llamados tiempos de renovaci\'on, referente a la independencia o renovaci\'on de la informaci\'on estoc\'astica en estos tiempos. Los $\xi_{n}$ son los tiempos de inter-renovaci\'on, y $N\left(t\right)$ es el n\'umero de renovaciones en el intervalo $\left[0,t\right)$
\end{Def}


\begin{Note}
Para definir un proceso de renovaci\'on para cualquier contexto, solamente hay que especificar una distribuci\'on $F$, con $F\left(0\right)=0$, para los tiempos de inter-renovaci\'on. La funci\'on $F$ en turno degune las otra variables aleatorias. De manera formal, existe un espacio de probabilidad y una sucesi\'on de variables aleatorias $\xi_{1},\xi_{2},\ldots$ definidas en este con distribuci\'on $F$. Entonces las otras cantidades son $T_{n}=\sum_{k=1}^{n}\xi_{k}$ y $N\left(t\right)=\sum_{n=1}^{\infty}\indora\left(T_{n}\leq t\right)$, donde $T_{n}\rightarrow\infty$ casi seguramente por la Ley Fuerte de los Grandes Números.
\end{Note}


%______________________________________________________________________
%\subsection*{Procesos de Renovaci\'on}
%______________________________________________________________________

\begin{Def}\label{Def.Tn}
Sean $0\leq T_{1}\leq T_{2}\leq \ldots$ son tiempos aleatorios infinitos en los cuales ocurren ciertos eventos. El n\'umero de tiempos $T_{n}$ en el intervalo $\left[0,t\right)$ es

\begin{eqnarray}
N\left(t\right)=\sum_{n=1}^{\infty}\indora\left(T_{n}\leq t\right),
\end{eqnarray}
para $t\geq0$.
\end{Def}

Si se consideran los puntos $T_{n}$ como elementos de $\rea_{+}$, y $N\left(t\right)$ es el n\'umero de puntos en $\rea$. El proceso denotado por $\left\{N\left(t\right):t\geq0\right\}$, denotado por $N\left(t\right)$, es un proceso puntual en $\rea_{+}$. Los $T_{n}$ son los tiempos de ocurrencia, el proceso puntual $N\left(t\right)$ es simple si su n\'umero de ocurrencias son distintas: $0<T_{1}<T_{2}<\ldots$ casi seguramente.

\begin{Def}
Un proceso puntual $N\left(t\right)$ es un proceso de renovaci\'on si los tiempos de interocurrencia $\xi_{n}=T_{n}-T_{n-1}$, para $n\geq1$, son independientes e identicamente distribuidos con distribuci\'on $F$, donde $F\left(0\right)=0$ y $T_{0}=0$. Los $T_{n}$ son llamados tiempos de renovaci\'on, referente a la independencia o renovaci\'on de la informaci\'on estoc\'astica en estos tiempos. Los $\xi_{n}$ son los tiempos de inter-renovaci\'on, y $N\left(t\right)$ es el n\'umero de renovaciones en el intervalo $\left[0,t\right)$
\end{Def}


\begin{Note}
Para definir un proceso de renovaci\'on para cualquier contexto, solamente hay que especificar una distribuci\'on $F$, con $F\left(0\right)=0$, para los tiempos de inter-renovaci\'on. La funci\'on $F$ en turno degune las otra variables aleatorias. De manera formal, existe un espacio de probabilidad y una sucesi\'on de variables aleatorias $\xi_{1},\xi_{2},\ldots$ definidas en este con distribuci\'on $F$. Entonces las otras cantidades son $T_{n}=\sum_{k=1}^{n}\xi_{k}$ y $N\left(t\right)=\sum_{n=1}^{\infty}\indora\left(T_{n}\leq t\right)$, donde $T_{n}\rightarrow\infty$ casi seguramente por la Ley Fuerte de los Grandes Números.
\end{Note}
%___________________________________________________________________________________________
%
\section{Renewal and Regenerative Processes: Serfozo\cite{Serfozo}}\label{Appendix.E}
%___________________________________________________________________________________________
%
\begin{Def}\label{Def.Tn}
Sean $0\leq T_{1}\leq T_{2}\leq \ldots$ son tiempos aleatorios infinitos en los cuales ocurren ciertos eventos. El n\'umero de tiempos $T_{n}$ en el intervalo $\left[0,t\right)$ es

\begin{eqnarray}
N\left(t\right)=\sum_{n=1}^{\infty}\indora\left(T_{n}\leq t\right),
\end{eqnarray}
para $t\geq0$.
\end{Def}

Si se consideran los puntos $T_{n}$ como elementos de $\rea_{+}$, y $N\left(t\right)$ es el n\'umero de puntos en $\rea$. El proceso denotado por $\left\{N\left(t\right):t\geq0\right\}$, denotado por $N\left(t\right)$, es un proceso puntual en $\rea_{+}$. Los $T_{n}$ son los tiempos de ocurrencia, el proceso puntual $N\left(t\right)$ es simple si su n\'umero de ocurrencias son distintas: $0<T_{1}<T_{2}<\ldots$ casi seguramente.

\begin{Def}
Un proceso puntual $N\left(t\right)$ es un proceso de renovaci\'on si los tiempos de interocurrencia $\xi_{n}=T_{n}-T_{n-1}$, para $n\geq1$, son independientes e identicamente distribuidos con distribuci\'on $F$, donde $F\left(0\right)=0$ y $T_{0}=0$. Los $T_{n}$ son llamados tiempos de renovaci\'on, referente a la independencia o renovaci\'on de la informaci\'on estoc\'astica en estos tiempos. Los $\xi_{n}$ son los tiempos de inter-renovaci\'on, y $N\left(t\right)$ es el n\'umero de renovaciones en el intervalo $\left[0,t\right)$
\end{Def}


\begin{Note}
Para definir un proceso de renovaci\'on para cualquier contexto, solamente hay que especificar una distribuci\'on $F$, con $F\left(0\right)=0$, para los tiempos de inter-renovaci\'on. La funci\'on $F$ en turno degune las otra variables aleatorias. De manera formal, existe un espacio de probabilidad y una sucesi\'on de variables aleatorias $\xi_{1},\xi_{2},\ldots$ definidas en este con distribuci\'on $F$. Entonces las otras cantidades son $T_{n}=\sum_{k=1}^{n}\xi_{k}$ y $N\left(t\right)=\sum_{n=1}^{\infty}\indora\left(T_{n}\leq t\right)$, donde $T_{n}\rightarrow\infty$ casi seguramente por la Ley Fuerte de los Grandes N\'umeros.
\end{Note}







Los tiempos $T_{n}$ est\'an relacionados con los conteos de $N\left(t\right)$ por

\begin{eqnarray*}
\left\{N\left(t\right)\geq n\right\}&=&\left\{T_{n}\leq t\right\}\\
T_{N\left(t\right)}\leq &t&<T_{N\left(t\right)+1},
\end{eqnarray*}

adem\'as $N\left(T_{n}\right)=n$, y 

\begin{eqnarray*}
N\left(t\right)=\max\left\{n:T_{n}\leq t\right\}=\min\left\{n:T_{n+1}>t\right\}
\end{eqnarray*}

Por propiedades de la convoluci\'on se sabe que

\begin{eqnarray*}
P\left\{T_{n}\leq t\right\}=F^{n\star}\left(t\right)
\end{eqnarray*}
que es la $n$-\'esima convoluci\'on de $F$. Entonces 

\begin{eqnarray*}
\left\{N\left(t\right)\geq n\right\}&=&\left\{T_{n}\leq t\right\}\\
P\left\{N\left(t\right)\leq n\right\}&=&1-F^{\left(n+1\right)\star}\left(t\right)
\end{eqnarray*}

Adem\'as usando el hecho de que $\esp\left[N\left(t\right)\right]=\sum_{n=1}^{\infty}P\left\{N\left(t\right)\geq n\right\}$
se tiene que

\begin{eqnarray*}
\esp\left[N\left(t\right)\right]=\sum_{n=1}^{\infty}F^{n\star}\left(t\right)
\end{eqnarray*}

\begin{Prop}
Para cada $t\geq0$, la funci\'on generadora de momentos $\esp\left[e^{\alpha N\left(t\right)}\right]$ existe para alguna $\alpha$ en una vecindad del 0, y de aqu\'i que $\esp\left[N\left(t\right)^{m}\right]<\infty$, para $m\geq1$.
\end{Prop}

\begin{Ejem}[\textbf{Proceso Poisson}]

Suponga que se tienen tiempos de inter-renovaci\'on \textit{i.i.d.} del proceso de renovaci\'on $N\left(t\right)$ tienen distribuci\'on exponencial $F\left(t\right)=q-e^{-\lambda t}$ con tasa $\lambda$. Entonces $N\left(t\right)$ es un proceso Poisson con tasa $\lambda$.

\end{Ejem}


\begin{Note}
Si el primer tiempo de renovaci\'on $\xi_{1}$ no tiene la misma distribuci\'on que el resto de las $\xi_{n}$, para $n\geq2$, a $N\left(t\right)$ se le llama Proceso de Renovaci\'on retardado, donde si $\xi$ tiene distribuci\'on $G$, entonces el tiempo $T_{n}$ de la $n$-\'esima renovaci\'on tiene distribuci\'on $G\star F^{\left(n-1\right)\star}\left(t\right)$
\end{Note}


\begin{Teo}
Para una constante $\mu\leq\infty$ ( o variable aleatoria), las siguientes expresiones son equivalentes:

\begin{eqnarray}
lim_{n\rightarrow\infty}n^{-1}T_{n}&=&\mu,\textrm{ c.s.}\\
lim_{t\rightarrow\infty}t^{-1}N\left(t\right)&=&1/\mu,\textrm{ c.s.}
\end{eqnarray}
\end{Teo}


Es decir, $T_{n}$ satisface la Ley Fuerte de los Grandes N\'umeros s\'i y s\'olo s\'i $N\left/t\right)$ la cumple.


\begin{Coro}[Ley Fuerte de los Grandes N\'umeros para Procesos de Renovaci\'on]
Si $N\left(t\right)$ es un proceso de renovaci\'on cuyos tiempos de inter-renovaci\'on tienen media $\mu\leq\infty$, entonces
\begin{eqnarray}
t^{-1}N\left(t\right)\rightarrow 1/\mu,\textrm{ c.s. cuando }t\rightarrow\infty.
\end{eqnarray}

\end{Coro}


Considerar el proceso estoc\'astico de valores reales $\left\{Z\left(t\right):t\geq0\right\}$ en el mismo espacio de probabilidad que $N\left(t\right)$

\begin{Def}
Para el proceso $\left\{Z\left(t\right):t\geq0\right\}$ se define la fluctuaci\'on m\'axima de $Z\left(t\right)$ en el intervalo $\left(T_{n-1},T_{n}\right]$:
\begin{eqnarray*}
M_{n}=\sup_{T_{n-1}<t\leq T_{n}}|Z\left(t\right)-Z\left(T_{n-1}\right)|
\end{eqnarray*}
\end{Def}

\begin{Teo}
Sup\'ongase que $n^{-1}T_{n}\rightarrow\mu$ c.s. cuando $n\rightarrow\infty$, donde $\mu\leq\infty$ es una constante o variable aleatoria. Sea $a$ una constante o variable aleatoria que puede ser infinita cuando $\mu$ es finita, y considere las expresiones l\'imite:
\begin{eqnarray}
lim_{n\rightarrow\infty}n^{-1}Z\left(T_{n}\right)&=&a,\textrm{ c.s.}\\
lim_{t\rightarrow\infty}t^{-1}Z\left(t\right)&=&a/\mu,\textrm{ c.s.}
\end{eqnarray}
La segunda expresi\'on implica la primera. Conversamente, la primera implica la segunda si el proceso $Z\left(t\right)$ es creciente, o si $lim_{n\rightarrow\infty}n^{-1}M_{n}=0$ c.s.
\end{Teo}

\begin{Coro}
Si $N\left(t\right)$ es un proceso de renovaci\'on, y $\left(Z\left(T_{n}\right)-Z\left(T_{n-1}\right),M_{n}\right)$, para $n\geq1$, son variables aleatorias independientes e id\'enticamente distribuidas con media finita, entonces,
\begin{eqnarray}
lim_{t\rightarrow\infty}t^{-1}Z\left(t\right)\rightarrow\frac{\esp\left[Z\left(T_{1}\right)-Z\left(T_{0}\right)\right]}{\esp\left[T_{1}\right]},\textrm{ c.s. cuando  }t\rightarrow\infty.
\end{eqnarray}
\end{Coro}


Sup\'ongase que $N\left(t\right)$ es un proceso de renovaci\'on con distribuci\'on $F$ con media finita $\mu$.

\begin{Def}
La funci\'on de renovaci\'on asociada con la distribuci\'on $F$, del proceso $N\left(t\right)$, es
\begin{eqnarray*}
U\left(t\right)=\sum_{n=1}^{\infty}F^{n\star}\left(t\right),\textrm{   }t\geq0,
\end{eqnarray*}
donde $F^{0\star}\left(t\right)=\indora\left(t\geq0\right)$.
\end{Def}


\begin{Prop}
Sup\'ongase que la distribuci\'on de inter-renovaci\'on $F$ tiene densidad $f$. Entonces $U\left(t\right)$ tambi\'en tiene densidad, para $t>0$, y es $U^{'}\left(t\right)=\sum_{n=0}^{\infty}f^{n\star}\left(t\right)$. Adem\'as
\begin{eqnarray*}
\prob\left\{N\left(t\right)>N\left(t-\right)\right\}=0\textrm{,   }t\geq0.
\end{eqnarray*}
\end{Prop}

\begin{Def}
La Transformada de Laplace-Stieljes de $F$ est\'a dada por

\begin{eqnarray*}
\hat{F}\left(\alpha\right)=\int_{\rea_{+}}e^{-\alpha t}dF\left(t\right)\textrm{,  }\alpha\geq0.
\end{eqnarray*}
\end{Def}

Entonces

\begin{eqnarray*}
\hat{U}\left(\alpha\right)=\sum_{n=0}^{\infty}\hat{F^{n\star}}\left(\alpha\right)=\sum_{n=0}^{\infty}\hat{F}\left(\alpha\right)^{n}=\frac{1}{1-\hat{F}\left(\alpha\right)}.
\end{eqnarray*}


\begin{Prop}
La Transformada de Laplace $\hat{U}\left(\alpha\right)$ y $\hat{F}\left(\alpha\right)$ determina una a la otra de manera \'unica por la relaci\'on $\hat{U}\left(\alpha\right)=\frac{1}{1-\hat{F}\left(\alpha\right)}$.
\end{Prop}


\begin{Note}
Un proceso de renovaci\'on $N\left(t\right)$ cuyos tiempos de inter-renovaci\'on tienen media finita, es un proceso Poisson con tasa $\lambda$ si y s\'olo s\'i $\esp\left[U\left(t\right)\right]=\lambda t$, para $t\geq0$.
\end{Note}


\begin{Teo}
Sea $N\left(t\right)$ un proceso puntual simple con puntos de localizaci\'on $T_{n}$ tal que $\eta\left(t\right)=\esp\left[N\left(\right)\right]$ es finita para cada $t$. Entonces para cualquier funci\'on $f:\rea_{+}\rightarrow\rea$,
\begin{eqnarray*}
\esp\left[\sum_{n=1}^{N\left(\right)}f\left(T_{n}\right)\right]=\int_{\left(0,t\right]}f\left(s\right)d\eta\left(s\right)\textrm{,  }t\geq0,
\end{eqnarray*}
suponiendo que la integral exista. Adem\'as si $X_{1},X_{2},\ldots$ son variables aleatorias definidas en el mismo espacio de probabilidad que el proceso $N\left(t\right)$ tal que $\esp\left[X_{n}|T_{n}=s\right]=f\left(s\right)$, independiente de $n$. Entonces
\begin{eqnarray*}
\esp\left[\sum_{n=1}^{N\left(t\right)}X_{n}\right]=\int_{\left(0,t\right]}f\left(s\right)d\eta\left(s\right)\textrm{,  }t\geq0,
\end{eqnarray*} 
suponiendo que la integral exista. 
\end{Teo}

\begin{Coro}[Identidad de Wald para Renovaciones]
Para el proceso de renovaci\'on $N\left(t\right)$,
\begin{eqnarray*}
\esp\left[T_{N\left(t\right)+1}\right]=\mu\esp\left[N\left(t\right)+1\right]\textrm{,  }t\geq0,
\end{eqnarray*}  
\end{Coro}


\begin{Def}
Sea $h\left(t\right)$ funci\'on de valores reales en $\rea$ acotada en intervalos finitos e igual a cero para $t<0$ La ecuaci\'on de renovaci\'on para $h\left(t\right)$ y la distribuci\'on $F$ es

\begin{eqnarray}\label{Ec.Renovacion}
H\left(t\right)=h\left(t\right)+\int_{\left[0,t\right]}H\left(t-s\right)dF\left(s\right)\textrm{,    }t\geq0,
\end{eqnarray}
donde $H\left(t\right)$ es una funci\'on de valores reales. Esto es $H=h+F\star H$. Decimos que $H\left(t\right)$ es soluci\'on de esta ecuaci\'on si satisface la ecuaci\'on, y es acotada en intervalos finitos e iguales a cero para $t<0$.
\end{Def}

\begin{Prop}
La funci\'on $U\star h\left(t\right)$ es la \'unica soluci\'on de la ecuaci\'on de renovaci\'on (\ref{Ec.Renovacion}).
\end{Prop}

\begin{Teo}[Teorema Renovaci\'on Elemental]
\begin{eqnarray*}
t^{-1}U\left(t\right)\rightarrow 1/\mu\textrm{,    cuando }t\rightarrow\infty.
\end{eqnarray*}
\end{Teo}



Sup\'ongase que $N\left(t\right)$ es un proceso de renovaci\'on con distribuci\'on $F$ con media finita $\mu$.

\begin{Def}
La funci\'on de renovaci\'on asociada con la distribuci\'on $F$, del proceso $N\left(t\right)$, es
\begin{eqnarray*}
U\left(t\right)=\sum_{n=1}^{\infty}F^{n\star}\left(t\right),\textrm{   }t\geq0,
\end{eqnarray*}
donde $F^{0\star}\left(t\right)=\indora\left(t\geq0\right)$.
\end{Def}


\begin{Prop}
Sup\'ongase que la distribuci\'on de inter-renovaci\'on $F$ tiene densidad $f$. Entonces $U\left(t\right)$ tambi\'en tiene densidad, para $t>0$, y es $U^{'}\left(t\right)=\sum_{n=0}^{\infty}f^{n\star}\left(t\right)$. Adem\'as
\begin{eqnarray*}
\prob\left\{N\left(t\right)>N\left(t-\right)\right\}=0\textrm{,   }t\geq0.
\end{eqnarray*}
\end{Prop}

\begin{Def}
La Transformada de Laplace-Stieljes de $F$ est\'a dada por

\begin{eqnarray*}
\hat{F}\left(\alpha\right)=\int_{\rea_{+}}e^{-\alpha t}dF\left(t\right)\textrm{,  }\alpha\geq0.
\end{eqnarray*}
\end{Def}

Entonces

\begin{eqnarray*}
\hat{U}\left(\alpha\right)=\sum_{n=0}^{\infty}\hat{F^{n\star}}\left(\alpha\right)=\sum_{n=0}^{\infty}\hat{F}\left(\alpha\right)^{n}=\frac{1}{1-\hat{F}\left(\alpha\right)}.
\end{eqnarray*}


\begin{Prop}
La Transformada de Laplace $\hat{U}\left(\alpha\right)$ y $\hat{F}\left(\alpha\right)$ determina una a la otra de manera \'unica por la relaci\'on $\hat{U}\left(\alpha\right)=\frac{1}{1-\hat{F}\left(\alpha\right)}$.
\end{Prop}


\begin{Note}
Un proceso de renovaci\'on $N\left(t\right)$ cuyos tiempos de inter-renovaci\'on tienen media finita, es un proceso Poisson con tasa $\lambda$ si y s\'olo s\'i $\esp\left[U\left(t\right)\right]=\lambda t$, para $t\geq0$.
\end{Note}


\begin{Teo}
Sea $N\left(t\right)$ un proceso puntual simple con puntos de localizaci\'on $T_{n}$ tal que $\eta\left(t\right)=\esp\left[N\left(\right)\right]$ es finita para cada $t$. Entonces para cualquier funci\'on $f:\rea_{+}\rightarrow\rea$,
\begin{eqnarray*}
\esp\left[\sum_{n=1}^{N\left(\right)}f\left(T_{n}\right)\right]=\int_{\left(0,t\right]}f\left(s\right)d\eta\left(s\right)\textrm{,  }t\geq0,
\end{eqnarray*}
suponiendo que la integral exista. Adem\'as si $X_{1},X_{2},\ldots$ son variables aleatorias definidas en el mismo espacio de probabilidad que el proceso $N\left(t\right)$ tal que $\esp\left[X_{n}|T_{n}=s\right]=f\left(s\right)$, independiente de $n$. Entonces
\begin{eqnarray*}
\esp\left[\sum_{n=1}^{N\left(t\right)}X_{n}\right]=\int_{\left(0,t\right]}f\left(s\right)d\eta\left(s\right)\textrm{,  }t\geq0,
\end{eqnarray*} 
suponiendo que la integral exista. 
\end{Teo}

\begin{Coro}[Identidad de Wald para Renovaciones]
Para el proceso de renovaci\'on $N\left(t\right)$,
\begin{eqnarray*}
\esp\left[T_{N\left(t\right)+1}\right]=\mu\esp\left[N\left(t\right)+1\right]\textrm{,  }t\geq0,
\end{eqnarray*}  
\end{Coro}


\begin{Def}
Sea $h\left(t\right)$ funci\'on de valores reales en $\rea$ acotada en intervalos finitos e igual a cero para $t<0$ La ecuaci\'on de renovaci\'on para $h\left(t\right)$ y la distribuci\'on $F$ es

\begin{eqnarray}\label{Ec.Renovacion}
H\left(t\right)=h\left(t\right)+\int_{\left[0,t\right]}H\left(t-s\right)dF\left(s\right)\textrm{,    }t\geq0,
\end{eqnarray}
donde $H\left(t\right)$ es una funci\'on de valores reales. Esto es $H=h+F\star H$. Decimos que $H\left(t\right)$ es soluci\'on de esta ecuaci\'on si satisface la ecuaci\'on, y es acotada en intervalos finitos e iguales a cero para $t<0$.
\end{Def}

\begin{Prop}
La funci\'on $U\star h\left(t\right)$ es la \'unica soluci\'on de la ecuaci\'on de renovaci\'on (\ref{Ec.Renovacion}).
\end{Prop}

\begin{Teo}[Teorema Renovaci\'on Elemental]
\begin{eqnarray*}
t^{-1}U\left(t\right)\rightarrow 1/\mu\textrm{,    cuando }t\rightarrow\infty.
\end{eqnarray*}
\end{Teo}


\begin{Note} Una funci\'on $h:\rea_{+}\rightarrow\rea$ es Directamente Riemann Integrable en los siguientes casos:
\begin{itemize}
\item[a)] $h\left(t\right)\geq0$ es decreciente y Riemann Integrable.
\item[b)] $h$ es continua excepto posiblemente en un conjunto de Lebesgue de medida 0, y $|h\left(t\right)|\leq b\left(t\right)$, donde $b$ es DRI.
\end{itemize}
\end{Note}

\begin{Teo}[Teorema Principal de Renovaci\'on]
Si $F$ es no aritm\'etica y $h\left(t\right)$ es Directamente Riemann Integrable (DRI), entonces

\begin{eqnarray*}
lim_{t\rightarrow\infty}U\star h=\frac{1}{\mu}\int_{\rea_{+}}h\left(s\right)ds.
\end{eqnarray*}
\end{Teo}

\begin{Prop}
Cualquier funci\'on $H\left(t\right)$ acotada en intervalos finitos y que es 0 para $t<0$ puede expresarse como
\begin{eqnarray*}
H\left(t\right)=U\star h\left(t\right)\textrm{,  donde }h\left(t\right)=H\left(t\right)-F\star H\left(t\right)
\end{eqnarray*}
\end{Prop}

\begin{Def}
Un proceso estoc\'astico $X\left(t\right)$ es crudamente regenerativo en un tiempo aleatorio positivo $T$ si
\begin{eqnarray*}
\esp\left[X\left(T+t\right)|T\right]=\esp\left[X\left(t\right)\right]\textrm{, para }t\geq0,\end{eqnarray*}
y con las esperanzas anteriores finitas.
\end{Def}

\begin{Prop}
Sup\'ongase que $X\left(t\right)$ es un proceso crudamente regenerativo en $T$, que tiene distribuci\'on $F$. Si $\esp\left[X\left(t\right)\right]$ es acotado en intervalos finitos, entonces
\begin{eqnarray*}
\esp\left[X\left(t\right)\right]=U\star h\left(t\right)\textrm{,  donde }h\left(t\right)=\esp\left[X\left(t\right)\indora\left(T>t\right)\right].
\end{eqnarray*}
\end{Prop}

\begin{Teo}[Regeneraci\'on Cruda]
Sup\'ongase que $X\left(t\right)$ es un proceso con valores positivo crudamente regenerativo en $T$, y def\'inase $M=\sup\left\{|X\left(t\right)|:t\leq T\right\}$. Si $T$ es no aritm\'etico y $M$ y $MT$ tienen media finita, entonces
\begin{eqnarray*}
lim_{t\rightarrow\infty}\esp\left[X\left(t\right)\right]=\frac{1}{\mu}\int_{\rea_{+}}h\left(s\right)ds,
\end{eqnarray*}
donde $h\left(t\right)=\esp\left[X\left(t\right)\indora\left(T>t\right)\right]$.
\end{Teo}


\begin{Note} Una funci\'on $h:\rea_{+}\rightarrow\rea$ es Directamente Riemann Integrable en los siguientes casos:
\begin{itemize}
\item[a)] $h\left(t\right)\geq0$ es decreciente y Riemann Integrable.
\item[b)] $h$ es continua excepto posiblemente en un conjunto de Lebesgue de medida 0, y $|h\left(t\right)|\leq b\left(t\right)$, donde $b$ es DRI.
\end{itemize}
\end{Note}

\begin{Teo}[Teorema Principal de Renovaci\'on]
Si $F$ es no aritm\'etica y $h\left(t\right)$ es Directamente Riemann Integrable (DRI), entonces

\begin{eqnarray*}
lim_{t\rightarrow\infty}U\star h=\frac{1}{\mu}\int_{\rea_{+}}h\left(s\right)ds.
\end{eqnarray*}
\end{Teo}

\begin{Prop}
Cualquier funci\'on $H\left(t\right)$ acotada en intervalos finitos y que es 0 para $t<0$ puede expresarse como
\begin{eqnarray*}
H\left(t\right)=U\star h\left(t\right)\textrm{,  donde }h\left(t\right)=H\left(t\right)-F\star H\left(t\right)
\end{eqnarray*}
\end{Prop}

\begin{Def}
Un proceso estoc\'astico $X\left(t\right)$ es crudamente regenerativo en un tiempo aleatorio positivo $T$ si
\begin{eqnarray*}
\esp\left[X\left(T+t\right)|T\right]=\esp\left[X\left(t\right)\right]\textrm{, para }t\geq0,\end{eqnarray*}
y con las esperanzas anteriores finitas.
\end{Def}

\begin{Prop}
Sup\'ongase que $X\left(t\right)$ es un proceso crudamente regenerativo en $T$, que tiene distribuci\'on $F$. Si $\esp\left[X\left(t\right)\right]$ es acotado en intervalos finitos, entonces
\begin{eqnarray*}
\esp\left[X\left(t\right)\right]=U\star h\left(t\right)\textrm{,  donde }h\left(t\right)=\esp\left[X\left(t\right)\indora\left(T>t\right)\right].
\end{eqnarray*}
\end{Prop}

\begin{Teo}[Regeneraci\'on Cruda]
Sup\'ongase que $X\left(t\right)$ es un proceso con valores positivo crudamente regenerativo en $T$, y def\'inase $M=\sup\left\{|X\left(t\right)|:t\leq T\right\}$. Si $T$ es no aritm\'etico y $M$ y $MT$ tienen media finita, entonces
\begin{eqnarray*}
lim_{t\rightarrow\infty}\esp\left[X\left(t\right)\right]=\frac{1}{\mu}\int_{\rea_{+}}h\left(s\right)ds,
\end{eqnarray*}
donde $h\left(t\right)=\esp\left[X\left(t\right)\indora\left(T>t\right)\right]$.
\end{Teo}

\begin{Def}
Para el proceso $\left\{\left(N\left(t\right),X\left(t\right)\right):t\geq0\right\}$, sus trayectoria muestrales en el intervalo de tiempo $\left[T_{n-1},T_{n}\right)$ est\'an descritas por
\begin{eqnarray*}
\zeta_{n}=\left(\xi_{n},\left\{X\left(T_{n-1}+t\right):0\leq t<\xi_{n}\right\}\right)
\end{eqnarray*}
Este $\zeta_{n}$ es el $n$-\'esimo segmento del proceso. El proceso es regenerativo sobre los tiempos $T_{n}$ si sus segmentos $\zeta_{n}$ son independientes e id\'enticamennte distribuidos.
\end{Def}


\begin{Note}
Si $\tilde{X}\left(t\right)$ con espacio de estados $\tilde{S}$ es regenerativo sobre $T_{n}$, entonces $X\left(t\right)=f\left(\tilde{X}\left(t\right)\right)$ tambi\'en es regenerativo sobre $T_{n}$, para cualquier funci\'on $f:\tilde{S}\rightarrow S$.
\end{Note}

\begin{Note}
Los procesos regenerativos son crudamente regenerativos, pero no al rev\'es.
\end{Note}


\begin{Note}
Un proceso estoc\'astico a tiempo continuo o discreto es regenerativo si existe un proceso de renovaci\'on  tal que los segmentos del proceso entre tiempos de renovaci\'on sucesivos son i.i.d., es decir, para $\left\{X\left(t\right):t\geq0\right\}$ proceso estoc\'astico a tiempo continuo con espacio de estados $S$, espacio m\'etrico.
\end{Note}

Para $\left\{X\left(t\right):t\geq0\right\}$ Proceso Estoc\'astico a tiempo continuo con estado de espacios $S$, que es un espacio m\'etrico, con trayectorias continuas por la derecha y con l\'imites por la izquierda c.s. Sea $N\left(t\right)$ un proceso de renovaci\'on en $\rea_{+}$ definido en el mismo espacio de probabilidad que $X\left(t\right)$, con tiempos de renovaci\'on $T$ y tiempos de inter-renovaci\'on $\xi_{n}=T_{n}-T_{n-1}$, con misma distribuci\'on $F$ de media finita $\mu$.



\begin{Def}
Para el proceso $\left\{\left(N\left(t\right),X\left(t\right)\right):t\geq0\right\}$, sus trayectoria muestrales en el intervalo de tiempo $\left[T_{n-1},T_{n}\right)$ est\'an descritas por
\begin{eqnarray*}
\zeta_{n}=\left(\xi_{n},\left\{X\left(T_{n-1}+t\right):0\leq t<\xi_{n}\right\}\right)
\end{eqnarray*}
Este $\zeta_{n}$ es el $n$-\'esimo segmento del proceso. El proceso es regenerativo sobre los tiempos $T_{n}$ si sus segmentos $\zeta_{n}$ son independientes e id\'enticamennte distribuidos.
\end{Def}

\begin{Note}
Un proceso regenerativo con media de la longitud de ciclo finita es llamado positivo recurrente.
\end{Note}

\begin{Teo}[Procesos Regenerativos]
Suponga que el proceso
\end{Teo}


\begin{Def}[Renewal Process Trinity]
Para un proceso de renovaci\'on $N\left(t\right)$, los siguientes procesos proveen de informaci\'on sobre los tiempos de renovaci\'on.
\begin{itemize}
\item $A\left(t\right)=t-T_{N\left(t\right)}$, el tiempo de recurrencia hacia atr\'as al tiempo $t$, que es el tiempo desde la \'ultima renovaci\'on para $t$.

\item $B\left(t\right)=T_{N\left(t\right)+1}-t$, el tiempo de recurrencia hacia adelante al tiempo $t$, residual del tiempo de renovaci\'on, que es el tiempo para la pr\'oxima renovaci\'on despu\'es de $t$.

\item $L\left(t\right)=\xi_{N\left(t\right)+1}=A\left(t\right)+B\left(t\right)$, la longitud del intervalo de renovaci\'on que contiene a $t$.
\end{itemize}
\end{Def}

\begin{Note}
El proceso tridimensional $\left(A\left(t\right),B\left(t\right),L\left(t\right)\right)$ es regenerativo sobre $T_{n}$, y por ende cada proceso lo es. Cada proceso $A\left(t\right)$ y $B\left(t\right)$ son procesos de MArkov a tiempo continuo con trayectorias continuas por partes en el espacio de estados $\rea_{+}$. Una expresi\'on conveniente para su distribuci\'on conjunta es, para $0\leq x<t,y\geq0$
\begin{equation}\label{NoRenovacion}
P\left\{A\left(t\right)>x,B\left(t\right)>y\right\}=
P\left\{N\left(t+y\right)-N\left((t-x)\right)=0\right\}
\end{equation}
\end{Note}

\begin{Ejem}[Tiempos de recurrencia Poisson]
Si $N\left(t\right)$ es un proceso Poisson con tasa $\lambda$, entonces de la expresi\'on (\ref{NoRenovacion}) se tiene que

\begin{eqnarray*}
\begin{array}{lc}
P\left\{A\left(t\right)>x,B\left(t\right)>y\right\}=e^{-\lambda\left(x+y\right)},&0\leq x<t,y\geq0,
\end{array}
\end{eqnarray*}
que es la probabilidad Poisson de no renovaciones en un intervalo de longitud $x+y$.

\end{Ejem}

\begin{Note}
Una cadena de Markov erg\'odica tiene la propiedad de ser estacionaria si la distribución de su estado al tiempo $0$ es su distribuci\'on estacionaria.
\end{Note}


\begin{Def}
Un proceso estoc\'astico a tiempo continuo $\left\{X\left(t\right):t\geq0\right\}$ en un espacio general es estacionario si sus distribuciones finito dimensionales son invariantes bajo cualquier  traslado: para cada $0\leq s_{1}<s_{2}<\cdots<s_{k}$ y $t\geq0$,
\begin{eqnarray*}
\left(X\left(s_{1}+t\right),\ldots,X\left(s_{k}+t\right)\right)=_{d}\left(X\left(s_{1}\right),\ldots,X\left(s_{k}\right)\right).
\end{eqnarray*}
\end{Def}

\begin{Note}
Un proceso de Markov es estacionario si $X\left(t\right)=_{d}X\left(0\right)$, $t\geq0$.
\end{Note}

Considerese el proceso $N\left(t\right)=\sum_{n}\indora\left(\tau_{n}\leq t\right)$ en $\rea_{+}$, con puntos $0<\tau_{1}<\tau_{2}<\cdots$.

\begin{Prop}
Si $N$ es un proceso puntual estacionario y $\esp\left[N\left(1\right)\right]<\infty$, entonces $\esp\left[N\left(t\right)\right]=t\esp\left[N\left(1\right)\right]$, $t\geq0$

\end{Prop}

\begin{Teo}
Los siguientes enunciados son equivalentes
\begin{itemize}
\item[i)] El proceso retardado de renovaci\'on $N$ es estacionario.

\item[ii)] EL proceso de tiempos de recurrencia hacia adelante $B\left(t\right)$ es estacionario.


\item[iii)] $\esp\left[N\left(t\right)\right]=t/\mu$,


\item[iv)] $G\left(t\right)=F_{e}\left(t\right)=\frac{1}{\mu}\int_{0}^{t}\left[1-F\left(s\right)\right]ds$
\end{itemize}
Cuando estos enunciados son ciertos, $P\left\{B\left(t\right)\leq x\right\}=F_{e}\left(x\right)$, para $t,x\geq0$.

\end{Teo}

\begin{Note}
Una consecuencia del teorema anterior es que el Proceso Poisson es el \'unico proceso sin retardo que es estacionario.
\end{Note}

\begin{Coro}
El proceso de renovaci\'on $N\left(t\right)$ sin retardo, y cuyos tiempos de inter renonaci\'on tienen media finita, es estacionario si y s\'olo si es un proceso Poisson.

\end{Coro}





%___________________________________________________________________________________________
%
%\subsection*{Renewal and Regenerative Processes: Serfozo\cite{Serfozo}}
%___________________________________________________________________________________________
%
\begin{Def}%\label{Def.Tn}
Sean $0\leq T_{1}\leq T_{2}\leq \ldots$ son tiempos aleatorios infinitos en los cuales ocurren ciertos eventos. El n\'umero de tiempos $T_{n}$ en el intervalo $\left[0,t\right)$ es

\begin{eqnarray}
N\left(t\right)=\sum_{n=1}^{\infty}\indora\left(T_{n}\leq t\right),
\end{eqnarray}
para $t\geq0$.
\end{Def}

Si se consideran los puntos $T_{n}$ como elementos de $\rea_{+}$, y $N\left(t\right)$ es el n\'umero de puntos en $\rea$. El proceso denotado por $\left\{N\left(t\right):t\geq0\right\}$, denotado por $N\left(t\right)$, es un proceso puntual en $\rea_{+}$. Los $T_{n}$ son los tiempos de ocurrencia, el proceso puntual $N\left(t\right)$ es simple si su n\'umero de ocurrencias son distintas: $0<T_{1}<T_{2}<\ldots$ casi seguramente.

\begin{Def}
Un proceso puntual $N\left(t\right)$ es un proceso de renovaci\'on si los tiempos de interocurrencia $\xi_{n}=T_{n}-T_{n-1}$, para $n\geq1$, son independientes e identicamente distribuidos con distribuci\'on $F$, donde $F\left(0\right)=0$ y $T_{0}=0$. Los $T_{n}$ son llamados tiempos de renovaci\'on, referente a la independencia o renovaci\'on de la informaci\'on estoc\'astica en estos tiempos. Los $\xi_{n}$ son los tiempos de inter-renovaci\'on, y $N\left(t\right)$ es el n\'umero de renovaciones en el intervalo $\left[0,t\right)$
\end{Def}


\begin{Note}
Para definir un proceso de renovaci\'on para cualquier contexto, solamente hay que especificar una distribuci\'on $F$, con $F\left(0\right)=0$, para los tiempos de inter-renovaci\'on. La funci\'on $F$ en turno degune las otra variables aleatorias. De manera formal, existe un espacio de probabilidad y una sucesi\'on de variables aleatorias $\xi_{1},\xi_{2},\ldots$ definidas en este con distribuci\'on $F$. Entonces las otras cantidades son $T_{n}=\sum_{k=1}^{n}\xi_{k}$ y $N\left(t\right)=\sum_{n=1}^{\infty}\indora\left(T_{n}\leq t\right)$, donde $T_{n}\rightarrow\infty$ casi seguramente por la Ley Fuerte de los Grandes N\'umeros.
\end{Note}







Los tiempos $T_{n}$ est\'an relacionados con los conteos de $N\left(t\right)$ por

\begin{eqnarray*}
\left\{N\left(t\right)\geq n\right\}&=&\left\{T_{n}\leq t\right\}\\
T_{N\left(t\right)}\leq &t&<T_{N\left(t\right)+1},
\end{eqnarray*}

adem\'as $N\left(T_{n}\right)=n$, y 

\begin{eqnarray*}
N\left(t\right)=\max\left\{n:T_{n}\leq t\right\}=\min\left\{n:T_{n+1}>t\right\}
\end{eqnarray*}

Por propiedades de la convoluci\'on se sabe que

\begin{eqnarray*}
P\left\{T_{n}\leq t\right\}=F^{n\star}\left(t\right)
\end{eqnarray*}
que es la $n$-\'esima convoluci\'on de $F$. Entonces 

\begin{eqnarray*}
\left\{N\left(t\right)\geq n\right\}&=&\left\{T_{n}\leq t\right\}\\
P\left\{N\left(t\right)\leq n\right\}&=&1-F^{\left(n+1\right)\star}\left(t\right)
\end{eqnarray*}

Adem\'as usando el hecho de que $\esp\left[N\left(t\right)\right]=\sum_{n=1}^{\infty}P\left\{N\left(t\right)\geq n\right\}$
se tiene que

\begin{eqnarray*}
\esp\left[N\left(t\right)\right]=\sum_{n=1}^{\infty}F^{n\star}\left(t\right)
\end{eqnarray*}

\begin{Prop}
Para cada $t\geq0$, la funci\'on generadora de momentos $\esp\left[e^{\alpha N\left(t\right)}\right]$ existe para alguna $\alpha$ en una vecindad del 0, y de aqu\'i que $\esp\left[N\left(t\right)^{m}\right]<\infty$, para $m\geq1$.
\end{Prop}

\begin{Ejem}[\textbf{Proceso Poisson}]

Suponga que se tienen tiempos de inter-renovaci\'on \textit{i.i.d.} del proceso de renovaci\'on $N\left(t\right)$ tienen distribuci\'on exponencial $F\left(t\right)=q-e^{-\lambda t}$ con tasa $\lambda$. Entonces $N\left(t\right)$ es un proceso Poisson con tasa $\lambda$.

\end{Ejem}


\begin{Note}
Si el primer tiempo de renovaci\'on $\xi_{1}$ no tiene la misma distribuci\'on que el resto de las $\xi_{n}$, para $n\geq2$, a $N\left(t\right)$ se le llama Proceso de Renovaci\'on retardado, donde si $\xi$ tiene distribuci\'on $G$, entonces el tiempo $T_{n}$ de la $n$-\'esima renovaci\'on tiene distribuci\'on $G\star F^{\left(n-1\right)\star}\left(t\right)$
\end{Note}


\begin{Teo}
Para una constante $\mu\leq\infty$ ( o variable aleatoria), las siguientes expresiones son equivalentes:

\begin{eqnarray}
lim_{n\rightarrow\infty}n^{-1}T_{n}&=&\mu,\textrm{ c.s.}\\
lim_{t\rightarrow\infty}t^{-1}N\left(t\right)&=&1/\mu,\textrm{ c.s.}
\end{eqnarray}
\end{Teo}


Es decir, $T_{n}$ satisface la Ley Fuerte de los Grandes N\'umeros s\'i y s\'olo s\'i $N\left/t\right)$ la cumple.


\begin{Coro}[Ley Fuerte de los Grandes N\'umeros para Procesos de Renovaci\'on]
Si $N\left(t\right)$ es un proceso de renovaci\'on cuyos tiempos de inter-renovaci\'on tienen media $\mu\leq\infty$, entonces
\begin{eqnarray}
t^{-1}N\left(t\right)\rightarrow 1/\mu,\textrm{ c.s. cuando }t\rightarrow\infty.
\end{eqnarray}

\end{Coro}


Considerar el proceso estoc\'astico de valores reales $\left\{Z\left(t\right):t\geq0\right\}$ en el mismo espacio de probabilidad que $N\left(t\right)$

\begin{Def}
Para el proceso $\left\{Z\left(t\right):t\geq0\right\}$ se define la fluctuaci\'on m\'axima de $Z\left(t\right)$ en el intervalo $\left(T_{n-1},T_{n}\right]$:
\begin{eqnarray*}
M_{n}=\sup_{T_{n-1}<t\leq T_{n}}|Z\left(t\right)-Z\left(T_{n-1}\right)|
\end{eqnarray*}
\end{Def}

\begin{Teo}
Sup\'ongase que $n^{-1}T_{n}\rightarrow\mu$ c.s. cuando $n\rightarrow\infty$, donde $\mu\leq\infty$ es una constante o variable aleatoria. Sea $a$ una constante o variable aleatoria que puede ser infinita cuando $\mu$ es finita, y considere las expresiones l\'imite:
\begin{eqnarray}
lim_{n\rightarrow\infty}n^{-1}Z\left(T_{n}\right)&=&a,\textrm{ c.s.}\\
lim_{t\rightarrow\infty}t^{-1}Z\left(t\right)&=&a/\mu,\textrm{ c.s.}
\end{eqnarray}
La segunda expresi\'on implica la primera. Conversamente, la primera implica la segunda si el proceso $Z\left(t\right)$ es creciente, o si $lim_{n\rightarrow\infty}n^{-1}M_{n}=0$ c.s.
\end{Teo}

\begin{Coro}
Si $N\left(t\right)$ es un proceso de renovaci\'on, y $\left(Z\left(T_{n}\right)-Z\left(T_{n-1}\right),M_{n}\right)$, para $n\geq1$, son variables aleatorias independientes e id\'enticamente distribuidas con media finita, entonces,
\begin{eqnarray}
lim_{t\rightarrow\infty}t^{-1}Z\left(t\right)\rightarrow\frac{\esp\left[Z\left(T_{1}\right)-Z\left(T_{0}\right)\right]}{\esp\left[T_{1}\right]},\textrm{ c.s. cuando  }t\rightarrow\infty.
\end{eqnarray}
\end{Coro}


Sup\'ongase que $N\left(t\right)$ es un proceso de renovaci\'on con distribuci\'on $F$ con media finita $\mu$.

\begin{Def}
La funci\'on de renovaci\'on asociada con la distribuci\'on $F$, del proceso $N\left(t\right)$, es
\begin{eqnarray*}
U\left(t\right)=\sum_{n=1}^{\infty}F^{n\star}\left(t\right),\textrm{   }t\geq0,
\end{eqnarray*}
donde $F^{0\star}\left(t\right)=\indora\left(t\geq0\right)$.
\end{Def}


\begin{Prop}
Sup\'ongase que la distribuci\'on de inter-renovaci\'on $F$ tiene densidad $f$. Entonces $U\left(t\right)$ tambi\'en tiene densidad, para $t>0$, y es $U^{'}\left(t\right)=\sum_{n=0}^{\infty}f^{n\star}\left(t\right)$. Adem\'as
\begin{eqnarray*}
\prob\left\{N\left(t\right)>N\left(t-\right)\right\}=0\textrm{,   }t\geq0.
\end{eqnarray*}
\end{Prop}

\begin{Def}
La Transformada de Laplace-Stieljes de $F$ est\'a dada por

\begin{eqnarray*}
\hat{F}\left(\alpha\right)=\int_{\rea_{+}}e^{-\alpha t}dF\left(t\right)\textrm{,  }\alpha\geq0.
\end{eqnarray*}
\end{Def}

Entonces

\begin{eqnarray*}
\hat{U}\left(\alpha\right)=\sum_{n=0}^{\infty}\hat{F^{n\star}}\left(\alpha\right)=\sum_{n=0}^{\infty}\hat{F}\left(\alpha\right)^{n}=\frac{1}{1-\hat{F}\left(\alpha\right)}.
\end{eqnarray*}


\begin{Prop}
La Transformada de Laplace $\hat{U}\left(\alpha\right)$ y $\hat{F}\left(\alpha\right)$ determina una a la otra de manera \'unica por la relaci\'on $\hat{U}\left(\alpha\right)=\frac{1}{1-\hat{F}\left(\alpha\right)}$.
\end{Prop}


\begin{Note}
Un proceso de renovaci\'on $N\left(t\right)$ cuyos tiempos de inter-renovaci\'on tienen media finita, es un proceso Poisson con tasa $\lambda$ si y s\'olo s\'i $\esp\left[U\left(t\right)\right]=\lambda t$, para $t\geq0$.
\end{Note}


\begin{Teo}
Sea $N\left(t\right)$ un proceso puntual simple con puntos de localizaci\'on $T_{n}$ tal que $\eta\left(t\right)=\esp\left[N\left(\right)\right]$ es finita para cada $t$. Entonces para cualquier funci\'on $f:\rea_{+}\rightarrow\rea$,
\begin{eqnarray*}
\esp\left[\sum_{n=1}^{N\left(\right)}f\left(T_{n}\right)\right]=\int_{\left(0,t\right]}f\left(s\right)d\eta\left(s\right)\textrm{,  }t\geq0,
\end{eqnarray*}
suponiendo que la integral exista. Adem\'as si $X_{1},X_{2},\ldots$ son variables aleatorias definidas en el mismo espacio de probabilidad que el proceso $N\left(t\right)$ tal que $\esp\left[X_{n}|T_{n}=s\right]=f\left(s\right)$, independiente de $n$. Entonces
\begin{eqnarray*}
\esp\left[\sum_{n=1}^{N\left(t\right)}X_{n}\right]=\int_{\left(0,t\right]}f\left(s\right)d\eta\left(s\right)\textrm{,  }t\geq0,
\end{eqnarray*} 
suponiendo que la integral exista. 
\end{Teo}

\begin{Coro}[Identidad de Wald para Renovaciones]
Para el proceso de renovaci\'on $N\left(t\right)$,
\begin{eqnarray*}
\esp\left[T_{N\left(t\right)+1}\right]=\mu\esp\left[N\left(t\right)+1\right]\textrm{,  }t\geq0,
\end{eqnarray*}  
\end{Coro}


\begin{Def}
Sea $h\left(t\right)$ funci\'on de valores reales en $\rea$ acotada en intervalos finitos e igual a cero para $t<0$ La ecuaci\'on de renovaci\'on para $h\left(t\right)$ y la distribuci\'on $F$ es

\begin{eqnarray}%\label{Ec.Renovacion}
H\left(t\right)=h\left(t\right)+\int_{\left[0,t\right]}H\left(t-s\right)dF\left(s\right)\textrm{,    }t\geq0,
\end{eqnarray}
donde $H\left(t\right)$ es una funci\'on de valores reales. Esto es $H=h+F\star H$. Decimos que $H\left(t\right)$ es soluci\'on de esta ecuaci\'on si satisface la ecuaci\'on, y es acotada en intervalos finitos e iguales a cero para $t<0$.
\end{Def}

\begin{Prop}
La funci\'on $U\star h\left(t\right)$ es la \'unica soluci\'on de la ecuaci\'on de renovaci\'on (\ref{Ec.Renovacion}).
\end{Prop}

\begin{Teo}[Teorema Renovaci\'on Elemental]
\begin{eqnarray*}
t^{-1}U\left(t\right)\rightarrow 1/\mu\textrm{,    cuando }t\rightarrow\infty.
\end{eqnarray*}
\end{Teo}



Sup\'ongase que $N\left(t\right)$ es un proceso de renovaci\'on con distribuci\'on $F$ con media finita $\mu$.

\begin{Def}
La funci\'on de renovaci\'on asociada con la distribuci\'on $F$, del proceso $N\left(t\right)$, es
\begin{eqnarray*}
U\left(t\right)=\sum_{n=1}^{\infty}F^{n\star}\left(t\right),\textrm{   }t\geq0,
\end{eqnarray*}
donde $F^{0\star}\left(t\right)=\indora\left(t\geq0\right)$.
\end{Def}


\begin{Prop}
Sup\'ongase que la distribuci\'on de inter-renovaci\'on $F$ tiene densidad $f$. Entonces $U\left(t\right)$ tambi\'en tiene densidad, para $t>0$, y es $U^{'}\left(t\right)=\sum_{n=0}^{\infty}f^{n\star}\left(t\right)$. Adem\'as
\begin{eqnarray*}
\prob\left\{N\left(t\right)>N\left(t-\right)\right\}=0\textrm{,   }t\geq0.
\end{eqnarray*}
\end{Prop}

\begin{Def}
La Transformada de Laplace-Stieljes de $F$ est\'a dada por

\begin{eqnarray*}
\hat{F}\left(\alpha\right)=\int_{\rea_{+}}e^{-\alpha t}dF\left(t\right)\textrm{,  }\alpha\geq0.
\end{eqnarray*}
\end{Def}

Entonces

\begin{eqnarray*}
\hat{U}\left(\alpha\right)=\sum_{n=0}^{\infty}\hat{F^{n\star}}\left(\alpha\right)=\sum_{n=0}^{\infty}\hat{F}\left(\alpha\right)^{n}=\frac{1}{1-\hat{F}\left(\alpha\right)}.
\end{eqnarray*}


\begin{Prop}
La Transformada de Laplace $\hat{U}\left(\alpha\right)$ y $\hat{F}\left(\alpha\right)$ determina una a la otra de manera \'unica por la relaci\'on $\hat{U}\left(\alpha\right)=\frac{1}{1-\hat{F}\left(\alpha\right)}$.
\end{Prop}


\begin{Note}
Un proceso de renovaci\'on $N\left(t\right)$ cuyos tiempos de inter-renovaci\'on tienen media finita, es un proceso Poisson con tasa $\lambda$ si y s\'olo s\'i $\esp\left[U\left(t\right)\right]=\lambda t$, para $t\geq0$.
\end{Note}


\begin{Teo}
Sea $N\left(t\right)$ un proceso puntual simple con puntos de localizaci\'on $T_{n}$ tal que $\eta\left(t\right)=\esp\left[N\left(\right)\right]$ es finita para cada $t$. Entonces para cualquier funci\'on $f:\rea_{+}\rightarrow\rea$,
\begin{eqnarray*}
\esp\left[\sum_{n=1}^{N\left(\right)}f\left(T_{n}\right)\right]=\int_{\left(0,t\right]}f\left(s\right)d\eta\left(s\right)\textrm{,  }t\geq0,
\end{eqnarray*}
suponiendo que la integral exista. Adem\'as si $X_{1},X_{2},\ldots$ son variables aleatorias definidas en el mismo espacio de probabilidad que el proceso $N\left(t\right)$ tal que $\esp\left[X_{n}|T_{n}=s\right]=f\left(s\right)$, independiente de $n$. Entonces
\begin{eqnarray*}
\esp\left[\sum_{n=1}^{N\left(t\right)}X_{n}\right]=\int_{\left(0,t\right]}f\left(s\right)d\eta\left(s\right)\textrm{,  }t\geq0,
\end{eqnarray*} 
suponiendo que la integral exista. 
\end{Teo}

\begin{Coro}[Identidad de Wald para Renovaciones]
Para el proceso de renovaci\'on $N\left(t\right)$,
\begin{eqnarray*}
\esp\left[T_{N\left(t\right)+1}\right]=\mu\esp\left[N\left(t\right)+1\right]\textrm{,  }t\geq0,
\end{eqnarray*}  
\end{Coro}


\begin{Def}
Sea $h\left(t\right)$ funci\'on de valores reales en $\rea$ acotada en intervalos finitos e igual a cero para $t<0$ La ecuaci\'on de renovaci\'on para $h\left(t\right)$ y la distribuci\'on $F$ es

\begin{eqnarray}%\label{Ec.Renovacion}
H\left(t\right)=h\left(t\right)+\int_{\left[0,t\right]}H\left(t-s\right)dF\left(s\right)\textrm{,    }t\geq0,
\end{eqnarray}
donde $H\left(t\right)$ es una funci\'on de valores reales. Esto es $H=h+F\star H$. Decimos que $H\left(t\right)$ es soluci\'on de esta ecuaci\'on si satisface la ecuaci\'on, y es acotada en intervalos finitos e iguales a cero para $t<0$.
\end{Def}

\begin{Prop}
La funci\'on $U\star h\left(t\right)$ es la \'unica soluci\'on de la ecuaci\'on de renovaci\'on (\ref{Ec.Renovacion}).
\end{Prop}

\begin{Teo}[Teorema Renovaci\'on Elemental]
\begin{eqnarray*}
t^{-1}U\left(t\right)\rightarrow 1/\mu\textrm{,    cuando }t\rightarrow\infty.
\end{eqnarray*}
\end{Teo}


\begin{Note} Una funci\'on $h:\rea_{+}\rightarrow\rea$ es Directamente Riemann Integrable en los siguientes casos:
\begin{itemize}
\item[a)] $h\left(t\right)\geq0$ es decreciente y Riemann Integrable.
\item[b)] $h$ es continua excepto posiblemente en un conjunto de Lebesgue de medida 0, y $|h\left(t\right)|\leq b\left(t\right)$, donde $b$ es DRI.
\end{itemize}
\end{Note}

\begin{Teo}[Teorema Principal de Renovaci\'on]
Si $F$ es no aritm\'etica y $h\left(t\right)$ es Directamente Riemann Integrable (DRI), entonces

\begin{eqnarray*}
lim_{t\rightarrow\infty}U\star h=\frac{1}{\mu}\int_{\rea_{+}}h\left(s\right)ds.
\end{eqnarray*}
\end{Teo}

\begin{Prop}
Cualquier funci\'on $H\left(t\right)$ acotada en intervalos finitos y que es 0 para $t<0$ puede expresarse como
\begin{eqnarray*}
H\left(t\right)=U\star h\left(t\right)\textrm{,  donde }h\left(t\right)=H\left(t\right)-F\star H\left(t\right)
\end{eqnarray*}
\end{Prop}

\begin{Def}
Un proceso estoc\'astico $X\left(t\right)$ es crudamente regenerativo en un tiempo aleatorio positivo $T$ si
\begin{eqnarray*}
\esp\left[X\left(T+t\right)|T\right]=\esp\left[X\left(t\right)\right]\textrm{, para }t\geq0,\end{eqnarray*}
y con las esperanzas anteriores finitas.
\end{Def}

\begin{Prop}
Sup\'ongase que $X\left(t\right)$ es un proceso crudamente regenerativo en $T$, que tiene distribuci\'on $F$. Si $\esp\left[X\left(t\right)\right]$ es acotado en intervalos finitos, entonces
\begin{eqnarray*}
\esp\left[X\left(t\right)\right]=U\star h\left(t\right)\textrm{,  donde }h\left(t\right)=\esp\left[X\left(t\right)\indora\left(T>t\right)\right].
\end{eqnarray*}
\end{Prop}

\begin{Teo}[Regeneraci\'on Cruda]
Sup\'ongase que $X\left(t\right)$ es un proceso con valores positivo crudamente regenerativo en $T$, y def\'inase $M=\sup\left\{|X\left(t\right)|:t\leq T\right\}$. Si $T$ es no aritm\'etico y $M$ y $MT$ tienen media finita, entonces
\begin{eqnarray*}
lim_{t\rightarrow\infty}\esp\left[X\left(t\right)\right]=\frac{1}{\mu}\int_{\rea_{+}}h\left(s\right)ds,
\end{eqnarray*}
donde $h\left(t\right)=\esp\left[X\left(t\right)\indora\left(T>t\right)\right]$.
\end{Teo}


\begin{Note} Una funci\'on $h:\rea_{+}\rightarrow\rea$ es Directamente Riemann Integrable en los siguientes casos:
\begin{itemize}
\item[a)] $h\left(t\right)\geq0$ es decreciente y Riemann Integrable.
\item[b)] $h$ es continua excepto posiblemente en un conjunto de Lebesgue de medida 0, y $|h\left(t\right)|\leq b\left(t\right)$, donde $b$ es DRI.
\end{itemize}
\end{Note}

\begin{Teo}[Teorema Principal de Renovaci\'on]
Si $F$ es no aritm\'etica y $h\left(t\right)$ es Directamente Riemann Integrable (DRI), entonces

\begin{eqnarray*}
lim_{t\rightarrow\infty}U\star h=\frac{1}{\mu}\int_{\rea_{+}}h\left(s\right)ds.
\end{eqnarray*}
\end{Teo}

\begin{Prop}
Cualquier funci\'on $H\left(t\right)$ acotada en intervalos finitos y que es 0 para $t<0$ puede expresarse como
\begin{eqnarray*}
H\left(t\right)=U\star h\left(t\right)\textrm{,  donde }h\left(t\right)=H\left(t\right)-F\star H\left(t\right)
\end{eqnarray*}
\end{Prop}

\begin{Def}
Un proceso estoc\'astico $X\left(t\right)$ es crudamente regenerativo en un tiempo aleatorio positivo $T$ si
\begin{eqnarray*}
\esp\left[X\left(T+t\right)|T\right]=\esp\left[X\left(t\right)\right]\textrm{, para }t\geq0,\end{eqnarray*}
y con las esperanzas anteriores finitas.
\end{Def}

\begin{Prop}
Sup\'ongase que $X\left(t\right)$ es un proceso crudamente regenerativo en $T$, que tiene distribuci\'on $F$. Si $\esp\left[X\left(t\right)\right]$ es acotado en intervalos finitos, entonces
\begin{eqnarray*}
\esp\left[X\left(t\right)\right]=U\star h\left(t\right)\textrm{,  donde }h\left(t\right)=\esp\left[X\left(t\right)\indora\left(T>t\right)\right].
\end{eqnarray*}
\end{Prop}

\begin{Teo}[Regeneraci\'on Cruda]
Sup\'ongase que $X\left(t\right)$ es un proceso con valores positivo crudamente regenerativo en $T$, y def\'inase $M=\sup\left\{|X\left(t\right)|:t\leq T\right\}$. Si $T$ es no aritm\'etico y $M$ y $MT$ tienen media finita, entonces
\begin{eqnarray*}
lim_{t\rightarrow\infty}\esp\left[X\left(t\right)\right]=\frac{1}{\mu}\int_{\rea_{+}}h\left(s\right)ds,
\end{eqnarray*}
donde $h\left(t\right)=\esp\left[X\left(t\right)\indora\left(T>t\right)\right]$.
\end{Teo}

\begin{Def}
Para el proceso $\left\{\left(N\left(t\right),X\left(t\right)\right):t\geq0\right\}$, sus trayectoria muestrales en el intervalo de tiempo $\left[T_{n-1},T_{n}\right)$ est\'an descritas por
\begin{eqnarray*}
\zeta_{n}=\left(\xi_{n},\left\{X\left(T_{n-1}+t\right):0\leq t<\xi_{n}\right\}\right)
\end{eqnarray*}
Este $\zeta_{n}$ es el $n$-\'esimo segmento del proceso. El proceso es regenerativo sobre los tiempos $T_{n}$ si sus segmentos $\zeta_{n}$ son independientes e id\'enticamennte distribuidos.
\end{Def}


\begin{Note}
Si $\tilde{X}\left(t\right)$ con espacio de estados $\tilde{S}$ es regenerativo sobre $T_{n}$, entonces $X\left(t\right)=f\left(\tilde{X}\left(t\right)\right)$ tambi\'en es regenerativo sobre $T_{n}$, para cualquier funci\'on $f:\tilde{S}\rightarrow S$.
\end{Note}

\begin{Note}
Los procesos regenerativos son crudamente regenerativos, pero no al rev\'es.
\end{Note}


\begin{Note}
Un proceso estoc\'astico a tiempo continuo o discreto es regenerativo si existe un proceso de renovaci\'on  tal que los segmentos del proceso entre tiempos de renovaci\'on sucesivos son i.i.d., es decir, para $\left\{X\left(t\right):t\geq0\right\}$ proceso estoc\'astico a tiempo continuo con espacio de estados $S$, espacio m\'etrico.
\end{Note}

Para $\left\{X\left(t\right):t\geq0\right\}$ Proceso Estoc\'astico a tiempo continuo con estado de espacios $S$, que es un espacio m\'etrico, con trayectorias continuas por la derecha y con l\'imites por la izquierda c.s. Sea $N\left(t\right)$ un proceso de renovaci\'on en $\rea_{+}$ definido en el mismo espacio de probabilidad que $X\left(t\right)$, con tiempos de renovaci\'on $T$ y tiempos de inter-renovaci\'on $\xi_{n}=T_{n}-T_{n-1}$, con misma distribuci\'on $F$ de media finita $\mu$.



\begin{Def}
Para el proceso $\left\{\left(N\left(t\right),X\left(t\right)\right):t\geq0\right\}$, sus trayectoria muestrales en el intervalo de tiempo $\left[T_{n-1},T_{n}\right)$ est\'an descritas por
\begin{eqnarray*}
\zeta_{n}=\left(\xi_{n},\left\{X\left(T_{n-1}+t\right):0\leq t<\xi_{n}\right\}\right)
\end{eqnarray*}
Este $\zeta_{n}$ es el $n$-\'esimo segmento del proceso. El proceso es regenerativo sobre los tiempos $T_{n}$ si sus segmentos $\zeta_{n}$ son independientes e id\'enticamennte distribuidos.
\end{Def}

\begin{Note}
Un proceso regenerativo con media de la longitud de ciclo finita es llamado positivo recurrente.
\end{Note}

\begin{Teo}[Procesos Regenerativos]
Suponga que el proceso
\end{Teo}


\begin{Def}[Renewal Process Trinity]
Para un proceso de renovaci\'on $N\left(t\right)$, los siguientes procesos proveen de informaci\'on sobre los tiempos de renovaci\'on.
\begin{itemize}
\item $A\left(t\right)=t-T_{N\left(t\right)}$, el tiempo de recurrencia hacia atr\'as al tiempo $t$, que es el tiempo desde la \'ultima renovaci\'on para $t$.

\item $B\left(t\right)=T_{N\left(t\right)+1}-t$, el tiempo de recurrencia hacia adelante al tiempo $t$, residual del tiempo de renovaci\'on, que es el tiempo para la pr\'oxima renovaci\'on despu\'es de $t$.

\item $L\left(t\right)=\xi_{N\left(t\right)+1}=A\left(t\right)+B\left(t\right)$, la longitud del intervalo de renovaci\'on que contiene a $t$.
\end{itemize}
\end{Def}

\begin{Note}
El proceso tridimensional $\left(A\left(t\right),B\left(t\right),L\left(t\right)\right)$ es regenerativo sobre $T_{n}$, y por ende cada proceso lo es. Cada proceso $A\left(t\right)$ y $B\left(t\right)$ son procesos de MArkov a tiempo continuo con trayectorias continuas por partes en el espacio de estados $\rea_{+}$. Una expresi\'on conveniente para su distribuci\'on conjunta es, para $0\leq x<t,y\geq0$
\begin{equation}\label{NoRenovacion}
P\left\{A\left(t\right)>x,B\left(t\right)>y\right\}=
P\left\{N\left(t+y\right)-N\left((t-x)\right)=0\right\}
\end{equation}
\end{Note}

\begin{Ejem}[Tiempos de recurrencia Poisson]
Si $N\left(t\right)$ es un proceso Poisson con tasa $\lambda$, entonces de la expresi\'on (\ref{NoRenovacion}) se tiene que

\begin{eqnarray*}
\begin{array}{lc}
P\left\{A\left(t\right)>x,B\left(t\right)>y\right\}=e^{-\lambda\left(x+y\right)},&0\leq x<t,y\geq0,
\end{array}
\end{eqnarray*}
que es la probabilidad Poisson de no renovaciones en un intervalo de longitud $x+y$.

\end{Ejem}

\begin{Note}
Una cadena de Markov erg\'odica tiene la propiedad de ser estacionaria si la distribución de su estado al tiempo $0$ es su distribuci\'on estacionaria.
\end{Note}


\begin{Def}
Un proceso estoc\'astico a tiempo continuo $\left\{X\left(t\right):t\geq0\right\}$ en un espacio general es estacionario si sus distribuciones finito dimensionales son invariantes bajo cualquier  traslado: para cada $0\leq s_{1}<s_{2}<\cdots<s_{k}$ y $t\geq0$,
\begin{eqnarray*}
\left(X\left(s_{1}+t\right),\ldots,X\left(s_{k}+t\right)\right)=_{d}\left(X\left(s_{1}\right),\ldots,X\left(s_{k}\right)\right).
\end{eqnarray*}
\end{Def}

\begin{Note}
Un proceso de Markov es estacionario si $X\left(t\right)=_{d}X\left(0\right)$, $t\geq0$.
\end{Note}

Considerese el proceso $N\left(t\right)=\sum_{n}\indora\left(\tau_{n}\leq t\right)$ en $\rea_{+}$, con puntos $0<\tau_{1}<\tau_{2}<\cdots$.

\begin{Prop}
Si $N$ es un proceso puntual estacionario y $\esp\left[N\left(1\right)\right]<\infty$, entonces $\esp\left[N\left(t\right)\right]=t\esp\left[N\left(1\right)\right]$, $t\geq0$

\end{Prop}

\begin{Teo}
Los siguientes enunciados son equivalentes
\begin{itemize}
\item[i)] El proceso retardado de renovaci\'on $N$ es estacionario.

\item[ii)] EL proceso de tiempos de recurrencia hacia adelante $B\left(t\right)$ es estacionario.


\item[iii)] $\esp\left[N\left(t\right)\right]=t/\mu$,


\item[iv)] $G\left(t\right)=F_{e}\left(t\right)=\frac{1}{\mu}\int_{0}^{t}\left[1-F\left(s\right)\right]ds$
\end{itemize}
Cuando estos enunciados son ciertos, $P\left\{B\left(t\right)\leq x\right\}=F_{e}\left(x\right)$, para $t,x\geq0$.

\end{Teo}

\begin{Note}
Una consecuencia del teorema anterior es que el Proceso Poisson es el \'unico proceso sin retardo que es estacionario.
\end{Note}

\begin{Coro}
El proceso de renovaci\'on $N\left(t\right)$ sin retardo, y cuyos tiempos de inter renonaci\'on tienen media finita, es estacionario si y s\'olo si es un proceso Poisson.

\end{Coro}




%______________________________________________________________________
\section{Resultados para Procesos de Salida}
%______________________________________________________________________
En Sigman, Thorison y Wolff \cite{Sigman2} prueban que para la existencia de un una sucesi\'on infinita no decreciente de tiempos de regeneraci\'on $\tau_{1}\leq\tau_{2}\leq\cdots$ en los cuales el proceso se regenera, basta un tiempo de regeneraci\'on $R_{1}$, donde $R_{j}=\tau_{j}-\tau_{j-1}$. Para tal efecto se requiere la existencia de un espacio de probabilidad $\left(\Omega,\mathcal{F},\prob\right)$, y proceso estoc\'astico $\textit{X}=\left\{X\left(t\right):t\geq0\right\}$ con espacio de estados $\left(S,\mathcal{R}\right)$, con $\mathcal{R}$ $\sigma$-\'algebra.

\begin{Prop}
Si existe una variable aleatoria no negativa $R_{1}$ tal que $\theta_{R\footnotesize{1}}X=_{D}X$, entonces $\left(\Omega,\mathcal{F},\prob\right)$ puede extenderse para soportar una sucesi\'on estacionaria de variables aleatorias $R=\left\{R_{k}:k\geq1\right\}$, tal que para $k\geq1$,
\begin{eqnarray*}
\theta_{k}\left(X,R\right)=_{D}\left(X,R\right).
\end{eqnarray*}

Adem\'as, para $k\geq1$, $\theta_{k}R$ es condicionalmente independiente de $\left(X,R_{1},\ldots,R_{k}\right)$, dado $\theta_{\tau k}X$.

\end{Prop}


\begin{itemize}
\item Doob en 1953 demostr\'o que el estado estacionario de un proceso de partida en un sistema de espera $M/G/\infty$, es Poisson con la misma tasa que el proceso de arribos.

\item Burke en 1968, fue el primero en demostrar que el estado estacionario de un proceso de salida de una cola $M/M/s$ es un proceso Poisson.

\item Disney en 1973 obtuvo el siguiente resultado:

\begin{Teo}
Para el sistema de espera $M/G/1/L$ con disciplina FIFO, el proceso $\textbf{I}$ es un proceso de renovaci\'on si y s\'olo si el proceso denominado longitud de la cola es estacionario y se cumple cualquiera de los siguientes casos:

\begin{itemize}
\item[a)] Los tiempos de servicio son identicamente cero;
\item[b)] $L=0$, para cualquier proceso de servicio $S$;
\item[c)] $L=1$ y $G=D$;
\item[d)] $L=\infty$ y $G=M$.
\end{itemize}
En estos casos, respectivamente, las distribuciones de interpartida $P\left\{T_{n+1}-T_{n}\leq t\right\}$ son


\begin{itemize}
\item[a)] $1-e^{-\lambda t}$, $t\geq0$;
\item[b)] $1-e^{-\lambda t}*F\left(t\right)$, $t\geq0$;
\item[c)] $1-e^{-\lambda t}*\indora_{d}\left(t\right)$, $t\geq0$;
\item[d)] $1-e^{-\lambda t}*F\left(t\right)$, $t\geq0$.
\end{itemize}
\end{Teo}


\item Finch (1959) mostr\'o que para los sistemas $M/G/1/L$, con $1\leq L\leq \infty$ con distribuciones de servicio dos veces diferenciable, solamente el sistema $M/M/1/\infty$ tiene proceso de salida de renovaci\'on estacionario.

\item King (1971) demostro que un sistema de colas estacionario $M/G/1/1$ tiene sus tiempos de interpartida sucesivas $D_{n}$ y $D_{n+1}$ son independientes, si y s\'olo si, $G=D$, en cuyo caso le proceso de salida es de renovaci\'on.

\item Disney (1973) demostr\'o que el \'unico sistema estacionario $M/G/1/L$, que tiene proceso de salida de renovaci\'on  son los sistemas $M/M/1$ y $M/D/1/1$.



\item El siguiente resultado es de Disney y Koning (1985)
\begin{Teo}
En un sistema de espera $M/G/s$, el estado estacionario del proceso de salida es un proceso Poisson para cualquier distribuci\'on de los tiempos de servicio si el sistema tiene cualquiera de las siguientes cuatro propiedades.

\begin{itemize}
\item[a)] $s=\infty$
\item[b)] La disciplina de servicio es de procesador compartido.
\item[c)] La disciplina de servicio es LCFS y preemptive resume, esto se cumple para $L<\infty$
\item[d)] $G=M$.
\end{itemize}

\end{Teo}

\item El siguiente resultado es de Alamatsaz (1983)

\begin{Teo}
En cualquier sistema de colas $GI/G/1/L$ con $1\leq L<\infty$ y distribuci\'on de interarribos $A$ y distribuci\'on de los tiempos de servicio $B$, tal que $A\left(0\right)=0$, $A\left(t\right)\left(1-B\left(t\right)\right)>0$ para alguna $t>0$ y $B\left(t\right)$ para toda $t>0$, es imposible que el proceso de salida estacionario sea de renovaci\'on.
\end{Teo}

\end{itemize}

Estos resultados aparecen en Daley (1968) \cite{Daley68} para $\left\{T_{n}\right\}$ intervalos de inter-arribo, $\left\{D_{n}\right\}$ intervalos de inter-salida y $\left\{S_{n}\right\}$ tiempos de servicio.

\begin{itemize}
\item Si el proceso $\left\{T_{n}\right\}$ es Poisson, el proceso $\left\{D_{n}\right\}$ es no correlacionado si y s\'olo si es un proceso Poisso, lo cual ocurre si y s\'olo si $\left\{S_{n}\right\}$ son exponenciales negativas.

\item Si $\left\{S_{n}\right\}$ son exponenciales negativas, $\left\{D_{n}\right\}$ es un proceso de renovaci\'on  si y s\'olo si es un proceso Poisson, lo cual ocurre si y s\'olo si $\left\{T_{n}\right\}$ es un proceso Poisson.

\item $\esp\left(D_{n}\right)=\esp\left(T_{n}\right)$.

\item Para un sistema de visitas $GI/M/1$ se tiene el siguiente teorema:

\begin{Teo}
En un sistema estacionario $GI/M/1$ los intervalos de interpartida tienen
\begin{eqnarray*}
\esp\left(e^{-\theta D_{n}}\right)&=&\mu\left(\mu+\theta\right)^{-1}\left[\delta\theta
-\mu\left(1-\delta\right)\alpha\left(\theta\right)\right]
\left[\theta-\mu\left(1-\delta\right)^{-1}\right]\\
\alpha\left(\theta\right)&=&\esp\left[e^{-\theta T_{0}}\right]\\
var\left(D_{n}\right)&=&var\left(T_{0}\right)-\left(\tau^{-1}-\delta^{-1}\right)
2\delta\left(\esp\left(S_{0}\right)\right)^{2}\left(1-\delta\right)^{-1}.
\end{eqnarray*}
\end{Teo}



\begin{Teo}
El proceso de salida de un sistema de colas estacionario $GI/M/1$ es un proceso de renovaci\'on si y s\'olo si el proceso de entrada es un proceso Poisson, en cuyo caso el proceso de salida es un proceso Poisson.
\end{Teo}


\begin{Teo}
Los intervalos de interpartida $\left\{D_{n}\right\}$ de un sistema $M/G/1$ estacionario son no correlacionados si y s\'olo si la distribuci\'on de los tiempos de servicio es exponencial negativa, es decir, el sistema es de tipo  $M/M/1$.

\end{Teo}



\end{itemize}





\begin{thebibliography}{99}

\bibitem{ISL}
James, G., Witten, D., Hastie, T., and Tibshirani, R. (2013). \textit{An Introduction to Statistical Learning: with Applications in R}. Springer.

\bibitem{Logistic}
Hosmer, D. W., Lemeshow, S., and Sturdivant, R. X. (2013). \textit{Applied Logistic Regression} (3rd ed.). Wiley.

\bibitem{PatternRecognition}
Bishop, C. M. (2006). \textit{Pattern Recognition and Machine Learning}. Springer.

\bibitem{Harrell}
Harrell, F. E. (2015). \textit{Regression Modeling Strategies: With Applications to Linear Models, Logistic and Ordinal Regression, and Survival Analysis}. Springer.

\bibitem{RDocumentation}
R Documentation and Tutorials: \url{https://cran.r-project.org/manuals.html}

\bibitem{RBlogger}
Tutorials on R-bloggers: \url{https://www.r-bloggers.com/}

\bibitem{CourseraML}
Coursera: \textit{Machine Learning} by Andrew Ng.

\bibitem{edXDS}
edX: \textit{Data Science and Machine Learning Essentials} by Microsoft.

% Libros adicionales
\bibitem{Ross}
Ross, S. M. (2014). \textit{Introduction to Probability and Statistics for Engineers and Scientists}. Academic Press.

\bibitem{DeGroot}
DeGroot, M. H., and Schervish, M. J. (2012). \textit{Probability and Statistics} (4th ed.). Pearson.

\bibitem{Hogg}
Hogg, R. V., McKean, J., and Craig, A. T. (2019). \textit{Introduction to Mathematical Statistics} (8th ed.). Pearson.

\bibitem{Kleinbaum}
Kleinbaum, D. G., and Klein, M. (2010). \textit{Logistic Regression: A Self-Learning Text} (3rd ed.). Springer.

% Artículos y tutoriales adicionales
\bibitem{Wasserman}
Wasserman, L. (2004). \textit{All of Statistics: A Concise Course in Statistical Inference}. Springer.

\bibitem{KhanAcademy}
Probability and Statistics Tutorials on Khan Academy: \url{https://www.khanacademy.org/math/statistics-probability}

\bibitem{OnlineStatBook}
Online Statistics Education: \url{http://onlinestatbook.com/}

\bibitem{Peng}
Peng, C. Y. J., Lee, K. L., and Ingersoll, G. M. (2002). \textit{An Introduction to Logistic Regression Analysis and Reporting}. The Journal of Educational Research.

\bibitem{Agresti}
Agresti, A. (2007). \textit{An Introduction to Categorical Data Analysis} (2nd ed.). Wiley.

\bibitem{Han}
Han, J., Pei, J., and Kamber, M. (2011). \textit{Data Mining: Concepts and Techniques}. Morgan Kaufmann.

\bibitem{TowardsDataScience}
Data Cleaning and Preprocessing on Towards Data Science: \url{https://towardsdatascience.com/data-cleaning-and-preprocessing}

\bibitem{Molinaro}
Molinaro, A. M., Simon, R., and Pfeiffer, R. M. (2005). \textit{Prediction error estimation: a comparison of resampling methods}. Bioinformatics.

\bibitem{EvaluatingModels}
Evaluating Machine Learning Models on Towards Data Science: \url{https://towardsdatascience.com/evaluating-machine-learning-models}

\bibitem{LogisticRegressionGuide}
Practical Guide to Logistic Regression in R on Towards Data Science: \url{https://towardsdatascience.com/practical-guide-to-logistic-regression-in-r}

% Cursos en línea adicionales
\bibitem{CourseraStatistics}
Coursera: \textit{Statistics with R} by Duke University.

\bibitem{edXProbability}
edX: \textit{Data Science: Probability} by Harvard University.

\bibitem{CourseraLogistic}
Coursera: \textit{Logistic Regression} by Stanford University.

\bibitem{edXInference}
edX: \textit{Data Science: Inference and Modeling} by Harvard University.

\bibitem{CourseraWrangling}
Coursera: \textit{Data Science: Wrangling and Cleaning} by Johns Hopkins University.

\bibitem{edXRBasics}
edX: \textit{Data Science: R Basics} by Harvard University.

\bibitem{CourseraRegression}
Coursera: \textit{Regression Models} by Johns Hopkins University.

\bibitem{edXStatInference}
edX: \textit{Data Science: Statistical Inference} by Harvard University.

\bibitem{SurvivalAnalysis}
An Introduction to Survival Analysis on Towards Data Science: \url{https://towardsdatascience.com/an-introduction-to-survival-analysis}

\bibitem{MultinomialLogistic}
Multinomial Logistic Regression on DataCamp: \url{https://www.datacamp.com/community/tutorials/multinomial-logistic-regression-R}

\bibitem{CourseraSurvival}
Coursera: \textit{Survival Analysis} by Johns Hopkins University.

\bibitem{edXHighthroughput}
edX: \textit{Data Science: Statistical Inference and Modeling for High-throughput Experiments} by Harvard University.

\end{thebibliography}


\end{document}



\chapter{Procesos Estocasticos}

\section{Procesos Estocásticos: Introducción}

\begin{Def}
Sea $\left(\Omega,\mathcal{F},\prob\right)$ un espacio de probabilidad y $\mathbf{E}$ un conjunto no vacío, finito o numerable. Una sucesión de variables aleatorias $\left\{X_{n}:\Omega\rightarrow\mathbf{E},n\geq0\right\}$ se le llama \textit{Cadena de Markov} con espacio de estados $\mathbf{E}$ si satisface la condición de Markov, esto es, si para todo $n\geq1$ y toda sucesión $x_{0},x_{1},\ldots,x_{n},x,y\in\mathbf{E}$ se cumple que 

\begin{equation}
P\left\{X_{n}=y|X_{n-1}=x,\ldots,X_{0}=x_{0}\right\}=P\left\{X_{n}=x_{n}|X_{n-1}=x_{n-1}\right\}
\end{equation}
La distribución de $X_{0}$ se llama distribución inicial y se denotará por $\pi$.
\end{Def}

Las probabilidades condicionales $P\left\{X_{n}=y|X_{n-1}=x\right\}$ se les llama \textit{probabilidades condicionales}


En este trabajo se considerarán solamente aquellas cadenas de Markov con probabilidades de transición estacionarias, es decir, aquellas que no dependen del valor de $n$ (se dice que es una cadena homogénea), es decir, cuando se diga $X_{n},n\geq0$ es cadena de Markov, se entiende que es una sucesión de variables aleatorias que satisfacen la propiedad de Markov y que tienen probabilidades de transición estacionarias.

\begin{Note}
Para una cadena de Markov Homogénea se tiene la siguiente denotación
\begin{equation}
P\left\{X_{n}=y|X_{n-1}=x\right\}=P_{x,y}
\end{equation}
\end{Note}

\begin{Note}
Para $m\geq1$ se denotará por $P^{(m)}_{x,y}$ a $P\left\{X_{n+m}=y|X_{n}=x\right\}$, que significa la probabilidad de ir en $m$ pasos o unidades de tiempo de $x$ a $y$, y se le llama \textit{probabilidad de transición en $m$ pasos}.
\end{Note}

\begin{Note}
Para $x,y\in\mathbf{E}$ se define a $P^{(0)}_{x,y}$ como $\delta_{x,y}$, donde $\delta_{x,y}$ es la delta de Kronecker, es decir, vale 1 si $x=y$ y 0 en otro caso.
\end{Note}


\begin{Note}
En el caso de que $\mathbf{E}$ sea finito, se considera la matrix $P=\left(P_{x,y}\right)_{x,y\in \mathbf{E}}$ y se le llama \textit{matriz de transición}.
\end{Note}


\begin{Note}
Si la distribución inicial $\pi$ es igual al vector $\left(\delta_{x,y}\right)_{y\in\mathbf{E}}$, es decir
\begin{eqnarray*}
P\left(X_{0}=x)=1\right)\textrm{ y }P\left(X_{0}\neq x\right)=0,
\end{eqnarray*}
entonces se toma la notación 
\begin{eqnarray}
&&P_{x}\left(A\right)=P\left(A|X_{0}=x\right),A\in\mathcal{F},
\end{eqnarray}
y se dice que la cadena empieza en $A$. Se puede demostrar que $P_{x}$ es una nueva medida de probabilidad en el espacio $\left(\Omega,\mathcal{F}\right)$.
\end{Note}


\begin{Note}
La suma de las entradas de los renglones de la matriz de transición es igual a uno, es decir, para todo $x\in \mathbf{E}$ se tiene $\sum_{y\in\mathbf{E}}P_{x,y}=1$.
\end{Note}

Para poder obtener uno de los resultados más importantes en cadenas de Markov, la \textit{ecuación de Chapman-kolmogorov} se requieren los siguientes resultados:

\begin{Lema}
Sean $x,y,z\in\Eb$ y $0\leq m\leq n-1$, entonces se cumple que
\begin{equation}
P\left(X_{n+1}=y|X_{n}=z,X_{m}=x\right)=P_{z,y}.
\end{equation}
\end{Lema}


\begin{Prop}
Si $x_{0},x_{1},\ldots,x_{n}\in \Eb$ y $\pi\left(x_{0}\right)=P\left(X_{0}=x_{0}\right)$, entonces
\begin{equation}
P\left(X_{1}=x_{1},\ldots,X_{n}=x_{n},X_{0}=x_{0}\right)=\pi\left(x_{0}\right)P_{x_{0},x_{1}}\cdot P_{x_{1},x_{2}}\cdots P_{x_{n-1},x_{n}}
\end{equation}
\end{Prop}

De la proposición anterior se tiene
\begin{equation}
P\left(X_{1}=x_{1},\ldots,X_{n}=x_{n}|X_{0}=x_{0}\right)=P_{x_{0},x_{1}}\cdot P_{x_{1},x_{2}}\cdots P_{x_{n-1},x_{n}}.
\end{equation}

finalmente tenemos la siguiente proposición

\begin{Prop}
Sean $n,k\in\nat$ fijos y $x_{0},x_{1},\ldots,x_{n},\ldots,x_{n+k}\in\Eb$, entonces
\begin{eqnarray*}
&&P\left(X_{n+1}=x_{n+1},\ldots,X_{n+k}=x_{n+k}|X_{n}=x_{n},\ldots,X_{0}=x_{0}\right)\\
&=&P\left(X_{1}=x_{n+1},X_{2}=x_{n+2},\cdots,X_{k}=x_{n+k}|X_{0}=x_{n}\right)
\end{eqnarray*}
\end{Prop}


\begin{Ejem}
Sea $X_{n}$ una variable aleatoria al tiempo $n$ tal que
\begin{eqnarray}
P\left(X_{n+1}=1|X_{n}=0\right)&=&p\\
P\left(X_{n+1}=0|X_{n}=1\right)&=&q=1-p\\
P\left(X_{0}=0\right)&=&\pi_{0}\left(0\right).
\end{eqnarray}
\end{Ejem}
Se puede demostrar que
\begin{eqnarray}
P\left(X_{n}=0\right)&=&\frac{q}{p+q}\\
P\left(X_{n}=1\right)&=&\frac{p}{p+q}
\end{eqnarray}

\begin{Ejem}
El problema de la Caminata Aleatoria
\end{Ejem}

\begin{Ejem}
El problema de la ruina del jugador
\end{Ejem}

\begin{Ejem}
Sea $\left\{Y_{i}\right\}_{i=0}^{\infty}$ sucesión de variables aleatorias independientes e identicamente distribuidas, definidas sobre un espacio de probabilidad $\left(\Omega,\mathcal{F},\prob\right)$ y que toman valores enteros, se tiene que la sucesión $\left\{X_{i}\right\}_{i=0}^{\infty}$ definida por $X_{j}\sum_{}i=0^{j}Y_{j}$ es una cadena de Markov en el conjunto de los números enteros.
\end{Ejem}

\begin{Prop}
Para una cadena de Markov $\left(X_{n}\right)_{n\in\nat}$ con espacio de estados $\Eb$ y para todo $n,m\in \nat$ y toda pareja $x,y\in\Eb$ se cumple
\begin{equation}
P\left(X_{n+m}=y|X_{0}=x\right)=\sum_{z\in\Eb}P_{x,z}^{(m)}P_{z,y}^{(n)}=P_{x,y}^{(n+m)}
\end{equation}
\end{Prop}

\begin{Note}
Para una cadena de Markov con un número finito de estados, se puede pensar a $P^{n}$ como la $n$-ésima potencia de la matriz $P$. Sea $\pi_{0}$ distribución inicial de la cadena de Markov, como 
\begin{eqnarray}
P\left(X_{n}=y\right)=\sum_{x} P\left(X_{0}=x,X_{n}=y\right)=\sum_{x} P\left(X_{0}=x\right)P\left(X_{n}=y|X_{0}=x\right)
\end{eqnarray}
se puede comprobar que 

\begin{eqnarray}
P\left(X_{n}=y\right)=\sum_{x} \pi_{0}\left(X\right)P^{n}\left(x,y\right).
\end{eqnarray}
\end{Note}

Con lo anterior es posible calcular la distribuición de $X_{n}$ en términos de la distribución inicial $\pi_{0}$ y la función de transición de $n$-pasos $P^{n}$,
\begin{eqnarray}
P\left(X_{n+1}=y\right)=\sum_{x} P\left(X_{n}=x\right)P\left(x,y\right).
\end{eqnarray}

Si se conoce la distribución de $X_{0}$ se puede conocer la distribución de $X_{1}$.

\section{Clasificación de Estados}

\begin{Def}
Para $A$ conjunto en el espacio de estados, se define un tiempo de paro $T_{A}$ de $A$ como
\begin{equation}
T_{A}=min_{n>0}\left(X_{n}\in A\right)
\end{equation}
\end{Def}

\begin{Note}
Si $X_{n}\notin A$ para toda $n>0$, $T_{A}=\infty$, es decir,  $T_{A}$ es el primer tiempo positivo que la cadena de Markov está en $A$.
\end{Note}

Una vez que se tiene la definición anterior se puede demostrar la siguiente igualdad:

\begin{Prop}
$P^{n}\left(x,y\right)=\sum_{m=1}^{n}P_{x}\left(T_{y}=m\right)P^{n.m}\left(y,x\right), n\geq1$
\end{Prop}

\begin{Def}
En una cadena de Markov $\left(X_{n}\right)_{n\in\nat}$ con espacio de estados $\Eb$, matriz de transición $\left(P_{x,y}\right)_{x,y\in\Eb}$ y para $x,y\in\Eb$,  se dice que
\begin{enumerate}
\item  De $x$ se accede a $y$ si existe $n\geq0$ tal que $P_{x,y}^{(n)}>0$ y se denota por $\left(x\rightarrow y\right)$

\item $x$ y $y$ se comunican entre sí, lo que se denota por $\left(x\leftrightarrow y\right)$, si se cumplen $\left(x\rightarrow y\right)$ y $\left(y\rightarrow x\right)$.

\item Un estado $x\in\Eb$ es estado recurrente si $$P\left(X_{n}=x\textrm{ para algún }n\in\nat|X_{0}=x \right)\equiv1.$$ 

\item Un estado $x\in\Eb$ es estado transitorio si $$P\left(X_{n}=x\textrm{ para algún }n\in\nat|X_{0}=x \right)<1.$$ 

\item Un estado $x\in\Eb$ se llama absorbente si $P_{x,x}\equiv1$.
\end{enumerate}
\end{Def}


Se tiene el siguiente resultado:

\begin{Prop}
$x\leftrightarrow y$ es una relación de equivalencia y da lugar a una partición del espacio de estados $\Eb$
\end{Prop}


\begin{Def}
\begin{enumerate}
\item[1.  ] Se dice que $C\subset \Eb$ es una clase de comunicación si cualesquiera dos estados de $C$ se comunicán entre sí

\item[2.  ] Dado $x\in\Eb$, su clase de comunicación se denota por: $C\left(x\right)=\left\{y\in\Eb:x\leftrightarrow y\right\}$.

\item[3.  ] Se dice que un conjunto de estados  $C\subset \Eb$ es cerrado si ningún estado de $\Eb-C$ puede ser accedido desde un estado de $C$.
\end{enumerate}
\end{Def}


\begin{Def}
Se dice que la cadena es irreducible si cualquiera de las siguientes condiciones, equivalentes entre sí,  se cumplen
\begin{enumerate}
\item[a) ] Desde cualquier estado de $\Eb$ se puede acceder a cualquier otro.

\item[b) ] Todos los estados se comunican entre sí.

\item[c) ] $C\left(x\right)=\Eb$ para algún $x\in\Eb$.

\item[d) ] $C\left(x\right)=\Eb$ para todo $x\in\Eb$.

\item[e) ] El único conjunto cerrado es el total.
\end{enumerate}
\end{Def}

\begin{Prop}
\begin{enumerate}
\item[a) ] Un estado $x\in\Eb$ es recurrente si y sólo si $P\left(T_{x}<\infty|x_{0}=x\right)=1$.

\item[b) ] Un estado $x\in\Eb$ es transitorio si y sólo si $P\left(T_{x}<\infty|x_{0}=x\right)<1$.

\item[c) ] Un estado $x\in\Eb$ es absorbente si y sólo si $P\left(T_{x}=1|x_{0}=x\right)=1$.


\end{enumerate}
\end{Prop}




\chapter{Procesos Regenerativos}
%___________________________________________________________________________________________
%
% Cap 1
%
%___________________________________________________________________________________________
\vspace{5.5cm}
\section{Procesos Regenerativos}
\vspace{-1.0cm}
%___________________________________________________________________________________________


\begin{Def}
Sea $X$ un conjunto y $\mathcal{F}$ una $\sigma$-\'algebra de
subconjuntos de $X$, la pareja $\left(X,\mathcal{F}\right)$ es
llamado espacio medible. Un subconjunto $A$ de $X$ es llamado
medible, o medible con respecto a $\mathcal{F}$, si
$A\in\mathcal{F}$.
\end{Def}

\begin{Def}
Sea $\left(X,\mathcal{F},\mu\right)$ espacio de medida. Se dice
que la medida $\mu$ es $\sigma$-finita si se puede escribir
$X=\bigcup_{n\geq1}X_{n}$ con $X_{n}\in\mathcal{F}$ y
$\mu\left(X_{n}\right)<\infty$.
\end{Def}

\begin{Def}\label{Cto.Borel}
Sea $X$ el conjunto de los \'umeros reales $\rea$. El \'algebra de
Borel es la $\sigma$-\'algebra $B$ generada por los intervalos
abiertos $\left(a,b\right)\in\rea$. Cualquier conjunto en $B$ es
llamado {\em Conjunto de Borel}.
\end{Def}

\begin{Def}\label{Funcion.Medible}
Una funci\'on $f:X\rightarrow\rea$, es medible si para cualquier
n\'umero real $\alpha$ el conjunto
\[\left\{x\in X:f\left(x\right)>\alpha\right\}\]
pertenece a $X$. Equivalentemente, se dice que $f$ es medible si
\[f^{-1}\left(\left(\alpha,\infty\right)\right)=\left\{x\in X:f\left(x\right)>\alpha\right\}\in\mathcal{F}.\]
\end{Def}


\begin{Def}\label{Def.Cilindros}
Sean $\left(\Omega_{i},\mathcal{F}_{i}\right)$, $i=1,2,\ldots,$
espacios medibles y $\Omega=\prod_{i=1}^{\infty}\Omega_{i}$ el
conjunto de todas las sucesiones
$\left(\omega_{1},\omega_{2},\ldots,\right)$ tales que
$\omega_{i}\in\Omega_{i}$, $i=1,2,\ldots,$. Si
$B^{n}\subset\prod_{i=1}^{\infty}\Omega_{i}$, definimos
$B_{n}=\left\{\omega\in\Omega:\left(\omega_{1},\omega_{2},\ldots,\omega_{n}\right)\in
B^{n}\right\}$. Al conjunto $B_{n}$ se le llama {\em cilindro} con
base $B^{n}$, el cilindro es llamado medible si
$B^{n}\in\prod_{i=1}^{\infty}\mathcal{F}_{i}$.
\end{Def}


\begin{Def}\label{Def.Proc.Adaptado}[TSP, Ash \cite{RBA}]
Sea $X\left(t\right),t\geq0$ proceso estoc\'astico, el proceso es
adaptado a la familia de $\sigma$-\'algebras $\mathcal{F}_{t}$,
para $t\geq0$, si para $s<t$ implica que
$\mathcal{F}_{s}\subset\mathcal{F}_{t}$, y $X\left(t\right)$ es
$\mathcal{F}_{t}$-medible para cada $t$. Si no se especifica
$\mathcal{F}_{t}$ entonces se toma $\mathcal{F}_{t}$ como
$\mathcal{F}\left(X\left(s\right),s\leq t\right)$, la m\'as
peque\~na $\sigma$-\'algebra de subconjuntos de $\Omega$ que hace
que cada $X\left(s\right)$, con $s\leq t$ sea Borel medible.
\end{Def}


\begin{Def}\label{Def.Tiempo.Paro}[TSP, Ash \cite{RBA}]
Sea $\left\{\mathcal{F}\left(t\right),t\geq0\right\}$ familia
creciente de sub $\sigma$-\'algebras. es decir,
$\mathcal{F}\left(s\right)\subset\mathcal{F}\left(t\right)$ para
$s\leq t$. Un tiempo de paro para $\mathcal{F}\left(t\right)$ es
una funci\'on $T:\Omega\rightarrow\left[0,\infty\right]$ tal que
$\left\{T\leq t\right\}\in\mathcal{F}\left(t\right)$ para cada
$t\geq0$. Un tiempo de paro para el proceso estoc\'astico
$X\left(t\right),t\geq0$ es un tiempo de paro para las
$\sigma$-\'algebras
$\mathcal{F}\left(t\right)=\mathcal{F}\left(X\left(s\right)\right)$.
\end{Def}

\begin{Def}
Sea $X\left(t\right),t\geq0$ proceso estoc\'astico, con
$\left(S,\chi\right)$ espacio de estados. Se dice que el proceso
es adaptado a $\left\{\mathcal{F}\left(t\right)\right\}$, es
decir, si para cualquier $s,t\in I$, $I$ conjunto de \'indices,
$s<t$, se tiene que
$\mathcal{F}\left(s\right)\subset\mathcal{F}\left(t\right)$ y
$X\left(t\right)$ es $\mathcal{F}\left(t\right)$-medible,
\end{Def}

\begin{Def}
Sea $X\left(t\right),t\geq0$ proceso estoc\'astico, se dice que es
un Proceso de Markov relativo a $\mathcal{F}\left(t\right)$ o que
$\left\{X\left(t\right),\mathcal{F}\left(t\right)\right\}$ es de
Markov si y s\'olo si para cualquier conjunto $B\in\chi$,  y
$s,t\in I$, $s<t$ se cumple que
\begin{equation}\label{Prop.Markov}
P\left\{X\left(t\right)\in
B|\mathcal{F}\left(s\right)\right\}=P\left\{X\left(t\right)\in
B|X\left(s\right)\right\}.
\end{equation}
\end{Def}
\begin{Note}
Si se dice que $\left\{X\left(t\right)\right\}$ es un Proceso de
Markov sin mencionar $\mathcal{F}\left(t\right)$, se asumir\'a que
\begin{eqnarray*}
\mathcal{F}\left(t\right)=\mathcal{F}_{0}\left(t\right)=\mathcal{F}\left(X\left(r\right),r\leq
t\right),
\end{eqnarray*}
entonces la ecuaci\'on (\ref{Prop.Markov}) se puede escribir como
\begin{equation}
P\left\{X\left(t\right)\in B|X\left(r\right),r\leq s\right\} =
P\left\{X\left(t\right)\in B|X\left(s\right)\right\}
\end{equation}
\end{Note}

\begin{Teo}
Sea $\left(X_{n},\mathcal{F}_{n},n=0,1,\ldots,\right\}$ Proceso de
Markov con espacio de estados $\left(S_{0},\chi_{0}\right)$
generado por una distribuici\'on inicial $P_{o}$ y probabilidad de
transici\'on $p_{mn}$, para $m,n=0,1,\ldots,$ $m<n$, que por
notaci\'on se escribir\'a como $p\left(m,n,x,B\right)\rightarrow
p_{mn}\left(x,B\right)$. Sea $S$ tiempo de paro relativo a la
$\sigma$-\'algebra $\mathcal{F}_{n}$. Sea $T$ funci\'on medible,
$T:\Omega\rightarrow\left\{0,1,\ldots,\right\}$. Sup\'ongase que
$T\geq S$, entonces $T$ es tiempo de paro. Si $B\in\chi_{0}$,
entonces
\begin{equation}\label{Prop.Fuerte.Markov}
P\left\{X\left(T\right)\in
B,T<\infty|\mathcal{F}\left(S\right)\right\} =
p\left(S,T,X\left(s\right),B\right)
\end{equation}
en $\left\{T<\infty\right\}$.
\end{Teo}


\begin{Def}
Sea $X$ un conjunto y $\mathcal{F}$ una $\sigma$-\'algebra de
subconjuntos de $X$, la pareja $\left(X,\mathcal{F}\right)$ es
llamado espacio medible. Un subconjunto $A$ de $X$ es llamado
medible, o medible con respecto a $\mathcal{F}$, si
$A\in\mathcal{F}$.
\end{Def}

\begin{Def}
Sea $\left(X,\mathcal{F},\mu\right)$ espacio de medida. Se dice
que la medida $\mu$ es $\sigma$-finita si se puede escribir
$X=\bigcup_{n\geq1}X_{n}$ con $X_{n}\in\mathcal{F}$ y
$\mu\left(X_{n}\right)<\infty$.
\end{Def}

\begin{Def}\label{Cto.Borel}
Sea $X$ el conjunto de los n\'umeros reales $\rea$. El \'algebra
de Borel es la $\sigma$-\'algebra $B$ generada por los intervalos
abiertos $\left(a,b\right)\in\rea$. Cualquier conjunto en $B$ es
llamado {\em Conjunto de Borel}.
\end{Def}

\begin{Def}\label{Funcion.Medible}
Una funci\'on $f:X\rightarrow\rea$, es medible si para cualquier
n\'umero real $\alpha$ el conjunto
\[\left\{x\in X:f\left(x\right)>\alpha\right\}\]
pertenece a $\mathcal{F}$. Equivalentemente, se dice que $f$ es
medible si
\[f^{-1}\left(\left(\alpha,\infty\right)\right)=\left\{x\in X:f\left(x\right)>\alpha\right\}\in\mathcal{F}.\]
\end{Def}


\begin{Def}\label{Def.Cilindros}
Sean $\left(\Omega_{i},\mathcal{F}_{i}\right)$, $i=1,2,\ldots,$
espacios medibles y $\Omega=\prod_{i=1}^{\infty}\Omega_{i}$ el
conjunto de todas las sucesiones
$\left(\omega_{1},\omega_{2},\ldots,\right)$ tales que
$\omega_{i}\in\Omega_{i}$, $i=1,2,\ldots,$. Si
$B^{n}\subset\prod_{i=1}^{\infty}\Omega_{i}$, definimos
$B_{n}=\left\{\omega\in\Omega:\left(\omega_{1},\omega_{2},\ldots,\omega_{n}\right)\in
B^{n}\right\}$. Al conjunto $B_{n}$ se le llama {\em cilindro} con
base $B^{n}$, el cilindro es llamado medible si
$B^{n}\in\prod_{i=1}^{\infty}\mathcal{F}_{i}$.
\end{Def}


\begin{Def}\label{Def.Proc.Adaptado}[TSP, Ash \cite{RBA}]
Sea $X\left(t\right),t\geq0$ proceso estoc\'astico, el proceso es
adaptado a la familia de $\sigma$-\'algebras $\mathcal{F}_{t}$,
para $t\geq0$, si para $s<t$ implica que
$\mathcal{F}_{s}\subset\mathcal{F}_{t}$, y $X\left(t\right)$ es
$\mathcal{F}_{t}$-medible para cada $t$. Si no se especifica
$\mathcal{F}_{t}$ entonces se toma $\mathcal{F}_{t}$ como
$\mathcal{F}\left(X\left(s\right),s\leq t\right)$, la m\'as
peque\~na $\sigma$-\'algebra de subconjuntos de $\Omega$ que hace
que cada $X\left(s\right)$, con $s\leq t$ sea Borel medible.
\end{Def}


\begin{Def}\label{Def.Tiempo.Paro}[TSP, Ash \cite{RBA}]
Sea $\left\{\mathcal{F}\left(t\right),t\geq0\right\}$ familia
creciente de sub $\sigma$-\'algebras. es decir,
$\mathcal{F}\left(s\right)\subset\mathcal{F}\left(t\right)$ para
$s\leq t$. Un tiempo de paro para $\mathcal{F}\left(t\right)$ es
una funci\'on $T:\Omega\rightarrow\left[0,\infty\right]$ tal que
$\left\{T\leq t\right\}\in\mathcal{F}\left(t\right)$ para cada
$t\geq0$. Un tiempo de paro para el proceso estoc\'astico
$X\left(t\right),t\geq0$ es un tiempo de paro para las
$\sigma$-\'algebras
$\mathcal{F}\left(t\right)=\mathcal{F}\left(X\left(s\right)\right)$.
\end{Def}

\begin{Def}
Sea $X\left(t\right),t\geq0$ proceso estoc\'astico, con
$\left(S,\chi\right)$ espacio de estados. Se dice que el proceso
es adaptado a $\left\{\mathcal{F}\left(t\right)\right\}$, es
decir, si para cualquier $s,t\in I$, $I$ conjunto de \'indices,
$s<t$, se tiene que
$\mathcal{F}\left(s\right)\subset\mathcal{F}\left(t\right)$ y
$X\left(t\right)$ es $\mathcal{F}\left(t\right)$-medible,
\end{Def}

\begin{Def}
Sea $X\left(t\right),t\geq0$ proceso estoc\'astico, se dice que es
un Proceso de Markov relativo a $\mathcal{F}\left(t\right)$ o que
$\left\{X\left(t\right),\mathcal{F}\left(t\right)\right\}$ es de
Markov si y s\'olo si para cualquier conjunto $B\in\chi$,  y
$s,t\in I$, $s<t$ se cumple que
\begin{equation}\label{Prop.Markov}
P\left\{X\left(t\right)\in
B|\mathcal{F}\left(s\right)\right\}=P\left\{X\left(t\right)\in
B|X\left(s\right)\right\}.
\end{equation}
\end{Def}
\begin{Note}
Si se dice que $\left\{X\left(t\right)\right\}$ es un Proceso de
Markov sin mencionar $\mathcal{F}\left(t\right)$, se asumir\'a que
\begin{eqnarray*}
\mathcal{F}\left(t\right)=\mathcal{F}_{0}\left(t\right)=\mathcal{F}\left(X\left(r\right),r\leq
t\right),
\end{eqnarray*}
entonces la ecuaci\'on (\ref{Prop.Markov}) se puede escribir como
\begin{equation}
P\left\{X\left(t\right)\in B|X\left(r\right),r\leq s\right\} =
P\left\{X\left(t\right)\in B|X\left(s\right)\right\}
\end{equation}
\end{Note}


%\newpage
%_______________________________________________________________
%\subsection{Procesos de Estados de Markov}
%_______________________________________________________________

\begin{Teo}
Sea $\left(X_{n},\mathcal{F}_{n},n=0,1,\ldots,\right\}$ Proceso de
Markov con espacio de estados $\left(S_{0},\chi_{0}\right)$
generado por una distribuici\'on inicial $P_{o}$ y probabilidad de
transici\'on $p_{mn}$, para $m,n=0,1,\ldots,$ $m<n$, que por
notaci\'on se escribir\'a como $p\left(m,n,x,B\right)\rightarrow
p_{mn}\left(x,B\right)$. Sea $S$ tiempo de paro relativo a la
$\sigma$-\'algebra $\mathcal{F}_{n}$. Sea $T$ funci\'on medible,
$T:\Omega\rightarrow\left\{0,1,\ldots,\right\}$. Sup\'ongase que
$T\geq S$, entonces $T$ es tiempo de paro. Si $B\in\chi_{0}$,
entonces
\begin{equation}\label{Prop.Fuerte.Markov}
P\left\{X\left(T\right)\in
B,T<\infty|\mathcal{F}\left(S\right)\right\} =
p\left(S,T,X\left(s\right),B\right)
\end{equation}
en $\left\{T<\infty\right\}$.
\end{Teo}


Sea $K$ conjunto numerable y sea $d:K\rightarrow\nat$ funci\'on.
Para $v\in K$, $M_{v}$ es un conjunto abierto de
$\rea^{d\left(v\right)}$. Entonces \[E=\bigcup_{v\in
K}M_{v}=\left\{\left(v,\zeta\right):v\in K,\zeta\in
M_{v}\right\}.\]

Sea $\mathcal{E}$ la clase de conjuntos medibles en $E$:
\[\mathcal{E}=\left\{\bigcup_{v\in K}A_{v}:A_{v}\in \mathcal{M}_{v}\right\}.\]

donde $l{M}$ son los conjuntos de Borel de $M_{v}$.
Entonces $\left(E,\mathcal{E}\right)$ es un espacio de Borel. El
estado del proceso se denotar\'a por
$\mathbf{x}_{t}=\left(v_{t},\zeta_{t}\right)$. La distribuci\'on
de $\left(\mathbf{x}_{t}\right)$ est\'a determinada por por los
siguientes objetos:

\begin{itemize}
\item[i)] Los campos vectoriales $\left(\mathcal{H}_{v},v\in
K\right)$. \item[ii)] Una funci\'on medible $\lambda:E\rightarrow
\rea_{+}$. \item[iii)] Una medida de transici\'on
$Q:\mathcal{E}\times\left(E\cup\Gamma^{*}\right)\rightarrow\left[0,1\right]$
donde
\begin{equation}
\Gamma^{*}=\bigcup_{v\in K}\partial^{*}M_{v}.
\end{equation}
y
\begin{equation}
\partial^{*}M_{v}=\left\{z\in\partial M_{v}:\mathbf{\mathbf{\phi}_{v}\left(t,\zeta\right)=\mathbf{z}}\textrm{ para alguna }\left(t,\zeta\right)\in\rea_{+}\times M_{v}\right\}.
\end{equation}
$\partial M_{v}$ denota  la frontera de $M_{v}$.
\end{itemize}

El campo vectorial $\left(\mathcal{H}_{v},v\in K\right)$ se supone
tal que para cada $\mathbf{z}\in M_{v}$ existe una \'unica curva
integral $\mathbf{\phi}_{v}\left(t,\zeta\right)$ que satisface la
ecuaci\'on

\begin{equation}
\frac{d}{dt}f\left(\zeta_{t}\right)=\mathcal{H}f\left(\zeta_{t}\right),
\end{equation}
con $\zeta_{0}=\mathbf{z}$, para cualquier funci\'on suave
$f:\rea^{d}\rightarrow\rea$ y $\mathcal{H}$ denota el operador
diferencial de primer orden, con $\mathcal{H}=\mathcal{H}_{v}$ y
$\zeta_{t}=\mathbf{\phi}\left(t,\mathbf{z}\right)$. Adem\'as se
supone que $\mathcal{H}_{v}$ es conservativo, es decir, las curvas
integrales est\'an definidas para todo $t>0$.

Para $\mathbf{x}=\left(v,\zeta\right)\in E$ se denota
\[t^{*}\mathbf{x}=inf\left\{t>0:\mathbf{\phi}_{v}\left(t,\zeta\right)\in\partial^{*}M_{v}\right\}\]

En lo que respecta a la funci\'on $\lambda$, se supondr\'a que
para cada $\left(v,\zeta\right)\in E$ existe un $\epsilon>0$ tal
que la funci\'on
$s\rightarrow\lambda\left(v,\phi_{v}\left(s,\zeta\right)\right)\in
E$ es integrable para $s\in\left[0,\epsilon\right)$. La medida de
transici\'on $Q\left(A;\mathbf{x}\right)$ es una funci\'on medible
de $\mathbf{x}$ para cada $A\in\mathcal{E}$, definida para
$\mathbf{x}\in E\cup\Gamma^{*}$ y es una medida de probabilidad en
$\left(E,\mathcal{E}\right)$ para cada $\mathbf{x}\in E$.

El movimiento del proceso $\left(\mathbf{x}_{t}\right)$ comenzando
en $\mathbf{x}=\left(n,\mathbf{z}\right)\in E$ se puede construir
de la siguiente manera, def\'inase la funci\'on $F$ por

\begin{equation}
F\left(t\right)=\left\{\begin{array}{ll}\\
exp\left(-\int_{0}^{t}\lambda\left(n,\phi_{n}\left(s,\mathbf{z}\right)\right)ds\right), & t<t^{*}\left(\mathbf{x}\right),\\
0, & t\geq t^{*}\left(\mathbf{x}\right)
\end{array}\right.
\end{equation}

Sea $T_{1}$ una variable aleatoria tal que
$\prob\left[T_{1}>t\right]=F\left(t\right)$, ahora sea la variable
aleatoria $\left(N,Z\right)$ con distribuici\'on
$Q\left(\cdot;\phi_{n}\left(T_{1},\mathbf{z}\right)\right)$. La
trayectoria de $\left(\mathbf{x}_{t}\right)$ para $t\leq T_{1}$ es
\begin{eqnarray*}
\mathbf{x}_{t}=\left(v_{t},\zeta_{t}\right)=\left\{\begin{array}{ll}
\left(n,\phi_{n}\left(t,\mathbf{z}\right)\right), & t<T_{1},\\
\left(N,\mathbf{Z}\right), & t=t_{1}.
\end{array}\right.
\end{eqnarray*}

Comenzando en $\mathbf{x}_{T_{1}}$ se selecciona el siguiente
tiempo de intersalto $T_{2}-T_{1}$ lugar del post-salto
$\mathbf{x}_{T_{2}}$ de manera similar y as\'i sucesivamente. Este
procedimiento nos da una trayectoria determinista por partes
$\mathbf{x}_{t}$ con tiempos de salto $T_{1},T_{2},\ldots$. Bajo
las condiciones enunciadas para $\lambda,T_{1}>0$  y
$T_{1}-T_{2}>0$ para cada $i$, con probabilidad 1. Se supone que
se cumple la siguiente condici\'on.

\begin{Sup}[Supuesto 3.1, Davis \cite{Davis}]\label{Sup3.1.Davis}
Sea $N_{t}:=\sum_{t}\indora_{\left(t\geq t\right)}$ el n\'umero de
saltos en $\left[0,t\right]$. Entonces
\begin{equation}
\esp\left[N_{t}\right]<\infty\textrm{ para toda }t.
\end{equation}
\end{Sup}

es un proceso de Markov, m\'as a\'un, es un Proceso Fuerte de
Markov, es decir, la Propiedad Fuerte de Markov\footnote{Revisar
p\'agina 362, y 364 de Davis \cite{Davis}.} se cumple para
cualquier tiempo de paro.
%_________________________________________________________________________
%\subsection{Teor\'ia General de Procesos Estoc\'asticos}
%_________________________________________________________________________
En esta secci\'on se har\'an las siguientes consideraciones: $E$
es un espacio m\'etrico separable y la m\'etrica $d$ es compatible
con la topolog\'ia.

\begin{Def}
Una medida finita, $\lambda$ en la $\sigma$-\'algebra de Borel de
un espacio metrizable $E$ se dice cerrada si
\begin{equation}\label{Eq.A2.3}
\lambda\left(E\right)=sup\left\{\lambda\left(K\right):K\textrm{ es
compacto en }E\right\}.
\end{equation}
\end{Def}

\begin{Def}
$E$ es un espacio de Rad\'on si cada medida finita en
$\left(E,\mathcal{B}\left(E\right)\right)$ es regular interior o cerrada,
{\em tight}.
\end{Def}


El siguiente teorema nos permite tener una mejor caracterizaci\'on de los espacios de Rad\'on:
\begin{Teo}\label{Tma.A2.2}
Sea $E$ espacio separable metrizable. Entonces $E$ es de Rad\'on
si y s\'olo s\'i cada medida finita en
$\left(E,\mathcal{B}\left(E\right)\right)$ es cerrada.
\end{Teo}

%_________________________________________________________________________________________
%\subsection{Propiedades de Markov}
%_________________________________________________________________________________________

Sea $E$ espacio de estados, tal que $E$ es un espacio de Rad\'on, $\mathcal{B}\left(E\right)$ $\sigma$-\'algebra de Borel en $E$, que se denotar\'a por $\mathcal{E}$.

Sea $\left(X,\mathcal{G},\prob\right)$ espacio de probabilidad,
$I\subset\rea$ conjunto de índices. Sea $\mathcal{F}_{\leq t}$ la
$\sigma$-\'algebra natural definida como
$\sigma\left\{f\left(X_{r}\right):r\in I, r\leq
t,f\in\mathcal{E}\right\}$. Se considerar\'a una
$\sigma$-\'algebra m\'as general\footnote{qu\'e se quiere decir
con el t\'ermino: m\'as general?}, $ \left(\mathcal{G}_{t}\right)$
tal que $\left(X_{t}\right)$ sea $\mathcal{E}$-adaptado.

\begin{Def}
Una familia $\left(P_{s,t}\right)$ de kernels de Markov en $\left(E,\mathcal{E}\right)$ indexada por pares $s,t\in I$, con $s\leq t$ es una funci\'on de transici\'on en $\ER$, si  para todo $r\leq s< t$ en $I$ y todo $x\in E$, $B\in\mathcal{E}$
\begin{equation}\label{Eq.Kernels}
P_{r,t}\left(x,B\right)=\int_{E}P_{r,s}\left(x,dy\right)P_{s,t}\left(y,B\right)\footnote{Ecuaci\'on de Chapman-Kolmogorov}.
\end{equation}
\end{Def}

Se dice que la funci\'on de transici\'on $\KM$ en $\ER$ es la funci\'on de transici\'on para un proceso $\PE$  con valores en $E$ y que satisface la propiedad de Markov\footnote{\begin{equation}\label{Eq.1.4.S}
\prob\left\{H|\mathcal{G}_{t}\right\}=\prob\left\{H|X_{t}\right\}\textrm{ }H\in p\mathcal{F}_{\geq t}.
\end{equation}} (\ref{Eq.1.4.S}) relativa a $\left(\mathcal{G}_{t}\right)$ si

\begin{equation}\label{Eq.1.6.S}
\prob\left\{f\left(X_{t}\right)|\mathcal{G}_{s}\right\}=P_{s,t}f\left(X_{t}\right)\textrm{ }s\leq t\in I,\textrm{ }f\in b\mathcal{E}.
\end{equation}

\begin{Def}
Una familia $\left(P_{t}\right)_{t\geq0}$ de kernels de Markov en $\ER$ es llamada {\em Semigrupo de Transici\'on de Markov} o {\em Semigrupo de Transici\'on} si
\[P_{t+s}f\left(x\right)=P_{t}\left(P_{s}f\right)\left(x\right),\textrm{ }t,s\geq0,\textrm{ }x\in E\textrm{ }f\in b\mathcal{E}\footnote{Definir los t\'ermino $b\mathcal{E}$ y $p\mathcal{E}$}.\]
\end{Def}
\begin{Note}
Si la funci\'on de transici\'on $\KM$ es llamada homog\'enea si $P_{s,t}=P_{t-s}$.
\end{Note}

Un proceso de Markov que satisface la ecuaci\'on (\ref{Eq.1.6.S}) con funci\'on de transici\'on homog\'enea $\left(P_{t}\right)$ tiene la propiedad caracter\'istica
\begin{equation}\label{Eq.1.8.S}
\prob\left\{f\left(X_{t+s}\right)|\mathcal{G}_{t}\right\}=P_{s}f\left(X_{t}\right)\textrm{ }t,s\geq0,\textrm{ }f\in b\mathcal{E}.
\end{equation}
La ecuaci\'on anterior es la {\em Propiedad Simple de Markov} de $X$ relativa a $\left(P_{t}\right)$.

En este sentido el proceso $\PE$ cumple con la propiedad de Markov (\ref{Eq.1.8.S}) relativa a $\left(\Omega,\mathcal{G},\mathcal{G}_{t},\prob\right)$ con semigrupo de transici\'on $\left(P_{t}\right)$.
%_________________________________________________________________________________________
%\subsection{Primer Condici\'on de Regularidad}
%_________________________________________________________________________________________
%\newcommand{\EM}{\left(\Omega,\mathcal{G},\prob\right)}
%\newcommand{\E4}{\left(\Omega,\mathcal{G},\mathcal{G}_{t},\prob\right)}
\begin{Def}
Un proceso estoc\'astico $\PE$ definido en
$\left(\Omega,\mathcal{G},\prob\right)$ con valores en el espacio
topol\'ogico $E$ es continuo por la derecha si cada trayectoria
muestral $t\rightarrow X_{t}\left(w\right)$ es un mapeo continuo
por la derecha de $I$ en $E$.
\end{Def}

\begin{Def}[HD1]\label{Eq.2.1.S}
Un semigrupo de Markov $\left(P_{t}\right)$ en un espacio de
Rad\'on $E$ se dice que satisface la condici\'on {\em HD1} si,
dada una medida de probabilidad $\mu$ en $E$, existe una
$\sigma$-\'algebra $\mathcal{E^{*}}$ con
$\mathcal{E}\subset\mathcal{E}^{*}$ y
$P_{t}\left(b\mathcal{E}^{*}\right)\subset b\mathcal{E}^{*}$, y un
$\mathcal{E}^{*}$-proceso $E$-valuado continuo por la derecha
$\PE$ en alg\'un espacio de probabilidad filtrado
$\left(\Omega,\mathcal{G},\mathcal{G}_{t},\prob\right)$ tal que
$X=\left(\Omega,\mathcal{G},\mathcal{G}_{t},\prob\right)$ es de
Markov (Homog\'eneo) con semigrupo de transici\'on $(P_{t})$ y
distribuci\'on inicial $\mu$.
\end{Def}

Consid\'erese la colecci\'on de variables aleatorias $X_{t}$
definidas en alg\'un espacio de probabilidad, y una colecci\'on de
medidas $\mathbf{P}^{x}$ tales que
$\mathbf{P}^{x}\left\{X_{0}=x\right\}$, y bajo cualquier
$\mathbf{P}^{x}$, $X_{t}$ es de Markov con semigrupo
$\left(P_{t}\right)$. $\mathbf{P}^{x}$ puede considerarse como la
distribuci\'on condicional de $\mathbf{P}$ dado $X_{0}=x$.

\begin{Def}\label{Def.2.2.S}
Sea $E$ espacio de Rad\'on, $\SG$ semigrupo de Markov en $\ER$. La colecci\'on $\mathbf{X}=\left(\Omega,\mathcal{G},\mathcal{G}_{t},X_{t},\theta_{t},\CM\right)$ es un proceso $\mathcal{E}$-Markov continuo por la derecha simple, con espacio de estados $E$ y semigrupo de transici\'on $\SG$ en caso de que $\mathbf{X}$ satisfaga las siguientes condiciones:
\begin{itemize}
\item[i)] $\left(\Omega,\mathcal{G},\mathcal{G}_{t}\right)$ es un espacio de medida filtrado, y $X_{t}$ es un proceso $E$-valuado continuo por la derecha $\mathcal{E}^{*}$-adaptado a $\left(\mathcal{G}_{t}\right)$;

\item[ii)] $\left(\theta_{t}\right)_{t\geq0}$ es una colecci\'on de operadores {\em shift} para $X$, es decir, mapea $\Omega$ en s\'i mismo satisfaciendo para $t,s\geq0$,

\begin{equation}\label{Eq.Shift}
\theta_{t}\circ\theta_{s}=\theta_{t+s}\textrm{ y }X_{t}\circ\theta_{t}=X_{t+s};
\end{equation}

\item[iii)] Para cualquier $x\in E$,$\CM\left\{X_{0}=x\right\}=1$, y el proceso $\PE$ tiene la propiedad de Markov (\ref{Eq.1.8.S}) con semigrupo de transici\'on $\SG$ relativo a $\left(\Omega,\mathcal{G},\mathcal{G}_{t},\CM\right)$.
\end{itemize}
\end{Def}

\begin{Def}[HD2]\label{Eq.2.2.S}
Para cualquier $\alpha>0$ y cualquier $f\in S^{\alpha}$, el proceso $t\rightarrow f\left(X_{t}\right)$ es continuo por la derecha casi seguramente.
\end{Def}

\begin{Def}\label{Def.PD}
Un sistema $\mathbf{X}=\left(\Omega,\mathcal{G},\mathcal{G}_{t},X_{t},\theta_{t},\CM\right)$ es un proceso derecho en el espacio de Rad\'on $E$ con semigrupo de transici\'on $\SG$ provisto de:
\begin{itemize}
\item[i)] $\mathbf{X}$ es una realizaci\'on  continua por la derecha, \ref{Def.2.2.S}, de $\SG$.

\item[ii)] $\mathbf{X}$ satisface la condicion HD2, \ref{Eq.2.2.S}, relativa a $\mathcal{G}_{t}$.

\item[iii)] $\mathcal{G}_{t}$ es aumentado y continuo por la derecha.
\end{itemize}
\end{Def}

%__________________________________________________________________________________________
\subsection{Procesos Regenerativos Estacionarios - Stidham \cite{Stidham}}
%__________________________________________________________________________________________


Un proceso estoc\'astico a tiempo continuo $\left\{V\left(t\right),t\geq0\right\}$ es un proceso regenerativo si existe una sucesi\'on de variables aleatorias independientes e id\'enticamente distribuidas $\left\{X_{1},X_{2},\ldots\right\}$, sucesi\'on de renovaci\'on, tal que para cualquier conjunto de Borel $A$, 

\begin{eqnarray*}
\prob\left\{V\left(t\right)\in A|X_{1}+X_{2}+\cdots+X_{R\left(t\right)}=s,\left\{V\left(\tau\right),\tau<s\right\}\right\}=\prob\left\{V\left(t-s\right)\in A|X_{1}>t-s\right\},
\end{eqnarray*}
para todo $0\leq s\leq t$, donde $R\left(t\right)=\max\left\{X_{1}+X_{2}+\cdots+X_{j}\leq t\right\}=$n\'umero de renovaciones ({\emph{puntos de regeneraci\'on}}) que ocurren en $\left[0,t\right]$. El intervalo $\left[0,X_{1}\right)$ es llamado {\emph{primer ciclo de regeneraci\'on}} de $\left\{V\left(t \right),t\geq0\right\}$, $\left[X_{1},X_{1}+X_{2}\right)$ el {\emph{segundo ciclo de regeneraci\'on}}, y as\'i sucesivamente.

Sea $X=X_{1}$ y sea $F$ la funci\'on de distrbuci\'on de $X$


\begin{Def}
Se define el proceso estacionario, $\left\{V^{*}\left(t\right),t\geq0\right\}$, para $\left\{V\left(t\right),t\geq0\right\}$ por

\begin{eqnarray*}
\prob\left\{V\left(t\right)\in A\right\}=\frac{1}{\esp\left[X\right]}\int_{0}^{\infty}\prob\left\{V\left(t+x\right)\in A|X>x\right\}\left(1-F\left(x\right)\right)dx,
\end{eqnarray*} 
para todo $t\geq0$ y todo conjunto de Borel $A$.
\end{Def}

\begin{Def}
Una distribuci\'on se dice que es {\emph{aritm\'etica}} si todos sus puntos de incremento son m\'ultiplos de la forma $0,\lambda, 2\lambda,\ldots$ para alguna $\lambda>0$ entera.
\end{Def}


\begin{Def}
Una modificaci\'on medible de un proceso $\left\{V\left(t\right),t\geq0\right\}$, es una versi\'on de este, $\left\{V\left(t,w\right)\right\}$ conjuntamente medible para $t\geq0$ y para $w\in S$, $S$ espacio de estados para $\left\{V\left(t\right),t\geq0\right\}$.
\end{Def}

\begin{Teo}
Sea $\left\{V\left(t\right),t\geq\right\}$ un proceso regenerativo no negativo con modificaci\'on medible. Sea $\esp\left[X\right]<\infty$. Entonces el proceso estacionario dado por la ecuaci\'on anterior est\'a bien definido y tiene funci\'on de distribuci\'on independiente de $t$, adem\'as
\begin{itemize}
\item[i)] \begin{eqnarray*}
\esp\left[V^{*}\left(0\right)\right]&=&\frac{\esp\left[\int_{0}^{X}V\left(s\right)ds\right]}{\esp\left[X\right]}\end{eqnarray*}
\item[ii)] Si $\esp\left[V^{*}\left(0\right)\right]<\infty$, equivalentemente, si $\esp\left[\int_{0}^{X}V\left(s\right)ds\right]<\infty$,entonces
\begin{eqnarray*}
\frac{\int_{0}^{t}V\left(s\right)ds}{t}\rightarrow\frac{\esp\left[\int_{0}^{X}V\left(s\right)ds\right]}{\esp\left[X\right]}
\end{eqnarray*}
con probabilidad 1 y en media, cuando $t\rightarrow\infty$.
\end{itemize}
\end{Teo}

\begin{Coro}
Sea $\left\{V\left(t\right),t\geq0\right\}$ un proceso regenerativo no negativo, con modificaci\'on medible. Si $\esp <\infty$, $F$ es no-aritm\'etica, y para todo $x\geq0$, $P\left\{V\left(t\right)\leq x,C>x\right\}$ es de variaci\'on acotada como funci\'on de $t$ en cada intervalo finito $\left[0,\tau\right]$, entonces $V\left(t\right)$ converge en distribuci\'on  cuando $t\rightarrow\infty$ y $$\esp V=\frac{\esp \int_{0}^{X}V\left(s\right)ds}{\esp X}$$
Donde $V$ tiene la distribuci\'on l\'imite de $V\left(t\right)$ cuando $t\rightarrow\infty$.

\end{Coro}

Para el caso discreto se tienen resultados similares.



%______________________________________________________________________
\subsection{Procesos de Renovaci\'on}
%______________________________________________________________________

\begin{Def}%\label{Def.Tn}
Sean $0\leq T_{1}\leq T_{2}\leq \ldots$ son tiempos aleatorios infinitos en los cuales ocurren ciertos eventos. El n\'umero de tiempos $T_{n}$ en el intervalo $\left[0,t\right)$ es

\begin{eqnarray}
N\left(t\right)=\sum_{n=1}^{\infty}\indora\left(T_{n}\leq t\right),
\end{eqnarray}
para $t\geq0$.
\end{Def}

Si se consideran los puntos $T_{n}$ como elementos de $\rea_{+}$, y $N\left(t\right)$ es el n\'umero de puntos en $\rea$. El proceso denotado por $\left\{N\left(t\right):t\geq0\right\}$, denotado por $N\left(t\right)$, es un proceso puntual en $\rea_{+}$. Los $T_{n}$ son los tiempos de ocurrencia, el proceso puntual $N\left(t\right)$ es simple si su n\'umero de ocurrencias son distintas: $0<T_{1}<T_{2}<\ldots$ casi seguramente.

\begin{Def}
Un proceso puntual $N\left(t\right)$ es un proceso de renovaci\'on si los tiempos de interocurrencia $\xi_{n}=T_{n}-T_{n-1}$, para $n\geq1$, son independientes e identicamente distribuidos con distribuci\'on $F$, donde $F\left(0\right)=0$ y $T_{0}=0$. Los $T_{n}$ son llamados tiempos de renovaci\'on, referente a la independencia o renovaci\'on de la informaci\'on estoc\'astica en estos tiempos. Los $\xi_{n}$ son los tiempos de inter-renovaci\'on, y $N\left(t\right)$ es el n\'umero de renovaciones en el intervalo $\left[0,t\right)$
\end{Def}


\begin{Note}
Para definir un proceso de renovaci\'on para cualquier contexto, solamente hay que especificar una distribuci\'on $F$, con $F\left(0\right)=0$, para los tiempos de inter-renovaci\'on. La funci\'on $F$ en turno degune las otra variables aleatorias. De manera formal, existe un espacio de probabilidad y una sucesi\'on de variables aleatorias $\xi_{1},\xi_{2},\ldots$ definidas en este con distribuci\'on $F$. Entonces las otras cantidades son $T_{n}=\sum_{k=1}^{n}\xi_{k}$ y $N\left(t\right)=\sum_{n=1}^{\infty}\indora\left(T_{n}\leq t\right)$, donde $T_{n}\rightarrow\infty$ casi seguramente por la Ley Fuerte de los Grandes Números.
\end{Note}

%___________________________________________________________________________________________
%
\subsection{Teorema Principal de Renovaci\'on}
%___________________________________________________________________________________________
%

\begin{Note} Una funci\'on $h:\rea_{+}\rightarrow\rea$ es Directamente Riemann Integrable en los siguientes casos:
\begin{itemize}
\item[a)] $h\left(t\right)\geq0$ es decreciente y Riemann Integrable.
\item[b)] $h$ es continua excepto posiblemente en un conjunto de Lebesgue de medida 0, y $|h\left(t\right)|\leq b\left(t\right)$, donde $b$ es DRI.
\end{itemize}
\end{Note}

\begin{Teo}[Teorema Principal de Renovaci\'on]
Si $F$ es no aritm\'etica y $h\left(t\right)$ es Directamente Riemann Integrable (DRI), entonces

\begin{eqnarray*}
lim_{t\rightarrow\infty}U\star h=\frac{1}{\mu}\int_{\rea_{+}}h\left(s\right)ds.
\end{eqnarray*}
\end{Teo}

\begin{Prop}
Cualquier funci\'on $H\left(t\right)$ acotada en intervalos finitos y que es 0 para $t<0$ puede expresarse como
\begin{eqnarray*}
H\left(t\right)=U\star h\left(t\right)\textrm{,  donde }h\left(t\right)=H\left(t\right)-F\star H\left(t\right)
\end{eqnarray*}
\end{Prop}

\begin{Def}
Un proceso estoc\'astico $X\left(t\right)$ es crudamente regenerativo en un tiempo aleatorio positivo $T$ si
\begin{eqnarray*}
\esp\left[X\left(T+t\right)|T\right]=\esp\left[X\left(t\right)\right]\textrm{, para }t\geq0,\end{eqnarray*}
y con las esperanzas anteriores finitas.
\end{Def}

\begin{Prop}
Sup\'ongase que $X\left(t\right)$ es un proceso crudamente regenerativo en $T$, que tiene distribuci\'on $F$. Si $\esp\left[X\left(t\right)\right]$ es acotado en intervalos finitos, entonces
\begin{eqnarray*}
\esp\left[X\left(t\right)\right]=U\star h\left(t\right)\textrm{,  donde }h\left(t\right)=\esp\left[X\left(t\right)\indora\left(T>t\right)\right].
\end{eqnarray*}
\end{Prop}

\begin{Teo}[Regeneraci\'on Cruda]
Sup\'ongase que $X\left(t\right)$ es un proceso con valores positivo crudamente regenerativo en $T$, y def\'inase $M=\sup\left\{|X\left(t\right)|:t\leq T\right\}$. Si $T$ es no aritm\'etico y $M$ y $MT$ tienen media finita, entonces
\begin{eqnarray*}
lim_{t\rightarrow\infty}\esp\left[X\left(t\right)\right]=\frac{1}{\mu}\int_{\rea_{+}}h\left(s\right)ds,
\end{eqnarray*}
donde $h\left(t\right)=\esp\left[X\left(t\right)\indora\left(T>t\right)\right]$.
\end{Teo}

%___________________________________________________________________________________________
%
\subsection{Propiedades de los Procesos de Renovaci\'on}
%___________________________________________________________________________________________
%

Los tiempos $T_{n}$ est\'an relacionados con los conteos de $N\left(t\right)$ por

\begin{eqnarray*}
\left\{N\left(t\right)\geq n\right\}&=&\left\{T_{n}\leq t\right\}\\
T_{N\left(t\right)}\leq &t&<T_{N\left(t\right)+1},
\end{eqnarray*}

adem\'as $N\left(T_{n}\right)=n$, y 

\begin{eqnarray*}
N\left(t\right)=\max\left\{n:T_{n}\leq t\right\}=\min\left\{n:T_{n+1}>t\right\}
\end{eqnarray*}

Por propiedades de la convoluci\'on se sabe que

\begin{eqnarray*}
P\left\{T_{n}\leq t\right\}=F^{n\star}\left(t\right)
\end{eqnarray*}
que es la $n$-\'esima convoluci\'on de $F$. Entonces 

\begin{eqnarray*}
\left\{N\left(t\right)\geq n\right\}&=&\left\{T_{n}\leq t\right\}\\
P\left\{N\left(t\right)\leq n\right\}&=&1-F^{\left(n+1\right)\star}\left(t\right)
\end{eqnarray*}

Adem\'as usando el hecho de que $\esp\left[N\left(t\right)\right]=\sum_{n=1}^{\infty}P\left\{N\left(t\right)\geq n\right\}$
se tiene que

\begin{eqnarray*}
\esp\left[N\left(t\right)\right]=\sum_{n=1}^{\infty}F^{n\star}\left(t\right)
\end{eqnarray*}

\begin{Prop}
Para cada $t\geq0$, la funci\'on generadora de momentos $\esp\left[e^{\alpha N\left(t\right)}\right]$ existe para alguna $\alpha$ en una vecindad del 0, y de aqu\'i que $\esp\left[N\left(t\right)^{m}\right]<\infty$, para $m\geq1$.
\end{Prop}


\begin{Note}
Si el primer tiempo de renovaci\'on $\xi_{1}$ no tiene la misma distribuci\'on que el resto de las $\xi_{n}$, para $n\geq2$, a $N\left(t\right)$ se le llama Proceso de Renovaci\'on retardado, donde si $\xi$ tiene distribuci\'on $G$, entonces el tiempo $T_{n}$ de la $n$-\'esima renovaci\'on tiene distribuci\'on $G\star F^{\left(n-1\right)\star}\left(t\right)$
\end{Note}


\begin{Teo}
Para una constante $\mu\leq\infty$ ( o variable aleatoria), las siguientes expresiones son equivalentes:

\begin{eqnarray}
lim_{n\rightarrow\infty}n^{-1}T_{n}&=&\mu,\textrm{ c.s.}\\
lim_{t\rightarrow\infty}t^{-1}N\left(t\right)&=&1/\mu,\textrm{ c.s.}
\end{eqnarray}
\end{Teo}


Es decir, $T_{n}$ satisface la Ley Fuerte de los Grandes N\'umeros s\'i y s\'olo s\'i $N\left/t\right)$ la cumple.


\begin{Coro}[Ley Fuerte de los Grandes N\'umeros para Procesos de Renovaci\'on]
Si $N\left(t\right)$ es un proceso de renovaci\'on cuyos tiempos de inter-renovaci\'on tienen media $\mu\leq\infty$, entonces
\begin{eqnarray}
t^{-1}N\left(t\right)\rightarrow 1/\mu,\textrm{ c.s. cuando }t\rightarrow\infty.
\end{eqnarray}

\end{Coro}


Considerar el proceso estoc\'astico de valores reales $\left\{Z\left(t\right):t\geq0\right\}$ en el mismo espacio de probabilidad que $N\left(t\right)$

\begin{Def}
Para el proceso $\left\{Z\left(t\right):t\geq0\right\}$ se define la fluctuaci\'on m\'axima de $Z\left(t\right)$ en el intervalo $\left(T_{n-1},T_{n}\right]$:
\begin{eqnarray*}
M_{n}=\sup_{T_{n-1}<t\leq T_{n}}|Z\left(t\right)-Z\left(T_{n-1}\right)|
\end{eqnarray*}
\end{Def}

\begin{Teo}
Sup\'ongase que $n^{-1}T_{n}\rightarrow\mu$ c.s. cuando $n\rightarrow\infty$, donde $\mu\leq\infty$ es una constante o variable aleatoria. Sea $a$ una constante o variable aleatoria que puede ser infinita cuando $\mu$ es finita, y considere las expresiones l\'imite:
\begin{eqnarray}
lim_{n\rightarrow\infty}n^{-1}Z\left(T_{n}\right)&=&a,\textrm{ c.s.}\\
lim_{t\rightarrow\infty}t^{-1}Z\left(t\right)&=&a/\mu,\textrm{ c.s.}
\end{eqnarray}
La segunda expresi\'on implica la primera. Conversamente, la primera implica la segunda si el proceso $Z\left(t\right)$ es creciente, o si $lim_{n\rightarrow\infty}n^{-1}M_{n}=0$ c.s.
\end{Teo}

\begin{Coro}
Si $N\left(t\right)$ es un proceso de renovaci\'on, y $\left(Z\left(T_{n}\right)-Z\left(T_{n-1}\right),M_{n}\right)$, para $n\geq1$, son variables aleatorias independientes e id\'enticamente distribuidas con media finita, entonces,
\begin{eqnarray}
lim_{t\rightarrow\infty}t^{-1}Z\left(t\right)\rightarrow\frac{\esp\left[Z\left(T_{1}\right)-Z\left(T_{0}\right)\right]}{\esp\left[T_{1}\right]},\textrm{ c.s. cuando  }t\rightarrow\infty.
\end{eqnarray}
\end{Coro}



%___________________________________________________________________________________________
%
\subsection{Propiedades de los Procesos de Renovaci\'on}
%___________________________________________________________________________________________
%

Los tiempos $T_{n}$ est\'an relacionados con los conteos de $N\left(t\right)$ por

\begin{eqnarray*}
\left\{N\left(t\right)\geq n\right\}&=&\left\{T_{n}\leq t\right\}\\
T_{N\left(t\right)}\leq &t&<T_{N\left(t\right)+1},
\end{eqnarray*}

adem\'as $N\left(T_{n}\right)=n$, y 

\begin{eqnarray*}
N\left(t\right)=\max\left\{n:T_{n}\leq t\right\}=\min\left\{n:T_{n+1}>t\right\}
\end{eqnarray*}

Por propiedades de la convoluci\'on se sabe que

\begin{eqnarray*}
P\left\{T_{n}\leq t\right\}=F^{n\star}\left(t\right)
\end{eqnarray*}
que es la $n$-\'esima convoluci\'on de $F$. Entonces 

\begin{eqnarray*}
\left\{N\left(t\right)\geq n\right\}&=&\left\{T_{n}\leq t\right\}\\
P\left\{N\left(t\right)\leq n\right\}&=&1-F^{\left(n+1\right)\star}\left(t\right)
\end{eqnarray*}

Adem\'as usando el hecho de que $\esp\left[N\left(t\right)\right]=\sum_{n=1}^{\infty}P\left\{N\left(t\right)\geq n\right\}$
se tiene que

\begin{eqnarray*}
\esp\left[N\left(t\right)\right]=\sum_{n=1}^{\infty}F^{n\star}\left(t\right)
\end{eqnarray*}

\begin{Prop}
Para cada $t\geq0$, la funci\'on generadora de momentos $\esp\left[e^{\alpha N\left(t\right)}\right]$ existe para alguna $\alpha$ en una vecindad del 0, y de aqu\'i que $\esp\left[N\left(t\right)^{m}\right]<\infty$, para $m\geq1$.
\end{Prop}


\begin{Note}
Si el primer tiempo de renovaci\'on $\xi_{1}$ no tiene la misma distribuci\'on que el resto de las $\xi_{n}$, para $n\geq2$, a $N\left(t\right)$ se le llama Proceso de Renovaci\'on retardado, donde si $\xi$ tiene distribuci\'on $G$, entonces el tiempo $T_{n}$ de la $n$-\'esima renovaci\'on tiene distribuci\'on $G\star F^{\left(n-1\right)\star}\left(t\right)$
\end{Note}


\begin{Teo}
Para una constante $\mu\leq\infty$ ( o variable aleatoria), las siguientes expresiones son equivalentes:

\begin{eqnarray}
lim_{n\rightarrow\infty}n^{-1}T_{n}&=&\mu,\textrm{ c.s.}\\
lim_{t\rightarrow\infty}t^{-1}N\left(t\right)&=&1/\mu,\textrm{ c.s.}
\end{eqnarray}
\end{Teo}


Es decir, $T_{n}$ satisface la Ley Fuerte de los Grandes N\'umeros s\'i y s\'olo s\'i $N\left/t\right)$ la cumple.


\begin{Coro}[Ley Fuerte de los Grandes N\'umeros para Procesos de Renovaci\'on]
Si $N\left(t\right)$ es un proceso de renovaci\'on cuyos tiempos de inter-renovaci\'on tienen media $\mu\leq\infty$, entonces
\begin{eqnarray}
t^{-1}N\left(t\right)\rightarrow 1/\mu,\textrm{ c.s. cuando }t\rightarrow\infty.
\end{eqnarray}

\end{Coro}


Considerar el proceso estoc\'astico de valores reales $\left\{Z\left(t\right):t\geq0\right\}$ en el mismo espacio de probabilidad que $N\left(t\right)$

\begin{Def}
Para el proceso $\left\{Z\left(t\right):t\geq0\right\}$ se define la fluctuaci\'on m\'axima de $Z\left(t\right)$ en el intervalo $\left(T_{n-1},T_{n}\right]$:
\begin{eqnarray*}
M_{n}=\sup_{T_{n-1}<t\leq T_{n}}|Z\left(t\right)-Z\left(T_{n-1}\right)|
\end{eqnarray*}
\end{Def}

\begin{Teo}
Sup\'ongase que $n^{-1}T_{n}\rightarrow\mu$ c.s. cuando $n\rightarrow\infty$, donde $\mu\leq\infty$ es una constante o variable aleatoria. Sea $a$ una constante o variable aleatoria que puede ser infinita cuando $\mu$ es finita, y considere las expresiones l\'imite:
\begin{eqnarray}
lim_{n\rightarrow\infty}n^{-1}Z\left(T_{n}\right)&=&a,\textrm{ c.s.}\\
lim_{t\rightarrow\infty}t^{-1}Z\left(t\right)&=&a/\mu,\textrm{ c.s.}
\end{eqnarray}
La segunda expresi\'on implica la primera. Conversamente, la primera implica la segunda si el proceso $Z\left(t\right)$ es creciente, o si $lim_{n\rightarrow\infty}n^{-1}M_{n}=0$ c.s.
\end{Teo}

\begin{Coro}
Si $N\left(t\right)$ es un proceso de renovaci\'on, y $\left(Z\left(T_{n}\right)-Z\left(T_{n-1}\right),M_{n}\right)$, para $n\geq1$, son variables aleatorias independientes e id\'enticamente distribuidas con media finita, entonces,
\begin{eqnarray}
lim_{t\rightarrow\infty}t^{-1}Z\left(t\right)\rightarrow\frac{\esp\left[Z\left(T_{1}\right)-Z\left(T_{0}\right)\right]}{\esp\left[T_{1}\right]},\textrm{ c.s. cuando  }t\rightarrow\infty.
\end{eqnarray}
\end{Coro}


%___________________________________________________________________________________________
%
\subsection{Propiedades de los Procesos de Renovaci\'on}
%___________________________________________________________________________________________
%

Los tiempos $T_{n}$ est\'an relacionados con los conteos de $N\left(t\right)$ por

\begin{eqnarray*}
\left\{N\left(t\right)\geq n\right\}&=&\left\{T_{n}\leq t\right\}\\
T_{N\left(t\right)}\leq &t&<T_{N\left(t\right)+1},
\end{eqnarray*}

adem\'as $N\left(T_{n}\right)=n$, y 

\begin{eqnarray*}
N\left(t\right)=\max\left\{n:T_{n}\leq t\right\}=\min\left\{n:T_{n+1}>t\right\}
\end{eqnarray*}

Por propiedades de la convoluci\'on se sabe que

\begin{eqnarray*}
P\left\{T_{n}\leq t\right\}=F^{n\star}\left(t\right)
\end{eqnarray*}
que es la $n$-\'esima convoluci\'on de $F$. Entonces 

\begin{eqnarray*}
\left\{N\left(t\right)\geq n\right\}&=&\left\{T_{n}\leq t\right\}\\
P\left\{N\left(t\right)\leq n\right\}&=&1-F^{\left(n+1\right)\star}\left(t\right)
\end{eqnarray*}

Adem\'as usando el hecho de que $\esp\left[N\left(t\right)\right]=\sum_{n=1}^{\infty}P\left\{N\left(t\right)\geq n\right\}$
se tiene que

\begin{eqnarray*}
\esp\left[N\left(t\right)\right]=\sum_{n=1}^{\infty}F^{n\star}\left(t\right)
\end{eqnarray*}

\begin{Prop}
Para cada $t\geq0$, la funci\'on generadora de momentos $\esp\left[e^{\alpha N\left(t\right)}\right]$ existe para alguna $\alpha$ en una vecindad del 0, y de aqu\'i que $\esp\left[N\left(t\right)^{m}\right]<\infty$, para $m\geq1$.
\end{Prop}


\begin{Note}
Si el primer tiempo de renovaci\'on $\xi_{1}$ no tiene la misma distribuci\'on que el resto de las $\xi_{n}$, para $n\geq2$, a $N\left(t\right)$ se le llama Proceso de Renovaci\'on retardado, donde si $\xi$ tiene distribuci\'on $G$, entonces el tiempo $T_{n}$ de la $n$-\'esima renovaci\'on tiene distribuci\'on $G\star F^{\left(n-1\right)\star}\left(t\right)$
\end{Note}


\begin{Teo}
Para una constante $\mu\leq\infty$ ( o variable aleatoria), las siguientes expresiones son equivalentes:

\begin{eqnarray}
lim_{n\rightarrow\infty}n^{-1}T_{n}&=&\mu,\textrm{ c.s.}\\
lim_{t\rightarrow\infty}t^{-1}N\left(t\right)&=&1/\mu,\textrm{ c.s.}
\end{eqnarray}
\end{Teo}


Es decir, $T_{n}$ satisface la Ley Fuerte de los Grandes N\'umeros s\'i y s\'olo s\'i $N\left/t\right)$ la cumple.


\begin{Coro}[Ley Fuerte de los Grandes N\'umeros para Procesos de Renovaci\'on]
Si $N\left(t\right)$ es un proceso de renovaci\'on cuyos tiempos de inter-renovaci\'on tienen media $\mu\leq\infty$, entonces
\begin{eqnarray}
t^{-1}N\left(t\right)\rightarrow 1/\mu,\textrm{ c.s. cuando }t\rightarrow\infty.
\end{eqnarray}

\end{Coro}


Considerar el proceso estoc\'astico de valores reales $\left\{Z\left(t\right):t\geq0\right\}$ en el mismo espacio de probabilidad que $N\left(t\right)$

\begin{Def}
Para el proceso $\left\{Z\left(t\right):t\geq0\right\}$ se define la fluctuaci\'on m\'axima de $Z\left(t\right)$ en el intervalo $\left(T_{n-1},T_{n}\right]$:
\begin{eqnarray*}
M_{n}=\sup_{T_{n-1}<t\leq T_{n}}|Z\left(t\right)-Z\left(T_{n-1}\right)|
\end{eqnarray*}
\end{Def}

\begin{Teo}
Sup\'ongase que $n^{-1}T_{n}\rightarrow\mu$ c.s. cuando $n\rightarrow\infty$, donde $\mu\leq\infty$ es una constante o variable aleatoria. Sea $a$ una constante o variable aleatoria que puede ser infinita cuando $\mu$ es finita, y considere las expresiones l\'imite:
\begin{eqnarray}
lim_{n\rightarrow\infty}n^{-1}Z\left(T_{n}\right)&=&a,\textrm{ c.s.}\\
lim_{t\rightarrow\infty}t^{-1}Z\left(t\right)&=&a/\mu,\textrm{ c.s.}
\end{eqnarray}
La segunda expresi\'on implica la primera. Conversamente, la primera implica la segunda si el proceso $Z\left(t\right)$ es creciente, o si $lim_{n\rightarrow\infty}n^{-1}M_{n}=0$ c.s.
\end{Teo}

\begin{Coro}
Si $N\left(t\right)$ es un proceso de renovaci\'on, y $\left(Z\left(T_{n}\right)-Z\left(T_{n-1}\right),M_{n}\right)$, para $n\geq1$, son variables aleatorias independientes e id\'enticamente distribuidas con media finita, entonces,
\begin{eqnarray}
lim_{t\rightarrow\infty}t^{-1}Z\left(t\right)\rightarrow\frac{\esp\left[Z\left(T_{1}\right)-Z\left(T_{0}\right)\right]}{\esp\left[T_{1}\right]},\textrm{ c.s. cuando  }t\rightarrow\infty.
\end{eqnarray}
\end{Coro}

%___________________________________________________________________________________________
%
\subsection{Propiedades de los Procesos de Renovaci\'on}
%___________________________________________________________________________________________
%

Los tiempos $T_{n}$ est\'an relacionados con los conteos de $N\left(t\right)$ por

\begin{eqnarray*}
\left\{N\left(t\right)\geq n\right\}&=&\left\{T_{n}\leq t\right\}\\
T_{N\left(t\right)}\leq &t&<T_{N\left(t\right)+1},
\end{eqnarray*}

adem\'as $N\left(T_{n}\right)=n$, y 

\begin{eqnarray*}
N\left(t\right)=\max\left\{n:T_{n}\leq t\right\}=\min\left\{n:T_{n+1}>t\right\}
\end{eqnarray*}

Por propiedades de la convoluci\'on se sabe que

\begin{eqnarray*}
P\left\{T_{n}\leq t\right\}=F^{n\star}\left(t\right)
\end{eqnarray*}
que es la $n$-\'esima convoluci\'on de $F$. Entonces 

\begin{eqnarray*}
\left\{N\left(t\right)\geq n\right\}&=&\left\{T_{n}\leq t\right\}\\
P\left\{N\left(t\right)\leq n\right\}&=&1-F^{\left(n+1\right)\star}\left(t\right)
\end{eqnarray*}

Adem\'as usando el hecho de que $\esp\left[N\left(t\right)\right]=\sum_{n=1}^{\infty}P\left\{N\left(t\right)\geq n\right\}$
se tiene que

\begin{eqnarray*}
\esp\left[N\left(t\right)\right]=\sum_{n=1}^{\infty}F^{n\star}\left(t\right)
\end{eqnarray*}

\begin{Prop}
Para cada $t\geq0$, la funci\'on generadora de momentos $\esp\left[e^{\alpha N\left(t\right)}\right]$ existe para alguna $\alpha$ en una vecindad del 0, y de aqu\'i que $\esp\left[N\left(t\right)^{m}\right]<\infty$, para $m\geq1$.
\end{Prop}


\begin{Note}
Si el primer tiempo de renovaci\'on $\xi_{1}$ no tiene la misma distribuci\'on que el resto de las $\xi_{n}$, para $n\geq2$, a $N\left(t\right)$ se le llama Proceso de Renovaci\'on retardado, donde si $\xi$ tiene distribuci\'on $G$, entonces el tiempo $T_{n}$ de la $n$-\'esima renovaci\'on tiene distribuci\'on $G\star F^{\left(n-1\right)\star}\left(t\right)$
\end{Note}


\begin{Teo}
Para una constante $\mu\leq\infty$ ( o variable aleatoria), las siguientes expresiones son equivalentes:

\begin{eqnarray}
lim_{n\rightarrow\infty}n^{-1}T_{n}&=&\mu,\textrm{ c.s.}\\
lim_{t\rightarrow\infty}t^{-1}N\left(t\right)&=&1/\mu,\textrm{ c.s.}
\end{eqnarray}
\end{Teo}


Es decir, $T_{n}$ satisface la Ley Fuerte de los Grandes N\'umeros s\'i y s\'olo s\'i $N\left/t\right)$ la cumple.


\begin{Coro}[Ley Fuerte de los Grandes N\'umeros para Procesos de Renovaci\'on]
Si $N\left(t\right)$ es un proceso de renovaci\'on cuyos tiempos de inter-renovaci\'on tienen media $\mu\leq\infty$, entonces
\begin{eqnarray}
t^{-1}N\left(t\right)\rightarrow 1/\mu,\textrm{ c.s. cuando }t\rightarrow\infty.
\end{eqnarray}

\end{Coro}


Considerar el proceso estoc\'astico de valores reales $\left\{Z\left(t\right):t\geq0\right\}$ en el mismo espacio de probabilidad que $N\left(t\right)$

\begin{Def}
Para el proceso $\left\{Z\left(t\right):t\geq0\right\}$ se define la fluctuaci\'on m\'axima de $Z\left(t\right)$ en el intervalo $\left(T_{n-1},T_{n}\right]$:
\begin{eqnarray*}
M_{n}=\sup_{T_{n-1}<t\leq T_{n}}|Z\left(t\right)-Z\left(T_{n-1}\right)|
\end{eqnarray*}
\end{Def}

\begin{Teo}
Sup\'ongase que $n^{-1}T_{n}\rightarrow\mu$ c.s. cuando $n\rightarrow\infty$, donde $\mu\leq\infty$ es una constante o variable aleatoria. Sea $a$ una constante o variable aleatoria que puede ser infinita cuando $\mu$ es finita, y considere las expresiones l\'imite:
\begin{eqnarray}
lim_{n\rightarrow\infty}n^{-1}Z\left(T_{n}\right)&=&a,\textrm{ c.s.}\\
lim_{t\rightarrow\infty}t^{-1}Z\left(t\right)&=&a/\mu,\textrm{ c.s.}
\end{eqnarray}
La segunda expresi\'on implica la primera. Conversamente, la primera implica la segunda si el proceso $Z\left(t\right)$ es creciente, o si $lim_{n\rightarrow\infty}n^{-1}M_{n}=0$ c.s.
\end{Teo}

\begin{Coro}
Si $N\left(t\right)$ es un proceso de renovaci\'on, y $\left(Z\left(T_{n}\right)-Z\left(T_{n-1}\right),M_{n}\right)$, para $n\geq1$, son variables aleatorias independientes e id\'enticamente distribuidas con media finita, entonces,
\begin{eqnarray}
lim_{t\rightarrow\infty}t^{-1}Z\left(t\right)\rightarrow\frac{\esp\left[Z\left(T_{1}\right)-Z\left(T_{0}\right)\right]}{\esp\left[T_{1}\right]},\textrm{ c.s. cuando  }t\rightarrow\infty.
\end{eqnarray}
\end{Coro}
%___________________________________________________________________________________________
%
\subsection{Propiedades de los Procesos de Renovaci\'on}
%___________________________________________________________________________________________
%

Los tiempos $T_{n}$ est\'an relacionados con los conteos de $N\left(t\right)$ por

\begin{eqnarray*}
\left\{N\left(t\right)\geq n\right\}&=&\left\{T_{n}\leq t\right\}\\
T_{N\left(t\right)}\leq &t&<T_{N\left(t\right)+1},
\end{eqnarray*}

adem\'as $N\left(T_{n}\right)=n$, y 

\begin{eqnarray*}
N\left(t\right)=\max\left\{n:T_{n}\leq t\right\}=\min\left\{n:T_{n+1}>t\right\}
\end{eqnarray*}

Por propiedades de la convoluci\'on se sabe que

\begin{eqnarray*}
P\left\{T_{n}\leq t\right\}=F^{n\star}\left(t\right)
\end{eqnarray*}
que es la $n$-\'esima convoluci\'on de $F$. Entonces 

\begin{eqnarray*}
\left\{N\left(t\right)\geq n\right\}&=&\left\{T_{n}\leq t\right\}\\
P\left\{N\left(t\right)\leq n\right\}&=&1-F^{\left(n+1\right)\star}\left(t\right)
\end{eqnarray*}

Adem\'as usando el hecho de que $\esp\left[N\left(t\right)\right]=\sum_{n=1}^{\infty}P\left\{N\left(t\right)\geq n\right\}$
se tiene que

\begin{eqnarray*}
\esp\left[N\left(t\right)\right]=\sum_{n=1}^{\infty}F^{n\star}\left(t\right)
\end{eqnarray*}

\begin{Prop}
Para cada $t\geq0$, la funci\'on generadora de momentos $\esp\left[e^{\alpha N\left(t\right)}\right]$ existe para alguna $\alpha$ en una vecindad del 0, y de aqu\'i que $\esp\left[N\left(t\right)^{m}\right]<\infty$, para $m\geq1$.
\end{Prop}


\begin{Note}
Si el primer tiempo de renovaci\'on $\xi_{1}$ no tiene la misma distribuci\'on que el resto de las $\xi_{n}$, para $n\geq2$, a $N\left(t\right)$ se le llama Proceso de Renovaci\'on retardado, donde si $\xi$ tiene distribuci\'on $G$, entonces el tiempo $T_{n}$ de la $n$-\'esima renovaci\'on tiene distribuci\'on $G\star F^{\left(n-1\right)\star}\left(t\right)$
\end{Note}


\begin{Teo}
Para una constante $\mu\leq\infty$ ( o variable aleatoria), las siguientes expresiones son equivalentes:

\begin{eqnarray}
lim_{n\rightarrow\infty}n^{-1}T_{n}&=&\mu,\textrm{ c.s.}\\
lim_{t\rightarrow\infty}t^{-1}N\left(t\right)&=&1/\mu,\textrm{ c.s.}
\end{eqnarray}
\end{Teo}


Es decir, $T_{n}$ satisface la Ley Fuerte de los Grandes N\'umeros s\'i y s\'olo s\'i $N\left/t\right)$ la cumple.


\begin{Coro}[Ley Fuerte de los Grandes N\'umeros para Procesos de Renovaci\'on]
Si $N\left(t\right)$ es un proceso de renovaci\'on cuyos tiempos de inter-renovaci\'on tienen media $\mu\leq\infty$, entonces
\begin{eqnarray}
t^{-1}N\left(t\right)\rightarrow 1/\mu,\textrm{ c.s. cuando }t\rightarrow\infty.
\end{eqnarray}

\end{Coro}


Considerar el proceso estoc\'astico de valores reales $\left\{Z\left(t\right):t\geq0\right\}$ en el mismo espacio de probabilidad que $N\left(t\right)$

\begin{Def}
Para el proceso $\left\{Z\left(t\right):t\geq0\right\}$ se define la fluctuaci\'on m\'axima de $Z\left(t\right)$ en el intervalo $\left(T_{n-1},T_{n}\right]$:
\begin{eqnarray*}
M_{n}=\sup_{T_{n-1}<t\leq T_{n}}|Z\left(t\right)-Z\left(T_{n-1}\right)|
\end{eqnarray*}
\end{Def}

\begin{Teo}
Sup\'ongase que $n^{-1}T_{n}\rightarrow\mu$ c.s. cuando $n\rightarrow\infty$, donde $\mu\leq\infty$ es una constante o variable aleatoria. Sea $a$ una constante o variable aleatoria que puede ser infinita cuando $\mu$ es finita, y considere las expresiones l\'imite:
\begin{eqnarray}
lim_{n\rightarrow\infty}n^{-1}Z\left(T_{n}\right)&=&a,\textrm{ c.s.}\\
lim_{t\rightarrow\infty}t^{-1}Z\left(t\right)&=&a/\mu,\textrm{ c.s.}
\end{eqnarray}
La segunda expresi\'on implica la primera. Conversamente, la primera implica la segunda si el proceso $Z\left(t\right)$ es creciente, o si $lim_{n\rightarrow\infty}n^{-1}M_{n}=0$ c.s.
\end{Teo}

\begin{Coro}
Si $N\left(t\right)$ es un proceso de renovaci\'on, y $\left(Z\left(T_{n}\right)-Z\left(T_{n-1}\right),M_{n}\right)$, para $n\geq1$, son variables aleatorias independientes e id\'enticamente distribuidas con media finita, entonces,
\begin{eqnarray}
lim_{t\rightarrow\infty}t^{-1}Z\left(t\right)\rightarrow\frac{\esp\left[Z\left(T_{1}\right)-Z\left(T_{0}\right)\right]}{\esp\left[T_{1}\right]},\textrm{ c.s. cuando  }t\rightarrow\infty.
\end{eqnarray}
\end{Coro}


%___________________________________________________________________________________________
%
\subsection{Funci\'on de Renovaci\'on}
%___________________________________________________________________________________________
%


\begin{Def}
Sea $h\left(t\right)$ funci\'on de valores reales en $\rea$ acotada en intervalos finitos e igual a cero para $t<0$ La ecuaci\'on de renovaci\'on para $h\left(t\right)$ y la distribuci\'on $F$ es

\begin{eqnarray}%\label{Ec.Renovacion}
H\left(t\right)=h\left(t\right)+\int_{\left[0,t\right]}H\left(t-s\right)dF\left(s\right)\textrm{,    }t\geq0,
\end{eqnarray}
donde $H\left(t\right)$ es una funci\'on de valores reales. Esto es $H=h+F\star H$. Decimos que $H\left(t\right)$ es soluci\'on de esta ecuaci\'on si satisface la ecuaci\'on, y es acotada en intervalos finitos e iguales a cero para $t<0$.
\end{Def}

\begin{Prop}
La funci\'on $U\star h\left(t\right)$ es la \'unica soluci\'on de la ecuaci\'on de renovaci\'on (\ref{Ec.Renovacion}).
\end{Prop}

\begin{Teo}[Teorema Renovaci\'on Elemental]
\begin{eqnarray*}
t^{-1}U\left(t\right)\rightarrow 1/\mu\textrm{,    cuando }t\rightarrow\infty.
\end{eqnarray*}
\end{Teo}

%___________________________________________________________________________________________
%
\subsection{Funci\'on de Renovaci\'on}
%___________________________________________________________________________________________
%


Sup\'ongase que $N\left(t\right)$ es un proceso de renovaci\'on con distribuci\'on $F$ con media finita $\mu$.

\begin{Def}
La funci\'on de renovaci\'on asociada con la distribuci\'on $F$, del proceso $N\left(t\right)$, es
\begin{eqnarray*}
U\left(t\right)=\sum_{n=1}^{\infty}F^{n\star}\left(t\right),\textrm{   }t\geq0,
\end{eqnarray*}
donde $F^{0\star}\left(t\right)=\indora\left(t\geq0\right)$.
\end{Def}


\begin{Prop}
Sup\'ongase que la distribuci\'on de inter-renovaci\'on $F$ tiene densidad $f$. Entonces $U\left(t\right)$ tambi\'en tiene densidad, para $t>0$, y es $U^{'}\left(t\right)=\sum_{n=0}^{\infty}f^{n\star}\left(t\right)$. Adem\'as
\begin{eqnarray*}
\prob\left\{N\left(t\right)>N\left(t-\right)\right\}=0\textrm{,   }t\geq0.
\end{eqnarray*}
\end{Prop}

\begin{Def}
La Transformada de Laplace-Stieljes de $F$ est\'a dada por

\begin{eqnarray*}
\hat{F}\left(\alpha\right)=\int_{\rea_{+}}e^{-\alpha t}dF\left(t\right)\textrm{,  }\alpha\geq0.
\end{eqnarray*}
\end{Def}

Entonces

\begin{eqnarray*}
\hat{U}\left(\alpha\right)=\sum_{n=0}^{\infty}\hat{F^{n\star}}\left(\alpha\right)=\sum_{n=0}^{\infty}\hat{F}\left(\alpha\right)^{n}=\frac{1}{1-\hat{F}\left(\alpha\right)}.
\end{eqnarray*}


\begin{Prop}
La Transformada de Laplace $\hat{U}\left(\alpha\right)$ y $\hat{F}\left(\alpha\right)$ determina una a la otra de manera \'unica por la relaci\'on $\hat{U}\left(\alpha\right)=\frac{1}{1-\hat{F}\left(\alpha\right)}$.
\end{Prop}


\begin{Note}
Un proceso de renovaci\'on $N\left(t\right)$ cuyos tiempos de inter-renovaci\'on tienen media finita, es un proceso Poisson con tasa $\lambda$ si y s\'olo s\'i $\esp\left[U\left(t\right)\right]=\lambda t$, para $t\geq0$.
\end{Note}


\begin{Teo}
Sea $N\left(t\right)$ un proceso puntual simple con puntos de localizaci\'on $T_{n}$ tal que $\eta\left(t\right)=\esp\left[N\left(\right)\right]$ es finita para cada $t$. Entonces para cualquier funci\'on $f:\rea_{+}\rightarrow\rea$,
\begin{eqnarray*}
\esp\left[\sum_{n=1}^{N\left(\right)}f\left(T_{n}\right)\right]=\int_{\left(0,t\right]}f\left(s\right)d\eta\left(s\right)\textrm{,  }t\geq0,
\end{eqnarray*}
suponiendo que la integral exista. Adem\'as si $X_{1},X_{2},\ldots$ son variables aleatorias definidas en el mismo espacio de probabilidad que el proceso $N\left(t\right)$ tal que $\esp\left[X_{n}|T_{n}=s\right]=f\left(s\right)$, independiente de $n$. Entonces
\begin{eqnarray*}
\esp\left[\sum_{n=1}^{N\left(t\right)}X_{n}\right]=\int_{\left(0,t\right]}f\left(s\right)d\eta\left(s\right)\textrm{,  }t\geq0,
\end{eqnarray*} 
suponiendo que la integral exista. 
\end{Teo}

\begin{Coro}[Identidad de Wald para Renovaciones]
Para el proceso de renovaci\'on $N\left(t\right)$,
\begin{eqnarray*}
\esp\left[T_{N\left(t\right)+1}\right]=\mu\esp\left[N\left(t\right)+1\right]\textrm{,  }t\geq0,
\end{eqnarray*}  
\end{Coro}

%______________________________________________________________________
\subsection{Procesos de Renovaci\'on}
%______________________________________________________________________

\begin{Def}%\label{Def.Tn}
Sean $0\leq T_{1}\leq T_{2}\leq \ldots$ son tiempos aleatorios infinitos en los cuales ocurren ciertos eventos. El n\'umero de tiempos $T_{n}$ en el intervalo $\left[0,t\right)$ es

\begin{eqnarray}
N\left(t\right)=\sum_{n=1}^{\infty}\indora\left(T_{n}\leq t\right),
\end{eqnarray}
para $t\geq0$.
\end{Def}

Si se consideran los puntos $T_{n}$ como elementos de $\rea_{+}$, y $N\left(t\right)$ es el n\'umero de puntos en $\rea$. El proceso denotado por $\left\{N\left(t\right):t\geq0\right\}$, denotado por $N\left(t\right)$, es un proceso puntual en $\rea_{+}$. Los $T_{n}$ son los tiempos de ocurrencia, el proceso puntual $N\left(t\right)$ es simple si su n\'umero de ocurrencias son distintas: $0<T_{1}<T_{2}<\ldots$ casi seguramente.

\begin{Def}
Un proceso puntual $N\left(t\right)$ es un proceso de renovaci\'on si los tiempos de interocurrencia $\xi_{n}=T_{n}-T_{n-1}$, para $n\geq1$, son independientes e identicamente distribuidos con distribuci\'on $F$, donde $F\left(0\right)=0$ y $T_{0}=0$. Los $T_{n}$ son llamados tiempos de renovaci\'on, referente a la independencia o renovaci\'on de la informaci\'on estoc\'astica en estos tiempos. Los $\xi_{n}$ son los tiempos de inter-renovaci\'on, y $N\left(t\right)$ es el n\'umero de renovaciones en el intervalo $\left[0,t\right)$
\end{Def}


\begin{Note}
Para definir un proceso de renovaci\'on para cualquier contexto, solamente hay que especificar una distribuci\'on $F$, con $F\left(0\right)=0$, para los tiempos de inter-renovaci\'on. La funci\'on $F$ en turno degune las otra variables aleatorias. De manera formal, existe un espacio de probabilidad y una sucesi\'on de variables aleatorias $\xi_{1},\xi_{2},\ldots$ definidas en este con distribuci\'on $F$. Entonces las otras cantidades son $T_{n}=\sum_{k=1}^{n}\xi_{k}$ y $N\left(t\right)=\sum_{n=1}^{\infty}\indora\left(T_{n}\leq t\right)$, donde $T_{n}\rightarrow\infty$ casi seguramente por la Ley Fuerte de los Grandes Números.
\end{Note}

%___________________________________________________________________________________________
%
\subsection{Renewal and Regenerative Processes: Serfozo\cite{Serfozo}}
%___________________________________________________________________________________________
%
\begin{Def}%\label{Def.Tn}
Sean $0\leq T_{1}\leq T_{2}\leq \ldots$ son tiempos aleatorios infinitos en los cuales ocurren ciertos eventos. El n\'umero de tiempos $T_{n}$ en el intervalo $\left[0,t\right)$ es

\begin{eqnarray}
N\left(t\right)=\sum_{n=1}^{\infty}\indora\left(T_{n}\leq t\right),
\end{eqnarray}
para $t\geq0$.
\end{Def}

Si se consideran los puntos $T_{n}$ como elementos de $\rea_{+}$, y $N\left(t\right)$ es el n\'umero de puntos en $\rea$. El proceso denotado por $\left\{N\left(t\right):t\geq0\right\}$, denotado por $N\left(t\right)$, es un proceso puntual en $\rea_{+}$. Los $T_{n}$ son los tiempos de ocurrencia, el proceso puntual $N\left(t\right)$ es simple si su n\'umero de ocurrencias son distintas: $0<T_{1}<T_{2}<\ldots$ casi seguramente.

\begin{Def}
Un proceso puntual $N\left(t\right)$ es un proceso de renovaci\'on si los tiempos de interocurrencia $\xi_{n}=T_{n}-T_{n-1}$, para $n\geq1$, son independientes e identicamente distribuidos con distribuci\'on $F$, donde $F\left(0\right)=0$ y $T_{0}=0$. Los $T_{n}$ son llamados tiempos de renovaci\'on, referente a la independencia o renovaci\'on de la informaci\'on estoc\'astica en estos tiempos. Los $\xi_{n}$ son los tiempos de inter-renovaci\'on, y $N\left(t\right)$ es el n\'umero de renovaciones en el intervalo $\left[0,t\right)$
\end{Def}


\begin{Note}
Para definir un proceso de renovaci\'on para cualquier contexto, solamente hay que especificar una distribuci\'on $F$, con $F\left(0\right)=0$, para los tiempos de inter-renovaci\'on. La funci\'on $F$ en turno degune las otra variables aleatorias. De manera formal, existe un espacio de probabilidad y una sucesi\'on de variables aleatorias $\xi_{1},\xi_{2},\ldots$ definidas en este con distribuci\'on $F$. Entonces las otras cantidades son $T_{n}=\sum_{k=1}^{n}\xi_{k}$ y $N\left(t\right)=\sum_{n=1}^{\infty}\indora\left(T_{n}\leq t\right)$, donde $T_{n}\rightarrow\infty$ casi seguramente por la Ley Fuerte de los Grandes N\'umeros.
\end{Note}

Los tiempos $T_{n}$ est\'an relacionados con los conteos de $N\left(t\right)$ por

\begin{eqnarray*}
\left\{N\left(t\right)\geq n\right\}&=&\left\{T_{n}\leq t\right\}\\
T_{N\left(t\right)}\leq &t&<T_{N\left(t\right)+1},
\end{eqnarray*}

adem\'as $N\left(T_{n}\right)=n$, y 

\begin{eqnarray*}
N\left(t\right)=\max\left\{n:T_{n}\leq t\right\}=\min\left\{n:T_{n+1}>t\right\}
\end{eqnarray*}

Por propiedades de la convoluci\'on se sabe que

\begin{eqnarray*}
P\left\{T_{n}\leq t\right\}=F^{n\star}\left(t\right)
\end{eqnarray*}
que es la $n$-\'esima convoluci\'on de $F$. Entonces 

\begin{eqnarray*}
\left\{N\left(t\right)\geq n\right\}&=&\left\{T_{n}\leq t\right\}\\
P\left\{N\left(t\right)\leq n\right\}&=&1-F^{\left(n+1\right)\star}\left(t\right)
\end{eqnarray*}

Adem\'as usando el hecho de que $\esp\left[N\left(t\right)\right]=\sum_{n=1}^{\infty}P\left\{N\left(t\right)\geq n\right\}$
se tiene que

\begin{eqnarray*}
\esp\left[N\left(t\right)\right]=\sum_{n=1}^{\infty}F^{n\star}\left(t\right)
\end{eqnarray*}

\begin{Prop}
Para cada $t\geq0$, la funci\'on generadora de momentos $\esp\left[e^{\alpha N\left(t\right)}\right]$ existe para alguna $\alpha$ en una vecindad del 0, y de aqu\'i que $\esp\left[N\left(t\right)^{m}\right]<\infty$, para $m\geq1$.
\end{Prop}

\begin{Ejem}[\textbf{Proceso Poisson}]

Suponga que se tienen tiempos de inter-renovaci\'on \textit{i.i.d.} del proceso de renovaci\'on $N\left(t\right)$ tienen distribuci\'on exponencial $F\left(t\right)=q-e^{-\lambda t}$ con tasa $\lambda$. Entonces $N\left(t\right)$ es un proceso Poisson con tasa $\lambda$.

\end{Ejem}


\begin{Note}
Si el primer tiempo de renovaci\'on $\xi_{1}$ no tiene la misma distribuci\'on que el resto de las $\xi_{n}$, para $n\geq2$, a $N\left(t\right)$ se le llama Proceso de Renovaci\'on retardado, donde si $\xi$ tiene distribuci\'on $G$, entonces el tiempo $T_{n}$ de la $n$-\'esima renovaci\'on tiene distribuci\'on $G\star F^{\left(n-1\right)\star}\left(t\right)$
\end{Note}


\begin{Teo}
Para una constante $\mu\leq\infty$ ( o variable aleatoria), las siguientes expresiones son equivalentes:

\begin{eqnarray}
lim_{n\rightarrow\infty}n^{-1}T_{n}&=&\mu,\textrm{ c.s.}\\
lim_{t\rightarrow\infty}t^{-1}N\left(t\right)&=&1/\mu,\textrm{ c.s.}
\end{eqnarray}
\end{Teo}


Es decir, $T_{n}$ satisface la Ley Fuerte de los Grandes N\'umeros s\'i y s\'olo s\'i $N\left/t\right)$ la cumple.


\begin{Coro}[Ley Fuerte de los Grandes N\'umeros para Procesos de Renovaci\'on]
Si $N\left(t\right)$ es un proceso de renovaci\'on cuyos tiempos de inter-renovaci\'on tienen media $\mu\leq\infty$, entonces
\begin{eqnarray}
t^{-1}N\left(t\right)\rightarrow 1/\mu,\textrm{ c.s. cuando }t\rightarrow\infty.
\end{eqnarray}

\end{Coro}


Considerar el proceso estoc\'astico de valores reales $\left\{Z\left(t\right):t\geq0\right\}$ en el mismo espacio de probabilidad que $N\left(t\right)$

\begin{Def}
Para el proceso $\left\{Z\left(t\right):t\geq0\right\}$ se define la fluctuaci\'on m\'axima de $Z\left(t\right)$ en el intervalo $\left(T_{n-1},T_{n}\right]$:
\begin{eqnarray*}
M_{n}=\sup_{T_{n-1}<t\leq T_{n}}|Z\left(t\right)-Z\left(T_{n-1}\right)|
\end{eqnarray*}
\end{Def}

\begin{Teo}
Sup\'ongase que $n^{-1}T_{n}\rightarrow\mu$ c.s. cuando $n\rightarrow\infty$, donde $\mu\leq\infty$ es una constante o variable aleatoria. Sea $a$ una constante o variable aleatoria que puede ser infinita cuando $\mu$ es finita, y considere las expresiones l\'imite:
\begin{eqnarray}
lim_{n\rightarrow\infty}n^{-1}Z\left(T_{n}\right)&=&a,\textrm{ c.s.}\\
lim_{t\rightarrow\infty}t^{-1}Z\left(t\right)&=&a/\mu,\textrm{ c.s.}
\end{eqnarray}
La segunda expresi\'on implica la primera. Conversamente, la primera implica la segunda si el proceso $Z\left(t\right)$ es creciente, o si $lim_{n\rightarrow\infty}n^{-1}M_{n}=0$ c.s.
\end{Teo}

\begin{Coro}
Si $N\left(t\right)$ es un proceso de renovaci\'on, y $\left(Z\left(T_{n}\right)-Z\left(T_{n-1}\right),M_{n}\right)$, para $n\geq1$, son variables aleatorias independientes e id\'enticamente distribuidas con media finita, entonces,
\begin{eqnarray}
lim_{t\rightarrow\infty}t^{-1}Z\left(t\right)\rightarrow\frac{\esp\left[Z\left(T_{1}\right)-Z\left(T_{0}\right)\right]}{\esp\left[T_{1}\right]},\textrm{ c.s. cuando  }t\rightarrow\infty.
\end{eqnarray}
\end{Coro}


Sup\'ongase que $N\left(t\right)$ es un proceso de renovaci\'on con distribuci\'on $F$ con media finita $\mu$.

\begin{Def}
La funci\'on de renovaci\'on asociada con la distribuci\'on $F$, del proceso $N\left(t\right)$, es
\begin{eqnarray*}
U\left(t\right)=\sum_{n=1}^{\infty}F^{n\star}\left(t\right),\textrm{   }t\geq0,
\end{eqnarray*}
donde $F^{0\star}\left(t\right)=\indora\left(t\geq0\right)$.
\end{Def}


\begin{Prop}
Sup\'ongase que la distribuci\'on de inter-renovaci\'on $F$ tiene densidad $f$. Entonces $U\left(t\right)$ tambi\'en tiene densidad, para $t>0$, y es $U^{'}\left(t\right)=\sum_{n=0}^{\infty}f^{n\star}\left(t\right)$. Adem\'as
\begin{eqnarray*}
\prob\left\{N\left(t\right)>N\left(t-\right)\right\}=0\textrm{,   }t\geq0.
\end{eqnarray*}
\end{Prop}

\begin{Def}
La Transformada de Laplace-Stieljes de $F$ est\'a dada por

\begin{eqnarray*}
\hat{F}\left(\alpha\right)=\int_{\rea_{+}}e^{-\alpha t}dF\left(t\right)\textrm{,  }\alpha\geq0.
\end{eqnarray*}
\end{Def}

Entonces

\begin{eqnarray*}
\hat{U}\left(\alpha\right)=\sum_{n=0}^{\infty}\hat{F^{n\star}}\left(\alpha\right)=\sum_{n=0}^{\infty}\hat{F}\left(\alpha\right)^{n}=\frac{1}{1-\hat{F}\left(\alpha\right)}.
\end{eqnarray*}


\begin{Prop}
La Transformada de Laplace $\hat{U}\left(\alpha\right)$ y $\hat{F}\left(\alpha\right)$ determina una a la otra de manera \'unica por la relaci\'on $\hat{U}\left(\alpha\right)=\frac{1}{1-\hat{F}\left(\alpha\right)}$.
\end{Prop}


\begin{Note}
Un proceso de renovaci\'on $N\left(t\right)$ cuyos tiempos de inter-renovaci\'on tienen media finita, es un proceso Poisson con tasa $\lambda$ si y s\'olo s\'i $\esp\left[U\left(t\right)\right]=\lambda t$, para $t\geq0$.
\end{Note}


\begin{Teo}
Sea $N\left(t\right)$ un proceso puntual simple con puntos de localizaci\'on $T_{n}$ tal que $\eta\left(t\right)=\esp\left[N\left(\right)\right]$ es finita para cada $t$. Entonces para cualquier funci\'on $f:\rea_{+}\rightarrow\rea$,
\begin{eqnarray*}
\esp\left[\sum_{n=1}^{N\left(\right)}f\left(T_{n}\right)\right]=\int_{\left(0,t\right]}f\left(s\right)d\eta\left(s\right)\textrm{,  }t\geq0,
\end{eqnarray*}
suponiendo que la integral exista. Adem\'as si $X_{1},X_{2},\ldots$ son variables aleatorias definidas en el mismo espacio de probabilidad que el proceso $N\left(t\right)$ tal que $\esp\left[X_{n}|T_{n}=s\right]=f\left(s\right)$, independiente de $n$. Entonces
\begin{eqnarray*}
\esp\left[\sum_{n=1}^{N\left(t\right)}X_{n}\right]=\int_{\left(0,t\right]}f\left(s\right)d\eta\left(s\right)\textrm{,  }t\geq0,
\end{eqnarray*} 
suponiendo que la integral exista. 
\end{Teo}

\begin{Coro}[Identidad de Wald para Renovaciones]
Para el proceso de renovaci\'on $N\left(t\right)$,
\begin{eqnarray*}
\esp\left[T_{N\left(t\right)+1}\right]=\mu\esp\left[N\left(t\right)+1\right]\textrm{,  }t\geq0,
\end{eqnarray*}  
\end{Coro}


\begin{Def}
Sea $h\left(t\right)$ funci\'on de valores reales en $\rea$ acotada en intervalos finitos e igual a cero para $t<0$ La ecuaci\'on de renovaci\'on para $h\left(t\right)$ y la distribuci\'on $F$ es

\begin{eqnarray}%\label{Ec.Renovacion}
H\left(t\right)=h\left(t\right)+\int_{\left[0,t\right]}H\left(t-s\right)dF\left(s\right)\textrm{,    }t\geq0,
\end{eqnarray}
donde $H\left(t\right)$ es una funci\'on de valores reales. Esto es $H=h+F\star H$. Decimos que $H\left(t\right)$ es soluci\'on de esta ecuaci\'on si satisface la ecuaci\'on, y es acotada en intervalos finitos e iguales a cero para $t<0$.
\end{Def}

\begin{Prop}
La funci\'on $U\star h\left(t\right)$ es la \'unica soluci\'on de la ecuaci\'on de renovaci\'on (\ref{Ec.Renovacion}).
\end{Prop}

\begin{Teo}[Teorema Renovaci\'on Elemental]
\begin{eqnarray*}
t^{-1}U\left(t\right)\rightarrow 1/\mu\textrm{,    cuando }t\rightarrow\infty.
\end{eqnarray*}
\end{Teo}



Sup\'ongase que $N\left(t\right)$ es un proceso de renovaci\'on con distribuci\'on $F$ con media finita $\mu$.

\begin{Def}
La funci\'on de renovaci\'on asociada con la distribuci\'on $F$, del proceso $N\left(t\right)$, es
\begin{eqnarray*}
U\left(t\right)=\sum_{n=1}^{\infty}F^{n\star}\left(t\right),\textrm{   }t\geq0,
\end{eqnarray*}
donde $F^{0\star}\left(t\right)=\indora\left(t\geq0\right)$.
\end{Def}


\begin{Prop}
Sup\'ongase que la distribuci\'on de inter-renovaci\'on $F$ tiene densidad $f$. Entonces $U\left(t\right)$ tambi\'en tiene densidad, para $t>0$, y es $U^{'}\left(t\right)=\sum_{n=0}^{\infty}f^{n\star}\left(t\right)$. Adem\'as
\begin{eqnarray*}
\prob\left\{N\left(t\right)>N\left(t-\right)\right\}=0\textrm{,   }t\geq0.
\end{eqnarray*}
\end{Prop}

\begin{Def}
La Transformada de Laplace-Stieljes de $F$ est\'a dada por

\begin{eqnarray*}
\hat{F}\left(\alpha\right)=\int_{\rea_{+}}e^{-\alpha t}dF\left(t\right)\textrm{,  }\alpha\geq0.
\end{eqnarray*}
\end{Def}

Entonces

\begin{eqnarray*}
\hat{U}\left(\alpha\right)=\sum_{n=0}^{\infty}\hat{F^{n\star}}\left(\alpha\right)=\sum_{n=0}^{\infty}\hat{F}\left(\alpha\right)^{n}=\frac{1}{1-\hat{F}\left(\alpha\right)}.
\end{eqnarray*}


\begin{Prop}
La Transformada de Laplace $\hat{U}\left(\alpha\right)$ y $\hat{F}\left(\alpha\right)$ determina una a la otra de manera \'unica por la relaci\'on $\hat{U}\left(\alpha\right)=\frac{1}{1-\hat{F}\left(\alpha\right)}$.
\end{Prop}


\begin{Note}
Un proceso de renovaci\'on $N\left(t\right)$ cuyos tiempos de inter-renovaci\'on tienen media finita, es un proceso Poisson con tasa $\lambda$ si y s\'olo s\'i $\esp\left[U\left(t\right)\right]=\lambda t$, para $t\geq0$.
\end{Note}


\begin{Teo}
Sea $N\left(t\right)$ un proceso puntual simple con puntos de localizaci\'on $T_{n}$ tal que $\eta\left(t\right)=\esp\left[N\left(\right)\right]$ es finita para cada $t$. Entonces para cualquier funci\'on $f:\rea_{+}\rightarrow\rea$,
\begin{eqnarray*}
\esp\left[\sum_{n=1}^{N\left(\right)}f\left(T_{n}\right)\right]=\int_{\left(0,t\right]}f\left(s\right)d\eta\left(s\right)\textrm{,  }t\geq0,
\end{eqnarray*}
suponiendo que la integral exista. Adem\'as si $X_{1},X_{2},\ldots$ son variables aleatorias definidas en el mismo espacio de probabilidad que el proceso $N\left(t\right)$ tal que $\esp\left[X_{n}|T_{n}=s\right]=f\left(s\right)$, independiente de $n$. Entonces
\begin{eqnarray*}
\esp\left[\sum_{n=1}^{N\left(t\right)}X_{n}\right]=\int_{\left(0,t\right]}f\left(s\right)d\eta\left(s\right)\textrm{,  }t\geq0,
\end{eqnarray*} 
suponiendo que la integral exista. 
\end{Teo}

\begin{Coro}[Identidad de Wald para Renovaciones]
Para el proceso de renovaci\'on $N\left(t\right)$,
\begin{eqnarray*}
\esp\left[T_{N\left(t\right)+1}\right]=\mu\esp\left[N\left(t\right)+1\right]\textrm{,  }t\geq0,
\end{eqnarray*}  
\end{Coro}


\begin{Def}
Sea $h\left(t\right)$ funci\'on de valores reales en $\rea$ acotada en intervalos finitos e igual a cero para $t<0$ La ecuaci\'on de renovaci\'on para $h\left(t\right)$ y la distribuci\'on $F$ es

\begin{eqnarray}%\label{Ec.Renovacion}
H\left(t\right)=h\left(t\right)+\int_{\left[0,t\right]}H\left(t-s\right)dF\left(s\right)\textrm{,    }t\geq0,
\end{eqnarray}
donde $H\left(t\right)$ es una funci\'on de valores reales. Esto es $H=h+F\star H$. Decimos que $H\left(t\right)$ es soluci\'on de esta ecuaci\'on si satisface la ecuaci\'on, y es acotada en intervalos finitos e iguales a cero para $t<0$.
\end{Def}

\begin{Prop}
La funci\'on $U\star h\left(t\right)$ es la \'unica soluci\'on de la ecuaci\'on de renovaci\'on (\ref{Ec.Renovacion}).
\end{Prop}

\begin{Teo}[Teorema Renovaci\'on Elemental]
\begin{eqnarray*}
t^{-1}U\left(t\right)\rightarrow 1/\mu\textrm{,    cuando }t\rightarrow\infty.
\end{eqnarray*}
\end{Teo}


\begin{Note} Una funci\'on $h:\rea_{+}\rightarrow\rea$ es Directamente Riemann Integrable en los siguientes casos:
\begin{itemize}
\item[a)] $h\left(t\right)\geq0$ es decreciente y Riemann Integrable.
\item[b)] $h$ es continua excepto posiblemente en un conjunto de Lebesgue de medida 0, y $|h\left(t\right)|\leq b\left(t\right)$, donde $b$ es DRI.
\end{itemize}
\end{Note}

\begin{Teo}[Teorema Principal de Renovaci\'on]
Si $F$ es no aritm\'etica y $h\left(t\right)$ es Directamente Riemann Integrable (DRI), entonces

\begin{eqnarray*}
lim_{t\rightarrow\infty}U\star h=\frac{1}{\mu}\int_{\rea_{+}}h\left(s\right)ds.
\end{eqnarray*}
\end{Teo}

\begin{Prop}
Cualquier funci\'on $H\left(t\right)$ acotada en intervalos finitos y que es 0 para $t<0$ puede expresarse como
\begin{eqnarray*}
H\left(t\right)=U\star h\left(t\right)\textrm{,  donde }h\left(t\right)=H\left(t\right)-F\star H\left(t\right)
\end{eqnarray*}
\end{Prop}

\begin{Def}
Un proceso estoc\'astico $X\left(t\right)$ es crudamente regenerativo en un tiempo aleatorio positivo $T$ si
\begin{eqnarray*}
\esp\left[X\left(T+t\right)|T\right]=\esp\left[X\left(t\right)\right]\textrm{, para }t\geq0,\end{eqnarray*}
y con las esperanzas anteriores finitas.
\end{Def}

\begin{Prop}
Sup\'ongase que $X\left(t\right)$ es un proceso crudamente regenerativo en $T$, que tiene distribuci\'on $F$. Si $\esp\left[X\left(t\right)\right]$ es acotado en intervalos finitos, entonces
\begin{eqnarray*}
\esp\left[X\left(t\right)\right]=U\star h\left(t\right)\textrm{,  donde }h\left(t\right)=\esp\left[X\left(t\right)\indora\left(T>t\right)\right].
\end{eqnarray*}
\end{Prop}

\begin{Teo}[Regeneraci\'on Cruda]
Sup\'ongase que $X\left(t\right)$ es un proceso con valores positivo crudamente regenerativo en $T$, y def\'inase $M=\sup\left\{|X\left(t\right)|:t\leq T\right\}$. Si $T$ es no aritm\'etico y $M$ y $MT$ tienen media finita, entonces
\begin{eqnarray*}
lim_{t\rightarrow\infty}\esp\left[X\left(t\right)\right]=\frac{1}{\mu}\int_{\rea_{+}}h\left(s\right)ds,
\end{eqnarray*}
donde $h\left(t\right)=\esp\left[X\left(t\right)\indora\left(T>t\right)\right]$.
\end{Teo}


\begin{Note} Una funci\'on $h:\rea_{+}\rightarrow\rea$ es Directamente Riemann Integrable en los siguientes casos:
\begin{itemize}
\item[a)] $h\left(t\right)\geq0$ es decreciente y Riemann Integrable.
\item[b)] $h$ es continua excepto posiblemente en un conjunto de Lebesgue de medida 0, y $|h\left(t\right)|\leq b\left(t\right)$, donde $b$ es DRI.
\end{itemize}
\end{Note}

\begin{Teo}[Teorema Principal de Renovaci\'on]
Si $F$ es no aritm\'etica y $h\left(t\right)$ es Directamente Riemann Integrable (DRI), entonces

\begin{eqnarray*}
lim_{t\rightarrow\infty}U\star h=\frac{1}{\mu}\int_{\rea_{+}}h\left(s\right)ds.
\end{eqnarray*}
\end{Teo}

\begin{Prop}
Cualquier funci\'on $H\left(t\right)$ acotada en intervalos finitos y que es 0 para $t<0$ puede expresarse como
\begin{eqnarray*}
H\left(t\right)=U\star h\left(t\right)\textrm{,  donde }h\left(t\right)=H\left(t\right)-F\star H\left(t\right)
\end{eqnarray*}
\end{Prop}

\begin{Def}
Un proceso estoc\'astico $X\left(t\right)$ es crudamente regenerativo en un tiempo aleatorio positivo $T$ si
\begin{eqnarray*}
\esp\left[X\left(T+t\right)|T\right]=\esp\left[X\left(t\right)\right]\textrm{, para }t\geq0,\end{eqnarray*}
y con las esperanzas anteriores finitas.
\end{Def}

\begin{Prop}
Sup\'ongase que $X\left(t\right)$ es un proceso crudamente regenerativo en $T$, que tiene distribuci\'on $F$. Si $\esp\left[X\left(t\right)\right]$ es acotado en intervalos finitos, entonces
\begin{eqnarray*}
\esp\left[X\left(t\right)\right]=U\star h\left(t\right)\textrm{,  donde }h\left(t\right)=\esp\left[X\left(t\right)\indora\left(T>t\right)\right].
\end{eqnarray*}
\end{Prop}

\begin{Teo}[Regeneraci\'on Cruda]
Sup\'ongase que $X\left(t\right)$ es un proceso con valores positivo crudamente regenerativo en $T$, y def\'inase $M=\sup\left\{|X\left(t\right)|:t\leq T\right\}$. Si $T$ es no aritm\'etico y $M$ y $MT$ tienen media finita, entonces
\begin{eqnarray*}
lim_{t\rightarrow\infty}\esp\left[X\left(t\right)\right]=\frac{1}{\mu}\int_{\rea_{+}}h\left(s\right)ds,
\end{eqnarray*}
donde $h\left(t\right)=\esp\left[X\left(t\right)\indora\left(T>t\right)\right]$.
\end{Teo}

\begin{Def}
Para el proceso $\left\{\left(N\left(t\right),X\left(t\right)\right):t\geq0\right\}$, sus trayectoria muestrales en el intervalo de tiempo $\left[T_{n-1},T_{n}\right)$ est\'an descritas por
\begin{eqnarray*}
\zeta_{n}=\left(\xi_{n},\left\{X\left(T_{n-1}+t\right):0\leq t<\xi_{n}\right\}\right)
\end{eqnarray*}
Este $\zeta_{n}$ es el $n$-\'esimo segmento del proceso. El proceso es regenerativo sobre los tiempos $T_{n}$ si sus segmentos $\zeta_{n}$ son independientes e id\'enticamennte distribuidos.
\end{Def}


\begin{Note}
Si $\tilde{X}\left(t\right)$ con espacio de estados $\tilde{S}$ es regenerativo sobre $T_{n}$, entonces $X\left(t\right)=f\left(\tilde{X}\left(t\right)\right)$ tambi\'en es regenerativo sobre $T_{n}$, para cualquier funci\'on $f:\tilde{S}\rightarrow S$.
\end{Note}

\begin{Note}
Los procesos regenerativos son crudamente regenerativos, pero no al rev\'es.
\end{Note}


\begin{Note}
Un proceso estoc\'astico a tiempo continuo o discreto es regenerativo si existe un proceso de renovaci\'on  tal que los segmentos del proceso entre tiempos de renovaci\'on sucesivos son i.i.d., es decir, para $\left\{X\left(t\right):t\geq0\right\}$ proceso estoc\'astico a tiempo continuo con espacio de estados $S$, espacio m\'etrico.
\end{Note}

Para $\left\{X\left(t\right):t\geq0\right\}$ Proceso Estoc\'astico a tiempo continuo con estado de espacios $S$, que es un espacio m\'etrico, con trayectorias continuas por la derecha y con l\'imites por la izquierda c.s. Sea $N\left(t\right)$ un proceso de renovaci\'on en $\rea_{+}$ definido en el mismo espacio de probabilidad que $X\left(t\right)$, con tiempos de renovaci\'on $T$ y tiempos de inter-renovaci\'on $\xi_{n}=T_{n}-T_{n-1}$, con misma distribuci\'on $F$ de media finita $\mu$.



\begin{Def}
Para el proceso $\left\{\left(N\left(t\right),X\left(t\right)\right):t\geq0\right\}$, sus trayectoria muestrales en el intervalo de tiempo $\left[T_{n-1},T_{n}\right)$ est\'an descritas por
\begin{eqnarray*}
\zeta_{n}=\left(\xi_{n},\left\{X\left(T_{n-1}+t\right):0\leq t<\xi_{n}\right\}\right)
\end{eqnarray*}
Este $\zeta_{n}$ es el $n$-\'esimo segmento del proceso. El proceso es regenerativo sobre los tiempos $T_{n}$ si sus segmentos $\zeta_{n}$ son independientes e id\'enticamennte distribuidos.
\end{Def}

\begin{Note}
Un proceso regenerativo con media de la longitud de ciclo finita es llamado positivo recurrente.
\end{Note}

\begin{Teo}[Procesos Regenerativos]
Suponga que el proceso
\end{Teo}


\begin{Def}[Renewal Process Trinity]
Para un proceso de renovaci\'on $N\left(t\right)$, los siguientes procesos proveen de informaci\'on sobre los tiempos de renovaci\'on.
\begin{itemize}
\item $A\left(t\right)=t-T_{N\left(t\right)}$, el tiempo de recurrencia hacia atr\'as al tiempo $t$, que es el tiempo desde la \'ultima renovaci\'on para $t$.

\item $B\left(t\right)=T_{N\left(t\right)+1}-t$, el tiempo de recurrencia hacia adelante al tiempo $t$, residual del tiempo de renovaci\'on, que es el tiempo para la pr\'oxima renovaci\'on despu\'es de $t$.

\item $L\left(t\right)=\xi_{N\left(t\right)+1}=A\left(t\right)+B\left(t\right)$, la longitud del intervalo de renovaci\'on que contiene a $t$.
\end{itemize}
\end{Def}

\begin{Note}
El proceso tridimensional $\left(A\left(t\right),B\left(t\right),L\left(t\right)\right)$ es regenerativo sobre $T_{n}$, y por ende cada proceso lo es. Cada proceso $A\left(t\right)$ y $B\left(t\right)$ son procesos de MArkov a tiempo continuo con trayectorias continuas por partes en el espacio de estados $\rea_{+}$. Una expresi\'on conveniente para su distribuci\'on conjunta es, para $0\leq x<t,y\geq0$
\begin{equation}\label{NoRenovacion}
P\left\{A\left(t\right)>x,B\left(t\right)>y\right\}=
P\left\{N\left(t+y\right)-N\left((t-x)\right)=0\right\}
\end{equation}
\end{Note}

\begin{Ejem}[Tiempos de recurrencia Poisson]
Si $N\left(t\right)$ es un proceso Poisson con tasa $\lambda$, entonces de la expresi\'on (\ref{NoRenovacion}) se tiene que

\begin{eqnarray*}
\begin{array}{lc}
P\left\{A\left(t\right)>x,B\left(t\right)>y\right\}=e^{-\lambda\left(x+y\right)},&0\leq x<t,y\geq0,
\end{array}
\end{eqnarray*}
que es la probabilidad Poisson de no renovaciones en un intervalo de longitud $x+y$.

\end{Ejem}

\begin{Note}
Una cadena de Markov erg\'odica tiene la propiedad de ser estacionaria si la distribuci\'on de su estado al tiempo $0$ es su distribuci\'on estacionaria.
\end{Note}


\begin{Def}
Un proceso estoc\'astico a tiempo continuo $\left\{X\left(t\right):t\geq0\right\}$ en un espacio general es estacionario si sus distribuciones finito dimensionales son invariantes bajo cualquier  traslado: para cada $0\leq s_{1}<s_{2}<\cdots<s_{k}$ y $t\geq0$,
\begin{eqnarray*}
\left(X\left(s_{1}+t\right),\ldots,X\left(s_{k}+t\right)\right)=_{d}\left(X\left(s_{1}\right),\ldots,X\left(s_{k}\right)\right).
\end{eqnarray*}
\end{Def}

\begin{Note}
Un proceso de Markov es estacionario si $X\left(t\right)=_{d}X\left(0\right)$, $t\geq0$.
\end{Note}

Considerese el proceso $N\left(t\right)=\sum_{n}\indora\left(\tau_{n}\leq t\right)$ en $\rea_{+}$, con puntos $0<\tau_{1}<\tau_{2}<\cdots$.

\begin{Prop}
Si $N$ es un proceso puntual estacionario y $\esp\left[N\left(1\right)\right]<\infty$, entonces $\esp\left[N\left(t\right)\right]=t\esp\left[N\left(1\right)\right]$, $t\geq0$

\end{Prop}

\begin{Teo}
Los siguientes enunciados son equivalentes
\begin{itemize}
\item[i)] El proceso retardado de renovaci\'on $N$ es estacionario.

\item[ii)] EL proceso de tiempos de recurrencia hacia adelante $B\left(t\right)$ es estacionario.


\item[iii)] $\esp\left[N\left(t\right)\right]=t/\mu$,


\item[iv)] $G\left(t\right)=F_{e}\left(t\right)=\frac{1}{\mu}\int_{0}^{t}\left[1-F\left(s\right)\right]ds$
\end{itemize}
Cuando estos enunciados son ciertos, $P\left\{B\left(t\right)\leq x\right\}=F_{e}\left(x\right)$, para $t,x\geq0$.

\end{Teo}

\begin{Note}
Una consecuencia del teorema anterior es que el Proceso Poisson es el \'unico proceso sin retardo que es estacionario.
\end{Note}

\begin{Coro}
El proceso de renovaci\'on $N\left(t\right)$ sin retardo, y cuyos tiempos de inter renonaci\'on tienen media finita, es estacionario si y s\'olo si es un proceso Poisson.

\end{Coro}


%________________________________________________________________________
\subsection{Procesos Regenerativos}
%________________________________________________________________________



\begin{Note}
Si $\tilde{X}\left(t\right)$ con espacio de estados $\tilde{S}$ es regenerativo sobre $T_{n}$, entonces $X\left(t\right)=f\left(\tilde{X}\left(t\right)\right)$ tambi\'en es regenerativo sobre $T_{n}$, para cualquier funci\'on $f:\tilde{S}\rightarrow S$.
\end{Note}

\begin{Note}
Los procesos regenerativos son crudamente regenerativos, pero no al rev\'es.
\end{Note}
%\subsection*{Procesos Regenerativos: Sigman\cite{Sigman1}}
\begin{Def}[Definici\'on Cl\'asica]
Un proceso estoc\'astico $X=\left\{X\left(t\right):t\geq0\right\}$ es llamado regenerativo is existe una variable aleatoria $R_{1}>0$ tal que
\begin{itemize}
\item[i)] $\left\{X\left(t+R_{1}\right):t\geq0\right\}$ es independiente de $\left\{\left\{X\left(t\right):t<R_{1}\right\},\right\}$
\item[ii)] $\left\{X\left(t+R_{1}\right):t\geq0\right\}$ es estoc\'asticamente equivalente a $\left\{X\left(t\right):t>0\right\}$
\end{itemize}

Llamamos a $R_{1}$ tiempo de regeneraci\'on, y decimos que $X$ se regenera en este punto.
\end{Def}

$\left\{X\left(t+R_{1}\right)\right\}$ es regenerativo con tiempo de regeneraci\'on $R_{2}$, independiente de $R_{1}$ pero con la misma distribuci\'on que $R_{1}$. Procediendo de esta manera se obtiene una secuencia de variables aleatorias independientes e id\'enticamente distribuidas $\left\{R_{n}\right\}$ llamados longitudes de ciclo. Si definimos a $Z_{k}\equiv R_{1}+R_{2}+\cdots+R_{k}$, se tiene un proceso de renovaci\'on llamado proceso de renovaci\'on encajado para $X$.




\begin{Def}
Para $x$ fijo y para cada $t\geq0$, sea $I_{x}\left(t\right)=1$ si $X\left(t\right)\leq x$,  $I_{x}\left(t\right)=0$ en caso contrario, y def\'inanse los tiempos promedio
\begin{eqnarray*}
\overline{X}&=&lim_{t\rightarrow\infty}\frac{1}{t}\int_{0}^{\infty}X\left(u\right)du\\
\prob\left(X_{\infty}\leq x\right)&=&lim_{t\rightarrow\infty}\frac{1}{t}\int_{0}^{\infty}I_{x}\left(u\right)du,
\end{eqnarray*}
cuando estos l\'imites existan.
\end{Def}

Como consecuencia del teorema de Renovaci\'on-Recompensa, se tiene que el primer l\'imite  existe y es igual a la constante
\begin{eqnarray*}
\overline{X}&=&\frac{\esp\left[\int_{0}^{R_{1}}X\left(t\right)dt\right]}{\esp\left[R_{1}\right]},
\end{eqnarray*}
suponiendo que ambas esperanzas son finitas.

\begin{Note}
\begin{itemize}
\item[a)] Si el proceso regenerativo $X$ es positivo recurrente y tiene trayectorias muestrales no negativas, entonces la ecuaci\'on anterior es v\'alida.
\item[b)] Si $X$ es positivo recurrente regenerativo, podemos construir una \'unica versi\'on estacionaria de este proceso, $X_{e}=\left\{X_{e}\left(t\right)\right\}$, donde $X_{e}$ es un proceso estoc\'astico regenerativo y estrictamente estacionario, con distribuci\'on marginal distribuida como $X_{\infty}$
\end{itemize}
\end{Note}

%________________________________________________________________________
\subsection{Procesos Regenerativos}
%________________________________________________________________________

Para $\left\{X\left(t\right):t\geq0\right\}$ Proceso Estoc\'astico a tiempo continuo con estado de espacios $S$, que es un espacio m\'etrico, con trayectorias continuas por la derecha y con l\'imites por la izquierda c.s. Sea $N\left(t\right)$ un proceso de renovaci\'on en $\rea_{+}$ definido en el mismo espacio de probabilidad que $X\left(t\right)$, con tiempos de renovaci\'on $T$ y tiempos de inter-renovaci\'on $\xi_{n}=T_{n}-T_{n-1}$, con misma distribuci\'on $F$ de media finita $\mu$.



\begin{Def}
Para el proceso $\left\{\left(N\left(t\right),X\left(t\right)\right):t\geq0\right\}$, sus trayectoria muestrales en el intervalo de tiempo $\left[T_{n-1},T_{n}\right)$ est\'an descritas por
\begin{eqnarray*}
\zeta_{n}=\left(\xi_{n},\left\{X\left(T_{n-1}+t\right):0\leq t<\xi_{n}\right\}\right)
\end{eqnarray*}
Este $\zeta_{n}$ es el $n$-\'esimo segmento del proceso. El proceso es regenerativo sobre los tiempos $T_{n}$ si sus segmentos $\zeta_{n}$ son independientes e id\'enticamennte distribuidos.
\end{Def}


\begin{Note}
Si $\tilde{X}\left(t\right)$ con espacio de estados $\tilde{S}$ es regenerativo sobre $T_{n}$, entonces $X\left(t\right)=f\left(\tilde{X}\left(t\right)\right)$ tambi\'en es regenerativo sobre $T_{n}$, para cualquier funci\'on $f:\tilde{S}\rightarrow S$.
\end{Note}

\begin{Note}
Los procesos regenerativos son crudamente regenerativos, pero no al rev\'es.
\end{Note}

\begin{Def}[Definici\'on Cl\'asica]
Un proceso estoc\'astico $X=\left\{X\left(t\right):t\geq0\right\}$ es llamado regenerativo is existe una variable aleatoria $R_{1}>0$ tal que
\begin{itemize}
\item[i)] $\left\{X\left(t+R_{1}\right):t\geq0\right\}$ es independiente de $\left\{\left\{X\left(t\right):t<R_{1}\right\},\right\}$
\item[ii)] $\left\{X\left(t+R_{1}\right):t\geq0\right\}$ es estoc\'asticamente equivalente a $\left\{X\left(t\right):t>0\right\}$
\end{itemize}

Llamamos a $R_{1}$ tiempo de regeneraci\'on, y decimos que $X$ se regenera en este punto.
\end{Def}

$\left\{X\left(t+R_{1}\right)\right\}$ es regenerativo con tiempo de regeneraci\'on $R_{2}$, independiente de $R_{1}$ pero con la misma distribuci\'on que $R_{1}$. Procediendo de esta manera se obtiene una secuencia de variables aleatorias independientes e id\'enticamente distribuidas $\left\{R_{n}\right\}$ llamados longitudes de ciclo. Si definimos a $Z_{k}\equiv R_{1}+R_{2}+\cdots+R_{k}$, se tiene un proceso de renovaci\'on llamado proceso de renovaci\'on encajado para $X$.

\begin{Note}
Un proceso regenerativo con media de la longitud de ciclo finita es llamado positivo recurrente.
\end{Note}


\begin{Def}
Para $x$ fijo y para cada $t\geq0$, sea $I_{x}\left(t\right)=1$ si $X\left(t\right)\leq x$,  $I_{x}\left(t\right)=0$ en caso contrario, y def\'inanse los tiempos promedio
\begin{eqnarray*}
\overline{X}&=&lim_{t\rightarrow\infty}\frac{1}{t}\int_{0}^{\infty}X\left(u\right)du\\
\prob\left(X_{\infty}\leq x\right)&=&lim_{t\rightarrow\infty}\frac{1}{t}\int_{0}^{\infty}I_{x}\left(u\right)du,
\end{eqnarray*}
cuando estos l\'imites existan.
\end{Def}

Como consecuencia del teorema de Renovaci\'on-Recompensa, se tiene que el primer l\'imite  existe y es igual a la constante
\begin{eqnarray*}
\overline{X}&=&\frac{\esp\left[\int_{0}^{R_{1}}X\left(t\right)dt\right]}{\esp\left[R_{1}\right]},
\end{eqnarray*}
suponiendo que ambas esperanzas son finitas.

\begin{Note}
\begin{itemize}
\item[a)] Si el proceso regenerativo $X$ es positivo recurrente y tiene trayectorias muestrales no negativas, entonces la ecuaci\'on anterior es v\'alida.
\item[b)] Si $X$ es positivo recurrente regenerativo, podemos construir una \'unica versi\'on estacionaria de este proceso, $X_{e}=\left\{X_{e}\left(t\right)\right\}$, donde $X_{e}$ es un proceso estoc\'astico regenerativo y estrictamente estacionario, con distribuci\'on marginal distribuida como $X_{\infty}$
\end{itemize}
\end{Note}

%__________________________________________________________________________________________
\subsection{Procesos Regenerativos Estacionarios - Stidham \cite{Stidham}}
%__________________________________________________________________________________________


Un proceso estoc\'astico a tiempo continuo $\left\{V\left(t\right),t\geq0\right\}$ es un proceso regenerativo si existe una sucesi\'on de variables aleatorias independientes e id\'enticamente distribuidas $\left\{X_{1},X_{2},\ldots\right\}$, sucesi\'on de renovaci\'on, tal que para cualquier conjunto de Borel $A$, 

\begin{eqnarray*}
\prob\left\{V\left(t\right)\in A|X_{1}+X_{2}+\cdots+X_{R\left(t\right)}=s,\left\{V\left(\tau\right),\tau<s\right\}\right\}=\prob\left\{V\left(t-s\right)\in A|X_{1}>t-s\right\},
\end{eqnarray*}
para todo $0\leq s\leq t$, donde $R\left(t\right)=\max\left\{X_{1}+X_{2}+\cdots+X_{j}\leq t\right\}=$n\'umero de renovaciones ({\emph{puntos de regeneraci\'on}}) que ocurren en $\left[0,t\right]$. El intervalo $\left[0,X_{1}\right)$ es llamado {\emph{primer ciclo de regeneraci\'on}} de $\left\{V\left(t \right),t\geq0\right\}$, $\left[X_{1},X_{1}+X_{2}\right)$ el {\emph{segundo ciclo de regeneraci\'on}}, y as\'i sucesivamente.

Sea $X=X_{1}$ y sea $F$ la funci\'on de distrbuci\'on de $X$


\begin{Def}
Se define el proceso estacionario, $\left\{V^{*}\left(t\right),t\geq0\right\}$, para $\left\{V\left(t\right),t\geq0\right\}$ por

\begin{eqnarray*}
\prob\left\{V\left(t\right)\in A\right\}=\frac{1}{\esp\left[X\right]}\int_{0}^{\infty}\prob\left\{V\left(t+x\right)\in A|X>x\right\}\left(1-F\left(x\right)\right)dx,
\end{eqnarray*} 
para todo $t\geq0$ y todo conjunto de Borel $A$.
\end{Def}

\begin{Def}
Una distribuci\'on se dice que es {\emph{aritm\'etica}} si todos sus puntos de incremento son m\'ultiplos de la forma $0,\lambda, 2\lambda,\ldots$ para alguna $\lambda>0$ entera.
\end{Def}


\begin{Def}
Una modificaci\'on medible de un proceso $\left\{V\left(t\right),t\geq0\right\}$, es una versi\'on de este, $\left\{V\left(t,w\right)\right\}$ conjuntamente medible para $t\geq0$ y para $w\in S$, $S$ espacio de estados para $\left\{V\left(t\right),t\geq0\right\}$.
\end{Def}

\begin{Teo}
Sea $\left\{V\left(t\right),t\geq\right\}$ un proceso regenerativo no negativo con modificaci\'on medible. Sea $\esp\left[X\right]<\infty$. Entonces el proceso estacionario dado por la ecuaci\'on anterior est\'a bien definido y tiene funci\'on de distribuci\'on independiente de $t$, adem\'as
\begin{itemize}
\item[i)] \begin{eqnarray*}
\esp\left[V^{*}\left(0\right)\right]&=&\frac{\esp\left[\int_{0}^{X}V\left(s\right)ds\right]}{\esp\left[X\right]}\end{eqnarray*}
\item[ii)] Si $\esp\left[V^{*}\left(0\right)\right]<\infty$, equivalentemente, si $\esp\left[\int_{0}^{X}V\left(s\right)ds\right]<\infty$,entonces
\begin{eqnarray*}
\frac{\int_{0}^{t}V\left(s\right)ds}{t}\rightarrow\frac{\esp\left[\int_{0}^{X}V\left(s\right)ds\right]}{\esp\left[X\right]}
\end{eqnarray*}
con probabilidad 1 y en media, cuando $t\rightarrow\infty$.
\end{itemize}
\end{Teo}

%__________________________________________________________________________________________
\subsection{Procesos Regenerativos Estacionarios - Stidham \cite{Stidham}}
%__________________________________________________________________________________________


Un proceso estoc\'astico a tiempo continuo $\left\{V\left(t\right),t\geq0\right\}$ es un proceso regenerativo si existe una sucesi\'on de variables aleatorias independientes e id\'enticamente distribuidas $\left\{X_{1},X_{2},\ldots\right\}$, sucesi\'on de renovaci\'on, tal que para cualquier conjunto de Borel $A$, 

\begin{eqnarray*}
\prob\left\{V\left(t\right)\in A|X_{1}+X_{2}+\cdots+X_{R\left(t\right)}=s,\left\{V\left(\tau\right),\tau<s\right\}\right\}=\prob\left\{V\left(t-s\right)\in A|X_{1}>t-s\right\},
\end{eqnarray*}
para todo $0\leq s\leq t$, donde $R\left(t\right)=\max\left\{X_{1}+X_{2}+\cdots+X_{j}\leq t\right\}=$n\'umero de renovaciones ({\emph{puntos de regeneraci\'on}}) que ocurren en $\left[0,t\right]$. El intervalo $\left[0,X_{1}\right)$ es llamado {\emph{primer ciclo de regeneraci\'on}} de $\left\{V\left(t \right),t\geq0\right\}$, $\left[X_{1},X_{1}+X_{2}\right)$ el {\emph{segundo ciclo de regeneraci\'on}}, y as\'i sucesivamente.

Sea $X=X_{1}$ y sea $F$ la funci\'on de distrbuci\'on de $X$


\begin{Def}
Se define el proceso estacionario, $\left\{V^{*}\left(t\right),t\geq0\right\}$, para $\left\{V\left(t\right),t\geq0\right\}$ por

\begin{eqnarray*}
\prob\left\{V\left(t\right)\in A\right\}=\frac{1}{\esp\left[X\right]}\int_{0}^{\infty}\prob\left\{V\left(t+x\right)\in A|X>x\right\}\left(1-F\left(x\right)\right)dx,
\end{eqnarray*} 
para todo $t\geq0$ y todo conjunto de Borel $A$.
\end{Def}

\begin{Def}
Una distribuci\'on se dice que es {\emph{aritm\'etica}} si todos sus puntos de incremento son m\'ultiplos de la forma $0,\lambda, 2\lambda,\ldots$ para alguna $\lambda>0$ entera.
\end{Def}


\begin{Def}
Una modificaci\'on medible de un proceso $\left\{V\left(t\right),t\geq0\right\}$, es una versi\'on de este, $\left\{V\left(t,w\right)\right\}$ conjuntamente medible para $t\geq0$ y para $w\in S$, $S$ espacio de estados para $\left\{V\left(t\right),t\geq0\right\}$.
\end{Def}

\begin{Teo}
Sea $\left\{V\left(t\right),t\geq\right\}$ un proceso regenerativo no negativo con modificaci\'on medible. Sea $\esp\left[X\right]<\infty$. Entonces el proceso estacionario dado por la ecuaci\'on anterior est\'a bien definido y tiene funci\'on de distribuci\'on independiente de $t$, adem\'as
\begin{itemize}
\item[i)] \begin{eqnarray*}
\esp\left[V^{*}\left(0\right)\right]&=&\frac{\esp\left[\int_{0}^{X}V\left(s\right)ds\right]}{\esp\left[X\right]}\end{eqnarray*}
\item[ii)] Si $\esp\left[V^{*}\left(0\right)\right]<\infty$, equivalentemente, si $\esp\left[\int_{0}^{X}V\left(s\right)ds\right]<\infty$,entonces
\begin{eqnarray*}
\frac{\int_{0}^{t}V\left(s\right)ds}{t}\rightarrow\frac{\esp\left[\int_{0}^{X}V\left(s\right)ds\right]}{\esp\left[X\right]}
\end{eqnarray*}
con probabilidad 1 y en media, cuando $t\rightarrow\infty$.
\end{itemize}
\end{Teo}

Para $\left\{X\left(t\right):t\geq0\right\}$ Proceso Estoc\'astico a tiempo continuo con estado de espacios $S$, que es un espacio m\'etrico, con trayectorias continuas por la derecha y con l\'imites por la izquierda c.s. Sea $N\left(t\right)$ un proceso de renovaci\'on en $\rea_{+}$ definido en el mismo espacio de probabilidad que $X\left(t\right)$, con tiempos de renovaci\'on $T$ y tiempos de inter-renovaci\'on $\xi_{n}=T_{n}-T_{n-1}$, con misma distribuci\'on $F$ de media finita $\mu$.



\subsection{Renewal and Regenerative Processes: Serfozo\cite{Serfozo}}
\begin{Def}\label{Def.Tn}
Sean $0\leq T_{1}\leq T_{2}\leq \ldots$ son tiempos aleatorios infinitos en los cuales ocurren ciertos eventos. El n\'umero de tiempos $T_{n}$ en el intervalo $\left[0,t\right)$ es

\begin{eqnarray}
N\left(t\right)=\sum_{n=1}^{\infty}\indora\left(T_{n}\leq t\right),
\end{eqnarray}
para $t\geq0$.
\end{Def}

Si se consideran los puntos $T_{n}$ como elementos de $\rea_{+}$, y $N\left(t\right)$ es el n\'umero de puntos en $\rea$. El proceso denotado por $\left\{N\left(t\right):t\geq0\right\}$, denotado por $N\left(t\right)$, es un proceso puntual en $\rea_{+}$. Los $T_{n}$ son los tiempos de ocurrencia, el proceso puntual $N\left(t\right)$ es simple si su n\'umero de ocurrencias son distintas: $0<T_{1}<T_{2}<\ldots$ casi seguramente.

\begin{Def}
Un proceso puntual $N\left(t\right)$ es un proceso de renovaci\'on si los tiempos de interocurrencia $\xi_{n}=T_{n}-T_{n-1}$, para $n\geq1$, son independientes e identicamente distribuidos con distribuci\'on $F$, donde $F\left(0\right)=0$ y $T_{0}=0$. Los $T_{n}$ son llamados tiempos de renovaci\'on, referente a la independencia o renovaci\'on de la informaci\'on estoc\'astica en estos tiempos. Los $\xi_{n}$ son los tiempos de inter-renovaci\'on, y $N\left(t\right)$ es el n\'umero de renovaciones en el intervalo $\left[0,t\right)$
\end{Def}


\begin{Note}
Para definir un proceso de renovaci\'on para cualquier contexto, solamente hay que especificar una distribuci\'on $F$, con $F\left(0\right)=0$, para los tiempos de inter-renovaci\'on. La funci\'on $F$ en turno degune las otra variables aleatorias. De manera formal, existe un espacio de probabilidad y una sucesi\'on de variables aleatorias $\xi_{1},\xi_{2},\ldots$ definidas en este con distribuci\'on $F$. Entonces las otras cantidades son $T_{n}=\sum_{k=1}^{n}\xi_{k}$ y $N\left(t\right)=\sum_{n=1}^{\infty}\indora\left(T_{n}\leq t\right)$, donde $T_{n}\rightarrow\infty$ casi seguramente por la Ley Fuerte de los Grandes N\'umeros.
\end{Note}


Los tiempos $T_{n}$ est\'an relacionados con los conteos de $N\left(t\right)$ por

\begin{eqnarray*}
\left\{N\left(t\right)\geq n\right\}&=&\left\{T_{n}\leq t\right\}\\
T_{N\left(t\right)}\leq &t&<T_{N\left(t\right)+1},
\end{eqnarray*}

adem\'as $N\left(T_{n}\right)=n$, y 

\begin{eqnarray*}
N\left(t\right)=\max\left\{n:T_{n}\leq t\right\}=\min\left\{n:T_{n+1}>t\right\}
\end{eqnarray*}

Por propiedades de la convoluci\'on se sabe que

\begin{eqnarray*}
P\left\{T_{n}\leq t\right\}=F^{n\star}\left(t\right)
\end{eqnarray*}
que es la $n$-\'esima convoluci\'on de $F$. Entonces 

\begin{eqnarray*}
\left\{N\left(t\right)\geq n\right\}&=&\left\{T_{n}\leq t\right\}\\
P\left\{N\left(t\right)\leq n\right\}&=&1-F^{\left(n+1\right)\star}\left(t\right)
\end{eqnarray*}

Adem\'as usando el hecho de que $\esp\left[N\left(t\right)\right]=\sum_{n=1}^{\infty}P\left\{N\left(t\right)\geq n\right\}$
se tiene que

\begin{eqnarray*}
\esp\left[N\left(t\right)\right]=\sum_{n=1}^{\infty}F^{n\star}\left(t\right)
\end{eqnarray*}

\begin{Prop}
Para cada $t\geq0$, la funci\'on generadora de momentos $\esp\left[e^{\alpha N\left(t\right)}\right]$ existe para alguna $\alpha$ en una vecindad del 0, y de aqu\'i que $\esp\left[N\left(t\right)^{m}\right]<\infty$, para $m\geq1$.
\end{Prop}


\begin{Note}
Si el primer tiempo de renovaci\'on $\xi_{1}$ no tiene la misma distribuci\'on que el resto de las $\xi_{n}$, para $n\geq2$, a $N\left(t\right)$ se le llama Proceso de Renovaci\'on retardado, donde si $\xi$ tiene distribuci\'on $G$, entonces el tiempo $T_{n}$ de la $n$-\'esima renovaci\'on tiene distribuci\'on $G\star F^{\left(n-1\right)\star}\left(t\right)$
\end{Note}


\begin{Teo}
Para una constante $\mu\leq\infty$ ( o variable aleatoria), las siguientes expresiones son equivalentes:

\begin{eqnarray}
lim_{n\rightarrow\infty}n^{-1}T_{n}&=&\mu,\textrm{ c.s.}\\
lim_{t\rightarrow\infty}t^{-1}N\left(t\right)&=&1/\mu,\textrm{ c.s.}
\end{eqnarray}
\end{Teo}


Es decir, $T_{n}$ satisface la Ley Fuerte de los Grandes N\'umeros s\'i y s\'olo s\'i $N\left/t\right)$ la cumple.


\begin{Coro}[Ley Fuerte de los Grandes N\'umeros para Procesos de Renovaci\'on]
Si $N\left(t\right)$ es un proceso de renovaci\'on cuyos tiempos de inter-renovaci\'on tienen media $\mu\leq\infty$, entonces
\begin{eqnarray}
t^{-1}N\left(t\right)\rightarrow 1/\mu,\textrm{ c.s. cuando }t\rightarrow\infty.
\end{eqnarray}

\end{Coro}


Considerar el proceso estoc\'astico de valores reales $\left\{Z\left(t\right):t\geq0\right\}$ en el mismo espacio de probabilidad que $N\left(t\right)$

\begin{Def}
Para el proceso $\left\{Z\left(t\right):t\geq0\right\}$ se define la fluctuaci\'on m\'axima de $Z\left(t\right)$ en el intervalo $\left(T_{n-1},T_{n}\right]$:
\begin{eqnarray*}
M_{n}=\sup_{T_{n-1}<t\leq T_{n}}|Z\left(t\right)-Z\left(T_{n-1}\right)|
\end{eqnarray*}
\end{Def}

\begin{Teo}
Sup\'ongase que $n^{-1}T_{n}\rightarrow\mu$ c.s. cuando $n\rightarrow\infty$, donde $\mu\leq\infty$ es una constante o variable aleatoria. Sea $a$ una constante o variable aleatoria que puede ser infinita cuando $\mu$ es finita, y considere las expresiones l\'imite:
\begin{eqnarray}
lim_{n\rightarrow\infty}n^{-1}Z\left(T_{n}\right)&=&a,\textrm{ c.s.}\\
lim_{t\rightarrow\infty}t^{-1}Z\left(t\right)&=&a/\mu,\textrm{ c.s.}
\end{eqnarray}
La segunda expresi\'on implica la primera. Conversamente, la primera implica la segunda si el proceso $Z\left(t\right)$ es creciente, o si $lim_{n\rightarrow\infty}n^{-1}M_{n}=0$ c.s.
\end{Teo}

\begin{Coro}
Si $N\left(t\right)$ es un proceso de renovaci\'on, y $\left(Z\left(T_{n}\right)-Z\left(T_{n-1}\right),M_{n}\right)$, para $n\geq1$, son variables aleatorias independientes e id\'enticamente distribuidas con media finita, entonces,
\begin{eqnarray}
lim_{t\rightarrow\infty}t^{-1}Z\left(t\right)\rightarrow\frac{\esp\left[Z\left(T_{1}\right)-Z\left(T_{0}\right)\right]}{\esp\left[T_{1}\right]},\textrm{ c.s. cuando  }t\rightarrow\infty.
\end{eqnarray}
\end{Coro}


Sup\'ongase que $N\left(t\right)$ es un proceso de renovaci\'on con distribuci\'on $F$ con media finita $\mu$.

\begin{Def}
La funci\'on de renovaci\'on asociada con la distribuci\'on $F$, del proceso $N\left(t\right)$, es
\begin{eqnarray*}
U\left(t\right)=\sum_{n=1}^{\infty}F^{n\star}\left(t\right),\textrm{   }t\geq0,
\end{eqnarray*}
donde $F^{0\star}\left(t\right)=\indora\left(t\geq0\right)$.
\end{Def}


\begin{Prop}
Sup\'ongase que la distribuci\'on de inter-renovaci\'on $F$ tiene densidad $f$. Entonces $U\left(t\right)$ tambi\'en tiene densidad, para $t>0$, y es $U^{'}\left(t\right)=\sum_{n=0}^{\infty}f^{n\star}\left(t\right)$. Adem\'as
\begin{eqnarray*}
\prob\left\{N\left(t\right)>N\left(t-\right)\right\}=0\textrm{,   }t\geq0.
\end{eqnarray*}
\end{Prop}

\begin{Def}
La Transformada de Laplace-Stieljes de $F$ est\'a dada por

\begin{eqnarray*}
\hat{F}\left(\alpha\right)=\int_{\rea_{+}}e^{-\alpha t}dF\left(t\right)\textrm{,  }\alpha\geq0.
\end{eqnarray*}
\end{Def}

Entonces

\begin{eqnarray*}
\hat{U}\left(\alpha\right)=\sum_{n=0}^{\infty}\hat{F^{n\star}}\left(\alpha\right)=\sum_{n=0}^{\infty}\hat{F}\left(\alpha\right)^{n}=\frac{1}{1-\hat{F}\left(\alpha\right)}.
\end{eqnarray*}


\begin{Prop}
La Transformada de Laplace $\hat{U}\left(\alpha\right)$ y $\hat{F}\left(\alpha\right)$ determina una a la otra de manera \'unica por la relaci\'on $\hat{U}\left(\alpha\right)=\frac{1}{1-\hat{F}\left(\alpha\right)}$.
\end{Prop}


\begin{Note}
Un proceso de renovaci\'on $N\left(t\right)$ cuyos tiempos de inter-renovaci\'on tienen media finita, es un proceso Poisson con tasa $\lambda$ si y s\'olo s\'i $\esp\left[U\left(t\right)\right]=\lambda t$, para $t\geq0$.
\end{Note}


\begin{Teo}
Sea $N\left(t\right)$ un proceso puntual simple con puntos de localizaci\'on $T_{n}$ tal que $\eta\left(t\right)=\esp\left[N\left(\right)\right]$ es finita para cada $t$. Entonces para cualquier funci\'on $f:\rea_{+}\rightarrow\rea$,
\begin{eqnarray*}
\esp\left[\sum_{n=1}^{N\left(\right)}f\left(T_{n}\right)\right]=\int_{\left(0,t\right]}f\left(s\right)d\eta\left(s\right)\textrm{,  }t\geq0,
\end{eqnarray*}
suponiendo que la integral exista. Adem\'as si $X_{1},X_{2},\ldots$ son variables aleatorias definidas en el mismo espacio de probabilidad que el proceso $N\left(t\right)$ tal que $\esp\left[X_{n}|T_{n}=s\right]=f\left(s\right)$, independiente de $n$. Entonces
\begin{eqnarray*}
\esp\left[\sum_{n=1}^{N\left(t\right)}X_{n}\right]=\int_{\left(0,t\right]}f\left(s\right)d\eta\left(s\right)\textrm{,  }t\geq0,
\end{eqnarray*} 
suponiendo que la integral exista. 
\end{Teo}

\begin{Coro}[Identidad de Wald para Renovaciones]
Para el proceso de renovaci\'on $N\left(t\right)$,
\begin{eqnarray*}
\esp\left[T_{N\left(t\right)+1}\right]=\mu\esp\left[N\left(t\right)+1\right]\textrm{,  }t\geq0,
\end{eqnarray*}  
\end{Coro}


\begin{Def}
Sea $h\left(t\right)$ funci\'on de valores reales en $\rea$ acotada en intervalos finitos e igual a cero para $t<0$ La ecuaci\'on de renovaci\'on para $h\left(t\right)$ y la distribuci\'on $F$ es

\begin{eqnarray}\label{Ec.Renovacion}
H\left(t\right)=h\left(t\right)+\int_{\left[0,t\right]}H\left(t-s\right)dF\left(s\right)\textrm{,    }t\geq0,
\end{eqnarray}
donde $H\left(t\right)$ es una funci\'on de valores reales. Esto es $H=h+F\star H$. Decimos que $H\left(t\right)$ es soluci\'on de esta ecuaci\'on si satisface la ecuaci\'on, y es acotada en intervalos finitos e iguales a cero para $t<0$.
\end{Def}

\begin{Prop}
La funci\'on $U\star h\left(t\right)$ es la \'unica soluci\'on de la ecuaci\'on de renovaci\'on (\ref{Ec.Renovacion}).
\end{Prop}

\begin{Teo}[Teorema Renovaci\'on Elemental]
\begin{eqnarray*}
t^{-1}U\left(t\right)\rightarrow 1/\mu\textrm{,    cuando }t\rightarrow\infty.
\end{eqnarray*}
\end{Teo}



Sup\'ongase que $N\left(t\right)$ es un proceso de renovaci\'on con distribuci\'on $F$ con media finita $\mu$.

\begin{Def}
La funci\'on de renovaci\'on asociada con la distribuci\'on $F$, del proceso $N\left(t\right)$, es
\begin{eqnarray*}
U\left(t\right)=\sum_{n=1}^{\infty}F^{n\star}\left(t\right),\textrm{   }t\geq0,
\end{eqnarray*}
donde $F^{0\star}\left(t\right)=\indora\left(t\geq0\right)$.
\end{Def}


\begin{Prop}
Sup\'ongase que la distribuci\'on de inter-renovaci\'on $F$ tiene densidad $f$. Entonces $U\left(t\right)$ tambi\'en tiene densidad, para $t>0$, y es $U^{'}\left(t\right)=\sum_{n=0}^{\infty}f^{n\star}\left(t\right)$. Adem\'as
\begin{eqnarray*}
\prob\left\{N\left(t\right)>N\left(t-\right)\right\}=0\textrm{,   }t\geq0.
\end{eqnarray*}
\end{Prop}

\begin{Def}
La Transformada de Laplace-Stieljes de $F$ est\'a dada por

\begin{eqnarray*}
\hat{F}\left(\alpha\right)=\int_{\rea_{+}}e^{-\alpha t}dF\left(t\right)\textrm{,  }\alpha\geq0.
\end{eqnarray*}
\end{Def}

Entonces

\begin{eqnarray*}
\hat{U}\left(\alpha\right)=\sum_{n=0}^{\infty}\hat{F^{n\star}}\left(\alpha\right)=\sum_{n=0}^{\infty}\hat{F}\left(\alpha\right)^{n}=\frac{1}{1-\hat{F}\left(\alpha\right)}.
\end{eqnarray*}


\begin{Prop}
La Transformada de Laplace $\hat{U}\left(\alpha\right)$ y $\hat{F}\left(\alpha\right)$ determina una a la otra de manera \'unica por la relaci\'on $\hat{U}\left(\alpha\right)=\frac{1}{1-\hat{F}\left(\alpha\right)}$.
\end{Prop}


\begin{Note}
Un proceso de renovaci\'on $N\left(t\right)$ cuyos tiempos de inter-renovaci\'on tienen media finita, es un proceso Poisson con tasa $\lambda$ si y s\'olo s\'i $\esp\left[U\left(t\right)\right]=\lambda t$, para $t\geq0$.
\end{Note}


\begin{Teo}
Sea $N\left(t\right)$ un proceso puntual simple con puntos de localizaci\'on $T_{n}$ tal que $\eta\left(t\right)=\esp\left[N\left(\right)\right]$ es finita para cada $t$. Entonces para cualquier funci\'on $f:\rea_{+}\rightarrow\rea$,
\begin{eqnarray*}
\esp\left[\sum_{n=1}^{N\left(\right)}f\left(T_{n}\right)\right]=\int_{\left(0,t\right]}f\left(s\right)d\eta\left(s\right)\textrm{,  }t\geq0,
\end{eqnarray*}
suponiendo que la integral exista. Adem\'as si $X_{1},X_{2},\ldots$ son variables aleatorias definidas en el mismo espacio de probabilidad que el proceso $N\left(t\right)$ tal que $\esp\left[X_{n}|T_{n}=s\right]=f\left(s\right)$, independiente de $n$. Entonces
\begin{eqnarray*}
\esp\left[\sum_{n=1}^{N\left(t\right)}X_{n}\right]=\int_{\left(0,t\right]}f\left(s\right)d\eta\left(s\right)\textrm{,  }t\geq0,
\end{eqnarray*} 
suponiendo que la integral exista. 
\end{Teo}

\begin{Coro}[Identidad de Wald para Renovaciones]
Para el proceso de renovaci\'on $N\left(t\right)$,
\begin{eqnarray*}
\esp\left[T_{N\left(t\right)+1}\right]=\mu\esp\left[N\left(t\right)+1\right]\textrm{,  }t\geq0,
\end{eqnarray*}  
\end{Coro}


\begin{Def}
Sea $h\left(t\right)$ funci\'on de valores reales en $\rea$ acotada en intervalos finitos e igual a cero para $t<0$ La ecuaci\'on de renovaci\'on para $h\left(t\right)$ y la distribuci\'on $F$ es

\begin{eqnarray}\label{Ec.Renovacion}
H\left(t\right)=h\left(t\right)+\int_{\left[0,t\right]}H\left(t-s\right)dF\left(s\right)\textrm{,    }t\geq0,
\end{eqnarray}
donde $H\left(t\right)$ es una funci\'on de valores reales. Esto es $H=h+F\star H$. Decimos que $H\left(t\right)$ es soluci\'on de esta ecuaci\'on si satisface la ecuaci\'on, y es acotada en intervalos finitos e iguales a cero para $t<0$.
\end{Def}

\begin{Prop}
La funci\'on $U\star h\left(t\right)$ es la \'unica soluci\'on de la ecuaci\'on de renovaci\'on (\ref{Ec.Renovacion}).
\end{Prop}

\begin{Teo}[Teorema Renovaci\'on Elemental]
\begin{eqnarray*}
t^{-1}U\left(t\right)\rightarrow 1/\mu\textrm{,    cuando }t\rightarrow\infty.
\end{eqnarray*}
\end{Teo}


\begin{Note} Una funci\'on $h:\rea_{+}\rightarrow\rea$ es Directamente Riemann Integrable en los siguientes casos:
\begin{itemize}
\item[a)] $h\left(t\right)\geq0$ es decreciente y Riemann Integrable.
\item[b)] $h$ es continua excepto posiblemente en un conjunto de Lebesgue de medida 0, y $|h\left(t\right)|\leq b\left(t\right)$, donde $b$ es DRI.
\end{itemize}
\end{Note}

\begin{Teo}[Teorema Principal de Renovaci\'on]
Si $F$ es no aritm\'etica y $h\left(t\right)$ es Directamente Riemann Integrable (DRI), entonces

\begin{eqnarray*}
lim_{t\rightarrow\infty}U\star h=\frac{1}{\mu}\int_{\rea_{+}}h\left(s\right)ds.
\end{eqnarray*}
\end{Teo}

\begin{Prop}
Cualquier funci\'on $H\left(t\right)$ acotada en intervalos finitos y que es 0 para $t<0$ puede expresarse como
\begin{eqnarray*}
H\left(t\right)=U\star h\left(t\right)\textrm{,  donde }h\left(t\right)=H\left(t\right)-F\star H\left(t\right)
\end{eqnarray*}
\end{Prop}

\begin{Def}
Un proceso estoc\'astico $X\left(t\right)$ es crudamente regenerativo en un tiempo aleatorio positivo $T$ si
\begin{eqnarray*}
\esp\left[X\left(T+t\right)|T\right]=\esp\left[X\left(t\right)\right]\textrm{, para }t\geq0,\end{eqnarray*}
y con las esperanzas anteriores finitas.
\end{Def}

\begin{Prop}
Sup\'ongase que $X\left(t\right)$ es un proceso crudamente regenerativo en $T$, que tiene distribuci\'on $F$. Si $\esp\left[X\left(t\right)\right]$ es acotado en intervalos finitos, entonces
\begin{eqnarray*}
\esp\left[X\left(t\right)\right]=U\star h\left(t\right)\textrm{,  donde }h\left(t\right)=\esp\left[X\left(t\right)\indora\left(T>t\right)\right].
\end{eqnarray*}
\end{Prop}

\begin{Teo}[Regeneraci\'on Cruda]
Sup\'ongase que $X\left(t\right)$ es un proceso con valores positivo crudamente regenerativo en $T$, y def\'inase $M=\sup\left\{|X\left(t\right)|:t\leq T\right\}$. Si $T$ es no aritm\'etico y $M$ y $MT$ tienen media finita, entonces
\begin{eqnarray*}
lim_{t\rightarrow\infty}\esp\left[X\left(t\right)\right]=\frac{1}{\mu}\int_{\rea_{+}}h\left(s\right)ds,
\end{eqnarray*}
donde $h\left(t\right)=\esp\left[X\left(t\right)\indora\left(T>t\right)\right]$.
\end{Teo}


\begin{Note} Una funci\'on $h:\rea_{+}\rightarrow\rea$ es Directamente Riemann Integrable en los siguientes casos:
\begin{itemize}
\item[a)] $h\left(t\right)\geq0$ es decreciente y Riemann Integrable.
\item[b)] $h$ es continua excepto posiblemente en un conjunto de Lebesgue de medida 0, y $|h\left(t\right)|\leq b\left(t\right)$, donde $b$ es DRI.
\end{itemize}
\end{Note}

\begin{Teo}[Teorema Principal de Renovaci\'on]
Si $F$ es no aritm\'etica y $h\left(t\right)$ es Directamente Riemann Integrable (DRI), entonces

\begin{eqnarray*}
lim_{t\rightarrow\infty}U\star h=\frac{1}{\mu}\int_{\rea_{+}}h\left(s\right)ds.
\end{eqnarray*}
\end{Teo}

\begin{Prop}
Cualquier funci\'on $H\left(t\right)$ acotada en intervalos finitos y que es 0 para $t<0$ puede expresarse como
\begin{eqnarray*}
H\left(t\right)=U\star h\left(t\right)\textrm{,  donde }h\left(t\right)=H\left(t\right)-F\star H\left(t\right)
\end{eqnarray*}
\end{Prop}

\begin{Def}
Un proceso estoc\'astico $X\left(t\right)$ es crudamente regenerativo en un tiempo aleatorio positivo $T$ si
\begin{eqnarray*}
\esp\left[X\left(T+t\right)|T\right]=\esp\left[X\left(t\right)\right]\textrm{, para }t\geq0,\end{eqnarray*}
y con las esperanzas anteriores finitas.
\end{Def}

\begin{Prop}
Sup\'ongase que $X\left(t\right)$ es un proceso crudamente regenerativo en $T$, que tiene distribuci\'on $F$. Si $\esp\left[X\left(t\right)\right]$ es acotado en intervalos finitos, entonces
\begin{eqnarray*}
\esp\left[X\left(t\right)\right]=U\star h\left(t\right)\textrm{,  donde }h\left(t\right)=\esp\left[X\left(t\right)\indora\left(T>t\right)\right].
\end{eqnarray*}
\end{Prop}

\begin{Teo}[Regeneraci\'on Cruda]
Sup\'ongase que $X\left(t\right)$ es un proceso con valores positivo crudamente regenerativo en $T$, y def\'inase $M=\sup\left\{|X\left(t\right)|:t\leq T\right\}$. Si $T$ es no aritm\'etico y $M$ y $MT$ tienen media finita, entonces
\begin{eqnarray*}
lim_{t\rightarrow\infty}\esp\left[X\left(t\right)\right]=\frac{1}{\mu}\int_{\rea_{+}}h\left(s\right)ds,
\end{eqnarray*}
donde $h\left(t\right)=\esp\left[X\left(t\right)\indora\left(T>t\right)\right]$.
\end{Teo}

%________________________________________________________________________
\subsection{Procesos Regenerativos}
%________________________________________________________________________

Para $\left\{X\left(t\right):t\geq0\right\}$ Proceso Estoc\'astico a tiempo continuo con estado de espacios $S$, que es un espacio m\'etrico, con trayectorias continuas por la derecha y con l\'imites por la izquierda c.s. Sea $N\left(t\right)$ un proceso de renovaci\'on en $\rea_{+}$ definido en el mismo espacio de probabilidad que $X\left(t\right)$, con tiempos de renovaci\'on $T$ y tiempos de inter-renovaci\'on $\xi_{n}=T_{n}-T_{n-1}$, con misma distribuci\'on $F$ de media finita $\mu$.



\begin{Def}
Para el proceso $\left\{\left(N\left(t\right),X\left(t\right)\right):t\geq0\right\}$, sus trayectoria muestrales en el intervalo de tiempo $\left[T_{n-1},T_{n}\right)$ est\'an descritas por
\begin{eqnarray*}
\zeta_{n}=\left(\xi_{n},\left\{X\left(T_{n-1}+t\right):0\leq t<\xi_{n}\right\}\right)
\end{eqnarray*}
Este $\zeta_{n}$ es el $n$-\'esimo segmento del proceso. El proceso es regenerativo sobre los tiempos $T_{n}$ si sus segmentos $\zeta_{n}$ son independientes e id\'enticamennte distribuidos.
\end{Def}


\begin{Obs}
Si $\tilde{X}\left(t\right)$ con espacio de estados $\tilde{S}$ es regenerativo sobre $T_{n}$, entonces $X\left(t\right)=f\left(\tilde{X}\left(t\right)\right)$ tambi\'en es regenerativo sobre $T_{n}$, para cualquier funci\'on $f:\tilde{S}\rightarrow S$.
\end{Obs}

\begin{Obs}
Los procesos regenerativos son crudamente regenerativos, pero no al rev\'es.
\end{Obs}

\begin{Def}[Definici\'on Cl\'asica]
Un proceso estoc\'astico $X=\left\{X\left(t\right):t\geq0\right\}$ es llamado regenerativo is existe una variable aleatoria $R_{1}>0$ tal que
\begin{itemize}
\item[i)] $\left\{X\left(t+R_{1}\right):t\geq0\right\}$ es independiente de $\left\{\left\{X\left(t\right):t<R_{1}\right\},\right\}$
\item[ii)] $\left\{X\left(t+R_{1}\right):t\geq0\right\}$ es estoc\'asticamente equivalente a $\left\{X\left(t\right):t>0\right\}$
\end{itemize}

Llamamos a $R_{1}$ tiempo de regeneraci\'on, y decimos que $X$ se regenera en este punto.
\end{Def}

$\left\{X\left(t+R_{1}\right)\right\}$ es regenerativo con tiempo de regeneraci\'on $R_{2}$, independiente de $R_{1}$ pero con la misma distribuci\'on que $R_{1}$. Procediendo de esta manera se obtiene una secuencia de variables aleatorias independientes e id\'enticamente distribuidas $\left\{R_{n}\right\}$ llamados longitudes de ciclo. Si definimos a $Z_{k}\equiv R_{1}+R_{2}+\cdots+R_{k}$, se tiene un proceso de renovaci\'on llamado proceso de renovaci\'on encajado para $X$.

\begin{Note}
Un proceso regenerativo con media de la longitud de ciclo finita es llamado positivo recurrente.
\end{Note}


\begin{Def}
Para $x$ fijo y para cada $t\geq0$, sea $I_{x}\left(t\right)=1$ si $X\left(t\right)\leq x$,  $I_{x}\left(t\right)=0$ en caso contrario, y def\'inanse los tiempos promedio
\begin{eqnarray*}
\overline{X}&=&lim_{t\rightarrow\infty}\frac{1}{t}\int_{0}^{\infty}X\left(u\right)du\\
\prob\left(X_{\infty}\leq x\right)&=&lim_{t\rightarrow\infty}\frac{1}{t}\int_{0}^{\infty}I_{x}\left(u\right)du,
\end{eqnarray*}
cuando estos l\'imites existan.
\end{Def}

Como consecuencia del teorema de Renovaci\'on-Recompensa, se tiene que el primer l\'imite  existe y es igual a la constante
\begin{eqnarray*}
\overline{X}&=&\frac{\esp\left[\int_{0}^{R_{1}}X\left(t\right)dt\right]}{\esp\left[R_{1}\right]},
\end{eqnarray*}
suponiendo que ambas esperanzas son finitas.

\begin{Note}
\begin{itemize}
\item[a)] Si el proceso regenerativo $X$ es positivo recurrente y tiene trayectorias muestrales no negativas, entonces la ecuaci\'on anterior es v\'alida.
\item[b)] Si $X$ es positivo recurrente regenerativo, podemos construir una \'unica versi\'on estacionaria de este proceso, $X_{e}=\left\{X_{e}\left(t\right)\right\}$, donde $X_{e}$ es un proceso estoc\'astico regenerativo y estrictamente estacionario, con distribuci\'on marginal distribuida como $X_{\infty}$
\end{itemize}
\end{Note}

%________________________________________________________________________
\subsection{Procesos Regenerativos}
%________________________________________________________________________

Para $\left\{X\left(t\right):t\geq0\right\}$ Proceso Estoc\'astico a tiempo continuo con estado de espacios $S$, que es un espacio m\'etrico, con trayectorias continuas por la derecha y con l\'imites por la izquierda c.s. Sea $N\left(t\right)$ un proceso de renovaci\'on en $\rea_{+}$ definido en el mismo espacio de probabilidad que $X\left(t\right)$, con tiempos de renovaci\'on $T$ y tiempos de inter-renovaci\'on $\xi_{n}=T_{n}-T_{n-1}$, con misma distribuci\'on $F$ de media finita $\mu$.



\begin{Def}
Para el proceso $\left\{\left(N\left(t\right),X\left(t\right)\right):t\geq0\right\}$, sus trayectoria muestrales en el intervalo de tiempo $\left[T_{n-1},T_{n}\right)$ est\'an descritas por
\begin{eqnarray*}
\zeta_{n}=\left(\xi_{n},\left\{X\left(T_{n-1}+t\right):0\leq t<\xi_{n}\right\}\right)
\end{eqnarray*}
Este $\zeta_{n}$ es el $n$-\'esimo segmento del proceso. El proceso es regenerativo sobre los tiempos $T_{n}$ si sus segmentos $\zeta_{n}$ son independientes e id\'enticamennte distribuidos.
\end{Def}


\begin{Obs}
Si $\tilde{X}\left(t\right)$ con espacio de estados $\tilde{S}$ es regenerativo sobre $T_{n}$, entonces $X\left(t\right)=f\left(\tilde{X}\left(t\right)\right)$ tambi\'en es regenerativo sobre $T_{n}$, para cualquier funci\'on $f:\tilde{S}\rightarrow S$.
\end{Obs}

\begin{Obs}
Los procesos regenerativos son crudamente regenerativos, pero no al rev\'es.
\end{Obs}

\begin{Def}[Definici\'on Cl\'asica]
Un proceso estoc\'astico $X=\left\{X\left(t\right):t\geq0\right\}$ es llamado regenerativo is existe una variable aleatoria $R_{1}>0$ tal que
\begin{itemize}
\item[i)] $\left\{X\left(t+R_{1}\right):t\geq0\right\}$ es independiente de $\left\{\left\{X\left(t\right):t<R_{1}\right\},\right\}$
\item[ii)] $\left\{X\left(t+R_{1}\right):t\geq0\right\}$ es estoc\'asticamente equivalente a $\left\{X\left(t\right):t>0\right\}$
\end{itemize}

Llamamos a $R_{1}$ tiempo de regeneraci\'on, y decimos que $X$ se regenera en este punto.
\end{Def}

$\left\{X\left(t+R_{1}\right)\right\}$ es regenerativo con tiempo de regeneraci\'on $R_{2}$, independiente de $R_{1}$ pero con la misma distribuci\'on que $R_{1}$. Procediendo de esta manera se obtiene una secuencia de variables aleatorias independientes e id\'enticamente distribuidas $\left\{R_{n}\right\}$ llamados longitudes de ciclo. Si definimos a $Z_{k}\equiv R_{1}+R_{2}+\cdots+R_{k}$, se tiene un proceso de renovaci\'on llamado proceso de renovaci\'on encajado para $X$.

\begin{Note}
Un proceso regenerativo con media de la longitud de ciclo finita es llamado positivo recurrente.
\end{Note}


\begin{Def}
Para $x$ fijo y para cada $t\geq0$, sea $I_{x}\left(t\right)=1$ si $X\left(t\right)\leq x$,  $I_{x}\left(t\right)=0$ en caso contrario, y def\'inanse los tiempos promedio
\begin{eqnarray*}
\overline{X}&=&lim_{t\rightarrow\infty}\frac{1}{t}\int_{0}^{\infty}X\left(u\right)du\\
\prob\left(X_{\infty}\leq x\right)&=&lim_{t\rightarrow\infty}\frac{1}{t}\int_{0}^{\infty}I_{x}\left(u\right)du,
\end{eqnarray*}
cuando estos l\'imites existan.
\end{Def}

Como consecuencia del teorema de Renovaci\'on-Recompensa, se tiene que el primer l\'imite  existe y es igual a la constante
\begin{eqnarray*}
\overline{X}&=&\frac{\esp\left[\int_{0}^{R_{1}}X\left(t\right)dt\right]}{\esp\left[R_{1}\right]},
\end{eqnarray*}
suponiendo que ambas esperanzas son finitas.

\begin{Note}
\begin{itemize}
\item[a)] Si el proceso regenerativo $X$ es positivo recurrente y tiene trayectorias muestrales no negativas, entonces la ecuaci\'on anterior es v\'alida.
\item[b)] Si $X$ es positivo recurrente regenerativo, podemos construir una \'unica versi\'on estacionaria de este proceso, $X_{e}=\left\{X_{e}\left(t\right)\right\}$, donde $X_{e}$ es un proceso estoc\'astico regenerativo y estrictamente estacionario, con distribuci\'on marginal distribuida como $X_{\infty}$
\end{itemize}
\end{Note}
%__________________________________________________________________________________________
\subsection{Procesos Regenerativos Estacionarios - Stidham \cite{Stidham}}
%__________________________________________________________________________________________


Un proceso estoc\'astico a tiempo continuo $\left\{V\left(t\right),t\geq0\right\}$ es un proceso regenerativo si existe una sucesi\'on de variables aleatorias independientes e id\'enticamente distribuidas $\left\{X_{1},X_{2},\ldots\right\}$, sucesi\'on de renovaci\'on, tal que para cualquier conjunto de Borel $A$, 

\begin{eqnarray*}
\prob\left\{V\left(t\right)\in A|X_{1}+X_{2}+\cdots+X_{R\left(t\right)}=s,\left\{V\left(\tau\right),\tau<s\right\}\right\}=\prob\left\{V\left(t-s\right)\in A|X_{1}>t-s\right\},
\end{eqnarray*}
para todo $0\leq s\leq t$, donde $R\left(t\right)=\max\left\{X_{1}+X_{2}+\cdots+X_{j}\leq t\right\}=$n\'umero de renovaciones ({\emph{puntos de regeneraci\'on}}) que ocurren en $\left[0,t\right]$. El intervalo $\left[0,X_{1}\right)$ es llamado {\emph{primer ciclo de regeneraci\'on}} de $\left\{V\left(t \right),t\geq0\right\}$, $\left[X_{1},X_{1}+X_{2}\right)$ el {\emph{segundo ciclo de regeneraci\'on}}, y as\'i sucesivamente.

Sea $X=X_{1}$ y sea $F$ la funci\'on de distrbuci\'on de $X$


\begin{Def}
Se define el proceso estacionario, $\left\{V^{*}\left(t\right),t\geq0\right\}$, para $\left\{V\left(t\right),t\geq0\right\}$ por

\begin{eqnarray*}
\prob\left\{V\left(t\right)\in A\right\}=\frac{1}{\esp\left[X\right]}\int_{0}^{\infty}\prob\left\{V\left(t+x\right)\in A|X>x\right\}\left(1-F\left(x\right)\right)dx,
\end{eqnarray*} 
para todo $t\geq0$ y todo conjunto de Borel $A$.
\end{Def}

\begin{Def}
Una distribuci\'on se dice que es {\emph{aritm\'etica}} si todos sus puntos de incremento son m\'ultiplos de la forma $0,\lambda, 2\lambda,\ldots$ para alguna $\lambda>0$ entera.
\end{Def}


\begin{Def}
Una modificaci\'on medible de un proceso $\left\{V\left(t\right),t\geq0\right\}$, es una versi\'on de este, $\left\{V\left(t,w\right)\right\}$ conjuntamente medible para $t\geq0$ y para $w\in S$, $S$ espacio de estados para $\left\{V\left(t\right),t\geq0\right\}$.
\end{Def}

\begin{Teo}
Sea $\left\{V\left(t\right),t\geq\right\}$ un proceso regenerativo no negativo con modificaci\'on medible. Sea $\esp\left[X\right]<\infty$. Entonces el proceso estacionario dado por la ecuaci\'on anterior est\'a bien definido y tiene funci\'on de distribuci\'on independiente de $t$, adem\'as
\begin{itemize}
\item[i)] \begin{eqnarray*}
\esp\left[V^{*}\left(0\right)\right]&=&\frac{\esp\left[\int_{0}^{X}V\left(s\right)ds\right]}{\esp\left[X\right]}\end{eqnarray*}
\item[ii)] Si $\esp\left[V^{*}\left(0\right)\right]<\infty$, equivalentemente, si $\esp\left[\int_{0}^{X}V\left(s\right)ds\right]<\infty$,entonces
\begin{eqnarray*}
\frac{\int_{0}^{t}V\left(s\right)ds}{t}\rightarrow\frac{\esp\left[\int_{0}^{X}V\left(s\right)ds\right]}{\esp\left[X\right]}
\end{eqnarray*}
con probabilidad 1 y en media, cuando $t\rightarrow\infty$.
\end{itemize}
\end{Teo}


%__________________________________________________________________________________________
\subsection{Procesos Regenerativos Estacionarios - Stidham \cite{Stidham}}
%__________________________________________________________________________________________


Un proceso estoc\'astico a tiempo continuo $\left\{V\left(t\right),t\geq0\right\}$ es un proceso regenerativo si existe una sucesi\'on de variables aleatorias independientes e id\'enticamente distribuidas $\left\{X_{1},X_{2},\ldots\right\}$, sucesi\'on de renovaci\'on, tal que para cualquier conjunto de Borel $A$, 

\begin{eqnarray*}
\prob\left\{V\left(t\right)\in A|X_{1}+X_{2}+\cdots+X_{R\left(t\right)}=s,\left\{V\left(\tau\right),\tau<s\right\}\right\}=\prob\left\{V\left(t-s\right)\in A|X_{1}>t-s\right\},
\end{eqnarray*}
para todo $0\leq s\leq t$, donde $R\left(t\right)=\max\left\{X_{1}+X_{2}+\cdots+X_{j}\leq t\right\}=$n\'umero de renovaciones ({\emph{puntos de regeneraci\'on}}) que ocurren en $\left[0,t\right]$. El intervalo $\left[0,X_{1}\right)$ es llamado {\emph{primer ciclo de regeneraci\'on}} de $\left\{V\left(t \right),t\geq0\right\}$, $\left[X_{1},X_{1}+X_{2}\right)$ el {\emph{segundo ciclo de regeneraci\'on}}, y as\'i sucesivamente.

Sea $X=X_{1}$ y sea $F$ la funci\'on de distrbuci\'on de $X$


\begin{Def}
Se define el proceso estacionario, $\left\{V^{*}\left(t\right),t\geq0\right\}$, para $\left\{V\left(t\right),t\geq0\right\}$ por

\begin{eqnarray*}
\prob\left\{V\left(t\right)\in A\right\}=\frac{1}{\esp\left[X\right]}\int_{0}^{\infty}\prob\left\{V\left(t+x\right)\in A|X>x\right\}\left(1-F\left(x\right)\right)dx,
\end{eqnarray*} 
para todo $t\geq0$ y todo conjunto de Borel $A$.
\end{Def}

\begin{Def}
Una distribuci\'on se dice que es {\emph{aritm\'etica}} si todos sus puntos de incremento son m\'ultiplos de la forma $0,\lambda, 2\lambda,\ldots$ para alguna $\lambda>0$ entera.
\end{Def}


\begin{Def}
Una modificaci\'on medible de un proceso $\left\{V\left(t\right),t\geq0\right\}$, es una versi\'on de este, $\left\{V\left(t,w\right)\right\}$ conjuntamente medible para $t\geq0$ y para $w\in S$, $S$ espacio de estados para $\left\{V\left(t\right),t\geq0\right\}$.
\end{Def}

\begin{Teo}
Sea $\left\{V\left(t\right),t\geq\right\}$ un proceso regenerativo no negativo con modificaci\'on medible. Sea $\esp\left[X\right]<\infty$. Entonces el proceso estacionario dado por la ecuaci\'on anterior est\'a bien definido y tiene funci\'on de distribuci\'on independiente de $t$, adem\'as
\begin{itemize}
\item[i)] \begin{eqnarray*}
\esp\left[V^{*}\left(0\right)\right]&=&\frac{\esp\left[\int_{0}^{X}V\left(s\right)ds\right]}{\esp\left[X\right]}\end{eqnarray*}
\item[ii)] Si $\esp\left[V^{*}\left(0\right)\right]<\infty$, equivalentemente, si $\esp\left[\int_{0}^{X}V\left(s\right)ds\right]<\infty$,entonces
\begin{eqnarray*}
\frac{\int_{0}^{t}V\left(s\right)ds}{t}\rightarrow\frac{\esp\left[\int_{0}^{X}V\left(s\right)ds\right]}{\esp\left[X\right]}
\end{eqnarray*}
con probabilidad 1 y en media, cuando $t\rightarrow\infty$.
\end{itemize}
\end{Teo}
%___________________________________________________________________________________________
%
\subsection{Propiedades de los Procesos de Renovaci\'on}
%___________________________________________________________________________________________
%

Los tiempos $T_{n}$ est\'an relacionados con los conteos de $N\left(t\right)$ por

\begin{eqnarray*}
\left\{N\left(t\right)\geq n\right\}&=&\left\{T_{n}\leq t\right\}\\
T_{N\left(t\right)}\leq &t&<T_{N\left(t\right)+1},
\end{eqnarray*}

adem\'as $N\left(T_{n}\right)=n$, y 

\begin{eqnarray*}
N\left(t\right)=\max\left\{n:T_{n}\leq t\right\}=\min\left\{n:T_{n+1}>t\right\}
\end{eqnarray*}

Por propiedades de la convoluci\'on se sabe que

\begin{eqnarray*}
P\left\{T_{n}\leq t\right\}=F^{n\star}\left(t\right)
\end{eqnarray*}
que es la $n$-\'esima convoluci\'on de $F$. Entonces 

\begin{eqnarray*}
\left\{N\left(t\right)\geq n\right\}&=&\left\{T_{n}\leq t\right\}\\
P\left\{N\left(t\right)\leq n\right\}&=&1-F^{\left(n+1\right)\star}\left(t\right)
\end{eqnarray*}

Adem\'as usando el hecho de que $\esp\left[N\left(t\right)\right]=\sum_{n=1}^{\infty}P\left\{N\left(t\right)\geq n\right\}$
se tiene que

\begin{eqnarray*}
\esp\left[N\left(t\right)\right]=\sum_{n=1}^{\infty}F^{n\star}\left(t\right)
\end{eqnarray*}

\begin{Prop}
Para cada $t\geq0$, la funci\'on generadora de momentos $\esp\left[e^{\alpha N\left(t\right)}\right]$ existe para alguna $\alpha$ en una vecindad del 0, y de aqu\'i que $\esp\left[N\left(t\right)^{m}\right]<\infty$, para $m\geq1$.
\end{Prop}


\begin{Note}
Si el primer tiempo de renovaci\'on $\xi_{1}$ no tiene la misma distribuci\'on que el resto de las $\xi_{n}$, para $n\geq2$, a $N\left(t\right)$ se le llama Proceso de Renovaci\'on retardado, donde si $\xi$ tiene distribuci\'on $G$, entonces el tiempo $T_{n}$ de la $n$-\'esima renovaci\'on tiene distribuci\'on $G\star F^{\left(n-1\right)\star}\left(t\right)$
\end{Note}


\begin{Teo}
Para una constante $\mu\leq\infty$ ( o variable aleatoria), las siguientes expresiones son equivalentes:

\begin{eqnarray}
lim_{n\rightarrow\infty}n^{-1}T_{n}&=&\mu,\textrm{ c.s.}\\
lim_{t\rightarrow\infty}t^{-1}N\left(t\right)&=&1/\mu,\textrm{ c.s.}
\end{eqnarray}
\end{Teo}


Es decir, $T_{n}$ satisface la Ley Fuerte de los Grandes N\'umeros s\'i y s\'olo s\'i $N\left/t\right)$ la cumple.


\begin{Coro}[Ley Fuerte de los Grandes N\'umeros para Procesos de Renovaci\'on]
Si $N\left(t\right)$ es un proceso de renovaci\'on cuyos tiempos de inter-renovaci\'on tienen media $\mu\leq\infty$, entonces
\begin{eqnarray}
t^{-1}N\left(t\right)\rightarrow 1/\mu,\textrm{ c.s. cuando }t\rightarrow\infty.
\end{eqnarray}

\end{Coro}


Considerar el proceso estoc\'astico de valores reales $\left\{Z\left(t\right):t\geq0\right\}$ en el mismo espacio de probabilidad que $N\left(t\right)$

\begin{Def}
Para el proceso $\left\{Z\left(t\right):t\geq0\right\}$ se define la fluctuaci\'on m\'axima de $Z\left(t\right)$ en el intervalo $\left(T_{n-1},T_{n}\right]$:
\begin{eqnarray*}
M_{n}=\sup_{T_{n-1}<t\leq T_{n}}|Z\left(t\right)-Z\left(T_{n-1}\right)|
\end{eqnarray*}
\end{Def}

\begin{Teo}
Sup\'ongase que $n^{-1}T_{n}\rightarrow\mu$ c.s. cuando $n\rightarrow\infty$, donde $\mu\leq\infty$ es una constante o variable aleatoria. Sea $a$ una constante o variable aleatoria que puede ser infinita cuando $\mu$ es finita, y considere las expresiones l\'imite:
\begin{eqnarray}
lim_{n\rightarrow\infty}n^{-1}Z\left(T_{n}\right)&=&a,\textrm{ c.s.}\\
lim_{t\rightarrow\infty}t^{-1}Z\left(t\right)&=&a/\mu,\textrm{ c.s.}
\end{eqnarray}
La segunda expresi\'on implica la primera. Conversamente, la primera implica la segunda si el proceso $Z\left(t\right)$ es creciente, o si $lim_{n\rightarrow\infty}n^{-1}M_{n}=0$ c.s.
\end{Teo}

\begin{Coro}
Si $N\left(t\right)$ es un proceso de renovaci\'on, y $\left(Z\left(T_{n}\right)-Z\left(T_{n-1}\right),M_{n}\right)$, para $n\geq1$, son variables aleatorias independientes e id\'enticamente distribuidas con media finita, entonces,
\begin{eqnarray}
lim_{t\rightarrow\infty}t^{-1}Z\left(t\right)\rightarrow\frac{\esp\left[Z\left(T_{1}\right)-Z\left(T_{0}\right)\right]}{\esp\left[T_{1}\right]},\textrm{ c.s. cuando  }t\rightarrow\infty.
\end{eqnarray}
\end{Coro}


%___________________________________________________________________________________________
%
\subsection{Propiedades de los Procesos de Renovaci\'on}
%___________________________________________________________________________________________
%

Los tiempos $T_{n}$ est\'an relacionados con los conteos de $N\left(t\right)$ por

\begin{eqnarray*}
\left\{N\left(t\right)\geq n\right\}&=&\left\{T_{n}\leq t\right\}\\
T_{N\left(t\right)}\leq &t&<T_{N\left(t\right)+1},
\end{eqnarray*}

adem\'as $N\left(T_{n}\right)=n$, y 

\begin{eqnarray*}
N\left(t\right)=\max\left\{n:T_{n}\leq t\right\}=\min\left\{n:T_{n+1}>t\right\}
\end{eqnarray*}

Por propiedades de la convoluci\'on se sabe que

\begin{eqnarray*}
P\left\{T_{n}\leq t\right\}=F^{n\star}\left(t\right)
\end{eqnarray*}
que es la $n$-\'esima convoluci\'on de $F$. Entonces 

\begin{eqnarray*}
\left\{N\left(t\right)\geq n\right\}&=&\left\{T_{n}\leq t\right\}\\
P\left\{N\left(t\right)\leq n\right\}&=&1-F^{\left(n+1\right)\star}\left(t\right)
\end{eqnarray*}

Adem\'as usando el hecho de que $\esp\left[N\left(t\right)\right]=\sum_{n=1}^{\infty}P\left\{N\left(t\right)\geq n\right\}$
se tiene que

\begin{eqnarray*}
\esp\left[N\left(t\right)\right]=\sum_{n=1}^{\infty}F^{n\star}\left(t\right)
\end{eqnarray*}

\begin{Prop}
Para cada $t\geq0$, la funci\'on generadora de momentos $\esp\left[e^{\alpha N\left(t\right)}\right]$ existe para alguna $\alpha$ en una vecindad del 0, y de aqu\'i que $\esp\left[N\left(t\right)^{m}\right]<\infty$, para $m\geq1$.
\end{Prop}


\begin{Note}
Si el primer tiempo de renovaci\'on $\xi_{1}$ no tiene la misma distribuci\'on que el resto de las $\xi_{n}$, para $n\geq2$, a $N\left(t\right)$ se le llama Proceso de Renovaci\'on retardado, donde si $\xi$ tiene distribuci\'on $G$, entonces el tiempo $T_{n}$ de la $n$-\'esima renovaci\'on tiene distribuci\'on $G\star F^{\left(n-1\right)\star}\left(t\right)$
\end{Note}


\begin{Teo}
Para una constante $\mu\leq\infty$ ( o variable aleatoria), las siguientes expresiones son equivalentes:

\begin{eqnarray}
lim_{n\rightarrow\infty}n^{-1}T_{n}&=&\mu,\textrm{ c.s.}\\
lim_{t\rightarrow\infty}t^{-1}N\left(t\right)&=&1/\mu,\textrm{ c.s.}
\end{eqnarray}
\end{Teo}


Es decir, $T_{n}$ satisface la Ley Fuerte de los Grandes N\'umeros s\'i y s\'olo s\'i $N\left/t\right)$ la cumple.


\begin{Coro}[Ley Fuerte de los Grandes N\'umeros para Procesos de Renovaci\'on]
Si $N\left(t\right)$ es un proceso de renovaci\'on cuyos tiempos de inter-renovaci\'on tienen media $\mu\leq\infty$, entonces
\begin{eqnarray}
t^{-1}N\left(t\right)\rightarrow 1/\mu,\textrm{ c.s. cuando }t\rightarrow\infty.
\end{eqnarray}

\end{Coro}


Considerar el proceso estoc\'astico de valores reales $\left\{Z\left(t\right):t\geq0\right\}$ en el mismo espacio de probabilidad que $N\left(t\right)$

\begin{Def}
Para el proceso $\left\{Z\left(t\right):t\geq0\right\}$ se define la fluctuaci\'on m\'axima de $Z\left(t\right)$ en el intervalo $\left(T_{n-1},T_{n}\right]$:
\begin{eqnarray*}
M_{n}=\sup_{T_{n-1}<t\leq T_{n}}|Z\left(t\right)-Z\left(T_{n-1}\right)|
\end{eqnarray*}
\end{Def}

\begin{Teo}
Sup\'ongase que $n^{-1}T_{n}\rightarrow\mu$ c.s. cuando $n\rightarrow\infty$, donde $\mu\leq\infty$ es una constante o variable aleatoria. Sea $a$ una constante o variable aleatoria que puede ser infinita cuando $\mu$ es finita, y considere las expresiones l\'imite:
\begin{eqnarray}
lim_{n\rightarrow\infty}n^{-1}Z\left(T_{n}\right)&=&a,\textrm{ c.s.}\\
lim_{t\rightarrow\infty}t^{-1}Z\left(t\right)&=&a/\mu,\textrm{ c.s.}
\end{eqnarray}
La segunda expresi\'on implica la primera. Conversamente, la primera implica la segunda si el proceso $Z\left(t\right)$ es creciente, o si $lim_{n\rightarrow\infty}n^{-1}M_{n}=0$ c.s.
\end{Teo}

\begin{Coro}
Si $N\left(t\right)$ es un proceso de renovaci\'on, y $\left(Z\left(T_{n}\right)-Z\left(T_{n-1}\right),M_{n}\right)$, para $n\geq1$, son variables aleatorias independientes e id\'enticamente distribuidas con media finita, entonces,
\begin{eqnarray}
lim_{t\rightarrow\infty}t^{-1}Z\left(t\right)\rightarrow\frac{\esp\left[Z\left(T_{1}\right)-Z\left(T_{0}\right)\right]}{\esp\left[T_{1}\right]},\textrm{ c.s. cuando  }t\rightarrow\infty.
\end{eqnarray}
\end{Coro}

%___________________________________________________________________________________________
%
\subsection{Propiedades de los Procesos de Renovaci\'on}
%___________________________________________________________________________________________
%

Los tiempos $T_{n}$ est\'an relacionados con los conteos de $N\left(t\right)$ por

\begin{eqnarray*}
\left\{N\left(t\right)\geq n\right\}&=&\left\{T_{n}\leq t\right\}\\
T_{N\left(t\right)}\leq &t&<T_{N\left(t\right)+1},
\end{eqnarray*}

adem\'as $N\left(T_{n}\right)=n$, y 

\begin{eqnarray*}
N\left(t\right)=\max\left\{n:T_{n}\leq t\right\}=\min\left\{n:T_{n+1}>t\right\}
\end{eqnarray*}

Por propiedades de la convoluci\'on se sabe que

\begin{eqnarray*}
P\left\{T_{n}\leq t\right\}=F^{n\star}\left(t\right)
\end{eqnarray*}
que es la $n$-\'esima convoluci\'on de $F$. Entonces 

\begin{eqnarray*}
\left\{N\left(t\right)\geq n\right\}&=&\left\{T_{n}\leq t\right\}\\
P\left\{N\left(t\right)\leq n\right\}&=&1-F^{\left(n+1\right)\star}\left(t\right)
\end{eqnarray*}

Adem\'as usando el hecho de que $\esp\left[N\left(t\right)\right]=\sum_{n=1}^{\infty}P\left\{N\left(t\right)\geq n\right\}$
se tiene que

\begin{eqnarray*}
\esp\left[N\left(t\right)\right]=\sum_{n=1}^{\infty}F^{n\star}\left(t\right)
\end{eqnarray*}

\begin{Prop}
Para cada $t\geq0$, la funci\'on generadora de momentos $\esp\left[e^{\alpha N\left(t\right)}\right]$ existe para alguna $\alpha$ en una vecindad del 0, y de aqu\'i que $\esp\left[N\left(t\right)^{m}\right]<\infty$, para $m\geq1$.
\end{Prop}


\begin{Note}
Si el primer tiempo de renovaci\'on $\xi_{1}$ no tiene la misma distribuci\'on que el resto de las $\xi_{n}$, para $n\geq2$, a $N\left(t\right)$ se le llama Proceso de Renovaci\'on retardado, donde si $\xi$ tiene distribuci\'on $G$, entonces el tiempo $T_{n}$ de la $n$-\'esima renovaci\'on tiene distribuci\'on $G\star F^{\left(n-1\right)\star}\left(t\right)$
\end{Note}


\begin{Teo}
Para una constante $\mu\leq\infty$ ( o variable aleatoria), las siguientes expresiones son equivalentes:

\begin{eqnarray}
lim_{n\rightarrow\infty}n^{-1}T_{n}&=&\mu,\textrm{ c.s.}\\
lim_{t\rightarrow\infty}t^{-1}N\left(t\right)&=&1/\mu,\textrm{ c.s.}
\end{eqnarray}
\end{Teo}


Es decir, $T_{n}$ satisface la Ley Fuerte de los Grandes N\'umeros s\'i y s\'olo s\'i $N\left/t\right)$ la cumple.


\begin{Coro}[Ley Fuerte de los Grandes N\'umeros para Procesos de Renovaci\'on]
Si $N\left(t\right)$ es un proceso de renovaci\'on cuyos tiempos de inter-renovaci\'on tienen media $\mu\leq\infty$, entonces
\begin{eqnarray}
t^{-1}N\left(t\right)\rightarrow 1/\mu,\textrm{ c.s. cuando }t\rightarrow\infty.
\end{eqnarray}

\end{Coro}


Considerar el proceso estoc\'astico de valores reales $\left\{Z\left(t\right):t\geq0\right\}$ en el mismo espacio de probabilidad que $N\left(t\right)$

\begin{Def}
Para el proceso $\left\{Z\left(t\right):t\geq0\right\}$ se define la fluctuaci\'on m\'axima de $Z\left(t\right)$ en el intervalo $\left(T_{n-1},T_{n}\right]$:
\begin{eqnarray*}
M_{n}=\sup_{T_{n-1}<t\leq T_{n}}|Z\left(t\right)-Z\left(T_{n-1}\right)|
\end{eqnarray*}
\end{Def}

\begin{Teo}
Sup\'ongase que $n^{-1}T_{n}\rightarrow\mu$ c.s. cuando $n\rightarrow\infty$, donde $\mu\leq\infty$ es una constante o variable aleatoria. Sea $a$ una constante o variable aleatoria que puede ser infinita cuando $\mu$ es finita, y considere las expresiones l\'imite:
\begin{eqnarray}
lim_{n\rightarrow\infty}n^{-1}Z\left(T_{n}\right)&=&a,\textrm{ c.s.}\\
lim_{t\rightarrow\infty}t^{-1}Z\left(t\right)&=&a/\mu,\textrm{ c.s.}
\end{eqnarray}
La segunda expresi\'on implica la primera. Conversamente, la primera implica la segunda si el proceso $Z\left(t\right)$ es creciente, o si $lim_{n\rightarrow\infty}n^{-1}M_{n}=0$ c.s.
\end{Teo}

\begin{Coro}
Si $N\left(t\right)$ es un proceso de renovaci\'on, y $\left(Z\left(T_{n}\right)-Z\left(T_{n-1}\right),M_{n}\right)$, para $n\geq1$, son variables aleatorias independientes e id\'enticamente distribuidas con media finita, entonces,
\begin{eqnarray}
lim_{t\rightarrow\infty}t^{-1}Z\left(t\right)\rightarrow\frac{\esp\left[Z\left(T_{1}\right)-Z\left(T_{0}\right)\right]}{\esp\left[T_{1}\right]},\textrm{ c.s. cuando  }t\rightarrow\infty.
\end{eqnarray}
\end{Coro}

%___________________________________________________________________________________________
%
\subsection{Propiedades de los Procesos de Renovaci\'on}
%___________________________________________________________________________________________
%

Los tiempos $T_{n}$ est\'an relacionados con los conteos de $N\left(t\right)$ por

\begin{eqnarray*}
\left\{N\left(t\right)\geq n\right\}&=&\left\{T_{n}\leq t\right\}\\
T_{N\left(t\right)}\leq &t&<T_{N\left(t\right)+1},
\end{eqnarray*}

adem\'as $N\left(T_{n}\right)=n$, y 

\begin{eqnarray*}
N\left(t\right)=\max\left\{n:T_{n}\leq t\right\}=\min\left\{n:T_{n+1}>t\right\}
\end{eqnarray*}

Por propiedades de la convoluci\'on se sabe que

\begin{eqnarray*}
P\left\{T_{n}\leq t\right\}=F^{n\star}\left(t\right)
\end{eqnarray*}
que es la $n$-\'esima convoluci\'on de $F$. Entonces 

\begin{eqnarray*}
\left\{N\left(t\right)\geq n\right\}&=&\left\{T_{n}\leq t\right\}\\
P\left\{N\left(t\right)\leq n\right\}&=&1-F^{\left(n+1\right)\star}\left(t\right)
\end{eqnarray*}

Adem\'as usando el hecho de que $\esp\left[N\left(t\right)\right]=\sum_{n=1}^{\infty}P\left\{N\left(t\right)\geq n\right\}$
se tiene que

\begin{eqnarray*}
\esp\left[N\left(t\right)\right]=\sum_{n=1}^{\infty}F^{n\star}\left(t\right)
\end{eqnarray*}

\begin{Prop}
Para cada $t\geq0$, la funci\'on generadora de momentos $\esp\left[e^{\alpha N\left(t\right)}\right]$ existe para alguna $\alpha$ en una vecindad del 0, y de aqu\'i que $\esp\left[N\left(t\right)^{m}\right]<\infty$, para $m\geq1$.
\end{Prop}


\begin{Note}
Si el primer tiempo de renovaci\'on $\xi_{1}$ no tiene la misma distribuci\'on que el resto de las $\xi_{n}$, para $n\geq2$, a $N\left(t\right)$ se le llama Proceso de Renovaci\'on retardado, donde si $\xi$ tiene distribuci\'on $G$, entonces el tiempo $T_{n}$ de la $n$-\'esima renovaci\'on tiene distribuci\'on $G\star F^{\left(n-1\right)\star}\left(t\right)$
\end{Note}


\begin{Teo}
Para una constante $\mu\leq\infty$ ( o variable aleatoria), las siguientes expresiones son equivalentes:

\begin{eqnarray}
lim_{n\rightarrow\infty}n^{-1}T_{n}&=&\mu,\textrm{ c.s.}\\
lim_{t\rightarrow\infty}t^{-1}N\left(t\right)&=&1/\mu,\textrm{ c.s.}
\end{eqnarray}
\end{Teo}


Es decir, $T_{n}$ satisface la Ley Fuerte de los Grandes N\'umeros s\'i y s\'olo s\'i $N\left/t\right)$ la cumple.


\begin{Coro}[Ley Fuerte de los Grandes N\'umeros para Procesos de Renovaci\'on]
Si $N\left(t\right)$ es un proceso de renovaci\'on cuyos tiempos de inter-renovaci\'on tienen media $\mu\leq\infty$, entonces
\begin{eqnarray}
t^{-1}N\left(t\right)\rightarrow 1/\mu,\textrm{ c.s. cuando }t\rightarrow\infty.
\end{eqnarray}

\end{Coro}


Considerar el proceso estoc\'astico de valores reales $\left\{Z\left(t\right):t\geq0\right\}$ en el mismo espacio de probabilidad que $N\left(t\right)$

\begin{Def}
Para el proceso $\left\{Z\left(t\right):t\geq0\right\}$ se define la fluctuaci\'on m\'axima de $Z\left(t\right)$ en el intervalo $\left(T_{n-1},T_{n}\right]$:
\begin{eqnarray*}
M_{n}=\sup_{T_{n-1}<t\leq T_{n}}|Z\left(t\right)-Z\left(T_{n-1}\right)|
\end{eqnarray*}
\end{Def}

\begin{Teo}
Sup\'ongase que $n^{-1}T_{n}\rightarrow\mu$ c.s. cuando $n\rightarrow\infty$, donde $\mu\leq\infty$ es una constante o variable aleatoria. Sea $a$ una constante o variable aleatoria que puede ser infinita cuando $\mu$ es finita, y considere las expresiones l\'imite:
\begin{eqnarray}
lim_{n\rightarrow\infty}n^{-1}Z\left(T_{n}\right)&=&a,\textrm{ c.s.}\\
lim_{t\rightarrow\infty}t^{-1}Z\left(t\right)&=&a/\mu,\textrm{ c.s.}
\end{eqnarray}
La segunda expresi\'on implica la primera. Conversamente, la primera implica la segunda si el proceso $Z\left(t\right)$ es creciente, o si $lim_{n\rightarrow\infty}n^{-1}M_{n}=0$ c.s.
\end{Teo}

\begin{Coro}
Si $N\left(t\right)$ es un proceso de renovaci\'on, y $\left(Z\left(T_{n}\right)-Z\left(T_{n-1}\right),M_{n}\right)$, para $n\geq1$, son variables aleatorias independientes e id\'enticamente distribuidas con media finita, entonces,
\begin{eqnarray}
lim_{t\rightarrow\infty}t^{-1}Z\left(t\right)\rightarrow\frac{\esp\left[Z\left(T_{1}\right)-Z\left(T_{0}\right)\right]}{\esp\left[T_{1}\right]},\textrm{ c.s. cuando  }t\rightarrow\infty.
\end{eqnarray}
\end{Coro}


%__________________________________________________________________________________________
\subsection{Procesos Regenerativos Estacionarios - Stidham \cite{Stidham}}
%__________________________________________________________________________________________


Un proceso estoc\'astico a tiempo continuo $\left\{V\left(t\right),t\geq0\right\}$ es un proceso regenerativo si existe una sucesi\'on de variables aleatorias independientes e id\'enticamente distribuidas $\left\{X_{1},X_{2},\ldots\right\}$, sucesi\'on de renovaci\'on, tal que para cualquier conjunto de Borel $A$, 

\begin{eqnarray*}
\prob\left\{V\left(t\right)\in A|X_{1}+X_{2}+\cdots+X_{R\left(t\right)}=s,\left\{V\left(\tau\right),\tau<s\right\}\right\}=\prob\left\{V\left(t-s\right)\in A|X_{1}>t-s\right\},
\end{eqnarray*}
para todo $0\leq s\leq t$, donde $R\left(t\right)=\max\left\{X_{1}+X_{2}+\cdots+X_{j}\leq t\right\}=$n\'umero de renovaciones ({\emph{puntos de regeneraci\'on}}) que ocurren en $\left[0,t\right]$. El intervalo $\left[0,X_{1}\right)$ es llamado {\emph{primer ciclo de regeneraci\'on}} de $\left\{V\left(t \right),t\geq0\right\}$, $\left[X_{1},X_{1}+X_{2}\right)$ el {\emph{segundo ciclo de regeneraci\'on}}, y as\'i sucesivamente.

Sea $X=X_{1}$ y sea $F$ la funci\'on de distrbuci\'on de $X$


\begin{Def}
Se define el proceso estacionario, $\left\{V^{*}\left(t\right),t\geq0\right\}$, para $\left\{V\left(t\right),t\geq0\right\}$ por

\begin{eqnarray*}
\prob\left\{V\left(t\right)\in A\right\}=\frac{1}{\esp\left[X\right]}\int_{0}^{\infty}\prob\left\{V\left(t+x\right)\in A|X>x\right\}\left(1-F\left(x\right)\right)dx,
\end{eqnarray*} 
para todo $t\geq0$ y todo conjunto de Borel $A$.
\end{Def}

\begin{Def}
Una distribuci\'on se dice que es {\emph{aritm\'etica}} si todos sus puntos de incremento son m\'ultiplos de la forma $0,\lambda, 2\lambda,\ldots$ para alguna $\lambda>0$ entera.
\end{Def}


\begin{Def}
Una modificaci\'on medible de un proceso $\left\{V\left(t\right),t\geq0\right\}$, es una versi\'on de este, $\left\{V\left(t,w\right)\right\}$ conjuntamente medible para $t\geq0$ y para $w\in S$, $S$ espacio de estados para $\left\{V\left(t\right),t\geq0\right\}$.
\end{Def}

\begin{Teo}
Sea $\left\{V\left(t\right),t\geq\right\}$ un proceso regenerativo no negativo con modificaci\'on medible. Sea $\esp\left[X\right]<\infty$. Entonces el proceso estacionario dado por la ecuaci\'on anterior est\'a bien definido y tiene funci\'on de distribuci\'on independiente de $t$, adem\'as
\begin{itemize}
\item[i)] \begin{eqnarray*}
\esp\left[V^{*}\left(0\right)\right]&=&\frac{\esp\left[\int_{0}^{X}V\left(s\right)ds\right]}{\esp\left[X\right]}\end{eqnarray*}
\item[ii)] Si $\esp\left[V^{*}\left(0\right)\right]<\infty$, equivalentemente, si $\esp\left[\int_{0}^{X}V\left(s\right)ds\right]<\infty$,entonces
\begin{eqnarray*}
\frac{\int_{0}^{t}V\left(s\right)ds}{t}\rightarrow\frac{\esp\left[\int_{0}^{X}V\left(s\right)ds\right]}{\esp\left[X\right]}
\end{eqnarray*}
con probabilidad 1 y en media, cuando $t\rightarrow\infty$.
\end{itemize}
\end{Teo}

%______________________________________________________________________
\subsection{Procesos de Renovaci\'on}
%______________________________________________________________________

\begin{Def}\label{Def.Tn}
Sean $0\leq T_{1}\leq T_{2}\leq \ldots$ son tiempos aleatorios infinitos en los cuales ocurren ciertos eventos. El n\'umero de tiempos $T_{n}$ en el intervalo $\left[0,t\right)$ es

\begin{eqnarray}
N\left(t\right)=\sum_{n=1}^{\infty}\indora\left(T_{n}\leq t\right),
\end{eqnarray}
para $t\geq0$.
\end{Def}

Si se consideran los puntos $T_{n}$ como elementos de $\rea_{+}$, y $N\left(t\right)$ es el n\'umero de puntos en $\rea$. El proceso denotado por $\left\{N\left(t\right):t\geq0\right\}$, denotado por $N\left(t\right)$, es un proceso puntual en $\rea_{+}$. Los $T_{n}$ son los tiempos de ocurrencia, el proceso puntual $N\left(t\right)$ es simple si su n\'umero de ocurrencias son distintas: $0<T_{1}<T_{2}<\ldots$ casi seguramente.

\begin{Def}
Un proceso puntual $N\left(t\right)$ es un proceso de renovaci\'on si los tiempos de interocurrencia $\xi_{n}=T_{n}-T_{n-1}$, para $n\geq1$, son independientes e identicamente distribuidos con distribuci\'on $F$, donde $F\left(0\right)=0$ y $T_{0}=0$. Los $T_{n}$ son llamados tiempos de renovaci\'on, referente a la independencia o renovaci\'on de la informaci\'on estoc\'astica en estos tiempos. Los $\xi_{n}$ son los tiempos de inter-renovaci\'on, y $N\left(t\right)$ es el n\'umero de renovaciones en el intervalo $\left[0,t\right)$
\end{Def}


\begin{Note}
Para definir un proceso de renovaci\'on para cualquier contexto, solamente hay que especificar una distribuci\'on $F$, con $F\left(0\right)=0$, para los tiempos de inter-renovaci\'on. La funci\'on $F$ en turno degune las otra variables aleatorias. De manera formal, existe un espacio de probabilidad y una sucesi\'on de variables aleatorias $\xi_{1},\xi_{2},\ldots$ definidas en este con distribuci\'on $F$. Entonces las otras cantidades son $T_{n}=\sum_{k=1}^{n}\xi_{k}$ y $N\left(t\right)=\sum_{n=1}^{\infty}\indora\left(T_{n}\leq t\right)$, donde $T_{n}\rightarrow\infty$ casi seguramente por la Ley Fuerte de los Grandes Números.
\end{Note}

%___________________________________________________________________________________________
%
\subsection{Teorema Principal de Renovaci\'on}
%___________________________________________________________________________________________
%

\begin{Note} Una funci\'on $h:\rea_{+}\rightarrow\rea$ es Directamente Riemann Integrable en los siguientes casos:
\begin{itemize}
\item[a)] $h\left(t\right)\geq0$ es decreciente y Riemann Integrable.
\item[b)] $h$ es continua excepto posiblemente en un conjunto de Lebesgue de medida 0, y $|h\left(t\right)|\leq b\left(t\right)$, donde $b$ es DRI.
\end{itemize}
\end{Note}

\begin{Teo}[Teorema Principal de Renovaci\'on]
Si $F$ es no aritm\'etica y $h\left(t\right)$ es Directamente Riemann Integrable (DRI), entonces

\begin{eqnarray*}
lim_{t\rightarrow\infty}U\star h=\frac{1}{\mu}\int_{\rea_{+}}h\left(s\right)ds.
\end{eqnarray*}
\end{Teo}

\begin{Prop}
Cualquier funci\'on $H\left(t\right)$ acotada en intervalos finitos y que es 0 para $t<0$ puede expresarse como
\begin{eqnarray*}
H\left(t\right)=U\star h\left(t\right)\textrm{,  donde }h\left(t\right)=H\left(t\right)-F\star H\left(t\right)
\end{eqnarray*}
\end{Prop}

\begin{Def}
Un proceso estoc\'astico $X\left(t\right)$ es crudamente regenerativo en un tiempo aleatorio positivo $T$ si
\begin{eqnarray*}
\esp\left[X\left(T+t\right)|T\right]=\esp\left[X\left(t\right)\right]\textrm{, para }t\geq0,\end{eqnarray*}
y con las esperanzas anteriores finitas.
\end{Def}

\begin{Prop}
Sup\'ongase que $X\left(t\right)$ es un proceso crudamente regenerativo en $T$, que tiene distribuci\'on $F$. Si $\esp\left[X\left(t\right)\right]$ es acotado en intervalos finitos, entonces
\begin{eqnarray*}
\esp\left[X\left(t\right)\right]=U\star h\left(t\right)\textrm{,  donde }h\left(t\right)=\esp\left[X\left(t\right)\indora\left(T>t\right)\right].
\end{eqnarray*}
\end{Prop}

\begin{Teo}[Regeneraci\'on Cruda]
Sup\'ongase que $X\left(t\right)$ es un proceso con valores positivo crudamente regenerativo en $T$, y def\'inase $M=\sup\left\{|X\left(t\right)|:t\leq T\right\}$. Si $T$ es no aritm\'etico y $M$ y $MT$ tienen media finita, entonces
\begin{eqnarray*}
lim_{t\rightarrow\infty}\esp\left[X\left(t\right)\right]=\frac{1}{\mu}\int_{\rea_{+}}h\left(s\right)ds,
\end{eqnarray*}
donde $h\left(t\right)=\esp\left[X\left(t\right)\indora\left(T>t\right)\right]$.
\end{Teo}



%___________________________________________________________________________________________
%
\subsection{Funci\'on de Renovaci\'on}
%___________________________________________________________________________________________
%


\begin{Def}
Sea $h\left(t\right)$ funci\'on de valores reales en $\rea$ acotada en intervalos finitos e igual a cero para $t<0$ La ecuaci\'on de renovaci\'on para $h\left(t\right)$ y la distribuci\'on $F$ es

\begin{eqnarray}\label{Ec.Renovacion}
H\left(t\right)=h\left(t\right)+\int_{\left[0,t\right]}H\left(t-s\right)dF\left(s\right)\textrm{,    }t\geq0,
\end{eqnarray}
donde $H\left(t\right)$ es una funci\'on de valores reales. Esto es $H=h+F\star H$. Decimos que $H\left(t\right)$ es soluci\'on de esta ecuaci\'on si satisface la ecuaci\'on, y es acotada en intervalos finitos e iguales a cero para $t<0$.
\end{Def}

\begin{Prop}
La funci\'on $U\star h\left(t\right)$ es la \'unica soluci\'on de la ecuaci\'on de renovaci\'on (\ref{Ec.Renovacion}).
\end{Prop}

\begin{Teo}[Teorema Renovaci\'on Elemental]
\begin{eqnarray*}
t^{-1}U\left(t\right)\rightarrow 1/\mu\textrm{,    cuando }t\rightarrow\infty.
\end{eqnarray*}
\end{Teo}

%___________________________________________________________________________________________
%
\subsection{Propiedades de los Procesos de Renovaci\'on}
%___________________________________________________________________________________________
%

Los tiempos $T_{n}$ est\'an relacionados con los conteos de $N\left(t\right)$ por

\begin{eqnarray*}
\left\{N\left(t\right)\geq n\right\}&=&\left\{T_{n}\leq t\right\}\\
T_{N\left(t\right)}\leq &t&<T_{N\left(t\right)+1},
\end{eqnarray*}

adem\'as $N\left(T_{n}\right)=n$, y 

\begin{eqnarray*}
N\left(t\right)=\max\left\{n:T_{n}\leq t\right\}=\min\left\{n:T_{n+1}>t\right\}
\end{eqnarray*}

Por propiedades de la convoluci\'on se sabe que

\begin{eqnarray*}
P\left\{T_{n}\leq t\right\}=F^{n\star}\left(t\right)
\end{eqnarray*}
que es la $n$-\'esima convoluci\'on de $F$. Entonces 

\begin{eqnarray*}
\left\{N\left(t\right)\geq n\right\}&=&\left\{T_{n}\leq t\right\}\\
P\left\{N\left(t\right)\leq n\right\}&=&1-F^{\left(n+1\right)\star}\left(t\right)
\end{eqnarray*}

Adem\'as usando el hecho de que $\esp\left[N\left(t\right)\right]=\sum_{n=1}^{\infty}P\left\{N\left(t\right)\geq n\right\}$
se tiene que

\begin{eqnarray*}
\esp\left[N\left(t\right)\right]=\sum_{n=1}^{\infty}F^{n\star}\left(t\right)
\end{eqnarray*}

\begin{Prop}
Para cada $t\geq0$, la funci\'on generadora de momentos $\esp\left[e^{\alpha N\left(t\right)}\right]$ existe para alguna $\alpha$ en una vecindad del 0, y de aqu\'i que $\esp\left[N\left(t\right)^{m}\right]<\infty$, para $m\geq1$.
\end{Prop}


\begin{Note}
Si el primer tiempo de renovaci\'on $\xi_{1}$ no tiene la misma distribuci\'on que el resto de las $\xi_{n}$, para $n\geq2$, a $N\left(t\right)$ se le llama Proceso de Renovaci\'on retardado, donde si $\xi$ tiene distribuci\'on $G$, entonces el tiempo $T_{n}$ de la $n$-\'esima renovaci\'on tiene distribuci\'on $G\star F^{\left(n-1\right)\star}\left(t\right)$
\end{Note}


\begin{Teo}
Para una constante $\mu\leq\infty$ ( o variable aleatoria), las siguientes expresiones son equivalentes:

\begin{eqnarray}
lim_{n\rightarrow\infty}n^{-1}T_{n}&=&\mu,\textrm{ c.s.}\\
lim_{t\rightarrow\infty}t^{-1}N\left(t\right)&=&1/\mu,\textrm{ c.s.}
\end{eqnarray}
\end{Teo}


Es decir, $T_{n}$ satisface la Ley Fuerte de los Grandes N\'umeros s\'i y s\'olo s\'i $N\left/t\right)$ la cumple.


\begin{Coro}[Ley Fuerte de los Grandes N\'umeros para Procesos de Renovaci\'on]
Si $N\left(t\right)$ es un proceso de renovaci\'on cuyos tiempos de inter-renovaci\'on tienen media $\mu\leq\infty$, entonces
\begin{eqnarray}
t^{-1}N\left(t\right)\rightarrow 1/\mu,\textrm{ c.s. cuando }t\rightarrow\infty.
\end{eqnarray}

\end{Coro}


Considerar el proceso estoc\'astico de valores reales $\left\{Z\left(t\right):t\geq0\right\}$ en el mismo espacio de probabilidad que $N\left(t\right)$

\begin{Def}
Para el proceso $\left\{Z\left(t\right):t\geq0\right\}$ se define la fluctuaci\'on m\'axima de $Z\left(t\right)$ en el intervalo $\left(T_{n-1},T_{n}\right]$:
\begin{eqnarray*}
M_{n}=\sup_{T_{n-1}<t\leq T_{n}}|Z\left(t\right)-Z\left(T_{n-1}\right)|
\end{eqnarray*}
\end{Def}

\begin{Teo}
Sup\'ongase que $n^{-1}T_{n}\rightarrow\mu$ c.s. cuando $n\rightarrow\infty$, donde $\mu\leq\infty$ es una constante o variable aleatoria. Sea $a$ una constante o variable aleatoria que puede ser infinita cuando $\mu$ es finita, y considere las expresiones l\'imite:
\begin{eqnarray}
lim_{n\rightarrow\infty}n^{-1}Z\left(T_{n}\right)&=&a,\textrm{ c.s.}\\
lim_{t\rightarrow\infty}t^{-1}Z\left(t\right)&=&a/\mu,\textrm{ c.s.}
\end{eqnarray}
La segunda expresi\'on implica la primera. Conversamente, la primera implica la segunda si el proceso $Z\left(t\right)$ es creciente, o si $lim_{n\rightarrow\infty}n^{-1}M_{n}=0$ c.s.
\end{Teo}

\begin{Coro}
Si $N\left(t\right)$ es un proceso de renovaci\'on, y $\left(Z\left(T_{n}\right)-Z\left(T_{n-1}\right),M_{n}\right)$, para $n\geq1$, son variables aleatorias independientes e id\'enticamente distribuidas con media finita, entonces,
\begin{eqnarray}
lim_{t\rightarrow\infty}t^{-1}Z\left(t\right)\rightarrow\frac{\esp\left[Z\left(T_{1}\right)-Z\left(T_{0}\right)\right]}{\esp\left[T_{1}\right]},\textrm{ c.s. cuando  }t\rightarrow\infty.
\end{eqnarray}
\end{Coro}

%___________________________________________________________________________________________
%
\subsection{Funci\'on de Renovaci\'on}
%___________________________________________________________________________________________
%


Sup\'ongase que $N\left(t\right)$ es un proceso de renovaci\'on con distribuci\'on $F$ con media finita $\mu$.

\begin{Def}
La funci\'on de renovaci\'on asociada con la distribuci\'on $F$, del proceso $N\left(t\right)$, es
\begin{eqnarray*}
U\left(t\right)=\sum_{n=1}^{\infty}F^{n\star}\left(t\right),\textrm{   }t\geq0,
\end{eqnarray*}
donde $F^{0\star}\left(t\right)=\indora\left(t\geq0\right)$.
\end{Def}


\begin{Prop}
Sup\'ongase que la distribuci\'on de inter-renovaci\'on $F$ tiene densidad $f$. Entonces $U\left(t\right)$ tambi\'en tiene densidad, para $t>0$, y es $U^{'}\left(t\right)=\sum_{n=0}^{\infty}f^{n\star}\left(t\right)$. Adem\'as
\begin{eqnarray*}
\prob\left\{N\left(t\right)>N\left(t-\right)\right\}=0\textrm{,   }t\geq0.
\end{eqnarray*}
\end{Prop}

\begin{Def}
La Transformada de Laplace-Stieljes de $F$ est\'a dada por

\begin{eqnarray*}
\hat{F}\left(\alpha\right)=\int_{\rea_{+}}e^{-\alpha t}dF\left(t\right)\textrm{,  }\alpha\geq0.
\end{eqnarray*}
\end{Def}

Entonces

\begin{eqnarray*}
\hat{U}\left(\alpha\right)=\sum_{n=0}^{\infty}\hat{F^{n\star}}\left(\alpha\right)=\sum_{n=0}^{\infty}\hat{F}\left(\alpha\right)^{n}=\frac{1}{1-\hat{F}\left(\alpha\right)}.
\end{eqnarray*}


\begin{Prop}
La Transformada de Laplace $\hat{U}\left(\alpha\right)$ y $\hat{F}\left(\alpha\right)$ determina una a la otra de manera \'unica por la relaci\'on $\hat{U}\left(\alpha\right)=\frac{1}{1-\hat{F}\left(\alpha\right)}$.
\end{Prop}


\begin{Note}
Un proceso de renovaci\'on $N\left(t\right)$ cuyos tiempos de inter-renovaci\'on tienen media finita, es un proceso Poisson con tasa $\lambda$ si y s\'olo s\'i $\esp\left[U\left(t\right)\right]=\lambda t$, para $t\geq0$.
\end{Note}


\begin{Teo}
Sea $N\left(t\right)$ un proceso puntual simple con puntos de localizaci\'on $T_{n}$ tal que $\eta\left(t\right)=\esp\left[N\left(\right)\right]$ es finita para cada $t$. Entonces para cualquier funci\'on $f:\rea_{+}\rightarrow\rea$,
\begin{eqnarray*}
\esp\left[\sum_{n=1}^{N\left(\right)}f\left(T_{n}\right)\right]=\int_{\left(0,t\right]}f\left(s\right)d\eta\left(s\right)\textrm{,  }t\geq0,
\end{eqnarray*}
suponiendo que la integral exista. Adem\'as si $X_{1},X_{2},\ldots$ son variables aleatorias definidas en el mismo espacio de probabilidad que el proceso $N\left(t\right)$ tal que $\esp\left[X_{n}|T_{n}=s\right]=f\left(s\right)$, independiente de $n$. Entonces
\begin{eqnarray*}
\esp\left[\sum_{n=1}^{N\left(t\right)}X_{n}\right]=\int_{\left(0,t\right]}f\left(s\right)d\eta\left(s\right)\textrm{,  }t\geq0,
\end{eqnarray*} 
suponiendo que la integral exista. 
\end{Teo}

\begin{Coro}[Identidad de Wald para Renovaciones]
Para el proceso de renovaci\'on $N\left(t\right)$,
\begin{eqnarray*}
\esp\left[T_{N\left(t\right)+1}\right]=\mu\esp\left[N\left(t\right)+1\right]\textrm{,  }t\geq0,
\end{eqnarray*}  
\end{Coro}

%______________________________________________________________________
\subsection{Procesos de Renovaci\'on}
%______________________________________________________________________

\begin{Def}\label{Def.Tn}
Sean $0\leq T_{1}\leq T_{2}\leq \ldots$ son tiempos aleatorios infinitos en los cuales ocurren ciertos eventos. El n\'umero de tiempos $T_{n}$ en el intervalo $\left[0,t\right)$ es

\begin{eqnarray}
N\left(t\right)=\sum_{n=1}^{\infty}\indora\left(T_{n}\leq t\right),
\end{eqnarray}
para $t\geq0$.
\end{Def}

Si se consideran los puntos $T_{n}$ como elementos de $\rea_{+}$, y $N\left(t\right)$ es el n\'umero de puntos en $\rea$. El proceso denotado por $\left\{N\left(t\right):t\geq0\right\}$, denotado por $N\left(t\right)$, es un proceso puntual en $\rea_{+}$. Los $T_{n}$ son los tiempos de ocurrencia, el proceso puntual $N\left(t\right)$ es simple si su n\'umero de ocurrencias son distintas: $0<T_{1}<T_{2}<\ldots$ casi seguramente.

\begin{Def}
Un proceso puntual $N\left(t\right)$ es un proceso de renovaci\'on si los tiempos de interocurrencia $\xi_{n}=T_{n}-T_{n-1}$, para $n\geq1$, son independientes e identicamente distribuidos con distribuci\'on $F$, donde $F\left(0\right)=0$ y $T_{0}=0$. Los $T_{n}$ son llamados tiempos de renovaci\'on, referente a la independencia o renovaci\'on de la informaci\'on estoc\'astica en estos tiempos. Los $\xi_{n}$ son los tiempos de inter-renovaci\'on, y $N\left(t\right)$ es el n\'umero de renovaciones en el intervalo $\left[0,t\right)$
\end{Def}


\begin{Note}
Para definir un proceso de renovaci\'on para cualquier contexto, solamente hay que especificar una distribuci\'on $F$, con $F\left(0\right)=0$, para los tiempos de inter-renovaci\'on. La funci\'on $F$ en turno degune las otra variables aleatorias. De manera formal, existe un espacio de probabilidad y una sucesi\'on de variables aleatorias $\xi_{1},\xi_{2},\ldots$ definidas en este con distribuci\'on $F$. Entonces las otras cantidades son $T_{n}=\sum_{k=1}^{n}\xi_{k}$ y $N\left(t\right)=\sum_{n=1}^{\infty}\indora\left(T_{n}\leq t\right)$, donde $T_{n}\rightarrow\infty$ casi seguramente por la Ley Fuerte de los Grandes Números.
\end{Note}
%_____________________________________________________
\subsection{Puntos de Renovaci\'on}
%_____________________________________________________

Para cada cola $Q_{i}$ se tienen los procesos de arribo a la cola, para estas, los tiempos de arribo est\'an dados por $$\left\{T_{1}^{i},T_{2}^{i},\ldots,T_{k}^{i},\ldots\right\},$$ entonces, consideremos solamente los primeros tiempos de arribo a cada una de las colas, es decir, $$\left\{T_{1}^{1},T_{1}^{2},T_{1}^{3},T_{1}^{4}\right\},$$ se sabe que cada uno de estos tiempos se distribuye de manera exponencial con par\'ametro $1/mu_{i}$. Adem\'as se sabe que para $$T^{*}=\min\left\{T_{1}^{1},T_{1}^{2},T_{1}^{3},T_{1}^{4}\right\},$$ $T^{*}$ se distribuye de manera exponencial con par\'ametro $$\mu^{*}=\sum_{i=1}^{4}\mu_{i}.$$ Ahora, dado que 
\begin{center}
\begin{tabular}{lcl}
$\tilde{r}=r_{1}+r_{2}$ & y &$\hat{r}=r_{3}+r_{4}.$
\end{tabular}
\end{center}


Supongamos que $$\tilde{r},\hat{r}<\mu^{*},$$ entonces si tomamos $$r^{*}=\min\left\{\tilde{r},\hat{r}\right\},$$ se tiene que para  $$t^{*}\in\left(0,r^{*}\right)$$ se cumple que 
\begin{center}
\begin{tabular}{lcl}
$\tau_{1}\left(1\right)=0$ & y por tanto & $\overline{\tau}_{1}=0,$
\end{tabular}
\end{center}
entonces para la segunda cola en este primer ciclo se cumple que $$\tau_{2}=\overline{\tau}_{1}+r_{1}=r_{1}<\mu^{*},$$ y por tanto se tiene que  $$\overline{\tau}_{2}=\tau_{2}.$$ Por lo tanto, nuevamente para la primer cola en el segundo ciclo $$\tau_{1}\left(2\right)=\tau_{2}\left(1\right)+r_{2}=\tilde{r}<\mu^{*}.$$ An\'alogamente para el segundo sistema se tiene que ambas colas est\'an vac\'ias, es decir, existe un valor $t^{*}$ tal que en el intervalo $\left(0,t^{*}\right)$ no ha llegado ning\'un usuario, es decir, $$L_{i}\left(t^{*}\right)=0$$ para $i=1,2,3,4$.



%________________________________________________________________________
\subsection{Procesos Regenerativos}
%________________________________________________________________________

Para $\left\{X\left(t\right):t\geq0\right\}$ Proceso Estoc\'astico a tiempo continuo con estado de espacios $S$, que es un espacio m\'etrico, con trayectorias continuas por la derecha y con l\'imites por la izquierda c.s. Sea $N\left(t\right)$ un proceso de renovaci\'on en $\rea_{+}$ definido en el mismo espacio de probabilidad que $X\left(t\right)$, con tiempos de renovaci\'on $T$ y tiempos de inter-renovaci\'on $\xi_{n}=T_{n}-T_{n-1}$, con misma distribuci\'on $F$ de media finita $\mu$.



\begin{Def}
Para el proceso $\left\{\left(N\left(t\right),X\left(t\right)\right):t\geq0\right\}$, sus trayectoria muestrales en el intervalo de tiempo $\left[T_{n-1},T_{n}\right)$ est\'an descritas por
\begin{eqnarray*}
\zeta_{n}=\left(\xi_{n},\left\{X\left(T_{n-1}+t\right):0\leq t<\xi_{n}\right\}\right)
\end{eqnarray*}
Este $\zeta_{n}$ es el $n$-\'esimo segmento del proceso. El proceso es regenerativo sobre los tiempos $T_{n}$ si sus segmentos $\zeta_{n}$ son independientes e id\'enticamennte distribuidos.
\end{Def}


\begin{Obs}
Si $\tilde{X}\left(t\right)$ con espacio de estados $\tilde{S}$ es regenerativo sobre $T_{n}$, entonces $X\left(t\right)=f\left(\tilde{X}\left(t\right)\right)$ tambi\'en es regenerativo sobre $T_{n}$, para cualquier funci\'on $f:\tilde{S}\rightarrow S$.
\end{Obs}

\begin{Obs}
Los procesos regenerativos son crudamente regenerativos, pero no al rev\'es.
\end{Obs}

\begin{Def}[Definici\'on Cl\'asica]
Un proceso estoc\'astico $X=\left\{X\left(t\right):t\geq0\right\}$ es llamado regenerativo is existe una variable aleatoria $R_{1}>0$ tal que
\begin{itemize}
\item[i)] $\left\{X\left(t+R_{1}\right):t\geq0\right\}$ es independiente de $\left\{\left\{X\left(t\right):t<R_{1}\right\},\right\}$
\item[ii)] $\left\{X\left(t+R_{1}\right):t\geq0\right\}$ es estoc\'asticamente equivalente a $\left\{X\left(t\right):t>0\right\}$
\end{itemize}

Llamamos a $R_{1}$ tiempo de regeneraci\'on, y decimos que $X$ se regenera en este punto.
\end{Def}

$\left\{X\left(t+R_{1}\right)\right\}$ es regenerativo con tiempo de regeneraci\'on $R_{2}$, independiente de $R_{1}$ pero con la misma distribuci\'on que $R_{1}$. Procediendo de esta manera se obtiene una secuencia de variables aleatorias independientes e id\'enticamente distribuidas $\left\{R_{n}\right\}$ llamados longitudes de ciclo. Si definimos a $Z_{k}\equiv R_{1}+R_{2}+\cdots+R_{k}$, se tiene un proceso de renovaci\'on llamado proceso de renovaci\'on encajado para $X$.

\begin{Note}
Un proceso regenerativo con media de la longitud de ciclo finita es llamado positivo recurrente.
\end{Note}


\begin{Def}
Para $x$ fijo y para cada $t\geq0$, sea $I_{x}\left(t\right)=1$ si $X\left(t\right)\leq x$,  $I_{x}\left(t\right)=0$ en caso contrario, y def\'inanse los tiempos promedio
\begin{eqnarray*}
\overline{X}&=&lim_{t\rightarrow\infty}\frac{1}{t}\int_{0}^{\infty}X\left(u\right)du\\
\prob\left(X_{\infty}\leq x\right)&=&lim_{t\rightarrow\infty}\frac{1}{t}\int_{0}^{\infty}I_{x}\left(u\right)du,
\end{eqnarray*}
cuando estos l\'imites existan.
\end{Def}

Como consecuencia del teorema de Renovaci\'on-Recompensa, se tiene que el primer l\'imite  existe y es igual a la constante
\begin{eqnarray*}
\overline{X}&=&\frac{\esp\left[\int_{0}^{R_{1}}X\left(t\right)dt\right]}{\esp\left[R_{1}\right]},
\end{eqnarray*}
suponiendo que ambas esperanzas son finitas.

\begin{Note}
\begin{itemize}
\item[a)] Si el proceso regenerativo $X$ es positivo recurrente y tiene trayectorias muestrales no negativas, entonces la ecuaci\'on anterior es v\'alida.
\item[b)] Si $X$ es positivo recurrente regenerativo, podemos construir una \'unica versi\'on estacionaria de este proceso, $X_{e}=\left\{X_{e}\left(t\right)\right\}$, donde $X_{e}$ es un proceso estoc\'astico regenerativo y estrictamente estacionario, con distribuci\'on marginal distribuida como $X_{\infty}$
\end{itemize}
\end{Note}

\subsection{Renewal and Regenerative Processes: Serfozo\cite{Serfozo}}
\begin{Def}\label{Def.Tn}
Sean $0\leq T_{1}\leq T_{2}\leq \ldots$ son tiempos aleatorios infinitos en los cuales ocurren ciertos eventos. El n\'umero de tiempos $T_{n}$ en el intervalo $\left[0,t\right)$ es

\begin{eqnarray}
N\left(t\right)=\sum_{n=1}^{\infty}\indora\left(T_{n}\leq t\right),
\end{eqnarray}
para $t\geq0$.
\end{Def}

Si se consideran los puntos $T_{n}$ como elementos de $\rea_{+}$, y $N\left(t\right)$ es el n\'umero de puntos en $\rea$. El proceso denotado por $\left\{N\left(t\right):t\geq0\right\}$, denotado por $N\left(t\right)$, es un proceso puntual en $\rea_{+}$. Los $T_{n}$ son los tiempos de ocurrencia, el proceso puntual $N\left(t\right)$ es simple si su n\'umero de ocurrencias son distintas: $0<T_{1}<T_{2}<\ldots$ casi seguramente.

\begin{Def}
Un proceso puntual $N\left(t\right)$ es un proceso de renovaci\'on si los tiempos de interocurrencia $\xi_{n}=T_{n}-T_{n-1}$, para $n\geq1$, son independientes e identicamente distribuidos con distribuci\'on $F$, donde $F\left(0\right)=0$ y $T_{0}=0$. Los $T_{n}$ son llamados tiempos de renovaci\'on, referente a la independencia o renovaci\'on de la informaci\'on estoc\'astica en estos tiempos. Los $\xi_{n}$ son los tiempos de inter-renovaci\'on, y $N\left(t\right)$ es el n\'umero de renovaciones en el intervalo $\left[0,t\right)$
\end{Def}


\begin{Note}
Para definir un proceso de renovaci\'on para cualquier contexto, solamente hay que especificar una distribuci\'on $F$, con $F\left(0\right)=0$, para los tiempos de inter-renovaci\'on. La funci\'on $F$ en turno degune las otra variables aleatorias. De manera formal, existe un espacio de probabilidad y una sucesi\'on de variables aleatorias $\xi_{1},\xi_{2},\ldots$ definidas en este con distribuci\'on $F$. Entonces las otras cantidades son $T_{n}=\sum_{k=1}^{n}\xi_{k}$ y $N\left(t\right)=\sum_{n=1}^{\infty}\indora\left(T_{n}\leq t\right)$, donde $T_{n}\rightarrow\infty$ casi seguramente por la Ley Fuerte de los Grandes N\'umeros.
\end{Note}







Los tiempos $T_{n}$ est\'an relacionados con los conteos de $N\left(t\right)$ por

\begin{eqnarray*}
\left\{N\left(t\right)\geq n\right\}&=&\left\{T_{n}\leq t\right\}\\
T_{N\left(t\right)}\leq &t&<T_{N\left(t\right)+1},
\end{eqnarray*}

adem\'as $N\left(T_{n}\right)=n$, y 

\begin{eqnarray*}
N\left(t\right)=\max\left\{n:T_{n}\leq t\right\}=\min\left\{n:T_{n+1}>t\right\}
\end{eqnarray*}

Por propiedades de la convoluci\'on se sabe que

\begin{eqnarray*}
P\left\{T_{n}\leq t\right\}=F^{n\star}\left(t\right)
\end{eqnarray*}
que es la $n$-\'esima convoluci\'on de $F$. Entonces 

\begin{eqnarray*}
\left\{N\left(t\right)\geq n\right\}&=&\left\{T_{n}\leq t\right\}\\
P\left\{N\left(t\right)\leq n\right\}&=&1-F^{\left(n+1\right)\star}\left(t\right)
\end{eqnarray*}

Adem\'as usando el hecho de que $\esp\left[N\left(t\right)\right]=\sum_{n=1}^{\infty}P\left\{N\left(t\right)\geq n\right\}$
se tiene que

\begin{eqnarray*}
\esp\left[N\left(t\right)\right]=\sum_{n=1}^{\infty}F^{n\star}\left(t\right)
\end{eqnarray*}

\begin{Prop}
Para cada $t\geq0$, la funci\'on generadora de momentos $\esp\left[e^{\alpha N\left(t\right)}\right]$ existe para alguna $\alpha$ en una vecindad del 0, y de aqu\'i que $\esp\left[N\left(t\right)^{m}\right]<\infty$, para $m\geq1$.
\end{Prop}


\begin{Note}
Si el primer tiempo de renovaci\'on $\xi_{1}$ no tiene la misma distribuci\'on que el resto de las $\xi_{n}$, para $n\geq2$, a $N\left(t\right)$ se le llama Proceso de Renovaci\'on retardado, donde si $\xi$ tiene distribuci\'on $G$, entonces el tiempo $T_{n}$ de la $n$-\'esima renovaci\'on tiene distribuci\'on $G\star F^{\left(n-1\right)\star}\left(t\right)$
\end{Note}


\begin{Teo}
Para una constante $\mu\leq\infty$ ( o variable aleatoria), las siguientes expresiones son equivalentes:

\begin{eqnarray}
lim_{n\rightarrow\infty}n^{-1}T_{n}&=&\mu,\textrm{ c.s.}\\
lim_{t\rightarrow\infty}t^{-1}N\left(t\right)&=&1/\mu,\textrm{ c.s.}
\end{eqnarray}
\end{Teo}


Es decir, $T_{n}$ satisface la Ley Fuerte de los Grandes N\'umeros s\'i y s\'olo s\'i $N\left/t\right)$ la cumple.


\begin{Coro}[Ley Fuerte de los Grandes N\'umeros para Procesos de Renovaci\'on]
Si $N\left(t\right)$ es un proceso de renovaci\'on cuyos tiempos de inter-renovaci\'on tienen media $\mu\leq\infty$, entonces
\begin{eqnarray}
t^{-1}N\left(t\right)\rightarrow 1/\mu,\textrm{ c.s. cuando }t\rightarrow\infty.
\end{eqnarray}

\end{Coro}


Considerar el proceso estoc\'astico de valores reales $\left\{Z\left(t\right):t\geq0\right\}$ en el mismo espacio de probabilidad que $N\left(t\right)$

\begin{Def}
Para el proceso $\left\{Z\left(t\right):t\geq0\right\}$ se define la fluctuaci\'on m\'axima de $Z\left(t\right)$ en el intervalo $\left(T_{n-1},T_{n}\right]$:
\begin{eqnarray*}
M_{n}=\sup_{T_{n-1}<t\leq T_{n}}|Z\left(t\right)-Z\left(T_{n-1}\right)|
\end{eqnarray*}
\end{Def}

\begin{Teo}
Sup\'ongase que $n^{-1}T_{n}\rightarrow\mu$ c.s. cuando $n\rightarrow\infty$, donde $\mu\leq\infty$ es una constante o variable aleatoria. Sea $a$ una constante o variable aleatoria que puede ser infinita cuando $\mu$ es finita, y considere las expresiones l\'imite:
\begin{eqnarray}
lim_{n\rightarrow\infty}n^{-1}Z\left(T_{n}\right)&=&a,\textrm{ c.s.}\\
lim_{t\rightarrow\infty}t^{-1}Z\left(t\right)&=&a/\mu,\textrm{ c.s.}
\end{eqnarray}
La segunda expresi\'on implica la primera. Conversamente, la primera implica la segunda si el proceso $Z\left(t\right)$ es creciente, o si $lim_{n\rightarrow\infty}n^{-1}M_{n}=0$ c.s.
\end{Teo}

\begin{Coro}
Si $N\left(t\right)$ es un proceso de renovaci\'on, y $\left(Z\left(T_{n}\right)-Z\left(T_{n-1}\right),M_{n}\right)$, para $n\geq1$, son variables aleatorias independientes e id\'enticamente distribuidas con media finita, entonces,
\begin{eqnarray}
lim_{t\rightarrow\infty}t^{-1}Z\left(t\right)\rightarrow\frac{\esp\left[Z\left(T_{1}\right)-Z\left(T_{0}\right)\right]}{\esp\left[T_{1}\right]},\textrm{ c.s. cuando  }t\rightarrow\infty.
\end{eqnarray}
\end{Coro}


Sup\'ongase que $N\left(t\right)$ es un proceso de renovaci\'on con distribuci\'on $F$ con media finita $\mu$.

\begin{Def}
La funci\'on de renovaci\'on asociada con la distribuci\'on $F$, del proceso $N\left(t\right)$, es
\begin{eqnarray*}
U\left(t\right)=\sum_{n=1}^{\infty}F^{n\star}\left(t\right),\textrm{   }t\geq0,
\end{eqnarray*}
donde $F^{0\star}\left(t\right)=\indora\left(t\geq0\right)$.
\end{Def}


\begin{Prop}
Sup\'ongase que la distribuci\'on de inter-renovaci\'on $F$ tiene densidad $f$. Entonces $U\left(t\right)$ tambi\'en tiene densidad, para $t>0$, y es $U^{'}\left(t\right)=\sum_{n=0}^{\infty}f^{n\star}\left(t\right)$. Adem\'as
\begin{eqnarray*}
\prob\left\{N\left(t\right)>N\left(t-\right)\right\}=0\textrm{,   }t\geq0.
\end{eqnarray*}
\end{Prop}

\begin{Def}
La Transformada de Laplace-Stieljes de $F$ est\'a dada por

\begin{eqnarray*}
\hat{F}\left(\alpha\right)=\int_{\rea_{+}}e^{-\alpha t}dF\left(t\right)\textrm{,  }\alpha\geq0.
\end{eqnarray*}
\end{Def}

Entonces

\begin{eqnarray*}
\hat{U}\left(\alpha\right)=\sum_{n=0}^{\infty}\hat{F^{n\star}}\left(\alpha\right)=\sum_{n=0}^{\infty}\hat{F}\left(\alpha\right)^{n}=\frac{1}{1-\hat{F}\left(\alpha\right)}.
\end{eqnarray*}


\begin{Prop}
La Transformada de Laplace $\hat{U}\left(\alpha\right)$ y $\hat{F}\left(\alpha\right)$ determina una a la otra de manera \'unica por la relaci\'on $\hat{U}\left(\alpha\right)=\frac{1}{1-\hat{F}\left(\alpha\right)}$.
\end{Prop}


\begin{Note}
Un proceso de renovaci\'on $N\left(t\right)$ cuyos tiempos de inter-renovaci\'on tienen media finita, es un proceso Poisson con tasa $\lambda$ si y s\'olo s\'i $\esp\left[U\left(t\right)\right]=\lambda t$, para $t\geq0$.
\end{Note}


\begin{Teo}
Sea $N\left(t\right)$ un proceso puntual simple con puntos de localizaci\'on $T_{n}$ tal que $\eta\left(t\right)=\esp\left[N\left(\right)\right]$ es finita para cada $t$. Entonces para cualquier funci\'on $f:\rea_{+}\rightarrow\rea$,
\begin{eqnarray*}
\esp\left[\sum_{n=1}^{N\left(\right)}f\left(T_{n}\right)\right]=\int_{\left(0,t\right]}f\left(s\right)d\eta\left(s\right)\textrm{,  }t\geq0,
\end{eqnarray*}
suponiendo que la integral exista. Adem\'as si $X_{1},X_{2},\ldots$ son variables aleatorias definidas en el mismo espacio de probabilidad que el proceso $N\left(t\right)$ tal que $\esp\left[X_{n}|T_{n}=s\right]=f\left(s\right)$, independiente de $n$. Entonces
\begin{eqnarray*}
\esp\left[\sum_{n=1}^{N\left(t\right)}X_{n}\right]=\int_{\left(0,t\right]}f\left(s\right)d\eta\left(s\right)\textrm{,  }t\geq0,
\end{eqnarray*} 
suponiendo que la integral exista. 
\end{Teo}

\begin{Coro}[Identidad de Wald para Renovaciones]
Para el proceso de renovaci\'on $N\left(t\right)$,
\begin{eqnarray*}
\esp\left[T_{N\left(t\right)+1}\right]=\mu\esp\left[N\left(t\right)+1\right]\textrm{,  }t\geq0,
\end{eqnarray*}  
\end{Coro}


\begin{Def}
Sea $h\left(t\right)$ funci\'on de valores reales en $\rea$ acotada en intervalos finitos e igual a cero para $t<0$ La ecuaci\'on de renovaci\'on para $h\left(t\right)$ y la distribuci\'on $F$ es

\begin{eqnarray}\label{Ec.Renovacion}
H\left(t\right)=h\left(t\right)+\int_{\left[0,t\right]}H\left(t-s\right)dF\left(s\right)\textrm{,    }t\geq0,
\end{eqnarray}
donde $H\left(t\right)$ es una funci\'on de valores reales. Esto es $H=h+F\star H$. Decimos que $H\left(t\right)$ es soluci\'on de esta ecuaci\'on si satisface la ecuaci\'on, y es acotada en intervalos finitos e iguales a cero para $t<0$.
\end{Def}

\begin{Prop}
La funci\'on $U\star h\left(t\right)$ es la \'unica soluci\'on de la ecuaci\'on de renovaci\'on (\ref{Ec.Renovacion}).
\end{Prop}

\begin{Teo}[Teorema Renovaci\'on Elemental]
\begin{eqnarray*}
t^{-1}U\left(t\right)\rightarrow 1/\mu\textrm{,    cuando }t\rightarrow\infty.
\end{eqnarray*}
\end{Teo}



Sup\'ongase que $N\left(t\right)$ es un proceso de renovaci\'on con distribuci\'on $F$ con media finita $\mu$.

\begin{Def}
La funci\'on de renovaci\'on asociada con la distribuci\'on $F$, del proceso $N\left(t\right)$, es
\begin{eqnarray*}
U\left(t\right)=\sum_{n=1}^{\infty}F^{n\star}\left(t\right),\textrm{   }t\geq0,
\end{eqnarray*}
donde $F^{0\star}\left(t\right)=\indora\left(t\geq0\right)$.
\end{Def}


\begin{Prop}
Sup\'ongase que la distribuci\'on de inter-renovaci\'on $F$ tiene densidad $f$. Entonces $U\left(t\right)$ tambi\'en tiene densidad, para $t>0$, y es $U^{'}\left(t\right)=\sum_{n=0}^{\infty}f^{n\star}\left(t\right)$. Adem\'as
\begin{eqnarray*}
\prob\left\{N\left(t\right)>N\left(t-\right)\right\}=0\textrm{,   }t\geq0.
\end{eqnarray*}
\end{Prop}

\begin{Def}
La Transformada de Laplace-Stieljes de $F$ est\'a dada por

\begin{eqnarray*}
\hat{F}\left(\alpha\right)=\int_{\rea_{+}}e^{-\alpha t}dF\left(t\right)\textrm{,  }\alpha\geq0.
\end{eqnarray*}
\end{Def}

Entonces

\begin{eqnarray*}
\hat{U}\left(\alpha\right)=\sum_{n=0}^{\infty}\hat{F^{n\star}}\left(\alpha\right)=\sum_{n=0}^{\infty}\hat{F}\left(\alpha\right)^{n}=\frac{1}{1-\hat{F}\left(\alpha\right)}.
\end{eqnarray*}


\begin{Prop}
La Transformada de Laplace $\hat{U}\left(\alpha\right)$ y $\hat{F}\left(\alpha\right)$ determina una a la otra de manera \'unica por la relaci\'on $\hat{U}\left(\alpha\right)=\frac{1}{1-\hat{F}\left(\alpha\right)}$.
\end{Prop}


\begin{Note}
Un proceso de renovaci\'on $N\left(t\right)$ cuyos tiempos de inter-renovaci\'on tienen media finita, es un proceso Poisson con tasa $\lambda$ si y s\'olo s\'i $\esp\left[U\left(t\right)\right]=\lambda t$, para $t\geq0$.
\end{Note}


\begin{Teo}
Sea $N\left(t\right)$ un proceso puntual simple con puntos de localizaci\'on $T_{n}$ tal que $\eta\left(t\right)=\esp\left[N\left(\right)\right]$ es finita para cada $t$. Entonces para cualquier funci\'on $f:\rea_{+}\rightarrow\rea$,
\begin{eqnarray*}
\esp\left[\sum_{n=1}^{N\left(\right)}f\left(T_{n}\right)\right]=\int_{\left(0,t\right]}f\left(s\right)d\eta\left(s\right)\textrm{,  }t\geq0,
\end{eqnarray*}
suponiendo que la integral exista. Adem\'as si $X_{1},X_{2},\ldots$ son variables aleatorias definidas en el mismo espacio de probabilidad que el proceso $N\left(t\right)$ tal que $\esp\left[X_{n}|T_{n}=s\right]=f\left(s\right)$, independiente de $n$. Entonces
\begin{eqnarray*}
\esp\left[\sum_{n=1}^{N\left(t\right)}X_{n}\right]=\int_{\left(0,t\right]}f\left(s\right)d\eta\left(s\right)\textrm{,  }t\geq0,
\end{eqnarray*} 
suponiendo que la integral exista. 
\end{Teo}

\begin{Coro}[Identidad de Wald para Renovaciones]
Para el proceso de renovaci\'on $N\left(t\right)$,
\begin{eqnarray*}
\esp\left[T_{N\left(t\right)+1}\right]=\mu\esp\left[N\left(t\right)+1\right]\textrm{,  }t\geq0,
\end{eqnarray*}  
\end{Coro}


\begin{Def}
Sea $h\left(t\right)$ funci\'on de valores reales en $\rea$ acotada en intervalos finitos e igual a cero para $t<0$ La ecuaci\'on de renovaci\'on para $h\left(t\right)$ y la distribuci\'on $F$ es

\begin{eqnarray}\label{Ec.Renovacion}
H\left(t\right)=h\left(t\right)+\int_{\left[0,t\right]}H\left(t-s\right)dF\left(s\right)\textrm{,    }t\geq0,
\end{eqnarray}
donde $H\left(t\right)$ es una funci\'on de valores reales. Esto es $H=h+F\star H$. Decimos que $H\left(t\right)$ es soluci\'on de esta ecuaci\'on si satisface la ecuaci\'on, y es acotada en intervalos finitos e iguales a cero para $t<0$.
\end{Def}

\begin{Prop}
La funci\'on $U\star h\left(t\right)$ es la \'unica soluci\'on de la ecuaci\'on de renovaci\'on (\ref{Ec.Renovacion}).
\end{Prop}

\begin{Teo}[Teorema Renovaci\'on Elemental]
\begin{eqnarray*}
t^{-1}U\left(t\right)\rightarrow 1/\mu\textrm{,    cuando }t\rightarrow\infty.
\end{eqnarray*}
\end{Teo}


\begin{Note} Una funci\'on $h:\rea_{+}\rightarrow\rea$ es Directamente Riemann Integrable en los siguientes casos:
\begin{itemize}
\item[a)] $h\left(t\right)\geq0$ es decreciente y Riemann Integrable.
\item[b)] $h$ es continua excepto posiblemente en un conjunto de Lebesgue de medida 0, y $|h\left(t\right)|\leq b\left(t\right)$, donde $b$ es DRI.
\end{itemize}
\end{Note}

\begin{Teo}[Teorema Principal de Renovaci\'on]
Si $F$ es no aritm\'etica y $h\left(t\right)$ es Directamente Riemann Integrable (DRI), entonces

\begin{eqnarray*}
lim_{t\rightarrow\infty}U\star h=\frac{1}{\mu}\int_{\rea_{+}}h\left(s\right)ds.
\end{eqnarray*}
\end{Teo}

\begin{Prop}
Cualquier funci\'on $H\left(t\right)$ acotada en intervalos finitos y que es 0 para $t<0$ puede expresarse como
\begin{eqnarray*}
H\left(t\right)=U\star h\left(t\right)\textrm{,  donde }h\left(t\right)=H\left(t\right)-F\star H\left(t\right)
\end{eqnarray*}
\end{Prop}

\begin{Def}
Un proceso estoc\'astico $X\left(t\right)$ es crudamente regenerativo en un tiempo aleatorio positivo $T$ si
\begin{eqnarray*}
\esp\left[X\left(T+t\right)|T\right]=\esp\left[X\left(t\right)\right]\textrm{, para }t\geq0,\end{eqnarray*}
y con las esperanzas anteriores finitas.
\end{Def}

\begin{Prop}
Sup\'ongase que $X\left(t\right)$ es un proceso crudamente regenerativo en $T$, que tiene distribuci\'on $F$. Si $\esp\left[X\left(t\right)\right]$ es acotado en intervalos finitos, entonces
\begin{eqnarray*}
\esp\left[X\left(t\right)\right]=U\star h\left(t\right)\textrm{,  donde }h\left(t\right)=\esp\left[X\left(t\right)\indora\left(T>t\right)\right].
\end{eqnarray*}
\end{Prop}

\begin{Teo}[Regeneraci\'on Cruda]
Sup\'ongase que $X\left(t\right)$ es un proceso con valores positivo crudamente regenerativo en $T$, y def\'inase $M=\sup\left\{|X\left(t\right)|:t\leq T\right\}$. Si $T$ es no aritm\'etico y $M$ y $MT$ tienen media finita, entonces
\begin{eqnarray*}
lim_{t\rightarrow\infty}\esp\left[X\left(t\right)\right]=\frac{1}{\mu}\int_{\rea_{+}}h\left(s\right)ds,
\end{eqnarray*}
donde $h\left(t\right)=\esp\left[X\left(t\right)\indora\left(T>t\right)\right]$.
\end{Teo}


\begin{Note} Una funci\'on $h:\rea_{+}\rightarrow\rea$ es Directamente Riemann Integrable en los siguientes casos:
\begin{itemize}
\item[a)] $h\left(t\right)\geq0$ es decreciente y Riemann Integrable.
\item[b)] $h$ es continua excepto posiblemente en un conjunto de Lebesgue de medida 0, y $|h\left(t\right)|\leq b\left(t\right)$, donde $b$ es DRI.
\end{itemize}
\end{Note}

\begin{Teo}[Teorema Principal de Renovaci\'on]
Si $F$ es no aritm\'etica y $h\left(t\right)$ es Directamente Riemann Integrable (DRI), entonces

\begin{eqnarray*}
lim_{t\rightarrow\infty}U\star h=\frac{1}{\mu}\int_{\rea_{+}}h\left(s\right)ds.
\end{eqnarray*}
\end{Teo}

\begin{Prop}
Cualquier funci\'on $H\left(t\right)$ acotada en intervalos finitos y que es 0 para $t<0$ puede expresarse como
\begin{eqnarray*}
H\left(t\right)=U\star h\left(t\right)\textrm{,  donde }h\left(t\right)=H\left(t\right)-F\star H\left(t\right)
\end{eqnarray*}
\end{Prop}

\begin{Def}
Un proceso estoc\'astico $X\left(t\right)$ es crudamente regenerativo en un tiempo aleatorio positivo $T$ si
\begin{eqnarray*}
\esp\left[X\left(T+t\right)|T\right]=\esp\left[X\left(t\right)\right]\textrm{, para }t\geq0,\end{eqnarray*}
y con las esperanzas anteriores finitas.
\end{Def}

\begin{Prop}
Sup\'ongase que $X\left(t\right)$ es un proceso crudamente regenerativo en $T$, que tiene distribuci\'on $F$. Si $\esp\left[X\left(t\right)\right]$ es acotado en intervalos finitos, entonces
\begin{eqnarray*}
\esp\left[X\left(t\right)\right]=U\star h\left(t\right)\textrm{,  donde }h\left(t\right)=\esp\left[X\left(t\right)\indora\left(T>t\right)\right].
\end{eqnarray*}
\end{Prop}

\begin{Teo}[Regeneraci\'on Cruda]
Sup\'ongase que $X\left(t\right)$ es un proceso con valores positivo crudamente regenerativo en $T$, y def\'inase $M=\sup\left\{|X\left(t\right)|:t\leq T\right\}$. Si $T$ es no aritm\'etico y $M$ y $MT$ tienen media finita, entonces
\begin{eqnarray*}
lim_{t\rightarrow\infty}\esp\left[X\left(t\right)\right]=\frac{1}{\mu}\int_{\rea_{+}}h\left(s\right)ds,
\end{eqnarray*}
donde $h\left(t\right)=\esp\left[X\left(t\right)\indora\left(T>t\right)\right]$.
\end{Teo}

%________________________________________________________________________
\subsection{Procesos Regenerativos}
%________________________________________________________________________

Para $\left\{X\left(t\right):t\geq0\right\}$ Proceso Estoc\'astico a tiempo continuo con estado de espacios $S$, que es un espacio m\'etrico, con trayectorias continuas por la derecha y con l\'imites por la izquierda c.s. Sea $N\left(t\right)$ un proceso de renovaci\'on en $\rea_{+}$ definido en el mismo espacio de probabilidad que $X\left(t\right)$, con tiempos de renovaci\'on $T$ y tiempos de inter-renovaci\'on $\xi_{n}=T_{n}-T_{n-1}$, con misma distribuci\'on $F$ de media finita $\mu$.



\begin{Def}
Para el proceso $\left\{\left(N\left(t\right),X\left(t\right)\right):t\geq0\right\}$, sus trayectoria muestrales en el intervalo de tiempo $\left[T_{n-1},T_{n}\right)$ est\'an descritas por
\begin{eqnarray*}
\zeta_{n}=\left(\xi_{n},\left\{X\left(T_{n-1}+t\right):0\leq t<\xi_{n}\right\}\right)
\end{eqnarray*}
Este $\zeta_{n}$ es el $n$-\'esimo segmento del proceso. El proceso es regenerativo sobre los tiempos $T_{n}$ si sus segmentos $\zeta_{n}$ son independientes e id\'enticamennte distribuidos.
\end{Def}


\begin{Obs}
Si $\tilde{X}\left(t\right)$ con espacio de estados $\tilde{S}$ es regenerativo sobre $T_{n}$, entonces $X\left(t\right)=f\left(\tilde{X}\left(t\right)\right)$ tambi\'en es regenerativo sobre $T_{n}$, para cualquier funci\'on $f:\tilde{S}\rightarrow S$.
\end{Obs}

\begin{Obs}
Los procesos regenerativos son crudamente regenerativos, pero no al rev\'es.
\end{Obs}

\begin{Def}[Definici\'on Cl\'asica]
Un proceso estoc\'astico $X=\left\{X\left(t\right):t\geq0\right\}$ es llamado regenerativo is existe una variable aleatoria $R_{1}>0$ tal que
\begin{itemize}
\item[i)] $\left\{X\left(t+R_{1}\right):t\geq0\right\}$ es independiente de $\left\{\left\{X\left(t\right):t<R_{1}\right\},\right\}$
\item[ii)] $\left\{X\left(t+R_{1}\right):t\geq0\right\}$ es estoc\'asticamente equivalente a $\left\{X\left(t\right):t>0\right\}$
\end{itemize}

Llamamos a $R_{1}$ tiempo de regeneraci\'on, y decimos que $X$ se regenera en este punto.
\end{Def}

$\left\{X\left(t+R_{1}\right)\right\}$ es regenerativo con tiempo de regeneraci\'on $R_{2}$, independiente de $R_{1}$ pero con la misma distribuci\'on que $R_{1}$. Procediendo de esta manera se obtiene una secuencia de variables aleatorias independientes e id\'enticamente distribuidas $\left\{R_{n}\right\}$ llamados longitudes de ciclo. Si definimos a $Z_{k}\equiv R_{1}+R_{2}+\cdots+R_{k}$, se tiene un proceso de renovaci\'on llamado proceso de renovaci\'on encajado para $X$.

\begin{Note}
Un proceso regenerativo con media de la longitud de ciclo finita es llamado positivo recurrente.
\end{Note}


\begin{Def}
Para $x$ fijo y para cada $t\geq0$, sea $I_{x}\left(t\right)=1$ si $X\left(t\right)\leq x$,  $I_{x}\left(t\right)=0$ en caso contrario, y def\'inanse los tiempos promedio
\begin{eqnarray*}
\overline{X}&=&lim_{t\rightarrow\infty}\frac{1}{t}\int_{0}^{\infty}X\left(u\right)du\\
\prob\left(X_{\infty}\leq x\right)&=&lim_{t\rightarrow\infty}\frac{1}{t}\int_{0}^{\infty}I_{x}\left(u\right)du,
\end{eqnarray*}
cuando estos l\'imites existan.
\end{Def}

Como consecuencia del teorema de Renovaci\'on-Recompensa, se tiene que el primer l\'imite  existe y es igual a la constante
\begin{eqnarray*}
\overline{X}&=&\frac{\esp\left[\int_{0}^{R_{1}}X\left(t\right)dt\right]}{\esp\left[R_{1}\right]},
\end{eqnarray*}
suponiendo que ambas esperanzas son finitas.

\begin{Note}
\begin{itemize}
\item[a)] Si el proceso regenerativo $X$ es positivo recurrente y tiene trayectorias muestrales no negativas, entonces la ecuaci\'on anterior es v\'alida.
\item[b)] Si $X$ es positivo recurrente regenerativo, podemos construir una \'unica versi\'on estacionaria de este proceso, $X_{e}=\left\{X_{e}\left(t\right)\right\}$, donde $X_{e}$ es un proceso estoc\'astico regenerativo y estrictamente estacionario, con distribuci\'on marginal distribuida como $X_{\infty}$
\end{itemize}
\end{Note}

%________________________________________________________________________
\subsection{Procesos Regenerativos}
%________________________________________________________________________

Para $\left\{X\left(t\right):t\geq0\right\}$ Proceso Estoc\'astico a tiempo continuo con estado de espacios $S$, que es un espacio m\'etrico, con trayectorias continuas por la derecha y con l\'imites por la izquierda c.s. Sea $N\left(t\right)$ un proceso de renovaci\'on en $\rea_{+}$ definido en el mismo espacio de probabilidad que $X\left(t\right)$, con tiempos de renovaci\'on $T$ y tiempos de inter-renovaci\'on $\xi_{n}=T_{n}-T_{n-1}$, con misma distribuci\'on $F$ de media finita $\mu$.



\begin{Def}
Para el proceso $\left\{\left(N\left(t\right),X\left(t\right)\right):t\geq0\right\}$, sus trayectoria muestrales en el intervalo de tiempo $\left[T_{n-1},T_{n}\right)$ est\'an descritas por
\begin{eqnarray*}
\zeta_{n}=\left(\xi_{n},\left\{X\left(T_{n-1}+t\right):0\leq t<\xi_{n}\right\}\right)
\end{eqnarray*}
Este $\zeta_{n}$ es el $n$-\'esimo segmento del proceso. El proceso es regenerativo sobre los tiempos $T_{n}$ si sus segmentos $\zeta_{n}$ son independientes e id\'enticamennte distribuidos.
\end{Def}


\begin{Obs}
Si $\tilde{X}\left(t\right)$ con espacio de estados $\tilde{S}$ es regenerativo sobre $T_{n}$, entonces $X\left(t\right)=f\left(\tilde{X}\left(t\right)\right)$ tambi\'en es regenerativo sobre $T_{n}$, para cualquier funci\'on $f:\tilde{S}\rightarrow S$.
\end{Obs}

\begin{Obs}
Los procesos regenerativos son crudamente regenerativos, pero no al rev\'es.
\end{Obs}

\begin{Def}[Definici\'on Cl\'asica]
Un proceso estoc\'astico $X=\left\{X\left(t\right):t\geq0\right\}$ es llamado regenerativo is existe una variable aleatoria $R_{1}>0$ tal que
\begin{itemize}
\item[i)] $\left\{X\left(t+R_{1}\right):t\geq0\right\}$ es independiente de $\left\{\left\{X\left(t\right):t<R_{1}\right\},\right\}$
\item[ii)] $\left\{X\left(t+R_{1}\right):t\geq0\right\}$ es estoc\'asticamente equivalente a $\left\{X\left(t\right):t>0\right\}$
\end{itemize}

Llamamos a $R_{1}$ tiempo de regeneraci\'on, y decimos que $X$ se regenera en este punto.
\end{Def}

$\left\{X\left(t+R_{1}\right)\right\}$ es regenerativo con tiempo de regeneraci\'on $R_{2}$, independiente de $R_{1}$ pero con la misma distribuci\'on que $R_{1}$. Procediendo de esta manera se obtiene una secuencia de variables aleatorias independientes e id\'enticamente distribuidas $\left\{R_{n}\right\}$ llamados longitudes de ciclo. Si definimos a $Z_{k}\equiv R_{1}+R_{2}+\cdots+R_{k}$, se tiene un proceso de renovaci\'on llamado proceso de renovaci\'on encajado para $X$.

\begin{Note}
Un proceso regenerativo con media de la longitud de ciclo finita es llamado positivo recurrente.
\end{Note}


\begin{Def}
Para $x$ fijo y para cada $t\geq0$, sea $I_{x}\left(t\right)=1$ si $X\left(t\right)\leq x$,  $I_{x}\left(t\right)=0$ en caso contrario, y def\'inanse los tiempos promedio
\begin{eqnarray*}
\overline{X}&=&lim_{t\rightarrow\infty}\frac{1}{t}\int_{0}^{\infty}X\left(u\right)du\\
\prob\left(X_{\infty}\leq x\right)&=&lim_{t\rightarrow\infty}\frac{1}{t}\int_{0}^{\infty}I_{x}\left(u\right)du,
\end{eqnarray*}
cuando estos l\'imites existan.
\end{Def}

Como consecuencia del teorema de Renovaci\'on-Recompensa, se tiene que el primer l\'imite  existe y es igual a la constante
\begin{eqnarray*}
\overline{X}&=&\frac{\esp\left[\int_{0}^{R_{1}}X\left(t\right)dt\right]}{\esp\left[R_{1}\right]},
\end{eqnarray*}
suponiendo que ambas esperanzas son finitas.

\begin{Note}
\begin{itemize}
\item[a)] Si el proceso regenerativo $X$ es positivo recurrente y tiene trayectorias muestrales no negativas, entonces la ecuaci\'on anterior es v\'alida.
\item[b)] Si $X$ es positivo recurrente regenerativo, podemos construir una \'unica versi\'on estacionaria de este proceso, $X_{e}=\left\{X_{e}\left(t\right)\right\}$, donde $X_{e}$ es un proceso estoc\'astico regenerativo y estrictamente estacionario, con distribuci\'on marginal distribuida como $X_{\infty}$
\end{itemize}
\end{Note}
%__________________________________________________________________________________________
\subsection{Procesos Regenerativos Estacionarios - Stidham \cite{Stidham}}
%__________________________________________________________________________________________


Un proceso estoc\'astico a tiempo continuo $\left\{V\left(t\right),t\geq0\right\}$ es un proceso regenerativo si existe una sucesi\'on de variables aleatorias independientes e id\'enticamente distribuidas $\left\{X_{1},X_{2},\ldots\right\}$, sucesi\'on de renovaci\'on, tal que para cualquier conjunto de Borel $A$, 

\begin{eqnarray*}
\prob\left\{V\left(t\right)\in A|X_{1}+X_{2}+\cdots+X_{R\left(t\right)}=s,\left\{V\left(\tau\right),\tau<s\right\}\right\}=\prob\left\{V\left(t-s\right)\in A|X_{1}>t-s\right\},
\end{eqnarray*}
para todo $0\leq s\leq t$, donde $R\left(t\right)=\max\left\{X_{1}+X_{2}+\cdots+X_{j}\leq t\right\}=$n\'umero de renovaciones ({\emph{puntos de regeneraci\'on}}) que ocurren en $\left[0,t\right]$. El intervalo $\left[0,X_{1}\right)$ es llamado {\emph{primer ciclo de regeneraci\'on}} de $\left\{V\left(t \right),t\geq0\right\}$, $\left[X_{1},X_{1}+X_{2}\right)$ el {\emph{segundo ciclo de regeneraci\'on}}, y as\'i sucesivamente.

Sea $X=X_{1}$ y sea $F$ la funci\'on de distrbuci\'on de $X$


\begin{Def}
Se define el proceso estacionario, $\left\{V^{*}\left(t\right),t\geq0\right\}$, para $\left\{V\left(t\right),t\geq0\right\}$ por

\begin{eqnarray*}
\prob\left\{V\left(t\right)\in A\right\}=\frac{1}{\esp\left[X\right]}\int_{0}^{\infty}\prob\left\{V\left(t+x\right)\in A|X>x\right\}\left(1-F\left(x\right)\right)dx,
\end{eqnarray*} 
para todo $t\geq0$ y todo conjunto de Borel $A$.
\end{Def}

\begin{Def}
Una distribuci\'on se dice que es {\emph{aritm\'etica}} si todos sus puntos de incremento son m\'ultiplos de la forma $0,\lambda, 2\lambda,\ldots$ para alguna $\lambda>0$ entera.
\end{Def}


\begin{Def}
Una modificaci\'on medible de un proceso $\left\{V\left(t\right),t\geq0\right\}$, es una versi\'on de este, $\left\{V\left(t,w\right)\right\}$ conjuntamente medible para $t\geq0$ y para $w\in S$, $S$ espacio de estados para $\left\{V\left(t\right),t\geq0\right\}$.
\end{Def}

\begin{Teo}
Sea $\left\{V\left(t\right),t\geq\right\}$ un proceso regenerativo no negativo con modificaci\'on medible. Sea $\esp\left[X\right]<\infty$. Entonces el proceso estacionario dado por la ecuaci\'on anterior est\'a bien definido y tiene funci\'on de distribuci\'on independiente de $t$, adem\'as
\begin{itemize}
\item[i)] \begin{eqnarray*}
\esp\left[V^{*}\left(0\right)\right]&=&\frac{\esp\left[\int_{0}^{X}V\left(s\right)ds\right]}{\esp\left[X\right]}\end{eqnarray*}
\item[ii)] Si $\esp\left[V^{*}\left(0\right)\right]<\infty$, equivalentemente, si $\esp\left[\int_{0}^{X}V\left(s\right)ds\right]<\infty$,entonces
\begin{eqnarray*}
\frac{\int_{0}^{t}V\left(s\right)ds}{t}\rightarrow\frac{\esp\left[\int_{0}^{X}V\left(s\right)ds\right]}{\esp\left[X\right]}
\end{eqnarray*}
con probabilidad 1 y en media, cuando $t\rightarrow\infty$.
\end{itemize}
\end{Teo}


%__________________________________________________________________________________________
\subsection{Procesos Regenerativos Estacionarios - Stidham \cite{Stidham}}
%__________________________________________________________________________________________


Un proceso estoc\'astico a tiempo continuo $\left\{V\left(t\right),t\geq0\right\}$ es un proceso regenerativo si existe una sucesi\'on de variables aleatorias independientes e id\'enticamente distribuidas $\left\{X_{1},X_{2},\ldots\right\}$, sucesi\'on de renovaci\'on, tal que para cualquier conjunto de Borel $A$, 

\begin{eqnarray*}
\prob\left\{V\left(t\right)\in A|X_{1}+X_{2}+\cdots+X_{R\left(t\right)}=s,\left\{V\left(\tau\right),\tau<s\right\}\right\}=\prob\left\{V\left(t-s\right)\in A|X_{1}>t-s\right\},
\end{eqnarray*}
para todo $0\leq s\leq t$, donde $R\left(t\right)=\max\left\{X_{1}+X_{2}+\cdots+X_{j}\leq t\right\}=$n\'umero de renovaciones ({\emph{puntos de regeneraci\'on}}) que ocurren en $\left[0,t\right]$. El intervalo $\left[0,X_{1}\right)$ es llamado {\emph{primer ciclo de regeneraci\'on}} de $\left\{V\left(t \right),t\geq0\right\}$, $\left[X_{1},X_{1}+X_{2}\right)$ el {\emph{segundo ciclo de regeneraci\'on}}, y as\'i sucesivamente.

Sea $X=X_{1}$ y sea $F$ la funci\'on de distrbuci\'on de $X$


\begin{Def}
Se define el proceso estacionario, $\left\{V^{*}\left(t\right),t\geq0\right\}$, para $\left\{V\left(t\right),t\geq0\right\}$ por

\begin{eqnarray*}
\prob\left\{V\left(t\right)\in A\right\}=\frac{1}{\esp\left[X\right]}\int_{0}^{\infty}\prob\left\{V\left(t+x\right)\in A|X>x\right\}\left(1-F\left(x\right)\right)dx,
\end{eqnarray*} 
para todo $t\geq0$ y todo conjunto de Borel $A$.
\end{Def}

\begin{Def}
Una distribuci\'on se dice que es {\emph{aritm\'etica}} si todos sus puntos de incremento son m\'ultiplos de la forma $0,\lambda, 2\lambda,\ldots$ para alguna $\lambda>0$ entera.
\end{Def}


\begin{Def}
Una modificaci\'on medible de un proceso $\left\{V\left(t\right),t\geq0\right\}$, es una versi\'on de este, $\left\{V\left(t,w\right)\right\}$ conjuntamente medible para $t\geq0$ y para $w\in S$, $S$ espacio de estados para $\left\{V\left(t\right),t\geq0\right\}$.
\end{Def}

\begin{Teo}
Sea $\left\{V\left(t\right),t\geq\right\}$ un proceso regenerativo no negativo con modificaci\'on medible. Sea $\esp\left[X\right]<\infty$. Entonces el proceso estacionario dado por la ecuaci\'on anterior est\'a bien definido y tiene funci\'on de distribuci\'on independiente de $t$, adem\'as
\begin{itemize}
\item[i)] \begin{eqnarray*}
\esp\left[V^{*}\left(0\right)\right]&=&\frac{\esp\left[\int_{0}^{X}V\left(s\right)ds\right]}{\esp\left[X\right]}\end{eqnarray*}
\item[ii)] Si $\esp\left[V^{*}\left(0\right)\right]<\infty$, equivalentemente, si $\esp\left[\int_{0}^{X}V\left(s\right)ds\right]<\infty$,entonces
\begin{eqnarray*}
\frac{\int_{0}^{t}V\left(s\right)ds}{t}\rightarrow\frac{\esp\left[\int_{0}^{X}V\left(s\right)ds\right]}{\esp\left[X\right]}
\end{eqnarray*}
con probabilidad 1 y en media, cuando $t\rightarrow\infty$.
\end{itemize}
\end{Teo}
%
%___________________________________________________________________________________________
%\vspace{5.5cm}
%\chapter{Cadenas de Markov estacionarias}
%\vspace{-1.0cm}
%___________________________________________________________________________________________
%
\subsection{Propiedades de los Procesos de Renovaci\'on}
%___________________________________________________________________________________________
%

Los tiempos $T_{n}$ est\'an relacionados con los conteos de $N\left(t\right)$ por

\begin{eqnarray*}
\left\{N\left(t\right)\geq n\right\}&=&\left\{T_{n}\leq t\right\}\\
T_{N\left(t\right)}\leq &t&<T_{N\left(t\right)+1},
\end{eqnarray*}

adem\'as $N\left(T_{n}\right)=n$, y 

\begin{eqnarray*}
N\left(t\right)=\max\left\{n:T_{n}\leq t\right\}=\min\left\{n:T_{n+1}>t\right\}
\end{eqnarray*}

Por propiedades de la convoluci\'on se sabe que

\begin{eqnarray*}
P\left\{T_{n}\leq t\right\}=F^{n\star}\left(t\right)
\end{eqnarray*}
que es la $n$-\'esima convoluci\'on de $F$. Entonces 

\begin{eqnarray*}
\left\{N\left(t\right)\geq n\right\}&=&\left\{T_{n}\leq t\right\}\\
P\left\{N\left(t\right)\leq n\right\}&=&1-F^{\left(n+1\right)\star}\left(t\right)
\end{eqnarray*}

Adem\'as usando el hecho de que $\esp\left[N\left(t\right)\right]=\sum_{n=1}^{\infty}P\left\{N\left(t\right)\geq n\right\}$
se tiene que

\begin{eqnarray*}
\esp\left[N\left(t\right)\right]=\sum_{n=1}^{\infty}F^{n\star}\left(t\right)
\end{eqnarray*}

\begin{Prop}
Para cada $t\geq0$, la funci\'on generadora de momentos $\esp\left[e^{\alpha N\left(t\right)}\right]$ existe para alguna $\alpha$ en una vecindad del 0, y de aqu\'i que $\esp\left[N\left(t\right)^{m}\right]<\infty$, para $m\geq1$.
\end{Prop}


\begin{Note}
Si el primer tiempo de renovaci\'on $\xi_{1}$ no tiene la misma distribuci\'on que el resto de las $\xi_{n}$, para $n\geq2$, a $N\left(t\right)$ se le llama Proceso de Renovaci\'on retardado, donde si $\xi$ tiene distribuci\'on $G$, entonces el tiempo $T_{n}$ de la $n$-\'esima renovaci\'on tiene distribuci\'on $G\star F^{\left(n-1\right)\star}\left(t\right)$
\end{Note}


\begin{Teo}
Para una constante $\mu\leq\infty$ ( o variable aleatoria), las siguientes expresiones son equivalentes:

\begin{eqnarray}
lim_{n\rightarrow\infty}n^{-1}T_{n}&=&\mu,\textrm{ c.s.}\\
lim_{t\rightarrow\infty}t^{-1}N\left(t\right)&=&1/\mu,\textrm{ c.s.}
\end{eqnarray}
\end{Teo}


Es decir, $T_{n}$ satisface la Ley Fuerte de los Grandes N\'umeros s\'i y s\'olo s\'i $N\left/t\right)$ la cumple.


\begin{Coro}[Ley Fuerte de los Grandes N\'umeros para Procesos de Renovaci\'on]
Si $N\left(t\right)$ es un proceso de renovaci\'on cuyos tiempos de inter-renovaci\'on tienen media $\mu\leq\infty$, entonces
\begin{eqnarray}
t^{-1}N\left(t\right)\rightarrow 1/\mu,\textrm{ c.s. cuando }t\rightarrow\infty.
\end{eqnarray}

\end{Coro}


Considerar el proceso estoc\'astico de valores reales $\left\{Z\left(t\right):t\geq0\right\}$ en el mismo espacio de probabilidad que $N\left(t\right)$

\begin{Def}
Para el proceso $\left\{Z\left(t\right):t\geq0\right\}$ se define la fluctuaci\'on m\'axima de $Z\left(t\right)$ en el intervalo $\left(T_{n-1},T_{n}\right]$:
\begin{eqnarray*}
M_{n}=\sup_{T_{n-1}<t\leq T_{n}}|Z\left(t\right)-Z\left(T_{n-1}\right)|
\end{eqnarray*}
\end{Def}

\begin{Teo}
Sup\'ongase que $n^{-1}T_{n}\rightarrow\mu$ c.s. cuando $n\rightarrow\infty$, donde $\mu\leq\infty$ es una constante o variable aleatoria. Sea $a$ una constante o variable aleatoria que puede ser infinita cuando $\mu$ es finita, y considere las expresiones l\'imite:
\begin{eqnarray}
lim_{n\rightarrow\infty}n^{-1}Z\left(T_{n}\right)&=&a,\textrm{ c.s.}\\
lim_{t\rightarrow\infty}t^{-1}Z\left(t\right)&=&a/\mu,\textrm{ c.s.}
\end{eqnarray}
La segunda expresi\'on implica la primera. Conversamente, la primera implica la segunda si el proceso $Z\left(t\right)$ es creciente, o si $lim_{n\rightarrow\infty}n^{-1}M_{n}=0$ c.s.
\end{Teo}

\begin{Coro}
Si $N\left(t\right)$ es un proceso de renovaci\'on, y $\left(Z\left(T_{n}\right)-Z\left(T_{n-1}\right),M_{n}\right)$, para $n\geq1$, son variables aleatorias independientes e id\'enticamente distribuidas con media finita, entonces,
\begin{eqnarray}
lim_{t\rightarrow\infty}t^{-1}Z\left(t\right)\rightarrow\frac{\esp\left[Z\left(T_{1}\right)-Z\left(T_{0}\right)\right]}{\esp\left[T_{1}\right]},\textrm{ c.s. cuando  }t\rightarrow\infty.
\end{eqnarray}
\end{Coro}


%___________________________________________________________________________________________
%
%\subsection{Propiedades de los Procesos de Renovaci\'on}
%___________________________________________________________________________________________
%

Los tiempos $T_{n}$ est\'an relacionados con los conteos de $N\left(t\right)$ por

\begin{eqnarray*}
\left\{N\left(t\right)\geq n\right\}&=&\left\{T_{n}\leq t\right\}\\
T_{N\left(t\right)}\leq &t&<T_{N\left(t\right)+1},
\end{eqnarray*}

adem\'as $N\left(T_{n}\right)=n$, y 

\begin{eqnarray*}
N\left(t\right)=\max\left\{n:T_{n}\leq t\right\}=\min\left\{n:T_{n+1}>t\right\}
\end{eqnarray*}

Por propiedades de la convoluci\'on se sabe que

\begin{eqnarray*}
P\left\{T_{n}\leq t\right\}=F^{n\star}\left(t\right)
\end{eqnarray*}
que es la $n$-\'esima convoluci\'on de $F$. Entonces 

\begin{eqnarray*}
\left\{N\left(t\right)\geq n\right\}&=&\left\{T_{n}\leq t\right\}\\
P\left\{N\left(t\right)\leq n\right\}&=&1-F^{\left(n+1\right)\star}\left(t\right)
\end{eqnarray*}

Adem\'as usando el hecho de que $\esp\left[N\left(t\right)\right]=\sum_{n=1}^{\infty}P\left\{N\left(t\right)\geq n\right\}$
se tiene que

\begin{eqnarray*}
\esp\left[N\left(t\right)\right]=\sum_{n=1}^{\infty}F^{n\star}\left(t\right)
\end{eqnarray*}

\begin{Prop}
Para cada $t\geq0$, la funci\'on generadora de momentos $\esp\left[e^{\alpha N\left(t\right)}\right]$ existe para alguna $\alpha$ en una vecindad del 0, y de aqu\'i que $\esp\left[N\left(t\right)^{m}\right]<\infty$, para $m\geq1$.
\end{Prop}


\begin{Note}
Si el primer tiempo de renovaci\'on $\xi_{1}$ no tiene la misma distribuci\'on que el resto de las $\xi_{n}$, para $n\geq2$, a $N\left(t\right)$ se le llama Proceso de Renovaci\'on retardado, donde si $\xi$ tiene distribuci\'on $G$, entonces el tiempo $T_{n}$ de la $n$-\'esima renovaci\'on tiene distribuci\'on $G\star F^{\left(n-1\right)\star}\left(t\right)$
\end{Note}


\begin{Teo}
Para una constante $\mu\leq\infty$ ( o variable aleatoria), las siguientes expresiones son equivalentes:

\begin{eqnarray}
lim_{n\rightarrow\infty}n^{-1}T_{n}&=&\mu,\textrm{ c.s.}\\
lim_{t\rightarrow\infty}t^{-1}N\left(t\right)&=&1/\mu,\textrm{ c.s.}
\end{eqnarray}
\end{Teo}


Es decir, $T_{n}$ satisface la Ley Fuerte de los Grandes N\'umeros s\'i y s\'olo s\'i $N\left/t\right)$ la cumple.


\begin{Coro}[Ley Fuerte de los Grandes N\'umeros para Procesos de Renovaci\'on]
Si $N\left(t\right)$ es un proceso de renovaci\'on cuyos tiempos de inter-renovaci\'on tienen media $\mu\leq\infty$, entonces
\begin{eqnarray}
t^{-1}N\left(t\right)\rightarrow 1/\mu,\textrm{ c.s. cuando }t\rightarrow\infty.
\end{eqnarray}

\end{Coro}


Considerar el proceso estoc\'astico de valores reales $\left\{Z\left(t\right):t\geq0\right\}$ en el mismo espacio de probabilidad que $N\left(t\right)$

\begin{Def}
Para el proceso $\left\{Z\left(t\right):t\geq0\right\}$ se define la fluctuaci\'on m\'axima de $Z\left(t\right)$ en el intervalo $\left(T_{n-1},T_{n}\right]$:
\begin{eqnarray*}
M_{n}=\sup_{T_{n-1}<t\leq T_{n}}|Z\left(t\right)-Z\left(T_{n-1}\right)|
\end{eqnarray*}
\end{Def}

\begin{Teo}
Sup\'ongase que $n^{-1}T_{n}\rightarrow\mu$ c.s. cuando $n\rightarrow\infty$, donde $\mu\leq\infty$ es una constante o variable aleatoria. Sea $a$ una constante o variable aleatoria que puede ser infinita cuando $\mu$ es finita, y considere las expresiones l\'imite:
\begin{eqnarray}
lim_{n\rightarrow\infty}n^{-1}Z\left(T_{n}\right)&=&a,\textrm{ c.s.}\\
lim_{t\rightarrow\infty}t^{-1}Z\left(t\right)&=&a/\mu,\textrm{ c.s.}
\end{eqnarray}
La segunda expresi\'on implica la primera. Conversamente, la primera implica la segunda si el proceso $Z\left(t\right)$ es creciente, o si $lim_{n\rightarrow\infty}n^{-1}M_{n}=0$ c.s.
\end{Teo}

\begin{Coro}
Si $N\left(t\right)$ es un proceso de renovaci\'on, y $\left(Z\left(T_{n}\right)-Z\left(T_{n-1}\right),M_{n}\right)$, para $n\geq1$, son variables aleatorias independientes e id\'enticamente distribuidas con media finita, entonces,
\begin{eqnarray}
lim_{t\rightarrow\infty}t^{-1}Z\left(t\right)\rightarrow\frac{\esp\left[Z\left(T_{1}\right)-Z\left(T_{0}\right)\right]}{\esp\left[T_{1}\right]},\textrm{ c.s. cuando  }t\rightarrow\infty.
\end{eqnarray}
\end{Coro}

%___________________________________________________________________________________________
%
%\subsection{Propiedades de los Procesos de Renovaci\'on}
%___________________________________________________________________________________________
%

Los tiempos $T_{n}$ est\'an relacionados con los conteos de $N\left(t\right)$ por

\begin{eqnarray*}
\left\{N\left(t\right)\geq n\right\}&=&\left\{T_{n}\leq t\right\}\\
T_{N\left(t\right)}\leq &t&<T_{N\left(t\right)+1},
\end{eqnarray*}

adem\'as $N\left(T_{n}\right)=n$, y 

\begin{eqnarray*}
N\left(t\right)=\max\left\{n:T_{n}\leq t\right\}=\min\left\{n:T_{n+1}>t\right\}
\end{eqnarray*}

Por propiedades de la convoluci\'on se sabe que

\begin{eqnarray*}
P\left\{T_{n}\leq t\right\}=F^{n\star}\left(t\right)
\end{eqnarray*}
que es la $n$-\'esima convoluci\'on de $F$. Entonces 

\begin{eqnarray*}
\left\{N\left(t\right)\geq n\right\}&=&\left\{T_{n}\leq t\right\}\\
P\left\{N\left(t\right)\leq n\right\}&=&1-F^{\left(n+1\right)\star}\left(t\right)
\end{eqnarray*}

Adem\'as usando el hecho de que $\esp\left[N\left(t\right)\right]=\sum_{n=1}^{\infty}P\left\{N\left(t\right)\geq n\right\}$
se tiene que

\begin{eqnarray*}
\esp\left[N\left(t\right)\right]=\sum_{n=1}^{\infty}F^{n\star}\left(t\right)
\end{eqnarray*}

\begin{Prop}
Para cada $t\geq0$, la funci\'on generadora de momentos $\esp\left[e^{\alpha N\left(t\right)}\right]$ existe para alguna $\alpha$ en una vecindad del 0, y de aqu\'i que $\esp\left[N\left(t\right)^{m}\right]<\infty$, para $m\geq1$.
\end{Prop}


\begin{Note}
Si el primer tiempo de renovaci\'on $\xi_{1}$ no tiene la misma distribuci\'on que el resto de las $\xi_{n}$, para $n\geq2$, a $N\left(t\right)$ se le llama Proceso de Renovaci\'on retardado, donde si $\xi$ tiene distribuci\'on $G$, entonces el tiempo $T_{n}$ de la $n$-\'esima renovaci\'on tiene distribuci\'on $G\star F^{\left(n-1\right)\star}\left(t\right)$
\end{Note}


\begin{Teo}
Para una constante $\mu\leq\infty$ ( o variable aleatoria), las siguientes expresiones son equivalentes:

\begin{eqnarray}
lim_{n\rightarrow\infty}n^{-1}T_{n}&=&\mu,\textrm{ c.s.}\\
lim_{t\rightarrow\infty}t^{-1}N\left(t\right)&=&1/\mu,\textrm{ c.s.}
\end{eqnarray}
\end{Teo}


Es decir, $T_{n}$ satisface la Ley Fuerte de los Grandes N\'umeros s\'i y s\'olo s\'i $N\left/t\right)$ la cumple.


\begin{Coro}[Ley Fuerte de los Grandes N\'umeros para Procesos de Renovaci\'on]
Si $N\left(t\right)$ es un proceso de renovaci\'on cuyos tiempos de inter-renovaci\'on tienen media $\mu\leq\infty$, entonces
\begin{eqnarray}
t^{-1}N\left(t\right)\rightarrow 1/\mu,\textrm{ c.s. cuando }t\rightarrow\infty.
\end{eqnarray}

\end{Coro}


Considerar el proceso estoc\'astico de valores reales $\left\{Z\left(t\right):t\geq0\right\}$ en el mismo espacio de probabilidad que $N\left(t\right)$

\begin{Def}
Para el proceso $\left\{Z\left(t\right):t\geq0\right\}$ se define la fluctuaci\'on m\'axima de $Z\left(t\right)$ en el intervalo $\left(T_{n-1},T_{n}\right]$:
\begin{eqnarray*}
M_{n}=\sup_{T_{n-1}<t\leq T_{n}}|Z\left(t\right)-Z\left(T_{n-1}\right)|
\end{eqnarray*}
\end{Def}

\begin{Teo}
Sup\'ongase que $n^{-1}T_{n}\rightarrow\mu$ c.s. cuando $n\rightarrow\infty$, donde $\mu\leq\infty$ es una constante o variable aleatoria. Sea $a$ una constante o variable aleatoria que puede ser infinita cuando $\mu$ es finita, y considere las expresiones l\'imite:
\begin{eqnarray}
lim_{n\rightarrow\infty}n^{-1}Z\left(T_{n}\right)&=&a,\textrm{ c.s.}\\
lim_{t\rightarrow\infty}t^{-1}Z\left(t\right)&=&a/\mu,\textrm{ c.s.}
\end{eqnarray}
La segunda expresi\'on implica la primera. Conversamente, la primera implica la segunda si el proceso $Z\left(t\right)$ es creciente, o si $lim_{n\rightarrow\infty}n^{-1}M_{n}=0$ c.s.
\end{Teo}

\begin{Coro}
Si $N\left(t\right)$ es un proceso de renovaci\'on, y $\left(Z\left(T_{n}\right)-Z\left(T_{n-1}\right),M_{n}\right)$, para $n\geq1$, son variables aleatorias independientes e id\'enticamente distribuidas con media finita, entonces,
\begin{eqnarray}
lim_{t\rightarrow\infty}t^{-1}Z\left(t\right)\rightarrow\frac{\esp\left[Z\left(T_{1}\right)-Z\left(T_{0}\right)\right]}{\esp\left[T_{1}\right]},\textrm{ c.s. cuando  }t\rightarrow\infty.
\end{eqnarray}
\end{Coro}



%___________________________________________________________________________________________
%
\subsection{Propiedades de los Procesos de Renovaci\'on}
%___________________________________________________________________________________________
%

Los tiempos $T_{n}$ est\'an relacionados con los conteos de $N\left(t\right)$ por

\begin{eqnarray*}
\left\{N\left(t\right)\geq n\right\}&=&\left\{T_{n}\leq t\right\}\\
T_{N\left(t\right)}\leq &t&<T_{N\left(t\right)+1},
\end{eqnarray*}

adem\'as $N\left(T_{n}\right)=n$, y 

\begin{eqnarray*}
N\left(t\right)=\max\left\{n:T_{n}\leq t\right\}=\min\left\{n:T_{n+1}>t\right\}
\end{eqnarray*}

Por propiedades de la convoluci\'on se sabe que

\begin{eqnarray*}
P\left\{T_{n}\leq t\right\}=F^{n\star}\left(t\right)
\end{eqnarray*}
que es la $n$-\'esima convoluci\'on de $F$. Entonces 

\begin{eqnarray*}
\left\{N\left(t\right)\geq n\right\}&=&\left\{T_{n}\leq t\right\}\\
P\left\{N\left(t\right)\leq n\right\}&=&1-F^{\left(n+1\right)\star}\left(t\right)
\end{eqnarray*}

Adem\'as usando el hecho de que $\esp\left[N\left(t\right)\right]=\sum_{n=1}^{\infty}P\left\{N\left(t\right)\geq n\right\}$
se tiene que

\begin{eqnarray*}
\esp\left[N\left(t\right)\right]=\sum_{n=1}^{\infty}F^{n\star}\left(t\right)
\end{eqnarray*}

\begin{Prop}
Para cada $t\geq0$, la funci\'on generadora de momentos $\esp\left[e^{\alpha N\left(t\right)}\right]$ existe para alguna $\alpha$ en una vecindad del 0, y de aqu\'i que $\esp\left[N\left(t\right)^{m}\right]<\infty$, para $m\geq1$.
\end{Prop}


\begin{Note}
Si el primer tiempo de renovaci\'on $\xi_{1}$ no tiene la misma distribuci\'on que el resto de las $\xi_{n}$, para $n\geq2$, a $N\left(t\right)$ se le llama Proceso de Renovaci\'on retardado, donde si $\xi$ tiene distribuci\'on $G$, entonces el tiempo $T_{n}$ de la $n$-\'esima renovaci\'on tiene distribuci\'on $G\star F^{\left(n-1\right)\star}\left(t\right)$
\end{Note}


\begin{Teo}
Para una constante $\mu\leq\infty$ ( o variable aleatoria), las siguientes expresiones son equivalentes:

\begin{eqnarray}
lim_{n\rightarrow\infty}n^{-1}T_{n}&=&\mu,\textrm{ c.s.}\\
lim_{t\rightarrow\infty}t^{-1}N\left(t\right)&=&1/\mu,\textrm{ c.s.}
\end{eqnarray}
\end{Teo}


Es decir, $T_{n}$ satisface la Ley Fuerte de los Grandes N\'umeros s\'i y s\'olo s\'i $N\left/t\right)$ la cumple.


\begin{Coro}[Ley Fuerte de los Grandes N\'umeros para Procesos de Renovaci\'on]
Si $N\left(t\right)$ es un proceso de renovaci\'on cuyos tiempos de inter-renovaci\'on tienen media $\mu\leq\infty$, entonces
\begin{eqnarray}
t^{-1}N\left(t\right)\rightarrow 1/\mu,\textrm{ c.s. cuando }t\rightarrow\infty.
\end{eqnarray}

\end{Coro}


Considerar el proceso estoc\'astico de valores reales $\left\{Z\left(t\right):t\geq0\right\}$ en el mismo espacio de probabilidad que $N\left(t\right)$

\begin{Def}
Para el proceso $\left\{Z\left(t\right):t\geq0\right\}$ se define la fluctuaci\'on m\'axima de $Z\left(t\right)$ en el intervalo $\left(T_{n-1},T_{n}\right]$:
\begin{eqnarray*}
M_{n}=\sup_{T_{n-1}<t\leq T_{n}}|Z\left(t\right)-Z\left(T_{n-1}\right)|
\end{eqnarray*}
\end{Def}

\begin{Teo}
Sup\'ongase que $n^{-1}T_{n}\rightarrow\mu$ c.s. cuando $n\rightarrow\infty$, donde $\mu\leq\infty$ es una constante o variable aleatoria. Sea $a$ una constante o variable aleatoria que puede ser infinita cuando $\mu$ es finita, y considere las expresiones l\'imite:
\begin{eqnarray}
lim_{n\rightarrow\infty}n^{-1}Z\left(T_{n}\right)&=&a,\textrm{ c.s.}\\
lim_{t\rightarrow\infty}t^{-1}Z\left(t\right)&=&a/\mu,\textrm{ c.s.}
\end{eqnarray}
La segunda expresi\'on implica la primera. Conversamente, la primera implica la segunda si el proceso $Z\left(t\right)$ es creciente, o si $lim_{n\rightarrow\infty}n^{-1}M_{n}=0$ c.s.
\end{Teo}

\begin{Coro}
Si $N\left(t\right)$ es un proceso de renovaci\'on, y $\left(Z\left(T_{n}\right)-Z\left(T_{n-1}\right),M_{n}\right)$, para $n\geq1$, son variables aleatorias independientes e id\'enticamente distribuidas con media finita, entonces,
\begin{eqnarray}
lim_{t\rightarrow\infty}t^{-1}Z\left(t\right)\rightarrow\frac{\esp\left[Z\left(T_{1}\right)-Z\left(T_{0}\right)\right]}{\esp\left[T_{1}\right]},\textrm{ c.s. cuando  }t\rightarrow\infty.
\end{eqnarray}
\end{Coro}




%__________________________________________________________________________________________
\subsection{Procesos Regenerativos Estacionarios - Stidham \cite{Stidham}}
%__________________________________________________________________________________________


Un proceso estoc\'astico a tiempo continuo $\left\{V\left(t\right),t\geq0\right\}$ es un proceso regenerativo si existe una sucesi\'on de variables aleatorias independientes e id\'enticamente distribuidas $\left\{X_{1},X_{2},\ldots\right\}$, sucesi\'on de renovaci\'on, tal que para cualquier conjunto de Borel $A$, 

\begin{eqnarray*}
\prob\left\{V\left(t\right)\in A|X_{1}+X_{2}+\cdots+X_{R\left(t\right)}=s,\left\{V\left(\tau\right),\tau<s\right\}\right\}=\prob\left\{V\left(t-s\right)\in A|X_{1}>t-s\right\},
\end{eqnarray*}
para todo $0\leq s\leq t$, donde $R\left(t\right)=\max\left\{X_{1}+X_{2}+\cdots+X_{j}\leq t\right\}=$n\'umero de renovaciones ({\emph{puntos de regeneraci\'on}}) que ocurren en $\left[0,t\right]$. El intervalo $\left[0,X_{1}\right)$ es llamado {\emph{primer ciclo de regeneraci\'on}} de $\left\{V\left(t \right),t\geq0\right\}$, $\left[X_{1},X_{1}+X_{2}\right)$ el {\emph{segundo ciclo de regeneraci\'on}}, y as\'i sucesivamente.

Sea $X=X_{1}$ y sea $F$ la funci\'on de distrbuci\'on de $X$


\begin{Def}
Se define el proceso estacionario, $\left\{V^{*}\left(t\right),t\geq0\right\}$, para $\left\{V\left(t\right),t\geq0\right\}$ por

\begin{eqnarray*}
\prob\left\{V\left(t\right)\in A\right\}=\frac{1}{\esp\left[X\right]}\int_{0}^{\infty}\prob\left\{V\left(t+x\right)\in A|X>x\right\}\left(1-F\left(x\right)\right)dx,
\end{eqnarray*} 
para todo $t\geq0$ y todo conjunto de Borel $A$.
\end{Def}

\begin{Def}
Una distribuci\'on se dice que es {\emph{aritm\'etica}} si todos sus puntos de incremento son m\'ultiplos de la forma $0,\lambda, 2\lambda,\ldots$ para alguna $\lambda>0$ entera.
\end{Def}


\begin{Def}
Una modificaci\'on medible de un proceso $\left\{V\left(t\right),t\geq0\right\}$, es una versi\'on de este, $\left\{V\left(t,w\right)\right\}$ conjuntamente medible para $t\geq0$ y para $w\in S$, $S$ espacio de estados para $\left\{V\left(t\right),t\geq0\right\}$.
\end{Def}

\begin{Teo}
Sea $\left\{V\left(t\right),t\geq\right\}$ un proceso regenerativo no negativo con modificaci\'on medible. Sea $\esp\left[X\right]<\infty$. Entonces el proceso estacionario dado por la ecuaci\'on anterior est\'a bien definido y tiene funci\'on de distribuci\'on independiente de $t$, adem\'as
\begin{itemize}
\item[i)] \begin{eqnarray*}
\esp\left[V^{*}\left(0\right)\right]&=&\frac{\esp\left[\int_{0}^{X}V\left(s\right)ds\right]}{\esp\left[X\right]}\end{eqnarray*}
\item[ii)] Si $\esp\left[V^{*}\left(0\right)\right]<\infty$, equivalentemente, si $\esp\left[\int_{0}^{X}V\left(s\right)ds\right]<\infty$,entonces
\begin{eqnarray*}
\frac{\int_{0}^{t}V\left(s\right)ds}{t}\rightarrow\frac{\esp\left[\int_{0}^{X}V\left(s\right)ds\right]}{\esp\left[X\right]}
\end{eqnarray*}
con probabilidad 1 y en media, cuando $t\rightarrow\infty$.
\end{itemize}
\end{Teo}

%______________________________________________________________________
\subsection{Procesos de Renovaci\'on}
%______________________________________________________________________

\begin{Def}\label{Def.Tn}
Sean $0\leq T_{1}\leq T_{2}\leq \ldots$ son tiempos aleatorios infinitos en los cuales ocurren ciertos eventos. El n\'umero de tiempos $T_{n}$ en el intervalo $\left[0,t\right)$ es

\begin{eqnarray}
N\left(t\right)=\sum_{n=1}^{\infty}\indora\left(T_{n}\leq t\right),
\end{eqnarray}
para $t\geq0$.
\end{Def}

Si se consideran los puntos $T_{n}$ como elementos de $\rea_{+}$, y $N\left(t\right)$ es el n\'umero de puntos en $\rea$. El proceso denotado por $\left\{N\left(t\right):t\geq0\right\}$, denotado por $N\left(t\right)$, es un proceso puntual en $\rea_{+}$. Los $T_{n}$ son los tiempos de ocurrencia, el proceso puntual $N\left(t\right)$ es simple si su n\'umero de ocurrencias son distintas: $0<T_{1}<T_{2}<\ldots$ casi seguramente.

\begin{Def}
Un proceso puntual $N\left(t\right)$ es un proceso de renovaci\'on si los tiempos de interocurrencia $\xi_{n}=T_{n}-T_{n-1}$, para $n\geq1$, son independientes e identicamente distribuidos con distribuci\'on $F$, donde $F\left(0\right)=0$ y $T_{0}=0$. Los $T_{n}$ son llamados tiempos de renovaci\'on, referente a la independencia o renovaci\'on de la informaci\'on estoc\'astica en estos tiempos. Los $\xi_{n}$ son los tiempos de inter-renovaci\'on, y $N\left(t\right)$ es el n\'umero de renovaciones en el intervalo $\left[0,t\right)$
\end{Def}


\begin{Note}
Para definir un proceso de renovaci\'on para cualquier contexto, solamente hay que especificar una distribuci\'on $F$, con $F\left(0\right)=0$, para los tiempos de inter-renovaci\'on. La funci\'on $F$ en turno degune las otra variables aleatorias. De manera formal, existe un espacio de probabilidad y una sucesi\'on de variables aleatorias $\xi_{1},\xi_{2},\ldots$ definidas en este con distribuci\'on $F$. Entonces las otras cantidades son $T_{n}=\sum_{k=1}^{n}\xi_{k}$ y $N\left(t\right)=\sum_{n=1}^{\infty}\indora\left(T_{n}\leq t\right)$, donde $T_{n}\rightarrow\infty$ casi seguramente por la Ley Fuerte de los Grandes Números.
\end{Note}

%___________________________________________________________________________________________
%
\subsection{Teorema Principal de Renovaci\'on}
%___________________________________________________________________________________________
%

\begin{Note} Una funci\'on $h:\rea_{+}\rightarrow\rea$ es Directamente Riemann Integrable en los siguientes casos:
\begin{itemize}
\item[a)] $h\left(t\right)\geq0$ es decreciente y Riemann Integrable.
\item[b)] $h$ es continua excepto posiblemente en un conjunto de Lebesgue de medida 0, y $|h\left(t\right)|\leq b\left(t\right)$, donde $b$ es DRI.
\end{itemize}
\end{Note}

\begin{Teo}[Teorema Principal de Renovaci\'on]
Si $F$ es no aritm\'etica y $h\left(t\right)$ es Directamente Riemann Integrable (DRI), entonces

\begin{eqnarray*}
lim_{t\rightarrow\infty}U\star h=\frac{1}{\mu}\int_{\rea_{+}}h\left(s\right)ds.
\end{eqnarray*}
\end{Teo}

\begin{Prop}
Cualquier funci\'on $H\left(t\right)$ acotada en intervalos finitos y que es 0 para $t<0$ puede expresarse como
\begin{eqnarray*}
H\left(t\right)=U\star h\left(t\right)\textrm{,  donde }h\left(t\right)=H\left(t\right)-F\star H\left(t\right)
\end{eqnarray*}
\end{Prop}

\begin{Def}
Un proceso estoc\'astico $X\left(t\right)$ es crudamente regenerativo en un tiempo aleatorio positivo $T$ si
\begin{eqnarray*}
\esp\left[X\left(T+t\right)|T\right]=\esp\left[X\left(t\right)\right]\textrm{, para }t\geq0,\end{eqnarray*}
y con las esperanzas anteriores finitas.
\end{Def}

\begin{Prop}
Sup\'ongase que $X\left(t\right)$ es un proceso crudamente regenerativo en $T$, que tiene distribuci\'on $F$. Si $\esp\left[X\left(t\right)\right]$ es acotado en intervalos finitos, entonces
\begin{eqnarray*}
\esp\left[X\left(t\right)\right]=U\star h\left(t\right)\textrm{,  donde }h\left(t\right)=\esp\left[X\left(t\right)\indora\left(T>t\right)\right].
\end{eqnarray*}
\end{Prop}

\begin{Teo}[Regeneraci\'on Cruda]
Sup\'ongase que $X\left(t\right)$ es un proceso con valores positivo crudamente regenerativo en $T$, y def\'inase $M=\sup\left\{|X\left(t\right)|:t\leq T\right\}$. Si $T$ es no aritm\'etico y $M$ y $MT$ tienen media finita, entonces
\begin{eqnarray*}
lim_{t\rightarrow\infty}\esp\left[X\left(t\right)\right]=\frac{1}{\mu}\int_{\rea_{+}}h\left(s\right)ds,
\end{eqnarray*}
donde $h\left(t\right)=\esp\left[X\left(t\right)\indora\left(T>t\right)\right]$.
\end{Teo}



%___________________________________________________________________________________________
%
\subsection{Funci\'on de Renovaci\'on}
%___________________________________________________________________________________________
%


\begin{Def}
Sea $h\left(t\right)$ funci\'on de valores reales en $\rea$ acotada en intervalos finitos e igual a cero para $t<0$ La ecuaci\'on de renovaci\'on para $h\left(t\right)$ y la distribuci\'on $F$ es

\begin{eqnarray}\label{Ec.Renovacion}
H\left(t\right)=h\left(t\right)+\int_{\left[0,t\right]}H\left(t-s\right)dF\left(s\right)\textrm{,    }t\geq0,
\end{eqnarray}
donde $H\left(t\right)$ es una funci\'on de valores reales. Esto es $H=h+F\star H$. Decimos que $H\left(t\right)$ es soluci\'on de esta ecuaci\'on si satisface la ecuaci\'on, y es acotada en intervalos finitos e iguales a cero para $t<0$.
\end{Def}

\begin{Prop}
La funci\'on $U\star h\left(t\right)$ es la \'unica soluci\'on de la ecuaci\'on de renovaci\'on (\ref{Ec.Renovacion}).
\end{Prop}

\begin{Teo}[Teorema Renovaci\'on Elemental]
\begin{eqnarray*}
t^{-1}U\left(t\right)\rightarrow 1/\mu\textrm{,    cuando }t\rightarrow\infty.
\end{eqnarray*}
\end{Teo}

%___________________________________________________________________________________________
%
\subsection{Propiedades de los Procesos de Renovaci\'on}
%___________________________________________________________________________________________
%

Los tiempos $T_{n}$ est\'an relacionados con los conteos de $N\left(t\right)$ por

\begin{eqnarray*}
\left\{N\left(t\right)\geq n\right\}&=&\left\{T_{n}\leq t\right\}\\
T_{N\left(t\right)}\leq &t&<T_{N\left(t\right)+1},
\end{eqnarray*}

adem\'as $N\left(T_{n}\right)=n$, y 

\begin{eqnarray*}
N\left(t\right)=\max\left\{n:T_{n}\leq t\right\}=\min\left\{n:T_{n+1}>t\right\}
\end{eqnarray*}

Por propiedades de la convoluci\'on se sabe que

\begin{eqnarray*}
P\left\{T_{n}\leq t\right\}=F^{n\star}\left(t\right)
\end{eqnarray*}
que es la $n$-\'esima convoluci\'on de $F$. Entonces 

\begin{eqnarray*}
\left\{N\left(t\right)\geq n\right\}&=&\left\{T_{n}\leq t\right\}\\
P\left\{N\left(t\right)\leq n\right\}&=&1-F^{\left(n+1\right)\star}\left(t\right)
\end{eqnarray*}

Adem\'as usando el hecho de que $\esp\left[N\left(t\right)\right]=\sum_{n=1}^{\infty}P\left\{N\left(t\right)\geq n\right\}$
se tiene que

\begin{eqnarray*}
\esp\left[N\left(t\right)\right]=\sum_{n=1}^{\infty}F^{n\star}\left(t\right)
\end{eqnarray*}

\begin{Prop}
Para cada $t\geq0$, la funci\'on generadora de momentos $\esp\left[e^{\alpha N\left(t\right)}\right]$ existe para alguna $\alpha$ en una vecindad del 0, y de aqu\'i que $\esp\left[N\left(t\right)^{m}\right]<\infty$, para $m\geq1$.
\end{Prop}


\begin{Note}
Si el primer tiempo de renovaci\'on $\xi_{1}$ no tiene la misma distribuci\'on que el resto de las $\xi_{n}$, para $n\geq2$, a $N\left(t\right)$ se le llama Proceso de Renovaci\'on retardado, donde si $\xi$ tiene distribuci\'on $G$, entonces el tiempo $T_{n}$ de la $n$-\'esima renovaci\'on tiene distribuci\'on $G\star F^{\left(n-1\right)\star}\left(t\right)$
\end{Note}


\begin{Teo}
Para una constante $\mu\leq\infty$ ( o variable aleatoria), las siguientes expresiones son equivalentes:

\begin{eqnarray}
lim_{n\rightarrow\infty}n^{-1}T_{n}&=&\mu,\textrm{ c.s.}\\
lim_{t\rightarrow\infty}t^{-1}N\left(t\right)&=&1/\mu,\textrm{ c.s.}
\end{eqnarray}
\end{Teo}


Es decir, $T_{n}$ satisface la Ley Fuerte de los Grandes N\'umeros s\'i y s\'olo s\'i $N\left/t\right)$ la cumple.


\begin{Coro}[Ley Fuerte de los Grandes N\'umeros para Procesos de Renovaci\'on]
Si $N\left(t\right)$ es un proceso de renovaci\'on cuyos tiempos de inter-renovaci\'on tienen media $\mu\leq\infty$, entonces
\begin{eqnarray}
t^{-1}N\left(t\right)\rightarrow 1/\mu,\textrm{ c.s. cuando }t\rightarrow\infty.
\end{eqnarray}

\end{Coro}


Considerar el proceso estoc\'astico de valores reales $\left\{Z\left(t\right):t\geq0\right\}$ en el mismo espacio de probabilidad que $N\left(t\right)$

\begin{Def}
Para el proceso $\left\{Z\left(t\right):t\geq0\right\}$ se define la fluctuaci\'on m\'axima de $Z\left(t\right)$ en el intervalo $\left(T_{n-1},T_{n}\right]$:
\begin{eqnarray*}
M_{n}=\sup_{T_{n-1}<t\leq T_{n}}|Z\left(t\right)-Z\left(T_{n-1}\right)|
\end{eqnarray*}
\end{Def}

\begin{Teo}
Sup\'ongase que $n^{-1}T_{n}\rightarrow\mu$ c.s. cuando $n\rightarrow\infty$, donde $\mu\leq\infty$ es una constante o variable aleatoria. Sea $a$ una constante o variable aleatoria que puede ser infinita cuando $\mu$ es finita, y considere las expresiones l\'imite:
\begin{eqnarray}
lim_{n\rightarrow\infty}n^{-1}Z\left(T_{n}\right)&=&a,\textrm{ c.s.}\\
lim_{t\rightarrow\infty}t^{-1}Z\left(t\right)&=&a/\mu,\textrm{ c.s.}
\end{eqnarray}
La segunda expresi\'on implica la primera. Conversamente, la primera implica la segunda si el proceso $Z\left(t\right)$ es creciente, o si $lim_{n\rightarrow\infty}n^{-1}M_{n}=0$ c.s.
\end{Teo}

\begin{Coro}
Si $N\left(t\right)$ es un proceso de renovaci\'on, y $\left(Z\left(T_{n}\right)-Z\left(T_{n-1}\right),M_{n}\right)$, para $n\geq1$, son variables aleatorias independientes e id\'enticamente distribuidas con media finita, entonces,
\begin{eqnarray}
lim_{t\rightarrow\infty}t^{-1}Z\left(t\right)\rightarrow\frac{\esp\left[Z\left(T_{1}\right)-Z\left(T_{0}\right)\right]}{\esp\left[T_{1}\right]},\textrm{ c.s. cuando  }t\rightarrow\infty.
\end{eqnarray}
\end{Coro}

%___________________________________________________________________________________________
%
\subsection{Funci\'on de Renovaci\'on}
%___________________________________________________________________________________________
%


Sup\'ongase que $N\left(t\right)$ es un proceso de renovaci\'on con distribuci\'on $F$ con media finita $\mu$.

\begin{Def}
La funci\'on de renovaci\'on asociada con la distribuci\'on $F$, del proceso $N\left(t\right)$, es
\begin{eqnarray*}
U\left(t\right)=\sum_{n=1}^{\infty}F^{n\star}\left(t\right),\textrm{   }t\geq0,
\end{eqnarray*}
donde $F^{0\star}\left(t\right)=\indora\left(t\geq0\right)$.
\end{Def}


\begin{Prop}
Sup\'ongase que la distribuci\'on de inter-renovaci\'on $F$ tiene densidad $f$. Entonces $U\left(t\right)$ tambi\'en tiene densidad, para $t>0$, y es $U^{'}\left(t\right)=\sum_{n=0}^{\infty}f^{n\star}\left(t\right)$. Adem\'as
\begin{eqnarray*}
\prob\left\{N\left(t\right)>N\left(t-\right)\right\}=0\textrm{,   }t\geq0.
\end{eqnarray*}
\end{Prop}

\begin{Def}
La Transformada de Laplace-Stieljes de $F$ est\'a dada por

\begin{eqnarray*}
\hat{F}\left(\alpha\right)=\int_{\rea_{+}}e^{-\alpha t}dF\left(t\right)\textrm{,  }\alpha\geq0.
\end{eqnarray*}
\end{Def}

Entonces

\begin{eqnarray*}
\hat{U}\left(\alpha\right)=\sum_{n=0}^{\infty}\hat{F^{n\star}}\left(\alpha\right)=\sum_{n=0}^{\infty}\hat{F}\left(\alpha\right)^{n}=\frac{1}{1-\hat{F}\left(\alpha\right)}.
\end{eqnarray*}


\begin{Prop}
La Transformada de Laplace $\hat{U}\left(\alpha\right)$ y $\hat{F}\left(\alpha\right)$ determina una a la otra de manera \'unica por la relaci\'on $\hat{U}\left(\alpha\right)=\frac{1}{1-\hat{F}\left(\alpha\right)}$.
\end{Prop}


\begin{Note}
Un proceso de renovaci\'on $N\left(t\right)$ cuyos tiempos de inter-renovaci\'on tienen media finita, es un proceso Poisson con tasa $\lambda$ si y s\'olo s\'i $\esp\left[U\left(t\right)\right]=\lambda t$, para $t\geq0$.
\end{Note}


\begin{Teo}
Sea $N\left(t\right)$ un proceso puntual simple con puntos de localizaci\'on $T_{n}$ tal que $\eta\left(t\right)=\esp\left[N\left(\right)\right]$ es finita para cada $t$. Entonces para cualquier funci\'on $f:\rea_{+}\rightarrow\rea$,
\begin{eqnarray*}
\esp\left[\sum_{n=1}^{N\left(\right)}f\left(T_{n}\right)\right]=\int_{\left(0,t\right]}f\left(s\right)d\eta\left(s\right)\textrm{,  }t\geq0,
\end{eqnarray*}
suponiendo que la integral exista. Adem\'as si $X_{1},X_{2},\ldots$ son variables aleatorias definidas en el mismo espacio de probabilidad que el proceso $N\left(t\right)$ tal que $\esp\left[X_{n}|T_{n}=s\right]=f\left(s\right)$, independiente de $n$. Entonces
\begin{eqnarray*}
\esp\left[\sum_{n=1}^{N\left(t\right)}X_{n}\right]=\int_{\left(0,t\right]}f\left(s\right)d\eta\left(s\right)\textrm{,  }t\geq0,
\end{eqnarray*} 
suponiendo que la integral exista. 
\end{Teo}

\begin{Coro}[Identidad de Wald para Renovaciones]
Para el proceso de renovaci\'on $N\left(t\right)$,
\begin{eqnarray*}
\esp\left[T_{N\left(t\right)+1}\right]=\mu\esp\left[N\left(t\right)+1\right]\textrm{,  }t\geq0,
\end{eqnarray*}  
\end{Coro}

%______________________________________________________________________
\subsection{Procesos de Renovaci\'on}
%______________________________________________________________________

\begin{Def}\label{Def.Tn}
Sean $0\leq T_{1}\leq T_{2}\leq \ldots$ son tiempos aleatorios infinitos en los cuales ocurren ciertos eventos. El n\'umero de tiempos $T_{n}$ en el intervalo $\left[0,t\right)$ es

\begin{eqnarray}
N\left(t\right)=\sum_{n=1}^{\infty}\indora\left(T_{n}\leq t\right),
\end{eqnarray}
para $t\geq0$.
\end{Def}

Si se consideran los puntos $T_{n}$ como elementos de $\rea_{+}$, y $N\left(t\right)$ es el n\'umero de puntos en $\rea$. El proceso denotado por $\left\{N\left(t\right):t\geq0\right\}$, denotado por $N\left(t\right)$, es un proceso puntual en $\rea_{+}$. Los $T_{n}$ son los tiempos de ocurrencia, el proceso puntual $N\left(t\right)$ es simple si su n\'umero de ocurrencias son distintas: $0<T_{1}<T_{2}<\ldots$ casi seguramente.

\begin{Def}
Un proceso puntual $N\left(t\right)$ es un proceso de renovaci\'on si los tiempos de interocurrencia $\xi_{n}=T_{n}-T_{n-1}$, para $n\geq1$, son independientes e identicamente distribuidos con distribuci\'on $F$, donde $F\left(0\right)=0$ y $T_{0}=0$. Los $T_{n}$ son llamados tiempos de renovaci\'on, referente a la independencia o renovaci\'on de la informaci\'on estoc\'astica en estos tiempos. Los $\xi_{n}$ son los tiempos de inter-renovaci\'on, y $N\left(t\right)$ es el n\'umero de renovaciones en el intervalo $\left[0,t\right)$
\end{Def}


\begin{Note}
Para definir un proceso de renovaci\'on para cualquier contexto, solamente hay que especificar una distribuci\'on $F$, con $F\left(0\right)=0$, para los tiempos de inter-renovaci\'on. La funci\'on $F$ en turno degune las otra variables aleatorias. De manera formal, existe un espacio de probabilidad y una sucesi\'on de variables aleatorias $\xi_{1},\xi_{2},\ldots$ definidas en este con distribuci\'on $F$. Entonces las otras cantidades son $T_{n}=\sum_{k=1}^{n}\xi_{k}$ y $N\left(t\right)=\sum_{n=1}^{\infty}\indora\left(T_{n}\leq t\right)$, donde $T_{n}\rightarrow\infty$ casi seguramente por la Ley Fuerte de los Grandes Números.
\end{Note}
%_____________________________________________________
\subsection{Puntos de Renovaci\'on}
%_____________________________________________________

Para cada cola $Q_{i}$ se tienen los procesos de arribo a la cola, para estas, los tiempos de arribo est\'an dados por $$\left\{T_{1}^{i},T_{2}^{i},\ldots,T_{k}^{i},\ldots\right\},$$ entonces, consideremos solamente los primeros tiempos de arribo a cada una de las colas, es decir, $$\left\{T_{1}^{1},T_{1}^{2},T_{1}^{3},T_{1}^{4}\right\},$$ se sabe que cada uno de estos tiempos se distribuye de manera exponencial con par\'ametro $1/mu_{i}$. Adem\'as se sabe que para $$T^{*}=\min\left\{T_{1}^{1},T_{1}^{2},T_{1}^{3},T_{1}^{4}\right\},$$ $T^{*}$ se distribuye de manera exponencial con par\'ametro $$\mu^{*}=\sum_{i=1}^{4}\mu_{i}.$$ Ahora, dado que 
\begin{center}
\begin{tabular}{lcl}
$\tilde{r}=r_{1}+r_{2}$ & y &$\hat{r}=r_{3}+r_{4}.$
\end{tabular}
\end{center}


Supongamos que $$\tilde{r},\hat{r}<\mu^{*},$$ entonces si tomamos $$r^{*}=\min\left\{\tilde{r},\hat{r}\right\},$$ se tiene que para  $$t^{*}\in\left(0,r^{*}\right)$$ se cumple que 
\begin{center}
\begin{tabular}{lcl}
$\tau_{1}\left(1\right)=0$ & y por tanto & $\overline{\tau}_{1}=0,$
\end{tabular}
\end{center}
entonces para la segunda cola en este primer ciclo se cumple que $$\tau_{2}=\overline{\tau}_{1}+r_{1}=r_{1}<\mu^{*},$$ y por tanto se tiene que  $$\overline{\tau}_{2}=\tau_{2}.$$ Por lo tanto, nuevamente para la primer cola en el segundo ciclo $$\tau_{1}\left(2\right)=\tau_{2}\left(1\right)+r_{2}=\tilde{r}<\mu^{*}.$$ An\'alogamente para el segundo sistema se tiene que ambas colas est\'an vac\'ias, es decir, existe un valor $t^{*}$ tal que en el intervalo $\left(0,t^{*}\right)$ no ha llegado ning\'un usuario, es decir, $$L_{i}\left(t^{*}\right)=0$$ para $i=1,2,3,4$.




%_______________________________________________________________________________________________________
\subsection{Ya revisado}
%_______________________________________________________________________________________________________


Def\'inanse los puntos de regenaraci\'on  en el proceso $\left[L_{1}\left(t\right),L_{2}\left(t\right),\ldots,L_{N}\left(t\right)\right]$. Los puntos cuando la cola $i$ es visitada y todos los $L_{j}\left(\tau_{i}\left(m\right)\right)=0$ para $i=1,2$  son puntos de regeneraci\'on. Se llama ciclo regenerativo al intervalo entre dos puntos regenerativos sucesivos.

Sea $M_{i}$  el n\'umero de ciclos de visita en un ciclo regenerativo, y sea $C_{i}^{(m)}$, para $m=1,2,\ldots,M_{i}$ la duraci\'on del $m$-\'esimo ciclo de visita en un ciclo regenerativo. Se define el ciclo del tiempo de visita promedio $\esp\left[C_{i}\right]$ como

\begin{eqnarray*}
\esp\left[C_{i}\right]&=&\frac{\esp\left[\sum_{m=1}^{M_{i}}C_{i}^{(m)}\right]}{\esp\left[M_{i}\right]}
\end{eqnarray*}




Sea la funci\'on generadora de momentos para $L_{i}$, el n\'umero de usuarios en la cola $Q_{i}\left(z\right)$ en cualquier momento, est\'a dada por el tiempo promedio de $z^{L_{i}\left(t\right)}$ sobre el ciclo regenerativo definido anteriormente:

\begin{eqnarray*}
Q_{i}\left(z\right)&=&\esp\left[z^{L_{i}\left(t\right)}\right]=\frac{\esp\left[\sum_{m=1}^{M_{i}}\sum_{t=\tau_{i}\left(m\right)}^{\tau_{i}\left(m+1\right)-1}z^{L_{i}\left(t\right)}\right]}{\esp\left[\sum_{m=1}^{M_{i}}\tau_{i}\left(m+1\right)-\tau_{i}\left(m\right)\right]}
\end{eqnarray*}

$M_{i}$ es un tiempo de paro en el proceso regenerativo con $\esp\left[M_{i}\right]<\infty$, se sigue del lema de Wald que:


\begin{eqnarray*}
\esp\left[\sum_{m=1}^{M_{i}}\sum_{t=\tau_{i}\left(m\right)}^{\tau_{i}\left(m+1\right)-1}z^{L_{i}\left(t\right)}\right]&=&\esp\left[M_{i}\right]\esp\left[\sum_{t=\tau_{i}\left(m\right)}^{\tau_{i}\left(m+1\right)-1}z^{L_{i}\left(t\right)}\right]\\
\esp\left[\sum_{m=1}^{M_{i}}\tau_{i}\left(m+1\right)-\tau_{i}\left(m\right)\right]&=&\esp\left[M_{i}\right]\esp\left[\tau_{i}\left(m+1\right)-\tau_{i}\left(m\right)\right]
\end{eqnarray*}

por tanto se tiene que


\begin{eqnarray*}
Q_{i}\left(z\right)&=&\frac{\esp\left[\sum_{t=\tau_{i}\left(m\right)}^{\tau_{i}\left(m+1\right)-1}z^{L_{i}\left(t\right)}\right]}{\esp\left[\tau_{i}\left(m+1\right)-\tau_{i}\left(m\right)\right]}
\end{eqnarray*}

observar que el denominador es simplemente la duraci\'on promedio del tiempo del ciclo.


Se puede demostrar (ver Hideaki Takagi 1986) que

\begin{eqnarray*}
\esp\left[\sum_{t=\tau_{i}\left(m\right)}^{\tau_{i}\left(m+1\right)-1}z^{L_{i}\left(t\right)}\right]=z\frac{F_{i}\left(z\right)-1}{z-P_{i}\left(z\right)}
\end{eqnarray*}

Durante el tiempo de intervisita para la cola $i$, $L_{i}\left(t\right)$ solamente se incrementa de manera que el incremento por intervalo de tiempo est\'a dado por la funci\'on generadora de probabilidades de $P_{i}\left(z\right)$, por tanto la suma sobre el tiempo de intervisita puede evaluarse como:

\begin{eqnarray*}
\esp\left[\sum_{t=\tau_{i}\left(m\right)}^{\tau_{i}\left(m+1\right)-1}z^{L_{i}\left(t\right)}\right]&=&\esp\left[\sum_{t=\tau_{i}\left(m\right)}^{\tau_{i}\left(m+1\right)-1}\left\{P_{i}\left(z\right)\right\}^{t-\overline{\tau}_{i}\left(m\right)}\right]=\frac{1-\esp\left[\left\{P_{i}\left(z\right)\right\}^{\tau_{i}\left(m+1\right)-\overline{\tau}_{i}\left(m\right)}\right]}{1-P_{i}\left(z\right)}\\
&=&\frac{1-I_{i}\left[P_{i}\left(z\right)\right]}{1-P_{i}\left(z\right)}
\end{eqnarray*}
por tanto

\begin{eqnarray*}
\esp\left[\sum_{t=\tau_{i}\left(m\right)}^{\tau_{i}\left(m+1\right)-1}z^{L_{i}\left(t\right)}\right]&=&\frac{1-F_{i}\left(z\right)}{1-P_{i}\left(z\right)}
\end{eqnarray*}

Haciendo uso de lo hasta ahora desarrollado se tiene que

\begin{eqnarray*}
Q_{i}\left(z\right)&=&\frac{1}{\esp\left[C_{i}\right]}\cdot\frac{1-F_{i}\left(z\right)}{P_{i}\left(z\right)-z}\cdot\frac{\left(1-z\right)P_{i}\left(z\right)}{1-P_{i}\left(z\right)}\\
&=&\frac{\mu_{i}\left(1-\mu_{i}\right)}{f_{i}\left(i\right)}\cdot\frac{1-F_{i}\left(z\right)}{P_{i}\left(z\right)-z}\cdot\frac{\left(1-z\right)P_{i}\left(z\right)}{1-P_{i}\left(z\right)}
\end{eqnarray*}

\begin{Def}
Sea $L_{i}^{*}$el n\'umero de usuarios en la cola $Q_{i}$ cuando es visitada por el servidor para dar servicio, entonces

\begin{eqnarray}
\esp\left[L_{i}^{*}\right]&=&f_{i}\left(i\right)\\
Var\left[L_{i}^{*}\right]&=&f_{i}\left(i,i\right)+\esp\left[L_{i}^{*}\right]-\esp\left[L_{i}^{*}\right]^{2}.
\end{eqnarray}

\end{Def}


\begin{Def}
El tiempo de intervisita $I_{i}$ es el periodo de tiempo que comienza cuando se ha completado el servicio en un ciclo y termina cuando es visitada nuevamente en el pr\'oximo ciclo. Su  duraci\'on del mismo est\'a dada por $\tau_{i}\left(m+1\right)-\overline{\tau}_{i}\left(m\right)$.
\end{Def}


Recordemos las siguientes expresiones:

\begin{eqnarray*}
S_{i}\left(z\right)&=&\esp\left[z^{\overline{\tau}_{i}\left(m\right)-\tau_{i}\left(m\right)}\right]=F_{i}\left(\theta\left(z\right)\right),\\
F\left(z\right)&=&\esp\left[z^{L_{0}}\right],\\
P\left(z\right)&=&\esp\left[z^{X_{n}}\right],\\
F_{i}\left(z\right)&=&\esp\left[z^{L_{i}\left(\tau_{i}\left(m\right)\right)}\right],
\theta_{i}\left(z\right)-zP_{i}
\end{eqnarray*}

entonces 

\begin{eqnarray*}
\esp\left[S_{i}\right]&=&\frac{\esp\left[L_{i}^{*}\right]}{1-\mu_{i}}=\frac{f_{i}\left(i\right)}{1-\mu_{i}},\\
Var\left[S_{i}\right]&=&\frac{Var\left[L_{i}^{*}\right]}{\left(1-\mu_{i}\right)^{2}}+\frac{\sigma^{2}\esp\left[L_{i}^{*}\right]}{\left(1-\mu_{i}\right)^{3}}
\end{eqnarray*}

donde recordemos que

\begin{eqnarray*}
Var\left[L_{i}^{*}\right]&=&f_{i}\left(i,i\right)+f_{i}\left(i\right)-f_{i}\left(i\right)^{2}.
\end{eqnarray*}

La duraci\'on del tiempo de intervisita es $\tau_{i}\left(m+1\right)-\overline{\tau}\left(m\right)$. Dado que el n\'umero de usuarios presentes en $Q_{i}$ al tiempo $t=\tau_{i}\left(m+1\right)$ es igual al n\'umero de arribos durante el intervalo de tiempo $\left[\overline{\tau}\left(m\right),\tau_{i}\left(m+1\right)\right]$ se tiene que


\begin{eqnarray*}
\esp\left[z_{i}^{L_{i}\left(\tau_{i}\left(m+1\right)\right)}\right]=\esp\left[\left\{P_{i}\left(z_{i}\right)\right\}^{\tau_{i}\left(m+1\right)-\overline{\tau}\left(m\right)}\right]
\end{eqnarray*}

entonces, si \begin{eqnarray*}I_{i}\left(z\right)&=&\esp\left[z^{\tau_{i}\left(m+1\right)-\overline{\tau}\left(m\right)}\right]\end{eqnarray*} se tienen que

\begin{eqnarray*}
F_{i}\left(z\right)=I_{i}\left[P_{i}\left(z\right)\right]
\end{eqnarray*}
para $i=1,2$, por tanto



\begin{eqnarray*}
\esp\left[L_{i}^{*}\right]&=&\mu_{i}\esp\left[I_{i}\right]\\
Var\left[L_{i}^{*}\right]&=&\mu_{i}^{2}Var\left[I_{i}\right]+\sigma^{2}\esp\left[I_{i}\right]
\end{eqnarray*}
para $i=1,2$, por tanto


\begin{eqnarray*}
\esp\left[I_{i}\right]&=&\frac{f_{i}\left(i\right)}{\mu_{i}},
\end{eqnarray*}
adem\'as

\begin{eqnarray*}
Var\left[I_{i}\right]&=&\frac{Var\left[L_{i}^{*}\right]}{\mu_{i}^{2}}-\frac{\sigma_{i}^{2}}{\mu_{i}^{2}}f_{i}\left(i\right).
\end{eqnarray*}


Si  $C_{i}\left(z\right)=\esp\left[z^{\overline{\tau}\left(m+1\right)-\overline{\tau}_{i}\left(m\right)}\right]$el tiempo de duraci\'on del ciclo, entonces, por lo hasta ahora establecido, se tiene que

\begin{eqnarray*}
C_{i}\left(z\right)=I_{i}\left[\theta_{i}\left(z\right)\right],
\end{eqnarray*}
entonces

\begin{eqnarray*}
\esp\left[C_{i}\right]&=&\esp\left[I_{i}\right]\esp\left[\theta_{i}\left(z\right)\right]=\frac{\esp\left[L_{i}^{*}\right]}{\mu_{i}}\frac{1}{1-\mu_{i}}=\frac{f_{i}\left(i\right)}{\mu_{i}\left(1-\mu_{i}\right)}\\
Var\left[C_{i}\right]&=&\frac{Var\left[L_{i}^{*}\right]}{\mu_{i}^{2}\left(1-\mu_{i}\right)^{2}}.
\end{eqnarray*}

Por tanto se tienen las siguientes igualdades


\begin{eqnarray*}
\esp\left[S_{i}\right]&=&\mu_{i}\esp\left[C_{i}\right],\\
\esp\left[I_{i}\right]&=&\left(1-\mu_{i}\right)\esp\left[C_{i}\right]\\
\end{eqnarray*}

derivando con respecto a $z$



\begin{eqnarray*}
\frac{d Q_{i}\left(z\right)}{d z}&=&\frac{\left(1-F_{i}\left(z\right)\right)P_{i}\left(z\right)}{\esp\left[C_{i}\right]\left(1-P_{i}\left(z\right)\right)\left(P_{i}\left(z\right)-z\right)}\\
&-&\frac{\left(1-z\right)P_{i}\left(z\right)F_{i}^{'}\left(z\right)}{\esp\left[C_{i}\right]\left(1-P_{i}\left(z\right)\right)\left(P_{i}\left(z\right)-z\right)}\\
&-&\frac{\left(1-z\right)\left(1-F_{i}\left(z\right)\right)P_{i}\left(z\right)\left(P_{i}^{'}\left(z\right)-1\right)}{\esp\left[C_{i}\right]\left(1-P_{i}\left(z\right)\right)\left(P_{i}\left(z\right)-z\right)^{2}}\\
&+&\frac{\left(1-z\right)\left(1-F_{i}\left(z\right)\right)P_{i}^{'}\left(z\right)}{\esp\left[C_{i}\right]\left(1-P_{i}\left(z\right)\right)\left(P_{i}\left(z\right)-z\right)}\\
&+&\frac{\left(1-z\right)\left(1-F_{i}\left(z\right)\right)P_{i}\left(z\right)P_{i}^{'}\left(z\right)}{\esp\left[C_{i}\right]\left(1-P_{i}\left(z\right)\right)^{2}\left(P_{i}\left(z\right)-z\right)}
\end{eqnarray*}

Calculando el l\'imite cuando $z\rightarrow1^{+}$:
\begin{eqnarray}
Q_{i}^{(1)}\left(z\right)=\lim_{z\rightarrow1^{+}}\frac{d Q_{i}\left(z\right)}{dz}&=&\lim_{z\rightarrow1}\frac{\left(1-F_{i}\left(z\right)\right)P_{i}\left(z\right)}{\esp\left[C_{i}\right]\left(1-P_{i}\left(z\right)\right)\left(P_{i}\left(z\right)-z\right)}\\
&-&\lim_{z\rightarrow1^{+}}\frac{\left(1-z\right)P_{i}\left(z\right)F_{i}^{'}\left(z\right)}{\esp\left[C_{i}\right]\left(1-P_{i}\left(z\right)\right)\left(P_{i}\left(z\right)-z\right)}\\
&-&\lim_{z\rightarrow1^{+}}\frac{\left(1-z\right)\left(1-F_{i}\left(z\right)\right)P_{i}\left(z\right)\left(P_{i}^{'}\left(z\right)-1\right)}{\esp\left[C_{i}\right]\left(1-P_{i}\left(z\right)\right)\left(P_{i}\left(z\right)-z\right)^{2}}\\
&+&\lim_{z\rightarrow1^{+}}\frac{\left(1-z\right)\left(1-F_{i}\left(z\right)\right)P_{i}^{'}\left(z\right)}{\esp\left[C_{i}\right]\left(1-P_{i}\left(z\right)\right)\left(P_{i}\left(z\right)-z\right)}\\
&+&\lim_{z\rightarrow1^{+}}\frac{\left(1-z\right)\left(1-F_{i}\left(z\right)\right)P_{i}\left(z\right)P_{i}^{'}\left(z\right)}{\esp\left[C_{i}\right]\left(1-P_{i}\left(z\right)\right)^{2}\left(P_{i}\left(z\right)-z\right)}
\end{eqnarray}

Entonces:
%______________________________________________________

\begin{eqnarray*}
\lim_{z\rightarrow1^{+}}\frac{\left(1-F_{i}\left(z\right)\right)P_{i}\left(z\right)}{\left(1-P_{i}\left(z\right)\right)\left(P_{i}\left(z\right)-z\right)}&=&\lim_{z\rightarrow1^{+}}\frac{\frac{d}{dz}\left[\left(1-F_{i}\left(z\right)\right)P_{i}\left(z\right)\right]}{\frac{d}{dz}\left[\left(1-P_{i}\left(z\right)\right)\left(-z+P_{i}\left(z\right)\right)\right]}\\
&=&\lim_{z\rightarrow1^{+}}\frac{-P_{i}\left(z\right)F_{i}^{'}\left(z\right)+\left(1-F_{i}\left(z\right)\right)P_{i}^{'}\left(z\right)}{\left(1-P_{i}\left(z\right)\right)\left(-1+P_{i}^{'}\left(z\right)\right)-\left(-z+P_{i}\left(z\right)\right)P_{i}^{'}\left(z\right)}
\end{eqnarray*}


%______________________________________________________


\begin{eqnarray*}
\lim_{z\rightarrow1^{+}}\frac{\left(1-z\right)P_{i}\left(z\right)F_{i}^{'}\left(z\right)}{\left(1-P_{i}\left(z\right)\right)\left(P_{i}\left(z\right)-z\right)}&=&\lim_{z\rightarrow1^{+}}\frac{\frac{d}{dz}\left[\left(1-z\right)P_{i}\left(z\right)F_{i}^{'}\left(z\right)\right]}{\frac{d}{dz}\left[\left(1-P_{i}\left(z\right)\right)\left(P_{i}\left(z\right)-z\right)\right]}\\
&=&\lim_{z\rightarrow1^{+}}\frac{-P_{i}\left(z\right) F_{i}^{'}\left(z\right)+(1-z) F_{i}^{'}\left(z\right) P_{i}^{'}\left(z\right)+(1-z) P_{i}\left(z\right)F_{i}^{''}\left(z\right)}{\left(1-P_{i}\left(z\right)\right)\left(-1+P_{i}^{'}\left(z\right)\right)-\left(-z+P_{i}\left(z\right)\right)P_{i}^{'}\left(z\right)}
\end{eqnarray*}


%______________________________________________________

\begin{eqnarray*}
&&\lim_{z\rightarrow1^{+}}\frac{\left(1-z\right)\left(1-F_{i}\left(z\right)\right)P_{i}\left(z\right)\left(P_{i}^{'}\left(z\right)-1\right)}{\left(1-P_{i}\left(z\right)\right)\left(P_{i}\left(z\right)-z\right)^{2}}=\lim_{z\rightarrow1^{+}}\frac{\frac{d}{dz}\left[\left(1-z\right)\left(1-F_{i}\left(z\right)\right)P_{i}\left(z\right)\left(P_{i}^{'}\left(z\right)-1\right)\right]}{\frac{d}{dz}\left[\left(1-P_{i}\left(z\right)\right)\left(P_{i}\left(z\right)-z\right)^{2}\right]}\\
&=&\lim_{z\rightarrow1^{+}}\frac{-\left(1-F_{i}\left(z\right)\right) P_{i}\left(z\right)\left(-1+P_{i}^{'}\left(z\right)\right)-(1-z) P_{i}\left(z\right)F_{i}^{'}\left(z\right)\left(-1+P_{i}^{'}\left(z\right)\right)}{2\left(1-P_{i}\left(z\right)\right)\left(-z+P_{i}\left(z\right)\right) \left(-1+P_{i}^{'}\left(z\right)\right)-\left(-z+P_{i}\left(z\right)\right)^2 P_{i}^{'}\left(z\right)}\\
&+&\lim_{z\rightarrow1^{+}}\frac{+(1-z) \left(1-F_{i}\left(z\right)\right) \left(-1+P_{i}^{'}\left(z\right)\right) P_{i}^{'}\left(z\right)}{{2\left(1-P_{i}\left(z\right)\right)\left(-z+P_{i}\left(z\right)\right) \left(-1+P_{i}^{'}\left(z\right)\right)-\left(-z+P_{i}\left(z\right)\right)^2 P_{i}^{'}\left(z\right)}}\\
&+&\lim_{z\rightarrow1^{+}}\frac{+(1-z) \left(1-F_{i}\left(z\right)\right) P_{i}\left(z\right)P_{i}^{''}\left(z\right)}{{2\left(1-P_{i}\left(z\right)\right)\left(-z+P_{i}\left(z\right)\right) \left(-1+P_{i}^{'}\left(z\right)\right)-\left(-z+P_{i}\left(z\right)\right)^2 P_{i}^{'}\left(z\right)}}
\end{eqnarray*}











%______________________________________________________
\begin{eqnarray*}
&&\lim_{z\rightarrow1^{+}}\frac{\left(1-z\right)\left(1-F_{i}\left(z\right)\right)P_{i}^{'}\left(z\right)}{\left(1-P_{i}\left(z\right)\right)\left(P_{i}\left(z\right)-z\right)}=\lim_{z\rightarrow1^{+}}\frac{\frac{d}{dz}\left[\left(1-z\right)\left(1-F_{i}\left(z\right)\right)P_{i}^{'}\left(z\right)\right]}{\frac{d}{dz}\left[\left(1-P_{i}\left(z\right)\right)\left(P_{i}\left(z\right)-z\right)\right]}\\
&=&\lim_{z\rightarrow1^{+}}\frac{-\left(1-F_{i}\left(z\right)\right) P_{i}^{'}\left(z\right)-(1-z) F_{i}^{'}\left(z\right) P_{i}^{'}\left(z\right)+(1-z) \left(1-F_{i}\left(z\right)\right) P_{i}^{''}\left(z\right)}{\left(1-P_{i}\left(z\right)\right) \left(-1+P_{i}^{'}\left(z\right)\right)-\left(-z+P_{i}\left(z\right)\right) P_{i}^{'}\left(z\right)}\frac{}{}
\end{eqnarray*}

%______________________________________________________
\begin{eqnarray*}
&&\lim_{z\rightarrow1^{+}}\frac{\left(1-z\right)\left(1-F_{i}\left(z\right)\right)P_{i}\left(z\right)P_{i}^{'}\left(z\right)}{\left(1-P_{i}\left(z\right)\right)^{2}\left(P_{i}\left(z\right)-z\right)}=\lim_{z\rightarrow1^{+}}\frac{\frac{d}{dz}\left[\left(1-z\right)\left(1-F_{i}\left(z\right)\right)P_{i}\left(z\right)P_{i}^{'}\left(z\right)\right]}{\frac{d}{dz}\left[\left(1-P_{i}\left(z\right)\right)^{2}\left(P_{i}\left(z\right)-z\right)\right]}\\
&=&\lim_{z\rightarrow1^{+}}\frac{-\left(1-F_{i}\left(z\right)\right) P_{i}\left(z\right) P_{i}^{'}\left(z\right)-(1-z) P_{i}\left(z\right) F_{i}^{'}\left(z\right)P_i'[z]}{\left(1-P_{i}\left(z\right)\right)^2 \left(-1+P_{i}^{'}\left(z\right)\right)-2 \left(1-P_{i}\left(z\right)\right) \left(-z+P_{i}\left(z\right)\right) P_{i}^{'}\left(z\right)}\\
&+&\lim_{z\rightarrow1^{+}}\frac{(1-z) \left(1-F_{i}\left(z\right)\right) P_{i}^{'}\left(z\right)^2+(1-z) \left(1-F_{i}\left(z\right)\right) P_{i}\left(z\right) P_{i}^{''}\left(z\right)}{\left(1-P_{i}\left(z\right)\right)^2 \left(-1+P_{i}^{'}\left(z\right)\right)-2 \left(1-P_{i}\left(z\right)\right) \left(-z+P_{i}\left(z\right)\right) P_{i}^{'}\left(z\right)}\\
\end{eqnarray*}



En nuestra notaci\'on $V\left(t\right)\equiv C_{i}$ y $X_{i}=C_{i}^{(m)}$ para nuestra segunda definici\'on, mientras que para la primera la notaci\'on es: $X\left(t\right)\equiv C_{i}$ y $R_{i}\equiv C_{i}^{(m)}$.


%___________________________________________________________________________________________
%\section{Tiempos de Ciclo e Intervisita}
%___________________________________________________________________________________________


\begin{Def}
Sea $L_{i}^{*}$el n\'umero de usuarios en la cola $Q_{i}$ cuando es visitada por el servidor para dar servicio, entonces

\begin{eqnarray}
\esp\left[L_{i}^{*}\right]&=&f_{i}\left(i\right)\\
Var\left[L_{i}^{*}\right]&=&f_{i}\left(i,i\right)+\esp\left[L_{i}^{*}\right]-\esp\left[L_{i}^{*}\right]^{2}.
\end{eqnarray}

\end{Def}

\begin{Def}
El tiempo de Ciclo $C_{i}$ es e periodo de tiempo que comienza cuando la cola $i$ es visitada por primera vez en un ciclo, y termina cuando es visitado nuevamente en el pr\'oximo ciclo. La duraci\'on del mismo est\'a dada por $\tau_{i}\left(m+1\right)-\tau_{i}\left(m\right)$, o equivalentemente $\overline{\tau}_{i}\left(m+1\right)-\overline{\tau}_{i}\left(m\right)$ bajo condiciones de estabilidad.
\end{Def}

\begin{Def}
El tiempo de intervisita $I_{i}$ es el periodo de tiempo que comienza cuando se ha completado el servicio en un ciclo y termina cuando es visitada nuevamente en el pr\'oximo ciclo. Su  duraci\'on del mismo est\'a dada por $\tau_{i}\left(m+1\right)-\overline{\tau}_{i}\left(m\right)$.
\end{Def}


Recordemos las siguientes expresiones:

\begin{eqnarray*}
S_{i}\left(z\right)&=&\esp\left[z^{\overline{\tau}_{i}\left(m\right)-\tau_{i}\left(m\right)}\right]=F_{i}\left(\theta\left(z\right)\right),\\
F\left(z\right)&=&\esp\left[z^{L_{0}}\right],\\
P\left(z\right)&=&\esp\left[z^{X_{n}}\right],\\
F_{i}\left(z\right)&=&\esp\left[z^{L_{i}\left(\tau_{i}\left(m\right)\right)}\right],
\theta_{i}\left(z\right)-zP_{i}
\end{eqnarray*}

entonces 

\begin{eqnarray*}
\esp\left[S_{i}\right]&=&\frac{\esp\left[L_{i}^{*}\right]}{1-\mu_{i}}=\frac{f_{i}\left(i\right)}{1-\mu_{i}},\\
Var\left[S_{i}\right]&=&\frac{Var\left[L_{i}^{*}\right]}{\left(1-\mu_{i}\right)^{2}}+\frac{\sigma^{2}\esp\left[L_{i}^{*}\right]}{\left(1-\mu_{i}\right)^{3}}
\end{eqnarray*}

donde recordemos que

\begin{eqnarray*}
Var\left[L_{i}^{*}\right]&=&f_{i}\left(i,i\right)+f_{i}\left(i\right)-f_{i}\left(i\right)^{2}.
\end{eqnarray*}

La duraci\'on del tiempo de intervisita es $\tau_{i}\left(m+1\right)-\overline{\tau}\left(m\right)$. Dado que el n\'umero de usuarios presentes en $Q_{i}$ al tiempo $t=\tau_{i}\left(m+1\right)$ es igual al n\'umero de arribos durante el intervalo de tiempo $\left[\overline{\tau}\left(m\right),\tau_{i}\left(m+1\right)\right]$ se tiene que


\begin{eqnarray*}
\esp\left[z_{i}^{L_{i}\left(\tau_{i}\left(m+1\right)\right)}\right]=\esp\left[\left\{P_{i}\left(z_{i}\right)\right\}^{\tau_{i}\left(m+1\right)-\overline{\tau}\left(m\right)}\right]
\end{eqnarray*}

entonces, si \begin{eqnarray*}I_{i}\left(z\right)&=&\esp\left[z^{\tau_{i}\left(m+1\right)-\overline{\tau}\left(m\right)}\right]\end{eqnarray*} se tienen que

\begin{eqnarray*}
F_{i}\left(z\right)=I_{i}\left[P_{i}\left(z\right)\right]
\end{eqnarray*}
para $i=1,2$, por tanto



\begin{eqnarray*}
\esp\left[L_{i}^{*}\right]&=&\mu_{i}\esp\left[I_{i}\right]\\
Var\left[L_{i}^{*}\right]&=&\mu_{i}^{2}Var\left[I_{i}\right]+\sigma^{2}\esp\left[I_{i}\right]
\end{eqnarray*}
para $i=1,2$, por tanto


\begin{eqnarray*}
\esp\left[I_{i}\right]&=&\frac{f_{i}\left(i\right)}{\mu_{i}},
\end{eqnarray*}
adem\'as

\begin{eqnarray*}
Var\left[I_{i}\right]&=&\frac{Var\left[L_{i}^{*}\right]}{\mu_{i}^{2}}-\frac{\sigma_{i}^{2}}{\mu_{i}^{2}}f_{i}\left(i\right).
\end{eqnarray*}


Si  $C_{i}\left(z\right)=\esp\left[z^{\overline{\tau}\left(m+1\right)-\overline{\tau}_{i}\left(m\right)}\right]$el tiempo de duraci\'on del ciclo, entonces, por lo hasta ahora establecido, se tiene que

\begin{eqnarray*}
C_{i}\left(z\right)=I_{i}\left[\theta_{i}\left(z\right)\right],
\end{eqnarray*}
entonces

\begin{eqnarray*}
\esp\left[C_{i}\right]&=&\esp\left[I_{i}\right]\esp\left[\theta_{i}\left(z\right)\right]=\frac{\esp\left[L_{i}^{*}\right]}{\mu_{i}}\frac{1}{1-\mu_{i}}=\frac{f_{i}\left(i\right)}{\mu_{i}\left(1-\mu_{i}\right)}\\
Var\left[C_{i}\right]&=&\frac{Var\left[L_{i}^{*}\right]}{\mu_{i}^{2}\left(1-\mu_{i}\right)^{2}}.
\end{eqnarray*}

Por tanto se tienen las siguientes igualdades


\begin{eqnarray*}
\esp\left[S_{i}\right]&=&\mu_{i}\esp\left[C_{i}\right],\\
\esp\left[I_{i}\right]&=&\left(1-\mu_{i}\right)\esp\left[C_{i}\right]\\
\end{eqnarray*}

Def\'inanse los puntos de regenaraci\'on  en el proceso $\left[L_{1}\left(t\right),L_{2}\left(t\right),\ldots,L_{N}\left(t\right)\right]$. Los puntos cuando la cola $i$ es visitada y todos los $L_{j}\left(\tau_{i}\left(m\right)\right)=0$ para $i=1,2$  son puntos de regeneraci\'on. Se llama ciclo regenerativo al intervalo entre dos puntos regenerativos sucesivos.

Sea $M_{i}$  el n\'umero de ciclos de visita en un ciclo regenerativo, y sea $C_{i}^{(m)}$, para $m=1,2,\ldots,M_{i}$ la duraci\'on del $m$-\'esimo ciclo de visita en un ciclo regenerativo. Se define el ciclo del tiempo de visita promedio $\esp\left[C_{i}\right]$ como

\begin{eqnarray*}
\esp\left[C_{i}\right]&=&\frac{\esp\left[\sum_{m=1}^{M_{i}}C_{i}^{(m)}\right]}{\esp\left[M_{i}\right]}
\end{eqnarray*}


En Stid72 y Heym82 se muestra que una condici\'on suficiente para que el proceso regenerativo 
estacionario sea un procesoo estacionario es que el valor esperado del tiempo del ciclo regenerativo sea finito:

\begin{eqnarray*}
\esp\left[\sum_{m=1}^{M_{i}}C_{i}^{(m)}\right]<\infty.
\end{eqnarray*}

como cada $C_{i}^{(m)}$ contiene intervalos de r\'eplica positivos, se tiene que $\esp\left[M_{i}\right]<\infty$, adem\'as, como $M_{i}>0$, se tiene que la condici\'on anterior es equivalente a tener que 

\begin{eqnarray*}
\esp\left[C_{i}\right]<\infty,
\end{eqnarray*}
por lo tanto una condici\'on suficiente para la existencia del proceso regenerativo est\'a dada por

\begin{eqnarray*}
\sum_{k=1}^{N}\mu_{k}<1.
\end{eqnarray*}



\begin{Note}\label{Cita1.Stidham}
En Stidham\cite{Stidham} y Heyman  se muestra que una condici\'on suficiente para que el proceso regenerativo 
estacionario sea un procesoo estacionario es que el valor esperado del tiempo del ciclo regenerativo sea finito:

\begin{eqnarray*}
\esp\left[\sum_{m=1}^{M_{i}}C_{i}^{(m)}\right]<\infty.
\end{eqnarray*}

como cada $C_{i}^{(m)}$ contiene intervalos de r\'eplica positivos, se tiene que $\esp\left[M_{i}\right]<\infty$, adem\'as, como $M_{i}>0$, se tiene que la condici\'on anterior es equivalente a tener que 

\begin{eqnarray*}
\esp\left[C_{i}\right]<\infty,
\end{eqnarray*}
por lo tanto una condici\'on suficiente para la existencia del proceso regenerativo est\'a dada por

\begin{eqnarray*}
\sum_{k=1}^{N}\mu_{k}<1.
\end{eqnarray*}

{\centering{\Huge{\textbf{Nota incompleta!!}}}}
\end{Note}

%_______________________________________________________________________________________
\subsection{Procesos de Renovaci\'on y Regenerativos}
%_______________________________________________________________________________________



Se puede demostrar (ver Hideaki Takagi 1986) que

\begin{eqnarray*}
\esp\left[\sum_{t=\tau_{i}\left(m\right)}^{\tau_{i}\left(m+1\right)-1}z^{L_{i}\left(t\right)}\right]=z\frac{F_{i}\left(z\right)-1}{z-P_{i}\left(z\right)}
\end{eqnarray*}

Durante el tiempo de intervisita para la cola $i$, $L_{i}\left(t\right)$ solamente se incrementa de manera que el incremento por intervalo de tiempo est\'a dado por la funci\'on generadora de probabilidades de $P_{i}\left(z\right)$, por tanto la suma sobre el tiempo de intervisita puede evaluarse como:

\begin{eqnarray*}
\esp\left[\sum_{t=\tau_{i}\left(m\right)}^{\tau_{i}\left(m+1\right)-1}z^{L_{i}\left(t\right)}\right]&=&\esp\left[\sum_{t=\tau_{i}\left(m\right)}^{\tau_{i}\left(m+1\right)-1}\left\{P_{i}\left(z\right)\right\}^{t-\overline{\tau}_{i}\left(m\right)}\right]=\frac{1-\esp\left[\left\{P_{i}\left(z\right)\right\}^{\tau_{i}\left(m+1\right)-\overline{\tau}_{i}\left(m\right)}\right]}{1-P_{i}\left(z\right)}\\
&=&\frac{1-I_{i}\left[P_{i}\left(z\right)\right]}{1-P_{i}\left(z\right)}
\end{eqnarray*}
por tanto

\begin{eqnarray*}
\esp\left[\sum_{t=\tau_{i}\left(m\right)}^{\tau_{i}\left(m+1\right)-1}z^{L_{i}\left(t\right)}\right]&=&\frac{1-F_{i}\left(z\right)}{1-P_{i}\left(z\right)}
\end{eqnarray*}

Haciendo uso de lo hasta ahora desarrollado se tiene que



%___________________________________________________________________________________________
%\subsection{Longitudes de la Cola en cualquier tiempo}
%___________________________________________________________________________________________
Sea 
\begin{eqnarray*}
Q_{i}\left(z\right)&=&\frac{1}{\esp\left[C_{i}\right]}\cdot\frac{1-F_{i}\left(z\right)}{P_{i}\left(z\right)-z}\cdot\frac{\left(1-z\right)P_{i}\left(z\right)}{1-P_{i}\left(z\right)}
\end{eqnarray*}

Consideremos una cola de la red de sistemas de visitas c\'iclicas fija, $Q_{l}$.


Conforme a la definici\'on dada al principio del cap\'itulo, definici\'on (\ref{Def.Tn}), sean $T_{1},T_{2},\ldots$ los puntos donde las longitudes de las colas de la red de sistemas de visitas c\'iclicas son cero simult\'aneamente, cuando la cola $Q_{l}$ es visitada por el servidor para dar servicio, es decir, $L_{1}\left(T_{i}\right)=0,L_{2}\left(T_{i}\right)=0,\hat{L}_{1}\left(T_{i}\right)=0$ y $\hat{L}_{2}\left(T_{i}\right)=0$, a estos puntos se les denominar\'a puntos regenerativos. Entonces, 

\begin{Def}
Al intervalo de tiempo entre dos puntos regenerativos se le llamar\'a ciclo regenerativo.
\end{Def}

\begin{Def}
Para $T_{i}$ se define, $M_{i}$, el n\'umero de ciclos de visita a la cola $Q_{l}$, durante el ciclo regenerativo, es decir, $M_{i}$ es un proceso de renovaci\'on.
\end{Def}

\begin{Def}
Para cada uno de los $M_{i}$'s, se definen a su vez la duraci\'on de cada uno de estos ciclos de visita en el ciclo regenerativo, $C_{i}^{(m)}$, para $m=1,2,\ldots,M_{i}$, que a su vez, tambi\'en es n proceso de renovaci\'on.
\end{Def}

En nuestra notaci\'on $V\left(t\right)\equiv C_{i}$ y $X_{i}=C_{i}^{(m)}$ para nuestra segunda definici\'on, mientras que para la primera la notaci\'on es: $X\left(t\right)\equiv C_{i}$ y $R_{i}\equiv C_{i}^{(m)}$.


%___________________________________________________________________________________________
%\subsection{Tiempos de Ciclo e Intervisita}
%___________________________________________________________________________________________


\begin{Def}
Sea $L_{i}^{*}$el n\'umero de usuarios en la cola $Q_{i}$ cuando es visitada por el servidor para dar servicio, entonces

\begin{eqnarray}
\esp\left[L_{i}^{*}\right]&=&f_{i}\left(i\right)\\
Var\left[L_{i}^{*}\right]&=&f_{i}\left(i,i\right)+\esp\left[L_{i}^{*}\right]-\esp\left[L_{i}^{*}\right]^{2}.
\end{eqnarray}

\end{Def}

\begin{Def}
El tiempo de Ciclo $C_{i}$ es e periodo de tiempo que comienza cuando la cola $i$ es visitada por primera vez en un ciclo, y termina cuando es visitado nuevamente en el pr\'oximo ciclo. La duraci\'on del mismo est\'a dada por $\tau_{i}\left(m+1\right)-\tau_{i}\left(m\right)$, o equivalentemente $\overline{\tau}_{i}\left(m+1\right)-\overline{\tau}_{i}\left(m\right)$ bajo condiciones de estabilidad.
\end{Def}



Recordemos las siguientes expresiones:

\begin{eqnarray*}
S_{i}\left(z\right)&=&\esp\left[z^{\overline{\tau}_{i}\left(m\right)-\tau_{i}\left(m\right)}\right]=F_{i}\left(\theta\left(z\right)\right),\\
F\left(z\right)&=&\esp\left[z^{L_{0}}\right],\\
P\left(z\right)&=&\esp\left[z^{X_{n}}\right],\\
F_{i}\left(z\right)&=&\esp\left[z^{L_{i}\left(\tau_{i}\left(m\right)\right)}\right],
\theta_{i}\left(z\right)-zP_{i}
\end{eqnarray*}

entonces 

\begin{eqnarray*}
\esp\left[S_{i}\right]&=&\frac{\esp\left[L_{i}^{*}\right]}{1-\mu_{i}}=\frac{f_{i}\left(i\right)}{1-\mu_{i}},\\
Var\left[S_{i}\right]&=&\frac{Var\left[L_{i}^{*}\right]}{\left(1-\mu_{i}\right)^{2}}+\frac{\sigma^{2}\esp\left[L_{i}^{*}\right]}{\left(1-\mu_{i}\right)^{3}}
\end{eqnarray*}

donde recordemos que

\begin{eqnarray*}
Var\left[L_{i}^{*}\right]&=&f_{i}\left(i,i\right)+f_{i}\left(i\right)-f_{i}\left(i\right)^{2}.
\end{eqnarray*}

 por tanto


\begin{eqnarray*}
\esp\left[I_{i}\right]&=&\frac{f_{i}\left(i\right)}{\mu_{i}},
\end{eqnarray*}
adem\'as

\begin{eqnarray*}
Var\left[I_{i}\right]&=&\frac{Var\left[L_{i}^{*}\right]}{\mu_{i}^{2}}-\frac{\sigma_{i}^{2}}{\mu_{i}^{2}}f_{i}\left(i\right).
\end{eqnarray*}


Si  $C_{i}\left(z\right)=\esp\left[z^{\overline{\tau}\left(m+1\right)-\overline{\tau}_{i}\left(m\right)}\right]$el tiempo de duraci\'on del ciclo, entonces, por lo hasta ahora establecido, se tiene que

\begin{eqnarray*}
C_{i}\left(z\right)=I_{i}\left[\theta_{i}\left(z\right)\right],
\end{eqnarray*}
entonces

\begin{eqnarray*}
\esp\left[C_{i}\right]&=&\esp\left[I_{i}\right]\esp\left[\theta_{i}\left(z\right)\right]=\frac{\esp\left[L_{i}^{*}\right]}{\mu_{i}}\frac{1}{1-\mu_{i}}=\frac{f_{i}\left(i\right)}{\mu_{i}\left(1-\mu_{i}\right)}\\
Var\left[C_{i}\right]&=&\frac{Var\left[L_{i}^{*}\right]}{\mu_{i}^{2}\left(1-\mu_{i}\right)^{2}}.
\end{eqnarray*}

Por tanto se tienen las siguientes igualdades


\begin{eqnarray*}
\esp\left[S_{i}\right]&=&\mu_{i}\esp\left[C_{i}\right],\\
\esp\left[I_{i}\right]&=&\left(1-\mu_{i}\right)\esp\left[C_{i}\right]\\
\end{eqnarray*}

Def\'inanse los puntos de regenaraci\'on  en el proceso $\left[L_{1}\left(t\right),L_{2}\left(t\right),\ldots,L_{N}\left(t\right)\right]$. Los puntos cuando la cola $i$ es visitada y todos los $L_{j}\left(\tau_{i}\left(m\right)\right)=0$ para $i=1,2$  son puntos de regeneraci\'on. Se llama ciclo regenerativo al intervalo entre dos puntos regenerativos sucesivos.

Sea $M_{i}$  el n\'umero de ciclos de visita en un ciclo regenerativo, y sea $C_{i}^{(m)}$, para $m=1,2,\ldots,M_{i}$ la duraci\'on del $m$-\'esimo ciclo de visita en un ciclo regenerativo. Se define el ciclo del tiempo de visita promedio $\esp\left[C_{i}\right]$ como

\begin{eqnarray*}
\esp\left[C_{i}\right]&=&\frac{\esp\left[\sum_{m=1}^{M_{i}}C_{i}^{(m)}\right]}{\esp\left[M_{i}\right]}
\end{eqnarray*}


En Stid72 y Heym82 se muestra que una condici\'on suficiente para que el proceso regenerativo 
estacionario sea un procesoo estacionario es que el valor esperado del tiempo del ciclo regenerativo sea finito:

\begin{eqnarray*}
\esp\left[\sum_{m=1}^{M_{i}}C_{i}^{(m)}\right]<\infty.
\end{eqnarray*}

como cada $C_{i}^{(m)}$ contiene intervalos de r\'eplica positivos, se tiene que $\esp\left[M_{i}\right]<\infty$, adem\'as, como $M_{i}>0$, se tiene que la condici\'on anterior es equivalente a tener que 

\begin{eqnarray*}
\esp\left[C_{i}\right]<\infty,
\end{eqnarray*}
por lo tanto una condici\'on suficiente para la existencia del proceso regenerativo est\'a dada por

\begin{eqnarray*}
\sum_{k=1}^{N}\mu_{k}<1.
\end{eqnarray*}

Sea la funci\'on generadora de momentos para $L_{i}$, el n\'umero de usuarios en la cola $Q_{i}\left(z\right)$ en cualquier momento, est\'a dada por el tiempo promedio de $z^{L_{i}\left(t\right)}$ sobre el ciclo regenerativo definido anteriormente:

\begin{eqnarray*}
Q_{i}\left(z\right)&=&\esp\left[z^{L_{i}\left(t\right)}\right]=\frac{\esp\left[\sum_{m=1}^{M_{i}}\sum_{t=\tau_{i}\left(m\right)}^{\tau_{i}\left(m+1\right)-1}z^{L_{i}\left(t\right)}\right]}{\esp\left[\sum_{m=1}^{M_{i}}\tau_{i}\left(m+1\right)-\tau_{i}\left(m\right)\right]}
\end{eqnarray*}

$M_{i}$ es un tiempo de paro en el proceso regenerativo con $\esp\left[M_{i}\right]<\infty$, se sigue del lema de Wald que:


\begin{eqnarray*}
\esp\left[\sum_{m=1}^{M_{i}}\sum_{t=\tau_{i}\left(m\right)}^{\tau_{i}\left(m+1\right)-1}z^{L_{i}\left(t\right)}\right]&=&\esp\left[M_{i}\right]\esp\left[\sum_{t=\tau_{i}\left(m\right)}^{\tau_{i}\left(m+1\right)-1}z^{L_{i}\left(t\right)}\right]\\
\esp\left[\sum_{m=1}^{M_{i}}\tau_{i}\left(m+1\right)-\tau_{i}\left(m\right)\right]&=&\esp\left[M_{i}\right]\esp\left[\tau_{i}\left(m+1\right)-\tau_{i}\left(m\right)\right]
\end{eqnarray*}

por tanto se tiene que


\begin{eqnarray*}
Q_{i}\left(z\right)&=&\frac{\esp\left[\sum_{t=\tau_{i}\left(m\right)}^{\tau_{i}\left(m+1\right)-1}z^{L_{i}\left(t\right)}\right]}{\esp\left[\tau_{i}\left(m+1\right)-\tau_{i}\left(m\right)\right]}
\end{eqnarray*}

observar que el denominador es simplemente la duraci\'on promedio del tiempo del ciclo.


Se puede demostrar (ver Hideaki Takagi 1986) que

\begin{eqnarray*}
\esp\left[\sum_{t=\tau_{i}\left(m\right)}^{\tau_{i}\left(m+1\right)-1}z^{L_{i}\left(t\right)}\right]=z\frac{F_{i}\left(z\right)-1}{z-P_{i}\left(z\right)}
\end{eqnarray*}

Durante el tiempo de intervisita para la cola $i$, $L_{i}\left(t\right)$ solamente se incrementa de manera que el incremento por intervalo de tiempo est\'a dado por la funci\'on generadora de probabilidades de $P_{i}\left(z\right)$, por tanto la suma sobre el tiempo de intervisita puede evaluarse como:

\begin{eqnarray*}
\esp\left[\sum_{t=\tau_{i}\left(m\right)}^{\tau_{i}\left(m+1\right)-1}z^{L_{i}\left(t\right)}\right]&=&\esp\left[\sum_{t=\tau_{i}\left(m\right)}^{\tau_{i}\left(m+1\right)-1}\left\{P_{i}\left(z\right)\right\}^{t-\overline{\tau}_{i}\left(m\right)}\right]=\frac{1-\esp\left[\left\{P_{i}\left(z\right)\right\}^{\tau_{i}\left(m+1\right)-\overline{\tau}_{i}\left(m\right)}\right]}{1-P_{i}\left(z\right)}\\
&=&\frac{1-I_{i}\left[P_{i}\left(z\right)\right]}{1-P_{i}\left(z\right)}
\end{eqnarray*}
por tanto

\begin{eqnarray*}
\esp\left[\sum_{t=\tau_{i}\left(m\right)}^{\tau_{i}\left(m+1\right)-1}z^{L_{i}\left(t\right)}\right]&=&\frac{1-F_{i}\left(z\right)}{1-P_{i}\left(z\right)}
\end{eqnarray*}

Haciendo uso de lo hasta ahora desarrollado se tiene que

\begin{eqnarray*}
Q_{i}\left(z\right)&=&\frac{1}{\esp\left[C_{i}\right]}\cdot\frac{1-F_{i}\left(z\right)}{P_{i}\left(z\right)-z}\cdot\frac{\left(1-z\right)P_{i}\left(z\right)}{1-P_{i}\left(z\right)}\\
&=&\frac{\mu_{i}\left(1-\mu_{i}\right)}{f_{i}\left(i\right)}\cdot\frac{1-F_{i}\left(z\right)}{P_{i}\left(z\right)-z}\cdot\frac{\left(1-z\right)P_{i}\left(z\right)}{1-P_{i}\left(z\right)}
\end{eqnarray*}

derivando con respecto a $z$



\begin{eqnarray*}
\frac{d Q_{i}\left(z\right)}{d z}&=&\frac{\left(1-F_{i}\left(z\right)\right)P_{i}\left(z\right)}{\esp\left[C_{i}\right]\left(1-P_{i}\left(z\right)\right)\left(P_{i}\left(z\right)-z\right)}\\
&-&\frac{\left(1-z\right)P_{i}\left(z\right)F_{i}^{'}\left(z\right)}{\esp\left[C_{i}\right]\left(1-P_{i}\left(z\right)\right)\left(P_{i}\left(z\right)-z\right)}\\
&-&\frac{\left(1-z\right)\left(1-F_{i}\left(z\right)\right)P_{i}\left(z\right)\left(P_{i}^{'}\left(z\right)-1\right)}{\esp\left[C_{i}\right]\left(1-P_{i}\left(z\right)\right)\left(P_{i}\left(z\right)-z\right)^{2}}\\
&+&\frac{\left(1-z\right)\left(1-F_{i}\left(z\right)\right)P_{i}^{'}\left(z\right)}{\esp\left[C_{i}\right]\left(1-P_{i}\left(z\right)\right)\left(P_{i}\left(z\right)-z\right)}\\
&+&\frac{\left(1-z\right)\left(1-F_{i}\left(z\right)\right)P_{i}\left(z\right)P_{i}^{'}\left(z\right)}{\esp\left[C_{i}\right]\left(1-P_{i}\left(z\right)\right)^{2}\left(P_{i}\left(z\right)-z\right)}
\end{eqnarray*}

Calculando el l\'imite cuando $z\rightarrow1^{+}$:
\begin{eqnarray}
Q_{i}^{(1)}\left(z\right)=\lim_{z\rightarrow1^{+}}\frac{d Q_{i}\left(z\right)}{dz}&=&\lim_{z\rightarrow1}\frac{\left(1-F_{i}\left(z\right)\right)P_{i}\left(z\right)}{\esp\left[C_{i}\right]\left(1-P_{i}\left(z\right)\right)\left(P_{i}\left(z\right)-z\right)}\\
&-&\lim_{z\rightarrow1^{+}}\frac{\left(1-z\right)P_{i}\left(z\right)F_{i}^{'}\left(z\right)}{\esp\left[C_{i}\right]\left(1-P_{i}\left(z\right)\right)\left(P_{i}\left(z\right)-z\right)}\\
&-&\lim_{z\rightarrow1^{+}}\frac{\left(1-z\right)\left(1-F_{i}\left(z\right)\right)P_{i}\left(z\right)\left(P_{i}^{'}\left(z\right)-1\right)}{\esp\left[C_{i}\right]\left(1-P_{i}\left(z\right)\right)\left(P_{i}\left(z\right)-z\right)^{2}}\\
&+&\lim_{z\rightarrow1^{+}}\frac{\left(1-z\right)\left(1-F_{i}\left(z\right)\right)P_{i}^{'}\left(z\right)}{\esp\left[C_{i}\right]\left(1-P_{i}\left(z\right)\right)\left(P_{i}\left(z\right)-z\right)}\\
&+&\lim_{z\rightarrow1^{+}}\frac{\left(1-z\right)\left(1-F_{i}\left(z\right)\right)P_{i}\left(z\right)P_{i}^{'}\left(z\right)}{\esp\left[C_{i}\right]\left(1-P_{i}\left(z\right)\right)^{2}\left(P_{i}\left(z\right)-z\right)}
\end{eqnarray}

Entonces:
%______________________________________________________

\begin{eqnarray*}
\lim_{z\rightarrow1^{+}}\frac{\left(1-F_{i}\left(z\right)\right)P_{i}\left(z\right)}{\left(1-P_{i}\left(z\right)\right)\left(P_{i}\left(z\right)-z\right)}&=&\lim_{z\rightarrow1^{+}}\frac{\frac{d}{dz}\left[\left(1-F_{i}\left(z\right)\right)P_{i}\left(z\right)\right]}{\frac{d}{dz}\left[\left(1-P_{i}\left(z\right)\right)\left(-z+P_{i}\left(z\right)\right)\right]}\\
&=&\lim_{z\rightarrow1^{+}}\frac{-P_{i}\left(z\right)F_{i}^{'}\left(z\right)+\left(1-F_{i}\left(z\right)\right)P_{i}^{'}\left(z\right)}{\left(1-P_{i}\left(z\right)\right)\left(-1+P_{i}^{'}\left(z\right)\right)-\left(-z+P_{i}\left(z\right)\right)P_{i}^{'}\left(z\right)}
\end{eqnarray*}


%______________________________________________________


\begin{eqnarray*}
\lim_{z\rightarrow1^{+}}\frac{\left(1-z\right)P_{i}\left(z\right)F_{i}^{'}\left(z\right)}{\left(1-P_{i}\left(z\right)\right)\left(P_{i}\left(z\right)-z\right)}&=&\lim_{z\rightarrow1^{+}}\frac{\frac{d}{dz}\left[\left(1-z\right)P_{i}\left(z\right)F_{i}^{'}\left(z\right)\right]}{\frac{d}{dz}\left[\left(1-P_{i}\left(z\right)\right)\left(P_{i}\left(z\right)-z\right)\right]}\\
&=&\lim_{z\rightarrow1^{+}}\frac{-P_{i}\left(z\right) F_{i}^{'}\left(z\right)+(1-z) F_{i}^{'}\left(z\right) P_{i}^{'}\left(z\right)+(1-z) P_{i}\left(z\right)F_{i}^{''}\left(z\right)}{\left(1-P_{i}\left(z\right)\right)\left(-1+P_{i}^{'}\left(z\right)\right)-\left(-z+P_{i}\left(z\right)\right)P_{i}^{'}\left(z\right)}
\end{eqnarray*}


%______________________________________________________

\begin{eqnarray*}
&&\lim_{z\rightarrow1^{+}}\frac{\left(1-z\right)\left(1-F_{i}\left(z\right)\right)P_{i}\left(z\right)\left(P_{i}^{'}\left(z\right)-1\right)}{\left(1-P_{i}\left(z\right)\right)\left(P_{i}\left(z\right)-z\right)^{2}}=\lim_{z\rightarrow1^{+}}\frac{\frac{d}{dz}\left[\left(1-z\right)\left(1-F_{i}\left(z\right)\right)P_{i}\left(z\right)\left(P_{i}^{'}\left(z\right)-1\right)\right]}{\frac{d}{dz}\left[\left(1-P_{i}\left(z\right)\right)\left(P_{i}\left(z\right)-z\right)^{2}\right]}\\
&=&\lim_{z\rightarrow1^{+}}\frac{-\left(1-F_{i}\left(z\right)\right) P_{i}\left(z\right)\left(-1+P_{i}^{'}\left(z\right)\right)-(1-z) P_{i}\left(z\right)F_{i}^{'}\left(z\right)\left(-1+P_{i}^{'}\left(z\right)\right)}{2\left(1-P_{i}\left(z\right)\right)\left(-z+P_{i}\left(z\right)\right) \left(-1+P_{i}^{'}\left(z\right)\right)-\left(-z+P_{i}\left(z\right)\right)^2 P_{i}^{'}\left(z\right)}\\
&+&\lim_{z\rightarrow1^{+}}\frac{+(1-z) \left(1-F_{i}\left(z\right)\right) \left(-1+P_{i}^{'}\left(z\right)\right) P_{i}^{'}\left(z\right)}{{2\left(1-P_{i}\left(z\right)\right)\left(-z+P_{i}\left(z\right)\right) \left(-1+P_{i}^{'}\left(z\right)\right)-\left(-z+P_{i}\left(z\right)\right)^2 P_{i}^{'}\left(z\right)}}\\
&+&\lim_{z\rightarrow1^{+}}\frac{+(1-z) \left(1-F_{i}\left(z\right)\right) P_{i}\left(z\right)P_{i}^{''}\left(z\right)}{{2\left(1-P_{i}\left(z\right)\right)\left(-z+P_{i}\left(z\right)\right) \left(-1+P_{i}^{'}\left(z\right)\right)-\left(-z+P_{i}\left(z\right)\right)^2 P_{i}^{'}\left(z\right)}}
\end{eqnarray*}











%______________________________________________________
\begin{eqnarray*}
&&\lim_{z\rightarrow1^{+}}\frac{\left(1-z\right)\left(1-F_{i}\left(z\right)\right)P_{i}^{'}\left(z\right)}{\left(1-P_{i}\left(z\right)\right)\left(P_{i}\left(z\right)-z\right)}=\lim_{z\rightarrow1^{+}}\frac{\frac{d}{dz}\left[\left(1-z\right)\left(1-F_{i}\left(z\right)\right)P_{i}^{'}\left(z\right)\right]}{\frac{d}{dz}\left[\left(1-P_{i}\left(z\right)\right)\left(P_{i}\left(z\right)-z\right)\right]}\\
&=&\lim_{z\rightarrow1^{+}}\frac{-\left(1-F_{i}\left(z\right)\right) P_{i}^{'}\left(z\right)-(1-z) F_{i}^{'}\left(z\right) P_{i}^{'}\left(z\right)+(1-z) \left(1-F_{i}\left(z\right)\right) P_{i}^{''}\left(z\right)}{\left(1-P_{i}\left(z\right)\right) \left(-1+P_{i}^{'}\left(z\right)\right)-\left(-z+P_{i}\left(z\right)\right) P_{i}^{'}\left(z\right)}\frac{}{}
\end{eqnarray*}

%______________________________________________________
\begin{eqnarray*}
&&\lim_{z\rightarrow1^{+}}\frac{\left(1-z\right)\left(1-F_{i}\left(z\right)\right)P_{i}\left(z\right)P_{i}^{'}\left(z\right)}{\left(1-P_{i}\left(z\right)\right)^{2}\left(P_{i}\left(z\right)-z\right)}=\lim_{z\rightarrow1^{+}}\frac{\frac{d}{dz}\left[\left(1-z\right)\left(1-F_{i}\left(z\right)\right)P_{i}\left(z\right)P_{i}^{'}\left(z\right)\right]}{\frac{d}{dz}\left[\left(1-P_{i}\left(z\right)\right)^{2}\left(P_{i}\left(z\right)-z\right)\right]}\\
&=&\lim_{z\rightarrow1^{+}}\frac{-\left(1-F_{i}\left(z\right)\right) P_{i}\left(z\right) P_{i}^{'}\left(z\right)-(1-z) P_{i}\left(z\right) F_{i}^{'}\left(z\right)P_i'[z]}{\left(1-P_{i}\left(z\right)\right)^2 \left(-1+P_{i}^{'}\left(z\right)\right)-2 \left(1-P_{i}\left(z\right)\right) \left(-z+P_{i}\left(z\right)\right) P_{i}^{'}\left(z\right)}\\
&+&\lim_{z\rightarrow1^{+}}\frac{(1-z) \left(1-F_{i}\left(z\right)\right) P_{i}^{'}\left(z\right)^2+(1-z) \left(1-F_{i}\left(z\right)\right) P_{i}\left(z\right) P_{i}^{''}\left(z\right)}{\left(1-P_{i}\left(z\right)\right)^2 \left(-1+P_{i}^{'}\left(z\right)\right)-2 \left(1-P_{i}\left(z\right)\right) \left(-z+P_{i}\left(z\right)\right) P_{i}^{'}\left(z\right)}\\
\end{eqnarray*}
%___________________________________________________________________________________________
%\subsection{Tiempos de Ciclo e Intervisita}
%___________________________________________________________________________________________


\begin{Def}
Sea $L_{i}^{*}$el n\'umero de usuarios en la cola $Q_{i}$ cuando es visitada por el servidor para dar servicio, entonces

\begin{eqnarray}
\esp\left[L_{i}^{*}\right]&=&f_{i}\left(i\right)\\
Var\left[L_{i}^{*}\right]&=&f_{i}\left(i,i\right)+\esp\left[L_{i}^{*}\right]-\esp\left[L_{i}^{*}\right]^{2}.
\end{eqnarray}

\end{Def}

\begin{Def}
El tiempo de Ciclo $C_{i}$ es e periodo de tiempo que comienza cuando la cola $i$ es visitada por primera vez en un ciclo, y termina cuando es visitado nuevamente en el pr\'oximo ciclo. La duraci\'on del mismo est\'a dada por $\tau_{i}\left(m+1\right)-\tau_{i}\left(m\right)$, o equivalentemente $\overline{\tau}_{i}\left(m+1\right)-\overline{\tau}_{i}\left(m\right)$ bajo condiciones de estabilidad.
\end{Def}

\begin{Def}
El tiempo de intervisita $I_{i}$ es el periodo de tiempo que comienza cuando se ha completado el servicio en un ciclo y termina cuando es visitada nuevamente en el pr\'oximo ciclo. Su  duraci\'on del mismo est\'a dada por $\tau_{i}\left(m+1\right)-\overline{\tau}_{i}\left(m\right)$.
\end{Def}


Recordemos las siguientes expresiones:

\begin{eqnarray*}
S_{i}\left(z\right)&=&\esp\left[z^{\overline{\tau}_{i}\left(m\right)-\tau_{i}\left(m\right)}\right]=F_{i}\left(\theta\left(z\right)\right),\\
F\left(z\right)&=&\esp\left[z^{L_{0}}\right],\\
P\left(z\right)&=&\esp\left[z^{X_{n}}\right],\\
F_{i}\left(z\right)&=&\esp\left[z^{L_{i}\left(\tau_{i}\left(m\right)\right)}\right],
\theta_{i}\left(z\right)-zP_{i}
\end{eqnarray*}

entonces 

\begin{eqnarray*}
\esp\left[S_{i}\right]&=&\frac{\esp\left[L_{i}^{*}\right]}{1-\mu_{i}}=\frac{f_{i}\left(i\right)}{1-\mu_{i}},\\
Var\left[S_{i}\right]&=&\frac{Var\left[L_{i}^{*}\right]}{\left(1-\mu_{i}\right)^{2}}+\frac{\sigma^{2}\esp\left[L_{i}^{*}\right]}{\left(1-\mu_{i}\right)^{3}}
\end{eqnarray*}

donde recordemos que

\begin{eqnarray*}
Var\left[L_{i}^{*}\right]&=&f_{i}\left(i,i\right)+f_{i}\left(i\right)-f_{i}\left(i\right)^{2}.
\end{eqnarray*}

La duraci\'on del tiempo de intervisita es $\tau_{i}\left(m+1\right)-\overline{\tau}\left(m\right)$. Dado que el n\'umero de usuarios presentes en $Q_{i}$ al tiempo $t=\tau_{i}\left(m+1\right)$ es igual al n\'umero de arribos durante el intervalo de tiempo $\left[\overline{\tau}\left(m\right),\tau_{i}\left(m+1\right)\right]$ se tiene que


\begin{eqnarray*}
\esp\left[z_{i}^{L_{i}\left(\tau_{i}\left(m+1\right)\right)}\right]=\esp\left[\left\{P_{i}\left(z_{i}\right)\right\}^{\tau_{i}\left(m+1\right)-\overline{\tau}\left(m\right)}\right]
\end{eqnarray*}

entonces, si \begin{eqnarray*}I_{i}\left(z\right)&=&\esp\left[z^{\tau_{i}\left(m+1\right)-\overline{\tau}\left(m\right)}\right]\end{eqnarray*} se tienen que

\begin{eqnarray*}
F_{i}\left(z\right)=I_{i}\left[P_{i}\left(z\right)\right]
\end{eqnarray*}
para $i=1,2$, por tanto



\begin{eqnarray*}
\esp\left[L_{i}^{*}\right]&=&\mu_{i}\esp\left[I_{i}\right]\\
Var\left[L_{i}^{*}\right]&=&\mu_{i}^{2}Var\left[I_{i}\right]+\sigma^{2}\esp\left[I_{i}\right]
\end{eqnarray*}
para $i=1,2$, por tanto


\begin{eqnarray*}
\esp\left[I_{i}\right]&=&\frac{f_{i}\left(i\right)}{\mu_{i}},
\end{eqnarray*}
adem\'as

\begin{eqnarray*}
Var\left[I_{i}\right]&=&\frac{Var\left[L_{i}^{*}\right]}{\mu_{i}^{2}}-\frac{\sigma_{i}^{2}}{\mu_{i}^{2}}f_{i}\left(i\right).
\end{eqnarray*}


Si  $C_{i}\left(z\right)=\esp\left[z^{\overline{\tau}\left(m+1\right)-\overline{\tau}_{i}\left(m\right)}\right]$el tiempo de duraci\'on del ciclo, entonces, por lo hasta ahora establecido, se tiene que

\begin{eqnarray*}
C_{i}\left(z\right)=I_{i}\left[\theta_{i}\left(z\right)\right],
\end{eqnarray*}
entonces

\begin{eqnarray*}
\esp\left[C_{i}\right]&=&\esp\left[I_{i}\right]\esp\left[\theta_{i}\left(z\right)\right]=\frac{\esp\left[L_{i}^{*}\right]}{\mu_{i}}\frac{1}{1-\mu_{i}}=\frac{f_{i}\left(i\right)}{\mu_{i}\left(1-\mu_{i}\right)}\\
Var\left[C_{i}\right]&=&\frac{Var\left[L_{i}^{*}\right]}{\mu_{i}^{2}\left(1-\mu_{i}\right)^{2}}.
\end{eqnarray*}

Por tanto se tienen las siguientes igualdades


\begin{eqnarray*}
\esp\left[S_{i}\right]&=&\mu_{i}\esp\left[C_{i}\right],\\
\esp\left[I_{i}\right]&=&\left(1-\mu_{i}\right)\esp\left[C_{i}\right]\\
\end{eqnarray*}

Def\'inanse los puntos de regenaraci\'on  en el proceso $\left[L_{1}\left(t\right),L_{2}\left(t\right),\ldots,L_{N}\left(t\right)\right]$. Los puntos cuando la cola $i$ es visitada y todos los $L_{j}\left(\tau_{i}\left(m\right)\right)=0$ para $i=1,2$  son puntos de regeneraci\'on. Se llama ciclo regenerativo al intervalo entre dos puntos regenerativos sucesivos.

Sea $M_{i}$  el n\'umero de ciclos de visita en un ciclo regenerativo, y sea $C_{i}^{(m)}$, para $m=1,2,\ldots,M_{i}$ la duraci\'on del $m$-\'esimo ciclo de visita en un ciclo regenerativo. Se define el ciclo del tiempo de visita promedio $\esp\left[C_{i}\right]$ como

\begin{eqnarray*}
\esp\left[C_{i}\right]&=&\frac{\esp\left[\sum_{m=1}^{M_{i}}C_{i}^{(m)}\right]}{\esp\left[M_{i}\right]}
\end{eqnarray*}


En Stid72 y Heym82 se muestra que una condici\'on suficiente para que el proceso regenerativo 
estacionario sea un procesoo estacionario es que el valor esperado del tiempo del ciclo regenerativo sea finito:

\begin{eqnarray*}
\esp\left[\sum_{m=1}^{M_{i}}C_{i}^{(m)}\right]<\infty.
\end{eqnarray*}

como cada $C_{i}^{(m)}$ contiene intervalos de r\'eplica positivos, se tiene que $\esp\left[M_{i}\right]<\infty$, adem\'as, como $M_{i}>0$, se tiene que la condici\'on anterior es equivalente a tener que 

\begin{eqnarray*}
\esp\left[C_{i}\right]<\infty,
\end{eqnarray*}
por lo tanto una condici\'on suficiente para la existencia del proceso regenerativo est\'a dada por

\begin{eqnarray*}
\sum_{k=1}^{N}\mu_{k}<1.
\end{eqnarray*}

Sea la funci\'on generadora de momentos para $L_{i}$, el n\'umero de usuarios en la cola $Q_{i}\left(z\right)$ en cualquier momento, est\'a dada por el tiempo promedio de $z^{L_{i}\left(t\right)}$ sobre el ciclo regenerativo definido anteriormente:

\begin{eqnarray*}
Q_{i}\left(z\right)&=&\esp\left[z^{L_{i}\left(t\right)}\right]=\frac{\esp\left[\sum_{m=1}^{M_{i}}\sum_{t=\tau_{i}\left(m\right)}^{\tau_{i}\left(m+1\right)-1}z^{L_{i}\left(t\right)}\right]}{\esp\left[\sum_{m=1}^{M_{i}}\tau_{i}\left(m+1\right)-\tau_{i}\left(m\right)\right]}
\end{eqnarray*}

$M_{i}$ es un tiempo de paro en el proceso regenerativo con $\esp\left[M_{i}\right]<\infty$, se sigue del lema de Wald que:


\begin{eqnarray*}
\esp\left[\sum_{m=1}^{M_{i}}\sum_{t=\tau_{i}\left(m\right)}^{\tau_{i}\left(m+1\right)-1}z^{L_{i}\left(t\right)}\right]&=&\esp\left[M_{i}\right]\esp\left[\sum_{t=\tau_{i}\left(m\right)}^{\tau_{i}\left(m+1\right)-1}z^{L_{i}\left(t\right)}\right]\\
\esp\left[\sum_{m=1}^{M_{i}}\tau_{i}\left(m+1\right)-\tau_{i}\left(m\right)\right]&=&\esp\left[M_{i}\right]\esp\left[\tau_{i}\left(m+1\right)-\tau_{i}\left(m\right)\right]
\end{eqnarray*}

por tanto se tiene que


\begin{eqnarray*}
Q_{i}\left(z\right)&=&\frac{\esp\left[\sum_{t=\tau_{i}\left(m\right)}^{\tau_{i}\left(m+1\right)-1}z^{L_{i}\left(t\right)}\right]}{\esp\left[\tau_{i}\left(m+1\right)-\tau_{i}\left(m\right)\right]}
\end{eqnarray*}

observar que el denominador es simplemente la duraci\'on promedio del tiempo del ciclo.


Se puede demostrar (ver Hideaki Takagi 1986) que

\begin{eqnarray*}
\esp\left[\sum_{t=\tau_{i}\left(m\right)}^{\tau_{i}\left(m+1\right)-1}z^{L_{i}\left(t\right)}\right]=z\frac{F_{i}\left(z\right)-1}{z-P_{i}\left(z\right)}
\end{eqnarray*}

Durante el tiempo de intervisita para la cola $i$, $L_{i}\left(t\right)$ solamente se incrementa de manera que el incremento por intervalo de tiempo est\'a dado por la funci\'on generadora de probabilidades de $P_{i}\left(z\right)$, por tanto la suma sobre el tiempo de intervisita puede evaluarse como:

\begin{eqnarray*}
\esp\left[\sum_{t=\tau_{i}\left(m\right)}^{\tau_{i}\left(m+1\right)-1}z^{L_{i}\left(t\right)}\right]&=&\esp\left[\sum_{t=\tau_{i}\left(m\right)}^{\tau_{i}\left(m+1\right)-1}\left\{P_{i}\left(z\right)\right\}^{t-\overline{\tau}_{i}\left(m\right)}\right]=\frac{1-\esp\left[\left\{P_{i}\left(z\right)\right\}^{\tau_{i}\left(m+1\right)-\overline{\tau}_{i}\left(m\right)}\right]}{1-P_{i}\left(z\right)}\\
&=&\frac{1-I_{i}\left[P_{i}\left(z\right)\right]}{1-P_{i}\left(z\right)}
\end{eqnarray*}
por tanto

\begin{eqnarray*}
\esp\left[\sum_{t=\tau_{i}\left(m\right)}^{\tau_{i}\left(m+1\right)-1}z^{L_{i}\left(t\right)}\right]&=&\frac{1-F_{i}\left(z\right)}{1-P_{i}\left(z\right)}
\end{eqnarray*}

Haciendo uso de lo hasta ahora desarrollado se tiene que

\begin{eqnarray*}
Q_{i}\left(z\right)&=&\frac{1}{\esp\left[C_{i}\right]}\cdot\frac{1-F_{i}\left(z\right)}{P_{i}\left(z\right)-z}\cdot\frac{\left(1-z\right)P_{i}\left(z\right)}{1-P_{i}\left(z\right)}\\
&=&\frac{\mu_{i}\left(1-\mu_{i}\right)}{f_{i}\left(i\right)}\cdot\frac{1-F_{i}\left(z\right)}{P_{i}\left(z\right)-z}\cdot\frac{\left(1-z\right)P_{i}\left(z\right)}{1-P_{i}\left(z\right)}
\end{eqnarray*}


%___________________________________________________________________________________________
%\subsection{Longitudes de la Cola en cualquier tiempo}
%___________________________________________________________________________________________

Sea
$V_{i}\left(z\right)=\frac{1}{\esp\left[C_{i}\right]}\frac{I_{i}\left(z\right)-1}{z-P_{i}\left(z\right)}$

%{\esp\lef[I_{i}\right]}\frac{1-\mu_{i}}{z-P_{i}\left(z\right)}

\begin{eqnarray*}
\frac{\partial V_{i}\left(z\right)}{\partial z}&=&\frac{1}{\esp\left[C_{i}\right]}\left[\frac{I_{i}{'}\left(z\right)\left(z-P_{i}\left(z\right)\right)}{z-P_{i}\left(z\right)}-\frac{\left(I_{i}\left(z\right)-1\right)\left(1-P_{i}{'}\left(z\right)\right)}{\left(z-P_{i}\left(z\right)\right)^{2}}\right]
\end{eqnarray*}


La FGP para el tiempo de espera para cualquier usuario en la cola est\'a dada por:
\[U_{i}\left(z\right)=\frac{1}{\esp\left[C_{i}\right]}\cdot\frac{1-P_{i}\left(z\right)}{z-P_{i}\left(z\right)}\cdot\frac{I_{i}\left(z\right)-1}{1-z}\]

entonces


\begin{eqnarray*}
\frac{d}{dz}V_{i}\left(z\right)&=&\frac{1}{\esp\left[C_{i}\right]}\left\{\frac{d}{dz}\left(\frac{1-P_{i}\left(z\right)}{z-P_{i}\left(z\right)}\right)\frac{I_{i}\left(z\right)-1}{1-z}+\frac{1-P_{i}\left(z\right)}{z-P_{i}\left(z\right)}\frac{d}{dz}\left(\frac{I_{i}\left(z\right)-1}{1-z}\right)\right\}\\
&=&\frac{1}{\esp\left[C_{i}\right]}\left\{\frac{-P_{i}\left(z\right)\left(z-P_{i}\left(z\right)\right)-\left(1-P_{i}\left(z\right)\right)\left(1-P_{i}^{'}\left(z\right)\right)}{\left(z-P_{i}\left(z\right)\right)^{2}}\cdot\frac{I_{i}\left(z\right)-1}{1-z}\right\}\\
&+&\frac{1}{\esp\left[C_{i}\right]}\left\{\frac{1-P_{i}\left(z\right)}{z-P_{i}\left(z\right)}\cdot\frac{I_{i}^{'}\left(z\right)\left(1-z\right)+\left(I_{i}\left(z\right)-1\right)}{\left(1-z\right)^{2}}\right\}
\end{eqnarray*}
%\frac{I_{i}\left(z\right)-1}{1-z}
%+\frac{1-P_{i}\left(z\right)}{z-P_{i}\frac{d}{dz}\left(\frac{I_{i}\left(z\right)-1}{1-z}\right)


\begin{eqnarray*}
\frac{\partial U_{i}\left(z\right)}{\partial z}&=&\frac{(-1+I_{i}[z]) (1-P_{i}[z])}{(1-z)^2 \esp[I_{i}] (z-P_{i}[z])}+\frac{(1-P_{i}[z]) I_{i}^{'}[z]}{(1-z) \esp[I_{i}] (z-P_{i}[z])}-\frac{(-1+I_{i}[z]) (1-P_{i}[z])\left(1-P{'}[z]\right)}{(1-z) \esp[I_{i}] (z-P_{i}[z])^2}\\
&-&\frac{(-1+I_{i}[z]) P_{i}{'}[z]}{(1-z) \esp[I_{i}](z-P_{i}[z])}
\end{eqnarray*}
%___________________________________________________________________________________________
%\subsection{Longitudes de la Cola en cualquier tiempo}
%___________________________________________________________________________________________
Sea 
\begin{eqnarray*}
Q_{i}\left(z\right)&=&\frac{1}{\esp\left[C_{i}\right]}\cdot\frac{1-F_{i}\left(z\right)}{P_{i}\left(z\right)-z}\cdot\frac{\left(1-z\right)P_{i}\left(z\right)}{1-P_{i}\left(z\right)}
\end{eqnarray*}

derivando con respecto a $z$



\begin{eqnarray*}
\frac{d Q_{i}\left(z\right)}{d z}&=&\frac{\left(1-F_{i}\left(z\right)\right)P_{i}\left(z\right)}{\esp\left[C_{i}\right]\left(1-P_{i}\left(z\right)\right)\left(P_{i}\left(z\right)-z\right)}\\
&-&\frac{\left(1-z\right)P_{i}\left(z\right)F_{i}^{'}\left(z\right)}{\esp\left[C_{i}\right]\left(1-P_{i}\left(z\right)\right)\left(P_{i}\left(z\right)-z\right)}\\
&-&\frac{\left(1-z\right)\left(1-F_{i}\left(z\right)\right)P_{i}\left(z\right)\left(P_{i}^{'}\left(z\right)-1\right)}{\esp\left[C_{i}\right]\left(1-P_{i}\left(z\right)\right)\left(P_{i}\left(z\right)-z\right)^{2}}\\
&+&\frac{\left(1-z\right)\left(1-F_{i}\left(z\right)\right)P_{i}^{'}\left(z\right)}{\esp\left[C_{i}\right]\left(1-P_{i}\left(z\right)\right)\left(P_{i}\left(z\right)-z\right)}\\
&+&\frac{\left(1-z\right)\left(1-F_{i}\left(z\right)\right)P_{i}\left(z\right)P_{i}^{'}\left(z\right)}{\esp\left[C_{i}\right]\left(1-P_{i}\left(z\right)\right)^{2}\left(P_{i}\left(z\right)-z\right)}
\end{eqnarray*}

Calculando el l\'imite cuando $z\rightarrow1^{+}$:
\begin{eqnarray}
Q_{i}^{(1)}\left(z\right)=\lim_{z\rightarrow1^{+}}\frac{d Q_{i}\left(z\right)}{dz}&=&\lim_{z\rightarrow1}\frac{\left(1-F_{i}\left(z\right)\right)P_{i}\left(z\right)}{\esp\left[C_{i}\right]\left(1-P_{i}\left(z\right)\right)\left(P_{i}\left(z\right)-z\right)}\\
&-&\lim_{z\rightarrow1^{+}}\frac{\left(1-z\right)P_{i}\left(z\right)F_{i}^{'}\left(z\right)}{\esp\left[C_{i}\right]\left(1-P_{i}\left(z\right)\right)\left(P_{i}\left(z\right)-z\right)}\\
&-&\lim_{z\rightarrow1^{+}}\frac{\left(1-z\right)\left(1-F_{i}\left(z\right)\right)P_{i}\left(z\right)\left(P_{i}^{'}\left(z\right)-1\right)}{\esp\left[C_{i}\right]\left(1-P_{i}\left(z\right)\right)\left(P_{i}\left(z\right)-z\right)^{2}}\\
&+&\lim_{z\rightarrow1^{+}}\frac{\left(1-z\right)\left(1-F_{i}\left(z\right)\right)P_{i}^{'}\left(z\right)}{\esp\left[C_{i}\right]\left(1-P_{i}\left(z\right)\right)\left(P_{i}\left(z\right)-z\right)}\\
&+&\lim_{z\rightarrow1^{+}}\frac{\left(1-z\right)\left(1-F_{i}\left(z\right)\right)P_{i}\left(z\right)P_{i}^{'}\left(z\right)}{\esp\left[C_{i}\right]\left(1-P_{i}\left(z\right)\right)^{2}\left(P_{i}\left(z\right)-z\right)}
\end{eqnarray}

Entonces:
%______________________________________________________

\begin{eqnarray*}
\lim_{z\rightarrow1^{+}}\frac{\left(1-F_{i}\left(z\right)\right)P_{i}\left(z\right)}{\left(1-P_{i}\left(z\right)\right)\left(P_{i}\left(z\right)-z\right)}&=&\lim_{z\rightarrow1^{+}}\frac{\frac{d}{dz}\left[\left(1-F_{i}\left(z\right)\right)P_{i}\left(z\right)\right]}{\frac{d}{dz}\left[\left(1-P_{i}\left(z\right)\right)\left(-z+P_{i}\left(z\right)\right)\right]}\\
&=&\lim_{z\rightarrow1^{+}}\frac{-P_{i}\left(z\right)F_{i}^{'}\left(z\right)+\left(1-F_{i}\left(z\right)\right)P_{i}^{'}\left(z\right)}{\left(1-P_{i}\left(z\right)\right)\left(-1+P_{i}^{'}\left(z\right)\right)-\left(-z+P_{i}\left(z\right)\right)P_{i}^{'}\left(z\right)}
\end{eqnarray*}


%______________________________________________________


\begin{eqnarray*}
\lim_{z\rightarrow1^{+}}\frac{\left(1-z\right)P_{i}\left(z\right)F_{i}^{'}\left(z\right)}{\left(1-P_{i}\left(z\right)\right)\left(P_{i}\left(z\right)-z\right)}&=&\lim_{z\rightarrow1^{+}}\frac{\frac{d}{dz}\left[\left(1-z\right)P_{i}\left(z\right)F_{i}^{'}\left(z\right)\right]}{\frac{d}{dz}\left[\left(1-P_{i}\left(z\right)\right)\left(P_{i}\left(z\right)-z\right)\right]}\\
&=&\lim_{z\rightarrow1^{+}}\frac{-P_{i}\left(z\right) F_{i}^{'}\left(z\right)+(1-z) F_{i}^{'}\left(z\right) P_{i}^{'}\left(z\right)+(1-z) P_{i}\left(z\right)F_{i}^{''}\left(z\right)}{\left(1-P_{i}\left(z\right)\right)\left(-1+P_{i}^{'}\left(z\right)\right)-\left(-z+P_{i}\left(z\right)\right)P_{i}^{'}\left(z\right)}
\end{eqnarray*}


%______________________________________________________

\begin{eqnarray*}
&&\lim_{z\rightarrow1^{+}}\frac{\left(1-z\right)\left(1-F_{i}\left(z\right)\right)P_{i}\left(z\right)\left(P_{i}^{'}\left(z\right)-1\right)}{\left(1-P_{i}\left(z\right)\right)\left(P_{i}\left(z\right)-z\right)^{2}}=\lim_{z\rightarrow1^{+}}\frac{\frac{d}{dz}\left[\left(1-z\right)\left(1-F_{i}\left(z\right)\right)P_{i}\left(z\right)\left(P_{i}^{'}\left(z\right)-1\right)\right]}{\frac{d}{dz}\left[\left(1-P_{i}\left(z\right)\right)\left(P_{i}\left(z\right)-z\right)^{2}\right]}\\
&=&\lim_{z\rightarrow1^{+}}\frac{-\left(1-F_{i}\left(z\right)\right) P_{i}\left(z\right)\left(-1+P_{i}^{'}\left(z\right)\right)-(1-z) P_{i}\left(z\right)F_{i}^{'}\left(z\right)\left(-1+P_{i}^{'}\left(z\right)\right)}{2\left(1-P_{i}\left(z\right)\right)\left(-z+P_{i}\left(z\right)\right) \left(-1+P_{i}^{'}\left(z\right)\right)-\left(-z+P_{i}\left(z\right)\right)^2 P_{i}^{'}\left(z\right)}\\
&+&\lim_{z\rightarrow1^{+}}\frac{+(1-z) \left(1-F_{i}\left(z\right)\right) \left(-1+P_{i}^{'}\left(z\right)\right) P_{i}^{'}\left(z\right)}{{2\left(1-P_{i}\left(z\right)\right)\left(-z+P_{i}\left(z\right)\right) \left(-1+P_{i}^{'}\left(z\right)\right)-\left(-z+P_{i}\left(z\right)\right)^2 P_{i}^{'}\left(z\right)}}\\
&+&\lim_{z\rightarrow1^{+}}\frac{+(1-z) \left(1-F_{i}\left(z\right)\right) P_{i}\left(z\right)P_{i}^{''}\left(z\right)}{{2\left(1-P_{i}\left(z\right)\right)\left(-z+P_{i}\left(z\right)\right) \left(-1+P_{i}^{'}\left(z\right)\right)-\left(-z+P_{i}\left(z\right)\right)^2 P_{i}^{'}\left(z\right)}}
\end{eqnarray*}











%______________________________________________________
\begin{eqnarray*}
&&\lim_{z\rightarrow1^{+}}\frac{\left(1-z\right)\left(1-F_{i}\left(z\right)\right)P_{i}^{'}\left(z\right)}{\left(1-P_{i}\left(z\right)\right)\left(P_{i}\left(z\right)-z\right)}=\lim_{z\rightarrow1^{+}}\frac{\frac{d}{dz}\left[\left(1-z\right)\left(1-F_{i}\left(z\right)\right)P_{i}^{'}\left(z\right)\right]}{\frac{d}{dz}\left[\left(1-P_{i}\left(z\right)\right)\left(P_{i}\left(z\right)-z\right)\right]}\\
&=&\lim_{z\rightarrow1^{+}}\frac{-\left(1-F_{i}\left(z\right)\right) P_{i}^{'}\left(z\right)-(1-z) F_{i}^{'}\left(z\right) P_{i}^{'}\left(z\right)+(1-z) \left(1-F_{i}\left(z\right)\right) P_{i}^{''}\left(z\right)}{\left(1-P_{i}\left(z\right)\right) \left(-1+P_{i}^{'}\left(z\right)\right)-\left(-z+P_{i}\left(z\right)\right) P_{i}^{'}\left(z\right)}\frac{}{}
\end{eqnarray*}

%______________________________________________________
\begin{eqnarray*}
&&\lim_{z\rightarrow1^{+}}\frac{\left(1-z\right)\left(1-F_{i}\left(z\right)\right)P_{i}\left(z\right)P_{i}^{'}\left(z\right)}{\left(1-P_{i}\left(z\right)\right)^{2}\left(P_{i}\left(z\right)-z\right)}=\lim_{z\rightarrow1^{+}}\frac{\frac{d}{dz}\left[\left(1-z\right)\left(1-F_{i}\left(z\right)\right)P_{i}\left(z\right)P_{i}^{'}\left(z\right)\right]}{\frac{d}{dz}\left[\left(1-P_{i}\left(z\right)\right)^{2}\left(P_{i}\left(z\right)-z\right)\right]}\\
&=&\lim_{z\rightarrow1^{+}}\frac{-\left(1-F_{i}\left(z\right)\right) P_{i}\left(z\right) P_{i}^{'}\left(z\right)-(1-z) P_{i}\left(z\right) F_{i}^{'}\left(z\right)P_i'[z]}{\left(1-P_{i}\left(z\right)\right)^2 \left(-1+P_{i}^{'}\left(z\right)\right)-2 \left(1-P_{i}\left(z\right)\right) \left(-z+P_{i}\left(z\right)\right) P_{i}^{'}\left(z\right)}\\
&+&\lim_{z\rightarrow1^{+}}\frac{(1-z) \left(1-F_{i}\left(z\right)\right) P_{i}^{'}\left(z\right)^2+(1-z) \left(1-F_{i}\left(z\right)\right) P_{i}\left(z\right) P_{i}^{''}\left(z\right)}{\left(1-P_{i}\left(z\right)\right)^2 \left(-1+P_{i}^{'}\left(z\right)\right)-2 \left(1-P_{i}\left(z\right)\right) \left(-z+P_{i}\left(z\right)\right) P_{i}^{'}\left(z\right)}\\
\end{eqnarray*}




%_______________________________________________________________________________________________________
\subsection{Tiempo de Ciclo Promedio}
%_______________________________________________________________________________________________________

Consideremos una cola de la red de sistemas de visitas c\'iclicas fija, $Q_{l}$.


Conforme a la definici\'on dada al principio del cap\'itulo, definici\'on (\ref{Def.Tn}), sean $T_{1},T_{2},\ldots$ los puntos donde las longitudes de las colas de la red de sistemas de visitas c\'iclicas son cero simult\'aneamente, cuando la cola $Q_{l}$ es visitada por el servidor para dar servicio, es decir, $L_{1}\left(T_{i}\right)=0,L_{2}\left(T_{i}\right)=0,\hat{L}_{1}\left(T_{i}\right)=0$ y $\hat{L}_{2}\left(T_{i}\right)=0$, a estos puntos se les denominar\'a puntos regenerativos. Entonces, 

\begin{Def}
Al intervalo de tiempo entre dos puntos regenerativos se le llamar\'a ciclo regenerativo.
\end{Def}

\begin{Def}
Para $T_{i}$ se define, $M_{i}$, el n\'umero de ciclos de visita a la cola $Q_{l}$, durante el ciclo regenerativo, es decir, $M_{i}$ es un proceso de renovaci\'on.
\end{Def}

\begin{Def}
Para cada uno de los $M_{i}$'s, se definen a su vez la duraci\'on de cada uno de estos ciclos de visita en el ciclo regenerativo, $C_{i}^{(m)}$, para $m=1,2,\ldots,M_{i}$, que a su vez, tambi\'en es n proceso de renovaci\'on.
\end{Def}

En nuestra notaci\'on $V\left(t\right)\equiv C_{i}$ y $X_{i}=C_{i}^{(m)}$ para nuestra segunda definici\'on, mientras que para la primera la notaci\'on es: $X\left(t\right)\equiv C_{i}$ y $R_{i}\equiv C_{i}^{(m)}$.


%___________________________________________________________________________________________
\subsection{Tiempos de Ciclo e Intervisita}
%___________________________________________________________________________________________


\begin{Def}
Sea $L_{i}^{*}$el n\'umero de usuarios en la cola $Q_{i}$ cuando es visitada por el servidor para dar servicio, entonces

\begin{eqnarray}
\esp\left[L_{i}^{*}\right]&=&f_{i}\left(i\right)\\
Var\left[L_{i}^{*}\right]&=&f_{i}\left(i,i\right)+\esp\left[L_{i}^{*}\right]-\esp\left[L_{i}^{*}\right]^{2}.
\end{eqnarray}

\end{Def}

\begin{Def}
El tiempo de Ciclo $C_{i}$ es e periodo de tiempo que comienza cuando la cola $i$ es visitada por primera vez en un ciclo, y termina cuando es visitado nuevamente en el pr\'oximo ciclo. La duraci\'on del mismo est\'a dada por $\tau_{i}\left(m+1\right)-\tau_{i}\left(m\right)$, o equivalentemente $\overline{\tau}_{i}\left(m+1\right)-\overline{\tau}_{i}\left(m\right)$ bajo condiciones de estabilidad.
\end{Def}

\begin{Def}
El tiempo de intervisita $I_{i}$ es el periodo de tiempo que comienza cuando se ha completado el servicio en un ciclo y termina cuando es visitada nuevamente en el pr\'oximo ciclo. Su  duraci\'on del mismo est\'a dada por $\tau_{i}\left(m+1\right)-\overline{\tau}_{i}\left(m\right)$.
\end{Def}


Recordemos las siguientes expresiones:

\begin{eqnarray*}
S_{i}\left(z\right)&=&\esp\left[z^{\overline{\tau}_{i}\left(m\right)-\tau_{i}\left(m\right)}\right]=F_{i}\left(\theta\left(z\right)\right),\\
F\left(z\right)&=&\esp\left[z^{L_{0}}\right],\\
P\left(z\right)&=&\esp\left[z^{X_{n}}\right],\\
F_{i}\left(z\right)&=&\esp\left[z^{L_{i}\left(\tau_{i}\left(m\right)\right)}\right],
\theta_{i}\left(z\right)-zP_{i}
\end{eqnarray*}

entonces 

\begin{eqnarray*}
\esp\left[S_{i}\right]&=&\frac{\esp\left[L_{i}^{*}\right]}{1-\mu_{i}}=\frac{f_{i}\left(i\right)}{1-\mu_{i}},\\
Var\left[S_{i}\right]&=&\frac{Var\left[L_{i}^{*}\right]}{\left(1-\mu_{i}\right)^{2}}+\frac{\sigma^{2}\esp\left[L_{i}^{*}\right]}{\left(1-\mu_{i}\right)^{3}}
\end{eqnarray*}

donde recordemos que

\begin{eqnarray*}
Var\left[L_{i}^{*}\right]&=&f_{i}\left(i,i\right)+f_{i}\left(i\right)-f_{i}\left(i\right)^{2}.
\end{eqnarray*}

La duraci\'on del tiempo de intervisita es $\tau_{i}\left(m+1\right)-\overline{\tau}\left(m\right)$. Dado que el n\'umero de usuarios presentes en $Q_{i}$ al tiempo $t=\tau_{i}\left(m+1\right)$ es igual al n\'umero de arribos durante el intervalo de tiempo $\left[\overline{\tau}\left(m\right),\tau_{i}\left(m+1\right)\right]$ se tiene que


\begin{eqnarray*}
\esp\left[z_{i}^{L_{i}\left(\tau_{i}\left(m+1\right)\right)}\right]=\esp\left[\left\{P_{i}\left(z_{i}\right)\right\}^{\tau_{i}\left(m+1\right)-\overline{\tau}\left(m\right)}\right]
\end{eqnarray*}

entonces, si \begin{eqnarray*}I_{i}\left(z\right)&=&\esp\left[z^{\tau_{i}\left(m+1\right)-\overline{\tau}\left(m\right)}\right]\end{eqnarray*} se tienen que

\begin{eqnarray*}
F_{i}\left(z\right)=I_{i}\left[P_{i}\left(z\right)\right]
\end{eqnarray*}
para $i=1,2$, por tanto



\begin{eqnarray*}
\esp\left[L_{i}^{*}\right]&=&\mu_{i}\esp\left[I_{i}\right]\\
Var\left[L_{i}^{*}\right]&=&\mu_{i}^{2}Var\left[I_{i}\right]+\sigma^{2}\esp\left[I_{i}\right]
\end{eqnarray*}
para $i=1,2$, por tanto


\begin{eqnarray*}
\esp\left[I_{i}\right]&=&\frac{f_{i}\left(i\right)}{\mu_{i}},
\end{eqnarray*}
adem\'as

\begin{eqnarray*}
Var\left[I_{i}\right]&=&\frac{Var\left[L_{i}^{*}\right]}{\mu_{i}^{2}}-\frac{\sigma_{i}^{2}}{\mu_{i}^{2}}f_{i}\left(i\right).
\end{eqnarray*}


Si  $C_{i}\left(z\right)=\esp\left[z^{\overline{\tau}\left(m+1\right)-\overline{\tau}_{i}\left(m\right)}\right]$el tiempo de duraci\'on del ciclo, entonces, por lo hasta ahora establecido, se tiene que

\begin{eqnarray*}
C_{i}\left(z\right)=I_{i}\left[\theta_{i}\left(z\right)\right],
\end{eqnarray*}
entonces

\begin{eqnarray*}
\esp\left[C_{i}\right]&=&\esp\left[I_{i}\right]\esp\left[\theta_{i}\left(z\right)\right]=\frac{\esp\left[L_{i}^{*}\right]}{\mu_{i}}\frac{1}{1-\mu_{i}}=\frac{f_{i}\left(i\right)}{\mu_{i}\left(1-\mu_{i}\right)}\\
Var\left[C_{i}\right]&=&\frac{Var\left[L_{i}^{*}\right]}{\mu_{i}^{2}\left(1-\mu_{i}\right)^{2}}.
\end{eqnarray*}

Por tanto se tienen las siguientes igualdades


\begin{eqnarray*}
\esp\left[S_{i}\right]&=&\mu_{i}\esp\left[C_{i}\right],\\
\esp\left[I_{i}\right]&=&\left(1-\mu_{i}\right)\esp\left[C_{i}\right]\\
\end{eqnarray*}

Def\'inanse los puntos de regenaraci\'on  en el proceso $\left[L_{1}\left(t\right),L_{2}\left(t\right),\ldots,L_{N}\left(t\right)\right]$. Los puntos cuando la cola $i$ es visitada y todos los $L_{j}\left(\tau_{i}\left(m\right)\right)=0$ para $i=1,2$  son puntos de regeneraci\'on. Se llama ciclo regenerativo al intervalo entre dos puntos regenerativos sucesivos.

Sea $M_{i}$  el n\'umero de ciclos de visita en un ciclo regenerativo, y sea $C_{i}^{(m)}$, para $m=1,2,\ldots,M_{i}$ la duraci\'on del $m$-\'esimo ciclo de visita en un ciclo regenerativo. Se define el ciclo del tiempo de visita promedio $\esp\left[C_{i}\right]$ como

\begin{eqnarray*}
\esp\left[C_{i}\right]&=&\frac{\esp\left[\sum_{m=1}^{M_{i}}C_{i}^{(m)}\right]}{\esp\left[M_{i}\right]}
\end{eqnarray*}


En Stid72 y Heym82 se muestra que una condici\'on suficiente para que el proceso regenerativo 
estacionario sea un procesoo estacionario es que el valor esperado del tiempo del ciclo regenerativo sea finito:

\begin{eqnarray*}
\esp\left[\sum_{m=1}^{M_{i}}C_{i}^{(m)}\right]<\infty.
\end{eqnarray*}

como cada $C_{i}^{(m)}$ contiene intervalos de r\'eplica positivos, se tiene que $\esp\left[M_{i}\right]<\infty$, adem\'as, como $M_{i}>0$, se tiene que la condici\'on anterior es equivalente a tener que 

\begin{eqnarray*}
\esp\left[C_{i}\right]<\infty,
\end{eqnarray*}
por lo tanto una condici\'on suficiente para la existencia del proceso regenerativo est\'a dada por

\begin{eqnarray*}
\sum_{k=1}^{N}\mu_{k}<1.
\end{eqnarray*}

Sea la funci\'on generadora de momentos para $L_{i}$, el n\'umero de usuarios en la cola $Q_{i}\left(z\right)$ en cualquier momento, est\'a dada por el tiempo promedio de $z^{L_{i}\left(t\right)}$ sobre el ciclo regenerativo definido anteriormente:

\begin{eqnarray*}
Q_{i}\left(z\right)&=&\esp\left[z^{L_{i}\left(t\right)}\right]=\frac{\esp\left[\sum_{m=1}^{M_{i}}\sum_{t=\tau_{i}\left(m\right)}^{\tau_{i}\left(m+1\right)-1}z^{L_{i}\left(t\right)}\right]}{\esp\left[\sum_{m=1}^{M_{i}}\tau_{i}\left(m+1\right)-\tau_{i}\left(m\right)\right]}
\end{eqnarray*}

$M_{i}$ es un tiempo de paro en el proceso regenerativo con $\esp\left[M_{i}\right]<\infty$, se sigue del lema de Wald que:


\begin{eqnarray*}
\esp\left[\sum_{m=1}^{M_{i}}\sum_{t=\tau_{i}\left(m\right)}^{\tau_{i}\left(m+1\right)-1}z^{L_{i}\left(t\right)}\right]&=&\esp\left[M_{i}\right]\esp\left[\sum_{t=\tau_{i}\left(m\right)}^{\tau_{i}\left(m+1\right)-1}z^{L_{i}\left(t\right)}\right]\\
\esp\left[\sum_{m=1}^{M_{i}}\tau_{i}\left(m+1\right)-\tau_{i}\left(m\right)\right]&=&\esp\left[M_{i}\right]\esp\left[\tau_{i}\left(m+1\right)-\tau_{i}\left(m\right)\right]
\end{eqnarray*}

por tanto se tiene que


\begin{eqnarray*}
Q_{i}\left(z\right)&=&\frac{\esp\left[\sum_{t=\tau_{i}\left(m\right)}^{\tau_{i}\left(m+1\right)-1}z^{L_{i}\left(t\right)}\right]}{\esp\left[\tau_{i}\left(m+1\right)-\tau_{i}\left(m\right)\right]}
\end{eqnarray*}

observar que el denominador es simplemente la duraci\'on promedio del tiempo del ciclo.


Se puede demostrar (ver Hideaki Takagi 1986) que

\begin{eqnarray*}
\esp\left[\sum_{t=\tau_{i}\left(m\right)}^{\tau_{i}\left(m+1\right)-1}z^{L_{i}\left(t\right)}\right]=z\frac{F_{i}\left(z\right)-1}{z-P_{i}\left(z\right)}
\end{eqnarray*}

Durante el tiempo de intervisita para la cola $i$, $L_{i}\left(t\right)$ solamente se incrementa de manera que el incremento por intervalo de tiempo est\'a dado por la funci\'on generadora de probabilidades de $P_{i}\left(z\right)$, por tanto la suma sobre el tiempo de intervisita puede evaluarse como:

\begin{eqnarray*}
\esp\left[\sum_{t=\tau_{i}\left(m\right)}^{\tau_{i}\left(m+1\right)-1}z^{L_{i}\left(t\right)}\right]&=&\esp\left[\sum_{t=\tau_{i}\left(m\right)}^{\tau_{i}\left(m+1\right)-1}\left\{P_{i}\left(z\right)\right\}^{t-\overline{\tau}_{i}\left(m\right)}\right]=\frac{1-\esp\left[\left\{P_{i}\left(z\right)\right\}^{\tau_{i}\left(m+1\right)-\overline{\tau}_{i}\left(m\right)}\right]}{1-P_{i}\left(z\right)}\\
&=&\frac{1-I_{i}\left[P_{i}\left(z\right)\right]}{1-P_{i}\left(z\right)}
\end{eqnarray*}
por tanto

\begin{eqnarray*}
\esp\left[\sum_{t=\tau_{i}\left(m\right)}^{\tau_{i}\left(m+1\right)-1}z^{L_{i}\left(t\right)}\right]&=&\frac{1-F_{i}\left(z\right)}{1-P_{i}\left(z\right)}
\end{eqnarray*}

Haciendo uso de lo hasta ahora desarrollado se tiene que

\begin{eqnarray*}
Q_{i}\left(z\right)&=&\frac{1}{\esp\left[C_{i}\right]}\cdot\frac{1-F_{i}\left(z\right)}{P_{i}\left(z\right)-z}\cdot\frac{\left(1-z\right)P_{i}\left(z\right)}{1-P_{i}\left(z\right)}\\
&=&\frac{\mu_{i}\left(1-\mu_{i}\right)}{f_{i}\left(i\right)}\cdot\frac{1-F_{i}\left(z\right)}{P_{i}\left(z\right)-z}\cdot\frac{\left(1-z\right)P_{i}\left(z\right)}{1-P_{i}\left(z\right)}
\end{eqnarray*}


%___________________________________________________________________________________________
\subsection{Longitudes de la Cola en cualquier tiempo}
%___________________________________________________________________________________________
Sea 
\begin{eqnarray*}
Q_{i}\left(z\right)&=&\frac{1}{\esp\left[C_{i}\right]}\cdot\frac{1-F_{i}\left(z\right)}{P_{i}\left(z\right)-z}\cdot\frac{\left(1-z\right)P_{i}\left(z\right)}{1-P_{i}\left(z\right)}
\end{eqnarray*}

derivando con respecto a $z$



\begin{eqnarray*}
\frac{d Q_{i}\left(z\right)}{d z}&=&\frac{\left(1-F_{i}\left(z\right)\right)P_{i}\left(z\right)}{\esp\left[C_{i}\right]\left(1-P_{i}\left(z\right)\right)\left(P_{i}\left(z\right)-z\right)}\\
&-&\frac{\left(1-z\right)P_{i}\left(z\right)F_{i}^{'}\left(z\right)}{\esp\left[C_{i}\right]\left(1-P_{i}\left(z\right)\right)\left(P_{i}\left(z\right)-z\right)}\\
&-&\frac{\left(1-z\right)\left(1-F_{i}\left(z\right)\right)P_{i}\left(z\right)\left(P_{i}^{'}\left(z\right)-1\right)}{\esp\left[C_{i}\right]\left(1-P_{i}\left(z\right)\right)\left(P_{i}\left(z\right)-z\right)^{2}}\\
&+&\frac{\left(1-z\right)\left(1-F_{i}\left(z\right)\right)P_{i}^{'}\left(z\right)}{\esp\left[C_{i}\right]\left(1-P_{i}\left(z\right)\right)\left(P_{i}\left(z\right)-z\right)}\\
&+&\frac{\left(1-z\right)\left(1-F_{i}\left(z\right)\right)P_{i}\left(z\right)P_{i}^{'}\left(z\right)}{\esp\left[C_{i}\right]\left(1-P_{i}\left(z\right)\right)^{2}\left(P_{i}\left(z\right)-z\right)}
\end{eqnarray*}

Calculando el l\'imite cuando $z\rightarrow1^{+}$:
\begin{eqnarray}
Q_{i}^{(1)}\left(z\right)=\lim_{z\rightarrow1^{+}}\frac{d Q_{i}\left(z\right)}{dz}&=&\lim_{z\rightarrow1}\frac{\left(1-F_{i}\left(z\right)\right)P_{i}\left(z\right)}{\esp\left[C_{i}\right]\left(1-P_{i}\left(z\right)\right)\left(P_{i}\left(z\right)-z\right)}\\
&-&\lim_{z\rightarrow1^{+}}\frac{\left(1-z\right)P_{i}\left(z\right)F_{i}^{'}\left(z\right)}{\esp\left[C_{i}\right]\left(1-P_{i}\left(z\right)\right)\left(P_{i}\left(z\right)-z\right)}\\
&-&\lim_{z\rightarrow1^{+}}\frac{\left(1-z\right)\left(1-F_{i}\left(z\right)\right)P_{i}\left(z\right)\left(P_{i}^{'}\left(z\right)-1\right)}{\esp\left[C_{i}\right]\left(1-P_{i}\left(z\right)\right)\left(P_{i}\left(z\right)-z\right)^{2}}\\
&+&\lim_{z\rightarrow1^{+}}\frac{\left(1-z\right)\left(1-F_{i}\left(z\right)\right)P_{i}^{'}\left(z\right)}{\esp\left[C_{i}\right]\left(1-P_{i}\left(z\right)\right)\left(P_{i}\left(z\right)-z\right)}\\
&+&\lim_{z\rightarrow1^{+}}\frac{\left(1-z\right)\left(1-F_{i}\left(z\right)\right)P_{i}\left(z\right)P_{i}^{'}\left(z\right)}{\esp\left[C_{i}\right]\left(1-P_{i}\left(z\right)\right)^{2}\left(P_{i}\left(z\right)-z\right)}
\end{eqnarray}

Entonces:
%______________________________________________________

\begin{eqnarray*}
\lim_{z\rightarrow1^{+}}\frac{\left(1-F_{i}\left(z\right)\right)P_{i}\left(z\right)}{\left(1-P_{i}\left(z\right)\right)\left(P_{i}\left(z\right)-z\right)}&=&\lim_{z\rightarrow1^{+}}\frac{\frac{d}{dz}\left[\left(1-F_{i}\left(z\right)\right)P_{i}\left(z\right)\right]}{\frac{d}{dz}\left[\left(1-P_{i}\left(z\right)\right)\left(-z+P_{i}\left(z\right)\right)\right]}\\
&=&\lim_{z\rightarrow1^{+}}\frac{-P_{i}\left(z\right)F_{i}^{'}\left(z\right)+\left(1-F_{i}\left(z\right)\right)P_{i}^{'}\left(z\right)}{\left(1-P_{i}\left(z\right)\right)\left(-1+P_{i}^{'}\left(z\right)\right)-\left(-z+P_{i}\left(z\right)\right)P_{i}^{'}\left(z\right)}
\end{eqnarray*}


%______________________________________________________


\begin{eqnarray*}
\lim_{z\rightarrow1^{+}}\frac{\left(1-z\right)P_{i}\left(z\right)F_{i}^{'}\left(z\right)}{\left(1-P_{i}\left(z\right)\right)\left(P_{i}\left(z\right)-z\right)}&=&\lim_{z\rightarrow1^{+}}\frac{\frac{d}{dz}\left[\left(1-z\right)P_{i}\left(z\right)F_{i}^{'}\left(z\right)\right]}{\frac{d}{dz}\left[\left(1-P_{i}\left(z\right)\right)\left(P_{i}\left(z\right)-z\right)\right]}\\
&=&\lim_{z\rightarrow1^{+}}\frac{-P_{i}\left(z\right) F_{i}^{'}\left(z\right)+(1-z) F_{i}^{'}\left(z\right) P_{i}^{'}\left(z\right)+(1-z) P_{i}\left(z\right)F_{i}^{''}\left(z\right)}{\left(1-P_{i}\left(z\right)\right)\left(-1+P_{i}^{'}\left(z\right)\right)-\left(-z+P_{i}\left(z\right)\right)P_{i}^{'}\left(z\right)}
\end{eqnarray*}


%______________________________________________________

\begin{eqnarray*}
&&\lim_{z\rightarrow1^{+}}\frac{\left(1-z\right)\left(1-F_{i}\left(z\right)\right)P_{i}\left(z\right)\left(P_{i}^{'}\left(z\right)-1\right)}{\left(1-P_{i}\left(z\right)\right)\left(P_{i}\left(z\right)-z\right)^{2}}=\lim_{z\rightarrow1^{+}}\frac{\frac{d}{dz}\left[\left(1-z\right)\left(1-F_{i}\left(z\right)\right)P_{i}\left(z\right)\left(P_{i}^{'}\left(z\right)-1\right)\right]}{\frac{d}{dz}\left[\left(1-P_{i}\left(z\right)\right)\left(P_{i}\left(z\right)-z\right)^{2}\right]}\\
&=&\lim_{z\rightarrow1^{+}}\frac{-\left(1-F_{i}\left(z\right)\right) P_{i}\left(z\right)\left(-1+P_{i}^{'}\left(z\right)\right)-(1-z) P_{i}\left(z\right)F_{i}^{'}\left(z\right)\left(-1+P_{i}^{'}\left(z\right)\right)}{2\left(1-P_{i}\left(z\right)\right)\left(-z+P_{i}\left(z\right)\right) \left(-1+P_{i}^{'}\left(z\right)\right)-\left(-z+P_{i}\left(z\right)\right)^2 P_{i}^{'}\left(z\right)}\\
&+&\lim_{z\rightarrow1^{+}}\frac{+(1-z) \left(1-F_{i}\left(z\right)\right) \left(-1+P_{i}^{'}\left(z\right)\right) P_{i}^{'}\left(z\right)}{{2\left(1-P_{i}\left(z\right)\right)\left(-z+P_{i}\left(z\right)\right) \left(-1+P_{i}^{'}\left(z\right)\right)-\left(-z+P_{i}\left(z\right)\right)^2 P_{i}^{'}\left(z\right)}}\\
&+&\lim_{z\rightarrow1^{+}}\frac{+(1-z) \left(1-F_{i}\left(z\right)\right) P_{i}\left(z\right)P_{i}^{''}\left(z\right)}{{2\left(1-P_{i}\left(z\right)\right)\left(-z+P_{i}\left(z\right)\right) \left(-1+P_{i}^{'}\left(z\right)\right)-\left(-z+P_{i}\left(z\right)\right)^2 P_{i}^{'}\left(z\right)}}
\end{eqnarray*}











%______________________________________________________
\begin{eqnarray*}
&&\lim_{z\rightarrow1^{+}}\frac{\left(1-z\right)\left(1-F_{i}\left(z\right)\right)P_{i}^{'}\left(z\right)}{\left(1-P_{i}\left(z\right)\right)\left(P_{i}\left(z\right)-z\right)}=\lim_{z\rightarrow1^{+}}\frac{\frac{d}{dz}\left[\left(1-z\right)\left(1-F_{i}\left(z\right)\right)P_{i}^{'}\left(z\right)\right]}{\frac{d}{dz}\left[\left(1-P_{i}\left(z\right)\right)\left(P_{i}\left(z\right)-z\right)\right]}\\
&=&\lim_{z\rightarrow1^{+}}\frac{-\left(1-F_{i}\left(z\right)\right) P_{i}^{'}\left(z\right)-(1-z) F_{i}^{'}\left(z\right) P_{i}^{'}\left(z\right)+(1-z) \left(1-F_{i}\left(z\right)\right) P_{i}^{''}\left(z\right)}{\left(1-P_{i}\left(z\right)\right) \left(-1+P_{i}^{'}\left(z\right)\right)-\left(-z+P_{i}\left(z\right)\right) P_{i}^{'}\left(z\right)}\frac{}{}
\end{eqnarray*}

%______________________________________________________
\begin{eqnarray*}
&&\lim_{z\rightarrow1^{+}}\frac{\left(1-z\right)\left(1-F_{i}\left(z\right)\right)P_{i}\left(z\right)P_{i}^{'}\left(z\right)}{\left(1-P_{i}\left(z\right)\right)^{2}\left(P_{i}\left(z\right)-z\right)}=\lim_{z\rightarrow1^{+}}\frac{\frac{d}{dz}\left[\left(1-z\right)\left(1-F_{i}\left(z\right)\right)P_{i}\left(z\right)P_{i}^{'}\left(z\right)\right]}{\frac{d}{dz}\left[\left(1-P_{i}\left(z\right)\right)^{2}\left(P_{i}\left(z\right)-z\right)\right]}\\
&=&\lim_{z\rightarrow1^{+}}\frac{-\left(1-F_{i}\left(z\right)\right) P_{i}\left(z\right) P_{i}^{'}\left(z\right)-(1-z) P_{i}\left(z\right) F_{i}^{'}\left(z\right)P_i'[z]}{\left(1-P_{i}\left(z\right)\right)^2 \left(-1+P_{i}^{'}\left(z\right)\right)-2 \left(1-P_{i}\left(z\right)\right) \left(-z+P_{i}\left(z\right)\right) P_{i}^{'}\left(z\right)}\\
&+&\lim_{z\rightarrow1^{+}}\frac{(1-z) \left(1-F_{i}\left(z\right)\right) P_{i}^{'}\left(z\right)^2+(1-z) \left(1-F_{i}\left(z\right)\right) P_{i}\left(z\right) P_{i}^{''}\left(z\right)}{\left(1-P_{i}\left(z\right)\right)^2 \left(-1+P_{i}^{'}\left(z\right)\right)-2 \left(1-P_{i}\left(z\right)\right) \left(-z+P_{i}\left(z\right)\right) P_{i}^{'}\left(z\right)}\\
\end{eqnarray*}

\subsection{Por resolver}



\begin{eqnarray*}
&&\frac{\partial Q_{i}\left(z\right)}{\partial z}=\frac{1}{\esp\left[C_{i}\right]}\frac{\partial}{\partial z}\left\{\frac{1-F_{i}\left(z\right)}{P_{i}\left(z\right)-z}\cdot\frac{\left(1-z\right)P_{i}\left(z\right)}{1-P_{i}\left(z\right)}\right\}\\
&=&\frac{1}{\esp\left[C_{i}\right]}\left\{\frac{\partial}{\partial z}\left(\frac{1-F_{i}\left(z\right)}{P_{i}\left(z\right)-z}\right)\cdot\frac{\left(1-z\right)P_{i}\left(z\right)}{1-P_{i}\left(z\right)}+\frac{1-F_{i}\left(z\right)}{P_{i}\left(z\right)-z}\cdot\frac{\partial}{\partial z}\left(\frac{\left(1-z\right)P_{i}\left(z\right)}{1-P_{i}\left(z\right)}\right)\right\}\\
&=&\frac{1}{\esp\left[C_{i}\right]}\cdot\frac{\left(1-z\right)P_{i}\left(z\right)}{1-P_{i}\left(z\right)}\cdot\frac{\partial}{\partial z}\left(\frac{1-F_{i}\left(z\right)}{P_{i}\left(z\right)-z}\right)+\frac{1}{\esp\left[C_{i}\right]}\cdot\frac{1-F_{i}\left(z\right)}{P_{i}\left(z\right)-z}\cdot\frac{\partial}{\partial z}\left(\frac{\left(1-z\right)P_{i}\left(z\right)}{1-P_{i}\left(z\right)}\right)\\
&=&\frac{1}{\esp\left[C_{i}\right]}\cdot\frac{\left(1-z\right)P_{i}\left(z\right)}{1-P_{i}\left(z\right)}\cdot\frac{-F_{i}^{'}\left(z\right)\left(P_{i}\left(z\right)-z\right)-\left(1-F_{i}\left(z\right)\right)\left(P_{i}^{'}\left(z\right)-1\right)}{\left(P_{i}\left(z\right)-z\right)^{2}}\\
&+&\frac{1}{\esp\left[C_{i}\right]}\cdot\frac{1-F_{i}\left(z\right)}{P_{i}\left(z\right)-z}\cdot\frac{\left(1-z\right)P_{i}^{'}\left(z\right)-P_{i}\left(z\right)}{\left(1-P_{i}\left(z\right)\right)^{2}}
\end{eqnarray*}



\begin{eqnarray*}
Q_{i}^{(1)}\left(z\right)&=& \frac{\left(1-F_{i}\left(z\right)\right)P_{i}\left(z\right)}{\esp\left[C_{i}\right]\left(1-P_{i}\left(z\right)\right)\left(P_{i}\left(z\right)-z\right)}
-\frac{\left(1-z\right)P_{i}\left(z\right)F_{i}^{'}\left(z\right)}{\esp\left[C_{i}\right]\left(1-P_{i}\left(z\right)\right)\left(P_{i}\left(z\right)-z\right)}\\
&-&\frac{\left(1-z\right)\left(1-F_{i}\left(z\right)\right)P_{i}\left(z\right)\left(P_{i}^{'}\left(z\right)-1\right)}{\esp\left[C_{i}\right]\left(1-P_{i}\left(z\right)\right)\left(P_{i}\left(z\right)-z\right)^{2}}+\frac{\left(1-z\right)\left(1-F_{i}\left(z\right)\right)P_{i}^{'}\left(z\right)}{\esp\left[C_{i}\right]\left(1-P_{i}\left(z\right)\right)\left(P_{i}\left(z\right)-z\right)}\\
&+&\frac{\left(1-z\right)\left(1-F_{i}\left(z\right)\right)P_{i}\left(z\right)P_{i}^{'}\left(z\right)}{\esp\left[C_{i}\right]\left(1-P_{i}\left(z\right)\right)^{2}\left(P_{i}\left(z\right)-z\right)}
\end{eqnarray*}
%___________________________________________________________________________________________
%\subsection{Operaciones Matemathica: Tiempos de Espera}
%___________________________________________________________________________________________
Sea
$V_{i}\left(z\right)=\frac{1}{\esp\left[C_{i}\right]}\frac{I_{i}\left(z\right)-1}{z-P_{i}\left(z\right)}$

%{\esp\lef[I_{i}\right]}\frac{1-\mu_{i}}{z-P_{i}\left(z\right)}

\begin{eqnarray*}
\frac{\partial V_{i}\left(z\right)}{\partial z}&=&\frac{1}{\esp\left[C_{i}\right]}\left[\frac{I_{i}{'}\left(z\right)\left(z-P_{i}\left(z\right)\right)}{z-P_{i}\left(z\right)}-\frac{\left(I_{i}\left(z\right)-1\right)\left(1-P_{i}{'}\left(z\right)\right)}{\left(z-P_{i}\left(z\right)\right)^{2}}\right]
\end{eqnarray*}


La FGP para el tiempo de espera para cualquier usuario en la cola est\'a dada por:
\[U_{i}\left(z\right)=\frac{1}{\esp\left[C_{i}\right]}\cdot\frac{1-P_{i}\left(z\right)}{z-P_{i}\left(z\right)}\cdot\frac{I_{i}\left(z\right)-1}{1-z}\]

entonces


\begin{eqnarray*}
\frac{d}{dz}V_{i}\left(z\right)&=&\frac{1}{\esp\left[C_{i}\right]}\left\{\frac{d}{dz}\left(\frac{1-P_{i}\left(z\right)}{z-P_{i}\left(z\right)}\right)\frac{I_{i}\left(z\right)-1}{1-z}+\frac{1-P_{i}\left(z\right)}{z-P_{i}\left(z\right)}\frac{d}{dz}\left(\frac{I_{i}\left(z\right)-1}{1-z}\right)\right\}\\
&=&\frac{1}{\esp\left[C_{i}\right]}\left\{\frac{-P_{i}\left(z\right)\left(z-P_{i}\left(z\right)\right)-\left(1-P_{i}\left(z\right)\right)\left(1-P_{i}^{'}\left(z\right)\right)}{\left(z-P_{i}\left(z\right)\right)^{2}}\cdot\frac{I_{i}\left(z\right)-1}{1-z}\right\}\\
&+&\frac{1}{\esp\left[C_{i}\right]}\left\{\frac{1-P_{i}\left(z\right)}{z-P_{i}\left(z\right)}\cdot\frac{I_{i}^{'}\left(z\right)\left(1-z\right)+\left(I_{i}\left(z\right)-1\right)}{\left(1-z\right)^{2}}\right\}
\end{eqnarray*}
%\frac{I_{i}\left(z\right)-1}{1-z}
%+\frac{1-P_{i}\left(z\right)}{z-P_{i}\frac{d}{dz}\left(\frac{I_{i}\left(z\right)-1}{1-z}\right)


\begin{eqnarray*}
\frac{\partial U_{i}\left(z\right)}{\partial z}&=&\frac{(-1+I_{i}[z]) (1-P_{i}[z])}{(1-z)^2 \esp[I_{i}] (z-P_{i}[z])}+\frac{(1-P_{i}[z]) I_{i}^{'}[z]}{(1-z) \esp[I_{i}] (z-P_{i}[z])}-\frac{(-1+I_{i}[z]) (1-P_{i}[z])\left(1-P{'}[z]\right)}{(1-z) \esp[I_{i}] (z-P_{i}[z])^2}\\
&-&\frac{(-1+I_{i}[z]) P_{i}{'}[z]}{(1-z) \esp[I_{i}](z-P_{i}[z])}
\end{eqnarray*}

%___________________________________________________________________________________________
\subsection{Tiempos de Ciclo e Intervisita}
%___________________________________________________________________________________________


\begin{Def}
Sea $L_{i}^{*}$el n\'umero de usuarios en la cola $Q_{i}$ cuando es visitada por el servidor para dar servicio, entonces

\begin{eqnarray}
\esp\left[L_{i}^{*}\right]&=&f_{i}\left(i\right)\\
Var\left[L_{i}^{*}\right]&=&f_{i}\left(i,i\right)+\esp\left[L_{i}^{*}\right]-\esp\left[L_{i}^{*}\right]^{2}.
\end{eqnarray}

\end{Def}

\begin{Def}
El tiempo de Ciclo $C_{i}$ es e periodo de tiempo que comienza cuando la cola $i$ es visitada por primera vez en un ciclo, y termina cuando es visitado nuevamente en el pr\'oximo ciclo. La duraci\'on del mismo est\'a dada por $\tau_{i}\left(m+1\right)-\tau_{i}\left(m\right)$, o equivalentemente $\overline{\tau}_{i}\left(m+1\right)-\overline{\tau}_{i}\left(m\right)$ bajo condiciones de estabilidad.
\end{Def}

\begin{Def}
El tiempo de intervisita $I_{i}$ es el periodo de tiempo que comienza cuando se ha completado el servicio en un ciclo y termina cuando es visitada nuevamente en el pr\'oximo ciclo. Su  duraci\'on del mismo est\'a dada por $\tau_{i}\left(m+1\right)-\overline{\tau}_{i}\left(m\right)$.
\end{Def}


Recordemos las siguientes expresiones:

\begin{eqnarray*}
S_{i}\left(z\right)&=&\esp\left[z^{\overline{\tau}_{i}\left(m\right)-\tau_{i}\left(m\right)}\right]=F_{i}\left(\theta\left(z\right)\right),\\
F\left(z\right)&=&\esp\left[z^{L_{0}}\right],\\
P\left(z\right)&=&\esp\left[z^{X_{n}}\right],\\
F_{i}\left(z\right)&=&\esp\left[z^{L_{i}\left(\tau_{i}\left(m\right)\right)}\right],
\theta_{i}\left(z\right)-zP_{i}
\end{eqnarray*}

entonces 

\begin{eqnarray*}
\esp\left[S_{i}\right]&=&\frac{\esp\left[L_{i}^{*}\right]}{1-\mu_{i}}=\frac{f_{i}\left(i\right)}{1-\mu_{i}},\\
Var\left[S_{i}\right]&=&\frac{Var\left[L_{i}^{*}\right]}{\left(1-\mu_{i}\right)^{2}}+\frac{\sigma^{2}\esp\left[L_{i}^{*}\right]}{\left(1-\mu_{i}\right)^{3}}
\end{eqnarray*}

donde recordemos que

\begin{eqnarray*}
Var\left[L_{i}^{*}\right]&=&f_{i}\left(i,i\right)+f_{i}\left(i\right)-f_{i}\left(i\right)^{2}.
\end{eqnarray*}

La duraci\'on del tiempo de intervisita es $\tau_{i}\left(m+1\right)-\overline{\tau}\left(m\right)$. Dado que el n\'umero de usuarios presentes en $Q_{i}$ al tiempo $t=\tau_{i}\left(m+1\right)$ es igual al n\'umero de arribos durante el intervalo de tiempo $\left[\overline{\tau}\left(m\right),\tau_{i}\left(m+1\right)\right]$ se tiene que


\begin{eqnarray*}
\esp\left[z_{i}^{L_{i}\left(\tau_{i}\left(m+1\right)\right)}\right]=\esp\left[\left\{P_{i}\left(z_{i}\right)\right\}^{\tau_{i}\left(m+1\right)-\overline{\tau}\left(m\right)}\right]
\end{eqnarray*}

entonces, si \begin{eqnarray*}I_{i}\left(z\right)&=&\esp\left[z^{\tau_{i}\left(m+1\right)-\overline{\tau}\left(m\right)}\right]\end{eqnarray*} se tienen que

\begin{eqnarray*}
F_{i}\left(z\right)=I_{i}\left[P_{i}\left(z\right)\right]
\end{eqnarray*}
para $i=1,2$, por tanto



\begin{eqnarray*}
\esp\left[L_{i}^{*}\right]&=&\mu_{i}\esp\left[I_{i}\right]\\
Var\left[L_{i}^{*}\right]&=&\mu_{i}^{2}Var\left[I_{i}\right]+\sigma^{2}\esp\left[I_{i}\right]
\end{eqnarray*}
para $i=1,2$, por tanto


\begin{eqnarray*}
\esp\left[I_{i}\right]&=&\frac{f_{i}\left(i\right)}{\mu_{i}},
\end{eqnarray*}
adem\'as

\begin{eqnarray*}
Var\left[I_{i}\right]&=&\frac{Var\left[L_{i}^{*}\right]}{\mu_{i}^{2}}-\frac{\sigma_{i}^{2}}{\mu_{i}^{2}}f_{i}\left(i\right).
\end{eqnarray*}


Si  $C_{i}\left(z\right)=\esp\left[z^{\overline{\tau}\left(m+1\right)-\overline{\tau}_{i}\left(m\right)}\right]$el tiempo de duraci\'on del ciclo, entonces, por lo hasta ahora establecido, se tiene que

\begin{eqnarray*}
C_{i}\left(z\right)=I_{i}\left[\theta_{i}\left(z\right)\right],
\end{eqnarray*}
entonces

\begin{eqnarray*}
\esp\left[C_{i}\right]&=&\esp\left[I_{i}\right]\esp\left[\theta_{i}\left(z\right)\right]=\frac{\esp\left[L_{i}^{*}\right]}{\mu_{i}}\frac{1}{1-\mu_{i}}=\frac{f_{i}\left(i\right)}{\mu_{i}\left(1-\mu_{i}\right)}\\
Var\left[C_{i}\right]&=&\frac{Var\left[L_{i}^{*}\right]}{\mu_{i}^{2}\left(1-\mu_{i}\right)^{2}}.
\end{eqnarray*}

Por tanto se tienen las siguientes igualdades


\begin{eqnarray*}
\esp\left[S_{i}\right]&=&\mu_{i}\esp\left[C_{i}\right],\\
\esp\left[I_{i}\right]&=&\left(1-\mu_{i}\right)\esp\left[C_{i}\right]\\
\end{eqnarray*}

Def\'inanse los puntos de regenaraci\'on  en el proceso $\left[L_{1}\left(t\right),L_{2}\left(t\right),\ldots,L_{N}\left(t\right)\right]$. Los puntos cuando la cola $i$ es visitada y todos los $L_{j}\left(\tau_{i}\left(m\right)\right)=0$ para $i=1,2$  son puntos de regeneraci\'on. Se llama ciclo regenerativo al intervalo entre dos puntos regenerativos sucesivos.

Sea $M_{i}$  el n\'umero de ciclos de visita en un ciclo regenerativo, y sea $C_{i}^{(m)}$, para $m=1,2,\ldots,M_{i}$ la duraci\'on del $m$-\'esimo ciclo de visita en un ciclo regenerativo. Se define el ciclo del tiempo de visita promedio $\esp\left[C_{i}\right]$ como

\begin{eqnarray*}
\esp\left[C_{i}\right]&=&\frac{\esp\left[\sum_{m=1}^{M_{i}}C_{i}^{(m)}\right]}{\esp\left[M_{i}\right]}
\end{eqnarray*}


En Stid72 y Heym82 se muestra que una condici\'on suficiente para que el proceso regenerativo 
estacionario sea un procesoo estacionario es que el valor esperado del tiempo del ciclo regenerativo sea finito:

\begin{eqnarray*}
\esp\left[\sum_{m=1}^{M_{i}}C_{i}^{(m)}\right]<\infty.
\end{eqnarray*}

como cada $C_{i}^{(m)}$ contiene intervalos de r\'eplica positivos, se tiene que $\esp\left[M_{i}\right]<\infty$, adem\'as, como $M_{i}>0$, se tiene que la condici\'on anterior es equivalente a tener que 

\begin{eqnarray*}
\esp\left[C_{i}\right]<\infty,
\end{eqnarray*}
por lo tanto una condici\'on suficiente para la existencia del proceso regenerativo est\'a dada por

\begin{eqnarray*}
\sum_{k=1}^{N}\mu_{k}<1.
\end{eqnarray*}

Sea la funci\'on generadora de momentos para $L_{i}$, el n\'umero de usuarios en la cola $Q_{i}\left(z\right)$ en cualquier momento, est\'a dada por el tiempo promedio de $z^{L_{i}\left(t\right)}$ sobre el ciclo regenerativo definido anteriormente:

\begin{eqnarray*}
Q_{i}\left(z\right)&=&\esp\left[z^{L_{i}\left(t\right)}\right]=\frac{\esp\left[\sum_{m=1}^{M_{i}}\sum_{t=\tau_{i}\left(m\right)}^{\tau_{i}\left(m+1\right)-1}z^{L_{i}\left(t\right)}\right]}{\esp\left[\sum_{m=1}^{M_{i}}\tau_{i}\left(m+1\right)-\tau_{i}\left(m\right)\right]}
\end{eqnarray*}

$M_{i}$ es un tiempo de paro en el proceso regenerativo con $\esp\left[M_{i}\right]<\infty$, se sigue del lema de Wald que:


\begin{eqnarray*}
\esp\left[\sum_{m=1}^{M_{i}}\sum_{t=\tau_{i}\left(m\right)}^{\tau_{i}\left(m+1\right)-1}z^{L_{i}\left(t\right)}\right]&=&\esp\left[M_{i}\right]\esp\left[\sum_{t=\tau_{i}\left(m\right)}^{\tau_{i}\left(m+1\right)-1}z^{L_{i}\left(t\right)}\right]\\
\esp\left[\sum_{m=1}^{M_{i}}\tau_{i}\left(m+1\right)-\tau_{i}\left(m\right)\right]&=&\esp\left[M_{i}\right]\esp\left[\tau_{i}\left(m+1\right)-\tau_{i}\left(m\right)\right]
\end{eqnarray*}

por tanto se tiene que


\begin{eqnarray*}
Q_{i}\left(z\right)&=&\frac{\esp\left[\sum_{t=\tau_{i}\left(m\right)}^{\tau_{i}\left(m+1\right)-1}z^{L_{i}\left(t\right)}\right]}{\esp\left[\tau_{i}\left(m+1\right)-\tau_{i}\left(m\right)\right]}
\end{eqnarray*}

observar que el denominador es simplemente la duraci\'on promedio del tiempo del ciclo.


Se puede demostrar (ver Hideaki Takagi 1986) que

\begin{eqnarray*}
\esp\left[\sum_{t=\tau_{i}\left(m\right)}^{\tau_{i}\left(m+1\right)-1}z^{L_{i}\left(t\right)}\right]=z\frac{F_{i}\left(z\right)-1}{z-P_{i}\left(z\right)}
\end{eqnarray*}

Durante el tiempo de intervisita para la cola $i$, $L_{i}\left(t\right)$ solamente se incrementa de manera que el incremento por intervalo de tiempo est\'a dado por la funci\'on generadora de probabilidades de $P_{i}\left(z\right)$, por tanto la suma sobre el tiempo de intervisita puede evaluarse como:

\begin{eqnarray*}
\esp\left[\sum_{t=\tau_{i}\left(m\right)}^{\tau_{i}\left(m+1\right)-1}z^{L_{i}\left(t\right)}\right]&=&\esp\left[\sum_{t=\tau_{i}\left(m\right)}^{\tau_{i}\left(m+1\right)-1}\left\{P_{i}\left(z\right)\right\}^{t-\overline{\tau}_{i}\left(m\right)}\right]=\frac{1-\esp\left[\left\{P_{i}\left(z\right)\right\}^{\tau_{i}\left(m+1\right)-\overline{\tau}_{i}\left(m\right)}\right]}{1-P_{i}\left(z\right)}\\
&=&\frac{1-I_{i}\left[P_{i}\left(z\right)\right]}{1-P_{i}\left(z\right)}
\end{eqnarray*}
por tanto

\begin{eqnarray*}
\esp\left[\sum_{t=\tau_{i}\left(m\right)}^{\tau_{i}\left(m+1\right)-1}z^{L_{i}\left(t\right)}\right]&=&\frac{1-F_{i}\left(z\right)}{1-P_{i}\left(z\right)}
\end{eqnarray*}

Haciendo uso de lo hasta ahora desarrollado se tiene que

\begin{eqnarray*}
Q_{i}\left(z\right)&=&\frac{1}{\esp\left[C_{i}\right]}\cdot\frac{1-F_{i}\left(z\right)}{P_{i}\left(z\right)-z}\cdot\frac{\left(1-z\right)P_{i}\left(z\right)}{1-P_{i}\left(z\right)}\\
&=&\frac{\mu_{i}\left(1-\mu_{i}\right)}{f_{i}\left(i\right)}\cdot\frac{1-F_{i}\left(z\right)}{P_{i}\left(z\right)-z}\cdot\frac{\left(1-z\right)P_{i}\left(z\right)}{1-P_{i}\left(z\right)}
\end{eqnarray*}

derivando con respecto a $z$



\begin{eqnarray*}
\frac{d Q_{i}\left(z\right)}{d z}&=&\frac{\left(1-F_{i}\left(z\right)\right)P_{i}\left(z\right)}{\esp\left[C_{i}\right]\left(1-P_{i}\left(z\right)\right)\left(P_{i}\left(z\right)-z\right)}\\
&-&\frac{\left(1-z\right)P_{i}\left(z\right)F_{i}^{'}\left(z\right)}{\esp\left[C_{i}\right]\left(1-P_{i}\left(z\right)\right)\left(P_{i}\left(z\right)-z\right)}\\
&-&\frac{\left(1-z\right)\left(1-F_{i}\left(z\right)\right)P_{i}\left(z\right)\left(P_{i}^{'}\left(z\right)-1\right)}{\esp\left[C_{i}\right]\left(1-P_{i}\left(z\right)\right)\left(P_{i}\left(z\right)-z\right)^{2}}\\
&+&\frac{\left(1-z\right)\left(1-F_{i}\left(z\right)\right)P_{i}^{'}\left(z\right)}{\esp\left[C_{i}\right]\left(1-P_{i}\left(z\right)\right)\left(P_{i}\left(z\right)-z\right)}\\
&+&\frac{\left(1-z\right)\left(1-F_{i}\left(z\right)\right)P_{i}\left(z\right)P_{i}^{'}\left(z\right)}{\esp\left[C_{i}\right]\left(1-P_{i}\left(z\right)\right)^{2}\left(P_{i}\left(z\right)-z\right)}
\end{eqnarray*}

Calculando el l\'imite cuando $z\rightarrow1^{+}$:
\begin{eqnarray}
Q_{i}^{(1)}\left(z\right)=\lim_{z\rightarrow1^{+}}\frac{d Q_{i}\left(z\right)}{dz}&=&\lim_{z\rightarrow1}\frac{\left(1-F_{i}\left(z\right)\right)P_{i}\left(z\right)}{\esp\left[C_{i}\right]\left(1-P_{i}\left(z\right)\right)\left(P_{i}\left(z\right)-z\right)}\\
&-&\lim_{z\rightarrow1^{+}}\frac{\left(1-z\right)P_{i}\left(z\right)F_{i}^{'}\left(z\right)}{\esp\left[C_{i}\right]\left(1-P_{i}\left(z\right)\right)\left(P_{i}\left(z\right)-z\right)}\\
&-&\lim_{z\rightarrow1^{+}}\frac{\left(1-z\right)\left(1-F_{i}\left(z\right)\right)P_{i}\left(z\right)\left(P_{i}^{'}\left(z\right)-1\right)}{\esp\left[C_{i}\right]\left(1-P_{i}\left(z\right)\right)\left(P_{i}\left(z\right)-z\right)^{2}}\\
&+&\lim_{z\rightarrow1^{+}}\frac{\left(1-z\right)\left(1-F_{i}\left(z\right)\right)P_{i}^{'}\left(z\right)}{\esp\left[C_{i}\right]\left(1-P_{i}\left(z\right)\right)\left(P_{i}\left(z\right)-z\right)}\\
&+&\lim_{z\rightarrow1^{+}}\frac{\left(1-z\right)\left(1-F_{i}\left(z\right)\right)P_{i}\left(z\right)P_{i}^{'}\left(z\right)}{\esp\left[C_{i}\right]\left(1-P_{i}\left(z\right)\right)^{2}\left(P_{i}\left(z\right)-z\right)}
\end{eqnarray}

Entonces:
%______________________________________________________

\begin{eqnarray*}
\lim_{z\rightarrow1^{+}}\frac{\left(1-F_{i}\left(z\right)\right)P_{i}\left(z\right)}{\left(1-P_{i}\left(z\right)\right)\left(P_{i}\left(z\right)-z\right)}&=&\lim_{z\rightarrow1^{+}}\frac{\frac{d}{dz}\left[\left(1-F_{i}\left(z\right)\right)P_{i}\left(z\right)\right]}{\frac{d}{dz}\left[\left(1-P_{i}\left(z\right)\right)\left(-z+P_{i}\left(z\right)\right)\right]}\\
&=&\lim_{z\rightarrow1^{+}}\frac{-P_{i}\left(z\right)F_{i}^{'}\left(z\right)+\left(1-F_{i}\left(z\right)\right)P_{i}^{'}\left(z\right)}{\left(1-P_{i}\left(z\right)\right)\left(-1+P_{i}^{'}\left(z\right)\right)-\left(-z+P_{i}\left(z\right)\right)P_{i}^{'}\left(z\right)}
\end{eqnarray*}


%______________________________________________________


\begin{eqnarray*}
\lim_{z\rightarrow1^{+}}\frac{\left(1-z\right)P_{i}\left(z\right)F_{i}^{'}\left(z\right)}{\left(1-P_{i}\left(z\right)\right)\left(P_{i}\left(z\right)-z\right)}&=&\lim_{z\rightarrow1^{+}}\frac{\frac{d}{dz}\left[\left(1-z\right)P_{i}\left(z\right)F_{i}^{'}\left(z\right)\right]}{\frac{d}{dz}\left[\left(1-P_{i}\left(z\right)\right)\left(P_{i}\left(z\right)-z\right)\right]}\\
&=&\lim_{z\rightarrow1^{+}}\frac{-P_{i}\left(z\right) F_{i}^{'}\left(z\right)+(1-z) F_{i}^{'}\left(z\right) P_{i}^{'}\left(z\right)+(1-z) P_{i}\left(z\right)F_{i}^{''}\left(z\right)}{\left(1-P_{i}\left(z\right)\right)\left(-1+P_{i}^{'}\left(z\right)\right)-\left(-z+P_{i}\left(z\right)\right)P_{i}^{'}\left(z\right)}
\end{eqnarray*}


%______________________________________________________

\begin{eqnarray*}
&&\lim_{z\rightarrow1^{+}}\frac{\left(1-z\right)\left(1-F_{i}\left(z\right)\right)P_{i}\left(z\right)\left(P_{i}^{'}\left(z\right)-1\right)}{\left(1-P_{i}\left(z\right)\right)\left(P_{i}\left(z\right)-z\right)^{2}}=\lim_{z\rightarrow1^{+}}\frac{\frac{d}{dz}\left[\left(1-z\right)\left(1-F_{i}\left(z\right)\right)P_{i}\left(z\right)\left(P_{i}^{'}\left(z\right)-1\right)\right]}{\frac{d}{dz}\left[\left(1-P_{i}\left(z\right)\right)\left(P_{i}\left(z\right)-z\right)^{2}\right]}\\
&=&\lim_{z\rightarrow1^{+}}\frac{-\left(1-F_{i}\left(z\right)\right) P_{i}\left(z\right)\left(-1+P_{i}^{'}\left(z\right)\right)-(1-z) P_{i}\left(z\right)F_{i}^{'}\left(z\right)\left(-1+P_{i}^{'}\left(z\right)\right)}{2\left(1-P_{i}\left(z\right)\right)\left(-z+P_{i}\left(z\right)\right) \left(-1+P_{i}^{'}\left(z\right)\right)-\left(-z+P_{i}\left(z\right)\right)^2 P_{i}^{'}\left(z\right)}\\
&+&\lim_{z\rightarrow1^{+}}\frac{+(1-z) \left(1-F_{i}\left(z\right)\right) \left(-1+P_{i}^{'}\left(z\right)\right) P_{i}^{'}\left(z\right)}{{2\left(1-P_{i}\left(z\right)\right)\left(-z+P_{i}\left(z\right)\right) \left(-1+P_{i}^{'}\left(z\right)\right)-\left(-z+P_{i}\left(z\right)\right)^2 P_{i}^{'}\left(z\right)}}\\
&+&\lim_{z\rightarrow1^{+}}\frac{+(1-z) \left(1-F_{i}\left(z\right)\right) P_{i}\left(z\right)P_{i}^{''}\left(z\right)}{{2\left(1-P_{i}\left(z\right)\right)\left(-z+P_{i}\left(z\right)\right) \left(-1+P_{i}^{'}\left(z\right)\right)-\left(-z+P_{i}\left(z\right)\right)^2 P_{i}^{'}\left(z\right)}}
\end{eqnarray*}











%______________________________________________________
\begin{eqnarray*}
&&\lim_{z\rightarrow1^{+}}\frac{\left(1-z\right)\left(1-F_{i}\left(z\right)\right)P_{i}^{'}\left(z\right)}{\left(1-P_{i}\left(z\right)\right)\left(P_{i}\left(z\right)-z\right)}=\lim_{z\rightarrow1^{+}}\frac{\frac{d}{dz}\left[\left(1-z\right)\left(1-F_{i}\left(z\right)\right)P_{i}^{'}\left(z\right)\right]}{\frac{d}{dz}\left[\left(1-P_{i}\left(z\right)\right)\left(P_{i}\left(z\right)-z\right)\right]}\\
&=&\lim_{z\rightarrow1^{+}}\frac{-\left(1-F_{i}\left(z\right)\right) P_{i}^{'}\left(z\right)-(1-z) F_{i}^{'}\left(z\right) P_{i}^{'}\left(z\right)+(1-z) \left(1-F_{i}\left(z\right)\right) P_{i}^{''}\left(z\right)}{\left(1-P_{i}\left(z\right)\right) \left(-1+P_{i}^{'}\left(z\right)\right)-\left(-z+P_{i}\left(z\right)\right) P_{i}^{'}\left(z\right)}\frac{}{}
\end{eqnarray*}

%______________________________________________________
\begin{eqnarray*}
&&\lim_{z\rightarrow1^{+}}\frac{\left(1-z\right)\left(1-F_{i}\left(z\right)\right)P_{i}\left(z\right)P_{i}^{'}\left(z\right)}{\left(1-P_{i}\left(z\right)\right)^{2}\left(P_{i}\left(z\right)-z\right)}=\lim_{z\rightarrow1^{+}}\frac{\frac{d}{dz}\left[\left(1-z\right)\left(1-F_{i}\left(z\right)\right)P_{i}\left(z\right)P_{i}^{'}\left(z\right)\right]}{\frac{d}{dz}\left[\left(1-P_{i}\left(z\right)\right)^{2}\left(P_{i}\left(z\right)-z\right)\right]}\\
&=&\lim_{z\rightarrow1^{+}}\frac{-\left(1-F_{i}\left(z\right)\right) P_{i}\left(z\right) P_{i}^{'}\left(z\right)-(1-z) P_{i}\left(z\right) F_{i}^{'}\left(z\right)P_i'[z]}{\left(1-P_{i}\left(z\right)\right)^2 \left(-1+P_{i}^{'}\left(z\right)\right)-2 \left(1-P_{i}\left(z\right)\right) \left(-z+P_{i}\left(z\right)\right) P_{i}^{'}\left(z\right)}\\
&+&\lim_{z\rightarrow1^{+}}\frac{(1-z) \left(1-F_{i}\left(z\right)\right) P_{i}^{'}\left(z\right)^2+(1-z) \left(1-F_{i}\left(z\right)\right) P_{i}\left(z\right) P_{i}^{''}\left(z\right)}{\left(1-P_{i}\left(z\right)\right)^2 \left(-1+P_{i}^{'}\left(z\right)\right)-2 \left(1-P_{i}\left(z\right)\right) \left(-z+P_{i}\left(z\right)\right) P_{i}^{'}\left(z\right)}\\
\end{eqnarray*}

%___________________________________________________________________________________________
\subsection{Longitudes de la Cola en cualquier tiempo}
%___________________________________________________________________________________________

Sea
$V_{i}\left(z\right)=\frac{1}{\esp\left[C_{i}\right]}\frac{I_{i}\left(z\right)-1}{z-P_{i}\left(z\right)}$

%{\esp\lef[I_{i}\right]}\frac{1-\mu_{i}}{z-P_{i}\left(z\right)}

\begin{eqnarray*}
\frac{\partial V_{i}\left(z\right)}{\partial z}&=&\frac{1}{\esp\left[C_{i}\right]}\left[\frac{I_{i}{'}\left(z\right)\left(z-P_{i}\left(z\right)\right)}{z-P_{i}\left(z\right)}-\frac{\left(I_{i}\left(z\right)-1\right)\left(1-P_{i}{'}\left(z\right)\right)}{\left(z-P_{i}\left(z\right)\right)^{2}}\right]
\end{eqnarray*}


La FGP para el tiempo de espera para cualquier usuario en la cola est\'a dada por:
\[U_{i}\left(z\right)=\frac{1}{\esp\left[C_{i}\right]}\cdot\frac{1-P_{i}\left(z\right)}{z-P_{i}\left(z\right)}\cdot\frac{I_{i}\left(z\right)-1}{1-z}\]

entonces


\begin{eqnarray*}
\frac{d}{dz}V_{i}\left(z\right)&=&\frac{1}{\esp\left[C_{i}\right]}\left\{\frac{d}{dz}\left(\frac{1-P_{i}\left(z\right)}{z-P_{i}\left(z\right)}\right)\frac{I_{i}\left(z\right)-1}{1-z}+\frac{1-P_{i}\left(z\right)}{z-P_{i}\left(z\right)}\frac{d}{dz}\left(\frac{I_{i}\left(z\right)-1}{1-z}\right)\right\}\\
&=&\frac{1}{\esp\left[C_{i}\right]}\left\{\frac{-P_{i}\left(z\right)\left(z-P_{i}\left(z\right)\right)-\left(1-P_{i}\left(z\right)\right)\left(1-P_{i}^{'}\left(z\right)\right)}{\left(z-P_{i}\left(z\right)\right)^{2}}\cdot\frac{I_{i}\left(z\right)-1}{1-z}\right\}\\
&+&\frac{1}{\esp\left[C_{i}\right]}\left\{\frac{1-P_{i}\left(z\right)}{z-P_{i}\left(z\right)}\cdot\frac{I_{i}^{'}\left(z\right)\left(1-z\right)+\left(I_{i}\left(z\right)-1\right)}{\left(1-z\right)^{2}}\right\}
\end{eqnarray*}
%\frac{I_{i}\left(z\right)-1}{1-z}
%+\frac{1-P_{i}\left(z\right)}{z-P_{i}\frac{d}{dz}\left(\frac{I_{i}\left(z\right)-1}{1-z}\right)


\begin{eqnarray*}
\frac{\partial U_{i}\left(z\right)}{\partial z}&=&\frac{(-1+I_{i}[z]) (1-P_{i}[z])}{(1-z)^2 \esp[I_{i}] (z-P_{i}[z])}+\frac{(1-P_{i}[z]) I_{i}^{'}[z]}{(1-z) \esp[I_{i}] (z-P_{i}[z])}-\frac{(-1+I_{i}[z]) (1-P_{i}[z])\left(1-P{'}[z]\right)}{(1-z) \esp[I_{i}] (z-P_{i}[z])^2}\\
&-&\frac{(-1+I_{i}[z]) P_{i}{'}[z]}{(1-z) \esp[I_{i}](z-P_{i}[z])}
\end{eqnarray*}


\subsection{Material por agregar}


\begin{Teo}
Dada una Red de Sistemas de Visitas C\'iclicas (RSVC), conformada por dos Sistemas de Visitas C\'iclicas (SVC), donde cada uno de ellos consta de dos colas tipo $M/M/1$. Los dos sistemas est\'an comunicados entre s\'i por medio de la transferencia de usuarios entre las colas $Q_{1}\leftrightarrow Q_{3}$ y $Q_{2}\leftrightarrow Q_{4}$. Se definen los eventos para los procesos de arribos al tiempo $t$, $A_{j}\left(t\right)=\left\{0 \textrm{ arribos en }Q_{j}\textrm{ al tiempo }t\right\}$ para alg\'un tiempo $t\geq0$ y $Q_{j}$ la cola $j$-\'esima en la RSVC, para $j=1,2,3,4$.  Existe un intervalo $I\neq\emptyset$ tal que para $T^{*}\in I$, tal que $\prob\left\{A_{1}\left(T^{*}\right),A_{2}\left(Tt^{*}\right),
A_{3}\left(T^{*}\right),A_{4}\left(T^{*}\right)|T^{*}\in I\right\}>0$.
\end{Teo}



\begin{proof}
Sin p\'erdida de generalidad podemos considerar como base del an\'alisis a la cola $Q_{1}$ del primer sistema que conforma la RSVC.\medskip 

Sea $n\geq1$, ciclo en el primer sistema en el que se sabe que $L_{j}\left(\overline{\tau}_{1}\left(n\right)\right)=0$, pues la pol\'itica de servicio con que atienden los servidores es la exhaustiva. Como es sabido, para trasladarse a la siguiente cola, el servidor incurre en un tiempo de traslado $r_{1}\left(n\right)>0$, entonces tenemos que $\tau_{2}\left(n\right)=\overline{\tau}_{1}\left(n\right)+r_{1}\left(n\right)$.\medskip 


Definamos el intervalo $I_{1}\equiv\left[\overline{\tau}_{1}\left(n\right),\tau_{2}\left(n\right)\right]$ de longitud $\xi_{1}=r_{1}\left(n\right)$.

Dado que los tiempos entre arribo son exponenciales con tasa $\tilde{\mu}_{1}=\mu_{1}+\hat{\mu}_{1}$ ($\mu_{1}$ son los arribos a $Q_{1}$ por primera vez al sistema, mientras que $\hat{\mu}_{1}$ son los arribos de traslado procedentes de $Q_{3}$) se tiene que la probabilidad del evento $A_{1}\left(t\right)$ est\'a dada por 

\begin{equation}
\prob\left\{A_{1}\left(t\right)|t\in I_{1}\left(n\right)\right\}=e^{-\tilde{\mu}_{1}\xi_{1}\left(n\right)}.
\end{equation} 


Por otra parte, para la cola $Q_{2}$ el tiempo $\overline{\tau}_{2}\left(n-1\right)$ es tal que $L_{2}\left(\overline{\tau}_{2}\left(n-1\right)\right)=0$, es decir, es el tiempo en que la cola queda totalmente vac\'ia en el ciclo anterior a $n$. \medskip 


Entonces tenemos un sgundo intervalo $I_{2}\equiv\left[\overline{\tau}_{2}\left(n-1\right),\tau_{2}\left(n\right)\right]$. Por lo tanto la probabilidad del evento $A_{2}\left(t\right)$ tiene probabilidad dada por

\begin{eqnarray}
\prob\left\{A_{2}\left(t\right)|t\in I_{2}\left(n\right)\right\}=e^{-\tilde{\mu}_{2}\xi_{2}\left(n\right)},\\
\xi_{2}\left(n\right)=\tau_{2}\left(n\right)-\overline{\tau}_{2}\left(n-1\right)
\end{eqnarray}
%\end{equation} 

%donde $$.

Ahora, dado que $I_{1}\left(n\right)\subset I_{2}\left(n\right)$, se tiene que

\begin{eqnarray*}
\xi_{1}\left(n\right)\leq\xi_{2}\left(n\right)&\Leftrightarrow& -\xi_{1}\left(n\right)\geq-\xi_{2}\left(n\right)
\\
-\tilde{\mu}_{2}\xi_{1}\left(n\right)\geq-\tilde{\mu}_{2}\xi_{2}\left(n\right)&\Leftrightarrow&
e^{-\tilde{\mu}_{2}\xi_{1}\left(n\right)}\geq e^{-\tilde{\mu}_{2}\xi_{2}\left(n\right)}\\
\prob\left\{A_{2}\left(t\right)|t\in I_{1}\left(n\right)\right\}&\geq&
\prob\left\{A_{2}\left(t\right)|t\in I_{2}\left(n\right)\right\}.
\end{eqnarray*}


Entonces se tiene que
\small{
\begin{eqnarray*}
\prob\left\{A_{1}\left(t\right),A_{2}\left(t\right)|t\in I_{1}\left(n\right)\right\}&=&
\prob\left\{A_{1}\left(t\right)|t\in I_{1}\left(n\right)\right\}
\prob\left\{A_{2}\left(t\right)|t\in I_{1}\left(n\right)\right\}\\
&\geq&
\prob\left\{A_{1}\left(t\right)|t\in I_{1}\left(n\right)\right\}
\prob\left\{A_{2}\left(t\right)|t\in I_{2}\left(n\right)\right\}\\
&=&e^{-\tilde{\mu}_{1}\xi_{1}\left(n\right)}e^{-\tilde{\mu}_{2}\xi_{2}\left(n\right)}
=e^{-\left[\tilde{\mu}_{1}\xi_{1}\left(n\right)+\tilde{\mu}_{2}\xi_{2}\left(n\right)\right]}.
\end{eqnarray*}}


Es decir, 

\begin{equation}
\prob\left\{A_{1}\left(t\right),A_{2}\left(t\right)|t\in I_{1}\left(n\right)\right\}
=e^{-\left[\tilde{\mu}_{1}\xi_{1}\left(n\right)+\tilde{\mu}_{2}\xi_{2}
\left(n\right)\right]}>0.
\end{equation}
En lo que respecta a la relaci\'on entre los dos SVC que conforman la RSVC para alg\'un $m\geq1$ se tiene que $\tau_{3}\left(m\right)<\tau_{2}\left(n\right)<\tau_{4}\left(m\right)$ por lo tanto se cumple cualquiera de los siguientes cuatro casos
\begin{itemize}
\item[a)] $\tau_{3}\left(m\right)<\tau_{2}\left(n\right)<\overline{\tau}_{3}\left(m\right)$

\item[b)] $\overline{\tau}_{3}\left(m\right)<\tau_{2}\left(n\right)
<\tau_{4}\left(m\right)$

\item[c)] $\tau_{4}\left(m\right)<\tau_{2}\left(n\right)<
\overline{\tau}_{4}\left(m\right)$

\item[d)] $\overline{\tau}_{4}\left(m\right)<\tau_{2}\left(n\right)
<\tau_{3}\left(m+1\right)$
\end{itemize}


Sea el intervalo $I_{3}\left(m\right)\equiv\left[\tau_{3}\left(m\right),\overline{\tau}_{3}\left(m\right)\right]$ tal que $\tau_{2}\left(n\right)\in I_{3}\left(m\right)$, con longitud de intervalo $\xi_{3}\equiv\overline{\tau}_{3}\left(m\right)-\tau_{3}\left(m\right)$, entonces se tiene que para $Q_{3}$
\begin{equation}
\prob\left\{A_{3}\left(t\right)|t\in I_{3}\left(m\right)\right\}=e^{-\tilde{\mu}_{3}\xi_{3}\left(m\right)}.
\end{equation} 

mientras que para $Q_{4}$ consideremos el intervalo $I_{4}\left(m\right)\equiv\left[\tau_{4}\left(m-1\right),\overline{\tau}_{3}\left(m\right)\right]$, entonces por construcci\'on  $I_{3}\left(m\right)\subset I_{4}\left(m\right)$, por lo tanto


\begin{eqnarray*}
\xi_{3}\left(m\right)\leq\xi_{4}\left(m\right)&\Leftrightarrow& -\xi_{3}\left(m\right)\geq-\xi_{4}\left(m\right)
\\
-\tilde{\mu}_{4}\xi_{3}\left(m\right)\geq-\tilde{\mu}_{4}\xi_{4}\left(m\right)&\Leftrightarrow&
e^{-\tilde{\mu}_{4}\xi_{3}\left(m\right)}\geq e^{-\tilde{\mu}_{4}\xi_{4}\left(n\right)}\\
\prob\left\{A_{4}\left(t\right)|t\in I_{3}\left(m\right)\right\}&\geq&
\prob\left\{A_{4}\left(t\right)|t\in I_{4}\left(m\right)\right\}.
\end{eqnarray*}



Entonces se tiene que
\small{
\begin{eqnarray*}
\prob\left\{A_{3}\left(t\right),A_{4}\left(t\right)|t\in I_{3}\left(m\right)\right\}&=&
\prob\left\{A_{3}\left(t\right)|t\in I_{3}\left(m\right)\right\}
\prob\left\{A_{4}\left(t\right)|t\in I_{3}\left(m\right)\right\}\\
&\geq&
\prob\left\{A_{3}\left(t\right)|t\in I_{3}\left(m\right)\right\}
\prob\left\{A_{4}\left(t\right)|t\in I_{4}\left(m\right)\right\}\\
&=&e^{-\tilde{\mu}_{3}\xi_{3}\left(m\right)}e^{-\tilde{\mu}_{4}\xi_{4}
\left(m\right)}
=e^{-\left(\tilde{\mu}_{3}\xi_{3}\left(m\right)+\tilde{\mu}_{4}\xi_{4}\left(m\right)\right)}.
\end{eqnarray*}}

Es decir, 

\begin{equation}
\prob\left\{A_{3}\left(t\right),A_{4}\left(t\right)|t\in I_{3}\left(m\right)\right\}\geq
e^{-\left(\tilde{\mu}_{3}\xi_{3}\left(m\right)+\tilde{\mu}_{4}\xi_{4}\left(m\right)\right)}>0.
\end{equation}


Sea el intervalo $I_{3}\left(m\right)\equiv\left[\overline{\tau}_{3}\left(m\right),\tau_{4}\left(m\right)\right]$ con longitud $\xi_{3}\equiv\tau_{4}\left(m\right)-\overline{\tau}_{3}\left(m\right)$, entonces se tiene que para $Q_{3}$
\begin{equation}
\prob\left\{A_{3}\left(t\right)|t\in I_{3}\left(m\right)\right\}=e^{-\tilde{\mu}_{3}\xi_{3}\left(m\right)}.
\end{equation} 

mientras que para $Q_{4}$ consideremos el intervalo $I_{4}\left(m\right)\equiv\left[\overline{\tau}_{4}\left(m-1\right),\tau_{4}\left(m\right)\right]$, entonces por construcci\'on  $I_{3}\left(m\right)\subset I_{4}\left(m\right)$, y al igual que en el caso anterior se tiene que 

\begin{eqnarray*}
\xi_{3}\left(m\right)\leq\xi_{4}\left(m\right)&\Leftrightarrow& -\xi_{3}\left(m\right)\geq-\xi_{4}\left(m\right)
\\
-\tilde{\mu}_{4}\xi_{3}\left(m\right)\geq-\tilde{\mu}_{4}\xi_{4}\left(m\right)&\Leftrightarrow&
e^{-\tilde{\mu}_{4}\xi_{3}\left(m\right)}\geq e^{-\tilde{\mu}_{4}\xi_{4}\left(n\right)}\\
\prob\left\{A_{4}\left(t\right)|t\in I_{3}\left(m\right)\right\}&\geq&
\prob\left\{A_{4}\left(t\right)|t\in I_{4}\left(m\right)\right\}.
\end{eqnarray*}


Entonces se tiene que
\small{
\begin{eqnarray*}
\prob\left\{A_{3}\left(t\right),A_{4}\left(t\right)|t\in I_{3}\left(m\right)\right\}&=&
\prob\left\{A_{3}\left(t\right)|t\in I_{3}\left(m\right)\right\}
\prob\left\{A_{4}\left(t\right)|t\in I_{3}\left(m\right)\right\}\\
&\geq&
\prob\left\{A_{3}\left(t\right)|t\in I_{3}\left(m\right)\right\}
\prob\left\{A_{4}\left(t\right)|t\in I_{4}\left(m\right)\right\}\\
&=&e^{-\tilde{\mu}_{3}\xi_{3}\left(m\right)}e^{-\tilde{\mu}_{4}\xi_{4}\left(m\right)}
=e^{-\left(\tilde{\mu}_{3}\xi_{3}\left(m\right)+\tilde{\mu}_{4}\xi_{4}\left(m\right)\right)}.
\end{eqnarray*}}

Es decir, 

\begin{equation}
\prob\left\{A_{3}\left(t\right),A_{4}\left(t\right)|t\in I_{4}\left(m\right)\right\}\geq
e^{-\left(\tilde{\mu}_{3}+\tilde{\mu}_{4}\right)\xi_{3}\left(m\right)}>0.
\end{equation}


Para el intervalo $I_{3}\left(m\right)=\left[\tau_{4}\left(m\right),\overline{\tau}_{4}\left(m\right)\right]$, se tiene que este caso es an\'alogo al caso (a).


Para el intevalo $I_{3}\left(m\right)\equiv\left[\overline{\tau}_{4}\left(m\right),\tau_{4}\left(m+1\right)\right]$, se tiene que es an\'alogo al caso (b).


Por construcci\'on se tiene que $I\left(n,m\right)\equiv I_{1}\left(n\right)\cap I_{3}\left(m\right)\neq\emptyset$,entonces en particular se tienen las contenciones $I\left(n,m\right)\subseteq I_{1}\left(n\right)$ y $I\left(n,m\right)\subseteq I_{3}\left(m\right)$, por lo tanto si definimos $\xi_{n,m}\equiv\ell\left(I\left(n,m\right)\right)$ tenemos que

\begin{eqnarray*}
\xi_{n,m}\leq\xi_{1}\left(n\right)\textrm{ y }\xi_{n,m}\leq\xi_{3}\left(m\right)\textrm{ entonces }\\
-\xi_{n,m}\geq-\xi_{1}\left(n\right)\textrm{ y }-\xi_{n,m}\leq-\xi_{3}\left(m\right)\\
\end{eqnarray*}
por lo tanto tenemos las desigualdades 


\begin{eqnarray*}
\begin{array}{ll}
-\tilde{\mu}_{1}\xi_{n,m}\geq-\tilde{\mu}_{1}\xi_{1}\left(n\right),&
-\tilde{\mu}_{2}\xi_{n,m}\geq-\tilde{\mu}_{2}\xi_{1}\left(n\right)
\geq-\tilde{\mu}_{2}\xi_{2}\left(n\right),\\
-\tilde{\mu}_{3}\xi_{n,m}\geq-\tilde{\mu}_{3}\xi_{3}\left(m\right),&
-\tilde{\mu}_{4}\xi_{n,m}\geq-\tilde{\mu}_{4}\xi_{3}\left(m\right)
\geq-\tilde{\mu}_{4}\xi_{4}\left(m\right).
\end{array}
\end{eqnarray*}

Sea $T^{*}\in I\left(n,m\right)$, entonces dado que en particular $T^{*}\in I_{1}\left(n\right)$, se cumple con probabilidad positiva que no hay arribos a las colas $Q_{1}$ y $Q_{2}$, en consecuencia, tampoco hay usuarios de transferencia para $Q_{3}$ y $Q_{4}$, es decir, $\tilde{\mu}_{1}=\mu_{1}$, $\tilde{\mu}_{2}=\mu_{2}$, $\tilde{\mu}_{3}=\mu_{3}$, $\tilde{\mu}_{4}=\mu_{4}$, es decir, los eventos $Q_{1}$ y $Q_{3}$ son condicionalmente independientes en el intervalo $I\left(n,m\right)$; lo mismo ocurre para las colas $Q_{2}$ y $Q_{4}$, por lo tanto tenemos que
%\small{
\begin{eqnarray}
\begin{array}{l}
\prob\left\{A_{1}\left(T^{*}\right),A_{2}\left(T^{*}\right),
A_{3}\left(T^{*}\right),A_{4}\left(T^{*}\right)|T^{*}\in I\left(n,m\right)\right\}\\
=\prod_{j=1}^{4}\prob\left\{A_{j}\left(T^{*}\right)|T^{*}\in I\left(n,m\right)\right\}\\
\geq\prob\left\{A_{1}\left(T^{*}\right)|T^{*}\in I_{1}\left(n\right)\right\}
\prob\left\{A_{2}\left(T^{*}\right)|T^{*}\in I_{2}\left(n\right)\right\}\\
\prob\left\{A_{3}\left(T^{*}\right)|T^{*}\in I_{3}\left(m\right)\right\}
\prob\left\{A_{4}\left(T^{*}\right)|T^{*}\in I_{4}\left(m\right)\right\}\\
=e^{-\mu_{1}\xi_{1}\left(n\right)}
e^{-\mu_{2}\xi_{2}\left(n\right)}
e^{-\mu_{3}\xi_{3}\left(m\right)}
e^{-\mu_{4}\xi_{4}\left(m\right)}\\
=e^{-\left[\tilde{\mu}_{1}\xi_{1}\left(n\right)
+\tilde{\mu}_{2}\xi_{2}\left(n\right)
+\tilde{\mu}_{3}\xi_{3}\left(m\right)
+\tilde{\mu}_{4}\xi_{4}
\left(m\right)\right]}>0.
\end{array}
\end{eqnarray}


Ahora solo resta demostrar que para $n\ge1$, existe $m\geq1$ tal que se cumplen cualquiera de los cuatro casos arriba mencionados: 

\begin{itemize}
\item[a)] $\tau_{3}\left(m\right)<\tau_{2}\left(n\right)<\overline{\tau}_{3}\left(m\right)$

\item[b)] $\overline{\tau}_{3}\left(m\right)<\tau_{2}\left(n\right)
<\tau_{4}\left(m\right)$

\item[c)] $\tau_{4}\left(m\right)<\tau_{2}\left(n\right)<
\overline{\tau}_{4}\left(m\right)$

\item[d)] $\overline{\tau}_{4}\left(m\right)<\tau_{2}\left(n\right)
<\tau_{3}\left(m+1\right)$
\end{itemize}

Consideremos nuevamente el primer caso. Supongamos que no existe $m\geq1$, tal que $I_{1}\left(n\right)\cap I_{3}\left(m\right)\neq\emptyset$, es decir, para toda $m\geq1$, $I_{1}\left(n\right)\cap I_{3}\left(m\right)=\emptyset$, entonces se tiene que ocurren cualquiera de los dos casos

\begin{itemize}
\item[a)] $\tau_{2}\left(n\right)\leq\tau_{3}\left(m\right)$: Recordemos que $\tau_{2}\left(m\right)=\overline{\tau}_{1}+r_{1}\left(m\right)$ donde cada una de las variables aleatorias son tales que $\esp\left[\overline{\tau}_{1}\left(n\right)-\tau_{1}\left(n\right)\right]<\infty$, $\esp\left[R_{1}\right]<\infty$ y $\esp\left[\tau_{3}\left(m\right)\right]<\infty$, lo cual contradice el hecho de que no exista un ciclo $m\geq1$ que satisfaga la condici\'on deseada.

\item[b)] $\tau_{2}\left(n\right)\geq\overline{\tau}_{3}\left(m\right)$: por un argumento similar al anterior se tiene que no es posible que no exista un ciclo $m\geq1$ tal que satisaface la condici\'on deseada.

\end{itemize}

Para el resto de los casos la demostraci\'on es an\'aloga. Por lo tanto, se tiene que efectivamente existe $m\geq1$ tal que $\tau_{3}\left(m\right)<\tau_{2}\left(n\right)<\tau_{4}\left(m\right)$.
\end{proof}
\newpage

En Sigman, Thorison y Wolff \cite{Sigman2} prueban que para la existencia de un una sucesi\'on infinita no decreciente de tiempos de regeneraci\'on $\tau_{1}\leq\tau_{2}\leq\cdots$ en los cuales el proceso se regenera, basta un tiempo de regeneraci\'on $R_{1}$, donde $R_{j}=\tau_{j}-\tau_{j-1}$. Para tal efecto se requiere la existencia de un espacio de probabilidad $\left(\Omega,\mathcal{F},\prob\right)$, y proceso estoc\'astico $\textit{X}=\left\{X\left(t\right):t\geq0\right\}$ con espacio de estados $\left(S,\mathcal{R}\right)$, con $\mathcal{R}$ $\sigma$-\'algebra.

\begin{Prop}
Si existe una variable aleatoria no negativa $R_{1}$ tal que $\theta_{R1}X=_{D}X$, entonces $\left(\Omega,\mathcal{F},\prob\right)$ puede extenderse para soportar una sucesi\'on estacionaria de variables aleatorias $R=\left\{R_{k}:k\geq1\right\}$, tal que para $k\geq1$,
\begin{eqnarray*}
\theta_{k}\left(X,R\right)=_{D}\left(X,R\right).
\end{eqnarray*}

Adem\'as, para $k\geq1$, $\theta_{k}R$ es condicionalmente independiente de $\left(X,R_{1},\ldots,R_{k}\right)$, dado $\theta_{\tau k}X$.

\end{Prop}


\begin{itemize}
\item Doob en 1953 demostr\'o que el estado estacionario de un proceso de partida en un sistema de espera $M/G/\infty$, es Poisson con la misma tasa que el proceso de arribos.

\item Burke en 1968, fue el primero en demostrar que el estado estacionario de un proceso de salida de una cola $M/M/s$ es un proceso Poisson.

\item Disney en 1973 obtuvo el siguiente resultado:

\begin{Teo}
Para el sistema de espera $M/G/1/L$ con disciplina FIFO, el proceso $\textbf{I}$ es un proceso de renovaci\'on si y s\'olo si el proceso denominado longitud de la cola es estacionario y se cumple cualquiera de los siguientes casos:

\begin{itemize}
\item[a)] Los tiempos de servicio son identicamente cero;
\item[b)] $L=0$, para cualquier proceso de servicio $S$;
\item[c)] $L=1$ y $G=D$;
\item[d)] $L=\infty$ y $G=M$.
\end{itemize}
En estos casos, respectivamente, las distribuciones de interpartida $P\left\{T_{n+1}-T_{n}\leq t\right\}$ son


\begin{itemize}
\item[a)] $1-e^{-\lambda t}$, $t\geq0$;
\item[b)] $1-e^{-\lambda t}*F\left(t\right)$, $t\geq0$;
\item[c)] $1-e^{-\lambda t}*\indora_{d}\left(t\right)$, $t\geq0$;
\item[d)] $1-e^{-\lambda t}*F\left(t\right)$, $t\geq0$.
\end{itemize}
\end{Teo}


\item Finch (1959) mostr\'o que para los sistemas $M/G/1/L$, con $1\leq L\leq \infty$ con distribuciones de servicio dos veces diferenciable, solamente el sistema $M/M/1/\infty$ tiene proceso de salida de renovaci\'on estacionario.

\item King (1971) demostro que un sistema de colas estacionario $M/G/1/1$ tiene sus tiempos de interpartida sucesivas $D_{n}$ y $D_{n+1}$ son independientes, si y s\'olo si, $G=D$, en cuyo caso le proceso de salida es de renovaci\'on.

\item Disney (1973) demostr\'o que el \'unico sistema estacionario $M/G/1/L$, que tiene proceso de salida de renovaci\'on  son los sistemas $M/M/1$ y $M/D/1/1$.



\item El siguiente resultado es de Disney y Koning (1985)
\begin{Teo}
En un sistema de espera $M/G/s$, el estado estacionario del proceso de salida es un proceso Poisson para cualquier distribuci\'on de los tiempos de servicio si el sistema tiene cualquiera de las siguientes cuatro propiedades.

\begin{itemize}
\item[a)] $s=\infty$
\item[b)] La disciplina de servicio es de procesador compartido.
\item[c)] La disciplina de servicio es LCFS y preemptive resume, esto se cumple para $L<\infty$
\item[d)] $G=M$.
\end{itemize}

\end{Teo}

\item El siguiente resultado es de Alamatsaz (1983)

\begin{Teo}
En cualquier sistema de colas $GI/G/1/L$ con $1\leq L<\infty$ y distribuci\'on de interarribos $A$ y distribuci\'on de los tiempos de servicio $B$, tal que $A\left(0\right)=0$, $A\left(t\right)\left(1-B\left(t\right)\right)>0$ para alguna $t>0$ y $B\left(t\right)$ para toda $t>0$, es imposible que el proceso de salida estacionario sea de renovaci\'on.
\end{Teo}

\end{itemize}



%________________________________________________________________________
%\subsection{Procesos Regenerativos Sigman, Thorisson y Wolff \cite{Sigman1}}
%________________________________________________________________________


\begin{Def}[Definici\'on Cl\'asica]
Un proceso estoc\'astico $X=\left\{X\left(t\right):t\geq0\right\}$ es llamado regenerativo is existe una variable aleatoria $R_{1}>0$ tal que
\begin{itemize}
\item[i)] $\left\{X\left(t+R_{1}\right):t\geq0\right\}$ es independiente de $\left\{\left\{X\left(t\right):t<R_{1}\right\},\right\}$
\item[ii)] $\left\{X\left(t+R_{1}\right):t\geq0\right\}$ es estoc\'asticamente equivalente a $\left\{X\left(t\right):t>0\right\}$
\end{itemize}

Llamamos a $R_{1}$ tiempo de regeneraci\'on, y decimos que $X$ se regenera en este punto.
\end{Def}

$\left\{X\left(t+R_{1}\right)\right\}$ es regenerativo con tiempo de regeneraci\'on $R_{2}$, independiente de $R_{1}$ pero con la misma distribuci\'on que $R_{1}$. Procediendo de esta manera se obtiene una secuencia de variables aleatorias independientes e id\'enticamente distribuidas $\left\{R_{n}\right\}$ llamados longitudes de ciclo. Si definimos a $Z_{k}\equiv R_{1}+R_{2}+\cdots+R_{k}$, se tiene un proceso de renovaci\'on llamado proceso de renovaci\'on encajado para $X$.


\begin{Note}
La existencia de un primer tiempo de regeneraci\'on, $R_{1}$, implica la existencia de una sucesi\'on completa de estos tiempos $R_{1},R_{2}\ldots,$ que satisfacen la propiedad deseada \cite{Sigman2}.
\end{Note}


\begin{Note} Para la cola $GI/GI/1$ los usuarios arriban con tiempos $t_{n}$ y son atendidos con tiempos de servicio $S_{n}$, los tiempos de arribo forman un proceso de renovaci\'on  con tiempos entre arribos independientes e identicamente distribuidos (\texttt{i.i.d.})$T_{n}=t_{n}-t_{n-1}$, adem\'as los tiempos de servicio son \texttt{i.i.d.} e independientes de los procesos de arribo. Por \textit{estable} se entiende que $\esp S_{n}<\esp T_{n}<\infty$.
\end{Note}
 


\begin{Def}
Para $x$ fijo y para cada $t\geq0$, sea $I_{x}\left(t\right)=1$ si $X\left(t\right)\leq x$,  $I_{x}\left(t\right)=0$ en caso contrario, y def\'inanse los tiempos promedio
\begin{eqnarray*}
\overline{X}&=&lim_{t\rightarrow\infty}\frac{1}{t}\int_{0}^{\infty}X\left(u\right)du\\
\prob\left(X_{\infty}\leq x\right)&=&lim_{t\rightarrow\infty}\frac{1}{t}\int_{0}^{\infty}I_{x}\left(u\right)du,
\end{eqnarray*}
cuando estos l\'imites existan.
\end{Def}

Como consecuencia del teorema de Renovaci\'on-Recompensa, se tiene que el primer l\'imite  existe y es igual a la constante
\begin{eqnarray*}
\overline{X}&=&\frac{\esp\left[\int_{0}^{R_{1}}X\left(t\right)dt\right]}{\esp\left[R_{1}\right]},
\end{eqnarray*}
suponiendo que ambas esperanzas son finitas.
 
\begin{Note}
Funciones de procesos regenerativos son regenerativas, es decir, si $X\left(t\right)$ es regenerativo y se define el proceso $Y\left(t\right)$ por $Y\left(t\right)=f\left(X\left(t\right)\right)$ para alguna funci\'on Borel medible $f\left(\cdot\right)$. Adem\'as $Y$ es regenerativo con los mismos tiempos de renovaci\'on que $X$. 

En general, los tiempos de renovaci\'on, $Z_{k}$ de un proceso regenerativo no requieren ser tiempos de paro con respecto a la evoluci\'on de $X\left(t\right)$.
\end{Note} 

\begin{Note}
Una funci\'on de un proceso de Markov, usualmente no ser\'a un proceso de Markov, sin embargo ser\'a regenerativo si el proceso de Markov lo es.
\end{Note}

 
\begin{Note}
Un proceso regenerativo con media de la longitud de ciclo finita es llamado positivo recurrente.
\end{Note}


\begin{Note}
\begin{itemize}
\item[a)] Si el proceso regenerativo $X$ es positivo recurrente y tiene trayectorias muestrales no negativas, entonces la ecuaci\'on anterior es v\'alida.
\item[b)] Si $X$ es positivo recurrente regenerativo, podemos construir una \'unica versi\'on estacionaria de este proceso, $X_{e}=\left\{X_{e}\left(t\right)\right\}$, donde $X_{e}$ es un proceso estoc\'astico regenerativo y estrictamente estacionario, con distribuci\'on marginal distribuida como $X_{\infty}$
\end{itemize}
\end{Note}


%__________________________________________________________________________________________
%\subsection{Procesos Regenerativos Estacionarios - Stidham \cite{Stidham}}
%__________________________________________________________________________________________


Un proceso estoc\'astico a tiempo continuo $\left\{V\left(t\right),t\geq0\right\}$ es un proceso regenerativo si existe una sucesi\'on de variables aleatorias independientes e id\'enticamente distribuidas $\left\{X_{1},X_{2},\ldots\right\}$, sucesi\'on de renovaci\'on, tal que para cualquier conjunto de Borel $A$, 

\begin{eqnarray*}
\prob\left\{V\left(t\right)\in A|X_{1}+X_{2}+\cdots+X_{R\left(t\right)}=s,\left\{V\left(\tau\right),\tau<s\right\}\right\}=\prob\left\{V\left(t-s\right)\in A|X_{1}>t-s\right\},
\end{eqnarray*}
para todo $0\leq s\leq t$, donde $R\left(t\right)=\max\left\{X_{1}+X_{2}+\cdots+X_{j}\leq t\right\}=$n\'umero de renovaciones ({\emph{puntos de regeneraci\'on}}) que ocurren en $\left[0,t\right]$. El intervalo $\left[0,X_{1}\right)$ es llamado {\emph{primer ciclo de regeneraci\'on}} de $\left\{V\left(t \right),t\geq0\right\}$, $\left[X_{1},X_{1}+X_{2}\right)$ el {\emph{segundo ciclo de regeneraci\'on}}, y as\'i sucesivamente.

Sea $X=X_{1}$ y sea $F$ la funci\'on de distrbuci\'on de $X$


\begin{Def}
Se define el proceso estacionario, $\left\{V^{*}\left(t\right),t\geq0\right\}$, para $\left\{V\left(t\right),t\geq0\right\}$ por

\begin{eqnarray*}
\prob\left\{V\left(t\right)\in A\right\}=\frac{1}{\esp\left[X\right]}\int_{0}^{\infty}\prob\left\{V\left(t+x\right)\in A|X>x\right\}\left(1-F\left(x\right)\right)dx,
\end{eqnarray*} 
para todo $t\geq0$ y todo conjunto de Borel $A$.
\end{Def}

\begin{Def}
Una distribuci\'on se dice que es {\emph{aritm\'etica}} si todos sus puntos de incremento son m\'ultiplos de la forma $0,\lambda, 2\lambda,\ldots$ para alguna $\lambda>0$ entera.
\end{Def}


\begin{Def}
Una modificaci\'on medible de un proceso $\left\{V\left(t\right),t\geq0\right\}$, es una versi\'on de este, $\left\{V\left(t,w\right)\right\}$ conjuntamente medible para $t\geq0$ y para $w\in S$, $S$ espacio de estados para $\left\{V\left(t\right),t\geq0\right\}$.
\end{Def}

\begin{Teo}
Sea $\left\{V\left(t\right),t\geq\right\}$ un proceso regenerativo no negativo con modificaci\'on medible. Sea $\esp\left[X\right]<\infty$. Entonces el proceso estacionario dado por la ecuaci\'on anterior est\'a bien definido y tiene funci\'on de distribuci\'on independiente de $t$, adem\'as
\begin{itemize}
\item[i)] \begin{eqnarray*}
\esp\left[V^{*}\left(0\right)\right]&=&\frac{\esp\left[\int_{0}^{X}V\left(s\right)ds\right]}{\esp\left[X\right]}\end{eqnarray*}
\item[ii)] Si $\esp\left[V^{*}\left(0\right)\right]<\infty$, equivalentemente, si $\esp\left[\int_{0}^{X}V\left(s\right)ds\right]<\infty$,entonces
\begin{eqnarray*}
\frac{\int_{0}^{t}V\left(s\right)ds}{t}\rightarrow\frac{\esp\left[\int_{0}^{X}V\left(s\right)ds\right]}{\esp\left[X\right]}
\end{eqnarray*}
con probabilidad 1 y en media, cuando $t\rightarrow\infty$.
\end{itemize}
\end{Teo}

\begin{Coro}
Sea $\left\{V\left(t\right),t\geq0\right\}$ un proceso regenerativo no negativo, con modificaci\'on medible. Si $\esp <\infty$, $F$ es no-aritm\'etica, y para todo $x\geq0$, $P\left\{V\left(t\right)\leq x,C>x\right\}$ es de variaci\'on acotada como funci\'on de $t$ en cada intervalo finito $\left[0,\tau\right]$, entonces $V\left(t\right)$ converge en distribuci\'on  cuando $t\rightarrow\infty$ y $$\esp V=\frac{\esp \int_{0}^{X}V\left(s\right)ds}{\esp X}$$
Donde $V$ tiene la distribuci\'on l\'imite de $V\left(t\right)$ cuando $t\rightarrow\infty$.

\end{Coro}

Para el caso discreto se tienen resultados similares.



%______________________________________________________________________
%\subsection{Procesos de Renovaci\'on}
%______________________________________________________________________

\begin{Def}%\label{Def.Tn}
Sean $0\leq T_{1}\leq T_{2}\leq \ldots$ son tiempos aleatorios infinitos en los cuales ocurren ciertos eventos. El n\'umero de tiempos $T_{n}$ en el intervalo $\left[0,t\right)$ es

\begin{eqnarray}
N\left(t\right)=\sum_{n=1}^{\infty}\indora\left(T_{n}\leq t\right),
\end{eqnarray}
para $t\geq0$.
\end{Def}

Si se consideran los puntos $T_{n}$ como elementos de $\rea_{+}$, y $N\left(t\right)$ es el n\'umero de puntos en $\rea$. El proceso denotado por $\left\{N\left(t\right):t\geq0\right\}$, denotado por $N\left(t\right)$, es un proceso puntual en $\rea_{+}$. Los $T_{n}$ son los tiempos de ocurrencia, el proceso puntual $N\left(t\right)$ es simple si su n\'umero de ocurrencias son distintas: $0<T_{1}<T_{2}<\ldots$ casi seguramente.

\begin{Def}
Un proceso puntual $N\left(t\right)$ es un proceso de renovaci\'on si los tiempos de interocurrencia $\xi_{n}=T_{n}-T_{n-1}$, para $n\geq1$, son independientes e identicamente distribuidos con distribuci\'on $F$, donde $F\left(0\right)=0$ y $T_{0}=0$. Los $T_{n}$ son llamados tiempos de renovaci\'on, referente a la independencia o renovaci\'on de la informaci\'on estoc\'astica en estos tiempos. Los $\xi_{n}$ son los tiempos de inter-renovaci\'on, y $N\left(t\right)$ es el n\'umero de renovaciones en el intervalo $\left[0,t\right)$
\end{Def}


\begin{Note}
Para definir un proceso de renovaci\'on para cualquier contexto, solamente hay que especificar una distribuci\'on $F$, con $F\left(0\right)=0$, para los tiempos de inter-renovaci\'on. La funci\'on $F$ en turno degune las otra variables aleatorias. De manera formal, existe un espacio de probabilidad y una sucesi\'on de variables aleatorias $\xi_{1},\xi_{2},\ldots$ definidas en este con distribuci\'on $F$. Entonces las otras cantidades son $T_{n}=\sum_{k=1}^{n}\xi_{k}$ y $N\left(t\right)=\sum_{n=1}^{\infty}\indora\left(T_{n}\leq t\right)$, donde $T_{n}\rightarrow\infty$ casi seguramente por la Ley Fuerte de los Grandes Números.
\end{Note}

%___________________________________________________________________________________________
%
%\subsection{Teorema Principal de Renovaci\'on}
%___________________________________________________________________________________________
%

\begin{Note} Una funci\'on $h:\rea_{+}\rightarrow\rea$ es Directamente Riemann Integrable en los siguientes casos:
\begin{itemize}
\item[a)] $h\left(t\right)\geq0$ es decreciente y Riemann Integrable.
\item[b)] $h$ es continua excepto posiblemente en un conjunto de Lebesgue de medida 0, y $|h\left(t\right)|\leq b\left(t\right)$, donde $b$ es DRI.
\end{itemize}
\end{Note}

\begin{Teo}[Teorema Principal de Renovaci\'on]
Si $F$ es no aritm\'etica y $h\left(t\right)$ es Directamente Riemann Integrable (DRI), entonces

\begin{eqnarray*}
lim_{t\rightarrow\infty}U\star h=\frac{1}{\mu}\int_{\rea_{+}}h\left(s\right)ds.
\end{eqnarray*}
\end{Teo}

\begin{Prop}
Cualquier funci\'on $H\left(t\right)$ acotada en intervalos finitos y que es 0 para $t<0$ puede expresarse como
\begin{eqnarray*}
H\left(t\right)=U\star h\left(t\right)\textrm{,  donde }h\left(t\right)=H\left(t\right)-F\star H\left(t\right)
\end{eqnarray*}
\end{Prop}

\begin{Def}
Un proceso estoc\'astico $X\left(t\right)$ es crudamente regenerativo en un tiempo aleatorio positivo $T$ si
\begin{eqnarray*}
\esp\left[X\left(T+t\right)|T\right]=\esp\left[X\left(t\right)\right]\textrm{, para }t\geq0,\end{eqnarray*}
y con las esperanzas anteriores finitas.
\end{Def}

\begin{Prop}
Sup\'ongase que $X\left(t\right)$ es un proceso crudamente regenerativo en $T$, que tiene distribuci\'on $F$. Si $\esp\left[X\left(t\right)\right]$ es acotado en intervalos finitos, entonces
\begin{eqnarray*}
\esp\left[X\left(t\right)\right]=U\star h\left(t\right)\textrm{,  donde }h\left(t\right)=\esp\left[X\left(t\right)\indora\left(T>t\right)\right].
\end{eqnarray*}
\end{Prop}

\begin{Teo}[Regeneraci\'on Cruda]
Sup\'ongase que $X\left(t\right)$ es un proceso con valores positivo crudamente regenerativo en $T$, y def\'inase $M=\sup\left\{|X\left(t\right)|:t\leq T\right\}$. Si $T$ es no aritm\'etico y $M$ y $MT$ tienen media finita, entonces
\begin{eqnarray*}
lim_{t\rightarrow\infty}\esp\left[X\left(t\right)\right]=\frac{1}{\mu}\int_{\rea_{+}}h\left(s\right)ds,
\end{eqnarray*}
donde $h\left(t\right)=\esp\left[X\left(t\right)\indora\left(T>t\right)\right]$.
\end{Teo}

%___________________________________________________________________________________________
%
%\subsection{Propiedades de los Procesos de Renovaci\'on}
%___________________________________________________________________________________________
%

Los tiempos $T_{n}$ est\'an relacionados con los conteos de $N\left(t\right)$ por

\begin{eqnarray*}
\left\{N\left(t\right)\geq n\right\}&=&\left\{T_{n}\leq t\right\}\\
T_{N\left(t\right)}\leq &t&<T_{N\left(t\right)+1},
\end{eqnarray*}

adem\'as $N\left(T_{n}\right)=n$, y 

\begin{eqnarray*}
N\left(t\right)=\max\left\{n:T_{n}\leq t\right\}=\min\left\{n:T_{n+1}>t\right\}
\end{eqnarray*}

Por propiedades de la convoluci\'on se sabe que

\begin{eqnarray*}
P\left\{T_{n}\leq t\right\}=F^{n\star}\left(t\right)
\end{eqnarray*}
que es la $n$-\'esima convoluci\'on de $F$. Entonces 

\begin{eqnarray*}
\left\{N\left(t\right)\geq n\right\}&=&\left\{T_{n}\leq t\right\}\\
P\left\{N\left(t\right)\leq n\right\}&=&1-F^{\left(n+1\right)\star}\left(t\right)
\end{eqnarray*}

Adem\'as usando el hecho de que $\esp\left[N\left(t\right)\right]=\sum_{n=1}^{\infty}P\left\{N\left(t\right)\geq n\right\}$
se tiene que

\begin{eqnarray*}
\esp\left[N\left(t\right)\right]=\sum_{n=1}^{\infty}F^{n\star}\left(t\right)
\end{eqnarray*}

\begin{Prop}
Para cada $t\geq0$, la funci\'on generadora de momentos $\esp\left[e^{\alpha N\left(t\right)}\right]$ existe para alguna $\alpha$ en una vecindad del 0, y de aqu\'i que $\esp\left[N\left(t\right)^{m}\right]<\infty$, para $m\geq1$.
\end{Prop}


\begin{Note}
Si el primer tiempo de renovaci\'on $\xi_{1}$ no tiene la misma distribuci\'on que el resto de las $\xi_{n}$, para $n\geq2$, a $N\left(t\right)$ se le llama Proceso de Renovaci\'on retardado, donde si $\xi$ tiene distribuci\'on $G$, entonces el tiempo $T_{n}$ de la $n$-\'esima renovaci\'on tiene distribuci\'on $G\star F^{\left(n-1\right)\star}\left(t\right)$
\end{Note}


\begin{Teo}
Para una constante $\mu\leq\infty$ ( o variable aleatoria), las siguientes expresiones son equivalentes:

\begin{eqnarray}
lim_{n\rightarrow\infty}n^{-1}T_{n}&=&\mu,\textrm{ c.s.}\\
lim_{t\rightarrow\infty}t^{-1}N\left(t\right)&=&1/\mu,\textrm{ c.s.}
\end{eqnarray}
\end{Teo}


Es decir, $T_{n}$ satisface la Ley Fuerte de los Grandes N\'umeros s\'i y s\'olo s\'i $N\left/t\right)$ la cumple.


\begin{Coro}[Ley Fuerte de los Grandes N\'umeros para Procesos de Renovaci\'on]
Si $N\left(t\right)$ es un proceso de renovaci\'on cuyos tiempos de inter-renovaci\'on tienen media $\mu\leq\infty$, entonces
\begin{eqnarray}
t^{-1}N\left(t\right)\rightarrow 1/\mu,\textrm{ c.s. cuando }t\rightarrow\infty.
\end{eqnarray}

\end{Coro}


Considerar el proceso estoc\'astico de valores reales $\left\{Z\left(t\right):t\geq0\right\}$ en el mismo espacio de probabilidad que $N\left(t\right)$

\begin{Def}
Para el proceso $\left\{Z\left(t\right):t\geq0\right\}$ se define la fluctuaci\'on m\'axima de $Z\left(t\right)$ en el intervalo $\left(T_{n-1},T_{n}\right]$:
\begin{eqnarray*}
M_{n}=\sup_{T_{n-1}<t\leq T_{n}}|Z\left(t\right)-Z\left(T_{n-1}\right)|
\end{eqnarray*}
\end{Def}

\begin{Teo}
Sup\'ongase que $n^{-1}T_{n}\rightarrow\mu$ c.s. cuando $n\rightarrow\infty$, donde $\mu\leq\infty$ es una constante o variable aleatoria. Sea $a$ una constante o variable aleatoria que puede ser infinita cuando $\mu$ es finita, y considere las expresiones l\'imite:
\begin{eqnarray}
lim_{n\rightarrow\infty}n^{-1}Z\left(T_{n}\right)&=&a,\textrm{ c.s.}\\
lim_{t\rightarrow\infty}t^{-1}Z\left(t\right)&=&a/\mu,\textrm{ c.s.}
\end{eqnarray}
La segunda expresi\'on implica la primera. Conversamente, la primera implica la segunda si el proceso $Z\left(t\right)$ es creciente, o si $lim_{n\rightarrow\infty}n^{-1}M_{n}=0$ c.s.
\end{Teo}

\begin{Coro}
Si $N\left(t\right)$ es un proceso de renovaci\'on, y $\left(Z\left(T_{n}\right)-Z\left(T_{n-1}\right),M_{n}\right)$, para $n\geq1$, son variables aleatorias independientes e id\'enticamente distribuidas con media finita, entonces,
\begin{eqnarray}
lim_{t\rightarrow\infty}t^{-1}Z\left(t\right)\rightarrow\frac{\esp\left[Z\left(T_{1}\right)-Z\left(T_{0}\right)\right]}{\esp\left[T_{1}\right]},\textrm{ c.s. cuando  }t\rightarrow\infty.
\end{eqnarray}
\end{Coro}



%___________________________________________________________________________________________
%
%\subsection{Propiedades de los Procesos de Renovaci\'on}
%___________________________________________________________________________________________
%

Los tiempos $T_{n}$ est\'an relacionados con los conteos de $N\left(t\right)$ por

\begin{eqnarray*}
\left\{N\left(t\right)\geq n\right\}&=&\left\{T_{n}\leq t\right\}\\
T_{N\left(t\right)}\leq &t&<T_{N\left(t\right)+1},
\end{eqnarray*}

adem\'as $N\left(T_{n}\right)=n$, y 

\begin{eqnarray*}
N\left(t\right)=\max\left\{n:T_{n}\leq t\right\}=\min\left\{n:T_{n+1}>t\right\}
\end{eqnarray*}

Por propiedades de la convoluci\'on se sabe que

\begin{eqnarray*}
P\left\{T_{n}\leq t\right\}=F^{n\star}\left(t\right)
\end{eqnarray*}
que es la $n$-\'esima convoluci\'on de $F$. Entonces 

\begin{eqnarray*}
\left\{N\left(t\right)\geq n\right\}&=&\left\{T_{n}\leq t\right\}\\
P\left\{N\left(t\right)\leq n\right\}&=&1-F^{\left(n+1\right)\star}\left(t\right)
\end{eqnarray*}

Adem\'as usando el hecho de que $\esp\left[N\left(t\right)\right]=\sum_{n=1}^{\infty}P\left\{N\left(t\right)\geq n\right\}$
se tiene que

\begin{eqnarray*}
\esp\left[N\left(t\right)\right]=\sum_{n=1}^{\infty}F^{n\star}\left(t\right)
\end{eqnarray*}

\begin{Prop}
Para cada $t\geq0$, la funci\'on generadora de momentos $\esp\left[e^{\alpha N\left(t\right)}\right]$ existe para alguna $\alpha$ en una vecindad del 0, y de aqu\'i que $\esp\left[N\left(t\right)^{m}\right]<\infty$, para $m\geq1$.
\end{Prop}


\begin{Note}
Si el primer tiempo de renovaci\'on $\xi_{1}$ no tiene la misma distribuci\'on que el resto de las $\xi_{n}$, para $n\geq2$, a $N\left(t\right)$ se le llama Proceso de Renovaci\'on retardado, donde si $\xi$ tiene distribuci\'on $G$, entonces el tiempo $T_{n}$ de la $n$-\'esima renovaci\'on tiene distribuci\'on $G\star F^{\left(n-1\right)\star}\left(t\right)$
\end{Note}


\begin{Teo}
Para una constante $\mu\leq\infty$ ( o variable aleatoria), las siguientes expresiones son equivalentes:

\begin{eqnarray}
lim_{n\rightarrow\infty}n^{-1}T_{n}&=&\mu,\textrm{ c.s.}\\
lim_{t\rightarrow\infty}t^{-1}N\left(t\right)&=&1/\mu,\textrm{ c.s.}
\end{eqnarray}
\end{Teo}


Es decir, $T_{n}$ satisface la Ley Fuerte de los Grandes N\'umeros s\'i y s\'olo s\'i $N\left/t\right)$ la cumple.


\begin{Coro}[Ley Fuerte de los Grandes N\'umeros para Procesos de Renovaci\'on]
Si $N\left(t\right)$ es un proceso de renovaci\'on cuyos tiempos de inter-renovaci\'on tienen media $\mu\leq\infty$, entonces
\begin{eqnarray}
t^{-1}N\left(t\right)\rightarrow 1/\mu,\textrm{ c.s. cuando }t\rightarrow\infty.
\end{eqnarray}

\end{Coro}


Considerar el proceso estoc\'astico de valores reales $\left\{Z\left(t\right):t\geq0\right\}$ en el mismo espacio de probabilidad que $N\left(t\right)$

\begin{Def}
Para el proceso $\left\{Z\left(t\right):t\geq0\right\}$ se define la fluctuaci\'on m\'axima de $Z\left(t\right)$ en el intervalo $\left(T_{n-1},T_{n}\right]$:
\begin{eqnarray*}
M_{n}=\sup_{T_{n-1}<t\leq T_{n}}|Z\left(t\right)-Z\left(T_{n-1}\right)|
\end{eqnarray*}
\end{Def}

\begin{Teo}
Sup\'ongase que $n^{-1}T_{n}\rightarrow\mu$ c.s. cuando $n\rightarrow\infty$, donde $\mu\leq\infty$ es una constante o variable aleatoria. Sea $a$ una constante o variable aleatoria que puede ser infinita cuando $\mu$ es finita, y considere las expresiones l\'imite:
\begin{eqnarray}
lim_{n\rightarrow\infty}n^{-1}Z\left(T_{n}\right)&=&a,\textrm{ c.s.}\\
lim_{t\rightarrow\infty}t^{-1}Z\left(t\right)&=&a/\mu,\textrm{ c.s.}
\end{eqnarray}
La segunda expresi\'on implica la primera. Conversamente, la primera implica la segunda si el proceso $Z\left(t\right)$ es creciente, o si $lim_{n\rightarrow\infty}n^{-1}M_{n}=0$ c.s.
\end{Teo}

\begin{Coro}
Si $N\left(t\right)$ es un proceso de renovaci\'on, y $\left(Z\left(T_{n}\right)-Z\left(T_{n-1}\right),M_{n}\right)$, para $n\geq1$, son variables aleatorias independientes e id\'enticamente distribuidas con media finita, entonces,
\begin{eqnarray}
lim_{t\rightarrow\infty}t^{-1}Z\left(t\right)\rightarrow\frac{\esp\left[Z\left(T_{1}\right)-Z\left(T_{0}\right)\right]}{\esp\left[T_{1}\right]},\textrm{ c.s. cuando  }t\rightarrow\infty.
\end{eqnarray}
\end{Coro}


%___________________________________________________________________________________________
%
%\subsection{Propiedades de los Procesos de Renovaci\'on}
%___________________________________________________________________________________________
%

Los tiempos $T_{n}$ est\'an relacionados con los conteos de $N\left(t\right)$ por

\begin{eqnarray*}
\left\{N\left(t\right)\geq n\right\}&=&\left\{T_{n}\leq t\right\}\\
T_{N\left(t\right)}\leq &t&<T_{N\left(t\right)+1},
\end{eqnarray*}

adem\'as $N\left(T_{n}\right)=n$, y 

\begin{eqnarray*}
N\left(t\right)=\max\left\{n:T_{n}\leq t\right\}=\min\left\{n:T_{n+1}>t\right\}
\end{eqnarray*}

Por propiedades de la convoluci\'on se sabe que

\begin{eqnarray*}
P\left\{T_{n}\leq t\right\}=F^{n\star}\left(t\right)
\end{eqnarray*}
que es la $n$-\'esima convoluci\'on de $F$. Entonces 

\begin{eqnarray*}
\left\{N\left(t\right)\geq n\right\}&=&\left\{T_{n}\leq t\right\}\\
P\left\{N\left(t\right)\leq n\right\}&=&1-F^{\left(n+1\right)\star}\left(t\right)
\end{eqnarray*}

Adem\'as usando el hecho de que $\esp\left[N\left(t\right)\right]=\sum_{n=1}^{\infty}P\left\{N\left(t\right)\geq n\right\}$
se tiene que

\begin{eqnarray*}
\esp\left[N\left(t\right)\right]=\sum_{n=1}^{\infty}F^{n\star}\left(t\right)
\end{eqnarray*}

\begin{Prop}
Para cada $t\geq0$, la funci\'on generadora de momentos $\esp\left[e^{\alpha N\left(t\right)}\right]$ existe para alguna $\alpha$ en una vecindad del 0, y de aqu\'i que $\esp\left[N\left(t\right)^{m}\right]<\infty$, para $m\geq1$.
\end{Prop}


\begin{Note}
Si el primer tiempo de renovaci\'on $\xi_{1}$ no tiene la misma distribuci\'on que el resto de las $\xi_{n}$, para $n\geq2$, a $N\left(t\right)$ se le llama Proceso de Renovaci\'on retardado, donde si $\xi$ tiene distribuci\'on $G$, entonces el tiempo $T_{n}$ de la $n$-\'esima renovaci\'on tiene distribuci\'on $G\star F^{\left(n-1\right)\star}\left(t\right)$
\end{Note}


\begin{Teo}
Para una constante $\mu\leq\infty$ ( o variable aleatoria), las siguientes expresiones son equivalentes:

\begin{eqnarray}
lim_{n\rightarrow\infty}n^{-1}T_{n}&=&\mu,\textrm{ c.s.}\\
lim_{t\rightarrow\infty}t^{-1}N\left(t\right)&=&1/\mu,\textrm{ c.s.}
\end{eqnarray}
\end{Teo}


Es decir, $T_{n}$ satisface la Ley Fuerte de los Grandes N\'umeros s\'i y s\'olo s\'i $N\left/t\right)$ la cumple.


\begin{Coro}[Ley Fuerte de los Grandes N\'umeros para Procesos de Renovaci\'on]
Si $N\left(t\right)$ es un proceso de renovaci\'on cuyos tiempos de inter-renovaci\'on tienen media $\mu\leq\infty$, entonces
\begin{eqnarray}
t^{-1}N\left(t\right)\rightarrow 1/\mu,\textrm{ c.s. cuando }t\rightarrow\infty.
\end{eqnarray}

\end{Coro}


Considerar el proceso estoc\'astico de valores reales $\left\{Z\left(t\right):t\geq0\right\}$ en el mismo espacio de probabilidad que $N\left(t\right)$

\begin{Def}
Para el proceso $\left\{Z\left(t\right):t\geq0\right\}$ se define la fluctuaci\'on m\'axima de $Z\left(t\right)$ en el intervalo $\left(T_{n-1},T_{n}\right]$:
\begin{eqnarray*}
M_{n}=\sup_{T_{n-1}<t\leq T_{n}}|Z\left(t\right)-Z\left(T_{n-1}\right)|
\end{eqnarray*}
\end{Def}

\begin{Teo}
Sup\'ongase que $n^{-1}T_{n}\rightarrow\mu$ c.s. cuando $n\rightarrow\infty$, donde $\mu\leq\infty$ es una constante o variable aleatoria. Sea $a$ una constante o variable aleatoria que puede ser infinita cuando $\mu$ es finita, y considere las expresiones l\'imite:
\begin{eqnarray}
lim_{n\rightarrow\infty}n^{-1}Z\left(T_{n}\right)&=&a,\textrm{ c.s.}\\
lim_{t\rightarrow\infty}t^{-1}Z\left(t\right)&=&a/\mu,\textrm{ c.s.}
\end{eqnarray}
La segunda expresi\'on implica la primera. Conversamente, la primera implica la segunda si el proceso $Z\left(t\right)$ es creciente, o si $lim_{n\rightarrow\infty}n^{-1}M_{n}=0$ c.s.
\end{Teo}

\begin{Coro}
Si $N\left(t\right)$ es un proceso de renovaci\'on, y $\left(Z\left(T_{n}\right)-Z\left(T_{n-1}\right),M_{n}\right)$, para $n\geq1$, son variables aleatorias independientes e id\'enticamente distribuidas con media finita, entonces,
\begin{eqnarray}
lim_{t\rightarrow\infty}t^{-1}Z\left(t\right)\rightarrow\frac{\esp\left[Z\left(T_{1}\right)-Z\left(T_{0}\right)\right]}{\esp\left[T_{1}\right]},\textrm{ c.s. cuando  }t\rightarrow\infty.
\end{eqnarray}
\end{Coro}

%___________________________________________________________________________________________
%
%\subsection{Propiedades de los Procesos de Renovaci\'on}
%___________________________________________________________________________________________
%

Los tiempos $T_{n}$ est\'an relacionados con los conteos de $N\left(t\right)$ por

\begin{eqnarray*}
\left\{N\left(t\right)\geq n\right\}&=&\left\{T_{n}\leq t\right\}\\
T_{N\left(t\right)}\leq &t&<T_{N\left(t\right)+1},
\end{eqnarray*}

adem\'as $N\left(T_{n}\right)=n$, y 

\begin{eqnarray*}
N\left(t\right)=\max\left\{n:T_{n}\leq t\right\}=\min\left\{n:T_{n+1}>t\right\}
\end{eqnarray*}

Por propiedades de la convoluci\'on se sabe que

\begin{eqnarray*}
P\left\{T_{n}\leq t\right\}=F^{n\star}\left(t\right)
\end{eqnarray*}
que es la $n$-\'esima convoluci\'on de $F$. Entonces 

\begin{eqnarray*}
\left\{N\left(t\right)\geq n\right\}&=&\left\{T_{n}\leq t\right\}\\
P\left\{N\left(t\right)\leq n\right\}&=&1-F^{\left(n+1\right)\star}\left(t\right)
\end{eqnarray*}

Adem\'as usando el hecho de que $\esp\left[N\left(t\right)\right]=\sum_{n=1}^{\infty}P\left\{N\left(t\right)\geq n\right\}$
se tiene que

\begin{eqnarray*}
\esp\left[N\left(t\right)\right]=\sum_{n=1}^{\infty}F^{n\star}\left(t\right)
\end{eqnarray*}

\begin{Prop}
Para cada $t\geq0$, la funci\'on generadora de momentos $\esp\left[e^{\alpha N\left(t\right)}\right]$ existe para alguna $\alpha$ en una vecindad del 0, y de aqu\'i que $\esp\left[N\left(t\right)^{m}\right]<\infty$, para $m\geq1$.
\end{Prop}


\begin{Note}
Si el primer tiempo de renovaci\'on $\xi_{1}$ no tiene la misma distribuci\'on que el resto de las $\xi_{n}$, para $n\geq2$, a $N\left(t\right)$ se le llama Proceso de Renovaci\'on retardado, donde si $\xi$ tiene distribuci\'on $G$, entonces el tiempo $T_{n}$ de la $n$-\'esima renovaci\'on tiene distribuci\'on $G\star F^{\left(n-1\right)\star}\left(t\right)$
\end{Note}


\begin{Teo}
Para una constante $\mu\leq\infty$ ( o variable aleatoria), las siguientes expresiones son equivalentes:

\begin{eqnarray}
lim_{n\rightarrow\infty}n^{-1}T_{n}&=&\mu,\textrm{ c.s.}\\
lim_{t\rightarrow\infty}t^{-1}N\left(t\right)&=&1/\mu,\textrm{ c.s.}
\end{eqnarray}
\end{Teo}


Es decir, $T_{n}$ satisface la Ley Fuerte de los Grandes N\'umeros s\'i y s\'olo s\'i $N\left/t\right)$ la cumple.


\begin{Coro}[Ley Fuerte de los Grandes N\'umeros para Procesos de Renovaci\'on]
Si $N\left(t\right)$ es un proceso de renovaci\'on cuyos tiempos de inter-renovaci\'on tienen media $\mu\leq\infty$, entonces
\begin{eqnarray}
t^{-1}N\left(t\right)\rightarrow 1/\mu,\textrm{ c.s. cuando }t\rightarrow\infty.
\end{eqnarray}

\end{Coro}


Considerar el proceso estoc\'astico de valores reales $\left\{Z\left(t\right):t\geq0\right\}$ en el mismo espacio de probabilidad que $N\left(t\right)$

\begin{Def}
Para el proceso $\left\{Z\left(t\right):t\geq0\right\}$ se define la fluctuaci\'on m\'axima de $Z\left(t\right)$ en el intervalo $\left(T_{n-1},T_{n}\right]$:
\begin{eqnarray*}
M_{n}=\sup_{T_{n-1}<t\leq T_{n}}|Z\left(t\right)-Z\left(T_{n-1}\right)|
\end{eqnarray*}
\end{Def}

\begin{Teo}
Sup\'ongase que $n^{-1}T_{n}\rightarrow\mu$ c.s. cuando $n\rightarrow\infty$, donde $\mu\leq\infty$ es una constante o variable aleatoria. Sea $a$ una constante o variable aleatoria que puede ser infinita cuando $\mu$ es finita, y considere las expresiones l\'imite:
\begin{eqnarray}
lim_{n\rightarrow\infty}n^{-1}Z\left(T_{n}\right)&=&a,\textrm{ c.s.}\\
lim_{t\rightarrow\infty}t^{-1}Z\left(t\right)&=&a/\mu,\textrm{ c.s.}
\end{eqnarray}
La segunda expresi\'on implica la primera. Conversamente, la primera implica la segunda si el proceso $Z\left(t\right)$ es creciente, o si $lim_{n\rightarrow\infty}n^{-1}M_{n}=0$ c.s.
\end{Teo}

\begin{Coro}
Si $N\left(t\right)$ es un proceso de renovaci\'on, y $\left(Z\left(T_{n}\right)-Z\left(T_{n-1}\right),M_{n}\right)$, para $n\geq1$, son variables aleatorias independientes e id\'enticamente distribuidas con media finita, entonces,
\begin{eqnarray}
lim_{t\rightarrow\infty}t^{-1}Z\left(t\right)\rightarrow\frac{\esp\left[Z\left(T_{1}\right)-Z\left(T_{0}\right)\right]}{\esp\left[T_{1}\right]},\textrm{ c.s. cuando  }t\rightarrow\infty.
\end{eqnarray}
\end{Coro}
%___________________________________________________________________________________________
%
%\subsection{Propiedades de los Procesos de Renovaci\'on}
%___________________________________________________________________________________________
%

Los tiempos $T_{n}$ est\'an relacionados con los conteos de $N\left(t\right)$ por

\begin{eqnarray*}
\left\{N\left(t\right)\geq n\right\}&=&\left\{T_{n}\leq t\right\}\\
T_{N\left(t\right)}\leq &t&<T_{N\left(t\right)+1},
\end{eqnarray*}

adem\'as $N\left(T_{n}\right)=n$, y 

\begin{eqnarray*}
N\left(t\right)=\max\left\{n:T_{n}\leq t\right\}=\min\left\{n:T_{n+1}>t\right\}
\end{eqnarray*}

Por propiedades de la convoluci\'on se sabe que

\begin{eqnarray*}
P\left\{T_{n}\leq t\right\}=F^{n\star}\left(t\right)
\end{eqnarray*}
que es la $n$-\'esima convoluci\'on de $F$. Entonces 

\begin{eqnarray*}
\left\{N\left(t\right)\geq n\right\}&=&\left\{T_{n}\leq t\right\}\\
P\left\{N\left(t\right)\leq n\right\}&=&1-F^{\left(n+1\right)\star}\left(t\right)
\end{eqnarray*}

Adem\'as usando el hecho de que $\esp\left[N\left(t\right)\right]=\sum_{n=1}^{\infty}P\left\{N\left(t\right)\geq n\right\}$
se tiene que

\begin{eqnarray*}
\esp\left[N\left(t\right)\right]=\sum_{n=1}^{\infty}F^{n\star}\left(t\right)
\end{eqnarray*}

\begin{Prop}
Para cada $t\geq0$, la funci\'on generadora de momentos $\esp\left[e^{\alpha N\left(t\right)}\right]$ existe para alguna $\alpha$ en una vecindad del 0, y de aqu\'i que $\esp\left[N\left(t\right)^{m}\right]<\infty$, para $m\geq1$.
\end{Prop}


\begin{Note}
Si el primer tiempo de renovaci\'on $\xi_{1}$ no tiene la misma distribuci\'on que el resto de las $\xi_{n}$, para $n\geq2$, a $N\left(t\right)$ se le llama Proceso de Renovaci\'on retardado, donde si $\xi$ tiene distribuci\'on $G$, entonces el tiempo $T_{n}$ de la $n$-\'esima renovaci\'on tiene distribuci\'on $G\star F^{\left(n-1\right)\star}\left(t\right)$
\end{Note}


\begin{Teo}
Para una constante $\mu\leq\infty$ ( o variable aleatoria), las siguientes expresiones son equivalentes:

\begin{eqnarray}
lim_{n\rightarrow\infty}n^{-1}T_{n}&=&\mu,\textrm{ c.s.}\\
lim_{t\rightarrow\infty}t^{-1}N\left(t\right)&=&1/\mu,\textrm{ c.s.}
\end{eqnarray}
\end{Teo}


Es decir, $T_{n}$ satisface la Ley Fuerte de los Grandes N\'umeros s\'i y s\'olo s\'i $N\left/t\right)$ la cumple.


\begin{Coro}[Ley Fuerte de los Grandes N\'umeros para Procesos de Renovaci\'on]
Si $N\left(t\right)$ es un proceso de renovaci\'on cuyos tiempos de inter-renovaci\'on tienen media $\mu\leq\infty$, entonces
\begin{eqnarray}
t^{-1}N\left(t\right)\rightarrow 1/\mu,\textrm{ c.s. cuando }t\rightarrow\infty.
\end{eqnarray}

\end{Coro}


Considerar el proceso estoc\'astico de valores reales $\left\{Z\left(t\right):t\geq0\right\}$ en el mismo espacio de probabilidad que $N\left(t\right)$

\begin{Def}
Para el proceso $\left\{Z\left(t\right):t\geq0\right\}$ se define la fluctuaci\'on m\'axima de $Z\left(t\right)$ en el intervalo $\left(T_{n-1},T_{n}\right]$:
\begin{eqnarray*}
M_{n}=\sup_{T_{n-1}<t\leq T_{n}}|Z\left(t\right)-Z\left(T_{n-1}\right)|
\end{eqnarray*}
\end{Def}

\begin{Teo}
Sup\'ongase que $n^{-1}T_{n}\rightarrow\mu$ c.s. cuando $n\rightarrow\infty$, donde $\mu\leq\infty$ es una constante o variable aleatoria. Sea $a$ una constante o variable aleatoria que puede ser infinita cuando $\mu$ es finita, y considere las expresiones l\'imite:
\begin{eqnarray}
lim_{n\rightarrow\infty}n^{-1}Z\left(T_{n}\right)&=&a,\textrm{ c.s.}\\
lim_{t\rightarrow\infty}t^{-1}Z\left(t\right)&=&a/\mu,\textrm{ c.s.}
\end{eqnarray}
La segunda expresi\'on implica la primera. Conversamente, la primera implica la segunda si el proceso $Z\left(t\right)$ es creciente, o si $lim_{n\rightarrow\infty}n^{-1}M_{n}=0$ c.s.
\end{Teo}

\begin{Coro}
Si $N\left(t\right)$ es un proceso de renovaci\'on, y $\left(Z\left(T_{n}\right)-Z\left(T_{n-1}\right),M_{n}\right)$, para $n\geq1$, son variables aleatorias independientes e id\'enticamente distribuidas con media finita, entonces,
\begin{eqnarray}
lim_{t\rightarrow\infty}t^{-1}Z\left(t\right)\rightarrow\frac{\esp\left[Z\left(T_{1}\right)-Z\left(T_{0}\right)\right]}{\esp\left[T_{1}\right]},\textrm{ c.s. cuando  }t\rightarrow\infty.
\end{eqnarray}
\end{Coro}


%___________________________________________________________________________________________
%
%\subsection{Funci\'on de Renovaci\'on}
%___________________________________________________________________________________________
%


\begin{Def}
Sea $h\left(t\right)$ funci\'on de valores reales en $\rea$ acotada en intervalos finitos e igual a cero para $t<0$ La ecuaci\'on de renovaci\'on para $h\left(t\right)$ y la distribuci\'on $F$ es

\begin{eqnarray}%\label{Ec.Renovacion}
H\left(t\right)=h\left(t\right)+\int_{\left[0,t\right]}H\left(t-s\right)dF\left(s\right)\textrm{,    }t\geq0,
\end{eqnarray}
donde $H\left(t\right)$ es una funci\'on de valores reales. Esto es $H=h+F\star H$. Decimos que $H\left(t\right)$ es soluci\'on de esta ecuaci\'on si satisface la ecuaci\'on, y es acotada en intervalos finitos e iguales a cero para $t<0$.
\end{Def}

\begin{Prop}
La funci\'on $U\star h\left(t\right)$ es la \'unica soluci\'on de la ecuaci\'on de renovaci\'on (\ref{Ec.Renovacion}).
\end{Prop}

\begin{Teo}[Teorema Renovaci\'on Elemental]
\begin{eqnarray*}
t^{-1}U\left(t\right)\rightarrow 1/\mu\textrm{,    cuando }t\rightarrow\infty.
\end{eqnarray*}
\end{Teo}

%___________________________________________________________________________________________
%
%\subsection{Funci\'on de Renovaci\'on}
%___________________________________________________________________________________________
%


Sup\'ongase que $N\left(t\right)$ es un proceso de renovaci\'on con distribuci\'on $F$ con media finita $\mu$.

\begin{Def}
La funci\'on de renovaci\'on asociada con la distribuci\'on $F$, del proceso $N\left(t\right)$, es
\begin{eqnarray*}
U\left(t\right)=\sum_{n=1}^{\infty}F^{n\star}\left(t\right),\textrm{   }t\geq0,
\end{eqnarray*}
donde $F^{0\star}\left(t\right)=\indora\left(t\geq0\right)$.
\end{Def}


\begin{Prop}
Sup\'ongase que la distribuci\'on de inter-renovaci\'on $F$ tiene densidad $f$. Entonces $U\left(t\right)$ tambi\'en tiene densidad, para $t>0$, y es $U^{'}\left(t\right)=\sum_{n=0}^{\infty}f^{n\star}\left(t\right)$. Adem\'as
\begin{eqnarray*}
\prob\left\{N\left(t\right)>N\left(t-\right)\right\}=0\textrm{,   }t\geq0.
\end{eqnarray*}
\end{Prop}

\begin{Def}
La Transformada de Laplace-Stieljes de $F$ est\'a dada por

\begin{eqnarray*}
\hat{F}\left(\alpha\right)=\int_{\rea_{+}}e^{-\alpha t}dF\left(t\right)\textrm{,  }\alpha\geq0.
\end{eqnarray*}
\end{Def}

Entonces

\begin{eqnarray*}
\hat{U}\left(\alpha\right)=\sum_{n=0}^{\infty}\hat{F^{n\star}}\left(\alpha\right)=\sum_{n=0}^{\infty}\hat{F}\left(\alpha\right)^{n}=\frac{1}{1-\hat{F}\left(\alpha\right)}.
\end{eqnarray*}


\begin{Prop}
La Transformada de Laplace $\hat{U}\left(\alpha\right)$ y $\hat{F}\left(\alpha\right)$ determina una a la otra de manera \'unica por la relaci\'on $\hat{U}\left(\alpha\right)=\frac{1}{1-\hat{F}\left(\alpha\right)}$.
\end{Prop}


\begin{Note}
Un proceso de renovaci\'on $N\left(t\right)$ cuyos tiempos de inter-renovaci\'on tienen media finita, es un proceso Poisson con tasa $\lambda$ si y s\'olo s\'i $\esp\left[U\left(t\right)\right]=\lambda t$, para $t\geq0$.
\end{Note}


\begin{Teo}
Sea $N\left(t\right)$ un proceso puntual simple con puntos de localizaci\'on $T_{n}$ tal que $\eta\left(t\right)=\esp\left[N\left(\right)\right]$ es finita para cada $t$. Entonces para cualquier funci\'on $f:\rea_{+}\rightarrow\rea$,
\begin{eqnarray*}
\esp\left[\sum_{n=1}^{N\left(\right)}f\left(T_{n}\right)\right]=\int_{\left(0,t\right]}f\left(s\right)d\eta\left(s\right)\textrm{,  }t\geq0,
\end{eqnarray*}
suponiendo que la integral exista. Adem\'as si $X_{1},X_{2},\ldots$ son variables aleatorias definidas en el mismo espacio de probabilidad que el proceso $N\left(t\right)$ tal que $\esp\left[X_{n}|T_{n}=s\right]=f\left(s\right)$, independiente de $n$. Entonces
\begin{eqnarray*}
\esp\left[\sum_{n=1}^{N\left(t\right)}X_{n}\right]=\int_{\left(0,t\right]}f\left(s\right)d\eta\left(s\right)\textrm{,  }t\geq0,
\end{eqnarray*} 
suponiendo que la integral exista. 
\end{Teo}

\begin{Coro}[Identidad de Wald para Renovaciones]
Para el proceso de renovaci\'on $N\left(t\right)$,
\begin{eqnarray*}
\esp\left[T_{N\left(t\right)+1}\right]=\mu\esp\left[N\left(t\right)+1\right]\textrm{,  }t\geq0,
\end{eqnarray*}  
\end{Coro}

%______________________________________________________________________
%\subsection{Procesos de Renovaci\'on}
%______________________________________________________________________

\begin{Def}%\label{Def.Tn}
Sean $0\leq T_{1}\leq T_{2}\leq \ldots$ son tiempos aleatorios infinitos en los cuales ocurren ciertos eventos. El n\'umero de tiempos $T_{n}$ en el intervalo $\left[0,t\right)$ es

\begin{eqnarray}
N\left(t\right)=\sum_{n=1}^{\infty}\indora\left(T_{n}\leq t\right),
\end{eqnarray}
para $t\geq0$.
\end{Def}

Si se consideran los puntos $T_{n}$ como elementos de $\rea_{+}$, y $N\left(t\right)$ es el n\'umero de puntos en $\rea$. El proceso denotado por $\left\{N\left(t\right):t\geq0\right\}$, denotado por $N\left(t\right)$, es un proceso puntual en $\rea_{+}$. Los $T_{n}$ son los tiempos de ocurrencia, el proceso puntual $N\left(t\right)$ es simple si su n\'umero de ocurrencias son distintas: $0<T_{1}<T_{2}<\ldots$ casi seguramente.

\begin{Def}
Un proceso puntual $N\left(t\right)$ es un proceso de renovaci\'on si los tiempos de interocurrencia $\xi_{n}=T_{n}-T_{n-1}$, para $n\geq1$, son independientes e identicamente distribuidos con distribuci\'on $F$, donde $F\left(0\right)=0$ y $T_{0}=0$. Los $T_{n}$ son llamados tiempos de renovaci\'on, referente a la independencia o renovaci\'on de la informaci\'on estoc\'astica en estos tiempos. Los $\xi_{n}$ son los tiempos de inter-renovaci\'on, y $N\left(t\right)$ es el n\'umero de renovaciones en el intervalo $\left[0,t\right)$
\end{Def}


\begin{Note}
Para definir un proceso de renovaci\'on para cualquier contexto, solamente hay que especificar una distribuci\'on $F$, con $F\left(0\right)=0$, para los tiempos de inter-renovaci\'on. La funci\'on $F$ en turno degune las otra variables aleatorias. De manera formal, existe un espacio de probabilidad y una sucesi\'on de variables aleatorias $\xi_{1},\xi_{2},\ldots$ definidas en este con distribuci\'on $F$. Entonces las otras cantidades son $T_{n}=\sum_{k=1}^{n}\xi_{k}$ y $N\left(t\right)=\sum_{n=1}^{\infty}\indora\left(T_{n}\leq t\right)$, donde $T_{n}\rightarrow\infty$ casi seguramente por la Ley Fuerte de los Grandes Números.
\end{Note}

%___________________________________________________________________________________________
%
%\subsection{Renewal and Regenerative Processes: Serfozo\cite{Serfozo}}
%___________________________________________________________________________________________
%
\begin{Def}%\label{Def.Tn}
Sean $0\leq T_{1}\leq T_{2}\leq \ldots$ son tiempos aleatorios infinitos en los cuales ocurren ciertos eventos. El n\'umero de tiempos $T_{n}$ en el intervalo $\left[0,t\right)$ es

\begin{eqnarray}
N\left(t\right)=\sum_{n=1}^{\infty}\indora\left(T_{n}\leq t\right),
\end{eqnarray}
para $t\geq0$.
\end{Def}

Si se consideran los puntos $T_{n}$ como elementos de $\rea_{+}$, y $N\left(t\right)$ es el n\'umero de puntos en $\rea$. El proceso denotado por $\left\{N\left(t\right):t\geq0\right\}$, denotado por $N\left(t\right)$, es un proceso puntual en $\rea_{+}$. Los $T_{n}$ son los tiempos de ocurrencia, el proceso puntual $N\left(t\right)$ es simple si su n\'umero de ocurrencias son distintas: $0<T_{1}<T_{2}<\ldots$ casi seguramente.

\begin{Def}
Un proceso puntual $N\left(t\right)$ es un proceso de renovaci\'on si los tiempos de interocurrencia $\xi_{n}=T_{n}-T_{n-1}$, para $n\geq1$, son independientes e identicamente distribuidos con distribuci\'on $F$, donde $F\left(0\right)=0$ y $T_{0}=0$. Los $T_{n}$ son llamados tiempos de renovaci\'on, referente a la independencia o renovaci\'on de la informaci\'on estoc\'astica en estos tiempos. Los $\xi_{n}$ son los tiempos de inter-renovaci\'on, y $N\left(t\right)$ es el n\'umero de renovaciones en el intervalo $\left[0,t\right)$
\end{Def}


\begin{Note}
Para definir un proceso de renovaci\'on para cualquier contexto, solamente hay que especificar una distribuci\'on $F$, con $F\left(0\right)=0$, para los tiempos de inter-renovaci\'on. La funci\'on $F$ en turno degune las otra variables aleatorias. De manera formal, existe un espacio de probabilidad y una sucesi\'on de variables aleatorias $\xi_{1},\xi_{2},\ldots$ definidas en este con distribuci\'on $F$. Entonces las otras cantidades son $T_{n}=\sum_{k=1}^{n}\xi_{k}$ y $N\left(t\right)=\sum_{n=1}^{\infty}\indora\left(T_{n}\leq t\right)$, donde $T_{n}\rightarrow\infty$ casi seguramente por la Ley Fuerte de los Grandes N\'umeros.
\end{Note}







Los tiempos $T_{n}$ est\'an relacionados con los conteos de $N\left(t\right)$ por

\begin{eqnarray*}
\left\{N\left(t\right)\geq n\right\}&=&\left\{T_{n}\leq t\right\}\\
T_{N\left(t\right)}\leq &t&<T_{N\left(t\right)+1},
\end{eqnarray*}

adem\'as $N\left(T_{n}\right)=n$, y 

\begin{eqnarray*}
N\left(t\right)=\max\left\{n:T_{n}\leq t\right\}=\min\left\{n:T_{n+1}>t\right\}
\end{eqnarray*}

Por propiedades de la convoluci\'on se sabe que

\begin{eqnarray*}
P\left\{T_{n}\leq t\right\}=F^{n\star}\left(t\right)
\end{eqnarray*}
que es la $n$-\'esima convoluci\'on de $F$. Entonces 

\begin{eqnarray*}
\left\{N\left(t\right)\geq n\right\}&=&\left\{T_{n}\leq t\right\}\\
P\left\{N\left(t\right)\leq n\right\}&=&1-F^{\left(n+1\right)\star}\left(t\right)
\end{eqnarray*}

Adem\'as usando el hecho de que $\esp\left[N\left(t\right)\right]=\sum_{n=1}^{\infty}P\left\{N\left(t\right)\geq n\right\}$
se tiene que

\begin{eqnarray*}
\esp\left[N\left(t\right)\right]=\sum_{n=1}^{\infty}F^{n\star}\left(t\right)
\end{eqnarray*}

\begin{Prop}
Para cada $t\geq0$, la funci\'on generadora de momentos $\esp\left[e^{\alpha N\left(t\right)}\right]$ existe para alguna $\alpha$ en una vecindad del 0, y de aqu\'i que $\esp\left[N\left(t\right)^{m}\right]<\infty$, para $m\geq1$.
\end{Prop}

\begin{Ejem}[\textbf{Proceso Poisson}]

Suponga que se tienen tiempos de inter-renovaci\'on \textit{i.i.d.} del proceso de renovaci\'on $N\left(t\right)$ tienen distribuci\'on exponencial $F\left(t\right)=q-e^{-\lambda t}$ con tasa $\lambda$. Entonces $N\left(t\right)$ es un proceso Poisson con tasa $\lambda$.

\end{Ejem}


\begin{Note}
Si el primer tiempo de renovaci\'on $\xi_{1}$ no tiene la misma distribuci\'on que el resto de las $\xi_{n}$, para $n\geq2$, a $N\left(t\right)$ se le llama Proceso de Renovaci\'on retardado, donde si $\xi$ tiene distribuci\'on $G$, entonces el tiempo $T_{n}$ de la $n$-\'esima renovaci\'on tiene distribuci\'on $G\star F^{\left(n-1\right)\star}\left(t\right)$
\end{Note}


\begin{Teo}
Para una constante $\mu\leq\infty$ ( o variable aleatoria), las siguientes expresiones son equivalentes:

\begin{eqnarray}
lim_{n\rightarrow\infty}n^{-1}T_{n}&=&\mu,\textrm{ c.s.}\\
lim_{t\rightarrow\infty}t^{-1}N\left(t\right)&=&1/\mu,\textrm{ c.s.}
\end{eqnarray}
\end{Teo}


Es decir, $T_{n}$ satisface la Ley Fuerte de los Grandes N\'umeros s\'i y s\'olo s\'i $N\left/t\right)$ la cumple.


\begin{Coro}[Ley Fuerte de los Grandes N\'umeros para Procesos de Renovaci\'on]
Si $N\left(t\right)$ es un proceso de renovaci\'on cuyos tiempos de inter-renovaci\'on tienen media $\mu\leq\infty$, entonces
\begin{eqnarray}
t^{-1}N\left(t\right)\rightarrow 1/\mu,\textrm{ c.s. cuando }t\rightarrow\infty.
\end{eqnarray}

\end{Coro}


Considerar el proceso estoc\'astico de valores reales $\left\{Z\left(t\right):t\geq0\right\}$ en el mismo espacio de probabilidad que $N\left(t\right)$

\begin{Def}
Para el proceso $\left\{Z\left(t\right):t\geq0\right\}$ se define la fluctuaci\'on m\'axima de $Z\left(t\right)$ en el intervalo $\left(T_{n-1},T_{n}\right]$:
\begin{eqnarray*}
M_{n}=\sup_{T_{n-1}<t\leq T_{n}}|Z\left(t\right)-Z\left(T_{n-1}\right)|
\end{eqnarray*}
\end{Def}

\begin{Teo}
Sup\'ongase que $n^{-1}T_{n}\rightarrow\mu$ c.s. cuando $n\rightarrow\infty$, donde $\mu\leq\infty$ es una constante o variable aleatoria. Sea $a$ una constante o variable aleatoria que puede ser infinita cuando $\mu$ es finita, y considere las expresiones l\'imite:
\begin{eqnarray}
lim_{n\rightarrow\infty}n^{-1}Z\left(T_{n}\right)&=&a,\textrm{ c.s.}\\
lim_{t\rightarrow\infty}t^{-1}Z\left(t\right)&=&a/\mu,\textrm{ c.s.}
\end{eqnarray}
La segunda expresi\'on implica la primera. Conversamente, la primera implica la segunda si el proceso $Z\left(t\right)$ es creciente, o si $lim_{n\rightarrow\infty}n^{-1}M_{n}=0$ c.s.
\end{Teo}

\begin{Coro}
Si $N\left(t\right)$ es un proceso de renovaci\'on, y $\left(Z\left(T_{n}\right)-Z\left(T_{n-1}\right),M_{n}\right)$, para $n\geq1$, son variables aleatorias independientes e id\'enticamente distribuidas con media finita, entonces,
\begin{eqnarray}
lim_{t\rightarrow\infty}t^{-1}Z\left(t\right)\rightarrow\frac{\esp\left[Z\left(T_{1}\right)-Z\left(T_{0}\right)\right]}{\esp\left[T_{1}\right]},\textrm{ c.s. cuando  }t\rightarrow\infty.
\end{eqnarray}
\end{Coro}


Sup\'ongase que $N\left(t\right)$ es un proceso de renovaci\'on con distribuci\'on $F$ con media finita $\mu$.

\begin{Def}
La funci\'on de renovaci\'on asociada con la distribuci\'on $F$, del proceso $N\left(t\right)$, es
\begin{eqnarray*}
U\left(t\right)=\sum_{n=1}^{\infty}F^{n\star}\left(t\right),\textrm{   }t\geq0,
\end{eqnarray*}
donde $F^{0\star}\left(t\right)=\indora\left(t\geq0\right)$.
\end{Def}


\begin{Prop}
Sup\'ongase que la distribuci\'on de inter-renovaci\'on $F$ tiene densidad $f$. Entonces $U\left(t\right)$ tambi\'en tiene densidad, para $t>0$, y es $U^{'}\left(t\right)=\sum_{n=0}^{\infty}f^{n\star}\left(t\right)$. Adem\'as
\begin{eqnarray*}
\prob\left\{N\left(t\right)>N\left(t-\right)\right\}=0\textrm{,   }t\geq0.
\end{eqnarray*}
\end{Prop}

\begin{Def}
La Transformada de Laplace-Stieljes de $F$ est\'a dada por

\begin{eqnarray*}
\hat{F}\left(\alpha\right)=\int_{\rea_{+}}e^{-\alpha t}dF\left(t\right)\textrm{,  }\alpha\geq0.
\end{eqnarray*}
\end{Def}

Entonces

\begin{eqnarray*}
\hat{U}\left(\alpha\right)=\sum_{n=0}^{\infty}\hat{F^{n\star}}\left(\alpha\right)=\sum_{n=0}^{\infty}\hat{F}\left(\alpha\right)^{n}=\frac{1}{1-\hat{F}\left(\alpha\right)}.
\end{eqnarray*}


\begin{Prop}
La Transformada de Laplace $\hat{U}\left(\alpha\right)$ y $\hat{F}\left(\alpha\right)$ determina una a la otra de manera \'unica por la relaci\'on $\hat{U}\left(\alpha\right)=\frac{1}{1-\hat{F}\left(\alpha\right)}$.
\end{Prop}


\begin{Note}
Un proceso de renovaci\'on $N\left(t\right)$ cuyos tiempos de inter-renovaci\'on tienen media finita, es un proceso Poisson con tasa $\lambda$ si y s\'olo s\'i $\esp\left[U\left(t\right)\right]=\lambda t$, para $t\geq0$.
\end{Note}


\begin{Teo}
Sea $N\left(t\right)$ un proceso puntual simple con puntos de localizaci\'on $T_{n}$ tal que $\eta\left(t\right)=\esp\left[N\left(\right)\right]$ es finita para cada $t$. Entonces para cualquier funci\'on $f:\rea_{+}\rightarrow\rea$,
\begin{eqnarray*}
\esp\left[\sum_{n=1}^{N\left(\right)}f\left(T_{n}\right)\right]=\int_{\left(0,t\right]}f\left(s\right)d\eta\left(s\right)\textrm{,  }t\geq0,
\end{eqnarray*}
suponiendo que la integral exista. Adem\'as si $X_{1},X_{2},\ldots$ son variables aleatorias definidas en el mismo espacio de probabilidad que el proceso $N\left(t\right)$ tal que $\esp\left[X_{n}|T_{n}=s\right]=f\left(s\right)$, independiente de $n$. Entonces
\begin{eqnarray*}
\esp\left[\sum_{n=1}^{N\left(t\right)}X_{n}\right]=\int_{\left(0,t\right]}f\left(s\right)d\eta\left(s\right)\textrm{,  }t\geq0,
\end{eqnarray*} 
suponiendo que la integral exista. 
\end{Teo}

\begin{Coro}[Identidad de Wald para Renovaciones]
Para el proceso de renovaci\'on $N\left(t\right)$,
\begin{eqnarray*}
\esp\left[T_{N\left(t\right)+1}\right]=\mu\esp\left[N\left(t\right)+1\right]\textrm{,  }t\geq0,
\end{eqnarray*}  
\end{Coro}


\begin{Def}
Sea $h\left(t\right)$ funci\'on de valores reales en $\rea$ acotada en intervalos finitos e igual a cero para $t<0$ La ecuaci\'on de renovaci\'on para $h\left(t\right)$ y la distribuci\'on $F$ es

\begin{eqnarray}%\label{Ec.Renovacion}
H\left(t\right)=h\left(t\right)+\int_{\left[0,t\right]}H\left(t-s\right)dF\left(s\right)\textrm{,    }t\geq0,
\end{eqnarray}
donde $H\left(t\right)$ es una funci\'on de valores reales. Esto es $H=h+F\star H$. Decimos que $H\left(t\right)$ es soluci\'on de esta ecuaci\'on si satisface la ecuaci\'on, y es acotada en intervalos finitos e iguales a cero para $t<0$.
\end{Def}

\begin{Prop}
La funci\'on $U\star h\left(t\right)$ es la \'unica soluci\'on de la ecuaci\'on de renovaci\'on (\ref{Ec.Renovacion}).
\end{Prop}

\begin{Teo}[Teorema Renovaci\'on Elemental]
\begin{eqnarray*}
t^{-1}U\left(t\right)\rightarrow 1/\mu\textrm{,    cuando }t\rightarrow\infty.
\end{eqnarray*}
\end{Teo}



Sup\'ongase que $N\left(t\right)$ es un proceso de renovaci\'on con distribuci\'on $F$ con media finita $\mu$.

\begin{Def}
La funci\'on de renovaci\'on asociada con la distribuci\'on $F$, del proceso $N\left(t\right)$, es
\begin{eqnarray*}
U\left(t\right)=\sum_{n=1}^{\infty}F^{n\star}\left(t\right),\textrm{   }t\geq0,
\end{eqnarray*}
donde $F^{0\star}\left(t\right)=\indora\left(t\geq0\right)$.
\end{Def}


\begin{Prop}
Sup\'ongase que la distribuci\'on de inter-renovaci\'on $F$ tiene densidad $f$. Entonces $U\left(t\right)$ tambi\'en tiene densidad, para $t>0$, y es $U^{'}\left(t\right)=\sum_{n=0}^{\infty}f^{n\star}\left(t\right)$. Adem\'as
\begin{eqnarray*}
\prob\left\{N\left(t\right)>N\left(t-\right)\right\}=0\textrm{,   }t\geq0.
\end{eqnarray*}
\end{Prop}

\begin{Def}
La Transformada de Laplace-Stieljes de $F$ est\'a dada por

\begin{eqnarray*}
\hat{F}\left(\alpha\right)=\int_{\rea_{+}}e^{-\alpha t}dF\left(t\right)\textrm{,  }\alpha\geq0.
\end{eqnarray*}
\end{Def}

Entonces

\begin{eqnarray*}
\hat{U}\left(\alpha\right)=\sum_{n=0}^{\infty}\hat{F^{n\star}}\left(\alpha\right)=\sum_{n=0}^{\infty}\hat{F}\left(\alpha\right)^{n}=\frac{1}{1-\hat{F}\left(\alpha\right)}.
\end{eqnarray*}


\begin{Prop}
La Transformada de Laplace $\hat{U}\left(\alpha\right)$ y $\hat{F}\left(\alpha\right)$ determina una a la otra de manera \'unica por la relaci\'on $\hat{U}\left(\alpha\right)=\frac{1}{1-\hat{F}\left(\alpha\right)}$.
\end{Prop}


\begin{Note}
Un proceso de renovaci\'on $N\left(t\right)$ cuyos tiempos de inter-renovaci\'on tienen media finita, es un proceso Poisson con tasa $\lambda$ si y s\'olo s\'i $\esp\left[U\left(t\right)\right]=\lambda t$, para $t\geq0$.
\end{Note}


\begin{Teo}
Sea $N\left(t\right)$ un proceso puntual simple con puntos de localizaci\'on $T_{n}$ tal que $\eta\left(t\right)=\esp\left[N\left(\right)\right]$ es finita para cada $t$. Entonces para cualquier funci\'on $f:\rea_{+}\rightarrow\rea$,
\begin{eqnarray*}
\esp\left[\sum_{n=1}^{N\left(\right)}f\left(T_{n}\right)\right]=\int_{\left(0,t\right]}f\left(s\right)d\eta\left(s\right)\textrm{,  }t\geq0,
\end{eqnarray*}
suponiendo que la integral exista. Adem\'as si $X_{1},X_{2},\ldots$ son variables aleatorias definidas en el mismo espacio de probabilidad que el proceso $N\left(t\right)$ tal que $\esp\left[X_{n}|T_{n}=s\right]=f\left(s\right)$, independiente de $n$. Entonces
\begin{eqnarray*}
\esp\left[\sum_{n=1}^{N\left(t\right)}X_{n}\right]=\int_{\left(0,t\right]}f\left(s\right)d\eta\left(s\right)\textrm{,  }t\geq0,
\end{eqnarray*} 
suponiendo que la integral exista. 
\end{Teo}

\begin{Coro}[Identidad de Wald para Renovaciones]
Para el proceso de renovaci\'on $N\left(t\right)$,
\begin{eqnarray*}
\esp\left[T_{N\left(t\right)+1}\right]=\mu\esp\left[N\left(t\right)+1\right]\textrm{,  }t\geq0,
\end{eqnarray*}  
\end{Coro}


\begin{Def}
Sea $h\left(t\right)$ funci\'on de valores reales en $\rea$ acotada en intervalos finitos e igual a cero para $t<0$ La ecuaci\'on de renovaci\'on para $h\left(t\right)$ y la distribuci\'on $F$ es

\begin{eqnarray}%\label{Ec.Renovacion}
H\left(t\right)=h\left(t\right)+\int_{\left[0,t\right]}H\left(t-s\right)dF\left(s\right)\textrm{,    }t\geq0,
\end{eqnarray}
donde $H\left(t\right)$ es una funci\'on de valores reales. Esto es $H=h+F\star H$. Decimos que $H\left(t\right)$ es soluci\'on de esta ecuaci\'on si satisface la ecuaci\'on, y es acotada en intervalos finitos e iguales a cero para $t<0$.
\end{Def}

\begin{Prop}
La funci\'on $U\star h\left(t\right)$ es la \'unica soluci\'on de la ecuaci\'on de renovaci\'on (\ref{Ec.Renovacion}).
\end{Prop}

\begin{Teo}[Teorema Renovaci\'on Elemental]
\begin{eqnarray*}
t^{-1}U\left(t\right)\rightarrow 1/\mu\textrm{,    cuando }t\rightarrow\infty.
\end{eqnarray*}
\end{Teo}


\begin{Note} Una funci\'on $h:\rea_{+}\rightarrow\rea$ es Directamente Riemann Integrable en los siguientes casos:
\begin{itemize}
\item[a)] $h\left(t\right)\geq0$ es decreciente y Riemann Integrable.
\item[b)] $h$ es continua excepto posiblemente en un conjunto de Lebesgue de medida 0, y $|h\left(t\right)|\leq b\left(t\right)$, donde $b$ es DRI.
\end{itemize}
\end{Note}

\begin{Teo}[Teorema Principal de Renovaci\'on]
Si $F$ es no aritm\'etica y $h\left(t\right)$ es Directamente Riemann Integrable (DRI), entonces

\begin{eqnarray*}
lim_{t\rightarrow\infty}U\star h=\frac{1}{\mu}\int_{\rea_{+}}h\left(s\right)ds.
\end{eqnarray*}
\end{Teo}

\begin{Prop}
Cualquier funci\'on $H\left(t\right)$ acotada en intervalos finitos y que es 0 para $t<0$ puede expresarse como
\begin{eqnarray*}
H\left(t\right)=U\star h\left(t\right)\textrm{,  donde }h\left(t\right)=H\left(t\right)-F\star H\left(t\right)
\end{eqnarray*}
\end{Prop}

\begin{Def}
Un proceso estoc\'astico $X\left(t\right)$ es crudamente regenerativo en un tiempo aleatorio positivo $T$ si
\begin{eqnarray*}
\esp\left[X\left(T+t\right)|T\right]=\esp\left[X\left(t\right)\right]\textrm{, para }t\geq0,\end{eqnarray*}
y con las esperanzas anteriores finitas.
\end{Def}

\begin{Prop}
Sup\'ongase que $X\left(t\right)$ es un proceso crudamente regenerativo en $T$, que tiene distribuci\'on $F$. Si $\esp\left[X\left(t\right)\right]$ es acotado en intervalos finitos, entonces
\begin{eqnarray*}
\esp\left[X\left(t\right)\right]=U\star h\left(t\right)\textrm{,  donde }h\left(t\right)=\esp\left[X\left(t\right)\indora\left(T>t\right)\right].
\end{eqnarray*}
\end{Prop}

\begin{Teo}[Regeneraci\'on Cruda]
Sup\'ongase que $X\left(t\right)$ es un proceso con valores positivo crudamente regenerativo en $T$, y def\'inase $M=\sup\left\{|X\left(t\right)|:t\leq T\right\}$. Si $T$ es no aritm\'etico y $M$ y $MT$ tienen media finita, entonces
\begin{eqnarray*}
lim_{t\rightarrow\infty}\esp\left[X\left(t\right)\right]=\frac{1}{\mu}\int_{\rea_{+}}h\left(s\right)ds,
\end{eqnarray*}
donde $h\left(t\right)=\esp\left[X\left(t\right)\indora\left(T>t\right)\right]$.
\end{Teo}


\begin{Note} Una funci\'on $h:\rea_{+}\rightarrow\rea$ es Directamente Riemann Integrable en los siguientes casos:
\begin{itemize}
\item[a)] $h\left(t\right)\geq0$ es decreciente y Riemann Integrable.
\item[b)] $h$ es continua excepto posiblemente en un conjunto de Lebesgue de medida 0, y $|h\left(t\right)|\leq b\left(t\right)$, donde $b$ es DRI.
\end{itemize}
\end{Note}

\begin{Teo}[Teorema Principal de Renovaci\'on]
Si $F$ es no aritm\'etica y $h\left(t\right)$ es Directamente Riemann Integrable (DRI), entonces

\begin{eqnarray*}
lim_{t\rightarrow\infty}U\star h=\frac{1}{\mu}\int_{\rea_{+}}h\left(s\right)ds.
\end{eqnarray*}
\end{Teo}

\begin{Prop}
Cualquier funci\'on $H\left(t\right)$ acotada en intervalos finitos y que es 0 para $t<0$ puede expresarse como
\begin{eqnarray*}
H\left(t\right)=U\star h\left(t\right)\textrm{,  donde }h\left(t\right)=H\left(t\right)-F\star H\left(t\right)
\end{eqnarray*}
\end{Prop}

\begin{Def}
Un proceso estoc\'astico $X\left(t\right)$ es crudamente regenerativo en un tiempo aleatorio positivo $T$ si
\begin{eqnarray*}
\esp\left[X\left(T+t\right)|T\right]=\esp\left[X\left(t\right)\right]\textrm{, para }t\geq0,\end{eqnarray*}
y con las esperanzas anteriores finitas.
\end{Def}

\begin{Prop}
Sup\'ongase que $X\left(t\right)$ es un proceso crudamente regenerativo en $T$, que tiene distribuci\'on $F$. Si $\esp\left[X\left(t\right)\right]$ es acotado en intervalos finitos, entonces
\begin{eqnarray*}
\esp\left[X\left(t\right)\right]=U\star h\left(t\right)\textrm{,  donde }h\left(t\right)=\esp\left[X\left(t\right)\indora\left(T>t\right)\right].
\end{eqnarray*}
\end{Prop}

\begin{Teo}[Regeneraci\'on Cruda]
Sup\'ongase que $X\left(t\right)$ es un proceso con valores positivo crudamente regenerativo en $T$, y def\'inase $M=\sup\left\{|X\left(t\right)|:t\leq T\right\}$. Si $T$ es no aritm\'etico y $M$ y $MT$ tienen media finita, entonces
\begin{eqnarray*}
lim_{t\rightarrow\infty}\esp\left[X\left(t\right)\right]=\frac{1}{\mu}\int_{\rea_{+}}h\left(s\right)ds,
\end{eqnarray*}
donde $h\left(t\right)=\esp\left[X\left(t\right)\indora\left(T>t\right)\right]$.
\end{Teo}

\begin{Def}
Para el proceso $\left\{\left(N\left(t\right),X\left(t\right)\right):t\geq0\right\}$, sus trayectoria muestrales en el intervalo de tiempo $\left[T_{n-1},T_{n}\right)$ est\'an descritas por
\begin{eqnarray*}
\zeta_{n}=\left(\xi_{n},\left\{X\left(T_{n-1}+t\right):0\leq t<\xi_{n}\right\}\right)
\end{eqnarray*}
Este $\zeta_{n}$ es el $n$-\'esimo segmento del proceso. El proceso es regenerativo sobre los tiempos $T_{n}$ si sus segmentos $\zeta_{n}$ son independientes e id\'enticamennte distribuidos.
\end{Def}


\begin{Note}
Si $\tilde{X}\left(t\right)$ con espacio de estados $\tilde{S}$ es regenerativo sobre $T_{n}$, entonces $X\left(t\right)=f\left(\tilde{X}\left(t\right)\right)$ tambi\'en es regenerativo sobre $T_{n}$, para cualquier funci\'on $f:\tilde{S}\rightarrow S$.
\end{Note}

\begin{Note}
Los procesos regenerativos son crudamente regenerativos, pero no al rev\'es.
\end{Note}


\begin{Note}
Un proceso estoc\'astico a tiempo continuo o discreto es regenerativo si existe un proceso de renovaci\'on  tal que los segmentos del proceso entre tiempos de renovaci\'on sucesivos son i.i.d., es decir, para $\left\{X\left(t\right):t\geq0\right\}$ proceso estoc\'astico a tiempo continuo con espacio de estados $S$, espacio m\'etrico.
\end{Note}

Para $\left\{X\left(t\right):t\geq0\right\}$ Proceso Estoc\'astico a tiempo continuo con estado de espacios $S$, que es un espacio m\'etrico, con trayectorias continuas por la derecha y con l\'imites por la izquierda c.s. Sea $N\left(t\right)$ un proceso de renovaci\'on en $\rea_{+}$ definido en el mismo espacio de probabilidad que $X\left(t\right)$, con tiempos de renovaci\'on $T$ y tiempos de inter-renovaci\'on $\xi_{n}=T_{n}-T_{n-1}$, con misma distribuci\'on $F$ de media finita $\mu$.



\begin{Def}
Para el proceso $\left\{\left(N\left(t\right),X\left(t\right)\right):t\geq0\right\}$, sus trayectoria muestrales en el intervalo de tiempo $\left[T_{n-1},T_{n}\right)$ est\'an descritas por
\begin{eqnarray*}
\zeta_{n}=\left(\xi_{n},\left\{X\left(T_{n-1}+t\right):0\leq t<\xi_{n}\right\}\right)
\end{eqnarray*}
Este $\zeta_{n}$ es el $n$-\'esimo segmento del proceso. El proceso es regenerativo sobre los tiempos $T_{n}$ si sus segmentos $\zeta_{n}$ son independientes e id\'enticamennte distribuidos.
\end{Def}

\begin{Note}
Un proceso regenerativo con media de la longitud de ciclo finita es llamado positivo recurrente.
\end{Note}

\begin{Teo}[Procesos Regenerativos]
Suponga que el proceso
\end{Teo}


\begin{Def}[Renewal Process Trinity]
Para un proceso de renovaci\'on $N\left(t\right)$, los siguientes procesos proveen de informaci\'on sobre los tiempos de renovaci\'on.
\begin{itemize}
\item $A\left(t\right)=t-T_{N\left(t\right)}$, el tiempo de recurrencia hacia atr\'as al tiempo $t$, que es el tiempo desde la \'ultima renovaci\'on para $t$.

\item $B\left(t\right)=T_{N\left(t\right)+1}-t$, el tiempo de recurrencia hacia adelante al tiempo $t$, residual del tiempo de renovaci\'on, que es el tiempo para la pr\'oxima renovaci\'on despu\'es de $t$.

\item $L\left(t\right)=\xi_{N\left(t\right)+1}=A\left(t\right)+B\left(t\right)$, la longitud del intervalo de renovaci\'on que contiene a $t$.
\end{itemize}
\end{Def}

\begin{Note}
El proceso tridimensional $\left(A\left(t\right),B\left(t\right),L\left(t\right)\right)$ es regenerativo sobre $T_{n}$, y por ende cada proceso lo es. Cada proceso $A\left(t\right)$ y $B\left(t\right)$ son procesos de MArkov a tiempo continuo con trayectorias continuas por partes en el espacio de estados $\rea_{+}$. Una expresi\'on conveniente para su distribuci\'on conjunta es, para $0\leq x<t,y\geq0$
\begin{equation}\label{NoRenovacion}
P\left\{A\left(t\right)>x,B\left(t\right)>y\right\}=
P\left\{N\left(t+y\right)-N\left((t-x)\right)=0\right\}
\end{equation}
\end{Note}

\begin{Ejem}[Tiempos de recurrencia Poisson]
Si $N\left(t\right)$ es un proceso Poisson con tasa $\lambda$, entonces de la expresi\'on (\ref{NoRenovacion}) se tiene que

\begin{eqnarray*}
\begin{array}{lc}
P\left\{A\left(t\right)>x,B\left(t\right)>y\right\}=e^{-\lambda\left(x+y\right)},&0\leq x<t,y\geq0,
\end{array}
\end{eqnarray*}
que es la probabilidad Poisson de no renovaciones en un intervalo de longitud $x+y$.

\end{Ejem}

\begin{Note}
Una cadena de Markov erg\'odica tiene la propiedad de ser estacionaria si la distribuci\'on de su estado al tiempo $0$ es su distribuci\'on estacionaria.
\end{Note}


\begin{Def}
Un proceso estoc\'astico a tiempo continuo $\left\{X\left(t\right):t\geq0\right\}$ en un espacio general es estacionario si sus distribuciones finito dimensionales son invariantes bajo cualquier  traslado: para cada $0\leq s_{1}<s_{2}<\cdots<s_{k}$ y $t\geq0$,
\begin{eqnarray*}
\left(X\left(s_{1}+t\right),\ldots,X\left(s_{k}+t\right)\right)=_{d}\left(X\left(s_{1}\right),\ldots,X\left(s_{k}\right)\right).
\end{eqnarray*}
\end{Def}

\begin{Note}
Un proceso de Markov es estacionario si $X\left(t\right)=_{d}X\left(0\right)$, $t\geq0$.
\end{Note}

Considerese el proceso $N\left(t\right)=\sum_{n}\indora\left(\tau_{n}\leq t\right)$ en $\rea_{+}$, con puntos $0<\tau_{1}<\tau_{2}<\cdots$.

\begin{Prop}
Si $N$ es un proceso puntual estacionario y $\esp\left[N\left(1\right)\right]<\infty$, entonces $\esp\left[N\left(t\right)\right]=t\esp\left[N\left(1\right)\right]$, $t\geq0$

\end{Prop}

\begin{Teo}
Los siguientes enunciados son equivalentes
\begin{itemize}
\item[i)] El proceso retardado de renovaci\'on $N$ es estacionario.

\item[ii)] EL proceso de tiempos de recurrencia hacia adelante $B\left(t\right)$ es estacionario.


\item[iii)] $\esp\left[N\left(t\right)\right]=t/\mu$,


\item[iv)] $G\left(t\right)=F_{e}\left(t\right)=\frac{1}{\mu}\int_{0}^{t}\left[1-F\left(s\right)\right]ds$
\end{itemize}
Cuando estos enunciados son ciertos, $P\left\{B\left(t\right)\leq x\right\}=F_{e}\left(x\right)$, para $t,x\geq0$.

\end{Teo}

\begin{Note}
Una consecuencia del teorema anterior es que el Proceso Poisson es el \'unico proceso sin retardo que es estacionario.
\end{Note}

\begin{Coro}
El proceso de renovaci\'on $N\left(t\right)$ sin retardo, y cuyos tiempos de inter renonaci\'on tienen media finita, es estacionario si y s\'olo si es un proceso Poisson.

\end{Coro}


%________________________________________________________________________
%\subsection{Procesos Regenerativos}
%________________________________________________________________________



\begin{Note}
Si $\tilde{X}\left(t\right)$ con espacio de estados $\tilde{S}$ es regenerativo sobre $T_{n}$, entonces $X\left(t\right)=f\left(\tilde{X}\left(t\right)\right)$ tambi\'en es regenerativo sobre $T_{n}$, para cualquier funci\'on $f:\tilde{S}\rightarrow S$.
\end{Note}

\begin{Note}
Los procesos regenerativos son crudamente regenerativos, pero no al rev\'es.
\end{Note}
%\subsection*{Procesos Regenerativos: Sigman\cite{Sigman1}}
\begin{Def}[Definici\'on Cl\'asica]
Un proceso estoc\'astico $X=\left\{X\left(t\right):t\geq0\right\}$ es llamado regenerativo is existe una variable aleatoria $R_{1}>0$ tal que
\begin{itemize}
\item[i)] $\left\{X\left(t+R_{1}\right):t\geq0\right\}$ es independiente de $\left\{\left\{X\left(t\right):t<R_{1}\right\},\right\}$
\item[ii)] $\left\{X\left(t+R_{1}\right):t\geq0\right\}$ es estoc\'asticamente equivalente a $\left\{X\left(t\right):t>0\right\}$
\end{itemize}

Llamamos a $R_{1}$ tiempo de regeneraci\'on, y decimos que $X$ se regenera en este punto.
\end{Def}

$\left\{X\left(t+R_{1}\right)\right\}$ es regenerativo con tiempo de regeneraci\'on $R_{2}$, independiente de $R_{1}$ pero con la misma distribuci\'on que $R_{1}$. Procediendo de esta manera se obtiene una secuencia de variables aleatorias independientes e id\'enticamente distribuidas $\left\{R_{n}\right\}$ llamados longitudes de ciclo. Si definimos a $Z_{k}\equiv R_{1}+R_{2}+\cdots+R_{k}$, se tiene un proceso de renovaci\'on llamado proceso de renovaci\'on encajado para $X$.




\begin{Def}
Para $x$ fijo y para cada $t\geq0$, sea $I_{x}\left(t\right)=1$ si $X\left(t\right)\leq x$,  $I_{x}\left(t\right)=0$ en caso contrario, y def\'inanse los tiempos promedio
\begin{eqnarray*}
\overline{X}&=&lim_{t\rightarrow\infty}\frac{1}{t}\int_{0}^{\infty}X\left(u\right)du\\
\prob\left(X_{\infty}\leq x\right)&=&lim_{t\rightarrow\infty}\frac{1}{t}\int_{0}^{\infty}I_{x}\left(u\right)du,
\end{eqnarray*}
cuando estos l\'imites existan.
\end{Def}

Como consecuencia del teorema de Renovaci\'on-Recompensa, se tiene que el primer l\'imite  existe y es igual a la constante
\begin{eqnarray*}
\overline{X}&=&\frac{\esp\left[\int_{0}^{R_{1}}X\left(t\right)dt\right]}{\esp\left[R_{1}\right]},
\end{eqnarray*}
suponiendo que ambas esperanzas son finitas.

\begin{Note}
\begin{itemize}
\item[a)] Si el proceso regenerativo $X$ es positivo recurrente y tiene trayectorias muestrales no negativas, entonces la ecuaci\'on anterior es v\'alida.
\item[b)] Si $X$ es positivo recurrente regenerativo, podemos construir una \'unica versi\'on estacionaria de este proceso, $X_{e}=\left\{X_{e}\left(t\right)\right\}$, donde $X_{e}$ es un proceso estoc\'astico regenerativo y estrictamente estacionario, con distribuci\'on marginal distribuida como $X_{\infty}$
\end{itemize}
\end{Note}

%________________________________________________________________________
%\subsection{Procesos Regenerativos}
%________________________________________________________________________

Para $\left\{X\left(t\right):t\geq0\right\}$ Proceso Estoc\'astico a tiempo continuo con estado de espacios $S$, que es un espacio m\'etrico, con trayectorias continuas por la derecha y con l\'imites por la izquierda c.s. Sea $N\left(t\right)$ un proceso de renovaci\'on en $\rea_{+}$ definido en el mismo espacio de probabilidad que $X\left(t\right)$, con tiempos de renovaci\'on $T$ y tiempos de inter-renovaci\'on $\xi_{n}=T_{n}-T_{n-1}$, con misma distribuci\'on $F$ de media finita $\mu$.



\begin{Def}
Para el proceso $\left\{\left(N\left(t\right),X\left(t\right)\right):t\geq0\right\}$, sus trayectoria muestrales en el intervalo de tiempo $\left[T_{n-1},T_{n}\right)$ est\'an descritas por
\begin{eqnarray*}
\zeta_{n}=\left(\xi_{n},\left\{X\left(T_{n-1}+t\right):0\leq t<\xi_{n}\right\}\right)
\end{eqnarray*}
Este $\zeta_{n}$ es el $n$-\'esimo segmento del proceso. El proceso es regenerativo sobre los tiempos $T_{n}$ si sus segmentos $\zeta_{n}$ son independientes e id\'enticamennte distribuidos.
\end{Def}


\begin{Note}
Si $\tilde{X}\left(t\right)$ con espacio de estados $\tilde{S}$ es regenerativo sobre $T_{n}$, entonces $X\left(t\right)=f\left(\tilde{X}\left(t\right)\right)$ tambi\'en es regenerativo sobre $T_{n}$, para cualquier funci\'on $f:\tilde{S}\rightarrow S$.
\end{Note}

\begin{Note}
Los procesos regenerativos son crudamente regenerativos, pero no al rev\'es.
\end{Note}

\begin{Def}[Definici\'on Cl\'asica]
Un proceso estoc\'astico $X=\left\{X\left(t\right):t\geq0\right\}$ es llamado regenerativo is existe una variable aleatoria $R_{1}>0$ tal que
\begin{itemize}
\item[i)] $\left\{X\left(t+R_{1}\right):t\geq0\right\}$ es independiente de $\left\{\left\{X\left(t\right):t<R_{1}\right\},\right\}$
\item[ii)] $\left\{X\left(t+R_{1}\right):t\geq0\right\}$ es estoc\'asticamente equivalente a $\left\{X\left(t\right):t>0\right\}$
\end{itemize}

Llamamos a $R_{1}$ tiempo de regeneraci\'on, y decimos que $X$ se regenera en este punto.
\end{Def}

$\left\{X\left(t+R_{1}\right)\right\}$ es regenerativo con tiempo de regeneraci\'on $R_{2}$, independiente de $R_{1}$ pero con la misma distribuci\'on que $R_{1}$. Procediendo de esta manera se obtiene una secuencia de variables aleatorias independientes e id\'enticamente distribuidas $\left\{R_{n}\right\}$ llamados longitudes de ciclo. Si definimos a $Z_{k}\equiv R_{1}+R_{2}+\cdots+R_{k}$, se tiene un proceso de renovaci\'on llamado proceso de renovaci\'on encajado para $X$.

\begin{Note}
Un proceso regenerativo con media de la longitud de ciclo finita es llamado positivo recurrente.
\end{Note}


\begin{Def}
Para $x$ fijo y para cada $t\geq0$, sea $I_{x}\left(t\right)=1$ si $X\left(t\right)\leq x$,  $I_{x}\left(t\right)=0$ en caso contrario, y def\'inanse los tiempos promedio
\begin{eqnarray*}
\overline{X}&=&lim_{t\rightarrow\infty}\frac{1}{t}\int_{0}^{\infty}X\left(u\right)du\\
\prob\left(X_{\infty}\leq x\right)&=&lim_{t\rightarrow\infty}\frac{1}{t}\int_{0}^{\infty}I_{x}\left(u\right)du,
\end{eqnarray*}
cuando estos l\'imites existan.
\end{Def}

Como consecuencia del teorema de Renovaci\'on-Recompensa, se tiene que el primer l\'imite  existe y es igual a la constante
\begin{eqnarray*}
\overline{X}&=&\frac{\esp\left[\int_{0}^{R_{1}}X\left(t\right)dt\right]}{\esp\left[R_{1}\right]},
\end{eqnarray*}
suponiendo que ambas esperanzas son finitas.

\begin{Note}
\begin{itemize}
\item[a)] Si el proceso regenerativo $X$ es positivo recurrente y tiene trayectorias muestrales no negativas, entonces la ecuaci\'on anterior es v\'alida.
\item[b)] Si $X$ es positivo recurrente regenerativo, podemos construir una \'unica versi\'on estacionaria de este proceso, $X_{e}=\left\{X_{e}\left(t\right)\right\}$, donde $X_{e}$ es un proceso estoc\'astico regenerativo y estrictamente estacionario, con distribuci\'on marginal distribuida como $X_{\infty}$
\end{itemize}
\end{Note}

%__________________________________________________________________________________________
%\subsection{Procesos Regenerativos Estacionarios - Stidham \cite{Stidham}}
%__________________________________________________________________________________________


Un proceso estoc\'astico a tiempo continuo $\left\{V\left(t\right),t\geq0\right\}$ es un proceso regenerativo si existe una sucesi\'on de variables aleatorias independientes e id\'enticamente distribuidas $\left\{X_{1},X_{2},\ldots\right\}$, sucesi\'on de renovaci\'on, tal que para cualquier conjunto de Borel $A$, 

\begin{eqnarray*}
\prob\left\{V\left(t\right)\in A|X_{1}+X_{2}+\cdots+X_{R\left(t\right)}=s,\left\{V\left(\tau\right),\tau<s\right\}\right\}=\prob\left\{V\left(t-s\right)\in A|X_{1}>t-s\right\},
\end{eqnarray*}
para todo $0\leq s\leq t$, donde $R\left(t\right)=\max\left\{X_{1}+X_{2}+\cdots+X_{j}\leq t\right\}=$n\'umero de renovaciones ({\emph{puntos de regeneraci\'on}}) que ocurren en $\left[0,t\right]$. El intervalo $\left[0,X_{1}\right)$ es llamado {\emph{primer ciclo de regeneraci\'on}} de $\left\{V\left(t \right),t\geq0\right\}$, $\left[X_{1},X_{1}+X_{2}\right)$ el {\emph{segundo ciclo de regeneraci\'on}}, y as\'i sucesivamente.

Sea $X=X_{1}$ y sea $F$ la funci\'on de distrbuci\'on de $X$


\begin{Def}
Se define el proceso estacionario, $\left\{V^{*}\left(t\right),t\geq0\right\}$, para $\left\{V\left(t\right),t\geq0\right\}$ por

\begin{eqnarray*}
\prob\left\{V\left(t\right)\in A\right\}=\frac{1}{\esp\left[X\right]}\int_{0}^{\infty}\prob\left\{V\left(t+x\right)\in A|X>x\right\}\left(1-F\left(x\right)\right)dx,
\end{eqnarray*} 
para todo $t\geq0$ y todo conjunto de Borel $A$.
\end{Def}

\begin{Def}
Una distribuci\'on se dice que es {\emph{aritm\'etica}} si todos sus puntos de incremento son m\'ultiplos de la forma $0,\lambda, 2\lambda,\ldots$ para alguna $\lambda>0$ entera.
\end{Def}


\begin{Def}
Una modificaci\'on medible de un proceso $\left\{V\left(t\right),t\geq0\right\}$, es una versi\'on de este, $\left\{V\left(t,w\right)\right\}$ conjuntamente medible para $t\geq0$ y para $w\in S$, $S$ espacio de estados para $\left\{V\left(t\right),t\geq0\right\}$.
\end{Def}

\begin{Teo}
Sea $\left\{V\left(t\right),t\geq\right\}$ un proceso regenerativo no negativo con modificaci\'on medible. Sea $\esp\left[X\right]<\infty$. Entonces el proceso estacionario dado por la ecuaci\'on anterior est\'a bien definido y tiene funci\'on de distribuci\'on independiente de $t$, adem\'as
\begin{itemize}
\item[i)] \begin{eqnarray*}
\esp\left[V^{*}\left(0\right)\right]&=&\frac{\esp\left[\int_{0}^{X}V\left(s\right)ds\right]}{\esp\left[X\right]}\end{eqnarray*}
\item[ii)] Si $\esp\left[V^{*}\left(0\right)\right]<\infty$, equivalentemente, si $\esp\left[\int_{0}^{X}V\left(s\right)ds\right]<\infty$,entonces
\begin{eqnarray*}
\frac{\int_{0}^{t}V\left(s\right)ds}{t}\rightarrow\frac{\esp\left[\int_{0}^{X}V\left(s\right)ds\right]}{\esp\left[X\right]}
\end{eqnarray*}
con probabilidad 1 y en media, cuando $t\rightarrow\infty$.
\end{itemize}
\end{Teo}
%
%___________________________________________________________________________________________
%\vspace{5.5cm}
%\chapter{Cadenas de Markov estacionarias}
%\vspace{-1.0cm}


%__________________________________________________________________________________________
%\subsection{Procesos Regenerativos Estacionarios - Stidham \cite{Stidham}}
%__________________________________________________________________________________________


Un proceso estoc\'astico a tiempo continuo $\left\{V\left(t\right),t\geq0\right\}$ es un proceso regenerativo si existe una sucesi\'on de variables aleatorias independientes e id\'enticamente distribuidas $\left\{X_{1},X_{2},\ldots\right\}$, sucesi\'on de renovaci\'on, tal que para cualquier conjunto de Borel $A$, 

\begin{eqnarray*}
\prob\left\{V\left(t\right)\in A|X_{1}+X_{2}+\cdots+X_{R\left(t\right)}=s,\left\{V\left(\tau\right),\tau<s\right\}\right\}=\prob\left\{V\left(t-s\right)\in A|X_{1}>t-s\right\},
\end{eqnarray*}
para todo $0\leq s\leq t$, donde $R\left(t\right)=\max\left\{X_{1}+X_{2}+\cdots+X_{j}\leq t\right\}=$n\'umero de renovaciones ({\emph{puntos de regeneraci\'on}}) que ocurren en $\left[0,t\right]$. El intervalo $\left[0,X_{1}\right)$ es llamado {\emph{primer ciclo de regeneraci\'on}} de $\left\{V\left(t \right),t\geq0\right\}$, $\left[X_{1},X_{1}+X_{2}\right)$ el {\emph{segundo ciclo de regeneraci\'on}}, y as\'i sucesivamente.

Sea $X=X_{1}$ y sea $F$ la funci\'on de distrbuci\'on de $X$


\begin{Def}
Se define el proceso estacionario, $\left\{V^{*}\left(t\right),t\geq0\right\}$, para $\left\{V\left(t\right),t\geq0\right\}$ por

\begin{eqnarray*}
\prob\left\{V\left(t\right)\in A\right\}=\frac{1}{\esp\left[X\right]}\int_{0}^{\infty}\prob\left\{V\left(t+x\right)\in A|X>x\right\}\left(1-F\left(x\right)\right)dx,
\end{eqnarray*} 
para todo $t\geq0$ y todo conjunto de Borel $A$.
\end{Def}

\begin{Def}
Una distribuci\'on se dice que es {\emph{aritm\'etica}} si todos sus puntos de incremento son m\'ultiplos de la forma $0,\lambda, 2\lambda,\ldots$ para alguna $\lambda>0$ entera.
\end{Def}


\begin{Def}
Una modificaci\'on medible de un proceso $\left\{V\left(t\right),t\geq0\right\}$, es una versi\'on de este, $\left\{V\left(t,w\right)\right\}$ conjuntamente medible para $t\geq0$ y para $w\in S$, $S$ espacio de estados para $\left\{V\left(t\right),t\geq0\right\}$.
\end{Def}

\begin{Teo}
Sea $\left\{V\left(t\right),t\geq\right\}$ un proceso regenerativo no negativo con modificaci\'on medible. Sea $\esp\left[X\right]<\infty$. Entonces el proceso estacionario dado por la ecuaci\'on anterior est\'a bien definido y tiene funci\'on de distribuci\'on independiente de $t$, adem\'as
\begin{itemize}
\item[i)] \begin{eqnarray*}
\esp\left[V^{*}\left(0\right)\right]&=&\frac{\esp\left[\int_{0}^{X}V\left(s\right)ds\right]}{\esp\left[X\right]}\end{eqnarray*}
\item[ii)] Si $\esp\left[V^{*}\left(0\right)\right]<\infty$, equivalentemente, si $\esp\left[\int_{0}^{X}V\left(s\right)ds\right]<\infty$,entonces
\begin{eqnarray*}
\frac{\int_{0}^{t}V\left(s\right)ds}{t}\rightarrow\frac{\esp\left[\int_{0}^{X}V\left(s\right)ds\right]}{\esp\left[X\right]}
\end{eqnarray*}
con probabilidad 1 y en media, cuando $t\rightarrow\infty$.
\end{itemize}
\end{Teo}

Para $\left\{X\left(t\right):t\geq0\right\}$ Proceso Estoc\'astico a tiempo continuo con estado de espacios $S$, que es un espacio m\'etrico, con trayectorias continuas por la derecha y con l\'imites por la izquierda c.s. Sea $N\left(t\right)$ un proceso de renovaci\'on en $\rea_{+}$ definido en el mismo espacio de probabilidad que $X\left(t\right)$, con tiempos de renovaci\'on $T$ y tiempos de inter-renovaci\'on $\xi_{n}=T_{n}-T_{n-1}$, con misma distribuci\'on $F$ de media finita $\mu$.


%______________________________________________________________________
%\subsection{Ejemplos, Notas importantes}


Sean $T_{1},T_{2},\ldots$ los puntos donde las longitudes de las colas de la red de sistemas de visitas c\'iclicas son cero simult\'aneamente, cuando la cola $Q_{j}$ es visitada por el servidor para dar servicio, es decir, $L_{1}\left(T_{i}\right)=0,L_{2}\left(T_{i}\right)=0,\hat{L}_{1}\left(T_{i}\right)=0$ y $\hat{L}_{2}\left(T_{i}\right)=0$, a estos puntos se les denominar\'a puntos regenerativos. Sea la funci\'on generadora de momentos para $L_{i}$, el n\'umero de usuarios en la cola $Q_{i}\left(z\right)$ en cualquier momento, est\'a dada por el tiempo promedio de $z^{L_{i}\left(t\right)}$ sobre el ciclo regenerativo definido anteriormente:

\begin{eqnarray*}
Q_{i}\left(z\right)&=&\esp\left[z^{L_{i}\left(t\right)}\right]=\frac{\esp\left[\sum_{m=1}^{M_{i}}\sum_{t=\tau_{i}\left(m\right)}^{\tau_{i}\left(m+1\right)-1}z^{L_{i}\left(t\right)}\right]}{\esp\left[\sum_{m=1}^{M_{i}}\tau_{i}\left(m+1\right)-\tau_{i}\left(m\right)\right]}
\end{eqnarray*}

$M_{i}$ es un tiempo de paro en el proceso regenerativo con $\esp\left[M_{i}\right]<\infty$\footnote{En Stidham\cite{Stidham} y Heyman  se muestra que una condici\'on suficiente para que el proceso regenerativo 
estacionario sea un procesoo estacionario es que el valor esperado del tiempo del ciclo regenerativo sea finito, es decir: $\esp\left[\sum_{m=1}^{M_{i}}C_{i}^{(m)}\right]<\infty$, como cada $C_{i}^{(m)}$ contiene intervalos de r\'eplica positivos, se tiene que $\esp\left[M_{i}\right]<\infty$, adem\'as, como $M_{i}>0$, se tiene que la condici\'on anterior es equivalente a tener que $\esp\left[C_{i}\right]<\infty$,
por lo tanto una condici\'on suficiente para la existencia del proceso regenerativo est\'a dada por $\sum_{k=1}^{N}\mu_{k}<1.$}, se sigue del lema de Wald que:


\begin{eqnarray*}
\esp\left[\sum_{m=1}^{M_{i}}\sum_{t=\tau_{i}\left(m\right)}^{\tau_{i}\left(m+1\right)-1}z^{L_{i}\left(t\right)}\right]&=&\esp\left[M_{i}\right]\esp\left[\sum_{t=\tau_{i}\left(m\right)}^{\tau_{i}\left(m+1\right)-1}z^{L_{i}\left(t\right)}\right]\\
\esp\left[\sum_{m=1}^{M_{i}}\tau_{i}\left(m+1\right)-\tau_{i}\left(m\right)\right]&=&\esp\left[M_{i}\right]\esp\left[\tau_{i}\left(m+1\right)-\tau_{i}\left(m\right)\right]
\end{eqnarray*}

por tanto se tiene que


\begin{eqnarray*}
Q_{i}\left(z\right)&=&\frac{\esp\left[\sum_{t=\tau_{i}\left(m\right)}^{\tau_{i}\left(m+1\right)-1}z^{L_{i}\left(t\right)}\right]}{\esp\left[\tau_{i}\left(m+1\right)-\tau_{i}\left(m\right)\right]}
\end{eqnarray*}

observar que el denominador es simplemente la duraci\'on promedio del tiempo del ciclo.


Haciendo las siguientes sustituciones en la ecuaci\'on (\ref{Corolario2}): $n\rightarrow t-\tau_{i}\left(m\right)$, $T \rightarrow \overline{\tau}_{i}\left(m\right)-\tau_{i}\left(m\right)$, $L_{n}\rightarrow L_{i}\left(t\right)$ y $F\left(z\right)=\esp\left[z^{L_{0}}\right]\rightarrow F_{i}\left(z\right)=\esp\left[z^{L_{i}\tau_{i}\left(m\right)}\right]$, se puede ver que

\begin{eqnarray}\label{Eq.Arribos.Primera}
\esp\left[\sum_{n=0}^{T-1}z^{L_{n}}\right]=
\esp\left[\sum_{t=\tau_{i}\left(m\right)}^{\overline{\tau}_{i}\left(m\right)-1}z^{L_{i}\left(t\right)}\right]
=z\frac{F_{i}\left(z\right)-1}{z-P_{i}\left(z\right)}
\end{eqnarray}

Por otra parte durante el tiempo de intervisita para la cola $i$, $L_{i}\left(t\right)$ solamente se incrementa de manera que el incremento por intervalo de tiempo est\'a dado por la funci\'on generadora de probabilidades de $P_{i}\left(z\right)$, por tanto la suma sobre el tiempo de intervisita puede evaluarse como:

\begin{eqnarray*}
\esp\left[\sum_{t=\tau_{i}\left(m\right)}^{\tau_{i}\left(m+1\right)-1}z^{L_{i}\left(t\right)}\right]&=&\esp\left[\sum_{t=\tau_{i}\left(m\right)}^{\tau_{i}\left(m+1\right)-1}\left\{P_{i}\left(z\right)\right\}^{t-\overline{\tau}_{i}\left(m\right)}\right]=\frac{1-\esp\left[\left\{P_{i}\left(z\right)\right\}^{\tau_{i}\left(m+1\right)-\overline{\tau}_{i}\left(m\right)}\right]}{1-P_{i}\left(z\right)}\\
&=&\frac{1-I_{i}\left[P_{i}\left(z\right)\right]}{1-P_{i}\left(z\right)}
\end{eqnarray*}
por tanto

\begin{eqnarray*}
\esp\left[\sum_{t=\tau_{i}\left(m\right)}^{\tau_{i}\left(m+1\right)-1}z^{L_{i}\left(t\right)}\right]&=&
\frac{1-F_{i}\left(z\right)}{1-P_{i}\left(z\right)}
\end{eqnarray*}

Por lo tanto

\begin{eqnarray*}
Q_{i}\left(z\right)&=&\frac{\esp\left[\sum_{t=\tau_{i}\left(m\right)}^{\tau_{i}
\left(m+1\right)-1}z^{L_{i}\left(t\right)}\right]}{\esp\left[\tau_{i}\left(m+1\right)-\tau_{i}\left(m\right)\right]}\\
&=&\frac{1}{\esp\left[\tau_{i}\left(m+1\right)-\tau_{i}\left(m\right)\right]}
\left\{
\esp\left[\sum_{t=\tau_{i}\left(m\right)}^{\overline{\tau}_{i}\left(m\right)-1}
z^{L_{i}\left(t\right)}\right]
+\esp\left[\sum_{t=\overline{\tau}_{i}\left(m\right)}^{\tau_{i}\left(m+1\right)-1}
z^{L_{i}\left(t\right)}\right]\right\}\\
&=&\frac{1}{\esp\left[\tau_{i}\left(m+1\right)-\tau_{i}\left(m\right)\right]}
\left\{
z\frac{F_{i}\left(z\right)-1}{z-P_{i}\left(z\right)}+\frac{1-F_{i}\left(z\right)}
{1-P_{i}\left(z\right)}
\right\}
\end{eqnarray*}

es decir

\begin{equation}
Q_{i}\left(z\right)=\frac{1}{\esp\left[C_{i}\right]}\cdot\frac{1-F_{i}\left(z\right)}{P_{i}\left(z\right)-z}\cdot\frac{\left(1-z\right)P_{i}\left(z\right)}{1-P_{i}\left(z\right)}
\end{equation}

\begin{Teo}
Dada una Red de Sistemas de Visitas C\'iclicas (RSVC), conformada por dos Sistemas de Visitas C\'iclicas (SVC), donde cada uno de ellos consta de dos colas tipo $M/M/1$. Los dos sistemas est\'an comunicados entre s\'i por medio de la transferencia de usuarios entre las colas $Q_{1}\leftrightarrow Q_{3}$ y $Q_{2}\leftrightarrow Q_{4}$. Se definen los eventos para los procesos de arribos al tiempo $t$, $A_{j}\left(t\right)=\left\{0 \textrm{ arribos en }Q_{j}\textrm{ al tiempo }t\right\}$ para alg\'un tiempo $t\geq0$ y $Q_{j}$ la cola $j$-\'esima en la RSVC, para $j=1,2,3,4$.  Existe un intervalo $I\neq\emptyset$ tal que para $T^{*}\in I$, tal que $\prob\left\{A_{1}\left(T^{*}\right),A_{2}\left(Tt^{*}\right),
A_{3}\left(T^{*}\right),A_{4}\left(T^{*}\right)|T^{*}\in I\right\}>0$.
\end{Teo}

\begin{proof}
Sin p\'erdida de generalidad podemos considerar como base del an\'alisis a la cola $Q_{1}$ del primer sistema que conforma la RSVC.

Sea $n>0$, ciclo en el primer sistema en el que se sabe que $L_{j}\left(\overline{\tau}_{1}\left(n\right)\right)=0$, pues la pol\'itica de servicio con que atienden los servidores es la exhaustiva. Como es sabido, para trasladarse a la siguiente cola, el servidor incurre en un tiempo de traslado $r_{1}\left(n\right)>0$, entonces tenemos que $\tau_{2}\left(n\right)=\overline{\tau}_{1}\left(n\right)+r_{1}\left(n\right)$.


Definamos el intervalo $I_{1}\equiv\left[\overline{\tau}_{1}\left(n\right),\tau_{2}\left(n\right)\right]$ de longitud $\xi_{1}=r_{1}\left(n\right)$. Dado que los tiempos entre arribo son exponenciales con tasa $\tilde{\mu}_{1}=\mu_{1}+\hat{\mu}_{1}$ ($\mu_{1}$ son los arribos a $Q_{1}$ por primera vez al sistema, mientras que $\hat{\mu}_{1}$ son los arribos de traslado procedentes de $Q_{3}$) se tiene que la probabilidad del evento $A_{1}\left(t\right)$ est\'a dada por 

\begin{equation}
\prob\left\{A_{1}\left(t\right)|t\in I_{1}\left(n\right)\right\}=e^{-\tilde{\mu}_{1}\xi_{1}\left(n\right)}.
\end{equation} 

Por otra parte, para la cola $Q_{2}$, el tiempo $\overline{\tau}_{2}\left(n-1\right)$ es tal que $L_{2}\left(\overline{\tau}_{2}\left(n-1\right)\right)=0$, es decir, es el tiempo en que la cola queda totalmente vac\'ia en el ciclo anterior a $n$. Entonces tenemos un sgundo intervalo $I_{2}\equiv\left[\overline{\tau}_{2}\left(n-1\right),\tau_{2}\left(n\right)\right]$. Por lo tanto la probabilidad del evento $A_{2}\left(t\right)$ tiene probabilidad dada por

\begin{equation}
\prob\left\{A_{2}\left(t\right)|t\in I_{2}\left(n\right)\right\}=e^{-\tilde{\mu}_{2}\xi_{2}\left(n\right)},
\end{equation} 

donde $\xi_{2}\left(n\right)=\tau_{2}\left(n\right)-\overline{\tau}_{2}\left(n-1\right)$.



Entonces, se tiene que

\begin{eqnarray*}
\prob\left\{A_{1}\left(t\right),A_{2}\left(t\right)|t\in I_{1}\left(n\right)\right\}&=&
\prob\left\{A_{1}\left(t\right)|t\in I_{1}\left(n\right)\right\}
\prob\left\{A_{2}\left(t\right)|t\in I_{1}\left(n\right)\right\}\\
&\geq&
\prob\left\{A_{1}\left(t\right)|t\in I_{1}\left(n\right)\right\}
\prob\left\{A_{2}\left(t\right)|t\in I_{2}\left(n\right)\right\}\\
&=&e^{-\tilde{\mu}_{1}\xi_{1}\left(n\right)}e^{-\tilde{\mu}_{2}\xi_{2}\left(n\right)}
=e^{-\left[\tilde{\mu}_{1}\xi_{1}\left(n\right)+\tilde{\mu}_{2}\xi_{2}\left(n\right)\right]}.
\end{eqnarray*}


es decir, 

\begin{equation}
\prob\left\{A_{1}\left(t\right),A_{2}\left(t\right)|t\in I_{1}\left(n\right)\right\}
=e^{-\left[\tilde{\mu}_{1}\xi_{1}\left(n\right)+\tilde{\mu}_{2}\xi_{2}
\left(n\right)\right]}>0.
\end{equation}

En lo que respecta a la relaci\'on entre los dos SVC que conforman la RSVC, se afirma que existe $m>0$ tal que $\overline{\tau}_{3}\left(m\right)<\tau_{2}\left(n\right)<\tau_{4}\left(m\right)$.

Para $Q_{3}$ sea $I_{3}=\left[\overline{\tau}_{3}\left(m\right),\tau_{4}\left(m\right)\right]$ con longitud  $\xi_{3}\left(m\right)=r_{3}\left(m\right)$, entonces 

\begin{equation}
\prob\left\{A_{3}\left(t\right)|t\in I_{3}\left(n\right)\right\}=e^{-\tilde{\mu}_{3}\xi_{3}\left(n\right)}.
\end{equation} 

An\'alogamente que como se hizo para $Q_{2}$, tenemos que para $Q_{4}$ se tiene el intervalo $I_{4}=\left[\overline{\tau}_{4}\left(m-1\right),\tau_{4}\left(m\right)\right]$ con longitud $\xi_{4}\left(m\right)=\tau_{4}\left(m\right)-\overline{\tau}_{4}\left(m-1\right)$, entonces


\begin{equation}
\prob\left\{A_{4}\left(t\right)|t\in I_{4}\left(m\right)\right\}=e^{-\tilde{\mu}_{4}\xi_{4}\left(n\right)}.
\end{equation} 

Al igual que para el primer sistema, dado que $I_{3}\left(m\right)\subset I_{4}\left(m\right)$, se tiene que

\begin{eqnarray*}
\xi_{3}\left(m\right)\leq\xi_{4}\left(m\right)&\Leftrightarrow& -\xi_{3}\left(m\right)\geq-\xi_{4}\left(m\right)
\\
-\tilde{\mu}_{4}\xi_{3}\left(m\right)\geq-\tilde{\mu}_{4}\xi_{4}\left(m\right)&\Leftrightarrow&
e^{-\tilde{\mu}_{4}\xi_{3}\left(m\right)}\geq e^{-\tilde{\mu}_{4}\xi_{4}\left(m\right)}\\
\prob\left\{A_{4}\left(t\right)|t\in I_{3}\left(m\right)\right\}&\geq&
\prob\left\{A_{4}\left(t\right)|t\in I_{4}\left(m\right)\right\}
\end{eqnarray*}

Entonces, dado que los eventos $A_{3}$ y $A_{4}$ son independientes, se tiene que

\begin{eqnarray*}
\prob\left\{A_{3}\left(t\right),A_{4}\left(t\right)|t\in I_{3}\left(m\right)\right\}&=&
\prob\left\{A_{3}\left(t\right)|t\in I_{3}\left(m\right)\right\}
\prob\left\{A_{4}\left(t\right)|t\in I_{3}\left(m\right)\right\}\\
&\geq&
\prob\left\{A_{3}\left(t\right)|t\in I_{3}\left(n\right)\right\}
\prob\left\{A_{4}\left(t\right)|t\in I_{4}\left(n\right)\right\}\\
&=&e^{-\tilde{\mu}_{3}\xi_{3}\left(m\right)}e^{-\tilde{\mu}_{4}\xi_{4}
\left(m\right)}
=e^{-\left[\tilde{\mu}_{3}\xi_{3}\left(m\right)+\tilde{\mu}_{4}\xi_{4}
\left(m\right)\right]}.
\end{eqnarray*}


es decir, 

\begin{equation}
\prob\left\{A_{3}\left(t\right),A_{4}\left(t\right)|t\in I_{3}\left(m\right)\right\}
=e^{-\left[\tilde{\mu}_{3}\xi_{3}\left(m\right)+\tilde{\mu}_{4}\xi_{4}
\left(m\right)\right]}>0.
\end{equation}

Por construcci\'on se tiene que $I\left(n,m\right)\equiv I_{1}\left(n\right)\cap I_{3}\left(m\right)\neq\emptyset$,entonces en particular se tienen las contenciones $I\left(n,m\right)\subseteq I_{1}\left(n\right)$ y $I\left(n,m\right)\subseteq I_{3}\left(m\right)$, por lo tanto si definimos $\xi_{n,m}\equiv\ell\left(I\left(n,m\right)\right)$ tenemos que

\begin{eqnarray*}
\xi_{n,m}\leq\xi_{1}\left(n\right)\textrm{ y }\xi_{n,m}\leq\xi_{3}\left(m\right)\textrm{ entonces }
-\xi_{n,m}\geq-\xi_{1}\left(n\right)\textrm{ y }-\xi_{n,m}\leq-\xi_{3}\left(m\right)\\
\end{eqnarray*}
por lo tanto tenemos las desigualdades 



\begin{eqnarray*}
\begin{array}{ll}
-\tilde{\mu}_{1}\xi_{n,m}\geq-\tilde{\mu}_{1}\xi_{1}\left(n\right),&
-\tilde{\mu}_{2}\xi_{n,m}\geq-\tilde{\mu}_{2}\xi_{1}\left(n\right)
\geq-\tilde{\mu}_{2}\xi_{2}\left(n\right),\\
-\tilde{\mu}_{3}\xi_{n,m}\geq-\tilde{\mu}_{3}\xi_{3}\left(m\right),&
-\tilde{\mu}_{4}\xi_{n,m}\geq-\tilde{\mu}_{4}\xi_{3}\left(m\right)
\geq-\tilde{\mu}_{4}\xi_{4}\left(m\right).
\end{array}
\end{eqnarray*}

Sea $T^{*}\in I_{n,m}$, entonces dado que en particular $T^{*}\in I_{1}\left(n\right)$ se cumple con probabilidad positiva que no hay arribos a las colas $Q_{1}$ y $Q_{2}$, en consecuencia, tampoco hay usuarios de transferencia para $Q_{3}$ y $Q_{4}$, es decir, $\tilde{\mu}_{1}=\mu_{1}$, $\tilde{\mu}_{2}=\mu_{2}$, $\tilde{\mu}_{3}=\mu_{3}$, $\tilde{\mu}_{4}=\mu_{4}$, es decir, los eventos $Q_{1}$ y $Q_{3}$ son condicionalmente independientes en el intervalo $I_{n,m}$; lo mismo ocurre para las colas $Q_{2}$ y $Q_{4}$, por lo tanto tenemos que


\begin{eqnarray}
\begin{array}{l}
\prob\left\{A_{1}\left(T^{*}\right),A_{2}\left(T^{*}\right),
A_{3}\left(T^{*}\right),A_{4}\left(T^{*}\right)|T^{*}\in I_{n,m}\right\}
=\prod_{j=1}^{4}\prob\left\{A_{j}\left(T^{*}\right)|T^{*}\in I_{n,m}\right\}\\
\geq\prob\left\{A_{1}\left(T^{*}\right)|T^{*}\in I_{1}\left(n\right)\right\}
\prob\left\{A_{2}\left(T^{*}\right)|T^{*}\in I_{2}\left(n\right)\right\}
\prob\left\{A_{3}\left(T^{*}\right)|T^{*}\in I_{3}\left(m\right)\right\}
\prob\left\{A_{4}\left(T^{*}\right)|T^{*}\in I_{4}\left(m\right)\right\}\\
=e^{-\mu_{1}\xi_{1}\left(n\right)}
e^{-\mu_{2}\xi_{2}\left(n\right)}
e^{-\mu_{3}\xi_{3}\left(m\right)}
e^{-\mu_{4}\xi_{4}\left(m\right)}
=e^{-\left[\tilde{\mu}_{1}\xi_{1}\left(n\right)
+\tilde{\mu}_{2}\xi_{2}\left(n\right)
+\tilde{\mu}_{3}\xi_{3}\left(m\right)
+\tilde{\mu}_{4}\xi_{4}
\left(m\right)\right]}>0.
\end{array}
\end{eqnarray}
\end{proof}


Estos resultados aparecen en Daley (1968) \cite{Daley68} para $\left\{T_{n}\right\}$ intervalos de inter-arribo, $\left\{D_{n}\right\}$ intervalos de inter-salida y $\left\{S_{n}\right\}$ tiempos de servicio.

\begin{itemize}
\item Si el proceso $\left\{T_{n}\right\}$ es Poisson, el proceso $\left\{D_{n}\right\}$ es no correlacionado si y s\'olo si es un proceso Poisso, lo cual ocurre si y s\'olo si $\left\{S_{n}\right\}$ son exponenciales negativas.

\item Si $\left\{S_{n}\right\}$ son exponenciales negativas, $\left\{D_{n}\right\}$ es un proceso de renovaci\'on  si y s\'olo si es un proceso Poisson, lo cual ocurre si y s\'olo si $\left\{T_{n}\right\}$ es un proceso Poisson.

\item $\esp\left(D_{n}\right)=\esp\left(T_{n}\right)$.

\item Para un sistema de visitas $GI/M/1$ se tiene el siguiente teorema:

\begin{Teo}
En un sistema estacionario $GI/M/1$ los intervalos de interpartida tienen
\begin{eqnarray*}
\esp\left(e^{-\theta D_{n}}\right)&=&\mu\left(\mu+\theta\right)^{-1}\left[\delta\theta
-\mu\left(1-\delta\right)\alpha\left(\theta\right)\right]
\left[\theta-\mu\left(1-\delta\right)^{-1}\right]\\
\alpha\left(\theta\right)&=&\esp\left[e^{-\theta T_{0}}\right]\\
var\left(D_{n}\right)&=&var\left(T_{0}\right)-\left(\tau^{-1}-\delta^{-1}\right)
2\delta\left(\esp\left(S_{0}\right)\right)^{2}\left(1-\delta\right)^{-1}.
\end{eqnarray*}
\end{Teo}



\begin{Teo}
El proceso de salida de un sistema de colas estacionario $GI/M/1$ es un proceso de renovaci\'on si y s\'olo si el proceso de entrada es un proceso Poisson, en cuyo caso el proceso de salida es un proceso Poisson.
\end{Teo}


\begin{Teo}
Los intervalos de interpartida $\left\{D_{n}\right\}$ de un sistema $M/G/1$ estacionario son no correlacionados si y s\'olo si la distribuci\'on de los tiempos de servicio es exponencial negativa, es decir, el sistema es de tipo  $M/M/1$.

\end{Teo}



\end{itemize}


%\section{Resultados para Procesos de Salida}

En Sigman, Thorison y Wolff \cite{Sigman2} prueban que para la existencia de un una sucesi\'on infinita no decreciente de tiempos de regeneraci\'on $\tau_{1}\leq\tau_{2}\leq\cdots$ en los cuales el proceso se regenera, basta un tiempo de regeneraci\'on $R_{1}$, donde $R_{j}=\tau_{j}-\tau_{j-1}$. Para tal efecto se requiere la existencia de un espacio de probabilidad $\left(\Omega,\mathcal{F},\prob\right)$, y proceso estoc\'astico $\textit{X}=\left\{X\left(t\right):t\geq0\right\}$ con espacio de estados $\left(S,\mathcal{R}\right)$, con $\mathcal{R}$ $\sigma$-\'algebra.

\begin{Prop}
Si existe una variable aleatoria no negativa $R_{1}$ tal que $\theta_{R\footnotesize{1}}X=_{D}X$, entonces $\left(\Omega,\mathcal{F},\prob\right)$ puede extenderse para soportar una sucesi\'on estacionaria de variables aleatorias $R=\left\{R_{k}:k\geq1\right\}$, tal que para $k\geq1$,
\begin{eqnarray*}
\theta_{k}\left(X,R\right)=_{D}\left(X,R\right).
\end{eqnarray*}

Adem\'as, para $k\geq1$, $\theta_{k}R$ es condicionalmente independiente de $\left(X,R_{1},\ldots,R_{k}\right)$, dado $\theta_{\tau k}X$.

\end{Prop}


\begin{itemize}
\item Doob en 1953 demostr\'o que el estado estacionario de un proceso de partida en un sistema de espera $M/G/\infty$, es Poisson con la misma tasa que el proceso de arribos.

\item Burke en 1968, fue el primero en demostrar que el estado estacionario de un proceso de salida de una cola $M/M/s$ es un proceso Poisson.

\item Disney en 1973 obtuvo el siguiente resultado:

\begin{Teo}
Para el sistema de espera $M/G/1/L$ con disciplina FIFO, el proceso $\textbf{I}$ es un proceso de renovaci\'on si y s\'olo si el proceso denominado longitud de la cola es estacionario y se cumple cualquiera de los siguientes casos:

\begin{itemize}
\item[a)] Los tiempos de servicio son identicamente cero;
\item[b)] $L=0$, para cualquier proceso de servicio $S$;
\item[c)] $L=1$ y $G=D$;
\item[d)] $L=\infty$ y $G=M$.
\end{itemize}
En estos casos, respectivamente, las distribuciones de interpartida $P\left\{T_{n+1}-T_{n}\leq t\right\}$ son


\begin{itemize}
\item[a)] $1-e^{-\lambda t}$, $t\geq0$;
\item[b)] $1-e^{-\lambda t}*F\left(t\right)$, $t\geq0$;
\item[c)] $1-e^{-\lambda t}*\indora_{d}\left(t\right)$, $t\geq0$;
\item[d)] $1-e^{-\lambda t}*F\left(t\right)$, $t\geq0$.
\end{itemize}
\end{Teo}


\item Finch (1959) mostr\'o que para los sistemas $M/G/1/L$, con $1\leq L\leq \infty$ con distribuciones de servicio dos veces diferenciable, solamente el sistema $M/M/1/\infty$ tiene proceso de salida de renovaci\'on estacionario.

\item King (1971) demostro que un sistema de colas estacionario $M/G/1/1$ tiene sus tiempos de interpartida sucesivas $D_{n}$ y $D_{n+1}$ son independientes, si y s\'olo si, $G=D$, en cuyo caso le proceso de salida es de renovaci\'on.

\item Disney (1973) demostr\'o que el \'unico sistema estacionario $M/G/1/L$, que tiene proceso de salida de renovaci\'on  son los sistemas $M/M/1$ y $M/D/1/1$.



\item El siguiente resultado es de Disney y Koning (1985)
\begin{Teo}
En un sistema de espera $M/G/s$, el estado estacionario del proceso de salida es un proceso Poisson para cualquier distribuci\'on de los tiempos de servicio si el sistema tiene cualquiera de las siguientes cuatro propiedades.

\begin{itemize}
\item[a)] $s=\infty$
\item[b)] La disciplina de servicio es de procesador compartido.
\item[c)] La disciplina de servicio es LCFS y preemptive resume, esto se cumple para $L<\infty$
\item[d)] $G=M$.
\end{itemize}

\end{Teo}

\item El siguiente resultado es de Alamatsaz (1983)

\begin{Teo}
En cualquier sistema de colas $GI/G/1/L$ con $1\leq L<\infty$ y distribuci\'on de interarribos $A$ y distribuci\'on de los tiempos de servicio $B$, tal que $A\left(0\right)=0$, $A\left(t\right)\left(1-B\left(t\right)\right)>0$ para alguna $t>0$ y $B\left(t\right)$ para toda $t>0$, es imposible que el proceso de salida estacionario sea de renovaci\'on.
\end{Teo}

\end{itemize}

Estos resultados aparecen en Daley (1968) \cite{Daley68} para $\left\{T_{n}\right\}$ intervalos de inter-arribo, $\left\{D_{n}\right\}$ intervalos de inter-salida y $\left\{S_{n}\right\}$ tiempos de servicio.

\begin{itemize}
\item Si el proceso $\left\{T_{n}\right\}$ es Poisson, el proceso $\left\{D_{n}\right\}$ es no correlacionado si y s\'olo si es un proceso Poisso, lo cual ocurre si y s\'olo si $\left\{S_{n}\right\}$ son exponenciales negativas.

\item Si $\left\{S_{n}\right\}$ son exponenciales negativas, $\left\{D_{n}\right\}$ es un proceso de renovaci\'on  si y s\'olo si es un proceso Poisson, lo cual ocurre si y s\'olo si $\left\{T_{n}\right\}$ es un proceso Poisson.

\item $\esp\left(D_{n}\right)=\esp\left(T_{n}\right)$.

\item Para un sistema de visitas $GI/M/1$ se tiene el siguiente teorema:

\begin{Teo}
En un sistema estacionario $GI/M/1$ los intervalos de interpartida tienen
\begin{eqnarray*}
\esp\left(e^{-\theta D_{n}}\right)&=&\mu\left(\mu+\theta\right)^{-1}\left[\delta\theta
-\mu\left(1-\delta\right)\alpha\left(\theta\right)\right]
\left[\theta-\mu\left(1-\delta\right)^{-1}\right]\\
\alpha\left(\theta\right)&=&\esp\left[e^{-\theta T_{0}}\right]\\
var\left(D_{n}\right)&=&var\left(T_{0}\right)-\left(\tau^{-1}-\delta^{-1}\right)
2\delta\left(\esp\left(S_{0}\right)\right)^{2}\left(1-\delta\right)^{-1}.
\end{eqnarray*}
\end{Teo}



\begin{Teo}
El proceso de salida de un sistema de colas estacionario $GI/M/1$ es un proceso de renovaci\'on si y s\'olo si el proceso de entrada es un proceso Poisson, en cuyo caso el proceso de salida es un proceso Poisson.
\end{Teo}


\begin{Teo}
Los intervalos de interpartida $\left\{D_{n}\right\}$ de un sistema $M/G/1$ estacionario son no correlacionados si y s\'olo si la distribuci\'on de los tiempos de servicio es exponencial negativa, es decir, el sistema es de tipo  $M/M/1$.

\end{Teo}



\end{itemize}
%\newpage
%________________________________________________________________________
%\section{Redes de Sistemas de Visitas C\'iclicas}
%________________________________________________________________________

Sean $Q_{1},Q_{2},Q_{3}$ y $Q_{4}$ en una Red de Sistemas de Visitas C\'iclicas (RSVC). Supongamos que cada una de las colas es del tipo $M/M/1$ con tasa de arribo $\mu_{i}$ y que la transferencia de usuarios entre los dos sistemas ocurre entre $Q_{1}\leftrightarrow Q_{3}$ y $Q_{2}\leftrightarrow Q_{4}$ con respectiva tasa de arribo igual a la tasa de salida $\hat{\mu}_{i}=\mu_{i}$, esto se sabe por lo desarrollado en la secci\'on anterior.  

Consideremos, sin p\'erdida de generalidad como base del an\'alisis, la cola $Q_{1}$ adem\'as supongamos al servidor lo comenzamos a observar una vez que termina de atender a la misma para desplazarse y llegar a $Q_{2}$, es decir al tiempo $\tau_{2}$.

Sea $n\in\nat$, $n>0$, ciclo del servidor en que regresa a $Q_{1}$ para dar servicio y atender conforme a la pol\'itica exhaustiva, entonces se tiene que $\overline{\tau}_{1}\left(n\right)$ es el tiempo del servidor en el sistema 1 en que termina de dar servicio a todos los usuarios presentes en la cola, por lo tanto se cumple que $L_{1}\left(\overline{\tau}_{1}\left(n\right)\right)=0$, entonces el servidor para llegar a $Q_{2}$ incurre en un tiempo de traslado $r_{1}$ y por tanto se cumple que $\tau_{2}\left(n\right)=\overline{\tau}_{1}\left(n\right)+r_{1}$. Dado que los tiempos entre arribos son exponenciales se cumple que 

\begin{eqnarray*}
\prob\left\{0 \textrm{ arribos en }Q_{1}\textrm{ en el intervalo }\left[\overline{\tau}_{1}\left(n\right),\overline{\tau}_{1}\left(n\right)+r_{1}\right]\right\}=e^{-\tilde{\mu}_{1}r_{1}},\\
\prob\left\{0 \textrm{ arribos en }Q_{2}\textrm{ en el intervalo }\left[\overline{\tau}_{1}\left(n\right),\overline{\tau}_{1}\left(n\right)+r_{1}\right]\right\}=e^{-\tilde{\mu}_{2}r_{1}}.
\end{eqnarray*}

El evento que nos interesa consiste en que no haya arribos desde que el servidor abandon\'o $Q_{2}$ y regresa nuevamente para dar servicio, es decir en el intervalo de tiempo $\left[\overline{\tau}_{2}\left(n-1\right),\tau_{2}\left(n\right)\right]$. Entonces, si hacemos


\begin{eqnarray*}
\varphi_{1}\left(n\right)&\equiv&\overline{\tau}_{1}\left(n\right)+r_{1}=\overline{\tau}_{2}\left(n-1\right)+r_{1}+r_{2}+\overline{\tau}_{1}\left(n\right)-\tau_{1}\left(n\right)\\
&=&\overline{\tau}_{2}\left(n-1\right)+\overline{\tau}_{1}\left(n\right)-\tau_{1}\left(n\right)+r,,
\end{eqnarray*}

y longitud del intervalo

\begin{eqnarray*}
\xi&\equiv&\overline{\tau}_{1}\left(n\right)+r_{1}-\overline{\tau}_{2}\left(n-1\right)
=\overline{\tau}_{2}\left(n-1\right)+\overline{\tau}_{1}\left(n\right)-\tau_{1}\left(n\right)+r-\overline{\tau}_{2}\left(n-1\right)\\
&=&\overline{\tau}_{1}\left(n\right)-\tau_{1}\left(n\right)+r.
\end{eqnarray*}


Entonces, determinemos la probabilidad del evento no arribos a $Q_{2}$ en $\left[\overline{\tau}_{2}\left(n-1\right),\varphi_{1}\left(n\right)\right]$:

\begin{eqnarray}
\prob\left\{0 \textrm{ arribos en }Q_{2}\textrm{ en el intervalo }\left[\overline{\tau}_{2}\left(n-1\right),\varphi_{1}\left(n\right)\right]\right\}
=e^{-\tilde{\mu}_{2}\xi}.
\end{eqnarray}

De manera an\'aloga, tenemos que la probabilidad de no arribos a $Q_{1}$ en $\left[\overline{\tau}_{2}\left(n-1\right),\varphi_{1}\left(n\right)\right]$ esta dada por

\begin{eqnarray}
\prob\left\{0 \textrm{ arribos en }Q_{1}\textrm{ en el intervalo }\left[\overline{\tau}_{2}\left(n-1\right),\varphi_{1}\left(n\right)\right]\right\}
=e^{-\tilde{\mu}_{1}\xi},
\end{eqnarray}

\begin{eqnarray}
\prob\left\{0 \textrm{ arribos en }Q_{2}\textrm{ en el intervalo }\left[\overline{\tau}_{2}\left(n-1\right),\varphi_{1}\left(n\right)\right]\right\}
=e^{-\tilde{\mu}_{2}\xi}.
\end{eqnarray}

Por tanto 

\begin{eqnarray}
\begin{array}{l}
\prob\left\{0 \textrm{ arribos en }Q_{1}\textrm{ y }Q_{2}\textrm{ en el intervalo }\left[\overline{\tau}_{2}\left(n-1\right),\varphi_{1}\left(n\right)\right]\right\}\\
=\prob\left\{0 \textrm{ arribos en }Q_{1}\textrm{ en el intervalo }\left[\overline{\tau}_{2}\left(n-1\right),\varphi_{1}\left(n\right)\right]\right\}\\
\times
\prob\left\{0 \textrm{ arribos en }Q_{2}\textrm{ en el intervalo }\left[\overline{\tau}_{2}\left(n-1\right),\varphi_{1}\left(n\right)\right]\right\}=e^{-\tilde{\mu}_{1}\xi}e^{-\tilde{\mu}_{2}\xi}
=e^{-\tilde{\mu}\xi}.
\end{array}
\end{eqnarray}

Para el segundo sistema, consideremos nuevamente $\overline{\tau}_{1}\left(n\right)+r_{1}$, sin p\'erdida de generalidad podemos suponer que existe $m>0$ tal que $\overline{\tau}_{3}\left(m\right)<\overline{\tau}_{1}+r_{1}<\tau_{4}\left(m\right)$, entonces

\begin{eqnarray}
\prob\left\{0 \textrm{ arribos en }Q_{3}\textrm{ en el intervalo }\left[\overline{\tau}_{3}\left(m\right),\overline{\tau}_{1}\left(n\right)+r_{1}\right]\right\}
=e^{-\tilde{\mu}_{3}\xi_{3}},
\end{eqnarray}
donde 
\begin{eqnarray}
\xi_{3}=\overline{\tau}_{1}\left(n\right)+r_{1}-\overline{\tau}_{3}\left(m\right)=
\overline{\tau}_{1}\left(n\right)-\overline{\tau}_{3}\left(m\right)+r_{1},
\end{eqnarray}

mientras que para $Q_{4}$ al igual que con $Q_{2}$ escribiremos $\tau_{4}\left(m\right)$ en t\'erminos de $\overline{\tau}_{4}\left(m-1\right)$:

$\varphi_{2}\equiv\tau_{4}\left(m\right)=\overline{\tau}_{4}\left(m-1\right)+r_{4}+\overline{\tau}_{3}\left(m\right)
-\tau_{3}\left(m\right)+r_{3}=\overline{\tau}_{4}\left(m-1\right)+\overline{\tau}_{3}\left(m\right)
-\tau_{3}\left(m\right)+\hat{r}$, adem\'as,

$\xi_{2}\equiv\varphi_{2}\left(m\right)-\overline{\tau}_{1}-r_{1}=\overline{\tau}_{4}\left(m-1\right)+\overline{\tau}_{3}\left(m\right)
-\tau_{3}\left(m\right)-\overline{\tau}_{1}\left(n\right)+\hat{r}-r_{1}$. 

Entonces


\begin{eqnarray}
\prob\left\{0 \textrm{ arribos en }Q_{4}\textrm{ en el intervalo }\left[\overline{\tau}_{1}\left(n\right)+r_{1},\varphi_{2}\left(m\right)\right]\right\}
=e^{-\tilde{\mu}_{4}\xi_{2}},
\end{eqnarray}

mientras que para $Q_{3}$ se tiene que 

\begin{eqnarray}
\prob\left\{0 \textrm{ arribos en }Q_{3}\textrm{ en el intervalo }\left[\overline{\tau}_{1}\left(n\right)+r_{1},\varphi_{2}\left(m\right)\right]\right\}
=e^{-\tilde{\mu}_{3}\xi_{2}}
\end{eqnarray}

Por tanto

\begin{eqnarray}
\prob\left\{0 \textrm{ arribos en }Q_{3}\wedge Q_{4}\textrm{ en el intervalo }\left[\overline{\tau}_{1}\left(n\right)+r_{1},\varphi_{2}\left(m\right)\right]\right\}
=e^{-\hat{\mu}\xi_{2}}
\end{eqnarray}
donde $\hat{\mu}=\tilde{\mu}_{3}+\tilde{\mu}_{4}$.

Ahora, definamos los intervalos $\mathcal{I}_{1}=\left[\overline{\tau}_{1}\left(n\right)+r_{1},\varphi_{1}\left(n\right)\right]$  y $\mathcal{I}_{2}=\left[\overline{\tau}_{1}\left(n\right)+r_{1},\varphi_{2}\left(m\right)\right]$, entonces, sea $\mathcal{I}=\mathcal{I}_{1}\cap\mathcal{I}_{2}$ el intervalo donde cada una de las colas se encuentran vac\'ias, es decir, si tomamos $T^{*}\in\mathcal{I}$, entonces  $L_{1}\left(T^{*}\right)=L_{2}\left(T^{*}\right)=L_{3}\left(T^{*}\right)=L_{4}\left(T^{*}\right)=0$.

Ahora, dado que por construcci\'on $\mathcal{I}\neq\emptyset$ y que para $T^{*}\in\mathcal{I}$ en ninguna de las colas han llegado usuarios, se tiene que no hay transferencia entre las colas, por lo tanto, el sistema 1 y el sistema 2 son condicionalmente independientes en $\mathcal{I}$, es decir

\begin{eqnarray}
\prob\left\{L_{1}\left(T^{*}\right),L_{2}\left(T^{*}\right),L_{3}\left(T^{*}\right),L_{4}\left(T^{*}\right)|T^{*}\in\mathcal{I}\right\}=\prod_{j=1}^{4}\prob\left\{L_{j}\left(T^{*}\right)\right\},
\end{eqnarray}

para $T^{*}\in\mathcal{I}$. 

%\newpage























%________________________________________________________________________
%\section{Procesos Regenerativos}
%________________________________________________________________________

%________________________________________________________________________
%\subsection*{Procesos Regenerativos Sigman, Thorisson y Wolff \cite{Sigman1}}
%________________________________________________________________________


\begin{Def}[Definici\'on Cl\'asica]
Un proceso estoc\'astico $X=\left\{X\left(t\right):t\geq0\right\}$ es llamado regenerativo is existe una variable aleatoria $R_{1}>0$ tal que
\begin{itemize}
\item[i)] $\left\{X\left(t+R_{1}\right):t\geq0\right\}$ es independiente de $\left\{\left\{X\left(t\right):t<R_{1}\right\},\right\}$
\item[ii)] $\left\{X\left(t+R_{1}\right):t\geq0\right\}$ es estoc\'asticamente equivalente a $\left\{X\left(t\right):t>0\right\}$
\end{itemize}

Llamamos a $R_{1}$ tiempo de regeneraci\'on, y decimos que $X$ se regenera en este punto.
\end{Def}

$\left\{X\left(t+R_{1}\right)\right\}$ es regenerativo con tiempo de regeneraci\'on $R_{2}$, independiente de $R_{1}$ pero con la misma distribuci\'on que $R_{1}$. Procediendo de esta manera se obtiene una secuencia de variables aleatorias independientes e id\'enticamente distribuidas $\left\{R_{n}\right\}$ llamados longitudes de ciclo. Si definimos a $Z_{k}\equiv R_{1}+R_{2}+\cdots+R_{k}$, se tiene un proceso de renovaci\'on llamado proceso de renovaci\'on encajado para $X$.


\begin{Note}
La existencia de un primer tiempo de regeneraci\'on, $R_{1}$, implica la existencia de una sucesi\'on completa de estos tiempos $R_{1},R_{2}\ldots,$ que satisfacen la propiedad deseada \cite{Sigman2}.
\end{Note}


\begin{Note} Para la cola $GI/GI/1$ los usuarios arriban con tiempos $t_{n}$ y son atendidos con tiempos de servicio $S_{n}$, los tiempos de arribo forman un proceso de renovaci\'on  con tiempos entre arribos independientes e identicamente distribuidos (\texttt{i.i.d.})$T_{n}=t_{n}-t_{n-1}$, adem\'as los tiempos de servicio son \texttt{i.i.d.} e independientes de los procesos de arribo. Por \textit{estable} se entiende que $\esp S_{n}<\esp T_{n}<\infty$.
\end{Note}
 


\begin{Def}
Para $x$ fijo y para cada $t\geq0$, sea $I_{x}\left(t\right)=1$ si $X\left(t\right)\leq x$,  $I_{x}\left(t\right)=0$ en caso contrario, y def\'inanse los tiempos promedio
\begin{eqnarray*}
\overline{X}&=&lim_{t\rightarrow\infty}\frac{1}{t}\int_{0}^{\infty}X\left(u\right)du\\
\prob\left(X_{\infty}\leq x\right)&=&lim_{t\rightarrow\infty}\frac{1}{t}\int_{0}^{\infty}I_{x}\left(u\right)du,
\end{eqnarray*}
cuando estos l\'imites existan.
\end{Def}

Como consecuencia del teorema de Renovaci\'on-Recompensa, se tiene que el primer l\'imite  existe y es igual a la constante
\begin{eqnarray*}
\overline{X}&=&\frac{\esp\left[\int_{0}^{R_{1}}X\left(t\right)dt\right]}{\esp\left[R_{1}\right]},
\end{eqnarray*}
suponiendo que ambas esperanzas son finitas.
 
\begin{Note}
Funciones de procesos regenerativos son regenerativas, es decir, si $X\left(t\right)$ es regenerativo y se define el proceso $Y\left(t\right)$ por $Y\left(t\right)=f\left(X\left(t\right)\right)$ para alguna funci\'on Borel medible $f\left(\cdot\right)$. Adem\'as $Y$ es regenerativo con los mismos tiempos de renovaci\'on que $X$. 

En general, los tiempos de renovaci\'on, $Z_{k}$ de un proceso regenerativo no requieren ser tiempos de paro con respecto a la evoluci\'on de $X\left(t\right)$.
\end{Note} 

\begin{Note}
Una funci\'on de un proceso de Markov, usualmente no ser\'a un proceso de Markov, sin embargo ser\'a regenerativo si el proceso de Markov lo es.
\end{Note}

 
\begin{Note}
Un proceso regenerativo con media de la longitud de ciclo finita es llamado positivo recurrente.
\end{Note}


\begin{Note}
\begin{itemize}
\item[a)] Si el proceso regenerativo $X$ es positivo recurrente y tiene trayectorias muestrales no negativas, entonces la ecuaci\'on anterior es v\'alida.
\item[b)] Si $X$ es positivo recurrente regenerativo, podemos construir una \'unica versi\'on estacionaria de este proceso, $X_{e}=\left\{X_{e}\left(t\right)\right\}$, donde $X_{e}$ es un proceso estoc\'astico regenerativo y estrictamente estacionario, con distribuci\'on marginal distribuida como $X_{\infty}$
\end{itemize}
\end{Note}


%__________________________________________________________________________________________
%\subsection*{Procesos Regenerativos Estacionarios - Stidham \cite{Stidham}}
%__________________________________________________________________________________________


Un proceso estoc\'astico a tiempo continuo $\left\{V\left(t\right),t\geq0\right\}$ es un proceso regenerativo si existe una sucesi\'on de variables aleatorias independientes e id\'enticamente distribuidas $\left\{X_{1},X_{2},\ldots\right\}$, sucesi\'on de renovaci\'on, tal que para cualquier conjunto de Borel $A$, 

\begin{eqnarray*}
\prob\left\{V\left(t\right)\in A|X_{1}+X_{2}+\cdots+X_{R\left(t\right)}=s,\left\{V\left(\tau\right),\tau<s\right\}\right\}=\prob\left\{V\left(t-s\right)\in A|X_{1}>t-s\right\},
\end{eqnarray*}
para todo $0\leq s\leq t$, donde $R\left(t\right)=\max\left\{X_{1}+X_{2}+\cdots+X_{j}\leq t\right\}=$n\'umero de renovaciones ({\emph{puntos de regeneraci\'on}}) que ocurren en $\left[0,t\right]$. El intervalo $\left[0,X_{1}\right)$ es llamado {\emph{primer ciclo de regeneraci\'on}} de $\left\{V\left(t \right),t\geq0\right\}$, $\left[X_{1},X_{1}+X_{2}\right)$ el {\emph{segundo ciclo de regeneraci\'on}}, y as\'i sucesivamente.

Sea $X=X_{1}$ y sea $F$ la funci\'on de distrbuci\'on de $X$


\begin{Def}
Se define el proceso estacionario, $\left\{V^{*}\left(t\right),t\geq0\right\}$, para $\left\{V\left(t\right),t\geq0\right\}$ por

\begin{eqnarray*}
\prob\left\{V\left(t\right)\in A\right\}=\frac{1}{\esp\left[X\right]}\int_{0}^{\infty}\prob\left\{V\left(t+x\right)\in A|X>x\right\}\left(1-F\left(x\right)\right)dx,
\end{eqnarray*} 
para todo $t\geq0$ y todo conjunto de Borel $A$.
\end{Def}

\begin{Def}
Una distribuci\'on se dice que es {\emph{aritm\'etica}} si todos sus puntos de incremento son m\'ultiplos de la forma $0,\lambda, 2\lambda,\ldots$ para alguna $\lambda>0$ entera.
\end{Def}


\begin{Def}
Una modificaci\'on medible de un proceso $\left\{V\left(t\right),t\geq0\right\}$, es una versi\'on de este, $\left\{V\left(t,w\right)\right\}$ conjuntamente medible para $t\geq0$ y para $w\in S$, $S$ espacio de estados para $\left\{V\left(t\right),t\geq0\right\}$.
\end{Def}

\begin{Teo}
Sea $\left\{V\left(t\right),t\geq\right\}$ un proceso regenerativo no negativo con modificaci\'on medible. Sea $\esp\left[X\right]<\infty$. Entonces el proceso estacionario dado por la ecuaci\'on anterior est\'a bien definido y tiene funci\'on de distribuci\'on independiente de $t$, adem\'as
\begin{itemize}
\item[i)] \begin{eqnarray*}
\esp\left[V^{*}\left(0\right)\right]&=&\frac{\esp\left[\int_{0}^{X}V\left(s\right)ds\right]}{\esp\left[X\right]}\end{eqnarray*}
\item[ii)] Si $\esp\left[V^{*}\left(0\right)\right]<\infty$, equivalentemente, si $\esp\left[\int_{0}^{X}V\left(s\right)ds\right]<\infty$,entonces
\begin{eqnarray*}
\frac{\int_{0}^{t}V\left(s\right)ds}{t}\rightarrow\frac{\esp\left[\int_{0}^{X}V\left(s\right)ds\right]}{\esp\left[X\right]}
\end{eqnarray*}
con probabilidad 1 y en media, cuando $t\rightarrow\infty$.
\end{itemize}
\end{Teo}

\begin{Coro}
Sea $\left\{V\left(t\right),t\geq0\right\}$ un proceso regenerativo no negativo, con modificaci\'on medible. Si $\esp <\infty$, $F$ es no-aritm\'etica, y para todo $x\geq0$, $P\left\{V\left(t\right)\leq x,C>x\right\}$ es de variaci\'on acotada como funci\'on de $t$ en cada intervalo finito $\left[0,\tau\right]$, entonces $V\left(t\right)$ converge en distribuci\'on  cuando $t\rightarrow\infty$ y $$\esp V=\frac{\esp \int_{0}^{X}V\left(s\right)ds}{\esp X}$$
Donde $V$ tiene la distribuci\'on l\'imite de $V\left(t\right)$ cuando $t\rightarrow\infty$.

\end{Coro}

Para el caso discreto se tienen resultados similares.



%______________________________________________________________________
%\section{Procesos de Renovaci\'on}
%______________________________________________________________________

\begin{Def}\label{Def.Tn}
Sean $0\leq T_{1}\leq T_{2}\leq \ldots$ son tiempos aleatorios infinitos en los cuales ocurren ciertos eventos. El n\'umero de tiempos $T_{n}$ en el intervalo $\left[0,t\right)$ es

\begin{eqnarray}
N\left(t\right)=\sum_{n=1}^{\infty}\indora\left(T_{n}\leq t\right),
\end{eqnarray}
para $t\geq0$.
\end{Def}

Si se consideran los puntos $T_{n}$ como elementos de $\rea_{+}$, y $N\left(t\right)$ es el n\'umero de puntos en $\rea$. El proceso denotado por $\left\{N\left(t\right):t\geq0\right\}$, denotado por $N\left(t\right)$, es un proceso puntual en $\rea_{+}$. Los $T_{n}$ son los tiempos de ocurrencia, el proceso puntual $N\left(t\right)$ es simple si su n\'umero de ocurrencias son distintas: $0<T_{1}<T_{2}<\ldots$ casi seguramente.

\begin{Def}
Un proceso puntual $N\left(t\right)$ es un proceso de renovaci\'on si los tiempos de interocurrencia $\xi_{n}=T_{n}-T_{n-1}$, para $n\geq1$, son independientes e identicamente distribuidos con distribuci\'on $F$, donde $F\left(0\right)=0$ y $T_{0}=0$. Los $T_{n}$ son llamados tiempos de renovaci\'on, referente a la independencia o renovaci\'on de la informaci\'on estoc\'astica en estos tiempos. Los $\xi_{n}$ son los tiempos de inter-renovaci\'on, y $N\left(t\right)$ es el n\'umero de renovaciones en el intervalo $\left[0,t\right)$
\end{Def}


\begin{Note}
Para definir un proceso de renovaci\'on para cualquier contexto, solamente hay que especificar una distribuci\'on $F$, con $F\left(0\right)=0$, para los tiempos de inter-renovaci\'on. La funci\'on $F$ en turno degune las otra variables aleatorias. De manera formal, existe un espacio de probabilidad y una sucesi\'on de variables aleatorias $\xi_{1},\xi_{2},\ldots$ definidas en este con distribuci\'on $F$. Entonces las otras cantidades son $T_{n}=\sum_{k=1}^{n}\xi_{k}$ y $N\left(t\right)=\sum_{n=1}^{\infty}\indora\left(T_{n}\leq t\right)$, donde $T_{n}\rightarrow\infty$ casi seguramente por la Ley Fuerte de los Grandes Números.
\end{Note}

%___________________________________________________________________________________________
%
%\subsection*{Teorema Principal de Renovaci\'on}
%___________________________________________________________________________________________
%

\begin{Note} Una funci\'on $h:\rea_{+}\rightarrow\rea$ es Directamente Riemann Integrable en los siguientes casos:
\begin{itemize}
\item[a)] $h\left(t\right)\geq0$ es decreciente y Riemann Integrable.
\item[b)] $h$ es continua excepto posiblemente en un conjunto de Lebesgue de medida 0, y $|h\left(t\right)|\leq b\left(t\right)$, donde $b$ es DRI.
\end{itemize}
\end{Note}

\begin{Teo}[Teorema Principal de Renovaci\'on]
Si $F$ es no aritm\'etica y $h\left(t\right)$ es Directamente Riemann Integrable (DRI), entonces

\begin{eqnarray*}
lim_{t\rightarrow\infty}U\star h=\frac{1}{\mu}\int_{\rea_{+}}h\left(s\right)ds.
\end{eqnarray*}
\end{Teo}

\begin{Prop}
Cualquier funci\'on $H\left(t\right)$ acotada en intervalos finitos y que es 0 para $t<0$ puede expresarse como
\begin{eqnarray*}
H\left(t\right)=U\star h\left(t\right)\textrm{,  donde }h\left(t\right)=H\left(t\right)-F\star H\left(t\right)
\end{eqnarray*}
\end{Prop}

\begin{Def}
Un proceso estoc\'astico $X\left(t\right)$ es crudamente regenerativo en un tiempo aleatorio positivo $T$ si
\begin{eqnarray*}
\esp\left[X\left(T+t\right)|T\right]=\esp\left[X\left(t\right)\right]\textrm{, para }t\geq0,\end{eqnarray*}
y con las esperanzas anteriores finitas.
\end{Def}

\begin{Prop}
Sup\'ongase que $X\left(t\right)$ es un proceso crudamente regenerativo en $T$, que tiene distribuci\'on $F$. Si $\esp\left[X\left(t\right)\right]$ es acotado en intervalos finitos, entonces
\begin{eqnarray*}
\esp\left[X\left(t\right)\right]=U\star h\left(t\right)\textrm{,  donde }h\left(t\right)=\esp\left[X\left(t\right)\indora\left(T>t\right)\right].
\end{eqnarray*}
\end{Prop}

\begin{Teo}[Regeneraci\'on Cruda]
Sup\'ongase que $X\left(t\right)$ es un proceso con valores positivo crudamente regenerativo en $T$, y def\'inase $M=\sup\left\{|X\left(t\right)|:t\leq T\right\}$. Si $T$ es no aritm\'etico y $M$ y $MT$ tienen media finita, entonces
\begin{eqnarray*}
lim_{t\rightarrow\infty}\esp\left[X\left(t\right)\right]=\frac{1}{\mu}\int_{\rea_{+}}h\left(s\right)ds,
\end{eqnarray*}
donde $h\left(t\right)=\esp\left[X\left(t\right)\indora\left(T>t\right)\right]$.
\end{Teo}

%___________________________________________________________________________________________
%
%\subsection*{Propiedades de los Procesos de Renovaci\'on}
%___________________________________________________________________________________________
%

Los tiempos $T_{n}$ est\'an relacionados con los conteos de $N\left(t\right)$ por

\begin{eqnarray*}
\left\{N\left(t\right)\geq n\right\}&=&\left\{T_{n}\leq t\right\}\\
T_{N\left(t\right)}\leq &t&<T_{N\left(t\right)+1},
\end{eqnarray*}

adem\'as $N\left(T_{n}\right)=n$, y 

\begin{eqnarray*}
N\left(t\right)=\max\left\{n:T_{n}\leq t\right\}=\min\left\{n:T_{n+1}>t\right\}
\end{eqnarray*}

Por propiedades de la convoluci\'on se sabe que

\begin{eqnarray*}
P\left\{T_{n}\leq t\right\}=F^{n\star}\left(t\right)
\end{eqnarray*}
que es la $n$-\'esima convoluci\'on de $F$. Entonces 

\begin{eqnarray*}
\left\{N\left(t\right)\geq n\right\}&=&\left\{T_{n}\leq t\right\}\\
P\left\{N\left(t\right)\leq n\right\}&=&1-F^{\left(n+1\right)\star}\left(t\right)
\end{eqnarray*}

Adem\'as usando el hecho de que $\esp\left[N\left(t\right)\right]=\sum_{n=1}^{\infty}P\left\{N\left(t\right)\geq n\right\}$
se tiene que

\begin{eqnarray*}
\esp\left[N\left(t\right)\right]=\sum_{n=1}^{\infty}F^{n\star}\left(t\right)
\end{eqnarray*}

\begin{Prop}
Para cada $t\geq0$, la funci\'on generadora de momentos $\esp\left[e^{\alpha N\left(t\right)}\right]$ existe para alguna $\alpha$ en una vecindad del 0, y de aqu\'i que $\esp\left[N\left(t\right)^{m}\right]<\infty$, para $m\geq1$.
\end{Prop}


\begin{Note}
Si el primer tiempo de renovaci\'on $\xi_{1}$ no tiene la misma distribuci\'on que el resto de las $\xi_{n}$, para $n\geq2$, a $N\left(t\right)$ se le llama Proceso de Renovaci\'on retardado, donde si $\xi$ tiene distribuci\'on $G$, entonces el tiempo $T_{n}$ de la $n$-\'esima renovaci\'on tiene distribuci\'on $G\star F^{\left(n-1\right)\star}\left(t\right)$
\end{Note}


\begin{Teo}
Para una constante $\mu\leq\infty$ ( o variable aleatoria), las siguientes expresiones son equivalentes:

\begin{eqnarray}
lim_{n\rightarrow\infty}n^{-1}T_{n}&=&\mu,\textrm{ c.s.}\\
lim_{t\rightarrow\infty}t^{-1}N\left(t\right)&=&1/\mu,\textrm{ c.s.}
\end{eqnarray}
\end{Teo}


Es decir, $T_{n}$ satisface la Ley Fuerte de los Grandes N\'umeros s\'i y s\'olo s\'i $N\left/t\right)$ la cumple.


\begin{Coro}[Ley Fuerte de los Grandes N\'umeros para Procesos de Renovaci\'on]
Si $N\left(t\right)$ es un proceso de renovaci\'on cuyos tiempos de inter-renovaci\'on tienen media $\mu\leq\infty$, entonces
\begin{eqnarray}
t^{-1}N\left(t\right)\rightarrow 1/\mu,\textrm{ c.s. cuando }t\rightarrow\infty.
\end{eqnarray}

\end{Coro}


Considerar el proceso estoc\'astico de valores reales $\left\{Z\left(t\right):t\geq0\right\}$ en el mismo espacio de probabilidad que $N\left(t\right)$

\begin{Def}
Para el proceso $\left\{Z\left(t\right):t\geq0\right\}$ se define la fluctuaci\'on m\'axima de $Z\left(t\right)$ en el intervalo $\left(T_{n-1},T_{n}\right]$:
\begin{eqnarray*}
M_{n}=\sup_{T_{n-1}<t\leq T_{n}}|Z\left(t\right)-Z\left(T_{n-1}\right)|
\end{eqnarray*}
\end{Def}

\begin{Teo}
Sup\'ongase que $n^{-1}T_{n}\rightarrow\mu$ c.s. cuando $n\rightarrow\infty$, donde $\mu\leq\infty$ es una constante o variable aleatoria. Sea $a$ una constante o variable aleatoria que puede ser infinita cuando $\mu$ es finita, y considere las expresiones l\'imite:
\begin{eqnarray}
lim_{n\rightarrow\infty}n^{-1}Z\left(T_{n}\right)&=&a,\textrm{ c.s.}\\
lim_{t\rightarrow\infty}t^{-1}Z\left(t\right)&=&a/\mu,\textrm{ c.s.}
\end{eqnarray}
La segunda expresi\'on implica la primera. Conversamente, la primera implica la segunda si el proceso $Z\left(t\right)$ es creciente, o si $lim_{n\rightarrow\infty}n^{-1}M_{n}=0$ c.s.
\end{Teo}

\begin{Coro}
Si $N\left(t\right)$ es un proceso de renovaci\'on, y $\left(Z\left(T_{n}\right)-Z\left(T_{n-1}\right),M_{n}\right)$, para $n\geq1$, son variables aleatorias independientes e id\'enticamente distribuidas con media finita, entonces,
\begin{eqnarray}
lim_{t\rightarrow\infty}t^{-1}Z\left(t\right)\rightarrow\frac{\esp\left[Z\left(T_{1}\right)-Z\left(T_{0}\right)\right]}{\esp\left[T_{1}\right]},\textrm{ c.s. cuando  }t\rightarrow\infty.
\end{eqnarray}
\end{Coro}



%___________________________________________________________________________________________
%
%\subsection{Propiedades de los Procesos de Renovaci\'on}
%___________________________________________________________________________________________
%

Los tiempos $T_{n}$ est\'an relacionados con los conteos de $N\left(t\right)$ por

\begin{eqnarray*}
\left\{N\left(t\right)\geq n\right\}&=&\left\{T_{n}\leq t\right\}\\
T_{N\left(t\right)}\leq &t&<T_{N\left(t\right)+1},
\end{eqnarray*}

adem\'as $N\left(T_{n}\right)=n$, y 

\begin{eqnarray*}
N\left(t\right)=\max\left\{n:T_{n}\leq t\right\}=\min\left\{n:T_{n+1}>t\right\}
\end{eqnarray*}

Por propiedades de la convoluci\'on se sabe que

\begin{eqnarray*}
P\left\{T_{n}\leq t\right\}=F^{n\star}\left(t\right)
\end{eqnarray*}
que es la $n$-\'esima convoluci\'on de $F$. Entonces 

\begin{eqnarray*}
\left\{N\left(t\right)\geq n\right\}&=&\left\{T_{n}\leq t\right\}\\
P\left\{N\left(t\right)\leq n\right\}&=&1-F^{\left(n+1\right)\star}\left(t\right)
\end{eqnarray*}

Adem\'as usando el hecho de que $\esp\left[N\left(t\right)\right]=\sum_{n=1}^{\infty}P\left\{N\left(t\right)\geq n\right\}$
se tiene que

\begin{eqnarray*}
\esp\left[N\left(t\right)\right]=\sum_{n=1}^{\infty}F^{n\star}\left(t\right)
\end{eqnarray*}

\begin{Prop}
Para cada $t\geq0$, la funci\'on generadora de momentos $\esp\left[e^{\alpha N\left(t\right)}\right]$ existe para alguna $\alpha$ en una vecindad del 0, y de aqu\'i que $\esp\left[N\left(t\right)^{m}\right]<\infty$, para $m\geq1$.
\end{Prop}


\begin{Note}
Si el primer tiempo de renovaci\'on $\xi_{1}$ no tiene la misma distribuci\'on que el resto de las $\xi_{n}$, para $n\geq2$, a $N\left(t\right)$ se le llama Proceso de Renovaci\'on retardado, donde si $\xi$ tiene distribuci\'on $G$, entonces el tiempo $T_{n}$ de la $n$-\'esima renovaci\'on tiene distribuci\'on $G\star F^{\left(n-1\right)\star}\left(t\right)$
\end{Note}


\begin{Teo}
Para una constante $\mu\leq\infty$ ( o variable aleatoria), las siguientes expresiones son equivalentes:

\begin{eqnarray}
lim_{n\rightarrow\infty}n^{-1}T_{n}&=&\mu,\textrm{ c.s.}\\
lim_{t\rightarrow\infty}t^{-1}N\left(t\right)&=&1/\mu,\textrm{ c.s.}
\end{eqnarray}
\end{Teo}


Es decir, $T_{n}$ satisface la Ley Fuerte de los Grandes N\'umeros s\'i y s\'olo s\'i $N\left/t\right)$ la cumple.


\begin{Coro}[Ley Fuerte de los Grandes N\'umeros para Procesos de Renovaci\'on]
Si $N\left(t\right)$ es un proceso de renovaci\'on cuyos tiempos de inter-renovaci\'on tienen media $\mu\leq\infty$, entonces
\begin{eqnarray}
t^{-1}N\left(t\right)\rightarrow 1/\mu,\textrm{ c.s. cuando }t\rightarrow\infty.
\end{eqnarray}

\end{Coro}


Considerar el proceso estoc\'astico de valores reales $\left\{Z\left(t\right):t\geq0\right\}$ en el mismo espacio de probabilidad que $N\left(t\right)$

\begin{Def}
Para el proceso $\left\{Z\left(t\right):t\geq0\right\}$ se define la fluctuaci\'on m\'axima de $Z\left(t\right)$ en el intervalo $\left(T_{n-1},T_{n}\right]$:
\begin{eqnarray*}
M_{n}=\sup_{T_{n-1}<t\leq T_{n}}|Z\left(t\right)-Z\left(T_{n-1}\right)|
\end{eqnarray*}
\end{Def}

\begin{Teo}
Sup\'ongase que $n^{-1}T_{n}\rightarrow\mu$ c.s. cuando $n\rightarrow\infty$, donde $\mu\leq\infty$ es una constante o variable aleatoria. Sea $a$ una constante o variable aleatoria que puede ser infinita cuando $\mu$ es finita, y considere las expresiones l\'imite:
\begin{eqnarray}
lim_{n\rightarrow\infty}n^{-1}Z\left(T_{n}\right)&=&a,\textrm{ c.s.}\\
lim_{t\rightarrow\infty}t^{-1}Z\left(t\right)&=&a/\mu,\textrm{ c.s.}
\end{eqnarray}
La segunda expresi\'on implica la primera. Conversamente, la primera implica la segunda si el proceso $Z\left(t\right)$ es creciente, o si $lim_{n\rightarrow\infty}n^{-1}M_{n}=0$ c.s.
\end{Teo}

\begin{Coro}
Si $N\left(t\right)$ es un proceso de renovaci\'on, y $\left(Z\left(T_{n}\right)-Z\left(T_{n-1}\right),M_{n}\right)$, para $n\geq1$, son variables aleatorias independientes e id\'enticamente distribuidas con media finita, entonces,
\begin{eqnarray}
lim_{t\rightarrow\infty}t^{-1}Z\left(t\right)\rightarrow\frac{\esp\left[Z\left(T_{1}\right)-Z\left(T_{0}\right)\right]}{\esp\left[T_{1}\right]},\textrm{ c.s. cuando  }t\rightarrow\infty.
\end{eqnarray}
\end{Coro}


%___________________________________________________________________________________________
%
%\subsection{Propiedades de los Procesos de Renovaci\'on}
%___________________________________________________________________________________________
%

Los tiempos $T_{n}$ est\'an relacionados con los conteos de $N\left(t\right)$ por

\begin{eqnarray*}
\left\{N\left(t\right)\geq n\right\}&=&\left\{T_{n}\leq t\right\}\\
T_{N\left(t\right)}\leq &t&<T_{N\left(t\right)+1},
\end{eqnarray*}

adem\'as $N\left(T_{n}\right)=n$, y 

\begin{eqnarray*}
N\left(t\right)=\max\left\{n:T_{n}\leq t\right\}=\min\left\{n:T_{n+1}>t\right\}
\end{eqnarray*}

Por propiedades de la convoluci\'on se sabe que

\begin{eqnarray*}
P\left\{T_{n}\leq t\right\}=F^{n\star}\left(t\right)
\end{eqnarray*}
que es la $n$-\'esima convoluci\'on de $F$. Entonces 

\begin{eqnarray*}
\left\{N\left(t\right)\geq n\right\}&=&\left\{T_{n}\leq t\right\}\\
P\left\{N\left(t\right)\leq n\right\}&=&1-F^{\left(n+1\right)\star}\left(t\right)
\end{eqnarray*}

Adem\'as usando el hecho de que $\esp\left[N\left(t\right)\right]=\sum_{n=1}^{\infty}P\left\{N\left(t\right)\geq n\right\}$
se tiene que

\begin{eqnarray*}
\esp\left[N\left(t\right)\right]=\sum_{n=1}^{\infty}F^{n\star}\left(t\right)
\end{eqnarray*}

\begin{Prop}
Para cada $t\geq0$, la funci\'on generadora de momentos $\esp\left[e^{\alpha N\left(t\right)}\right]$ existe para alguna $\alpha$ en una vecindad del 0, y de aqu\'i que $\esp\left[N\left(t\right)^{m}\right]<\infty$, para $m\geq1$.
\end{Prop}


\begin{Note}
Si el primer tiempo de renovaci\'on $\xi_{1}$ no tiene la misma distribuci\'on que el resto de las $\xi_{n}$, para $n\geq2$, a $N\left(t\right)$ se le llama Proceso de Renovaci\'on retardado, donde si $\xi$ tiene distribuci\'on $G$, entonces el tiempo $T_{n}$ de la $n$-\'esima renovaci\'on tiene distribuci\'on $G\star F^{\left(n-1\right)\star}\left(t\right)$
\end{Note}


\begin{Teo}
Para una constante $\mu\leq\infty$ ( o variable aleatoria), las siguientes expresiones son equivalentes:

\begin{eqnarray}
lim_{n\rightarrow\infty}n^{-1}T_{n}&=&\mu,\textrm{ c.s.}\\
lim_{t\rightarrow\infty}t^{-1}N\left(t\right)&=&1/\mu,\textrm{ c.s.}
\end{eqnarray}
\end{Teo}


Es decir, $T_{n}$ satisface la Ley Fuerte de los Grandes N\'umeros s\'i y s\'olo s\'i $N\left/t\right)$ la cumple.


\begin{Coro}[Ley Fuerte de los Grandes N\'umeros para Procesos de Renovaci\'on]
Si $N\left(t\right)$ es un proceso de renovaci\'on cuyos tiempos de inter-renovaci\'on tienen media $\mu\leq\infty$, entonces
\begin{eqnarray}
t^{-1}N\left(t\right)\rightarrow 1/\mu,\textrm{ c.s. cuando }t\rightarrow\infty.
\end{eqnarray}

\end{Coro}


Considerar el proceso estoc\'astico de valores reales $\left\{Z\left(t\right):t\geq0\right\}$ en el mismo espacio de probabilidad que $N\left(t\right)$

\begin{Def}
Para el proceso $\left\{Z\left(t\right):t\geq0\right\}$ se define la fluctuaci\'on m\'axima de $Z\left(t\right)$ en el intervalo $\left(T_{n-1},T_{n}\right]$:
\begin{eqnarray*}
M_{n}=\sup_{T_{n-1}<t\leq T_{n}}|Z\left(t\right)-Z\left(T_{n-1}\right)|
\end{eqnarray*}
\end{Def}

\begin{Teo}
Sup\'ongase que $n^{-1}T_{n}\rightarrow\mu$ c.s. cuando $n\rightarrow\infty$, donde $\mu\leq\infty$ es una constante o variable aleatoria. Sea $a$ una constante o variable aleatoria que puede ser infinita cuando $\mu$ es finita, y considere las expresiones l\'imite:
\begin{eqnarray}
lim_{n\rightarrow\infty}n^{-1}Z\left(T_{n}\right)&=&a,\textrm{ c.s.}\\
lim_{t\rightarrow\infty}t^{-1}Z\left(t\right)&=&a/\mu,\textrm{ c.s.}
\end{eqnarray}
La segunda expresi\'on implica la primera. Conversamente, la primera implica la segunda si el proceso $Z\left(t\right)$ es creciente, o si $lim_{n\rightarrow\infty}n^{-1}M_{n}=0$ c.s.
\end{Teo}

\begin{Coro}
Si $N\left(t\right)$ es un proceso de renovaci\'on, y $\left(Z\left(T_{n}\right)-Z\left(T_{n-1}\right),M_{n}\right)$, para $n\geq1$, son variables aleatorias independientes e id\'enticamente distribuidas con media finita, entonces,
\begin{eqnarray}
lim_{t\rightarrow\infty}t^{-1}Z\left(t\right)\rightarrow\frac{\esp\left[Z\left(T_{1}\right)-Z\left(T_{0}\right)\right]}{\esp\left[T_{1}\right]},\textrm{ c.s. cuando  }t\rightarrow\infty.
\end{eqnarray}
\end{Coro}

%___________________________________________________________________________________________
%
%\subsection{Propiedades de los Procesos de Renovaci\'on}
%___________________________________________________________________________________________
%

Los tiempos $T_{n}$ est\'an relacionados con los conteos de $N\left(t\right)$ por

\begin{eqnarray*}
\left\{N\left(t\right)\geq n\right\}&=&\left\{T_{n}\leq t\right\}\\
T_{N\left(t\right)}\leq &t&<T_{N\left(t\right)+1},
\end{eqnarray*}

adem\'as $N\left(T_{n}\right)=n$, y 

\begin{eqnarray*}
N\left(t\right)=\max\left\{n:T_{n}\leq t\right\}=\min\left\{n:T_{n+1}>t\right\}
\end{eqnarray*}

Por propiedades de la convoluci\'on se sabe que

\begin{eqnarray*}
P\left\{T_{n}\leq t\right\}=F^{n\star}\left(t\right)
\end{eqnarray*}
que es la $n$-\'esima convoluci\'on de $F$. Entonces 

\begin{eqnarray*}
\left\{N\left(t\right)\geq n\right\}&=&\left\{T_{n}\leq t\right\}\\
P\left\{N\left(t\right)\leq n\right\}&=&1-F^{\left(n+1\right)\star}\left(t\right)
\end{eqnarray*}

Adem\'as usando el hecho de que $\esp\left[N\left(t\right)\right]=\sum_{n=1}^{\infty}P\left\{N\left(t\right)\geq n\right\}$
se tiene que

\begin{eqnarray*}
\esp\left[N\left(t\right)\right]=\sum_{n=1}^{\infty}F^{n\star}\left(t\right)
\end{eqnarray*}

\begin{Prop}
Para cada $t\geq0$, la funci\'on generadora de momentos $\esp\left[e^{\alpha N\left(t\right)}\right]$ existe para alguna $\alpha$ en una vecindad del 0, y de aqu\'i que $\esp\left[N\left(t\right)^{m}\right]<\infty$, para $m\geq1$.
\end{Prop}


\begin{Note}
Si el primer tiempo de renovaci\'on $\xi_{1}$ no tiene la misma distribuci\'on que el resto de las $\xi_{n}$, para $n\geq2$, a $N\left(t\right)$ se le llama Proceso de Renovaci\'on retardado, donde si $\xi$ tiene distribuci\'on $G$, entonces el tiempo $T_{n}$ de la $n$-\'esima renovaci\'on tiene distribuci\'on $G\star F^{\left(n-1\right)\star}\left(t\right)$
\end{Note}


\begin{Teo}
Para una constante $\mu\leq\infty$ ( o variable aleatoria), las siguientes expresiones son equivalentes:

\begin{eqnarray}
lim_{n\rightarrow\infty}n^{-1}T_{n}&=&\mu,\textrm{ c.s.}\\
lim_{t\rightarrow\infty}t^{-1}N\left(t\right)&=&1/\mu,\textrm{ c.s.}
\end{eqnarray}
\end{Teo}


Es decir, $T_{n}$ satisface la Ley Fuerte de los Grandes N\'umeros s\'i y s\'olo s\'i $N\left/t\right)$ la cumple.


\begin{Coro}[Ley Fuerte de los Grandes N\'umeros para Procesos de Renovaci\'on]
Si $N\left(t\right)$ es un proceso de renovaci\'on cuyos tiempos de inter-renovaci\'on tienen media $\mu\leq\infty$, entonces
\begin{eqnarray}
t^{-1}N\left(t\right)\rightarrow 1/\mu,\textrm{ c.s. cuando }t\rightarrow\infty.
\end{eqnarray}

\end{Coro}


Considerar el proceso estoc\'astico de valores reales $\left\{Z\left(t\right):t\geq0\right\}$ en el mismo espacio de probabilidad que $N\left(t\right)$

\begin{Def}
Para el proceso $\left\{Z\left(t\right):t\geq0\right\}$ se define la fluctuaci\'on m\'axima de $Z\left(t\right)$ en el intervalo $\left(T_{n-1},T_{n}\right]$:
\begin{eqnarray*}
M_{n}=\sup_{T_{n-1}<t\leq T_{n}}|Z\left(t\right)-Z\left(T_{n-1}\right)|
\end{eqnarray*}
\end{Def}

\begin{Teo}
Sup\'ongase que $n^{-1}T_{n}\rightarrow\mu$ c.s. cuando $n\rightarrow\infty$, donde $\mu\leq\infty$ es una constante o variable aleatoria. Sea $a$ una constante o variable aleatoria que puede ser infinita cuando $\mu$ es finita, y considere las expresiones l\'imite:
\begin{eqnarray}
lim_{n\rightarrow\infty}n^{-1}Z\left(T_{n}\right)&=&a,\textrm{ c.s.}\\
lim_{t\rightarrow\infty}t^{-1}Z\left(t\right)&=&a/\mu,\textrm{ c.s.}
\end{eqnarray}
La segunda expresi\'on implica la primera. Conversamente, la primera implica la segunda si el proceso $Z\left(t\right)$ es creciente, o si $lim_{n\rightarrow\infty}n^{-1}M_{n}=0$ c.s.
\end{Teo}

\begin{Coro}
Si $N\left(t\right)$ es un proceso de renovaci\'on, y $\left(Z\left(T_{n}\right)-Z\left(T_{n-1}\right),M_{n}\right)$, para $n\geq1$, son variables aleatorias independientes e id\'enticamente distribuidas con media finita, entonces,
\begin{eqnarray}
lim_{t\rightarrow\infty}t^{-1}Z\left(t\right)\rightarrow\frac{\esp\left[Z\left(T_{1}\right)-Z\left(T_{0}\right)\right]}{\esp\left[T_{1}\right]},\textrm{ c.s. cuando  }t\rightarrow\infty.
\end{eqnarray}
\end{Coro}
%___________________________________________________________________________________________
%
%\subsection{Propiedades de los Procesos de Renovaci\'on}
%___________________________________________________________________________________________
%

Los tiempos $T_{n}$ est\'an relacionados con los conteos de $N\left(t\right)$ por

\begin{eqnarray*}
\left\{N\left(t\right)\geq n\right\}&=&\left\{T_{n}\leq t\right\}\\
T_{N\left(t\right)}\leq &t&<T_{N\left(t\right)+1},
\end{eqnarray*}

adem\'as $N\left(T_{n}\right)=n$, y 

\begin{eqnarray*}
N\left(t\right)=\max\left\{n:T_{n}\leq t\right\}=\min\left\{n:T_{n+1}>t\right\}
\end{eqnarray*}

Por propiedades de la convoluci\'on se sabe que

\begin{eqnarray*}
P\left\{T_{n}\leq t\right\}=F^{n\star}\left(t\right)
\end{eqnarray*}
que es la $n$-\'esima convoluci\'on de $F$. Entonces 

\begin{eqnarray*}
\left\{N\left(t\right)\geq n\right\}&=&\left\{T_{n}\leq t\right\}\\
P\left\{N\left(t\right)\leq n\right\}&=&1-F^{\left(n+1\right)\star}\left(t\right)
\end{eqnarray*}

Adem\'as usando el hecho de que $\esp\left[N\left(t\right)\right]=\sum_{n=1}^{\infty}P\left\{N\left(t\right)\geq n\right\}$
se tiene que

\begin{eqnarray*}
\esp\left[N\left(t\right)\right]=\sum_{n=1}^{\infty}F^{n\star}\left(t\right)
\end{eqnarray*}

\begin{Prop}
Para cada $t\geq0$, la funci\'on generadora de momentos $\esp\left[e^{\alpha N\left(t\right)}\right]$ existe para alguna $\alpha$ en una vecindad del 0, y de aqu\'i que $\esp\left[N\left(t\right)^{m}\right]<\infty$, para $m\geq1$.
\end{Prop}


\begin{Note}
Si el primer tiempo de renovaci\'on $\xi_{1}$ no tiene la misma distribuci\'on que el resto de las $\xi_{n}$, para $n\geq2$, a $N\left(t\right)$ se le llama Proceso de Renovaci\'on retardado, donde si $\xi$ tiene distribuci\'on $G$, entonces el tiempo $T_{n}$ de la $n$-\'esima renovaci\'on tiene distribuci\'on $G\star F^{\left(n-1\right)\star}\left(t\right)$
\end{Note}


\begin{Teo}
Para una constante $\mu\leq\infty$ ( o variable aleatoria), las siguientes expresiones son equivalentes:

\begin{eqnarray}
lim_{n\rightarrow\infty}n^{-1}T_{n}&=&\mu,\textrm{ c.s.}\\
lim_{t\rightarrow\infty}t^{-1}N\left(t\right)&=&1/\mu,\textrm{ c.s.}
\end{eqnarray}
\end{Teo}


Es decir, $T_{n}$ satisface la Ley Fuerte de los Grandes N\'umeros s\'i y s\'olo s\'i $N\left/t\right)$ la cumple.


\begin{Coro}[Ley Fuerte de los Grandes N\'umeros para Procesos de Renovaci\'on]
Si $N\left(t\right)$ es un proceso de renovaci\'on cuyos tiempos de inter-renovaci\'on tienen media $\mu\leq\infty$, entonces
\begin{eqnarray}
t^{-1}N\left(t\right)\rightarrow 1/\mu,\textrm{ c.s. cuando }t\rightarrow\infty.
\end{eqnarray}

\end{Coro}


Considerar el proceso estoc\'astico de valores reales $\left\{Z\left(t\right):t\geq0\right\}$ en el mismo espacio de probabilidad que $N\left(t\right)$

\begin{Def}
Para el proceso $\left\{Z\left(t\right):t\geq0\right\}$ se define la fluctuaci\'on m\'axima de $Z\left(t\right)$ en el intervalo $\left(T_{n-1},T_{n}\right]$:
\begin{eqnarray*}
M_{n}=\sup_{T_{n-1}<t\leq T_{n}}|Z\left(t\right)-Z\left(T_{n-1}\right)|
\end{eqnarray*}
\end{Def}

\begin{Teo}
Sup\'ongase que $n^{-1}T_{n}\rightarrow\mu$ c.s. cuando $n\rightarrow\infty$, donde $\mu\leq\infty$ es una constante o variable aleatoria. Sea $a$ una constante o variable aleatoria que puede ser infinita cuando $\mu$ es finita, y considere las expresiones l\'imite:
\begin{eqnarray}
lim_{n\rightarrow\infty}n^{-1}Z\left(T_{n}\right)&=&a,\textrm{ c.s.}\\
lim_{t\rightarrow\infty}t^{-1}Z\left(t\right)&=&a/\mu,\textrm{ c.s.}
\end{eqnarray}
La segunda expresi\'on implica la primera. Conversamente, la primera implica la segunda si el proceso $Z\left(t\right)$ es creciente, o si $lim_{n\rightarrow\infty}n^{-1}M_{n}=0$ c.s.
\end{Teo}

\begin{Coro}
Si $N\left(t\right)$ es un proceso de renovaci\'on, y $\left(Z\left(T_{n}\right)-Z\left(T_{n-1}\right),M_{n}\right)$, para $n\geq1$, son variables aleatorias independientes e id\'enticamente distribuidas con media finita, entonces,
\begin{eqnarray}
lim_{t\rightarrow\infty}t^{-1}Z\left(t\right)\rightarrow\frac{\esp\left[Z\left(T_{1}\right)-Z\left(T_{0}\right)\right]}{\esp\left[T_{1}\right]},\textrm{ c.s. cuando  }t\rightarrow\infty.
\end{eqnarray}
\end{Coro}


%___________________________________________________________________________________________
%
%\subsection*{Funci\'on de Renovaci\'on}
%___________________________________________________________________________________________
%


\begin{Def}
Sea $h\left(t\right)$ funci\'on de valores reales en $\rea$ acotada en intervalos finitos e igual a cero para $t<0$ La ecuaci\'on de renovaci\'on para $h\left(t\right)$ y la distribuci\'on $F$ es

\begin{eqnarray}\label{Ec.Renovacion}
H\left(t\right)=h\left(t\right)+\int_{\left[0,t\right]}H\left(t-s\right)dF\left(s\right)\textrm{,    }t\geq0,
\end{eqnarray}
donde $H\left(t\right)$ es una funci\'on de valores reales. Esto es $H=h+F\star H$. Decimos que $H\left(t\right)$ es soluci\'on de esta ecuaci\'on si satisface la ecuaci\'on, y es acotada en intervalos finitos e iguales a cero para $t<0$.
\end{Def}

\begin{Prop}
La funci\'on $U\star h\left(t\right)$ es la \'unica soluci\'on de la ecuaci\'on de renovaci\'on (\ref{Ec.Renovacion}).
\end{Prop}

\begin{Teo}[Teorema Renovaci\'on Elemental]
\begin{eqnarray*}
t^{-1}U\left(t\right)\rightarrow 1/\mu\textrm{,    cuando }t\rightarrow\infty.
\end{eqnarray*}
\end{Teo}

%___________________________________________________________________________________________
%
%\subsection{Funci\'on de Renovaci\'on}
%___________________________________________________________________________________________
%


Sup\'ongase que $N\left(t\right)$ es un proceso de renovaci\'on con distribuci\'on $F$ con media finita $\mu$.

\begin{Def}
La funci\'on de renovaci\'on asociada con la distribuci\'on $F$, del proceso $N\left(t\right)$, es
\begin{eqnarray*}
U\left(t\right)=\sum_{n=1}^{\infty}F^{n\star}\left(t\right),\textrm{   }t\geq0,
\end{eqnarray*}
donde $F^{0\star}\left(t\right)=\indora\left(t\geq0\right)$.
\end{Def}


\begin{Prop}
Sup\'ongase que la distribuci\'on de inter-renovaci\'on $F$ tiene densidad $f$. Entonces $U\left(t\right)$ tambi\'en tiene densidad, para $t>0$, y es $U^{'}\left(t\right)=\sum_{n=0}^{\infty}f^{n\star}\left(t\right)$. Adem\'as
\begin{eqnarray*}
\prob\left\{N\left(t\right)>N\left(t-\right)\right\}=0\textrm{,   }t\geq0.
\end{eqnarray*}
\end{Prop}

\begin{Def}
La Transformada de Laplace-Stieljes de $F$ est\'a dada por

\begin{eqnarray*}
\hat{F}\left(\alpha\right)=\int_{\rea_{+}}e^{-\alpha t}dF\left(t\right)\textrm{,  }\alpha\geq0.
\end{eqnarray*}
\end{Def}

Entonces

\begin{eqnarray*}
\hat{U}\left(\alpha\right)=\sum_{n=0}^{\infty}\hat{F^{n\star}}\left(\alpha\right)=\sum_{n=0}^{\infty}\hat{F}\left(\alpha\right)^{n}=\frac{1}{1-\hat{F}\left(\alpha\right)}.
\end{eqnarray*}


\begin{Prop}
La Transformada de Laplace $\hat{U}\left(\alpha\right)$ y $\hat{F}\left(\alpha\right)$ determina una a la otra de manera \'unica por la relaci\'on $\hat{U}\left(\alpha\right)=\frac{1}{1-\hat{F}\left(\alpha\right)}$.
\end{Prop}


\begin{Note}
Un proceso de renovaci\'on $N\left(t\right)$ cuyos tiempos de inter-renovaci\'on tienen media finita, es un proceso Poisson con tasa $\lambda$ si y s\'olo s\'i $\esp\left[U\left(t\right)\right]=\lambda t$, para $t\geq0$.
\end{Note}


\begin{Teo}
Sea $N\left(t\right)$ un proceso puntual simple con puntos de localizaci\'on $T_{n}$ tal que $\eta\left(t\right)=\esp\left[N\left(\right)\right]$ es finita para cada $t$. Entonces para cualquier funci\'on $f:\rea_{+}\rightarrow\rea$,
\begin{eqnarray*}
\esp\left[\sum_{n=1}^{N\left(\right)}f\left(T_{n}\right)\right]=\int_{\left(0,t\right]}f\left(s\right)d\eta\left(s\right)\textrm{,  }t\geq0,
\end{eqnarray*}
suponiendo que la integral exista. Adem\'as si $X_{1},X_{2},\ldots$ son variables aleatorias definidas en el mismo espacio de probabilidad que el proceso $N\left(t\right)$ tal que $\esp\left[X_{n}|T_{n}=s\right]=f\left(s\right)$, independiente de $n$. Entonces
\begin{eqnarray*}
\esp\left[\sum_{n=1}^{N\left(t\right)}X_{n}\right]=\int_{\left(0,t\right]}f\left(s\right)d\eta\left(s\right)\textrm{,  }t\geq0,
\end{eqnarray*} 
suponiendo que la integral exista. 
\end{Teo}

\begin{Coro}[Identidad de Wald para Renovaciones]
Para el proceso de renovaci\'on $N\left(t\right)$,
\begin{eqnarray*}
\esp\left[T_{N\left(t\right)+1}\right]=\mu\esp\left[N\left(t\right)+1\right]\textrm{,  }t\geq0,
\end{eqnarray*}  
\end{Coro}

%______________________________________________________________________
%\subsection{Procesos de Renovaci\'on}
%______________________________________________________________________

\begin{Def}\label{Def.Tn}
Sean $0\leq T_{1}\leq T_{2}\leq \ldots$ son tiempos aleatorios infinitos en los cuales ocurren ciertos eventos. El n\'umero de tiempos $T_{n}$ en el intervalo $\left[0,t\right)$ es

\begin{eqnarray}
N\left(t\right)=\sum_{n=1}^{\infty}\indora\left(T_{n}\leq t\right),
\end{eqnarray}
para $t\geq0$.
\end{Def}

Si se consideran los puntos $T_{n}$ como elementos de $\rea_{+}$, y $N\left(t\right)$ es el n\'umero de puntos en $\rea$. El proceso denotado por $\left\{N\left(t\right):t\geq0\right\}$, denotado por $N\left(t\right)$, es un proceso puntual en $\rea_{+}$. Los $T_{n}$ son los tiempos de ocurrencia, el proceso puntual $N\left(t\right)$ es simple si su n\'umero de ocurrencias son distintas: $0<T_{1}<T_{2}<\ldots$ casi seguramente.

\begin{Def}
Un proceso puntual $N\left(t\right)$ es un proceso de renovaci\'on si los tiempos de interocurrencia $\xi_{n}=T_{n}-T_{n-1}$, para $n\geq1$, son independientes e identicamente distribuidos con distribuci\'on $F$, donde $F\left(0\right)=0$ y $T_{0}=0$. Los $T_{n}$ son llamados tiempos de renovaci\'on, referente a la independencia o renovaci\'on de la informaci\'on estoc\'astica en estos tiempos. Los $\xi_{n}$ son los tiempos de inter-renovaci\'on, y $N\left(t\right)$ es el n\'umero de renovaciones en el intervalo $\left[0,t\right)$
\end{Def}


\begin{Note}
Para definir un proceso de renovaci\'on para cualquier contexto, solamente hay que especificar una distribuci\'on $F$, con $F\left(0\right)=0$, para los tiempos de inter-renovaci\'on. La funci\'on $F$ en turno degune las otra variables aleatorias. De manera formal, existe un espacio de probabilidad y una sucesi\'on de variables aleatorias $\xi_{1},\xi_{2},\ldots$ definidas en este con distribuci\'on $F$. Entonces las otras cantidades son $T_{n}=\sum_{k=1}^{n}\xi_{k}$ y $N\left(t\right)=\sum_{n=1}^{\infty}\indora\left(T_{n}\leq t\right)$, donde $T_{n}\rightarrow\infty$ casi seguramente por la Ley Fuerte de los Grandes Números.
\end{Note}

\begin{Def}\label{Def.Tn}
Sean $0\leq T_{1}\leq T_{2}\leq \ldots$ son tiempos aleatorios infinitos en los cuales ocurren ciertos eventos. El n\'umero de tiempos $T_{n}$ en el intervalo $\left[0,t\right)$ es

\begin{eqnarray}
N\left(t\right)=\sum_{n=1}^{\infty}\indora\left(T_{n}\leq t\right),
\end{eqnarray}
para $t\geq0$.
\end{Def}

Si se consideran los puntos $T_{n}$ como elementos de $\rea_{+}$, y $N\left(t\right)$ es el n\'umero de puntos en $\rea$. El proceso denotado por $\left\{N\left(t\right):t\geq0\right\}$, denotado por $N\left(t\right)$, es un proceso puntual en $\rea_{+}$. Los $T_{n}$ son los tiempos de ocurrencia, el proceso puntual $N\left(t\right)$ es simple si su n\'umero de ocurrencias son distintas: $0<T_{1}<T_{2}<\ldots$ casi seguramente.

\begin{Def}
Un proceso puntual $N\left(t\right)$ es un proceso de renovaci\'on si los tiempos de interocurrencia $\xi_{n}=T_{n}-T_{n-1}$, para $n\geq1$, son independientes e identicamente distribuidos con distribuci\'on $F$, donde $F\left(0\right)=0$ y $T_{0}=0$. Los $T_{n}$ son llamados tiempos de renovaci\'on, referente a la independencia o renovaci\'on de la informaci\'on estoc\'astica en estos tiempos. Los $\xi_{n}$ son los tiempos de inter-renovaci\'on, y $N\left(t\right)$ es el n\'umero de renovaciones en el intervalo $\left[0,t\right)$
\end{Def}


\begin{Note}
Para definir un proceso de renovaci\'on para cualquier contexto, solamente hay que especificar una distribuci\'on $F$, con $F\left(0\right)=0$, para los tiempos de inter-renovaci\'on. La funci\'on $F$ en turno degune las otra variables aleatorias. De manera formal, existe un espacio de probabilidad y una sucesi\'on de variables aleatorias $\xi_{1},\xi_{2},\ldots$ definidas en este con distribuci\'on $F$. Entonces las otras cantidades son $T_{n}=\sum_{k=1}^{n}\xi_{k}$ y $N\left(t\right)=\sum_{n=1}^{\infty}\indora\left(T_{n}\leq t\right)$, donde $T_{n}\rightarrow\infty$ casi seguramente por la Ley Fuerte de los Grandes N\'umeros.
\end{Note}







Los tiempos $T_{n}$ est\'an relacionados con los conteos de $N\left(t\right)$ por

\begin{eqnarray*}
\left\{N\left(t\right)\geq n\right\}&=&\left\{T_{n}\leq t\right\}\\
T_{N\left(t\right)}\leq &t&<T_{N\left(t\right)+1},
\end{eqnarray*}

adem\'as $N\left(T_{n}\right)=n$, y 

\begin{eqnarray*}
N\left(t\right)=\max\left\{n:T_{n}\leq t\right\}=\min\left\{n:T_{n+1}>t\right\}
\end{eqnarray*}

Por propiedades de la convoluci\'on se sabe que

\begin{eqnarray*}
P\left\{T_{n}\leq t\right\}=F^{n\star}\left(t\right)
\end{eqnarray*}
que es la $n$-\'esima convoluci\'on de $F$. Entonces 

\begin{eqnarray*}
\left\{N\left(t\right)\geq n\right\}&=&\left\{T_{n}\leq t\right\}\\
P\left\{N\left(t\right)\leq n\right\}&=&1-F^{\left(n+1\right)\star}\left(t\right)
\end{eqnarray*}

Adem\'as usando el hecho de que $\esp\left[N\left(t\right)\right]=\sum_{n=1}^{\infty}P\left\{N\left(t\right)\geq n\right\}$
se tiene que

\begin{eqnarray*}
\esp\left[N\left(t\right)\right]=\sum_{n=1}^{\infty}F^{n\star}\left(t\right)
\end{eqnarray*}

\begin{Prop}
Para cada $t\geq0$, la funci\'on generadora de momentos $\esp\left[e^{\alpha N\left(t\right)}\right]$ existe para alguna $\alpha$ en una vecindad del 0, y de aqu\'i que $\esp\left[N\left(t\right)^{m}\right]<\infty$, para $m\geq1$.
\end{Prop}

\begin{Ejem}[\textbf{Proceso Poisson}]

Suponga que se tienen tiempos de inter-renovaci\'on \textit{i.i.d.} del proceso de renovaci\'on $N\left(t\right)$ tienen distribuci\'on exponencial $F\left(t\right)=q-e^{-\lambda t}$ con tasa $\lambda$. Entonces $N\left(t\right)$ es un proceso Poisson con tasa $\lambda$.

\end{Ejem}


\begin{Note}
Si el primer tiempo de renovaci\'on $\xi_{1}$ no tiene la misma distribuci\'on que el resto de las $\xi_{n}$, para $n\geq2$, a $N\left(t\right)$ se le llama Proceso de Renovaci\'on retardado, donde si $\xi$ tiene distribuci\'on $G$, entonces el tiempo $T_{n}$ de la $n$-\'esima renovaci\'on tiene distribuci\'on $G\star F^{\left(n-1\right)\star}\left(t\right)$
\end{Note}


\begin{Teo}
Para una constante $\mu\leq\infty$ ( o variable aleatoria), las siguientes expresiones son equivalentes:

\begin{eqnarray}
lim_{n\rightarrow\infty}n^{-1}T_{n}&=&\mu,\textrm{ c.s.}\\
lim_{t\rightarrow\infty}t^{-1}N\left(t\right)&=&1/\mu,\textrm{ c.s.}
\end{eqnarray}
\end{Teo}


Es decir, $T_{n}$ satisface la Ley Fuerte de los Grandes N\'umeros s\'i y s\'olo s\'i $N\left/t\right)$ la cumple.


\begin{Coro}[Ley Fuerte de los Grandes N\'umeros para Procesos de Renovaci\'on]
Si $N\left(t\right)$ es un proceso de renovaci\'on cuyos tiempos de inter-renovaci\'on tienen media $\mu\leq\infty$, entonces
\begin{eqnarray}
t^{-1}N\left(t\right)\rightarrow 1/\mu,\textrm{ c.s. cuando }t\rightarrow\infty.
\end{eqnarray}

\end{Coro}


Considerar el proceso estoc\'astico de valores reales $\left\{Z\left(t\right):t\geq0\right\}$ en el mismo espacio de probabilidad que $N\left(t\right)$

\begin{Def}
Para el proceso $\left\{Z\left(t\right):t\geq0\right\}$ se define la fluctuaci\'on m\'axima de $Z\left(t\right)$ en el intervalo $\left(T_{n-1},T_{n}\right]$:
\begin{eqnarray*}
M_{n}=\sup_{T_{n-1}<t\leq T_{n}}|Z\left(t\right)-Z\left(T_{n-1}\right)|
\end{eqnarray*}
\end{Def}

\begin{Teo}
Sup\'ongase que $n^{-1}T_{n}\rightarrow\mu$ c.s. cuando $n\rightarrow\infty$, donde $\mu\leq\infty$ es una constante o variable aleatoria. Sea $a$ una constante o variable aleatoria que puede ser infinita cuando $\mu$ es finita, y considere las expresiones l\'imite:
\begin{eqnarray}
lim_{n\rightarrow\infty}n^{-1}Z\left(T_{n}\right)&=&a,\textrm{ c.s.}\\
lim_{t\rightarrow\infty}t^{-1}Z\left(t\right)&=&a/\mu,\textrm{ c.s.}
\end{eqnarray}
La segunda expresi\'on implica la primera. Conversamente, la primera implica la segunda si el proceso $Z\left(t\right)$ es creciente, o si $lim_{n\rightarrow\infty}n^{-1}M_{n}=0$ c.s.
\end{Teo}

\begin{Coro}
Si $N\left(t\right)$ es un proceso de renovaci\'on, y $\left(Z\left(T_{n}\right)-Z\left(T_{n-1}\right),M_{n}\right)$, para $n\geq1$, son variables aleatorias independientes e id\'enticamente distribuidas con media finita, entonces,
\begin{eqnarray}
lim_{t\rightarrow\infty}t^{-1}Z\left(t\right)\rightarrow\frac{\esp\left[Z\left(T_{1}\right)-Z\left(T_{0}\right)\right]}{\esp\left[T_{1}\right]},\textrm{ c.s. cuando  }t\rightarrow\infty.
\end{eqnarray}
\end{Coro}


Sup\'ongase que $N\left(t\right)$ es un proceso de renovaci\'on con distribuci\'on $F$ con media finita $\mu$.

\begin{Def}
La funci\'on de renovaci\'on asociada con la distribuci\'on $F$, del proceso $N\left(t\right)$, es
\begin{eqnarray*}
U\left(t\right)=\sum_{n=1}^{\infty}F^{n\star}\left(t\right),\textrm{   }t\geq0,
\end{eqnarray*}
donde $F^{0\star}\left(t\right)=\indora\left(t\geq0\right)$.
\end{Def}


\begin{Prop}
Sup\'ongase que la distribuci\'on de inter-renovaci\'on $F$ tiene densidad $f$. Entonces $U\left(t\right)$ tambi\'en tiene densidad, para $t>0$, y es $U^{'}\left(t\right)=\sum_{n=0}^{\infty}f^{n\star}\left(t\right)$. Adem\'as
\begin{eqnarray*}
\prob\left\{N\left(t\right)>N\left(t-\right)\right\}=0\textrm{,   }t\geq0.
\end{eqnarray*}
\end{Prop}

\begin{Def}
La Transformada de Laplace-Stieljes de $F$ est\'a dada por

\begin{eqnarray*}
\hat{F}\left(\alpha\right)=\int_{\rea_{+}}e^{-\alpha t}dF\left(t\right)\textrm{,  }\alpha\geq0.
\end{eqnarray*}
\end{Def}

Entonces

\begin{eqnarray*}
\hat{U}\left(\alpha\right)=\sum_{n=0}^{\infty}\hat{F^{n\star}}\left(\alpha\right)=\sum_{n=0}^{\infty}\hat{F}\left(\alpha\right)^{n}=\frac{1}{1-\hat{F}\left(\alpha\right)}.
\end{eqnarray*}


\begin{Prop}
La Transformada de Laplace $\hat{U}\left(\alpha\right)$ y $\hat{F}\left(\alpha\right)$ determina una a la otra de manera \'unica por la relaci\'on $\hat{U}\left(\alpha\right)=\frac{1}{1-\hat{F}\left(\alpha\right)}$.
\end{Prop}


\begin{Note}
Un proceso de renovaci\'on $N\left(t\right)$ cuyos tiempos de inter-renovaci\'on tienen media finita, es un proceso Poisson con tasa $\lambda$ si y s\'olo s\'i $\esp\left[U\left(t\right)\right]=\lambda t$, para $t\geq0$.
\end{Note}


\begin{Teo}
Sea $N\left(t\right)$ un proceso puntual simple con puntos de localizaci\'on $T_{n}$ tal que $\eta\left(t\right)=\esp\left[N\left(\right)\right]$ es finita para cada $t$. Entonces para cualquier funci\'on $f:\rea_{+}\rightarrow\rea$,
\begin{eqnarray*}
\esp\left[\sum_{n=1}^{N\left(\right)}f\left(T_{n}\right)\right]=\int_{\left(0,t\right]}f\left(s\right)d\eta\left(s\right)\textrm{,  }t\geq0,
\end{eqnarray*}
suponiendo que la integral exista. Adem\'as si $X_{1},X_{2},\ldots$ son variables aleatorias definidas en el mismo espacio de probabilidad que el proceso $N\left(t\right)$ tal que $\esp\left[X_{n}|T_{n}=s\right]=f\left(s\right)$, independiente de $n$. Entonces
\begin{eqnarray*}
\esp\left[\sum_{n=1}^{N\left(t\right)}X_{n}\right]=\int_{\left(0,t\right]}f\left(s\right)d\eta\left(s\right)\textrm{,  }t\geq0,
\end{eqnarray*} 
suponiendo que la integral exista. 
\end{Teo}

\begin{Coro}[Identidad de Wald para Renovaciones]
Para el proceso de renovaci\'on $N\left(t\right)$,
\begin{eqnarray*}
\esp\left[T_{N\left(t\right)+1}\right]=\mu\esp\left[N\left(t\right)+1\right]\textrm{,  }t\geq0,
\end{eqnarray*}  
\end{Coro}


\begin{Def}
Sea $h\left(t\right)$ funci\'on de valores reales en $\rea$ acotada en intervalos finitos e igual a cero para $t<0$ La ecuaci\'on de renovaci\'on para $h\left(t\right)$ y la distribuci\'on $F$ es

\begin{eqnarray}\label{Ec.Renovacion}
H\left(t\right)=h\left(t\right)+\int_{\left[0,t\right]}H\left(t-s\right)dF\left(s\right)\textrm{,    }t\geq0,
\end{eqnarray}
donde $H\left(t\right)$ es una funci\'on de valores reales. Esto es $H=h+F\star H$. Decimos que $H\left(t\right)$ es soluci\'on de esta ecuaci\'on si satisface la ecuaci\'on, y es acotada en intervalos finitos e iguales a cero para $t<0$.
\end{Def}

\begin{Prop}
La funci\'on $U\star h\left(t\right)$ es la \'unica soluci\'on de la ecuaci\'on de renovaci\'on (\ref{Ec.Renovacion}).
\end{Prop}

\begin{Teo}[Teorema Renovaci\'on Elemental]
\begin{eqnarray*}
t^{-1}U\left(t\right)\rightarrow 1/\mu\textrm{,    cuando }t\rightarrow\infty.
\end{eqnarray*}
\end{Teo}



Sup\'ongase que $N\left(t\right)$ es un proceso de renovaci\'on con distribuci\'on $F$ con media finita $\mu$.

\begin{Def}
La funci\'on de renovaci\'on asociada con la distribuci\'on $F$, del proceso $N\left(t\right)$, es
\begin{eqnarray*}
U\left(t\right)=\sum_{n=1}^{\infty}F^{n\star}\left(t\right),\textrm{   }t\geq0,
\end{eqnarray*}
donde $F^{0\star}\left(t\right)=\indora\left(t\geq0\right)$.
\end{Def}


\begin{Prop}
Sup\'ongase que la distribuci\'on de inter-renovaci\'on $F$ tiene densidad $f$. Entonces $U\left(t\right)$ tambi\'en tiene densidad, para $t>0$, y es $U^{'}\left(t\right)=\sum_{n=0}^{\infty}f^{n\star}\left(t\right)$. Adem\'as
\begin{eqnarray*}
\prob\left\{N\left(t\right)>N\left(t-\right)\right\}=0\textrm{,   }t\geq0.
\end{eqnarray*}
\end{Prop}

\begin{Def}
La Transformada de Laplace-Stieljes de $F$ est\'a dada por

\begin{eqnarray*}
\hat{F}\left(\alpha\right)=\int_{\rea_{+}}e^{-\alpha t}dF\left(t\right)\textrm{,  }\alpha\geq0.
\end{eqnarray*}
\end{Def}

Entonces

\begin{eqnarray*}
\hat{U}\left(\alpha\right)=\sum_{n=0}^{\infty}\hat{F^{n\star}}\left(\alpha\right)=\sum_{n=0}^{\infty}\hat{F}\left(\alpha\right)^{n}=\frac{1}{1-\hat{F}\left(\alpha\right)}.
\end{eqnarray*}


\begin{Prop}
La Transformada de Laplace $\hat{U}\left(\alpha\right)$ y $\hat{F}\left(\alpha\right)$ determina una a la otra de manera \'unica por la relaci\'on $\hat{U}\left(\alpha\right)=\frac{1}{1-\hat{F}\left(\alpha\right)}$.
\end{Prop}


\begin{Note}
Un proceso de renovaci\'on $N\left(t\right)$ cuyos tiempos de inter-renovaci\'on tienen media finita, es un proceso Poisson con tasa $\lambda$ si y s\'olo s\'i $\esp\left[U\left(t\right)\right]=\lambda t$, para $t\geq0$.
\end{Note}


\begin{Teo}
Sea $N\left(t\right)$ un proceso puntual simple con puntos de localizaci\'on $T_{n}$ tal que $\eta\left(t\right)=\esp\left[N\left(\right)\right]$ es finita para cada $t$. Entonces para cualquier funci\'on $f:\rea_{+}\rightarrow\rea$,
\begin{eqnarray*}
\esp\left[\sum_{n=1}^{N\left(\right)}f\left(T_{n}\right)\right]=\int_{\left(0,t\right]}f\left(s\right)d\eta\left(s\right)\textrm{,  }t\geq0,
\end{eqnarray*}
suponiendo que la integral exista. Adem\'as si $X_{1},X_{2},\ldots$ son variables aleatorias definidas en el mismo espacio de probabilidad que el proceso $N\left(t\right)$ tal que $\esp\left[X_{n}|T_{n}=s\right]=f\left(s\right)$, independiente de $n$. Entonces
\begin{eqnarray*}
\esp\left[\sum_{n=1}^{N\left(t\right)}X_{n}\right]=\int_{\left(0,t\right]}f\left(s\right)d\eta\left(s\right)\textrm{,  }t\geq0,
\end{eqnarray*} 
suponiendo que la integral exista. 
\end{Teo}

\begin{Coro}[Identidad de Wald para Renovaciones]
Para el proceso de renovaci\'on $N\left(t\right)$,
\begin{eqnarray*}
\esp\left[T_{N\left(t\right)+1}\right]=\mu\esp\left[N\left(t\right)+1\right]\textrm{,  }t\geq0,
\end{eqnarray*}  
\end{Coro}


\begin{Def}
Sea $h\left(t\right)$ funci\'on de valores reales en $\rea$ acotada en intervalos finitos e igual a cero para $t<0$ La ecuaci\'on de renovaci\'on para $h\left(t\right)$ y la distribuci\'on $F$ es

\begin{eqnarray}\label{Ec.Renovacion}
H\left(t\right)=h\left(t\right)+\int_{\left[0,t\right]}H\left(t-s\right)dF\left(s\right)\textrm{,    }t\geq0,
\end{eqnarray}
donde $H\left(t\right)$ es una funci\'on de valores reales. Esto es $H=h+F\star H$. Decimos que $H\left(t\right)$ es soluci\'on de esta ecuaci\'on si satisface la ecuaci\'on, y es acotada en intervalos finitos e iguales a cero para $t<0$.
\end{Def}

\begin{Prop}
La funci\'on $U\star h\left(t\right)$ es la \'unica soluci\'on de la ecuaci\'on de renovaci\'on (\ref{Ec.Renovacion}).
\end{Prop}

\begin{Teo}[Teorema Renovaci\'on Elemental]
\begin{eqnarray*}
t^{-1}U\left(t\right)\rightarrow 1/\mu\textrm{,    cuando }t\rightarrow\infty.
\end{eqnarray*}
\end{Teo}


\begin{Note} Una funci\'on $h:\rea_{+}\rightarrow\rea$ es Directamente Riemann Integrable en los siguientes casos:
\begin{itemize}
\item[a)] $h\left(t\right)\geq0$ es decreciente y Riemann Integrable.
\item[b)] $h$ es continua excepto posiblemente en un conjunto de Lebesgue de medida 0, y $|h\left(t\right)|\leq b\left(t\right)$, donde $b$ es DRI.
\end{itemize}
\end{Note}

\begin{Teo}[Teorema Principal de Renovaci\'on]
Si $F$ es no aritm\'etica y $h\left(t\right)$ es Directamente Riemann Integrable (DRI), entonces

\begin{eqnarray*}
lim_{t\rightarrow\infty}U\star h=\frac{1}{\mu}\int_{\rea_{+}}h\left(s\right)ds.
\end{eqnarray*}
\end{Teo}

\begin{Prop}
Cualquier funci\'on $H\left(t\right)$ acotada en intervalos finitos y que es 0 para $t<0$ puede expresarse como
\begin{eqnarray*}
H\left(t\right)=U\star h\left(t\right)\textrm{,  donde }h\left(t\right)=H\left(t\right)-F\star H\left(t\right)
\end{eqnarray*}
\end{Prop}

\begin{Def}
Un proceso estoc\'astico $X\left(t\right)$ es crudamente regenerativo en un tiempo aleatorio positivo $T$ si
\begin{eqnarray*}
\esp\left[X\left(T+t\right)|T\right]=\esp\left[X\left(t\right)\right]\textrm{, para }t\geq0,\end{eqnarray*}
y con las esperanzas anteriores finitas.
\end{Def}

\begin{Prop}
Sup\'ongase que $X\left(t\right)$ es un proceso crudamente regenerativo en $T$, que tiene distribuci\'on $F$. Si $\esp\left[X\left(t\right)\right]$ es acotado en intervalos finitos, entonces
\begin{eqnarray*}
\esp\left[X\left(t\right)\right]=U\star h\left(t\right)\textrm{,  donde }h\left(t\right)=\esp\left[X\left(t\right)\indora\left(T>t\right)\right].
\end{eqnarray*}
\end{Prop}

\begin{Teo}[Regeneraci\'on Cruda]
Sup\'ongase que $X\left(t\right)$ es un proceso con valores positivo crudamente regenerativo en $T$, y def\'inase $M=\sup\left\{|X\left(t\right)|:t\leq T\right\}$. Si $T$ es no aritm\'etico y $M$ y $MT$ tienen media finita, entonces
\begin{eqnarray*}
lim_{t\rightarrow\infty}\esp\left[X\left(t\right)\right]=\frac{1}{\mu}\int_{\rea_{+}}h\left(s\right)ds,
\end{eqnarray*}
donde $h\left(t\right)=\esp\left[X\left(t\right)\indora\left(T>t\right)\right]$.
\end{Teo}


\begin{Note} Una funci\'on $h:\rea_{+}\rightarrow\rea$ es Directamente Riemann Integrable en los siguientes casos:
\begin{itemize}
\item[a)] $h\left(t\right)\geq0$ es decreciente y Riemann Integrable.
\item[b)] $h$ es continua excepto posiblemente en un conjunto de Lebesgue de medida 0, y $|h\left(t\right)|\leq b\left(t\right)$, donde $b$ es DRI.
\end{itemize}
\end{Note}

\begin{Teo}[Teorema Principal de Renovaci\'on]
Si $F$ es no aritm\'etica y $h\left(t\right)$ es Directamente Riemann Integrable (DRI), entonces

\begin{eqnarray*}
lim_{t\rightarrow\infty}U\star h=\frac{1}{\mu}\int_{\rea_{+}}h\left(s\right)ds.
\end{eqnarray*}
\end{Teo}

\begin{Prop}
Cualquier funci\'on $H\left(t\right)$ acotada en intervalos finitos y que es 0 para $t<0$ puede expresarse como
\begin{eqnarray*}
H\left(t\right)=U\star h\left(t\right)\textrm{,  donde }h\left(t\right)=H\left(t\right)-F\star H\left(t\right)
\end{eqnarray*}
\end{Prop}

\begin{Def}
Un proceso estoc\'astico $X\left(t\right)$ es crudamente regenerativo en un tiempo aleatorio positivo $T$ si
\begin{eqnarray*}
\esp\left[X\left(T+t\right)|T\right]=\esp\left[X\left(t\right)\right]\textrm{, para }t\geq0,\end{eqnarray*}
y con las esperanzas anteriores finitas.
\end{Def}

\begin{Prop}
Sup\'ongase que $X\left(t\right)$ es un proceso crudamente regenerativo en $T$, que tiene distribuci\'on $F$. Si $\esp\left[X\left(t\right)\right]$ es acotado en intervalos finitos, entonces
\begin{eqnarray*}
\esp\left[X\left(t\right)\right]=U\star h\left(t\right)\textrm{,  donde }h\left(t\right)=\esp\left[X\left(t\right)\indora\left(T>t\right)\right].
\end{eqnarray*}
\end{Prop}

\begin{Teo}[Regeneraci\'on Cruda]
Sup\'ongase que $X\left(t\right)$ es un proceso con valores positivo crudamente regenerativo en $T$, y def\'inase $M=\sup\left\{|X\left(t\right)|:t\leq T\right\}$. Si $T$ es no aritm\'etico y $M$ y $MT$ tienen media finita, entonces
\begin{eqnarray*}
lim_{t\rightarrow\infty}\esp\left[X\left(t\right)\right]=\frac{1}{\mu}\int_{\rea_{+}}h\left(s\right)ds,
\end{eqnarray*}
donde $h\left(t\right)=\esp\left[X\left(t\right)\indora\left(T>t\right)\right]$.
\end{Teo}

\begin{Def}
Para el proceso $\left\{\left(N\left(t\right),X\left(t\right)\right):t\geq0\right\}$, sus trayectoria muestrales en el intervalo de tiempo $\left[T_{n-1},T_{n}\right)$ est\'an descritas por
\begin{eqnarray*}
\zeta_{n}=\left(\xi_{n},\left\{X\left(T_{n-1}+t\right):0\leq t<\xi_{n}\right\}\right)
\end{eqnarray*}
Este $\zeta_{n}$ es el $n$-\'esimo segmento del proceso. El proceso es regenerativo sobre los tiempos $T_{n}$ si sus segmentos $\zeta_{n}$ son independientes e id\'enticamennte distribuidos.
\end{Def}


\begin{Note}
Si $\tilde{X}\left(t\right)$ con espacio de estados $\tilde{S}$ es regenerativo sobre $T_{n}$, entonces $X\left(t\right)=f\left(\tilde{X}\left(t\right)\right)$ tambi\'en es regenerativo sobre $T_{n}$, para cualquier funci\'on $f:\tilde{S}\rightarrow S$.
\end{Note}

\begin{Note}
Los procesos regenerativos son crudamente regenerativos, pero no al rev\'es.
\end{Note}


\begin{Note}
Un proceso estoc\'astico a tiempo continuo o discreto es regenerativo si existe un proceso de renovaci\'on  tal que los segmentos del proceso entre tiempos de renovaci\'on sucesivos son i.i.d., es decir, para $\left\{X\left(t\right):t\geq0\right\}$ proceso estoc\'astico a tiempo continuo con espacio de estados $S$, espacio m\'etrico.
\end{Note}

Para $\left\{X\left(t\right):t\geq0\right\}$ Proceso Estoc\'astico a tiempo continuo con estado de espacios $S$, que es un espacio m\'etrico, con trayectorias continuas por la derecha y con l\'imites por la izquierda c.s. Sea $N\left(t\right)$ un proceso de renovaci\'on en $\rea_{+}$ definido en el mismo espacio de probabilidad que $X\left(t\right)$, con tiempos de renovaci\'on $T$ y tiempos de inter-renovaci\'on $\xi_{n}=T_{n}-T_{n-1}$, con misma distribuci\'on $F$ de media finita $\mu$.



\begin{Def}
Para el proceso $\left\{\left(N\left(t\right),X\left(t\right)\right):t\geq0\right\}$, sus trayectoria muestrales en el intervalo de tiempo $\left[T_{n-1},T_{n}\right)$ est\'an descritas por
\begin{eqnarray*}
\zeta_{n}=\left(\xi_{n},\left\{X\left(T_{n-1}+t\right):0\leq t<\xi_{n}\right\}\right)
\end{eqnarray*}
Este $\zeta_{n}$ es el $n$-\'esimo segmento del proceso. El proceso es regenerativo sobre los tiempos $T_{n}$ si sus segmentos $\zeta_{n}$ son independientes e id\'enticamennte distribuidos.
\end{Def}

\begin{Note}
Un proceso regenerativo con media de la longitud de ciclo finita es llamado positivo recurrente.
\end{Note}

\begin{Teo}[Procesos Regenerativos]
Suponga que el proceso
\end{Teo}


\begin{Def}[Renewal Process Trinity]
Para un proceso de renovaci\'on $N\left(t\right)$, los siguientes procesos proveen de informaci\'on sobre los tiempos de renovaci\'on.
\begin{itemize}
\item $A\left(t\right)=t-T_{N\left(t\right)}$, el tiempo de recurrencia hacia atr\'as al tiempo $t$, que es el tiempo desde la \'ultima renovaci\'on para $t$.

\item $B\left(t\right)=T_{N\left(t\right)+1}-t$, el tiempo de recurrencia hacia adelante al tiempo $t$, residual del tiempo de renovaci\'on, que es el tiempo para la pr\'oxima renovaci\'on despu\'es de $t$.

\item $L\left(t\right)=\xi_{N\left(t\right)+1}=A\left(t\right)+B\left(t\right)$, la longitud del intervalo de renovaci\'on que contiene a $t$.
\end{itemize}
\end{Def}

\begin{Note}
El proceso tridimensional $\left(A\left(t\right),B\left(t\right),L\left(t\right)\right)$ es regenerativo sobre $T_{n}$, y por ende cada proceso lo es. Cada proceso $A\left(t\right)$ y $B\left(t\right)$ son procesos de MArkov a tiempo continuo con trayectorias continuas por partes en el espacio de estados $\rea_{+}$. Una expresi\'on conveniente para su distribuci\'on conjunta es, para $0\leq x<t,y\geq0$
\begin{equation}\label{NoRenovacion}
P\left\{A\left(t\right)>x,B\left(t\right)>y\right\}=
P\left\{N\left(t+y\right)-N\left((t-x)\right)=0\right\}
\end{equation}
\end{Note}

\begin{Ejem}[Tiempos de recurrencia Poisson]
Si $N\left(t\right)$ es un proceso Poisson con tasa $\lambda$, entonces de la expresi\'on (\ref{NoRenovacion}) se tiene que

\begin{eqnarray*}
\begin{array}{lc}
P\left\{A\left(t\right)>x,B\left(t\right)>y\right\}=e^{-\lambda\left(x+y\right)},&0\leq x<t,y\geq0,
\end{array}
\end{eqnarray*}
que es la probabilidad Poisson de no renovaciones en un intervalo de longitud $x+y$.

\end{Ejem}

\begin{Note}
Una cadena de Markov erg\'odica tiene la propiedad de ser estacionaria si la distribuci\'on de su estado al tiempo $0$ es su distribuci\'on estacionaria.
\end{Note}


\begin{Def}
Un proceso estoc\'astico a tiempo continuo $\left\{X\left(t\right):t\geq0\right\}$ en un espacio general es estacionario si sus distribuciones finito dimensionales son invariantes bajo cualquier  traslado: para cada $0\leq s_{1}<s_{2}<\cdots<s_{k}$ y $t\geq0$,
\begin{eqnarray*}
\left(X\left(s_{1}+t\right),\ldots,X\left(s_{k}+t\right)\right)=_{d}\left(X\left(s_{1}\right),\ldots,X\left(s_{k}\right)\right).
\end{eqnarray*}
\end{Def}

\begin{Note}
Un proceso de Markov es estacionario si $X\left(t\right)=_{d}X\left(0\right)$, $t\geq0$.
\end{Note}

Considerese el proceso $N\left(t\right)=\sum_{n}\indora\left(\tau_{n}\leq t\right)$ en $\rea_{+}$, con puntos $0<\tau_{1}<\tau_{2}<\cdots$.

\begin{Prop}
Si $N$ es un proceso puntual estacionario y $\esp\left[N\left(1\right)\right]<\infty$, entonces $\esp\left[N\left(t\right)\right]=t\esp\left[N\left(1\right)\right]$, $t\geq0$

\end{Prop}

\begin{Teo}
Los siguientes enunciados son equivalentes
\begin{itemize}
\item[i)] El proceso retardado de renovaci\'on $N$ es estacionario.

\item[ii)] EL proceso de tiempos de recurrencia hacia adelante $B\left(t\right)$ es estacionario.


\item[iii)] $\esp\left[N\left(t\right)\right]=t/\mu$,


\item[iv)] $G\left(t\right)=F_{e}\left(t\right)=\frac{1}{\mu}\int_{0}^{t}\left[1-F\left(s\right)\right]ds$
\end{itemize}
Cuando estos enunciados son ciertos, $P\left\{B\left(t\right)\leq x\right\}=F_{e}\left(x\right)$, para $t,x\geq0$.

\end{Teo}

\begin{Note}
Una consecuencia del teorema anterior es que el Proceso Poisson es el \'unico proceso sin retardo que es estacionario.
\end{Note}

\begin{Coro}
El proceso de renovaci\'on $N\left(t\right)$ sin retardo, y cuyos tiempos de inter renonaci\'on tienen media finita, es estacionario si y s\'olo si es un proceso Poisson.

\end{Coro}

%______________________________________________________________________

%\section{Ejemplos, Notas importantes}
%______________________________________________________________________
%\section*{Ap\'endice A}
%__________________________________________________________________

%________________________________________________________________________
%\subsection*{Procesos Regenerativos}
%________________________________________________________________________



\begin{Note}
Si $\tilde{X}\left(t\right)$ con espacio de estados $\tilde{S}$ es regenerativo sobre $T_{n}$, entonces $X\left(t\right)=f\left(\tilde{X}\left(t\right)\right)$ tambi\'en es regenerativo sobre $T_{n}$, para cualquier funci\'on $f:\tilde{S}\rightarrow S$.
\end{Note}

\begin{Note}
Los procesos regenerativos son crudamente regenerativos, pero no al rev\'es.
\end{Note}
%\subsection*{Procesos Regenerativos: Sigman\cite{Sigman1}}
\begin{Def}[Definici\'on Cl\'asica]
Un proceso estoc\'astico $X=\left\{X\left(t\right):t\geq0\right\}$ es llamado regenerativo is existe una variable aleatoria $R_{1}>0$ tal que
\begin{itemize}
\item[i)] $\left\{X\left(t+R_{1}\right):t\geq0\right\}$ es independiente de $\left\{\left\{X\left(t\right):t<R_{1}\right\},\right\}$
\item[ii)] $\left\{X\left(t+R_{1}\right):t\geq0\right\}$ es estoc\'asticamente equivalente a $\left\{X\left(t\right):t>0\right\}$
\end{itemize}

Llamamos a $R_{1}$ tiempo de regeneraci\'on, y decimos que $X$ se regenera en este punto.
\end{Def}

$\left\{X\left(t+R_{1}\right)\right\}$ es regenerativo con tiempo de regeneraci\'on $R_{2}$, independiente de $R_{1}$ pero con la misma distribuci\'on que $R_{1}$. Procediendo de esta manera se obtiene una secuencia de variables aleatorias independientes e id\'enticamente distribuidas $\left\{R_{n}\right\}$ llamados longitudes de ciclo. Si definimos a $Z_{k}\equiv R_{1}+R_{2}+\cdots+R_{k}$, se tiene un proceso de renovaci\'on llamado proceso de renovaci\'on encajado para $X$.




\begin{Def}
Para $x$ fijo y para cada $t\geq0$, sea $I_{x}\left(t\right)=1$ si $X\left(t\right)\leq x$,  $I_{x}\left(t\right)=0$ en caso contrario, y def\'inanse los tiempos promedio
\begin{eqnarray*}
\overline{X}&=&lim_{t\rightarrow\infty}\frac{1}{t}\int_{0}^{\infty}X\left(u\right)du\\
\prob\left(X_{\infty}\leq x\right)&=&lim_{t\rightarrow\infty}\frac{1}{t}\int_{0}^{\infty}I_{x}\left(u\right)du,
\end{eqnarray*}
cuando estos l\'imites existan.
\end{Def}

Como consecuencia del teorema de Renovaci\'on-Recompensa, se tiene que el primer l\'imite  existe y es igual a la constante
\begin{eqnarray*}
\overline{X}&=&\frac{\esp\left[\int_{0}^{R_{1}}X\left(t\right)dt\right]}{\esp\left[R_{1}\right]},
\end{eqnarray*}
suponiendo que ambas esperanzas son finitas.

\begin{Note}
\begin{itemize}
\item[a)] Si el proceso regenerativo $X$ es positivo recurrente y tiene trayectorias muestrales no negativas, entonces la ecuaci\'on anterior es v\'alida.
\item[b)] Si $X$ es positivo recurrente regenerativo, podemos construir una \'unica versi\'on estacionaria de este proceso, $X_{e}=\left\{X_{e}\left(t\right)\right\}$, donde $X_{e}$ es un proceso estoc\'astico regenerativo y estrictamente estacionario, con distribuci\'on marginal distribuida como $X_{\infty}$
\end{itemize}
\end{Note}

Para $\left\{X\left(t\right):t\geq0\right\}$ Proceso Estoc\'astico a tiempo continuo con estado de espacios $S$, que es un espacio m\'etrico, con trayectorias continuas por la derecha y con l\'imites por la izquierda c.s. Sea $N\left(t\right)$ un proceso de renovaci\'on en $\rea_{+}$ definido en el mismo espacio de probabilidad que $X\left(t\right)$, con tiempos de renovaci\'on $T$ y tiempos de inter-renovaci\'on $\xi_{n}=T_{n}-T_{n-1}$, con misma distribuci\'on $F$ de media finita $\mu$.


\begin{Def}
Para el proceso $\left\{\left(N\left(t\right),X\left(t\right)\right):t\geq0\right\}$, sus trayectoria muestrales en el intervalo de tiempo $\left[T_{n-1},T_{n}\right)$ est\'an descritas por
\begin{eqnarray*}
\zeta_{n}=\left(\xi_{n},\left\{X\left(T_{n-1}+t\right):0\leq t<\xi_{n}\right\}\right)
\end{eqnarray*}
Este $\zeta_{n}$ es el $n$-\'esimo segmento del proceso. El proceso es regenerativo sobre los tiempos $T_{n}$ si sus segmentos $\zeta_{n}$ son independientes e id\'enticamennte distribuidos.
\end{Def}


\begin{Note}
Si $\tilde{X}\left(t\right)$ con espacio de estados $\tilde{S}$ es regenerativo sobre $T_{n}$, entonces $X\left(t\right)=f\left(\tilde{X}\left(t\right)\right)$ tambi\'en es regenerativo sobre $T_{n}$, para cualquier funci\'on $f:\tilde{S}\rightarrow S$.
\end{Note}

\begin{Note}
Los procesos regenerativos son crudamente regenerativos, pero no al rev\'es.
\end{Note}

\begin{Def}[Definici\'on Cl\'asica]
Un proceso estoc\'astico $X=\left\{X\left(t\right):t\geq0\right\}$ es llamado regenerativo is existe una variable aleatoria $R_{1}>0$ tal que
\begin{itemize}
\item[i)] $\left\{X\left(t+R_{1}\right):t\geq0\right\}$ es independiente de $\left\{\left\{X\left(t\right):t<R_{1}\right\},\right\}$
\item[ii)] $\left\{X\left(t+R_{1}\right):t\geq0\right\}$ es estoc\'asticamente equivalente a $\left\{X\left(t\right):t>0\right\}$
\end{itemize}

Llamamos a $R_{1}$ tiempo de regeneraci\'on, y decimos que $X$ se regenera en este punto.
\end{Def}

$\left\{X\left(t+R_{1}\right)\right\}$ es regenerativo con tiempo de regeneraci\'on $R_{2}$, independiente de $R_{1}$ pero con la misma distribuci\'on que $R_{1}$. Procediendo de esta manera se obtiene una secuencia de variables aleatorias independientes e id\'enticamente distribuidas $\left\{R_{n}\right\}$ llamados longitudes de ciclo. Si definimos a $Z_{k}\equiv R_{1}+R_{2}+\cdots+R_{k}$, se tiene un proceso de renovaci\'on llamado proceso de renovaci\'on encajado para $X$.

\begin{Note}
Un proceso regenerativo con media de la longitud de ciclo finita es llamado positivo recurrente.
\end{Note}


\begin{Def}
Para $x$ fijo y para cada $t\geq0$, sea $I_{x}\left(t\right)=1$ si $X\left(t\right)\leq x$,  $I_{x}\left(t\right)=0$ en caso contrario, y def\'inanse los tiempos promedio
\begin{eqnarray*}
\overline{X}&=&lim_{t\rightarrow\infty}\frac{1}{t}\int_{0}^{\infty}X\left(u\right)du\\
\prob\left(X_{\infty}\leq x\right)&=&lim_{t\rightarrow\infty}\frac{1}{t}\int_{0}^{\infty}I_{x}\left(u\right)du,
\end{eqnarray*}
cuando estos l\'imites existan.
\end{Def}

Como consecuencia del teorema de Renovaci\'on-Recompensa, se tiene que el primer l\'imite  existe y es igual a la constante
\begin{eqnarray*}
\overline{X}&=&\frac{\esp\left[\int_{0}^{R_{1}}X\left(t\right)dt\right]}{\esp\left[R_{1}\right]},
\end{eqnarray*}
suponiendo que ambas esperanzas son finitas.

\begin{Note}
\begin{itemize}
\item[a)] Si el proceso regenerativo $X$ es positivo recurrente y tiene trayectorias muestrales no negativas, entonces la ecuaci\'on anterior es v\'alida.
\item[b)] Si $X$ es positivo recurrente regenerativo, podemos construir una \'unica versi\'on estacionaria de este proceso, $X_{e}=\left\{X_{e}\left(t\right)\right\}$, donde $X_{e}$ es un proceso estoc\'astico regenerativo y estrictamente estacionario, con distribuci\'on marginal distribuida como $X_{\infty}$
\end{itemize}
\end{Note}

%__________________________________________________________________________________________
%\subsection{Procesos Regenerativos Estacionarios - Stidham \cite{Stidham}}
%__________________________________________________________________________________________


Un proceso estoc\'astico a tiempo continuo $\left\{V\left(t\right),t\geq0\right\}$ es un proceso regenerativo si existe una sucesi\'on de variables aleatorias independientes e id\'enticamente distribuidas $\left\{X_{1},X_{2},\ldots\right\}$, sucesi\'on de renovaci\'on, tal que para cualquier conjunto de Borel $A$, 

\begin{eqnarray*}
\prob\left\{V\left(t\right)\in A|X_{1}+X_{2}+\cdots+X_{R\left(t\right)}=s,\left\{V\left(\tau\right),\tau<s\right\}\right\}=\prob\left\{V\left(t-s\right)\in A|X_{1}>t-s\right\},
\end{eqnarray*}
para todo $0\leq s\leq t$, donde $R\left(t\right)=\max\left\{X_{1}+X_{2}+\cdots+X_{j}\leq t\right\}=$n\'umero de renovaciones ({\emph{puntos de regeneraci\'on}}) que ocurren en $\left[0,t\right]$. El intervalo $\left[0,X_{1}\right)$ es llamado {\emph{primer ciclo de regeneraci\'on}} de $\left\{V\left(t \right),t\geq0\right\}$, $\left[X_{1},X_{1}+X_{2}\right)$ el {\emph{segundo ciclo de regeneraci\'on}}, y as\'i sucesivamente.

Sea $X=X_{1}$ y sea $F$ la funci\'on de distrbuci\'on de $X$


\begin{Def}
Se define el proceso estacionario, $\left\{V^{*}\left(t\right),t\geq0\right\}$, para $\left\{V\left(t\right),t\geq0\right\}$ por

\begin{eqnarray*}
\prob\left\{V\left(t\right)\in A\right\}=\frac{1}{\esp\left[X\right]}\int_{0}^{\infty}\prob\left\{V\left(t+x\right)\in A|X>x\right\}\left(1-F\left(x\right)\right)dx,
\end{eqnarray*} 
para todo $t\geq0$ y todo conjunto de Borel $A$.
\end{Def}

\begin{Def}
Una distribuci\'on se dice que es {\emph{aritm\'etica}} si todos sus puntos de incremento son m\'ultiplos de la forma $0,\lambda, 2\lambda,\ldots$ para alguna $\lambda>0$ entera.
\end{Def}


\begin{Def}
Una modificaci\'on medible de un proceso $\left\{V\left(t\right),t\geq0\right\}$, es una versi\'on de este, $\left\{V\left(t,w\right)\right\}$ conjuntamente medible para $t\geq0$ y para $w\in S$, $S$ espacio de estados para $\left\{V\left(t\right),t\geq0\right\}$.
\end{Def}

\begin{Teo}
Sea $\left\{V\left(t\right),t\geq\right\}$ un proceso regenerativo no negativo con modificaci\'on medible. Sea $\esp\left[X\right]<\infty$. Entonces el proceso estacionario dado por la ecuaci\'on anterior est\'a bien definido y tiene funci\'on de distribuci\'on independiente de $t$, adem\'as
\begin{itemize}
\item[i)] \begin{eqnarray*}
\esp\left[V^{*}\left(0\right)\right]&=&\frac{\esp\left[\int_{0}^{X}V\left(s\right)ds\right]}{\esp\left[X\right]}\end{eqnarray*}
\item[ii)] Si $\esp\left[V^{*}\left(0\right)\right]<\infty$, equivalentemente, si $\esp\left[\int_{0}^{X}V\left(s\right)ds\right]<\infty$,entonces
\begin{eqnarray*}
\frac{\int_{0}^{t}V\left(s\right)ds}{t}\rightarrow\frac{\esp\left[\int_{0}^{X}V\left(s\right)ds\right]}{\esp\left[X\right]}
\end{eqnarray*}
con probabilidad 1 y en media, cuando $t\rightarrow\infty$.
\end{itemize}
\end{Teo}
%
%___________________________________________________________________________________________
%\vspace{5.5cm}
%\chapter{Cadenas de Markov estacionarias}
%\vspace{-1.0cm}


%__________________________________________________________________________________________
%\subsection{Procesos Regenerativos Estacionarios - Stidham \cite{Stidham}}
%__________________________________________________________________________________________


Un proceso estoc\'astico a tiempo continuo $\left\{V\left(t\right),t\geq0\right\}$ es un proceso regenerativo si existe una sucesi\'on de variables aleatorias independientes e id\'enticamente distribuidas $\left\{X_{1},X_{2},\ldots\right\}$, sucesi\'on de renovaci\'on, tal que para cualquier conjunto de Borel $A$, 

\begin{eqnarray*}
\prob\left\{V\left(t\right)\in A|X_{1}+X_{2}+\cdots+X_{R\left(t\right)}=s,\left\{V\left(\tau\right),\tau<s\right\}\right\}=\prob\left\{V\left(t-s\right)\in A|X_{1}>t-s\right\},
\end{eqnarray*}
para todo $0\leq s\leq t$, donde $R\left(t\right)=\max\left\{X_{1}+X_{2}+\cdots+X_{j}\leq t\right\}=$n\'umero de renovaciones ({\emph{puntos de regeneraci\'on}}) que ocurren en $\left[0,t\right]$. El intervalo $\left[0,X_{1}\right)$ es llamado {\emph{primer ciclo de regeneraci\'on}} de $\left\{V\left(t \right),t\geq0\right\}$, $\left[X_{1},X_{1}+X_{2}\right)$ el {\emph{segundo ciclo de regeneraci\'on}}, y as\'i sucesivamente.

Sea $X=X_{1}$ y sea $F$ la funci\'on de distrbuci\'on de $X$


\begin{Def}
Se define el proceso estacionario, $\left\{V^{*}\left(t\right),t\geq0\right\}$, para $\left\{V\left(t\right),t\geq0\right\}$ por

\begin{eqnarray*}
\prob\left\{V\left(t\right)\in A\right\}=\frac{1}{\esp\left[X\right]}\int_{0}^{\infty}\prob\left\{V\left(t+x\right)\in A|X>x\right\}\left(1-F\left(x\right)\right)dx,
\end{eqnarray*} 
para todo $t\geq0$ y todo conjunto de Borel $A$.
\end{Def}

\begin{Def}
Una distribuci\'on se dice que es {\emph{aritm\'etica}} si todos sus puntos de incremento son m\'ultiplos de la forma $0,\lambda, 2\lambda,\ldots$ para alguna $\lambda>0$ entera.
\end{Def}


\begin{Def}
Una modificaci\'on medible de un proceso $\left\{V\left(t\right),t\geq0\right\}$, es una versi\'on de este, $\left\{V\left(t,w\right)\right\}$ conjuntamente medible para $t\geq0$ y para $w\in S$, $S$ espacio de estados para $\left\{V\left(t\right),t\geq0\right\}$.
\end{Def}

\begin{Teo}
Sea $\left\{V\left(t\right),t\geq\right\}$ un proceso regenerativo no negativo con modificaci\'on medible. Sea $\esp\left[X\right]<\infty$. Entonces el proceso estacionario dado por la ecuaci\'on anterior est\'a bien definido y tiene funci\'on de distribuci\'on independiente de $t$, adem\'as
\begin{itemize}
\item[i)] \begin{eqnarray*}
\esp\left[V^{*}\left(0\right)\right]&=&\frac{\esp\left[\int_{0}^{X}V\left(s\right)ds\right]}{\esp\left[X\right]}\end{eqnarray*}
\item[ii)] Si $\esp\left[V^{*}\left(0\right)\right]<\infty$, equivalentemente, si $\esp\left[\int_{0}^{X}V\left(s\right)ds\right]<\infty$,entonces
\begin{eqnarray*}
\frac{\int_{0}^{t}V\left(s\right)ds}{t}\rightarrow\frac{\esp\left[\int_{0}^{X}V\left(s\right)ds\right]}{\esp\left[X\right]}
\end{eqnarray*}
con probabilidad 1 y en media, cuando $t\rightarrow\infty$.
\end{itemize}
\end{Teo}

Sea la funci\'on generadora de momentos para $L_{i}$, el n\'umero de usuarios en la cola $Q_{i}\left(z\right)$ en cualquier momento, est\'a dada por el tiempo promedio de $z^{L_{i}\left(t\right)}$ sobre el ciclo regenerativo definido anteriormente. Entonces 



Es decir, es posible determinar las longitudes de las colas a cualquier tiempo $t$. Entonces, determinando el primer momento es posible ver que


\begin{Def}
El tiempo de Ciclo $C_{i}$ es el periodo de tiempo que comienza cuando la cola $i$ es visitada por primera vez en un ciclo, y termina cuando es visitado nuevamente en el pr\'oximo ciclo. La duraci\'on del mismo est\'a dada por $\tau_{i}\left(m+1\right)-\tau_{i}\left(m\right)$, o equivalentemente $\overline{\tau}_{i}\left(m+1\right)-\overline{\tau}_{i}\left(m\right)$ bajo condiciones de estabilidad.
\end{Def}


\begin{Def}
El tiempo de intervisita $I_{i}$ es el periodo de tiempo que comienza cuando se ha completado el servicio en un ciclo y termina cuando es visitada nuevamente en el pr\'oximo ciclo. Su  duraci\'on del mismo est\'a dada por $\tau_{i}\left(m+1\right)-\overline{\tau}_{i}\left(m\right)$.
\end{Def}

La duraci\'on del tiempo de intervisita es $\tau_{i}\left(m+1\right)-\overline{\tau}\left(m\right)$. Dado que el n\'umero de usuarios presentes en $Q_{i}$ al tiempo $t=\tau_{i}\left(m+1\right)$ es igual al n\'umero de arribos durante el intervalo de tiempo $\left[\overline{\tau}\left(m\right),\tau_{i}\left(m+1\right)\right]$ se tiene que


\begin{eqnarray*}
\esp\left[z_{i}^{L_{i}\left(\tau_{i}\left(m+1\right)\right)}\right]=\esp\left[\left\{P_{i}\left(z_{i}\right)\right\}^{\tau_{i}\left(m+1\right)-\overline{\tau}\left(m\right)}\right]
\end{eqnarray*}

entonces, si $I_{i}\left(z\right)=\esp\left[z^{\tau_{i}\left(m+1\right)-\overline{\tau}\left(m\right)}\right]$
se tiene que $F_{i}\left(z\right)=I_{i}\left[P_{i}\left(z\right)\right]$
para $i=1,2$.

Conforme a la definici\'on dada al principio del cap\'itulo, definici\'on (\ref{Def.Tn}), sean $T_{1},T_{2},\ldots$ los puntos donde las longitudes de las colas de la red de sistemas de visitas c\'iclicas son cero simult\'aneamente, cuando la cola $Q_{j}$ es visitada por el servidor para dar servicio, es decir, $L_{1}\left(T_{i}\right)=0,L_{2}\left(T_{i}\right)=0,\hat{L}_{1}\left(T_{i}\right)=0$ y $\hat{L}_{2}\left(T_{i}\right)=0$, a estos puntos se les denominar\'a puntos regenerativos. Entonces, 

\begin{Def}
Al intervalo de tiempo entre dos puntos regenerativos se le llamar\'a ciclo regenerativo.
\end{Def}

\begin{Def}
Para $T_{i}$ se define, $M_{i}$, el n\'umero de ciclos de visita a la cola $Q_{l}$, durante el ciclo regenerativo, es decir, $M_{i}$ es un proceso de renovaci\'on.
\end{Def}

\begin{Def}
Para cada uno de los $M_{i}$'s, se definen a su vez la duraci\'on de cada uno de estos ciclos de visita en el ciclo regenerativo, $C_{i}^{(m)}$, para $m=1,2,\ldots,M_{i}$, que a su vez, tambi\'en es n proceso de renovaci\'on.
\end{Def}

\footnote{In Stidham and  Heyman \cite{Stidham} shows that is sufficient for the regenerative process to be stationary that the mean regenerative cycle time is finite: $\esp\left[\sum_{m=1}^{M_{i}}C_{i}^{(m)}\right]<\infty$, 


 como cada $C_{i}^{(m)}$ contiene intervalos de r\'eplica positivos, se tiene que $\esp\left[M_{i}\right]<\infty$, adem\'as, como $M_{i}>0$, se tiene que la condici\'on anterior es equivalente a tener que $\esp\left[C_{i}\right]<\infty$,
por lo tanto una condici\'on suficiente para la existencia del proceso regenerativo est\'a dada por $\sum_{k=1}^{N}\mu_{k}<1.$}

Para $\left\{X\left(t\right):t\geq0\right\}$ Proceso Estoc\'astico a tiempo continuo con estado de espacios $S$, que es un espacio m\'etrico, con trayectorias continuas por la derecha y con l\'imites por la izquierda c.s. Sea $N\left(t\right)$ un proceso de renovaci\'on en $\rea_{+}$ definido en el mismo espacio de probabilidad que $X\left(t\right)$, con tiempos de renovaci\'on $T$ y tiempos de inter-renovaci\'on $\xi_{n}=T_{n}-T_{n-1}$, con misma distribuci\'on $F$ de media finita $\mu$.

\begin{Def}
Un elemento aleatorio en un espacio medible $\left(E,\mathcal{E}\right)$ en un espacio de probabilidad $\left(\Omega,\mathcal{F},\prob\right)$ a $\left(E,\mathcal{E}\right)$, es decir,
para $A\in \mathcal{E}$,  se tiene que $\left\{Y\in A\right\}\in\mathcal{F}$, donde $\left\{Y\in A\right\}:=\left\{w\in\Omega:Y\left(w\right)\in A\right\}=:Y^{-1}A$.
\end{Def}

\begin{Note}
Tambi\'en se dice que $Y$ est\'a soportado por el espacio de probabilidad $\left(\Omega,\mathcal{F},\prob\right)$ y que $Y$ es un mapeo medible de $\Omega$ en $E$, es decir, es $\mathcal{F}/\mathcal{E}$ medible.
\end{Note}

\begin{Def}
Para cada $i\in \mathbb{I}$ sea $P_{i}$ una medida de probabilidad en un espacio medible $\left(E_{i},\mathcal{E}_{i}\right)$. Se define el espacio producto
$\otimes_{i\in\mathbb{I}}\left(E_{i},\mathcal{E}_{i}\right):=\left(\prod_{i\in\mathbb{I}}E_{i},\otimes_{i\in\mathbb{I}}\mathcal{E}_{i}\right)$, donde $\prod_{i\in\mathbb{I}}E_{i}$ es el producto cartesiano de los $E_{i}$'s, y $\otimes_{i\in\mathbb{I}}\mathcal{E}_{i}$ es la $\sigma$-\'algebra producto, es decir, es la $\sigma$-\'algebra m\'as peque\~na en $\prod_{i\in\mathbb{I}}E_{i}$ que hace al $i$-\'esimo mapeo proyecci\'on en $E_{i}$ medible para toda $i\in\mathbb{I}$ es la $\sigma$-\'algebra inducida por los mapeos proyecci\'on. $$\otimes_{i\in\mathbb{I}}\mathcal{E}_{i}:=\sigma\left\{\left\{y:y_{i}\in A\right\}:i\in\mathbb{I}\textrm{ y }A\in\mathcal{E}_{i}\right\}.$$
\end{Def}

\begin{Def}
Un espacio de probabilidad $\left(\tilde{\Omega},\tilde{\mathcal{F}},\tilde{\prob}\right)$ es una extensi\'on de otro espacio de probabilidad $\left(\Omega,\mathcal{F},\prob\right)$ si $\left(\tilde{\Omega},\tilde{\mathcal{F}},\tilde{\prob}\right)$ soporta un elemento aleatorio $\xi\in\left(\Omega,\mathcal{F}\right)$ que tienen a $\prob$ como distribuci\'on.
\end{Def}

\begin{Teo}
Sea $\mathbb{I}$ un conjunto de \'indices arbitrario. Para cada $i\in\mathbb{I}$ sea $P_{i}$ una medida de probabilidad en un espacio medible $\left(E_{i},\mathcal{E}_{i}\right)$. Entonces existe una \'unica medida de probabilidad $\otimes_{i\in\mathbb{I}}P_{i}$ en $\otimes_{i\in\mathbb{I}}\left(E_{i},\mathcal{E}_{i}\right)$ tal que 

\begin{eqnarray*}
\otimes_{i\in\mathbb{I}}P_{i}\left(y\in\prod_{i\in\mathbb{I}}E_{i}:y_{i}\in A_{i_{1}},\ldots,y_{n}\in A_{i_{n}}\right)=P_{i_{1}}\left(A_{i_{n}}\right)\cdots P_{i_{n}}\left(A_{i_{n}}\right)
\end{eqnarray*}
para todos los enteros $n>0$, toda $i_{1},\ldots,i_{n}\in\mathbb{I}$ y todo $A_{i_{1}}\in\mathcal{E}_{i_{1}},\ldots,A_{i_{n}}\in\mathcal{E}_{i_{n}}$
\end{Teo}

La medida $\otimes_{i\in\mathbb{I}}P_{i}$ es llamada la medida producto y $\otimes_{i\in\mathbb{I}}\left(E_{i},\mathcal{E}_{i},P_{i}\right):=\left(\prod_{i\in\mathbb{I}},E_{i},\otimes_{i\in\mathbb{I}}\mathcal{E}_{i},\otimes_{i\in\mathbb{I}}P_{i}\right)$, es llamado espacio de probabilidad producto.


\begin{Def}
Un espacio medible $\left(E,\mathcal{E}\right)$ es \textit{Polaco} si existe una m\'etrica en $E$ tal que $E$ es completo, es decir cada sucesi\'on de Cauchy converge a un l\'imite en $E$, y \textit{separable}, $E$ tienen un subconjunto denso numerable, y tal que $\mathcal{E}$ es generado por conjuntos abiertos.
\end{Def}


\begin{Def}
Dos espacios medibles $\left(E,\mathcal{E}\right)$ y $\left(G,\mathcal{G}\right)$ son Borel equivalentes \textit{isomorfos} si existe una biyecci\'on $f:E\rightarrow G$ tal que $f$ es $\mathcal{E}/\mathcal{G}$ medible y su inversa $f^{-1}$ es $\mathcal{G}/\mathcal{E}$ medible. La biyecci\'on es una equivalencia de Borel.
\end{Def}

\begin{Def}
Un espacio medible  $\left(E,\mathcal{E}\right)$ es un \textit{espacio est\'andar} si es Borel equivalente a $\left(G,\mathcal{G}\right)$, donde $G$ es un subconjunto de Borel de $\left[0,1\right]$ y $\mathcal{G}$ son los subconjuntos de Borel de $G$.
\end{Def}

\begin{Note}
Cualquier espacio Polaco es un espacio est\'andar.
\end{Note}


\begin{Def}
Un proceso estoc\'astico con conjunto de \'indices $\mathbb{I}$ y espacio de estados $\left(E,\mathcal{E}\right)$ es una familia $Z=\left(\mathbb{Z}_{s}\right)_{s\in\mathbb{I}}$ donde $\mathbb{Z}_{s}$ son elementos aleatorios definidos en un espacio de probabilidad com\'un $\left(\Omega,\mathcal{F},\prob\right)$ y todos toman valores en $\left(E,\mathcal{E}\right)$.
\end{Def}

\begin{Def}
Un proceso estoc\'astico \textit{one-sided contiuous time} (\textbf{PEOSCT}) es un proceso estoc\'astico con conjunto de \'indices $\mathbb{I}=\left[0,\infty\right)$.
\end{Def}


Sea $\left(E^{\mathbb{I}},\mathcal{E}^{\mathbb{I}}\right)$ denota el espacio producto $\left(E^{\mathbb{I}},\mathcal{E}^{\mathbb{I}}\right):=\otimes_{s\in\mathbb{I}}\left(E,\mathcal{E}\right)$. Vamos a considerar $\mathbb{Z}$ como un mapeo aleatorio, es decir, como un elemento aleatorio en $\left(E^{\mathbb{I}},\mathcal{E}^{\mathbb{I}}\right)$ definido por $Z\left(w\right)=\left(Z_{s}\left(w\right)\right)_{s\in\mathbb{I}}$ y $w\in\Omega$.

\begin{Note}
La distribuci\'on de un proceso estoc\'astico $Z$ es la distribuci\'on de $Z$ como un elemento aleatorio en $\left(E^{\mathbb{I}},\mathcal{E}^{\mathbb{I}}\right)$. La distribuci\'on de $Z$ esta determinada de manera \'unica por las distribuciones finito dimensionales.
\end{Note}

\begin{Note}
En particular cuando $Z$ toma valores reales, es decir, $\left(E,\mathcal{E}\right)=\left(\mathbb{R},\mathcal{B}\right)$ las distribuciones finito dimensionales est\'an determinadas por las funciones de distribuci\'on finito dimensionales

\begin{eqnarray}
\prob\left(Z_{t_{1}}\leq x_{1},\ldots,Z_{t_{n}}\leq x_{n}\right),x_{1},\ldots,x_{n}\in\mathbb{R},t_{1},\ldots,t_{n}\in\mathbb{I},n\geq1.
\end{eqnarray}
\end{Note}

\begin{Note}
Para espacios polacos $\left(E,\mathcal{E}\right)$ el Teorema de Consistencia de Kolmogorov asegura que dada una colecci\'on de distribuciones finito dimensionales consistentes, siempre existe un proceso estoc\'astico que posee tales distribuciones finito dimensionales.
\end{Note}


\begin{Def}
Las trayectorias de $Z$ son las realizaciones $Z\left(w\right)$ para $w\in\Omega$ del mapeo aleatorio $Z$.
\end{Def}

\begin{Note}
Algunas restricciones se imponen sobre las trayectorias, por ejemplo que sean continuas por la derecha, o continuas por la derecha con l\'imites por la izquierda, o de manera m\'as general, se pedir\'a que caigan en alg\'un subconjunto $H$ de $E^{\mathbb{I}}$. En este caso es natural considerar a $Z$ como un elemento aleatorio que no est\'a en $\left(E^{\mathbb{I}},\mathcal{E}^{\mathbb{I}}\right)$ sino en $\left(H,\mathcal{H}\right)$, donde $\mathcal{H}$ es la $\sigma$-\'algebra generada por los mapeos proyecci\'on que toman a $z\in H$ a $z_{t}\in E$ para $t\in\mathbb{I}$. A $\mathcal{H}$ se le conoce como la traza de $H$ en $E^{\mathbb{I}}$, es decir,
\begin{eqnarray}
\mathcal{H}:=E^{\mathbb{I}}\cap H:=\left\{A\cap H:A\in E^{\mathbb{I}}\right\}.
\end{eqnarray}
\end{Note}


\begin{Note}
$Z$ tiene trayectorias con valores en $H$ y cada $Z_{t}$ es un mapeo medible de $\left(\Omega,\mathcal{F}\right)$ a $\left(H,\mathcal{H}\right)$. Cuando se considera un espacio de trayectorias en particular $H$, al espacio $\left(H,\mathcal{H}\right)$ se le llama el espacio de trayectorias de $Z$.
\end{Note}

\begin{Note}
La distribuci\'on del proceso estoc\'astico $Z$ con espacio de trayectorias $\left(H,\mathcal{H}\right)$ es la distribuci\'on de $Z$ como  un elemento aleatorio en $\left(H,\mathcal{H}\right)$. La distribuci\'on, nuevemente, est\'a determinada de manera \'unica por las distribuciones finito dimensionales.
\end{Note}


\begin{Def}
Sea $Z$ un PEOSCT  con espacio de estados $\left(E,\mathcal{E}\right)$ y sea $T$ un tiempo aleatorio en $\left[0,\infty\right)$. Por $Z_{T}$ se entiende el mapeo con valores en $E$ definido en $\Omega$ en la manera obvia:
\begin{eqnarray*}
Z_{T}\left(w\right):=Z_{T\left(w\right)}\left(w\right). w\in\Omega.
\end{eqnarray*}
\end{Def}

\begin{Def}
Un PEOSCT $Z$ es conjuntamente medible (\textbf{CM}) si el mapeo que toma $\left(w,t\right)\in\Omega\times\left[0,\infty\right)$ a $Z_{t}\left(w\right)\in E$ es $\mathcal{F}\otimes\mathcal{B}\left[0,\infty\right)/\mathcal{E}$ medible.
\end{Def}

\begin{Note}
Un PEOSCT-CM implica que el proceso es medible, dado que $Z_{T}$ es una composici\'on  de dos mapeos continuos: el primero que toma $w$ en $\left(w,T\left(w\right)\right)$ es $\mathcal{F}/\mathcal{F}\otimes\mathcal{B}\left[0,\infty\right)$ medible, mientras que el segundo toma $\left(w,T\left(w\right)\right)$ en $Z_{T\left(w\right)}\left(w\right)$ es $\mathcal{F}\otimes\mathcal{B}\left[0,\infty\right)/\mathcal{E}$ medible.
\end{Note}


\begin{Def}
Un PEOSCT con espacio de estados $\left(H,\mathcal{H}\right)$ es can\'onicamente conjuntamente medible (\textbf{CCM}) si el mapeo $\left(z,t\right)\in H\times\left[0,\infty\right)$ en $Z_{t}\in E$ es $\mathcal{H}\otimes\mathcal{B}\left[0,\infty\right)/\mathcal{E}$ medible.
\end{Def}

\begin{Note}
Un PEOSCTCCM implica que el proceso es CM, dado que un PECCM $Z$ es un mapeo de $\Omega\times\left[0,\infty\right)$ a $E$, es la composici\'on de dos mapeos medibles: el primero, toma $\left(w,t\right)$ en $\left(Z\left(w\right),t\right)$ es $\mathcal{F}\otimes\mathcal{B}\left[0,\infty\right)/\mathcal{H}\otimes\mathcal{B}\left[0,\infty\right)$ medible, y el segundo que toma $\left(Z\left(w\right),t\right)$  en $Z_{t}\left(w\right)$ es $\mathcal{H}\otimes\mathcal{B}\left[0,\infty\right)/\mathcal{E}$ medible. Por tanto CCM es una condici\'on m\'as fuerte que CM.
\end{Note}

\begin{Def}
Un conjunto de trayectorias $H$ de un PEOSCT $Z$, es internamente shift-invariante (\textbf{ISI}) si 
\begin{eqnarray*}
\left\{\left(z_{t+s}\right)_{s\in\left[0,\infty\right)}:z\in H\right\}=H\textrm{, }t\in\left[0,\infty\right).
\end{eqnarray*}
\end{Def}


\begin{Def}
Dado un PEOSCTISI, se define el mapeo-shift $\theta_{t}$, $t\in\left[0,\infty\right)$, de $H$ a $H$ por 
\begin{eqnarray*}
\theta_{t}z=\left(z_{t+s}\right)_{s\in\left[0,\infty\right)}\textrm{, }z\in H.
\end{eqnarray*}
\end{Def}

\begin{Def}
Se dice que un proceso $Z$ es shift-medible (\textbf{SM}) si $Z$ tiene un conjunto de trayectorias $H$ que es ISI y adem\'as el mapeo que toma $\left(z,t\right)\in H\times\left[0,\infty\right)$ en $\theta_{t}z\in H$ es $\mathcal{H}\otimes\mathcal{B}\left[0,\infty\right)/\mathcal{H}$ medible.
\end{Def}

\begin{Note}
Un proceso estoc\'astico con conjunto de trayectorias $H$ ISI es shift-medible si y s\'olo si es CCM
\end{Note}

\begin{Note}
\begin{itemize}
\item Dado el espacio polaco $\left(E,\mathcal{E}\right)$ se tiene el  conjunto de trayectorias $D_{E}\left[0,\infty\right)$ que es ISI, entonces cumpe con ser CCM.

\item Si $G$ es abierto, podemos cubrirlo por bolas abiertas cuay cerradura este contenida en $G$, y como $G$ es segundo numerable como subespacio de $E$, lo podemos cubrir por una cantidad numerable de bolas abiertas.

\end{itemize}
\end{Note}


\begin{Note}
Los procesos estoc\'asticos $Z$ a tiempo discreto con espacio de estados polaco, tambi\'en tiene un espacio de trayectorias polaco y por tanto tiene distribuciones condicionales regulares.
\end{Note}

\begin{Teo}
El producto numerable de espacios polacos es polaco.
\end{Teo}


\begin{Def}
Sea $\left(\Omega,\mathcal{F},\prob\right)$ espacio de probabilidad que soporta al proceso $Z=\left(Z_{s}\right)_{s\in\left[0,\infty\right)}$ y $S=\left(S_{k}\right)_{0}^{\infty}$ donde $Z$ es un PEOSCTM con espacio de estados $\left(E,\mathcal{E}\right)$  y espacio de trayectorias $\left(H,\mathcal{H}\right)$  y adem\'as $S$ es una sucesi\'on de tiempos aleatorios one-sided que satisfacen la condici\'on $0\leq S_{0}<S_{1}<\cdots\rightarrow\infty$. Considerando $S$ como un mapeo medible de $\left(\Omega,\mathcal{F}\right)$ al espacio sucesi\'on $\left(L,\mathcal{L}\right)$, donde 
\begin{eqnarray*}
L=\left\{\left(s_{k}\right)_{0}^{\infty}\in\left[0,\infty\right)^{\left\{0,1,\ldots\right\}}:s_{0}<s_{1}<\cdots\rightarrow\infty\right\},
\end{eqnarray*}
donde $\mathcal{L}$ son los subconjuntos de Borel de $L$, es decir, $\mathcal{L}=L\cap\mathcal{B}^{\left\{0,1,\ldots\right\}}$.

As\'i el par $\left(Z,S\right)$ es un mapeo medible de  $\left(\Omega,\mathcal{F}\right)$ en $\left(H\times L,\mathcal{H}\otimes\mathcal{L}\right)$. El par $\mathcal{H}\otimes\mathcal{L}^{+}$ denotar\'a la clase de todas las funciones medibles de $\left(H\times L,\mathcal{H}\otimes\mathcal{L}\right)$ en $\left(\left[0,\infty\right),\mathcal{B}\left[0,\infty\right)\right)$.
\end{Def}


\begin{Def}
Sea $\theta_{t}$ el mapeo-shift conjunto de $H\times L$ en $H\times L$ dado por
\begin{eqnarray*}
\theta_{t}\left(z,\left(s_{k}\right)_{0}^{\infty}\right)=\theta_{t}\left(z,\left(s_{n_{t-}+k}-t\right)_{0}^{\infty}\right)
\end{eqnarray*}
donde 
$n_{t-}=inf\left\{n\geq1:s_{n}\geq t\right\}$.
\end{Def}

\begin{Note}
Con la finalidad de poder realizar los shift's sin complicaciones de medibilidad, se supondr\'a que $Z$ es shit-medible, es decir, el conjunto de trayectorias $H$ es invariante bajo shifts del tiempo y el mapeo que toma $\left(z,t\right)\in H\times\left[0,\infty\right)$ en $z_{t}\in E$ es $\mathcal{H}\otimes\mathcal{B}\left[0,\infty\right)/\mathcal{E}$ medible.
\end{Note}

\begin{Def}
Dado un proceso \textbf{PEOSSM} (Proceso Estoc\'astico One Side Shift Medible) $Z$, se dice regenerativo cl\'asico con tiempos de regeneraci\'on $S$ si 

\begin{eqnarray*}
\theta_{S_{n}}\left(Z,S\right)=\left(Z^{0},S^{0}\right),n\geq0
\end{eqnarray*}
y adem\'as $\theta_{S_{n}}\left(Z,S\right)$ es independiente de $\left(\left(Z_{s}\right)s\in\left[0,S_{n}\right),S_{0},\ldots,S_{n}\right)$
Si lo anterior se cumple, al par $\left(Z,S\right)$ se le llama regenerativo cl\'asico.
\end{Def}

\begin{Note}
Si el par $\left(Z,S\right)$ es regenerativo cl\'asico, entonces las longitudes de los ciclos $X_{1},X_{2},\ldots,$ son i.i.d. e independientes de la longitud del retraso $S_{0}$, es decir, $S$ es un proceso de renovaci\'on. Las longitudes de los ciclos tambi\'en son llamados tiempos de inter-regeneraci\'on y tiempos de ocurrencia.

\end{Note}

\begin{Teo}
Sup\'ongase que el par $\left(Z,S\right)$ es regenerativo cl\'asico con $\esp\left[X_{1}\right]<\infty$. Entonces $\left(Z^{*},S^{*}\right)$ en el teorema 2.1 es una versi\'on estacionaria de $\left(Z,S\right)$. Adem\'as, si $X_{1}$ es lattice con span $d$, entonces $\left(Z^{**},S^{**}\right)$ en el teorema 2.2 es una versi\'on periodicamente estacionaria de $\left(Z,S\right)$ con periodo $d$.

\end{Teo}

\begin{Def}
Una variable aleatoria $X_{1}$ es \textit{spread out} si existe una $n\geq1$ y una  funci\'on $f\in\mathcal{B}^{+}$ tal que $\int_{\rea}f\left(x\right)dx>0$ con $X_{2},X_{3},\ldots,X_{n}$ copias i.i.d  de $X_{1}$, $$\prob\left(X_{1}+\cdots+X_{n}\in B\right)\geq\int_{B}f\left(x\right)dx$$ para $B\in\mathcal{B}$.

\end{Def}



\begin{Def}
Dado un proceso estoc\'astico $Z$ se le llama \textit{wide-sense regenerative} (\textbf{WSR}) con tiempos de regeneraci\'on $S$ si $\theta_{S_{n}}\left(Z,S\right)=\left(Z^{0},S^{0}\right)$ para $n\geq0$ en distribuci\'on y $\theta_{S_{n}}\left(Z,S\right)$ es independiente de $\left(S_{0},S_{1},\ldots,S_{n}\right)$ para $n\geq0$.
Se dice que el par $\left(Z,S\right)$ es WSR si lo anterior se cumple.
\end{Def}


\begin{Note}
\begin{itemize}
\item El proceso de trayectorias $\left(\theta_{s}Z\right)_{s\in\left[0,\infty\right)}$ es WSR con tiempos de regeneraci\'on $S$ pero no es regenerativo cl\'asico.

\item Si $Z$ es cualquier proceso estacionario y $S$ es un proceso de renovaci\'on que es independiente de $Z$, entonces $\left(Z,S\right)$ es WSR pero en general no es regenerativo cl\'asico

\end{itemize}

\end{Note}


\begin{Note}
Para cualquier proceso estoc\'astico $Z$, el proceso de trayectorias $\left(\theta_{s}Z\right)_{s\in\left[0,\infty\right)}$ es siempre un proceso de Markov.
\end{Note}



\begin{Teo}
Supongase que el par $\left(Z,S\right)$ es WSR con $\esp\left[X_{1}\right]<\infty$. Entonces $\left(Z^{*},S^{*}\right)$ en el teorema 2.1 es una versi\'on estacionaria de 
$\left(Z,S\right)$.
\end{Teo}


\begin{Teo}
Supongase que $\left(Z,S\right)$ es cycle-stationary con $\esp\left[X_{1}\right]<\infty$. Sea $U$ distribuida uniformemente en $\left[0,1\right)$ e independiente de $\left(Z^{0},S^{0}\right)$ y sea $\prob^{*}$ la medida de probabilidad en $\left(\Omega,\prob\right)$ definida por $$d\prob^{*}=\frac{X_{1}}{\esp\left[X_{1}\right]}d\prob$$. Sea $\left(Z^{*},S^{*}\right)$ con distribuci\'on $\prob^{*}\left(\theta_{UX_{1}}\left(Z^{0},S^{0}\right)\in\cdot\right)$. Entonces $\left(Z^{}*,S^{*}\right)$ es estacionario,
\begin{eqnarray*}
\esp\left[f\left(Z^{*},S^{*}\right)\right]=\esp\left[\int_{0}^{X_{1}}f\left(\theta_{s}\left(Z^{0},S^{0}\right)\right)ds\right]/\esp\left[X_{1}\right]
\end{eqnarray*}
$f\in\mathcal{H}\otimes\mathcal{L}^{+}$, and $S_{0}^{*}$ es continuo con funci\'on distribuci\'on $G_{\infty}$ definida por $$G_{\infty}\left(x\right):=\frac{\esp\left[X_{1}\right]\wedge x}{\esp\left[X_{1}\right]}$$ para $x\geq0$ y densidad $\prob\left[X_{1}>x\right]/\esp\left[X_{1}\right]$, con $x\geq0$.

\end{Teo}


\begin{Teo}
Sea $Z$ un Proceso Estoc\'astico un lado shift-medible \textit{one-sided shift-measurable stochastic process}, (PEOSSM),
y $S_{0}$ y $S_{1}$ tiempos aleatorios tales que $0\leq S_{0}<S_{1}$ y
\begin{equation}
\theta_{S_{1}}Z=\theta_{S_{0}}Z\textrm{ en distribuci\'on}.
\end{equation}

Entonces el espacio de probabilidad subyacente $\left(\Omega,\mathcal{F},\prob\right)$ puede extenderse para soportar una sucesi\'on de tiempos aleatorios $S$ tales que

\begin{eqnarray}
\theta_{S_{n}}\left(Z,S\right)=\left(Z^{0},S^{0}\right),n\geq0,\textrm{ en distribuci\'on},\\
\left(Z,S_{0},S_{1}\right)\textrm{ depende de }\left(X_{2},X_{3},\ldots\right)\textrm{ solamente a traves de }\theta_{S_{1}}Z.
\end{eqnarray}
\end{Teo}





%_________________________________________________________________________
%
%\subsection{Una vez que se tiene estabilidad}
%_________________________________________________________________________
%

Also the intervisit time $I_{i}$ is defined as the period beginning at the time of its service completion in a cycle and ending at the time when it is polled in the next cycle; its duration is given by $\tau_{i}\left(m+1\right)-\overline{\tau}_{i}\left(m\right)$.

So we the following are still true 

\begin{eqnarray}
\begin{array}{ll}
\esp\left[L_{i}\right]=\mu_{i}\esp\left[I_{i}\right], &
\esp\left[C_{i}\right]=\frac{f_{i}\left(i\right)}{\mu_{i}\left(1-\mu_{i}\right)},\\
\esp\left[S_{i}\right]=\mu_{i}\esp\left[C_{i}\right],&
\esp\left[I_{i}\right]=\left(1-\mu_{i}\right)\esp\left[C_{i}\right],\\
Var\left[L_{i}\right]= \mu_{i}^{2}Var\left[I_{i}\right]+\sigma^{2}\esp\left[I_{i}\right],& 
Var\left[C_{i}\right]=\frac{Var\left[L_{i}^{*}\right]}{\mu_{i}^{2}\left(1-\mu_{i}\right)^{2}},\\
Var\left[S_{i}\right]= \frac{Var\left[L_{i}^{*}\right]}{\left(1-\mu_{i}\right)^{2}}+\frac{\sigma^{2}\esp\left[L_{i}^{*}\right]}{\left(1-\mu_{i}\right)^{3}},&
Var\left[I_{i}\right]= \frac{Var\left[L_{i}^{*}\right]}{\mu_{i}^{2}}-\frac{\sigma_{i}^{2}}{\mu_{i}^{2}}f_{i}\left(i\right).
\end{array}
\end{eqnarray}
\begin{Def}
El tiempo de Ciclo $C_{i}$ es el periodo de tiempo que comienza cuando la cola $i$ es visitada por primera vez en un ciclo, y termina cuando es visitado nuevamente en el pr\'oximo ciclo. La duraci\'on del mismo est\'a dada por $\tau_{i}\left(m+1\right)-\tau_{i}\left(m\right)$, o equivalentemente $\overline{\tau}_{i}\left(m+1\right)-\overline{\tau}_{i}\left(m\right)$ bajo condiciones de estabilidad.
\end{Def}


\begin{Def}
El tiempo de intervisita $I_{i}$ es el periodo de tiempo que comienza cuando se ha completado el servicio en un ciclo y termina cuando es visitada nuevamente en el pr\'oximo ciclo. Su  duraci\'on del mismo est\'a dada por $\tau_{i}\left(m+1\right)-\overline{\tau}_{i}\left(m\right)$.
\end{Def}

La duraci\'on del tiempo de intervisita es $\tau_{i}\left(m+1\right)-\overline{\tau}\left(m\right)$. Dado que el n\'umero de usuarios presentes en $Q_{i}$ al tiempo $t=\tau_{i}\left(m+1\right)$ es igual al n\'umero de arribos durante el intervalo de tiempo $\left[\overline{\tau}\left(m\right),\tau_{i}\left(m+1\right)\right]$ se tiene que


\begin{eqnarray*}
\esp\left[z_{i}^{L_{i}\left(\tau_{i}\left(m+1\right)\right)}\right]=\esp\left[\left\{P_{i}\left(z_{i}\right)\right\}^{\tau_{i}\left(m+1\right)-\overline{\tau}\left(m\right)}\right]
\end{eqnarray*}

entonces, si $I_{i}\left(z\right)=\esp\left[z^{\tau_{i}\left(m+1\right)-\overline{\tau}\left(m\right)}\right]$
se tiene que $F_{i}\left(z\right)=I_{i}\left[P_{i}\left(z\right)\right]$
para $i=1,2$.

Conforme a la definici\'on dada al principio del cap\'itulo, definici\'on (\ref{Def.Tn}), sean $T_{1},T_{2},\ldots$ los puntos donde las longitudes de las colas de la red de sistemas de visitas c\'iclicas son cero simult\'aneamente, cuando la cola $Q_{j}$ es visitada por el servidor para dar servicio, es decir, $L_{1}\left(T_{i}\right)=0,L_{2}\left(T_{i}\right)=0,\hat{L}_{1}\left(T_{i}\right)=0$ y $\hat{L}_{2}\left(T_{i}\right)=0$, a estos puntos se les denominar\'a puntos regenerativos. Entonces, 

\begin{Def}
Al intervalo de tiempo entre dos puntos regenerativos se le llamar\'a ciclo regenerativo.
\end{Def}

\begin{Def}
Para $T_{i}$ se define, $M_{i}$, el n\'umero de ciclos de visita a la cola $Q_{l}$, durante el ciclo regenerativo, es decir, $M_{i}$ es un proceso de renovaci\'on.
\end{Def}

\begin{Def}
Para cada uno de los $M_{i}$'s, se definen a su vez la duraci\'on de cada uno de estos ciclos de visita en el ciclo regenerativo, $C_{i}^{(m)}$, para $m=1,2,\ldots,M_{i}$, que a su vez, tambi\'en es n proceso de renovaci\'on.
\end{Def}


Sea la funci\'on generadora de momentos para $L_{i}$, el n\'umero de usuarios en la cola $Q_{i}\left(z\right)$ en cualquier momento, est\'a dada por el tiempo promedio de $z^{L_{i}\left(t\right)}$ sobre el ciclo regenerativo definido anteriormente:

\begin{eqnarray*}
Q_{i}\left(z\right)&=&\esp\left[z^{L_{i}\left(t\right)}\right]=\frac{\esp\left[\sum_{m=1}^{M_{i}}\sum_{t=\tau_{i}\left(m\right)}^{\tau_{i}\left(m+1\right)-1}z^{L_{i}\left(t\right)}\right]}{\esp\left[\sum_{m=1}^{M_{i}}\tau_{i}\left(m+1\right)-\tau_{i}\left(m\right)\right]}
\end{eqnarray*}

$M_{i}$ es un tiempo de paro en el proceso regenerativo con $\esp\left[M_{i}\right]<\infty$\footnote{En Stidham\cite{Stidham} y Heyman  se muestra que una condici\'on suficiente para que el proceso regenerativo 
estacionario sea un procesoo estacionario es que el valor esperado del tiempo del ciclo regenerativo sea finito, es decir: $\esp\left[\sum_{m=1}^{M_{i}}C_{i}^{(m)}\right]<\infty$, como cada $C_{i}^{(m)}$ contiene intervalos de r\'eplica positivos, se tiene que $\esp\left[M_{i}\right]<\infty$, adem\'as, como $M_{i}>0$, se tiene que la condici\'on anterior es equivalente a tener que $\esp\left[C_{i}\right]<\infty$,
por lo tanto una condici\'on suficiente para la existencia del proceso regenerativo est\'a dada por $\sum_{k=1}^{N}\mu_{k}<1.$}, se sigue del lema de Wald que:


\begin{eqnarray*}
\esp\left[\sum_{m=1}^{M_{i}}\sum_{t=\tau_{i}\left(m\right)}^{\tau_{i}\left(m+1\right)-1}z^{L_{i}\left(t\right)}\right]&=&\esp\left[M_{i}\right]\esp\left[\sum_{t=\tau_{i}\left(m\right)}^{\tau_{i}\left(m+1\right)-1}z^{L_{i}\left(t\right)}\right]\\
\esp\left[\sum_{m=1}^{M_{i}}\tau_{i}\left(m+1\right)-\tau_{i}\left(m\right)\right]&=&\esp\left[M_{i}\right]\esp\left[\tau_{i}\left(m+1\right)-\tau_{i}\left(m\right)\right]
\end{eqnarray*}

por tanto se tiene que


\begin{eqnarray*}
Q_{i}\left(z\right)&=&\frac{\esp\left[\sum_{t=\tau_{i}\left(m\right)}^{\tau_{i}\left(m+1\right)-1}z^{L_{i}\left(t\right)}\right]}{\esp\left[\tau_{i}\left(m+1\right)-\tau_{i}\left(m\right)\right]}
\end{eqnarray*}

observar que el denominador es simplemente la duraci\'on promedio del tiempo del ciclo.


Haciendo las siguientes sustituciones en la ecuaci\'on (\ref{Corolario2}): $n\rightarrow t-\tau_{i}\left(m\right)$, $T \rightarrow \overline{\tau}_{i}\left(m\right)-\tau_{i}\left(m\right)$, $L_{n}\rightarrow L_{i}\left(t\right)$ y $F\left(z\right)=\esp\left[z^{L_{0}}\right]\rightarrow F_{i}\left(z\right)=\esp\left[z^{L_{i}\tau_{i}\left(m\right)}\right]$, se puede ver que

\begin{eqnarray}\label{Eq.Arribos.Primera}
\esp\left[\sum_{n=0}^{T-1}z^{L_{n}}\right]=
\esp\left[\sum_{t=\tau_{i}\left(m\right)}^{\overline{\tau}_{i}\left(m\right)-1}z^{L_{i}\left(t\right)}\right]
=z\frac{F_{i}\left(z\right)-1}{z-P_{i}\left(z\right)}
\end{eqnarray}

Por otra parte durante el tiempo de intervisita para la cola $i$, $L_{i}\left(t\right)$ solamente se incrementa de manera que el incremento por intervalo de tiempo est\'a dado por la funci\'on generadora de probabilidades de $P_{i}\left(z\right)$, por tanto la suma sobre el tiempo de intervisita puede evaluarse como:

\begin{eqnarray*}
\esp\left[\sum_{t=\tau_{i}\left(m\right)}^{\tau_{i}\left(m+1\right)-1}z^{L_{i}\left(t\right)}\right]&=&\esp\left[\sum_{t=\tau_{i}\left(m\right)}^{\tau_{i}\left(m+1\right)-1}\left\{P_{i}\left(z\right)\right\}^{t-\overline{\tau}_{i}\left(m\right)}\right]=\frac{1-\esp\left[\left\{P_{i}\left(z\right)\right\}^{\tau_{i}\left(m+1\right)-\overline{\tau}_{i}\left(m\right)}\right]}{1-P_{i}\left(z\right)}\\
&=&\frac{1-I_{i}\left[P_{i}\left(z\right)\right]}{1-P_{i}\left(z\right)}
\end{eqnarray*}
por tanto

\begin{eqnarray*}
\esp\left[\sum_{t=\tau_{i}\left(m\right)}^{\tau_{i}\left(m+1\right)-1}z^{L_{i}\left(t\right)}\right]&=&
\frac{1-F_{i}\left(z\right)}{1-P_{i}\left(z\right)}
\end{eqnarray*}

Por lo tanto

\begin{eqnarray*}
Q_{i}\left(z\right)&=&\frac{\esp\left[\sum_{t=\tau_{i}\left(m\right)}^{\tau_{i}\left(m+1\right)-1}z^{L_{i}\left(t\right)}\right]}{\esp\left[\tau_{i}\left(m+1\right)-\tau_{i}\left(m\right)\right]}
=\frac{1}{\esp\left[\tau_{i}\left(m+1\right)-\tau_{i}\left(m\right)\right]}
\esp\left[\sum_{t=\tau_{i}\left(m\right)}^{\tau_{i}\left(m+1\right)-1}z^{L_{i}\left(t\right)}\right]\\
&=&\frac{1}{\esp\left[\tau_{i}\left(m+1\right)-\tau_{i}\left(m\right)\right]}
\esp\left[\sum_{t=\tau_{i}\left(m\right)}^{\overline{\tau}_{i}\left(m\right)-1}z^{L_{i}\left(t\right)}
+\sum_{t=\overline{\tau}_{i}\left(m\right)}^{\tau_{i}\left(m+1\right)-1}z^{L_{i}\left(t\right)}\right]\\
&=&\frac{1}{\esp\left[\tau_{i}\left(m+1\right)-\tau_{i}\left(m\right)\right]}\left\{
\esp\left[\sum_{t=\tau_{i}\left(m\right)}^{\overline{\tau}_{i}\left(m\right)-1}z^{L_{i}\left(t\right)}\right]
+\esp\left[\sum_{t=\overline{\tau}_{i}\left(m\right)}^{\tau_{i}\left(m+1\right)-1}z^{L_{i}\left(t\right)}\right]\right\}\\
&=&\frac{1}{\esp\left[\tau_{i}\left(m+1\right)-\tau_{i}\left(m\right)\right]}\left\{
z\frac{F_{i}\left(z\right)-1}{z-P_{i}\left(z\right)}+\frac{1-F_{i}\left(z\right)}{1-P_{i}\left(z\right)}
\right\}\\
&=&\frac{1}{\esp\left[C_{i}\right]}\cdot\frac{1-F_{i}\left(z\right)}{P_{i}\left(z\right)-z}\cdot\frac{\left(1-z\right)P_{i}\left(z\right)}{1-P_{i}\left(z\right)}
\end{eqnarray*}

es decir

\begin{equation}
Q_{i}\left(z\right)=\frac{1}{\esp\left[C_{i}\right]}\cdot\frac{1-F_{i}\left(z\right)}{P_{i}\left(z\right)-z}\cdot\frac{\left(1-z\right)P_{i}\left(z\right)}{1-P_{i}\left(z\right)}
\end{equation}


Si hacemos:

\begin{eqnarray}
S\left(z\right)&=&1-F\left(z\right)\\
T\left(z\right)&=&z-P\left(z\right)\\
U\left(z\right)&=&1-P\left(z\right)
\end{eqnarray}
entonces 

\begin{eqnarray}
\esp\left[C_{i}\right]Q\left(z\right)=\frac{\left(z-1\right)S\left(z\right)P\left(z\right)}{T\left(z\right)U\left(z\right)}
\end{eqnarray}

A saber, si $a_{k}=P\left\{L\left(t\right)=k\right\}$
\begin{eqnarray*}
S\left(z\right)=1-F\left(z\right)=1-\sum_{k=0}^{+\infty}a_{k}z^{k}
\end{eqnarray*}
entonces

%\begin{eqnarray}
%\begin{array}{ll}
%S^{'}\left(z\right)=-\sum_{k=1}^{+\infty}ka_{k}z^{k-1},& %S^{(1)}\left(1\right)=-\sum_{k=1}^{+\infty}ka_{k}=-\esp\left[L\left(t\right)\right],\\
%S^{''}\left(z\right)=-\sum_{k=2}^{+\infty}k(k-1)a_{k}z^{k-2},& S^{(2)}\left(1\right)=-\sum_{k=2}^{+\infty}k(k-1)a_{k}=\esp\left[L\left(L-1\right)\right],\\
%S^{'''}\left(z\right)=-\sum_{k=3}^{+\infty}k(k-1)(k-2)a_{k}z^{k-3},&
%S^{(3)}\left(1\right)=-\sum_{k=3}^{+\infty}k(k-1)(k-2)a_{k}\\
%&=-\esp\left[L\left(L-1\right)\left(L-2\right)\right]\\
%&=-\esp\left[L^{3}\right]+3-\esp\left[L^{2}\right]-2-\esp\left[L\right];
%\end{array}
%\end{eqnarray}

$S^{'}\left(z\right)=-\sum_{k=1}^{+\infty}ka_{k}z^{k-1}$, por tanto $S^{(1)}\left(1\right)=-\sum_{k=1}^{+\infty}ka_{k}=-\esp\left[L\left(t\right)\right]$,
luego $S^{''}\left(z\right)=-\sum_{k=2}^{+\infty}k(k-1)a_{k}z^{k-2}$ y $S^{(2)}\left(1\right)=-\sum_{k=2}^{+\infty}k(k-1)a_{k}=\esp\left[L\left(L-1\right)\right]$;
de la misma manera $S^{'''}\left(z\right)=-\sum_{k=3}^{+\infty}k(k-1)(k-2)a_{k}z^{k-3}$ y $S^{(3)}\left(1\right)=-\sum_{k=3}^{+\infty}k(k-1)(k-2)a_{k}=-\esp\left[L\left(L-1\right)\left(L-2\right)\right]
=-\esp\left[L^{3}\right]+3-\esp\left[L^{2}\right]-2-\esp\left[L\right]$. 

Es decir

\begin{eqnarray*}
S^{(1)}\left(1\right)&=&-\esp\left[L\left(t\right)\right],\\ S^{(2)}\left(1\right)&=&-\esp\left[L\left(L-1\right)\right]
=-\esp\left[L^{2}\right]+\esp\left[L\right],\\
S^{(3)}\left(1\right)&=&-\esp\left[L\left(L-1\right)\left(L-2\right)\right]
=-\esp\left[L^{3}\right]+3\esp\left[L^{2}\right]-2\esp\left[L\right].
\end{eqnarray*}


Expandiendo alrededor de $z=1$

\begin{eqnarray*}
S\left(z\right)&=&S\left(1\right)+\frac{S^{'}\left(1\right)}{1!}\left(z-1\right)+\frac{S^{''}\left(1\right)}{2!}\left(z-1\right)^{2}+\frac{S^{'''}\left(1\right)}{3!}\left(z-1\right)^{3}+\ldots+\\
&=&\left(z-1\right)\left\{S^{'}\left(1\right)+\frac{S^{''}\left(1\right)}{2!}\left(z-1\right)+\frac{S^{'''}\left(1\right)}{3!}\left(z-1\right)^{2}+\ldots+\right\}\\
&=&\left(z-1\right)R_{1}\left(z\right)
\end{eqnarray*}
con $R_{1}\left(z\right)\neq0$, pues

\begin{eqnarray}\label{Eq.R1}
R_{1}\left(z\right)=-\esp\left[L\right]
\end{eqnarray}
entonces

\begin{eqnarray}
R_{1}\left(z\right)&=&S^{'}\left(1\right)+\frac{S^{''}\left(1\right)}{2!}\left(z-1\right)+\frac{S^{'''}\left(1\right)}{3!}\left(z-1\right)^{2}+\frac{S^{iv}\left(1\right)}{4!}\left(z-1\right)^{3}+\ldots+
\end{eqnarray}
Calculando las derivadas y evaluando en $z=1$

\begin{eqnarray}
R_{1}\left(1\right)&=&S^{(1)}\left(1\right)=-\esp\left[L\right]\\
R_{1}^{(1)}\left(1\right)&=&\frac{1}{2}S^{(2)}\left(1\right)=-\frac{1}{2}\esp\left[L^{2}\right]+\frac{1}{2}\esp\left[L\right]\\
R_{1}^{(2)}\left(1\right)&=&\frac{2}{3!}S^{(3)}\left(1\right)
=-\frac{1}{3}\esp\left[L^{3}\right]+\esp\left[L^{2}\right]-\frac{2}{3}\esp\left[L\right]
\end{eqnarray}

De manera an\'aloga se puede ver que para $T\left(z\right)=z-P\left(z\right)$ se puede encontrar una expanci\'on alrededor de $z=1$

Expandiendo alrededor de $z=1$

\begin{eqnarray*}
T\left(z\right)&=&T\left(1\right)+\frac{T^{'}\left(1\right)}{1!}\left(z-1\right)+\frac{T^{''}\left(1\right)}{2!}\left(z-1\right)^{2}+\frac{T^{'''}\left(1\right)}{3!}\left(z-1\right)^{3}+\ldots+\\
&=&\left(z-1\right)\left\{T^{'}\left(1\right)+\frac{T^{''}\left(1\right)}{2!}\left(z-1\right)+\frac{T^{'''}\left(1\right)}{3!}\left(z-1\right)^{2}+\ldots+\right\}\\
&=&\left(z-1\right)R_{2}\left(z\right)
\end{eqnarray*}

donde 
\begin{eqnarray*}
T^{(1)}\left(1\right)&=&-\esp\left[X\left(t\right)\right]=-\mu,\\ T^{(2)}\left(1\right)&=&-\esp\left[X\left(X-1\right)\right]
=-\esp\left[X^{2}\right]+\esp\left[X\right]=-\esp\left[X^{2}\right]+\mu,\\
T^{(3)}\left(1\right)&=&-\esp\left[X\left(X-1\right)\left(X-2\right)\right]
=-\esp\left[X^{3}\right]+3\esp\left[X^{2}\right]-2\esp\left[X\right]\\
&=&-\esp\left[X^{3}\right]+3\esp\left[X^{2}\right]-2\mu.
\end{eqnarray*}

Por lo tanto $R_{2}\left(1\right)\neq0$, pues

\begin{eqnarray}\label{Eq.R2}
R_{2}\left(1\right)=1-\esp\left[X\right]=1-\mu
\end{eqnarray}
entonces

\begin{eqnarray}
R_{2}\left(z\right)&=&T^{'}\left(1\right)+\frac{T^{''}\left(1\right)}{2!}\left(z-1\right)+\frac{T^{'''}\left(1\right)}{3!}\left(z-1\right)^{2}+\frac{T^{(iv)}\left(1\right)}{4!}\left(z-1\right)^{3}+\ldots+
\end{eqnarray}
Calculando las derivadas y evaluando en $z=1$

\begin{eqnarray}
R_{2}\left(1\right)&=&T^{(1)}\left(1\right)=1-\mu\\
R_{2}^{(1)}\left(1\right)&=&\frac{1}{2}T^{(2)}\left(1\right)=-\frac{1}{2}\esp\left[X^{2}\right]+\frac{1}{2}\mu\\
R_{2}^{(2)}\left(1\right)&=&\frac{2}{3!}T^{(3)}\left(1\right)
=-\frac{1}{3}\esp\left[X^{3}\right]+\esp\left[X^{2}\right]-\frac{2}{3}\mu
\end{eqnarray}

Finalmente para de manera an\'aloga se puede ver que para $U\left(z\right)=1-P\left(z\right)$ se puede encontrar una expanci\'on alrededor de $z=1$

\begin{eqnarray*}
U\left(z\right)&=&U\left(1\right)+\frac{U^{'}\left(1\right)}{1!}\left(z-1\right)+\frac{U^{''}\left(1\right)}{2!}\left(z-1\right)^{2}+\frac{U^{'''}\left(1\right)}{3!}\left(z-1\right)^{3}+\ldots+\\
&=&\left(z-1\right)\left\{U^{'}\left(1\right)+\frac{U^{''}\left(1\right)}{2!}\left(z-1\right)+\frac{U^{'''}\left(1\right)}{3!}\left(z-1\right)^{2}+\ldots+\right\}\\
&=&\left(z-1\right)R_{3}\left(z\right)
\end{eqnarray*}

donde 
\begin{eqnarray*}
U^{(1)}\left(1\right)&=&-\esp\left[X\left(t\right)\right]=-\mu,\\ U^{(2)}\left(1\right)&=&-\esp\left[X\left(X-1\right)\right]
=-\esp\left[X^{2}\right]+\esp\left[X\right]=-\esp\left[X^{2}\right]+\mu,\\
U^{(3)}\left(1\right)&=&-\esp\left[X\left(X-1\right)\left(X-2\right)\right]
=-\esp\left[X^{3}\right]+3\esp\left[X^{2}\right]-2\esp\left[X\right]\\
&=&-\esp\left[X^{3}\right]+3\esp\left[X^{2}\right]-2\mu.
\end{eqnarray*}

Por lo tanto $R_{3}\left(1\right)\neq0$, pues

\begin{eqnarray}\label{Eq.R2}
R_{3}\left(1\right)=-\esp\left[X\right]=-\mu
\end{eqnarray}
entonces

\begin{eqnarray}
R_{3}\left(z\right)&=&U^{'}\left(1\right)+\frac{U^{''}\left(1\right)}{2!}\left(z-1\right)+\frac{U^{'''}\left(1\right)}{3!}\left(z-1\right)^{2}+\frac{U^{(iv)}\left(1\right)}{4!}\left(z-1\right)^{3}+\ldots+
\end{eqnarray}

Calculando las derivadas y evaluando en $z=1$

\begin{eqnarray}
R_{3}\left(1\right)&=&U^{(1)}\left(1\right)=-\mu\\
R_{3}^{(1)}\left(1\right)&=&\frac{1}{2}U^{(2)}\left(1\right)=-\frac{1}{2}\esp\left[X^{2}\right]+\frac{1}{2}\mu\\
R_{3}^{(2)}\left(1\right)&=&\frac{2}{3!}U^{(3)}\left(1\right)
=-\frac{1}{3}\esp\left[X^{3}\right]+\esp\left[X^{2}\right]-\frac{2}{3}\mu
\end{eqnarray}

Por lo tanto

\begin{eqnarray}
\esp\left[C_{i}\right]Q\left(z\right)&=&\frac{\left(z-1\right)\left(z-1\right)R_{1}\left(z\right)P\left(z\right)}{\left(z-1\right)R_{2}\left(z\right)\left(z-1\right)R_{3}\left(z\right)}
=\frac{R_{1}\left(z\right)P\left(z\right)}{R_{2}\left(z\right)R_{3}\left(z\right)}\equiv\frac{R_{1}P}{R_{2}R_{3}}
\end{eqnarray}

Entonces

\begin{eqnarray}\label{Eq.Primer.Derivada.Q}
\left[\frac{R_{1}\left(z\right)P\left(z\right)}{R_{2}\left(z\right)R_{3}\left(z\right)}\right]^{'}&=&\frac{PR_{2}R_{3}R_{1}^{'}
+R_{1}R_{2}R_{3}P^{'}-R_{3}R_{1}PR_{2}-R_{2}R_{1}PR_{3}^{'}}{\left(R_{2}R_{3}\right)^{2}}
\end{eqnarray}
Evaluando en $z=1$
\begin{eqnarray*}
&=&\frac{R_{2}(1)R_{3}(1)R_{1}^{(1)}(1)+R_{1}(1)R_{2}(1)R_{3}(1)P^{'}(1)-R_{3}(1)R_{1}(1)R_{2}(1)^{(1)}-R_{2}(1)R_{1}(1)R_{3}^{'}(1)}{\left(R_{2}(1)R_{3}(1)\right)^{2}}\\
&=&\frac{1}{\left(1-\mu\right)^{2}\mu^{2}}\left\{\left(-\frac{1}{2}\esp L^{2}+\frac{1}{2}\esp L\right)\left(1-\mu\right)\left(-\mu\right)+\left(-\esp L\right)\left(1-\mu\right)\left(-\mu\right)\mu\right.\\
&&\left.-\left(-\frac{1}{2}\esp X^{2}+\frac{1}{2}\mu\right)\left(-\mu\right)\left(-\esp L\right)-\left(1-\mu\right)\left(-\esp L\right)\left(-\frac{1}{2}\esp X^{2}+\frac{1}{2}\mu\right)\right\}\\
&=&\frac{1}{\left(1-\mu\right)^{2}\mu^{2}}\left\{\left(-\frac{1}{2}\esp L^{2}+\frac{1}{2}\esp L\right)\left(\mu^{2}-\mu\right)
+\left(\mu^{2}-\mu^{3}\right)\esp L\right.\\
&&\left.-\mu\esp L\left(-\frac{1}{2}\esp X^{2}+\frac{1}{2}\mu\right)
+\left(\esp L-\mu\esp L\right)\left(-\frac{1}{2}\esp X^{2}+\frac{1}{2}\mu\right)\right\}\\
&=&\frac{1}{\left(1-\mu\right)^{2}\mu^{2}}\left\{-\frac{1}{2}\mu^{2}\esp L^{2}
+\frac{1}{2}\mu\esp L^{2}
+\frac{1}{2}\mu^{2}\esp L
-\mu^{3}\esp L
+\mu\esp L\esp X^{2}
-\frac{1}{2}\esp L\esp X^{2}\right\}\\
&=&\frac{1}{\left(1-\mu\right)^{2}\mu^{2}}\left\{
\frac{1}{2}\mu\esp L^{2}\left(1-\mu\right)
+\esp L\left(\frac{1}{2}-\mu\right)\left(\mu^{2}-\esp X^{2}\right)\right\}\\
&=&\frac{1}{2\mu\left(1-\mu\right)}\esp L^{2}-\frac{\frac{1}{2}-\mu}{\left(1-\mu\right)^{2}\mu^{2}}\sigma^{2}\esp L
\end{eqnarray*}

por lo tanto (para Takagi)

\begin{eqnarray*}
Q^{(1)}=\frac{1}{\esp C}\left\{\frac{1}{2\mu\left(1-\mu\right)}\esp L^{2}-\frac{\frac{1}{2}-\mu}{\left(1-\mu\right)^{2}\mu^{2}}\sigma^{2}\esp L\right\}
\end{eqnarray*}
donde 

\begin{eqnarray*}
\esp C = \frac{\esp L}{\mu\left(1-\mu\right)}
\end{eqnarray*}
entonces

\begin{eqnarray*}
Q^{(1)}&=&\frac{1}{2}\frac{\esp L^{2}}{\esp L}-\frac{\frac{1}{2}-\mu}{\left(1-\mu\right)\mu}\sigma^{2}
=\frac{\esp L^{2}}{2\esp L}-\frac{\sigma^{2}}{2}\left\{\frac{2\mu-1}{\left(1-\mu\right)\mu}\right\}\\
&=&\frac{\esp L^{2}}{2\esp L}+\frac{\sigma^{2}}{2}\left\{\frac{1}{1-\mu}+\frac{1}{\mu}\right\}
\end{eqnarray*}

Mientras que para nosotros

\begin{eqnarray*}
Q^{(1)}=\frac{1}{\mu\left(1-\mu\right)}\frac{\esp L^{2}}{2\esp C}
-\sigma^{2}\frac{\esp L}{2\esp C}\cdot\frac{1-2\mu}{\left(1-\mu\right)^{2}\mu^{2}}
\end{eqnarray*}

Retomando la ecuaci\'on (\ref{Eq.Primer.Derivada.Q})

\begin{eqnarray*}
\left[\frac{R_{1}\left(z\right)P\left(z\right)}{R_{2}\left(z\right)R_{3}\left(z\right)}\right]^{'}&=&\frac{PR_{2}R_{3}R_{1}^{'}
+R_{1}R_{2}R_{3}P^{'}-R_{3}R_{1}PR_{2}-R_{2}R_{1}PR_{3}^{'}}{\left(R_{2}R_{3}\right)^{2}}
=\frac{F\left(z\right)}{G\left(z\right)}
\end{eqnarray*}

donde 

\begin{eqnarray*}
F\left(z\right)&=&PR_{2}R_{3}R_{1}^{'}
+R_{1}R_{2}R_{3}P^{'}-R_{3}R_{1}PR_{2}^{'}-R_{2}R_{1}PR_{3}^{'}\\
G\left(z\right)&=&R_{2}^{2}R_{3}^{2}\\
G^{2}\left(z\right)&=&R_{2}^{4}R_{3}^{4}=\left(1-\mu\right)^{4}\mu^{4}
\end{eqnarray*}
y por tanto

\begin{eqnarray*}
G^{'}\left(z\right)&=&2R_{2}R_{3}\left[R_{2}^{'}R_{3}+R_{2}R_{3}^{'}\right]\\
G^{'}\left(1\right)&=&-2\left(1-\mu\right)\mu\left[\left(-\frac{1}{2}\esp\left[X^{2}\right]+\frac{1}{2}\mu\right)\left(-\mu\right)+\left(1-\mu\right)\left(-\frac{1}{2}\esp\left[X^{2}\right]+\frac{1}{2}\mu\right)\right]
\end{eqnarray*}


\begin{eqnarray*}
F^{'}\left(z\right)&=&\left[\left(R_{2}R_{3}\right)R_{1}^{''}
-\left(R_{1}R_{3}\right)R_{2}^{''}
-\left(R_{1}R_{2}\right)R_{3}^{''}
-2\left(R_{2}^{'}R_{3}^{'}\right)R_{1}\right]P
+2\left(R_{2}R_{3}\right)R_{1}^{'}P^{'}
+\left(R_{1}R_{2}R_{3}\right)P^{''}
\end{eqnarray*}

Por lo tanto, encontremos $F^{'}\left(z\right)G\left(z\right)+F\left(z\right)G^{'}\left(z\right)$:

\begin{eqnarray*}
&&F^{'}\left(z\right)G\left(z\right)+F\left(z\right)G^{'}\left(z\right)=
\left\{\left[\left(R_{2}R_{3}\right)R_{1}^{''}
-\left(R_{1}R_{3}\right)R_{2}^{''}
-\left(R_{1}R_{2}\right)R_{3}^{''}
-2\left(R_{2}^{'}R_{3}^{'}\right)R_{1}\right]P\right.\\
&&\left.+2\left(R_{2}R_{3}\right)R_{1}^{'}P^{'}
+\left(R_{1}R_{2}R_{3}\right)P^{''}\right\}R_{2}^{2}R_{3}^{2}
-\left\{\left[PR_{2}R_{3}R_{1}^{'}+R_{1}R_{2}R_{3}P^{'}
-R_{3}R_{1}PR_{2}^{'}\right.\right.\\
&&\left.\left.
-R_{2}R_{1}PR_{3}^{'}\right]\left[2R_{2}R_{3}\left(R_{2}^{'}R_{3}+R_{2}R_{3}^{'}\right)\right]\right\}
\end{eqnarray*}
Evaluando en $z=1$

\begin{eqnarray*}
&=&\left(1+R_{3}\right)^{3}R_{3}^{3}R_{1}^{''}-\left(1+R_{3}\right)^{2}R_{1}R_{3}^{3}R_{3}^{''}
-\left(1+R_{3}\right)^{3}R_{3}^{2}R_{1}R_{3}^{''}-2\left(1+R_{3}\right)^{2}R_{3}^{2}
\left(R_{3}^{'}\right)^{2}\\
&+&2\left(1+R_{3}\right)^{3}R_{3}^{3}R_{1}^{'}P^{'}
+\left(1+R_{3}\right)^{3}R_{3}^{3}R_{1}P^{''}
-2\left(1+R_{3}\right)^{2}R_{3}^{2}\left(1+2R_{3}\right)R_{3}^{'}R_{1}^{'}\\
&-&2\left(1+R_{3}\right)^{2}R_{3}^{2}R_{1}R_{3}^{'}\left(1+2R_{3}\right)P^{'}
+2\left(1+R_{3}\right)\left(1+2R_{3}\right)R_{3}^{3}R_{1}\left(R_{3}^{'}\right)^{2}\\
&+&2\left(1+R_{3}\right)^{2}\left(1+2R_{3}\right)R_{1}R_{3}R_{3}^{'}\\
&=&-\left(1-\mu\right)^{3}\mu^{3}R_{1}^{''}-\left(1-\mu\right)^{2}\mu^{2}R_{1}\left(1-2\mu\right)R_{3}^{''}
-\left(1-\mu\right)^{3}\mu^{3}R_{1}P^{''}\\
&+&2\left(1-\mu\right)\mu^{2}\left[\left(1-2\mu\right)R_{1}-\left(1-\mu\right)\right]\left(R_{3}^{'}\right)^{2}
-2\left(1-\mu\right)^{2}\mu R_{1}\left(1-2\mu\right)R_{3}^{'}\\
&-&2\left(1-\mu\right)^{3}\mu^{4}R_{1}^{'}-2\mu\left(1-\mu\right)\left(1-2\mu\right)R_{3}^{'}R_{1}^{'}
-2\mu^{3}\left(1-\mu\right)^{2}\left(1-2\mu\right)R_{1}R_{1}^{'}
\end{eqnarray*}

por tanto

\begin{eqnarray*}
\left[\frac{F\left(z\right)}{G\left(z\right)}\right]^{'}&=&\frac{1}{\mu^{3}\left(1-\mu\right)^{3}}\left\{
-\left(1-\mu\right)^{2}\mu^{2}R_{1}^{''}-\mu\left(1-\mu\right)\left(1-2\mu\right)R_{1}R_{3}^{''}
-\mu^{2}\left(1-\mu\right)^{2}R_{1}P^{''}\right.\\
&&\left.+2\mu\left[\left(1-2\mu\right)R_{1}-\left(1-\mu\right)\right]\left(R_{3}^{'}\right)^{2}
-2\left(1-\mu\right)\left(1-2\mu\right)R_{1}R_{3}^{'}-2\mu^{3}\left(1-\mu\right)^{2}R_{1}^{'}\right.\\
&&\left.-2\left(1-2\mu\right)R_{3}^{'}R_{1}^{'}-2\mu^{2}\left(1-\mu\right)\left(1-2\mu\right)R_{1}R_{1}^{'}\right\}
\end{eqnarray*}

recordemos que


\begin{eqnarray*}
R_{1}&=&-\esp L\\
R_{3}&=& -\mu\\
R_{1}^{'}&=&-\frac{1}{2}\esp L^{2}+\frac{1}{2}\esp L\\
R_{3}^{'}&=&-\frac{1}{2}\esp X^{2}+\frac{1}{2}\mu\\
R_{1}^{''}&=&-\frac{1}{3}\esp L^{3}+\esp L^{2}-\frac{2}{3}\esp L\\
R_{3}^{''}&=&-\frac{1}{3}\esp X^{3}+\esp X^{2}-\frac{2}{3}\mu\\
R_{1}R_{3}^{'}&=&\frac{1}{2}\esp X^{2}\esp L-\frac{1}{2}\esp X\esp L\\
R_{1}R_{1}^{'}&=&\frac{1}{2}\esp L^{2}\esp L+\frac{1}{2}\esp^{2}L\\
R_{3}^{'}R_{1}^{'}&=&\frac{1}{4}\esp X^{2}\esp L^{2}-\frac{1}{4}\esp X^{2}\esp L-\frac{1}{4}\esp L^{2}\esp X+\frac{1}{4}\esp X\esp L\\
R_{1}R_{3}^{''}&=&\frac{1}{6}\esp X^{3}\esp L^{2}-\frac{1}{6}\esp X^{3}\esp L-\frac{1}{2}\esp L^{2}\esp X^{2}+\frac{1}{2}\esp X^{2}\esp L+\frac{1}{3}\esp X\esp L^{2}-\frac{1}{3}\esp X\esp L\\
R_{1}P^{''}&=&-\esp X^{2}\esp L\\
\left(R_{3}^{'}\right)^{2}&=&\frac{1}{4}\esp^{2}X^{2}-\frac{1}{2}\esp X^{2}\esp X+\frac{1}{4}\esp^{2} X
\end{eqnarray*}




\begin{Def}
Let $L_{i}^{*}$ be the number of users at queue $Q_{i}$ when it is polled, then
\begin{eqnarray}
\begin{array}{cc}
\esp\left[L_{i}^{*}\right]=f_{i}\left(i\right), &
Var\left[L_{i}^{*}\right]=f_{i}\left(i,i\right)+\esp\left[L_{i}^{*}\right]-\esp\left[L_{i}^{*}\right]^{2}.
\end{array}
\end{eqnarray}
\end{Def}

\begin{Def}
The cycle time $C_{i}$ for the queue $Q_{i}$ is the period beginning at the time when it is polled in a cycle and ending at the time when it is polled in the next cycle; it's duration is given by $\tau_{i}\left(m+1\right)-\tau_{i}\left(m\right)$, equivalently $\overline{\tau}_{i}\left(m+1\right)-\overline{\tau}_{i}\left(m\right)$ under steady state assumption.
\end{Def}

\begin{Def}
The intervisit time $I_{i}$ is defined as the period beginning at the time of its service completion in a cycle and ending at the time when it is polled in the next cycle; its duration is given by $\tau_{i}\left(m+1\right)-\overline{\tau}_{i}\left(m\right)$.
\end{Def}

The intervisit time duration $\tau_{i}\left(m+1\right)-\overline{\tau}\left(m\right)$ given the number of users found at queue $Q_{i}$ at time $t=\tau_{i}\left(m+1\right)$ is equal to the number of arrivals during the preceding intervisit time $\left[\overline{\tau}\left(m\right),\tau_{i}\left(m+1\right)\right]$. 

So we have



\begin{eqnarray*}
\esp\left[z_{i}^{L_{i}\left(\tau_{i}\left(m+1\right)\right)}\right]=\esp\left[\left\{P_{i}\left(z_{i}\right)\right\}^{\tau_{i}\left(m+1\right)-\overline{\tau}\left(m\right)}\right]
\end{eqnarray*}

if $I_{i}\left(z\right)=\esp\left[z^{\tau_{i}\left(m+1\right)-\overline{\tau}\left(m\right)}\right]$
we have $F_{i}\left(z\right)=I_{i}\left[P_{i}\left(z\right)\right]$
for $i=1,2$. Futhermore can be proved that

\begin{eqnarray}
\begin{array}{ll}
\esp\left[L_{i}\right]=\mu_{i}\esp\left[I_{i}\right], &
\esp\left[C_{i}\right]=\frac{f_{i}\left(i\right)}{\mu_{i}\left(1-\mu_{i}\right)},\\
\esp\left[S_{i}\right]=\mu_{i}\esp\left[C_{i}\right],&
\esp\left[I_{i}\right]=\left(1-\mu_{i}\right)\esp\left[C_{i}\right],\\
Var\left[L_{i}\right]= \mu_{i}^{2}Var\left[I_{i}\right]+\sigma^{2}\esp\left[I_{i}\right],& 
Var\left[C_{i}\right]=\frac{Var\left[L_{i}^{*}\right]}{\mu_{i}^{2}\left(1-\mu_{i}\right)^{2}},\\
Var\left[S_{i}\right]= \frac{Var\left[L_{i}^{*}\right]}{\left(1-\mu_{i}\right)^{2}}+\frac{\sigma^{2}\esp\left[L_{i}^{*}\right]}{\left(1-\mu_{i}\right)^{3}},&
Var\left[I_{i}\right]= \frac{Var\left[L_{i}^{*}\right]}{\mu_{i}^{2}}-\frac{\sigma_{i}^{2}}{\mu_{i}^{2}}f_{i}\left(i\right).
\end{array}
\end{eqnarray}

Let consider the points when the process $\left[L_{1}\left(1\right),L_{2}\left(1\right),L_{3}\left(1\right),L_{4}\left(1\right)
\right]$ becomes zero at the same time, this points, $T_{1},T_{2},\ldots$ will be denoted as regeneration points, then we have that

\begin{Def}
the interval between two such succesive regeneration points will be called regenerative cycle.
\end{Def}

\begin{Def}
Para $T_{i}$ se define, $M_{i}$, el n\'umero de ciclos de visita a la cola $Q_{l}$, durante el ciclo regenerativo, es decir, $M_{i}$ es un proceso de renovaci\'on.
\end{Def}

\begin{Def}
Para cada uno de los $M_{i}$'s, se definen a su vez la duraci\'on de cada uno de estos ciclos de visita en el ciclo regenerativo, $C_{i}^{(m)}$, para $m=1,2,\ldots,M_{i}$, que a su vez, tambi\'en es n proceso de renovaci\'on.
\end{Def}



Sea la funci\'on generadora de momentos para $L_{i}$, el n\'umero de usuarios en la cola $Q_{i}\left(z\right)$ en cualquier momento, est\'a dada por el tiempo promedio de $z^{L_{i}\left(t\right)}$ sobre el ciclo regenerativo definido anteriormente. Entonces 

\begin{equation}\label{Eq.Longitud.Tiempo.t}
Q_{i}\left(z\right)=\frac{1}{\esp\left[C_{i}\right]}\cdot\frac{1-F_{i}\left(z\right)}{P_{i}\left(z\right)-z}\cdot\frac{\left(1-z\right)P_{i}\left(z\right)}{1-P_{i}\left(z\right)}.
\end{equation}

Es decir, es posible determinar las longitudes de las colas a cualquier tiempo $t$. Entonces, determinando el primer momento es posible ver que


$M_{i}$ is an stopping time for the regenerative process with $\esp\left[M_{i}\right]<\infty$, from Wald's lemma follows that:


\begin{eqnarray*}
\esp\left[\sum_{m=1}^{M_{i}}\sum_{t=\tau_{i}\left(m\right)}^{\tau_{i}\left(m+1\right)-1}z^{L_{i}\left(t\right)}\right]&=&\esp\left[M_{i}\right]\esp\left[\sum_{t=\tau_{i}\left(m\right)}^{\tau_{i}\left(m+1\right)-1}z^{L_{i}\left(t\right)}\right]\\
\esp\left[\sum_{m=1}^{M_{i}}\tau_{i}\left(m+1\right)-\tau_{i}\left(m\right)\right]&=&\esp\left[M_{i}\right]\esp\left[\tau_{i}\left(m+1\right)-\tau_{i}\left(m\right)\right]
\end{eqnarray*}
therefore 

\begin{eqnarray*}
Q_{i}\left(z\right)&=&\frac{\esp\left[\sum_{t=\tau_{i}\left(m\right)}^{\tau_{i}\left(m+1\right)-1}z^{L_{i}\left(t\right)}\right]}{\esp\left[\tau_{i}\left(m+1\right)-\tau_{i}\left(m\right)\right]}
\end{eqnarray*}

Doing the following substitutions en (\ref{Corolario2}): $n\rightarrow t-\tau_{i}\left(m\right)$, $T \rightarrow \overline{\tau}_{i}\left(m\right)-\tau_{i}\left(m\right)$, $L_{n}\rightarrow L_{i}\left(t\right)$ and $F\left(z\right)=\esp\left[z^{L_{0}}\right]\rightarrow F_{i}\left(z\right)=\esp\left[z^{L_{i}\tau_{i}\left(m\right)}\right]$, 
we obtain

\begin{eqnarray}\label{Eq.Arribos.Primera}
\esp\left[\sum_{n=0}^{T-1}z^{L_{n}}\right]=
\esp\left[\sum_{t=\tau_{i}\left(m\right)}^{\overline{\tau}_{i}\left(m\right)-1}z^{L_{i}\left(t\right)}\right]
=z\frac{F_{i}\left(z\right)-1}{z-P_{i}\left(z\right)}
\end{eqnarray}



Por otra parte durante el tiempo de intervisita para la cola $i$, $L_{i}\left(t\right)$ solamente se incrementa de manera que el incremento por intervalo de tiempo est\'a dado por la funci\'on generadora de probabilidades de $P_{i}\left(z\right)$, por tanto la suma sobre el tiempo de intervisita puede evaluarse como:

\begin{eqnarray*}
\esp\left[\sum_{t=\tau_{i}\left(m\right)}^{\tau_{i}\left(m+1\right)-1}z^{L_{i}\left(t\right)}\right]&=&\esp\left[\sum_{t=\tau_{i}\left(m\right)}^{\tau_{i}\left(m+1\right)-1}\left\{P_{i}\left(z\right)\right\}^{t-\overline{\tau}_{i}\left(m\right)}\right]=\frac{1-\esp\left[\left\{P_{i}\left(z\right)\right\}^{\tau_{i}\left(m+1\right)-\overline{\tau}_{i}\left(m\right)}\right]}{1-P_{i}\left(z\right)}\\
&=&\frac{1-I_{i}\left[P_{i}\left(z\right)\right]}{1-P_{i}\left(z\right)}
\end{eqnarray*}
por tanto

\begin{eqnarray*}
\esp\left[\sum_{t=\tau_{i}\left(m\right)}^{\tau_{i}\left(m+1\right)-1}z^{L_{i}\left(t\right)}\right]&=&
\frac{1-F_{i}\left(z\right)}{1-P_{i}\left(z\right)}
\end{eqnarray*}

Por lo tanto

\begin{eqnarray*}
Q_{i}\left(z\right)&=&\frac{\esp\left[\sum_{t=\tau_{i}\left(m\right)}^{\tau_{i}\left(m+1\right)-1}z^{L_{i}\left(t\right)}\right]}{\esp\left[\tau_{i}\left(m+1\right)-\tau_{i}\left(m\right)\right]}
=\frac{1}{\esp\left[\tau_{i}\left(m+1\right)-\tau_{i}\left(m\right)\right]}
\esp\left[\sum_{t=\tau_{i}\left(m\right)}^{\tau_{i}\left(m+1\right)-1}z^{L_{i}\left(t\right)}\right]\\
&=&\frac{1}{\esp\left[\tau_{i}\left(m+1\right)-\tau_{i}\left(m\right)\right]}
\esp\left[\sum_{t=\tau_{i}\left(m\right)}^{\overline{\tau}_{i}\left(m\right)-1}z^{L_{i}\left(t\right)}
+\sum_{t=\overline{\tau}_{i}\left(m\right)}^{\tau_{i}\left(m+1\right)-1}z^{L_{i}\left(t\right)}\right]\\
&=&\frac{1}{\esp\left[\tau_{i}\left(m+1\right)-\tau_{i}\left(m\right)\right]}\left\{
\esp\left[\sum_{t=\tau_{i}\left(m\right)}^{\overline{\tau}_{i}\left(m\right)-1}z^{L_{i}\left(t\right)}\right]
+\esp\left[\sum_{t=\overline{\tau}_{i}\left(m\right)}^{\tau_{i}\left(m+1\right)-1}z^{L_{i}\left(t\right)}\right]\right\}\\
&=&\frac{1}{\esp\left[\tau_{i}\left(m+1\right)-\tau_{i}\left(m\right)\right]}\left\{
z\frac{F_{i}\left(z\right)-1}{z-P_{i}\left(z\right)}+\frac{1-F_{i}\left(z\right)}{1-P_{i}\left(z\right)}
\right\}\\
&=&\frac{1}{\esp\left[C_{i}\right]}\cdot\frac{1-F_{i}\left(z\right)}{P_{i}\left(z\right)-z}\cdot\frac{\left(1-z\right)P_{i}\left(z\right)}{1-P_{i}\left(z\right)}
\end{eqnarray*}

es decir

\begin{eqnarray}
\begin{array}{ll}
S^{'}\left(z\right)=-\sum_{k=1}^{+\infty}ka_{k}z^{k-1},& S^{(1)}\left(1\right)=-\sum_{k=1}^{+\infty}ka_{k}=-\esp\left[L\left(t\right)\right],\\
S^{''}\left(z\right)=-\sum_{k=2}^{+\infty}k(k-1)a_{k}z^{k-2},& S^{(2)}\left(1\right)=-\sum_{k=2}^{+\infty}k(k-1)a_{k}=\esp\left[L\left(L-1\right)\right],\\
S^{'''}\left(z\right)=-\sum_{k=3}^{+\infty}k(k-1)(k-2)a_{k}z^{k-3},&
S^{(3)}\left(1\right)=-\sum_{k=3}^{+\infty}k(k-1)(k-2)a_{k}\\
&=-\esp\left[L\left(L-1\right)\left(L-2\right)\right]\\
&=-\esp\left[L^{3}\right]+3-\esp\left[L^{2}\right]-2-\esp\left[L\right];
\end{array}
\end{eqnarray}







%________________________________________________________________________
\subsection{Procesos Regenerativos Sigman, Thorisson y Wolff \cite{Sigman1}}
%________________________________________________________________________


\begin{Def}[Definici\'on Cl\'asica]
Un proceso estoc\'astico $X=\left\{X\left(t\right):t\geq0\right\}$ es llamado regenerativo is existe una variable aleatoria $R_{1}>0$ tal que
\begin{itemize}
\item[i)] $\left\{X\left(t+R_{1}\right):t\geq0\right\}$ es independiente de $\left\{\left\{X\left(t\right):t<R_{1}\right\},\right\}$
\item[ii)] $\left\{X\left(t+R_{1}\right):t\geq0\right\}$ es estoc\'asticamente equivalente a $\left\{X\left(t\right):t>0\right\}$
\end{itemize}

Llamamos a $R_{1}$ tiempo de regeneraci\'on, y decimos que $X$ se regenera en este punto.
\end{Def}

$\left\{X\left(t+R_{1}\right)\right\}$ es regenerativo con tiempo de regeneraci\'on $R_{2}$, independiente de $R_{1}$ pero con la misma distribuci\'on que $R_{1}$. Procediendo de esta manera se obtiene una secuencia de variables aleatorias independientes e id\'enticamente distribuidas $\left\{R_{n}\right\}$ llamados longitudes de ciclo. Si definimos a $Z_{k}\equiv R_{1}+R_{2}+\cdots+R_{k}$, se tiene un proceso de renovaci\'on llamado proceso de renovaci\'on encajado para $X$.


\begin{Note}
La existencia de un primer tiempo de regeneraci\'on, $R_{1}$, implica la existencia de una sucesi\'on completa de estos tiempos $R_{1},R_{2}\ldots,$ que satisfacen la propiedad deseada \cite{Sigman2}.
\end{Note}


\begin{Note} Para la cola $GI/GI/1$ los usuarios arriban con tiempos $t_{n}$ y son atendidos con tiempos de servicio $S_{n}$, los tiempos de arribo forman un proceso de renovaci\'on  con tiempos entre arribos independientes e identicamente distribuidos (\texttt{i.i.d.})$T_{n}=t_{n}-t_{n-1}$, adem\'as los tiempos de servicio son \texttt{i.i.d.} e independientes de los procesos de arribo. Por \textit{estable} se entiende que $\esp S_{n}<\esp T_{n}<\infty$.
\end{Note}
 


\begin{Def}
Para $x$ fijo y para cada $t\geq0$, sea $I_{x}\left(t\right)=1$ si $X\left(t\right)\leq x$,  $I_{x}\left(t\right)=0$ en caso contrario, y def\'inanse los tiempos promedio
\begin{eqnarray*}
\overline{X}&=&lim_{t\rightarrow\infty}\frac{1}{t}\int_{0}^{\infty}X\left(u\right)du\\
\prob\left(X_{\infty}\leq x\right)&=&lim_{t\rightarrow\infty}\frac{1}{t}\int_{0}^{\infty}I_{x}\left(u\right)du,
\end{eqnarray*}
cuando estos l\'imites existan.
\end{Def}

Como consecuencia del teorema de Renovaci\'on-Recompensa, se tiene que el primer l\'imite  existe y es igual a la constante
\begin{eqnarray*}
\overline{X}&=&\frac{\esp\left[\int_{0}^{R_{1}}X\left(t\right)dt\right]}{\esp\left[R_{1}\right]},
\end{eqnarray*}
suponiendo que ambas esperanzas son finitas.
 
\begin{Note}
Funciones de procesos regenerativos son regenerativas, es decir, si $X\left(t\right)$ es regenerativo y se define el proceso $Y\left(t\right)$ por $Y\left(t\right)=f\left(X\left(t\right)\right)$ para alguna funci\'on Borel medible $f\left(\cdot\right)$. Adem\'as $Y$ es regenerativo con los mismos tiempos de renovaci\'on que $X$. 

En general, los tiempos de renovaci\'on, $Z_{k}$ de un proceso regenerativo no requieren ser tiempos de paro con respecto a la evoluci\'on de $X\left(t\right)$.
\end{Note} 

\begin{Note}
Una funci\'on de un proceso de Markov, usualmente no ser\'a un proceso de Markov, sin embargo ser\'a regenerativo si el proceso de Markov lo es.
\end{Note}

 
\begin{Note}
Un proceso regenerativo con media de la longitud de ciclo finita es llamado positivo recurrente.
\end{Note}


\begin{Note}
\begin{itemize}
\item[a)] Si el proceso regenerativo $X$ es positivo recurrente y tiene trayectorias muestrales no negativas, entonces la ecuaci\'on anterior es v\'alida.
\item[b)] Si $X$ es positivo recurrente regenerativo, podemos construir una \'unica versi\'on estacionaria de este proceso, $X_{e}=\left\{X_{e}\left(t\right)\right\}$, donde $X_{e}$ es un proceso estoc\'astico regenerativo y estrictamente estacionario, con distribuci\'on marginal distribuida como $X_{\infty}$
\end{itemize}
\end{Note}


%__________________________________________________________________________________________
%\subsection{Procesos Regenerativos Estacionarios - Stidham \cite{Stidham}}
%__________________________________________________________________________________________


Un proceso estoc\'astico a tiempo continuo $\left\{V\left(t\right),t\geq0\right\}$ es un proceso regenerativo si existe una sucesi\'on de variables aleatorias independientes e id\'enticamente distribuidas $\left\{X_{1},X_{2},\ldots\right\}$, sucesi\'on de renovaci\'on, tal que para cualquier conjunto de Borel $A$, 

\begin{eqnarray*}
\prob\left\{V\left(t\right)\in A|X_{1}+X_{2}+\cdots+X_{R\left(t\right)}=s,\left\{V\left(\tau\right),\tau<s\right\}\right\}=\prob\left\{V\left(t-s\right)\in A|X_{1}>t-s\right\},
\end{eqnarray*}
para todo $0\leq s\leq t$, donde $R\left(t\right)=\max\left\{X_{1}+X_{2}+\cdots+X_{j}\leq t\right\}=$n\'umero de renovaciones ({\emph{puntos de regeneraci\'on}}) que ocurren en $\left[0,t\right]$. El intervalo $\left[0,X_{1}\right)$ es llamado {\emph{primer ciclo de regeneraci\'on}} de $\left\{V\left(t \right),t\geq0\right\}$, $\left[X_{1},X_{1}+X_{2}\right)$ el {\emph{segundo ciclo de regeneraci\'on}}, y as\'i sucesivamente.

Sea $X=X_{1}$ y sea $F$ la funci\'on de distrbuci\'on de $X$


\begin{Def}
Se define el proceso estacionario, $\left\{V^{*}\left(t\right),t\geq0\right\}$, para $\left\{V\left(t\right),t\geq0\right\}$ por

\begin{eqnarray*}
\prob\left\{V\left(t\right)\in A\right\}=\frac{1}{\esp\left[X\right]}\int_{0}^{\infty}\prob\left\{V\left(t+x\right)\in A|X>x\right\}\left(1-F\left(x\right)\right)dx,
\end{eqnarray*} 
para todo $t\geq0$ y todo conjunto de Borel $A$.
\end{Def}

\begin{Def}
Una distribuci\'on se dice que es {\emph{aritm\'etica}} si todos sus puntos de incremento son m\'ultiplos de la forma $0,\lambda, 2\lambda,\ldots$ para alguna $\lambda>0$ entera.
\end{Def}


\begin{Def}
Una modificaci\'on medible de un proceso $\left\{V\left(t\right),t\geq0\right\}$, es una versi\'on de este, $\left\{V\left(t,w\right)\right\}$ conjuntamente medible para $t\geq0$ y para $w\in S$, $S$ espacio de estados para $\left\{V\left(t\right),t\geq0\right\}$.
\end{Def}

\begin{Teo}
Sea $\left\{V\left(t\right),t\geq\right\}$ un proceso regenerativo no negativo con modificaci\'on medible. Sea $\esp\left[X\right]<\infty$. Entonces el proceso estacionario dado por la ecuaci\'on anterior est\'a bien definido y tiene funci\'on de distribuci\'on independiente de $t$, adem\'as
\begin{itemize}
\item[i)] \begin{eqnarray*}
\esp\left[V^{*}\left(0\right)\right]&=&\frac{\esp\left[\int_{0}^{X}V\left(s\right)ds\right]}{\esp\left[X\right]}\end{eqnarray*}
\item[ii)] Si $\esp\left[V^{*}\left(0\right)\right]<\infty$, equivalentemente, si $\esp\left[\int_{0}^{X}V\left(s\right)ds\right]<\infty$,entonces
\begin{eqnarray*}
\frac{\int_{0}^{t}V\left(s\right)ds}{t}\rightarrow\frac{\esp\left[\int_{0}^{X}V\left(s\right)ds\right]}{\esp\left[X\right]}
\end{eqnarray*}
con probabilidad 1 y en media, cuando $t\rightarrow\infty$.
\end{itemize}
\end{Teo}

\begin{Coro}
Sea $\left\{V\left(t\right),t\geq0\right\}$ un proceso regenerativo no negativo, con modificaci\'on medible. Si $\esp <\infty$, $F$ es no-aritm\'etica, y para todo $x\geq0$, $P\left\{V\left(t\right)\leq x,C>x\right\}$ es de variaci\'on acotada como funci\'on de $t$ en cada intervalo finito $\left[0,\tau\right]$, entonces $V\left(t\right)$ converge en distribuci\'on  cuando $t\rightarrow\infty$ y $$\esp V=\frac{\esp \int_{0}^{X}V\left(s\right)ds}{\esp X}$$
Donde $V$ tiene la distribuci\'on l\'imite de $V\left(t\right)$ cuando $t\rightarrow\infty$.

\end{Coro}

Para el caso discreto se tienen resultados similares.



%______________________________________________________________________
%\subsection{Procesos de Renovaci\'on}
%______________________________________________________________________

\begin{Def}%\label{Def.Tn}
Sean $0\leq T_{1}\leq T_{2}\leq \ldots$ son tiempos aleatorios infinitos en los cuales ocurren ciertos eventos. El n\'umero de tiempos $T_{n}$ en el intervalo $\left[0,t\right)$ es

\begin{eqnarray}
N\left(t\right)=\sum_{n=1}^{\infty}\indora\left(T_{n}\leq t\right),
\end{eqnarray}
para $t\geq0$.
\end{Def}

Si se consideran los puntos $T_{n}$ como elementos de $\rea_{+}$, y $N\left(t\right)$ es el n\'umero de puntos en $\rea$. El proceso denotado por $\left\{N\left(t\right):t\geq0\right\}$, denotado por $N\left(t\right)$, es un proceso puntual en $\rea_{+}$. Los $T_{n}$ son los tiempos de ocurrencia, el proceso puntual $N\left(t\right)$ es simple si su n\'umero de ocurrencias son distintas: $0<T_{1}<T_{2}<\ldots$ casi seguramente.

\begin{Def}
Un proceso puntual $N\left(t\right)$ es un proceso de renovaci\'on si los tiempos de interocurrencia $\xi_{n}=T_{n}-T_{n-1}$, para $n\geq1$, son independientes e identicamente distribuidos con distribuci\'on $F$, donde $F\left(0\right)=0$ y $T_{0}=0$. Los $T_{n}$ son llamados tiempos de renovaci\'on, referente a la independencia o renovaci\'on de la informaci\'on estoc\'astica en estos tiempos. Los $\xi_{n}$ son los tiempos de inter-renovaci\'on, y $N\left(t\right)$ es el n\'umero de renovaciones en el intervalo $\left[0,t\right)$
\end{Def}


\begin{Note}
Para definir un proceso de renovaci\'on para cualquier contexto, solamente hay que especificar una distribuci\'on $F$, con $F\left(0\right)=0$, para los tiempos de inter-renovaci\'on. La funci\'on $F$ en turno degune las otra variables aleatorias. De manera formal, existe un espacio de probabilidad y una sucesi\'on de variables aleatorias $\xi_{1},\xi_{2},\ldots$ definidas en este con distribuci\'on $F$. Entonces las otras cantidades son $T_{n}=\sum_{k=1}^{n}\xi_{k}$ y $N\left(t\right)=\sum_{n=1}^{\infty}\indora\left(T_{n}\leq t\right)$, donde $T_{n}\rightarrow\infty$ casi seguramente por la Ley Fuerte de los Grandes Números.
\end{Note}

%___________________________________________________________________________________________
%
%\subsection{Teorema Principal de Renovaci\'on}
%___________________________________________________________________________________________
%

\begin{Note} Una funci\'on $h:\rea_{+}\rightarrow\rea$ es Directamente Riemann Integrable en los siguientes casos:
\begin{itemize}
\item[a)] $h\left(t\right)\geq0$ es decreciente y Riemann Integrable.
\item[b)] $h$ es continua excepto posiblemente en un conjunto de Lebesgue de medida 0, y $|h\left(t\right)|\leq b\left(t\right)$, donde $b$ es DRI.
\end{itemize}
\end{Note}

\begin{Teo}[Teorema Principal de Renovaci\'on]
Si $F$ es no aritm\'etica y $h\left(t\right)$ es Directamente Riemann Integrable (DRI), entonces

\begin{eqnarray*}
lim_{t\rightarrow\infty}U\star h=\frac{1}{\mu}\int_{\rea_{+}}h\left(s\right)ds.
\end{eqnarray*}
\end{Teo}

\begin{Prop}
Cualquier funci\'on $H\left(t\right)$ acotada en intervalos finitos y que es 0 para $t<0$ puede expresarse como
\begin{eqnarray*}
H\left(t\right)=U\star h\left(t\right)\textrm{,  donde }h\left(t\right)=H\left(t\right)-F\star H\left(t\right)
\end{eqnarray*}
\end{Prop}

\begin{Def}
Un proceso estoc\'astico $X\left(t\right)$ es crudamente regenerativo en un tiempo aleatorio positivo $T$ si
\begin{eqnarray*}
\esp\left[X\left(T+t\right)|T\right]=\esp\left[X\left(t\right)\right]\textrm{, para }t\geq0,\end{eqnarray*}
y con las esperanzas anteriores finitas.
\end{Def}

\begin{Prop}
Sup\'ongase que $X\left(t\right)$ es un proceso crudamente regenerativo en $T$, que tiene distribuci\'on $F$. Si $\esp\left[X\left(t\right)\right]$ es acotado en intervalos finitos, entonces
\begin{eqnarray*}
\esp\left[X\left(t\right)\right]=U\star h\left(t\right)\textrm{,  donde }h\left(t\right)=\esp\left[X\left(t\right)\indora\left(T>t\right)\right].
\end{eqnarray*}
\end{Prop}

\begin{Teo}[Regeneraci\'on Cruda]
Sup\'ongase que $X\left(t\right)$ es un proceso con valores positivo crudamente regenerativo en $T$, y def\'inase $M=\sup\left\{|X\left(t\right)|:t\leq T\right\}$. Si $T$ es no aritm\'etico y $M$ y $MT$ tienen media finita, entonces
\begin{eqnarray*}
lim_{t\rightarrow\infty}\esp\left[X\left(t\right)\right]=\frac{1}{\mu}\int_{\rea_{+}}h\left(s\right)ds,
\end{eqnarray*}
donde $h\left(t\right)=\esp\left[X\left(t\right)\indora\left(T>t\right)\right]$.
\end{Teo}

%___________________________________________________________________________________________
%
%\subsection{Propiedades de los Procesos de Renovaci\'on}
%___________________________________________________________________________________________
%

Los tiempos $T_{n}$ est\'an relacionados con los conteos de $N\left(t\right)$ por

\begin{eqnarray*}
\left\{N\left(t\right)\geq n\right\}&=&\left\{T_{n}\leq t\right\}\\
T_{N\left(t\right)}\leq &t&<T_{N\left(t\right)+1},
\end{eqnarray*}

adem\'as $N\left(T_{n}\right)=n$, y 

\begin{eqnarray*}
N\left(t\right)=\max\left\{n:T_{n}\leq t\right\}=\min\left\{n:T_{n+1}>t\right\}
\end{eqnarray*}

Por propiedades de la convoluci\'on se sabe que

\begin{eqnarray*}
P\left\{T_{n}\leq t\right\}=F^{n\star}\left(t\right)
\end{eqnarray*}
que es la $n$-\'esima convoluci\'on de $F$. Entonces 

\begin{eqnarray*}
\left\{N\left(t\right)\geq n\right\}&=&\left\{T_{n}\leq t\right\}\\
P\left\{N\left(t\right)\leq n\right\}&=&1-F^{\left(n+1\right)\star}\left(t\right)
\end{eqnarray*}

Adem\'as usando el hecho de que $\esp\left[N\left(t\right)\right]=\sum_{n=1}^{\infty}P\left\{N\left(t\right)\geq n\right\}$
se tiene que

\begin{eqnarray*}
\esp\left[N\left(t\right)\right]=\sum_{n=1}^{\infty}F^{n\star}\left(t\right)
\end{eqnarray*}

\begin{Prop}
Para cada $t\geq0$, la funci\'on generadora de momentos $\esp\left[e^{\alpha N\left(t\right)}\right]$ existe para alguna $\alpha$ en una vecindad del 0, y de aqu\'i que $\esp\left[N\left(t\right)^{m}\right]<\infty$, para $m\geq1$.
\end{Prop}


\begin{Note}
Si el primer tiempo de renovaci\'on $\xi_{1}$ no tiene la misma distribuci\'on que el resto de las $\xi_{n}$, para $n\geq2$, a $N\left(t\right)$ se le llama Proceso de Renovaci\'on retardado, donde si $\xi$ tiene distribuci\'on $G$, entonces el tiempo $T_{n}$ de la $n$-\'esima renovaci\'on tiene distribuci\'on $G\star F^{\left(n-1\right)\star}\left(t\right)$
\end{Note}


\begin{Teo}
Para una constante $\mu\leq\infty$ ( o variable aleatoria), las siguientes expresiones son equivalentes:

\begin{eqnarray}
lim_{n\rightarrow\infty}n^{-1}T_{n}&=&\mu,\textrm{ c.s.}\\
lim_{t\rightarrow\infty}t^{-1}N\left(t\right)&=&1/\mu,\textrm{ c.s.}
\end{eqnarray}
\end{Teo}


Es decir, $T_{n}$ satisface la Ley Fuerte de los Grandes N\'umeros s\'i y s\'olo s\'i $N\left/t\right)$ la cumple.


\begin{Coro}[Ley Fuerte de los Grandes N\'umeros para Procesos de Renovaci\'on]
Si $N\left(t\right)$ es un proceso de renovaci\'on cuyos tiempos de inter-renovaci\'on tienen media $\mu\leq\infty$, entonces
\begin{eqnarray}
t^{-1}N\left(t\right)\rightarrow 1/\mu,\textrm{ c.s. cuando }t\rightarrow\infty.
\end{eqnarray}

\end{Coro}


Considerar el proceso estoc\'astico de valores reales $\left\{Z\left(t\right):t\geq0\right\}$ en el mismo espacio de probabilidad que $N\left(t\right)$

\begin{Def}
Para el proceso $\left\{Z\left(t\right):t\geq0\right\}$ se define la fluctuaci\'on m\'axima de $Z\left(t\right)$ en el intervalo $\left(T_{n-1},T_{n}\right]$:
\begin{eqnarray*}
M_{n}=\sup_{T_{n-1}<t\leq T_{n}}|Z\left(t\right)-Z\left(T_{n-1}\right)|
\end{eqnarray*}
\end{Def}

\begin{Teo}
Sup\'ongase que $n^{-1}T_{n}\rightarrow\mu$ c.s. cuando $n\rightarrow\infty$, donde $\mu\leq\infty$ es una constante o variable aleatoria. Sea $a$ una constante o variable aleatoria que puede ser infinita cuando $\mu$ es finita, y considere las expresiones l\'imite:
\begin{eqnarray}
lim_{n\rightarrow\infty}n^{-1}Z\left(T_{n}\right)&=&a,\textrm{ c.s.}\\
lim_{t\rightarrow\infty}t^{-1}Z\left(t\right)&=&a/\mu,\textrm{ c.s.}
\end{eqnarray}
La segunda expresi\'on implica la primera. Conversamente, la primera implica la segunda si el proceso $Z\left(t\right)$ es creciente, o si $lim_{n\rightarrow\infty}n^{-1}M_{n}=0$ c.s.
\end{Teo}

\begin{Coro}
Si $N\left(t\right)$ es un proceso de renovaci\'on, y $\left(Z\left(T_{n}\right)-Z\left(T_{n-1}\right),M_{n}\right)$, para $n\geq1$, son variables aleatorias independientes e id\'enticamente distribuidas con media finita, entonces,
\begin{eqnarray}
lim_{t\rightarrow\infty}t^{-1}Z\left(t\right)\rightarrow\frac{\esp\left[Z\left(T_{1}\right)-Z\left(T_{0}\right)\right]}{\esp\left[T_{1}\right]},\textrm{ c.s. cuando  }t\rightarrow\infty.
\end{eqnarray}
\end{Coro}



%___________________________________________________________________________________________
%
%\subsection{Propiedades de los Procesos de Renovaci\'on}
%___________________________________________________________________________________________
%

Los tiempos $T_{n}$ est\'an relacionados con los conteos de $N\left(t\right)$ por

\begin{eqnarray*}
\left\{N\left(t\right)\geq n\right\}&=&\left\{T_{n}\leq t\right\}\\
T_{N\left(t\right)}\leq &t&<T_{N\left(t\right)+1},
\end{eqnarray*}

adem\'as $N\left(T_{n}\right)=n$, y 

\begin{eqnarray*}
N\left(t\right)=\max\left\{n:T_{n}\leq t\right\}=\min\left\{n:T_{n+1}>t\right\}
\end{eqnarray*}

Por propiedades de la convoluci\'on se sabe que

\begin{eqnarray*}
P\left\{T_{n}\leq t\right\}=F^{n\star}\left(t\right)
\end{eqnarray*}
que es la $n$-\'esima convoluci\'on de $F$. Entonces 

\begin{eqnarray*}
\left\{N\left(t\right)\geq n\right\}&=&\left\{T_{n}\leq t\right\}\\
P\left\{N\left(t\right)\leq n\right\}&=&1-F^{\left(n+1\right)\star}\left(t\right)
\end{eqnarray*}

Adem\'as usando el hecho de que $\esp\left[N\left(t\right)\right]=\sum_{n=1}^{\infty}P\left\{N\left(t\right)\geq n\right\}$
se tiene que

\begin{eqnarray*}
\esp\left[N\left(t\right)\right]=\sum_{n=1}^{\infty}F^{n\star}\left(t\right)
\end{eqnarray*}

\begin{Prop}
Para cada $t\geq0$, la funci\'on generadora de momentos $\esp\left[e^{\alpha N\left(t\right)}\right]$ existe para alguna $\alpha$ en una vecindad del 0, y de aqu\'i que $\esp\left[N\left(t\right)^{m}\right]<\infty$, para $m\geq1$.
\end{Prop}


\begin{Note}
Si el primer tiempo de renovaci\'on $\xi_{1}$ no tiene la misma distribuci\'on que el resto de las $\xi_{n}$, para $n\geq2$, a $N\left(t\right)$ se le llama Proceso de Renovaci\'on retardado, donde si $\xi$ tiene distribuci\'on $G$, entonces el tiempo $T_{n}$ de la $n$-\'esima renovaci\'on tiene distribuci\'on $G\star F^{\left(n-1\right)\star}\left(t\right)$
\end{Note}


\begin{Teo}
Para una constante $\mu\leq\infty$ ( o variable aleatoria), las siguientes expresiones son equivalentes:

\begin{eqnarray}
lim_{n\rightarrow\infty}n^{-1}T_{n}&=&\mu,\textrm{ c.s.}\\
lim_{t\rightarrow\infty}t^{-1}N\left(t\right)&=&1/\mu,\textrm{ c.s.}
\end{eqnarray}
\end{Teo}


Es decir, $T_{n}$ satisface la Ley Fuerte de los Grandes N\'umeros s\'i y s\'olo s\'i $N\left/t\right)$ la cumple.


\begin{Coro}[Ley Fuerte de los Grandes N\'umeros para Procesos de Renovaci\'on]
Si $N\left(t\right)$ es un proceso de renovaci\'on cuyos tiempos de inter-renovaci\'on tienen media $\mu\leq\infty$, entonces
\begin{eqnarray}
t^{-1}N\left(t\right)\rightarrow 1/\mu,\textrm{ c.s. cuando }t\rightarrow\infty.
\end{eqnarray}

\end{Coro}


Considerar el proceso estoc\'astico de valores reales $\left\{Z\left(t\right):t\geq0\right\}$ en el mismo espacio de probabilidad que $N\left(t\right)$

\begin{Def}
Para el proceso $\left\{Z\left(t\right):t\geq0\right\}$ se define la fluctuaci\'on m\'axima de $Z\left(t\right)$ en el intervalo $\left(T_{n-1},T_{n}\right]$:
\begin{eqnarray*}
M_{n}=\sup_{T_{n-1}<t\leq T_{n}}|Z\left(t\right)-Z\left(T_{n-1}\right)|
\end{eqnarray*}
\end{Def}

\begin{Teo}
Sup\'ongase que $n^{-1}T_{n}\rightarrow\mu$ c.s. cuando $n\rightarrow\infty$, donde $\mu\leq\infty$ es una constante o variable aleatoria. Sea $a$ una constante o variable aleatoria que puede ser infinita cuando $\mu$ es finita, y considere las expresiones l\'imite:
\begin{eqnarray}
lim_{n\rightarrow\infty}n^{-1}Z\left(T_{n}\right)&=&a,\textrm{ c.s.}\\
lim_{t\rightarrow\infty}t^{-1}Z\left(t\right)&=&a/\mu,\textrm{ c.s.}
\end{eqnarray}
La segunda expresi\'on implica la primera. Conversamente, la primera implica la segunda si el proceso $Z\left(t\right)$ es creciente, o si $lim_{n\rightarrow\infty}n^{-1}M_{n}=0$ c.s.
\end{Teo}

\begin{Coro}
Si $N\left(t\right)$ es un proceso de renovaci\'on, y $\left(Z\left(T_{n}\right)-Z\left(T_{n-1}\right),M_{n}\right)$, para $n\geq1$, son variables aleatorias independientes e id\'enticamente distribuidas con media finita, entonces,
\begin{eqnarray}
lim_{t\rightarrow\infty}t^{-1}Z\left(t\right)\rightarrow\frac{\esp\left[Z\left(T_{1}\right)-Z\left(T_{0}\right)\right]}{\esp\left[T_{1}\right]},\textrm{ c.s. cuando  }t\rightarrow\infty.
\end{eqnarray}
\end{Coro}


%___________________________________________________________________________________________
%
%\subsection{Propiedades de los Procesos de Renovaci\'on}
%___________________________________________________________________________________________
%

Los tiempos $T_{n}$ est\'an relacionados con los conteos de $N\left(t\right)$ por

\begin{eqnarray*}
\left\{N\left(t\right)\geq n\right\}&=&\left\{T_{n}\leq t\right\}\\
T_{N\left(t\right)}\leq &t&<T_{N\left(t\right)+1},
\end{eqnarray*}

adem\'as $N\left(T_{n}\right)=n$, y 

\begin{eqnarray*}
N\left(t\right)=\max\left\{n:T_{n}\leq t\right\}=\min\left\{n:T_{n+1}>t\right\}
\end{eqnarray*}

Por propiedades de la convoluci\'on se sabe que

\begin{eqnarray*}
P\left\{T_{n}\leq t\right\}=F^{n\star}\left(t\right)
\end{eqnarray*}
que es la $n$-\'esima convoluci\'on de $F$. Entonces 

\begin{eqnarray*}
\left\{N\left(t\right)\geq n\right\}&=&\left\{T_{n}\leq t\right\}\\
P\left\{N\left(t\right)\leq n\right\}&=&1-F^{\left(n+1\right)\star}\left(t\right)
\end{eqnarray*}

Adem\'as usando el hecho de que $\esp\left[N\left(t\right)\right]=\sum_{n=1}^{\infty}P\left\{N\left(t\right)\geq n\right\}$
se tiene que

\begin{eqnarray*}
\esp\left[N\left(t\right)\right]=\sum_{n=1}^{\infty}F^{n\star}\left(t\right)
\end{eqnarray*}

\begin{Prop}
Para cada $t\geq0$, la funci\'on generadora de momentos $\esp\left[e^{\alpha N\left(t\right)}\right]$ existe para alguna $\alpha$ en una vecindad del 0, y de aqu\'i que $\esp\left[N\left(t\right)^{m}\right]<\infty$, para $m\geq1$.
\end{Prop}


\begin{Note}
Si el primer tiempo de renovaci\'on $\xi_{1}$ no tiene la misma distribuci\'on que el resto de las $\xi_{n}$, para $n\geq2$, a $N\left(t\right)$ se le llama Proceso de Renovaci\'on retardado, donde si $\xi$ tiene distribuci\'on $G$, entonces el tiempo $T_{n}$ de la $n$-\'esima renovaci\'on tiene distribuci\'on $G\star F^{\left(n-1\right)\star}\left(t\right)$
\end{Note}


\begin{Teo}
Para una constante $\mu\leq\infty$ ( o variable aleatoria), las siguientes expresiones son equivalentes:

\begin{eqnarray}
lim_{n\rightarrow\infty}n^{-1}T_{n}&=&\mu,\textrm{ c.s.}\\
lim_{t\rightarrow\infty}t^{-1}N\left(t\right)&=&1/\mu,\textrm{ c.s.}
\end{eqnarray}
\end{Teo}


Es decir, $T_{n}$ satisface la Ley Fuerte de los Grandes N\'umeros s\'i y s\'olo s\'i $N\left/t\right)$ la cumple.


\begin{Coro}[Ley Fuerte de los Grandes N\'umeros para Procesos de Renovaci\'on]
Si $N\left(t\right)$ es un proceso de renovaci\'on cuyos tiempos de inter-renovaci\'on tienen media $\mu\leq\infty$, entonces
\begin{eqnarray}
t^{-1}N\left(t\right)\rightarrow 1/\mu,\textrm{ c.s. cuando }t\rightarrow\infty.
\end{eqnarray}

\end{Coro}


Considerar el proceso estoc\'astico de valores reales $\left\{Z\left(t\right):t\geq0\right\}$ en el mismo espacio de probabilidad que $N\left(t\right)$

\begin{Def}
Para el proceso $\left\{Z\left(t\right):t\geq0\right\}$ se define la fluctuaci\'on m\'axima de $Z\left(t\right)$ en el intervalo $\left(T_{n-1},T_{n}\right]$:
\begin{eqnarray*}
M_{n}=\sup_{T_{n-1}<t\leq T_{n}}|Z\left(t\right)-Z\left(T_{n-1}\right)|
\end{eqnarray*}
\end{Def}

\begin{Teo}
Sup\'ongase que $n^{-1}T_{n}\rightarrow\mu$ c.s. cuando $n\rightarrow\infty$, donde $\mu\leq\infty$ es una constante o variable aleatoria. Sea $a$ una constante o variable aleatoria que puede ser infinita cuando $\mu$ es finita, y considere las expresiones l\'imite:
\begin{eqnarray}
lim_{n\rightarrow\infty}n^{-1}Z\left(T_{n}\right)&=&a,\textrm{ c.s.}\\
lim_{t\rightarrow\infty}t^{-1}Z\left(t\right)&=&a/\mu,\textrm{ c.s.}
\end{eqnarray}
La segunda expresi\'on implica la primera. Conversamente, la primera implica la segunda si el proceso $Z\left(t\right)$ es creciente, o si $lim_{n\rightarrow\infty}n^{-1}M_{n}=0$ c.s.
\end{Teo}

\begin{Coro}
Si $N\left(t\right)$ es un proceso de renovaci\'on, y $\left(Z\left(T_{n}\right)-Z\left(T_{n-1}\right),M_{n}\right)$, para $n\geq1$, son variables aleatorias independientes e id\'enticamente distribuidas con media finita, entonces,
\begin{eqnarray}
lim_{t\rightarrow\infty}t^{-1}Z\left(t\right)\rightarrow\frac{\esp\left[Z\left(T_{1}\right)-Z\left(T_{0}\right)\right]}{\esp\left[T_{1}\right]},\textrm{ c.s. cuando  }t\rightarrow\infty.
\end{eqnarray}
\end{Coro}

%___________________________________________________________________________________________
%
%\subsection{Propiedades de los Procesos de Renovaci\'on}
%___________________________________________________________________________________________
%

Los tiempos $T_{n}$ est\'an relacionados con los conteos de $N\left(t\right)$ por

\begin{eqnarray*}
\left\{N\left(t\right)\geq n\right\}&=&\left\{T_{n}\leq t\right\}\\
T_{N\left(t\right)}\leq &t&<T_{N\left(t\right)+1},
\end{eqnarray*}

adem\'as $N\left(T_{n}\right)=n$, y 

\begin{eqnarray*}
N\left(t\right)=\max\left\{n:T_{n}\leq t\right\}=\min\left\{n:T_{n+1}>t\right\}
\end{eqnarray*}

Por propiedades de la convoluci\'on se sabe que

\begin{eqnarray*}
P\left\{T_{n}\leq t\right\}=F^{n\star}\left(t\right)
\end{eqnarray*}
que es la $n$-\'esima convoluci\'on de $F$. Entonces 

\begin{eqnarray*}
\left\{N\left(t\right)\geq n\right\}&=&\left\{T_{n}\leq t\right\}\\
P\left\{N\left(t\right)\leq n\right\}&=&1-F^{\left(n+1\right)\star}\left(t\right)
\end{eqnarray*}

Adem\'as usando el hecho de que $\esp\left[N\left(t\right)\right]=\sum_{n=1}^{\infty}P\left\{N\left(t\right)\geq n\right\}$
se tiene que

\begin{eqnarray*}
\esp\left[N\left(t\right)\right]=\sum_{n=1}^{\infty}F^{n\star}\left(t\right)
\end{eqnarray*}

\begin{Prop}
Para cada $t\geq0$, la funci\'on generadora de momentos $\esp\left[e^{\alpha N\left(t\right)}\right]$ existe para alguna $\alpha$ en una vecindad del 0, y de aqu\'i que $\esp\left[N\left(t\right)^{m}\right]<\infty$, para $m\geq1$.
\end{Prop}


\begin{Note}
Si el primer tiempo de renovaci\'on $\xi_{1}$ no tiene la misma distribuci\'on que el resto de las $\xi_{n}$, para $n\geq2$, a $N\left(t\right)$ se le llama Proceso de Renovaci\'on retardado, donde si $\xi$ tiene distribuci\'on $G$, entonces el tiempo $T_{n}$ de la $n$-\'esima renovaci\'on tiene distribuci\'on $G\star F^{\left(n-1\right)\star}\left(t\right)$
\end{Note}


\begin{Teo}
Para una constante $\mu\leq\infty$ ( o variable aleatoria), las siguientes expresiones son equivalentes:

\begin{eqnarray}
lim_{n\rightarrow\infty}n^{-1}T_{n}&=&\mu,\textrm{ c.s.}\\
lim_{t\rightarrow\infty}t^{-1}N\left(t\right)&=&1/\mu,\textrm{ c.s.}
\end{eqnarray}
\end{Teo}


Es decir, $T_{n}$ satisface la Ley Fuerte de los Grandes N\'umeros s\'i y s\'olo s\'i $N\left/t\right)$ la cumple.


\begin{Coro}[Ley Fuerte de los Grandes N\'umeros para Procesos de Renovaci\'on]
Si $N\left(t\right)$ es un proceso de renovaci\'on cuyos tiempos de inter-renovaci\'on tienen media $\mu\leq\infty$, entonces
\begin{eqnarray}
t^{-1}N\left(t\right)\rightarrow 1/\mu,\textrm{ c.s. cuando }t\rightarrow\infty.
\end{eqnarray}

\end{Coro}


Considerar el proceso estoc\'astico de valores reales $\left\{Z\left(t\right):t\geq0\right\}$ en el mismo espacio de probabilidad que $N\left(t\right)$

\begin{Def}
Para el proceso $\left\{Z\left(t\right):t\geq0\right\}$ se define la fluctuaci\'on m\'axima de $Z\left(t\right)$ en el intervalo $\left(T_{n-1},T_{n}\right]$:
\begin{eqnarray*}
M_{n}=\sup_{T_{n-1}<t\leq T_{n}}|Z\left(t\right)-Z\left(T_{n-1}\right)|
\end{eqnarray*}
\end{Def}

\begin{Teo}
Sup\'ongase que $n^{-1}T_{n}\rightarrow\mu$ c.s. cuando $n\rightarrow\infty$, donde $\mu\leq\infty$ es una constante o variable aleatoria. Sea $a$ una constante o variable aleatoria que puede ser infinita cuando $\mu$ es finita, y considere las expresiones l\'imite:
\begin{eqnarray}
lim_{n\rightarrow\infty}n^{-1}Z\left(T_{n}\right)&=&a,\textrm{ c.s.}\\
lim_{t\rightarrow\infty}t^{-1}Z\left(t\right)&=&a/\mu,\textrm{ c.s.}
\end{eqnarray}
La segunda expresi\'on implica la primera. Conversamente, la primera implica la segunda si el proceso $Z\left(t\right)$ es creciente, o si $lim_{n\rightarrow\infty}n^{-1}M_{n}=0$ c.s.
\end{Teo}

\begin{Coro}
Si $N\left(t\right)$ es un proceso de renovaci\'on, y $\left(Z\left(T_{n}\right)-Z\left(T_{n-1}\right),M_{n}\right)$, para $n\geq1$, son variables aleatorias independientes e id\'enticamente distribuidas con media finita, entonces,
\begin{eqnarray}
lim_{t\rightarrow\infty}t^{-1}Z\left(t\right)\rightarrow\frac{\esp\left[Z\left(T_{1}\right)-Z\left(T_{0}\right)\right]}{\esp\left[T_{1}\right]},\textrm{ c.s. cuando  }t\rightarrow\infty.
\end{eqnarray}
\end{Coro}
%___________________________________________________________________________________________
%
%\subsection{Propiedades de los Procesos de Renovaci\'on}
%___________________________________________________________________________________________
%

Los tiempos $T_{n}$ est\'an relacionados con los conteos de $N\left(t\right)$ por

\begin{eqnarray*}
\left\{N\left(t\right)\geq n\right\}&=&\left\{T_{n}\leq t\right\}\\
T_{N\left(t\right)}\leq &t&<T_{N\left(t\right)+1},
\end{eqnarray*}

adem\'as $N\left(T_{n}\right)=n$, y 

\begin{eqnarray*}
N\left(t\right)=\max\left\{n:T_{n}\leq t\right\}=\min\left\{n:T_{n+1}>t\right\}
\end{eqnarray*}

Por propiedades de la convoluci\'on se sabe que

\begin{eqnarray*}
P\left\{T_{n}\leq t\right\}=F^{n\star}\left(t\right)
\end{eqnarray*}
que es la $n$-\'esima convoluci\'on de $F$. Entonces 

\begin{eqnarray*}
\left\{N\left(t\right)\geq n\right\}&=&\left\{T_{n}\leq t\right\}\\
P\left\{N\left(t\right)\leq n\right\}&=&1-F^{\left(n+1\right)\star}\left(t\right)
\end{eqnarray*}

Adem\'as usando el hecho de que $\esp\left[N\left(t\right)\right]=\sum_{n=1}^{\infty}P\left\{N\left(t\right)\geq n\right\}$
se tiene que

\begin{eqnarray*}
\esp\left[N\left(t\right)\right]=\sum_{n=1}^{\infty}F^{n\star}\left(t\right)
\end{eqnarray*}

\begin{Prop}
Para cada $t\geq0$, la funci\'on generadora de momentos $\esp\left[e^{\alpha N\left(t\right)}\right]$ existe para alguna $\alpha$ en una vecindad del 0, y de aqu\'i que $\esp\left[N\left(t\right)^{m}\right]<\infty$, para $m\geq1$.
\end{Prop}


\begin{Note}
Si el primer tiempo de renovaci\'on $\xi_{1}$ no tiene la misma distribuci\'on que el resto de las $\xi_{n}$, para $n\geq2$, a $N\left(t\right)$ se le llama Proceso de Renovaci\'on retardado, donde si $\xi$ tiene distribuci\'on $G$, entonces el tiempo $T_{n}$ de la $n$-\'esima renovaci\'on tiene distribuci\'on $G\star F^{\left(n-1\right)\star}\left(t\right)$
\end{Note}


\begin{Teo}
Para una constante $\mu\leq\infty$ ( o variable aleatoria), las siguientes expresiones son equivalentes:

\begin{eqnarray}
lim_{n\rightarrow\infty}n^{-1}T_{n}&=&\mu,\textrm{ c.s.}\\
lim_{t\rightarrow\infty}t^{-1}N\left(t\right)&=&1/\mu,\textrm{ c.s.}
\end{eqnarray}
\end{Teo}


Es decir, $T_{n}$ satisface la Ley Fuerte de los Grandes N\'umeros s\'i y s\'olo s\'i $N\left/t\right)$ la cumple.


\begin{Coro}[Ley Fuerte de los Grandes N\'umeros para Procesos de Renovaci\'on]
Si $N\left(t\right)$ es un proceso de renovaci\'on cuyos tiempos de inter-renovaci\'on tienen media $\mu\leq\infty$, entonces
\begin{eqnarray}
t^{-1}N\left(t\right)\rightarrow 1/\mu,\textrm{ c.s. cuando }t\rightarrow\infty.
\end{eqnarray}

\end{Coro}


Considerar el proceso estoc\'astico de valores reales $\left\{Z\left(t\right):t\geq0\right\}$ en el mismo espacio de probabilidad que $N\left(t\right)$

\begin{Def}
Para el proceso $\left\{Z\left(t\right):t\geq0\right\}$ se define la fluctuaci\'on m\'axima de $Z\left(t\right)$ en el intervalo $\left(T_{n-1},T_{n}\right]$:
\begin{eqnarray*}
M_{n}=\sup_{T_{n-1}<t\leq T_{n}}|Z\left(t\right)-Z\left(T_{n-1}\right)|
\end{eqnarray*}
\end{Def}

\begin{Teo}
Sup\'ongase que $n^{-1}T_{n}\rightarrow\mu$ c.s. cuando $n\rightarrow\infty$, donde $\mu\leq\infty$ es una constante o variable aleatoria. Sea $a$ una constante o variable aleatoria que puede ser infinita cuando $\mu$ es finita, y considere las expresiones l\'imite:
\begin{eqnarray}
lim_{n\rightarrow\infty}n^{-1}Z\left(T_{n}\right)&=&a,\textrm{ c.s.}\\
lim_{t\rightarrow\infty}t^{-1}Z\left(t\right)&=&a/\mu,\textrm{ c.s.}
\end{eqnarray}
La segunda expresi\'on implica la primera. Conversamente, la primera implica la segunda si el proceso $Z\left(t\right)$ es creciente, o si $lim_{n\rightarrow\infty}n^{-1}M_{n}=0$ c.s.
\end{Teo}

\begin{Coro}
Si $N\left(t\right)$ es un proceso de renovaci\'on, y $\left(Z\left(T_{n}\right)-Z\left(T_{n-1}\right),M_{n}\right)$, para $n\geq1$, son variables aleatorias independientes e id\'enticamente distribuidas con media finita, entonces,
\begin{eqnarray}
lim_{t\rightarrow\infty}t^{-1}Z\left(t\right)\rightarrow\frac{\esp\left[Z\left(T_{1}\right)-Z\left(T_{0}\right)\right]}{\esp\left[T_{1}\right]},\textrm{ c.s. cuando  }t\rightarrow\infty.
\end{eqnarray}
\end{Coro}


%___________________________________________________________________________________________
%
%\subsection{Funci\'on de Renovaci\'on}
%___________________________________________________________________________________________
%


\begin{Def}
Sea $h\left(t\right)$ funci\'on de valores reales en $\rea$ acotada en intervalos finitos e igual a cero para $t<0$ La ecuaci\'on de renovaci\'on para $h\left(t\right)$ y la distribuci\'on $F$ es

\begin{eqnarray}%\label{Ec.Renovacion}
H\left(t\right)=h\left(t\right)+\int_{\left[0,t\right]}H\left(t-s\right)dF\left(s\right)\textrm{,    }t\geq0,
\end{eqnarray}
donde $H\left(t\right)$ es una funci\'on de valores reales. Esto es $H=h+F\star H$. Decimos que $H\left(t\right)$ es soluci\'on de esta ecuaci\'on si satisface la ecuaci\'on, y es acotada en intervalos finitos e iguales a cero para $t<0$.
\end{Def}

\begin{Prop}
La funci\'on $U\star h\left(t\right)$ es la \'unica soluci\'on de la ecuaci\'on de renovaci\'on (\ref{Ec.Renovacion}).
\end{Prop}

\begin{Teo}[Teorema Renovaci\'on Elemental]
\begin{eqnarray*}
t^{-1}U\left(t\right)\rightarrow 1/\mu\textrm{,    cuando }t\rightarrow\infty.
\end{eqnarray*}
\end{Teo}

%___________________________________________________________________________________________
%
%\subsection{Funci\'on de Renovaci\'on}
%___________________________________________________________________________________________
%


Sup\'ongase que $N\left(t\right)$ es un proceso de renovaci\'on con distribuci\'on $F$ con media finita $\mu$.

\begin{Def}
La funci\'on de renovaci\'on asociada con la distribuci\'on $F$, del proceso $N\left(t\right)$, es
\begin{eqnarray*}
U\left(t\right)=\sum_{n=1}^{\infty}F^{n\star}\left(t\right),\textrm{   }t\geq0,
\end{eqnarray*}
donde $F^{0\star}\left(t\right)=\indora\left(t\geq0\right)$.
\end{Def}


\begin{Prop}
Sup\'ongase que la distribuci\'on de inter-renovaci\'on $F$ tiene densidad $f$. Entonces $U\left(t\right)$ tambi\'en tiene densidad, para $t>0$, y es $U^{'}\left(t\right)=\sum_{n=0}^{\infty}f^{n\star}\left(t\right)$. Adem\'as
\begin{eqnarray*}
\prob\left\{N\left(t\right)>N\left(t-\right)\right\}=0\textrm{,   }t\geq0.
\end{eqnarray*}
\end{Prop}

\begin{Def}
La Transformada de Laplace-Stieljes de $F$ est\'a dada por

\begin{eqnarray*}
\hat{F}\left(\alpha\right)=\int_{\rea_{+}}e^{-\alpha t}dF\left(t\right)\textrm{,  }\alpha\geq0.
\end{eqnarray*}
\end{Def}

Entonces

\begin{eqnarray*}
\hat{U}\left(\alpha\right)=\sum_{n=0}^{\infty}\hat{F^{n\star}}\left(\alpha\right)=\sum_{n=0}^{\infty}\hat{F}\left(\alpha\right)^{n}=\frac{1}{1-\hat{F}\left(\alpha\right)}.
\end{eqnarray*}


\begin{Prop}
La Transformada de Laplace $\hat{U}\left(\alpha\right)$ y $\hat{F}\left(\alpha\right)$ determina una a la otra de manera \'unica por la relaci\'on $\hat{U}\left(\alpha\right)=\frac{1}{1-\hat{F}\left(\alpha\right)}$.
\end{Prop}


\begin{Note}
Un proceso de renovaci\'on $N\left(t\right)$ cuyos tiempos de inter-renovaci\'on tienen media finita, es un proceso Poisson con tasa $\lambda$ si y s\'olo s\'i $\esp\left[U\left(t\right)\right]=\lambda t$, para $t\geq0$.
\end{Note}


\begin{Teo}
Sea $N\left(t\right)$ un proceso puntual simple con puntos de localizaci\'on $T_{n}$ tal que $\eta\left(t\right)=\esp\left[N\left(\right)\right]$ es finita para cada $t$. Entonces para cualquier funci\'on $f:\rea_{+}\rightarrow\rea$,
\begin{eqnarray*}
\esp\left[\sum_{n=1}^{N\left(\right)}f\left(T_{n}\right)\right]=\int_{\left(0,t\right]}f\left(s\right)d\eta\left(s\right)\textrm{,  }t\geq0,
\end{eqnarray*}
suponiendo que la integral exista. Adem\'as si $X_{1},X_{2},\ldots$ son variables aleatorias definidas en el mismo espacio de probabilidad que el proceso $N\left(t\right)$ tal que $\esp\left[X_{n}|T_{n}=s\right]=f\left(s\right)$, independiente de $n$. Entonces
\begin{eqnarray*}
\esp\left[\sum_{n=1}^{N\left(t\right)}X_{n}\right]=\int_{\left(0,t\right]}f\left(s\right)d\eta\left(s\right)\textrm{,  }t\geq0,
\end{eqnarray*} 
suponiendo que la integral exista. 
\end{Teo}

\begin{Coro}[Identidad de Wald para Renovaciones]
Para el proceso de renovaci\'on $N\left(t\right)$,
\begin{eqnarray*}
\esp\left[T_{N\left(t\right)+1}\right]=\mu\esp\left[N\left(t\right)+1\right]\textrm{,  }t\geq0,
\end{eqnarray*}  
\end{Coro}

%______________________________________________________________________
%\subsection{Procesos de Renovaci\'on}
%______________________________________________________________________

\begin{Def}%\label{Def.Tn}
Sean $0\leq T_{1}\leq T_{2}\leq \ldots$ son tiempos aleatorios infinitos en los cuales ocurren ciertos eventos. El n\'umero de tiempos $T_{n}$ en el intervalo $\left[0,t\right)$ es

\begin{eqnarray}
N\left(t\right)=\sum_{n=1}^{\infty}\indora\left(T_{n}\leq t\right),
\end{eqnarray}
para $t\geq0$.
\end{Def}

Si se consideran los puntos $T_{n}$ como elementos de $\rea_{+}$, y $N\left(t\right)$ es el n\'umero de puntos en $\rea$. El proceso denotado por $\left\{N\left(t\right):t\geq0\right\}$, denotado por $N\left(t\right)$, es un proceso puntual en $\rea_{+}$. Los $T_{n}$ son los tiempos de ocurrencia, el proceso puntual $N\left(t\right)$ es simple si su n\'umero de ocurrencias son distintas: $0<T_{1}<T_{2}<\ldots$ casi seguramente.

\begin{Def}
Un proceso puntual $N\left(t\right)$ es un proceso de renovaci\'on si los tiempos de interocurrencia $\xi_{n}=T_{n}-T_{n-1}$, para $n\geq1$, son independientes e identicamente distribuidos con distribuci\'on $F$, donde $F\left(0\right)=0$ y $T_{0}=0$. Los $T_{n}$ son llamados tiempos de renovaci\'on, referente a la independencia o renovaci\'on de la informaci\'on estoc\'astica en estos tiempos. Los $\xi_{n}$ son los tiempos de inter-renovaci\'on, y $N\left(t\right)$ es el n\'umero de renovaciones en el intervalo $\left[0,t\right)$
\end{Def}


\begin{Note}
Para definir un proceso de renovaci\'on para cualquier contexto, solamente hay que especificar una distribuci\'on $F$, con $F\left(0\right)=0$, para los tiempos de inter-renovaci\'on. La funci\'on $F$ en turno degune las otra variables aleatorias. De manera formal, existe un espacio de probabilidad y una sucesi\'on de variables aleatorias $\xi_{1},\xi_{2},\ldots$ definidas en este con distribuci\'on $F$. Entonces las otras cantidades son $T_{n}=\sum_{k=1}^{n}\xi_{k}$ y $N\left(t\right)=\sum_{n=1}^{\infty}\indora\left(T_{n}\leq t\right)$, donde $T_{n}\rightarrow\infty$ casi seguramente por la Ley Fuerte de los Grandes Números.
\end{Note}

%___________________________________________________________________________________________
%
%\subsection{Renewal and Regenerative Processes: Serfozo\cite{Serfozo}}
%___________________________________________________________________________________________
%
\begin{Def}%\label{Def.Tn}
Sean $0\leq T_{1}\leq T_{2}\leq \ldots$ son tiempos aleatorios infinitos en los cuales ocurren ciertos eventos. El n\'umero de tiempos $T_{n}$ en el intervalo $\left[0,t\right)$ es

\begin{eqnarray}
N\left(t\right)=\sum_{n=1}^{\infty}\indora\left(T_{n}\leq t\right),
\end{eqnarray}
para $t\geq0$.
\end{Def}

Si se consideran los puntos $T_{n}$ como elementos de $\rea_{+}$, y $N\left(t\right)$ es el n\'umero de puntos en $\rea$. El proceso denotado por $\left\{N\left(t\right):t\geq0\right\}$, denotado por $N\left(t\right)$, es un proceso puntual en $\rea_{+}$. Los $T_{n}$ son los tiempos de ocurrencia, el proceso puntual $N\left(t\right)$ es simple si su n\'umero de ocurrencias son distintas: $0<T_{1}<T_{2}<\ldots$ casi seguramente.

\begin{Def}
Un proceso puntual $N\left(t\right)$ es un proceso de renovaci\'on si los tiempos de interocurrencia $\xi_{n}=T_{n}-T_{n-1}$, para $n\geq1$, son independientes e identicamente distribuidos con distribuci\'on $F$, donde $F\left(0\right)=0$ y $T_{0}=0$. Los $T_{n}$ son llamados tiempos de renovaci\'on, referente a la independencia o renovaci\'on de la informaci\'on estoc\'astica en estos tiempos. Los $\xi_{n}$ son los tiempos de inter-renovaci\'on, y $N\left(t\right)$ es el n\'umero de renovaciones en el intervalo $\left[0,t\right)$
\end{Def}


\begin{Note}
Para definir un proceso de renovaci\'on para cualquier contexto, solamente hay que especificar una distribuci\'on $F$, con $F\left(0\right)=0$, para los tiempos de inter-renovaci\'on. La funci\'on $F$ en turno degune las otra variables aleatorias. De manera formal, existe un espacio de probabilidad y una sucesi\'on de variables aleatorias $\xi_{1},\xi_{2},\ldots$ definidas en este con distribuci\'on $F$. Entonces las otras cantidades son $T_{n}=\sum_{k=1}^{n}\xi_{k}$ y $N\left(t\right)=\sum_{n=1}^{\infty}\indora\left(T_{n}\leq t\right)$, donde $T_{n}\rightarrow\infty$ casi seguramente por la Ley Fuerte de los Grandes N\'umeros.
\end{Note}







Los tiempos $T_{n}$ est\'an relacionados con los conteos de $N\left(t\right)$ por

\begin{eqnarray*}
\left\{N\left(t\right)\geq n\right\}&=&\left\{T_{n}\leq t\right\}\\
T_{N\left(t\right)}\leq &t&<T_{N\left(t\right)+1},
\end{eqnarray*}

adem\'as $N\left(T_{n}\right)=n$, y 

\begin{eqnarray*}
N\left(t\right)=\max\left\{n:T_{n}\leq t\right\}=\min\left\{n:T_{n+1}>t\right\}
\end{eqnarray*}

Por propiedades de la convoluci\'on se sabe que

\begin{eqnarray*}
P\left\{T_{n}\leq t\right\}=F^{n\star}\left(t\right)
\end{eqnarray*}
que es la $n$-\'esima convoluci\'on de $F$. Entonces 

\begin{eqnarray*}
\left\{N\left(t\right)\geq n\right\}&=&\left\{T_{n}\leq t\right\}\\
P\left\{N\left(t\right)\leq n\right\}&=&1-F^{\left(n+1\right)\star}\left(t\right)
\end{eqnarray*}

Adem\'as usando el hecho de que $\esp\left[N\left(t\right)\right]=\sum_{n=1}^{\infty}P\left\{N\left(t\right)\geq n\right\}$
se tiene que

\begin{eqnarray*}
\esp\left[N\left(t\right)\right]=\sum_{n=1}^{\infty}F^{n\star}\left(t\right)
\end{eqnarray*}

\begin{Prop}
Para cada $t\geq0$, la funci\'on generadora de momentos $\esp\left[e^{\alpha N\left(t\right)}\right]$ existe para alguna $\alpha$ en una vecindad del 0, y de aqu\'i que $\esp\left[N\left(t\right)^{m}\right]<\infty$, para $m\geq1$.
\end{Prop}

\begin{Ejem}[\textbf{Proceso Poisson}]

Suponga que se tienen tiempos de inter-renovaci\'on \textit{i.i.d.} del proceso de renovaci\'on $N\left(t\right)$ tienen distribuci\'on exponencial $F\left(t\right)=q-e^{-\lambda t}$ con tasa $\lambda$. Entonces $N\left(t\right)$ es un proceso Poisson con tasa $\lambda$.

\end{Ejem}


\begin{Note}
Si el primer tiempo de renovaci\'on $\xi_{1}$ no tiene la misma distribuci\'on que el resto de las $\xi_{n}$, para $n\geq2$, a $N\left(t\right)$ se le llama Proceso de Renovaci\'on retardado, donde si $\xi$ tiene distribuci\'on $G$, entonces el tiempo $T_{n}$ de la $n$-\'esima renovaci\'on tiene distribuci\'on $G\star F^{\left(n-1\right)\star}\left(t\right)$
\end{Note}


\begin{Teo}
Para una constante $\mu\leq\infty$ ( o variable aleatoria), las siguientes expresiones son equivalentes:

\begin{eqnarray}
lim_{n\rightarrow\infty}n^{-1}T_{n}&=&\mu,\textrm{ c.s.}\\
lim_{t\rightarrow\infty}t^{-1}N\left(t\right)&=&1/\mu,\textrm{ c.s.}
\end{eqnarray}
\end{Teo}


Es decir, $T_{n}$ satisface la Ley Fuerte de los Grandes N\'umeros s\'i y s\'olo s\'i $N\left/t\right)$ la cumple.


\begin{Coro}[Ley Fuerte de los Grandes N\'umeros para Procesos de Renovaci\'on]
Si $N\left(t\right)$ es un proceso de renovaci\'on cuyos tiempos de inter-renovaci\'on tienen media $\mu\leq\infty$, entonces
\begin{eqnarray}
t^{-1}N\left(t\right)\rightarrow 1/\mu,\textrm{ c.s. cuando }t\rightarrow\infty.
\end{eqnarray}

\end{Coro}


Considerar el proceso estoc\'astico de valores reales $\left\{Z\left(t\right):t\geq0\right\}$ en el mismo espacio de probabilidad que $N\left(t\right)$

\begin{Def}
Para el proceso $\left\{Z\left(t\right):t\geq0\right\}$ se define la fluctuaci\'on m\'axima de $Z\left(t\right)$ en el intervalo $\left(T_{n-1},T_{n}\right]$:
\begin{eqnarray*}
M_{n}=\sup_{T_{n-1}<t\leq T_{n}}|Z\left(t\right)-Z\left(T_{n-1}\right)|
\end{eqnarray*}
\end{Def}

\begin{Teo}
Sup\'ongase que $n^{-1}T_{n}\rightarrow\mu$ c.s. cuando $n\rightarrow\infty$, donde $\mu\leq\infty$ es una constante o variable aleatoria. Sea $a$ una constante o variable aleatoria que puede ser infinita cuando $\mu$ es finita, y considere las expresiones l\'imite:
\begin{eqnarray}
lim_{n\rightarrow\infty}n^{-1}Z\left(T_{n}\right)&=&a,\textrm{ c.s.}\\
lim_{t\rightarrow\infty}t^{-1}Z\left(t\right)&=&a/\mu,\textrm{ c.s.}
\end{eqnarray}
La segunda expresi\'on implica la primera. Conversamente, la primera implica la segunda si el proceso $Z\left(t\right)$ es creciente, o si $lim_{n\rightarrow\infty}n^{-1}M_{n}=0$ c.s.
\end{Teo}

\begin{Coro}
Si $N\left(t\right)$ es un proceso de renovaci\'on, y $\left(Z\left(T_{n}\right)-Z\left(T_{n-1}\right),M_{n}\right)$, para $n\geq1$, son variables aleatorias independientes e id\'enticamente distribuidas con media finita, entonces,
\begin{eqnarray}
lim_{t\rightarrow\infty}t^{-1}Z\left(t\right)\rightarrow\frac{\esp\left[Z\left(T_{1}\right)-Z\left(T_{0}\right)\right]}{\esp\left[T_{1}\right]},\textrm{ c.s. cuando  }t\rightarrow\infty.
\end{eqnarray}
\end{Coro}


Sup\'ongase que $N\left(t\right)$ es un proceso de renovaci\'on con distribuci\'on $F$ con media finita $\mu$.

\begin{Def}
La funci\'on de renovaci\'on asociada con la distribuci\'on $F$, del proceso $N\left(t\right)$, es
\begin{eqnarray*}
U\left(t\right)=\sum_{n=1}^{\infty}F^{n\star}\left(t\right),\textrm{   }t\geq0,
\end{eqnarray*}
donde $F^{0\star}\left(t\right)=\indora\left(t\geq0\right)$.
\end{Def}


\begin{Prop}
Sup\'ongase que la distribuci\'on de inter-renovaci\'on $F$ tiene densidad $f$. Entonces $U\left(t\right)$ tambi\'en tiene densidad, para $t>0$, y es $U^{'}\left(t\right)=\sum_{n=0}^{\infty}f^{n\star}\left(t\right)$. Adem\'as
\begin{eqnarray*}
\prob\left\{N\left(t\right)>N\left(t-\right)\right\}=0\textrm{,   }t\geq0.
\end{eqnarray*}
\end{Prop}

\begin{Def}
La Transformada de Laplace-Stieljes de $F$ est\'a dada por

\begin{eqnarray*}
\hat{F}\left(\alpha\right)=\int_{\rea_{+}}e^{-\alpha t}dF\left(t\right)\textrm{,  }\alpha\geq0.
\end{eqnarray*}
\end{Def}

Entonces

\begin{eqnarray*}
\hat{U}\left(\alpha\right)=\sum_{n=0}^{\infty}\hat{F^{n\star}}\left(\alpha\right)=\sum_{n=0}^{\infty}\hat{F}\left(\alpha\right)^{n}=\frac{1}{1-\hat{F}\left(\alpha\right)}.
\end{eqnarray*}


\begin{Prop}
La Transformada de Laplace $\hat{U}\left(\alpha\right)$ y $\hat{F}\left(\alpha\right)$ determina una a la otra de manera \'unica por la relaci\'on $\hat{U}\left(\alpha\right)=\frac{1}{1-\hat{F}\left(\alpha\right)}$.
\end{Prop}


\begin{Note}
Un proceso de renovaci\'on $N\left(t\right)$ cuyos tiempos de inter-renovaci\'on tienen media finita, es un proceso Poisson con tasa $\lambda$ si y s\'olo s\'i $\esp\left[U\left(t\right)\right]=\lambda t$, para $t\geq0$.
\end{Note}


\begin{Teo}
Sea $N\left(t\right)$ un proceso puntual simple con puntos de localizaci\'on $T_{n}$ tal que $\eta\left(t\right)=\esp\left[N\left(\right)\right]$ es finita para cada $t$. Entonces para cualquier funci\'on $f:\rea_{+}\rightarrow\rea$,
\begin{eqnarray*}
\esp\left[\sum_{n=1}^{N\left(\right)}f\left(T_{n}\right)\right]=\int_{\left(0,t\right]}f\left(s\right)d\eta\left(s\right)\textrm{,  }t\geq0,
\end{eqnarray*}
suponiendo que la integral exista. Adem\'as si $X_{1},X_{2},\ldots$ son variables aleatorias definidas en el mismo espacio de probabilidad que el proceso $N\left(t\right)$ tal que $\esp\left[X_{n}|T_{n}=s\right]=f\left(s\right)$, independiente de $n$. Entonces
\begin{eqnarray*}
\esp\left[\sum_{n=1}^{N\left(t\right)}X_{n}\right]=\int_{\left(0,t\right]}f\left(s\right)d\eta\left(s\right)\textrm{,  }t\geq0,
\end{eqnarray*} 
suponiendo que la integral exista. 
\end{Teo}

\begin{Coro}[Identidad de Wald para Renovaciones]
Para el proceso de renovaci\'on $N\left(t\right)$,
\begin{eqnarray*}
\esp\left[T_{N\left(t\right)+1}\right]=\mu\esp\left[N\left(t\right)+1\right]\textrm{,  }t\geq0,
\end{eqnarray*}  
\end{Coro}


\begin{Def}
Sea $h\left(t\right)$ funci\'on de valores reales en $\rea$ acotada en intervalos finitos e igual a cero para $t<0$ La ecuaci\'on de renovaci\'on para $h\left(t\right)$ y la distribuci\'on $F$ es

\begin{eqnarray}%\label{Ec.Renovacion}
H\left(t\right)=h\left(t\right)+\int_{\left[0,t\right]}H\left(t-s\right)dF\left(s\right)\textrm{,    }t\geq0,
\end{eqnarray}
donde $H\left(t\right)$ es una funci\'on de valores reales. Esto es $H=h+F\star H$. Decimos que $H\left(t\right)$ es soluci\'on de esta ecuaci\'on si satisface la ecuaci\'on, y es acotada en intervalos finitos e iguales a cero para $t<0$.
\end{Def}

\begin{Prop}
La funci\'on $U\star h\left(t\right)$ es la \'unica soluci\'on de la ecuaci\'on de renovaci\'on (\ref{Ec.Renovacion}).
\end{Prop}

\begin{Teo}[Teorema Renovaci\'on Elemental]
\begin{eqnarray*}
t^{-1}U\left(t\right)\rightarrow 1/\mu\textrm{,    cuando }t\rightarrow\infty.
\end{eqnarray*}
\end{Teo}



Sup\'ongase que $N\left(t\right)$ es un proceso de renovaci\'on con distribuci\'on $F$ con media finita $\mu$.

\begin{Def}
La funci\'on de renovaci\'on asociada con la distribuci\'on $F$, del proceso $N\left(t\right)$, es
\begin{eqnarray*}
U\left(t\right)=\sum_{n=1}^{\infty}F^{n\star}\left(t\right),\textrm{   }t\geq0,
\end{eqnarray*}
donde $F^{0\star}\left(t\right)=\indora\left(t\geq0\right)$.
\end{Def}


\begin{Prop}
Sup\'ongase que la distribuci\'on de inter-renovaci\'on $F$ tiene densidad $f$. Entonces $U\left(t\right)$ tambi\'en tiene densidad, para $t>0$, y es $U^{'}\left(t\right)=\sum_{n=0}^{\infty}f^{n\star}\left(t\right)$. Adem\'as
\begin{eqnarray*}
\prob\left\{N\left(t\right)>N\left(t-\right)\right\}=0\textrm{,   }t\geq0.
\end{eqnarray*}
\end{Prop}

\begin{Def}
La Transformada de Laplace-Stieljes de $F$ est\'a dada por

\begin{eqnarray*}
\hat{F}\left(\alpha\right)=\int_{\rea_{+}}e^{-\alpha t}dF\left(t\right)\textrm{,  }\alpha\geq0.
\end{eqnarray*}
\end{Def}

Entonces

\begin{eqnarray*}
\hat{U}\left(\alpha\right)=\sum_{n=0}^{\infty}\hat{F^{n\star}}\left(\alpha\right)=\sum_{n=0}^{\infty}\hat{F}\left(\alpha\right)^{n}=\frac{1}{1-\hat{F}\left(\alpha\right)}.
\end{eqnarray*}


\begin{Prop}
La Transformada de Laplace $\hat{U}\left(\alpha\right)$ y $\hat{F}\left(\alpha\right)$ determina una a la otra de manera \'unica por la relaci\'on $\hat{U}\left(\alpha\right)=\frac{1}{1-\hat{F}\left(\alpha\right)}$.
\end{Prop}


\begin{Note}
Un proceso de renovaci\'on $N\left(t\right)$ cuyos tiempos de inter-renovaci\'on tienen media finita, es un proceso Poisson con tasa $\lambda$ si y s\'olo s\'i $\esp\left[U\left(t\right)\right]=\lambda t$, para $t\geq0$.
\end{Note}


\begin{Teo}
Sea $N\left(t\right)$ un proceso puntual simple con puntos de localizaci\'on $T_{n}$ tal que $\eta\left(t\right)=\esp\left[N\left(\right)\right]$ es finita para cada $t$. Entonces para cualquier funci\'on $f:\rea_{+}\rightarrow\rea$,
\begin{eqnarray*}
\esp\left[\sum_{n=1}^{N\left(\right)}f\left(T_{n}\right)\right]=\int_{\left(0,t\right]}f\left(s\right)d\eta\left(s\right)\textrm{,  }t\geq0,
\end{eqnarray*}
suponiendo que la integral exista. Adem\'as si $X_{1},X_{2},\ldots$ son variables aleatorias definidas en el mismo espacio de probabilidad que el proceso $N\left(t\right)$ tal que $\esp\left[X_{n}|T_{n}=s\right]=f\left(s\right)$, independiente de $n$. Entonces
\begin{eqnarray*}
\esp\left[\sum_{n=1}^{N\left(t\right)}X_{n}\right]=\int_{\left(0,t\right]}f\left(s\right)d\eta\left(s\right)\textrm{,  }t\geq0,
\end{eqnarray*} 
suponiendo que la integral exista. 
\end{Teo}

\begin{Coro}[Identidad de Wald para Renovaciones]
Para el proceso de renovaci\'on $N\left(t\right)$,
\begin{eqnarray*}
\esp\left[T_{N\left(t\right)+1}\right]=\mu\esp\left[N\left(t\right)+1\right]\textrm{,  }t\geq0,
\end{eqnarray*}  
\end{Coro}


\begin{Def}
Sea $h\left(t\right)$ funci\'on de valores reales en $\rea$ acotada en intervalos finitos e igual a cero para $t<0$ La ecuaci\'on de renovaci\'on para $h\left(t\right)$ y la distribuci\'on $F$ es

\begin{eqnarray}%\label{Ec.Renovacion}
H\left(t\right)=h\left(t\right)+\int_{\left[0,t\right]}H\left(t-s\right)dF\left(s\right)\textrm{,    }t\geq0,
\end{eqnarray}
donde $H\left(t\right)$ es una funci\'on de valores reales. Esto es $H=h+F\star H$. Decimos que $H\left(t\right)$ es soluci\'on de esta ecuaci\'on si satisface la ecuaci\'on, y es acotada en intervalos finitos e iguales a cero para $t<0$.
\end{Def}

\begin{Prop}
La funci\'on $U\star h\left(t\right)$ es la \'unica soluci\'on de la ecuaci\'on de renovaci\'on (\ref{Ec.Renovacion}).
\end{Prop}

\begin{Teo}[Teorema Renovaci\'on Elemental]
\begin{eqnarray*}
t^{-1}U\left(t\right)\rightarrow 1/\mu\textrm{,    cuando }t\rightarrow\infty.
\end{eqnarray*}
\end{Teo}


\begin{Note} Una funci\'on $h:\rea_{+}\rightarrow\rea$ es Directamente Riemann Integrable en los siguientes casos:
\begin{itemize}
\item[a)] $h\left(t\right)\geq0$ es decreciente y Riemann Integrable.
\item[b)] $h$ es continua excepto posiblemente en un conjunto de Lebesgue de medida 0, y $|h\left(t\right)|\leq b\left(t\right)$, donde $b$ es DRI.
\end{itemize}
\end{Note}

\begin{Teo}[Teorema Principal de Renovaci\'on]
Si $F$ es no aritm\'etica y $h\left(t\right)$ es Directamente Riemann Integrable (DRI), entonces

\begin{eqnarray*}
lim_{t\rightarrow\infty}U\star h=\frac{1}{\mu}\int_{\rea_{+}}h\left(s\right)ds.
\end{eqnarray*}
\end{Teo}

\begin{Prop}
Cualquier funci\'on $H\left(t\right)$ acotada en intervalos finitos y que es 0 para $t<0$ puede expresarse como
\begin{eqnarray*}
H\left(t\right)=U\star h\left(t\right)\textrm{,  donde }h\left(t\right)=H\left(t\right)-F\star H\left(t\right)
\end{eqnarray*}
\end{Prop}

\begin{Def}
Un proceso estoc\'astico $X\left(t\right)$ es crudamente regenerativo en un tiempo aleatorio positivo $T$ si
\begin{eqnarray*}
\esp\left[X\left(T+t\right)|T\right]=\esp\left[X\left(t\right)\right]\textrm{, para }t\geq0,\end{eqnarray*}
y con las esperanzas anteriores finitas.
\end{Def}

\begin{Prop}
Sup\'ongase que $X\left(t\right)$ es un proceso crudamente regenerativo en $T$, que tiene distribuci\'on $F$. Si $\esp\left[X\left(t\right)\right]$ es acotado en intervalos finitos, entonces
\begin{eqnarray*}
\esp\left[X\left(t\right)\right]=U\star h\left(t\right)\textrm{,  donde }h\left(t\right)=\esp\left[X\left(t\right)\indora\left(T>t\right)\right].
\end{eqnarray*}
\end{Prop}

\begin{Teo}[Regeneraci\'on Cruda]
Sup\'ongase que $X\left(t\right)$ es un proceso con valores positivo crudamente regenerativo en $T$, y def\'inase $M=\sup\left\{|X\left(t\right)|:t\leq T\right\}$. Si $T$ es no aritm\'etico y $M$ y $MT$ tienen media finita, entonces
\begin{eqnarray*}
lim_{t\rightarrow\infty}\esp\left[X\left(t\right)\right]=\frac{1}{\mu}\int_{\rea_{+}}h\left(s\right)ds,
\end{eqnarray*}
donde $h\left(t\right)=\esp\left[X\left(t\right)\indora\left(T>t\right)\right]$.
\end{Teo}


\begin{Note} Una funci\'on $h:\rea_{+}\rightarrow\rea$ es Directamente Riemann Integrable en los siguientes casos:
\begin{itemize}
\item[a)] $h\left(t\right)\geq0$ es decreciente y Riemann Integrable.
\item[b)] $h$ es continua excepto posiblemente en un conjunto de Lebesgue de medida 0, y $|h\left(t\right)|\leq b\left(t\right)$, donde $b$ es DRI.
\end{itemize}
\end{Note}

\begin{Teo}[Teorema Principal de Renovaci\'on]
Si $F$ es no aritm\'etica y $h\left(t\right)$ es Directamente Riemann Integrable (DRI), entonces

\begin{eqnarray*}
lim_{t\rightarrow\infty}U\star h=\frac{1}{\mu}\int_{\rea_{+}}h\left(s\right)ds.
\end{eqnarray*}
\end{Teo}

\begin{Prop}
Cualquier funci\'on $H\left(t\right)$ acotada en intervalos finitos y que es 0 para $t<0$ puede expresarse como
\begin{eqnarray*}
H\left(t\right)=U\star h\left(t\right)\textrm{,  donde }h\left(t\right)=H\left(t\right)-F\star H\left(t\right)
\end{eqnarray*}
\end{Prop}

\begin{Def}
Un proceso estoc\'astico $X\left(t\right)$ es crudamente regenerativo en un tiempo aleatorio positivo $T$ si
\begin{eqnarray*}
\esp\left[X\left(T+t\right)|T\right]=\esp\left[X\left(t\right)\right]\textrm{, para }t\geq0,\end{eqnarray*}
y con las esperanzas anteriores finitas.
\end{Def}

\begin{Prop}
Sup\'ongase que $X\left(t\right)$ es un proceso crudamente regenerativo en $T$, que tiene distribuci\'on $F$. Si $\esp\left[X\left(t\right)\right]$ es acotado en intervalos finitos, entonces
\begin{eqnarray*}
\esp\left[X\left(t\right)\right]=U\star h\left(t\right)\textrm{,  donde }h\left(t\right)=\esp\left[X\left(t\right)\indora\left(T>t\right)\right].
\end{eqnarray*}
\end{Prop}

\begin{Teo}[Regeneraci\'on Cruda]
Sup\'ongase que $X\left(t\right)$ es un proceso con valores positivo crudamente regenerativo en $T$, y def\'inase $M=\sup\left\{|X\left(t\right)|:t\leq T\right\}$. Si $T$ es no aritm\'etico y $M$ y $MT$ tienen media finita, entonces
\begin{eqnarray*}
lim_{t\rightarrow\infty}\esp\left[X\left(t\right)\right]=\frac{1}{\mu}\int_{\rea_{+}}h\left(s\right)ds,
\end{eqnarray*}
donde $h\left(t\right)=\esp\left[X\left(t\right)\indora\left(T>t\right)\right]$.
\end{Teo}

\begin{Def}
Para el proceso $\left\{\left(N\left(t\right),X\left(t\right)\right):t\geq0\right\}$, sus trayectoria muestrales en el intervalo de tiempo $\left[T_{n-1},T_{n}\right)$ est\'an descritas por
\begin{eqnarray*}
\zeta_{n}=\left(\xi_{n},\left\{X\left(T_{n-1}+t\right):0\leq t<\xi_{n}\right\}\right)
\end{eqnarray*}
Este $\zeta_{n}$ es el $n$-\'esimo segmento del proceso. El proceso es regenerativo sobre los tiempos $T_{n}$ si sus segmentos $\zeta_{n}$ son independientes e id\'enticamennte distribuidos.
\end{Def}


\begin{Note}
Si $\tilde{X}\left(t\right)$ con espacio de estados $\tilde{S}$ es regenerativo sobre $T_{n}$, entonces $X\left(t\right)=f\left(\tilde{X}\left(t\right)\right)$ tambi\'en es regenerativo sobre $T_{n}$, para cualquier funci\'on $f:\tilde{S}\rightarrow S$.
\end{Note}

\begin{Note}
Los procesos regenerativos son crudamente regenerativos, pero no al rev\'es.
\end{Note}


\begin{Note}
Un proceso estoc\'astico a tiempo continuo o discreto es regenerativo si existe un proceso de renovaci\'on  tal que los segmentos del proceso entre tiempos de renovaci\'on sucesivos son i.i.d., es decir, para $\left\{X\left(t\right):t\geq0\right\}$ proceso estoc\'astico a tiempo continuo con espacio de estados $S$, espacio m\'etrico.
\end{Note}

Para $\left\{X\left(t\right):t\geq0\right\}$ Proceso Estoc\'astico a tiempo continuo con estado de espacios $S$, que es un espacio m\'etrico, con trayectorias continuas por la derecha y con l\'imites por la izquierda c.s. Sea $N\left(t\right)$ un proceso de renovaci\'on en $\rea_{+}$ definido en el mismo espacio de probabilidad que $X\left(t\right)$, con tiempos de renovaci\'on $T$ y tiempos de inter-renovaci\'on $\xi_{n}=T_{n}-T_{n-1}$, con misma distribuci\'on $F$ de media finita $\mu$.



\begin{Def}
Para el proceso $\left\{\left(N\left(t\right),X\left(t\right)\right):t\geq0\right\}$, sus trayectoria muestrales en el intervalo de tiempo $\left[T_{n-1},T_{n}\right)$ est\'an descritas por
\begin{eqnarray*}
\zeta_{n}=\left(\xi_{n},\left\{X\left(T_{n-1}+t\right):0\leq t<\xi_{n}\right\}\right)
\end{eqnarray*}
Este $\zeta_{n}$ es el $n$-\'esimo segmento del proceso. El proceso es regenerativo sobre los tiempos $T_{n}$ si sus segmentos $\zeta_{n}$ son independientes e id\'enticamennte distribuidos.
\end{Def}

\begin{Note}
Un proceso regenerativo con media de la longitud de ciclo finita es llamado positivo recurrente.
\end{Note}

\begin{Teo}[Procesos Regenerativos]
Suponga que el proceso
\end{Teo}


\begin{Def}[Renewal Process Trinity]
Para un proceso de renovaci\'on $N\left(t\right)$, los siguientes procesos proveen de informaci\'on sobre los tiempos de renovaci\'on.
\begin{itemize}
\item $A\left(t\right)=t-T_{N\left(t\right)}$, el tiempo de recurrencia hacia atr\'as al tiempo $t$, que es el tiempo desde la \'ultima renovaci\'on para $t$.

\item $B\left(t\right)=T_{N\left(t\right)+1}-t$, el tiempo de recurrencia hacia adelante al tiempo $t$, residual del tiempo de renovaci\'on, que es el tiempo para la pr\'oxima renovaci\'on despu\'es de $t$.

\item $L\left(t\right)=\xi_{N\left(t\right)+1}=A\left(t\right)+B\left(t\right)$, la longitud del intervalo de renovaci\'on que contiene a $t$.
\end{itemize}
\end{Def}

\begin{Note}
El proceso tridimensional $\left(A\left(t\right),B\left(t\right),L\left(t\right)\right)$ es regenerativo sobre $T_{n}$, y por ende cada proceso lo es. Cada proceso $A\left(t\right)$ y $B\left(t\right)$ son procesos de MArkov a tiempo continuo con trayectorias continuas por partes en el espacio de estados $\rea_{+}$. Una expresi\'on conveniente para su distribuci\'on conjunta es, para $0\leq x<t,y\geq0$
\begin{equation}\label{NoRenovacion}
P\left\{A\left(t\right)>x,B\left(t\right)>y\right\}=
P\left\{N\left(t+y\right)-N\left((t-x)\right)=0\right\}
\end{equation}
\end{Note}

\begin{Ejem}[Tiempos de recurrencia Poisson]
Si $N\left(t\right)$ es un proceso Poisson con tasa $\lambda$, entonces de la expresi\'on (\ref{NoRenovacion}) se tiene que

\begin{eqnarray*}
\begin{array}{lc}
P\left\{A\left(t\right)>x,B\left(t\right)>y\right\}=e^{-\lambda\left(x+y\right)},&0\leq x<t,y\geq0,
\end{array}
\end{eqnarray*}
que es la probabilidad Poisson de no renovaciones en un intervalo de longitud $x+y$.

\end{Ejem}

\begin{Note}
Una cadena de Markov erg\'odica tiene la propiedad de ser estacionaria si la distribuci\'on de su estado al tiempo $0$ es su distribuci\'on estacionaria.
\end{Note}


\begin{Def}
Un proceso estoc\'astico a tiempo continuo $\left\{X\left(t\right):t\geq0\right\}$ en un espacio general es estacionario si sus distribuciones finito dimensionales son invariantes bajo cualquier  traslado: para cada $0\leq s_{1}<s_{2}<\cdots<s_{k}$ y $t\geq0$,
\begin{eqnarray*}
\left(X\left(s_{1}+t\right),\ldots,X\left(s_{k}+t\right)\right)=_{d}\left(X\left(s_{1}\right),\ldots,X\left(s_{k}\right)\right).
\end{eqnarray*}
\end{Def}

\begin{Note}
Un proceso de Markov es estacionario si $X\left(t\right)=_{d}X\left(0\right)$, $t\geq0$.
\end{Note}

Considerese el proceso $N\left(t\right)=\sum_{n}\indora\left(\tau_{n}\leq t\right)$ en $\rea_{+}$, con puntos $0<\tau_{1}<\tau_{2}<\cdots$.

\begin{Prop}
Si $N$ es un proceso puntual estacionario y $\esp\left[N\left(1\right)\right]<\infty$, entonces $\esp\left[N\left(t\right)\right]=t\esp\left[N\left(1\right)\right]$, $t\geq0$

\end{Prop}

\begin{Teo}
Los siguientes enunciados son equivalentes
\begin{itemize}
\item[i)] El proceso retardado de renovaci\'on $N$ es estacionario.

\item[ii)] EL proceso de tiempos de recurrencia hacia adelante $B\left(t\right)$ es estacionario.


\item[iii)] $\esp\left[N\left(t\right)\right]=t/\mu$,


\item[iv)] $G\left(t\right)=F_{e}\left(t\right)=\frac{1}{\mu}\int_{0}^{t}\left[1-F\left(s\right)\right]ds$
\end{itemize}
Cuando estos enunciados son ciertos, $P\left\{B\left(t\right)\leq x\right\}=F_{e}\left(x\right)$, para $t,x\geq0$.

\end{Teo}

\begin{Note}
Una consecuencia del teorema anterior es que el Proceso Poisson es el \'unico proceso sin retardo que es estacionario.
\end{Note}

\begin{Coro}
El proceso de renovaci\'on $N\left(t\right)$ sin retardo, y cuyos tiempos de inter renonaci\'on tienen media finita, es estacionario si y s\'olo si es un proceso Poisson.

\end{Coro}


%________________________________________________________________________
%\subsection{Procesos Regenerativos}
%________________________________________________________________________



\begin{Note}
Si $\tilde{X}\left(t\right)$ con espacio de estados $\tilde{S}$ es regenerativo sobre $T_{n}$, entonces $X\left(t\right)=f\left(\tilde{X}\left(t\right)\right)$ tambi\'en es regenerativo sobre $T_{n}$, para cualquier funci\'on $f:\tilde{S}\rightarrow S$.
\end{Note}

\begin{Note}
Los procesos regenerativos son crudamente regenerativos, pero no al rev\'es.
\end{Note}
%\subsection*{Procesos Regenerativos: Sigman\cite{Sigman1}}
\begin{Def}[Definici\'on Cl\'asica]
Un proceso estoc\'astico $X=\left\{X\left(t\right):t\geq0\right\}$ es llamado regenerativo is existe una variable aleatoria $R_{1}>0$ tal que
\begin{itemize}
\item[i)] $\left\{X\left(t+R_{1}\right):t\geq0\right\}$ es independiente de $\left\{\left\{X\left(t\right):t<R_{1}\right\},\right\}$
\item[ii)] $\left\{X\left(t+R_{1}\right):t\geq0\right\}$ es estoc\'asticamente equivalente a $\left\{X\left(t\right):t>0\right\}$
\end{itemize}

Llamamos a $R_{1}$ tiempo de regeneraci\'on, y decimos que $X$ se regenera en este punto.
\end{Def}

$\left\{X\left(t+R_{1}\right)\right\}$ es regenerativo con tiempo de regeneraci\'on $R_{2}$, independiente de $R_{1}$ pero con la misma distribuci\'on que $R_{1}$. Procediendo de esta manera se obtiene una secuencia de variables aleatorias independientes e id\'enticamente distribuidas $\left\{R_{n}\right\}$ llamados longitudes de ciclo. Si definimos a $Z_{k}\equiv R_{1}+R_{2}+\cdots+R_{k}$, se tiene un proceso de renovaci\'on llamado proceso de renovaci\'on encajado para $X$.




\begin{Def}
Para $x$ fijo y para cada $t\geq0$, sea $I_{x}\left(t\right)=1$ si $X\left(t\right)\leq x$,  $I_{x}\left(t\right)=0$ en caso contrario, y def\'inanse los tiempos promedio
\begin{eqnarray*}
\overline{X}&=&lim_{t\rightarrow\infty}\frac{1}{t}\int_{0}^{\infty}X\left(u\right)du\\
\prob\left(X_{\infty}\leq x\right)&=&lim_{t\rightarrow\infty}\frac{1}{t}\int_{0}^{\infty}I_{x}\left(u\right)du,
\end{eqnarray*}
cuando estos l\'imites existan.
\end{Def}

Como consecuencia del teorema de Renovaci\'on-Recompensa, se tiene que el primer l\'imite  existe y es igual a la constante
\begin{eqnarray*}
\overline{X}&=&\frac{\esp\left[\int_{0}^{R_{1}}X\left(t\right)dt\right]}{\esp\left[R_{1}\right]},
\end{eqnarray*}
suponiendo que ambas esperanzas son finitas.

\begin{Note}
\begin{itemize}
\item[a)] Si el proceso regenerativo $X$ es positivo recurrente y tiene trayectorias muestrales no negativas, entonces la ecuaci\'on anterior es v\'alida.
\item[b)] Si $X$ es positivo recurrente regenerativo, podemos construir una \'unica versi\'on estacionaria de este proceso, $X_{e}=\left\{X_{e}\left(t\right)\right\}$, donde $X_{e}$ es un proceso estoc\'astico regenerativo y estrictamente estacionario, con distribuci\'on marginal distribuida como $X_{\infty}$
\end{itemize}
\end{Note}

%________________________________________________________________________
%\subsection{Procesos Regenerativos}
%________________________________________________________________________

Para $\left\{X\left(t\right):t\geq0\right\}$ Proceso Estoc\'astico a tiempo continuo con estado de espacios $S$, que es un espacio m\'etrico, con trayectorias continuas por la derecha y con l\'imites por la izquierda c.s. Sea $N\left(t\right)$ un proceso de renovaci\'on en $\rea_{+}$ definido en el mismo espacio de probabilidad que $X\left(t\right)$, con tiempos de renovaci\'on $T$ y tiempos de inter-renovaci\'on $\xi_{n}=T_{n}-T_{n-1}$, con misma distribuci\'on $F$ de media finita $\mu$.



\begin{Def}
Para el proceso $\left\{\left(N\left(t\right),X\left(t\right)\right):t\geq0\right\}$, sus trayectoria muestrales en el intervalo de tiempo $\left[T_{n-1},T_{n}\right)$ est\'an descritas por
\begin{eqnarray*}
\zeta_{n}=\left(\xi_{n},\left\{X\left(T_{n-1}+t\right):0\leq t<\xi_{n}\right\}\right)
\end{eqnarray*}
Este $\zeta_{n}$ es el $n$-\'esimo segmento del proceso. El proceso es regenerativo sobre los tiempos $T_{n}$ si sus segmentos $\zeta_{n}$ son independientes e id\'enticamennte distribuidos.
\end{Def}


\begin{Note}
Si $\tilde{X}\left(t\right)$ con espacio de estados $\tilde{S}$ es regenerativo sobre $T_{n}$, entonces $X\left(t\right)=f\left(\tilde{X}\left(t\right)\right)$ tambi\'en es regenerativo sobre $T_{n}$, para cualquier funci\'on $f:\tilde{S}\rightarrow S$.
\end{Note}

\begin{Note}
Los procesos regenerativos son crudamente regenerativos, pero no al rev\'es.
\end{Note}

\begin{Def}[Definici\'on Cl\'asica]
Un proceso estoc\'astico $X=\left\{X\left(t\right):t\geq0\right\}$ es llamado regenerativo is existe una variable aleatoria $R_{1}>0$ tal que
\begin{itemize}
\item[i)] $\left\{X\left(t+R_{1}\right):t\geq0\right\}$ es independiente de $\left\{\left\{X\left(t\right):t<R_{1}\right\},\right\}$
\item[ii)] $\left\{X\left(t+R_{1}\right):t\geq0\right\}$ es estoc\'asticamente equivalente a $\left\{X\left(t\right):t>0\right\}$
\end{itemize}

Llamamos a $R_{1}$ tiempo de regeneraci\'on, y decimos que $X$ se regenera en este punto.
\end{Def}

$\left\{X\left(t+R_{1}\right)\right\}$ es regenerativo con tiempo de regeneraci\'on $R_{2}$, independiente de $R_{1}$ pero con la misma distribuci\'on que $R_{1}$. Procediendo de esta manera se obtiene una secuencia de variables aleatorias independientes e id\'enticamente distribuidas $\left\{R_{n}\right\}$ llamados longitudes de ciclo. Si definimos a $Z_{k}\equiv R_{1}+R_{2}+\cdots+R_{k}$, se tiene un proceso de renovaci\'on llamado proceso de renovaci\'on encajado para $X$.

\begin{Note}
Un proceso regenerativo con media de la longitud de ciclo finita es llamado positivo recurrente.
\end{Note}


\begin{Def}
Para $x$ fijo y para cada $t\geq0$, sea $I_{x}\left(t\right)=1$ si $X\left(t\right)\leq x$,  $I_{x}\left(t\right)=0$ en caso contrario, y def\'inanse los tiempos promedio
\begin{eqnarray*}
\overline{X}&=&lim_{t\rightarrow\infty}\frac{1}{t}\int_{0}^{\infty}X\left(u\right)du\\
\prob\left(X_{\infty}\leq x\right)&=&lim_{t\rightarrow\infty}\frac{1}{t}\int_{0}^{\infty}I_{x}\left(u\right)du,
\end{eqnarray*}
cuando estos l\'imites existan.
\end{Def}

Como consecuencia del teorema de Renovaci\'on-Recompensa, se tiene que el primer l\'imite  existe y es igual a la constante
\begin{eqnarray*}
\overline{X}&=&\frac{\esp\left[\int_{0}^{R_{1}}X\left(t\right)dt\right]}{\esp\left[R_{1}\right]},
\end{eqnarray*}
suponiendo que ambas esperanzas son finitas.

\begin{Note}
\begin{itemize}
\item[a)] Si el proceso regenerativo $X$ es positivo recurrente y tiene trayectorias muestrales no negativas, entonces la ecuaci\'on anterior es v\'alida.
\item[b)] Si $X$ es positivo recurrente regenerativo, podemos construir una \'unica versi\'on estacionaria de este proceso, $X_{e}=\left\{X_{e}\left(t\right)\right\}$, donde $X_{e}$ es un proceso estoc\'astico regenerativo y estrictamente estacionario, con distribuci\'on marginal distribuida como $X_{\infty}$
\end{itemize}
\end{Note}

%__________________________________________________________________________________________
%\subsection{Procesos Regenerativos Estacionarios - Stidham \cite{Stidham}}
%__________________________________________________________________________________________


Un proceso estoc\'astico a tiempo continuo $\left\{V\left(t\right),t\geq0\right\}$ es un proceso regenerativo si existe una sucesi\'on de variables aleatorias independientes e id\'enticamente distribuidas $\left\{X_{1},X_{2},\ldots\right\}$, sucesi\'on de renovaci\'on, tal que para cualquier conjunto de Borel $A$, 

\begin{eqnarray*}
\prob\left\{V\left(t\right)\in A|X_{1}+X_{2}+\cdots+X_{R\left(t\right)}=s,\left\{V\left(\tau\right),\tau<s\right\}\right\}=\prob\left\{V\left(t-s\right)\in A|X_{1}>t-s\right\},
\end{eqnarray*}
para todo $0\leq s\leq t$, donde $R\left(t\right)=\max\left\{X_{1}+X_{2}+\cdots+X_{j}\leq t\right\}=$n\'umero de renovaciones ({\emph{puntos de regeneraci\'on}}) que ocurren en $\left[0,t\right]$. El intervalo $\left[0,X_{1}\right)$ es llamado {\emph{primer ciclo de regeneraci\'on}} de $\left\{V\left(t \right),t\geq0\right\}$, $\left[X_{1},X_{1}+X_{2}\right)$ el {\emph{segundo ciclo de regeneraci\'on}}, y as\'i sucesivamente.

Sea $X=X_{1}$ y sea $F$ la funci\'on de distrbuci\'on de $X$


\begin{Def}
Se define el proceso estacionario, $\left\{V^{*}\left(t\right),t\geq0\right\}$, para $\left\{V\left(t\right),t\geq0\right\}$ por

\begin{eqnarray*}
\prob\left\{V\left(t\right)\in A\right\}=\frac{1}{\esp\left[X\right]}\int_{0}^{\infty}\prob\left\{V\left(t+x\right)\in A|X>x\right\}\left(1-F\left(x\right)\right)dx,
\end{eqnarray*} 
para todo $t\geq0$ y todo conjunto de Borel $A$.
\end{Def}

\begin{Def}
Una distribuci\'on se dice que es {\emph{aritm\'etica}} si todos sus puntos de incremento son m\'ultiplos de la forma $0,\lambda, 2\lambda,\ldots$ para alguna $\lambda>0$ entera.
\end{Def}


\begin{Def}
Una modificaci\'on medible de un proceso $\left\{V\left(t\right),t\geq0\right\}$, es una versi\'on de este, $\left\{V\left(t,w\right)\right\}$ conjuntamente medible para $t\geq0$ y para $w\in S$, $S$ espacio de estados para $\left\{V\left(t\right),t\geq0\right\}$.
\end{Def}

\begin{Teo}
Sea $\left\{V\left(t\right),t\geq\right\}$ un proceso regenerativo no negativo con modificaci\'on medible. Sea $\esp\left[X\right]<\infty$. Entonces el proceso estacionario dado por la ecuaci\'on anterior est\'a bien definido y tiene funci\'on de distribuci\'on independiente de $t$, adem\'as
\begin{itemize}
\item[i)] \begin{eqnarray*}
\esp\left[V^{*}\left(0\right)\right]&=&\frac{\esp\left[\int_{0}^{X}V\left(s\right)ds\right]}{\esp\left[X\right]}\end{eqnarray*}
\item[ii)] Si $\esp\left[V^{*}\left(0\right)\right]<\infty$, equivalentemente, si $\esp\left[\int_{0}^{X}V\left(s\right)ds\right]<\infty$,entonces
\begin{eqnarray*}
\frac{\int_{0}^{t}V\left(s\right)ds}{t}\rightarrow\frac{\esp\left[\int_{0}^{X}V\left(s\right)ds\right]}{\esp\left[X\right]}
\end{eqnarray*}
con probabilidad 1 y en media, cuando $t\rightarrow\infty$.
\end{itemize}
\end{Teo}
%
%___________________________________________________________________________________________
%\vspace{5.5cm}
%\chapter{Cadenas de Markov estacionarias}
%\vspace{-1.0cm}


%__________________________________________________________________________________________
%\subsection{Procesos Regenerativos Estacionarios - Stidham \cite{Stidham}}
%__________________________________________________________________________________________


Un proceso estoc\'astico a tiempo continuo $\left\{V\left(t\right),t\geq0\right\}$ es un proceso regenerativo si existe una sucesi\'on de variables aleatorias independientes e id\'enticamente distribuidas $\left\{X_{1},X_{2},\ldots\right\}$, sucesi\'on de renovaci\'on, tal que para cualquier conjunto de Borel $A$, 

\begin{eqnarray*}
\prob\left\{V\left(t\right)\in A|X_{1}+X_{2}+\cdots+X_{R\left(t\right)}=s,\left\{V\left(\tau\right),\tau<s\right\}\right\}=\prob\left\{V\left(t-s\right)\in A|X_{1}>t-s\right\},
\end{eqnarray*}
para todo $0\leq s\leq t$, donde $R\left(t\right)=\max\left\{X_{1}+X_{2}+\cdots+X_{j}\leq t\right\}=$n\'umero de renovaciones ({\emph{puntos de regeneraci\'on}}) que ocurren en $\left[0,t\right]$. El intervalo $\left[0,X_{1}\right)$ es llamado {\emph{primer ciclo de regeneraci\'on}} de $\left\{V\left(t \right),t\geq0\right\}$, $\left[X_{1},X_{1}+X_{2}\right)$ el {\emph{segundo ciclo de regeneraci\'on}}, y as\'i sucesivamente.

Sea $X=X_{1}$ y sea $F$ la funci\'on de distrbuci\'on de $X$


\begin{Def}
Se define el proceso estacionario, $\left\{V^{*}\left(t\right),t\geq0\right\}$, para $\left\{V\left(t\right),t\geq0\right\}$ por

\begin{eqnarray*}
\prob\left\{V\left(t\right)\in A\right\}=\frac{1}{\esp\left[X\right]}\int_{0}^{\infty}\prob\left\{V\left(t+x\right)\in A|X>x\right\}\left(1-F\left(x\right)\right)dx,
\end{eqnarray*} 
para todo $t\geq0$ y todo conjunto de Borel $A$.
\end{Def}

\begin{Def}
Una distribuci\'on se dice que es {\emph{aritm\'etica}} si todos sus puntos de incremento son m\'ultiplos de la forma $0,\lambda, 2\lambda,\ldots$ para alguna $\lambda>0$ entera.
\end{Def}


\begin{Def}
Una modificaci\'on medible de un proceso $\left\{V\left(t\right),t\geq0\right\}$, es una versi\'on de este, $\left\{V\left(t,w\right)\right\}$ conjuntamente medible para $t\geq0$ y para $w\in S$, $S$ espacio de estados para $\left\{V\left(t\right),t\geq0\right\}$.
\end{Def}

\begin{Teo}
Sea $\left\{V\left(t\right),t\geq\right\}$ un proceso regenerativo no negativo con modificaci\'on medible. Sea $\esp\left[X\right]<\infty$. Entonces el proceso estacionario dado por la ecuaci\'on anterior est\'a bien definido y tiene funci\'on de distribuci\'on independiente de $t$, adem\'as
\begin{itemize}
\item[i)] \begin{eqnarray*}
\esp\left[V^{*}\left(0\right)\right]&=&\frac{\esp\left[\int_{0}^{X}V\left(s\right)ds\right]}{\esp\left[X\right]}\end{eqnarray*}
\item[ii)] Si $\esp\left[V^{*}\left(0\right)\right]<\infty$, equivalentemente, si $\esp\left[\int_{0}^{X}V\left(s\right)ds\right]<\infty$,entonces
\begin{eqnarray*}
\frac{\int_{0}^{t}V\left(s\right)ds}{t}\rightarrow\frac{\esp\left[\int_{0}^{X}V\left(s\right)ds\right]}{\esp\left[X\right]}
\end{eqnarray*}
con probabilidad 1 y en media, cuando $t\rightarrow\infty$.
\end{itemize}
\end{Teo}

Para $\left\{X\left(t\right):t\geq0\right\}$ Proceso Estoc\'astico a tiempo continuo con estado de espacios $S$, que es un espacio m\'etrico, con trayectorias continuas por la derecha y con l\'imites por la izquierda c.s. Sea $N\left(t\right)$ un proceso de renovaci\'on en $\rea_{+}$ definido en el mismo espacio de probabilidad que $X\left(t\right)$, con tiempos de renovaci\'on $T$ y tiempos de inter-renovaci\'on $\xi_{n}=T_{n}-T_{n-1}$, con misma distribuci\'on $F$ de media finita $\mu$.


%______________________________________________________________________
%\subsection{Ejemplos, Notas importantes}


Sean $T_{1},T_{2},\ldots$ los puntos donde las longitudes de las colas de la red de sistemas de visitas c\'iclicas son cero simult\'aneamente, cuando la cola $Q_{j}$ es visitada por el servidor para dar servicio, es decir, $L_{1}\left(T_{i}\right)=0,L_{2}\left(T_{i}\right)=0,\hat{L}_{1}\left(T_{i}\right)=0$ y $\hat{L}_{2}\left(T_{i}\right)=0$, a estos puntos se les denominar\'a puntos regenerativos. Sea la funci\'on generadora de momentos para $L_{i}$, el n\'umero de usuarios en la cola $Q_{i}\left(z\right)$ en cualquier momento, est\'a dada por el tiempo promedio de $z^{L_{i}\left(t\right)}$ sobre el ciclo regenerativo definido anteriormente:

\begin{eqnarray*}
Q_{i}\left(z\right)&=&\esp\left[z^{L_{i}\left(t\right)}\right]=\frac{\esp\left[\sum_{m=1}^{M_{i}}\sum_{t=\tau_{i}\left(m\right)}^{\tau_{i}\left(m+1\right)-1}z^{L_{i}\left(t\right)}\right]}{\esp\left[\sum_{m=1}^{M_{i}}\tau_{i}\left(m+1\right)-\tau_{i}\left(m\right)\right]}
\end{eqnarray*}

$M_{i}$ es un tiempo de paro en el proceso regenerativo con $\esp\left[M_{i}\right]<\infty$\footnote{En Stidham\cite{Stidham} y Heyman  se muestra que una condici\'on suficiente para que el proceso regenerativo 
estacionario sea un procesoo estacionario es que el valor esperado del tiempo del ciclo regenerativo sea finito, es decir: $\esp\left[\sum_{m=1}^{M_{i}}C_{i}^{(m)}\right]<\infty$, como cada $C_{i}^{(m)}$ contiene intervalos de r\'eplica positivos, se tiene que $\esp\left[M_{i}\right]<\infty$, adem\'as, como $M_{i}>0$, se tiene que la condici\'on anterior es equivalente a tener que $\esp\left[C_{i}\right]<\infty$,
por lo tanto una condici\'on suficiente para la existencia del proceso regenerativo est\'a dada por $\sum_{k=1}^{N}\mu_{k}<1.$}, se sigue del lema de Wald que:


\begin{eqnarray*}
\esp\left[\sum_{m=1}^{M_{i}}\sum_{t=\tau_{i}\left(m\right)}^{\tau_{i}\left(m+1\right)-1}z^{L_{i}\left(t\right)}\right]&=&\esp\left[M_{i}\right]\esp\left[\sum_{t=\tau_{i}\left(m\right)}^{\tau_{i}\left(m+1\right)-1}z^{L_{i}\left(t\right)}\right]\\
\esp\left[\sum_{m=1}^{M_{i}}\tau_{i}\left(m+1\right)-\tau_{i}\left(m\right)\right]&=&\esp\left[M_{i}\right]\esp\left[\tau_{i}\left(m+1\right)-\tau_{i}\left(m\right)\right]
\end{eqnarray*}

por tanto se tiene que


\begin{eqnarray*}
Q_{i}\left(z\right)&=&\frac{\esp\left[\sum_{t=\tau_{i}\left(m\right)}^{\tau_{i}\left(m+1\right)-1}z^{L_{i}\left(t\right)}\right]}{\esp\left[\tau_{i}\left(m+1\right)-\tau_{i}\left(m\right)\right]}
\end{eqnarray*}

observar que el denominador es simplemente la duraci\'on promedio del tiempo del ciclo.


Haciendo las siguientes sustituciones en la ecuaci\'on (\ref{Corolario2}): $n\rightarrow t-\tau_{i}\left(m\right)$, $T \rightarrow \overline{\tau}_{i}\left(m\right)-\tau_{i}\left(m\right)$, $L_{n}\rightarrow L_{i}\left(t\right)$ y $F\left(z\right)=\esp\left[z^{L_{0}}\right]\rightarrow F_{i}\left(z\right)=\esp\left[z^{L_{i}\tau_{i}\left(m\right)}\right]$, se puede ver que

\begin{eqnarray}\label{Eq.Arribos.Primera}
\esp\left[\sum_{n=0}^{T-1}z^{L_{n}}\right]=
\esp\left[\sum_{t=\tau_{i}\left(m\right)}^{\overline{\tau}_{i}\left(m\right)-1}z^{L_{i}\left(t\right)}\right]
=z\frac{F_{i}\left(z\right)-1}{z-P_{i}\left(z\right)}
\end{eqnarray}

Por otra parte durante el tiempo de intervisita para la cola $i$, $L_{i}\left(t\right)$ solamente se incrementa de manera que el incremento por intervalo de tiempo est\'a dado por la funci\'on generadora de probabilidades de $P_{i}\left(z\right)$, por tanto la suma sobre el tiempo de intervisita puede evaluarse como:

\begin{eqnarray*}
\esp\left[\sum_{t=\tau_{i}\left(m\right)}^{\tau_{i}\left(m+1\right)-1}z^{L_{i}\left(t\right)}\right]&=&\esp\left[\sum_{t=\tau_{i}\left(m\right)}^{\tau_{i}\left(m+1\right)-1}\left\{P_{i}\left(z\right)\right\}^{t-\overline{\tau}_{i}\left(m\right)}\right]=\frac{1-\esp\left[\left\{P_{i}\left(z\right)\right\}^{\tau_{i}\left(m+1\right)-\overline{\tau}_{i}\left(m\right)}\right]}{1-P_{i}\left(z\right)}\\
&=&\frac{1-I_{i}\left[P_{i}\left(z\right)\right]}{1-P_{i}\left(z\right)}
\end{eqnarray*}
por tanto

\begin{eqnarray*}
\esp\left[\sum_{t=\tau_{i}\left(m\right)}^{\tau_{i}\left(m+1\right)-1}z^{L_{i}\left(t\right)}\right]&=&
\frac{1-F_{i}\left(z\right)}{1-P_{i}\left(z\right)}
\end{eqnarray*}

Por lo tanto

\begin{eqnarray*}
Q_{i}\left(z\right)&=&\frac{\esp\left[\sum_{t=\tau_{i}\left(m\right)}^{\tau_{i}
\left(m+1\right)-1}z^{L_{i}\left(t\right)}\right]}{\esp\left[\tau_{i}\left(m+1\right)-\tau_{i}\left(m\right)\right]}\\
&=&\frac{1}{\esp\left[\tau_{i}\left(m+1\right)-\tau_{i}\left(m\right)\right]}
\left\{
\esp\left[\sum_{t=\tau_{i}\left(m\right)}^{\overline{\tau}_{i}\left(m\right)-1}
z^{L_{i}\left(t\right)}\right]
+\esp\left[\sum_{t=\overline{\tau}_{i}\left(m\right)}^{\tau_{i}\left(m+1\right)-1}
z^{L_{i}\left(t\right)}\right]\right\}\\
&=&\frac{1}{\esp\left[\tau_{i}\left(m+1\right)-\tau_{i}\left(m\right)\right]}
\left\{
z\frac{F_{i}\left(z\right)-1}{z-P_{i}\left(z\right)}+\frac{1-F_{i}\left(z\right)}
{1-P_{i}\left(z\right)}
\right\}
\end{eqnarray*}

es decir

\begin{equation}
Q_{i}\left(z\right)=\frac{1}{\esp\left[C_{i}\right]}\cdot\frac{1-F_{i}\left(z\right)}{P_{i}\left(z\right)-z}\cdot\frac{\left(1-z\right)P_{i}\left(z\right)}{1-P_{i}\left(z\right)}
\end{equation}

\begin{Teo}
Dada una Red de Sistemas de Visitas C\'iclicas (RSVC), conformada por dos Sistemas de Visitas C\'iclicas (SVC), donde cada uno de ellos consta de dos colas tipo $M/M/1$. Los dos sistemas est\'an comunicados entre s\'i por medio de la transferencia de usuarios entre las colas $Q_{1}\leftrightarrow Q_{3}$ y $Q_{2}\leftrightarrow Q_{4}$. Se definen los eventos para los procesos de arribos al tiempo $t$, $A_{j}\left(t\right)=\left\{0 \textrm{ arribos en }Q_{j}\textrm{ al tiempo }t\right\}$ para alg\'un tiempo $t\geq0$ y $Q_{j}$ la cola $j$-\'esima en la RSVC, para $j=1,2,3,4$.  Existe un intervalo $I\neq\emptyset$ tal que para $T^{*}\in I$, tal que $\prob\left\{A_{1}\left(T^{*}\right),A_{2}\left(Tt^{*}\right),
A_{3}\left(T^{*}\right),A_{4}\left(T^{*}\right)|T^{*}\in I\right\}>0$.
\end{Teo}

\begin{proof}
Sin p\'erdida de generalidad podemos considerar como base del an\'alisis a la cola $Q_{1}$ del primer sistema que conforma la RSVC.

Sea $n>0$, ciclo en el primer sistema en el que se sabe que $L_{j}\left(\overline{\tau}_{1}\left(n\right)\right)=0$, pues la pol\'itica de servicio con que atienden los servidores es la exhaustiva. Como es sabido, para trasladarse a la siguiente cola, el servidor incurre en un tiempo de traslado $r_{1}\left(n\right)>0$, entonces tenemos que $\tau_{2}\left(n\right)=\overline{\tau}_{1}\left(n\right)+r_{1}\left(n\right)$.


Definamos el intervalo $I_{1}\equiv\left[\overline{\tau}_{1}\left(n\right),\tau_{2}\left(n\right)\right]$ de longitud $\xi_{1}=r_{1}\left(n\right)$. Dado que los tiempos entre arribo son exponenciales con tasa $\tilde{\mu}_{1}=\mu_{1}+\hat{\mu}_{1}$ ($\mu_{1}$ son los arribos a $Q_{1}$ por primera vez al sistema, mientras que $\hat{\mu}_{1}$ son los arribos de traslado procedentes de $Q_{3}$) se tiene que la probabilidad del evento $A_{1}\left(t\right)$ est\'a dada por 

\begin{equation}
\prob\left\{A_{1}\left(t\right)|t\in I_{1}\left(n\right)\right\}=e^{-\tilde{\mu}_{1}\xi_{1}\left(n\right)}.
\end{equation} 

Por otra parte, para la cola $Q_{2}$, el tiempo $\overline{\tau}_{2}\left(n-1\right)$ es tal que $L_{2}\left(\overline{\tau}_{2}\left(n-1\right)\right)=0$, es decir, es el tiempo en que la cola queda totalmente vac\'ia en el ciclo anterior a $n$. Entonces tenemos un sgundo intervalo $I_{2}\equiv\left[\overline{\tau}_{2}\left(n-1\right),\tau_{2}\left(n\right)\right]$. Por lo tanto la probabilidad del evento $A_{2}\left(t\right)$ tiene probabilidad dada por

\begin{equation}
\prob\left\{A_{2}\left(t\right)|t\in I_{2}\left(n\right)\right\}=e^{-\tilde{\mu}_{2}\xi_{2}\left(n\right)},
\end{equation} 

donde $\xi_{2}\left(n\right)=\tau_{2}\left(n\right)-\overline{\tau}_{2}\left(n-1\right)$.



Entonces, se tiene que

\begin{eqnarray*}
\prob\left\{A_{1}\left(t\right),A_{2}\left(t\right)|t\in I_{1}\left(n\right)\right\}&=&
\prob\left\{A_{1}\left(t\right)|t\in I_{1}\left(n\right)\right\}
\prob\left\{A_{2}\left(t\right)|t\in I_{1}\left(n\right)\right\}\\
&\geq&
\prob\left\{A_{1}\left(t\right)|t\in I_{1}\left(n\right)\right\}
\prob\left\{A_{2}\left(t\right)|t\in I_{2}\left(n\right)\right\}\\
&=&e^{-\tilde{\mu}_{1}\xi_{1}\left(n\right)}e^{-\tilde{\mu}_{2}\xi_{2}\left(n\right)}
=e^{-\left[\tilde{\mu}_{1}\xi_{1}\left(n\right)+\tilde{\mu}_{2}\xi_{2}\left(n\right)\right]}.
\end{eqnarray*}


es decir, 

\begin{equation}
\prob\left\{A_{1}\left(t\right),A_{2}\left(t\right)|t\in I_{1}\left(n\right)\right\}
=e^{-\left[\tilde{\mu}_{1}\xi_{1}\left(n\right)+\tilde{\mu}_{2}\xi_{2}
\left(n\right)\right]}>0.
\end{equation}

En lo que respecta a la relaci\'on entre los dos SVC que conforman la RSVC, se afirma que existe $m>0$ tal que $\overline{\tau}_{3}\left(m\right)<\tau_{2}\left(n\right)<\tau_{4}\left(m\right)$.

Para $Q_{3}$ sea $I_{3}=\left[\overline{\tau}_{3}\left(m\right),\tau_{4}\left(m\right)\right]$ con longitud  $\xi_{3}\left(m\right)=r_{3}\left(m\right)$, entonces 

\begin{equation}
\prob\left\{A_{3}\left(t\right)|t\in I_{3}\left(n\right)\right\}=e^{-\tilde{\mu}_{3}\xi_{3}\left(n\right)}.
\end{equation} 

An\'alogamente que como se hizo para $Q_{2}$, tenemos que para $Q_{4}$ se tiene el intervalo $I_{4}=\left[\overline{\tau}_{4}\left(m-1\right),\tau_{4}\left(m\right)\right]$ con longitud $\xi_{4}\left(m\right)=\tau_{4}\left(m\right)-\overline{\tau}_{4}\left(m-1\right)$, entonces


\begin{equation}
\prob\left\{A_{4}\left(t\right)|t\in I_{4}\left(m\right)\right\}=e^{-\tilde{\mu}_{4}\xi_{4}\left(n\right)}.
\end{equation} 

Al igual que para el primer sistema, dado que $I_{3}\left(m\right)\subset I_{4}\left(m\right)$, se tiene que

\begin{eqnarray*}
\xi_{3}\left(m\right)\leq\xi_{4}\left(m\right)&\Leftrightarrow& -\xi_{3}\left(m\right)\geq-\xi_{4}\left(m\right)
\\
-\tilde{\mu}_{4}\xi_{3}\left(m\right)\geq-\tilde{\mu}_{4}\xi_{4}\left(m\right)&\Leftrightarrow&
e^{-\tilde{\mu}_{4}\xi_{3}\left(m\right)}\geq e^{-\tilde{\mu}_{4}\xi_{4}\left(m\right)}\\
\prob\left\{A_{4}\left(t\right)|t\in I_{3}\left(m\right)\right\}&\geq&
\prob\left\{A_{4}\left(t\right)|t\in I_{4}\left(m\right)\right\}
\end{eqnarray*}

Entonces, dado que los eventos $A_{3}$ y $A_{4}$ son independientes, se tiene que

\begin{eqnarray*}
\prob\left\{A_{3}\left(t\right),A_{4}\left(t\right)|t\in I_{3}\left(m\right)\right\}&=&
\prob\left\{A_{3}\left(t\right)|t\in I_{3}\left(m\right)\right\}
\prob\left\{A_{4}\left(t\right)|t\in I_{3}\left(m\right)\right\}\\
&\geq&
\prob\left\{A_{3}\left(t\right)|t\in I_{3}\left(n\right)\right\}
\prob\left\{A_{4}\left(t\right)|t\in I_{4}\left(n\right)\right\}\\
&=&e^{-\tilde{\mu}_{3}\xi_{3}\left(m\right)}e^{-\tilde{\mu}_{4}\xi_{4}
\left(m\right)}
=e^{-\left[\tilde{\mu}_{3}\xi_{3}\left(m\right)+\tilde{\mu}_{4}\xi_{4}
\left(m\right)\right]}.
\end{eqnarray*}


es decir, 

\begin{equation}
\prob\left\{A_{3}\left(t\right),A_{4}\left(t\right)|t\in I_{3}\left(m\right)\right\}
=e^{-\left[\tilde{\mu}_{3}\xi_{3}\left(m\right)+\tilde{\mu}_{4}\xi_{4}
\left(m\right)\right]}>0.
\end{equation}

Por construcci\'on se tiene que $I\left(n,m\right)\equiv I_{1}\left(n\right)\cap I_{3}\left(m\right)\neq\emptyset$,entonces en particular se tienen las contenciones $I\left(n,m\right)\subseteq I_{1}\left(n\right)$ y $I\left(n,m\right)\subseteq I_{3}\left(m\right)$, por lo tanto si definimos $\xi_{n,m}\equiv\ell\left(I\left(n,m\right)\right)$ tenemos que

\begin{eqnarray*}
\xi_{n,m}\leq\xi_{1}\left(n\right)\textrm{ y }\xi_{n,m}\leq\xi_{3}\left(m\right)\textrm{ entonces }
-\xi_{n,m}\geq-\xi_{1}\left(n\right)\textrm{ y }-\xi_{n,m}\leq-\xi_{3}\left(m\right)\\
\end{eqnarray*}
por lo tanto tenemos las desigualdades 



\begin{eqnarray*}
\begin{array}{ll}
-\tilde{\mu}_{1}\xi_{n,m}\geq-\tilde{\mu}_{1}\xi_{1}\left(n\right),&
-\tilde{\mu}_{2}\xi_{n,m}\geq-\tilde{\mu}_{2}\xi_{1}\left(n\right)
\geq-\tilde{\mu}_{2}\xi_{2}\left(n\right),\\
-\tilde{\mu}_{3}\xi_{n,m}\geq-\tilde{\mu}_{3}\xi_{3}\left(m\right),&
-\tilde{\mu}_{4}\xi_{n,m}\geq-\tilde{\mu}_{4}\xi_{3}\left(m\right)
\geq-\tilde{\mu}_{4}\xi_{4}\left(m\right).
\end{array}
\end{eqnarray*}

Sea $T^{*}\in I_{n,m}$, entonces dado que en particular $T^{*}\in I_{1}\left(n\right)$ se cumple con probabilidad positiva que no hay arribos a las colas $Q_{1}$ y $Q_{2}$, en consecuencia, tampoco hay usuarios de transferencia para $Q_{3}$ y $Q_{4}$, es decir, $\tilde{\mu}_{1}=\mu_{1}$, $\tilde{\mu}_{2}=\mu_{2}$, $\tilde{\mu}_{3}=\mu_{3}$, $\tilde{\mu}_{4}=\mu_{4}$, es decir, los eventos $Q_{1}$ y $Q_{3}$ son condicionalmente independientes en el intervalo $I_{n,m}$; lo mismo ocurre para las colas $Q_{2}$ y $Q_{4}$, por lo tanto tenemos que


\begin{eqnarray}
\begin{array}{l}
\prob\left\{A_{1}\left(T^{*}\right),A_{2}\left(T^{*}\right),
A_{3}\left(T^{*}\right),A_{4}\left(T^{*}\right)|T^{*}\in I_{n,m}\right\}
=\prod_{j=1}^{4}\prob\left\{A_{j}\left(T^{*}\right)|T^{*}\in I_{n,m}\right\}\\
\geq\prob\left\{A_{1}\left(T^{*}\right)|T^{*}\in I_{1}\left(n\right)\right\}
\prob\left\{A_{2}\left(T^{*}\right)|T^{*}\in I_{2}\left(n\right)\right\}
\prob\left\{A_{3}\left(T^{*}\right)|T^{*}\in I_{3}\left(m\right)\right\}
\prob\left\{A_{4}\left(T^{*}\right)|T^{*}\in I_{4}\left(m\right)\right\}\\
=e^{-\mu_{1}\xi_{1}\left(n\right)}
e^{-\mu_{2}\xi_{2}\left(n\right)}
e^{-\mu_{3}\xi_{3}\left(m\right)}
e^{-\mu_{4}\xi_{4}\left(m\right)}
=e^{-\left[\tilde{\mu}_{1}\xi_{1}\left(n\right)
+\tilde{\mu}_{2}\xi_{2}\left(n\right)
+\tilde{\mu}_{3}\xi_{3}\left(m\right)
+\tilde{\mu}_{4}\xi_{4}
\left(m\right)\right]}>0.
\end{array}
\end{eqnarray}
\end{proof}


Estos resultados aparecen en Daley (1968) \cite{Daley68} para $\left\{T_{n}\right\}$ intervalos de inter-arribo, $\left\{D_{n}\right\}$ intervalos de inter-salida y $\left\{S_{n}\right\}$ tiempos de servicio.

\begin{itemize}
\item Si el proceso $\left\{T_{n}\right\}$ es Poisson, el proceso $\left\{D_{n}\right\}$ es no correlacionado si y s\'olo si es un proceso Poisso, lo cual ocurre si y s\'olo si $\left\{S_{n}\right\}$ son exponenciales negativas.

\item Si $\left\{S_{n}\right\}$ son exponenciales negativas, $\left\{D_{n}\right\}$ es un proceso de renovaci\'on  si y s\'olo si es un proceso Poisson, lo cual ocurre si y s\'olo si $\left\{T_{n}\right\}$ es un proceso Poisson.

\item $\esp\left(D_{n}\right)=\esp\left(T_{n}\right)$.

\item Para un sistema de visitas $GI/M/1$ se tiene el siguiente teorema:

\begin{Teo}
En un sistema estacionario $GI/M/1$ los intervalos de interpartida tienen
\begin{eqnarray*}
\esp\left(e^{-\theta D_{n}}\right)&=&\mu\left(\mu+\theta\right)^{-1}\left[\delta\theta
-\mu\left(1-\delta\right)\alpha\left(\theta\right)\right]
\left[\theta-\mu\left(1-\delta\right)^{-1}\right]\\
\alpha\left(\theta\right)&=&\esp\left[e^{-\theta T_{0}}\right]\\
var\left(D_{n}\right)&=&var\left(T_{0}\right)-\left(\tau^{-1}-\delta^{-1}\right)
2\delta\left(\esp\left(S_{0}\right)\right)^{2}\left(1-\delta\right)^{-1}.
\end{eqnarray*}
\end{Teo}



\begin{Teo}
El proceso de salida de un sistema de colas estacionario $GI/M/1$ es un proceso de renovaci\'on si y s\'olo si el proceso de entrada es un proceso Poisson, en cuyo caso el proceso de salida es un proceso Poisson.
\end{Teo}


\begin{Teo}
Los intervalos de interpartida $\left\{D_{n}\right\}$ de un sistema $M/G/1$ estacionario son no correlacionados si y s\'olo si la distribuci\'on de los tiempos de servicio es exponencial negativa, es decir, el sistema es de tipo  $M/M/1$.

\end{Teo}



\end{itemize}


%\section{Resultados para Procesos de Salida}

En Sigman, Thorison y Wolff \cite{Sigman2} prueban que para la existencia de un una sucesi\'on infinita no decreciente de tiempos de regeneraci\'on $\tau_{1}\leq\tau_{2}\leq\cdots$ en los cuales el proceso se regenera, basta un tiempo de regeneraci\'on $R_{1}$, donde $R_{j}=\tau_{j}-\tau_{j-1}$. Para tal efecto se requiere la existencia de un espacio de probabilidad $\left(\Omega,\mathcal{F},\prob\right)$, y proceso estoc\'astico $\textit{X}=\left\{X\left(t\right):t\geq0\right\}$ con espacio de estados $\left(S,\mathcal{R}\right)$, con $\mathcal{R}$ $\sigma$-\'algebra.

\begin{Prop}
Si existe una variable aleatoria no negativa $R_{1}$ tal que $\theta_{R\footnotesize{1}}X=_{D}X$, entonces $\left(\Omega,\mathcal{F},\prob\right)$ puede extenderse para soportar una sucesi\'on estacionaria de variables aleatorias $R=\left\{R_{k}:k\geq1\right\}$, tal que para $k\geq1$,
\begin{eqnarray*}
\theta_{k}\left(X,R\right)=_{D}\left(X,R\right).
\end{eqnarray*}

Adem\'as, para $k\geq1$, $\theta_{k}R$ es condicionalmente independiente de $\left(X,R_{1},\ldots,R_{k}\right)$, dado $\theta_{\tau k}X$.

\end{Prop}


\begin{itemize}
\item Doob en 1953 demostr\'o que el estado estacionario de un proceso de partida en un sistema de espera $M/G/\infty$, es Poisson con la misma tasa que el proceso de arribos.

\item Burke en 1968, fue el primero en demostrar que el estado estacionario de un proceso de salida de una cola $M/M/s$ es un proceso Poisson.

\item Disney en 1973 obtuvo el siguiente resultado:

\begin{Teo}
Para el sistema de espera $M/G/1/L$ con disciplina FIFO, el proceso $\textbf{I}$ es un proceso de renovaci\'on si y s\'olo si el proceso denominado longitud de la cola es estacionario y se cumple cualquiera de los siguientes casos:

\begin{itemize}
\item[a)] Los tiempos de servicio son identicamente cero;
\item[b)] $L=0$, para cualquier proceso de servicio $S$;
\item[c)] $L=1$ y $G=D$;
\item[d)] $L=\infty$ y $G=M$.
\end{itemize}
En estos casos, respectivamente, las distribuciones de interpartida $P\left\{T_{n+1}-T_{n}\leq t\right\}$ son


\begin{itemize}
\item[a)] $1-e^{-\lambda t}$, $t\geq0$;
\item[b)] $1-e^{-\lambda t}*F\left(t\right)$, $t\geq0$;
\item[c)] $1-e^{-\lambda t}*\indora_{d}\left(t\right)$, $t\geq0$;
\item[d)] $1-e^{-\lambda t}*F\left(t\right)$, $t\geq0$.
\end{itemize}
\end{Teo}


\item Finch (1959) mostr\'o que para los sistemas $M/G/1/L$, con $1\leq L\leq \infty$ con distribuciones de servicio dos veces diferenciable, solamente el sistema $M/M/1/\infty$ tiene proceso de salida de renovaci\'on estacionario.

\item King (1971) demostro que un sistema de colas estacionario $M/G/1/1$ tiene sus tiempos de interpartida sucesivas $D_{n}$ y $D_{n+1}$ son independientes, si y s\'olo si, $G=D$, en cuyo caso le proceso de salida es de renovaci\'on.

\item Disney (1973) demostr\'o que el \'unico sistema estacionario $M/G/1/L$, que tiene proceso de salida de renovaci\'on  son los sistemas $M/M/1$ y $M/D/1/1$.



\item El siguiente resultado es de Disney y Koning (1985)
\begin{Teo}
En un sistema de espera $M/G/s$, el estado estacionario del proceso de salida es un proceso Poisson para cualquier distribuci\'on de los tiempos de servicio si el sistema tiene cualquiera de las siguientes cuatro propiedades.

\begin{itemize}
\item[a)] $s=\infty$
\item[b)] La disciplina de servicio es de procesador compartido.
\item[c)] La disciplina de servicio es LCFS y preemptive resume, esto se cumple para $L<\infty$
\item[d)] $G=M$.
\end{itemize}

\end{Teo}

\item El siguiente resultado es de Alamatsaz (1983)

\begin{Teo}
En cualquier sistema de colas $GI/G/1/L$ con $1\leq L<\infty$ y distribuci\'on de interarribos $A$ y distribuci\'on de los tiempos de servicio $B$, tal que $A\left(0\right)=0$, $A\left(t\right)\left(1-B\left(t\right)\right)>0$ para alguna $t>0$ y $B\left(t\right)$ para toda $t>0$, es imposible que el proceso de salida estacionario sea de renovaci\'on.
\end{Teo}

\end{itemize}

Estos resultados aparecen en Daley (1968) \cite{Daley68} para $\left\{T_{n}\right\}$ intervalos de inter-arribo, $\left\{D_{n}\right\}$ intervalos de inter-salida y $\left\{S_{n}\right\}$ tiempos de servicio.

\begin{itemize}
\item Si el proceso $\left\{T_{n}\right\}$ es Poisson, el proceso $\left\{D_{n}\right\}$ es no correlacionado si y s\'olo si es un proceso Poisso, lo cual ocurre si y s\'olo si $\left\{S_{n}\right\}$ son exponenciales negativas.

\item Si $\left\{S_{n}\right\}$ son exponenciales negativas, $\left\{D_{n}\right\}$ es un proceso de renovaci\'on  si y s\'olo si es un proceso Poisson, lo cual ocurre si y s\'olo si $\left\{T_{n}\right\}$ es un proceso Poisson.

\item $\esp\left(D_{n}\right)=\esp\left(T_{n}\right)$.

\item Para un sistema de visitas $GI/M/1$ se tiene el siguiente teorema:

\begin{Teo}
En un sistema estacionario $GI/M/1$ los intervalos de interpartida tienen
\begin{eqnarray*}
\esp\left(e^{-\theta D_{n}}\right)&=&\mu\left(\mu+\theta\right)^{-1}\left[\delta\theta
-\mu\left(1-\delta\right)\alpha\left(\theta\right)\right]
\left[\theta-\mu\left(1-\delta\right)^{-1}\right]\\
\alpha\left(\theta\right)&=&\esp\left[e^{-\theta T_{0}}\right]\\
var\left(D_{n}\right)&=&var\left(T_{0}\right)-\left(\tau^{-1}-\delta^{-1}\right)
2\delta\left(\esp\left(S_{0}\right)\right)^{2}\left(1-\delta\right)^{-1}.
\end{eqnarray*}
\end{Teo}



\begin{Teo}
El proceso de salida de un sistema de colas estacionario $GI/M/1$ es un proceso de renovaci\'on si y s\'olo si el proceso de entrada es un proceso Poisson, en cuyo caso el proceso de salida es un proceso Poisson.
\end{Teo}


\begin{Teo}
Los intervalos de interpartida $\left\{D_{n}\right\}$ de un sistema $M/G/1$ estacionario son no correlacionados si y s\'olo si la distribuci\'on de los tiempos de servicio es exponencial negativa, es decir, el sistema es de tipo  $M/M/1$.

\end{Teo}



\end{itemize}
%\newpage
%________________________________________________________________________
%\section{Redes de Sistemas de Visitas C\'iclicas}
%________________________________________________________________________

Sean $Q_{1},Q_{2},Q_{3}$ y $Q_{4}$ en una Red de Sistemas de Visitas C\'iclicas (RSVC). Supongamos que cada una de las colas es del tipo $M/M/1$ con tasa de arribo $\mu_{i}$ y que la transferencia de usuarios entre los dos sistemas ocurre entre $Q_{1}\leftrightarrow Q_{3}$ y $Q_{2}\leftrightarrow Q_{4}$ con respectiva tasa de arribo igual a la tasa de salida $\hat{\mu}_{i}=\mu_{i}$, esto se sabe por lo desarrollado en la secci\'on anterior.  

Consideremos, sin p\'erdida de generalidad como base del an\'alisis, la cola $Q_{1}$ adem\'as supongamos al servidor lo comenzamos a observar una vez que termina de atender a la misma para desplazarse y llegar a $Q_{2}$, es decir al tiempo $\tau_{2}$.

Sea $n\in\nat$, $n>0$, ciclo del servidor en que regresa a $Q_{1}$ para dar servicio y atender conforme a la pol\'itica exhaustiva, entonces se tiene que $\overline{\tau}_{1}\left(n\right)$ es el tiempo del servidor en el sistema 1 en que termina de dar servicio a todos los usuarios presentes en la cola, por lo tanto se cumple que $L_{1}\left(\overline{\tau}_{1}\left(n\right)\right)=0$, entonces el servidor para llegar a $Q_{2}$ incurre en un tiempo de traslado $r_{1}$ y por tanto se cumple que $\tau_{2}\left(n\right)=\overline{\tau}_{1}\left(n\right)+r_{1}$. Dado que los tiempos entre arribos son exponenciales se cumple que 

\begin{eqnarray*}
\prob\left\{0 \textrm{ arribos en }Q_{1}\textrm{ en el intervalo }\left[\overline{\tau}_{1}\left(n\right),\overline{\tau}_{1}\left(n\right)+r_{1}\right]\right\}=e^{-\tilde{\mu}_{1}r_{1}},\\
\prob\left\{0 \textrm{ arribos en }Q_{2}\textrm{ en el intervalo }\left[\overline{\tau}_{1}\left(n\right),\overline{\tau}_{1}\left(n\right)+r_{1}\right]\right\}=e^{-\tilde{\mu}_{2}r_{1}}.
\end{eqnarray*}

El evento que nos interesa consiste en que no haya arribos desde que el servidor abandon\'o $Q_{2}$ y regresa nuevamente para dar servicio, es decir en el intervalo de tiempo $\left[\overline{\tau}_{2}\left(n-1\right),\tau_{2}\left(n\right)\right]$. Entonces, si hacemos


\begin{eqnarray*}
\varphi_{1}\left(n\right)&\equiv&\overline{\tau}_{1}\left(n\right)+r_{1}=\overline{\tau}_{2}\left(n-1\right)+r_{1}+r_{2}+\overline{\tau}_{1}\left(n\right)-\tau_{1}\left(n\right)\\
&=&\overline{\tau}_{2}\left(n-1\right)+\overline{\tau}_{1}\left(n\right)-\tau_{1}\left(n\right)+r,,
\end{eqnarray*}

y longitud del intervalo

\begin{eqnarray*}
\xi&\equiv&\overline{\tau}_{1}\left(n\right)+r_{1}-\overline{\tau}_{2}\left(n-1\right)
=\overline{\tau}_{2}\left(n-1\right)+\overline{\tau}_{1}\left(n\right)-\tau_{1}\left(n\right)+r-\overline{\tau}_{2}\left(n-1\right)\\
&=&\overline{\tau}_{1}\left(n\right)-\tau_{1}\left(n\right)+r.
\end{eqnarray*}


Entonces, determinemos la probabilidad del evento no arribos a $Q_{2}$ en $\left[\overline{\tau}_{2}\left(n-1\right),\varphi_{1}\left(n\right)\right]$:

\begin{eqnarray}
\prob\left\{0 \textrm{ arribos en }Q_{2}\textrm{ en el intervalo }\left[\overline{\tau}_{2}\left(n-1\right),\varphi_{1}\left(n\right)\right]\right\}
=e^{-\tilde{\mu}_{2}\xi}.
\end{eqnarray}

De manera an\'aloga, tenemos que la probabilidad de no arribos a $Q_{1}$ en $\left[\overline{\tau}_{2}\left(n-1\right),\varphi_{1}\left(n\right)\right]$ esta dada por

\begin{eqnarray}
\prob\left\{0 \textrm{ arribos en }Q_{1}\textrm{ en el intervalo }\left[\overline{\tau}_{2}\left(n-1\right),\varphi_{1}\left(n\right)\right]\right\}
=e^{-\tilde{\mu}_{1}\xi},
\end{eqnarray}

\begin{eqnarray}
\prob\left\{0 \textrm{ arribos en }Q_{2}\textrm{ en el intervalo }\left[\overline{\tau}_{2}\left(n-1\right),\varphi_{1}\left(n\right)\right]\right\}
=e^{-\tilde{\mu}_{2}\xi}.
\end{eqnarray}

Por tanto 

\begin{eqnarray}
\begin{array}{l}
\prob\left\{0 \textrm{ arribos en }Q_{1}\textrm{ y }Q_{2}\textrm{ en el intervalo }\left[\overline{\tau}_{2}\left(n-1\right),\varphi_{1}\left(n\right)\right]\right\}\\
=\prob\left\{0 \textrm{ arribos en }Q_{1}\textrm{ en el intervalo }\left[\overline{\tau}_{2}\left(n-1\right),\varphi_{1}\left(n\right)\right]\right\}\\
\times
\prob\left\{0 \textrm{ arribos en }Q_{2}\textrm{ en el intervalo }\left[\overline{\tau}_{2}\left(n-1\right),\varphi_{1}\left(n\right)\right]\right\}=e^{-\tilde{\mu}_{1}\xi}e^{-\tilde{\mu}_{2}\xi}
=e^{-\tilde{\mu}\xi}.
\end{array}
\end{eqnarray}

Para el segundo sistema, consideremos nuevamente $\overline{\tau}_{1}\left(n\right)+r_{1}$, sin p\'erdida de generalidad podemos suponer que existe $m>0$ tal que $\overline{\tau}_{3}\left(m\right)<\overline{\tau}_{1}+r_{1}<\tau_{4}\left(m\right)$, entonces

\begin{eqnarray}
\prob\left\{0 \textrm{ arribos en }Q_{3}\textrm{ en el intervalo }\left[\overline{\tau}_{3}\left(m\right),\overline{\tau}_{1}\left(n\right)+r_{1}\right]\right\}
=e^{-\tilde{\mu}_{3}\xi_{3}},
\end{eqnarray}
donde 
\begin{eqnarray}
\xi_{3}=\overline{\tau}_{1}\left(n\right)+r_{1}-\overline{\tau}_{3}\left(m\right)=
\overline{\tau}_{1}\left(n\right)-\overline{\tau}_{3}\left(m\right)+r_{1},
\end{eqnarray}

mientras que para $Q_{4}$ al igual que con $Q_{2}$ escribiremos $\tau_{4}\left(m\right)$ en t\'erminos de $\overline{\tau}_{4}\left(m-1\right)$:

$\varphi_{2}\equiv\tau_{4}\left(m\right)=\overline{\tau}_{4}\left(m-1\right)+r_{4}+\overline{\tau}_{3}\left(m\right)
-\tau_{3}\left(m\right)+r_{3}=\overline{\tau}_{4}\left(m-1\right)+\overline{\tau}_{3}\left(m\right)
-\tau_{3}\left(m\right)+\hat{r}$, adem\'as,

$\xi_{2}\equiv\varphi_{2}\left(m\right)-\overline{\tau}_{1}-r_{1}=\overline{\tau}_{4}\left(m-1\right)+\overline{\tau}_{3}\left(m\right)
-\tau_{3}\left(m\right)-\overline{\tau}_{1}\left(n\right)+\hat{r}-r_{1}$. 

Entonces


\begin{eqnarray}
\prob\left\{0 \textrm{ arribos en }Q_{4}\textrm{ en el intervalo }\left[\overline{\tau}_{1}\left(n\right)+r_{1},\varphi_{2}\left(m\right)\right]\right\}
=e^{-\tilde{\mu}_{4}\xi_{2}},
\end{eqnarray}

mientras que para $Q_{3}$ se tiene que 

\begin{eqnarray}
\prob\left\{0 \textrm{ arribos en }Q_{3}\textrm{ en el intervalo }\left[\overline{\tau}_{1}\left(n\right)+r_{1},\varphi_{2}\left(m\right)\right]\right\}
=e^{-\tilde{\mu}_{3}\xi_{2}}
\end{eqnarray}

Por tanto

\begin{eqnarray}
\prob\left\{0 \textrm{ arribos en }Q_{3}\wedge Q_{4}\textrm{ en el intervalo }\left[\overline{\tau}_{1}\left(n\right)+r_{1},\varphi_{2}\left(m\right)\right]\right\}
=e^{-\hat{\mu}\xi_{2}}
\end{eqnarray}
donde $\hat{\mu}=\tilde{\mu}_{3}+\tilde{\mu}_{4}$.

Ahora, definamos los intervalos $\mathcal{I}_{1}=\left[\overline{\tau}_{1}\left(n\right)+r_{1},\varphi_{1}\left(n\right)\right]$  y $\mathcal{I}_{2}=\left[\overline{\tau}_{1}\left(n\right)+r_{1},\varphi_{2}\left(m\right)\right]$, entonces, sea $\mathcal{I}=\mathcal{I}_{1}\cap\mathcal{I}_{2}$ el intervalo donde cada una de las colas se encuentran vac\'ias, es decir, si tomamos $T^{*}\in\mathcal{I}$, entonces  $L_{1}\left(T^{*}\right)=L_{2}\left(T^{*}\right)=L_{3}\left(T^{*}\right)=L_{4}\left(T^{*}\right)=0$.

Ahora, dado que por construcci\'on $\mathcal{I}\neq\emptyset$ y que para $T^{*}\in\mathcal{I}$ en ninguna de las colas han llegado usuarios, se tiene que no hay transferencia entre las colas, por lo tanto, el sistema 1 y el sistema 2 son condicionalmente independientes en $\mathcal{I}$, es decir

\begin{eqnarray}
\prob\left\{L_{1}\left(T^{*}\right),L_{2}\left(T^{*}\right),L_{3}\left(T^{*}\right),L_{4}\left(T^{*}\right)|T^{*}\in\mathcal{I}\right\}=\prod_{j=1}^{4}\prob\left\{L_{j}\left(T^{*}\right)\right\},
\end{eqnarray}

para $T^{*}\in\mathcal{I}$. 

%\newpage























%________________________________________________________________________
%\section{Procesos Regenerativos}
%________________________________________________________________________

%________________________________________________________________________
%\subsection*{Procesos Regenerativos Sigman, Thorisson y Wolff \cite{Sigman1}}
%________________________________________________________________________


\begin{Def}[Definici\'on Cl\'asica]
Un proceso estoc\'astico $X=\left\{X\left(t\right):t\geq0\right\}$ es llamado regenerativo is existe una variable aleatoria $R_{1}>0$ tal que
\begin{itemize}
\item[i)] $\left\{X\left(t+R_{1}\right):t\geq0\right\}$ es independiente de $\left\{\left\{X\left(t\right):t<R_{1}\right\},\right\}$
\item[ii)] $\left\{X\left(t+R_{1}\right):t\geq0\right\}$ es estoc\'asticamente equivalente a $\left\{X\left(t\right):t>0\right\}$
\end{itemize}

Llamamos a $R_{1}$ tiempo de regeneraci\'on, y decimos que $X$ se regenera en este punto.
\end{Def}

$\left\{X\left(t+R_{1}\right)\right\}$ es regenerativo con tiempo de regeneraci\'on $R_{2}$, independiente de $R_{1}$ pero con la misma distribuci\'on que $R_{1}$. Procediendo de esta manera se obtiene una secuencia de variables aleatorias independientes e id\'enticamente distribuidas $\left\{R_{n}\right\}$ llamados longitudes de ciclo. Si definimos a $Z_{k}\equiv R_{1}+R_{2}+\cdots+R_{k}$, se tiene un proceso de renovaci\'on llamado proceso de renovaci\'on encajado para $X$.


\begin{Note}
La existencia de un primer tiempo de regeneraci\'on, $R_{1}$, implica la existencia de una sucesi\'on completa de estos tiempos $R_{1},R_{2}\ldots,$ que satisfacen la propiedad deseada \cite{Sigman2}.
\end{Note}


\begin{Note} Para la cola $GI/GI/1$ los usuarios arriban con tiempos $t_{n}$ y son atendidos con tiempos de servicio $S_{n}$, los tiempos de arribo forman un proceso de renovaci\'on  con tiempos entre arribos independientes e identicamente distribuidos (\texttt{i.i.d.})$T_{n}=t_{n}-t_{n-1}$, adem\'as los tiempos de servicio son \texttt{i.i.d.} e independientes de los procesos de arribo. Por \textit{estable} se entiende que $\esp S_{n}<\esp T_{n}<\infty$.
\end{Note}
 


\begin{Def}
Para $x$ fijo y para cada $t\geq0$, sea $I_{x}\left(t\right)=1$ si $X\left(t\right)\leq x$,  $I_{x}\left(t\right)=0$ en caso contrario, y def\'inanse los tiempos promedio
\begin{eqnarray*}
\overline{X}&=&lim_{t\rightarrow\infty}\frac{1}{t}\int_{0}^{\infty}X\left(u\right)du\\
\prob\left(X_{\infty}\leq x\right)&=&lim_{t\rightarrow\infty}\frac{1}{t}\int_{0}^{\infty}I_{x}\left(u\right)du,
\end{eqnarray*}
cuando estos l\'imites existan.
\end{Def}

Como consecuencia del teorema de Renovaci\'on-Recompensa, se tiene que el primer l\'imite  existe y es igual a la constante
\begin{eqnarray*}
\overline{X}&=&\frac{\esp\left[\int_{0}^{R_{1}}X\left(t\right)dt\right]}{\esp\left[R_{1}\right]},
\end{eqnarray*}
suponiendo que ambas esperanzas son finitas.
 
\begin{Note}
Funciones de procesos regenerativos son regenerativas, es decir, si $X\left(t\right)$ es regenerativo y se define el proceso $Y\left(t\right)$ por $Y\left(t\right)=f\left(X\left(t\right)\right)$ para alguna funci\'on Borel medible $f\left(\cdot\right)$. Adem\'as $Y$ es regenerativo con los mismos tiempos de renovaci\'on que $X$. 

En general, los tiempos de renovaci\'on, $Z_{k}$ de un proceso regenerativo no requieren ser tiempos de paro con respecto a la evoluci\'on de $X\left(t\right)$.
\end{Note} 

\begin{Note}
Una funci\'on de un proceso de Markov, usualmente no ser\'a un proceso de Markov, sin embargo ser\'a regenerativo si el proceso de Markov lo es.
\end{Note}

 
\begin{Note}
Un proceso regenerativo con media de la longitud de ciclo finita es llamado positivo recurrente.
\end{Note}


\begin{Note}
\begin{itemize}
\item[a)] Si el proceso regenerativo $X$ es positivo recurrente y tiene trayectorias muestrales no negativas, entonces la ecuaci\'on anterior es v\'alida.
\item[b)] Si $X$ es positivo recurrente regenerativo, podemos construir una \'unica versi\'on estacionaria de este proceso, $X_{e}=\left\{X_{e}\left(t\right)\right\}$, donde $X_{e}$ es un proceso estoc\'astico regenerativo y estrictamente estacionario, con distribuci\'on marginal distribuida como $X_{\infty}$
\end{itemize}
\end{Note}


%__________________________________________________________________________________________
%\subsection*{Procesos Regenerativos Estacionarios - Stidham \cite{Stidham}}
%__________________________________________________________________________________________


Un proceso estoc\'astico a tiempo continuo $\left\{V\left(t\right),t\geq0\right\}$ es un proceso regenerativo si existe una sucesi\'on de variables aleatorias independientes e id\'enticamente distribuidas $\left\{X_{1},X_{2},\ldots\right\}$, sucesi\'on de renovaci\'on, tal que para cualquier conjunto de Borel $A$, 

\begin{eqnarray*}
\prob\left\{V\left(t\right)\in A|X_{1}+X_{2}+\cdots+X_{R\left(t\right)}=s,\left\{V\left(\tau\right),\tau<s\right\}\right\}=\prob\left\{V\left(t-s\right)\in A|X_{1}>t-s\right\},
\end{eqnarray*}
para todo $0\leq s\leq t$, donde $R\left(t\right)=\max\left\{X_{1}+X_{2}+\cdots+X_{j}\leq t\right\}=$n\'umero de renovaciones ({\emph{puntos de regeneraci\'on}}) que ocurren en $\left[0,t\right]$. El intervalo $\left[0,X_{1}\right)$ es llamado {\emph{primer ciclo de regeneraci\'on}} de $\left\{V\left(t \right),t\geq0\right\}$, $\left[X_{1},X_{1}+X_{2}\right)$ el {\emph{segundo ciclo de regeneraci\'on}}, y as\'i sucesivamente.

Sea $X=X_{1}$ y sea $F$ la funci\'on de distrbuci\'on de $X$


\begin{Def}
Se define el proceso estacionario, $\left\{V^{*}\left(t\right),t\geq0\right\}$, para $\left\{V\left(t\right),t\geq0\right\}$ por

\begin{eqnarray*}
\prob\left\{V\left(t\right)\in A\right\}=\frac{1}{\esp\left[X\right]}\int_{0}^{\infty}\prob\left\{V\left(t+x\right)\in A|X>x\right\}\left(1-F\left(x\right)\right)dx,
\end{eqnarray*} 
para todo $t\geq0$ y todo conjunto de Borel $A$.
\end{Def}

\begin{Def}
Una distribuci\'on se dice que es {\emph{aritm\'etica}} si todos sus puntos de incremento son m\'ultiplos de la forma $0,\lambda, 2\lambda,\ldots$ para alguna $\lambda>0$ entera.
\end{Def}


\begin{Def}
Una modificaci\'on medible de un proceso $\left\{V\left(t\right),t\geq0\right\}$, es una versi\'on de este, $\left\{V\left(t,w\right)\right\}$ conjuntamente medible para $t\geq0$ y para $w\in S$, $S$ espacio de estados para $\left\{V\left(t\right),t\geq0\right\}$.
\end{Def}

\begin{Teo}
Sea $\left\{V\left(t\right),t\geq\right\}$ un proceso regenerativo no negativo con modificaci\'on medible. Sea $\esp\left[X\right]<\infty$. Entonces el proceso estacionario dado por la ecuaci\'on anterior est\'a bien definido y tiene funci\'on de distribuci\'on independiente de $t$, adem\'as
\begin{itemize}
\item[i)] \begin{eqnarray*}
\esp\left[V^{*}\left(0\right)\right]&=&\frac{\esp\left[\int_{0}^{X}V\left(s\right)ds\right]}{\esp\left[X\right]}\end{eqnarray*}
\item[ii)] Si $\esp\left[V^{*}\left(0\right)\right]<\infty$, equivalentemente, si $\esp\left[\int_{0}^{X}V\left(s\right)ds\right]<\infty$,entonces
\begin{eqnarray*}
\frac{\int_{0}^{t}V\left(s\right)ds}{t}\rightarrow\frac{\esp\left[\int_{0}^{X}V\left(s\right)ds\right]}{\esp\left[X\right]}
\end{eqnarray*}
con probabilidad 1 y en media, cuando $t\rightarrow\infty$.
\end{itemize}
\end{Teo}

\begin{Coro}
Sea $\left\{V\left(t\right),t\geq0\right\}$ un proceso regenerativo no negativo, con modificaci\'on medible. Si $\esp <\infty$, $F$ es no-aritm\'etica, y para todo $x\geq0$, $P\left\{V\left(t\right)\leq x,C>x\right\}$ es de variaci\'on acotada como funci\'on de $t$ en cada intervalo finito $\left[0,\tau\right]$, entonces $V\left(t\right)$ converge en distribuci\'on  cuando $t\rightarrow\infty$ y $$\esp V=\frac{\esp \int_{0}^{X}V\left(s\right)ds}{\esp X}$$
Donde $V$ tiene la distribuci\'on l\'imite de $V\left(t\right)$ cuando $t\rightarrow\infty$.

\end{Coro}

Para el caso discreto se tienen resultados similares.



%______________________________________________________________________
%\section{Procesos de Renovaci\'on}
%______________________________________________________________________

\begin{Def}\label{Def.Tn}
Sean $0\leq T_{1}\leq T_{2}\leq \ldots$ son tiempos aleatorios infinitos en los cuales ocurren ciertos eventos. El n\'umero de tiempos $T_{n}$ en el intervalo $\left[0,t\right)$ es

\begin{eqnarray}
N\left(t\right)=\sum_{n=1}^{\infty}\indora\left(T_{n}\leq t\right),
\end{eqnarray}
para $t\geq0$.
\end{Def}

Si se consideran los puntos $T_{n}$ como elementos de $\rea_{+}$, y $N\left(t\right)$ es el n\'umero de puntos en $\rea$. El proceso denotado por $\left\{N\left(t\right):t\geq0\right\}$, denotado por $N\left(t\right)$, es un proceso puntual en $\rea_{+}$. Los $T_{n}$ son los tiempos de ocurrencia, el proceso puntual $N\left(t\right)$ es simple si su n\'umero de ocurrencias son distintas: $0<T_{1}<T_{2}<\ldots$ casi seguramente.

\begin{Def}
Un proceso puntual $N\left(t\right)$ es un proceso de renovaci\'on si los tiempos de interocurrencia $\xi_{n}=T_{n}-T_{n-1}$, para $n\geq1$, son independientes e identicamente distribuidos con distribuci\'on $F$, donde $F\left(0\right)=0$ y $T_{0}=0$. Los $T_{n}$ son llamados tiempos de renovaci\'on, referente a la independencia o renovaci\'on de la informaci\'on estoc\'astica en estos tiempos. Los $\xi_{n}$ son los tiempos de inter-renovaci\'on, y $N\left(t\right)$ es el n\'umero de renovaciones en el intervalo $\left[0,t\right)$
\end{Def}


\begin{Note}
Para definir un proceso de renovaci\'on para cualquier contexto, solamente hay que especificar una distribuci\'on $F$, con $F\left(0\right)=0$, para los tiempos de inter-renovaci\'on. La funci\'on $F$ en turno degune las otra variables aleatorias. De manera formal, existe un espacio de probabilidad y una sucesi\'on de variables aleatorias $\xi_{1},\xi_{2},\ldots$ definidas en este con distribuci\'on $F$. Entonces las otras cantidades son $T_{n}=\sum_{k=1}^{n}\xi_{k}$ y $N\left(t\right)=\sum_{n=1}^{\infty}\indora\left(T_{n}\leq t\right)$, donde $T_{n}\rightarrow\infty$ casi seguramente por la Ley Fuerte de los Grandes Números.
\end{Note}

%___________________________________________________________________________________________
%
%\subsection*{Teorema Principal de Renovaci\'on}
%___________________________________________________________________________________________
%

\begin{Note} Una funci\'on $h:\rea_{+}\rightarrow\rea$ es Directamente Riemann Integrable en los siguientes casos:
\begin{itemize}
\item[a)] $h\left(t\right)\geq0$ es decreciente y Riemann Integrable.
\item[b)] $h$ es continua excepto posiblemente en un conjunto de Lebesgue de medida 0, y $|h\left(t\right)|\leq b\left(t\right)$, donde $b$ es DRI.
\end{itemize}
\end{Note}

\begin{Teo}[Teorema Principal de Renovaci\'on]
Si $F$ es no aritm\'etica y $h\left(t\right)$ es Directamente Riemann Integrable (DRI), entonces

\begin{eqnarray*}
lim_{t\rightarrow\infty}U\star h=\frac{1}{\mu}\int_{\rea_{+}}h\left(s\right)ds.
\end{eqnarray*}
\end{Teo}

\begin{Prop}
Cualquier funci\'on $H\left(t\right)$ acotada en intervalos finitos y que es 0 para $t<0$ puede expresarse como
\begin{eqnarray*}
H\left(t\right)=U\star h\left(t\right)\textrm{,  donde }h\left(t\right)=H\left(t\right)-F\star H\left(t\right)
\end{eqnarray*}
\end{Prop}

\begin{Def}
Un proceso estoc\'astico $X\left(t\right)$ es crudamente regenerativo en un tiempo aleatorio positivo $T$ si
\begin{eqnarray*}
\esp\left[X\left(T+t\right)|T\right]=\esp\left[X\left(t\right)\right]\textrm{, para }t\geq0,\end{eqnarray*}
y con las esperanzas anteriores finitas.
\end{Def}

\begin{Prop}
Sup\'ongase que $X\left(t\right)$ es un proceso crudamente regenerativo en $T$, que tiene distribuci\'on $F$. Si $\esp\left[X\left(t\right)\right]$ es acotado en intervalos finitos, entonces
\begin{eqnarray*}
\esp\left[X\left(t\right)\right]=U\star h\left(t\right)\textrm{,  donde }h\left(t\right)=\esp\left[X\left(t\right)\indora\left(T>t\right)\right].
\end{eqnarray*}
\end{Prop}

\begin{Teo}[Regeneraci\'on Cruda]
Sup\'ongase que $X\left(t\right)$ es un proceso con valores positivo crudamente regenerativo en $T$, y def\'inase $M=\sup\left\{|X\left(t\right)|:t\leq T\right\}$. Si $T$ es no aritm\'etico y $M$ y $MT$ tienen media finita, entonces
\begin{eqnarray*}
lim_{t\rightarrow\infty}\esp\left[X\left(t\right)\right]=\frac{1}{\mu}\int_{\rea_{+}}h\left(s\right)ds,
\end{eqnarray*}
donde $h\left(t\right)=\esp\left[X\left(t\right)\indora\left(T>t\right)\right]$.
\end{Teo}

%___________________________________________________________________________________________
%
%\subsection*{Propiedades de los Procesos de Renovaci\'on}
%___________________________________________________________________________________________
%

Los tiempos $T_{n}$ est\'an relacionados con los conteos de $N\left(t\right)$ por

\begin{eqnarray*}
\left\{N\left(t\right)\geq n\right\}&=&\left\{T_{n}\leq t\right\}\\
T_{N\left(t\right)}\leq &t&<T_{N\left(t\right)+1},
\end{eqnarray*}

adem\'as $N\left(T_{n}\right)=n$, y 

\begin{eqnarray*}
N\left(t\right)=\max\left\{n:T_{n}\leq t\right\}=\min\left\{n:T_{n+1}>t\right\}
\end{eqnarray*}

Por propiedades de la convoluci\'on se sabe que

\begin{eqnarray*}
P\left\{T_{n}\leq t\right\}=F^{n\star}\left(t\right)
\end{eqnarray*}
que es la $n$-\'esima convoluci\'on de $F$. Entonces 

\begin{eqnarray*}
\left\{N\left(t\right)\geq n\right\}&=&\left\{T_{n}\leq t\right\}\\
P\left\{N\left(t\right)\leq n\right\}&=&1-F^{\left(n+1\right)\star}\left(t\right)
\end{eqnarray*}

Adem\'as usando el hecho de que $\esp\left[N\left(t\right)\right]=\sum_{n=1}^{\infty}P\left\{N\left(t\right)\geq n\right\}$
se tiene que

\begin{eqnarray*}
\esp\left[N\left(t\right)\right]=\sum_{n=1}^{\infty}F^{n\star}\left(t\right)
\end{eqnarray*}

\begin{Prop}
Para cada $t\geq0$, la funci\'on generadora de momentos $\esp\left[e^{\alpha N\left(t\right)}\right]$ existe para alguna $\alpha$ en una vecindad del 0, y de aqu\'i que $\esp\left[N\left(t\right)^{m}\right]<\infty$, para $m\geq1$.
\end{Prop}


\begin{Note}
Si el primer tiempo de renovaci\'on $\xi_{1}$ no tiene la misma distribuci\'on que el resto de las $\xi_{n}$, para $n\geq2$, a $N\left(t\right)$ se le llama Proceso de Renovaci\'on retardado, donde si $\xi$ tiene distribuci\'on $G$, entonces el tiempo $T_{n}$ de la $n$-\'esima renovaci\'on tiene distribuci\'on $G\star F^{\left(n-1\right)\star}\left(t\right)$
\end{Note}


\begin{Teo}
Para una constante $\mu\leq\infty$ ( o variable aleatoria), las siguientes expresiones son equivalentes:

\begin{eqnarray}
lim_{n\rightarrow\infty}n^{-1}T_{n}&=&\mu,\textrm{ c.s.}\\
lim_{t\rightarrow\infty}t^{-1}N\left(t\right)&=&1/\mu,\textrm{ c.s.}
\end{eqnarray}
\end{Teo}


Es decir, $T_{n}$ satisface la Ley Fuerte de los Grandes N\'umeros s\'i y s\'olo s\'i $N\left/t\right)$ la cumple.


\begin{Coro}[Ley Fuerte de los Grandes N\'umeros para Procesos de Renovaci\'on]
Si $N\left(t\right)$ es un proceso de renovaci\'on cuyos tiempos de inter-renovaci\'on tienen media $\mu\leq\infty$, entonces
\begin{eqnarray}
t^{-1}N\left(t\right)\rightarrow 1/\mu,\textrm{ c.s. cuando }t\rightarrow\infty.
\end{eqnarray}

\end{Coro}


Considerar el proceso estoc\'astico de valores reales $\left\{Z\left(t\right):t\geq0\right\}$ en el mismo espacio de probabilidad que $N\left(t\right)$

\begin{Def}
Para el proceso $\left\{Z\left(t\right):t\geq0\right\}$ se define la fluctuaci\'on m\'axima de $Z\left(t\right)$ en el intervalo $\left(T_{n-1},T_{n}\right]$:
\begin{eqnarray*}
M_{n}=\sup_{T_{n-1}<t\leq T_{n}}|Z\left(t\right)-Z\left(T_{n-1}\right)|
\end{eqnarray*}
\end{Def}

\begin{Teo}
Sup\'ongase que $n^{-1}T_{n}\rightarrow\mu$ c.s. cuando $n\rightarrow\infty$, donde $\mu\leq\infty$ es una constante o variable aleatoria. Sea $a$ una constante o variable aleatoria que puede ser infinita cuando $\mu$ es finita, y considere las expresiones l\'imite:
\begin{eqnarray}
lim_{n\rightarrow\infty}n^{-1}Z\left(T_{n}\right)&=&a,\textrm{ c.s.}\\
lim_{t\rightarrow\infty}t^{-1}Z\left(t\right)&=&a/\mu,\textrm{ c.s.}
\end{eqnarray}
La segunda expresi\'on implica la primera. Conversamente, la primera implica la segunda si el proceso $Z\left(t\right)$ es creciente, o si $lim_{n\rightarrow\infty}n^{-1}M_{n}=0$ c.s.
\end{Teo}

\begin{Coro}
Si $N\left(t\right)$ es un proceso de renovaci\'on, y $\left(Z\left(T_{n}\right)-Z\left(T_{n-1}\right),M_{n}\right)$, para $n\geq1$, son variables aleatorias independientes e id\'enticamente distribuidas con media finita, entonces,
\begin{eqnarray}
lim_{t\rightarrow\infty}t^{-1}Z\left(t\right)\rightarrow\frac{\esp\left[Z\left(T_{1}\right)-Z\left(T_{0}\right)\right]}{\esp\left[T_{1}\right]},\textrm{ c.s. cuando  }t\rightarrow\infty.
\end{eqnarray}
\end{Coro}



%___________________________________________________________________________________________
%
%\subsection{Propiedades de los Procesos de Renovaci\'on}
%___________________________________________________________________________________________
%

Los tiempos $T_{n}$ est\'an relacionados con los conteos de $N\left(t\right)$ por

\begin{eqnarray*}
\left\{N\left(t\right)\geq n\right\}&=&\left\{T_{n}\leq t\right\}\\
T_{N\left(t\right)}\leq &t&<T_{N\left(t\right)+1},
\end{eqnarray*}

adem\'as $N\left(T_{n}\right)=n$, y 

\begin{eqnarray*}
N\left(t\right)=\max\left\{n:T_{n}\leq t\right\}=\min\left\{n:T_{n+1}>t\right\}
\end{eqnarray*}

Por propiedades de la convoluci\'on se sabe que

\begin{eqnarray*}
P\left\{T_{n}\leq t\right\}=F^{n\star}\left(t\right)
\end{eqnarray*}
que es la $n$-\'esima convoluci\'on de $F$. Entonces 

\begin{eqnarray*}
\left\{N\left(t\right)\geq n\right\}&=&\left\{T_{n}\leq t\right\}\\
P\left\{N\left(t\right)\leq n\right\}&=&1-F^{\left(n+1\right)\star}\left(t\right)
\end{eqnarray*}

Adem\'as usando el hecho de que $\esp\left[N\left(t\right)\right]=\sum_{n=1}^{\infty}P\left\{N\left(t\right)\geq n\right\}$
se tiene que

\begin{eqnarray*}
\esp\left[N\left(t\right)\right]=\sum_{n=1}^{\infty}F^{n\star}\left(t\right)
\end{eqnarray*}

\begin{Prop}
Para cada $t\geq0$, la funci\'on generadora de momentos $\esp\left[e^{\alpha N\left(t\right)}\right]$ existe para alguna $\alpha$ en una vecindad del 0, y de aqu\'i que $\esp\left[N\left(t\right)^{m}\right]<\infty$, para $m\geq1$.
\end{Prop}


\begin{Note}
Si el primer tiempo de renovaci\'on $\xi_{1}$ no tiene la misma distribuci\'on que el resto de las $\xi_{n}$, para $n\geq2$, a $N\left(t\right)$ se le llama Proceso de Renovaci\'on retardado, donde si $\xi$ tiene distribuci\'on $G$, entonces el tiempo $T_{n}$ de la $n$-\'esima renovaci\'on tiene distribuci\'on $G\star F^{\left(n-1\right)\star}\left(t\right)$
\end{Note}


\begin{Teo}
Para una constante $\mu\leq\infty$ ( o variable aleatoria), las siguientes expresiones son equivalentes:

\begin{eqnarray}
lim_{n\rightarrow\infty}n^{-1}T_{n}&=&\mu,\textrm{ c.s.}\\
lim_{t\rightarrow\infty}t^{-1}N\left(t\right)&=&1/\mu,\textrm{ c.s.}
\end{eqnarray}
\end{Teo}


Es decir, $T_{n}$ satisface la Ley Fuerte de los Grandes N\'umeros s\'i y s\'olo s\'i $N\left/t\right)$ la cumple.


\begin{Coro}[Ley Fuerte de los Grandes N\'umeros para Procesos de Renovaci\'on]
Si $N\left(t\right)$ es un proceso de renovaci\'on cuyos tiempos de inter-renovaci\'on tienen media $\mu\leq\infty$, entonces
\begin{eqnarray}
t^{-1}N\left(t\right)\rightarrow 1/\mu,\textrm{ c.s. cuando }t\rightarrow\infty.
\end{eqnarray}

\end{Coro}


Considerar el proceso estoc\'astico de valores reales $\left\{Z\left(t\right):t\geq0\right\}$ en el mismo espacio de probabilidad que $N\left(t\right)$

\begin{Def}
Para el proceso $\left\{Z\left(t\right):t\geq0\right\}$ se define la fluctuaci\'on m\'axima de $Z\left(t\right)$ en el intervalo $\left(T_{n-1},T_{n}\right]$:
\begin{eqnarray*}
M_{n}=\sup_{T_{n-1}<t\leq T_{n}}|Z\left(t\right)-Z\left(T_{n-1}\right)|
\end{eqnarray*}
\end{Def}

\begin{Teo}
Sup\'ongase que $n^{-1}T_{n}\rightarrow\mu$ c.s. cuando $n\rightarrow\infty$, donde $\mu\leq\infty$ es una constante o variable aleatoria. Sea $a$ una constante o variable aleatoria que puede ser infinita cuando $\mu$ es finita, y considere las expresiones l\'imite:
\begin{eqnarray}
lim_{n\rightarrow\infty}n^{-1}Z\left(T_{n}\right)&=&a,\textrm{ c.s.}\\
lim_{t\rightarrow\infty}t^{-1}Z\left(t\right)&=&a/\mu,\textrm{ c.s.}
\end{eqnarray}
La segunda expresi\'on implica la primera. Conversamente, la primera implica la segunda si el proceso $Z\left(t\right)$ es creciente, o si $lim_{n\rightarrow\infty}n^{-1}M_{n}=0$ c.s.
\end{Teo}

\begin{Coro}
Si $N\left(t\right)$ es un proceso de renovaci\'on, y $\left(Z\left(T_{n}\right)-Z\left(T_{n-1}\right),M_{n}\right)$, para $n\geq1$, son variables aleatorias independientes e id\'enticamente distribuidas con media finita, entonces,
\begin{eqnarray}
lim_{t\rightarrow\infty}t^{-1}Z\left(t\right)\rightarrow\frac{\esp\left[Z\left(T_{1}\right)-Z\left(T_{0}\right)\right]}{\esp\left[T_{1}\right]},\textrm{ c.s. cuando  }t\rightarrow\infty.
\end{eqnarray}
\end{Coro}


%___________________________________________________________________________________________
%
%\subsection{Propiedades de los Procesos de Renovaci\'on}
%___________________________________________________________________________________________
%

Los tiempos $T_{n}$ est\'an relacionados con los conteos de $N\left(t\right)$ por

\begin{eqnarray*}
\left\{N\left(t\right)\geq n\right\}&=&\left\{T_{n}\leq t\right\}\\
T_{N\left(t\right)}\leq &t&<T_{N\left(t\right)+1},
\end{eqnarray*}

adem\'as $N\left(T_{n}\right)=n$, y 

\begin{eqnarray*}
N\left(t\right)=\max\left\{n:T_{n}\leq t\right\}=\min\left\{n:T_{n+1}>t\right\}
\end{eqnarray*}

Por propiedades de la convoluci\'on se sabe que

\begin{eqnarray*}
P\left\{T_{n}\leq t\right\}=F^{n\star}\left(t\right)
\end{eqnarray*}
que es la $n$-\'esima convoluci\'on de $F$. Entonces 

\begin{eqnarray*}
\left\{N\left(t\right)\geq n\right\}&=&\left\{T_{n}\leq t\right\}\\
P\left\{N\left(t\right)\leq n\right\}&=&1-F^{\left(n+1\right)\star}\left(t\right)
\end{eqnarray*}

Adem\'as usando el hecho de que $\esp\left[N\left(t\right)\right]=\sum_{n=1}^{\infty}P\left\{N\left(t\right)\geq n\right\}$
se tiene que

\begin{eqnarray*}
\esp\left[N\left(t\right)\right]=\sum_{n=1}^{\infty}F^{n\star}\left(t\right)
\end{eqnarray*}

\begin{Prop}
Para cada $t\geq0$, la funci\'on generadora de momentos $\esp\left[e^{\alpha N\left(t\right)}\right]$ existe para alguna $\alpha$ en una vecindad del 0, y de aqu\'i que $\esp\left[N\left(t\right)^{m}\right]<\infty$, para $m\geq1$.
\end{Prop}


\begin{Note}
Si el primer tiempo de renovaci\'on $\xi_{1}$ no tiene la misma distribuci\'on que el resto de las $\xi_{n}$, para $n\geq2$, a $N\left(t\right)$ se le llama Proceso de Renovaci\'on retardado, donde si $\xi$ tiene distribuci\'on $G$, entonces el tiempo $T_{n}$ de la $n$-\'esima renovaci\'on tiene distribuci\'on $G\star F^{\left(n-1\right)\star}\left(t\right)$
\end{Note}


\begin{Teo}
Para una constante $\mu\leq\infty$ ( o variable aleatoria), las siguientes expresiones son equivalentes:

\begin{eqnarray}
lim_{n\rightarrow\infty}n^{-1}T_{n}&=&\mu,\textrm{ c.s.}\\
lim_{t\rightarrow\infty}t^{-1}N\left(t\right)&=&1/\mu,\textrm{ c.s.}
\end{eqnarray}
\end{Teo}


Es decir, $T_{n}$ satisface la Ley Fuerte de los Grandes N\'umeros s\'i y s\'olo s\'i $N\left/t\right)$ la cumple.


\begin{Coro}[Ley Fuerte de los Grandes N\'umeros para Procesos de Renovaci\'on]
Si $N\left(t\right)$ es un proceso de renovaci\'on cuyos tiempos de inter-renovaci\'on tienen media $\mu\leq\infty$, entonces
\begin{eqnarray}
t^{-1}N\left(t\right)\rightarrow 1/\mu,\textrm{ c.s. cuando }t\rightarrow\infty.
\end{eqnarray}

\end{Coro}


Considerar el proceso estoc\'astico de valores reales $\left\{Z\left(t\right):t\geq0\right\}$ en el mismo espacio de probabilidad que $N\left(t\right)$

\begin{Def}
Para el proceso $\left\{Z\left(t\right):t\geq0\right\}$ se define la fluctuaci\'on m\'axima de $Z\left(t\right)$ en el intervalo $\left(T_{n-1},T_{n}\right]$:
\begin{eqnarray*}
M_{n}=\sup_{T_{n-1}<t\leq T_{n}}|Z\left(t\right)-Z\left(T_{n-1}\right)|
\end{eqnarray*}
\end{Def}

\begin{Teo}
Sup\'ongase que $n^{-1}T_{n}\rightarrow\mu$ c.s. cuando $n\rightarrow\infty$, donde $\mu\leq\infty$ es una constante o variable aleatoria. Sea $a$ una constante o variable aleatoria que puede ser infinita cuando $\mu$ es finita, y considere las expresiones l\'imite:
\begin{eqnarray}
lim_{n\rightarrow\infty}n^{-1}Z\left(T_{n}\right)&=&a,\textrm{ c.s.}\\
lim_{t\rightarrow\infty}t^{-1}Z\left(t\right)&=&a/\mu,\textrm{ c.s.}
\end{eqnarray}
La segunda expresi\'on implica la primera. Conversamente, la primera implica la segunda si el proceso $Z\left(t\right)$ es creciente, o si $lim_{n\rightarrow\infty}n^{-1}M_{n}=0$ c.s.
\end{Teo}

\begin{Coro}
Si $N\left(t\right)$ es un proceso de renovaci\'on, y $\left(Z\left(T_{n}\right)-Z\left(T_{n-1}\right),M_{n}\right)$, para $n\geq1$, son variables aleatorias independientes e id\'enticamente distribuidas con media finita, entonces,
\begin{eqnarray}
lim_{t\rightarrow\infty}t^{-1}Z\left(t\right)\rightarrow\frac{\esp\left[Z\left(T_{1}\right)-Z\left(T_{0}\right)\right]}{\esp\left[T_{1}\right]},\textrm{ c.s. cuando  }t\rightarrow\infty.
\end{eqnarray}
\end{Coro}

%___________________________________________________________________________________________
%
%\subsection{Propiedades de los Procesos de Renovaci\'on}
%___________________________________________________________________________________________
%

Los tiempos $T_{n}$ est\'an relacionados con los conteos de $N\left(t\right)$ por

\begin{eqnarray*}
\left\{N\left(t\right)\geq n\right\}&=&\left\{T_{n}\leq t\right\}\\
T_{N\left(t\right)}\leq &t&<T_{N\left(t\right)+1},
\end{eqnarray*}

adem\'as $N\left(T_{n}\right)=n$, y 

\begin{eqnarray*}
N\left(t\right)=\max\left\{n:T_{n}\leq t\right\}=\min\left\{n:T_{n+1}>t\right\}
\end{eqnarray*}

Por propiedades de la convoluci\'on se sabe que

\begin{eqnarray*}
P\left\{T_{n}\leq t\right\}=F^{n\star}\left(t\right)
\end{eqnarray*}
que es la $n$-\'esima convoluci\'on de $F$. Entonces 

\begin{eqnarray*}
\left\{N\left(t\right)\geq n\right\}&=&\left\{T_{n}\leq t\right\}\\
P\left\{N\left(t\right)\leq n\right\}&=&1-F^{\left(n+1\right)\star}\left(t\right)
\end{eqnarray*}

Adem\'as usando el hecho de que $\esp\left[N\left(t\right)\right]=\sum_{n=1}^{\infty}P\left\{N\left(t\right)\geq n\right\}$
se tiene que

\begin{eqnarray*}
\esp\left[N\left(t\right)\right]=\sum_{n=1}^{\infty}F^{n\star}\left(t\right)
\end{eqnarray*}

\begin{Prop}
Para cada $t\geq0$, la funci\'on generadora de momentos $\esp\left[e^{\alpha N\left(t\right)}\right]$ existe para alguna $\alpha$ en una vecindad del 0, y de aqu\'i que $\esp\left[N\left(t\right)^{m}\right]<\infty$, para $m\geq1$.
\end{Prop}


\begin{Note}
Si el primer tiempo de renovaci\'on $\xi_{1}$ no tiene la misma distribuci\'on que el resto de las $\xi_{n}$, para $n\geq2$, a $N\left(t\right)$ se le llama Proceso de Renovaci\'on retardado, donde si $\xi$ tiene distribuci\'on $G$, entonces el tiempo $T_{n}$ de la $n$-\'esima renovaci\'on tiene distribuci\'on $G\star F^{\left(n-1\right)\star}\left(t\right)$
\end{Note}


\begin{Teo}
Para una constante $\mu\leq\infty$ ( o variable aleatoria), las siguientes expresiones son equivalentes:

\begin{eqnarray}
lim_{n\rightarrow\infty}n^{-1}T_{n}&=&\mu,\textrm{ c.s.}\\
lim_{t\rightarrow\infty}t^{-1}N\left(t\right)&=&1/\mu,\textrm{ c.s.}
\end{eqnarray}
\end{Teo}


Es decir, $T_{n}$ satisface la Ley Fuerte de los Grandes N\'umeros s\'i y s\'olo s\'i $N\left/t\right)$ la cumple.


\begin{Coro}[Ley Fuerte de los Grandes N\'umeros para Procesos de Renovaci\'on]
Si $N\left(t\right)$ es un proceso de renovaci\'on cuyos tiempos de inter-renovaci\'on tienen media $\mu\leq\infty$, entonces
\begin{eqnarray}
t^{-1}N\left(t\right)\rightarrow 1/\mu,\textrm{ c.s. cuando }t\rightarrow\infty.
\end{eqnarray}

\end{Coro}


Considerar el proceso estoc\'astico de valores reales $\left\{Z\left(t\right):t\geq0\right\}$ en el mismo espacio de probabilidad que $N\left(t\right)$

\begin{Def}
Para el proceso $\left\{Z\left(t\right):t\geq0\right\}$ se define la fluctuaci\'on m\'axima de $Z\left(t\right)$ en el intervalo $\left(T_{n-1},T_{n}\right]$:
\begin{eqnarray*}
M_{n}=\sup_{T_{n-1}<t\leq T_{n}}|Z\left(t\right)-Z\left(T_{n-1}\right)|
\end{eqnarray*}
\end{Def}

\begin{Teo}
Sup\'ongase que $n^{-1}T_{n}\rightarrow\mu$ c.s. cuando $n\rightarrow\infty$, donde $\mu\leq\infty$ es una constante o variable aleatoria. Sea $a$ una constante o variable aleatoria que puede ser infinita cuando $\mu$ es finita, y considere las expresiones l\'imite:
\begin{eqnarray}
lim_{n\rightarrow\infty}n^{-1}Z\left(T_{n}\right)&=&a,\textrm{ c.s.}\\
lim_{t\rightarrow\infty}t^{-1}Z\left(t\right)&=&a/\mu,\textrm{ c.s.}
\end{eqnarray}
La segunda expresi\'on implica la primera. Conversamente, la primera implica la segunda si el proceso $Z\left(t\right)$ es creciente, o si $lim_{n\rightarrow\infty}n^{-1}M_{n}=0$ c.s.
\end{Teo}

\begin{Coro}
Si $N\left(t\right)$ es un proceso de renovaci\'on, y $\left(Z\left(T_{n}\right)-Z\left(T_{n-1}\right),M_{n}\right)$, para $n\geq1$, son variables aleatorias independientes e id\'enticamente distribuidas con media finita, entonces,
\begin{eqnarray}
lim_{t\rightarrow\infty}t^{-1}Z\left(t\right)\rightarrow\frac{\esp\left[Z\left(T_{1}\right)-Z\left(T_{0}\right)\right]}{\esp\left[T_{1}\right]},\textrm{ c.s. cuando  }t\rightarrow\infty.
\end{eqnarray}
\end{Coro}
%___________________________________________________________________________________________
%
%\subsection{Propiedades de los Procesos de Renovaci\'on}
%___________________________________________________________________________________________
%

Los tiempos $T_{n}$ est\'an relacionados con los conteos de $N\left(t\right)$ por

\begin{eqnarray*}
\left\{N\left(t\right)\geq n\right\}&=&\left\{T_{n}\leq t\right\}\\
T_{N\left(t\right)}\leq &t&<T_{N\left(t\right)+1},
\end{eqnarray*}

adem\'as $N\left(T_{n}\right)=n$, y 

\begin{eqnarray*}
N\left(t\right)=\max\left\{n:T_{n}\leq t\right\}=\min\left\{n:T_{n+1}>t\right\}
\end{eqnarray*}

Por propiedades de la convoluci\'on se sabe que

\begin{eqnarray*}
P\left\{T_{n}\leq t\right\}=F^{n\star}\left(t\right)
\end{eqnarray*}
que es la $n$-\'esima convoluci\'on de $F$. Entonces 

\begin{eqnarray*}
\left\{N\left(t\right)\geq n\right\}&=&\left\{T_{n}\leq t\right\}\\
P\left\{N\left(t\right)\leq n\right\}&=&1-F^{\left(n+1\right)\star}\left(t\right)
\end{eqnarray*}

Adem\'as usando el hecho de que $\esp\left[N\left(t\right)\right]=\sum_{n=1}^{\infty}P\left\{N\left(t\right)\geq n\right\}$
se tiene que

\begin{eqnarray*}
\esp\left[N\left(t\right)\right]=\sum_{n=1}^{\infty}F^{n\star}\left(t\right)
\end{eqnarray*}

\begin{Prop}
Para cada $t\geq0$, la funci\'on generadora de momentos $\esp\left[e^{\alpha N\left(t\right)}\right]$ existe para alguna $\alpha$ en una vecindad del 0, y de aqu\'i que $\esp\left[N\left(t\right)^{m}\right]<\infty$, para $m\geq1$.
\end{Prop}


\begin{Note}
Si el primer tiempo de renovaci\'on $\xi_{1}$ no tiene la misma distribuci\'on que el resto de las $\xi_{n}$, para $n\geq2$, a $N\left(t\right)$ se le llama Proceso de Renovaci\'on retardado, donde si $\xi$ tiene distribuci\'on $G$, entonces el tiempo $T_{n}$ de la $n$-\'esima renovaci\'on tiene distribuci\'on $G\star F^{\left(n-1\right)\star}\left(t\right)$
\end{Note}


\begin{Teo}
Para una constante $\mu\leq\infty$ ( o variable aleatoria), las siguientes expresiones son equivalentes:

\begin{eqnarray}
lim_{n\rightarrow\infty}n^{-1}T_{n}&=&\mu,\textrm{ c.s.}\\
lim_{t\rightarrow\infty}t^{-1}N\left(t\right)&=&1/\mu,\textrm{ c.s.}
\end{eqnarray}
\end{Teo}


Es decir, $T_{n}$ satisface la Ley Fuerte de los Grandes N\'umeros s\'i y s\'olo s\'i $N\left/t\right)$ la cumple.


\begin{Coro}[Ley Fuerte de los Grandes N\'umeros para Procesos de Renovaci\'on]
Si $N\left(t\right)$ es un proceso de renovaci\'on cuyos tiempos de inter-renovaci\'on tienen media $\mu\leq\infty$, entonces
\begin{eqnarray}
t^{-1}N\left(t\right)\rightarrow 1/\mu,\textrm{ c.s. cuando }t\rightarrow\infty.
\end{eqnarray}

\end{Coro}


Considerar el proceso estoc\'astico de valores reales $\left\{Z\left(t\right):t\geq0\right\}$ en el mismo espacio de probabilidad que $N\left(t\right)$

\begin{Def}
Para el proceso $\left\{Z\left(t\right):t\geq0\right\}$ se define la fluctuaci\'on m\'axima de $Z\left(t\right)$ en el intervalo $\left(T_{n-1},T_{n}\right]$:
\begin{eqnarray*}
M_{n}=\sup_{T_{n-1}<t\leq T_{n}}|Z\left(t\right)-Z\left(T_{n-1}\right)|
\end{eqnarray*}
\end{Def}

\begin{Teo}
Sup\'ongase que $n^{-1}T_{n}\rightarrow\mu$ c.s. cuando $n\rightarrow\infty$, donde $\mu\leq\infty$ es una constante o variable aleatoria. Sea $a$ una constante o variable aleatoria que puede ser infinita cuando $\mu$ es finita, y considere las expresiones l\'imite:
\begin{eqnarray}
lim_{n\rightarrow\infty}n^{-1}Z\left(T_{n}\right)&=&a,\textrm{ c.s.}\\
lim_{t\rightarrow\infty}t^{-1}Z\left(t\right)&=&a/\mu,\textrm{ c.s.}
\end{eqnarray}
La segunda expresi\'on implica la primera. Conversamente, la primera implica la segunda si el proceso $Z\left(t\right)$ es creciente, o si $lim_{n\rightarrow\infty}n^{-1}M_{n}=0$ c.s.
\end{Teo}

\begin{Coro}
Si $N\left(t\right)$ es un proceso de renovaci\'on, y $\left(Z\left(T_{n}\right)-Z\left(T_{n-1}\right),M_{n}\right)$, para $n\geq1$, son variables aleatorias independientes e id\'enticamente distribuidas con media finita, entonces,
\begin{eqnarray}
lim_{t\rightarrow\infty}t^{-1}Z\left(t\right)\rightarrow\frac{\esp\left[Z\left(T_{1}\right)-Z\left(T_{0}\right)\right]}{\esp\left[T_{1}\right]},\textrm{ c.s. cuando  }t\rightarrow\infty.
\end{eqnarray}
\end{Coro}


%___________________________________________________________________________________________
%
%\subsection*{Funci\'on de Renovaci\'on}
%___________________________________________________________________________________________
%


\begin{Def}
Sea $h\left(t\right)$ funci\'on de valores reales en $\rea$ acotada en intervalos finitos e igual a cero para $t<0$ La ecuaci\'on de renovaci\'on para $h\left(t\right)$ y la distribuci\'on $F$ es

\begin{eqnarray}\label{Ec.Renovacion}
H\left(t\right)=h\left(t\right)+\int_{\left[0,t\right]}H\left(t-s\right)dF\left(s\right)\textrm{,    }t\geq0,
\end{eqnarray}
donde $H\left(t\right)$ es una funci\'on de valores reales. Esto es $H=h+F\star H$. Decimos que $H\left(t\right)$ es soluci\'on de esta ecuaci\'on si satisface la ecuaci\'on, y es acotada en intervalos finitos e iguales a cero para $t<0$.
\end{Def}

\begin{Prop}
La funci\'on $U\star h\left(t\right)$ es la \'unica soluci\'on de la ecuaci\'on de renovaci\'on (\ref{Ec.Renovacion}).
\end{Prop}

\begin{Teo}[Teorema Renovaci\'on Elemental]
\begin{eqnarray*}
t^{-1}U\left(t\right)\rightarrow 1/\mu\textrm{,    cuando }t\rightarrow\infty.
\end{eqnarray*}
\end{Teo}

%___________________________________________________________________________________________
%
%\subsection{Funci\'on de Renovaci\'on}
%___________________________________________________________________________________________
%


Sup\'ongase que $N\left(t\right)$ es un proceso de renovaci\'on con distribuci\'on $F$ con media finita $\mu$.

\begin{Def}
La funci\'on de renovaci\'on asociada con la distribuci\'on $F$, del proceso $N\left(t\right)$, es
\begin{eqnarray*}
U\left(t\right)=\sum_{n=1}^{\infty}F^{n\star}\left(t\right),\textrm{   }t\geq0,
\end{eqnarray*}
donde $F^{0\star}\left(t\right)=\indora\left(t\geq0\right)$.
\end{Def}


\begin{Prop}
Sup\'ongase que la distribuci\'on de inter-renovaci\'on $F$ tiene densidad $f$. Entonces $U\left(t\right)$ tambi\'en tiene densidad, para $t>0$, y es $U^{'}\left(t\right)=\sum_{n=0}^{\infty}f^{n\star}\left(t\right)$. Adem\'as
\begin{eqnarray*}
\prob\left\{N\left(t\right)>N\left(t-\right)\right\}=0\textrm{,   }t\geq0.
\end{eqnarray*}
\end{Prop}

\begin{Def}
La Transformada de Laplace-Stieljes de $F$ est\'a dada por

\begin{eqnarray*}
\hat{F}\left(\alpha\right)=\int_{\rea_{+}}e^{-\alpha t}dF\left(t\right)\textrm{,  }\alpha\geq0.
\end{eqnarray*}
\end{Def}

Entonces

\begin{eqnarray*}
\hat{U}\left(\alpha\right)=\sum_{n=0}^{\infty}\hat{F^{n\star}}\left(\alpha\right)=\sum_{n=0}^{\infty}\hat{F}\left(\alpha\right)^{n}=\frac{1}{1-\hat{F}\left(\alpha\right)}.
\end{eqnarray*}


\begin{Prop}
La Transformada de Laplace $\hat{U}\left(\alpha\right)$ y $\hat{F}\left(\alpha\right)$ determina una a la otra de manera \'unica por la relaci\'on $\hat{U}\left(\alpha\right)=\frac{1}{1-\hat{F}\left(\alpha\right)}$.
\end{Prop}


\begin{Note}
Un proceso de renovaci\'on $N\left(t\right)$ cuyos tiempos de inter-renovaci\'on tienen media finita, es un proceso Poisson con tasa $\lambda$ si y s\'olo s\'i $\esp\left[U\left(t\right)\right]=\lambda t$, para $t\geq0$.
\end{Note}


\begin{Teo}
Sea $N\left(t\right)$ un proceso puntual simple con puntos de localizaci\'on $T_{n}$ tal que $\eta\left(t\right)=\esp\left[N\left(\right)\right]$ es finita para cada $t$. Entonces para cualquier funci\'on $f:\rea_{+}\rightarrow\rea$,
\begin{eqnarray*}
\esp\left[\sum_{n=1}^{N\left(\right)}f\left(T_{n}\right)\right]=\int_{\left(0,t\right]}f\left(s\right)d\eta\left(s\right)\textrm{,  }t\geq0,
\end{eqnarray*}
suponiendo que la integral exista. Adem\'as si $X_{1},X_{2},\ldots$ son variables aleatorias definidas en el mismo espacio de probabilidad que el proceso $N\left(t\right)$ tal que $\esp\left[X_{n}|T_{n}=s\right]=f\left(s\right)$, independiente de $n$. Entonces
\begin{eqnarray*}
\esp\left[\sum_{n=1}^{N\left(t\right)}X_{n}\right]=\int_{\left(0,t\right]}f\left(s\right)d\eta\left(s\right)\textrm{,  }t\geq0,
\end{eqnarray*} 
suponiendo que la integral exista. 
\end{Teo}

\begin{Coro}[Identidad de Wald para Renovaciones]
Para el proceso de renovaci\'on $N\left(t\right)$,
\begin{eqnarray*}
\esp\left[T_{N\left(t\right)+1}\right]=\mu\esp\left[N\left(t\right)+1\right]\textrm{,  }t\geq0,
\end{eqnarray*}  
\end{Coro}

%______________________________________________________________________
%\subsection{Procesos de Renovaci\'on}
%______________________________________________________________________

\begin{Def}\label{Def.Tn}
Sean $0\leq T_{1}\leq T_{2}\leq \ldots$ son tiempos aleatorios infinitos en los cuales ocurren ciertos eventos. El n\'umero de tiempos $T_{n}$ en el intervalo $\left[0,t\right)$ es

\begin{eqnarray}
N\left(t\right)=\sum_{n=1}^{\infty}\indora\left(T_{n}\leq t\right),
\end{eqnarray}
para $t\geq0$.
\end{Def}

Si se consideran los puntos $T_{n}$ como elementos de $\rea_{+}$, y $N\left(t\right)$ es el n\'umero de puntos en $\rea$. El proceso denotado por $\left\{N\left(t\right):t\geq0\right\}$, denotado por $N\left(t\right)$, es un proceso puntual en $\rea_{+}$. Los $T_{n}$ son los tiempos de ocurrencia, el proceso puntual $N\left(t\right)$ es simple si su n\'umero de ocurrencias son distintas: $0<T_{1}<T_{2}<\ldots$ casi seguramente.

\begin{Def}
Un proceso puntual $N\left(t\right)$ es un proceso de renovaci\'on si los tiempos de interocurrencia $\xi_{n}=T_{n}-T_{n-1}$, para $n\geq1$, son independientes e identicamente distribuidos con distribuci\'on $F$, donde $F\left(0\right)=0$ y $T_{0}=0$. Los $T_{n}$ son llamados tiempos de renovaci\'on, referente a la independencia o renovaci\'on de la informaci\'on estoc\'astica en estos tiempos. Los $\xi_{n}$ son los tiempos de inter-renovaci\'on, y $N\left(t\right)$ es el n\'umero de renovaciones en el intervalo $\left[0,t\right)$
\end{Def}


\begin{Note}
Para definir un proceso de renovaci\'on para cualquier contexto, solamente hay que especificar una distribuci\'on $F$, con $F\left(0\right)=0$, para los tiempos de inter-renovaci\'on. La funci\'on $F$ en turno degune las otra variables aleatorias. De manera formal, existe un espacio de probabilidad y una sucesi\'on de variables aleatorias $\xi_{1},\xi_{2},\ldots$ definidas en este con distribuci\'on $F$. Entonces las otras cantidades son $T_{n}=\sum_{k=1}^{n}\xi_{k}$ y $N\left(t\right)=\sum_{n=1}^{\infty}\indora\left(T_{n}\leq t\right)$, donde $T_{n}\rightarrow\infty$ casi seguramente por la Ley Fuerte de los Grandes Números.
\end{Note}

%___________________________________________________________________________________________
%
\subsection{Renewal and Regenerative Processes: Serfozo\cite{Serfozo}}
%___________________________________________________________________________________________
%
\begin{Def}\label{Def.Tn}
Sean $0\leq T_{1}\leq T_{2}\leq \ldots$ son tiempos aleatorios infinitos en los cuales ocurren ciertos eventos. El n\'umero de tiempos $T_{n}$ en el intervalo $\left[0,t\right)$ es

\begin{eqnarray}
N\left(t\right)=\sum_{n=1}^{\infty}\indora\left(T_{n}\leq t\right),
\end{eqnarray}
para $t\geq0$.
\end{Def}

Si se consideran los puntos $T_{n}$ como elementos de $\rea_{+}$, y $N\left(t\right)$ es el n\'umero de puntos en $\rea$. El proceso denotado por $\left\{N\left(t\right):t\geq0\right\}$, denotado por $N\left(t\right)$, es un proceso puntual en $\rea_{+}$. Los $T_{n}$ son los tiempos de ocurrencia, el proceso puntual $N\left(t\right)$ es simple si su n\'umero de ocurrencias son distintas: $0<T_{1}<T_{2}<\ldots$ casi seguramente.

\begin{Def}
Un proceso puntual $N\left(t\right)$ es un proceso de renovaci\'on si los tiempos de interocurrencia $\xi_{n}=T_{n}-T_{n-1}$, para $n\geq1$, son independientes e identicamente distribuidos con distribuci\'on $F$, donde $F\left(0\right)=0$ y $T_{0}=0$. Los $T_{n}$ son llamados tiempos de renovaci\'on, referente a la independencia o renovaci\'on de la informaci\'on estoc\'astica en estos tiempos. Los $\xi_{n}$ son los tiempos de inter-renovaci\'on, y $N\left(t\right)$ es el n\'umero de renovaciones en el intervalo $\left[0,t\right)$
\end{Def}


\begin{Note}
Para definir un proceso de renovaci\'on para cualquier contexto, solamente hay que especificar una distribuci\'on $F$, con $F\left(0\right)=0$, para los tiempos de inter-renovaci\'on. La funci\'on $F$ en turno degune las otra variables aleatorias. De manera formal, existe un espacio de probabilidad y una sucesi\'on de variables aleatorias $\xi_{1},\xi_{2},\ldots$ definidas en este con distribuci\'on $F$. Entonces las otras cantidades son $T_{n}=\sum_{k=1}^{n}\xi_{k}$ y $N\left(t\right)=\sum_{n=1}^{\infty}\indora\left(T_{n}\leq t\right)$, donde $T_{n}\rightarrow\infty$ casi seguramente por la Ley Fuerte de los Grandes N\'umeros.
\end{Note}







Los tiempos $T_{n}$ est\'an relacionados con los conteos de $N\left(t\right)$ por

\begin{eqnarray*}
\left\{N\left(t\right)\geq n\right\}&=&\left\{T_{n}\leq t\right\}\\
T_{N\left(t\right)}\leq &t&<T_{N\left(t\right)+1},
\end{eqnarray*}

adem\'as $N\left(T_{n}\right)=n$, y 

\begin{eqnarray*}
N\left(t\right)=\max\left\{n:T_{n}\leq t\right\}=\min\left\{n:T_{n+1}>t\right\}
\end{eqnarray*}

Por propiedades de la convoluci\'on se sabe que

\begin{eqnarray*}
P\left\{T_{n}\leq t\right\}=F^{n\star}\left(t\right)
\end{eqnarray*}
que es la $n$-\'esima convoluci\'on de $F$. Entonces 

\begin{eqnarray*}
\left\{N\left(t\right)\geq n\right\}&=&\left\{T_{n}\leq t\right\}\\
P\left\{N\left(t\right)\leq n\right\}&=&1-F^{\left(n+1\right)\star}\left(t\right)
\end{eqnarray*}

Adem\'as usando el hecho de que $\esp\left[N\left(t\right)\right]=\sum_{n=1}^{\infty}P\left\{N\left(t\right)\geq n\right\}$
se tiene que

\begin{eqnarray*}
\esp\left[N\left(t\right)\right]=\sum_{n=1}^{\infty}F^{n\star}\left(t\right)
\end{eqnarray*}

\begin{Prop}
Para cada $t\geq0$, la funci\'on generadora de momentos $\esp\left[e^{\alpha N\left(t\right)}\right]$ existe para alguna $\alpha$ en una vecindad del 0, y de aqu\'i que $\esp\left[N\left(t\right)^{m}\right]<\infty$, para $m\geq1$.
\end{Prop}

\begin{Ejem}[\textbf{Proceso Poisson}]

Suponga que se tienen tiempos de inter-renovaci\'on \textit{i.i.d.} del proceso de renovaci\'on $N\left(t\right)$ tienen distribuci\'on exponencial $F\left(t\right)=q-e^{-\lambda t}$ con tasa $\lambda$. Entonces $N\left(t\right)$ es un proceso Poisson con tasa $\lambda$.

\end{Ejem}


\begin{Note}
Si el primer tiempo de renovaci\'on $\xi_{1}$ no tiene la misma distribuci\'on que el resto de las $\xi_{n}$, para $n\geq2$, a $N\left(t\right)$ se le llama Proceso de Renovaci\'on retardado, donde si $\xi$ tiene distribuci\'on $G$, entonces el tiempo $T_{n}$ de la $n$-\'esima renovaci\'on tiene distribuci\'on $G\star F^{\left(n-1\right)\star}\left(t\right)$
\end{Note}


\begin{Teo}
Para una constante $\mu\leq\infty$ ( o variable aleatoria), las siguientes expresiones son equivalentes:

\begin{eqnarray}
lim_{n\rightarrow\infty}n^{-1}T_{n}&=&\mu,\textrm{ c.s.}\\
lim_{t\rightarrow\infty}t^{-1}N\left(t\right)&=&1/\mu,\textrm{ c.s.}
\end{eqnarray}
\end{Teo}


Es decir, $T_{n}$ satisface la Ley Fuerte de los Grandes N\'umeros s\'i y s\'olo s\'i $N\left/t\right)$ la cumple.


\begin{Coro}[Ley Fuerte de los Grandes N\'umeros para Procesos de Renovaci\'on]
Si $N\left(t\right)$ es un proceso de renovaci\'on cuyos tiempos de inter-renovaci\'on tienen media $\mu\leq\infty$, entonces
\begin{eqnarray}
t^{-1}N\left(t\right)\rightarrow 1/\mu,\textrm{ c.s. cuando }t\rightarrow\infty.
\end{eqnarray}

\end{Coro}


Considerar el proceso estoc\'astico de valores reales $\left\{Z\left(t\right):t\geq0\right\}$ en el mismo espacio de probabilidad que $N\left(t\right)$

\begin{Def}
Para el proceso $\left\{Z\left(t\right):t\geq0\right\}$ se define la fluctuaci\'on m\'axima de $Z\left(t\right)$ en el intervalo $\left(T_{n-1},T_{n}\right]$:
\begin{eqnarray*}
M_{n}=\sup_{T_{n-1}<t\leq T_{n}}|Z\left(t\right)-Z\left(T_{n-1}\right)|
\end{eqnarray*}
\end{Def}

\begin{Teo}
Sup\'ongase que $n^{-1}T_{n}\rightarrow\mu$ c.s. cuando $n\rightarrow\infty$, donde $\mu\leq\infty$ es una constante o variable aleatoria. Sea $a$ una constante o variable aleatoria que puede ser infinita cuando $\mu$ es finita, y considere las expresiones l\'imite:
\begin{eqnarray}
lim_{n\rightarrow\infty}n^{-1}Z\left(T_{n}\right)&=&a,\textrm{ c.s.}\\
lim_{t\rightarrow\infty}t^{-1}Z\left(t\right)&=&a/\mu,\textrm{ c.s.}
\end{eqnarray}
La segunda expresi\'on implica la primera. Conversamente, la primera implica la segunda si el proceso $Z\left(t\right)$ es creciente, o si $lim_{n\rightarrow\infty}n^{-1}M_{n}=0$ c.s.
\end{Teo}

\begin{Coro}
Si $N\left(t\right)$ es un proceso de renovaci\'on, y $\left(Z\left(T_{n}\right)-Z\left(T_{n-1}\right),M_{n}\right)$, para $n\geq1$, son variables aleatorias independientes e id\'enticamente distribuidas con media finita, entonces,
\begin{eqnarray}
lim_{t\rightarrow\infty}t^{-1}Z\left(t\right)\rightarrow\frac{\esp\left[Z\left(T_{1}\right)-Z\left(T_{0}\right)\right]}{\esp\left[T_{1}\right]},\textrm{ c.s. cuando  }t\rightarrow\infty.
\end{eqnarray}
\end{Coro}


Sup\'ongase que $N\left(t\right)$ es un proceso de renovaci\'on con distribuci\'on $F$ con media finita $\mu$.

\begin{Def}
La funci\'on de renovaci\'on asociada con la distribuci\'on $F$, del proceso $N\left(t\right)$, es
\begin{eqnarray*}
U\left(t\right)=\sum_{n=1}^{\infty}F^{n\star}\left(t\right),\textrm{   }t\geq0,
\end{eqnarray*}
donde $F^{0\star}\left(t\right)=\indora\left(t\geq0\right)$.
\end{Def}


\begin{Prop}
Sup\'ongase que la distribuci\'on de inter-renovaci\'on $F$ tiene densidad $f$. Entonces $U\left(t\right)$ tambi\'en tiene densidad, para $t>0$, y es $U^{'}\left(t\right)=\sum_{n=0}^{\infty}f^{n\star}\left(t\right)$. Adem\'as
\begin{eqnarray*}
\prob\left\{N\left(t\right)>N\left(t-\right)\right\}=0\textrm{,   }t\geq0.
\end{eqnarray*}
\end{Prop}

\begin{Def}
La Transformada de Laplace-Stieljes de $F$ est\'a dada por

\begin{eqnarray*}
\hat{F}\left(\alpha\right)=\int_{\rea_{+}}e^{-\alpha t}dF\left(t\right)\textrm{,  }\alpha\geq0.
\end{eqnarray*}
\end{Def}

Entonces

\begin{eqnarray*}
\hat{U}\left(\alpha\right)=\sum_{n=0}^{\infty}\hat{F^{n\star}}\left(\alpha\right)=\sum_{n=0}^{\infty}\hat{F}\left(\alpha\right)^{n}=\frac{1}{1-\hat{F}\left(\alpha\right)}.
\end{eqnarray*}


\begin{Prop}
La Transformada de Laplace $\hat{U}\left(\alpha\right)$ y $\hat{F}\left(\alpha\right)$ determina una a la otra de manera \'unica por la relaci\'on $\hat{U}\left(\alpha\right)=\frac{1}{1-\hat{F}\left(\alpha\right)}$.
\end{Prop}


\begin{Note}
Un proceso de renovaci\'on $N\left(t\right)$ cuyos tiempos de inter-renovaci\'on tienen media finita, es un proceso Poisson con tasa $\lambda$ si y s\'olo s\'i $\esp\left[U\left(t\right)\right]=\lambda t$, para $t\geq0$.
\end{Note}


\begin{Teo}
Sea $N\left(t\right)$ un proceso puntual simple con puntos de localizaci\'on $T_{n}$ tal que $\eta\left(t\right)=\esp\left[N\left(\right)\right]$ es finita para cada $t$. Entonces para cualquier funci\'on $f:\rea_{+}\rightarrow\rea$,
\begin{eqnarray*}
\esp\left[\sum_{n=1}^{N\left(\right)}f\left(T_{n}\right)\right]=\int_{\left(0,t\right]}f\left(s\right)d\eta\left(s\right)\textrm{,  }t\geq0,
\end{eqnarray*}
suponiendo que la integral exista. Adem\'as si $X_{1},X_{2},\ldots$ son variables aleatorias definidas en el mismo espacio de probabilidad que el proceso $N\left(t\right)$ tal que $\esp\left[X_{n}|T_{n}=s\right]=f\left(s\right)$, independiente de $n$. Entonces
\begin{eqnarray*}
\esp\left[\sum_{n=1}^{N\left(t\right)}X_{n}\right]=\int_{\left(0,t\right]}f\left(s\right)d\eta\left(s\right)\textrm{,  }t\geq0,
\end{eqnarray*} 
suponiendo que la integral exista. 
\end{Teo}

\begin{Coro}[Identidad de Wald para Renovaciones]
Para el proceso de renovaci\'on $N\left(t\right)$,
\begin{eqnarray*}
\esp\left[T_{N\left(t\right)+1}\right]=\mu\esp\left[N\left(t\right)+1\right]\textrm{,  }t\geq0,
\end{eqnarray*}  
\end{Coro}


\begin{Def}
Sea $h\left(t\right)$ funci\'on de valores reales en $\rea$ acotada en intervalos finitos e igual a cero para $t<0$ La ecuaci\'on de renovaci\'on para $h\left(t\right)$ y la distribuci\'on $F$ es

\begin{eqnarray}\label{Ec.Renovacion}
H\left(t\right)=h\left(t\right)+\int_{\left[0,t\right]}H\left(t-s\right)dF\left(s\right)\textrm{,    }t\geq0,
\end{eqnarray}
donde $H\left(t\right)$ es una funci\'on de valores reales. Esto es $H=h+F\star H$. Decimos que $H\left(t\right)$ es soluci\'on de esta ecuaci\'on si satisface la ecuaci\'on, y es acotada en intervalos finitos e iguales a cero para $t<0$.
\end{Def}

\begin{Prop}
La funci\'on $U\star h\left(t\right)$ es la \'unica soluci\'on de la ecuaci\'on de renovaci\'on (\ref{Ec.Renovacion}).
\end{Prop}

\begin{Teo}[Teorema Renovaci\'on Elemental]
\begin{eqnarray*}
t^{-1}U\left(t\right)\rightarrow 1/\mu\textrm{,    cuando }t\rightarrow\infty.
\end{eqnarray*}
\end{Teo}



Sup\'ongase que $N\left(t\right)$ es un proceso de renovaci\'on con distribuci\'on $F$ con media finita $\mu$.

\begin{Def}
La funci\'on de renovaci\'on asociada con la distribuci\'on $F$, del proceso $N\left(t\right)$, es
\begin{eqnarray*}
U\left(t\right)=\sum_{n=1}^{\infty}F^{n\star}\left(t\right),\textrm{   }t\geq0,
\end{eqnarray*}
donde $F^{0\star}\left(t\right)=\indora\left(t\geq0\right)$.
\end{Def}


\begin{Prop}
Sup\'ongase que la distribuci\'on de inter-renovaci\'on $F$ tiene densidad $f$. Entonces $U\left(t\right)$ tambi\'en tiene densidad, para $t>0$, y es $U^{'}\left(t\right)=\sum_{n=0}^{\infty}f^{n\star}\left(t\right)$. Adem\'as
\begin{eqnarray*}
\prob\left\{N\left(t\right)>N\left(t-\right)\right\}=0\textrm{,   }t\geq0.
\end{eqnarray*}
\end{Prop}

\begin{Def}
La Transformada de Laplace-Stieljes de $F$ est\'a dada por

\begin{eqnarray*}
\hat{F}\left(\alpha\right)=\int_{\rea_{+}}e^{-\alpha t}dF\left(t\right)\textrm{,  }\alpha\geq0.
\end{eqnarray*}
\end{Def}

Entonces

\begin{eqnarray*}
\hat{U}\left(\alpha\right)=\sum_{n=0}^{\infty}\hat{F^{n\star}}\left(\alpha\right)=\sum_{n=0}^{\infty}\hat{F}\left(\alpha\right)^{n}=\frac{1}{1-\hat{F}\left(\alpha\right)}.
\end{eqnarray*}


\begin{Prop}
La Transformada de Laplace $\hat{U}\left(\alpha\right)$ y $\hat{F}\left(\alpha\right)$ determina una a la otra de manera \'unica por la relaci\'on $\hat{U}\left(\alpha\right)=\frac{1}{1-\hat{F}\left(\alpha\right)}$.
\end{Prop}


\begin{Note}
Un proceso de renovaci\'on $N\left(t\right)$ cuyos tiempos de inter-renovaci\'on tienen media finita, es un proceso Poisson con tasa $\lambda$ si y s\'olo s\'i $\esp\left[U\left(t\right)\right]=\lambda t$, para $t\geq0$.
\end{Note}


\begin{Teo}
Sea $N\left(t\right)$ un proceso puntual simple con puntos de localizaci\'on $T_{n}$ tal que $\eta\left(t\right)=\esp\left[N\left(\right)\right]$ es finita para cada $t$. Entonces para cualquier funci\'on $f:\rea_{+}\rightarrow\rea$,
\begin{eqnarray*}
\esp\left[\sum_{n=1}^{N\left(\right)}f\left(T_{n}\right)\right]=\int_{\left(0,t\right]}f\left(s\right)d\eta\left(s\right)\textrm{,  }t\geq0,
\end{eqnarray*}
suponiendo que la integral exista. Adem\'as si $X_{1},X_{2},\ldots$ son variables aleatorias definidas en el mismo espacio de probabilidad que el proceso $N\left(t\right)$ tal que $\esp\left[X_{n}|T_{n}=s\right]=f\left(s\right)$, independiente de $n$. Entonces
\begin{eqnarray*}
\esp\left[\sum_{n=1}^{N\left(t\right)}X_{n}\right]=\int_{\left(0,t\right]}f\left(s\right)d\eta\left(s\right)\textrm{,  }t\geq0,
\end{eqnarray*} 
suponiendo que la integral exista. 
\end{Teo}

\begin{Coro}[Identidad de Wald para Renovaciones]
Para el proceso de renovaci\'on $N\left(t\right)$,
\begin{eqnarray*}
\esp\left[T_{N\left(t\right)+1}\right]=\mu\esp\left[N\left(t\right)+1\right]\textrm{,  }t\geq0,
\end{eqnarray*}  
\end{Coro}


\begin{Def}
Sea $h\left(t\right)$ funci\'on de valores reales en $\rea$ acotada en intervalos finitos e igual a cero para $t<0$ La ecuaci\'on de renovaci\'on para $h\left(t\right)$ y la distribuci\'on $F$ es

\begin{eqnarray}\label{Ec.Renovacion}
H\left(t\right)=h\left(t\right)+\int_{\left[0,t\right]}H\left(t-s\right)dF\left(s\right)\textrm{,    }t\geq0,
\end{eqnarray}
donde $H\left(t\right)$ es una funci\'on de valores reales. Esto es $H=h+F\star H$. Decimos que $H\left(t\right)$ es soluci\'on de esta ecuaci\'on si satisface la ecuaci\'on, y es acotada en intervalos finitos e iguales a cero para $t<0$.
\end{Def}

\begin{Prop}
La funci\'on $U\star h\left(t\right)$ es la \'unica soluci\'on de la ecuaci\'on de renovaci\'on (\ref{Ec.Renovacion}).
\end{Prop}

\begin{Teo}[Teorema Renovaci\'on Elemental]
\begin{eqnarray*}
t^{-1}U\left(t\right)\rightarrow 1/\mu\textrm{,    cuando }t\rightarrow\infty.
\end{eqnarray*}
\end{Teo}


\begin{Note} Una funci\'on $h:\rea_{+}\rightarrow\rea$ es Directamente Riemann Integrable en los siguientes casos:
\begin{itemize}
\item[a)] $h\left(t\right)\geq0$ es decreciente y Riemann Integrable.
\item[b)] $h$ es continua excepto posiblemente en un conjunto de Lebesgue de medida 0, y $|h\left(t\right)|\leq b\left(t\right)$, donde $b$ es DRI.
\end{itemize}
\end{Note}

\begin{Teo}[Teorema Principal de Renovaci\'on]
Si $F$ es no aritm\'etica y $h\left(t\right)$ es Directamente Riemann Integrable (DRI), entonces

\begin{eqnarray*}
lim_{t\rightarrow\infty}U\star h=\frac{1}{\mu}\int_{\rea_{+}}h\left(s\right)ds.
\end{eqnarray*}
\end{Teo}

\begin{Prop}
Cualquier funci\'on $H\left(t\right)$ acotada en intervalos finitos y que es 0 para $t<0$ puede expresarse como
\begin{eqnarray*}
H\left(t\right)=U\star h\left(t\right)\textrm{,  donde }h\left(t\right)=H\left(t\right)-F\star H\left(t\right)
\end{eqnarray*}
\end{Prop}

\begin{Def}
Un proceso estoc\'astico $X\left(t\right)$ es crudamente regenerativo en un tiempo aleatorio positivo $T$ si
\begin{eqnarray*}
\esp\left[X\left(T+t\right)|T\right]=\esp\left[X\left(t\right)\right]\textrm{, para }t\geq0,\end{eqnarray*}
y con las esperanzas anteriores finitas.
\end{Def}

\begin{Prop}
Sup\'ongase que $X\left(t\right)$ es un proceso crudamente regenerativo en $T$, que tiene distribuci\'on $F$. Si $\esp\left[X\left(t\right)\right]$ es acotado en intervalos finitos, entonces
\begin{eqnarray*}
\esp\left[X\left(t\right)\right]=U\star h\left(t\right)\textrm{,  donde }h\left(t\right)=\esp\left[X\left(t\right)\indora\left(T>t\right)\right].
\end{eqnarray*}
\end{Prop}

\begin{Teo}[Regeneraci\'on Cruda]
Sup\'ongase que $X\left(t\right)$ es un proceso con valores positivo crudamente regenerativo en $T$, y def\'inase $M=\sup\left\{|X\left(t\right)|:t\leq T\right\}$. Si $T$ es no aritm\'etico y $M$ y $MT$ tienen media finita, entonces
\begin{eqnarray*}
lim_{t\rightarrow\infty}\esp\left[X\left(t\right)\right]=\frac{1}{\mu}\int_{\rea_{+}}h\left(s\right)ds,
\end{eqnarray*}
donde $h\left(t\right)=\esp\left[X\left(t\right)\indora\left(T>t\right)\right]$.
\end{Teo}


\begin{Note} Una funci\'on $h:\rea_{+}\rightarrow\rea$ es Directamente Riemann Integrable en los siguientes casos:
\begin{itemize}
\item[a)] $h\left(t\right)\geq0$ es decreciente y Riemann Integrable.
\item[b)] $h$ es continua excepto posiblemente en un conjunto de Lebesgue de medida 0, y $|h\left(t\right)|\leq b\left(t\right)$, donde $b$ es DRI.
\end{itemize}
\end{Note}

\begin{Teo}[Teorema Principal de Renovaci\'on]
Si $F$ es no aritm\'etica y $h\left(t\right)$ es Directamente Riemann Integrable (DRI), entonces

\begin{eqnarray*}
lim_{t\rightarrow\infty}U\star h=\frac{1}{\mu}\int_{\rea_{+}}h\left(s\right)ds.
\end{eqnarray*}
\end{Teo}

\begin{Prop}
Cualquier funci\'on $H\left(t\right)$ acotada en intervalos finitos y que es 0 para $t<0$ puede expresarse como
\begin{eqnarray*}
H\left(t\right)=U\star h\left(t\right)\textrm{,  donde }h\left(t\right)=H\left(t\right)-F\star H\left(t\right)
\end{eqnarray*}
\end{Prop}

\begin{Def}
Un proceso estoc\'astico $X\left(t\right)$ es crudamente regenerativo en un tiempo aleatorio positivo $T$ si
\begin{eqnarray*}
\esp\left[X\left(T+t\right)|T\right]=\esp\left[X\left(t\right)\right]\textrm{, para }t\geq0,\end{eqnarray*}
y con las esperanzas anteriores finitas.
\end{Def}

\begin{Prop}
Sup\'ongase que $X\left(t\right)$ es un proceso crudamente regenerativo en $T$, que tiene distribuci\'on $F$. Si $\esp\left[X\left(t\right)\right]$ es acotado en intervalos finitos, entonces
\begin{eqnarray*}
\esp\left[X\left(t\right)\right]=U\star h\left(t\right)\textrm{,  donde }h\left(t\right)=\esp\left[X\left(t\right)\indora\left(T>t\right)\right].
\end{eqnarray*}
\end{Prop}

\begin{Teo}[Regeneraci\'on Cruda]
Sup\'ongase que $X\left(t\right)$ es un proceso con valores positivo crudamente regenerativo en $T$, y def\'inase $M=\sup\left\{|X\left(t\right)|:t\leq T\right\}$. Si $T$ es no aritm\'etico y $M$ y $MT$ tienen media finita, entonces
\begin{eqnarray*}
lim_{t\rightarrow\infty}\esp\left[X\left(t\right)\right]=\frac{1}{\mu}\int_{\rea_{+}}h\left(s\right)ds,
\end{eqnarray*}
donde $h\left(t\right)=\esp\left[X\left(t\right)\indora\left(T>t\right)\right]$.
\end{Teo}

\begin{Def}
Para el proceso $\left\{\left(N\left(t\right),X\left(t\right)\right):t\geq0\right\}$, sus trayectoria muestrales en el intervalo de tiempo $\left[T_{n-1},T_{n}\right)$ est\'an descritas por
\begin{eqnarray*}
\zeta_{n}=\left(\xi_{n},\left\{X\left(T_{n-1}+t\right):0\leq t<\xi_{n}\right\}\right)
\end{eqnarray*}
Este $\zeta_{n}$ es el $n$-\'esimo segmento del proceso. El proceso es regenerativo sobre los tiempos $T_{n}$ si sus segmentos $\zeta_{n}$ son independientes e id\'enticamennte distribuidos.
\end{Def}


\begin{Note}
Si $\tilde{X}\left(t\right)$ con espacio de estados $\tilde{S}$ es regenerativo sobre $T_{n}$, entonces $X\left(t\right)=f\left(\tilde{X}\left(t\right)\right)$ tambi\'en es regenerativo sobre $T_{n}$, para cualquier funci\'on $f:\tilde{S}\rightarrow S$.
\end{Note}

\begin{Note}
Los procesos regenerativos son crudamente regenerativos, pero no al rev\'es.
\end{Note}


\begin{Note}
Un proceso estoc\'astico a tiempo continuo o discreto es regenerativo si existe un proceso de renovaci\'on  tal que los segmentos del proceso entre tiempos de renovaci\'on sucesivos son i.i.d., es decir, para $\left\{X\left(t\right):t\geq0\right\}$ proceso estoc\'astico a tiempo continuo con espacio de estados $S$, espacio m\'etrico.
\end{Note}

Para $\left\{X\left(t\right):t\geq0\right\}$ Proceso Estoc\'astico a tiempo continuo con estado de espacios $S$, que es un espacio m\'etrico, con trayectorias continuas por la derecha y con l\'imites por la izquierda c.s. Sea $N\left(t\right)$ un proceso de renovaci\'on en $\rea_{+}$ definido en el mismo espacio de probabilidad que $X\left(t\right)$, con tiempos de renovaci\'on $T$ y tiempos de inter-renovaci\'on $\xi_{n}=T_{n}-T_{n-1}$, con misma distribuci\'on $F$ de media finita $\mu$.



\begin{Def}
Para el proceso $\left\{\left(N\left(t\right),X\left(t\right)\right):t\geq0\right\}$, sus trayectoria muestrales en el intervalo de tiempo $\left[T_{n-1},T_{n}\right)$ est\'an descritas por
\begin{eqnarray*}
\zeta_{n}=\left(\xi_{n},\left\{X\left(T_{n-1}+t\right):0\leq t<\xi_{n}\right\}\right)
\end{eqnarray*}
Este $\zeta_{n}$ es el $n$-\'esimo segmento del proceso. El proceso es regenerativo sobre los tiempos $T_{n}$ si sus segmentos $\zeta_{n}$ son independientes e id\'enticamennte distribuidos.
\end{Def}

\begin{Note}
Un proceso regenerativo con media de la longitud de ciclo finita es llamado positivo recurrente.
\end{Note}

\begin{Teo}[Procesos Regenerativos]
Suponga que el proceso
\end{Teo}


\begin{Def}[Renewal Process Trinity]
Para un proceso de renovaci\'on $N\left(t\right)$, los siguientes procesos proveen de informaci\'on sobre los tiempos de renovaci\'on.
\begin{itemize}
\item $A\left(t\right)=t-T_{N\left(t\right)}$, el tiempo de recurrencia hacia atr\'as al tiempo $t$, que es el tiempo desde la \'ultima renovaci\'on para $t$.

\item $B\left(t\right)=T_{N\left(t\right)+1}-t$, el tiempo de recurrencia hacia adelante al tiempo $t$, residual del tiempo de renovaci\'on, que es el tiempo para la pr\'oxima renovaci\'on despu\'es de $t$.

\item $L\left(t\right)=\xi_{N\left(t\right)+1}=A\left(t\right)+B\left(t\right)$, la longitud del intervalo de renovaci\'on que contiene a $t$.
\end{itemize}
\end{Def}

\begin{Note}
El proceso tridimensional $\left(A\left(t\right),B\left(t\right),L\left(t\right)\right)$ es regenerativo sobre $T_{n}$, y por ende cada proceso lo es. Cada proceso $A\left(t\right)$ y $B\left(t\right)$ son procesos de MArkov a tiempo continuo con trayectorias continuas por partes en el espacio de estados $\rea_{+}$. Una expresi\'on conveniente para su distribuci\'on conjunta es, para $0\leq x<t,y\geq0$
\begin{equation}\label{NoRenovacion}
P\left\{A\left(t\right)>x,B\left(t\right)>y\right\}=
P\left\{N\left(t+y\right)-N\left((t-x)\right)=0\right\}
\end{equation}
\end{Note}

\begin{Ejem}[Tiempos de recurrencia Poisson]
Si $N\left(t\right)$ es un proceso Poisson con tasa $\lambda$, entonces de la expresi\'on (\ref{NoRenovacion}) se tiene que

\begin{eqnarray*}
\begin{array}{lc}
P\left\{A\left(t\right)>x,B\left(t\right)>y\right\}=e^{-\lambda\left(x+y\right)},&0\leq x<t,y\geq0,
\end{array}
\end{eqnarray*}
que es la probabilidad Poisson de no renovaciones en un intervalo de longitud $x+y$.

\end{Ejem}

\begin{Note}
Una cadena de Markov erg\'odica tiene la propiedad de ser estacionaria si la distribuci\'on de su estado al tiempo $0$ es su distribuci\'on estacionaria.
\end{Note}


\begin{Def}
Un proceso estoc\'astico a tiempo continuo $\left\{X\left(t\right):t\geq0\right\}$ en un espacio general es estacionario si sus distribuciones finito dimensionales son invariantes bajo cualquier  traslado: para cada $0\leq s_{1}<s_{2}<\cdots<s_{k}$ y $t\geq0$,
\begin{eqnarray*}
\left(X\left(s_{1}+t\right),\ldots,X\left(s_{k}+t\right)\right)=_{d}\left(X\left(s_{1}\right),\ldots,X\left(s_{k}\right)\right).
\end{eqnarray*}
\end{Def}

\begin{Note}
Un proceso de Markov es estacionario si $X\left(t\right)=_{d}X\left(0\right)$, $t\geq0$.
\end{Note}

Considerese el proceso $N\left(t\right)=\sum_{n}\indora\left(\tau_{n}\leq t\right)$ en $\rea_{+}$, con puntos $0<\tau_{1}<\tau_{2}<\cdots$.

\begin{Prop}
Si $N$ es un proceso puntual estacionario y $\esp\left[N\left(1\right)\right]<\infty$, entonces $\esp\left[N\left(t\right)\right]=t\esp\left[N\left(1\right)\right]$, $t\geq0$

\end{Prop}

\begin{Teo}
Los siguientes enunciados son equivalentes
\begin{itemize}
\item[i)] El proceso retardado de renovaci\'on $N$ es estacionario.

\item[ii)] EL proceso de tiempos de recurrencia hacia adelante $B\left(t\right)$ es estacionario.


\item[iii)] $\esp\left[N\left(t\right)\right]=t/\mu$,


\item[iv)] $G\left(t\right)=F_{e}\left(t\right)=\frac{1}{\mu}\int_{0}^{t}\left[1-F\left(s\right)\right]ds$
\end{itemize}
Cuando estos enunciados son ciertos, $P\left\{B\left(t\right)\leq x\right\}=F_{e}\left(x\right)$, para $t,x\geq0$.

\end{Teo}

\begin{Note}
Una consecuencia del teorema anterior es que el Proceso Poisson es el \'unico proceso sin retardo que es estacionario.
\end{Note}

\begin{Coro}
El proceso de renovaci\'on $N\left(t\right)$ sin retardo, y cuyos tiempos de inter renonaci\'on tienen media finita, es estacionario si y s\'olo si es un proceso Poisson.

\end{Coro}

%______________________________________________________________________

%\section{Ejemplos, Notas importantes}
%______________________________________________________________________
%\section*{Ap\'endice A}
%__________________________________________________________________

%________________________________________________________________________
%\subsection*{Procesos Regenerativos}
%________________________________________________________________________



\begin{Note}
Si $\tilde{X}\left(t\right)$ con espacio de estados $\tilde{S}$ es regenerativo sobre $T_{n}$, entonces $X\left(t\right)=f\left(\tilde{X}\left(t\right)\right)$ tambi\'en es regenerativo sobre $T_{n}$, para cualquier funci\'on $f:\tilde{S}\rightarrow S$.
\end{Note}

\begin{Note}
Los procesos regenerativos son crudamente regenerativos, pero no al rev\'es.
\end{Note}
%\subsection*{Procesos Regenerativos: Sigman\cite{Sigman1}}
\begin{Def}[Definici\'on Cl\'asica]
Un proceso estoc\'astico $X=\left\{X\left(t\right):t\geq0\right\}$ es llamado regenerativo is existe una variable aleatoria $R_{1}>0$ tal que
\begin{itemize}
\item[i)] $\left\{X\left(t+R_{1}\right):t\geq0\right\}$ es independiente de $\left\{\left\{X\left(t\right):t<R_{1}\right\},\right\}$
\item[ii)] $\left\{X\left(t+R_{1}\right):t\geq0\right\}$ es estoc\'asticamente equivalente a $\left\{X\left(t\right):t>0\right\}$
\end{itemize}

Llamamos a $R_{1}$ tiempo de regeneraci\'on, y decimos que $X$ se regenera en este punto.
\end{Def}

$\left\{X\left(t+R_{1}\right)\right\}$ es regenerativo con tiempo de regeneraci\'on $R_{2}$, independiente de $R_{1}$ pero con la misma distribuci\'on que $R_{1}$. Procediendo de esta manera se obtiene una secuencia de variables aleatorias independientes e id\'enticamente distribuidas $\left\{R_{n}\right\}$ llamados longitudes de ciclo. Si definimos a $Z_{k}\equiv R_{1}+R_{2}+\cdots+R_{k}$, se tiene un proceso de renovaci\'on llamado proceso de renovaci\'on encajado para $X$.




\begin{Def}
Para $x$ fijo y para cada $t\geq0$, sea $I_{x}\left(t\right)=1$ si $X\left(t\right)\leq x$,  $I_{x}\left(t\right)=0$ en caso contrario, y def\'inanse los tiempos promedio
\begin{eqnarray*}
\overline{X}&=&lim_{t\rightarrow\infty}\frac{1}{t}\int_{0}^{\infty}X\left(u\right)du\\
\prob\left(X_{\infty}\leq x\right)&=&lim_{t\rightarrow\infty}\frac{1}{t}\int_{0}^{\infty}I_{x}\left(u\right)du,
\end{eqnarray*}
cuando estos l\'imites existan.
\end{Def}

Como consecuencia del teorema de Renovaci\'on-Recompensa, se tiene que el primer l\'imite  existe y es igual a la constante
\begin{eqnarray*}
\overline{X}&=&\frac{\esp\left[\int_{0}^{R_{1}}X\left(t\right)dt\right]}{\esp\left[R_{1}\right]},
\end{eqnarray*}
suponiendo que ambas esperanzas son finitas.

\begin{Note}
\begin{itemize}
\item[a)] Si el proceso regenerativo $X$ es positivo recurrente y tiene trayectorias muestrales no negativas, entonces la ecuaci\'on anterior es v\'alida.
\item[b)] Si $X$ es positivo recurrente regenerativo, podemos construir una \'unica versi\'on estacionaria de este proceso, $X_{e}=\left\{X_{e}\left(t\right)\right\}$, donde $X_{e}$ es un proceso estoc\'astico regenerativo y estrictamente estacionario, con distribuci\'on marginal distribuida como $X_{\infty}$
\end{itemize}
\end{Note}

Para $\left\{X\left(t\right):t\geq0\right\}$ Proceso Estoc\'astico a tiempo continuo con estado de espacios $S$, que es un espacio m\'etrico, con trayectorias continuas por la derecha y con l\'imites por la izquierda c.s. Sea $N\left(t\right)$ un proceso de renovaci\'on en $\rea_{+}$ definido en el mismo espacio de probabilidad que $X\left(t\right)$, con tiempos de renovaci\'on $T$ y tiempos de inter-renovaci\'on $\xi_{n}=T_{n}-T_{n-1}$, con misma distribuci\'on $F$ de media finita $\mu$.


\begin{Def}
Para el proceso $\left\{\left(N\left(t\right),X\left(t\right)\right):t\geq0\right\}$, sus trayectoria muestrales en el intervalo de tiempo $\left[T_{n-1},T_{n}\right)$ est\'an descritas por
\begin{eqnarray*}
\zeta_{n}=\left(\xi_{n},\left\{X\left(T_{n-1}+t\right):0\leq t<\xi_{n}\right\}\right)
\end{eqnarray*}
Este $\zeta_{n}$ es el $n$-\'esimo segmento del proceso. El proceso es regenerativo sobre los tiempos $T_{n}$ si sus segmentos $\zeta_{n}$ son independientes e id\'enticamennte distribuidos.
\end{Def}


\begin{Note}
Si $\tilde{X}\left(t\right)$ con espacio de estados $\tilde{S}$ es regenerativo sobre $T_{n}$, entonces $X\left(t\right)=f\left(\tilde{X}\left(t\right)\right)$ tambi\'en es regenerativo sobre $T_{n}$, para cualquier funci\'on $f:\tilde{S}\rightarrow S$.
\end{Note}

\begin{Note}
Los procesos regenerativos son crudamente regenerativos, pero no al rev\'es.
\end{Note}

\begin{Def}[Definici\'on Cl\'asica]
Un proceso estoc\'astico $X=\left\{X\left(t\right):t\geq0\right\}$ es llamado regenerativo is existe una variable aleatoria $R_{1}>0$ tal que
\begin{itemize}
\item[i)] $\left\{X\left(t+R_{1}\right):t\geq0\right\}$ es independiente de $\left\{\left\{X\left(t\right):t<R_{1}\right\},\right\}$
\item[ii)] $\left\{X\left(t+R_{1}\right):t\geq0\right\}$ es estoc\'asticamente equivalente a $\left\{X\left(t\right):t>0\right\}$
\end{itemize}

Llamamos a $R_{1}$ tiempo de regeneraci\'on, y decimos que $X$ se regenera en este punto.
\end{Def}

$\left\{X\left(t+R_{1}\right)\right\}$ es regenerativo con tiempo de regeneraci\'on $R_{2}$, independiente de $R_{1}$ pero con la misma distribuci\'on que $R_{1}$. Procediendo de esta manera se obtiene una secuencia de variables aleatorias independientes e id\'enticamente distribuidas $\left\{R_{n}\right\}$ llamados longitudes de ciclo. Si definimos a $Z_{k}\equiv R_{1}+R_{2}+\cdots+R_{k}$, se tiene un proceso de renovaci\'on llamado proceso de renovaci\'on encajado para $X$.

\begin{Note}
Un proceso regenerativo con media de la longitud de ciclo finita es llamado positivo recurrente.
\end{Note}


\begin{Def}
Para $x$ fijo y para cada $t\geq0$, sea $I_{x}\left(t\right)=1$ si $X\left(t\right)\leq x$,  $I_{x}\left(t\right)=0$ en caso contrario, y def\'inanse los tiempos promedio
\begin{eqnarray*}
\overline{X}&=&lim_{t\rightarrow\infty}\frac{1}{t}\int_{0}^{\infty}X\left(u\right)du\\
\prob\left(X_{\infty}\leq x\right)&=&lim_{t\rightarrow\infty}\frac{1}{t}\int_{0}^{\infty}I_{x}\left(u\right)du,
\end{eqnarray*}
cuando estos l\'imites existan.
\end{Def}

Como consecuencia del teorema de Renovaci\'on-Recompensa, se tiene que el primer l\'imite  existe y es igual a la constante
\begin{eqnarray*}
\overline{X}&=&\frac{\esp\left[\int_{0}^{R_{1}}X\left(t\right)dt\right]}{\esp\left[R_{1}\right]},
\end{eqnarray*}
suponiendo que ambas esperanzas son finitas.

\begin{Note}
\begin{itemize}
\item[a)] Si el proceso regenerativo $X$ es positivo recurrente y tiene trayectorias muestrales no negativas, entonces la ecuaci\'on anterior es v\'alida.
\item[b)] Si $X$ es positivo recurrente regenerativo, podemos construir una \'unica versi\'on estacionaria de este proceso, $X_{e}=\left\{X_{e}\left(t\right)\right\}$, donde $X_{e}$ es un proceso estoc\'astico regenerativo y estrictamente estacionario, con distribuci\'on marginal distribuida como $X_{\infty}$
\end{itemize}
\end{Note}

%__________________________________________________________________________________________
%\subsection{Procesos Regenerativos Estacionarios - Stidham \cite{Stidham}}
%__________________________________________________________________________________________


Un proceso estoc\'astico a tiempo continuo $\left\{V\left(t\right),t\geq0\right\}$ es un proceso regenerativo si existe una sucesi\'on de variables aleatorias independientes e id\'enticamente distribuidas $\left\{X_{1},X_{2},\ldots\right\}$, sucesi\'on de renovaci\'on, tal que para cualquier conjunto de Borel $A$, 

\begin{eqnarray*}
\prob\left\{V\left(t\right)\in A|X_{1}+X_{2}+\cdots+X_{R\left(t\right)}=s,\left\{V\left(\tau\right),\tau<s\right\}\right\}=\prob\left\{V\left(t-s\right)\in A|X_{1}>t-s\right\},
\end{eqnarray*}
para todo $0\leq s\leq t$, donde $R\left(t\right)=\max\left\{X_{1}+X_{2}+\cdots+X_{j}\leq t\right\}=$n\'umero de renovaciones ({\emph{puntos de regeneraci\'on}}) que ocurren en $\left[0,t\right]$. El intervalo $\left[0,X_{1}\right)$ es llamado {\emph{primer ciclo de regeneraci\'on}} de $\left\{V\left(t \right),t\geq0\right\}$, $\left[X_{1},X_{1}+X_{2}\right)$ el {\emph{segundo ciclo de regeneraci\'on}}, y as\'i sucesivamente.

Sea $X=X_{1}$ y sea $F$ la funci\'on de distrbuci\'on de $X$


\begin{Def}
Se define el proceso estacionario, $\left\{V^{*}\left(t\right),t\geq0\right\}$, para $\left\{V\left(t\right),t\geq0\right\}$ por

\begin{eqnarray*}
\prob\left\{V\left(t\right)\in A\right\}=\frac{1}{\esp\left[X\right]}\int_{0}^{\infty}\prob\left\{V\left(t+x\right)\in A|X>x\right\}\left(1-F\left(x\right)\right)dx,
\end{eqnarray*} 
para todo $t\geq0$ y todo conjunto de Borel $A$.
\end{Def}

\begin{Def}
Una distribuci\'on se dice que es {\emph{aritm\'etica}} si todos sus puntos de incremento son m\'ultiplos de la forma $0,\lambda, 2\lambda,\ldots$ para alguna $\lambda>0$ entera.
\end{Def}


\begin{Def}
Una modificaci\'on medible de un proceso $\left\{V\left(t\right),t\geq0\right\}$, es una versi\'on de este, $\left\{V\left(t,w\right)\right\}$ conjuntamente medible para $t\geq0$ y para $w\in S$, $S$ espacio de estados para $\left\{V\left(t\right),t\geq0\right\}$.
\end{Def}

\begin{Teo}
Sea $\left\{V\left(t\right),t\geq\right\}$ un proceso regenerativo no negativo con modificaci\'on medible. Sea $\esp\left[X\right]<\infty$. Entonces el proceso estacionario dado por la ecuaci\'on anterior est\'a bien definido y tiene funci\'on de distribuci\'on independiente de $t$, adem\'as
\begin{itemize}
\item[i)] \begin{eqnarray*}
\esp\left[V^{*}\left(0\right)\right]&=&\frac{\esp\left[\int_{0}^{X}V\left(s\right)ds\right]}{\esp\left[X\right]}\end{eqnarray*}
\item[ii)] Si $\esp\left[V^{*}\left(0\right)\right]<\infty$, equivalentemente, si $\esp\left[\int_{0}^{X}V\left(s\right)ds\right]<\infty$,entonces
\begin{eqnarray*}
\frac{\int_{0}^{t}V\left(s\right)ds}{t}\rightarrow\frac{\esp\left[\int_{0}^{X}V\left(s\right)ds\right]}{\esp\left[X\right]}
\end{eqnarray*}
con probabilidad 1 y en media, cuando $t\rightarrow\infty$.
\end{itemize}
\end{Teo}



%_______________________________________________________________________________________
\subsection*{Procesos Regenerativos: Sigman\cite{Sigman1}}
\begin{Def}[Definici\'on Cl\'asica]
Un proceso estoc\'astico $X=\left\{X\left(t\right):t\geq0\right\}$ es llamado regenerativo is existe una variable aleatoria $R_{1}>0$ tal que
\begin{itemize}
\item[i)] $\left\{X\left(t+R_{1}\right):t\geq0\right\}$ es independiente de $\left\{\left\{X\left(t\right):t<R_{1}\right\},\right\}$
\item[ii)] $\left\{X\left(t+R_{1}\right):t\geq0\right\}$ es estoc\'asticamente equivalente a $\left\{X\left(t\right):t>0\right\}$
\end{itemize}

Llamamos a $R_{1}$ tiempo de regeneraci\'on, y decimos que $X$ se regenera en este punto.
\end{Def}

$\left\{X\left(t+R_{1}\right)\right\}$ es regenerativo con tiempo de regeneraci\'on $R_{2}$, independiente de $R_{1}$ pero con la misma distribuci\'on que $R_{1}$. Procediendo de esta manera se obtiene una secuencia de variables aleatorias independientes e id\'enticamente distribuidas $\left\{R_{n}\right\}$ llamados longitudes de ciclo. Si definimos a $Z_{k}\equiv R_{1}+R_{2}+\cdots+R_{k}$, se tiene un proceso de renovaci\'on llamado proceso de renovaci\'on encajado para $X$.




\begin{Def}
Para $x$ fijo y para cada $t\geq0$, sea $I_{x}\left(t\right)=1$ si $X\left(t\right)\leq x$,  $I_{x}\left(t\right)=0$ en caso contrario, y def\'inanse los tiempos promedio
\begin{eqnarray*}
\overline{X}&=&lim_{t\rightarrow\infty}\frac{1}{t}\int_{0}^{\infty}X\left(u\right)du\\
\prob\left(X_{\infty}\leq x\right)&=&lim_{t\rightarrow\infty}\frac{1}{t}\int_{0}^{\infty}I_{x}\left(u\right)du,
\end{eqnarray*}
cuando estos l\'imites existan.
\end{Def}

Como consecuencia del teorema de Renovaci\'on-Recompensa, se tiene que el primer l\'imite  existe y es igual a la constante
\begin{eqnarray*}
\overline{X}&=&\frac{\esp\left[\int_{0}^{R_{1}}X\left(t\right)dt\right]}{\esp\left[R_{1}\right]},
\end{eqnarray*}
suponiendo que ambas esperanzas son finitas.

\begin{Note}
\begin{itemize}
\item[a)] Si el proceso regenerativo $X$ es positivo recurrente y tiene trayectorias muestrales no negativas, entonces la ecuaci\'on anterior es v\'alida.
\item[b)] Si $X$ es positivo recurrente regenerativo, podemos construir una \'unica versi\'on estacionaria de este proceso, $X_{e}=\left\{X_{e}\left(t\right)\right\}$, donde $X_{e}$ es un proceso estoc\'astico regenerativo y estrictamente estacionario, con distribuci\'on marginal distribuida como $X_{\infty}$
\end{itemize}
\end{Note}


%______________________________________________________________________
\subsection{Procesos de Renovaci\'on}
%______________________________________________________________________

\begin{Def}\label{Def.Tn}
Sean $0\leq T_{1}\leq T_{2}\leq \ldots$ son tiempos aleatorios infinitos en los cuales ocurren ciertos eventos. El n\'umero de tiempos $T_{n}$ en el intervalo $\left[0,t\right)$ es

\begin{eqnarray}
N\left(t\right)=\sum_{n=1}^{\infty}\indora\left(T_{n}\leq t\right),
\end{eqnarray}
para $t\geq0$.
\end{Def}

Si se consideran los puntos $T_{n}$ como elementos de $\rea_{+}$, y $N\left(t\right)$ es el n\'umero de puntos en $\rea$. El proceso denotado por $\left\{N\left(t\right):t\geq0\right\}$, denotado por $N\left(t\right)$, es un proceso puntual en $\rea_{+}$. Los $T_{n}$ son los tiempos de ocurrencia, el proceso puntual $N\left(t\right)$ es simple si su n\'umero de ocurrencias son distintas: $0<T_{1}<T_{2}<\ldots$ casi seguramente.

\begin{Def}
Un proceso puntual $N\left(t\right)$ es un proceso de renovaci\'on si los tiempos de interocurrencia $\xi_{n}=T_{n}-T_{n-1}$, para $n\geq1$, son independientes e identicamente distribuidos con distribuci\'on $F$, donde $F\left(0\right)=0$ y $T_{0}=0$. Los $T_{n}$ son llamados tiempos de renovaci\'on, referente a la independencia o renovaci\'on de la informaci\'on estoc\'astica en estos tiempos. Los $\xi_{n}$ son los tiempos de inter-renovaci\'on, y $N\left(t\right)$ es el n\'umero de renovaciones en el intervalo $\left[0,t\right)$
\end{Def}


\begin{Note}
Para definir un proceso de renovaci\'on para cualquier contexto, solamente hay que especificar una distribuci\'on $F$, con $F\left(0\right)=0$, para los tiempos de inter-renovaci\'on. La funci\'on $F$ en turno degune las otra variables aleatorias. De manera formal, existe un espacio de probabilidad y una sucesi\'on de variables aleatorias $\xi_{1},\xi_{2},\ldots$ definidas en este con distribuci\'on $F$. Entonces las otras cantidades son $T_{n}=\sum_{k=1}^{n}\xi_{k}$ y $N\left(t\right)=\sum_{n=1}^{\infty}\indora\left(T_{n}\leq t\right)$, donde $T_{n}\rightarrow\infty$ casi seguramente por la Ley Fuerte de los Grandes Números.
\end{Note}



%______________________________________________________________________
\subsection{Procesos de Renovaci\'on}
%______________________________________________________________________

\begin{Def}\label{Def.Tn}
Sean $0\leq T_{1}\leq T_{2}\leq \ldots$ son tiempos aleatorios infinitos en los cuales ocurren ciertos eventos. El n\'umero de tiempos $T_{n}$ en el intervalo $\left[0,t\right)$ es

\begin{eqnarray}
N\left(t\right)=\sum_{n=1}^{\infty}\indora\left(T_{n}\leq t\right),
\end{eqnarray}
para $t\geq0$.
\end{Def}

Si se consideran los puntos $T_{n}$ como elementos de $\rea_{+}$, y $N\left(t\right)$ es el n\'umero de puntos en $\rea$. El proceso denotado por $\left\{N\left(t\right):t\geq0\right\}$, denotado por $N\left(t\right)$, es un proceso puntual en $\rea_{+}$. Los $T_{n}$ son los tiempos de ocurrencia, el proceso puntual $N\left(t\right)$ es simple si su n\'umero de ocurrencias son distintas: $0<T_{1}<T_{2}<\ldots$ casi seguramente.

\begin{Def}
Un proceso puntual $N\left(t\right)$ es un proceso de renovaci\'on si los tiempos de interocurrencia $\xi_{n}=T_{n}-T_{n-1}$, para $n\geq1$, son independientes e identicamente distribuidos con distribuci\'on $F$, donde $F\left(0\right)=0$ y $T_{0}=0$. Los $T_{n}$ son llamados tiempos de renovaci\'on, referente a la independencia o renovaci\'on de la informaci\'on estoc\'astica en estos tiempos. Los $\xi_{n}$ son los tiempos de inter-renovaci\'on, y $N\left(t\right)$ es el n\'umero de renovaciones en el intervalo $\left[0,t\right)$
\end{Def}


\begin{Note}
Para definir un proceso de renovaci\'on para cualquier contexto, solamente hay que especificar una distribuci\'on $F$, con $F\left(0\right)=0$, para los tiempos de inter-renovaci\'on. La funci\'on $F$ en turno degune las otra variables aleatorias. De manera formal, existe un espacio de probabilidad y una sucesi\'on de variables aleatorias $\xi_{1},\xi_{2},\ldots$ definidas en este con distribuci\'on $F$. Entonces las otras cantidades son $T_{n}=\sum_{k=1}^{n}\xi_{k}$ y $N\left(t\right)=\sum_{n=1}^{\infty}\indora\left(T_{n}\leq t\right)$, donde $T_{n}\rightarrow\infty$ casi seguramente por la Ley Fuerte de los Grandes Números.
\end{Note}

%______________________________________________________________________
\subsection{Procesos de Renovaci\'on}
%______________________________________________________________________

\begin{Def}%\label{Def.Tn}
Sean $0\leq T_{1}\leq T_{2}\leq \ldots$ son tiempos aleatorios infinitos en los cuales ocurren ciertos eventos. El n\'umero de tiempos $T_{n}$ en el intervalo $\left[0,t\right)$ es

\begin{eqnarray}
N\left(t\right)=\sum_{n=1}^{\infty}\indora\left(T_{n}\leq t\right),
\end{eqnarray}
para $t\geq0$.
\end{Def}

Si se consideran los puntos $T_{n}$ como elementos de $\rea_{+}$, y $N\left(t\right)$ es el n\'umero de puntos en $\rea$. El proceso denotado por $\left\{N\left(t\right):t\geq0\right\}$, denotado por $N\left(t\right)$, es un proceso puntual en $\rea_{+}$. Los $T_{n}$ son los tiempos de ocurrencia, el proceso puntual $N\left(t\right)$ es simple si su n\'umero de ocurrencias son distintas: $0<T_{1}<T_{2}<\ldots$ casi seguramente.

\begin{Def}
Un proceso puntual $N\left(t\right)$ es un proceso de renovaci\'on si los tiempos de interocurrencia $\xi_{n}=T_{n}-T_{n-1}$, para $n\geq1$, son independientes e identicamente distribuidos con distribuci\'on $F$, donde $F\left(0\right)=0$ y $T_{0}=0$. Los $T_{n}$ son llamados tiempos de renovaci\'on, referente a la independencia o renovaci\'on de la informaci\'on estoc\'astica en estos tiempos. Los $\xi_{n}$ son los tiempos de inter-renovaci\'on, y $N\left(t\right)$ es el n\'umero de renovaciones en el intervalo $\left[0,t\right)$
\end{Def}


\begin{Note}
Para definir un proceso de renovaci\'on para cualquier contexto, solamente hay que especificar una distribuci\'on $F$, con $F\left(0\right)=0$, para los tiempos de inter-renovaci\'on. La funci\'on $F$ en turno degune las otra variables aleatorias. De manera formal, existe un espacio de probabilidad y una sucesi\'on de variables aleatorias $\xi_{1},\xi_{2},\ldots$ definidas en este con distribuci\'on $F$. Entonces las otras cantidades son $T_{n}=\sum_{k=1}^{n}\xi_{k}$ y $N\left(t\right)=\sum_{n=1}^{\infty}\indora\left(T_{n}\leq t\right)$, donde $T_{n}\rightarrow\infty$ casi seguramente por la Ley Fuerte de los Grandes Números.
\end{Note}
%______________________________________________________________________
\subsection{Procesos de Renovaci\'on}
%______________________________________________________________________

\begin{Def}\label{Def.Tn}
Sean $0\leq T_{1}\leq T_{2}\leq \ldots$ son tiempos aleatorios infinitos en los cuales ocurren ciertos eventos. El n\'umero de tiempos $T_{n}$ en el intervalo $\left[0,t\right)$ es

\begin{eqnarray}
N\left(t\right)=\sum_{n=1}^{\infty}\indora\left(T_{n}\leq t\right),
\end{eqnarray}
para $t\geq0$.
\end{Def}

Si se consideran los puntos $T_{n}$ como elementos de $\rea_{+}$, y $N\left(t\right)$ es el n\'umero de puntos en $\rea$. El proceso denotado por $\left\{N\left(t\right):t\geq0\right\}$, denotado por $N\left(t\right)$, es un proceso puntual en $\rea_{+}$. Los $T_{n}$ son los tiempos de ocurrencia, el proceso puntual $N\left(t\right)$ es simple si su n\'umero de ocurrencias son distintas: $0<T_{1}<T_{2}<\ldots$ casi seguramente.

\begin{Def}
Un proceso puntual $N\left(t\right)$ es un proceso de renovaci\'on si los tiempos de interocurrencia $\xi_{n}=T_{n}-T_{n-1}$, para $n\geq1$, son independientes e identicamente distribuidos con distribuci\'on $F$, donde $F\left(0\right)=0$ y $T_{0}=0$. Los $T_{n}$ son llamados tiempos de renovaci\'on, referente a la independencia o renovaci\'on de la informaci\'on estoc\'astica en estos tiempos. Los $\xi_{n}$ son los tiempos de inter-renovaci\'on, y $N\left(t\right)$ es el n\'umero de renovaciones en el intervalo $\left[0,t\right)$
\end{Def}


\begin{Note}
Para definir un proceso de renovaci\'on para cualquier contexto, solamente hay que especificar una distribuci\'on $F$, con $F\left(0\right)=0$, para los tiempos de inter-renovaci\'on. La funci\'on $F$ en turno degune las otra variables aleatorias. De manera formal, existe un espacio de probabilidad y una sucesi\'on de variables aleatorias $\xi_{1},\xi_{2},\ldots$ definidas en este con distribuci\'on $F$. Entonces las otras cantidades son $T_{n}=\sum_{k=1}^{n}\xi_{k}$ y $N\left(t\right)=\sum_{n=1}^{\infty}\indora\left(T_{n}\leq t\right)$, donde $T_{n}\rightarrow\infty$ casi seguramente por la Ley Fuerte de los Grandes Números.
\end{Note}

%______________________________________________________________________
\subsection{Procesos de Renovaci\'on}
%______________________________________________________________________

\begin{Def}%\label{Def.Tn}
Sean $0\leq T_{1}\leq T_{2}\leq \ldots$ son tiempos aleatorios infinitos en los cuales ocurren ciertos eventos. El n\'umero de tiempos $T_{n}$ en el intervalo $\left[0,t\right)$ es

\begin{eqnarray}
N\left(t\right)=\sum_{n=1}^{\infty}\indora\left(T_{n}\leq t\right),
\end{eqnarray}
para $t\geq0$.
\end{Def}

Si se consideran los puntos $T_{n}$ como elementos de $\rea_{+}$, y $N\left(t\right)$ es el n\'umero de puntos en $\rea$. El proceso denotado por $\left\{N\left(t\right):t\geq0\right\}$, denotado por $N\left(t\right)$, es un proceso puntual en $\rea_{+}$. Los $T_{n}$ son los tiempos de ocurrencia, el proceso puntual $N\left(t\right)$ es simple si su n\'umero de ocurrencias son distintas: $0<T_{1}<T_{2}<\ldots$ casi seguramente.

\begin{Def}
Un proceso puntual $N\left(t\right)$ es un proceso de renovaci\'on si los tiempos de interocurrencia $\xi_{n}=T_{n}-T_{n-1}$, para $n\geq1$, son independientes e identicamente distribuidos con distribuci\'on $F$, donde $F\left(0\right)=0$ y $T_{0}=0$. Los $T_{n}$ son llamados tiempos de renovaci\'on, referente a la independencia o renovaci\'on de la informaci\'on estoc\'astica en estos tiempos. Los $\xi_{n}$ son los tiempos de inter-renovaci\'on, y $N\left(t\right)$ es el n\'umero de renovaciones en el intervalo $\left[0,t\right)$
\end{Def}


\begin{Note}
Para definir un proceso de renovaci\'on para cualquier contexto, solamente hay que especificar una distribuci\'on $F$, con $F\left(0\right)=0$, para los tiempos de inter-renovaci\'on. La funci\'on $F$ en turno degune las otra variables aleatorias. De manera formal, existe un espacio de probabilidad y una sucesi\'on de variables aleatorias $\xi_{1},\xi_{2},\ldots$ definidas en este con distribuci\'on $F$. Entonces las otras cantidades son $T_{n}=\sum_{k=1}^{n}\xi_{k}$ y $N\left(t\right)=\sum_{n=1}^{\infty}\indora\left(T_{n}\leq t\right)$, donde $T_{n}\rightarrow\infty$ casi seguramente por la Ley Fuerte de los Grandes Números.
\end{Note}
%______________________________________________________________________
\subsection{Procesos Regenerativos Estacionarios: Visi\'on cl\'asica}
%______________________________________________________________________

\begin{Def}\label{Def.Tn}
Sean $0\leq T_{1}\leq T_{2}\leq \ldots$ son tiempos aleatorios infinitos en los cuales ocurren ciertos eventos. El n\'umero de tiempos $T_{n}$ en el intervalo $\left[0,t\right)$ es

\begin{eqnarray}
N\left(t\right)=\sum_{n=1}^{\infty}\indora\left(T_{n}\leq t\right),
\end{eqnarray}
para $t\geq0$.
\end{Def}

Si se consideran los puntos $T_{n}$ como elementos de $\rea_{+}$, y $N\left(t\right)$ es el n\'umero de puntos en $\rea$. El proceso denotado por $\left\{N\left(t\right):t\geq0\right\}$, denotado por $N\left(t\right)$, es un proceso puntual en $\rea_{+}$. Los $T_{n}$ son los tiempos de ocurrencia, el proceso puntual $N\left(t\right)$ es simple si su n\'umero de ocurrencias son distintas: $0<T_{1}<T_{2}<\ldots$ casi seguramente.

\begin{Def}
Un proceso puntual $N\left(t\right)$ es un proceso de renovaci\'on si los tiempos de interocurrencia $\xi_{n}=T_{n}-T_{n-1}$, para $n\geq1$, son independientes e identicamente distribuidos con distribuci\'on $F$, donde $F\left(0\right)=0$ y $T_{0}=0$. Los $T_{n}$ son llamados tiempos de renovaci\'on, referente a la independencia o renovaci\'on de la informaci\'on estoc\'astica en estos tiempos. Los $\xi_{n}$ son los tiempos de inter-renovaci\'on, y $N\left(t\right)$ es el n\'umero de renovaciones en el intervalo $\left[0,t\right)$
\end{Def}


\begin{Note}
Para definir un proceso de renovaci\'on para cualquier contexto, solamente hay que especificar una distribuci\'on $F$, con $F\left(0\right)=0$, para los tiempos de inter-renovaci\'on. La funci\'on $F$ en turno degune las otra variables aleatorias. De manera formal, existe un espacio de probabilidad y una sucesi\'on de variables aleatorias $\xi_{1},\xi_{2},\ldots$ definidas en este con distribuci\'on $F$. Entonces las otras cantidades son $T_{n}=\sum_{k=1}^{n}\xi_{k}$ y $N\left(t\right)=\sum_{n=1}^{\infty}\indora\left(T_{n}\leq t\right)$, donde $T_{n}\rightarrow\infty$ casi seguramente por la Ley Fuerte de los Grandes Números.
\end{Note}

%___________________________________________________________________________________________
%
\subsection{Teorema Principal de Renovaci\'on}
%___________________________________________________________________________________________
%

\begin{Note} Una funci\'on $h:\rea_{+}\rightarrow\rea$ es Directamente Riemann Integrable en los siguientes casos:
\begin{itemize}
\item[a)] $h\left(t\right)\geq0$ es decreciente y Riemann Integrable.
\item[b)] $h$ es continua excepto posiblemente en un conjunto de Lebesgue de medida 0, y $|h\left(t\right)|\leq b\left(t\right)$, donde $b$ es DRI.
\end{itemize}
\end{Note}

\begin{Teo}[Teorema Principal de Renovaci\'on]
Si $F$ es no aritm\'etica y $h\left(t\right)$ es Directamente Riemann Integrable (DRI), entonces

\begin{eqnarray*}
lim_{t\rightarrow\infty}U\star h=\frac{1}{\mu}\int_{\rea_{+}}h\left(s\right)ds.
\end{eqnarray*}
\end{Teo}

\begin{Prop}
Cualquier funci\'on $H\left(t\right)$ acotada en intervalos finitos y que es 0 para $t<0$ puede expresarse como
\begin{eqnarray*}
H\left(t\right)=U\star h\left(t\right)\textrm{,  donde }h\left(t\right)=H\left(t\right)-F\star H\left(t\right)
\end{eqnarray*}
\end{Prop}

\begin{Def}
Un proceso estoc\'astico $X\left(t\right)$ es crudamente regenerativo en un tiempo aleatorio positivo $T$ si
\begin{eqnarray*}
\esp\left[X\left(T+t\right)|T\right]=\esp\left[X\left(t\right)\right]\textrm{, para }t\geq0,\end{eqnarray*}
y con las esperanzas anteriores finitas.
\end{Def}

\begin{Prop}
Sup\'ongase que $X\left(t\right)$ es un proceso crudamente regenerativo en $T$, que tiene distribuci\'on $F$. Si $\esp\left[X\left(t\right)\right]$ es acotado en intervalos finitos, entonces
\begin{eqnarray*}
\esp\left[X\left(t\right)\right]=U\star h\left(t\right)\textrm{,  donde }h\left(t\right)=\esp\left[X\left(t\right)\indora\left(T>t\right)\right].
\end{eqnarray*}
\end{Prop}

\begin{Teo}[Regeneraci\'on Cruda]
Sup\'ongase que $X\left(t\right)$ es un proceso con valores positivo crudamente regenerativo en $T$, y def\'inase $M=\sup\left\{|X\left(t\right)|:t\leq T\right\}$. Si $T$ es no aritm\'etico y $M$ y $MT$ tienen media finita, entonces
\begin{eqnarray*}
lim_{t\rightarrow\infty}\esp\left[X\left(t\right)\right]=\frac{1}{\mu}\int_{\rea_{+}}h\left(s\right)ds,
\end{eqnarray*}
donde $h\left(t\right)=\esp\left[X\left(t\right)\indora\left(T>t\right)\right]$.
\end{Teo}

%___________________________________________________________________________________________
%
\subsection{Propiedades de los Procesos de Renovaci\'on}
%___________________________________________________________________________________________
%

Los tiempos $T_{n}$ est\'an relacionados con los conteos de $N\left(t\right)$ por

\begin{eqnarray*}
\left\{N\left(t\right)\geq n\right\}&=&\left\{T_{n}\leq t\right\}\\
T_{N\left(t\right)}\leq &t&<T_{N\left(t\right)+1},
\end{eqnarray*}

adem\'as $N\left(T_{n}\right)=n$, y 

\begin{eqnarray*}
N\left(t\right)=\max\left\{n:T_{n}\leq t\right\}=\min\left\{n:T_{n+1}>t\right\}
\end{eqnarray*}

Por propiedades de la convoluci\'on se sabe que

\begin{eqnarray*}
P\left\{T_{n}\leq t\right\}=F^{n\star}\left(t\right)
\end{eqnarray*}
que es la $n$-\'esima convoluci\'on de $F$. Entonces 

\begin{eqnarray*}
\left\{N\left(t\right)\geq n\right\}&=&\left\{T_{n}\leq t\right\}\\
P\left\{N\left(t\right)\leq n\right\}&=&1-F^{\left(n+1\right)\star}\left(t\right)
\end{eqnarray*}

Adem\'as usando el hecho de que $\esp\left[N\left(t\right)\right]=\sum_{n=1}^{\infty}P\left\{N\left(t\right)\geq n\right\}$
se tiene que

\begin{eqnarray*}
\esp\left[N\left(t\right)\right]=\sum_{n=1}^{\infty}F^{n\star}\left(t\right)
\end{eqnarray*}

\begin{Prop}
Para cada $t\geq0$, la funci\'on generadora de momentos $\esp\left[e^{\alpha N\left(t\right)}\right]$ existe para alguna $\alpha$ en una vecindad del 0, y de aqu\'i que $\esp\left[N\left(t\right)^{m}\right]<\infty$, para $m\geq1$.
\end{Prop}


\begin{Note}
Si el primer tiempo de renovaci\'on $\xi_{1}$ no tiene la misma distribuci\'on que el resto de las $\xi_{n}$, para $n\geq2$, a $N\left(t\right)$ se le llama Proceso de Renovaci\'on retardado, donde si $\xi$ tiene distribuci\'on $G$, entonces el tiempo $T_{n}$ de la $n$-\'esima renovaci\'on tiene distribuci\'on $G\star F^{\left(n-1\right)\star}\left(t\right)$
\end{Note}


\begin{Teo}
Para una constante $\mu\leq\infty$ ( o variable aleatoria), las siguientes expresiones son equivalentes:

\begin{eqnarray}
lim_{n\rightarrow\infty}n^{-1}T_{n}&=&\mu,\textrm{ c.s.}\\
lim_{t\rightarrow\infty}t^{-1}N\left(t\right)&=&1/\mu,\textrm{ c.s.}
\end{eqnarray}
\end{Teo}


Es decir, $T_{n}$ satisface la Ley Fuerte de los Grandes N\'umeros s\'i y s\'olo s\'i $N\left/t\right)$ la cumple.


\begin{Coro}[Ley Fuerte de los Grandes N\'umeros para Procesos de Renovaci\'on]
Si $N\left(t\right)$ es un proceso de renovaci\'on cuyos tiempos de inter-renovaci\'on tienen media $\mu\leq\infty$, entonces
\begin{eqnarray}
t^{-1}N\left(t\right)\rightarrow 1/\mu,\textrm{ c.s. cuando }t\rightarrow\infty.
\end{eqnarray}

\end{Coro}


Considerar el proceso estoc\'astico de valores reales $\left\{Z\left(t\right):t\geq0\right\}$ en el mismo espacio de probabilidad que $N\left(t\right)$

\begin{Def}
Para el proceso $\left\{Z\left(t\right):t\geq0\right\}$ se define la fluctuaci\'on m\'axima de $Z\left(t\right)$ en el intervalo $\left(T_{n-1},T_{n}\right]$:
\begin{eqnarray*}
M_{n}=\sup_{T_{n-1}<t\leq T_{n}}|Z\left(t\right)-Z\left(T_{n-1}\right)|
\end{eqnarray*}
\end{Def}

\begin{Teo}
Sup\'ongase que $n^{-1}T_{n}\rightarrow\mu$ c.s. cuando $n\rightarrow\infty$, donde $\mu\leq\infty$ es una constante o variable aleatoria. Sea $a$ una constante o variable aleatoria que puede ser infinita cuando $\mu$ es finita, y considere las expresiones l\'imite:
\begin{eqnarray}
lim_{n\rightarrow\infty}n^{-1}Z\left(T_{n}\right)&=&a,\textrm{ c.s.}\\
lim_{t\rightarrow\infty}t^{-1}Z\left(t\right)&=&a/\mu,\textrm{ c.s.}
\end{eqnarray}
La segunda expresi\'on implica la primera. Conversamente, la primera implica la segunda si el proceso $Z\left(t\right)$ es creciente, o si $lim_{n\rightarrow\infty}n^{-1}M_{n}=0$ c.s.
\end{Teo}

\begin{Coro}
Si $N\left(t\right)$ es un proceso de renovaci\'on, y $\left(Z\left(T_{n}\right)-Z\left(T_{n-1}\right),M_{n}\right)$, para $n\geq1$, son variables aleatorias independientes e id\'enticamente distribuidas con media finita, entonces,
\begin{eqnarray}
lim_{t\rightarrow\infty}t^{-1}Z\left(t\right)\rightarrow\frac{\esp\left[Z\left(T_{1}\right)-Z\left(T_{0}\right)\right]}{\esp\left[T_{1}\right]},\textrm{ c.s. cuando  }t\rightarrow\infty.
\end{eqnarray}
\end{Coro}

%___________________________________________________________________________________________
%
\subsection{Funci\'on de Renovaci\'on}
%___________________________________________________________________________________________
%


\begin{Def}
Sea $h\left(t\right)$ funci\'on de valores reales en $\rea$ acotada en intervalos finitos e igual a cero para $t<0$ La ecuaci\'on de renovaci\'on para $h\left(t\right)$ y la distribuci\'on $F$ es

\begin{eqnarray}\label{Ec.Renovacion}
H\left(t\right)=h\left(t\right)+\int_{\left[0,t\right]}H\left(t-s\right)dF\left(s\right)\textrm{,    }t\geq0,
\end{eqnarray}
donde $H\left(t\right)$ es una funci\'on de valores reales. Esto es $H=h+F\star H$. Decimos que $H\left(t\right)$ es soluci\'on de esta ecuaci\'on si satisface la ecuaci\'on, y es acotada en intervalos finitos e iguales a cero para $t<0$.
\end{Def}

\begin{Prop}
La funci\'on $U\star h\left(t\right)$ es la \'unica soluci\'on de la ecuaci\'on de renovaci\'on (\ref{Ec.Renovacion}).
\end{Prop}

\begin{Teo}[Teorema Renovaci\'on Elemental]
\begin{eqnarray*}
t^{-1}U\left(t\right)\rightarrow 1/\mu\textrm{,    cuando }t\rightarrow\infty.
\end{eqnarray*}
\end{Teo}

%______________________________________________________________________
\subsection{Procesos de Renovaci\'on}
%______________________________________________________________________

\begin{Def}\label{Def.Tn}
Sean $0\leq T_{1}\leq T_{2}\leq \ldots$ son tiempos aleatorios infinitos en los cuales ocurren ciertos eventos. El n\'umero de tiempos $T_{n}$ en el intervalo $\left[0,t\right)$ es

\begin{eqnarray}
N\left(t\right)=\sum_{n=1}^{\infty}\indora\left(T_{n}\leq t\right),
\end{eqnarray}
para $t\geq0$.
\end{Def}

Si se consideran los puntos $T_{n}$ como elementos de $\rea_{+}$, y $N\left(t\right)$ es el n\'umero de puntos en $\rea$. El proceso denotado por $\left\{N\left(t\right):t\geq0\right\}$, denotado por $N\left(t\right)$, es un proceso puntual en $\rea_{+}$. Los $T_{n}$ son los tiempos de ocurrencia, el proceso puntual $N\left(t\right)$ es simple si su n\'umero de ocurrencias son distintas: $0<T_{1}<T_{2}<\ldots$ casi seguramente.

\begin{Def}
Un proceso puntual $N\left(t\right)$ es un proceso de renovaci\'on si los tiempos de interocurrencia $\xi_{n}=T_{n}-T_{n-1}$, para $n\geq1$, son independientes e identicamente distribuidos con distribuci\'on $F$, donde $F\left(0\right)=0$ y $T_{0}=0$. Los $T_{n}$ son llamados tiempos de renovaci\'on, referente a la independencia o renovaci\'on de la informaci\'on estoc\'astica en estos tiempos. Los $\xi_{n}$ son los tiempos de inter-renovaci\'on, y $N\left(t\right)$ es el n\'umero de renovaciones en el intervalo $\left[0,t\right)$
\end{Def}


\begin{Note}
Para definir un proceso de renovaci\'on para cualquier contexto, solamente hay que especificar una distribuci\'on $F$, con $F\left(0\right)=0$, para los tiempos de inter-renovaci\'on. La funci\'on $F$ en turno degune las otra variables aleatorias. De manera formal, existe un espacio de probabilidad y una sucesi\'on de variables aleatorias $\xi_{1},\xi_{2},\ldots$ definidas en este con distribuci\'on $F$. Entonces las otras cantidades son $T_{n}=\sum_{k=1}^{n}\xi_{k}$ y $N\left(t\right)=\sum_{n=1}^{\infty}\indora\left(T_{n}\leq t\right)$, donde $T_{n}\rightarrow\infty$ casi seguramente por la Ley Fuerte de los Grandes Números.
\end{Note}

%___________________________________________________________________________________________
%
\subsection{Renewal and Regenerative Processes: Serfozo\cite{Serfozo}}
%___________________________________________________________________________________________
%
\begin{Def}\label{Def.Tn}
Sean $0\leq T_{1}\leq T_{2}\leq \ldots$ son tiempos aleatorios infinitos en los cuales ocurren ciertos eventos. El n\'umero de tiempos $T_{n}$ en el intervalo $\left[0,t\right)$ es

\begin{eqnarray}
N\left(t\right)=\sum_{n=1}^{\infty}\indora\left(T_{n}\leq t\right),
\end{eqnarray}
para $t\geq0$.
\end{Def}

Si se consideran los puntos $T_{n}$ como elementos de $\rea_{+}$, y $N\left(t\right)$ es el n\'umero de puntos en $\rea$. El proceso denotado por $\left\{N\left(t\right):t\geq0\right\}$, denotado por $N\left(t\right)$, es un proceso puntual en $\rea_{+}$. Los $T_{n}$ son los tiempos de ocurrencia, el proceso puntual $N\left(t\right)$ es simple si su n\'umero de ocurrencias son distintas: $0<T_{1}<T_{2}<\ldots$ casi seguramente.

\begin{Def}
Un proceso puntual $N\left(t\right)$ es un proceso de renovaci\'on si los tiempos de interocurrencia $\xi_{n}=T_{n}-T_{n-1}$, para $n\geq1$, son independientes e identicamente distribuidos con distribuci\'on $F$, donde $F\left(0\right)=0$ y $T_{0}=0$. Los $T_{n}$ son llamados tiempos de renovaci\'on, referente a la independencia o renovaci\'on de la informaci\'on estoc\'astica en estos tiempos. Los $\xi_{n}$ son los tiempos de inter-renovaci\'on, y $N\left(t\right)$ es el n\'umero de renovaciones en el intervalo $\left[0,t\right)$
\end{Def}


\begin{Note}
Para definir un proceso de renovaci\'on para cualquier contexto, solamente hay que especificar una distribuci\'on $F$, con $F\left(0\right)=0$, para los tiempos de inter-renovaci\'on. La funci\'on $F$ en turno degune las otra variables aleatorias. De manera formal, existe un espacio de probabilidad y una sucesi\'on de variables aleatorias $\xi_{1},\xi_{2},\ldots$ definidas en este con distribuci\'on $F$. Entonces las otras cantidades son $T_{n}=\sum_{k=1}^{n}\xi_{k}$ y $N\left(t\right)=\sum_{n=1}^{\infty}\indora\left(T_{n}\leq t\right)$, donde $T_{n}\rightarrow\infty$ casi seguramente por la Ley Fuerte de los Grandes N\'umeros.
\end{Note}







Los tiempos $T_{n}$ est\'an relacionados con los conteos de $N\left(t\right)$ por

\begin{eqnarray*}
\left\{N\left(t\right)\geq n\right\}&=&\left\{T_{n}\leq t\right\}\\
T_{N\left(t\right)}\leq &t&<T_{N\left(t\right)+1},
\end{eqnarray*}

adem\'as $N\left(T_{n}\right)=n$, y 

\begin{eqnarray*}
N\left(t\right)=\max\left\{n:T_{n}\leq t\right\}=\min\left\{n:T_{n+1}>t\right\}
\end{eqnarray*}

Por propiedades de la convoluci\'on se sabe que

\begin{eqnarray*}
P\left\{T_{n}\leq t\right\}=F^{n\star}\left(t\right)
\end{eqnarray*}
que es la $n$-\'esima convoluci\'on de $F$. Entonces 

\begin{eqnarray*}
\left\{N\left(t\right)\geq n\right\}&=&\left\{T_{n}\leq t\right\}\\
P\left\{N\left(t\right)\leq n\right\}&=&1-F^{\left(n+1\right)\star}\left(t\right)
\end{eqnarray*}

Adem\'as usando el hecho de que $\esp\left[N\left(t\right)\right]=\sum_{n=1}^{\infty}P\left\{N\left(t\right)\geq n\right\}$
se tiene que

\begin{eqnarray*}
\esp\left[N\left(t\right)\right]=\sum_{n=1}^{\infty}F^{n\star}\left(t\right)
\end{eqnarray*}

\begin{Prop}
Para cada $t\geq0$, la funci\'on generadora de momentos $\esp\left[e^{\alpha N\left(t\right)}\right]$ existe para alguna $\alpha$ en una vecindad del 0, y de aqu\'i que $\esp\left[N\left(t\right)^{m}\right]<\infty$, para $m\geq1$.
\end{Prop}

\begin{Ejem}[\textbf{Proceso Poisson}]

Suponga que se tienen tiempos de inter-renovaci\'on \textit{i.i.d.} del proceso de renovaci\'on $N\left(t\right)$ tienen distribuci\'on exponencial $F\left(t\right)=q-e^{-\lambda t}$ con tasa $\lambda$. Entonces $N\left(t\right)$ es un proceso Poisson con tasa $\lambda$.

\end{Ejem}


\begin{Note}
Si el primer tiempo de renovaci\'on $\xi_{1}$ no tiene la misma distribuci\'on que el resto de las $\xi_{n}$, para $n\geq2$, a $N\left(t\right)$ se le llama Proceso de Renovaci\'on retardado, donde si $\xi$ tiene distribuci\'on $G$, entonces el tiempo $T_{n}$ de la $n$-\'esima renovaci\'on tiene distribuci\'on $G\star F^{\left(n-1\right)\star}\left(t\right)$
\end{Note}


\begin{Teo}
Para una constante $\mu\leq\infty$ ( o variable aleatoria), las siguientes expresiones son equivalentes:

\begin{eqnarray}
lim_{n\rightarrow\infty}n^{-1}T_{n}&=&\mu,\textrm{ c.s.}\\
lim_{t\rightarrow\infty}t^{-1}N\left(t\right)&=&1/\mu,\textrm{ c.s.}
\end{eqnarray}
\end{Teo}


Es decir, $T_{n}$ satisface la Ley Fuerte de los Grandes N\'umeros s\'i y s\'olo s\'i $N\left/t\right)$ la cumple.


\begin{Coro}[Ley Fuerte de los Grandes N\'umeros para Procesos de Renovaci\'on]
Si $N\left(t\right)$ es un proceso de renovaci\'on cuyos tiempos de inter-renovaci\'on tienen media $\mu\leq\infty$, entonces
\begin{eqnarray}
t^{-1}N\left(t\right)\rightarrow 1/\mu,\textrm{ c.s. cuando }t\rightarrow\infty.
\end{eqnarray}

\end{Coro}


Considerar el proceso estoc\'astico de valores reales $\left\{Z\left(t\right):t\geq0\right\}$ en el mismo espacio de probabilidad que $N\left(t\right)$

\begin{Def}
Para el proceso $\left\{Z\left(t\right):t\geq0\right\}$ se define la fluctuaci\'on m\'axima de $Z\left(t\right)$ en el intervalo $\left(T_{n-1},T_{n}\right]$:
\begin{eqnarray*}
M_{n}=\sup_{T_{n-1}<t\leq T_{n}}|Z\left(t\right)-Z\left(T_{n-1}\right)|
\end{eqnarray*}
\end{Def}

\begin{Teo}
Sup\'ongase que $n^{-1}T_{n}\rightarrow\mu$ c.s. cuando $n\rightarrow\infty$, donde $\mu\leq\infty$ es una constante o variable aleatoria. Sea $a$ una constante o variable aleatoria que puede ser infinita cuando $\mu$ es finita, y considere las expresiones l\'imite:
\begin{eqnarray}
lim_{n\rightarrow\infty}n^{-1}Z\left(T_{n}\right)&=&a,\textrm{ c.s.}\\
lim_{t\rightarrow\infty}t^{-1}Z\left(t\right)&=&a/\mu,\textrm{ c.s.}
\end{eqnarray}
La segunda expresi\'on implica la primera. Conversamente, la primera implica la segunda si el proceso $Z\left(t\right)$ es creciente, o si $lim_{n\rightarrow\infty}n^{-1}M_{n}=0$ c.s.
\end{Teo}

\begin{Coro}
Si $N\left(t\right)$ es un proceso de renovaci\'on, y $\left(Z\left(T_{n}\right)-Z\left(T_{n-1}\right),M_{n}\right)$, para $n\geq1$, son variables aleatorias independientes e id\'enticamente distribuidas con media finita, entonces,
\begin{eqnarray}
lim_{t\rightarrow\infty}t^{-1}Z\left(t\right)\rightarrow\frac{\esp\left[Z\left(T_{1}\right)-Z\left(T_{0}\right)\right]}{\esp\left[T_{1}\right]},\textrm{ c.s. cuando  }t\rightarrow\infty.
\end{eqnarray}
\end{Coro}


Sup\'ongase que $N\left(t\right)$ es un proceso de renovaci\'on con distribuci\'on $F$ con media finita $\mu$.

\begin{Def}
La funci\'on de renovaci\'on asociada con la distribuci\'on $F$, del proceso $N\left(t\right)$, es
\begin{eqnarray*}
U\left(t\right)=\sum_{n=1}^{\infty}F^{n\star}\left(t\right),\textrm{   }t\geq0,
\end{eqnarray*}
donde $F^{0\star}\left(t\right)=\indora\left(t\geq0\right)$.
\end{Def}


\begin{Prop}
Sup\'ongase que la distribuci\'on de inter-renovaci\'on $F$ tiene densidad $f$. Entonces $U\left(t\right)$ tambi\'en tiene densidad, para $t>0$, y es $U^{'}\left(t\right)=\sum_{n=0}^{\infty}f^{n\star}\left(t\right)$. Adem\'as
\begin{eqnarray*}
\prob\left\{N\left(t\right)>N\left(t-\right)\right\}=0\textrm{,   }t\geq0.
\end{eqnarray*}
\end{Prop}

\begin{Def}
La Transformada de Laplace-Stieljes de $F$ est\'a dada por

\begin{eqnarray*}
\hat{F}\left(\alpha\right)=\int_{\rea_{+}}e^{-\alpha t}dF\left(t\right)\textrm{,  }\alpha\geq0.
\end{eqnarray*}
\end{Def}

Entonces

\begin{eqnarray*}
\hat{U}\left(\alpha\right)=\sum_{n=0}^{\infty}\hat{F^{n\star}}\left(\alpha\right)=\sum_{n=0}^{\infty}\hat{F}\left(\alpha\right)^{n}=\frac{1}{1-\hat{F}\left(\alpha\right)}.
\end{eqnarray*}


\begin{Prop}
La Transformada de Laplace $\hat{U}\left(\alpha\right)$ y $\hat{F}\left(\alpha\right)$ determina una a la otra de manera \'unica por la relaci\'on $\hat{U}\left(\alpha\right)=\frac{1}{1-\hat{F}\left(\alpha\right)}$.
\end{Prop}


\begin{Note}
Un proceso de renovaci\'on $N\left(t\right)$ cuyos tiempos de inter-renovaci\'on tienen media finita, es un proceso Poisson con tasa $\lambda$ si y s\'olo s\'i $\esp\left[U\left(t\right)\right]=\lambda t$, para $t\geq0$.
\end{Note}


\begin{Teo}
Sea $N\left(t\right)$ un proceso puntual simple con puntos de localizaci\'on $T_{n}$ tal que $\eta\left(t\right)=\esp\left[N\left(\right)\right]$ es finita para cada $t$. Entonces para cualquier funci\'on $f:\rea_{+}\rightarrow\rea$,
\begin{eqnarray*}
\esp\left[\sum_{n=1}^{N\left(\right)}f\left(T_{n}\right)\right]=\int_{\left(0,t\right]}f\left(s\right)d\eta\left(s\right)\textrm{,  }t\geq0,
\end{eqnarray*}
suponiendo que la integral exista. Adem\'as si $X_{1},X_{2},\ldots$ son variables aleatorias definidas en el mismo espacio de probabilidad que el proceso $N\left(t\right)$ tal que $\esp\left[X_{n}|T_{n}=s\right]=f\left(s\right)$, independiente de $n$. Entonces
\begin{eqnarray*}
\esp\left[\sum_{n=1}^{N\left(t\right)}X_{n}\right]=\int_{\left(0,t\right]}f\left(s\right)d\eta\left(s\right)\textrm{,  }t\geq0,
\end{eqnarray*} 
suponiendo que la integral exista. 
\end{Teo}

\begin{Coro}[Identidad de Wald para Renovaciones]
Para el proceso de renovaci\'on $N\left(t\right)$,
\begin{eqnarray*}
\esp\left[T_{N\left(t\right)+1}\right]=\mu\esp\left[N\left(t\right)+1\right]\textrm{,  }t\geq0,
\end{eqnarray*}  
\end{Coro}


\begin{Def}
Sea $h\left(t\right)$ funci\'on de valores reales en $\rea$ acotada en intervalos finitos e igual a cero para $t<0$ La ecuaci\'on de renovaci\'on para $h\left(t\right)$ y la distribuci\'on $F$ es

\begin{eqnarray}\label{Ec.Renovacion}
H\left(t\right)=h\left(t\right)+\int_{\left[0,t\right]}H\left(t-s\right)dF\left(s\right)\textrm{,    }t\geq0,
\end{eqnarray}
donde $H\left(t\right)$ es una funci\'on de valores reales. Esto es $H=h+F\star H$. Decimos que $H\left(t\right)$ es soluci\'on de esta ecuaci\'on si satisface la ecuaci\'on, y es acotada en intervalos finitos e iguales a cero para $t<0$.
\end{Def}

\begin{Prop}
La funci\'on $U\star h\left(t\right)$ es la \'unica soluci\'on de la ecuaci\'on de renovaci\'on (\ref{Ec.Renovacion}).
\end{Prop}

\begin{Teo}[Teorema Renovaci\'on Elemental]
\begin{eqnarray*}
t^{-1}U\left(t\right)\rightarrow 1/\mu\textrm{,    cuando }t\rightarrow\infty.
\end{eqnarray*}
\end{Teo}



Sup\'ongase que $N\left(t\right)$ es un proceso de renovaci\'on con distribuci\'on $F$ con media finita $\mu$.

\begin{Def}
La funci\'on de renovaci\'on asociada con la distribuci\'on $F$, del proceso $N\left(t\right)$, es
\begin{eqnarray*}
U\left(t\right)=\sum_{n=1}^{\infty}F^{n\star}\left(t\right),\textrm{   }t\geq0,
\end{eqnarray*}
donde $F^{0\star}\left(t\right)=\indora\left(t\geq0\right)$.
\end{Def}


\begin{Prop}
Sup\'ongase que la distribuci\'on de inter-renovaci\'on $F$ tiene densidad $f$. Entonces $U\left(t\right)$ tambi\'en tiene densidad, para $t>0$, y es $U^{'}\left(t\right)=\sum_{n=0}^{\infty}f^{n\star}\left(t\right)$. Adem\'as
\begin{eqnarray*}
\prob\left\{N\left(t\right)>N\left(t-\right)\right\}=0\textrm{,   }t\geq0.
\end{eqnarray*}
\end{Prop}

\begin{Def}
La Transformada de Laplace-Stieljes de $F$ est\'a dada por

\begin{eqnarray*}
\hat{F}\left(\alpha\right)=\int_{\rea_{+}}e^{-\alpha t}dF\left(t\right)\textrm{,  }\alpha\geq0.
\end{eqnarray*}
\end{Def}

Entonces

\begin{eqnarray*}
\hat{U}\left(\alpha\right)=\sum_{n=0}^{\infty}\hat{F^{n\star}}\left(\alpha\right)=\sum_{n=0}^{\infty}\hat{F}\left(\alpha\right)^{n}=\frac{1}{1-\hat{F}\left(\alpha\right)}.
\end{eqnarray*}


\begin{Prop}
La Transformada de Laplace $\hat{U}\left(\alpha\right)$ y $\hat{F}\left(\alpha\right)$ determina una a la otra de manera \'unica por la relaci\'on $\hat{U}\left(\alpha\right)=\frac{1}{1-\hat{F}\left(\alpha\right)}$.
\end{Prop}


\begin{Note}
Un proceso de renovaci\'on $N\left(t\right)$ cuyos tiempos de inter-renovaci\'on tienen media finita, es un proceso Poisson con tasa $\lambda$ si y s\'olo s\'i $\esp\left[U\left(t\right)\right]=\lambda t$, para $t\geq0$.
\end{Note}


\begin{Teo}
Sea $N\left(t\right)$ un proceso puntual simple con puntos de localizaci\'on $T_{n}$ tal que $\eta\left(t\right)=\esp\left[N\left(\right)\right]$ es finita para cada $t$. Entonces para cualquier funci\'on $f:\rea_{+}\rightarrow\rea$,
\begin{eqnarray*}
\esp\left[\sum_{n=1}^{N\left(\right)}f\left(T_{n}\right)\right]=\int_{\left(0,t\right]}f\left(s\right)d\eta\left(s\right)\textrm{,  }t\geq0,
\end{eqnarray*}
suponiendo que la integral exista. Adem\'as si $X_{1},X_{2},\ldots$ son variables aleatorias definidas en el mismo espacio de probabilidad que el proceso $N\left(t\right)$ tal que $\esp\left[X_{n}|T_{n}=s\right]=f\left(s\right)$, independiente de $n$. Entonces
\begin{eqnarray*}
\esp\left[\sum_{n=1}^{N\left(t\right)}X_{n}\right]=\int_{\left(0,t\right]}f\left(s\right)d\eta\left(s\right)\textrm{,  }t\geq0,
\end{eqnarray*} 
suponiendo que la integral exista. 
\end{Teo}

\begin{Coro}[Identidad de Wald para Renovaciones]
Para el proceso de renovaci\'on $N\left(t\right)$,
\begin{eqnarray*}
\esp\left[T_{N\left(t\right)+1}\right]=\mu\esp\left[N\left(t\right)+1\right]\textrm{,  }t\geq0,
\end{eqnarray*}  
\end{Coro}


\begin{Def}
Sea $h\left(t\right)$ funci\'on de valores reales en $\rea$ acotada en intervalos finitos e igual a cero para $t<0$ La ecuaci\'on de renovaci\'on para $h\left(t\right)$ y la distribuci\'on $F$ es

\begin{eqnarray}\label{Ec.Renovacion}
H\left(t\right)=h\left(t\right)+\int_{\left[0,t\right]}H\left(t-s\right)dF\left(s\right)\textrm{,    }t\geq0,
\end{eqnarray}
donde $H\left(t\right)$ es una funci\'on de valores reales. Esto es $H=h+F\star H$. Decimos que $H\left(t\right)$ es soluci\'on de esta ecuaci\'on si satisface la ecuaci\'on, y es acotada en intervalos finitos e iguales a cero para $t<0$.
\end{Def}

\begin{Prop}
La funci\'on $U\star h\left(t\right)$ es la \'unica soluci\'on de la ecuaci\'on de renovaci\'on (\ref{Ec.Renovacion}).
\end{Prop}

\begin{Teo}[Teorema Renovaci\'on Elemental]
\begin{eqnarray*}
t^{-1}U\left(t\right)\rightarrow 1/\mu\textrm{,    cuando }t\rightarrow\infty.
\end{eqnarray*}
\end{Teo}


\begin{Note} Una funci\'on $h:\rea_{+}\rightarrow\rea$ es Directamente Riemann Integrable en los siguientes casos:
\begin{itemize}
\item[a)] $h\left(t\right)\geq0$ es decreciente y Riemann Integrable.
\item[b)] $h$ es continua excepto posiblemente en un conjunto de Lebesgue de medida 0, y $|h\left(t\right)|\leq b\left(t\right)$, donde $b$ es DRI.
\end{itemize}
\end{Note}

\begin{Teo}[Teorema Principal de Renovaci\'on]
Si $F$ es no aritm\'etica y $h\left(t\right)$ es Directamente Riemann Integrable (DRI), entonces

\begin{eqnarray*}
lim_{t\rightarrow\infty}U\star h=\frac{1}{\mu}\int_{\rea_{+}}h\left(s\right)ds.
\end{eqnarray*}
\end{Teo}

\begin{Prop}
Cualquier funci\'on $H\left(t\right)$ acotada en intervalos finitos y que es 0 para $t<0$ puede expresarse como
\begin{eqnarray*}
H\left(t\right)=U\star h\left(t\right)\textrm{,  donde }h\left(t\right)=H\left(t\right)-F\star H\left(t\right)
\end{eqnarray*}
\end{Prop}

\begin{Def}
Un proceso estoc\'astico $X\left(t\right)$ es crudamente regenerativo en un tiempo aleatorio positivo $T$ si
\begin{eqnarray*}
\esp\left[X\left(T+t\right)|T\right]=\esp\left[X\left(t\right)\right]\textrm{, para }t\geq0,\end{eqnarray*}
y con las esperanzas anteriores finitas.
\end{Def}

\begin{Prop}
Sup\'ongase que $X\left(t\right)$ es un proceso crudamente regenerativo en $T$, que tiene distribuci\'on $F$. Si $\esp\left[X\left(t\right)\right]$ es acotado en intervalos finitos, entonces
\begin{eqnarray*}
\esp\left[X\left(t\right)\right]=U\star h\left(t\right)\textrm{,  donde }h\left(t\right)=\esp\left[X\left(t\right)\indora\left(T>t\right)\right].
\end{eqnarray*}
\end{Prop}

\begin{Teo}[Regeneraci\'on Cruda]
Sup\'ongase que $X\left(t\right)$ es un proceso con valores positivo crudamente regenerativo en $T$, y def\'inase $M=\sup\left\{|X\left(t\right)|:t\leq T\right\}$. Si $T$ es no aritm\'etico y $M$ y $MT$ tienen media finita, entonces
\begin{eqnarray*}
lim_{t\rightarrow\infty}\esp\left[X\left(t\right)\right]=\frac{1}{\mu}\int_{\rea_{+}}h\left(s\right)ds,
\end{eqnarray*}
donde $h\left(t\right)=\esp\left[X\left(t\right)\indora\left(T>t\right)\right]$.
\end{Teo}


\begin{Note} Una funci\'on $h:\rea_{+}\rightarrow\rea$ es Directamente Riemann Integrable en los siguientes casos:
\begin{itemize}
\item[a)] $h\left(t\right)\geq0$ es decreciente y Riemann Integrable.
\item[b)] $h$ es continua excepto posiblemente en un conjunto de Lebesgue de medida 0, y $|h\left(t\right)|\leq b\left(t\right)$, donde $b$ es DRI.
\end{itemize}
\end{Note}

\begin{Teo}[Teorema Principal de Renovaci\'on]
Si $F$ es no aritm\'etica y $h\left(t\right)$ es Directamente Riemann Integrable (DRI), entonces

\begin{eqnarray*}
lim_{t\rightarrow\infty}U\star h=\frac{1}{\mu}\int_{\rea_{+}}h\left(s\right)ds.
\end{eqnarray*}
\end{Teo}

\begin{Prop}
Cualquier funci\'on $H\left(t\right)$ acotada en intervalos finitos y que es 0 para $t<0$ puede expresarse como
\begin{eqnarray*}
H\left(t\right)=U\star h\left(t\right)\textrm{,  donde }h\left(t\right)=H\left(t\right)-F\star H\left(t\right)
\end{eqnarray*}
\end{Prop}

\begin{Def}
Un proceso estoc\'astico $X\left(t\right)$ es crudamente regenerativo en un tiempo aleatorio positivo $T$ si
\begin{eqnarray*}
\esp\left[X\left(T+t\right)|T\right]=\esp\left[X\left(t\right)\right]\textrm{, para }t\geq0,\end{eqnarray*}
y con las esperanzas anteriores finitas.
\end{Def}

\begin{Prop}
Sup\'ongase que $X\left(t\right)$ es un proceso crudamente regenerativo en $T$, que tiene distribuci\'on $F$. Si $\esp\left[X\left(t\right)\right]$ es acotado en intervalos finitos, entonces
\begin{eqnarray*}
\esp\left[X\left(t\right)\right]=U\star h\left(t\right)\textrm{,  donde }h\left(t\right)=\esp\left[X\left(t\right)\indora\left(T>t\right)\right].
\end{eqnarray*}
\end{Prop}

\begin{Teo}[Regeneraci\'on Cruda]
Sup\'ongase que $X\left(t\right)$ es un proceso con valores positivo crudamente regenerativo en $T$, y def\'inase $M=\sup\left\{|X\left(t\right)|:t\leq T\right\}$. Si $T$ es no aritm\'etico y $M$ y $MT$ tienen media finita, entonces
\begin{eqnarray*}
lim_{t\rightarrow\infty}\esp\left[X\left(t\right)\right]=\frac{1}{\mu}\int_{\rea_{+}}h\left(s\right)ds,
\end{eqnarray*}
donde $h\left(t\right)=\esp\left[X\left(t\right)\indora\left(T>t\right)\right]$.
\end{Teo}

\begin{Def}
Para el proceso $\left\{\left(N\left(t\right),X\left(t\right)\right):t\geq0\right\}$, sus trayectoria muestrales en el intervalo de tiempo $\left[T_{n-1},T_{n}\right)$ est\'an descritas por
\begin{eqnarray*}
\zeta_{n}=\left(\xi_{n},\left\{X\left(T_{n-1}+t\right):0\leq t<\xi_{n}\right\}\right)
\end{eqnarray*}
Este $\zeta_{n}$ es el $n$-\'esimo segmento del proceso. El proceso es regenerativo sobre los tiempos $T_{n}$ si sus segmentos $\zeta_{n}$ son independientes e id\'enticamennte distribuidos.
\end{Def}


\begin{Note}
Si $\tilde{X}\left(t\right)$ con espacio de estados $\tilde{S}$ es regenerativo sobre $T_{n}$, entonces $X\left(t\right)=f\left(\tilde{X}\left(t\right)\right)$ tambi\'en es regenerativo sobre $T_{n}$, para cualquier funci\'on $f:\tilde{S}\rightarrow S$.
\end{Note}

\begin{Note}
Los procesos regenerativos son crudamente regenerativos, pero no al rev\'es.
\end{Note}


\begin{Note}
Un proceso estoc\'astico a tiempo continuo o discreto es regenerativo si existe un proceso de renovaci\'on  tal que los segmentos del proceso entre tiempos de renovaci\'on sucesivos son i.i.d., es decir, para $\left\{X\left(t\right):t\geq0\right\}$ proceso estoc\'astico a tiempo continuo con espacio de estados $S$, espacio m\'etrico.
\end{Note}

Para $\left\{X\left(t\right):t\geq0\right\}$ Proceso Estoc\'astico a tiempo continuo con estado de espacios $S$, que es un espacio m\'etrico, con trayectorias continuas por la derecha y con l\'imites por la izquierda c.s. Sea $N\left(t\right)$ un proceso de renovaci\'on en $\rea_{+}$ definido en el mismo espacio de probabilidad que $X\left(t\right)$, con tiempos de renovaci\'on $T$ y tiempos de inter-renovaci\'on $\xi_{n}=T_{n}-T_{n-1}$, con misma distribuci\'on $F$ de media finita $\mu$.



\begin{Def}
Para el proceso $\left\{\left(N\left(t\right),X\left(t\right)\right):t\geq0\right\}$, sus trayectoria muestrales en el intervalo de tiempo $\left[T_{n-1},T_{n}\right)$ est\'an descritas por
\begin{eqnarray*}
\zeta_{n}=\left(\xi_{n},\left\{X\left(T_{n-1}+t\right):0\leq t<\xi_{n}\right\}\right)
\end{eqnarray*}
Este $\zeta_{n}$ es el $n$-\'esimo segmento del proceso. El proceso es regenerativo sobre los tiempos $T_{n}$ si sus segmentos $\zeta_{n}$ son independientes e id\'enticamennte distribuidos.
\end{Def}

\begin{Note}
Un proceso regenerativo con media de la longitud de ciclo finita es llamado positivo recurrente.
\end{Note}

\begin{Teo}[Procesos Regenerativos]
Suponga que el proceso
\end{Teo}


\begin{Def}[Renewal Process Trinity]
Para un proceso de renovaci\'on $N\left(t\right)$, los siguientes procesos proveen de informaci\'on sobre los tiempos de renovaci\'on.
\begin{itemize}
\item $A\left(t\right)=t-T_{N\left(t\right)}$, el tiempo de recurrencia hacia atr\'as al tiempo $t$, que es el tiempo desde la \'ultima renovaci\'on para $t$.

\item $B\left(t\right)=T_{N\left(t\right)+1}-t$, el tiempo de recurrencia hacia adelante al tiempo $t$, residual del tiempo de renovaci\'on, que es el tiempo para la pr\'oxima renovaci\'on despu\'es de $t$.

\item $L\left(t\right)=\xi_{N\left(t\right)+1}=A\left(t\right)+B\left(t\right)$, la longitud del intervalo de renovaci\'on que contiene a $t$.
\end{itemize}
\end{Def}

\begin{Note}
El proceso tridimensional $\left(A\left(t\right),B\left(t\right),L\left(t\right)\right)$ es regenerativo sobre $T_{n}$, y por ende cada proceso lo es. Cada proceso $A\left(t\right)$ y $B\left(t\right)$ son procesos de MArkov a tiempo continuo con trayectorias continuas por partes en el espacio de estados $\rea_{+}$. Una expresi\'on conveniente para su distribuci\'on conjunta es, para $0\leq x<t,y\geq0$
\begin{equation}\label{NoRenovacion}
P\left\{A\left(t\right)>x,B\left(t\right)>y\right\}=
P\left\{N\left(t+y\right)-N\left((t-x)\right)=0\right\}
\end{equation}
\end{Note}

\begin{Ejem}[Tiempos de recurrencia Poisson]
Si $N\left(t\right)$ es un proceso Poisson con tasa $\lambda$, entonces de la expresi\'on (\ref{NoRenovacion}) se tiene que

\begin{eqnarray*}
\begin{array}{lc}
P\left\{A\left(t\right)>x,B\left(t\right)>y\right\}=e^{-\lambda\left(x+y\right)},&0\leq x<t,y\geq0,
\end{array}
\end{eqnarray*}
que es la probabilidad Poisson de no renovaciones en un intervalo de longitud $x+y$.

\end{Ejem}

\begin{Note}
Una cadena de Markov erg\'odica tiene la propiedad de ser estacionaria si la distribuci\'on de su estado al tiempo $0$ es su distribuci\'on estacionaria.
\end{Note}


\begin{Def}
Un proceso estoc\'astico a tiempo continuo $\left\{X\left(t\right):t\geq0\right\}$ en un espacio general es estacionario si sus distribuciones finito dimensionales son invariantes bajo cualquier  traslado: para cada $0\leq s_{1}<s_{2}<\cdots<s_{k}$ y $t\geq0$,
\begin{eqnarray*}
\left(X\left(s_{1}+t\right),\ldots,X\left(s_{k}+t\right)\right)=_{d}\left(X\left(s_{1}\right),\ldots,X\left(s_{k}\right)\right).
\end{eqnarray*}
\end{Def}

\begin{Note}
Un proceso de Markov es estacionario si $X\left(t\right)=_{d}X\left(0\right)$, $t\geq0$.
\end{Note}

Considerese el proceso $N\left(t\right)=\sum_{n}\indora\left(\tau_{n}\leq t\right)$ en $\rea_{+}$, con puntos $0<\tau_{1}<\tau_{2}<\cdots$.

\begin{Prop}
Si $N$ es un proceso puntual estacionario y $\esp\left[N\left(1\right)\right]<\infty$, entonces $\esp\left[N\left(t\right)\right]=t\esp\left[N\left(1\right)\right]$, $t\geq0$

\end{Prop}

\begin{Teo}
Los siguientes enunciados son equivalentes
\begin{itemize}
\item[i)] El proceso retardado de renovaci\'on $N$ es estacionario.

\item[ii)] EL proceso de tiempos de recurrencia hacia adelante $B\left(t\right)$ es estacionario.


\item[iii)] $\esp\left[N\left(t\right)\right]=t/\mu$,


\item[iv)] $G\left(t\right)=F_{e}\left(t\right)=\frac{1}{\mu}\int_{0}^{t}\left[1-F\left(s\right)\right]ds$
\end{itemize}
Cuando estos enunciados son ciertos, $P\left\{B\left(t\right)\leq x\right\}=F_{e}\left(x\right)$, para $t,x\geq0$.

\end{Teo}

\begin{Note}
Una consecuencia del teorema anterior es que el Proceso Poisson es el \'unico proceso sin retardo que es estacionario.
\end{Note}

\begin{Coro}
El proceso de renovaci\'on $N\left(t\right)$ sin retardo, y cuyos tiempos de inter renonaci\'on tienen media finita, es estacionario si y s\'olo si es un proceso Poisson.

\end{Coro}

%______________________________________________________________________

%\section{Ejemplos, Notas importantes}
%______________________________________________________________________
%\section*{Ap\'endice A}
%__________________________________________________________________

%________________________________________________________________________
%\subsection*{Procesos Regenerativos}
%________________________________________________________________________



\begin{Note}
Si $\tilde{X}\left(t\right)$ con espacio de estados $\tilde{S}$ es regenerativo sobre $T_{n}$, entonces $X\left(t\right)=f\left(\tilde{X}\left(t\right)\right)$ tambi\'en es regenerativo sobre $T_{n}$, para cualquier funci\'on $f:\tilde{S}\rightarrow S$.
\end{Note}

\begin{Note}
Los procesos regenerativos son crudamente regenerativos, pero no al rev\'es.
\end{Note}
%\subsection*{Procesos Regenerativos: Sigman\cite{Sigman1}}
\begin{Def}[Definici\'on Cl\'asica]
Un proceso estoc\'astico $X=\left\{X\left(t\right):t\geq0\right\}$ es llamado regenerativo is existe una variable aleatoria $R_{1}>0$ tal que
\begin{itemize}
\item[i)] $\left\{X\left(t+R_{1}\right):t\geq0\right\}$ es independiente de $\left\{\left\{X\left(t\right):t<R_{1}\right\},\right\}$
\item[ii)] $\left\{X\left(t+R_{1}\right):t\geq0\right\}$ es estoc\'asticamente equivalente a $\left\{X\left(t\right):t>0\right\}$
\end{itemize}

Llamamos a $R_{1}$ tiempo de regeneraci\'on, y decimos que $X$ se regenera en este punto.
\end{Def}

$\left\{X\left(t+R_{1}\right)\right\}$ es regenerativo con tiempo de regeneraci\'on $R_{2}$, independiente de $R_{1}$ pero con la misma distribuci\'on que $R_{1}$. Procediendo de esta manera se obtiene una secuencia de variables aleatorias independientes e id\'enticamente distribuidas $\left\{R_{n}\right\}$ llamados longitudes de ciclo. Si definimos a $Z_{k}\equiv R_{1}+R_{2}+\cdots+R_{k}$, se tiene un proceso de renovaci\'on llamado proceso de renovaci\'on encajado para $X$.




\begin{Def}
Para $x$ fijo y para cada $t\geq0$, sea $I_{x}\left(t\right)=1$ si $X\left(t\right)\leq x$,  $I_{x}\left(t\right)=0$ en caso contrario, y def\'inanse los tiempos promedio
\begin{eqnarray*}
\overline{X}&=&lim_{t\rightarrow\infty}\frac{1}{t}\int_{0}^{\infty}X\left(u\right)du\\
\prob\left(X_{\infty}\leq x\right)&=&lim_{t\rightarrow\infty}\frac{1}{t}\int_{0}^{\infty}I_{x}\left(u\right)du,
\end{eqnarray*}
cuando estos l\'imites existan.
\end{Def}

Como consecuencia del teorema de Renovaci\'on-Recompensa, se tiene que el primer l\'imite  existe y es igual a la constante
\begin{eqnarray*}
\overline{X}&=&\frac{\esp\left[\int_{0}^{R_{1}}X\left(t\right)dt\right]}{\esp\left[R_{1}\right]},
\end{eqnarray*}
suponiendo que ambas esperanzas son finitas.

\begin{Note}
\begin{itemize}
\item[a)] Si el proceso regenerativo $X$ es positivo recurrente y tiene trayectorias muestrales no negativas, entonces la ecuaci\'on anterior es v\'alida.
\item[b)] Si $X$ es positivo recurrente regenerativo, podemos construir una \'unica versi\'on estacionaria de este proceso, $X_{e}=\left\{X_{e}\left(t\right)\right\}$, donde $X_{e}$ es un proceso estoc\'astico regenerativo y estrictamente estacionario, con distribuci\'on marginal distribuida como $X_{\infty}$
\end{itemize}
\end{Note}

Para $\left\{X\left(t\right):t\geq0\right\}$ Proceso Estoc\'astico a tiempo continuo con estado de espacios $S$, que es un espacio m\'etrico, con trayectorias continuas por la derecha y con l\'imites por la izquierda c.s. Sea $N\left(t\right)$ un proceso de renovaci\'on en $\rea_{+}$ definido en el mismo espacio de probabilidad que $X\left(t\right)$, con tiempos de renovaci\'on $T$ y tiempos de inter-renovaci\'on $\xi_{n}=T_{n}-T_{n-1}$, con misma distribuci\'on $F$ de media finita $\mu$.


\begin{Def}
Para el proceso $\left\{\left(N\left(t\right),X\left(t\right)\right):t\geq0\right\}$, sus trayectoria muestrales en el intervalo de tiempo $\left[T_{n-1},T_{n}\right)$ est\'an descritas por
\begin{eqnarray*}
\zeta_{n}=\left(\xi_{n},\left\{X\left(T_{n-1}+t\right):0\leq t<\xi_{n}\right\}\right)
\end{eqnarray*}
Este $\zeta_{n}$ es el $n$-\'esimo segmento del proceso. El proceso es regenerativo sobre los tiempos $T_{n}$ si sus segmentos $\zeta_{n}$ son independientes e id\'enticamennte distribuidos.
\end{Def}


\begin{Note}
Si $\tilde{X}\left(t\right)$ con espacio de estados $\tilde{S}$ es regenerativo sobre $T_{n}$, entonces $X\left(t\right)=f\left(\tilde{X}\left(t\right)\right)$ tambi\'en es regenerativo sobre $T_{n}$, para cualquier funci\'on $f:\tilde{S}\rightarrow S$.
\end{Note}

\begin{Note}
Los procesos regenerativos son crudamente regenerativos, pero no al rev\'es.
\end{Note}

\begin{Def}[Definici\'on Cl\'asica]
Un proceso estoc\'astico $X=\left\{X\left(t\right):t\geq0\right\}$ es llamado regenerativo is existe una variable aleatoria $R_{1}>0$ tal que
\begin{itemize}
\item[i)] $\left\{X\left(t+R_{1}\right):t\geq0\right\}$ es independiente de $\left\{\left\{X\left(t\right):t<R_{1}\right\},\right\}$
\item[ii)] $\left\{X\left(t+R_{1}\right):t\geq0\right\}$ es estoc\'asticamente equivalente a $\left\{X\left(t\right):t>0\right\}$
\end{itemize}

Llamamos a $R_{1}$ tiempo de regeneraci\'on, y decimos que $X$ se regenera en este punto.
\end{Def}

$\left\{X\left(t+R_{1}\right)\right\}$ es regenerativo con tiempo de regeneraci\'on $R_{2}$, independiente de $R_{1}$ pero con la misma distribuci\'on que $R_{1}$. Procediendo de esta manera se obtiene una secuencia de variables aleatorias independientes e id\'enticamente distribuidas $\left\{R_{n}\right\}$ llamados longitudes de ciclo. Si definimos a $Z_{k}\equiv R_{1}+R_{2}+\cdots+R_{k}$, se tiene un proceso de renovaci\'on llamado proceso de renovaci\'on encajado para $X$.

\begin{Note}
Un proceso regenerativo con media de la longitud de ciclo finita es llamado positivo recurrente.
\end{Note}


\begin{Def}
Para $x$ fijo y para cada $t\geq0$, sea $I_{x}\left(t\right)=1$ si $X\left(t\right)\leq x$,  $I_{x}\left(t\right)=0$ en caso contrario, y def\'inanse los tiempos promedio
\begin{eqnarray*}
\overline{X}&=&lim_{t\rightarrow\infty}\frac{1}{t}\int_{0}^{\infty}X\left(u\right)du\\
\prob\left(X_{\infty}\leq x\right)&=&lim_{t\rightarrow\infty}\frac{1}{t}\int_{0}^{\infty}I_{x}\left(u\right)du,
\end{eqnarray*}
cuando estos l\'imites existan.
\end{Def}

Como consecuencia del teorema de Renovaci\'on-Recompensa, se tiene que el primer l\'imite  existe y es igual a la constante
\begin{eqnarray*}
\overline{X}&=&\frac{\esp\left[\int_{0}^{R_{1}}X\left(t\right)dt\right]}{\esp\left[R_{1}\right]},
\end{eqnarray*}
suponiendo que ambas esperanzas son finitas.

\begin{Note}
\begin{itemize}
\item[a)] Si el proceso regenerativo $X$ es positivo recurrente y tiene trayectorias muestrales no negativas, entonces la ecuaci\'on anterior es v\'alida.
\item[b)] Si $X$ es positivo recurrente regenerativo, podemos construir una \'unica versi\'on estacionaria de este proceso, $X_{e}=\left\{X_{e}\left(t\right)\right\}$, donde $X_{e}$ es un proceso estoc\'astico regenerativo y estrictamente estacionario, con distribuci\'on marginal distribuida como $X_{\infty}$
\end{itemize}
\end{Note}

%__________________________________________________________________________________________
%\subsection{Procesos Regenerativos Estacionarios - Stidham \cite{Stidham}}
%__________________________________________________________________________________________


Un proceso estoc\'astico a tiempo continuo $\left\{V\left(t\right),t\geq0\right\}$ es un proceso regenerativo si existe una sucesi\'on de variables aleatorias independientes e id\'enticamente distribuidas $\left\{X_{1},X_{2},\ldots\right\}$, sucesi\'on de renovaci\'on, tal que para cualquier conjunto de Borel $A$, 

\begin{eqnarray*}
\prob\left\{V\left(t\right)\in A|X_{1}+X_{2}+\cdots+X_{R\left(t\right)}=s,\left\{V\left(\tau\right),\tau<s\right\}\right\}=\prob\left\{V\left(t-s\right)\in A|X_{1}>t-s\right\},
\end{eqnarray*}
para todo $0\leq s\leq t$, donde $R\left(t\right)=\max\left\{X_{1}+X_{2}+\cdots+X_{j}\leq t\right\}=$n\'umero de renovaciones ({\emph{puntos de regeneraci\'on}}) que ocurren en $\left[0,t\right]$. El intervalo $\left[0,X_{1}\right)$ es llamado {\emph{primer ciclo de regeneraci\'on}} de $\left\{V\left(t \right),t\geq0\right\}$, $\left[X_{1},X_{1}+X_{2}\right)$ el {\emph{segundo ciclo de regeneraci\'on}}, y as\'i sucesivamente.

Sea $X=X_{1}$ y sea $F$ la funci\'on de distrbuci\'on de $X$


\begin{Def}
Se define el proceso estacionario, $\left\{V^{*}\left(t\right),t\geq0\right\}$, para $\left\{V\left(t\right),t\geq0\right\}$ por

\begin{eqnarray*}
\prob\left\{V\left(t\right)\in A\right\}=\frac{1}{\esp\left[X\right]}\int_{0}^{\infty}\prob\left\{V\left(t+x\right)\in A|X>x\right\}\left(1-F\left(x\right)\right)dx,
\end{eqnarray*} 
para todo $t\geq0$ y todo conjunto de Borel $A$.
\end{Def}

\begin{Def}
Una distribuci\'on se dice que es {\emph{aritm\'etica}} si todos sus puntos de incremento son m\'ultiplos de la forma $0,\lambda, 2\lambda,\ldots$ para alguna $\lambda>0$ entera.
\end{Def}


\begin{Def}
Una modificaci\'on medible de un proceso $\left\{V\left(t\right),t\geq0\right\}$, es una versi\'on de este, $\left\{V\left(t,w\right)\right\}$ conjuntamente medible para $t\geq0$ y para $w\in S$, $S$ espacio de estados para $\left\{V\left(t\right),t\geq0\right\}$.
\end{Def}

\begin{Teo}
Sea $\left\{V\left(t\right),t\geq\right\}$ un proceso regenerativo no negativo con modificaci\'on medible. Sea $\esp\left[X\right]<\infty$. Entonces el proceso estacionario dado por la ecuaci\'on anterior est\'a bien definido y tiene funci\'on de distribuci\'on independiente de $t$, adem\'as
\begin{itemize}
\item[i)] \begin{eqnarray*}
\esp\left[V^{*}\left(0\right)\right]&=&\frac{\esp\left[\int_{0}^{X}V\left(s\right)ds\right]}{\esp\left[X\right]}\end{eqnarray*}
\item[ii)] Si $\esp\left[V^{*}\left(0\right)\right]<\infty$, equivalentemente, si $\esp\left[\int_{0}^{X}V\left(s\right)ds\right]<\infty$,entonces
\begin{eqnarray*}
\frac{\int_{0}^{t}V\left(s\right)ds}{t}\rightarrow\frac{\esp\left[\int_{0}^{X}V\left(s\right)ds\right]}{\esp\left[X\right]}
\end{eqnarray*}
con probabilidad 1 y en media, cuando $t\rightarrow\infty$.
\end{itemize}
\end{Teo}
%
%___________________________________________________________________________________________
%\vspace{5.5cm}
%\chapter{Cadenas de Markov estacionarias}
%\vspace{-1.0cm}


%__________________________________________________________________________________________
%\subsection{Procesos Regenerativos Estacionarios - Stidham \cite{Stidham}}
%__________________________________________________________________________________________


Un proceso estoc\'astico a tiempo continuo $\left\{V\left(t\right),t\geq0\right\}$ es un proceso regenerativo si existe una sucesi\'on de variables aleatorias independientes e id\'enticamente distribuidas $\left\{X_{1},X_{2},\ldots\right\}$, sucesi\'on de renovaci\'on, tal que para cualquier conjunto de Borel $A$, 

\begin{eqnarray*}
\prob\left\{V\left(t\right)\in A|X_{1}+X_{2}+\cdots+X_{R\left(t\right)}=s,\left\{V\left(\tau\right),\tau<s\right\}\right\}=\prob\left\{V\left(t-s\right)\in A|X_{1}>t-s\right\},
\end{eqnarray*}
para todo $0\leq s\leq t$, donde $R\left(t\right)=\max\left\{X_{1}+X_{2}+\cdots+X_{j}\leq t\right\}=$n\'umero de renovaciones ({\emph{puntos de regeneraci\'on}}) que ocurren en $\left[0,t\right]$. El intervalo $\left[0,X_{1}\right)$ es llamado {\emph{primer ciclo de regeneraci\'on}} de $\left\{V\left(t \right),t\geq0\right\}$, $\left[X_{1},X_{1}+X_{2}\right)$ el {\emph{segundo ciclo de regeneraci\'on}}, y as\'i sucesivamente.

Sea $X=X_{1}$ y sea $F$ la funci\'on de distrbuci\'on de $X$


\begin{Def}
Se define el proceso estacionario, $\left\{V^{*}\left(t\right),t\geq0\right\}$, para $\left\{V\left(t\right),t\geq0\right\}$ por

\begin{eqnarray*}
\prob\left\{V\left(t\right)\in A\right\}=\frac{1}{\esp\left[X\right]}\int_{0}^{\infty}\prob\left\{V\left(t+x\right)\in A|X>x\right\}\left(1-F\left(x\right)\right)dx,
\end{eqnarray*} 
para todo $t\geq0$ y todo conjunto de Borel $A$.
\end{Def}

\begin{Def}
Una distribuci\'on se dice que es {\emph{aritm\'etica}} si todos sus puntos de incremento son m\'ultiplos de la forma $0,\lambda, 2\lambda,\ldots$ para alguna $\lambda>0$ entera.
\end{Def}


\begin{Def}
Una modificaci\'on medible de un proceso $\left\{V\left(t\right),t\geq0\right\}$, es una versi\'on de este, $\left\{V\left(t,w\right)\right\}$ conjuntamente medible para $t\geq0$ y para $w\in S$, $S$ espacio de estados para $\left\{V\left(t\right),t\geq0\right\}$.
\end{Def}

\begin{Teo}
Sea $\left\{V\left(t\right),t\geq\right\}$ un proceso regenerativo no negativo con modificaci\'on medible. Sea $\esp\left[X\right]<\infty$. Entonces el proceso estacionario dado por la ecuaci\'on anterior est\'a bien definido y tiene funci\'on de distribuci\'on independiente de $t$, adem\'as
\begin{itemize}
\item[i)] \begin{eqnarray*}
\esp\left[V^{*}\left(0\right)\right]&=&\frac{\esp\left[\int_{0}^{X}V\left(s\right)ds\right]}{\esp\left[X\right]}\end{eqnarray*}
\item[ii)] Si $\esp\left[V^{*}\left(0\right)\right]<\infty$, equivalentemente, si $\esp\left[\int_{0}^{X}V\left(s\right)ds\right]<\infty$,entonces
\begin{eqnarray*}
\frac{\int_{0}^{t}V\left(s\right)ds}{t}\rightarrow\frac{\esp\left[\int_{0}^{X}V\left(s\right)ds\right]}{\esp\left[X\right]}
\end{eqnarray*}
con probabilidad 1 y en media, cuando $t\rightarrow\infty$.
\end{itemize}
\end{Teo}

Sea la funci\'on generadora de momentos para $L_{i}$, el n\'umero de usuarios en la cola $Q_{i}\left(z\right)$ en cualquier momento, est\'a dada por el tiempo promedio de $z^{L_{i}\left(t\right)}$ sobre el ciclo regenerativo definido anteriormente. Entonces 



Es decir, es posible determinar las longitudes de las colas a cualquier tiempo $t$. Entonces, determinando el primer momento es posible ver que


\begin{Def}
El tiempo de Ciclo $C_{i}$ es el periodo de tiempo que comienza cuando la cola $i$ es visitada por primera vez en un ciclo, y termina cuando es visitado nuevamente en el pr\'oximo ciclo. La duraci\'on del mismo est\'a dada por $\tau_{i}\left(m+1\right)-\tau_{i}\left(m\right)$, o equivalentemente $\overline{\tau}_{i}\left(m+1\right)-\overline{\tau}_{i}\left(m\right)$ bajo condiciones de estabilidad.
\end{Def}


\begin{Def}
El tiempo de intervisita $I_{i}$ es el periodo de tiempo que comienza cuando se ha completado el servicio en un ciclo y termina cuando es visitada nuevamente en el pr\'oximo ciclo. Su  duraci\'on del mismo est\'a dada por $\tau_{i}\left(m+1\right)-\overline{\tau}_{i}\left(m\right)$.
\end{Def}

La duraci\'on del tiempo de intervisita es $\tau_{i}\left(m+1\right)-\overline{\tau}\left(m\right)$. Dado que el n\'umero de usuarios presentes en $Q_{i}$ al tiempo $t=\tau_{i}\left(m+1\right)$ es igual al n\'umero de arribos durante el intervalo de tiempo $\left[\overline{\tau}\left(m\right),\tau_{i}\left(m+1\right)\right]$ se tiene que


\begin{eqnarray*}
\esp\left[z_{i}^{L_{i}\left(\tau_{i}\left(m+1\right)\right)}\right]=\esp\left[\left\{P_{i}\left(z_{i}\right)\right\}^{\tau_{i}\left(m+1\right)-\overline{\tau}\left(m\right)}\right]
\end{eqnarray*}

entonces, si $I_{i}\left(z\right)=\esp\left[z^{\tau_{i}\left(m+1\right)-\overline{\tau}\left(m\right)}\right]$
se tiene que $F_{i}\left(z\right)=I_{i}\left[P_{i}\left(z\right)\right]$
para $i=1,2$.

Conforme a la definici\'on dada al principio del cap\'itulo, definici\'on (\ref{Def.Tn}), sean $T_{1},T_{2},\ldots$ los puntos donde las longitudes de las colas de la red de sistemas de visitas c\'iclicas son cero simult\'aneamente, cuando la cola $Q_{j}$ es visitada por el servidor para dar servicio, es decir, $L_{1}\left(T_{i}\right)=0,L_{2}\left(T_{i}\right)=0,\hat{L}_{1}\left(T_{i}\right)=0$ y $\hat{L}_{2}\left(T_{i}\right)=0$, a estos puntos se les denominar\'a puntos regenerativos. Entonces, 

\begin{Def}
Al intervalo de tiempo entre dos puntos regenerativos se le llamar\'a ciclo regenerativo.
\end{Def}

\begin{Def}
Para $T_{i}$ se define, $M_{i}$, el n\'umero de ciclos de visita a la cola $Q_{l}$, durante el ciclo regenerativo, es decir, $M_{i}$ es un proceso de renovaci\'on.
\end{Def}

\begin{Def}
Para cada uno de los $M_{i}$'s, se definen a su vez la duraci\'on de cada uno de estos ciclos de visita en el ciclo regenerativo, $C_{i}^{(m)}$, para $m=1,2,\ldots,M_{i}$, que a su vez, tambi\'en es n proceso de renovaci\'on.
\end{Def}

\footnote{In Stidham and  Heyman \cite{Stidham} shows that is sufficient for the regenerative process to be stationary that the mean regenerative cycle time is finite: $\esp\left[\sum_{m=1}^{M_{i}}C_{i}^{(m)}\right]<\infty$, 


 como cada $C_{i}^{(m)}$ contiene intervalos de r\'eplica positivos, se tiene que $\esp\left[M_{i}\right]<\infty$, adem\'as, como $M_{i}>0$, se tiene que la condici\'on anterior es equivalente a tener que $\esp\left[C_{i}\right]<\infty$,
por lo tanto una condici\'on suficiente para la existencia del proceso regenerativo est\'a dada por $\sum_{k=1}^{N}\mu_{k}<1.$}
%________________________________________________________________________
\subsection{Procesos Regenerativos Sigman, Thorisson y Wolff \cite{Sigman2}}
%________________________________________________________________________


\begin{Def}[Definici\'on Cl\'asica]
Un proceso estoc\'astico $X=\left\{X\left(t\right):t\geq0\right\}$ es llamado regenerativo is existe una variable aleatoria $R_{1}>0$ tal que
\begin{itemize}
\item[i)] $\left\{X\left(t+R_{1}\right):t\geq0\right\}$ es independiente de $\left\{\left\{X\left(t\right):t<R_{1}\right\},\right\}$
\item[ii)] $\left\{X\left(t+R_{1}\right):t\geq0\right\}$ es estoc\'asticamente equivalente a $\left\{X\left(t\right):t>0\right\}$
\end{itemize}

Llamamos a $R_{1}$ tiempo de regeneraci\'on, y decimos que $X$ se regenera en este punto.
\end{Def}

$\left\{X\left(t+R_{1}\right)\right\}$ es regenerativo con tiempo de regeneraci\'on $R_{2}$, independiente de $R_{1}$ pero con la misma distribuci\'on que $R_{1}$. Procediendo de esta manera se obtiene una secuencia de variables aleatorias independientes e id\'enticamente distribuidas $\left\{R_{n}\right\}$ llamados longitudes de ciclo. Si definimos a $Z_{k}\equiv R_{1}+R_{2}+\cdots+R_{k}$, se tiene un proceso de renovaci\'on llamado proceso de renovaci\'on encajado para $X$.


\begin{Note}
La existencia de un primer tiempo de regeneraci\'on, $R_{1}$, implica la existencia de una sucesi\'on completa de estos tiempos $R_{1},R_{2}\ldots,$ que satisfacen la propiedad deseada \cite{Sigman2}.
\end{Note}


\begin{Note} Para la cola $GI/GI/1$ los usuarios arriban con tiempos $t_{n}$ y son atendidos con tiempos de servicio $S_{n}$, los tiempos de arribo forman un proceso de renovaci\'on  con tiempos entre arribos independientes e identicamente distribuidos (\texttt{i.i.d.})$T_{n}=t_{n}-t_{n-1}$, adem\'as los tiempos de servicio son \texttt{i.i.d.} e independientes de los procesos de arribo. Por \textit{estable} se entiende que $\esp S_{n}<\esp T_{n}<\infty$.
\end{Note}
 


\begin{Def}
Para $x$ fijo y para cada $t\geq0$, sea $I_{x}\left(t\right)=1$ si $X\left(t\right)\leq x$,  $I_{x}\left(t\right)=0$ en caso contrario, y def\'inanse los tiempos promedio
\begin{eqnarray*}
\overline{X}&=&lim_{t\rightarrow\infty}\frac{1}{t}\int_{0}^{\infty}X\left(u\right)du\\
\prob\left(X_{\infty}\leq x\right)&=&lim_{t\rightarrow\infty}\frac{1}{t}\int_{0}^{\infty}I_{x}\left(u\right)du,
\end{eqnarray*}
cuando estos l\'imites existan.
\end{Def}

Como consecuencia del teorema de Renovaci\'on-Recompensa, se tiene que el primer l\'imite  existe y es igual a la constante
\begin{eqnarray*}
\overline{X}&=&\frac{\esp\left[\int_{0}^{R_{1}}X\left(t\right)dt\right]}{\esp\left[R_{1}\right]},
\end{eqnarray*}
suponiendo que ambas esperanzas son finitas.
 
\begin{Note}
Funciones de procesos regenerativos son regenerativas, es decir, si $X\left(t\right)$ es regenerativo y se define el proceso $Y\left(t\right)$ por $Y\left(t\right)=f\left(X\left(t\right)\right)$ para alguna funci\'on Borel medible $f\left(\cdot\right)$. Adem\'as $Y$ es regenerativo con los mismos tiempos de renovaci\'on que $X$. 

En general, los tiempos de renovaci\'on, $Z_{k}$ de un proceso regenerativo no requieren ser tiempos de paro con respecto a la evoluci\'on de $X\left(t\right)$.
\end{Note} 

\begin{Note}
Una funci\'on de un proceso de Markov, usualmente no ser\'a un proceso de Markov, sin embargo ser\'a regenerativo si el proceso de Markov lo es.
\end{Note}

 
\begin{Note}
Un proceso regenerativo con media de la longitud de ciclo finita es llamado positivo recurrente.
\end{Note}


\begin{Note}
\begin{itemize}
\item[a)] Si el proceso regenerativo $X$ es positivo recurrente y tiene trayectorias muestrales no negativas, entonces la ecuaci\'on anterior es v\'alida.
\item[b)] Si $X$ es positivo recurrente regenerativo, podemos construir una \'unica versi\'on estacionaria de este proceso, $X_{e}=\left\{X_{e}\left(t\right)\right\}$, donde $X_{e}$ es un proceso estoc\'astico regenerativo y estrictamente estacionario, con distribuci\'on marginal distribuida como $X_{\infty}$
\end{itemize}
\end{Note}



%________________________________________________________________________
\subsection{Procesos Regenerativos}
%________________________________________________________________________



\begin{Note}
Si $\tilde{X}\left(t\right)$ con espacio de estados $\tilde{S}$ es regenerativo sobre $T_{n}$, entonces $X\left(t\right)=f\left(\tilde{X}\left(t\right)\right)$ tambi\'en es regenerativo sobre $T_{n}$, para cualquier funci\'on $f:\tilde{S}\rightarrow S$.
\end{Note}

\begin{Note}
Los procesos regenerativos son crudamente regenerativos, pero no al rev\'es.
\end{Note}

%______________________________________________________________________
\subsection*{Procesos Regenerativos: Sigman\cite{Sigman1}}
%______________________________________________________________________
\begin{Def}[Definici\'on Cl\'asica]
Un proceso estoc\'astico $X=\left\{X\left(t\right):t\geq0\right\}$ es llamado regenerativo is existe una variable aleatoria $R_{1}>0$ tal que
\begin{itemize}
\item[i)] $\left\{X\left(t+R_{1}\right):t\geq0\right\}$ es independiente de $\left\{\left\{X\left(t\right):t<R_{1}\right\},\right\}$
\item[ii)] $\left\{X\left(t+R_{1}\right):t\geq0\right\}$ es estoc\'asticamente equivalente a $\left\{X\left(t\right):t>0\right\}$
\end{itemize}

Llamamos a $R_{1}$ tiempo de regeneraci\'on, y decimos que $X$ se regenera en este punto.
\end{Def}

$\left\{X\left(t+R_{1}\right)\right\}$ es regenerativo con tiempo de regeneraci\'on $R_{2}$, independiente de $R_{1}$ pero con la misma distribuci\'on que $R_{1}$. Procediendo de esta manera se obtiene una secuencia de variables aleatorias independientes e id\'enticamente distribuidas $\left\{R_{n}\right\}$ llamados longitudes de ciclo. Si definimos a $Z_{k}\equiv R_{1}+R_{2}+\cdots+R_{k}$, se tiene un proceso de renovaci\'on llamado proceso de renovaci\'on encajado para $X$.




\begin{Def}
Para $x$ fijo y para cada $t\geq0$, sea $I_{x}\left(t\right)=1$ si $X\left(t\right)\leq x$,  $I_{x}\left(t\right)=0$ en caso contrario, y def\'inanse los tiempos promedio
\begin{eqnarray*}
\overline{X}&=&lim_{t\rightarrow\infty}\frac{1}{t}\int_{0}^{\infty}X\left(u\right)du\\
\prob\left(X_{\infty}\leq x\right)&=&lim_{t\rightarrow\infty}\frac{1}{t}\int_{0}^{\infty}I_{x}\left(u\right)du,
\end{eqnarray*}
cuando estos l\'imites existan.
\end{Def}

Como consecuencia del teorema de Renovaci\'on-Recompensa, se tiene que el primer l\'imite  existe y es igual a la constante
\begin{eqnarray*}
\overline{X}&=&\frac{\esp\left[\int_{0}^{R_{1}}X\left(t\right)dt\right]}{\esp\left[R_{1}\right]},
\end{eqnarray*}
suponiendo que ambas esperanzas son finitas.

\begin{Note}
\begin{itemize}
\item[a)] Si el proceso regenerativo $X$ es positivo recurrente y tiene trayectorias muestrales no negativas, entonces la ecuaci\'on anterior es v\'alida.
\item[b)] Si $X$ es positivo recurrente regenerativo, podemos construir una \'unica versi\'on estacionaria de este proceso, $X_{e}=\left\{X_{e}\left(t\right)\right\}$, donde $X_{e}$ es un proceso estoc\'astico regenerativo y estrictamente estacionario, con distribuci\'on marginal distribuida como $X_{\infty}$
\end{itemize}
\end{Note}


%__________________________________________________________________________________________
\subsection{Procesos Regenerativos Estacionarios - Stidham \cite{Stidham}}
%__________________________________________________________________________________________


Un proceso estoc\'astico a tiempo continuo $\left\{V\left(t\right),t\geq0\right\}$ es un proceso regenerativo si existe una sucesi\'on de variables aleatorias independientes e id\'enticamente distribuidas $\left\{X_{1},X_{2},\ldots\right\}$, sucesi\'on de renovaci\'on, tal que para cualquier conjunto de Borel $A$, 

\begin{eqnarray*}
\prob\left\{V\left(t\right)\in A|X_{1}+X_{2}+\cdots+X_{R\left(t\right)}=s,\left\{V\left(\tau\right),\tau<s\right\}\right\}=\prob\left\{V\left(t-s\right)\in A|X_{1}>t-s\right\},
\end{eqnarray*}
para todo $0\leq s\leq t$, donde $R\left(t\right)=\max\left\{X_{1}+X_{2}+\cdots+X_{j}\leq t\right\}=$n\'umero de renovaciones ({\emph{puntos de regeneraci\'on}}) que ocurren en $\left[0,t\right]$. El intervalo $\left[0,X_{1}\right)$ es llamado {\emph{primer ciclo de regeneraci\'on}} de $\left\{V\left(t \right),t\geq0\right\}$, $\left[X_{1},X_{1}+X_{2}\right)$ el {\emph{segundo ciclo de regeneraci\'on}}, y as\'i sucesivamente.

Sea $X=X_{1}$ y sea $F$ la funci\'on de distrbuci\'on de $X$


\begin{Def}
Se define el proceso estacionario, $\left\{V^{*}\left(t\right),t\geq0\right\}$, para $\left\{V\left(t\right),t\geq0\right\}$ por

\begin{eqnarray*}
\prob\left\{V\left(t\right)\in A\right\}=\frac{1}{\esp\left[X\right]}\int_{0}^{\infty}\prob\left\{V\left(t+x\right)\in A|X>x\right\}\left(1-F\left(x\right)\right)dx,
\end{eqnarray*} 
para todo $t\geq0$ y todo conjunto de Borel $A$.
\end{Def}

\begin{Def}
Una distribuci\'on se dice que es {\emph{aritm\'etica}} si todos sus puntos de incremento son m\'ultiplos de la forma $0,\lambda, 2\lambda,\ldots$ para alguna $\lambda>0$ entera.
\end{Def}


\begin{Def}
Una modificaci\'on medible de un proceso $\left\{V\left(t\right),t\geq0\right\}$, es una versi\'on de este, $\left\{V\left(t,w\right)\right\}$ conjuntamente medible para $t\geq0$ y para $w\in S$, $S$ espacio de estados para $\left\{V\left(t\right),t\geq0\right\}$.
\end{Def}

\begin{Teo}
Sea $\left\{V\left(t\right),t\geq\right\}$ un proceso regenerativo no negativo con modificaci\'on medible. Sea $\esp\left[X\right]<\infty$. Entonces el proceso estacionario dado por la ecuaci\'on anterior est\'a bien definido y tiene funci\'on de distribuci\'on independiente de $t$, adem\'as
\begin{itemize}
\item[i)] \begin{eqnarray*}
\esp\left[V^{*}\left(0\right)\right]&=&\frac{\esp\left[\int_{0}^{X}V\left(s\right)ds\right]}{\esp\left[X\right]}\end{eqnarray*}
\item[ii)] Si $\esp\left[V^{*}\left(0\right)\right]<\infty$, equivalentemente, si $\esp\left[\int_{0}^{X}V\left(s\right)ds\right]<\infty$,entonces
\begin{eqnarray*}
\frac{\int_{0}^{t}V\left(s\right)ds}{t}\rightarrow\frac{\esp\left[\int_{0}^{X}V\left(s\right)ds\right]}{\esp\left[X\right]}
\end{eqnarray*}
con probabilidad 1 y en media, cuando $t\rightarrow\infty$.
\end{itemize}
\end{Teo}

%________________________________________________________________________
\subsection{Procesos Regenerativos}
%________________________________________________________________________

Para $\left\{X\left(t\right):t\geq0\right\}$ Proceso Estoc\'astico a tiempo continuo con estado de espacios $S$, que es un espacio m\'etrico, con trayectorias continuas por la derecha y con l\'imites por la izquierda c.s. Sea $N\left(t\right)$ un proceso de renovaci\'on en $\rea_{+}$ definido en el mismo espacio de probabilidad que $X\left(t\right)$, con tiempos de renovaci\'on $T$ y tiempos de inter-renovaci\'on $\xi_{n}=T_{n}-T_{n-1}$, con misma distribuci\'on $F$ de media finita $\mu$.



\begin{Def}
Para el proceso $\left\{\left(N\left(t\right),X\left(t\right)\right):t\geq0\right\}$, sus trayectoria muestrales en el intervalo de tiempo $\left[T_{n-1},T_{n}\right)$ est\'an descritas por
\begin{eqnarray*}
\zeta_{n}=\left(\xi_{n},\left\{X\left(T_{n-1}+t\right):0\leq t<\xi_{n}\right\}\right)
\end{eqnarray*}
Este $\zeta_{n}$ es el $n$-\'esimo segmento del proceso. El proceso es regenerativo sobre los tiempos $T_{n}$ si sus segmentos $\zeta_{n}$ son independientes e id\'enticamennte distribuidos.
\end{Def}


\begin{Obs}
Si $\tilde{X}\left(t\right)$ con espacio de estados $\tilde{S}$ es regenerativo sobre $T_{n}$, entonces $X\left(t\right)=f\left(\tilde{X}\left(t\right)\right)$ tambi\'en es regenerativo sobre $T_{n}$, para cualquier funci\'on $f:\tilde{S}\rightarrow S$.
\end{Obs}

\begin{Obs}
Los procesos regenerativos son crudamente regenerativos, pero no al rev\'es.
\end{Obs}

\begin{Def}[Definici\'on Cl\'asica]
Un proceso estoc\'astico $X=\left\{X\left(t\right):t\geq0\right\}$ es llamado regenerativo is existe una variable aleatoria $R_{1}>0$ tal que
\begin{itemize}
\item[i)] $\left\{X\left(t+R_{1}\right):t\geq0\right\}$ es independiente de $\left\{\left\{X\left(t\right):t<R_{1}\right\},\right\}$
\item[ii)] $\left\{X\left(t+R_{1}\right):t\geq0\right\}$ es estoc\'asticamente equivalente a $\left\{X\left(t\right):t>0\right\}$
\end{itemize}

Llamamos a $R_{1}$ tiempo de regeneraci\'on, y decimos que $X$ se regenera en este punto.
\end{Def}

$\left\{X\left(t+R_{1}\right)\right\}$ es regenerativo con tiempo de regeneraci\'on $R_{2}$, independiente de $R_{1}$ pero con la misma distribuci\'on que $R_{1}$. Procediendo de esta manera se obtiene una secuencia de variables aleatorias independientes e id\'enticamente distribuidas $\left\{R_{n}\right\}$ llamados longitudes de ciclo. Si definimos a $Z_{k}\equiv R_{1}+R_{2}+\cdots+R_{k}$, se tiene un proceso de renovaci\'on llamado proceso de renovaci\'on encajado para $X$.

\begin{Note}
Un proceso regenerativo con media de la longitud de ciclo finita es llamado positivo recurrente.
\end{Note}


\begin{Def}
Para $x$ fijo y para cada $t\geq0$, sea $I_{x}\left(t\right)=1$ si $X\left(t\right)\leq x$,  $I_{x}\left(t\right)=0$ en caso contrario, y def\'inanse los tiempos promedio
\begin{eqnarray*}
\overline{X}&=&lim_{t\rightarrow\infty}\frac{1}{t}\int_{0}^{\infty}X\left(u\right)du\\
\prob\left(X_{\infty}\leq x\right)&=&lim_{t\rightarrow\infty}\frac{1}{t}\int_{0}^{\infty}I_{x}\left(u\right)du,
\end{eqnarray*}
cuando estos l\'imites existan.
\end{Def}

Como consecuencia del teorema de Renovaci\'on-Recompensa, se tiene que el primer l\'imite  existe y es igual a la constante
\begin{eqnarray*}
\overline{X}&=&\frac{\esp\left[\int_{0}^{R_{1}}X\left(t\right)dt\right]}{\esp\left[R_{1}\right]},
\end{eqnarray*}
suponiendo que ambas esperanzas son finitas.

\begin{Note}
\begin{itemize}
\item[a)] Si el proceso regenerativo $X$ es positivo recurrente y tiene trayectorias muestrales no negativas, entonces la ecuaci\'on anterior es v\'alida.
\item[b)] Si $X$ es positivo recurrente regenerativo, podemos construir una \'unica versi\'on estacionaria de este proceso, $X_{e}=\left\{X_{e}\left(t\right)\right\}$, donde $X_{e}$ es un proceso estoc\'astico regenerativo y estrictamente estacionario, con distribuci\'on marginal distribuida como $X_{\infty}$
\end{itemize}
\end{Note}


%______________________________________________________________________
\subsection{Procesos Regenerativos Estacionarios: Visi\'on cl\'asica}
%______________________________________________________________________

\begin{Def}\label{Def.Tn}
Sean $0\leq T_{1}\leq T_{2}\leq \ldots$ son tiempos aleatorios infinitos en los cuales ocurren ciertos eventos. El n\'umero de tiempos $T_{n}$ en el intervalo $\left[0,t\right)$ es

\begin{eqnarray}
N\left(t\right)=\sum_{n=1}^{\infty}\indora\left(T_{n}\leq t\right),
\end{eqnarray}
para $t\geq0$.
\end{Def}

Si se consideran los puntos $T_{n}$ como elementos de $\rea_{+}$, y $N\left(t\right)$ es el n\'umero de puntos en $\rea$. El proceso denotado por $\left\{N\left(t\right):t\geq0\right\}$, denotado por $N\left(t\right)$, es un proceso puntual en $\rea_{+}$. Los $T_{n}$ son los tiempos de ocurrencia, el proceso puntual $N\left(t\right)$ es simple si su n\'umero de ocurrencias son distintas: $0<T_{1}<T_{2}<\ldots$ casi seguramente.

\begin{Def}
Un proceso puntual $N\left(t\right)$ es un proceso de renovaci\'on si los tiempos de interocurrencia $\xi_{n}=T_{n}-T_{n-1}$, para $n\geq1$, son independientes e identicamente distribuidos con distribuci\'on $F$, donde $F\left(0\right)=0$ y $T_{0}=0$. Los $T_{n}$ son llamados tiempos de renovaci\'on, referente a la independencia o renovaci\'on de la informaci\'on estoc\'astica en estos tiempos. Los $\xi_{n}$ son los tiempos de inter-renovaci\'on, y $N\left(t\right)$ es el n\'umero de renovaciones en el intervalo $\left[0,t\right)$
\end{Def}


\begin{Note}
Para definir un proceso de renovaci\'on para cualquier contexto, solamente hay que especificar una distribuci\'on $F$, con $F\left(0\right)=0$, para los tiempos de inter-renovaci\'on. La funci\'on $F$ en turno degune las otra variables aleatorias. De manera formal, existe un espacio de probabilidad y una sucesi\'on de variables aleatorias $\xi_{1},\xi_{2},\ldots$ definidas en este con distribuci\'on $F$. Entonces las otras cantidades son $T_{n}=\sum_{k=1}^{n}\xi_{k}$ y $N\left(t\right)=\sum_{n=1}^{\infty}\indora\left(T_{n}\leq t\right)$, donde $T_{n}\rightarrow\infty$ casi seguramente por la Ley Fuerte de los Grandes Números.
\end{Note}

%___________________________________________________________________________________________
%
\subsection{Teorema Principal de Renovaci\'on}
%___________________________________________________________________________________________
%

\begin{Note} Una funci\'on $h:\rea_{+}\rightarrow\rea$ es Directamente Riemann Integrable en los siguientes casos:
\begin{itemize}
\item[a)] $h\left(t\right)\geq0$ es decreciente y Riemann Integrable.
\item[b)] $h$ es continua excepto posiblemente en un conjunto de Lebesgue de medida 0, y $|h\left(t\right)|\leq b\left(t\right)$, donde $b$ es DRI.
\end{itemize}
\end{Note}

\begin{Teo}[Teorema Principal de Renovaci\'on]
Si $F$ es no aritm\'etica y $h\left(t\right)$ es Directamente Riemann Integrable (DRI), entonces

\begin{eqnarray*}
lim_{t\rightarrow\infty}U\star h=\frac{1}{\mu}\int_{\rea_{+}}h\left(s\right)ds.
\end{eqnarray*}
\end{Teo}

\begin{Prop}
Cualquier funci\'on $H\left(t\right)$ acotada en intervalos finitos y que es 0 para $t<0$ puede expresarse como
\begin{eqnarray*}
H\left(t\right)=U\star h\left(t\right)\textrm{,  donde }h\left(t\right)=H\left(t\right)-F\star H\left(t\right)
\end{eqnarray*}
\end{Prop}

\begin{Def}
Un proceso estoc\'astico $X\left(t\right)$ es crudamente regenerativo en un tiempo aleatorio positivo $T$ si
\begin{eqnarray*}
\esp\left[X\left(T+t\right)|T\right]=\esp\left[X\left(t\right)\right]\textrm{, para }t\geq0,\end{eqnarray*}
y con las esperanzas anteriores finitas.
\end{Def}

\begin{Prop}
Sup\'ongase que $X\left(t\right)$ es un proceso crudamente regenerativo en $T$, que tiene distribuci\'on $F$. Si $\esp\left[X\left(t\right)\right]$ es acotado en intervalos finitos, entonces
\begin{eqnarray*}
\esp\left[X\left(t\right)\right]=U\star h\left(t\right)\textrm{,  donde }h\left(t\right)=\esp\left[X\left(t\right)\indora\left(T>t\right)\right].
\end{eqnarray*}
\end{Prop}

\begin{Teo}[Regeneraci\'on Cruda]
Sup\'ongase que $X\left(t\right)$ es un proceso con valores positivo crudamente regenerativo en $T$, y def\'inase $M=\sup\left\{|X\left(t\right)|:t\leq T\right\}$. Si $T$ es no aritm\'etico y $M$ y $MT$ tienen media finita, entonces
\begin{eqnarray*}
lim_{t\rightarrow\infty}\esp\left[X\left(t\right)\right]=\frac{1}{\mu}\int_{\rea_{+}}h\left(s\right)ds,
\end{eqnarray*}
donde $h\left(t\right)=\esp\left[X\left(t\right)\indora\left(T>t\right)\right]$.
\end{Teo}

%___________________________________________________________________________________________
%
\subsection{Propiedades de los Procesos de Renovaci\'on}
%___________________________________________________________________________________________
%

Los tiempos $T_{n}$ est\'an relacionados con los conteos de $N\left(t\right)$ por

\begin{eqnarray*}
\left\{N\left(t\right)\geq n\right\}&=&\left\{T_{n}\leq t\right\}\\
T_{N\left(t\right)}\leq &t&<T_{N\left(t\right)+1},
\end{eqnarray*}

adem\'as $N\left(T_{n}\right)=n$, y 

\begin{eqnarray*}
N\left(t\right)=\max\left\{n:T_{n}\leq t\right\}=\min\left\{n:T_{n+1}>t\right\}
\end{eqnarray*}

Por propiedades de la convoluci\'on se sabe que

\begin{eqnarray*}
P\left\{T_{n}\leq t\right\}=F^{n\star}\left(t\right)
\end{eqnarray*}
que es la $n$-\'esima convoluci\'on de $F$. Entonces 

\begin{eqnarray*}
\left\{N\left(t\right)\geq n\right\}&=&\left\{T_{n}\leq t\right\}\\
P\left\{N\left(t\right)\leq n\right\}&=&1-F^{\left(n+1\right)\star}\left(t\right)
\end{eqnarray*}

Adem\'as usando el hecho de que $\esp\left[N\left(t\right)\right]=\sum_{n=1}^{\infty}P\left\{N\left(t\right)\geq n\right\}$
se tiene que

\begin{eqnarray*}
\esp\left[N\left(t\right)\right]=\sum_{n=1}^{\infty}F^{n\star}\left(t\right)
\end{eqnarray*}

\begin{Prop}
Para cada $t\geq0$, la funci\'on generadora de momentos $\esp\left[e^{\alpha N\left(t\right)}\right]$ existe para alguna $\alpha$ en una vecindad del 0, y de aqu\'i que $\esp\left[N\left(t\right)^{m}\right]<\infty$, para $m\geq1$.
\end{Prop}


\begin{Note}
Si el primer tiempo de renovaci\'on $\xi_{1}$ no tiene la misma distribuci\'on que el resto de las $\xi_{n}$, para $n\geq2$, a $N\left(t\right)$ se le llama Proceso de Renovaci\'on retardado, donde si $\xi$ tiene distribuci\'on $G$, entonces el tiempo $T_{n}$ de la $n$-\'esima renovaci\'on tiene distribuci\'on $G\star F^{\left(n-1\right)\star}\left(t\right)$
\end{Note}


\begin{Teo}
Para una constante $\mu\leq\infty$ ( o variable aleatoria), las siguientes expresiones son equivalentes:

\begin{eqnarray}
lim_{n\rightarrow\infty}n^{-1}T_{n}&=&\mu,\textrm{ c.s.}\\
lim_{t\rightarrow\infty}t^{-1}N\left(t\right)&=&1/\mu,\textrm{ c.s.}
\end{eqnarray}
\end{Teo}


Es decir, $T_{n}$ satisface la Ley Fuerte de los Grandes N\'umeros s\'i y s\'olo s\'i $N\left/t\right)$ la cumple.


\begin{Coro}[Ley Fuerte de los Grandes N\'umeros para Procesos de Renovaci\'on]
Si $N\left(t\right)$ es un proceso de renovaci\'on cuyos tiempos de inter-renovaci\'on tienen media $\mu\leq\infty$, entonces
\begin{eqnarray}
t^{-1}N\left(t\right)\rightarrow 1/\mu,\textrm{ c.s. cuando }t\rightarrow\infty.
\end{eqnarray}

\end{Coro}


Considerar el proceso estoc\'astico de valores reales $\left\{Z\left(t\right):t\geq0\right\}$ en el mismo espacio de probabilidad que $N\left(t\right)$

\begin{Def}
Para el proceso $\left\{Z\left(t\right):t\geq0\right\}$ se define la fluctuaci\'on m\'axima de $Z\left(t\right)$ en el intervalo $\left(T_{n-1},T_{n}\right]$:
\begin{eqnarray*}
M_{n}=\sup_{T_{n-1}<t\leq T_{n}}|Z\left(t\right)-Z\left(T_{n-1}\right)|
\end{eqnarray*}
\end{Def}

\begin{Teo}
Sup\'ongase que $n^{-1}T_{n}\rightarrow\mu$ c.s. cuando $n\rightarrow\infty$, donde $\mu\leq\infty$ es una constante o variable aleatoria. Sea $a$ una constante o variable aleatoria que puede ser infinita cuando $\mu$ es finita, y considere las expresiones l\'imite:
\begin{eqnarray}
lim_{n\rightarrow\infty}n^{-1}Z\left(T_{n}\right)&=&a,\textrm{ c.s.}\\
lim_{t\rightarrow\infty}t^{-1}Z\left(t\right)&=&a/\mu,\textrm{ c.s.}
\end{eqnarray}
La segunda expresi\'on implica la primera. Conversamente, la primera implica la segunda si el proceso $Z\left(t\right)$ es creciente, o si $lim_{n\rightarrow\infty}n^{-1}M_{n}=0$ c.s.
\end{Teo}

\begin{Coro}
Si $N\left(t\right)$ es un proceso de renovaci\'on, y $\left(Z\left(T_{n}\right)-Z\left(T_{n-1}\right),M_{n}\right)$, para $n\geq1$, son variables aleatorias independientes e id\'enticamente distribuidas con media finita, entonces,
\begin{eqnarray}
lim_{t\rightarrow\infty}t^{-1}Z\left(t\right)\rightarrow\frac{\esp\left[Z\left(T_{1}\right)-Z\left(T_{0}\right)\right]}{\esp\left[T_{1}\right]},\textrm{ c.s. cuando  }t\rightarrow\infty.
\end{eqnarray}
\end{Coro}

%___________________________________________________________________________________________
%
\subsection{Funci\'on de Renovaci\'on}
%___________________________________________________________________________________________
%


\begin{Def}
Sea $h\left(t\right)$ funci\'on de valores reales en $\rea$ acotada en intervalos finitos e igual a cero para $t<0$ La ecuaci\'on de renovaci\'on para $h\left(t\right)$ y la distribuci\'on $F$ es

\begin{eqnarray}\label{Ec.Renovacion}
H\left(t\right)=h\left(t\right)+\int_{\left[0,t\right]}H\left(t-s\right)dF\left(s\right)\textrm{,    }t\geq0,
\end{eqnarray}
donde $H\left(t\right)$ es una funci\'on de valores reales. Esto es $H=h+F\star H$. Decimos que $H\left(t\right)$ es soluci\'on de esta ecuaci\'on si satisface la ecuaci\'on, y es acotada en intervalos finitos e iguales a cero para $t<0$.
\end{Def}

\begin{Prop}
La funci\'on $U\star h\left(t\right)$ es la \'unica soluci\'on de la ecuaci\'on de renovaci\'on (\ref{Ec.Renovacion}).
\end{Prop}

\begin{Teo}[Teorema Renovaci\'on Elemental]
\begin{eqnarray*}
t^{-1}U\left(t\right)\rightarrow 1/\mu\textrm{,    cuando }t\rightarrow\infty.
\end{eqnarray*}
\end{Teo}

%______________________________________________________________________
\subsection{Procesos de Renovaci\'on}
%______________________________________________________________________

\begin{Def}\label{Def.Tn}
Sean $0\leq T_{1}\leq T_{2}\leq \ldots$ son tiempos aleatorios infinitos en los cuales ocurren ciertos eventos. El n\'umero de tiempos $T_{n}$ en el intervalo $\left[0,t\right)$ es

\begin{eqnarray}
N\left(t\right)=\sum_{n=1}^{\infty}\indora\left(T_{n}\leq t\right),
\end{eqnarray}
para $t\geq0$.
\end{Def}

Si se consideran los puntos $T_{n}$ como elementos de $\rea_{+}$, y $N\left(t\right)$ es el n\'umero de puntos en $\rea$. El proceso denotado por $\left\{N\left(t\right):t\geq0\right\}$, denotado por $N\left(t\right)$, es un proceso puntual en $\rea_{+}$. Los $T_{n}$ son los tiempos de ocurrencia, el proceso puntual $N\left(t\right)$ es simple si su n\'umero de ocurrencias son distintas: $0<T_{1}<T_{2}<\ldots$ casi seguramente.

\begin{Def}
Un proceso puntual $N\left(t\right)$ es un proceso de renovaci\'on si los tiempos de interocurrencia $\xi_{n}=T_{n}-T_{n-1}$, para $n\geq1$, son independientes e identicamente distribuidos con distribuci\'on $F$, donde $F\left(0\right)=0$ y $T_{0}=0$. Los $T_{n}$ son llamados tiempos de renovaci\'on, referente a la independencia o renovaci\'on de la informaci\'on estoc\'astica en estos tiempos. Los $\xi_{n}$ son los tiempos de inter-renovaci\'on, y $N\left(t\right)$ es el n\'umero de renovaciones en el intervalo $\left[0,t\right)$
\end{Def}


\begin{Note}
Para definir un proceso de renovaci\'on para cualquier contexto, solamente hay que especificar una distribuci\'on $F$, con $F\left(0\right)=0$, para los tiempos de inter-renovaci\'on. La funci\'on $F$ en turno degune las otra variables aleatorias. De manera formal, existe un espacio de probabilidad y una sucesi\'on de variables aleatorias $\xi_{1},\xi_{2},\ldots$ definidas en este con distribuci\'on $F$. Entonces las otras cantidades son $T_{n}=\sum_{k=1}^{n}\xi_{k}$ y $N\left(t\right)=\sum_{n=1}^{\infty}\indora\left(T_{n}\leq t\right)$, donde $T_{n}\rightarrow\infty$ casi seguramente por la Ley Fuerte de los Grandes Números.
\end{Note}

%___________________________________________________________________________________________
%
\subsection{Renewal and Regenerative Processes: Serfozo\cite{Serfozo}}
%___________________________________________________________________________________________
%
\begin{Def}\label{Def.Tn}
Sean $0\leq T_{1}\leq T_{2}\leq \ldots$ son tiempos aleatorios infinitos en los cuales ocurren ciertos eventos. El n\'umero de tiempos $T_{n}$ en el intervalo $\left[0,t\right)$ es

\begin{eqnarray}
N\left(t\right)=\sum_{n=1}^{\infty}\indora\left(T_{n}\leq t\right),
\end{eqnarray}
para $t\geq0$.
\end{Def}

Si se consideran los puntos $T_{n}$ como elementos de $\rea_{+}$, y $N\left(t\right)$ es el n\'umero de puntos en $\rea$. El proceso denotado por $\left\{N\left(t\right):t\geq0\right\}$, denotado por $N\left(t\right)$, es un proceso puntual en $\rea_{+}$. Los $T_{n}$ son los tiempos de ocurrencia, el proceso puntual $N\left(t\right)$ es simple si su n\'umero de ocurrencias son distintas: $0<T_{1}<T_{2}<\ldots$ casi seguramente.

\begin{Def}
Un proceso puntual $N\left(t\right)$ es un proceso de renovaci\'on si los tiempos de interocurrencia $\xi_{n}=T_{n}-T_{n-1}$, para $n\geq1$, son independientes e identicamente distribuidos con distribuci\'on $F$, donde $F\left(0\right)=0$ y $T_{0}=0$. Los $T_{n}$ son llamados tiempos de renovaci\'on, referente a la independencia o renovaci\'on de la informaci\'on estoc\'astica en estos tiempos. Los $\xi_{n}$ son los tiempos de inter-renovaci\'on, y $N\left(t\right)$ es el n\'umero de renovaciones en el intervalo $\left[0,t\right)$
\end{Def}


\begin{Note}
Para definir un proceso de renovaci\'on para cualquier contexto, solamente hay que especificar una distribuci\'on $F$, con $F\left(0\right)=0$, para los tiempos de inter-renovaci\'on. La funci\'on $F$ en turno degune las otra variables aleatorias. De manera formal, existe un espacio de probabilidad y una sucesi\'on de variables aleatorias $\xi_{1},\xi_{2},\ldots$ definidas en este con distribuci\'on $F$. Entonces las otras cantidades son $T_{n}=\sum_{k=1}^{n}\xi_{k}$ y $N\left(t\right)=\sum_{n=1}^{\infty}\indora\left(T_{n}\leq t\right)$, donde $T_{n}\rightarrow\infty$ casi seguramente por la Ley Fuerte de los Grandes N\'umeros.
\end{Note}







Los tiempos $T_{n}$ est\'an relacionados con los conteos de $N\left(t\right)$ por

\begin{eqnarray*}
\left\{N\left(t\right)\geq n\right\}&=&\left\{T_{n}\leq t\right\}\\
T_{N\left(t\right)}\leq &t&<T_{N\left(t\right)+1},
\end{eqnarray*}

adem\'as $N\left(T_{n}\right)=n$, y 

\begin{eqnarray*}
N\left(t\right)=\max\left\{n:T_{n}\leq t\right\}=\min\left\{n:T_{n+1}>t\right\}
\end{eqnarray*}

Por propiedades de la convoluci\'on se sabe que

\begin{eqnarray*}
P\left\{T_{n}\leq t\right\}=F^{n\star}\left(t\right)
\end{eqnarray*}
que es la $n$-\'esima convoluci\'on de $F$. Entonces 

\begin{eqnarray*}
\left\{N\left(t\right)\geq n\right\}&=&\left\{T_{n}\leq t\right\}\\
P\left\{N\left(t\right)\leq n\right\}&=&1-F^{\left(n+1\right)\star}\left(t\right)
\end{eqnarray*}

Adem\'as usando el hecho de que $\esp\left[N\left(t\right)\right]=\sum_{n=1}^{\infty}P\left\{N\left(t\right)\geq n\right\}$
se tiene que

\begin{eqnarray*}
\esp\left[N\left(t\right)\right]=\sum_{n=1}^{\infty}F^{n\star}\left(t\right)
\end{eqnarray*}

\begin{Prop}
Para cada $t\geq0$, la funci\'on generadora de momentos $\esp\left[e^{\alpha N\left(t\right)}\right]$ existe para alguna $\alpha$ en una vecindad del 0, y de aqu\'i que $\esp\left[N\left(t\right)^{m}\right]<\infty$, para $m\geq1$.
\end{Prop}

\begin{Ejem}[\textbf{Proceso Poisson}]

Suponga que se tienen tiempos de inter-renovaci\'on \textit{i.i.d.} del proceso de renovaci\'on $N\left(t\right)$ tienen distribuci\'on exponencial $F\left(t\right)=q-e^{-\lambda t}$ con tasa $\lambda$. Entonces $N\left(t\right)$ es un proceso Poisson con tasa $\lambda$.

\end{Ejem}


\begin{Note}
Si el primer tiempo de renovaci\'on $\xi_{1}$ no tiene la misma distribuci\'on que el resto de las $\xi_{n}$, para $n\geq2$, a $N\left(t\right)$ se le llama Proceso de Renovaci\'on retardado, donde si $\xi$ tiene distribuci\'on $G$, entonces el tiempo $T_{n}$ de la $n$-\'esima renovaci\'on tiene distribuci\'on $G\star F^{\left(n-1\right)\star}\left(t\right)$
\end{Note}


\begin{Teo}
Para una constante $\mu\leq\infty$ ( o variable aleatoria), las siguientes expresiones son equivalentes:

\begin{eqnarray}
lim_{n\rightarrow\infty}n^{-1}T_{n}&=&\mu,\textrm{ c.s.}\\
lim_{t\rightarrow\infty}t^{-1}N\left(t\right)&=&1/\mu,\textrm{ c.s.}
\end{eqnarray}
\end{Teo}


Es decir, $T_{n}$ satisface la Ley Fuerte de los Grandes N\'umeros s\'i y s\'olo s\'i $N\left/t\right)$ la cumple.


\begin{Coro}[Ley Fuerte de los Grandes N\'umeros para Procesos de Renovaci\'on]
Si $N\left(t\right)$ es un proceso de renovaci\'on cuyos tiempos de inter-renovaci\'on tienen media $\mu\leq\infty$, entonces
\begin{eqnarray}
t^{-1}N\left(t\right)\rightarrow 1/\mu,\textrm{ c.s. cuando }t\rightarrow\infty.
\end{eqnarray}

\end{Coro}


Considerar el proceso estoc\'astico de valores reales $\left\{Z\left(t\right):t\geq0\right\}$ en el mismo espacio de probabilidad que $N\left(t\right)$

\begin{Def}
Para el proceso $\left\{Z\left(t\right):t\geq0\right\}$ se define la fluctuaci\'on m\'axima de $Z\left(t\right)$ en el intervalo $\left(T_{n-1},T_{n}\right]$:
\begin{eqnarray*}
M_{n}=\sup_{T_{n-1}<t\leq T_{n}}|Z\left(t\right)-Z\left(T_{n-1}\right)|
\end{eqnarray*}
\end{Def}

\begin{Teo}
Sup\'ongase que $n^{-1}T_{n}\rightarrow\mu$ c.s. cuando $n\rightarrow\infty$, donde $\mu\leq\infty$ es una constante o variable aleatoria. Sea $a$ una constante o variable aleatoria que puede ser infinita cuando $\mu$ es finita, y considere las expresiones l\'imite:
\begin{eqnarray}
lim_{n\rightarrow\infty}n^{-1}Z\left(T_{n}\right)&=&a,\textrm{ c.s.}\\
lim_{t\rightarrow\infty}t^{-1}Z\left(t\right)&=&a/\mu,\textrm{ c.s.}
\end{eqnarray}
La segunda expresi\'on implica la primera. Conversamente, la primera implica la segunda si el proceso $Z\left(t\right)$ es creciente, o si $lim_{n\rightarrow\infty}n^{-1}M_{n}=0$ c.s.
\end{Teo}

\begin{Coro}
Si $N\left(t\right)$ es un proceso de renovaci\'on, y $\left(Z\left(T_{n}\right)-Z\left(T_{n-1}\right),M_{n}\right)$, para $n\geq1$, son variables aleatorias independientes e id\'enticamente distribuidas con media finita, entonces,
\begin{eqnarray}
lim_{t\rightarrow\infty}t^{-1}Z\left(t\right)\rightarrow\frac{\esp\left[Z\left(T_{1}\right)-Z\left(T_{0}\right)\right]}{\esp\left[T_{1}\right]},\textrm{ c.s. cuando  }t\rightarrow\infty.
\end{eqnarray}
\end{Coro}


Sup\'ongase que $N\left(t\right)$ es un proceso de renovaci\'on con distribuci\'on $F$ con media finita $\mu$.

\begin{Def}
La funci\'on de renovaci\'on asociada con la distribuci\'on $F$, del proceso $N\left(t\right)$, es
\begin{eqnarray*}
U\left(t\right)=\sum_{n=1}^{\infty}F^{n\star}\left(t\right),\textrm{   }t\geq0,
\end{eqnarray*}
donde $F^{0\star}\left(t\right)=\indora\left(t\geq0\right)$.
\end{Def}


\begin{Prop}
Sup\'ongase que la distribuci\'on de inter-renovaci\'on $F$ tiene densidad $f$. Entonces $U\left(t\right)$ tambi\'en tiene densidad, para $t>0$, y es $U^{'}\left(t\right)=\sum_{n=0}^{\infty}f^{n\star}\left(t\right)$. Adem\'as
\begin{eqnarray*}
\prob\left\{N\left(t\right)>N\left(t-\right)\right\}=0\textrm{,   }t\geq0.
\end{eqnarray*}
\end{Prop}

\begin{Def}
La Transformada de Laplace-Stieljes de $F$ est\'a dada por

\begin{eqnarray*}
\hat{F}\left(\alpha\right)=\int_{\rea_{+}}e^{-\alpha t}dF\left(t\right)\textrm{,  }\alpha\geq0.
\end{eqnarray*}
\end{Def}

Entonces

\begin{eqnarray*}
\hat{U}\left(\alpha\right)=\sum_{n=0}^{\infty}\hat{F^{n\star}}\left(\alpha\right)=\sum_{n=0}^{\infty}\hat{F}\left(\alpha\right)^{n}=\frac{1}{1-\hat{F}\left(\alpha\right)}.
\end{eqnarray*}


\begin{Prop}
La Transformada de Laplace $\hat{U}\left(\alpha\right)$ y $\hat{F}\left(\alpha\right)$ determina una a la otra de manera \'unica por la relaci\'on $\hat{U}\left(\alpha\right)=\frac{1}{1-\hat{F}\left(\alpha\right)}$.
\end{Prop}


\begin{Note}
Un proceso de renovaci\'on $N\left(t\right)$ cuyos tiempos de inter-renovaci\'on tienen media finita, es un proceso Poisson con tasa $\lambda$ si y s\'olo s\'i $\esp\left[U\left(t\right)\right]=\lambda t$, para $t\geq0$.
\end{Note}


\begin{Teo}
Sea $N\left(t\right)$ un proceso puntual simple con puntos de localizaci\'on $T_{n}$ tal que $\eta\left(t\right)=\esp\left[N\left(\right)\right]$ es finita para cada $t$. Entonces para cualquier funci\'on $f:\rea_{+}\rightarrow\rea$,
\begin{eqnarray*}
\esp\left[\sum_{n=1}^{N\left(\right)}f\left(T_{n}\right)\right]=\int_{\left(0,t\right]}f\left(s\right)d\eta\left(s\right)\textrm{,  }t\geq0,
\end{eqnarray*}
suponiendo que la integral exista. Adem\'as si $X_{1},X_{2},\ldots$ son variables aleatorias definidas en el mismo espacio de probabilidad que el proceso $N\left(t\right)$ tal que $\esp\left[X_{n}|T_{n}=s\right]=f\left(s\right)$, independiente de $n$. Entonces
\begin{eqnarray*}
\esp\left[\sum_{n=1}^{N\left(t\right)}X_{n}\right]=\int_{\left(0,t\right]}f\left(s\right)d\eta\left(s\right)\textrm{,  }t\geq0,
\end{eqnarray*} 
suponiendo que la integral exista. 
\end{Teo}

\begin{Coro}[Identidad de Wald para Renovaciones]
Para el proceso de renovaci\'on $N\left(t\right)$,
\begin{eqnarray*}
\esp\left[T_{N\left(t\right)+1}\right]=\mu\esp\left[N\left(t\right)+1\right]\textrm{,  }t\geq0,
\end{eqnarray*}  
\end{Coro}


\begin{Def}
Sea $h\left(t\right)$ funci\'on de valores reales en $\rea$ acotada en intervalos finitos e igual a cero para $t<0$ La ecuaci\'on de renovaci\'on para $h\left(t\right)$ y la distribuci\'on $F$ es

\begin{eqnarray}\label{Ec.Renovacion}
H\left(t\right)=h\left(t\right)+\int_{\left[0,t\right]}H\left(t-s\right)dF\left(s\right)\textrm{,    }t\geq0,
\end{eqnarray}
donde $H\left(t\right)$ es una funci\'on de valores reales. Esto es $H=h+F\star H$. Decimos que $H\left(t\right)$ es soluci\'on de esta ecuaci\'on si satisface la ecuaci\'on, y es acotada en intervalos finitos e iguales a cero para $t<0$.
\end{Def}

\begin{Prop}
La funci\'on $U\star h\left(t\right)$ es la \'unica soluci\'on de la ecuaci\'on de renovaci\'on (\ref{Ec.Renovacion}).
\end{Prop}

\begin{Teo}[Teorema Renovaci\'on Elemental]
\begin{eqnarray*}
t^{-1}U\left(t\right)\rightarrow 1/\mu\textrm{,    cuando }t\rightarrow\infty.
\end{eqnarray*}
\end{Teo}



Sup\'ongase que $N\left(t\right)$ es un proceso de renovaci\'on con distribuci\'on $F$ con media finita $\mu$.

\begin{Def}
La funci\'on de renovaci\'on asociada con la distribuci\'on $F$, del proceso $N\left(t\right)$, es
\begin{eqnarray*}
U\left(t\right)=\sum_{n=1}^{\infty}F^{n\star}\left(t\right),\textrm{   }t\geq0,
\end{eqnarray*}
donde $F^{0\star}\left(t\right)=\indora\left(t\geq0\right)$.
\end{Def}


\begin{Prop}
Sup\'ongase que la distribuci\'on de inter-renovaci\'on $F$ tiene densidad $f$. Entonces $U\left(t\right)$ tambi\'en tiene densidad, para $t>0$, y es $U^{'}\left(t\right)=\sum_{n=0}^{\infty}f^{n\star}\left(t\right)$. Adem\'as
\begin{eqnarray*}
\prob\left\{N\left(t\right)>N\left(t-\right)\right\}=0\textrm{,   }t\geq0.
\end{eqnarray*}
\end{Prop}

\begin{Def}
La Transformada de Laplace-Stieljes de $F$ est\'a dada por

\begin{eqnarray*}
\hat{F}\left(\alpha\right)=\int_{\rea_{+}}e^{-\alpha t}dF\left(t\right)\textrm{,  }\alpha\geq0.
\end{eqnarray*}
\end{Def}

Entonces

\begin{eqnarray*}
\hat{U}\left(\alpha\right)=\sum_{n=0}^{\infty}\hat{F^{n\star}}\left(\alpha\right)=\sum_{n=0}^{\infty}\hat{F}\left(\alpha\right)^{n}=\frac{1}{1-\hat{F}\left(\alpha\right)}.
\end{eqnarray*}


\begin{Prop}
La Transformada de Laplace $\hat{U}\left(\alpha\right)$ y $\hat{F}\left(\alpha\right)$ determina una a la otra de manera \'unica por la relaci\'on $\hat{U}\left(\alpha\right)=\frac{1}{1-\hat{F}\left(\alpha\right)}$.
\end{Prop}


\begin{Note}
Un proceso de renovaci\'on $N\left(t\right)$ cuyos tiempos de inter-renovaci\'on tienen media finita, es un proceso Poisson con tasa $\lambda$ si y s\'olo s\'i $\esp\left[U\left(t\right)\right]=\lambda t$, para $t\geq0$.
\end{Note}


\begin{Teo}
Sea $N\left(t\right)$ un proceso puntual simple con puntos de localizaci\'on $T_{n}$ tal que $\eta\left(t\right)=\esp\left[N\left(\right)\right]$ es finita para cada $t$. Entonces para cualquier funci\'on $f:\rea_{+}\rightarrow\rea$,
\begin{eqnarray*}
\esp\left[\sum_{n=1}^{N\left(\right)}f\left(T_{n}\right)\right]=\int_{\left(0,t\right]}f\left(s\right)d\eta\left(s\right)\textrm{,  }t\geq0,
\end{eqnarray*}
suponiendo que la integral exista. Adem\'as si $X_{1},X_{2},\ldots$ son variables aleatorias definidas en el mismo espacio de probabilidad que el proceso $N\left(t\right)$ tal que $\esp\left[X_{n}|T_{n}=s\right]=f\left(s\right)$, independiente de $n$. Entonces
\begin{eqnarray*}
\esp\left[\sum_{n=1}^{N\left(t\right)}X_{n}\right]=\int_{\left(0,t\right]}f\left(s\right)d\eta\left(s\right)\textrm{,  }t\geq0,
\end{eqnarray*} 
suponiendo que la integral exista. 
\end{Teo}

\begin{Coro}[Identidad de Wald para Renovaciones]
Para el proceso de renovaci\'on $N\left(t\right)$,
\begin{eqnarray*}
\esp\left[T_{N\left(t\right)+1}\right]=\mu\esp\left[N\left(t\right)+1\right]\textrm{,  }t\geq0,
\end{eqnarray*}  
\end{Coro}


\begin{Def}
Sea $h\left(t\right)$ funci\'on de valores reales en $\rea$ acotada en intervalos finitos e igual a cero para $t<0$ La ecuaci\'on de renovaci\'on para $h\left(t\right)$ y la distribuci\'on $F$ es

\begin{eqnarray}\label{Ec.Renovacion}
H\left(t\right)=h\left(t\right)+\int_{\left[0,t\right]}H\left(t-s\right)dF\left(s\right)\textrm{,    }t\geq0,
\end{eqnarray}
donde $H\left(t\right)$ es una funci\'on de valores reales. Esto es $H=h+F\star H$. Decimos que $H\left(t\right)$ es soluci\'on de esta ecuaci\'on si satisface la ecuaci\'on, y es acotada en intervalos finitos e iguales a cero para $t<0$.
\end{Def}

\begin{Prop}
La funci\'on $U\star h\left(t\right)$ es la \'unica soluci\'on de la ecuaci\'on de renovaci\'on (\ref{Ec.Renovacion}).
\end{Prop}

\begin{Teo}[Teorema Renovaci\'on Elemental]
\begin{eqnarray*}
t^{-1}U\left(t\right)\rightarrow 1/\mu\textrm{,    cuando }t\rightarrow\infty.
\end{eqnarray*}
\end{Teo}


\begin{Note} Una funci\'on $h:\rea_{+}\rightarrow\rea$ es Directamente Riemann Integrable en los siguientes casos:
\begin{itemize}
\item[a)] $h\left(t\right)\geq0$ es decreciente y Riemann Integrable.
\item[b)] $h$ es continua excepto posiblemente en un conjunto de Lebesgue de medida 0, y $|h\left(t\right)|\leq b\left(t\right)$, donde $b$ es DRI.
\end{itemize}
\end{Note}

\begin{Teo}[Teorema Principal de Renovaci\'on]
Si $F$ es no aritm\'etica y $h\left(t\right)$ es Directamente Riemann Integrable (DRI), entonces

\begin{eqnarray*}
lim_{t\rightarrow\infty}U\star h=\frac{1}{\mu}\int_{\rea_{+}}h\left(s\right)ds.
\end{eqnarray*}
\end{Teo}

\begin{Prop}
Cualquier funci\'on $H\left(t\right)$ acotada en intervalos finitos y que es 0 para $t<0$ puede expresarse como
\begin{eqnarray*}
H\left(t\right)=U\star h\left(t\right)\textrm{,  donde }h\left(t\right)=H\left(t\right)-F\star H\left(t\right)
\end{eqnarray*}
\end{Prop}

\begin{Def}
Un proceso estoc\'astico $X\left(t\right)$ es crudamente regenerativo en un tiempo aleatorio positivo $T$ si
\begin{eqnarray*}
\esp\left[X\left(T+t\right)|T\right]=\esp\left[X\left(t\right)\right]\textrm{, para }t\geq0,\end{eqnarray*}
y con las esperanzas anteriores finitas.
\end{Def}

\begin{Prop}
Sup\'ongase que $X\left(t\right)$ es un proceso crudamente regenerativo en $T$, que tiene distribuci\'on $F$. Si $\esp\left[X\left(t\right)\right]$ es acotado en intervalos finitos, entonces
\begin{eqnarray*}
\esp\left[X\left(t\right)\right]=U\star h\left(t\right)\textrm{,  donde }h\left(t\right)=\esp\left[X\left(t\right)\indora\left(T>t\right)\right].
\end{eqnarray*}
\end{Prop}

\begin{Teo}[Regeneraci\'on Cruda]
Sup\'ongase que $X\left(t\right)$ es un proceso con valores positivo crudamente regenerativo en $T$, y def\'inase $M=\sup\left\{|X\left(t\right)|:t\leq T\right\}$. Si $T$ es no aritm\'etico y $M$ y $MT$ tienen media finita, entonces
\begin{eqnarray*}
lim_{t\rightarrow\infty}\esp\left[X\left(t\right)\right]=\frac{1}{\mu}\int_{\rea_{+}}h\left(s\right)ds,
\end{eqnarray*}
donde $h\left(t\right)=\esp\left[X\left(t\right)\indora\left(T>t\right)\right]$.
\end{Teo}


\begin{Note} Una funci\'on $h:\rea_{+}\rightarrow\rea$ es Directamente Riemann Integrable en los siguientes casos:
\begin{itemize}
\item[a)] $h\left(t\right)\geq0$ es decreciente y Riemann Integrable.
\item[b)] $h$ es continua excepto posiblemente en un conjunto de Lebesgue de medida 0, y $|h\left(t\right)|\leq b\left(t\right)$, donde $b$ es DRI.
\end{itemize}
\end{Note}

\begin{Teo}[Teorema Principal de Renovaci\'on]
Si $F$ es no aritm\'etica y $h\left(t\right)$ es Directamente Riemann Integrable (DRI), entonces

\begin{eqnarray*}
lim_{t\rightarrow\infty}U\star h=\frac{1}{\mu}\int_{\rea_{+}}h\left(s\right)ds.
\end{eqnarray*}
\end{Teo}

\begin{Prop}
Cualquier funci\'on $H\left(t\right)$ acotada en intervalos finitos y que es 0 para $t<0$ puede expresarse como
\begin{eqnarray*}
H\left(t\right)=U\star h\left(t\right)\textrm{,  donde }h\left(t\right)=H\left(t\right)-F\star H\left(t\right)
\end{eqnarray*}
\end{Prop}

\begin{Def}
Un proceso estoc\'astico $X\left(t\right)$ es crudamente regenerativo en un tiempo aleatorio positivo $T$ si
\begin{eqnarray*}
\esp\left[X\left(T+t\right)|T\right]=\esp\left[X\left(t\right)\right]\textrm{, para }t\geq0,\end{eqnarray*}
y con las esperanzas anteriores finitas.
\end{Def}

\begin{Prop}
Sup\'ongase que $X\left(t\right)$ es un proceso crudamente regenerativo en $T$, que tiene distribuci\'on $F$. Si $\esp\left[X\left(t\right)\right]$ es acotado en intervalos finitos, entonces
\begin{eqnarray*}
\esp\left[X\left(t\right)\right]=U\star h\left(t\right)\textrm{,  donde }h\left(t\right)=\esp\left[X\left(t\right)\indora\left(T>t\right)\right].
\end{eqnarray*}
\end{Prop}

\begin{Teo}[Regeneraci\'on Cruda]
Sup\'ongase que $X\left(t\right)$ es un proceso con valores positivo crudamente regenerativo en $T$, y def\'inase $M=\sup\left\{|X\left(t\right)|:t\leq T\right\}$. Si $T$ es no aritm\'etico y $M$ y $MT$ tienen media finita, entonces
\begin{eqnarray*}
lim_{t\rightarrow\infty}\esp\left[X\left(t\right)\right]=\frac{1}{\mu}\int_{\rea_{+}}h\left(s\right)ds,
\end{eqnarray*}
donde $h\left(t\right)=\esp\left[X\left(t\right)\indora\left(T>t\right)\right]$.
\end{Teo}

\begin{Def}
Para el proceso $\left\{\left(N\left(t\right),X\left(t\right)\right):t\geq0\right\}$, sus trayectoria muestrales en el intervalo de tiempo $\left[T_{n-1},T_{n}\right)$ est\'an descritas por
\begin{eqnarray*}
\zeta_{n}=\left(\xi_{n},\left\{X\left(T_{n-1}+t\right):0\leq t<\xi_{n}\right\}\right)
\end{eqnarray*}
Este $\zeta_{n}$ es el $n$-\'esimo segmento del proceso. El proceso es regenerativo sobre los tiempos $T_{n}$ si sus segmentos $\zeta_{n}$ son independientes e id\'enticamennte distribuidos.
\end{Def}


\begin{Note}
Si $\tilde{X}\left(t\right)$ con espacio de estados $\tilde{S}$ es regenerativo sobre $T_{n}$, entonces $X\left(t\right)=f\left(\tilde{X}\left(t\right)\right)$ tambi\'en es regenerativo sobre $T_{n}$, para cualquier funci\'on $f:\tilde{S}\rightarrow S$.
\end{Note}

\begin{Note}
Los procesos regenerativos son crudamente regenerativos, pero no al rev\'es.
\end{Note}


\begin{Note}
Un proceso estoc\'astico a tiempo continuo o discreto es regenerativo si existe un proceso de renovaci\'on  tal que los segmentos del proceso entre tiempos de renovaci\'on sucesivos son i.i.d., es decir, para $\left\{X\left(t\right):t\geq0\right\}$ proceso estoc\'astico a tiempo continuo con espacio de estados $S$, espacio m\'etrico.
\end{Note}

Para $\left\{X\left(t\right):t\geq0\right\}$ Proceso Estoc\'astico a tiempo continuo con estado de espacios $S$, que es un espacio m\'etrico, con trayectorias continuas por la derecha y con l\'imites por la izquierda c.s. Sea $N\left(t\right)$ un proceso de renovaci\'on en $\rea_{+}$ definido en el mismo espacio de probabilidad que $X\left(t\right)$, con tiempos de renovaci\'on $T$ y tiempos de inter-renovaci\'on $\xi_{n}=T_{n}-T_{n-1}$, con misma distribuci\'on $F$ de media finita $\mu$.



\begin{Def}
Para el proceso $\left\{\left(N\left(t\right),X\left(t\right)\right):t\geq0\right\}$, sus trayectoria muestrales en el intervalo de tiempo $\left[T_{n-1},T_{n}\right)$ est\'an descritas por
\begin{eqnarray*}
\zeta_{n}=\left(\xi_{n},\left\{X\left(T_{n-1}+t\right):0\leq t<\xi_{n}\right\}\right)
\end{eqnarray*}
Este $\zeta_{n}$ es el $n$-\'esimo segmento del proceso. El proceso es regenerativo sobre los tiempos $T_{n}$ si sus segmentos $\zeta_{n}$ son independientes e id\'enticamennte distribuidos.
\end{Def}

\begin{Note}
Un proceso regenerativo con media de la longitud de ciclo finita es llamado positivo recurrente.
\end{Note}

\begin{Teo}[Procesos Regenerativos]
Suponga que el proceso
\end{Teo}


\begin{Def}[Renewal Process Trinity]
Para un proceso de renovaci\'on $N\left(t\right)$, los siguientes procesos proveen de informaci\'on sobre los tiempos de renovaci\'on.
\begin{itemize}
\item $A\left(t\right)=t-T_{N\left(t\right)}$, el tiempo de recurrencia hacia atr\'as al tiempo $t$, que es el tiempo desde la \'ultima renovaci\'on para $t$.

\item $B\left(t\right)=T_{N\left(t\right)+1}-t$, el tiempo de recurrencia hacia adelante al tiempo $t$, residual del tiempo de renovaci\'on, que es el tiempo para la pr\'oxima renovaci\'on despu\'es de $t$.

\item $L\left(t\right)=\xi_{N\left(t\right)+1}=A\left(t\right)+B\left(t\right)$, la longitud del intervalo de renovaci\'on que contiene a $t$.
\end{itemize}
\end{Def}

\begin{Note}
El proceso tridimensional $\left(A\left(t\right),B\left(t\right),L\left(t\right)\right)$ es regenerativo sobre $T_{n}$, y por ende cada proceso lo es. Cada proceso $A\left(t\right)$ y $B\left(t\right)$ son procesos de MArkov a tiempo continuo con trayectorias continuas por partes en el espacio de estados $\rea_{+}$. Una expresi\'on conveniente para su distribuci\'on conjunta es, para $0\leq x<t,y\geq0$
\begin{equation}\label{NoRenovacion}
P\left\{A\left(t\right)>x,B\left(t\right)>y\right\}=
P\left\{N\left(t+y\right)-N\left((t-x)\right)=0\right\}
\end{equation}
\end{Note}

\begin{Ejem}[Tiempos de recurrencia Poisson]
Si $N\left(t\right)$ es un proceso Poisson con tasa $\lambda$, entonces de la expresi\'on (\ref{NoRenovacion}) se tiene que

\begin{eqnarray*}
\begin{array}{lc}
P\left\{A\left(t\right)>x,B\left(t\right)>y\right\}=e^{-\lambda\left(x+y\right)},&0\leq x<t,y\geq0,
\end{array}
\end{eqnarray*}
que es la probabilidad Poisson de no renovaciones en un intervalo de longitud $x+y$.

\end{Ejem}

\begin{Note}
Una cadena de Markov erg\'odica tiene la propiedad de ser estacionaria si la distribuci\'on de su estado al tiempo $0$ es su distribuci\'on estacionaria.
\end{Note}


\begin{Def}
Un proceso estoc\'astico a tiempo continuo $\left\{X\left(t\right):t\geq0\right\}$ en un espacio general es estacionario si sus distribuciones finito dimensionales son invariantes bajo cualquier  traslado: para cada $0\leq s_{1}<s_{2}<\cdots<s_{k}$ y $t\geq0$,
\begin{eqnarray*}
\left(X\left(s_{1}+t\right),\ldots,X\left(s_{k}+t\right)\right)=_{d}\left(X\left(s_{1}\right),\ldots,X\left(s_{k}\right)\right).
\end{eqnarray*}
\end{Def}

\begin{Note}
Un proceso de Markov es estacionario si $X\left(t\right)=_{d}X\left(0\right)$, $t\geq0$.
\end{Note}

Considerese el proceso $N\left(t\right)=\sum_{n}\indora\left(\tau_{n}\leq t\right)$ en $\rea_{+}$, con puntos $0<\tau_{1}<\tau_{2}<\cdots$.

\begin{Prop}
Si $N$ es un proceso puntual estacionario y $\esp\left[N\left(1\right)\right]<\infty$, entonces $\esp\left[N\left(t\right)\right]=t\esp\left[N\left(1\right)\right]$, $t\geq0$

\end{Prop}

\begin{Teo}
Los siguientes enunciados son equivalentes
\begin{itemize}
\item[i)] El proceso retardado de renovaci\'on $N$ es estacionario.

\item[ii)] EL proceso de tiempos de recurrencia hacia adelante $B\left(t\right)$ es estacionario.


\item[iii)] $\esp\left[N\left(t\right)\right]=t/\mu$,


\item[iv)] $G\left(t\right)=F_{e}\left(t\right)=\frac{1}{\mu}\int_{0}^{t}\left[1-F\left(s\right)\right]ds$
\end{itemize}
Cuando estos enunciados son ciertos, $P\left\{B\left(t\right)\leq x\right\}=F_{e}\left(x\right)$, para $t,x\geq0$.

\end{Teo}

\begin{Note}
Una consecuencia del teorema anterior es que el Proceso Poisson es el \'unico proceso sin retardo que es estacionario.
\end{Note}

\begin{Coro}
El proceso de renovaci\'on $N\left(t\right)$ sin retardo, y cuyos tiempos de inter renonaci\'on tienen media finita, es estacionario si y s\'olo si es un proceso Poisson.

\end{Coro}

%______________________________________________________________________

%\section{Ejemplos, Notas importantes}
%______________________________________________________________________
%\section*{Ap\'endice A}
%__________________________________________________________________

%________________________________________________________________________
%\subsection*{Procesos Regenerativos}
%________________________________________________________________________



\begin{Note}
Si $\tilde{X}\left(t\right)$ con espacio de estados $\tilde{S}$ es regenerativo sobre $T_{n}$, entonces $X\left(t\right)=f\left(\tilde{X}\left(t\right)\right)$ tambi\'en es regenerativo sobre $T_{n}$, para cualquier funci\'on $f:\tilde{S}\rightarrow S$.
\end{Note}

\begin{Note}
Los procesos regenerativos son crudamente regenerativos, pero no al rev\'es.
\end{Note}
%\subsection*{Procesos Regenerativos: Sigman\cite{Sigman1}}
\begin{Def}[Definici\'on Cl\'asica]
Un proceso estoc\'astico $X=\left\{X\left(t\right):t\geq0\right\}$ es llamado regenerativo is existe una variable aleatoria $R_{1}>0$ tal que
\begin{itemize}
\item[i)] $\left\{X\left(t+R_{1}\right):t\geq0\right\}$ es independiente de $\left\{\left\{X\left(t\right):t<R_{1}\right\},\right\}$
\item[ii)] $\left\{X\left(t+R_{1}\right):t\geq0\right\}$ es estoc\'asticamente equivalente a $\left\{X\left(t\right):t>0\right\}$
\end{itemize}

Llamamos a $R_{1}$ tiempo de regeneraci\'on, y decimos que $X$ se regenera en este punto.
\end{Def}

$\left\{X\left(t+R_{1}\right)\right\}$ es regenerativo con tiempo de regeneraci\'on $R_{2}$, independiente de $R_{1}$ pero con la misma distribuci\'on que $R_{1}$. Procediendo de esta manera se obtiene una secuencia de variables aleatorias independientes e id\'enticamente distribuidas $\left\{R_{n}\right\}$ llamados longitudes de ciclo. Si definimos a $Z_{k}\equiv R_{1}+R_{2}+\cdots+R_{k}$, se tiene un proceso de renovaci\'on llamado proceso de renovaci\'on encajado para $X$.




\begin{Def}
Para $x$ fijo y para cada $t\geq0$, sea $I_{x}\left(t\right)=1$ si $X\left(t\right)\leq x$,  $I_{x}\left(t\right)=0$ en caso contrario, y def\'inanse los tiempos promedio
\begin{eqnarray*}
\overline{X}&=&lim_{t\rightarrow\infty}\frac{1}{t}\int_{0}^{\infty}X\left(u\right)du\\
\prob\left(X_{\infty}\leq x\right)&=&lim_{t\rightarrow\infty}\frac{1}{t}\int_{0}^{\infty}I_{x}\left(u\right)du,
\end{eqnarray*}
cuando estos l\'imites existan.
\end{Def}

Como consecuencia del teorema de Renovaci\'on-Recompensa, se tiene que el primer l\'imite  existe y es igual a la constante
\begin{eqnarray*}
\overline{X}&=&\frac{\esp\left[\int_{0}^{R_{1}}X\left(t\right)dt\right]}{\esp\left[R_{1}\right]},
\end{eqnarray*}
suponiendo que ambas esperanzas son finitas.

\begin{Note}
\begin{itemize}
\item[a)] Si el proceso regenerativo $X$ es positivo recurrente y tiene trayectorias muestrales no negativas, entonces la ecuaci\'on anterior es v\'alida.
\item[b)] Si $X$ es positivo recurrente regenerativo, podemos construir una \'unica versi\'on estacionaria de este proceso, $X_{e}=\left\{X_{e}\left(t\right)\right\}$, donde $X_{e}$ es un proceso estoc\'astico regenerativo y estrictamente estacionario, con distribuci\'on marginal distribuida como $X_{\infty}$
\end{itemize}
\end{Note}

Para $\left\{X\left(t\right):t\geq0\right\}$ Proceso Estoc\'astico a tiempo continuo con estado de espacios $S$, que es un espacio m\'etrico, con trayectorias continuas por la derecha y con l\'imites por la izquierda c.s. Sea $N\left(t\right)$ un proceso de renovaci\'on en $\rea_{+}$ definido en el mismo espacio de probabilidad que $X\left(t\right)$, con tiempos de renovaci\'on $T$ y tiempos de inter-renovaci\'on $\xi_{n}=T_{n}-T_{n-1}$, con misma distribuci\'on $F$ de media finita $\mu$.


\begin{Def}
Para el proceso $\left\{\left(N\left(t\right),X\left(t\right)\right):t\geq0\right\}$, sus trayectoria muestrales en el intervalo de tiempo $\left[T_{n-1},T_{n}\right)$ est\'an descritas por
\begin{eqnarray*}
\zeta_{n}=\left(\xi_{n},\left\{X\left(T_{n-1}+t\right):0\leq t<\xi_{n}\right\}\right)
\end{eqnarray*}
Este $\zeta_{n}$ es el $n$-\'esimo segmento del proceso. El proceso es regenerativo sobre los tiempos $T_{n}$ si sus segmentos $\zeta_{n}$ son independientes e id\'enticamennte distribuidos.
\end{Def}


\begin{Note}
Si $\tilde{X}\left(t\right)$ con espacio de estados $\tilde{S}$ es regenerativo sobre $T_{n}$, entonces $X\left(t\right)=f\left(\tilde{X}\left(t\right)\right)$ tambi\'en es regenerativo sobre $T_{n}$, para cualquier funci\'on $f:\tilde{S}\rightarrow S$.
\end{Note}

\begin{Note}
Los procesos regenerativos son crudamente regenerativos, pero no al rev\'es.
\end{Note}

\begin{Def}[Definici\'on Cl\'asica]
Un proceso estoc\'astico $X=\left\{X\left(t\right):t\geq0\right\}$ es llamado regenerativo is existe una variable aleatoria $R_{1}>0$ tal que
\begin{itemize}
\item[i)] $\left\{X\left(t+R_{1}\right):t\geq0\right\}$ es independiente de $\left\{\left\{X\left(t\right):t<R_{1}\right\},\right\}$
\item[ii)] $\left\{X\left(t+R_{1}\right):t\geq0\right\}$ es estoc\'asticamente equivalente a $\left\{X\left(t\right):t>0\right\}$
\end{itemize}

Llamamos a $R_{1}$ tiempo de regeneraci\'on, y decimos que $X$ se regenera en este punto.
\end{Def}

$\left\{X\left(t+R_{1}\right)\right\}$ es regenerativo con tiempo de regeneraci\'on $R_{2}$, independiente de $R_{1}$ pero con la misma distribuci\'on que $R_{1}$. Procediendo de esta manera se obtiene una secuencia de variables aleatorias independientes e id\'enticamente distribuidas $\left\{R_{n}\right\}$ llamados longitudes de ciclo. Si definimos a $Z_{k}\equiv R_{1}+R_{2}+\cdots+R_{k}$, se tiene un proceso de renovaci\'on llamado proceso de renovaci\'on encajado para $X$.

\begin{Note}
Un proceso regenerativo con media de la longitud de ciclo finita es llamado positivo recurrente.
\end{Note}


\begin{Def}
Para $x$ fijo y para cada $t\geq0$, sea $I_{x}\left(t\right)=1$ si $X\left(t\right)\leq x$,  $I_{x}\left(t\right)=0$ en caso contrario, y def\'inanse los tiempos promedio
\begin{eqnarray*}
\overline{X}&=&lim_{t\rightarrow\infty}\frac{1}{t}\int_{0}^{\infty}X\left(u\right)du\\
\prob\left(X_{\infty}\leq x\right)&=&lim_{t\rightarrow\infty}\frac{1}{t}\int_{0}^{\infty}I_{x}\left(u\right)du,
\end{eqnarray*}
cuando estos l\'imites existan.
\end{Def}

Como consecuencia del teorema de Renovaci\'on-Recompensa, se tiene que el primer l\'imite  existe y es igual a la constante
\begin{eqnarray*}
\overline{X}&=&\frac{\esp\left[\int_{0}^{R_{1}}X\left(t\right)dt\right]}{\esp\left[R_{1}\right]},
\end{eqnarray*}
suponiendo que ambas esperanzas son finitas.

\begin{Note}
\begin{itemize}
\item[a)] Si el proceso regenerativo $X$ es positivo recurrente y tiene trayectorias muestrales no negativas, entonces la ecuaci\'on anterior es v\'alida.
\item[b)] Si $X$ es positivo recurrente regenerativo, podemos construir una \'unica versi\'on estacionaria de este proceso, $X_{e}=\left\{X_{e}\left(t\right)\right\}$, donde $X_{e}$ es un proceso estoc\'astico regenerativo y estrictamente estacionario, con distribuci\'on marginal distribuida como $X_{\infty}$
\end{itemize}
\end{Note}

%__________________________________________________________________________________________
%\subsection{Procesos Regenerativos Estacionarios - Stidham \cite{Stidham}}
%__________________________________________________________________________________________


Un proceso estoc\'astico a tiempo continuo $\left\{V\left(t\right),t\geq0\right\}$ es un proceso regenerativo si existe una sucesi\'on de variables aleatorias independientes e id\'enticamente distribuidas $\left\{X_{1},X_{2},\ldots\right\}$, sucesi\'on de renovaci\'on, tal que para cualquier conjunto de Borel $A$, 

\begin{eqnarray*}
\prob\left\{V\left(t\right)\in A|X_{1}+X_{2}+\cdots+X_{R\left(t\right)}=s,\left\{V\left(\tau\right),\tau<s\right\}\right\}=\prob\left\{V\left(t-s\right)\in A|X_{1}>t-s\right\},
\end{eqnarray*}
para todo $0\leq s\leq t$, donde $R\left(t\right)=\max\left\{X_{1}+X_{2}+\cdots+X_{j}\leq t\right\}=$n\'umero de renovaciones ({\emph{puntos de regeneraci\'on}}) que ocurren en $\left[0,t\right]$. El intervalo $\left[0,X_{1}\right)$ es llamado {\emph{primer ciclo de regeneraci\'on}} de $\left\{V\left(t \right),t\geq0\right\}$, $\left[X_{1},X_{1}+X_{2}\right)$ el {\emph{segundo ciclo de regeneraci\'on}}, y as\'i sucesivamente.

Sea $X=X_{1}$ y sea $F$ la funci\'on de distrbuci\'on de $X$


\begin{Def}
Se define el proceso estacionario, $\left\{V^{*}\left(t\right),t\geq0\right\}$, para $\left\{V\left(t\right),t\geq0\right\}$ por

\begin{eqnarray*}
\prob\left\{V\left(t\right)\in A\right\}=\frac{1}{\esp\left[X\right]}\int_{0}^{\infty}\prob\left\{V\left(t+x\right)\in A|X>x\right\}\left(1-F\left(x\right)\right)dx,
\end{eqnarray*} 
para todo $t\geq0$ y todo conjunto de Borel $A$.
\end{Def}

\begin{Def}
Una distribuci\'on se dice que es {\emph{aritm\'etica}} si todos sus puntos de incremento son m\'ultiplos de la forma $0,\lambda, 2\lambda,\ldots$ para alguna $\lambda>0$ entera.
\end{Def}


\begin{Def}
Una modificaci\'on medible de un proceso $\left\{V\left(t\right),t\geq0\right\}$, es una versi\'on de este, $\left\{V\left(t,w\right)\right\}$ conjuntamente medible para $t\geq0$ y para $w\in S$, $S$ espacio de estados para $\left\{V\left(t\right),t\geq0\right\}$.
\end{Def}

\begin{Teo}
Sea $\left\{V\left(t\right),t\geq\right\}$ un proceso regenerativo no negativo con modificaci\'on medible. Sea $\esp\left[X\right]<\infty$. Entonces el proceso estacionario dado por la ecuaci\'on anterior est\'a bien definido y tiene funci\'on de distribuci\'on independiente de $t$, adem\'as
\begin{itemize}
\item[i)] \begin{eqnarray*}
\esp\left[V^{*}\left(0\right)\right]&=&\frac{\esp\left[\int_{0}^{X}V\left(s\right)ds\right]}{\esp\left[X\right]}\end{eqnarray*}
\item[ii)] Si $\esp\left[V^{*}\left(0\right)\right]<\infty$, equivalentemente, si $\esp\left[\int_{0}^{X}V\left(s\right)ds\right]<\infty$,entonces
\begin{eqnarray*}
\frac{\int_{0}^{t}V\left(s\right)ds}{t}\rightarrow\frac{\esp\left[\int_{0}^{X}V\left(s\right)ds\right]}{\esp\left[X\right]}
\end{eqnarray*}
con probabilidad 1 y en media, cuando $t\rightarrow\infty$.
\end{itemize}
\end{Teo}
%
%___________________________________________________________________________________________
%\vspace{5.5cm}
%\chapter{Cadenas de Markov estacionarias}
%\vspace{-1.0cm}


%__________________________________________________________________________________________
%\subsection{Procesos Regenerativos Estacionarios - Stidham \cite{Stidham}}
%__________________________________________________________________________________________


Un proceso estoc\'astico a tiempo continuo $\left\{V\left(t\right),t\geq0\right\}$ es un proceso regenerativo si existe una sucesi\'on de variables aleatorias independientes e id\'enticamente distribuidas $\left\{X_{1},X_{2},\ldots\right\}$, sucesi\'on de renovaci\'on, tal que para cualquier conjunto de Borel $A$, 

\begin{eqnarray*}
\prob\left\{V\left(t\right)\in A|X_{1}+X_{2}+\cdots+X_{R\left(t\right)}=s,\left\{V\left(\tau\right),\tau<s\right\}\right\}=\prob\left\{V\left(t-s\right)\in A|X_{1}>t-s\right\},
\end{eqnarray*}
para todo $0\leq s\leq t$, donde $R\left(t\right)=\max\left\{X_{1}+X_{2}+\cdots+X_{j}\leq t\right\}=$n\'umero de renovaciones ({\emph{puntos de regeneraci\'on}}) que ocurren en $\left[0,t\right]$. El intervalo $\left[0,X_{1}\right)$ es llamado {\emph{primer ciclo de regeneraci\'on}} de $\left\{V\left(t \right),t\geq0\right\}$, $\left[X_{1},X_{1}+X_{2}\right)$ el {\emph{segundo ciclo de regeneraci\'on}}, y as\'i sucesivamente.

Sea $X=X_{1}$ y sea $F$ la funci\'on de distrbuci\'on de $X$


\begin{Def}
Se define el proceso estacionario, $\left\{V^{*}\left(t\right),t\geq0\right\}$, para $\left\{V\left(t\right),t\geq0\right\}$ por

\begin{eqnarray*}
\prob\left\{V\left(t\right)\in A\right\}=\frac{1}{\esp\left[X\right]}\int_{0}^{\infty}\prob\left\{V\left(t+x\right)\in A|X>x\right\}\left(1-F\left(x\right)\right)dx,
\end{eqnarray*} 
para todo $t\geq0$ y todo conjunto de Borel $A$.
\end{Def}

\begin{Def}
Una distribuci\'on se dice que es {\emph{aritm\'etica}} si todos sus puntos de incremento son m\'ultiplos de la forma $0,\lambda, 2\lambda,\ldots$ para alguna $\lambda>0$ entera.
\end{Def}


\begin{Def}
Una modificaci\'on medible de un proceso $\left\{V\left(t\right),t\geq0\right\}$, es una versi\'on de este, $\left\{V\left(t,w\right)\right\}$ conjuntamente medible para $t\geq0$ y para $w\in S$, $S$ espacio de estados para $\left\{V\left(t\right),t\geq0\right\}$.
\end{Def}

\begin{Teo}
Sea $\left\{V\left(t\right),t\geq\right\}$ un proceso regenerativo no negativo con modificaci\'on medible. Sea $\esp\left[X\right]<\infty$. Entonces el proceso estacionario dado por la ecuaci\'on anterior est\'a bien definido y tiene funci\'on de distribuci\'on independiente de $t$, adem\'as
\begin{itemize}
\item[i)] \begin{eqnarray*}
\esp\left[V^{*}\left(0\right)\right]&=&\frac{\esp\left[\int_{0}^{X}V\left(s\right)ds\right]}{\esp\left[X\right]}\end{eqnarray*}
\item[ii)] Si $\esp\left[V^{*}\left(0\right)\right]<\infty$, equivalentemente, si $\esp\left[\int_{0}^{X}V\left(s\right)ds\right]<\infty$,entonces
\begin{eqnarray*}
\frac{\int_{0}^{t}V\left(s\right)ds}{t}\rightarrow\frac{\esp\left[\int_{0}^{X}V\left(s\right)ds\right]}{\esp\left[X\right]}
\end{eqnarray*}
con probabilidad 1 y en media, cuando $t\rightarrow\infty$.
\end{itemize}
\end{Teo}

Sea la funci\'on generadora de momentos para $L_{i}$, el n\'umero de usuarios en la cola $Q_{i}\left(z\right)$ en cualquier momento, est\'a dada por el tiempo promedio de $z^{L_{i}\left(t\right)}$ sobre el ciclo regenerativo definido anteriormente. Entonces 



Es decir, es posible determinar las longitudes de las colas a cualquier tiempo $t$. Entonces, determinando el primer momento es posible ver que


\begin{Def}
El tiempo de Ciclo $C_{i}$ es el periodo de tiempo que comienza cuando la cola $i$ es visitada por primera vez en un ciclo, y termina cuando es visitado nuevamente en el pr\'oximo ciclo. La duraci\'on del mismo est\'a dada por $\tau_{i}\left(m+1\right)-\tau_{i}\left(m\right)$, o equivalentemente $\overline{\tau}_{i}\left(m+1\right)-\overline{\tau}_{i}\left(m\right)$ bajo condiciones de estabilidad.
\end{Def}


\begin{Def}
El tiempo de intervisita $I_{i}$ es el periodo de tiempo que comienza cuando se ha completado el servicio en un ciclo y termina cuando es visitada nuevamente en el pr\'oximo ciclo. Su  duraci\'on del mismo est\'a dada por $\tau_{i}\left(m+1\right)-\overline{\tau}_{i}\left(m\right)$.
\end{Def}

La duraci\'on del tiempo de intervisita es $\tau_{i}\left(m+1\right)-\overline{\tau}\left(m\right)$. Dado que el n\'umero de usuarios presentes en $Q_{i}$ al tiempo $t=\tau_{i}\left(m+1\right)$ es igual al n\'umero de arribos durante el intervalo de tiempo $\left[\overline{\tau}\left(m\right),\tau_{i}\left(m+1\right)\right]$ se tiene que


\begin{eqnarray*}
\esp\left[z_{i}^{L_{i}\left(\tau_{i}\left(m+1\right)\right)}\right]=\esp\left[\left\{P_{i}\left(z_{i}\right)\right\}^{\tau_{i}\left(m+1\right)-\overline{\tau}\left(m\right)}\right]
\end{eqnarray*}

entonces, si $I_{i}\left(z\right)=\esp\left[z^{\tau_{i}\left(m+1\right)-\overline{\tau}\left(m\right)}\right]$
se tiene que $F_{i}\left(z\right)=I_{i}\left[P_{i}\left(z\right)\right]$
para $i=1,2$.

Conforme a la definici\'on dada al principio del cap\'itulo, definici\'on (\ref{Def.Tn}), sean $T_{1},T_{2},\ldots$ los puntos donde las longitudes de las colas de la red de sistemas de visitas c\'iclicas son cero simult\'aneamente, cuando la cola $Q_{j}$ es visitada por el servidor para dar servicio, es decir, $L_{1}\left(T_{i}\right)=0,L_{2}\left(T_{i}\right)=0,\hat{L}_{1}\left(T_{i}\right)=0$ y $\hat{L}_{2}\left(T_{i}\right)=0$, a estos puntos se les denominar\'a puntos regenerativos. Entonces, 

\begin{Def}
Al intervalo de tiempo entre dos puntos regenerativos se le llamar\'a ciclo regenerativo.
\end{Def}

\begin{Def}
Para $T_{i}$ se define, $M_{i}$, el n\'umero de ciclos de visita a la cola $Q_{l}$, durante el ciclo regenerativo, es decir, $M_{i}$ es un proceso de renovaci\'on.
\end{Def}

\begin{Def}
Para cada uno de los $M_{i}$'s, se definen a su vez la duraci\'on de cada uno de estos ciclos de visita en el ciclo regenerativo, $C_{i}^{(m)}$, para $m=1,2,\ldots,M_{i}$, que a su vez, tambi\'en es n proceso de renovaci\'on.
\end{Def}

\footnote{In Stidham and  Heyman \cite{Stidham} shows that is sufficient for the regenerative process to be stationary that the mean regenerative cycle time is finite: $\esp\left[\sum_{m=1}^{M_{i}}C_{i}^{(m)}\right]<\infty$, 


 como cada $C_{i}^{(m)}$ contiene intervalos de r\'eplica positivos, se tiene que $\esp\left[M_{i}\right]<\infty$, adem\'as, como $M_{i}>0$, se tiene que la condici\'on anterior es equivalente a tener que $\esp\left[C_{i}\right]<\infty$,
por lo tanto una condici\'on suficiente para la existencia del proceso regenerativo est\'a dada por $\sum_{k=1}^{N}\mu_{k}<1.$}
%________________________________________________________________________
\subsection{Procesos Regenerativos Sigman, Thorisson y Wolff \cite{Sigman2}}
%________________________________________________________________________


\begin{Def}[Definici\'on Cl\'asica]
Un proceso estoc\'astico $X=\left\{X\left(t\right):t\geq0\right\}$ es llamado regenerativo is existe una variable aleatoria $R_{1}>0$ tal que
\begin{itemize}
\item[i)] $\left\{X\left(t+R_{1}\right):t\geq0\right\}$ es independiente de $\left\{\left\{X\left(t\right):t<R_{1}\right\},\right\}$
\item[ii)] $\left\{X\left(t+R_{1}\right):t\geq0\right\}$ es estoc\'asticamente equivalente a $\left\{X\left(t\right):t>0\right\}$
\end{itemize}

Llamamos a $R_{1}$ tiempo de regeneraci\'on, y decimos que $X$ se regenera en este punto.
\end{Def}

$\left\{X\left(t+R_{1}\right)\right\}$ es regenerativo con tiempo de regeneraci\'on $R_{2}$, independiente de $R_{1}$ pero con la misma distribuci\'on que $R_{1}$. Procediendo de esta manera se obtiene una secuencia de variables aleatorias independientes e id\'enticamente distribuidas $\left\{R_{n}\right\}$ llamados longitudes de ciclo. Si definimos a $Z_{k}\equiv R_{1}+R_{2}+\cdots+R_{k}$, se tiene un proceso de renovaci\'on llamado proceso de renovaci\'on encajado para $X$.


\begin{Note}
La existencia de un primer tiempo de regeneraci\'on, $R_{1}$, implica la existencia de una sucesi\'on completa de estos tiempos $R_{1},R_{2}\ldots,$ que satisfacen la propiedad deseada \cite{Sigman2}.
\end{Note}


\begin{Note} Para la cola $GI/GI/1$ los usuarios arriban con tiempos $t_{n}$ y son atendidos con tiempos de servicio $S_{n}$, los tiempos de arribo forman un proceso de renovaci\'on  con tiempos entre arribos independientes e identicamente distribuidos (\texttt{i.i.d.})$T_{n}=t_{n}-t_{n-1}$, adem\'as los tiempos de servicio son \texttt{i.i.d.} e independientes de los procesos de arribo. Por \textit{estable} se entiende que $\esp S_{n}<\esp T_{n}<\infty$.
\end{Note}
 


\begin{Def}
Para $x$ fijo y para cada $t\geq0$, sea $I_{x}\left(t\right)=1$ si $X\left(t\right)\leq x$,  $I_{x}\left(t\right)=0$ en caso contrario, y def\'inanse los tiempos promedio
\begin{eqnarray*}
\overline{X}&=&lim_{t\rightarrow\infty}\frac{1}{t}\int_{0}^{\infty}X\left(u\right)du\\
\prob\left(X_{\infty}\leq x\right)&=&lim_{t\rightarrow\infty}\frac{1}{t}\int_{0}^{\infty}I_{x}\left(u\right)du,
\end{eqnarray*}
cuando estos l\'imites existan.
\end{Def}

Como consecuencia del teorema de Renovaci\'on-Recompensa, se tiene que el primer l\'imite  existe y es igual a la constante
\begin{eqnarray*}
\overline{X}&=&\frac{\esp\left[\int_{0}^{R_{1}}X\left(t\right)dt\right]}{\esp\left[R_{1}\right]},
\end{eqnarray*}
suponiendo que ambas esperanzas son finitas.
 
\begin{Note}
Funciones de procesos regenerativos son regenerativas, es decir, si $X\left(t\right)$ es regenerativo y se define el proceso $Y\left(t\right)$ por $Y\left(t\right)=f\left(X\left(t\right)\right)$ para alguna funci\'on Borel medible $f\left(\cdot\right)$. Adem\'as $Y$ es regenerativo con los mismos tiempos de renovaci\'on que $X$. 

En general, los tiempos de renovaci\'on, $Z_{k}$ de un proceso regenerativo no requieren ser tiempos de paro con respecto a la evoluci\'on de $X\left(t\right)$.
\end{Note} 

\begin{Note}
Una funci\'on de un proceso de Markov, usualmente no ser\'a un proceso de Markov, sin embargo ser\'a regenerativo si el proceso de Markov lo es.
\end{Note}

 
\begin{Note}
Un proceso regenerativo con media de la longitud de ciclo finita es llamado positivo recurrente.
\end{Note}


\begin{Note}
\begin{itemize}
\item[a)] Si el proceso regenerativo $X$ es positivo recurrente y tiene trayectorias muestrales no negativas, entonces la ecuaci\'on anterior es v\'alida.
\item[b)] Si $X$ es positivo recurrente regenerativo, podemos construir una \'unica versi\'on estacionaria de este proceso, $X_{e}=\left\{X_{e}\left(t\right)\right\}$, donde $X_{e}$ es un proceso estoc\'astico regenerativo y estrictamente estacionario, con distribuci\'on marginal distribuida como $X_{\infty}$
\end{itemize}
\end{Note}



%________________________________________________________________________
\subsection{Procesos Regenerativos}
%________________________________________________________________________



\begin{Note}
Si $\tilde{X}\left(t\right)$ con espacio de estados $\tilde{S}$ es regenerativo sobre $T_{n}$, entonces $X\left(t\right)=f\left(\tilde{X}\left(t\right)\right)$ tambi\'en es regenerativo sobre $T_{n}$, para cualquier funci\'on $f:\tilde{S}\rightarrow S$.
\end{Note}

\begin{Note}
Los procesos regenerativos son crudamente regenerativos, pero no al rev\'es.
\end{Note}

%______________________________________________________________________
\subsection*{Procesos Regenerativos: Sigman\cite{Sigman1}}
%______________________________________________________________________
\begin{Def}[Definici\'on Cl\'asica]
Un proceso estoc\'astico $X=\left\{X\left(t\right):t\geq0\right\}$ es llamado regenerativo is existe una variable aleatoria $R_{1}>0$ tal que
\begin{itemize}
\item[i)] $\left\{X\left(t+R_{1}\right):t\geq0\right\}$ es independiente de $\left\{\left\{X\left(t\right):t<R_{1}\right\},\right\}$
\item[ii)] $\left\{X\left(t+R_{1}\right):t\geq0\right\}$ es estoc\'asticamente equivalente a $\left\{X\left(t\right):t>0\right\}$
\end{itemize}

Llamamos a $R_{1}$ tiempo de regeneraci\'on, y decimos que $X$ se regenera en este punto.
\end{Def}

$\left\{X\left(t+R_{1}\right)\right\}$ es regenerativo con tiempo de regeneraci\'on $R_{2}$, independiente de $R_{1}$ pero con la misma distribuci\'on que $R_{1}$. Procediendo de esta manera se obtiene una secuencia de variables aleatorias independientes e id\'enticamente distribuidas $\left\{R_{n}\right\}$ llamados longitudes de ciclo. Si definimos a $Z_{k}\equiv R_{1}+R_{2}+\cdots+R_{k}$, se tiene un proceso de renovaci\'on llamado proceso de renovaci\'on encajado para $X$.




\begin{Def}
Para $x$ fijo y para cada $t\geq0$, sea $I_{x}\left(t\right)=1$ si $X\left(t\right)\leq x$,  $I_{x}\left(t\right)=0$ en caso contrario, y def\'inanse los tiempos promedio
\begin{eqnarray*}
\overline{X}&=&lim_{t\rightarrow\infty}\frac{1}{t}\int_{0}^{\infty}X\left(u\right)du\\
\prob\left(X_{\infty}\leq x\right)&=&lim_{t\rightarrow\infty}\frac{1}{t}\int_{0}^{\infty}I_{x}\left(u\right)du,
\end{eqnarray*}
cuando estos l\'imites existan.
\end{Def}

Como consecuencia del teorema de Renovaci\'on-Recompensa, se tiene que el primer l\'imite  existe y es igual a la constante
\begin{eqnarray*}
\overline{X}&=&\frac{\esp\left[\int_{0}^{R_{1}}X\left(t\right)dt\right]}{\esp\left[R_{1}\right]},
\end{eqnarray*}
suponiendo que ambas esperanzas son finitas.

\begin{Note}
\begin{itemize}
\item[a)] Si el proceso regenerativo $X$ es positivo recurrente y tiene trayectorias muestrales no negativas, entonces la ecuaci\'on anterior es v\'alida.
\item[b)] Si $X$ es positivo recurrente regenerativo, podemos construir una \'unica versi\'on estacionaria de este proceso, $X_{e}=\left\{X_{e}\left(t\right)\right\}$, donde $X_{e}$ es un proceso estoc\'astico regenerativo y estrictamente estacionario, con distribuci\'on marginal distribuida como $X_{\infty}$
\end{itemize}
\end{Note}


%__________________________________________________________________________________________
\subsection{Procesos Regenerativos Estacionarios - Stidham \cite{Stidham}}
%__________________________________________________________________________________________


Un proceso estoc\'astico a tiempo continuo $\left\{V\left(t\right),t\geq0\right\}$ es un proceso regenerativo si existe una sucesi\'on de variables aleatorias independientes e id\'enticamente distribuidas $\left\{X_{1},X_{2},\ldots\right\}$, sucesi\'on de renovaci\'on, tal que para cualquier conjunto de Borel $A$, 

\begin{eqnarray*}
\prob\left\{V\left(t\right)\in A|X_{1}+X_{2}+\cdots+X_{R\left(t\right)}=s,\left\{V\left(\tau\right),\tau<s\right\}\right\}=\prob\left\{V\left(t-s\right)\in A|X_{1}>t-s\right\},
\end{eqnarray*}
para todo $0\leq s\leq t$, donde $R\left(t\right)=\max\left\{X_{1}+X_{2}+\cdots+X_{j}\leq t\right\}=$n\'umero de renovaciones ({\emph{puntos de regeneraci\'on}}) que ocurren en $\left[0,t\right]$. El intervalo $\left[0,X_{1}\right)$ es llamado {\emph{primer ciclo de regeneraci\'on}} de $\left\{V\left(t \right),t\geq0\right\}$, $\left[X_{1},X_{1}+X_{2}\right)$ el {\emph{segundo ciclo de regeneraci\'on}}, y as\'i sucesivamente.

Sea $X=X_{1}$ y sea $F$ la funci\'on de distrbuci\'on de $X$


\begin{Def}
Se define el proceso estacionario, $\left\{V^{*}\left(t\right),t\geq0\right\}$, para $\left\{V\left(t\right),t\geq0\right\}$ por

\begin{eqnarray*}
\prob\left\{V\left(t\right)\in A\right\}=\frac{1}{\esp\left[X\right]}\int_{0}^{\infty}\prob\left\{V\left(t+x\right)\in A|X>x\right\}\left(1-F\left(x\right)\right)dx,
\end{eqnarray*} 
para todo $t\geq0$ y todo conjunto de Borel $A$.
\end{Def}

\begin{Def}
Una distribuci\'on se dice que es {\emph{aritm\'etica}} si todos sus puntos de incremento son m\'ultiplos de la forma $0,\lambda, 2\lambda,\ldots$ para alguna $\lambda>0$ entera.
\end{Def}


\begin{Def}
Una modificaci\'on medible de un proceso $\left\{V\left(t\right),t\geq0\right\}$, es una versi\'on de este, $\left\{V\left(t,w\right)\right\}$ conjuntamente medible para $t\geq0$ y para $w\in S$, $S$ espacio de estados para $\left\{V\left(t\right),t\geq0\right\}$.
\end{Def}

\begin{Teo}
Sea $\left\{V\left(t\right),t\geq\right\}$ un proceso regenerativo no negativo con modificaci\'on medible. Sea $\esp\left[X\right]<\infty$. Entonces el proceso estacionario dado por la ecuaci\'on anterior est\'a bien definido y tiene funci\'on de distribuci\'on independiente de $t$, adem\'as
\begin{itemize}
\item[i)] \begin{eqnarray*}
\esp\left[V^{*}\left(0\right)\right]&=&\frac{\esp\left[\int_{0}^{X}V\left(s\right)ds\right]}{\esp\left[X\right]}\end{eqnarray*}
\item[ii)] Si $\esp\left[V^{*}\left(0\right)\right]<\infty$, equivalentemente, si $\esp\left[\int_{0}^{X}V\left(s\right)ds\right]<\infty$,entonces
\begin{eqnarray*}
\frac{\int_{0}^{t}V\left(s\right)ds}{t}\rightarrow\frac{\esp\left[\int_{0}^{X}V\left(s\right)ds\right]}{\esp\left[X\right]}
\end{eqnarray*}
con probabilidad 1 y en media, cuando $t\rightarrow\infty$.
\end{itemize}
\end{Teo}

%________________________________________________________________________
\subsection{Procesos Regenerativos}
%________________________________________________________________________

Para $\left\{X\left(t\right):t\geq0\right\}$ Proceso Estoc\'astico a tiempo continuo con estado de espacios $S$, que es un espacio m\'etrico, con trayectorias continuas por la derecha y con l\'imites por la izquierda c.s. Sea $N\left(t\right)$ un proceso de renovaci\'on en $\rea_{+}$ definido en el mismo espacio de probabilidad que $X\left(t\right)$, con tiempos de renovaci\'on $T$ y tiempos de inter-renovaci\'on $\xi_{n}=T_{n}-T_{n-1}$, con misma distribuci\'on $F$ de media finita $\mu$.



\begin{Def}
Para el proceso $\left\{\left(N\left(t\right),X\left(t\right)\right):t\geq0\right\}$, sus trayectoria muestrales en el intervalo de tiempo $\left[T_{n-1},T_{n}\right)$ est\'an descritas por
\begin{eqnarray*}
\zeta_{n}=\left(\xi_{n},\left\{X\left(T_{n-1}+t\right):0\leq t<\xi_{n}\right\}\right)
\end{eqnarray*}
Este $\zeta_{n}$ es el $n$-\'esimo segmento del proceso. El proceso es regenerativo sobre los tiempos $T_{n}$ si sus segmentos $\zeta_{n}$ son independientes e id\'enticamennte distribuidos.
\end{Def}


\begin{Obs}
Si $\tilde{X}\left(t\right)$ con espacio de estados $\tilde{S}$ es regenerativo sobre $T_{n}$, entonces $X\left(t\right)=f\left(\tilde{X}\left(t\right)\right)$ tambi\'en es regenerativo sobre $T_{n}$, para cualquier funci\'on $f:\tilde{S}\rightarrow S$.
\end{Obs}

\begin{Obs}
Los procesos regenerativos son crudamente regenerativos, pero no al rev\'es.
\end{Obs}

\begin{Def}[Definici\'on Cl\'asica]
Un proceso estoc\'astico $X=\left\{X\left(t\right):t\geq0\right\}$ es llamado regenerativo is existe una variable aleatoria $R_{1}>0$ tal que
\begin{itemize}
\item[i)] $\left\{X\left(t+R_{1}\right):t\geq0\right\}$ es independiente de $\left\{\left\{X\left(t\right):t<R_{1}\right\},\right\}$
\item[ii)] $\left\{X\left(t+R_{1}\right):t\geq0\right\}$ es estoc\'asticamente equivalente a $\left\{X\left(t\right):t>0\right\}$
\end{itemize}

Llamamos a $R_{1}$ tiempo de regeneraci\'on, y decimos que $X$ se regenera en este punto.
\end{Def}

$\left\{X\left(t+R_{1}\right)\right\}$ es regenerativo con tiempo de regeneraci\'on $R_{2}$, independiente de $R_{1}$ pero con la misma distribuci\'on que $R_{1}$. Procediendo de esta manera se obtiene una secuencia de variables aleatorias independientes e id\'enticamente distribuidas $\left\{R_{n}\right\}$ llamados longitudes de ciclo. Si definimos a $Z_{k}\equiv R_{1}+R_{2}+\cdots+R_{k}$, se tiene un proceso de renovaci\'on llamado proceso de renovaci\'on encajado para $X$.

\begin{Note}
Un proceso regenerativo con media de la longitud de ciclo finita es llamado positivo recurrente.
\end{Note}


\begin{Def}
Para $x$ fijo y para cada $t\geq0$, sea $I_{x}\left(t\right)=1$ si $X\left(t\right)\leq x$,  $I_{x}\left(t\right)=0$ en caso contrario, y def\'inanse los tiempos promedio
\begin{eqnarray*}
\overline{X}&=&lim_{t\rightarrow\infty}\frac{1}{t}\int_{0}^{\infty}X\left(u\right)du\\
\prob\left(X_{\infty}\leq x\right)&=&lim_{t\rightarrow\infty}\frac{1}{t}\int_{0}^{\infty}I_{x}\left(u\right)du,
\end{eqnarray*}
cuando estos l\'imites existan.
\end{Def}

Como consecuencia del teorema de Renovaci\'on-Recompensa, se tiene que el primer l\'imite  existe y es igual a la constante
\begin{eqnarray*}
\overline{X}&=&\frac{\esp\left[\int_{0}^{R_{1}}X\left(t\right)dt\right]}{\esp\left[R_{1}\right]},
\end{eqnarray*}
suponiendo que ambas esperanzas son finitas.

\begin{Note}
\begin{itemize}
\item[a)] Si el proceso regenerativo $X$ es positivo recurrente y tiene trayectorias muestrales no negativas, entonces la ecuaci\'on anterior es v\'alida.
\item[b)] Si $X$ es positivo recurrente regenerativo, podemos construir una \'unica versi\'on estacionaria de este proceso, $X_{e}=\left\{X_{e}\left(t\right)\right\}$, donde $X_{e}$ es un proceso estoc\'astico regenerativo y estrictamente estacionario, con distribuci\'on marginal distribuida como $X_{\infty}$
\end{itemize}
\end{Note}




%________________________________________________________________________
\subsection{Procesos Regenerativos Sigman, Thorisson y Wolff \cite{Sigman2}}
%________________________________________________________________________


\begin{Def}[Definici\'on Cl\'asica]
Un proceso estoc\'astico $X=\left\{X\left(t\right):t\geq0\right\}$ es llamado regenerativo is existe una variable aleatoria $R_{1}>0$ tal que
\begin{itemize}
\item[i)] $\left\{X\left(t+R_{1}\right):t\geq0\right\}$ es independiente de $\left\{\left\{X\left(t\right):t<R_{1}\right\},\right\}$
\item[ii)] $\left\{X\left(t+R_{1}\right):t\geq0\right\}$ es estoc\'asticamente equivalente a $\left\{X\left(t\right):t>0\right\}$
\end{itemize}

Llamamos a $R_{1}$ tiempo de regeneraci\'on, y decimos que $X$ se regenera en este punto.
\end{Def}

$\left\{X\left(t+R_{1}\right)\right\}$ es regenerativo con tiempo de regeneraci\'on $R_{2}$, independiente de $R_{1}$ pero con la misma distribuci\'on que $R_{1}$. Procediendo de esta manera se obtiene una secuencia de variables aleatorias independientes e id\'enticamente distribuidas $\left\{R_{n}\right\}$ llamados longitudes de ciclo. Si definimos a $Z_{k}\equiv R_{1}+R_{2}+\cdots+R_{k}$, se tiene un proceso de renovaci\'on llamado proceso de renovaci\'on encajado para $X$.


\begin{Note}
La existencia de un primer tiempo de regeneraci\'on, $R_{1}$, implica la existencia de una sucesi\'on completa de estos tiempos $R_{1},R_{2}\ldots,$ que satisfacen la propiedad deseada \cite{Sigman2}.
\end{Note}


\begin{Note} Para la cola $GI/GI/1$ los usuarios arriban con tiempos $t_{n}$ y son atendidos con tiempos de servicio $S_{n}$, los tiempos de arribo forman un proceso de renovaci\'on  con tiempos entre arribos independientes e identicamente distribuidos (\texttt{i.i.d.})$T_{n}=t_{n}-t_{n-1}$, adem\'as los tiempos de servicio son \texttt{i.i.d.} e independientes de los procesos de arribo. Por \textit{estable} se entiende que $\esp S_{n}<\esp T_{n}<\infty$.
\end{Note}
 


\begin{Def}
Para $x$ fijo y para cada $t\geq0$, sea $I_{x}\left(t\right)=1$ si $X\left(t\right)\leq x$,  $I_{x}\left(t\right)=0$ en caso contrario, y def\'inanse los tiempos promedio
\begin{eqnarray*}
\overline{X}&=&lim_{t\rightarrow\infty}\frac{1}{t}\int_{0}^{\infty}X\left(u\right)du\\
\prob\left(X_{\infty}\leq x\right)&=&lim_{t\rightarrow\infty}\frac{1}{t}\int_{0}^{\infty}I_{x}\left(u\right)du,
\end{eqnarray*}
cuando estos l\'imites existan.
\end{Def}

Como consecuencia del teorema de Renovaci\'on-Recompensa, se tiene que el primer l\'imite  existe y es igual a la constante
\begin{eqnarray*}
\overline{X}&=&\frac{\esp\left[\int_{0}^{R_{1}}X\left(t\right)dt\right]}{\esp\left[R_{1}\right]},
\end{eqnarray*}
suponiendo que ambas esperanzas son finitas.
 
\begin{Note}
Funciones de procesos regenerativos son regenerativas, es decir, si $X\left(t\right)$ es regenerativo y se define el proceso $Y\left(t\right)$ por $Y\left(t\right)=f\left(X\left(t\right)\right)$ para alguna funci\'on Borel medible $f\left(\cdot\right)$. Adem\'as $Y$ es regenerativo con los mismos tiempos de renovaci\'on que $X$. 

En general, los tiempos de renovaci\'on, $Z_{k}$ de un proceso regenerativo no requieren ser tiempos de paro con respecto a la evoluci\'on de $X\left(t\right)$.
\end{Note} 

\begin{Note}
Una funci\'on de un proceso de Markov, usualmente no ser\'a un proceso de Markov, sin embargo ser\'a regenerativo si el proceso de Markov lo es.
\end{Note}

 
\begin{Note}
Un proceso regenerativo con media de la longitud de ciclo finita es llamado positivo recurrente.
\end{Note}


\begin{Note}
\begin{itemize}
\item[a)] Si el proceso regenerativo $X$ es positivo recurrente y tiene trayectorias muestrales no negativas, entonces la ecuaci\'on anterior es v\'alida.
\item[b)] Si $X$ es positivo recurrente regenerativo, podemos construir una \'unica versi\'on estacionaria de este proceso, $X_{e}=\left\{X_{e}\left(t\right)\right\}$, donde $X_{e}$ es un proceso estoc\'astico regenerativo y estrictamente estacionario, con distribuci\'on marginal distribuida como $X_{\infty}$
\end{itemize}
\end{Note}


%__________________________________________________________________________________________
\subsection{Procesos Regenerativos Estacionarios - Stidham \cite{Stidham}}
%__________________________________________________________________________________________


Un proceso estoc\'astico a tiempo continuo $\left\{V\left(t\right),t\geq0\right\}$ es un proceso regenerativo si existe una sucesi\'on de variables aleatorias independientes e id\'enticamente distribuidas $\left\{X_{1},X_{2},\ldots\right\}$, sucesi\'on de renovaci\'on, tal que para cualquier conjunto de Borel $A$, 

\begin{eqnarray*}
\prob\left\{V\left(t\right)\in A|X_{1}+X_{2}+\cdots+X_{R\left(t\right)}=s,\left\{V\left(\tau\right),\tau<s\right\}\right\}=\prob\left\{V\left(t-s\right)\in A|X_{1}>t-s\right\},
\end{eqnarray*}
para todo $0\leq s\leq t$, donde $R\left(t\right)=\max\left\{X_{1}+X_{2}+\cdots+X_{j}\leq t\right\}=$n\'umero de renovaciones ({\emph{puntos de regeneraci\'on}}) que ocurren en $\left[0,t\right]$. El intervalo $\left[0,X_{1}\right)$ es llamado {\emph{primer ciclo de regeneraci\'on}} de $\left\{V\left(t \right),t\geq0\right\}$, $\left[X_{1},X_{1}+X_{2}\right)$ el {\emph{segundo ciclo de regeneraci\'on}}, y as\'i sucesivamente.

Sea $X=X_{1}$ y sea $F$ la funci\'on de distrbuci\'on de $X$


\begin{Def}
Se define el proceso estacionario, $\left\{V^{*}\left(t\right),t\geq0\right\}$, para $\left\{V\left(t\right),t\geq0\right\}$ por

\begin{eqnarray*}
\prob\left\{V\left(t\right)\in A\right\}=\frac{1}{\esp\left[X\right]}\int_{0}^{\infty}\prob\left\{V\left(t+x\right)\in A|X>x\right\}\left(1-F\left(x\right)\right)dx,
\end{eqnarray*} 
para todo $t\geq0$ y todo conjunto de Borel $A$.
\end{Def}

\begin{Def}
Una distribuci\'on se dice que es {\emph{aritm\'etica}} si todos sus puntos de incremento son m\'ultiplos de la forma $0,\lambda, 2\lambda,\ldots$ para alguna $\lambda>0$ entera.
\end{Def}


\begin{Def}
Una modificaci\'on medible de un proceso $\left\{V\left(t\right),t\geq0\right\}$, es una versi\'on de este, $\left\{V\left(t,w\right)\right\}$ conjuntamente medible para $t\geq0$ y para $w\in S$, $S$ espacio de estados para $\left\{V\left(t\right),t\geq0\right\}$.
\end{Def}

\begin{Teo}
Sea $\left\{V\left(t\right),t\geq\right\}$ un proceso regenerativo no negativo con modificaci\'on medible. Sea $\esp\left[X\right]<\infty$. Entonces el proceso estacionario dado por la ecuaci\'on anterior est\'a bien definido y tiene funci\'on de distribuci\'on independiente de $t$, adem\'as
\begin{itemize}
\item[i)] \begin{eqnarray*}
\esp\left[V^{*}\left(0\right)\right]&=&\frac{\esp\left[\int_{0}^{X}V\left(s\right)ds\right]}{\esp\left[X\right]}\end{eqnarray*}
\item[ii)] Si $\esp\left[V^{*}\left(0\right)\right]<\infty$, equivalentemente, si $\esp\left[\int_{0}^{X}V\left(s\right)ds\right]<\infty$,entonces
\begin{eqnarray*}
\frac{\int_{0}^{t}V\left(s\right)ds}{t}\rightarrow\frac{\esp\left[\int_{0}^{X}V\left(s\right)ds\right]}{\esp\left[X\right]}
\end{eqnarray*}
con probabilidad 1 y en media, cuando $t\rightarrow\infty$.
\end{itemize}
\end{Teo}

\begin{Coro}
Sea $\left\{V\left(t\right),t\geq0\right\}$ un proceso regenerativo no negativo, con modificaci\'on medible. Si $\esp <\infty$, $F$ es no-aritm\'etica, y para todo $x\geq0$, $P\left\{V\left(t\right)\leq x,C>x\right\}$ es de variaci\'on acotada como funci\'on de $t$ en cada intervalo finito $\left[0,\tau\right]$, entonces $V\left(t\right)$ converge en distribuci\'on  cuando $t\rightarrow\infty$ y $$\esp V=\frac{\esp \int_{0}^{X}V\left(s\right)ds}{\esp X}$$
Donde $V$ tiene la distribuci\'on l\'imite de $V\left(t\right)$ cuando $t\rightarrow\infty$.

\end{Coro}

Para el caso discreto se tienen resultados similares.



%______________________________________________________________________
\subsection{Procesos de Renovaci\'on}
%______________________________________________________________________

\begin{Def}%\label{Def.Tn}
Sean $0\leq T_{1}\leq T_{2}\leq \ldots$ son tiempos aleatorios infinitos en los cuales ocurren ciertos eventos. El n\'umero de tiempos $T_{n}$ en el intervalo $\left[0,t\right)$ es

\begin{eqnarray}
N\left(t\right)=\sum_{n=1}^{\infty}\indora\left(T_{n}\leq t\right),
\end{eqnarray}
para $t\geq0$.
\end{Def}

Si se consideran los puntos $T_{n}$ como elementos de $\rea_{+}$, y $N\left(t\right)$ es el n\'umero de puntos en $\rea$. El proceso denotado por $\left\{N\left(t\right):t\geq0\right\}$, denotado por $N\left(t\right)$, es un proceso puntual en $\rea_{+}$. Los $T_{n}$ son los tiempos de ocurrencia, el proceso puntual $N\left(t\right)$ es simple si su n\'umero de ocurrencias son distintas: $0<T_{1}<T_{2}<\ldots$ casi seguramente.

\begin{Def}
Un proceso puntual $N\left(t\right)$ es un proceso de renovaci\'on si los tiempos de interocurrencia $\xi_{n}=T_{n}-T_{n-1}$, para $n\geq1$, son independientes e identicamente distribuidos con distribuci\'on $F$, donde $F\left(0\right)=0$ y $T_{0}=0$. Los $T_{n}$ son llamados tiempos de renovaci\'on, referente a la independencia o renovaci\'on de la informaci\'on estoc\'astica en estos tiempos. Los $\xi_{n}$ son los tiempos de inter-renovaci\'on, y $N\left(t\right)$ es el n\'umero de renovaciones en el intervalo $\left[0,t\right)$
\end{Def}


\begin{Note}
Para definir un proceso de renovaci\'on para cualquier contexto, solamente hay que especificar una distribuci\'on $F$, con $F\left(0\right)=0$, para los tiempos de inter-renovaci\'on. La funci\'on $F$ en turno degune las otra variables aleatorias. De manera formal, existe un espacio de probabilidad y una sucesi\'on de variables aleatorias $\xi_{1},\xi_{2},\ldots$ definidas en este con distribuci\'on $F$. Entonces las otras cantidades son $T_{n}=\sum_{k=1}^{n}\xi_{k}$ y $N\left(t\right)=\sum_{n=1}^{\infty}\indora\left(T_{n}\leq t\right)$, donde $T_{n}\rightarrow\infty$ casi seguramente por la Ley Fuerte de los Grandes Números.
\end{Note}

%___________________________________________________________________________________________
%
\subsection{Teorema Principal de Renovaci\'on}
%___________________________________________________________________________________________
%

\begin{Note} Una funci\'on $h:\rea_{+}\rightarrow\rea$ es Directamente Riemann Integrable en los siguientes casos:
\begin{itemize}
\item[a)] $h\left(t\right)\geq0$ es decreciente y Riemann Integrable.
\item[b)] $h$ es continua excepto posiblemente en un conjunto de Lebesgue de medida 0, y $|h\left(t\right)|\leq b\left(t\right)$, donde $b$ es DRI.
\end{itemize}
\end{Note}

\begin{Teo}[Teorema Principal de Renovaci\'on]
Si $F$ es no aritm\'etica y $h\left(t\right)$ es Directamente Riemann Integrable (DRI), entonces

\begin{eqnarray*}
lim_{t\rightarrow\infty}U\star h=\frac{1}{\mu}\int_{\rea_{+}}h\left(s\right)ds.
\end{eqnarray*}
\end{Teo}

\begin{Prop}
Cualquier funci\'on $H\left(t\right)$ acotada en intervalos finitos y que es 0 para $t<0$ puede expresarse como
\begin{eqnarray*}
H\left(t\right)=U\star h\left(t\right)\textrm{,  donde }h\left(t\right)=H\left(t\right)-F\star H\left(t\right)
\end{eqnarray*}
\end{Prop}

\begin{Def}
Un proceso estoc\'astico $X\left(t\right)$ es crudamente regenerativo en un tiempo aleatorio positivo $T$ si
\begin{eqnarray*}
\esp\left[X\left(T+t\right)|T\right]=\esp\left[X\left(t\right)\right]\textrm{, para }t\geq0,\end{eqnarray*}
y con las esperanzas anteriores finitas.
\end{Def}

\begin{Prop}
Sup\'ongase que $X\left(t\right)$ es un proceso crudamente regenerativo en $T$, que tiene distribuci\'on $F$. Si $\esp\left[X\left(t\right)\right]$ es acotado en intervalos finitos, entonces
\begin{eqnarray*}
\esp\left[X\left(t\right)\right]=U\star h\left(t\right)\textrm{,  donde }h\left(t\right)=\esp\left[X\left(t\right)\indora\left(T>t\right)\right].
\end{eqnarray*}
\end{Prop}

\begin{Teo}[Regeneraci\'on Cruda]
Sup\'ongase que $X\left(t\right)$ es un proceso con valores positivo crudamente regenerativo en $T$, y def\'inase $M=\sup\left\{|X\left(t\right)|:t\leq T\right\}$. Si $T$ es no aritm\'etico y $M$ y $MT$ tienen media finita, entonces
\begin{eqnarray*}
lim_{t\rightarrow\infty}\esp\left[X\left(t\right)\right]=\frac{1}{\mu}\int_{\rea_{+}}h\left(s\right)ds,
\end{eqnarray*}
donde $h\left(t\right)=\esp\left[X\left(t\right)\indora\left(T>t\right)\right]$.
\end{Teo}

%___________________________________________________________________________________________
%
\subsection{Propiedades de los Procesos de Renovaci\'on}
%___________________________________________________________________________________________
%

Los tiempos $T_{n}$ est\'an relacionados con los conteos de $N\left(t\right)$ por

\begin{eqnarray*}
\left\{N\left(t\right)\geq n\right\}&=&\left\{T_{n}\leq t\right\}\\
T_{N\left(t\right)}\leq &t&<T_{N\left(t\right)+1},
\end{eqnarray*}

adem\'as $N\left(T_{n}\right)=n$, y 

\begin{eqnarray*}
N\left(t\right)=\max\left\{n:T_{n}\leq t\right\}=\min\left\{n:T_{n+1}>t\right\}
\end{eqnarray*}

Por propiedades de la convoluci\'on se sabe que

\begin{eqnarray*}
P\left\{T_{n}\leq t\right\}=F^{n\star}\left(t\right)
\end{eqnarray*}
que es la $n$-\'esima convoluci\'on de $F$. Entonces 

\begin{eqnarray*}
\left\{N\left(t\right)\geq n\right\}&=&\left\{T_{n}\leq t\right\}\\
P\left\{N\left(t\right)\leq n\right\}&=&1-F^{\left(n+1\right)\star}\left(t\right)
\end{eqnarray*}

Adem\'as usando el hecho de que $\esp\left[N\left(t\right)\right]=\sum_{n=1}^{\infty}P\left\{N\left(t\right)\geq n\right\}$
se tiene que

\begin{eqnarray*}
\esp\left[N\left(t\right)\right]=\sum_{n=1}^{\infty}F^{n\star}\left(t\right)
\end{eqnarray*}

\begin{Prop}
Para cada $t\geq0$, la funci\'on generadora de momentos $\esp\left[e^{\alpha N\left(t\right)}\right]$ existe para alguna $\alpha$ en una vecindad del 0, y de aqu\'i que $\esp\left[N\left(t\right)^{m}\right]<\infty$, para $m\geq1$.
\end{Prop}


\begin{Note}
Si el primer tiempo de renovaci\'on $\xi_{1}$ no tiene la misma distribuci\'on que el resto de las $\xi_{n}$, para $n\geq2$, a $N\left(t\right)$ se le llama Proceso de Renovaci\'on retardado, donde si $\xi$ tiene distribuci\'on $G$, entonces el tiempo $T_{n}$ de la $n$-\'esima renovaci\'on tiene distribuci\'on $G\star F^{\left(n-1\right)\star}\left(t\right)$
\end{Note}


\begin{Teo}
Para una constante $\mu\leq\infty$ ( o variable aleatoria), las siguientes expresiones son equivalentes:

\begin{eqnarray}
lim_{n\rightarrow\infty}n^{-1}T_{n}&=&\mu,\textrm{ c.s.}\\
lim_{t\rightarrow\infty}t^{-1}N\left(t\right)&=&1/\mu,\textrm{ c.s.}
\end{eqnarray}
\end{Teo}


Es decir, $T_{n}$ satisface la Ley Fuerte de los Grandes N\'umeros s\'i y s\'olo s\'i $N\left/t\right)$ la cumple.


\begin{Coro}[Ley Fuerte de los Grandes N\'umeros para Procesos de Renovaci\'on]
Si $N\left(t\right)$ es un proceso de renovaci\'on cuyos tiempos de inter-renovaci\'on tienen media $\mu\leq\infty$, entonces
\begin{eqnarray}
t^{-1}N\left(t\right)\rightarrow 1/\mu,\textrm{ c.s. cuando }t\rightarrow\infty.
\end{eqnarray}

\end{Coro}


Considerar el proceso estoc\'astico de valores reales $\left\{Z\left(t\right):t\geq0\right\}$ en el mismo espacio de probabilidad que $N\left(t\right)$

\begin{Def}
Para el proceso $\left\{Z\left(t\right):t\geq0\right\}$ se define la fluctuaci\'on m\'axima de $Z\left(t\right)$ en el intervalo $\left(T_{n-1},T_{n}\right]$:
\begin{eqnarray*}
M_{n}=\sup_{T_{n-1}<t\leq T_{n}}|Z\left(t\right)-Z\left(T_{n-1}\right)|
\end{eqnarray*}
\end{Def}

\begin{Teo}
Sup\'ongase que $n^{-1}T_{n}\rightarrow\mu$ c.s. cuando $n\rightarrow\infty$, donde $\mu\leq\infty$ es una constante o variable aleatoria. Sea $a$ una constante o variable aleatoria que puede ser infinita cuando $\mu$ es finita, y considere las expresiones l\'imite:
\begin{eqnarray}
lim_{n\rightarrow\infty}n^{-1}Z\left(T_{n}\right)&=&a,\textrm{ c.s.}\\
lim_{t\rightarrow\infty}t^{-1}Z\left(t\right)&=&a/\mu,\textrm{ c.s.}
\end{eqnarray}
La segunda expresi\'on implica la primera. Conversamente, la primera implica la segunda si el proceso $Z\left(t\right)$ es creciente, o si $lim_{n\rightarrow\infty}n^{-1}M_{n}=0$ c.s.
\end{Teo}

\begin{Coro}
Si $N\left(t\right)$ es un proceso de renovaci\'on, y $\left(Z\left(T_{n}\right)-Z\left(T_{n-1}\right),M_{n}\right)$, para $n\geq1$, son variables aleatorias independientes e id\'enticamente distribuidas con media finita, entonces,
\begin{eqnarray}
lim_{t\rightarrow\infty}t^{-1}Z\left(t\right)\rightarrow\frac{\esp\left[Z\left(T_{1}\right)-Z\left(T_{0}\right)\right]}{\esp\left[T_{1}\right]},\textrm{ c.s. cuando  }t\rightarrow\infty.
\end{eqnarray}
\end{Coro}



%___________________________________________________________________________________________
%
%\subsection{Propiedades de los Procesos de Renovaci\'on}
%___________________________________________________________________________________________
%

Los tiempos $T_{n}$ est\'an relacionados con los conteos de $N\left(t\right)$ por

\begin{eqnarray*}
\left\{N\left(t\right)\geq n\right\}&=&\left\{T_{n}\leq t\right\}\\
T_{N\left(t\right)}\leq &t&<T_{N\left(t\right)+1},
\end{eqnarray*}

adem\'as $N\left(T_{n}\right)=n$, y 

\begin{eqnarray*}
N\left(t\right)=\max\left\{n:T_{n}\leq t\right\}=\min\left\{n:T_{n+1}>t\right\}
\end{eqnarray*}

Por propiedades de la convoluci\'on se sabe que

\begin{eqnarray*}
P\left\{T_{n}\leq t\right\}=F^{n\star}\left(t\right)
\end{eqnarray*}
que es la $n$-\'esima convoluci\'on de $F$. Entonces 

\begin{eqnarray*}
\left\{N\left(t\right)\geq n\right\}&=&\left\{T_{n}\leq t\right\}\\
P\left\{N\left(t\right)\leq n\right\}&=&1-F^{\left(n+1\right)\star}\left(t\right)
\end{eqnarray*}

Adem\'as usando el hecho de que $\esp\left[N\left(t\right)\right]=\sum_{n=1}^{\infty}P\left\{N\left(t\right)\geq n\right\}$
se tiene que

\begin{eqnarray*}
\esp\left[N\left(t\right)\right]=\sum_{n=1}^{\infty}F^{n\star}\left(t\right)
\end{eqnarray*}

\begin{Prop}
Para cada $t\geq0$, la funci\'on generadora de momentos $\esp\left[e^{\alpha N\left(t\right)}\right]$ existe para alguna $\alpha$ en una vecindad del 0, y de aqu\'i que $\esp\left[N\left(t\right)^{m}\right]<\infty$, para $m\geq1$.
\end{Prop}


\begin{Note}
Si el primer tiempo de renovaci\'on $\xi_{1}$ no tiene la misma distribuci\'on que el resto de las $\xi_{n}$, para $n\geq2$, a $N\left(t\right)$ se le llama Proceso de Renovaci\'on retardado, donde si $\xi$ tiene distribuci\'on $G$, entonces el tiempo $T_{n}$ de la $n$-\'esima renovaci\'on tiene distribuci\'on $G\star F^{\left(n-1\right)\star}\left(t\right)$
\end{Note}


\begin{Teo}
Para una constante $\mu\leq\infty$ ( o variable aleatoria), las siguientes expresiones son equivalentes:

\begin{eqnarray}
lim_{n\rightarrow\infty}n^{-1}T_{n}&=&\mu,\textrm{ c.s.}\\
lim_{t\rightarrow\infty}t^{-1}N\left(t\right)&=&1/\mu,\textrm{ c.s.}
\end{eqnarray}
\end{Teo}


Es decir, $T_{n}$ satisface la Ley Fuerte de los Grandes N\'umeros s\'i y s\'olo s\'i $N\left/t\right)$ la cumple.


\begin{Coro}[Ley Fuerte de los Grandes N\'umeros para Procesos de Renovaci\'on]
Si $N\left(t\right)$ es un proceso de renovaci\'on cuyos tiempos de inter-renovaci\'on tienen media $\mu\leq\infty$, entonces
\begin{eqnarray}
t^{-1}N\left(t\right)\rightarrow 1/\mu,\textrm{ c.s. cuando }t\rightarrow\infty.
\end{eqnarray}

\end{Coro}


Considerar el proceso estoc\'astico de valores reales $\left\{Z\left(t\right):t\geq0\right\}$ en el mismo espacio de probabilidad que $N\left(t\right)$

\begin{Def}
Para el proceso $\left\{Z\left(t\right):t\geq0\right\}$ se define la fluctuaci\'on m\'axima de $Z\left(t\right)$ en el intervalo $\left(T_{n-1},T_{n}\right]$:
\begin{eqnarray*}
M_{n}=\sup_{T_{n-1}<t\leq T_{n}}|Z\left(t\right)-Z\left(T_{n-1}\right)|
\end{eqnarray*}
\end{Def}

\begin{Teo}
Sup\'ongase que $n^{-1}T_{n}\rightarrow\mu$ c.s. cuando $n\rightarrow\infty$, donde $\mu\leq\infty$ es una constante o variable aleatoria. Sea $a$ una constante o variable aleatoria que puede ser infinita cuando $\mu$ es finita, y considere las expresiones l\'imite:
\begin{eqnarray}
lim_{n\rightarrow\infty}n^{-1}Z\left(T_{n}\right)&=&a,\textrm{ c.s.}\\
lim_{t\rightarrow\infty}t^{-1}Z\left(t\right)&=&a/\mu,\textrm{ c.s.}
\end{eqnarray}
La segunda expresi\'on implica la primera. Conversamente, la primera implica la segunda si el proceso $Z\left(t\right)$ es creciente, o si $lim_{n\rightarrow\infty}n^{-1}M_{n}=0$ c.s.
\end{Teo}

\begin{Coro}
Si $N\left(t\right)$ es un proceso de renovaci\'on, y $\left(Z\left(T_{n}\right)-Z\left(T_{n-1}\right),M_{n}\right)$, para $n\geq1$, son variables aleatorias independientes e id\'enticamente distribuidas con media finita, entonces,
\begin{eqnarray}
lim_{t\rightarrow\infty}t^{-1}Z\left(t\right)\rightarrow\frac{\esp\left[Z\left(T_{1}\right)-Z\left(T_{0}\right)\right]}{\esp\left[T_{1}\right]},\textrm{ c.s. cuando  }t\rightarrow\infty.
\end{eqnarray}
\end{Coro}


%___________________________________________________________________________________________
%
%\subsection{Propiedades de los Procesos de Renovaci\'on}
%___________________________________________________________________________________________
%

Los tiempos $T_{n}$ est\'an relacionados con los conteos de $N\left(t\right)$ por

\begin{eqnarray*}
\left\{N\left(t\right)\geq n\right\}&=&\left\{T_{n}\leq t\right\}\\
T_{N\left(t\right)}\leq &t&<T_{N\left(t\right)+1},
\end{eqnarray*}

adem\'as $N\left(T_{n}\right)=n$, y 

\begin{eqnarray*}
N\left(t\right)=\max\left\{n:T_{n}\leq t\right\}=\min\left\{n:T_{n+1}>t\right\}
\end{eqnarray*}

Por propiedades de la convoluci\'on se sabe que

\begin{eqnarray*}
P\left\{T_{n}\leq t\right\}=F^{n\star}\left(t\right)
\end{eqnarray*}
que es la $n$-\'esima convoluci\'on de $F$. Entonces 

\begin{eqnarray*}
\left\{N\left(t\right)\geq n\right\}&=&\left\{T_{n}\leq t\right\}\\
P\left\{N\left(t\right)\leq n\right\}&=&1-F^{\left(n+1\right)\star}\left(t\right)
\end{eqnarray*}

Adem\'as usando el hecho de que $\esp\left[N\left(t\right)\right]=\sum_{n=1}^{\infty}P\left\{N\left(t\right)\geq n\right\}$
se tiene que

\begin{eqnarray*}
\esp\left[N\left(t\right)\right]=\sum_{n=1}^{\infty}F^{n\star}\left(t\right)
\end{eqnarray*}

\begin{Prop}
Para cada $t\geq0$, la funci\'on generadora de momentos $\esp\left[e^{\alpha N\left(t\right)}\right]$ existe para alguna $\alpha$ en una vecindad del 0, y de aqu\'i que $\esp\left[N\left(t\right)^{m}\right]<\infty$, para $m\geq1$.
\end{Prop}


\begin{Note}
Si el primer tiempo de renovaci\'on $\xi_{1}$ no tiene la misma distribuci\'on que el resto de las $\xi_{n}$, para $n\geq2$, a $N\left(t\right)$ se le llama Proceso de Renovaci\'on retardado, donde si $\xi$ tiene distribuci\'on $G$, entonces el tiempo $T_{n}$ de la $n$-\'esima renovaci\'on tiene distribuci\'on $G\star F^{\left(n-1\right)\star}\left(t\right)$
\end{Note}


\begin{Teo}
Para una constante $\mu\leq\infty$ ( o variable aleatoria), las siguientes expresiones son equivalentes:

\begin{eqnarray}
lim_{n\rightarrow\infty}n^{-1}T_{n}&=&\mu,\textrm{ c.s.}\\
lim_{t\rightarrow\infty}t^{-1}N\left(t\right)&=&1/\mu,\textrm{ c.s.}
\end{eqnarray}
\end{Teo}


Es decir, $T_{n}$ satisface la Ley Fuerte de los Grandes N\'umeros s\'i y s\'olo s\'i $N\left/t\right)$ la cumple.


\begin{Coro}[Ley Fuerte de los Grandes N\'umeros para Procesos de Renovaci\'on]
Si $N\left(t\right)$ es un proceso de renovaci\'on cuyos tiempos de inter-renovaci\'on tienen media $\mu\leq\infty$, entonces
\begin{eqnarray}
t^{-1}N\left(t\right)\rightarrow 1/\mu,\textrm{ c.s. cuando }t\rightarrow\infty.
\end{eqnarray}

\end{Coro}


Considerar el proceso estoc\'astico de valores reales $\left\{Z\left(t\right):t\geq0\right\}$ en el mismo espacio de probabilidad que $N\left(t\right)$

\begin{Def}
Para el proceso $\left\{Z\left(t\right):t\geq0\right\}$ se define la fluctuaci\'on m\'axima de $Z\left(t\right)$ en el intervalo $\left(T_{n-1},T_{n}\right]$:
\begin{eqnarray*}
M_{n}=\sup_{T_{n-1}<t\leq T_{n}}|Z\left(t\right)-Z\left(T_{n-1}\right)|
\end{eqnarray*}
\end{Def}

\begin{Teo}
Sup\'ongase que $n^{-1}T_{n}\rightarrow\mu$ c.s. cuando $n\rightarrow\infty$, donde $\mu\leq\infty$ es una constante o variable aleatoria. Sea $a$ una constante o variable aleatoria que puede ser infinita cuando $\mu$ es finita, y considere las expresiones l\'imite:
\begin{eqnarray}
lim_{n\rightarrow\infty}n^{-1}Z\left(T_{n}\right)&=&a,\textrm{ c.s.}\\
lim_{t\rightarrow\infty}t^{-1}Z\left(t\right)&=&a/\mu,\textrm{ c.s.}
\end{eqnarray}
La segunda expresi\'on implica la primera. Conversamente, la primera implica la segunda si el proceso $Z\left(t\right)$ es creciente, o si $lim_{n\rightarrow\infty}n^{-1}M_{n}=0$ c.s.
\end{Teo}

\begin{Coro}
Si $N\left(t\right)$ es un proceso de renovaci\'on, y $\left(Z\left(T_{n}\right)-Z\left(T_{n-1}\right),M_{n}\right)$, para $n\geq1$, son variables aleatorias independientes e id\'enticamente distribuidas con media finita, entonces,
\begin{eqnarray}
lim_{t\rightarrow\infty}t^{-1}Z\left(t\right)\rightarrow\frac{\esp\left[Z\left(T_{1}\right)-Z\left(T_{0}\right)\right]}{\esp\left[T_{1}\right]},\textrm{ c.s. cuando  }t\rightarrow\infty.
\end{eqnarray}
\end{Coro}

%___________________________________________________________________________________________
%
%\subsection{Propiedades de los Procesos de Renovaci\'on}
%___________________________________________________________________________________________
%

Los tiempos $T_{n}$ est\'an relacionados con los conteos de $N\left(t\right)$ por

\begin{eqnarray*}
\left\{N\left(t\right)\geq n\right\}&=&\left\{T_{n}\leq t\right\}\\
T_{N\left(t\right)}\leq &t&<T_{N\left(t\right)+1},
\end{eqnarray*}

adem\'as $N\left(T_{n}\right)=n$, y 

\begin{eqnarray*}
N\left(t\right)=\max\left\{n:T_{n}\leq t\right\}=\min\left\{n:T_{n+1}>t\right\}
\end{eqnarray*}

Por propiedades de la convoluci\'on se sabe que

\begin{eqnarray*}
P\left\{T_{n}\leq t\right\}=F^{n\star}\left(t\right)
\end{eqnarray*}
que es la $n$-\'esima convoluci\'on de $F$. Entonces 

\begin{eqnarray*}
\left\{N\left(t\right)\geq n\right\}&=&\left\{T_{n}\leq t\right\}\\
P\left\{N\left(t\right)\leq n\right\}&=&1-F^{\left(n+1\right)\star}\left(t\right)
\end{eqnarray*}

Adem\'as usando el hecho de que $\esp\left[N\left(t\right)\right]=\sum_{n=1}^{\infty}P\left\{N\left(t\right)\geq n\right\}$
se tiene que

\begin{eqnarray*}
\esp\left[N\left(t\right)\right]=\sum_{n=1}^{\infty}F^{n\star}\left(t\right)
\end{eqnarray*}

\begin{Prop}
Para cada $t\geq0$, la funci\'on generadora de momentos $\esp\left[e^{\alpha N\left(t\right)}\right]$ existe para alguna $\alpha$ en una vecindad del 0, y de aqu\'i que $\esp\left[N\left(t\right)^{m}\right]<\infty$, para $m\geq1$.
\end{Prop}


\begin{Note}
Si el primer tiempo de renovaci\'on $\xi_{1}$ no tiene la misma distribuci\'on que el resto de las $\xi_{n}$, para $n\geq2$, a $N\left(t\right)$ se le llama Proceso de Renovaci\'on retardado, donde si $\xi$ tiene distribuci\'on $G$, entonces el tiempo $T_{n}$ de la $n$-\'esima renovaci\'on tiene distribuci\'on $G\star F^{\left(n-1\right)\star}\left(t\right)$
\end{Note}


\begin{Teo}
Para una constante $\mu\leq\infty$ ( o variable aleatoria), las siguientes expresiones son equivalentes:

\begin{eqnarray}
lim_{n\rightarrow\infty}n^{-1}T_{n}&=&\mu,\textrm{ c.s.}\\
lim_{t\rightarrow\infty}t^{-1}N\left(t\right)&=&1/\mu,\textrm{ c.s.}
\end{eqnarray}
\end{Teo}


Es decir, $T_{n}$ satisface la Ley Fuerte de los Grandes N\'umeros s\'i y s\'olo s\'i $N\left/t\right)$ la cumple.


\begin{Coro}[Ley Fuerte de los Grandes N\'umeros para Procesos de Renovaci\'on]
Si $N\left(t\right)$ es un proceso de renovaci\'on cuyos tiempos de inter-renovaci\'on tienen media $\mu\leq\infty$, entonces
\begin{eqnarray}
t^{-1}N\left(t\right)\rightarrow 1/\mu,\textrm{ c.s. cuando }t\rightarrow\infty.
\end{eqnarray}

\end{Coro}


Considerar el proceso estoc\'astico de valores reales $\left\{Z\left(t\right):t\geq0\right\}$ en el mismo espacio de probabilidad que $N\left(t\right)$

\begin{Def}
Para el proceso $\left\{Z\left(t\right):t\geq0\right\}$ se define la fluctuaci\'on m\'axima de $Z\left(t\right)$ en el intervalo $\left(T_{n-1},T_{n}\right]$:
\begin{eqnarray*}
M_{n}=\sup_{T_{n-1}<t\leq T_{n}}|Z\left(t\right)-Z\left(T_{n-1}\right)|
\end{eqnarray*}
\end{Def}

\begin{Teo}
Sup\'ongase que $n^{-1}T_{n}\rightarrow\mu$ c.s. cuando $n\rightarrow\infty$, donde $\mu\leq\infty$ es una constante o variable aleatoria. Sea $a$ una constante o variable aleatoria que puede ser infinita cuando $\mu$ es finita, y considere las expresiones l\'imite:
\begin{eqnarray}
lim_{n\rightarrow\infty}n^{-1}Z\left(T_{n}\right)&=&a,\textrm{ c.s.}\\
lim_{t\rightarrow\infty}t^{-1}Z\left(t\right)&=&a/\mu,\textrm{ c.s.}
\end{eqnarray}
La segunda expresi\'on implica la primera. Conversamente, la primera implica la segunda si el proceso $Z\left(t\right)$ es creciente, o si $lim_{n\rightarrow\infty}n^{-1}M_{n}=0$ c.s.
\end{Teo}

\begin{Coro}
Si $N\left(t\right)$ es un proceso de renovaci\'on, y $\left(Z\left(T_{n}\right)-Z\left(T_{n-1}\right),M_{n}\right)$, para $n\geq1$, son variables aleatorias independientes e id\'enticamente distribuidas con media finita, entonces,
\begin{eqnarray}
lim_{t\rightarrow\infty}t^{-1}Z\left(t\right)\rightarrow\frac{\esp\left[Z\left(T_{1}\right)-Z\left(T_{0}\right)\right]}{\esp\left[T_{1}\right]},\textrm{ c.s. cuando  }t\rightarrow\infty.
\end{eqnarray}
\end{Coro}
%___________________________________________________________________________________________
%
%\subsection{Propiedades de los Procesos de Renovaci\'on}
%___________________________________________________________________________________________
%

Los tiempos $T_{n}$ est\'an relacionados con los conteos de $N\left(t\right)$ por

\begin{eqnarray*}
\left\{N\left(t\right)\geq n\right\}&=&\left\{T_{n}\leq t\right\}\\
T_{N\left(t\right)}\leq &t&<T_{N\left(t\right)+1},
\end{eqnarray*}

adem\'as $N\left(T_{n}\right)=n$, y 

\begin{eqnarray*}
N\left(t\right)=\max\left\{n:T_{n}\leq t\right\}=\min\left\{n:T_{n+1}>t\right\}
\end{eqnarray*}

Por propiedades de la convoluci\'on se sabe que

\begin{eqnarray*}
P\left\{T_{n}\leq t\right\}=F^{n\star}\left(t\right)
\end{eqnarray*}
que es la $n$-\'esima convoluci\'on de $F$. Entonces 

\begin{eqnarray*}
\left\{N\left(t\right)\geq n\right\}&=&\left\{T_{n}\leq t\right\}\\
P\left\{N\left(t\right)\leq n\right\}&=&1-F^{\left(n+1\right)\star}\left(t\right)
\end{eqnarray*}

Adem\'as usando el hecho de que $\esp\left[N\left(t\right)\right]=\sum_{n=1}^{\infty}P\left\{N\left(t\right)\geq n\right\}$
se tiene que

\begin{eqnarray*}
\esp\left[N\left(t\right)\right]=\sum_{n=1}^{\infty}F^{n\star}\left(t\right)
\end{eqnarray*}

\begin{Prop}
Para cada $t\geq0$, la funci\'on generadora de momentos $\esp\left[e^{\alpha N\left(t\right)}\right]$ existe para alguna $\alpha$ en una vecindad del 0, y de aqu\'i que $\esp\left[N\left(t\right)^{m}\right]<\infty$, para $m\geq1$.
\end{Prop}


\begin{Note}
Si el primer tiempo de renovaci\'on $\xi_{1}$ no tiene la misma distribuci\'on que el resto de las $\xi_{n}$, para $n\geq2$, a $N\left(t\right)$ se le llama Proceso de Renovaci\'on retardado, donde si $\xi$ tiene distribuci\'on $G$, entonces el tiempo $T_{n}$ de la $n$-\'esima renovaci\'on tiene distribuci\'on $G\star F^{\left(n-1\right)\star}\left(t\right)$
\end{Note}


\begin{Teo}
Para una constante $\mu\leq\infty$ ( o variable aleatoria), las siguientes expresiones son equivalentes:

\begin{eqnarray}
lim_{n\rightarrow\infty}n^{-1}T_{n}&=&\mu,\textrm{ c.s.}\\
lim_{t\rightarrow\infty}t^{-1}N\left(t\right)&=&1/\mu,\textrm{ c.s.}
\end{eqnarray}
\end{Teo}


Es decir, $T_{n}$ satisface la Ley Fuerte de los Grandes N\'umeros s\'i y s\'olo s\'i $N\left/t\right)$ la cumple.


\begin{Coro}[Ley Fuerte de los Grandes N\'umeros para Procesos de Renovaci\'on]
Si $N\left(t\right)$ es un proceso de renovaci\'on cuyos tiempos de inter-renovaci\'on tienen media $\mu\leq\infty$, entonces
\begin{eqnarray}
t^{-1}N\left(t\right)\rightarrow 1/\mu,\textrm{ c.s. cuando }t\rightarrow\infty.
\end{eqnarray}

\end{Coro}


Considerar el proceso estoc\'astico de valores reales $\left\{Z\left(t\right):t\geq0\right\}$ en el mismo espacio de probabilidad que $N\left(t\right)$

\begin{Def}
Para el proceso $\left\{Z\left(t\right):t\geq0\right\}$ se define la fluctuaci\'on m\'axima de $Z\left(t\right)$ en el intervalo $\left(T_{n-1},T_{n}\right]$:
\begin{eqnarray*}
M_{n}=\sup_{T_{n-1}<t\leq T_{n}}|Z\left(t\right)-Z\left(T_{n-1}\right)|
\end{eqnarray*}
\end{Def}

\begin{Teo}
Sup\'ongase que $n^{-1}T_{n}\rightarrow\mu$ c.s. cuando $n\rightarrow\infty$, donde $\mu\leq\infty$ es una constante o variable aleatoria. Sea $a$ una constante o variable aleatoria que puede ser infinita cuando $\mu$ es finita, y considere las expresiones l\'imite:
\begin{eqnarray}
lim_{n\rightarrow\infty}n^{-1}Z\left(T_{n}\right)&=&a,\textrm{ c.s.}\\
lim_{t\rightarrow\infty}t^{-1}Z\left(t\right)&=&a/\mu,\textrm{ c.s.}
\end{eqnarray}
La segunda expresi\'on implica la primera. Conversamente, la primera implica la segunda si el proceso $Z\left(t\right)$ es creciente, o si $lim_{n\rightarrow\infty}n^{-1}M_{n}=0$ c.s.
\end{Teo}

\begin{Coro}
Si $N\left(t\right)$ es un proceso de renovaci\'on, y $\left(Z\left(T_{n}\right)-Z\left(T_{n-1}\right),M_{n}\right)$, para $n\geq1$, son variables aleatorias independientes e id\'enticamente distribuidas con media finita, entonces,
\begin{eqnarray}
lim_{t\rightarrow\infty}t^{-1}Z\left(t\right)\rightarrow\frac{\esp\left[Z\left(T_{1}\right)-Z\left(T_{0}\right)\right]}{\esp\left[T_{1}\right]},\textrm{ c.s. cuando  }t\rightarrow\infty.
\end{eqnarray}
\end{Coro}


%___________________________________________________________________________________________
%
\subsection{Funci\'on de Renovaci\'on}
%___________________________________________________________________________________________
%


\begin{Def}
Sea $h\left(t\right)$ funci\'on de valores reales en $\rea$ acotada en intervalos finitos e igual a cero para $t<0$ La ecuaci\'on de renovaci\'on para $h\left(t\right)$ y la distribuci\'on $F$ es

\begin{eqnarray}%\label{Ec.Renovacion}
H\left(t\right)=h\left(t\right)+\int_{\left[0,t\right]}H\left(t-s\right)dF\left(s\right)\textrm{,    }t\geq0,
\end{eqnarray}
donde $H\left(t\right)$ es una funci\'on de valores reales. Esto es $H=h+F\star H$. Decimos que $H\left(t\right)$ es soluci\'on de esta ecuaci\'on si satisface la ecuaci\'on, y es acotada en intervalos finitos e iguales a cero para $t<0$.
\end{Def}

\begin{Prop}
La funci\'on $U\star h\left(t\right)$ es la \'unica soluci\'on de la ecuaci\'on de renovaci\'on (\ref{Ec.Renovacion}).
\end{Prop}

\begin{Teo}[Teorema Renovaci\'on Elemental]
\begin{eqnarray*}
t^{-1}U\left(t\right)\rightarrow 1/\mu\textrm{,    cuando }t\rightarrow\infty.
\end{eqnarray*}
\end{Teo}

%___________________________________________________________________________________________
%
%\subsection{Funci\'on de Renovaci\'on}
%___________________________________________________________________________________________
%


Sup\'ongase que $N\left(t\right)$ es un proceso de renovaci\'on con distribuci\'on $F$ con media finita $\mu$.

\begin{Def}
La funci\'on de renovaci\'on asociada con la distribuci\'on $F$, del proceso $N\left(t\right)$, es
\begin{eqnarray*}
U\left(t\right)=\sum_{n=1}^{\infty}F^{n\star}\left(t\right),\textrm{   }t\geq0,
\end{eqnarray*}
donde $F^{0\star}\left(t\right)=\indora\left(t\geq0\right)$.
\end{Def}


\begin{Prop}
Sup\'ongase que la distribuci\'on de inter-renovaci\'on $F$ tiene densidad $f$. Entonces $U\left(t\right)$ tambi\'en tiene densidad, para $t>0$, y es $U^{'}\left(t\right)=\sum_{n=0}^{\infty}f^{n\star}\left(t\right)$. Adem\'as
\begin{eqnarray*}
\prob\left\{N\left(t\right)>N\left(t-\right)\right\}=0\textrm{,   }t\geq0.
\end{eqnarray*}
\end{Prop}

\begin{Def}
La Transformada de Laplace-Stieljes de $F$ est\'a dada por

\begin{eqnarray*}
\hat{F}\left(\alpha\right)=\int_{\rea_{+}}e^{-\alpha t}dF\left(t\right)\textrm{,  }\alpha\geq0.
\end{eqnarray*}
\end{Def}

Entonces

\begin{eqnarray*}
\hat{U}\left(\alpha\right)=\sum_{n=0}^{\infty}\hat{F^{n\star}}\left(\alpha\right)=\sum_{n=0}^{\infty}\hat{F}\left(\alpha\right)^{n}=\frac{1}{1-\hat{F}\left(\alpha\right)}.
\end{eqnarray*}


\begin{Prop}
La Transformada de Laplace $\hat{U}\left(\alpha\right)$ y $\hat{F}\left(\alpha\right)$ determina una a la otra de manera \'unica por la relaci\'on $\hat{U}\left(\alpha\right)=\frac{1}{1-\hat{F}\left(\alpha\right)}$.
\end{Prop}


\begin{Note}
Un proceso de renovaci\'on $N\left(t\right)$ cuyos tiempos de inter-renovaci\'on tienen media finita, es un proceso Poisson con tasa $\lambda$ si y s\'olo s\'i $\esp\left[U\left(t\right)\right]=\lambda t$, para $t\geq0$.
\end{Note}


\begin{Teo}
Sea $N\left(t\right)$ un proceso puntual simple con puntos de localizaci\'on $T_{n}$ tal que $\eta\left(t\right)=\esp\left[N\left(\right)\right]$ es finita para cada $t$. Entonces para cualquier funci\'on $f:\rea_{+}\rightarrow\rea$,
\begin{eqnarray*}
\esp\left[\sum_{n=1}^{N\left(\right)}f\left(T_{n}\right)\right]=\int_{\left(0,t\right]}f\left(s\right)d\eta\left(s\right)\textrm{,  }t\geq0,
\end{eqnarray*}
suponiendo que la integral exista. Adem\'as si $X_{1},X_{2},\ldots$ son variables aleatorias definidas en el mismo espacio de probabilidad que el proceso $N\left(t\right)$ tal que $\esp\left[X_{n}|T_{n}=s\right]=f\left(s\right)$, independiente de $n$. Entonces
\begin{eqnarray*}
\esp\left[\sum_{n=1}^{N\left(t\right)}X_{n}\right]=\int_{\left(0,t\right]}f\left(s\right)d\eta\left(s\right)\textrm{,  }t\geq0,
\end{eqnarray*} 
suponiendo que la integral exista. 
\end{Teo}

\begin{Coro}[Identidad de Wald para Renovaciones]
Para el proceso de renovaci\'on $N\left(t\right)$,
\begin{eqnarray*}
\esp\left[T_{N\left(t\right)+1}\right]=\mu\esp\left[N\left(t\right)+1\right]\textrm{,  }t\geq0,
\end{eqnarray*}  
\end{Coro}

%______________________________________________________________________
%\subsection{Procesos de Renovaci\'on}
%______________________________________________________________________

\begin{Def}%\label{Def.Tn}
Sean $0\leq T_{1}\leq T_{2}\leq \ldots$ son tiempos aleatorios infinitos en los cuales ocurren ciertos eventos. El n\'umero de tiempos $T_{n}$ en el intervalo $\left[0,t\right)$ es

\begin{eqnarray}
N\left(t\right)=\sum_{n=1}^{\infty}\indora\left(T_{n}\leq t\right),
\end{eqnarray}
para $t\geq0$.
\end{Def}

Si se consideran los puntos $T_{n}$ como elementos de $\rea_{+}$, y $N\left(t\right)$ es el n\'umero de puntos en $\rea$. El proceso denotado por $\left\{N\left(t\right):t\geq0\right\}$, denotado por $N\left(t\right)$, es un proceso puntual en $\rea_{+}$. Los $T_{n}$ son los tiempos de ocurrencia, el proceso puntual $N\left(t\right)$ es simple si su n\'umero de ocurrencias son distintas: $0<T_{1}<T_{2}<\ldots$ casi seguramente.

\begin{Def}
Un proceso puntual $N\left(t\right)$ es un proceso de renovaci\'on si los tiempos de interocurrencia $\xi_{n}=T_{n}-T_{n-1}$, para $n\geq1$, son independientes e identicamente distribuidos con distribuci\'on $F$, donde $F\left(0\right)=0$ y $T_{0}=0$. Los $T_{n}$ son llamados tiempos de renovaci\'on, referente a la independencia o renovaci\'on de la informaci\'on estoc\'astica en estos tiempos. Los $\xi_{n}$ son los tiempos de inter-renovaci\'on, y $N\left(t\right)$ es el n\'umero de renovaciones en el intervalo $\left[0,t\right)$
\end{Def}


\begin{Note}
Para definir un proceso de renovaci\'on para cualquier contexto, solamente hay que especificar una distribuci\'on $F$, con $F\left(0\right)=0$, para los tiempos de inter-renovaci\'on. La funci\'on $F$ en turno degune las otra variables aleatorias. De manera formal, existe un espacio de probabilidad y una sucesi\'on de variables aleatorias $\xi_{1},\xi_{2},\ldots$ definidas en este con distribuci\'on $F$. Entonces las otras cantidades son $T_{n}=\sum_{k=1}^{n}\xi_{k}$ y $N\left(t\right)=\sum_{n=1}^{\infty}\indora\left(T_{n}\leq t\right)$, donde $T_{n}\rightarrow\infty$ casi seguramente por la Ley Fuerte de los Grandes Números.
\end{Note}

%___________________________________________________________________________________________
%
\subsection{Renewal and Regenerative Processes: Serfozo\cite{Serfozo}}
%___________________________________________________________________________________________
%
\begin{Def}%\label{Def.Tn}
Sean $0\leq T_{1}\leq T_{2}\leq \ldots$ son tiempos aleatorios infinitos en los cuales ocurren ciertos eventos. El n\'umero de tiempos $T_{n}$ en el intervalo $\left[0,t\right)$ es

\begin{eqnarray}
N\left(t\right)=\sum_{n=1}^{\infty}\indora\left(T_{n}\leq t\right),
\end{eqnarray}
para $t\geq0$.
\end{Def}

Si se consideran los puntos $T_{n}$ como elementos de $\rea_{+}$, y $N\left(t\right)$ es el n\'umero de puntos en $\rea$. El proceso denotado por $\left\{N\left(t\right):t\geq0\right\}$, denotado por $N\left(t\right)$, es un proceso puntual en $\rea_{+}$. Los $T_{n}$ son los tiempos de ocurrencia, el proceso puntual $N\left(t\right)$ es simple si su n\'umero de ocurrencias son distintas: $0<T_{1}<T_{2}<\ldots$ casi seguramente.

\begin{Def}
Un proceso puntual $N\left(t\right)$ es un proceso de renovaci\'on si los tiempos de interocurrencia $\xi_{n}=T_{n}-T_{n-1}$, para $n\geq1$, son independientes e identicamente distribuidos con distribuci\'on $F$, donde $F\left(0\right)=0$ y $T_{0}=0$. Los $T_{n}$ son llamados tiempos de renovaci\'on, referente a la independencia o renovaci\'on de la informaci\'on estoc\'astica en estos tiempos. Los $\xi_{n}$ son los tiempos de inter-renovaci\'on, y $N\left(t\right)$ es el n\'umero de renovaciones en el intervalo $\left[0,t\right)$
\end{Def}


\begin{Note}
Para definir un proceso de renovaci\'on para cualquier contexto, solamente hay que especificar una distribuci\'on $F$, con $F\left(0\right)=0$, para los tiempos de inter-renovaci\'on. La funci\'on $F$ en turno degune las otra variables aleatorias. De manera formal, existe un espacio de probabilidad y una sucesi\'on de variables aleatorias $\xi_{1},\xi_{2},\ldots$ definidas en este con distribuci\'on $F$. Entonces las otras cantidades son $T_{n}=\sum_{k=1}^{n}\xi_{k}$ y $N\left(t\right)=\sum_{n=1}^{\infty}\indora\left(T_{n}\leq t\right)$, donde $T_{n}\rightarrow\infty$ casi seguramente por la Ley Fuerte de los Grandes N\'umeros.
\end{Note}

Los tiempos $T_{n}$ est\'an relacionados con los conteos de $N\left(t\right)$ por

\begin{eqnarray*}
\left\{N\left(t\right)\geq n\right\}&=&\left\{T_{n}\leq t\right\}\\
T_{N\left(t\right)}\leq &t&<T_{N\left(t\right)+1},
\end{eqnarray*}

adem\'as $N\left(T_{n}\right)=n$, y 

\begin{eqnarray*}
N\left(t\right)=\max\left\{n:T_{n}\leq t\right\}=\min\left\{n:T_{n+1}>t\right\}
\end{eqnarray*}

Por propiedades de la convoluci\'on se sabe que

\begin{eqnarray*}
P\left\{T_{n}\leq t\right\}=F^{n\star}\left(t\right)
\end{eqnarray*}
que es la $n$-\'esima convoluci\'on de $F$. Entonces 

\begin{eqnarray*}
\left\{N\left(t\right)\geq n\right\}&=&\left\{T_{n}\leq t\right\}\\
P\left\{N\left(t\right)\leq n\right\}&=&1-F^{\left(n+1\right)\star}\left(t\right)
\end{eqnarray*}

Adem\'as usando el hecho de que $\esp\left[N\left(t\right)\right]=\sum_{n=1}^{\infty}P\left\{N\left(t\right)\geq n\right\}$
se tiene que

\begin{eqnarray*}
\esp\left[N\left(t\right)\right]=\sum_{n=1}^{\infty}F^{n\star}\left(t\right)
\end{eqnarray*}

\begin{Prop}
Para cada $t\geq0$, la funci\'on generadora de momentos $\esp\left[e^{\alpha N\left(t\right)}\right]$ existe para alguna $\alpha$ en una vecindad del 0, y de aqu\'i que $\esp\left[N\left(t\right)^{m}\right]<\infty$, para $m\geq1$.
\end{Prop}

\begin{Ejem}[\textbf{Proceso Poisson}]

Suponga que se tienen tiempos de inter-renovaci\'on \textit{i.i.d.} del proceso de renovaci\'on $N\left(t\right)$ tienen distribuci\'on exponencial $F\left(t\right)=q-e^{-\lambda t}$ con tasa $\lambda$. Entonces $N\left(t\right)$ es un proceso Poisson con tasa $\lambda$.

\end{Ejem}


\begin{Note}
Si el primer tiempo de renovaci\'on $\xi_{1}$ no tiene la misma distribuci\'on que el resto de las $\xi_{n}$, para $n\geq2$, a $N\left(t\right)$ se le llama Proceso de Renovaci\'on retardado, donde si $\xi$ tiene distribuci\'on $G$, entonces el tiempo $T_{n}$ de la $n$-\'esima renovaci\'on tiene distribuci\'on $G\star F^{\left(n-1\right)\star}\left(t\right)$
\end{Note}


\begin{Teo}
Para una constante $\mu\leq\infty$ ( o variable aleatoria), las siguientes expresiones son equivalentes:

\begin{eqnarray}
lim_{n\rightarrow\infty}n^{-1}T_{n}&=&\mu,\textrm{ c.s.}\\
lim_{t\rightarrow\infty}t^{-1}N\left(t\right)&=&1/\mu,\textrm{ c.s.}
\end{eqnarray}
\end{Teo}


Es decir, $T_{n}$ satisface la Ley Fuerte de los Grandes N\'umeros s\'i y s\'olo s\'i $N\left/t\right)$ la cumple.


\begin{Coro}[Ley Fuerte de los Grandes N\'umeros para Procesos de Renovaci\'on]
Si $N\left(t\right)$ es un proceso de renovaci\'on cuyos tiempos de inter-renovaci\'on tienen media $\mu\leq\infty$, entonces
\begin{eqnarray}
t^{-1}N\left(t\right)\rightarrow 1/\mu,\textrm{ c.s. cuando }t\rightarrow\infty.
\end{eqnarray}

\end{Coro}


Considerar el proceso estoc\'astico de valores reales $\left\{Z\left(t\right):t\geq0\right\}$ en el mismo espacio de probabilidad que $N\left(t\right)$

\begin{Def}
Para el proceso $\left\{Z\left(t\right):t\geq0\right\}$ se define la fluctuaci\'on m\'axima de $Z\left(t\right)$ en el intervalo $\left(T_{n-1},T_{n}\right]$:
\begin{eqnarray*}
M_{n}=\sup_{T_{n-1}<t\leq T_{n}}|Z\left(t\right)-Z\left(T_{n-1}\right)|
\end{eqnarray*}
\end{Def}

\begin{Teo}
Sup\'ongase que $n^{-1}T_{n}\rightarrow\mu$ c.s. cuando $n\rightarrow\infty$, donde $\mu\leq\infty$ es una constante o variable aleatoria. Sea $a$ una constante o variable aleatoria que puede ser infinita cuando $\mu$ es finita, y considere las expresiones l\'imite:
\begin{eqnarray}
lim_{n\rightarrow\infty}n^{-1}Z\left(T_{n}\right)&=&a,\textrm{ c.s.}\\
lim_{t\rightarrow\infty}t^{-1}Z\left(t\right)&=&a/\mu,\textrm{ c.s.}
\end{eqnarray}
La segunda expresi\'on implica la primera. Conversamente, la primera implica la segunda si el proceso $Z\left(t\right)$ es creciente, o si $lim_{n\rightarrow\infty}n^{-1}M_{n}=0$ c.s.
\end{Teo}

\begin{Coro}
Si $N\left(t\right)$ es un proceso de renovaci\'on, y $\left(Z\left(T_{n}\right)-Z\left(T_{n-1}\right),M_{n}\right)$, para $n\geq1$, son variables aleatorias independientes e id\'enticamente distribuidas con media finita, entonces,
\begin{eqnarray}
lim_{t\rightarrow\infty}t^{-1}Z\left(t\right)\rightarrow\frac{\esp\left[Z\left(T_{1}\right)-Z\left(T_{0}\right)\right]}{\esp\left[T_{1}\right]},\textrm{ c.s. cuando  }t\rightarrow\infty.
\end{eqnarray}
\end{Coro}


Sup\'ongase que $N\left(t\right)$ es un proceso de renovaci\'on con distribuci\'on $F$ con media finita $\mu$.

\begin{Def}
La funci\'on de renovaci\'on asociada con la distribuci\'on $F$, del proceso $N\left(t\right)$, es
\begin{eqnarray*}
U\left(t\right)=\sum_{n=1}^{\infty}F^{n\star}\left(t\right),\textrm{   }t\geq0,
\end{eqnarray*}
donde $F^{0\star}\left(t\right)=\indora\left(t\geq0\right)$.
\end{Def}


\begin{Prop}
Sup\'ongase que la distribuci\'on de inter-renovaci\'on $F$ tiene densidad $f$. Entonces $U\left(t\right)$ tambi\'en tiene densidad, para $t>0$, y es $U^{'}\left(t\right)=\sum_{n=0}^{\infty}f^{n\star}\left(t\right)$. Adem\'as
\begin{eqnarray*}
\prob\left\{N\left(t\right)>N\left(t-\right)\right\}=0\textrm{,   }t\geq0.
\end{eqnarray*}
\end{Prop}

\begin{Def}
La Transformada de Laplace-Stieljes de $F$ est\'a dada por

\begin{eqnarray*}
\hat{F}\left(\alpha\right)=\int_{\rea_{+}}e^{-\alpha t}dF\left(t\right)\textrm{,  }\alpha\geq0.
\end{eqnarray*}
\end{Def}

Entonces

\begin{eqnarray*}
\hat{U}\left(\alpha\right)=\sum_{n=0}^{\infty}\hat{F^{n\star}}\left(\alpha\right)=\sum_{n=0}^{\infty}\hat{F}\left(\alpha\right)^{n}=\frac{1}{1-\hat{F}\left(\alpha\right)}.
\end{eqnarray*}


\begin{Prop}
La Transformada de Laplace $\hat{U}\left(\alpha\right)$ y $\hat{F}\left(\alpha\right)$ determina una a la otra de manera \'unica por la relaci\'on $\hat{U}\left(\alpha\right)=\frac{1}{1-\hat{F}\left(\alpha\right)}$.
\end{Prop}


\begin{Note}
Un proceso de renovaci\'on $N\left(t\right)$ cuyos tiempos de inter-renovaci\'on tienen media finita, es un proceso Poisson con tasa $\lambda$ si y s\'olo s\'i $\esp\left[U\left(t\right)\right]=\lambda t$, para $t\geq0$.
\end{Note}


\begin{Teo}
Sea $N\left(t\right)$ un proceso puntual simple con puntos de localizaci\'on $T_{n}$ tal que $\eta\left(t\right)=\esp\left[N\left(\right)\right]$ es finita para cada $t$. Entonces para cualquier funci\'on $f:\rea_{+}\rightarrow\rea$,
\begin{eqnarray*}
\esp\left[\sum_{n=1}^{N\left(\right)}f\left(T_{n}\right)\right]=\int_{\left(0,t\right]}f\left(s\right)d\eta\left(s\right)\textrm{,  }t\geq0,
\end{eqnarray*}
suponiendo que la integral exista. Adem\'as si $X_{1},X_{2},\ldots$ son variables aleatorias definidas en el mismo espacio de probabilidad que el proceso $N\left(t\right)$ tal que $\esp\left[X_{n}|T_{n}=s\right]=f\left(s\right)$, independiente de $n$. Entonces
\begin{eqnarray*}
\esp\left[\sum_{n=1}^{N\left(t\right)}X_{n}\right]=\int_{\left(0,t\right]}f\left(s\right)d\eta\left(s\right)\textrm{,  }t\geq0,
\end{eqnarray*} 
suponiendo que la integral exista. 
\end{Teo}

\begin{Coro}[Identidad de Wald para Renovaciones]
Para el proceso de renovaci\'on $N\left(t\right)$,
\begin{eqnarray*}
\esp\left[T_{N\left(t\right)+1}\right]=\mu\esp\left[N\left(t\right)+1\right]\textrm{,  }t\geq0,
\end{eqnarray*}  
\end{Coro}


\begin{Def}
Sea $h\left(t\right)$ funci\'on de valores reales en $\rea$ acotada en intervalos finitos e igual a cero para $t<0$ La ecuaci\'on de renovaci\'on para $h\left(t\right)$ y la distribuci\'on $F$ es

\begin{eqnarray}%\label{Ec.Renovacion}
H\left(t\right)=h\left(t\right)+\int_{\left[0,t\right]}H\left(t-s\right)dF\left(s\right)\textrm{,    }t\geq0,
\end{eqnarray}
donde $H\left(t\right)$ es una funci\'on de valores reales. Esto es $H=h+F\star H$. Decimos que $H\left(t\right)$ es soluci\'on de esta ecuaci\'on si satisface la ecuaci\'on, y es acotada en intervalos finitos e iguales a cero para $t<0$.
\end{Def}

\begin{Prop}
La funci\'on $U\star h\left(t\right)$ es la \'unica soluci\'on de la ecuaci\'on de renovaci\'on (\ref{Ec.Renovacion}).
\end{Prop}

\begin{Teo}[Teorema Renovaci\'on Elemental]
\begin{eqnarray*}
t^{-1}U\left(t\right)\rightarrow 1/\mu\textrm{,    cuando }t\rightarrow\infty.
\end{eqnarray*}
\end{Teo}



Sup\'ongase que $N\left(t\right)$ es un proceso de renovaci\'on con distribuci\'on $F$ con media finita $\mu$.

\begin{Def}
La funci\'on de renovaci\'on asociada con la distribuci\'on $F$, del proceso $N\left(t\right)$, es
\begin{eqnarray*}
U\left(t\right)=\sum_{n=1}^{\infty}F^{n\star}\left(t\right),\textrm{   }t\geq0,
\end{eqnarray*}
donde $F^{0\star}\left(t\right)=\indora\left(t\geq0\right)$.
\end{Def}


\begin{Prop}
Sup\'ongase que la distribuci\'on de inter-renovaci\'on $F$ tiene densidad $f$. Entonces $U\left(t\right)$ tambi\'en tiene densidad, para $t>0$, y es $U^{'}\left(t\right)=\sum_{n=0}^{\infty}f^{n\star}\left(t\right)$. Adem\'as
\begin{eqnarray*}
\prob\left\{N\left(t\right)>N\left(t-\right)\right\}=0\textrm{,   }t\geq0.
\end{eqnarray*}
\end{Prop}

\begin{Def}
La Transformada de Laplace-Stieljes de $F$ est\'a dada por

\begin{eqnarray*}
\hat{F}\left(\alpha\right)=\int_{\rea_{+}}e^{-\alpha t}dF\left(t\right)\textrm{,  }\alpha\geq0.
\end{eqnarray*}
\end{Def}

Entonces

\begin{eqnarray*}
\hat{U}\left(\alpha\right)=\sum_{n=0}^{\infty}\hat{F^{n\star}}\left(\alpha\right)=\sum_{n=0}^{\infty}\hat{F}\left(\alpha\right)^{n}=\frac{1}{1-\hat{F}\left(\alpha\right)}.
\end{eqnarray*}


\begin{Prop}
La Transformada de Laplace $\hat{U}\left(\alpha\right)$ y $\hat{F}\left(\alpha\right)$ determina una a la otra de manera \'unica por la relaci\'on $\hat{U}\left(\alpha\right)=\frac{1}{1-\hat{F}\left(\alpha\right)}$.
\end{Prop}


\begin{Note}
Un proceso de renovaci\'on $N\left(t\right)$ cuyos tiempos de inter-renovaci\'on tienen media finita, es un proceso Poisson con tasa $\lambda$ si y s\'olo s\'i $\esp\left[U\left(t\right)\right]=\lambda t$, para $t\geq0$.
\end{Note}


\begin{Teo}
Sea $N\left(t\right)$ un proceso puntual simple con puntos de localizaci\'on $T_{n}$ tal que $\eta\left(t\right)=\esp\left[N\left(\right)\right]$ es finita para cada $t$. Entonces para cualquier funci\'on $f:\rea_{+}\rightarrow\rea$,
\begin{eqnarray*}
\esp\left[\sum_{n=1}^{N\left(\right)}f\left(T_{n}\right)\right]=\int_{\left(0,t\right]}f\left(s\right)d\eta\left(s\right)\textrm{,  }t\geq0,
\end{eqnarray*}
suponiendo que la integral exista. Adem\'as si $X_{1},X_{2},\ldots$ son variables aleatorias definidas en el mismo espacio de probabilidad que el proceso $N\left(t\right)$ tal que $\esp\left[X_{n}|T_{n}=s\right]=f\left(s\right)$, independiente de $n$. Entonces
\begin{eqnarray*}
\esp\left[\sum_{n=1}^{N\left(t\right)}X_{n}\right]=\int_{\left(0,t\right]}f\left(s\right)d\eta\left(s\right)\textrm{,  }t\geq0,
\end{eqnarray*} 
suponiendo que la integral exista. 
\end{Teo}

\begin{Coro}[Identidad de Wald para Renovaciones]
Para el proceso de renovaci\'on $N\left(t\right)$,
\begin{eqnarray*}
\esp\left[T_{N\left(t\right)+1}\right]=\mu\esp\left[N\left(t\right)+1\right]\textrm{,  }t\geq0,
\end{eqnarray*}  
\end{Coro}


\begin{Def}
Sea $h\left(t\right)$ funci\'on de valores reales en $\rea$ acotada en intervalos finitos e igual a cero para $t<0$ La ecuaci\'on de renovaci\'on para $h\left(t\right)$ y la distribuci\'on $F$ es

\begin{eqnarray}%\label{Ec.Renovacion}
H\left(t\right)=h\left(t\right)+\int_{\left[0,t\right]}H\left(t-s\right)dF\left(s\right)\textrm{,    }t\geq0,
\end{eqnarray}
donde $H\left(t\right)$ es una funci\'on de valores reales. Esto es $H=h+F\star H$. Decimos que $H\left(t\right)$ es soluci\'on de esta ecuaci\'on si satisface la ecuaci\'on, y es acotada en intervalos finitos e iguales a cero para $t<0$.
\end{Def}

\begin{Prop}
La funci\'on $U\star h\left(t\right)$ es la \'unica soluci\'on de la ecuaci\'on de renovaci\'on (\ref{Ec.Renovacion}).
\end{Prop}

\begin{Teo}[Teorema Renovaci\'on Elemental]
\begin{eqnarray*}
t^{-1}U\left(t\right)\rightarrow 1/\mu\textrm{,    cuando }t\rightarrow\infty.
\end{eqnarray*}
\end{Teo}


\begin{Note} Una funci\'on $h:\rea_{+}\rightarrow\rea$ es Directamente Riemann Integrable en los siguientes casos:
\begin{itemize}
\item[a)] $h\left(t\right)\geq0$ es decreciente y Riemann Integrable.
\item[b)] $h$ es continua excepto posiblemente en un conjunto de Lebesgue de medida 0, y $|h\left(t\right)|\leq b\left(t\right)$, donde $b$ es DRI.
\end{itemize}
\end{Note}

\begin{Teo}[Teorema Principal de Renovaci\'on]
Si $F$ es no aritm\'etica y $h\left(t\right)$ es Directamente Riemann Integrable (DRI), entonces

\begin{eqnarray*}
lim_{t\rightarrow\infty}U\star h=\frac{1}{\mu}\int_{\rea_{+}}h\left(s\right)ds.
\end{eqnarray*}
\end{Teo}

\begin{Prop}
Cualquier funci\'on $H\left(t\right)$ acotada en intervalos finitos y que es 0 para $t<0$ puede expresarse como
\begin{eqnarray*}
H\left(t\right)=U\star h\left(t\right)\textrm{,  donde }h\left(t\right)=H\left(t\right)-F\star H\left(t\right)
\end{eqnarray*}
\end{Prop}

\begin{Def}
Un proceso estoc\'astico $X\left(t\right)$ es crudamente regenerativo en un tiempo aleatorio positivo $T$ si
\begin{eqnarray*}
\esp\left[X\left(T+t\right)|T\right]=\esp\left[X\left(t\right)\right]\textrm{, para }t\geq0,\end{eqnarray*}
y con las esperanzas anteriores finitas.
\end{Def}

\begin{Prop}
Sup\'ongase que $X\left(t\right)$ es un proceso crudamente regenerativo en $T$, que tiene distribuci\'on $F$. Si $\esp\left[X\left(t\right)\right]$ es acotado en intervalos finitos, entonces
\begin{eqnarray*}
\esp\left[X\left(t\right)\right]=U\star h\left(t\right)\textrm{,  donde }h\left(t\right)=\esp\left[X\left(t\right)\indora\left(T>t\right)\right].
\end{eqnarray*}
\end{Prop}

\begin{Teo}[Regeneraci\'on Cruda]
Sup\'ongase que $X\left(t\right)$ es un proceso con valores positivo crudamente regenerativo en $T$, y def\'inase $M=\sup\left\{|X\left(t\right)|:t\leq T\right\}$. Si $T$ es no aritm\'etico y $M$ y $MT$ tienen media finita, entonces
\begin{eqnarray*}
lim_{t\rightarrow\infty}\esp\left[X\left(t\right)\right]=\frac{1}{\mu}\int_{\rea_{+}}h\left(s\right)ds,
\end{eqnarray*}
donde $h\left(t\right)=\esp\left[X\left(t\right)\indora\left(T>t\right)\right]$.
\end{Teo}


\begin{Note} Una funci\'on $h:\rea_{+}\rightarrow\rea$ es Directamente Riemann Integrable en los siguientes casos:
\begin{itemize}
\item[a)] $h\left(t\right)\geq0$ es decreciente y Riemann Integrable.
\item[b)] $h$ es continua excepto posiblemente en un conjunto de Lebesgue de medida 0, y $|h\left(t\right)|\leq b\left(t\right)$, donde $b$ es DRI.
\end{itemize}
\end{Note}

\begin{Teo}[Teorema Principal de Renovaci\'on]
Si $F$ es no aritm\'etica y $h\left(t\right)$ es Directamente Riemann Integrable (DRI), entonces

\begin{eqnarray*}
lim_{t\rightarrow\infty}U\star h=\frac{1}{\mu}\int_{\rea_{+}}h\left(s\right)ds.
\end{eqnarray*}
\end{Teo}

\begin{Prop}
Cualquier funci\'on $H\left(t\right)$ acotada en intervalos finitos y que es 0 para $t<0$ puede expresarse como
\begin{eqnarray*}
H\left(t\right)=U\star h\left(t\right)\textrm{,  donde }h\left(t\right)=H\left(t\right)-F\star H\left(t\right)
\end{eqnarray*}
\end{Prop}

\begin{Def}
Un proceso estoc\'astico $X\left(t\right)$ es crudamente regenerativo en un tiempo aleatorio positivo $T$ si
\begin{eqnarray*}
\esp\left[X\left(T+t\right)|T\right]=\esp\left[X\left(t\right)\right]\textrm{, para }t\geq0,\end{eqnarray*}
y con las esperanzas anteriores finitas.
\end{Def}

\begin{Prop}
Sup\'ongase que $X\left(t\right)$ es un proceso crudamente regenerativo en $T$, que tiene distribuci\'on $F$. Si $\esp\left[X\left(t\right)\right]$ es acotado en intervalos finitos, entonces
\begin{eqnarray*}
\esp\left[X\left(t\right)\right]=U\star h\left(t\right)\textrm{,  donde }h\left(t\right)=\esp\left[X\left(t\right)\indora\left(T>t\right)\right].
\end{eqnarray*}
\end{Prop}

\begin{Teo}[Regeneraci\'on Cruda]
Sup\'ongase que $X\left(t\right)$ es un proceso con valores positivo crudamente regenerativo en $T$, y def\'inase $M=\sup\left\{|X\left(t\right)|:t\leq T\right\}$. Si $T$ es no aritm\'etico y $M$ y $MT$ tienen media finita, entonces
\begin{eqnarray*}
lim_{t\rightarrow\infty}\esp\left[X\left(t\right)\right]=\frac{1}{\mu}\int_{\rea_{+}}h\left(s\right)ds,
\end{eqnarray*}
donde $h\left(t\right)=\esp\left[X\left(t\right)\indora\left(T>t\right)\right]$.
\end{Teo}

\begin{Def}
Para el proceso $\left\{\left(N\left(t\right),X\left(t\right)\right):t\geq0\right\}$, sus trayectoria muestrales en el intervalo de tiempo $\left[T_{n-1},T_{n}\right)$ est\'an descritas por
\begin{eqnarray*}
\zeta_{n}=\left(\xi_{n},\left\{X\left(T_{n-1}+t\right):0\leq t<\xi_{n}\right\}\right)
\end{eqnarray*}
Este $\zeta_{n}$ es el $n$-\'esimo segmento del proceso. El proceso es regenerativo sobre los tiempos $T_{n}$ si sus segmentos $\zeta_{n}$ son independientes e id\'enticamennte distribuidos.
\end{Def}


\begin{Note}
Si $\tilde{X}\left(t\right)$ con espacio de estados $\tilde{S}$ es regenerativo sobre $T_{n}$, entonces $X\left(t\right)=f\left(\tilde{X}\left(t\right)\right)$ tambi\'en es regenerativo sobre $T_{n}$, para cualquier funci\'on $f:\tilde{S}\rightarrow S$.
\end{Note}

\begin{Note}
Los procesos regenerativos son crudamente regenerativos, pero no al rev\'es.
\end{Note}


\begin{Note}
Un proceso estoc\'astico a tiempo continuo o discreto es regenerativo si existe un proceso de renovaci\'on  tal que los segmentos del proceso entre tiempos de renovaci\'on sucesivos son i.i.d., es decir, para $\left\{X\left(t\right):t\geq0\right\}$ proceso estoc\'astico a tiempo continuo con espacio de estados $S$, espacio m\'etrico.
\end{Note}

Para $\left\{X\left(t\right):t\geq0\right\}$ Proceso Estoc\'astico a tiempo continuo con estado de espacios $S$, que es un espacio m\'etrico, con trayectorias continuas por la derecha y con l\'imites por la izquierda c.s. Sea $N\left(t\right)$ un proceso de renovaci\'on en $\rea_{+}$ definido en el mismo espacio de probabilidad que $X\left(t\right)$, con tiempos de renovaci\'on $T$ y tiempos de inter-renovaci\'on $\xi_{n}=T_{n}-T_{n-1}$, con misma distribuci\'on $F$ de media finita $\mu$.



\begin{Def}
Para el proceso $\left\{\left(N\left(t\right),X\left(t\right)\right):t\geq0\right\}$, sus trayectoria muestrales en el intervalo de tiempo $\left[T_{n-1},T_{n}\right)$ est\'an descritas por
\begin{eqnarray*}
\zeta_{n}=\left(\xi_{n},\left\{X\left(T_{n-1}+t\right):0\leq t<\xi_{n}\right\}\right)
\end{eqnarray*}
Este $\zeta_{n}$ es el $n$-\'esimo segmento del proceso. El proceso es regenerativo sobre los tiempos $T_{n}$ si sus segmentos $\zeta_{n}$ son independientes e id\'enticamennte distribuidos.
\end{Def}

\begin{Note}
Un proceso regenerativo con media de la longitud de ciclo finita es llamado positivo recurrente.
\end{Note}

\begin{Teo}[Procesos Regenerativos]
Suponga que el proceso
\end{Teo}


\begin{Def}[Renewal Process Trinity]
Para un proceso de renovaci\'on $N\left(t\right)$, los siguientes procesos proveen de informaci\'on sobre los tiempos de renovaci\'on.
\begin{itemize}
\item $A\left(t\right)=t-T_{N\left(t\right)}$, el tiempo de recurrencia hacia atr\'as al tiempo $t$, que es el tiempo desde la \'ultima renovaci\'on para $t$.

\item $B\left(t\right)=T_{N\left(t\right)+1}-t$, el tiempo de recurrencia hacia adelante al tiempo $t$, residual del tiempo de renovaci\'on, que es el tiempo para la pr\'oxima renovaci\'on despu\'es de $t$.

\item $L\left(t\right)=\xi_{N\left(t\right)+1}=A\left(t\right)+B\left(t\right)$, la longitud del intervalo de renovaci\'on que contiene a $t$.
\end{itemize}
\end{Def}

\begin{Note}
El proceso tridimensional $\left(A\left(t\right),B\left(t\right),L\left(t\right)\right)$ es regenerativo sobre $T_{n}$, y por ende cada proceso lo es. Cada proceso $A\left(t\right)$ y $B\left(t\right)$ son procesos de MArkov a tiempo continuo con trayectorias continuas por partes en el espacio de estados $\rea_{+}$. Una expresi\'on conveniente para su distribuci\'on conjunta es, para $0\leq x<t,y\geq0$
\begin{equation}\label{NoRenovacion}
P\left\{A\left(t\right)>x,B\left(t\right)>y\right\}=
P\left\{N\left(t+y\right)-N\left((t-x)\right)=0\right\}
\end{equation}
\end{Note}

\begin{Ejem}[Tiempos de recurrencia Poisson]
Si $N\left(t\right)$ es un proceso Poisson con tasa $\lambda$, entonces de la expresi\'on (\ref{NoRenovacion}) se tiene que

\begin{eqnarray*}
\begin{array}{lc}
P\left\{A\left(t\right)>x,B\left(t\right)>y\right\}=e^{-\lambda\left(x+y\right)},&0\leq x<t,y\geq0,
\end{array}
\end{eqnarray*}
que es la probabilidad Poisson de no renovaciones en un intervalo de longitud $x+y$.

\end{Ejem}

%\begin{Note}
Una cadena de Markov erg\'odica tiene la propiedad de ser estacionaria si la distribuci\'on de su estado al tiempo $0$ es su distribuci\'on estacionaria.
%\end{Note}


\begin{Def}
Un proceso estoc\'astico a tiempo continuo $\left\{X\left(t\right):t\geq0\right\}$ en un espacio general es estacionario si sus distribuciones finito dimensionales son invariantes bajo cualquier  traslado: para cada $0\leq s_{1}<s_{2}<\cdots<s_{k}$ y $t\geq0$,
\begin{eqnarray*}
\left(X\left(s_{1}+t\right),\ldots,X\left(s_{k}+t\right)\right)=_{d}\left(X\left(s_{1}\right),\ldots,X\left(s_{k}\right)\right).
\end{eqnarray*}
\end{Def}

\begin{Note}
Un proceso de Markov es estacionario si $X\left(t\right)=_{d}X\left(0\right)$, $t\geq0$.
\end{Note}

Considerese el proceso $N\left(t\right)=\sum_{n}\indora\left(\tau_{n}\leq t\right)$ en $\rea_{+}$, con puntos $0<\tau_{1}<\tau_{2}<\cdots$.

\begin{Prop}
Si $N$ es un proceso puntual estacionario y $\esp\left[N\left(1\right)\right]<\infty$, entonces $\esp\left[N\left(t\right)\right]=t\esp\left[N\left(1\right)\right]$, $t\geq0$

\end{Prop}

\begin{Teo}
Los siguientes enunciados son equivalentes
\begin{itemize}
\item[i)] El proceso retardado de renovaci\'on $N$ es estacionario.

\item[ii)] EL proceso de tiempos de recurrencia hacia adelante $B\left(t\right)$ es estacionario.


\item[iii)] $\esp\left[N\left(t\right)\right]=t/\mu$,


\item[iv)] $G\left(t\right)=F_{e}\left(t\right)=\frac{1}{\mu}\int_{0}^{t}\left[1-F\left(s\right)\right]ds$
\end{itemize}
Cuando estos enunciados son ciertos, $P\left\{B\left(t\right)\leq x\right\}=F_{e}\left(x\right)$, para $t,x\geq0$.

\end{Teo}

\begin{Note}
Una consecuencia del teorema anterior es que el Proceso Poisson es el \'unico proceso sin retardo que es estacionario.
\end{Note}

\begin{Coro}
El proceso de renovaci\'on $N\left(t\right)$ sin retardo, y cuyos tiempos de inter renonaci\'on tienen media finita, es estacionario si y s\'olo si es un proceso Poisson.

\end{Coro}


%________________________________________________________________________
\subsection{Procesos Regenerativos}
%________________________________________________________________________



\begin{Note}
Si $\tilde{X}\left(t\right)$ con espacio de estados $\tilde{S}$ es regenerativo sobre $T_{n}$, entonces $X\left(t\right)=f\left(\tilde{X}\left(t\right)\right)$ tambi\'en es regenerativo sobre $T_{n}$, para cualquier funci\'on $f:\tilde{S}\rightarrow S$.
\end{Note}

\begin{Note}
Los procesos regenerativos son crudamente regenerativos, pero no al rev\'es.
\end{Note}
%\subsection*{Procesos Regenerativos: Sigman\cite{Sigman1}}
\begin{Def}[Definici\'on Cl\'asica]
Un proceso estoc\'astico $X=\left\{X\left(t\right):t\geq0\right\}$ es llamado regenerativo is existe una variable aleatoria $R_{1}>0$ tal que
\begin{itemize}
\item[i)] $\left\{X\left(t+R_{1}\right):t\geq0\right\}$ es independiente de $\left\{\left\{X\left(t\right):t<R_{1}\right\},\right\}$
\item[ii)] $\left\{X\left(t+R_{1}\right):t\geq0\right\}$ es estoc\'asticamente equivalente a $\left\{X\left(t\right):t>0\right\}$
\end{itemize}

Llamamos a $R_{1}$ tiempo de regeneraci\'on, y decimos que $X$ se regenera en este punto.
\end{Def}

$\left\{X\left(t+R_{1}\right)\right\}$ es regenerativo con tiempo de regeneraci\'on $R_{2}$, independiente de $R_{1}$ pero con la misma distribuci\'on que $R_{1}$. Procediendo de esta manera se obtiene una secuencia de variables aleatorias independientes e id\'enticamente distribuidas $\left\{R_{n}\right\}$ llamados longitudes de ciclo. Si definimos a $Z_{k}\equiv R_{1}+R_{2}+\cdots+R_{k}$, se tiene un proceso de renovaci\'on llamado proceso de renovaci\'on encajado para $X$.




\begin{Def}
Para $x$ fijo y para cada $t\geq0$, sea $I_{x}\left(t\right)=1$ si $X\left(t\right)\leq x$,  $I_{x}\left(t\right)=0$ en caso contrario, y def\'inanse los tiempos promedio
\begin{eqnarray*}
\overline{X}&=&lim_{t\rightarrow\infty}\frac{1}{t}\int_{0}^{\infty}X\left(u\right)du\\
\prob\left(X_{\infty}\leq x\right)&=&lim_{t\rightarrow\infty}\frac{1}{t}\int_{0}^{\infty}I_{x}\left(u\right)du,
\end{eqnarray*}
cuando estos l\'imites existan.
\end{Def}

Como consecuencia del teorema de Renovaci\'on-Recompensa, se tiene que el primer l\'imite  existe y es igual a la constante
\begin{eqnarray*}
\overline{X}&=&\frac{\esp\left[\int_{0}^{R_{1}}X\left(t\right)dt\right]}{\esp\left[R_{1}\right]},
\end{eqnarray*}
suponiendo que ambas esperanzas son finitas.

\begin{Note}
\begin{itemize}
\item[a)] Si el proceso regenerativo $X$ es positivo recurrente y tiene trayectorias muestrales no negativas, entonces la ecuaci\'on anterior es v\'alida.
\item[b)] Si $X$ es positivo recurrente regenerativo, podemos construir una \'unica versi\'on estacionaria de este proceso, $X_{e}=\left\{X_{e}\left(t\right)\right\}$, donde $X_{e}$ es un proceso estoc\'astico regenerativo y estrictamente estacionario, con distribuci\'on marginal distribuida como $X_{\infty}$
\end{itemize}
\end{Note}

%________________________________________________________________________
%\subsection{Procesos Regenerativos}
%________________________________________________________________________

Para $\left\{X\left(t\right):t\geq0\right\}$ Proceso Estoc\'astico a tiempo continuo con estado de espacios $S$, que es un espacio m\'etrico, con trayectorias continuas por la derecha y con l\'imites por la izquierda c.s. Sea $N\left(t\right)$ un proceso de renovaci\'on en $\rea_{+}$ definido en el mismo espacio de probabilidad que $X\left(t\right)$, con tiempos de renovaci\'on $T$ y tiempos de inter-renovaci\'on $\xi_{n}=T_{n}-T_{n-1}$, con misma distribuci\'on $F$ de media finita $\mu$.



\begin{Def}
Para el proceso $\left\{\left(N\left(t\right),X\left(t\right)\right):t\geq0\right\}$, sus trayectoria muestrales en el intervalo de tiempo $\left[T_{n-1},T_{n}\right)$ est\'an descritas por
\begin{eqnarray*}
\zeta_{n}=\left(\xi_{n},\left\{X\left(T_{n-1}+t\right):0\leq t<\xi_{n}\right\}\right)
\end{eqnarray*}
Este $\zeta_{n}$ es el $n$-\'esimo segmento del proceso. El proceso es regenerativo sobre los tiempos $T_{n}$ si sus segmentos $\zeta_{n}$ son independientes e id\'enticamennte distribuidos.
\end{Def}


\begin{Note}
Si $\tilde{X}\left(t\right)$ con espacio de estados $\tilde{S}$ es regenerativo sobre $T_{n}$, entonces $X\left(t\right)=f\left(\tilde{X}\left(t\right)\right)$ tambi\'en es regenerativo sobre $T_{n}$, para cualquier funci\'on $f:\tilde{S}\rightarrow S$.
\end{Note}

\begin{Note}
Los procesos regenerativos son crudamente regenerativos, pero no al rev\'es.
\end{Note}

\begin{Def}[Definici\'on Cl\'asica]
Un proceso estoc\'astico $X=\left\{X\left(t\right):t\geq0\right\}$ es llamado regenerativo is existe una variable aleatoria $R_{1}>0$ tal que
\begin{itemize}
\item[i)] $\left\{X\left(t+R_{1}\right):t\geq0\right\}$ es independiente de $\left\{\left\{X\left(t\right):t<R_{1}\right\},\right\}$
\item[ii)] $\left\{X\left(t+R_{1}\right):t\geq0\right\}$ es estoc\'asticamente equivalente a $\left\{X\left(t\right):t>0\right\}$
\end{itemize}

Llamamos a $R_{1}$ tiempo de regeneraci\'on, y decimos que $X$ se regenera en este punto.
\end{Def}

$\left\{X\left(t+R_{1}\right)\right\}$ es regenerativo con tiempo de regeneraci\'on $R_{2}$, independiente de $R_{1}$ pero con la misma distribuci\'on que $R_{1}$. Procediendo de esta manera se obtiene una secuencia de variables aleatorias independientes e id\'enticamente distribuidas $\left\{R_{n}\right\}$ llamados longitudes de ciclo. Si definimos a $Z_{k}\equiv R_{1}+R_{2}+\cdots+R_{k}$, se tiene un proceso de renovaci\'on llamado proceso de renovaci\'on encajado para $X$.

\begin{Note}
Un proceso regenerativo con media de la longitud de ciclo finita es llamado positivo recurrente.
\end{Note}


\begin{Def}
Para $x$ fijo y para cada $t\geq0$, sea $I_{x}\left(t\right)=1$ si $X\left(t\right)\leq x$,  $I_{x}\left(t\right)=0$ en caso contrario, y def\'inanse los tiempos promedio
\begin{eqnarray*}
\overline{X}&=&lim_{t\rightarrow\infty}\frac{1}{t}\int_{0}^{\infty}X\left(u\right)du\\
\prob\left(X_{\infty}\leq x\right)&=&lim_{t\rightarrow\infty}\frac{1}{t}\int_{0}^{\infty}I_{x}\left(u\right)du,
\end{eqnarray*}
cuando estos l\'imites existan.
\end{Def}

Como consecuencia del teorema de Renovaci\'on-Recompensa, se tiene que el primer l\'imite  existe y es igual a la constante
\begin{eqnarray*}
\overline{X}&=&\frac{\esp\left[\int_{0}^{R_{1}}X\left(t\right)dt\right]}{\esp\left[R_{1}\right]},
\end{eqnarray*}
suponiendo que ambas esperanzas son finitas.

\begin{Note}
\begin{itemize}
\item[a)] Si el proceso regenerativo $X$ es positivo recurrente y tiene trayectorias muestrales no negativas, entonces la ecuaci\'on anterior es v\'alida.
\item[b)] Si $X$ es positivo recurrente regenerativo, podemos construir una \'unica versi\'on estacionaria de este proceso, $X_{e}=\left\{X_{e}\left(t\right)\right\}$, donde $X_{e}$ es un proceso estoc\'astico regenerativo y estrictamente estacionario, con distribuci\'on marginal distribuida como $X_{\infty}$
\end{itemize}
\end{Note}

%__________________________________________________________________________________________
%\subsection{Procesos Regenerativos Estacionarios - Stidham \cite{Stidham}}
%__________________________________________________________________________________________


Un proceso estoc\'astico a tiempo continuo $\left\{V\left(t\right),t\geq0\right\}$ es un proceso regenerativo si existe una sucesi\'on de variables aleatorias independientes e id\'enticamente distribuidas $\left\{X_{1},X_{2},\ldots\right\}$, sucesi\'on de renovaci\'on, tal que para cualquier conjunto de Borel $A$, 

\begin{eqnarray*}
\prob\left\{V\left(t\right)\in A|X_{1}+X_{2}+\cdots+X_{R\left(t\right)}=s,\left\{V\left(\tau\right),\tau<s\right\}\right\}=\prob\left\{V\left(t-s\right)\in A|X_{1}>t-s\right\},
\end{eqnarray*}
para todo $0\leq s\leq t$, donde $R\left(t\right)=\max\left\{X_{1}+X_{2}+\cdots+X_{j}\leq t\right\}=$n\'umero de renovaciones ({\emph{puntos de regeneraci\'on}}) que ocurren en $\left[0,t\right]$. El intervalo $\left[0,X_{1}\right)$ es llamado {\emph{primer ciclo de regeneraci\'on}} de $\left\{V\left(t \right),t\geq0\right\}$, $\left[X_{1},X_{1}+X_{2}\right)$ el {\emph{segundo ciclo de regeneraci\'on}}, y as\'i sucesivamente.

Sea $X=X_{1}$ y sea $F$ la funci\'on de distrbuci\'on de $X$


\begin{Def}
Se define el proceso estacionario, $\left\{V^{*}\left(t\right),t\geq0\right\}$, para $\left\{V\left(t\right),t\geq0\right\}$ por

\begin{eqnarray*}
\prob\left\{V\left(t\right)\in A\right\}=\frac{1}{\esp\left[X\right]}\int_{0}^{\infty}\prob\left\{V\left(t+x\right)\in A|X>x\right\}\left(1-F\left(x\right)\right)dx,
\end{eqnarray*} 
para todo $t\geq0$ y todo conjunto de Borel $A$.
\end{Def}

\begin{Def}
Una distribuci\'on se dice que es {\emph{aritm\'etica}} si todos sus puntos de incremento son m\'ultiplos de la forma $0,\lambda, 2\lambda,\ldots$ para alguna $\lambda>0$ entera.
\end{Def}


\begin{Def}
Una modificaci\'on medible de un proceso $\left\{V\left(t\right),t\geq0\right\}$, es una versi\'on de este, $\left\{V\left(t,w\right)\right\}$ conjuntamente medible para $t\geq0$ y para $w\in S$, $S$ espacio de estados para $\left\{V\left(t\right),t\geq0\right\}$.
\end{Def}

\begin{Teo}
Sea $\left\{V\left(t\right),t\geq\right\}$ un proceso regenerativo no negativo con modificaci\'on medible. Sea $\esp\left[X\right]<\infty$. Entonces el proceso estacionario dado por la ecuaci\'on anterior est\'a bien definido y tiene funci\'on de distribuci\'on independiente de $t$, adem\'as
\begin{itemize}
\item[i)] \begin{eqnarray*}
\esp\left[V^{*}\left(0\right)\right]&=&\frac{\esp\left[\int_{0}^{X}V\left(s\right)ds\right]}{\esp\left[X\right]}\end{eqnarray*}
\item[ii)] Si $\esp\left[V^{*}\left(0\right)\right]<\infty$, equivalentemente, si $\esp\left[\int_{0}^{X}V\left(s\right)ds\right]<\infty$,entonces
\begin{eqnarray*}
\frac{\int_{0}^{t}V\left(s\right)ds}{t}\rightarrow\frac{\esp\left[\int_{0}^{X}V\left(s\right)ds\right]}{\esp\left[X\right]}
\end{eqnarray*}
con probabilidad 1 y en media, cuando $t\rightarrow\infty$.
\end{itemize}
\end{Teo}

%__________________________________________________________________________________________
%\subsection{Procesos Regenerativos Estacionarios - Stidham \cite{Stidham}}
%__________________________________________________________________________________________


Un proceso estoc\'astico a tiempo continuo $\left\{V\left(t\right),t\geq0\right\}$ es un proceso regenerativo si existe una sucesi\'on de variables aleatorias independientes e id\'enticamente distribuidas $\left\{X_{1},X_{2},\ldots\right\}$, sucesi\'on de renovaci\'on, tal que para cualquier conjunto de Borel $A$, 

\begin{eqnarray*}
\prob\left\{V\left(t\right)\in A|X_{1}+X_{2}+\cdots+X_{R\left(t\right)}=s,\left\{V\left(\tau\right),\tau<s\right\}\right\}=\prob\left\{V\left(t-s\right)\in A|X_{1}>t-s\right\},
\end{eqnarray*}
para todo $0\leq s\leq t$, donde $R\left(t\right)=\max\left\{X_{1}+X_{2}+\cdots+X_{j}\leq t\right\}=$n\'umero de renovaciones ({\emph{puntos de regeneraci\'on}}) que ocurren en $\left[0,t\right]$. El intervalo $\left[0,X_{1}\right)$ es llamado {\emph{primer ciclo de regeneraci\'on}} de $\left\{V\left(t \right),t\geq0\right\}$, $\left[X_{1},X_{1}+X_{2}\right)$ el {\emph{segundo ciclo de regeneraci\'on}}, y as\'i sucesivamente.

Sea $X=X_{1}$ y sea $F$ la funci\'on de distrbuci\'on de $X$


\begin{Def}
Se define el proceso estacionario, $\left\{V^{*}\left(t\right),t\geq0\right\}$, para $\left\{V\left(t\right),t\geq0\right\}$ por

\begin{eqnarray*}
\prob\left\{V\left(t\right)\in A\right\}=\frac{1}{\esp\left[X\right]}\int_{0}^{\infty}\prob\left\{V\left(t+x\right)\in A|X>x\right\}\left(1-F\left(x\right)\right)dx,
\end{eqnarray*} 
para todo $t\geq0$ y todo conjunto de Borel $A$.
\end{Def}

\begin{Def}
Una distribuci\'on se dice que es {\emph{aritm\'etica}} si todos sus puntos de incremento son m\'ultiplos de la forma $0,\lambda, 2\lambda,\ldots$ para alguna $\lambda>0$ entera.
\end{Def}


\begin{Def}
Una modificaci\'on medible de un proceso $\left\{V\left(t\right),t\geq0\right\}$, es una versi\'on de este, $\left\{V\left(t,w\right)\right\}$ conjuntamente medible para $t\geq0$ y para $w\in S$, $S$ espacio de estados para $\left\{V\left(t\right),t\geq0\right\}$.
\end{Def}

\begin{Teo}
Sea $\left\{V\left(t\right),t\geq\right\}$ un proceso regenerativo no negativo con modificaci\'on medible. Sea $\esp\left[X\right]<\infty$. Entonces el proceso estacionario dado por la ecuaci\'on anterior est\'a bien definido y tiene funci\'on de distribuci\'on independiente de $t$, adem\'as
\begin{itemize}
\item[i)] \begin{eqnarray*}
\esp\left[V^{*}\left(0\right)\right]&=&\frac{\esp\left[\int_{0}^{X}V\left(s\right)ds\right]}{\esp\left[X\right]}\end{eqnarray*}
\item[ii)] Si $\esp\left[V^{*}\left(0\right)\right]<\infty$, equivalentemente, si $\esp\left[\int_{0}^{X}V\left(s\right)ds\right]<\infty$,entonces
\begin{eqnarray*}
\frac{\int_{0}^{t}V\left(s\right)ds}{t}\rightarrow\frac{\esp\left[\int_{0}^{X}V\left(s\right)ds\right]}{\esp\left[X\right]}
\end{eqnarray*}
con probabilidad 1 y en media, cuando $t\rightarrow\infty$.
\end{itemize}
\end{Teo}

Para $\left\{X\left(t\right):t\geq0\right\}$ Proceso Estoc\'astico a tiempo continuo con estado de espacios $S$, que es un espacio m\'etrico, con trayectorias continuas por la derecha y con l\'imites por la izquierda c.s. Sea $N\left(t\right)$ un proceso de renovaci\'on en $\rea_{+}$ definido en el mismo espacio de probabilidad que $X\left(t\right)$, con tiempos de renovaci\'on $T$ y tiempos de inter-renovaci\'on $\xi_{n}=T_{n}-T_{n-1}$, con misma distribuci\'on $F$ de media finita $\mu$.
%_____________________________________________________
\subsection{Puntos de Renovaci\'on}
%_____________________________________________________

Para cada cola $Q_{i}$ se tienen los procesos de arribo a la cola, para estas, los tiempos de arribo est\'an dados por $$\left\{T_{1}^{i},T_{2}^{i},\ldots,T_{k}^{i},\ldots\right\},$$ entonces, consideremos solamente los primeros tiempos de arribo a cada una de las colas, es decir, $$\left\{T_{1}^{1},T_{1}^{2},T_{1}^{3},T_{1}^{4}\right\},$$ se sabe que cada uno de estos tiempos se distribuye de manera exponencial con par\'ametro $1/mu_{i}$. Adem\'as se sabe que para $$T^{*}=\min\left\{T_{1}^{1},T_{1}^{2},T_{1}^{3},T_{1}^{4}\right\},$$ $T^{*}$ se distribuye de manera exponencial con par\'ametro $$\mu^{*}=\sum_{i=1}^{4}\mu_{i}.$$ Ahora, dado que 
\begin{center}
\begin{tabular}{lcl}
$\tilde{r}=r_{1}+r_{2}$ & y &$\hat{r}=r_{3}+r_{4}.$
\end{tabular}
\end{center}


Supongamos que $$\tilde{r},\hat{r}<\mu^{*},$$ entonces si tomamos $$r^{*}=\min\left\{\tilde{r},\hat{r}\right\},$$ se tiene que para  $$t^{*}\in\left(0,r^{*}\right)$$ se cumple que 
\begin{center}
\begin{tabular}{lcl}
$\tau_{1}\left(1\right)=0$ & y por tanto & $\overline{\tau}_{1}=0,$
\end{tabular}
\end{center}
entonces para la segunda cola en este primer ciclo se cumple que $$\tau_{2}=\overline{\tau}_{1}+r_{1}=r_{1}<\mu^{*},$$ y por tanto se tiene que  $$\overline{\tau}_{2}=\tau_{2}.$$ Por lo tanto, nuevamente para la primer cola en el segundo ciclo $$\tau_{1}\left(2\right)=\tau_{2}\left(1\right)+r_{2}=\tilde{r}<\mu^{*}.$$ An\'alogamente para el segundo sistema se tiene que ambas colas est\'an vac\'ias, es decir, existe un valor $t^{*}$ tal que en el intervalo $\left(0,t^{*}\right)$ no ha llegado ning\'un usuario, es decir, $$L_{i}\left(t^{*}\right)=0$$ para $i=1,2,3,4$.





%________________________________________________________________________
\subsection{Procesos Regenerativos}
%________________________________________________________________________

Para $\left\{X\left(t\right):t\geq0\right\}$ Proceso Estoc\'astico a tiempo continuo con estado de espacios $S$, que es un espacio m\'etrico, con trayectorias continuas por la derecha y con l\'imites por la izquierda c.s. Sea $N\left(t\right)$ un proceso de renovaci\'on en $\rea_{+}$ definido en el mismo espacio de probabilidad que $X\left(t\right)$, con tiempos de renovaci\'on $T$ y tiempos de inter-renovaci\'on $\xi_{n}=T_{n}-T_{n-1}$, con misma distribuci\'on $F$ de media finita $\mu$.



\begin{Def}
Para el proceso $\left\{\left(N\left(t\right),X\left(t\right)\right):t\geq0\right\}$, sus trayectoria muestrales en el intervalo de tiempo $\left[T_{n-1},T_{n}\right)$ est\'an descritas por
\begin{eqnarray*}
\zeta_{n}=\left(\xi_{n},\left\{X\left(T_{n-1}+t\right):0\leq t<\xi_{n}\right\}\right)
\end{eqnarray*}
Este $\zeta_{n}$ es el $n$-\'esimo segmento del proceso. El proceso es regenerativo sobre los tiempos $T_{n}$ si sus segmentos $\zeta_{n}$ son independientes e id\'enticamennte distribuidos.
\end{Def}


\begin{Obs}
Si $\tilde{X}\left(t\right)$ con espacio de estados $\tilde{S}$ es regenerativo sobre $T_{n}$, entonces $X\left(t\right)=f\left(\tilde{X}\left(t\right)\right)$ tambi\'en es regenerativo sobre $T_{n}$, para cualquier funci\'on $f:\tilde{S}\rightarrow S$.
\end{Obs}

\begin{Obs}
Los procesos regenerativos son crudamente regenerativos, pero no al rev\'es.
\end{Obs}

\begin{Def}[Definici\'on Cl\'asica]
Un proceso estoc\'astico $X=\left\{X\left(t\right):t\geq0\right\}$ es llamado regenerativo is existe una variable aleatoria $R_{1}>0$ tal que
\begin{itemize}
\item[i)] $\left\{X\left(t+R_{1}\right):t\geq0\right\}$ es independiente de $\left\{\left\{X\left(t\right):t<R_{1}\right\},\right\}$
\item[ii)] $\left\{X\left(t+R_{1}\right):t\geq0\right\}$ es estoc\'asticamente equivalente a $\left\{X\left(t\right):t>0\right\}$
\end{itemize}

Llamamos a $R_{1}$ tiempo de regeneraci\'on, y decimos que $X$ se regenera en este punto.
\end{Def}

$\left\{X\left(t+R_{1}\right)\right\}$ es regenerativo con tiempo de regeneraci\'on $R_{2}$, independiente de $R_{1}$ pero con la misma distribuci\'on que $R_{1}$. Procediendo de esta manera se obtiene una secuencia de variables aleatorias independientes e id\'enticamente distribuidas $\left\{R_{n}\right\}$ llamados longitudes de ciclo. Si definimos a $Z_{k}\equiv R_{1}+R_{2}+\cdots+R_{k}$, se tiene un proceso de renovaci\'on llamado proceso de renovaci\'on encajado para $X$.

\begin{Note}
Un proceso regenerativo con media de la longitud de ciclo finita es llamado positivo recurrente.
\end{Note}


\begin{Def}
Para $x$ fijo y para cada $t\geq0$, sea $I_{x}\left(t\right)=1$ si $X\left(t\right)\leq x$,  $I_{x}\left(t\right)=0$ en caso contrario, y def\'inanse los tiempos promedio
\begin{eqnarray*}
\overline{X}&=&lim_{t\rightarrow\infty}\frac{1}{t}\int_{0}^{\infty}X\left(u\right)du\\
\prob\left(X_{\infty}\leq x\right)&=&lim_{t\rightarrow\infty}\frac{1}{t}\int_{0}^{\infty}I_{x}\left(u\right)du,
\end{eqnarray*}
cuando estos l\'imites existan.
\end{Def}

Como consecuencia del teorema de Renovaci\'on-Recompensa, se tiene que el primer l\'imite  existe y es igual a la constante
\begin{eqnarray*}
\overline{X}&=&\frac{\esp\left[\int_{0}^{R_{1}}X\left(t\right)dt\right]}{\esp\left[R_{1}\right]},
\end{eqnarray*}
suponiendo que ambas esperanzas son finitas.

\begin{Note}
\begin{itemize}
\item[a)] Si el proceso regenerativo $X$ es positivo recurrente y tiene trayectorias muestrales no negativas, entonces la ecuaci\'on anterior es v\'alida.
\item[b)] Si $X$ es positivo recurrente regenerativo, podemos construir una \'unica versi\'on estacionaria de este proceso, $X_{e}=\left\{X_{e}\left(t\right)\right\}$, donde $X_{e}$ es un proceso estoc\'astico regenerativo y estrictamente estacionario, con distribuci\'on marginal distribuida como $X_{\infty}$
\end{itemize}
\end{Note}

\subsection{Renewal and Regenerative Processes: Serfozo\cite{Serfozo}}
\begin{Def}\label{Def.Tn}
Sean $0\leq T_{1}\leq T_{2}\leq \ldots$ son tiempos aleatorios infinitos en los cuales ocurren ciertos eventos. El n\'umero de tiempos $T_{n}$ en el intervalo $\left[0,t\right)$ es

\begin{eqnarray}
N\left(t\right)=\sum_{n=1}^{\infty}\indora\left(T_{n}\leq t\right),
\end{eqnarray}
para $t\geq0$.
\end{Def}

Si se consideran los puntos $T_{n}$ como elementos de $\rea_{+}$, y $N\left(t\right)$ es el n\'umero de puntos en $\rea$. El proceso denotado por $\left\{N\left(t\right):t\geq0\right\}$, denotado por $N\left(t\right)$, es un proceso puntual en $\rea_{+}$. Los $T_{n}$ son los tiempos de ocurrencia, el proceso puntual $N\left(t\right)$ es simple si su n\'umero de ocurrencias son distintas: $0<T_{1}<T_{2}<\ldots$ casi seguramente.

\begin{Def}
Un proceso puntual $N\left(t\right)$ es un proceso de renovaci\'on si los tiempos de interocurrencia $\xi_{n}=T_{n}-T_{n-1}$, para $n\geq1$, son independientes e identicamente distribuidos con distribuci\'on $F$, donde $F\left(0\right)=0$ y $T_{0}=0$. Los $T_{n}$ son llamados tiempos de renovaci\'on, referente a la independencia o renovaci\'on de la informaci\'on estoc\'astica en estos tiempos. Los $\xi_{n}$ son los tiempos de inter-renovaci\'on, y $N\left(t\right)$ es el n\'umero de renovaciones en el intervalo $\left[0,t\right)$
\end{Def}


\begin{Note}
Para definir un proceso de renovaci\'on para cualquier contexto, solamente hay que especificar una distribuci\'on $F$, con $F\left(0\right)=0$, para los tiempos de inter-renovaci\'on. La funci\'on $F$ en turno degune las otra variables aleatorias. De manera formal, existe un espacio de probabilidad y una sucesi\'on de variables aleatorias $\xi_{1},\xi_{2},\ldots$ definidas en este con distribuci\'on $F$. Entonces las otras cantidades son $T_{n}=\sum_{k=1}^{n}\xi_{k}$ y $N\left(t\right)=\sum_{n=1}^{\infty}\indora\left(T_{n}\leq t\right)$, donde $T_{n}\rightarrow\infty$ casi seguramente por la Ley Fuerte de los Grandes N\'umeros.
\end{Note}


Los tiempos $T_{n}$ est\'an relacionados con los conteos de $N\left(t\right)$ por

\begin{eqnarray*}
\left\{N\left(t\right)\geq n\right\}&=&\left\{T_{n}\leq t\right\}\\
T_{N\left(t\right)}\leq &t&<T_{N\left(t\right)+1},
\end{eqnarray*}

adem\'as $N\left(T_{n}\right)=n$, y 

\begin{eqnarray*}
N\left(t\right)=\max\left\{n:T_{n}\leq t\right\}=\min\left\{n:T_{n+1}>t\right\}
\end{eqnarray*}

Por propiedades de la convoluci\'on se sabe que

\begin{eqnarray*}
P\left\{T_{n}\leq t\right\}=F^{n\star}\left(t\right)
\end{eqnarray*}
que es la $n$-\'esima convoluci\'on de $F$. Entonces 

\begin{eqnarray*}
\left\{N\left(t\right)\geq n\right\}&=&\left\{T_{n}\leq t\right\}\\
P\left\{N\left(t\right)\leq n\right\}&=&1-F^{\left(n+1\right)\star}\left(t\right)
\end{eqnarray*}

Adem\'as usando el hecho de que $\esp\left[N\left(t\right)\right]=\sum_{n=1}^{\infty}P\left\{N\left(t\right)\geq n\right\}$
se tiene que

\begin{eqnarray*}
\esp\left[N\left(t\right)\right]=\sum_{n=1}^{\infty}F^{n\star}\left(t\right)
\end{eqnarray*}

\begin{Prop}
Para cada $t\geq0$, la funci\'on generadora de momentos $\esp\left[e^{\alpha N\left(t\right)}\right]$ existe para alguna $\alpha$ en una vecindad del 0, y de aqu\'i que $\esp\left[N\left(t\right)^{m}\right]<\infty$, para $m\geq1$.
\end{Prop}


\begin{Note}
Si el primer tiempo de renovaci\'on $\xi_{1}$ no tiene la misma distribuci\'on que el resto de las $\xi_{n}$, para $n\geq2$, a $N\left(t\right)$ se le llama Proceso de Renovaci\'on retardado, donde si $\xi$ tiene distribuci\'on $G$, entonces el tiempo $T_{n}$ de la $n$-\'esima renovaci\'on tiene distribuci\'on $G\star F^{\left(n-1\right)\star}\left(t\right)$
\end{Note}


\begin{Teo}
Para una constante $\mu\leq\infty$ ( o variable aleatoria), las siguientes expresiones son equivalentes:

\begin{eqnarray}
lim_{n\rightarrow\infty}n^{-1}T_{n}&=&\mu,\textrm{ c.s.}\\
lim_{t\rightarrow\infty}t^{-1}N\left(t\right)&=&1/\mu,\textrm{ c.s.}
\end{eqnarray}
\end{Teo}


Es decir, $T_{n}$ satisface la Ley Fuerte de los Grandes N\'umeros s\'i y s\'olo s\'i $N\left/t\right)$ la cumple.


\begin{Coro}[Ley Fuerte de los Grandes N\'umeros para Procesos de Renovaci\'on]
Si $N\left(t\right)$ es un proceso de renovaci\'on cuyos tiempos de inter-renovaci\'on tienen media $\mu\leq\infty$, entonces
\begin{eqnarray}
t^{-1}N\left(t\right)\rightarrow 1/\mu,\textrm{ c.s. cuando }t\rightarrow\infty.
\end{eqnarray}

\end{Coro}


Considerar el proceso estoc\'astico de valores reales $\left\{Z\left(t\right):t\geq0\right\}$ en el mismo espacio de probabilidad que $N\left(t\right)$

\begin{Def}
Para el proceso $\left\{Z\left(t\right):t\geq0\right\}$ se define la fluctuaci\'on m\'axima de $Z\left(t\right)$ en el intervalo $\left(T_{n-1},T_{n}\right]$:
\begin{eqnarray*}
M_{n}=\sup_{T_{n-1}<t\leq T_{n}}|Z\left(t\right)-Z\left(T_{n-1}\right)|
\end{eqnarray*}
\end{Def}

\begin{Teo}
Sup\'ongase que $n^{-1}T_{n}\rightarrow\mu$ c.s. cuando $n\rightarrow\infty$, donde $\mu\leq\infty$ es una constante o variable aleatoria. Sea $a$ una constante o variable aleatoria que puede ser infinita cuando $\mu$ es finita, y considere las expresiones l\'imite:
\begin{eqnarray}
lim_{n\rightarrow\infty}n^{-1}Z\left(T_{n}\right)&=&a,\textrm{ c.s.}\\
lim_{t\rightarrow\infty}t^{-1}Z\left(t\right)&=&a/\mu,\textrm{ c.s.}
\end{eqnarray}
La segunda expresi\'on implica la primera. Conversamente, la primera implica la segunda si el proceso $Z\left(t\right)$ es creciente, o si $lim_{n\rightarrow\infty}n^{-1}M_{n}=0$ c.s.
\end{Teo}

\begin{Coro}
Si $N\left(t\right)$ es un proceso de renovaci\'on, y $\left(Z\left(T_{n}\right)-Z\left(T_{n-1}\right),M_{n}\right)$, para $n\geq1$, son variables aleatorias independientes e id\'enticamente distribuidas con media finita, entonces,
\begin{eqnarray}
lim_{t\rightarrow\infty}t^{-1}Z\left(t\right)\rightarrow\frac{\esp\left[Z\left(T_{1}\right)-Z\left(T_{0}\right)\right]}{\esp\left[T_{1}\right]},\textrm{ c.s. cuando  }t\rightarrow\infty.
\end{eqnarray}
\end{Coro}


Sup\'ongase que $N\left(t\right)$ es un proceso de renovaci\'on con distribuci\'on $F$ con media finita $\mu$.

\begin{Def}
La funci\'on de renovaci\'on asociada con la distribuci\'on $F$, del proceso $N\left(t\right)$, es
\begin{eqnarray*}
U\left(t\right)=\sum_{n=1}^{\infty}F^{n\star}\left(t\right),\textrm{   }t\geq0,
\end{eqnarray*}
donde $F^{0\star}\left(t\right)=\indora\left(t\geq0\right)$.
\end{Def}


\begin{Prop}
Sup\'ongase que la distribuci\'on de inter-renovaci\'on $F$ tiene densidad $f$. Entonces $U\left(t\right)$ tambi\'en tiene densidad, para $t>0$, y es $U^{'}\left(t\right)=\sum_{n=0}^{\infty}f^{n\star}\left(t\right)$. Adem\'as
\begin{eqnarray*}
\prob\left\{N\left(t\right)>N\left(t-\right)\right\}=0\textrm{,   }t\geq0.
\end{eqnarray*}
\end{Prop}

\begin{Def}
La Transformada de Laplace-Stieljes de $F$ est\'a dada por

\begin{eqnarray*}
\hat{F}\left(\alpha\right)=\int_{\rea_{+}}e^{-\alpha t}dF\left(t\right)\textrm{,  }\alpha\geq0.
\end{eqnarray*}
\end{Def}

Entonces

\begin{eqnarray*}
\hat{U}\left(\alpha\right)=\sum_{n=0}^{\infty}\hat{F^{n\star}}\left(\alpha\right)=\sum_{n=0}^{\infty}\hat{F}\left(\alpha\right)^{n}=\frac{1}{1-\hat{F}\left(\alpha\right)}.
\end{eqnarray*}


\begin{Prop}
La Transformada de Laplace $\hat{U}\left(\alpha\right)$ y $\hat{F}\left(\alpha\right)$ determina una a la otra de manera \'unica por la relaci\'on $\hat{U}\left(\alpha\right)=\frac{1}{1-\hat{F}\left(\alpha\right)}$.
\end{Prop}


\begin{Note}
Un proceso de renovaci\'on $N\left(t\right)$ cuyos tiempos de inter-renovaci\'on tienen media finita, es un proceso Poisson con tasa $\lambda$ si y s\'olo s\'i $\esp\left[U\left(t\right)\right]=\lambda t$, para $t\geq0$.
\end{Note}


\begin{Teo}
Sea $N\left(t\right)$ un proceso puntual simple con puntos de localizaci\'on $T_{n}$ tal que $\eta\left(t\right)=\esp\left[N\left(\right)\right]$ es finita para cada $t$. Entonces para cualquier funci\'on $f:\rea_{+}\rightarrow\rea$,
\begin{eqnarray*}
\esp\left[\sum_{n=1}^{N\left(\right)}f\left(T_{n}\right)\right]=\int_{\left(0,t\right]}f\left(s\right)d\eta\left(s\right)\textrm{,  }t\geq0,
\end{eqnarray*}
suponiendo que la integral exista. Adem\'as si $X_{1},X_{2},\ldots$ son variables aleatorias definidas en el mismo espacio de probabilidad que el proceso $N\left(t\right)$ tal que $\esp\left[X_{n}|T_{n}=s\right]=f\left(s\right)$, independiente de $n$. Entonces
\begin{eqnarray*}
\esp\left[\sum_{n=1}^{N\left(t\right)}X_{n}\right]=\int_{\left(0,t\right]}f\left(s\right)d\eta\left(s\right)\textrm{,  }t\geq0,
\end{eqnarray*} 
suponiendo que la integral exista. 
\end{Teo}

\begin{Coro}[Identidad de Wald para Renovaciones]
Para el proceso de renovaci\'on $N\left(t\right)$,
\begin{eqnarray*}
\esp\left[T_{N\left(t\right)+1}\right]=\mu\esp\left[N\left(t\right)+1\right]\textrm{,  }t\geq0,
\end{eqnarray*}  
\end{Coro}


\begin{Def}
Sea $h\left(t\right)$ funci\'on de valores reales en $\rea$ acotada en intervalos finitos e igual a cero para $t<0$ La ecuaci\'on de renovaci\'on para $h\left(t\right)$ y la distribuci\'on $F$ es

\begin{eqnarray}\label{Ec.Renovacion}
H\left(t\right)=h\left(t\right)+\int_{\left[0,t\right]}H\left(t-s\right)dF\left(s\right)\textrm{,    }t\geq0,
\end{eqnarray}
donde $H\left(t\right)$ es una funci\'on de valores reales. Esto es $H=h+F\star H$. Decimos que $H\left(t\right)$ es soluci\'on de esta ecuaci\'on si satisface la ecuaci\'on, y es acotada en intervalos finitos e iguales a cero para $t<0$.
\end{Def}

\begin{Prop}
La funci\'on $U\star h\left(t\right)$ es la \'unica soluci\'on de la ecuaci\'on de renovaci\'on (\ref{Ec.Renovacion}).
\end{Prop}

\begin{Teo}[Teorema Renovaci\'on Elemental]
\begin{eqnarray*}
t^{-1}U\left(t\right)\rightarrow 1/\mu\textrm{,    cuando }t\rightarrow\infty.
\end{eqnarray*}
\end{Teo}



Sup\'ongase que $N\left(t\right)$ es un proceso de renovaci\'on con distribuci\'on $F$ con media finita $\mu$.

\begin{Def}
La funci\'on de renovaci\'on asociada con la distribuci\'on $F$, del proceso $N\left(t\right)$, es
\begin{eqnarray*}
U\left(t\right)=\sum_{n=1}^{\infty}F^{n\star}\left(t\right),\textrm{   }t\geq0,
\end{eqnarray*}
donde $F^{0\star}\left(t\right)=\indora\left(t\geq0\right)$.
\end{Def}


\begin{Prop}
Sup\'ongase que la distribuci\'on de inter-renovaci\'on $F$ tiene densidad $f$. Entonces $U\left(t\right)$ tambi\'en tiene densidad, para $t>0$, y es $U^{'}\left(t\right)=\sum_{n=0}^{\infty}f^{n\star}\left(t\right)$. Adem\'as
\begin{eqnarray*}
\prob\left\{N\left(t\right)>N\left(t-\right)\right\}=0\textrm{,   }t\geq0.
\end{eqnarray*}
\end{Prop}

\begin{Def}
La Transformada de Laplace-Stieljes de $F$ est\'a dada por

\begin{eqnarray*}
\hat{F}\left(\alpha\right)=\int_{\rea_{+}}e^{-\alpha t}dF\left(t\right)\textrm{,  }\alpha\geq0.
\end{eqnarray*}
\end{Def}

Entonces

\begin{eqnarray*}
\hat{U}\left(\alpha\right)=\sum_{n=0}^{\infty}\hat{F^{n\star}}\left(\alpha\right)=\sum_{n=0}^{\infty}\hat{F}\left(\alpha\right)^{n}=\frac{1}{1-\hat{F}\left(\alpha\right)}.
\end{eqnarray*}


\begin{Prop}
La Transformada de Laplace $\hat{U}\left(\alpha\right)$ y $\hat{F}\left(\alpha\right)$ determina una a la otra de manera \'unica por la relaci\'on $\hat{U}\left(\alpha\right)=\frac{1}{1-\hat{F}\left(\alpha\right)}$.
\end{Prop}


\begin{Note}
Un proceso de renovaci\'on $N\left(t\right)$ cuyos tiempos de inter-renovaci\'on tienen media finita, es un proceso Poisson con tasa $\lambda$ si y s\'olo s\'i $\esp\left[U\left(t\right)\right]=\lambda t$, para $t\geq0$.
\end{Note}


\begin{Teo}
Sea $N\left(t\right)$ un proceso puntual simple con puntos de localizaci\'on $T_{n}$ tal que $\eta\left(t\right)=\esp\left[N\left(\right)\right]$ es finita para cada $t$. Entonces para cualquier funci\'on $f:\rea_{+}\rightarrow\rea$,
\begin{eqnarray*}
\esp\left[\sum_{n=1}^{N\left(\right)}f\left(T_{n}\right)\right]=\int_{\left(0,t\right]}f\left(s\right)d\eta\left(s\right)\textrm{,  }t\geq0,
\end{eqnarray*}
suponiendo que la integral exista. Adem\'as si $X_{1},X_{2},\ldots$ son variables aleatorias definidas en el mismo espacio de probabilidad que el proceso $N\left(t\right)$ tal que $\esp\left[X_{n}|T_{n}=s\right]=f\left(s\right)$, independiente de $n$. Entonces
\begin{eqnarray*}
\esp\left[\sum_{n=1}^{N\left(t\right)}X_{n}\right]=\int_{\left(0,t\right]}f\left(s\right)d\eta\left(s\right)\textrm{,  }t\geq0,
\end{eqnarray*} 
suponiendo que la integral exista. 
\end{Teo}

\begin{Coro}[Identidad de Wald para Renovaciones]
Para el proceso de renovaci\'on $N\left(t\right)$,
\begin{eqnarray*}
\esp\left[T_{N\left(t\right)+1}\right]=\mu\esp\left[N\left(t\right)+1\right]\textrm{,  }t\geq0,
\end{eqnarray*}  
\end{Coro}


\begin{Def}
Sea $h\left(t\right)$ funci\'on de valores reales en $\rea$ acotada en intervalos finitos e igual a cero para $t<0$ La ecuaci\'on de renovaci\'on para $h\left(t\right)$ y la distribuci\'on $F$ es

\begin{eqnarray}\label{Ec.Renovacion}
H\left(t\right)=h\left(t\right)+\int_{\left[0,t\right]}H\left(t-s\right)dF\left(s\right)\textrm{,    }t\geq0,
\end{eqnarray}
donde $H\left(t\right)$ es una funci\'on de valores reales. Esto es $H=h+F\star H$. Decimos que $H\left(t\right)$ es soluci\'on de esta ecuaci\'on si satisface la ecuaci\'on, y es acotada en intervalos finitos e iguales a cero para $t<0$.
\end{Def}

\begin{Prop}
La funci\'on $U\star h\left(t\right)$ es la \'unica soluci\'on de la ecuaci\'on de renovaci\'on (\ref{Ec.Renovacion}).
\end{Prop}

\begin{Teo}[Teorema Renovaci\'on Elemental]
\begin{eqnarray*}
t^{-1}U\left(t\right)\rightarrow 1/\mu\textrm{,    cuando }t\rightarrow\infty.
\end{eqnarray*}
\end{Teo}


\begin{Note} Una funci\'on $h:\rea_{+}\rightarrow\rea$ es Directamente Riemann Integrable en los siguientes casos:
\begin{itemize}
\item[a)] $h\left(t\right)\geq0$ es decreciente y Riemann Integrable.
\item[b)] $h$ es continua excepto posiblemente en un conjunto de Lebesgue de medida 0, y $|h\left(t\right)|\leq b\left(t\right)$, donde $b$ es DRI.
\end{itemize}
\end{Note}

\begin{Teo}[Teorema Principal de Renovaci\'on]
Si $F$ es no aritm\'etica y $h\left(t\right)$ es Directamente Riemann Integrable (DRI), entonces

\begin{eqnarray*}
lim_{t\rightarrow\infty}U\star h=\frac{1}{\mu}\int_{\rea_{+}}h\left(s\right)ds.
\end{eqnarray*}
\end{Teo}

\begin{Prop}
Cualquier funci\'on $H\left(t\right)$ acotada en intervalos finitos y que es 0 para $t<0$ puede expresarse como
\begin{eqnarray*}
H\left(t\right)=U\star h\left(t\right)\textrm{,  donde }h\left(t\right)=H\left(t\right)-F\star H\left(t\right)
\end{eqnarray*}
\end{Prop}

\begin{Def}
Un proceso estoc\'astico $X\left(t\right)$ es crudamente regenerativo en un tiempo aleatorio positivo $T$ si
\begin{eqnarray*}
\esp\left[X\left(T+t\right)|T\right]=\esp\left[X\left(t\right)\right]\textrm{, para }t\geq0,\end{eqnarray*}
y con las esperanzas anteriores finitas.
\end{Def}

\begin{Prop}
Sup\'ongase que $X\left(t\right)$ es un proceso crudamente regenerativo en $T$, que tiene distribuci\'on $F$. Si $\esp\left[X\left(t\right)\right]$ es acotado en intervalos finitos, entonces
\begin{eqnarray*}
\esp\left[X\left(t\right)\right]=U\star h\left(t\right)\textrm{,  donde }h\left(t\right)=\esp\left[X\left(t\right)\indora\left(T>t\right)\right].
\end{eqnarray*}
\end{Prop}

\begin{Teo}[Regeneraci\'on Cruda]
Sup\'ongase que $X\left(t\right)$ es un proceso con valores positivo crudamente regenerativo en $T$, y def\'inase $M=\sup\left\{|X\left(t\right)|:t\leq T\right\}$. Si $T$ es no aritm\'etico y $M$ y $MT$ tienen media finita, entonces
\begin{eqnarray*}
lim_{t\rightarrow\infty}\esp\left[X\left(t\right)\right]=\frac{1}{\mu}\int_{\rea_{+}}h\left(s\right)ds,
\end{eqnarray*}
donde $h\left(t\right)=\esp\left[X\left(t\right)\indora\left(T>t\right)\right]$.
\end{Teo}


\begin{Note} Una funci\'on $h:\rea_{+}\rightarrow\rea$ es Directamente Riemann Integrable en los siguientes casos:
\begin{itemize}
\item[a)] $h\left(t\right)\geq0$ es decreciente y Riemann Integrable.
\item[b)] $h$ es continua excepto posiblemente en un conjunto de Lebesgue de medida 0, y $|h\left(t\right)|\leq b\left(t\right)$, donde $b$ es DRI.
\end{itemize}
\end{Note}

\begin{Teo}[Teorema Principal de Renovaci\'on]
Si $F$ es no aritm\'etica y $h\left(t\right)$ es Directamente Riemann Integrable (DRI), entonces

\begin{eqnarray*}
lim_{t\rightarrow\infty}U\star h=\frac{1}{\mu}\int_{\rea_{+}}h\left(s\right)ds.
\end{eqnarray*}
\end{Teo}

\begin{Prop}
Cualquier funci\'on $H\left(t\right)$ acotada en intervalos finitos y que es 0 para $t<0$ puede expresarse como
\begin{eqnarray*}
H\left(t\right)=U\star h\left(t\right)\textrm{,  donde }h\left(t\right)=H\left(t\right)-F\star H\left(t\right)
\end{eqnarray*}
\end{Prop}

\begin{Def}
Un proceso estoc\'astico $X\left(t\right)$ es crudamente regenerativo en un tiempo aleatorio positivo $T$ si
\begin{eqnarray*}
\esp\left[X\left(T+t\right)|T\right]=\esp\left[X\left(t\right)\right]\textrm{, para }t\geq0,\end{eqnarray*}
y con las esperanzas anteriores finitas.
\end{Def}

\begin{Prop}
Sup\'ongase que $X\left(t\right)$ es un proceso crudamente regenerativo en $T$, que tiene distribuci\'on $F$. Si $\esp\left[X\left(t\right)\right]$ es acotado en intervalos finitos, entonces
\begin{eqnarray*}
\esp\left[X\left(t\right)\right]=U\star h\left(t\right)\textrm{,  donde }h\left(t\right)=\esp\left[X\left(t\right)\indora\left(T>t\right)\right].
\end{eqnarray*}
\end{Prop}

\begin{Teo}[Regeneraci\'on Cruda]
Sup\'ongase que $X\left(t\right)$ es un proceso con valores positivo crudamente regenerativo en $T$, y def\'inase $M=\sup\left\{|X\left(t\right)|:t\leq T\right\}$. Si $T$ es no aritm\'etico y $M$ y $MT$ tienen media finita, entonces
\begin{eqnarray*}
lim_{t\rightarrow\infty}\esp\left[X\left(t\right)\right]=\frac{1}{\mu}\int_{\rea_{+}}h\left(s\right)ds,
\end{eqnarray*}
donde $h\left(t\right)=\esp\left[X\left(t\right)\indora\left(T>t\right)\right]$.
\end{Teo}

%________________________________________________________________________
\subsection{Procesos Regenerativos}
%________________________________________________________________________

Para $\left\{X\left(t\right):t\geq0\right\}$ Proceso Estoc\'astico a tiempo continuo con estado de espacios $S$, que es un espacio m\'etrico, con trayectorias continuas por la derecha y con l\'imites por la izquierda c.s. Sea $N\left(t\right)$ un proceso de renovaci\'on en $\rea_{+}$ definido en el mismo espacio de probabilidad que $X\left(t\right)$, con tiempos de renovaci\'on $T$ y tiempos de inter-renovaci\'on $\xi_{n}=T_{n}-T_{n-1}$, con misma distribuci\'on $F$ de media finita $\mu$.



\begin{Def}
Para el proceso $\left\{\left(N\left(t\right),X\left(t\right)\right):t\geq0\right\}$, sus trayectoria muestrales en el intervalo de tiempo $\left[T_{n-1},T_{n}\right)$ est\'an descritas por
\begin{eqnarray*}
\zeta_{n}=\left(\xi_{n},\left\{X\left(T_{n-1}+t\right):0\leq t<\xi_{n}\right\}\right)
\end{eqnarray*}
Este $\zeta_{n}$ es el $n$-\'esimo segmento del proceso. El proceso es regenerativo sobre los tiempos $T_{n}$ si sus segmentos $\zeta_{n}$ son independientes e id\'enticamennte distribuidos.
\end{Def}


\begin{Obs}
Si $\tilde{X}\left(t\right)$ con espacio de estados $\tilde{S}$ es regenerativo sobre $T_{n}$, entonces $X\left(t\right)=f\left(\tilde{X}\left(t\right)\right)$ tambi\'en es regenerativo sobre $T_{n}$, para cualquier funci\'on $f:\tilde{S}\rightarrow S$.
\end{Obs}

\begin{Obs}
Los procesos regenerativos son crudamente regenerativos, pero no al rev\'es.
\end{Obs}

\begin{Def}[Definici\'on Cl\'asica]
Un proceso estoc\'astico $X=\left\{X\left(t\right):t\geq0\right\}$ es llamado regenerativo is existe una variable aleatoria $R_{1}>0$ tal que
\begin{itemize}
\item[i)] $\left\{X\left(t+R_{1}\right):t\geq0\right\}$ es independiente de $\left\{\left\{X\left(t\right):t<R_{1}\right\},\right\}$
\item[ii)] $\left\{X\left(t+R_{1}\right):t\geq0\right\}$ es estoc\'asticamente equivalente a $\left\{X\left(t\right):t>0\right\}$
\end{itemize}

Llamamos a $R_{1}$ tiempo de regeneraci\'on, y decimos que $X$ se regenera en este punto.
\end{Def}

$\left\{X\left(t+R_{1}\right)\right\}$ es regenerativo con tiempo de regeneraci\'on $R_{2}$, independiente de $R_{1}$ pero con la misma distribuci\'on que $R_{1}$. Procediendo de esta manera se obtiene una secuencia de variables aleatorias independientes e id\'enticamente distribuidas $\left\{R_{n}\right\}$ llamados longitudes de ciclo. Si definimos a $Z_{k}\equiv R_{1}+R_{2}+\cdots+R_{k}$, se tiene un proceso de renovaci\'on llamado proceso de renovaci\'on encajado para $X$.

\begin{Note}
Un proceso regenerativo con media de la longitud de ciclo finita es llamado positivo recurrente.
\end{Note}


\begin{Def}
Para $x$ fijo y para cada $t\geq0$, sea $I_{x}\left(t\right)=1$ si $X\left(t\right)\leq x$,  $I_{x}\left(t\right)=0$ en caso contrario, y def\'inanse los tiempos promedio
\begin{eqnarray*}
\overline{X}&=&lim_{t\rightarrow\infty}\frac{1}{t}\int_{0}^{\infty}X\left(u\right)du\\
\prob\left(X_{\infty}\leq x\right)&=&lim_{t\rightarrow\infty}\frac{1}{t}\int_{0}^{\infty}I_{x}\left(u\right)du,
\end{eqnarray*}
cuando estos l\'imites existan.
\end{Def}

Como consecuencia del teorema de Renovaci\'on-Recompensa, se tiene que el primer l\'imite  existe y es igual a la constante
\begin{eqnarray*}
\overline{X}&=&\frac{\esp\left[\int_{0}^{R_{1}}X\left(t\right)dt\right]}{\esp\left[R_{1}\right]},
\end{eqnarray*}
suponiendo que ambas esperanzas son finitas.

\begin{Note}
\begin{itemize}
\item[a)] Si el proceso regenerativo $X$ es positivo recurrente y tiene trayectorias muestrales no negativas, entonces la ecuaci\'on anterior es v\'alida.
\item[b)] Si $X$ es positivo recurrente regenerativo, podemos construir una \'unica versi\'on estacionaria de este proceso, $X_{e}=\left\{X_{e}\left(t\right)\right\}$, donde $X_{e}$ es un proceso estoc\'astico regenerativo y estrictamente estacionario, con distribuci\'on marginal distribuida como $X_{\infty}$
\end{itemize}
\end{Note}

%________________________________________________________________________
\subsection{Procesos Regenerativos}
%________________________________________________________________________

Para $\left\{X\left(t\right):t\geq0\right\}$ Proceso Estoc\'astico a tiempo continuo con estado de espacios $S$, que es un espacio m\'etrico, con trayectorias continuas por la derecha y con l\'imites por la izquierda c.s. Sea $N\left(t\right)$ un proceso de renovaci\'on en $\rea_{+}$ definido en el mismo espacio de probabilidad que $X\left(t\right)$, con tiempos de renovaci\'on $T$ y tiempos de inter-renovaci\'on $\xi_{n}=T_{n}-T_{n-1}$, con misma distribuci\'on $F$ de media finita $\mu$.



\begin{Def}
Para el proceso $\left\{\left(N\left(t\right),X\left(t\right)\right):t\geq0\right\}$, sus trayectoria muestrales en el intervalo de tiempo $\left[T_{n-1},T_{n}\right)$ est\'an descritas por
\begin{eqnarray*}
\zeta_{n}=\left(\xi_{n},\left\{X\left(T_{n-1}+t\right):0\leq t<\xi_{n}\right\}\right)
\end{eqnarray*}
Este $\zeta_{n}$ es el $n$-\'esimo segmento del proceso. El proceso es regenerativo sobre los tiempos $T_{n}$ si sus segmentos $\zeta_{n}$ son independientes e id\'enticamennte distribuidos.
\end{Def}


\begin{Obs}
Si $\tilde{X}\left(t\right)$ con espacio de estados $\tilde{S}$ es regenerativo sobre $T_{n}$, entonces $X\left(t\right)=f\left(\tilde{X}\left(t\right)\right)$ tambi\'en es regenerativo sobre $T_{n}$, para cualquier funci\'on $f:\tilde{S}\rightarrow S$.
\end{Obs}

\begin{Obs}
Los procesos regenerativos son crudamente regenerativos, pero no al rev\'es.
\end{Obs}

\begin{Def}[Definici\'on Cl\'asica]
Un proceso estoc\'astico $X=\left\{X\left(t\right):t\geq0\right\}$ es llamado regenerativo is existe una variable aleatoria $R_{1}>0$ tal que
\begin{itemize}
\item[i)] $\left\{X\left(t+R_{1}\right):t\geq0\right\}$ es independiente de $\left\{\left\{X\left(t\right):t<R_{1}\right\},\right\}$
\item[ii)] $\left\{X\left(t+R_{1}\right):t\geq0\right\}$ es estoc\'asticamente equivalente a $\left\{X\left(t\right):t>0\right\}$
\end{itemize}

Llamamos a $R_{1}$ tiempo de regeneraci\'on, y decimos que $X$ se regenera en este punto.
\end{Def}

$\left\{X\left(t+R_{1}\right)\right\}$ es regenerativo con tiempo de regeneraci\'on $R_{2}$, independiente de $R_{1}$ pero con la misma distribuci\'on que $R_{1}$. Procediendo de esta manera se obtiene una secuencia de variables aleatorias independientes e id\'enticamente distribuidas $\left\{R_{n}\right\}$ llamados longitudes de ciclo. Si definimos a $Z_{k}\equiv R_{1}+R_{2}+\cdots+R_{k}$, se tiene un proceso de renovaci\'on llamado proceso de renovaci\'on encajado para $X$.

\begin{Note}
Un proceso regenerativo con media de la longitud de ciclo finita es llamado positivo recurrente.
\end{Note}


\begin{Def}
Para $x$ fijo y para cada $t\geq0$, sea $I_{x}\left(t\right)=1$ si $X\left(t\right)\leq x$,  $I_{x}\left(t\right)=0$ en caso contrario, y def\'inanse los tiempos promedio
\begin{eqnarray*}
\overline{X}&=&lim_{t\rightarrow\infty}\frac{1}{t}\int_{0}^{\infty}X\left(u\right)du\\
\prob\left(X_{\infty}\leq x\right)&=&lim_{t\rightarrow\infty}\frac{1}{t}\int_{0}^{\infty}I_{x}\left(u\right)du,
\end{eqnarray*}
cuando estos l\'imites existan.
\end{Def}

Como consecuencia del teorema de Renovaci\'on-Recompensa, se tiene que el primer l\'imite  existe y es igual a la constante
\begin{eqnarray*}
\overline{X}&=&\frac{\esp\left[\int_{0}^{R_{1}}X\left(t\right)dt\right]}{\esp\left[R_{1}\right]},
\end{eqnarray*}
suponiendo que ambas esperanzas son finitas.

\begin{Note}
\begin{itemize}
\item[a)] Si el proceso regenerativo $X$ es positivo recurrente y tiene trayectorias muestrales no negativas, entonces la ecuaci\'on anterior es v\'alida.
\item[b)] Si $X$ es positivo recurrente regenerativo, podemos construir una \'unica versi\'on estacionaria de este proceso, $X_{e}=\left\{X_{e}\left(t\right)\right\}$, donde $X_{e}$ es un proceso estoc\'astico regenerativo y estrictamente estacionario, con distribuci\'on marginal distribuida como $X_{\infty}$
\end{itemize}
\end{Note}
%__________________________________________________________________________________________
\subsection{Procesos Regenerativos Estacionarios - Stidham \cite{Stidham}}
%__________________________________________________________________________________________


Un proceso estoc\'astico a tiempo continuo $\left\{V\left(t\right),t\geq0\right\}$ es un proceso regenerativo si existe una sucesi\'on de variables aleatorias independientes e id\'enticamente distribuidas $\left\{X_{1},X_{2},\ldots\right\}$, sucesi\'on de renovaci\'on, tal que para cualquier conjunto de Borel $A$, 

\begin{eqnarray*}
\prob\left\{V\left(t\right)\in A|X_{1}+X_{2}+\cdots+X_{R\left(t\right)}=s,\left\{V\left(\tau\right),\tau<s\right\}\right\}=\prob\left\{V\left(t-s\right)\in A|X_{1}>t-s\right\},
\end{eqnarray*}
para todo $0\leq s\leq t$, donde $R\left(t\right)=\max\left\{X_{1}+X_{2}+\cdots+X_{j}\leq t\right\}=$n\'umero de renovaciones ({\emph{puntos de regeneraci\'on}}) que ocurren en $\left[0,t\right]$. El intervalo $\left[0,X_{1}\right)$ es llamado {\emph{primer ciclo de regeneraci\'on}} de $\left\{V\left(t \right),t\geq0\right\}$, $\left[X_{1},X_{1}+X_{2}\right)$ el {\emph{segundo ciclo de regeneraci\'on}}, y as\'i sucesivamente.

Sea $X=X_{1}$ y sea $F$ la funci\'on de distrbuci\'on de $X$


\begin{Def}
Se define el proceso estacionario, $\left\{V^{*}\left(t\right),t\geq0\right\}$, para $\left\{V\left(t\right),t\geq0\right\}$ por

\begin{eqnarray*}
\prob\left\{V\left(t\right)\in A\right\}=\frac{1}{\esp\left[X\right]}\int_{0}^{\infty}\prob\left\{V\left(t+x\right)\in A|X>x\right\}\left(1-F\left(x\right)\right)dx,
\end{eqnarray*} 
para todo $t\geq0$ y todo conjunto de Borel $A$.
\end{Def}

\begin{Def}
Una distribuci\'on se dice que es {\emph{aritm\'etica}} si todos sus puntos de incremento son m\'ultiplos de la forma $0,\lambda, 2\lambda,\ldots$ para alguna $\lambda>0$ entera.
\end{Def}


\begin{Def}
Una modificaci\'on medible de un proceso $\left\{V\left(t\right),t\geq0\right\}$, es una versi\'on de este, $\left\{V\left(t,w\right)\right\}$ conjuntamente medible para $t\geq0$ y para $w\in S$, $S$ espacio de estados para $\left\{V\left(t\right),t\geq0\right\}$.
\end{Def}

\begin{Teo}
Sea $\left\{V\left(t\right),t\geq\right\}$ un proceso regenerativo no negativo con modificaci\'on medible. Sea $\esp\left[X\right]<\infty$. Entonces el proceso estacionario dado por la ecuaci\'on anterior est\'a bien definido y tiene funci\'on de distribuci\'on independiente de $t$, adem\'as
\begin{itemize}
\item[i)] \begin{eqnarray*}
\esp\left[V^{*}\left(0\right)\right]&=&\frac{\esp\left[\int_{0}^{X}V\left(s\right)ds\right]}{\esp\left[X\right]}\end{eqnarray*}
\item[ii)] Si $\esp\left[V^{*}\left(0\right)\right]<\infty$, equivalentemente, si $\esp\left[\int_{0}^{X}V\left(s\right)ds\right]<\infty$,entonces
\begin{eqnarray*}
\frac{\int_{0}^{t}V\left(s\right)ds}{t}\rightarrow\frac{\esp\left[\int_{0}^{X}V\left(s\right)ds\right]}{\esp\left[X\right]}
\end{eqnarray*}
con probabilidad 1 y en media, cuando $t\rightarrow\infty$.
\end{itemize}
\end{Teo}


%__________________________________________________________________________________________
\subsection{Procesos Regenerativos Estacionarios - Stidham \cite{Stidham}}
%__________________________________________________________________________________________


Un proceso estoc\'astico a tiempo continuo $\left\{V\left(t\right),t\geq0\right\}$ es un proceso regenerativo si existe una sucesi\'on de variables aleatorias independientes e id\'enticamente distribuidas $\left\{X_{1},X_{2},\ldots\right\}$, sucesi\'on de renovaci\'on, tal que para cualquier conjunto de Borel $A$, 

\begin{eqnarray*}
\prob\left\{V\left(t\right)\in A|X_{1}+X_{2}+\cdots+X_{R\left(t\right)}=s,\left\{V\left(\tau\right),\tau<s\right\}\right\}=\prob\left\{V\left(t-s\right)\in A|X_{1}>t-s\right\},
\end{eqnarray*}
para todo $0\leq s\leq t$, donde $R\left(t\right)=\max\left\{X_{1}+X_{2}+\cdots+X_{j}\leq t\right\}=$n\'umero de renovaciones ({\emph{puntos de regeneraci\'on}}) que ocurren en $\left[0,t\right]$. El intervalo $\left[0,X_{1}\right)$ es llamado {\emph{primer ciclo de regeneraci\'on}} de $\left\{V\left(t \right),t\geq0\right\}$, $\left[X_{1},X_{1}+X_{2}\right)$ el {\emph{segundo ciclo de regeneraci\'on}}, y as\'i sucesivamente.

Sea $X=X_{1}$ y sea $F$ la funci\'on de distrbuci\'on de $X$


\begin{Def}
Se define el proceso estacionario, $\left\{V^{*}\left(t\right),t\geq0\right\}$, para $\left\{V\left(t\right),t\geq0\right\}$ por

\begin{eqnarray*}
\prob\left\{V\left(t\right)\in A\right\}=\frac{1}{\esp\left[X\right]}\int_{0}^{\infty}\prob\left\{V\left(t+x\right)\in A|X>x\right\}\left(1-F\left(x\right)\right)dx,
\end{eqnarray*} 
para todo $t\geq0$ y todo conjunto de Borel $A$.
\end{Def}

\begin{Def}
Una distribuci\'on se dice que es {\emph{aritm\'etica}} si todos sus puntos de incremento son m\'ultiplos de la forma $0,\lambda, 2\lambda,\ldots$ para alguna $\lambda>0$ entera.
\end{Def}


\begin{Def}
Una modificaci\'on medible de un proceso $\left\{V\left(t\right),t\geq0\right\}$, es una versi\'on de este, $\left\{V\left(t,w\right)\right\}$ conjuntamente medible para $t\geq0$ y para $w\in S$, $S$ espacio de estados para $\left\{V\left(t\right),t\geq0\right\}$.
\end{Def}

\begin{Teo}
Sea $\left\{V\left(t\right),t\geq\right\}$ un proceso regenerativo no negativo con modificaci\'on medible. Sea $\esp\left[X\right]<\infty$. Entonces el proceso estacionario dado por la ecuaci\'on anterior est\'a bien definido y tiene funci\'on de distribuci\'on independiente de $t$, adem\'as
\begin{itemize}
\item[i)] \begin{eqnarray*}
\esp\left[V^{*}\left(0\right)\right]&=&\frac{\esp\left[\int_{0}^{X}V\left(s\right)ds\right]}{\esp\left[X\right]}\end{eqnarray*}
\item[ii)] Si $\esp\left[V^{*}\left(0\right)\right]<\infty$, equivalentemente, si $\esp\left[\int_{0}^{X}V\left(s\right)ds\right]<\infty$,entonces
\begin{eqnarray*}
\frac{\int_{0}^{t}V\left(s\right)ds}{t}\rightarrow\frac{\esp\left[\int_{0}^{X}V\left(s\right)ds\right]}{\esp\left[X\right]}
\end{eqnarray*}
con probabilidad 1 y en media, cuando $t\rightarrow\infty$.
\end{itemize}
\end{Teo}
%___________________________________________________________________________________________
%
\subsection{Propiedades de los Procesos de Renovaci\'on}
%___________________________________________________________________________________________
%

Los tiempos $T_{n}$ est\'an relacionados con los conteos de $N\left(t\right)$ por

\begin{eqnarray*}
\left\{N\left(t\right)\geq n\right\}&=&\left\{T_{n}\leq t\right\}\\
T_{N\left(t\right)}\leq &t&<T_{N\left(t\right)+1},
\end{eqnarray*}

adem\'as $N\left(T_{n}\right)=n$, y 

\begin{eqnarray*}
N\left(t\right)=\max\left\{n:T_{n}\leq t\right\}=\min\left\{n:T_{n+1}>t\right\}
\end{eqnarray*}

Por propiedades de la convoluci\'on se sabe que

\begin{eqnarray*}
P\left\{T_{n}\leq t\right\}=F^{n\star}\left(t\right)
\end{eqnarray*}
que es la $n$-\'esima convoluci\'on de $F$. Entonces 

\begin{eqnarray*}
\left\{N\left(t\right)\geq n\right\}&=&\left\{T_{n}\leq t\right\}\\
P\left\{N\left(t\right)\leq n\right\}&=&1-F^{\left(n+1\right)\star}\left(t\right)
\end{eqnarray*}

Adem\'as usando el hecho de que $\esp\left[N\left(t\right)\right]=\sum_{n=1}^{\infty}P\left\{N\left(t\right)\geq n\right\}$
se tiene que

\begin{eqnarray*}
\esp\left[N\left(t\right)\right]=\sum_{n=1}^{\infty}F^{n\star}\left(t\right)
\end{eqnarray*}

\begin{Prop}
Para cada $t\geq0$, la funci\'on generadora de momentos $\esp\left[e^{\alpha N\left(t\right)}\right]$ existe para alguna $\alpha$ en una vecindad del 0, y de aqu\'i que $\esp\left[N\left(t\right)^{m}\right]<\infty$, para $m\geq1$.
\end{Prop}


\begin{Note}
Si el primer tiempo de renovaci\'on $\xi_{1}$ no tiene la misma distribuci\'on que el resto de las $\xi_{n}$, para $n\geq2$, a $N\left(t\right)$ se le llama Proceso de Renovaci\'on retardado, donde si $\xi$ tiene distribuci\'on $G$, entonces el tiempo $T_{n}$ de la $n$-\'esima renovaci\'on tiene distribuci\'on $G\star F^{\left(n-1\right)\star}\left(t\right)$
\end{Note}


\begin{Teo}
Para una constante $\mu\leq\infty$ ( o variable aleatoria), las siguientes expresiones son equivalentes:

\begin{eqnarray}
lim_{n\rightarrow\infty}n^{-1}T_{n}&=&\mu,\textrm{ c.s.}\\
lim_{t\rightarrow\infty}t^{-1}N\left(t\right)&=&1/\mu,\textrm{ c.s.}
\end{eqnarray}
\end{Teo}


Es decir, $T_{n}$ satisface la Ley Fuerte de los Grandes N\'umeros s\'i y s\'olo s\'i $N\left/t\right)$ la cumple.


\begin{Coro}[Ley Fuerte de los Grandes N\'umeros para Procesos de Renovaci\'on]
Si $N\left(t\right)$ es un proceso de renovaci\'on cuyos tiempos de inter-renovaci\'on tienen media $\mu\leq\infty$, entonces
\begin{eqnarray}
t^{-1}N\left(t\right)\rightarrow 1/\mu,\textrm{ c.s. cuando }t\rightarrow\infty.
\end{eqnarray}

\end{Coro}


Considerar el proceso estoc\'astico de valores reales $\left\{Z\left(t\right):t\geq0\right\}$ en el mismo espacio de probabilidad que $N\left(t\right)$

\begin{Def}
Para el proceso $\left\{Z\left(t\right):t\geq0\right\}$ se define la fluctuaci\'on m\'axima de $Z\left(t\right)$ en el intervalo $\left(T_{n-1},T_{n}\right]$:
\begin{eqnarray*}
M_{n}=\sup_{T_{n-1}<t\leq T_{n}}|Z\left(t\right)-Z\left(T_{n-1}\right)|
\end{eqnarray*}
\end{Def}

\begin{Teo}
Sup\'ongase que $n^{-1}T_{n}\rightarrow\mu$ c.s. cuando $n\rightarrow\infty$, donde $\mu\leq\infty$ es una constante o variable aleatoria. Sea $a$ una constante o variable aleatoria que puede ser infinita cuando $\mu$ es finita, y considere las expresiones l\'imite:
\begin{eqnarray}
lim_{n\rightarrow\infty}n^{-1}Z\left(T_{n}\right)&=&a,\textrm{ c.s.}\\
lim_{t\rightarrow\infty}t^{-1}Z\left(t\right)&=&a/\mu,\textrm{ c.s.}
\end{eqnarray}
La segunda expresi\'on implica la primera. Conversamente, la primera implica la segunda si el proceso $Z\left(t\right)$ es creciente, o si $lim_{n\rightarrow\infty}n^{-1}M_{n}=0$ c.s.
\end{Teo}

\begin{Coro}
Si $N\left(t\right)$ es un proceso de renovaci\'on, y $\left(Z\left(T_{n}\right)-Z\left(T_{n-1}\right),M_{n}\right)$, para $n\geq1$, son variables aleatorias independientes e id\'enticamente distribuidas con media finita, entonces,
\begin{eqnarray}
lim_{t\rightarrow\infty}t^{-1}Z\left(t\right)\rightarrow\frac{\esp\left[Z\left(T_{1}\right)-Z\left(T_{0}\right)\right]}{\esp\left[T_{1}\right]},\textrm{ c.s. cuando  }t\rightarrow\infty.
\end{eqnarray}
\end{Coro}


%___________________________________________________________________________________________
%
\subsection{Propiedades de los Procesos de Renovaci\'on}
%___________________________________________________________________________________________
%

Los tiempos $T_{n}$ est\'an relacionados con los conteos de $N\left(t\right)$ por

\begin{eqnarray*}
\left\{N\left(t\right)\geq n\right\}&=&\left\{T_{n}\leq t\right\}\\
T_{N\left(t\right)}\leq &t&<T_{N\left(t\right)+1},
\end{eqnarray*}

adem\'as $N\left(T_{n}\right)=n$, y 

\begin{eqnarray*}
N\left(t\right)=\max\left\{n:T_{n}\leq t\right\}=\min\left\{n:T_{n+1}>t\right\}
\end{eqnarray*}

Por propiedades de la convoluci\'on se sabe que

\begin{eqnarray*}
P\left\{T_{n}\leq t\right\}=F^{n\star}\left(t\right)
\end{eqnarray*}
que es la $n$-\'esima convoluci\'on de $F$. Entonces 

\begin{eqnarray*}
\left\{N\left(t\right)\geq n\right\}&=&\left\{T_{n}\leq t\right\}\\
P\left\{N\left(t\right)\leq n\right\}&=&1-F^{\left(n+1\right)\star}\left(t\right)
\end{eqnarray*}

Adem\'as usando el hecho de que $\esp\left[N\left(t\right)\right]=\sum_{n=1}^{\infty}P\left\{N\left(t\right)\geq n\right\}$
se tiene que

\begin{eqnarray*}
\esp\left[N\left(t\right)\right]=\sum_{n=1}^{\infty}F^{n\star}\left(t\right)
\end{eqnarray*}

\begin{Prop}
Para cada $t\geq0$, la funci\'on generadora de momentos $\esp\left[e^{\alpha N\left(t\right)}\right]$ existe para alguna $\alpha$ en una vecindad del 0, y de aqu\'i que $\esp\left[N\left(t\right)^{m}\right]<\infty$, para $m\geq1$.
\end{Prop}


\begin{Note}
Si el primer tiempo de renovaci\'on $\xi_{1}$ no tiene la misma distribuci\'on que el resto de las $\xi_{n}$, para $n\geq2$, a $N\left(t\right)$ se le llama Proceso de Renovaci\'on retardado, donde si $\xi$ tiene distribuci\'on $G$, entonces el tiempo $T_{n}$ de la $n$-\'esima renovaci\'on tiene distribuci\'on $G\star F^{\left(n-1\right)\star}\left(t\right)$
\end{Note}


\begin{Teo}
Para una constante $\mu\leq\infty$ ( o variable aleatoria), las siguientes expresiones son equivalentes:

\begin{eqnarray}
lim_{n\rightarrow\infty}n^{-1}T_{n}&=&\mu,\textrm{ c.s.}\\
lim_{t\rightarrow\infty}t^{-1}N\left(t\right)&=&1/\mu,\textrm{ c.s.}
\end{eqnarray}
\end{Teo}


Es decir, $T_{n}$ satisface la Ley Fuerte de los Grandes N\'umeros s\'i y s\'olo s\'i $N\left/t\right)$ la cumple.


\begin{Coro}[Ley Fuerte de los Grandes N\'umeros para Procesos de Renovaci\'on]
Si $N\left(t\right)$ es un proceso de renovaci\'on cuyos tiempos de inter-renovaci\'on tienen media $\mu\leq\infty$, entonces
\begin{eqnarray}
t^{-1}N\left(t\right)\rightarrow 1/\mu,\textrm{ c.s. cuando }t\rightarrow\infty.
\end{eqnarray}

\end{Coro}


Considerar el proceso estoc\'astico de valores reales $\left\{Z\left(t\right):t\geq0\right\}$ en el mismo espacio de probabilidad que $N\left(t\right)$

\begin{Def}
Para el proceso $\left\{Z\left(t\right):t\geq0\right\}$ se define la fluctuaci\'on m\'axima de $Z\left(t\right)$ en el intervalo $\left(T_{n-1},T_{n}\right]$:
\begin{eqnarray*}
M_{n}=\sup_{T_{n-1}<t\leq T_{n}}|Z\left(t\right)-Z\left(T_{n-1}\right)|
\end{eqnarray*}
\end{Def}

\begin{Teo}
Sup\'ongase que $n^{-1}T_{n}\rightarrow\mu$ c.s. cuando $n\rightarrow\infty$, donde $\mu\leq\infty$ es una constante o variable aleatoria. Sea $a$ una constante o variable aleatoria que puede ser infinita cuando $\mu$ es finita, y considere las expresiones l\'imite:
\begin{eqnarray}
lim_{n\rightarrow\infty}n^{-1}Z\left(T_{n}\right)&=&a,\textrm{ c.s.}\\
lim_{t\rightarrow\infty}t^{-1}Z\left(t\right)&=&a/\mu,\textrm{ c.s.}
\end{eqnarray}
La segunda expresi\'on implica la primera. Conversamente, la primera implica la segunda si el proceso $Z\left(t\right)$ es creciente, o si $lim_{n\rightarrow\infty}n^{-1}M_{n}=0$ c.s.
\end{Teo}

\begin{Coro}
Si $N\left(t\right)$ es un proceso de renovaci\'on, y $\left(Z\left(T_{n}\right)-Z\left(T_{n-1}\right),M_{n}\right)$, para $n\geq1$, son variables aleatorias independientes e id\'enticamente distribuidas con media finita, entonces,
\begin{eqnarray}
lim_{t\rightarrow\infty}t^{-1}Z\left(t\right)\rightarrow\frac{\esp\left[Z\left(T_{1}\right)-Z\left(T_{0}\right)\right]}{\esp\left[T_{1}\right]},\textrm{ c.s. cuando  }t\rightarrow\infty.
\end{eqnarray}
\end{Coro}

%___________________________________________________________________________________________
%
\subsection{Propiedades de los Procesos de Renovaci\'on}
%___________________________________________________________________________________________
%

Los tiempos $T_{n}$ est\'an relacionados con los conteos de $N\left(t\right)$ por

\begin{eqnarray*}
\left\{N\left(t\right)\geq n\right\}&=&\left\{T_{n}\leq t\right\}\\
T_{N\left(t\right)}\leq &t&<T_{N\left(t\right)+1},
\end{eqnarray*}

adem\'as $N\left(T_{n}\right)=n$, y 

\begin{eqnarray*}
N\left(t\right)=\max\left\{n:T_{n}\leq t\right\}=\min\left\{n:T_{n+1}>t\right\}
\end{eqnarray*}

Por propiedades de la convoluci\'on se sabe que

\begin{eqnarray*}
P\left\{T_{n}\leq t\right\}=F^{n\star}\left(t\right)
\end{eqnarray*}
que es la $n$-\'esima convoluci\'on de $F$. Entonces 

\begin{eqnarray*}
\left\{N\left(t\right)\geq n\right\}&=&\left\{T_{n}\leq t\right\}\\
P\left\{N\left(t\right)\leq n\right\}&=&1-F^{\left(n+1\right)\star}\left(t\right)
\end{eqnarray*}

Adem\'as usando el hecho de que $\esp\left[N\left(t\right)\right]=\sum_{n=1}^{\infty}P\left\{N\left(t\right)\geq n\right\}$
se tiene que

\begin{eqnarray*}
\esp\left[N\left(t\right)\right]=\sum_{n=1}^{\infty}F^{n\star}\left(t\right)
\end{eqnarray*}

\begin{Prop}
Para cada $t\geq0$, la funci\'on generadora de momentos $\esp\left[e^{\alpha N\left(t\right)}\right]$ existe para alguna $\alpha$ en una vecindad del 0, y de aqu\'i que $\esp\left[N\left(t\right)^{m}\right]<\infty$, para $m\geq1$.
\end{Prop}


\begin{Note}
Si el primer tiempo de renovaci\'on $\xi_{1}$ no tiene la misma distribuci\'on que el resto de las $\xi_{n}$, para $n\geq2$, a $N\left(t\right)$ se le llama Proceso de Renovaci\'on retardado, donde si $\xi$ tiene distribuci\'on $G$, entonces el tiempo $T_{n}$ de la $n$-\'esima renovaci\'on tiene distribuci\'on $G\star F^{\left(n-1\right)\star}\left(t\right)$
\end{Note}


\begin{Teo}
Para una constante $\mu\leq\infty$ ( o variable aleatoria), las siguientes expresiones son equivalentes:

\begin{eqnarray}
lim_{n\rightarrow\infty}n^{-1}T_{n}&=&\mu,\textrm{ c.s.}\\
lim_{t\rightarrow\infty}t^{-1}N\left(t\right)&=&1/\mu,\textrm{ c.s.}
\end{eqnarray}
\end{Teo}


Es decir, $T_{n}$ satisface la Ley Fuerte de los Grandes N\'umeros s\'i y s\'olo s\'i $N\left/t\right)$ la cumple.


\begin{Coro}[Ley Fuerte de los Grandes N\'umeros para Procesos de Renovaci\'on]
Si $N\left(t\right)$ es un proceso de renovaci\'on cuyos tiempos de inter-renovaci\'on tienen media $\mu\leq\infty$, entonces
\begin{eqnarray}
t^{-1}N\left(t\right)\rightarrow 1/\mu,\textrm{ c.s. cuando }t\rightarrow\infty.
\end{eqnarray}

\end{Coro}


Considerar el proceso estoc\'astico de valores reales $\left\{Z\left(t\right):t\geq0\right\}$ en el mismo espacio de probabilidad que $N\left(t\right)$

\begin{Def}
Para el proceso $\left\{Z\left(t\right):t\geq0\right\}$ se define la fluctuaci\'on m\'axima de $Z\left(t\right)$ en el intervalo $\left(T_{n-1},T_{n}\right]$:
\begin{eqnarray*}
M_{n}=\sup_{T_{n-1}<t\leq T_{n}}|Z\left(t\right)-Z\left(T_{n-1}\right)|
\end{eqnarray*}
\end{Def}

\begin{Teo}
Sup\'ongase que $n^{-1}T_{n}\rightarrow\mu$ c.s. cuando $n\rightarrow\infty$, donde $\mu\leq\infty$ es una constante o variable aleatoria. Sea $a$ una constante o variable aleatoria que puede ser infinita cuando $\mu$ es finita, y considere las expresiones l\'imite:
\begin{eqnarray}
lim_{n\rightarrow\infty}n^{-1}Z\left(T_{n}\right)&=&a,\textrm{ c.s.}\\
lim_{t\rightarrow\infty}t^{-1}Z\left(t\right)&=&a/\mu,\textrm{ c.s.}
\end{eqnarray}
La segunda expresi\'on implica la primera. Conversamente, la primera implica la segunda si el proceso $Z\left(t\right)$ es creciente, o si $lim_{n\rightarrow\infty}n^{-1}M_{n}=0$ c.s.
\end{Teo}

\begin{Coro}
Si $N\left(t\right)$ es un proceso de renovaci\'on, y $\left(Z\left(T_{n}\right)-Z\left(T_{n-1}\right),M_{n}\right)$, para $n\geq1$, son variables aleatorias independientes e id\'enticamente distribuidas con media finita, entonces,
\begin{eqnarray}
lim_{t\rightarrow\infty}t^{-1}Z\left(t\right)\rightarrow\frac{\esp\left[Z\left(T_{1}\right)-Z\left(T_{0}\right)\right]}{\esp\left[T_{1}\right]},\textrm{ c.s. cuando  }t\rightarrow\infty.
\end{eqnarray}
\end{Coro}

%___________________________________________________________________________________________
%
\subsection{Propiedades de los Procesos de Renovaci\'on}
%___________________________________________________________________________________________
%

Los tiempos $T_{n}$ est\'an relacionados con los conteos de $N\left(t\right)$ por

\begin{eqnarray*}
\left\{N\left(t\right)\geq n\right\}&=&\left\{T_{n}\leq t\right\}\\
T_{N\left(t\right)}\leq &t&<T_{N\left(t\right)+1},
\end{eqnarray*}

adem\'as $N\left(T_{n}\right)=n$, y 

\begin{eqnarray*}
N\left(t\right)=\max\left\{n:T_{n}\leq t\right\}=\min\left\{n:T_{n+1}>t\right\}
\end{eqnarray*}

Por propiedades de la convoluci\'on se sabe que

\begin{eqnarray*}
P\left\{T_{n}\leq t\right\}=F^{n\star}\left(t\right)
\end{eqnarray*}
que es la $n$-\'esima convoluci\'on de $F$. Entonces 

\begin{eqnarray*}
\left\{N\left(t\right)\geq n\right\}&=&\left\{T_{n}\leq t\right\}\\
P\left\{N\left(t\right)\leq n\right\}&=&1-F^{\left(n+1\right)\star}\left(t\right)
\end{eqnarray*}

Adem\'as usando el hecho de que $\esp\left[N\left(t\right)\right]=\sum_{n=1}^{\infty}P\left\{N\left(t\right)\geq n\right\}$
se tiene que

\begin{eqnarray*}
\esp\left[N\left(t\right)\right]=\sum_{n=1}^{\infty}F^{n\star}\left(t\right)
\end{eqnarray*}

\begin{Prop}
Para cada $t\geq0$, la funci\'on generadora de momentos $\esp\left[e^{\alpha N\left(t\right)}\right]$ existe para alguna $\alpha$ en una vecindad del 0, y de aqu\'i que $\esp\left[N\left(t\right)^{m}\right]<\infty$, para $m\geq1$.
\end{Prop}


\begin{Note}
Si el primer tiempo de renovaci\'on $\xi_{1}$ no tiene la misma distribuci\'on que el resto de las $\xi_{n}$, para $n\geq2$, a $N\left(t\right)$ se le llama Proceso de Renovaci\'on retardado, donde si $\xi$ tiene distribuci\'on $G$, entonces el tiempo $T_{n}$ de la $n$-\'esima renovaci\'on tiene distribuci\'on $G\star F^{\left(n-1\right)\star}\left(t\right)$
\end{Note}


\begin{Teo}
Para una constante $\mu\leq\infty$ ( o variable aleatoria), las siguientes expresiones son equivalentes:

\begin{eqnarray}
lim_{n\rightarrow\infty}n^{-1}T_{n}&=&\mu,\textrm{ c.s.}\\
lim_{t\rightarrow\infty}t^{-1}N\left(t\right)&=&1/\mu,\textrm{ c.s.}
\end{eqnarray}
\end{Teo}


Es decir, $T_{n}$ satisface la Ley Fuerte de los Grandes N\'umeros s\'i y s\'olo s\'i $N\left/t\right)$ la cumple.


\begin{Coro}[Ley Fuerte de los Grandes N\'umeros para Procesos de Renovaci\'on]
Si $N\left(t\right)$ es un proceso de renovaci\'on cuyos tiempos de inter-renovaci\'on tienen media $\mu\leq\infty$, entonces
\begin{eqnarray}
t^{-1}N\left(t\right)\rightarrow 1/\mu,\textrm{ c.s. cuando }t\rightarrow\infty.
\end{eqnarray}

\end{Coro}


Considerar el proceso estoc\'astico de valores reales $\left\{Z\left(t\right):t\geq0\right\}$ en el mismo espacio de probabilidad que $N\left(t\right)$

\begin{Def}
Para el proceso $\left\{Z\left(t\right):t\geq0\right\}$ se define la fluctuaci\'on m\'axima de $Z\left(t\right)$ en el intervalo $\left(T_{n-1},T_{n}\right]$:
\begin{eqnarray*}
M_{n}=\sup_{T_{n-1}<t\leq T_{n}}|Z\left(t\right)-Z\left(T_{n-1}\right)|
\end{eqnarray*}
\end{Def}

\begin{Teo}
Sup\'ongase que $n^{-1}T_{n}\rightarrow\mu$ c.s. cuando $n\rightarrow\infty$, donde $\mu\leq\infty$ es una constante o variable aleatoria. Sea $a$ una constante o variable aleatoria que puede ser infinita cuando $\mu$ es finita, y considere las expresiones l\'imite:
\begin{eqnarray}
lim_{n\rightarrow\infty}n^{-1}Z\left(T_{n}\right)&=&a,\textrm{ c.s.}\\
lim_{t\rightarrow\infty}t^{-1}Z\left(t\right)&=&a/\mu,\textrm{ c.s.}
\end{eqnarray}
La segunda expresi\'on implica la primera. Conversamente, la primera implica la segunda si el proceso $Z\left(t\right)$ es creciente, o si $lim_{n\rightarrow\infty}n^{-1}M_{n}=0$ c.s.
\end{Teo}

\begin{Coro}
Si $N\left(t\right)$ es un proceso de renovaci\'on, y $\left(Z\left(T_{n}\right)-Z\left(T_{n-1}\right),M_{n}\right)$, para $n\geq1$, son variables aleatorias independientes e id\'enticamente distribuidas con media finita, entonces,
\begin{eqnarray}
lim_{t\rightarrow\infty}t^{-1}Z\left(t\right)\rightarrow\frac{\esp\left[Z\left(T_{1}\right)-Z\left(T_{0}\right)\right]}{\esp\left[T_{1}\right]},\textrm{ c.s. cuando  }t\rightarrow\infty.
\end{eqnarray}
\end{Coro}


%__________________________________________________________________________________________
\subsection{Procesos Regenerativos Estacionarios - Stidham \cite{Stidham}}
%__________________________________________________________________________________________


Un proceso estoc\'astico a tiempo continuo $\left\{V\left(t\right),t\geq0\right\}$ es un proceso regenerativo si existe una sucesi\'on de variables aleatorias independientes e id\'enticamente distribuidas $\left\{X_{1},X_{2},\ldots\right\}$, sucesi\'on de renovaci\'on, tal que para cualquier conjunto de Borel $A$, 

\begin{eqnarray*}
\prob\left\{V\left(t\right)\in A|X_{1}+X_{2}+\cdots+X_{R\left(t\right)}=s,\left\{V\left(\tau\right),\tau<s\right\}\right\}=\prob\left\{V\left(t-s\right)\in A|X_{1}>t-s\right\},
\end{eqnarray*}
para todo $0\leq s\leq t$, donde $R\left(t\right)=\max\left\{X_{1}+X_{2}+\cdots+X_{j}\leq t\right\}=$n\'umero de renovaciones ({\emph{puntos de regeneraci\'on}}) que ocurren en $\left[0,t\right]$. El intervalo $\left[0,X_{1}\right)$ es llamado {\emph{primer ciclo de regeneraci\'on}} de $\left\{V\left(t \right),t\geq0\right\}$, $\left[X_{1},X_{1}+X_{2}\right)$ el {\emph{segundo ciclo de regeneraci\'on}}, y as\'i sucesivamente.

Sea $X=X_{1}$ y sea $F$ la funci\'on de distrbuci\'on de $X$


\begin{Def}
Se define el proceso estacionario, $\left\{V^{*}\left(t\right),t\geq0\right\}$, para $\left\{V\left(t\right),t\geq0\right\}$ por

\begin{eqnarray*}
\prob\left\{V\left(t\right)\in A\right\}=\frac{1}{\esp\left[X\right]}\int_{0}^{\infty}\prob\left\{V\left(t+x\right)\in A|X>x\right\}\left(1-F\left(x\right)\right)dx,
\end{eqnarray*} 
para todo $t\geq0$ y todo conjunto de Borel $A$.
\end{Def}

\begin{Def}
Una distribuci\'on se dice que es {\emph{aritm\'etica}} si todos sus puntos de incremento son m\'ultiplos de la forma $0,\lambda, 2\lambda,\ldots$ para alguna $\lambda>0$ entera.
\end{Def}


\begin{Def}
Una modificaci\'on medible de un proceso $\left\{V\left(t\right),t\geq0\right\}$, es una versi\'on de este, $\left\{V\left(t,w\right)\right\}$ conjuntamente medible para $t\geq0$ y para $w\in S$, $S$ espacio de estados para $\left\{V\left(t\right),t\geq0\right\}$.
\end{Def}

\begin{Teo}
Sea $\left\{V\left(t\right),t\geq\right\}$ un proceso regenerativo no negativo con modificaci\'on medible. Sea $\esp\left[X\right]<\infty$. Entonces el proceso estacionario dado por la ecuaci\'on anterior est\'a bien definido y tiene funci\'on de distribuci\'on independiente de $t$, adem\'as
\begin{itemize}
\item[i)] \begin{eqnarray*}
\esp\left[V^{*}\left(0\right)\right]&=&\frac{\esp\left[\int_{0}^{X}V\left(s\right)ds\right]}{\esp\left[X\right]}\end{eqnarray*}
\item[ii)] Si $\esp\left[V^{*}\left(0\right)\right]<\infty$, equivalentemente, si $\esp\left[\int_{0}^{X}V\left(s\right)ds\right]<\infty$,entonces
\begin{eqnarray*}
\frac{\int_{0}^{t}V\left(s\right)ds}{t}\rightarrow\frac{\esp\left[\int_{0}^{X}V\left(s\right)ds\right]}{\esp\left[X\right]}
\end{eqnarray*}
con probabilidad 1 y en media, cuando $t\rightarrow\infty$.
\end{itemize}
\end{Teo}

%______________________________________________________________________
\subsection{Procesos de Renovaci\'on}
%______________________________________________________________________

\begin{Def}\label{Def.Tn}
Sean $0\leq T_{1}\leq T_{2}\leq \ldots$ son tiempos aleatorios infinitos en los cuales ocurren ciertos eventos. El n\'umero de tiempos $T_{n}$ en el intervalo $\left[0,t\right)$ es

\begin{eqnarray}
N\left(t\right)=\sum_{n=1}^{\infty}\indora\left(T_{n}\leq t\right),
\end{eqnarray}
para $t\geq0$.
\end{Def}

Si se consideran los puntos $T_{n}$ como elementos de $\rea_{+}$, y $N\left(t\right)$ es el n\'umero de puntos en $\rea$. El proceso denotado por $\left\{N\left(t\right):t\geq0\right\}$, denotado por $N\left(t\right)$, es un proceso puntual en $\rea_{+}$. Los $T_{n}$ son los tiempos de ocurrencia, el proceso puntual $N\left(t\right)$ es simple si su n\'umero de ocurrencias son distintas: $0<T_{1}<T_{2}<\ldots$ casi seguramente.

\begin{Def}
Un proceso puntual $N\left(t\right)$ es un proceso de renovaci\'on si los tiempos de interocurrencia $\xi_{n}=T_{n}-T_{n-1}$, para $n\geq1$, son independientes e identicamente distribuidos con distribuci\'on $F$, donde $F\left(0\right)=0$ y $T_{0}=0$. Los $T_{n}$ son llamados tiempos de renovaci\'on, referente a la independencia o renovaci\'on de la informaci\'on estoc\'astica en estos tiempos. Los $\xi_{n}$ son los tiempos de inter-renovaci\'on, y $N\left(t\right)$ es el n\'umero de renovaciones en el intervalo $\left[0,t\right)$
\end{Def}


\begin{Note}
Para definir un proceso de renovaci\'on para cualquier contexto, solamente hay que especificar una distribuci\'on $F$, con $F\left(0\right)=0$, para los tiempos de inter-renovaci\'on. La funci\'on $F$ en turno degune las otra variables aleatorias. De manera formal, existe un espacio de probabilidad y una sucesi\'on de variables aleatorias $\xi_{1},\xi_{2},\ldots$ definidas en este con distribuci\'on $F$. Entonces las otras cantidades son $T_{n}=\sum_{k=1}^{n}\xi_{k}$ y $N\left(t\right)=\sum_{n=1}^{\infty}\indora\left(T_{n}\leq t\right)$, donde $T_{n}\rightarrow\infty$ casi seguramente por la Ley Fuerte de los Grandes Números.
\end{Note}

%___________________________________________________________________________________________
%
\subsection{Teorema Principal de Renovaci\'on}
%___________________________________________________________________________________________
%

\begin{Note} Una funci\'on $h:\rea_{+}\rightarrow\rea$ es Directamente Riemann Integrable en los siguientes casos:
\begin{itemize}
\item[a)] $h\left(t\right)\geq0$ es decreciente y Riemann Integrable.
\item[b)] $h$ es continua excepto posiblemente en un conjunto de Lebesgue de medida 0, y $|h\left(t\right)|\leq b\left(t\right)$, donde $b$ es DRI.
\end{itemize}
\end{Note}

\begin{Teo}[Teorema Principal de Renovaci\'on]
Si $F$ es no aritm\'etica y $h\left(t\right)$ es Directamente Riemann Integrable (DRI), entonces

\begin{eqnarray*}
lim_{t\rightarrow\infty}U\star h=\frac{1}{\mu}\int_{\rea_{+}}h\left(s\right)ds.
\end{eqnarray*}
\end{Teo}

\begin{Prop}
Cualquier funci\'on $H\left(t\right)$ acotada en intervalos finitos y que es 0 para $t<0$ puede expresarse como
\begin{eqnarray*}
H\left(t\right)=U\star h\left(t\right)\textrm{,  donde }h\left(t\right)=H\left(t\right)-F\star H\left(t\right)
\end{eqnarray*}
\end{Prop}

\begin{Def}
Un proceso estoc\'astico $X\left(t\right)$ es crudamente regenerativo en un tiempo aleatorio positivo $T$ si
\begin{eqnarray*}
\esp\left[X\left(T+t\right)|T\right]=\esp\left[X\left(t\right)\right]\textrm{, para }t\geq0,\end{eqnarray*}
y con las esperanzas anteriores finitas.
\end{Def}

\begin{Prop}
Sup\'ongase que $X\left(t\right)$ es un proceso crudamente regenerativo en $T$, que tiene distribuci\'on $F$. Si $\esp\left[X\left(t\right)\right]$ es acotado en intervalos finitos, entonces
\begin{eqnarray*}
\esp\left[X\left(t\right)\right]=U\star h\left(t\right)\textrm{,  donde }h\left(t\right)=\esp\left[X\left(t\right)\indora\left(T>t\right)\right].
\end{eqnarray*}
\end{Prop}

\begin{Teo}[Regeneraci\'on Cruda]
Sup\'ongase que $X\left(t\right)$ es un proceso con valores positivo crudamente regenerativo en $T$, y def\'inase $M=\sup\left\{|X\left(t\right)|:t\leq T\right\}$. Si $T$ es no aritm\'etico y $M$ y $MT$ tienen media finita, entonces
\begin{eqnarray*}
lim_{t\rightarrow\infty}\esp\left[X\left(t\right)\right]=\frac{1}{\mu}\int_{\rea_{+}}h\left(s\right)ds,
\end{eqnarray*}
donde $h\left(t\right)=\esp\left[X\left(t\right)\indora\left(T>t\right)\right]$.
\end{Teo}



%___________________________________________________________________________________________
%
\subsection{Funci\'on de Renovaci\'on}
%___________________________________________________________________________________________
%


\begin{Def}
Sea $h\left(t\right)$ funci\'on de valores reales en $\rea$ acotada en intervalos finitos e igual a cero para $t<0$ La ecuaci\'on de renovaci\'on para $h\left(t\right)$ y la distribuci\'on $F$ es

\begin{eqnarray}\label{Ec.Renovacion}
H\left(t\right)=h\left(t\right)+\int_{\left[0,t\right]}H\left(t-s\right)dF\left(s\right)\textrm{,    }t\geq0,
\end{eqnarray}
donde $H\left(t\right)$ es una funci\'on de valores reales. Esto es $H=h+F\star H$. Decimos que $H\left(t\right)$ es soluci\'on de esta ecuaci\'on si satisface la ecuaci\'on, y es acotada en intervalos finitos e iguales a cero para $t<0$.
\end{Def}

\begin{Prop}
La funci\'on $U\star h\left(t\right)$ es la \'unica soluci\'on de la ecuaci\'on de renovaci\'on (\ref{Ec.Renovacion}).
\end{Prop}

\begin{Teo}[Teorema Renovaci\'on Elemental]
\begin{eqnarray*}
t^{-1}U\left(t\right)\rightarrow 1/\mu\textrm{,    cuando }t\rightarrow\infty.
\end{eqnarray*}
\end{Teo}

%___________________________________________________________________________________________
%
\subsection{Propiedades de los Procesos de Renovaci\'on}
%___________________________________________________________________________________________
%

Los tiempos $T_{n}$ est\'an relacionados con los conteos de $N\left(t\right)$ por

\begin{eqnarray*}
\left\{N\left(t\right)\geq n\right\}&=&\left\{T_{n}\leq t\right\}\\
T_{N\left(t\right)}\leq &t&<T_{N\left(t\right)+1},
\end{eqnarray*}

adem\'as $N\left(T_{n}\right)=n$, y 

\begin{eqnarray*}
N\left(t\right)=\max\left\{n:T_{n}\leq t\right\}=\min\left\{n:T_{n+1}>t\right\}
\end{eqnarray*}

Por propiedades de la convoluci\'on se sabe que

\begin{eqnarray*}
P\left\{T_{n}\leq t\right\}=F^{n\star}\left(t\right)
\end{eqnarray*}
que es la $n$-\'esima convoluci\'on de $F$. Entonces 

\begin{eqnarray*}
\left\{N\left(t\right)\geq n\right\}&=&\left\{T_{n}\leq t\right\}\\
P\left\{N\left(t\right)\leq n\right\}&=&1-F^{\left(n+1\right)\star}\left(t\right)
\end{eqnarray*}

Adem\'as usando el hecho de que $\esp\left[N\left(t\right)\right]=\sum_{n=1}^{\infty}P\left\{N\left(t\right)\geq n\right\}$
se tiene que

\begin{eqnarray*}
\esp\left[N\left(t\right)\right]=\sum_{n=1}^{\infty}F^{n\star}\left(t\right)
\end{eqnarray*}

\begin{Prop}
Para cada $t\geq0$, la funci\'on generadora de momentos $\esp\left[e^{\alpha N\left(t\right)}\right]$ existe para alguna $\alpha$ en una vecindad del 0, y de aqu\'i que $\esp\left[N\left(t\right)^{m}\right]<\infty$, para $m\geq1$.
\end{Prop}


\begin{Note}
Si el primer tiempo de renovaci\'on $\xi_{1}$ no tiene la misma distribuci\'on que el resto de las $\xi_{n}$, para $n\geq2$, a $N\left(t\right)$ se le llama Proceso de Renovaci\'on retardado, donde si $\xi$ tiene distribuci\'on $G$, entonces el tiempo $T_{n}$ de la $n$-\'esima renovaci\'on tiene distribuci\'on $G\star F^{\left(n-1\right)\star}\left(t\right)$
\end{Note}


\begin{Teo}
Para una constante $\mu\leq\infty$ ( o variable aleatoria), las siguientes expresiones son equivalentes:

\begin{eqnarray}
lim_{n\rightarrow\infty}n^{-1}T_{n}&=&\mu,\textrm{ c.s.}\\
lim_{t\rightarrow\infty}t^{-1}N\left(t\right)&=&1/\mu,\textrm{ c.s.}
\end{eqnarray}
\end{Teo}


Es decir, $T_{n}$ satisface la Ley Fuerte de los Grandes N\'umeros s\'i y s\'olo s\'i $N\left/t\right)$ la cumple.


\begin{Coro}[Ley Fuerte de los Grandes N\'umeros para Procesos de Renovaci\'on]
Si $N\left(t\right)$ es un proceso de renovaci\'on cuyos tiempos de inter-renovaci\'on tienen media $\mu\leq\infty$, entonces
\begin{eqnarray}
t^{-1}N\left(t\right)\rightarrow 1/\mu,\textrm{ c.s. cuando }t\rightarrow\infty.
\end{eqnarray}

\end{Coro}


Considerar el proceso estoc\'astico de valores reales $\left\{Z\left(t\right):t\geq0\right\}$ en el mismo espacio de probabilidad que $N\left(t\right)$

\begin{Def}
Para el proceso $\left\{Z\left(t\right):t\geq0\right\}$ se define la fluctuaci\'on m\'axima de $Z\left(t\right)$ en el intervalo $\left(T_{n-1},T_{n}\right]$:
\begin{eqnarray*}
M_{n}=\sup_{T_{n-1}<t\leq T_{n}}|Z\left(t\right)-Z\left(T_{n-1}\right)|
\end{eqnarray*}
\end{Def}

\begin{Teo}
Sup\'ongase que $n^{-1}T_{n}\rightarrow\mu$ c.s. cuando $n\rightarrow\infty$, donde $\mu\leq\infty$ es una constante o variable aleatoria. Sea $a$ una constante o variable aleatoria que puede ser infinita cuando $\mu$ es finita, y considere las expresiones l\'imite:
\begin{eqnarray}
lim_{n\rightarrow\infty}n^{-1}Z\left(T_{n}\right)&=&a,\textrm{ c.s.}\\
lim_{t\rightarrow\infty}t^{-1}Z\left(t\right)&=&a/\mu,\textrm{ c.s.}
\end{eqnarray}
La segunda expresi\'on implica la primera. Conversamente, la primera implica la segunda si el proceso $Z\left(t\right)$ es creciente, o si $lim_{n\rightarrow\infty}n^{-1}M_{n}=0$ c.s.
\end{Teo}

\begin{Coro}
Si $N\left(t\right)$ es un proceso de renovaci\'on, y $\left(Z\left(T_{n}\right)-Z\left(T_{n-1}\right),M_{n}\right)$, para $n\geq1$, son variables aleatorias independientes e id\'enticamente distribuidas con media finita, entonces,
\begin{eqnarray}
lim_{t\rightarrow\infty}t^{-1}Z\left(t\right)\rightarrow\frac{\esp\left[Z\left(T_{1}\right)-Z\left(T_{0}\right)\right]}{\esp\left[T_{1}\right]},\textrm{ c.s. cuando  }t\rightarrow\infty.
\end{eqnarray}
\end{Coro}

%___________________________________________________________________________________________
%
\subsection{Funci\'on de Renovaci\'on}
%___________________________________________________________________________________________
%


Sup\'ongase que $N\left(t\right)$ es un proceso de renovaci\'on con distribuci\'on $F$ con media finita $\mu$.

\begin{Def}
La funci\'on de renovaci\'on asociada con la distribuci\'on $F$, del proceso $N\left(t\right)$, es
\begin{eqnarray*}
U\left(t\right)=\sum_{n=1}^{\infty}F^{n\star}\left(t\right),\textrm{   }t\geq0,
\end{eqnarray*}
donde $F^{0\star}\left(t\right)=\indora\left(t\geq0\right)$.
\end{Def}


\begin{Prop}
Sup\'ongase que la distribuci\'on de inter-renovaci\'on $F$ tiene densidad $f$. Entonces $U\left(t\right)$ tambi\'en tiene densidad, para $t>0$, y es $U^{'}\left(t\right)=\sum_{n=0}^{\infty}f^{n\star}\left(t\right)$. Adem\'as
\begin{eqnarray*}
\prob\left\{N\left(t\right)>N\left(t-\right)\right\}=0\textrm{,   }t\geq0.
\end{eqnarray*}
\end{Prop}

\begin{Def}
La Transformada de Laplace-Stieljes de $F$ est\'a dada por

\begin{eqnarray*}
\hat{F}\left(\alpha\right)=\int_{\rea_{+}}e^{-\alpha t}dF\left(t\right)\textrm{,  }\alpha\geq0.
\end{eqnarray*}
\end{Def}

Entonces

\begin{eqnarray*}
\hat{U}\left(\alpha\right)=\sum_{n=0}^{\infty}\hat{F^{n\star}}\left(\alpha\right)=\sum_{n=0}^{\infty}\hat{F}\left(\alpha\right)^{n}=\frac{1}{1-\hat{F}\left(\alpha\right)}.
\end{eqnarray*}


\begin{Prop}
La Transformada de Laplace $\hat{U}\left(\alpha\right)$ y $\hat{F}\left(\alpha\right)$ determina una a la otra de manera \'unica por la relaci\'on $\hat{U}\left(\alpha\right)=\frac{1}{1-\hat{F}\left(\alpha\right)}$.
\end{Prop}


\begin{Note}
Un proceso de renovaci\'on $N\left(t\right)$ cuyos tiempos de inter-renovaci\'on tienen media finita, es un proceso Poisson con tasa $\lambda$ si y s\'olo s\'i $\esp\left[U\left(t\right)\right]=\lambda t$, para $t\geq0$.
\end{Note}


\begin{Teo}
Sea $N\left(t\right)$ un proceso puntual simple con puntos de localizaci\'on $T_{n}$ tal que $\eta\left(t\right)=\esp\left[N\left(\right)\right]$ es finita para cada $t$. Entonces para cualquier funci\'on $f:\rea_{+}\rightarrow\rea$,
\begin{eqnarray*}
\esp\left[\sum_{n=1}^{N\left(\right)}f\left(T_{n}\right)\right]=\int_{\left(0,t\right]}f\left(s\right)d\eta\left(s\right)\textrm{,  }t\geq0,
\end{eqnarray*}
suponiendo que la integral exista. Adem\'as si $X_{1},X_{2},\ldots$ son variables aleatorias definidas en el mismo espacio de probabilidad que el proceso $N\left(t\right)$ tal que $\esp\left[X_{n}|T_{n}=s\right]=f\left(s\right)$, independiente de $n$. Entonces
\begin{eqnarray*}
\esp\left[\sum_{n=1}^{N\left(t\right)}X_{n}\right]=\int_{\left(0,t\right]}f\left(s\right)d\eta\left(s\right)\textrm{,  }t\geq0,
\end{eqnarray*} 
suponiendo que la integral exista. 
\end{Teo}

\begin{Coro}[Identidad de Wald para Renovaciones]
Para el proceso de renovaci\'on $N\left(t\right)$,
\begin{eqnarray*}
\esp\left[T_{N\left(t\right)+1}\right]=\mu\esp\left[N\left(t\right)+1\right]\textrm{,  }t\geq0,
\end{eqnarray*}  
\end{Coro}

%______________________________________________________________________
\subsection{Procesos de Renovaci\'on}
%______________________________________________________________________

\begin{Def}\label{Def.Tn}
Sean $0\leq T_{1}\leq T_{2}\leq \ldots$ son tiempos aleatorios infinitos en los cuales ocurren ciertos eventos. El n\'umero de tiempos $T_{n}$ en el intervalo $\left[0,t\right)$ es

\begin{eqnarray}
N\left(t\right)=\sum_{n=1}^{\infty}\indora\left(T_{n}\leq t\right),
\end{eqnarray}
para $t\geq0$.
\end{Def}

Si se consideran los puntos $T_{n}$ como elementos de $\rea_{+}$, y $N\left(t\right)$ es el n\'umero de puntos en $\rea$. El proceso denotado por $\left\{N\left(t\right):t\geq0\right\}$, denotado por $N\left(t\right)$, es un proceso puntual en $\rea_{+}$. Los $T_{n}$ son los tiempos de ocurrencia, el proceso puntual $N\left(t\right)$ es simple si su n\'umero de ocurrencias son distintas: $0<T_{1}<T_{2}<\ldots$ casi seguramente.

\begin{Def}
Un proceso puntual $N\left(t\right)$ es un proceso de renovaci\'on si los tiempos de interocurrencia $\xi_{n}=T_{n}-T_{n-1}$, para $n\geq1$, son independientes e identicamente distribuidos con distribuci\'on $F$, donde $F\left(0\right)=0$ y $T_{0}=0$. Los $T_{n}$ son llamados tiempos de renovaci\'on, referente a la independencia o renovaci\'on de la informaci\'on estoc\'astica en estos tiempos. Los $\xi_{n}$ son los tiempos de inter-renovaci\'on, y $N\left(t\right)$ es el n\'umero de renovaciones en el intervalo $\left[0,t\right)$
\end{Def}


\begin{Note}
Para definir un proceso de renovaci\'on para cualquier contexto, solamente hay que especificar una distribuci\'on $F$, con $F\left(0\right)=0$, para los tiempos de inter-renovaci\'on. La funci\'on $F$ en turno degune las otra variables aleatorias. De manera formal, existe un espacio de probabilidad y una sucesi\'on de variables aleatorias $\xi_{1},\xi_{2},\ldots$ definidas en este con distribuci\'on $F$. Entonces las otras cantidades son $T_{n}=\sum_{k=1}^{n}\xi_{k}$ y $N\left(t\right)=\sum_{n=1}^{\infty}\indora\left(T_{n}\leq t\right)$, donde $T_{n}\rightarrow\infty$ casi seguramente por la Ley Fuerte de los Grandes Números.
\end{Note}
%_____________________________________________________
\subsection{Puntos de Renovaci\'on}
%_____________________________________________________

Para cada cola $Q_{i}$ se tienen los procesos de arribo a la cola, para estas, los tiempos de arribo est\'an dados por $$\left\{T_{1}^{i},T_{2}^{i},\ldots,T_{k}^{i},\ldots\right\},$$ entonces, consideremos solamente los primeros tiempos de arribo a cada una de las colas, es decir, $$\left\{T_{1}^{1},T_{1}^{2},T_{1}^{3},T_{1}^{4}\right\},$$ se sabe que cada uno de estos tiempos se distribuye de manera exponencial con par\'ametro $1/mu_{i}$. Adem\'as se sabe que para $$T^{*}=\min\left\{T_{1}^{1},T_{1}^{2},T_{1}^{3},T_{1}^{4}\right\},$$ $T^{*}$ se distribuye de manera exponencial con par\'ametro $$\mu^{*}=\sum_{i=1}^{4}\mu_{i}.$$ Ahora, dado que 
\begin{center}
\begin{tabular}{lcl}
$\tilde{r}=r_{1}+r_{2}$ & y &$\hat{r}=r_{3}+r_{4}.$
\end{tabular}
\end{center}


Supongamos que $$\tilde{r},\hat{r}<\mu^{*},$$ entonces si tomamos $$r^{*}=\min\left\{\tilde{r},\hat{r}\right\},$$ se tiene que para  $$t^{*}\in\left(0,r^{*}\right)$$ se cumple que 
\begin{center}
\begin{tabular}{lcl}
$\tau_{1}\left(1\right)=0$ & y por tanto & $\overline{\tau}_{1}=0,$
\end{tabular}
\end{center}
entonces para la segunda cola en este primer ciclo se cumple que $$\tau_{2}=\overline{\tau}_{1}+r_{1}=r_{1}<\mu^{*},$$ y por tanto se tiene que  $$\overline{\tau}_{2}=\tau_{2}.$$ Por lo tanto, nuevamente para la primer cola en el segundo ciclo $$\tau_{1}\left(2\right)=\tau_{2}\left(1\right)+r_{2}=\tilde{r}<\mu^{*}.$$ An\'alogamente para el segundo sistema se tiene que ambas colas est\'an vac\'ias, es decir, existe un valor $t^{*}$ tal que en el intervalo $\left(0,t^{*}\right)$ no ha llegado ning\'un usuario, es decir, $$L_{i}\left(t^{*}\right)=0$$ para $i=1,2,3,4$.



%________________________________________________________________________
\subsection{Procesos Regenerativos}
%________________________________________________________________________

Para $\left\{X\left(t\right):t\geq0\right\}$ Proceso Estoc\'astico a tiempo continuo con estado de espacios $S$, que es un espacio m\'etrico, con trayectorias continuas por la derecha y con l\'imites por la izquierda c.s. Sea $N\left(t\right)$ un proceso de renovaci\'on en $\rea_{+}$ definido en el mismo espacio de probabilidad que $X\left(t\right)$, con tiempos de renovaci\'on $T$ y tiempos de inter-renovaci\'on $\xi_{n}=T_{n}-T_{n-1}$, con misma distribuci\'on $F$ de media finita $\mu$.



\begin{Def}
Para el proceso $\left\{\left(N\left(t\right),X\left(t\right)\right):t\geq0\right\}$, sus trayectoria muestrales en el intervalo de tiempo $\left[T_{n-1},T_{n}\right)$ est\'an descritas por
\begin{eqnarray*}
\zeta_{n}=\left(\xi_{n},\left\{X\left(T_{n-1}+t\right):0\leq t<\xi_{n}\right\}\right)
\end{eqnarray*}
Este $\zeta_{n}$ es el $n$-\'esimo segmento del proceso. El proceso es regenerativo sobre los tiempos $T_{n}$ si sus segmentos $\zeta_{n}$ son independientes e id\'enticamennte distribuidos.
\end{Def}


\begin{Obs}
Si $\tilde{X}\left(t\right)$ con espacio de estados $\tilde{S}$ es regenerativo sobre $T_{n}$, entonces $X\left(t\right)=f\left(\tilde{X}\left(t\right)\right)$ tambi\'en es regenerativo sobre $T_{n}$, para cualquier funci\'on $f:\tilde{S}\rightarrow S$.
\end{Obs}

\begin{Obs}
Los procesos regenerativos son crudamente regenerativos, pero no al rev\'es.
\end{Obs}

\begin{Def}[Definici\'on Cl\'asica]
Un proceso estoc\'astico $X=\left\{X\left(t\right):t\geq0\right\}$ es llamado regenerativo is existe una variable aleatoria $R_{1}>0$ tal que
\begin{itemize}
\item[i)] $\left\{X\left(t+R_{1}\right):t\geq0\right\}$ es independiente de $\left\{\left\{X\left(t\right):t<R_{1}\right\},\right\}$
\item[ii)] $\left\{X\left(t+R_{1}\right):t\geq0\right\}$ es estoc\'asticamente equivalente a $\left\{X\left(t\right):t>0\right\}$
\end{itemize}

Llamamos a $R_{1}$ tiempo de regeneraci\'on, y decimos que $X$ se regenera en este punto.
\end{Def}

$\left\{X\left(t+R_{1}\right)\right\}$ es regenerativo con tiempo de regeneraci\'on $R_{2}$, independiente de $R_{1}$ pero con la misma distribuci\'on que $R_{1}$. Procediendo de esta manera se obtiene una secuencia de variables aleatorias independientes e id\'enticamente distribuidas $\left\{R_{n}\right\}$ llamados longitudes de ciclo. Si definimos a $Z_{k}\equiv R_{1}+R_{2}+\cdots+R_{k}$, se tiene un proceso de renovaci\'on llamado proceso de renovaci\'on encajado para $X$.

\begin{Note}
Un proceso regenerativo con media de la longitud de ciclo finita es llamado positivo recurrente.
\end{Note}


\begin{Def}
Para $x$ fijo y para cada $t\geq0$, sea $I_{x}\left(t\right)=1$ si $X\left(t\right)\leq x$,  $I_{x}\left(t\right)=0$ en caso contrario, y def\'inanse los tiempos promedio
\begin{eqnarray*}
\overline{X}&=&lim_{t\rightarrow\infty}\frac{1}{t}\int_{0}^{\infty}X\left(u\right)du\\
\prob\left(X_{\infty}\leq x\right)&=&lim_{t\rightarrow\infty}\frac{1}{t}\int_{0}^{\infty}I_{x}\left(u\right)du,
\end{eqnarray*}
cuando estos l\'imites existan.
\end{Def}

Como consecuencia del teorema de Renovaci\'on-Recompensa, se tiene que el primer l\'imite  existe y es igual a la constante
\begin{eqnarray*}
\overline{X}&=&\frac{\esp\left[\int_{0}^{R_{1}}X\left(t\right)dt\right]}{\esp\left[R_{1}\right]},
\end{eqnarray*}
suponiendo que ambas esperanzas son finitas.

\begin{Note}
\begin{itemize}
\item[a)] Si el proceso regenerativo $X$ es positivo recurrente y tiene trayectorias muestrales no negativas, entonces la ecuaci\'on anterior es v\'alida.
\item[b)] Si $X$ es positivo recurrente regenerativo, podemos construir una \'unica versi\'on estacionaria de este proceso, $X_{e}=\left\{X_{e}\left(t\right)\right\}$, donde $X_{e}$ es un proceso estoc\'astico regenerativo y estrictamente estacionario, con distribuci\'on marginal distribuida como $X_{\infty}$
\end{itemize}
\end{Note}

\subsection{Renewal and Regenerative Processes: Serfozo\cite{Serfozo}}
\begin{Def}\label{Def.Tn}
Sean $0\leq T_{1}\leq T_{2}\leq \ldots$ son tiempos aleatorios infinitos en los cuales ocurren ciertos eventos. El n\'umero de tiempos $T_{n}$ en el intervalo $\left[0,t\right)$ es

\begin{eqnarray}
N\left(t\right)=\sum_{n=1}^{\infty}\indora\left(T_{n}\leq t\right),
\end{eqnarray}
para $t\geq0$.
\end{Def}

Si se consideran los puntos $T_{n}$ como elementos de $\rea_{+}$, y $N\left(t\right)$ es el n\'umero de puntos en $\rea$. El proceso denotado por $\left\{N\left(t\right):t\geq0\right\}$, denotado por $N\left(t\right)$, es un proceso puntual en $\rea_{+}$. Los $T_{n}$ son los tiempos de ocurrencia, el proceso puntual $N\left(t\right)$ es simple si su n\'umero de ocurrencias son distintas: $0<T_{1}<T_{2}<\ldots$ casi seguramente.

\begin{Def}
Un proceso puntual $N\left(t\right)$ es un proceso de renovaci\'on si los tiempos de interocurrencia $\xi_{n}=T_{n}-T_{n-1}$, para $n\geq1$, son independientes e identicamente distribuidos con distribuci\'on $F$, donde $F\left(0\right)=0$ y $T_{0}=0$. Los $T_{n}$ son llamados tiempos de renovaci\'on, referente a la independencia o renovaci\'on de la informaci\'on estoc\'astica en estos tiempos. Los $\xi_{n}$ son los tiempos de inter-renovaci\'on, y $N\left(t\right)$ es el n\'umero de renovaciones en el intervalo $\left[0,t\right)$
\end{Def}


\begin{Note}
Para definir un proceso de renovaci\'on para cualquier contexto, solamente hay que especificar una distribuci\'on $F$, con $F\left(0\right)=0$, para los tiempos de inter-renovaci\'on. La funci\'on $F$ en turno degune las otra variables aleatorias. De manera formal, existe un espacio de probabilidad y una sucesi\'on de variables aleatorias $\xi_{1},\xi_{2},\ldots$ definidas en este con distribuci\'on $F$. Entonces las otras cantidades son $T_{n}=\sum_{k=1}^{n}\xi_{k}$ y $N\left(t\right)=\sum_{n=1}^{\infty}\indora\left(T_{n}\leq t\right)$, donde $T_{n}\rightarrow\infty$ casi seguramente por la Ley Fuerte de los Grandes N\'umeros.
\end{Note}







Los tiempos $T_{n}$ est\'an relacionados con los conteos de $N\left(t\right)$ por

\begin{eqnarray*}
\left\{N\left(t\right)\geq n\right\}&=&\left\{T_{n}\leq t\right\}\\
T_{N\left(t\right)}\leq &t&<T_{N\left(t\right)+1},
\end{eqnarray*}

adem\'as $N\left(T_{n}\right)=n$, y 

\begin{eqnarray*}
N\left(t\right)=\max\left\{n:T_{n}\leq t\right\}=\min\left\{n:T_{n+1}>t\right\}
\end{eqnarray*}

Por propiedades de la convoluci\'on se sabe que

\begin{eqnarray*}
P\left\{T_{n}\leq t\right\}=F^{n\star}\left(t\right)
\end{eqnarray*}
que es la $n$-\'esima convoluci\'on de $F$. Entonces 

\begin{eqnarray*}
\left\{N\left(t\right)\geq n\right\}&=&\left\{T_{n}\leq t\right\}\\
P\left\{N\left(t\right)\leq n\right\}&=&1-F^{\left(n+1\right)\star}\left(t\right)
\end{eqnarray*}

Adem\'as usando el hecho de que $\esp\left[N\left(t\right)\right]=\sum_{n=1}^{\infty}P\left\{N\left(t\right)\geq n\right\}$
se tiene que

\begin{eqnarray*}
\esp\left[N\left(t\right)\right]=\sum_{n=1}^{\infty}F^{n\star}\left(t\right)
\end{eqnarray*}

\begin{Prop}
Para cada $t\geq0$, la funci\'on generadora de momentos $\esp\left[e^{\alpha N\left(t\right)}\right]$ existe para alguna $\alpha$ en una vecindad del 0, y de aqu\'i que $\esp\left[N\left(t\right)^{m}\right]<\infty$, para $m\geq1$.
\end{Prop}


\begin{Note}
Si el primer tiempo de renovaci\'on $\xi_{1}$ no tiene la misma distribuci\'on que el resto de las $\xi_{n}$, para $n\geq2$, a $N\left(t\right)$ se le llama Proceso de Renovaci\'on retardado, donde si $\xi$ tiene distribuci\'on $G$, entonces el tiempo $T_{n}$ de la $n$-\'esima renovaci\'on tiene distribuci\'on $G\star F^{\left(n-1\right)\star}\left(t\right)$
\end{Note}


\begin{Teo}
Para una constante $\mu\leq\infty$ ( o variable aleatoria), las siguientes expresiones son equivalentes:

\begin{eqnarray}
lim_{n\rightarrow\infty}n^{-1}T_{n}&=&\mu,\textrm{ c.s.}\\
lim_{t\rightarrow\infty}t^{-1}N\left(t\right)&=&1/\mu,\textrm{ c.s.}
\end{eqnarray}
\end{Teo}


Es decir, $T_{n}$ satisface la Ley Fuerte de los Grandes N\'umeros s\'i y s\'olo s\'i $N\left/t\right)$ la cumple.


\begin{Coro}[Ley Fuerte de los Grandes N\'umeros para Procesos de Renovaci\'on]
Si $N\left(t\right)$ es un proceso de renovaci\'on cuyos tiempos de inter-renovaci\'on tienen media $\mu\leq\infty$, entonces
\begin{eqnarray}
t^{-1}N\left(t\right)\rightarrow 1/\mu,\textrm{ c.s. cuando }t\rightarrow\infty.
\end{eqnarray}

\end{Coro}


Considerar el proceso estoc\'astico de valores reales $\left\{Z\left(t\right):t\geq0\right\}$ en el mismo espacio de probabilidad que $N\left(t\right)$

\begin{Def}
Para el proceso $\left\{Z\left(t\right):t\geq0\right\}$ se define la fluctuaci\'on m\'axima de $Z\left(t\right)$ en el intervalo $\left(T_{n-1},T_{n}\right]$:
\begin{eqnarray*}
M_{n}=\sup_{T_{n-1}<t\leq T_{n}}|Z\left(t\right)-Z\left(T_{n-1}\right)|
\end{eqnarray*}
\end{Def}

\begin{Teo}
Sup\'ongase que $n^{-1}T_{n}\rightarrow\mu$ c.s. cuando $n\rightarrow\infty$, donde $\mu\leq\infty$ es una constante o variable aleatoria. Sea $a$ una constante o variable aleatoria que puede ser infinita cuando $\mu$ es finita, y considere las expresiones l\'imite:
\begin{eqnarray}
lim_{n\rightarrow\infty}n^{-1}Z\left(T_{n}\right)&=&a,\textrm{ c.s.}\\
lim_{t\rightarrow\infty}t^{-1}Z\left(t\right)&=&a/\mu,\textrm{ c.s.}
\end{eqnarray}
La segunda expresi\'on implica la primera. Conversamente, la primera implica la segunda si el proceso $Z\left(t\right)$ es creciente, o si $lim_{n\rightarrow\infty}n^{-1}M_{n}=0$ c.s.
\end{Teo}

\begin{Coro}
Si $N\left(t\right)$ es un proceso de renovaci\'on, y $\left(Z\left(T_{n}\right)-Z\left(T_{n-1}\right),M_{n}\right)$, para $n\geq1$, son variables aleatorias independientes e id\'enticamente distribuidas con media finita, entonces,
\begin{eqnarray}
lim_{t\rightarrow\infty}t^{-1}Z\left(t\right)\rightarrow\frac{\esp\left[Z\left(T_{1}\right)-Z\left(T_{0}\right)\right]}{\esp\left[T_{1}\right]},\textrm{ c.s. cuando  }t\rightarrow\infty.
\end{eqnarray}
\end{Coro}


Sup\'ongase que $N\left(t\right)$ es un proceso de renovaci\'on con distribuci\'on $F$ con media finita $\mu$.

\begin{Def}
La funci\'on de renovaci\'on asociada con la distribuci\'on $F$, del proceso $N\left(t\right)$, es
\begin{eqnarray*}
U\left(t\right)=\sum_{n=1}^{\infty}F^{n\star}\left(t\right),\textrm{   }t\geq0,
\end{eqnarray*}
donde $F^{0\star}\left(t\right)=\indora\left(t\geq0\right)$.
\end{Def}


\begin{Prop}
Sup\'ongase que la distribuci\'on de inter-renovaci\'on $F$ tiene densidad $f$. Entonces $U\left(t\right)$ tambi\'en tiene densidad, para $t>0$, y es $U^{'}\left(t\right)=\sum_{n=0}^{\infty}f^{n\star}\left(t\right)$. Adem\'as
\begin{eqnarray*}
\prob\left\{N\left(t\right)>N\left(t-\right)\right\}=0\textrm{,   }t\geq0.
\end{eqnarray*}
\end{Prop}

\begin{Def}
La Transformada de Laplace-Stieljes de $F$ est\'a dada por

\begin{eqnarray*}
\hat{F}\left(\alpha\right)=\int_{\rea_{+}}e^{-\alpha t}dF\left(t\right)\textrm{,  }\alpha\geq0.
\end{eqnarray*}
\end{Def}

Entonces

\begin{eqnarray*}
\hat{U}\left(\alpha\right)=\sum_{n=0}^{\infty}\hat{F^{n\star}}\left(\alpha\right)=\sum_{n=0}^{\infty}\hat{F}\left(\alpha\right)^{n}=\frac{1}{1-\hat{F}\left(\alpha\right)}.
\end{eqnarray*}


\begin{Prop}
La Transformada de Laplace $\hat{U}\left(\alpha\right)$ y $\hat{F}\left(\alpha\right)$ determina una a la otra de manera \'unica por la relaci\'on $\hat{U}\left(\alpha\right)=\frac{1}{1-\hat{F}\left(\alpha\right)}$.
\end{Prop}


\begin{Note}
Un proceso de renovaci\'on $N\left(t\right)$ cuyos tiempos de inter-renovaci\'on tienen media finita, es un proceso Poisson con tasa $\lambda$ si y s\'olo s\'i $\esp\left[U\left(t\right)\right]=\lambda t$, para $t\geq0$.
\end{Note}


\begin{Teo}
Sea $N\left(t\right)$ un proceso puntual simple con puntos de localizaci\'on $T_{n}$ tal que $\eta\left(t\right)=\esp\left[N\left(\right)\right]$ es finita para cada $t$. Entonces para cualquier funci\'on $f:\rea_{+}\rightarrow\rea$,
\begin{eqnarray*}
\esp\left[\sum_{n=1}^{N\left(\right)}f\left(T_{n}\right)\right]=\int_{\left(0,t\right]}f\left(s\right)d\eta\left(s\right)\textrm{,  }t\geq0,
\end{eqnarray*}
suponiendo que la integral exista. Adem\'as si $X_{1},X_{2},\ldots$ son variables aleatorias definidas en el mismo espacio de probabilidad que el proceso $N\left(t\right)$ tal que $\esp\left[X_{n}|T_{n}=s\right]=f\left(s\right)$, independiente de $n$. Entonces
\begin{eqnarray*}
\esp\left[\sum_{n=1}^{N\left(t\right)}X_{n}\right]=\int_{\left(0,t\right]}f\left(s\right)d\eta\left(s\right)\textrm{,  }t\geq0,
\end{eqnarray*} 
suponiendo que la integral exista. 
\end{Teo}

\begin{Coro}[Identidad de Wald para Renovaciones]
Para el proceso de renovaci\'on $N\left(t\right)$,
\begin{eqnarray*}
\esp\left[T_{N\left(t\right)+1}\right]=\mu\esp\left[N\left(t\right)+1\right]\textrm{,  }t\geq0,
\end{eqnarray*}  
\end{Coro}


\begin{Def}
Sea $h\left(t\right)$ funci\'on de valores reales en $\rea$ acotada en intervalos finitos e igual a cero para $t<0$ La ecuaci\'on de renovaci\'on para $h\left(t\right)$ y la distribuci\'on $F$ es

\begin{eqnarray}\label{Ec.Renovacion}
H\left(t\right)=h\left(t\right)+\int_{\left[0,t\right]}H\left(t-s\right)dF\left(s\right)\textrm{,    }t\geq0,
\end{eqnarray}
donde $H\left(t\right)$ es una funci\'on de valores reales. Esto es $H=h+F\star H$. Decimos que $H\left(t\right)$ es soluci\'on de esta ecuaci\'on si satisface la ecuaci\'on, y es acotada en intervalos finitos e iguales a cero para $t<0$.
\end{Def}

\begin{Prop}
La funci\'on $U\star h\left(t\right)$ es la \'unica soluci\'on de la ecuaci\'on de renovaci\'on (\ref{Ec.Renovacion}).
\end{Prop}

\begin{Teo}[Teorema Renovaci\'on Elemental]
\begin{eqnarray*}
t^{-1}U\left(t\right)\rightarrow 1/\mu\textrm{,    cuando }t\rightarrow\infty.
\end{eqnarray*}
\end{Teo}



Sup\'ongase que $N\left(t\right)$ es un proceso de renovaci\'on con distribuci\'on $F$ con media finita $\mu$.

\begin{Def}
La funci\'on de renovaci\'on asociada con la distribuci\'on $F$, del proceso $N\left(t\right)$, es
\begin{eqnarray*}
U\left(t\right)=\sum_{n=1}^{\infty}F^{n\star}\left(t\right),\textrm{   }t\geq0,
\end{eqnarray*}
donde $F^{0\star}\left(t\right)=\indora\left(t\geq0\right)$.
\end{Def}


\begin{Prop}
Sup\'ongase que la distribuci\'on de inter-renovaci\'on $F$ tiene densidad $f$. Entonces $U\left(t\right)$ tambi\'en tiene densidad, para $t>0$, y es $U^{'}\left(t\right)=\sum_{n=0}^{\infty}f^{n\star}\left(t\right)$. Adem\'as
\begin{eqnarray*}
\prob\left\{N\left(t\right)>N\left(t-\right)\right\}=0\textrm{,   }t\geq0.
\end{eqnarray*}
\end{Prop}

\begin{Def}
La Transformada de Laplace-Stieljes de $F$ est\'a dada por

\begin{eqnarray*}
\hat{F}\left(\alpha\right)=\int_{\rea_{+}}e^{-\alpha t}dF\left(t\right)\textrm{,  }\alpha\geq0.
\end{eqnarray*}
\end{Def}

Entonces

\begin{eqnarray*}
\hat{U}\left(\alpha\right)=\sum_{n=0}^{\infty}\hat{F^{n\star}}\left(\alpha\right)=\sum_{n=0}^{\infty}\hat{F}\left(\alpha\right)^{n}=\frac{1}{1-\hat{F}\left(\alpha\right)}.
\end{eqnarray*}


\begin{Prop}
La Transformada de Laplace $\hat{U}\left(\alpha\right)$ y $\hat{F}\left(\alpha\right)$ determina una a la otra de manera \'unica por la relaci\'on $\hat{U}\left(\alpha\right)=\frac{1}{1-\hat{F}\left(\alpha\right)}$.
\end{Prop}


\begin{Note}
Un proceso de renovaci\'on $N\left(t\right)$ cuyos tiempos de inter-renovaci\'on tienen media finita, es un proceso Poisson con tasa $\lambda$ si y s\'olo s\'i $\esp\left[U\left(t\right)\right]=\lambda t$, para $t\geq0$.
\end{Note}


\begin{Teo}
Sea $N\left(t\right)$ un proceso puntual simple con puntos de localizaci\'on $T_{n}$ tal que $\eta\left(t\right)=\esp\left[N\left(\right)\right]$ es finita para cada $t$. Entonces para cualquier funci\'on $f:\rea_{+}\rightarrow\rea$,
\begin{eqnarray*}
\esp\left[\sum_{n=1}^{N\left(\right)}f\left(T_{n}\right)\right]=\int_{\left(0,t\right]}f\left(s\right)d\eta\left(s\right)\textrm{,  }t\geq0,
\end{eqnarray*}
suponiendo que la integral exista. Adem\'as si $X_{1},X_{2},\ldots$ son variables aleatorias definidas en el mismo espacio de probabilidad que el proceso $N\left(t\right)$ tal que $\esp\left[X_{n}|T_{n}=s\right]=f\left(s\right)$, independiente de $n$. Entonces
\begin{eqnarray*}
\esp\left[\sum_{n=1}^{N\left(t\right)}X_{n}\right]=\int_{\left(0,t\right]}f\left(s\right)d\eta\left(s\right)\textrm{,  }t\geq0,
\end{eqnarray*} 
suponiendo que la integral exista. 
\end{Teo}

\begin{Coro}[Identidad de Wald para Renovaciones]
Para el proceso de renovaci\'on $N\left(t\right)$,
\begin{eqnarray*}
\esp\left[T_{N\left(t\right)+1}\right]=\mu\esp\left[N\left(t\right)+1\right]\textrm{,  }t\geq0,
\end{eqnarray*}  
\end{Coro}


\begin{Def}
Sea $h\left(t\right)$ funci\'on de valores reales en $\rea$ acotada en intervalos finitos e igual a cero para $t<0$ La ecuaci\'on de renovaci\'on para $h\left(t\right)$ y la distribuci\'on $F$ es

\begin{eqnarray}\label{Ec.Renovacion}
H\left(t\right)=h\left(t\right)+\int_{\left[0,t\right]}H\left(t-s\right)dF\left(s\right)\textrm{,    }t\geq0,
\end{eqnarray}
donde $H\left(t\right)$ es una funci\'on de valores reales. Esto es $H=h+F\star H$. Decimos que $H\left(t\right)$ es soluci\'on de esta ecuaci\'on si satisface la ecuaci\'on, y es acotada en intervalos finitos e iguales a cero para $t<0$.
\end{Def}

\begin{Prop}
La funci\'on $U\star h\left(t\right)$ es la \'unica soluci\'on de la ecuaci\'on de renovaci\'on (\ref{Ec.Renovacion}).
\end{Prop}

\begin{Teo}[Teorema Renovaci\'on Elemental]
\begin{eqnarray*}
t^{-1}U\left(t\right)\rightarrow 1/\mu\textrm{,    cuando }t\rightarrow\infty.
\end{eqnarray*}
\end{Teo}


\begin{Note} Una funci\'on $h:\rea_{+}\rightarrow\rea$ es Directamente Riemann Integrable en los siguientes casos:
\begin{itemize}
\item[a)] $h\left(t\right)\geq0$ es decreciente y Riemann Integrable.
\item[b)] $h$ es continua excepto posiblemente en un conjunto de Lebesgue de medida 0, y $|h\left(t\right)|\leq b\left(t\right)$, donde $b$ es DRI.
\end{itemize}
\end{Note}

\begin{Teo}[Teorema Principal de Renovaci\'on]
Si $F$ es no aritm\'etica y $h\left(t\right)$ es Directamente Riemann Integrable (DRI), entonces

\begin{eqnarray*}
lim_{t\rightarrow\infty}U\star h=\frac{1}{\mu}\int_{\rea_{+}}h\left(s\right)ds.
\end{eqnarray*}
\end{Teo}

\begin{Prop}
Cualquier funci\'on $H\left(t\right)$ acotada en intervalos finitos y que es 0 para $t<0$ puede expresarse como
\begin{eqnarray*}
H\left(t\right)=U\star h\left(t\right)\textrm{,  donde }h\left(t\right)=H\left(t\right)-F\star H\left(t\right)
\end{eqnarray*}
\end{Prop}

\begin{Def}
Un proceso estoc\'astico $X\left(t\right)$ es crudamente regenerativo en un tiempo aleatorio positivo $T$ si
\begin{eqnarray*}
\esp\left[X\left(T+t\right)|T\right]=\esp\left[X\left(t\right)\right]\textrm{, para }t\geq0,\end{eqnarray*}
y con las esperanzas anteriores finitas.
\end{Def}

\begin{Prop}
Sup\'ongase que $X\left(t\right)$ es un proceso crudamente regenerativo en $T$, que tiene distribuci\'on $F$. Si $\esp\left[X\left(t\right)\right]$ es acotado en intervalos finitos, entonces
\begin{eqnarray*}
\esp\left[X\left(t\right)\right]=U\star h\left(t\right)\textrm{,  donde }h\left(t\right)=\esp\left[X\left(t\right)\indora\left(T>t\right)\right].
\end{eqnarray*}
\end{Prop}

\begin{Teo}[Regeneraci\'on Cruda]
Sup\'ongase que $X\left(t\right)$ es un proceso con valores positivo crudamente regenerativo en $T$, y def\'inase $M=\sup\left\{|X\left(t\right)|:t\leq T\right\}$. Si $T$ es no aritm\'etico y $M$ y $MT$ tienen media finita, entonces
\begin{eqnarray*}
lim_{t\rightarrow\infty}\esp\left[X\left(t\right)\right]=\frac{1}{\mu}\int_{\rea_{+}}h\left(s\right)ds,
\end{eqnarray*}
donde $h\left(t\right)=\esp\left[X\left(t\right)\indora\left(T>t\right)\right]$.
\end{Teo}


\begin{Note} Una funci\'on $h:\rea_{+}\rightarrow\rea$ es Directamente Riemann Integrable en los siguientes casos:
\begin{itemize}
\item[a)] $h\left(t\right)\geq0$ es decreciente y Riemann Integrable.
\item[b)] $h$ es continua excepto posiblemente en un conjunto de Lebesgue de medida 0, y $|h\left(t\right)|\leq b\left(t\right)$, donde $b$ es DRI.
\end{itemize}
\end{Note}

\begin{Teo}[Teorema Principal de Renovaci\'on]
Si $F$ es no aritm\'etica y $h\left(t\right)$ es Directamente Riemann Integrable (DRI), entonces

\begin{eqnarray*}
lim_{t\rightarrow\infty}U\star h=\frac{1}{\mu}\int_{\rea_{+}}h\left(s\right)ds.
\end{eqnarray*}
\end{Teo}

\begin{Prop}
Cualquier funci\'on $H\left(t\right)$ acotada en intervalos finitos y que es 0 para $t<0$ puede expresarse como
\begin{eqnarray*}
H\left(t\right)=U\star h\left(t\right)\textrm{,  donde }h\left(t\right)=H\left(t\right)-F\star H\left(t\right)
\end{eqnarray*}
\end{Prop}

\begin{Def}
Un proceso estoc\'astico $X\left(t\right)$ es crudamente regenerativo en un tiempo aleatorio positivo $T$ si
\begin{eqnarray*}
\esp\left[X\left(T+t\right)|T\right]=\esp\left[X\left(t\right)\right]\textrm{, para }t\geq0,\end{eqnarray*}
y con las esperanzas anteriores finitas.
\end{Def}

\begin{Prop}
Sup\'ongase que $X\left(t\right)$ es un proceso crudamente regenerativo en $T$, que tiene distribuci\'on $F$. Si $\esp\left[X\left(t\right)\right]$ es acotado en intervalos finitos, entonces
\begin{eqnarray*}
\esp\left[X\left(t\right)\right]=U\star h\left(t\right)\textrm{,  donde }h\left(t\right)=\esp\left[X\left(t\right)\indora\left(T>t\right)\right].
\end{eqnarray*}
\end{Prop}

\begin{Teo}[Regeneraci\'on Cruda]
Sup\'ongase que $X\left(t\right)$ es un proceso con valores positivo crudamente regenerativo en $T$, y def\'inase $M=\sup\left\{|X\left(t\right)|:t\leq T\right\}$. Si $T$ es no aritm\'etico y $M$ y $MT$ tienen media finita, entonces
\begin{eqnarray*}
lim_{t\rightarrow\infty}\esp\left[X\left(t\right)\right]=\frac{1}{\mu}\int_{\rea_{+}}h\left(s\right)ds,
\end{eqnarray*}
donde $h\left(t\right)=\esp\left[X\left(t\right)\indora\left(T>t\right)\right]$.
\end{Teo}

%________________________________________________________________________
\subsection{Procesos Regenerativos}
%________________________________________________________________________

Para $\left\{X\left(t\right):t\geq0\right\}$ Proceso Estoc\'astico a tiempo continuo con estado de espacios $S$, que es un espacio m\'etrico, con trayectorias continuas por la derecha y con l\'imites por la izquierda c.s. Sea $N\left(t\right)$ un proceso de renovaci\'on en $\rea_{+}$ definido en el mismo espacio de probabilidad que $X\left(t\right)$, con tiempos de renovaci\'on $T$ y tiempos de inter-renovaci\'on $\xi_{n}=T_{n}-T_{n-1}$, con misma distribuci\'on $F$ de media finita $\mu$.



\begin{Def}
Para el proceso $\left\{\left(N\left(t\right),X\left(t\right)\right):t\geq0\right\}$, sus trayectoria muestrales en el intervalo de tiempo $\left[T_{n-1},T_{n}\right)$ est\'an descritas por
\begin{eqnarray*}
\zeta_{n}=\left(\xi_{n},\left\{X\left(T_{n-1}+t\right):0\leq t<\xi_{n}\right\}\right)
\end{eqnarray*}
Este $\zeta_{n}$ es el $n$-\'esimo segmento del proceso. El proceso es regenerativo sobre los tiempos $T_{n}$ si sus segmentos $\zeta_{n}$ son independientes e id\'enticamennte distribuidos.
\end{Def}


\begin{Obs}
Si $\tilde{X}\left(t\right)$ con espacio de estados $\tilde{S}$ es regenerativo sobre $T_{n}$, entonces $X\left(t\right)=f\left(\tilde{X}\left(t\right)\right)$ tambi\'en es regenerativo sobre $T_{n}$, para cualquier funci\'on $f:\tilde{S}\rightarrow S$.
\end{Obs}

\begin{Obs}
Los procesos regenerativos son crudamente regenerativos, pero no al rev\'es.
\end{Obs}

\begin{Def}[Definici\'on Cl\'asica]
Un proceso estoc\'astico $X=\left\{X\left(t\right):t\geq0\right\}$ es llamado regenerativo is existe una variable aleatoria $R_{1}>0$ tal que
\begin{itemize}
\item[i)] $\left\{X\left(t+R_{1}\right):t\geq0\right\}$ es independiente de $\left\{\left\{X\left(t\right):t<R_{1}\right\},\right\}$
\item[ii)] $\left\{X\left(t+R_{1}\right):t\geq0\right\}$ es estoc\'asticamente equivalente a $\left\{X\left(t\right):t>0\right\}$
\end{itemize}

Llamamos a $R_{1}$ tiempo de regeneraci\'on, y decimos que $X$ se regenera en este punto.
\end{Def}

$\left\{X\left(t+R_{1}\right)\right\}$ es regenerativo con tiempo de regeneraci\'on $R_{2}$, independiente de $R_{1}$ pero con la misma distribuci\'on que $R_{1}$. Procediendo de esta manera se obtiene una secuencia de variables aleatorias independientes e id\'enticamente distribuidas $\left\{R_{n}\right\}$ llamados longitudes de ciclo. Si definimos a $Z_{k}\equiv R_{1}+R_{2}+\cdots+R_{k}$, se tiene un proceso de renovaci\'on llamado proceso de renovaci\'on encajado para $X$.

\begin{Note}
Un proceso regenerativo con media de la longitud de ciclo finita es llamado positivo recurrente.
\end{Note}


\begin{Def}
Para $x$ fijo y para cada $t\geq0$, sea $I_{x}\left(t\right)=1$ si $X\left(t\right)\leq x$,  $I_{x}\left(t\right)=0$ en caso contrario, y def\'inanse los tiempos promedio
\begin{eqnarray*}
\overline{X}&=&lim_{t\rightarrow\infty}\frac{1}{t}\int_{0}^{\infty}X\left(u\right)du\\
\prob\left(X_{\infty}\leq x\right)&=&lim_{t\rightarrow\infty}\frac{1}{t}\int_{0}^{\infty}I_{x}\left(u\right)du,
\end{eqnarray*}
cuando estos l\'imites existan.
\end{Def}

Como consecuencia del teorema de Renovaci\'on-Recompensa, se tiene que el primer l\'imite  existe y es igual a la constante
\begin{eqnarray*}
\overline{X}&=&\frac{\esp\left[\int_{0}^{R_{1}}X\left(t\right)dt\right]}{\esp\left[R_{1}\right]},
\end{eqnarray*}
suponiendo que ambas esperanzas son finitas.

\begin{Note}
\begin{itemize}
\item[a)] Si el proceso regenerativo $X$ es positivo recurrente y tiene trayectorias muestrales no negativas, entonces la ecuaci\'on anterior es v\'alida.
\item[b)] Si $X$ es positivo recurrente regenerativo, podemos construir una \'unica versi\'on estacionaria de este proceso, $X_{e}=\left\{X_{e}\left(t\right)\right\}$, donde $X_{e}$ es un proceso estoc\'astico regenerativo y estrictamente estacionario, con distribuci\'on marginal distribuida como $X_{\infty}$
\end{itemize}
\end{Note}

%________________________________________________________________________
\subsection{Procesos Regenerativos}
%________________________________________________________________________

Para $\left\{X\left(t\right):t\geq0\right\}$ Proceso Estoc\'astico a tiempo continuo con estado de espacios $S$, que es un espacio m\'etrico, con trayectorias continuas por la derecha y con l\'imites por la izquierda c.s. Sea $N\left(t\right)$ un proceso de renovaci\'on en $\rea_{+}$ definido en el mismo espacio de probabilidad que $X\left(t\right)$, con tiempos de renovaci\'on $T$ y tiempos de inter-renovaci\'on $\xi_{n}=T_{n}-T_{n-1}$, con misma distribuci\'on $F$ de media finita $\mu$.



\begin{Def}
Para el proceso $\left\{\left(N\left(t\right),X\left(t\right)\right):t\geq0\right\}$, sus trayectoria muestrales en el intervalo de tiempo $\left[T_{n-1},T_{n}\right)$ est\'an descritas por
\begin{eqnarray*}
\zeta_{n}=\left(\xi_{n},\left\{X\left(T_{n-1}+t\right):0\leq t<\xi_{n}\right\}\right)
\end{eqnarray*}
Este $\zeta_{n}$ es el $n$-\'esimo segmento del proceso. El proceso es regenerativo sobre los tiempos $T_{n}$ si sus segmentos $\zeta_{n}$ son independientes e id\'enticamennte distribuidos.
\end{Def}


\begin{Obs}
Si $\tilde{X}\left(t\right)$ con espacio de estados $\tilde{S}$ es regenerativo sobre $T_{n}$, entonces $X\left(t\right)=f\left(\tilde{X}\left(t\right)\right)$ tambi\'en es regenerativo sobre $T_{n}$, para cualquier funci\'on $f:\tilde{S}\rightarrow S$.
\end{Obs}

\begin{Obs}
Los procesos regenerativos son crudamente regenerativos, pero no al rev\'es.
\end{Obs}

\begin{Def}[Definici\'on Cl\'asica]
Un proceso estoc\'astico $X=\left\{X\left(t\right):t\geq0\right\}$ es llamado regenerativo is existe una variable aleatoria $R_{1}>0$ tal que
\begin{itemize}
\item[i)] $\left\{X\left(t+R_{1}\right):t\geq0\right\}$ es independiente de $\left\{\left\{X\left(t\right):t<R_{1}\right\},\right\}$
\item[ii)] $\left\{X\left(t+R_{1}\right):t\geq0\right\}$ es estoc\'asticamente equivalente a $\left\{X\left(t\right):t>0\right\}$
\end{itemize}

Llamamos a $R_{1}$ tiempo de regeneraci\'on, y decimos que $X$ se regenera en este punto.
\end{Def}

$\left\{X\left(t+R_{1}\right)\right\}$ es regenerativo con tiempo de regeneraci\'on $R_{2}$, independiente de $R_{1}$ pero con la misma distribuci\'on que $R_{1}$. Procediendo de esta manera se obtiene una secuencia de variables aleatorias independientes e id\'enticamente distribuidas $\left\{R_{n}\right\}$ llamados longitudes de ciclo. Si definimos a $Z_{k}\equiv R_{1}+R_{2}+\cdots+R_{k}$, se tiene un proceso de renovaci\'on llamado proceso de renovaci\'on encajado para $X$.

\begin{Note}
Un proceso regenerativo con media de la longitud de ciclo finita es llamado positivo recurrente.
\end{Note}


\begin{Def}
Para $x$ fijo y para cada $t\geq0$, sea $I_{x}\left(t\right)=1$ si $X\left(t\right)\leq x$,  $I_{x}\left(t\right)=0$ en caso contrario, y def\'inanse los tiempos promedio
\begin{eqnarray*}
\overline{X}&=&lim_{t\rightarrow\infty}\frac{1}{t}\int_{0}^{\infty}X\left(u\right)du\\
\prob\left(X_{\infty}\leq x\right)&=&lim_{t\rightarrow\infty}\frac{1}{t}\int_{0}^{\infty}I_{x}\left(u\right)du,
\end{eqnarray*}
cuando estos l\'imites existan.
\end{Def}

Como consecuencia del teorema de Renovaci\'on-Recompensa, se tiene que el primer l\'imite  existe y es igual a la constante
\begin{eqnarray*}
\overline{X}&=&\frac{\esp\left[\int_{0}^{R_{1}}X\left(t\right)dt\right]}{\esp\left[R_{1}\right]},
\end{eqnarray*}
suponiendo que ambas esperanzas son finitas.

\begin{Note}
\begin{itemize}
\item[a)] Si el proceso regenerativo $X$ es positivo recurrente y tiene trayectorias muestrales no negativas, entonces la ecuaci\'on anterior es v\'alida.
\item[b)] Si $X$ es positivo recurrente regenerativo, podemos construir una \'unica versi\'on estacionaria de este proceso, $X_{e}=\left\{X_{e}\left(t\right)\right\}$, donde $X_{e}$ es un proceso estoc\'astico regenerativo y estrictamente estacionario, con distribuci\'on marginal distribuida como $X_{\infty}$
\end{itemize}
\end{Note}
%__________________________________________________________________________________________
\subsection{Procesos Regenerativos Estacionarios - Stidham \cite{Stidham}}
%__________________________________________________________________________________________


Un proceso estoc\'astico a tiempo continuo $\left\{V\left(t\right),t\geq0\right\}$ es un proceso regenerativo si existe una sucesi\'on de variables aleatorias independientes e id\'enticamente distribuidas $\left\{X_{1},X_{2},\ldots\right\}$, sucesi\'on de renovaci\'on, tal que para cualquier conjunto de Borel $A$, 

\begin{eqnarray*}
\prob\left\{V\left(t\right)\in A|X_{1}+X_{2}+\cdots+X_{R\left(t\right)}=s,\left\{V\left(\tau\right),\tau<s\right\}\right\}=\prob\left\{V\left(t-s\right)\in A|X_{1}>t-s\right\},
\end{eqnarray*}
para todo $0\leq s\leq t$, donde $R\left(t\right)=\max\left\{X_{1}+X_{2}+\cdots+X_{j}\leq t\right\}=$n\'umero de renovaciones ({\emph{puntos de regeneraci\'on}}) que ocurren en $\left[0,t\right]$. El intervalo $\left[0,X_{1}\right)$ es llamado {\emph{primer ciclo de regeneraci\'on}} de $\left\{V\left(t \right),t\geq0\right\}$, $\left[X_{1},X_{1}+X_{2}\right)$ el {\emph{segundo ciclo de regeneraci\'on}}, y as\'i sucesivamente.

Sea $X=X_{1}$ y sea $F$ la funci\'on de distrbuci\'on de $X$


\begin{Def}
Se define el proceso estacionario, $\left\{V^{*}\left(t\right),t\geq0\right\}$, para $\left\{V\left(t\right),t\geq0\right\}$ por

\begin{eqnarray*}
\prob\left\{V\left(t\right)\in A\right\}=\frac{1}{\esp\left[X\right]}\int_{0}^{\infty}\prob\left\{V\left(t+x\right)\in A|X>x\right\}\left(1-F\left(x\right)\right)dx,
\end{eqnarray*} 
para todo $t\geq0$ y todo conjunto de Borel $A$.
\end{Def}

\begin{Def}
Una distribuci\'on se dice que es {\emph{aritm\'etica}} si todos sus puntos de incremento son m\'ultiplos de la forma $0,\lambda, 2\lambda,\ldots$ para alguna $\lambda>0$ entera.
\end{Def}


\begin{Def}
Una modificaci\'on medible de un proceso $\left\{V\left(t\right),t\geq0\right\}$, es una versi\'on de este, $\left\{V\left(t,w\right)\right\}$ conjuntamente medible para $t\geq0$ y para $w\in S$, $S$ espacio de estados para $\left\{V\left(t\right),t\geq0\right\}$.
\end{Def}

\begin{Teo}
Sea $\left\{V\left(t\right),t\geq\right\}$ un proceso regenerativo no negativo con modificaci\'on medible. Sea $\esp\left[X\right]<\infty$. Entonces el proceso estacionario dado por la ecuaci\'on anterior est\'a bien definido y tiene funci\'on de distribuci\'on independiente de $t$, adem\'as
\begin{itemize}
\item[i)] \begin{eqnarray*}
\esp\left[V^{*}\left(0\right)\right]&=&\frac{\esp\left[\int_{0}^{X}V\left(s\right)ds\right]}{\esp\left[X\right]}\end{eqnarray*}
\item[ii)] Si $\esp\left[V^{*}\left(0\right)\right]<\infty$, equivalentemente, si $\esp\left[\int_{0}^{X}V\left(s\right)ds\right]<\infty$,entonces
\begin{eqnarray*}
\frac{\int_{0}^{t}V\left(s\right)ds}{t}\rightarrow\frac{\esp\left[\int_{0}^{X}V\left(s\right)ds\right]}{\esp\left[X\right]}
\end{eqnarray*}
con probabilidad 1 y en media, cuando $t\rightarrow\infty$.
\end{itemize}
\end{Teo}


%__________________________________________________________________________________________
\subsection{Procesos Regenerativos Estacionarios - Stidham \cite{Stidham}}
%__________________________________________________________________________________________


Un proceso estoc\'astico a tiempo continuo $\left\{V\left(t\right),t\geq0\right\}$ es un proceso regenerativo si existe una sucesi\'on de variables aleatorias independientes e id\'enticamente distribuidas $\left\{X_{1},X_{2},\ldots\right\}$, sucesi\'on de renovaci\'on, tal que para cualquier conjunto de Borel $A$, 

\begin{eqnarray*}
\prob\left\{V\left(t\right)\in A|X_{1}+X_{2}+\cdots+X_{R\left(t\right)}=s,\left\{V\left(\tau\right),\tau<s\right\}\right\}=\prob\left\{V\left(t-s\right)\in A|X_{1}>t-s\right\},
\end{eqnarray*}
para todo $0\leq s\leq t$, donde $R\left(t\right)=\max\left\{X_{1}+X_{2}+\cdots+X_{j}\leq t\right\}=$n\'umero de renovaciones ({\emph{puntos de regeneraci\'on}}) que ocurren en $\left[0,t\right]$. El intervalo $\left[0,X_{1}\right)$ es llamado {\emph{primer ciclo de regeneraci\'on}} de $\left\{V\left(t \right),t\geq0\right\}$, $\left[X_{1},X_{1}+X_{2}\right)$ el {\emph{segundo ciclo de regeneraci\'on}}, y as\'i sucesivamente.

Sea $X=X_{1}$ y sea $F$ la funci\'on de distrbuci\'on de $X$


\begin{Def}
Se define el proceso estacionario, $\left\{V^{*}\left(t\right),t\geq0\right\}$, para $\left\{V\left(t\right),t\geq0\right\}$ por

\begin{eqnarray*}
\prob\left\{V\left(t\right)\in A\right\}=\frac{1}{\esp\left[X\right]}\int_{0}^{\infty}\prob\left\{V\left(t+x\right)\in A|X>x\right\}\left(1-F\left(x\right)\right)dx,
\end{eqnarray*} 
para todo $t\geq0$ y todo conjunto de Borel $A$.
\end{Def}

\begin{Def}
Una distribuci\'on se dice que es {\emph{aritm\'etica}} si todos sus puntos de incremento son m\'ultiplos de la forma $0,\lambda, 2\lambda,\ldots$ para alguna $\lambda>0$ entera.
\end{Def}


\begin{Def}
Una modificaci\'on medible de un proceso $\left\{V\left(t\right),t\geq0\right\}$, es una versi\'on de este, $\left\{V\left(t,w\right)\right\}$ conjuntamente medible para $t\geq0$ y para $w\in S$, $S$ espacio de estados para $\left\{V\left(t\right),t\geq0\right\}$.
\end{Def}

\begin{Teo}
Sea $\left\{V\left(t\right),t\geq\right\}$ un proceso regenerativo no negativo con modificaci\'on medible. Sea $\esp\left[X\right]<\infty$. Entonces el proceso estacionario dado por la ecuaci\'on anterior est\'a bien definido y tiene funci\'on de distribuci\'on independiente de $t$, adem\'as
\begin{itemize}
\item[i)] \begin{eqnarray*}
\esp\left[V^{*}\left(0\right)\right]&=&\frac{\esp\left[\int_{0}^{X}V\left(s\right)ds\right]}{\esp\left[X\right]}\end{eqnarray*}
\item[ii)] Si $\esp\left[V^{*}\left(0\right)\right]<\infty$, equivalentemente, si $\esp\left[\int_{0}^{X}V\left(s\right)ds\right]<\infty$,entonces
\begin{eqnarray*}
\frac{\int_{0}^{t}V\left(s\right)ds}{t}\rightarrow\frac{\esp\left[\int_{0}^{X}V\left(s\right)ds\right]}{\esp\left[X\right]}
\end{eqnarray*}
con probabilidad 1 y en media, cuando $t\rightarrow\infty$.
\end{itemize}
\end{Teo}
%
%___________________________________________________________________________________________
%\vspace{5.5cm}
%\chapter{Cadenas de Markov estacionarias}
%\vspace{-1.0cm}
%___________________________________________________________________________________________
%
\subsection{Propiedades de los Procesos de Renovaci\'on}
%___________________________________________________________________________________________
%

Los tiempos $T_{n}$ est\'an relacionados con los conteos de $N\left(t\right)$ por

\begin{eqnarray*}
\left\{N\left(t\right)\geq n\right\}&=&\left\{T_{n}\leq t\right\}\\
T_{N\left(t\right)}\leq &t&<T_{N\left(t\right)+1},
\end{eqnarray*}

adem\'as $N\left(T_{n}\right)=n$, y 

\begin{eqnarray*}
N\left(t\right)=\max\left\{n:T_{n}\leq t\right\}=\min\left\{n:T_{n+1}>t\right\}
\end{eqnarray*}

Por propiedades de la convoluci\'on se sabe que

\begin{eqnarray*}
P\left\{T_{n}\leq t\right\}=F^{n\star}\left(t\right)
\end{eqnarray*}
que es la $n$-\'esima convoluci\'on de $F$. Entonces 

\begin{eqnarray*}
\left\{N\left(t\right)\geq n\right\}&=&\left\{T_{n}\leq t\right\}\\
P\left\{N\left(t\right)\leq n\right\}&=&1-F^{\left(n+1\right)\star}\left(t\right)
\end{eqnarray*}

Adem\'as usando el hecho de que $\esp\left[N\left(t\right)\right]=\sum_{n=1}^{\infty}P\left\{N\left(t\right)\geq n\right\}$
se tiene que

\begin{eqnarray*}
\esp\left[N\left(t\right)\right]=\sum_{n=1}^{\infty}F^{n\star}\left(t\right)
\end{eqnarray*}

\begin{Prop}
Para cada $t\geq0$, la funci\'on generadora de momentos $\esp\left[e^{\alpha N\left(t\right)}\right]$ existe para alguna $\alpha$ en una vecindad del 0, y de aqu\'i que $\esp\left[N\left(t\right)^{m}\right]<\infty$, para $m\geq1$.
\end{Prop}


\begin{Note}
Si el primer tiempo de renovaci\'on $\xi_{1}$ no tiene la misma distribuci\'on que el resto de las $\xi_{n}$, para $n\geq2$, a $N\left(t\right)$ se le llama Proceso de Renovaci\'on retardado, donde si $\xi$ tiene distribuci\'on $G$, entonces el tiempo $T_{n}$ de la $n$-\'esima renovaci\'on tiene distribuci\'on $G\star F^{\left(n-1\right)\star}\left(t\right)$
\end{Note}


\begin{Teo}
Para una constante $\mu\leq\infty$ ( o variable aleatoria), las siguientes expresiones son equivalentes:

\begin{eqnarray}
lim_{n\rightarrow\infty}n^{-1}T_{n}&=&\mu,\textrm{ c.s.}\\
lim_{t\rightarrow\infty}t^{-1}N\left(t\right)&=&1/\mu,\textrm{ c.s.}
\end{eqnarray}
\end{Teo}


Es decir, $T_{n}$ satisface la Ley Fuerte de los Grandes N\'umeros s\'i y s\'olo s\'i $N\left/t\right)$ la cumple.


\begin{Coro}[Ley Fuerte de los Grandes N\'umeros para Procesos de Renovaci\'on]
Si $N\left(t\right)$ es un proceso de renovaci\'on cuyos tiempos de inter-renovaci\'on tienen media $\mu\leq\infty$, entonces
\begin{eqnarray}
t^{-1}N\left(t\right)\rightarrow 1/\mu,\textrm{ c.s. cuando }t\rightarrow\infty.
\end{eqnarray}

\end{Coro}


Considerar el proceso estoc\'astico de valores reales $\left\{Z\left(t\right):t\geq0\right\}$ en el mismo espacio de probabilidad que $N\left(t\right)$

\begin{Def}
Para el proceso $\left\{Z\left(t\right):t\geq0\right\}$ se define la fluctuaci\'on m\'axima de $Z\left(t\right)$ en el intervalo $\left(T_{n-1},T_{n}\right]$:
\begin{eqnarray*}
M_{n}=\sup_{T_{n-1}<t\leq T_{n}}|Z\left(t\right)-Z\left(T_{n-1}\right)|
\end{eqnarray*}
\end{Def}

\begin{Teo}
Sup\'ongase que $n^{-1}T_{n}\rightarrow\mu$ c.s. cuando $n\rightarrow\infty$, donde $\mu\leq\infty$ es una constante o variable aleatoria. Sea $a$ una constante o variable aleatoria que puede ser infinita cuando $\mu$ es finita, y considere las expresiones l\'imite:
\begin{eqnarray}
lim_{n\rightarrow\infty}n^{-1}Z\left(T_{n}\right)&=&a,\textrm{ c.s.}\\
lim_{t\rightarrow\infty}t^{-1}Z\left(t\right)&=&a/\mu,\textrm{ c.s.}
\end{eqnarray}
La segunda expresi\'on implica la primera. Conversamente, la primera implica la segunda si el proceso $Z\left(t\right)$ es creciente, o si $lim_{n\rightarrow\infty}n^{-1}M_{n}=0$ c.s.
\end{Teo}

\begin{Coro}
Si $N\left(t\right)$ es un proceso de renovaci\'on, y $\left(Z\left(T_{n}\right)-Z\left(T_{n-1}\right),M_{n}\right)$, para $n\geq1$, son variables aleatorias independientes e id\'enticamente distribuidas con media finita, entonces,
\begin{eqnarray}
lim_{t\rightarrow\infty}t^{-1}Z\left(t\right)\rightarrow\frac{\esp\left[Z\left(T_{1}\right)-Z\left(T_{0}\right)\right]}{\esp\left[T_{1}\right]},\textrm{ c.s. cuando  }t\rightarrow\infty.
\end{eqnarray}
\end{Coro}


%___________________________________________________________________________________________
%
%\subsection{Propiedades de los Procesos de Renovaci\'on}
%___________________________________________________________________________________________
%

Los tiempos $T_{n}$ est\'an relacionados con los conteos de $N\left(t\right)$ por

\begin{eqnarray*}
\left\{N\left(t\right)\geq n\right\}&=&\left\{T_{n}\leq t\right\}\\
T_{N\left(t\right)}\leq &t&<T_{N\left(t\right)+1},
\end{eqnarray*}

adem\'as $N\left(T_{n}\right)=n$, y 

\begin{eqnarray*}
N\left(t\right)=\max\left\{n:T_{n}\leq t\right\}=\min\left\{n:T_{n+1}>t\right\}
\end{eqnarray*}

Por propiedades de la convoluci\'on se sabe que

\begin{eqnarray*}
P\left\{T_{n}\leq t\right\}=F^{n\star}\left(t\right)
\end{eqnarray*}
que es la $n$-\'esima convoluci\'on de $F$. Entonces 

\begin{eqnarray*}
\left\{N\left(t\right)\geq n\right\}&=&\left\{T_{n}\leq t\right\}\\
P\left\{N\left(t\right)\leq n\right\}&=&1-F^{\left(n+1\right)\star}\left(t\right)
\end{eqnarray*}

Adem\'as usando el hecho de que $\esp\left[N\left(t\right)\right]=\sum_{n=1}^{\infty}P\left\{N\left(t\right)\geq n\right\}$
se tiene que

\begin{eqnarray*}
\esp\left[N\left(t\right)\right]=\sum_{n=1}^{\infty}F^{n\star}\left(t\right)
\end{eqnarray*}

\begin{Prop}
Para cada $t\geq0$, la funci\'on generadora de momentos $\esp\left[e^{\alpha N\left(t\right)}\right]$ existe para alguna $\alpha$ en una vecindad del 0, y de aqu\'i que $\esp\left[N\left(t\right)^{m}\right]<\infty$, para $m\geq1$.
\end{Prop}


\begin{Note}
Si el primer tiempo de renovaci\'on $\xi_{1}$ no tiene la misma distribuci\'on que el resto de las $\xi_{n}$, para $n\geq2$, a $N\left(t\right)$ se le llama Proceso de Renovaci\'on retardado, donde si $\xi$ tiene distribuci\'on $G$, entonces el tiempo $T_{n}$ de la $n$-\'esima renovaci\'on tiene distribuci\'on $G\star F^{\left(n-1\right)\star}\left(t\right)$
\end{Note}


\begin{Teo}
Para una constante $\mu\leq\infty$ ( o variable aleatoria), las siguientes expresiones son equivalentes:

\begin{eqnarray}
lim_{n\rightarrow\infty}n^{-1}T_{n}&=&\mu,\textrm{ c.s.}\\
lim_{t\rightarrow\infty}t^{-1}N\left(t\right)&=&1/\mu,\textrm{ c.s.}
\end{eqnarray}
\end{Teo}


Es decir, $T_{n}$ satisface la Ley Fuerte de los Grandes N\'umeros s\'i y s\'olo s\'i $N\left/t\right)$ la cumple.


\begin{Coro}[Ley Fuerte de los Grandes N\'umeros para Procesos de Renovaci\'on]
Si $N\left(t\right)$ es un proceso de renovaci\'on cuyos tiempos de inter-renovaci\'on tienen media $\mu\leq\infty$, entonces
\begin{eqnarray}
t^{-1}N\left(t\right)\rightarrow 1/\mu,\textrm{ c.s. cuando }t\rightarrow\infty.
\end{eqnarray}

\end{Coro}


Considerar el proceso estoc\'astico de valores reales $\left\{Z\left(t\right):t\geq0\right\}$ en el mismo espacio de probabilidad que $N\left(t\right)$

\begin{Def}
Para el proceso $\left\{Z\left(t\right):t\geq0\right\}$ se define la fluctuaci\'on m\'axima de $Z\left(t\right)$ en el intervalo $\left(T_{n-1},T_{n}\right]$:
\begin{eqnarray*}
M_{n}=\sup_{T_{n-1}<t\leq T_{n}}|Z\left(t\right)-Z\left(T_{n-1}\right)|
\end{eqnarray*}
\end{Def}

\begin{Teo}
Sup\'ongase que $n^{-1}T_{n}\rightarrow\mu$ c.s. cuando $n\rightarrow\infty$, donde $\mu\leq\infty$ es una constante o variable aleatoria. Sea $a$ una constante o variable aleatoria que puede ser infinita cuando $\mu$ es finita, y considere las expresiones l\'imite:
\begin{eqnarray}
lim_{n\rightarrow\infty}n^{-1}Z\left(T_{n}\right)&=&a,\textrm{ c.s.}\\
lim_{t\rightarrow\infty}t^{-1}Z\left(t\right)&=&a/\mu,\textrm{ c.s.}
\end{eqnarray}
La segunda expresi\'on implica la primera. Conversamente, la primera implica la segunda si el proceso $Z\left(t\right)$ es creciente, o si $lim_{n\rightarrow\infty}n^{-1}M_{n}=0$ c.s.
\end{Teo}

\begin{Coro}
Si $N\left(t\right)$ es un proceso de renovaci\'on, y $\left(Z\left(T_{n}\right)-Z\left(T_{n-1}\right),M_{n}\right)$, para $n\geq1$, son variables aleatorias independientes e id\'enticamente distribuidas con media finita, entonces,
\begin{eqnarray}
lim_{t\rightarrow\infty}t^{-1}Z\left(t\right)\rightarrow\frac{\esp\left[Z\left(T_{1}\right)-Z\left(T_{0}\right)\right]}{\esp\left[T_{1}\right]},\textrm{ c.s. cuando  }t\rightarrow\infty.
\end{eqnarray}
\end{Coro}

%___________________________________________________________________________________________
%
%\subsection{Propiedades de los Procesos de Renovaci\'on}
%___________________________________________________________________________________________
%

Los tiempos $T_{n}$ est\'an relacionados con los conteos de $N\left(t\right)$ por

\begin{eqnarray*}
\left\{N\left(t\right)\geq n\right\}&=&\left\{T_{n}\leq t\right\}\\
T_{N\left(t\right)}\leq &t&<T_{N\left(t\right)+1},
\end{eqnarray*}

adem\'as $N\left(T_{n}\right)=n$, y 

\begin{eqnarray*}
N\left(t\right)=\max\left\{n:T_{n}\leq t\right\}=\min\left\{n:T_{n+1}>t\right\}
\end{eqnarray*}

Por propiedades de la convoluci\'on se sabe que

\begin{eqnarray*}
P\left\{T_{n}\leq t\right\}=F^{n\star}\left(t\right)
\end{eqnarray*}
que es la $n$-\'esima convoluci\'on de $F$. Entonces 

\begin{eqnarray*}
\left\{N\left(t\right)\geq n\right\}&=&\left\{T_{n}\leq t\right\}\\
P\left\{N\left(t\right)\leq n\right\}&=&1-F^{\left(n+1\right)\star}\left(t\right)
\end{eqnarray*}

Adem\'as usando el hecho de que $\esp\left[N\left(t\right)\right]=\sum_{n=1}^{\infty}P\left\{N\left(t\right)\geq n\right\}$
se tiene que

\begin{eqnarray*}
\esp\left[N\left(t\right)\right]=\sum_{n=1}^{\infty}F^{n\star}\left(t\right)
\end{eqnarray*}

\begin{Prop}
Para cada $t\geq0$, la funci\'on generadora de momentos $\esp\left[e^{\alpha N\left(t\right)}\right]$ existe para alguna $\alpha$ en una vecindad del 0, y de aqu\'i que $\esp\left[N\left(t\right)^{m}\right]<\infty$, para $m\geq1$.
\end{Prop}


\begin{Note}
Si el primer tiempo de renovaci\'on $\xi_{1}$ no tiene la misma distribuci\'on que el resto de las $\xi_{n}$, para $n\geq2$, a $N\left(t\right)$ se le llama Proceso de Renovaci\'on retardado, donde si $\xi$ tiene distribuci\'on $G$, entonces el tiempo $T_{n}$ de la $n$-\'esima renovaci\'on tiene distribuci\'on $G\star F^{\left(n-1\right)\star}\left(t\right)$
\end{Note}


\begin{Teo}
Para una constante $\mu\leq\infty$ ( o variable aleatoria), las siguientes expresiones son equivalentes:

\begin{eqnarray}
lim_{n\rightarrow\infty}n^{-1}T_{n}&=&\mu,\textrm{ c.s.}\\
lim_{t\rightarrow\infty}t^{-1}N\left(t\right)&=&1/\mu,\textrm{ c.s.}
\end{eqnarray}
\end{Teo}


Es decir, $T_{n}$ satisface la Ley Fuerte de los Grandes N\'umeros s\'i y s\'olo s\'i $N\left/t\right)$ la cumple.


\begin{Coro}[Ley Fuerte de los Grandes N\'umeros para Procesos de Renovaci\'on]
Si $N\left(t\right)$ es un proceso de renovaci\'on cuyos tiempos de inter-renovaci\'on tienen media $\mu\leq\infty$, entonces
\begin{eqnarray}
t^{-1}N\left(t\right)\rightarrow 1/\mu,\textrm{ c.s. cuando }t\rightarrow\infty.
\end{eqnarray}

\end{Coro}


Considerar el proceso estoc\'astico de valores reales $\left\{Z\left(t\right):t\geq0\right\}$ en el mismo espacio de probabilidad que $N\left(t\right)$

\begin{Def}
Para el proceso $\left\{Z\left(t\right):t\geq0\right\}$ se define la fluctuaci\'on m\'axima de $Z\left(t\right)$ en el intervalo $\left(T_{n-1},T_{n}\right]$:
\begin{eqnarray*}
M_{n}=\sup_{T_{n-1}<t\leq T_{n}}|Z\left(t\right)-Z\left(T_{n-1}\right)|
\end{eqnarray*}
\end{Def}

\begin{Teo}
Sup\'ongase que $n^{-1}T_{n}\rightarrow\mu$ c.s. cuando $n\rightarrow\infty$, donde $\mu\leq\infty$ es una constante o variable aleatoria. Sea $a$ una constante o variable aleatoria que puede ser infinita cuando $\mu$ es finita, y considere las expresiones l\'imite:
\begin{eqnarray}
lim_{n\rightarrow\infty}n^{-1}Z\left(T_{n}\right)&=&a,\textrm{ c.s.}\\
lim_{t\rightarrow\infty}t^{-1}Z\left(t\right)&=&a/\mu,\textrm{ c.s.}
\end{eqnarray}
La segunda expresi\'on implica la primera. Conversamente, la primera implica la segunda si el proceso $Z\left(t\right)$ es creciente, o si $lim_{n\rightarrow\infty}n^{-1}M_{n}=0$ c.s.
\end{Teo}

\begin{Coro}
Si $N\left(t\right)$ es un proceso de renovaci\'on, y $\left(Z\left(T_{n}\right)-Z\left(T_{n-1}\right),M_{n}\right)$, para $n\geq1$, son variables aleatorias independientes e id\'enticamente distribuidas con media finita, entonces,
\begin{eqnarray}
lim_{t\rightarrow\infty}t^{-1}Z\left(t\right)\rightarrow\frac{\esp\left[Z\left(T_{1}\right)-Z\left(T_{0}\right)\right]}{\esp\left[T_{1}\right]},\textrm{ c.s. cuando  }t\rightarrow\infty.
\end{eqnarray}
\end{Coro}



%___________________________________________________________________________________________
%
\subsection{Propiedades de los Procesos de Renovaci\'on}
%___________________________________________________________________________________________
%

Los tiempos $T_{n}$ est\'an relacionados con los conteos de $N\left(t\right)$ por

\begin{eqnarray*}
\left\{N\left(t\right)\geq n\right\}&=&\left\{T_{n}\leq t\right\}\\
T_{N\left(t\right)}\leq &t&<T_{N\left(t\right)+1},
\end{eqnarray*}

adem\'as $N\left(T_{n}\right)=n$, y 

\begin{eqnarray*}
N\left(t\right)=\max\left\{n:T_{n}\leq t\right\}=\min\left\{n:T_{n+1}>t\right\}
\end{eqnarray*}

Por propiedades de la convoluci\'on se sabe que

\begin{eqnarray*}
P\left\{T_{n}\leq t\right\}=F^{n\star}\left(t\right)
\end{eqnarray*}
que es la $n$-\'esima convoluci\'on de $F$. Entonces 

\begin{eqnarray*}
\left\{N\left(t\right)\geq n\right\}&=&\left\{T_{n}\leq t\right\}\\
P\left\{N\left(t\right)\leq n\right\}&=&1-F^{\left(n+1\right)\star}\left(t\right)
\end{eqnarray*}

Adem\'as usando el hecho de que $\esp\left[N\left(t\right)\right]=\sum_{n=1}^{\infty}P\left\{N\left(t\right)\geq n\right\}$
se tiene que

\begin{eqnarray*}
\esp\left[N\left(t\right)\right]=\sum_{n=1}^{\infty}F^{n\star}\left(t\right)
\end{eqnarray*}

\begin{Prop}
Para cada $t\geq0$, la funci\'on generadora de momentos $\esp\left[e^{\alpha N\left(t\right)}\right]$ existe para alguna $\alpha$ en una vecindad del 0, y de aqu\'i que $\esp\left[N\left(t\right)^{m}\right]<\infty$, para $m\geq1$.
\end{Prop}


\begin{Note}
Si el primer tiempo de renovaci\'on $\xi_{1}$ no tiene la misma distribuci\'on que el resto de las $\xi_{n}$, para $n\geq2$, a $N\left(t\right)$ se le llama Proceso de Renovaci\'on retardado, donde si $\xi$ tiene distribuci\'on $G$, entonces el tiempo $T_{n}$ de la $n$-\'esima renovaci\'on tiene distribuci\'on $G\star F^{\left(n-1\right)\star}\left(t\right)$
\end{Note}


\begin{Teo}
Para una constante $\mu\leq\infty$ ( o variable aleatoria), las siguientes expresiones son equivalentes:

\begin{eqnarray}
lim_{n\rightarrow\infty}n^{-1}T_{n}&=&\mu,\textrm{ c.s.}\\
lim_{t\rightarrow\infty}t^{-1}N\left(t\right)&=&1/\mu,\textrm{ c.s.}
\end{eqnarray}
\end{Teo}


Es decir, $T_{n}$ satisface la Ley Fuerte de los Grandes N\'umeros s\'i y s\'olo s\'i $N\left/t\right)$ la cumple.


\begin{Coro}[Ley Fuerte de los Grandes N\'umeros para Procesos de Renovaci\'on]
Si $N\left(t\right)$ es un proceso de renovaci\'on cuyos tiempos de inter-renovaci\'on tienen media $\mu\leq\infty$, entonces
\begin{eqnarray}
t^{-1}N\left(t\right)\rightarrow 1/\mu,\textrm{ c.s. cuando }t\rightarrow\infty.
\end{eqnarray}

\end{Coro}


Considerar el proceso estoc\'astico de valores reales $\left\{Z\left(t\right):t\geq0\right\}$ en el mismo espacio de probabilidad que $N\left(t\right)$

\begin{Def}
Para el proceso $\left\{Z\left(t\right):t\geq0\right\}$ se define la fluctuaci\'on m\'axima de $Z\left(t\right)$ en el intervalo $\left(T_{n-1},T_{n}\right]$:
\begin{eqnarray*}
M_{n}=\sup_{T_{n-1}<t\leq T_{n}}|Z\left(t\right)-Z\left(T_{n-1}\right)|
\end{eqnarray*}
\end{Def}

\begin{Teo}
Sup\'ongase que $n^{-1}T_{n}\rightarrow\mu$ c.s. cuando $n\rightarrow\infty$, donde $\mu\leq\infty$ es una constante o variable aleatoria. Sea $a$ una constante o variable aleatoria que puede ser infinita cuando $\mu$ es finita, y considere las expresiones l\'imite:
\begin{eqnarray}
lim_{n\rightarrow\infty}n^{-1}Z\left(T_{n}\right)&=&a,\textrm{ c.s.}\\
lim_{t\rightarrow\infty}t^{-1}Z\left(t\right)&=&a/\mu,\textrm{ c.s.}
\end{eqnarray}
La segunda expresi\'on implica la primera. Conversamente, la primera implica la segunda si el proceso $Z\left(t\right)$ es creciente, o si $lim_{n\rightarrow\infty}n^{-1}M_{n}=0$ c.s.
\end{Teo}

\begin{Coro}
Si $N\left(t\right)$ es un proceso de renovaci\'on, y $\left(Z\left(T_{n}\right)-Z\left(T_{n-1}\right),M_{n}\right)$, para $n\geq1$, son variables aleatorias independientes e id\'enticamente distribuidas con media finita, entonces,
\begin{eqnarray}
lim_{t\rightarrow\infty}t^{-1}Z\left(t\right)\rightarrow\frac{\esp\left[Z\left(T_{1}\right)-Z\left(T_{0}\right)\right]}{\esp\left[T_{1}\right]},\textrm{ c.s. cuando  }t\rightarrow\infty.
\end{eqnarray}
\end{Coro}




%__________________________________________________________________________________________
\subsection{Procesos Regenerativos Estacionarios - Stidham \cite{Stidham}}
%__________________________________________________________________________________________


Un proceso estoc\'astico a tiempo continuo $\left\{V\left(t\right),t\geq0\right\}$ es un proceso regenerativo si existe una sucesi\'on de variables aleatorias independientes e id\'enticamente distribuidas $\left\{X_{1},X_{2},\ldots\right\}$, sucesi\'on de renovaci\'on, tal que para cualquier conjunto de Borel $A$, 

\begin{eqnarray*}
\prob\left\{V\left(t\right)\in A|X_{1}+X_{2}+\cdots+X_{R\left(t\right)}=s,\left\{V\left(\tau\right),\tau<s\right\}\right\}=\prob\left\{V\left(t-s\right)\in A|X_{1}>t-s\right\},
\end{eqnarray*}
para todo $0\leq s\leq t$, donde $R\left(t\right)=\max\left\{X_{1}+X_{2}+\cdots+X_{j}\leq t\right\}=$n\'umero de renovaciones ({\emph{puntos de regeneraci\'on}}) que ocurren en $\left[0,t\right]$. El intervalo $\left[0,X_{1}\right)$ es llamado {\emph{primer ciclo de regeneraci\'on}} de $\left\{V\left(t \right),t\geq0\right\}$, $\left[X_{1},X_{1}+X_{2}\right)$ el {\emph{segundo ciclo de regeneraci\'on}}, y as\'i sucesivamente.

Sea $X=X_{1}$ y sea $F$ la funci\'on de distrbuci\'on de $X$


\begin{Def}
Se define el proceso estacionario, $\left\{V^{*}\left(t\right),t\geq0\right\}$, para $\left\{V\left(t\right),t\geq0\right\}$ por

\begin{eqnarray*}
\prob\left\{V\left(t\right)\in A\right\}=\frac{1}{\esp\left[X\right]}\int_{0}^{\infty}\prob\left\{V\left(t+x\right)\in A|X>x\right\}\left(1-F\left(x\right)\right)dx,
\end{eqnarray*} 
para todo $t\geq0$ y todo conjunto de Borel $A$.
\end{Def}

\begin{Def}
Una distribuci\'on se dice que es {\emph{aritm\'etica}} si todos sus puntos de incremento son m\'ultiplos de la forma $0,\lambda, 2\lambda,\ldots$ para alguna $\lambda>0$ entera.
\end{Def}


\begin{Def}
Una modificaci\'on medible de un proceso $\left\{V\left(t\right),t\geq0\right\}$, es una versi\'on de este, $\left\{V\left(t,w\right)\right\}$ conjuntamente medible para $t\geq0$ y para $w\in S$, $S$ espacio de estados para $\left\{V\left(t\right),t\geq0\right\}$.
\end{Def}

\begin{Teo}
Sea $\left\{V\left(t\right),t\geq\right\}$ un proceso regenerativo no negativo con modificaci\'on medible. Sea $\esp\left[X\right]<\infty$. Entonces el proceso estacionario dado por la ecuaci\'on anterior est\'a bien definido y tiene funci\'on de distribuci\'on independiente de $t$, adem\'as
\begin{itemize}
\item[i)] \begin{eqnarray*}
\esp\left[V^{*}\left(0\right)\right]&=&\frac{\esp\left[\int_{0}^{X}V\left(s\right)ds\right]}{\esp\left[X\right]}\end{eqnarray*}
\item[ii)] Si $\esp\left[V^{*}\left(0\right)\right]<\infty$, equivalentemente, si $\esp\left[\int_{0}^{X}V\left(s\right)ds\right]<\infty$,entonces
\begin{eqnarray*}
\frac{\int_{0}^{t}V\left(s\right)ds}{t}\rightarrow\frac{\esp\left[\int_{0}^{X}V\left(s\right)ds\right]}{\esp\left[X\right]}
\end{eqnarray*}
con probabilidad 1 y en media, cuando $t\rightarrow\infty$.
\end{itemize}
\end{Teo}

%______________________________________________________________________
\subsection{Procesos de Renovaci\'on}
%______________________________________________________________________

\begin{Def}\label{Def.Tn}
Sean $0\leq T_{1}\leq T_{2}\leq \ldots$ son tiempos aleatorios infinitos en los cuales ocurren ciertos eventos. El n\'umero de tiempos $T_{n}$ en el intervalo $\left[0,t\right)$ es

\begin{eqnarray}
N\left(t\right)=\sum_{n=1}^{\infty}\indora\left(T_{n}\leq t\right),
\end{eqnarray}
para $t\geq0$.
\end{Def}

Si se consideran los puntos $T_{n}$ como elementos de $\rea_{+}$, y $N\left(t\right)$ es el n\'umero de puntos en $\rea$. El proceso denotado por $\left\{N\left(t\right):t\geq0\right\}$, denotado por $N\left(t\right)$, es un proceso puntual en $\rea_{+}$. Los $T_{n}$ son los tiempos de ocurrencia, el proceso puntual $N\left(t\right)$ es simple si su n\'umero de ocurrencias son distintas: $0<T_{1}<T_{2}<\ldots$ casi seguramente.

\begin{Def}
Un proceso puntual $N\left(t\right)$ es un proceso de renovaci\'on si los tiempos de interocurrencia $\xi_{n}=T_{n}-T_{n-1}$, para $n\geq1$, son independientes e identicamente distribuidos con distribuci\'on $F$, donde $F\left(0\right)=0$ y $T_{0}=0$. Los $T_{n}$ son llamados tiempos de renovaci\'on, referente a la independencia o renovaci\'on de la informaci\'on estoc\'astica en estos tiempos. Los $\xi_{n}$ son los tiempos de inter-renovaci\'on, y $N\left(t\right)$ es el n\'umero de renovaciones en el intervalo $\left[0,t\right)$
\end{Def}


\begin{Note}
Para definir un proceso de renovaci\'on para cualquier contexto, solamente hay que especificar una distribuci\'on $F$, con $F\left(0\right)=0$, para los tiempos de inter-renovaci\'on. La funci\'on $F$ en turno degune las otra variables aleatorias. De manera formal, existe un espacio de probabilidad y una sucesi\'on de variables aleatorias $\xi_{1},\xi_{2},\ldots$ definidas en este con distribuci\'on $F$. Entonces las otras cantidades son $T_{n}=\sum_{k=1}^{n}\xi_{k}$ y $N\left(t\right)=\sum_{n=1}^{\infty}\indora\left(T_{n}\leq t\right)$, donde $T_{n}\rightarrow\infty$ casi seguramente por la Ley Fuerte de los Grandes Números.
\end{Note}

%___________________________________________________________________________________________
%
\subsection{Teorema Principal de Renovaci\'on}
%___________________________________________________________________________________________
%

\begin{Note} Una funci\'on $h:\rea_{+}\rightarrow\rea$ es Directamente Riemann Integrable en los siguientes casos:
\begin{itemize}
\item[a)] $h\left(t\right)\geq0$ es decreciente y Riemann Integrable.
\item[b)] $h$ es continua excepto posiblemente en un conjunto de Lebesgue de medida 0, y $|h\left(t\right)|\leq b\left(t\right)$, donde $b$ es DRI.
\end{itemize}
\end{Note}

\begin{Teo}[Teorema Principal de Renovaci\'on]
Si $F$ es no aritm\'etica y $h\left(t\right)$ es Directamente Riemann Integrable (DRI), entonces

\begin{eqnarray*}
lim_{t\rightarrow\infty}U\star h=\frac{1}{\mu}\int_{\rea_{+}}h\left(s\right)ds.
\end{eqnarray*}
\end{Teo}

\begin{Prop}
Cualquier funci\'on $H\left(t\right)$ acotada en intervalos finitos y que es 0 para $t<0$ puede expresarse como
\begin{eqnarray*}
H\left(t\right)=U\star h\left(t\right)\textrm{,  donde }h\left(t\right)=H\left(t\right)-F\star H\left(t\right)
\end{eqnarray*}
\end{Prop}

\begin{Def}
Un proceso estoc\'astico $X\left(t\right)$ es crudamente regenerativo en un tiempo aleatorio positivo $T$ si
\begin{eqnarray*}
\esp\left[X\left(T+t\right)|T\right]=\esp\left[X\left(t\right)\right]\textrm{, para }t\geq0,\end{eqnarray*}
y con las esperanzas anteriores finitas.
\end{Def}

\begin{Prop}
Sup\'ongase que $X\left(t\right)$ es un proceso crudamente regenerativo en $T$, que tiene distribuci\'on $F$. Si $\esp\left[X\left(t\right)\right]$ es acotado en intervalos finitos, entonces
\begin{eqnarray*}
\esp\left[X\left(t\right)\right]=U\star h\left(t\right)\textrm{,  donde }h\left(t\right)=\esp\left[X\left(t\right)\indora\left(T>t\right)\right].
\end{eqnarray*}
\end{Prop}

\begin{Teo}[Regeneraci\'on Cruda]
Sup\'ongase que $X\left(t\right)$ es un proceso con valores positivo crudamente regenerativo en $T$, y def\'inase $M=\sup\left\{|X\left(t\right)|:t\leq T\right\}$. Si $T$ es no aritm\'etico y $M$ y $MT$ tienen media finita, entonces
\begin{eqnarray*}
lim_{t\rightarrow\infty}\esp\left[X\left(t\right)\right]=\frac{1}{\mu}\int_{\rea_{+}}h\left(s\right)ds,
\end{eqnarray*}
donde $h\left(t\right)=\esp\left[X\left(t\right)\indora\left(T>t\right)\right]$.
\end{Teo}



%___________________________________________________________________________________________
%
\subsection{Funci\'on de Renovaci\'on}
%___________________________________________________________________________________________
%


\begin{Def}
Sea $h\left(t\right)$ funci\'on de valores reales en $\rea$ acotada en intervalos finitos e igual a cero para $t<0$ La ecuaci\'on de renovaci\'on para $h\left(t\right)$ y la distribuci\'on $F$ es

\begin{eqnarray}\label{Ec.Renovacion}
H\left(t\right)=h\left(t\right)+\int_{\left[0,t\right]}H\left(t-s\right)dF\left(s\right)\textrm{,    }t\geq0,
\end{eqnarray}
donde $H\left(t\right)$ es una funci\'on de valores reales. Esto es $H=h+F\star H$. Decimos que $H\left(t\right)$ es soluci\'on de esta ecuaci\'on si satisface la ecuaci\'on, y es acotada en intervalos finitos e iguales a cero para $t<0$.
\end{Def}

\begin{Prop}
La funci\'on $U\star h\left(t\right)$ es la \'unica soluci\'on de la ecuaci\'on de renovaci\'on (\ref{Ec.Renovacion}).
\end{Prop}

\begin{Teo}[Teorema Renovaci\'on Elemental]
\begin{eqnarray*}
t^{-1}U\left(t\right)\rightarrow 1/\mu\textrm{,    cuando }t\rightarrow\infty.
\end{eqnarray*}
\end{Teo}

%___________________________________________________________________________________________
%
\subsection{Propiedades de los Procesos de Renovaci\'on}
%___________________________________________________________________________________________
%

Los tiempos $T_{n}$ est\'an relacionados con los conteos de $N\left(t\right)$ por

\begin{eqnarray*}
\left\{N\left(t\right)\geq n\right\}&=&\left\{T_{n}\leq t\right\}\\
T_{N\left(t\right)}\leq &t&<T_{N\left(t\right)+1},
\end{eqnarray*}

adem\'as $N\left(T_{n}\right)=n$, y 

\begin{eqnarray*}
N\left(t\right)=\max\left\{n:T_{n}\leq t\right\}=\min\left\{n:T_{n+1}>t\right\}
\end{eqnarray*}

Por propiedades de la convoluci\'on se sabe que

\begin{eqnarray*}
P\left\{T_{n}\leq t\right\}=F^{n\star}\left(t\right)
\end{eqnarray*}
que es la $n$-\'esima convoluci\'on de $F$. Entonces 

\begin{eqnarray*}
\left\{N\left(t\right)\geq n\right\}&=&\left\{T_{n}\leq t\right\}\\
P\left\{N\left(t\right)\leq n\right\}&=&1-F^{\left(n+1\right)\star}\left(t\right)
\end{eqnarray*}

Adem\'as usando el hecho de que $\esp\left[N\left(t\right)\right]=\sum_{n=1}^{\infty}P\left\{N\left(t\right)\geq n\right\}$
se tiene que

\begin{eqnarray*}
\esp\left[N\left(t\right)\right]=\sum_{n=1}^{\infty}F^{n\star}\left(t\right)
\end{eqnarray*}

\begin{Prop}
Para cada $t\geq0$, la funci\'on generadora de momentos $\esp\left[e^{\alpha N\left(t\right)}\right]$ existe para alguna $\alpha$ en una vecindad del 0, y de aqu\'i que $\esp\left[N\left(t\right)^{m}\right]<\infty$, para $m\geq1$.
\end{Prop}


\begin{Note}
Si el primer tiempo de renovaci\'on $\xi_{1}$ no tiene la misma distribuci\'on que el resto de las $\xi_{n}$, para $n\geq2$, a $N\left(t\right)$ se le llama Proceso de Renovaci\'on retardado, donde si $\xi$ tiene distribuci\'on $G$, entonces el tiempo $T_{n}$ de la $n$-\'esima renovaci\'on tiene distribuci\'on $G\star F^{\left(n-1\right)\star}\left(t\right)$
\end{Note}


\begin{Teo}
Para una constante $\mu\leq\infty$ ( o variable aleatoria), las siguientes expresiones son equivalentes:

\begin{eqnarray}
lim_{n\rightarrow\infty}n^{-1}T_{n}&=&\mu,\textrm{ c.s.}\\
lim_{t\rightarrow\infty}t^{-1}N\left(t\right)&=&1/\mu,\textrm{ c.s.}
\end{eqnarray}
\end{Teo}


Es decir, $T_{n}$ satisface la Ley Fuerte de los Grandes N\'umeros s\'i y s\'olo s\'i $N\left/t\right)$ la cumple.


\begin{Coro}[Ley Fuerte de los Grandes N\'umeros para Procesos de Renovaci\'on]
Si $N\left(t\right)$ es un proceso de renovaci\'on cuyos tiempos de inter-renovaci\'on tienen media $\mu\leq\infty$, entonces
\begin{eqnarray}
t^{-1}N\left(t\right)\rightarrow 1/\mu,\textrm{ c.s. cuando }t\rightarrow\infty.
\end{eqnarray}

\end{Coro}


Considerar el proceso estoc\'astico de valores reales $\left\{Z\left(t\right):t\geq0\right\}$ en el mismo espacio de probabilidad que $N\left(t\right)$

\begin{Def}
Para el proceso $\left\{Z\left(t\right):t\geq0\right\}$ se define la fluctuaci\'on m\'axima de $Z\left(t\right)$ en el intervalo $\left(T_{n-1},T_{n}\right]$:
\begin{eqnarray*}
M_{n}=\sup_{T_{n-1}<t\leq T_{n}}|Z\left(t\right)-Z\left(T_{n-1}\right)|
\end{eqnarray*}
\end{Def}

\begin{Teo}
Sup\'ongase que $n^{-1}T_{n}\rightarrow\mu$ c.s. cuando $n\rightarrow\infty$, donde $\mu\leq\infty$ es una constante o variable aleatoria. Sea $a$ una constante o variable aleatoria que puede ser infinita cuando $\mu$ es finita, y considere las expresiones l\'imite:
\begin{eqnarray}
lim_{n\rightarrow\infty}n^{-1}Z\left(T_{n}\right)&=&a,\textrm{ c.s.}\\
lim_{t\rightarrow\infty}t^{-1}Z\left(t\right)&=&a/\mu,\textrm{ c.s.}
\end{eqnarray}
La segunda expresi\'on implica la primera. Conversamente, la primera implica la segunda si el proceso $Z\left(t\right)$ es creciente, o si $lim_{n\rightarrow\infty}n^{-1}M_{n}=0$ c.s.
\end{Teo}

\begin{Coro}
Si $N\left(t\right)$ es un proceso de renovaci\'on, y $\left(Z\left(T_{n}\right)-Z\left(T_{n-1}\right),M_{n}\right)$, para $n\geq1$, son variables aleatorias independientes e id\'enticamente distribuidas con media finita, entonces,
\begin{eqnarray}
lim_{t\rightarrow\infty}t^{-1}Z\left(t\right)\rightarrow\frac{\esp\left[Z\left(T_{1}\right)-Z\left(T_{0}\right)\right]}{\esp\left[T_{1}\right]},\textrm{ c.s. cuando  }t\rightarrow\infty.
\end{eqnarray}
\end{Coro}

%___________________________________________________________________________________________
%
\subsection{Funci\'on de Renovaci\'on}
%___________________________________________________________________________________________
%


Sup\'ongase que $N\left(t\right)$ es un proceso de renovaci\'on con distribuci\'on $F$ con media finita $\mu$.

\begin{Def}
La funci\'on de renovaci\'on asociada con la distribuci\'on $F$, del proceso $N\left(t\right)$, es
\begin{eqnarray*}
U\left(t\right)=\sum_{n=1}^{\infty}F^{n\star}\left(t\right),\textrm{   }t\geq0,
\end{eqnarray*}
donde $F^{0\star}\left(t\right)=\indora\left(t\geq0\right)$.
\end{Def}


\begin{Prop}
Sup\'ongase que la distribuci\'on de inter-renovaci\'on $F$ tiene densidad $f$. Entonces $U\left(t\right)$ tambi\'en tiene densidad, para $t>0$, y es $U^{'}\left(t\right)=\sum_{n=0}^{\infty}f^{n\star}\left(t\right)$. Adem\'as
\begin{eqnarray*}
\prob\left\{N\left(t\right)>N\left(t-\right)\right\}=0\textrm{,   }t\geq0.
\end{eqnarray*}
\end{Prop}

\begin{Def}
La Transformada de Laplace-Stieljes de $F$ est\'a dada por

\begin{eqnarray*}
\hat{F}\left(\alpha\right)=\int_{\rea_{+}}e^{-\alpha t}dF\left(t\right)\textrm{,  }\alpha\geq0.
\end{eqnarray*}
\end{Def}

Entonces

\begin{eqnarray*}
\hat{U}\left(\alpha\right)=\sum_{n=0}^{\infty}\hat{F^{n\star}}\left(\alpha\right)=\sum_{n=0}^{\infty}\hat{F}\left(\alpha\right)^{n}=\frac{1}{1-\hat{F}\left(\alpha\right)}.
\end{eqnarray*}


\begin{Prop}
La Transformada de Laplace $\hat{U}\left(\alpha\right)$ y $\hat{F}\left(\alpha\right)$ determina una a la otra de manera \'unica por la relaci\'on $\hat{U}\left(\alpha\right)=\frac{1}{1-\hat{F}\left(\alpha\right)}$.
\end{Prop}


\begin{Note}
Un proceso de renovaci\'on $N\left(t\right)$ cuyos tiempos de inter-renovaci\'on tienen media finita, es un proceso Poisson con tasa $\lambda$ si y s\'olo s\'i $\esp\left[U\left(t\right)\right]=\lambda t$, para $t\geq0$.
\end{Note}


\begin{Teo}
Sea $N\left(t\right)$ un proceso puntual simple con puntos de localizaci\'on $T_{n}$ tal que $\eta\left(t\right)=\esp\left[N\left(\right)\right]$ es finita para cada $t$. Entonces para cualquier funci\'on $f:\rea_{+}\rightarrow\rea$,
\begin{eqnarray*}
\esp\left[\sum_{n=1}^{N\left(\right)}f\left(T_{n}\right)\right]=\int_{\left(0,t\right]}f\left(s\right)d\eta\left(s\right)\textrm{,  }t\geq0,
\end{eqnarray*}
suponiendo que la integral exista. Adem\'as si $X_{1},X_{2},\ldots$ son variables aleatorias definidas en el mismo espacio de probabilidad que el proceso $N\left(t\right)$ tal que $\esp\left[X_{n}|T_{n}=s\right]=f\left(s\right)$, independiente de $n$. Entonces
\begin{eqnarray*}
\esp\left[\sum_{n=1}^{N\left(t\right)}X_{n}\right]=\int_{\left(0,t\right]}f\left(s\right)d\eta\left(s\right)\textrm{,  }t\geq0,
\end{eqnarray*} 
suponiendo que la integral exista. 
\end{Teo}

\begin{Coro}[Identidad de Wald para Renovaciones]
Para el proceso de renovaci\'on $N\left(t\right)$,
\begin{eqnarray*}
\esp\left[T_{N\left(t\right)+1}\right]=\mu\esp\left[N\left(t\right)+1\right]\textrm{,  }t\geq0,
\end{eqnarray*}  
\end{Coro}

%______________________________________________________________________
\subsection{Procesos de Renovaci\'on}
%______________________________________________________________________

\begin{Def}\label{Def.Tn}
Sean $0\leq T_{1}\leq T_{2}\leq \ldots$ son tiempos aleatorios infinitos en los cuales ocurren ciertos eventos. El n\'umero de tiempos $T_{n}$ en el intervalo $\left[0,t\right)$ es

\begin{eqnarray}
N\left(t\right)=\sum_{n=1}^{\infty}\indora\left(T_{n}\leq t\right),
\end{eqnarray}
para $t\geq0$.
\end{Def}

Si se consideran los puntos $T_{n}$ como elementos de $\rea_{+}$, y $N\left(t\right)$ es el n\'umero de puntos en $\rea$. El proceso denotado por $\left\{N\left(t\right):t\geq0\right\}$, denotado por $N\left(t\right)$, es un proceso puntual en $\rea_{+}$. Los $T_{n}$ son los tiempos de ocurrencia, el proceso puntual $N\left(t\right)$ es simple si su n\'umero de ocurrencias son distintas: $0<T_{1}<T_{2}<\ldots$ casi seguramente.

\begin{Def}
Un proceso puntual $N\left(t\right)$ es un proceso de renovaci\'on si los tiempos de interocurrencia $\xi_{n}=T_{n}-T_{n-1}$, para $n\geq1$, son independientes e identicamente distribuidos con distribuci\'on $F$, donde $F\left(0\right)=0$ y $T_{0}=0$. Los $T_{n}$ son llamados tiempos de renovaci\'on, referente a la independencia o renovaci\'on de la informaci\'on estoc\'astica en estos tiempos. Los $\xi_{n}$ son los tiempos de inter-renovaci\'on, y $N\left(t\right)$ es el n\'umero de renovaciones en el intervalo $\left[0,t\right)$
\end{Def}


\begin{Note}
Para definir un proceso de renovaci\'on para cualquier contexto, solamente hay que especificar una distribuci\'on $F$, con $F\left(0\right)=0$, para los tiempos de inter-renovaci\'on. La funci\'on $F$ en turno degune las otra variables aleatorias. De manera formal, existe un espacio de probabilidad y una sucesi\'on de variables aleatorias $\xi_{1},\xi_{2},\ldots$ definidas en este con distribuci\'on $F$. Entonces las otras cantidades son $T_{n}=\sum_{k=1}^{n}\xi_{k}$ y $N\left(t\right)=\sum_{n=1}^{\infty}\indora\left(T_{n}\leq t\right)$, donde $T_{n}\rightarrow\infty$ casi seguramente por la Ley Fuerte de los Grandes Números.
\end{Note}
%_____________________________________________________
\subsection{Puntos de Renovaci\'on}
%_____________________________________________________

Para cada cola $Q_{i}$ se tienen los procesos de arribo a la cola, para estas, los tiempos de arribo est\'an dados por $$\left\{T_{1}^{i},T_{2}^{i},\ldots,T_{k}^{i},\ldots\right\},$$ entonces, consideremos solamente los primeros tiempos de arribo a cada una de las colas, es decir, $$\left\{T_{1}^{1},T_{1}^{2},T_{1}^{3},T_{1}^{4}\right\},$$ se sabe que cada uno de estos tiempos se distribuye de manera exponencial con par\'ametro $1/mu_{i}$. Adem\'as se sabe que para $$T^{*}=\min\left\{T_{1}^{1},T_{1}^{2},T_{1}^{3},T_{1}^{4}\right\},$$ $T^{*}$ se distribuye de manera exponencial con par\'ametro $$\mu^{*}=\sum_{i=1}^{4}\mu_{i}.$$ Ahora, dado que 
\begin{center}
\begin{tabular}{lcl}
$\tilde{r}=r_{1}+r_{2}$ & y &$\hat{r}=r_{3}+r_{4}.$
\end{tabular}
\end{center}


Supongamos que $$\tilde{r},\hat{r}<\mu^{*},$$ entonces si tomamos $$r^{*}=\min\left\{\tilde{r},\hat{r}\right\},$$ se tiene que para  $$t^{*}\in\left(0,r^{*}\right)$$ se cumple que 
\begin{center}
\begin{tabular}{lcl}
$\tau_{1}\left(1\right)=0$ & y por tanto & $\overline{\tau}_{1}=0,$
\end{tabular}
\end{center}
entonces para la segunda cola en este primer ciclo se cumple que $$\tau_{2}=\overline{\tau}_{1}+r_{1}=r_{1}<\mu^{*},$$ y por tanto se tiene que  $$\overline{\tau}_{2}=\tau_{2}.$$ Por lo tanto, nuevamente para la primer cola en el segundo ciclo $$\tau_{1}\left(2\right)=\tau_{2}\left(1\right)+r_{2}=\tilde{r}<\mu^{*}.$$ An\'alogamente para el segundo sistema se tiene que ambas colas est\'an vac\'ias, es decir, existe un valor $t^{*}$ tal que en el intervalo $\left(0,t^{*}\right)$ no ha llegado ning\'un usuario, es decir, $$L_{i}\left(t^{*}\right)=0$$ para $i=1,2,3,4$.




Sup\'ongase que $N\left(t\right)$ es un proceso de renovaci\'on con distribuci\'on $F$ con media finita $\mu$.

\begin{Def}
La funci\'on de renovaci\'on asociada con la distribuci\'on $F$, del proceso $N\left(t\right)$, es
\begin{eqnarray*}
U\left(t\right)=\sum_{n=1}^{\infty}F^{n\star}\left(t\right),\textrm{   }t\geq0,
\end{eqnarray*}
donde $F^{0\star}\left(t\right)=\indora\left(t\geq0\right)$.
\end{Def}


\begin{Prop}
Sup\'ongase que la distribuci\'on de inter-renovaci\'on $F$ tiene densidad $f$. Entonces $U\left(t\right)$ tambi\'en tiene densidad, para $t>0$, y es $U^{'}\left(t\right)=\sum_{n=0}^{\infty}f^{n\star}\left(t\right)$. Adem\'as
\begin{eqnarray*}
\prob\left\{N\left(t\right)>N\left(t-\right)\right\}=0\textrm{,   }t\geq0.
\end{eqnarray*}
\end{Prop}

\begin{Def}
La Transformada de Laplace-Stieljes de $F$ est\'a dada por

\begin{eqnarray*}
\hat{F}\left(\alpha\right)=\int_{\rea_{+}}e^{-\alpha t}dF\left(t\right)\textrm{,  }\alpha\geq0.
\end{eqnarray*}
\end{Def}

Entonces

\begin{eqnarray*}
\hat{U}\left(\alpha\right)=\sum_{n=0}^{\infty}\hat{F^{n\star}}\left(\alpha\right)=\sum_{n=0}^{\infty}\hat{F}\left(\alpha\right)^{n}=\frac{1}{1-\hat{F}\left(\alpha\right)}.
\end{eqnarray*}


\begin{Prop}
La Transformada de Laplace $\hat{U}\left(\alpha\right)$ y $\hat{F}\left(\alpha\right)$ determina una a la otra de manera \'unica por la relaci\'on $\hat{U}\left(\alpha\right)=\frac{1}{1-\hat{F}\left(\alpha\right)}$.
\end{Prop}


\begin{Note}
Un proceso de renovaci\'on $N\left(t\right)$ cuyos tiempos de inter-renovaci\'on tienen media finita, es un proceso Poisson con tasa $\lambda$ si y s\'olo s\'i $\esp\left[U\left(t\right)\right]=\lambda t$, para $t\geq0$.
\end{Note}


\begin{Teo}
Sea $N\left(t\right)$ un proceso puntual simple con puntos de localizaci\'on $T_{n}$ tal que $\eta\left(t\right)=\esp\left[N\left(\right)\right]$ es finita para cada $t$. Entonces para cualquier funci\'on $f:\rea_{+}\rightarrow\rea$,
\begin{eqnarray*}
\esp\left[\sum_{n=1}^{N\left(\right)}f\left(T_{n}\right)\right]=\int_{\left(0,t\right]}f\left(s\right)d\eta\left(s\right)\textrm{,  }t\geq0,
\end{eqnarray*}
suponiendo que la integral exista. Adem\'as si $X_{1},X_{2},\ldots$ son variables aleatorias definidas en el mismo espacio de probabilidad que el proceso $N\left(t\right)$ tal que $\esp\left[X_{n}|T_{n}=s\right]=f\left(s\right)$, independiente de $n$. Entonces
\begin{eqnarray*}
\esp\left[\sum_{n=1}^{N\left(t\right)}X_{n}\right]=\int_{\left(0,t\right]}f\left(s\right)d\eta\left(s\right)\textrm{,  }t\geq0,
\end{eqnarray*} 
suponiendo que la integral exista. 
\end{Teo}

\begin{Coro}[Identidad de Wald para Renovaciones]
Para el proceso de renovaci\'on $N\left(t\right)$,
\begin{eqnarray*}
\esp\left[T_{N\left(t\right)+1}\right]=\mu\esp\left[N\left(t\right)+1\right]\textrm{,  }t\geq0,
\end{eqnarray*}  
\end{Coro}

%___________________________________________________________________________________________
%
\subsection{Funci\'on de Renovaci\'on}
%___________________________________________________________________________________________
%


Sup\'ongase que $N\left(t\right)$ es un proceso de renovaci\'on con distribuci\'on $F$ con media finita $\mu$.

\begin{Def}
La funci\'on de renovaci\'on asociada con la distribuci\'on $F$, del proceso $N\left(t\right)$, es
\begin{eqnarray*}
U\left(t\right)=\sum_{n=1}^{\infty}F^{n\star}\left(t\right),\textrm{   }t\geq0,
\end{eqnarray*}
donde $F^{0\star}\left(t\right)=\indora\left(t\geq0\right)$.
\end{Def}


\begin{Prop}
Sup\'ongase que la distribuci\'on de inter-renovaci\'on $F$ tiene densidad $f$. Entonces $U\left(t\right)$ tambi\'en tiene densidad, para $t>0$, y es $U^{'}\left(t\right)=\sum_{n=0}^{\infty}f^{n\star}\left(t\right)$. Adem\'as
\begin{eqnarray*}
\prob\left\{N\left(t\right)>N\left(t-\right)\right\}=0\textrm{,   }t\geq0.
\end{eqnarray*}
\end{Prop}

\begin{Def}
La Transformada de Laplace-Stieljes de $F$ est\'a dada por

\begin{eqnarray*}
\hat{F}\left(\alpha\right)=\int_{\rea_{+}}e^{-\alpha t}dF\left(t\right)\textrm{,  }\alpha\geq0.
\end{eqnarray*}
\end{Def}

Entonces

\begin{eqnarray*}
\hat{U}\left(\alpha\right)=\sum_{n=0}^{\infty}\hat{F^{n\star}}\left(\alpha\right)=\sum_{n=0}^{\infty}\hat{F}\left(\alpha\right)^{n}=\frac{1}{1-\hat{F}\left(\alpha\right)}.
\end{eqnarray*}


\begin{Prop}
La Transformada de Laplace $\hat{U}\left(\alpha\right)$ y $\hat{F}\left(\alpha\right)$ determina una a la otra de manera \'unica por la relaci\'on $\hat{U}\left(\alpha\right)=\frac{1}{1-\hat{F}\left(\alpha\right)}$.
\end{Prop}


\begin{Note}
Un proceso de renovaci\'on $N\left(t\right)$ cuyos tiempos de inter-renovaci\'on tienen media finita, es un proceso Poisson con tasa $\lambda$ si y s\'olo s\'i $\esp\left[U\left(t\right)\right]=\lambda t$, para $t\geq0$.
\end{Note}


\begin{Teo}
Sea $N\left(t\right)$ un proceso puntual simple con puntos de localizaci\'on $T_{n}$ tal que $\eta\left(t\right)=\esp\left[N\left(\right)\right]$ es finita para cada $t$. Entonces para cualquier funci\'on $f:\rea_{+}\rightarrow\rea$,
\begin{eqnarray*}
\esp\left[\sum_{n=1}^{N\left(\right)}f\left(T_{n}\right)\right]=\int_{\left(0,t\right]}f\left(s\right)d\eta\left(s\right)\textrm{,  }t\geq0,
\end{eqnarray*}
suponiendo que la integral exista. Adem\'as si $X_{1},X_{2},\ldots$ son variables aleatorias definidas en el mismo espacio de probabilidad que el proceso $N\left(t\right)$ tal que $\esp\left[X_{n}|T_{n}=s\right]=f\left(s\right)$, independiente de $n$. Entonces
\begin{eqnarray*}
\esp\left[\sum_{n=1}^{N\left(t\right)}X_{n}\right]=\int_{\left(0,t\right]}f\left(s\right)d\eta\left(s\right)\textrm{,  }t\geq0,
\end{eqnarray*}
suponiendo que la integral exista.
\end{Teo}

\begin{Coro}[Identidad de Wald para Renovaciones]
Para el proceso de renovaci\'on $N\left(t\right)$,
\begin{eqnarray*}
\esp\left[T_{N\left(t\right)+1}\right]=\mu\esp\left[N\left(t\right)+1\right]\textrm{,  }t\geq0,
\end{eqnarray*}
\end{Coro}
%___________________________________________________________________________________________
%
\subsection{Funci\'on de Renovaci\'on}
%___________________________________________________________________________________________
%


\begin{Def}
Sea $h\left(t\right)$ funci\'on de valores reales en $\rea$ acotada en intervalos finitos e igual a cero para $t<0$ La ecuaci\'on de renovaci\'on para $h\left(t\right)$ y la distribuci\'on $F$ es

\begin{eqnarray}\label{Ec.Renovacion}
H\left(t\right)=h\left(t\right)+\int_{\left[0,t\right]}H\left(t-s\right)dF\left(s\right)\textrm{,    }t\geq0,
\end{eqnarray}
donde $H\left(t\right)$ es una funci\'on de valores reales. Esto es $H=h+F\star H$. Decimos que $H\left(t\right)$ es soluci\'on de esta ecuaci\'on si satisface la ecuaci\'on, y es acotada en intervalos finitos e iguales a cero para $t<0$.
\end{Def}

\begin{Prop}
La funci\'on $U\star h\left(t\right)$ es la \'unica soluci\'on de la ecuaci\'on de renovaci\'on (\ref{Ec.Renovacion}).
\end{Prop}

\begin{Teo}[Teorema Renovaci\'on Elemental]
\begin{eqnarray*}
t^{-1}U\left(t\right)\rightarrow 1/\mu\textrm{,    cuando }t\rightarrow\infty.
\end{eqnarray*}
\end{Teo}


%___________________________________________________________________________________________
%
\subsection{Funci\'on de Renovaci\'on}
%___________________________________________________________________________________________
%


\begin{Def}
Sea $h\left(t\right)$ funci\'on de valores reales en $\rea$ acotada en intervalos finitos e igual a cero para $t<0$ La ecuaci\'on de renovaci\'on para $h\left(t\right)$ y la distribuci\'on $F$ es

\begin{eqnarray}\label{Ec.Renovacion}
H\left(t\right)=h\left(t\right)+\int_{\left[0,t\right]}H\left(t-s\right)dF\left(s\right)\textrm{,    }t\geq0,
\end{eqnarray}
donde $H\left(t\right)$ es una funci\'on de valores reales. Esto es $H=h+F\star H$. Decimos que $H\left(t\right)$ es soluci\'on de esta ecuaci\'on si satisface la ecuaci\'on, y es acotada en intervalos finitos e iguales a cero para $t<0$.
\end{Def}

\begin{Prop}
La funci\'on $U\star h\left(t\right)$ es la \'unica soluci\'on de la ecuaci\'on de renovaci\'on (\ref{Ec.Renovacion}).
\end{Prop}

\begin{Teo}[Teorema Renovaci\'on Elemental]
\begin{eqnarray*}
t^{-1}U\left(t\right)\rightarrow 1/\mu\textrm{,    cuando }t\rightarrow\infty.
\end{eqnarray*}
\end{Teo}


%___________________________________________________________________________________________
%
\subsection{Funci\'on de Renovaci\'on}
%___________________________________________________________________________________________
%


\begin{Def}
Sea $h\left(t\right)$ funci\'on de valores reales en $\rea$ acotada en intervalos finitos e igual a cero para $t<0$ La ecuaci\'on de renovaci\'on para $h\left(t\right)$ y la distribuci\'on $F$ es

\begin{eqnarray}%\label{Ec.Renovacion}
H\left(t\right)=h\left(t\right)+\int_{\left[0,t\right]}H\left(t-s\right)dF\left(s\right)\textrm{,    }t\geq0,
\end{eqnarray}
donde $H\left(t\right)$ es una funci\'on de valores reales. Esto es $H=h+F\star H$. Decimos que $H\left(t\right)$ es soluci\'on de esta ecuaci\'on si satisface la ecuaci\'on, y es acotada en intervalos finitos e iguales a cero para $t<0$.
\end{Def}

\begin{Prop}
La funci\'on $U\star h\left(t\right)$ es la \'unica soluci\'on de la ecuaci\'on de renovaci\'on (\ref{Ec.Renovacion}).
\end{Prop}

\begin{Teo}[Teorema Renovaci\'on Elemental]
\begin{eqnarray*}
t^{-1}U\left(t\right)\rightarrow 1/\mu\textrm{,    cuando }t\rightarrow\infty.
\end{eqnarray*}
\end{Teo}

%___________________________________________________________________________________________
%
\subsection{Funci\'on de Renovaci\'on}
%___________________________________________________________________________________________
%


Sup\'ongase que $N\left(t\right)$ es un proceso de renovaci\'on con distribuci\'on $F$ con media finita $\mu$.

\begin{Def}
La funci\'on de renovaci\'on asociada con la distribuci\'on $F$, del proceso $N\left(t\right)$, es
\begin{eqnarray*}
U\left(t\right)=\sum_{n=1}^{\infty}F^{n\star}\left(t\right),\textrm{   }t\geq0,
\end{eqnarray*}
donde $F^{0\star}\left(t\right)=\indora\left(t\geq0\right)$.
\end{Def}


\begin{Prop}
Sup\'ongase que la distribuci\'on de inter-renovaci\'on $F$ tiene densidad $f$. Entonces $U\left(t\right)$ tambi\'en tiene densidad, para $t>0$, y es $U^{'}\left(t\right)=\sum_{n=0}^{\infty}f^{n\star}\left(t\right)$. Adem\'as
\begin{eqnarray*}
\prob\left\{N\left(t\right)>N\left(t-\right)\right\}=0\textrm{,   }t\geq0.
\end{eqnarray*}
\end{Prop}

\begin{Def}
La Transformada de Laplace-Stieljes de $F$ est\'a dada por

\begin{eqnarray*}
\hat{F}\left(\alpha\right)=\int_{\rea_{+}}e^{-\alpha t}dF\left(t\right)\textrm{,  }\alpha\geq0.
\end{eqnarray*}
\end{Def}

Entonces

\begin{eqnarray*}
\hat{U}\left(\alpha\right)=\sum_{n=0}^{\infty}\hat{F^{n\star}}\left(\alpha\right)=\sum_{n=0}^{\infty}\hat{F}\left(\alpha\right)^{n}=\frac{1}{1-\hat{F}\left(\alpha\right)}.
\end{eqnarray*}


\begin{Prop}
La Transformada de Laplace $\hat{U}\left(\alpha\right)$ y $\hat{F}\left(\alpha\right)$ determina una a la otra de manera \'unica por la relaci\'on $\hat{U}\left(\alpha\right)=\frac{1}{1-\hat{F}\left(\alpha\right)}$.
\end{Prop}


\begin{Note}
Un proceso de renovaci\'on $N\left(t\right)$ cuyos tiempos de inter-renovaci\'on tienen media finita, es un proceso Poisson con tasa $\lambda$ si y s\'olo s\'i $\esp\left[U\left(t\right)\right]=\lambda t$, para $t\geq0$.
\end{Note}


\begin{Teo}
Sea $N\left(t\right)$ un proceso puntual simple con puntos de localizaci\'on $T_{n}$ tal que $\eta\left(t\right)=\esp\left[N\left(\right)\right]$ es finita para cada $t$. Entonces para cualquier funci\'on $f:\rea_{+}\rightarrow\rea$,
\begin{eqnarray*}
\esp\left[\sum_{n=1}^{N\left(\right)}f\left(T_{n}\right)\right]=\int_{\left(0,t\right]}f\left(s\right)d\eta\left(s\right)\textrm{,  }t\geq0,
\end{eqnarray*}
suponiendo que la integral exista. Adem\'as si $X_{1},X_{2},\ldots$ son variables aleatorias definidas en el mismo espacio de probabilidad que el proceso $N\left(t\right)$ tal que $\esp\left[X_{n}|T_{n}=s\right]=f\left(s\right)$, independiente de $n$. Entonces
\begin{eqnarray*}
\esp\left[\sum_{n=1}^{N\left(t\right)}X_{n}\right]=\int_{\left(0,t\right]}f\left(s\right)d\eta\left(s\right)\textrm{,  }t\geq0,
\end{eqnarray*} 
suponiendo que la integral exista. 
\end{Teo}

\begin{Coro}[Identidad de Wald para Renovaciones]
Para el proceso de renovaci\'on $N\left(t\right)$,
\begin{eqnarray*}
\esp\left[T_{N\left(t\right)+1}\right]=\mu\esp\left[N\left(t\right)+1\right]\textrm{,  }t\geq0,
\end{eqnarray*}  
\end{Coro}

%___________________________________________________________________________________________
%
\subsection{Funci\'on de Renovaci\'on}
%___________________________________________________________________________________________
%


\begin{Def}
Sea $h\left(t\right)$ funci\'on de valores reales en $\rea$ acotada en intervalos finitos e igual a cero para $t<0$ La ecuaci\'on de renovaci\'on para $h\left(t\right)$ y la distribuci\'on $F$ es

\begin{eqnarray}%\label{Ec.Renovacion}
H\left(t\right)=h\left(t\right)+\int_{\left[0,t\right]}H\left(t-s\right)dF\left(s\right)\textrm{,    }t\geq0,
\end{eqnarray}
donde $H\left(t\right)$ es una funci\'on de valores reales. Esto es $H=h+F\star H$. Decimos que $H\left(t\right)$ es soluci\'on de esta ecuaci\'on si satisface la ecuaci\'on, y es acotada en intervalos finitos e iguales a cero para $t<0$.
\end{Def}

\begin{Prop}
La funci\'on $U\star h\left(t\right)$ es la \'unica soluci\'on de la ecuaci\'on de renovaci\'on (\ref{Ec.Renovacion}).
\end{Prop}

\begin{Teo}[Teorema Renovaci\'on Elemental]
\begin{eqnarray*}
t^{-1}U\left(t\right)\rightarrow 1/\mu\textrm{,    cuando }t\rightarrow\infty.
\end{eqnarray*}
\end{Teo}

%___________________________________________________________________________________________
%
\subsection{Funci\'on de Renovaci\'on}
%___________________________________________________________________________________________
%


Sup\'ongase que $N\left(t\right)$ es un proceso de renovaci\'on con distribuci\'on $F$ con media finita $\mu$.

\begin{Def}
La funci\'on de renovaci\'on asociada con la distribuci\'on $F$, del proceso $N\left(t\right)$, es
\begin{eqnarray*}
U\left(t\right)=\sum_{n=1}^{\infty}F^{n\star}\left(t\right),\textrm{   }t\geq0,
\end{eqnarray*}
donde $F^{0\star}\left(t\right)=\indora\left(t\geq0\right)$.
\end{Def}


\begin{Prop}
Sup\'ongase que la distribuci\'on de inter-renovaci\'on $F$ tiene densidad $f$. Entonces $U\left(t\right)$ tambi\'en tiene densidad, para $t>0$, y es $U^{'}\left(t\right)=\sum_{n=0}^{\infty}f^{n\star}\left(t\right)$. Adem\'as
\begin{eqnarray*}
\prob\left\{N\left(t\right)>N\left(t-\right)\right\}=0\textrm{,   }t\geq0.
\end{eqnarray*}
\end{Prop}

\begin{Def}
La Transformada de Laplace-Stieljes de $F$ est\'a dada por

\begin{eqnarray*}
\hat{F}\left(\alpha\right)=\int_{\rea_{+}}e^{-\alpha t}dF\left(t\right)\textrm{,  }\alpha\geq0.
\end{eqnarray*}
\end{Def}

Entonces

\begin{eqnarray*}
\hat{U}\left(\alpha\right)=\sum_{n=0}^{\infty}\hat{F^{n\star}}\left(\alpha\right)=\sum_{n=0}^{\infty}\hat{F}\left(\alpha\right)^{n}=\frac{1}{1-\hat{F}\left(\alpha\right)}.
\end{eqnarray*}


\begin{Prop}
La Transformada de Laplace $\hat{U}\left(\alpha\right)$ y $\hat{F}\left(\alpha\right)$ determina una a la otra de manera \'unica por la relaci\'on $\hat{U}\left(\alpha\right)=\frac{1}{1-\hat{F}\left(\alpha\right)}$.
\end{Prop}


\begin{Note}
Un proceso de renovaci\'on $N\left(t\right)$ cuyos tiempos de inter-renovaci\'on tienen media finita, es un proceso Poisson con tasa $\lambda$ si y s\'olo s\'i $\esp\left[U\left(t\right)\right]=\lambda t$, para $t\geq0$.
\end{Note}


\begin{Teo}
Sea $N\left(t\right)$ un proceso puntual simple con puntos de localizaci\'on $T_{n}$ tal que $\eta\left(t\right)=\esp\left[N\left(\right)\right]$ es finita para cada $t$. Entonces para cualquier funci\'on $f:\rea_{+}\rightarrow\rea$,
\begin{eqnarray*}
\esp\left[\sum_{n=1}^{N\left(\right)}f\left(T_{n}\right)\right]=\int_{\left(0,t\right]}f\left(s\right)d\eta\left(s\right)\textrm{,  }t\geq0,
\end{eqnarray*}
suponiendo que la integral exista. Adem\'as si $X_{1},X_{2},\ldots$ son variables aleatorias definidas en el mismo espacio de probabilidad que el proceso $N\left(t\right)$ tal que $\esp\left[X_{n}|T_{n}=s\right]=f\left(s\right)$, independiente de $n$. Entonces
\begin{eqnarray*}
\esp\left[\sum_{n=1}^{N\left(t\right)}X_{n}\right]=\int_{\left(0,t\right]}f\left(s\right)d\eta\left(s\right)\textrm{,  }t\geq0,
\end{eqnarray*} 
suponiendo que la integral exista. 
\end{Teo}

\begin{Coro}[Identidad de Wald para Renovaciones]
Para el proceso de renovaci\'on $N\left(t\right)$,
\begin{eqnarray*}
\esp\left[T_{N\left(t\right)+1}\right]=\mu\esp\left[N\left(t\right)+1\right]\textrm{,  }t\geq0,
\end{eqnarray*}  
\end{Coro}


%________________________________________________________________________
\subsection{Procesos Regenerativos}
%________________________________________________________________________

Para $\left\{X\left(t\right):t\geq0\right\}$ Proceso Estoc\'astico a tiempo continuo con estado de espacios $S$, que es un espacio m\'etrico, con trayectorias continuas por la derecha y con l\'imites por la izquierda c.s. Sea $N\left(t\right)$ un proceso de renovaci\'on en $\rea_{+}$ definido en el mismo espacio de probabilidad que $X\left(t\right)$, con tiempos de renovaci\'on $T$ y tiempos de inter-renovaci\'on $\xi_{n}=T_{n}-T_{n-1}$, con misma distribuci\'on $F$ de media finita $\mu$.



\begin{Def}
Para el proceso $\left\{\left(N\left(t\right),X\left(t\right)\right):t\geq0\right\}$, sus trayectoria muestrales en el intervalo de tiempo $\left[T_{n-1},T_{n}\right)$ est\'an descritas por
\begin{eqnarray*}
\zeta_{n}=\left(\xi_{n},\left\{X\left(T_{n-1}+t\right):0\leq t<\xi_{n}\right\}\right)
\end{eqnarray*}
Este $\zeta_{n}$ es el $n$-\'esimo segmento del proceso. El proceso es regenerativo sobre los tiempos $T_{n}$ si sus segmentos $\zeta_{n}$ son independientes e id\'enticamennte distribuidos.
\end{Def}


\begin{Obs}
Si $\tilde{X}\left(t\right)$ con espacio de estados $\tilde{S}$ es regenerativo sobre $T_{n}$, entonces $X\left(t\right)=f\left(\tilde{X}\left(t\right)\right)$ tambi\'en es regenerativo sobre $T_{n}$, para cualquier funci\'on $f:\tilde{S}\rightarrow S$.
\end{Obs}

\begin{Obs}
Los procesos regenerativos son crudamente regenerativos, pero no al rev\'es.
\end{Obs}

\begin{Def}[Definici\'on Cl\'asica]
Un proceso estoc\'astico $X=\left\{X\left(t\right):t\geq0\right\}$ es llamado regenerativo is existe una variable aleatoria $R_{1}>0$ tal que
\begin{itemize}
\item[i)] $\left\{X\left(t+R_{1}\right):t\geq0\right\}$ es independiente de $\left\{\left\{X\left(t\right):t<R_{1}\right\},\right\}$
\item[ii)] $\left\{X\left(t+R_{1}\right):t\geq0\right\}$ es estoc\'asticamente equivalente a $\left\{X\left(t\right):t>0\right\}$
\end{itemize}

Llamamos a $R_{1}$ tiempo de regeneraci\'on, y decimos que $X$ se regenera en este punto.
\end{Def}

$\left\{X\left(t+R_{1}\right)\right\}$ es regenerativo con tiempo de regeneraci\'on $R_{2}$, independiente de $R_{1}$ pero con la misma distribuci\'on que $R_{1}$. Procediendo de esta manera se obtiene una secuencia de variables aleatorias independientes e id\'enticamente distribuidas $\left\{R_{n}\right\}$ llamados longitudes de ciclo. Si definimos a $Z_{k}\equiv R_{1}+R_{2}+\cdots+R_{k}$, se tiene un proceso de renovaci\'on llamado proceso de renovaci\'on encajado para $X$.

\begin{Note}
Un proceso regenerativo con media de la longitud de ciclo finita es llamado positivo recurrente.
\end{Note}


\begin{Def}
Para $x$ fijo y para cada $t\geq0$, sea $I_{x}\left(t\right)=1$ si $X\left(t\right)\leq x$,  $I_{x}\left(t\right)=0$ en caso contrario, y def\'inanse los tiempos promedio
\begin{eqnarray*}
\overline{X}&=&lim_{t\rightarrow\infty}\frac{1}{t}\int_{0}^{\infty}X\left(u\right)du\\
\prob\left(X_{\infty}\leq x\right)&=&lim_{t\rightarrow\infty}\frac{1}{t}\int_{0}^{\infty}I_{x}\left(u\right)du,
\end{eqnarray*}
cuando estos l\'imites existan.
\end{Def}

Como consecuencia del teorema de Renovaci\'on-Recompensa, se tiene que el primer l\'imite  existe y es igual a la constante
\begin{eqnarray*}
\overline{X}&=&\frac{\esp\left[\int_{0}^{R_{1}}X\left(t\right)dt\right]}{\esp\left[R_{1}\right]},
\end{eqnarray*}
suponiendo que ambas esperanzas son finitas.

\begin{Note}
\begin{itemize}
\item[a)] Si el proceso regenerativo $X$ es positivo recurrente y tiene trayectorias muestrales no negativas, entonces la ecuaci\'on anterior es v\'alida.
\item[b)] Si $X$ es positivo recurrente regenerativo, podemos construir una \'unica versi\'on estacionaria de este proceso, $X_{e}=\left\{X_{e}\left(t\right)\right\}$, donde $X_{e}$ es un proceso estoc\'astico regenerativo y estrictamente estacionario, con distribuci\'on marginal distribuida como $X_{\infty}$
\end{itemize}
\end{Note}



%________________________________________________________________________
%\subsection{Procesos Regenerativos Sigman, Thorisson y Wolff \cite{Sigman1}}
%________________________________________________________________________


\begin{Def}[Definici\'on Cl\'asica]
Un proceso estoc\'astico $X=\left\{X\left(t\right):t\geq0\right\}$ es llamado regenerativo is existe una variable aleatoria $R_{1}>0$ tal que
\begin{itemize}
\item[i)] $\left\{X\left(t+R_{1}\right):t\geq0\right\}$ es independiente de $\left\{\left\{X\left(t\right):t<R_{1}\right\},\right\}$
\item[ii)] $\left\{X\left(t+R_{1}\right):t\geq0\right\}$ es estoc\'asticamente equivalente a $\left\{X\left(t\right):t>0\right\}$
\end{itemize}

Llamamos a $R_{1}$ tiempo de regeneraci\'on, y decimos que $X$ se regenera en este punto.
\end{Def}

$\left\{X\left(t+R_{1}\right)\right\}$ es regenerativo con tiempo de regeneraci\'on $R_{2}$, independiente de $R_{1}$ pero con la misma distribuci\'on que $R_{1}$. Procediendo de esta manera se obtiene una secuencia de variables aleatorias independientes e id\'enticamente distribuidas $\left\{R_{n}\right\}$ llamados longitudes de ciclo. Si definimos a $Z_{k}\equiv R_{1}+R_{2}+\cdots+R_{k}$, se tiene un proceso de renovaci\'on llamado proceso de renovaci\'on encajado para $X$.


\begin{Note}
La existencia de un primer tiempo de regeneraci\'on, $R_{1}$, implica la existencia de una sucesi\'on completa de estos tiempos $R_{1},R_{2}\ldots,$ que satisfacen la propiedad deseada \cite{Sigman2}.
\end{Note}


\begin{Note} Para la cola $GI/GI/1$ los usuarios arriban con tiempos $t_{n}$ y son atendidos con tiempos de servicio $S_{n}$, los tiempos de arribo forman un proceso de renovaci\'on  con tiempos entre arribos independientes e identicamente distribuidos (\texttt{i.i.d.})$T_{n}=t_{n}-t_{n-1}$, adem\'as los tiempos de servicio son \texttt{i.i.d.} e independientes de los procesos de arribo. Por \textit{estable} se entiende que $\esp S_{n}<\esp T_{n}<\infty$.
\end{Note}
 


\begin{Def}
Para $x$ fijo y para cada $t\geq0$, sea $I_{x}\left(t\right)=1$ si $X\left(t\right)\leq x$,  $I_{x}\left(t\right)=0$ en caso contrario, y def\'inanse los tiempos promedio
\begin{eqnarray*}
\overline{X}&=&lim_{t\rightarrow\infty}\frac{1}{t}\int_{0}^{\infty}X\left(u\right)du\\
\prob\left(X_{\infty}\leq x\right)&=&lim_{t\rightarrow\infty}\frac{1}{t}\int_{0}^{\infty}I_{x}\left(u\right)du,
\end{eqnarray*}
cuando estos l\'imites existan.
\end{Def}

Como consecuencia del teorema de Renovaci\'on-Recompensa, se tiene que el primer l\'imite  existe y es igual a la constante
\begin{eqnarray*}
\overline{X}&=&\frac{\esp\left[\int_{0}^{R_{1}}X\left(t\right)dt\right]}{\esp\left[R_{1}\right]},
\end{eqnarray*}
suponiendo que ambas esperanzas son finitas.
 
\begin{Note}
Funciones de procesos regenerativos son regenerativas, es decir, si $X\left(t\right)$ es regenerativo y se define el proceso $Y\left(t\right)$ por $Y\left(t\right)=f\left(X\left(t\right)\right)$ para alguna funci\'on Borel medible $f\left(\cdot\right)$. Adem\'as $Y$ es regenerativo con los mismos tiempos de renovaci\'on que $X$. 

En general, los tiempos de renovaci\'on, $Z_{k}$ de un proceso regenerativo no requieren ser tiempos de paro con respecto a la evoluci\'on de $X\left(t\right)$.
\end{Note} 

\begin{Note}
Una funci\'on de un proceso de Markov, usualmente no ser\'a un proceso de Markov, sin embargo ser\'a regenerativo si el proceso de Markov lo es.
\end{Note}

 
\begin{Note}
Un proceso regenerativo con media de la longitud de ciclo finita es llamado positivo recurrente.
\end{Note}


\begin{Note}
\begin{itemize}
\item[a)] Si el proceso regenerativo $X$ es positivo recurrente y tiene trayectorias muestrales no negativas, entonces la ecuaci\'on anterior es v\'alida.
\item[b)] Si $X$ es positivo recurrente regenerativo, podemos construir una \'unica versi\'on estacionaria de este proceso, $X_{e}=\left\{X_{e}\left(t\right)\right\}$, donde $X_{e}$ es un proceso estoc\'astico regenerativo y estrictamente estacionario, con distribuci\'on marginal distribuida como $X_{\infty}$
\end{itemize}
\end{Note}


%__________________________________________________________________________________________
%\subsection{Procesos Regenerativos Estacionarios - Stidham \cite{Stidham}}
%__________________________________________________________________________________________


Un proceso estoc\'astico a tiempo continuo $\left\{V\left(t\right),t\geq0\right\}$ es un proceso regenerativo si existe una sucesi\'on de variables aleatorias independientes e id\'enticamente distribuidas $\left\{X_{1},X_{2},\ldots\right\}$, sucesi\'on de renovaci\'on, tal que para cualquier conjunto de Borel $A$, 

\begin{eqnarray*}
\prob\left\{V\left(t\right)\in A|X_{1}+X_{2}+\cdots+X_{R\left(t\right)}=s,\left\{V\left(\tau\right),\tau<s\right\}\right\}=\prob\left\{V\left(t-s\right)\in A|X_{1}>t-s\right\},
\end{eqnarray*}
para todo $0\leq s\leq t$, donde $R\left(t\right)=\max\left\{X_{1}+X_{2}+\cdots+X_{j}\leq t\right\}=$n\'umero de renovaciones ({\emph{puntos de regeneraci\'on}}) que ocurren en $\left[0,t\right]$. El intervalo $\left[0,X_{1}\right)$ es llamado {\emph{primer ciclo de regeneraci\'on}} de $\left\{V\left(t \right),t\geq0\right\}$, $\left[X_{1},X_{1}+X_{2}\right)$ el {\emph{segundo ciclo de regeneraci\'on}}, y as\'i sucesivamente.

Sea $X=X_{1}$ y sea $F$ la funci\'on de distrbuci\'on de $X$


\begin{Def}
Se define el proceso estacionario, $\left\{V^{*}\left(t\right),t\geq0\right\}$, para $\left\{V\left(t\right),t\geq0\right\}$ por

\begin{eqnarray*}
\prob\left\{V\left(t\right)\in A\right\}=\frac{1}{\esp\left[X\right]}\int_{0}^{\infty}\prob\left\{V\left(t+x\right)\in A|X>x\right\}\left(1-F\left(x\right)\right)dx,
\end{eqnarray*} 
para todo $t\geq0$ y todo conjunto de Borel $A$.
\end{Def}

\begin{Def}
Una distribuci\'on se dice que es {\emph{aritm\'etica}} si todos sus puntos de incremento son m\'ultiplos de la forma $0,\lambda, 2\lambda,\ldots$ para alguna $\lambda>0$ entera.
\end{Def}


\begin{Def}
Una modificaci\'on medible de un proceso $\left\{V\left(t\right),t\geq0\right\}$, es una versi\'on de este, $\left\{V\left(t,w\right)\right\}$ conjuntamente medible para $t\geq0$ y para $w\in S$, $S$ espacio de estados para $\left\{V\left(t\right),t\geq0\right\}$.
\end{Def}

\begin{Teo}
Sea $\left\{V\left(t\right),t\geq\right\}$ un proceso regenerativo no negativo con modificaci\'on medible. Sea $\esp\left[X\right]<\infty$. Entonces el proceso estacionario dado por la ecuaci\'on anterior est\'a bien definido y tiene funci\'on de distribuci\'on independiente de $t$, adem\'as
\begin{itemize}
\item[i)] \begin{eqnarray*}
\esp\left[V^{*}\left(0\right)\right]&=&\frac{\esp\left[\int_{0}^{X}V\left(s\right)ds\right]}{\esp\left[X\right]}\end{eqnarray*}
\item[ii)] Si $\esp\left[V^{*}\left(0\right)\right]<\infty$, equivalentemente, si $\esp\left[\int_{0}^{X}V\left(s\right)ds\right]<\infty$,entonces
\begin{eqnarray*}
\frac{\int_{0}^{t}V\left(s\right)ds}{t}\rightarrow\frac{\esp\left[\int_{0}^{X}V\left(s\right)ds\right]}{\esp\left[X\right]}
\end{eqnarray*}
con probabilidad 1 y en media, cuando $t\rightarrow\infty$.
\end{itemize}
\end{Teo}

\begin{Coro}
Sea $\left\{V\left(t\right),t\geq0\right\}$ un proceso regenerativo no negativo, con modificaci\'on medible. Si $\esp <\infty$, $F$ es no-aritm\'etica, y para todo $x\geq0$, $P\left\{V\left(t\right)\leq x,C>x\right\}$ es de variaci\'on acotada como funci\'on de $t$ en cada intervalo finito $\left[0,\tau\right]$, entonces $V\left(t\right)$ converge en distribuci\'on  cuando $t\rightarrow\infty$ y $$\esp V=\frac{\esp \int_{0}^{X}V\left(s\right)ds}{\esp X}$$
Donde $V$ tiene la distribuci\'on l\'imite de $V\left(t\right)$ cuando $t\rightarrow\infty$.

\end{Coro}

Para el caso discreto se tienen resultados similares.



%______________________________________________________________________
%\subsection{Procesos de Renovaci\'on}
%______________________________________________________________________

\begin{Def}%\label{Def.Tn}
Sean $0\leq T_{1}\leq T_{2}\leq \ldots$ son tiempos aleatorios infinitos en los cuales ocurren ciertos eventos. El n\'umero de tiempos $T_{n}$ en el intervalo $\left[0,t\right)$ es

\begin{eqnarray}
N\left(t\right)=\sum_{n=1}^{\infty}\indora\left(T_{n}\leq t\right),
\end{eqnarray}
para $t\geq0$.
\end{Def}

Si se consideran los puntos $T_{n}$ como elementos de $\rea_{+}$, y $N\left(t\right)$ es el n\'umero de puntos en $\rea$. El proceso denotado por $\left\{N\left(t\right):t\geq0\right\}$, denotado por $N\left(t\right)$, es un proceso puntual en $\rea_{+}$. Los $T_{n}$ son los tiempos de ocurrencia, el proceso puntual $N\left(t\right)$ es simple si su n\'umero de ocurrencias son distintas: $0<T_{1}<T_{2}<\ldots$ casi seguramente.

\begin{Def}
Un proceso puntual $N\left(t\right)$ es un proceso de renovaci\'on si los tiempos de interocurrencia $\xi_{n}=T_{n}-T_{n-1}$, para $n\geq1$, son independientes e identicamente distribuidos con distribuci\'on $F$, donde $F\left(0\right)=0$ y $T_{0}=0$. Los $T_{n}$ son llamados tiempos de renovaci\'on, referente a la independencia o renovaci\'on de la informaci\'on estoc\'astica en estos tiempos. Los $\xi_{n}$ son los tiempos de inter-renovaci\'on, y $N\left(t\right)$ es el n\'umero de renovaciones en el intervalo $\left[0,t\right)$
\end{Def}


\begin{Note}
Para definir un proceso de renovaci\'on para cualquier contexto, solamente hay que especificar una distribuci\'on $F$, con $F\left(0\right)=0$, para los tiempos de inter-renovaci\'on. La funci\'on $F$ en turno degune las otra variables aleatorias. De manera formal, existe un espacio de probabilidad y una sucesi\'on de variables aleatorias $\xi_{1},\xi_{2},\ldots$ definidas en este con distribuci\'on $F$. Entonces las otras cantidades son $T_{n}=\sum_{k=1}^{n}\xi_{k}$ y $N\left(t\right)=\sum_{n=1}^{\infty}\indora\left(T_{n}\leq t\right)$, donde $T_{n}\rightarrow\infty$ casi seguramente por la Ley Fuerte de los Grandes Números.
\end{Note}

%___________________________________________________________________________________________
%
%\subsection{Teorema Principal de Renovaci\'on}
%___________________________________________________________________________________________
%

\begin{Note} Una funci\'on $h:\rea_{+}\rightarrow\rea$ es Directamente Riemann Integrable en los siguientes casos:
\begin{itemize}
\item[a)] $h\left(t\right)\geq0$ es decreciente y Riemann Integrable.
\item[b)] $h$ es continua excepto posiblemente en un conjunto de Lebesgue de medida 0, y $|h\left(t\right)|\leq b\left(t\right)$, donde $b$ es DRI.
\end{itemize}
\end{Note}

\begin{Teo}[Teorema Principal de Renovaci\'on]
Si $F$ es no aritm\'etica y $h\left(t\right)$ es Directamente Riemann Integrable (DRI), entonces

\begin{eqnarray*}
lim_{t\rightarrow\infty}U\star h=\frac{1}{\mu}\int_{\rea_{+}}h\left(s\right)ds.
\end{eqnarray*}
\end{Teo}

\begin{Prop}
Cualquier funci\'on $H\left(t\right)$ acotada en intervalos finitos y que es 0 para $t<0$ puede expresarse como
\begin{eqnarray*}
H\left(t\right)=U\star h\left(t\right)\textrm{,  donde }h\left(t\right)=H\left(t\right)-F\star H\left(t\right)
\end{eqnarray*}
\end{Prop}

\begin{Def}
Un proceso estoc\'astico $X\left(t\right)$ es crudamente regenerativo en un tiempo aleatorio positivo $T$ si
\begin{eqnarray*}
\esp\left[X\left(T+t\right)|T\right]=\esp\left[X\left(t\right)\right]\textrm{, para }t\geq0,\end{eqnarray*}
y con las esperanzas anteriores finitas.
\end{Def}

\begin{Prop}
Sup\'ongase que $X\left(t\right)$ es un proceso crudamente regenerativo en $T$, que tiene distribuci\'on $F$. Si $\esp\left[X\left(t\right)\right]$ es acotado en intervalos finitos, entonces
\begin{eqnarray*}
\esp\left[X\left(t\right)\right]=U\star h\left(t\right)\textrm{,  donde }h\left(t\right)=\esp\left[X\left(t\right)\indora\left(T>t\right)\right].
\end{eqnarray*}
\end{Prop}

\begin{Teo}[Regeneraci\'on Cruda]
Sup\'ongase que $X\left(t\right)$ es un proceso con valores positivo crudamente regenerativo en $T$, y def\'inase $M=\sup\left\{|X\left(t\right)|:t\leq T\right\}$. Si $T$ es no aritm\'etico y $M$ y $MT$ tienen media finita, entonces
\begin{eqnarray*}
lim_{t\rightarrow\infty}\esp\left[X\left(t\right)\right]=\frac{1}{\mu}\int_{\rea_{+}}h\left(s\right)ds,
\end{eqnarray*}
donde $h\left(t\right)=\esp\left[X\left(t\right)\indora\left(T>t\right)\right]$.
\end{Teo}

%___________________________________________________________________________________________
%
%\subsection{Propiedades de los Procesos de Renovaci\'on}
%___________________________________________________________________________________________
%

Los tiempos $T_{n}$ est\'an relacionados con los conteos de $N\left(t\right)$ por

\begin{eqnarray*}
\left\{N\left(t\right)\geq n\right\}&=&\left\{T_{n}\leq t\right\}\\
T_{N\left(t\right)}\leq &t&<T_{N\left(t\right)+1},
\end{eqnarray*}

adem\'as $N\left(T_{n}\right)=n$, y 

\begin{eqnarray*}
N\left(t\right)=\max\left\{n:T_{n}\leq t\right\}=\min\left\{n:T_{n+1}>t\right\}
\end{eqnarray*}

Por propiedades de la convoluci\'on se sabe que

\begin{eqnarray*}
P\left\{T_{n}\leq t\right\}=F^{n\star}\left(t\right)
\end{eqnarray*}
que es la $n$-\'esima convoluci\'on de $F$. Entonces 

\begin{eqnarray*}
\left\{N\left(t\right)\geq n\right\}&=&\left\{T_{n}\leq t\right\}\\
P\left\{N\left(t\right)\leq n\right\}&=&1-F^{\left(n+1\right)\star}\left(t\right)
\end{eqnarray*}

Adem\'as usando el hecho de que $\esp\left[N\left(t\right)\right]=\sum_{n=1}^{\infty}P\left\{N\left(t\right)\geq n\right\}$
se tiene que

\begin{eqnarray*}
\esp\left[N\left(t\right)\right]=\sum_{n=1}^{\infty}F^{n\star}\left(t\right)
\end{eqnarray*}

\begin{Prop}
Para cada $t\geq0$, la funci\'on generadora de momentos $\esp\left[e^{\alpha N\left(t\right)}\right]$ existe para alguna $\alpha$ en una vecindad del 0, y de aqu\'i que $\esp\left[N\left(t\right)^{m}\right]<\infty$, para $m\geq1$.
\end{Prop}


\begin{Note}
Si el primer tiempo de renovaci\'on $\xi_{1}$ no tiene la misma distribuci\'on que el resto de las $\xi_{n}$, para $n\geq2$, a $N\left(t\right)$ se le llama Proceso de Renovaci\'on retardado, donde si $\xi$ tiene distribuci\'on $G$, entonces el tiempo $T_{n}$ de la $n$-\'esima renovaci\'on tiene distribuci\'on $G\star F^{\left(n-1\right)\star}\left(t\right)$
\end{Note}


\begin{Teo}
Para una constante $\mu\leq\infty$ ( o variable aleatoria), las siguientes expresiones son equivalentes:

\begin{eqnarray}
lim_{n\rightarrow\infty}n^{-1}T_{n}&=&\mu,\textrm{ c.s.}\\
lim_{t\rightarrow\infty}t^{-1}N\left(t\right)&=&1/\mu,\textrm{ c.s.}
\end{eqnarray}
\end{Teo}


Es decir, $T_{n}$ satisface la Ley Fuerte de los Grandes N\'umeros s\'i y s\'olo s\'i $N\left/t\right)$ la cumple.


\begin{Coro}[Ley Fuerte de los Grandes N\'umeros para Procesos de Renovaci\'on]
Si $N\left(t\right)$ es un proceso de renovaci\'on cuyos tiempos de inter-renovaci\'on tienen media $\mu\leq\infty$, entonces
\begin{eqnarray}
t^{-1}N\left(t\right)\rightarrow 1/\mu,\textrm{ c.s. cuando }t\rightarrow\infty.
\end{eqnarray}

\end{Coro}


Considerar el proceso estoc\'astico de valores reales $\left\{Z\left(t\right):t\geq0\right\}$ en el mismo espacio de probabilidad que $N\left(t\right)$

\begin{Def}
Para el proceso $\left\{Z\left(t\right):t\geq0\right\}$ se define la fluctuaci\'on m\'axima de $Z\left(t\right)$ en el intervalo $\left(T_{n-1},T_{n}\right]$:
\begin{eqnarray*}
M_{n}=\sup_{T_{n-1}<t\leq T_{n}}|Z\left(t\right)-Z\left(T_{n-1}\right)|
\end{eqnarray*}
\end{Def}

\begin{Teo}
Sup\'ongase que $n^{-1}T_{n}\rightarrow\mu$ c.s. cuando $n\rightarrow\infty$, donde $\mu\leq\infty$ es una constante o variable aleatoria. Sea $a$ una constante o variable aleatoria que puede ser infinita cuando $\mu$ es finita, y considere las expresiones l\'imite:
\begin{eqnarray}
lim_{n\rightarrow\infty}n^{-1}Z\left(T_{n}\right)&=&a,\textrm{ c.s.}\\
lim_{t\rightarrow\infty}t^{-1}Z\left(t\right)&=&a/\mu,\textrm{ c.s.}
\end{eqnarray}
La segunda expresi\'on implica la primera. Conversamente, la primera implica la segunda si el proceso $Z\left(t\right)$ es creciente, o si $lim_{n\rightarrow\infty}n^{-1}M_{n}=0$ c.s.
\end{Teo}

\begin{Coro}
Si $N\left(t\right)$ es un proceso de renovaci\'on, y $\left(Z\left(T_{n}\right)-Z\left(T_{n-1}\right),M_{n}\right)$, para $n\geq1$, son variables aleatorias independientes e id\'enticamente distribuidas con media finita, entonces,
\begin{eqnarray}
lim_{t\rightarrow\infty}t^{-1}Z\left(t\right)\rightarrow\frac{\esp\left[Z\left(T_{1}\right)-Z\left(T_{0}\right)\right]}{\esp\left[T_{1}\right]},\textrm{ c.s. cuando  }t\rightarrow\infty.
\end{eqnarray}
\end{Coro}



%___________________________________________________________________________________________
%
%\subsection{Propiedades de los Procesos de Renovaci\'on}
%___________________________________________________________________________________________
%

Los tiempos $T_{n}$ est\'an relacionados con los conteos de $N\left(t\right)$ por

\begin{eqnarray*}
\left\{N\left(t\right)\geq n\right\}&=&\left\{T_{n}\leq t\right\}\\
T_{N\left(t\right)}\leq &t&<T_{N\left(t\right)+1},
\end{eqnarray*}

adem\'as $N\left(T_{n}\right)=n$, y 

\begin{eqnarray*}
N\left(t\right)=\max\left\{n:T_{n}\leq t\right\}=\min\left\{n:T_{n+1}>t\right\}
\end{eqnarray*}

Por propiedades de la convoluci\'on se sabe que

\begin{eqnarray*}
P\left\{T_{n}\leq t\right\}=F^{n\star}\left(t\right)
\end{eqnarray*}
que es la $n$-\'esima convoluci\'on de $F$. Entonces 

\begin{eqnarray*}
\left\{N\left(t\right)\geq n\right\}&=&\left\{T_{n}\leq t\right\}\\
P\left\{N\left(t\right)\leq n\right\}&=&1-F^{\left(n+1\right)\star}\left(t\right)
\end{eqnarray*}

Adem\'as usando el hecho de que $\esp\left[N\left(t\right)\right]=\sum_{n=1}^{\infty}P\left\{N\left(t\right)\geq n\right\}$
se tiene que

\begin{eqnarray*}
\esp\left[N\left(t\right)\right]=\sum_{n=1}^{\infty}F^{n\star}\left(t\right)
\end{eqnarray*}

\begin{Prop}
Para cada $t\geq0$, la funci\'on generadora de momentos $\esp\left[e^{\alpha N\left(t\right)}\right]$ existe para alguna $\alpha$ en una vecindad del 0, y de aqu\'i que $\esp\left[N\left(t\right)^{m}\right]<\infty$, para $m\geq1$.
\end{Prop}


\begin{Note}
Si el primer tiempo de renovaci\'on $\xi_{1}$ no tiene la misma distribuci\'on que el resto de las $\xi_{n}$, para $n\geq2$, a $N\left(t\right)$ se le llama Proceso de Renovaci\'on retardado, donde si $\xi$ tiene distribuci\'on $G$, entonces el tiempo $T_{n}$ de la $n$-\'esima renovaci\'on tiene distribuci\'on $G\star F^{\left(n-1\right)\star}\left(t\right)$
\end{Note}


\begin{Teo}
Para una constante $\mu\leq\infty$ ( o variable aleatoria), las siguientes expresiones son equivalentes:

\begin{eqnarray}
lim_{n\rightarrow\infty}n^{-1}T_{n}&=&\mu,\textrm{ c.s.}\\
lim_{t\rightarrow\infty}t^{-1}N\left(t\right)&=&1/\mu,\textrm{ c.s.}
\end{eqnarray}
\end{Teo}


Es decir, $T_{n}$ satisface la Ley Fuerte de los Grandes N\'umeros s\'i y s\'olo s\'i $N\left/t\right)$ la cumple.


\begin{Coro}[Ley Fuerte de los Grandes N\'umeros para Procesos de Renovaci\'on]
Si $N\left(t\right)$ es un proceso de renovaci\'on cuyos tiempos de inter-renovaci\'on tienen media $\mu\leq\infty$, entonces
\begin{eqnarray}
t^{-1}N\left(t\right)\rightarrow 1/\mu,\textrm{ c.s. cuando }t\rightarrow\infty.
\end{eqnarray}

\end{Coro}


Considerar el proceso estoc\'astico de valores reales $\left\{Z\left(t\right):t\geq0\right\}$ en el mismo espacio de probabilidad que $N\left(t\right)$

\begin{Def}
Para el proceso $\left\{Z\left(t\right):t\geq0\right\}$ se define la fluctuaci\'on m\'axima de $Z\left(t\right)$ en el intervalo $\left(T_{n-1},T_{n}\right]$:
\begin{eqnarray*}
M_{n}=\sup_{T_{n-1}<t\leq T_{n}}|Z\left(t\right)-Z\left(T_{n-1}\right)|
\end{eqnarray*}
\end{Def}

\begin{Teo}
Sup\'ongase que $n^{-1}T_{n}\rightarrow\mu$ c.s. cuando $n\rightarrow\infty$, donde $\mu\leq\infty$ es una constante o variable aleatoria. Sea $a$ una constante o variable aleatoria que puede ser infinita cuando $\mu$ es finita, y considere las expresiones l\'imite:
\begin{eqnarray}
lim_{n\rightarrow\infty}n^{-1}Z\left(T_{n}\right)&=&a,\textrm{ c.s.}\\
lim_{t\rightarrow\infty}t^{-1}Z\left(t\right)&=&a/\mu,\textrm{ c.s.}
\end{eqnarray}
La segunda expresi\'on implica la primera. Conversamente, la primera implica la segunda si el proceso $Z\left(t\right)$ es creciente, o si $lim_{n\rightarrow\infty}n^{-1}M_{n}=0$ c.s.
\end{Teo}

\begin{Coro}
Si $N\left(t\right)$ es un proceso de renovaci\'on, y $\left(Z\left(T_{n}\right)-Z\left(T_{n-1}\right),M_{n}\right)$, para $n\geq1$, son variables aleatorias independientes e id\'enticamente distribuidas con media finita, entonces,
\begin{eqnarray}
lim_{t\rightarrow\infty}t^{-1}Z\left(t\right)\rightarrow\frac{\esp\left[Z\left(T_{1}\right)-Z\left(T_{0}\right)\right]}{\esp\left[T_{1}\right]},\textrm{ c.s. cuando  }t\rightarrow\infty.
\end{eqnarray}
\end{Coro}


%___________________________________________________________________________________________
%
%\subsection{Propiedades de los Procesos de Renovaci\'on}
%___________________________________________________________________________________________
%

Los tiempos $T_{n}$ est\'an relacionados con los conteos de $N\left(t\right)$ por

\begin{eqnarray*}
\left\{N\left(t\right)\geq n\right\}&=&\left\{T_{n}\leq t\right\}\\
T_{N\left(t\right)}\leq &t&<T_{N\left(t\right)+1},
\end{eqnarray*}

adem\'as $N\left(T_{n}\right)=n$, y 

\begin{eqnarray*}
N\left(t\right)=\max\left\{n:T_{n}\leq t\right\}=\min\left\{n:T_{n+1}>t\right\}
\end{eqnarray*}

Por propiedades de la convoluci\'on se sabe que

\begin{eqnarray*}
P\left\{T_{n}\leq t\right\}=F^{n\star}\left(t\right)
\end{eqnarray*}
que es la $n$-\'esima convoluci\'on de $F$. Entonces 

\begin{eqnarray*}
\left\{N\left(t\right)\geq n\right\}&=&\left\{T_{n}\leq t\right\}\\
P\left\{N\left(t\right)\leq n\right\}&=&1-F^{\left(n+1\right)\star}\left(t\right)
\end{eqnarray*}

Adem\'as usando el hecho de que $\esp\left[N\left(t\right)\right]=\sum_{n=1}^{\infty}P\left\{N\left(t\right)\geq n\right\}$
se tiene que

\begin{eqnarray*}
\esp\left[N\left(t\right)\right]=\sum_{n=1}^{\infty}F^{n\star}\left(t\right)
\end{eqnarray*}

\begin{Prop}
Para cada $t\geq0$, la funci\'on generadora de momentos $\esp\left[e^{\alpha N\left(t\right)}\right]$ existe para alguna $\alpha$ en una vecindad del 0, y de aqu\'i que $\esp\left[N\left(t\right)^{m}\right]<\infty$, para $m\geq1$.
\end{Prop}


\begin{Note}
Si el primer tiempo de renovaci\'on $\xi_{1}$ no tiene la misma distribuci\'on que el resto de las $\xi_{n}$, para $n\geq2$, a $N\left(t\right)$ se le llama Proceso de Renovaci\'on retardado, donde si $\xi$ tiene distribuci\'on $G$, entonces el tiempo $T_{n}$ de la $n$-\'esima renovaci\'on tiene distribuci\'on $G\star F^{\left(n-1\right)\star}\left(t\right)$
\end{Note}


\begin{Teo}
Para una constante $\mu\leq\infty$ ( o variable aleatoria), las siguientes expresiones son equivalentes:

\begin{eqnarray}
lim_{n\rightarrow\infty}n^{-1}T_{n}&=&\mu,\textrm{ c.s.}\\
lim_{t\rightarrow\infty}t^{-1}N\left(t\right)&=&1/\mu,\textrm{ c.s.}
\end{eqnarray}
\end{Teo}


Es decir, $T_{n}$ satisface la Ley Fuerte de los Grandes N\'umeros s\'i y s\'olo s\'i $N\left/t\right)$ la cumple.


\begin{Coro}[Ley Fuerte de los Grandes N\'umeros para Procesos de Renovaci\'on]
Si $N\left(t\right)$ es un proceso de renovaci\'on cuyos tiempos de inter-renovaci\'on tienen media $\mu\leq\infty$, entonces
\begin{eqnarray}
t^{-1}N\left(t\right)\rightarrow 1/\mu,\textrm{ c.s. cuando }t\rightarrow\infty.
\end{eqnarray}

\end{Coro}


Considerar el proceso estoc\'astico de valores reales $\left\{Z\left(t\right):t\geq0\right\}$ en el mismo espacio de probabilidad que $N\left(t\right)$

\begin{Def}
Para el proceso $\left\{Z\left(t\right):t\geq0\right\}$ se define la fluctuaci\'on m\'axima de $Z\left(t\right)$ en el intervalo $\left(T_{n-1},T_{n}\right]$:
\begin{eqnarray*}
M_{n}=\sup_{T_{n-1}<t\leq T_{n}}|Z\left(t\right)-Z\left(T_{n-1}\right)|
\end{eqnarray*}
\end{Def}

\begin{Teo}
Sup\'ongase que $n^{-1}T_{n}\rightarrow\mu$ c.s. cuando $n\rightarrow\infty$, donde $\mu\leq\infty$ es una constante o variable aleatoria. Sea $a$ una constante o variable aleatoria que puede ser infinita cuando $\mu$ es finita, y considere las expresiones l\'imite:
\begin{eqnarray}
lim_{n\rightarrow\infty}n^{-1}Z\left(T_{n}\right)&=&a,\textrm{ c.s.}\\
lim_{t\rightarrow\infty}t^{-1}Z\left(t\right)&=&a/\mu,\textrm{ c.s.}
\end{eqnarray}
La segunda expresi\'on implica la primera. Conversamente, la primera implica la segunda si el proceso $Z\left(t\right)$ es creciente, o si $lim_{n\rightarrow\infty}n^{-1}M_{n}=0$ c.s.
\end{Teo}

\begin{Coro}
Si $N\left(t\right)$ es un proceso de renovaci\'on, y $\left(Z\left(T_{n}\right)-Z\left(T_{n-1}\right),M_{n}\right)$, para $n\geq1$, son variables aleatorias independientes e id\'enticamente distribuidas con media finita, entonces,
\begin{eqnarray}
lim_{t\rightarrow\infty}t^{-1}Z\left(t\right)\rightarrow\frac{\esp\left[Z\left(T_{1}\right)-Z\left(T_{0}\right)\right]}{\esp\left[T_{1}\right]},\textrm{ c.s. cuando  }t\rightarrow\infty.
\end{eqnarray}
\end{Coro}

%___________________________________________________________________________________________
%
%\subsection{Propiedades de los Procesos de Renovaci\'on}
%___________________________________________________________________________________________
%

Los tiempos $T_{n}$ est\'an relacionados con los conteos de $N\left(t\right)$ por

\begin{eqnarray*}
\left\{N\left(t\right)\geq n\right\}&=&\left\{T_{n}\leq t\right\}\\
T_{N\left(t\right)}\leq &t&<T_{N\left(t\right)+1},
\end{eqnarray*}

adem\'as $N\left(T_{n}\right)=n$, y 

\begin{eqnarray*}
N\left(t\right)=\max\left\{n:T_{n}\leq t\right\}=\min\left\{n:T_{n+1}>t\right\}
\end{eqnarray*}

Por propiedades de la convoluci\'on se sabe que

\begin{eqnarray*}
P\left\{T_{n}\leq t\right\}=F^{n\star}\left(t\right)
\end{eqnarray*}
que es la $n$-\'esima convoluci\'on de $F$. Entonces 

\begin{eqnarray*}
\left\{N\left(t\right)\geq n\right\}&=&\left\{T_{n}\leq t\right\}\\
P\left\{N\left(t\right)\leq n\right\}&=&1-F^{\left(n+1\right)\star}\left(t\right)
\end{eqnarray*}

Adem\'as usando el hecho de que $\esp\left[N\left(t\right)\right]=\sum_{n=1}^{\infty}P\left\{N\left(t\right)\geq n\right\}$
se tiene que

\begin{eqnarray*}
\esp\left[N\left(t\right)\right]=\sum_{n=1}^{\infty}F^{n\star}\left(t\right)
\end{eqnarray*}

\begin{Prop}
Para cada $t\geq0$, la funci\'on generadora de momentos $\esp\left[e^{\alpha N\left(t\right)}\right]$ existe para alguna $\alpha$ en una vecindad del 0, y de aqu\'i que $\esp\left[N\left(t\right)^{m}\right]<\infty$, para $m\geq1$.
\end{Prop}


\begin{Note}
Si el primer tiempo de renovaci\'on $\xi_{1}$ no tiene la misma distribuci\'on que el resto de las $\xi_{n}$, para $n\geq2$, a $N\left(t\right)$ se le llama Proceso de Renovaci\'on retardado, donde si $\xi$ tiene distribuci\'on $G$, entonces el tiempo $T_{n}$ de la $n$-\'esima renovaci\'on tiene distribuci\'on $G\star F^{\left(n-1\right)\star}\left(t\right)$
\end{Note}


\begin{Teo}
Para una constante $\mu\leq\infty$ ( o variable aleatoria), las siguientes expresiones son equivalentes:

\begin{eqnarray}
lim_{n\rightarrow\infty}n^{-1}T_{n}&=&\mu,\textrm{ c.s.}\\
lim_{t\rightarrow\infty}t^{-1}N\left(t\right)&=&1/\mu,\textrm{ c.s.}
\end{eqnarray}
\end{Teo}


Es decir, $T_{n}$ satisface la Ley Fuerte de los Grandes N\'umeros s\'i y s\'olo s\'i $N\left/t\right)$ la cumple.


\begin{Coro}[Ley Fuerte de los Grandes N\'umeros para Procesos de Renovaci\'on]
Si $N\left(t\right)$ es un proceso de renovaci\'on cuyos tiempos de inter-renovaci\'on tienen media $\mu\leq\infty$, entonces
\begin{eqnarray}
t^{-1}N\left(t\right)\rightarrow 1/\mu,\textrm{ c.s. cuando }t\rightarrow\infty.
\end{eqnarray}

\end{Coro}


Considerar el proceso estoc\'astico de valores reales $\left\{Z\left(t\right):t\geq0\right\}$ en el mismo espacio de probabilidad que $N\left(t\right)$

\begin{Def}
Para el proceso $\left\{Z\left(t\right):t\geq0\right\}$ se define la fluctuaci\'on m\'axima de $Z\left(t\right)$ en el intervalo $\left(T_{n-1},T_{n}\right]$:
\begin{eqnarray*}
M_{n}=\sup_{T_{n-1}<t\leq T_{n}}|Z\left(t\right)-Z\left(T_{n-1}\right)|
\end{eqnarray*}
\end{Def}

\begin{Teo}
Sup\'ongase que $n^{-1}T_{n}\rightarrow\mu$ c.s. cuando $n\rightarrow\infty$, donde $\mu\leq\infty$ es una constante o variable aleatoria. Sea $a$ una constante o variable aleatoria que puede ser infinita cuando $\mu$ es finita, y considere las expresiones l\'imite:
\begin{eqnarray}
lim_{n\rightarrow\infty}n^{-1}Z\left(T_{n}\right)&=&a,\textrm{ c.s.}\\
lim_{t\rightarrow\infty}t^{-1}Z\left(t\right)&=&a/\mu,\textrm{ c.s.}
\end{eqnarray}
La segunda expresi\'on implica la primera. Conversamente, la primera implica la segunda si el proceso $Z\left(t\right)$ es creciente, o si $lim_{n\rightarrow\infty}n^{-1}M_{n}=0$ c.s.
\end{Teo}

\begin{Coro}
Si $N\left(t\right)$ es un proceso de renovaci\'on, y $\left(Z\left(T_{n}\right)-Z\left(T_{n-1}\right),M_{n}\right)$, para $n\geq1$, son variables aleatorias independientes e id\'enticamente distribuidas con media finita, entonces,
\begin{eqnarray}
lim_{t\rightarrow\infty}t^{-1}Z\left(t\right)\rightarrow\frac{\esp\left[Z\left(T_{1}\right)-Z\left(T_{0}\right)\right]}{\esp\left[T_{1}\right]},\textrm{ c.s. cuando  }t\rightarrow\infty.
\end{eqnarray}
\end{Coro}
%___________________________________________________________________________________________
%
%\subsection{Propiedades de los Procesos de Renovaci\'on}
%___________________________________________________________________________________________
%

Los tiempos $T_{n}$ est\'an relacionados con los conteos de $N\left(t\right)$ por

\begin{eqnarray*}
\left\{N\left(t\right)\geq n\right\}&=&\left\{T_{n}\leq t\right\}\\
T_{N\left(t\right)}\leq &t&<T_{N\left(t\right)+1},
\end{eqnarray*}

adem\'as $N\left(T_{n}\right)=n$, y 

\begin{eqnarray*}
N\left(t\right)=\max\left\{n:T_{n}\leq t\right\}=\min\left\{n:T_{n+1}>t\right\}
\end{eqnarray*}

Por propiedades de la convoluci\'on se sabe que

\begin{eqnarray*}
P\left\{T_{n}\leq t\right\}=F^{n\star}\left(t\right)
\end{eqnarray*}
que es la $n$-\'esima convoluci\'on de $F$. Entonces 

\begin{eqnarray*}
\left\{N\left(t\right)\geq n\right\}&=&\left\{T_{n}\leq t\right\}\\
P\left\{N\left(t\right)\leq n\right\}&=&1-F^{\left(n+1\right)\star}\left(t\right)
\end{eqnarray*}

Adem\'as usando el hecho de que $\esp\left[N\left(t\right)\right]=\sum_{n=1}^{\infty}P\left\{N\left(t\right)\geq n\right\}$
se tiene que

\begin{eqnarray*}
\esp\left[N\left(t\right)\right]=\sum_{n=1}^{\infty}F^{n\star}\left(t\right)
\end{eqnarray*}

\begin{Prop}
Para cada $t\geq0$, la funci\'on generadora de momentos $\esp\left[e^{\alpha N\left(t\right)}\right]$ existe para alguna $\alpha$ en una vecindad del 0, y de aqu\'i que $\esp\left[N\left(t\right)^{m}\right]<\infty$, para $m\geq1$.
\end{Prop}


\begin{Note}
Si el primer tiempo de renovaci\'on $\xi_{1}$ no tiene la misma distribuci\'on que el resto de las $\xi_{n}$, para $n\geq2$, a $N\left(t\right)$ se le llama Proceso de Renovaci\'on retardado, donde si $\xi$ tiene distribuci\'on $G$, entonces el tiempo $T_{n}$ de la $n$-\'esima renovaci\'on tiene distribuci\'on $G\star F^{\left(n-1\right)\star}\left(t\right)$
\end{Note}


\begin{Teo}
Para una constante $\mu\leq\infty$ ( o variable aleatoria), las siguientes expresiones son equivalentes:

\begin{eqnarray}
lim_{n\rightarrow\infty}n^{-1}T_{n}&=&\mu,\textrm{ c.s.}\\
lim_{t\rightarrow\infty}t^{-1}N\left(t\right)&=&1/\mu,\textrm{ c.s.}
\end{eqnarray}
\end{Teo}


Es decir, $T_{n}$ satisface la Ley Fuerte de los Grandes N\'umeros s\'i y s\'olo s\'i $N\left/t\right)$ la cumple.


\begin{Coro}[Ley Fuerte de los Grandes N\'umeros para Procesos de Renovaci\'on]
Si $N\left(t\right)$ es un proceso de renovaci\'on cuyos tiempos de inter-renovaci\'on tienen media $\mu\leq\infty$, entonces
\begin{eqnarray}
t^{-1}N\left(t\right)\rightarrow 1/\mu,\textrm{ c.s. cuando }t\rightarrow\infty.
\end{eqnarray}

\end{Coro}


Considerar el proceso estoc\'astico de valores reales $\left\{Z\left(t\right):t\geq0\right\}$ en el mismo espacio de probabilidad que $N\left(t\right)$

\begin{Def}
Para el proceso $\left\{Z\left(t\right):t\geq0\right\}$ se define la fluctuaci\'on m\'axima de $Z\left(t\right)$ en el intervalo $\left(T_{n-1},T_{n}\right]$:
\begin{eqnarray*}
M_{n}=\sup_{T_{n-1}<t\leq T_{n}}|Z\left(t\right)-Z\left(T_{n-1}\right)|
\end{eqnarray*}
\end{Def}

\begin{Teo}
Sup\'ongase que $n^{-1}T_{n}\rightarrow\mu$ c.s. cuando $n\rightarrow\infty$, donde $\mu\leq\infty$ es una constante o variable aleatoria. Sea $a$ una constante o variable aleatoria que puede ser infinita cuando $\mu$ es finita, y considere las expresiones l\'imite:
\begin{eqnarray}
lim_{n\rightarrow\infty}n^{-1}Z\left(T_{n}\right)&=&a,\textrm{ c.s.}\\
lim_{t\rightarrow\infty}t^{-1}Z\left(t\right)&=&a/\mu,\textrm{ c.s.}
\end{eqnarray}
La segunda expresi\'on implica la primera. Conversamente, la primera implica la segunda si el proceso $Z\left(t\right)$ es creciente, o si $lim_{n\rightarrow\infty}n^{-1}M_{n}=0$ c.s.
\end{Teo}

\begin{Coro}
Si $N\left(t\right)$ es un proceso de renovaci\'on, y $\left(Z\left(T_{n}\right)-Z\left(T_{n-1}\right),M_{n}\right)$, para $n\geq1$, son variables aleatorias independientes e id\'enticamente distribuidas con media finita, entonces,
\begin{eqnarray}
lim_{t\rightarrow\infty}t^{-1}Z\left(t\right)\rightarrow\frac{\esp\left[Z\left(T_{1}\right)-Z\left(T_{0}\right)\right]}{\esp\left[T_{1}\right]},\textrm{ c.s. cuando  }t\rightarrow\infty.
\end{eqnarray}
\end{Coro}


%___________________________________________________________________________________________
%
%\subsection{Funci\'on de Renovaci\'on}
%___________________________________________________________________________________________
%


\begin{Def}
Sea $h\left(t\right)$ funci\'on de valores reales en $\rea$ acotada en intervalos finitos e igual a cero para $t<0$ La ecuaci\'on de renovaci\'on para $h\left(t\right)$ y la distribuci\'on $F$ es

\begin{eqnarray}%\label{Ec.Renovacion}
H\left(t\right)=h\left(t\right)+\int_{\left[0,t\right]}H\left(t-s\right)dF\left(s\right)\textrm{,    }t\geq0,
\end{eqnarray}
donde $H\left(t\right)$ es una funci\'on de valores reales. Esto es $H=h+F\star H$. Decimos que $H\left(t\right)$ es soluci\'on de esta ecuaci\'on si satisface la ecuaci\'on, y es acotada en intervalos finitos e iguales a cero para $t<0$.
\end{Def}

\begin{Prop}
La funci\'on $U\star h\left(t\right)$ es la \'unica soluci\'on de la ecuaci\'on de renovaci\'on (\ref{Ec.Renovacion}).
\end{Prop}

\begin{Teo}[Teorema Renovaci\'on Elemental]
\begin{eqnarray*}
t^{-1}U\left(t\right)\rightarrow 1/\mu\textrm{,    cuando }t\rightarrow\infty.
\end{eqnarray*}
\end{Teo}

%___________________________________________________________________________________________
%
%\subsection{Funci\'on de Renovaci\'on}
%___________________________________________________________________________________________
%


Sup\'ongase que $N\left(t\right)$ es un proceso de renovaci\'on con distribuci\'on $F$ con media finita $\mu$.

\begin{Def}
La funci\'on de renovaci\'on asociada con la distribuci\'on $F$, del proceso $N\left(t\right)$, es
\begin{eqnarray*}
U\left(t\right)=\sum_{n=1}^{\infty}F^{n\star}\left(t\right),\textrm{   }t\geq0,
\end{eqnarray*}
donde $F^{0\star}\left(t\right)=\indora\left(t\geq0\right)$.
\end{Def}


\begin{Prop}
Sup\'ongase que la distribuci\'on de inter-renovaci\'on $F$ tiene densidad $f$. Entonces $U\left(t\right)$ tambi\'en tiene densidad, para $t>0$, y es $U^{'}\left(t\right)=\sum_{n=0}^{\infty}f^{n\star}\left(t\right)$. Adem\'as
\begin{eqnarray*}
\prob\left\{N\left(t\right)>N\left(t-\right)\right\}=0\textrm{,   }t\geq0.
\end{eqnarray*}
\end{Prop}

\begin{Def}
La Transformada de Laplace-Stieljes de $F$ est\'a dada por

\begin{eqnarray*}
\hat{F}\left(\alpha\right)=\int_{\rea_{+}}e^{-\alpha t}dF\left(t\right)\textrm{,  }\alpha\geq0.
\end{eqnarray*}
\end{Def}

Entonces

\begin{eqnarray*}
\hat{U}\left(\alpha\right)=\sum_{n=0}^{\infty}\hat{F^{n\star}}\left(\alpha\right)=\sum_{n=0}^{\infty}\hat{F}\left(\alpha\right)^{n}=\frac{1}{1-\hat{F}\left(\alpha\right)}.
\end{eqnarray*}


\begin{Prop}
La Transformada de Laplace $\hat{U}\left(\alpha\right)$ y $\hat{F}\left(\alpha\right)$ determina una a la otra de manera \'unica por la relaci\'on $\hat{U}\left(\alpha\right)=\frac{1}{1-\hat{F}\left(\alpha\right)}$.
\end{Prop}


\begin{Note}
Un proceso de renovaci\'on $N\left(t\right)$ cuyos tiempos de inter-renovaci\'on tienen media finita, es un proceso Poisson con tasa $\lambda$ si y s\'olo s\'i $\esp\left[U\left(t\right)\right]=\lambda t$, para $t\geq0$.
\end{Note}


\begin{Teo}
Sea $N\left(t\right)$ un proceso puntual simple con puntos de localizaci\'on $T_{n}$ tal que $\eta\left(t\right)=\esp\left[N\left(\right)\right]$ es finita para cada $t$. Entonces para cualquier funci\'on $f:\rea_{+}\rightarrow\rea$,
\begin{eqnarray*}
\esp\left[\sum_{n=1}^{N\left(\right)}f\left(T_{n}\right)\right]=\int_{\left(0,t\right]}f\left(s\right)d\eta\left(s\right)\textrm{,  }t\geq0,
\end{eqnarray*}
suponiendo que la integral exista. Adem\'as si $X_{1},X_{2},\ldots$ son variables aleatorias definidas en el mismo espacio de probabilidad que el proceso $N\left(t\right)$ tal que $\esp\left[X_{n}|T_{n}=s\right]=f\left(s\right)$, independiente de $n$. Entonces
\begin{eqnarray*}
\esp\left[\sum_{n=1}^{N\left(t\right)}X_{n}\right]=\int_{\left(0,t\right]}f\left(s\right)d\eta\left(s\right)\textrm{,  }t\geq0,
\end{eqnarray*} 
suponiendo que la integral exista. 
\end{Teo}

\begin{Coro}[Identidad de Wald para Renovaciones]
Para el proceso de renovaci\'on $N\left(t\right)$,
\begin{eqnarray*}
\esp\left[T_{N\left(t\right)+1}\right]=\mu\esp\left[N\left(t\right)+1\right]\textrm{,  }t\geq0,
\end{eqnarray*}  
\end{Coro}

%______________________________________________________________________
%\subsection{Procesos de Renovaci\'on}
%______________________________________________________________________

\begin{Def}%\label{Def.Tn}
Sean $0\leq T_{1}\leq T_{2}\leq \ldots$ son tiempos aleatorios infinitos en los cuales ocurren ciertos eventos. El n\'umero de tiempos $T_{n}$ en el intervalo $\left[0,t\right)$ es

\begin{eqnarray}
N\left(t\right)=\sum_{n=1}^{\infty}\indora\left(T_{n}\leq t\right),
\end{eqnarray}
para $t\geq0$.
\end{Def}

Si se consideran los puntos $T_{n}$ como elementos de $\rea_{+}$, y $N\left(t\right)$ es el n\'umero de puntos en $\rea$. El proceso denotado por $\left\{N\left(t\right):t\geq0\right\}$, denotado por $N\left(t\right)$, es un proceso puntual en $\rea_{+}$. Los $T_{n}$ son los tiempos de ocurrencia, el proceso puntual $N\left(t\right)$ es simple si su n\'umero de ocurrencias son distintas: $0<T_{1}<T_{2}<\ldots$ casi seguramente.

\begin{Def}
Un proceso puntual $N\left(t\right)$ es un proceso de renovaci\'on si los tiempos de interocurrencia $\xi_{n}=T_{n}-T_{n-1}$, para $n\geq1$, son independientes e identicamente distribuidos con distribuci\'on $F$, donde $F\left(0\right)=0$ y $T_{0}=0$. Los $T_{n}$ son llamados tiempos de renovaci\'on, referente a la independencia o renovaci\'on de la informaci\'on estoc\'astica en estos tiempos. Los $\xi_{n}$ son los tiempos de inter-renovaci\'on, y $N\left(t\right)$ es el n\'umero de renovaciones en el intervalo $\left[0,t\right)$
\end{Def}


\begin{Note}
Para definir un proceso de renovaci\'on para cualquier contexto, solamente hay que especificar una distribuci\'on $F$, con $F\left(0\right)=0$, para los tiempos de inter-renovaci\'on. La funci\'on $F$ en turno degune las otra variables aleatorias. De manera formal, existe un espacio de probabilidad y una sucesi\'on de variables aleatorias $\xi_{1},\xi_{2},\ldots$ definidas en este con distribuci\'on $F$. Entonces las otras cantidades son $T_{n}=\sum_{k=1}^{n}\xi_{k}$ y $N\left(t\right)=\sum_{n=1}^{\infty}\indora\left(T_{n}\leq t\right)$, donde $T_{n}\rightarrow\infty$ casi seguramente por la Ley Fuerte de los Grandes Números.
\end{Note}

%___________________________________________________________________________________________
%
%\subsection{Renewal and Regenerative Processes: Serfozo\cite{Serfozo}}
%___________________________________________________________________________________________
%
\begin{Def}%\label{Def.Tn}
Sean $0\leq T_{1}\leq T_{2}\leq \ldots$ son tiempos aleatorios infinitos en los cuales ocurren ciertos eventos. El n\'umero de tiempos $T_{n}$ en el intervalo $\left[0,t\right)$ es

\begin{eqnarray}
N\left(t\right)=\sum_{n=1}^{\infty}\indora\left(T_{n}\leq t\right),
\end{eqnarray}
para $t\geq0$.
\end{Def}

Si se consideran los puntos $T_{n}$ como elementos de $\rea_{+}$, y $N\left(t\right)$ es el n\'umero de puntos en $\rea$. El proceso denotado por $\left\{N\left(t\right):t\geq0\right\}$, denotado por $N\left(t\right)$, es un proceso puntual en $\rea_{+}$. Los $T_{n}$ son los tiempos de ocurrencia, el proceso puntual $N\left(t\right)$ es simple si su n\'umero de ocurrencias son distintas: $0<T_{1}<T_{2}<\ldots$ casi seguramente.

\begin{Def}
Un proceso puntual $N\left(t\right)$ es un proceso de renovaci\'on si los tiempos de interocurrencia $\xi_{n}=T_{n}-T_{n-1}$, para $n\geq1$, son independientes e identicamente distribuidos con distribuci\'on $F$, donde $F\left(0\right)=0$ y $T_{0}=0$. Los $T_{n}$ son llamados tiempos de renovaci\'on, referente a la independencia o renovaci\'on de la informaci\'on estoc\'astica en estos tiempos. Los $\xi_{n}$ son los tiempos de inter-renovaci\'on, y $N\left(t\right)$ es el n\'umero de renovaciones en el intervalo $\left[0,t\right)$
\end{Def}


\begin{Note}
Para definir un proceso de renovaci\'on para cualquier contexto, solamente hay que especificar una distribuci\'on $F$, con $F\left(0\right)=0$, para los tiempos de inter-renovaci\'on. La funci\'on $F$ en turno degune las otra variables aleatorias. De manera formal, existe un espacio de probabilidad y una sucesi\'on de variables aleatorias $\xi_{1},\xi_{2},\ldots$ definidas en este con distribuci\'on $F$. Entonces las otras cantidades son $T_{n}=\sum_{k=1}^{n}\xi_{k}$ y $N\left(t\right)=\sum_{n=1}^{\infty}\indora\left(T_{n}\leq t\right)$, donde $T_{n}\rightarrow\infty$ casi seguramente por la Ley Fuerte de los Grandes N\'umeros.
\end{Note}







Los tiempos $T_{n}$ est\'an relacionados con los conteos de $N\left(t\right)$ por

\begin{eqnarray*}
\left\{N\left(t\right)\geq n\right\}&=&\left\{T_{n}\leq t\right\}\\
T_{N\left(t\right)}\leq &t&<T_{N\left(t\right)+1},
\end{eqnarray*}

adem\'as $N\left(T_{n}\right)=n$, y 

\begin{eqnarray*}
N\left(t\right)=\max\left\{n:T_{n}\leq t\right\}=\min\left\{n:T_{n+1}>t\right\}
\end{eqnarray*}

Por propiedades de la convoluci\'on se sabe que

\begin{eqnarray*}
P\left\{T_{n}\leq t\right\}=F^{n\star}\left(t\right)
\end{eqnarray*}
que es la $n$-\'esima convoluci\'on de $F$. Entonces 

\begin{eqnarray*}
\left\{N\left(t\right)\geq n\right\}&=&\left\{T_{n}\leq t\right\}\\
P\left\{N\left(t\right)\leq n\right\}&=&1-F^{\left(n+1\right)\star}\left(t\right)
\end{eqnarray*}

Adem\'as usando el hecho de que $\esp\left[N\left(t\right)\right]=\sum_{n=1}^{\infty}P\left\{N\left(t\right)\geq n\right\}$
se tiene que

\begin{eqnarray*}
\esp\left[N\left(t\right)\right]=\sum_{n=1}^{\infty}F^{n\star}\left(t\right)
\end{eqnarray*}

\begin{Prop}
Para cada $t\geq0$, la funci\'on generadora de momentos $\esp\left[e^{\alpha N\left(t\right)}\right]$ existe para alguna $\alpha$ en una vecindad del 0, y de aqu\'i que $\esp\left[N\left(t\right)^{m}\right]<\infty$, para $m\geq1$.
\end{Prop}

\begin{Ejem}[\textbf{Proceso Poisson}]

Suponga que se tienen tiempos de inter-renovaci\'on \textit{i.i.d.} del proceso de renovaci\'on $N\left(t\right)$ tienen distribuci\'on exponencial $F\left(t\right)=q-e^{-\lambda t}$ con tasa $\lambda$. Entonces $N\left(t\right)$ es un proceso Poisson con tasa $\lambda$.

\end{Ejem}


\begin{Note}
Si el primer tiempo de renovaci\'on $\xi_{1}$ no tiene la misma distribuci\'on que el resto de las $\xi_{n}$, para $n\geq2$, a $N\left(t\right)$ se le llama Proceso de Renovaci\'on retardado, donde si $\xi$ tiene distribuci\'on $G$, entonces el tiempo $T_{n}$ de la $n$-\'esima renovaci\'on tiene distribuci\'on $G\star F^{\left(n-1\right)\star}\left(t\right)$
\end{Note}


\begin{Teo}
Para una constante $\mu\leq\infty$ ( o variable aleatoria), las siguientes expresiones son equivalentes:

\begin{eqnarray}
lim_{n\rightarrow\infty}n^{-1}T_{n}&=&\mu,\textrm{ c.s.}\\
lim_{t\rightarrow\infty}t^{-1}N\left(t\right)&=&1/\mu,\textrm{ c.s.}
\end{eqnarray}
\end{Teo}


Es decir, $T_{n}$ satisface la Ley Fuerte de los Grandes N\'umeros s\'i y s\'olo s\'i $N\left/t\right)$ la cumple.


\begin{Coro}[Ley Fuerte de los Grandes N\'umeros para Procesos de Renovaci\'on]
Si $N\left(t\right)$ es un proceso de renovaci\'on cuyos tiempos de inter-renovaci\'on tienen media $\mu\leq\infty$, entonces
\begin{eqnarray}
t^{-1}N\left(t\right)\rightarrow 1/\mu,\textrm{ c.s. cuando }t\rightarrow\infty.
\end{eqnarray}

\end{Coro}


Considerar el proceso estoc\'astico de valores reales $\left\{Z\left(t\right):t\geq0\right\}$ en el mismo espacio de probabilidad que $N\left(t\right)$

\begin{Def}
Para el proceso $\left\{Z\left(t\right):t\geq0\right\}$ se define la fluctuaci\'on m\'axima de $Z\left(t\right)$ en el intervalo $\left(T_{n-1},T_{n}\right]$:
\begin{eqnarray*}
M_{n}=\sup_{T_{n-1}<t\leq T_{n}}|Z\left(t\right)-Z\left(T_{n-1}\right)|
\end{eqnarray*}
\end{Def}

\begin{Teo}
Sup\'ongase que $n^{-1}T_{n}\rightarrow\mu$ c.s. cuando $n\rightarrow\infty$, donde $\mu\leq\infty$ es una constante o variable aleatoria. Sea $a$ una constante o variable aleatoria que puede ser infinita cuando $\mu$ es finita, y considere las expresiones l\'imite:
\begin{eqnarray}
lim_{n\rightarrow\infty}n^{-1}Z\left(T_{n}\right)&=&a,\textrm{ c.s.}\\
lim_{t\rightarrow\infty}t^{-1}Z\left(t\right)&=&a/\mu,\textrm{ c.s.}
\end{eqnarray}
La segunda expresi\'on implica la primera. Conversamente, la primera implica la segunda si el proceso $Z\left(t\right)$ es creciente, o si $lim_{n\rightarrow\infty}n^{-1}M_{n}=0$ c.s.
\end{Teo}

\begin{Coro}
Si $N\left(t\right)$ es un proceso de renovaci\'on, y $\left(Z\left(T_{n}\right)-Z\left(T_{n-1}\right),M_{n}\right)$, para $n\geq1$, son variables aleatorias independientes e id\'enticamente distribuidas con media finita, entonces,
\begin{eqnarray}
lim_{t\rightarrow\infty}t^{-1}Z\left(t\right)\rightarrow\frac{\esp\left[Z\left(T_{1}\right)-Z\left(T_{0}\right)\right]}{\esp\left[T_{1}\right]},\textrm{ c.s. cuando  }t\rightarrow\infty.
\end{eqnarray}
\end{Coro}


Sup\'ongase que $N\left(t\right)$ es un proceso de renovaci\'on con distribuci\'on $F$ con media finita $\mu$.

\begin{Def}
La funci\'on de renovaci\'on asociada con la distribuci\'on $F$, del proceso $N\left(t\right)$, es
\begin{eqnarray*}
U\left(t\right)=\sum_{n=1}^{\infty}F^{n\star}\left(t\right),\textrm{   }t\geq0,
\end{eqnarray*}
donde $F^{0\star}\left(t\right)=\indora\left(t\geq0\right)$.
\end{Def}


\begin{Prop}
Sup\'ongase que la distribuci\'on de inter-renovaci\'on $F$ tiene densidad $f$. Entonces $U\left(t\right)$ tambi\'en tiene densidad, para $t>0$, y es $U^{'}\left(t\right)=\sum_{n=0}^{\infty}f^{n\star}\left(t\right)$. Adem\'as
\begin{eqnarray*}
\prob\left\{N\left(t\right)>N\left(t-\right)\right\}=0\textrm{,   }t\geq0.
\end{eqnarray*}
\end{Prop}

\begin{Def}
La Transformada de Laplace-Stieljes de $F$ est\'a dada por

\begin{eqnarray*}
\hat{F}\left(\alpha\right)=\int_{\rea_{+}}e^{-\alpha t}dF\left(t\right)\textrm{,  }\alpha\geq0.
\end{eqnarray*}
\end{Def}

Entonces

\begin{eqnarray*}
\hat{U}\left(\alpha\right)=\sum_{n=0}^{\infty}\hat{F^{n\star}}\left(\alpha\right)=\sum_{n=0}^{\infty}\hat{F}\left(\alpha\right)^{n}=\frac{1}{1-\hat{F}\left(\alpha\right)}.
\end{eqnarray*}


\begin{Prop}
La Transformada de Laplace $\hat{U}\left(\alpha\right)$ y $\hat{F}\left(\alpha\right)$ determina una a la otra de manera \'unica por la relaci\'on $\hat{U}\left(\alpha\right)=\frac{1}{1-\hat{F}\left(\alpha\right)}$.
\end{Prop}


\begin{Note}
Un proceso de renovaci\'on $N\left(t\right)$ cuyos tiempos de inter-renovaci\'on tienen media finita, es un proceso Poisson con tasa $\lambda$ si y s\'olo s\'i $\esp\left[U\left(t\right)\right]=\lambda t$, para $t\geq0$.
\end{Note}


\begin{Teo}
Sea $N\left(t\right)$ un proceso puntual simple con puntos de localizaci\'on $T_{n}$ tal que $\eta\left(t\right)=\esp\left[N\left(\right)\right]$ es finita para cada $t$. Entonces para cualquier funci\'on $f:\rea_{+}\rightarrow\rea$,
\begin{eqnarray*}
\esp\left[\sum_{n=1}^{N\left(\right)}f\left(T_{n}\right)\right]=\int_{\left(0,t\right]}f\left(s\right)d\eta\left(s\right)\textrm{,  }t\geq0,
\end{eqnarray*}
suponiendo que la integral exista. Adem\'as si $X_{1},X_{2},\ldots$ son variables aleatorias definidas en el mismo espacio de probabilidad que el proceso $N\left(t\right)$ tal que $\esp\left[X_{n}|T_{n}=s\right]=f\left(s\right)$, independiente de $n$. Entonces
\begin{eqnarray*}
\esp\left[\sum_{n=1}^{N\left(t\right)}X_{n}\right]=\int_{\left(0,t\right]}f\left(s\right)d\eta\left(s\right)\textrm{,  }t\geq0,
\end{eqnarray*} 
suponiendo que la integral exista. 
\end{Teo}

\begin{Coro}[Identidad de Wald para Renovaciones]
Para el proceso de renovaci\'on $N\left(t\right)$,
\begin{eqnarray*}
\esp\left[T_{N\left(t\right)+1}\right]=\mu\esp\left[N\left(t\right)+1\right]\textrm{,  }t\geq0,
\end{eqnarray*}  
\end{Coro}


\begin{Def}
Sea $h\left(t\right)$ funci\'on de valores reales en $\rea$ acotada en intervalos finitos e igual a cero para $t<0$ La ecuaci\'on de renovaci\'on para $h\left(t\right)$ y la distribuci\'on $F$ es

\begin{eqnarray}%\label{Ec.Renovacion}
H\left(t\right)=h\left(t\right)+\int_{\left[0,t\right]}H\left(t-s\right)dF\left(s\right)\textrm{,    }t\geq0,
\end{eqnarray}
donde $H\left(t\right)$ es una funci\'on de valores reales. Esto es $H=h+F\star H$. Decimos que $H\left(t\right)$ es soluci\'on de esta ecuaci\'on si satisface la ecuaci\'on, y es acotada en intervalos finitos e iguales a cero para $t<0$.
\end{Def}

\begin{Prop}
La funci\'on $U\star h\left(t\right)$ es la \'unica soluci\'on de la ecuaci\'on de renovaci\'on (\ref{Ec.Renovacion}).
\end{Prop}

\begin{Teo}[Teorema Renovaci\'on Elemental]
\begin{eqnarray*}
t^{-1}U\left(t\right)\rightarrow 1/\mu\textrm{,    cuando }t\rightarrow\infty.
\end{eqnarray*}
\end{Teo}



Sup\'ongase que $N\left(t\right)$ es un proceso de renovaci\'on con distribuci\'on $F$ con media finita $\mu$.

\begin{Def}
La funci\'on de renovaci\'on asociada con la distribuci\'on $F$, del proceso $N\left(t\right)$, es
\begin{eqnarray*}
U\left(t\right)=\sum_{n=1}^{\infty}F^{n\star}\left(t\right),\textrm{   }t\geq0,
\end{eqnarray*}
donde $F^{0\star}\left(t\right)=\indora\left(t\geq0\right)$.
\end{Def}


\begin{Prop}
Sup\'ongase que la distribuci\'on de inter-renovaci\'on $F$ tiene densidad $f$. Entonces $U\left(t\right)$ tambi\'en tiene densidad, para $t>0$, y es $U^{'}\left(t\right)=\sum_{n=0}^{\infty}f^{n\star}\left(t\right)$. Adem\'as
\begin{eqnarray*}
\prob\left\{N\left(t\right)>N\left(t-\right)\right\}=0\textrm{,   }t\geq0.
\end{eqnarray*}
\end{Prop}

\begin{Def}
La Transformada de Laplace-Stieljes de $F$ est\'a dada por

\begin{eqnarray*}
\hat{F}\left(\alpha\right)=\int_{\rea_{+}}e^{-\alpha t}dF\left(t\right)\textrm{,  }\alpha\geq0.
\end{eqnarray*}
\end{Def}

Entonces

\begin{eqnarray*}
\hat{U}\left(\alpha\right)=\sum_{n=0}^{\infty}\hat{F^{n\star}}\left(\alpha\right)=\sum_{n=0}^{\infty}\hat{F}\left(\alpha\right)^{n}=\frac{1}{1-\hat{F}\left(\alpha\right)}.
\end{eqnarray*}


\begin{Prop}
La Transformada de Laplace $\hat{U}\left(\alpha\right)$ y $\hat{F}\left(\alpha\right)$ determina una a la otra de manera \'unica por la relaci\'on $\hat{U}\left(\alpha\right)=\frac{1}{1-\hat{F}\left(\alpha\right)}$.
\end{Prop}


\begin{Note}
Un proceso de renovaci\'on $N\left(t\right)$ cuyos tiempos de inter-renovaci\'on tienen media finita, es un proceso Poisson con tasa $\lambda$ si y s\'olo s\'i $\esp\left[U\left(t\right)\right]=\lambda t$, para $t\geq0$.
\end{Note}


\begin{Teo}
Sea $N\left(t\right)$ un proceso puntual simple con puntos de localizaci\'on $T_{n}$ tal que $\eta\left(t\right)=\esp\left[N\left(\right)\right]$ es finita para cada $t$. Entonces para cualquier funci\'on $f:\rea_{+}\rightarrow\rea$,
\begin{eqnarray*}
\esp\left[\sum_{n=1}^{N\left(\right)}f\left(T_{n}\right)\right]=\int_{\left(0,t\right]}f\left(s\right)d\eta\left(s\right)\textrm{,  }t\geq0,
\end{eqnarray*}
suponiendo que la integral exista. Adem\'as si $X_{1},X_{2},\ldots$ son variables aleatorias definidas en el mismo espacio de probabilidad que el proceso $N\left(t\right)$ tal que $\esp\left[X_{n}|T_{n}=s\right]=f\left(s\right)$, independiente de $n$. Entonces
\begin{eqnarray*}
\esp\left[\sum_{n=1}^{N\left(t\right)}X_{n}\right]=\int_{\left(0,t\right]}f\left(s\right)d\eta\left(s\right)\textrm{,  }t\geq0,
\end{eqnarray*} 
suponiendo que la integral exista. 
\end{Teo}

\begin{Coro}[Identidad de Wald para Renovaciones]
Para el proceso de renovaci\'on $N\left(t\right)$,
\begin{eqnarray*}
\esp\left[T_{N\left(t\right)+1}\right]=\mu\esp\left[N\left(t\right)+1\right]\textrm{,  }t\geq0,
\end{eqnarray*}  
\end{Coro}


\begin{Def}
Sea $h\left(t\right)$ funci\'on de valores reales en $\rea$ acotada en intervalos finitos e igual a cero para $t<0$ La ecuaci\'on de renovaci\'on para $h\left(t\right)$ y la distribuci\'on $F$ es

\begin{eqnarray}%\label{Ec.Renovacion}
H\left(t\right)=h\left(t\right)+\int_{\left[0,t\right]}H\left(t-s\right)dF\left(s\right)\textrm{,    }t\geq0,
\end{eqnarray}
donde $H\left(t\right)$ es una funci\'on de valores reales. Esto es $H=h+F\star H$. Decimos que $H\left(t\right)$ es soluci\'on de esta ecuaci\'on si satisface la ecuaci\'on, y es acotada en intervalos finitos e iguales a cero para $t<0$.
\end{Def}

\begin{Prop}
La funci\'on $U\star h\left(t\right)$ es la \'unica soluci\'on de la ecuaci\'on de renovaci\'on (\ref{Ec.Renovacion}).
\end{Prop}

\begin{Teo}[Teorema Renovaci\'on Elemental]
\begin{eqnarray*}
t^{-1}U\left(t\right)\rightarrow 1/\mu\textrm{,    cuando }t\rightarrow\infty.
\end{eqnarray*}
\end{Teo}


\begin{Note} Una funci\'on $h:\rea_{+}\rightarrow\rea$ es Directamente Riemann Integrable en los siguientes casos:
\begin{itemize}
\item[a)] $h\left(t\right)\geq0$ es decreciente y Riemann Integrable.
\item[b)] $h$ es continua excepto posiblemente en un conjunto de Lebesgue de medida 0, y $|h\left(t\right)|\leq b\left(t\right)$, donde $b$ es DRI.
\end{itemize}
\end{Note}

\begin{Teo}[Teorema Principal de Renovaci\'on]
Si $F$ es no aritm\'etica y $h\left(t\right)$ es Directamente Riemann Integrable (DRI), entonces

\begin{eqnarray*}
lim_{t\rightarrow\infty}U\star h=\frac{1}{\mu}\int_{\rea_{+}}h\left(s\right)ds.
\end{eqnarray*}
\end{Teo}

\begin{Prop}
Cualquier funci\'on $H\left(t\right)$ acotada en intervalos finitos y que es 0 para $t<0$ puede expresarse como
\begin{eqnarray*}
H\left(t\right)=U\star h\left(t\right)\textrm{,  donde }h\left(t\right)=H\left(t\right)-F\star H\left(t\right)
\end{eqnarray*}
\end{Prop}

\begin{Def}
Un proceso estoc\'astico $X\left(t\right)$ es crudamente regenerativo en un tiempo aleatorio positivo $T$ si
\begin{eqnarray*}
\esp\left[X\left(T+t\right)|T\right]=\esp\left[X\left(t\right)\right]\textrm{, para }t\geq0,\end{eqnarray*}
y con las esperanzas anteriores finitas.
\end{Def}

\begin{Prop}
Sup\'ongase que $X\left(t\right)$ es un proceso crudamente regenerativo en $T$, que tiene distribuci\'on $F$. Si $\esp\left[X\left(t\right)\right]$ es acotado en intervalos finitos, entonces
\begin{eqnarray*}
\esp\left[X\left(t\right)\right]=U\star h\left(t\right)\textrm{,  donde }h\left(t\right)=\esp\left[X\left(t\right)\indora\left(T>t\right)\right].
\end{eqnarray*}
\end{Prop}

\begin{Teo}[Regeneraci\'on Cruda]
Sup\'ongase que $X\left(t\right)$ es un proceso con valores positivo crudamente regenerativo en $T$, y def\'inase $M=\sup\left\{|X\left(t\right)|:t\leq T\right\}$. Si $T$ es no aritm\'etico y $M$ y $MT$ tienen media finita, entonces
\begin{eqnarray*}
lim_{t\rightarrow\infty}\esp\left[X\left(t\right)\right]=\frac{1}{\mu}\int_{\rea_{+}}h\left(s\right)ds,
\end{eqnarray*}
donde $h\left(t\right)=\esp\left[X\left(t\right)\indora\left(T>t\right)\right]$.
\end{Teo}


\begin{Note} Una funci\'on $h:\rea_{+}\rightarrow\rea$ es Directamente Riemann Integrable en los siguientes casos:
\begin{itemize}
\item[a)] $h\left(t\right)\geq0$ es decreciente y Riemann Integrable.
\item[b)] $h$ es continua excepto posiblemente en un conjunto de Lebesgue de medida 0, y $|h\left(t\right)|\leq b\left(t\right)$, donde $b$ es DRI.
\end{itemize}
\end{Note}

\begin{Teo}[Teorema Principal de Renovaci\'on]
Si $F$ es no aritm\'etica y $h\left(t\right)$ es Directamente Riemann Integrable (DRI), entonces

\begin{eqnarray*}
lim_{t\rightarrow\infty}U\star h=\frac{1}{\mu}\int_{\rea_{+}}h\left(s\right)ds.
\end{eqnarray*}
\end{Teo}

\begin{Prop}
Cualquier funci\'on $H\left(t\right)$ acotada en intervalos finitos y que es 0 para $t<0$ puede expresarse como
\begin{eqnarray*}
H\left(t\right)=U\star h\left(t\right)\textrm{,  donde }h\left(t\right)=H\left(t\right)-F\star H\left(t\right)
\end{eqnarray*}
\end{Prop}

\begin{Def}
Un proceso estoc\'astico $X\left(t\right)$ es crudamente regenerativo en un tiempo aleatorio positivo $T$ si
\begin{eqnarray*}
\esp\left[X\left(T+t\right)|T\right]=\esp\left[X\left(t\right)\right]\textrm{, para }t\geq0,\end{eqnarray*}
y con las esperanzas anteriores finitas.
\end{Def}

\begin{Prop}
Sup\'ongase que $X\left(t\right)$ es un proceso crudamente regenerativo en $T$, que tiene distribuci\'on $F$. Si $\esp\left[X\left(t\right)\right]$ es acotado en intervalos finitos, entonces
\begin{eqnarray*}
\esp\left[X\left(t\right)\right]=U\star h\left(t\right)\textrm{,  donde }h\left(t\right)=\esp\left[X\left(t\right)\indora\left(T>t\right)\right].
\end{eqnarray*}
\end{Prop}

\begin{Teo}[Regeneraci\'on Cruda]
Sup\'ongase que $X\left(t\right)$ es un proceso con valores positivo crudamente regenerativo en $T$, y def\'inase $M=\sup\left\{|X\left(t\right)|:t\leq T\right\}$. Si $T$ es no aritm\'etico y $M$ y $MT$ tienen media finita, entonces
\begin{eqnarray*}
lim_{t\rightarrow\infty}\esp\left[X\left(t\right)\right]=\frac{1}{\mu}\int_{\rea_{+}}h\left(s\right)ds,
\end{eqnarray*}
donde $h\left(t\right)=\esp\left[X\left(t\right)\indora\left(T>t\right)\right]$.
\end{Teo}

\begin{Def}
Para el proceso $\left\{\left(N\left(t\right),X\left(t\right)\right):t\geq0\right\}$, sus trayectoria muestrales en el intervalo de tiempo $\left[T_{n-1},T_{n}\right)$ est\'an descritas por
\begin{eqnarray*}
\zeta_{n}=\left(\xi_{n},\left\{X\left(T_{n-1}+t\right):0\leq t<\xi_{n}\right\}\right)
\end{eqnarray*}
Este $\zeta_{n}$ es el $n$-\'esimo segmento del proceso. El proceso es regenerativo sobre los tiempos $T_{n}$ si sus segmentos $\zeta_{n}$ son independientes e id\'enticamennte distribuidos.
\end{Def}


\begin{Note}
Si $\tilde{X}\left(t\right)$ con espacio de estados $\tilde{S}$ es regenerativo sobre $T_{n}$, entonces $X\left(t\right)=f\left(\tilde{X}\left(t\right)\right)$ tambi\'en es regenerativo sobre $T_{n}$, para cualquier funci\'on $f:\tilde{S}\rightarrow S$.
\end{Note}

\begin{Note}
Los procesos regenerativos son crudamente regenerativos, pero no al rev\'es.
\end{Note}


\begin{Note}
Un proceso estoc\'astico a tiempo continuo o discreto es regenerativo si existe un proceso de renovaci\'on  tal que los segmentos del proceso entre tiempos de renovaci\'on sucesivos son i.i.d., es decir, para $\left\{X\left(t\right):t\geq0\right\}$ proceso estoc\'astico a tiempo continuo con espacio de estados $S$, espacio m\'etrico.
\end{Note}

Para $\left\{X\left(t\right):t\geq0\right\}$ Proceso Estoc\'astico a tiempo continuo con estado de espacios $S$, que es un espacio m\'etrico, con trayectorias continuas por la derecha y con l\'imites por la izquierda c.s. Sea $N\left(t\right)$ un proceso de renovaci\'on en $\rea_{+}$ definido en el mismo espacio de probabilidad que $X\left(t\right)$, con tiempos de renovaci\'on $T$ y tiempos de inter-renovaci\'on $\xi_{n}=T_{n}-T_{n-1}$, con misma distribuci\'on $F$ de media finita $\mu$.



\begin{Def}
Para el proceso $\left\{\left(N\left(t\right),X\left(t\right)\right):t\geq0\right\}$, sus trayectoria muestrales en el intervalo de tiempo $\left[T_{n-1},T_{n}\right)$ est\'an descritas por
\begin{eqnarray*}
\zeta_{n}=\left(\xi_{n},\left\{X\left(T_{n-1}+t\right):0\leq t<\xi_{n}\right\}\right)
\end{eqnarray*}
Este $\zeta_{n}$ es el $n$-\'esimo segmento del proceso. El proceso es regenerativo sobre los tiempos $T_{n}$ si sus segmentos $\zeta_{n}$ son independientes e id\'enticamennte distribuidos.
\end{Def}

\begin{Note}
Un proceso regenerativo con media de la longitud de ciclo finita es llamado positivo recurrente.
\end{Note}

\begin{Teo}[Procesos Regenerativos]
Suponga que el proceso
\end{Teo}


\begin{Def}[Renewal Process Trinity]
Para un proceso de renovaci\'on $N\left(t\right)$, los siguientes procesos proveen de informaci\'on sobre los tiempos de renovaci\'on.
\begin{itemize}
\item $A\left(t\right)=t-T_{N\left(t\right)}$, el tiempo de recurrencia hacia atr\'as al tiempo $t$, que es el tiempo desde la \'ultima renovaci\'on para $t$.

\item $B\left(t\right)=T_{N\left(t\right)+1}-t$, el tiempo de recurrencia hacia adelante al tiempo $t$, residual del tiempo de renovaci\'on, que es el tiempo para la pr\'oxima renovaci\'on despu\'es de $t$.

\item $L\left(t\right)=\xi_{N\left(t\right)+1}=A\left(t\right)+B\left(t\right)$, la longitud del intervalo de renovaci\'on que contiene a $t$.
\end{itemize}
\end{Def}

\begin{Note}
El proceso tridimensional $\left(A\left(t\right),B\left(t\right),L\left(t\right)\right)$ es regenerativo sobre $T_{n}$, y por ende cada proceso lo es. Cada proceso $A\left(t\right)$ y $B\left(t\right)$ son procesos de MArkov a tiempo continuo con trayectorias continuas por partes en el espacio de estados $\rea_{+}$. Una expresi\'on conveniente para su distribuci\'on conjunta es, para $0\leq x<t,y\geq0$
\begin{equation}\label{NoRenovacion}
P\left\{A\left(t\right)>x,B\left(t\right)>y\right\}=
P\left\{N\left(t+y\right)-N\left((t-x)\right)=0\right\}
\end{equation}
\end{Note}

\begin{Ejem}[Tiempos de recurrencia Poisson]
Si $N\left(t\right)$ es un proceso Poisson con tasa $\lambda$, entonces de la expresi\'on (\ref{NoRenovacion}) se tiene que

\begin{eqnarray*}
\begin{array}{lc}
P\left\{A\left(t\right)>x,B\left(t\right)>y\right\}=e^{-\lambda\left(x+y\right)},&0\leq x<t,y\geq0,
\end{array}
\end{eqnarray*}
que es la probabilidad Poisson de no renovaciones en un intervalo de longitud $x+y$.

\end{Ejem}

\begin{Note}
Una cadena de Markov erg\'odica tiene la propiedad de ser estacionaria si la distribuci\'on de su estado al tiempo $0$ es su distribuci\'on estacionaria.
\end{Note}


\begin{Def}
Un proceso estoc\'astico a tiempo continuo $\left\{X\left(t\right):t\geq0\right\}$ en un espacio general es estacionario si sus distribuciones finito dimensionales son invariantes bajo cualquier  traslado: para cada $0\leq s_{1}<s_{2}<\cdots<s_{k}$ y $t\geq0$,
\begin{eqnarray*}
\left(X\left(s_{1}+t\right),\ldots,X\left(s_{k}+t\right)\right)=_{d}\left(X\left(s_{1}\right),\ldots,X\left(s_{k}\right)\right).
\end{eqnarray*}
\end{Def}

\begin{Note}
Un proceso de Markov es estacionario si $X\left(t\right)=_{d}X\left(0\right)$, $t\geq0$.
\end{Note}

Considerese el proceso $N\left(t\right)=\sum_{n}\indora\left(\tau_{n}\leq t\right)$ en $\rea_{+}$, con puntos $0<\tau_{1}<\tau_{2}<\cdots$.

\begin{Prop}
Si $N$ es un proceso puntual estacionario y $\esp\left[N\left(1\right)\right]<\infty$, entonces $\esp\left[N\left(t\right)\right]=t\esp\left[N\left(1\right)\right]$, $t\geq0$

\end{Prop}

\begin{Teo}
Los siguientes enunciados son equivalentes
\begin{itemize}
\item[i)] El proceso retardado de renovaci\'on $N$ es estacionario.

\item[ii)] EL proceso de tiempos de recurrencia hacia adelante $B\left(t\right)$ es estacionario.


\item[iii)] $\esp\left[N\left(t\right)\right]=t/\mu$,


\item[iv)] $G\left(t\right)=F_{e}\left(t\right)=\frac{1}{\mu}\int_{0}^{t}\left[1-F\left(s\right)\right]ds$
\end{itemize}
Cuando estos enunciados son ciertos, $P\left\{B\left(t\right)\leq x\right\}=F_{e}\left(x\right)$, para $t,x\geq0$.

\end{Teo}

\begin{Note}
Una consecuencia del teorema anterior es que el Proceso Poisson es el \'unico proceso sin retardo que es estacionario.
\end{Note}

\begin{Coro}
El proceso de renovaci\'on $N\left(t\right)$ sin retardo, y cuyos tiempos de inter renonaci\'on tienen media finita, es estacionario si y s\'olo si es un proceso Poisson.

\end{Coro}


%________________________________________________________________________
%\subsection{Procesos Regenerativos}
%________________________________________________________________________



\begin{Note}
Si $\tilde{X}\left(t\right)$ con espacio de estados $\tilde{S}$ es regenerativo sobre $T_{n}$, entonces $X\left(t\right)=f\left(\tilde{X}\left(t\right)\right)$ tambi\'en es regenerativo sobre $T_{n}$, para cualquier funci\'on $f:\tilde{S}\rightarrow S$.
\end{Note}

\begin{Note}
Los procesos regenerativos son crudamente regenerativos, pero no al rev\'es.
\end{Note}
%\subsection*{Procesos Regenerativos: Sigman\cite{Sigman1}}
\begin{Def}[Definici\'on Cl\'asica]
Un proceso estoc\'astico $X=\left\{X\left(t\right):t\geq0\right\}$ es llamado regenerativo is existe una variable aleatoria $R_{1}>0$ tal que
\begin{itemize}
\item[i)] $\left\{X\left(t+R_{1}\right):t\geq0\right\}$ es independiente de $\left\{\left\{X\left(t\right):t<R_{1}\right\},\right\}$
\item[ii)] $\left\{X\left(t+R_{1}\right):t\geq0\right\}$ es estoc\'asticamente equivalente a $\left\{X\left(t\right):t>0\right\}$
\end{itemize}

Llamamos a $R_{1}$ tiempo de regeneraci\'on, y decimos que $X$ se regenera en este punto.
\end{Def}

$\left\{X\left(t+R_{1}\right)\right\}$ es regenerativo con tiempo de regeneraci\'on $R_{2}$, independiente de $R_{1}$ pero con la misma distribuci\'on que $R_{1}$. Procediendo de esta manera se obtiene una secuencia de variables aleatorias independientes e id\'enticamente distribuidas $\left\{R_{n}\right\}$ llamados longitudes de ciclo. Si definimos a $Z_{k}\equiv R_{1}+R_{2}+\cdots+R_{k}$, se tiene un proceso de renovaci\'on llamado proceso de renovaci\'on encajado para $X$.




\begin{Def}
Para $x$ fijo y para cada $t\geq0$, sea $I_{x}\left(t\right)=1$ si $X\left(t\right)\leq x$,  $I_{x}\left(t\right)=0$ en caso contrario, y def\'inanse los tiempos promedio
\begin{eqnarray*}
\overline{X}&=&lim_{t\rightarrow\infty}\frac{1}{t}\int_{0}^{\infty}X\left(u\right)du\\
\prob\left(X_{\infty}\leq x\right)&=&lim_{t\rightarrow\infty}\frac{1}{t}\int_{0}^{\infty}I_{x}\left(u\right)du,
\end{eqnarray*}
cuando estos l\'imites existan.
\end{Def}

Como consecuencia del teorema de Renovaci\'on-Recompensa, se tiene que el primer l\'imite  existe y es igual a la constante
\begin{eqnarray*}
\overline{X}&=&\frac{\esp\left[\int_{0}^{R_{1}}X\left(t\right)dt\right]}{\esp\left[R_{1}\right]},
\end{eqnarray*}
suponiendo que ambas esperanzas son finitas.

\begin{Note}
\begin{itemize}
\item[a)] Si el proceso regenerativo $X$ es positivo recurrente y tiene trayectorias muestrales no negativas, entonces la ecuaci\'on anterior es v\'alida.
\item[b)] Si $X$ es positivo recurrente regenerativo, podemos construir una \'unica versi\'on estacionaria de este proceso, $X_{e}=\left\{X_{e}\left(t\right)\right\}$, donde $X_{e}$ es un proceso estoc\'astico regenerativo y estrictamente estacionario, con distribuci\'on marginal distribuida como $X_{\infty}$
\end{itemize}
\end{Note}

%________________________________________________________________________
%\subsection{Procesos Regenerativos}
%________________________________________________________________________

Para $\left\{X\left(t\right):t\geq0\right\}$ Proceso Estoc\'astico a tiempo continuo con estado de espacios $S$, que es un espacio m\'etrico, con trayectorias continuas por la derecha y con l\'imites por la izquierda c.s. Sea $N\left(t\right)$ un proceso de renovaci\'on en $\rea_{+}$ definido en el mismo espacio de probabilidad que $X\left(t\right)$, con tiempos de renovaci\'on $T$ y tiempos de inter-renovaci\'on $\xi_{n}=T_{n}-T_{n-1}$, con misma distribuci\'on $F$ de media finita $\mu$.



\begin{Def}
Para el proceso $\left\{\left(N\left(t\right),X\left(t\right)\right):t\geq0\right\}$, sus trayectoria muestrales en el intervalo de tiempo $\left[T_{n-1},T_{n}\right)$ est\'an descritas por
\begin{eqnarray*}
\zeta_{n}=\left(\xi_{n},\left\{X\left(T_{n-1}+t\right):0\leq t<\xi_{n}\right\}\right)
\end{eqnarray*}
Este $\zeta_{n}$ es el $n$-\'esimo segmento del proceso. El proceso es regenerativo sobre los tiempos $T_{n}$ si sus segmentos $\zeta_{n}$ son independientes e id\'enticamennte distribuidos.
\end{Def}


\begin{Note}
Si $\tilde{X}\left(t\right)$ con espacio de estados $\tilde{S}$ es regenerativo sobre $T_{n}$, entonces $X\left(t\right)=f\left(\tilde{X}\left(t\right)\right)$ tambi\'en es regenerativo sobre $T_{n}$, para cualquier funci\'on $f:\tilde{S}\rightarrow S$.
\end{Note}

\begin{Note}
Los procesos regenerativos son crudamente regenerativos, pero no al rev\'es.
\end{Note}

\begin{Def}[Definici\'on Cl\'asica]
Un proceso estoc\'astico $X=\left\{X\left(t\right):t\geq0\right\}$ es llamado regenerativo is existe una variable aleatoria $R_{1}>0$ tal que
\begin{itemize}
\item[i)] $\left\{X\left(t+R_{1}\right):t\geq0\right\}$ es independiente de $\left\{\left\{X\left(t\right):t<R_{1}\right\},\right\}$
\item[ii)] $\left\{X\left(t+R_{1}\right):t\geq0\right\}$ es estoc\'asticamente equivalente a $\left\{X\left(t\right):t>0\right\}$
\end{itemize}

Llamamos a $R_{1}$ tiempo de regeneraci\'on, y decimos que $X$ se regenera en este punto.
\end{Def}

$\left\{X\left(t+R_{1}\right)\right\}$ es regenerativo con tiempo de regeneraci\'on $R_{2}$, independiente de $R_{1}$ pero con la misma distribuci\'on que $R_{1}$. Procediendo de esta manera se obtiene una secuencia de variables aleatorias independientes e id\'enticamente distribuidas $\left\{R_{n}\right\}$ llamados longitudes de ciclo. Si definimos a $Z_{k}\equiv R_{1}+R_{2}+\cdots+R_{k}$, se tiene un proceso de renovaci\'on llamado proceso de renovaci\'on encajado para $X$.

\begin{Note}
Un proceso regenerativo con media de la longitud de ciclo finita es llamado positivo recurrente.
\end{Note}


\begin{Def}
Para $x$ fijo y para cada $t\geq0$, sea $I_{x}\left(t\right)=1$ si $X\left(t\right)\leq x$,  $I_{x}\left(t\right)=0$ en caso contrario, y def\'inanse los tiempos promedio
\begin{eqnarray*}
\overline{X}&=&lim_{t\rightarrow\infty}\frac{1}{t}\int_{0}^{\infty}X\left(u\right)du\\
\prob\left(X_{\infty}\leq x\right)&=&lim_{t\rightarrow\infty}\frac{1}{t}\int_{0}^{\infty}I_{x}\left(u\right)du,
\end{eqnarray*}
cuando estos l\'imites existan.
\end{Def}

Como consecuencia del teorema de Renovaci\'on-Recompensa, se tiene que el primer l\'imite  existe y es igual a la constante
\begin{eqnarray*}
\overline{X}&=&\frac{\esp\left[\int_{0}^{R_{1}}X\left(t\right)dt\right]}{\esp\left[R_{1}\right]},
\end{eqnarray*}
suponiendo que ambas esperanzas son finitas.

\begin{Note}
\begin{itemize}
\item[a)] Si el proceso regenerativo $X$ es positivo recurrente y tiene trayectorias muestrales no negativas, entonces la ecuaci\'on anterior es v\'alida.
\item[b)] Si $X$ es positivo recurrente regenerativo, podemos construir una \'unica versi\'on estacionaria de este proceso, $X_{e}=\left\{X_{e}\left(t\right)\right\}$, donde $X_{e}$ es un proceso estoc\'astico regenerativo y estrictamente estacionario, con distribuci\'on marginal distribuida como $X_{\infty}$
\end{itemize}
\end{Note}

%__________________________________________________________________________________________
%\subsection{Procesos Regenerativos Estacionarios - Stidham \cite{Stidham}}
%__________________________________________________________________________________________


Un proceso estoc\'astico a tiempo continuo $\left\{V\left(t\right),t\geq0\right\}$ es un proceso regenerativo si existe una sucesi\'on de variables aleatorias independientes e id\'enticamente distribuidas $\left\{X_{1},X_{2},\ldots\right\}$, sucesi\'on de renovaci\'on, tal que para cualquier conjunto de Borel $A$, 

\begin{eqnarray*}
\prob\left\{V\left(t\right)\in A|X_{1}+X_{2}+\cdots+X_{R\left(t\right)}=s,\left\{V\left(\tau\right),\tau<s\right\}\right\}=\prob\left\{V\left(t-s\right)\in A|X_{1}>t-s\right\},
\end{eqnarray*}
para todo $0\leq s\leq t$, donde $R\left(t\right)=\max\left\{X_{1}+X_{2}+\cdots+X_{j}\leq t\right\}=$n\'umero de renovaciones ({\emph{puntos de regeneraci\'on}}) que ocurren en $\left[0,t\right]$. El intervalo $\left[0,X_{1}\right)$ es llamado {\emph{primer ciclo de regeneraci\'on}} de $\left\{V\left(t \right),t\geq0\right\}$, $\left[X_{1},X_{1}+X_{2}\right)$ el {\emph{segundo ciclo de regeneraci\'on}}, y as\'i sucesivamente.

Sea $X=X_{1}$ y sea $F$ la funci\'on de distrbuci\'on de $X$


\begin{Def}
Se define el proceso estacionario, $\left\{V^{*}\left(t\right),t\geq0\right\}$, para $\left\{V\left(t\right),t\geq0\right\}$ por

\begin{eqnarray*}
\prob\left\{V\left(t\right)\in A\right\}=\frac{1}{\esp\left[X\right]}\int_{0}^{\infty}\prob\left\{V\left(t+x\right)\in A|X>x\right\}\left(1-F\left(x\right)\right)dx,
\end{eqnarray*} 
para todo $t\geq0$ y todo conjunto de Borel $A$.
\end{Def}

\begin{Def}
Una distribuci\'on se dice que es {\emph{aritm\'etica}} si todos sus puntos de incremento son m\'ultiplos de la forma $0,\lambda, 2\lambda,\ldots$ para alguna $\lambda>0$ entera.
\end{Def}


\begin{Def}
Una modificaci\'on medible de un proceso $\left\{V\left(t\right),t\geq0\right\}$, es una versi\'on de este, $\left\{V\left(t,w\right)\right\}$ conjuntamente medible para $t\geq0$ y para $w\in S$, $S$ espacio de estados para $\left\{V\left(t\right),t\geq0\right\}$.
\end{Def}

\begin{Teo}
Sea $\left\{V\left(t\right),t\geq\right\}$ un proceso regenerativo no negativo con modificaci\'on medible. Sea $\esp\left[X\right]<\infty$. Entonces el proceso estacionario dado por la ecuaci\'on anterior est\'a bien definido y tiene funci\'on de distribuci\'on independiente de $t$, adem\'as
\begin{itemize}
\item[i)] \begin{eqnarray*}
\esp\left[V^{*}\left(0\right)\right]&=&\frac{\esp\left[\int_{0}^{X}V\left(s\right)ds\right]}{\esp\left[X\right]}\end{eqnarray*}
\item[ii)] Si $\esp\left[V^{*}\left(0\right)\right]<\infty$, equivalentemente, si $\esp\left[\int_{0}^{X}V\left(s\right)ds\right]<\infty$,entonces
\begin{eqnarray*}
\frac{\int_{0}^{t}V\left(s\right)ds}{t}\rightarrow\frac{\esp\left[\int_{0}^{X}V\left(s\right)ds\right]}{\esp\left[X\right]}
\end{eqnarray*}
con probabilidad 1 y en media, cuando $t\rightarrow\infty$.
\end{itemize}
\end{Teo}
%
%___________________________________________________________________________________________
%\vspace{5.5cm}
%\chapter{Cadenas de Markov estacionarias}
%\vspace{-1.0cm}


%__________________________________________________________________________________________
%\subsection{Procesos Regenerativos Estacionarios - Stidham \cite{Stidham}}
%__________________________________________________________________________________________


Un proceso estoc\'astico a tiempo continuo $\left\{V\left(t\right),t\geq0\right\}$ es un proceso regenerativo si existe una sucesi\'on de variables aleatorias independientes e id\'enticamente distribuidas $\left\{X_{1},X_{2},\ldots\right\}$, sucesi\'on de renovaci\'on, tal que para cualquier conjunto de Borel $A$, 

\begin{eqnarray*}
\prob\left\{V\left(t\right)\in A|X_{1}+X_{2}+\cdots+X_{R\left(t\right)}=s,\left\{V\left(\tau\right),\tau<s\right\}\right\}=\prob\left\{V\left(t-s\right)\in A|X_{1}>t-s\right\},
\end{eqnarray*}
para todo $0\leq s\leq t$, donde $R\left(t\right)=\max\left\{X_{1}+X_{2}+\cdots+X_{j}\leq t\right\}=$n\'umero de renovaciones ({\emph{puntos de regeneraci\'on}}) que ocurren en $\left[0,t\right]$. El intervalo $\left[0,X_{1}\right)$ es llamado {\emph{primer ciclo de regeneraci\'on}} de $\left\{V\left(t \right),t\geq0\right\}$, $\left[X_{1},X_{1}+X_{2}\right)$ el {\emph{segundo ciclo de regeneraci\'on}}, y as\'i sucesivamente.

Sea $X=X_{1}$ y sea $F$ la funci\'on de distrbuci\'on de $X$


\begin{Def}
Se define el proceso estacionario, $\left\{V^{*}\left(t\right),t\geq0\right\}$, para $\left\{V\left(t\right),t\geq0\right\}$ por

\begin{eqnarray*}
\prob\left\{V\left(t\right)\in A\right\}=\frac{1}{\esp\left[X\right]}\int_{0}^{\infty}\prob\left\{V\left(t+x\right)\in A|X>x\right\}\left(1-F\left(x\right)\right)dx,
\end{eqnarray*} 
para todo $t\geq0$ y todo conjunto de Borel $A$.
\end{Def}

\begin{Def}
Una distribuci\'on se dice que es {\emph{aritm\'etica}} si todos sus puntos de incremento son m\'ultiplos de la forma $0,\lambda, 2\lambda,\ldots$ para alguna $\lambda>0$ entera.
\end{Def}


\begin{Def}
Una modificaci\'on medible de un proceso $\left\{V\left(t\right),t\geq0\right\}$, es una versi\'on de este, $\left\{V\left(t,w\right)\right\}$ conjuntamente medible para $t\geq0$ y para $w\in S$, $S$ espacio de estados para $\left\{V\left(t\right),t\geq0\right\}$.
\end{Def}

\begin{Teo}
Sea $\left\{V\left(t\right),t\geq\right\}$ un proceso regenerativo no negativo con modificaci\'on medible. Sea $\esp\left[X\right]<\infty$. Entonces el proceso estacionario dado por la ecuaci\'on anterior est\'a bien definido y tiene funci\'on de distribuci\'on independiente de $t$, adem\'as
\begin{itemize}
\item[i)] \begin{eqnarray*}
\esp\left[V^{*}\left(0\right)\right]&=&\frac{\esp\left[\int_{0}^{X}V\left(s\right)ds\right]}{\esp\left[X\right]}\end{eqnarray*}
\item[ii)] Si $\esp\left[V^{*}\left(0\right)\right]<\infty$, equivalentemente, si $\esp\left[\int_{0}^{X}V\left(s\right)ds\right]<\infty$,entonces
\begin{eqnarray*}
\frac{\int_{0}^{t}V\left(s\right)ds}{t}\rightarrow\frac{\esp\left[\int_{0}^{X}V\left(s\right)ds\right]}{\esp\left[X\right]}
\end{eqnarray*}
con probabilidad 1 y en media, cuando $t\rightarrow\infty$.
\end{itemize}
\end{Teo}

Para $\left\{X\left(t\right):t\geq0\right\}$ Proceso Estoc\'astico a tiempo continuo con estado de espacios $S$, que es un espacio m\'etrico, con trayectorias continuas por la derecha y con l\'imites por la izquierda c.s. Sea $N\left(t\right)$ un proceso de renovaci\'on en $\rea_{+}$ definido en el mismo espacio de probabilidad que $X\left(t\right)$, con tiempos de renovaci\'on $T$ y tiempos de inter-renovaci\'on $\xi_{n}=T_{n}-T_{n-1}$, con misma distribuci\'on $F$ de media finita $\mu$.


%______________________________________________________________________
%\subsection{Ejemplos, Notas importantes}


Sean $T_{1},T_{2},\ldots$ los puntos donde las longitudes de las colas de la red de sistemas de visitas c\'iclicas son cero simult\'aneamente, cuando la cola $Q_{j}$ es visitada por el servidor para dar servicio, es decir, $L_{1}\left(T_{i}\right)=0,L_{2}\left(T_{i}\right)=0,\hat{L}_{1}\left(T_{i}\right)=0$ y $\hat{L}_{2}\left(T_{i}\right)=0$, a estos puntos se les denominar\'a puntos regenerativos. Sea la funci\'on generadora de momentos para $L_{i}$, el n\'umero de usuarios en la cola $Q_{i}\left(z\right)$ en cualquier momento, est\'a dada por el tiempo promedio de $z^{L_{i}\left(t\right)}$ sobre el ciclo regenerativo definido anteriormente:

\begin{eqnarray*}
Q_{i}\left(z\right)&=&\esp\left[z^{L_{i}\left(t\right)}\right]=\frac{\esp\left[\sum_{m=1}^{M_{i}}\sum_{t=\tau_{i}\left(m\right)}^{\tau_{i}\left(m+1\right)-1}z^{L_{i}\left(t\right)}\right]}{\esp\left[\sum_{m=1}^{M_{i}}\tau_{i}\left(m+1\right)-\tau_{i}\left(m\right)\right]}
\end{eqnarray*}

$M_{i}$ es un tiempo de paro en el proceso regenerativo con $\esp\left[M_{i}\right]<\infty$\footnote{En Stidham\cite{Stidham} y Heyman  se muestra que una condici\'on suficiente para que el proceso regenerativo 
estacionario sea un procesoo estacionario es que el valor esperado del tiempo del ciclo regenerativo sea finito, es decir: $\esp\left[\sum_{m=1}^{M_{i}}C_{i}^{(m)}\right]<\infty$, como cada $C_{i}^{(m)}$ contiene intervalos de r\'eplica positivos, se tiene que $\esp\left[M_{i}\right]<\infty$, adem\'as, como $M_{i}>0$, se tiene que la condici\'on anterior es equivalente a tener que $\esp\left[C_{i}\right]<\infty$,
por lo tanto una condici\'on suficiente para la existencia del proceso regenerativo est\'a dada por $\sum_{k=1}^{N}\mu_{k}<1.$}, se sigue del lema de Wald que:


\begin{eqnarray*}
\esp\left[\sum_{m=1}^{M_{i}}\sum_{t=\tau_{i}\left(m\right)}^{\tau_{i}\left(m+1\right)-1}z^{L_{i}\left(t\right)}\right]&=&\esp\left[M_{i}\right]\esp\left[\sum_{t=\tau_{i}\left(m\right)}^{\tau_{i}\left(m+1\right)-1}z^{L_{i}\left(t\right)}\right]\\
\esp\left[\sum_{m=1}^{M_{i}}\tau_{i}\left(m+1\right)-\tau_{i}\left(m\right)\right]&=&\esp\left[M_{i}\right]\esp\left[\tau_{i}\left(m+1\right)-\tau_{i}\left(m\right)\right]
\end{eqnarray*}

por tanto se tiene que


\begin{eqnarray*}
Q_{i}\left(z\right)&=&\frac{\esp\left[\sum_{t=\tau_{i}\left(m\right)}^{\tau_{i}\left(m+1\right)-1}z^{L_{i}\left(t\right)}\right]}{\esp\left[\tau_{i}\left(m+1\right)-\tau_{i}\left(m\right)\right]}
\end{eqnarray*}

observar que el denominador es simplemente la duraci\'on promedio del tiempo del ciclo.


Haciendo las siguientes sustituciones en la ecuaci\'on (\ref{Corolario2}): $n\rightarrow t-\tau_{i}\left(m\right)$, $T \rightarrow \overline{\tau}_{i}\left(m\right)-\tau_{i}\left(m\right)$, $L_{n}\rightarrow L_{i}\left(t\right)$ y $F\left(z\right)=\esp\left[z^{L_{0}}\right]\rightarrow F_{i}\left(z\right)=\esp\left[z^{L_{i}\tau_{i}\left(m\right)}\right]$, se puede ver que

\begin{eqnarray}\label{Eq.Arribos.Primera}
\esp\left[\sum_{n=0}^{T-1}z^{L_{n}}\right]=
\esp\left[\sum_{t=\tau_{i}\left(m\right)}^{\overline{\tau}_{i}\left(m\right)-1}z^{L_{i}\left(t\right)}\right]
=z\frac{F_{i}\left(z\right)-1}{z-P_{i}\left(z\right)}
\end{eqnarray}

Por otra parte durante el tiempo de intervisita para la cola $i$, $L_{i}\left(t\right)$ solamente se incrementa de manera que el incremento por intervalo de tiempo est\'a dado por la funci\'on generadora de probabilidades de $P_{i}\left(z\right)$, por tanto la suma sobre el tiempo de intervisita puede evaluarse como:

\begin{eqnarray*}
\esp\left[\sum_{t=\tau_{i}\left(m\right)}^{\tau_{i}\left(m+1\right)-1}z^{L_{i}\left(t\right)}\right]&=&\esp\left[\sum_{t=\tau_{i}\left(m\right)}^{\tau_{i}\left(m+1\right)-1}\left\{P_{i}\left(z\right)\right\}^{t-\overline{\tau}_{i}\left(m\right)}\right]=\frac{1-\esp\left[\left\{P_{i}\left(z\right)\right\}^{\tau_{i}\left(m+1\right)-\overline{\tau}_{i}\left(m\right)}\right]}{1-P_{i}\left(z\right)}\\
&=&\frac{1-I_{i}\left[P_{i}\left(z\right)\right]}{1-P_{i}\left(z\right)}
\end{eqnarray*}
por tanto

\begin{eqnarray*}
\esp\left[\sum_{t=\tau_{i}\left(m\right)}^{\tau_{i}\left(m+1\right)-1}z^{L_{i}\left(t\right)}\right]&=&
\frac{1-F_{i}\left(z\right)}{1-P_{i}\left(z\right)}
\end{eqnarray*}

Por lo tanto

\begin{eqnarray*}
Q_{i}\left(z\right)&=&\frac{\esp\left[\sum_{t=\tau_{i}\left(m\right)}^{\tau_{i}
\left(m+1\right)-1}z^{L_{i}\left(t\right)}\right]}{\esp\left[\tau_{i}\left(m+1\right)-\tau_{i}\left(m\right)\right]}\\
&=&\frac{1}{\esp\left[\tau_{i}\left(m+1\right)-\tau_{i}\left(m\right)\right]}
\left\{
\esp\left[\sum_{t=\tau_{i}\left(m\right)}^{\overline{\tau}_{i}\left(m\right)-1}
z^{L_{i}\left(t\right)}\right]
+\esp\left[\sum_{t=\overline{\tau}_{i}\left(m\right)}^{\tau_{i}\left(m+1\right)-1}
z^{L_{i}\left(t\right)}\right]\right\}\\
&=&\frac{1}{\esp\left[\tau_{i}\left(m+1\right)-\tau_{i}\left(m\right)\right]}
\left\{
z\frac{F_{i}\left(z\right)-1}{z-P_{i}\left(z\right)}+\frac{1-F_{i}\left(z\right)}
{1-P_{i}\left(z\right)}
\right\}
\end{eqnarray*}

es decir

\begin{equation}
Q_{i}\left(z\right)=\frac{1}{\esp\left[C_{i}\right]}\cdot\frac{1-F_{i}\left(z\right)}{P_{i}\left(z\right)-z}\cdot\frac{\left(1-z\right)P_{i}\left(z\right)}{1-P_{i}\left(z\right)}
\end{equation}

\begin{Teo}
Dada una Red de Sistemas de Visitas C\'iclicas (RSVC), conformada por dos Sistemas de Visitas C\'iclicas (SVC), donde cada uno de ellos consta de dos colas tipo $M/M/1$. Los dos sistemas est\'an comunicados entre s\'i por medio de la transferencia de usuarios entre las colas $Q_{1}\leftrightarrow Q_{3}$ y $Q_{2}\leftrightarrow Q_{4}$. Se definen los eventos para los procesos de arribos al tiempo $t$, $A_{j}\left(t\right)=\left\{0 \textrm{ arribos en }Q_{j}\textrm{ al tiempo }t\right\}$ para alg\'un tiempo $t\geq0$ y $Q_{j}$ la cola $j$-\'esima en la RSVC, para $j=1,2,3,4$.  Existe un intervalo $I\neq\emptyset$ tal que para $T^{*}\in I$, tal que $\prob\left\{A_{1}\left(T^{*}\right),A_{2}\left(Tt^{*}\right),
A_{3}\left(T^{*}\right),A_{4}\left(T^{*}\right)|T^{*}\in I\right\}>0$.
\end{Teo}

\begin{proof}
Sin p\'erdida de generalidad podemos considerar como base del an\'alisis a la cola $Q_{1}$ del primer sistema que conforma la RSVC.

Sea $n>0$, ciclo en el primer sistema en el que se sabe que $L_{j}\left(\overline{\tau}_{1}\left(n\right)\right)=0$, pues la pol\'itica de servicio con que atienden los servidores es la exhaustiva. Como es sabido, para trasladarse a la siguiente cola, el servidor incurre en un tiempo de traslado $r_{1}\left(n\right)>0$, entonces tenemos que $\tau_{2}\left(n\right)=\overline{\tau}_{1}\left(n\right)+r_{1}\left(n\right)$.


Definamos el intervalo $I_{1}\equiv\left[\overline{\tau}_{1}\left(n\right),\tau_{2}\left(n\right)\right]$ de longitud $\xi_{1}=r_{1}\left(n\right)$. Dado que los tiempos entre arribo son exponenciales con tasa $\tilde{\mu}_{1}=\mu_{1}+\hat{\mu}_{1}$ ($\mu_{1}$ son los arribos a $Q_{1}$ por primera vez al sistema, mientras que $\hat{\mu}_{1}$ son los arribos de traslado procedentes de $Q_{3}$) se tiene que la probabilidad del evento $A_{1}\left(t\right)$ est\'a dada por 

\begin{equation}
\prob\left\{A_{1}\left(t\right)|t\in I_{1}\left(n\right)\right\}=e^{-\tilde{\mu}_{1}\xi_{1}\left(n\right)}.
\end{equation} 

Por otra parte, para la cola $Q_{2}$, el tiempo $\overline{\tau}_{2}\left(n-1\right)$ es tal que $L_{2}\left(\overline{\tau}_{2}\left(n-1\right)\right)=0$, es decir, es el tiempo en que la cola queda totalmente vac\'ia en el ciclo anterior a $n$. Entonces tenemos un sgundo intervalo $I_{2}\equiv\left[\overline{\tau}_{2}\left(n-1\right),\tau_{2}\left(n\right)\right]$. Por lo tanto la probabilidad del evento $A_{2}\left(t\right)$ tiene probabilidad dada por

\begin{equation}
\prob\left\{A_{2}\left(t\right)|t\in I_{2}\left(n\right)\right\}=e^{-\tilde{\mu}_{2}\xi_{2}\left(n\right)},
\end{equation} 

donde $\xi_{2}\left(n\right)=\tau_{2}\left(n\right)-\overline{\tau}_{2}\left(n-1\right)$.



Entonces, se tiene que

\begin{eqnarray*}
\prob\left\{A_{1}\left(t\right),A_{2}\left(t\right)|t\in I_{1}\left(n\right)\right\}&=&
\prob\left\{A_{1}\left(t\right)|t\in I_{1}\left(n\right)\right\}
\prob\left\{A_{2}\left(t\right)|t\in I_{1}\left(n\right)\right\}\\
&\geq&
\prob\left\{A_{1}\left(t\right)|t\in I_{1}\left(n\right)\right\}
\prob\left\{A_{2}\left(t\right)|t\in I_{2}\left(n\right)\right\}\\
&=&e^{-\tilde{\mu}_{1}\xi_{1}\left(n\right)}e^{-\tilde{\mu}_{2}\xi_{2}\left(n\right)}
=e^{-\left[\tilde{\mu}_{1}\xi_{1}\left(n\right)+\tilde{\mu}_{2}\xi_{2}\left(n\right)\right]}.
\end{eqnarray*}


es decir, 

\begin{equation}
\prob\left\{A_{1}\left(t\right),A_{2}\left(t\right)|t\in I_{1}\left(n\right)\right\}
=e^{-\left[\tilde{\mu}_{1}\xi_{1}\left(n\right)+\tilde{\mu}_{2}\xi_{2}
\left(n\right)\right]}>0.
\end{equation}

En lo que respecta a la relaci\'on entre los dos SVC que conforman la RSVC, se afirma que existe $m>0$ tal que $\overline{\tau}_{3}\left(m\right)<\tau_{2}\left(n\right)<\tau_{4}\left(m\right)$.

Para $Q_{3}$ sea $I_{3}=\left[\overline{\tau}_{3}\left(m\right),\tau_{4}\left(m\right)\right]$ con longitud  $\xi_{3}\left(m\right)=r_{3}\left(m\right)$, entonces 

\begin{equation}
\prob\left\{A_{3}\left(t\right)|t\in I_{3}\left(n\right)\right\}=e^{-\tilde{\mu}_{3}\xi_{3}\left(n\right)}.
\end{equation} 

An\'alogamente que como se hizo para $Q_{2}$, tenemos que para $Q_{4}$ se tiene el intervalo $I_{4}=\left[\overline{\tau}_{4}\left(m-1\right),\tau_{4}\left(m\right)\right]$ con longitud $\xi_{4}\left(m\right)=\tau_{4}\left(m\right)-\overline{\tau}_{4}\left(m-1\right)$, entonces


\begin{equation}
\prob\left\{A_{4}\left(t\right)|t\in I_{4}\left(m\right)\right\}=e^{-\tilde{\mu}_{4}\xi_{4}\left(n\right)}.
\end{equation} 

Al igual que para el primer sistema, dado que $I_{3}\left(m\right)\subset I_{4}\left(m\right)$, se tiene que

\begin{eqnarray*}
\xi_{3}\left(m\right)\leq\xi_{4}\left(m\right)&\Leftrightarrow& -\xi_{3}\left(m\right)\geq-\xi_{4}\left(m\right)
\\
-\tilde{\mu}_{4}\xi_{3}\left(m\right)\geq-\tilde{\mu}_{4}\xi_{4}\left(m\right)&\Leftrightarrow&
e^{-\tilde{\mu}_{4}\xi_{3}\left(m\right)}\geq e^{-\tilde{\mu}_{4}\xi_{4}\left(m\right)}\\
\prob\left\{A_{4}\left(t\right)|t\in I_{3}\left(m\right)\right\}&\geq&
\prob\left\{A_{4}\left(t\right)|t\in I_{4}\left(m\right)\right\}
\end{eqnarray*}

Entonces, dado que los eventos $A_{3}$ y $A_{4}$ son independientes, se tiene que

\begin{eqnarray*}
\prob\left\{A_{3}\left(t\right),A_{4}\left(t\right)|t\in I_{3}\left(m\right)\right\}&=&
\prob\left\{A_{3}\left(t\right)|t\in I_{3}\left(m\right)\right\}
\prob\left\{A_{4}\left(t\right)|t\in I_{3}\left(m\right)\right\}\\
&\geq&
\prob\left\{A_{3}\left(t\right)|t\in I_{3}\left(n\right)\right\}
\prob\left\{A_{4}\left(t\right)|t\in I_{4}\left(n\right)\right\}\\
&=&e^{-\tilde{\mu}_{3}\xi_{3}\left(m\right)}e^{-\tilde{\mu}_{4}\xi_{4}
\left(m\right)}
=e^{-\left[\tilde{\mu}_{3}\xi_{3}\left(m\right)+\tilde{\mu}_{4}\xi_{4}
\left(m\right)\right]}.
\end{eqnarray*}


es decir, 

\begin{equation}
\prob\left\{A_{3}\left(t\right),A_{4}\left(t\right)|t\in I_{3}\left(m\right)\right\}
=e^{-\left[\tilde{\mu}_{3}\xi_{3}\left(m\right)+\tilde{\mu}_{4}\xi_{4}
\left(m\right)\right]}>0.
\end{equation}

Por construcci\'on se tiene que $I\left(n,m\right)\equiv I_{1}\left(n\right)\cap I_{3}\left(m\right)\neq\emptyset$,entonces en particular se tienen las contenciones $I\left(n,m\right)\subseteq I_{1}\left(n\right)$ y $I\left(n,m\right)\subseteq I_{3}\left(m\right)$, por lo tanto si definimos $\xi_{n,m}\equiv\ell\left(I\left(n,m\right)\right)$ tenemos que

\begin{eqnarray*}
\xi_{n,m}\leq\xi_{1}\left(n\right)\textrm{ y }\xi_{n,m}\leq\xi_{3}\left(m\right)\textrm{ entonces }
-\xi_{n,m}\geq-\xi_{1}\left(n\right)\textrm{ y }-\xi_{n,m}\leq-\xi_{3}\left(m\right)\\
\end{eqnarray*}
por lo tanto tenemos las desigualdades 



\begin{eqnarray*}
\begin{array}{ll}
-\tilde{\mu}_{1}\xi_{n,m}\geq-\tilde{\mu}_{1}\xi_{1}\left(n\right),&
-\tilde{\mu}_{2}\xi_{n,m}\geq-\tilde{\mu}_{2}\xi_{1}\left(n\right)
\geq-\tilde{\mu}_{2}\xi_{2}\left(n\right),\\
-\tilde{\mu}_{3}\xi_{n,m}\geq-\tilde{\mu}_{3}\xi_{3}\left(m\right),&
-\tilde{\mu}_{4}\xi_{n,m}\geq-\tilde{\mu}_{4}\xi_{3}\left(m\right)
\geq-\tilde{\mu}_{4}\xi_{4}\left(m\right).
\end{array}
\end{eqnarray*}

Sea $T^{*}\in I_{n,m}$, entonces dado que en particular $T^{*}\in I_{1}\left(n\right)$ se cumple con probabilidad positiva que no hay arribos a las colas $Q_{1}$ y $Q_{2}$, en consecuencia, tampoco hay usuarios de transferencia para $Q_{3}$ y $Q_{4}$, es decir, $\tilde{\mu}_{1}=\mu_{1}$, $\tilde{\mu}_{2}=\mu_{2}$, $\tilde{\mu}_{3}=\mu_{3}$, $\tilde{\mu}_{4}=\mu_{4}$, es decir, los eventos $Q_{1}$ y $Q_{3}$ son condicionalmente independientes en el intervalo $I_{n,m}$; lo mismo ocurre para las colas $Q_{2}$ y $Q_{4}$, por lo tanto tenemos que


\begin{eqnarray}
\begin{array}{l}
\prob\left\{A_{1}\left(T^{*}\right),A_{2}\left(T^{*}\right),
A_{3}\left(T^{*}\right),A_{4}\left(T^{*}\right)|T^{*}\in I_{n,m}\right\}
=\prod_{j=1}^{4}\prob\left\{A_{j}\left(T^{*}\right)|T^{*}\in I_{n,m}\right\}\\
\geq\prob\left\{A_{1}\left(T^{*}\right)|T^{*}\in I_{1}\left(n\right)\right\}
\prob\left\{A_{2}\left(T^{*}\right)|T^{*}\in I_{2}\left(n\right)\right\}
\prob\left\{A_{3}\left(T^{*}\right)|T^{*}\in I_{3}\left(m\right)\right\}
\prob\left\{A_{4}\left(T^{*}\right)|T^{*}\in I_{4}\left(m\right)\right\}\\
=e^{-\mu_{1}\xi_{1}\left(n\right)}
e^{-\mu_{2}\xi_{2}\left(n\right)}
e^{-\mu_{3}\xi_{3}\left(m\right)}
e^{-\mu_{4}\xi_{4}\left(m\right)}
=e^{-\left[\tilde{\mu}_{1}\xi_{1}\left(n\right)
+\tilde{\mu}_{2}\xi_{2}\left(n\right)
+\tilde{\mu}_{3}\xi_{3}\left(m\right)
+\tilde{\mu}_{4}\xi_{4}
\left(m\right)\right]}>0.
\end{array}
\end{eqnarray}
\end{proof}


Estos resultados aparecen en Daley (1968) \cite{Daley68} para $\left\{T_{n}\right\}$ intervalos de inter-arribo, $\left\{D_{n}\right\}$ intervalos de inter-salida y $\left\{S_{n}\right\}$ tiempos de servicio.

\begin{itemize}
\item Si el proceso $\left\{T_{n}\right\}$ es Poisson, el proceso $\left\{D_{n}\right\}$ es no correlacionado si y s\'olo si es un proceso Poisso, lo cual ocurre si y s\'olo si $\left\{S_{n}\right\}$ son exponenciales negativas.

\item Si $\left\{S_{n}\right\}$ son exponenciales negativas, $\left\{D_{n}\right\}$ es un proceso de renovaci\'on  si y s\'olo si es un proceso Poisson, lo cual ocurre si y s\'olo si $\left\{T_{n}\right\}$ es un proceso Poisson.

\item $\esp\left(D_{n}\right)=\esp\left(T_{n}\right)$.

\item Para un sistema de visitas $GI/M/1$ se tiene el siguiente teorema:

\begin{Teo}
En un sistema estacionario $GI/M/1$ los intervalos de interpartida tienen
\begin{eqnarray*}
\esp\left(e^{-\theta D_{n}}\right)&=&\mu\left(\mu+\theta\right)^{-1}\left[\delta\theta
-\mu\left(1-\delta\right)\alpha\left(\theta\right)\right]
\left[\theta-\mu\left(1-\delta\right)^{-1}\right]\\
\alpha\left(\theta\right)&=&\esp\left[e^{-\theta T_{0}}\right]\\
var\left(D_{n}\right)&=&var\left(T_{0}\right)-\left(\tau^{-1}-\delta^{-1}\right)
2\delta\left(\esp\left(S_{0}\right)\right)^{2}\left(1-\delta\right)^{-1}.
\end{eqnarray*}
\end{Teo}



\begin{Teo}
El proceso de salida de un sistema de colas estacionario $GI/M/1$ es un proceso de renovaci\'on si y s\'olo si el proceso de entrada es un proceso Poisson, en cuyo caso el proceso de salida es un proceso Poisson.
\end{Teo}


\begin{Teo}
Los intervalos de interpartida $\left\{D_{n}\right\}$ de un sistema $M/G/1$ estacionario son no correlacionados si y s\'olo si la distribuci\'on de los tiempos de servicio es exponencial negativa, es decir, el sistema es de tipo  $M/M/1$.

\end{Teo}



\end{itemize}


%\section{Resultados para Procesos de Salida}

En Sigman, Thorison y Wolff \cite{Sigman2} prueban que para la existencia de un una sucesi\'on infinita no decreciente de tiempos de regeneraci\'on $\tau_{1}\leq\tau_{2}\leq\cdots$ en los cuales el proceso se regenera, basta un tiempo de regeneraci\'on $R_{1}$, donde $R_{j}=\tau_{j}-\tau_{j-1}$. Para tal efecto se requiere la existencia de un espacio de probabilidad $\left(\Omega,\mathcal{F},\prob\right)$, y proceso estoc\'astico $\textit{X}=\left\{X\left(t\right):t\geq0\right\}$ con espacio de estados $\left(S,\mathcal{R}\right)$, con $\mathcal{R}$ $\sigma$-\'algebra.

\begin{Prop}
Si existe una variable aleatoria no negativa $R_{1}$ tal que $\theta_{R\footnotesize{1}}X=_{D}X$, entonces $\left(\Omega,\mathcal{F},\prob\right)$ puede extenderse para soportar una sucesi\'on estacionaria de variables aleatorias $R=\left\{R_{k}:k\geq1\right\}$, tal que para $k\geq1$,
\begin{eqnarray*}
\theta_{k}\left(X,R\right)=_{D}\left(X,R\right).
\end{eqnarray*}

Adem\'as, para $k\geq1$, $\theta_{k}R$ es condicionalmente independiente de $\left(X,R_{1},\ldots,R_{k}\right)$, dado $\theta_{\tau k}X$.

\end{Prop}


\begin{itemize}
\item Doob en 1953 demostr\'o que el estado estacionario de un proceso de partida en un sistema de espera $M/G/\infty$, es Poisson con la misma tasa que el proceso de arribos.

\item Burke en 1968, fue el primero en demostrar que el estado estacionario de un proceso de salida de una cola $M/M/s$ es un proceso Poisson.

\item Disney en 1973 obtuvo el siguiente resultado:

\begin{Teo}
Para el sistema de espera $M/G/1/L$ con disciplina FIFO, el proceso $\textbf{I}$ es un proceso de renovaci\'on si y s\'olo si el proceso denominado longitud de la cola es estacionario y se cumple cualquiera de los siguientes casos:

\begin{itemize}
\item[a)] Los tiempos de servicio son identicamente cero;
\item[b)] $L=0$, para cualquier proceso de servicio $S$;
\item[c)] $L=1$ y $G=D$;
\item[d)] $L=\infty$ y $G=M$.
\end{itemize}
En estos casos, respectivamente, las distribuciones de interpartida $P\left\{T_{n+1}-T_{n}\leq t\right\}$ son


\begin{itemize}
\item[a)] $1-e^{-\lambda t}$, $t\geq0$;
\item[b)] $1-e^{-\lambda t}*F\left(t\right)$, $t\geq0$;
\item[c)] $1-e^{-\lambda t}*\indora_{d}\left(t\right)$, $t\geq0$;
\item[d)] $1-e^{-\lambda t}*F\left(t\right)$, $t\geq0$.
\end{itemize}
\end{Teo}


\item Finch (1959) mostr\'o que para los sistemas $M/G/1/L$, con $1\leq L\leq \infty$ con distribuciones de servicio dos veces diferenciable, solamente el sistema $M/M/1/\infty$ tiene proceso de salida de renovaci\'on estacionario.

\item King (1971) demostro que un sistema de colas estacionario $M/G/1/1$ tiene sus tiempos de interpartida sucesivas $D_{n}$ y $D_{n+1}$ son independientes, si y s\'olo si, $G=D$, en cuyo caso le proceso de salida es de renovaci\'on.

\item Disney (1973) demostr\'o que el \'unico sistema estacionario $M/G/1/L$, que tiene proceso de salida de renovaci\'on  son los sistemas $M/M/1$ y $M/D/1/1$.



\item El siguiente resultado es de Disney y Koning (1985)
\begin{Teo}
En un sistema de espera $M/G/s$, el estado estacionario del proceso de salida es un proceso Poisson para cualquier distribuci\'on de los tiempos de servicio si el sistema tiene cualquiera de las siguientes cuatro propiedades.

\begin{itemize}
\item[a)] $s=\infty$
\item[b)] La disciplina de servicio es de procesador compartido.
\item[c)] La disciplina de servicio es LCFS y preemptive resume, esto se cumple para $L<\infty$
\item[d)] $G=M$.
\end{itemize}

\end{Teo}

\item El siguiente resultado es de Alamatsaz (1983)

\begin{Teo}
En cualquier sistema de colas $GI/G/1/L$ con $1\leq L<\infty$ y distribuci\'on de interarribos $A$ y distribuci\'on de los tiempos de servicio $B$, tal que $A\left(0\right)=0$, $A\left(t\right)\left(1-B\left(t\right)\right)>0$ para alguna $t>0$ y $B\left(t\right)$ para toda $t>0$, es imposible que el proceso de salida estacionario sea de renovaci\'on.
\end{Teo}

\end{itemize}

Estos resultados aparecen en Daley (1968) \cite{Daley68} para $\left\{T_{n}\right\}$ intervalos de inter-arribo, $\left\{D_{n}\right\}$ intervalos de inter-salida y $\left\{S_{n}\right\}$ tiempos de servicio.

\begin{itemize}
\item Si el proceso $\left\{T_{n}\right\}$ es Poisson, el proceso $\left\{D_{n}\right\}$ es no correlacionado si y s\'olo si es un proceso Poisso, lo cual ocurre si y s\'olo si $\left\{S_{n}\right\}$ son exponenciales negativas.

\item Si $\left\{S_{n}\right\}$ son exponenciales negativas, $\left\{D_{n}\right\}$ es un proceso de renovaci\'on  si y s\'olo si es un proceso Poisson, lo cual ocurre si y s\'olo si $\left\{T_{n}\right\}$ es un proceso Poisson.

\item $\esp\left(D_{n}\right)=\esp\left(T_{n}\right)$.

\item Para un sistema de visitas $GI/M/1$ se tiene el siguiente teorema:

\begin{Teo}
En un sistema estacionario $GI/M/1$ los intervalos de interpartida tienen
\begin{eqnarray*}
\esp\left(e^{-\theta D_{n}}\right)&=&\mu\left(\mu+\theta\right)^{-1}\left[\delta\theta
-\mu\left(1-\delta\right)\alpha\left(\theta\right)\right]
\left[\theta-\mu\left(1-\delta\right)^{-1}\right]\\
\alpha\left(\theta\right)&=&\esp\left[e^{-\theta T_{0}}\right]\\
var\left(D_{n}\right)&=&var\left(T_{0}\right)-\left(\tau^{-1}-\delta^{-1}\right)
2\delta\left(\esp\left(S_{0}\right)\right)^{2}\left(1-\delta\right)^{-1}.
\end{eqnarray*}
\end{Teo}



\begin{Teo}
El proceso de salida de un sistema de colas estacionario $GI/M/1$ es un proceso de renovaci\'on si y s\'olo si el proceso de entrada es un proceso Poisson, en cuyo caso el proceso de salida es un proceso Poisson.
\end{Teo}


\begin{Teo}
Los intervalos de interpartida $\left\{D_{n}\right\}$ de un sistema $M/G/1$ estacionario son no correlacionados si y s\'olo si la distribuci\'on de los tiempos de servicio es exponencial negativa, es decir, el sistema es de tipo  $M/M/1$.

\end{Teo}



\end{itemize}
%\newpage
%________________________________________________________________________
%\section{Redes de Sistemas de Visitas C\'iclicas}
%________________________________________________________________________

Sean $Q_{1},Q_{2},Q_{3}$ y $Q_{4}$ en una Red de Sistemas de Visitas C\'iclicas (RSVC). Supongamos que cada una de las colas es del tipo $M/M/1$ con tasa de arribo $\mu_{i}$ y que la transferencia de usuarios entre los dos sistemas ocurre entre $Q_{1}\leftrightarrow Q_{3}$ y $Q_{2}\leftrightarrow Q_{4}$ con respectiva tasa de arribo igual a la tasa de salida $\hat{\mu}_{i}=\mu_{i}$, esto se sabe por lo desarrollado en la secci\'on anterior.  

Consideremos, sin p\'erdida de generalidad como base del an\'alisis, la cola $Q_{1}$ adem\'as supongamos al servidor lo comenzamos a observar una vez que termina de atender a la misma para desplazarse y llegar a $Q_{2}$, es decir al tiempo $\tau_{2}$.

Sea $n\in\nat$, $n>0$, ciclo del servidor en que regresa a $Q_{1}$ para dar servicio y atender conforme a la pol\'itica exhaustiva, entonces se tiene que $\overline{\tau}_{1}\left(n\right)$ es el tiempo del servidor en el sistema 1 en que termina de dar servicio a todos los usuarios presentes en la cola, por lo tanto se cumple que $L_{1}\left(\overline{\tau}_{1}\left(n\right)\right)=0$, entonces el servidor para llegar a $Q_{2}$ incurre en un tiempo de traslado $r_{1}$ y por tanto se cumple que $\tau_{2}\left(n\right)=\overline{\tau}_{1}\left(n\right)+r_{1}$. Dado que los tiempos entre arribos son exponenciales se cumple que 

\begin{eqnarray*}
\prob\left\{0 \textrm{ arribos en }Q_{1}\textrm{ en el intervalo }\left[\overline{\tau}_{1}\left(n\right),\overline{\tau}_{1}\left(n\right)+r_{1}\right]\right\}=e^{-\tilde{\mu}_{1}r_{1}},\\
\prob\left\{0 \textrm{ arribos en }Q_{2}\textrm{ en el intervalo }\left[\overline{\tau}_{1}\left(n\right),\overline{\tau}_{1}\left(n\right)+r_{1}\right]\right\}=e^{-\tilde{\mu}_{2}r_{1}}.
\end{eqnarray*}

El evento que nos interesa consiste en que no haya arribos desde que el servidor abandon\'o $Q_{2}$ y regresa nuevamente para dar servicio, es decir en el intervalo de tiempo $\left[\overline{\tau}_{2}\left(n-1\right),\tau_{2}\left(n\right)\right]$. Entonces, si hacemos


\begin{eqnarray*}
\varphi_{1}\left(n\right)&\equiv&\overline{\tau}_{1}\left(n\right)+r_{1}=\overline{\tau}_{2}\left(n-1\right)+r_{1}+r_{2}+\overline{\tau}_{1}\left(n\right)-\tau_{1}\left(n\right)\\
&=&\overline{\tau}_{2}\left(n-1\right)+\overline{\tau}_{1}\left(n\right)-\tau_{1}\left(n\right)+r,,
\end{eqnarray*}

y longitud del intervalo

\begin{eqnarray*}
\xi&\equiv&\overline{\tau}_{1}\left(n\right)+r_{1}-\overline{\tau}_{2}\left(n-1\right)
=\overline{\tau}_{2}\left(n-1\right)+\overline{\tau}_{1}\left(n\right)-\tau_{1}\left(n\right)+r-\overline{\tau}_{2}\left(n-1\right)\\
&=&\overline{\tau}_{1}\left(n\right)-\tau_{1}\left(n\right)+r.
\end{eqnarray*}


Entonces, determinemos la probabilidad del evento no arribos a $Q_{2}$ en $\left[\overline{\tau}_{2}\left(n-1\right),\varphi_{1}\left(n\right)\right]$:

\begin{eqnarray}
\prob\left\{0 \textrm{ arribos en }Q_{2}\textrm{ en el intervalo }\left[\overline{\tau}_{2}\left(n-1\right),\varphi_{1}\left(n\right)\right]\right\}
=e^{-\tilde{\mu}_{2}\xi}.
\end{eqnarray}

De manera an\'aloga, tenemos que la probabilidad de no arribos a $Q_{1}$ en $\left[\overline{\tau}_{2}\left(n-1\right),\varphi_{1}\left(n\right)\right]$ esta dada por

\begin{eqnarray}
\prob\left\{0 \textrm{ arribos en }Q_{1}\textrm{ en el intervalo }\left[\overline{\tau}_{2}\left(n-1\right),\varphi_{1}\left(n\right)\right]\right\}
=e^{-\tilde{\mu}_{1}\xi},
\end{eqnarray}

\begin{eqnarray}
\prob\left\{0 \textrm{ arribos en }Q_{2}\textrm{ en el intervalo }\left[\overline{\tau}_{2}\left(n-1\right),\varphi_{1}\left(n\right)\right]\right\}
=e^{-\tilde{\mu}_{2}\xi}.
\end{eqnarray}

Por tanto 

\begin{eqnarray}
\begin{array}{l}
\prob\left\{0 \textrm{ arribos en }Q_{1}\textrm{ y }Q_{2}\textrm{ en el intervalo }\left[\overline{\tau}_{2}\left(n-1\right),\varphi_{1}\left(n\right)\right]\right\}\\
=\prob\left\{0 \textrm{ arribos en }Q_{1}\textrm{ en el intervalo }\left[\overline{\tau}_{2}\left(n-1\right),\varphi_{1}\left(n\right)\right]\right\}\\
\times
\prob\left\{0 \textrm{ arribos en }Q_{2}\textrm{ en el intervalo }\left[\overline{\tau}_{2}\left(n-1\right),\varphi_{1}\left(n\right)\right]\right\}=e^{-\tilde{\mu}_{1}\xi}e^{-\tilde{\mu}_{2}\xi}
=e^{-\tilde{\mu}\xi}.
\end{array}
\end{eqnarray}

Para el segundo sistema, consideremos nuevamente $\overline{\tau}_{1}\left(n\right)+r_{1}$, sin p\'erdida de generalidad podemos suponer que existe $m>0$ tal que $\overline{\tau}_{3}\left(m\right)<\overline{\tau}_{1}+r_{1}<\tau_{4}\left(m\right)$, entonces

\begin{eqnarray}
\prob\left\{0 \textrm{ arribos en }Q_{3}\textrm{ en el intervalo }\left[\overline{\tau}_{3}\left(m\right),\overline{\tau}_{1}\left(n\right)+r_{1}\right]\right\}
=e^{-\tilde{\mu}_{3}\xi_{3}},
\end{eqnarray}
donde 
\begin{eqnarray}
\xi_{3}=\overline{\tau}_{1}\left(n\right)+r_{1}-\overline{\tau}_{3}\left(m\right)=
\overline{\tau}_{1}\left(n\right)-\overline{\tau}_{3}\left(m\right)+r_{1},
\end{eqnarray}

mientras que para $Q_{4}$ al igual que con $Q_{2}$ escribiremos $\tau_{4}\left(m\right)$ en t\'erminos de $\overline{\tau}_{4}\left(m-1\right)$:

$\varphi_{2}\equiv\tau_{4}\left(m\right)=\overline{\tau}_{4}\left(m-1\right)+r_{4}+\overline{\tau}_{3}\left(m\right)
-\tau_{3}\left(m\right)+r_{3}=\overline{\tau}_{4}\left(m-1\right)+\overline{\tau}_{3}\left(m\right)
-\tau_{3}\left(m\right)+\hat{r}$, adem\'as,

$\xi_{2}\equiv\varphi_{2}\left(m\right)-\overline{\tau}_{1}-r_{1}=\overline{\tau}_{4}\left(m-1\right)+\overline{\tau}_{3}\left(m\right)
-\tau_{3}\left(m\right)-\overline{\tau}_{1}\left(n\right)+\hat{r}-r_{1}$. 

Entonces


\begin{eqnarray}
\prob\left\{0 \textrm{ arribos en }Q_{4}\textrm{ en el intervalo }\left[\overline{\tau}_{1}\left(n\right)+r_{1},\varphi_{2}\left(m\right)\right]\right\}
=e^{-\tilde{\mu}_{4}\xi_{2}},
\end{eqnarray}

mientras que para $Q_{3}$ se tiene que 

\begin{eqnarray}
\prob\left\{0 \textrm{ arribos en }Q_{3}\textrm{ en el intervalo }\left[\overline{\tau}_{1}\left(n\right)+r_{1},\varphi_{2}\left(m\right)\right]\right\}
=e^{-\tilde{\mu}_{3}\xi_{2}}
\end{eqnarray}

Por tanto

\begin{eqnarray}
\prob\left\{0 \textrm{ arribos en }Q_{3}\wedge Q_{4}\textrm{ en el intervalo }\left[\overline{\tau}_{1}\left(n\right)+r_{1},\varphi_{2}\left(m\right)\right]\right\}
=e^{-\hat{\mu}\xi_{2}}
\end{eqnarray}
donde $\hat{\mu}=\tilde{\mu}_{3}+\tilde{\mu}_{4}$.

Ahora, definamos los intervalos $\mathcal{I}_{1}=\left[\overline{\tau}_{1}\left(n\right)+r_{1},\varphi_{1}\left(n\right)\right]$  y $\mathcal{I}_{2}=\left[\overline{\tau}_{1}\left(n\right)+r_{1},\varphi_{2}\left(m\right)\right]$, entonces, sea $\mathcal{I}=\mathcal{I}_{1}\cap\mathcal{I}_{2}$ el intervalo donde cada una de las colas se encuentran vac\'ias, es decir, si tomamos $T^{*}\in\mathcal{I}$, entonces  $L_{1}\left(T^{*}\right)=L_{2}\left(T^{*}\right)=L_{3}\left(T^{*}\right)=L_{4}\left(T^{*}\right)=0$.

Ahora, dado que por construcci\'on $\mathcal{I}\neq\emptyset$ y que para $T^{*}\in\mathcal{I}$ en ninguna de las colas han llegado usuarios, se tiene que no hay transferencia entre las colas, por lo tanto, el sistema 1 y el sistema 2 son condicionalmente independientes en $\mathcal{I}$, es decir

\begin{eqnarray}
\prob\left\{L_{1}\left(T^{*}\right),L_{2}\left(T^{*}\right),L_{3}\left(T^{*}\right),L_{4}\left(T^{*}\right)|T^{*}\in\mathcal{I}\right\}=\prod_{j=1}^{4}\prob\left\{L_{j}\left(T^{*}\right)\right\},
\end{eqnarray}

para $T^{*}\in\mathcal{I}$. 

%\newpage























%________________________________________________________________________
%\section{Procesos Regenerativos}
%________________________________________________________________________

%________________________________________________________________________
%\subsection*{Procesos Regenerativos Sigman, Thorisson y Wolff \cite{Sigman1}}
%________________________________________________________________________


\begin{Def}[Definici\'on Cl\'asica]
Un proceso estoc\'astico $X=\left\{X\left(t\right):t\geq0\right\}$ es llamado regenerativo is existe una variable aleatoria $R_{1}>0$ tal que
\begin{itemize}
\item[i)] $\left\{X\left(t+R_{1}\right):t\geq0\right\}$ es independiente de $\left\{\left\{X\left(t\right):t<R_{1}\right\},\right\}$
\item[ii)] $\left\{X\left(t+R_{1}\right):t\geq0\right\}$ es estoc\'asticamente equivalente a $\left\{X\left(t\right):t>0\right\}$
\end{itemize}

Llamamos a $R_{1}$ tiempo de regeneraci\'on, y decimos que $X$ se regenera en este punto.
\end{Def}

$\left\{X\left(t+R_{1}\right)\right\}$ es regenerativo con tiempo de regeneraci\'on $R_{2}$, independiente de $R_{1}$ pero con la misma distribuci\'on que $R_{1}$. Procediendo de esta manera se obtiene una secuencia de variables aleatorias independientes e id\'enticamente distribuidas $\left\{R_{n}\right\}$ llamados longitudes de ciclo. Si definimos a $Z_{k}\equiv R_{1}+R_{2}+\cdots+R_{k}$, se tiene un proceso de renovaci\'on llamado proceso de renovaci\'on encajado para $X$.


\begin{Note}
La existencia de un primer tiempo de regeneraci\'on, $R_{1}$, implica la existencia de una sucesi\'on completa de estos tiempos $R_{1},R_{2}\ldots,$ que satisfacen la propiedad deseada \cite{Sigman2}.
\end{Note}


\begin{Note} Para la cola $GI/GI/1$ los usuarios arriban con tiempos $t_{n}$ y son atendidos con tiempos de servicio $S_{n}$, los tiempos de arribo forman un proceso de renovaci\'on  con tiempos entre arribos independientes e identicamente distribuidos (\texttt{i.i.d.})$T_{n}=t_{n}-t_{n-1}$, adem\'as los tiempos de servicio son \texttt{i.i.d.} e independientes de los procesos de arribo. Por \textit{estable} se entiende que $\esp S_{n}<\esp T_{n}<\infty$.
\end{Note}
 


\begin{Def}
Para $x$ fijo y para cada $t\geq0$, sea $I_{x}\left(t\right)=1$ si $X\left(t\right)\leq x$,  $I_{x}\left(t\right)=0$ en caso contrario, y def\'inanse los tiempos promedio
\begin{eqnarray*}
\overline{X}&=&lim_{t\rightarrow\infty}\frac{1}{t}\int_{0}^{\infty}X\left(u\right)du\\
\prob\left(X_{\infty}\leq x\right)&=&lim_{t\rightarrow\infty}\frac{1}{t}\int_{0}^{\infty}I_{x}\left(u\right)du,
\end{eqnarray*}
cuando estos l\'imites existan.
\end{Def}

Como consecuencia del teorema de Renovaci\'on-Recompensa, se tiene que el primer l\'imite  existe y es igual a la constante
\begin{eqnarray*}
\overline{X}&=&\frac{\esp\left[\int_{0}^{R_{1}}X\left(t\right)dt\right]}{\esp\left[R_{1}\right]},
\end{eqnarray*}
suponiendo que ambas esperanzas son finitas.
 
\begin{Note}
Funciones de procesos regenerativos son regenerativas, es decir, si $X\left(t\right)$ es regenerativo y se define el proceso $Y\left(t\right)$ por $Y\left(t\right)=f\left(X\left(t\right)\right)$ para alguna funci\'on Borel medible $f\left(\cdot\right)$. Adem\'as $Y$ es regenerativo con los mismos tiempos de renovaci\'on que $X$. 

En general, los tiempos de renovaci\'on, $Z_{k}$ de un proceso regenerativo no requieren ser tiempos de paro con respecto a la evoluci\'on de $X\left(t\right)$.
\end{Note} 

\begin{Note}
Una funci\'on de un proceso de Markov, usualmente no ser\'a un proceso de Markov, sin embargo ser\'a regenerativo si el proceso de Markov lo es.
\end{Note}

 
\begin{Note}
Un proceso regenerativo con media de la longitud de ciclo finita es llamado positivo recurrente.
\end{Note}


\begin{Note}
\begin{itemize}
\item[a)] Si el proceso regenerativo $X$ es positivo recurrente y tiene trayectorias muestrales no negativas, entonces la ecuaci\'on anterior es v\'alida.
\item[b)] Si $X$ es positivo recurrente regenerativo, podemos construir una \'unica versi\'on estacionaria de este proceso, $X_{e}=\left\{X_{e}\left(t\right)\right\}$, donde $X_{e}$ es un proceso estoc\'astico regenerativo y estrictamente estacionario, con distribuci\'on marginal distribuida como $X_{\infty}$
\end{itemize}
\end{Note}


%__________________________________________________________________________________________
%\subsection*{Procesos Regenerativos Estacionarios - Stidham \cite{Stidham}}
%__________________________________________________________________________________________


Un proceso estoc\'astico a tiempo continuo $\left\{V\left(t\right),t\geq0\right\}$ es un proceso regenerativo si existe una sucesi\'on de variables aleatorias independientes e id\'enticamente distribuidas $\left\{X_{1},X_{2},\ldots\right\}$, sucesi\'on de renovaci\'on, tal que para cualquier conjunto de Borel $A$, 

\begin{eqnarray*}
\prob\left\{V\left(t\right)\in A|X_{1}+X_{2}+\cdots+X_{R\left(t\right)}=s,\left\{V\left(\tau\right),\tau<s\right\}\right\}=\prob\left\{V\left(t-s\right)\in A|X_{1}>t-s\right\},
\end{eqnarray*}
para todo $0\leq s\leq t$, donde $R\left(t\right)=\max\left\{X_{1}+X_{2}+\cdots+X_{j}\leq t\right\}=$n\'umero de renovaciones ({\emph{puntos de regeneraci\'on}}) que ocurren en $\left[0,t\right]$. El intervalo $\left[0,X_{1}\right)$ es llamado {\emph{primer ciclo de regeneraci\'on}} de $\left\{V\left(t \right),t\geq0\right\}$, $\left[X_{1},X_{1}+X_{2}\right)$ el {\emph{segundo ciclo de regeneraci\'on}}, y as\'i sucesivamente.

Sea $X=X_{1}$ y sea $F$ la funci\'on de distrbuci\'on de $X$


\begin{Def}
Se define el proceso estacionario, $\left\{V^{*}\left(t\right),t\geq0\right\}$, para $\left\{V\left(t\right),t\geq0\right\}$ por

\begin{eqnarray*}
\prob\left\{V\left(t\right)\in A\right\}=\frac{1}{\esp\left[X\right]}\int_{0}^{\infty}\prob\left\{V\left(t+x\right)\in A|X>x\right\}\left(1-F\left(x\right)\right)dx,
\end{eqnarray*} 
para todo $t\geq0$ y todo conjunto de Borel $A$.
\end{Def}

\begin{Def}
Una distribuci\'on se dice que es {\emph{aritm\'etica}} si todos sus puntos de incremento son m\'ultiplos de la forma $0,\lambda, 2\lambda,\ldots$ para alguna $\lambda>0$ entera.
\end{Def}


\begin{Def}
Una modificaci\'on medible de un proceso $\left\{V\left(t\right),t\geq0\right\}$, es una versi\'on de este, $\left\{V\left(t,w\right)\right\}$ conjuntamente medible para $t\geq0$ y para $w\in S$, $S$ espacio de estados para $\left\{V\left(t\right),t\geq0\right\}$.
\end{Def}

\begin{Teo}
Sea $\left\{V\left(t\right),t\geq\right\}$ un proceso regenerativo no negativo con modificaci\'on medible. Sea $\esp\left[X\right]<\infty$. Entonces el proceso estacionario dado por la ecuaci\'on anterior est\'a bien definido y tiene funci\'on de distribuci\'on independiente de $t$, adem\'as
\begin{itemize}
\item[i)] \begin{eqnarray*}
\esp\left[V^{*}\left(0\right)\right]&=&\frac{\esp\left[\int_{0}^{X}V\left(s\right)ds\right]}{\esp\left[X\right]}\end{eqnarray*}
\item[ii)] Si $\esp\left[V^{*}\left(0\right)\right]<\infty$, equivalentemente, si $\esp\left[\int_{0}^{X}V\left(s\right)ds\right]<\infty$,entonces
\begin{eqnarray*}
\frac{\int_{0}^{t}V\left(s\right)ds}{t}\rightarrow\frac{\esp\left[\int_{0}^{X}V\left(s\right)ds\right]}{\esp\left[X\right]}
\end{eqnarray*}
con probabilidad 1 y en media, cuando $t\rightarrow\infty$.
\end{itemize}
\end{Teo}

\begin{Coro}
Sea $\left\{V\left(t\right),t\geq0\right\}$ un proceso regenerativo no negativo, con modificaci\'on medible. Si $\esp <\infty$, $F$ es no-aritm\'etica, y para todo $x\geq0$, $P\left\{V\left(t\right)\leq x,C>x\right\}$ es de variaci\'on acotada como funci\'on de $t$ en cada intervalo finito $\left[0,\tau\right]$, entonces $V\left(t\right)$ converge en distribuci\'on  cuando $t\rightarrow\infty$ y $$\esp V=\frac{\esp \int_{0}^{X}V\left(s\right)ds}{\esp X}$$
Donde $V$ tiene la distribuci\'on l\'imite de $V\left(t\right)$ cuando $t\rightarrow\infty$.

\end{Coro}

Para el caso discreto se tienen resultados similares.



%______________________________________________________________________
%\section{Procesos de Renovaci\'on}
%______________________________________________________________________

\begin{Def}\label{Def.Tn}
Sean $0\leq T_{1}\leq T_{2}\leq \ldots$ son tiempos aleatorios infinitos en los cuales ocurren ciertos eventos. El n\'umero de tiempos $T_{n}$ en el intervalo $\left[0,t\right)$ es

\begin{eqnarray}
N\left(t\right)=\sum_{n=1}^{\infty}\indora\left(T_{n}\leq t\right),
\end{eqnarray}
para $t\geq0$.
\end{Def}

Si se consideran los puntos $T_{n}$ como elementos de $\rea_{+}$, y $N\left(t\right)$ es el n\'umero de puntos en $\rea$. El proceso denotado por $\left\{N\left(t\right):t\geq0\right\}$, denotado por $N\left(t\right)$, es un proceso puntual en $\rea_{+}$. Los $T_{n}$ son los tiempos de ocurrencia, el proceso puntual $N\left(t\right)$ es simple si su n\'umero de ocurrencias son distintas: $0<T_{1}<T_{2}<\ldots$ casi seguramente.

\begin{Def}
Un proceso puntual $N\left(t\right)$ es un proceso de renovaci\'on si los tiempos de interocurrencia $\xi_{n}=T_{n}-T_{n-1}$, para $n\geq1$, son independientes e identicamente distribuidos con distribuci\'on $F$, donde $F\left(0\right)=0$ y $T_{0}=0$. Los $T_{n}$ son llamados tiempos de renovaci\'on, referente a la independencia o renovaci\'on de la informaci\'on estoc\'astica en estos tiempos. Los $\xi_{n}$ son los tiempos de inter-renovaci\'on, y $N\left(t\right)$ es el n\'umero de renovaciones en el intervalo $\left[0,t\right)$
\end{Def}


\begin{Note}
Para definir un proceso de renovaci\'on para cualquier contexto, solamente hay que especificar una distribuci\'on $F$, con $F\left(0\right)=0$, para los tiempos de inter-renovaci\'on. La funci\'on $F$ en turno degune las otra variables aleatorias. De manera formal, existe un espacio de probabilidad y una sucesi\'on de variables aleatorias $\xi_{1},\xi_{2},\ldots$ definidas en este con distribuci\'on $F$. Entonces las otras cantidades son $T_{n}=\sum_{k=1}^{n}\xi_{k}$ y $N\left(t\right)=\sum_{n=1}^{\infty}\indora\left(T_{n}\leq t\right)$, donde $T_{n}\rightarrow\infty$ casi seguramente por la Ley Fuerte de los Grandes Números.
\end{Note}

%___________________________________________________________________________________________
%
%\subsection*{Teorema Principal de Renovaci\'on}
%___________________________________________________________________________________________
%

\begin{Note} Una funci\'on $h:\rea_{+}\rightarrow\rea$ es Directamente Riemann Integrable en los siguientes casos:
\begin{itemize}
\item[a)] $h\left(t\right)\geq0$ es decreciente y Riemann Integrable.
\item[b)] $h$ es continua excepto posiblemente en un conjunto de Lebesgue de medida 0, y $|h\left(t\right)|\leq b\left(t\right)$, donde $b$ es DRI.
\end{itemize}
\end{Note}

\begin{Teo}[Teorema Principal de Renovaci\'on]
Si $F$ es no aritm\'etica y $h\left(t\right)$ es Directamente Riemann Integrable (DRI), entonces

\begin{eqnarray*}
lim_{t\rightarrow\infty}U\star h=\frac{1}{\mu}\int_{\rea_{+}}h\left(s\right)ds.
\end{eqnarray*}
\end{Teo}

\begin{Prop}
Cualquier funci\'on $H\left(t\right)$ acotada en intervalos finitos y que es 0 para $t<0$ puede expresarse como
\begin{eqnarray*}
H\left(t\right)=U\star h\left(t\right)\textrm{,  donde }h\left(t\right)=H\left(t\right)-F\star H\left(t\right)
\end{eqnarray*}
\end{Prop}

\begin{Def}
Un proceso estoc\'astico $X\left(t\right)$ es crudamente regenerativo en un tiempo aleatorio positivo $T$ si
\begin{eqnarray*}
\esp\left[X\left(T+t\right)|T\right]=\esp\left[X\left(t\right)\right]\textrm{, para }t\geq0,\end{eqnarray*}
y con las esperanzas anteriores finitas.
\end{Def}

\begin{Prop}
Sup\'ongase que $X\left(t\right)$ es un proceso crudamente regenerativo en $T$, que tiene distribuci\'on $F$. Si $\esp\left[X\left(t\right)\right]$ es acotado en intervalos finitos, entonces
\begin{eqnarray*}
\esp\left[X\left(t\right)\right]=U\star h\left(t\right)\textrm{,  donde }h\left(t\right)=\esp\left[X\left(t\right)\indora\left(T>t\right)\right].
\end{eqnarray*}
\end{Prop}

\begin{Teo}[Regeneraci\'on Cruda]
Sup\'ongase que $X\left(t\right)$ es un proceso con valores positivo crudamente regenerativo en $T$, y def\'inase $M=\sup\left\{|X\left(t\right)|:t\leq T\right\}$. Si $T$ es no aritm\'etico y $M$ y $MT$ tienen media finita, entonces
\begin{eqnarray*}
lim_{t\rightarrow\infty}\esp\left[X\left(t\right)\right]=\frac{1}{\mu}\int_{\rea_{+}}h\left(s\right)ds,
\end{eqnarray*}
donde $h\left(t\right)=\esp\left[X\left(t\right)\indora\left(T>t\right)\right]$.
\end{Teo}

%___________________________________________________________________________________________
%
%\subsection*{Propiedades de los Procesos de Renovaci\'on}
%___________________________________________________________________________________________
%

Los tiempos $T_{n}$ est\'an relacionados con los conteos de $N\left(t\right)$ por

\begin{eqnarray*}
\left\{N\left(t\right)\geq n\right\}&=&\left\{T_{n}\leq t\right\}\\
T_{N\left(t\right)}\leq &t&<T_{N\left(t\right)+1},
\end{eqnarray*}

adem\'as $N\left(T_{n}\right)=n$, y 

\begin{eqnarray*}
N\left(t\right)=\max\left\{n:T_{n}\leq t\right\}=\min\left\{n:T_{n+1}>t\right\}
\end{eqnarray*}

Por propiedades de la convoluci\'on se sabe que

\begin{eqnarray*}
P\left\{T_{n}\leq t\right\}=F^{n\star}\left(t\right)
\end{eqnarray*}
que es la $n$-\'esima convoluci\'on de $F$. Entonces 

\begin{eqnarray*}
\left\{N\left(t\right)\geq n\right\}&=&\left\{T_{n}\leq t\right\}\\
P\left\{N\left(t\right)\leq n\right\}&=&1-F^{\left(n+1\right)\star}\left(t\right)
\end{eqnarray*}

Adem\'as usando el hecho de que $\esp\left[N\left(t\right)\right]=\sum_{n=1}^{\infty}P\left\{N\left(t\right)\geq n\right\}$
se tiene que

\begin{eqnarray*}
\esp\left[N\left(t\right)\right]=\sum_{n=1}^{\infty}F^{n\star}\left(t\right)
\end{eqnarray*}

\begin{Prop}
Para cada $t\geq0$, la funci\'on generadora de momentos $\esp\left[e^{\alpha N\left(t\right)}\right]$ existe para alguna $\alpha$ en una vecindad del 0, y de aqu\'i que $\esp\left[N\left(t\right)^{m}\right]<\infty$, para $m\geq1$.
\end{Prop}


\begin{Note}
Si el primer tiempo de renovaci\'on $\xi_{1}$ no tiene la misma distribuci\'on que el resto de las $\xi_{n}$, para $n\geq2$, a $N\left(t\right)$ se le llama Proceso de Renovaci\'on retardado, donde si $\xi$ tiene distribuci\'on $G$, entonces el tiempo $T_{n}$ de la $n$-\'esima renovaci\'on tiene distribuci\'on $G\star F^{\left(n-1\right)\star}\left(t\right)$
\end{Note}


\begin{Teo}
Para una constante $\mu\leq\infty$ ( o variable aleatoria), las siguientes expresiones son equivalentes:

\begin{eqnarray}
lim_{n\rightarrow\infty}n^{-1}T_{n}&=&\mu,\textrm{ c.s.}\\
lim_{t\rightarrow\infty}t^{-1}N\left(t\right)&=&1/\mu,\textrm{ c.s.}
\end{eqnarray}
\end{Teo}


Es decir, $T_{n}$ satisface la Ley Fuerte de los Grandes N\'umeros s\'i y s\'olo s\'i $N\left/t\right)$ la cumple.


\begin{Coro}[Ley Fuerte de los Grandes N\'umeros para Procesos de Renovaci\'on]
Si $N\left(t\right)$ es un proceso de renovaci\'on cuyos tiempos de inter-renovaci\'on tienen media $\mu\leq\infty$, entonces
\begin{eqnarray}
t^{-1}N\left(t\right)\rightarrow 1/\mu,\textrm{ c.s. cuando }t\rightarrow\infty.
\end{eqnarray}

\end{Coro}


Considerar el proceso estoc\'astico de valores reales $\left\{Z\left(t\right):t\geq0\right\}$ en el mismo espacio de probabilidad que $N\left(t\right)$

\begin{Def}
Para el proceso $\left\{Z\left(t\right):t\geq0\right\}$ se define la fluctuaci\'on m\'axima de $Z\left(t\right)$ en el intervalo $\left(T_{n-1},T_{n}\right]$:
\begin{eqnarray*}
M_{n}=\sup_{T_{n-1}<t\leq T_{n}}|Z\left(t\right)-Z\left(T_{n-1}\right)|
\end{eqnarray*}
\end{Def}

\begin{Teo}
Sup\'ongase que $n^{-1}T_{n}\rightarrow\mu$ c.s. cuando $n\rightarrow\infty$, donde $\mu\leq\infty$ es una constante o variable aleatoria. Sea $a$ una constante o variable aleatoria que puede ser infinita cuando $\mu$ es finita, y considere las expresiones l\'imite:
\begin{eqnarray}
lim_{n\rightarrow\infty}n^{-1}Z\left(T_{n}\right)&=&a,\textrm{ c.s.}\\
lim_{t\rightarrow\infty}t^{-1}Z\left(t\right)&=&a/\mu,\textrm{ c.s.}
\end{eqnarray}
La segunda expresi\'on implica la primera. Conversamente, la primera implica la segunda si el proceso $Z\left(t\right)$ es creciente, o si $lim_{n\rightarrow\infty}n^{-1}M_{n}=0$ c.s.
\end{Teo}

\begin{Coro}
Si $N\left(t\right)$ es un proceso de renovaci\'on, y $\left(Z\left(T_{n}\right)-Z\left(T_{n-1}\right),M_{n}\right)$, para $n\geq1$, son variables aleatorias independientes e id\'enticamente distribuidas con media finita, entonces,
\begin{eqnarray}
lim_{t\rightarrow\infty}t^{-1}Z\left(t\right)\rightarrow\frac{\esp\left[Z\left(T_{1}\right)-Z\left(T_{0}\right)\right]}{\esp\left[T_{1}\right]},\textrm{ c.s. cuando  }t\rightarrow\infty.
\end{eqnarray}
\end{Coro}



%___________________________________________________________________________________________
%
%\subsection{Propiedades de los Procesos de Renovaci\'on}
%___________________________________________________________________________________________
%

Los tiempos $T_{n}$ est\'an relacionados con los conteos de $N\left(t\right)$ por

\begin{eqnarray*}
\left\{N\left(t\right)\geq n\right\}&=&\left\{T_{n}\leq t\right\}\\
T_{N\left(t\right)}\leq &t&<T_{N\left(t\right)+1},
\end{eqnarray*}

adem\'as $N\left(T_{n}\right)=n$, y 

\begin{eqnarray*}
N\left(t\right)=\max\left\{n:T_{n}\leq t\right\}=\min\left\{n:T_{n+1}>t\right\}
\end{eqnarray*}

Por propiedades de la convoluci\'on se sabe que

\begin{eqnarray*}
P\left\{T_{n}\leq t\right\}=F^{n\star}\left(t\right)
\end{eqnarray*}
que es la $n$-\'esima convoluci\'on de $F$. Entonces 

\begin{eqnarray*}
\left\{N\left(t\right)\geq n\right\}&=&\left\{T_{n}\leq t\right\}\\
P\left\{N\left(t\right)\leq n\right\}&=&1-F^{\left(n+1\right)\star}\left(t\right)
\end{eqnarray*}

Adem\'as usando el hecho de que $\esp\left[N\left(t\right)\right]=\sum_{n=1}^{\infty}P\left\{N\left(t\right)\geq n\right\}$
se tiene que

\begin{eqnarray*}
\esp\left[N\left(t\right)\right]=\sum_{n=1}^{\infty}F^{n\star}\left(t\right)
\end{eqnarray*}

\begin{Prop}
Para cada $t\geq0$, la funci\'on generadora de momentos $\esp\left[e^{\alpha N\left(t\right)}\right]$ existe para alguna $\alpha$ en una vecindad del 0, y de aqu\'i que $\esp\left[N\left(t\right)^{m}\right]<\infty$, para $m\geq1$.
\end{Prop}


\begin{Note}
Si el primer tiempo de renovaci\'on $\xi_{1}$ no tiene la misma distribuci\'on que el resto de las $\xi_{n}$, para $n\geq2$, a $N\left(t\right)$ se le llama Proceso de Renovaci\'on retardado, donde si $\xi$ tiene distribuci\'on $G$, entonces el tiempo $T_{n}$ de la $n$-\'esima renovaci\'on tiene distribuci\'on $G\star F^{\left(n-1\right)\star}\left(t\right)$
\end{Note}


\begin{Teo}
Para una constante $\mu\leq\infty$ ( o variable aleatoria), las siguientes expresiones son equivalentes:

\begin{eqnarray}
lim_{n\rightarrow\infty}n^{-1}T_{n}&=&\mu,\textrm{ c.s.}\\
lim_{t\rightarrow\infty}t^{-1}N\left(t\right)&=&1/\mu,\textrm{ c.s.}
\end{eqnarray}
\end{Teo}


Es decir, $T_{n}$ satisface la Ley Fuerte de los Grandes N\'umeros s\'i y s\'olo s\'i $N\left/t\right)$ la cumple.


\begin{Coro}[Ley Fuerte de los Grandes N\'umeros para Procesos de Renovaci\'on]
Si $N\left(t\right)$ es un proceso de renovaci\'on cuyos tiempos de inter-renovaci\'on tienen media $\mu\leq\infty$, entonces
\begin{eqnarray}
t^{-1}N\left(t\right)\rightarrow 1/\mu,\textrm{ c.s. cuando }t\rightarrow\infty.
\end{eqnarray}

\end{Coro}


Considerar el proceso estoc\'astico de valores reales $\left\{Z\left(t\right):t\geq0\right\}$ en el mismo espacio de probabilidad que $N\left(t\right)$

\begin{Def}
Para el proceso $\left\{Z\left(t\right):t\geq0\right\}$ se define la fluctuaci\'on m\'axima de $Z\left(t\right)$ en el intervalo $\left(T_{n-1},T_{n}\right]$:
\begin{eqnarray*}
M_{n}=\sup_{T_{n-1}<t\leq T_{n}}|Z\left(t\right)-Z\left(T_{n-1}\right)|
\end{eqnarray*}
\end{Def}

\begin{Teo}
Sup\'ongase que $n^{-1}T_{n}\rightarrow\mu$ c.s. cuando $n\rightarrow\infty$, donde $\mu\leq\infty$ es una constante o variable aleatoria. Sea $a$ una constante o variable aleatoria que puede ser infinita cuando $\mu$ es finita, y considere las expresiones l\'imite:
\begin{eqnarray}
lim_{n\rightarrow\infty}n^{-1}Z\left(T_{n}\right)&=&a,\textrm{ c.s.}\\
lim_{t\rightarrow\infty}t^{-1}Z\left(t\right)&=&a/\mu,\textrm{ c.s.}
\end{eqnarray}
La segunda expresi\'on implica la primera. Conversamente, la primera implica la segunda si el proceso $Z\left(t\right)$ es creciente, o si $lim_{n\rightarrow\infty}n^{-1}M_{n}=0$ c.s.
\end{Teo}

\begin{Coro}
Si $N\left(t\right)$ es un proceso de renovaci\'on, y $\left(Z\left(T_{n}\right)-Z\left(T_{n-1}\right),M_{n}\right)$, para $n\geq1$, son variables aleatorias independientes e id\'enticamente distribuidas con media finita, entonces,
\begin{eqnarray}
lim_{t\rightarrow\infty}t^{-1}Z\left(t\right)\rightarrow\frac{\esp\left[Z\left(T_{1}\right)-Z\left(T_{0}\right)\right]}{\esp\left[T_{1}\right]},\textrm{ c.s. cuando  }t\rightarrow\infty.
\end{eqnarray}
\end{Coro}


%___________________________________________________________________________________________
%
%\subsection{Propiedades de los Procesos de Renovaci\'on}
%___________________________________________________________________________________________
%

Los tiempos $T_{n}$ est\'an relacionados con los conteos de $N\left(t\right)$ por

\begin{eqnarray*}
\left\{N\left(t\right)\geq n\right\}&=&\left\{T_{n}\leq t\right\}\\
T_{N\left(t\right)}\leq &t&<T_{N\left(t\right)+1},
\end{eqnarray*}

adem\'as $N\left(T_{n}\right)=n$, y 

\begin{eqnarray*}
N\left(t\right)=\max\left\{n:T_{n}\leq t\right\}=\min\left\{n:T_{n+1}>t\right\}
\end{eqnarray*}

Por propiedades de la convoluci\'on se sabe que

\begin{eqnarray*}
P\left\{T_{n}\leq t\right\}=F^{n\star}\left(t\right)
\end{eqnarray*}
que es la $n$-\'esima convoluci\'on de $F$. Entonces 

\begin{eqnarray*}
\left\{N\left(t\right)\geq n\right\}&=&\left\{T_{n}\leq t\right\}\\
P\left\{N\left(t\right)\leq n\right\}&=&1-F^{\left(n+1\right)\star}\left(t\right)
\end{eqnarray*}

Adem\'as usando el hecho de que $\esp\left[N\left(t\right)\right]=\sum_{n=1}^{\infty}P\left\{N\left(t\right)\geq n\right\}$
se tiene que

\begin{eqnarray*}
\esp\left[N\left(t\right)\right]=\sum_{n=1}^{\infty}F^{n\star}\left(t\right)
\end{eqnarray*}

\begin{Prop}
Para cada $t\geq0$, la funci\'on generadora de momentos $\esp\left[e^{\alpha N\left(t\right)}\right]$ existe para alguna $\alpha$ en una vecindad del 0, y de aqu\'i que $\esp\left[N\left(t\right)^{m}\right]<\infty$, para $m\geq1$.
\end{Prop}


\begin{Note}
Si el primer tiempo de renovaci\'on $\xi_{1}$ no tiene la misma distribuci\'on que el resto de las $\xi_{n}$, para $n\geq2$, a $N\left(t\right)$ se le llama Proceso de Renovaci\'on retardado, donde si $\xi$ tiene distribuci\'on $G$, entonces el tiempo $T_{n}$ de la $n$-\'esima renovaci\'on tiene distribuci\'on $G\star F^{\left(n-1\right)\star}\left(t\right)$
\end{Note}


\begin{Teo}
Para una constante $\mu\leq\infty$ ( o variable aleatoria), las siguientes expresiones son equivalentes:

\begin{eqnarray}
lim_{n\rightarrow\infty}n^{-1}T_{n}&=&\mu,\textrm{ c.s.}\\
lim_{t\rightarrow\infty}t^{-1}N\left(t\right)&=&1/\mu,\textrm{ c.s.}
\end{eqnarray}
\end{Teo}


Es decir, $T_{n}$ satisface la Ley Fuerte de los Grandes N\'umeros s\'i y s\'olo s\'i $N\left/t\right)$ la cumple.


\begin{Coro}[Ley Fuerte de los Grandes N\'umeros para Procesos de Renovaci\'on]
Si $N\left(t\right)$ es un proceso de renovaci\'on cuyos tiempos de inter-renovaci\'on tienen media $\mu\leq\infty$, entonces
\begin{eqnarray}
t^{-1}N\left(t\right)\rightarrow 1/\mu,\textrm{ c.s. cuando }t\rightarrow\infty.
\end{eqnarray}

\end{Coro}


Considerar el proceso estoc\'astico de valores reales $\left\{Z\left(t\right):t\geq0\right\}$ en el mismo espacio de probabilidad que $N\left(t\right)$

\begin{Def}
Para el proceso $\left\{Z\left(t\right):t\geq0\right\}$ se define la fluctuaci\'on m\'axima de $Z\left(t\right)$ en el intervalo $\left(T_{n-1},T_{n}\right]$:
\begin{eqnarray*}
M_{n}=\sup_{T_{n-1}<t\leq T_{n}}|Z\left(t\right)-Z\left(T_{n-1}\right)|
\end{eqnarray*}
\end{Def}

\begin{Teo}
Sup\'ongase que $n^{-1}T_{n}\rightarrow\mu$ c.s. cuando $n\rightarrow\infty$, donde $\mu\leq\infty$ es una constante o variable aleatoria. Sea $a$ una constante o variable aleatoria que puede ser infinita cuando $\mu$ es finita, y considere las expresiones l\'imite:
\begin{eqnarray}
lim_{n\rightarrow\infty}n^{-1}Z\left(T_{n}\right)&=&a,\textrm{ c.s.}\\
lim_{t\rightarrow\infty}t^{-1}Z\left(t\right)&=&a/\mu,\textrm{ c.s.}
\end{eqnarray}
La segunda expresi\'on implica la primera. Conversamente, la primera implica la segunda si el proceso $Z\left(t\right)$ es creciente, o si $lim_{n\rightarrow\infty}n^{-1}M_{n}=0$ c.s.
\end{Teo}

\begin{Coro}
Si $N\left(t\right)$ es un proceso de renovaci\'on, y $\left(Z\left(T_{n}\right)-Z\left(T_{n-1}\right),M_{n}\right)$, para $n\geq1$, son variables aleatorias independientes e id\'enticamente distribuidas con media finita, entonces,
\begin{eqnarray}
lim_{t\rightarrow\infty}t^{-1}Z\left(t\right)\rightarrow\frac{\esp\left[Z\left(T_{1}\right)-Z\left(T_{0}\right)\right]}{\esp\left[T_{1}\right]},\textrm{ c.s. cuando  }t\rightarrow\infty.
\end{eqnarray}
\end{Coro}

%___________________________________________________________________________________________
%
%\subsection{Propiedades de los Procesos de Renovaci\'on}
%___________________________________________________________________________________________
%

Los tiempos $T_{n}$ est\'an relacionados con los conteos de $N\left(t\right)$ por

\begin{eqnarray*}
\left\{N\left(t\right)\geq n\right\}&=&\left\{T_{n}\leq t\right\}\\
T_{N\left(t\right)}\leq &t&<T_{N\left(t\right)+1},
\end{eqnarray*}

adem\'as $N\left(T_{n}\right)=n$, y 

\begin{eqnarray*}
N\left(t\right)=\max\left\{n:T_{n}\leq t\right\}=\min\left\{n:T_{n+1}>t\right\}
\end{eqnarray*}

Por propiedades de la convoluci\'on se sabe que

\begin{eqnarray*}
P\left\{T_{n}\leq t\right\}=F^{n\star}\left(t\right)
\end{eqnarray*}
que es la $n$-\'esima convoluci\'on de $F$. Entonces 

\begin{eqnarray*}
\left\{N\left(t\right)\geq n\right\}&=&\left\{T_{n}\leq t\right\}\\
P\left\{N\left(t\right)\leq n\right\}&=&1-F^{\left(n+1\right)\star}\left(t\right)
\end{eqnarray*}

Adem\'as usando el hecho de que $\esp\left[N\left(t\right)\right]=\sum_{n=1}^{\infty}P\left\{N\left(t\right)\geq n\right\}$
se tiene que

\begin{eqnarray*}
\esp\left[N\left(t\right)\right]=\sum_{n=1}^{\infty}F^{n\star}\left(t\right)
\end{eqnarray*}

\begin{Prop}
Para cada $t\geq0$, la funci\'on generadora de momentos $\esp\left[e^{\alpha N\left(t\right)}\right]$ existe para alguna $\alpha$ en una vecindad del 0, y de aqu\'i que $\esp\left[N\left(t\right)^{m}\right]<\infty$, para $m\geq1$.
\end{Prop}


\begin{Note}
Si el primer tiempo de renovaci\'on $\xi_{1}$ no tiene la misma distribuci\'on que el resto de las $\xi_{n}$, para $n\geq2$, a $N\left(t\right)$ se le llama Proceso de Renovaci\'on retardado, donde si $\xi$ tiene distribuci\'on $G$, entonces el tiempo $T_{n}$ de la $n$-\'esima renovaci\'on tiene distribuci\'on $G\star F^{\left(n-1\right)\star}\left(t\right)$
\end{Note}


\begin{Teo}
Para una constante $\mu\leq\infty$ ( o variable aleatoria), las siguientes expresiones son equivalentes:

\begin{eqnarray}
lim_{n\rightarrow\infty}n^{-1}T_{n}&=&\mu,\textrm{ c.s.}\\
lim_{t\rightarrow\infty}t^{-1}N\left(t\right)&=&1/\mu,\textrm{ c.s.}
\end{eqnarray}
\end{Teo}


Es decir, $T_{n}$ satisface la Ley Fuerte de los Grandes N\'umeros s\'i y s\'olo s\'i $N\left/t\right)$ la cumple.


\begin{Coro}[Ley Fuerte de los Grandes N\'umeros para Procesos de Renovaci\'on]
Si $N\left(t\right)$ es un proceso de renovaci\'on cuyos tiempos de inter-renovaci\'on tienen media $\mu\leq\infty$, entonces
\begin{eqnarray}
t^{-1}N\left(t\right)\rightarrow 1/\mu,\textrm{ c.s. cuando }t\rightarrow\infty.
\end{eqnarray}

\end{Coro}


Considerar el proceso estoc\'astico de valores reales $\left\{Z\left(t\right):t\geq0\right\}$ en el mismo espacio de probabilidad que $N\left(t\right)$

\begin{Def}
Para el proceso $\left\{Z\left(t\right):t\geq0\right\}$ se define la fluctuaci\'on m\'axima de $Z\left(t\right)$ en el intervalo $\left(T_{n-1},T_{n}\right]$:
\begin{eqnarray*}
M_{n}=\sup_{T_{n-1}<t\leq T_{n}}|Z\left(t\right)-Z\left(T_{n-1}\right)|
\end{eqnarray*}
\end{Def}

\begin{Teo}
Sup\'ongase que $n^{-1}T_{n}\rightarrow\mu$ c.s. cuando $n\rightarrow\infty$, donde $\mu\leq\infty$ es una constante o variable aleatoria. Sea $a$ una constante o variable aleatoria que puede ser infinita cuando $\mu$ es finita, y considere las expresiones l\'imite:
\begin{eqnarray}
lim_{n\rightarrow\infty}n^{-1}Z\left(T_{n}\right)&=&a,\textrm{ c.s.}\\
lim_{t\rightarrow\infty}t^{-1}Z\left(t\right)&=&a/\mu,\textrm{ c.s.}
\end{eqnarray}
La segunda expresi\'on implica la primera. Conversamente, la primera implica la segunda si el proceso $Z\left(t\right)$ es creciente, o si $lim_{n\rightarrow\infty}n^{-1}M_{n}=0$ c.s.
\end{Teo}

\begin{Coro}
Si $N\left(t\right)$ es un proceso de renovaci\'on, y $\left(Z\left(T_{n}\right)-Z\left(T_{n-1}\right),M_{n}\right)$, para $n\geq1$, son variables aleatorias independientes e id\'enticamente distribuidas con media finita, entonces,
\begin{eqnarray}
lim_{t\rightarrow\infty}t^{-1}Z\left(t\right)\rightarrow\frac{\esp\left[Z\left(T_{1}\right)-Z\left(T_{0}\right)\right]}{\esp\left[T_{1}\right]},\textrm{ c.s. cuando  }t\rightarrow\infty.
\end{eqnarray}
\end{Coro}
%___________________________________________________________________________________________
%
%\subsection{Propiedades de los Procesos de Renovaci\'on}
%___________________________________________________________________________________________
%

Los tiempos $T_{n}$ est\'an relacionados con los conteos de $N\left(t\right)$ por

\begin{eqnarray*}
\left\{N\left(t\right)\geq n\right\}&=&\left\{T_{n}\leq t\right\}\\
T_{N\left(t\right)}\leq &t&<T_{N\left(t\right)+1},
\end{eqnarray*}

adem\'as $N\left(T_{n}\right)=n$, y 

\begin{eqnarray*}
N\left(t\right)=\max\left\{n:T_{n}\leq t\right\}=\min\left\{n:T_{n+1}>t\right\}
\end{eqnarray*}

Por propiedades de la convoluci\'on se sabe que

\begin{eqnarray*}
P\left\{T_{n}\leq t\right\}=F^{n\star}\left(t\right)
\end{eqnarray*}
que es la $n$-\'esima convoluci\'on de $F$. Entonces 

\begin{eqnarray*}
\left\{N\left(t\right)\geq n\right\}&=&\left\{T_{n}\leq t\right\}\\
P\left\{N\left(t\right)\leq n\right\}&=&1-F^{\left(n+1\right)\star}\left(t\right)
\end{eqnarray*}

Adem\'as usando el hecho de que $\esp\left[N\left(t\right)\right]=\sum_{n=1}^{\infty}P\left\{N\left(t\right)\geq n\right\}$
se tiene que

\begin{eqnarray*}
\esp\left[N\left(t\right)\right]=\sum_{n=1}^{\infty}F^{n\star}\left(t\right)
\end{eqnarray*}

\begin{Prop}
Para cada $t\geq0$, la funci\'on generadora de momentos $\esp\left[e^{\alpha N\left(t\right)}\right]$ existe para alguna $\alpha$ en una vecindad del 0, y de aqu\'i que $\esp\left[N\left(t\right)^{m}\right]<\infty$, para $m\geq1$.
\end{Prop}


\begin{Note}
Si el primer tiempo de renovaci\'on $\xi_{1}$ no tiene la misma distribuci\'on que el resto de las $\xi_{n}$, para $n\geq2$, a $N\left(t\right)$ se le llama Proceso de Renovaci\'on retardado, donde si $\xi$ tiene distribuci\'on $G$, entonces el tiempo $T_{n}$ de la $n$-\'esima renovaci\'on tiene distribuci\'on $G\star F^{\left(n-1\right)\star}\left(t\right)$
\end{Note}


\begin{Teo}
Para una constante $\mu\leq\infty$ ( o variable aleatoria), las siguientes expresiones son equivalentes:

\begin{eqnarray}
lim_{n\rightarrow\infty}n^{-1}T_{n}&=&\mu,\textrm{ c.s.}\\
lim_{t\rightarrow\infty}t^{-1}N\left(t\right)&=&1/\mu,\textrm{ c.s.}
\end{eqnarray}
\end{Teo}


Es decir, $T_{n}$ satisface la Ley Fuerte de los Grandes N\'umeros s\'i y s\'olo s\'i $N\left/t\right)$ la cumple.


\begin{Coro}[Ley Fuerte de los Grandes N\'umeros para Procesos de Renovaci\'on]
Si $N\left(t\right)$ es un proceso de renovaci\'on cuyos tiempos de inter-renovaci\'on tienen media $\mu\leq\infty$, entonces
\begin{eqnarray}
t^{-1}N\left(t\right)\rightarrow 1/\mu,\textrm{ c.s. cuando }t\rightarrow\infty.
\end{eqnarray}

\end{Coro}


Considerar el proceso estoc\'astico de valores reales $\left\{Z\left(t\right):t\geq0\right\}$ en el mismo espacio de probabilidad que $N\left(t\right)$

\begin{Def}
Para el proceso $\left\{Z\left(t\right):t\geq0\right\}$ se define la fluctuaci\'on m\'axima de $Z\left(t\right)$ en el intervalo $\left(T_{n-1},T_{n}\right]$:
\begin{eqnarray*}
M_{n}=\sup_{T_{n-1}<t\leq T_{n}}|Z\left(t\right)-Z\left(T_{n-1}\right)|
\end{eqnarray*}
\end{Def}

\begin{Teo}
Sup\'ongase que $n^{-1}T_{n}\rightarrow\mu$ c.s. cuando $n\rightarrow\infty$, donde $\mu\leq\infty$ es una constante o variable aleatoria. Sea $a$ una constante o variable aleatoria que puede ser infinita cuando $\mu$ es finita, y considere las expresiones l\'imite:
\begin{eqnarray}
lim_{n\rightarrow\infty}n^{-1}Z\left(T_{n}\right)&=&a,\textrm{ c.s.}\\
lim_{t\rightarrow\infty}t^{-1}Z\left(t\right)&=&a/\mu,\textrm{ c.s.}
\end{eqnarray}
La segunda expresi\'on implica la primera. Conversamente, la primera implica la segunda si el proceso $Z\left(t\right)$ es creciente, o si $lim_{n\rightarrow\infty}n^{-1}M_{n}=0$ c.s.
\end{Teo}

\begin{Coro}
Si $N\left(t\right)$ es un proceso de renovaci\'on, y $\left(Z\left(T_{n}\right)-Z\left(T_{n-1}\right),M_{n}\right)$, para $n\geq1$, son variables aleatorias independientes e id\'enticamente distribuidas con media finita, entonces,
\begin{eqnarray}
lim_{t\rightarrow\infty}t^{-1}Z\left(t\right)\rightarrow\frac{\esp\left[Z\left(T_{1}\right)-Z\left(T_{0}\right)\right]}{\esp\left[T_{1}\right]},\textrm{ c.s. cuando  }t\rightarrow\infty.
\end{eqnarray}
\end{Coro}


%___________________________________________________________________________________________
%
%\subsection*{Funci\'on de Renovaci\'on}
%___________________________________________________________________________________________
%


\begin{Def}
Sea $h\left(t\right)$ funci\'on de valores reales en $\rea$ acotada en intervalos finitos e igual a cero para $t<0$ La ecuaci\'on de renovaci\'on para $h\left(t\right)$ y la distribuci\'on $F$ es

\begin{eqnarray}\label{Ec.Renovacion}
H\left(t\right)=h\left(t\right)+\int_{\left[0,t\right]}H\left(t-s\right)dF\left(s\right)\textrm{,    }t\geq0,
\end{eqnarray}
donde $H\left(t\right)$ es una funci\'on de valores reales. Esto es $H=h+F\star H$. Decimos que $H\left(t\right)$ es soluci\'on de esta ecuaci\'on si satisface la ecuaci\'on, y es acotada en intervalos finitos e iguales a cero para $t<0$.
\end{Def}

\begin{Prop}
La funci\'on $U\star h\left(t\right)$ es la \'unica soluci\'on de la ecuaci\'on de renovaci\'on (\ref{Ec.Renovacion}).
\end{Prop}

\begin{Teo}[Teorema Renovaci\'on Elemental]
\begin{eqnarray*}
t^{-1}U\left(t\right)\rightarrow 1/\mu\textrm{,    cuando }t\rightarrow\infty.
\end{eqnarray*}
\end{Teo}

%___________________________________________________________________________________________
%
%\subsection{Funci\'on de Renovaci\'on}
%___________________________________________________________________________________________
%


Sup\'ongase que $N\left(t\right)$ es un proceso de renovaci\'on con distribuci\'on $F$ con media finita $\mu$.

\begin{Def}
La funci\'on de renovaci\'on asociada con la distribuci\'on $F$, del proceso $N\left(t\right)$, es
\begin{eqnarray*}
U\left(t\right)=\sum_{n=1}^{\infty}F^{n\star}\left(t\right),\textrm{   }t\geq0,
\end{eqnarray*}
donde $F^{0\star}\left(t\right)=\indora\left(t\geq0\right)$.
\end{Def}


\begin{Prop}
Sup\'ongase que la distribuci\'on de inter-renovaci\'on $F$ tiene densidad $f$. Entonces $U\left(t\right)$ tambi\'en tiene densidad, para $t>0$, y es $U^{'}\left(t\right)=\sum_{n=0}^{\infty}f^{n\star}\left(t\right)$. Adem\'as
\begin{eqnarray*}
\prob\left\{N\left(t\right)>N\left(t-\right)\right\}=0\textrm{,   }t\geq0.
\end{eqnarray*}
\end{Prop}

\begin{Def}
La Transformada de Laplace-Stieljes de $F$ est\'a dada por

\begin{eqnarray*}
\hat{F}\left(\alpha\right)=\int_{\rea_{+}}e^{-\alpha t}dF\left(t\right)\textrm{,  }\alpha\geq0.
\end{eqnarray*}
\end{Def}

Entonces

\begin{eqnarray*}
\hat{U}\left(\alpha\right)=\sum_{n=0}^{\infty}\hat{F^{n\star}}\left(\alpha\right)=\sum_{n=0}^{\infty}\hat{F}\left(\alpha\right)^{n}=\frac{1}{1-\hat{F}\left(\alpha\right)}.
\end{eqnarray*}


\begin{Prop}
La Transformada de Laplace $\hat{U}\left(\alpha\right)$ y $\hat{F}\left(\alpha\right)$ determina una a la otra de manera \'unica por la relaci\'on $\hat{U}\left(\alpha\right)=\frac{1}{1-\hat{F}\left(\alpha\right)}$.
\end{Prop}


\begin{Note}
Un proceso de renovaci\'on $N\left(t\right)$ cuyos tiempos de inter-renovaci\'on tienen media finita, es un proceso Poisson con tasa $\lambda$ si y s\'olo s\'i $\esp\left[U\left(t\right)\right]=\lambda t$, para $t\geq0$.
\end{Note}


\begin{Teo}
Sea $N\left(t\right)$ un proceso puntual simple con puntos de localizaci\'on $T_{n}$ tal que $\eta\left(t\right)=\esp\left[N\left(\right)\right]$ es finita para cada $t$. Entonces para cualquier funci\'on $f:\rea_{+}\rightarrow\rea$,
\begin{eqnarray*}
\esp\left[\sum_{n=1}^{N\left(\right)}f\left(T_{n}\right)\right]=\int_{\left(0,t\right]}f\left(s\right)d\eta\left(s\right)\textrm{,  }t\geq0,
\end{eqnarray*}
suponiendo que la integral exista. Adem\'as si $X_{1},X_{2},\ldots$ son variables aleatorias definidas en el mismo espacio de probabilidad que el proceso $N\left(t\right)$ tal que $\esp\left[X_{n}|T_{n}=s\right]=f\left(s\right)$, independiente de $n$. Entonces
\begin{eqnarray*}
\esp\left[\sum_{n=1}^{N\left(t\right)}X_{n}\right]=\int_{\left(0,t\right]}f\left(s\right)d\eta\left(s\right)\textrm{,  }t\geq0,
\end{eqnarray*} 
suponiendo que la integral exista. 
\end{Teo}

\begin{Coro}[Identidad de Wald para Renovaciones]
Para el proceso de renovaci\'on $N\left(t\right)$,
\begin{eqnarray*}
\esp\left[T_{N\left(t\right)+1}\right]=\mu\esp\left[N\left(t\right)+1\right]\textrm{,  }t\geq0,
\end{eqnarray*}  
\end{Coro}

%______________________________________________________________________
%\subsection{Procesos de Renovaci\'on}
%______________________________________________________________________

\begin{Def}\label{Def.Tn}
Sean $0\leq T_{1}\leq T_{2}\leq \ldots$ son tiempos aleatorios infinitos en los cuales ocurren ciertos eventos. El n\'umero de tiempos $T_{n}$ en el intervalo $\left[0,t\right)$ es

\begin{eqnarray}
N\left(t\right)=\sum_{n=1}^{\infty}\indora\left(T_{n}\leq t\right),
\end{eqnarray}
para $t\geq0$.
\end{Def}

Si se consideran los puntos $T_{n}$ como elementos de $\rea_{+}$, y $N\left(t\right)$ es el n\'umero de puntos en $\rea$. El proceso denotado por $\left\{N\left(t\right):t\geq0\right\}$, denotado por $N\left(t\right)$, es un proceso puntual en $\rea_{+}$. Los $T_{n}$ son los tiempos de ocurrencia, el proceso puntual $N\left(t\right)$ es simple si su n\'umero de ocurrencias son distintas: $0<T_{1}<T_{2}<\ldots$ casi seguramente.

\begin{Def}
Un proceso puntual $N\left(t\right)$ es un proceso de renovaci\'on si los tiempos de interocurrencia $\xi_{n}=T_{n}-T_{n-1}$, para $n\geq1$, son independientes e identicamente distribuidos con distribuci\'on $F$, donde $F\left(0\right)=0$ y $T_{0}=0$. Los $T_{n}$ son llamados tiempos de renovaci\'on, referente a la independencia o renovaci\'on de la informaci\'on estoc\'astica en estos tiempos. Los $\xi_{n}$ son los tiempos de inter-renovaci\'on, y $N\left(t\right)$ es el n\'umero de renovaciones en el intervalo $\left[0,t\right)$
\end{Def}


\begin{Note}
Para definir un proceso de renovaci\'on para cualquier contexto, solamente hay que especificar una distribuci\'on $F$, con $F\left(0\right)=0$, para los tiempos de inter-renovaci\'on. La funci\'on $F$ en turno degune las otra variables aleatorias. De manera formal, existe un espacio de probabilidad y una sucesi\'on de variables aleatorias $\xi_{1},\xi_{2},\ldots$ definidas en este con distribuci\'on $F$. Entonces las otras cantidades son $T_{n}=\sum_{k=1}^{n}\xi_{k}$ y $N\left(t\right)=\sum_{n=1}^{\infty}\indora\left(T_{n}\leq t\right)$, donde $T_{n}\rightarrow\infty$ casi seguramente por la Ley Fuerte de los Grandes Números.
\end{Note}

\begin{Def}\label{Def.Tn}
Sean $0\leq T_{1}\leq T_{2}\leq \ldots$ son tiempos aleatorios infinitos en los cuales ocurren ciertos eventos. El n\'umero de tiempos $T_{n}$ en el intervalo $\left[0,t\right)$ es

\begin{eqnarray}
N\left(t\right)=\sum_{n=1}^{\infty}\indora\left(T_{n}\leq t\right),
\end{eqnarray}
para $t\geq0$.
\end{Def}

Si se consideran los puntos $T_{n}$ como elementos de $\rea_{+}$, y $N\left(t\right)$ es el n\'umero de puntos en $\rea$. El proceso denotado por $\left\{N\left(t\right):t\geq0\right\}$, denotado por $N\left(t\right)$, es un proceso puntual en $\rea_{+}$. Los $T_{n}$ son los tiempos de ocurrencia, el proceso puntual $N\left(t\right)$ es simple si su n\'umero de ocurrencias son distintas: $0<T_{1}<T_{2}<\ldots$ casi seguramente.

\begin{Def}
Un proceso puntual $N\left(t\right)$ es un proceso de renovaci\'on si los tiempos de interocurrencia $\xi_{n}=T_{n}-T_{n-1}$, para $n\geq1$, son independientes e identicamente distribuidos con distribuci\'on $F$, donde $F\left(0\right)=0$ y $T_{0}=0$. Los $T_{n}$ son llamados tiempos de renovaci\'on, referente a la independencia o renovaci\'on de la informaci\'on estoc\'astica en estos tiempos. Los $\xi_{n}$ son los tiempos de inter-renovaci\'on, y $N\left(t\right)$ es el n\'umero de renovaciones en el intervalo $\left[0,t\right)$
\end{Def}


\begin{Note}
Para definir un proceso de renovaci\'on para cualquier contexto, solamente hay que especificar una distribuci\'on $F$, con $F\left(0\right)=0$, para los tiempos de inter-renovaci\'on. La funci\'on $F$ en turno degune las otra variables aleatorias. De manera formal, existe un espacio de probabilidad y una sucesi\'on de variables aleatorias $\xi_{1},\xi_{2},\ldots$ definidas en este con distribuci\'on $F$. Entonces las otras cantidades son $T_{n}=\sum_{k=1}^{n}\xi_{k}$ y $N\left(t\right)=\sum_{n=1}^{\infty}\indora\left(T_{n}\leq t\right)$, donde $T_{n}\rightarrow\infty$ casi seguramente por la Ley Fuerte de los Grandes N\'umeros.
\end{Note}







Los tiempos $T_{n}$ est\'an relacionados con los conteos de $N\left(t\right)$ por

\begin{eqnarray*}
\left\{N\left(t\right)\geq n\right\}&=&\left\{T_{n}\leq t\right\}\\
T_{N\left(t\right)}\leq &t&<T_{N\left(t\right)+1},
\end{eqnarray*}

adem\'as $N\left(T_{n}\right)=n$, y 

\begin{eqnarray*}
N\left(t\right)=\max\left\{n:T_{n}\leq t\right\}=\min\left\{n:T_{n+1}>t\right\}
\end{eqnarray*}

Por propiedades de la convoluci\'on se sabe que

\begin{eqnarray*}
P\left\{T_{n}\leq t\right\}=F^{n\star}\left(t\right)
\end{eqnarray*}
que es la $n$-\'esima convoluci\'on de $F$. Entonces 

\begin{eqnarray*}
\left\{N\left(t\right)\geq n\right\}&=&\left\{T_{n}\leq t\right\}\\
P\left\{N\left(t\right)\leq n\right\}&=&1-F^{\left(n+1\right)\star}\left(t\right)
\end{eqnarray*}

Adem\'as usando el hecho de que $\esp\left[N\left(t\right)\right]=\sum_{n=1}^{\infty}P\left\{N\left(t\right)\geq n\right\}$
se tiene que

\begin{eqnarray*}
\esp\left[N\left(t\right)\right]=\sum_{n=1}^{\infty}F^{n\star}\left(t\right)
\end{eqnarray*}

\begin{Prop}
Para cada $t\geq0$, la funci\'on generadora de momentos $\esp\left[e^{\alpha N\left(t\right)}\right]$ existe para alguna $\alpha$ en una vecindad del 0, y de aqu\'i que $\esp\left[N\left(t\right)^{m}\right]<\infty$, para $m\geq1$.
\end{Prop}

\begin{Ejem}[\textbf{Proceso Poisson}]

Suponga que se tienen tiempos de inter-renovaci\'on \textit{i.i.d.} del proceso de renovaci\'on $N\left(t\right)$ tienen distribuci\'on exponencial $F\left(t\right)=q-e^{-\lambda t}$ con tasa $\lambda$. Entonces $N\left(t\right)$ es un proceso Poisson con tasa $\lambda$.

\end{Ejem}


\begin{Note}
Si el primer tiempo de renovaci\'on $\xi_{1}$ no tiene la misma distribuci\'on que el resto de las $\xi_{n}$, para $n\geq2$, a $N\left(t\right)$ se le llama Proceso de Renovaci\'on retardado, donde si $\xi$ tiene distribuci\'on $G$, entonces el tiempo $T_{n}$ de la $n$-\'esima renovaci\'on tiene distribuci\'on $G\star F^{\left(n-1\right)\star}\left(t\right)$
\end{Note}


\begin{Teo}
Para una constante $\mu\leq\infty$ ( o variable aleatoria), las siguientes expresiones son equivalentes:

\begin{eqnarray}
lim_{n\rightarrow\infty}n^{-1}T_{n}&=&\mu,\textrm{ c.s.}\\
lim_{t\rightarrow\infty}t^{-1}N\left(t\right)&=&1/\mu,\textrm{ c.s.}
\end{eqnarray}
\end{Teo}


Es decir, $T_{n}$ satisface la Ley Fuerte de los Grandes N\'umeros s\'i y s\'olo s\'i $N\left/t\right)$ la cumple.


\begin{Coro}[Ley Fuerte de los Grandes N\'umeros para Procesos de Renovaci\'on]
Si $N\left(t\right)$ es un proceso de renovaci\'on cuyos tiempos de inter-renovaci\'on tienen media $\mu\leq\infty$, entonces
\begin{eqnarray}
t^{-1}N\left(t\right)\rightarrow 1/\mu,\textrm{ c.s. cuando }t\rightarrow\infty.
\end{eqnarray}

\end{Coro}


Considerar el proceso estoc\'astico de valores reales $\left\{Z\left(t\right):t\geq0\right\}$ en el mismo espacio de probabilidad que $N\left(t\right)$

\begin{Def}
Para el proceso $\left\{Z\left(t\right):t\geq0\right\}$ se define la fluctuaci\'on m\'axima de $Z\left(t\right)$ en el intervalo $\left(T_{n-1},T_{n}\right]$:
\begin{eqnarray*}
M_{n}=\sup_{T_{n-1}<t\leq T_{n}}|Z\left(t\right)-Z\left(T_{n-1}\right)|
\end{eqnarray*}
\end{Def}

\begin{Teo}
Sup\'ongase que $n^{-1}T_{n}\rightarrow\mu$ c.s. cuando $n\rightarrow\infty$, donde $\mu\leq\infty$ es una constante o variable aleatoria. Sea $a$ una constante o variable aleatoria que puede ser infinita cuando $\mu$ es finita, y considere las expresiones l\'imite:
\begin{eqnarray}
lim_{n\rightarrow\infty}n^{-1}Z\left(T_{n}\right)&=&a,\textrm{ c.s.}\\
lim_{t\rightarrow\infty}t^{-1}Z\left(t\right)&=&a/\mu,\textrm{ c.s.}
\end{eqnarray}
La segunda expresi\'on implica la primera. Conversamente, la primera implica la segunda si el proceso $Z\left(t\right)$ es creciente, o si $lim_{n\rightarrow\infty}n^{-1}M_{n}=0$ c.s.
\end{Teo}

\begin{Coro}
Si $N\left(t\right)$ es un proceso de renovaci\'on, y $\left(Z\left(T_{n}\right)-Z\left(T_{n-1}\right),M_{n}\right)$, para $n\geq1$, son variables aleatorias independientes e id\'enticamente distribuidas con media finita, entonces,
\begin{eqnarray}
lim_{t\rightarrow\infty}t^{-1}Z\left(t\right)\rightarrow\frac{\esp\left[Z\left(T_{1}\right)-Z\left(T_{0}\right)\right]}{\esp\left[T_{1}\right]},\textrm{ c.s. cuando  }t\rightarrow\infty.
\end{eqnarray}
\end{Coro}


Sup\'ongase que $N\left(t\right)$ es un proceso de renovaci\'on con distribuci\'on $F$ con media finita $\mu$.

\begin{Def}
La funci\'on de renovaci\'on asociada con la distribuci\'on $F$, del proceso $N\left(t\right)$, es
\begin{eqnarray*}
U\left(t\right)=\sum_{n=1}^{\infty}F^{n\star}\left(t\right),\textrm{   }t\geq0,
\end{eqnarray*}
donde $F^{0\star}\left(t\right)=\indora\left(t\geq0\right)$.
\end{Def}


\begin{Prop}
Sup\'ongase que la distribuci\'on de inter-renovaci\'on $F$ tiene densidad $f$. Entonces $U\left(t\right)$ tambi\'en tiene densidad, para $t>0$, y es $U^{'}\left(t\right)=\sum_{n=0}^{\infty}f^{n\star}\left(t\right)$. Adem\'as
\begin{eqnarray*}
\prob\left\{N\left(t\right)>N\left(t-\right)\right\}=0\textrm{,   }t\geq0.
\end{eqnarray*}
\end{Prop}

\begin{Def}
La Transformada de Laplace-Stieljes de $F$ est\'a dada por

\begin{eqnarray*}
\hat{F}\left(\alpha\right)=\int_{\rea_{+}}e^{-\alpha t}dF\left(t\right)\textrm{,  }\alpha\geq0.
\end{eqnarray*}
\end{Def}

Entonces

\begin{eqnarray*}
\hat{U}\left(\alpha\right)=\sum_{n=0}^{\infty}\hat{F^{n\star}}\left(\alpha\right)=\sum_{n=0}^{\infty}\hat{F}\left(\alpha\right)^{n}=\frac{1}{1-\hat{F}\left(\alpha\right)}.
\end{eqnarray*}


\begin{Prop}
La Transformada de Laplace $\hat{U}\left(\alpha\right)$ y $\hat{F}\left(\alpha\right)$ determina una a la otra de manera \'unica por la relaci\'on $\hat{U}\left(\alpha\right)=\frac{1}{1-\hat{F}\left(\alpha\right)}$.
\end{Prop}


\begin{Note}
Un proceso de renovaci\'on $N\left(t\right)$ cuyos tiempos de inter-renovaci\'on tienen media finita, es un proceso Poisson con tasa $\lambda$ si y s\'olo s\'i $\esp\left[U\left(t\right)\right]=\lambda t$, para $t\geq0$.
\end{Note}


\begin{Teo}
Sea $N\left(t\right)$ un proceso puntual simple con puntos de localizaci\'on $T_{n}$ tal que $\eta\left(t\right)=\esp\left[N\left(\right)\right]$ es finita para cada $t$. Entonces para cualquier funci\'on $f:\rea_{+}\rightarrow\rea$,
\begin{eqnarray*}
\esp\left[\sum_{n=1}^{N\left(\right)}f\left(T_{n}\right)\right]=\int_{\left(0,t\right]}f\left(s\right)d\eta\left(s\right)\textrm{,  }t\geq0,
\end{eqnarray*}
suponiendo que la integral exista. Adem\'as si $X_{1},X_{2},\ldots$ son variables aleatorias definidas en el mismo espacio de probabilidad que el proceso $N\left(t\right)$ tal que $\esp\left[X_{n}|T_{n}=s\right]=f\left(s\right)$, independiente de $n$. Entonces
\begin{eqnarray*}
\esp\left[\sum_{n=1}^{N\left(t\right)}X_{n}\right]=\int_{\left(0,t\right]}f\left(s\right)d\eta\left(s\right)\textrm{,  }t\geq0,
\end{eqnarray*} 
suponiendo que la integral exista. 
\end{Teo}

\begin{Coro}[Identidad de Wald para Renovaciones]
Para el proceso de renovaci\'on $N\left(t\right)$,
\begin{eqnarray*}
\esp\left[T_{N\left(t\right)+1}\right]=\mu\esp\left[N\left(t\right)+1\right]\textrm{,  }t\geq0,
\end{eqnarray*}  
\end{Coro}


\begin{Def}
Sea $h\left(t\right)$ funci\'on de valores reales en $\rea$ acotada en intervalos finitos e igual a cero para $t<0$ La ecuaci\'on de renovaci\'on para $h\left(t\right)$ y la distribuci\'on $F$ es

\begin{eqnarray}\label{Ec.Renovacion}
H\left(t\right)=h\left(t\right)+\int_{\left[0,t\right]}H\left(t-s\right)dF\left(s\right)\textrm{,    }t\geq0,
\end{eqnarray}
donde $H\left(t\right)$ es una funci\'on de valores reales. Esto es $H=h+F\star H$. Decimos que $H\left(t\right)$ es soluci\'on de esta ecuaci\'on si satisface la ecuaci\'on, y es acotada en intervalos finitos e iguales a cero para $t<0$.
\end{Def}

\begin{Prop}
La funci\'on $U\star h\left(t\right)$ es la \'unica soluci\'on de la ecuaci\'on de renovaci\'on (\ref{Ec.Renovacion}).
\end{Prop}

\begin{Teo}[Teorema Renovaci\'on Elemental]
\begin{eqnarray*}
t^{-1}U\left(t\right)\rightarrow 1/\mu\textrm{,    cuando }t\rightarrow\infty.
\end{eqnarray*}
\end{Teo}



Sup\'ongase que $N\left(t\right)$ es un proceso de renovaci\'on con distribuci\'on $F$ con media finita $\mu$.

\begin{Def}
La funci\'on de renovaci\'on asociada con la distribuci\'on $F$, del proceso $N\left(t\right)$, es
\begin{eqnarray*}
U\left(t\right)=\sum_{n=1}^{\infty}F^{n\star}\left(t\right),\textrm{   }t\geq0,
\end{eqnarray*}
donde $F^{0\star}\left(t\right)=\indora\left(t\geq0\right)$.
\end{Def}


\begin{Prop}
Sup\'ongase que la distribuci\'on de inter-renovaci\'on $F$ tiene densidad $f$. Entonces $U\left(t\right)$ tambi\'en tiene densidad, para $t>0$, y es $U^{'}\left(t\right)=\sum_{n=0}^{\infty}f^{n\star}\left(t\right)$. Adem\'as
\begin{eqnarray*}
\prob\left\{N\left(t\right)>N\left(t-\right)\right\}=0\textrm{,   }t\geq0.
\end{eqnarray*}
\end{Prop}

\begin{Def}
La Transformada de Laplace-Stieljes de $F$ est\'a dada por

\begin{eqnarray*}
\hat{F}\left(\alpha\right)=\int_{\rea_{+}}e^{-\alpha t}dF\left(t\right)\textrm{,  }\alpha\geq0.
\end{eqnarray*}
\end{Def}

Entonces

\begin{eqnarray*}
\hat{U}\left(\alpha\right)=\sum_{n=0}^{\infty}\hat{F^{n\star}}\left(\alpha\right)=\sum_{n=0}^{\infty}\hat{F}\left(\alpha\right)^{n}=\frac{1}{1-\hat{F}\left(\alpha\right)}.
\end{eqnarray*}


\begin{Prop}
La Transformada de Laplace $\hat{U}\left(\alpha\right)$ y $\hat{F}\left(\alpha\right)$ determina una a la otra de manera \'unica por la relaci\'on $\hat{U}\left(\alpha\right)=\frac{1}{1-\hat{F}\left(\alpha\right)}$.
\end{Prop}


\begin{Note}
Un proceso de renovaci\'on $N\left(t\right)$ cuyos tiempos de inter-renovaci\'on tienen media finita, es un proceso Poisson con tasa $\lambda$ si y s\'olo s\'i $\esp\left[U\left(t\right)\right]=\lambda t$, para $t\geq0$.
\end{Note}


\begin{Teo}
Sea $N\left(t\right)$ un proceso puntual simple con puntos de localizaci\'on $T_{n}$ tal que $\eta\left(t\right)=\esp\left[N\left(\right)\right]$ es finita para cada $t$. Entonces para cualquier funci\'on $f:\rea_{+}\rightarrow\rea$,
\begin{eqnarray*}
\esp\left[\sum_{n=1}^{N\left(\right)}f\left(T_{n}\right)\right]=\int_{\left(0,t\right]}f\left(s\right)d\eta\left(s\right)\textrm{,  }t\geq0,
\end{eqnarray*}
suponiendo que la integral exista. Adem\'as si $X_{1},X_{2},\ldots$ son variables aleatorias definidas en el mismo espacio de probabilidad que el proceso $N\left(t\right)$ tal que $\esp\left[X_{n}|T_{n}=s\right]=f\left(s\right)$, independiente de $n$. Entonces
\begin{eqnarray*}
\esp\left[\sum_{n=1}^{N\left(t\right)}X_{n}\right]=\int_{\left(0,t\right]}f\left(s\right)d\eta\left(s\right)\textrm{,  }t\geq0,
\end{eqnarray*} 
suponiendo que la integral exista. 
\end{Teo}

\begin{Coro}[Identidad de Wald para Renovaciones]
Para el proceso de renovaci\'on $N\left(t\right)$,
\begin{eqnarray*}
\esp\left[T_{N\left(t\right)+1}\right]=\mu\esp\left[N\left(t\right)+1\right]\textrm{,  }t\geq0,
\end{eqnarray*}  
\end{Coro}


\begin{Def}
Sea $h\left(t\right)$ funci\'on de valores reales en $\rea$ acotada en intervalos finitos e igual a cero para $t<0$ La ecuaci\'on de renovaci\'on para $h\left(t\right)$ y la distribuci\'on $F$ es

\begin{eqnarray}\label{Ec.Renovacion}
H\left(t\right)=h\left(t\right)+\int_{\left[0,t\right]}H\left(t-s\right)dF\left(s\right)\textrm{,    }t\geq0,
\end{eqnarray}
donde $H\left(t\right)$ es una funci\'on de valores reales. Esto es $H=h+F\star H$. Decimos que $H\left(t\right)$ es soluci\'on de esta ecuaci\'on si satisface la ecuaci\'on, y es acotada en intervalos finitos e iguales a cero para $t<0$.
\end{Def}

\begin{Prop}
La funci\'on $U\star h\left(t\right)$ es la \'unica soluci\'on de la ecuaci\'on de renovaci\'on (\ref{Ec.Renovacion}).
\end{Prop}

\begin{Teo}[Teorema Renovaci\'on Elemental]
\begin{eqnarray*}
t^{-1}U\left(t\right)\rightarrow 1/\mu\textrm{,    cuando }t\rightarrow\infty.
\end{eqnarray*}
\end{Teo}


\begin{Note} Una funci\'on $h:\rea_{+}\rightarrow\rea$ es Directamente Riemann Integrable en los siguientes casos:
\begin{itemize}
\item[a)] $h\left(t\right)\geq0$ es decreciente y Riemann Integrable.
\item[b)] $h$ es continua excepto posiblemente en un conjunto de Lebesgue de medida 0, y $|h\left(t\right)|\leq b\left(t\right)$, donde $b$ es DRI.
\end{itemize}
\end{Note}

\begin{Teo}[Teorema Principal de Renovaci\'on]
Si $F$ es no aritm\'etica y $h\left(t\right)$ es Directamente Riemann Integrable (DRI), entonces

\begin{eqnarray*}
lim_{t\rightarrow\infty}U\star h=\frac{1}{\mu}\int_{\rea_{+}}h\left(s\right)ds.
\end{eqnarray*}
\end{Teo}

\begin{Prop}
Cualquier funci\'on $H\left(t\right)$ acotada en intervalos finitos y que es 0 para $t<0$ puede expresarse como
\begin{eqnarray*}
H\left(t\right)=U\star h\left(t\right)\textrm{,  donde }h\left(t\right)=H\left(t\right)-F\star H\left(t\right)
\end{eqnarray*}
\end{Prop}

\begin{Def}
Un proceso estoc\'astico $X\left(t\right)$ es crudamente regenerativo en un tiempo aleatorio positivo $T$ si
\begin{eqnarray*}
\esp\left[X\left(T+t\right)|T\right]=\esp\left[X\left(t\right)\right]\textrm{, para }t\geq0,\end{eqnarray*}
y con las esperanzas anteriores finitas.
\end{Def}

\begin{Prop}
Sup\'ongase que $X\left(t\right)$ es un proceso crudamente regenerativo en $T$, que tiene distribuci\'on $F$. Si $\esp\left[X\left(t\right)\right]$ es acotado en intervalos finitos, entonces
\begin{eqnarray*}
\esp\left[X\left(t\right)\right]=U\star h\left(t\right)\textrm{,  donde }h\left(t\right)=\esp\left[X\left(t\right)\indora\left(T>t\right)\right].
\end{eqnarray*}
\end{Prop}

\begin{Teo}[Regeneraci\'on Cruda]
Sup\'ongase que $X\left(t\right)$ es un proceso con valores positivo crudamente regenerativo en $T$, y def\'inase $M=\sup\left\{|X\left(t\right)|:t\leq T\right\}$. Si $T$ es no aritm\'etico y $M$ y $MT$ tienen media finita, entonces
\begin{eqnarray*}
lim_{t\rightarrow\infty}\esp\left[X\left(t\right)\right]=\frac{1}{\mu}\int_{\rea_{+}}h\left(s\right)ds,
\end{eqnarray*}
donde $h\left(t\right)=\esp\left[X\left(t\right)\indora\left(T>t\right)\right]$.
\end{Teo}


\begin{Note} Una funci\'on $h:\rea_{+}\rightarrow\rea$ es Directamente Riemann Integrable en los siguientes casos:
\begin{itemize}
\item[a)] $h\left(t\right)\geq0$ es decreciente y Riemann Integrable.
\item[b)] $h$ es continua excepto posiblemente en un conjunto de Lebesgue de medida 0, y $|h\left(t\right)|\leq b\left(t\right)$, donde $b$ es DRI.
\end{itemize}
\end{Note}

\begin{Teo}[Teorema Principal de Renovaci\'on]
Si $F$ es no aritm\'etica y $h\left(t\right)$ es Directamente Riemann Integrable (DRI), entonces

\begin{eqnarray*}
lim_{t\rightarrow\infty}U\star h=\frac{1}{\mu}\int_{\rea_{+}}h\left(s\right)ds.
\end{eqnarray*}
\end{Teo}

\begin{Prop}
Cualquier funci\'on $H\left(t\right)$ acotada en intervalos finitos y que es 0 para $t<0$ puede expresarse como
\begin{eqnarray*}
H\left(t\right)=U\star h\left(t\right)\textrm{,  donde }h\left(t\right)=H\left(t\right)-F\star H\left(t\right)
\end{eqnarray*}
\end{Prop}

\begin{Def}
Un proceso estoc\'astico $X\left(t\right)$ es crudamente regenerativo en un tiempo aleatorio positivo $T$ si
\begin{eqnarray*}
\esp\left[X\left(T+t\right)|T\right]=\esp\left[X\left(t\right)\right]\textrm{, para }t\geq0,\end{eqnarray*}
y con las esperanzas anteriores finitas.
\end{Def}

\begin{Prop}
Sup\'ongase que $X\left(t\right)$ es un proceso crudamente regenerativo en $T$, que tiene distribuci\'on $F$. Si $\esp\left[X\left(t\right)\right]$ es acotado en intervalos finitos, entonces
\begin{eqnarray*}
\esp\left[X\left(t\right)\right]=U\star h\left(t\right)\textrm{,  donde }h\left(t\right)=\esp\left[X\left(t\right)\indora\left(T>t\right)\right].
\end{eqnarray*}
\end{Prop}

\begin{Teo}[Regeneraci\'on Cruda]
Sup\'ongase que $X\left(t\right)$ es un proceso con valores positivo crudamente regenerativo en $T$, y def\'inase $M=\sup\left\{|X\left(t\right)|:t\leq T\right\}$. Si $T$ es no aritm\'etico y $M$ y $MT$ tienen media finita, entonces
\begin{eqnarray*}
lim_{t\rightarrow\infty}\esp\left[X\left(t\right)\right]=\frac{1}{\mu}\int_{\rea_{+}}h\left(s\right)ds,
\end{eqnarray*}
donde $h\left(t\right)=\esp\left[X\left(t\right)\indora\left(T>t\right)\right]$.
\end{Teo}

\begin{Def}
Para el proceso $\left\{\left(N\left(t\right),X\left(t\right)\right):t\geq0\right\}$, sus trayectoria muestrales en el intervalo de tiempo $\left[T_{n-1},T_{n}\right)$ est\'an descritas por
\begin{eqnarray*}
\zeta_{n}=\left(\xi_{n},\left\{X\left(T_{n-1}+t\right):0\leq t<\xi_{n}\right\}\right)
\end{eqnarray*}
Este $\zeta_{n}$ es el $n$-\'esimo segmento del proceso. El proceso es regenerativo sobre los tiempos $T_{n}$ si sus segmentos $\zeta_{n}$ son independientes e id\'enticamennte distribuidos.
\end{Def}


\begin{Note}
Si $\tilde{X}\left(t\right)$ con espacio de estados $\tilde{S}$ es regenerativo sobre $T_{n}$, entonces $X\left(t\right)=f\left(\tilde{X}\left(t\right)\right)$ tambi\'en es regenerativo sobre $T_{n}$, para cualquier funci\'on $f:\tilde{S}\rightarrow S$.
\end{Note}

\begin{Note}
Los procesos regenerativos son crudamente regenerativos, pero no al rev\'es.
\end{Note}


\begin{Note}
Un proceso estoc\'astico a tiempo continuo o discreto es regenerativo si existe un proceso de renovaci\'on  tal que los segmentos del proceso entre tiempos de renovaci\'on sucesivos son i.i.d., es decir, para $\left\{X\left(t\right):t\geq0\right\}$ proceso estoc\'astico a tiempo continuo con espacio de estados $S$, espacio m\'etrico.
\end{Note}

Para $\left\{X\left(t\right):t\geq0\right\}$ Proceso Estoc\'astico a tiempo continuo con estado de espacios $S$, que es un espacio m\'etrico, con trayectorias continuas por la derecha y con l\'imites por la izquierda c.s. Sea $N\left(t\right)$ un proceso de renovaci\'on en $\rea_{+}$ definido en el mismo espacio de probabilidad que $X\left(t\right)$, con tiempos de renovaci\'on $T$ y tiempos de inter-renovaci\'on $\xi_{n}=T_{n}-T_{n-1}$, con misma distribuci\'on $F$ de media finita $\mu$.



\begin{Def}
Para el proceso $\left\{\left(N\left(t\right),X\left(t\right)\right):t\geq0\right\}$, sus trayectoria muestrales en el intervalo de tiempo $\left[T_{n-1},T_{n}\right)$ est\'an descritas por
\begin{eqnarray*}
\zeta_{n}=\left(\xi_{n},\left\{X\left(T_{n-1}+t\right):0\leq t<\xi_{n}\right\}\right)
\end{eqnarray*}
Este $\zeta_{n}$ es el $n$-\'esimo segmento del proceso. El proceso es regenerativo sobre los tiempos $T_{n}$ si sus segmentos $\zeta_{n}$ son independientes e id\'enticamennte distribuidos.
\end{Def}

\begin{Note}
Un proceso regenerativo con media de la longitud de ciclo finita es llamado positivo recurrente.
\end{Note}

\begin{Teo}[Procesos Regenerativos]
Suponga que el proceso
\end{Teo}


\begin{Def}[Renewal Process Trinity]
Para un proceso de renovaci\'on $N\left(t\right)$, los siguientes procesos proveen de informaci\'on sobre los tiempos de renovaci\'on.
\begin{itemize}
\item $A\left(t\right)=t-T_{N\left(t\right)}$, el tiempo de recurrencia hacia atr\'as al tiempo $t$, que es el tiempo desde la \'ultima renovaci\'on para $t$.

\item $B\left(t\right)=T_{N\left(t\right)+1}-t$, el tiempo de recurrencia hacia adelante al tiempo $t$, residual del tiempo de renovaci\'on, que es el tiempo para la pr\'oxima renovaci\'on despu\'es de $t$.

\item $L\left(t\right)=\xi_{N\left(t\right)+1}=A\left(t\right)+B\left(t\right)$, la longitud del intervalo de renovaci\'on que contiene a $t$.
\end{itemize}
\end{Def}

\begin{Note}
El proceso tridimensional $\left(A\left(t\right),B\left(t\right),L\left(t\right)\right)$ es regenerativo sobre $T_{n}$, y por ende cada proceso lo es. Cada proceso $A\left(t\right)$ y $B\left(t\right)$ son procesos de MArkov a tiempo continuo con trayectorias continuas por partes en el espacio de estados $\rea_{+}$. Una expresi\'on conveniente para su distribuci\'on conjunta es, para $0\leq x<t,y\geq0$
\begin{equation}\label{NoRenovacion}
P\left\{A\left(t\right)>x,B\left(t\right)>y\right\}=
P\left\{N\left(t+y\right)-N\left((t-x)\right)=0\right\}
\end{equation}
\end{Note}

\begin{Ejem}[Tiempos de recurrencia Poisson]
Si $N\left(t\right)$ es un proceso Poisson con tasa $\lambda$, entonces de la expresi\'on (\ref{NoRenovacion}) se tiene que

\begin{eqnarray*}
\begin{array}{lc}
P\left\{A\left(t\right)>x,B\left(t\right)>y\right\}=e^{-\lambda\left(x+y\right)},&0\leq x<t,y\geq0,
\end{array}
\end{eqnarray*}
que es la probabilidad Poisson de no renovaciones en un intervalo de longitud $x+y$.

\end{Ejem}

\begin{Note}
Una cadena de Markov erg\'odica tiene la propiedad de ser estacionaria si la distribuci\'on de su estado al tiempo $0$ es su distribuci\'on estacionaria.
\end{Note}


\begin{Def}
Un proceso estoc\'astico a tiempo continuo $\left\{X\left(t\right):t\geq0\right\}$ en un espacio general es estacionario si sus distribuciones finito dimensionales son invariantes bajo cualquier  traslado: para cada $0\leq s_{1}<s_{2}<\cdots<s_{k}$ y $t\geq0$,
\begin{eqnarray*}
\left(X\left(s_{1}+t\right),\ldots,X\left(s_{k}+t\right)\right)=_{d}\left(X\left(s_{1}\right),\ldots,X\left(s_{k}\right)\right).
\end{eqnarray*}
\end{Def}

\begin{Note}
Un proceso de Markov es estacionario si $X\left(t\right)=_{d}X\left(0\right)$, $t\geq0$.
\end{Note}

Considerese el proceso $N\left(t\right)=\sum_{n}\indora\left(\tau_{n}\leq t\right)$ en $\rea_{+}$, con puntos $0<\tau_{1}<\tau_{2}<\cdots$.

\begin{Prop}
Si $N$ es un proceso puntual estacionario y $\esp\left[N\left(1\right)\right]<\infty$, entonces $\esp\left[N\left(t\right)\right]=t\esp\left[N\left(1\right)\right]$, $t\geq0$

\end{Prop}

\begin{Teo}
Los siguientes enunciados son equivalentes
\begin{itemize}
\item[i)] El proceso retardado de renovaci\'on $N$ es estacionario.

\item[ii)] EL proceso de tiempos de recurrencia hacia adelante $B\left(t\right)$ es estacionario.


\item[iii)] $\esp\left[N\left(t\right)\right]=t/\mu$,


\item[iv)] $G\left(t\right)=F_{e}\left(t\right)=\frac{1}{\mu}\int_{0}^{t}\left[1-F\left(s\right)\right]ds$
\end{itemize}
Cuando estos enunciados son ciertos, $P\left\{B\left(t\right)\leq x\right\}=F_{e}\left(x\right)$, para $t,x\geq0$.

\end{Teo}

\begin{Note}
Una consecuencia del teorema anterior es que el Proceso Poisson es el \'unico proceso sin retardo que es estacionario.
\end{Note}

\begin{Coro}
El proceso de renovaci\'on $N\left(t\right)$ sin retardo, y cuyos tiempos de inter renonaci\'on tienen media finita, es estacionario si y s\'olo si es un proceso Poisson.

\end{Coro}

%______________________________________________________________________

%\section{Ejemplos, Notas importantes}
%______________________________________________________________________
%\section*{Ap\'endice A}
%__________________________________________________________________

%________________________________________________________________________
%\subsection*{Procesos Regenerativos}
%________________________________________________________________________



\begin{Note}
Si $\tilde{X}\left(t\right)$ con espacio de estados $\tilde{S}$ es regenerativo sobre $T_{n}$, entonces $X\left(t\right)=f\left(\tilde{X}\left(t\right)\right)$ tambi\'en es regenerativo sobre $T_{n}$, para cualquier funci\'on $f:\tilde{S}\rightarrow S$.
\end{Note}

\begin{Note}
Los procesos regenerativos son crudamente regenerativos, pero no al rev\'es.
\end{Note}
%\subsection*{Procesos Regenerativos: Sigman\cite{Sigman1}}
\begin{Def}[Definici\'on Cl\'asica]
Un proceso estoc\'astico $X=\left\{X\left(t\right):t\geq0\right\}$ es llamado regenerativo is existe una variable aleatoria $R_{1}>0$ tal que
\begin{itemize}
\item[i)] $\left\{X\left(t+R_{1}\right):t\geq0\right\}$ es independiente de $\left\{\left\{X\left(t\right):t<R_{1}\right\},\right\}$
\item[ii)] $\left\{X\left(t+R_{1}\right):t\geq0\right\}$ es estoc\'asticamente equivalente a $\left\{X\left(t\right):t>0\right\}$
\end{itemize}

Llamamos a $R_{1}$ tiempo de regeneraci\'on, y decimos que $X$ se regenera en este punto.
\end{Def}

$\left\{X\left(t+R_{1}\right)\right\}$ es regenerativo con tiempo de regeneraci\'on $R_{2}$, independiente de $R_{1}$ pero con la misma distribuci\'on que $R_{1}$. Procediendo de esta manera se obtiene una secuencia de variables aleatorias independientes e id\'enticamente distribuidas $\left\{R_{n}\right\}$ llamados longitudes de ciclo. Si definimos a $Z_{k}\equiv R_{1}+R_{2}+\cdots+R_{k}$, se tiene un proceso de renovaci\'on llamado proceso de renovaci\'on encajado para $X$.




\begin{Def}
Para $x$ fijo y para cada $t\geq0$, sea $I_{x}\left(t\right)=1$ si $X\left(t\right)\leq x$,  $I_{x}\left(t\right)=0$ en caso contrario, y def\'inanse los tiempos promedio
\begin{eqnarray*}
\overline{X}&=&lim_{t\rightarrow\infty}\frac{1}{t}\int_{0}^{\infty}X\left(u\right)du\\
\prob\left(X_{\infty}\leq x\right)&=&lim_{t\rightarrow\infty}\frac{1}{t}\int_{0}^{\infty}I_{x}\left(u\right)du,
\end{eqnarray*}
cuando estos l\'imites existan.
\end{Def}

Como consecuencia del teorema de Renovaci\'on-Recompensa, se tiene que el primer l\'imite  existe y es igual a la constante
\begin{eqnarray*}
\overline{X}&=&\frac{\esp\left[\int_{0}^{R_{1}}X\left(t\right)dt\right]}{\esp\left[R_{1}\right]},
\end{eqnarray*}
suponiendo que ambas esperanzas son finitas.

\begin{Note}
\begin{itemize}
\item[a)] Si el proceso regenerativo $X$ es positivo recurrente y tiene trayectorias muestrales no negativas, entonces la ecuaci\'on anterior es v\'alida.
\item[b)] Si $X$ es positivo recurrente regenerativo, podemos construir una \'unica versi\'on estacionaria de este proceso, $X_{e}=\left\{X_{e}\left(t\right)\right\}$, donde $X_{e}$ es un proceso estoc\'astico regenerativo y estrictamente estacionario, con distribuci\'on marginal distribuida como $X_{\infty}$
\end{itemize}
\end{Note}

Para $\left\{X\left(t\right):t\geq0\right\}$ Proceso Estoc\'astico a tiempo continuo con estado de espacios $S$, que es un espacio m\'etrico, con trayectorias continuas por la derecha y con l\'imites por la izquierda c.s. Sea $N\left(t\right)$ un proceso de renovaci\'on en $\rea_{+}$ definido en el mismo espacio de probabilidad que $X\left(t\right)$, con tiempos de renovaci\'on $T$ y tiempos de inter-renovaci\'on $\xi_{n}=T_{n}-T_{n-1}$, con misma distribuci\'on $F$ de media finita $\mu$.


\begin{Def}
Para el proceso $\left\{\left(N\left(t\right),X\left(t\right)\right):t\geq0\right\}$, sus trayectoria muestrales en el intervalo de tiempo $\left[T_{n-1},T_{n}\right)$ est\'an descritas por
\begin{eqnarray*}
\zeta_{n}=\left(\xi_{n},\left\{X\left(T_{n-1}+t\right):0\leq t<\xi_{n}\right\}\right)
\end{eqnarray*}
Este $\zeta_{n}$ es el $n$-\'esimo segmento del proceso. El proceso es regenerativo sobre los tiempos $T_{n}$ si sus segmentos $\zeta_{n}$ son independientes e id\'enticamennte distribuidos.
\end{Def}


\begin{Note}
Si $\tilde{X}\left(t\right)$ con espacio de estados $\tilde{S}$ es regenerativo sobre $T_{n}$, entonces $X\left(t\right)=f\left(\tilde{X}\left(t\right)\right)$ tambi\'en es regenerativo sobre $T_{n}$, para cualquier funci\'on $f:\tilde{S}\rightarrow S$.
\end{Note}

\begin{Note}
Los procesos regenerativos son crudamente regenerativos, pero no al rev\'es.
\end{Note}

\begin{Def}[Definici\'on Cl\'asica]
Un proceso estoc\'astico $X=\left\{X\left(t\right):t\geq0\right\}$ es llamado regenerativo is existe una variable aleatoria $R_{1}>0$ tal que
\begin{itemize}
\item[i)] $\left\{X\left(t+R_{1}\right):t\geq0\right\}$ es independiente de $\left\{\left\{X\left(t\right):t<R_{1}\right\},\right\}$
\item[ii)] $\left\{X\left(t+R_{1}\right):t\geq0\right\}$ es estoc\'asticamente equivalente a $\left\{X\left(t\right):t>0\right\}$
\end{itemize}

Llamamos a $R_{1}$ tiempo de regeneraci\'on, y decimos que $X$ se regenera en este punto.
\end{Def}

$\left\{X\left(t+R_{1}\right)\right\}$ es regenerativo con tiempo de regeneraci\'on $R_{2}$, independiente de $R_{1}$ pero con la misma distribuci\'on que $R_{1}$. Procediendo de esta manera se obtiene una secuencia de variables aleatorias independientes e id\'enticamente distribuidas $\left\{R_{n}\right\}$ llamados longitudes de ciclo. Si definimos a $Z_{k}\equiv R_{1}+R_{2}+\cdots+R_{k}$, se tiene un proceso de renovaci\'on llamado proceso de renovaci\'on encajado para $X$.

\begin{Note}
Un proceso regenerativo con media de la longitud de ciclo finita es llamado positivo recurrente.
\end{Note}


\begin{Def}
Para $x$ fijo y para cada $t\geq0$, sea $I_{x}\left(t\right)=1$ si $X\left(t\right)\leq x$,  $I_{x}\left(t\right)=0$ en caso contrario, y def\'inanse los tiempos promedio
\begin{eqnarray*}
\overline{X}&=&lim_{t\rightarrow\infty}\frac{1}{t}\int_{0}^{\infty}X\left(u\right)du\\
\prob\left(X_{\infty}\leq x\right)&=&lim_{t\rightarrow\infty}\frac{1}{t}\int_{0}^{\infty}I_{x}\left(u\right)du,
\end{eqnarray*}
cuando estos l\'imites existan.
\end{Def}

Como consecuencia del teorema de Renovaci\'on-Recompensa, se tiene que el primer l\'imite  existe y es igual a la constante
\begin{eqnarray*}
\overline{X}&=&\frac{\esp\left[\int_{0}^{R_{1}}X\left(t\right)dt\right]}{\esp\left[R_{1}\right]},
\end{eqnarray*}
suponiendo que ambas esperanzas son finitas.

\begin{Note}
\begin{itemize}
\item[a)] Si el proceso regenerativo $X$ es positivo recurrente y tiene trayectorias muestrales no negativas, entonces la ecuaci\'on anterior es v\'alida.
\item[b)] Si $X$ es positivo recurrente regenerativo, podemos construir una \'unica versi\'on estacionaria de este proceso, $X_{e}=\left\{X_{e}\left(t\right)\right\}$, donde $X_{e}$ es un proceso estoc\'astico regenerativo y estrictamente estacionario, con distribuci\'on marginal distribuida como $X_{\infty}$
\end{itemize}
\end{Note}

%__________________________________________________________________________________________
%\subsection{Procesos Regenerativos Estacionarios - Stidham \cite{Stidham}}
%__________________________________________________________________________________________


Un proceso estoc\'astico a tiempo continuo $\left\{V\left(t\right),t\geq0\right\}$ es un proceso regenerativo si existe una sucesi\'on de variables aleatorias independientes e id\'enticamente distribuidas $\left\{X_{1},X_{2},\ldots\right\}$, sucesi\'on de renovaci\'on, tal que para cualquier conjunto de Borel $A$, 

\begin{eqnarray*}
\prob\left\{V\left(t\right)\in A|X_{1}+X_{2}+\cdots+X_{R\left(t\right)}=s,\left\{V\left(\tau\right),\tau<s\right\}\right\}=\prob\left\{V\left(t-s\right)\in A|X_{1}>t-s\right\},
\end{eqnarray*}
para todo $0\leq s\leq t$, donde $R\left(t\right)=\max\left\{X_{1}+X_{2}+\cdots+X_{j}\leq t\right\}=$n\'umero de renovaciones ({\emph{puntos de regeneraci\'on}}) que ocurren en $\left[0,t\right]$. El intervalo $\left[0,X_{1}\right)$ es llamado {\emph{primer ciclo de regeneraci\'on}} de $\left\{V\left(t \right),t\geq0\right\}$, $\left[X_{1},X_{1}+X_{2}\right)$ el {\emph{segundo ciclo de regeneraci\'on}}, y as\'i sucesivamente.

Sea $X=X_{1}$ y sea $F$ la funci\'on de distrbuci\'on de $X$


\begin{Def}
Se define el proceso estacionario, $\left\{V^{*}\left(t\right),t\geq0\right\}$, para $\left\{V\left(t\right),t\geq0\right\}$ por

\begin{eqnarray*}
\prob\left\{V\left(t\right)\in A\right\}=\frac{1}{\esp\left[X\right]}\int_{0}^{\infty}\prob\left\{V\left(t+x\right)\in A|X>x\right\}\left(1-F\left(x\right)\right)dx,
\end{eqnarray*} 
para todo $t\geq0$ y todo conjunto de Borel $A$.
\end{Def}

\begin{Def}
Una distribuci\'on se dice que es {\emph{aritm\'etica}} si todos sus puntos de incremento son m\'ultiplos de la forma $0,\lambda, 2\lambda,\ldots$ para alguna $\lambda>0$ entera.
\end{Def}


\begin{Def}
Una modificaci\'on medible de un proceso $\left\{V\left(t\right),t\geq0\right\}$, es una versi\'on de este, $\left\{V\left(t,w\right)\right\}$ conjuntamente medible para $t\geq0$ y para $w\in S$, $S$ espacio de estados para $\left\{V\left(t\right),t\geq0\right\}$.
\end{Def}

\begin{Teo}
Sea $\left\{V\left(t\right),t\geq\right\}$ un proceso regenerativo no negativo con modificaci\'on medible. Sea $\esp\left[X\right]<\infty$. Entonces el proceso estacionario dado por la ecuaci\'on anterior est\'a bien definido y tiene funci\'on de distribuci\'on independiente de $t$, adem\'as
\begin{itemize}
\item[i)] \begin{eqnarray*}
\esp\left[V^{*}\left(0\right)\right]&=&\frac{\esp\left[\int_{0}^{X}V\left(s\right)ds\right]}{\esp\left[X\right]}\end{eqnarray*}
\item[ii)] Si $\esp\left[V^{*}\left(0\right)\right]<\infty$, equivalentemente, si $\esp\left[\int_{0}^{X}V\left(s\right)ds\right]<\infty$,entonces
\begin{eqnarray*}
\frac{\int_{0}^{t}V\left(s\right)ds}{t}\rightarrow\frac{\esp\left[\int_{0}^{X}V\left(s\right)ds\right]}{\esp\left[X\right]}
\end{eqnarray*}
con probabilidad 1 y en media, cuando $t\rightarrow\infty$.
\end{itemize}
\end{Teo}
%
%___________________________________________________________________________________________
%\vspace{5.5cm}
%\chapter{Cadenas de Markov estacionarias}
%\vspace{-1.0cm}


%__________________________________________________________________________________________
%\subsection{Procesos Regenerativos Estacionarios - Stidham \cite{Stidham}}
%__________________________________________________________________________________________


Un proceso estoc\'astico a tiempo continuo $\left\{V\left(t\right),t\geq0\right\}$ es un proceso regenerativo si existe una sucesi\'on de variables aleatorias independientes e id\'enticamente distribuidas $\left\{X_{1},X_{2},\ldots\right\}$, sucesi\'on de renovaci\'on, tal que para cualquier conjunto de Borel $A$, 

\begin{eqnarray*}
\prob\left\{V\left(t\right)\in A|X_{1}+X_{2}+\cdots+X_{R\left(t\right)}=s,\left\{V\left(\tau\right),\tau<s\right\}\right\}=\prob\left\{V\left(t-s\right)\in A|X_{1}>t-s\right\},
\end{eqnarray*}
para todo $0\leq s\leq t$, donde $R\left(t\right)=\max\left\{X_{1}+X_{2}+\cdots+X_{j}\leq t\right\}=$n\'umero de renovaciones ({\emph{puntos de regeneraci\'on}}) que ocurren en $\left[0,t\right]$. El intervalo $\left[0,X_{1}\right)$ es llamado {\emph{primer ciclo de regeneraci\'on}} de $\left\{V\left(t \right),t\geq0\right\}$, $\left[X_{1},X_{1}+X_{2}\right)$ el {\emph{segundo ciclo de regeneraci\'on}}, y as\'i sucesivamente.

Sea $X=X_{1}$ y sea $F$ la funci\'on de distrbuci\'on de $X$


\begin{Def}
Se define el proceso estacionario, $\left\{V^{*}\left(t\right),t\geq0\right\}$, para $\left\{V\left(t\right),t\geq0\right\}$ por

\begin{eqnarray*}
\prob\left\{V\left(t\right)\in A\right\}=\frac{1}{\esp\left[X\right]}\int_{0}^{\infty}\prob\left\{V\left(t+x\right)\in A|X>x\right\}\left(1-F\left(x\right)\right)dx,
\end{eqnarray*} 
para todo $t\geq0$ y todo conjunto de Borel $A$.
\end{Def}

\begin{Def}
Una distribuci\'on se dice que es {\emph{aritm\'etica}} si todos sus puntos de incremento son m\'ultiplos de la forma $0,\lambda, 2\lambda,\ldots$ para alguna $\lambda>0$ entera.
\end{Def}


\begin{Def}
Una modificaci\'on medible de un proceso $\left\{V\left(t\right),t\geq0\right\}$, es una versi\'on de este, $\left\{V\left(t,w\right)\right\}$ conjuntamente medible para $t\geq0$ y para $w\in S$, $S$ espacio de estados para $\left\{V\left(t\right),t\geq0\right\}$.
\end{Def}

\begin{Teo}
Sea $\left\{V\left(t\right),t\geq\right\}$ un proceso regenerativo no negativo con modificaci\'on medible. Sea $\esp\left[X\right]<\infty$. Entonces el proceso estacionario dado por la ecuaci\'on anterior est\'a bien definido y tiene funci\'on de distribuci\'on independiente de $t$, adem\'as
\begin{itemize}
\item[i)] \begin{eqnarray*}
\esp\left[V^{*}\left(0\right)\right]&=&\frac{\esp\left[\int_{0}^{X}V\left(s\right)ds\right]}{\esp\left[X\right]}\end{eqnarray*}
\item[ii)] Si $\esp\left[V^{*}\left(0\right)\right]<\infty$, equivalentemente, si $\esp\left[\int_{0}^{X}V\left(s\right)ds\right]<\infty$,entonces
\begin{eqnarray*}
\frac{\int_{0}^{t}V\left(s\right)ds}{t}\rightarrow\frac{\esp\left[\int_{0}^{X}V\left(s\right)ds\right]}{\esp\left[X\right]}
\end{eqnarray*}
con probabilidad 1 y en media, cuando $t\rightarrow\infty$.
\end{itemize}
\end{Teo}

Sea la funci\'on generadora de momentos para $L_{i}$, el n\'umero de usuarios en la cola $Q_{i}\left(z\right)$ en cualquier momento, est\'a dada por el tiempo promedio de $z^{L_{i}\left(t\right)}$ sobre el ciclo regenerativo definido anteriormente. Entonces 



Es decir, es posible determinar las longitudes de las colas a cualquier tiempo $t$. Entonces, determinando el primer momento es posible ver que


\begin{Def}
El tiempo de Ciclo $C_{i}$ es el periodo de tiempo que comienza cuando la cola $i$ es visitada por primera vez en un ciclo, y termina cuando es visitado nuevamente en el pr\'oximo ciclo. La duraci\'on del mismo est\'a dada por $\tau_{i}\left(m+1\right)-\tau_{i}\left(m\right)$, o equivalentemente $\overline{\tau}_{i}\left(m+1\right)-\overline{\tau}_{i}\left(m\right)$ bajo condiciones de estabilidad.
\end{Def}


\begin{Def}
El tiempo de intervisita $I_{i}$ es el periodo de tiempo que comienza cuando se ha completado el servicio en un ciclo y termina cuando es visitada nuevamente en el pr\'oximo ciclo. Su  duraci\'on del mismo est\'a dada por $\tau_{i}\left(m+1\right)-\overline{\tau}_{i}\left(m\right)$.
\end{Def}

La duraci\'on del tiempo de intervisita es $\tau_{i}\left(m+1\right)-\overline{\tau}\left(m\right)$. Dado que el n\'umero de usuarios presentes en $Q_{i}$ al tiempo $t=\tau_{i}\left(m+1\right)$ es igual al n\'umero de arribos durante el intervalo de tiempo $\left[\overline{\tau}\left(m\right),\tau_{i}\left(m+1\right)\right]$ se tiene que


\begin{eqnarray*}
\esp\left[z_{i}^{L_{i}\left(\tau_{i}\left(m+1\right)\right)}\right]=\esp\left[\left\{P_{i}\left(z_{i}\right)\right\}^{\tau_{i}\left(m+1\right)-\overline{\tau}\left(m\right)}\right]
\end{eqnarray*}

entonces, si $I_{i}\left(z\right)=\esp\left[z^{\tau_{i}\left(m+1\right)-\overline{\tau}\left(m\right)}\right]$
se tiene que $F_{i}\left(z\right)=I_{i}\left[P_{i}\left(z\right)\right]$
para $i=1,2$.

Conforme a la definici\'on dada al principio del cap\'itulo, definici\'on (\ref{Def.Tn}), sean $T_{1},T_{2},\ldots$ los puntos donde las longitudes de las colas de la red de sistemas de visitas c\'iclicas son cero simult\'aneamente, cuando la cola $Q_{j}$ es visitada por el servidor para dar servicio, es decir, $L_{1}\left(T_{i}\right)=0,L_{2}\left(T_{i}\right)=0,\hat{L}_{1}\left(T_{i}\right)=0$ y $\hat{L}_{2}\left(T_{i}\right)=0$, a estos puntos se les denominar\'a puntos regenerativos. Entonces, 

\begin{Def}
Al intervalo de tiempo entre dos puntos regenerativos se le llamar\'a ciclo regenerativo.
\end{Def}

\begin{Def}
Para $T_{i}$ se define, $M_{i}$, el n\'umero de ciclos de visita a la cola $Q_{l}$, durante el ciclo regenerativo, es decir, $M_{i}$ es un proceso de renovaci\'on.
\end{Def}

\begin{Def}
Para cada uno de los $M_{i}$'s, se definen a su vez la duraci\'on de cada uno de estos ciclos de visita en el ciclo regenerativo, $C_{i}^{(m)}$, para $m=1,2,\ldots,M_{i}$, que a su vez, tambi\'en es n proceso de renovaci\'on.
\end{Def}

\footnote{In Stidham and  Heyman \cite{Stidham} shows that is sufficient for the regenerative process to be stationary that the mean regenerative cycle time is finite: $\esp\left[\sum_{m=1}^{M_{i}}C_{i}^{(m)}\right]<\infty$, 


 como cada $C_{i}^{(m)}$ contiene intervalos de r\'eplica positivos, se tiene que $\esp\left[M_{i}\right]<\infty$, adem\'as, como $M_{i}>0$, se tiene que la condici\'on anterior es equivalente a tener que $\esp\left[C_{i}\right]<\infty$,
por lo tanto una condici\'on suficiente para la existencia del proceso regenerativo est\'a dada por $\sum_{k=1}^{N}\mu_{k}<1.$}

Para $\left\{X\left(t\right):t\geq0\right\}$ Proceso Estoc\'astico a tiempo continuo con estado de espacios $S$, que es un espacio m\'etrico, con trayectorias continuas por la derecha y con l\'imites por la izquierda c.s. Sea $N\left(t\right)$ un proceso de renovaci\'on en $\rea_{+}$ definido en el mismo espacio de probabilidad que $X\left(t\right)$, con tiempos de renovaci\'on $T$ y tiempos de inter-renovaci\'on $\xi_{n}=T_{n}-T_{n-1}$, con misma distribuci\'on $F$ de media finita $\mu$.

\begin{Def}
Un elemento aleatorio en un espacio medible $\left(E,\mathcal{E}\right)$ en un espacio de probabilidad $\left(\Omega,\mathcal{F},\prob\right)$ a $\left(E,\mathcal{E}\right)$, es decir,
para $A\in \mathcal{E}$,  se tiene que $\left\{Y\in A\right\}\in\mathcal{F}$, donde $\left\{Y\in A\right\}:=\left\{w\in\Omega:Y\left(w\right)\in A\right\}=:Y^{-1}A$.
\end{Def}

\begin{Note}
Tambi\'en se dice que $Y$ est\'a soportado por el espacio de probabilidad $\left(\Omega,\mathcal{F},\prob\right)$ y que $Y$ es un mapeo medible de $\Omega$ en $E$, es decir, es $\mathcal{F}/\mathcal{E}$ medible.
\end{Note}

\begin{Def}
Para cada $i\in \mathbb{I}$ sea $P_{i}$ una medida de probabilidad en un espacio medible $\left(E_{i},\mathcal{E}_{i}\right)$. Se define el espacio producto
$\otimes_{i\in\mathbb{I}}\left(E_{i},\mathcal{E}_{i}\right):=\left(\prod_{i\in\mathbb{I}}E_{i},\otimes_{i\in\mathbb{I}}\mathcal{E}_{i}\right)$, donde $\prod_{i\in\mathbb{I}}E_{i}$ es el producto cartesiano de los $E_{i}$'s, y $\otimes_{i\in\mathbb{I}}\mathcal{E}_{i}$ es la $\sigma$-\'algebra producto, es decir, es la $\sigma$-\'algebra m\'as peque\~na en $\prod_{i\in\mathbb{I}}E_{i}$ que hace al $i$-\'esimo mapeo proyecci\'on en $E_{i}$ medible para toda $i\in\mathbb{I}$ es la $\sigma$-\'algebra inducida por los mapeos proyecci\'on. $$\otimes_{i\in\mathbb{I}}\mathcal{E}_{i}:=\sigma\left\{\left\{y:y_{i}\in A\right\}:i\in\mathbb{I}\textrm{ y }A\in\mathcal{E}_{i}\right\}.$$
\end{Def}

\begin{Def}
Un espacio de probabilidad $\left(\tilde{\Omega},\tilde{\mathcal{F}},\tilde{\prob}\right)$ es una extensi\'on de otro espacio de probabilidad $\left(\Omega,\mathcal{F},\prob\right)$ si $\left(\tilde{\Omega},\tilde{\mathcal{F}},\tilde{\prob}\right)$ soporta un elemento aleatorio $\xi\in\left(\Omega,\mathcal{F}\right)$ que tienen a $\prob$ como distribuci\'on.
\end{Def}

\begin{Teo}
Sea $\mathbb{I}$ un conjunto de \'indices arbitrario. Para cada $i\in\mathbb{I}$ sea $P_{i}$ una medida de probabilidad en un espacio medible $\left(E_{i},\mathcal{E}_{i}\right)$. Entonces existe una \'unica medida de probabilidad $\otimes_{i\in\mathbb{I}}P_{i}$ en $\otimes_{i\in\mathbb{I}}\left(E_{i},\mathcal{E}_{i}\right)$ tal que 

\begin{eqnarray*}
\otimes_{i\in\mathbb{I}}P_{i}\left(y\in\prod_{i\in\mathbb{I}}E_{i}:y_{i}\in A_{i_{1}},\ldots,y_{n}\in A_{i_{n}}\right)=P_{i_{1}}\left(A_{i_{n}}\right)\cdots P_{i_{n}}\left(A_{i_{n}}\right)
\end{eqnarray*}
para todos los enteros $n>0$, toda $i_{1},\ldots,i_{n}\in\mathbb{I}$ y todo $A_{i_{1}}\in\mathcal{E}_{i_{1}},\ldots,A_{i_{n}}\in\mathcal{E}_{i_{n}}$
\end{Teo}

La medida $\otimes_{i\in\mathbb{I}}P_{i}$ es llamada la medida producto y $\otimes_{i\in\mathbb{I}}\left(E_{i},\mathcal{E}_{i},P_{i}\right):=\left(\prod_{i\in\mathbb{I}},E_{i},\otimes_{i\in\mathbb{I}}\mathcal{E}_{i},\otimes_{i\in\mathbb{I}}P_{i}\right)$, es llamado espacio de probabilidad producto.


\begin{Def}
Un espacio medible $\left(E,\mathcal{E}\right)$ es \textit{Polaco} si existe una m\'etrica en $E$ tal que $E$ es completo, es decir cada sucesi\'on de Cauchy converge a un l\'imite en $E$, y \textit{separable}, $E$ tienen un subconjunto denso numerable, y tal que $\mathcal{E}$ es generado por conjuntos abiertos.
\end{Def}


\begin{Def}
Dos espacios medibles $\left(E,\mathcal{E}\right)$ y $\left(G,\mathcal{G}\right)$ son Borel equivalentes \textit{isomorfos} si existe una biyecci\'on $f:E\rightarrow G$ tal que $f$ es $\mathcal{E}/\mathcal{G}$ medible y su inversa $f^{-1}$ es $\mathcal{G}/\mathcal{E}$ medible. La biyecci\'on es una equivalencia de Borel.
\end{Def}

\begin{Def}
Un espacio medible  $\left(E,\mathcal{E}\right)$ es un \textit{espacio est\'andar} si es Borel equivalente a $\left(G,\mathcal{G}\right)$, donde $G$ es un subconjunto de Borel de $\left[0,1\right]$ y $\mathcal{G}$ son los subconjuntos de Borel de $G$.
\end{Def}

\begin{Note}
Cualquier espacio Polaco es un espacio est\'andar.
\end{Note}


\begin{Def}
Un proceso estoc\'astico con conjunto de \'indices $\mathbb{I}$ y espacio de estados $\left(E,\mathcal{E}\right)$ es una familia $Z=\left(\mathbb{Z}_{s}\right)_{s\in\mathbb{I}}$ donde $\mathbb{Z}_{s}$ son elementos aleatorios definidos en un espacio de probabilidad com\'un $\left(\Omega,\mathcal{F},\prob\right)$ y todos toman valores en $\left(E,\mathcal{E}\right)$.
\end{Def}

\begin{Def}
Un proceso estoc\'astico \textit{one-sided contiuous time} (\textbf{PEOSCT}) es un proceso estoc\'astico con conjunto de \'indices $\mathbb{I}=\left[0,\infty\right)$.
\end{Def}


Sea $\left(E^{\mathbb{I}},\mathcal{E}^{\mathbb{I}}\right)$ denota el espacio producto $\left(E^{\mathbb{I}},\mathcal{E}^{\mathbb{I}}\right):=\otimes_{s\in\mathbb{I}}\left(E,\mathcal{E}\right)$. Vamos a considerar $\mathbb{Z}$ como un mapeo aleatorio, es decir, como un elemento aleatorio en $\left(E^{\mathbb{I}},\mathcal{E}^{\mathbb{I}}\right)$ definido por $Z\left(w\right)=\left(Z_{s}\left(w\right)\right)_{s\in\mathbb{I}}$ y $w\in\Omega$.

\begin{Note}
La distribuci\'on de un proceso estoc\'astico $Z$ es la distribuci\'on de $Z$ como un elemento aleatorio en $\left(E^{\mathbb{I}},\mathcal{E}^{\mathbb{I}}\right)$. La distribuci\'on de $Z$ esta determinada de manera \'unica por las distribuciones finito dimensionales.
\end{Note}

\begin{Note}
En particular cuando $Z$ toma valores reales, es decir, $\left(E,\mathcal{E}\right)=\left(\mathbb{R},\mathcal{B}\right)$ las distribuciones finito dimensionales est\'an determinadas por las funciones de distribuci\'on finito dimensionales

\begin{eqnarray}
\prob\left(Z_{t_{1}}\leq x_{1},\ldots,Z_{t_{n}}\leq x_{n}\right),x_{1},\ldots,x_{n}\in\mathbb{R},t_{1},\ldots,t_{n}\in\mathbb{I},n\geq1.
\end{eqnarray}
\end{Note}

\begin{Note}
Para espacios polacos $\left(E,\mathcal{E}\right)$ el Teorema de Consistencia de Kolmogorov asegura que dada una colecci\'on de distribuciones finito dimensionales consistentes, siempre existe un proceso estoc\'astico que posee tales distribuciones finito dimensionales.
\end{Note}


\begin{Def}
Las trayectorias de $Z$ son las realizaciones $Z\left(w\right)$ para $w\in\Omega$ del mapeo aleatorio $Z$.
\end{Def}

\begin{Note}
Algunas restricciones se imponen sobre las trayectorias, por ejemplo que sean continuas por la derecha, o continuas por la derecha con l\'imites por la izquierda, o de manera m\'as general, se pedir\'a que caigan en alg\'un subconjunto $H$ de $E^{\mathbb{I}}$. En este caso es natural considerar a $Z$ como un elemento aleatorio que no est\'a en $\left(E^{\mathbb{I}},\mathcal{E}^{\mathbb{I}}\right)$ sino en $\left(H,\mathcal{H}\right)$, donde $\mathcal{H}$ es la $\sigma$-\'algebra generada por los mapeos proyecci\'on que toman a $z\in H$ a $z_{t}\in E$ para $t\in\mathbb{I}$. A $\mathcal{H}$ se le conoce como la traza de $H$ en $E^{\mathbb{I}}$, es decir,
\begin{eqnarray}
\mathcal{H}:=E^{\mathbb{I}}\cap H:=\left\{A\cap H:A\in E^{\mathbb{I}}\right\}.
\end{eqnarray}
\end{Note}


\begin{Note}
$Z$ tiene trayectorias con valores en $H$ y cada $Z_{t}$ es un mapeo medible de $\left(\Omega,\mathcal{F}\right)$ a $\left(H,\mathcal{H}\right)$. Cuando se considera un espacio de trayectorias en particular $H$, al espacio $\left(H,\mathcal{H}\right)$ se le llama el espacio de trayectorias de $Z$.
\end{Note}

\begin{Note}
La distribuci\'on del proceso estoc\'astico $Z$ con espacio de trayectorias $\left(H,\mathcal{H}\right)$ es la distribuci\'on de $Z$ como  un elemento aleatorio en $\left(H,\mathcal{H}\right)$. La distribuci\'on, nuevemente, est\'a determinada de manera \'unica por las distribuciones finito dimensionales.
\end{Note}


\begin{Def}
Sea $Z$ un PEOSCT  con espacio de estados $\left(E,\mathcal{E}\right)$ y sea $T$ un tiempo aleatorio en $\left[0,\infty\right)$. Por $Z_{T}$ se entiende el mapeo con valores en $E$ definido en $\Omega$ en la manera obvia:
\begin{eqnarray*}
Z_{T}\left(w\right):=Z_{T\left(w\right)}\left(w\right). w\in\Omega.
\end{eqnarray*}
\end{Def}

\begin{Def}
Un PEOSCT $Z$ es conjuntamente medible (\textbf{CM}) si el mapeo que toma $\left(w,t\right)\in\Omega\times\left[0,\infty\right)$ a $Z_{t}\left(w\right)\in E$ es $\mathcal{F}\otimes\mathcal{B}\left[0,\infty\right)/\mathcal{E}$ medible.
\end{Def}

\begin{Note}
Un PEOSCT-CM implica que el proceso es medible, dado que $Z_{T}$ es una composici\'on  de dos mapeos continuos: el primero que toma $w$ en $\left(w,T\left(w\right)\right)$ es $\mathcal{F}/\mathcal{F}\otimes\mathcal{B}\left[0,\infty\right)$ medible, mientras que el segundo toma $\left(w,T\left(w\right)\right)$ en $Z_{T\left(w\right)}\left(w\right)$ es $\mathcal{F}\otimes\mathcal{B}\left[0,\infty\right)/\mathcal{E}$ medible.
\end{Note}


\begin{Def}
Un PEOSCT con espacio de estados $\left(H,\mathcal{H}\right)$ es can\'onicamente conjuntamente medible (\textbf{CCM}) si el mapeo $\left(z,t\right)\in H\times\left[0,\infty\right)$ en $Z_{t}\in E$ es $\mathcal{H}\otimes\mathcal{B}\left[0,\infty\right)/\mathcal{E}$ medible.
\end{Def}

\begin{Note}
Un PEOSCTCCM implica que el proceso es CM, dado que un PECCM $Z$ es un mapeo de $\Omega\times\left[0,\infty\right)$ a $E$, es la composici\'on de dos mapeos medibles: el primero, toma $\left(w,t\right)$ en $\left(Z\left(w\right),t\right)$ es $\mathcal{F}\otimes\mathcal{B}\left[0,\infty\right)/\mathcal{H}\otimes\mathcal{B}\left[0,\infty\right)$ medible, y el segundo que toma $\left(Z\left(w\right),t\right)$  en $Z_{t}\left(w\right)$ es $\mathcal{H}\otimes\mathcal{B}\left[0,\infty\right)/\mathcal{E}$ medible. Por tanto CCM es una condici\'on m\'as fuerte que CM.
\end{Note}

\begin{Def}
Un conjunto de trayectorias $H$ de un PEOSCT $Z$, es internamente shift-invariante (\textbf{ISI}) si 
\begin{eqnarray*}
\left\{\left(z_{t+s}\right)_{s\in\left[0,\infty\right)}:z\in H\right\}=H\textrm{, }t\in\left[0,\infty\right).
\end{eqnarray*}
\end{Def}


\begin{Def}
Dado un PEOSCTISI, se define el mapeo-shift $\theta_{t}$, $t\in\left[0,\infty\right)$, de $H$ a $H$ por 
\begin{eqnarray*}
\theta_{t}z=\left(z_{t+s}\right)_{s\in\left[0,\infty\right)}\textrm{, }z\in H.
\end{eqnarray*}
\end{Def}

\begin{Def}
Se dice que un proceso $Z$ es shift-medible (\textbf{SM}) si $Z$ tiene un conjunto de trayectorias $H$ que es ISI y adem\'as el mapeo que toma $\left(z,t\right)\in H\times\left[0,\infty\right)$ en $\theta_{t}z\in H$ es $\mathcal{H}\otimes\mathcal{B}\left[0,\infty\right)/\mathcal{H}$ medible.
\end{Def}

\begin{Note}
Un proceso estoc\'astico con conjunto de trayectorias $H$ ISI es shift-medible si y s\'olo si es CCM
\end{Note}

\begin{Note}
\begin{itemize}
\item Dado el espacio polaco $\left(E,\mathcal{E}\right)$ se tiene el  conjunto de trayectorias $D_{E}\left[0,\infty\right)$ que es ISI, entonces cumpe con ser CCM.

\item Si $G$ es abierto, podemos cubrirlo por bolas abiertas cuay cerradura este contenida en $G$, y como $G$ es segundo numerable como subespacio de $E$, lo podemos cubrir por una cantidad numerable de bolas abiertas.

\end{itemize}
\end{Note}


\begin{Note}
Los procesos estoc\'asticos $Z$ a tiempo discreto con espacio de estados polaco, tambi\'en tiene un espacio de trayectorias polaco y por tanto tiene distribuciones condicionales regulares.
\end{Note}

\begin{Teo}
El producto numerable de espacios polacos es polaco.
\end{Teo}


\begin{Def}
Sea $\left(\Omega,\mathcal{F},\prob\right)$ espacio de probabilidad que soporta al proceso $Z=\left(Z_{s}\right)_{s\in\left[0,\infty\right)}$ y $S=\left(S_{k}\right)_{0}^{\infty}$ donde $Z$ es un PEOSCTM con espacio de estados $\left(E,\mathcal{E}\right)$  y espacio de trayectorias $\left(H,\mathcal{H}\right)$  y adem\'as $S$ es una sucesi\'on de tiempos aleatorios one-sided que satisfacen la condici\'on $0\leq S_{0}<S_{1}<\cdots\rightarrow\infty$. Considerando $S$ como un mapeo medible de $\left(\Omega,\mathcal{F}\right)$ al espacio sucesi\'on $\left(L,\mathcal{L}\right)$, donde 
\begin{eqnarray*}
L=\left\{\left(s_{k}\right)_{0}^{\infty}\in\left[0,\infty\right)^{\left\{0,1,\ldots\right\}}:s_{0}<s_{1}<\cdots\rightarrow\infty\right\},
\end{eqnarray*}
donde $\mathcal{L}$ son los subconjuntos de Borel de $L$, es decir, $\mathcal{L}=L\cap\mathcal{B}^{\left\{0,1,\ldots\right\}}$.

As\'i el par $\left(Z,S\right)$ es un mapeo medible de  $\left(\Omega,\mathcal{F}\right)$ en $\left(H\times L,\mathcal{H}\otimes\mathcal{L}\right)$. El par $\mathcal{H}\otimes\mathcal{L}^{+}$ denotar\'a la clase de todas las funciones medibles de $\left(H\times L,\mathcal{H}\otimes\mathcal{L}\right)$ en $\left(\left[0,\infty\right),\mathcal{B}\left[0,\infty\right)\right)$.
\end{Def}


\begin{Def}
Sea $\theta_{t}$ el mapeo-shift conjunto de $H\times L$ en $H\times L$ dado por
\begin{eqnarray*}
\theta_{t}\left(z,\left(s_{k}\right)_{0}^{\infty}\right)=\theta_{t}\left(z,\left(s_{n_{t-}+k}-t\right)_{0}^{\infty}\right)
\end{eqnarray*}
donde 
$n_{t-}=inf\left\{n\geq1:s_{n}\geq t\right\}$.
\end{Def}

\begin{Note}
Con la finalidad de poder realizar los shift's sin complicaciones de medibilidad, se supondr\'a que $Z$ es shit-medible, es decir, el conjunto de trayectorias $H$ es invariante bajo shifts del tiempo y el mapeo que toma $\left(z,t\right)\in H\times\left[0,\infty\right)$ en $z_{t}\in E$ es $\mathcal{H}\otimes\mathcal{B}\left[0,\infty\right)/\mathcal{E}$ medible.
\end{Note}

\begin{Def}
Dado un proceso \textbf{PEOSSM} (Proceso Estoc\'astico One Side Shift Medible) $Z$, se dice regenerativo cl\'asico con tiempos de regeneraci\'on $S$ si 

\begin{eqnarray*}
\theta_{S_{n}}\left(Z,S\right)=\left(Z^{0},S^{0}\right),n\geq0
\end{eqnarray*}
y adem\'as $\theta_{S_{n}}\left(Z,S\right)$ es independiente de $\left(\left(Z_{s}\right)s\in\left[0,S_{n}\right),S_{0},\ldots,S_{n}\right)$
Si lo anterior se cumple, al par $\left(Z,S\right)$ se le llama regenerativo cl\'asico.
\end{Def}

\begin{Note}
Si el par $\left(Z,S\right)$ es regenerativo cl\'asico, entonces las longitudes de los ciclos $X_{1},X_{2},\ldots,$ son i.i.d. e independientes de la longitud del retraso $S_{0}$, es decir, $S$ es un proceso de renovaci\'on. Las longitudes de los ciclos tambi\'en son llamados tiempos de inter-regeneraci\'on y tiempos de ocurrencia.

\end{Note}

\begin{Teo}
Sup\'ongase que el par $\left(Z,S\right)$ es regenerativo cl\'asico con $\esp\left[X_{1}\right]<\infty$. Entonces $\left(Z^{*},S^{*}\right)$ en el teorema 2.1 es una versi\'on estacionaria de $\left(Z,S\right)$. Adem\'as, si $X_{1}$ es lattice con span $d$, entonces $\left(Z^{**},S^{**}\right)$ en el teorema 2.2 es una versi\'on periodicamente estacionaria de $\left(Z,S\right)$ con periodo $d$.

\end{Teo}

\begin{Def}
Una variable aleatoria $X_{1}$ es \textit{spread out} si existe una $n\geq1$ y una  funci\'on $f\in\mathcal{B}^{+}$ tal que $\int_{\rea}f\left(x\right)dx>0$ con $X_{2},X_{3},\ldots,X_{n}$ copias i.i.d  de $X_{1}$, $$\prob\left(X_{1}+\cdots+X_{n}\in B\right)\geq\int_{B}f\left(x\right)dx$$ para $B\in\mathcal{B}$.

\end{Def}



\begin{Def}
Dado un proceso estoc\'astico $Z$ se le llama \textit{wide-sense regenerative} (\textbf{WSR}) con tiempos de regeneraci\'on $S$ si $\theta_{S_{n}}\left(Z,S\right)=\left(Z^{0},S^{0}\right)$ para $n\geq0$ en distribuci\'on y $\theta_{S_{n}}\left(Z,S\right)$ es independiente de $\left(S_{0},S_{1},\ldots,S_{n}\right)$ para $n\geq0$.
Se dice que el par $\left(Z,S\right)$ es WSR si lo anterior se cumple.
\end{Def}


\begin{Note}
\begin{itemize}
\item El proceso de trayectorias $\left(\theta_{s}Z\right)_{s\in\left[0,\infty\right)}$ es WSR con tiempos de regeneraci\'on $S$ pero no es regenerativo cl\'asico.

\item Si $Z$ es cualquier proceso estacionario y $S$ es un proceso de renovaci\'on que es independiente de $Z$, entonces $\left(Z,S\right)$ es WSR pero en general no es regenerativo cl\'asico

\end{itemize}

\end{Note}


\begin{Note}
Para cualquier proceso estoc\'astico $Z$, el proceso de trayectorias $\left(\theta_{s}Z\right)_{s\in\left[0,\infty\right)}$ es siempre un proceso de Markov.
\end{Note}



\begin{Teo}
Supongase que el par $\left(Z,S\right)$ es WSR con $\esp\left[X_{1}\right]<\infty$. Entonces $\left(Z^{*},S^{*}\right)$ en el teorema 2.1 es una versi\'on estacionaria de 
$\left(Z,S\right)$.
\end{Teo}


\begin{Teo}
Supongase que $\left(Z,S\right)$ es cycle-stationary con $\esp\left[X_{1}\right]<\infty$. Sea $U$ distribuida uniformemente en $\left[0,1\right)$ e independiente de $\left(Z^{0},S^{0}\right)$ y sea $\prob^{*}$ la medida de probabilidad en $\left(\Omega,\prob\right)$ definida por $$d\prob^{*}=\frac{X_{1}}{\esp\left[X_{1}\right]}d\prob$$. Sea $\left(Z^{*},S^{*}\right)$ con distribuci\'on $\prob^{*}\left(\theta_{UX_{1}}\left(Z^{0},S^{0}\right)\in\cdot\right)$. Entonces $\left(Z^{}*,S^{*}\right)$ es estacionario,
\begin{eqnarray*}
\esp\left[f\left(Z^{*},S^{*}\right)\right]=\esp\left[\int_{0}^{X_{1}}f\left(\theta_{s}\left(Z^{0},S^{0}\right)\right)ds\right]/\esp\left[X_{1}\right]
\end{eqnarray*}
$f\in\mathcal{H}\otimes\mathcal{L}^{+}$, and $S_{0}^{*}$ es continuo con funci\'on distribuci\'on $G_{\infty}$ definida por $$G_{\infty}\left(x\right):=\frac{\esp\left[X_{1}\right]\wedge x}{\esp\left[X_{1}\right]}$$ para $x\geq0$ y densidad $\prob\left[X_{1}>x\right]/\esp\left[X_{1}\right]$, con $x\geq0$.

\end{Teo}


\begin{Teo}
Sea $Z$ un Proceso Estoc\'astico un lado shift-medible \textit{one-sided shift-measurable stochastic process}, (PEOSSM),
y $S_{0}$ y $S_{1}$ tiempos aleatorios tales que $0\leq S_{0}<S_{1}$ y
\begin{equation}
\theta_{S_{1}}Z=\theta_{S_{0}}Z\textrm{ en distribuci\'on}.
\end{equation}

Entonces el espacio de probabilidad subyacente $\left(\Omega,\mathcal{F},\prob\right)$ puede extenderse para soportar una sucesi\'on de tiempos aleatorios $S$ tales que

\begin{eqnarray}
\theta_{S_{n}}\left(Z,S\right)=\left(Z^{0},S^{0}\right),n\geq0,\textrm{ en distribuci\'on},\\
\left(Z,S_{0},S_{1}\right)\textrm{ depende de }\left(X_{2},X_{3},\ldots\right)\textrm{ solamente a traves de }\theta_{S_{1}}Z.
\end{eqnarray}
\end{Teo}





%_________________________________________________________________________
%
%\subsection{Una vez que se tiene estabilidad}
%_________________________________________________________________________
%

Also the intervisit time $I_{i}$ is defined as the period beginning at the time of its service completion in a cycle and ending at the time when it is polled in the next cycle; its duration is given by $\tau_{i}\left(m+1\right)-\overline{\tau}_{i}\left(m\right)$.

So we the following are still true 

\begin{eqnarray}
\begin{array}{ll}
\esp\left[L_{i}\right]=\mu_{i}\esp\left[I_{i}\right], &
\esp\left[C_{i}\right]=\frac{f_{i}\left(i\right)}{\mu_{i}\left(1-\mu_{i}\right)},\\
\esp\left[S_{i}\right]=\mu_{i}\esp\left[C_{i}\right],&
\esp\left[I_{i}\right]=\left(1-\mu_{i}\right)\esp\left[C_{i}\right],\\
Var\left[L_{i}\right]= \mu_{i}^{2}Var\left[I_{i}\right]+\sigma^{2}\esp\left[I_{i}\right],& 
Var\left[C_{i}\right]=\frac{Var\left[L_{i}^{*}\right]}{\mu_{i}^{2}\left(1-\mu_{i}\right)^{2}},\\
Var\left[S_{i}\right]= \frac{Var\left[L_{i}^{*}\right]}{\left(1-\mu_{i}\right)^{2}}+\frac{\sigma^{2}\esp\left[L_{i}^{*}\right]}{\left(1-\mu_{i}\right)^{3}},&
Var\left[I_{i}\right]= \frac{Var\left[L_{i}^{*}\right]}{\mu_{i}^{2}}-\frac{\sigma_{i}^{2}}{\mu_{i}^{2}}f_{i}\left(i\right).
\end{array}
\end{eqnarray}
\begin{Def}
El tiempo de Ciclo $C_{i}$ es el periodo de tiempo que comienza cuando la cola $i$ es visitada por primera vez en un ciclo, y termina cuando es visitado nuevamente en el pr\'oximo ciclo. La duraci\'on del mismo est\'a dada por $\tau_{i}\left(m+1\right)-\tau_{i}\left(m\right)$, o equivalentemente $\overline{\tau}_{i}\left(m+1\right)-\overline{\tau}_{i}\left(m\right)$ bajo condiciones de estabilidad.
\end{Def}


\begin{Def}
El tiempo de intervisita $I_{i}$ es el periodo de tiempo que comienza cuando se ha completado el servicio en un ciclo y termina cuando es visitada nuevamente en el pr\'oximo ciclo. Su  duraci\'on del mismo est\'a dada por $\tau_{i}\left(m+1\right)-\overline{\tau}_{i}\left(m\right)$.
\end{Def}

La duraci\'on del tiempo de intervisita es $\tau_{i}\left(m+1\right)-\overline{\tau}\left(m\right)$. Dado que el n\'umero de usuarios presentes en $Q_{i}$ al tiempo $t=\tau_{i}\left(m+1\right)$ es igual al n\'umero de arribos durante el intervalo de tiempo $\left[\overline{\tau}\left(m\right),\tau_{i}\left(m+1\right)\right]$ se tiene que


\begin{eqnarray*}
\esp\left[z_{i}^{L_{i}\left(\tau_{i}\left(m+1\right)\right)}\right]=\esp\left[\left\{P_{i}\left(z_{i}\right)\right\}^{\tau_{i}\left(m+1\right)-\overline{\tau}\left(m\right)}\right]
\end{eqnarray*}

entonces, si $I_{i}\left(z\right)=\esp\left[z^{\tau_{i}\left(m+1\right)-\overline{\tau}\left(m\right)}\right]$
se tiene que $F_{i}\left(z\right)=I_{i}\left[P_{i}\left(z\right)\right]$
para $i=1,2$.

Conforme a la definici\'on dada al principio del cap\'itulo, definici\'on (\ref{Def.Tn}), sean $T_{1},T_{2},\ldots$ los puntos donde las longitudes de las colas de la red de sistemas de visitas c\'iclicas son cero simult\'aneamente, cuando la cola $Q_{j}$ es visitada por el servidor para dar servicio, es decir, $L_{1}\left(T_{i}\right)=0,L_{2}\left(T_{i}\right)=0,\hat{L}_{1}\left(T_{i}\right)=0$ y $\hat{L}_{2}\left(T_{i}\right)=0$, a estos puntos se les denominar\'a puntos regenerativos. Entonces, 

\begin{Def}
Al intervalo de tiempo entre dos puntos regenerativos se le llamar\'a ciclo regenerativo.
\end{Def}

\begin{Def}
Para $T_{i}$ se define, $M_{i}$, el n\'umero de ciclos de visita a la cola $Q_{l}$, durante el ciclo regenerativo, es decir, $M_{i}$ es un proceso de renovaci\'on.
\end{Def}

\begin{Def}
Para cada uno de los $M_{i}$'s, se definen a su vez la duraci\'on de cada uno de estos ciclos de visita en el ciclo regenerativo, $C_{i}^{(m)}$, para $m=1,2,\ldots,M_{i}$, que a su vez, tambi\'en es n proceso de renovaci\'on.
\end{Def}


Sea la funci\'on generadora de momentos para $L_{i}$, el n\'umero de usuarios en la cola $Q_{i}\left(z\right)$ en cualquier momento, est\'a dada por el tiempo promedio de $z^{L_{i}\left(t\right)}$ sobre el ciclo regenerativo definido anteriormente:

\begin{eqnarray*}
Q_{i}\left(z\right)&=&\esp\left[z^{L_{i}\left(t\right)}\right]=\frac{\esp\left[\sum_{m=1}^{M_{i}}\sum_{t=\tau_{i}\left(m\right)}^{\tau_{i}\left(m+1\right)-1}z^{L_{i}\left(t\right)}\right]}{\esp\left[\sum_{m=1}^{M_{i}}\tau_{i}\left(m+1\right)-\tau_{i}\left(m\right)\right]}
\end{eqnarray*}

$M_{i}$ es un tiempo de paro en el proceso regenerativo con $\esp\left[M_{i}\right]<\infty$\footnote{En Stidham\cite{Stidham} y Heyman  se muestra que una condici\'on suficiente para que el proceso regenerativo 
estacionario sea un procesoo estacionario es que el valor esperado del tiempo del ciclo regenerativo sea finito, es decir: $\esp\left[\sum_{m=1}^{M_{i}}C_{i}^{(m)}\right]<\infty$, como cada $C_{i}^{(m)}$ contiene intervalos de r\'eplica positivos, se tiene que $\esp\left[M_{i}\right]<\infty$, adem\'as, como $M_{i}>0$, se tiene que la condici\'on anterior es equivalente a tener que $\esp\left[C_{i}\right]<\infty$,
por lo tanto una condici\'on suficiente para la existencia del proceso regenerativo est\'a dada por $\sum_{k=1}^{N}\mu_{k}<1.$}, se sigue del lema de Wald que:


\begin{eqnarray*}
\esp\left[\sum_{m=1}^{M_{i}}\sum_{t=\tau_{i}\left(m\right)}^{\tau_{i}\left(m+1\right)-1}z^{L_{i}\left(t\right)}\right]&=&\esp\left[M_{i}\right]\esp\left[\sum_{t=\tau_{i}\left(m\right)}^{\tau_{i}\left(m+1\right)-1}z^{L_{i}\left(t\right)}\right]\\
\esp\left[\sum_{m=1}^{M_{i}}\tau_{i}\left(m+1\right)-\tau_{i}\left(m\right)\right]&=&\esp\left[M_{i}\right]\esp\left[\tau_{i}\left(m+1\right)-\tau_{i}\left(m\right)\right]
\end{eqnarray*}

por tanto se tiene que


\begin{eqnarray*}
Q_{i}\left(z\right)&=&\frac{\esp\left[\sum_{t=\tau_{i}\left(m\right)}^{\tau_{i}\left(m+1\right)-1}z^{L_{i}\left(t\right)}\right]}{\esp\left[\tau_{i}\left(m+1\right)-\tau_{i}\left(m\right)\right]}
\end{eqnarray*}

observar que el denominador es simplemente la duraci\'on promedio del tiempo del ciclo.


Haciendo las siguientes sustituciones en la ecuaci\'on (\ref{Corolario2}): $n\rightarrow t-\tau_{i}\left(m\right)$, $T \rightarrow \overline{\tau}_{i}\left(m\right)-\tau_{i}\left(m\right)$, $L_{n}\rightarrow L_{i}\left(t\right)$ y $F\left(z\right)=\esp\left[z^{L_{0}}\right]\rightarrow F_{i}\left(z\right)=\esp\left[z^{L_{i}\tau_{i}\left(m\right)}\right]$, se puede ver que

\begin{eqnarray}\label{Eq.Arribos.Primera}
\esp\left[\sum_{n=0}^{T-1}z^{L_{n}}\right]=
\esp\left[\sum_{t=\tau_{i}\left(m\right)}^{\overline{\tau}_{i}\left(m\right)-1}z^{L_{i}\left(t\right)}\right]
=z\frac{F_{i}\left(z\right)-1}{z-P_{i}\left(z\right)}
\end{eqnarray}

Por otra parte durante el tiempo de intervisita para la cola $i$, $L_{i}\left(t\right)$ solamente se incrementa de manera que el incremento por intervalo de tiempo est\'a dado por la funci\'on generadora de probabilidades de $P_{i}\left(z\right)$, por tanto la suma sobre el tiempo de intervisita puede evaluarse como:

\begin{eqnarray*}
\esp\left[\sum_{t=\tau_{i}\left(m\right)}^{\tau_{i}\left(m+1\right)-1}z^{L_{i}\left(t\right)}\right]&=&\esp\left[\sum_{t=\tau_{i}\left(m\right)}^{\tau_{i}\left(m+1\right)-1}\left\{P_{i}\left(z\right)\right\}^{t-\overline{\tau}_{i}\left(m\right)}\right]=\frac{1-\esp\left[\left\{P_{i}\left(z\right)\right\}^{\tau_{i}\left(m+1\right)-\overline{\tau}_{i}\left(m\right)}\right]}{1-P_{i}\left(z\right)}\\
&=&\frac{1-I_{i}\left[P_{i}\left(z\right)\right]}{1-P_{i}\left(z\right)}
\end{eqnarray*}
por tanto

\begin{eqnarray*}
\esp\left[\sum_{t=\tau_{i}\left(m\right)}^{\tau_{i}\left(m+1\right)-1}z^{L_{i}\left(t\right)}\right]&=&
\frac{1-F_{i}\left(z\right)}{1-P_{i}\left(z\right)}
\end{eqnarray*}

Por lo tanto

\begin{eqnarray*}
Q_{i}\left(z\right)&=&\frac{\esp\left[\sum_{t=\tau_{i}\left(m\right)}^{\tau_{i}\left(m+1\right)-1}z^{L_{i}\left(t\right)}\right]}{\esp\left[\tau_{i}\left(m+1\right)-\tau_{i}\left(m\right)\right]}
=\frac{1}{\esp\left[\tau_{i}\left(m+1\right)-\tau_{i}\left(m\right)\right]}
\esp\left[\sum_{t=\tau_{i}\left(m\right)}^{\tau_{i}\left(m+1\right)-1}z^{L_{i}\left(t\right)}\right]\\
&=&\frac{1}{\esp\left[\tau_{i}\left(m+1\right)-\tau_{i}\left(m\right)\right]}
\esp\left[\sum_{t=\tau_{i}\left(m\right)}^{\overline{\tau}_{i}\left(m\right)-1}z^{L_{i}\left(t\right)}
+\sum_{t=\overline{\tau}_{i}\left(m\right)}^{\tau_{i}\left(m+1\right)-1}z^{L_{i}\left(t\right)}\right]\\
&=&\frac{1}{\esp\left[\tau_{i}\left(m+1\right)-\tau_{i}\left(m\right)\right]}\left\{
\esp\left[\sum_{t=\tau_{i}\left(m\right)}^{\overline{\tau}_{i}\left(m\right)-1}z^{L_{i}\left(t\right)}\right]
+\esp\left[\sum_{t=\overline{\tau}_{i}\left(m\right)}^{\tau_{i}\left(m+1\right)-1}z^{L_{i}\left(t\right)}\right]\right\}\\
&=&\frac{1}{\esp\left[\tau_{i}\left(m+1\right)-\tau_{i}\left(m\right)\right]}\left\{
z\frac{F_{i}\left(z\right)-1}{z-P_{i}\left(z\right)}+\frac{1-F_{i}\left(z\right)}{1-P_{i}\left(z\right)}
\right\}\\
&=&\frac{1}{\esp\left[C_{i}\right]}\cdot\frac{1-F_{i}\left(z\right)}{P_{i}\left(z\right)-z}\cdot\frac{\left(1-z\right)P_{i}\left(z\right)}{1-P_{i}\left(z\right)}
\end{eqnarray*}

es decir

\begin{equation}
Q_{i}\left(z\right)=\frac{1}{\esp\left[C_{i}\right]}\cdot\frac{1-F_{i}\left(z\right)}{P_{i}\left(z\right)-z}\cdot\frac{\left(1-z\right)P_{i}\left(z\right)}{1-P_{i}\left(z\right)}
\end{equation}


Si hacemos:

\begin{eqnarray}
S\left(z\right)&=&1-F\left(z\right)\\
T\left(z\right)&=&z-P\left(z\right)\\
U\left(z\right)&=&1-P\left(z\right)
\end{eqnarray}
entonces 

\begin{eqnarray}
\esp\left[C_{i}\right]Q\left(z\right)=\frac{\left(z-1\right)S\left(z\right)P\left(z\right)}{T\left(z\right)U\left(z\right)}
\end{eqnarray}

A saber, si $a_{k}=P\left\{L\left(t\right)=k\right\}$
\begin{eqnarray*}
S\left(z\right)=1-F\left(z\right)=1-\sum_{k=0}^{+\infty}a_{k}z^{k}
\end{eqnarray*}
entonces

%\begin{eqnarray}
%\begin{array}{ll}
%S^{'}\left(z\right)=-\sum_{k=1}^{+\infty}ka_{k}z^{k-1},& %S^{(1)}\left(1\right)=-\sum_{k=1}^{+\infty}ka_{k}=-\esp\left[L\left(t\right)\right],\\
%S^{''}\left(z\right)=-\sum_{k=2}^{+\infty}k(k-1)a_{k}z^{k-2},& S^{(2)}\left(1\right)=-\sum_{k=2}^{+\infty}k(k-1)a_{k}=\esp\left[L\left(L-1\right)\right],\\
%S^{'''}\left(z\right)=-\sum_{k=3}^{+\infty}k(k-1)(k-2)a_{k}z^{k-3},&
%S^{(3)}\left(1\right)=-\sum_{k=3}^{+\infty}k(k-1)(k-2)a_{k}\\
%&=-\esp\left[L\left(L-1\right)\left(L-2\right)\right]\\
%&=-\esp\left[L^{3}\right]+3-\esp\left[L^{2}\right]-2-\esp\left[L\right];
%\end{array}
%\end{eqnarray}

$S^{'}\left(z\right)=-\sum_{k=1}^{+\infty}ka_{k}z^{k-1}$, por tanto $S^{(1)}\left(1\right)=-\sum_{k=1}^{+\infty}ka_{k}=-\esp\left[L\left(t\right)\right]$,
luego $S^{''}\left(z\right)=-\sum_{k=2}^{+\infty}k(k-1)a_{k}z^{k-2}$ y $S^{(2)}\left(1\right)=-\sum_{k=2}^{+\infty}k(k-1)a_{k}=\esp\left[L\left(L-1\right)\right]$;
de la misma manera $S^{'''}\left(z\right)=-\sum_{k=3}^{+\infty}k(k-1)(k-2)a_{k}z^{k-3}$ y $S^{(3)}\left(1\right)=-\sum_{k=3}^{+\infty}k(k-1)(k-2)a_{k}=-\esp\left[L\left(L-1\right)\left(L-2\right)\right]
=-\esp\left[L^{3}\right]+3-\esp\left[L^{2}\right]-2-\esp\left[L\right]$. 

Es decir

\begin{eqnarray*}
S^{(1)}\left(1\right)&=&-\esp\left[L\left(t\right)\right],\\ S^{(2)}\left(1\right)&=&-\esp\left[L\left(L-1\right)\right]
=-\esp\left[L^{2}\right]+\esp\left[L\right],\\
S^{(3)}\left(1\right)&=&-\esp\left[L\left(L-1\right)\left(L-2\right)\right]
=-\esp\left[L^{3}\right]+3\esp\left[L^{2}\right]-2\esp\left[L\right].
\end{eqnarray*}


Expandiendo alrededor de $z=1$

\begin{eqnarray*}
S\left(z\right)&=&S\left(1\right)+\frac{S^{'}\left(1\right)}{1!}\left(z-1\right)+\frac{S^{''}\left(1\right)}{2!}\left(z-1\right)^{2}+\frac{S^{'''}\left(1\right)}{3!}\left(z-1\right)^{3}+\ldots+\\
&=&\left(z-1\right)\left\{S^{'}\left(1\right)+\frac{S^{''}\left(1\right)}{2!}\left(z-1\right)+\frac{S^{'''}\left(1\right)}{3!}\left(z-1\right)^{2}+\ldots+\right\}\\
&=&\left(z-1\right)R_{1}\left(z\right)
\end{eqnarray*}
con $R_{1}\left(z\right)\neq0$, pues

\begin{eqnarray}\label{Eq.R1}
R_{1}\left(z\right)=-\esp\left[L\right]
\end{eqnarray}
entonces

\begin{eqnarray}
R_{1}\left(z\right)&=&S^{'}\left(1\right)+\frac{S^{''}\left(1\right)}{2!}\left(z-1\right)+\frac{S^{'''}\left(1\right)}{3!}\left(z-1\right)^{2}+\frac{S^{iv}\left(1\right)}{4!}\left(z-1\right)^{3}+\ldots+
\end{eqnarray}
Calculando las derivadas y evaluando en $z=1$

\begin{eqnarray}
R_{1}\left(1\right)&=&S^{(1)}\left(1\right)=-\esp\left[L\right]\\
R_{1}^{(1)}\left(1\right)&=&\frac{1}{2}S^{(2)}\left(1\right)=-\frac{1}{2}\esp\left[L^{2}\right]+\frac{1}{2}\esp\left[L\right]\\
R_{1}^{(2)}\left(1\right)&=&\frac{2}{3!}S^{(3)}\left(1\right)
=-\frac{1}{3}\esp\left[L^{3}\right]+\esp\left[L^{2}\right]-\frac{2}{3}\esp\left[L\right]
\end{eqnarray}

De manera an\'aloga se puede ver que para $T\left(z\right)=z-P\left(z\right)$ se puede encontrar una expanci\'on alrededor de $z=1$

Expandiendo alrededor de $z=1$

\begin{eqnarray*}
T\left(z\right)&=&T\left(1\right)+\frac{T^{'}\left(1\right)}{1!}\left(z-1\right)+\frac{T^{''}\left(1\right)}{2!}\left(z-1\right)^{2}+\frac{T^{'''}\left(1\right)}{3!}\left(z-1\right)^{3}+\ldots+\\
&=&\left(z-1\right)\left\{T^{'}\left(1\right)+\frac{T^{''}\left(1\right)}{2!}\left(z-1\right)+\frac{T^{'''}\left(1\right)}{3!}\left(z-1\right)^{2}+\ldots+\right\}\\
&=&\left(z-1\right)R_{2}\left(z\right)
\end{eqnarray*}

donde 
\begin{eqnarray*}
T^{(1)}\left(1\right)&=&-\esp\left[X\left(t\right)\right]=-\mu,\\ T^{(2)}\left(1\right)&=&-\esp\left[X\left(X-1\right)\right]
=-\esp\left[X^{2}\right]+\esp\left[X\right]=-\esp\left[X^{2}\right]+\mu,\\
T^{(3)}\left(1\right)&=&-\esp\left[X\left(X-1\right)\left(X-2\right)\right]
=-\esp\left[X^{3}\right]+3\esp\left[X^{2}\right]-2\esp\left[X\right]\\
&=&-\esp\left[X^{3}\right]+3\esp\left[X^{2}\right]-2\mu.
\end{eqnarray*}

Por lo tanto $R_{2}\left(1\right)\neq0$, pues

\begin{eqnarray}\label{Eq.R2}
R_{2}\left(1\right)=1-\esp\left[X\right]=1-\mu
\end{eqnarray}
entonces

\begin{eqnarray}
R_{2}\left(z\right)&=&T^{'}\left(1\right)+\frac{T^{''}\left(1\right)}{2!}\left(z-1\right)+\frac{T^{'''}\left(1\right)}{3!}\left(z-1\right)^{2}+\frac{T^{(iv)}\left(1\right)}{4!}\left(z-1\right)^{3}+\ldots+
\end{eqnarray}
Calculando las derivadas y evaluando en $z=1$

\begin{eqnarray}
R_{2}\left(1\right)&=&T^{(1)}\left(1\right)=1-\mu\\
R_{2}^{(1)}\left(1\right)&=&\frac{1}{2}T^{(2)}\left(1\right)=-\frac{1}{2}\esp\left[X^{2}\right]+\frac{1}{2}\mu\\
R_{2}^{(2)}\left(1\right)&=&\frac{2}{3!}T^{(3)}\left(1\right)
=-\frac{1}{3}\esp\left[X^{3}\right]+\esp\left[X^{2}\right]-\frac{2}{3}\mu
\end{eqnarray}

Finalmente para de manera an\'aloga se puede ver que para $U\left(z\right)=1-P\left(z\right)$ se puede encontrar una expanci\'on alrededor de $z=1$

\begin{eqnarray*}
U\left(z\right)&=&U\left(1\right)+\frac{U^{'}\left(1\right)}{1!}\left(z-1\right)+\frac{U^{''}\left(1\right)}{2!}\left(z-1\right)^{2}+\frac{U^{'''}\left(1\right)}{3!}\left(z-1\right)^{3}+\ldots+\\
&=&\left(z-1\right)\left\{U^{'}\left(1\right)+\frac{U^{''}\left(1\right)}{2!}\left(z-1\right)+\frac{U^{'''}\left(1\right)}{3!}\left(z-1\right)^{2}+\ldots+\right\}\\
&=&\left(z-1\right)R_{3}\left(z\right)
\end{eqnarray*}

donde 
\begin{eqnarray*}
U^{(1)}\left(1\right)&=&-\esp\left[X\left(t\right)\right]=-\mu,\\ U^{(2)}\left(1\right)&=&-\esp\left[X\left(X-1\right)\right]
=-\esp\left[X^{2}\right]+\esp\left[X\right]=-\esp\left[X^{2}\right]+\mu,\\
U^{(3)}\left(1\right)&=&-\esp\left[X\left(X-1\right)\left(X-2\right)\right]
=-\esp\left[X^{3}\right]+3\esp\left[X^{2}\right]-2\esp\left[X\right]\\
&=&-\esp\left[X^{3}\right]+3\esp\left[X^{2}\right]-2\mu.
\end{eqnarray*}

Por lo tanto $R_{3}\left(1\right)\neq0$, pues

\begin{eqnarray}\label{Eq.R2}
R_{3}\left(1\right)=-\esp\left[X\right]=-\mu
\end{eqnarray}
entonces

\begin{eqnarray}
R_{3}\left(z\right)&=&U^{'}\left(1\right)+\frac{U^{''}\left(1\right)}{2!}\left(z-1\right)+\frac{U^{'''}\left(1\right)}{3!}\left(z-1\right)^{2}+\frac{U^{(iv)}\left(1\right)}{4!}\left(z-1\right)^{3}+\ldots+
\end{eqnarray}

Calculando las derivadas y evaluando en $z=1$

\begin{eqnarray}
R_{3}\left(1\right)&=&U^{(1)}\left(1\right)=-\mu\\
R_{3}^{(1)}\left(1\right)&=&\frac{1}{2}U^{(2)}\left(1\right)=-\frac{1}{2}\esp\left[X^{2}\right]+\frac{1}{2}\mu\\
R_{3}^{(2)}\left(1\right)&=&\frac{2}{3!}U^{(3)}\left(1\right)
=-\frac{1}{3}\esp\left[X^{3}\right]+\esp\left[X^{2}\right]-\frac{2}{3}\mu
\end{eqnarray}

Por lo tanto

\begin{eqnarray}
\esp\left[C_{i}\right]Q\left(z\right)&=&\frac{\left(z-1\right)\left(z-1\right)R_{1}\left(z\right)P\left(z\right)}{\left(z-1\right)R_{2}\left(z\right)\left(z-1\right)R_{3}\left(z\right)}
=\frac{R_{1}\left(z\right)P\left(z\right)}{R_{2}\left(z\right)R_{3}\left(z\right)}\equiv\frac{R_{1}P}{R_{2}R_{3}}
\end{eqnarray}

Entonces

\begin{eqnarray}\label{Eq.Primer.Derivada.Q}
\left[\frac{R_{1}\left(z\right)P\left(z\right)}{R_{2}\left(z\right)R_{3}\left(z\right)}\right]^{'}&=&\frac{PR_{2}R_{3}R_{1}^{'}
+R_{1}R_{2}R_{3}P^{'}-R_{3}R_{1}PR_{2}-R_{2}R_{1}PR_{3}^{'}}{\left(R_{2}R_{3}\right)^{2}}
\end{eqnarray}
Evaluando en $z=1$
\begin{eqnarray*}
&=&\frac{R_{2}(1)R_{3}(1)R_{1}^{(1)}(1)+R_{1}(1)R_{2}(1)R_{3}(1)P^{'}(1)-R_{3}(1)R_{1}(1)R_{2}(1)^{(1)}-R_{2}(1)R_{1}(1)R_{3}^{'}(1)}{\left(R_{2}(1)R_{3}(1)\right)^{2}}\\
&=&\frac{1}{\left(1-\mu\right)^{2}\mu^{2}}\left\{\left(-\frac{1}{2}\esp L^{2}+\frac{1}{2}\esp L\right)\left(1-\mu\right)\left(-\mu\right)+\left(-\esp L\right)\left(1-\mu\right)\left(-\mu\right)\mu\right.\\
&&\left.-\left(-\frac{1}{2}\esp X^{2}+\frac{1}{2}\mu\right)\left(-\mu\right)\left(-\esp L\right)-\left(1-\mu\right)\left(-\esp L\right)\left(-\frac{1}{2}\esp X^{2}+\frac{1}{2}\mu\right)\right\}\\
&=&\frac{1}{\left(1-\mu\right)^{2}\mu^{2}}\left\{\left(-\frac{1}{2}\esp L^{2}+\frac{1}{2}\esp L\right)\left(\mu^{2}-\mu\right)
+\left(\mu^{2}-\mu^{3}\right)\esp L\right.\\
&&\left.-\mu\esp L\left(-\frac{1}{2}\esp X^{2}+\frac{1}{2}\mu\right)
+\left(\esp L-\mu\esp L\right)\left(-\frac{1}{2}\esp X^{2}+\frac{1}{2}\mu\right)\right\}\\
&=&\frac{1}{\left(1-\mu\right)^{2}\mu^{2}}\left\{-\frac{1}{2}\mu^{2}\esp L^{2}
+\frac{1}{2}\mu\esp L^{2}
+\frac{1}{2}\mu^{2}\esp L
-\mu^{3}\esp L
+\mu\esp L\esp X^{2}
-\frac{1}{2}\esp L\esp X^{2}\right\}\\
&=&\frac{1}{\left(1-\mu\right)^{2}\mu^{2}}\left\{
\frac{1}{2}\mu\esp L^{2}\left(1-\mu\right)
+\esp L\left(\frac{1}{2}-\mu\right)\left(\mu^{2}-\esp X^{2}\right)\right\}\\
&=&\frac{1}{2\mu\left(1-\mu\right)}\esp L^{2}-\frac{\frac{1}{2}-\mu}{\left(1-\mu\right)^{2}\mu^{2}}\sigma^{2}\esp L
\end{eqnarray*}

por lo tanto (para Takagi)

\begin{eqnarray*}
Q^{(1)}=\frac{1}{\esp C}\left\{\frac{1}{2\mu\left(1-\mu\right)}\esp L^{2}-\frac{\frac{1}{2}-\mu}{\left(1-\mu\right)^{2}\mu^{2}}\sigma^{2}\esp L\right\}
\end{eqnarray*}
donde 

\begin{eqnarray*}
\esp C = \frac{\esp L}{\mu\left(1-\mu\right)}
\end{eqnarray*}
entonces

\begin{eqnarray*}
Q^{(1)}&=&\frac{1}{2}\frac{\esp L^{2}}{\esp L}-\frac{\frac{1}{2}-\mu}{\left(1-\mu\right)\mu}\sigma^{2}
=\frac{\esp L^{2}}{2\esp L}-\frac{\sigma^{2}}{2}\left\{\frac{2\mu-1}{\left(1-\mu\right)\mu}\right\}\\
&=&\frac{\esp L^{2}}{2\esp L}+\frac{\sigma^{2}}{2}\left\{\frac{1}{1-\mu}+\frac{1}{\mu}\right\}
\end{eqnarray*}

Mientras que para nosotros

\begin{eqnarray*}
Q^{(1)}=\frac{1}{\mu\left(1-\mu\right)}\frac{\esp L^{2}}{2\esp C}
-\sigma^{2}\frac{\esp L}{2\esp C}\cdot\frac{1-2\mu}{\left(1-\mu\right)^{2}\mu^{2}}
\end{eqnarray*}

Retomando la ecuaci\'on (\ref{Eq.Primer.Derivada.Q})

\begin{eqnarray*}
\left[\frac{R_{1}\left(z\right)P\left(z\right)}{R_{2}\left(z\right)R_{3}\left(z\right)}\right]^{'}&=&\frac{PR_{2}R_{3}R_{1}^{'}
+R_{1}R_{2}R_{3}P^{'}-R_{3}R_{1}PR_{2}-R_{2}R_{1}PR_{3}^{'}}{\left(R_{2}R_{3}\right)^{2}}
=\frac{F\left(z\right)}{G\left(z\right)}
\end{eqnarray*}

donde 

\begin{eqnarray*}
F\left(z\right)&=&PR_{2}R_{3}R_{1}^{'}
+R_{1}R_{2}R_{3}P^{'}-R_{3}R_{1}PR_{2}^{'}-R_{2}R_{1}PR_{3}^{'}\\
G\left(z\right)&=&R_{2}^{2}R_{3}^{2}\\
G^{2}\left(z\right)&=&R_{2}^{4}R_{3}^{4}=\left(1-\mu\right)^{4}\mu^{4}
\end{eqnarray*}
y por tanto

\begin{eqnarray*}
G^{'}\left(z\right)&=&2R_{2}R_{3}\left[R_{2}^{'}R_{3}+R_{2}R_{3}^{'}\right]\\
G^{'}\left(1\right)&=&-2\left(1-\mu\right)\mu\left[\left(-\frac{1}{2}\esp\left[X^{2}\right]+\frac{1}{2}\mu\right)\left(-\mu\right)+\left(1-\mu\right)\left(-\frac{1}{2}\esp\left[X^{2}\right]+\frac{1}{2}\mu\right)\right]
\end{eqnarray*}


\begin{eqnarray*}
F^{'}\left(z\right)&=&\left[\left(R_{2}R_{3}\right)R_{1}^{''}
-\left(R_{1}R_{3}\right)R_{2}^{''}
-\left(R_{1}R_{2}\right)R_{3}^{''}
-2\left(R_{2}^{'}R_{3}^{'}\right)R_{1}\right]P
+2\left(R_{2}R_{3}\right)R_{1}^{'}P^{'}
+\left(R_{1}R_{2}R_{3}\right)P^{''}
\end{eqnarray*}

Por lo tanto, encontremos $F^{'}\left(z\right)G\left(z\right)+F\left(z\right)G^{'}\left(z\right)$:

\begin{eqnarray*}
&&F^{'}\left(z\right)G\left(z\right)+F\left(z\right)G^{'}\left(z\right)=
\left\{\left[\left(R_{2}R_{3}\right)R_{1}^{''}
-\left(R_{1}R_{3}\right)R_{2}^{''}
-\left(R_{1}R_{2}\right)R_{3}^{''}
-2\left(R_{2}^{'}R_{3}^{'}\right)R_{1}\right]P\right.\\
&&\left.+2\left(R_{2}R_{3}\right)R_{1}^{'}P^{'}
+\left(R_{1}R_{2}R_{3}\right)P^{''}\right\}R_{2}^{2}R_{3}^{2}
-\left\{\left[PR_{2}R_{3}R_{1}^{'}+R_{1}R_{2}R_{3}P^{'}
-R_{3}R_{1}PR_{2}^{'}\right.\right.\\
&&\left.\left.
-R_{2}R_{1}PR_{3}^{'}\right]\left[2R_{2}R_{3}\left(R_{2}^{'}R_{3}+R_{2}R_{3}^{'}\right)\right]\right\}
\end{eqnarray*}
Evaluando en $z=1$

\begin{eqnarray*}
&=&\left(1+R_{3}\right)^{3}R_{3}^{3}R_{1}^{''}-\left(1+R_{3}\right)^{2}R_{1}R_{3}^{3}R_{3}^{''}
-\left(1+R_{3}\right)^{3}R_{3}^{2}R_{1}R_{3}^{''}-2\left(1+R_{3}\right)^{2}R_{3}^{2}
\left(R_{3}^{'}\right)^{2}\\
&+&2\left(1+R_{3}\right)^{3}R_{3}^{3}R_{1}^{'}P^{'}
+\left(1+R_{3}\right)^{3}R_{3}^{3}R_{1}P^{''}
-2\left(1+R_{3}\right)^{2}R_{3}^{2}\left(1+2R_{3}\right)R_{3}^{'}R_{1}^{'}\\
&-&2\left(1+R_{3}\right)^{2}R_{3}^{2}R_{1}R_{3}^{'}\left(1+2R_{3}\right)P^{'}
+2\left(1+R_{3}\right)\left(1+2R_{3}\right)R_{3}^{3}R_{1}\left(R_{3}^{'}\right)^{2}\\
&+&2\left(1+R_{3}\right)^{2}\left(1+2R_{3}\right)R_{1}R_{3}R_{3}^{'}\\
&=&-\left(1-\mu\right)^{3}\mu^{3}R_{1}^{''}-\left(1-\mu\right)^{2}\mu^{2}R_{1}\left(1-2\mu\right)R_{3}^{''}
-\left(1-\mu\right)^{3}\mu^{3}R_{1}P^{''}\\
&+&2\left(1-\mu\right)\mu^{2}\left[\left(1-2\mu\right)R_{1}-\left(1-\mu\right)\right]\left(R_{3}^{'}\right)^{2}
-2\left(1-\mu\right)^{2}\mu R_{1}\left(1-2\mu\right)R_{3}^{'}\\
&-&2\left(1-\mu\right)^{3}\mu^{4}R_{1}^{'}-2\mu\left(1-\mu\right)\left(1-2\mu\right)R_{3}^{'}R_{1}^{'}
-2\mu^{3}\left(1-\mu\right)^{2}\left(1-2\mu\right)R_{1}R_{1}^{'}
\end{eqnarray*}

por tanto

\begin{eqnarray*}
\left[\frac{F\left(z\right)}{G\left(z\right)}\right]^{'}&=&\frac{1}{\mu^{3}\left(1-\mu\right)^{3}}\left\{
-\left(1-\mu\right)^{2}\mu^{2}R_{1}^{''}-\mu\left(1-\mu\right)\left(1-2\mu\right)R_{1}R_{3}^{''}
-\mu^{2}\left(1-\mu\right)^{2}R_{1}P^{''}\right.\\
&&\left.+2\mu\left[\left(1-2\mu\right)R_{1}-\left(1-\mu\right)\right]\left(R_{3}^{'}\right)^{2}
-2\left(1-\mu\right)\left(1-2\mu\right)R_{1}R_{3}^{'}-2\mu^{3}\left(1-\mu\right)^{2}R_{1}^{'}\right.\\
&&\left.-2\left(1-2\mu\right)R_{3}^{'}R_{1}^{'}-2\mu^{2}\left(1-\mu\right)\left(1-2\mu\right)R_{1}R_{1}^{'}\right\}
\end{eqnarray*}

recordemos que


\begin{eqnarray*}
R_{1}&=&-\esp L\\
R_{3}&=& -\mu\\
R_{1}^{'}&=&-\frac{1}{2}\esp L^{2}+\frac{1}{2}\esp L\\
R_{3}^{'}&=&-\frac{1}{2}\esp X^{2}+\frac{1}{2}\mu\\
R_{1}^{''}&=&-\frac{1}{3}\esp L^{3}+\esp L^{2}-\frac{2}{3}\esp L\\
R_{3}^{''}&=&-\frac{1}{3}\esp X^{3}+\esp X^{2}-\frac{2}{3}\mu\\
R_{1}R_{3}^{'}&=&\frac{1}{2}\esp X^{2}\esp L-\frac{1}{2}\esp X\esp L\\
R_{1}R_{1}^{'}&=&\frac{1}{2}\esp L^{2}\esp L+\frac{1}{2}\esp^{2}L\\
R_{3}^{'}R_{1}^{'}&=&\frac{1}{4}\esp X^{2}\esp L^{2}-\frac{1}{4}\esp X^{2}\esp L-\frac{1}{4}\esp L^{2}\esp X+\frac{1}{4}\esp X\esp L\\
R_{1}R_{3}^{''}&=&\frac{1}{6}\esp X^{3}\esp L^{2}-\frac{1}{6}\esp X^{3}\esp L-\frac{1}{2}\esp L^{2}\esp X^{2}+\frac{1}{2}\esp X^{2}\esp L+\frac{1}{3}\esp X\esp L^{2}-\frac{1}{3}\esp X\esp L\\
R_{1}P^{''}&=&-\esp X^{2}\esp L\\
\left(R_{3}^{'}\right)^{2}&=&\frac{1}{4}\esp^{2}X^{2}-\frac{1}{2}\esp X^{2}\esp X+\frac{1}{4}\esp^{2} X
\end{eqnarray*}




\begin{Def}
Let $L_{i}^{*}$ be the number of users at queue $Q_{i}$ when it is polled, then
\begin{eqnarray}
\begin{array}{cc}
\esp\left[L_{i}^{*}\right]=f_{i}\left(i\right), &
Var\left[L_{i}^{*}\right]=f_{i}\left(i,i\right)+\esp\left[L_{i}^{*}\right]-\esp\left[L_{i}^{*}\right]^{2}.
\end{array}
\end{eqnarray}
\end{Def}

\begin{Def}
The cycle time $C_{i}$ for the queue $Q_{i}$ is the period beginning at the time when it is polled in a cycle and ending at the time when it is polled in the next cycle; it's duration is given by $\tau_{i}\left(m+1\right)-\tau_{i}\left(m\right)$, equivalently $\overline{\tau}_{i}\left(m+1\right)-\overline{\tau}_{i}\left(m\right)$ under steady state assumption.
\end{Def}

\begin{Def}
The intervisit time $I_{i}$ is defined as the period beginning at the time of its service completion in a cycle and ending at the time when it is polled in the next cycle; its duration is given by $\tau_{i}\left(m+1\right)-\overline{\tau}_{i}\left(m\right)$.
\end{Def}

The intervisit time duration $\tau_{i}\left(m+1\right)-\overline{\tau}\left(m\right)$ given the number of users found at queue $Q_{i}$ at time $t=\tau_{i}\left(m+1\right)$ is equal to the number of arrivals during the preceding intervisit time $\left[\overline{\tau}\left(m\right),\tau_{i}\left(m+1\right)\right]$. 

So we have



\begin{eqnarray*}
\esp\left[z_{i}^{L_{i}\left(\tau_{i}\left(m+1\right)\right)}\right]=\esp\left[\left\{P_{i}\left(z_{i}\right)\right\}^{\tau_{i}\left(m+1\right)-\overline{\tau}\left(m\right)}\right]
\end{eqnarray*}

if $I_{i}\left(z\right)=\esp\left[z^{\tau_{i}\left(m+1\right)-\overline{\tau}\left(m\right)}\right]$
we have $F_{i}\left(z\right)=I_{i}\left[P_{i}\left(z\right)\right]$
for $i=1,2$. Futhermore can be proved that

\begin{eqnarray}
\begin{array}{ll}
\esp\left[L_{i}\right]=\mu_{i}\esp\left[I_{i}\right], &
\esp\left[C_{i}\right]=\frac{f_{i}\left(i\right)}{\mu_{i}\left(1-\mu_{i}\right)},\\
\esp\left[S_{i}\right]=\mu_{i}\esp\left[C_{i}\right],&
\esp\left[I_{i}\right]=\left(1-\mu_{i}\right)\esp\left[C_{i}\right],\\
Var\left[L_{i}\right]= \mu_{i}^{2}Var\left[I_{i}\right]+\sigma^{2}\esp\left[I_{i}\right],& 
Var\left[C_{i}\right]=\frac{Var\left[L_{i}^{*}\right]}{\mu_{i}^{2}\left(1-\mu_{i}\right)^{2}},\\
Var\left[S_{i}\right]= \frac{Var\left[L_{i}^{*}\right]}{\left(1-\mu_{i}\right)^{2}}+\frac{\sigma^{2}\esp\left[L_{i}^{*}\right]}{\left(1-\mu_{i}\right)^{3}},&
Var\left[I_{i}\right]= \frac{Var\left[L_{i}^{*}\right]}{\mu_{i}^{2}}-\frac{\sigma_{i}^{2}}{\mu_{i}^{2}}f_{i}\left(i\right).
\end{array}
\end{eqnarray}

Let consider the points when the process $\left[L_{1}\left(1\right),L_{2}\left(1\right),L_{3}\left(1\right),L_{4}\left(1\right)
\right]$ becomes zero at the same time, this points, $T_{1},T_{2},\ldots$ will be denoted as regeneration points, then we have that

\begin{Def}
the interval between two such succesive regeneration points will be called regenerative cycle.
\end{Def}

\begin{Def}
Para $T_{i}$ se define, $M_{i}$, el n\'umero de ciclos de visita a la cola $Q_{l}$, durante el ciclo regenerativo, es decir, $M_{i}$ es un proceso de renovaci\'on.
\end{Def}

\begin{Def}
Para cada uno de los $M_{i}$'s, se definen a su vez la duraci\'on de cada uno de estos ciclos de visita en el ciclo regenerativo, $C_{i}^{(m)}$, para $m=1,2,\ldots,M_{i}$, que a su vez, tambi\'en es n proceso de renovaci\'on.
\end{Def}



Sea la funci\'on generadora de momentos para $L_{i}$, el n\'umero de usuarios en la cola $Q_{i}\left(z\right)$ en cualquier momento, est\'a dada por el tiempo promedio de $z^{L_{i}\left(t\right)}$ sobre el ciclo regenerativo definido anteriormente. Entonces 

\begin{equation}\label{Eq.Longitud.Tiempo.t}
Q_{i}\left(z\right)=\frac{1}{\esp\left[C_{i}\right]}\cdot\frac{1-F_{i}\left(z\right)}{P_{i}\left(z\right)-z}\cdot\frac{\left(1-z\right)P_{i}\left(z\right)}{1-P_{i}\left(z\right)}.
\end{equation}

Es decir, es posible determinar las longitudes de las colas a cualquier tiempo $t$. Entonces, determinando el primer momento es posible ver que


$M_{i}$ is an stopping time for the regenerative process with $\esp\left[M_{i}\right]<\infty$, from Wald's lemma follows that:


\begin{eqnarray*}
\esp\left[\sum_{m=1}^{M_{i}}\sum_{t=\tau_{i}\left(m\right)}^{\tau_{i}\left(m+1\right)-1}z^{L_{i}\left(t\right)}\right]&=&\esp\left[M_{i}\right]\esp\left[\sum_{t=\tau_{i}\left(m\right)}^{\tau_{i}\left(m+1\right)-1}z^{L_{i}\left(t\right)}\right]\\
\esp\left[\sum_{m=1}^{M_{i}}\tau_{i}\left(m+1\right)-\tau_{i}\left(m\right)\right]&=&\esp\left[M_{i}\right]\esp\left[\tau_{i}\left(m+1\right)-\tau_{i}\left(m\right)\right]
\end{eqnarray*}
therefore 

\begin{eqnarray*}
Q_{i}\left(z\right)&=&\frac{\esp\left[\sum_{t=\tau_{i}\left(m\right)}^{\tau_{i}\left(m+1\right)-1}z^{L_{i}\left(t\right)}\right]}{\esp\left[\tau_{i}\left(m+1\right)-\tau_{i}\left(m\right)\right]}
\end{eqnarray*}

Doing the following substitutions en (\ref{Corolario2}): $n\rightarrow t-\tau_{i}\left(m\right)$, $T \rightarrow \overline{\tau}_{i}\left(m\right)-\tau_{i}\left(m\right)$, $L_{n}\rightarrow L_{i}\left(t\right)$ and $F\left(z\right)=\esp\left[z^{L_{0}}\right]\rightarrow F_{i}\left(z\right)=\esp\left[z^{L_{i}\tau_{i}\left(m\right)}\right]$, 
we obtain

\begin{eqnarray}\label{Eq.Arribos.Primera}
\esp\left[\sum_{n=0}^{T-1}z^{L_{n}}\right]=
\esp\left[\sum_{t=\tau_{i}\left(m\right)}^{\overline{\tau}_{i}\left(m\right)-1}z^{L_{i}\left(t\right)}\right]
=z\frac{F_{i}\left(z\right)-1}{z-P_{i}\left(z\right)}
\end{eqnarray}



Por otra parte durante el tiempo de intervisita para la cola $i$, $L_{i}\left(t\right)$ solamente se incrementa de manera que el incremento por intervalo de tiempo est\'a dado por la funci\'on generadora de probabilidades de $P_{i}\left(z\right)$, por tanto la suma sobre el tiempo de intervisita puede evaluarse como:

\begin{eqnarray*}
\esp\left[\sum_{t=\tau_{i}\left(m\right)}^{\tau_{i}\left(m+1\right)-1}z^{L_{i}\left(t\right)}\right]&=&\esp\left[\sum_{t=\tau_{i}\left(m\right)}^{\tau_{i}\left(m+1\right)-1}\left\{P_{i}\left(z\right)\right\}^{t-\overline{\tau}_{i}\left(m\right)}\right]=\frac{1-\esp\left[\left\{P_{i}\left(z\right)\right\}^{\tau_{i}\left(m+1\right)-\overline{\tau}_{i}\left(m\right)}\right]}{1-P_{i}\left(z\right)}\\
&=&\frac{1-I_{i}\left[P_{i}\left(z\right)\right]}{1-P_{i}\left(z\right)}
\end{eqnarray*}
por tanto

\begin{eqnarray*}
\esp\left[\sum_{t=\tau_{i}\left(m\right)}^{\tau_{i}\left(m+1\right)-1}z^{L_{i}\left(t\right)}\right]&=&
\frac{1-F_{i}\left(z\right)}{1-P_{i}\left(z\right)}
\end{eqnarray*}

Por lo tanto

\begin{eqnarray*}
Q_{i}\left(z\right)&=&\frac{\esp\left[\sum_{t=\tau_{i}\left(m\right)}^{\tau_{i}\left(m+1\right)-1}z^{L_{i}\left(t\right)}\right]}{\esp\left[\tau_{i}\left(m+1\right)-\tau_{i}\left(m\right)\right]}
=\frac{1}{\esp\left[\tau_{i}\left(m+1\right)-\tau_{i}\left(m\right)\right]}
\esp\left[\sum_{t=\tau_{i}\left(m\right)}^{\tau_{i}\left(m+1\right)-1}z^{L_{i}\left(t\right)}\right]\\
&=&\frac{1}{\esp\left[\tau_{i}\left(m+1\right)-\tau_{i}\left(m\right)\right]}
\esp\left[\sum_{t=\tau_{i}\left(m\right)}^{\overline{\tau}_{i}\left(m\right)-1}z^{L_{i}\left(t\right)}
+\sum_{t=\overline{\tau}_{i}\left(m\right)}^{\tau_{i}\left(m+1\right)-1}z^{L_{i}\left(t\right)}\right]\\
&=&\frac{1}{\esp\left[\tau_{i}\left(m+1\right)-\tau_{i}\left(m\right)\right]}\left\{
\esp\left[\sum_{t=\tau_{i}\left(m\right)}^{\overline{\tau}_{i}\left(m\right)-1}z^{L_{i}\left(t\right)}\right]
+\esp\left[\sum_{t=\overline{\tau}_{i}\left(m\right)}^{\tau_{i}\left(m+1\right)-1}z^{L_{i}\left(t\right)}\right]\right\}\\
&=&\frac{1}{\esp\left[\tau_{i}\left(m+1\right)-\tau_{i}\left(m\right)\right]}\left\{
z\frac{F_{i}\left(z\right)-1}{z-P_{i}\left(z\right)}+\frac{1-F_{i}\left(z\right)}{1-P_{i}\left(z\right)}
\right\}\\
&=&\frac{1}{\esp\left[C_{i}\right]}\cdot\frac{1-F_{i}\left(z\right)}{P_{i}\left(z\right)-z}\cdot\frac{\left(1-z\right)P_{i}\left(z\right)}{1-P_{i}\left(z\right)}
\end{eqnarray*}

es decir

\begin{eqnarray}
\begin{array}{ll}
S^{'}\left(z\right)=-\sum_{k=1}^{+\infty}ka_{k}z^{k-1},& S^{(1)}\left(1\right)=-\sum_{k=1}^{+\infty}ka_{k}=-\esp\left[L\left(t\right)\right],\\
S^{''}\left(z\right)=-\sum_{k=2}^{+\infty}k(k-1)a_{k}z^{k-2},& S^{(2)}\left(1\right)=-\sum_{k=2}^{+\infty}k(k-1)a_{k}=\esp\left[L\left(L-1\right)\right],\\
S^{'''}\left(z\right)=-\sum_{k=3}^{+\infty}k(k-1)(k-2)a_{k}z^{k-3},&
S^{(3)}\left(1\right)=-\sum_{k=3}^{+\infty}k(k-1)(k-2)a_{k}\\
&=-\esp\left[L\left(L-1\right)\left(L-2\right)\right]\\
&=-\esp\left[L^{3}\right]+3-\esp\left[L^{2}\right]-2-\esp\left[L\right];
\end{array}
\end{eqnarray}








%________________________________________________________________________
%\subsection{Procesos Regenerativos Sigman, Thorisson y Wolff \cite{Sigman1}}
%________________________________________________________________________


\begin{Def}[Definici\'on Cl\'asica]
Un proceso estoc\'astico $X=\left\{X\left(t\right):t\geq0\right\}$ es llamado regenerativo is existe una variable aleatoria $R_{1}>0$ tal que
\begin{itemize}
\item[i)] $\left\{X\left(t+R_{1}\right):t\geq0\right\}$ es independiente de $\left\{\left\{X\left(t\right):t<R_{1}\right\},\right\}$
\item[ii)] $\left\{X\left(t+R_{1}\right):t\geq0\right\}$ es estoc\'asticamente equivalente a $\left\{X\left(t\right):t>0\right\}$
\end{itemize}

Llamamos a $R_{1}$ tiempo de regeneraci\'on, y decimos que $X$ se regenera en este punto.
\end{Def}

$\left\{X\left(t+R_{1}\right)\right\}$ es regenerativo con tiempo de regeneraci\'on $R_{2}$, independiente de $R_{1}$ pero con la misma distribuci\'on que $R_{1}$. Procediendo de esta manera se obtiene una secuencia de variables aleatorias independientes e id\'enticamente distribuidas $\left\{R_{n}\right\}$ llamados longitudes de ciclo. Si definimos a $Z_{k}\equiv R_{1}+R_{2}+\cdots+R_{k}$, se tiene un proceso de renovaci\'on llamado proceso de renovaci\'on encajado para $X$.


\begin{Note}
La existencia de un primer tiempo de regeneraci\'on, $R_{1}$, implica la existencia de una sucesi\'on completa de estos tiempos $R_{1},R_{2}\ldots,$ que satisfacen la propiedad deseada \cite{Sigman2}.
\end{Note}


\begin{Note} Para la cola $GI/GI/1$ los usuarios arriban con tiempos $t_{n}$ y son atendidos con tiempos de servicio $S_{n}$, los tiempos de arribo forman un proceso de renovaci\'on  con tiempos entre arribos independientes e identicamente distribuidos (\texttt{i.i.d.})$T_{n}=t_{n}-t_{n-1}$, adem\'as los tiempos de servicio son \texttt{i.i.d.} e independientes de los procesos de arribo. Por \textit{estable} se entiende que $\esp S_{n}<\esp T_{n}<\infty$.
\end{Note}
 


\begin{Def}
Para $x$ fijo y para cada $t\geq0$, sea $I_{x}\left(t\right)=1$ si $X\left(t\right)\leq x$,  $I_{x}\left(t\right)=0$ en caso contrario, y def\'inanse los tiempos promedio
\begin{eqnarray*}
\overline{X}&=&lim_{t\rightarrow\infty}\frac{1}{t}\int_{0}^{\infty}X\left(u\right)du\\
\prob\left(X_{\infty}\leq x\right)&=&lim_{t\rightarrow\infty}\frac{1}{t}\int_{0}^{\infty}I_{x}\left(u\right)du,
\end{eqnarray*}
cuando estos l\'imites existan.
\end{Def}

Como consecuencia del teorema de Renovaci\'on-Recompensa, se tiene que el primer l\'imite  existe y es igual a la constante
\begin{eqnarray*}
\overline{X}&=&\frac{\esp\left[\int_{0}^{R_{1}}X\left(t\right)dt\right]}{\esp\left[R_{1}\right]},
\end{eqnarray*}
suponiendo que ambas esperanzas son finitas.
 
\begin{Note}
Funciones de procesos regenerativos son regenerativas, es decir, si $X\left(t\right)$ es regenerativo y se define el proceso $Y\left(t\right)$ por $Y\left(t\right)=f\left(X\left(t\right)\right)$ para alguna funci\'on Borel medible $f\left(\cdot\right)$. Adem\'as $Y$ es regenerativo con los mismos tiempos de renovaci\'on que $X$. 

En general, los tiempos de renovaci\'on, $Z_{k}$ de un proceso regenerativo no requieren ser tiempos de paro con respecto a la evoluci\'on de $X\left(t\right)$.
\end{Note} 

\begin{Note}
Una funci\'on de un proceso de Markov, usualmente no ser\'a un proceso de Markov, sin embargo ser\'a regenerativo si el proceso de Markov lo es.
\end{Note}

 
\begin{Note}
Un proceso regenerativo con media de la longitud de ciclo finita es llamado positivo recurrente.
\end{Note}


\begin{Note}
\begin{itemize}
\item[a)] Si el proceso regenerativo $X$ es positivo recurrente y tiene trayectorias muestrales no negativas, entonces la ecuaci\'on anterior es v\'alida.
\item[b)] Si $X$ es positivo recurrente regenerativo, podemos construir una \'unica versi\'on estacionaria de este proceso, $X_{e}=\left\{X_{e}\left(t\right)\right\}$, donde $X_{e}$ es un proceso estoc\'astico regenerativo y estrictamente estacionario, con distribuci\'on marginal distribuida como $X_{\infty}$
\end{itemize}
\end{Note}


%__________________________________________________________________________________________
%\subsection{Procesos Regenerativos Estacionarios - Stidham \cite{Stidham}}
%__________________________________________________________________________________________


Un proceso estoc\'astico a tiempo continuo $\left\{V\left(t\right),t\geq0\right\}$ es un proceso regenerativo si existe una sucesi\'on de variables aleatorias independientes e id\'enticamente distribuidas $\left\{X_{1},X_{2},\ldots\right\}$, sucesi\'on de renovaci\'on, tal que para cualquier conjunto de Borel $A$, 

\begin{eqnarray*}
\prob\left\{V\left(t\right)\in A|X_{1}+X_{2}+\cdots+X_{R\left(t\right)}=s,\left\{V\left(\tau\right),\tau<s\right\}\right\}=\prob\left\{V\left(t-s\right)\in A|X_{1}>t-s\right\},
\end{eqnarray*}
para todo $0\leq s\leq t$, donde $R\left(t\right)=\max\left\{X_{1}+X_{2}+\cdots+X_{j}\leq t\right\}=$n\'umero de renovaciones ({\emph{puntos de regeneraci\'on}}) que ocurren en $\left[0,t\right]$. El intervalo $\left[0,X_{1}\right)$ es llamado {\emph{primer ciclo de regeneraci\'on}} de $\left\{V\left(t \right),t\geq0\right\}$, $\left[X_{1},X_{1}+X_{2}\right)$ el {\emph{segundo ciclo de regeneraci\'on}}, y as\'i sucesivamente.

Sea $X=X_{1}$ y sea $F$ la funci\'on de distrbuci\'on de $X$


\begin{Def}
Se define el proceso estacionario, $\left\{V^{*}\left(t\right),t\geq0\right\}$, para $\left\{V\left(t\right),t\geq0\right\}$ por

\begin{eqnarray*}
\prob\left\{V\left(t\right)\in A\right\}=\frac{1}{\esp\left[X\right]}\int_{0}^{\infty}\prob\left\{V\left(t+x\right)\in A|X>x\right\}\left(1-F\left(x\right)\right)dx,
\end{eqnarray*} 
para todo $t\geq0$ y todo conjunto de Borel $A$.
\end{Def}

\begin{Def}
Una distribuci\'on se dice que es {\emph{aritm\'etica}} si todos sus puntos de incremento son m\'ultiplos de la forma $0,\lambda, 2\lambda,\ldots$ para alguna $\lambda>0$ entera.
\end{Def}


\begin{Def}
Una modificaci\'on medible de un proceso $\left\{V\left(t\right),t\geq0\right\}$, es una versi\'on de este, $\left\{V\left(t,w\right)\right\}$ conjuntamente medible para $t\geq0$ y para $w\in S$, $S$ espacio de estados para $\left\{V\left(t\right),t\geq0\right\}$.
\end{Def}

\begin{Teo}
Sea $\left\{V\left(t\right),t\geq\right\}$ un proceso regenerativo no negativo con modificaci\'on medible. Sea $\esp\left[X\right]<\infty$. Entonces el proceso estacionario dado por la ecuaci\'on anterior est\'a bien definido y tiene funci\'on de distribuci\'on independiente de $t$, adem\'as
\begin{itemize}
\item[i)] \begin{eqnarray*}
\esp\left[V^{*}\left(0\right)\right]&=&\frac{\esp\left[\int_{0}^{X}V\left(s\right)ds\right]}{\esp\left[X\right]}\end{eqnarray*}
\item[ii)] Si $\esp\left[V^{*}\left(0\right)\right]<\infty$, equivalentemente, si $\esp\left[\int_{0}^{X}V\left(s\right)ds\right]<\infty$,entonces
\begin{eqnarray*}
\frac{\int_{0}^{t}V\left(s\right)ds}{t}\rightarrow\frac{\esp\left[\int_{0}^{X}V\left(s\right)ds\right]}{\esp\left[X\right]}
\end{eqnarray*}
con probabilidad 1 y en media, cuando $t\rightarrow\infty$.
\end{itemize}
\end{Teo}

\begin{Coro}
Sea $\left\{V\left(t\right),t\geq0\right\}$ un proceso regenerativo no negativo, con modificaci\'on medible. Si $\esp <\infty$, $F$ es no-aritm\'etica, y para todo $x\geq0$, $P\left\{V\left(t\right)\leq x,C>x\right\}$ es de variaci\'on acotada como funci\'on de $t$ en cada intervalo finito $\left[0,\tau\right]$, entonces $V\left(t\right)$ converge en distribuci\'on  cuando $t\rightarrow\infty$ y $$\esp V=\frac{\esp \int_{0}^{X}V\left(s\right)ds}{\esp X}$$
Donde $V$ tiene la distribuci\'on l\'imite de $V\left(t\right)$ cuando $t\rightarrow\infty$.

\end{Coro}

Para el caso discreto se tienen resultados similares.



%______________________________________________________________________
%\subsection{Procesos de Renovaci\'on}
%______________________________________________________________________

\begin{Def}%\label{Def.Tn}
Sean $0\leq T_{1}\leq T_{2}\leq \ldots$ son tiempos aleatorios infinitos en los cuales ocurren ciertos eventos. El n\'umero de tiempos $T_{n}$ en el intervalo $\left[0,t\right)$ es

\begin{eqnarray}
N\left(t\right)=\sum_{n=1}^{\infty}\indora\left(T_{n}\leq t\right),
\end{eqnarray}
para $t\geq0$.
\end{Def}

Si se consideran los puntos $T_{n}$ como elementos de $\rea_{+}$, y $N\left(t\right)$ es el n\'umero de puntos en $\rea$. El proceso denotado por $\left\{N\left(t\right):t\geq0\right\}$, denotado por $N\left(t\right)$, es un proceso puntual en $\rea_{+}$. Los $T_{n}$ son los tiempos de ocurrencia, el proceso puntual $N\left(t\right)$ es simple si su n\'umero de ocurrencias son distintas: $0<T_{1}<T_{2}<\ldots$ casi seguramente.

\begin{Def}
Un proceso puntual $N\left(t\right)$ es un proceso de renovaci\'on si los tiempos de interocurrencia $\xi_{n}=T_{n}-T_{n-1}$, para $n\geq1$, son independientes e identicamente distribuidos con distribuci\'on $F$, donde $F\left(0\right)=0$ y $T_{0}=0$. Los $T_{n}$ son llamados tiempos de renovaci\'on, referente a la independencia o renovaci\'on de la informaci\'on estoc\'astica en estos tiempos. Los $\xi_{n}$ son los tiempos de inter-renovaci\'on, y $N\left(t\right)$ es el n\'umero de renovaciones en el intervalo $\left[0,t\right)$
\end{Def}


\begin{Note}
Para definir un proceso de renovaci\'on para cualquier contexto, solamente hay que especificar una distribuci\'on $F$, con $F\left(0\right)=0$, para los tiempos de inter-renovaci\'on. La funci\'on $F$ en turno degune las otra variables aleatorias. De manera formal, existe un espacio de probabilidad y una sucesi\'on de variables aleatorias $\xi_{1},\xi_{2},\ldots$ definidas en este con distribuci\'on $F$. Entonces las otras cantidades son $T_{n}=\sum_{k=1}^{n}\xi_{k}$ y $N\left(t\right)=\sum_{n=1}^{\infty}\indora\left(T_{n}\leq t\right)$, donde $T_{n}\rightarrow\infty$ casi seguramente por la Ley Fuerte de los Grandes Números.
\end{Note}

%___________________________________________________________________________________________
%
%\subsection{Teorema Principal de Renovaci\'on}
%___________________________________________________________________________________________
%

\begin{Note} Una funci\'on $h:\rea_{+}\rightarrow\rea$ es Directamente Riemann Integrable en los siguientes casos:
\begin{itemize}
\item[a)] $h\left(t\right)\geq0$ es decreciente y Riemann Integrable.
\item[b)] $h$ es continua excepto posiblemente en un conjunto de Lebesgue de medida 0, y $|h\left(t\right)|\leq b\left(t\right)$, donde $b$ es DRI.
\end{itemize}
\end{Note}

\begin{Teo}[Teorema Principal de Renovaci\'on]
Si $F$ es no aritm\'etica y $h\left(t\right)$ es Directamente Riemann Integrable (DRI), entonces

\begin{eqnarray*}
lim_{t\rightarrow\infty}U\star h=\frac{1}{\mu}\int_{\rea_{+}}h\left(s\right)ds.
\end{eqnarray*}
\end{Teo}

\begin{Prop}
Cualquier funci\'on $H\left(t\right)$ acotada en intervalos finitos y que es 0 para $t<0$ puede expresarse como
\begin{eqnarray*}
H\left(t\right)=U\star h\left(t\right)\textrm{,  donde }h\left(t\right)=H\left(t\right)-F\star H\left(t\right)
\end{eqnarray*}
\end{Prop}

\begin{Def}
Un proceso estoc\'astico $X\left(t\right)$ es crudamente regenerativo en un tiempo aleatorio positivo $T$ si
\begin{eqnarray*}
\esp\left[X\left(T+t\right)|T\right]=\esp\left[X\left(t\right)\right]\textrm{, para }t\geq0,\end{eqnarray*}
y con las esperanzas anteriores finitas.
\end{Def}

\begin{Prop}
Sup\'ongase que $X\left(t\right)$ es un proceso crudamente regenerativo en $T$, que tiene distribuci\'on $F$. Si $\esp\left[X\left(t\right)\right]$ es acotado en intervalos finitos, entonces
\begin{eqnarray*}
\esp\left[X\left(t\right)\right]=U\star h\left(t\right)\textrm{,  donde }h\left(t\right)=\esp\left[X\left(t\right)\indora\left(T>t\right)\right].
\end{eqnarray*}
\end{Prop}

\begin{Teo}[Regeneraci\'on Cruda]
Sup\'ongase que $X\left(t\right)$ es un proceso con valores positivo crudamente regenerativo en $T$, y def\'inase $M=\sup\left\{|X\left(t\right)|:t\leq T\right\}$. Si $T$ es no aritm\'etico y $M$ y $MT$ tienen media finita, entonces
\begin{eqnarray*}
lim_{t\rightarrow\infty}\esp\left[X\left(t\right)\right]=\frac{1}{\mu}\int_{\rea_{+}}h\left(s\right)ds,
\end{eqnarray*}
donde $h\left(t\right)=\esp\left[X\left(t\right)\indora\left(T>t\right)\right]$.
\end{Teo}

%___________________________________________________________________________________________
%
%\subsection{Propiedades de los Procesos de Renovaci\'on}
%___________________________________________________________________________________________
%

Los tiempos $T_{n}$ est\'an relacionados con los conteos de $N\left(t\right)$ por

\begin{eqnarray*}
\left\{N\left(t\right)\geq n\right\}&=&\left\{T_{n}\leq t\right\}\\
T_{N\left(t\right)}\leq &t&<T_{N\left(t\right)+1},
\end{eqnarray*}

adem\'as $N\left(T_{n}\right)=n$, y 

\begin{eqnarray*}
N\left(t\right)=\max\left\{n:T_{n}\leq t\right\}=\min\left\{n:T_{n+1}>t\right\}
\end{eqnarray*}

Por propiedades de la convoluci\'on se sabe que

\begin{eqnarray*}
P\left\{T_{n}\leq t\right\}=F^{n\star}\left(t\right)
\end{eqnarray*}
que es la $n$-\'esima convoluci\'on de $F$. Entonces 

\begin{eqnarray*}
\left\{N\left(t\right)\geq n\right\}&=&\left\{T_{n}\leq t\right\}\\
P\left\{N\left(t\right)\leq n\right\}&=&1-F^{\left(n+1\right)\star}\left(t\right)
\end{eqnarray*}

Adem\'as usando el hecho de que $\esp\left[N\left(t\right)\right]=\sum_{n=1}^{\infty}P\left\{N\left(t\right)\geq n\right\}$
se tiene que

\begin{eqnarray*}
\esp\left[N\left(t\right)\right]=\sum_{n=1}^{\infty}F^{n\star}\left(t\right)
\end{eqnarray*}

\begin{Prop}
Para cada $t\geq0$, la funci\'on generadora de momentos $\esp\left[e^{\alpha N\left(t\right)}\right]$ existe para alguna $\alpha$ en una vecindad del 0, y de aqu\'i que $\esp\left[N\left(t\right)^{m}\right]<\infty$, para $m\geq1$.
\end{Prop}


\begin{Note}
Si el primer tiempo de renovaci\'on $\xi_{1}$ no tiene la misma distribuci\'on que el resto de las $\xi_{n}$, para $n\geq2$, a $N\left(t\right)$ se le llama Proceso de Renovaci\'on retardado, donde si $\xi$ tiene distribuci\'on $G$, entonces el tiempo $T_{n}$ de la $n$-\'esima renovaci\'on tiene distribuci\'on $G\star F^{\left(n-1\right)\star}\left(t\right)$
\end{Note}


\begin{Teo}
Para una constante $\mu\leq\infty$ ( o variable aleatoria), las siguientes expresiones son equivalentes:

\begin{eqnarray}
lim_{n\rightarrow\infty}n^{-1}T_{n}&=&\mu,\textrm{ c.s.}\\
lim_{t\rightarrow\infty}t^{-1}N\left(t\right)&=&1/\mu,\textrm{ c.s.}
\end{eqnarray}
\end{Teo}


Es decir, $T_{n}$ satisface la Ley Fuerte de los Grandes N\'umeros s\'i y s\'olo s\'i $N\left/t\right)$ la cumple.


\begin{Coro}[Ley Fuerte de los Grandes N\'umeros para Procesos de Renovaci\'on]
Si $N\left(t\right)$ es un proceso de renovaci\'on cuyos tiempos de inter-renovaci\'on tienen media $\mu\leq\infty$, entonces
\begin{eqnarray}
t^{-1}N\left(t\right)\rightarrow 1/\mu,\textrm{ c.s. cuando }t\rightarrow\infty.
\end{eqnarray}

\end{Coro}


Considerar el proceso estoc\'astico de valores reales $\left\{Z\left(t\right):t\geq0\right\}$ en el mismo espacio de probabilidad que $N\left(t\right)$

\begin{Def}
Para el proceso $\left\{Z\left(t\right):t\geq0\right\}$ se define la fluctuaci\'on m\'axima de $Z\left(t\right)$ en el intervalo $\left(T_{n-1},T_{n}\right]$:
\begin{eqnarray*}
M_{n}=\sup_{T_{n-1}<t\leq T_{n}}|Z\left(t\right)-Z\left(T_{n-1}\right)|
\end{eqnarray*}
\end{Def}

\begin{Teo}
Sup\'ongase que $n^{-1}T_{n}\rightarrow\mu$ c.s. cuando $n\rightarrow\infty$, donde $\mu\leq\infty$ es una constante o variable aleatoria. Sea $a$ una constante o variable aleatoria que puede ser infinita cuando $\mu$ es finita, y considere las expresiones l\'imite:
\begin{eqnarray}
lim_{n\rightarrow\infty}n^{-1}Z\left(T_{n}\right)&=&a,\textrm{ c.s.}\\
lim_{t\rightarrow\infty}t^{-1}Z\left(t\right)&=&a/\mu,\textrm{ c.s.}
\end{eqnarray}
La segunda expresi\'on implica la primera. Conversamente, la primera implica la segunda si el proceso $Z\left(t\right)$ es creciente, o si $lim_{n\rightarrow\infty}n^{-1}M_{n}=0$ c.s.
\end{Teo}

\begin{Coro}
Si $N\left(t\right)$ es un proceso de renovaci\'on, y $\left(Z\left(T_{n}\right)-Z\left(T_{n-1}\right),M_{n}\right)$, para $n\geq1$, son variables aleatorias independientes e id\'enticamente distribuidas con media finita, entonces,
\begin{eqnarray}
lim_{t\rightarrow\infty}t^{-1}Z\left(t\right)\rightarrow\frac{\esp\left[Z\left(T_{1}\right)-Z\left(T_{0}\right)\right]}{\esp\left[T_{1}\right]},\textrm{ c.s. cuando  }t\rightarrow\infty.
\end{eqnarray}
\end{Coro}



%___________________________________________________________________________________________
%
%\subsection{Propiedades de los Procesos de Renovaci\'on}
%___________________________________________________________________________________________
%

Los tiempos $T_{n}$ est\'an relacionados con los conteos de $N\left(t\right)$ por

\begin{eqnarray*}
\left\{N\left(t\right)\geq n\right\}&=&\left\{T_{n}\leq t\right\}\\
T_{N\left(t\right)}\leq &t&<T_{N\left(t\right)+1},
\end{eqnarray*}

adem\'as $N\left(T_{n}\right)=n$, y 

\begin{eqnarray*}
N\left(t\right)=\max\left\{n:T_{n}\leq t\right\}=\min\left\{n:T_{n+1}>t\right\}
\end{eqnarray*}

Por propiedades de la convoluci\'on se sabe que

\begin{eqnarray*}
P\left\{T_{n}\leq t\right\}=F^{n\star}\left(t\right)
\end{eqnarray*}
que es la $n$-\'esima convoluci\'on de $F$. Entonces 

\begin{eqnarray*}
\left\{N\left(t\right)\geq n\right\}&=&\left\{T_{n}\leq t\right\}\\
P\left\{N\left(t\right)\leq n\right\}&=&1-F^{\left(n+1\right)\star}\left(t\right)
\end{eqnarray*}

Adem\'as usando el hecho de que $\esp\left[N\left(t\right)\right]=\sum_{n=1}^{\infty}P\left\{N\left(t\right)\geq n\right\}$
se tiene que

\begin{eqnarray*}
\esp\left[N\left(t\right)\right]=\sum_{n=1}^{\infty}F^{n\star}\left(t\right)
\end{eqnarray*}

\begin{Prop}
Para cada $t\geq0$, la funci\'on generadora de momentos $\esp\left[e^{\alpha N\left(t\right)}\right]$ existe para alguna $\alpha$ en una vecindad del 0, y de aqu\'i que $\esp\left[N\left(t\right)^{m}\right]<\infty$, para $m\geq1$.
\end{Prop}


\begin{Note}
Si el primer tiempo de renovaci\'on $\xi_{1}$ no tiene la misma distribuci\'on que el resto de las $\xi_{n}$, para $n\geq2$, a $N\left(t\right)$ se le llama Proceso de Renovaci\'on retardado, donde si $\xi$ tiene distribuci\'on $G$, entonces el tiempo $T_{n}$ de la $n$-\'esima renovaci\'on tiene distribuci\'on $G\star F^{\left(n-1\right)\star}\left(t\right)$
\end{Note}


\begin{Teo}
Para una constante $\mu\leq\infty$ ( o variable aleatoria), las siguientes expresiones son equivalentes:

\begin{eqnarray}
lim_{n\rightarrow\infty}n^{-1}T_{n}&=&\mu,\textrm{ c.s.}\\
lim_{t\rightarrow\infty}t^{-1}N\left(t\right)&=&1/\mu,\textrm{ c.s.}
\end{eqnarray}
\end{Teo}


Es decir, $T_{n}$ satisface la Ley Fuerte de los Grandes N\'umeros s\'i y s\'olo s\'i $N\left/t\right)$ la cumple.


\begin{Coro}[Ley Fuerte de los Grandes N\'umeros para Procesos de Renovaci\'on]
Si $N\left(t\right)$ es un proceso de renovaci\'on cuyos tiempos de inter-renovaci\'on tienen media $\mu\leq\infty$, entonces
\begin{eqnarray}
t^{-1}N\left(t\right)\rightarrow 1/\mu,\textrm{ c.s. cuando }t\rightarrow\infty.
\end{eqnarray}

\end{Coro}


Considerar el proceso estoc\'astico de valores reales $\left\{Z\left(t\right):t\geq0\right\}$ en el mismo espacio de probabilidad que $N\left(t\right)$

\begin{Def}
Para el proceso $\left\{Z\left(t\right):t\geq0\right\}$ se define la fluctuaci\'on m\'axima de $Z\left(t\right)$ en el intervalo $\left(T_{n-1},T_{n}\right]$:
\begin{eqnarray*}
M_{n}=\sup_{T_{n-1}<t\leq T_{n}}|Z\left(t\right)-Z\left(T_{n-1}\right)|
\end{eqnarray*}
\end{Def}

\begin{Teo}
Sup\'ongase que $n^{-1}T_{n}\rightarrow\mu$ c.s. cuando $n\rightarrow\infty$, donde $\mu\leq\infty$ es una constante o variable aleatoria. Sea $a$ una constante o variable aleatoria que puede ser infinita cuando $\mu$ es finita, y considere las expresiones l\'imite:
\begin{eqnarray}
lim_{n\rightarrow\infty}n^{-1}Z\left(T_{n}\right)&=&a,\textrm{ c.s.}\\
lim_{t\rightarrow\infty}t^{-1}Z\left(t\right)&=&a/\mu,\textrm{ c.s.}
\end{eqnarray}
La segunda expresi\'on implica la primera. Conversamente, la primera implica la segunda si el proceso $Z\left(t\right)$ es creciente, o si $lim_{n\rightarrow\infty}n^{-1}M_{n}=0$ c.s.
\end{Teo}

\begin{Coro}
Si $N\left(t\right)$ es un proceso de renovaci\'on, y $\left(Z\left(T_{n}\right)-Z\left(T_{n-1}\right),M_{n}\right)$, para $n\geq1$, son variables aleatorias independientes e id\'enticamente distribuidas con media finita, entonces,
\begin{eqnarray}
lim_{t\rightarrow\infty}t^{-1}Z\left(t\right)\rightarrow\frac{\esp\left[Z\left(T_{1}\right)-Z\left(T_{0}\right)\right]}{\esp\left[T_{1}\right]},\textrm{ c.s. cuando  }t\rightarrow\infty.
\end{eqnarray}
\end{Coro}


%___________________________________________________________________________________________
%
%\subsection{Propiedades de los Procesos de Renovaci\'on}
%___________________________________________________________________________________________
%

Los tiempos $T_{n}$ est\'an relacionados con los conteos de $N\left(t\right)$ por

\begin{eqnarray*}
\left\{N\left(t\right)\geq n\right\}&=&\left\{T_{n}\leq t\right\}\\
T_{N\left(t\right)}\leq &t&<T_{N\left(t\right)+1},
\end{eqnarray*}

adem\'as $N\left(T_{n}\right)=n$, y 

\begin{eqnarray*}
N\left(t\right)=\max\left\{n:T_{n}\leq t\right\}=\min\left\{n:T_{n+1}>t\right\}
\end{eqnarray*}

Por propiedades de la convoluci\'on se sabe que

\begin{eqnarray*}
P\left\{T_{n}\leq t\right\}=F^{n\star}\left(t\right)
\end{eqnarray*}
que es la $n$-\'esima convoluci\'on de $F$. Entonces 

\begin{eqnarray*}
\left\{N\left(t\right)\geq n\right\}&=&\left\{T_{n}\leq t\right\}\\
P\left\{N\left(t\right)\leq n\right\}&=&1-F^{\left(n+1\right)\star}\left(t\right)
\end{eqnarray*}

Adem\'as usando el hecho de que $\esp\left[N\left(t\right)\right]=\sum_{n=1}^{\infty}P\left\{N\left(t\right)\geq n\right\}$
se tiene que

\begin{eqnarray*}
\esp\left[N\left(t\right)\right]=\sum_{n=1}^{\infty}F^{n\star}\left(t\right)
\end{eqnarray*}

\begin{Prop}
Para cada $t\geq0$, la funci\'on generadora de momentos $\esp\left[e^{\alpha N\left(t\right)}\right]$ existe para alguna $\alpha$ en una vecindad del 0, y de aqu\'i que $\esp\left[N\left(t\right)^{m}\right]<\infty$, para $m\geq1$.
\end{Prop}


\begin{Note}
Si el primer tiempo de renovaci\'on $\xi_{1}$ no tiene la misma distribuci\'on que el resto de las $\xi_{n}$, para $n\geq2$, a $N\left(t\right)$ se le llama Proceso de Renovaci\'on retardado, donde si $\xi$ tiene distribuci\'on $G$, entonces el tiempo $T_{n}$ de la $n$-\'esima renovaci\'on tiene distribuci\'on $G\star F^{\left(n-1\right)\star}\left(t\right)$
\end{Note}


\begin{Teo}
Para una constante $\mu\leq\infty$ ( o variable aleatoria), las siguientes expresiones son equivalentes:

\begin{eqnarray}
lim_{n\rightarrow\infty}n^{-1}T_{n}&=&\mu,\textrm{ c.s.}\\
lim_{t\rightarrow\infty}t^{-1}N\left(t\right)&=&1/\mu,\textrm{ c.s.}
\end{eqnarray}
\end{Teo}


Es decir, $T_{n}$ satisface la Ley Fuerte de los Grandes N\'umeros s\'i y s\'olo s\'i $N\left/t\right)$ la cumple.


\begin{Coro}[Ley Fuerte de los Grandes N\'umeros para Procesos de Renovaci\'on]
Si $N\left(t\right)$ es un proceso de renovaci\'on cuyos tiempos de inter-renovaci\'on tienen media $\mu\leq\infty$, entonces
\begin{eqnarray}
t^{-1}N\left(t\right)\rightarrow 1/\mu,\textrm{ c.s. cuando }t\rightarrow\infty.
\end{eqnarray}

\end{Coro}


Considerar el proceso estoc\'astico de valores reales $\left\{Z\left(t\right):t\geq0\right\}$ en el mismo espacio de probabilidad que $N\left(t\right)$

\begin{Def}
Para el proceso $\left\{Z\left(t\right):t\geq0\right\}$ se define la fluctuaci\'on m\'axima de $Z\left(t\right)$ en el intervalo $\left(T_{n-1},T_{n}\right]$:
\begin{eqnarray*}
M_{n}=\sup_{T_{n-1}<t\leq T_{n}}|Z\left(t\right)-Z\left(T_{n-1}\right)|
\end{eqnarray*}
\end{Def}

\begin{Teo}
Sup\'ongase que $n^{-1}T_{n}\rightarrow\mu$ c.s. cuando $n\rightarrow\infty$, donde $\mu\leq\infty$ es una constante o variable aleatoria. Sea $a$ una constante o variable aleatoria que puede ser infinita cuando $\mu$ es finita, y considere las expresiones l\'imite:
\begin{eqnarray}
lim_{n\rightarrow\infty}n^{-1}Z\left(T_{n}\right)&=&a,\textrm{ c.s.}\\
lim_{t\rightarrow\infty}t^{-1}Z\left(t\right)&=&a/\mu,\textrm{ c.s.}
\end{eqnarray}
La segunda expresi\'on implica la primera. Conversamente, la primera implica la segunda si el proceso $Z\left(t\right)$ es creciente, o si $lim_{n\rightarrow\infty}n^{-1}M_{n}=0$ c.s.
\end{Teo}

\begin{Coro}
Si $N\left(t\right)$ es un proceso de renovaci\'on, y $\left(Z\left(T_{n}\right)-Z\left(T_{n-1}\right),M_{n}\right)$, para $n\geq1$, son variables aleatorias independientes e id\'enticamente distribuidas con media finita, entonces,
\begin{eqnarray}
lim_{t\rightarrow\infty}t^{-1}Z\left(t\right)\rightarrow\frac{\esp\left[Z\left(T_{1}\right)-Z\left(T_{0}\right)\right]}{\esp\left[T_{1}\right]},\textrm{ c.s. cuando  }t\rightarrow\infty.
\end{eqnarray}
\end{Coro}

%___________________________________________________________________________________________
%
%\subsection{Propiedades de los Procesos de Renovaci\'on}
%___________________________________________________________________________________________
%

Los tiempos $T_{n}$ est\'an relacionados con los conteos de $N\left(t\right)$ por

\begin{eqnarray*}
\left\{N\left(t\right)\geq n\right\}&=&\left\{T_{n}\leq t\right\}\\
T_{N\left(t\right)}\leq &t&<T_{N\left(t\right)+1},
\end{eqnarray*}

adem\'as $N\left(T_{n}\right)=n$, y 

\begin{eqnarray*}
N\left(t\right)=\max\left\{n:T_{n}\leq t\right\}=\min\left\{n:T_{n+1}>t\right\}
\end{eqnarray*}

Por propiedades de la convoluci\'on se sabe que

\begin{eqnarray*}
P\left\{T_{n}\leq t\right\}=F^{n\star}\left(t\right)
\end{eqnarray*}
que es la $n$-\'esima convoluci\'on de $F$. Entonces 

\begin{eqnarray*}
\left\{N\left(t\right)\geq n\right\}&=&\left\{T_{n}\leq t\right\}\\
P\left\{N\left(t\right)\leq n\right\}&=&1-F^{\left(n+1\right)\star}\left(t\right)
\end{eqnarray*}

Adem\'as usando el hecho de que $\esp\left[N\left(t\right)\right]=\sum_{n=1}^{\infty}P\left\{N\left(t\right)\geq n\right\}$
se tiene que

\begin{eqnarray*}
\esp\left[N\left(t\right)\right]=\sum_{n=1}^{\infty}F^{n\star}\left(t\right)
\end{eqnarray*}

\begin{Prop}
Para cada $t\geq0$, la funci\'on generadora de momentos $\esp\left[e^{\alpha N\left(t\right)}\right]$ existe para alguna $\alpha$ en una vecindad del 0, y de aqu\'i que $\esp\left[N\left(t\right)^{m}\right]<\infty$, para $m\geq1$.
\end{Prop}


\begin{Note}
Si el primer tiempo de renovaci\'on $\xi_{1}$ no tiene la misma distribuci\'on que el resto de las $\xi_{n}$, para $n\geq2$, a $N\left(t\right)$ se le llama Proceso de Renovaci\'on retardado, donde si $\xi$ tiene distribuci\'on $G$, entonces el tiempo $T_{n}$ de la $n$-\'esima renovaci\'on tiene distribuci\'on $G\star F^{\left(n-1\right)\star}\left(t\right)$
\end{Note}


\begin{Teo}
Para una constante $\mu\leq\infty$ ( o variable aleatoria), las siguientes expresiones son equivalentes:

\begin{eqnarray}
lim_{n\rightarrow\infty}n^{-1}T_{n}&=&\mu,\textrm{ c.s.}\\
lim_{t\rightarrow\infty}t^{-1}N\left(t\right)&=&1/\mu,\textrm{ c.s.}
\end{eqnarray}
\end{Teo}


Es decir, $T_{n}$ satisface la Ley Fuerte de los Grandes N\'umeros s\'i y s\'olo s\'i $N\left/t\right)$ la cumple.


\begin{Coro}[Ley Fuerte de los Grandes N\'umeros para Procesos de Renovaci\'on]
Si $N\left(t\right)$ es un proceso de renovaci\'on cuyos tiempos de inter-renovaci\'on tienen media $\mu\leq\infty$, entonces
\begin{eqnarray}
t^{-1}N\left(t\right)\rightarrow 1/\mu,\textrm{ c.s. cuando }t\rightarrow\infty.
\end{eqnarray}

\end{Coro}


Considerar el proceso estoc\'astico de valores reales $\left\{Z\left(t\right):t\geq0\right\}$ en el mismo espacio de probabilidad que $N\left(t\right)$

\begin{Def}
Para el proceso $\left\{Z\left(t\right):t\geq0\right\}$ se define la fluctuaci\'on m\'axima de $Z\left(t\right)$ en el intervalo $\left(T_{n-1},T_{n}\right]$:
\begin{eqnarray*}
M_{n}=\sup_{T_{n-1}<t\leq T_{n}}|Z\left(t\right)-Z\left(T_{n-1}\right)|
\end{eqnarray*}
\end{Def}

\begin{Teo}
Sup\'ongase que $n^{-1}T_{n}\rightarrow\mu$ c.s. cuando $n\rightarrow\infty$, donde $\mu\leq\infty$ es una constante o variable aleatoria. Sea $a$ una constante o variable aleatoria que puede ser infinita cuando $\mu$ es finita, y considere las expresiones l\'imite:
\begin{eqnarray}
lim_{n\rightarrow\infty}n^{-1}Z\left(T_{n}\right)&=&a,\textrm{ c.s.}\\
lim_{t\rightarrow\infty}t^{-1}Z\left(t\right)&=&a/\mu,\textrm{ c.s.}
\end{eqnarray}
La segunda expresi\'on implica la primera. Conversamente, la primera implica la segunda si el proceso $Z\left(t\right)$ es creciente, o si $lim_{n\rightarrow\infty}n^{-1}M_{n}=0$ c.s.
\end{Teo}

\begin{Coro}
Si $N\left(t\right)$ es un proceso de renovaci\'on, y $\left(Z\left(T_{n}\right)-Z\left(T_{n-1}\right),M_{n}\right)$, para $n\geq1$, son variables aleatorias independientes e id\'enticamente distribuidas con media finita, entonces,
\begin{eqnarray}
lim_{t\rightarrow\infty}t^{-1}Z\left(t\right)\rightarrow\frac{\esp\left[Z\left(T_{1}\right)-Z\left(T_{0}\right)\right]}{\esp\left[T_{1}\right]},\textrm{ c.s. cuando  }t\rightarrow\infty.
\end{eqnarray}
\end{Coro}
%___________________________________________________________________________________________
%
%\subsection{Propiedades de los Procesos de Renovaci\'on}
%___________________________________________________________________________________________
%

Los tiempos $T_{n}$ est\'an relacionados con los conteos de $N\left(t\right)$ por

\begin{eqnarray*}
\left\{N\left(t\right)\geq n\right\}&=&\left\{T_{n}\leq t\right\}\\
T_{N\left(t\right)}\leq &t&<T_{N\left(t\right)+1},
\end{eqnarray*}

adem\'as $N\left(T_{n}\right)=n$, y 

\begin{eqnarray*}
N\left(t\right)=\max\left\{n:T_{n}\leq t\right\}=\min\left\{n:T_{n+1}>t\right\}
\end{eqnarray*}

Por propiedades de la convoluci\'on se sabe que

\begin{eqnarray*}
P\left\{T_{n}\leq t\right\}=F^{n\star}\left(t\right)
\end{eqnarray*}
que es la $n$-\'esima convoluci\'on de $F$. Entonces 

\begin{eqnarray*}
\left\{N\left(t\right)\geq n\right\}&=&\left\{T_{n}\leq t\right\}\\
P\left\{N\left(t\right)\leq n\right\}&=&1-F^{\left(n+1\right)\star}\left(t\right)
\end{eqnarray*}

Adem\'as usando el hecho de que $\esp\left[N\left(t\right)\right]=\sum_{n=1}^{\infty}P\left\{N\left(t\right)\geq n\right\}$
se tiene que

\begin{eqnarray*}
\esp\left[N\left(t\right)\right]=\sum_{n=1}^{\infty}F^{n\star}\left(t\right)
\end{eqnarray*}

\begin{Prop}
Para cada $t\geq0$, la funci\'on generadora de momentos $\esp\left[e^{\alpha N\left(t\right)}\right]$ existe para alguna $\alpha$ en una vecindad del 0, y de aqu\'i que $\esp\left[N\left(t\right)^{m}\right]<\infty$, para $m\geq1$.
\end{Prop}


\begin{Note}
Si el primer tiempo de renovaci\'on $\xi_{1}$ no tiene la misma distribuci\'on que el resto de las $\xi_{n}$, para $n\geq2$, a $N\left(t\right)$ se le llama Proceso de Renovaci\'on retardado, donde si $\xi$ tiene distribuci\'on $G$, entonces el tiempo $T_{n}$ de la $n$-\'esima renovaci\'on tiene distribuci\'on $G\star F^{\left(n-1\right)\star}\left(t\right)$
\end{Note}


\begin{Teo}
Para una constante $\mu\leq\infty$ ( o variable aleatoria), las siguientes expresiones son equivalentes:

\begin{eqnarray}
lim_{n\rightarrow\infty}n^{-1}T_{n}&=&\mu,\textrm{ c.s.}\\
lim_{t\rightarrow\infty}t^{-1}N\left(t\right)&=&1/\mu,\textrm{ c.s.}
\end{eqnarray}
\end{Teo}


Es decir, $T_{n}$ satisface la Ley Fuerte de los Grandes N\'umeros s\'i y s\'olo s\'i $N\left/t\right)$ la cumple.


\begin{Coro}[Ley Fuerte de los Grandes N\'umeros para Procesos de Renovaci\'on]
Si $N\left(t\right)$ es un proceso de renovaci\'on cuyos tiempos de inter-renovaci\'on tienen media $\mu\leq\infty$, entonces
\begin{eqnarray}
t^{-1}N\left(t\right)\rightarrow 1/\mu,\textrm{ c.s. cuando }t\rightarrow\infty.
\end{eqnarray}

\end{Coro}


Considerar el proceso estoc\'astico de valores reales $\left\{Z\left(t\right):t\geq0\right\}$ en el mismo espacio de probabilidad que $N\left(t\right)$

\begin{Def}
Para el proceso $\left\{Z\left(t\right):t\geq0\right\}$ se define la fluctuaci\'on m\'axima de $Z\left(t\right)$ en el intervalo $\left(T_{n-1},T_{n}\right]$:
\begin{eqnarray*}
M_{n}=\sup_{T_{n-1}<t\leq T_{n}}|Z\left(t\right)-Z\left(T_{n-1}\right)|
\end{eqnarray*}
\end{Def}

\begin{Teo}
Sup\'ongase que $n^{-1}T_{n}\rightarrow\mu$ c.s. cuando $n\rightarrow\infty$, donde $\mu\leq\infty$ es una constante o variable aleatoria. Sea $a$ una constante o variable aleatoria que puede ser infinita cuando $\mu$ es finita, y considere las expresiones l\'imite:
\begin{eqnarray}
lim_{n\rightarrow\infty}n^{-1}Z\left(T_{n}\right)&=&a,\textrm{ c.s.}\\
lim_{t\rightarrow\infty}t^{-1}Z\left(t\right)&=&a/\mu,\textrm{ c.s.}
\end{eqnarray}
La segunda expresi\'on implica la primera. Conversamente, la primera implica la segunda si el proceso $Z\left(t\right)$ es creciente, o si $lim_{n\rightarrow\infty}n^{-1}M_{n}=0$ c.s.
\end{Teo}

\begin{Coro}
Si $N\left(t\right)$ es un proceso de renovaci\'on, y $\left(Z\left(T_{n}\right)-Z\left(T_{n-1}\right),M_{n}\right)$, para $n\geq1$, son variables aleatorias independientes e id\'enticamente distribuidas con media finita, entonces,
\begin{eqnarray}
lim_{t\rightarrow\infty}t^{-1}Z\left(t\right)\rightarrow\frac{\esp\left[Z\left(T_{1}\right)-Z\left(T_{0}\right)\right]}{\esp\left[T_{1}\right]},\textrm{ c.s. cuando  }t\rightarrow\infty.
\end{eqnarray}
\end{Coro}


%___________________________________________________________________________________________
%
%\subsection{Funci\'on de Renovaci\'on}
%___________________________________________________________________________________________
%


\begin{Def}
Sea $h\left(t\right)$ funci\'on de valores reales en $\rea$ acotada en intervalos finitos e igual a cero para $t<0$ La ecuaci\'on de renovaci\'on para $h\left(t\right)$ y la distribuci\'on $F$ es

\begin{eqnarray}%\label{Ec.Renovacion}
H\left(t\right)=h\left(t\right)+\int_{\left[0,t\right]}H\left(t-s\right)dF\left(s\right)\textrm{,    }t\geq0,
\end{eqnarray}
donde $H\left(t\right)$ es una funci\'on de valores reales. Esto es $H=h+F\star H$. Decimos que $H\left(t\right)$ es soluci\'on de esta ecuaci\'on si satisface la ecuaci\'on, y es acotada en intervalos finitos e iguales a cero para $t<0$.
\end{Def}

\begin{Prop}
La funci\'on $U\star h\left(t\right)$ es la \'unica soluci\'on de la ecuaci\'on de renovaci\'on (\ref{Ec.Renovacion}).
\end{Prop}

\begin{Teo}[Teorema Renovaci\'on Elemental]
\begin{eqnarray*}
t^{-1}U\left(t\right)\rightarrow 1/\mu\textrm{,    cuando }t\rightarrow\infty.
\end{eqnarray*}
\end{Teo}

%___________________________________________________________________________________________
%
%\subsection{Funci\'on de Renovaci\'on}
%___________________________________________________________________________________________
%


Sup\'ongase que $N\left(t\right)$ es un proceso de renovaci\'on con distribuci\'on $F$ con media finita $\mu$.

\begin{Def}
La funci\'on de renovaci\'on asociada con la distribuci\'on $F$, del proceso $N\left(t\right)$, es
\begin{eqnarray*}
U\left(t\right)=\sum_{n=1}^{\infty}F^{n\star}\left(t\right),\textrm{   }t\geq0,
\end{eqnarray*}
donde $F^{0\star}\left(t\right)=\indora\left(t\geq0\right)$.
\end{Def}


\begin{Prop}
Sup\'ongase que la distribuci\'on de inter-renovaci\'on $F$ tiene densidad $f$. Entonces $U\left(t\right)$ tambi\'en tiene densidad, para $t>0$, y es $U^{'}\left(t\right)=\sum_{n=0}^{\infty}f^{n\star}\left(t\right)$. Adem\'as
\begin{eqnarray*}
\prob\left\{N\left(t\right)>N\left(t-\right)\right\}=0\textrm{,   }t\geq0.
\end{eqnarray*}
\end{Prop}

\begin{Def}
La Transformada de Laplace-Stieljes de $F$ est\'a dada por

\begin{eqnarray*}
\hat{F}\left(\alpha\right)=\int_{\rea_{+}}e^{-\alpha t}dF\left(t\right)\textrm{,  }\alpha\geq0.
\end{eqnarray*}
\end{Def}

Entonces

\begin{eqnarray*}
\hat{U}\left(\alpha\right)=\sum_{n=0}^{\infty}\hat{F^{n\star}}\left(\alpha\right)=\sum_{n=0}^{\infty}\hat{F}\left(\alpha\right)^{n}=\frac{1}{1-\hat{F}\left(\alpha\right)}.
\end{eqnarray*}


\begin{Prop}
La Transformada de Laplace $\hat{U}\left(\alpha\right)$ y $\hat{F}\left(\alpha\right)$ determina una a la otra de manera \'unica por la relaci\'on $\hat{U}\left(\alpha\right)=\frac{1}{1-\hat{F}\left(\alpha\right)}$.
\end{Prop}


\begin{Note}
Un proceso de renovaci\'on $N\left(t\right)$ cuyos tiempos de inter-renovaci\'on tienen media finita, es un proceso Poisson con tasa $\lambda$ si y s\'olo s\'i $\esp\left[U\left(t\right)\right]=\lambda t$, para $t\geq0$.
\end{Note}


\begin{Teo}
Sea $N\left(t\right)$ un proceso puntual simple con puntos de localizaci\'on $T_{n}$ tal que $\eta\left(t\right)=\esp\left[N\left(\right)\right]$ es finita para cada $t$. Entonces para cualquier funci\'on $f:\rea_{+}\rightarrow\rea$,
\begin{eqnarray*}
\esp\left[\sum_{n=1}^{N\left(\right)}f\left(T_{n}\right)\right]=\int_{\left(0,t\right]}f\left(s\right)d\eta\left(s\right)\textrm{,  }t\geq0,
\end{eqnarray*}
suponiendo que la integral exista. Adem\'as si $X_{1},X_{2},\ldots$ son variables aleatorias definidas en el mismo espacio de probabilidad que el proceso $N\left(t\right)$ tal que $\esp\left[X_{n}|T_{n}=s\right]=f\left(s\right)$, independiente de $n$. Entonces
\begin{eqnarray*}
\esp\left[\sum_{n=1}^{N\left(t\right)}X_{n}\right]=\int_{\left(0,t\right]}f\left(s\right)d\eta\left(s\right)\textrm{,  }t\geq0,
\end{eqnarray*} 
suponiendo que la integral exista. 
\end{Teo}

\begin{Coro}[Identidad de Wald para Renovaciones]
Para el proceso de renovaci\'on $N\left(t\right)$,
\begin{eqnarray*}
\esp\left[T_{N\left(t\right)+1}\right]=\mu\esp\left[N\left(t\right)+1\right]\textrm{,  }t\geq0,
\end{eqnarray*}  
\end{Coro}

%______________________________________________________________________
%\subsection{Procesos de Renovaci\'on}
%______________________________________________________________________

\begin{Def}%\label{Def.Tn}
Sean $0\leq T_{1}\leq T_{2}\leq \ldots$ son tiempos aleatorios infinitos en los cuales ocurren ciertos eventos. El n\'umero de tiempos $T_{n}$ en el intervalo $\left[0,t\right)$ es

\begin{eqnarray}
N\left(t\right)=\sum_{n=1}^{\infty}\indora\left(T_{n}\leq t\right),
\end{eqnarray}
para $t\geq0$.
\end{Def}

Si se consideran los puntos $T_{n}$ como elementos de $\rea_{+}$, y $N\left(t\right)$ es el n\'umero de puntos en $\rea$. El proceso denotado por $\left\{N\left(t\right):t\geq0\right\}$, denotado por $N\left(t\right)$, es un proceso puntual en $\rea_{+}$. Los $T_{n}$ son los tiempos de ocurrencia, el proceso puntual $N\left(t\right)$ es simple si su n\'umero de ocurrencias son distintas: $0<T_{1}<T_{2}<\ldots$ casi seguramente.

\begin{Def}
Un proceso puntual $N\left(t\right)$ es un proceso de renovaci\'on si los tiempos de interocurrencia $\xi_{n}=T_{n}-T_{n-1}$, para $n\geq1$, son independientes e identicamente distribuidos con distribuci\'on $F$, donde $F\left(0\right)=0$ y $T_{0}=0$. Los $T_{n}$ son llamados tiempos de renovaci\'on, referente a la independencia o renovaci\'on de la informaci\'on estoc\'astica en estos tiempos. Los $\xi_{n}$ son los tiempos de inter-renovaci\'on, y $N\left(t\right)$ es el n\'umero de renovaciones en el intervalo $\left[0,t\right)$
\end{Def}


\begin{Note}
Para definir un proceso de renovaci\'on para cualquier contexto, solamente hay que especificar una distribuci\'on $F$, con $F\left(0\right)=0$, para los tiempos de inter-renovaci\'on. La funci\'on $F$ en turno degune las otra variables aleatorias. De manera formal, existe un espacio de probabilidad y una sucesi\'on de variables aleatorias $\xi_{1},\xi_{2},\ldots$ definidas en este con distribuci\'on $F$. Entonces las otras cantidades son $T_{n}=\sum_{k=1}^{n}\xi_{k}$ y $N\left(t\right)=\sum_{n=1}^{\infty}\indora\left(T_{n}\leq t\right)$, donde $T_{n}\rightarrow\infty$ casi seguramente por la Ley Fuerte de los Grandes Números.
\end{Note}

%___________________________________________________________________________________________
%
%\subsection{Renewal and Regenerative Processes: Serfozo\cite{Serfozo}}
%___________________________________________________________________________________________
%
\begin{Def}%\label{Def.Tn}
Sean $0\leq T_{1}\leq T_{2}\leq \ldots$ son tiempos aleatorios infinitos en los cuales ocurren ciertos eventos. El n\'umero de tiempos $T_{n}$ en el intervalo $\left[0,t\right)$ es

\begin{eqnarray}
N\left(t\right)=\sum_{n=1}^{\infty}\indora\left(T_{n}\leq t\right),
\end{eqnarray}
para $t\geq0$.
\end{Def}

Si se consideran los puntos $T_{n}$ como elementos de $\rea_{+}$, y $N\left(t\right)$ es el n\'umero de puntos en $\rea$. El proceso denotado por $\left\{N\left(t\right):t\geq0\right\}$, denotado por $N\left(t\right)$, es un proceso puntual en $\rea_{+}$. Los $T_{n}$ son los tiempos de ocurrencia, el proceso puntual $N\left(t\right)$ es simple si su n\'umero de ocurrencias son distintas: $0<T_{1}<T_{2}<\ldots$ casi seguramente.

\begin{Def}
Un proceso puntual $N\left(t\right)$ es un proceso de renovaci\'on si los tiempos de interocurrencia $\xi_{n}=T_{n}-T_{n-1}$, para $n\geq1$, son independientes e identicamente distribuidos con distribuci\'on $F$, donde $F\left(0\right)=0$ y $T_{0}=0$. Los $T_{n}$ son llamados tiempos de renovaci\'on, referente a la independencia o renovaci\'on de la informaci\'on estoc\'astica en estos tiempos. Los $\xi_{n}$ son los tiempos de inter-renovaci\'on, y $N\left(t\right)$ es el n\'umero de renovaciones en el intervalo $\left[0,t\right)$
\end{Def}


\begin{Note}
Para definir un proceso de renovaci\'on para cualquier contexto, solamente hay que especificar una distribuci\'on $F$, con $F\left(0\right)=0$, para los tiempos de inter-renovaci\'on. La funci\'on $F$ en turno degune las otra variables aleatorias. De manera formal, existe un espacio de probabilidad y una sucesi\'on de variables aleatorias $\xi_{1},\xi_{2},\ldots$ definidas en este con distribuci\'on $F$. Entonces las otras cantidades son $T_{n}=\sum_{k=1}^{n}\xi_{k}$ y $N\left(t\right)=\sum_{n=1}^{\infty}\indora\left(T_{n}\leq t\right)$, donde $T_{n}\rightarrow\infty$ casi seguramente por la Ley Fuerte de los Grandes N\'umeros.
\end{Note}







Los tiempos $T_{n}$ est\'an relacionados con los conteos de $N\left(t\right)$ por

\begin{eqnarray*}
\left\{N\left(t\right)\geq n\right\}&=&\left\{T_{n}\leq t\right\}\\
T_{N\left(t\right)}\leq &t&<T_{N\left(t\right)+1},
\end{eqnarray*}

adem\'as $N\left(T_{n}\right)=n$, y 

\begin{eqnarray*}
N\left(t\right)=\max\left\{n:T_{n}\leq t\right\}=\min\left\{n:T_{n+1}>t\right\}
\end{eqnarray*}

Por propiedades de la convoluci\'on se sabe que

\begin{eqnarray*}
P\left\{T_{n}\leq t\right\}=F^{n\star}\left(t\right)
\end{eqnarray*}
que es la $n$-\'esima convoluci\'on de $F$. Entonces 

\begin{eqnarray*}
\left\{N\left(t\right)\geq n\right\}&=&\left\{T_{n}\leq t\right\}\\
P\left\{N\left(t\right)\leq n\right\}&=&1-F^{\left(n+1\right)\star}\left(t\right)
\end{eqnarray*}

Adem\'as usando el hecho de que $\esp\left[N\left(t\right)\right]=\sum_{n=1}^{\infty}P\left\{N\left(t\right)\geq n\right\}$
se tiene que

\begin{eqnarray*}
\esp\left[N\left(t\right)\right]=\sum_{n=1}^{\infty}F^{n\star}\left(t\right)
\end{eqnarray*}

\begin{Prop}
Para cada $t\geq0$, la funci\'on generadora de momentos $\esp\left[e^{\alpha N\left(t\right)}\right]$ existe para alguna $\alpha$ en una vecindad del 0, y de aqu\'i que $\esp\left[N\left(t\right)^{m}\right]<\infty$, para $m\geq1$.
\end{Prop}

\begin{Ejem}[\textbf{Proceso Poisson}]

Suponga que se tienen tiempos de inter-renovaci\'on \textit{i.i.d.} del proceso de renovaci\'on $N\left(t\right)$ tienen distribuci\'on exponencial $F\left(t\right)=q-e^{-\lambda t}$ con tasa $\lambda$. Entonces $N\left(t\right)$ es un proceso Poisson con tasa $\lambda$.

\end{Ejem}


\begin{Note}
Si el primer tiempo de renovaci\'on $\xi_{1}$ no tiene la misma distribuci\'on que el resto de las $\xi_{n}$, para $n\geq2$, a $N\left(t\right)$ se le llama Proceso de Renovaci\'on retardado, donde si $\xi$ tiene distribuci\'on $G$, entonces el tiempo $T_{n}$ de la $n$-\'esima renovaci\'on tiene distribuci\'on $G\star F^{\left(n-1\right)\star}\left(t\right)$
\end{Note}


\begin{Teo}
Para una constante $\mu\leq\infty$ ( o variable aleatoria), las siguientes expresiones son equivalentes:

\begin{eqnarray}
lim_{n\rightarrow\infty}n^{-1}T_{n}&=&\mu,\textrm{ c.s.}\\
lim_{t\rightarrow\infty}t^{-1}N\left(t\right)&=&1/\mu,\textrm{ c.s.}
\end{eqnarray}
\end{Teo}


Es decir, $T_{n}$ satisface la Ley Fuerte de los Grandes N\'umeros s\'i y s\'olo s\'i $N\left/t\right)$ la cumple.


\begin{Coro}[Ley Fuerte de los Grandes N\'umeros para Procesos de Renovaci\'on]
Si $N\left(t\right)$ es un proceso de renovaci\'on cuyos tiempos de inter-renovaci\'on tienen media $\mu\leq\infty$, entonces
\begin{eqnarray}
t^{-1}N\left(t\right)\rightarrow 1/\mu,\textrm{ c.s. cuando }t\rightarrow\infty.
\end{eqnarray}

\end{Coro}


Considerar el proceso estoc\'astico de valores reales $\left\{Z\left(t\right):t\geq0\right\}$ en el mismo espacio de probabilidad que $N\left(t\right)$

\begin{Def}
Para el proceso $\left\{Z\left(t\right):t\geq0\right\}$ se define la fluctuaci\'on m\'axima de $Z\left(t\right)$ en el intervalo $\left(T_{n-1},T_{n}\right]$:
\begin{eqnarray*}
M_{n}=\sup_{T_{n-1}<t\leq T_{n}}|Z\left(t\right)-Z\left(T_{n-1}\right)|
\end{eqnarray*}
\end{Def}

\begin{Teo}
Sup\'ongase que $n^{-1}T_{n}\rightarrow\mu$ c.s. cuando $n\rightarrow\infty$, donde $\mu\leq\infty$ es una constante o variable aleatoria. Sea $a$ una constante o variable aleatoria que puede ser infinita cuando $\mu$ es finita, y considere las expresiones l\'imite:
\begin{eqnarray}
lim_{n\rightarrow\infty}n^{-1}Z\left(T_{n}\right)&=&a,\textrm{ c.s.}\\
lim_{t\rightarrow\infty}t^{-1}Z\left(t\right)&=&a/\mu,\textrm{ c.s.}
\end{eqnarray}
La segunda expresi\'on implica la primera. Conversamente, la primera implica la segunda si el proceso $Z\left(t\right)$ es creciente, o si $lim_{n\rightarrow\infty}n^{-1}M_{n}=0$ c.s.
\end{Teo}

\begin{Coro}
Si $N\left(t\right)$ es un proceso de renovaci\'on, y $\left(Z\left(T_{n}\right)-Z\left(T_{n-1}\right),M_{n}\right)$, para $n\geq1$, son variables aleatorias independientes e id\'enticamente distribuidas con media finita, entonces,
\begin{eqnarray}
lim_{t\rightarrow\infty}t^{-1}Z\left(t\right)\rightarrow\frac{\esp\left[Z\left(T_{1}\right)-Z\left(T_{0}\right)\right]}{\esp\left[T_{1}\right]},\textrm{ c.s. cuando  }t\rightarrow\infty.
\end{eqnarray}
\end{Coro}


Sup\'ongase que $N\left(t\right)$ es un proceso de renovaci\'on con distribuci\'on $F$ con media finita $\mu$.

\begin{Def}
La funci\'on de renovaci\'on asociada con la distribuci\'on $F$, del proceso $N\left(t\right)$, es
\begin{eqnarray*}
U\left(t\right)=\sum_{n=1}^{\infty}F^{n\star}\left(t\right),\textrm{   }t\geq0,
\end{eqnarray*}
donde $F^{0\star}\left(t\right)=\indora\left(t\geq0\right)$.
\end{Def}


\begin{Prop}
Sup\'ongase que la distribuci\'on de inter-renovaci\'on $F$ tiene densidad $f$. Entonces $U\left(t\right)$ tambi\'en tiene densidad, para $t>0$, y es $U^{'}\left(t\right)=\sum_{n=0}^{\infty}f^{n\star}\left(t\right)$. Adem\'as
\begin{eqnarray*}
\prob\left\{N\left(t\right)>N\left(t-\right)\right\}=0\textrm{,   }t\geq0.
\end{eqnarray*}
\end{Prop}

\begin{Def}
La Transformada de Laplace-Stieljes de $F$ est\'a dada por

\begin{eqnarray*}
\hat{F}\left(\alpha\right)=\int_{\rea_{+}}e^{-\alpha t}dF\left(t\right)\textrm{,  }\alpha\geq0.
\end{eqnarray*}
\end{Def}

Entonces

\begin{eqnarray*}
\hat{U}\left(\alpha\right)=\sum_{n=0}^{\infty}\hat{F^{n\star}}\left(\alpha\right)=\sum_{n=0}^{\infty}\hat{F}\left(\alpha\right)^{n}=\frac{1}{1-\hat{F}\left(\alpha\right)}.
\end{eqnarray*}


\begin{Prop}
La Transformada de Laplace $\hat{U}\left(\alpha\right)$ y $\hat{F}\left(\alpha\right)$ determina una a la otra de manera \'unica por la relaci\'on $\hat{U}\left(\alpha\right)=\frac{1}{1-\hat{F}\left(\alpha\right)}$.
\end{Prop}


\begin{Note}
Un proceso de renovaci\'on $N\left(t\right)$ cuyos tiempos de inter-renovaci\'on tienen media finita, es un proceso Poisson con tasa $\lambda$ si y s\'olo s\'i $\esp\left[U\left(t\right)\right]=\lambda t$, para $t\geq0$.
\end{Note}


\begin{Teo}
Sea $N\left(t\right)$ un proceso puntual simple con puntos de localizaci\'on $T_{n}$ tal que $\eta\left(t\right)=\esp\left[N\left(\right)\right]$ es finita para cada $t$. Entonces para cualquier funci\'on $f:\rea_{+}\rightarrow\rea$,
\begin{eqnarray*}
\esp\left[\sum_{n=1}^{N\left(\right)}f\left(T_{n}\right)\right]=\int_{\left(0,t\right]}f\left(s\right)d\eta\left(s\right)\textrm{,  }t\geq0,
\end{eqnarray*}
suponiendo que la integral exista. Adem\'as si $X_{1},X_{2},\ldots$ son variables aleatorias definidas en el mismo espacio de probabilidad que el proceso $N\left(t\right)$ tal que $\esp\left[X_{n}|T_{n}=s\right]=f\left(s\right)$, independiente de $n$. Entonces
\begin{eqnarray*}
\esp\left[\sum_{n=1}^{N\left(t\right)}X_{n}\right]=\int_{\left(0,t\right]}f\left(s\right)d\eta\left(s\right)\textrm{,  }t\geq0,
\end{eqnarray*} 
suponiendo que la integral exista. 
\end{Teo}

\begin{Coro}[Identidad de Wald para Renovaciones]
Para el proceso de renovaci\'on $N\left(t\right)$,
\begin{eqnarray*}
\esp\left[T_{N\left(t\right)+1}\right]=\mu\esp\left[N\left(t\right)+1\right]\textrm{,  }t\geq0,
\end{eqnarray*}  
\end{Coro}


\begin{Def}
Sea $h\left(t\right)$ funci\'on de valores reales en $\rea$ acotada en intervalos finitos e igual a cero para $t<0$ La ecuaci\'on de renovaci\'on para $h\left(t\right)$ y la distribuci\'on $F$ es

\begin{eqnarray}%\label{Ec.Renovacion}
H\left(t\right)=h\left(t\right)+\int_{\left[0,t\right]}H\left(t-s\right)dF\left(s\right)\textrm{,    }t\geq0,
\end{eqnarray}
donde $H\left(t\right)$ es una funci\'on de valores reales. Esto es $H=h+F\star H$. Decimos que $H\left(t\right)$ es soluci\'on de esta ecuaci\'on si satisface la ecuaci\'on, y es acotada en intervalos finitos e iguales a cero para $t<0$.
\end{Def}

\begin{Prop}
La funci\'on $U\star h\left(t\right)$ es la \'unica soluci\'on de la ecuaci\'on de renovaci\'on (\ref{Ec.Renovacion}).
\end{Prop}

\begin{Teo}[Teorema Renovaci\'on Elemental]
\begin{eqnarray*}
t^{-1}U\left(t\right)\rightarrow 1/\mu\textrm{,    cuando }t\rightarrow\infty.
\end{eqnarray*}
\end{Teo}



Sup\'ongase que $N\left(t\right)$ es un proceso de renovaci\'on con distribuci\'on $F$ con media finita $\mu$.

\begin{Def}
La funci\'on de renovaci\'on asociada con la distribuci\'on $F$, del proceso $N\left(t\right)$, es
\begin{eqnarray*}
U\left(t\right)=\sum_{n=1}^{\infty}F^{n\star}\left(t\right),\textrm{   }t\geq0,
\end{eqnarray*}
donde $F^{0\star}\left(t\right)=\indora\left(t\geq0\right)$.
\end{Def}


\begin{Prop}
Sup\'ongase que la distribuci\'on de inter-renovaci\'on $F$ tiene densidad $f$. Entonces $U\left(t\right)$ tambi\'en tiene densidad, para $t>0$, y es $U^{'}\left(t\right)=\sum_{n=0}^{\infty}f^{n\star}\left(t\right)$. Adem\'as
\begin{eqnarray*}
\prob\left\{N\left(t\right)>N\left(t-\right)\right\}=0\textrm{,   }t\geq0.
\end{eqnarray*}
\end{Prop}

\begin{Def}
La Transformada de Laplace-Stieljes de $F$ est\'a dada por

\begin{eqnarray*}
\hat{F}\left(\alpha\right)=\int_{\rea_{+}}e^{-\alpha t}dF\left(t\right)\textrm{,  }\alpha\geq0.
\end{eqnarray*}
\end{Def}

Entonces

\begin{eqnarray*}
\hat{U}\left(\alpha\right)=\sum_{n=0}^{\infty}\hat{F^{n\star}}\left(\alpha\right)=\sum_{n=0}^{\infty}\hat{F}\left(\alpha\right)^{n}=\frac{1}{1-\hat{F}\left(\alpha\right)}.
\end{eqnarray*}


\begin{Prop}
La Transformada de Laplace $\hat{U}\left(\alpha\right)$ y $\hat{F}\left(\alpha\right)$ determina una a la otra de manera \'unica por la relaci\'on $\hat{U}\left(\alpha\right)=\frac{1}{1-\hat{F}\left(\alpha\right)}$.
\end{Prop}


\begin{Note}
Un proceso de renovaci\'on $N\left(t\right)$ cuyos tiempos de inter-renovaci\'on tienen media finita, es un proceso Poisson con tasa $\lambda$ si y s\'olo s\'i $\esp\left[U\left(t\right)\right]=\lambda t$, para $t\geq0$.
\end{Note}


\begin{Teo}
Sea $N\left(t\right)$ un proceso puntual simple con puntos de localizaci\'on $T_{n}$ tal que $\eta\left(t\right)=\esp\left[N\left(\right)\right]$ es finita para cada $t$. Entonces para cualquier funci\'on $f:\rea_{+}\rightarrow\rea$,
\begin{eqnarray*}
\esp\left[\sum_{n=1}^{N\left(\right)}f\left(T_{n}\right)\right]=\int_{\left(0,t\right]}f\left(s\right)d\eta\left(s\right)\textrm{,  }t\geq0,
\end{eqnarray*}
suponiendo que la integral exista. Adem\'as si $X_{1},X_{2},\ldots$ son variables aleatorias definidas en el mismo espacio de probabilidad que el proceso $N\left(t\right)$ tal que $\esp\left[X_{n}|T_{n}=s\right]=f\left(s\right)$, independiente de $n$. Entonces
\begin{eqnarray*}
\esp\left[\sum_{n=1}^{N\left(t\right)}X_{n}\right]=\int_{\left(0,t\right]}f\left(s\right)d\eta\left(s\right)\textrm{,  }t\geq0,
\end{eqnarray*} 
suponiendo que la integral exista. 
\end{Teo}

\begin{Coro}[Identidad de Wald para Renovaciones]
Para el proceso de renovaci\'on $N\left(t\right)$,
\begin{eqnarray*}
\esp\left[T_{N\left(t\right)+1}\right]=\mu\esp\left[N\left(t\right)+1\right]\textrm{,  }t\geq0,
\end{eqnarray*}  
\end{Coro}


\begin{Def}
Sea $h\left(t\right)$ funci\'on de valores reales en $\rea$ acotada en intervalos finitos e igual a cero para $t<0$ La ecuaci\'on de renovaci\'on para $h\left(t\right)$ y la distribuci\'on $F$ es

\begin{eqnarray}%\label{Ec.Renovacion}
H\left(t\right)=h\left(t\right)+\int_{\left[0,t\right]}H\left(t-s\right)dF\left(s\right)\textrm{,    }t\geq0,
\end{eqnarray}
donde $H\left(t\right)$ es una funci\'on de valores reales. Esto es $H=h+F\star H$. Decimos que $H\left(t\right)$ es soluci\'on de esta ecuaci\'on si satisface la ecuaci\'on, y es acotada en intervalos finitos e iguales a cero para $t<0$.
\end{Def}

\begin{Prop}
La funci\'on $U\star h\left(t\right)$ es la \'unica soluci\'on de la ecuaci\'on de renovaci\'on (\ref{Ec.Renovacion}).
\end{Prop}

\begin{Teo}[Teorema Renovaci\'on Elemental]
\begin{eqnarray*}
t^{-1}U\left(t\right)\rightarrow 1/\mu\textrm{,    cuando }t\rightarrow\infty.
\end{eqnarray*}
\end{Teo}


\begin{Note} Una funci\'on $h:\rea_{+}\rightarrow\rea$ es Directamente Riemann Integrable en los siguientes casos:
\begin{itemize}
\item[a)] $h\left(t\right)\geq0$ es decreciente y Riemann Integrable.
\item[b)] $h$ es continua excepto posiblemente en un conjunto de Lebesgue de medida 0, y $|h\left(t\right)|\leq b\left(t\right)$, donde $b$ es DRI.
\end{itemize}
\end{Note}

\begin{Teo}[Teorema Principal de Renovaci\'on]
Si $F$ es no aritm\'etica y $h\left(t\right)$ es Directamente Riemann Integrable (DRI), entonces

\begin{eqnarray*}
lim_{t\rightarrow\infty}U\star h=\frac{1}{\mu}\int_{\rea_{+}}h\left(s\right)ds.
\end{eqnarray*}
\end{Teo}

\begin{Prop}
Cualquier funci\'on $H\left(t\right)$ acotada en intervalos finitos y que es 0 para $t<0$ puede expresarse como
\begin{eqnarray*}
H\left(t\right)=U\star h\left(t\right)\textrm{,  donde }h\left(t\right)=H\left(t\right)-F\star H\left(t\right)
\end{eqnarray*}
\end{Prop}

\begin{Def}
Un proceso estoc\'astico $X\left(t\right)$ es crudamente regenerativo en un tiempo aleatorio positivo $T$ si
\begin{eqnarray*}
\esp\left[X\left(T+t\right)|T\right]=\esp\left[X\left(t\right)\right]\textrm{, para }t\geq0,\end{eqnarray*}
y con las esperanzas anteriores finitas.
\end{Def}

\begin{Prop}
Sup\'ongase que $X\left(t\right)$ es un proceso crudamente regenerativo en $T$, que tiene distribuci\'on $F$. Si $\esp\left[X\left(t\right)\right]$ es acotado en intervalos finitos, entonces
\begin{eqnarray*}
\esp\left[X\left(t\right)\right]=U\star h\left(t\right)\textrm{,  donde }h\left(t\right)=\esp\left[X\left(t\right)\indora\left(T>t\right)\right].
\end{eqnarray*}
\end{Prop}

\begin{Teo}[Regeneraci\'on Cruda]
Sup\'ongase que $X\left(t\right)$ es un proceso con valores positivo crudamente regenerativo en $T$, y def\'inase $M=\sup\left\{|X\left(t\right)|:t\leq T\right\}$. Si $T$ es no aritm\'etico y $M$ y $MT$ tienen media finita, entonces
\begin{eqnarray*}
lim_{t\rightarrow\infty}\esp\left[X\left(t\right)\right]=\frac{1}{\mu}\int_{\rea_{+}}h\left(s\right)ds,
\end{eqnarray*}
donde $h\left(t\right)=\esp\left[X\left(t\right)\indora\left(T>t\right)\right]$.
\end{Teo}


\begin{Note} Una funci\'on $h:\rea_{+}\rightarrow\rea$ es Directamente Riemann Integrable en los siguientes casos:
\begin{itemize}
\item[a)] $h\left(t\right)\geq0$ es decreciente y Riemann Integrable.
\item[b)] $h$ es continua excepto posiblemente en un conjunto de Lebesgue de medida 0, y $|h\left(t\right)|\leq b\left(t\right)$, donde $b$ es DRI.
\end{itemize}
\end{Note}

\begin{Teo}[Teorema Principal de Renovaci\'on]
Si $F$ es no aritm\'etica y $h\left(t\right)$ es Directamente Riemann Integrable (DRI), entonces

\begin{eqnarray*}
lim_{t\rightarrow\infty}U\star h=\frac{1}{\mu}\int_{\rea_{+}}h\left(s\right)ds.
\end{eqnarray*}
\end{Teo}

\begin{Prop}
Cualquier funci\'on $H\left(t\right)$ acotada en intervalos finitos y que es 0 para $t<0$ puede expresarse como
\begin{eqnarray*}
H\left(t\right)=U\star h\left(t\right)\textrm{,  donde }h\left(t\right)=H\left(t\right)-F\star H\left(t\right)
\end{eqnarray*}
\end{Prop}

\begin{Def}
Un proceso estoc\'astico $X\left(t\right)$ es crudamente regenerativo en un tiempo aleatorio positivo $T$ si
\begin{eqnarray*}
\esp\left[X\left(T+t\right)|T\right]=\esp\left[X\left(t\right)\right]\textrm{, para }t\geq0,\end{eqnarray*}
y con las esperanzas anteriores finitas.
\end{Def}

\begin{Prop}
Sup\'ongase que $X\left(t\right)$ es un proceso crudamente regenerativo en $T$, que tiene distribuci\'on $F$. Si $\esp\left[X\left(t\right)\right]$ es acotado en intervalos finitos, entonces
\begin{eqnarray*}
\esp\left[X\left(t\right)\right]=U\star h\left(t\right)\textrm{,  donde }h\left(t\right)=\esp\left[X\left(t\right)\indora\left(T>t\right)\right].
\end{eqnarray*}
\end{Prop}

\begin{Teo}[Regeneraci\'on Cruda]
Sup\'ongase que $X\left(t\right)$ es un proceso con valores positivo crudamente regenerativo en $T$, y def\'inase $M=\sup\left\{|X\left(t\right)|:t\leq T\right\}$. Si $T$ es no aritm\'etico y $M$ y $MT$ tienen media finita, entonces
\begin{eqnarray*}
lim_{t\rightarrow\infty}\esp\left[X\left(t\right)\right]=\frac{1}{\mu}\int_{\rea_{+}}h\left(s\right)ds,
\end{eqnarray*}
donde $h\left(t\right)=\esp\left[X\left(t\right)\indora\left(T>t\right)\right]$.
\end{Teo}

\begin{Def}
Para el proceso $\left\{\left(N\left(t\right),X\left(t\right)\right):t\geq0\right\}$, sus trayectoria muestrales en el intervalo de tiempo $\left[T_{n-1},T_{n}\right)$ est\'an descritas por
\begin{eqnarray*}
\zeta_{n}=\left(\xi_{n},\left\{X\left(T_{n-1}+t\right):0\leq t<\xi_{n}\right\}\right)
\end{eqnarray*}
Este $\zeta_{n}$ es el $n$-\'esimo segmento del proceso. El proceso es regenerativo sobre los tiempos $T_{n}$ si sus segmentos $\zeta_{n}$ son independientes e id\'enticamennte distribuidos.
\end{Def}


\begin{Note}
Si $\tilde{X}\left(t\right)$ con espacio de estados $\tilde{S}$ es regenerativo sobre $T_{n}$, entonces $X\left(t\right)=f\left(\tilde{X}\left(t\right)\right)$ tambi\'en es regenerativo sobre $T_{n}$, para cualquier funci\'on $f:\tilde{S}\rightarrow S$.
\end{Note}

\begin{Note}
Los procesos regenerativos son crudamente regenerativos, pero no al rev\'es.
\end{Note}


\begin{Note}
Un proceso estoc\'astico a tiempo continuo o discreto es regenerativo si existe un proceso de renovaci\'on  tal que los segmentos del proceso entre tiempos de renovaci\'on sucesivos son i.i.d., es decir, para $\left\{X\left(t\right):t\geq0\right\}$ proceso estoc\'astico a tiempo continuo con espacio de estados $S$, espacio m\'etrico.
\end{Note}

Para $\left\{X\left(t\right):t\geq0\right\}$ Proceso Estoc\'astico a tiempo continuo con estado de espacios $S$, que es un espacio m\'etrico, con trayectorias continuas por la derecha y con l\'imites por la izquierda c.s. Sea $N\left(t\right)$ un proceso de renovaci\'on en $\rea_{+}$ definido en el mismo espacio de probabilidad que $X\left(t\right)$, con tiempos de renovaci\'on $T$ y tiempos de inter-renovaci\'on $\xi_{n}=T_{n}-T_{n-1}$, con misma distribuci\'on $F$ de media finita $\mu$.



\begin{Def}
Para el proceso $\left\{\left(N\left(t\right),X\left(t\right)\right):t\geq0\right\}$, sus trayectoria muestrales en el intervalo de tiempo $\left[T_{n-1},T_{n}\right)$ est\'an descritas por
\begin{eqnarray*}
\zeta_{n}=\left(\xi_{n},\left\{X\left(T_{n-1}+t\right):0\leq t<\xi_{n}\right\}\right)
\end{eqnarray*}
Este $\zeta_{n}$ es el $n$-\'esimo segmento del proceso. El proceso es regenerativo sobre los tiempos $T_{n}$ si sus segmentos $\zeta_{n}$ son independientes e id\'enticamennte distribuidos.
\end{Def}

\begin{Note}
Un proceso regenerativo con media de la longitud de ciclo finita es llamado positivo recurrente.
\end{Note}

\begin{Teo}[Procesos Regenerativos]
Suponga que el proceso
\end{Teo}


\begin{Def}[Renewal Process Trinity]
Para un proceso de renovaci\'on $N\left(t\right)$, los siguientes procesos proveen de informaci\'on sobre los tiempos de renovaci\'on.
\begin{itemize}
\item $A\left(t\right)=t-T_{N\left(t\right)}$, el tiempo de recurrencia hacia atr\'as al tiempo $t$, que es el tiempo desde la \'ultima renovaci\'on para $t$.

\item $B\left(t\right)=T_{N\left(t\right)+1}-t$, el tiempo de recurrencia hacia adelante al tiempo $t$, residual del tiempo de renovaci\'on, que es el tiempo para la pr\'oxima renovaci\'on despu\'es de $t$.

\item $L\left(t\right)=\xi_{N\left(t\right)+1}=A\left(t\right)+B\left(t\right)$, la longitud del intervalo de renovaci\'on que contiene a $t$.
\end{itemize}
\end{Def}

\begin{Note}
El proceso tridimensional $\left(A\left(t\right),B\left(t\right),L\left(t\right)\right)$ es regenerativo sobre $T_{n}$, y por ende cada proceso lo es. Cada proceso $A\left(t\right)$ y $B\left(t\right)$ son procesos de MArkov a tiempo continuo con trayectorias continuas por partes en el espacio de estados $\rea_{+}$. Una expresi\'on conveniente para su distribuci\'on conjunta es, para $0\leq x<t,y\geq0$
\begin{equation}\label{NoRenovacion}
P\left\{A\left(t\right)>x,B\left(t\right)>y\right\}=
P\left\{N\left(t+y\right)-N\left((t-x)\right)=0\right\}
\end{equation}
\end{Note}

\begin{Ejem}[Tiempos de recurrencia Poisson]
Si $N\left(t\right)$ es un proceso Poisson con tasa $\lambda$, entonces de la expresi\'on (\ref{NoRenovacion}) se tiene que

\begin{eqnarray*}
\begin{array}{lc}
P\left\{A\left(t\right)>x,B\left(t\right)>y\right\}=e^{-\lambda\left(x+y\right)},&0\leq x<t,y\geq0,
\end{array}
\end{eqnarray*}
que es la probabilidad Poisson de no renovaciones en un intervalo de longitud $x+y$.

\end{Ejem}

\begin{Note}
Una cadena de Markov erg\'odica tiene la propiedad de ser estacionaria si la distribuci\'on de su estado al tiempo $0$ es su distribuci\'on estacionaria.
\end{Note}


\begin{Def}
Un proceso estoc\'astico a tiempo continuo $\left\{X\left(t\right):t\geq0\right\}$ en un espacio general es estacionario si sus distribuciones finito dimensionales son invariantes bajo cualquier  traslado: para cada $0\leq s_{1}<s_{2}<\cdots<s_{k}$ y $t\geq0$,
\begin{eqnarray*}
\left(X\left(s_{1}+t\right),\ldots,X\left(s_{k}+t\right)\right)=_{d}\left(X\left(s_{1}\right),\ldots,X\left(s_{k}\right)\right).
\end{eqnarray*}
\end{Def}

\begin{Note}
Un proceso de Markov es estacionario si $X\left(t\right)=_{d}X\left(0\right)$, $t\geq0$.
\end{Note}

Considerese el proceso $N\left(t\right)=\sum_{n}\indora\left(\tau_{n}\leq t\right)$ en $\rea_{+}$, con puntos $0<\tau_{1}<\tau_{2}<\cdots$.

\begin{Prop}
Si $N$ es un proceso puntual estacionario y $\esp\left[N\left(1\right)\right]<\infty$, entonces $\esp\left[N\left(t\right)\right]=t\esp\left[N\left(1\right)\right]$, $t\geq0$

\end{Prop}

\begin{Teo}
Los siguientes enunciados son equivalentes
\begin{itemize}
\item[i)] El proceso retardado de renovaci\'on $N$ es estacionario.

\item[ii)] EL proceso de tiempos de recurrencia hacia adelante $B\left(t\right)$ es estacionario.


\item[iii)] $\esp\left[N\left(t\right)\right]=t/\mu$,


\item[iv)] $G\left(t\right)=F_{e}\left(t\right)=\frac{1}{\mu}\int_{0}^{t}\left[1-F\left(s\right)\right]ds$
\end{itemize}
Cuando estos enunciados son ciertos, $P\left\{B\left(t\right)\leq x\right\}=F_{e}\left(x\right)$, para $t,x\geq0$.

\end{Teo}

\begin{Note}
Una consecuencia del teorema anterior es que el Proceso Poisson es el \'unico proceso sin retardo que es estacionario.
\end{Note}

\begin{Coro}
El proceso de renovaci\'on $N\left(t\right)$ sin retardo, y cuyos tiempos de inter renonaci\'on tienen media finita, es estacionario si y s\'olo si es un proceso Poisson.

\end{Coro}


%________________________________________________________________________
%\subsection{Procesos Regenerativos}
%________________________________________________________________________



\begin{Note}
Si $\tilde{X}\left(t\right)$ con espacio de estados $\tilde{S}$ es regenerativo sobre $T_{n}$, entonces $X\left(t\right)=f\left(\tilde{X}\left(t\right)\right)$ tambi\'en es regenerativo sobre $T_{n}$, para cualquier funci\'on $f:\tilde{S}\rightarrow S$.
\end{Note}

\begin{Note}
Los procesos regenerativos son crudamente regenerativos, pero no al rev\'es.
\end{Note}
%\subsection*{Procesos Regenerativos: Sigman\cite{Sigman1}}
\begin{Def}[Definici\'on Cl\'asica]
Un proceso estoc\'astico $X=\left\{X\left(t\right):t\geq0\right\}$ es llamado regenerativo is existe una variable aleatoria $R_{1}>0$ tal que
\begin{itemize}
\item[i)] $\left\{X\left(t+R_{1}\right):t\geq0\right\}$ es independiente de $\left\{\left\{X\left(t\right):t<R_{1}\right\},\right\}$
\item[ii)] $\left\{X\left(t+R_{1}\right):t\geq0\right\}$ es estoc\'asticamente equivalente a $\left\{X\left(t\right):t>0\right\}$
\end{itemize}

Llamamos a $R_{1}$ tiempo de regeneraci\'on, y decimos que $X$ se regenera en este punto.
\end{Def}

$\left\{X\left(t+R_{1}\right)\right\}$ es regenerativo con tiempo de regeneraci\'on $R_{2}$, independiente de $R_{1}$ pero con la misma distribuci\'on que $R_{1}$. Procediendo de esta manera se obtiene una secuencia de variables aleatorias independientes e id\'enticamente distribuidas $\left\{R_{n}\right\}$ llamados longitudes de ciclo. Si definimos a $Z_{k}\equiv R_{1}+R_{2}+\cdots+R_{k}$, se tiene un proceso de renovaci\'on llamado proceso de renovaci\'on encajado para $X$.




\begin{Def}
Para $x$ fijo y para cada $t\geq0$, sea $I_{x}\left(t\right)=1$ si $X\left(t\right)\leq x$,  $I_{x}\left(t\right)=0$ en caso contrario, y def\'inanse los tiempos promedio
\begin{eqnarray*}
\overline{X}&=&lim_{t\rightarrow\infty}\frac{1}{t}\int_{0}^{\infty}X\left(u\right)du\\
\prob\left(X_{\infty}\leq x\right)&=&lim_{t\rightarrow\infty}\frac{1}{t}\int_{0}^{\infty}I_{x}\left(u\right)du,
\end{eqnarray*}
cuando estos l\'imites existan.
\end{Def}

Como consecuencia del teorema de Renovaci\'on-Recompensa, se tiene que el primer l\'imite  existe y es igual a la constante
\begin{eqnarray*}
\overline{X}&=&\frac{\esp\left[\int_{0}^{R_{1}}X\left(t\right)dt\right]}{\esp\left[R_{1}\right]},
\end{eqnarray*}
suponiendo que ambas esperanzas son finitas.

\begin{Note}
\begin{itemize}
\item[a)] Si el proceso regenerativo $X$ es positivo recurrente y tiene trayectorias muestrales no negativas, entonces la ecuaci\'on anterior es v\'alida.
\item[b)] Si $X$ es positivo recurrente regenerativo, podemos construir una \'unica versi\'on estacionaria de este proceso, $X_{e}=\left\{X_{e}\left(t\right)\right\}$, donde $X_{e}$ es un proceso estoc\'astico regenerativo y estrictamente estacionario, con distribuci\'on marginal distribuida como $X_{\infty}$
\end{itemize}
\end{Note}

%________________________________________________________________________
%\subsection{Procesos Regenerativos}
%________________________________________________________________________

Para $\left\{X\left(t\right):t\geq0\right\}$ Proceso Estoc\'astico a tiempo continuo con estado de espacios $S$, que es un espacio m\'etrico, con trayectorias continuas por la derecha y con l\'imites por la izquierda c.s. Sea $N\left(t\right)$ un proceso de renovaci\'on en $\rea_{+}$ definido en el mismo espacio de probabilidad que $X\left(t\right)$, con tiempos de renovaci\'on $T$ y tiempos de inter-renovaci\'on $\xi_{n}=T_{n}-T_{n-1}$, con misma distribuci\'on $F$ de media finita $\mu$.



\begin{Def}
Para el proceso $\left\{\left(N\left(t\right),X\left(t\right)\right):t\geq0\right\}$, sus trayectoria muestrales en el intervalo de tiempo $\left[T_{n-1},T_{n}\right)$ est\'an descritas por
\begin{eqnarray*}
\zeta_{n}=\left(\xi_{n},\left\{X\left(T_{n-1}+t\right):0\leq t<\xi_{n}\right\}\right)
\end{eqnarray*}
Este $\zeta_{n}$ es el $n$-\'esimo segmento del proceso. El proceso es regenerativo sobre los tiempos $T_{n}$ si sus segmentos $\zeta_{n}$ son independientes e id\'enticamennte distribuidos.
\end{Def}


\begin{Note}
Si $\tilde{X}\left(t\right)$ con espacio de estados $\tilde{S}$ es regenerativo sobre $T_{n}$, entonces $X\left(t\right)=f\left(\tilde{X}\left(t\right)\right)$ tambi\'en es regenerativo sobre $T_{n}$, para cualquier funci\'on $f:\tilde{S}\rightarrow S$.
\end{Note}

\begin{Note}
Los procesos regenerativos son crudamente regenerativos, pero no al rev\'es.
\end{Note}

\begin{Def}[Definici\'on Cl\'asica]
Un proceso estoc\'astico $X=\left\{X\left(t\right):t\geq0\right\}$ es llamado regenerativo is existe una variable aleatoria $R_{1}>0$ tal que
\begin{itemize}
\item[i)] $\left\{X\left(t+R_{1}\right):t\geq0\right\}$ es independiente de $\left\{\left\{X\left(t\right):t<R_{1}\right\},\right\}$
\item[ii)] $\left\{X\left(t+R_{1}\right):t\geq0\right\}$ es estoc\'asticamente equivalente a $\left\{X\left(t\right):t>0\right\}$
\end{itemize}

Llamamos a $R_{1}$ tiempo de regeneraci\'on, y decimos que $X$ se regenera en este punto.
\end{Def}

$\left\{X\left(t+R_{1}\right)\right\}$ es regenerativo con tiempo de regeneraci\'on $R_{2}$, independiente de $R_{1}$ pero con la misma distribuci\'on que $R_{1}$. Procediendo de esta manera se obtiene una secuencia de variables aleatorias independientes e id\'enticamente distribuidas $\left\{R_{n}\right\}$ llamados longitudes de ciclo. Si definimos a $Z_{k}\equiv R_{1}+R_{2}+\cdots+R_{k}$, se tiene un proceso de renovaci\'on llamado proceso de renovaci\'on encajado para $X$.

\begin{Note}
Un proceso regenerativo con media de la longitud de ciclo finita es llamado positivo recurrente.
\end{Note}


\begin{Def}
Para $x$ fijo y para cada $t\geq0$, sea $I_{x}\left(t\right)=1$ si $X\left(t\right)\leq x$,  $I_{x}\left(t\right)=0$ en caso contrario, y def\'inanse los tiempos promedio
\begin{eqnarray*}
\overline{X}&=&lim_{t\rightarrow\infty}\frac{1}{t}\int_{0}^{\infty}X\left(u\right)du\\
\prob\left(X_{\infty}\leq x\right)&=&lim_{t\rightarrow\infty}\frac{1}{t}\int_{0}^{\infty}I_{x}\left(u\right)du,
\end{eqnarray*}
cuando estos l\'imites existan.
\end{Def}

Como consecuencia del teorema de Renovaci\'on-Recompensa, se tiene que el primer l\'imite  existe y es igual a la constante
\begin{eqnarray*}
\overline{X}&=&\frac{\esp\left[\int_{0}^{R_{1}}X\left(t\right)dt\right]}{\esp\left[R_{1}\right]},
\end{eqnarray*}
suponiendo que ambas esperanzas son finitas.

\begin{Note}
\begin{itemize}
\item[a)] Si el proceso regenerativo $X$ es positivo recurrente y tiene trayectorias muestrales no negativas, entonces la ecuaci\'on anterior es v\'alida.
\item[b)] Si $X$ es positivo recurrente regenerativo, podemos construir una \'unica versi\'on estacionaria de este proceso, $X_{e}=\left\{X_{e}\left(t\right)\right\}$, donde $X_{e}$ es un proceso estoc\'astico regenerativo y estrictamente estacionario, con distribuci\'on marginal distribuida como $X_{\infty}$
\end{itemize}
\end{Note}

%__________________________________________________________________________________________
%\subsection{Procesos Regenerativos Estacionarios - Stidham \cite{Stidham}}
%__________________________________________________________________________________________


Un proceso estoc\'astico a tiempo continuo $\left\{V\left(t\right),t\geq0\right\}$ es un proceso regenerativo si existe una sucesi\'on de variables aleatorias independientes e id\'enticamente distribuidas $\left\{X_{1},X_{2},\ldots\right\}$, sucesi\'on de renovaci\'on, tal que para cualquier conjunto de Borel $A$, 

\begin{eqnarray*}
\prob\left\{V\left(t\right)\in A|X_{1}+X_{2}+\cdots+X_{R\left(t\right)}=s,\left\{V\left(\tau\right),\tau<s\right\}\right\}=\prob\left\{V\left(t-s\right)\in A|X_{1}>t-s\right\},
\end{eqnarray*}
para todo $0\leq s\leq t$, donde $R\left(t\right)=\max\left\{X_{1}+X_{2}+\cdots+X_{j}\leq t\right\}=$n\'umero de renovaciones ({\emph{puntos de regeneraci\'on}}) que ocurren en $\left[0,t\right]$. El intervalo $\left[0,X_{1}\right)$ es llamado {\emph{primer ciclo de regeneraci\'on}} de $\left\{V\left(t \right),t\geq0\right\}$, $\left[X_{1},X_{1}+X_{2}\right)$ el {\emph{segundo ciclo de regeneraci\'on}}, y as\'i sucesivamente.

Sea $X=X_{1}$ y sea $F$ la funci\'on de distrbuci\'on de $X$


\begin{Def}
Se define el proceso estacionario, $\left\{V^{*}\left(t\right),t\geq0\right\}$, para $\left\{V\left(t\right),t\geq0\right\}$ por

\begin{eqnarray*}
\prob\left\{V\left(t\right)\in A\right\}=\frac{1}{\esp\left[X\right]}\int_{0}^{\infty}\prob\left\{V\left(t+x\right)\in A|X>x\right\}\left(1-F\left(x\right)\right)dx,
\end{eqnarray*} 
para todo $t\geq0$ y todo conjunto de Borel $A$.
\end{Def}

\begin{Def}
Una distribuci\'on se dice que es {\emph{aritm\'etica}} si todos sus puntos de incremento son m\'ultiplos de la forma $0,\lambda, 2\lambda,\ldots$ para alguna $\lambda>0$ entera.
\end{Def}


\begin{Def}
Una modificaci\'on medible de un proceso $\left\{V\left(t\right),t\geq0\right\}$, es una versi\'on de este, $\left\{V\left(t,w\right)\right\}$ conjuntamente medible para $t\geq0$ y para $w\in S$, $S$ espacio de estados para $\left\{V\left(t\right),t\geq0\right\}$.
\end{Def}

\begin{Teo}
Sea $\left\{V\left(t\right),t\geq\right\}$ un proceso regenerativo no negativo con modificaci\'on medible. Sea $\esp\left[X\right]<\infty$. Entonces el proceso estacionario dado por la ecuaci\'on anterior est\'a bien definido y tiene funci\'on de distribuci\'on independiente de $t$, adem\'as
\begin{itemize}
\item[i)] \begin{eqnarray*}
\esp\left[V^{*}\left(0\right)\right]&=&\frac{\esp\left[\int_{0}^{X}V\left(s\right)ds\right]}{\esp\left[X\right]}\end{eqnarray*}
\item[ii)] Si $\esp\left[V^{*}\left(0\right)\right]<\infty$, equivalentemente, si $\esp\left[\int_{0}^{X}V\left(s\right)ds\right]<\infty$,entonces
\begin{eqnarray*}
\frac{\int_{0}^{t}V\left(s\right)ds}{t}\rightarrow\frac{\esp\left[\int_{0}^{X}V\left(s\right)ds\right]}{\esp\left[X\right]}
\end{eqnarray*}
con probabilidad 1 y en media, cuando $t\rightarrow\infty$.
\end{itemize}
\end{Teo}
%
%___________________________________________________________________________________________
%\vspace{5.5cm}
%\chapter{Cadenas de Markov estacionarias}
%\vspace{-1.0cm}


%__________________________________________________________________________________________
%\subsection{Procesos Regenerativos Estacionarios - Stidham \cite{Stidham}}
%__________________________________________________________________________________________


Un proceso estoc\'astico a tiempo continuo $\left\{V\left(t\right),t\geq0\right\}$ es un proceso regenerativo si existe una sucesi\'on de variables aleatorias independientes e id\'enticamente distribuidas $\left\{X_{1},X_{2},\ldots\right\}$, sucesi\'on de renovaci\'on, tal que para cualquier conjunto de Borel $A$, 

\begin{eqnarray*}
\prob\left\{V\left(t\right)\in A|X_{1}+X_{2}+\cdots+X_{R\left(t\right)}=s,\left\{V\left(\tau\right),\tau<s\right\}\right\}=\prob\left\{V\left(t-s\right)\in A|X_{1}>t-s\right\},
\end{eqnarray*}
para todo $0\leq s\leq t$, donde $R\left(t\right)=\max\left\{X_{1}+X_{2}+\cdots+X_{j}\leq t\right\}=$n\'umero de renovaciones ({\emph{puntos de regeneraci\'on}}) que ocurren en $\left[0,t\right]$. El intervalo $\left[0,X_{1}\right)$ es llamado {\emph{primer ciclo de regeneraci\'on}} de $\left\{V\left(t \right),t\geq0\right\}$, $\left[X_{1},X_{1}+X_{2}\right)$ el {\emph{segundo ciclo de regeneraci\'on}}, y as\'i sucesivamente.

Sea $X=X_{1}$ y sea $F$ la funci\'on de distrbuci\'on de $X$


\begin{Def}
Se define el proceso estacionario, $\left\{V^{*}\left(t\right),t\geq0\right\}$, para $\left\{V\left(t\right),t\geq0\right\}$ por

\begin{eqnarray*}
\prob\left\{V\left(t\right)\in A\right\}=\frac{1}{\esp\left[X\right]}\int_{0}^{\infty}\prob\left\{V\left(t+x\right)\in A|X>x\right\}\left(1-F\left(x\right)\right)dx,
\end{eqnarray*} 
para todo $t\geq0$ y todo conjunto de Borel $A$.
\end{Def}

\begin{Def}
Una distribuci\'on se dice que es {\emph{aritm\'etica}} si todos sus puntos de incremento son m\'ultiplos de la forma $0,\lambda, 2\lambda,\ldots$ para alguna $\lambda>0$ entera.
\end{Def}


\begin{Def}
Una modificaci\'on medible de un proceso $\left\{V\left(t\right),t\geq0\right\}$, es una versi\'on de este, $\left\{V\left(t,w\right)\right\}$ conjuntamente medible para $t\geq0$ y para $w\in S$, $S$ espacio de estados para $\left\{V\left(t\right),t\geq0\right\}$.
\end{Def}

\begin{Teo}
Sea $\left\{V\left(t\right),t\geq\right\}$ un proceso regenerativo no negativo con modificaci\'on medible. Sea $\esp\left[X\right]<\infty$. Entonces el proceso estacionario dado por la ecuaci\'on anterior est\'a bien definido y tiene funci\'on de distribuci\'on independiente de $t$, adem\'as
\begin{itemize}
\item[i)] \begin{eqnarray*}
\esp\left[V^{*}\left(0\right)\right]&=&\frac{\esp\left[\int_{0}^{X}V\left(s\right)ds\right]}{\esp\left[X\right]}\end{eqnarray*}
\item[ii)] Si $\esp\left[V^{*}\left(0\right)\right]<\infty$, equivalentemente, si $\esp\left[\int_{0}^{X}V\left(s\right)ds\right]<\infty$,entonces
\begin{eqnarray*}
\frac{\int_{0}^{t}V\left(s\right)ds}{t}\rightarrow\frac{\esp\left[\int_{0}^{X}V\left(s\right)ds\right]}{\esp\left[X\right]}
\end{eqnarray*}
con probabilidad 1 y en media, cuando $t\rightarrow\infty$.
\end{itemize}
\end{Teo}

Para $\left\{X\left(t\right):t\geq0\right\}$ Proceso Estoc\'astico a tiempo continuo con estado de espacios $S$, que es un espacio m\'etrico, con trayectorias continuas por la derecha y con l\'imites por la izquierda c.s. Sea $N\left(t\right)$ un proceso de renovaci\'on en $\rea_{+}$ definido en el mismo espacio de probabilidad que $X\left(t\right)$, con tiempos de renovaci\'on $T$ y tiempos de inter-renovaci\'on $\xi_{n}=T_{n}-T_{n-1}$, con misma distribuci\'on $F$ de media finita $\mu$.


%______________________________________________________________________
%\subsection{Ejemplos, Notas importantes}


Sean $T_{1},T_{2},\ldots$ los puntos donde las longitudes de las colas de la red de sistemas de visitas c\'iclicas son cero simult\'aneamente, cuando la cola $Q_{j}$ es visitada por el servidor para dar servicio, es decir, $L_{1}\left(T_{i}\right)=0,L_{2}\left(T_{i}\right)=0,\hat{L}_{1}\left(T_{i}\right)=0$ y $\hat{L}_{2}\left(T_{i}\right)=0$, a estos puntos se les denominar\'a puntos regenerativos. Sea la funci\'on generadora de momentos para $L_{i}$, el n\'umero de usuarios en la cola $Q_{i}\left(z\right)$ en cualquier momento, est\'a dada por el tiempo promedio de $z^{L_{i}\left(t\right)}$ sobre el ciclo regenerativo definido anteriormente:

\begin{eqnarray*}
Q_{i}\left(z\right)&=&\esp\left[z^{L_{i}\left(t\right)}\right]=\frac{\esp\left[\sum_{m=1}^{M_{i}}\sum_{t=\tau_{i}\left(m\right)}^{\tau_{i}\left(m+1\right)-1}z^{L_{i}\left(t\right)}\right]}{\esp\left[\sum_{m=1}^{M_{i}}\tau_{i}\left(m+1\right)-\tau_{i}\left(m\right)\right]}
\end{eqnarray*}

$M_{i}$ es un tiempo de paro en el proceso regenerativo con $\esp\left[M_{i}\right]<\infty$\footnote{En Stidham\cite{Stidham} y Heyman  se muestra que una condici\'on suficiente para que el proceso regenerativo 
estacionario sea un procesoo estacionario es que el valor esperado del tiempo del ciclo regenerativo sea finito, es decir: $\esp\left[\sum_{m=1}^{M_{i}}C_{i}^{(m)}\right]<\infty$, como cada $C_{i}^{(m)}$ contiene intervalos de r\'eplica positivos, se tiene que $\esp\left[M_{i}\right]<\infty$, adem\'as, como $M_{i}>0$, se tiene que la condici\'on anterior es equivalente a tener que $\esp\left[C_{i}\right]<\infty$,
por lo tanto una condici\'on suficiente para la existencia del proceso regenerativo est\'a dada por $\sum_{k=1}^{N}\mu_{k}<1.$}, se sigue del lema de Wald que:


\begin{eqnarray*}
\esp\left[\sum_{m=1}^{M_{i}}\sum_{t=\tau_{i}\left(m\right)}^{\tau_{i}\left(m+1\right)-1}z^{L_{i}\left(t\right)}\right]&=&\esp\left[M_{i}\right]\esp\left[\sum_{t=\tau_{i}\left(m\right)}^{\tau_{i}\left(m+1\right)-1}z^{L_{i}\left(t\right)}\right]\\
\esp\left[\sum_{m=1}^{M_{i}}\tau_{i}\left(m+1\right)-\tau_{i}\left(m\right)\right]&=&\esp\left[M_{i}\right]\esp\left[\tau_{i}\left(m+1\right)-\tau_{i}\left(m\right)\right]
\end{eqnarray*}

por tanto se tiene que


\begin{eqnarray*}
Q_{i}\left(z\right)&=&\frac{\esp\left[\sum_{t=\tau_{i}\left(m\right)}^{\tau_{i}\left(m+1\right)-1}z^{L_{i}\left(t\right)}\right]}{\esp\left[\tau_{i}\left(m+1\right)-\tau_{i}\left(m\right)\right]}
\end{eqnarray*}

observar que el denominador es simplemente la duraci\'on promedio del tiempo del ciclo.


Haciendo las siguientes sustituciones en la ecuaci\'on (\ref{Corolario2}): $n\rightarrow t-\tau_{i}\left(m\right)$, $T \rightarrow \overline{\tau}_{i}\left(m\right)-\tau_{i}\left(m\right)$, $L_{n}\rightarrow L_{i}\left(t\right)$ y $F\left(z\right)=\esp\left[z^{L_{0}}\right]\rightarrow F_{i}\left(z\right)=\esp\left[z^{L_{i}\tau_{i}\left(m\right)}\right]$, se puede ver que

\begin{eqnarray}\label{Eq.Arribos.Primera}
\esp\left[\sum_{n=0}^{T-1}z^{L_{n}}\right]=
\esp\left[\sum_{t=\tau_{i}\left(m\right)}^{\overline{\tau}_{i}\left(m\right)-1}z^{L_{i}\left(t\right)}\right]
=z\frac{F_{i}\left(z\right)-1}{z-P_{i}\left(z\right)}
\end{eqnarray}

Por otra parte durante el tiempo de intervisita para la cola $i$, $L_{i}\left(t\right)$ solamente se incrementa de manera que el incremento por intervalo de tiempo est\'a dado por la funci\'on generadora de probabilidades de $P_{i}\left(z\right)$, por tanto la suma sobre el tiempo de intervisita puede evaluarse como:

\begin{eqnarray*}
\esp\left[\sum_{t=\tau_{i}\left(m\right)}^{\tau_{i}\left(m+1\right)-1}z^{L_{i}\left(t\right)}\right]&=&\esp\left[\sum_{t=\tau_{i}\left(m\right)}^{\tau_{i}\left(m+1\right)-1}\left\{P_{i}\left(z\right)\right\}^{t-\overline{\tau}_{i}\left(m\right)}\right]=\frac{1-\esp\left[\left\{P_{i}\left(z\right)\right\}^{\tau_{i}\left(m+1\right)-\overline{\tau}_{i}\left(m\right)}\right]}{1-P_{i}\left(z\right)}\\
&=&\frac{1-I_{i}\left[P_{i}\left(z\right)\right]}{1-P_{i}\left(z\right)}
\end{eqnarray*}
por tanto

\begin{eqnarray*}
\esp\left[\sum_{t=\tau_{i}\left(m\right)}^{\tau_{i}\left(m+1\right)-1}z^{L_{i}\left(t\right)}\right]&=&
\frac{1-F_{i}\left(z\right)}{1-P_{i}\left(z\right)}
\end{eqnarray*}

Por lo tanto

\begin{eqnarray*}
Q_{i}\left(z\right)&=&\frac{\esp\left[\sum_{t=\tau_{i}\left(m\right)}^{\tau_{i}
\left(m+1\right)-1}z^{L_{i}\left(t\right)}\right]}{\esp\left[\tau_{i}\left(m+1\right)-\tau_{i}\left(m\right)\right]}\\
&=&\frac{1}{\esp\left[\tau_{i}\left(m+1\right)-\tau_{i}\left(m\right)\right]}
\left\{
\esp\left[\sum_{t=\tau_{i}\left(m\right)}^{\overline{\tau}_{i}\left(m\right)-1}
z^{L_{i}\left(t\right)}\right]
+\esp\left[\sum_{t=\overline{\tau}_{i}\left(m\right)}^{\tau_{i}\left(m+1\right)-1}
z^{L_{i}\left(t\right)}\right]\right\}\\
&=&\frac{1}{\esp\left[\tau_{i}\left(m+1\right)-\tau_{i}\left(m\right)\right]}
\left\{
z\frac{F_{i}\left(z\right)-1}{z-P_{i}\left(z\right)}+\frac{1-F_{i}\left(z\right)}
{1-P_{i}\left(z\right)}
\right\}
\end{eqnarray*}

es decir

\begin{equation}
Q_{i}\left(z\right)=\frac{1}{\esp\left[C_{i}\right]}\cdot\frac{1-F_{i}\left(z\right)}{P_{i}\left(z\right)-z}\cdot\frac{\left(1-z\right)P_{i}\left(z\right)}{1-P_{i}\left(z\right)}
\end{equation}

\begin{Teo}
Dada una Red de Sistemas de Visitas C\'iclicas (RSVC), conformada por dos Sistemas de Visitas C\'iclicas (SVC), donde cada uno de ellos consta de dos colas tipo $M/M/1$. Los dos sistemas est\'an comunicados entre s\'i por medio de la transferencia de usuarios entre las colas $Q_{1}\leftrightarrow Q_{3}$ y $Q_{2}\leftrightarrow Q_{4}$. Se definen los eventos para los procesos de arribos al tiempo $t$, $A_{j}\left(t\right)=\left\{0 \textrm{ arribos en }Q_{j}\textrm{ al tiempo }t\right\}$ para alg\'un tiempo $t\geq0$ y $Q_{j}$ la cola $j$-\'esima en la RSVC, para $j=1,2,3,4$.  Existe un intervalo $I\neq\emptyset$ tal que para $T^{*}\in I$, tal que $\prob\left\{A_{1}\left(T^{*}\right),A_{2}\left(Tt^{*}\right),
A_{3}\left(T^{*}\right),A_{4}\left(T^{*}\right)|T^{*}\in I\right\}>0$.
\end{Teo}

\begin{proof}
Sin p\'erdida de generalidad podemos considerar como base del an\'alisis a la cola $Q_{1}$ del primer sistema que conforma la RSVC.

Sea $n>0$, ciclo en el primer sistema en el que se sabe que $L_{j}\left(\overline{\tau}_{1}\left(n\right)\right)=0$, pues la pol\'itica de servicio con que atienden los servidores es la exhaustiva. Como es sabido, para trasladarse a la siguiente cola, el servidor incurre en un tiempo de traslado $r_{1}\left(n\right)>0$, entonces tenemos que $\tau_{2}\left(n\right)=\overline{\tau}_{1}\left(n\right)+r_{1}\left(n\right)$.


Definamos el intervalo $I_{1}\equiv\left[\overline{\tau}_{1}\left(n\right),\tau_{2}\left(n\right)\right]$ de longitud $\xi_{1}=r_{1}\left(n\right)$. Dado que los tiempos entre arribo son exponenciales con tasa $\tilde{\mu}_{1}=\mu_{1}+\hat{\mu}_{1}$ ($\mu_{1}$ son los arribos a $Q_{1}$ por primera vez al sistema, mientras que $\hat{\mu}_{1}$ son los arribos de traslado procedentes de $Q_{3}$) se tiene que la probabilidad del evento $A_{1}\left(t\right)$ est\'a dada por 

\begin{equation}
\prob\left\{A_{1}\left(t\right)|t\in I_{1}\left(n\right)\right\}=e^{-\tilde{\mu}_{1}\xi_{1}\left(n\right)}.
\end{equation} 

Por otra parte, para la cola $Q_{2}$, el tiempo $\overline{\tau}_{2}\left(n-1\right)$ es tal que $L_{2}\left(\overline{\tau}_{2}\left(n-1\right)\right)=0$, es decir, es el tiempo en que la cola queda totalmente vac\'ia en el ciclo anterior a $n$. Entonces tenemos un sgundo intervalo $I_{2}\equiv\left[\overline{\tau}_{2}\left(n-1\right),\tau_{2}\left(n\right)\right]$. Por lo tanto la probabilidad del evento $A_{2}\left(t\right)$ tiene probabilidad dada por

\begin{equation}
\prob\left\{A_{2}\left(t\right)|t\in I_{2}\left(n\right)\right\}=e^{-\tilde{\mu}_{2}\xi_{2}\left(n\right)},
\end{equation} 

donde $\xi_{2}\left(n\right)=\tau_{2}\left(n\right)-\overline{\tau}_{2}\left(n-1\right)$.



Entonces, se tiene que

\begin{eqnarray*}
\prob\left\{A_{1}\left(t\right),A_{2}\left(t\right)|t\in I_{1}\left(n\right)\right\}&=&
\prob\left\{A_{1}\left(t\right)|t\in I_{1}\left(n\right)\right\}
\prob\left\{A_{2}\left(t\right)|t\in I_{1}\left(n\right)\right\}\\
&\geq&
\prob\left\{A_{1}\left(t\right)|t\in I_{1}\left(n\right)\right\}
\prob\left\{A_{2}\left(t\right)|t\in I_{2}\left(n\right)\right\}\\
&=&e^{-\tilde{\mu}_{1}\xi_{1}\left(n\right)}e^{-\tilde{\mu}_{2}\xi_{2}\left(n\right)}
=e^{-\left[\tilde{\mu}_{1}\xi_{1}\left(n\right)+\tilde{\mu}_{2}\xi_{2}\left(n\right)\right]}.
\end{eqnarray*}


es decir, 

\begin{equation}
\prob\left\{A_{1}\left(t\right),A_{2}\left(t\right)|t\in I_{1}\left(n\right)\right\}
=e^{-\left[\tilde{\mu}_{1}\xi_{1}\left(n\right)+\tilde{\mu}_{2}\xi_{2}
\left(n\right)\right]}>0.
\end{equation}

En lo que respecta a la relaci\'on entre los dos SVC que conforman la RSVC, se afirma que existe $m>0$ tal que $\overline{\tau}_{3}\left(m\right)<\tau_{2}\left(n\right)<\tau_{4}\left(m\right)$.

Para $Q_{3}$ sea $I_{3}=\left[\overline{\tau}_{3}\left(m\right),\tau_{4}\left(m\right)\right]$ con longitud  $\xi_{3}\left(m\right)=r_{3}\left(m\right)$, entonces 

\begin{equation}
\prob\left\{A_{3}\left(t\right)|t\in I_{3}\left(n\right)\right\}=e^{-\tilde{\mu}_{3}\xi_{3}\left(n\right)}.
\end{equation} 

An\'alogamente que como se hizo para $Q_{2}$, tenemos que para $Q_{4}$ se tiene el intervalo $I_{4}=\left[\overline{\tau}_{4}\left(m-1\right),\tau_{4}\left(m\right)\right]$ con longitud $\xi_{4}\left(m\right)=\tau_{4}\left(m\right)-\overline{\tau}_{4}\left(m-1\right)$, entonces


\begin{equation}
\prob\left\{A_{4}\left(t\right)|t\in I_{4}\left(m\right)\right\}=e^{-\tilde{\mu}_{4}\xi_{4}\left(n\right)}.
\end{equation} 

Al igual que para el primer sistema, dado que $I_{3}\left(m\right)\subset I_{4}\left(m\right)$, se tiene que

\begin{eqnarray*}
\xi_{3}\left(m\right)\leq\xi_{4}\left(m\right)&\Leftrightarrow& -\xi_{3}\left(m\right)\geq-\xi_{4}\left(m\right)
\\
-\tilde{\mu}_{4}\xi_{3}\left(m\right)\geq-\tilde{\mu}_{4}\xi_{4}\left(m\right)&\Leftrightarrow&
e^{-\tilde{\mu}_{4}\xi_{3}\left(m\right)}\geq e^{-\tilde{\mu}_{4}\xi_{4}\left(m\right)}\\
\prob\left\{A_{4}\left(t\right)|t\in I_{3}\left(m\right)\right\}&\geq&
\prob\left\{A_{4}\left(t\right)|t\in I_{4}\left(m\right)\right\}
\end{eqnarray*}

Entonces, dado que los eventos $A_{3}$ y $A_{4}$ son independientes, se tiene que

\begin{eqnarray*}
\prob\left\{A_{3}\left(t\right),A_{4}\left(t\right)|t\in I_{3}\left(m\right)\right\}&=&
\prob\left\{A_{3}\left(t\right)|t\in I_{3}\left(m\right)\right\}
\prob\left\{A_{4}\left(t\right)|t\in I_{3}\left(m\right)\right\}\\
&\geq&
\prob\left\{A_{3}\left(t\right)|t\in I_{3}\left(n\right)\right\}
\prob\left\{A_{4}\left(t\right)|t\in I_{4}\left(n\right)\right\}\\
&=&e^{-\tilde{\mu}_{3}\xi_{3}\left(m\right)}e^{-\tilde{\mu}_{4}\xi_{4}
\left(m\right)}
=e^{-\left[\tilde{\mu}_{3}\xi_{3}\left(m\right)+\tilde{\mu}_{4}\xi_{4}
\left(m\right)\right]}.
\end{eqnarray*}


es decir, 

\begin{equation}
\prob\left\{A_{3}\left(t\right),A_{4}\left(t\right)|t\in I_{3}\left(m\right)\right\}
=e^{-\left[\tilde{\mu}_{3}\xi_{3}\left(m\right)+\tilde{\mu}_{4}\xi_{4}
\left(m\right)\right]}>0.
\end{equation}

Por construcci\'on se tiene que $I\left(n,m\right)\equiv I_{1}\left(n\right)\cap I_{3}\left(m\right)\neq\emptyset$,entonces en particular se tienen las contenciones $I\left(n,m\right)\subseteq I_{1}\left(n\right)$ y $I\left(n,m\right)\subseteq I_{3}\left(m\right)$, por lo tanto si definimos $\xi_{n,m}\equiv\ell\left(I\left(n,m\right)\right)$ tenemos que

\begin{eqnarray*}
\xi_{n,m}\leq\xi_{1}\left(n\right)\textrm{ y }\xi_{n,m}\leq\xi_{3}\left(m\right)\textrm{ entonces }
-\xi_{n,m}\geq-\xi_{1}\left(n\right)\textrm{ y }-\xi_{n,m}\leq-\xi_{3}\left(m\right)\\
\end{eqnarray*}
por lo tanto tenemos las desigualdades 



\begin{eqnarray*}
\begin{array}{ll}
-\tilde{\mu}_{1}\xi_{n,m}\geq-\tilde{\mu}_{1}\xi_{1}\left(n\right),&
-\tilde{\mu}_{2}\xi_{n,m}\geq-\tilde{\mu}_{2}\xi_{1}\left(n\right)
\geq-\tilde{\mu}_{2}\xi_{2}\left(n\right),\\
-\tilde{\mu}_{3}\xi_{n,m}\geq-\tilde{\mu}_{3}\xi_{3}\left(m\right),&
-\tilde{\mu}_{4}\xi_{n,m}\geq-\tilde{\mu}_{4}\xi_{3}\left(m\right)
\geq-\tilde{\mu}_{4}\xi_{4}\left(m\right).
\end{array}
\end{eqnarray*}

Sea $T^{*}\in I_{n,m}$, entonces dado que en particular $T^{*}\in I_{1}\left(n\right)$ se cumple con probabilidad positiva que no hay arribos a las colas $Q_{1}$ y $Q_{2}$, en consecuencia, tampoco hay usuarios de transferencia para $Q_{3}$ y $Q_{4}$, es decir, $\tilde{\mu}_{1}=\mu_{1}$, $\tilde{\mu}_{2}=\mu_{2}$, $\tilde{\mu}_{3}=\mu_{3}$, $\tilde{\mu}_{4}=\mu_{4}$, es decir, los eventos $Q_{1}$ y $Q_{3}$ son condicionalmente independientes en el intervalo $I_{n,m}$; lo mismo ocurre para las colas $Q_{2}$ y $Q_{4}$, por lo tanto tenemos que


\begin{eqnarray}
\begin{array}{l}
\prob\left\{A_{1}\left(T^{*}\right),A_{2}\left(T^{*}\right),
A_{3}\left(T^{*}\right),A_{4}\left(T^{*}\right)|T^{*}\in I_{n,m}\right\}
=\prod_{j=1}^{4}\prob\left\{A_{j}\left(T^{*}\right)|T^{*}\in I_{n,m}\right\}\\
\geq\prob\left\{A_{1}\left(T^{*}\right)|T^{*}\in I_{1}\left(n\right)\right\}
\prob\left\{A_{2}\left(T^{*}\right)|T^{*}\in I_{2}\left(n\right)\right\}
\prob\left\{A_{3}\left(T^{*}\right)|T^{*}\in I_{3}\left(m\right)\right\}
\prob\left\{A_{4}\left(T^{*}\right)|T^{*}\in I_{4}\left(m\right)\right\}\\
=e^{-\mu_{1}\xi_{1}\left(n\right)}
e^{-\mu_{2}\xi_{2}\left(n\right)}
e^{-\mu_{3}\xi_{3}\left(m\right)}
e^{-\mu_{4}\xi_{4}\left(m\right)}
=e^{-\left[\tilde{\mu}_{1}\xi_{1}\left(n\right)
+\tilde{\mu}_{2}\xi_{2}\left(n\right)
+\tilde{\mu}_{3}\xi_{3}\left(m\right)
+\tilde{\mu}_{4}\xi_{4}
\left(m\right)\right]}>0.
\end{array}
\end{eqnarray}
\end{proof}


Estos resultados aparecen en Daley (1968) \cite{Daley68} para $\left\{T_{n}\right\}$ intervalos de inter-arribo, $\left\{D_{n}\right\}$ intervalos de inter-salida y $\left\{S_{n}\right\}$ tiempos de servicio.

\begin{itemize}
\item Si el proceso $\left\{T_{n}\right\}$ es Poisson, el proceso $\left\{D_{n}\right\}$ es no correlacionado si y s\'olo si es un proceso Poisso, lo cual ocurre si y s\'olo si $\left\{S_{n}\right\}$ son exponenciales negativas.

\item Si $\left\{S_{n}\right\}$ son exponenciales negativas, $\left\{D_{n}\right\}$ es un proceso de renovaci\'on  si y s\'olo si es un proceso Poisson, lo cual ocurre si y s\'olo si $\left\{T_{n}\right\}$ es un proceso Poisson.

\item $\esp\left(D_{n}\right)=\esp\left(T_{n}\right)$.

\item Para un sistema de visitas $GI/M/1$ se tiene el siguiente teorema:

\begin{Teo}
En un sistema estacionario $GI/M/1$ los intervalos de interpartida tienen
\begin{eqnarray*}
\esp\left(e^{-\theta D_{n}}\right)&=&\mu\left(\mu+\theta\right)^{-1}\left[\delta\theta
-\mu\left(1-\delta\right)\alpha\left(\theta\right)\right]
\left[\theta-\mu\left(1-\delta\right)^{-1}\right]\\
\alpha\left(\theta\right)&=&\esp\left[e^{-\theta T_{0}}\right]\\
var\left(D_{n}\right)&=&var\left(T_{0}\right)-\left(\tau^{-1}-\delta^{-1}\right)
2\delta\left(\esp\left(S_{0}\right)\right)^{2}\left(1-\delta\right)^{-1}.
\end{eqnarray*}
\end{Teo}



\begin{Teo}
El proceso de salida de un sistema de colas estacionario $GI/M/1$ es un proceso de renovaci\'on si y s\'olo si el proceso de entrada es un proceso Poisson, en cuyo caso el proceso de salida es un proceso Poisson.
\end{Teo}


\begin{Teo}
Los intervalos de interpartida $\left\{D_{n}\right\}$ de un sistema $M/G/1$ estacionario son no correlacionados si y s\'olo si la distribuci\'on de los tiempos de servicio es exponencial negativa, es decir, el sistema es de tipo  $M/M/1$.

\end{Teo}



\end{itemize}


%\section{Resultados para Procesos de Salida}

En Sigman, Thorison y Wolff \cite{Sigman2} prueban que para la existencia de un una sucesi\'on infinita no decreciente de tiempos de regeneraci\'on $\tau_{1}\leq\tau_{2}\leq\cdots$ en los cuales el proceso se regenera, basta un tiempo de regeneraci\'on $R_{1}$, donde $R_{j}=\tau_{j}-\tau_{j-1}$. Para tal efecto se requiere la existencia de un espacio de probabilidad $\left(\Omega,\mathcal{F},\prob\right)$, y proceso estoc\'astico $\textit{X}=\left\{X\left(t\right):t\geq0\right\}$ con espacio de estados $\left(S,\mathcal{R}\right)$, con $\mathcal{R}$ $\sigma$-\'algebra.

\begin{Prop}
Si existe una variable aleatoria no negativa $R_{1}$ tal que $\theta_{R\footnotesize{1}}X=_{D}X$, entonces $\left(\Omega,\mathcal{F},\prob\right)$ puede extenderse para soportar una sucesi\'on estacionaria de variables aleatorias $R=\left\{R_{k}:k\geq1\right\}$, tal que para $k\geq1$,
\begin{eqnarray*}
\theta_{k}\left(X,R\right)=_{D}\left(X,R\right).
\end{eqnarray*}

Adem\'as, para $k\geq1$, $\theta_{k}R$ es condicionalmente independiente de $\left(X,R_{1},\ldots,R_{k}\right)$, dado $\theta_{\tau k}X$.

\end{Prop}


\begin{itemize}
\item Doob en 1953 demostr\'o que el estado estacionario de un proceso de partida en un sistema de espera $M/G/\infty$, es Poisson con la misma tasa que el proceso de arribos.

\item Burke en 1968, fue el primero en demostrar que el estado estacionario de un proceso de salida de una cola $M/M/s$ es un proceso Poisson.

\item Disney en 1973 obtuvo el siguiente resultado:

\begin{Teo}
Para el sistema de espera $M/G/1/L$ con disciplina FIFO, el proceso $\textbf{I}$ es un proceso de renovaci\'on si y s\'olo si el proceso denominado longitud de la cola es estacionario y se cumple cualquiera de los siguientes casos:

\begin{itemize}
\item[a)] Los tiempos de servicio son identicamente cero;
\item[b)] $L=0$, para cualquier proceso de servicio $S$;
\item[c)] $L=1$ y $G=D$;
\item[d)] $L=\infty$ y $G=M$.
\end{itemize}
En estos casos, respectivamente, las distribuciones de interpartida $P\left\{T_{n+1}-T_{n}\leq t\right\}$ son


\begin{itemize}
\item[a)] $1-e^{-\lambda t}$, $t\geq0$;
\item[b)] $1-e^{-\lambda t}*F\left(t\right)$, $t\geq0$;
\item[c)] $1-e^{-\lambda t}*\indora_{d}\left(t\right)$, $t\geq0$;
\item[d)] $1-e^{-\lambda t}*F\left(t\right)$, $t\geq0$.
\end{itemize}
\end{Teo}


\item Finch (1959) mostr\'o que para los sistemas $M/G/1/L$, con $1\leq L\leq \infty$ con distribuciones de servicio dos veces diferenciable, solamente el sistema $M/M/1/\infty$ tiene proceso de salida de renovaci\'on estacionario.

\item King (1971) demostro que un sistema de colas estacionario $M/G/1/1$ tiene sus tiempos de interpartida sucesivas $D_{n}$ y $D_{n+1}$ son independientes, si y s\'olo si, $G=D$, en cuyo caso le proceso de salida es de renovaci\'on.

\item Disney (1973) demostr\'o que el \'unico sistema estacionario $M/G/1/L$, que tiene proceso de salida de renovaci\'on  son los sistemas $M/M/1$ y $M/D/1/1$.



\item El siguiente resultado es de Disney y Koning (1985)
\begin{Teo}
En un sistema de espera $M/G/s$, el estado estacionario del proceso de salida es un proceso Poisson para cualquier distribuci\'on de los tiempos de servicio si el sistema tiene cualquiera de las siguientes cuatro propiedades.

\begin{itemize}
\item[a)] $s=\infty$
\item[b)] La disciplina de servicio es de procesador compartido.
\item[c)] La disciplina de servicio es LCFS y preemptive resume, esto se cumple para $L<\infty$
\item[d)] $G=M$.
\end{itemize}

\end{Teo}

\item El siguiente resultado es de Alamatsaz (1983)

\begin{Teo}
En cualquier sistema de colas $GI/G/1/L$ con $1\leq L<\infty$ y distribuci\'on de interarribos $A$ y distribuci\'on de los tiempos de servicio $B$, tal que $A\left(0\right)=0$, $A\left(t\right)\left(1-B\left(t\right)\right)>0$ para alguna $t>0$ y $B\left(t\right)$ para toda $t>0$, es imposible que el proceso de salida estacionario sea de renovaci\'on.
\end{Teo}

\end{itemize}

Estos resultados aparecen en Daley (1968) \cite{Daley68} para $\left\{T_{n}\right\}$ intervalos de inter-arribo, $\left\{D_{n}\right\}$ intervalos de inter-salida y $\left\{S_{n}\right\}$ tiempos de servicio.

\begin{itemize}
\item Si el proceso $\left\{T_{n}\right\}$ es Poisson, el proceso $\left\{D_{n}\right\}$ es no correlacionado si y s\'olo si es un proceso Poisso, lo cual ocurre si y s\'olo si $\left\{S_{n}\right\}$ son exponenciales negativas.

\item Si $\left\{S_{n}\right\}$ son exponenciales negativas, $\left\{D_{n}\right\}$ es un proceso de renovaci\'on  si y s\'olo si es un proceso Poisson, lo cual ocurre si y s\'olo si $\left\{T_{n}\right\}$ es un proceso Poisson.

\item $\esp\left(D_{n}\right)=\esp\left(T_{n}\right)$.

\item Para un sistema de visitas $GI/M/1$ se tiene el siguiente teorema:

\begin{Teo}
En un sistema estacionario $GI/M/1$ los intervalos de interpartida tienen
\begin{eqnarray*}
\esp\left(e^{-\theta D_{n}}\right)&=&\mu\left(\mu+\theta\right)^{-1}\left[\delta\theta
-\mu\left(1-\delta\right)\alpha\left(\theta\right)\right]
\left[\theta-\mu\left(1-\delta\right)^{-1}\right]\\
\alpha\left(\theta\right)&=&\esp\left[e^{-\theta T_{0}}\right]\\
var\left(D_{n}\right)&=&var\left(T_{0}\right)-\left(\tau^{-1}-\delta^{-1}\right)
2\delta\left(\esp\left(S_{0}\right)\right)^{2}\left(1-\delta\right)^{-1}.
\end{eqnarray*}
\end{Teo}



\begin{Teo}
El proceso de salida de un sistema de colas estacionario $GI/M/1$ es un proceso de renovaci\'on si y s\'olo si el proceso de entrada es un proceso Poisson, en cuyo caso el proceso de salida es un proceso Poisson.
\end{Teo}


\begin{Teo}
Los intervalos de interpartida $\left\{D_{n}\right\}$ de un sistema $M/G/1$ estacionario son no correlacionados si y s\'olo si la distribuci\'on de los tiempos de servicio es exponencial negativa, es decir, el sistema es de tipo  $M/M/1$.

\end{Teo}



\end{itemize}
%\newpage
%________________________________________________________________________
%\section{Redes de Sistemas de Visitas C\'iclicas}
%________________________________________________________________________

Sean $Q_{1},Q_{2},Q_{3}$ y $Q_{4}$ en una Red de Sistemas de Visitas C\'iclicas (RSVC). Supongamos que cada una de las colas es del tipo $M/M/1$ con tasa de arribo $\mu_{i}$ y que la transferencia de usuarios entre los dos sistemas ocurre entre $Q_{1}\leftrightarrow Q_{3}$ y $Q_{2}\leftrightarrow Q_{4}$ con respectiva tasa de arribo igual a la tasa de salida $\hat{\mu}_{i}=\mu_{i}$, esto se sabe por lo desarrollado en la secci\'on anterior.  

Consideremos, sin p\'erdida de generalidad como base del an\'alisis, la cola $Q_{1}$ adem\'as supongamos al servidor lo comenzamos a observar una vez que termina de atender a la misma para desplazarse y llegar a $Q_{2}$, es decir al tiempo $\tau_{2}$.

Sea $n\in\nat$, $n>0$, ciclo del servidor en que regresa a $Q_{1}$ para dar servicio y atender conforme a la pol\'itica exhaustiva, entonces se tiene que $\overline{\tau}_{1}\left(n\right)$ es el tiempo del servidor en el sistema 1 en que termina de dar servicio a todos los usuarios presentes en la cola, por lo tanto se cumple que $L_{1}\left(\overline{\tau}_{1}\left(n\right)\right)=0$, entonces el servidor para llegar a $Q_{2}$ incurre en un tiempo de traslado $r_{1}$ y por tanto se cumple que $\tau_{2}\left(n\right)=\overline{\tau}_{1}\left(n\right)+r_{1}$. Dado que los tiempos entre arribos son exponenciales se cumple que 

\begin{eqnarray*}
\prob\left\{0 \textrm{ arribos en }Q_{1}\textrm{ en el intervalo }\left[\overline{\tau}_{1}\left(n\right),\overline{\tau}_{1}\left(n\right)+r_{1}\right]\right\}=e^{-\tilde{\mu}_{1}r_{1}},\\
\prob\left\{0 \textrm{ arribos en }Q_{2}\textrm{ en el intervalo }\left[\overline{\tau}_{1}\left(n\right),\overline{\tau}_{1}\left(n\right)+r_{1}\right]\right\}=e^{-\tilde{\mu}_{2}r_{1}}.
\end{eqnarray*}

El evento que nos interesa consiste en que no haya arribos desde que el servidor abandon\'o $Q_{2}$ y regresa nuevamente para dar servicio, es decir en el intervalo de tiempo $\left[\overline{\tau}_{2}\left(n-1\right),\tau_{2}\left(n\right)\right]$. Entonces, si hacemos


\begin{eqnarray*}
\varphi_{1}\left(n\right)&\equiv&\overline{\tau}_{1}\left(n\right)+r_{1}=\overline{\tau}_{2}\left(n-1\right)+r_{1}+r_{2}+\overline{\tau}_{1}\left(n\right)-\tau_{1}\left(n\right)\\
&=&\overline{\tau}_{2}\left(n-1\right)+\overline{\tau}_{1}\left(n\right)-\tau_{1}\left(n\right)+r,,
\end{eqnarray*}

y longitud del intervalo

\begin{eqnarray*}
\xi&\equiv&\overline{\tau}_{1}\left(n\right)+r_{1}-\overline{\tau}_{2}\left(n-1\right)
=\overline{\tau}_{2}\left(n-1\right)+\overline{\tau}_{1}\left(n\right)-\tau_{1}\left(n\right)+r-\overline{\tau}_{2}\left(n-1\right)\\
&=&\overline{\tau}_{1}\left(n\right)-\tau_{1}\left(n\right)+r.
\end{eqnarray*}


Entonces, determinemos la probabilidad del evento no arribos a $Q_{2}$ en $\left[\overline{\tau}_{2}\left(n-1\right),\varphi_{1}\left(n\right)\right]$:

\begin{eqnarray}
\prob\left\{0 \textrm{ arribos en }Q_{2}\textrm{ en el intervalo }\left[\overline{\tau}_{2}\left(n-1\right),\varphi_{1}\left(n\right)\right]\right\}
=e^{-\tilde{\mu}_{2}\xi}.
\end{eqnarray}

De manera an\'aloga, tenemos que la probabilidad de no arribos a $Q_{1}$ en $\left[\overline{\tau}_{2}\left(n-1\right),\varphi_{1}\left(n\right)\right]$ esta dada por

\begin{eqnarray}
\prob\left\{0 \textrm{ arribos en }Q_{1}\textrm{ en el intervalo }\left[\overline{\tau}_{2}\left(n-1\right),\varphi_{1}\left(n\right)\right]\right\}
=e^{-\tilde{\mu}_{1}\xi},
\end{eqnarray}

\begin{eqnarray}
\prob\left\{0 \textrm{ arribos en }Q_{2}\textrm{ en el intervalo }\left[\overline{\tau}_{2}\left(n-1\right),\varphi_{1}\left(n\right)\right]\right\}
=e^{-\tilde{\mu}_{2}\xi}.
\end{eqnarray}

Por tanto 

\begin{eqnarray}
\begin{array}{l}
\prob\left\{0 \textrm{ arribos en }Q_{1}\textrm{ y }Q_{2}\textrm{ en el intervalo }\left[\overline{\tau}_{2}\left(n-1\right),\varphi_{1}\left(n\right)\right]\right\}\\
=\prob\left\{0 \textrm{ arribos en }Q_{1}\textrm{ en el intervalo }\left[\overline{\tau}_{2}\left(n-1\right),\varphi_{1}\left(n\right)\right]\right\}\\
\times
\prob\left\{0 \textrm{ arribos en }Q_{2}\textrm{ en el intervalo }\left[\overline{\tau}_{2}\left(n-1\right),\varphi_{1}\left(n\right)\right]\right\}=e^{-\tilde{\mu}_{1}\xi}e^{-\tilde{\mu}_{2}\xi}
=e^{-\tilde{\mu}\xi}.
\end{array}
\end{eqnarray}

Para el segundo sistema, consideremos nuevamente $\overline{\tau}_{1}\left(n\right)+r_{1}$, sin p\'erdida de generalidad podemos suponer que existe $m>0$ tal que $\overline{\tau}_{3}\left(m\right)<\overline{\tau}_{1}+r_{1}<\tau_{4}\left(m\right)$, entonces

\begin{eqnarray}
\prob\left\{0 \textrm{ arribos en }Q_{3}\textrm{ en el intervalo }\left[\overline{\tau}_{3}\left(m\right),\overline{\tau}_{1}\left(n\right)+r_{1}\right]\right\}
=e^{-\tilde{\mu}_{3}\xi_{3}},
\end{eqnarray}
donde 
\begin{eqnarray}
\xi_{3}=\overline{\tau}_{1}\left(n\right)+r_{1}-\overline{\tau}_{3}\left(m\right)=
\overline{\tau}_{1}\left(n\right)-\overline{\tau}_{3}\left(m\right)+r_{1},
\end{eqnarray}

mientras que para $Q_{4}$ al igual que con $Q_{2}$ escribiremos $\tau_{4}\left(m\right)$ en t\'erminos de $\overline{\tau}_{4}\left(m-1\right)$:

$\varphi_{2}\equiv\tau_{4}\left(m\right)=\overline{\tau}_{4}\left(m-1\right)+r_{4}+\overline{\tau}_{3}\left(m\right)
-\tau_{3}\left(m\right)+r_{3}=\overline{\tau}_{4}\left(m-1\right)+\overline{\tau}_{3}\left(m\right)
-\tau_{3}\left(m\right)+\hat{r}$, adem\'as,

$\xi_{2}\equiv\varphi_{2}\left(m\right)-\overline{\tau}_{1}-r_{1}=\overline{\tau}_{4}\left(m-1\right)+\overline{\tau}_{3}\left(m\right)
-\tau_{3}\left(m\right)-\overline{\tau}_{1}\left(n\right)+\hat{r}-r_{1}$. 

Entonces


\begin{eqnarray}
\prob\left\{0 \textrm{ arribos en }Q_{4}\textrm{ en el intervalo }\left[\overline{\tau}_{1}\left(n\right)+r_{1},\varphi_{2}\left(m\right)\right]\right\}
=e^{-\tilde{\mu}_{4}\xi_{2}},
\end{eqnarray}

mientras que para $Q_{3}$ se tiene que 

\begin{eqnarray}
\prob\left\{0 \textrm{ arribos en }Q_{3}\textrm{ en el intervalo }\left[\overline{\tau}_{1}\left(n\right)+r_{1},\varphi_{2}\left(m\right)\right]\right\}
=e^{-\tilde{\mu}_{3}\xi_{2}}
\end{eqnarray}

Por tanto

\begin{eqnarray}
\prob\left\{0 \textrm{ arribos en }Q_{3}\wedge Q_{4}\textrm{ en el intervalo }\left[\overline{\tau}_{1}\left(n\right)+r_{1},\varphi_{2}\left(m\right)\right]\right\}
=e^{-\hat{\mu}\xi_{2}}
\end{eqnarray}
donde $\hat{\mu}=\tilde{\mu}_{3}+\tilde{\mu}_{4}$.

Ahora, definamos los intervalos $\mathcal{I}_{1}=\left[\overline{\tau}_{1}\left(n\right)+r_{1},\varphi_{1}\left(n\right)\right]$  y $\mathcal{I}_{2}=\left[\overline{\tau}_{1}\left(n\right)+r_{1},\varphi_{2}\left(m\right)\right]$, entonces, sea $\mathcal{I}=\mathcal{I}_{1}\cap\mathcal{I}_{2}$ el intervalo donde cada una de las colas se encuentran vac\'ias, es decir, si tomamos $T^{*}\in\mathcal{I}$, entonces  $L_{1}\left(T^{*}\right)=L_{2}\left(T^{*}\right)=L_{3}\left(T^{*}\right)=L_{4}\left(T^{*}\right)=0$.

Ahora, dado que por construcci\'on $\mathcal{I}\neq\emptyset$ y que para $T^{*}\in\mathcal{I}$ en ninguna de las colas han llegado usuarios, se tiene que no hay transferencia entre las colas, por lo tanto, el sistema 1 y el sistema 2 son condicionalmente independientes en $\mathcal{I}$, es decir

\begin{eqnarray}
\prob\left\{L_{1}\left(T^{*}\right),L_{2}\left(T^{*}\right),L_{3}\left(T^{*}\right),L_{4}\left(T^{*}\right)|T^{*}\in\mathcal{I}\right\}=\prod_{j=1}^{4}\prob\left\{L_{j}\left(T^{*}\right)\right\},
\end{eqnarray}

para $T^{*}\in\mathcal{I}$. 

%\newpage























%________________________________________________________________________
%\section{Procesos Regenerativos}
%________________________________________________________________________

%________________________________________________________________________
%\subsection*{Procesos Regenerativos Sigman, Thorisson y Wolff \cite{Sigman1}}
%________________________________________________________________________


\begin{Def}[Definici\'on Cl\'asica]
Un proceso estoc\'astico $X=\left\{X\left(t\right):t\geq0\right\}$ es llamado regenerativo is existe una variable aleatoria $R_{1}>0$ tal que
\begin{itemize}
\item[i)] $\left\{X\left(t+R_{1}\right):t\geq0\right\}$ es independiente de $\left\{\left\{X\left(t\right):t<R_{1}\right\},\right\}$
\item[ii)] $\left\{X\left(t+R_{1}\right):t\geq0\right\}$ es estoc\'asticamente equivalente a $\left\{X\left(t\right):t>0\right\}$
\end{itemize}

Llamamos a $R_{1}$ tiempo de regeneraci\'on, y decimos que $X$ se regenera en este punto.
\end{Def}

$\left\{X\left(t+R_{1}\right)\right\}$ es regenerativo con tiempo de regeneraci\'on $R_{2}$, independiente de $R_{1}$ pero con la misma distribuci\'on que $R_{1}$. Procediendo de esta manera se obtiene una secuencia de variables aleatorias independientes e id\'enticamente distribuidas $\left\{R_{n}\right\}$ llamados longitudes de ciclo. Si definimos a $Z_{k}\equiv R_{1}+R_{2}+\cdots+R_{k}$, se tiene un proceso de renovaci\'on llamado proceso de renovaci\'on encajado para $X$.


\begin{Note}
La existencia de un primer tiempo de regeneraci\'on, $R_{1}$, implica la existencia de una sucesi\'on completa de estos tiempos $R_{1},R_{2}\ldots,$ que satisfacen la propiedad deseada \cite{Sigman2}.
\end{Note}


\begin{Note} Para la cola $GI/GI/1$ los usuarios arriban con tiempos $t_{n}$ y son atendidos con tiempos de servicio $S_{n}$, los tiempos de arribo forman un proceso de renovaci\'on  con tiempos entre arribos independientes e identicamente distribuidos (\texttt{i.i.d.})$T_{n}=t_{n}-t_{n-1}$, adem\'as los tiempos de servicio son \texttt{i.i.d.} e independientes de los procesos de arribo. Por \textit{estable} se entiende que $\esp S_{n}<\esp T_{n}<\infty$.
\end{Note}
 


\begin{Def}
Para $x$ fijo y para cada $t\geq0$, sea $I_{x}\left(t\right)=1$ si $X\left(t\right)\leq x$,  $I_{x}\left(t\right)=0$ en caso contrario, y def\'inanse los tiempos promedio
\begin{eqnarray*}
\overline{X}&=&lim_{t\rightarrow\infty}\frac{1}{t}\int_{0}^{\infty}X\left(u\right)du\\
\prob\left(X_{\infty}\leq x\right)&=&lim_{t\rightarrow\infty}\frac{1}{t}\int_{0}^{\infty}I_{x}\left(u\right)du,
\end{eqnarray*}
cuando estos l\'imites existan.
\end{Def}

Como consecuencia del teorema de Renovaci\'on-Recompensa, se tiene que el primer l\'imite  existe y es igual a la constante
\begin{eqnarray*}
\overline{X}&=&\frac{\esp\left[\int_{0}^{R_{1}}X\left(t\right)dt\right]}{\esp\left[R_{1}\right]},
\end{eqnarray*}
suponiendo que ambas esperanzas son finitas.
 
\begin{Note}
Funciones de procesos regenerativos son regenerativas, es decir, si $X\left(t\right)$ es regenerativo y se define el proceso $Y\left(t\right)$ por $Y\left(t\right)=f\left(X\left(t\right)\right)$ para alguna funci\'on Borel medible $f\left(\cdot\right)$. Adem\'as $Y$ es regenerativo con los mismos tiempos de renovaci\'on que $X$. 

En general, los tiempos de renovaci\'on, $Z_{k}$ de un proceso regenerativo no requieren ser tiempos de paro con respecto a la evoluci\'on de $X\left(t\right)$.
\end{Note} 

\begin{Note}
Una funci\'on de un proceso de Markov, usualmente no ser\'a un proceso de Markov, sin embargo ser\'a regenerativo si el proceso de Markov lo es.
\end{Note}

 
\begin{Note}
Un proceso regenerativo con media de la longitud de ciclo finita es llamado positivo recurrente.
\end{Note}


\begin{Note}
\begin{itemize}
\item[a)] Si el proceso regenerativo $X$ es positivo recurrente y tiene trayectorias muestrales no negativas, entonces la ecuaci\'on anterior es v\'alida.
\item[b)] Si $X$ es positivo recurrente regenerativo, podemos construir una \'unica versi\'on estacionaria de este proceso, $X_{e}=\left\{X_{e}\left(t\right)\right\}$, donde $X_{e}$ es un proceso estoc\'astico regenerativo y estrictamente estacionario, con distribuci\'on marginal distribuida como $X_{\infty}$
\end{itemize}
\end{Note}


%__________________________________________________________________________________________
%\subsection*{Procesos Regenerativos Estacionarios - Stidham \cite{Stidham}}
%__________________________________________________________________________________________


Un proceso estoc\'astico a tiempo continuo $\left\{V\left(t\right),t\geq0\right\}$ es un proceso regenerativo si existe una sucesi\'on de variables aleatorias independientes e id\'enticamente distribuidas $\left\{X_{1},X_{2},\ldots\right\}$, sucesi\'on de renovaci\'on, tal que para cualquier conjunto de Borel $A$, 

\begin{eqnarray*}
\prob\left\{V\left(t\right)\in A|X_{1}+X_{2}+\cdots+X_{R\left(t\right)}=s,\left\{V\left(\tau\right),\tau<s\right\}\right\}=\prob\left\{V\left(t-s\right)\in A|X_{1}>t-s\right\},
\end{eqnarray*}
para todo $0\leq s\leq t$, donde $R\left(t\right)=\max\left\{X_{1}+X_{2}+\cdots+X_{j}\leq t\right\}=$n\'umero de renovaciones ({\emph{puntos de regeneraci\'on}}) que ocurren en $\left[0,t\right]$. El intervalo $\left[0,X_{1}\right)$ es llamado {\emph{primer ciclo de regeneraci\'on}} de $\left\{V\left(t \right),t\geq0\right\}$, $\left[X_{1},X_{1}+X_{2}\right)$ el {\emph{segundo ciclo de regeneraci\'on}}, y as\'i sucesivamente.

Sea $X=X_{1}$ y sea $F$ la funci\'on de distrbuci\'on de $X$


\begin{Def}
Se define el proceso estacionario, $\left\{V^{*}\left(t\right),t\geq0\right\}$, para $\left\{V\left(t\right),t\geq0\right\}$ por

\begin{eqnarray*}
\prob\left\{V\left(t\right)\in A\right\}=\frac{1}{\esp\left[X\right]}\int_{0}^{\infty}\prob\left\{V\left(t+x\right)\in A|X>x\right\}\left(1-F\left(x\right)\right)dx,
\end{eqnarray*} 
para todo $t\geq0$ y todo conjunto de Borel $A$.
\end{Def}

\begin{Def}
Una distribuci\'on se dice que es {\emph{aritm\'etica}} si todos sus puntos de incremento son m\'ultiplos de la forma $0,\lambda, 2\lambda,\ldots$ para alguna $\lambda>0$ entera.
\end{Def}


\begin{Def}
Una modificaci\'on medible de un proceso $\left\{V\left(t\right),t\geq0\right\}$, es una versi\'on de este, $\left\{V\left(t,w\right)\right\}$ conjuntamente medible para $t\geq0$ y para $w\in S$, $S$ espacio de estados para $\left\{V\left(t\right),t\geq0\right\}$.
\end{Def}

\begin{Teo}
Sea $\left\{V\left(t\right),t\geq\right\}$ un proceso regenerativo no negativo con modificaci\'on medible. Sea $\esp\left[X\right]<\infty$. Entonces el proceso estacionario dado por la ecuaci\'on anterior est\'a bien definido y tiene funci\'on de distribuci\'on independiente de $t$, adem\'as
\begin{itemize}
\item[i)] \begin{eqnarray*}
\esp\left[V^{*}\left(0\right)\right]&=&\frac{\esp\left[\int_{0}^{X}V\left(s\right)ds\right]}{\esp\left[X\right]}\end{eqnarray*}
\item[ii)] Si $\esp\left[V^{*}\left(0\right)\right]<\infty$, equivalentemente, si $\esp\left[\int_{0}^{X}V\left(s\right)ds\right]<\infty$,entonces
\begin{eqnarray*}
\frac{\int_{0}^{t}V\left(s\right)ds}{t}\rightarrow\frac{\esp\left[\int_{0}^{X}V\left(s\right)ds\right]}{\esp\left[X\right]}
\end{eqnarray*}
con probabilidad 1 y en media, cuando $t\rightarrow\infty$.
\end{itemize}
\end{Teo}

\begin{Coro}
Sea $\left\{V\left(t\right),t\geq0\right\}$ un proceso regenerativo no negativo, con modificaci\'on medible. Si $\esp <\infty$, $F$ es no-aritm\'etica, y para todo $x\geq0$, $P\left\{V\left(t\right)\leq x,C>x\right\}$ es de variaci\'on acotada como funci\'on de $t$ en cada intervalo finito $\left[0,\tau\right]$, entonces $V\left(t\right)$ converge en distribuci\'on  cuando $t\rightarrow\infty$ y $$\esp V=\frac{\esp \int_{0}^{X}V\left(s\right)ds}{\esp X}$$
Donde $V$ tiene la distribuci\'on l\'imite de $V\left(t\right)$ cuando $t\rightarrow\infty$.

\end{Coro}

Para el caso discreto se tienen resultados similares.



%______________________________________________________________________
%\section{Procesos de Renovaci\'on}
%______________________________________________________________________

\begin{Def}\label{Def.Tn}
Sean $0\leq T_{1}\leq T_{2}\leq \ldots$ son tiempos aleatorios infinitos en los cuales ocurren ciertos eventos. El n\'umero de tiempos $T_{n}$ en el intervalo $\left[0,t\right)$ es

\begin{eqnarray}
N\left(t\right)=\sum_{n=1}^{\infty}\indora\left(T_{n}\leq t\right),
\end{eqnarray}
para $t\geq0$.
\end{Def}

Si se consideran los puntos $T_{n}$ como elementos de $\rea_{+}$, y $N\left(t\right)$ es el n\'umero de puntos en $\rea$. El proceso denotado por $\left\{N\left(t\right):t\geq0\right\}$, denotado por $N\left(t\right)$, es un proceso puntual en $\rea_{+}$. Los $T_{n}$ son los tiempos de ocurrencia, el proceso puntual $N\left(t\right)$ es simple si su n\'umero de ocurrencias son distintas: $0<T_{1}<T_{2}<\ldots$ casi seguramente.

\begin{Def}
Un proceso puntual $N\left(t\right)$ es un proceso de renovaci\'on si los tiempos de interocurrencia $\xi_{n}=T_{n}-T_{n-1}$, para $n\geq1$, son independientes e identicamente distribuidos con distribuci\'on $F$, donde $F\left(0\right)=0$ y $T_{0}=0$. Los $T_{n}$ son llamados tiempos de renovaci\'on, referente a la independencia o renovaci\'on de la informaci\'on estoc\'astica en estos tiempos. Los $\xi_{n}$ son los tiempos de inter-renovaci\'on, y $N\left(t\right)$ es el n\'umero de renovaciones en el intervalo $\left[0,t\right)$
\end{Def}


\begin{Note}
Para definir un proceso de renovaci\'on para cualquier contexto, solamente hay que especificar una distribuci\'on $F$, con $F\left(0\right)=0$, para los tiempos de inter-renovaci\'on. La funci\'on $F$ en turno degune las otra variables aleatorias. De manera formal, existe un espacio de probabilidad y una sucesi\'on de variables aleatorias $\xi_{1},\xi_{2},\ldots$ definidas en este con distribuci\'on $F$. Entonces las otras cantidades son $T_{n}=\sum_{k=1}^{n}\xi_{k}$ y $N\left(t\right)=\sum_{n=1}^{\infty}\indora\left(T_{n}\leq t\right)$, donde $T_{n}\rightarrow\infty$ casi seguramente por la Ley Fuerte de los Grandes Números.
\end{Note}

%___________________________________________________________________________________________
%
%\subsection*{Teorema Principal de Renovaci\'on}
%___________________________________________________________________________________________
%

\begin{Note} Una funci\'on $h:\rea_{+}\rightarrow\rea$ es Directamente Riemann Integrable en los siguientes casos:
\begin{itemize}
\item[a)] $h\left(t\right)\geq0$ es decreciente y Riemann Integrable.
\item[b)] $h$ es continua excepto posiblemente en un conjunto de Lebesgue de medida 0, y $|h\left(t\right)|\leq b\left(t\right)$, donde $b$ es DRI.
\end{itemize}
\end{Note}

\begin{Teo}[Teorema Principal de Renovaci\'on]
Si $F$ es no aritm\'etica y $h\left(t\right)$ es Directamente Riemann Integrable (DRI), entonces

\begin{eqnarray*}
lim_{t\rightarrow\infty}U\star h=\frac{1}{\mu}\int_{\rea_{+}}h\left(s\right)ds.
\end{eqnarray*}
\end{Teo}

\begin{Prop}
Cualquier funci\'on $H\left(t\right)$ acotada en intervalos finitos y que es 0 para $t<0$ puede expresarse como
\begin{eqnarray*}
H\left(t\right)=U\star h\left(t\right)\textrm{,  donde }h\left(t\right)=H\left(t\right)-F\star H\left(t\right)
\end{eqnarray*}
\end{Prop}

\begin{Def}
Un proceso estoc\'astico $X\left(t\right)$ es crudamente regenerativo en un tiempo aleatorio positivo $T$ si
\begin{eqnarray*}
\esp\left[X\left(T+t\right)|T\right]=\esp\left[X\left(t\right)\right]\textrm{, para }t\geq0,\end{eqnarray*}
y con las esperanzas anteriores finitas.
\end{Def}

\begin{Prop}
Sup\'ongase que $X\left(t\right)$ es un proceso crudamente regenerativo en $T$, que tiene distribuci\'on $F$. Si $\esp\left[X\left(t\right)\right]$ es acotado en intervalos finitos, entonces
\begin{eqnarray*}
\esp\left[X\left(t\right)\right]=U\star h\left(t\right)\textrm{,  donde }h\left(t\right)=\esp\left[X\left(t\right)\indora\left(T>t\right)\right].
\end{eqnarray*}
\end{Prop}

\begin{Teo}[Regeneraci\'on Cruda]
Sup\'ongase que $X\left(t\right)$ es un proceso con valores positivo crudamente regenerativo en $T$, y def\'inase $M=\sup\left\{|X\left(t\right)|:t\leq T\right\}$. Si $T$ es no aritm\'etico y $M$ y $MT$ tienen media finita, entonces
\begin{eqnarray*}
lim_{t\rightarrow\infty}\esp\left[X\left(t\right)\right]=\frac{1}{\mu}\int_{\rea_{+}}h\left(s\right)ds,
\end{eqnarray*}
donde $h\left(t\right)=\esp\left[X\left(t\right)\indora\left(T>t\right)\right]$.
\end{Teo}

%___________________________________________________________________________________________
%
%\subsection*{Propiedades de los Procesos de Renovaci\'on}
%___________________________________________________________________________________________
%

Los tiempos $T_{n}$ est\'an relacionados con los conteos de $N\left(t\right)$ por

\begin{eqnarray*}
\left\{N\left(t\right)\geq n\right\}&=&\left\{T_{n}\leq t\right\}\\
T_{N\left(t\right)}\leq &t&<T_{N\left(t\right)+1},
\end{eqnarray*}

adem\'as $N\left(T_{n}\right)=n$, y 

\begin{eqnarray*}
N\left(t\right)=\max\left\{n:T_{n}\leq t\right\}=\min\left\{n:T_{n+1}>t\right\}
\end{eqnarray*}

Por propiedades de la convoluci\'on se sabe que

\begin{eqnarray*}
P\left\{T_{n}\leq t\right\}=F^{n\star}\left(t\right)
\end{eqnarray*}
que es la $n$-\'esima convoluci\'on de $F$. Entonces 

\begin{eqnarray*}
\left\{N\left(t\right)\geq n\right\}&=&\left\{T_{n}\leq t\right\}\\
P\left\{N\left(t\right)\leq n\right\}&=&1-F^{\left(n+1\right)\star}\left(t\right)
\end{eqnarray*}

Adem\'as usando el hecho de que $\esp\left[N\left(t\right)\right]=\sum_{n=1}^{\infty}P\left\{N\left(t\right)\geq n\right\}$
se tiene que

\begin{eqnarray*}
\esp\left[N\left(t\right)\right]=\sum_{n=1}^{\infty}F^{n\star}\left(t\right)
\end{eqnarray*}

\begin{Prop}
Para cada $t\geq0$, la funci\'on generadora de momentos $\esp\left[e^{\alpha N\left(t\right)}\right]$ existe para alguna $\alpha$ en una vecindad del 0, y de aqu\'i que $\esp\left[N\left(t\right)^{m}\right]<\infty$, para $m\geq1$.
\end{Prop}


\begin{Note}
Si el primer tiempo de renovaci\'on $\xi_{1}$ no tiene la misma distribuci\'on que el resto de las $\xi_{n}$, para $n\geq2$, a $N\left(t\right)$ se le llama Proceso de Renovaci\'on retardado, donde si $\xi$ tiene distribuci\'on $G$, entonces el tiempo $T_{n}$ de la $n$-\'esima renovaci\'on tiene distribuci\'on $G\star F^{\left(n-1\right)\star}\left(t\right)$
\end{Note}


\begin{Teo}
Para una constante $\mu\leq\infty$ ( o variable aleatoria), las siguientes expresiones son equivalentes:

\begin{eqnarray}
lim_{n\rightarrow\infty}n^{-1}T_{n}&=&\mu,\textrm{ c.s.}\\
lim_{t\rightarrow\infty}t^{-1}N\left(t\right)&=&1/\mu,\textrm{ c.s.}
\end{eqnarray}
\end{Teo}


Es decir, $T_{n}$ satisface la Ley Fuerte de los Grandes N\'umeros s\'i y s\'olo s\'i $N\left/t\right)$ la cumple.


\begin{Coro}[Ley Fuerte de los Grandes N\'umeros para Procesos de Renovaci\'on]
Si $N\left(t\right)$ es un proceso de renovaci\'on cuyos tiempos de inter-renovaci\'on tienen media $\mu\leq\infty$, entonces
\begin{eqnarray}
t^{-1}N\left(t\right)\rightarrow 1/\mu,\textrm{ c.s. cuando }t\rightarrow\infty.
\end{eqnarray}

\end{Coro}


Considerar el proceso estoc\'astico de valores reales $\left\{Z\left(t\right):t\geq0\right\}$ en el mismo espacio de probabilidad que $N\left(t\right)$

\begin{Def}
Para el proceso $\left\{Z\left(t\right):t\geq0\right\}$ se define la fluctuaci\'on m\'axima de $Z\left(t\right)$ en el intervalo $\left(T_{n-1},T_{n}\right]$:
\begin{eqnarray*}
M_{n}=\sup_{T_{n-1}<t\leq T_{n}}|Z\left(t\right)-Z\left(T_{n-1}\right)|
\end{eqnarray*}
\end{Def}

\begin{Teo}
Sup\'ongase que $n^{-1}T_{n}\rightarrow\mu$ c.s. cuando $n\rightarrow\infty$, donde $\mu\leq\infty$ es una constante o variable aleatoria. Sea $a$ una constante o variable aleatoria que puede ser infinita cuando $\mu$ es finita, y considere las expresiones l\'imite:
\begin{eqnarray}
lim_{n\rightarrow\infty}n^{-1}Z\left(T_{n}\right)&=&a,\textrm{ c.s.}\\
lim_{t\rightarrow\infty}t^{-1}Z\left(t\right)&=&a/\mu,\textrm{ c.s.}
\end{eqnarray}
La segunda expresi\'on implica la primera. Conversamente, la primera implica la segunda si el proceso $Z\left(t\right)$ es creciente, o si $lim_{n\rightarrow\infty}n^{-1}M_{n}=0$ c.s.
\end{Teo}

\begin{Coro}
Si $N\left(t\right)$ es un proceso de renovaci\'on, y $\left(Z\left(T_{n}\right)-Z\left(T_{n-1}\right),M_{n}\right)$, para $n\geq1$, son variables aleatorias independientes e id\'enticamente distribuidas con media finita, entonces,
\begin{eqnarray}
lim_{t\rightarrow\infty}t^{-1}Z\left(t\right)\rightarrow\frac{\esp\left[Z\left(T_{1}\right)-Z\left(T_{0}\right)\right]}{\esp\left[T_{1}\right]},\textrm{ c.s. cuando  }t\rightarrow\infty.
\end{eqnarray}
\end{Coro}



%___________________________________________________________________________________________
%
%\subsection{Propiedades de los Procesos de Renovaci\'on}
%___________________________________________________________________________________________
%

Los tiempos $T_{n}$ est\'an relacionados con los conteos de $N\left(t\right)$ por

\begin{eqnarray*}
\left\{N\left(t\right)\geq n\right\}&=&\left\{T_{n}\leq t\right\}\\
T_{N\left(t\right)}\leq &t&<T_{N\left(t\right)+1},
\end{eqnarray*}

adem\'as $N\left(T_{n}\right)=n$, y 

\begin{eqnarray*}
N\left(t\right)=\max\left\{n:T_{n}\leq t\right\}=\min\left\{n:T_{n+1}>t\right\}
\end{eqnarray*}

Por propiedades de la convoluci\'on se sabe que

\begin{eqnarray*}
P\left\{T_{n}\leq t\right\}=F^{n\star}\left(t\right)
\end{eqnarray*}
que es la $n$-\'esima convoluci\'on de $F$. Entonces 

\begin{eqnarray*}
\left\{N\left(t\right)\geq n\right\}&=&\left\{T_{n}\leq t\right\}\\
P\left\{N\left(t\right)\leq n\right\}&=&1-F^{\left(n+1\right)\star}\left(t\right)
\end{eqnarray*}

Adem\'as usando el hecho de que $\esp\left[N\left(t\right)\right]=\sum_{n=1}^{\infty}P\left\{N\left(t\right)\geq n\right\}$
se tiene que

\begin{eqnarray*}
\esp\left[N\left(t\right)\right]=\sum_{n=1}^{\infty}F^{n\star}\left(t\right)
\end{eqnarray*}

\begin{Prop}
Para cada $t\geq0$, la funci\'on generadora de momentos $\esp\left[e^{\alpha N\left(t\right)}\right]$ existe para alguna $\alpha$ en una vecindad del 0, y de aqu\'i que $\esp\left[N\left(t\right)^{m}\right]<\infty$, para $m\geq1$.
\end{Prop}


\begin{Note}
Si el primer tiempo de renovaci\'on $\xi_{1}$ no tiene la misma distribuci\'on que el resto de las $\xi_{n}$, para $n\geq2$, a $N\left(t\right)$ se le llama Proceso de Renovaci\'on retardado, donde si $\xi$ tiene distribuci\'on $G$, entonces el tiempo $T_{n}$ de la $n$-\'esima renovaci\'on tiene distribuci\'on $G\star F^{\left(n-1\right)\star}\left(t\right)$
\end{Note}


\begin{Teo}
Para una constante $\mu\leq\infty$ ( o variable aleatoria), las siguientes expresiones son equivalentes:

\begin{eqnarray}
lim_{n\rightarrow\infty}n^{-1}T_{n}&=&\mu,\textrm{ c.s.}\\
lim_{t\rightarrow\infty}t^{-1}N\left(t\right)&=&1/\mu,\textrm{ c.s.}
\end{eqnarray}
\end{Teo}


Es decir, $T_{n}$ satisface la Ley Fuerte de los Grandes N\'umeros s\'i y s\'olo s\'i $N\left/t\right)$ la cumple.


\begin{Coro}[Ley Fuerte de los Grandes N\'umeros para Procesos de Renovaci\'on]
Si $N\left(t\right)$ es un proceso de renovaci\'on cuyos tiempos de inter-renovaci\'on tienen media $\mu\leq\infty$, entonces
\begin{eqnarray}
t^{-1}N\left(t\right)\rightarrow 1/\mu,\textrm{ c.s. cuando }t\rightarrow\infty.
\end{eqnarray}

\end{Coro}


Considerar el proceso estoc\'astico de valores reales $\left\{Z\left(t\right):t\geq0\right\}$ en el mismo espacio de probabilidad que $N\left(t\right)$

\begin{Def}
Para el proceso $\left\{Z\left(t\right):t\geq0\right\}$ se define la fluctuaci\'on m\'axima de $Z\left(t\right)$ en el intervalo $\left(T_{n-1},T_{n}\right]$:
\begin{eqnarray*}
M_{n}=\sup_{T_{n-1}<t\leq T_{n}}|Z\left(t\right)-Z\left(T_{n-1}\right)|
\end{eqnarray*}
\end{Def}

\begin{Teo}
Sup\'ongase que $n^{-1}T_{n}\rightarrow\mu$ c.s. cuando $n\rightarrow\infty$, donde $\mu\leq\infty$ es una constante o variable aleatoria. Sea $a$ una constante o variable aleatoria que puede ser infinita cuando $\mu$ es finita, y considere las expresiones l\'imite:
\begin{eqnarray}
lim_{n\rightarrow\infty}n^{-1}Z\left(T_{n}\right)&=&a,\textrm{ c.s.}\\
lim_{t\rightarrow\infty}t^{-1}Z\left(t\right)&=&a/\mu,\textrm{ c.s.}
\end{eqnarray}
La segunda expresi\'on implica la primera. Conversamente, la primera implica la segunda si el proceso $Z\left(t\right)$ es creciente, o si $lim_{n\rightarrow\infty}n^{-1}M_{n}=0$ c.s.
\end{Teo}

\begin{Coro}
Si $N\left(t\right)$ es un proceso de renovaci\'on, y $\left(Z\left(T_{n}\right)-Z\left(T_{n-1}\right),M_{n}\right)$, para $n\geq1$, son variables aleatorias independientes e id\'enticamente distribuidas con media finita, entonces,
\begin{eqnarray}
lim_{t\rightarrow\infty}t^{-1}Z\left(t\right)\rightarrow\frac{\esp\left[Z\left(T_{1}\right)-Z\left(T_{0}\right)\right]}{\esp\left[T_{1}\right]},\textrm{ c.s. cuando  }t\rightarrow\infty.
\end{eqnarray}
\end{Coro}


%___________________________________________________________________________________________
%
%\subsection{Propiedades de los Procesos de Renovaci\'on}
%___________________________________________________________________________________________
%

Los tiempos $T_{n}$ est\'an relacionados con los conteos de $N\left(t\right)$ por

\begin{eqnarray*}
\left\{N\left(t\right)\geq n\right\}&=&\left\{T_{n}\leq t\right\}\\
T_{N\left(t\right)}\leq &t&<T_{N\left(t\right)+1},
\end{eqnarray*}

adem\'as $N\left(T_{n}\right)=n$, y 

\begin{eqnarray*}
N\left(t\right)=\max\left\{n:T_{n}\leq t\right\}=\min\left\{n:T_{n+1}>t\right\}
\end{eqnarray*}

Por propiedades de la convoluci\'on se sabe que

\begin{eqnarray*}
P\left\{T_{n}\leq t\right\}=F^{n\star}\left(t\right)
\end{eqnarray*}
que es la $n$-\'esima convoluci\'on de $F$. Entonces 

\begin{eqnarray*}
\left\{N\left(t\right)\geq n\right\}&=&\left\{T_{n}\leq t\right\}\\
P\left\{N\left(t\right)\leq n\right\}&=&1-F^{\left(n+1\right)\star}\left(t\right)
\end{eqnarray*}

Adem\'as usando el hecho de que $\esp\left[N\left(t\right)\right]=\sum_{n=1}^{\infty}P\left\{N\left(t\right)\geq n\right\}$
se tiene que

\begin{eqnarray*}
\esp\left[N\left(t\right)\right]=\sum_{n=1}^{\infty}F^{n\star}\left(t\right)
\end{eqnarray*}

\begin{Prop}
Para cada $t\geq0$, la funci\'on generadora de momentos $\esp\left[e^{\alpha N\left(t\right)}\right]$ existe para alguna $\alpha$ en una vecindad del 0, y de aqu\'i que $\esp\left[N\left(t\right)^{m}\right]<\infty$, para $m\geq1$.
\end{Prop}


\begin{Note}
Si el primer tiempo de renovaci\'on $\xi_{1}$ no tiene la misma distribuci\'on que el resto de las $\xi_{n}$, para $n\geq2$, a $N\left(t\right)$ se le llama Proceso de Renovaci\'on retardado, donde si $\xi$ tiene distribuci\'on $G$, entonces el tiempo $T_{n}$ de la $n$-\'esima renovaci\'on tiene distribuci\'on $G\star F^{\left(n-1\right)\star}\left(t\right)$
\end{Note}


\begin{Teo}
Para una constante $\mu\leq\infty$ ( o variable aleatoria), las siguientes expresiones son equivalentes:

\begin{eqnarray}
lim_{n\rightarrow\infty}n^{-1}T_{n}&=&\mu,\textrm{ c.s.}\\
lim_{t\rightarrow\infty}t^{-1}N\left(t\right)&=&1/\mu,\textrm{ c.s.}
\end{eqnarray}
\end{Teo}


Es decir, $T_{n}$ satisface la Ley Fuerte de los Grandes N\'umeros s\'i y s\'olo s\'i $N\left/t\right)$ la cumple.


\begin{Coro}[Ley Fuerte de los Grandes N\'umeros para Procesos de Renovaci\'on]
Si $N\left(t\right)$ es un proceso de renovaci\'on cuyos tiempos de inter-renovaci\'on tienen media $\mu\leq\infty$, entonces
\begin{eqnarray}
t^{-1}N\left(t\right)\rightarrow 1/\mu,\textrm{ c.s. cuando }t\rightarrow\infty.
\end{eqnarray}

\end{Coro}


Considerar el proceso estoc\'astico de valores reales $\left\{Z\left(t\right):t\geq0\right\}$ en el mismo espacio de probabilidad que $N\left(t\right)$

\begin{Def}
Para el proceso $\left\{Z\left(t\right):t\geq0\right\}$ se define la fluctuaci\'on m\'axima de $Z\left(t\right)$ en el intervalo $\left(T_{n-1},T_{n}\right]$:
\begin{eqnarray*}
M_{n}=\sup_{T_{n-1}<t\leq T_{n}}|Z\left(t\right)-Z\left(T_{n-1}\right)|
\end{eqnarray*}
\end{Def}

\begin{Teo}
Sup\'ongase que $n^{-1}T_{n}\rightarrow\mu$ c.s. cuando $n\rightarrow\infty$, donde $\mu\leq\infty$ es una constante o variable aleatoria. Sea $a$ una constante o variable aleatoria que puede ser infinita cuando $\mu$ es finita, y considere las expresiones l\'imite:
\begin{eqnarray}
lim_{n\rightarrow\infty}n^{-1}Z\left(T_{n}\right)&=&a,\textrm{ c.s.}\\
lim_{t\rightarrow\infty}t^{-1}Z\left(t\right)&=&a/\mu,\textrm{ c.s.}
\end{eqnarray}
La segunda expresi\'on implica la primera. Conversamente, la primera implica la segunda si el proceso $Z\left(t\right)$ es creciente, o si $lim_{n\rightarrow\infty}n^{-1}M_{n}=0$ c.s.
\end{Teo}

\begin{Coro}
Si $N\left(t\right)$ es un proceso de renovaci\'on, y $\left(Z\left(T_{n}\right)-Z\left(T_{n-1}\right),M_{n}\right)$, para $n\geq1$, son variables aleatorias independientes e id\'enticamente distribuidas con media finita, entonces,
\begin{eqnarray}
lim_{t\rightarrow\infty}t^{-1}Z\left(t\right)\rightarrow\frac{\esp\left[Z\left(T_{1}\right)-Z\left(T_{0}\right)\right]}{\esp\left[T_{1}\right]},\textrm{ c.s. cuando  }t\rightarrow\infty.
\end{eqnarray}
\end{Coro}

%___________________________________________________________________________________________
%
%\subsection{Propiedades de los Procesos de Renovaci\'on}
%___________________________________________________________________________________________
%

Los tiempos $T_{n}$ est\'an relacionados con los conteos de $N\left(t\right)$ por

\begin{eqnarray*}
\left\{N\left(t\right)\geq n\right\}&=&\left\{T_{n}\leq t\right\}\\
T_{N\left(t\right)}\leq &t&<T_{N\left(t\right)+1},
\end{eqnarray*}

adem\'as $N\left(T_{n}\right)=n$, y 

\begin{eqnarray*}
N\left(t\right)=\max\left\{n:T_{n}\leq t\right\}=\min\left\{n:T_{n+1}>t\right\}
\end{eqnarray*}

Por propiedades de la convoluci\'on se sabe que

\begin{eqnarray*}
P\left\{T_{n}\leq t\right\}=F^{n\star}\left(t\right)
\end{eqnarray*}
que es la $n$-\'esima convoluci\'on de $F$. Entonces 

\begin{eqnarray*}
\left\{N\left(t\right)\geq n\right\}&=&\left\{T_{n}\leq t\right\}\\
P\left\{N\left(t\right)\leq n\right\}&=&1-F^{\left(n+1\right)\star}\left(t\right)
\end{eqnarray*}

Adem\'as usando el hecho de que $\esp\left[N\left(t\right)\right]=\sum_{n=1}^{\infty}P\left\{N\left(t\right)\geq n\right\}$
se tiene que

\begin{eqnarray*}
\esp\left[N\left(t\right)\right]=\sum_{n=1}^{\infty}F^{n\star}\left(t\right)
\end{eqnarray*}

\begin{Prop}
Para cada $t\geq0$, la funci\'on generadora de momentos $\esp\left[e^{\alpha N\left(t\right)}\right]$ existe para alguna $\alpha$ en una vecindad del 0, y de aqu\'i que $\esp\left[N\left(t\right)^{m}\right]<\infty$, para $m\geq1$.
\end{Prop}


\begin{Note}
Si el primer tiempo de renovaci\'on $\xi_{1}$ no tiene la misma distribuci\'on que el resto de las $\xi_{n}$, para $n\geq2$, a $N\left(t\right)$ se le llama Proceso de Renovaci\'on retardado, donde si $\xi$ tiene distribuci\'on $G$, entonces el tiempo $T_{n}$ de la $n$-\'esima renovaci\'on tiene distribuci\'on $G\star F^{\left(n-1\right)\star}\left(t\right)$
\end{Note}


\begin{Teo}
Para una constante $\mu\leq\infty$ ( o variable aleatoria), las siguientes expresiones son equivalentes:

\begin{eqnarray}
lim_{n\rightarrow\infty}n^{-1}T_{n}&=&\mu,\textrm{ c.s.}\\
lim_{t\rightarrow\infty}t^{-1}N\left(t\right)&=&1/\mu,\textrm{ c.s.}
\end{eqnarray}
\end{Teo}


Es decir, $T_{n}$ satisface la Ley Fuerte de los Grandes N\'umeros s\'i y s\'olo s\'i $N\left/t\right)$ la cumple.


\begin{Coro}[Ley Fuerte de los Grandes N\'umeros para Procesos de Renovaci\'on]
Si $N\left(t\right)$ es un proceso de renovaci\'on cuyos tiempos de inter-renovaci\'on tienen media $\mu\leq\infty$, entonces
\begin{eqnarray}
t^{-1}N\left(t\right)\rightarrow 1/\mu,\textrm{ c.s. cuando }t\rightarrow\infty.
\end{eqnarray}

\end{Coro}


Considerar el proceso estoc\'astico de valores reales $\left\{Z\left(t\right):t\geq0\right\}$ en el mismo espacio de probabilidad que $N\left(t\right)$

\begin{Def}
Para el proceso $\left\{Z\left(t\right):t\geq0\right\}$ se define la fluctuaci\'on m\'axima de $Z\left(t\right)$ en el intervalo $\left(T_{n-1},T_{n}\right]$:
\begin{eqnarray*}
M_{n}=\sup_{T_{n-1}<t\leq T_{n}}|Z\left(t\right)-Z\left(T_{n-1}\right)|
\end{eqnarray*}
\end{Def}

\begin{Teo}
Sup\'ongase que $n^{-1}T_{n}\rightarrow\mu$ c.s. cuando $n\rightarrow\infty$, donde $\mu\leq\infty$ es una constante o variable aleatoria. Sea $a$ una constante o variable aleatoria que puede ser infinita cuando $\mu$ es finita, y considere las expresiones l\'imite:
\begin{eqnarray}
lim_{n\rightarrow\infty}n^{-1}Z\left(T_{n}\right)&=&a,\textrm{ c.s.}\\
lim_{t\rightarrow\infty}t^{-1}Z\left(t\right)&=&a/\mu,\textrm{ c.s.}
\end{eqnarray}
La segunda expresi\'on implica la primera. Conversamente, la primera implica la segunda si el proceso $Z\left(t\right)$ es creciente, o si $lim_{n\rightarrow\infty}n^{-1}M_{n}=0$ c.s.
\end{Teo}

\begin{Coro}
Si $N\left(t\right)$ es un proceso de renovaci\'on, y $\left(Z\left(T_{n}\right)-Z\left(T_{n-1}\right),M_{n}\right)$, para $n\geq1$, son variables aleatorias independientes e id\'enticamente distribuidas con media finita, entonces,
\begin{eqnarray}
lim_{t\rightarrow\infty}t^{-1}Z\left(t\right)\rightarrow\frac{\esp\left[Z\left(T_{1}\right)-Z\left(T_{0}\right)\right]}{\esp\left[T_{1}\right]},\textrm{ c.s. cuando  }t\rightarrow\infty.
\end{eqnarray}
\end{Coro}
%___________________________________________________________________________________________
%
%\subsection{Propiedades de los Procesos de Renovaci\'on}
%___________________________________________________________________________________________
%

Los tiempos $T_{n}$ est\'an relacionados con los conteos de $N\left(t\right)$ por

\begin{eqnarray*}
\left\{N\left(t\right)\geq n\right\}&=&\left\{T_{n}\leq t\right\}\\
T_{N\left(t\right)}\leq &t&<T_{N\left(t\right)+1},
\end{eqnarray*}

adem\'as $N\left(T_{n}\right)=n$, y 

\begin{eqnarray*}
N\left(t\right)=\max\left\{n:T_{n}\leq t\right\}=\min\left\{n:T_{n+1}>t\right\}
\end{eqnarray*}

Por propiedades de la convoluci\'on se sabe que

\begin{eqnarray*}
P\left\{T_{n}\leq t\right\}=F^{n\star}\left(t\right)
\end{eqnarray*}
que es la $n$-\'esima convoluci\'on de $F$. Entonces 

\begin{eqnarray*}
\left\{N\left(t\right)\geq n\right\}&=&\left\{T_{n}\leq t\right\}\\
P\left\{N\left(t\right)\leq n\right\}&=&1-F^{\left(n+1\right)\star}\left(t\right)
\end{eqnarray*}

Adem\'as usando el hecho de que $\esp\left[N\left(t\right)\right]=\sum_{n=1}^{\infty}P\left\{N\left(t\right)\geq n\right\}$
se tiene que

\begin{eqnarray*}
\esp\left[N\left(t\right)\right]=\sum_{n=1}^{\infty}F^{n\star}\left(t\right)
\end{eqnarray*}

\begin{Prop}
Para cada $t\geq0$, la funci\'on generadora de momentos $\esp\left[e^{\alpha N\left(t\right)}\right]$ existe para alguna $\alpha$ en una vecindad del 0, y de aqu\'i que $\esp\left[N\left(t\right)^{m}\right]<\infty$, para $m\geq1$.
\end{Prop}


\begin{Note}
Si el primer tiempo de renovaci\'on $\xi_{1}$ no tiene la misma distribuci\'on que el resto de las $\xi_{n}$, para $n\geq2$, a $N\left(t\right)$ se le llama Proceso de Renovaci\'on retardado, donde si $\xi$ tiene distribuci\'on $G$, entonces el tiempo $T_{n}$ de la $n$-\'esima renovaci\'on tiene distribuci\'on $G\star F^{\left(n-1\right)\star}\left(t\right)$
\end{Note}


\begin{Teo}
Para una constante $\mu\leq\infty$ ( o variable aleatoria), las siguientes expresiones son equivalentes:

\begin{eqnarray}
lim_{n\rightarrow\infty}n^{-1}T_{n}&=&\mu,\textrm{ c.s.}\\
lim_{t\rightarrow\infty}t^{-1}N\left(t\right)&=&1/\mu,\textrm{ c.s.}
\end{eqnarray}
\end{Teo}


Es decir, $T_{n}$ satisface la Ley Fuerte de los Grandes N\'umeros s\'i y s\'olo s\'i $N\left/t\right)$ la cumple.


\begin{Coro}[Ley Fuerte de los Grandes N\'umeros para Procesos de Renovaci\'on]
Si $N\left(t\right)$ es un proceso de renovaci\'on cuyos tiempos de inter-renovaci\'on tienen media $\mu\leq\infty$, entonces
\begin{eqnarray}
t^{-1}N\left(t\right)\rightarrow 1/\mu,\textrm{ c.s. cuando }t\rightarrow\infty.
\end{eqnarray}

\end{Coro}


Considerar el proceso estoc\'astico de valores reales $\left\{Z\left(t\right):t\geq0\right\}$ en el mismo espacio de probabilidad que $N\left(t\right)$

\begin{Def}
Para el proceso $\left\{Z\left(t\right):t\geq0\right\}$ se define la fluctuaci\'on m\'axima de $Z\left(t\right)$ en el intervalo $\left(T_{n-1},T_{n}\right]$:
\begin{eqnarray*}
M_{n}=\sup_{T_{n-1}<t\leq T_{n}}|Z\left(t\right)-Z\left(T_{n-1}\right)|
\end{eqnarray*}
\end{Def}

\begin{Teo}
Sup\'ongase que $n^{-1}T_{n}\rightarrow\mu$ c.s. cuando $n\rightarrow\infty$, donde $\mu\leq\infty$ es una constante o variable aleatoria. Sea $a$ una constante o variable aleatoria que puede ser infinita cuando $\mu$ es finita, y considere las expresiones l\'imite:
\begin{eqnarray}
lim_{n\rightarrow\infty}n^{-1}Z\left(T_{n}\right)&=&a,\textrm{ c.s.}\\
lim_{t\rightarrow\infty}t^{-1}Z\left(t\right)&=&a/\mu,\textrm{ c.s.}
\end{eqnarray}
La segunda expresi\'on implica la primera. Conversamente, la primera implica la segunda si el proceso $Z\left(t\right)$ es creciente, o si $lim_{n\rightarrow\infty}n^{-1}M_{n}=0$ c.s.
\end{Teo}

\begin{Coro}
Si $N\left(t\right)$ es un proceso de renovaci\'on, y $\left(Z\left(T_{n}\right)-Z\left(T_{n-1}\right),M_{n}\right)$, para $n\geq1$, son variables aleatorias independientes e id\'enticamente distribuidas con media finita, entonces,
\begin{eqnarray}
lim_{t\rightarrow\infty}t^{-1}Z\left(t\right)\rightarrow\frac{\esp\left[Z\left(T_{1}\right)-Z\left(T_{0}\right)\right]}{\esp\left[T_{1}\right]},\textrm{ c.s. cuando  }t\rightarrow\infty.
\end{eqnarray}
\end{Coro}


%___________________________________________________________________________________________
%
%\subsection*{Funci\'on de Renovaci\'on}
%___________________________________________________________________________________________
%


\begin{Def}
Sea $h\left(t\right)$ funci\'on de valores reales en $\rea$ acotada en intervalos finitos e igual a cero para $t<0$ La ecuaci\'on de renovaci\'on para $h\left(t\right)$ y la distribuci\'on $F$ es

\begin{eqnarray}\label{Ec.Renovacion}
H\left(t\right)=h\left(t\right)+\int_{\left[0,t\right]}H\left(t-s\right)dF\left(s\right)\textrm{,    }t\geq0,
\end{eqnarray}
donde $H\left(t\right)$ es una funci\'on de valores reales. Esto es $H=h+F\star H$. Decimos que $H\left(t\right)$ es soluci\'on de esta ecuaci\'on si satisface la ecuaci\'on, y es acotada en intervalos finitos e iguales a cero para $t<0$.
\end{Def}

\begin{Prop}
La funci\'on $U\star h\left(t\right)$ es la \'unica soluci\'on de la ecuaci\'on de renovaci\'on (\ref{Ec.Renovacion}).
\end{Prop}

\begin{Teo}[Teorema Renovaci\'on Elemental]
\begin{eqnarray*}
t^{-1}U\left(t\right)\rightarrow 1/\mu\textrm{,    cuando }t\rightarrow\infty.
\end{eqnarray*}
\end{Teo}

%___________________________________________________________________________________________
%
%\subsection{Funci\'on de Renovaci\'on}
%___________________________________________________________________________________________
%


Sup\'ongase que $N\left(t\right)$ es un proceso de renovaci\'on con distribuci\'on $F$ con media finita $\mu$.

\begin{Def}
La funci\'on de renovaci\'on asociada con la distribuci\'on $F$, del proceso $N\left(t\right)$, es
\begin{eqnarray*}
U\left(t\right)=\sum_{n=1}^{\infty}F^{n\star}\left(t\right),\textrm{   }t\geq0,
\end{eqnarray*}
donde $F^{0\star}\left(t\right)=\indora\left(t\geq0\right)$.
\end{Def}


\begin{Prop}
Sup\'ongase que la distribuci\'on de inter-renovaci\'on $F$ tiene densidad $f$. Entonces $U\left(t\right)$ tambi\'en tiene densidad, para $t>0$, y es $U^{'}\left(t\right)=\sum_{n=0}^{\infty}f^{n\star}\left(t\right)$. Adem\'as
\begin{eqnarray*}
\prob\left\{N\left(t\right)>N\left(t-\right)\right\}=0\textrm{,   }t\geq0.
\end{eqnarray*}
\end{Prop}

\begin{Def}
La Transformada de Laplace-Stieljes de $F$ est\'a dada por

\begin{eqnarray*}
\hat{F}\left(\alpha\right)=\int_{\rea_{+}}e^{-\alpha t}dF\left(t\right)\textrm{,  }\alpha\geq0.
\end{eqnarray*}
\end{Def}

Entonces

\begin{eqnarray*}
\hat{U}\left(\alpha\right)=\sum_{n=0}^{\infty}\hat{F^{n\star}}\left(\alpha\right)=\sum_{n=0}^{\infty}\hat{F}\left(\alpha\right)^{n}=\frac{1}{1-\hat{F}\left(\alpha\right)}.
\end{eqnarray*}


\begin{Prop}
La Transformada de Laplace $\hat{U}\left(\alpha\right)$ y $\hat{F}\left(\alpha\right)$ determina una a la otra de manera \'unica por la relaci\'on $\hat{U}\left(\alpha\right)=\frac{1}{1-\hat{F}\left(\alpha\right)}$.
\end{Prop}


\begin{Note}
Un proceso de renovaci\'on $N\left(t\right)$ cuyos tiempos de inter-renovaci\'on tienen media finita, es un proceso Poisson con tasa $\lambda$ si y s\'olo s\'i $\esp\left[U\left(t\right)\right]=\lambda t$, para $t\geq0$.
\end{Note}


\begin{Teo}
Sea $N\left(t\right)$ un proceso puntual simple con puntos de localizaci\'on $T_{n}$ tal que $\eta\left(t\right)=\esp\left[N\left(\right)\right]$ es finita para cada $t$. Entonces para cualquier funci\'on $f:\rea_{+}\rightarrow\rea$,
\begin{eqnarray*}
\esp\left[\sum_{n=1}^{N\left(\right)}f\left(T_{n}\right)\right]=\int_{\left(0,t\right]}f\left(s\right)d\eta\left(s\right)\textrm{,  }t\geq0,
\end{eqnarray*}
suponiendo que la integral exista. Adem\'as si $X_{1},X_{2},\ldots$ son variables aleatorias definidas en el mismo espacio de probabilidad que el proceso $N\left(t\right)$ tal que $\esp\left[X_{n}|T_{n}=s\right]=f\left(s\right)$, independiente de $n$. Entonces
\begin{eqnarray*}
\esp\left[\sum_{n=1}^{N\left(t\right)}X_{n}\right]=\int_{\left(0,t\right]}f\left(s\right)d\eta\left(s\right)\textrm{,  }t\geq0,
\end{eqnarray*} 
suponiendo que la integral exista. 
\end{Teo}

\begin{Coro}[Identidad de Wald para Renovaciones]
Para el proceso de renovaci\'on $N\left(t\right)$,
\begin{eqnarray*}
\esp\left[T_{N\left(t\right)+1}\right]=\mu\esp\left[N\left(t\right)+1\right]\textrm{,  }t\geq0,
\end{eqnarray*}  
\end{Coro}

%______________________________________________________________________
%\subsection{Procesos de Renovaci\'on}
%______________________________________________________________________

\begin{Def}\label{Def.Tn}
Sean $0\leq T_{1}\leq T_{2}\leq \ldots$ son tiempos aleatorios infinitos en los cuales ocurren ciertos eventos. El n\'umero de tiempos $T_{n}$ en el intervalo $\left[0,t\right)$ es

\begin{eqnarray}
N\left(t\right)=\sum_{n=1}^{\infty}\indora\left(T_{n}\leq t\right),
\end{eqnarray}
para $t\geq0$.
\end{Def}

Si se consideran los puntos $T_{n}$ como elementos de $\rea_{+}$, y $N\left(t\right)$ es el n\'umero de puntos en $\rea$. El proceso denotado por $\left\{N\left(t\right):t\geq0\right\}$, denotado por $N\left(t\right)$, es un proceso puntual en $\rea_{+}$. Los $T_{n}$ son los tiempos de ocurrencia, el proceso puntual $N\left(t\right)$ es simple si su n\'umero de ocurrencias son distintas: $0<T_{1}<T_{2}<\ldots$ casi seguramente.

\begin{Def}
Un proceso puntual $N\left(t\right)$ es un proceso de renovaci\'on si los tiempos de interocurrencia $\xi_{n}=T_{n}-T_{n-1}$, para $n\geq1$, son independientes e identicamente distribuidos con distribuci\'on $F$, donde $F\left(0\right)=0$ y $T_{0}=0$. Los $T_{n}$ son llamados tiempos de renovaci\'on, referente a la independencia o renovaci\'on de la informaci\'on estoc\'astica en estos tiempos. Los $\xi_{n}$ son los tiempos de inter-renovaci\'on, y $N\left(t\right)$ es el n\'umero de renovaciones en el intervalo $\left[0,t\right)$
\end{Def}


\begin{Note}
Para definir un proceso de renovaci\'on para cualquier contexto, solamente hay que especificar una distribuci\'on $F$, con $F\left(0\right)=0$, para los tiempos de inter-renovaci\'on. La funci\'on $F$ en turno degune las otra variables aleatorias. De manera formal, existe un espacio de probabilidad y una sucesi\'on de variables aleatorias $\xi_{1},\xi_{2},\ldots$ definidas en este con distribuci\'on $F$. Entonces las otras cantidades son $T_{n}=\sum_{k=1}^{n}\xi_{k}$ y $N\left(t\right)=\sum_{n=1}^{\infty}\indora\left(T_{n}\leq t\right)$, donde $T_{n}\rightarrow\infty$ casi seguramente por la Ley Fuerte de los Grandes Números.
\end{Note}

%___________________________________________________________________________________________
%
\subsection{Renewal and Regenerative Processes: Serfozo\cite{Serfozo}}
%___________________________________________________________________________________________
%
\begin{Def}\label{Def.Tn}
Sean $0\leq T_{1}\leq T_{2}\leq \ldots$ son tiempos aleatorios infinitos en los cuales ocurren ciertos eventos. El n\'umero de tiempos $T_{n}$ en el intervalo $\left[0,t\right)$ es

\begin{eqnarray}
N\left(t\right)=\sum_{n=1}^{\infty}\indora\left(T_{n}\leq t\right),
\end{eqnarray}
para $t\geq0$.
\end{Def}

Si se consideran los puntos $T_{n}$ como elementos de $\rea_{+}$, y $N\left(t\right)$ es el n\'umero de puntos en $\rea$. El proceso denotado por $\left\{N\left(t\right):t\geq0\right\}$, denotado por $N\left(t\right)$, es un proceso puntual en $\rea_{+}$. Los $T_{n}$ son los tiempos de ocurrencia, el proceso puntual $N\left(t\right)$ es simple si su n\'umero de ocurrencias son distintas: $0<T_{1}<T_{2}<\ldots$ casi seguramente.

\begin{Def}
Un proceso puntual $N\left(t\right)$ es un proceso de renovaci\'on si los tiempos de interocurrencia $\xi_{n}=T_{n}-T_{n-1}$, para $n\geq1$, son independientes e identicamente distribuidos con distribuci\'on $F$, donde $F\left(0\right)=0$ y $T_{0}=0$. Los $T_{n}$ son llamados tiempos de renovaci\'on, referente a la independencia o renovaci\'on de la informaci\'on estoc\'astica en estos tiempos. Los $\xi_{n}$ son los tiempos de inter-renovaci\'on, y $N\left(t\right)$ es el n\'umero de renovaciones en el intervalo $\left[0,t\right)$
\end{Def}


\begin{Note}
Para definir un proceso de renovaci\'on para cualquier contexto, solamente hay que especificar una distribuci\'on $F$, con $F\left(0\right)=0$, para los tiempos de inter-renovaci\'on. La funci\'on $F$ en turno degune las otra variables aleatorias. De manera formal, existe un espacio de probabilidad y una sucesi\'on de variables aleatorias $\xi_{1},\xi_{2},\ldots$ definidas en este con distribuci\'on $F$. Entonces las otras cantidades son $T_{n}=\sum_{k=1}^{n}\xi_{k}$ y $N\left(t\right)=\sum_{n=1}^{\infty}\indora\left(T_{n}\leq t\right)$, donde $T_{n}\rightarrow\infty$ casi seguramente por la Ley Fuerte de los Grandes N\'umeros.
\end{Note}







Los tiempos $T_{n}$ est\'an relacionados con los conteos de $N\left(t\right)$ por

\begin{eqnarray*}
\left\{N\left(t\right)\geq n\right\}&=&\left\{T_{n}\leq t\right\}\\
T_{N\left(t\right)}\leq &t&<T_{N\left(t\right)+1},
\end{eqnarray*}

adem\'as $N\left(T_{n}\right)=n$, y 

\begin{eqnarray*}
N\left(t\right)=\max\left\{n:T_{n}\leq t\right\}=\min\left\{n:T_{n+1}>t\right\}
\end{eqnarray*}

Por propiedades de la convoluci\'on se sabe que

\begin{eqnarray*}
P\left\{T_{n}\leq t\right\}=F^{n\star}\left(t\right)
\end{eqnarray*}
que es la $n$-\'esima convoluci\'on de $F$. Entonces 

\begin{eqnarray*}
\left\{N\left(t\right)\geq n\right\}&=&\left\{T_{n}\leq t\right\}\\
P\left\{N\left(t\right)\leq n\right\}&=&1-F^{\left(n+1\right)\star}\left(t\right)
\end{eqnarray*}

Adem\'as usando el hecho de que $\esp\left[N\left(t\right)\right]=\sum_{n=1}^{\infty}P\left\{N\left(t\right)\geq n\right\}$
se tiene que

\begin{eqnarray*}
\esp\left[N\left(t\right)\right]=\sum_{n=1}^{\infty}F^{n\star}\left(t\right)
\end{eqnarray*}

\begin{Prop}
Para cada $t\geq0$, la funci\'on generadora de momentos $\esp\left[e^{\alpha N\left(t\right)}\right]$ existe para alguna $\alpha$ en una vecindad del 0, y de aqu\'i que $\esp\left[N\left(t\right)^{m}\right]<\infty$, para $m\geq1$.
\end{Prop}

\begin{Ejem}[\textbf{Proceso Poisson}]

Suponga que se tienen tiempos de inter-renovaci\'on \textit{i.i.d.} del proceso de renovaci\'on $N\left(t\right)$ tienen distribuci\'on exponencial $F\left(t\right)=q-e^{-\lambda t}$ con tasa $\lambda$. Entonces $N\left(t\right)$ es un proceso Poisson con tasa $\lambda$.

\end{Ejem}


\begin{Note}
Si el primer tiempo de renovaci\'on $\xi_{1}$ no tiene la misma distribuci\'on que el resto de las $\xi_{n}$, para $n\geq2$, a $N\left(t\right)$ se le llama Proceso de Renovaci\'on retardado, donde si $\xi$ tiene distribuci\'on $G$, entonces el tiempo $T_{n}$ de la $n$-\'esima renovaci\'on tiene distribuci\'on $G\star F^{\left(n-1\right)\star}\left(t\right)$
\end{Note}


\begin{Teo}
Para una constante $\mu\leq\infty$ ( o variable aleatoria), las siguientes expresiones son equivalentes:

\begin{eqnarray}
lim_{n\rightarrow\infty}n^{-1}T_{n}&=&\mu,\textrm{ c.s.}\\
lim_{t\rightarrow\infty}t^{-1}N\left(t\right)&=&1/\mu,\textrm{ c.s.}
\end{eqnarray}
\end{Teo}


Es decir, $T_{n}$ satisface la Ley Fuerte de los Grandes N\'umeros s\'i y s\'olo s\'i $N\left/t\right)$ la cumple.


\begin{Coro}[Ley Fuerte de los Grandes N\'umeros para Procesos de Renovaci\'on]
Si $N\left(t\right)$ es un proceso de renovaci\'on cuyos tiempos de inter-renovaci\'on tienen media $\mu\leq\infty$, entonces
\begin{eqnarray}
t^{-1}N\left(t\right)\rightarrow 1/\mu,\textrm{ c.s. cuando }t\rightarrow\infty.
\end{eqnarray}

\end{Coro}


Considerar el proceso estoc\'astico de valores reales $\left\{Z\left(t\right):t\geq0\right\}$ en el mismo espacio de probabilidad que $N\left(t\right)$

\begin{Def}
Para el proceso $\left\{Z\left(t\right):t\geq0\right\}$ se define la fluctuaci\'on m\'axima de $Z\left(t\right)$ en el intervalo $\left(T_{n-1},T_{n}\right]$:
\begin{eqnarray*}
M_{n}=\sup_{T_{n-1}<t\leq T_{n}}|Z\left(t\right)-Z\left(T_{n-1}\right)|
\end{eqnarray*}
\end{Def}

\begin{Teo}
Sup\'ongase que $n^{-1}T_{n}\rightarrow\mu$ c.s. cuando $n\rightarrow\infty$, donde $\mu\leq\infty$ es una constante o variable aleatoria. Sea $a$ una constante o variable aleatoria que puede ser infinita cuando $\mu$ es finita, y considere las expresiones l\'imite:
\begin{eqnarray}
lim_{n\rightarrow\infty}n^{-1}Z\left(T_{n}\right)&=&a,\textrm{ c.s.}\\
lim_{t\rightarrow\infty}t^{-1}Z\left(t\right)&=&a/\mu,\textrm{ c.s.}
\end{eqnarray}
La segunda expresi\'on implica la primera. Conversamente, la primera implica la segunda si el proceso $Z\left(t\right)$ es creciente, o si $lim_{n\rightarrow\infty}n^{-1}M_{n}=0$ c.s.
\end{Teo}

\begin{Coro}
Si $N\left(t\right)$ es un proceso de renovaci\'on, y $\left(Z\left(T_{n}\right)-Z\left(T_{n-1}\right),M_{n}\right)$, para $n\geq1$, son variables aleatorias independientes e id\'enticamente distribuidas con media finita, entonces,
\begin{eqnarray}
lim_{t\rightarrow\infty}t^{-1}Z\left(t\right)\rightarrow\frac{\esp\left[Z\left(T_{1}\right)-Z\left(T_{0}\right)\right]}{\esp\left[T_{1}\right]},\textrm{ c.s. cuando  }t\rightarrow\infty.
\end{eqnarray}
\end{Coro}


Sup\'ongase que $N\left(t\right)$ es un proceso de renovaci\'on con distribuci\'on $F$ con media finita $\mu$.

\begin{Def}
La funci\'on de renovaci\'on asociada con la distribuci\'on $F$, del proceso $N\left(t\right)$, es
\begin{eqnarray*}
U\left(t\right)=\sum_{n=1}^{\infty}F^{n\star}\left(t\right),\textrm{   }t\geq0,
\end{eqnarray*}
donde $F^{0\star}\left(t\right)=\indora\left(t\geq0\right)$.
\end{Def}


\begin{Prop}
Sup\'ongase que la distribuci\'on de inter-renovaci\'on $F$ tiene densidad $f$. Entonces $U\left(t\right)$ tambi\'en tiene densidad, para $t>0$, y es $U^{'}\left(t\right)=\sum_{n=0}^{\infty}f^{n\star}\left(t\right)$. Adem\'as
\begin{eqnarray*}
\prob\left\{N\left(t\right)>N\left(t-\right)\right\}=0\textrm{,   }t\geq0.
\end{eqnarray*}
\end{Prop}

\begin{Def}
La Transformada de Laplace-Stieljes de $F$ est\'a dada por

\begin{eqnarray*}
\hat{F}\left(\alpha\right)=\int_{\rea_{+}}e^{-\alpha t}dF\left(t\right)\textrm{,  }\alpha\geq0.
\end{eqnarray*}
\end{Def}

Entonces

\begin{eqnarray*}
\hat{U}\left(\alpha\right)=\sum_{n=0}^{\infty}\hat{F^{n\star}}\left(\alpha\right)=\sum_{n=0}^{\infty}\hat{F}\left(\alpha\right)^{n}=\frac{1}{1-\hat{F}\left(\alpha\right)}.
\end{eqnarray*}


\begin{Prop}
La Transformada de Laplace $\hat{U}\left(\alpha\right)$ y $\hat{F}\left(\alpha\right)$ determina una a la otra de manera \'unica por la relaci\'on $\hat{U}\left(\alpha\right)=\frac{1}{1-\hat{F}\left(\alpha\right)}$.
\end{Prop}


\begin{Note}
Un proceso de renovaci\'on $N\left(t\right)$ cuyos tiempos de inter-renovaci\'on tienen media finita, es un proceso Poisson con tasa $\lambda$ si y s\'olo s\'i $\esp\left[U\left(t\right)\right]=\lambda t$, para $t\geq0$.
\end{Note}


\begin{Teo}
Sea $N\left(t\right)$ un proceso puntual simple con puntos de localizaci\'on $T_{n}$ tal que $\eta\left(t\right)=\esp\left[N\left(\right)\right]$ es finita para cada $t$. Entonces para cualquier funci\'on $f:\rea_{+}\rightarrow\rea$,
\begin{eqnarray*}
\esp\left[\sum_{n=1}^{N\left(\right)}f\left(T_{n}\right)\right]=\int_{\left(0,t\right]}f\left(s\right)d\eta\left(s\right)\textrm{,  }t\geq0,
\end{eqnarray*}
suponiendo que la integral exista. Adem\'as si $X_{1},X_{2},\ldots$ son variables aleatorias definidas en el mismo espacio de probabilidad que el proceso $N\left(t\right)$ tal que $\esp\left[X_{n}|T_{n}=s\right]=f\left(s\right)$, independiente de $n$. Entonces
\begin{eqnarray*}
\esp\left[\sum_{n=1}^{N\left(t\right)}X_{n}\right]=\int_{\left(0,t\right]}f\left(s\right)d\eta\left(s\right)\textrm{,  }t\geq0,
\end{eqnarray*} 
suponiendo que la integral exista. 
\end{Teo}

\begin{Coro}[Identidad de Wald para Renovaciones]
Para el proceso de renovaci\'on $N\left(t\right)$,
\begin{eqnarray*}
\esp\left[T_{N\left(t\right)+1}\right]=\mu\esp\left[N\left(t\right)+1\right]\textrm{,  }t\geq0,
\end{eqnarray*}  
\end{Coro}


\begin{Def}
Sea $h\left(t\right)$ funci\'on de valores reales en $\rea$ acotada en intervalos finitos e igual a cero para $t<0$ La ecuaci\'on de renovaci\'on para $h\left(t\right)$ y la distribuci\'on $F$ es

\begin{eqnarray}\label{Ec.Renovacion}
H\left(t\right)=h\left(t\right)+\int_{\left[0,t\right]}H\left(t-s\right)dF\left(s\right)\textrm{,    }t\geq0,
\end{eqnarray}
donde $H\left(t\right)$ es una funci\'on de valores reales. Esto es $H=h+F\star H$. Decimos que $H\left(t\right)$ es soluci\'on de esta ecuaci\'on si satisface la ecuaci\'on, y es acotada en intervalos finitos e iguales a cero para $t<0$.
\end{Def}

\begin{Prop}
La funci\'on $U\star h\left(t\right)$ es la \'unica soluci\'on de la ecuaci\'on de renovaci\'on (\ref{Ec.Renovacion}).
\end{Prop}

\begin{Teo}[Teorema Renovaci\'on Elemental]
\begin{eqnarray*}
t^{-1}U\left(t\right)\rightarrow 1/\mu\textrm{,    cuando }t\rightarrow\infty.
\end{eqnarray*}
\end{Teo}



Sup\'ongase que $N\left(t\right)$ es un proceso de renovaci\'on con distribuci\'on $F$ con media finita $\mu$.

\begin{Def}
La funci\'on de renovaci\'on asociada con la distribuci\'on $F$, del proceso $N\left(t\right)$, es
\begin{eqnarray*}
U\left(t\right)=\sum_{n=1}^{\infty}F^{n\star}\left(t\right),\textrm{   }t\geq0,
\end{eqnarray*}
donde $F^{0\star}\left(t\right)=\indora\left(t\geq0\right)$.
\end{Def}


\begin{Prop}
Sup\'ongase que la distribuci\'on de inter-renovaci\'on $F$ tiene densidad $f$. Entonces $U\left(t\right)$ tambi\'en tiene densidad, para $t>0$, y es $U^{'}\left(t\right)=\sum_{n=0}^{\infty}f^{n\star}\left(t\right)$. Adem\'as
\begin{eqnarray*}
\prob\left\{N\left(t\right)>N\left(t-\right)\right\}=0\textrm{,   }t\geq0.
\end{eqnarray*}
\end{Prop}

\begin{Def}
La Transformada de Laplace-Stieljes de $F$ est\'a dada por

\begin{eqnarray*}
\hat{F}\left(\alpha\right)=\int_{\rea_{+}}e^{-\alpha t}dF\left(t\right)\textrm{,  }\alpha\geq0.
\end{eqnarray*}
\end{Def}

Entonces

\begin{eqnarray*}
\hat{U}\left(\alpha\right)=\sum_{n=0}^{\infty}\hat{F^{n\star}}\left(\alpha\right)=\sum_{n=0}^{\infty}\hat{F}\left(\alpha\right)^{n}=\frac{1}{1-\hat{F}\left(\alpha\right)}.
\end{eqnarray*}


\begin{Prop}
La Transformada de Laplace $\hat{U}\left(\alpha\right)$ y $\hat{F}\left(\alpha\right)$ determina una a la otra de manera \'unica por la relaci\'on $\hat{U}\left(\alpha\right)=\frac{1}{1-\hat{F}\left(\alpha\right)}$.
\end{Prop}


\begin{Note}
Un proceso de renovaci\'on $N\left(t\right)$ cuyos tiempos de inter-renovaci\'on tienen media finita, es un proceso Poisson con tasa $\lambda$ si y s\'olo s\'i $\esp\left[U\left(t\right)\right]=\lambda t$, para $t\geq0$.
\end{Note}


\begin{Teo}
Sea $N\left(t\right)$ un proceso puntual simple con puntos de localizaci\'on $T_{n}$ tal que $\eta\left(t\right)=\esp\left[N\left(\right)\right]$ es finita para cada $t$. Entonces para cualquier funci\'on $f:\rea_{+}\rightarrow\rea$,
\begin{eqnarray*}
\esp\left[\sum_{n=1}^{N\left(\right)}f\left(T_{n}\right)\right]=\int_{\left(0,t\right]}f\left(s\right)d\eta\left(s\right)\textrm{,  }t\geq0,
\end{eqnarray*}
suponiendo que la integral exista. Adem\'as si $X_{1},X_{2},\ldots$ son variables aleatorias definidas en el mismo espacio de probabilidad que el proceso $N\left(t\right)$ tal que $\esp\left[X_{n}|T_{n}=s\right]=f\left(s\right)$, independiente de $n$. Entonces
\begin{eqnarray*}
\esp\left[\sum_{n=1}^{N\left(t\right)}X_{n}\right]=\int_{\left(0,t\right]}f\left(s\right)d\eta\left(s\right)\textrm{,  }t\geq0,
\end{eqnarray*} 
suponiendo que la integral exista. 
\end{Teo}

\begin{Coro}[Identidad de Wald para Renovaciones]
Para el proceso de renovaci\'on $N\left(t\right)$,
\begin{eqnarray*}
\esp\left[T_{N\left(t\right)+1}\right]=\mu\esp\left[N\left(t\right)+1\right]\textrm{,  }t\geq0,
\end{eqnarray*}  
\end{Coro}


\begin{Def}
Sea $h\left(t\right)$ funci\'on de valores reales en $\rea$ acotada en intervalos finitos e igual a cero para $t<0$ La ecuaci\'on de renovaci\'on para $h\left(t\right)$ y la distribuci\'on $F$ es

\begin{eqnarray}\label{Ec.Renovacion}
H\left(t\right)=h\left(t\right)+\int_{\left[0,t\right]}H\left(t-s\right)dF\left(s\right)\textrm{,    }t\geq0,
\end{eqnarray}
donde $H\left(t\right)$ es una funci\'on de valores reales. Esto es $H=h+F\star H$. Decimos que $H\left(t\right)$ es soluci\'on de esta ecuaci\'on si satisface la ecuaci\'on, y es acotada en intervalos finitos e iguales a cero para $t<0$.
\end{Def}

\begin{Prop}
La funci\'on $U\star h\left(t\right)$ es la \'unica soluci\'on de la ecuaci\'on de renovaci\'on (\ref{Ec.Renovacion}).
\end{Prop}

\begin{Teo}[Teorema Renovaci\'on Elemental]
\begin{eqnarray*}
t^{-1}U\left(t\right)\rightarrow 1/\mu\textrm{,    cuando }t\rightarrow\infty.
\end{eqnarray*}
\end{Teo}


\begin{Note} Una funci\'on $h:\rea_{+}\rightarrow\rea$ es Directamente Riemann Integrable en los siguientes casos:
\begin{itemize}
\item[a)] $h\left(t\right)\geq0$ es decreciente y Riemann Integrable.
\item[b)] $h$ es continua excepto posiblemente en un conjunto de Lebesgue de medida 0, y $|h\left(t\right)|\leq b\left(t\right)$, donde $b$ es DRI.
\end{itemize}
\end{Note}

\begin{Teo}[Teorema Principal de Renovaci\'on]
Si $F$ es no aritm\'etica y $h\left(t\right)$ es Directamente Riemann Integrable (DRI), entonces

\begin{eqnarray*}
lim_{t\rightarrow\infty}U\star h=\frac{1}{\mu}\int_{\rea_{+}}h\left(s\right)ds.
\end{eqnarray*}
\end{Teo}

\begin{Prop}
Cualquier funci\'on $H\left(t\right)$ acotada en intervalos finitos y que es 0 para $t<0$ puede expresarse como
\begin{eqnarray*}
H\left(t\right)=U\star h\left(t\right)\textrm{,  donde }h\left(t\right)=H\left(t\right)-F\star H\left(t\right)
\end{eqnarray*}
\end{Prop}

\begin{Def}
Un proceso estoc\'astico $X\left(t\right)$ es crudamente regenerativo en un tiempo aleatorio positivo $T$ si
\begin{eqnarray*}
\esp\left[X\left(T+t\right)|T\right]=\esp\left[X\left(t\right)\right]\textrm{, para }t\geq0,\end{eqnarray*}
y con las esperanzas anteriores finitas.
\end{Def}

\begin{Prop}
Sup\'ongase que $X\left(t\right)$ es un proceso crudamente regenerativo en $T$, que tiene distribuci\'on $F$. Si $\esp\left[X\left(t\right)\right]$ es acotado en intervalos finitos, entonces
\begin{eqnarray*}
\esp\left[X\left(t\right)\right]=U\star h\left(t\right)\textrm{,  donde }h\left(t\right)=\esp\left[X\left(t\right)\indora\left(T>t\right)\right].
\end{eqnarray*}
\end{Prop}

\begin{Teo}[Regeneraci\'on Cruda]
Sup\'ongase que $X\left(t\right)$ es un proceso con valores positivo crudamente regenerativo en $T$, y def\'inase $M=\sup\left\{|X\left(t\right)|:t\leq T\right\}$. Si $T$ es no aritm\'etico y $M$ y $MT$ tienen media finita, entonces
\begin{eqnarray*}
lim_{t\rightarrow\infty}\esp\left[X\left(t\right)\right]=\frac{1}{\mu}\int_{\rea_{+}}h\left(s\right)ds,
\end{eqnarray*}
donde $h\left(t\right)=\esp\left[X\left(t\right)\indora\left(T>t\right)\right]$.
\end{Teo}


\begin{Note} Una funci\'on $h:\rea_{+}\rightarrow\rea$ es Directamente Riemann Integrable en los siguientes casos:
\begin{itemize}
\item[a)] $h\left(t\right)\geq0$ es decreciente y Riemann Integrable.
\item[b)] $h$ es continua excepto posiblemente en un conjunto de Lebesgue de medida 0, y $|h\left(t\right)|\leq b\left(t\right)$, donde $b$ es DRI.
\end{itemize}
\end{Note}

\begin{Teo}[Teorema Principal de Renovaci\'on]
Si $F$ es no aritm\'etica y $h\left(t\right)$ es Directamente Riemann Integrable (DRI), entonces

\begin{eqnarray*}
lim_{t\rightarrow\infty}U\star h=\frac{1}{\mu}\int_{\rea_{+}}h\left(s\right)ds.
\end{eqnarray*}
\end{Teo}

\begin{Prop}
Cualquier funci\'on $H\left(t\right)$ acotada en intervalos finitos y que es 0 para $t<0$ puede expresarse como
\begin{eqnarray*}
H\left(t\right)=U\star h\left(t\right)\textrm{,  donde }h\left(t\right)=H\left(t\right)-F\star H\left(t\right)
\end{eqnarray*}
\end{Prop}

\begin{Def}
Un proceso estoc\'astico $X\left(t\right)$ es crudamente regenerativo en un tiempo aleatorio positivo $T$ si
\begin{eqnarray*}
\esp\left[X\left(T+t\right)|T\right]=\esp\left[X\left(t\right)\right]\textrm{, para }t\geq0,\end{eqnarray*}
y con las esperanzas anteriores finitas.
\end{Def}

\begin{Prop}
Sup\'ongase que $X\left(t\right)$ es un proceso crudamente regenerativo en $T$, que tiene distribuci\'on $F$. Si $\esp\left[X\left(t\right)\right]$ es acotado en intervalos finitos, entonces
\begin{eqnarray*}
\esp\left[X\left(t\right)\right]=U\star h\left(t\right)\textrm{,  donde }h\left(t\right)=\esp\left[X\left(t\right)\indora\left(T>t\right)\right].
\end{eqnarray*}
\end{Prop}

\begin{Teo}[Regeneraci\'on Cruda]
Sup\'ongase que $X\left(t\right)$ es un proceso con valores positivo crudamente regenerativo en $T$, y def\'inase $M=\sup\left\{|X\left(t\right)|:t\leq T\right\}$. Si $T$ es no aritm\'etico y $M$ y $MT$ tienen media finita, entonces
\begin{eqnarray*}
lim_{t\rightarrow\infty}\esp\left[X\left(t\right)\right]=\frac{1}{\mu}\int_{\rea_{+}}h\left(s\right)ds,
\end{eqnarray*}
donde $h\left(t\right)=\esp\left[X\left(t\right)\indora\left(T>t\right)\right]$.
\end{Teo}

\begin{Def}
Para el proceso $\left\{\left(N\left(t\right),X\left(t\right)\right):t\geq0\right\}$, sus trayectoria muestrales en el intervalo de tiempo $\left[T_{n-1},T_{n}\right)$ est\'an descritas por
\begin{eqnarray*}
\zeta_{n}=\left(\xi_{n},\left\{X\left(T_{n-1}+t\right):0\leq t<\xi_{n}\right\}\right)
\end{eqnarray*}
Este $\zeta_{n}$ es el $n$-\'esimo segmento del proceso. El proceso es regenerativo sobre los tiempos $T_{n}$ si sus segmentos $\zeta_{n}$ son independientes e id\'enticamennte distribuidos.
\end{Def}


\begin{Note}
Si $\tilde{X}\left(t\right)$ con espacio de estados $\tilde{S}$ es regenerativo sobre $T_{n}$, entonces $X\left(t\right)=f\left(\tilde{X}\left(t\right)\right)$ tambi\'en es regenerativo sobre $T_{n}$, para cualquier funci\'on $f:\tilde{S}\rightarrow S$.
\end{Note}

\begin{Note}
Los procesos regenerativos son crudamente regenerativos, pero no al rev\'es.
\end{Note}


\begin{Note}
Un proceso estoc\'astico a tiempo continuo o discreto es regenerativo si existe un proceso de renovaci\'on  tal que los segmentos del proceso entre tiempos de renovaci\'on sucesivos son i.i.d., es decir, para $\left\{X\left(t\right):t\geq0\right\}$ proceso estoc\'astico a tiempo continuo con espacio de estados $S$, espacio m\'etrico.
\end{Note}

Para $\left\{X\left(t\right):t\geq0\right\}$ Proceso Estoc\'astico a tiempo continuo con estado de espacios $S$, que es un espacio m\'etrico, con trayectorias continuas por la derecha y con l\'imites por la izquierda c.s. Sea $N\left(t\right)$ un proceso de renovaci\'on en $\rea_{+}$ definido en el mismo espacio de probabilidad que $X\left(t\right)$, con tiempos de renovaci\'on $T$ y tiempos de inter-renovaci\'on $\xi_{n}=T_{n}-T_{n-1}$, con misma distribuci\'on $F$ de media finita $\mu$.



\begin{Def}
Para el proceso $\left\{\left(N\left(t\right),X\left(t\right)\right):t\geq0\right\}$, sus trayectoria muestrales en el intervalo de tiempo $\left[T_{n-1},T_{n}\right)$ est\'an descritas por
\begin{eqnarray*}
\zeta_{n}=\left(\xi_{n},\left\{X\left(T_{n-1}+t\right):0\leq t<\xi_{n}\right\}\right)
\end{eqnarray*}
Este $\zeta_{n}$ es el $n$-\'esimo segmento del proceso. El proceso es regenerativo sobre los tiempos $T_{n}$ si sus segmentos $\zeta_{n}$ son independientes e id\'enticamennte distribuidos.
\end{Def}

\begin{Note}
Un proceso regenerativo con media de la longitud de ciclo finita es llamado positivo recurrente.
\end{Note}

\begin{Teo}[Procesos Regenerativos]
Suponga que el proceso
\end{Teo}


\begin{Def}[Renewal Process Trinity]
Para un proceso de renovaci\'on $N\left(t\right)$, los siguientes procesos proveen de informaci\'on sobre los tiempos de renovaci\'on.
\begin{itemize}
\item $A\left(t\right)=t-T_{N\left(t\right)}$, el tiempo de recurrencia hacia atr\'as al tiempo $t$, que es el tiempo desde la \'ultima renovaci\'on para $t$.

\item $B\left(t\right)=T_{N\left(t\right)+1}-t$, el tiempo de recurrencia hacia adelante al tiempo $t$, residual del tiempo de renovaci\'on, que es el tiempo para la pr\'oxima renovaci\'on despu\'es de $t$.

\item $L\left(t\right)=\xi_{N\left(t\right)+1}=A\left(t\right)+B\left(t\right)$, la longitud del intervalo de renovaci\'on que contiene a $t$.
\end{itemize}
\end{Def}

\begin{Note}
El proceso tridimensional $\left(A\left(t\right),B\left(t\right),L\left(t\right)\right)$ es regenerativo sobre $T_{n}$, y por ende cada proceso lo es. Cada proceso $A\left(t\right)$ y $B\left(t\right)$ son procesos de MArkov a tiempo continuo con trayectorias continuas por partes en el espacio de estados $\rea_{+}$. Una expresi\'on conveniente para su distribuci\'on conjunta es, para $0\leq x<t,y\geq0$
\begin{equation}\label{NoRenovacion}
P\left\{A\left(t\right)>x,B\left(t\right)>y\right\}=
P\left\{N\left(t+y\right)-N\left((t-x)\right)=0\right\}
\end{equation}
\end{Note}

\begin{Ejem}[Tiempos de recurrencia Poisson]
Si $N\left(t\right)$ es un proceso Poisson con tasa $\lambda$, entonces de la expresi\'on (\ref{NoRenovacion}) se tiene que

\begin{eqnarray*}
\begin{array}{lc}
P\left\{A\left(t\right)>x,B\left(t\right)>y\right\}=e^{-\lambda\left(x+y\right)},&0\leq x<t,y\geq0,
\end{array}
\end{eqnarray*}
que es la probabilidad Poisson de no renovaciones en un intervalo de longitud $x+y$.

\end{Ejem}

\begin{Note}
Una cadena de Markov erg\'odica tiene la propiedad de ser estacionaria si la distribuci\'on de su estado al tiempo $0$ es su distribuci\'on estacionaria.
\end{Note}


\begin{Def}
Un proceso estoc\'astico a tiempo continuo $\left\{X\left(t\right):t\geq0\right\}$ en un espacio general es estacionario si sus distribuciones finito dimensionales son invariantes bajo cualquier  traslado: para cada $0\leq s_{1}<s_{2}<\cdots<s_{k}$ y $t\geq0$,
\begin{eqnarray*}
\left(X\left(s_{1}+t\right),\ldots,X\left(s_{k}+t\right)\right)=_{d}\left(X\left(s_{1}\right),\ldots,X\left(s_{k}\right)\right).
\end{eqnarray*}
\end{Def}

\begin{Note}
Un proceso de Markov es estacionario si $X\left(t\right)=_{d}X\left(0\right)$, $t\geq0$.
\end{Note}

Considerese el proceso $N\left(t\right)=\sum_{n}\indora\left(\tau_{n}\leq t\right)$ en $\rea_{+}$, con puntos $0<\tau_{1}<\tau_{2}<\cdots$.

\begin{Prop}
Si $N$ es un proceso puntual estacionario y $\esp\left[N\left(1\right)\right]<\infty$, entonces $\esp\left[N\left(t\right)\right]=t\esp\left[N\left(1\right)\right]$, $t\geq0$

\end{Prop}

\begin{Teo}
Los siguientes enunciados son equivalentes
\begin{itemize}
\item[i)] El proceso retardado de renovaci\'on $N$ es estacionario.

\item[ii)] EL proceso de tiempos de recurrencia hacia adelante $B\left(t\right)$ es estacionario.


\item[iii)] $\esp\left[N\left(t\right)\right]=t/\mu$,


\item[iv)] $G\left(t\right)=F_{e}\left(t\right)=\frac{1}{\mu}\int_{0}^{t}\left[1-F\left(s\right)\right]ds$
\end{itemize}
Cuando estos enunciados son ciertos, $P\left\{B\left(t\right)\leq x\right\}=F_{e}\left(x\right)$, para $t,x\geq0$.

\end{Teo}

\begin{Note}
Una consecuencia del teorema anterior es que el Proceso Poisson es el \'unico proceso sin retardo que es estacionario.
\end{Note}

\begin{Coro}
El proceso de renovaci\'on $N\left(t\right)$ sin retardo, y cuyos tiempos de inter renonaci\'on tienen media finita, es estacionario si y s\'olo si es un proceso Poisson.

\end{Coro}

%______________________________________________________________________

%\section{Ejemplos, Notas importantes}
%______________________________________________________________________
%\section*{Ap\'endice A}
%__________________________________________________________________

%________________________________________________________________________
%\subsection*{Procesos Regenerativos}
%________________________________________________________________________



\begin{Note}
Si $\tilde{X}\left(t\right)$ con espacio de estados $\tilde{S}$ es regenerativo sobre $T_{n}$, entonces $X\left(t\right)=f\left(\tilde{X}\left(t\right)\right)$ tambi\'en es regenerativo sobre $T_{n}$, para cualquier funci\'on $f:\tilde{S}\rightarrow S$.
\end{Note}

\begin{Note}
Los procesos regenerativos son crudamente regenerativos, pero no al rev\'es.
\end{Note}
%\subsection*{Procesos Regenerativos: Sigman\cite{Sigman1}}
\begin{Def}[Definici\'on Cl\'asica]
Un proceso estoc\'astico $X=\left\{X\left(t\right):t\geq0\right\}$ es llamado regenerativo is existe una variable aleatoria $R_{1}>0$ tal que
\begin{itemize}
\item[i)] $\left\{X\left(t+R_{1}\right):t\geq0\right\}$ es independiente de $\left\{\left\{X\left(t\right):t<R_{1}\right\},\right\}$
\item[ii)] $\left\{X\left(t+R_{1}\right):t\geq0\right\}$ es estoc\'asticamente equivalente a $\left\{X\left(t\right):t>0\right\}$
\end{itemize}

Llamamos a $R_{1}$ tiempo de regeneraci\'on, y decimos que $X$ se regenera en este punto.
\end{Def}

$\left\{X\left(t+R_{1}\right)\right\}$ es regenerativo con tiempo de regeneraci\'on $R_{2}$, independiente de $R_{1}$ pero con la misma distribuci\'on que $R_{1}$. Procediendo de esta manera se obtiene una secuencia de variables aleatorias independientes e id\'enticamente distribuidas $\left\{R_{n}\right\}$ llamados longitudes de ciclo. Si definimos a $Z_{k}\equiv R_{1}+R_{2}+\cdots+R_{k}$, se tiene un proceso de renovaci\'on llamado proceso de renovaci\'on encajado para $X$.




\begin{Def}
Para $x$ fijo y para cada $t\geq0$, sea $I_{x}\left(t\right)=1$ si $X\left(t\right)\leq x$,  $I_{x}\left(t\right)=0$ en caso contrario, y def\'inanse los tiempos promedio
\begin{eqnarray*}
\overline{X}&=&lim_{t\rightarrow\infty}\frac{1}{t}\int_{0}^{\infty}X\left(u\right)du\\
\prob\left(X_{\infty}\leq x\right)&=&lim_{t\rightarrow\infty}\frac{1}{t}\int_{0}^{\infty}I_{x}\left(u\right)du,
\end{eqnarray*}
cuando estos l\'imites existan.
\end{Def}

Como consecuencia del teorema de Renovaci\'on-Recompensa, se tiene que el primer l\'imite  existe y es igual a la constante
\begin{eqnarray*}
\overline{X}&=&\frac{\esp\left[\int_{0}^{R_{1}}X\left(t\right)dt\right]}{\esp\left[R_{1}\right]},
\end{eqnarray*}
suponiendo que ambas esperanzas son finitas.

\begin{Note}
\begin{itemize}
\item[a)] Si el proceso regenerativo $X$ es positivo recurrente y tiene trayectorias muestrales no negativas, entonces la ecuaci\'on anterior es v\'alida.
\item[b)] Si $X$ es positivo recurrente regenerativo, podemos construir una \'unica versi\'on estacionaria de este proceso, $X_{e}=\left\{X_{e}\left(t\right)\right\}$, donde $X_{e}$ es un proceso estoc\'astico regenerativo y estrictamente estacionario, con distribuci\'on marginal distribuida como $X_{\infty}$
\end{itemize}
\end{Note}

Para $\left\{X\left(t\right):t\geq0\right\}$ Proceso Estoc\'astico a tiempo continuo con estado de espacios $S$, que es un espacio m\'etrico, con trayectorias continuas por la derecha y con l\'imites por la izquierda c.s. Sea $N\left(t\right)$ un proceso de renovaci\'on en $\rea_{+}$ definido en el mismo espacio de probabilidad que $X\left(t\right)$, con tiempos de renovaci\'on $T$ y tiempos de inter-renovaci\'on $\xi_{n}=T_{n}-T_{n-1}$, con misma distribuci\'on $F$ de media finita $\mu$.


\begin{Def}
Para el proceso $\left\{\left(N\left(t\right),X\left(t\right)\right):t\geq0\right\}$, sus trayectoria muestrales en el intervalo de tiempo $\left[T_{n-1},T_{n}\right)$ est\'an descritas por
\begin{eqnarray*}
\zeta_{n}=\left(\xi_{n},\left\{X\left(T_{n-1}+t\right):0\leq t<\xi_{n}\right\}\right)
\end{eqnarray*}
Este $\zeta_{n}$ es el $n$-\'esimo segmento del proceso. El proceso es regenerativo sobre los tiempos $T_{n}$ si sus segmentos $\zeta_{n}$ son independientes e id\'enticamennte distribuidos.
\end{Def}


\begin{Note}
Si $\tilde{X}\left(t\right)$ con espacio de estados $\tilde{S}$ es regenerativo sobre $T_{n}$, entonces $X\left(t\right)=f\left(\tilde{X}\left(t\right)\right)$ tambi\'en es regenerativo sobre $T_{n}$, para cualquier funci\'on $f:\tilde{S}\rightarrow S$.
\end{Note}

\begin{Note}
Los procesos regenerativos son crudamente regenerativos, pero no al rev\'es.
\end{Note}

\begin{Def}[Definici\'on Cl\'asica]
Un proceso estoc\'astico $X=\left\{X\left(t\right):t\geq0\right\}$ es llamado regenerativo is existe una variable aleatoria $R_{1}>0$ tal que
\begin{itemize}
\item[i)] $\left\{X\left(t+R_{1}\right):t\geq0\right\}$ es independiente de $\left\{\left\{X\left(t\right):t<R_{1}\right\},\right\}$
\item[ii)] $\left\{X\left(t+R_{1}\right):t\geq0\right\}$ es estoc\'asticamente equivalente a $\left\{X\left(t\right):t>0\right\}$
\end{itemize}

Llamamos a $R_{1}$ tiempo de regeneraci\'on, y decimos que $X$ se regenera en este punto.
\end{Def}

$\left\{X\left(t+R_{1}\right)\right\}$ es regenerativo con tiempo de regeneraci\'on $R_{2}$, independiente de $R_{1}$ pero con la misma distribuci\'on que $R_{1}$. Procediendo de esta manera se obtiene una secuencia de variables aleatorias independientes e id\'enticamente distribuidas $\left\{R_{n}\right\}$ llamados longitudes de ciclo. Si definimos a $Z_{k}\equiv R_{1}+R_{2}+\cdots+R_{k}$, se tiene un proceso de renovaci\'on llamado proceso de renovaci\'on encajado para $X$.

\begin{Note}
Un proceso regenerativo con media de la longitud de ciclo finita es llamado positivo recurrente.
\end{Note}


\begin{Def}
Para $x$ fijo y para cada $t\geq0$, sea $I_{x}\left(t\right)=1$ si $X\left(t\right)\leq x$,  $I_{x}\left(t\right)=0$ en caso contrario, y def\'inanse los tiempos promedio
\begin{eqnarray*}
\overline{X}&=&lim_{t\rightarrow\infty}\frac{1}{t}\int_{0}^{\infty}X\left(u\right)du\\
\prob\left(X_{\infty}\leq x\right)&=&lim_{t\rightarrow\infty}\frac{1}{t}\int_{0}^{\infty}I_{x}\left(u\right)du,
\end{eqnarray*}
cuando estos l\'imites existan.
\end{Def}

Como consecuencia del teorema de Renovaci\'on-Recompensa, se tiene que el primer l\'imite  existe y es igual a la constante
\begin{eqnarray*}
\overline{X}&=&\frac{\esp\left[\int_{0}^{R_{1}}X\left(t\right)dt\right]}{\esp\left[R_{1}\right]},
\end{eqnarray*}
suponiendo que ambas esperanzas son finitas.

\begin{Note}
\begin{itemize}
\item[a)] Si el proceso regenerativo $X$ es positivo recurrente y tiene trayectorias muestrales no negativas, entonces la ecuaci\'on anterior es v\'alida.
\item[b)] Si $X$ es positivo recurrente regenerativo, podemos construir una \'unica versi\'on estacionaria de este proceso, $X_{e}=\left\{X_{e}\left(t\right)\right\}$, donde $X_{e}$ es un proceso estoc\'astico regenerativo y estrictamente estacionario, con distribuci\'on marginal distribuida como $X_{\infty}$
\end{itemize}
\end{Note}

%__________________________________________________________________________________________
%\subsection{Procesos Regenerativos Estacionarios - Stidham \cite{Stidham}}
%__________________________________________________________________________________________


Un proceso estoc\'astico a tiempo continuo $\left\{V\left(t\right),t\geq0\right\}$ es un proceso regenerativo si existe una sucesi\'on de variables aleatorias independientes e id\'enticamente distribuidas $\left\{X_{1},X_{2},\ldots\right\}$, sucesi\'on de renovaci\'on, tal que para cualquier conjunto de Borel $A$, 

\begin{eqnarray*}
\prob\left\{V\left(t\right)\in A|X_{1}+X_{2}+\cdots+X_{R\left(t\right)}=s,\left\{V\left(\tau\right),\tau<s\right\}\right\}=\prob\left\{V\left(t-s\right)\in A|X_{1}>t-s\right\},
\end{eqnarray*}
para todo $0\leq s\leq t$, donde $R\left(t\right)=\max\left\{X_{1}+X_{2}+\cdots+X_{j}\leq t\right\}=$n\'umero de renovaciones ({\emph{puntos de regeneraci\'on}}) que ocurren en $\left[0,t\right]$. El intervalo $\left[0,X_{1}\right)$ es llamado {\emph{primer ciclo de regeneraci\'on}} de $\left\{V\left(t \right),t\geq0\right\}$, $\left[X_{1},X_{1}+X_{2}\right)$ el {\emph{segundo ciclo de regeneraci\'on}}, y as\'i sucesivamente.

Sea $X=X_{1}$ y sea $F$ la funci\'on de distrbuci\'on de $X$


\begin{Def}
Se define el proceso estacionario, $\left\{V^{*}\left(t\right),t\geq0\right\}$, para $\left\{V\left(t\right),t\geq0\right\}$ por

\begin{eqnarray*}
\prob\left\{V\left(t\right)\in A\right\}=\frac{1}{\esp\left[X\right]}\int_{0}^{\infty}\prob\left\{V\left(t+x\right)\in A|X>x\right\}\left(1-F\left(x\right)\right)dx,
\end{eqnarray*} 
para todo $t\geq0$ y todo conjunto de Borel $A$.
\end{Def}

\begin{Def}
Una distribuci\'on se dice que es {\emph{aritm\'etica}} si todos sus puntos de incremento son m\'ultiplos de la forma $0,\lambda, 2\lambda,\ldots$ para alguna $\lambda>0$ entera.
\end{Def}


\begin{Def}
Una modificaci\'on medible de un proceso $\left\{V\left(t\right),t\geq0\right\}$, es una versi\'on de este, $\left\{V\left(t,w\right)\right\}$ conjuntamente medible para $t\geq0$ y para $w\in S$, $S$ espacio de estados para $\left\{V\left(t\right),t\geq0\right\}$.
\end{Def}

\begin{Teo}
Sea $\left\{V\left(t\right),t\geq\right\}$ un proceso regenerativo no negativo con modificaci\'on medible. Sea $\esp\left[X\right]<\infty$. Entonces el proceso estacionario dado por la ecuaci\'on anterior est\'a bien definido y tiene funci\'on de distribuci\'on independiente de $t$, adem\'as
\begin{itemize}
\item[i)] \begin{eqnarray*}
\esp\left[V^{*}\left(0\right)\right]&=&\frac{\esp\left[\int_{0}^{X}V\left(s\right)ds\right]}{\esp\left[X\right]}\end{eqnarray*}
\item[ii)] Si $\esp\left[V^{*}\left(0\right)\right]<\infty$, equivalentemente, si $\esp\left[\int_{0}^{X}V\left(s\right)ds\right]<\infty$,entonces
\begin{eqnarray*}
\frac{\int_{0}^{t}V\left(s\right)ds}{t}\rightarrow\frac{\esp\left[\int_{0}^{X}V\left(s\right)ds\right]}{\esp\left[X\right]}
\end{eqnarray*}
con probabilidad 1 y en media, cuando $t\rightarrow\infty$.
\end{itemize}
\end{Teo}



\chapter{Introducción a los Procesos de Renovacion}
\documentclass{article}
\usepackage[utf8]{inputenc}
\usepackage[spanish,english]{babel}
\usepackage{amsmath,amssymb,amsthm,amsfonts}
\usepackage{geometry}
\usepackage{hyperref}
\usepackage{fancyhdr}
\usepackage{titlesec}
\usepackage{listings}
\usepackage{graphicx,graphics}
\usepackage{multicol}
\usepackage{multirow}
\usepackage{color}
\usepackage{float} 
\usepackage{subfig}
\usepackage[figuresright]{rotating}
\usepackage{enumerate}
\usepackage{anysize} 
\usepackage{url}

\title{Procesos de Renovaci\'on: Revisi\'on}
\author{Carlos E. Martínez-Rodríguez}
\date{Julio 2024}

\geometry{
  a4paper,
  left=25mm,
  right=25mm,
  top=30mm,
  bottom=30mm,
}

% Configuración de encabezados y pies de página
\pagestyle{fancy}
\fancyhf{}
\fancyhead[L]{\leftmark}
\fancyfoot[C]{\thepage}
\fancyfoot[R]{\rightmark}
\fancyfoot[L]{Carlos E. Martínez-Rodríguez}

% Definiciones de nuevos entornos
\newtheorem{Algthm}{Algoritmo}
\newtheorem{Def}{Definición}
\newtheorem{Ejem}{Ejemplo}
\newtheorem{Teo}{Teorema}
\newtheorem{Dem}{Demostración}
\newtheorem{Note}{Nota}
\newtheorem{Sol}{Solución}
\newtheorem{Prop}{Proposición}
\newtheorem{Cor}{Corolario}
\newtheorem{Col}{Corolario}
\newtheorem{Coro}{Corolario}
\newtheorem{Lemma}{Lema}
\newtheorem{Lem}{Lema}
\newtheorem{Lema}{Lema}
\newtheorem{Sup}{Supuestos}
\newtheorem{Assumption}{Supuestos}
\newtheorem{Remark}{Observación}
\newtheorem{Condition}{Condición}
\newtheorem{Theorem}{Teorema}
\newtheorem{Corollary}{Corolario}
\newtheorem{Ejemplo}{Ejemplo}
\newtheorem{Example}{Ejemplo}
\newtheorem{Obs}{Observación}

% Nuevos comandos
\def\RR{\mathbb{R}}
\def\ZZ{\mathbb{Z}}
\newcommand{\nat}{\mathbb{N}}
\newcommand{\ent}{\mathbb{Z}}
\newcommand{\rea}{\mathbb{R}}
\newcommand{\Eb}{\mathbf{E}}
\newcommand{\esp}{\mathbb{E}}
\newcommand{\prob}{\mathbb{P}}
\newcommand{\indora}{\mbox{$1$\hspace{-0.8ex}$1$}}
\newcommand{\ER}{\left(E,\mathcal{E}\right)}
\newcommand{\KM}{\left(P_{s,t}\right)}
\newcommand{\Xt}{\left(X_{t}\right)_{t\in I}}
\newcommand{\PE}{\left(X_{t}\right)_{t\in I}}
\newcommand{\SG}{\left(P_{t}\right)}
\newcommand{\CM}{\mathbf{P}^{x}}
\newcommand\mypar{\par\vspace{\baselineskip}}

\begin{document}

\maketitle

\tableofcontents
%<>===<>==<>===<>==<>===<>==<>===<>==<>===<>==<>===<>==<>===<>==<>===<>==<>===

%___________________________________________________________________________________________
%
\section{Propiedades de los Procesos de Renovaci\'on}
%___________________________________________________________________________________________
%

Los tiempos $T_{n}$ est\'an relacionados con los conteos de $N\left(t\right)$ por

\begin{eqnarray*}
\left\{N\left(t\right)\geq n\right\}&=&\left\{T_{n}\leq t\right\}\\
T_{N\left(t\right)}\leq &t&<T_{N\left(t\right)+1},
\end{eqnarray*}

adem\'as $N\left(T_{n}\right)=n$, y

\begin{eqnarray*}
N\left(t\right)=\max\left\{n:T_{n}\leq t\right\}=\min\left\{n:T_{n+1}>t\right\}
\end{eqnarray*}

Por propiedades de la convoluci\'on se sabe que

\begin{eqnarray*}
P\left\{T_{n}\leq t\right\}=F^{n\star}\left(t\right)
\end{eqnarray*}
que es la $n$-\'esima convoluci\'on de $F$. Entonces

\begin{eqnarray*}
\left\{N\left(t\right)\geq n\right\}&=&\left\{T_{n}\leq t\right\}\\
P\left\{N\left(t\right)\leq n\right\}&=&1-F^{\left(n+1\right)\star}\left(t\right)
\end{eqnarray*}

Adem\'as usando el hecho de que $\esp\left[N\left(t\right)\right]=\sum_{n=1}^{\infty}P\left\{N\left(t\right)\geq n\right\}$
se tiene que

\begin{eqnarray*}
\esp\left[N\left(t\right)\right]=\sum_{n=1}^{\infty}F^{n\star}\left(t\right)
\end{eqnarray*}

\begin{Prop}
Para cada $t\geq0$, la funci\'on generadora de momentos $\esp\left[e^{\alpha N\left(t\right)}\right]$ existe para alguna $\alpha$ en una vecindad del 0, y de aqu\'i que $\esp\left[N\left(t\right)^{m}\right]<\infty$, para $m\geq1$.
\end{Prop}


\begin{Note}
Si el primer tiempo de renovaci\'on $\xi_{1}$ no tiene la misma distribuci\'on que el resto de las $\xi_{n}$, para $n\geq2$, a $N\left(t\right)$ se le llama Proceso de Renovaci\'on retardado, donde si $\xi$ tiene distribuci\'on $G$, entonces el tiempo $T_{n}$ de la $n$-\'esima renovaci\'on tiene distribuci\'on $G\star F^{\left(n-1\right)\star}\left(t\right)$
\end{Note}


\begin{Teo}
Para una constante $\mu\leq\infty$ ( o variable aleatoria), las siguientes expresiones son equivalentes:

\begin{eqnarray}
lim_{n\rightarrow\infty}n^{-1}T_{n}&=&\mu,\textrm{ c.s.}\\
lim_{t\rightarrow\infty}t^{-1}N\left(t\right)&=&1/\mu,\textrm{ c.s.}
\end{eqnarray}
\end{Teo}


Es decir, $T_{n}$ satisface la Ley Fuerte de los Grandes N\'umeros s\'i y s\'olo s\'i $N\left/t\right)$ la cumple.


\begin{Coro}[Ley Fuerte de los Grandes N\'umeros para Procesos de Renovaci\'on]
Si $N\left(t\right)$ es un proceso de renovaci\'on cuyos tiempos de inter-renovaci\'on tienen media $\mu\leq\infty$, entonces
\begin{eqnarray}
t^{-1}N\left(t\right)\rightarrow 1/\mu,\textrm{ c.s. cuando }t\rightarrow\infty.
\end{eqnarray}

\end{Coro}


Considerar el proceso estoc\'astico de valores reales $\left\{Z\left(t\right):t\geq0\right\}$ en el mismo espacio de probabilidad que $N\left(t\right)$

\begin{Def}
Para el proceso $\left\{Z\left(t\right):t\geq0\right\}$ se define la fluctuaci\'on m\'axima de $Z\left(t\right)$ en el intervalo $\left(T_{n-1},T_{n}\right]$:
\begin{eqnarray*}
M_{n}=\sup_{T_{n-1}<t\leq T_{n}}|Z\left(t\right)-Z\left(T_{n-1}\right)|
\end{eqnarray*}
\end{Def}

\begin{Teo}
Sup\'ongase que $n^{-1}T_{n}\rightarrow\mu$ c.s. cuando $n\rightarrow\infty$, donde $\mu\leq\infty$ es una constante o variable aleatoria. Sea $a$ una constante o variable aleatoria que puede ser infinita cuando $\mu$ es finita, y considere las expresiones l\'imite:
\begin{eqnarray}
lim_{n\rightarrow\infty}n^{-1}Z\left(T_{n}\right)&=&a,\textrm{ c.s.}\\
lim_{t\rightarrow\infty}t^{-1}Z\left(t\right)&=&a/\mu,\textrm{ c.s.}
\end{eqnarray}
La segunda expresi\'on implica la primera. Conversamente, la primera implica la segunda si el proceso $Z\left(t\right)$ es creciente, o si $lim_{n\rightarrow\infty}n^{-1}M_{n}=0$ c.s.
\end{Teo}

\begin{Coro}
Si $N\left(t\right)$ es un proceso de renovaci\'on, y $\left(Z\left(T_{n}\right)-Z\left(T_{n-1}\right),M_{n}\right)$, para $n\geq1$, son variables aleatorias independientes e id\'enticamente distribuidas con media finita, entonces,
\begin{eqnarray}
lim_{t\rightarrow\infty}t^{-1}Z\left(t\right)\rightarrow\frac{\esp\left[Z\left(T_{1}\right)-Z\left(T_{0}\right)\right]}{\esp\left[T_{1}\right]},\textrm{ c.s. cuando  }t\rightarrow\infty.
\end{eqnarray}
\end{Coro}

%___________________________________________________________________________________________
%
\subsubsection{Propiedades de los Procesos de Renovaci\'on}
%___________________________________________________________________________________________
%

Los tiempos $T_{n}$ est\'an relacionados con los conteos de $N\left(t\right)$ por

\begin{eqnarray*}
\left\{N\left(t\right)\geq n\right\}&=&\left\{T_{n}\leq t\right\}\\
T_{N\left(t\right)}\leq &t&<T_{N\left(t\right)+1},
\end{eqnarray*}

adem\'as $N\left(T_{n}\right)=n$, y 

\begin{eqnarray*}
N\left(t\right)=\max\left\{n:T_{n}\leq t\right\}=\min\left\{n:T_{n+1}>t\right\}
\end{eqnarray*}

Por propiedades de la convoluci\'on se sabe que

\begin{eqnarray*}
P\left\{T_{n}\leq t\right\}=F^{n\star}\left(t\right)
\end{eqnarray*}
que es la $n$-\'esima convoluci\'on de $F$. Entonces 

\begin{eqnarray*}
\left\{N\left(t\right)\geq n\right\}&=&\left\{T_{n}\leq t\right\}\\
P\left\{N\left(t\right)\leq n\right\}&=&1-F^{\left(n+1\right)\star}\left(t\right)
\end{eqnarray*}

Adem\'as usando el hecho de que $\esp\left[N\left(t\right)\right]=\sum_{n=1}^{\infty}P\left\{N\left(t\right)\geq n\right\}$
se tiene que

\begin{eqnarray*}
\esp\left[N\left(t\right)\right]=\sum_{n=1}^{\infty}F^{n\star}\left(t\right)
\end{eqnarray*}

\begin{Prop}
Para cada $t\geq0$, la funci\'on generadora de momentos $\esp\left[e^{\alpha N\left(t\right)}\right]$ existe para alguna $\alpha$ en una vecindad del 0, y de aqu\'i que $\esp\left[N\left(t\right)^{m}\right]<\infty$, para $m\geq1$.
\end{Prop}


\begin{Note}
Si el primer tiempo de renovaci\'on $\xi_{1}$ no tiene la misma distribuci\'on que el resto de las $\xi_{n}$, para $n\geq2$, a $N\left(t\right)$ se le llama Proceso de Renovaci\'on retardado, donde si $\xi$ tiene distribuci\'on $G$, entonces el tiempo $T_{n}$ de la $n$-\'esima renovaci\'on tiene distribuci\'on $G\star F^{\left(n-1\right)\star}\left(t\right)$
\end{Note}


\begin{Teo}
Para una constante $\mu\leq\infty$ ( o variable aleatoria), las siguientes expresiones son equivalentes:

\begin{eqnarray}
lim_{n\rightarrow\infty}n^{-1}T_{n}&=&\mu,\textrm{ c.s.}\\
lim_{t\rightarrow\infty}t^{-1}N\left(t\right)&=&1/\mu,\textrm{ c.s.}
\end{eqnarray}
\end{Teo}


Es decir, $T_{n}$ satisface la Ley Fuerte de los Grandes N\'umeros s\'i y s\'olo s\'i $N\left/t\right)$ la cumple.


\begin{Coro}[Ley Fuerte de los Grandes N\'umeros para Procesos de Renovaci\'on]
Si $N\left(t\right)$ es un proceso de renovaci\'on cuyos tiempos de inter-renovaci\'on tienen media $\mu\leq\infty$, entonces
\begin{eqnarray}
t^{-1}N\left(t\right)\rightarrow 1/\mu,\textrm{ c.s. cuando }t\rightarrow\infty.
\end{eqnarray}

\end{Coro}


Considerar el proceso estoc\'astico de valores reales $\left\{Z\left(t\right):t\geq0\right\}$ en el mismo espacio de probabilidad que $N\left(t\right)$

\begin{Def}
Para el proceso $\left\{Z\left(t\right):t\geq0\right\}$ se define la fluctuaci\'on m\'axima de $Z\left(t\right)$ en el intervalo $\left(T_{n-1},T_{n}\right]$:
\begin{eqnarray*}
M_{n}=\sup_{T_{n-1}<t\leq T_{n}}|Z\left(t\right)-Z\left(T_{n-1}\right)|
\end{eqnarray*}
\end{Def}

\begin{Teo}
Sup\'ongase que $n^{-1}T_{n}\rightarrow\mu$ c.s. cuando $n\rightarrow\infty$, donde $\mu\leq\infty$ es una constante o variable aleatoria. Sea $a$ una constante o variable aleatoria que puede ser infinita cuando $\mu$ es finita, y considere las expresiones l\'imite:
\begin{eqnarray}
lim_{n\rightarrow\infty}n^{-1}Z\left(T_{n}\right)&=&a,\textrm{ c.s.}\\
lim_{t\rightarrow\infty}t^{-1}Z\left(t\right)&=&a/\mu,\textrm{ c.s.}
\end{eqnarray}
La segunda expresi\'on implica la primera. Conversamente, la primera implica la segunda si el proceso $Z\left(t\right)$ es creciente, o si $lim_{n\rightarrow\infty}n^{-1}M_{n}=0$ c.s.
\end{Teo}

\begin{Coro}
Si $N\left(t\right)$ es un proceso de renovaci\'on, y $\left(Z\left(T_{n}\right)-Z\left(T_{n-1}\right),M_{n}\right)$, para $n\geq1$, son variables aleatorias independientes e id\'enticamente distribuidas con media finita, entonces,
\begin{eqnarray}
lim_{t\rightarrow\infty}t^{-1}Z\left(t\right)\rightarrow\frac{\esp\left[Z\left(T_{1}\right)-Z\left(T_{0}\right)\right]}{\esp\left[T_{1}\right]},\textrm{ c.s. cuando  }t\rightarrow\infty.
\end{eqnarray}
\end{Coro}



%___________________________________________________________________________________________
%
\subsection{Propiedades de los Procesos de Renovaci\'on}
%___________________________________________________________________________________________
%

Los tiempos $T_{n}$ est\'an relacionados con los conteos de $N\left(t\right)$ por

\begin{eqnarray*}
\left\{N\left(t\right)\geq n\right\}&=&\left\{T_{n}\leq t\right\}\\
T_{N\left(t\right)}\leq &t&<T_{N\left(t\right)+1},
\end{eqnarray*}

adem\'as $N\left(T_{n}\right)=n$, y 

\begin{eqnarray*}
N\left(t\right)=\max\left\{n:T_{n}\leq t\right\}=\min\left\{n:T_{n+1}>t\right\}
\end{eqnarray*}

Por propiedades de la convoluci\'on se sabe que

\begin{eqnarray*}
P\left\{T_{n}\leq t\right\}=F^{n\star}\left(t\right)
\end{eqnarray*}
que es la $n$-\'esima convoluci\'on de $F$. Entonces 

\begin{eqnarray*}
\left\{N\left(t\right)\geq n\right\}&=&\left\{T_{n}\leq t\right\}\\
P\left\{N\left(t\right)\leq n\right\}&=&1-F^{\left(n+1\right)\star}\left(t\right)
\end{eqnarray*}

Adem\'as usando el hecho de que $\esp\left[N\left(t\right)\right]=\sum_{n=1}^{\infty}P\left\{N\left(t\right)\geq n\right\}$
se tiene que

\begin{eqnarray*}
\esp\left[N\left(t\right)\right]=\sum_{n=1}^{\infty}F^{n\star}\left(t\right)
\end{eqnarray*}

\begin{Prop}
Para cada $t\geq0$, la funci\'on generadora de momentos $\esp\left[e^{\alpha N\left(t\right)}\right]$ existe para alguna $\alpha$ en una vecindad del 0, y de aqu\'i que $\esp\left[N\left(t\right)^{m}\right]<\infty$, para $m\geq1$.
\end{Prop}


\begin{Note}
Si el primer tiempo de renovaci\'on $\xi_{1}$ no tiene la misma distribuci\'on que el resto de las $\xi_{n}$, para $n\geq2$, a $N\left(t\right)$ se le llama Proceso de Renovaci\'on retardado, donde si $\xi$ tiene distribuci\'on $G$, entonces el tiempo $T_{n}$ de la $n$-\'esima renovaci\'on tiene distribuci\'on $G\star F^{\left(n-1\right)\star}\left(t\right)$
\end{Note}


\begin{Teo}
Para una constante $\mu\leq\infty$ ( o variable aleatoria), las siguientes expresiones son equivalentes:

\begin{eqnarray}
lim_{n\rightarrow\infty}n^{-1}T_{n}&=&\mu,\textrm{ c.s.}\\
lim_{t\rightarrow\infty}t^{-1}N\left(t\right)&=&1/\mu,\textrm{ c.s.}
\end{eqnarray}
\end{Teo}


Es decir, $T_{n}$ satisface la Ley Fuerte de los Grandes N\'umeros s\'i y s\'olo s\'i $N\left/t\right)$ la cumple.


\begin{Coro}[Ley Fuerte de los Grandes N\'umeros para Procesos de Renovaci\'on]
Si $N\left(t\right)$ es un proceso de renovaci\'on cuyos tiempos de inter-renovaci\'on tienen media $\mu\leq\infty$, entonces
\begin{eqnarray}
t^{-1}N\left(t\right)\rightarrow 1/\mu,\textrm{ c.s. cuando }t\rightarrow\infty.
\end{eqnarray}

\end{Coro}


Considerar el proceso estoc\'astico de valores reales $\left\{Z\left(t\right):t\geq0\right\}$ en el mismo espacio de probabilidad que $N\left(t\right)$

\begin{Def}
Para el proceso $\left\{Z\left(t\right):t\geq0\right\}$ se define la fluctuaci\'on m\'axima de $Z\left(t\right)$ en el intervalo $\left(T_{n-1},T_{n}\right]$:
\begin{eqnarray*}
M_{n}=\sup_{T_{n-1}<t\leq T_{n}}|Z\left(t\right)-Z\left(T_{n-1}\right)|
\end{eqnarray*}
\end{Def}

\begin{Teo}
Sup\'ongase que $n^{-1}T_{n}\rightarrow\mu$ c.s. cuando $n\rightarrow\infty$, donde $\mu\leq\infty$ es una constante o variable aleatoria. Sea $a$ una constante o variable aleatoria que puede ser infinita cuando $\mu$ es finita, y considere las expresiones l\'imite:
\begin{eqnarray}
lim_{n\rightarrow\infty}n^{-1}Z\left(T_{n}\right)&=&a,\textrm{ c.s.}\\
lim_{t\rightarrow\infty}t^{-1}Z\left(t\right)&=&a/\mu,\textrm{ c.s.}
\end{eqnarray}
La segunda expresi\'on implica la primera. Conversamente, la primera implica la segunda si el proceso $Z\left(t\right)$ es creciente, o si $lim_{n\rightarrow\infty}n^{-1}M_{n}=0$ c.s.
\end{Teo}

\begin{Coro}
Si $N\left(t\right)$ es un proceso de renovaci\'on, y $\left(Z\left(T_{n}\right)-Z\left(T_{n-1}\right),M_{n}\right)$, para $n\geq1$, son variables aleatorias independientes e id\'enticamente distribuidas con media finita, entonces,
\begin{eqnarray}
lim_{t\rightarrow\infty}t^{-1}Z\left(t\right)\rightarrow\frac{\esp\left[Z\left(T_{1}\right)-Z\left(T_{0}\right)\right]}{\esp\left[T_{1}\right]},\textrm{ c.s. cuando  }t\rightarrow\infty.
\end{eqnarray}
\end{Coro}


%___________________________________________________________________________________________
%
\subsection{Propiedades de los Procesos de Renovaci\'on}
%___________________________________________________________________________________________
%

Los tiempos $T_{n}$ est\'an relacionados con los conteos de $N\left(t\right)$ por

\begin{eqnarray*}
\left\{N\left(t\right)\geq n\right\}&=&\left\{T_{n}\leq t\right\}\\
T_{N\left(t\right)}\leq &t&<T_{N\left(t\right)+1},
\end{eqnarray*}

adem\'as $N\left(T_{n}\right)=n$, y 

\begin{eqnarray*}
N\left(t\right)=\max\left\{n:T_{n}\leq t\right\}=\min\left\{n:T_{n+1}>t\right\}
\end{eqnarray*}

Por propiedades de la convoluci\'on se sabe que

\begin{eqnarray*}
P\left\{T_{n}\leq t\right\}=F^{n\star}\left(t\right)
\end{eqnarray*}
que es la $n$-\'esima convoluci\'on de $F$. Entonces 

\begin{eqnarray*}
\left\{N\left(t\right)\geq n\right\}&=&\left\{T_{n}\leq t\right\}\\
P\left\{N\left(t\right)\leq n\right\}&=&1-F^{\left(n+1\right)\star}\left(t\right)
\end{eqnarray*}

Adem\'as usando el hecho de que $\esp\left[N\left(t\right)\right]=\sum_{n=1}^{\infty}P\left\{N\left(t\right)\geq n\right\}$
se tiene que

\begin{eqnarray*}
\esp\left[N\left(t\right)\right]=\sum_{n=1}^{\infty}F^{n\star}\left(t\right)
\end{eqnarray*}

\begin{Prop}
Para cada $t\geq0$, la funci\'on generadora de momentos $\esp\left[e^{\alpha N\left(t\right)}\right]$ existe para alguna $\alpha$ en una vecindad del 0, y de aqu\'i que $\esp\left[N\left(t\right)^{m}\right]<\infty$, para $m\geq1$.
\end{Prop}


\begin{Note}
Si el primer tiempo de renovaci\'on $\xi_{1}$ no tiene la misma distribuci\'on que el resto de las $\xi_{n}$, para $n\geq2$, a $N\left(t\right)$ se le llama Proceso de Renovaci\'on retardado, donde si $\xi$ tiene distribuci\'on $G$, entonces el tiempo $T_{n}$ de la $n$-\'esima renovaci\'on tiene distribuci\'on $G\star F^{\left(n-1\right)\star}\left(t\right)$
\end{Note}


\begin{Teo}
Para una constante $\mu\leq\infty$ ( o variable aleatoria), las siguientes expresiones son equivalentes:

\begin{eqnarray}
lim_{n\rightarrow\infty}n^{-1}T_{n}&=&\mu,\textrm{ c.s.}\\
lim_{t\rightarrow\infty}t^{-1}N\left(t\right)&=&1/\mu,\textrm{ c.s.}
\end{eqnarray}
\end{Teo}


Es decir, $T_{n}$ satisface la Ley Fuerte de los Grandes N\'umeros s\'i y s\'olo s\'i $N\left/t\right)$ la cumple.


\begin{Coro}[Ley Fuerte de los Grandes N\'umeros para Procesos de Renovaci\'on]
Si $N\left(t\right)$ es un proceso de renovaci\'on cuyos tiempos de inter-renovaci\'on tienen media $\mu\leq\infty$, entonces
\begin{eqnarray}
t^{-1}N\left(t\right)\rightarrow 1/\mu,\textrm{ c.s. cuando }t\rightarrow\infty.
\end{eqnarray}

\end{Coro}


Considerar el proceso estoc\'astico de valores reales $\left\{Z\left(t\right):t\geq0\right\}$ en el mismo espacio de probabilidad que $N\left(t\right)$

\begin{Def}
Para el proceso $\left\{Z\left(t\right):t\geq0\right\}$ se define la fluctuaci\'on m\'axima de $Z\left(t\right)$ en el intervalo $\left(T_{n-1},T_{n}\right]$:
\begin{eqnarray*}
M_{n}=\sup_{T_{n-1}<t\leq T_{n}}|Z\left(t\right)-Z\left(T_{n-1}\right)|
\end{eqnarray*}
\end{Def}

\begin{Teo}
Sup\'ongase que $n^{-1}T_{n}\rightarrow\mu$ c.s. cuando $n\rightarrow\infty$, donde $\mu\leq\infty$ es una constante o variable aleatoria. Sea $a$ una constante o variable aleatoria que puede ser infinita cuando $\mu$ es finita, y considere las expresiones l\'imite:
\begin{eqnarray}
lim_{n\rightarrow\infty}n^{-1}Z\left(T_{n}\right)&=&a,\textrm{ c.s.}\\
lim_{t\rightarrow\infty}t^{-1}Z\left(t\right)&=&a/\mu,\textrm{ c.s.}
\end{eqnarray}
La segunda expresi\'on implica la primera. Conversamente, la primera implica la segunda si el proceso $Z\left(t\right)$ es creciente, o si $lim_{n\rightarrow\infty}n^{-1}M_{n}=0$ c.s.
\end{Teo}

\begin{Coro}
Si $N\left(t\right)$ es un proceso de renovaci\'on, y $\left(Z\left(T_{n}\right)-Z\left(T_{n-1}\right),M_{n}\right)$, para $n\geq1$, son variables aleatorias independientes e id\'enticamente distribuidas con media finita, entonces,
\begin{eqnarray}
lim_{t\rightarrow\infty}t^{-1}Z\left(t\right)\rightarrow\frac{\esp\left[Z\left(T_{1}\right)-Z\left(T_{0}\right)\right]}{\esp\left[T_{1}\right]},\textrm{ c.s. cuando  }t\rightarrow\infty.
\end{eqnarray}
\end{Coro}

%___________________________________________________________________________________________
%
\subsection{Propiedades de los Procesos de Renovaci\'on}
%___________________________________________________________________________________________
%

Los tiempos $T_{n}$ est\'an relacionados con los conteos de $N\left(t\right)$ por

\begin{eqnarray*}
\left\{N\left(t\right)\geq n\right\}&=&\left\{T_{n}\leq t\right\}\\
T_{N\left(t\right)}\leq &t&<T_{N\left(t\right)+1},
\end{eqnarray*}

adem\'as $N\left(T_{n}\right)=n$, y 

\begin{eqnarray*}
N\left(t\right)=\max\left\{n:T_{n}\leq t\right\}=\min\left\{n:T_{n+1}>t\right\}
\end{eqnarray*}

Por propiedades de la convoluci\'on se sabe que

\begin{eqnarray*}
P\left\{T_{n}\leq t\right\}=F^{n\star}\left(t\right)
\end{eqnarray*}
que es la $n$-\'esima convoluci\'on de $F$. Entonces 

\begin{eqnarray*}
\left\{N\left(t\right)\geq n\right\}&=&\left\{T_{n}\leq t\right\}\\
P\left\{N\left(t\right)\leq n\right\}&=&1-F^{\left(n+1\right)\star}\left(t\right)
\end{eqnarray*}

Adem\'as usando el hecho de que $\esp\left[N\left(t\right)\right]=\sum_{n=1}^{\infty}P\left\{N\left(t\right)\geq n\right\}$
se tiene que

\begin{eqnarray*}
\esp\left[N\left(t\right)\right]=\sum_{n=1}^{\infty}F^{n\star}\left(t\right)
\end{eqnarray*}

\begin{Prop}
Para cada $t\geq0$, la funci\'on generadora de momentos $\esp\left[e^{\alpha N\left(t\right)}\right]$ existe para alguna $\alpha$ en una vecindad del 0, y de aqu\'i que $\esp\left[N\left(t\right)^{m}\right]<\infty$, para $m\geq1$.
\end{Prop}


\begin{Note}
Si el primer tiempo de renovaci\'on $\xi_{1}$ no tiene la misma distribuci\'on que el resto de las $\xi_{n}$, para $n\geq2$, a $N\left(t\right)$ se le llama Proceso de Renovaci\'on retardado, donde si $\xi$ tiene distribuci\'on $G$, entonces el tiempo $T_{n}$ de la $n$-\'esima renovaci\'on tiene distribuci\'on $G\star F^{\left(n-1\right)\star}\left(t\right)$
\end{Note}


\begin{Teo}
Para una constante $\mu\leq\infty$ ( o variable aleatoria), las siguientes expresiones son equivalentes:

\begin{eqnarray}
lim_{n\rightarrow\infty}n^{-1}T_{n}&=&\mu,\textrm{ c.s.}\\
lim_{t\rightarrow\infty}t^{-1}N\left(t\right)&=&1/\mu,\textrm{ c.s.}
\end{eqnarray}
\end{Teo}


Es decir, $T_{n}$ satisface la Ley Fuerte de los Grandes N\'umeros s\'i y s\'olo s\'i $N\left/t\right)$ la cumple.


\begin{Coro}[Ley Fuerte de los Grandes N\'umeros para Procesos de Renovaci\'on]
Si $N\left(t\right)$ es un proceso de renovaci\'on cuyos tiempos de inter-renovaci\'on tienen media $\mu\leq\infty$, entonces
\begin{eqnarray}
t^{-1}N\left(t\right)\rightarrow 1/\mu,\textrm{ c.s. cuando }t\rightarrow\infty.
\end{eqnarray}

\end{Coro}


Considerar el proceso estoc\'astico de valores reales $\left\{Z\left(t\right):t\geq0\right\}$ en el mismo espacio de probabilidad que $N\left(t\right)$

\begin{Def}
Para el proceso $\left\{Z\left(t\right):t\geq0\right\}$ se define la fluctuaci\'on m\'axima de $Z\left(t\right)$ en el intervalo $\left(T_{n-1},T_{n}\right]$:
\begin{eqnarray*}
M_{n}=\sup_{T_{n-1}<t\leq T_{n}}|Z\left(t\right)-Z\left(T_{n-1}\right)|
\end{eqnarray*}
\end{Def}

\begin{Teo}
Sup\'ongase que $n^{-1}T_{n}\rightarrow\mu$ c.s. cuando $n\rightarrow\infty$, donde $\mu\leq\infty$ es una constante o variable aleatoria. Sea $a$ una constante o variable aleatoria que puede ser infinita cuando $\mu$ es finita, y considere las expresiones l\'imite:
\begin{eqnarray}
lim_{n\rightarrow\infty}n^{-1}Z\left(T_{n}\right)&=&a,\textrm{ c.s.}\\
lim_{t\rightarrow\infty}t^{-1}Z\left(t\right)&=&a/\mu,\textrm{ c.s.}
\end{eqnarray}
La segunda expresi\'on implica la primera. Conversamente, la primera implica la segunda si el proceso $Z\left(t\right)$ es creciente, o si $lim_{n\rightarrow\infty}n^{-1}M_{n}=0$ c.s.
\end{Teo}

\begin{Coro}
Si $N\left(t\right)$ es un proceso de renovaci\'on, y $\left(Z\left(T_{n}\right)-Z\left(T_{n-1}\right),M_{n}\right)$, para $n\geq1$, son variables aleatorias independientes e id\'enticamente distribuidas con media finita, entonces,
\begin{eqnarray}
lim_{t\rightarrow\infty}t^{-1}Z\left(t\right)\rightarrow\frac{\esp\left[Z\left(T_{1}\right)-Z\left(T_{0}\right)\right]}{\esp\left[T_{1}\right]},\textrm{ c.s. cuando  }t\rightarrow\infty.
\end{eqnarray}
\end{Coro}
%___________________________________________________________________________________________
%
\subsection{Propiedades de los Procesos de Renovaci\'on}
%___________________________________________________________________________________________
%

Los tiempos $T_{n}$ est\'an relacionados con los conteos de $N\left(t\right)$ por

\begin{eqnarray*}
\left\{N\left(t\right)\geq n\right\}&=&\left\{T_{n}\leq t\right\}\\
T_{N\left(t\right)}\leq &t&<T_{N\left(t\right)+1},
\end{eqnarray*}

adem\'as $N\left(T_{n}\right)=n$, y 

\begin{eqnarray*}
N\left(t\right)=\max\left\{n:T_{n}\leq t\right\}=\min\left\{n:T_{n+1}>t\right\}
\end{eqnarray*}

Por propiedades de la convoluci\'on se sabe que

\begin{eqnarray*}
P\left\{T_{n}\leq t\right\}=F^{n\star}\left(t\right)
\end{eqnarray*}
que es la $n$-\'esima convoluci\'on de $F$. Entonces 

\begin{eqnarray*}
\left\{N\left(t\right)\geq n\right\}&=&\left\{T_{n}\leq t\right\}\\
P\left\{N\left(t\right)\leq n\right\}&=&1-F^{\left(n+1\right)\star}\left(t\right)
\end{eqnarray*}

Adem\'as usando el hecho de que $\esp\left[N\left(t\right)\right]=\sum_{n=1}^{\infty}P\left\{N\left(t\right)\geq n\right\}$
se tiene que

\begin{eqnarray*}
\esp\left[N\left(t\right)\right]=\sum_{n=1}^{\infty}F^{n\star}\left(t\right)
\end{eqnarray*}

\begin{Prop}
Para cada $t\geq0$, la funci\'on generadora de momentos $\esp\left[e^{\alpha N\left(t\right)}\right]$ existe para alguna $\alpha$ en una vecindad del 0, y de aqu\'i que $\esp\left[N\left(t\right)^{m}\right]<\infty$, para $m\geq1$.
\end{Prop}


\begin{Note}
Si el primer tiempo de renovaci\'on $\xi_{1}$ no tiene la misma distribuci\'on que el resto de las $\xi_{n}$, para $n\geq2$, a $N\left(t\right)$ se le llama Proceso de Renovaci\'on retardado, donde si $\xi$ tiene distribuci\'on $G$, entonces el tiempo $T_{n}$ de la $n$-\'esima renovaci\'on tiene distribuci\'on $G\star F^{\left(n-1\right)\star}\left(t\right)$
\end{Note}


\begin{Teo}
Para una constante $\mu\leq\infty$ ( o variable aleatoria), las siguientes expresiones son equivalentes:

\begin{eqnarray}
lim_{n\rightarrow\infty}n^{-1}T_{n}&=&\mu,\textrm{ c.s.}\\
lim_{t\rightarrow\infty}t^{-1}N\left(t\right)&=&1/\mu,\textrm{ c.s.}
\end{eqnarray}
\end{Teo}


Es decir, $T_{n}$ satisface la Ley Fuerte de los Grandes N\'umeros s\'i y s\'olo s\'i $N\left/t\right)$ la cumple.


\begin{Coro}[Ley Fuerte de los Grandes N\'umeros para Procesos de Renovaci\'on]
Si $N\left(t\right)$ es un proceso de renovaci\'on cuyos tiempos de inter-renovaci\'on tienen media $\mu\leq\infty$, entonces
\begin{eqnarray}
t^{-1}N\left(t\right)\rightarrow 1/\mu,\textrm{ c.s. cuando }t\rightarrow\infty.
\end{eqnarray}

\end{Coro}


Considerar el proceso estoc\'astico de valores reales $\left\{Z\left(t\right):t\geq0\right\}$ en el mismo espacio de probabilidad que $N\left(t\right)$

\begin{Def}
Para el proceso $\left\{Z\left(t\right):t\geq0\right\}$ se define la fluctuaci\'on m\'axima de $Z\left(t\right)$ en el intervalo $\left(T_{n-1},T_{n}\right]$:
\begin{eqnarray*}
M_{n}=\sup_{T_{n-1}<t\leq T_{n}}|Z\left(t\right)-Z\left(T_{n-1}\right)|
\end{eqnarray*}
\end{Def}

\begin{Teo}
Sup\'ongase que $n^{-1}T_{n}\rightarrow\mu$ c.s. cuando $n\rightarrow\infty$, donde $\mu\leq\infty$ es una constante o variable aleatoria. Sea $a$ una constante o variable aleatoria que puede ser infinita cuando $\mu$ es finita, y considere las expresiones l\'imite:
\begin{eqnarray}
lim_{n\rightarrow\infty}n^{-1}Z\left(T_{n}\right)&=&a,\textrm{ c.s.}\\
lim_{t\rightarrow\infty}t^{-1}Z\left(t\right)&=&a/\mu,\textrm{ c.s.}
\end{eqnarray}
La segunda expresi\'on implica la primera. Conversamente, la primera implica la segunda si el proceso $Z\left(t\right)$ es creciente, o si $lim_{n\rightarrow\infty}n^{-1}M_{n}=0$ c.s.
\end{Teo}

\begin{Coro}
Si $N\left(t\right)$ es un proceso de renovaci\'on, y $\left(Z\left(T_{n}\right)-Z\left(T_{n-1}\right),M_{n}\right)$, para $n\geq1$, son variables aleatorias independientes e id\'enticamente distribuidas con media finita, entonces,
\begin{eqnarray}
lim_{t\rightarrow\infty}t^{-1}Z\left(t\right)\rightarrow\frac{\esp\left[Z\left(T_{1}\right)-Z\left(T_{0}\right)\right]}{\esp\left[T_{1}\right]},\textrm{ c.s. cuando  }t\rightarrow\infty.
\end{eqnarray}
\end{Coro}


%___________________________________________________________________________________________
%
\subsection{Propiedades de los Procesos de Renovaci\'on}
%___________________________________________________________________________________________
%

Los tiempos $T_{n}$ est\'an relacionados con los conteos de $N\left(t\right)$ por

\begin{eqnarray*}
\left\{N\left(t\right)\geq n\right\}&=&\left\{T_{n}\leq t\right\}\\
T_{N\left(t\right)}\leq &t&<T_{N\left(t\right)+1},
\end{eqnarray*}

adem\'as $N\left(T_{n}\right)=n$, y 

\begin{eqnarray*}
N\left(t\right)=\max\left\{n:T_{n}\leq t\right\}=\min\left\{n:T_{n+1}>t\right\}
\end{eqnarray*}

Por propiedades de la convoluci\'on se sabe que

\begin{eqnarray*}
P\left\{T_{n}\leq t\right\}=F^{n\star}\left(t\right)
\end{eqnarray*}
que es la $n$-\'esima convoluci\'on de $F$. Entonces 

\begin{eqnarray*}
\left\{N\left(t\right)\geq n\right\}&=&\left\{T_{n}\leq t\right\}\\
P\left\{N\left(t\right)\leq n\right\}&=&1-F^{\left(n+1\right)\star}\left(t\right)
\end{eqnarray*}

Adem\'as usando el hecho de que $\esp\left[N\left(t\right)\right]=\sum_{n=1}^{\infty}P\left\{N\left(t\right)\geq n\right\}$
se tiene que

\begin{eqnarray*}
\esp\left[N\left(t\right)\right]=\sum_{n=1}^{\infty}F^{n\star}\left(t\right)
\end{eqnarray*}

\begin{Prop}
Para cada $t\geq0$, la funci\'on generadora de momentos $\esp\left[e^{\alpha N\left(t\right)}\right]$ existe para alguna $\alpha$ en una vecindad del 0, y de aqu\'i que $\esp\left[N\left(t\right)^{m}\right]<\infty$, para $m\geq1$.
\end{Prop}


\begin{Note}
Si el primer tiempo de renovaci\'on $\xi_{1}$ no tiene la misma distribuci\'on que el resto de las $\xi_{n}$, para $n\geq2$, a $N\left(t\right)$ se le llama Proceso de Renovaci\'on retardado, donde si $\xi$ tiene distribuci\'on $G$, entonces el tiempo $T_{n}$ de la $n$-\'esima renovaci\'on tiene distribuci\'on $G\star F^{\left(n-1\right)\star}\left(t\right)$
\end{Note}


\begin{Teo}
Para una constante $\mu\leq\infty$ ( o variable aleatoria), las siguientes expresiones son equivalentes:

\begin{eqnarray}
lim_{n\rightarrow\infty}n^{-1}T_{n}&=&\mu,\textrm{ c.s.}\\
lim_{t\rightarrow\infty}t^{-1}N\left(t\right)&=&1/\mu,\textrm{ c.s.}
\end{eqnarray}
\end{Teo}


Es decir, $T_{n}$ satisface la Ley Fuerte de los Grandes N\'umeros s\'i y s\'olo s\'i $N\left/t\right)$ la cumple.


\begin{Coro}[Ley Fuerte de los Grandes N\'umeros para Procesos de Renovaci\'on]
Si $N\left(t\right)$ es un proceso de renovaci\'on cuyos tiempos de inter-renovaci\'on tienen media $\mu\leq\infty$, entonces
\begin{eqnarray}
t^{-1}N\left(t\right)\rightarrow 1/\mu,\textrm{ c.s. cuando }t\rightarrow\infty.
\end{eqnarray}

\end{Coro}


Considerar el proceso estoc\'astico de valores reales $\left\{Z\left(t\right):t\geq0\right\}$ en el mismo espacio de probabilidad que $N\left(t\right)$

\begin{Def}
Para el proceso $\left\{Z\left(t\right):t\geq0\right\}$ se define la fluctuaci\'on m\'axima de $Z\left(t\right)$ en el intervalo $\left(T_{n-1},T_{n}\right]$:
\begin{eqnarray*}
M_{n}=\sup_{T_{n-1}<t\leq T_{n}}|Z\left(t\right)-Z\left(T_{n-1}\right)|
\end{eqnarray*}
\end{Def}

\begin{Teo}
Sup\'ongase que $n^{-1}T_{n}\rightarrow\mu$ c.s. cuando $n\rightarrow\infty$, donde $\mu\leq\infty$ es una constante o variable aleatoria. Sea $a$ una constante o variable aleatoria que puede ser infinita cuando $\mu$ es finita, y considere las expresiones l\'imite:
\begin{eqnarray}
lim_{n\rightarrow\infty}n^{-1}Z\left(T_{n}\right)&=&a,\textrm{ c.s.}\\
lim_{t\rightarrow\infty}t^{-1}Z\left(t\right)&=&a/\mu,\textrm{ c.s.}
\end{eqnarray}
La segunda expresi\'on implica la primera. Conversamente, la primera implica la segunda si el proceso $Z\left(t\right)$ es creciente, o si $lim_{n\rightarrow\infty}n^{-1}M_{n}=0$ c.s.
\end{Teo}

\begin{Coro}
Si $N\left(t\right)$ es un proceso de renovaci\'on, y $\left(Z\left(T_{n}\right)-Z\left(T_{n-1}\right),M_{n}\right)$, para $n\geq1$, son variables aleatorias independientes e id\'enticamente distribuidas con media finita, entonces,
\begin{eqnarray}
lim_{t\rightarrow\infty}t^{-1}Z\left(t\right)\rightarrow\frac{\esp\left[Z\left(T_{1}\right)-Z\left(T_{0}\right)\right]}{\esp\left[T_{1}\right]},\textrm{ c.s. cuando  }t\rightarrow\infty.
\end{eqnarray}
\end{Coro}

%___________________________________________________________________________________________
%
\subsection{Propiedades de los Procesos de Renovaci\'on}
%___________________________________________________________________________________________
%

Los tiempos $T_{n}$ est\'an relacionados con los conteos de $N\left(t\right)$ por

\begin{eqnarray*}
\left\{N\left(t\right)\geq n\right\}&=&\left\{T_{n}\leq t\right\}\\
T_{N\left(t\right)}\leq &t&<T_{N\left(t\right)+1},
\end{eqnarray*}

adem\'as $N\left(T_{n}\right)=n$, y 

\begin{eqnarray*}
N\left(t\right)=\max\left\{n:T_{n}\leq t\right\}=\min\left\{n:T_{n+1}>t\right\}
\end{eqnarray*}

Por propiedades de la convoluci\'on se sabe que

\begin{eqnarray*}
P\left\{T_{n}\leq t\right\}=F^{n\star}\left(t\right)
\end{eqnarray*}
que es la $n$-\'esima convoluci\'on de $F$. Entonces 

\begin{eqnarray*}
\left\{N\left(t\right)\geq n\right\}&=&\left\{T_{n}\leq t\right\}\\
P\left\{N\left(t\right)\leq n\right\}&=&1-F^{\left(n+1\right)\star}\left(t\right)
\end{eqnarray*}

Adem\'as usando el hecho de que $\esp\left[N\left(t\right)\right]=\sum_{n=1}^{\infty}P\left\{N\left(t\right)\geq n\right\}$
se tiene que

\begin{eqnarray*}
\esp\left[N\left(t\right)\right]=\sum_{n=1}^{\infty}F^{n\star}\left(t\right)
\end{eqnarray*}

\begin{Prop}
Para cada $t\geq0$, la funci\'on generadora de momentos $\esp\left[e^{\alpha N\left(t\right)}\right]$ existe para alguna $\alpha$ en una vecindad del 0, y de aqu\'i que $\esp\left[N\left(t\right)^{m}\right]<\infty$, para $m\geq1$.
\end{Prop}


\begin{Note}
Si el primer tiempo de renovaci\'on $\xi_{1}$ no tiene la misma distribuci\'on que el resto de las $\xi_{n}$, para $n\geq2$, a $N\left(t\right)$ se le llama Proceso de Renovaci\'on retardado, donde si $\xi$ tiene distribuci\'on $G$, entonces el tiempo $T_{n}$ de la $n$-\'esima renovaci\'on tiene distribuci\'on $G\star F^{\left(n-1\right)\star}\left(t\right)$
\end{Note}


\begin{Teo}
Para una constante $\mu\leq\infty$ ( o variable aleatoria), las siguientes expresiones son equivalentes:

\begin{eqnarray}
lim_{n\rightarrow\infty}n^{-1}T_{n}&=&\mu,\textrm{ c.s.}\\
lim_{t\rightarrow\infty}t^{-1}N\left(t\right)&=&1/\mu,\textrm{ c.s.}
\end{eqnarray}
\end{Teo}


Es decir, $T_{n}$ satisface la Ley Fuerte de los Grandes N\'umeros s\'i y s\'olo s\'i $N\left/t\right)$ la cumple.


\begin{Coro}[Ley Fuerte de los Grandes N\'umeros para Procesos de Renovaci\'on]
Si $N\left(t\right)$ es un proceso de renovaci\'on cuyos tiempos de inter-renovaci\'on tienen media $\mu\leq\infty$, entonces
\begin{eqnarray}
t^{-1}N\left(t\right)\rightarrow 1/\mu,\textrm{ c.s. cuando }t\rightarrow\infty.
\end{eqnarray}

\end{Coro}


Considerar el proceso estoc\'astico de valores reales $\left\{Z\left(t\right):t\geq0\right\}$ en el mismo espacio de probabilidad que $N\left(t\right)$

\begin{Def}
Para el proceso $\left\{Z\left(t\right):t\geq0\right\}$ se define la fluctuaci\'on m\'axima de $Z\left(t\right)$ en el intervalo $\left(T_{n-1},T_{n}\right]$:
\begin{eqnarray*}
M_{n}=\sup_{T_{n-1}<t\leq T_{n}}|Z\left(t\right)-Z\left(T_{n-1}\right)|
\end{eqnarray*}
\end{Def}

\begin{Teo}
Sup\'ongase que $n^{-1}T_{n}\rightarrow\mu$ c.s. cuando $n\rightarrow\infty$, donde $\mu\leq\infty$ es una constante o variable aleatoria. Sea $a$ una constante o variable aleatoria que puede ser infinita cuando $\mu$ es finita, y considere las expresiones l\'imite:
\begin{eqnarray}
lim_{n\rightarrow\infty}n^{-1}Z\left(T_{n}\right)&=&a,\textrm{ c.s.}\\
lim_{t\rightarrow\infty}t^{-1}Z\left(t\right)&=&a/\mu,\textrm{ c.s.}
\end{eqnarray}
La segunda expresi\'on implica la primera. Conversamente, la primera implica la segunda si el proceso $Z\left(t\right)$ es creciente, o si $lim_{n\rightarrow\infty}n^{-1}M_{n}=0$ c.s.
\end{Teo}

\begin{Coro}
Si $N\left(t\right)$ es un proceso de renovaci\'on, y $\left(Z\left(T_{n}\right)-Z\left(T_{n-1}\right),M_{n}\right)$, para $n\geq1$, son variables aleatorias independientes e id\'enticamente distribuidas con media finita, entonces,
\begin{eqnarray}
lim_{t\rightarrow\infty}t^{-1}Z\left(t\right)\rightarrow\frac{\esp\left[Z\left(T_{1}\right)-Z\left(T_{0}\right)\right]}{\esp\left[T_{1}\right]},\textrm{ c.s. cuando  }t\rightarrow\infty.
\end{eqnarray}
\end{Coro}


%___________________________________________________________________________________________
%
\subsection{Propiedades de los Procesos de Renovaci\'on}
%___________________________________________________________________________________________
%

Los tiempos $T_{n}$ est\'an relacionados con los conteos de $N\left(t\right)$ por

\begin{eqnarray*}
\left\{N\left(t\right)\geq n\right\}&=&\left\{T_{n}\leq t\right\}\\
T_{N\left(t\right)}\leq &t&<T_{N\left(t\right)+1},
\end{eqnarray*}

adem\'as $N\left(T_{n}\right)=n$, y 

\begin{eqnarray*}
N\left(t\right)=\max\left\{n:T_{n}\leq t\right\}=\min\left\{n:T_{n+1}>t\right\}
\end{eqnarray*}

Por propiedades de la convoluci\'on se sabe que

\begin{eqnarray*}
P\left\{T_{n}\leq t\right\}=F^{n\star}\left(t\right)
\end{eqnarray*}
que es la $n$-\'esima convoluci\'on de $F$. Entonces 

\begin{eqnarray*}
\left\{N\left(t\right)\geq n\right\}&=&\left\{T_{n}\leq t\right\}\\
P\left\{N\left(t\right)\leq n\right\}&=&1-F^{\left(n+1\right)\star}\left(t\right)
\end{eqnarray*}

Adem\'as usando el hecho de que $\esp\left[N\left(t\right)\right]=\sum_{n=1}^{\infty}P\left\{N\left(t\right)\geq n\right\}$
se tiene que

\begin{eqnarray*}
\esp\left[N\left(t\right)\right]=\sum_{n=1}^{\infty}F^{n\star}\left(t\right)
\end{eqnarray*}

\begin{Prop}
Para cada $t\geq0$, la funci\'on generadora de momentos $\esp\left[e^{\alpha N\left(t\right)}\right]$ existe para alguna $\alpha$ en una vecindad del 0, y de aqu\'i que $\esp\left[N\left(t\right)^{m}\right]<\infty$, para $m\geq1$.
\end{Prop}


\begin{Note}
Si el primer tiempo de renovaci\'on $\xi_{1}$ no tiene la misma distribuci\'on que el resto de las $\xi_{n}$, para $n\geq2$, a $N\left(t\right)$ se le llama Proceso de Renovaci\'on retardado, donde si $\xi$ tiene distribuci\'on $G$, entonces el tiempo $T_{n}$ de la $n$-\'esima renovaci\'on tiene distribuci\'on $G\star F^{\left(n-1\right)\star}\left(t\right)$
\end{Note}


\begin{Teo}
Para una constante $\mu\leq\infty$ ( o variable aleatoria), las siguientes expresiones son equivalentes:

\begin{eqnarray}
lim_{n\rightarrow\infty}n^{-1}T_{n}&=&\mu,\textrm{ c.s.}\\
lim_{t\rightarrow\infty}t^{-1}N\left(t\right)&=&1/\mu,\textrm{ c.s.}
\end{eqnarray}
\end{Teo}


Es decir, $T_{n}$ satisface la Ley Fuerte de los Grandes N\'umeros s\'i y s\'olo s\'i $N\left/t\right)$ la cumple.


\begin{Coro}[Ley Fuerte de los Grandes N\'umeros para Procesos de Renovaci\'on]
Si $N\left(t\right)$ es un proceso de renovaci\'on cuyos tiempos de inter-renovaci\'on tienen media $\mu\leq\infty$, entonces
\begin{eqnarray}
t^{-1}N\left(t\right)\rightarrow 1/\mu,\textrm{ c.s. cuando }t\rightarrow\infty.
\end{eqnarray}

\end{Coro}


Considerar el proceso estoc\'astico de valores reales $\left\{Z\left(t\right):t\geq0\right\}$ en el mismo espacio de probabilidad que $N\left(t\right)$

\begin{Def}
Para el proceso $\left\{Z\left(t\right):t\geq0\right\}$ se define la fluctuaci\'on m\'axima de $Z\left(t\right)$ en el intervalo $\left(T_{n-1},T_{n}\right]$:
\begin{eqnarray*}
M_{n}=\sup_{T_{n-1}<t\leq T_{n}}|Z\left(t\right)-Z\left(T_{n-1}\right)|
\end{eqnarray*}
\end{Def}

\begin{Teo}
Sup\'ongase que $n^{-1}T_{n}\rightarrow\mu$ c.s. cuando $n\rightarrow\infty$, donde $\mu\leq\infty$ es una constante o variable aleatoria. Sea $a$ una constante o variable aleatoria que puede ser infinita cuando $\mu$ es finita, y considere las expresiones l\'imite:
\begin{eqnarray}
lim_{n\rightarrow\infty}n^{-1}Z\left(T_{n}\right)&=&a,\textrm{ c.s.}\\
lim_{t\rightarrow\infty}t^{-1}Z\left(t\right)&=&a/\mu,\textrm{ c.s.}
\end{eqnarray}
La segunda expresi\'on implica la primera. Conversamente, la primera implica la segunda si el proceso $Z\left(t\right)$ es creciente, o si $lim_{n\rightarrow\infty}n^{-1}M_{n}=0$ c.s.
\end{Teo}

\begin{Coro}
Si $N\left(t\right)$ es un proceso de renovaci\'on, y $\left(Z\left(T_{n}\right)-Z\left(T_{n-1}\right),M_{n}\right)$, para $n\geq1$, son variables aleatorias independientes e id\'enticamente distribuidas con media finita, entonces,
\begin{eqnarray}
lim_{t\rightarrow\infty}t^{-1}Z\left(t\right)\rightarrow\frac{\esp\left[Z\left(T_{1}\right)-Z\left(T_{0}\right)\right]}{\esp\left[T_{1}\right]},\textrm{ c.s. cuando  }t\rightarrow\infty.
\end{eqnarray}
\end{Coro}

%___________________________________________________________________________________________
%
\subsection{Propiedades de los Procesos de Renovaci\'on}
%___________________________________________________________________________________________
%

Los tiempos $T_{n}$ est\'an relacionados con los conteos de $N\left(t\right)$ por

\begin{eqnarray*}
\left\{N\left(t\right)\geq n\right\}&=&\left\{T_{n}\leq t\right\}\\
T_{N\left(t\right)}\leq &t&<T_{N\left(t\right)+1},
\end{eqnarray*}

adem\'as $N\left(T_{n}\right)=n$, y 

\begin{eqnarray*}
N\left(t\right)=\max\left\{n:T_{n}\leq t\right\}=\min\left\{n:T_{n+1}>t\right\}
\end{eqnarray*}

Por propiedades de la convoluci\'on se sabe que

\begin{eqnarray*}
P\left\{T_{n}\leq t\right\}=F^{n\star}\left(t\right)
\end{eqnarray*}
que es la $n$-\'esima convoluci\'on de $F$. Entonces 

\begin{eqnarray*}
\left\{N\left(t\right)\geq n\right\}&=&\left\{T_{n}\leq t\right\}\\
P\left\{N\left(t\right)\leq n\right\}&=&1-F^{\left(n+1\right)\star}\left(t\right)
\end{eqnarray*}

Adem\'as usando el hecho de que $\esp\left[N\left(t\right)\right]=\sum_{n=1}^{\infty}P\left\{N\left(t\right)\geq n\right\}$
se tiene que

\begin{eqnarray*}
\esp\left[N\left(t\right)\right]=\sum_{n=1}^{\infty}F^{n\star}\left(t\right)
\end{eqnarray*}

\begin{Prop}
Para cada $t\geq0$, la funci\'on generadora de momentos $\esp\left[e^{\alpha N\left(t\right)}\right]$ existe para alguna $\alpha$ en una vecindad del 0, y de aqu\'i que $\esp\left[N\left(t\right)^{m}\right]<\infty$, para $m\geq1$.
\end{Prop}


\begin{Note}
Si el primer tiempo de renovaci\'on $\xi_{1}$ no tiene la misma distribuci\'on que el resto de las $\xi_{n}$, para $n\geq2$, a $N\left(t\right)$ se le llama Proceso de Renovaci\'on retardado, donde si $\xi$ tiene distribuci\'on $G$, entonces el tiempo $T_{n}$ de la $n$-\'esima renovaci\'on tiene distribuci\'on $G\star F^{\left(n-1\right)\star}\left(t\right)$
\end{Note}


\begin{Teo}
Para una constante $\mu\leq\infty$ ( o variable aleatoria), las siguientes expresiones son equivalentes:

\begin{eqnarray}
lim_{n\rightarrow\infty}n^{-1}T_{n}&=&\mu,\textrm{ c.s.}\\
lim_{t\rightarrow\infty}t^{-1}N\left(t\right)&=&1/\mu,\textrm{ c.s.}
\end{eqnarray}
\end{Teo}


Es decir, $T_{n}$ satisface la Ley Fuerte de los Grandes N\'umeros s\'i y s\'olo s\'i $N\left/t\right)$ la cumple.


\begin{Coro}[Ley Fuerte de los Grandes N\'umeros para Procesos de Renovaci\'on]
Si $N\left(t\right)$ es un proceso de renovaci\'on cuyos tiempos de inter-renovaci\'on tienen media $\mu\leq\infty$, entonces
\begin{eqnarray}
t^{-1}N\left(t\right)\rightarrow 1/\mu,\textrm{ c.s. cuando }t\rightarrow\infty.
\end{eqnarray}

\end{Coro}


Considerar el proceso estoc\'astico de valores reales $\left\{Z\left(t\right):t\geq0\right\}$ en el mismo espacio de probabilidad que $N\left(t\right)$

\begin{Def}
Para el proceso $\left\{Z\left(t\right):t\geq0\right\}$ se define la fluctuaci\'on m\'axima de $Z\left(t\right)$ en el intervalo $\left(T_{n-1},T_{n}\right]$:
\begin{eqnarray*}
M_{n}=\sup_{T_{n-1}<t\leq T_{n}}|Z\left(t\right)-Z\left(T_{n-1}\right)|
\end{eqnarray*}
\end{Def}

\begin{Teo}
Sup\'ongase que $n^{-1}T_{n}\rightarrow\mu$ c.s. cuando $n\rightarrow\infty$, donde $\mu\leq\infty$ es una constante o variable aleatoria. Sea $a$ una constante o variable aleatoria que puede ser infinita cuando $\mu$ es finita, y considere las expresiones l\'imite:
\begin{eqnarray}
lim_{n\rightarrow\infty}n^{-1}Z\left(T_{n}\right)&=&a,\textrm{ c.s.}\\
lim_{t\rightarrow\infty}t^{-1}Z\left(t\right)&=&a/\mu,\textrm{ c.s.}
\end{eqnarray}
La segunda expresi\'on implica la primera. Conversamente, la primera implica la segunda si el proceso $Z\left(t\right)$ es creciente, o si $lim_{n\rightarrow\infty}n^{-1}M_{n}=0$ c.s.
\end{Teo}

\begin{Coro}
Si $N\left(t\right)$ es un proceso de renovaci\'on, y $\left(Z\left(T_{n}\right)-Z\left(T_{n-1}\right),M_{n}\right)$, para $n\geq1$, son variables aleatorias independientes e id\'enticamente distribuidas con media finita, entonces,
\begin{eqnarray}
lim_{t\rightarrow\infty}t^{-1}Z\left(t\right)\rightarrow\frac{\esp\left[Z\left(T_{1}\right)-Z\left(T_{0}\right)\right]}{\esp\left[T_{1}\right]},\textrm{ c.s. cuando  }t\rightarrow\infty.
\end{eqnarray}
\end{Coro}
%___________________________________________________________________________________________
%
\subsection{Propiedades de los Procesos de Renovaci\'on}
%___________________________________________________________________________________________
%

Los tiempos $T_{n}$ est\'an relacionados con los conteos de $N\left(t\right)$ por

\begin{eqnarray*}
\left\{N\left(t\right)\geq n\right\}&=&\left\{T_{n}\leq t\right\}\\
T_{N\left(t\right)}\leq &t&<T_{N\left(t\right)+1},
\end{eqnarray*}

adem\'as $N\left(T_{n}\right)=n$, y 

\begin{eqnarray*}
N\left(t\right)=\max\left\{n:T_{n}\leq t\right\}=\min\left\{n:T_{n+1}>t\right\}
\end{eqnarray*}

Por propiedades de la convoluci\'on se sabe que

\begin{eqnarray*}
P\left\{T_{n}\leq t\right\}=F^{n\star}\left(t\right)
\end{eqnarray*}
que es la $n$-\'esima convoluci\'on de $F$. Entonces 

\begin{eqnarray*}
\left\{N\left(t\right)\geq n\right\}&=&\left\{T_{n}\leq t\right\}\\
P\left\{N\left(t\right)\leq n\right\}&=&1-F^{\left(n+1\right)\star}\left(t\right)
\end{eqnarray*}

Adem\'as usando el hecho de que $\esp\left[N\left(t\right)\right]=\sum_{n=1}^{\infty}P\left\{N\left(t\right)\geq n\right\}$
se tiene que

\begin{eqnarray*}
\esp\left[N\left(t\right)\right]=\sum_{n=1}^{\infty}F^{n\star}\left(t\right)
\end{eqnarray*}

\begin{Prop}
Para cada $t\geq0$, la funci\'on generadora de momentos $\esp\left[e^{\alpha N\left(t\right)}\right]$ existe para alguna $\alpha$ en una vecindad del 0, y de aqu\'i que $\esp\left[N\left(t\right)^{m}\right]<\infty$, para $m\geq1$.
\end{Prop}


\begin{Note}
Si el primer tiempo de renovaci\'on $\xi_{1}$ no tiene la misma distribuci\'on que el resto de las $\xi_{n}$, para $n\geq2$, a $N\left(t\right)$ se le llama Proceso de Renovaci\'on retardado, donde si $\xi$ tiene distribuci\'on $G$, entonces el tiempo $T_{n}$ de la $n$-\'esima renovaci\'on tiene distribuci\'on $G\star F^{\left(n-1\right)\star}\left(t\right)$
\end{Note}


\begin{Teo}
Para una constante $\mu\leq\infty$ ( o variable aleatoria), las siguientes expresiones son equivalentes:

\begin{eqnarray}
lim_{n\rightarrow\infty}n^{-1}T_{n}&=&\mu,\textrm{ c.s.}\\
lim_{t\rightarrow\infty}t^{-1}N\left(t\right)&=&1/\mu,\textrm{ c.s.}
\end{eqnarray}
\end{Teo}


Es decir, $T_{n}$ satisface la Ley Fuerte de los Grandes N\'umeros s\'i y s\'olo s\'i $N\left/t\right)$ la cumple.


\begin{Coro}[Ley Fuerte de los Grandes N\'umeros para Procesos de Renovaci\'on]
Si $N\left(t\right)$ es un proceso de renovaci\'on cuyos tiempos de inter-renovaci\'on tienen media $\mu\leq\infty$, entonces
\begin{eqnarray}
t^{-1}N\left(t\right)\rightarrow 1/\mu,\textrm{ c.s. cuando }t\rightarrow\infty.
\end{eqnarray}

\end{Coro}


Considerar el proceso estoc\'astico de valores reales $\left\{Z\left(t\right):t\geq0\right\}$ en el mismo espacio de probabilidad que $N\left(t\right)$

\begin{Def}
Para el proceso $\left\{Z\left(t\right):t\geq0\right\}$ se define la fluctuaci\'on m\'axima de $Z\left(t\right)$ en el intervalo $\left(T_{n-1},T_{n}\right]$:
\begin{eqnarray*}
M_{n}=\sup_{T_{n-1}<t\leq T_{n}}|Z\left(t\right)-Z\left(T_{n-1}\right)|
\end{eqnarray*}
\end{Def}

\begin{Teo}
Sup\'ongase que $n^{-1}T_{n}\rightarrow\mu$ c.s. cuando $n\rightarrow\infty$, donde $\mu\leq\infty$ es una constante o variable aleatoria. Sea $a$ una constante o variable aleatoria que puede ser infinita cuando $\mu$ es finita, y considere las expresiones l\'imite:
\begin{eqnarray}
lim_{n\rightarrow\infty}n^{-1}Z\left(T_{n}\right)&=&a,\textrm{ c.s.}\\
lim_{t\rightarrow\infty}t^{-1}Z\left(t\right)&=&a/\mu,\textrm{ c.s.}
\end{eqnarray}
La segunda expresi\'on implica la primera. Conversamente, la primera implica la segunda si el proceso $Z\left(t\right)$ es creciente, o si $lim_{n\rightarrow\infty}n^{-1}M_{n}=0$ c.s.
\end{Teo}

\begin{Coro}
Si $N\left(t\right)$ es un proceso de renovaci\'on, y $\left(Z\left(T_{n}\right)-Z\left(T_{n-1}\right),M_{n}\right)$, para $n\geq1$, son variables aleatorias independientes e id\'enticamente distribuidas con media finita, entonces,
\begin{eqnarray}
lim_{t\rightarrow\infty}t^{-1}Z\left(t\right)\rightarrow\frac{\esp\left[Z\left(T_{1}\right)-Z\left(T_{0}\right)\right]}{\esp\left[T_{1}\right]},\textrm{ c.s. cuando  }t\rightarrow\infty.
\end{eqnarray}
\end{Coro}

%___________________________________________________________________________________________
%
\subsection{Propiedades de los Procesos de Renovaci\'on}
%___________________________________________________________________________________________
%

Los tiempos $T_{n}$ est\'an relacionados con los conteos de $N\left(t\right)$ por

\begin{eqnarray*}
\left\{N\left(t\right)\geq n\right\}&=&\left\{T_{n}\leq t\right\}\\
T_{N\left(t\right)}\leq &t&<T_{N\left(t\right)+1},
\end{eqnarray*}

adem\'as $N\left(T_{n}\right)=n$, y 

\begin{eqnarray*}
N\left(t\right)=\max\left\{n:T_{n}\leq t\right\}=\min\left\{n:T_{n+1}>t\right\}
\end{eqnarray*}

Por propiedades de la convoluci\'on se sabe que

\begin{eqnarray*}
P\left\{T_{n}\leq t\right\}=F^{n\star}\left(t\right)
\end{eqnarray*}
que es la $n$-\'esima convoluci\'on de $F$. Entonces 

\begin{eqnarray*}
\left\{N\left(t\right)\geq n\right\}&=&\left\{T_{n}\leq t\right\}\\
P\left\{N\left(t\right)\leq n\right\}&=&1-F^{\left(n+1\right)\star}\left(t\right)
\end{eqnarray*}

Adem\'as usando el hecho de que $\esp\left[N\left(t\right)\right]=\sum_{n=1}^{\infty}P\left\{N\left(t\right)\geq n\right\}$
se tiene que

\begin{eqnarray*}
\esp\left[N\left(t\right)\right]=\sum_{n=1}^{\infty}F^{n\star}\left(t\right)
\end{eqnarray*}

\begin{Prop}
Para cada $t\geq0$, la funci\'on generadora de momentos $\esp\left[e^{\alpha N\left(t\right)}\right]$ existe para alguna $\alpha$ en una vecindad del 0, y de aqu\'i que $\esp\left[N\left(t\right)^{m}\right]<\infty$, para $m\geq1$.
\end{Prop}


\begin{Note}
Si el primer tiempo de renovaci\'on $\xi_{1}$ no tiene la misma distribuci\'on que el resto de las $\xi_{n}$, para $n\geq2$, a $N\left(t\right)$ se le llama Proceso de Renovaci\'on retardado, donde si $\xi$ tiene distribuci\'on $G$, entonces el tiempo $T_{n}$ de la $n$-\'esima renovaci\'on tiene distribuci\'on $G\star F^{\left(n-1\right)\star}\left(t\right)$
\end{Note}


\begin{Teo}
Para una constante $\mu\leq\infty$ ( o variable aleatoria), las siguientes expresiones son equivalentes:

\begin{eqnarray}
lim_{n\rightarrow\infty}n^{-1}T_{n}&=&\mu,\textrm{ c.s.}\\
lim_{t\rightarrow\infty}t^{-1}N\left(t\right)&=&1/\mu,\textrm{ c.s.}
\end{eqnarray}
\end{Teo}


Es decir, $T_{n}$ satisface la Ley Fuerte de los Grandes N\'umeros s\'i y s\'olo s\'i $N\left/t\right)$ la cumple.


\begin{Coro}[Ley Fuerte de los Grandes N\'umeros para Procesos de Renovaci\'on]
Si $N\left(t\right)$ es un proceso de renovaci\'on cuyos tiempos de inter-renovaci\'on tienen media $\mu\leq\infty$, entonces
\begin{eqnarray}
t^{-1}N\left(t\right)\rightarrow 1/\mu,\textrm{ c.s. cuando }t\rightarrow\infty.
\end{eqnarray}

\end{Coro}


Considerar el proceso estoc\'astico de valores reales $\left\{Z\left(t\right):t\geq0\right\}$ en el mismo espacio de probabilidad que $N\left(t\right)$

\begin{Def}
Para el proceso $\left\{Z\left(t\right):t\geq0\right\}$ se define la fluctuaci\'on m\'axima de $Z\left(t\right)$ en el intervalo $\left(T_{n-1},T_{n}\right]$:
\begin{eqnarray*}
M_{n}=\sup_{T_{n-1}<t\leq T_{n}}|Z\left(t\right)-Z\left(T_{n-1}\right)|
\end{eqnarray*}
\end{Def}

\begin{Teo}
Sup\'ongase que $n^{-1}T_{n}\rightarrow\mu$ c.s. cuando $n\rightarrow\infty$, donde $\mu\leq\infty$ es una constante o variable aleatoria. Sea $a$ una constante o variable aleatoria que puede ser infinita cuando $\mu$ es finita, y considere las expresiones l\'imite:
\begin{eqnarray}
lim_{n\rightarrow\infty}n^{-1}Z\left(T_{n}\right)&=&a,\textrm{ c.s.}\\
lim_{t\rightarrow\infty}t^{-1}Z\left(t\right)&=&a/\mu,\textrm{ c.s.}
\end{eqnarray}
La segunda expresi\'on implica la primera. Conversamente, la primera implica la segunda si el proceso $Z\left(t\right)$ es creciente, o si $lim_{n\rightarrow\infty}n^{-1}M_{n}=0$ c.s.
\end{Teo}

\begin{Coro}
Si $N\left(t\right)$ es un proceso de renovaci\'on, y $\left(Z\left(T_{n}\right)-Z\left(T_{n-1}\right),M_{n}\right)$, para $n\geq1$, son variables aleatorias independientes e id\'enticamente distribuidas con media finita, entonces,
\begin{eqnarray}
lim_{t\rightarrow\infty}t^{-1}Z\left(t\right)\rightarrow\frac{\esp\left[Z\left(T_{1}\right)-Z\left(T_{0}\right)\right]}{\esp\left[T_{1}\right]},\textrm{ c.s. cuando  }t\rightarrow\infty.
\end{eqnarray}
\end{Coro}


%___________________________________________________________________________________________
%
\subsection{Propiedades de los Procesos de Renovaci\'on}
%___________________________________________________________________________________________
%

Los tiempos $T_{n}$ est\'an relacionados con los conteos de $N\left(t\right)$ por

\begin{eqnarray*}
\left\{N\left(t\right)\geq n\right\}&=&\left\{T_{n}\leq t\right\}\\
T_{N\left(t\right)}\leq &t&<T_{N\left(t\right)+1},
\end{eqnarray*}

adem\'as $N\left(T_{n}\right)=n$, y 

\begin{eqnarray*}
N\left(t\right)=\max\left\{n:T_{n}\leq t\right\}=\min\left\{n:T_{n+1}>t\right\}
\end{eqnarray*}

Por propiedades de la convoluci\'on se sabe que

\begin{eqnarray*}
P\left\{T_{n}\leq t\right\}=F^{n\star}\left(t\right)
\end{eqnarray*}
que es la $n$-\'esima convoluci\'on de $F$. Entonces 

\begin{eqnarray*}
\left\{N\left(t\right)\geq n\right\}&=&\left\{T_{n}\leq t\right\}\\
P\left\{N\left(t\right)\leq n\right\}&=&1-F^{\left(n+1\right)\star}\left(t\right)
\end{eqnarray*}

Adem\'as usando el hecho de que $\esp\left[N\left(t\right)\right]=\sum_{n=1}^{\infty}P\left\{N\left(t\right)\geq n\right\}$
se tiene que

\begin{eqnarray*}
\esp\left[N\left(t\right)\right]=\sum_{n=1}^{\infty}F^{n\star}\left(t\right)
\end{eqnarray*}

\begin{Prop}
Para cada $t\geq0$, la funci\'on generadora de momentos $\esp\left[e^{\alpha N\left(t\right)}\right]$ existe para alguna $\alpha$ en una vecindad del 0, y de aqu\'i que $\esp\left[N\left(t\right)^{m}\right]<\infty$, para $m\geq1$.
\end{Prop}


\begin{Note}
Si el primer tiempo de renovaci\'on $\xi_{1}$ no tiene la misma distribuci\'on que el resto de las $\xi_{n}$, para $n\geq2$, a $N\left(t\right)$ se le llama Proceso de Renovaci\'on retardado, donde si $\xi$ tiene distribuci\'on $G$, entonces el tiempo $T_{n}$ de la $n$-\'esima renovaci\'on tiene distribuci\'on $G\star F^{\left(n-1\right)\star}\left(t\right)$
\end{Note}


\begin{Teo}
Para una constante $\mu\leq\infty$ ( o variable aleatoria), las siguientes expresiones son equivalentes:

\begin{eqnarray}
lim_{n\rightarrow\infty}n^{-1}T_{n}&=&\mu,\textrm{ c.s.}\\
lim_{t\rightarrow\infty}t^{-1}N\left(t\right)&=&1/\mu,\textrm{ c.s.}
\end{eqnarray}
\end{Teo}


Es decir, $T_{n}$ satisface la Ley Fuerte de los Grandes N\'umeros s\'i y s\'olo s\'i $N\left/t\right)$ la cumple.


\begin{Coro}[Ley Fuerte de los Grandes N\'umeros para Procesos de Renovaci\'on]
Si $N\left(t\right)$ es un proceso de renovaci\'on cuyos tiempos de inter-renovaci\'on tienen media $\mu\leq\infty$, entonces
\begin{eqnarray}
t^{-1}N\left(t\right)\rightarrow 1/\mu,\textrm{ c.s. cuando }t\rightarrow\infty.
\end{eqnarray}

\end{Coro}


Considerar el proceso estoc\'astico de valores reales $\left\{Z\left(t\right):t\geq0\right\}$ en el mismo espacio de probabilidad que $N\left(t\right)$

\begin{Def}
Para el proceso $\left\{Z\left(t\right):t\geq0\right\}$ se define la fluctuaci\'on m\'axima de $Z\left(t\right)$ en el intervalo $\left(T_{n-1},T_{n}\right]$:
\begin{eqnarray*}
M_{n}=\sup_{T_{n-1}<t\leq T_{n}}|Z\left(t\right)-Z\left(T_{n-1}\right)|
\end{eqnarray*}
\end{Def}

\begin{Teo}
Sup\'ongase que $n^{-1}T_{n}\rightarrow\mu$ c.s. cuando $n\rightarrow\infty$, donde $\mu\leq\infty$ es una constante o variable aleatoria. Sea $a$ una constante o variable aleatoria que puede ser infinita cuando $\mu$ es finita, y considere las expresiones l\'imite:
\begin{eqnarray}
lim_{n\rightarrow\infty}n^{-1}Z\left(T_{n}\right)&=&a,\textrm{ c.s.}\\
lim_{t\rightarrow\infty}t^{-1}Z\left(t\right)&=&a/\mu,\textrm{ c.s.}
\end{eqnarray}
La segunda expresi\'on implica la primera. Conversamente, la primera implica la segunda si el proceso $Z\left(t\right)$ es creciente, o si $lim_{n\rightarrow\infty}n^{-1}M_{n}=0$ c.s.
\end{Teo}

\begin{Coro}
Si $N\left(t\right)$ es un proceso de renovaci\'on, y $\left(Z\left(T_{n}\right)-Z\left(T_{n-1}\right),M_{n}\right)$, para $n\geq1$, son variables aleatorias independientes e id\'enticamente distribuidas con media finita, entonces,
\begin{eqnarray}
lim_{t\rightarrow\infty}t^{-1}Z\left(t\right)\rightarrow\frac{\esp\left[Z\left(T_{1}\right)-Z\left(T_{0}\right)\right]}{\esp\left[T_{1}\right]},\textrm{ c.s. cuando  }t\rightarrow\infty.
\end{eqnarray}
\end{Coro}

%___________________________________________________________________________________________
%
\subsection{Propiedades de los Procesos de Renovaci\'on}
%___________________________________________________________________________________________
%

Los tiempos $T_{n}$ est\'an relacionados con los conteos de $N\left(t\right)$ por

\begin{eqnarray*}
\left\{N\left(t\right)\geq n\right\}&=&\left\{T_{n}\leq t\right\}\\
T_{N\left(t\right)}\leq &t&<T_{N\left(t\right)+1},
\end{eqnarray*}

adem\'as $N\left(T_{n}\right)=n$, y 

\begin{eqnarray*}
N\left(t\right)=\max\left\{n:T_{n}\leq t\right\}=\min\left\{n:T_{n+1}>t\right\}
\end{eqnarray*}

Por propiedades de la convoluci\'on se sabe que

\begin{eqnarray*}
P\left\{T_{n}\leq t\right\}=F^{n\star}\left(t\right)
\end{eqnarray*}
que es la $n$-\'esima convoluci\'on de $F$. Entonces 

\begin{eqnarray*}
\left\{N\left(t\right)\geq n\right\}&=&\left\{T_{n}\leq t\right\}\\
P\left\{N\left(t\right)\leq n\right\}&=&1-F^{\left(n+1\right)\star}\left(t\right)
\end{eqnarray*}

Adem\'as usando el hecho de que $\esp\left[N\left(t\right)\right]=\sum_{n=1}^{\infty}P\left\{N\left(t\right)\geq n\right\}$
se tiene que

\begin{eqnarray*}
\esp\left[N\left(t\right)\right]=\sum_{n=1}^{\infty}F^{n\star}\left(t\right)
\end{eqnarray*}

\begin{Prop}
Para cada $t\geq0$, la funci\'on generadora de momentos $\esp\left[e^{\alpha N\left(t\right)}\right]$ existe para alguna $\alpha$ en una vecindad del 0, y de aqu\'i que $\esp\left[N\left(t\right)^{m}\right]<\infty$, para $m\geq1$.
\end{Prop}


\begin{Note}
Si el primer tiempo de renovaci\'on $\xi_{1}$ no tiene la misma distribuci\'on que el resto de las $\xi_{n}$, para $n\geq2$, a $N\left(t\right)$ se le llama Proceso de Renovaci\'on retardado, donde si $\xi$ tiene distribuci\'on $G$, entonces el tiempo $T_{n}$ de la $n$-\'esima renovaci\'on tiene distribuci\'on $G\star F^{\left(n-1\right)\star}\left(t\right)$
\end{Note}


\begin{Teo}
Para una constante $\mu\leq\infty$ ( o variable aleatoria), las siguientes expresiones son equivalentes:

\begin{eqnarray}
lim_{n\rightarrow\infty}n^{-1}T_{n}&=&\mu,\textrm{ c.s.}\\
lim_{t\rightarrow\infty}t^{-1}N\left(t\right)&=&1/\mu,\textrm{ c.s.}
\end{eqnarray}
\end{Teo}


Es decir, $T_{n}$ satisface la Ley Fuerte de los Grandes N\'umeros s\'i y s\'olo s\'i $N\left/t\right)$ la cumple.


\begin{Coro}[Ley Fuerte de los Grandes N\'umeros para Procesos de Renovaci\'on]
Si $N\left(t\right)$ es un proceso de renovaci\'on cuyos tiempos de inter-renovaci\'on tienen media $\mu\leq\infty$, entonces
\begin{eqnarray}
t^{-1}N\left(t\right)\rightarrow 1/\mu,\textrm{ c.s. cuando }t\rightarrow\infty.
\end{eqnarray}

\end{Coro}


Considerar el proceso estoc\'astico de valores reales $\left\{Z\left(t\right):t\geq0\right\}$ en el mismo espacio de probabilidad que $N\left(t\right)$

\begin{Def}
Para el proceso $\left\{Z\left(t\right):t\geq0\right\}$ se define la fluctuaci\'on m\'axima de $Z\left(t\right)$ en el intervalo $\left(T_{n-1},T_{n}\right]$:
\begin{eqnarray*}
M_{n}=\sup_{T_{n-1}<t\leq T_{n}}|Z\left(t\right)-Z\left(T_{n-1}\right)|
\end{eqnarray*}
\end{Def}

\begin{Teo}
Sup\'ongase que $n^{-1}T_{n}\rightarrow\mu$ c.s. cuando $n\rightarrow\infty$, donde $\mu\leq\infty$ es una constante o variable aleatoria. Sea $a$ una constante o variable aleatoria que puede ser infinita cuando $\mu$ es finita, y considere las expresiones l\'imite:
\begin{eqnarray}
lim_{n\rightarrow\infty}n^{-1}Z\left(T_{n}\right)&=&a,\textrm{ c.s.}\\
lim_{t\rightarrow\infty}t^{-1}Z\left(t\right)&=&a/\mu,\textrm{ c.s.}
\end{eqnarray}
La segunda expresi\'on implica la primera. Conversamente, la primera implica la segunda si el proceso $Z\left(t\right)$ es creciente, o si $lim_{n\rightarrow\infty}n^{-1}M_{n}=0$ c.s.
\end{Teo}

\begin{Coro}
Si $N\left(t\right)$ es un proceso de renovaci\'on, y $\left(Z\left(T_{n}\right)-Z\left(T_{n-1}\right),M_{n}\right)$, para $n\geq1$, son variables aleatorias independientes e id\'enticamente distribuidas con media finita, entonces,
\begin{eqnarray}
lim_{t\rightarrow\infty}t^{-1}Z\left(t\right)\rightarrow\frac{\esp\left[Z\left(T_{1}\right)-Z\left(T_{0}\right)\right]}{\esp\left[T_{1}\right]},\textrm{ c.s. cuando  }t\rightarrow\infty.
\end{eqnarray}
\end{Coro}



%___________________________________________________________________________________________
%
\subsection{Propiedades de los Procesos de Renovaci\'on}
%___________________________________________________________________________________________
%

Los tiempos $T_{n}$ est\'an relacionados con los conteos de $N\left(t\right)$ por

\begin{eqnarray*}
\left\{N\left(t\right)\geq n\right\}&=&\left\{T_{n}\leq t\right\}\\
T_{N\left(t\right)}\leq &t&<T_{N\left(t\right)+1},
\end{eqnarray*}

adem\'as $N\left(T_{n}\right)=n$, y 

\begin{eqnarray*}
N\left(t\right)=\max\left\{n:T_{n}\leq t\right\}=\min\left\{n:T_{n+1}>t\right\}
\end{eqnarray*}

Por propiedades de la convoluci\'on se sabe que

\begin{eqnarray*}
P\left\{T_{n}\leq t\right\}=F^{n\star}\left(t\right)
\end{eqnarray*}
que es la $n$-\'esima convoluci\'on de $F$. Entonces 

\begin{eqnarray*}
\left\{N\left(t\right)\geq n\right\}&=&\left\{T_{n}\leq t\right\}\\
P\left\{N\left(t\right)\leq n\right\}&=&1-F^{\left(n+1\right)\star}\left(t\right)
\end{eqnarray*}

Adem\'as usando el hecho de que $\esp\left[N\left(t\right)\right]=\sum_{n=1}^{\infty}P\left\{N\left(t\right)\geq n\right\}$
se tiene que

\begin{eqnarray*}
\esp\left[N\left(t\right)\right]=\sum_{n=1}^{\infty}F^{n\star}\left(t\right)
\end{eqnarray*}

\begin{Prop}
Para cada $t\geq0$, la funci\'on generadora de momentos $\esp\left[e^{\alpha N\left(t\right)}\right]$ existe para alguna $\alpha$ en una vecindad del 0, y de aqu\'i que $\esp\left[N\left(t\right)^{m}\right]<\infty$, para $m\geq1$.
\end{Prop}


\begin{Note}
Si el primer tiempo de renovaci\'on $\xi_{1}$ no tiene la misma distribuci\'on que el resto de las $\xi_{n}$, para $n\geq2$, a $N\left(t\right)$ se le llama Proceso de Renovaci\'on retardado, donde si $\xi$ tiene distribuci\'on $G$, entonces el tiempo $T_{n}$ de la $n$-\'esima renovaci\'on tiene distribuci\'on $G\star F^{\left(n-1\right)\star}\left(t\right)$
\end{Note}


\begin{Teo}
Para una constante $\mu\leq\infty$ ( o variable aleatoria), las siguientes expresiones son equivalentes:

\begin{eqnarray}
lim_{n\rightarrow\infty}n^{-1}T_{n}&=&\mu,\textrm{ c.s.}\\
lim_{t\rightarrow\infty}t^{-1}N\left(t\right)&=&1/\mu,\textrm{ c.s.}
\end{eqnarray}
\end{Teo}


Es decir, $T_{n}$ satisface la Ley Fuerte de los Grandes N\'umeros s\'i y s\'olo s\'i $N\left/t\right)$ la cumple.


\begin{Coro}[Ley Fuerte de los Grandes N\'umeros para Procesos de Renovaci\'on]
Si $N\left(t\right)$ es un proceso de renovaci\'on cuyos tiempos de inter-renovaci\'on tienen media $\mu\leq\infty$, entonces
\begin{eqnarray}
t^{-1}N\left(t\right)\rightarrow 1/\mu,\textrm{ c.s. cuando }t\rightarrow\infty.
\end{eqnarray}

\end{Coro}


Considerar el proceso estoc\'astico de valores reales $\left\{Z\left(t\right):t\geq0\right\}$ en el mismo espacio de probabilidad que $N\left(t\right)$

\begin{Def}
Para el proceso $\left\{Z\left(t\right):t\geq0\right\}$ se define la fluctuaci\'on m\'axima de $Z\left(t\right)$ en el intervalo $\left(T_{n-1},T_{n}\right]$:
\begin{eqnarray*}
M_{n}=\sup_{T_{n-1}<t\leq T_{n}}|Z\left(t\right)-Z\left(T_{n-1}\right)|
\end{eqnarray*}
\end{Def}

\begin{Teo}
Sup\'ongase que $n^{-1}T_{n}\rightarrow\mu$ c.s. cuando $n\rightarrow\infty$, donde $\mu\leq\infty$ es una constante o variable aleatoria. Sea $a$ una constante o variable aleatoria que puede ser infinita cuando $\mu$ es finita, y considere las expresiones l\'imite:
\begin{eqnarray}
lim_{n\rightarrow\infty}n^{-1}Z\left(T_{n}\right)&=&a,\textrm{ c.s.}\\
lim_{t\rightarrow\infty}t^{-1}Z\left(t\right)&=&a/\mu,\textrm{ c.s.}
\end{eqnarray}
La segunda expresi\'on implica la primera. Conversamente, la primera implica la segunda si el proceso $Z\left(t\right)$ es creciente, o si $lim_{n\rightarrow\infty}n^{-1}M_{n}=0$ c.s.
\end{Teo}

\begin{Coro}
Si $N\left(t\right)$ es un proceso de renovaci\'on, y $\left(Z\left(T_{n}\right)-Z\left(T_{n-1}\right),M_{n}\right)$, para $n\geq1$, son variables aleatorias independientes e id\'enticamente distribuidas con media finita, entonces,
\begin{eqnarray}
lim_{t\rightarrow\infty}t^{-1}Z\left(t\right)\rightarrow\frac{\esp\left[Z\left(T_{1}\right)-Z\left(T_{0}\right)\right]}{\esp\left[T_{1}\right]},\textrm{ c.s. cuando  }t\rightarrow\infty.
\end{eqnarray}
\end{Coro}

%___________________________________________________________________________________________
%
\subsection{Propiedades de los Procesos de Renovaci\'on}
%___________________________________________________________________________________________
%

Los tiempos $T_{n}$ est\'an relacionados con los conteos de $N\left(t\right)$ por

\begin{eqnarray*}
\left\{N\left(t\right)\geq n\right\}&=&\left\{T_{n}\leq t\right\}\\
T_{N\left(t\right)}\leq &t&<T_{N\left(t\right)+1},
\end{eqnarray*}

adem\'as $N\left(T_{n}\right)=n$, y 

\begin{eqnarray*}
N\left(t\right)=\max\left\{n:T_{n}\leq t\right\}=\min\left\{n:T_{n+1}>t\right\}
\end{eqnarray*}

Por propiedades de la convoluci\'on se sabe que

\begin{eqnarray*}
P\left\{T_{n}\leq t\right\}=F^{n\star}\left(t\right)
\end{eqnarray*}
que es la $n$-\'esima convoluci\'on de $F$. Entonces 

\begin{eqnarray*}
\left\{N\left(t\right)\geq n\right\}&=&\left\{T_{n}\leq t\right\}\\
P\left\{N\left(t\right)\leq n\right\}&=&1-F^{\left(n+1\right)\star}\left(t\right)
\end{eqnarray*}

Adem\'as usando el hecho de que $\esp\left[N\left(t\right)\right]=\sum_{n=1}^{\infty}P\left\{N\left(t\right)\geq n\right\}$
se tiene que

\begin{eqnarray*}
\esp\left[N\left(t\right)\right]=\sum_{n=1}^{\infty}F^{n\star}\left(t\right)
\end{eqnarray*}

\begin{Prop}
Para cada $t\geq0$, la funci\'on generadora de momentos $\esp\left[e^{\alpha N\left(t\right)}\right]$ existe para alguna $\alpha$ en una vecindad del 0, y de aqu\'i que $\esp\left[N\left(t\right)^{m}\right]<\infty$, para $m\geq1$.
\end{Prop}


\begin{Note}
Si el primer tiempo de renovaci\'on $\xi_{1}$ no tiene la misma distribuci\'on que el resto de las $\xi_{n}$, para $n\geq2$, a $N\left(t\right)$ se le llama Proceso de Renovaci\'on retardado, donde si $\xi$ tiene distribuci\'on $G$, entonces el tiempo $T_{n}$ de la $n$-\'esima renovaci\'on tiene distribuci\'on $G\star F^{\left(n-1\right)\star}\left(t\right)$
\end{Note}


\begin{Teo}
Para una constante $\mu\leq\infty$ ( o variable aleatoria), las siguientes expresiones son equivalentes:

\begin{eqnarray}
lim_{n\rightarrow\infty}n^{-1}T_{n}&=&\mu,\textrm{ c.s.}\\
lim_{t\rightarrow\infty}t^{-1}N\left(t\right)&=&1/\mu,\textrm{ c.s.}
\end{eqnarray}
\end{Teo}


Es decir, $T_{n}$ satisface la Ley Fuerte de los Grandes N\'umeros s\'i y s\'olo s\'i $N\left/t\right)$ la cumple.


\begin{Coro}[Ley Fuerte de los Grandes N\'umeros para Procesos de Renovaci\'on]
Si $N\left(t\right)$ es un proceso de renovaci\'on cuyos tiempos de inter-renovaci\'on tienen media $\mu\leq\infty$, entonces
\begin{eqnarray}
t^{-1}N\left(t\right)\rightarrow 1/\mu,\textrm{ c.s. cuando }t\rightarrow\infty.
\end{eqnarray}

\end{Coro}


Considerar el proceso estoc\'astico de valores reales $\left\{Z\left(t\right):t\geq0\right\}$ en el mismo espacio de probabilidad que $N\left(t\right)$

\begin{Def}
Para el proceso $\left\{Z\left(t\right):t\geq0\right\}$ se define la fluctuaci\'on m\'axima de $Z\left(t\right)$ en el intervalo $\left(T_{n-1},T_{n}\right]$:
\begin{eqnarray*}
M_{n}=\sup_{T_{n-1}<t\leq T_{n}}|Z\left(t\right)-Z\left(T_{n-1}\right)|
\end{eqnarray*}
\end{Def}

\begin{Teo}
Sup\'ongase que $n^{-1}T_{n}\rightarrow\mu$ c.s. cuando $n\rightarrow\infty$, donde $\mu\leq\infty$ es una constante o variable aleatoria. Sea $a$ una constante o variable aleatoria que puede ser infinita cuando $\mu$ es finita, y considere las expresiones l\'imite:
\begin{eqnarray}
lim_{n\rightarrow\infty}n^{-1}Z\left(T_{n}\right)&=&a,\textrm{ c.s.}\\
lim_{t\rightarrow\infty}t^{-1}Z\left(t\right)&=&a/\mu,\textrm{ c.s.}
\end{eqnarray}
La segunda expresi\'on implica la primera. Conversamente, la primera implica la segunda si el proceso $Z\left(t\right)$ es creciente, o si $lim_{n\rightarrow\infty}n^{-1}M_{n}=0$ c.s.
\end{Teo}

\begin{Coro}
Si $N\left(t\right)$ es un proceso de renovaci\'on, y $\left(Z\left(T_{n}\right)-Z\left(T_{n-1}\right),M_{n}\right)$, para $n\geq1$, son variables aleatorias independientes e id\'enticamente distribuidas con media finita, entonces,
\begin{eqnarray}
lim_{t\rightarrow\infty}t^{-1}Z\left(t\right)\rightarrow\frac{\esp\left[Z\left(T_{1}\right)-Z\left(T_{0}\right)\right]}{\esp\left[T_{1}\right]},\textrm{ c.s. cuando  }t\rightarrow\infty.
\end{eqnarray}
\end{Coro}



%___________________________________________________________________________________________
%
\subsection{Propiedades de los Procesos de Renovaci\'on}
%___________________________________________________________________________________________
%

Los tiempos $T_{n}$ est\'an relacionados con los conteos de $N\left(t\right)$ por

\begin{eqnarray*}
\left\{N\left(t\right)\geq n\right\}&=&\left\{T_{n}\leq t\right\}\\
T_{N\left(t\right)}\leq &t&<T_{N\left(t\right)+1},
\end{eqnarray*}

adem\'as $N\left(T_{n}\right)=n$, y 

\begin{eqnarray*}
N\left(t\right)=\max\left\{n:T_{n}\leq t\right\}=\min\left\{n:T_{n+1}>t\right\}
\end{eqnarray*}

Por propiedades de la convoluci\'on se sabe que

\begin{eqnarray*}
P\left\{T_{n}\leq t\right\}=F^{n\star}\left(t\right)
\end{eqnarray*}
que es la $n$-\'esima convoluci\'on de $F$. Entonces 

\begin{eqnarray*}
\left\{N\left(t\right)\geq n\right\}&=&\left\{T_{n}\leq t\right\}\\
P\left\{N\left(t\right)\leq n\right\}&=&1-F^{\left(n+1\right)\star}\left(t\right)
\end{eqnarray*}

Adem\'as usando el hecho de que $\esp\left[N\left(t\right)\right]=\sum_{n=1}^{\infty}P\left\{N\left(t\right)\geq n\right\}$
se tiene que

\begin{eqnarray*}
\esp\left[N\left(t\right)\right]=\sum_{n=1}^{\infty}F^{n\star}\left(t\right)
\end{eqnarray*}

\begin{Prop}
Para cada $t\geq0$, la funci\'on generadora de momentos $\esp\left[e^{\alpha N\left(t\right)}\right]$ existe para alguna $\alpha$ en una vecindad del 0, y de aqu\'i que $\esp\left[N\left(t\right)^{m}\right]<\infty$, para $m\geq1$.
\end{Prop}


\begin{Note}
Si el primer tiempo de renovaci\'on $\xi_{1}$ no tiene la misma distribuci\'on que el resto de las $\xi_{n}$, para $n\geq2$, a $N\left(t\right)$ se le llama Proceso de Renovaci\'on retardado, donde si $\xi$ tiene distribuci\'on $G$, entonces el tiempo $T_{n}$ de la $n$-\'esima renovaci\'on tiene distribuci\'on $G\star F^{\left(n-1\right)\star}\left(t\right)$
\end{Note}


\begin{Teo}
Para una constante $\mu\leq\infty$ ( o variable aleatoria), las siguientes expresiones son equivalentes:

\begin{eqnarray}
lim_{n\rightarrow\infty}n^{-1}T_{n}&=&\mu,\textrm{ c.s.}\\
lim_{t\rightarrow\infty}t^{-1}N\left(t\right)&=&1/\mu,\textrm{ c.s.}
\end{eqnarray}
\end{Teo}


Es decir, $T_{n}$ satisface la Ley Fuerte de los Grandes N\'umeros s\'i y s\'olo s\'i $N\left/t\right)$ la cumple.


\begin{Coro}[Ley Fuerte de los Grandes N\'umeros para Procesos de Renovaci\'on]
Si $N\left(t\right)$ es un proceso de renovaci\'on cuyos tiempos de inter-renovaci\'on tienen media $\mu\leq\infty$, entonces
\begin{eqnarray}
t^{-1}N\left(t\right)\rightarrow 1/\mu,\textrm{ c.s. cuando }t\rightarrow\infty.
\end{eqnarray}

\end{Coro}


Considerar el proceso estoc\'astico de valores reales $\left\{Z\left(t\right):t\geq0\right\}$ en el mismo espacio de probabilidad que $N\left(t\right)$

\begin{Def}
Para el proceso $\left\{Z\left(t\right):t\geq0\right\}$ se define la fluctuaci\'on m\'axima de $Z\left(t\right)$ en el intervalo $\left(T_{n-1},T_{n}\right]$:
\begin{eqnarray*}
M_{n}=\sup_{T_{n-1}<t\leq T_{n}}|Z\left(t\right)-Z\left(T_{n-1}\right)|
\end{eqnarray*}
\end{Def}

\begin{Teo}
Sup\'ongase que $n^{-1}T_{n}\rightarrow\mu$ c.s. cuando $n\rightarrow\infty$, donde $\mu\leq\infty$ es una constante o variable aleatoria. Sea $a$ una constante o variable aleatoria que puede ser infinita cuando $\mu$ es finita, y considere las expresiones l\'imite:
\begin{eqnarray}
lim_{n\rightarrow\infty}n^{-1}Z\left(T_{n}\right)&=&a,\textrm{ c.s.}\\
lim_{t\rightarrow\infty}t^{-1}Z\left(t\right)&=&a/\mu,\textrm{ c.s.}
\end{eqnarray}
La segunda expresi\'on implica la primera. Conversamente, la primera implica la segunda si el proceso $Z\left(t\right)$ es creciente, o si $lim_{n\rightarrow\infty}n^{-1}M_{n}=0$ c.s.
\end{Teo}

\begin{Coro}
Si $N\left(t\right)$ es un proceso de renovaci\'on, y $\left(Z\left(T_{n}\right)-Z\left(T_{n-1}\right),M_{n}\right)$, para $n\geq1$, son variables aleatorias independientes e id\'enticamente distribuidas con media finita, entonces,
\begin{eqnarray}
lim_{t\rightarrow\infty}t^{-1}Z\left(t\right)\rightarrow\frac{\esp\left[Z\left(T_{1}\right)-Z\left(T_{0}\right)\right]}{\esp\left[T_{1}\right]},\textrm{ c.s. cuando  }t\rightarrow\infty.
\end{eqnarray}
\end{Coro}


%___________________________________________________________________________________________
%
\subsection{Propiedades de los Procesos de Renovaci\'on}
%___________________________________________________________________________________________
%

Los tiempos $T_{n}$ est\'an relacionados con los conteos de $N\left(t\right)$ por

\begin{eqnarray*}
\left\{N\left(t\right)\geq n\right\}&=&\left\{T_{n}\leq t\right\}\\
T_{N\left(t\right)}\leq &t&<T_{N\left(t\right)+1},
\end{eqnarray*}

adem\'as $N\left(T_{n}\right)=n$, y 

\begin{eqnarray*}
N\left(t\right)=\max\left\{n:T_{n}\leq t\right\}=\min\left\{n:T_{n+1}>t\right\}
\end{eqnarray*}

Por propiedades de la convoluci\'on se sabe que

\begin{eqnarray*}
P\left\{T_{n}\leq t\right\}=F^{n\star}\left(t\right)
\end{eqnarray*}
que es la $n$-\'esima convoluci\'on de $F$. Entonces 

\begin{eqnarray*}
\left\{N\left(t\right)\geq n\right\}&=&\left\{T_{n}\leq t\right\}\\
P\left\{N\left(t\right)\leq n\right\}&=&1-F^{\left(n+1\right)\star}\left(t\right)
\end{eqnarray*}

Adem\'as usando el hecho de que $\esp\left[N\left(t\right)\right]=\sum_{n=1}^{\infty}P\left\{N\left(t\right)\geq n\right\}$
se tiene que

\begin{eqnarray*}
\esp\left[N\left(t\right)\right]=\sum_{n=1}^{\infty}F^{n\star}\left(t\right)
\end{eqnarray*}

\begin{Prop}
Para cada $t\geq0$, la funci\'on generadora de momentos $\esp\left[e^{\alpha N\left(t\right)}\right]$ existe para alguna $\alpha$ en una vecindad del 0, y de aqu\'i que $\esp\left[N\left(t\right)^{m}\right]<\infty$, para $m\geq1$.
\end{Prop}


\begin{Note}
Si el primer tiempo de renovaci\'on $\xi_{1}$ no tiene la misma distribuci\'on que el resto de las $\xi_{n}$, para $n\geq2$, a $N\left(t\right)$ se le llama Proceso de Renovaci\'on retardado, donde si $\xi$ tiene distribuci\'on $G$, entonces el tiempo $T_{n}$ de la $n$-\'esima renovaci\'on tiene distribuci\'on $G\star F^{\left(n-1\right)\star}\left(t\right)$
\end{Note}


\begin{Teo}
Para una constante $\mu\leq\infty$ ( o variable aleatoria), las siguientes expresiones son equivalentes:

\begin{eqnarray}
lim_{n\rightarrow\infty}n^{-1}T_{n}&=&\mu,\textrm{ c.s.}\\
lim_{t\rightarrow\infty}t^{-1}N\left(t\right)&=&1/\mu,\textrm{ c.s.}
\end{eqnarray}
\end{Teo}


Es decir, $T_{n}$ satisface la Ley Fuerte de los Grandes N\'umeros s\'i y s\'olo s\'i $N\left/t\right)$ la cumple.


\begin{Coro}[Ley Fuerte de los Grandes N\'umeros para Procesos de Renovaci\'on]
Si $N\left(t\right)$ es un proceso de renovaci\'on cuyos tiempos de inter-renovaci\'on tienen media $\mu\leq\infty$, entonces
\begin{eqnarray}
t^{-1}N\left(t\right)\rightarrow 1/\mu,\textrm{ c.s. cuando }t\rightarrow\infty.
\end{eqnarray}

\end{Coro}


Considerar el proceso estoc\'astico de valores reales $\left\{Z\left(t\right):t\geq0\right\}$ en el mismo espacio de probabilidad que $N\left(t\right)$

\begin{Def}
Para el proceso $\left\{Z\left(t\right):t\geq0\right\}$ se define la fluctuaci\'on m\'axima de $Z\left(t\right)$ en el intervalo $\left(T_{n-1},T_{n}\right]$:
\begin{eqnarray*}
M_{n}=\sup_{T_{n-1}<t\leq T_{n}}|Z\left(t\right)-Z\left(T_{n-1}\right)|
\end{eqnarray*}
\end{Def}

\begin{Teo}
Sup\'ongase que $n^{-1}T_{n}\rightarrow\mu$ c.s. cuando $n\rightarrow\infty$, donde $\mu\leq\infty$ es una constante o variable aleatoria. Sea $a$ una constante o variable aleatoria que puede ser infinita cuando $\mu$ es finita, y considere las expresiones l\'imite:
\begin{eqnarray}
lim_{n\rightarrow\infty}n^{-1}Z\left(T_{n}\right)&=&a,\textrm{ c.s.}\\
lim_{t\rightarrow\infty}t^{-1}Z\left(t\right)&=&a/\mu,\textrm{ c.s.}
\end{eqnarray}
La segunda expresi\'on implica la primera. Conversamente, la primera implica la segunda si el proceso $Z\left(t\right)$ es creciente, o si $lim_{n\rightarrow\infty}n^{-1}M_{n}=0$ c.s.
\end{Teo}

\begin{Coro}
Si $N\left(t\right)$ es un proceso de renovaci\'on, y $\left(Z\left(T_{n}\right)-Z\left(T_{n-1}\right),M_{n}\right)$, para $n\geq1$, son variables aleatorias independientes e id\'enticamente distribuidas con media finita, entonces,
\begin{eqnarray}
lim_{t\rightarrow\infty}t^{-1}Z\left(t\right)\rightarrow\frac{\esp\left[Z\left(T_{1}\right)-Z\left(T_{0}\right)\right]}{\esp\left[T_{1}\right]},\textrm{ c.s. cuando  }t\rightarrow\infty.
\end{eqnarray}
\end{Coro}

%___________________________________________________________________________________________
%
\subsection{Propiedades de los Procesos de Renovaci\'on}
%___________________________________________________________________________________________
%

Los tiempos $T_{n}$ est\'an relacionados con los conteos de $N\left(t\right)$ por

\begin{eqnarray*}
\left\{N\left(t\right)\geq n\right\}&=&\left\{T_{n}\leq t\right\}\\
T_{N\left(t\right)}\leq &t&<T_{N\left(t\right)+1},
\end{eqnarray*}

adem\'as $N\left(T_{n}\right)=n$, y 

\begin{eqnarray*}
N\left(t\right)=\max\left\{n:T_{n}\leq t\right\}=\min\left\{n:T_{n+1}>t\right\}
\end{eqnarray*}

Por propiedades de la convoluci\'on se sabe que

\begin{eqnarray*}
P\left\{T_{n}\leq t\right\}=F^{n\star}\left(t\right)
\end{eqnarray*}
que es la $n$-\'esima convoluci\'on de $F$. Entonces 

\begin{eqnarray*}
\left\{N\left(t\right)\geq n\right\}&=&\left\{T_{n}\leq t\right\}\\
P\left\{N\left(t\right)\leq n\right\}&=&1-F^{\left(n+1\right)\star}\left(t\right)
\end{eqnarray*}

Adem\'as usando el hecho de que $\esp\left[N\left(t\right)\right]=\sum_{n=1}^{\infty}P\left\{N\left(t\right)\geq n\right\}$
se tiene que

\begin{eqnarray*}
\esp\left[N\left(t\right)\right]=\sum_{n=1}^{\infty}F^{n\star}\left(t\right)
\end{eqnarray*}

\begin{Prop}
Para cada $t\geq0$, la funci\'on generadora de momentos $\esp\left[e^{\alpha N\left(t\right)}\right]$ existe para alguna $\alpha$ en una vecindad del 0, y de aqu\'i que $\esp\left[N\left(t\right)^{m}\right]<\infty$, para $m\geq1$.
\end{Prop}


\begin{Note}
Si el primer tiempo de renovaci\'on $\xi_{1}$ no tiene la misma distribuci\'on que el resto de las $\xi_{n}$, para $n\geq2$, a $N\left(t\right)$ se le llama Proceso de Renovaci\'on retardado, donde si $\xi$ tiene distribuci\'on $G$, entonces el tiempo $T_{n}$ de la $n$-\'esima renovaci\'on tiene distribuci\'on $G\star F^{\left(n-1\right)\star}\left(t\right)$
\end{Note}


\begin{Teo}
Para una constante $\mu\leq\infty$ ( o variable aleatoria), las siguientes expresiones son equivalentes:

\begin{eqnarray}
lim_{n\rightarrow\infty}n^{-1}T_{n}&=&\mu,\textrm{ c.s.}\\
lim_{t\rightarrow\infty}t^{-1}N\left(t\right)&=&1/\mu,\textrm{ c.s.}
\end{eqnarray}
\end{Teo}


Es decir, $T_{n}$ satisface la Ley Fuerte de los Grandes N\'umeros s\'i y s\'olo s\'i $N\left/t\right)$ la cumple.


\begin{Coro}[Ley Fuerte de los Grandes N\'umeros para Procesos de Renovaci\'on]
Si $N\left(t\right)$ es un proceso de renovaci\'on cuyos tiempos de inter-renovaci\'on tienen media $\mu\leq\infty$, entonces
\begin{eqnarray}
t^{-1}N\left(t\right)\rightarrow 1/\mu,\textrm{ c.s. cuando }t\rightarrow\infty.
\end{eqnarray}

\end{Coro}


Considerar el proceso estoc\'astico de valores reales $\left\{Z\left(t\right):t\geq0\right\}$ en el mismo espacio de probabilidad que $N\left(t\right)$

\begin{Def}
Para el proceso $\left\{Z\left(t\right):t\geq0\right\}$ se define la fluctuaci\'on m\'axima de $Z\left(t\right)$ en el intervalo $\left(T_{n-1},T_{n}\right]$:
\begin{eqnarray*}
M_{n}=\sup_{T_{n-1}<t\leq T_{n}}|Z\left(t\right)-Z\left(T_{n-1}\right)|
\end{eqnarray*}
\end{Def}

\begin{Teo}
Sup\'ongase que $n^{-1}T_{n}\rightarrow\mu$ c.s. cuando $n\rightarrow\infty$, donde $\mu\leq\infty$ es una constante o variable aleatoria. Sea $a$ una constante o variable aleatoria que puede ser infinita cuando $\mu$ es finita, y considere las expresiones l\'imite:
\begin{eqnarray}
lim_{n\rightarrow\infty}n^{-1}Z\left(T_{n}\right)&=&a,\textrm{ c.s.}\\
lim_{t\rightarrow\infty}t^{-1}Z\left(t\right)&=&a/\mu,\textrm{ c.s.}
\end{eqnarray}
La segunda expresi\'on implica la primera. Conversamente, la primera implica la segunda si el proceso $Z\left(t\right)$ es creciente, o si $lim_{n\rightarrow\infty}n^{-1}M_{n}=0$ c.s.
\end{Teo}

\begin{Coro}
Si $N\left(t\right)$ es un proceso de renovaci\'on, y $\left(Z\left(T_{n}\right)-Z\left(T_{n-1}\right),M_{n}\right)$, para $n\geq1$, son variables aleatorias independientes e id\'enticamente distribuidas con media finita, entonces,
\begin{eqnarray}
lim_{t\rightarrow\infty}t^{-1}Z\left(t\right)\rightarrow\frac{\esp\left[Z\left(T_{1}\right)-Z\left(T_{0}\right)\right]}{\esp\left[T_{1}\right]},\textrm{ c.s. cuando  }t\rightarrow\infty.
\end{eqnarray}
\end{Coro}
%___________________________________________________________________________________________
%
\subsection{Propiedades de los Procesos de Renovaci\'on}
%___________________________________________________________________________________________
%

Los tiempos $T_{n}$ est\'an relacionados con los conteos de $N\left(t\right)$ por

\begin{eqnarray*}
\left\{N\left(t\right)\geq n\right\}&=&\left\{T_{n}\leq t\right\}\\
T_{N\left(t\right)}\leq &t&<T_{N\left(t\right)+1},
\end{eqnarray*}

adem\'as $N\left(T_{n}\right)=n$, y 

\begin{eqnarray*}
N\left(t\right)=\max\left\{n:T_{n}\leq t\right\}=\min\left\{n:T_{n+1}>t\right\}
\end{eqnarray*}

Por propiedades de la convoluci\'on se sabe que

\begin{eqnarray*}
P\left\{T_{n}\leq t\right\}=F^{n\star}\left(t\right)
\end{eqnarray*}
que es la $n$-\'esima convoluci\'on de $F$. Entonces 

\begin{eqnarray*}
\left\{N\left(t\right)\geq n\right\}&=&\left\{T_{n}\leq t\right\}\\
P\left\{N\left(t\right)\leq n\right\}&=&1-F^{\left(n+1\right)\star}\left(t\right)
\end{eqnarray*}

Adem\'as usando el hecho de que $\esp\left[N\left(t\right)\right]=\sum_{n=1}^{\infty}P\left\{N\left(t\right)\geq n\right\}$
se tiene que

\begin{eqnarray*}
\esp\left[N\left(t\right)\right]=\sum_{n=1}^{\infty}F^{n\star}\left(t\right)
\end{eqnarray*}

\begin{Prop}
Para cada $t\geq0$, la funci\'on generadora de momentos $\esp\left[e^{\alpha N\left(t\right)}\right]$ existe para alguna $\alpha$ en una vecindad del 0, y de aqu\'i que $\esp\left[N\left(t\right)^{m}\right]<\infty$, para $m\geq1$.
\end{Prop}


\begin{Note}
Si el primer tiempo de renovaci\'on $\xi_{1}$ no tiene la misma distribuci\'on que el resto de las $\xi_{n}$, para $n\geq2$, a $N\left(t\right)$ se le llama Proceso de Renovaci\'on retardado, donde si $\xi$ tiene distribuci\'on $G$, entonces el tiempo $T_{n}$ de la $n$-\'esima renovaci\'on tiene distribuci\'on $G\star F^{\left(n-1\right)\star}\left(t\right)$
\end{Note}


\begin{Teo}
Para una constante $\mu\leq\infty$ ( o variable aleatoria), las siguientes expresiones son equivalentes:

\begin{eqnarray}
lim_{n\rightarrow\infty}n^{-1}T_{n}&=&\mu,\textrm{ c.s.}\\
lim_{t\rightarrow\infty}t^{-1}N\left(t\right)&=&1/\mu,\textrm{ c.s.}
\end{eqnarray}
\end{Teo}


Es decir, $T_{n}$ satisface la Ley Fuerte de los Grandes N\'umeros s\'i y s\'olo s\'i $N\left/t\right)$ la cumple.


\begin{Coro}[Ley Fuerte de los Grandes N\'umeros para Procesos de Renovaci\'on]
Si $N\left(t\right)$ es un proceso de renovaci\'on cuyos tiempos de inter-renovaci\'on tienen media $\mu\leq\infty$, entonces
\begin{eqnarray}
t^{-1}N\left(t\right)\rightarrow 1/\mu,\textrm{ c.s. cuando }t\rightarrow\infty.
\end{eqnarray}

\end{Coro}


Considerar el proceso estoc\'astico de valores reales $\left\{Z\left(t\right):t\geq0\right\}$ en el mismo espacio de probabilidad que $N\left(t\right)$

\begin{Def}
Para el proceso $\left\{Z\left(t\right):t\geq0\right\}$ se define la fluctuaci\'on m\'axima de $Z\left(t\right)$ en el intervalo $\left(T_{n-1},T_{n}\right]$:
\begin{eqnarray*}
M_{n}=\sup_{T_{n-1}<t\leq T_{n}}|Z\left(t\right)-Z\left(T_{n-1}\right)|
\end{eqnarray*}
\end{Def}

\begin{Teo}
Sup\'ongase que $n^{-1}T_{n}\rightarrow\mu$ c.s. cuando $n\rightarrow\infty$, donde $\mu\leq\infty$ es una constante o variable aleatoria. Sea $a$ una constante o variable aleatoria que puede ser infinita cuando $\mu$ es finita, y considere las expresiones l\'imite:
\begin{eqnarray}
lim_{n\rightarrow\infty}n^{-1}Z\left(T_{n}\right)&=&a,\textrm{ c.s.}\\
lim_{t\rightarrow\infty}t^{-1}Z\left(t\right)&=&a/\mu,\textrm{ c.s.}
\end{eqnarray}
La segunda expresi\'on implica la primera. Conversamente, la primera implica la segunda si el proceso $Z\left(t\right)$ es creciente, o si $lim_{n\rightarrow\infty}n^{-1}M_{n}=0$ c.s.
\end{Teo}

\begin{Coro}
Si $N\left(t\right)$ es un proceso de renovaci\'on, y $\left(Z\left(T_{n}\right)-Z\left(T_{n-1}\right),M_{n}\right)$, para $n\geq1$, son variables aleatorias independientes e id\'enticamente distribuidas con media finita, entonces,
\begin{eqnarray}
lim_{t\rightarrow\infty}t^{-1}Z\left(t\right)\rightarrow\frac{\esp\left[Z\left(T_{1}\right)-Z\left(T_{0}\right)\right]}{\esp\left[T_{1}\right]},\textrm{ c.s. cuando  }t\rightarrow\infty.
\end{eqnarray}
\end{Coro}



\begin{thebibliography}{99}

\bibitem{ISL}
James, G., Witten, D., Hastie, T., and Tibshirani, R. (2013). \textit{An Introduction to Statistical Learning: with Applications in R}. Springer.

\bibitem{Logistic}
Hosmer, D. W., Lemeshow, S., and Sturdivant, R. X. (2013). \textit{Applied Logistic Regression} (3rd ed.). Wiley.

\bibitem{PatternRecognition}
Bishop, C. M. (2006). \textit{Pattern Recognition and Machine Learning}. Springer.

\bibitem{Harrell}
Harrell, F. E. (2015). \textit{Regression Modeling Strategies: With Applications to Linear Models, Logistic and Ordinal Regression, and Survival Analysis}. Springer.

\bibitem{RDocumentation}
R Documentation and Tutorials: \url{https://cran.r-project.org/manuals.html}

\bibitem{RBlogger}
Tutorials on R-bloggers: \url{https://www.r-bloggers.com/}

\bibitem{CourseraML}
Coursera: \textit{Machine Learning} by Andrew Ng.

\bibitem{edXDS}
edX: \textit{Data Science and Machine Learning Essentials} by Microsoft.

% Libros adicionales
\bibitem{Ross}
Ross, S. M. (2014). \textit{Introduction to Probability and Statistics for Engineers and Scientists}. Academic Press.

\bibitem{DeGroot}
DeGroot, M. H., and Schervish, M. J. (2012). \textit{Probability and Statistics} (4th ed.). Pearson.

\bibitem{Hogg}
Hogg, R. V., McKean, J., and Craig, A. T. (2019). \textit{Introduction to Mathematical Statistics} (8th ed.). Pearson.

\bibitem{Kleinbaum}
Kleinbaum, D. G., and Klein, M. (2010). \textit{Logistic Regression: A Self-Learning Text} (3rd ed.). Springer.

% Artículos y tutoriales adicionales
\bibitem{Wasserman}
Wasserman, L. (2004). \textit{All of Statistics: A Concise Course in Statistical Inference}. Springer.

\bibitem{KhanAcademy}
Probability and Statistics Tutorials on Khan Academy: \url{https://www.khanacademy.org/math/statistics-probability}

\bibitem{OnlineStatBook}
Online Statistics Education: \url{http://onlinestatbook.com/}

\bibitem{Peng}
Peng, C. Y. J., Lee, K. L., and Ingersoll, G. M. (2002). \textit{An Introduction to Logistic Regression Analysis and Reporting}. The Journal of Educational Research.

\bibitem{Agresti}
Agresti, A. (2007). \textit{An Introduction to Categorical Data Analysis} (2nd ed.). Wiley.

\bibitem{Han}
Han, J., Pei, J., and Kamber, M. (2011). \textit{Data Mining: Concepts and Techniques}. Morgan Kaufmann.

\bibitem{TowardsDataScience}
Data Cleaning and Preprocessing on Towards Data Science: \url{https://towardsdatascience.com/data-cleaning-and-preprocessing}

\bibitem{Molinaro}
Molinaro, A. M., Simon, R., and Pfeiffer, R. M. (2005). \textit{Prediction error estimation: a comparison of resampling methods}. Bioinformatics.

\bibitem{EvaluatingModels}
Evaluating Machine Learning Models on Towards Data Science: \url{https://towardsdatascience.com/evaluating-machine-learning-models}

\bibitem{LogisticRegressionGuide}
Practical Guide to Logistic Regression in R on Towards Data Science: \url{https://towardsdatascience.com/practical-guide-to-logistic-regression-in-r}

% Cursos en línea adicionales
\bibitem{CourseraStatistics}
Coursera: \textit{Statistics with R} by Duke University.

\bibitem{edXProbability}
edX: \textit{Data Science: Probability} by Harvard University.

\bibitem{CourseraLogistic}
Coursera: \textit{Logistic Regression} by Stanford University.

\bibitem{edXInference}
edX: \textit{Data Science: Inference and Modeling} by Harvard University.

\bibitem{CourseraWrangling}
Coursera: \textit{Data Science: Wrangling and Cleaning} by Johns Hopkins University.

\bibitem{edXRBasics}
edX: \textit{Data Science: R Basics} by Harvard University.

\bibitem{CourseraRegression}
Coursera: \textit{Regression Models} by Johns Hopkins University.

\bibitem{edXStatInference}
edX: \textit{Data Science: Statistical Inference} by Harvard University.

\bibitem{SurvivalAnalysis}
An Introduction to Survival Analysis on Towards Data Science: \url{https://towardsdatascience.com/an-introduction-to-survival-analysis}

\bibitem{MultinomialLogistic}
Multinomial Logistic Regression on DataCamp: \url{https://www.datacamp.com/community/tutorials/multinomial-logistic-regression-R}

\bibitem{CourseraSurvival}
Coursera: \textit{Survival Analysis} by Johns Hopkins University.

\bibitem{edXHighthroughput}
edX: \textit{Data Science: Statistical Inference and Modeling for High-throughput Experiments} by Harvard University.

\end{thebibliography}


\end{document}



\chapter{Thorisson}
%___________________________________________________________
%
\section{Existencia de Tiempos de Regeneraci\'on}
%___________________________________________________________
%

%________________________________________________________________________
\subsection{Procesos Regenerativos: Thorisson}
%________________________________________________________________________

Para $\left\{X\left(t\right):t\geq0\right\}$ Proceso Estoc\'astico a tiempo continuo con estado de espacios $S$, que es un espacio m\'etrico, con trayectorias continuas por la derecha y con l\'imites por la izquierda c.s. Sea $N\left(t\right)$ un proceso de renovaci\'on en $\rea_{+}$ definido en el mismo espacio de probabilidad que $X\left(t\right)$, con tiempos de renovaci\'on $T$ y tiempos de inter-renovaci\'on $\xi_{n}=T_{n}-T_{n-1}$, con misma distribuci\'on $F$ de media finita $\mu$.

\begin{Def}
Un elemento aleatorio en un espacio medible $\left(E,\mathcal{E}\right)$ en un espacio de probabilidad $\left(\Omega,\mathcal{F},\prob\right)$ a $\left(E,\mathcal{E}\right)$, es decir,
para $A\in \mathcal{E}$,  se tiene que $\left\{Y\in A\right\}\in\mathcal{F}$, donde $\left\{Y\in A\right\}:=\left\{w\in\Omega:Y\left(w\right)\in A\right\}=:Y^{-1}A$.
\end{Def}

\begin{Note}
Tambi\'en se dice que $Y$ est\'a soportado por el espacio de probabilidad $\left(\Omega,\mathcal{F},\prob\right)$ y que $Y$ es un mapeo medible de $\Omega$ en $E$, es decir, es $\mathcal{F}/\mathcal{E}$ medible.
\end{Note}

\begin{Def}
Para cada $i\in \mathbb{I}$ sea $P_{i}$ una medida de probabilidad en un espacio medible $\left(E_{i},\mathcal{E}_{i}\right)$. Se define el espacio producto
$\otimes_{i\in\mathbb{I}}\left(E_{i},\mathcal{E}_{i}\right):=\left(\prod_{i\in\mathbb{I}}E_{i},\otimes_{i\in\mathbb{I}}\mathcal{E}_{i}\right)$, donde $\prod_{i\in\mathbb{I}}E_{i}$ es el producto cartesiano de los $E_{i}$'s, y $\otimes_{i\in\mathbb{I}}\mathcal{E}_{i}$ es la $\sigma$-\'algebra producto, es decir, es la $\sigma$-\'algebra m\'as peque\~na en $\prod_{i\in\mathbb{I}}E_{i}$ que hace al $i$-\'esimo mapeo proyecci\'on en $E_{i}$ medible para toda $i\in\mathbb{I}$ es la $\sigma$-\'algebra inducida por los mapeos proyecci\'on. $$\otimes_{i\in\mathbb{I}}\mathcal{E}_{i}:=\sigma\left\{\left\{y:y_{i}\in A\right\}:i\in\mathbb{I}\textrm{ y }A\in\mathcal{E}_{i}\right\}.$$
\end{Def}

\begin{Def}
Un espacio de probabilidad $\left(\tilde{\Omega},\tilde{\mathcal{F}},\tilde{\prob}\right)$ es una extensi\'on de otro espacio de probabilidad $\left(\Omega,\mathcal{F},\prob\right)$ si $\left(\tilde{\Omega},\tilde{\mathcal{F}},\tilde{\prob}\right)$ soporta un elemento aleatorio $\xi\in\left(\Omega,\mathcal{F}\right)$ que tienen a $\prob$ como distribuci\'on.
\end{Def}

\begin{Teo}
Sea $\mathbb{I}$ un conjunto de \'indices arbitrario. Para cada $i\in\mathbb{I}$ sea $P_{i}$ una medida de probabilidad en un espacio medible $\left(E_{i},\mathcal{E}_{i}\right)$. Entonces existe una \'unica medida de probabilidad $\otimes_{i\in\mathbb{I}}P_{i}$ en $\otimes_{i\in\mathbb{I}}\left(E_{i},\mathcal{E}_{i}\right)$ tal que 

\begin{eqnarray*}
\otimes_{i\in\mathbb{I}}P_{i}\left(y\in\prod_{i\in\mathbb{I}}E_{i}:y_{i}\in A_{i_{1}},\ldots,y_{n}\in A_{i_{n}}\right)=P_{i_{1}}\left(A_{i_{n}}\right)\cdots P_{i_{n}}\left(A_{i_{n}}\right)
\end{eqnarray*}
para todos los enteros $n>0$, toda $i_{1},\ldots,i_{n}\in\mathbb{I}$ y todo $A_{i_{1}}\in\mathcal{E}_{i_{1}},\ldots,A_{i_{n}}\in\mathcal{E}_{i_{n}}$
\end{Teo}

La medida $\otimes_{i\in\mathbb{I}}P_{i}$ es llamada la medida producto y $\otimes_{i\in\mathbb{I}}\left(E_{i},\mathcal{E}_{i},P_{i}\right):=\left(\prod_{i\in\mathbb{I}},E_{i},\otimes_{i\in\mathbb{I}}\mathcal{E}_{i},\otimes_{i\in\mathbb{I}}P_{i}\right)$, es llamado espacio de probabilidad producto.


\begin{Def}
Un espacio medible $\left(E,\mathcal{E}\right)$ es \textit{Polaco} si existe una m\'etrica en $E$ tal que $E$ es completo, es decir cada sucesi\'on de Cauchy converge a un l\'imite en $E$, y \textit{separable}, $E$ tienen un subconjunto denso numerable, y tal que $\mathcal{E}$ es generado por conjuntos abiertos.
\end{Def}


\begin{Def}
Dos espacios medibles $\left(E,\mathcal{E}\right)$ y $\left(G,\mathcal{G}\right)$ son Borel equivalentes \textit{isomorfos} si existe una biyecci\'on $f:E\rightarrow G$ tal que $f$ es $\mathcal{E}/\mathcal{G}$ medible y su inversa $f^{-1}$ es $\mathcal{G}/\mathcal{E}$ medible. La biyecci\'on es una equivalencia de Borel.
\end{Def}

\begin{Def}
Un espacio medible  $\left(E,\mathcal{E}\right)$ es un \textit{espacio est\'andar} si es Borel equivalente a $\left(G,\mathcal{G}\right)$, donde $G$ es un subconjunto de Borel de $\left[0,1\right]$ y $\mathcal{G}$ son los subconjuntos de Borel de $G$.
\end{Def}

\begin{Note}
Cualquier espacio Polaco es un espacio est\'andar.
\end{Note}


\begin{Def}
Un proceso estoc\'astico con conjunto de \'indices $\mathbb{I}$ y espacio de estados $\left(E,\mathcal{E}\right)$ es una familia $Z=\left(\mathbb{Z}_{s}\right)_{s\in\mathbb{I}}$ donde $\mathbb{Z}_{s}$ son elementos aleatorios definidos en un espacio de probabilidad com\'un $\left(\Omega,\mathcal{F},\prob\right)$ y todos toman valores en $\left(E,\mathcal{E}\right)$.
\end{Def}

\begin{Def}
Un proceso estoc\'astico \textit{one-sided contiuous time} (\textbf{PEOSCT}) es un proceso estoc\'astico con conjunto de \'indices $\mathbb{I}=\left[0,\infty\right)$.
\end{Def}


Sea $\left(E^{\mathbb{I}},\mathcal{E}^{\mathbb{I}}\right)$ denota el espacio producto $\left(E^{\mathbb{I}},\mathcal{E}^{\mathbb{I}}\right):=\otimes_{s\in\mathbb{I}}\left(E,\mathcal{E}\right)$. Vamos a considerar $\mathbb{Z}$ como un mapeo aleatorio, es decir, como un elemento aleatorio en $\left(E^{\mathbb{I}},\mathcal{E}^{\mathbb{I}}\right)$ definido por $Z\left(w\right)=\left(Z_{s}\left(w\right)\right)_{s\in\mathbb{I}}$ y $w\in\Omega$.

\begin{Note}
La distribuci\'on de un proceso estoc\'astico $Z$ es la distribuci\'on de $Z$ como un elemento aleatorio en $\left(E^{\mathbb{I}},\mathcal{E}^{\mathbb{I}}\right)$. La distribuci\'on de $Z$ esta determinada de manera \'unica por las distribuciones finito dimensionales.
\end{Note}

\begin{Note}
En particular cuando $Z$ toma valores reales, es decir, $\left(E,\mathcal{E}\right)=\left(\mathbb{R},\mathcal{B}\right)$ las distribuciones finito dimensionales est\'an determinadas por las funciones de distribuci\'on finito dimensionales

\begin{eqnarray}
\prob\left(Z_{t_{1}}\leq x_{1},\ldots,Z_{t_{n}}\leq x_{n}\right),x_{1},\ldots,x_{n}\in\mathbb{R},t_{1},\ldots,t_{n}\in\mathbb{I},n\geq1.
\end{eqnarray}
\end{Note}

\begin{Note}
Para espacios polacos $\left(E,\mathcal{E}\right)$ el Teorema de Consistencia de Kolmogorov asegura que dada una colecci\'on de distribuciones finito dimensionales consistentes, siempre existe un proceso estoc\'astico que posee tales distribuciones finito dimensionales.
\end{Note}


\begin{Def}
Las trayectorias de $Z$ son las realizaciones $Z\left(w\right)$ para $w\in\Omega$ del mapeo aleatorio $Z$.
\end{Def}

\begin{Note}
Algunas restricciones se imponen sobre las trayectorias, por ejemplo que sean continuas por la derecha, o continuas por la derecha con l\'imites por la izquierda, o de manera m\'as general, se pedir\'a que caigan en alg\'un subconjunto $H$ de $E^{\mathbb{I}}$. En este caso es natural considerar a $Z$ como un elemento aleatorio que no est\'a en $\left(E^{\mathbb{I}},\mathcal{E}^{\mathbb{I}}\right)$ sino en $\left(H,\mathcal{H}\right)$, donde $\mathcal{H}$ es la $\sigma$-\'algebra generada por los mapeos proyecci\'on que toman a $z\in H$ a $z_{t}\in E$ para $t\in\mathbb{I}$. A $\mathcal{H}$ se le conoce como la traza de $H$ en $E^{\mathbb{I}}$, es decir,
\begin{eqnarray}
\mathcal{H}:=E^{\mathbb{I}}\cap H:=\left\{A\cap H:A\in E^{\mathbb{I}}\right\}.
\end{eqnarray}
\end{Note}


\begin{Note}
$Z$ tiene trayectorias con valores en $H$ y cada $Z_{t}$ es un mapeo medible de $\left(\Omega,\mathcal{F}\right)$ a $\left(H,\mathcal{H}\right)$. Cuando se considera un espacio de trayectorias en particular $H$, al espacio $\left(H,\mathcal{H}\right)$ se le llama el espacio de trayectorias de $Z$.
\end{Note}

\begin{Note}
La distribuci\'on del proceso estoc\'astico $Z$ con espacio de trayectorias $\left(H,\mathcal{H}\right)$ es la distribuci\'on de $Z$ como  un elemento aleatorio en $\left(H,\mathcal{H}\right)$. La distribuci\'on, nuevemente, est\'a determinada de manera \'unica por las distribuciones finito dimensionales.
\end{Note}


\begin{Def}
Sea $Z$ un PEOSCT  con espacio de estados $\left(E,\mathcal{E}\right)$ y sea $T$ un tiempo aleatorio en $\left[0,\infty\right)$. Por $Z_{T}$ se entiende el mapeo con valores en $E$ definido en $\Omega$ en la manera obvia:
\begin{eqnarray*}
Z_{T}\left(w\right):=Z_{T\left(w\right)}\left(w\right). w\in\Omega.
\end{eqnarray*}
\end{Def}

\begin{Def}
Un PEOSCT $Z$ es conjuntamente medible (\textbf{CM}) si el mapeo que toma $\left(w,t\right)\in\Omega\times\left[0,\infty\right)$ a $Z_{t}\left(w\right)\in E$ es $\mathcal{F}\otimes\mathcal{B}\left[0,\infty\right)/\mathcal{E}$ medible.
\end{Def}

\begin{Note}
Un PEOSCT-CM implica que el proceso es medible, dado que $Z_{T}$ es una composici\'on  de dos mapeos continuos: el primero que toma $w$ en $\left(w,T\left(w\right)\right)$ es $\mathcal{F}/\mathcal{F}\otimes\mathcal{B}\left[0,\infty\right)$ medible, mientras que el segundo toma $\left(w,T\left(w\right)\right)$ en $Z_{T\left(w\right)}\left(w\right)$ es $\mathcal{F}\otimes\mathcal{B}\left[0,\infty\right)/\mathcal{E}$ medible.
\end{Note}


\begin{Def}
Un PEOSCT con espacio de estados $\left(H,\mathcal{H}\right)$ es can\'onicamente conjuntamente medible (\textbf{CCM}) si el mapeo $\left(z,t\right)\in H\times\left[0,\infty\right)$ en $Z_{t}\in E$ es $\mathcal{H}\otimes\mathcal{B}\left[0,\infty\right)/\mathcal{E}$ medible.
\end{Def}

\begin{Note}
Un PEOSCTCCM implica que el proceso es CM, dado que un PECCM $Z$ es un mapeo de $\Omega\times\left[0,\infty\right)$ a $E$, es la composici\'on de dos mapeos medibles: el primero, toma $\left(w,t\right)$ en $\left(Z\left(w\right),t\right)$ es $\mathcal{F}\otimes\mathcal{B}\left[0,\infty\right)/\mathcal{H}\otimes\mathcal{B}\left[0,\infty\right)$ medible, y el segundo que toma $\left(Z\left(w\right),t\right)$  en $Z_{t}\left(w\right)$ es $\mathcal{H}\otimes\mathcal{B}\left[0,\infty\right)/\mathcal{E}$ medible. Por tanto CCM es una condici\'on m\'as fuerte que CM.
\end{Note}

\begin{Def}
Un conjunto de trayectorias $H$ de un PEOSCT $Z$, es internamente shift-invariante (\textbf{ISI}) si 
\begin{eqnarray*}
\left\{\left(z_{t+s}\right)_{s\in\left[0,\infty\right)}:z\in H\right\}=H\textrm{, }t\in\left[0,\infty\right).
\end{eqnarray*}
\end{Def}


\begin{Def}
Dado un PEOSCTISI, se define el mapeo-shift $\theta_{t}$, $t\in\left[0,\infty\right)$, de $H$ a $H$ por 
\begin{eqnarray*}
\theta_{t}z=\left(z_{t+s}\right)_{s\in\left[0,\infty\right)}\textrm{, }z\in H.
\end{eqnarray*}
\end{Def}

\begin{Def}
Se dice que un proceso $Z$ es shift-medible (\textbf{SM}) si $Z$ tiene un conjunto de trayectorias $H$ que es ISI y adem\'as el mapeo que toma $\left(z,t\right)\in H\times\left[0,\infty\right)$ en $\theta_{t}z\in H$ es $\mathcal{H}\otimes\mathcal{B}\left[0,\infty\right)/\mathcal{H}$ medible.
\end{Def}

\begin{Note}
Un proceso estoc\'astico con conjunto de trayectorias $H$ ISI es shift-medible si y s\'olo si es CCM
\end{Note}

\begin{Note}
\begin{itemize}
\item Dado el espacio polaco $\left(E,\mathcal{E}\right)$ se tiene el  conjunto de trayectorias $D_{E}\left[0,\infty\right)$ que es ISI, entonces cumpe con ser CCM.

\item Si $G$ es abierto, podemos cubrirlo por bolas abiertas cuay cerradura este contenida en $G$, y como $G$ es segundo numerable como subespacio de $E$, lo podemos cubrir por una cantidad numerable de bolas abiertas.

\end{itemize}
\end{Note}


\begin{Note}
Los procesos estoc\'asticos $Z$ a tiempo discreto con espacio de estados polaco, tambi\'en tiene un espacio de trayectorias polaco y por tanto tiene distribuciones condicionales regulares.
\end{Note}

\begin{Teo}
El producto numerable de espacios polacos es polaco.
\end{Teo}


\begin{Def}
Sea $\left(\Omega,\mathcal{F},\prob\right)$ espacio de probabilidad que soporta al proceso $Z=\left(Z_{s}\right)_{s\in\left[0,\infty\right)}$ y $S=\left(S_{k}\right)_{0}^{\infty}$ donde $Z$ es un PEOSCTM con espacio de estados $\left(E,\mathcal{E}\right)$  y espacio de trayectorias $\left(H,\mathcal{H}\right)$  y adem\'as $S$ es una sucesi\'on de tiempos aleatorios one-sided que satisfacen la condici\'on $0\leq S_{0}<S_{1}<\cdots\rightarrow\infty$. Considerando $S$ como un mapeo medible de $\left(\Omega,\mathcal{F}\right)$ al espacio sucesi\'on $\left(L,\mathcal{L}\right)$, donde 
\begin{eqnarray*}
L=\left\{\left(s_{k}\right)_{0}^{\infty}\in\left[0,\infty\right)^{\left\{0,1,\ldots\right\}}:s_{0}<s_{1}<\cdots\rightarrow\infty\right\},
\end{eqnarray*}
donde $\mathcal{L}$ son los subconjuntos de Borel de $L$, es decir, $\mathcal{L}=L\cap\mathcal{B}^{\left\{0,1,\ldots\right\}}$.

As\'i el par $\left(Z,S\right)$ es un mapeo medible de  $\left(\Omega,\mathcal{F}\right)$ en $\left(H\times L,\mathcal{H}\otimes\mathcal{L}\right)$. El par $\mathcal{H}\otimes\mathcal{L}^{+}$ denotar\'a la clase de todas las funciones medibles de $\left(H\times L,\mathcal{H}\otimes\mathcal{L}\right)$ en $\left(\left[0,\infty\right),\mathcal{B}\left[0,\infty\right)\right)$.
\end{Def}


\begin{Def}
Sea $\theta_{t}$ el mapeo-shift conjunto de $H\times L$ en $H\times L$ dado por
\begin{eqnarray*}
\theta_{t}\left(z,\left(s_{k}\right)_{0}^{\infty}\right)=\theta_{t}\left(z,\left(s_{n_{t-}+k}-t\right)_{0}^{\infty}\right)
\end{eqnarray*}
donde 
$n_{t-}=inf\left\{n\geq1:s_{n}\geq t\right\}$.
\end{Def}

\begin{Note}
Con la finalidad de poder realizar los shift's sin complicaciones de medibilidad, se supondr\'a que $Z$ es shit-medible, es decir, el conjunto de trayectorias $H$ es invariante bajo shifts del tiempo y el mapeo que toma $\left(z,t\right)\in H\times\left[0,\infty\right)$ en $z_{t}\in E$ es $\mathcal{H}\otimes\mathcal{B}\left[0,\infty\right)/\mathcal{E}$ medible.
\end{Note}

\begin{Def}
Dado un proceso \textbf{PEOSSM} (Proceso Estoc\'astico One Side Shift Medible) $Z$, se dice regenerativo cl\'asico con tiempos de regeneraci\'on $S$ si 

\begin{eqnarray*}
\theta_{S_{n}}\left(Z,S\right)=\left(Z^{0},S^{0}\right),n\geq0
\end{eqnarray*}
y adem\'as $\theta_{S_{n}}\left(Z,S\right)$ es independiente de $\left(\left(Z_{s}\right)s\in\left[0,S_{n}\right),S_{0},\ldots,S_{n}\right)$
Si lo anterior se cumple, al par $\left(Z,S\right)$ se le llama regenerativo cl\'asico.
\end{Def}

\begin{Note}
Si el par $\left(Z,S\right)$ es regenerativo cl\'asico, entonces las longitudes de los ciclos $X_{1},X_{2},\ldots,$ son i.i.d. e independientes de la longitud del retraso $S_{0}$, es decir, $S$ es un proceso de renovaci\'on. Las longitudes de los ciclos tambi\'en son llamados tiempos de inter-regeneraci\'on y tiempos de ocurrencia.

\end{Note}

\begin{Teo}
Sup\'ongase que el par $\left(Z,S\right)$ es regenerativo cl\'asico con $\esp\left[X_{1}\right]<\infty$. Entonces $\left(Z^{*},S^{*}\right)$ en el teorema 2.1 es una versi\'on estacionaria de $\left(Z,S\right)$. Adem\'as, si $X_{1}$ es lattice con span $d$, entonces $\left(Z^{**},S^{**}\right)$ en el teorema 2.2 es una versi\'on periodicamente estacionaria de $\left(Z,S\right)$ con periodo $d$.

\end{Teo}

\begin{Def}
Una variable aleatoria $X_{1}$ es \textit{spread out} si existe una $n\geq1$ y una  funci\'on $f\in\mathcal{B}^{+}$ tal que $\int_{\rea}f\left(x\right)dx>0$ con $X_{2},X_{3},\ldots,X_{n}$ copias i.i.d  de $X_{1}$, $$\prob\left(X_{1}+\cdots+X_{n}\in B\right)\geq\int_{B}f\left(x\right)dx$$ para $B\in\mathcal{B}$.

\end{Def}



\begin{Def}
Dado un proceso estoc\'astico $Z$ se le llama \textit{wide-sense regenerative} (\textbf{WSR}) con tiempos de regeneraci\'on $S$ si $\theta_{S_{n}}\left(Z,S\right)=\left(Z^{0},S^{0}\right)$ para $n\geq0$ en distribuci\'on y $\theta_{S_{n}}\left(Z,S\right)$ es independiente de $\left(S_{0},S_{1},\ldots,S_{n}\right)$ para $n\geq0$.
Se dice que el par $\left(Z,S\right)$ es WSR si lo anterior se cumple.
\end{Def}


\begin{Note}
\begin{itemize}
\item El proceso de trayectorias $\left(\theta_{s}Z\right)_{s\in\left[0,\infty\right)}$ es WSR con tiempos de regeneraci\'on $S$ pero no es regenerativo cl\'asico.

\item Si $Z$ es cualquier proceso estacionario y $S$ es un proceso de renovaci\'on que es independiente de $Z$, entonces $\left(Z,S\right)$ es WSR pero en general no es regenerativo cl\'asico

\end{itemize}

\end{Note}


\begin{Note}
Para cualquier proceso estoc\'astico $Z$, el proceso de trayectorias $\left(\theta_{s}Z\right)_{s\in\left[0,\infty\right)}$ es siempre un proceso de Markov.
\end{Note}



\begin{Teo}
Supongase que el par $\left(Z,S\right)$ es WSR con $\esp\left[X_{1}\right]<\infty$. Entonces $\left(Z^{*},S^{*}\right)$ en el teorema 2.1 es una versi\'on estacionaria de 
$\left(Z,S\right)$.
\end{Teo}


\begin{Teo}
Supongase que $\left(Z,S\right)$ es cycle-stationary con $\esp\left[X_{1}\right]<\infty$. Sea $U$ distribuida uniformemente en $\left[0,1\right)$ e independiente de $\left(Z^{0},S^{0}\right)$ y sea $\prob^{*}$ la medida de probabilidad en $\left(\Omega,\prob\right)$ definida por $$d\prob^{*}=\frac{X_{1}}{\esp\left[X_{1}\right]}d\prob$$. Sea $\left(Z^{*},S^{*}\right)$ con distribuci\'on $\prob^{*}\left(\theta_{UX_{1}}\left(Z^{0},S^{0}\right)\in\cdot\right)$. Entonces $\left(Z^{}*,S^{*}\right)$ es estacionario,
\begin{eqnarray*}
\esp\left[f\left(Z^{*},S^{*}\right)\right]=\esp\left[\int_{0}^{X_{1}}f\left(\theta_{s}\left(Z^{0},S^{0}\right)\right)ds\right]/\esp\left[X_{1}\right]
\end{eqnarray*}
$f\in\mathcal{H}\otimes\mathcal{L}^{+}$, and $S_{0}^{*}$ es continuo con funci\'on distribuci\'on $G_{\infty}$ definida por $$G_{\infty}\left(x\right):=\frac{\esp\left[X_{1}\right]\wedge x}{\esp\left[X_{1}\right]}$$ para $x\geq0$ y densidad $\prob\left[X_{1}>x\right]/\esp\left[X_{1}\right]$, con $x\geq0$.

\end{Teo}


\begin{Teo}
Sea $Z$ un Proceso Estoc\'astico un lado shift-medible \textit{one-sided shift-measurable stochastic process}, (PEOSSM),
y $S_{0}$ y $S_{1}$ tiempos aleatorios tales que $0\leq S_{0}<S_{1}$ y
\begin{equation}
\theta_{S_{1}}Z=\theta_{S_{0}}Z\textrm{ en distribuci\'on}.
\end{equation}

Entonces el espacio de probabilidad subyacente $\left(\Omega,\mathcal{F},\prob\right)$ puede extenderse para soportar una sucesi\'on de tiempos aleatorios $S$ tales que

\begin{eqnarray}
\theta_{S_{n}}\left(Z,S\right)=\left(Z^{0},S^{0}\right),n\geq0,\textrm{ en distribuci\'on},\\
\left(Z,S_{0},S_{1}\right)\textrm{ depende de }\left(X_{2},X_{3},\ldots\right)\textrm{ solamente a traves de }\theta_{S_{1}}Z.
\end{eqnarray}
\end{Teo}


\begin{Def}
Un elemento aleatorio en un espacio medible $\left(E,\mathcal{E}\right)$ en un espacio de probabilidad $\left(\Omega,\mathcal{F},\prob\right)$ a $\left(E,\mathcal{E}\right)$, es decir,
para $A\in \mathcal{E}$,  se tiene que $\left\{Y\in A\right\}\in\mathcal{F}$, donde $\left\{Y\in A\right\}:=\left\{w\in\Omega:Y\left(w\right)\in A\right\}=:Y^{-1}A$.
\end{Def}

\begin{Note}
Tambi\'en se dice que $Y$ est\'a soportado por el espacio de probabilidad $\left(\Omega,\mathcal{F},\prob\right)$ y que $Y$ es un mapeo medible de $\Omega$ en $E$, es decir, es $\mathcal{F}/\mathcal{E}$ medible.
\end{Note}

\begin{Def}
Para cada $i\in \mathbb{I}$ sea $P_{i}$ una medida de probabilidad en un espacio medible $\left(E_{i},\mathcal{E}_{i}\right)$. Se define el espacio producto
$\otimes_{i\in\mathbb{I}}\left(E_{i},\mathcal{E}_{i}\right):=\left(\prod_{i\in\mathbb{I}}E_{i},\otimes_{i\in\mathbb{I}}\mathcal{E}_{i}\right)$, donde $\prod_{i\in\mathbb{I}}E_{i}$ es el producto cartesiano de los $E_{i}$'s, y $\otimes_{i\in\mathbb{I}}\mathcal{E}_{i}$ es la $\sigma$-\'algebra producto, es decir, es la $\sigma$-\'algebra m\'as peque\~na en $\prod_{i\in\mathbb{I}}E_{i}$ que hace al $i$-\'esimo mapeo proyecci\'on en $E_{i}$ medible para toda $i\in\mathbb{I}$ es la $\sigma$-\'algebra inducida por los mapeos proyecci\'on. $$\otimes_{i\in\mathbb{I}}\mathcal{E}_{i}:=\sigma\left\{\left\{y:y_{i}\in A\right\}:i\in\mathbb{I}\textrm{ y }A\in\mathcal{E}_{i}\right\}.$$
\end{Def}

\begin{Def}
Un espacio de probabilidad $\left(\tilde{\Omega},\tilde{\mathcal{F}},\tilde{\prob}\right)$ es una extensi\'on de otro espacio de probabilidad $\left(\Omega,\mathcal{F},\prob\right)$ si $\left(\tilde{\Omega},\tilde{\mathcal{F}},\tilde{\prob}\right)$ soporta un elemento aleatorio $\xi\in\left(\Omega,\mathcal{F}\right)$ que tienen a $\prob$ como distribuci\'on.
\end{Def}

\begin{Teo}
Sea $\mathbb{I}$ un conjunto de \'indices arbitrario. Para cada $i\in\mathbb{I}$ sea $P_{i}$ una medida de probabilidad en un espacio medible $\left(E_{i},\mathcal{E}_{i}\right)$. Entonces existe una \'unica medida de probabilidad $\otimes_{i\in\mathbb{I}}P_{i}$ en $\otimes_{i\in\mathbb{I}}\left(E_{i},\mathcal{E}_{i}\right)$ tal que 

\begin{eqnarray*}
\otimes_{i\in\mathbb{I}}P_{i}\left(y\in\prod_{i\in\mathbb{I}}E_{i}:y_{i}\in A_{i_{1}},\ldots,y_{n}\in A_{i_{n}}\right)=P_{i_{1}}\left(A_{i_{n}}\right)\cdots P_{i_{n}}\left(A_{i_{n}}\right)
\end{eqnarray*}
para todos los enteros $n>0$, toda $i_{1},\ldots,i_{n}\in\mathbb{I}$ y todo $A_{i_{1}}\in\mathcal{E}_{i_{1}},\ldots,A_{i_{n}}\in\mathcal{E}_{i_{n}}$
\end{Teo}

La medida $\otimes_{i\in\mathbb{I}}P_{i}$ es llamada la medida producto y $\otimes_{i\in\mathbb{I}}\left(E_{i},\mathcal{E}_{i},P_{i}\right):=\left(\prod_{i\in\mathbb{I}},E_{i},\otimes_{i\in\mathbb{I}}\mathcal{E}_{i},\otimes_{i\in\mathbb{I}}P_{i}\right)$, es llamado espacio de probabilidad producto.


\begin{Def}
Un espacio medible $\left(E,\mathcal{E}\right)$ es \textit{Polaco} si existe una m\'etrica en $E$ tal que $E$ es completo, es decir cada sucesi\'on de Cauchy converge a un l\'imite en $E$, y \textit{separable}, $E$ tienen un subconjunto denso numerable, y tal que $\mathcal{E}$ es generado por conjuntos abiertos.
\end{Def}


\begin{Def}
Dos espacios medibles $\left(E,\mathcal{E}\right)$ y $\left(G,\mathcal{G}\right)$ son Borel equivalentes \textit{isomorfos} si existe una biyecci\'on $f:E\rightarrow G$ tal que $f$ es $\mathcal{E}/\mathcal{G}$ medible y su inversa $f^{-1}$ es $\mathcal{G}/\mathcal{E}$ medible. La biyecci\'on es una equivalencia de Borel.
\end{Def}

\begin{Def}
Un espacio medible  $\left(E,\mathcal{E}\right)$ es un \textit{espacio est\'andar} si es Borel equivalente a $\left(G,\mathcal{G}\right)$, donde $G$ es un subconjunto de Borel de $\left[0,1\right]$ y $\mathcal{G}$ son los subconjuntos de Borel de $G$.
\end{Def}

\begin{Note}
Cualquier espacio Polaco es un espacio est\'andar.
\end{Note}


\begin{Def}
Un proceso estoc\'astico con conjunto de \'indices $\mathbb{I}$ y espacio de estados $\left(E,\mathcal{E}\right)$ es una familia $Z=\left(\mathbb{Z}_{s}\right)_{s\in\mathbb{I}}$ donde $\mathbb{Z}_{s}$ son elementos aleatorios definidos en un espacio de probabilidad com\'un $\left(\Omega,\mathcal{F},\prob\right)$ y todos toman valores en $\left(E,\mathcal{E}\right)$.
\end{Def}

\begin{Def}
Un proceso estoc\'astico \textit{one-sided contiuous time} (\textbf{PEOSCT}) es un proceso estoc\'astico con conjunto de \'indices $\mathbb{I}=\left[0,\infty\right)$.
\end{Def}


Sea $\left(E^{\mathbb{I}},\mathcal{E}^{\mathbb{I}}\right)$ denota el espacio producto $\left(E^{\mathbb{I}},\mathcal{E}^{\mathbb{I}}\right):=\otimes_{s\in\mathbb{I}}\left(E,\mathcal{E}\right)$. Vamos a considerar $\mathbb{Z}$ como un mapeo aleatorio, es decir, como un elemento aleatorio en $\left(E^{\mathbb{I}},\mathcal{E}^{\mathbb{I}}\right)$ definido por $Z\left(w\right)=\left(Z_{s}\left(w\right)\right)_{s\in\mathbb{I}}$ y $w\in\Omega$.

\begin{Note}
La distribuci\'on de un proceso estoc\'astico $Z$ es la distribuci\'on de $Z$ como un elemento aleatorio en $\left(E^{\mathbb{I}},\mathcal{E}^{\mathbb{I}}\right)$. La distribuci\'on de $Z$ esta determinada de manera \'unica por las distribuciones finito dimensionales.
\end{Note}

\begin{Note}
En particular cuando $Z$ toma valores reales, es decir, $\left(E,\mathcal{E}\right)=\left(\mathbb{R},\mathcal{B}\right)$ las distribuciones finito dimensionales est\'an determinadas por las funciones de distribuci\'on finito dimensionales

\begin{eqnarray}
\prob\left(Z_{t_{1}}\leq x_{1},\ldots,Z_{t_{n}}\leq x_{n}\right),x_{1},\ldots,x_{n}\in\mathbb{R},t_{1},\ldots,t_{n}\in\mathbb{I},n\geq1.
\end{eqnarray}
\end{Note}

\begin{Note}
Para espacios polacos $\left(E,\mathcal{E}\right)$ el Teorema de Consistencia de Kolmogorov asegura que dada una colecci\'on de distribuciones finito dimensionales consistentes, siempre existe un proceso estoc\'astico que posee tales distribuciones finito dimensionales.
\end{Note}


\begin{Def}
Las trayectorias de $Z$ son las realizaciones $Z\left(w\right)$ para $w\in\Omega$ del mapeo aleatorio $Z$.
\end{Def}

\begin{Note}
Algunas restricciones se imponen sobre las trayectorias, por ejemplo que sean continuas por la derecha, o continuas por la derecha con l\'imites por la izquierda, o de manera m\'as general, se pedir\'a que caigan en alg\'un subconjunto $H$ de $E^{\mathbb{I}}$. En este caso es natural considerar a $Z$ como un elemento aleatorio que no est\'a en $\left(E^{\mathbb{I}},\mathcal{E}^{\mathbb{I}}\right)$ sino en $\left(H,\mathcal{H}\right)$, donde $\mathcal{H}$ es la $\sigma$-\'algebra generada por los mapeos proyecci\'on que toman a $z\in H$ a $z_{t}\in E$ para $t\in\mathbb{I}$. A $\mathcal{H}$ se le conoce como la traza de $H$ en $E^{\mathbb{I}}$, es decir,
\begin{eqnarray}
\mathcal{H}:=E^{\mathbb{I}}\cap H:=\left\{A\cap H:A\in E^{\mathbb{I}}\right\}.
\end{eqnarray}
\end{Note}


\begin{Note}
$Z$ tiene trayectorias con valores en $H$ y cada $Z_{t}$ es un mapeo medible de $\left(\Omega,\mathcal{F}\right)$ a $\left(H,\mathcal{H}\right)$. Cuando se considera un espacio de trayectorias en particular $H$, al espacio $\left(H,\mathcal{H}\right)$ se le llama el espacio de trayectorias de $Z$.
\end{Note}

\begin{Note}
La distribuci\'on del proceso estoc\'astico $Z$ con espacio de trayectorias $\left(H,\mathcal{H}\right)$ es la distribuci\'on de $Z$ como  un elemento aleatorio en $\left(H,\mathcal{H}\right)$. La distribuci\'on, nuevemente, est\'a determinada de manera \'unica por las distribuciones finito dimensionales.
\end{Note}


\begin{Def}
Sea $Z$ un PEOSCT  con espacio de estados $\left(E,\mathcal{E}\right)$ y sea $T$ un tiempo aleatorio en $\left[0,\infty\right)$. Por $Z_{T}$ se entiende el mapeo con valores en $E$ definido en $\Omega$ en la manera obvia:
\begin{eqnarray*}
Z_{T}\left(w\right):=Z_{T\left(w\right)}\left(w\right). w\in\Omega.
\end{eqnarray*}
\end{Def}

\begin{Def}
Un PEOSCT $Z$ es conjuntamente medible (\textbf{CM}) si el mapeo que toma $\left(w,t\right)\in\Omega\times\left[0,\infty\right)$ a $Z_{t}\left(w\right)\in E$ es $\mathcal{F}\otimes\mathcal{B}\left[0,\infty\right)/\mathcal{E}$ medible.
\end{Def}

\begin{Note}
Un PEOSCT-CM implica que el proceso es medible, dado que $Z_{T}$ es una composici\'on  de dos mapeos continuos: el primero que toma $w$ en $\left(w,T\left(w\right)\right)$ es $\mathcal{F}/\mathcal{F}\otimes\mathcal{B}\left[0,\infty\right)$ medible, mientras que el segundo toma $\left(w,T\left(w\right)\right)$ en $Z_{T\left(w\right)}\left(w\right)$ es $\mathcal{F}\otimes\mathcal{B}\left[0,\infty\right)/\mathcal{E}$ medible.
\end{Note}


\begin{Def}
Un PEOSCT con espacio de estados $\left(H,\mathcal{H}\right)$ es can\'onicamente conjuntamente medible (\textbf{CCM}) si el mapeo $\left(z,t\right)\in H\times\left[0,\infty\right)$ en $Z_{t}\in E$ es $\mathcal{H}\otimes\mathcal{B}\left[0,\infty\right)/\mathcal{E}$ medible.
\end{Def}

\begin{Note}
Un PEOSCTCCM implica que el proceso es CM, dado que un PECCM $Z$ es un mapeo de $\Omega\times\left[0,\infty\right)$ a $E$, es la composici\'on de dos mapeos medibles: el primero, toma $\left(w,t\right)$ en $\left(Z\left(w\right),t\right)$ es $\mathcal{F}\otimes\mathcal{B}\left[0,\infty\right)/\mathcal{H}\otimes\mathcal{B}\left[0,\infty\right)$ medible, y el segundo que toma $\left(Z\left(w\right),t\right)$  en $Z_{t}\left(w\right)$ es $\mathcal{H}\otimes\mathcal{B}\left[0,\infty\right)/\mathcal{E}$ medible. Por tanto CCM es una condici\'on m\'as fuerte que CM.
\end{Note}

\begin{Def}
Un conjunto de trayectorias $H$ de un PEOSCT $Z$, es internamente shift-invariante (\textbf{ISI}) si 
\begin{eqnarray*}
\left\{\left(z_{t+s}\right)_{s\in\left[0,\infty\right)}:z\in H\right\}=H\textrm{, }t\in\left[0,\infty\right).
\end{eqnarray*}
\end{Def}


\begin{Def}
Dado un PEOSCTISI, se define el mapeo-shift $\theta_{t}$, $t\in\left[0,\infty\right)$, de $H$ a $H$ por 
\begin{eqnarray*}
\theta_{t}z=\left(z_{t+s}\right)_{s\in\left[0,\infty\right)}\textrm{, }z\in H.
\end{eqnarray*}
\end{Def}

\begin{Def}
Se dice que un proceso $Z$ es shift-medible (\textbf{SM}) si $Z$ tiene un conjunto de trayectorias $H$ que es ISI y adem\'as el mapeo que toma $\left(z,t\right)\in H\times\left[0,\infty\right)$ en $\theta_{t}z\in H$ es $\mathcal{H}\otimes\mathcal{B}\left[0,\infty\right)/\mathcal{H}$ medible.
\end{Def}

\begin{Note}
Un proceso estoc\'astico con conjunto de trayectorias $H$ ISI es shift-medible si y s\'olo si es CCM
\end{Note}

\begin{Note}
\begin{itemize}
\item Dado el espacio polaco $\left(E,\mathcal{E}\right)$ se tiene el  conjunto de trayectorias $D_{E}\left[0,\infty\right)$ que es ISI, entonces cumpe con ser CCM.

\item Si $G$ es abierto, podemos cubrirlo por bolas abiertas cuay cerradura este contenida en $G$, y como $G$ es segundo numerable como subespacio de $E$, lo podemos cubrir por una cantidad numerable de bolas abiertas.

\end{itemize}
\end{Note}


\begin{Note}
Los procesos estoc\'asticos $Z$ a tiempo discreto con espacio de estados polaco, tambi\'en tiene un espacio de trayectorias polaco y por tanto tiene distribuciones condicionales regulares.
\end{Note}

\begin{Teo}
El producto numerable de espacios polacos es polaco.
\end{Teo}


\begin{Def}
Sea $\left(\Omega,\mathcal{F},\prob\right)$ espacio de probabilidad que soporta al proceso $Z=\left(Z_{s}\right)_{s\in\left[0,\infty\right)}$ y $S=\left(S_{k}\right)_{0}^{\infty}$ donde $Z$ es un PEOSCTM con espacio de estados $\left(E,\mathcal{E}\right)$  y espacio de trayectorias $\left(H,\mathcal{H}\right)$  y adem\'as $S$ es una sucesi\'on de tiempos aleatorios one-sided que satisfacen la condici\'on $0\leq S_{0}<S_{1}<\cdots\rightarrow\infty$. Considerando $S$ como un mapeo medible de $\left(\Omega,\mathcal{F}\right)$ al espacio sucesi\'on $\left(L,\mathcal{L}\right)$, donde 
\begin{eqnarray*}
L=\left\{\left(s_{k}\right)_{0}^{\infty}\in\left[0,\infty\right)^{\left\{0,1,\ldots\right\}}:s_{0}<s_{1}<\cdots\rightarrow\infty\right\},
\end{eqnarray*}
donde $\mathcal{L}$ son los subconjuntos de Borel de $L$, es decir, $\mathcal{L}=L\cap\mathcal{B}^{\left\{0,1,\ldots\right\}}$.

As\'i el par $\left(Z,S\right)$ es un mapeo medible de  $\left(\Omega,\mathcal{F}\right)$ en $\left(H\times L,\mathcal{H}\otimes\mathcal{L}\right)$. El par $\mathcal{H}\otimes\mathcal{L}^{+}$ denotar\'a la clase de todas las funciones medibles de $\left(H\times L,\mathcal{H}\otimes\mathcal{L}\right)$ en $\left(\left[0,\infty\right),\mathcal{B}\left[0,\infty\right)\right)$.
\end{Def}


\begin{Def}
Sea $\theta_{t}$ el mapeo-shift conjunto de $H\times L$ en $H\times L$ dado por
\begin{eqnarray*}
\theta_{t}\left(z,\left(s_{k}\right)_{0}^{\infty}\right)=\theta_{t}\left(z,\left(s_{n_{t-}+k}-t\right)_{0}^{\infty}\right)
\end{eqnarray*}
donde 
$n_{t-}=inf\left\{n\geq1:s_{n}\geq t\right\}$.
\end{Def}

\begin{Note}
Con la finalidad de poder realizar los shift's sin complicaciones de medibilidad, se supondr\'a que $Z$ es shit-medible, es decir, el conjunto de trayectorias $H$ es invariante bajo shifts del tiempo y el mapeo que toma $\left(z,t\right)\in H\times\left[0,\infty\right)$ en $z_{t}\in E$ es $\mathcal{H}\otimes\mathcal{B}\left[0,\infty\right)/\mathcal{E}$ medible.
\end{Note}

\begin{Def}
Dado un proceso \textbf{PEOSSM} (Proceso Estoc\'astico One Side Shift Medible) $Z$, se dice regenerativo cl\'asico con tiempos de regeneraci\'on $S$ si 

\begin{eqnarray*}
\theta_{S_{n}}\left(Z,S\right)=\left(Z^{0},S^{0}\right),n\geq0
\end{eqnarray*}
y adem\'as $\theta_{S_{n}}\left(Z,S\right)$ es independiente de $\left(\left(Z_{s}\right)s\in\left[0,S_{n}\right),S_{0},\ldots,S_{n}\right)$
Si lo anterior se cumple, al par $\left(Z,S\right)$ se le llama regenerativo cl\'asico.
\end{Def}

\begin{Note}
Si el par $\left(Z,S\right)$ es regenerativo cl\'asico, entonces las longitudes de los ciclos $X_{1},X_{2},\ldots,$ son i.i.d. e independientes de la longitud del retraso $S_{0}$, es decir, $S$ es un proceso de renovaci\'on. Las longitudes de los ciclos tambi\'en son llamados tiempos de inter-regeneraci\'on y tiempos de ocurrencia.

\end{Note}

\begin{Teo}
Sup\'ongase que el par $\left(Z,S\right)$ es regenerativo cl\'asico con $\esp\left[X_{1}\right]<\infty$. Entonces $\left(Z^{*},S^{*}\right)$ en el teorema 2.1 es una versi\'on estacionaria de $\left(Z,S\right)$. Adem\'as, si $X_{1}$ es lattice con span $d$, entonces $\left(Z^{**},S^{**}\right)$ en el teorema 2.2 es una versi\'on periodicamente estacionaria de $\left(Z,S\right)$ con periodo $d$.

\end{Teo}

\begin{Def}
Una variable aleatoria $X_{1}$ es \textit{spread out} si existe una $n\geq1$ y una  funci\'on $f\in\mathcal{B}^{+}$ tal que $\int_{\rea}f\left(x\right)dx>0$ con $X_{2},X_{3},\ldots,X_{n}$ copias i.i.d  de $X_{1}$, $$\prob\left(X_{1}+\cdots+X_{n}\in B\right)\geq\int_{B}f\left(x\right)dx$$ para $B\in\mathcal{B}$.

\end{Def}



\begin{Def}
Dado un proceso estoc\'astico $Z$ se le llama \textit{wide-sense regenerative} (\textbf{WSR}) con tiempos de regeneraci\'on $S$ si $\theta_{S_{n}}\left(Z,S\right)=\left(Z^{0},S^{0}\right)$ para $n\geq0$ en distribuci\'on y $\theta_{S_{n}}\left(Z,S\right)$ es independiente de $\left(S_{0},S_{1},\ldots,S_{n}\right)$ para $n\geq0$.
Se dice que el par $\left(Z,S\right)$ es WSR si lo anterior se cumple.
\end{Def}


\begin{Note}
\begin{itemize}
\item El proceso de trayectorias $\left(\theta_{s}Z\right)_{s\in\left[0,\infty\right)}$ es WSR con tiempos de regeneraci\'on $S$ pero no es regenerativo cl\'asico.

\item Si $Z$ es cualquier proceso estacionario y $S$ es un proceso de renovaci\'on que es independiente de $Z$, entonces $\left(Z,S\right)$ es WSR pero en general no es regenerativo cl\'asico

\end{itemize}

\end{Note}


\begin{Note}
Para cualquier proceso estoc\'astico $Z$, el proceso de trayectorias $\left(\theta_{s}Z\right)_{s\in\left[0,\infty\right)}$ es siempre un proceso de Markov.
\end{Note}



\begin{Teo}
Supongase que el par $\left(Z,S\right)$ es WSR con $\esp\left[X_{1}\right]<\infty$. Entonces $\left(Z^{*},S^{*}\right)$ en el teorema 2.1 es una versi\'on estacionaria de 
$\left(Z,S\right)$.
\end{Teo}


\begin{Teo}
Supongase que $\left(Z,S\right)$ es cycle-stationary con $\esp\left[X_{1}\right]<\infty$. Sea $U$ distribuida uniformemente en $\left[0,1\right)$ e independiente de $\left(Z^{0},S^{0}\right)$ y sea $\prob^{*}$ la medida de probabilidad en $\left(\Omega,\prob\right)$ definida por $$d\prob^{*}=\frac{X_{1}}{\esp\left[X_{1}\right]}d\prob$$. Sea $\left(Z^{*},S^{*}\right)$ con distribuci\'on $\prob^{*}\left(\theta_{UX_{1}}\left(Z^{0},S^{0}\right)\in\cdot\right)$. Entonces $\left(Z^{}*,S^{*}\right)$ es estacionario,
\begin{eqnarray*}
\esp\left[f\left(Z^{*},S^{*}\right)\right]=\esp\left[\int_{0}^{X_{1}}f\left(\theta_{s}\left(Z^{0},S^{0}\right)\right)ds\right]/\esp\left[X_{1}\right]
\end{eqnarray*}
$f\in\mathcal{H}\otimes\mathcal{L}^{+}$, and $S_{0}^{*}$ es continuo con funci\'on distribuci\'on $G_{\infty}$ definida por $$G_{\infty}\left(x\right):=\frac{\esp\left[X_{1}\right]\wedge x}{\esp\left[X_{1}\right]}$$ para $x\geq0$ y densidad $\prob\left[X_{1}>x\right]/\esp\left[X_{1}\right]$, con $x\geq0$.

\end{Teo}


\begin{Teo}
Sea $Z$ un Proceso Estoc\'astico un lado shift-medible \textit{one-sided shift-measurable stochastic process}, (PEOSSM),
y $S_{0}$ y $S_{1}$ tiempos aleatorios tales que $0\leq S_{0}<S_{1}$ y
\begin{equation}
\theta_{S_{1}}Z=\theta_{S_{0}}Z\textrm{ en distribuci\'on}.
\end{equation}

Entonces el espacio de probabilidad subyacente $\left(\Omega,\mathcal{F},\prob\right)$ puede extenderse para soportar una sucesi\'on de tiempos aleatorios $S$ tales que

\begin{eqnarray}
\theta_{S_{n}}\left(Z,S\right)=\left(Z^{0},S^{0}\right),n\geq0,\textrm{ en distribuci\'on},\\
\left(Z,S_{0},S_{1}\right)\textrm{ depende de }\left(X_{2},X_{3},\ldots\right)\textrm{ solamente a traves de }\theta_{S_{1}}Z.
\end{eqnarray}
\end{Teo}

\begin{Def}
Un elemento aleatorio en un espacio medible $\left(E,\mathcal{E}\right)$ en un espacio de probabilidad $\left(\Omega,\mathcal{F},\prob\right)$ a $\left(E,\mathcal{E}\right)$, es decir,
para $A\in \mathcal{E}$,  se tiene que $\left\{Y\in A\right\}\in\mathcal{F}$, donde $\left\{Y\in A\right\}:=\left\{w\in\Omega:Y\left(w\right)\in A\right\}=:Y^{-1}A$.
\end{Def}

\begin{Note}
Tambi\'en se dice que $Y$ est\'a soportado por el espacio de probabilidad $\left(\Omega,\mathcal{F},\prob\right)$ y que $Y$ es un mapeo medible de $\Omega$ en $E$, es decir, es $\mathcal{F}/\mathcal{E}$ medible.
\end{Note}

\begin{Def}
Para cada $i\in \mathbb{I}$ sea $P_{i}$ una medida de probabilidad en un espacio medible $\left(E_{i},\mathcal{E}_{i}\right)$. Se define el espacio producto
$\otimes_{i\in\mathbb{I}}\left(E_{i},\mathcal{E}_{i}\right):=\left(\prod_{i\in\mathbb{I}}E_{i},\otimes_{i\in\mathbb{I}}\mathcal{E}_{i}\right)$, donde $\prod_{i\in\mathbb{I}}E_{i}$ es el producto cartesiano de los $E_{i}$'s, y $\otimes_{i\in\mathbb{I}}\mathcal{E}_{i}$ es la $\sigma$-\'algebra producto, es decir, es la $\sigma$-\'algebra m\'as peque\~na en $\prod_{i\in\mathbb{I}}E_{i}$ que hace al $i$-\'esimo mapeo proyecci\'on en $E_{i}$ medible para toda $i\in\mathbb{I}$ es la $\sigma$-\'algebra inducida por los mapeos proyecci\'on. $$\otimes_{i\in\mathbb{I}}\mathcal{E}_{i}:=\sigma\left\{\left\{y:y_{i}\in A\right\}:i\in\mathbb{I}\textrm{ y }A\in\mathcal{E}_{i}\right\}.$$
\end{Def}

\begin{Def}
Un espacio de probabilidad $\left(\tilde{\Omega},\tilde{\mathcal{F}},\tilde{\prob}\right)$ es una extensi\'on de otro espacio de probabilidad $\left(\Omega,\mathcal{F},\prob\right)$ si $\left(\tilde{\Omega},\tilde{\mathcal{F}},\tilde{\prob}\right)$ soporta un elemento aleatorio $\xi\in\left(\Omega,\mathcal{F}\right)$ que tienen a $\prob$ como distribuci\'on.
\end{Def}

\begin{Teo}
Sea $\mathbb{I}$ un conjunto de \'indices arbitrario. Para cada $i\in\mathbb{I}$ sea $P_{i}$ una medida de probabilidad en un espacio medible $\left(E_{i},\mathcal{E}_{i}\right)$. Entonces existe una \'unica medida de probabilidad $\otimes_{i\in\mathbb{I}}P_{i}$ en $\otimes_{i\in\mathbb{I}}\left(E_{i},\mathcal{E}_{i}\right)$ tal que 

\begin{eqnarray*}
\otimes_{i\in\mathbb{I}}P_{i}\left(y\in\prod_{i\in\mathbb{I}}E_{i}:y_{i}\in A_{i_{1}},\ldots,y_{n}\in A_{i_{n}}\right)=P_{i_{1}}\left(A_{i_{n}}\right)\cdots P_{i_{n}}\left(A_{i_{n}}\right)
\end{eqnarray*}
para todos los enteros $n>0$, toda $i_{1},\ldots,i_{n}\in\mathbb{I}$ y todo $A_{i_{1}}\in\mathcal{E}_{i_{1}},\ldots,A_{i_{n}}\in\mathcal{E}_{i_{n}}$
\end{Teo}

La medida $\otimes_{i\in\mathbb{I}}P_{i}$ es llamada la medida producto y $\otimes_{i\in\mathbb{I}}\left(E_{i},\mathcal{E}_{i},P_{i}\right):=\left(\prod_{i\in\mathbb{I}},E_{i},\otimes_{i\in\mathbb{I}}\mathcal{E}_{i},\otimes_{i\in\mathbb{I}}P_{i}\right)$, es llamado espacio de probabilidad producto.


\begin{Def}
Un espacio medible $\left(E,\mathcal{E}\right)$ es \textit{Polaco} si existe una m\'etrica en $E$ tal que $E$ es completo, es decir cada sucesi\'on de Cauchy converge a un l\'imite en $E$, y \textit{separable}, $E$ tienen un subconjunto denso numerable, y tal que $\mathcal{E}$ es generado por conjuntos abiertos.
\end{Def}


\begin{Def}
Dos espacios medibles $\left(E,\mathcal{E}\right)$ y $\left(G,\mathcal{G}\right)$ son Borel equivalentes \textit{isomorfos} si existe una biyecci\'on $f:E\rightarrow G$ tal que $f$ es $\mathcal{E}/\mathcal{G}$ medible y su inversa $f^{-1}$ es $\mathcal{G}/\mathcal{E}$ medible. La biyecci\'on es una equivalencia de Borel.
\end{Def}

\begin{Def}
Un espacio medible  $\left(E,\mathcal{E}\right)$ es un \textit{espacio est\'andar} si es Borel equivalente a $\left(G,\mathcal{G}\right)$, donde $G$ es un subconjunto de Borel de $\left[0,1\right]$ y $\mathcal{G}$ son los subconjuntos de Borel de $G$.
\end{Def}

\begin{Note}
Cualquier espacio Polaco es un espacio est\'andar.
\end{Note}


\begin{Def}
Un proceso estoc\'astico con conjunto de \'indices $\mathbb{I}$ y espacio de estados $\left(E,\mathcal{E}\right)$ es una familia $Z=\left(\mathbb{Z}_{s}\right)_{s\in\mathbb{I}}$ donde $\mathbb{Z}_{s}$ son elementos aleatorios definidos en un espacio de probabilidad com\'un $\left(\Omega,\mathcal{F},\prob\right)$ y todos toman valores en $\left(E,\mathcal{E}\right)$.
\end{Def}

\begin{Def}
Un proceso estoc\'astico \textit{one-sided contiuous time} (\textbf{PEOSCT}) es un proceso estoc\'astico con conjunto de \'indices $\mathbb{I}=\left[0,\infty\right)$.
\end{Def}


Sea $\left(E^{\mathbb{I}},\mathcal{E}^{\mathbb{I}}\right)$ denota el espacio producto $\left(E^{\mathbb{I}},\mathcal{E}^{\mathbb{I}}\right):=\otimes_{s\in\mathbb{I}}\left(E,\mathcal{E}\right)$. Vamos a considerar $\mathbb{Z}$ como un mapeo aleatorio, es decir, como un elemento aleatorio en $\left(E^{\mathbb{I}},\mathcal{E}^{\mathbb{I}}\right)$ definido por $Z\left(w\right)=\left(Z_{s}\left(w\right)\right)_{s\in\mathbb{I}}$ y $w\in\Omega$.

\begin{Note}
La distribuci\'on de un proceso estoc\'astico $Z$ es la distribuci\'on de $Z$ como un elemento aleatorio en $\left(E^{\mathbb{I}},\mathcal{E}^{\mathbb{I}}\right)$. La distribuci\'on de $Z$ esta determinada de manera \'unica por las distribuciones finito dimensionales.
\end{Note}

\begin{Note}
En particular cuando $Z$ toma valores reales, es decir, $\left(E,\mathcal{E}\right)=\left(\mathbb{R},\mathcal{B}\right)$ las distribuciones finito dimensionales est\'an determinadas por las funciones de distribuci\'on finito dimensionales

\begin{eqnarray}
\prob\left(Z_{t_{1}}\leq x_{1},\ldots,Z_{t_{n}}\leq x_{n}\right),x_{1},\ldots,x_{n}\in\mathbb{R},t_{1},\ldots,t_{n}\in\mathbb{I},n\geq1.
\end{eqnarray}
\end{Note}

\begin{Note}
Para espacios polacos $\left(E,\mathcal{E}\right)$ el Teorema de Consistencia de Kolmogorov asegura que dada una colecci\'on de distribuciones finito dimensionales consistentes, siempre existe un proceso estoc\'astico que posee tales distribuciones finito dimensionales.
\end{Note}


\begin{Def}
Las trayectorias de $Z$ son las realizaciones $Z\left(w\right)$ para $w\in\Omega$ del mapeo aleatorio $Z$.
\end{Def}

\begin{Note}
Algunas restricciones se imponen sobre las trayectorias, por ejemplo que sean continuas por la derecha, o continuas por la derecha con l\'imites por la izquierda, o de manera m\'as general, se pedir\'a que caigan en alg\'un subconjunto $H$ de $E^{\mathbb{I}}$. En este caso es natural considerar a $Z$ como un elemento aleatorio que no est\'a en $\left(E^{\mathbb{I}},\mathcal{E}^{\mathbb{I}}\right)$ sino en $\left(H,\mathcal{H}\right)$, donde $\mathcal{H}$ es la $\sigma$-\'algebra generada por los mapeos proyecci\'on que toman a $z\in H$ a $z_{t}\in E$ para $t\in\mathbb{I}$. A $\mathcal{H}$ se le conoce como la traza de $H$ en $E^{\mathbb{I}}$, es decir,
\begin{eqnarray}
\mathcal{H}:=E^{\mathbb{I}}\cap H:=\left\{A\cap H:A\in E^{\mathbb{I}}\right\}.
\end{eqnarray}
\end{Note}


\begin{Note}
$Z$ tiene trayectorias con valores en $H$ y cada $Z_{t}$ es un mapeo medible de $\left(\Omega,\mathcal{F}\right)$ a $\left(H,\mathcal{H}\right)$. Cuando se considera un espacio de trayectorias en particular $H$, al espacio $\left(H,\mathcal{H}\right)$ se le llama el espacio de trayectorias de $Z$.
\end{Note}

\begin{Note}
La distribuci\'on del proceso estoc\'astico $Z$ con espacio de trayectorias $\left(H,\mathcal{H}\right)$ es la distribuci\'on de $Z$ como  un elemento aleatorio en $\left(H,\mathcal{H}\right)$. La distribuci\'on, nuevemente, est\'a determinada de manera \'unica por las distribuciones finito dimensionales.
\end{Note}


\begin{Def}
Sea $Z$ un PEOSCT  con espacio de estados $\left(E,\mathcal{E}\right)$ y sea $T$ un tiempo aleatorio en $\left[0,\infty\right)$. Por $Z_{T}$ se entiende el mapeo con valores en $E$ definido en $\Omega$ en la manera obvia:
\begin{eqnarray*}
Z_{T}\left(w\right):=Z_{T\left(w\right)}\left(w\right). w\in\Omega.
\end{eqnarray*}
\end{Def}

\begin{Def}
Un PEOSCT $Z$ es conjuntamente medible (\textbf{CM}) si el mapeo que toma $\left(w,t\right)\in\Omega\times\left[0,\infty\right)$ a $Z_{t}\left(w\right)\in E$ es $\mathcal{F}\otimes\mathcal{B}\left[0,\infty\right)/\mathcal{E}$ medible.
\end{Def}

\begin{Note}
Un PEOSCT-CM implica que el proceso es medible, dado que $Z_{T}$ es una composici\'on  de dos mapeos continuos: el primero que toma $w$ en $\left(w,T\left(w\right)\right)$ es $\mathcal{F}/\mathcal{F}\otimes\mathcal{B}\left[0,\infty\right)$ medible, mientras que el segundo toma $\left(w,T\left(w\right)\right)$ en $Z_{T\left(w\right)}\left(w\right)$ es $\mathcal{F}\otimes\mathcal{B}\left[0,\infty\right)/\mathcal{E}$ medible.
\end{Note}


\begin{Def}
Un PEOSCT con espacio de estados $\left(H,\mathcal{H}\right)$ es can\'onicamente conjuntamente medible (\textbf{CCM}) si el mapeo $\left(z,t\right)\in H\times\left[0,\infty\right)$ en $Z_{t}\in E$ es $\mathcal{H}\otimes\mathcal{B}\left[0,\infty\right)/\mathcal{E}$ medible.
\end{Def}

\begin{Note}
Un PEOSCTCCM implica que el proceso es CM, dado que un PECCM $Z$ es un mapeo de $\Omega\times\left[0,\infty\right)$ a $E$, es la composici\'on de dos mapeos medibles: el primero, toma $\left(w,t\right)$ en $\left(Z\left(w\right),t\right)$ es $\mathcal{F}\otimes\mathcal{B}\left[0,\infty\right)/\mathcal{H}\otimes\mathcal{B}\left[0,\infty\right)$ medible, y el segundo que toma $\left(Z\left(w\right),t\right)$  en $Z_{t}\left(w\right)$ es $\mathcal{H}\otimes\mathcal{B}\left[0,\infty\right)/\mathcal{E}$ medible. Por tanto CCM es una condici\'on m\'as fuerte que CM.
\end{Note}

\begin{Def}
Un conjunto de trayectorias $H$ de un PEOSCT $Z$, es internamente shift-invariante (\textbf{ISI}) si 
\begin{eqnarray*}
\left\{\left(z_{t+s}\right)_{s\in\left[0,\infty\right)}:z\in H\right\}=H\textrm{, }t\in\left[0,\infty\right).
\end{eqnarray*}
\end{Def}


\begin{Def}
Dado un PEOSCTISI, se define el mapeo-shift $\theta_{t}$, $t\in\left[0,\infty\right)$, de $H$ a $H$ por 
\begin{eqnarray*}
\theta_{t}z=\left(z_{t+s}\right)_{s\in\left[0,\infty\right)}\textrm{, }z\in H.
\end{eqnarray*}
\end{Def}

\begin{Def}
Se dice que un proceso $Z$ es shift-medible (\textbf{SM}) si $Z$ tiene un conjunto de trayectorias $H$ que es ISI y adem\'as el mapeo que toma $\left(z,t\right)\in H\times\left[0,\infty\right)$ en $\theta_{t}z\in H$ es $\mathcal{H}\otimes\mathcal{B}\left[0,\infty\right)/\mathcal{H}$ medible.
\end{Def}

\begin{Note}
Un proceso estoc\'astico con conjunto de trayectorias $H$ ISI es shift-medible si y s\'olo si es CCM
\end{Note}

\begin{Note}
\begin{itemize}
\item Dado el espacio polaco $\left(E,\mathcal{E}\right)$ se tiene el  conjunto de trayectorias $D_{E}\left[0,\infty\right)$ que es ISI, entonces cumpe con ser CCM.

\item Si $G$ es abierto, podemos cubrirlo por bolas abiertas cuay cerradura este contenida en $G$, y como $G$ es segundo numerable como subespacio de $E$, lo podemos cubrir por una cantidad numerable de bolas abiertas.

\end{itemize}
\end{Note}


\begin{Note}
Los procesos estoc\'asticos $Z$ a tiempo discreto con espacio de estados polaco, tambi\'en tiene un espacio de trayectorias polaco y por tanto tiene distribuciones condicionales regulares.
\end{Note}

\begin{Teo}
El producto numerable de espacios polacos es polaco.
\end{Teo}


\begin{Def}
Sea $\left(\Omega,\mathcal{F},\prob\right)$ espacio de probabilidad que soporta al proceso $Z=\left(Z_{s}\right)_{s\in\left[0,\infty\right)}$ y $S=\left(S_{k}\right)_{0}^{\infty}$ donde $Z$ es un PEOSCTM con espacio de estados $\left(E,\mathcal{E}\right)$  y espacio de trayectorias $\left(H,\mathcal{H}\right)$  y adem\'as $S$ es una sucesi\'on de tiempos aleatorios one-sided que satisfacen la condici\'on $0\leq S_{0}<S_{1}<\cdots\rightarrow\infty$. Considerando $S$ como un mapeo medible de $\left(\Omega,\mathcal{F}\right)$ al espacio sucesi\'on $\left(L,\mathcal{L}\right)$, donde 
\begin{eqnarray*}
L=\left\{\left(s_{k}\right)_{0}^{\infty}\in\left[0,\infty\right)^{\left\{0,1,\ldots\right\}}:s_{0}<s_{1}<\cdots\rightarrow\infty\right\},
\end{eqnarray*}
donde $\mathcal{L}$ son los subconjuntos de Borel de $L$, es decir, $\mathcal{L}=L\cap\mathcal{B}^{\left\{0,1,\ldots\right\}}$.

As\'i el par $\left(Z,S\right)$ es un mapeo medible de  $\left(\Omega,\mathcal{F}\right)$ en $\left(H\times L,\mathcal{H}\otimes\mathcal{L}\right)$. El par $\mathcal{H}\otimes\mathcal{L}^{+}$ denotar\'a la clase de todas las funciones medibles de $\left(H\times L,\mathcal{H}\otimes\mathcal{L}\right)$ en $\left(\left[0,\infty\right),\mathcal{B}\left[0,\infty\right)\right)$.
\end{Def}


\begin{Def}
Sea $\theta_{t}$ el mapeo-shift conjunto de $H\times L$ en $H\times L$ dado por
\begin{eqnarray*}
\theta_{t}\left(z,\left(s_{k}\right)_{0}^{\infty}\right)=\theta_{t}\left(z,\left(s_{n_{t-}+k}-t\right)_{0}^{\infty}\right)
\end{eqnarray*}
donde 
$n_{t-}=inf\left\{n\geq1:s_{n}\geq t\right\}$.
\end{Def}

\begin{Note}
Con la finalidad de poder realizar los shift's sin complicaciones de medibilidad, se supondr\'a que $Z$ es shit-medible, es decir, el conjunto de trayectorias $H$ es invariante bajo shifts del tiempo y el mapeo que toma $\left(z,t\right)\in H\times\left[0,\infty\right)$ en $z_{t}\in E$ es $\mathcal{H}\otimes\mathcal{B}\left[0,\infty\right)/\mathcal{E}$ medible.
\end{Note}

\begin{Def}
Dado un proceso \textbf{PEOSSM} (Proceso Estoc\'astico One Side Shift Medible) $Z$, se dice regenerativo cl\'asico con tiempos de regeneraci\'on $S$ si 

\begin{eqnarray*}
\theta_{S_{n}}\left(Z,S\right)=\left(Z^{0},S^{0}\right),n\geq0
\end{eqnarray*}
y adem\'as $\theta_{S_{n}}\left(Z,S\right)$ es independiente de $\left(\left(Z_{s}\right)s\in\left[0,S_{n}\right),S_{0},\ldots,S_{n}\right)$
Si lo anterior se cumple, al par $\left(Z,S\right)$ se le llama regenerativo cl\'asico.
\end{Def}

\begin{Note}
Si el par $\left(Z,S\right)$ es regenerativo cl\'asico, entonces las longitudes de los ciclos $X_{1},X_{2},\ldots,$ son i.i.d. e independientes de la longitud del retraso $S_{0}$, es decir, $S$ es un proceso de renovaci\'on. Las longitudes de los ciclos tambi\'en son llamados tiempos de inter-regeneraci\'on y tiempos de ocurrencia.

\end{Note}

\begin{Teo}
Sup\'ongase que el par $\left(Z,S\right)$ es regenerativo cl\'asico con $\esp\left[X_{1}\right]<\infty$. Entonces $\left(Z^{*},S^{*}\right)$ en el teorema 2.1 es una versi\'on estacionaria de $\left(Z,S\right)$. Adem\'as, si $X_{1}$ es lattice con span $d$, entonces $\left(Z^{**},S^{**}\right)$ en el teorema 2.2 es una versi\'on periodicamente estacionaria de $\left(Z,S\right)$ con periodo $d$.

\end{Teo}

\begin{Def}
Una variable aleatoria $X_{1}$ es \textit{spread out} si existe una $n\geq1$ y una  funci\'on $f\in\mathcal{B}^{+}$ tal que $\int_{\rea}f\left(x\right)dx>0$ con $X_{2},X_{3},\ldots,X_{n}$ copias i.i.d  de $X_{1}$, $$\prob\left(X_{1}+\cdots+X_{n}\in B\right)\geq\int_{B}f\left(x\right)dx$$ para $B\in\mathcal{B}$.

\end{Def}



\begin{Def}
Dado un proceso estoc\'astico $Z$ se le llama \textit{wide-sense regenerative} (\textbf{WSR}) con tiempos de regeneraci\'on $S$ si $\theta_{S_{n}}\left(Z,S\right)=\left(Z^{0},S^{0}\right)$ para $n\geq0$ en distribuci\'on y $\theta_{S_{n}}\left(Z,S\right)$ es independiente de $\left(S_{0},S_{1},\ldots,S_{n}\right)$ para $n\geq0$.
Se dice que el par $\left(Z,S\right)$ es WSR si lo anterior se cumple.
\end{Def}


\begin{Note}
\begin{itemize}
\item El proceso de trayectorias $\left(\theta_{s}Z\right)_{s\in\left[0,\infty\right)}$ es WSR con tiempos de regeneraci\'on $S$ pero no es regenerativo cl\'asico.

\item Si $Z$ es cualquier proceso estacionario y $S$ es un proceso de renovaci\'on que es independiente de $Z$, entonces $\left(Z,S\right)$ es WSR pero en general no es regenerativo cl\'asico

\end{itemize}

\end{Note}


\begin{Note}
Para cualquier proceso estoc\'astico $Z$, el proceso de trayectorias $\left(\theta_{s}Z\right)_{s\in\left[0,\infty\right)}$ es siempre un proceso de Markov.
\end{Note}



\begin{Teo}
Supongase que el par $\left(Z,S\right)$ es WSR con $\esp\left[X_{1}\right]<\infty$. Entonces $\left(Z^{*},S^{*}\right)$ en el teorema 2.1 es una versi\'on estacionaria de 
$\left(Z,S\right)$.
\end{Teo}


\begin{Teo}
Supongase que $\left(Z,S\right)$ es cycle-stationary con $\esp\left[X_{1}\right]<\infty$. Sea $U$ distribuida uniformemente en $\left[0,1\right)$ e independiente de $\left(Z^{0},S^{0}\right)$ y sea $\prob^{*}$ la medida de probabilidad en $\left(\Omega,\prob\right)$ definida por $$d\prob^{*}=\frac{X_{1}}{\esp\left[X_{1}\right]}d\prob$$. Sea $\left(Z^{*},S^{*}\right)$ con distribuci\'on $\prob^{*}\left(\theta_{UX_{1}}\left(Z^{0},S^{0}\right)\in\cdot\right)$. Entonces $\left(Z^{}*,S^{*}\right)$ es estacionario,
\begin{eqnarray*}
\esp\left[f\left(Z^{*},S^{*}\right)\right]=\esp\left[\int_{0}^{X_{1}}f\left(\theta_{s}\left(Z^{0},S^{0}\right)\right)ds\right]/\esp\left[X_{1}\right]
\end{eqnarray*}
$f\in\mathcal{H}\otimes\mathcal{L}^{+}$, and $S_{0}^{*}$ es continuo con funci\'on distribuci\'on $G_{\infty}$ definida por $$G_{\infty}\left(x\right):=\frac{\esp\left[X_{1}\right]\wedge x}{\esp\left[X_{1}\right]}$$ para $x\geq0$ y densidad $\prob\left[X_{1}>x\right]/\esp\left[X_{1}\right]$, con $x\geq0$.

\end{Teo}


\begin{Teo}
Sea $Z$ un Proceso Estoc\'astico un lado shift-medible \textit{one-sided shift-measurable stochastic process}, (PEOSSM),
y $S_{0}$ y $S_{1}$ tiempos aleatorios tales que $0\leq S_{0}<S_{1}$ y
\begin{equation}
\theta_{S_{1}}Z=\theta_{S_{0}}Z\textrm{ en distribuci\'on}.
\end{equation}

Entonces el espacio de probabilidad subyacente $\left(\Omega,\mathcal{F},\prob\right)$ puede extenderse para soportar una sucesi\'on de tiempos aleatorios $S$ tales que

\begin{eqnarray}
\theta_{S_{n}}\left(Z,S\right)=\left(Z^{0},S^{0}\right),n\geq0,\textrm{ en distribuci\'on},\\
\left(Z,S_{0},S_{1}\right)\textrm{ depende de }\left(X_{2},X_{3},\ldots\right)\textrm{ solamente a traves de }\theta_{S_{1}}Z.
\end{eqnarray}
\end{Teo}




\begin{Def}
Un elemento aleatorio en un espacio medible $\left(E,\mathcal{E}\right)$ en un espacio de probabilidad $\left(\Omega,\mathcal{F},\prob\right)$ a $\left(E,\mathcal{E}\right)$, es decir,
para $A\in \mathcal{E}$,  se tiene que $\left\{Y\in A\right\}\in\mathcal{F}$, donde $\left\{Y\in A\right\}:=\left\{w\in\Omega:Y\left(w\right)\in A\right\}=:Y^{-1}A$.
\end{Def}

\begin{Note}
Tambi\'en se dice que $Y$ est\'a soportado por el espacio de probabilidad $\left(\Omega,\mathcal{F},\prob\right)$ y que $Y$ es un mapeo medible de $\Omega$ en $E$, es decir, es $\mathcal{F}/\mathcal{E}$ medible.
\end{Note}

\begin{Def}
Para cada $i\in \mathbb{I}$ sea $P_{i}$ una medida de probabilidad en un espacio medible $\left(E_{i},\mathcal{E}_{i}\right)$. Se define el espacio producto
$\otimes_{i\in\mathbb{I}}\left(E_{i},\mathcal{E}_{i}\right):=\left(\prod_{i\in\mathbb{I}}E_{i},\otimes_{i\in\mathbb{I}}\mathcal{E}_{i}\right)$, donde $\prod_{i\in\mathbb{I}}E_{i}$ es el producto cartesiano de los $E_{i}$'s, y $\otimes_{i\in\mathbb{I}}\mathcal{E}_{i}$ es la $\sigma$-\'algebra producto, es decir, es la $\sigma$-\'algebra m\'as peque\~na en $\prod_{i\in\mathbb{I}}E_{i}$ que hace al $i$-\'esimo mapeo proyecci\'on en $E_{i}$ medible para toda $i\in\mathbb{I}$ es la $\sigma$-\'algebra inducida por los mapeos proyecci\'on. $$\otimes_{i\in\mathbb{I}}\mathcal{E}_{i}:=\sigma\left\{\left\{y:y_{i}\in A\right\}:i\in\mathbb{I}\textrm{ y }A\in\mathcal{E}_{i}\right\}.$$
\end{Def}

\begin{Def}
Un espacio de probabilidad $\left(\tilde{\Omega},\tilde{\mathcal{F}},\tilde{\prob}\right)$ es una extensi\'on de otro espacio de probabilidad $\left(\Omega,\mathcal{F},\prob\right)$ si $\left(\tilde{\Omega},\tilde{\mathcal{F}},\tilde{\prob}\right)$ soporta un elemento aleatorio $\xi\in\left(\Omega,\mathcal{F}\right)$ que tienen a $\prob$ como distribuci\'on.
\end{Def}

\begin{Teo}
Sea $\mathbb{I}$ un conjunto de \'indices arbitrario. Para cada $i\in\mathbb{I}$ sea $P_{i}$ una medida de probabilidad en un espacio medible $\left(E_{i},\mathcal{E}_{i}\right)$. Entonces existe una \'unica medida de probabilidad $\otimes_{i\in\mathbb{I}}P_{i}$ en $\otimes_{i\in\mathbb{I}}\left(E_{i},\mathcal{E}_{i}\right)$ tal que 

\begin{eqnarray*}
\otimes_{i\in\mathbb{I}}P_{i}\left(y\in\prod_{i\in\mathbb{I}}E_{i}:y_{i}\in A_{i_{1}},\ldots,y_{n}\in A_{i_{n}}\right)=P_{i_{1}}\left(A_{i_{n}}\right)\cdots P_{i_{n}}\left(A_{i_{n}}\right)
\end{eqnarray*}
para todos los enteros $n>0$, toda $i_{1},\ldots,i_{n}\in\mathbb{I}$ y todo $A_{i_{1}}\in\mathcal{E}_{i_{1}},\ldots,A_{i_{n}}\in\mathcal{E}_{i_{n}}$
\end{Teo}

La medida $\otimes_{i\in\mathbb{I}}P_{i}$ es llamada la medida producto y $\otimes_{i\in\mathbb{I}}\left(E_{i},\mathcal{E}_{i},P_{i}\right):=\left(\prod_{i\in\mathbb{I}},E_{i},\otimes_{i\in\mathbb{I}}\mathcal{E}_{i},\otimes_{i\in\mathbb{I}}P_{i}\right)$, es llamado espacio de probabilidad producto.


\begin{Def}
Un espacio medible $\left(E,\mathcal{E}\right)$ es \textit{Polaco} si existe una m\'etrica en $E$ tal que $E$ es completo, es decir cada sucesi\'on de Cauchy converge a un l\'imite en $E$, y \textit{separable}, $E$ tienen un subconjunto denso numerable, y tal que $\mathcal{E}$ es generado por conjuntos abiertos.
\end{Def}


\begin{Def}
Dos espacios medibles $\left(E,\mathcal{E}\right)$ y $\left(G,\mathcal{G}\right)$ son Borel equivalentes \textit{isomorfos} si existe una biyecci\'on $f:E\rightarrow G$ tal que $f$ es $\mathcal{E}/\mathcal{G}$ medible y su inversa $f^{-1}$ es $\mathcal{G}/\mathcal{E}$ medible. La biyecci\'on es una equivalencia de Borel.
\end{Def}

\begin{Def}
Un espacio medible  $\left(E,\mathcal{E}\right)$ es un \textit{espacio est\'andar} si es Borel equivalente a $\left(G,\mathcal{G}\right)$, donde $G$ es un subconjunto de Borel de $\left[0,1\right]$ y $\mathcal{G}$ son los subconjuntos de Borel de $G$.
\end{Def}

\begin{Note}
Cualquier espacio Polaco es un espacio est\'andar.
\end{Note}


\begin{Def}
Un proceso estoc\'astico con conjunto de \'indices $\mathbb{I}$ y espacio de estados $\left(E,\mathcal{E}\right)$ es una familia $Z=\left(\mathbb{Z}_{s}\right)_{s\in\mathbb{I}}$ donde $\mathbb{Z}_{s}$ son elementos aleatorios definidos en un espacio de probabilidad com\'un $\left(\Omega,\mathcal{F},\prob\right)$ y todos toman valores en $\left(E,\mathcal{E}\right)$.
\end{Def}

\begin{Def}
Un proceso estoc\'astico \textit{one-sided contiuous time} (\textbf{PEOSCT}) es un proceso estoc\'astico con conjunto de \'indices $\mathbb{I}=\left[0,\infty\right)$.
\end{Def}


Sea $\left(E^{\mathbb{I}},\mathcal{E}^{\mathbb{I}}\right)$ denota el espacio producto $\left(E^{\mathbb{I}},\mathcal{E}^{\mathbb{I}}\right):=\otimes_{s\in\mathbb{I}}\left(E,\mathcal{E}\right)$. Vamos a considerar $\mathbb{Z}$ como un mapeo aleatorio, es decir, como un elemento aleatorio en $\left(E^{\mathbb{I}},\mathcal{E}^{\mathbb{I}}\right)$ definido por $Z\left(w\right)=\left(Z_{s}\left(w\right)\right)_{s\in\mathbb{I}}$ y $w\in\Omega$.

\begin{Note}
La distribuci\'on de un proceso estoc\'astico $Z$ es la distribuci\'on de $Z$ como un elemento aleatorio en $\left(E^{\mathbb{I}},\mathcal{E}^{\mathbb{I}}\right)$. La distribuci\'on de $Z$ esta determinada de manera \'unica por las distribuciones finito dimensionales.
\end{Note}

\begin{Note}
En particular cuando $Z$ toma valores reales, es decir, $\left(E,\mathcal{E}\right)=\left(\mathbb{R},\mathcal{B}\right)$ las distribuciones finito dimensionales est\'an determinadas por las funciones de distribuci\'on finito dimensionales

\begin{eqnarray}
\prob\left(Z_{t_{1}}\leq x_{1},\ldots,Z_{t_{n}}\leq x_{n}\right),x_{1},\ldots,x_{n}\in\mathbb{R},t_{1},\ldots,t_{n}\in\mathbb{I},n\geq1.
\end{eqnarray}
\end{Note}

\begin{Note}
Para espacios polacos $\left(E,\mathcal{E}\right)$ el Teorema de Consistencia de Kolmogorov asegura que dada una colecci\'on de distribuciones finito dimensionales consistentes, siempre existe un proceso estoc\'astico que posee tales distribuciones finito dimensionales.
\end{Note}


\begin{Def}
Las trayectorias de $Z$ son las realizaciones $Z\left(w\right)$ para $w\in\Omega$ del mapeo aleatorio $Z$.
\end{Def}

\begin{Note}
Algunas restricciones se imponen sobre las trayectorias, por ejemplo que sean continuas por la derecha, o continuas por la derecha con l\'imites por la izquierda, o de manera m\'as general, se pedir\'a que caigan en alg\'un subconjunto $H$ de $E^{\mathbb{I}}$. En este caso es natural considerar a $Z$ como un elemento aleatorio que no est\'a en $\left(E^{\mathbb{I}},\mathcal{E}^{\mathbb{I}}\right)$ sino en $\left(H,\mathcal{H}\right)$, donde $\mathcal{H}$ es la $\sigma$-\'algebra generada por los mapeos proyecci\'on que toman a $z\in H$ a $z_{t}\in E$ para $t\in\mathbb{I}$. A $\mathcal{H}$ se le conoce como la traza de $H$ en $E^{\mathbb{I}}$, es decir,
\begin{eqnarray}
\mathcal{H}:=E^{\mathbb{I}}\cap H:=\left\{A\cap H:A\in E^{\mathbb{I}}\right\}.
\end{eqnarray}
\end{Note}


\begin{Note}
$Z$ tiene trayectorias con valores en $H$ y cada $Z_{t}$ es un mapeo medible de $\left(\Omega,\mathcal{F}\right)$ a $\left(H,\mathcal{H}\right)$. Cuando se considera un espacio de trayectorias en particular $H$, al espacio $\left(H,\mathcal{H}\right)$ se le llama el espacio de trayectorias de $Z$.
\end{Note}

\begin{Note}
La distribuci\'on del proceso estoc\'astico $Z$ con espacio de trayectorias $\left(H,\mathcal{H}\right)$ es la distribuci\'on de $Z$ como  un elemento aleatorio en $\left(H,\mathcal{H}\right)$. La distribuci\'on, nuevemente, est\'a determinada de manera \'unica por las distribuciones finito dimensionales.
\end{Note}


\begin{Def}
Sea $Z$ un PEOSCT  con espacio de estados $\left(E,\mathcal{E}\right)$ y sea $T$ un tiempo aleatorio en $\left[0,\infty\right)$. Por $Z_{T}$ se entiende el mapeo con valores en $E$ definido en $\Omega$ en la manera obvia:
\begin{eqnarray*}
Z_{T}\left(w\right):=Z_{T\left(w\right)}\left(w\right). w\in\Omega.
\end{eqnarray*}
\end{Def}

\begin{Def}
Un PEOSCT $Z$ es conjuntamente medible (\textbf{CM}) si el mapeo que toma $\left(w,t\right)\in\Omega\times\left[0,\infty\right)$ a $Z_{t}\left(w\right)\in E$ es $\mathcal{F}\otimes\mathcal{B}\left[0,\infty\right)/\mathcal{E}$ medible.
\end{Def}

\begin{Note}
Un PEOSCT-CM implica que el proceso es medible, dado que $Z_{T}$ es una composici\'on  de dos mapeos continuos: el primero que toma $w$ en $\left(w,T\left(w\right)\right)$ es $\mathcal{F}/\mathcal{F}\otimes\mathcal{B}\left[0,\infty\right)$ medible, mientras que el segundo toma $\left(w,T\left(w\right)\right)$ en $Z_{T\left(w\right)}\left(w\right)$ es $\mathcal{F}\otimes\mathcal{B}\left[0,\infty\right)/\mathcal{E}$ medible.
\end{Note}


\begin{Def}
Un PEOSCT con espacio de estados $\left(H,\mathcal{H}\right)$ es can\'onicamente conjuntamente medible (\textbf{CCM}) si el mapeo $\left(z,t\right)\in H\times\left[0,\infty\right)$ en $Z_{t}\in E$ es $\mathcal{H}\otimes\mathcal{B}\left[0,\infty\right)/\mathcal{E}$ medible.
\end{Def}

\begin{Note}
Un PEOSCTCCM implica que el proceso es CM, dado que un PECCM $Z$ es un mapeo de $\Omega\times\left[0,\infty\right)$ a $E$, es la composici\'on de dos mapeos medibles: el primero, toma $\left(w,t\right)$ en $\left(Z\left(w\right),t\right)$ es $\mathcal{F}\otimes\mathcal{B}\left[0,\infty\right)/\mathcal{H}\otimes\mathcal{B}\left[0,\infty\right)$ medible, y el segundo que toma $\left(Z\left(w\right),t\right)$  en $Z_{t}\left(w\right)$ es $\mathcal{H}\otimes\mathcal{B}\left[0,\infty\right)/\mathcal{E}$ medible. Por tanto CCM es una condici\'on m\'as fuerte que CM.
\end{Note}

\begin{Def}
Un conjunto de trayectorias $H$ de un PEOSCT $Z$, es internamente shift-invariante (\textbf{ISI}) si 
\begin{eqnarray*}
\left\{\left(z_{t+s}\right)_{s\in\left[0,\infty\right)}:z\in H\right\}=H\textrm{, }t\in\left[0,\infty\right).
\end{eqnarray*}
\end{Def}


\begin{Def}
Dado un PEOSCTISI, se define el mapeo-shift $\theta_{t}$, $t\in\left[0,\infty\right)$, de $H$ a $H$ por 
\begin{eqnarray*}
\theta_{t}z=\left(z_{t+s}\right)_{s\in\left[0,\infty\right)}\textrm{, }z\in H.
\end{eqnarray*}
\end{Def}

\begin{Def}
Se dice que un proceso $Z$ es shift-medible (\textbf{SM}) si $Z$ tiene un conjunto de trayectorias $H$ que es ISI y adem\'as el mapeo que toma $\left(z,t\right)\in H\times\left[0,\infty\right)$ en $\theta_{t}z\in H$ es $\mathcal{H}\otimes\mathcal{B}\left[0,\infty\right)/\mathcal{H}$ medible.
\end{Def}

\begin{Note}
Un proceso estoc\'astico con conjunto de trayectorias $H$ ISI es shift-medible si y s\'olo si es CCM
\end{Note}

\begin{Note}
\begin{itemize}
\item Dado el espacio polaco $\left(E,\mathcal{E}\right)$ se tiene el  conjunto de trayectorias $D_{E}\left[0,\infty\right)$ que es ISI, entonces cumpe con ser CCM.

\item Si $G$ es abierto, podemos cubrirlo por bolas abiertas cuay cerradura este contenida en $G$, y como $G$ es segundo numerable como subespacio de $E$, lo podemos cubrir por una cantidad numerable de bolas abiertas.

\end{itemize}
\end{Note}


\begin{Note}
Los procesos estoc\'asticos $Z$ a tiempo discreto con espacio de estados polaco, tambi\'en tiene un espacio de trayectorias polaco y por tanto tiene distribuciones condicionales regulares.
\end{Note}

\begin{Teo}
El producto numerable de espacios polacos es polaco.
\end{Teo}


\begin{Def}
Sea $\left(\Omega,\mathcal{F},\prob\right)$ espacio de probabilidad que soporta al proceso $Z=\left(Z_{s}\right)_{s\in\left[0,\infty\right)}$ y $S=\left(S_{k}\right)_{0}^{\infty}$ donde $Z$ es un PEOSCTM con espacio de estados $\left(E,\mathcal{E}\right)$  y espacio de trayectorias $\left(H,\mathcal{H}\right)$  y adem\'as $S$ es una sucesi\'on de tiempos aleatorios one-sided que satisfacen la condici\'on $0\leq S_{0}<S_{1}<\cdots\rightarrow\infty$. Considerando $S$ como un mapeo medible de $\left(\Omega,\mathcal{F}\right)$ al espacio sucesi\'on $\left(L,\mathcal{L}\right)$, donde 
\begin{eqnarray*}
L=\left\{\left(s_{k}\right)_{0}^{\infty}\in\left[0,\infty\right)^{\left\{0,1,\ldots\right\}}:s_{0}<s_{1}<\cdots\rightarrow\infty\right\},
\end{eqnarray*}
donde $\mathcal{L}$ son los subconjuntos de Borel de $L$, es decir, $\mathcal{L}=L\cap\mathcal{B}^{\left\{0,1,\ldots\right\}}$.

As\'i el par $\left(Z,S\right)$ es un mapeo medible de  $\left(\Omega,\mathcal{F}\right)$ en $\left(H\times L,\mathcal{H}\otimes\mathcal{L}\right)$. El par $\mathcal{H}\otimes\mathcal{L}^{+}$ denotar\'a la clase de todas las funciones medibles de $\left(H\times L,\mathcal{H}\otimes\mathcal{L}\right)$ en $\left(\left[0,\infty\right),\mathcal{B}\left[0,\infty\right)\right)$.
\end{Def}


\begin{Def}
Sea $\theta_{t}$ el mapeo-shift conjunto de $H\times L$ en $H\times L$ dado por
\begin{eqnarray*}
\theta_{t}\left(z,\left(s_{k}\right)_{0}^{\infty}\right)=\theta_{t}\left(z,\left(s_{n_{t-}+k}-t\right)_{0}^{\infty}\right)
\end{eqnarray*}
donde 
$n_{t-}=inf\left\{n\geq1:s_{n}\geq t\right\}$.
\end{Def}

\begin{Note}
Con la finalidad de poder realizar los shift's sin complicaciones de medibilidad, se supondr\'a que $Z$ es shit-medible, es decir, el conjunto de trayectorias $H$ es invariante bajo shifts del tiempo y el mapeo que toma $\left(z,t\right)\in H\times\left[0,\infty\right)$ en $z_{t}\in E$ es $\mathcal{H}\otimes\mathcal{B}\left[0,\infty\right)/\mathcal{E}$ medible.
\end{Note}

\begin{Def}
Dado un proceso \textbf{PEOSSM} (Proceso Estoc\'astico One Side Shift Medible) $Z$, se dice regenerativo cl\'asico con tiempos de regeneraci\'on $S$ si 

\begin{eqnarray*}
\theta_{S_{n}}\left(Z,S\right)=\left(Z^{0},S^{0}\right),n\geq0
\end{eqnarray*}
y adem\'as $\theta_{S_{n}}\left(Z,S\right)$ es independiente de $\left(\left(Z_{s}\right)s\in\left[0,S_{n}\right),S_{0},\ldots,S_{n}\right)$
Si lo anterior se cumple, al par $\left(Z,S\right)$ se le llama regenerativo cl\'asico.
\end{Def}

\begin{Note}
Si el par $\left(Z,S\right)$ es regenerativo cl\'asico, entonces las longitudes de los ciclos $X_{1},X_{2},\ldots,$ son i.i.d. e independientes de la longitud del retraso $S_{0}$, es decir, $S$ es un proceso de renovaci\'on. Las longitudes de los ciclos tambi\'en son llamados tiempos de inter-regeneraci\'on y tiempos de ocurrencia.

\end{Note}

\begin{Teo}
Sup\'ongase que el par $\left(Z,S\right)$ es regenerativo cl\'asico con $\esp\left[X_{1}\right]<\infty$. Entonces $\left(Z^{*},S^{*}\right)$ en el teorema 2.1 es una versi\'on estacionaria de $\left(Z,S\right)$. Adem\'as, si $X_{1}$ es lattice con span $d$, entonces $\left(Z^{**},S^{**}\right)$ en el teorema 2.2 es una versi\'on periodicamente estacionaria de $\left(Z,S\right)$ con periodo $d$.

\end{Teo}

\begin{Def}
Una variable aleatoria $X_{1}$ es \textit{spread out} si existe una $n\geq1$ y una  funci\'on $f\in\mathcal{B}^{+}$ tal que $\int_{\rea}f\left(x\right)dx>0$ con $X_{2},X_{3},\ldots,X_{n}$ copias i.i.d  de $X_{1}$, $$\prob\left(X_{1}+\cdots+X_{n}\in B\right)\geq\int_{B}f\left(x\right)dx$$ para $B\in\mathcal{B}$.

\end{Def}



\begin{Def}
Dado un proceso estoc\'astico $Z$ se le llama \textit{wide-sense regenerative} (\textbf{WSR}) con tiempos de regeneraci\'on $S$ si $\theta_{S_{n}}\left(Z,S\right)=\left(Z^{0},S^{0}\right)$ para $n\geq0$ en distribuci\'on y $\theta_{S_{n}}\left(Z,S\right)$ es independiente de $\left(S_{0},S_{1},\ldots,S_{n}\right)$ para $n\geq0$.
Se dice que el par $\left(Z,S\right)$ es WSR si lo anterior se cumple.
\end{Def}


\begin{Note}
\begin{itemize}
\item El proceso de trayectorias $\left(\theta_{s}Z\right)_{s\in\left[0,\infty\right)}$ es WSR con tiempos de regeneraci\'on $S$ pero no es regenerativo cl\'asico.

\item Si $Z$ es cualquier proceso estacionario y $S$ es un proceso de renovaci\'on que es independiente de $Z$, entonces $\left(Z,S\right)$ es WSR pero en general no es regenerativo cl\'asico

\end{itemize}

\end{Note}


\begin{Note}
Para cualquier proceso estoc\'astico $Z$, el proceso de trayectorias $\left(\theta_{s}Z\right)_{s\in\left[0,\infty\right)}$ es siempre un proceso de Markov.
\end{Note}



\begin{Teo}
Supongase que el par $\left(Z,S\right)$ es WSR con $\esp\left[X_{1}\right]<\infty$. Entonces $\left(Z^{*},S^{*}\right)$ en el teorema 2.1 es una versi\'on estacionaria de 
$\left(Z,S\right)$.
\end{Teo}


\begin{Teo}
Supongase que $\left(Z,S\right)$ es cycle-stationary con $\esp\left[X_{1}\right]<\infty$. Sea $U$ distribuida uniformemente en $\left[0,1\right)$ e independiente de $\left(Z^{0},S^{0}\right)$ y sea $\prob^{*}$ la medida de probabilidad en $\left(\Omega,\prob\right)$ definida por $$d\prob^{*}=\frac{X_{1}}{\esp\left[X_{1}\right]}d\prob$$. Sea $\left(Z^{*},S^{*}\right)$ con distribuci\'on $\prob^{*}\left(\theta_{UX_{1}}\left(Z^{0},S^{0}\right)\in\cdot\right)$. Entonces $\left(Z^{}*,S^{*}\right)$ es estacionario,
\begin{eqnarray*}
\esp\left[f\left(Z^{*},S^{*}\right)\right]=\esp\left[\int_{0}^{X_{1}}f\left(\theta_{s}\left(Z^{0},S^{0}\right)\right)ds\right]/\esp\left[X_{1}\right]
\end{eqnarray*}
$f\in\mathcal{H}\otimes\mathcal{L}^{+}$, and $S_{0}^{*}$ es continuo con funci\'on distribuci\'on $G_{\infty}$ definida por $$G_{\infty}\left(x\right):=\frac{\esp\left[X_{1}\right]\wedge x}{\esp\left[X_{1}\right]}$$ para $x\geq0$ y densidad $\prob\left[X_{1}>x\right]/\esp\left[X_{1}\right]$, con $x\geq0$.

\end{Teo}


\begin{Teo}
Sea $Z$ un Proceso Estoc\'astico un lado shift-medible \textit{one-sided shift-measurable stochastic process}, (PEOSSM),
y $S_{0}$ y $S_{1}$ tiempos aleatorios tales que $0\leq S_{0}<S_{1}$ y
\begin{equation}
\theta_{S_{1}}Z=\theta_{S_{0}}Z\textrm{ en distribuci\'on}.
\end{equation}

Entonces el espacio de probabilidad subyacente $\left(\Omega,\mathcal{F},\prob\right)$ puede extenderse para soportar una sucesi\'on de tiempos aleatorios $S$ tales que

\begin{eqnarray}
\theta_{S_{n}}\left(Z,S\right)=\left(Z^{0},S^{0}\right),n\geq0,\textrm{ en distribuci\'on},\\
\left(Z,S_{0},S_{1}\right)\textrm{ depende de }\left(X_{2},X_{3},\ldots\right)\textrm{ solamente a traves de }\theta_{S_{1}}Z.
\end{eqnarray}
\end{Teo}
%______________________________________________________________________


\section{Procesos Regenerativos}
%________________________________________________________________________

%________________________________________________________________________
\subsection{Procesos Regenerativos Sigman, Thorisson y Wolff \cite{Sigman2}}
%________________________________________________________________________


\begin{Def}[Definici\'on Cl\'asica]
Un proceso estoc\'astico $X=\left\{X\left(t\right):t\geq0\right\}$ es llamado regenerativo is existe una variable aleatoria $R_{1}>0$ tal que
\begin{itemize}
\item[i)] $\left\{X\left(t+R_{1}\right):t\geq0\right\}$ es independiente de $\left\{\left\{X\left(t\right):t<R_{1}\right\},\right\}$
\item[ii)] $\left\{X\left(t+R_{1}\right):t\geq0\right\}$ es estoc\'asticamente equivalente a $\left\{X\left(t\right):t>0\right\}$
\end{itemize}

Llamamos a $R_{1}$ tiempo de regeneraci\'on, y decimos que $X$ se regenera en este punto.
\end{Def}

$\left\{X\left(t+R_{1}\right)\right\}$ es regenerativo con tiempo de regeneraci\'on $R_{2}$, independiente de $R_{1}$ pero con la misma distribuci\'on que $R_{1}$. Procediendo de esta manera se obtiene una secuencia de variables aleatorias independientes e id\'enticamente distribuidas $\left\{R_{n}\right\}$ llamados longitudes de ciclo. Si definimos a $Z_{k}\equiv R_{1}+R_{2}+\cdots+R_{k}$, se tiene un proceso de renovaci\'on llamado proceso de renovaci\'on encajado para $X$.


\begin{Note}
La existencia de un primer tiempo de regeneraci\'on, $R_{1}$, implica la existencia de una sucesi\'on completa de estos tiempos $R_{1},R_{2}\ldots,$ que satisfacen la propiedad deseada \cite{Sigman2}.
\end{Note}


\begin{Note} Para la cola $GI/GI/1$ los usuarios arriban con tiempos $t_{n}$ y son atendidos con tiempos de servicio $S_{n}$, los tiempos de arribo forman un proceso de renovaci\'on  con tiempos entre arribos independientes e identicamente distribuidos (\texttt{i.i.d.})$T_{n}=t_{n}-t_{n-1}$, adem\'as los tiempos de servicio son \texttt{i.i.d.} e independientes de los procesos de arribo. Por \textit{estable} se entiende que $\esp S_{n}<\esp T_{n}<\infty$.
\end{Note}
 


\begin{Def}
Para $x$ fijo y para cada $t\geq0$, sea $I_{x}\left(t\right)=1$ si $X\left(t\right)\leq x$,  $I_{x}\left(t\right)=0$ en caso contrario, y def\'inanse los tiempos promedio
\begin{eqnarray*}
\overline{X}&=&lim_{t\rightarrow\infty}\frac{1}{t}\int_{0}^{\infty}X\left(u\right)du\\
\prob\left(X_{\infty}\leq x\right)&=&lim_{t\rightarrow\infty}\frac{1}{t}\int_{0}^{\infty}I_{x}\left(u\right)du,
\end{eqnarray*}
cuando estos l\'imites existan.
\end{Def}

Como consecuencia del teorema de Renovaci\'on-Recompensa, se tiene que el primer l\'imite  existe y es igual a la constante
\begin{eqnarray*}
\overline{X}&=&\frac{\esp\left[\int_{0}^{R_{1}}X\left(t\right)dt\right]}{\esp\left[R_{1}\right]},
\end{eqnarray*}
suponiendo que ambas esperanzas son finitas.
 
\begin{Note}
Funciones de procesos regenerativos son regenerativas, es decir, si $X\left(t\right)$ es regenerativo y se define el proceso $Y\left(t\right)$ por $Y\left(t\right)=f\left(X\left(t\right)\right)$ para alguna funci\'on Borel medible $f\left(\cdot\right)$. Adem\'as $Y$ es regenerativo con los mismos tiempos de renovaci\'on que $X$. 

En general, los tiempos de renovaci\'on, $Z_{k}$ de un proceso regenerativo no requieren ser tiempos de paro con respecto a la evoluci\'on de $X\left(t\right)$.
\end{Note} 

\begin{Note}
Una funci\'on de un proceso de Markov, usualmente no ser\'a un proceso de Markov, sin embargo ser\'a regenerativo si el proceso de Markov lo es.
\end{Note}

 
\begin{Note}
Un proceso regenerativo con media de la longitud de ciclo finita es llamado positivo recurrente.
\end{Note}


\begin{Note}
\begin{itemize}
\item[a)] Si el proceso regenerativo $X$ es positivo recurrente y tiene trayectorias muestrales no negativas, entonces la ecuaci\'on anterior es v\'alida.
\item[b)] Si $X$ es positivo recurrente regenerativo, podemos construir una \'unica versi\'on estacionaria de este proceso, $X_{e}=\left\{X_{e}\left(t\right)\right\}$, donde $X_{e}$ es un proceso estoc\'astico regenerativo y estrictamente estacionario, con distribuci\'on marginal distribuida como $X_{\infty}$
\end{itemize}
\end{Note}


%__________________________________________________________________________________________
\subsection{Procesos Regenerativos Estacionarios - Stidham \cite{Stidham}}
%__________________________________________________________________________________________


Un proceso estoc\'astico a tiempo continuo $\left\{V\left(t\right),t\geq0\right\}$ es un proceso regenerativo si existe una sucesi\'on de variables aleatorias independientes e id\'enticamente distribuidas $\left\{X_{1},X_{2},\ldots\right\}$, sucesi\'on de renovaci\'on, tal que para cualquier conjunto de Borel $A$, 

\begin{eqnarray*}
\prob\left\{V\left(t\right)\in A|X_{1}+X_{2}+\cdots+X_{R\left(t\right)}=s,\left\{V\left(\tau\right),\tau<s\right\}\right\}=\prob\left\{V\left(t-s\right)\in A|X_{1}>t-s\right\},
\end{eqnarray*}
para todo $0\leq s\leq t$, donde $R\left(t\right)=\max\left\{X_{1}+X_{2}+\cdots+X_{j}\leq t\right\}=$n\'umero de renovaciones ({\emph{puntos de regeneraci\'on}}) que ocurren en $\left[0,t\right]$. El intervalo $\left[0,X_{1}\right)$ es llamado {\emph{primer ciclo de regeneraci\'on}} de $\left\{V\left(t \right),t\geq0\right\}$, $\left[X_{1},X_{1}+X_{2}\right)$ el {\emph{segundo ciclo de regeneraci\'on}}, y as\'i sucesivamente.

Sea $X=X_{1}$ y sea $F$ la funci\'on de distrbuci\'on de $X$


\begin{Def}
Se define el proceso estacionario, $\left\{V^{*}\left(t\right),t\geq0\right\}$, para $\left\{V\left(t\right),t\geq0\right\}$ por

\begin{eqnarray*}
\prob\left\{V\left(t\right)\in A\right\}=\frac{1}{\esp\left[X\right]}\int_{0}^{\infty}\prob\left\{V\left(t+x\right)\in A|X>x\right\}\left(1-F\left(x\right)\right)dx,
\end{eqnarray*} 
para todo $t\geq0$ y todo conjunto de Borel $A$.
\end{Def}

\begin{Def}
Una distribuci\'on se dice que es {\emph{aritm\'etica}} si todos sus puntos de incremento son m\'ultiplos de la forma $0,\lambda, 2\lambda,\ldots$ para alguna $\lambda>0$ entera.
\end{Def}


\begin{Def}
Una modificaci\'on medible de un proceso $\left\{V\left(t\right),t\geq0\right\}$, es una versi\'on de este, $\left\{V\left(t,w\right)\right\}$ conjuntamente medible para $t\geq0$ y para $w\in S$, $S$ espacio de estados para $\left\{V\left(t\right),t\geq0\right\}$.
\end{Def}

\begin{Teo}
Sea $\left\{V\left(t\right),t\geq\right\}$ un proceso regenerativo no negativo con modificaci\'on medible. Sea $\esp\left[X\right]<\infty$. Entonces el proceso estacionario dado por la ecuaci\'on anterior est\'a bien definido y tiene funci\'on de distribuci\'on independiente de $t$, adem\'as
\begin{itemize}
\item[i)] \begin{eqnarray*}
\esp\left[V^{*}\left(0\right)\right]&=&\frac{\esp\left[\int_{0}^{X}V\left(s\right)ds\right]}{\esp\left[X\right]}\end{eqnarray*}
\item[ii)] Si $\esp\left[V^{*}\left(0\right)\right]<\infty$, equivalentemente, si $\esp\left[\int_{0}^{X}V\left(s\right)ds\right]<\infty$,entonces
\begin{eqnarray*}
\frac{\int_{0}^{t}V\left(s\right)ds}{t}\rightarrow\frac{\esp\left[\int_{0}^{X}V\left(s\right)ds\right]}{\esp\left[X\right]}
\end{eqnarray*}
con probabilidad 1 y en media, cuando $t\rightarrow\infty$.
\end{itemize}
\end{Teo}

\begin{Coro}
Sea $\left\{V\left(t\right),t\geq0\right\}$ un proceso regenerativo no negativo, con modificaci\'on medible. Si $\esp <\infty$, $F$ es no-aritm\'etica, y para todo $x\geq0$, $P\left\{V\left(t\right)\leq x,C>x\right\}$ es de variaci\'on acotada como funci\'on de $t$ en cada intervalo finito $\left[0,\tau\right]$, entonces $V\left(t\right)$ converge en distribuci\'on  cuando $t\rightarrow\infty$ y $$\esp V=\frac{\esp \int_{0}^{X}V\left(s\right)ds}{\esp X}$$
Donde $V$ tiene la distribuci\'on l\'imite de $V\left(t\right)$ cuando $t\rightarrow\infty$.

\end{Coro}

Para el caso discreto se tienen resultados similares.



%______________________________________________________________________
\subsubsection{Procesos de Renovaci\'on}
%______________________________________________________________________

\begin{Def}%\label{Def.Tn}
Sean $0\leq T_{1}\leq T_{2}\leq \ldots$ son tiempos aleatorios infinitos en los cuales ocurren ciertos eventos. El n\'umero de tiempos $T_{n}$ en el intervalo $\left[0,t\right)$ es

\begin{eqnarray}
N\left(t\right)=\sum_{n=1}^{\infty}\indora\left(T_{n}\leq t\right),
\end{eqnarray}
para $t\geq0$.
\end{Def}

Si se consideran los puntos $T_{n}$ como elementos de $\rea_{+}$, y $N\left(t\right)$ es el n\'umero de puntos en $\rea$. El proceso denotado por $\left\{N\left(t\right):t\geq0\right\}$, denotado por $N\left(t\right)$, es un proceso puntual en $\rea_{+}$. Los $T_{n}$ son los tiempos de ocurrencia, el proceso puntual $N\left(t\right)$ es simple si su n\'umero de ocurrencias son distintas: $0<T_{1}<T_{2}<\ldots$ casi seguramente.

\begin{Def}
Un proceso puntual $N\left(t\right)$ es un proceso de renovaci\'on si los tiempos de interocurrencia $\xi_{n}=T_{n}-T_{n-1}$, para $n\geq1$, son independientes e identicamente distribuidos con distribuci\'on $F$, donde $F\left(0\right)=0$ y $T_{0}=0$. Los $T_{n}$ son llamados tiempos de renovaci\'on, referente a la independencia o renovaci\'on de la informaci\'on estoc\'astica en estos tiempos. Los $\xi_{n}$ son los tiempos de inter-renovaci\'on, y $N\left(t\right)$ es el n\'umero de renovaciones en el intervalo $\left[0,t\right)$
\end{Def}


\begin{Note}
Para definir un proceso de renovaci\'on para cualquier contexto, solamente hay que especificar una distribuci\'on $F$, con $F\left(0\right)=0$, para los tiempos de inter-renovaci\'on. La funci\'on $F$ en turno degune las otra variables aleatorias. De manera formal, existe un espacio de probabilidad y una sucesi\'on de variables aleatorias $\xi_{1},\xi_{2},\ldots$ definidas en este con distribuci\'on $F$. Entonces las otras cantidades son $T_{n}=\sum_{k=1}^{n}\xi_{k}$ y $N\left(t\right)=\sum_{n=1}^{\infty}\indora\left(T_{n}\leq t\right)$, donde $T_{n}\rightarrow\infty$ casi seguramente por la Ley Fuerte de los Grandes Números.
\end{Note}

%___________________________________________________________________________________________
%
\subsubsection{Teorema Principal de Renovaci\'on}
%___________________________________________________________________________________________
%

\begin{Note} Una funci\'on $h:\rea_{+}\rightarrow\rea$ es Directamente Riemann Integrable en los siguientes casos:
\begin{itemize}
\item[a)] $h\left(t\right)\geq0$ es decreciente y Riemann Integrable.
\item[b)] $h$ es continua excepto posiblemente en un conjunto de Lebesgue de medida 0, y $|h\left(t\right)|\leq b\left(t\right)$, donde $b$ es DRI.
\end{itemize}
\end{Note}

\begin{Teo}[Teorema Principal de Renovaci\'on]
Si $F$ es no aritm\'etica y $h\left(t\right)$ es Directamente Riemann Integrable (DRI), entonces

\begin{eqnarray*}
lim_{t\rightarrow\infty}U\star h=\frac{1}{\mu}\int_{\rea_{+}}h\left(s\right)ds.
\end{eqnarray*}
\end{Teo}

\begin{Prop}
Cualquier funci\'on $H\left(t\right)$ acotada en intervalos finitos y que es 0 para $t<0$ puede expresarse como
\begin{eqnarray*}
H\left(t\right)=U\star h\left(t\right)\textrm{,  donde }h\left(t\right)=H\left(t\right)-F\star H\left(t\right)
\end{eqnarray*}
\end{Prop}

\begin{Def}
Un proceso estoc\'astico $X\left(t\right)$ es crudamente regenerativo en un tiempo aleatorio positivo $T$ si
\begin{eqnarray*}
\esp\left[X\left(T+t\right)|T\right]=\esp\left[X\left(t\right)\right]\textrm{, para }t\geq0,\end{eqnarray*}
y con las esperanzas anteriores finitas.
\end{Def}

\begin{Prop}
Sup\'ongase que $X\left(t\right)$ es un proceso crudamente regenerativo en $T$, que tiene distribuci\'on $F$. Si $\esp\left[X\left(t\right)\right]$ es acotado en intervalos finitos, entonces
\begin{eqnarray*}
\esp\left[X\left(t\right)\right]=U\star h\left(t\right)\textrm{,  donde }h\left(t\right)=\esp\left[X\left(t\right)\indora\left(T>t\right)\right].
\end{eqnarray*}
\end{Prop}

\begin{Teo}[Regeneraci\'on Cruda]
Sup\'ongase que $X\left(t\right)$ es un proceso con valores positivo crudamente regenerativo en $T$, y def\'inase $M=\sup\left\{|X\left(t\right)|:t\leq T\right\}$. Si $T$ es no aritm\'etico y $M$ y $MT$ tienen media finita, entonces
\begin{eqnarray*}
lim_{t\rightarrow\infty}\esp\left[X\left(t\right)\right]=\frac{1}{\mu}\int_{\rea_{+}}h\left(s\right)ds,
\end{eqnarray*}
donde $h\left(t\right)=\esp\left[X\left(t\right)\indora\left(T>t\right)\right]$.
\end{Teo}

%___________________________________________________________________________________________
%
\subsubsection{Propiedades de los Procesos de Renovaci\'on}
%___________________________________________________________________________________________
%

Los tiempos $T_{n}$ est\'an relacionados con los conteos de $N\left(t\right)$ por

\begin{eqnarray*}
\left\{N\left(t\right)\geq n\right\}&=&\left\{T_{n}\leq t\right\}\\
T_{N\left(t\right)}\leq &t&<T_{N\left(t\right)+1},
\end{eqnarray*}

adem\'as $N\left(T_{n}\right)=n$, y 

\begin{eqnarray*}
N\left(t\right)=\max\left\{n:T_{n}\leq t\right\}=\min\left\{n:T_{n+1}>t\right\}
\end{eqnarray*}

Por propiedades de la convoluci\'on se sabe que

\begin{eqnarray*}
P\left\{T_{n}\leq t\right\}=F^{n\star}\left(t\right)
\end{eqnarray*}
que es la $n$-\'esima convoluci\'on de $F$. Entonces 

\begin{eqnarray*}
\left\{N\left(t\right)\geq n\right\}&=&\left\{T_{n}\leq t\right\}\\
P\left\{N\left(t\right)\leq n\right\}&=&1-F^{\left(n+1\right)\star}\left(t\right)
\end{eqnarray*}

Adem\'as usando el hecho de que $\esp\left[N\left(t\right)\right]=\sum_{n=1}^{\infty}P\left\{N\left(t\right)\geq n\right\}$
se tiene que

\begin{eqnarray*}
\esp\left[N\left(t\right)\right]=\sum_{n=1}^{\infty}F^{n\star}\left(t\right)
\end{eqnarray*}

\begin{Prop}
Para cada $t\geq0$, la funci\'on generadora de momentos $\esp\left[e^{\alpha N\left(t\right)}\right]$ existe para alguna $\alpha$ en una vecindad del 0, y de aqu\'i que $\esp\left[N\left(t\right)^{m}\right]<\infty$, para $m\geq1$.
\end{Prop}


\begin{Note}
Si el primer tiempo de renovaci\'on $\xi_{1}$ no tiene la misma distribuci\'on que el resto de las $\xi_{n}$, para $n\geq2$, a $N\left(t\right)$ se le llama Proceso de Renovaci\'on retardado, donde si $\xi$ tiene distribuci\'on $G$, entonces el tiempo $T_{n}$ de la $n$-\'esima renovaci\'on tiene distribuci\'on $G\star F^{\left(n-1\right)\star}\left(t\right)$
\end{Note}


\begin{Teo}
Para una constante $\mu\leq\infty$ ( o variable aleatoria), las siguientes expresiones son equivalentes:

\begin{eqnarray}
lim_{n\rightarrow\infty}n^{-1}T_{n}&=&\mu,\textrm{ c.s.}\\
lim_{t\rightarrow\infty}t^{-1}N\left(t\right)&=&1/\mu,\textrm{ c.s.}
\end{eqnarray}
\end{Teo}


Es decir, $T_{n}$ satisface la Ley Fuerte de los Grandes N\'umeros s\'i y s\'olo s\'i $N\left/t\right)$ la cumple.


\begin{Coro}[Ley Fuerte de los Grandes N\'umeros para Procesos de Renovaci\'on]
Si $N\left(t\right)$ es un proceso de renovaci\'on cuyos tiempos de inter-renovaci\'on tienen media $\mu\leq\infty$, entonces
\begin{eqnarray}
t^{-1}N\left(t\right)\rightarrow 1/\mu,\textrm{ c.s. cuando }t\rightarrow\infty.
\end{eqnarray}

\end{Coro}


Considerar el proceso estoc\'astico de valores reales $\left\{Z\left(t\right):t\geq0\right\}$ en el mismo espacio de probabilidad que $N\left(t\right)$

\begin{Def}
Para el proceso $\left\{Z\left(t\right):t\geq0\right\}$ se define la fluctuaci\'on m\'axima de $Z\left(t\right)$ en el intervalo $\left(T_{n-1},T_{n}\right]$:
\begin{eqnarray*}
M_{n}=\sup_{T_{n-1}<t\leq T_{n}}|Z\left(t\right)-Z\left(T_{n-1}\right)|
\end{eqnarray*}
\end{Def}

\begin{Teo}
Sup\'ongase que $n^{-1}T_{n}\rightarrow\mu$ c.s. cuando $n\rightarrow\infty$, donde $\mu\leq\infty$ es una constante o variable aleatoria. Sea $a$ una constante o variable aleatoria que puede ser infinita cuando $\mu$ es finita, y considere las expresiones l\'imite:
\begin{eqnarray}
lim_{n\rightarrow\infty}n^{-1}Z\left(T_{n}\right)&=&a,\textrm{ c.s.}\\
lim_{t\rightarrow\infty}t^{-1}Z\left(t\right)&=&a/\mu,\textrm{ c.s.}
\end{eqnarray}
La segunda expresi\'on implica la primera. Conversamente, la primera implica la segunda si el proceso $Z\left(t\right)$ es creciente, o si $lim_{n\rightarrow\infty}n^{-1}M_{n}=0$ c.s.
\end{Teo}

\begin{Coro}
Si $N\left(t\right)$ es un proceso de renovaci\'on, y $\left(Z\left(T_{n}\right)-Z\left(T_{n-1}\right),M_{n}\right)$, para $n\geq1$, son variables aleatorias independientes e id\'enticamente distribuidas con media finita, entonces,
\begin{eqnarray}
lim_{t\rightarrow\infty}t^{-1}Z\left(t\right)\rightarrow\frac{\esp\left[Z\left(T_{1}\right)-Z\left(T_{0}\right)\right]}{\esp\left[T_{1}\right]},\textrm{ c.s. cuando  }t\rightarrow\infty.
\end{eqnarray}
\end{Coro}



%___________________________________________________________________________________________
%
%\subsection{Propiedades de los Procesos de Renovaci\'on}
%___________________________________________________________________________________________
%

Los tiempos $T_{n}$ est\'an relacionados con los conteos de $N\left(t\right)$ por

\begin{eqnarray*}
\left\{N\left(t\right)\geq n\right\}&=&\left\{T_{n}\leq t\right\}\\
T_{N\left(t\right)}\leq &t&<T_{N\left(t\right)+1},
\end{eqnarray*}

adem\'as $N\left(T_{n}\right)=n$, y 

\begin{eqnarray*}
N\left(t\right)=\max\left\{n:T_{n}\leq t\right\}=\min\left\{n:T_{n+1}>t\right\}
\end{eqnarray*}

Por propiedades de la convoluci\'on se sabe que

\begin{eqnarray*}
P\left\{T_{n}\leq t\right\}=F^{n\star}\left(t\right)
\end{eqnarray*}
que es la $n$-\'esima convoluci\'on de $F$. Entonces 

\begin{eqnarray*}
\left\{N\left(t\right)\geq n\right\}&=&\left\{T_{n}\leq t\right\}\\
P\left\{N\left(t\right)\leq n\right\}&=&1-F^{\left(n+1\right)\star}\left(t\right)
\end{eqnarray*}

Adem\'as usando el hecho de que $\esp\left[N\left(t\right)\right]=\sum_{n=1}^{\infty}P\left\{N\left(t\right)\geq n\right\}$
se tiene que

\begin{eqnarray*}
\esp\left[N\left(t\right)\right]=\sum_{n=1}^{\infty}F^{n\star}\left(t\right)
\end{eqnarray*}

\begin{Prop}
Para cada $t\geq0$, la funci\'on generadora de momentos $\esp\left[e^{\alpha N\left(t\right)}\right]$ existe para alguna $\alpha$ en una vecindad del 0, y de aqu\'i que $\esp\left[N\left(t\right)^{m}\right]<\infty$, para $m\geq1$.
\end{Prop}


\begin{Note}
Si el primer tiempo de renovaci\'on $\xi_{1}$ no tiene la misma distribuci\'on que el resto de las $\xi_{n}$, para $n\geq2$, a $N\left(t\right)$ se le llama Proceso de Renovaci\'on retardado, donde si $\xi$ tiene distribuci\'on $G$, entonces el tiempo $T_{n}$ de la $n$-\'esima renovaci\'on tiene distribuci\'on $G\star F^{\left(n-1\right)\star}\left(t\right)$
\end{Note}


\begin{Teo}
Para una constante $\mu\leq\infty$ ( o variable aleatoria), las siguientes expresiones son equivalentes:

\begin{eqnarray}
lim_{n\rightarrow\infty}n^{-1}T_{n}&=&\mu,\textrm{ c.s.}\\
lim_{t\rightarrow\infty}t^{-1}N\left(t\right)&=&1/\mu,\textrm{ c.s.}
\end{eqnarray}
\end{Teo}


Es decir, $T_{n}$ satisface la Ley Fuerte de los Grandes N\'umeros s\'i y s\'olo s\'i $N\left/t\right)$ la cumple.


\begin{Coro}[Ley Fuerte de los Grandes N\'umeros para Procesos de Renovaci\'on]
Si $N\left(t\right)$ es un proceso de renovaci\'on cuyos tiempos de inter-renovaci\'on tienen media $\mu\leq\infty$, entonces
\begin{eqnarray}
t^{-1}N\left(t\right)\rightarrow 1/\mu,\textrm{ c.s. cuando }t\rightarrow\infty.
\end{eqnarray}

\end{Coro}


Considerar el proceso estoc\'astico de valores reales $\left\{Z\left(t\right):t\geq0\right\}$ en el mismo espacio de probabilidad que $N\left(t\right)$

\begin{Def}
Para el proceso $\left\{Z\left(t\right):t\geq0\right\}$ se define la fluctuaci\'on m\'axima de $Z\left(t\right)$ en el intervalo $\left(T_{n-1},T_{n}\right]$:
\begin{eqnarray*}
M_{n}=\sup_{T_{n-1}<t\leq T_{n}}|Z\left(t\right)-Z\left(T_{n-1}\right)|
\end{eqnarray*}
\end{Def}

\begin{Teo}
Sup\'ongase que $n^{-1}T_{n}\rightarrow\mu$ c.s. cuando $n\rightarrow\infty$, donde $\mu\leq\infty$ es una constante o variable aleatoria. Sea $a$ una constante o variable aleatoria que puede ser infinita cuando $\mu$ es finita, y considere las expresiones l\'imite:
\begin{eqnarray}
lim_{n\rightarrow\infty}n^{-1}Z\left(T_{n}\right)&=&a,\textrm{ c.s.}\\
lim_{t\rightarrow\infty}t^{-1}Z\left(t\right)&=&a/\mu,\textrm{ c.s.}
\end{eqnarray}
La segunda expresi\'on implica la primera. Conversamente, la primera implica la segunda si el proceso $Z\left(t\right)$ es creciente, o si $lim_{n\rightarrow\infty}n^{-1}M_{n}=0$ c.s.
\end{Teo}

\begin{Coro}
Si $N\left(t\right)$ es un proceso de renovaci\'on, y $\left(Z\left(T_{n}\right)-Z\left(T_{n-1}\right),M_{n}\right)$, para $n\geq1$, son variables aleatorias independientes e id\'enticamente distribuidas con media finita, entonces,
\begin{eqnarray}
lim_{t\rightarrow\infty}t^{-1}Z\left(t\right)\rightarrow\frac{\esp\left[Z\left(T_{1}\right)-Z\left(T_{0}\right)\right]}{\esp\left[T_{1}\right]},\textrm{ c.s. cuando  }t\rightarrow\infty.
\end{eqnarray}
\end{Coro}


%___________________________________________________________________________________________
%
%\subsection{Propiedades de los Procesos de Renovaci\'on}
%___________________________________________________________________________________________
%

Los tiempos $T_{n}$ est\'an relacionados con los conteos de $N\left(t\right)$ por

\begin{eqnarray*}
\left\{N\left(t\right)\geq n\right\}&=&\left\{T_{n}\leq t\right\}\\
T_{N\left(t\right)}\leq &t&<T_{N\left(t\right)+1},
\end{eqnarray*}

adem\'as $N\left(T_{n}\right)=n$, y 

\begin{eqnarray*}
N\left(t\right)=\max\left\{n:T_{n}\leq t\right\}=\min\left\{n:T_{n+1}>t\right\}
\end{eqnarray*}

Por propiedades de la convoluci\'on se sabe que

\begin{eqnarray*}
P\left\{T_{n}\leq t\right\}=F^{n\star}\left(t\right)
\end{eqnarray*}
que es la $n$-\'esima convoluci\'on de $F$. Entonces 

\begin{eqnarray*}
\left\{N\left(t\right)\geq n\right\}&=&\left\{T_{n}\leq t\right\}\\
P\left\{N\left(t\right)\leq n\right\}&=&1-F^{\left(n+1\right)\star}\left(t\right)
\end{eqnarray*}

Adem\'as usando el hecho de que $\esp\left[N\left(t\right)\right]=\sum_{n=1}^{\infty}P\left\{N\left(t\right)\geq n\right\}$
se tiene que

\begin{eqnarray*}
\esp\left[N\left(t\right)\right]=\sum_{n=1}^{\infty}F^{n\star}\left(t\right)
\end{eqnarray*}

\begin{Prop}
Para cada $t\geq0$, la funci\'on generadora de momentos $\esp\left[e^{\alpha N\left(t\right)}\right]$ existe para alguna $\alpha$ en una vecindad del 0, y de aqu\'i que $\esp\left[N\left(t\right)^{m}\right]<\infty$, para $m\geq1$.
\end{Prop}


\begin{Note}
Si el primer tiempo de renovaci\'on $\xi_{1}$ no tiene la misma distribuci\'on que el resto de las $\xi_{n}$, para $n\geq2$, a $N\left(t\right)$ se le llama Proceso de Renovaci\'on retardado, donde si $\xi$ tiene distribuci\'on $G$, entonces el tiempo $T_{n}$ de la $n$-\'esima renovaci\'on tiene distribuci\'on $G\star F^{\left(n-1\right)\star}\left(t\right)$
\end{Note}


\begin{Teo}
Para una constante $\mu\leq\infty$ ( o variable aleatoria), las siguientes expresiones son equivalentes:

\begin{eqnarray}
lim_{n\rightarrow\infty}n^{-1}T_{n}&=&\mu,\textrm{ c.s.}\\
lim_{t\rightarrow\infty}t^{-1}N\left(t\right)&=&1/\mu,\textrm{ c.s.}
\end{eqnarray}
\end{Teo}


Es decir, $T_{n}$ satisface la Ley Fuerte de los Grandes N\'umeros s\'i y s\'olo s\'i $N\left/t\right)$ la cumple.


\begin{Coro}[Ley Fuerte de los Grandes N\'umeros para Procesos de Renovaci\'on]
Si $N\left(t\right)$ es un proceso de renovaci\'on cuyos tiempos de inter-renovaci\'on tienen media $\mu\leq\infty$, entonces
\begin{eqnarray}
t^{-1}N\left(t\right)\rightarrow 1/\mu,\textrm{ c.s. cuando }t\rightarrow\infty.
\end{eqnarray}

\end{Coro}


Considerar el proceso estoc\'astico de valores reales $\left\{Z\left(t\right):t\geq0\right\}$ en el mismo espacio de probabilidad que $N\left(t\right)$

\begin{Def}
Para el proceso $\left\{Z\left(t\right):t\geq0\right\}$ se define la fluctuaci\'on m\'axima de $Z\left(t\right)$ en el intervalo $\left(T_{n-1},T_{n}\right]$:
\begin{eqnarray*}
M_{n}=\sup_{T_{n-1}<t\leq T_{n}}|Z\left(t\right)-Z\left(T_{n-1}\right)|
\end{eqnarray*}
\end{Def}

\begin{Teo}
Sup\'ongase que $n^{-1}T_{n}\rightarrow\mu$ c.s. cuando $n\rightarrow\infty$, donde $\mu\leq\infty$ es una constante o variable aleatoria. Sea $a$ una constante o variable aleatoria que puede ser infinita cuando $\mu$ es finita, y considere las expresiones l\'imite:
\begin{eqnarray}
lim_{n\rightarrow\infty}n^{-1}Z\left(T_{n}\right)&=&a,\textrm{ c.s.}\\
lim_{t\rightarrow\infty}t^{-1}Z\left(t\right)&=&a/\mu,\textrm{ c.s.}
\end{eqnarray}
La segunda expresi\'on implica la primera. Conversamente, la primera implica la segunda si el proceso $Z\left(t\right)$ es creciente, o si $lim_{n\rightarrow\infty}n^{-1}M_{n}=0$ c.s.
\end{Teo}

\begin{Coro}
Si $N\left(t\right)$ es un proceso de renovaci\'on, y $\left(Z\left(T_{n}\right)-Z\left(T_{n-1}\right),M_{n}\right)$, para $n\geq1$, son variables aleatorias independientes e id\'enticamente distribuidas con media finita, entonces,
\begin{eqnarray}
lim_{t\rightarrow\infty}t^{-1}Z\left(t\right)\rightarrow\frac{\esp\left[Z\left(T_{1}\right)-Z\left(T_{0}\right)\right]}{\esp\left[T_{1}\right]},\textrm{ c.s. cuando  }t\rightarrow\infty.
\end{eqnarray}
\end{Coro}

%___________________________________________________________________________________________
%
%\subsection{Propiedades de los Procesos de Renovaci\'on}
%___________________________________________________________________________________________
%

Los tiempos $T_{n}$ est\'an relacionados con los conteos de $N\left(t\right)$ por

\begin{eqnarray*}
\left\{N\left(t\right)\geq n\right\}&=&\left\{T_{n}\leq t\right\}\\
T_{N\left(t\right)}\leq &t&<T_{N\left(t\right)+1},
\end{eqnarray*}

adem\'as $N\left(T_{n}\right)=n$, y 

\begin{eqnarray*}
N\left(t\right)=\max\left\{n:T_{n}\leq t\right\}=\min\left\{n:T_{n+1}>t\right\}
\end{eqnarray*}

Por propiedades de la convoluci\'on se sabe que

\begin{eqnarray*}
P\left\{T_{n}\leq t\right\}=F^{n\star}\left(t\right)
\end{eqnarray*}
que es la $n$-\'esima convoluci\'on de $F$. Entonces 

\begin{eqnarray*}
\left\{N\left(t\right)\geq n\right\}&=&\left\{T_{n}\leq t\right\}\\
P\left\{N\left(t\right)\leq n\right\}&=&1-F^{\left(n+1\right)\star}\left(t\right)
\end{eqnarray*}

Adem\'as usando el hecho de que $\esp\left[N\left(t\right)\right]=\sum_{n=1}^{\infty}P\left\{N\left(t\right)\geq n\right\}$
se tiene que

\begin{eqnarray*}
\esp\left[N\left(t\right)\right]=\sum_{n=1}^{\infty}F^{n\star}\left(t\right)
\end{eqnarray*}

\begin{Prop}
Para cada $t\geq0$, la funci\'on generadora de momentos $\esp\left[e^{\alpha N\left(t\right)}\right]$ existe para alguna $\alpha$ en una vecindad del 0, y de aqu\'i que $\esp\left[N\left(t\right)^{m}\right]<\infty$, para $m\geq1$.
\end{Prop}


\begin{Note}
Si el primer tiempo de renovaci\'on $\xi_{1}$ no tiene la misma distribuci\'on que el resto de las $\xi_{n}$, para $n\geq2$, a $N\left(t\right)$ se le llama Proceso de Renovaci\'on retardado, donde si $\xi$ tiene distribuci\'on $G$, entonces el tiempo $T_{n}$ de la $n$-\'esima renovaci\'on tiene distribuci\'on $G\star F^{\left(n-1\right)\star}\left(t\right)$
\end{Note}


\begin{Teo}
Para una constante $\mu\leq\infty$ ( o variable aleatoria), las siguientes expresiones son equivalentes:

\begin{eqnarray}
lim_{n\rightarrow\infty}n^{-1}T_{n}&=&\mu,\textrm{ c.s.}\\
lim_{t\rightarrow\infty}t^{-1}N\left(t\right)&=&1/\mu,\textrm{ c.s.}
\end{eqnarray}
\end{Teo}


Es decir, $T_{n}$ satisface la Ley Fuerte de los Grandes N\'umeros s\'i y s\'olo s\'i $N\left/t\right)$ la cumple.


\begin{Coro}[Ley Fuerte de los Grandes N\'umeros para Procesos de Renovaci\'on]
Si $N\left(t\right)$ es un proceso de renovaci\'on cuyos tiempos de inter-renovaci\'on tienen media $\mu\leq\infty$, entonces
\begin{eqnarray}
t^{-1}N\left(t\right)\rightarrow 1/\mu,\textrm{ c.s. cuando }t\rightarrow\infty.
\end{eqnarray}

\end{Coro}


Considerar el proceso estoc\'astico de valores reales $\left\{Z\left(t\right):t\geq0\right\}$ en el mismo espacio de probabilidad que $N\left(t\right)$

\begin{Def}
Para el proceso $\left\{Z\left(t\right):t\geq0\right\}$ se define la fluctuaci\'on m\'axima de $Z\left(t\right)$ en el intervalo $\left(T_{n-1},T_{n}\right]$:
\begin{eqnarray*}
M_{n}=\sup_{T_{n-1}<t\leq T_{n}}|Z\left(t\right)-Z\left(T_{n-1}\right)|
\end{eqnarray*}
\end{Def}

\begin{Teo}
Sup\'ongase que $n^{-1}T_{n}\rightarrow\mu$ c.s. cuando $n\rightarrow\infty$, donde $\mu\leq\infty$ es una constante o variable aleatoria. Sea $a$ una constante o variable aleatoria que puede ser infinita cuando $\mu$ es finita, y considere las expresiones l\'imite:
\begin{eqnarray}
lim_{n\rightarrow\infty}n^{-1}Z\left(T_{n}\right)&=&a,\textrm{ c.s.}\\
lim_{t\rightarrow\infty}t^{-1}Z\left(t\right)&=&a/\mu,\textrm{ c.s.}
\end{eqnarray}
La segunda expresi\'on implica la primera. Conversamente, la primera implica la segunda si el proceso $Z\left(t\right)$ es creciente, o si $lim_{n\rightarrow\infty}n^{-1}M_{n}=0$ c.s.
\end{Teo}

\begin{Coro}
Si $N\left(t\right)$ es un proceso de renovaci\'on, y $\left(Z\left(T_{n}\right)-Z\left(T_{n-1}\right),M_{n}\right)$, para $n\geq1$, son variables aleatorias independientes e id\'enticamente distribuidas con media finita, entonces,
\begin{eqnarray}
lim_{t\rightarrow\infty}t^{-1}Z\left(t\right)\rightarrow\frac{\esp\left[Z\left(T_{1}\right)-Z\left(T_{0}\right)\right]}{\esp\left[T_{1}\right]},\textrm{ c.s. cuando  }t\rightarrow\infty.
\end{eqnarray}
\end{Coro}
%___________________________________________________________________________________________
%
%\subsection{Propiedades de los Procesos de Renovaci\'on}
%___________________________________________________________________________________________
%

Los tiempos $T_{n}$ est\'an relacionados con los conteos de $N\left(t\right)$ por

\begin{eqnarray*}
\left\{N\left(t\right)\geq n\right\}&=&\left\{T_{n}\leq t\right\}\\
T_{N\left(t\right)}\leq &t&<T_{N\left(t\right)+1},
\end{eqnarray*}

adem\'as $N\left(T_{n}\right)=n$, y 

\begin{eqnarray*}
N\left(t\right)=\max\left\{n:T_{n}\leq t\right\}=\min\left\{n:T_{n+1}>t\right\}
\end{eqnarray*}

Por propiedades de la convoluci\'on se sabe que

\begin{eqnarray*}
P\left\{T_{n}\leq t\right\}=F^{n\star}\left(t\right)
\end{eqnarray*}
que es la $n$-\'esima convoluci\'on de $F$. Entonces 

\begin{eqnarray*}
\left\{N\left(t\right)\geq n\right\}&=&\left\{T_{n}\leq t\right\}\\
P\left\{N\left(t\right)\leq n\right\}&=&1-F^{\left(n+1\right)\star}\left(t\right)
\end{eqnarray*}

Adem\'as usando el hecho de que $\esp\left[N\left(t\right)\right]=\sum_{n=1}^{\infty}P\left\{N\left(t\right)\geq n\right\}$
se tiene que

\begin{eqnarray*}
\esp\left[N\left(t\right)\right]=\sum_{n=1}^{\infty}F^{n\star}\left(t\right)
\end{eqnarray*}

\begin{Prop}
Para cada $t\geq0$, la funci\'on generadora de momentos $\esp\left[e^{\alpha N\left(t\right)}\right]$ existe para alguna $\alpha$ en una vecindad del 0, y de aqu\'i que $\esp\left[N\left(t\right)^{m}\right]<\infty$, para $m\geq1$.
\end{Prop}


\begin{Note}
Si el primer tiempo de renovaci\'on $\xi_{1}$ no tiene la misma distribuci\'on que el resto de las $\xi_{n}$, para $n\geq2$, a $N\left(t\right)$ se le llama Proceso de Renovaci\'on retardado, donde si $\xi$ tiene distribuci\'on $G$, entonces el tiempo $T_{n}$ de la $n$-\'esima renovaci\'on tiene distribuci\'on $G\star F^{\left(n-1\right)\star}\left(t\right)$
\end{Note}


\begin{Teo}
Para una constante $\mu\leq\infty$ ( o variable aleatoria), las siguientes expresiones son equivalentes:

\begin{eqnarray}
lim_{n\rightarrow\infty}n^{-1}T_{n}&=&\mu,\textrm{ c.s.}\\
lim_{t\rightarrow\infty}t^{-1}N\left(t\right)&=&1/\mu,\textrm{ c.s.}
\end{eqnarray}
\end{Teo}


Es decir, $T_{n}$ satisface la Ley Fuerte de los Grandes N\'umeros s\'i y s\'olo s\'i $N\left/t\right)$ la cumple.


\begin{Coro}[Ley Fuerte de los Grandes N\'umeros para Procesos de Renovaci\'on]
Si $N\left(t\right)$ es un proceso de renovaci\'on cuyos tiempos de inter-renovaci\'on tienen media $\mu\leq\infty$, entonces
\begin{eqnarray}
t^{-1}N\left(t\right)\rightarrow 1/\mu,\textrm{ c.s. cuando }t\rightarrow\infty.
\end{eqnarray}

\end{Coro}


Considerar el proceso estoc\'astico de valores reales $\left\{Z\left(t\right):t\geq0\right\}$ en el mismo espacio de probabilidad que $N\left(t\right)$

\begin{Def}
Para el proceso $\left\{Z\left(t\right):t\geq0\right\}$ se define la fluctuaci\'on m\'axima de $Z\left(t\right)$ en el intervalo $\left(T_{n-1},T_{n}\right]$:
\begin{eqnarray*}
M_{n}=\sup_{T_{n-1}<t\leq T_{n}}|Z\left(t\right)-Z\left(T_{n-1}\right)|
\end{eqnarray*}
\end{Def}

\begin{Teo}
Sup\'ongase que $n^{-1}T_{n}\rightarrow\mu$ c.s. cuando $n\rightarrow\infty$, donde $\mu\leq\infty$ es una constante o variable aleatoria. Sea $a$ una constante o variable aleatoria que puede ser infinita cuando $\mu$ es finita, y considere las expresiones l\'imite:
\begin{eqnarray}
lim_{n\rightarrow\infty}n^{-1}Z\left(T_{n}\right)&=&a,\textrm{ c.s.}\\
lim_{t\rightarrow\infty}t^{-1}Z\left(t\right)&=&a/\mu,\textrm{ c.s.}
\end{eqnarray}
La segunda expresi\'on implica la primera. Conversamente, la primera implica la segunda si el proceso $Z\left(t\right)$ es creciente, o si $lim_{n\rightarrow\infty}n^{-1}M_{n}=0$ c.s.
\end{Teo}

\begin{Coro}
Si $N\left(t\right)$ es un proceso de renovaci\'on, y $\left(Z\left(T_{n}\right)-Z\left(T_{n-1}\right),M_{n}\right)$, para $n\geq1$, son variables aleatorias independientes e id\'enticamente distribuidas con media finita, entonces,
\begin{eqnarray}
lim_{t\rightarrow\infty}t^{-1}Z\left(t\right)\rightarrow\frac{\esp\left[Z\left(T_{1}\right)-Z\left(T_{0}\right)\right]}{\esp\left[T_{1}\right]},\textrm{ c.s. cuando  }t\rightarrow\infty.
\end{eqnarray}
\end{Coro}


%___________________________________________________________________________________________
%
\subsubsection{Funci\'on de Renovaci\'on}
%___________________________________________________________________________________________
%


\begin{Def}
Sea $h\left(t\right)$ funci\'on de valores reales en $\rea$ acotada en intervalos finitos e igual a cero para $t<0$ La ecuaci\'on de renovaci\'on para $h\left(t\right)$ y la distribuci\'on $F$ es

\begin{eqnarray}%\label{Ec.Renovacion}
H\left(t\right)=h\left(t\right)+\int_{\left[0,t\right]}H\left(t-s\right)dF\left(s\right)\textrm{,    }t\geq0,
\end{eqnarray}
donde $H\left(t\right)$ es una funci\'on de valores reales. Esto es $H=h+F\star H$. Decimos que $H\left(t\right)$ es soluci\'on de esta ecuaci\'on si satisface la ecuaci\'on, y es acotada en intervalos finitos e iguales a cero para $t<0$.
\end{Def}

\begin{Prop}
La funci\'on $U\star h\left(t\right)$ es la \'unica soluci\'on de la ecuaci\'on de renovaci\'on (\ref{Ec.Renovacion}).
\end{Prop}

\begin{Teo}[Teorema Renovaci\'on Elemental]
\begin{eqnarray*}
t^{-1}U\left(t\right)\rightarrow 1/\mu\textrm{,    cuando }t\rightarrow\infty.
\end{eqnarray*}
\end{Teo}

%___________________________________________________________________________________________
%
%\subsection{Funci\'on de Renovaci\'on}
%___________________________________________________________________________________________
%


Sup\'ongase que $N\left(t\right)$ es un proceso de renovaci\'on con distribuci\'on $F$ con media finita $\mu$.

\begin{Def}
La funci\'on de renovaci\'on asociada con la distribuci\'on $F$, del proceso $N\left(t\right)$, es
\begin{eqnarray*}
U\left(t\right)=\sum_{n=1}^{\infty}F^{n\star}\left(t\right),\textrm{   }t\geq0,
\end{eqnarray*}
donde $F^{0\star}\left(t\right)=\indora\left(t\geq0\right)$.
\end{Def}


\begin{Prop}
Sup\'ongase que la distribuci\'on de inter-renovaci\'on $F$ tiene densidad $f$. Entonces $U\left(t\right)$ tambi\'en tiene densidad, para $t>0$, y es $U^{'}\left(t\right)=\sum_{n=0}^{\infty}f^{n\star}\left(t\right)$. Adem\'as
\begin{eqnarray*}
\prob\left\{N\left(t\right)>N\left(t-\right)\right\}=0\textrm{,   }t\geq0.
\end{eqnarray*}
\end{Prop}

\begin{Def}
La Transformada de Laplace-Stieljes de $F$ est\'a dada por

\begin{eqnarray*}
\hat{F}\left(\alpha\right)=\int_{\rea_{+}}e^{-\alpha t}dF\left(t\right)\textrm{,  }\alpha\geq0.
\end{eqnarray*}
\end{Def}

Entonces

\begin{eqnarray*}
\hat{U}\left(\alpha\right)=\sum_{n=0}^{\infty}\hat{F^{n\star}}\left(\alpha\right)=\sum_{n=0}^{\infty}\hat{F}\left(\alpha\right)^{n}=\frac{1}{1-\hat{F}\left(\alpha\right)}.
\end{eqnarray*}


\begin{Prop}
La Transformada de Laplace $\hat{U}\left(\alpha\right)$ y $\hat{F}\left(\alpha\right)$ determina una a la otra de manera \'unica por la relaci\'on $\hat{U}\left(\alpha\right)=\frac{1}{1-\hat{F}\left(\alpha\right)}$.
\end{Prop}


\begin{Note}
Un proceso de renovaci\'on $N\left(t\right)$ cuyos tiempos de inter-renovaci\'on tienen media finita, es un proceso Poisson con tasa $\lambda$ si y s\'olo s\'i $\esp\left[U\left(t\right)\right]=\lambda t$, para $t\geq0$.
\end{Note}


\begin{Teo}
Sea $N\left(t\right)$ un proceso puntual simple con puntos de localizaci\'on $T_{n}$ tal que $\eta\left(t\right)=\esp\left[N\left(\right)\right]$ es finita para cada $t$. Entonces para cualquier funci\'on $f:\rea_{+}\rightarrow\rea$,
\begin{eqnarray*}
\esp\left[\sum_{n=1}^{N\left(\right)}f\left(T_{n}\right)\right]=\int_{\left(0,t\right]}f\left(s\right)d\eta\left(s\right)\textrm{,  }t\geq0,
\end{eqnarray*}
suponiendo que la integral exista. Adem\'as si $X_{1},X_{2},\ldots$ son variables aleatorias definidas en el mismo espacio de probabilidad que el proceso $N\left(t\right)$ tal que $\esp\left[X_{n}|T_{n}=s\right]=f\left(s\right)$, independiente de $n$. Entonces
\begin{eqnarray*}
\esp\left[\sum_{n=1}^{N\left(t\right)}X_{n}\right]=\int_{\left(0,t\right]}f\left(s\right)d\eta\left(s\right)\textrm{,  }t\geq0,
\end{eqnarray*} 
suponiendo que la integral exista. 
\end{Teo}

\begin{Coro}[Identidad de Wald para Renovaciones]
Para el proceso de renovaci\'on $N\left(t\right)$,
\begin{eqnarray*}
\esp\left[T_{N\left(t\right)+1}\right]=\mu\esp\left[N\left(t\right)+1\right]\textrm{,  }t\geq0,
\end{eqnarray*}  
\end{Coro}

%______________________________________________________________________
%\subsection{Procesos de Renovaci\'on}
%______________________________________________________________________

\begin{Def}%\label{Def.Tn}
Sean $0\leq T_{1}\leq T_{2}\leq \ldots$ son tiempos aleatorios infinitos en los cuales ocurren ciertos eventos. El n\'umero de tiempos $T_{n}$ en el intervalo $\left[0,t\right)$ es

\begin{eqnarray}
N\left(t\right)=\sum_{n=1}^{\infty}\indora\left(T_{n}\leq t\right),
\end{eqnarray}
para $t\geq0$.
\end{Def}

Si se consideran los puntos $T_{n}$ como elementos de $\rea_{+}$, y $N\left(t\right)$ es el n\'umero de puntos en $\rea$. El proceso denotado por $\left\{N\left(t\right):t\geq0\right\}$, denotado por $N\left(t\right)$, es un proceso puntual en $\rea_{+}$. Los $T_{n}$ son los tiempos de ocurrencia, el proceso puntual $N\left(t\right)$ es simple si su n\'umero de ocurrencias son distintas: $0<T_{1}<T_{2}<\ldots$ casi seguramente.

\begin{Def}
Un proceso puntual $N\left(t\right)$ es un proceso de renovaci\'on si los tiempos de interocurrencia $\xi_{n}=T_{n}-T_{n-1}$, para $n\geq1$, son independientes e identicamente distribuidos con distribuci\'on $F$, donde $F\left(0\right)=0$ y $T_{0}=0$. Los $T_{n}$ son llamados tiempos de renovaci\'on, referente a la independencia o renovaci\'on de la informaci\'on estoc\'astica en estos tiempos. Los $\xi_{n}$ son los tiempos de inter-renovaci\'on, y $N\left(t\right)$ es el n\'umero de renovaciones en el intervalo $\left[0,t\right)$
\end{Def}


\begin{Note}
Para definir un proceso de renovaci\'on para cualquier contexto, solamente hay que especificar una distribuci\'on $F$, con $F\left(0\right)=0$, para los tiempos de inter-renovaci\'on. La funci\'on $F$ en turno degune las otra variables aleatorias. De manera formal, existe un espacio de probabilidad y una sucesi\'on de variables aleatorias $\xi_{1},\xi_{2},\ldots$ definidas en este con distribuci\'on $F$. Entonces las otras cantidades son $T_{n}=\sum_{k=1}^{n}\xi_{k}$ y $N\left(t\right)=\sum_{n=1}^{\infty}\indora\left(T_{n}\leq t\right)$, donde $T_{n}\rightarrow\infty$ casi seguramente por la Ley Fuerte de los Grandes Números.
\end{Note}

%___________________________________________________________________________________________
%
\subsubsection{Renewal and Regenerative Processes: Serfozo\cite{Serfozo}}
%___________________________________________________________________________________________
%
\begin{Def}%\label{Def.Tn}
Sean $0\leq T_{1}\leq T_{2}\leq \ldots$ son tiempos aleatorios infinitos en los cuales ocurren ciertos eventos. El n\'umero de tiempos $T_{n}$ en el intervalo $\left[0,t\right)$ es

\begin{eqnarray}
N\left(t\right)=\sum_{n=1}^{\infty}\indora\left(T_{n}\leq t\right),
\end{eqnarray}
para $t\geq0$.
\end{Def}

Si se consideran los puntos $T_{n}$ como elementos de $\rea_{+}$, y $N\left(t\right)$ es el n\'umero de puntos en $\rea$. El proceso denotado por $\left\{N\left(t\right):t\geq0\right\}$, denotado por $N\left(t\right)$, es un proceso puntual en $\rea_{+}$. Los $T_{n}$ son los tiempos de ocurrencia, el proceso puntual $N\left(t\right)$ es simple si su n\'umero de ocurrencias son distintas: $0<T_{1}<T_{2}<\ldots$ casi seguramente.

\begin{Def}
Un proceso puntual $N\left(t\right)$ es un proceso de renovaci\'on si los tiempos de interocurrencia $\xi_{n}=T_{n}-T_{n-1}$, para $n\geq1$, son independientes e identicamente distribuidos con distribuci\'on $F$, donde $F\left(0\right)=0$ y $T_{0}=0$. Los $T_{n}$ son llamados tiempos de renovaci\'on, referente a la independencia o renovaci\'on de la informaci\'on estoc\'astica en estos tiempos. Los $\xi_{n}$ son los tiempos de inter-renovaci\'on, y $N\left(t\right)$ es el n\'umero de renovaciones en el intervalo $\left[0,t\right)$
\end{Def}


\begin{Note}
Para definir un proceso de renovaci\'on para cualquier contexto, solamente hay que especificar una distribuci\'on $F$, con $F\left(0\right)=0$, para los tiempos de inter-renovaci\'on. La funci\'on $F$ en turno degune las otra variables aleatorias. De manera formal, existe un espacio de probabilidad y una sucesi\'on de variables aleatorias $\xi_{1},\xi_{2},\ldots$ definidas en este con distribuci\'on $F$. Entonces las otras cantidades son $T_{n}=\sum_{k=1}^{n}\xi_{k}$ y $N\left(t\right)=\sum_{n=1}^{\infty}\indora\left(T_{n}\leq t\right)$, donde $T_{n}\rightarrow\infty$ casi seguramente por la Ley Fuerte de los Grandes N\'umeros.
\end{Note}

Los tiempos $T_{n}$ est\'an relacionados con los conteos de $N\left(t\right)$ por

\begin{eqnarray*}
\left\{N\left(t\right)\geq n\right\}&=&\left\{T_{n}\leq t\right\}\\
T_{N\left(t\right)}\leq &t&<T_{N\left(t\right)+1},
\end{eqnarray*}

adem\'as $N\left(T_{n}\right)=n$, y 

\begin{eqnarray*}
N\left(t\right)=\max\left\{n:T_{n}\leq t\right\}=\min\left\{n:T_{n+1}>t\right\}
\end{eqnarray*}

Por propiedades de la convoluci\'on se sabe que

\begin{eqnarray*}
P\left\{T_{n}\leq t\right\}=F^{n\star}\left(t\right)
\end{eqnarray*}
que es la $n$-\'esima convoluci\'on de $F$. Entonces 

\begin{eqnarray*}
\left\{N\left(t\right)\geq n\right\}&=&\left\{T_{n}\leq t\right\}\\
P\left\{N\left(t\right)\leq n\right\}&=&1-F^{\left(n+1\right)\star}\left(t\right)
\end{eqnarray*}

Adem\'as usando el hecho de que $\esp\left[N\left(t\right)\right]=\sum_{n=1}^{\infty}P\left\{N\left(t\right)\geq n\right\}$
se tiene que

\begin{eqnarray*}
\esp\left[N\left(t\right)\right]=\sum_{n=1}^{\infty}F^{n\star}\left(t\right)
\end{eqnarray*}

\begin{Prop}
Para cada $t\geq0$, la funci\'on generadora de momentos $\esp\left[e^{\alpha N\left(t\right)}\right]$ existe para alguna $\alpha$ en una vecindad del 0, y de aqu\'i que $\esp\left[N\left(t\right)^{m}\right]<\infty$, para $m\geq1$.
\end{Prop}

\begin{Ejem}[\textbf{Proceso Poisson}]

Suponga que se tienen tiempos de inter-renovaci\'on \textit{i.i.d.} del proceso de renovaci\'on $N\left(t\right)$ tienen distribuci\'on exponencial $F\left(t\right)=q-e^{-\lambda t}$ con tasa $\lambda$. Entonces $N\left(t\right)$ es un proceso Poisson con tasa $\lambda$.

\end{Ejem}


\begin{Note}
Si el primer tiempo de renovaci\'on $\xi_{1}$ no tiene la misma distribuci\'on que el resto de las $\xi_{n}$, para $n\geq2$, a $N\left(t\right)$ se le llama Proceso de Renovaci\'on retardado, donde si $\xi$ tiene distribuci\'on $G$, entonces el tiempo $T_{n}$ de la $n$-\'esima renovaci\'on tiene distribuci\'on $G\star F^{\left(n-1\right)\star}\left(t\right)$
\end{Note}


\begin{Teo}
Para una constante $\mu\leq\infty$ ( o variable aleatoria), las siguientes expresiones son equivalentes:

\begin{eqnarray}
lim_{n\rightarrow\infty}n^{-1}T_{n}&=&\mu,\textrm{ c.s.}\\
lim_{t\rightarrow\infty}t^{-1}N\left(t\right)&=&1/\mu,\textrm{ c.s.}
\end{eqnarray}
\end{Teo}


Es decir, $T_{n}$ satisface la Ley Fuerte de los Grandes N\'umeros s\'i y s\'olo s\'i $N\left/t\right)$ la cumple.


\begin{Coro}[Ley Fuerte de los Grandes N\'umeros para Procesos de Renovaci\'on]
Si $N\left(t\right)$ es un proceso de renovaci\'on cuyos tiempos de inter-renovaci\'on tienen media $\mu\leq\infty$, entonces
\begin{eqnarray}
t^{-1}N\left(t\right)\rightarrow 1/\mu,\textrm{ c.s. cuando }t\rightarrow\infty.
\end{eqnarray}

\end{Coro}


Considerar el proceso estoc\'astico de valores reales $\left\{Z\left(t\right):t\geq0\right\}$ en el mismo espacio de probabilidad que $N\left(t\right)$

\begin{Def}
Para el proceso $\left\{Z\left(t\right):t\geq0\right\}$ se define la fluctuaci\'on m\'axima de $Z\left(t\right)$ en el intervalo $\left(T_{n-1},T_{n}\right]$:
\begin{eqnarray*}
M_{n}=\sup_{T_{n-1}<t\leq T_{n}}|Z\left(t\right)-Z\left(T_{n-1}\right)|
\end{eqnarray*}
\end{Def}

\begin{Teo}
Sup\'ongase que $n^{-1}T_{n}\rightarrow\mu$ c.s. cuando $n\rightarrow\infty$, donde $\mu\leq\infty$ es una constante o variable aleatoria. Sea $a$ una constante o variable aleatoria que puede ser infinita cuando $\mu$ es finita, y considere las expresiones l\'imite:
\begin{eqnarray}
lim_{n\rightarrow\infty}n^{-1}Z\left(T_{n}\right)&=&a,\textrm{ c.s.}\\
lim_{t\rightarrow\infty}t^{-1}Z\left(t\right)&=&a/\mu,\textrm{ c.s.}
\end{eqnarray}
La segunda expresi\'on implica la primera. Conversamente, la primera implica la segunda si el proceso $Z\left(t\right)$ es creciente, o si $lim_{n\rightarrow\infty}n^{-1}M_{n}=0$ c.s.
\end{Teo}

\begin{Coro}
Si $N\left(t\right)$ es un proceso de renovaci\'on, y $\left(Z\left(T_{n}\right)-Z\left(T_{n-1}\right),M_{n}\right)$, para $n\geq1$, son variables aleatorias independientes e id\'enticamente distribuidas con media finita, entonces,
\begin{eqnarray}
lim_{t\rightarrow\infty}t^{-1}Z\left(t\right)\rightarrow\frac{\esp\left[Z\left(T_{1}\right)-Z\left(T_{0}\right)\right]}{\esp\left[T_{1}\right]},\textrm{ c.s. cuando  }t\rightarrow\infty.
\end{eqnarray}
\end{Coro}


Sup\'ongase que $N\left(t\right)$ es un proceso de renovaci\'on con distribuci\'on $F$ con media finita $\mu$.

\begin{Def}
La funci\'on de renovaci\'on asociada con la distribuci\'on $F$, del proceso $N\left(t\right)$, es
\begin{eqnarray*}
U\left(t\right)=\sum_{n=1}^{\infty}F^{n\star}\left(t\right),\textrm{   }t\geq0,
\end{eqnarray*}
donde $F^{0\star}\left(t\right)=\indora\left(t\geq0\right)$.
\end{Def}


\begin{Prop}
Sup\'ongase que la distribuci\'on de inter-renovaci\'on $F$ tiene densidad $f$. Entonces $U\left(t\right)$ tambi\'en tiene densidad, para $t>0$, y es $U^{'}\left(t\right)=\sum_{n=0}^{\infty}f^{n\star}\left(t\right)$. Adem\'as
\begin{eqnarray*}
\prob\left\{N\left(t\right)>N\left(t-\right)\right\}=0\textrm{,   }t\geq0.
\end{eqnarray*}
\end{Prop}

\begin{Def}
La Transformada de Laplace-Stieljes de $F$ est\'a dada por

\begin{eqnarray*}
\hat{F}\left(\alpha\right)=\int_{\rea_{+}}e^{-\alpha t}dF\left(t\right)\textrm{,  }\alpha\geq0.
\end{eqnarray*}
\end{Def}

Entonces

\begin{eqnarray*}
\hat{U}\left(\alpha\right)=\sum_{n=0}^{\infty}\hat{F^{n\star}}\left(\alpha\right)=\sum_{n=0}^{\infty}\hat{F}\left(\alpha\right)^{n}=\frac{1}{1-\hat{F}\left(\alpha\right)}.
\end{eqnarray*}


\begin{Prop}
La Transformada de Laplace $\hat{U}\left(\alpha\right)$ y $\hat{F}\left(\alpha\right)$ determina una a la otra de manera \'unica por la relaci\'on $\hat{U}\left(\alpha\right)=\frac{1}{1-\hat{F}\left(\alpha\right)}$.
\end{Prop}


\begin{Note}
Un proceso de renovaci\'on $N\left(t\right)$ cuyos tiempos de inter-renovaci\'on tienen media finita, es un proceso Poisson con tasa $\lambda$ si y s\'olo s\'i $\esp\left[U\left(t\right)\right]=\lambda t$, para $t\geq0$.
\end{Note}


\begin{Teo}
Sea $N\left(t\right)$ un proceso puntual simple con puntos de localizaci\'on $T_{n}$ tal que $\eta\left(t\right)=\esp\left[N\left(\right)\right]$ es finita para cada $t$. Entonces para cualquier funci\'on $f:\rea_{+}\rightarrow\rea$,
\begin{eqnarray*}
\esp\left[\sum_{n=1}^{N\left(\right)}f\left(T_{n}\right)\right]=\int_{\left(0,t\right]}f\left(s\right)d\eta\left(s\right)\textrm{,  }t\geq0,
\end{eqnarray*}
suponiendo que la integral exista. Adem\'as si $X_{1},X_{2},\ldots$ son variables aleatorias definidas en el mismo espacio de probabilidad que el proceso $N\left(t\right)$ tal que $\esp\left[X_{n}|T_{n}=s\right]=f\left(s\right)$, independiente de $n$. Entonces
\begin{eqnarray*}
\esp\left[\sum_{n=1}^{N\left(t\right)}X_{n}\right]=\int_{\left(0,t\right]}f\left(s\right)d\eta\left(s\right)\textrm{,  }t\geq0,
\end{eqnarray*} 
suponiendo que la integral exista. 
\end{Teo}

\begin{Coro}[Identidad de Wald para Renovaciones]
Para el proceso de renovaci\'on $N\left(t\right)$,
\begin{eqnarray*}
\esp\left[T_{N\left(t\right)+1}\right]=\mu\esp\left[N\left(t\right)+1\right]\textrm{,  }t\geq0,
\end{eqnarray*}  
\end{Coro}


\begin{Def}
Sea $h\left(t\right)$ funci\'on de valores reales en $\rea$ acotada en intervalos finitos e igual a cero para $t<0$ La ecuaci\'on de renovaci\'on para $h\left(t\right)$ y la distribuci\'on $F$ es

\begin{eqnarray}%\label{Ec.Renovacion}
H\left(t\right)=h\left(t\right)+\int_{\left[0,t\right]}H\left(t-s\right)dF\left(s\right)\textrm{,    }t\geq0,
\end{eqnarray}
donde $H\left(t\right)$ es una funci\'on de valores reales. Esto es $H=h+F\star H$. Decimos que $H\left(t\right)$ es soluci\'on de esta ecuaci\'on si satisface la ecuaci\'on, y es acotada en intervalos finitos e iguales a cero para $t<0$.
\end{Def}

\begin{Prop}
La funci\'on $U\star h\left(t\right)$ es la \'unica soluci\'on de la ecuaci\'on de renovaci\'on (\ref{Ec.Renovacion}).
\end{Prop}

\begin{Teo}[Teorema Renovaci\'on Elemental]
\begin{eqnarray*}
t^{-1}U\left(t\right)\rightarrow 1/\mu\textrm{,    cuando }t\rightarrow\infty.
\end{eqnarray*}
\end{Teo}



Sup\'ongase que $N\left(t\right)$ es un proceso de renovaci\'on con distribuci\'on $F$ con media finita $\mu$.

\begin{Def}
La funci\'on de renovaci\'on asociada con la distribuci\'on $F$, del proceso $N\left(t\right)$, es
\begin{eqnarray*}
U\left(t\right)=\sum_{n=1}^{\infty}F^{n\star}\left(t\right),\textrm{   }t\geq0,
\end{eqnarray*}
donde $F^{0\star}\left(t\right)=\indora\left(t\geq0\right)$.
\end{Def}


\begin{Prop}
Sup\'ongase que la distribuci\'on de inter-renovaci\'on $F$ tiene densidad $f$. Entonces $U\left(t\right)$ tambi\'en tiene densidad, para $t>0$, y es $U^{'}\left(t\right)=\sum_{n=0}^{\infty}f^{n\star}\left(t\right)$. Adem\'as
\begin{eqnarray*}
\prob\left\{N\left(t\right)>N\left(t-\right)\right\}=0\textrm{,   }t\geq0.
\end{eqnarray*}
\end{Prop}

\begin{Def}
La Transformada de Laplace-Stieljes de $F$ est\'a dada por

\begin{eqnarray*}
\hat{F}\left(\alpha\right)=\int_{\rea_{+}}e^{-\alpha t}dF\left(t\right)\textrm{,  }\alpha\geq0.
\end{eqnarray*}
\end{Def}

Entonces

\begin{eqnarray*}
\hat{U}\left(\alpha\right)=\sum_{n=0}^{\infty}\hat{F^{n\star}}\left(\alpha\right)=\sum_{n=0}^{\infty}\hat{F}\left(\alpha\right)^{n}=\frac{1}{1-\hat{F}\left(\alpha\right)}.
\end{eqnarray*}


\begin{Prop}
La Transformada de Laplace $\hat{U}\left(\alpha\right)$ y $\hat{F}\left(\alpha\right)$ determina una a la otra de manera \'unica por la relaci\'on $\hat{U}\left(\alpha\right)=\frac{1}{1-\hat{F}\left(\alpha\right)}$.
\end{Prop}


\begin{Note}
Un proceso de renovaci\'on $N\left(t\right)$ cuyos tiempos de inter-renovaci\'on tienen media finita, es un proceso Poisson con tasa $\lambda$ si y s\'olo s\'i $\esp\left[U\left(t\right)\right]=\lambda t$, para $t\geq0$.
\end{Note}


\begin{Teo}
Sea $N\left(t\right)$ un proceso puntual simple con puntos de localizaci\'on $T_{n}$ tal que $\eta\left(t\right)=\esp\left[N\left(\right)\right]$ es finita para cada $t$. Entonces para cualquier funci\'on $f:\rea_{+}\rightarrow\rea$,
\begin{eqnarray*}
\esp\left[\sum_{n=1}^{N\left(\right)}f\left(T_{n}\right)\right]=\int_{\left(0,t\right]}f\left(s\right)d\eta\left(s\right)\textrm{,  }t\geq0,
\end{eqnarray*}
suponiendo que la integral exista. Adem\'as si $X_{1},X_{2},\ldots$ son variables aleatorias definidas en el mismo espacio de probabilidad que el proceso $N\left(t\right)$ tal que $\esp\left[X_{n}|T_{n}=s\right]=f\left(s\right)$, independiente de $n$. Entonces
\begin{eqnarray*}
\esp\left[\sum_{n=1}^{N\left(t\right)}X_{n}\right]=\int_{\left(0,t\right]}f\left(s\right)d\eta\left(s\right)\textrm{,  }t\geq0,
\end{eqnarray*} 
suponiendo que la integral exista. 
\end{Teo}

\begin{Coro}[Identidad de Wald para Renovaciones]
Para el proceso de renovaci\'on $N\left(t\right)$,
\begin{eqnarray*}
\esp\left[T_{N\left(t\right)+1}\right]=\mu\esp\left[N\left(t\right)+1\right]\textrm{,  }t\geq0,
\end{eqnarray*}  
\end{Coro}


\begin{Def}
Sea $h\left(t\right)$ funci\'on de valores reales en $\rea$ acotada en intervalos finitos e igual a cero para $t<0$ La ecuaci\'on de renovaci\'on para $h\left(t\right)$ y la distribuci\'on $F$ es

\begin{eqnarray}%\label{Ec.Renovacion}
H\left(t\right)=h\left(t\right)+\int_{\left[0,t\right]}H\left(t-s\right)dF\left(s\right)\textrm{,    }t\geq0,
\end{eqnarray}
donde $H\left(t\right)$ es una funci\'on de valores reales. Esto es $H=h+F\star H$. Decimos que $H\left(t\right)$ es soluci\'on de esta ecuaci\'on si satisface la ecuaci\'on, y es acotada en intervalos finitos e iguales a cero para $t<0$.
\end{Def}

\begin{Prop}
La funci\'on $U\star h\left(t\right)$ es la \'unica soluci\'on de la ecuaci\'on de renovaci\'on (\ref{Ec.Renovacion}).
\end{Prop}

\begin{Teo}[Teorema Renovaci\'on Elemental]
\begin{eqnarray*}
t^{-1}U\left(t\right)\rightarrow 1/\mu\textrm{,    cuando }t\rightarrow\infty.
\end{eqnarray*}
\end{Teo}


\begin{Note} Una funci\'on $h:\rea_{+}\rightarrow\rea$ es Directamente Riemann Integrable en los siguientes casos:
\begin{itemize}
\item[a)] $h\left(t\right)\geq0$ es decreciente y Riemann Integrable.
\item[b)] $h$ es continua excepto posiblemente en un conjunto de Lebesgue de medida 0, y $|h\left(t\right)|\leq b\left(t\right)$, donde $b$ es DRI.
\end{itemize}
\end{Note}

\begin{Teo}[Teorema Principal de Renovaci\'on]
Si $F$ es no aritm\'etica y $h\left(t\right)$ es Directamente Riemann Integrable (DRI), entonces

\begin{eqnarray*}
lim_{t\rightarrow\infty}U\star h=\frac{1}{\mu}\int_{\rea_{+}}h\left(s\right)ds.
\end{eqnarray*}
\end{Teo}

\begin{Prop}
Cualquier funci\'on $H\left(t\right)$ acotada en intervalos finitos y que es 0 para $t<0$ puede expresarse como
\begin{eqnarray*}
H\left(t\right)=U\star h\left(t\right)\textrm{,  donde }h\left(t\right)=H\left(t\right)-F\star H\left(t\right)
\end{eqnarray*}
\end{Prop}

\begin{Def}
Un proceso estoc\'astico $X\left(t\right)$ es crudamente regenerativo en un tiempo aleatorio positivo $T$ si
\begin{eqnarray*}
\esp\left[X\left(T+t\right)|T\right]=\esp\left[X\left(t\right)\right]\textrm{, para }t\geq0,\end{eqnarray*}
y con las esperanzas anteriores finitas.
\end{Def}

\begin{Prop}
Sup\'ongase que $X\left(t\right)$ es un proceso crudamente regenerativo en $T$, que tiene distribuci\'on $F$. Si $\esp\left[X\left(t\right)\right]$ es acotado en intervalos finitos, entonces
\begin{eqnarray*}
\esp\left[X\left(t\right)\right]=U\star h\left(t\right)\textrm{,  donde }h\left(t\right)=\esp\left[X\left(t\right)\indora\left(T>t\right)\right].
\end{eqnarray*}
\end{Prop}

\begin{Teo}[Regeneraci\'on Cruda]
Sup\'ongase que $X\left(t\right)$ es un proceso con valores positivo crudamente regenerativo en $T$, y def\'inase $M=\sup\left\{|X\left(t\right)|:t\leq T\right\}$. Si $T$ es no aritm\'etico y $M$ y $MT$ tienen media finita, entonces
\begin{eqnarray*}
lim_{t\rightarrow\infty}\esp\left[X\left(t\right)\right]=\frac{1}{\mu}\int_{\rea_{+}}h\left(s\right)ds,
\end{eqnarray*}
donde $h\left(t\right)=\esp\left[X\left(t\right)\indora\left(T>t\right)\right]$.
\end{Teo}


\begin{Note} Una funci\'on $h:\rea_{+}\rightarrow\rea$ es Directamente Riemann Integrable en los siguientes casos:
\begin{itemize}
\item[a)] $h\left(t\right)\geq0$ es decreciente y Riemann Integrable.
\item[b)] $h$ es continua excepto posiblemente en un conjunto de Lebesgue de medida 0, y $|h\left(t\right)|\leq b\left(t\right)$, donde $b$ es DRI.
\end{itemize}
\end{Note}

\begin{Teo}[Teorema Principal de Renovaci\'on]
Si $F$ es no aritm\'etica y $h\left(t\right)$ es Directamente Riemann Integrable (DRI), entonces

\begin{eqnarray*}
lim_{t\rightarrow\infty}U\star h=\frac{1}{\mu}\int_{\rea_{+}}h\left(s\right)ds.
\end{eqnarray*}
\end{Teo}

\begin{Prop}
Cualquier funci\'on $H\left(t\right)$ acotada en intervalos finitos y que es 0 para $t<0$ puede expresarse como
\begin{eqnarray*}
H\left(t\right)=U\star h\left(t\right)\textrm{,  donde }h\left(t\right)=H\left(t\right)-F\star H\left(t\right)
\end{eqnarray*}
\end{Prop}

\begin{Def}
Un proceso estoc\'astico $X\left(t\right)$ es crudamente regenerativo en un tiempo aleatorio positivo $T$ si
\begin{eqnarray*}
\esp\left[X\left(T+t\right)|T\right]=\esp\left[X\left(t\right)\right]\textrm{, para }t\geq0,\end{eqnarray*}
y con las esperanzas anteriores finitas.
\end{Def}

\begin{Prop}
Sup\'ongase que $X\left(t\right)$ es un proceso crudamente regenerativo en $T$, que tiene distribuci\'on $F$. Si $\esp\left[X\left(t\right)\right]$ es acotado en intervalos finitos, entonces
\begin{eqnarray*}
\esp\left[X\left(t\right)\right]=U\star h\left(t\right)\textrm{,  donde }h\left(t\right)=\esp\left[X\left(t\right)\indora\left(T>t\right)\right].
\end{eqnarray*}
\end{Prop}

\begin{Teo}[Regeneraci\'on Cruda]
Sup\'ongase que $X\left(t\right)$ es un proceso con valores positivo crudamente regenerativo en $T$, y def\'inase $M=\sup\left\{|X\left(t\right)|:t\leq T\right\}$. Si $T$ es no aritm\'etico y $M$ y $MT$ tienen media finita, entonces
\begin{eqnarray*}
lim_{t\rightarrow\infty}\esp\left[X\left(t\right)\right]=\frac{1}{\mu}\int_{\rea_{+}}h\left(s\right)ds,
\end{eqnarray*}
donde $h\left(t\right)=\esp\left[X\left(t\right)\indora\left(T>t\right)\right]$.
\end{Teo}

\begin{Def}
Para el proceso $\left\{\left(N\left(t\right),X\left(t\right)\right):t\geq0\right\}$, sus trayectoria muestrales en el intervalo de tiempo $\left[T_{n-1},T_{n}\right)$ est\'an descritas por
\begin{eqnarray*}
\zeta_{n}=\left(\xi_{n},\left\{X\left(T_{n-1}+t\right):0\leq t<\xi_{n}\right\}\right)
\end{eqnarray*}
Este $\zeta_{n}$ es el $n$-\'esimo segmento del proceso. El proceso es regenerativo sobre los tiempos $T_{n}$ si sus segmentos $\zeta_{n}$ son independientes e id\'enticamennte distribuidos.
\end{Def}


\begin{Note}
Si $\tilde{X}\left(t\right)$ con espacio de estados $\tilde{S}$ es regenerativo sobre $T_{n}$, entonces $X\left(t\right)=f\left(\tilde{X}\left(t\right)\right)$ tambi\'en es regenerativo sobre $T_{n}$, para cualquier funci\'on $f:\tilde{S}\rightarrow S$.
\end{Note}

\begin{Note}
Los procesos regenerativos son crudamente regenerativos, pero no al rev\'es.
\end{Note}


\begin{Note}
Un proceso estoc\'astico a tiempo continuo o discreto es regenerativo si existe un proceso de renovaci\'on  tal que los segmentos del proceso entre tiempos de renovaci\'on sucesivos son i.i.d., es decir, para $\left\{X\left(t\right):t\geq0\right\}$ proceso estoc\'astico a tiempo continuo con espacio de estados $S$, espacio m\'etrico.
\end{Note}

Para $\left\{X\left(t\right):t\geq0\right\}$ Proceso Estoc\'astico a tiempo continuo con estado de espacios $S$, que es un espacio m\'etrico, con trayectorias continuas por la derecha y con l\'imites por la izquierda c.s. Sea $N\left(t\right)$ un proceso de renovaci\'on en $\rea_{+}$ definido en el mismo espacio de probabilidad que $X\left(t\right)$, con tiempos de renovaci\'on $T$ y tiempos de inter-renovaci\'on $\xi_{n}=T_{n}-T_{n-1}$, con misma distribuci\'on $F$ de media finita $\mu$.



\begin{Def}
Para el proceso $\left\{\left(N\left(t\right),X\left(t\right)\right):t\geq0\right\}$, sus trayectoria muestrales en el intervalo de tiempo $\left[T_{n-1},T_{n}\right)$ est\'an descritas por
\begin{eqnarray*}
\zeta_{n}=\left(\xi_{n},\left\{X\left(T_{n-1}+t\right):0\leq t<\xi_{n}\right\}\right)
\end{eqnarray*}
Este $\zeta_{n}$ es el $n$-\'esimo segmento del proceso. El proceso es regenerativo sobre los tiempos $T_{n}$ si sus segmentos $\zeta_{n}$ son independientes e id\'enticamennte distribuidos.
\end{Def}

\begin{Note}
Un proceso regenerativo con media de la longitud de ciclo finita es llamado positivo recurrente.
\end{Note}

\begin{Teo}[Procesos Regenerativos]
Suponga que el proceso
\end{Teo}


\begin{Def}[Renewal Process Trinity]
Para un proceso de renovaci\'on $N\left(t\right)$, los siguientes procesos proveen de informaci\'on sobre los tiempos de renovaci\'on.
\begin{itemize}
\item $A\left(t\right)=t-T_{N\left(t\right)}$, el tiempo de recurrencia hacia atr\'as al tiempo $t$, que es el tiempo desde la \'ultima renovaci\'on para $t$.

\item $B\left(t\right)=T_{N\left(t\right)+1}-t$, el tiempo de recurrencia hacia adelante al tiempo $t$, residual del tiempo de renovaci\'on, que es el tiempo para la pr\'oxima renovaci\'on despu\'es de $t$.

\item $L\left(t\right)=\xi_{N\left(t\right)+1}=A\left(t\right)+B\left(t\right)$, la longitud del intervalo de renovaci\'on que contiene a $t$.
\end{itemize}
\end{Def}

\begin{Note}
El proceso tridimensional $\left(A\left(t\right),B\left(t\right),L\left(t\right)\right)$ es regenerativo sobre $T_{n}$, y por ende cada proceso lo es. Cada proceso $A\left(t\right)$ y $B\left(t\right)$ son procesos de MArkov a tiempo continuo con trayectorias continuas por partes en el espacio de estados $\rea_{+}$. Una expresi\'on conveniente para su distribuci\'on conjunta es, para $0\leq x<t,y\geq0$
\begin{equation}\label{NoRenovacion}
P\left\{A\left(t\right)>x,B\left(t\right)>y\right\}=
P\left\{N\left(t+y\right)-N\left((t-x)\right)=0\right\}
\end{equation}
\end{Note}

\begin{Ejem}[Tiempos de recurrencia Poisson]
Si $N\left(t\right)$ es un proceso Poisson con tasa $\lambda$, entonces de la expresi\'on (\ref{NoRenovacion}) se tiene que

\begin{eqnarray*}
\begin{array}{lc}
P\left\{A\left(t\right)>x,B\left(t\right)>y\right\}=e^{-\lambda\left(x+y\right)},&0\leq x<t,y\geq0,
\end{array}
\end{eqnarray*}
que es la probabilidad Poisson de no renovaciones en un intervalo de longitud $x+y$.

\end{Ejem}

%\begin{Note}
Una cadena de Markov erg\'odica tiene la propiedad de ser estacionaria si la distribuci\'on de su estado al tiempo $0$ es su distribuci\'on estacionaria.
%\end{Note}


\begin{Def}
Un proceso estoc\'astico a tiempo continuo $\left\{X\left(t\right):t\geq0\right\}$ en un espacio general es estacionario si sus distribuciones finito dimensionales son invariantes bajo cualquier  traslado: para cada $0\leq s_{1}<s_{2}<\cdots<s_{k}$ y $t\geq0$,
\begin{eqnarray*}
\left(X\left(s_{1}+t\right),\ldots,X\left(s_{k}+t\right)\right)=_{d}\left(X\left(s_{1}\right),\ldots,X\left(s_{k}\right)\right).
\end{eqnarray*}
\end{Def}

\begin{Note}
Un proceso de Markov es estacionario si $X\left(t\right)=_{d}X\left(0\right)$, $t\geq0$.
\end{Note}

Considerese el proceso $N\left(t\right)=\sum_{n}\indora\left(\tau_{n}\leq t\right)$ en $\rea_{+}$, con puntos $0<\tau_{1}<\tau_{2}<\cdots$.

\begin{Prop}
Si $N$ es un proceso puntual estacionario y $\esp\left[N\left(1\right)\right]<\infty$, entonces $\esp\left[N\left(t\right)\right]=t\esp\left[N\left(1\right)\right]$, $t\geq0$

\end{Prop}

\begin{Teo}
Los siguientes enunciados son equivalentes
\begin{itemize}
\item[i)] El proceso retardado de renovaci\'on $N$ es estacionario.

\item[ii)] EL proceso de tiempos de recurrencia hacia adelante $B\left(t\right)$ es estacionario.


\item[iii)] $\esp\left[N\left(t\right)\right]=t/\mu$,


\item[iv)] $G\left(t\right)=F_{e}\left(t\right)=\frac{1}{\mu}\int_{0}^{t}\left[1-F\left(s\right)\right]ds$
\end{itemize}
Cuando estos enunciados son ciertos, $P\left\{B\left(t\right)\leq x\right\}=F_{e}\left(x\right)$, para $t,x\geq0$.

\end{Teo}

\begin{Note}
Una consecuencia del teorema anterior es que el Proceso Poisson es el \'unico proceso sin retardo que es estacionario.
\end{Note}

\begin{Coro}
El proceso de renovaci\'on $N\left(t\right)$ sin retardo, y cuyos tiempos de inter renonaci\'on tienen media finita, es estacionario si y s\'olo si es un proceso Poisson.

\end{Coro}


%________________________________________________________________________
\subsubsection{Procesos Regenerativos}
%________________________________________________________________________



\begin{Note}
Si $\tilde{X}\left(t\right)$ con espacio de estados $\tilde{S}$ es regenerativo sobre $T_{n}$, entonces $X\left(t\right)=f\left(\tilde{X}\left(t\right)\right)$ tambi\'en es regenerativo sobre $T_{n}$, para cualquier funci\'on $f:\tilde{S}\rightarrow S$.
\end{Note}

\begin{Note}
Los procesos regenerativos son crudamente regenerativos, pero no al rev\'es.
\end{Note}
%\subsection*{Procesos Regenerativos: Sigman\cite{Sigman1}}
\begin{Def}[Definici\'on Cl\'asica]
Un proceso estoc\'astico $X=\left\{X\left(t\right):t\geq0\right\}$ es llamado regenerativo is existe una variable aleatoria $R_{1}>0$ tal que
\begin{itemize}
\item[i)] $\left\{X\left(t+R_{1}\right):t\geq0\right\}$ es independiente de $\left\{\left\{X\left(t\right):t<R_{1}\right\},\right\}$
\item[ii)] $\left\{X\left(t+R_{1}\right):t\geq0\right\}$ es estoc\'asticamente equivalente a $\left\{X\left(t\right):t>0\right\}$
\end{itemize}

Llamamos a $R_{1}$ tiempo de regeneraci\'on, y decimos que $X$ se regenera en este punto.
\end{Def}

$\left\{X\left(t+R_{1}\right)\right\}$ es regenerativo con tiempo de regeneraci\'on $R_{2}$, independiente de $R_{1}$ pero con la misma distribuci\'on que $R_{1}$. Procediendo de esta manera se obtiene una secuencia de variables aleatorias independientes e id\'enticamente distribuidas $\left\{R_{n}\right\}$ llamados longitudes de ciclo. Si definimos a $Z_{k}\equiv R_{1}+R_{2}+\cdots+R_{k}$, se tiene un proceso de renovaci\'on llamado proceso de renovaci\'on encajado para $X$.




\begin{Def}
Para $x$ fijo y para cada $t\geq0$, sea $I_{x}\left(t\right)=1$ si $X\left(t\right)\leq x$,  $I_{x}\left(t\right)=0$ en caso contrario, y def\'inanse los tiempos promedio
\begin{eqnarray*}
\overline{X}&=&lim_{t\rightarrow\infty}\frac{1}{t}\int_{0}^{\infty}X\left(u\right)du\\
\prob\left(X_{\infty}\leq x\right)&=&lim_{t\rightarrow\infty}\frac{1}{t}\int_{0}^{\infty}I_{x}\left(u\right)du,
\end{eqnarray*}
cuando estos l\'imites existan.
\end{Def}

Como consecuencia del teorema de Renovaci\'on-Recompensa, se tiene que el primer l\'imite  existe y es igual a la constante
\begin{eqnarray*}
\overline{X}&=&\frac{\esp\left[\int_{0}^{R_{1}}X\left(t\right)dt\right]}{\esp\left[R_{1}\right]},
\end{eqnarray*}
suponiendo que ambas esperanzas son finitas.

\begin{Note}
\begin{itemize}
\item[a)] Si el proceso regenerativo $X$ es positivo recurrente y tiene trayectorias muestrales no negativas, entonces la ecuaci\'on anterior es v\'alida.
\item[b)] Si $X$ es positivo recurrente regenerativo, podemos construir una \'unica versi\'on estacionaria de este proceso, $X_{e}=\left\{X_{e}\left(t\right)\right\}$, donde $X_{e}$ es un proceso estoc\'astico regenerativo y estrictamente estacionario, con distribuci\'on marginal distribuida como $X_{\infty}$
\end{itemize}
\end{Note}

%________________________________________________________________________
%\subsection{Procesos Regenerativos}
%________________________________________________________________________

Para $\left\{X\left(t\right):t\geq0\right\}$ Proceso Estoc\'astico a tiempo continuo con estado de espacios $S$, que es un espacio m\'etrico, con trayectorias continuas por la derecha y con l\'imites por la izquierda c.s. Sea $N\left(t\right)$ un proceso de renovaci\'on en $\rea_{+}$ definido en el mismo espacio de probabilidad que $X\left(t\right)$, con tiempos de renovaci\'on $T$ y tiempos de inter-renovaci\'on $\xi_{n}=T_{n}-T_{n-1}$, con misma distribuci\'on $F$ de media finita $\mu$.



\begin{Def}
Para el proceso $\left\{\left(N\left(t\right),X\left(t\right)\right):t\geq0\right\}$, sus trayectoria muestrales en el intervalo de tiempo $\left[T_{n-1},T_{n}\right)$ est\'an descritas por
\begin{eqnarray*}
\zeta_{n}=\left(\xi_{n},\left\{X\left(T_{n-1}+t\right):0\leq t<\xi_{n}\right\}\right)
\end{eqnarray*}
Este $\zeta_{n}$ es el $n$-\'esimo segmento del proceso. El proceso es regenerativo sobre los tiempos $T_{n}$ si sus segmentos $\zeta_{n}$ son independientes e id\'enticamennte distribuidos.
\end{Def}


\begin{Note}
Si $\tilde{X}\left(t\right)$ con espacio de estados $\tilde{S}$ es regenerativo sobre $T_{n}$, entonces $X\left(t\right)=f\left(\tilde{X}\left(t\right)\right)$ tambi\'en es regenerativo sobre $T_{n}$, para cualquier funci\'on $f:\tilde{S}\rightarrow S$.
\end{Note}

\begin{Note}
Los procesos regenerativos son crudamente regenerativos, pero no al rev\'es.
\end{Note}

\begin{Def}[Definici\'on Cl\'asica]
Un proceso estoc\'astico $X=\left\{X\left(t\right):t\geq0\right\}$ es llamado regenerativo is existe una variable aleatoria $R_{1}>0$ tal que
\begin{itemize}
\item[i)] $\left\{X\left(t+R_{1}\right):t\geq0\right\}$ es independiente de $\left\{\left\{X\left(t\right):t<R_{1}\right\},\right\}$
\item[ii)] $\left\{X\left(t+R_{1}\right):t\geq0\right\}$ es estoc\'asticamente equivalente a $\left\{X\left(t\right):t>0\right\}$
\end{itemize}

Llamamos a $R_{1}$ tiempo de regeneraci\'on, y decimos que $X$ se regenera en este punto.
\end{Def}

$\left\{X\left(t+R_{1}\right)\right\}$ es regenerativo con tiempo de regeneraci\'on $R_{2}$, independiente de $R_{1}$ pero con la misma distribuci\'on que $R_{1}$. Procediendo de esta manera se obtiene una secuencia de variables aleatorias independientes e id\'enticamente distribuidas $\left\{R_{n}\right\}$ llamados longitudes de ciclo. Si definimos a $Z_{k}\equiv R_{1}+R_{2}+\cdots+R_{k}$, se tiene un proceso de renovaci\'on llamado proceso de renovaci\'on encajado para $X$.

\begin{Note}
Un proceso regenerativo con media de la longitud de ciclo finita es llamado positivo recurrente.
\end{Note}


\begin{Def}
Para $x$ fijo y para cada $t\geq0$, sea $I_{x}\left(t\right)=1$ si $X\left(t\right)\leq x$,  $I_{x}\left(t\right)=0$ en caso contrario, y def\'inanse los tiempos promedio
\begin{eqnarray*}
\overline{X}&=&lim_{t\rightarrow\infty}\frac{1}{t}\int_{0}^{\infty}X\left(u\right)du\\
\prob\left(X_{\infty}\leq x\right)&=&lim_{t\rightarrow\infty}\frac{1}{t}\int_{0}^{\infty}I_{x}\left(u\right)du,
\end{eqnarray*}
cuando estos l\'imites existan.
\end{Def}

Como consecuencia del teorema de Renovaci\'on-Recompensa, se tiene que el primer l\'imite  existe y es igual a la constante
\begin{eqnarray*}
\overline{X}&=&\frac{\esp\left[\int_{0}^{R_{1}}X\left(t\right)dt\right]}{\esp\left[R_{1}\right]},
\end{eqnarray*}
suponiendo que ambas esperanzas son finitas.

\begin{Note}
\begin{itemize}
\item[a)] Si el proceso regenerativo $X$ es positivo recurrente y tiene trayectorias muestrales no negativas, entonces la ecuaci\'on anterior es v\'alida.
\item[b)] Si $X$ es positivo recurrente regenerativo, podemos construir una \'unica versi\'on estacionaria de este proceso, $X_{e}=\left\{X_{e}\left(t\right)\right\}$, donde $X_{e}$ es un proceso estoc\'astico regenerativo y estrictamente estacionario, con distribuci\'on marginal distribuida como $X_{\infty}$
\end{itemize}
\end{Note}

%__________________________________________________________________________________________
%\subsection{Procesos Regenerativos Estacionarios - Stidham \cite{Stidham}}
%__________________________________________________________________________________________


Un proceso estoc\'astico a tiempo continuo $\left\{V\left(t\right),t\geq0\right\}$ es un proceso regenerativo si existe una sucesi\'on de variables aleatorias independientes e id\'enticamente distribuidas $\left\{X_{1},X_{2},\ldots\right\}$, sucesi\'on de renovaci\'on, tal que para cualquier conjunto de Borel $A$, 

\begin{eqnarray*}
\prob\left\{V\left(t\right)\in A|X_{1}+X_{2}+\cdots+X_{R\left(t\right)}=s,\left\{V\left(\tau\right),\tau<s\right\}\right\}=\prob\left\{V\left(t-s\right)\in A|X_{1}>t-s\right\},
\end{eqnarray*}
para todo $0\leq s\leq t$, donde $R\left(t\right)=\max\left\{X_{1}+X_{2}+\cdots+X_{j}\leq t\right\}=$n\'umero de renovaciones ({\emph{puntos de regeneraci\'on}}) que ocurren en $\left[0,t\right]$. El intervalo $\left[0,X_{1}\right)$ es llamado {\emph{primer ciclo de regeneraci\'on}} de $\left\{V\left(t \right),t\geq0\right\}$, $\left[X_{1},X_{1}+X_{2}\right)$ el {\emph{segundo ciclo de regeneraci\'on}}, y as\'i sucesivamente.

Sea $X=X_{1}$ y sea $F$ la funci\'on de distrbuci\'on de $X$


\begin{Def}
Se define el proceso estacionario, $\left\{V^{*}\left(t\right),t\geq0\right\}$, para $\left\{V\left(t\right),t\geq0\right\}$ por

\begin{eqnarray*}
\prob\left\{V\left(t\right)\in A\right\}=\frac{1}{\esp\left[X\right]}\int_{0}^{\infty}\prob\left\{V\left(t+x\right)\in A|X>x\right\}\left(1-F\left(x\right)\right)dx,
\end{eqnarray*} 
para todo $t\geq0$ y todo conjunto de Borel $A$.
\end{Def}

\begin{Def}
Una distribuci\'on se dice que es {\emph{aritm\'etica}} si todos sus puntos de incremento son m\'ultiplos de la forma $0,\lambda, 2\lambda,\ldots$ para alguna $\lambda>0$ entera.
\end{Def}


\begin{Def}
Una modificaci\'on medible de un proceso $\left\{V\left(t\right),t\geq0\right\}$, es una versi\'on de este, $\left\{V\left(t,w\right)\right\}$ conjuntamente medible para $t\geq0$ y para $w\in S$, $S$ espacio de estados para $\left\{V\left(t\right),t\geq0\right\}$.
\end{Def}

\begin{Teo}
Sea $\left\{V\left(t\right),t\geq\right\}$ un proceso regenerativo no negativo con modificaci\'on medible. Sea $\esp\left[X\right]<\infty$. Entonces el proceso estacionario dado por la ecuaci\'on anterior est\'a bien definido y tiene funci\'on de distribuci\'on independiente de $t$, adem\'as
\begin{itemize}
\item[i)] \begin{eqnarray*}
\esp\left[V^{*}\left(0\right)\right]&=&\frac{\esp\left[\int_{0}^{X}V\left(s\right)ds\right]}{\esp\left[X\right]}\end{eqnarray*}
\item[ii)] Si $\esp\left[V^{*}\left(0\right)\right]<\infty$, equivalentemente, si $\esp\left[\int_{0}^{X}V\left(s\right)ds\right]<\infty$,entonces
\begin{eqnarray*}
\frac{\int_{0}^{t}V\left(s\right)ds}{t}\rightarrow\frac{\esp\left[\int_{0}^{X}V\left(s\right)ds\right]}{\esp\left[X\right]}
\end{eqnarray*}
con probabilidad 1 y en media, cuando $t\rightarrow\infty$.
\end{itemize}
\end{Teo}

%__________________________________________________________________________________________
%\subsection{Procesos Regenerativos Estacionarios - Stidham \cite{Stidham}}
%__________________________________________________________________________________________


Un proceso estoc\'astico a tiempo continuo $\left\{V\left(t\right),t\geq0\right\}$ es un proceso regenerativo si existe una sucesi\'on de variables aleatorias independientes e id\'enticamente distribuidas $\left\{X_{1},X_{2},\ldots\right\}$, sucesi\'on de renovaci\'on, tal que para cualquier conjunto de Borel $A$, 

\begin{eqnarray*}
\prob\left\{V\left(t\right)\in A|X_{1}+X_{2}+\cdots+X_{R\left(t\right)}=s,\left\{V\left(\tau\right),\tau<s\right\}\right\}=\prob\left\{V\left(t-s\right)\in A|X_{1}>t-s\right\},
\end{eqnarray*}
para todo $0\leq s\leq t$, donde $R\left(t\right)=\max\left\{X_{1}+X_{2}+\cdots+X_{j}\leq t\right\}=$n\'umero de renovaciones ({\emph{puntos de regeneraci\'on}}) que ocurren en $\left[0,t\right]$. El intervalo $\left[0,X_{1}\right)$ es llamado {\emph{primer ciclo de regeneraci\'on}} de $\left\{V\left(t \right),t\geq0\right\}$, $\left[X_{1},X_{1}+X_{2}\right)$ el {\emph{segundo ciclo de regeneraci\'on}}, y as\'i sucesivamente.

Sea $X=X_{1}$ y sea $F$ la funci\'on de distrbuci\'on de $X$


\begin{Def}
Se define el proceso estacionario, $\left\{V^{*}\left(t\right),t\geq0\right\}$, para $\left\{V\left(t\right),t\geq0\right\}$ por

\begin{eqnarray*}
\prob\left\{V\left(t\right)\in A\right\}=\frac{1}{\esp\left[X\right]}\int_{0}^{\infty}\prob\left\{V\left(t+x\right)\in A|X>x\right\}\left(1-F\left(x\right)\right)dx,
\end{eqnarray*} 
para todo $t\geq0$ y todo conjunto de Borel $A$.
\end{Def}

\begin{Def}
Una distribuci\'on se dice que es {\emph{aritm\'etica}} si todos sus puntos de incremento son m\'ultiplos de la forma $0,\lambda, 2\lambda,\ldots$ para alguna $\lambda>0$ entera.
\end{Def}


\begin{Def}
Una modificaci\'on medible de un proceso $\left\{V\left(t\right),t\geq0\right\}$, es una versi\'on de este, $\left\{V\left(t,w\right)\right\}$ conjuntamente medible para $t\geq0$ y para $w\in S$, $S$ espacio de estados para $\left\{V\left(t\right),t\geq0\right\}$.
\end{Def}

\begin{Teo}
Sea $\left\{V\left(t\right),t\geq\right\}$ un proceso regenerativo no negativo con modificaci\'on medible. Sea $\esp\left[X\right]<\infty$. Entonces el proceso estacionario dado por la ecuaci\'on anterior est\'a bien definido y tiene funci\'on de distribuci\'on independiente de $t$, adem\'as
\begin{itemize}
\item[i)] \begin{eqnarray*}
\esp\left[V^{*}\left(0\right)\right]&=&\frac{\esp\left[\int_{0}^{X}V\left(s\right)ds\right]}{\esp\left[X\right]}\end{eqnarray*}
\item[ii)] Si $\esp\left[V^{*}\left(0\right)\right]<\infty$, equivalentemente, si $\esp\left[\int_{0}^{X}V\left(s\right)ds\right]<\infty$,entonces
\begin{eqnarray*}
\frac{\int_{0}^{t}V\left(s\right)ds}{t}\rightarrow\frac{\esp\left[\int_{0}^{X}V\left(s\right)ds\right]}{\esp\left[X\right]}
\end{eqnarray*}
con probabilidad 1 y en media, cuando $t\rightarrow\infty$.
\end{itemize}
\end{Teo}

Para $\left\{X\left(t\right):t\geq0\right\}$ Proceso Estoc\'astico a tiempo continuo con estado de espacios $S$, que es un espacio m\'etrico, con trayectorias continuas por la derecha y con l\'imites por la izquierda c.s. Sea $N\left(t\right)$ un proceso de renovaci\'on en $\rea_{+}$ definido en el mismo espacio de probabilidad que $X\left(t\right)$, con tiempos de renovaci\'on $T$ y tiempos de inter-renovaci\'on $\xi_{n}=T_{n}-T_{n-1}$, con misma distribuci\'on $F$ de media finita $\mu$.
%_____________________________________________________
\subsection{Puntos de Renovaci\'on}
%_____________________________________________________

Para cada cola $Q_{i}$ se tienen los procesos de arribo a la cola, para estas, los tiempos de arribo est\'an dados por $$\left\{T_{1}^{i},T_{2}^{i},\ldots,T_{k}^{i},\ldots\right\},$$ entonces, consideremos solamente los primeros tiempos de arribo a cada una de las colas, es decir, $$\left\{T_{1}^{1},T_{1}^{2},T_{1}^{3},T_{1}^{4}\right\},$$ se sabe que cada uno de estos tiempos se distribuye de manera exponencial con par\'ametro $1/mu_{i}$. Adem\'as se sabe que para $$T^{*}=\min\left\{T_{1}^{1},T_{1}^{2},T_{1}^{3},T_{1}^{4}\right\},$$ $T^{*}$ se distribuye de manera exponencial con par\'ametro $$\mu^{*}=\sum_{i=1}^{4}\mu_{i}.$$ Ahora, dado que 
\begin{center}
\begin{tabular}{lcl}
$\tilde{r}=r_{1}+r_{2}$ & y &$\hat{r}=r_{3}+r_{4}.$
\end{tabular}
\end{center}


Supongamos que $$\tilde{r},\hat{r}<\mu^{*},$$ entonces si tomamos $$r^{*}=\min\left\{\tilde{r},\hat{r}\right\},$$ se tiene que para  $$t^{*}\in\left(0,r^{*}\right)$$ se cumple que 
\begin{center}
\begin{tabular}{lcl}
$\tau_{1}\left(1\right)=0$ & y por tanto & $\overline{\tau}_{1}=0,$
\end{tabular}
\end{center}
entonces para la segunda cola en este primer ciclo se cumple que $$\tau_{2}=\overline{\tau}_{1}+r_{1}=r_{1}<\mu^{*},$$ y por tanto se tiene que  $$\overline{\tau}_{2}=\tau_{2}.$$ Por lo tanto, nuevamente para la primer cola en el segundo ciclo $$\tau_{1}\left(2\right)=\tau_{2}\left(1\right)+r_{2}=\tilde{r}<\mu^{*}.$$ An\'alogamente para el segundo sistema se tiene que ambas colas est\'an vac\'ias, es decir, existe un valor $t^{*}$ tal que en el intervalo $\left(0,t^{*}\right)$ no ha llegado ning\'un usuario, es decir, $$L_{i}\left(t^{*}\right)=0$$ para $i=1,2,3,4$.

\subsection{Resultados para Procesos de Salida}




%________________________________________________________________________
\subsection{Procesos Regenerativos}
%________________________________________________________________________

Para $\left\{X\left(t\right):t\geq0\right\}$ Proceso Estoc\'astico a tiempo continuo con estado de espacios $S$, que es un espacio m\'etrico, con trayectorias continuas por la derecha y con l\'imites por la izquierda c.s. Sea $N\left(t\right)$ un proceso de renovaci\'on en $\rea_{+}$ definido en el mismo espacio de probabilidad que $X\left(t\right)$, con tiempos de renovaci\'on $T$ y tiempos de inter-renovaci\'on $\xi_{n}=T_{n}-T_{n-1}$, con misma distribuci\'on $F$ de media finita $\mu$.



\begin{Def}
Para el proceso $\left\{\left(N\left(t\right),X\left(t\right)\right):t\geq0\right\}$, sus trayectoria muestrales en el intervalo de tiempo $\left[T_{n-1},T_{n}\right)$ est\'an descritas por
\begin{eqnarray*}
\zeta_{n}=\left(\xi_{n},\left\{X\left(T_{n-1}+t\right):0\leq t<\xi_{n}\right\}\right)
\end{eqnarray*}
Este $\zeta_{n}$ es el $n$-\'esimo segmento del proceso. El proceso es regenerativo sobre los tiempos $T_{n}$ si sus segmentos $\zeta_{n}$ son independientes e id\'enticamennte distribuidos.
\end{Def}


\begin{Obs}
Si $\tilde{X}\left(t\right)$ con espacio de estados $\tilde{S}$ es regenerativo sobre $T_{n}$, entonces $X\left(t\right)=f\left(\tilde{X}\left(t\right)\right)$ tambi\'en es regenerativo sobre $T_{n}$, para cualquier funci\'on $f:\tilde{S}\rightarrow S$.
\end{Obs}

\begin{Obs}
Los procesos regenerativos son crudamente regenerativos, pero no al rev\'es.
\end{Obs}

\begin{Def}[Definici\'on Cl\'asica]
Un proceso estoc\'astico $X=\left\{X\left(t\right):t\geq0\right\}$ es llamado regenerativo is existe una variable aleatoria $R_{1}>0$ tal que
\begin{itemize}
\item[i)] $\left\{X\left(t+R_{1}\right):t\geq0\right\}$ es independiente de $\left\{\left\{X\left(t\right):t<R_{1}\right\},\right\}$
\item[ii)] $\left\{X\left(t+R_{1}\right):t\geq0\right\}$ es estoc\'asticamente equivalente a $\left\{X\left(t\right):t>0\right\}$
\end{itemize}

Llamamos a $R_{1}$ tiempo de regeneraci\'on, y decimos que $X$ se regenera en este punto.
\end{Def}

$\left\{X\left(t+R_{1}\right)\right\}$ es regenerativo con tiempo de regeneraci\'on $R_{2}$, independiente de $R_{1}$ pero con la misma distribuci\'on que $R_{1}$. Procediendo de esta manera se obtiene una secuencia de variables aleatorias independientes e id\'enticamente distribuidas $\left\{R_{n}\right\}$ llamados longitudes de ciclo. Si definimos a $Z_{k}\equiv R_{1}+R_{2}+\cdots+R_{k}$, se tiene un proceso de renovaci\'on llamado proceso de renovaci\'on encajado para $X$.

\begin{Note}
Un proceso regenerativo con media de la longitud de ciclo finita es llamado positivo recurrente.
\end{Note}


\begin{Def}
Para $x$ fijo y para cada $t\geq0$, sea $I_{x}\left(t\right)=1$ si $X\left(t\right)\leq x$,  $I_{x}\left(t\right)=0$ en caso contrario, y def\'inanse los tiempos promedio
\begin{eqnarray*}
\overline{X}&=&lim_{t\rightarrow\infty}\frac{1}{t}\int_{0}^{\infty}X\left(u\right)du\\
\prob\left(X_{\infty}\leq x\right)&=&lim_{t\rightarrow\infty}\frac{1}{t}\int_{0}^{\infty}I_{x}\left(u\right)du,
\end{eqnarray*}
cuando estos l\'imites existan.
\end{Def}

Como consecuencia del teorema de Renovaci\'on-Recompensa, se tiene que el primer l\'imite  existe y es igual a la constante
\begin{eqnarray*}
\overline{X}&=&\frac{\esp\left[\int_{0}^{R_{1}}X\left(t\right)dt\right]}{\esp\left[R_{1}\right]},
\end{eqnarray*}
suponiendo que ambas esperanzas son finitas.

\begin{Note}
\begin{itemize}
\item[a)] Si el proceso regenerativo $X$ es positivo recurrente y tiene trayectorias muestrales no negativas, entonces la ecuaci\'on anterior es v\'alida.
\item[b)] Si $X$ es positivo recurrente regenerativo, podemos construir una \'unica versi\'on estacionaria de este proceso, $X_{e}=\left\{X_{e}\left(t\right)\right\}$, donde $X_{e}$ es un proceso estoc\'astico regenerativo y estrictamente estacionario, con distribuci\'on marginal distribuida como $X_{\infty}$
\end{itemize}
\end{Note}

\subsection{Renewal and Regenerative Processes: Serfozo\cite{Serfozo}}
\begin{Def}\label{Def.Tn}
Sean $0\leq T_{1}\leq T_{2}\leq \ldots$ son tiempos aleatorios infinitos en los cuales ocurren ciertos eventos. El n\'umero de tiempos $T_{n}$ en el intervalo $\left[0,t\right)$ es

\begin{eqnarray}
N\left(t\right)=\sum_{n=1}^{\infty}\indora\left(T_{n}\leq t\right),
\end{eqnarray}
para $t\geq0$.
\end{Def}

Si se consideran los puntos $T_{n}$ como elementos de $\rea_{+}$, y $N\left(t\right)$ es el n\'umero de puntos en $\rea$. El proceso denotado por $\left\{N\left(t\right):t\geq0\right\}$, denotado por $N\left(t\right)$, es un proceso puntual en $\rea_{+}$. Los $T_{n}$ son los tiempos de ocurrencia, el proceso puntual $N\left(t\right)$ es simple si su n\'umero de ocurrencias son distintas: $0<T_{1}<T_{2}<\ldots$ casi seguramente.

\begin{Def}
Un proceso puntual $N\left(t\right)$ es un proceso de renovaci\'on si los tiempos de interocurrencia $\xi_{n}=T_{n}-T_{n-1}$, para $n\geq1$, son independientes e identicamente distribuidos con distribuci\'on $F$, donde $F\left(0\right)=0$ y $T_{0}=0$. Los $T_{n}$ son llamados tiempos de renovaci\'on, referente a la independencia o renovaci\'on de la informaci\'on estoc\'astica en estos tiempos. Los $\xi_{n}$ son los tiempos de inter-renovaci\'on, y $N\left(t\right)$ es el n\'umero de renovaciones en el intervalo $\left[0,t\right)$
\end{Def}


\begin{Note}
Para definir un proceso de renovaci\'on para cualquier contexto, solamente hay que especificar una distribuci\'on $F$, con $F\left(0\right)=0$, para los tiempos de inter-renovaci\'on. La funci\'on $F$ en turno degune las otra variables aleatorias. De manera formal, existe un espacio de probabilidad y una sucesi\'on de variables aleatorias $\xi_{1},\xi_{2},\ldots$ definidas en este con distribuci\'on $F$. Entonces las otras cantidades son $T_{n}=\sum_{k=1}^{n}\xi_{k}$ y $N\left(t\right)=\sum_{n=1}^{\infty}\indora\left(T_{n}\leq t\right)$, donde $T_{n}\rightarrow\infty$ casi seguramente por la Ley Fuerte de los Grandes N\'umeros.
\end{Note}


Los tiempos $T_{n}$ est\'an relacionados con los conteos de $N\left(t\right)$ por

\begin{eqnarray*}
\left\{N\left(t\right)\geq n\right\}&=&\left\{T_{n}\leq t\right\}\\
T_{N\left(t\right)}\leq &t&<T_{N\left(t\right)+1},
\end{eqnarray*}

adem\'as $N\left(T_{n}\right)=n$, y 

\begin{eqnarray*}
N\left(t\right)=\max\left\{n:T_{n}\leq t\right\}=\min\left\{n:T_{n+1}>t\right\}
\end{eqnarray*}

Por propiedades de la convoluci\'on se sabe que

\begin{eqnarray*}
P\left\{T_{n}\leq t\right\}=F^{n\star}\left(t\right)
\end{eqnarray*}
que es la $n$-\'esima convoluci\'on de $F$. Entonces 

\begin{eqnarray*}
\left\{N\left(t\right)\geq n\right\}&=&\left\{T_{n}\leq t\right\}\\
P\left\{N\left(t\right)\leq n\right\}&=&1-F^{\left(n+1\right)\star}\left(t\right)
\end{eqnarray*}

Adem\'as usando el hecho de que $\esp\left[N\left(t\right)\right]=\sum_{n=1}^{\infty}P\left\{N\left(t\right)\geq n\right\}$
se tiene que

\begin{eqnarray*}
\esp\left[N\left(t\right)\right]=\sum_{n=1}^{\infty}F^{n\star}\left(t\right)
\end{eqnarray*}

\begin{Prop}
Para cada $t\geq0$, la funci\'on generadora de momentos $\esp\left[e^{\alpha N\left(t\right)}\right]$ existe para alguna $\alpha$ en una vecindad del 0, y de aqu\'i que $\esp\left[N\left(t\right)^{m}\right]<\infty$, para $m\geq1$.
\end{Prop}


\begin{Note}
Si el primer tiempo de renovaci\'on $\xi_{1}$ no tiene la misma distribuci\'on que el resto de las $\xi_{n}$, para $n\geq2$, a $N\left(t\right)$ se le llama Proceso de Renovaci\'on retardado, donde si $\xi$ tiene distribuci\'on $G$, entonces el tiempo $T_{n}$ de la $n$-\'esima renovaci\'on tiene distribuci\'on $G\star F^{\left(n-1\right)\star}\left(t\right)$
\end{Note}


\begin{Teo}
Para una constante $\mu\leq\infty$ ( o variable aleatoria), las siguientes expresiones son equivalentes:

\begin{eqnarray}
lim_{n\rightarrow\infty}n^{-1}T_{n}&=&\mu,\textrm{ c.s.}\\
lim_{t\rightarrow\infty}t^{-1}N\left(t\right)&=&1/\mu,\textrm{ c.s.}
\end{eqnarray}
\end{Teo}


Es decir, $T_{n}$ satisface la Ley Fuerte de los Grandes N\'umeros s\'i y s\'olo s\'i $N\left/t\right)$ la cumple.


\begin{Coro}[Ley Fuerte de los Grandes N\'umeros para Procesos de Renovaci\'on]
Si $N\left(t\right)$ es un proceso de renovaci\'on cuyos tiempos de inter-renovaci\'on tienen media $\mu\leq\infty$, entonces
\begin{eqnarray}
t^{-1}N\left(t\right)\rightarrow 1/\mu,\textrm{ c.s. cuando }t\rightarrow\infty.
\end{eqnarray}

\end{Coro}


Considerar el proceso estoc\'astico de valores reales $\left\{Z\left(t\right):t\geq0\right\}$ en el mismo espacio de probabilidad que $N\left(t\right)$

\begin{Def}
Para el proceso $\left\{Z\left(t\right):t\geq0\right\}$ se define la fluctuaci\'on m\'axima de $Z\left(t\right)$ en el intervalo $\left(T_{n-1},T_{n}\right]$:
\begin{eqnarray*}
M_{n}=\sup_{T_{n-1}<t\leq T_{n}}|Z\left(t\right)-Z\left(T_{n-1}\right)|
\end{eqnarray*}
\end{Def}

\begin{Teo}
Sup\'ongase que $n^{-1}T_{n}\rightarrow\mu$ c.s. cuando $n\rightarrow\infty$, donde $\mu\leq\infty$ es una constante o variable aleatoria. Sea $a$ una constante o variable aleatoria que puede ser infinita cuando $\mu$ es finita, y considere las expresiones l\'imite:
\begin{eqnarray}
lim_{n\rightarrow\infty}n^{-1}Z\left(T_{n}\right)&=&a,\textrm{ c.s.}\\
lim_{t\rightarrow\infty}t^{-1}Z\left(t\right)&=&a/\mu,\textrm{ c.s.}
\end{eqnarray}
La segunda expresi\'on implica la primera. Conversamente, la primera implica la segunda si el proceso $Z\left(t\right)$ es creciente, o si $lim_{n\rightarrow\infty}n^{-1}M_{n}=0$ c.s.
\end{Teo}

\begin{Coro}
Si $N\left(t\right)$ es un proceso de renovaci\'on, y $\left(Z\left(T_{n}\right)-Z\left(T_{n-1}\right),M_{n}\right)$, para $n\geq1$, son variables aleatorias independientes e id\'enticamente distribuidas con media finita, entonces,
\begin{eqnarray}
lim_{t\rightarrow\infty}t^{-1}Z\left(t\right)\rightarrow\frac{\esp\left[Z\left(T_{1}\right)-Z\left(T_{0}\right)\right]}{\esp\left[T_{1}\right]},\textrm{ c.s. cuando  }t\rightarrow\infty.
\end{eqnarray}
\end{Coro}


Sup\'ongase que $N\left(t\right)$ es un proceso de renovaci\'on con distribuci\'on $F$ con media finita $\mu$.

\begin{Def}
La funci\'on de renovaci\'on asociada con la distribuci\'on $F$, del proceso $N\left(t\right)$, es
\begin{eqnarray*}
U\left(t\right)=\sum_{n=1}^{\infty}F^{n\star}\left(t\right),\textrm{   }t\geq0,
\end{eqnarray*}
donde $F^{0\star}\left(t\right)=\indora\left(t\geq0\right)$.
\end{Def}


\begin{Prop}
Sup\'ongase que la distribuci\'on de inter-renovaci\'on $F$ tiene densidad $f$. Entonces $U\left(t\right)$ tambi\'en tiene densidad, para $t>0$, y es $U^{'}\left(t\right)=\sum_{n=0}^{\infty}f^{n\star}\left(t\right)$. Adem\'as
\begin{eqnarray*}
\prob\left\{N\left(t\right)>N\left(t-\right)\right\}=0\textrm{,   }t\geq0.
\end{eqnarray*}
\end{Prop}

\begin{Def}
La Transformada de Laplace-Stieljes de $F$ est\'a dada por

\begin{eqnarray*}
\hat{F}\left(\alpha\right)=\int_{\rea_{+}}e^{-\alpha t}dF\left(t\right)\textrm{,  }\alpha\geq0.
\end{eqnarray*}
\end{Def}

Entonces

\begin{eqnarray*}
\hat{U}\left(\alpha\right)=\sum_{n=0}^{\infty}\hat{F^{n\star}}\left(\alpha\right)=\sum_{n=0}^{\infty}\hat{F}\left(\alpha\right)^{n}=\frac{1}{1-\hat{F}\left(\alpha\right)}.
\end{eqnarray*}


\begin{Prop}
La Transformada de Laplace $\hat{U}\left(\alpha\right)$ y $\hat{F}\left(\alpha\right)$ determina una a la otra de manera \'unica por la relaci\'on $\hat{U}\left(\alpha\right)=\frac{1}{1-\hat{F}\left(\alpha\right)}$.
\end{Prop}


\begin{Note}
Un proceso de renovaci\'on $N\left(t\right)$ cuyos tiempos de inter-renovaci\'on tienen media finita, es un proceso Poisson con tasa $\lambda$ si y s\'olo s\'i $\esp\left[U\left(t\right)\right]=\lambda t$, para $t\geq0$.
\end{Note}


\begin{Teo}
Sea $N\left(t\right)$ un proceso puntual simple con puntos de localizaci\'on $T_{n}$ tal que $\eta\left(t\right)=\esp\left[N\left(\right)\right]$ es finita para cada $t$. Entonces para cualquier funci\'on $f:\rea_{+}\rightarrow\rea$,
\begin{eqnarray*}
\esp\left[\sum_{n=1}^{N\left(\right)}f\left(T_{n}\right)\right]=\int_{\left(0,t\right]}f\left(s\right)d\eta\left(s\right)\textrm{,  }t\geq0,
\end{eqnarray*}
suponiendo que la integral exista. Adem\'as si $X_{1},X_{2},\ldots$ son variables aleatorias definidas en el mismo espacio de probabilidad que el proceso $N\left(t\right)$ tal que $\esp\left[X_{n}|T_{n}=s\right]=f\left(s\right)$, independiente de $n$. Entonces
\begin{eqnarray*}
\esp\left[\sum_{n=1}^{N\left(t\right)}X_{n}\right]=\int_{\left(0,t\right]}f\left(s\right)d\eta\left(s\right)\textrm{,  }t\geq0,
\end{eqnarray*} 
suponiendo que la integral exista. 
\end{Teo}

\begin{Coro}[Identidad de Wald para Renovaciones]
Para el proceso de renovaci\'on $N\left(t\right)$,
\begin{eqnarray*}
\esp\left[T_{N\left(t\right)+1}\right]=\mu\esp\left[N\left(t\right)+1\right]\textrm{,  }t\geq0,
\end{eqnarray*}  
\end{Coro}


\begin{Def}
Sea $h\left(t\right)$ funci\'on de valores reales en $\rea$ acotada en intervalos finitos e igual a cero para $t<0$ La ecuaci\'on de renovaci\'on para $h\left(t\right)$ y la distribuci\'on $F$ es

\begin{eqnarray}\label{Ec.Renovacion}
H\left(t\right)=h\left(t\right)+\int_{\left[0,t\right]}H\left(t-s\right)dF\left(s\right)\textrm{,    }t\geq0,
\end{eqnarray}
donde $H\left(t\right)$ es una funci\'on de valores reales. Esto es $H=h+F\star H$. Decimos que $H\left(t\right)$ es soluci\'on de esta ecuaci\'on si satisface la ecuaci\'on, y es acotada en intervalos finitos e iguales a cero para $t<0$.
\end{Def}

\begin{Prop}
La funci\'on $U\star h\left(t\right)$ es la \'unica soluci\'on de la ecuaci\'on de renovaci\'on (\ref{Ec.Renovacion}).
\end{Prop}

\begin{Teo}[Teorema Renovaci\'on Elemental]
\begin{eqnarray*}
t^{-1}U\left(t\right)\rightarrow 1/\mu\textrm{,    cuando }t\rightarrow\infty.
\end{eqnarray*}
\end{Teo}



Sup\'ongase que $N\left(t\right)$ es un proceso de renovaci\'on con distribuci\'on $F$ con media finita $\mu$.

\begin{Def}
La funci\'on de renovaci\'on asociada con la distribuci\'on $F$, del proceso $N\left(t\right)$, es
\begin{eqnarray*}
U\left(t\right)=\sum_{n=1}^{\infty}F^{n\star}\left(t\right),\textrm{   }t\geq0,
\end{eqnarray*}
donde $F^{0\star}\left(t\right)=\indora\left(t\geq0\right)$.
\end{Def}


\begin{Prop}
Sup\'ongase que la distribuci\'on de inter-renovaci\'on $F$ tiene densidad $f$. Entonces $U\left(t\right)$ tambi\'en tiene densidad, para $t>0$, y es $U^{'}\left(t\right)=\sum_{n=0}^{\infty}f^{n\star}\left(t\right)$. Adem\'as
\begin{eqnarray*}
\prob\left\{N\left(t\right)>N\left(t-\right)\right\}=0\textrm{,   }t\geq0.
\end{eqnarray*}
\end{Prop}

\begin{Def}
La Transformada de Laplace-Stieljes de $F$ est\'a dada por

\begin{eqnarray*}
\hat{F}\left(\alpha\right)=\int_{\rea_{+}}e^{-\alpha t}dF\left(t\right)\textrm{,  }\alpha\geq0.
\end{eqnarray*}
\end{Def}

Entonces

\begin{eqnarray*}
\hat{U}\left(\alpha\right)=\sum_{n=0}^{\infty}\hat{F^{n\star}}\left(\alpha\right)=\sum_{n=0}^{\infty}\hat{F}\left(\alpha\right)^{n}=\frac{1}{1-\hat{F}\left(\alpha\right)}.
\end{eqnarray*}


\begin{Prop}
La Transformada de Laplace $\hat{U}\left(\alpha\right)$ y $\hat{F}\left(\alpha\right)$ determina una a la otra de manera \'unica por la relaci\'on $\hat{U}\left(\alpha\right)=\frac{1}{1-\hat{F}\left(\alpha\right)}$.
\end{Prop}


\begin{Note}
Un proceso de renovaci\'on $N\left(t\right)$ cuyos tiempos de inter-renovaci\'on tienen media finita, es un proceso Poisson con tasa $\lambda$ si y s\'olo s\'i $\esp\left[U\left(t\right)\right]=\lambda t$, para $t\geq0$.
\end{Note}


\begin{Teo}
Sea $N\left(t\right)$ un proceso puntual simple con puntos de localizaci\'on $T_{n}$ tal que $\eta\left(t\right)=\esp\left[N\left(\right)\right]$ es finita para cada $t$. Entonces para cualquier funci\'on $f:\rea_{+}\rightarrow\rea$,
\begin{eqnarray*}
\esp\left[\sum_{n=1}^{N\left(\right)}f\left(T_{n}\right)\right]=\int_{\left(0,t\right]}f\left(s\right)d\eta\left(s\right)\textrm{,  }t\geq0,
\end{eqnarray*}
suponiendo que la integral exista. Adem\'as si $X_{1},X_{2},\ldots$ son variables aleatorias definidas en el mismo espacio de probabilidad que el proceso $N\left(t\right)$ tal que $\esp\left[X_{n}|T_{n}=s\right]=f\left(s\right)$, independiente de $n$. Entonces
\begin{eqnarray*}
\esp\left[\sum_{n=1}^{N\left(t\right)}X_{n}\right]=\int_{\left(0,t\right]}f\left(s\right)d\eta\left(s\right)\textrm{,  }t\geq0,
\end{eqnarray*} 
suponiendo que la integral exista. 
\end{Teo}

\begin{Coro}[Identidad de Wald para Renovaciones]
Para el proceso de renovaci\'on $N\left(t\right)$,
\begin{eqnarray*}
\esp\left[T_{N\left(t\right)+1}\right]=\mu\esp\left[N\left(t\right)+1\right]\textrm{,  }t\geq0,
\end{eqnarray*}  
\end{Coro}


\begin{Def}
Sea $h\left(t\right)$ funci\'on de valores reales en $\rea$ acotada en intervalos finitos e igual a cero para $t<0$ La ecuaci\'on de renovaci\'on para $h\left(t\right)$ y la distribuci\'on $F$ es

\begin{eqnarray}\label{Ec.Renovacion}
H\left(t\right)=h\left(t\right)+\int_{\left[0,t\right]}H\left(t-s\right)dF\left(s\right)\textrm{,    }t\geq0,
\end{eqnarray}
donde $H\left(t\right)$ es una funci\'on de valores reales. Esto es $H=h+F\star H$. Decimos que $H\left(t\right)$ es soluci\'on de esta ecuaci\'on si satisface la ecuaci\'on, y es acotada en intervalos finitos e iguales a cero para $t<0$.
\end{Def}

\begin{Prop}
La funci\'on $U\star h\left(t\right)$ es la \'unica soluci\'on de la ecuaci\'on de renovaci\'on (\ref{Ec.Renovacion}).
\end{Prop}

\begin{Teo}[Teorema Renovaci\'on Elemental]
\begin{eqnarray*}
t^{-1}U\left(t\right)\rightarrow 1/\mu\textrm{,    cuando }t\rightarrow\infty.
\end{eqnarray*}
\end{Teo}


\begin{Note} Una funci\'on $h:\rea_{+}\rightarrow\rea$ es Directamente Riemann Integrable en los siguientes casos:
\begin{itemize}
\item[a)] $h\left(t\right)\geq0$ es decreciente y Riemann Integrable.
\item[b)] $h$ es continua excepto posiblemente en un conjunto de Lebesgue de medida 0, y $|h\left(t\right)|\leq b\left(t\right)$, donde $b$ es DRI.
\end{itemize}
\end{Note}

\begin{Teo}[Teorema Principal de Renovaci\'on]
Si $F$ es no aritm\'etica y $h\left(t\right)$ es Directamente Riemann Integrable (DRI), entonces

\begin{eqnarray*}
lim_{t\rightarrow\infty}U\star h=\frac{1}{\mu}\int_{\rea_{+}}h\left(s\right)ds.
\end{eqnarray*}
\end{Teo}

\begin{Prop}
Cualquier funci\'on $H\left(t\right)$ acotada en intervalos finitos y que es 0 para $t<0$ puede expresarse como
\begin{eqnarray*}
H\left(t\right)=U\star h\left(t\right)\textrm{,  donde }h\left(t\right)=H\left(t\right)-F\star H\left(t\right)
\end{eqnarray*}
\end{Prop}

\begin{Def}
Un proceso estoc\'astico $X\left(t\right)$ es crudamente regenerativo en un tiempo aleatorio positivo $T$ si
\begin{eqnarray*}
\esp\left[X\left(T+t\right)|T\right]=\esp\left[X\left(t\right)\right]\textrm{, para }t\geq0,\end{eqnarray*}
y con las esperanzas anteriores finitas.
\end{Def}

\begin{Prop}
Sup\'ongase que $X\left(t\right)$ es un proceso crudamente regenerativo en $T$, que tiene distribuci\'on $F$. Si $\esp\left[X\left(t\right)\right]$ es acotado en intervalos finitos, entonces
\begin{eqnarray*}
\esp\left[X\left(t\right)\right]=U\star h\left(t\right)\textrm{,  donde }h\left(t\right)=\esp\left[X\left(t\right)\indora\left(T>t\right)\right].
\end{eqnarray*}
\end{Prop}

\begin{Teo}[Regeneraci\'on Cruda]
Sup\'ongase que $X\left(t\right)$ es un proceso con valores positivo crudamente regenerativo en $T$, y def\'inase $M=\sup\left\{|X\left(t\right)|:t\leq T\right\}$. Si $T$ es no aritm\'etico y $M$ y $MT$ tienen media finita, entonces
\begin{eqnarray*}
lim_{t\rightarrow\infty}\esp\left[X\left(t\right)\right]=\frac{1}{\mu}\int_{\rea_{+}}h\left(s\right)ds,
\end{eqnarray*}
donde $h\left(t\right)=\esp\left[X\left(t\right)\indora\left(T>t\right)\right]$.
\end{Teo}


\begin{Note} Una funci\'on $h:\rea_{+}\rightarrow\rea$ es Directamente Riemann Integrable en los siguientes casos:
\begin{itemize}
\item[a)] $h\left(t\right)\geq0$ es decreciente y Riemann Integrable.
\item[b)] $h$ es continua excepto posiblemente en un conjunto de Lebesgue de medida 0, y $|h\left(t\right)|\leq b\left(t\right)$, donde $b$ es DRI.
\end{itemize}
\end{Note}

\begin{Teo}[Teorema Principal de Renovaci\'on]
Si $F$ es no aritm\'etica y $h\left(t\right)$ es Directamente Riemann Integrable (DRI), entonces

\begin{eqnarray*}
lim_{t\rightarrow\infty}U\star h=\frac{1}{\mu}\int_{\rea_{+}}h\left(s\right)ds.
\end{eqnarray*}
\end{Teo}

\begin{Prop}
Cualquier funci\'on $H\left(t\right)$ acotada en intervalos finitos y que es 0 para $t<0$ puede expresarse como
\begin{eqnarray*}
H\left(t\right)=U\star h\left(t\right)\textrm{,  donde }h\left(t\right)=H\left(t\right)-F\star H\left(t\right)
\end{eqnarray*}
\end{Prop}

\begin{Def}
Un proceso estoc\'astico $X\left(t\right)$ es crudamente regenerativo en un tiempo aleatorio positivo $T$ si
\begin{eqnarray*}
\esp\left[X\left(T+t\right)|T\right]=\esp\left[X\left(t\right)\right]\textrm{, para }t\geq0,\end{eqnarray*}
y con las esperanzas anteriores finitas.
\end{Def}

\begin{Prop}
Sup\'ongase que $X\left(t\right)$ es un proceso crudamente regenerativo en $T$, que tiene distribuci\'on $F$. Si $\esp\left[X\left(t\right)\right]$ es acotado en intervalos finitos, entonces
\begin{eqnarray*}
\esp\left[X\left(t\right)\right]=U\star h\left(t\right)\textrm{,  donde }h\left(t\right)=\esp\left[X\left(t\right)\indora\left(T>t\right)\right].
\end{eqnarray*}
\end{Prop}

\begin{Teo}[Regeneraci\'on Cruda]
Sup\'ongase que $X\left(t\right)$ es un proceso con valores positivo crudamente regenerativo en $T$, y def\'inase $M=\sup\left\{|X\left(t\right)|:t\leq T\right\}$. Si $T$ es no aritm\'etico y $M$ y $MT$ tienen media finita, entonces
\begin{eqnarray*}
lim_{t\rightarrow\infty}\esp\left[X\left(t\right)\right]=\frac{1}{\mu}\int_{\rea_{+}}h\left(s\right)ds,
\end{eqnarray*}
donde $h\left(t\right)=\esp\left[X\left(t\right)\indora\left(T>t\right)\right]$.
\end{Teo}

%________________________________________________________________________
\subsection{Procesos Regenerativos}
%________________________________________________________________________

Para $\left\{X\left(t\right):t\geq0\right\}$ Proceso Estoc\'astico a tiempo continuo con estado de espacios $S$, que es un espacio m\'etrico, con trayectorias continuas por la derecha y con l\'imites por la izquierda c.s. Sea $N\left(t\right)$ un proceso de renovaci\'on en $\rea_{+}$ definido en el mismo espacio de probabilidad que $X\left(t\right)$, con tiempos de renovaci\'on $T$ y tiempos de inter-renovaci\'on $\xi_{n}=T_{n}-T_{n-1}$, con misma distribuci\'on $F$ de media finita $\mu$.



\begin{Def}
Para el proceso $\left\{\left(N\left(t\right),X\left(t\right)\right):t\geq0\right\}$, sus trayectoria muestrales en el intervalo de tiempo $\left[T_{n-1},T_{n}\right)$ est\'an descritas por
\begin{eqnarray*}
\zeta_{n}=\left(\xi_{n},\left\{X\left(T_{n-1}+t\right):0\leq t<\xi_{n}\right\}\right)
\end{eqnarray*}
Este $\zeta_{n}$ es el $n$-\'esimo segmento del proceso. El proceso es regenerativo sobre los tiempos $T_{n}$ si sus segmentos $\zeta_{n}$ son independientes e id\'enticamennte distribuidos.
\end{Def}


\begin{Obs}
Si $\tilde{X}\left(t\right)$ con espacio de estados $\tilde{S}$ es regenerativo sobre $T_{n}$, entonces $X\left(t\right)=f\left(\tilde{X}\left(t\right)\right)$ tambi\'en es regenerativo sobre $T_{n}$, para cualquier funci\'on $f:\tilde{S}\rightarrow S$.
\end{Obs}

\begin{Obs}
Los procesos regenerativos son crudamente regenerativos, pero no al rev\'es.
\end{Obs}

\begin{Def}[Definici\'on Cl\'asica]
Un proceso estoc\'astico $X=\left\{X\left(t\right):t\geq0\right\}$ es llamado regenerativo is existe una variable aleatoria $R_{1}>0$ tal que
\begin{itemize}
\item[i)] $\left\{X\left(t+R_{1}\right):t\geq0\right\}$ es independiente de $\left\{\left\{X\left(t\right):t<R_{1}\right\},\right\}$
\item[ii)] $\left\{X\left(t+R_{1}\right):t\geq0\right\}$ es estoc\'asticamente equivalente a $\left\{X\left(t\right):t>0\right\}$
\end{itemize}

Llamamos a $R_{1}$ tiempo de regeneraci\'on, y decimos que $X$ se regenera en este punto.
\end{Def}

$\left\{X\left(t+R_{1}\right)\right\}$ es regenerativo con tiempo de regeneraci\'on $R_{2}$, independiente de $R_{1}$ pero con la misma distribuci\'on que $R_{1}$. Procediendo de esta manera se obtiene una secuencia de variables aleatorias independientes e id\'enticamente distribuidas $\left\{R_{n}\right\}$ llamados longitudes de ciclo. Si definimos a $Z_{k}\equiv R_{1}+R_{2}+\cdots+R_{k}$, se tiene un proceso de renovaci\'on llamado proceso de renovaci\'on encajado para $X$.

\begin{Note}
Un proceso regenerativo con media de la longitud de ciclo finita es llamado positivo recurrente.
\end{Note}


\begin{Def}
Para $x$ fijo y para cada $t\geq0$, sea $I_{x}\left(t\right)=1$ si $X\left(t\right)\leq x$,  $I_{x}\left(t\right)=0$ en caso contrario, y def\'inanse los tiempos promedio
\begin{eqnarray*}
\overline{X}&=&lim_{t\rightarrow\infty}\frac{1}{t}\int_{0}^{\infty}X\left(u\right)du\\
\prob\left(X_{\infty}\leq x\right)&=&lim_{t\rightarrow\infty}\frac{1}{t}\int_{0}^{\infty}I_{x}\left(u\right)du,
\end{eqnarray*}
cuando estos l\'imites existan.
\end{Def}

Como consecuencia del teorema de Renovaci\'on-Recompensa, se tiene que el primer l\'imite  existe y es igual a la constante
\begin{eqnarray*}
\overline{X}&=&\frac{\esp\left[\int_{0}^{R_{1}}X\left(t\right)dt\right]}{\esp\left[R_{1}\right]},
\end{eqnarray*}
suponiendo que ambas esperanzas son finitas.

\begin{Note}
\begin{itemize}
\item[a)] Si el proceso regenerativo $X$ es positivo recurrente y tiene trayectorias muestrales no negativas, entonces la ecuaci\'on anterior es v\'alida.
\item[b)] Si $X$ es positivo recurrente regenerativo, podemos construir una \'unica versi\'on estacionaria de este proceso, $X_{e}=\left\{X_{e}\left(t\right)\right\}$, donde $X_{e}$ es un proceso estoc\'astico regenerativo y estrictamente estacionario, con distribuci\'on marginal distribuida como $X_{\infty}$
\end{itemize}
\end{Note}

%________________________________________________________________________
\subsection{Procesos Regenerativos}
%________________________________________________________________________

Para $\left\{X\left(t\right):t\geq0\right\}$ Proceso Estoc\'astico a tiempo continuo con estado de espacios $S$, que es un espacio m\'etrico, con trayectorias continuas por la derecha y con l\'imites por la izquierda c.s. Sea $N\left(t\right)$ un proceso de renovaci\'on en $\rea_{+}$ definido en el mismo espacio de probabilidad que $X\left(t\right)$, con tiempos de renovaci\'on $T$ y tiempos de inter-renovaci\'on $\xi_{n}=T_{n}-T_{n-1}$, con misma distribuci\'on $F$ de media finita $\mu$.



\begin{Def}
Para el proceso $\left\{\left(N\left(t\right),X\left(t\right)\right):t\geq0\right\}$, sus trayectoria muestrales en el intervalo de tiempo $\left[T_{n-1},T_{n}\right)$ est\'an descritas por
\begin{eqnarray*}
\zeta_{n}=\left(\xi_{n},\left\{X\left(T_{n-1}+t\right):0\leq t<\xi_{n}\right\}\right)
\end{eqnarray*}
Este $\zeta_{n}$ es el $n$-\'esimo segmento del proceso. El proceso es regenerativo sobre los tiempos $T_{n}$ si sus segmentos $\zeta_{n}$ son independientes e id\'enticamennte distribuidos.
\end{Def}


\begin{Obs}
Si $\tilde{X}\left(t\right)$ con espacio de estados $\tilde{S}$ es regenerativo sobre $T_{n}$, entonces $X\left(t\right)=f\left(\tilde{X}\left(t\right)\right)$ tambi\'en es regenerativo sobre $T_{n}$, para cualquier funci\'on $f:\tilde{S}\rightarrow S$.
\end{Obs}

\begin{Obs}
Los procesos regenerativos son crudamente regenerativos, pero no al rev\'es.
\end{Obs}

\begin{Def}[Definici\'on Cl\'asica]
Un proceso estoc\'astico $X=\left\{X\left(t\right):t\geq0\right\}$ es llamado regenerativo is existe una variable aleatoria $R_{1}>0$ tal que
\begin{itemize}
\item[i)] $\left\{X\left(t+R_{1}\right):t\geq0\right\}$ es independiente de $\left\{\left\{X\left(t\right):t<R_{1}\right\},\right\}$
\item[ii)] $\left\{X\left(t+R_{1}\right):t\geq0\right\}$ es estoc\'asticamente equivalente a $\left\{X\left(t\right):t>0\right\}$
\end{itemize}

Llamamos a $R_{1}$ tiempo de regeneraci\'on, y decimos que $X$ se regenera en este punto.
\end{Def}

$\left\{X\left(t+R_{1}\right)\right\}$ es regenerativo con tiempo de regeneraci\'on $R_{2}$, independiente de $R_{1}$ pero con la misma distribuci\'on que $R_{1}$. Procediendo de esta manera se obtiene una secuencia de variables aleatorias independientes e id\'enticamente distribuidas $\left\{R_{n}\right\}$ llamados longitudes de ciclo. Si definimos a $Z_{k}\equiv R_{1}+R_{2}+\cdots+R_{k}$, se tiene un proceso de renovaci\'on llamado proceso de renovaci\'on encajado para $X$.

\begin{Note}
Un proceso regenerativo con media de la longitud de ciclo finita es llamado positivo recurrente.
\end{Note}


\begin{Def}
Para $x$ fijo y para cada $t\geq0$, sea $I_{x}\left(t\right)=1$ si $X\left(t\right)\leq x$,  $I_{x}\left(t\right)=0$ en caso contrario, y def\'inanse los tiempos promedio
\begin{eqnarray*}
\overline{X}&=&lim_{t\rightarrow\infty}\frac{1}{t}\int_{0}^{\infty}X\left(u\right)du\\
\prob\left(X_{\infty}\leq x\right)&=&lim_{t\rightarrow\infty}\frac{1}{t}\int_{0}^{\infty}I_{x}\left(u\right)du,
\end{eqnarray*}
cuando estos l\'imites existan.
\end{Def}

Como consecuencia del teorema de Renovaci\'on-Recompensa, se tiene que el primer l\'imite  existe y es igual a la constante
\begin{eqnarray*}
\overline{X}&=&\frac{\esp\left[\int_{0}^{R_{1}}X\left(t\right)dt\right]}{\esp\left[R_{1}\right]},
\end{eqnarray*}
suponiendo que ambas esperanzas son finitas.

\begin{Note}
\begin{itemize}
\item[a)] Si el proceso regenerativo $X$ es positivo recurrente y tiene trayectorias muestrales no negativas, entonces la ecuaci\'on anterior es v\'alida.
\item[b)] Si $X$ es positivo recurrente regenerativo, podemos construir una \'unica versi\'on estacionaria de este proceso, $X_{e}=\left\{X_{e}\left(t\right)\right\}$, donde $X_{e}$ es un proceso estoc\'astico regenerativo y estrictamente estacionario, con distribuci\'on marginal distribuida como $X_{\infty}$
\end{itemize}
\end{Note}
%__________________________________________________________________________________________
\subsection{Procesos Regenerativos Estacionarios - Stidham \cite{Stidham}}
%__________________________________________________________________________________________


Un proceso estoc\'astico a tiempo continuo $\left\{V\left(t\right),t\geq0\right\}$ es un proceso regenerativo si existe una sucesi\'on de variables aleatorias independientes e id\'enticamente distribuidas $\left\{X_{1},X_{2},\ldots\right\}$, sucesi\'on de renovaci\'on, tal que para cualquier conjunto de Borel $A$, 

\begin{eqnarray*}
\prob\left\{V\left(t\right)\in A|X_{1}+X_{2}+\cdots+X_{R\left(t\right)}=s,\left\{V\left(\tau\right),\tau<s\right\}\right\}=\prob\left\{V\left(t-s\right)\in A|X_{1}>t-s\right\},
\end{eqnarray*}
para todo $0\leq s\leq t$, donde $R\left(t\right)=\max\left\{X_{1}+X_{2}+\cdots+X_{j}\leq t\right\}=$n\'umero de renovaciones ({\emph{puntos de regeneraci\'on}}) que ocurren en $\left[0,t\right]$. El intervalo $\left[0,X_{1}\right)$ es llamado {\emph{primer ciclo de regeneraci\'on}} de $\left\{V\left(t \right),t\geq0\right\}$, $\left[X_{1},X_{1}+X_{2}\right)$ el {\emph{segundo ciclo de regeneraci\'on}}, y as\'i sucesivamente.

Sea $X=X_{1}$ y sea $F$ la funci\'on de distrbuci\'on de $X$


\begin{Def}
Se define el proceso estacionario, $\left\{V^{*}\left(t\right),t\geq0\right\}$, para $\left\{V\left(t\right),t\geq0\right\}$ por

\begin{eqnarray*}
\prob\left\{V\left(t\right)\in A\right\}=\frac{1}{\esp\left[X\right]}\int_{0}^{\infty}\prob\left\{V\left(t+x\right)\in A|X>x\right\}\left(1-F\left(x\right)\right)dx,
\end{eqnarray*} 
para todo $t\geq0$ y todo conjunto de Borel $A$.
\end{Def}

\begin{Def}
Una distribuci\'on se dice que es {\emph{aritm\'etica}} si todos sus puntos de incremento son m\'ultiplos de la forma $0,\lambda, 2\lambda,\ldots$ para alguna $\lambda>0$ entera.
\end{Def}


\begin{Def}
Una modificaci\'on medible de un proceso $\left\{V\left(t\right),t\geq0\right\}$, es una versi\'on de este, $\left\{V\left(t,w\right)\right\}$ conjuntamente medible para $t\geq0$ y para $w\in S$, $S$ espacio de estados para $\left\{V\left(t\right),t\geq0\right\}$.
\end{Def}

\begin{Teo}
Sea $\left\{V\left(t\right),t\geq\right\}$ un proceso regenerativo no negativo con modificaci\'on medible. Sea $\esp\left[X\right]<\infty$. Entonces el proceso estacionario dado por la ecuaci\'on anterior est\'a bien definido y tiene funci\'on de distribuci\'on independiente de $t$, adem\'as
\begin{itemize}
\item[i)] \begin{eqnarray*}
\esp\left[V^{*}\left(0\right)\right]&=&\frac{\esp\left[\int_{0}^{X}V\left(s\right)ds\right]}{\esp\left[X\right]}\end{eqnarray*}
\item[ii)] Si $\esp\left[V^{*}\left(0\right)\right]<\infty$, equivalentemente, si $\esp\left[\int_{0}^{X}V\left(s\right)ds\right]<\infty$,entonces
\begin{eqnarray*}
\frac{\int_{0}^{t}V\left(s\right)ds}{t}\rightarrow\frac{\esp\left[\int_{0}^{X}V\left(s\right)ds\right]}{\esp\left[X\right]}
\end{eqnarray*}
con probabilidad 1 y en media, cuando $t\rightarrow\infty$.
\end{itemize}
\end{Teo}


%__________________________________________________________________________________________
\subsection{Procesos Regenerativos Estacionarios - Stidham \cite{Stidham}}
%__________________________________________________________________________________________


Un proceso estoc\'astico a tiempo continuo $\left\{V\left(t\right),t\geq0\right\}$ es un proceso regenerativo si existe una sucesi\'on de variables aleatorias independientes e id\'enticamente distribuidas $\left\{X_{1},X_{2},\ldots\right\}$, sucesi\'on de renovaci\'on, tal que para cualquier conjunto de Borel $A$, 

\begin{eqnarray*}
\prob\left\{V\left(t\right)\in A|X_{1}+X_{2}+\cdots+X_{R\left(t\right)}=s,\left\{V\left(\tau\right),\tau<s\right\}\right\}=\prob\left\{V\left(t-s\right)\in A|X_{1}>t-s\right\},
\end{eqnarray*}
para todo $0\leq s\leq t$, donde $R\left(t\right)=\max\left\{X_{1}+X_{2}+\cdots+X_{j}\leq t\right\}=$n\'umero de renovaciones ({\emph{puntos de regeneraci\'on}}) que ocurren en $\left[0,t\right]$. El intervalo $\left[0,X_{1}\right)$ es llamado {\emph{primer ciclo de regeneraci\'on}} de $\left\{V\left(t \right),t\geq0\right\}$, $\left[X_{1},X_{1}+X_{2}\right)$ el {\emph{segundo ciclo de regeneraci\'on}}, y as\'i sucesivamente.

Sea $X=X_{1}$ y sea $F$ la funci\'on de distrbuci\'on de $X$


\begin{Def}
Se define el proceso estacionario, $\left\{V^{*}\left(t\right),t\geq0\right\}$, para $\left\{V\left(t\right),t\geq0\right\}$ por

\begin{eqnarray*}
\prob\left\{V\left(t\right)\in A\right\}=\frac{1}{\esp\left[X\right]}\int_{0}^{\infty}\prob\left\{V\left(t+x\right)\in A|X>x\right\}\left(1-F\left(x\right)\right)dx,
\end{eqnarray*} 
para todo $t\geq0$ y todo conjunto de Borel $A$.
\end{Def}

\begin{Def}
Una distribuci\'on se dice que es {\emph{aritm\'etica}} si todos sus puntos de incremento son m\'ultiplos de la forma $0,\lambda, 2\lambda,\ldots$ para alguna $\lambda>0$ entera.
\end{Def}


\begin{Def}
Una modificaci\'on medible de un proceso $\left\{V\left(t\right),t\geq0\right\}$, es una versi\'on de este, $\left\{V\left(t,w\right)\right\}$ conjuntamente medible para $t\geq0$ y para $w\in S$, $S$ espacio de estados para $\left\{V\left(t\right),t\geq0\right\}$.
\end{Def}

\begin{Teo}
Sea $\left\{V\left(t\right),t\geq\right\}$ un proceso regenerativo no negativo con modificaci\'on medible. Sea $\esp\left[X\right]<\infty$. Entonces el proceso estacionario dado por la ecuaci\'on anterior est\'a bien definido y tiene funci\'on de distribuci\'on independiente de $t$, adem\'as
\begin{itemize}
\item[i)] \begin{eqnarray*}
\esp\left[V^{*}\left(0\right)\right]&=&\frac{\esp\left[\int_{0}^{X}V\left(s\right)ds\right]}{\esp\left[X\right]}\end{eqnarray*}
\item[ii)] Si $\esp\left[V^{*}\left(0\right)\right]<\infty$, equivalentemente, si $\esp\left[\int_{0}^{X}V\left(s\right)ds\right]<\infty$,entonces
\begin{eqnarray*}
\frac{\int_{0}^{t}V\left(s\right)ds}{t}\rightarrow\frac{\esp\left[\int_{0}^{X}V\left(s\right)ds\right]}{\esp\left[X\right]}
\end{eqnarray*}
con probabilidad 1 y en media, cuando $t\rightarrow\infty$.
\end{itemize}
\end{Teo}
%___________________________________________________________________________________________
%
\subsection{Propiedades de los Procesos de Renovaci\'on}
%___________________________________________________________________________________________
%

Los tiempos $T_{n}$ est\'an relacionados con los conteos de $N\left(t\right)$ por

\begin{eqnarray*}
\left\{N\left(t\right)\geq n\right\}&=&\left\{T_{n}\leq t\right\}\\
T_{N\left(t\right)}\leq &t&<T_{N\left(t\right)+1},
\end{eqnarray*}

adem\'as $N\left(T_{n}\right)=n$, y 

\begin{eqnarray*}
N\left(t\right)=\max\left\{n:T_{n}\leq t\right\}=\min\left\{n:T_{n+1}>t\right\}
\end{eqnarray*}

Por propiedades de la convoluci\'on se sabe que

\begin{eqnarray*}
P\left\{T_{n}\leq t\right\}=F^{n\star}\left(t\right)
\end{eqnarray*}
que es la $n$-\'esima convoluci\'on de $F$. Entonces 

\begin{eqnarray*}
\left\{N\left(t\right)\geq n\right\}&=&\left\{T_{n}\leq t\right\}\\
P\left\{N\left(t\right)\leq n\right\}&=&1-F^{\left(n+1\right)\star}\left(t\right)
\end{eqnarray*}

Adem\'as usando el hecho de que $\esp\left[N\left(t\right)\right]=\sum_{n=1}^{\infty}P\left\{N\left(t\right)\geq n\right\}$
se tiene que

\begin{eqnarray*}
\esp\left[N\left(t\right)\right]=\sum_{n=1}^{\infty}F^{n\star}\left(t\right)
\end{eqnarray*}

\begin{Prop}
Para cada $t\geq0$, la funci\'on generadora de momentos $\esp\left[e^{\alpha N\left(t\right)}\right]$ existe para alguna $\alpha$ en una vecindad del 0, y de aqu\'i que $\esp\left[N\left(t\right)^{m}\right]<\infty$, para $m\geq1$.
\end{Prop}


\begin{Note}
Si el primer tiempo de renovaci\'on $\xi_{1}$ no tiene la misma distribuci\'on que el resto de las $\xi_{n}$, para $n\geq2$, a $N\left(t\right)$ se le llama Proceso de Renovaci\'on retardado, donde si $\xi$ tiene distribuci\'on $G$, entonces el tiempo $T_{n}$ de la $n$-\'esima renovaci\'on tiene distribuci\'on $G\star F^{\left(n-1\right)\star}\left(t\right)$
\end{Note}


\begin{Teo}
Para una constante $\mu\leq\infty$ ( o variable aleatoria), las siguientes expresiones son equivalentes:

\begin{eqnarray}
lim_{n\rightarrow\infty}n^{-1}T_{n}&=&\mu,\textrm{ c.s.}\\
lim_{t\rightarrow\infty}t^{-1}N\left(t\right)&=&1/\mu,\textrm{ c.s.}
\end{eqnarray}
\end{Teo}


Es decir, $T_{n}$ satisface la Ley Fuerte de los Grandes N\'umeros s\'i y s\'olo s\'i $N\left/t\right)$ la cumple.


\begin{Coro}[Ley Fuerte de los Grandes N\'umeros para Procesos de Renovaci\'on]
Si $N\left(t\right)$ es un proceso de renovaci\'on cuyos tiempos de inter-renovaci\'on tienen media $\mu\leq\infty$, entonces
\begin{eqnarray}
t^{-1}N\left(t\right)\rightarrow 1/\mu,\textrm{ c.s. cuando }t\rightarrow\infty.
\end{eqnarray}

\end{Coro}


Considerar el proceso estoc\'astico de valores reales $\left\{Z\left(t\right):t\geq0\right\}$ en el mismo espacio de probabilidad que $N\left(t\right)$

\begin{Def}
Para el proceso $\left\{Z\left(t\right):t\geq0\right\}$ se define la fluctuaci\'on m\'axima de $Z\left(t\right)$ en el intervalo $\left(T_{n-1},T_{n}\right]$:
\begin{eqnarray*}
M_{n}=\sup_{T_{n-1}<t\leq T_{n}}|Z\left(t\right)-Z\left(T_{n-1}\right)|
\end{eqnarray*}
\end{Def}

\begin{Teo}
Sup\'ongase que $n^{-1}T_{n}\rightarrow\mu$ c.s. cuando $n\rightarrow\infty$, donde $\mu\leq\infty$ es una constante o variable aleatoria. Sea $a$ una constante o variable aleatoria que puede ser infinita cuando $\mu$ es finita, y considere las expresiones l\'imite:
\begin{eqnarray}
lim_{n\rightarrow\infty}n^{-1}Z\left(T_{n}\right)&=&a,\textrm{ c.s.}\\
lim_{t\rightarrow\infty}t^{-1}Z\left(t\right)&=&a/\mu,\textrm{ c.s.}
\end{eqnarray}
La segunda expresi\'on implica la primera. Conversamente, la primera implica la segunda si el proceso $Z\left(t\right)$ es creciente, o si $lim_{n\rightarrow\infty}n^{-1}M_{n}=0$ c.s.
\end{Teo}

\begin{Coro}
Si $N\left(t\right)$ es un proceso de renovaci\'on, y $\left(Z\left(T_{n}\right)-Z\left(T_{n-1}\right),M_{n}\right)$, para $n\geq1$, son variables aleatorias independientes e id\'enticamente distribuidas con media finita, entonces,
\begin{eqnarray}
lim_{t\rightarrow\infty}t^{-1}Z\left(t\right)\rightarrow\frac{\esp\left[Z\left(T_{1}\right)-Z\left(T_{0}\right)\right]}{\esp\left[T_{1}\right]},\textrm{ c.s. cuando  }t\rightarrow\infty.
\end{eqnarray}
\end{Coro}


%___________________________________________________________________________________________
%
\subsection{Propiedades de los Procesos de Renovaci\'on}
%___________________________________________________________________________________________
%

Los tiempos $T_{n}$ est\'an relacionados con los conteos de $N\left(t\right)$ por

\begin{eqnarray*}
\left\{N\left(t\right)\geq n\right\}&=&\left\{T_{n}\leq t\right\}\\
T_{N\left(t\right)}\leq &t&<T_{N\left(t\right)+1},
\end{eqnarray*}

adem\'as $N\left(T_{n}\right)=n$, y 

\begin{eqnarray*}
N\left(t\right)=\max\left\{n:T_{n}\leq t\right\}=\min\left\{n:T_{n+1}>t\right\}
\end{eqnarray*}

Por propiedades de la convoluci\'on se sabe que

\begin{eqnarray*}
P\left\{T_{n}\leq t\right\}=F^{n\star}\left(t\right)
\end{eqnarray*}
que es la $n$-\'esima convoluci\'on de $F$. Entonces 

\begin{eqnarray*}
\left\{N\left(t\right)\geq n\right\}&=&\left\{T_{n}\leq t\right\}\\
P\left\{N\left(t\right)\leq n\right\}&=&1-F^{\left(n+1\right)\star}\left(t\right)
\end{eqnarray*}

Adem\'as usando el hecho de que $\esp\left[N\left(t\right)\right]=\sum_{n=1}^{\infty}P\left\{N\left(t\right)\geq n\right\}$
se tiene que

\begin{eqnarray*}
\esp\left[N\left(t\right)\right]=\sum_{n=1}^{\infty}F^{n\star}\left(t\right)
\end{eqnarray*}

\begin{Prop}
Para cada $t\geq0$, la funci\'on generadora de momentos $\esp\left[e^{\alpha N\left(t\right)}\right]$ existe para alguna $\alpha$ en una vecindad del 0, y de aqu\'i que $\esp\left[N\left(t\right)^{m}\right]<\infty$, para $m\geq1$.
\end{Prop}


\begin{Note}
Si el primer tiempo de renovaci\'on $\xi_{1}$ no tiene la misma distribuci\'on que el resto de las $\xi_{n}$, para $n\geq2$, a $N\left(t\right)$ se le llama Proceso de Renovaci\'on retardado, donde si $\xi$ tiene distribuci\'on $G$, entonces el tiempo $T_{n}$ de la $n$-\'esima renovaci\'on tiene distribuci\'on $G\star F^{\left(n-1\right)\star}\left(t\right)$
\end{Note}


\begin{Teo}
Para una constante $\mu\leq\infty$ ( o variable aleatoria), las siguientes expresiones son equivalentes:

\begin{eqnarray}
lim_{n\rightarrow\infty}n^{-1}T_{n}&=&\mu,\textrm{ c.s.}\\
lim_{t\rightarrow\infty}t^{-1}N\left(t\right)&=&1/\mu,\textrm{ c.s.}
\end{eqnarray}
\end{Teo}


Es decir, $T_{n}$ satisface la Ley Fuerte de los Grandes N\'umeros s\'i y s\'olo s\'i $N\left/t\right)$ la cumple.


\begin{Coro}[Ley Fuerte de los Grandes N\'umeros para Procesos de Renovaci\'on]
Si $N\left(t\right)$ es un proceso de renovaci\'on cuyos tiempos de inter-renovaci\'on tienen media $\mu\leq\infty$, entonces
\begin{eqnarray}
t^{-1}N\left(t\right)\rightarrow 1/\mu,\textrm{ c.s. cuando }t\rightarrow\infty.
\end{eqnarray}

\end{Coro}


Considerar el proceso estoc\'astico de valores reales $\left\{Z\left(t\right):t\geq0\right\}$ en el mismo espacio de probabilidad que $N\left(t\right)$

\begin{Def}
Para el proceso $\left\{Z\left(t\right):t\geq0\right\}$ se define la fluctuaci\'on m\'axima de $Z\left(t\right)$ en el intervalo $\left(T_{n-1},T_{n}\right]$:
\begin{eqnarray*}
M_{n}=\sup_{T_{n-1}<t\leq T_{n}}|Z\left(t\right)-Z\left(T_{n-1}\right)|
\end{eqnarray*}
\end{Def}

\begin{Teo}
Sup\'ongase que $n^{-1}T_{n}\rightarrow\mu$ c.s. cuando $n\rightarrow\infty$, donde $\mu\leq\infty$ es una constante o variable aleatoria. Sea $a$ una constante o variable aleatoria que puede ser infinita cuando $\mu$ es finita, y considere las expresiones l\'imite:
\begin{eqnarray}
lim_{n\rightarrow\infty}n^{-1}Z\left(T_{n}\right)&=&a,\textrm{ c.s.}\\
lim_{t\rightarrow\infty}t^{-1}Z\left(t\right)&=&a/\mu,\textrm{ c.s.}
\end{eqnarray}
La segunda expresi\'on implica la primera. Conversamente, la primera implica la segunda si el proceso $Z\left(t\right)$ es creciente, o si $lim_{n\rightarrow\infty}n^{-1}M_{n}=0$ c.s.
\end{Teo}

\begin{Coro}
Si $N\left(t\right)$ es un proceso de renovaci\'on, y $\left(Z\left(T_{n}\right)-Z\left(T_{n-1}\right),M_{n}\right)$, para $n\geq1$, son variables aleatorias independientes e id\'enticamente distribuidas con media finita, entonces,
\begin{eqnarray}
lim_{t\rightarrow\infty}t^{-1}Z\left(t\right)\rightarrow\frac{\esp\left[Z\left(T_{1}\right)-Z\left(T_{0}\right)\right]}{\esp\left[T_{1}\right]},\textrm{ c.s. cuando  }t\rightarrow\infty.
\end{eqnarray}
\end{Coro}

%___________________________________________________________________________________________
%
\subsection{Propiedades de los Procesos de Renovaci\'on}
%___________________________________________________________________________________________
%

Los tiempos $T_{n}$ est\'an relacionados con los conteos de $N\left(t\right)$ por

\begin{eqnarray*}
\left\{N\left(t\right)\geq n\right\}&=&\left\{T_{n}\leq t\right\}\\
T_{N\left(t\right)}\leq &t&<T_{N\left(t\right)+1},
\end{eqnarray*}

adem\'as $N\left(T_{n}\right)=n$, y 

\begin{eqnarray*}
N\left(t\right)=\max\left\{n:T_{n}\leq t\right\}=\min\left\{n:T_{n+1}>t\right\}
\end{eqnarray*}

Por propiedades de la convoluci\'on se sabe que

\begin{eqnarray*}
P\left\{T_{n}\leq t\right\}=F^{n\star}\left(t\right)
\end{eqnarray*}
que es la $n$-\'esima convoluci\'on de $F$. Entonces 

\begin{eqnarray*}
\left\{N\left(t\right)\geq n\right\}&=&\left\{T_{n}\leq t\right\}\\
P\left\{N\left(t\right)\leq n\right\}&=&1-F^{\left(n+1\right)\star}\left(t\right)
\end{eqnarray*}

Adem\'as usando el hecho de que $\esp\left[N\left(t\right)\right]=\sum_{n=1}^{\infty}P\left\{N\left(t\right)\geq n\right\}$
se tiene que

\begin{eqnarray*}
\esp\left[N\left(t\right)\right]=\sum_{n=1}^{\infty}F^{n\star}\left(t\right)
\end{eqnarray*}

\begin{Prop}
Para cada $t\geq0$, la funci\'on generadora de momentos $\esp\left[e^{\alpha N\left(t\right)}\right]$ existe para alguna $\alpha$ en una vecindad del 0, y de aqu\'i que $\esp\left[N\left(t\right)^{m}\right]<\infty$, para $m\geq1$.
\end{Prop}


\begin{Note}
Si el primer tiempo de renovaci\'on $\xi_{1}$ no tiene la misma distribuci\'on que el resto de las $\xi_{n}$, para $n\geq2$, a $N\left(t\right)$ se le llama Proceso de Renovaci\'on retardado, donde si $\xi$ tiene distribuci\'on $G$, entonces el tiempo $T_{n}$ de la $n$-\'esima renovaci\'on tiene distribuci\'on $G\star F^{\left(n-1\right)\star}\left(t\right)$
\end{Note}


\begin{Teo}
Para una constante $\mu\leq\infty$ ( o variable aleatoria), las siguientes expresiones son equivalentes:

\begin{eqnarray}
lim_{n\rightarrow\infty}n^{-1}T_{n}&=&\mu,\textrm{ c.s.}\\
lim_{t\rightarrow\infty}t^{-1}N\left(t\right)&=&1/\mu,\textrm{ c.s.}
\end{eqnarray}
\end{Teo}


Es decir, $T_{n}$ satisface la Ley Fuerte de los Grandes N\'umeros s\'i y s\'olo s\'i $N\left/t\right)$ la cumple.


\begin{Coro}[Ley Fuerte de los Grandes N\'umeros para Procesos de Renovaci\'on]
Si $N\left(t\right)$ es un proceso de renovaci\'on cuyos tiempos de inter-renovaci\'on tienen media $\mu\leq\infty$, entonces
\begin{eqnarray}
t^{-1}N\left(t\right)\rightarrow 1/\mu,\textrm{ c.s. cuando }t\rightarrow\infty.
\end{eqnarray}

\end{Coro}


Considerar el proceso estoc\'astico de valores reales $\left\{Z\left(t\right):t\geq0\right\}$ en el mismo espacio de probabilidad que $N\left(t\right)$

\begin{Def}
Para el proceso $\left\{Z\left(t\right):t\geq0\right\}$ se define la fluctuaci\'on m\'axima de $Z\left(t\right)$ en el intervalo $\left(T_{n-1},T_{n}\right]$:
\begin{eqnarray*}
M_{n}=\sup_{T_{n-1}<t\leq T_{n}}|Z\left(t\right)-Z\left(T_{n-1}\right)|
\end{eqnarray*}
\end{Def}

\begin{Teo}
Sup\'ongase que $n^{-1}T_{n}\rightarrow\mu$ c.s. cuando $n\rightarrow\infty$, donde $\mu\leq\infty$ es una constante o variable aleatoria. Sea $a$ una constante o variable aleatoria que puede ser infinita cuando $\mu$ es finita, y considere las expresiones l\'imite:
\begin{eqnarray}
lim_{n\rightarrow\infty}n^{-1}Z\left(T_{n}\right)&=&a,\textrm{ c.s.}\\
lim_{t\rightarrow\infty}t^{-1}Z\left(t\right)&=&a/\mu,\textrm{ c.s.}
\end{eqnarray}
La segunda expresi\'on implica la primera. Conversamente, la primera implica la segunda si el proceso $Z\left(t\right)$ es creciente, o si $lim_{n\rightarrow\infty}n^{-1}M_{n}=0$ c.s.
\end{Teo}

\begin{Coro}
Si $N\left(t\right)$ es un proceso de renovaci\'on, y $\left(Z\left(T_{n}\right)-Z\left(T_{n-1}\right),M_{n}\right)$, para $n\geq1$, son variables aleatorias independientes e id\'enticamente distribuidas con media finita, entonces,
\begin{eqnarray}
lim_{t\rightarrow\infty}t^{-1}Z\left(t\right)\rightarrow\frac{\esp\left[Z\left(T_{1}\right)-Z\left(T_{0}\right)\right]}{\esp\left[T_{1}\right]},\textrm{ c.s. cuando  }t\rightarrow\infty.
\end{eqnarray}
\end{Coro}

%___________________________________________________________________________________________
%
\subsection{Propiedades de los Procesos de Renovaci\'on}
%___________________________________________________________________________________________
%

Los tiempos $T_{n}$ est\'an relacionados con los conteos de $N\left(t\right)$ por

\begin{eqnarray*}
\left\{N\left(t\right)\geq n\right\}&=&\left\{T_{n}\leq t\right\}\\
T_{N\left(t\right)}\leq &t&<T_{N\left(t\right)+1},
\end{eqnarray*}

adem\'as $N\left(T_{n}\right)=n$, y 

\begin{eqnarray*}
N\left(t\right)=\max\left\{n:T_{n}\leq t\right\}=\min\left\{n:T_{n+1}>t\right\}
\end{eqnarray*}

Por propiedades de la convoluci\'on se sabe que

\begin{eqnarray*}
P\left\{T_{n}\leq t\right\}=F^{n\star}\left(t\right)
\end{eqnarray*}
que es la $n$-\'esima convoluci\'on de $F$. Entonces 

\begin{eqnarray*}
\left\{N\left(t\right)\geq n\right\}&=&\left\{T_{n}\leq t\right\}\\
P\left\{N\left(t\right)\leq n\right\}&=&1-F^{\left(n+1\right)\star}\left(t\right)
\end{eqnarray*}

Adem\'as usando el hecho de que $\esp\left[N\left(t\right)\right]=\sum_{n=1}^{\infty}P\left\{N\left(t\right)\geq n\right\}$
se tiene que

\begin{eqnarray*}
\esp\left[N\left(t\right)\right]=\sum_{n=1}^{\infty}F^{n\star}\left(t\right)
\end{eqnarray*}

\begin{Prop}
Para cada $t\geq0$, la funci\'on generadora de momentos $\esp\left[e^{\alpha N\left(t\right)}\right]$ existe para alguna $\alpha$ en una vecindad del 0, y de aqu\'i que $\esp\left[N\left(t\right)^{m}\right]<\infty$, para $m\geq1$.
\end{Prop}


\begin{Note}
Si el primer tiempo de renovaci\'on $\xi_{1}$ no tiene la misma distribuci\'on que el resto de las $\xi_{n}$, para $n\geq2$, a $N\left(t\right)$ se le llama Proceso de Renovaci\'on retardado, donde si $\xi$ tiene distribuci\'on $G$, entonces el tiempo $T_{n}$ de la $n$-\'esima renovaci\'on tiene distribuci\'on $G\star F^{\left(n-1\right)\star}\left(t\right)$
\end{Note}


\begin{Teo}
Para una constante $\mu\leq\infty$ ( o variable aleatoria), las siguientes expresiones son equivalentes:

\begin{eqnarray}
lim_{n\rightarrow\infty}n^{-1}T_{n}&=&\mu,\textrm{ c.s.}\\
lim_{t\rightarrow\infty}t^{-1}N\left(t\right)&=&1/\mu,\textrm{ c.s.}
\end{eqnarray}
\end{Teo}


Es decir, $T_{n}$ satisface la Ley Fuerte de los Grandes N\'umeros s\'i y s\'olo s\'i $N\left/t\right)$ la cumple.


\begin{Coro}[Ley Fuerte de los Grandes N\'umeros para Procesos de Renovaci\'on]
Si $N\left(t\right)$ es un proceso de renovaci\'on cuyos tiempos de inter-renovaci\'on tienen media $\mu\leq\infty$, entonces
\begin{eqnarray}
t^{-1}N\left(t\right)\rightarrow 1/\mu,\textrm{ c.s. cuando }t\rightarrow\infty.
\end{eqnarray}

\end{Coro}


Considerar el proceso estoc\'astico de valores reales $\left\{Z\left(t\right):t\geq0\right\}$ en el mismo espacio de probabilidad que $N\left(t\right)$

\begin{Def}
Para el proceso $\left\{Z\left(t\right):t\geq0\right\}$ se define la fluctuaci\'on m\'axima de $Z\left(t\right)$ en el intervalo $\left(T_{n-1},T_{n}\right]$:
\begin{eqnarray*}
M_{n}=\sup_{T_{n-1}<t\leq T_{n}}|Z\left(t\right)-Z\left(T_{n-1}\right)|
\end{eqnarray*}
\end{Def}

\begin{Teo}
Sup\'ongase que $n^{-1}T_{n}\rightarrow\mu$ c.s. cuando $n\rightarrow\infty$, donde $\mu\leq\infty$ es una constante o variable aleatoria. Sea $a$ una constante o variable aleatoria que puede ser infinita cuando $\mu$ es finita, y considere las expresiones l\'imite:
\begin{eqnarray}
lim_{n\rightarrow\infty}n^{-1}Z\left(T_{n}\right)&=&a,\textrm{ c.s.}\\
lim_{t\rightarrow\infty}t^{-1}Z\left(t\right)&=&a/\mu,\textrm{ c.s.}
\end{eqnarray}
La segunda expresi\'on implica la primera. Conversamente, la primera implica la segunda si el proceso $Z\left(t\right)$ es creciente, o si $lim_{n\rightarrow\infty}n^{-1}M_{n}=0$ c.s.
\end{Teo}

\begin{Coro}
Si $N\left(t\right)$ es un proceso de renovaci\'on, y $\left(Z\left(T_{n}\right)-Z\left(T_{n-1}\right),M_{n}\right)$, para $n\geq1$, son variables aleatorias independientes e id\'enticamente distribuidas con media finita, entonces,
\begin{eqnarray}
lim_{t\rightarrow\infty}t^{-1}Z\left(t\right)\rightarrow\frac{\esp\left[Z\left(T_{1}\right)-Z\left(T_{0}\right)\right]}{\esp\left[T_{1}\right]},\textrm{ c.s. cuando  }t\rightarrow\infty.
\end{eqnarray}
\end{Coro}


%__________________________________________________________________________________________
\subsection{Procesos Regenerativos Estacionarios - Stidham \cite{Stidham}}
%__________________________________________________________________________________________


Un proceso estoc\'astico a tiempo continuo $\left\{V\left(t\right),t\geq0\right\}$ es un proceso regenerativo si existe una sucesi\'on de variables aleatorias independientes e id\'enticamente distribuidas $\left\{X_{1},X_{2},\ldots\right\}$, sucesi\'on de renovaci\'on, tal que para cualquier conjunto de Borel $A$, 

\begin{eqnarray*}
\prob\left\{V\left(t\right)\in A|X_{1}+X_{2}+\cdots+X_{R\left(t\right)}=s,\left\{V\left(\tau\right),\tau<s\right\}\right\}=\prob\left\{V\left(t-s\right)\in A|X_{1}>t-s\right\},
\end{eqnarray*}
para todo $0\leq s\leq t$, donde $R\left(t\right)=\max\left\{X_{1}+X_{2}+\cdots+X_{j}\leq t\right\}=$n\'umero de renovaciones ({\emph{puntos de regeneraci\'on}}) que ocurren en $\left[0,t\right]$. El intervalo $\left[0,X_{1}\right)$ es llamado {\emph{primer ciclo de regeneraci\'on}} de $\left\{V\left(t \right),t\geq0\right\}$, $\left[X_{1},X_{1}+X_{2}\right)$ el {\emph{segundo ciclo de regeneraci\'on}}, y as\'i sucesivamente.

Sea $X=X_{1}$ y sea $F$ la funci\'on de distrbuci\'on de $X$


\begin{Def}
Se define el proceso estacionario, $\left\{V^{*}\left(t\right),t\geq0\right\}$, para $\left\{V\left(t\right),t\geq0\right\}$ por

\begin{eqnarray*}
\prob\left\{V\left(t\right)\in A\right\}=\frac{1}{\esp\left[X\right]}\int_{0}^{\infty}\prob\left\{V\left(t+x\right)\in A|X>x\right\}\left(1-F\left(x\right)\right)dx,
\end{eqnarray*} 
para todo $t\geq0$ y todo conjunto de Borel $A$.
\end{Def}

\begin{Def}
Una distribuci\'on se dice que es {\emph{aritm\'etica}} si todos sus puntos de incremento son m\'ultiplos de la forma $0,\lambda, 2\lambda,\ldots$ para alguna $\lambda>0$ entera.
\end{Def}


\begin{Def}
Una modificaci\'on medible de un proceso $\left\{V\left(t\right),t\geq0\right\}$, es una versi\'on de este, $\left\{V\left(t,w\right)\right\}$ conjuntamente medible para $t\geq0$ y para $w\in S$, $S$ espacio de estados para $\left\{V\left(t\right),t\geq0\right\}$.
\end{Def}

\begin{Teo}
Sea $\left\{V\left(t\right),t\geq\right\}$ un proceso regenerativo no negativo con modificaci\'on medible. Sea $\esp\left[X\right]<\infty$. Entonces el proceso estacionario dado por la ecuaci\'on anterior est\'a bien definido y tiene funci\'on de distribuci\'on independiente de $t$, adem\'as
\begin{itemize}
\item[i)] \begin{eqnarray*}
\esp\left[V^{*}\left(0\right)\right]&=&\frac{\esp\left[\int_{0}^{X}V\left(s\right)ds\right]}{\esp\left[X\right]}\end{eqnarray*}
\item[ii)] Si $\esp\left[V^{*}\left(0\right)\right]<\infty$, equivalentemente, si $\esp\left[\int_{0}^{X}V\left(s\right)ds\right]<\infty$,entonces
\begin{eqnarray*}
\frac{\int_{0}^{t}V\left(s\right)ds}{t}\rightarrow\frac{\esp\left[\int_{0}^{X}V\left(s\right)ds\right]}{\esp\left[X\right]}
\end{eqnarray*}
con probabilidad 1 y en media, cuando $t\rightarrow\infty$.
\end{itemize}
\end{Teo}

%______________________________________________________________________
\subsection{Procesos de Renovaci\'on}
%______________________________________________________________________

\begin{Def}\label{Def.Tn}
Sean $0\leq T_{1}\leq T_{2}\leq \ldots$ son tiempos aleatorios infinitos en los cuales ocurren ciertos eventos. El n\'umero de tiempos $T_{n}$ en el intervalo $\left[0,t\right)$ es

\begin{eqnarray}
N\left(t\right)=\sum_{n=1}^{\infty}\indora\left(T_{n}\leq t\right),
\end{eqnarray}
para $t\geq0$.
\end{Def}

Si se consideran los puntos $T_{n}$ como elementos de $\rea_{+}$, y $N\left(t\right)$ es el n\'umero de puntos en $\rea$. El proceso denotado por $\left\{N\left(t\right):t\geq0\right\}$, denotado por $N\left(t\right)$, es un proceso puntual en $\rea_{+}$. Los $T_{n}$ son los tiempos de ocurrencia, el proceso puntual $N\left(t\right)$ es simple si su n\'umero de ocurrencias son distintas: $0<T_{1}<T_{2}<\ldots$ casi seguramente.

\begin{Def}
Un proceso puntual $N\left(t\right)$ es un proceso de renovaci\'on si los tiempos de interocurrencia $\xi_{n}=T_{n}-T_{n-1}$, para $n\geq1$, son independientes e identicamente distribuidos con distribuci\'on $F$, donde $F\left(0\right)=0$ y $T_{0}=0$. Los $T_{n}$ son llamados tiempos de renovaci\'on, referente a la independencia o renovaci\'on de la informaci\'on estoc\'astica en estos tiempos. Los $\xi_{n}$ son los tiempos de inter-renovaci\'on, y $N\left(t\right)$ es el n\'umero de renovaciones en el intervalo $\left[0,t\right)$
\end{Def}


\begin{Note}
Para definir un proceso de renovaci\'on para cualquier contexto, solamente hay que especificar una distribuci\'on $F$, con $F\left(0\right)=0$, para los tiempos de inter-renovaci\'on. La funci\'on $F$ en turno degune las otra variables aleatorias. De manera formal, existe un espacio de probabilidad y una sucesi\'on de variables aleatorias $\xi_{1},\xi_{2},\ldots$ definidas en este con distribuci\'on $F$. Entonces las otras cantidades son $T_{n}=\sum_{k=1}^{n}\xi_{k}$ y $N\left(t\right)=\sum_{n=1}^{\infty}\indora\left(T_{n}\leq t\right)$, donde $T_{n}\rightarrow\infty$ casi seguramente por la Ley Fuerte de los Grandes Números.
\end{Note}

%___________________________________________________________________________________________
%
\subsection{Teorema Principal de Renovaci\'on}
%___________________________________________________________________________________________
%

\begin{Note} Una funci\'on $h:\rea_{+}\rightarrow\rea$ es Directamente Riemann Integrable en los siguientes casos:
\begin{itemize}
\item[a)] $h\left(t\right)\geq0$ es decreciente y Riemann Integrable.
\item[b)] $h$ es continua excepto posiblemente en un conjunto de Lebesgue de medida 0, y $|h\left(t\right)|\leq b\left(t\right)$, donde $b$ es DRI.
\end{itemize}
\end{Note}

\begin{Teo}[Teorema Principal de Renovaci\'on]
Si $F$ es no aritm\'etica y $h\left(t\right)$ es Directamente Riemann Integrable (DRI), entonces

\begin{eqnarray*}
lim_{t\rightarrow\infty}U\star h=\frac{1}{\mu}\int_{\rea_{+}}h\left(s\right)ds.
\end{eqnarray*}
\end{Teo}

\begin{Prop}
Cualquier funci\'on $H\left(t\right)$ acotada en intervalos finitos y que es 0 para $t<0$ puede expresarse como
\begin{eqnarray*}
H\left(t\right)=U\star h\left(t\right)\textrm{,  donde }h\left(t\right)=H\left(t\right)-F\star H\left(t\right)
\end{eqnarray*}
\end{Prop}

\begin{Def}
Un proceso estoc\'astico $X\left(t\right)$ es crudamente regenerativo en un tiempo aleatorio positivo $T$ si
\begin{eqnarray*}
\esp\left[X\left(T+t\right)|T\right]=\esp\left[X\left(t\right)\right]\textrm{, para }t\geq0,\end{eqnarray*}
y con las esperanzas anteriores finitas.
\end{Def}

\begin{Prop}
Sup\'ongase que $X\left(t\right)$ es un proceso crudamente regenerativo en $T$, que tiene distribuci\'on $F$. Si $\esp\left[X\left(t\right)\right]$ es acotado en intervalos finitos, entonces
\begin{eqnarray*}
\esp\left[X\left(t\right)\right]=U\star h\left(t\right)\textrm{,  donde }h\left(t\right)=\esp\left[X\left(t\right)\indora\left(T>t\right)\right].
\end{eqnarray*}
\end{Prop}

\begin{Teo}[Regeneraci\'on Cruda]
Sup\'ongase que $X\left(t\right)$ es un proceso con valores positivo crudamente regenerativo en $T$, y def\'inase $M=\sup\left\{|X\left(t\right)|:t\leq T\right\}$. Si $T$ es no aritm\'etico y $M$ y $MT$ tienen media finita, entonces
\begin{eqnarray*}
lim_{t\rightarrow\infty}\esp\left[X\left(t\right)\right]=\frac{1}{\mu}\int_{\rea_{+}}h\left(s\right)ds,
\end{eqnarray*}
donde $h\left(t\right)=\esp\left[X\left(t\right)\indora\left(T>t\right)\right]$.
\end{Teo}



%___________________________________________________________________________________________
%
\subsection{Funci\'on de Renovaci\'on}
%___________________________________________________________________________________________
%


\begin{Def}
Sea $h\left(t\right)$ funci\'on de valores reales en $\rea$ acotada en intervalos finitos e igual a cero para $t<0$ La ecuaci\'on de renovaci\'on para $h\left(t\right)$ y la distribuci\'on $F$ es

\begin{eqnarray}\label{Ec.Renovacion}
H\left(t\right)=h\left(t\right)+\int_{\left[0,t\right]}H\left(t-s\right)dF\left(s\right)\textrm{,    }t\geq0,
\end{eqnarray}
donde $H\left(t\right)$ es una funci\'on de valores reales. Esto es $H=h+F\star H$. Decimos que $H\left(t\right)$ es soluci\'on de esta ecuaci\'on si satisface la ecuaci\'on, y es acotada en intervalos finitos e iguales a cero para $t<0$.
\end{Def}

\begin{Prop}
La funci\'on $U\star h\left(t\right)$ es la \'unica soluci\'on de la ecuaci\'on de renovaci\'on (\ref{Ec.Renovacion}).
\end{Prop}

\begin{Teo}[Teorema Renovaci\'on Elemental]
\begin{eqnarray*}
t^{-1}U\left(t\right)\rightarrow 1/\mu\textrm{,    cuando }t\rightarrow\infty.
\end{eqnarray*}
\end{Teo}

%___________________________________________________________________________________________
%
\subsection{Propiedades de los Procesos de Renovaci\'on}
%___________________________________________________________________________________________
%

Los tiempos $T_{n}$ est\'an relacionados con los conteos de $N\left(t\right)$ por

\begin{eqnarray*}
\left\{N\left(t\right)\geq n\right\}&=&\left\{T_{n}\leq t\right\}\\
T_{N\left(t\right)}\leq &t&<T_{N\left(t\right)+1},
\end{eqnarray*}

adem\'as $N\left(T_{n}\right)=n$, y 

\begin{eqnarray*}
N\left(t\right)=\max\left\{n:T_{n}\leq t\right\}=\min\left\{n:T_{n+1}>t\right\}
\end{eqnarray*}

Por propiedades de la convoluci\'on se sabe que

\begin{eqnarray*}
P\left\{T_{n}\leq t\right\}=F^{n\star}\left(t\right)
\end{eqnarray*}
que es la $n$-\'esima convoluci\'on de $F$. Entonces 

\begin{eqnarray*}
\left\{N\left(t\right)\geq n\right\}&=&\left\{T_{n}\leq t\right\}\\
P\left\{N\left(t\right)\leq n\right\}&=&1-F^{\left(n+1\right)\star}\left(t\right)
\end{eqnarray*}

Adem\'as usando el hecho de que $\esp\left[N\left(t\right)\right]=\sum_{n=1}^{\infty}P\left\{N\left(t\right)\geq n\right\}$
se tiene que

\begin{eqnarray*}
\esp\left[N\left(t\right)\right]=\sum_{n=1}^{\infty}F^{n\star}\left(t\right)
\end{eqnarray*}

\begin{Prop}
Para cada $t\geq0$, la funci\'on generadora de momentos $\esp\left[e^{\alpha N\left(t\right)}\right]$ existe para alguna $\alpha$ en una vecindad del 0, y de aqu\'i que $\esp\left[N\left(t\right)^{m}\right]<\infty$, para $m\geq1$.
\end{Prop}


\begin{Note}
Si el primer tiempo de renovaci\'on $\xi_{1}$ no tiene la misma distribuci\'on que el resto de las $\xi_{n}$, para $n\geq2$, a $N\left(t\right)$ se le llama Proceso de Renovaci\'on retardado, donde si $\xi$ tiene distribuci\'on $G$, entonces el tiempo $T_{n}$ de la $n$-\'esima renovaci\'on tiene distribuci\'on $G\star F^{\left(n-1\right)\star}\left(t\right)$
\end{Note}


\begin{Teo}
Para una constante $\mu\leq\infty$ ( o variable aleatoria), las siguientes expresiones son equivalentes:

\begin{eqnarray}
lim_{n\rightarrow\infty}n^{-1}T_{n}&=&\mu,\textrm{ c.s.}\\
lim_{t\rightarrow\infty}t^{-1}N\left(t\right)&=&1/\mu,\textrm{ c.s.}
\end{eqnarray}
\end{Teo}


Es decir, $T_{n}$ satisface la Ley Fuerte de los Grandes N\'umeros s\'i y s\'olo s\'i $N\left/t\right)$ la cumple.


\begin{Coro}[Ley Fuerte de los Grandes N\'umeros para Procesos de Renovaci\'on]
Si $N\left(t\right)$ es un proceso de renovaci\'on cuyos tiempos de inter-renovaci\'on tienen media $\mu\leq\infty$, entonces
\begin{eqnarray}
t^{-1}N\left(t\right)\rightarrow 1/\mu,\textrm{ c.s. cuando }t\rightarrow\infty.
\end{eqnarray}

\end{Coro}


Considerar el proceso estoc\'astico de valores reales $\left\{Z\left(t\right):t\geq0\right\}$ en el mismo espacio de probabilidad que $N\left(t\right)$

\begin{Def}
Para el proceso $\left\{Z\left(t\right):t\geq0\right\}$ se define la fluctuaci\'on m\'axima de $Z\left(t\right)$ en el intervalo $\left(T_{n-1},T_{n}\right]$:
\begin{eqnarray*}
M_{n}=\sup_{T_{n-1}<t\leq T_{n}}|Z\left(t\right)-Z\left(T_{n-1}\right)|
\end{eqnarray*}
\end{Def}

\begin{Teo}
Sup\'ongase que $n^{-1}T_{n}\rightarrow\mu$ c.s. cuando $n\rightarrow\infty$, donde $\mu\leq\infty$ es una constante o variable aleatoria. Sea $a$ una constante o variable aleatoria que puede ser infinita cuando $\mu$ es finita, y considere las expresiones l\'imite:
\begin{eqnarray}
lim_{n\rightarrow\infty}n^{-1}Z\left(T_{n}\right)&=&a,\textrm{ c.s.}\\
lim_{t\rightarrow\infty}t^{-1}Z\left(t\right)&=&a/\mu,\textrm{ c.s.}
\end{eqnarray}
La segunda expresi\'on implica la primera. Conversamente, la primera implica la segunda si el proceso $Z\left(t\right)$ es creciente, o si $lim_{n\rightarrow\infty}n^{-1}M_{n}=0$ c.s.
\end{Teo}

\begin{Coro}
Si $N\left(t\right)$ es un proceso de renovaci\'on, y $\left(Z\left(T_{n}\right)-Z\left(T_{n-1}\right),M_{n}\right)$, para $n\geq1$, son variables aleatorias independientes e id\'enticamente distribuidas con media finita, entonces,
\begin{eqnarray}
lim_{t\rightarrow\infty}t^{-1}Z\left(t\right)\rightarrow\frac{\esp\left[Z\left(T_{1}\right)-Z\left(T_{0}\right)\right]}{\esp\left[T_{1}\right]},\textrm{ c.s. cuando  }t\rightarrow\infty.
\end{eqnarray}
\end{Coro}

%___________________________________________________________________________________________
%
\subsection{Funci\'on de Renovaci\'on}
%___________________________________________________________________________________________
%


Sup\'ongase que $N\left(t\right)$ es un proceso de renovaci\'on con distribuci\'on $F$ con media finita $\mu$.

\begin{Def}
La funci\'on de renovaci\'on asociada con la distribuci\'on $F$, del proceso $N\left(t\right)$, es
\begin{eqnarray*}
U\left(t\right)=\sum_{n=1}^{\infty}F^{n\star}\left(t\right),\textrm{   }t\geq0,
\end{eqnarray*}
donde $F^{0\star}\left(t\right)=\indora\left(t\geq0\right)$.
\end{Def}


\begin{Prop}
Sup\'ongase que la distribuci\'on de inter-renovaci\'on $F$ tiene densidad $f$. Entonces $U\left(t\right)$ tambi\'en tiene densidad, para $t>0$, y es $U^{'}\left(t\right)=\sum_{n=0}^{\infty}f^{n\star}\left(t\right)$. Adem\'as
\begin{eqnarray*}
\prob\left\{N\left(t\right)>N\left(t-\right)\right\}=0\textrm{,   }t\geq0.
\end{eqnarray*}
\end{Prop}

\begin{Def}
La Transformada de Laplace-Stieljes de $F$ est\'a dada por

\begin{eqnarray*}
\hat{F}\left(\alpha\right)=\int_{\rea_{+}}e^{-\alpha t}dF\left(t\right)\textrm{,  }\alpha\geq0.
\end{eqnarray*}
\end{Def}

Entonces

\begin{eqnarray*}
\hat{U}\left(\alpha\right)=\sum_{n=0}^{\infty}\hat{F^{n\star}}\left(\alpha\right)=\sum_{n=0}^{\infty}\hat{F}\left(\alpha\right)^{n}=\frac{1}{1-\hat{F}\left(\alpha\right)}.
\end{eqnarray*}


\begin{Prop}
La Transformada de Laplace $\hat{U}\left(\alpha\right)$ y $\hat{F}\left(\alpha\right)$ determina una a la otra de manera \'unica por la relaci\'on $\hat{U}\left(\alpha\right)=\frac{1}{1-\hat{F}\left(\alpha\right)}$.
\end{Prop}


\begin{Note}
Un proceso de renovaci\'on $N\left(t\right)$ cuyos tiempos de inter-renovaci\'on tienen media finita, es un proceso Poisson con tasa $\lambda$ si y s\'olo s\'i $\esp\left[U\left(t\right)\right]=\lambda t$, para $t\geq0$.
\end{Note}


\begin{Teo}
Sea $N\left(t\right)$ un proceso puntual simple con puntos de localizaci\'on $T_{n}$ tal que $\eta\left(t\right)=\esp\left[N\left(\right)\right]$ es finita para cada $t$. Entonces para cualquier funci\'on $f:\rea_{+}\rightarrow\rea$,
\begin{eqnarray*}
\esp\left[\sum_{n=1}^{N\left(\right)}f\left(T_{n}\right)\right]=\int_{\left(0,t\right]}f\left(s\right)d\eta\left(s\right)\textrm{,  }t\geq0,
\end{eqnarray*}
suponiendo que la integral exista. Adem\'as si $X_{1},X_{2},\ldots$ son variables aleatorias definidas en el mismo espacio de probabilidad que el proceso $N\left(t\right)$ tal que $\esp\left[X_{n}|T_{n}=s\right]=f\left(s\right)$, independiente de $n$. Entonces
\begin{eqnarray*}
\esp\left[\sum_{n=1}^{N\left(t\right)}X_{n}\right]=\int_{\left(0,t\right]}f\left(s\right)d\eta\left(s\right)\textrm{,  }t\geq0,
\end{eqnarray*} 
suponiendo que la integral exista. 
\end{Teo}

\begin{Coro}[Identidad de Wald para Renovaciones]
Para el proceso de renovaci\'on $N\left(t\right)$,
\begin{eqnarray*}
\esp\left[T_{N\left(t\right)+1}\right]=\mu\esp\left[N\left(t\right)+1\right]\textrm{,  }t\geq0,
\end{eqnarray*}  
\end{Coro}

%______________________________________________________________________
\subsection{Procesos de Renovaci\'on}
%______________________________________________________________________

\begin{Def}\label{Def.Tn}
Sean $0\leq T_{1}\leq T_{2}\leq \ldots$ son tiempos aleatorios infinitos en los cuales ocurren ciertos eventos. El n\'umero de tiempos $T_{n}$ en el intervalo $\left[0,t\right)$ es

\begin{eqnarray}
N\left(t\right)=\sum_{n=1}^{\infty}\indora\left(T_{n}\leq t\right),
\end{eqnarray}
para $t\geq0$.
\end{Def}

Si se consideran los puntos $T_{n}$ como elementos de $\rea_{+}$, y $N\left(t\right)$ es el n\'umero de puntos en $\rea$. El proceso denotado por $\left\{N\left(t\right):t\geq0\right\}$, denotado por $N\left(t\right)$, es un proceso puntual en $\rea_{+}$. Los $T_{n}$ son los tiempos de ocurrencia, el proceso puntual $N\left(t\right)$ es simple si su n\'umero de ocurrencias son distintas: $0<T_{1}<T_{2}<\ldots$ casi seguramente.

\begin{Def}
Un proceso puntual $N\left(t\right)$ es un proceso de renovaci\'on si los tiempos de interocurrencia $\xi_{n}=T_{n}-T_{n-1}$, para $n\geq1$, son independientes e identicamente distribuidos con distribuci\'on $F$, donde $F\left(0\right)=0$ y $T_{0}=0$. Los $T_{n}$ son llamados tiempos de renovaci\'on, referente a la independencia o renovaci\'on de la informaci\'on estoc\'astica en estos tiempos. Los $\xi_{n}$ son los tiempos de inter-renovaci\'on, y $N\left(t\right)$ es el n\'umero de renovaciones en el intervalo $\left[0,t\right)$
\end{Def}


\begin{Note}
Para definir un proceso de renovaci\'on para cualquier contexto, solamente hay que especificar una distribuci\'on $F$, con $F\left(0\right)=0$, para los tiempos de inter-renovaci\'on. La funci\'on $F$ en turno degune las otra variables aleatorias. De manera formal, existe un espacio de probabilidad y una sucesi\'on de variables aleatorias $\xi_{1},\xi_{2},\ldots$ definidas en este con distribuci\'on $F$. Entonces las otras cantidades son $T_{n}=\sum_{k=1}^{n}\xi_{k}$ y $N\left(t\right)=\sum_{n=1}^{\infty}\indora\left(T_{n}\leq t\right)$, donde $T_{n}\rightarrow\infty$ casi seguramente por la Ley Fuerte de los Grandes Números.
\end{Note}
%_____________________________________________________
\subsection{Puntos de Renovaci\'on}
%_____________________________________________________

Para cada cola $Q_{i}$ se tienen los procesos de arribo a la cola, para estas, los tiempos de arribo est\'an dados por $$\left\{T_{1}^{i},T_{2}^{i},\ldots,T_{k}^{i},\ldots\right\},$$ entonces, consideremos solamente los primeros tiempos de arribo a cada una de las colas, es decir, $$\left\{T_{1}^{1},T_{1}^{2},T_{1}^{3},T_{1}^{4}\right\},$$ se sabe que cada uno de estos tiempos se distribuye de manera exponencial con par\'ametro $1/mu_{i}$. Adem\'as se sabe que para $$T^{*}=\min\left\{T_{1}^{1},T_{1}^{2},T_{1}^{3},T_{1}^{4}\right\},$$ $T^{*}$ se distribuye de manera exponencial con par\'ametro $$\mu^{*}=\sum_{i=1}^{4}\mu_{i}.$$ Ahora, dado que 
\begin{center}
\begin{tabular}{lcl}
$\tilde{r}=r_{1}+r_{2}$ & y &$\hat{r}=r_{3}+r_{4}.$
\end{tabular}
\end{center}


Supongamos que $$\tilde{r},\hat{r}<\mu^{*},$$ entonces si tomamos $$r^{*}=\min\left\{\tilde{r},\hat{r}\right\},$$ se tiene que para  $$t^{*}\in\left(0,r^{*}\right)$$ se cumple que 
\begin{center}
\begin{tabular}{lcl}
$\tau_{1}\left(1\right)=0$ & y por tanto & $\overline{\tau}_{1}=0,$
\end{tabular}
\end{center}
entonces para la segunda cola en este primer ciclo se cumple que $$\tau_{2}=\overline{\tau}_{1}+r_{1}=r_{1}<\mu^{*},$$ y por tanto se tiene que  $$\overline{\tau}_{2}=\tau_{2}.$$ Por lo tanto, nuevamente para la primer cola en el segundo ciclo $$\tau_{1}\left(2\right)=\tau_{2}\left(1\right)+r_{2}=\tilde{r}<\mu^{*}.$$ An\'alogamente para el segundo sistema se tiene que ambas colas est\'an vac\'ias, es decir, existe un valor $t^{*}$ tal que en el intervalo $\left(0,t^{*}\right)$ no ha llegado ning\'un usuario, es decir, $$L_{i}\left(t^{*}\right)=0$$ para $i=1,2,3,4$.

\subsection{Resultados para Procesos de Salida}

En \cite{Sigman2} prueban que para la existencia de un una sucesi\'on infinita no decreciente de tiempos de regeneraci\'on $\tau_{1}\leq\tau_{2}\leq\cdots$ en los cuales el proceso se regenera, basta un tiempo de regeneraci\'on $R_{1}$, donde $R_{j}=\tau_{j}-\tau_{j-1}$. Para tal efecto se requiere la existencia de un espacio de probabilidad $\left(\Omega,\mathcal{F},\prob\right)$, y proceso estoc\'astico $\textit{X}=\left\{X\left(t\right):t\geq0\right\}$ con espacio de estados $\left(S,\mathcal{R}\right)$, con $\mathcal{R}$ $\sigma$-\'algebra.

\begin{Prop}
Si existe una variable aleatoria no negativa $R_{1}$ tal que $\theta_{R\footnotesize{1}}X=_{D}X$, entonces $\left(\Omega,\mathcal{F},\prob\right)$ puede extenderse para soportar una sucesi\'on estacionaria de variables aleatorias $R=\left\{R_{k}:k\geq1\right\}$, tal que para $k\geq1$,
\begin{eqnarray*}
\theta_{k}\left(X,R\right)=_{D}\left(X,R\right).
\end{eqnarray*}

Adem\'as, para $k\geq1$, $\theta_{k}R$ es condicionalmente independiente de $\left(X,R_{1},\ldots,R_{k}\right)$, dado $\theta_{\tau k}X$.

\end{Prop}


\begin{itemize}
\item Doob en 1953 demostr\'o que el estado estacionario de un proceso de partida en un sistema de espera $M/G/\infty$, es Poisson con la misma tasa que el proceso de arribos.

\item Burke en 1968, fue el primero en demostrar que el estado estacionario de un proceso de salida de una cola $M/M/s$ es un proceso Poisson.

\item Disney en 1973 obtuvo el siguiente resultado:

\begin{Teo}
Para el sistema de espera $M/G/1/L$ con disciplina FIFO, el proceso $\textbf{I}$ es un proceso de renovaci\'on si y s\'olo si el proceso denominado longitud de la cola es estacionario y se cumple cualquiera de los siguientes casos:

\begin{itemize}
\item[a)] Los tiempos de servicio son identicamente cero;
\item[b)] $L=0$, para cualquier proceso de servicio $S$;
\item[c)] $L=1$ y $G=D$;
\item[d)] $L=\infty$ y $G=M$.
\end{itemize}
En estos casos, respectivamente, las distribuciones de interpartida $P\left\{T_{n+1}-T_{n}\leq t\right\}$ son


\begin{itemize}
\item[a)] $1-e^{-\lambda t}$, $t\geq0$;
\item[b)] $1-e^{-\lambda t}*F\left(t\right)$, $t\geq0$;
\item[c)] $1-e^{-\lambda t}*\indora_{d}\left(t\right)$, $t\geq0$;
\item[d)] $1-e^{-\lambda t}*F\left(t\right)$, $t\geq0$.
\end{itemize}
\end{Teo}


\item Finch (1959) mostr\'o que para los sistemas $M/G/1/L$, con $1\leq L\leq \infty$ con distribuciones de servicio dos veces diferenciable, solamente el sistema $M/M/1/\infty$ tiene proceso de salida de renovaci\'on estacionario.

\item King (1971) demostro que un sistema de colas estacionario $M/G/1/1$ tiene sus tiempos de interpartida sucesivas $D_{n}$ y $D_{n+1}$ son independientes, si y s\'olo si, $G=D$, en cuyo caso le proceso de salida es de renovaci\'on.

\item Disney (1973) demostr\'o que el \'unico sistema estacionario $M/G/1/L$, que tiene proceso de salida de renovaci\'on  son los sistemas $M/M/1$ y $M/D/1/1$.



\item El siguiente resultado es de Disney y Koning (1985)
\begin{Teo}
En un sistema de espera $M/G/s$, el estado estacionario del proceso de salida es un proceso Poisson para cualquier distribuci\'on de los tiempos de servicio si el sistema tiene cualquiera de las siguientes cuatro propiedades.

\begin{itemize}
\item[a)] $s=\infty$
\item[b)] La disciplina de servicio es de procesador compartido.
\item[c)] La disciplina de servicio es LCFS y preemptive resume, esto se cumple para $L<\infty$
\item[d)] $G=M$.
\end{itemize}

\end{Teo}

\item El siguiente resultado es de Alamatsaz (1983)

\begin{Teo}
En cualquier sistema de colas $GI/G/1/L$ con $1\leq L<\infty$ y distribuci\'on de interarribos $A$ y distribuci\'on de los tiempos de servicio $B$, tal que $A\left(0\right)=0$, $A\left(t\right)\left(1-B\left(t\right)\right)>0$ para alguna $t>0$ y $B\left(t\right)$ para toda $t>0$, es imposible que el proceso de salida estacionario sea de renovaci\'on.
\end{Teo}

\end{itemize}

Estos resultados aparecen en Daley (1968) \cite{Daley68} para $\left\{T_{n}\right\}$ intervalos de inter-arribo, $\left\{D_{n}\right\}$ intervalos de inter-salida y $\left\{S_{n}\right\}$ tiempos de servicio.

\begin{itemize}
\item Si el proceso $\left\{T_{n}\right\}$ es Poisson, el proceso $\left\{D_{n}\right\}$ es no correlacionado si y s\'olo si es un proceso Poisso, lo cual ocurre si y s\'olo si $\left\{S_{n}\right\}$ son exponenciales negativas.

\item Si $\left\{S_{n}\right\}$ son exponenciales negativas, $\left\{D_{n}\right\}$ es un proceso de renovaci\'on  si y s\'olo si es un proceso Poisson, lo cual ocurre si y s\'olo si $\left\{T_{n}\right\}$ es un proceso Poisson.

\item $\esp\left(D_{n}\right)=\esp\left(T_{n}\right)$.

\item Para un sistema de visitas $GI/M/1$ se tiene el siguiente teorema:

\begin{Teo}
En un sistema estacionario $GI/M/1$ los intervalos de interpartida tienen
\begin{eqnarray*}
\esp\left(e^{-\theta D_{n}}\right)&=&\mu\left(\mu+\theta\right)^{-1}\left[\delta\theta
-\mu\left(1-\delta\right)\alpha\left(\theta\right)\right]
\left[\theta-\mu\left(1-\delta\right)^{-1}\right]\\
\alpha\left(\theta\right)&=&\esp\left[e^{-\theta T_{0}}\right]\\
var\left(D_{n}\right)&=&var\left(T_{0}\right)-\left(\tau^{-1}-\delta^{-1}\right)
2\delta\left(\esp\left(S_{0}\right)\right)^{2}\left(1-\delta\right)^{-1}.
\end{eqnarray*}
\end{Teo}



\begin{Teo}
El proceso de salida de un sistema de colas estacionario $GI/M/1$ es un proceso de renovaci\'on si y s\'olo si el proceso de entrada es un proceso Poisson, en cuyo caso el proceso de salida es un proceso Poisson.
\end{Teo}


\begin{Teo}
Los intervalos de interpartida $\left\{D_{n}\right\}$ de un sistema $M/G/1$ estacionario son no correlacionados si y s\'olo si la distribuci\'on de los tiempos de servicio es exponencial negativa, es decir, el sistema es de tipo  $M/M/1$.

\end{Teo}



\end{itemize}


%________________________________________________________________________
\subsection{Procesos Regenerativos}
%________________________________________________________________________

Para $\left\{X\left(t\right):t\geq0\right\}$ Proceso Estoc\'astico a tiempo continuo con estado de espacios $S$, que es un espacio m\'etrico, con trayectorias continuas por la derecha y con l\'imites por la izquierda c.s. Sea $N\left(t\right)$ un proceso de renovaci\'on en $\rea_{+}$ definido en el mismo espacio de probabilidad que $X\left(t\right)$, con tiempos de renovaci\'on $T$ y tiempos de inter-renovaci\'on $\xi_{n}=T_{n}-T_{n-1}$, con misma distribuci\'on $F$ de media finita $\mu$.



\begin{Def}
Para el proceso $\left\{\left(N\left(t\right),X\left(t\right)\right):t\geq0\right\}$, sus trayectoria muestrales en el intervalo de tiempo $\left[T_{n-1},T_{n}\right)$ est\'an descritas por
\begin{eqnarray*}
\zeta_{n}=\left(\xi_{n},\left\{X\left(T_{n-1}+t\right):0\leq t<\xi_{n}\right\}\right)
\end{eqnarray*}
Este $\zeta_{n}$ es el $n$-\'esimo segmento del proceso. El proceso es regenerativo sobre los tiempos $T_{n}$ si sus segmentos $\zeta_{n}$ son independientes e id\'enticamennte distribuidos.
\end{Def}


\begin{Obs}
Si $\tilde{X}\left(t\right)$ con espacio de estados $\tilde{S}$ es regenerativo sobre $T_{n}$, entonces $X\left(t\right)=f\left(\tilde{X}\left(t\right)\right)$ tambi\'en es regenerativo sobre $T_{n}$, para cualquier funci\'on $f:\tilde{S}\rightarrow S$.
\end{Obs}

\begin{Obs}
Los procesos regenerativos son crudamente regenerativos, pero no al rev\'es.
\end{Obs}

\begin{Def}[Definici\'on Cl\'asica]
Un proceso estoc\'astico $X=\left\{X\left(t\right):t\geq0\right\}$ es llamado regenerativo is existe una variable aleatoria $R_{1}>0$ tal que
\begin{itemize}
\item[i)] $\left\{X\left(t+R_{1}\right):t\geq0\right\}$ es independiente de $\left\{\left\{X\left(t\right):t<R_{1}\right\},\right\}$
\item[ii)] $\left\{X\left(t+R_{1}\right):t\geq0\right\}$ es estoc\'asticamente equivalente a $\left\{X\left(t\right):t>0\right\}$
\end{itemize}

Llamamos a $R_{1}$ tiempo de regeneraci\'on, y decimos que $X$ se regenera en este punto.
\end{Def}

$\left\{X\left(t+R_{1}\right)\right\}$ es regenerativo con tiempo de regeneraci\'on $R_{2}$, independiente de $R_{1}$ pero con la misma distribuci\'on que $R_{1}$. Procediendo de esta manera se obtiene una secuencia de variables aleatorias independientes e id\'enticamente distribuidas $\left\{R_{n}\right\}$ llamados longitudes de ciclo. Si definimos a $Z_{k}\equiv R_{1}+R_{2}+\cdots+R_{k}$, se tiene un proceso de renovaci\'on llamado proceso de renovaci\'on encajado para $X$.

\begin{Note}
Un proceso regenerativo con media de la longitud de ciclo finita es llamado positivo recurrente.
\end{Note}


\begin{Def}
Para $x$ fijo y para cada $t\geq0$, sea $I_{x}\left(t\right)=1$ si $X\left(t\right)\leq x$,  $I_{x}\left(t\right)=0$ en caso contrario, y def\'inanse los tiempos promedio
\begin{eqnarray*}
\overline{X}&=&lim_{t\rightarrow\infty}\frac{1}{t}\int_{0}^{\infty}X\left(u\right)du\\
\prob\left(X_{\infty}\leq x\right)&=&lim_{t\rightarrow\infty}\frac{1}{t}\int_{0}^{\infty}I_{x}\left(u\right)du,
\end{eqnarray*}
cuando estos l\'imites existan.
\end{Def}

Como consecuencia del teorema de Renovaci\'on-Recompensa, se tiene que el primer l\'imite  existe y es igual a la constante
\begin{eqnarray*}
\overline{X}&=&\frac{\esp\left[\int_{0}^{R_{1}}X\left(t\right)dt\right]}{\esp\left[R_{1}\right]},
\end{eqnarray*}
suponiendo que ambas esperanzas son finitas.

\begin{Note}
\begin{itemize}
\item[a)] Si el proceso regenerativo $X$ es positivo recurrente y tiene trayectorias muestrales no negativas, entonces la ecuaci\'on anterior es v\'alida.
\item[b)] Si $X$ es positivo recurrente regenerativo, podemos construir una \'unica versi\'on estacionaria de este proceso, $X_{e}=\left\{X_{e}\left(t\right)\right\}$, donde $X_{e}$ es un proceso estoc\'astico regenerativo y estrictamente estacionario, con distribuci\'on marginal distribuida como $X_{\infty}$
\end{itemize}
\end{Note}

\subsection{Renewal and Regenerative Processes: Serfozo\cite{Serfozo}}
\begin{Def}\label{Def.Tn}
Sean $0\leq T_{1}\leq T_{2}\leq \ldots$ son tiempos aleatorios infinitos en los cuales ocurren ciertos eventos. El n\'umero de tiempos $T_{n}$ en el intervalo $\left[0,t\right)$ es

\begin{eqnarray}
N\left(t\right)=\sum_{n=1}^{\infty}\indora\left(T_{n}\leq t\right),
\end{eqnarray}
para $t\geq0$.
\end{Def}

Si se consideran los puntos $T_{n}$ como elementos de $\rea_{+}$, y $N\left(t\right)$ es el n\'umero de puntos en $\rea$. El proceso denotado por $\left\{N\left(t\right):t\geq0\right\}$, denotado por $N\left(t\right)$, es un proceso puntual en $\rea_{+}$. Los $T_{n}$ son los tiempos de ocurrencia, el proceso puntual $N\left(t\right)$ es simple si su n\'umero de ocurrencias son distintas: $0<T_{1}<T_{2}<\ldots$ casi seguramente.

\begin{Def}
Un proceso puntual $N\left(t\right)$ es un proceso de renovaci\'on si los tiempos de interocurrencia $\xi_{n}=T_{n}-T_{n-1}$, para $n\geq1$, son independientes e identicamente distribuidos con distribuci\'on $F$, donde $F\left(0\right)=0$ y $T_{0}=0$. Los $T_{n}$ son llamados tiempos de renovaci\'on, referente a la independencia o renovaci\'on de la informaci\'on estoc\'astica en estos tiempos. Los $\xi_{n}$ son los tiempos de inter-renovaci\'on, y $N\left(t\right)$ es el n\'umero de renovaciones en el intervalo $\left[0,t\right)$
\end{Def}


\begin{Note}
Para definir un proceso de renovaci\'on para cualquier contexto, solamente hay que especificar una distribuci\'on $F$, con $F\left(0\right)=0$, para los tiempos de inter-renovaci\'on. La funci\'on $F$ en turno degune las otra variables aleatorias. De manera formal, existe un espacio de probabilidad y una sucesi\'on de variables aleatorias $\xi_{1},\xi_{2},\ldots$ definidas en este con distribuci\'on $F$. Entonces las otras cantidades son $T_{n}=\sum_{k=1}^{n}\xi_{k}$ y $N\left(t\right)=\sum_{n=1}^{\infty}\indora\left(T_{n}\leq t\right)$, donde $T_{n}\rightarrow\infty$ casi seguramente por la Ley Fuerte de los Grandes N\'umeros.
\end{Note}







Los tiempos $T_{n}$ est\'an relacionados con los conteos de $N\left(t\right)$ por

\begin{eqnarray*}
\left\{N\left(t\right)\geq n\right\}&=&\left\{T_{n}\leq t\right\}\\
T_{N\left(t\right)}\leq &t&<T_{N\left(t\right)+1},
\end{eqnarray*}

adem\'as $N\left(T_{n}\right)=n$, y 

\begin{eqnarray*}
N\left(t\right)=\max\left\{n:T_{n}\leq t\right\}=\min\left\{n:T_{n+1}>t\right\}
\end{eqnarray*}

Por propiedades de la convoluci\'on se sabe que

\begin{eqnarray*}
P\left\{T_{n}\leq t\right\}=F^{n\star}\left(t\right)
\end{eqnarray*}
que es la $n$-\'esima convoluci\'on de $F$. Entonces 

\begin{eqnarray*}
\left\{N\left(t\right)\geq n\right\}&=&\left\{T_{n}\leq t\right\}\\
P\left\{N\left(t\right)\leq n\right\}&=&1-F^{\left(n+1\right)\star}\left(t\right)
\end{eqnarray*}

Adem\'as usando el hecho de que $\esp\left[N\left(t\right)\right]=\sum_{n=1}^{\infty}P\left\{N\left(t\right)\geq n\right\}$
se tiene que

\begin{eqnarray*}
\esp\left[N\left(t\right)\right]=\sum_{n=1}^{\infty}F^{n\star}\left(t\right)
\end{eqnarray*}

\begin{Prop}
Para cada $t\geq0$, la funci\'on generadora de momentos $\esp\left[e^{\alpha N\left(t\right)}\right]$ existe para alguna $\alpha$ en una vecindad del 0, y de aqu\'i que $\esp\left[N\left(t\right)^{m}\right]<\infty$, para $m\geq1$.
\end{Prop}


\begin{Note}
Si el primer tiempo de renovaci\'on $\xi_{1}$ no tiene la misma distribuci\'on que el resto de las $\xi_{n}$, para $n\geq2$, a $N\left(t\right)$ se le llama Proceso de Renovaci\'on retardado, donde si $\xi$ tiene distribuci\'on $G$, entonces el tiempo $T_{n}$ de la $n$-\'esima renovaci\'on tiene distribuci\'on $G\star F^{\left(n-1\right)\star}\left(t\right)$
\end{Note}


\begin{Teo}
Para una constante $\mu\leq\infty$ ( o variable aleatoria), las siguientes expresiones son equivalentes:

\begin{eqnarray}
lim_{n\rightarrow\infty}n^{-1}T_{n}&=&\mu,\textrm{ c.s.}\\
lim_{t\rightarrow\infty}t^{-1}N\left(t\right)&=&1/\mu,\textrm{ c.s.}
\end{eqnarray}
\end{Teo}


Es decir, $T_{n}$ satisface la Ley Fuerte de los Grandes N\'umeros s\'i y s\'olo s\'i $N\left/t\right)$ la cumple.


\begin{Coro}[Ley Fuerte de los Grandes N\'umeros para Procesos de Renovaci\'on]
Si $N\left(t\right)$ es un proceso de renovaci\'on cuyos tiempos de inter-renovaci\'on tienen media $\mu\leq\infty$, entonces
\begin{eqnarray}
t^{-1}N\left(t\right)\rightarrow 1/\mu,\textrm{ c.s. cuando }t\rightarrow\infty.
\end{eqnarray}

\end{Coro}


Considerar el proceso estoc\'astico de valores reales $\left\{Z\left(t\right):t\geq0\right\}$ en el mismo espacio de probabilidad que $N\left(t\right)$

\begin{Def}
Para el proceso $\left\{Z\left(t\right):t\geq0\right\}$ se define la fluctuaci\'on m\'axima de $Z\left(t\right)$ en el intervalo $\left(T_{n-1},T_{n}\right]$:
\begin{eqnarray*}
M_{n}=\sup_{T_{n-1}<t\leq T_{n}}|Z\left(t\right)-Z\left(T_{n-1}\right)|
\end{eqnarray*}
\end{Def}

\begin{Teo}
Sup\'ongase que $n^{-1}T_{n}\rightarrow\mu$ c.s. cuando $n\rightarrow\infty$, donde $\mu\leq\infty$ es una constante o variable aleatoria. Sea $a$ una constante o variable aleatoria que puede ser infinita cuando $\mu$ es finita, y considere las expresiones l\'imite:
\begin{eqnarray}
lim_{n\rightarrow\infty}n^{-1}Z\left(T_{n}\right)&=&a,\textrm{ c.s.}\\
lim_{t\rightarrow\infty}t^{-1}Z\left(t\right)&=&a/\mu,\textrm{ c.s.}
\end{eqnarray}
La segunda expresi\'on implica la primera. Conversamente, la primera implica la segunda si el proceso $Z\left(t\right)$ es creciente, o si $lim_{n\rightarrow\infty}n^{-1}M_{n}=0$ c.s.
\end{Teo}

\begin{Coro}
Si $N\left(t\right)$ es un proceso de renovaci\'on, y $\left(Z\left(T_{n}\right)-Z\left(T_{n-1}\right),M_{n}\right)$, para $n\geq1$, son variables aleatorias independientes e id\'enticamente distribuidas con media finita, entonces,
\begin{eqnarray}
lim_{t\rightarrow\infty}t^{-1}Z\left(t\right)\rightarrow\frac{\esp\left[Z\left(T_{1}\right)-Z\left(T_{0}\right)\right]}{\esp\left[T_{1}\right]},\textrm{ c.s. cuando  }t\rightarrow\infty.
\end{eqnarray}
\end{Coro}


Sup\'ongase que $N\left(t\right)$ es un proceso de renovaci\'on con distribuci\'on $F$ con media finita $\mu$.

\begin{Def}
La funci\'on de renovaci\'on asociada con la distribuci\'on $F$, del proceso $N\left(t\right)$, es
\begin{eqnarray*}
U\left(t\right)=\sum_{n=1}^{\infty}F^{n\star}\left(t\right),\textrm{   }t\geq0,
\end{eqnarray*}
donde $F^{0\star}\left(t\right)=\indora\left(t\geq0\right)$.
\end{Def}


\begin{Prop}
Sup\'ongase que la distribuci\'on de inter-renovaci\'on $F$ tiene densidad $f$. Entonces $U\left(t\right)$ tambi\'en tiene densidad, para $t>0$, y es $U^{'}\left(t\right)=\sum_{n=0}^{\infty}f^{n\star}\left(t\right)$. Adem\'as
\begin{eqnarray*}
\prob\left\{N\left(t\right)>N\left(t-\right)\right\}=0\textrm{,   }t\geq0.
\end{eqnarray*}
\end{Prop}

\begin{Def}
La Transformada de Laplace-Stieljes de $F$ est\'a dada por

\begin{eqnarray*}
\hat{F}\left(\alpha\right)=\int_{\rea_{+}}e^{-\alpha t}dF\left(t\right)\textrm{,  }\alpha\geq0.
\end{eqnarray*}
\end{Def}

Entonces

\begin{eqnarray*}
\hat{U}\left(\alpha\right)=\sum_{n=0}^{\infty}\hat{F^{n\star}}\left(\alpha\right)=\sum_{n=0}^{\infty}\hat{F}\left(\alpha\right)^{n}=\frac{1}{1-\hat{F}\left(\alpha\right)}.
\end{eqnarray*}


\begin{Prop}
La Transformada de Laplace $\hat{U}\left(\alpha\right)$ y $\hat{F}\left(\alpha\right)$ determina una a la otra de manera \'unica por la relaci\'on $\hat{U}\left(\alpha\right)=\frac{1}{1-\hat{F}\left(\alpha\right)}$.
\end{Prop}


\begin{Note}
Un proceso de renovaci\'on $N\left(t\right)$ cuyos tiempos de inter-renovaci\'on tienen media finita, es un proceso Poisson con tasa $\lambda$ si y s\'olo s\'i $\esp\left[U\left(t\right)\right]=\lambda t$, para $t\geq0$.
\end{Note}


\begin{Teo}
Sea $N\left(t\right)$ un proceso puntual simple con puntos de localizaci\'on $T_{n}$ tal que $\eta\left(t\right)=\esp\left[N\left(\right)\right]$ es finita para cada $t$. Entonces para cualquier funci\'on $f:\rea_{+}\rightarrow\rea$,
\begin{eqnarray*}
\esp\left[\sum_{n=1}^{N\left(\right)}f\left(T_{n}\right)\right]=\int_{\left(0,t\right]}f\left(s\right)d\eta\left(s\right)\textrm{,  }t\geq0,
\end{eqnarray*}
suponiendo que la integral exista. Adem\'as si $X_{1},X_{2},\ldots$ son variables aleatorias definidas en el mismo espacio de probabilidad que el proceso $N\left(t\right)$ tal que $\esp\left[X_{n}|T_{n}=s\right]=f\left(s\right)$, independiente de $n$. Entonces
\begin{eqnarray*}
\esp\left[\sum_{n=1}^{N\left(t\right)}X_{n}\right]=\int_{\left(0,t\right]}f\left(s\right)d\eta\left(s\right)\textrm{,  }t\geq0,
\end{eqnarray*} 
suponiendo que la integral exista. 
\end{Teo}

\begin{Coro}[Identidad de Wald para Renovaciones]
Para el proceso de renovaci\'on $N\left(t\right)$,
\begin{eqnarray*}
\esp\left[T_{N\left(t\right)+1}\right]=\mu\esp\left[N\left(t\right)+1\right]\textrm{,  }t\geq0,
\end{eqnarray*}  
\end{Coro}


\begin{Def}
Sea $h\left(t\right)$ funci\'on de valores reales en $\rea$ acotada en intervalos finitos e igual a cero para $t<0$ La ecuaci\'on de renovaci\'on para $h\left(t\right)$ y la distribuci\'on $F$ es

\begin{eqnarray}\label{Ec.Renovacion}
H\left(t\right)=h\left(t\right)+\int_{\left[0,t\right]}H\left(t-s\right)dF\left(s\right)\textrm{,    }t\geq0,
\end{eqnarray}
donde $H\left(t\right)$ es una funci\'on de valores reales. Esto es $H=h+F\star H$. Decimos que $H\left(t\right)$ es soluci\'on de esta ecuaci\'on si satisface la ecuaci\'on, y es acotada en intervalos finitos e iguales a cero para $t<0$.
\end{Def}

\begin{Prop}
La funci\'on $U\star h\left(t\right)$ es la \'unica soluci\'on de la ecuaci\'on de renovaci\'on (\ref{Ec.Renovacion}).
\end{Prop}

\begin{Teo}[Teorema Renovaci\'on Elemental]
\begin{eqnarray*}
t^{-1}U\left(t\right)\rightarrow 1/\mu\textrm{,    cuando }t\rightarrow\infty.
\end{eqnarray*}
\end{Teo}



Sup\'ongase que $N\left(t\right)$ es un proceso de renovaci\'on con distribuci\'on $F$ con media finita $\mu$.

\begin{Def}
La funci\'on de renovaci\'on asociada con la distribuci\'on $F$, del proceso $N\left(t\right)$, es
\begin{eqnarray*}
U\left(t\right)=\sum_{n=1}^{\infty}F^{n\star}\left(t\right),\textrm{   }t\geq0,
\end{eqnarray*}
donde $F^{0\star}\left(t\right)=\indora\left(t\geq0\right)$.
\end{Def}


\begin{Prop}
Sup\'ongase que la distribuci\'on de inter-renovaci\'on $F$ tiene densidad $f$. Entonces $U\left(t\right)$ tambi\'en tiene densidad, para $t>0$, y es $U^{'}\left(t\right)=\sum_{n=0}^{\infty}f^{n\star}\left(t\right)$. Adem\'as
\begin{eqnarray*}
\prob\left\{N\left(t\right)>N\left(t-\right)\right\}=0\textrm{,   }t\geq0.
\end{eqnarray*}
\end{Prop}

\begin{Def}
La Transformada de Laplace-Stieljes de $F$ est\'a dada por

\begin{eqnarray*}
\hat{F}\left(\alpha\right)=\int_{\rea_{+}}e^{-\alpha t}dF\left(t\right)\textrm{,  }\alpha\geq0.
\end{eqnarray*}
\end{Def}

Entonces

\begin{eqnarray*}
\hat{U}\left(\alpha\right)=\sum_{n=0}^{\infty}\hat{F^{n\star}}\left(\alpha\right)=\sum_{n=0}^{\infty}\hat{F}\left(\alpha\right)^{n}=\frac{1}{1-\hat{F}\left(\alpha\right)}.
\end{eqnarray*}


\begin{Prop}
La Transformada de Laplace $\hat{U}\left(\alpha\right)$ y $\hat{F}\left(\alpha\right)$ determina una a la otra de manera \'unica por la relaci\'on $\hat{U}\left(\alpha\right)=\frac{1}{1-\hat{F}\left(\alpha\right)}$.
\end{Prop}


\begin{Note}
Un proceso de renovaci\'on $N\left(t\right)$ cuyos tiempos de inter-renovaci\'on tienen media finita, es un proceso Poisson con tasa $\lambda$ si y s\'olo s\'i $\esp\left[U\left(t\right)\right]=\lambda t$, para $t\geq0$.
\end{Note}


\begin{Teo}
Sea $N\left(t\right)$ un proceso puntual simple con puntos de localizaci\'on $T_{n}$ tal que $\eta\left(t\right)=\esp\left[N\left(\right)\right]$ es finita para cada $t$. Entonces para cualquier funci\'on $f:\rea_{+}\rightarrow\rea$,
\begin{eqnarray*}
\esp\left[\sum_{n=1}^{N\left(\right)}f\left(T_{n}\right)\right]=\int_{\left(0,t\right]}f\left(s\right)d\eta\left(s\right)\textrm{,  }t\geq0,
\end{eqnarray*}
suponiendo que la integral exista. Adem\'as si $X_{1},X_{2},\ldots$ son variables aleatorias definidas en el mismo espacio de probabilidad que el proceso $N\left(t\right)$ tal que $\esp\left[X_{n}|T_{n}=s\right]=f\left(s\right)$, independiente de $n$. Entonces
\begin{eqnarray*}
\esp\left[\sum_{n=1}^{N\left(t\right)}X_{n}\right]=\int_{\left(0,t\right]}f\left(s\right)d\eta\left(s\right)\textrm{,  }t\geq0,
\end{eqnarray*} 
suponiendo que la integral exista. 
\end{Teo}

\begin{Coro}[Identidad de Wald para Renovaciones]
Para el proceso de renovaci\'on $N\left(t\right)$,
\begin{eqnarray*}
\esp\left[T_{N\left(t\right)+1}\right]=\mu\esp\left[N\left(t\right)+1\right]\textrm{,  }t\geq0,
\end{eqnarray*}  
\end{Coro}


\begin{Def}
Sea $h\left(t\right)$ funci\'on de valores reales en $\rea$ acotada en intervalos finitos e igual a cero para $t<0$ La ecuaci\'on de renovaci\'on para $h\left(t\right)$ y la distribuci\'on $F$ es

\begin{eqnarray}\label{Ec.Renovacion}
H\left(t\right)=h\left(t\right)+\int_{\left[0,t\right]}H\left(t-s\right)dF\left(s\right)\textrm{,    }t\geq0,
\end{eqnarray}
donde $H\left(t\right)$ es una funci\'on de valores reales. Esto es $H=h+F\star H$. Decimos que $H\left(t\right)$ es soluci\'on de esta ecuaci\'on si satisface la ecuaci\'on, y es acotada en intervalos finitos e iguales a cero para $t<0$.
\end{Def}

\begin{Prop}
La funci\'on $U\star h\left(t\right)$ es la \'unica soluci\'on de la ecuaci\'on de renovaci\'on (\ref{Ec.Renovacion}).
\end{Prop}

\begin{Teo}[Teorema Renovaci\'on Elemental]
\begin{eqnarray*}
t^{-1}U\left(t\right)\rightarrow 1/\mu\textrm{,    cuando }t\rightarrow\infty.
\end{eqnarray*}
\end{Teo}


\begin{Note} Una funci\'on $h:\rea_{+}\rightarrow\rea$ es Directamente Riemann Integrable en los siguientes casos:
\begin{itemize}
\item[a)] $h\left(t\right)\geq0$ es decreciente y Riemann Integrable.
\item[b)] $h$ es continua excepto posiblemente en un conjunto de Lebesgue de medida 0, y $|h\left(t\right)|\leq b\left(t\right)$, donde $b$ es DRI.
\end{itemize}
\end{Note}

\begin{Teo}[Teorema Principal de Renovaci\'on]
Si $F$ es no aritm\'etica y $h\left(t\right)$ es Directamente Riemann Integrable (DRI), entonces

\begin{eqnarray*}
lim_{t\rightarrow\infty}U\star h=\frac{1}{\mu}\int_{\rea_{+}}h\left(s\right)ds.
\end{eqnarray*}
\end{Teo}

\begin{Prop}
Cualquier funci\'on $H\left(t\right)$ acotada en intervalos finitos y que es 0 para $t<0$ puede expresarse como
\begin{eqnarray*}
H\left(t\right)=U\star h\left(t\right)\textrm{,  donde }h\left(t\right)=H\left(t\right)-F\star H\left(t\right)
\end{eqnarray*}
\end{Prop}

\begin{Def}
Un proceso estoc\'astico $X\left(t\right)$ es crudamente regenerativo en un tiempo aleatorio positivo $T$ si
\begin{eqnarray*}
\esp\left[X\left(T+t\right)|T\right]=\esp\left[X\left(t\right)\right]\textrm{, para }t\geq0,\end{eqnarray*}
y con las esperanzas anteriores finitas.
\end{Def}

\begin{Prop}
Sup\'ongase que $X\left(t\right)$ es un proceso crudamente regenerativo en $T$, que tiene distribuci\'on $F$. Si $\esp\left[X\left(t\right)\right]$ es acotado en intervalos finitos, entonces
\begin{eqnarray*}
\esp\left[X\left(t\right)\right]=U\star h\left(t\right)\textrm{,  donde }h\left(t\right)=\esp\left[X\left(t\right)\indora\left(T>t\right)\right].
\end{eqnarray*}
\end{Prop}

\begin{Teo}[Regeneraci\'on Cruda]
Sup\'ongase que $X\left(t\right)$ es un proceso con valores positivo crudamente regenerativo en $T$, y def\'inase $M=\sup\left\{|X\left(t\right)|:t\leq T\right\}$. Si $T$ es no aritm\'etico y $M$ y $MT$ tienen media finita, entonces
\begin{eqnarray*}
lim_{t\rightarrow\infty}\esp\left[X\left(t\right)\right]=\frac{1}{\mu}\int_{\rea_{+}}h\left(s\right)ds,
\end{eqnarray*}
donde $h\left(t\right)=\esp\left[X\left(t\right)\indora\left(T>t\right)\right]$.
\end{Teo}


\begin{Note} Una funci\'on $h:\rea_{+}\rightarrow\rea$ es Directamente Riemann Integrable en los siguientes casos:
\begin{itemize}
\item[a)] $h\left(t\right)\geq0$ es decreciente y Riemann Integrable.
\item[b)] $h$ es continua excepto posiblemente en un conjunto de Lebesgue de medida 0, y $|h\left(t\right)|\leq b\left(t\right)$, donde $b$ es DRI.
\end{itemize}
\end{Note}

\begin{Teo}[Teorema Principal de Renovaci\'on]
Si $F$ es no aritm\'etica y $h\left(t\right)$ es Directamente Riemann Integrable (DRI), entonces

\begin{eqnarray*}
lim_{t\rightarrow\infty}U\star h=\frac{1}{\mu}\int_{\rea_{+}}h\left(s\right)ds.
\end{eqnarray*}
\end{Teo}

\begin{Prop}
Cualquier funci\'on $H\left(t\right)$ acotada en intervalos finitos y que es 0 para $t<0$ puede expresarse como
\begin{eqnarray*}
H\left(t\right)=U\star h\left(t\right)\textrm{,  donde }h\left(t\right)=H\left(t\right)-F\star H\left(t\right)
\end{eqnarray*}
\end{Prop}

\begin{Def}
Un proceso estoc\'astico $X\left(t\right)$ es crudamente regenerativo en un tiempo aleatorio positivo $T$ si
\begin{eqnarray*}
\esp\left[X\left(T+t\right)|T\right]=\esp\left[X\left(t\right)\right]\textrm{, para }t\geq0,\end{eqnarray*}
y con las esperanzas anteriores finitas.
\end{Def}

\begin{Prop}
Sup\'ongase que $X\left(t\right)$ es un proceso crudamente regenerativo en $T$, que tiene distribuci\'on $F$. Si $\esp\left[X\left(t\right)\right]$ es acotado en intervalos finitos, entonces
\begin{eqnarray*}
\esp\left[X\left(t\right)\right]=U\star h\left(t\right)\textrm{,  donde }h\left(t\right)=\esp\left[X\left(t\right)\indora\left(T>t\right)\right].
\end{eqnarray*}
\end{Prop}

\begin{Teo}[Regeneraci\'on Cruda]
Sup\'ongase que $X\left(t\right)$ es un proceso con valores positivo crudamente regenerativo en $T$, y def\'inase $M=\sup\left\{|X\left(t\right)|:t\leq T\right\}$. Si $T$ es no aritm\'etico y $M$ y $MT$ tienen media finita, entonces
\begin{eqnarray*}
lim_{t\rightarrow\infty}\esp\left[X\left(t\right)\right]=\frac{1}{\mu}\int_{\rea_{+}}h\left(s\right)ds,
\end{eqnarray*}
donde $h\left(t\right)=\esp\left[X\left(t\right)\indora\left(T>t\right)\right]$.
\end{Teo}

%________________________________________________________________________
\subsection{Procesos Regenerativos}
%________________________________________________________________________

Para $\left\{X\left(t\right):t\geq0\right\}$ Proceso Estoc\'astico a tiempo continuo con estado de espacios $S$, que es un espacio m\'etrico, con trayectorias continuas por la derecha y con l\'imites por la izquierda c.s. Sea $N\left(t\right)$ un proceso de renovaci\'on en $\rea_{+}$ definido en el mismo espacio de probabilidad que $X\left(t\right)$, con tiempos de renovaci\'on $T$ y tiempos de inter-renovaci\'on $\xi_{n}=T_{n}-T_{n-1}$, con misma distribuci\'on $F$ de media finita $\mu$.



\begin{Def}
Para el proceso $\left\{\left(N\left(t\right),X\left(t\right)\right):t\geq0\right\}$, sus trayectoria muestrales en el intervalo de tiempo $\left[T_{n-1},T_{n}\right)$ est\'an descritas por
\begin{eqnarray*}
\zeta_{n}=\left(\xi_{n},\left\{X\left(T_{n-1}+t\right):0\leq t<\xi_{n}\right\}\right)
\end{eqnarray*}
Este $\zeta_{n}$ es el $n$-\'esimo segmento del proceso. El proceso es regenerativo sobre los tiempos $T_{n}$ si sus segmentos $\zeta_{n}$ son independientes e id\'enticamennte distribuidos.
\end{Def}


\begin{Obs}
Si $\tilde{X}\left(t\right)$ con espacio de estados $\tilde{S}$ es regenerativo sobre $T_{n}$, entonces $X\left(t\right)=f\left(\tilde{X}\left(t\right)\right)$ tambi\'en es regenerativo sobre $T_{n}$, para cualquier funci\'on $f:\tilde{S}\rightarrow S$.
\end{Obs}

\begin{Obs}
Los procesos regenerativos son crudamente regenerativos, pero no al rev\'es.
\end{Obs}

\begin{Def}[Definici\'on Cl\'asica]
Un proceso estoc\'astico $X=\left\{X\left(t\right):t\geq0\right\}$ es llamado regenerativo is existe una variable aleatoria $R_{1}>0$ tal que
\begin{itemize}
\item[i)] $\left\{X\left(t+R_{1}\right):t\geq0\right\}$ es independiente de $\left\{\left\{X\left(t\right):t<R_{1}\right\},\right\}$
\item[ii)] $\left\{X\left(t+R_{1}\right):t\geq0\right\}$ es estoc\'asticamente equivalente a $\left\{X\left(t\right):t>0\right\}$
\end{itemize}

Llamamos a $R_{1}$ tiempo de regeneraci\'on, y decimos que $X$ se regenera en este punto.
\end{Def}

$\left\{X\left(t+R_{1}\right)\right\}$ es regenerativo con tiempo de regeneraci\'on $R_{2}$, independiente de $R_{1}$ pero con la misma distribuci\'on que $R_{1}$. Procediendo de esta manera se obtiene una secuencia de variables aleatorias independientes e id\'enticamente distribuidas $\left\{R_{n}\right\}$ llamados longitudes de ciclo. Si definimos a $Z_{k}\equiv R_{1}+R_{2}+\cdots+R_{k}$, se tiene un proceso de renovaci\'on llamado proceso de renovaci\'on encajado para $X$.

\begin{Note}
Un proceso regenerativo con media de la longitud de ciclo finita es llamado positivo recurrente.
\end{Note}


\begin{Def}
Para $x$ fijo y para cada $t\geq0$, sea $I_{x}\left(t\right)=1$ si $X\left(t\right)\leq x$,  $I_{x}\left(t\right)=0$ en caso contrario, y def\'inanse los tiempos promedio
\begin{eqnarray*}
\overline{X}&=&lim_{t\rightarrow\infty}\frac{1}{t}\int_{0}^{\infty}X\left(u\right)du\\
\prob\left(X_{\infty}\leq x\right)&=&lim_{t\rightarrow\infty}\frac{1}{t}\int_{0}^{\infty}I_{x}\left(u\right)du,
\end{eqnarray*}
cuando estos l\'imites existan.
\end{Def}

Como consecuencia del teorema de Renovaci\'on-Recompensa, se tiene que el primer l\'imite  existe y es igual a la constante
\begin{eqnarray*}
\overline{X}&=&\frac{\esp\left[\int_{0}^{R_{1}}X\left(t\right)dt\right]}{\esp\left[R_{1}\right]},
\end{eqnarray*}
suponiendo que ambas esperanzas son finitas.

\begin{Note}
\begin{itemize}
\item[a)] Si el proceso regenerativo $X$ es positivo recurrente y tiene trayectorias muestrales no negativas, entonces la ecuaci\'on anterior es v\'alida.
\item[b)] Si $X$ es positivo recurrente regenerativo, podemos construir una \'unica versi\'on estacionaria de este proceso, $X_{e}=\left\{X_{e}\left(t\right)\right\}$, donde $X_{e}$ es un proceso estoc\'astico regenerativo y estrictamente estacionario, con distribuci\'on marginal distribuida como $X_{\infty}$
\end{itemize}
\end{Note}

%________________________________________________________________________
\subsection{Procesos Regenerativos}
%________________________________________________________________________

Para $\left\{X\left(t\right):t\geq0\right\}$ Proceso Estoc\'astico a tiempo continuo con estado de espacios $S$, que es un espacio m\'etrico, con trayectorias continuas por la derecha y con l\'imites por la izquierda c.s. Sea $N\left(t\right)$ un proceso de renovaci\'on en $\rea_{+}$ definido en el mismo espacio de probabilidad que $X\left(t\right)$, con tiempos de renovaci\'on $T$ y tiempos de inter-renovaci\'on $\xi_{n}=T_{n}-T_{n-1}$, con misma distribuci\'on $F$ de media finita $\mu$.



\begin{Def}
Para el proceso $\left\{\left(N\left(t\right),X\left(t\right)\right):t\geq0\right\}$, sus trayectoria muestrales en el intervalo de tiempo $\left[T_{n-1},T_{n}\right)$ est\'an descritas por
\begin{eqnarray*}
\zeta_{n}=\left(\xi_{n},\left\{X\left(T_{n-1}+t\right):0\leq t<\xi_{n}\right\}\right)
\end{eqnarray*}
Este $\zeta_{n}$ es el $n$-\'esimo segmento del proceso. El proceso es regenerativo sobre los tiempos $T_{n}$ si sus segmentos $\zeta_{n}$ son independientes e id\'enticamennte distribuidos.
\end{Def}


\begin{Obs}
Si $\tilde{X}\left(t\right)$ con espacio de estados $\tilde{S}$ es regenerativo sobre $T_{n}$, entonces $X\left(t\right)=f\left(\tilde{X}\left(t\right)\right)$ tambi\'en es regenerativo sobre $T_{n}$, para cualquier funci\'on $f:\tilde{S}\rightarrow S$.
\end{Obs}

\begin{Obs}
Los procesos regenerativos son crudamente regenerativos, pero no al rev\'es.
\end{Obs}

\begin{Def}[Definici\'on Cl\'asica]
Un proceso estoc\'astico $X=\left\{X\left(t\right):t\geq0\right\}$ es llamado regenerativo is existe una variable aleatoria $R_{1}>0$ tal que
\begin{itemize}
\item[i)] $\left\{X\left(t+R_{1}\right):t\geq0\right\}$ es independiente de $\left\{\left\{X\left(t\right):t<R_{1}\right\},\right\}$
\item[ii)] $\left\{X\left(t+R_{1}\right):t\geq0\right\}$ es estoc\'asticamente equivalente a $\left\{X\left(t\right):t>0\right\}$
\end{itemize}

Llamamos a $R_{1}$ tiempo de regeneraci\'on, y decimos que $X$ se regenera en este punto.
\end{Def}

$\left\{X\left(t+R_{1}\right)\right\}$ es regenerativo con tiempo de regeneraci\'on $R_{2}$, independiente de $R_{1}$ pero con la misma distribuci\'on que $R_{1}$. Procediendo de esta manera se obtiene una secuencia de variables aleatorias independientes e id\'enticamente distribuidas $\left\{R_{n}\right\}$ llamados longitudes de ciclo. Si definimos a $Z_{k}\equiv R_{1}+R_{2}+\cdots+R_{k}$, se tiene un proceso de renovaci\'on llamado proceso de renovaci\'on encajado para $X$.

\begin{Note}
Un proceso regenerativo con media de la longitud de ciclo finita es llamado positivo recurrente.
\end{Note}


\begin{Def}
Para $x$ fijo y para cada $t\geq0$, sea $I_{x}\left(t\right)=1$ si $X\left(t\right)\leq x$,  $I_{x}\left(t\right)=0$ en caso contrario, y def\'inanse los tiempos promedio
\begin{eqnarray*}
\overline{X}&=&lim_{t\rightarrow\infty}\frac{1}{t}\int_{0}^{\infty}X\left(u\right)du\\
\prob\left(X_{\infty}\leq x\right)&=&lim_{t\rightarrow\infty}\frac{1}{t}\int_{0}^{\infty}I_{x}\left(u\right)du,
\end{eqnarray*}
cuando estos l\'imites existan.
\end{Def}

Como consecuencia del teorema de Renovaci\'on-Recompensa, se tiene que el primer l\'imite  existe y es igual a la constante
\begin{eqnarray*}
\overline{X}&=&\frac{\esp\left[\int_{0}^{R_{1}}X\left(t\right)dt\right]}{\esp\left[R_{1}\right]},
\end{eqnarray*}
suponiendo que ambas esperanzas son finitas.

\begin{Note}
\begin{itemize}
\item[a)] Si el proceso regenerativo $X$ es positivo recurrente y tiene trayectorias muestrales no negativas, entonces la ecuaci\'on anterior es v\'alida.
\item[b)] Si $X$ es positivo recurrente regenerativo, podemos construir una \'unica versi\'on estacionaria de este proceso, $X_{e}=\left\{X_{e}\left(t\right)\right\}$, donde $X_{e}$ es un proceso estoc\'astico regenerativo y estrictamente estacionario, con distribuci\'on marginal distribuida como $X_{\infty}$
\end{itemize}
\end{Note}
%__________________________________________________________________________________________
\subsection{Procesos Regenerativos Estacionarios - Stidham \cite{Stidham}}
%__________________________________________________________________________________________


Un proceso estoc\'astico a tiempo continuo $\left\{V\left(t\right),t\geq0\right\}$ es un proceso regenerativo si existe una sucesi\'on de variables aleatorias independientes e id\'enticamente distribuidas $\left\{X_{1},X_{2},\ldots\right\}$, sucesi\'on de renovaci\'on, tal que para cualquier conjunto de Borel $A$, 

\begin{eqnarray*}
\prob\left\{V\left(t\right)\in A|X_{1}+X_{2}+\cdots+X_{R\left(t\right)}=s,\left\{V\left(\tau\right),\tau<s\right\}\right\}=\prob\left\{V\left(t-s\right)\in A|X_{1}>t-s\right\},
\end{eqnarray*}
para todo $0\leq s\leq t$, donde $R\left(t\right)=\max\left\{X_{1}+X_{2}+\cdots+X_{j}\leq t\right\}=$n\'umero de renovaciones ({\emph{puntos de regeneraci\'on}}) que ocurren en $\left[0,t\right]$. El intervalo $\left[0,X_{1}\right)$ es llamado {\emph{primer ciclo de regeneraci\'on}} de $\left\{V\left(t \right),t\geq0\right\}$, $\left[X_{1},X_{1}+X_{2}\right)$ el {\emph{segundo ciclo de regeneraci\'on}}, y as\'i sucesivamente.

Sea $X=X_{1}$ y sea $F$ la funci\'on de distrbuci\'on de $X$


\begin{Def}
Se define el proceso estacionario, $\left\{V^{*}\left(t\right),t\geq0\right\}$, para $\left\{V\left(t\right),t\geq0\right\}$ por

\begin{eqnarray*}
\prob\left\{V\left(t\right)\in A\right\}=\frac{1}{\esp\left[X\right]}\int_{0}^{\infty}\prob\left\{V\left(t+x\right)\in A|X>x\right\}\left(1-F\left(x\right)\right)dx,
\end{eqnarray*} 
para todo $t\geq0$ y todo conjunto de Borel $A$.
\end{Def}

\begin{Def}
Una distribuci\'on se dice que es {\emph{aritm\'etica}} si todos sus puntos de incremento son m\'ultiplos de la forma $0,\lambda, 2\lambda,\ldots$ para alguna $\lambda>0$ entera.
\end{Def}


\begin{Def}
Una modificaci\'on medible de un proceso $\left\{V\left(t\right),t\geq0\right\}$, es una versi\'on de este, $\left\{V\left(t,w\right)\right\}$ conjuntamente medible para $t\geq0$ y para $w\in S$, $S$ espacio de estados para $\left\{V\left(t\right),t\geq0\right\}$.
\end{Def}

\begin{Teo}
Sea $\left\{V\left(t\right),t\geq\right\}$ un proceso regenerativo no negativo con modificaci\'on medible. Sea $\esp\left[X\right]<\infty$. Entonces el proceso estacionario dado por la ecuaci\'on anterior est\'a bien definido y tiene funci\'on de distribuci\'on independiente de $t$, adem\'as
\begin{itemize}
\item[i)] \begin{eqnarray*}
\esp\left[V^{*}\left(0\right)\right]&=&\frac{\esp\left[\int_{0}^{X}V\left(s\right)ds\right]}{\esp\left[X\right]}\end{eqnarray*}
\item[ii)] Si $\esp\left[V^{*}\left(0\right)\right]<\infty$, equivalentemente, si $\esp\left[\int_{0}^{X}V\left(s\right)ds\right]<\infty$,entonces
\begin{eqnarray*}
\frac{\int_{0}^{t}V\left(s\right)ds}{t}\rightarrow\frac{\esp\left[\int_{0}^{X}V\left(s\right)ds\right]}{\esp\left[X\right]}
\end{eqnarray*}
con probabilidad 1 y en media, cuando $t\rightarrow\infty$.
\end{itemize}
\end{Teo}


%__________________________________________________________________________________________
\subsection{Procesos Regenerativos Estacionarios - Stidham \cite{Stidham}}
%__________________________________________________________________________________________


Un proceso estoc\'astico a tiempo continuo $\left\{V\left(t\right),t\geq0\right\}$ es un proceso regenerativo si existe una sucesi\'on de variables aleatorias independientes e id\'enticamente distribuidas $\left\{X_{1},X_{2},\ldots\right\}$, sucesi\'on de renovaci\'on, tal que para cualquier conjunto de Borel $A$, 

\begin{eqnarray*}
\prob\left\{V\left(t\right)\in A|X_{1}+X_{2}+\cdots+X_{R\left(t\right)}=s,\left\{V\left(\tau\right),\tau<s\right\}\right\}=\prob\left\{V\left(t-s\right)\in A|X_{1}>t-s\right\},
\end{eqnarray*}
para todo $0\leq s\leq t$, donde $R\left(t\right)=\max\left\{X_{1}+X_{2}+\cdots+X_{j}\leq t\right\}=$n\'umero de renovaciones ({\emph{puntos de regeneraci\'on}}) que ocurren en $\left[0,t\right]$. El intervalo $\left[0,X_{1}\right)$ es llamado {\emph{primer ciclo de regeneraci\'on}} de $\left\{V\left(t \right),t\geq0\right\}$, $\left[X_{1},X_{1}+X_{2}\right)$ el {\emph{segundo ciclo de regeneraci\'on}}, y as\'i sucesivamente.

Sea $X=X_{1}$ y sea $F$ la funci\'on de distrbuci\'on de $X$


\begin{Def}
Se define el proceso estacionario, $\left\{V^{*}\left(t\right),t\geq0\right\}$, para $\left\{V\left(t\right),t\geq0\right\}$ por

\begin{eqnarray*}
\prob\left\{V\left(t\right)\in A\right\}=\frac{1}{\esp\left[X\right]}\int_{0}^{\infty}\prob\left\{V\left(t+x\right)\in A|X>x\right\}\left(1-F\left(x\right)\right)dx,
\end{eqnarray*} 
para todo $t\geq0$ y todo conjunto de Borel $A$.
\end{Def}

\begin{Def}
Una distribuci\'on se dice que es {\emph{aritm\'etica}} si todos sus puntos de incremento son m\'ultiplos de la forma $0,\lambda, 2\lambda,\ldots$ para alguna $\lambda>0$ entera.
\end{Def}


\begin{Def}
Una modificaci\'on medible de un proceso $\left\{V\left(t\right),t\geq0\right\}$, es una versi\'on de este, $\left\{V\left(t,w\right)\right\}$ conjuntamente medible para $t\geq0$ y para $w\in S$, $S$ espacio de estados para $\left\{V\left(t\right),t\geq0\right\}$.
\end{Def}

\begin{Teo}
Sea $\left\{V\left(t\right),t\geq\right\}$ un proceso regenerativo no negativo con modificaci\'on medible. Sea $\esp\left[X\right]<\infty$. Entonces el proceso estacionario dado por la ecuaci\'on anterior est\'a bien definido y tiene funci\'on de distribuci\'on independiente de $t$, adem\'as
\begin{itemize}
\item[i)] \begin{eqnarray*}
\esp\left[V^{*}\left(0\right)\right]&=&\frac{\esp\left[\int_{0}^{X}V\left(s\right)ds\right]}{\esp\left[X\right]}\end{eqnarray*}
\item[ii)] Si $\esp\left[V^{*}\left(0\right)\right]<\infty$, equivalentemente, si $\esp\left[\int_{0}^{X}V\left(s\right)ds\right]<\infty$,entonces
\begin{eqnarray*}
\frac{\int_{0}^{t}V\left(s\right)ds}{t}\rightarrow\frac{\esp\left[\int_{0}^{X}V\left(s\right)ds\right]}{\esp\left[X\right]}
\end{eqnarray*}
con probabilidad 1 y en media, cuando $t\rightarrow\infty$.
\end{itemize}
\end{Teo}
%
%___________________________________________________________________________________________
%\vspace{5.5cm}
%\chapter{Cadenas de Markov estacionarias}
%\vspace{-1.0cm}
%___________________________________________________________________________________________
%
\subsection{Propiedades de los Procesos de Renovaci\'on}
%___________________________________________________________________________________________
%

Los tiempos $T_{n}$ est\'an relacionados con los conteos de $N\left(t\right)$ por

\begin{eqnarray*}
\left\{N\left(t\right)\geq n\right\}&=&\left\{T_{n}\leq t\right\}\\
T_{N\left(t\right)}\leq &t&<T_{N\left(t\right)+1},
\end{eqnarray*}

adem\'as $N\left(T_{n}\right)=n$, y 

\begin{eqnarray*}
N\left(t\right)=\max\left\{n:T_{n}\leq t\right\}=\min\left\{n:T_{n+1}>t\right\}
\end{eqnarray*}

Por propiedades de la convoluci\'on se sabe que

\begin{eqnarray*}
P\left\{T_{n}\leq t\right\}=F^{n\star}\left(t\right)
\end{eqnarray*}
que es la $n$-\'esima convoluci\'on de $F$. Entonces 

\begin{eqnarray*}
\left\{N\left(t\right)\geq n\right\}&=&\left\{T_{n}\leq t\right\}\\
P\left\{N\left(t\right)\leq n\right\}&=&1-F^{\left(n+1\right)\star}\left(t\right)
\end{eqnarray*}

Adem\'as usando el hecho de que $\esp\left[N\left(t\right)\right]=\sum_{n=1}^{\infty}P\left\{N\left(t\right)\geq n\right\}$
se tiene que

\begin{eqnarray*}
\esp\left[N\left(t\right)\right]=\sum_{n=1}^{\infty}F^{n\star}\left(t\right)
\end{eqnarray*}

\begin{Prop}
Para cada $t\geq0$, la funci\'on generadora de momentos $\esp\left[e^{\alpha N\left(t\right)}\right]$ existe para alguna $\alpha$ en una vecindad del 0, y de aqu\'i que $\esp\left[N\left(t\right)^{m}\right]<\infty$, para $m\geq1$.
\end{Prop}


\begin{Note}
Si el primer tiempo de renovaci\'on $\xi_{1}$ no tiene la misma distribuci\'on que el resto de las $\xi_{n}$, para $n\geq2$, a $N\left(t\right)$ se le llama Proceso de Renovaci\'on retardado, donde si $\xi$ tiene distribuci\'on $G$, entonces el tiempo $T_{n}$ de la $n$-\'esima renovaci\'on tiene distribuci\'on $G\star F^{\left(n-1\right)\star}\left(t\right)$
\end{Note}


\begin{Teo}
Para una constante $\mu\leq\infty$ ( o variable aleatoria), las siguientes expresiones son equivalentes:

\begin{eqnarray}
lim_{n\rightarrow\infty}n^{-1}T_{n}&=&\mu,\textrm{ c.s.}\\
lim_{t\rightarrow\infty}t^{-1}N\left(t\right)&=&1/\mu,\textrm{ c.s.}
\end{eqnarray}
\end{Teo}


Es decir, $T_{n}$ satisface la Ley Fuerte de los Grandes N\'umeros s\'i y s\'olo s\'i $N\left/t\right)$ la cumple.


\begin{Coro}[Ley Fuerte de los Grandes N\'umeros para Procesos de Renovaci\'on]
Si $N\left(t\right)$ es un proceso de renovaci\'on cuyos tiempos de inter-renovaci\'on tienen media $\mu\leq\infty$, entonces
\begin{eqnarray}
t^{-1}N\left(t\right)\rightarrow 1/\mu,\textrm{ c.s. cuando }t\rightarrow\infty.
\end{eqnarray}

\end{Coro}


Considerar el proceso estoc\'astico de valores reales $\left\{Z\left(t\right):t\geq0\right\}$ en el mismo espacio de probabilidad que $N\left(t\right)$

\begin{Def}
Para el proceso $\left\{Z\left(t\right):t\geq0\right\}$ se define la fluctuaci\'on m\'axima de $Z\left(t\right)$ en el intervalo $\left(T_{n-1},T_{n}\right]$:
\begin{eqnarray*}
M_{n}=\sup_{T_{n-1}<t\leq T_{n}}|Z\left(t\right)-Z\left(T_{n-1}\right)|
\end{eqnarray*}
\end{Def}

\begin{Teo}
Sup\'ongase que $n^{-1}T_{n}\rightarrow\mu$ c.s. cuando $n\rightarrow\infty$, donde $\mu\leq\infty$ es una constante o variable aleatoria. Sea $a$ una constante o variable aleatoria que puede ser infinita cuando $\mu$ es finita, y considere las expresiones l\'imite:
\begin{eqnarray}
lim_{n\rightarrow\infty}n^{-1}Z\left(T_{n}\right)&=&a,\textrm{ c.s.}\\
lim_{t\rightarrow\infty}t^{-1}Z\left(t\right)&=&a/\mu,\textrm{ c.s.}
\end{eqnarray}
La segunda expresi\'on implica la primera. Conversamente, la primera implica la segunda si el proceso $Z\left(t\right)$ es creciente, o si $lim_{n\rightarrow\infty}n^{-1}M_{n}=0$ c.s.
\end{Teo}

\begin{Coro}
Si $N\left(t\right)$ es un proceso de renovaci\'on, y $\left(Z\left(T_{n}\right)-Z\left(T_{n-1}\right),M_{n}\right)$, para $n\geq1$, son variables aleatorias independientes e id\'enticamente distribuidas con media finita, entonces,
\begin{eqnarray}
lim_{t\rightarrow\infty}t^{-1}Z\left(t\right)\rightarrow\frac{\esp\left[Z\left(T_{1}\right)-Z\left(T_{0}\right)\right]}{\esp\left[T_{1}\right]},\textrm{ c.s. cuando  }t\rightarrow\infty.
\end{eqnarray}
\end{Coro}


%___________________________________________________________________________________________
%
%\subsection{Propiedades de los Procesos de Renovaci\'on}
%___________________________________________________________________________________________
%

Los tiempos $T_{n}$ est\'an relacionados con los conteos de $N\left(t\right)$ por

\begin{eqnarray*}
\left\{N\left(t\right)\geq n\right\}&=&\left\{T_{n}\leq t\right\}\\
T_{N\left(t\right)}\leq &t&<T_{N\left(t\right)+1},
\end{eqnarray*}

adem\'as $N\left(T_{n}\right)=n$, y 

\begin{eqnarray*}
N\left(t\right)=\max\left\{n:T_{n}\leq t\right\}=\min\left\{n:T_{n+1}>t\right\}
\end{eqnarray*}

Por propiedades de la convoluci\'on se sabe que

\begin{eqnarray*}
P\left\{T_{n}\leq t\right\}=F^{n\star}\left(t\right)
\end{eqnarray*}
que es la $n$-\'esima convoluci\'on de $F$. Entonces 

\begin{eqnarray*}
\left\{N\left(t\right)\geq n\right\}&=&\left\{T_{n}\leq t\right\}\\
P\left\{N\left(t\right)\leq n\right\}&=&1-F^{\left(n+1\right)\star}\left(t\right)
\end{eqnarray*}

Adem\'as usando el hecho de que $\esp\left[N\left(t\right)\right]=\sum_{n=1}^{\infty}P\left\{N\left(t\right)\geq n\right\}$
se tiene que

\begin{eqnarray*}
\esp\left[N\left(t\right)\right]=\sum_{n=1}^{\infty}F^{n\star}\left(t\right)
\end{eqnarray*}

\begin{Prop}
Para cada $t\geq0$, la funci\'on generadora de momentos $\esp\left[e^{\alpha N\left(t\right)}\right]$ existe para alguna $\alpha$ en una vecindad del 0, y de aqu\'i que $\esp\left[N\left(t\right)^{m}\right]<\infty$, para $m\geq1$.
\end{Prop}


\begin{Note}
Si el primer tiempo de renovaci\'on $\xi_{1}$ no tiene la misma distribuci\'on que el resto de las $\xi_{n}$, para $n\geq2$, a $N\left(t\right)$ se le llama Proceso de Renovaci\'on retardado, donde si $\xi$ tiene distribuci\'on $G$, entonces el tiempo $T_{n}$ de la $n$-\'esima renovaci\'on tiene distribuci\'on $G\star F^{\left(n-1\right)\star}\left(t\right)$
\end{Note}


\begin{Teo}
Para una constante $\mu\leq\infty$ ( o variable aleatoria), las siguientes expresiones son equivalentes:

\begin{eqnarray}
lim_{n\rightarrow\infty}n^{-1}T_{n}&=&\mu,\textrm{ c.s.}\\
lim_{t\rightarrow\infty}t^{-1}N\left(t\right)&=&1/\mu,\textrm{ c.s.}
\end{eqnarray}
\end{Teo}


Es decir, $T_{n}$ satisface la Ley Fuerte de los Grandes N\'umeros s\'i y s\'olo s\'i $N\left/t\right)$ la cumple.


\begin{Coro}[Ley Fuerte de los Grandes N\'umeros para Procesos de Renovaci\'on]
Si $N\left(t\right)$ es un proceso de renovaci\'on cuyos tiempos de inter-renovaci\'on tienen media $\mu\leq\infty$, entonces
\begin{eqnarray}
t^{-1}N\left(t\right)\rightarrow 1/\mu,\textrm{ c.s. cuando }t\rightarrow\infty.
\end{eqnarray}

\end{Coro}


Considerar el proceso estoc\'astico de valores reales $\left\{Z\left(t\right):t\geq0\right\}$ en el mismo espacio de probabilidad que $N\left(t\right)$

\begin{Def}
Para el proceso $\left\{Z\left(t\right):t\geq0\right\}$ se define la fluctuaci\'on m\'axima de $Z\left(t\right)$ en el intervalo $\left(T_{n-1},T_{n}\right]$:
\begin{eqnarray*}
M_{n}=\sup_{T_{n-1}<t\leq T_{n}}|Z\left(t\right)-Z\left(T_{n-1}\right)|
\end{eqnarray*}
\end{Def}

\begin{Teo}
Sup\'ongase que $n^{-1}T_{n}\rightarrow\mu$ c.s. cuando $n\rightarrow\infty$, donde $\mu\leq\infty$ es una constante o variable aleatoria. Sea $a$ una constante o variable aleatoria que puede ser infinita cuando $\mu$ es finita, y considere las expresiones l\'imite:
\begin{eqnarray}
lim_{n\rightarrow\infty}n^{-1}Z\left(T_{n}\right)&=&a,\textrm{ c.s.}\\
lim_{t\rightarrow\infty}t^{-1}Z\left(t\right)&=&a/\mu,\textrm{ c.s.}
\end{eqnarray}
La segunda expresi\'on implica la primera. Conversamente, la primera implica la segunda si el proceso $Z\left(t\right)$ es creciente, o si $lim_{n\rightarrow\infty}n^{-1}M_{n}=0$ c.s.
\end{Teo}

\begin{Coro}
Si $N\left(t\right)$ es un proceso de renovaci\'on, y $\left(Z\left(T_{n}\right)-Z\left(T_{n-1}\right),M_{n}\right)$, para $n\geq1$, son variables aleatorias independientes e id\'enticamente distribuidas con media finita, entonces,
\begin{eqnarray}
lim_{t\rightarrow\infty}t^{-1}Z\left(t\right)\rightarrow\frac{\esp\left[Z\left(T_{1}\right)-Z\left(T_{0}\right)\right]}{\esp\left[T_{1}\right]},\textrm{ c.s. cuando  }t\rightarrow\infty.
\end{eqnarray}
\end{Coro}

%___________________________________________________________________________________________
%
%\subsection{Propiedades de los Procesos de Renovaci\'on}
%___________________________________________________________________________________________
%

Los tiempos $T_{n}$ est\'an relacionados con los conteos de $N\left(t\right)$ por

\begin{eqnarray*}
\left\{N\left(t\right)\geq n\right\}&=&\left\{T_{n}\leq t\right\}\\
T_{N\left(t\right)}\leq &t&<T_{N\left(t\right)+1},
\end{eqnarray*}

adem\'as $N\left(T_{n}\right)=n$, y 

\begin{eqnarray*}
N\left(t\right)=\max\left\{n:T_{n}\leq t\right\}=\min\left\{n:T_{n+1}>t\right\}
\end{eqnarray*}

Por propiedades de la convoluci\'on se sabe que

\begin{eqnarray*}
P\left\{T_{n}\leq t\right\}=F^{n\star}\left(t\right)
\end{eqnarray*}
que es la $n$-\'esima convoluci\'on de $F$. Entonces 

\begin{eqnarray*}
\left\{N\left(t\right)\geq n\right\}&=&\left\{T_{n}\leq t\right\}\\
P\left\{N\left(t\right)\leq n\right\}&=&1-F^{\left(n+1\right)\star}\left(t\right)
\end{eqnarray*}

Adem\'as usando el hecho de que $\esp\left[N\left(t\right)\right]=\sum_{n=1}^{\infty}P\left\{N\left(t\right)\geq n\right\}$
se tiene que

\begin{eqnarray*}
\esp\left[N\left(t\right)\right]=\sum_{n=1}^{\infty}F^{n\star}\left(t\right)
\end{eqnarray*}

\begin{Prop}
Para cada $t\geq0$, la funci\'on generadora de momentos $\esp\left[e^{\alpha N\left(t\right)}\right]$ existe para alguna $\alpha$ en una vecindad del 0, y de aqu\'i que $\esp\left[N\left(t\right)^{m}\right]<\infty$, para $m\geq1$.
\end{Prop}


\begin{Note}
Si el primer tiempo de renovaci\'on $\xi_{1}$ no tiene la misma distribuci\'on que el resto de las $\xi_{n}$, para $n\geq2$, a $N\left(t\right)$ se le llama Proceso de Renovaci\'on retardado, donde si $\xi$ tiene distribuci\'on $G$, entonces el tiempo $T_{n}$ de la $n$-\'esima renovaci\'on tiene distribuci\'on $G\star F^{\left(n-1\right)\star}\left(t\right)$
\end{Note}


\begin{Teo}
Para una constante $\mu\leq\infty$ ( o variable aleatoria), las siguientes expresiones son equivalentes:

\begin{eqnarray}
lim_{n\rightarrow\infty}n^{-1}T_{n}&=&\mu,\textrm{ c.s.}\\
lim_{t\rightarrow\infty}t^{-1}N\left(t\right)&=&1/\mu,\textrm{ c.s.}
\end{eqnarray}
\end{Teo}


Es decir, $T_{n}$ satisface la Ley Fuerte de los Grandes N\'umeros s\'i y s\'olo s\'i $N\left/t\right)$ la cumple.


\begin{Coro}[Ley Fuerte de los Grandes N\'umeros para Procesos de Renovaci\'on]
Si $N\left(t\right)$ es un proceso de renovaci\'on cuyos tiempos de inter-renovaci\'on tienen media $\mu\leq\infty$, entonces
\begin{eqnarray}
t^{-1}N\left(t\right)\rightarrow 1/\mu,\textrm{ c.s. cuando }t\rightarrow\infty.
\end{eqnarray}

\end{Coro}


Considerar el proceso estoc\'astico de valores reales $\left\{Z\left(t\right):t\geq0\right\}$ en el mismo espacio de probabilidad que $N\left(t\right)$

\begin{Def}
Para el proceso $\left\{Z\left(t\right):t\geq0\right\}$ se define la fluctuaci\'on m\'axima de $Z\left(t\right)$ en el intervalo $\left(T_{n-1},T_{n}\right]$:
\begin{eqnarray*}
M_{n}=\sup_{T_{n-1}<t\leq T_{n}}|Z\left(t\right)-Z\left(T_{n-1}\right)|
\end{eqnarray*}
\end{Def}

\begin{Teo}
Sup\'ongase que $n^{-1}T_{n}\rightarrow\mu$ c.s. cuando $n\rightarrow\infty$, donde $\mu\leq\infty$ es una constante o variable aleatoria. Sea $a$ una constante o variable aleatoria que puede ser infinita cuando $\mu$ es finita, y considere las expresiones l\'imite:
\begin{eqnarray}
lim_{n\rightarrow\infty}n^{-1}Z\left(T_{n}\right)&=&a,\textrm{ c.s.}\\
lim_{t\rightarrow\infty}t^{-1}Z\left(t\right)&=&a/\mu,\textrm{ c.s.}
\end{eqnarray}
La segunda expresi\'on implica la primera. Conversamente, la primera implica la segunda si el proceso $Z\left(t\right)$ es creciente, o si $lim_{n\rightarrow\infty}n^{-1}M_{n}=0$ c.s.
\end{Teo}

\begin{Coro}
Si $N\left(t\right)$ es un proceso de renovaci\'on, y $\left(Z\left(T_{n}\right)-Z\left(T_{n-1}\right),M_{n}\right)$, para $n\geq1$, son variables aleatorias independientes e id\'enticamente distribuidas con media finita, entonces,
\begin{eqnarray}
lim_{t\rightarrow\infty}t^{-1}Z\left(t\right)\rightarrow\frac{\esp\left[Z\left(T_{1}\right)-Z\left(T_{0}\right)\right]}{\esp\left[T_{1}\right]},\textrm{ c.s. cuando  }t\rightarrow\infty.
\end{eqnarray}
\end{Coro}



%___________________________________________________________________________________________
%
\subsection{Propiedades de los Procesos de Renovaci\'on}
%___________________________________________________________________________________________
%

Los tiempos $T_{n}$ est\'an relacionados con los conteos de $N\left(t\right)$ por

\begin{eqnarray*}
\left\{N\left(t\right)\geq n\right\}&=&\left\{T_{n}\leq t\right\}\\
T_{N\left(t\right)}\leq &t&<T_{N\left(t\right)+1},
\end{eqnarray*}

adem\'as $N\left(T_{n}\right)=n$, y 

\begin{eqnarray*}
N\left(t\right)=\max\left\{n:T_{n}\leq t\right\}=\min\left\{n:T_{n+1}>t\right\}
\end{eqnarray*}

Por propiedades de la convoluci\'on se sabe que

\begin{eqnarray*}
P\left\{T_{n}\leq t\right\}=F^{n\star}\left(t\right)
\end{eqnarray*}
que es la $n$-\'esima convoluci\'on de $F$. Entonces 

\begin{eqnarray*}
\left\{N\left(t\right)\geq n\right\}&=&\left\{T_{n}\leq t\right\}\\
P\left\{N\left(t\right)\leq n\right\}&=&1-F^{\left(n+1\right)\star}\left(t\right)
\end{eqnarray*}

Adem\'as usando el hecho de que $\esp\left[N\left(t\right)\right]=\sum_{n=1}^{\infty}P\left\{N\left(t\right)\geq n\right\}$
se tiene que

\begin{eqnarray*}
\esp\left[N\left(t\right)\right]=\sum_{n=1}^{\infty}F^{n\star}\left(t\right)
\end{eqnarray*}

\begin{Prop}
Para cada $t\geq0$, la funci\'on generadora de momentos $\esp\left[e^{\alpha N\left(t\right)}\right]$ existe para alguna $\alpha$ en una vecindad del 0, y de aqu\'i que $\esp\left[N\left(t\right)^{m}\right]<\infty$, para $m\geq1$.
\end{Prop}


\begin{Note}
Si el primer tiempo de renovaci\'on $\xi_{1}$ no tiene la misma distribuci\'on que el resto de las $\xi_{n}$, para $n\geq2$, a $N\left(t\right)$ se le llama Proceso de Renovaci\'on retardado, donde si $\xi$ tiene distribuci\'on $G$, entonces el tiempo $T_{n}$ de la $n$-\'esima renovaci\'on tiene distribuci\'on $G\star F^{\left(n-1\right)\star}\left(t\right)$
\end{Note}


\begin{Teo}
Para una constante $\mu\leq\infty$ ( o variable aleatoria), las siguientes expresiones son equivalentes:

\begin{eqnarray}
lim_{n\rightarrow\infty}n^{-1}T_{n}&=&\mu,\textrm{ c.s.}\\
lim_{t\rightarrow\infty}t^{-1}N\left(t\right)&=&1/\mu,\textrm{ c.s.}
\end{eqnarray}
\end{Teo}


Es decir, $T_{n}$ satisface la Ley Fuerte de los Grandes N\'umeros s\'i y s\'olo s\'i $N\left/t\right)$ la cumple.


\begin{Coro}[Ley Fuerte de los Grandes N\'umeros para Procesos de Renovaci\'on]
Si $N\left(t\right)$ es un proceso de renovaci\'on cuyos tiempos de inter-renovaci\'on tienen media $\mu\leq\infty$, entonces
\begin{eqnarray}
t^{-1}N\left(t\right)\rightarrow 1/\mu,\textrm{ c.s. cuando }t\rightarrow\infty.
\end{eqnarray}

\end{Coro}


Considerar el proceso estoc\'astico de valores reales $\left\{Z\left(t\right):t\geq0\right\}$ en el mismo espacio de probabilidad que $N\left(t\right)$

\begin{Def}
Para el proceso $\left\{Z\left(t\right):t\geq0\right\}$ se define la fluctuaci\'on m\'axima de $Z\left(t\right)$ en el intervalo $\left(T_{n-1},T_{n}\right]$:
\begin{eqnarray*}
M_{n}=\sup_{T_{n-1}<t\leq T_{n}}|Z\left(t\right)-Z\left(T_{n-1}\right)|
\end{eqnarray*}
\end{Def}

\begin{Teo}
Sup\'ongase que $n^{-1}T_{n}\rightarrow\mu$ c.s. cuando $n\rightarrow\infty$, donde $\mu\leq\infty$ es una constante o variable aleatoria. Sea $a$ una constante o variable aleatoria que puede ser infinita cuando $\mu$ es finita, y considere las expresiones l\'imite:
\begin{eqnarray}
lim_{n\rightarrow\infty}n^{-1}Z\left(T_{n}\right)&=&a,\textrm{ c.s.}\\
lim_{t\rightarrow\infty}t^{-1}Z\left(t\right)&=&a/\mu,\textrm{ c.s.}
\end{eqnarray}
La segunda expresi\'on implica la primera. Conversamente, la primera implica la segunda si el proceso $Z\left(t\right)$ es creciente, o si $lim_{n\rightarrow\infty}n^{-1}M_{n}=0$ c.s.
\end{Teo}

\begin{Coro}
Si $N\left(t\right)$ es un proceso de renovaci\'on, y $\left(Z\left(T_{n}\right)-Z\left(T_{n-1}\right),M_{n}\right)$, para $n\geq1$, son variables aleatorias independientes e id\'enticamente distribuidas con media finita, entonces,
\begin{eqnarray}
lim_{t\rightarrow\infty}t^{-1}Z\left(t\right)\rightarrow\frac{\esp\left[Z\left(T_{1}\right)-Z\left(T_{0}\right)\right]}{\esp\left[T_{1}\right]},\textrm{ c.s. cuando  }t\rightarrow\infty.
\end{eqnarray}
\end{Coro}




%__________________________________________________________________________________________
\subsection{Procesos Regenerativos Estacionarios - Stidham \cite{Stidham}}
%__________________________________________________________________________________________


Un proceso estoc\'astico a tiempo continuo $\left\{V\left(t\right),t\geq0\right\}$ es un proceso regenerativo si existe una sucesi\'on de variables aleatorias independientes e id\'enticamente distribuidas $\left\{X_{1},X_{2},\ldots\right\}$, sucesi\'on de renovaci\'on, tal que para cualquier conjunto de Borel $A$, 

\begin{eqnarray*}
\prob\left\{V\left(t\right)\in A|X_{1}+X_{2}+\cdots+X_{R\left(t\right)}=s,\left\{V\left(\tau\right),\tau<s\right\}\right\}=\prob\left\{V\left(t-s\right)\in A|X_{1}>t-s\right\},
\end{eqnarray*}
para todo $0\leq s\leq t$, donde $R\left(t\right)=\max\left\{X_{1}+X_{2}+\cdots+X_{j}\leq t\right\}=$n\'umero de renovaciones ({\emph{puntos de regeneraci\'on}}) que ocurren en $\left[0,t\right]$. El intervalo $\left[0,X_{1}\right)$ es llamado {\emph{primer ciclo de regeneraci\'on}} de $\left\{V\left(t \right),t\geq0\right\}$, $\left[X_{1},X_{1}+X_{2}\right)$ el {\emph{segundo ciclo de regeneraci\'on}}, y as\'i sucesivamente.

Sea $X=X_{1}$ y sea $F$ la funci\'on de distrbuci\'on de $X$


\begin{Def}
Se define el proceso estacionario, $\left\{V^{*}\left(t\right),t\geq0\right\}$, para $\left\{V\left(t\right),t\geq0\right\}$ por

\begin{eqnarray*}
\prob\left\{V\left(t\right)\in A\right\}=\frac{1}{\esp\left[X\right]}\int_{0}^{\infty}\prob\left\{V\left(t+x\right)\in A|X>x\right\}\left(1-F\left(x\right)\right)dx,
\end{eqnarray*} 
para todo $t\geq0$ y todo conjunto de Borel $A$.
\end{Def}

\begin{Def}
Una distribuci\'on se dice que es {\emph{aritm\'etica}} si todos sus puntos de incremento son m\'ultiplos de la forma $0,\lambda, 2\lambda,\ldots$ para alguna $\lambda>0$ entera.
\end{Def}


\begin{Def}
Una modificaci\'on medible de un proceso $\left\{V\left(t\right),t\geq0\right\}$, es una versi\'on de este, $\left\{V\left(t,w\right)\right\}$ conjuntamente medible para $t\geq0$ y para $w\in S$, $S$ espacio de estados para $\left\{V\left(t\right),t\geq0\right\}$.
\end{Def}

\begin{Teo}
Sea $\left\{V\left(t\right),t\geq\right\}$ un proceso regenerativo no negativo con modificaci\'on medible. Sea $\esp\left[X\right]<\infty$. Entonces el proceso estacionario dado por la ecuaci\'on anterior est\'a bien definido y tiene funci\'on de distribuci\'on independiente de $t$, adem\'as
\begin{itemize}
\item[i)] \begin{eqnarray*}
\esp\left[V^{*}\left(0\right)\right]&=&\frac{\esp\left[\int_{0}^{X}V\left(s\right)ds\right]}{\esp\left[X\right]}\end{eqnarray*}
\item[ii)] Si $\esp\left[V^{*}\left(0\right)\right]<\infty$, equivalentemente, si $\esp\left[\int_{0}^{X}V\left(s\right)ds\right]<\infty$,entonces
\begin{eqnarray*}
\frac{\int_{0}^{t}V\left(s\right)ds}{t}\rightarrow\frac{\esp\left[\int_{0}^{X}V\left(s\right)ds\right]}{\esp\left[X\right]}
\end{eqnarray*}
con probabilidad 1 y en media, cuando $t\rightarrow\infty$.
\end{itemize}
\end{Teo}

%______________________________________________________________________
\subsection{Procesos de Renovaci\'on}
%______________________________________________________________________

\begin{Def}\label{Def.Tn}
Sean $0\leq T_{1}\leq T_{2}\leq \ldots$ son tiempos aleatorios infinitos en los cuales ocurren ciertos eventos. El n\'umero de tiempos $T_{n}$ en el intervalo $\left[0,t\right)$ es

\begin{eqnarray}
N\left(t\right)=\sum_{n=1}^{\infty}\indora\left(T_{n}\leq t\right),
\end{eqnarray}
para $t\geq0$.
\end{Def}

Si se consideran los puntos $T_{n}$ como elementos de $\rea_{+}$, y $N\left(t\right)$ es el n\'umero de puntos en $\rea$. El proceso denotado por $\left\{N\left(t\right):t\geq0\right\}$, denotado por $N\left(t\right)$, es un proceso puntual en $\rea_{+}$. Los $T_{n}$ son los tiempos de ocurrencia, el proceso puntual $N\left(t\right)$ es simple si su n\'umero de ocurrencias son distintas: $0<T_{1}<T_{2}<\ldots$ casi seguramente.

\begin{Def}
Un proceso puntual $N\left(t\right)$ es un proceso de renovaci\'on si los tiempos de interocurrencia $\xi_{n}=T_{n}-T_{n-1}$, para $n\geq1$, son independientes e identicamente distribuidos con distribuci\'on $F$, donde $F\left(0\right)=0$ y $T_{0}=0$. Los $T_{n}$ son llamados tiempos de renovaci\'on, referente a la independencia o renovaci\'on de la informaci\'on estoc\'astica en estos tiempos. Los $\xi_{n}$ son los tiempos de inter-renovaci\'on, y $N\left(t\right)$ es el n\'umero de renovaciones en el intervalo $\left[0,t\right)$
\end{Def}


\begin{Note}
Para definir un proceso de renovaci\'on para cualquier contexto, solamente hay que especificar una distribuci\'on $F$, con $F\left(0\right)=0$, para los tiempos de inter-renovaci\'on. La funci\'on $F$ en turno degune las otra variables aleatorias. De manera formal, existe un espacio de probabilidad y una sucesi\'on de variables aleatorias $\xi_{1},\xi_{2},\ldots$ definidas en este con distribuci\'on $F$. Entonces las otras cantidades son $T_{n}=\sum_{k=1}^{n}\xi_{k}$ y $N\left(t\right)=\sum_{n=1}^{\infty}\indora\left(T_{n}\leq t\right)$, donde $T_{n}\rightarrow\infty$ casi seguramente por la Ley Fuerte de los Grandes Números.
\end{Note}

%___________________________________________________________________________________________
%
\subsection{Teorema Principal de Renovaci\'on}
%___________________________________________________________________________________________
%

\begin{Note} Una funci\'on $h:\rea_{+}\rightarrow\rea$ es Directamente Riemann Integrable en los siguientes casos:
\begin{itemize}
\item[a)] $h\left(t\right)\geq0$ es decreciente y Riemann Integrable.
\item[b)] $h$ es continua excepto posiblemente en un conjunto de Lebesgue de medida 0, y $|h\left(t\right)|\leq b\left(t\right)$, donde $b$ es DRI.
\end{itemize}
\end{Note}

\begin{Teo}[Teorema Principal de Renovaci\'on]
Si $F$ es no aritm\'etica y $h\left(t\right)$ es Directamente Riemann Integrable (DRI), entonces

\begin{eqnarray*}
lim_{t\rightarrow\infty}U\star h=\frac{1}{\mu}\int_{\rea_{+}}h\left(s\right)ds.
\end{eqnarray*}
\end{Teo}

\begin{Prop}
Cualquier funci\'on $H\left(t\right)$ acotada en intervalos finitos y que es 0 para $t<0$ puede expresarse como
\begin{eqnarray*}
H\left(t\right)=U\star h\left(t\right)\textrm{,  donde }h\left(t\right)=H\left(t\right)-F\star H\left(t\right)
\end{eqnarray*}
\end{Prop}

\begin{Def}
Un proceso estoc\'astico $X\left(t\right)$ es crudamente regenerativo en un tiempo aleatorio positivo $T$ si
\begin{eqnarray*}
\esp\left[X\left(T+t\right)|T\right]=\esp\left[X\left(t\right)\right]\textrm{, para }t\geq0,\end{eqnarray*}
y con las esperanzas anteriores finitas.
\end{Def}

\begin{Prop}
Sup\'ongase que $X\left(t\right)$ es un proceso crudamente regenerativo en $T$, que tiene distribuci\'on $F$. Si $\esp\left[X\left(t\right)\right]$ es acotado en intervalos finitos, entonces
\begin{eqnarray*}
\esp\left[X\left(t\right)\right]=U\star h\left(t\right)\textrm{,  donde }h\left(t\right)=\esp\left[X\left(t\right)\indora\left(T>t\right)\right].
\end{eqnarray*}
\end{Prop}

\begin{Teo}[Regeneraci\'on Cruda]
Sup\'ongase que $X\left(t\right)$ es un proceso con valores positivo crudamente regenerativo en $T$, y def\'inase $M=\sup\left\{|X\left(t\right)|:t\leq T\right\}$. Si $T$ es no aritm\'etico y $M$ y $MT$ tienen media finita, entonces
\begin{eqnarray*}
lim_{t\rightarrow\infty}\esp\left[X\left(t\right)\right]=\frac{1}{\mu}\int_{\rea_{+}}h\left(s\right)ds,
\end{eqnarray*}
donde $h\left(t\right)=\esp\left[X\left(t\right)\indora\left(T>t\right)\right]$.
\end{Teo}



%___________________________________________________________________________________________
%
\subsection{Funci\'on de Renovaci\'on}
%___________________________________________________________________________________________
%


\begin{Def}
Sea $h\left(t\right)$ funci\'on de valores reales en $\rea$ acotada en intervalos finitos e igual a cero para $t<0$ La ecuaci\'on de renovaci\'on para $h\left(t\right)$ y la distribuci\'on $F$ es

\begin{eqnarray}\label{Ec.Renovacion}
H\left(t\right)=h\left(t\right)+\int_{\left[0,t\right]}H\left(t-s\right)dF\left(s\right)\textrm{,    }t\geq0,
\end{eqnarray}
donde $H\left(t\right)$ es una funci\'on de valores reales. Esto es $H=h+F\star H$. Decimos que $H\left(t\right)$ es soluci\'on de esta ecuaci\'on si satisface la ecuaci\'on, y es acotada en intervalos finitos e iguales a cero para $t<0$.
\end{Def}

\begin{Prop}
La funci\'on $U\star h\left(t\right)$ es la \'unica soluci\'on de la ecuaci\'on de renovaci\'on (\ref{Ec.Renovacion}).
\end{Prop}

\begin{Teo}[Teorema Renovaci\'on Elemental]
\begin{eqnarray*}
t^{-1}U\left(t\right)\rightarrow 1/\mu\textrm{,    cuando }t\rightarrow\infty.
\end{eqnarray*}
\end{Teo}

%___________________________________________________________________________________________
%
\subsection{Propiedades de los Procesos de Renovaci\'on}
%___________________________________________________________________________________________
%

Los tiempos $T_{n}$ est\'an relacionados con los conteos de $N\left(t\right)$ por

\begin{eqnarray*}
\left\{N\left(t\right)\geq n\right\}&=&\left\{T_{n}\leq t\right\}\\
T_{N\left(t\right)}\leq &t&<T_{N\left(t\right)+1},
\end{eqnarray*}

adem\'as $N\left(T_{n}\right)=n$, y 

\begin{eqnarray*}
N\left(t\right)=\max\left\{n:T_{n}\leq t\right\}=\min\left\{n:T_{n+1}>t\right\}
\end{eqnarray*}

Por propiedades de la convoluci\'on se sabe que

\begin{eqnarray*}
P\left\{T_{n}\leq t\right\}=F^{n\star}\left(t\right)
\end{eqnarray*}
que es la $n$-\'esima convoluci\'on de $F$. Entonces 

\begin{eqnarray*}
\left\{N\left(t\right)\geq n\right\}&=&\left\{T_{n}\leq t\right\}\\
P\left\{N\left(t\right)\leq n\right\}&=&1-F^{\left(n+1\right)\star}\left(t\right)
\end{eqnarray*}

Adem\'as usando el hecho de que $\esp\left[N\left(t\right)\right]=\sum_{n=1}^{\infty}P\left\{N\left(t\right)\geq n\right\}$
se tiene que

\begin{eqnarray*}
\esp\left[N\left(t\right)\right]=\sum_{n=1}^{\infty}F^{n\star}\left(t\right)
\end{eqnarray*}

\begin{Prop}
Para cada $t\geq0$, la funci\'on generadora de momentos $\esp\left[e^{\alpha N\left(t\right)}\right]$ existe para alguna $\alpha$ en una vecindad del 0, y de aqu\'i que $\esp\left[N\left(t\right)^{m}\right]<\infty$, para $m\geq1$.
\end{Prop}


\begin{Note}
Si el primer tiempo de renovaci\'on $\xi_{1}$ no tiene la misma distribuci\'on que el resto de las $\xi_{n}$, para $n\geq2$, a $N\left(t\right)$ se le llama Proceso de Renovaci\'on retardado, donde si $\xi$ tiene distribuci\'on $G$, entonces el tiempo $T_{n}$ de la $n$-\'esima renovaci\'on tiene distribuci\'on $G\star F^{\left(n-1\right)\star}\left(t\right)$
\end{Note}


\begin{Teo}
Para una constante $\mu\leq\infty$ ( o variable aleatoria), las siguientes expresiones son equivalentes:

\begin{eqnarray}
lim_{n\rightarrow\infty}n^{-1}T_{n}&=&\mu,\textrm{ c.s.}\\
lim_{t\rightarrow\infty}t^{-1}N\left(t\right)&=&1/\mu,\textrm{ c.s.}
\end{eqnarray}
\end{Teo}


Es decir, $T_{n}$ satisface la Ley Fuerte de los Grandes N\'umeros s\'i y s\'olo s\'i $N\left/t\right)$ la cumple.


\begin{Coro}[Ley Fuerte de los Grandes N\'umeros para Procesos de Renovaci\'on]
Si $N\left(t\right)$ es un proceso de renovaci\'on cuyos tiempos de inter-renovaci\'on tienen media $\mu\leq\infty$, entonces
\begin{eqnarray}
t^{-1}N\left(t\right)\rightarrow 1/\mu,\textrm{ c.s. cuando }t\rightarrow\infty.
\end{eqnarray}

\end{Coro}


Considerar el proceso estoc\'astico de valores reales $\left\{Z\left(t\right):t\geq0\right\}$ en el mismo espacio de probabilidad que $N\left(t\right)$

\begin{Def}
Para el proceso $\left\{Z\left(t\right):t\geq0\right\}$ se define la fluctuaci\'on m\'axima de $Z\left(t\right)$ en el intervalo $\left(T_{n-1},T_{n}\right]$:
\begin{eqnarray*}
M_{n}=\sup_{T_{n-1}<t\leq T_{n}}|Z\left(t\right)-Z\left(T_{n-1}\right)|
\end{eqnarray*}
\end{Def}

\begin{Teo}
Sup\'ongase que $n^{-1}T_{n}\rightarrow\mu$ c.s. cuando $n\rightarrow\infty$, donde $\mu\leq\infty$ es una constante o variable aleatoria. Sea $a$ una constante o variable aleatoria que puede ser infinita cuando $\mu$ es finita, y considere las expresiones l\'imite:
\begin{eqnarray}
lim_{n\rightarrow\infty}n^{-1}Z\left(T_{n}\right)&=&a,\textrm{ c.s.}\\
lim_{t\rightarrow\infty}t^{-1}Z\left(t\right)&=&a/\mu,\textrm{ c.s.}
\end{eqnarray}
La segunda expresi\'on implica la primera. Conversamente, la primera implica la segunda si el proceso $Z\left(t\right)$ es creciente, o si $lim_{n\rightarrow\infty}n^{-1}M_{n}=0$ c.s.
\end{Teo}

\begin{Coro}
Si $N\left(t\right)$ es un proceso de renovaci\'on, y $\left(Z\left(T_{n}\right)-Z\left(T_{n-1}\right),M_{n}\right)$, para $n\geq1$, son variables aleatorias independientes e id\'enticamente distribuidas con media finita, entonces,
\begin{eqnarray}
lim_{t\rightarrow\infty}t^{-1}Z\left(t\right)\rightarrow\frac{\esp\left[Z\left(T_{1}\right)-Z\left(T_{0}\right)\right]}{\esp\left[T_{1}\right]},\textrm{ c.s. cuando  }t\rightarrow\infty.
\end{eqnarray}
\end{Coro}

%___________________________________________________________________________________________
%
\subsection{Funci\'on de Renovaci\'on}
%___________________________________________________________________________________________
%


Sup\'ongase que $N\left(t\right)$ es un proceso de renovaci\'on con distribuci\'on $F$ con media finita $\mu$.

\begin{Def}
La funci\'on de renovaci\'on asociada con la distribuci\'on $F$, del proceso $N\left(t\right)$, es
\begin{eqnarray*}
U\left(t\right)=\sum_{n=1}^{\infty}F^{n\star}\left(t\right),\textrm{   }t\geq0,
\end{eqnarray*}
donde $F^{0\star}\left(t\right)=\indora\left(t\geq0\right)$.
\end{Def}


\begin{Prop}
Sup\'ongase que la distribuci\'on de inter-renovaci\'on $F$ tiene densidad $f$. Entonces $U\left(t\right)$ tambi\'en tiene densidad, para $t>0$, y es $U^{'}\left(t\right)=\sum_{n=0}^{\infty}f^{n\star}\left(t\right)$. Adem\'as
\begin{eqnarray*}
\prob\left\{N\left(t\right)>N\left(t-\right)\right\}=0\textrm{,   }t\geq0.
\end{eqnarray*}
\end{Prop}

\begin{Def}
La Transformada de Laplace-Stieljes de $F$ est\'a dada por

\begin{eqnarray*}
\hat{F}\left(\alpha\right)=\int_{\rea_{+}}e^{-\alpha t}dF\left(t\right)\textrm{,  }\alpha\geq0.
\end{eqnarray*}
\end{Def}

Entonces

\begin{eqnarray*}
\hat{U}\left(\alpha\right)=\sum_{n=0}^{\infty}\hat{F^{n\star}}\left(\alpha\right)=\sum_{n=0}^{\infty}\hat{F}\left(\alpha\right)^{n}=\frac{1}{1-\hat{F}\left(\alpha\right)}.
\end{eqnarray*}


\begin{Prop}
La Transformada de Laplace $\hat{U}\left(\alpha\right)$ y $\hat{F}\left(\alpha\right)$ determina una a la otra de manera \'unica por la relaci\'on $\hat{U}\left(\alpha\right)=\frac{1}{1-\hat{F}\left(\alpha\right)}$.
\end{Prop}


\begin{Note}
Un proceso de renovaci\'on $N\left(t\right)$ cuyos tiempos de inter-renovaci\'on tienen media finita, es un proceso Poisson con tasa $\lambda$ si y s\'olo s\'i $\esp\left[U\left(t\right)\right]=\lambda t$, para $t\geq0$.
\end{Note}


\begin{Teo}
Sea $N\left(t\right)$ un proceso puntual simple con puntos de localizaci\'on $T_{n}$ tal que $\eta\left(t\right)=\esp\left[N\left(\right)\right]$ es finita para cada $t$. Entonces para cualquier funci\'on $f:\rea_{+}\rightarrow\rea$,
\begin{eqnarray*}
\esp\left[\sum_{n=1}^{N\left(\right)}f\left(T_{n}\right)\right]=\int_{\left(0,t\right]}f\left(s\right)d\eta\left(s\right)\textrm{,  }t\geq0,
\end{eqnarray*}
suponiendo que la integral exista. Adem\'as si $X_{1},X_{2},\ldots$ son variables aleatorias definidas en el mismo espacio de probabilidad que el proceso $N\left(t\right)$ tal que $\esp\left[X_{n}|T_{n}=s\right]=f\left(s\right)$, independiente de $n$. Entonces
\begin{eqnarray*}
\esp\left[\sum_{n=1}^{N\left(t\right)}X_{n}\right]=\int_{\left(0,t\right]}f\left(s\right)d\eta\left(s\right)\textrm{,  }t\geq0,
\end{eqnarray*} 
suponiendo que la integral exista. 
\end{Teo}

\begin{Coro}[Identidad de Wald para Renovaciones]
Para el proceso de renovaci\'on $N\left(t\right)$,
\begin{eqnarray*}
\esp\left[T_{N\left(t\right)+1}\right]=\mu\esp\left[N\left(t\right)+1\right]\textrm{,  }t\geq0,
\end{eqnarray*}  
\end{Coro}

%______________________________________________________________________
\subsection{Procesos de Renovaci\'on}
%______________________________________________________________________

\begin{Def}\label{Def.Tn}
Sean $0\leq T_{1}\leq T_{2}\leq \ldots$ son tiempos aleatorios infinitos en los cuales ocurren ciertos eventos. El n\'umero de tiempos $T_{n}$ en el intervalo $\left[0,t\right)$ es

\begin{eqnarray}
N\left(t\right)=\sum_{n=1}^{\infty}\indora\left(T_{n}\leq t\right),
\end{eqnarray}
para $t\geq0$.
\end{Def}

Si se consideran los puntos $T_{n}$ como elementos de $\rea_{+}$, y $N\left(t\right)$ es el n\'umero de puntos en $\rea$. El proceso denotado por $\left\{N\left(t\right):t\geq0\right\}$, denotado por $N\left(t\right)$, es un proceso puntual en $\rea_{+}$. Los $T_{n}$ son los tiempos de ocurrencia, el proceso puntual $N\left(t\right)$ es simple si su n\'umero de ocurrencias son distintas: $0<T_{1}<T_{2}<\ldots$ casi seguramente.

\begin{Def}
Un proceso puntual $N\left(t\right)$ es un proceso de renovaci\'on si los tiempos de interocurrencia $\xi_{n}=T_{n}-T_{n-1}$, para $n\geq1$, son independientes e identicamente distribuidos con distribuci\'on $F$, donde $F\left(0\right)=0$ y $T_{0}=0$. Los $T_{n}$ son llamados tiempos de renovaci\'on, referente a la independencia o renovaci\'on de la informaci\'on estoc\'astica en estos tiempos. Los $\xi_{n}$ son los tiempos de inter-renovaci\'on, y $N\left(t\right)$ es el n\'umero de renovaciones en el intervalo $\left[0,t\right)$
\end{Def}


\begin{Note}
Para definir un proceso de renovaci\'on para cualquier contexto, solamente hay que especificar una distribuci\'on $F$, con $F\left(0\right)=0$, para los tiempos de inter-renovaci\'on. La funci\'on $F$ en turno degune las otra variables aleatorias. De manera formal, existe un espacio de probabilidad y una sucesi\'on de variables aleatorias $\xi_{1},\xi_{2},\ldots$ definidas en este con distribuci\'on $F$. Entonces las otras cantidades son $T_{n}=\sum_{k=1}^{n}\xi_{k}$ y $N\left(t\right)=\sum_{n=1}^{\infty}\indora\left(T_{n}\leq t\right)$, donde $T_{n}\rightarrow\infty$ casi seguramente por la Ley Fuerte de los Grandes Números.
\end{Note}
%_____________________________________________________
\subsection{Puntos de Renovaci\'on}
%_____________________________________________________

Para cada cola $Q_{i}$ se tienen los procesos de arribo a la cola, para estas, los tiempos de arribo est\'an dados por $$\left\{T_{1}^{i},T_{2}^{i},\ldots,T_{k}^{i},\ldots\right\},$$ entonces, consideremos solamente los primeros tiempos de arribo a cada una de las colas, es decir, $$\left\{T_{1}^{1},T_{1}^{2},T_{1}^{3},T_{1}^{4}\right\},$$ se sabe que cada uno de estos tiempos se distribuye de manera exponencial con par\'ametro $1/mu_{i}$. Adem\'as se sabe que para $$T^{*}=\min\left\{T_{1}^{1},T_{1}^{2},T_{1}^{3},T_{1}^{4}\right\},$$ $T^{*}$ se distribuye de manera exponencial con par\'ametro $$\mu^{*}=\sum_{i=1}^{4}\mu_{i}.$$ Ahora, dado que 
\begin{center}
\begin{tabular}{lcl}
$\tilde{r}=r_{1}+r_{2}$ & y &$\hat{r}=r_{3}+r_{4}.$
\end{tabular}
\end{center}


Supongamos que $$\tilde{r},\hat{r}<\mu^{*},$$ entonces si tomamos $$r^{*}=\min\left\{\tilde{r},\hat{r}\right\},$$ se tiene que para  $$t^{*}\in\left(0,r^{*}\right)$$ se cumple que 
\begin{center}
\begin{tabular}{lcl}
$\tau_{1}\left(1\right)=0$ & y por tanto & $\overline{\tau}_{1}=0,$
\end{tabular}
\end{center}
entonces para la segunda cola en este primer ciclo se cumple que $$\tau_{2}=\overline{\tau}_{1}+r_{1}=r_{1}<\mu^{*},$$ y por tanto se tiene que  $$\overline{\tau}_{2}=\tau_{2}.$$ Por lo tanto, nuevamente para la primer cola en el segundo ciclo $$\tau_{1}\left(2\right)=\tau_{2}\left(1\right)+r_{2}=\tilde{r}<\mu^{*}.$$ An\'alogamente para el segundo sistema se tiene que ambas colas est\'an vac\'ias, es decir, existe un valor $t^{*}$ tal que en el intervalo $\left(0,t^{*}\right)$ no ha llegado ning\'un usuario, es decir, $$L_{i}\left(t^{*}\right)=0$$ para $i=1,2,3,4$.

\subsection{Resultados para Procesos de Salida}

En \cite{Sigman2} prueban que para la existencia de un una sucesi\'on infinita no decreciente de tiempos de regeneraci\'on $\tau_{1}\leq\tau_{2}\leq\cdots$ en los cuales el proceso se regenera, basta un tiempo de regeneraci\'on $R_{1}$, donde $R_{j}=\tau_{j}-\tau_{j-1}$. Para tal efecto se requiere la existencia de un espacio de probabilidad $\left(\Omega,\mathcal{F},\prob\right)$, y proceso estoc\'astico $\textit{X}=\left\{X\left(t\right):t\geq0\right\}$ con espacio de estados $\left(S,\mathcal{R}\right)$, con $\mathcal{R}$ $\sigma$-\'algebra.

\begin{Prop}
Si existe una variable aleatoria no negativa $R_{1}$ tal que $\theta_{R\footnotesize{1}}X=_{D}X$, entonces $\left(\Omega,\mathcal{F},\prob\right)$ puede extenderse para soportar una sucesi\'on estacionaria de variables aleatorias $R=\left\{R_{k}:k\geq1\right\}$, tal que para $k\geq1$,
\begin{eqnarray*}
\theta_{k}\left(X,R\right)=_{D}\left(X,R\right).
\end{eqnarray*}

Adem\'as, para $k\geq1$, $\theta_{k}R$ es condicionalmente independiente de $\left(X,R_{1},\ldots,R_{k}\right)$, dado $\theta_{\tau k}X$.

\end{Prop}


\begin{itemize}
\item Doob en 1953 demostr\'o que el estado estacionario de un proceso de partida en un sistema de espera $M/G/\infty$, es Poisson con la misma tasa que el proceso de arribos.

\item Burke en 1968, fue el primero en demostrar que el estado estacionario de un proceso de salida de una cola $M/M/s$ es un proceso Poisson.

\item Disney en 1973 obtuvo el siguiente resultado:

\begin{Teo}
Para el sistema de espera $M/G/1/L$ con disciplina FIFO, el proceso $\textbf{I}$ es un proceso de renovaci\'on si y s\'olo si el proceso denominado longitud de la cola es estacionario y se cumple cualquiera de los siguientes casos:

\begin{itemize}
\item[a)] Los tiempos de servicio son identicamente cero;
\item[b)] $L=0$, para cualquier proceso de servicio $S$;
\item[c)] $L=1$ y $G=D$;
\item[d)] $L=\infty$ y $G=M$.
\end{itemize}
En estos casos, respectivamente, las distribuciones de interpartida $P\left\{T_{n+1}-T_{n}\leq t\right\}$ son


\begin{itemize}
\item[a)] $1-e^{-\lambda t}$, $t\geq0$;
\item[b)] $1-e^{-\lambda t}*F\left(t\right)$, $t\geq0$;
\item[c)] $1-e^{-\lambda t}*\indora_{d}\left(t\right)$, $t\geq0$;
\item[d)] $1-e^{-\lambda t}*F\left(t\right)$, $t\geq0$.
\end{itemize}
\end{Teo}


\item Finch (1959) mostr\'o que para los sistemas $M/G/1/L$, con $1\leq L\leq \infty$ con distribuciones de servicio dos veces diferenciable, solamente el sistema $M/M/1/\infty$ tiene proceso de salida de renovaci\'on estacionario.

\item King (1971) demostro que un sistema de colas estacionario $M/G/1/1$ tiene sus tiempos de interpartida sucesivas $D_{n}$ y $D_{n+1}$ son independientes, si y s\'olo si, $G=D$, en cuyo caso le proceso de salida es de renovaci\'on.

\item Disney (1973) demostr\'o que el \'unico sistema estacionario $M/G/1/L$, que tiene proceso de salida de renovaci\'on  son los sistemas $M/M/1$ y $M/D/1/1$.



\item El siguiente resultado es de Disney y Koning (1985)
\begin{Teo}
En un sistema de espera $M/G/s$, el estado estacionario del proceso de salida es un proceso Poisson para cualquier distribuci\'on de los tiempos de servicio si el sistema tiene cualquiera de las siguientes cuatro propiedades.

\begin{itemize}
\item[a)] $s=\infty$
\item[b)] La disciplina de servicio es de procesador compartido.
\item[c)] La disciplina de servicio es LCFS y preemptive resume, esto se cumple para $L<\infty$
\item[d)] $G=M$.
\end{itemize}

\end{Teo}

\item El siguiente resultado es de Alamatsaz (1983)

\begin{Teo}
En cualquier sistema de colas $GI/G/1/L$ con $1\leq L<\infty$ y distribuci\'on de interarribos $A$ y distribuci\'on de los tiempos de servicio $B$, tal que $A\left(0\right)=0$, $A\left(t\right)\left(1-B\left(t\right)\right)>0$ para alguna $t>0$ y $B\left(t\right)$ para toda $t>0$, es imposible que el proceso de salida estacionario sea de renovaci\'on.
\end{Teo}

\end{itemize}

Estos resultados aparecen en Daley (1968) \cite{Daley68} para $\left\{T_{n}\right\}$ intervalos de inter-arribo, $\left\{D_{n}\right\}$ intervalos de inter-salida y $\left\{S_{n}\right\}$ tiempos de servicio.

\begin{itemize}
\item Si el proceso $\left\{T_{n}\right\}$ es Poisson, el proceso $\left\{D_{n}\right\}$ es no correlacionado si y s\'olo si es un proceso Poisso, lo cual ocurre si y s\'olo si $\left\{S_{n}\right\}$ son exponenciales negativas.

\item Si $\left\{S_{n}\right\}$ son exponenciales negativas, $\left\{D_{n}\right\}$ es un proceso de renovaci\'on  si y s\'olo si es un proceso Poisson, lo cual ocurre si y s\'olo si $\left\{T_{n}\right\}$ es un proceso Poisson.

\item $\esp\left(D_{n}\right)=\esp\left(T_{n}\right)$.

\item Para un sistema de visitas $GI/M/1$ se tiene el siguiente teorema:

\begin{Teo}
En un sistema estacionario $GI/M/1$ los intervalos de interpartida tienen
\begin{eqnarray*}
\esp\left(e^{-\theta D_{n}}\right)&=&\mu\left(\mu+\theta\right)^{-1}\left[\delta\theta
-\mu\left(1-\delta\right)\alpha\left(\theta\right)\right]
\left[\theta-\mu\left(1-\delta\right)^{-1}\right]\\
\alpha\left(\theta\right)&=&\esp\left[e^{-\theta T_{0}}\right]\\
var\left(D_{n}\right)&=&var\left(T_{0}\right)-\left(\tau^{-1}-\delta^{-1}\right)
2\delta\left(\esp\left(S_{0}\right)\right)^{2}\left(1-\delta\right)^{-1}.
\end{eqnarray*}
\end{Teo}



\begin{Teo}
El proceso de salida de un sistema de colas estacionario $GI/M/1$ es un proceso de renovaci\'on si y s\'olo si el proceso de entrada es un proceso Poisson, en cuyo caso el proceso de salida es un proceso Poisson.
\end{Teo}


\begin{Teo}
Los intervalos de interpartida $\left\{D_{n}\right\}$ de un sistema $M/G/1$ estacionario son no correlacionados si y s\'olo si la distribuci\'on de los tiempos de servicio es exponencial negativa, es decir, el sistema es de tipo  $M/M/1$.

\end{Teo}



\end{itemize}




%_______________________________________________________________________________________________________
\subsection{Ya revisado}
%_______________________________________________________________________________________________________


Def\'inanse los puntos de regenaraci\'on  en el proceso $\left[L_{1}\left(t\right),L_{2}\left(t\right),\ldots,L_{N}\left(t\right)\right]$. Los puntos cuando la cola $i$ es visitada y todos los $L_{j}\left(\tau_{i}\left(m\right)\right)=0$ para $i=1,2$  son puntos de regeneraci\'on. Se llama ciclo regenerativo al intervalo entre dos puntos regenerativos sucesivos.

Sea $M_{i}$  el n\'umero de ciclos de visita en un ciclo regenerativo, y sea $C_{i}^{(m)}$, para $m=1,2,\ldots,M_{i}$ la duraci\'on del $m$-\'esimo ciclo de visita en un ciclo regenerativo. Se define el ciclo del tiempo de visita promedio $\esp\left[C_{i}\right]$ como

\begin{eqnarray*}
\esp\left[C_{i}\right]&=&\frac{\esp\left[\sum_{m=1}^{M_{i}}C_{i}^{(m)}\right]}{\esp\left[M_{i}\right]}
\end{eqnarray*}




Sea la funci\'on generadora de momentos para $L_{i}$, el n\'umero de usuarios en la cola $Q_{i}\left(z\right)$ en cualquier momento, est\'a dada por el tiempo promedio de $z^{L_{i}\left(t\right)}$ sobre el ciclo regenerativo definido anteriormente:

\begin{eqnarray*}
Q_{i}\left(z\right)&=&\esp\left[z^{L_{i}\left(t\right)}\right]=\frac{\esp\left[\sum_{m=1}^{M_{i}}\sum_{t=\tau_{i}\left(m\right)}^{\tau_{i}\left(m+1\right)-1}z^{L_{i}\left(t\right)}\right]}{\esp\left[\sum_{m=1}^{M_{i}}\tau_{i}\left(m+1\right)-\tau_{i}\left(m\right)\right]}
\end{eqnarray*}

$M_{i}$ es un tiempo de paro en el proceso regenerativo con $\esp\left[M_{i}\right]<\infty$, se sigue del lema de Wald que:


\begin{eqnarray*}
\esp\left[\sum_{m=1}^{M_{i}}\sum_{t=\tau_{i}\left(m\right)}^{\tau_{i}\left(m+1\right)-1}z^{L_{i}\left(t\right)}\right]&=&\esp\left[M_{i}\right]\esp\left[\sum_{t=\tau_{i}\left(m\right)}^{\tau_{i}\left(m+1\right)-1}z^{L_{i}\left(t\right)}\right]\\
\esp\left[\sum_{m=1}^{M_{i}}\tau_{i}\left(m+1\right)-\tau_{i}\left(m\right)\right]&=&\esp\left[M_{i}\right]\esp\left[\tau_{i}\left(m+1\right)-\tau_{i}\left(m\right)\right]
\end{eqnarray*}

por tanto se tiene que


\begin{eqnarray*}
Q_{i}\left(z\right)&=&\frac{\esp\left[\sum_{t=\tau_{i}\left(m\right)}^{\tau_{i}\left(m+1\right)-1}z^{L_{i}\left(t\right)}\right]}{\esp\left[\tau_{i}\left(m+1\right)-\tau_{i}\left(m\right)\right]}
\end{eqnarray*}

observar que el denominador es simplemente la duraci\'on promedio del tiempo del ciclo.


Se puede demostrar (ver Hideaki Takagi 1986) que

\begin{eqnarray*}
\esp\left[\sum_{t=\tau_{i}\left(m\right)}^{\tau_{i}\left(m+1\right)-1}z^{L_{i}\left(t\right)}\right]=z\frac{F_{i}\left(z\right)-1}{z-P_{i}\left(z\right)}
\end{eqnarray*}

Durante el tiempo de intervisita para la cola $i$, $L_{i}\left(t\right)$ solamente se incrementa de manera que el incremento por intervalo de tiempo est\'a dado por la funci\'on generadora de probabilidades de $P_{i}\left(z\right)$, por tanto la suma sobre el tiempo de intervisita puede evaluarse como:

\begin{eqnarray*}
\esp\left[\sum_{t=\tau_{i}\left(m\right)}^{\tau_{i}\left(m+1\right)-1}z^{L_{i}\left(t\right)}\right]&=&\esp\left[\sum_{t=\tau_{i}\left(m\right)}^{\tau_{i}\left(m+1\right)-1}\left\{P_{i}\left(z\right)\right\}^{t-\overline{\tau}_{i}\left(m\right)}\right]=\frac{1-\esp\left[\left\{P_{i}\left(z\right)\right\}^{\tau_{i}\left(m+1\right)-\overline{\tau}_{i}\left(m\right)}\right]}{1-P_{i}\left(z\right)}\\
&=&\frac{1-I_{i}\left[P_{i}\left(z\right)\right]}{1-P_{i}\left(z\right)}
\end{eqnarray*}
por tanto

\begin{eqnarray*}
\esp\left[\sum_{t=\tau_{i}\left(m\right)}^{\tau_{i}\left(m+1\right)-1}z^{L_{i}\left(t\right)}\right]&=&\frac{1-F_{i}\left(z\right)}{1-P_{i}\left(z\right)}
\end{eqnarray*}

Haciendo uso de lo hasta ahora desarrollado se tiene que

\begin{eqnarray*}
Q_{i}\left(z\right)&=&\frac{1}{\esp\left[C_{i}\right]}\cdot\frac{1-F_{i}\left(z\right)}{P_{i}\left(z\right)-z}\cdot\frac{\left(1-z\right)P_{i}\left(z\right)}{1-P_{i}\left(z\right)}\\
&=&\frac{\mu_{i}\left(1-\mu_{i}\right)}{f_{i}\left(i\right)}\cdot\frac{1-F_{i}\left(z\right)}{P_{i}\left(z\right)-z}\cdot\frac{\left(1-z\right)P_{i}\left(z\right)}{1-P_{i}\left(z\right)}
\end{eqnarray*}

\begin{Def}
Sea $L_{i}^{*}$el n\'umero de usuarios en la cola $Q_{i}$ cuando es visitada por el servidor para dar servicio, entonces

\begin{eqnarray}
\esp\left[L_{i}^{*}\right]&=&f_{i}\left(i\right)\\
Var\left[L_{i}^{*}\right]&=&f_{i}\left(i,i\right)+\esp\left[L_{i}^{*}\right]-\esp\left[L_{i}^{*}\right]^{2}.
\end{eqnarray}

\end{Def}


\begin{Def}
El tiempo de intervisita $I_{i}$ es el periodo de tiempo que comienza cuando se ha completado el servicio en un ciclo y termina cuando es visitada nuevamente en el pr\'oximo ciclo. Su  duraci\'on del mismo est\'a dada por $\tau_{i}\left(m+1\right)-\overline{\tau}_{i}\left(m\right)$.
\end{Def}


Recordemos las siguientes expresiones:

\begin{eqnarray*}
S_{i}\left(z\right)&=&\esp\left[z^{\overline{\tau}_{i}\left(m\right)-\tau_{i}\left(m\right)}\right]=F_{i}\left(\theta\left(z\right)\right),\\
F\left(z\right)&=&\esp\left[z^{L_{0}}\right],\\
P\left(z\right)&=&\esp\left[z^{X_{n}}\right],\\
F_{i}\left(z\right)&=&\esp\left[z^{L_{i}\left(\tau_{i}\left(m\right)\right)}\right],
\theta_{i}\left(z\right)-zP_{i}
\end{eqnarray*}

entonces 

\begin{eqnarray*}
\esp\left[S_{i}\right]&=&\frac{\esp\left[L_{i}^{*}\right]}{1-\mu_{i}}=\frac{f_{i}\left(i\right)}{1-\mu_{i}},\\
Var\left[S_{i}\right]&=&\frac{Var\left[L_{i}^{*}\right]}{\left(1-\mu_{i}\right)^{2}}+\frac{\sigma^{2}\esp\left[L_{i}^{*}\right]}{\left(1-\mu_{i}\right)^{3}}
\end{eqnarray*}

donde recordemos que

\begin{eqnarray*}
Var\left[L_{i}^{*}\right]&=&f_{i}\left(i,i\right)+f_{i}\left(i\right)-f_{i}\left(i\right)^{2}.
\end{eqnarray*}

La duraci\'on del tiempo de intervisita es $\tau_{i}\left(m+1\right)-\overline{\tau}\left(m\right)$. Dado que el n\'umero de usuarios presentes en $Q_{i}$ al tiempo $t=\tau_{i}\left(m+1\right)$ es igual al n\'umero de arribos durante el intervalo de tiempo $\left[\overline{\tau}\left(m\right),\tau_{i}\left(m+1\right)\right]$ se tiene que


\begin{eqnarray*}
\esp\left[z_{i}^{L_{i}\left(\tau_{i}\left(m+1\right)\right)}\right]=\esp\left[\left\{P_{i}\left(z_{i}\right)\right\}^{\tau_{i}\left(m+1\right)-\overline{\tau}\left(m\right)}\right]
\end{eqnarray*}

entonces, si \begin{eqnarray*}I_{i}\left(z\right)&=&\esp\left[z^{\tau_{i}\left(m+1\right)-\overline{\tau}\left(m\right)}\right]\end{eqnarray*} se tienen que

\begin{eqnarray*}
F_{i}\left(z\right)=I_{i}\left[P_{i}\left(z\right)\right]
\end{eqnarray*}
para $i=1,2$, por tanto



\begin{eqnarray*}
\esp\left[L_{i}^{*}\right]&=&\mu_{i}\esp\left[I_{i}\right]\\
Var\left[L_{i}^{*}\right]&=&\mu_{i}^{2}Var\left[I_{i}\right]+\sigma^{2}\esp\left[I_{i}\right]
\end{eqnarray*}
para $i=1,2$, por tanto


\begin{eqnarray*}
\esp\left[I_{i}\right]&=&\frac{f_{i}\left(i\right)}{\mu_{i}},
\end{eqnarray*}
adem\'as

\begin{eqnarray*}
Var\left[I_{i}\right]&=&\frac{Var\left[L_{i}^{*}\right]}{\mu_{i}^{2}}-\frac{\sigma_{i}^{2}}{\mu_{i}^{2}}f_{i}\left(i\right).
\end{eqnarray*}


Si  $C_{i}\left(z\right)=\esp\left[z^{\overline{\tau}\left(m+1\right)-\overline{\tau}_{i}\left(m\right)}\right]$el tiempo de duraci\'on del ciclo, entonces, por lo hasta ahora establecido, se tiene que

\begin{eqnarray*}
C_{i}\left(z\right)=I_{i}\left[\theta_{i}\left(z\right)\right],
\end{eqnarray*}
entonces

\begin{eqnarray*}
\esp\left[C_{i}\right]&=&\esp\left[I_{i}\right]\esp\left[\theta_{i}\left(z\right)\right]=\frac{\esp\left[L_{i}^{*}\right]}{\mu_{i}}\frac{1}{1-\mu_{i}}=\frac{f_{i}\left(i\right)}{\mu_{i}\left(1-\mu_{i}\right)}\\
Var\left[C_{i}\right]&=&\frac{Var\left[L_{i}^{*}\right]}{\mu_{i}^{2}\left(1-\mu_{i}\right)^{2}}.
\end{eqnarray*}

Por tanto se tienen las siguientes igualdades


\begin{eqnarray*}
\esp\left[S_{i}\right]&=&\mu_{i}\esp\left[C_{i}\right],\\
\esp\left[I_{i}\right]&=&\left(1-\mu_{i}\right)\esp\left[C_{i}\right]\\
\end{eqnarray*}

derivando con respecto a $z$



\begin{eqnarray*}
\frac{d Q_{i}\left(z\right)}{d z}&=&\frac{\left(1-F_{i}\left(z\right)\right)P_{i}\left(z\right)}{\esp\left[C_{i}\right]\left(1-P_{i}\left(z\right)\right)\left(P_{i}\left(z\right)-z\right)}\\
&-&\frac{\left(1-z\right)P_{i}\left(z\right)F_{i}^{'}\left(z\right)}{\esp\left[C_{i}\right]\left(1-P_{i}\left(z\right)\right)\left(P_{i}\left(z\right)-z\right)}\\
&-&\frac{\left(1-z\right)\left(1-F_{i}\left(z\right)\right)P_{i}\left(z\right)\left(P_{i}^{'}\left(z\right)-1\right)}{\esp\left[C_{i}\right]\left(1-P_{i}\left(z\right)\right)\left(P_{i}\left(z\right)-z\right)^{2}}\\
&+&\frac{\left(1-z\right)\left(1-F_{i}\left(z\right)\right)P_{i}^{'}\left(z\right)}{\esp\left[C_{i}\right]\left(1-P_{i}\left(z\right)\right)\left(P_{i}\left(z\right)-z\right)}\\
&+&\frac{\left(1-z\right)\left(1-F_{i}\left(z\right)\right)P_{i}\left(z\right)P_{i}^{'}\left(z\right)}{\esp\left[C_{i}\right]\left(1-P_{i}\left(z\right)\right)^{2}\left(P_{i}\left(z\right)-z\right)}
\end{eqnarray*}

Calculando el l\'imite cuando $z\rightarrow1^{+}$:
\begin{eqnarray}
Q_{i}^{(1)}\left(z\right)=\lim_{z\rightarrow1^{+}}\frac{d Q_{i}\left(z\right)}{dz}&=&\lim_{z\rightarrow1}\frac{\left(1-F_{i}\left(z\right)\right)P_{i}\left(z\right)}{\esp\left[C_{i}\right]\left(1-P_{i}\left(z\right)\right)\left(P_{i}\left(z\right)-z\right)}\\
&-&\lim_{z\rightarrow1^{+}}\frac{\left(1-z\right)P_{i}\left(z\right)F_{i}^{'}\left(z\right)}{\esp\left[C_{i}\right]\left(1-P_{i}\left(z\right)\right)\left(P_{i}\left(z\right)-z\right)}\\
&-&\lim_{z\rightarrow1^{+}}\frac{\left(1-z\right)\left(1-F_{i}\left(z\right)\right)P_{i}\left(z\right)\left(P_{i}^{'}\left(z\right)-1\right)}{\esp\left[C_{i}\right]\left(1-P_{i}\left(z\right)\right)\left(P_{i}\left(z\right)-z\right)^{2}}\\
&+&\lim_{z\rightarrow1^{+}}\frac{\left(1-z\right)\left(1-F_{i}\left(z\right)\right)P_{i}^{'}\left(z\right)}{\esp\left[C_{i}\right]\left(1-P_{i}\left(z\right)\right)\left(P_{i}\left(z\right)-z\right)}\\
&+&\lim_{z\rightarrow1^{+}}\frac{\left(1-z\right)\left(1-F_{i}\left(z\right)\right)P_{i}\left(z\right)P_{i}^{'}\left(z\right)}{\esp\left[C_{i}\right]\left(1-P_{i}\left(z\right)\right)^{2}\left(P_{i}\left(z\right)-z\right)}
\end{eqnarray}

Entonces:
%______________________________________________________

\begin{eqnarray*}
\lim_{z\rightarrow1^{+}}\frac{\left(1-F_{i}\left(z\right)\right)P_{i}\left(z\right)}{\left(1-P_{i}\left(z\right)\right)\left(P_{i}\left(z\right)-z\right)}&=&\lim_{z\rightarrow1^{+}}\frac{\frac{d}{dz}\left[\left(1-F_{i}\left(z\right)\right)P_{i}\left(z\right)\right]}{\frac{d}{dz}\left[\left(1-P_{i}\left(z\right)\right)\left(-z+P_{i}\left(z\right)\right)\right]}\\
&=&\lim_{z\rightarrow1^{+}}\frac{-P_{i}\left(z\right)F_{i}^{'}\left(z\right)+\left(1-F_{i}\left(z\right)\right)P_{i}^{'}\left(z\right)}{\left(1-P_{i}\left(z\right)\right)\left(-1+P_{i}^{'}\left(z\right)\right)-\left(-z+P_{i}\left(z\right)\right)P_{i}^{'}\left(z\right)}
\end{eqnarray*}


%______________________________________________________


\begin{eqnarray*}
\lim_{z\rightarrow1^{+}}\frac{\left(1-z\right)P_{i}\left(z\right)F_{i}^{'}\left(z\right)}{\left(1-P_{i}\left(z\right)\right)\left(P_{i}\left(z\right)-z\right)}&=&\lim_{z\rightarrow1^{+}}\frac{\frac{d}{dz}\left[\left(1-z\right)P_{i}\left(z\right)F_{i}^{'}\left(z\right)\right]}{\frac{d}{dz}\left[\left(1-P_{i}\left(z\right)\right)\left(P_{i}\left(z\right)-z\right)\right]}\\
&=&\lim_{z\rightarrow1^{+}}\frac{-P_{i}\left(z\right) F_{i}^{'}\left(z\right)+(1-z) F_{i}^{'}\left(z\right) P_{i}^{'}\left(z\right)+(1-z) P_{i}\left(z\right)F_{i}^{''}\left(z\right)}{\left(1-P_{i}\left(z\right)\right)\left(-1+P_{i}^{'}\left(z\right)\right)-\left(-z+P_{i}\left(z\right)\right)P_{i}^{'}\left(z\right)}
\end{eqnarray*}


%______________________________________________________

\begin{eqnarray*}
&&\lim_{z\rightarrow1^{+}}\frac{\left(1-z\right)\left(1-F_{i}\left(z\right)\right)P_{i}\left(z\right)\left(P_{i}^{'}\left(z\right)-1\right)}{\left(1-P_{i}\left(z\right)\right)\left(P_{i}\left(z\right)-z\right)^{2}}=\lim_{z\rightarrow1^{+}}\frac{\frac{d}{dz}\left[\left(1-z\right)\left(1-F_{i}\left(z\right)\right)P_{i}\left(z\right)\left(P_{i}^{'}\left(z\right)-1\right)\right]}{\frac{d}{dz}\left[\left(1-P_{i}\left(z\right)\right)\left(P_{i}\left(z\right)-z\right)^{2}\right]}\\
&=&\lim_{z\rightarrow1^{+}}\frac{-\left(1-F_{i}\left(z\right)\right) P_{i}\left(z\right)\left(-1+P_{i}^{'}\left(z\right)\right)-(1-z) P_{i}\left(z\right)F_{i}^{'}\left(z\right)\left(-1+P_{i}^{'}\left(z\right)\right)}{2\left(1-P_{i}\left(z\right)\right)\left(-z+P_{i}\left(z\right)\right) \left(-1+P_{i}^{'}\left(z\right)\right)-\left(-z+P_{i}\left(z\right)\right)^2 P_{i}^{'}\left(z\right)}\\
&+&\lim_{z\rightarrow1^{+}}\frac{+(1-z) \left(1-F_{i}\left(z\right)\right) \left(-1+P_{i}^{'}\left(z\right)\right) P_{i}^{'}\left(z\right)}{{2\left(1-P_{i}\left(z\right)\right)\left(-z+P_{i}\left(z\right)\right) \left(-1+P_{i}^{'}\left(z\right)\right)-\left(-z+P_{i}\left(z\right)\right)^2 P_{i}^{'}\left(z\right)}}\\
&+&\lim_{z\rightarrow1^{+}}\frac{+(1-z) \left(1-F_{i}\left(z\right)\right) P_{i}\left(z\right)P_{i}^{''}\left(z\right)}{{2\left(1-P_{i}\left(z\right)\right)\left(-z+P_{i}\left(z\right)\right) \left(-1+P_{i}^{'}\left(z\right)\right)-\left(-z+P_{i}\left(z\right)\right)^2 P_{i}^{'}\left(z\right)}}
\end{eqnarray*}











%______________________________________________________
\begin{eqnarray*}
&&\lim_{z\rightarrow1^{+}}\frac{\left(1-z\right)\left(1-F_{i}\left(z\right)\right)P_{i}^{'}\left(z\right)}{\left(1-P_{i}\left(z\right)\right)\left(P_{i}\left(z\right)-z\right)}=\lim_{z\rightarrow1^{+}}\frac{\frac{d}{dz}\left[\left(1-z\right)\left(1-F_{i}\left(z\right)\right)P_{i}^{'}\left(z\right)\right]}{\frac{d}{dz}\left[\left(1-P_{i}\left(z\right)\right)\left(P_{i}\left(z\right)-z\right)\right]}\\
&=&\lim_{z\rightarrow1^{+}}\frac{-\left(1-F_{i}\left(z\right)\right) P_{i}^{'}\left(z\right)-(1-z) F_{i}^{'}\left(z\right) P_{i}^{'}\left(z\right)+(1-z) \left(1-F_{i}\left(z\right)\right) P_{i}^{''}\left(z\right)}{\left(1-P_{i}\left(z\right)\right) \left(-1+P_{i}^{'}\left(z\right)\right)-\left(-z+P_{i}\left(z\right)\right) P_{i}^{'}\left(z\right)}\frac{}{}
\end{eqnarray*}

%______________________________________________________
\begin{eqnarray*}
&&\lim_{z\rightarrow1^{+}}\frac{\left(1-z\right)\left(1-F_{i}\left(z\right)\right)P_{i}\left(z\right)P_{i}^{'}\left(z\right)}{\left(1-P_{i}\left(z\right)\right)^{2}\left(P_{i}\left(z\right)-z\right)}=\lim_{z\rightarrow1^{+}}\frac{\frac{d}{dz}\left[\left(1-z\right)\left(1-F_{i}\left(z\right)\right)P_{i}\left(z\right)P_{i}^{'}\left(z\right)\right]}{\frac{d}{dz}\left[\left(1-P_{i}\left(z\right)\right)^{2}\left(P_{i}\left(z\right)-z\right)\right]}\\
&=&\lim_{z\rightarrow1^{+}}\frac{-\left(1-F_{i}\left(z\right)\right) P_{i}\left(z\right) P_{i}^{'}\left(z\right)-(1-z) P_{i}\left(z\right) F_{i}^{'}\left(z\right)P_i'[z]}{\left(1-P_{i}\left(z\right)\right)^2 \left(-1+P_{i}^{'}\left(z\right)\right)-2 \left(1-P_{i}\left(z\right)\right) \left(-z+P_{i}\left(z\right)\right) P_{i}^{'}\left(z\right)}\\
&+&\lim_{z\rightarrow1^{+}}\frac{(1-z) \left(1-F_{i}\left(z\right)\right) P_{i}^{'}\left(z\right)^2+(1-z) \left(1-F_{i}\left(z\right)\right) P_{i}\left(z\right) P_{i}^{''}\left(z\right)}{\left(1-P_{i}\left(z\right)\right)^2 \left(-1+P_{i}^{'}\left(z\right)\right)-2 \left(1-P_{i}\left(z\right)\right) \left(-z+P_{i}\left(z\right)\right) P_{i}^{'}\left(z\right)}\\
\end{eqnarray*}



En nuestra notaci\'on $V\left(t\right)\equiv C_{i}$ y $X_{i}=C_{i}^{(m)}$ para nuestra segunda definici\'on, mientras que para la primera la notaci\'on es: $X\left(t\right)\equiv C_{i}$ y $R_{i}\equiv C_{i}^{(m)}$.


%___________________________________________________________________________________________
%\section{Tiempos de Ciclo e Intervisita}
%___________________________________________________________________________________________


\begin{Def}
Sea $L_{i}^{*}$el n\'umero de usuarios en la cola $Q_{i}$ cuando es visitada por el servidor para dar servicio, entonces

\begin{eqnarray}
\esp\left[L_{i}^{*}\right]&=&f_{i}\left(i\right)\\
Var\left[L_{i}^{*}\right]&=&f_{i}\left(i,i\right)+\esp\left[L_{i}^{*}\right]-\esp\left[L_{i}^{*}\right]^{2}.
\end{eqnarray}

\end{Def}

\begin{Def}
El tiempo de Ciclo $C_{i}$ es e periodo de tiempo que comienza cuando la cola $i$ es visitada por primera vez en un ciclo, y termina cuando es visitado nuevamente en el pr\'oximo ciclo. La duraci\'on del mismo est\'a dada por $\tau_{i}\left(m+1\right)-\tau_{i}\left(m\right)$, o equivalentemente $\overline{\tau}_{i}\left(m+1\right)-\overline{\tau}_{i}\left(m\right)$ bajo condiciones de estabilidad.
\end{Def}

\begin{Def}
El tiempo de intervisita $I_{i}$ es el periodo de tiempo que comienza cuando se ha completado el servicio en un ciclo y termina cuando es visitada nuevamente en el pr\'oximo ciclo. Su  duraci\'on del mismo est\'a dada por $\tau_{i}\left(m+1\right)-\overline{\tau}_{i}\left(m\right)$.
\end{Def}


Recordemos las siguientes expresiones:

\begin{eqnarray*}
S_{i}\left(z\right)&=&\esp\left[z^{\overline{\tau}_{i}\left(m\right)-\tau_{i}\left(m\right)}\right]=F_{i}\left(\theta\left(z\right)\right),\\
F\left(z\right)&=&\esp\left[z^{L_{0}}\right],\\
P\left(z\right)&=&\esp\left[z^{X_{n}}\right],\\
F_{i}\left(z\right)&=&\esp\left[z^{L_{i}\left(\tau_{i}\left(m\right)\right)}\right],
\theta_{i}\left(z\right)-zP_{i}
\end{eqnarray*}

entonces 

\begin{eqnarray*}
\esp\left[S_{i}\right]&=&\frac{\esp\left[L_{i}^{*}\right]}{1-\mu_{i}}=\frac{f_{i}\left(i\right)}{1-\mu_{i}},\\
Var\left[S_{i}\right]&=&\frac{Var\left[L_{i}^{*}\right]}{\left(1-\mu_{i}\right)^{2}}+\frac{\sigma^{2}\esp\left[L_{i}^{*}\right]}{\left(1-\mu_{i}\right)^{3}}
\end{eqnarray*}

donde recordemos que

\begin{eqnarray*}
Var\left[L_{i}^{*}\right]&=&f_{i}\left(i,i\right)+f_{i}\left(i\right)-f_{i}\left(i\right)^{2}.
\end{eqnarray*}

La duraci\'on del tiempo de intervisita es $\tau_{i}\left(m+1\right)-\overline{\tau}\left(m\right)$. Dado que el n\'umero de usuarios presentes en $Q_{i}$ al tiempo $t=\tau_{i}\left(m+1\right)$ es igual al n\'umero de arribos durante el intervalo de tiempo $\left[\overline{\tau}\left(m\right),\tau_{i}\left(m+1\right)\right]$ se tiene que


\begin{eqnarray*}
\esp\left[z_{i}^{L_{i}\left(\tau_{i}\left(m+1\right)\right)}\right]=\esp\left[\left\{P_{i}\left(z_{i}\right)\right\}^{\tau_{i}\left(m+1\right)-\overline{\tau}\left(m\right)}\right]
\end{eqnarray*}

entonces, si \begin{eqnarray*}I_{i}\left(z\right)&=&\esp\left[z^{\tau_{i}\left(m+1\right)-\overline{\tau}\left(m\right)}\right]\end{eqnarray*} se tienen que

\begin{eqnarray*}
F_{i}\left(z\right)=I_{i}\left[P_{i}\left(z\right)\right]
\end{eqnarray*}
para $i=1,2$, por tanto



\begin{eqnarray*}
\esp\left[L_{i}^{*}\right]&=&\mu_{i}\esp\left[I_{i}\right]\\
Var\left[L_{i}^{*}\right]&=&\mu_{i}^{2}Var\left[I_{i}\right]+\sigma^{2}\esp\left[I_{i}\right]
\end{eqnarray*}
para $i=1,2$, por tanto


\begin{eqnarray*}
\esp\left[I_{i}\right]&=&\frac{f_{i}\left(i\right)}{\mu_{i}},
\end{eqnarray*}
adem\'as

\begin{eqnarray*}
Var\left[I_{i}\right]&=&\frac{Var\left[L_{i}^{*}\right]}{\mu_{i}^{2}}-\frac{\sigma_{i}^{2}}{\mu_{i}^{2}}f_{i}\left(i\right).
\end{eqnarray*}


Si  $C_{i}\left(z\right)=\esp\left[z^{\overline{\tau}\left(m+1\right)-\overline{\tau}_{i}\left(m\right)}\right]$el tiempo de duraci\'on del ciclo, entonces, por lo hasta ahora establecido, se tiene que

\begin{eqnarray*}
C_{i}\left(z\right)=I_{i}\left[\theta_{i}\left(z\right)\right],
\end{eqnarray*}
entonces

\begin{eqnarray*}
\esp\left[C_{i}\right]&=&\esp\left[I_{i}\right]\esp\left[\theta_{i}\left(z\right)\right]=\frac{\esp\left[L_{i}^{*}\right]}{\mu_{i}}\frac{1}{1-\mu_{i}}=\frac{f_{i}\left(i\right)}{\mu_{i}\left(1-\mu_{i}\right)}\\
Var\left[C_{i}\right]&=&\frac{Var\left[L_{i}^{*}\right]}{\mu_{i}^{2}\left(1-\mu_{i}\right)^{2}}.
\end{eqnarray*}

Por tanto se tienen las siguientes igualdades


\begin{eqnarray*}
\esp\left[S_{i}\right]&=&\mu_{i}\esp\left[C_{i}\right],\\
\esp\left[I_{i}\right]&=&\left(1-\mu_{i}\right)\esp\left[C_{i}\right]\\
\end{eqnarray*}

Def\'inanse los puntos de regenaraci\'on  en el proceso $\left[L_{1}\left(t\right),L_{2}\left(t\right),\ldots,L_{N}\left(t\right)\right]$. Los puntos cuando la cola $i$ es visitada y todos los $L_{j}\left(\tau_{i}\left(m\right)\right)=0$ para $i=1,2$  son puntos de regeneraci\'on. Se llama ciclo regenerativo al intervalo entre dos puntos regenerativos sucesivos.

Sea $M_{i}$  el n\'umero de ciclos de visita en un ciclo regenerativo, y sea $C_{i}^{(m)}$, para $m=1,2,\ldots,M_{i}$ la duraci\'on del $m$-\'esimo ciclo de visita en un ciclo regenerativo. Se define el ciclo del tiempo de visita promedio $\esp\left[C_{i}\right]$ como

\begin{eqnarray*}
\esp\left[C_{i}\right]&=&\frac{\esp\left[\sum_{m=1}^{M_{i}}C_{i}^{(m)}\right]}{\esp\left[M_{i}\right]}
\end{eqnarray*}


En Stid72 y Heym82 se muestra que una condici\'on suficiente para que el proceso regenerativo 
estacionario sea un procesoo estacionario es que el valor esperado del tiempo del ciclo regenerativo sea finito:

\begin{eqnarray*}
\esp\left[\sum_{m=1}^{M_{i}}C_{i}^{(m)}\right]<\infty.
\end{eqnarray*}

como cada $C_{i}^{(m)}$ contiene intervalos de r\'eplica positivos, se tiene que $\esp\left[M_{i}\right]<\infty$, adem\'as, como $M_{i}>0$, se tiene que la condici\'on anterior es equivalente a tener que 

\begin{eqnarray*}
\esp\left[C_{i}\right]<\infty,
\end{eqnarray*}
por lo tanto una condici\'on suficiente para la existencia del proceso regenerativo est\'a dada por

\begin{eqnarray*}
\sum_{k=1}^{N}\mu_{k}<1.
\end{eqnarray*}



\begin{Note}\label{Cita1.Stidham}
En Stidham\cite{Stidham} y Heyman  se muestra que una condici\'on suficiente para que el proceso regenerativo 
estacionario sea un procesoo estacionario es que el valor esperado del tiempo del ciclo regenerativo sea finito:

\begin{eqnarray*}
\esp\left[\sum_{m=1}^{M_{i}}C_{i}^{(m)}\right]<\infty.
\end{eqnarray*}

como cada $C_{i}^{(m)}$ contiene intervalos de r\'eplica positivos, se tiene que $\esp\left[M_{i}\right]<\infty$, adem\'as, como $M_{i}>0$, se tiene que la condici\'on anterior es equivalente a tener que 

\begin{eqnarray*}
\esp\left[C_{i}\right]<\infty,
\end{eqnarray*}
por lo tanto una condici\'on suficiente para la existencia del proceso regenerativo est\'a dada por

\begin{eqnarray*}
\sum_{k=1}^{N}\mu_{k}<1.
\end{eqnarray*}

{\centering{\Huge{\textbf{Nota incompleta!!}}}}
\end{Note}

%_______________________________________________________________________________________
\subsection{Procesos de Renovaci\'on y Regenerativos}
%_______________________________________________________________________________________



Se puede demostrar (ver Hideaki Takagi 1986) que

\begin{eqnarray*}
\esp\left[\sum_{t=\tau_{i}\left(m\right)}^{\tau_{i}\left(m+1\right)-1}z^{L_{i}\left(t\right)}\right]=z\frac{F_{i}\left(z\right)-1}{z-P_{i}\left(z\right)}
\end{eqnarray*}

Durante el tiempo de intervisita para la cola $i$, $L_{i}\left(t\right)$ solamente se incrementa de manera que el incremento por intervalo de tiempo est\'a dado por la funci\'on generadora de probabilidades de $P_{i}\left(z\right)$, por tanto la suma sobre el tiempo de intervisita puede evaluarse como:

\begin{eqnarray*}
\esp\left[\sum_{t=\tau_{i}\left(m\right)}^{\tau_{i}\left(m+1\right)-1}z^{L_{i}\left(t\right)}\right]&=&\esp\left[\sum_{t=\tau_{i}\left(m\right)}^{\tau_{i}\left(m+1\right)-1}\left\{P_{i}\left(z\right)\right\}^{t-\overline{\tau}_{i}\left(m\right)}\right]=\frac{1-\esp\left[\left\{P_{i}\left(z\right)\right\}^{\tau_{i}\left(m+1\right)-\overline{\tau}_{i}\left(m\right)}\right]}{1-P_{i}\left(z\right)}\\
&=&\frac{1-I_{i}\left[P_{i}\left(z\right)\right]}{1-P_{i}\left(z\right)}
\end{eqnarray*}
por tanto

\begin{eqnarray*}
\esp\left[\sum_{t=\tau_{i}\left(m\right)}^{\tau_{i}\left(m+1\right)-1}z^{L_{i}\left(t\right)}\right]&=&\frac{1-F_{i}\left(z\right)}{1-P_{i}\left(z\right)}
\end{eqnarray*}

Haciendo uso de lo hasta ahora desarrollado se tiene que



%___________________________________________________________________________________________
%\subsection{Longitudes de la Cola en cualquier tiempo}
%___________________________________________________________________________________________
Sea 
\begin{eqnarray*}
Q_{i}\left(z\right)&=&\frac{1}{\esp\left[C_{i}\right]}\cdot\frac{1-F_{i}\left(z\right)}{P_{i}\left(z\right)-z}\cdot\frac{\left(1-z\right)P_{i}\left(z\right)}{1-P_{i}\left(z\right)}
\end{eqnarray*}

Consideremos una cola de la red de sistemas de visitas c\'iclicas fija, $Q_{l}$.


Conforme a la definici\'on dada al principio del cap\'itulo, definici\'on (\ref{Def.Tn}), sean $T_{1},T_{2},\ldots$ los puntos donde las longitudes de las colas de la red de sistemas de visitas c\'iclicas son cero simult\'aneamente, cuando la cola $Q_{l}$ es visitada por el servidor para dar servicio, es decir, $L_{1}\left(T_{i}\right)=0,L_{2}\left(T_{i}\right)=0,\hat{L}_{1}\left(T_{i}\right)=0$ y $\hat{L}_{2}\left(T_{i}\right)=0$, a estos puntos se les denominar\'a puntos regenerativos. Entonces, 

\begin{Def}
Al intervalo de tiempo entre dos puntos regenerativos se le llamar\'a ciclo regenerativo.
\end{Def}

\begin{Def}
Para $T_{i}$ se define, $M_{i}$, el n\'umero de ciclos de visita a la cola $Q_{l}$, durante el ciclo regenerativo, es decir, $M_{i}$ es un proceso de renovaci\'on.
\end{Def}

\begin{Def}
Para cada uno de los $M_{i}$'s, se definen a su vez la duraci\'on de cada uno de estos ciclos de visita en el ciclo regenerativo, $C_{i}^{(m)}$, para $m=1,2,\ldots,M_{i}$, que a su vez, tambi\'en es n proceso de renovaci\'on.
\end{Def}

En nuestra notaci\'on $V\left(t\right)\equiv C_{i}$ y $X_{i}=C_{i}^{(m)}$ para nuestra segunda definici\'on, mientras que para la primera la notaci\'on es: $X\left(t\right)\equiv C_{i}$ y $R_{i}\equiv C_{i}^{(m)}$.


%___________________________________________________________________________________________
%\subsection{Tiempos de Ciclo e Intervisita}
%___________________________________________________________________________________________


\begin{Def}
Sea $L_{i}^{*}$el n\'umero de usuarios en la cola $Q_{i}$ cuando es visitada por el servidor para dar servicio, entonces

\begin{eqnarray}
\esp\left[L_{i}^{*}\right]&=&f_{i}\left(i\right)\\
Var\left[L_{i}^{*}\right]&=&f_{i}\left(i,i\right)+\esp\left[L_{i}^{*}\right]-\esp\left[L_{i}^{*}\right]^{2}.
\end{eqnarray}

\end{Def}

\begin{Def}
El tiempo de Ciclo $C_{i}$ es e periodo de tiempo que comienza cuando la cola $i$ es visitada por primera vez en un ciclo, y termina cuando es visitado nuevamente en el pr\'oximo ciclo. La duraci\'on del mismo est\'a dada por $\tau_{i}\left(m+1\right)-\tau_{i}\left(m\right)$, o equivalentemente $\overline{\tau}_{i}\left(m+1\right)-\overline{\tau}_{i}\left(m\right)$ bajo condiciones de estabilidad.
\end{Def}



Recordemos las siguientes expresiones:

\begin{eqnarray*}
S_{i}\left(z\right)&=&\esp\left[z^{\overline{\tau}_{i}\left(m\right)-\tau_{i}\left(m\right)}\right]=F_{i}\left(\theta\left(z\right)\right),\\
F\left(z\right)&=&\esp\left[z^{L_{0}}\right],\\
P\left(z\right)&=&\esp\left[z^{X_{n}}\right],\\
F_{i}\left(z\right)&=&\esp\left[z^{L_{i}\left(\tau_{i}\left(m\right)\right)}\right],
\theta_{i}\left(z\right)-zP_{i}
\end{eqnarray*}

entonces 

\begin{eqnarray*}
\esp\left[S_{i}\right]&=&\frac{\esp\left[L_{i}^{*}\right]}{1-\mu_{i}}=\frac{f_{i}\left(i\right)}{1-\mu_{i}},\\
Var\left[S_{i}\right]&=&\frac{Var\left[L_{i}^{*}\right]}{\left(1-\mu_{i}\right)^{2}}+\frac{\sigma^{2}\esp\left[L_{i}^{*}\right]}{\left(1-\mu_{i}\right)^{3}}
\end{eqnarray*}

donde recordemos que

\begin{eqnarray*}
Var\left[L_{i}^{*}\right]&=&f_{i}\left(i,i\right)+f_{i}\left(i\right)-f_{i}\left(i\right)^{2}.
\end{eqnarray*}

 por tanto


\begin{eqnarray*}
\esp\left[I_{i}\right]&=&\frac{f_{i}\left(i\right)}{\mu_{i}},
\end{eqnarray*}
adem\'as

\begin{eqnarray*}
Var\left[I_{i}\right]&=&\frac{Var\left[L_{i}^{*}\right]}{\mu_{i}^{2}}-\frac{\sigma_{i}^{2}}{\mu_{i}^{2}}f_{i}\left(i\right).
\end{eqnarray*}


Si  $C_{i}\left(z\right)=\esp\left[z^{\overline{\tau}\left(m+1\right)-\overline{\tau}_{i}\left(m\right)}\right]$el tiempo de duraci\'on del ciclo, entonces, por lo hasta ahora establecido, se tiene que

\begin{eqnarray*}
C_{i}\left(z\right)=I_{i}\left[\theta_{i}\left(z\right)\right],
\end{eqnarray*}
entonces

\begin{eqnarray*}
\esp\left[C_{i}\right]&=&\esp\left[I_{i}\right]\esp\left[\theta_{i}\left(z\right)\right]=\frac{\esp\left[L_{i}^{*}\right]}{\mu_{i}}\frac{1}{1-\mu_{i}}=\frac{f_{i}\left(i\right)}{\mu_{i}\left(1-\mu_{i}\right)}\\
Var\left[C_{i}\right]&=&\frac{Var\left[L_{i}^{*}\right]}{\mu_{i}^{2}\left(1-\mu_{i}\right)^{2}}.
\end{eqnarray*}

Por tanto se tienen las siguientes igualdades


\begin{eqnarray*}
\esp\left[S_{i}\right]&=&\mu_{i}\esp\left[C_{i}\right],\\
\esp\left[I_{i}\right]&=&\left(1-\mu_{i}\right)\esp\left[C_{i}\right]\\
\end{eqnarray*}

Def\'inanse los puntos de regenaraci\'on  en el proceso $\left[L_{1}\left(t\right),L_{2}\left(t\right),\ldots,L_{N}\left(t\right)\right]$. Los puntos cuando la cola $i$ es visitada y todos los $L_{j}\left(\tau_{i}\left(m\right)\right)=0$ para $i=1,2$  son puntos de regeneraci\'on. Se llama ciclo regenerativo al intervalo entre dos puntos regenerativos sucesivos.

Sea $M_{i}$  el n\'umero de ciclos de visita en un ciclo regenerativo, y sea $C_{i}^{(m)}$, para $m=1,2,\ldots,M_{i}$ la duraci\'on del $m$-\'esimo ciclo de visita en un ciclo regenerativo. Se define el ciclo del tiempo de visita promedio $\esp\left[C_{i}\right]$ como

\begin{eqnarray*}
\esp\left[C_{i}\right]&=&\frac{\esp\left[\sum_{m=1}^{M_{i}}C_{i}^{(m)}\right]}{\esp\left[M_{i}\right]}
\end{eqnarray*}


En Stid72 y Heym82 se muestra que una condici\'on suficiente para que el proceso regenerativo 
estacionario sea un procesoo estacionario es que el valor esperado del tiempo del ciclo regenerativo sea finito:

\begin{eqnarray*}
\esp\left[\sum_{m=1}^{M_{i}}C_{i}^{(m)}\right]<\infty.
\end{eqnarray*}

como cada $C_{i}^{(m)}$ contiene intervalos de r\'eplica positivos, se tiene que $\esp\left[M_{i}\right]<\infty$, adem\'as, como $M_{i}>0$, se tiene que la condici\'on anterior es equivalente a tener que 

\begin{eqnarray*}
\esp\left[C_{i}\right]<\infty,
\end{eqnarray*}
por lo tanto una condici\'on suficiente para la existencia del proceso regenerativo est\'a dada por

\begin{eqnarray*}
\sum_{k=1}^{N}\mu_{k}<1.
\end{eqnarray*}

Sea la funci\'on generadora de momentos para $L_{i}$, el n\'umero de usuarios en la cola $Q_{i}\left(z\right)$ en cualquier momento, est\'a dada por el tiempo promedio de $z^{L_{i}\left(t\right)}$ sobre el ciclo regenerativo definido anteriormente:

\begin{eqnarray*}
Q_{i}\left(z\right)&=&\esp\left[z^{L_{i}\left(t\right)}\right]=\frac{\esp\left[\sum_{m=1}^{M_{i}}\sum_{t=\tau_{i}\left(m\right)}^{\tau_{i}\left(m+1\right)-1}z^{L_{i}\left(t\right)}\right]}{\esp\left[\sum_{m=1}^{M_{i}}\tau_{i}\left(m+1\right)-\tau_{i}\left(m\right)\right]}
\end{eqnarray*}

$M_{i}$ es un tiempo de paro en el proceso regenerativo con $\esp\left[M_{i}\right]<\infty$, se sigue del lema de Wald que:


\begin{eqnarray*}
\esp\left[\sum_{m=1}^{M_{i}}\sum_{t=\tau_{i}\left(m\right)}^{\tau_{i}\left(m+1\right)-1}z^{L_{i}\left(t\right)}\right]&=&\esp\left[M_{i}\right]\esp\left[\sum_{t=\tau_{i}\left(m\right)}^{\tau_{i}\left(m+1\right)-1}z^{L_{i}\left(t\right)}\right]\\
\esp\left[\sum_{m=1}^{M_{i}}\tau_{i}\left(m+1\right)-\tau_{i}\left(m\right)\right]&=&\esp\left[M_{i}\right]\esp\left[\tau_{i}\left(m+1\right)-\tau_{i}\left(m\right)\right]
\end{eqnarray*}

por tanto se tiene que


\begin{eqnarray*}
Q_{i}\left(z\right)&=&\frac{\esp\left[\sum_{t=\tau_{i}\left(m\right)}^{\tau_{i}\left(m+1\right)-1}z^{L_{i}\left(t\right)}\right]}{\esp\left[\tau_{i}\left(m+1\right)-\tau_{i}\left(m\right)\right]}
\end{eqnarray*}

observar que el denominador es simplemente la duraci\'on promedio del tiempo del ciclo.


Se puede demostrar (ver Hideaki Takagi 1986) que

\begin{eqnarray*}
\esp\left[\sum_{t=\tau_{i}\left(m\right)}^{\tau_{i}\left(m+1\right)-1}z^{L_{i}\left(t\right)}\right]=z\frac{F_{i}\left(z\right)-1}{z-P_{i}\left(z\right)}
\end{eqnarray*}

Durante el tiempo de intervisita para la cola $i$, $L_{i}\left(t\right)$ solamente se incrementa de manera que el incremento por intervalo de tiempo est\'a dado por la funci\'on generadora de probabilidades de $P_{i}\left(z\right)$, por tanto la suma sobre el tiempo de intervisita puede evaluarse como:

\begin{eqnarray*}
\esp\left[\sum_{t=\tau_{i}\left(m\right)}^{\tau_{i}\left(m+1\right)-1}z^{L_{i}\left(t\right)}\right]&=&\esp\left[\sum_{t=\tau_{i}\left(m\right)}^{\tau_{i}\left(m+1\right)-1}\left\{P_{i}\left(z\right)\right\}^{t-\overline{\tau}_{i}\left(m\right)}\right]=\frac{1-\esp\left[\left\{P_{i}\left(z\right)\right\}^{\tau_{i}\left(m+1\right)-\overline{\tau}_{i}\left(m\right)}\right]}{1-P_{i}\left(z\right)}\\
&=&\frac{1-I_{i}\left[P_{i}\left(z\right)\right]}{1-P_{i}\left(z\right)}
\end{eqnarray*}
por tanto

\begin{eqnarray*}
\esp\left[\sum_{t=\tau_{i}\left(m\right)}^{\tau_{i}\left(m+1\right)-1}z^{L_{i}\left(t\right)}\right]&=&\frac{1-F_{i}\left(z\right)}{1-P_{i}\left(z\right)}
\end{eqnarray*}

Haciendo uso de lo hasta ahora desarrollado se tiene que

\begin{eqnarray*}
Q_{i}\left(z\right)&=&\frac{1}{\esp\left[C_{i}\right]}\cdot\frac{1-F_{i}\left(z\right)}{P_{i}\left(z\right)-z}\cdot\frac{\left(1-z\right)P_{i}\left(z\right)}{1-P_{i}\left(z\right)}\\
&=&\frac{\mu_{i}\left(1-\mu_{i}\right)}{f_{i}\left(i\right)}\cdot\frac{1-F_{i}\left(z\right)}{P_{i}\left(z\right)-z}\cdot\frac{\left(1-z\right)P_{i}\left(z\right)}{1-P_{i}\left(z\right)}
\end{eqnarray*}

derivando con respecto a $z$



\begin{eqnarray*}
\frac{d Q_{i}\left(z\right)}{d z}&=&\frac{\left(1-F_{i}\left(z\right)\right)P_{i}\left(z\right)}{\esp\left[C_{i}\right]\left(1-P_{i}\left(z\right)\right)\left(P_{i}\left(z\right)-z\right)}\\
&-&\frac{\left(1-z\right)P_{i}\left(z\right)F_{i}^{'}\left(z\right)}{\esp\left[C_{i}\right]\left(1-P_{i}\left(z\right)\right)\left(P_{i}\left(z\right)-z\right)}\\
&-&\frac{\left(1-z\right)\left(1-F_{i}\left(z\right)\right)P_{i}\left(z\right)\left(P_{i}^{'}\left(z\right)-1\right)}{\esp\left[C_{i}\right]\left(1-P_{i}\left(z\right)\right)\left(P_{i}\left(z\right)-z\right)^{2}}\\
&+&\frac{\left(1-z\right)\left(1-F_{i}\left(z\right)\right)P_{i}^{'}\left(z\right)}{\esp\left[C_{i}\right]\left(1-P_{i}\left(z\right)\right)\left(P_{i}\left(z\right)-z\right)}\\
&+&\frac{\left(1-z\right)\left(1-F_{i}\left(z\right)\right)P_{i}\left(z\right)P_{i}^{'}\left(z\right)}{\esp\left[C_{i}\right]\left(1-P_{i}\left(z\right)\right)^{2}\left(P_{i}\left(z\right)-z\right)}
\end{eqnarray*}

Calculando el l\'imite cuando $z\rightarrow1^{+}$:
\begin{eqnarray}
Q_{i}^{(1)}\left(z\right)=\lim_{z\rightarrow1^{+}}\frac{d Q_{i}\left(z\right)}{dz}&=&\lim_{z\rightarrow1}\frac{\left(1-F_{i}\left(z\right)\right)P_{i}\left(z\right)}{\esp\left[C_{i}\right]\left(1-P_{i}\left(z\right)\right)\left(P_{i}\left(z\right)-z\right)}\\
&-&\lim_{z\rightarrow1^{+}}\frac{\left(1-z\right)P_{i}\left(z\right)F_{i}^{'}\left(z\right)}{\esp\left[C_{i}\right]\left(1-P_{i}\left(z\right)\right)\left(P_{i}\left(z\right)-z\right)}\\
&-&\lim_{z\rightarrow1^{+}}\frac{\left(1-z\right)\left(1-F_{i}\left(z\right)\right)P_{i}\left(z\right)\left(P_{i}^{'}\left(z\right)-1\right)}{\esp\left[C_{i}\right]\left(1-P_{i}\left(z\right)\right)\left(P_{i}\left(z\right)-z\right)^{2}}\\
&+&\lim_{z\rightarrow1^{+}}\frac{\left(1-z\right)\left(1-F_{i}\left(z\right)\right)P_{i}^{'}\left(z\right)}{\esp\left[C_{i}\right]\left(1-P_{i}\left(z\right)\right)\left(P_{i}\left(z\right)-z\right)}\\
&+&\lim_{z\rightarrow1^{+}}\frac{\left(1-z\right)\left(1-F_{i}\left(z\right)\right)P_{i}\left(z\right)P_{i}^{'}\left(z\right)}{\esp\left[C_{i}\right]\left(1-P_{i}\left(z\right)\right)^{2}\left(P_{i}\left(z\right)-z\right)}
\end{eqnarray}

Entonces:
%______________________________________________________

\begin{eqnarray*}
\lim_{z\rightarrow1^{+}}\frac{\left(1-F_{i}\left(z\right)\right)P_{i}\left(z\right)}{\left(1-P_{i}\left(z\right)\right)\left(P_{i}\left(z\right)-z\right)}&=&\lim_{z\rightarrow1^{+}}\frac{\frac{d}{dz}\left[\left(1-F_{i}\left(z\right)\right)P_{i}\left(z\right)\right]}{\frac{d}{dz}\left[\left(1-P_{i}\left(z\right)\right)\left(-z+P_{i}\left(z\right)\right)\right]}\\
&=&\lim_{z\rightarrow1^{+}}\frac{-P_{i}\left(z\right)F_{i}^{'}\left(z\right)+\left(1-F_{i}\left(z\right)\right)P_{i}^{'}\left(z\right)}{\left(1-P_{i}\left(z\right)\right)\left(-1+P_{i}^{'}\left(z\right)\right)-\left(-z+P_{i}\left(z\right)\right)P_{i}^{'}\left(z\right)}
\end{eqnarray*}


%______________________________________________________


\begin{eqnarray*}
\lim_{z\rightarrow1^{+}}\frac{\left(1-z\right)P_{i}\left(z\right)F_{i}^{'}\left(z\right)}{\left(1-P_{i}\left(z\right)\right)\left(P_{i}\left(z\right)-z\right)}&=&\lim_{z\rightarrow1^{+}}\frac{\frac{d}{dz}\left[\left(1-z\right)P_{i}\left(z\right)F_{i}^{'}\left(z\right)\right]}{\frac{d}{dz}\left[\left(1-P_{i}\left(z\right)\right)\left(P_{i}\left(z\right)-z\right)\right]}\\
&=&\lim_{z\rightarrow1^{+}}\frac{-P_{i}\left(z\right) F_{i}^{'}\left(z\right)+(1-z) F_{i}^{'}\left(z\right) P_{i}^{'}\left(z\right)+(1-z) P_{i}\left(z\right)F_{i}^{''}\left(z\right)}{\left(1-P_{i}\left(z\right)\right)\left(-1+P_{i}^{'}\left(z\right)\right)-\left(-z+P_{i}\left(z\right)\right)P_{i}^{'}\left(z\right)}
\end{eqnarray*}


%______________________________________________________

\begin{eqnarray*}
&&\lim_{z\rightarrow1^{+}}\frac{\left(1-z\right)\left(1-F_{i}\left(z\right)\right)P_{i}\left(z\right)\left(P_{i}^{'}\left(z\right)-1\right)}{\left(1-P_{i}\left(z\right)\right)\left(P_{i}\left(z\right)-z\right)^{2}}=\lim_{z\rightarrow1^{+}}\frac{\frac{d}{dz}\left[\left(1-z\right)\left(1-F_{i}\left(z\right)\right)P_{i}\left(z\right)\left(P_{i}^{'}\left(z\right)-1\right)\right]}{\frac{d}{dz}\left[\left(1-P_{i}\left(z\right)\right)\left(P_{i}\left(z\right)-z\right)^{2}\right]}\\
&=&\lim_{z\rightarrow1^{+}}\frac{-\left(1-F_{i}\left(z\right)\right) P_{i}\left(z\right)\left(-1+P_{i}^{'}\left(z\right)\right)-(1-z) P_{i}\left(z\right)F_{i}^{'}\left(z\right)\left(-1+P_{i}^{'}\left(z\right)\right)}{2\left(1-P_{i}\left(z\right)\right)\left(-z+P_{i}\left(z\right)\right) \left(-1+P_{i}^{'}\left(z\right)\right)-\left(-z+P_{i}\left(z\right)\right)^2 P_{i}^{'}\left(z\right)}\\
&+&\lim_{z\rightarrow1^{+}}\frac{+(1-z) \left(1-F_{i}\left(z\right)\right) \left(-1+P_{i}^{'}\left(z\right)\right) P_{i}^{'}\left(z\right)}{{2\left(1-P_{i}\left(z\right)\right)\left(-z+P_{i}\left(z\right)\right) \left(-1+P_{i}^{'}\left(z\right)\right)-\left(-z+P_{i}\left(z\right)\right)^2 P_{i}^{'}\left(z\right)}}\\
&+&\lim_{z\rightarrow1^{+}}\frac{+(1-z) \left(1-F_{i}\left(z\right)\right) P_{i}\left(z\right)P_{i}^{''}\left(z\right)}{{2\left(1-P_{i}\left(z\right)\right)\left(-z+P_{i}\left(z\right)\right) \left(-1+P_{i}^{'}\left(z\right)\right)-\left(-z+P_{i}\left(z\right)\right)^2 P_{i}^{'}\left(z\right)}}
\end{eqnarray*}











%______________________________________________________
\begin{eqnarray*}
&&\lim_{z\rightarrow1^{+}}\frac{\left(1-z\right)\left(1-F_{i}\left(z\right)\right)P_{i}^{'}\left(z\right)}{\left(1-P_{i}\left(z\right)\right)\left(P_{i}\left(z\right)-z\right)}=\lim_{z\rightarrow1^{+}}\frac{\frac{d}{dz}\left[\left(1-z\right)\left(1-F_{i}\left(z\right)\right)P_{i}^{'}\left(z\right)\right]}{\frac{d}{dz}\left[\left(1-P_{i}\left(z\right)\right)\left(P_{i}\left(z\right)-z\right)\right]}\\
&=&\lim_{z\rightarrow1^{+}}\frac{-\left(1-F_{i}\left(z\right)\right) P_{i}^{'}\left(z\right)-(1-z) F_{i}^{'}\left(z\right) P_{i}^{'}\left(z\right)+(1-z) \left(1-F_{i}\left(z\right)\right) P_{i}^{''}\left(z\right)}{\left(1-P_{i}\left(z\right)\right) \left(-1+P_{i}^{'}\left(z\right)\right)-\left(-z+P_{i}\left(z\right)\right) P_{i}^{'}\left(z\right)}\frac{}{}
\end{eqnarray*}

%______________________________________________________
\begin{eqnarray*}
&&\lim_{z\rightarrow1^{+}}\frac{\left(1-z\right)\left(1-F_{i}\left(z\right)\right)P_{i}\left(z\right)P_{i}^{'}\left(z\right)}{\left(1-P_{i}\left(z\right)\right)^{2}\left(P_{i}\left(z\right)-z\right)}=\lim_{z\rightarrow1^{+}}\frac{\frac{d}{dz}\left[\left(1-z\right)\left(1-F_{i}\left(z\right)\right)P_{i}\left(z\right)P_{i}^{'}\left(z\right)\right]}{\frac{d}{dz}\left[\left(1-P_{i}\left(z\right)\right)^{2}\left(P_{i}\left(z\right)-z\right)\right]}\\
&=&\lim_{z\rightarrow1^{+}}\frac{-\left(1-F_{i}\left(z\right)\right) P_{i}\left(z\right) P_{i}^{'}\left(z\right)-(1-z) P_{i}\left(z\right) F_{i}^{'}\left(z\right)P_i'[z]}{\left(1-P_{i}\left(z\right)\right)^2 \left(-1+P_{i}^{'}\left(z\right)\right)-2 \left(1-P_{i}\left(z\right)\right) \left(-z+P_{i}\left(z\right)\right) P_{i}^{'}\left(z\right)}\\
&+&\lim_{z\rightarrow1^{+}}\frac{(1-z) \left(1-F_{i}\left(z\right)\right) P_{i}^{'}\left(z\right)^2+(1-z) \left(1-F_{i}\left(z\right)\right) P_{i}\left(z\right) P_{i}^{''}\left(z\right)}{\left(1-P_{i}\left(z\right)\right)^2 \left(-1+P_{i}^{'}\left(z\right)\right)-2 \left(1-P_{i}\left(z\right)\right) \left(-z+P_{i}\left(z\right)\right) P_{i}^{'}\left(z\right)}\\
\end{eqnarray*}
%___________________________________________________________________________________________
%\subsection{Tiempos de Ciclo e Intervisita}
%___________________________________________________________________________________________


\begin{Def}
Sea $L_{i}^{*}$el n\'umero de usuarios en la cola $Q_{i}$ cuando es visitada por el servidor para dar servicio, entonces

\begin{eqnarray}
\esp\left[L_{i}^{*}\right]&=&f_{i}\left(i\right)\\
Var\left[L_{i}^{*}\right]&=&f_{i}\left(i,i\right)+\esp\left[L_{i}^{*}\right]-\esp\left[L_{i}^{*}\right]^{2}.
\end{eqnarray}

\end{Def}

\begin{Def}
El tiempo de Ciclo $C_{i}$ es e periodo de tiempo que comienza cuando la cola $i$ es visitada por primera vez en un ciclo, y termina cuando es visitado nuevamente en el pr\'oximo ciclo. La duraci\'on del mismo est\'a dada por $\tau_{i}\left(m+1\right)-\tau_{i}\left(m\right)$, o equivalentemente $\overline{\tau}_{i}\left(m+1\right)-\overline{\tau}_{i}\left(m\right)$ bajo condiciones de estabilidad.
\end{Def}

\begin{Def}
El tiempo de intervisita $I_{i}$ es el periodo de tiempo que comienza cuando se ha completado el servicio en un ciclo y termina cuando es visitada nuevamente en el pr\'oximo ciclo. Su  duraci\'on del mismo est\'a dada por $\tau_{i}\left(m+1\right)-\overline{\tau}_{i}\left(m\right)$.
\end{Def}


Recordemos las siguientes expresiones:

\begin{eqnarray*}
S_{i}\left(z\right)&=&\esp\left[z^{\overline{\tau}_{i}\left(m\right)-\tau_{i}\left(m\right)}\right]=F_{i}\left(\theta\left(z\right)\right),\\
F\left(z\right)&=&\esp\left[z^{L_{0}}\right],\\
P\left(z\right)&=&\esp\left[z^{X_{n}}\right],\\
F_{i}\left(z\right)&=&\esp\left[z^{L_{i}\left(\tau_{i}\left(m\right)\right)}\right],
\theta_{i}\left(z\right)-zP_{i}
\end{eqnarray*}

entonces 

\begin{eqnarray*}
\esp\left[S_{i}\right]&=&\frac{\esp\left[L_{i}^{*}\right]}{1-\mu_{i}}=\frac{f_{i}\left(i\right)}{1-\mu_{i}},\\
Var\left[S_{i}\right]&=&\frac{Var\left[L_{i}^{*}\right]}{\left(1-\mu_{i}\right)^{2}}+\frac{\sigma^{2}\esp\left[L_{i}^{*}\right]}{\left(1-\mu_{i}\right)^{3}}
\end{eqnarray*}

donde recordemos que

\begin{eqnarray*}
Var\left[L_{i}^{*}\right]&=&f_{i}\left(i,i\right)+f_{i}\left(i\right)-f_{i}\left(i\right)^{2}.
\end{eqnarray*}

La duraci\'on del tiempo de intervisita es $\tau_{i}\left(m+1\right)-\overline{\tau}\left(m\right)$. Dado que el n\'umero de usuarios presentes en $Q_{i}$ al tiempo $t=\tau_{i}\left(m+1\right)$ es igual al n\'umero de arribos durante el intervalo de tiempo $\left[\overline{\tau}\left(m\right),\tau_{i}\left(m+1\right)\right]$ se tiene que


\begin{eqnarray*}
\esp\left[z_{i}^{L_{i}\left(\tau_{i}\left(m+1\right)\right)}\right]=\esp\left[\left\{P_{i}\left(z_{i}\right)\right\}^{\tau_{i}\left(m+1\right)-\overline{\tau}\left(m\right)}\right]
\end{eqnarray*}

entonces, si \begin{eqnarray*}I_{i}\left(z\right)&=&\esp\left[z^{\tau_{i}\left(m+1\right)-\overline{\tau}\left(m\right)}\right]\end{eqnarray*} se tienen que

\begin{eqnarray*}
F_{i}\left(z\right)=I_{i}\left[P_{i}\left(z\right)\right]
\end{eqnarray*}
para $i=1,2$, por tanto



\begin{eqnarray*}
\esp\left[L_{i}^{*}\right]&=&\mu_{i}\esp\left[I_{i}\right]\\
Var\left[L_{i}^{*}\right]&=&\mu_{i}^{2}Var\left[I_{i}\right]+\sigma^{2}\esp\left[I_{i}\right]
\end{eqnarray*}
para $i=1,2$, por tanto


\begin{eqnarray*}
\esp\left[I_{i}\right]&=&\frac{f_{i}\left(i\right)}{\mu_{i}},
\end{eqnarray*}
adem\'as

\begin{eqnarray*}
Var\left[I_{i}\right]&=&\frac{Var\left[L_{i}^{*}\right]}{\mu_{i}^{2}}-\frac{\sigma_{i}^{2}}{\mu_{i}^{2}}f_{i}\left(i\right).
\end{eqnarray*}


Si  $C_{i}\left(z\right)=\esp\left[z^{\overline{\tau}\left(m+1\right)-\overline{\tau}_{i}\left(m\right)}\right]$el tiempo de duraci\'on del ciclo, entonces, por lo hasta ahora establecido, se tiene que

\begin{eqnarray*}
C_{i}\left(z\right)=I_{i}\left[\theta_{i}\left(z\right)\right],
\end{eqnarray*}
entonces

\begin{eqnarray*}
\esp\left[C_{i}\right]&=&\esp\left[I_{i}\right]\esp\left[\theta_{i}\left(z\right)\right]=\frac{\esp\left[L_{i}^{*}\right]}{\mu_{i}}\frac{1}{1-\mu_{i}}=\frac{f_{i}\left(i\right)}{\mu_{i}\left(1-\mu_{i}\right)}\\
Var\left[C_{i}\right]&=&\frac{Var\left[L_{i}^{*}\right]}{\mu_{i}^{2}\left(1-\mu_{i}\right)^{2}}.
\end{eqnarray*}

Por tanto se tienen las siguientes igualdades


\begin{eqnarray*}
\esp\left[S_{i}\right]&=&\mu_{i}\esp\left[C_{i}\right],\\
\esp\left[I_{i}\right]&=&\left(1-\mu_{i}\right)\esp\left[C_{i}\right]\\
\end{eqnarray*}

Def\'inanse los puntos de regenaraci\'on  en el proceso $\left[L_{1}\left(t\right),L_{2}\left(t\right),\ldots,L_{N}\left(t\right)\right]$. Los puntos cuando la cola $i$ es visitada y todos los $L_{j}\left(\tau_{i}\left(m\right)\right)=0$ para $i=1,2$  son puntos de regeneraci\'on. Se llama ciclo regenerativo al intervalo entre dos puntos regenerativos sucesivos.

Sea $M_{i}$  el n\'umero de ciclos de visita en un ciclo regenerativo, y sea $C_{i}^{(m)}$, para $m=1,2,\ldots,M_{i}$ la duraci\'on del $m$-\'esimo ciclo de visita en un ciclo regenerativo. Se define el ciclo del tiempo de visita promedio $\esp\left[C_{i}\right]$ como

\begin{eqnarray*}
\esp\left[C_{i}\right]&=&\frac{\esp\left[\sum_{m=1}^{M_{i}}C_{i}^{(m)}\right]}{\esp\left[M_{i}\right]}
\end{eqnarray*}


En Stid72 y Heym82 se muestra que una condici\'on suficiente para que el proceso regenerativo 
estacionario sea un procesoo estacionario es que el valor esperado del tiempo del ciclo regenerativo sea finito:

\begin{eqnarray*}
\esp\left[\sum_{m=1}^{M_{i}}C_{i}^{(m)}\right]<\infty.
\end{eqnarray*}

como cada $C_{i}^{(m)}$ contiene intervalos de r\'eplica positivos, se tiene que $\esp\left[M_{i}\right]<\infty$, adem\'as, como $M_{i}>0$, se tiene que la condici\'on anterior es equivalente a tener que 

\begin{eqnarray*}
\esp\left[C_{i}\right]<\infty,
\end{eqnarray*}
por lo tanto una condici\'on suficiente para la existencia del proceso regenerativo est\'a dada por

\begin{eqnarray*}
\sum_{k=1}^{N}\mu_{k}<1.
\end{eqnarray*}

Sea la funci\'on generadora de momentos para $L_{i}$, el n\'umero de usuarios en la cola $Q_{i}\left(z\right)$ en cualquier momento, est\'a dada por el tiempo promedio de $z^{L_{i}\left(t\right)}$ sobre el ciclo regenerativo definido anteriormente:

\begin{eqnarray*}
Q_{i}\left(z\right)&=&\esp\left[z^{L_{i}\left(t\right)}\right]=\frac{\esp\left[\sum_{m=1}^{M_{i}}\sum_{t=\tau_{i}\left(m\right)}^{\tau_{i}\left(m+1\right)-1}z^{L_{i}\left(t\right)}\right]}{\esp\left[\sum_{m=1}^{M_{i}}\tau_{i}\left(m+1\right)-\tau_{i}\left(m\right)\right]}
\end{eqnarray*}

$M_{i}$ es un tiempo de paro en el proceso regenerativo con $\esp\left[M_{i}\right]<\infty$, se sigue del lema de Wald que:


\begin{eqnarray*}
\esp\left[\sum_{m=1}^{M_{i}}\sum_{t=\tau_{i}\left(m\right)}^{\tau_{i}\left(m+1\right)-1}z^{L_{i}\left(t\right)}\right]&=&\esp\left[M_{i}\right]\esp\left[\sum_{t=\tau_{i}\left(m\right)}^{\tau_{i}\left(m+1\right)-1}z^{L_{i}\left(t\right)}\right]\\
\esp\left[\sum_{m=1}^{M_{i}}\tau_{i}\left(m+1\right)-\tau_{i}\left(m\right)\right]&=&\esp\left[M_{i}\right]\esp\left[\tau_{i}\left(m+1\right)-\tau_{i}\left(m\right)\right]
\end{eqnarray*}

por tanto se tiene que


\begin{eqnarray*}
Q_{i}\left(z\right)&=&\frac{\esp\left[\sum_{t=\tau_{i}\left(m\right)}^{\tau_{i}\left(m+1\right)-1}z^{L_{i}\left(t\right)}\right]}{\esp\left[\tau_{i}\left(m+1\right)-\tau_{i}\left(m\right)\right]}
\end{eqnarray*}

observar que el denominador es simplemente la duraci\'on promedio del tiempo del ciclo.


Se puede demostrar (ver Hideaki Takagi 1986) que

\begin{eqnarray*}
\esp\left[\sum_{t=\tau_{i}\left(m\right)}^{\tau_{i}\left(m+1\right)-1}z^{L_{i}\left(t\right)}\right]=z\frac{F_{i}\left(z\right)-1}{z-P_{i}\left(z\right)}
\end{eqnarray*}

Durante el tiempo de intervisita para la cola $i$, $L_{i}\left(t\right)$ solamente se incrementa de manera que el incremento por intervalo de tiempo est\'a dado por la funci\'on generadora de probabilidades de $P_{i}\left(z\right)$, por tanto la suma sobre el tiempo de intervisita puede evaluarse como:

\begin{eqnarray*}
\esp\left[\sum_{t=\tau_{i}\left(m\right)}^{\tau_{i}\left(m+1\right)-1}z^{L_{i}\left(t\right)}\right]&=&\esp\left[\sum_{t=\tau_{i}\left(m\right)}^{\tau_{i}\left(m+1\right)-1}\left\{P_{i}\left(z\right)\right\}^{t-\overline{\tau}_{i}\left(m\right)}\right]=\frac{1-\esp\left[\left\{P_{i}\left(z\right)\right\}^{\tau_{i}\left(m+1\right)-\overline{\tau}_{i}\left(m\right)}\right]}{1-P_{i}\left(z\right)}\\
&=&\frac{1-I_{i}\left[P_{i}\left(z\right)\right]}{1-P_{i}\left(z\right)}
\end{eqnarray*}
por tanto

\begin{eqnarray*}
\esp\left[\sum_{t=\tau_{i}\left(m\right)}^{\tau_{i}\left(m+1\right)-1}z^{L_{i}\left(t\right)}\right]&=&\frac{1-F_{i}\left(z\right)}{1-P_{i}\left(z\right)}
\end{eqnarray*}

Haciendo uso de lo hasta ahora desarrollado se tiene que

\begin{eqnarray*}
Q_{i}\left(z\right)&=&\frac{1}{\esp\left[C_{i}\right]}\cdot\frac{1-F_{i}\left(z\right)}{P_{i}\left(z\right)-z}\cdot\frac{\left(1-z\right)P_{i}\left(z\right)}{1-P_{i}\left(z\right)}\\
&=&\frac{\mu_{i}\left(1-\mu_{i}\right)}{f_{i}\left(i\right)}\cdot\frac{1-F_{i}\left(z\right)}{P_{i}\left(z\right)-z}\cdot\frac{\left(1-z\right)P_{i}\left(z\right)}{1-P_{i}\left(z\right)}
\end{eqnarray*}


%___________________________________________________________________________________________
%\subsection{Longitudes de la Cola en cualquier tiempo}
%___________________________________________________________________________________________

Sea
$V_{i}\left(z\right)=\frac{1}{\esp\left[C_{i}\right]}\frac{I_{i}\left(z\right)-1}{z-P_{i}\left(z\right)}$

%{\esp\lef[I_{i}\right]}\frac{1-\mu_{i}}{z-P_{i}\left(z\right)}

\begin{eqnarray*}
\frac{\partial V_{i}\left(z\right)}{\partial z}&=&\frac{1}{\esp\left[C_{i}\right]}\left[\frac{I_{i}{'}\left(z\right)\left(z-P_{i}\left(z\right)\right)}{z-P_{i}\left(z\right)}-\frac{\left(I_{i}\left(z\right)-1\right)\left(1-P_{i}{'}\left(z\right)\right)}{\left(z-P_{i}\left(z\right)\right)^{2}}\right]
\end{eqnarray*}


La FGP para el tiempo de espera para cualquier usuario en la cola est\'a dada por:
\[U_{i}\left(z\right)=\frac{1}{\esp\left[C_{i}\right]}\cdot\frac{1-P_{i}\left(z\right)}{z-P_{i}\left(z\right)}\cdot\frac{I_{i}\left(z\right)-1}{1-z}\]

entonces


\begin{eqnarray*}
\frac{d}{dz}V_{i}\left(z\right)&=&\frac{1}{\esp\left[C_{i}\right]}\left\{\frac{d}{dz}\left(\frac{1-P_{i}\left(z\right)}{z-P_{i}\left(z\right)}\right)\frac{I_{i}\left(z\right)-1}{1-z}+\frac{1-P_{i}\left(z\right)}{z-P_{i}\left(z\right)}\frac{d}{dz}\left(\frac{I_{i}\left(z\right)-1}{1-z}\right)\right\}\\
&=&\frac{1}{\esp\left[C_{i}\right]}\left\{\frac{-P_{i}\left(z\right)\left(z-P_{i}\left(z\right)\right)-\left(1-P_{i}\left(z\right)\right)\left(1-P_{i}^{'}\left(z\right)\right)}{\left(z-P_{i}\left(z\right)\right)^{2}}\cdot\frac{I_{i}\left(z\right)-1}{1-z}\right\}\\
&+&\frac{1}{\esp\left[C_{i}\right]}\left\{\frac{1-P_{i}\left(z\right)}{z-P_{i}\left(z\right)}\cdot\frac{I_{i}^{'}\left(z\right)\left(1-z\right)+\left(I_{i}\left(z\right)-1\right)}{\left(1-z\right)^{2}}\right\}
\end{eqnarray*}
%\frac{I_{i}\left(z\right)-1}{1-z}
%+\frac{1-P_{i}\left(z\right)}{z-P_{i}\frac{d}{dz}\left(\frac{I_{i}\left(z\right)-1}{1-z}\right)


\begin{eqnarray*}
\frac{\partial U_{i}\left(z\right)}{\partial z}&=&\frac{(-1+I_{i}[z]) (1-P_{i}[z])}{(1-z)^2 \esp[I_{i}] (z-P_{i}[z])}+\frac{(1-P_{i}[z]) I_{i}^{'}[z]}{(1-z) \esp[I_{i}] (z-P_{i}[z])}-\frac{(-1+I_{i}[z]) (1-P_{i}[z])\left(1-P{'}[z]\right)}{(1-z) \esp[I_{i}] (z-P_{i}[z])^2}\\
&-&\frac{(-1+I_{i}[z]) P_{i}{'}[z]}{(1-z) \esp[I_{i}](z-P_{i}[z])}
\end{eqnarray*}
%___________________________________________________________________________________________
%\subsection{Longitudes de la Cola en cualquier tiempo}
%___________________________________________________________________________________________
Sea 
\begin{eqnarray*}
Q_{i}\left(z\right)&=&\frac{1}{\esp\left[C_{i}\right]}\cdot\frac{1-F_{i}\left(z\right)}{P_{i}\left(z\right)-z}\cdot\frac{\left(1-z\right)P_{i}\left(z\right)}{1-P_{i}\left(z\right)}
\end{eqnarray*}

derivando con respecto a $z$



\begin{eqnarray*}
\frac{d Q_{i}\left(z\right)}{d z}&=&\frac{\left(1-F_{i}\left(z\right)\right)P_{i}\left(z\right)}{\esp\left[C_{i}\right]\left(1-P_{i}\left(z\right)\right)\left(P_{i}\left(z\right)-z\right)}\\
&-&\frac{\left(1-z\right)P_{i}\left(z\right)F_{i}^{'}\left(z\right)}{\esp\left[C_{i}\right]\left(1-P_{i}\left(z\right)\right)\left(P_{i}\left(z\right)-z\right)}\\
&-&\frac{\left(1-z\right)\left(1-F_{i}\left(z\right)\right)P_{i}\left(z\right)\left(P_{i}^{'}\left(z\right)-1\right)}{\esp\left[C_{i}\right]\left(1-P_{i}\left(z\right)\right)\left(P_{i}\left(z\right)-z\right)^{2}}\\
&+&\frac{\left(1-z\right)\left(1-F_{i}\left(z\right)\right)P_{i}^{'}\left(z\right)}{\esp\left[C_{i}\right]\left(1-P_{i}\left(z\right)\right)\left(P_{i}\left(z\right)-z\right)}\\
&+&\frac{\left(1-z\right)\left(1-F_{i}\left(z\right)\right)P_{i}\left(z\right)P_{i}^{'}\left(z\right)}{\esp\left[C_{i}\right]\left(1-P_{i}\left(z\right)\right)^{2}\left(P_{i}\left(z\right)-z\right)}
\end{eqnarray*}

Calculando el l\'imite cuando $z\rightarrow1^{+}$:
\begin{eqnarray}
Q_{i}^{(1)}\left(z\right)=\lim_{z\rightarrow1^{+}}\frac{d Q_{i}\left(z\right)}{dz}&=&\lim_{z\rightarrow1}\frac{\left(1-F_{i}\left(z\right)\right)P_{i}\left(z\right)}{\esp\left[C_{i}\right]\left(1-P_{i}\left(z\right)\right)\left(P_{i}\left(z\right)-z\right)}\\
&-&\lim_{z\rightarrow1^{+}}\frac{\left(1-z\right)P_{i}\left(z\right)F_{i}^{'}\left(z\right)}{\esp\left[C_{i}\right]\left(1-P_{i}\left(z\right)\right)\left(P_{i}\left(z\right)-z\right)}\\
&-&\lim_{z\rightarrow1^{+}}\frac{\left(1-z\right)\left(1-F_{i}\left(z\right)\right)P_{i}\left(z\right)\left(P_{i}^{'}\left(z\right)-1\right)}{\esp\left[C_{i}\right]\left(1-P_{i}\left(z\right)\right)\left(P_{i}\left(z\right)-z\right)^{2}}\\
&+&\lim_{z\rightarrow1^{+}}\frac{\left(1-z\right)\left(1-F_{i}\left(z\right)\right)P_{i}^{'}\left(z\right)}{\esp\left[C_{i}\right]\left(1-P_{i}\left(z\right)\right)\left(P_{i}\left(z\right)-z\right)}\\
&+&\lim_{z\rightarrow1^{+}}\frac{\left(1-z\right)\left(1-F_{i}\left(z\right)\right)P_{i}\left(z\right)P_{i}^{'}\left(z\right)}{\esp\left[C_{i}\right]\left(1-P_{i}\left(z\right)\right)^{2}\left(P_{i}\left(z\right)-z\right)}
\end{eqnarray}

Entonces:
%______________________________________________________

\begin{eqnarray*}
\lim_{z\rightarrow1^{+}}\frac{\left(1-F_{i}\left(z\right)\right)P_{i}\left(z\right)}{\left(1-P_{i}\left(z\right)\right)\left(P_{i}\left(z\right)-z\right)}&=&\lim_{z\rightarrow1^{+}}\frac{\frac{d}{dz}\left[\left(1-F_{i}\left(z\right)\right)P_{i}\left(z\right)\right]}{\frac{d}{dz}\left[\left(1-P_{i}\left(z\right)\right)\left(-z+P_{i}\left(z\right)\right)\right]}\\
&=&\lim_{z\rightarrow1^{+}}\frac{-P_{i}\left(z\right)F_{i}^{'}\left(z\right)+\left(1-F_{i}\left(z\right)\right)P_{i}^{'}\left(z\right)}{\left(1-P_{i}\left(z\right)\right)\left(-1+P_{i}^{'}\left(z\right)\right)-\left(-z+P_{i}\left(z\right)\right)P_{i}^{'}\left(z\right)}
\end{eqnarray*}


%______________________________________________________


\begin{eqnarray*}
\lim_{z\rightarrow1^{+}}\frac{\left(1-z\right)P_{i}\left(z\right)F_{i}^{'}\left(z\right)}{\left(1-P_{i}\left(z\right)\right)\left(P_{i}\left(z\right)-z\right)}&=&\lim_{z\rightarrow1^{+}}\frac{\frac{d}{dz}\left[\left(1-z\right)P_{i}\left(z\right)F_{i}^{'}\left(z\right)\right]}{\frac{d}{dz}\left[\left(1-P_{i}\left(z\right)\right)\left(P_{i}\left(z\right)-z\right)\right]}\\
&=&\lim_{z\rightarrow1^{+}}\frac{-P_{i}\left(z\right) F_{i}^{'}\left(z\right)+(1-z) F_{i}^{'}\left(z\right) P_{i}^{'}\left(z\right)+(1-z) P_{i}\left(z\right)F_{i}^{''}\left(z\right)}{\left(1-P_{i}\left(z\right)\right)\left(-1+P_{i}^{'}\left(z\right)\right)-\left(-z+P_{i}\left(z\right)\right)P_{i}^{'}\left(z\right)}
\end{eqnarray*}


%______________________________________________________

\begin{eqnarray*}
&&\lim_{z\rightarrow1^{+}}\frac{\left(1-z\right)\left(1-F_{i}\left(z\right)\right)P_{i}\left(z\right)\left(P_{i}^{'}\left(z\right)-1\right)}{\left(1-P_{i}\left(z\right)\right)\left(P_{i}\left(z\right)-z\right)^{2}}=\lim_{z\rightarrow1^{+}}\frac{\frac{d}{dz}\left[\left(1-z\right)\left(1-F_{i}\left(z\right)\right)P_{i}\left(z\right)\left(P_{i}^{'}\left(z\right)-1\right)\right]}{\frac{d}{dz}\left[\left(1-P_{i}\left(z\right)\right)\left(P_{i}\left(z\right)-z\right)^{2}\right]}\\
&=&\lim_{z\rightarrow1^{+}}\frac{-\left(1-F_{i}\left(z\right)\right) P_{i}\left(z\right)\left(-1+P_{i}^{'}\left(z\right)\right)-(1-z) P_{i}\left(z\right)F_{i}^{'}\left(z\right)\left(-1+P_{i}^{'}\left(z\right)\right)}{2\left(1-P_{i}\left(z\right)\right)\left(-z+P_{i}\left(z\right)\right) \left(-1+P_{i}^{'}\left(z\right)\right)-\left(-z+P_{i}\left(z\right)\right)^2 P_{i}^{'}\left(z\right)}\\
&+&\lim_{z\rightarrow1^{+}}\frac{+(1-z) \left(1-F_{i}\left(z\right)\right) \left(-1+P_{i}^{'}\left(z\right)\right) P_{i}^{'}\left(z\right)}{{2\left(1-P_{i}\left(z\right)\right)\left(-z+P_{i}\left(z\right)\right) \left(-1+P_{i}^{'}\left(z\right)\right)-\left(-z+P_{i}\left(z\right)\right)^2 P_{i}^{'}\left(z\right)}}\\
&+&\lim_{z\rightarrow1^{+}}\frac{+(1-z) \left(1-F_{i}\left(z\right)\right) P_{i}\left(z\right)P_{i}^{''}\left(z\right)}{{2\left(1-P_{i}\left(z\right)\right)\left(-z+P_{i}\left(z\right)\right) \left(-1+P_{i}^{'}\left(z\right)\right)-\left(-z+P_{i}\left(z\right)\right)^2 P_{i}^{'}\left(z\right)}}
\end{eqnarray*}











%______________________________________________________
\begin{eqnarray*}
&&\lim_{z\rightarrow1^{+}}\frac{\left(1-z\right)\left(1-F_{i}\left(z\right)\right)P_{i}^{'}\left(z\right)}{\left(1-P_{i}\left(z\right)\right)\left(P_{i}\left(z\right)-z\right)}=\lim_{z\rightarrow1^{+}}\frac{\frac{d}{dz}\left[\left(1-z\right)\left(1-F_{i}\left(z\right)\right)P_{i}^{'}\left(z\right)\right]}{\frac{d}{dz}\left[\left(1-P_{i}\left(z\right)\right)\left(P_{i}\left(z\right)-z\right)\right]}\\
&=&\lim_{z\rightarrow1^{+}}\frac{-\left(1-F_{i}\left(z\right)\right) P_{i}^{'}\left(z\right)-(1-z) F_{i}^{'}\left(z\right) P_{i}^{'}\left(z\right)+(1-z) \left(1-F_{i}\left(z\right)\right) P_{i}^{''}\left(z\right)}{\left(1-P_{i}\left(z\right)\right) \left(-1+P_{i}^{'}\left(z\right)\right)-\left(-z+P_{i}\left(z\right)\right) P_{i}^{'}\left(z\right)}\frac{}{}
\end{eqnarray*}

%______________________________________________________
\begin{eqnarray*}
&&\lim_{z\rightarrow1^{+}}\frac{\left(1-z\right)\left(1-F_{i}\left(z\right)\right)P_{i}\left(z\right)P_{i}^{'}\left(z\right)}{\left(1-P_{i}\left(z\right)\right)^{2}\left(P_{i}\left(z\right)-z\right)}=\lim_{z\rightarrow1^{+}}\frac{\frac{d}{dz}\left[\left(1-z\right)\left(1-F_{i}\left(z\right)\right)P_{i}\left(z\right)P_{i}^{'}\left(z\right)\right]}{\frac{d}{dz}\left[\left(1-P_{i}\left(z\right)\right)^{2}\left(P_{i}\left(z\right)-z\right)\right]}\\
&=&\lim_{z\rightarrow1^{+}}\frac{-\left(1-F_{i}\left(z\right)\right) P_{i}\left(z\right) P_{i}^{'}\left(z\right)-(1-z) P_{i}\left(z\right) F_{i}^{'}\left(z\right)P_i'[z]}{\left(1-P_{i}\left(z\right)\right)^2 \left(-1+P_{i}^{'}\left(z\right)\right)-2 \left(1-P_{i}\left(z\right)\right) \left(-z+P_{i}\left(z\right)\right) P_{i}^{'}\left(z\right)}\\
&+&\lim_{z\rightarrow1^{+}}\frac{(1-z) \left(1-F_{i}\left(z\right)\right) P_{i}^{'}\left(z\right)^2+(1-z) \left(1-F_{i}\left(z\right)\right) P_{i}\left(z\right) P_{i}^{''}\left(z\right)}{\left(1-P_{i}\left(z\right)\right)^2 \left(-1+P_{i}^{'}\left(z\right)\right)-2 \left(1-P_{i}\left(z\right)\right) \left(-z+P_{i}\left(z\right)\right) P_{i}^{'}\left(z\right)}\\
\end{eqnarray*}




%_______________________________________________________________________________________________________
\subsection{Tiempo de Ciclo Promedio}
%_______________________________________________________________________________________________________

Consideremos una cola de la red de sistemas de visitas c\'iclicas fija, $Q_{l}$.


Conforme a la definici\'on dada al principio del cap\'itulo, definici\'on (\ref{Def.Tn}), sean $T_{1},T_{2},\ldots$ los puntos donde las longitudes de las colas de la red de sistemas de visitas c\'iclicas son cero simult\'aneamente, cuando la cola $Q_{l}$ es visitada por el servidor para dar servicio, es decir, $L_{1}\left(T_{i}\right)=0,L_{2}\left(T_{i}\right)=0,\hat{L}_{1}\left(T_{i}\right)=0$ y $\hat{L}_{2}\left(T_{i}\right)=0$, a estos puntos se les denominar\'a puntos regenerativos. Entonces, 

\begin{Def}
Al intervalo de tiempo entre dos puntos regenerativos se le llamar\'a ciclo regenerativo.
\end{Def}

\begin{Def}
Para $T_{i}$ se define, $M_{i}$, el n\'umero de ciclos de visita a la cola $Q_{l}$, durante el ciclo regenerativo, es decir, $M_{i}$ es un proceso de renovaci\'on.
\end{Def}

\begin{Def}
Para cada uno de los $M_{i}$'s, se definen a su vez la duraci\'on de cada uno de estos ciclos de visita en el ciclo regenerativo, $C_{i}^{(m)}$, para $m=1,2,\ldots,M_{i}$, que a su vez, tambi\'en es n proceso de renovaci\'on.
\end{Def}

En nuestra notaci\'on $V\left(t\right)\equiv C_{i}$ y $X_{i}=C_{i}^{(m)}$ para nuestra segunda definici\'on, mientras que para la primera la notaci\'on es: $X\left(t\right)\equiv C_{i}$ y $R_{i}\equiv C_{i}^{(m)}$.


%___________________________________________________________________________________________
\subsection{Tiempos de Ciclo e Intervisita}
%___________________________________________________________________________________________


\begin{Def}
Sea $L_{i}^{*}$el n\'umero de usuarios en la cola $Q_{i}$ cuando es visitada por el servidor para dar servicio, entonces

\begin{eqnarray}
\esp\left[L_{i}^{*}\right]&=&f_{i}\left(i\right)\\
Var\left[L_{i}^{*}\right]&=&f_{i}\left(i,i\right)+\esp\left[L_{i}^{*}\right]-\esp\left[L_{i}^{*}\right]^{2}.
\end{eqnarray}

\end{Def}

\begin{Def}
El tiempo de Ciclo $C_{i}$ es e periodo de tiempo que comienza cuando la cola $i$ es visitada por primera vez en un ciclo, y termina cuando es visitado nuevamente en el pr\'oximo ciclo. La duraci\'on del mismo est\'a dada por $\tau_{i}\left(m+1\right)-\tau_{i}\left(m\right)$, o equivalentemente $\overline{\tau}_{i}\left(m+1\right)-\overline{\tau}_{i}\left(m\right)$ bajo condiciones de estabilidad.
\end{Def}

\begin{Def}
El tiempo de intervisita $I_{i}$ es el periodo de tiempo que comienza cuando se ha completado el servicio en un ciclo y termina cuando es visitada nuevamente en el pr\'oximo ciclo. Su  duraci\'on del mismo est\'a dada por $\tau_{i}\left(m+1\right)-\overline{\tau}_{i}\left(m\right)$.
\end{Def}


Recordemos las siguientes expresiones:

\begin{eqnarray*}
S_{i}\left(z\right)&=&\esp\left[z^{\overline{\tau}_{i}\left(m\right)-\tau_{i}\left(m\right)}\right]=F_{i}\left(\theta\left(z\right)\right),\\
F\left(z\right)&=&\esp\left[z^{L_{0}}\right],\\
P\left(z\right)&=&\esp\left[z^{X_{n}}\right],\\
F_{i}\left(z\right)&=&\esp\left[z^{L_{i}\left(\tau_{i}\left(m\right)\right)}\right],
\theta_{i}\left(z\right)-zP_{i}
\end{eqnarray*}

entonces 

\begin{eqnarray*}
\esp\left[S_{i}\right]&=&\frac{\esp\left[L_{i}^{*}\right]}{1-\mu_{i}}=\frac{f_{i}\left(i\right)}{1-\mu_{i}},\\
Var\left[S_{i}\right]&=&\frac{Var\left[L_{i}^{*}\right]}{\left(1-\mu_{i}\right)^{2}}+\frac{\sigma^{2}\esp\left[L_{i}^{*}\right]}{\left(1-\mu_{i}\right)^{3}}
\end{eqnarray*}

donde recordemos que

\begin{eqnarray*}
Var\left[L_{i}^{*}\right]&=&f_{i}\left(i,i\right)+f_{i}\left(i\right)-f_{i}\left(i\right)^{2}.
\end{eqnarray*}

La duraci\'on del tiempo de intervisita es $\tau_{i}\left(m+1\right)-\overline{\tau}\left(m\right)$. Dado que el n\'umero de usuarios presentes en $Q_{i}$ al tiempo $t=\tau_{i}\left(m+1\right)$ es igual al n\'umero de arribos durante el intervalo de tiempo $\left[\overline{\tau}\left(m\right),\tau_{i}\left(m+1\right)\right]$ se tiene que


\begin{eqnarray*}
\esp\left[z_{i}^{L_{i}\left(\tau_{i}\left(m+1\right)\right)}\right]=\esp\left[\left\{P_{i}\left(z_{i}\right)\right\}^{\tau_{i}\left(m+1\right)-\overline{\tau}\left(m\right)}\right]
\end{eqnarray*}

entonces, si \begin{eqnarray*}I_{i}\left(z\right)&=&\esp\left[z^{\tau_{i}\left(m+1\right)-\overline{\tau}\left(m\right)}\right]\end{eqnarray*} se tienen que

\begin{eqnarray*}
F_{i}\left(z\right)=I_{i}\left[P_{i}\left(z\right)\right]
\end{eqnarray*}
para $i=1,2$, por tanto



\begin{eqnarray*}
\esp\left[L_{i}^{*}\right]&=&\mu_{i}\esp\left[I_{i}\right]\\
Var\left[L_{i}^{*}\right]&=&\mu_{i}^{2}Var\left[I_{i}\right]+\sigma^{2}\esp\left[I_{i}\right]
\end{eqnarray*}
para $i=1,2$, por tanto


\begin{eqnarray*}
\esp\left[I_{i}\right]&=&\frac{f_{i}\left(i\right)}{\mu_{i}},
\end{eqnarray*}
adem\'as

\begin{eqnarray*}
Var\left[I_{i}\right]&=&\frac{Var\left[L_{i}^{*}\right]}{\mu_{i}^{2}}-\frac{\sigma_{i}^{2}}{\mu_{i}^{2}}f_{i}\left(i\right).
\end{eqnarray*}


Si  $C_{i}\left(z\right)=\esp\left[z^{\overline{\tau}\left(m+1\right)-\overline{\tau}_{i}\left(m\right)}\right]$el tiempo de duraci\'on del ciclo, entonces, por lo hasta ahora establecido, se tiene que

\begin{eqnarray*}
C_{i}\left(z\right)=I_{i}\left[\theta_{i}\left(z\right)\right],
\end{eqnarray*}
entonces

\begin{eqnarray*}
\esp\left[C_{i}\right]&=&\esp\left[I_{i}\right]\esp\left[\theta_{i}\left(z\right)\right]=\frac{\esp\left[L_{i}^{*}\right]}{\mu_{i}}\frac{1}{1-\mu_{i}}=\frac{f_{i}\left(i\right)}{\mu_{i}\left(1-\mu_{i}\right)}\\
Var\left[C_{i}\right]&=&\frac{Var\left[L_{i}^{*}\right]}{\mu_{i}^{2}\left(1-\mu_{i}\right)^{2}}.
\end{eqnarray*}

Por tanto se tienen las siguientes igualdades


\begin{eqnarray*}
\esp\left[S_{i}\right]&=&\mu_{i}\esp\left[C_{i}\right],\\
\esp\left[I_{i}\right]&=&\left(1-\mu_{i}\right)\esp\left[C_{i}\right]\\
\end{eqnarray*}

Def\'inanse los puntos de regenaraci\'on  en el proceso $\left[L_{1}\left(t\right),L_{2}\left(t\right),\ldots,L_{N}\left(t\right)\right]$. Los puntos cuando la cola $i$ es visitada y todos los $L_{j}\left(\tau_{i}\left(m\right)\right)=0$ para $i=1,2$  son puntos de regeneraci\'on. Se llama ciclo regenerativo al intervalo entre dos puntos regenerativos sucesivos.

Sea $M_{i}$  el n\'umero de ciclos de visita en un ciclo regenerativo, y sea $C_{i}^{(m)}$, para $m=1,2,\ldots,M_{i}$ la duraci\'on del $m$-\'esimo ciclo de visita en un ciclo regenerativo. Se define el ciclo del tiempo de visita promedio $\esp\left[C_{i}\right]$ como

\begin{eqnarray*}
\esp\left[C_{i}\right]&=&\frac{\esp\left[\sum_{m=1}^{M_{i}}C_{i}^{(m)}\right]}{\esp\left[M_{i}\right]}
\end{eqnarray*}


En Stid72 y Heym82 se muestra que una condici\'on suficiente para que el proceso regenerativo 
estacionario sea un procesoo estacionario es que el valor esperado del tiempo del ciclo regenerativo sea finito:

\begin{eqnarray*}
\esp\left[\sum_{m=1}^{M_{i}}C_{i}^{(m)}\right]<\infty.
\end{eqnarray*}

como cada $C_{i}^{(m)}$ contiene intervalos de r\'eplica positivos, se tiene que $\esp\left[M_{i}\right]<\infty$, adem\'as, como $M_{i}>0$, se tiene que la condici\'on anterior es equivalente a tener que 

\begin{eqnarray*}
\esp\left[C_{i}\right]<\infty,
\end{eqnarray*}
por lo tanto una condici\'on suficiente para la existencia del proceso regenerativo est\'a dada por

\begin{eqnarray*}
\sum_{k=1}^{N}\mu_{k}<1.
\end{eqnarray*}

Sea la funci\'on generadora de momentos para $L_{i}$, el n\'umero de usuarios en la cola $Q_{i}\left(z\right)$ en cualquier momento, est\'a dada por el tiempo promedio de $z^{L_{i}\left(t\right)}$ sobre el ciclo regenerativo definido anteriormente:

\begin{eqnarray*}
Q_{i}\left(z\right)&=&\esp\left[z^{L_{i}\left(t\right)}\right]=\frac{\esp\left[\sum_{m=1}^{M_{i}}\sum_{t=\tau_{i}\left(m\right)}^{\tau_{i}\left(m+1\right)-1}z^{L_{i}\left(t\right)}\right]}{\esp\left[\sum_{m=1}^{M_{i}}\tau_{i}\left(m+1\right)-\tau_{i}\left(m\right)\right]}
\end{eqnarray*}

$M_{i}$ es un tiempo de paro en el proceso regenerativo con $\esp\left[M_{i}\right]<\infty$, se sigue del lema de Wald que:


\begin{eqnarray*}
\esp\left[\sum_{m=1}^{M_{i}}\sum_{t=\tau_{i}\left(m\right)}^{\tau_{i}\left(m+1\right)-1}z^{L_{i}\left(t\right)}\right]&=&\esp\left[M_{i}\right]\esp\left[\sum_{t=\tau_{i}\left(m\right)}^{\tau_{i}\left(m+1\right)-1}z^{L_{i}\left(t\right)}\right]\\
\esp\left[\sum_{m=1}^{M_{i}}\tau_{i}\left(m+1\right)-\tau_{i}\left(m\right)\right]&=&\esp\left[M_{i}\right]\esp\left[\tau_{i}\left(m+1\right)-\tau_{i}\left(m\right)\right]
\end{eqnarray*}

por tanto se tiene que


\begin{eqnarray*}
Q_{i}\left(z\right)&=&\frac{\esp\left[\sum_{t=\tau_{i}\left(m\right)}^{\tau_{i}\left(m+1\right)-1}z^{L_{i}\left(t\right)}\right]}{\esp\left[\tau_{i}\left(m+1\right)-\tau_{i}\left(m\right)\right]}
\end{eqnarray*}

observar que el denominador es simplemente la duraci\'on promedio del tiempo del ciclo.


Se puede demostrar (ver Hideaki Takagi 1986) que

\begin{eqnarray*}
\esp\left[\sum_{t=\tau_{i}\left(m\right)}^{\tau_{i}\left(m+1\right)-1}z^{L_{i}\left(t\right)}\right]=z\frac{F_{i}\left(z\right)-1}{z-P_{i}\left(z\right)}
\end{eqnarray*}

Durante el tiempo de intervisita para la cola $i$, $L_{i}\left(t\right)$ solamente se incrementa de manera que el incremento por intervalo de tiempo est\'a dado por la funci\'on generadora de probabilidades de $P_{i}\left(z\right)$, por tanto la suma sobre el tiempo de intervisita puede evaluarse como:

\begin{eqnarray*}
\esp\left[\sum_{t=\tau_{i}\left(m\right)}^{\tau_{i}\left(m+1\right)-1}z^{L_{i}\left(t\right)}\right]&=&\esp\left[\sum_{t=\tau_{i}\left(m\right)}^{\tau_{i}\left(m+1\right)-1}\left\{P_{i}\left(z\right)\right\}^{t-\overline{\tau}_{i}\left(m\right)}\right]=\frac{1-\esp\left[\left\{P_{i}\left(z\right)\right\}^{\tau_{i}\left(m+1\right)-\overline{\tau}_{i}\left(m\right)}\right]}{1-P_{i}\left(z\right)}\\
&=&\frac{1-I_{i}\left[P_{i}\left(z\right)\right]}{1-P_{i}\left(z\right)}
\end{eqnarray*}
por tanto

\begin{eqnarray*}
\esp\left[\sum_{t=\tau_{i}\left(m\right)}^{\tau_{i}\left(m+1\right)-1}z^{L_{i}\left(t\right)}\right]&=&\frac{1-F_{i}\left(z\right)}{1-P_{i}\left(z\right)}
\end{eqnarray*}

Haciendo uso de lo hasta ahora desarrollado se tiene que

\begin{eqnarray*}
Q_{i}\left(z\right)&=&\frac{1}{\esp\left[C_{i}\right]}\cdot\frac{1-F_{i}\left(z\right)}{P_{i}\left(z\right)-z}\cdot\frac{\left(1-z\right)P_{i}\left(z\right)}{1-P_{i}\left(z\right)}\\
&=&\frac{\mu_{i}\left(1-\mu_{i}\right)}{f_{i}\left(i\right)}\cdot\frac{1-F_{i}\left(z\right)}{P_{i}\left(z\right)-z}\cdot\frac{\left(1-z\right)P_{i}\left(z\right)}{1-P_{i}\left(z\right)}
\end{eqnarray*}


%___________________________________________________________________________________________
\subsection{Longitudes de la Cola en cualquier tiempo}
%___________________________________________________________________________________________
Sea 
\begin{eqnarray*}
Q_{i}\left(z\right)&=&\frac{1}{\esp\left[C_{i}\right]}\cdot\frac{1-F_{i}\left(z\right)}{P_{i}\left(z\right)-z}\cdot\frac{\left(1-z\right)P_{i}\left(z\right)}{1-P_{i}\left(z\right)}
\end{eqnarray*}

derivando con respecto a $z$



\begin{eqnarray*}
\frac{d Q_{i}\left(z\right)}{d z}&=&\frac{\left(1-F_{i}\left(z\right)\right)P_{i}\left(z\right)}{\esp\left[C_{i}\right]\left(1-P_{i}\left(z\right)\right)\left(P_{i}\left(z\right)-z\right)}\\
&-&\frac{\left(1-z\right)P_{i}\left(z\right)F_{i}^{'}\left(z\right)}{\esp\left[C_{i}\right]\left(1-P_{i}\left(z\right)\right)\left(P_{i}\left(z\right)-z\right)}\\
&-&\frac{\left(1-z\right)\left(1-F_{i}\left(z\right)\right)P_{i}\left(z\right)\left(P_{i}^{'}\left(z\right)-1\right)}{\esp\left[C_{i}\right]\left(1-P_{i}\left(z\right)\right)\left(P_{i}\left(z\right)-z\right)^{2}}\\
&+&\frac{\left(1-z\right)\left(1-F_{i}\left(z\right)\right)P_{i}^{'}\left(z\right)}{\esp\left[C_{i}\right]\left(1-P_{i}\left(z\right)\right)\left(P_{i}\left(z\right)-z\right)}\\
&+&\frac{\left(1-z\right)\left(1-F_{i}\left(z\right)\right)P_{i}\left(z\right)P_{i}^{'}\left(z\right)}{\esp\left[C_{i}\right]\left(1-P_{i}\left(z\right)\right)^{2}\left(P_{i}\left(z\right)-z\right)}
\end{eqnarray*}

Calculando el l\'imite cuando $z\rightarrow1^{+}$:
\begin{eqnarray}
Q_{i}^{(1)}\left(z\right)=\lim_{z\rightarrow1^{+}}\frac{d Q_{i}\left(z\right)}{dz}&=&\lim_{z\rightarrow1}\frac{\left(1-F_{i}\left(z\right)\right)P_{i}\left(z\right)}{\esp\left[C_{i}\right]\left(1-P_{i}\left(z\right)\right)\left(P_{i}\left(z\right)-z\right)}\\
&-&\lim_{z\rightarrow1^{+}}\frac{\left(1-z\right)P_{i}\left(z\right)F_{i}^{'}\left(z\right)}{\esp\left[C_{i}\right]\left(1-P_{i}\left(z\right)\right)\left(P_{i}\left(z\right)-z\right)}\\
&-&\lim_{z\rightarrow1^{+}}\frac{\left(1-z\right)\left(1-F_{i}\left(z\right)\right)P_{i}\left(z\right)\left(P_{i}^{'}\left(z\right)-1\right)}{\esp\left[C_{i}\right]\left(1-P_{i}\left(z\right)\right)\left(P_{i}\left(z\right)-z\right)^{2}}\\
&+&\lim_{z\rightarrow1^{+}}\frac{\left(1-z\right)\left(1-F_{i}\left(z\right)\right)P_{i}^{'}\left(z\right)}{\esp\left[C_{i}\right]\left(1-P_{i}\left(z\right)\right)\left(P_{i}\left(z\right)-z\right)}\\
&+&\lim_{z\rightarrow1^{+}}\frac{\left(1-z\right)\left(1-F_{i}\left(z\right)\right)P_{i}\left(z\right)P_{i}^{'}\left(z\right)}{\esp\left[C_{i}\right]\left(1-P_{i}\left(z\right)\right)^{2}\left(P_{i}\left(z\right)-z\right)}
\end{eqnarray}

Entonces:
%______________________________________________________

\begin{eqnarray*}
\lim_{z\rightarrow1^{+}}\frac{\left(1-F_{i}\left(z\right)\right)P_{i}\left(z\right)}{\left(1-P_{i}\left(z\right)\right)\left(P_{i}\left(z\right)-z\right)}&=&\lim_{z\rightarrow1^{+}}\frac{\frac{d}{dz}\left[\left(1-F_{i}\left(z\right)\right)P_{i}\left(z\right)\right]}{\frac{d}{dz}\left[\left(1-P_{i}\left(z\right)\right)\left(-z+P_{i}\left(z\right)\right)\right]}\\
&=&\lim_{z\rightarrow1^{+}}\frac{-P_{i}\left(z\right)F_{i}^{'}\left(z\right)+\left(1-F_{i}\left(z\right)\right)P_{i}^{'}\left(z\right)}{\left(1-P_{i}\left(z\right)\right)\left(-1+P_{i}^{'}\left(z\right)\right)-\left(-z+P_{i}\left(z\right)\right)P_{i}^{'}\left(z\right)}
\end{eqnarray*}


%______________________________________________________


\begin{eqnarray*}
\lim_{z\rightarrow1^{+}}\frac{\left(1-z\right)P_{i}\left(z\right)F_{i}^{'}\left(z\right)}{\left(1-P_{i}\left(z\right)\right)\left(P_{i}\left(z\right)-z\right)}&=&\lim_{z\rightarrow1^{+}}\frac{\frac{d}{dz}\left[\left(1-z\right)P_{i}\left(z\right)F_{i}^{'}\left(z\right)\right]}{\frac{d}{dz}\left[\left(1-P_{i}\left(z\right)\right)\left(P_{i}\left(z\right)-z\right)\right]}\\
&=&\lim_{z\rightarrow1^{+}}\frac{-P_{i}\left(z\right) F_{i}^{'}\left(z\right)+(1-z) F_{i}^{'}\left(z\right) P_{i}^{'}\left(z\right)+(1-z) P_{i}\left(z\right)F_{i}^{''}\left(z\right)}{\left(1-P_{i}\left(z\right)\right)\left(-1+P_{i}^{'}\left(z\right)\right)-\left(-z+P_{i}\left(z\right)\right)P_{i}^{'}\left(z\right)}
\end{eqnarray*}


%______________________________________________________

\begin{eqnarray*}
&&\lim_{z\rightarrow1^{+}}\frac{\left(1-z\right)\left(1-F_{i}\left(z\right)\right)P_{i}\left(z\right)\left(P_{i}^{'}\left(z\right)-1\right)}{\left(1-P_{i}\left(z\right)\right)\left(P_{i}\left(z\right)-z\right)^{2}}=\lim_{z\rightarrow1^{+}}\frac{\frac{d}{dz}\left[\left(1-z\right)\left(1-F_{i}\left(z\right)\right)P_{i}\left(z\right)\left(P_{i}^{'}\left(z\right)-1\right)\right]}{\frac{d}{dz}\left[\left(1-P_{i}\left(z\right)\right)\left(P_{i}\left(z\right)-z\right)^{2}\right]}\\
&=&\lim_{z\rightarrow1^{+}}\frac{-\left(1-F_{i}\left(z\right)\right) P_{i}\left(z\right)\left(-1+P_{i}^{'}\left(z\right)\right)-(1-z) P_{i}\left(z\right)F_{i}^{'}\left(z\right)\left(-1+P_{i}^{'}\left(z\right)\right)}{2\left(1-P_{i}\left(z\right)\right)\left(-z+P_{i}\left(z\right)\right) \left(-1+P_{i}^{'}\left(z\right)\right)-\left(-z+P_{i}\left(z\right)\right)^2 P_{i}^{'}\left(z\right)}\\
&+&\lim_{z\rightarrow1^{+}}\frac{+(1-z) \left(1-F_{i}\left(z\right)\right) \left(-1+P_{i}^{'}\left(z\right)\right) P_{i}^{'}\left(z\right)}{{2\left(1-P_{i}\left(z\right)\right)\left(-z+P_{i}\left(z\right)\right) \left(-1+P_{i}^{'}\left(z\right)\right)-\left(-z+P_{i}\left(z\right)\right)^2 P_{i}^{'}\left(z\right)}}\\
&+&\lim_{z\rightarrow1^{+}}\frac{+(1-z) \left(1-F_{i}\left(z\right)\right) P_{i}\left(z\right)P_{i}^{''}\left(z\right)}{{2\left(1-P_{i}\left(z\right)\right)\left(-z+P_{i}\left(z\right)\right) \left(-1+P_{i}^{'}\left(z\right)\right)-\left(-z+P_{i}\left(z\right)\right)^2 P_{i}^{'}\left(z\right)}}
\end{eqnarray*}











%______________________________________________________
\begin{eqnarray*}
&&\lim_{z\rightarrow1^{+}}\frac{\left(1-z\right)\left(1-F_{i}\left(z\right)\right)P_{i}^{'}\left(z\right)}{\left(1-P_{i}\left(z\right)\right)\left(P_{i}\left(z\right)-z\right)}=\lim_{z\rightarrow1^{+}}\frac{\frac{d}{dz}\left[\left(1-z\right)\left(1-F_{i}\left(z\right)\right)P_{i}^{'}\left(z\right)\right]}{\frac{d}{dz}\left[\left(1-P_{i}\left(z\right)\right)\left(P_{i}\left(z\right)-z\right)\right]}\\
&=&\lim_{z\rightarrow1^{+}}\frac{-\left(1-F_{i}\left(z\right)\right) P_{i}^{'}\left(z\right)-(1-z) F_{i}^{'}\left(z\right) P_{i}^{'}\left(z\right)+(1-z) \left(1-F_{i}\left(z\right)\right) P_{i}^{''}\left(z\right)}{\left(1-P_{i}\left(z\right)\right) \left(-1+P_{i}^{'}\left(z\right)\right)-\left(-z+P_{i}\left(z\right)\right) P_{i}^{'}\left(z\right)}\frac{}{}
\end{eqnarray*}

%______________________________________________________
\begin{eqnarray*}
&&\lim_{z\rightarrow1^{+}}\frac{\left(1-z\right)\left(1-F_{i}\left(z\right)\right)P_{i}\left(z\right)P_{i}^{'}\left(z\right)}{\left(1-P_{i}\left(z\right)\right)^{2}\left(P_{i}\left(z\right)-z\right)}=\lim_{z\rightarrow1^{+}}\frac{\frac{d}{dz}\left[\left(1-z\right)\left(1-F_{i}\left(z\right)\right)P_{i}\left(z\right)P_{i}^{'}\left(z\right)\right]}{\frac{d}{dz}\left[\left(1-P_{i}\left(z\right)\right)^{2}\left(P_{i}\left(z\right)-z\right)\right]}\\
&=&\lim_{z\rightarrow1^{+}}\frac{-\left(1-F_{i}\left(z\right)\right) P_{i}\left(z\right) P_{i}^{'}\left(z\right)-(1-z) P_{i}\left(z\right) F_{i}^{'}\left(z\right)P_i'[z]}{\left(1-P_{i}\left(z\right)\right)^2 \left(-1+P_{i}^{'}\left(z\right)\right)-2 \left(1-P_{i}\left(z\right)\right) \left(-z+P_{i}\left(z\right)\right) P_{i}^{'}\left(z\right)}\\
&+&\lim_{z\rightarrow1^{+}}\frac{(1-z) \left(1-F_{i}\left(z\right)\right) P_{i}^{'}\left(z\right)^2+(1-z) \left(1-F_{i}\left(z\right)\right) P_{i}\left(z\right) P_{i}^{''}\left(z\right)}{\left(1-P_{i}\left(z\right)\right)^2 \left(-1+P_{i}^{'}\left(z\right)\right)-2 \left(1-P_{i}\left(z\right)\right) \left(-z+P_{i}\left(z\right)\right) P_{i}^{'}\left(z\right)}\\
\end{eqnarray*}

\subsection{Por resolver}



\begin{eqnarray*}
&&\frac{\partial Q_{i}\left(z\right)}{\partial z}=\frac{1}{\esp\left[C_{i}\right]}\frac{\partial}{\partial z}\left\{\frac{1-F_{i}\left(z\right)}{P_{i}\left(z\right)-z}\cdot\frac{\left(1-z\right)P_{i}\left(z\right)}{1-P_{i}\left(z\right)}\right\}\\
&=&\frac{1}{\esp\left[C_{i}\right]}\left\{\frac{\partial}{\partial z}\left(\frac{1-F_{i}\left(z\right)}{P_{i}\left(z\right)-z}\right)\cdot\frac{\left(1-z\right)P_{i}\left(z\right)}{1-P_{i}\left(z\right)}+\frac{1-F_{i}\left(z\right)}{P_{i}\left(z\right)-z}\cdot\frac{\partial}{\partial z}\left(\frac{\left(1-z\right)P_{i}\left(z\right)}{1-P_{i}\left(z\right)}\right)\right\}\\
&=&\frac{1}{\esp\left[C_{i}\right]}\cdot\frac{\left(1-z\right)P_{i}\left(z\right)}{1-P_{i}\left(z\right)}\cdot\frac{\partial}{\partial z}\left(\frac{1-F_{i}\left(z\right)}{P_{i}\left(z\right)-z}\right)+\frac{1}{\esp\left[C_{i}\right]}\cdot\frac{1-F_{i}\left(z\right)}{P_{i}\left(z\right)-z}\cdot\frac{\partial}{\partial z}\left(\frac{\left(1-z\right)P_{i}\left(z\right)}{1-P_{i}\left(z\right)}\right)\\
&=&\frac{1}{\esp\left[C_{i}\right]}\cdot\frac{\left(1-z\right)P_{i}\left(z\right)}{1-P_{i}\left(z\right)}\cdot\frac{-F_{i}^{'}\left(z\right)\left(P_{i}\left(z\right)-z\right)-\left(1-F_{i}\left(z\right)\right)\left(P_{i}^{'}\left(z\right)-1\right)}{\left(P_{i}\left(z\right)-z\right)^{2}}\\
&+&\frac{1}{\esp\left[C_{i}\right]}\cdot\frac{1-F_{i}\left(z\right)}{P_{i}\left(z\right)-z}\cdot\frac{\left(1-z\right)P_{i}^{'}\left(z\right)-P_{i}\left(z\right)}{\left(1-P_{i}\left(z\right)\right)^{2}}
\end{eqnarray*}



\begin{eqnarray*}
Q_{i}^{(1)}\left(z\right)&=& \frac{\left(1-F_{i}\left(z\right)\right)P_{i}\left(z\right)}{\esp\left[C_{i}\right]\left(1-P_{i}\left(z\right)\right)\left(P_{i}\left(z\right)-z\right)}
-\frac{\left(1-z\right)P_{i}\left(z\right)F_{i}^{'}\left(z\right)}{\esp\left[C_{i}\right]\left(1-P_{i}\left(z\right)\right)\left(P_{i}\left(z\right)-z\right)}\\
&-&\frac{\left(1-z\right)\left(1-F_{i}\left(z\right)\right)P_{i}\left(z\right)\left(P_{i}^{'}\left(z\right)-1\right)}{\esp\left[C_{i}\right]\left(1-P_{i}\left(z\right)\right)\left(P_{i}\left(z\right)-z\right)^{2}}+\frac{\left(1-z\right)\left(1-F_{i}\left(z\right)\right)P_{i}^{'}\left(z\right)}{\esp\left[C_{i}\right]\left(1-P_{i}\left(z\right)\right)\left(P_{i}\left(z\right)-z\right)}\\
&+&\frac{\left(1-z\right)\left(1-F_{i}\left(z\right)\right)P_{i}\left(z\right)P_{i}^{'}\left(z\right)}{\esp\left[C_{i}\right]\left(1-P_{i}\left(z\right)\right)^{2}\left(P_{i}\left(z\right)-z\right)}
\end{eqnarray*}
%___________________________________________________________________________________________
%\subsection{Operaciones Matemathica: Tiempos de Espera}
%___________________________________________________________________________________________
Sea
$V_{i}\left(z\right)=\frac{1}{\esp\left[C_{i}\right]}\frac{I_{i}\left(z\right)-1}{z-P_{i}\left(z\right)}$

%{\esp\lef[I_{i}\right]}\frac{1-\mu_{i}}{z-P_{i}\left(z\right)}

\begin{eqnarray*}
\frac{\partial V_{i}\left(z\right)}{\partial z}&=&\frac{1}{\esp\left[C_{i}\right]}\left[\frac{I_{i}{'}\left(z\right)\left(z-P_{i}\left(z\right)\right)}{z-P_{i}\left(z\right)}-\frac{\left(I_{i}\left(z\right)-1\right)\left(1-P_{i}{'}\left(z\right)\right)}{\left(z-P_{i}\left(z\right)\right)^{2}}\right]
\end{eqnarray*}


La FGP para el tiempo de espera para cualquier usuario en la cola est\'a dada por:
\[U_{i}\left(z\right)=\frac{1}{\esp\left[C_{i}\right]}\cdot\frac{1-P_{i}\left(z\right)}{z-P_{i}\left(z\right)}\cdot\frac{I_{i}\left(z\right)-1}{1-z}\]

entonces


\begin{eqnarray*}
\frac{d}{dz}V_{i}\left(z\right)&=&\frac{1}{\esp\left[C_{i}\right]}\left\{\frac{d}{dz}\left(\frac{1-P_{i}\left(z\right)}{z-P_{i}\left(z\right)}\right)\frac{I_{i}\left(z\right)-1}{1-z}+\frac{1-P_{i}\left(z\right)}{z-P_{i}\left(z\right)}\frac{d}{dz}\left(\frac{I_{i}\left(z\right)-1}{1-z}\right)\right\}\\
&=&\frac{1}{\esp\left[C_{i}\right]}\left\{\frac{-P_{i}\left(z\right)\left(z-P_{i}\left(z\right)\right)-\left(1-P_{i}\left(z\right)\right)\left(1-P_{i}^{'}\left(z\right)\right)}{\left(z-P_{i}\left(z\right)\right)^{2}}\cdot\frac{I_{i}\left(z\right)-1}{1-z}\right\}\\
&+&\frac{1}{\esp\left[C_{i}\right]}\left\{\frac{1-P_{i}\left(z\right)}{z-P_{i}\left(z\right)}\cdot\frac{I_{i}^{'}\left(z\right)\left(1-z\right)+\left(I_{i}\left(z\right)-1\right)}{\left(1-z\right)^{2}}\right\}
\end{eqnarray*}
%\frac{I_{i}\left(z\right)-1}{1-z}
%+\frac{1-P_{i}\left(z\right)}{z-P_{i}\frac{d}{dz}\left(\frac{I_{i}\left(z\right)-1}{1-z}\right)


\begin{eqnarray*}
\frac{\partial U_{i}\left(z\right)}{\partial z}&=&\frac{(-1+I_{i}[z]) (1-P_{i}[z])}{(1-z)^2 \esp[I_{i}] (z-P_{i}[z])}+\frac{(1-P_{i}[z]) I_{i}^{'}[z]}{(1-z) \esp[I_{i}] (z-P_{i}[z])}-\frac{(-1+I_{i}[z]) (1-P_{i}[z])\left(1-P{'}[z]\right)}{(1-z) \esp[I_{i}] (z-P_{i}[z])^2}\\
&-&\frac{(-1+I_{i}[z]) P_{i}{'}[z]}{(1-z) \esp[I_{i}](z-P_{i}[z])}
\end{eqnarray*}

%___________________________________________________________________________________________
\subsection{Tiempos de Ciclo e Intervisita}
%___________________________________________________________________________________________


\begin{Def}
Sea $L_{i}^{*}$el n\'umero de usuarios en la cola $Q_{i}$ cuando es visitada por el servidor para dar servicio, entonces

\begin{eqnarray}
\esp\left[L_{i}^{*}\right]&=&f_{i}\left(i\right)\\
Var\left[L_{i}^{*}\right]&=&f_{i}\left(i,i\right)+\esp\left[L_{i}^{*}\right]-\esp\left[L_{i}^{*}\right]^{2}.
\end{eqnarray}

\end{Def}

\begin{Def}
El tiempo de Ciclo $C_{i}$ es e periodo de tiempo que comienza cuando la cola $i$ es visitada por primera vez en un ciclo, y termina cuando es visitado nuevamente en el pr\'oximo ciclo. La duraci\'on del mismo est\'a dada por $\tau_{i}\left(m+1\right)-\tau_{i}\left(m\right)$, o equivalentemente $\overline{\tau}_{i}\left(m+1\right)-\overline{\tau}_{i}\left(m\right)$ bajo condiciones de estabilidad.
\end{Def}

\begin{Def}
El tiempo de intervisita $I_{i}$ es el periodo de tiempo que comienza cuando se ha completado el servicio en un ciclo y termina cuando es visitada nuevamente en el pr\'oximo ciclo. Su  duraci\'on del mismo est\'a dada por $\tau_{i}\left(m+1\right)-\overline{\tau}_{i}\left(m\right)$.
\end{Def}


Recordemos las siguientes expresiones:

\begin{eqnarray*}
S_{i}\left(z\right)&=&\esp\left[z^{\overline{\tau}_{i}\left(m\right)-\tau_{i}\left(m\right)}\right]=F_{i}\left(\theta\left(z\right)\right),\\
F\left(z\right)&=&\esp\left[z^{L_{0}}\right],\\
P\left(z\right)&=&\esp\left[z^{X_{n}}\right],\\
F_{i}\left(z\right)&=&\esp\left[z^{L_{i}\left(\tau_{i}\left(m\right)\right)}\right],
\theta_{i}\left(z\right)-zP_{i}
\end{eqnarray*}

entonces 

\begin{eqnarray*}
\esp\left[S_{i}\right]&=&\frac{\esp\left[L_{i}^{*}\right]}{1-\mu_{i}}=\frac{f_{i}\left(i\right)}{1-\mu_{i}},\\
Var\left[S_{i}\right]&=&\frac{Var\left[L_{i}^{*}\right]}{\left(1-\mu_{i}\right)^{2}}+\frac{\sigma^{2}\esp\left[L_{i}^{*}\right]}{\left(1-\mu_{i}\right)^{3}}
\end{eqnarray*}

donde recordemos que

\begin{eqnarray*}
Var\left[L_{i}^{*}\right]&=&f_{i}\left(i,i\right)+f_{i}\left(i\right)-f_{i}\left(i\right)^{2}.
\end{eqnarray*}

La duraci\'on del tiempo de intervisita es $\tau_{i}\left(m+1\right)-\overline{\tau}\left(m\right)$. Dado que el n\'umero de usuarios presentes en $Q_{i}$ al tiempo $t=\tau_{i}\left(m+1\right)$ es igual al n\'umero de arribos durante el intervalo de tiempo $\left[\overline{\tau}\left(m\right),\tau_{i}\left(m+1\right)\right]$ se tiene que


\begin{eqnarray*}
\esp\left[z_{i}^{L_{i}\left(\tau_{i}\left(m+1\right)\right)}\right]=\esp\left[\left\{P_{i}\left(z_{i}\right)\right\}^{\tau_{i}\left(m+1\right)-\overline{\tau}\left(m\right)}\right]
\end{eqnarray*}

entonces, si \begin{eqnarray*}I_{i}\left(z\right)&=&\esp\left[z^{\tau_{i}\left(m+1\right)-\overline{\tau}\left(m\right)}\right]\end{eqnarray*} se tienen que

\begin{eqnarray*}
F_{i}\left(z\right)=I_{i}\left[P_{i}\left(z\right)\right]
\end{eqnarray*}
para $i=1,2$, por tanto



\begin{eqnarray*}
\esp\left[L_{i}^{*}\right]&=&\mu_{i}\esp\left[I_{i}\right]\\
Var\left[L_{i}^{*}\right]&=&\mu_{i}^{2}Var\left[I_{i}\right]+\sigma^{2}\esp\left[I_{i}\right]
\end{eqnarray*}
para $i=1,2$, por tanto


\begin{eqnarray*}
\esp\left[I_{i}\right]&=&\frac{f_{i}\left(i\right)}{\mu_{i}},
\end{eqnarray*}
adem\'as

\begin{eqnarray*}
Var\left[I_{i}\right]&=&\frac{Var\left[L_{i}^{*}\right]}{\mu_{i}^{2}}-\frac{\sigma_{i}^{2}}{\mu_{i}^{2}}f_{i}\left(i\right).
\end{eqnarray*}


Si  $C_{i}\left(z\right)=\esp\left[z^{\overline{\tau}\left(m+1\right)-\overline{\tau}_{i}\left(m\right)}\right]$el tiempo de duraci\'on del ciclo, entonces, por lo hasta ahora establecido, se tiene que

\begin{eqnarray*}
C_{i}\left(z\right)=I_{i}\left[\theta_{i}\left(z\right)\right],
\end{eqnarray*}
entonces

\begin{eqnarray*}
\esp\left[C_{i}\right]&=&\esp\left[I_{i}\right]\esp\left[\theta_{i}\left(z\right)\right]=\frac{\esp\left[L_{i}^{*}\right]}{\mu_{i}}\frac{1}{1-\mu_{i}}=\frac{f_{i}\left(i\right)}{\mu_{i}\left(1-\mu_{i}\right)}\\
Var\left[C_{i}\right]&=&\frac{Var\left[L_{i}^{*}\right]}{\mu_{i}^{2}\left(1-\mu_{i}\right)^{2}}.
\end{eqnarray*}

Por tanto se tienen las siguientes igualdades


\begin{eqnarray*}
\esp\left[S_{i}\right]&=&\mu_{i}\esp\left[C_{i}\right],\\
\esp\left[I_{i}\right]&=&\left(1-\mu_{i}\right)\esp\left[C_{i}\right]\\
\end{eqnarray*}

Def\'inanse los puntos de regenaraci\'on  en el proceso $\left[L_{1}\left(t\right),L_{2}\left(t\right),\ldots,L_{N}\left(t\right)\right]$. Los puntos cuando la cola $i$ es visitada y todos los $L_{j}\left(\tau_{i}\left(m\right)\right)=0$ para $i=1,2$  son puntos de regeneraci\'on. Se llama ciclo regenerativo al intervalo entre dos puntos regenerativos sucesivos.

Sea $M_{i}$  el n\'umero de ciclos de visita en un ciclo regenerativo, y sea $C_{i}^{(m)}$, para $m=1,2,\ldots,M_{i}$ la duraci\'on del $m$-\'esimo ciclo de visita en un ciclo regenerativo. Se define el ciclo del tiempo de visita promedio $\esp\left[C_{i}\right]$ como

\begin{eqnarray*}
\esp\left[C_{i}\right]&=&\frac{\esp\left[\sum_{m=1}^{M_{i}}C_{i}^{(m)}\right]}{\esp\left[M_{i}\right]}
\end{eqnarray*}


En Stid72 y Heym82 se muestra que una condici\'on suficiente para que el proceso regenerativo 
estacionario sea un procesoo estacionario es que el valor esperado del tiempo del ciclo regenerativo sea finito:

\begin{eqnarray*}
\esp\left[\sum_{m=1}^{M_{i}}C_{i}^{(m)}\right]<\infty.
\end{eqnarray*}

como cada $C_{i}^{(m)}$ contiene intervalos de r\'eplica positivos, se tiene que $\esp\left[M_{i}\right]<\infty$, adem\'as, como $M_{i}>0$, se tiene que la condici\'on anterior es equivalente a tener que 

\begin{eqnarray*}
\esp\left[C_{i}\right]<\infty,
\end{eqnarray*}
por lo tanto una condici\'on suficiente para la existencia del proceso regenerativo est\'a dada por

\begin{eqnarray*}
\sum_{k=1}^{N}\mu_{k}<1.
\end{eqnarray*}

Sea la funci\'on generadora de momentos para $L_{i}$, el n\'umero de usuarios en la cola $Q_{i}\left(z\right)$ en cualquier momento, est\'a dada por el tiempo promedio de $z^{L_{i}\left(t\right)}$ sobre el ciclo regenerativo definido anteriormente:

\begin{eqnarray*}
Q_{i}\left(z\right)&=&\esp\left[z^{L_{i}\left(t\right)}\right]=\frac{\esp\left[\sum_{m=1}^{M_{i}}\sum_{t=\tau_{i}\left(m\right)}^{\tau_{i}\left(m+1\right)-1}z^{L_{i}\left(t\right)}\right]}{\esp\left[\sum_{m=1}^{M_{i}}\tau_{i}\left(m+1\right)-\tau_{i}\left(m\right)\right]}
\end{eqnarray*}

$M_{i}$ es un tiempo de paro en el proceso regenerativo con $\esp\left[M_{i}\right]<\infty$, se sigue del lema de Wald que:


\begin{eqnarray*}
\esp\left[\sum_{m=1}^{M_{i}}\sum_{t=\tau_{i}\left(m\right)}^{\tau_{i}\left(m+1\right)-1}z^{L_{i}\left(t\right)}\right]&=&\esp\left[M_{i}\right]\esp\left[\sum_{t=\tau_{i}\left(m\right)}^{\tau_{i}\left(m+1\right)-1}z^{L_{i}\left(t\right)}\right]\\
\esp\left[\sum_{m=1}^{M_{i}}\tau_{i}\left(m+1\right)-\tau_{i}\left(m\right)\right]&=&\esp\left[M_{i}\right]\esp\left[\tau_{i}\left(m+1\right)-\tau_{i}\left(m\right)\right]
\end{eqnarray*}

por tanto se tiene que


\begin{eqnarray*}
Q_{i}\left(z\right)&=&\frac{\esp\left[\sum_{t=\tau_{i}\left(m\right)}^{\tau_{i}\left(m+1\right)-1}z^{L_{i}\left(t\right)}\right]}{\esp\left[\tau_{i}\left(m+1\right)-\tau_{i}\left(m\right)\right]}
\end{eqnarray*}

observar que el denominador es simplemente la duraci\'on promedio del tiempo del ciclo.


Se puede demostrar (ver Hideaki Takagi 1986) que

\begin{eqnarray*}
\esp\left[\sum_{t=\tau_{i}\left(m\right)}^{\tau_{i}\left(m+1\right)-1}z^{L_{i}\left(t\right)}\right]=z\frac{F_{i}\left(z\right)-1}{z-P_{i}\left(z\right)}
\end{eqnarray*}

Durante el tiempo de intervisita para la cola $i$, $L_{i}\left(t\right)$ solamente se incrementa de manera que el incremento por intervalo de tiempo est\'a dado por la funci\'on generadora de probabilidades de $P_{i}\left(z\right)$, por tanto la suma sobre el tiempo de intervisita puede evaluarse como:

\begin{eqnarray*}
\esp\left[\sum_{t=\tau_{i}\left(m\right)}^{\tau_{i}\left(m+1\right)-1}z^{L_{i}\left(t\right)}\right]&=&\esp\left[\sum_{t=\tau_{i}\left(m\right)}^{\tau_{i}\left(m+1\right)-1}\left\{P_{i}\left(z\right)\right\}^{t-\overline{\tau}_{i}\left(m\right)}\right]=\frac{1-\esp\left[\left\{P_{i}\left(z\right)\right\}^{\tau_{i}\left(m+1\right)-\overline{\tau}_{i}\left(m\right)}\right]}{1-P_{i}\left(z\right)}\\
&=&\frac{1-I_{i}\left[P_{i}\left(z\right)\right]}{1-P_{i}\left(z\right)}
\end{eqnarray*}
por tanto

\begin{eqnarray*}
\esp\left[\sum_{t=\tau_{i}\left(m\right)}^{\tau_{i}\left(m+1\right)-1}z^{L_{i}\left(t\right)}\right]&=&\frac{1-F_{i}\left(z\right)}{1-P_{i}\left(z\right)}
\end{eqnarray*}

Haciendo uso de lo hasta ahora desarrollado se tiene que

\begin{eqnarray*}
Q_{i}\left(z\right)&=&\frac{1}{\esp\left[C_{i}\right]}\cdot\frac{1-F_{i}\left(z\right)}{P_{i}\left(z\right)-z}\cdot\frac{\left(1-z\right)P_{i}\left(z\right)}{1-P_{i}\left(z\right)}\\
&=&\frac{\mu_{i}\left(1-\mu_{i}\right)}{f_{i}\left(i\right)}\cdot\frac{1-F_{i}\left(z\right)}{P_{i}\left(z\right)-z}\cdot\frac{\left(1-z\right)P_{i}\left(z\right)}{1-P_{i}\left(z\right)}
\end{eqnarray*}

derivando con respecto a $z$



\begin{eqnarray*}
\frac{d Q_{i}\left(z\right)}{d z}&=&\frac{\left(1-F_{i}\left(z\right)\right)P_{i}\left(z\right)}{\esp\left[C_{i}\right]\left(1-P_{i}\left(z\right)\right)\left(P_{i}\left(z\right)-z\right)}\\
&-&\frac{\left(1-z\right)P_{i}\left(z\right)F_{i}^{'}\left(z\right)}{\esp\left[C_{i}\right]\left(1-P_{i}\left(z\right)\right)\left(P_{i}\left(z\right)-z\right)}\\
&-&\frac{\left(1-z\right)\left(1-F_{i}\left(z\right)\right)P_{i}\left(z\right)\left(P_{i}^{'}\left(z\right)-1\right)}{\esp\left[C_{i}\right]\left(1-P_{i}\left(z\right)\right)\left(P_{i}\left(z\right)-z\right)^{2}}\\
&+&\frac{\left(1-z\right)\left(1-F_{i}\left(z\right)\right)P_{i}^{'}\left(z\right)}{\esp\left[C_{i}\right]\left(1-P_{i}\left(z\right)\right)\left(P_{i}\left(z\right)-z\right)}\\
&+&\frac{\left(1-z\right)\left(1-F_{i}\left(z\right)\right)P_{i}\left(z\right)P_{i}^{'}\left(z\right)}{\esp\left[C_{i}\right]\left(1-P_{i}\left(z\right)\right)^{2}\left(P_{i}\left(z\right)-z\right)}
\end{eqnarray*}

Calculando el l\'imite cuando $z\rightarrow1^{+}$:
\begin{eqnarray}
Q_{i}^{(1)}\left(z\right)=\lim_{z\rightarrow1^{+}}\frac{d Q_{i}\left(z\right)}{dz}&=&\lim_{z\rightarrow1}\frac{\left(1-F_{i}\left(z\right)\right)P_{i}\left(z\right)}{\esp\left[C_{i}\right]\left(1-P_{i}\left(z\right)\right)\left(P_{i}\left(z\right)-z\right)}\\
&-&\lim_{z\rightarrow1^{+}}\frac{\left(1-z\right)P_{i}\left(z\right)F_{i}^{'}\left(z\right)}{\esp\left[C_{i}\right]\left(1-P_{i}\left(z\right)\right)\left(P_{i}\left(z\right)-z\right)}\\
&-&\lim_{z\rightarrow1^{+}}\frac{\left(1-z\right)\left(1-F_{i}\left(z\right)\right)P_{i}\left(z\right)\left(P_{i}^{'}\left(z\right)-1\right)}{\esp\left[C_{i}\right]\left(1-P_{i}\left(z\right)\right)\left(P_{i}\left(z\right)-z\right)^{2}}\\
&+&\lim_{z\rightarrow1^{+}}\frac{\left(1-z\right)\left(1-F_{i}\left(z\right)\right)P_{i}^{'}\left(z\right)}{\esp\left[C_{i}\right]\left(1-P_{i}\left(z\right)\right)\left(P_{i}\left(z\right)-z\right)}\\
&+&\lim_{z\rightarrow1^{+}}\frac{\left(1-z\right)\left(1-F_{i}\left(z\right)\right)P_{i}\left(z\right)P_{i}^{'}\left(z\right)}{\esp\left[C_{i}\right]\left(1-P_{i}\left(z\right)\right)^{2}\left(P_{i}\left(z\right)-z\right)}
\end{eqnarray}

Entonces:
%______________________________________________________

\begin{eqnarray*}
\lim_{z\rightarrow1^{+}}\frac{\left(1-F_{i}\left(z\right)\right)P_{i}\left(z\right)}{\left(1-P_{i}\left(z\right)\right)\left(P_{i}\left(z\right)-z\right)}&=&\lim_{z\rightarrow1^{+}}\frac{\frac{d}{dz}\left[\left(1-F_{i}\left(z\right)\right)P_{i}\left(z\right)\right]}{\frac{d}{dz}\left[\left(1-P_{i}\left(z\right)\right)\left(-z+P_{i}\left(z\right)\right)\right]}\\
&=&\lim_{z\rightarrow1^{+}}\frac{-P_{i}\left(z\right)F_{i}^{'}\left(z\right)+\left(1-F_{i}\left(z\right)\right)P_{i}^{'}\left(z\right)}{\left(1-P_{i}\left(z\right)\right)\left(-1+P_{i}^{'}\left(z\right)\right)-\left(-z+P_{i}\left(z\right)\right)P_{i}^{'}\left(z\right)}
\end{eqnarray*}


%______________________________________________________


\begin{eqnarray*}
\lim_{z\rightarrow1^{+}}\frac{\left(1-z\right)P_{i}\left(z\right)F_{i}^{'}\left(z\right)}{\left(1-P_{i}\left(z\right)\right)\left(P_{i}\left(z\right)-z\right)}&=&\lim_{z\rightarrow1^{+}}\frac{\frac{d}{dz}\left[\left(1-z\right)P_{i}\left(z\right)F_{i}^{'}\left(z\right)\right]}{\frac{d}{dz}\left[\left(1-P_{i}\left(z\right)\right)\left(P_{i}\left(z\right)-z\right)\right]}\\
&=&\lim_{z\rightarrow1^{+}}\frac{-P_{i}\left(z\right) F_{i}^{'}\left(z\right)+(1-z) F_{i}^{'}\left(z\right) P_{i}^{'}\left(z\right)+(1-z) P_{i}\left(z\right)F_{i}^{''}\left(z\right)}{\left(1-P_{i}\left(z\right)\right)\left(-1+P_{i}^{'}\left(z\right)\right)-\left(-z+P_{i}\left(z\right)\right)P_{i}^{'}\left(z\right)}
\end{eqnarray*}


%______________________________________________________

\begin{eqnarray*}
&&\lim_{z\rightarrow1^{+}}\frac{\left(1-z\right)\left(1-F_{i}\left(z\right)\right)P_{i}\left(z\right)\left(P_{i}^{'}\left(z\right)-1\right)}{\left(1-P_{i}\left(z\right)\right)\left(P_{i}\left(z\right)-z\right)^{2}}=\lim_{z\rightarrow1^{+}}\frac{\frac{d}{dz}\left[\left(1-z\right)\left(1-F_{i}\left(z\right)\right)P_{i}\left(z\right)\left(P_{i}^{'}\left(z\right)-1\right)\right]}{\frac{d}{dz}\left[\left(1-P_{i}\left(z\right)\right)\left(P_{i}\left(z\right)-z\right)^{2}\right]}\\
&=&\lim_{z\rightarrow1^{+}}\frac{-\left(1-F_{i}\left(z\right)\right) P_{i}\left(z\right)\left(-1+P_{i}^{'}\left(z\right)\right)-(1-z) P_{i}\left(z\right)F_{i}^{'}\left(z\right)\left(-1+P_{i}^{'}\left(z\right)\right)}{2\left(1-P_{i}\left(z\right)\right)\left(-z+P_{i}\left(z\right)\right) \left(-1+P_{i}^{'}\left(z\right)\right)-\left(-z+P_{i}\left(z\right)\right)^2 P_{i}^{'}\left(z\right)}\\
&+&\lim_{z\rightarrow1^{+}}\frac{+(1-z) \left(1-F_{i}\left(z\right)\right) \left(-1+P_{i}^{'}\left(z\right)\right) P_{i}^{'}\left(z\right)}{{2\left(1-P_{i}\left(z\right)\right)\left(-z+P_{i}\left(z\right)\right) \left(-1+P_{i}^{'}\left(z\right)\right)-\left(-z+P_{i}\left(z\right)\right)^2 P_{i}^{'}\left(z\right)}}\\
&+&\lim_{z\rightarrow1^{+}}\frac{+(1-z) \left(1-F_{i}\left(z\right)\right) P_{i}\left(z\right)P_{i}^{''}\left(z\right)}{{2\left(1-P_{i}\left(z\right)\right)\left(-z+P_{i}\left(z\right)\right) \left(-1+P_{i}^{'}\left(z\right)\right)-\left(-z+P_{i}\left(z\right)\right)^2 P_{i}^{'}\left(z\right)}}
\end{eqnarray*}











%______________________________________________________
\begin{eqnarray*}
&&\lim_{z\rightarrow1^{+}}\frac{\left(1-z\right)\left(1-F_{i}\left(z\right)\right)P_{i}^{'}\left(z\right)}{\left(1-P_{i}\left(z\right)\right)\left(P_{i}\left(z\right)-z\right)}=\lim_{z\rightarrow1^{+}}\frac{\frac{d}{dz}\left[\left(1-z\right)\left(1-F_{i}\left(z\right)\right)P_{i}^{'}\left(z\right)\right]}{\frac{d}{dz}\left[\left(1-P_{i}\left(z\right)\right)\left(P_{i}\left(z\right)-z\right)\right]}\\
&=&\lim_{z\rightarrow1^{+}}\frac{-\left(1-F_{i}\left(z\right)\right) P_{i}^{'}\left(z\right)-(1-z) F_{i}^{'}\left(z\right) P_{i}^{'}\left(z\right)+(1-z) \left(1-F_{i}\left(z\right)\right) P_{i}^{''}\left(z\right)}{\left(1-P_{i}\left(z\right)\right) \left(-1+P_{i}^{'}\left(z\right)\right)-\left(-z+P_{i}\left(z\right)\right) P_{i}^{'}\left(z\right)}\frac{}{}
\end{eqnarray*}

%______________________________________________________
\begin{eqnarray*}
&&\lim_{z\rightarrow1^{+}}\frac{\left(1-z\right)\left(1-F_{i}\left(z\right)\right)P_{i}\left(z\right)P_{i}^{'}\left(z\right)}{\left(1-P_{i}\left(z\right)\right)^{2}\left(P_{i}\left(z\right)-z\right)}=\lim_{z\rightarrow1^{+}}\frac{\frac{d}{dz}\left[\left(1-z\right)\left(1-F_{i}\left(z\right)\right)P_{i}\left(z\right)P_{i}^{'}\left(z\right)\right]}{\frac{d}{dz}\left[\left(1-P_{i}\left(z\right)\right)^{2}\left(P_{i}\left(z\right)-z\right)\right]}\\
&=&\lim_{z\rightarrow1^{+}}\frac{-\left(1-F_{i}\left(z\right)\right) P_{i}\left(z\right) P_{i}^{'}\left(z\right)-(1-z) P_{i}\left(z\right) F_{i}^{'}\left(z\right)P_i'[z]}{\left(1-P_{i}\left(z\right)\right)^2 \left(-1+P_{i}^{'}\left(z\right)\right)-2 \left(1-P_{i}\left(z\right)\right) \left(-z+P_{i}\left(z\right)\right) P_{i}^{'}\left(z\right)}\\
&+&\lim_{z\rightarrow1^{+}}\frac{(1-z) \left(1-F_{i}\left(z\right)\right) P_{i}^{'}\left(z\right)^2+(1-z) \left(1-F_{i}\left(z\right)\right) P_{i}\left(z\right) P_{i}^{''}\left(z\right)}{\left(1-P_{i}\left(z\right)\right)^2 \left(-1+P_{i}^{'}\left(z\right)\right)-2 \left(1-P_{i}\left(z\right)\right) \left(-z+P_{i}\left(z\right)\right) P_{i}^{'}\left(z\right)}\\
\end{eqnarray*}

%___________________________________________________________________________________________
\subsection{Longitudes de la Cola en cualquier tiempo}
%___________________________________________________________________________________________

Sea
$V_{i}\left(z\right)=\frac{1}{\esp\left[C_{i}\right]}\frac{I_{i}\left(z\right)-1}{z-P_{i}\left(z\right)}$

%{\esp\lef[I_{i}\right]}\frac{1-\mu_{i}}{z-P_{i}\left(z\right)}

\begin{eqnarray*}
\frac{\partial V_{i}\left(z\right)}{\partial z}&=&\frac{1}{\esp\left[C_{i}\right]}\left[\frac{I_{i}{'}\left(z\right)\left(z-P_{i}\left(z\right)\right)}{z-P_{i}\left(z\right)}-\frac{\left(I_{i}\left(z\right)-1\right)\left(1-P_{i}{'}\left(z\right)\right)}{\left(z-P_{i}\left(z\right)\right)^{2}}\right]
\end{eqnarray*}


La FGP para el tiempo de espera para cualquier usuario en la cola est\'a dada por:
\[U_{i}\left(z\right)=\frac{1}{\esp\left[C_{i}\right]}\cdot\frac{1-P_{i}\left(z\right)}{z-P_{i}\left(z\right)}\cdot\frac{I_{i}\left(z\right)-1}{1-z}\]

entonces


\begin{eqnarray*}
\frac{d}{dz}V_{i}\left(z\right)&=&\frac{1}{\esp\left[C_{i}\right]}\left\{\frac{d}{dz}\left(\frac{1-P_{i}\left(z\right)}{z-P_{i}\left(z\right)}\right)\frac{I_{i}\left(z\right)-1}{1-z}+\frac{1-P_{i}\left(z\right)}{z-P_{i}\left(z\right)}\frac{d}{dz}\left(\frac{I_{i}\left(z\right)-1}{1-z}\right)\right\}\\
&=&\frac{1}{\esp\left[C_{i}\right]}\left\{\frac{-P_{i}\left(z\right)\left(z-P_{i}\left(z\right)\right)-\left(1-P_{i}\left(z\right)\right)\left(1-P_{i}^{'}\left(z\right)\right)}{\left(z-P_{i}\left(z\right)\right)^{2}}\cdot\frac{I_{i}\left(z\right)-1}{1-z}\right\}\\
&+&\frac{1}{\esp\left[C_{i}\right]}\left\{\frac{1-P_{i}\left(z\right)}{z-P_{i}\left(z\right)}\cdot\frac{I_{i}^{'}\left(z\right)\left(1-z\right)+\left(I_{i}\left(z\right)-1\right)}{\left(1-z\right)^{2}}\right\}
\end{eqnarray*}
%\frac{I_{i}\left(z\right)-1}{1-z}
%+\frac{1-P_{i}\left(z\right)}{z-P_{i}\frac{d}{dz}\left(\frac{I_{i}\left(z\right)-1}{1-z}\right)


\begin{eqnarray*}
\frac{\partial U_{i}\left(z\right)}{\partial z}&=&\frac{(-1+I_{i}[z]) (1-P_{i}[z])}{(1-z)^2 \esp[I_{i}] (z-P_{i}[z])}+\frac{(1-P_{i}[z]) I_{i}^{'}[z]}{(1-z) \esp[I_{i}] (z-P_{i}[z])}-\frac{(-1+I_{i}[z]) (1-P_{i}[z])\left(1-P{'}[z]\right)}{(1-z) \esp[I_{i}] (z-P_{i}[z])^2}\\
&-&\frac{(-1+I_{i}[z]) P_{i}{'}[z]}{(1-z) \esp[I_{i}](z-P_{i}[z])}
\end{eqnarray*}


\subsection{Material por agregar}


\begin{Teo}
Dada una Red de Sistemas de Visitas C\'iclicas (RSVC), conformada por dos Sistemas de Visitas C\'iclicas (SVC), donde cada uno de ellos consta de dos colas tipo $M/M/1$. Los dos sistemas est\'an comunicados entre s\'i por medio de la transferencia de usuarios entre las colas $Q_{1}\leftrightarrow Q_{3}$ y $Q_{2}\leftrightarrow Q_{4}$. Se definen los eventos para los procesos de arribos al tiempo $t$, $A_{j}\left(t\right)=\left\{0 \textrm{ arribos en }Q_{j}\textrm{ al tiempo }t\right\}$ para alg\'un tiempo $t\geq0$ y $Q_{j}$ la cola $j$-\'esima en la RSVC, para $j=1,2,3,4$.  Existe un intervalo $I\neq\emptyset$ tal que para $T^{*}\in I$, tal que $\prob\left\{A_{1}\left(T^{*}\right),A_{2}\left(Tt^{*}\right),
A_{3}\left(T^{*}\right),A_{4}\left(T^{*}\right)|T^{*}\in I\right\}>0$.
\end{Teo}



\begin{proof}
Sin p\'erdida de generalidad podemos considerar como base del an\'alisis a la cola $Q_{1}$ del primer sistema que conforma la RSVC.\medskip 

Sea $n\geq1$, ciclo en el primer sistema en el que se sabe que $L_{j}\left(\overline{\tau}_{1}\left(n\right)\right)=0$, pues la pol\'itica de servicio con que atienden los servidores es la exhaustiva. Como es sabido, para trasladarse a la siguiente cola, el servidor incurre en un tiempo de traslado $r_{1}\left(n\right)>0$, entonces tenemos que $\tau_{2}\left(n\right)=\overline{\tau}_{1}\left(n\right)+r_{1}\left(n\right)$.\medskip 


Definamos el intervalo $I_{1}\equiv\left[\overline{\tau}_{1}\left(n\right),\tau_{2}\left(n\right)\right]$ de longitud $\xi_{1}=r_{1}\left(n\right)$.

Dado que los tiempos entre arribo son exponenciales con tasa $\tilde{\mu}_{1}=\mu_{1}+\hat{\mu}_{1}$ ($\mu_{1}$ son los arribos a $Q_{1}$ por primera vez al sistema, mientras que $\hat{\mu}_{1}$ son los arribos de traslado procedentes de $Q_{3}$) se tiene que la probabilidad del evento $A_{1}\left(t\right)$ est\'a dada por 

\begin{equation}
\prob\left\{A_{1}\left(t\right)|t\in I_{1}\left(n\right)\right\}=e^{-\tilde{\mu}_{1}\xi_{1}\left(n\right)}.
\end{equation} 


Por otra parte, para la cola $Q_{2}$ el tiempo $\overline{\tau}_{2}\left(n-1\right)$ es tal que $L_{2}\left(\overline{\tau}_{2}\left(n-1\right)\right)=0$, es decir, es el tiempo en que la cola queda totalmente vac\'ia en el ciclo anterior a $n$. \medskip 


Entonces tenemos un sgundo intervalo $I_{2}\equiv\left[\overline{\tau}_{2}\left(n-1\right),\tau_{2}\left(n\right)\right]$. Por lo tanto la probabilidad del evento $A_{2}\left(t\right)$ tiene probabilidad dada por

\begin{eqnarray}
\prob\left\{A_{2}\left(t\right)|t\in I_{2}\left(n\right)\right\}=e^{-\tilde{\mu}_{2}\xi_{2}\left(n\right)},\\
\xi_{2}\left(n\right)=\tau_{2}\left(n\right)-\overline{\tau}_{2}\left(n-1\right)
\end{eqnarray}
%\end{equation} 

%donde $$.

Ahora, dado que $I_{1}\left(n\right)\subset I_{2}\left(n\right)$, se tiene que

\begin{eqnarray*}
\xi_{1}\left(n\right)\leq\xi_{2}\left(n\right)&\Leftrightarrow& -\xi_{1}\left(n\right)\geq-\xi_{2}\left(n\right)
\\
-\tilde{\mu}_{2}\xi_{1}\left(n\right)\geq-\tilde{\mu}_{2}\xi_{2}\left(n\right)&\Leftrightarrow&
e^{-\tilde{\mu}_{2}\xi_{1}\left(n\right)}\geq e^{-\tilde{\mu}_{2}\xi_{2}\left(n\right)}\\
\prob\left\{A_{2}\left(t\right)|t\in I_{1}\left(n\right)\right\}&\geq&
\prob\left\{A_{2}\left(t\right)|t\in I_{2}\left(n\right)\right\}.
\end{eqnarray*}


Entonces se tiene que
\small{
\begin{eqnarray*}
\prob\left\{A_{1}\left(t\right),A_{2}\left(t\right)|t\in I_{1}\left(n\right)\right\}&=&
\prob\left\{A_{1}\left(t\right)|t\in I_{1}\left(n\right)\right\}
\prob\left\{A_{2}\left(t\right)|t\in I_{1}\left(n\right)\right\}\\
&\geq&
\prob\left\{A_{1}\left(t\right)|t\in I_{1}\left(n\right)\right\}
\prob\left\{A_{2}\left(t\right)|t\in I_{2}\left(n\right)\right\}\\
&=&e^{-\tilde{\mu}_{1}\xi_{1}\left(n\right)}e^{-\tilde{\mu}_{2}\xi_{2}\left(n\right)}
=e^{-\left[\tilde{\mu}_{1}\xi_{1}\left(n\right)+\tilde{\mu}_{2}\xi_{2}\left(n\right)\right]}.
\end{eqnarray*}}


Es decir, 

\begin{equation}
\prob\left\{A_{1}\left(t\right),A_{2}\left(t\right)|t\in I_{1}\left(n\right)\right\}
=e^{-\left[\tilde{\mu}_{1}\xi_{1}\left(n\right)+\tilde{\mu}_{2}\xi_{2}
\left(n\right)\right]}>0.
\end{equation}
En lo que respecta a la relaci\'on entre los dos SVC que conforman la RSVC para alg\'un $m\geq1$ se tiene que $\tau_{3}\left(m\right)<\tau_{2}\left(n\right)<\tau_{4}\left(m\right)$ por lo tanto se cumple cualquiera de los siguientes cuatro casos
\begin{itemize}
\item[a)] $\tau_{3}\left(m\right)<\tau_{2}\left(n\right)<\overline{\tau}_{3}\left(m\right)$

\item[b)] $\overline{\tau}_{3}\left(m\right)<\tau_{2}\left(n\right)
<\tau_{4}\left(m\right)$

\item[c)] $\tau_{4}\left(m\right)<\tau_{2}\left(n\right)<
\overline{\tau}_{4}\left(m\right)$

\item[d)] $\overline{\tau}_{4}\left(m\right)<\tau_{2}\left(n\right)
<\tau_{3}\left(m+1\right)$
\end{itemize}


Sea el intervalo $I_{3}\left(m\right)\equiv\left[\tau_{3}\left(m\right),\overline{\tau}_{3}\left(m\right)\right]$ tal que $\tau_{2}\left(n\right)\in I_{3}\left(m\right)$, con longitud de intervalo $\xi_{3}\equiv\overline{\tau}_{3}\left(m\right)-\tau_{3}\left(m\right)$, entonces se tiene que para $Q_{3}$
\begin{equation}
\prob\left\{A_{3}\left(t\right)|t\in I_{3}\left(m\right)\right\}=e^{-\tilde{\mu}_{3}\xi_{3}\left(m\right)}.
\end{equation} 

mientras que para $Q_{4}$ consideremos el intervalo $I_{4}\left(m\right)\equiv\left[\tau_{4}\left(m-1\right),\overline{\tau}_{3}\left(m\right)\right]$, entonces por construcci\'on  $I_{3}\left(m\right)\subset I_{4}\left(m\right)$, por lo tanto


\begin{eqnarray*}
\xi_{3}\left(m\right)\leq\xi_{4}\left(m\right)&\Leftrightarrow& -\xi_{3}\left(m\right)\geq-\xi_{4}\left(m\right)
\\
-\tilde{\mu}_{4}\xi_{3}\left(m\right)\geq-\tilde{\mu}_{4}\xi_{4}\left(m\right)&\Leftrightarrow&
e^{-\tilde{\mu}_{4}\xi_{3}\left(m\right)}\geq e^{-\tilde{\mu}_{4}\xi_{4}\left(n\right)}\\
\prob\left\{A_{4}\left(t\right)|t\in I_{3}\left(m\right)\right\}&\geq&
\prob\left\{A_{4}\left(t\right)|t\in I_{4}\left(m\right)\right\}.
\end{eqnarray*}



Entonces se tiene que
\small{
\begin{eqnarray*}
\prob\left\{A_{3}\left(t\right),A_{4}\left(t\right)|t\in I_{3}\left(m\right)\right\}&=&
\prob\left\{A_{3}\left(t\right)|t\in I_{3}\left(m\right)\right\}
\prob\left\{A_{4}\left(t\right)|t\in I_{3}\left(m\right)\right\}\\
&\geq&
\prob\left\{A_{3}\left(t\right)|t\in I_{3}\left(m\right)\right\}
\prob\left\{A_{4}\left(t\right)|t\in I_{4}\left(m\right)\right\}\\
&=&e^{-\tilde{\mu}_{3}\xi_{3}\left(m\right)}e^{-\tilde{\mu}_{4}\xi_{4}
\left(m\right)}
=e^{-\left(\tilde{\mu}_{3}\xi_{3}\left(m\right)+\tilde{\mu}_{4}\xi_{4}\left(m\right)\right)}.
\end{eqnarray*}}

Es decir, 

\begin{equation}
\prob\left\{A_{3}\left(t\right),A_{4}\left(t\right)|t\in I_{3}\left(m\right)\right\}\geq
e^{-\left(\tilde{\mu}_{3}\xi_{3}\left(m\right)+\tilde{\mu}_{4}\xi_{4}\left(m\right)\right)}>0.
\end{equation}


Sea el intervalo $I_{3}\left(m\right)\equiv\left[\overline{\tau}_{3}\left(m\right),\tau_{4}\left(m\right)\right]$ con longitud $\xi_{3}\equiv\tau_{4}\left(m\right)-\overline{\tau}_{3}\left(m\right)$, entonces se tiene que para $Q_{3}$
\begin{equation}
\prob\left\{A_{3}\left(t\right)|t\in I_{3}\left(m\right)\right\}=e^{-\tilde{\mu}_{3}\xi_{3}\left(m\right)}.
\end{equation} 

mientras que para $Q_{4}$ consideremos el intervalo $I_{4}\left(m\right)\equiv\left[\overline{\tau}_{4}\left(m-1\right),\tau_{4}\left(m\right)\right]$, entonces por construcci\'on  $I_{3}\left(m\right)\subset I_{4}\left(m\right)$, y al igual que en el caso anterior se tiene que 

\begin{eqnarray*}
\xi_{3}\left(m\right)\leq\xi_{4}\left(m\right)&\Leftrightarrow& -\xi_{3}\left(m\right)\geq-\xi_{4}\left(m\right)
\\
-\tilde{\mu}_{4}\xi_{3}\left(m\right)\geq-\tilde{\mu}_{4}\xi_{4}\left(m\right)&\Leftrightarrow&
e^{-\tilde{\mu}_{4}\xi_{3}\left(m\right)}\geq e^{-\tilde{\mu}_{4}\xi_{4}\left(n\right)}\\
\prob\left\{A_{4}\left(t\right)|t\in I_{3}\left(m\right)\right\}&\geq&
\prob\left\{A_{4}\left(t\right)|t\in I_{4}\left(m\right)\right\}.
\end{eqnarray*}


Entonces se tiene que
\small{
\begin{eqnarray*}
\prob\left\{A_{3}\left(t\right),A_{4}\left(t\right)|t\in I_{3}\left(m\right)\right\}&=&
\prob\left\{A_{3}\left(t\right)|t\in I_{3}\left(m\right)\right\}
\prob\left\{A_{4}\left(t\right)|t\in I_{3}\left(m\right)\right\}\\
&\geq&
\prob\left\{A_{3}\left(t\right)|t\in I_{3}\left(m\right)\right\}
\prob\left\{A_{4}\left(t\right)|t\in I_{4}\left(m\right)\right\}\\
&=&e^{-\tilde{\mu}_{3}\xi_{3}\left(m\right)}e^{-\tilde{\mu}_{4}\xi_{4}\left(m\right)}
=e^{-\left(\tilde{\mu}_{3}\xi_{3}\left(m\right)+\tilde{\mu}_{4}\xi_{4}\left(m\right)\right)}.
\end{eqnarray*}}

Es decir, 

\begin{equation}
\prob\left\{A_{3}\left(t\right),A_{4}\left(t\right)|t\in I_{4}\left(m\right)\right\}\geq
e^{-\left(\tilde{\mu}_{3}+\tilde{\mu}_{4}\right)\xi_{3}\left(m\right)}>0.
\end{equation}


Para el intervalo $I_{3}\left(m\right)=\left[\tau_{4}\left(m\right),\overline{\tau}_{4}\left(m\right)\right]$, se tiene que este caso es an\'alogo al caso (a).


Para el intevalo $I_{3}\left(m\right)\equiv\left[\overline{\tau}_{4}\left(m\right),\tau_{4}\left(m+1\right)\right]$, se tiene que es an\'alogo al caso (b).


Por construcci\'on se tiene que $I\left(n,m\right)\equiv I_{1}\left(n\right)\cap I_{3}\left(m\right)\neq\emptyset$,entonces en particular se tienen las contenciones $I\left(n,m\right)\subseteq I_{1}\left(n\right)$ y $I\left(n,m\right)\subseteq I_{3}\left(m\right)$, por lo tanto si definimos $\xi_{n,m}\equiv\ell\left(I\left(n,m\right)\right)$ tenemos que

\begin{eqnarray*}
\xi_{n,m}\leq\xi_{1}\left(n\right)\textrm{ y }\xi_{n,m}\leq\xi_{3}\left(m\right)\textrm{ entonces }\\
-\xi_{n,m}\geq-\xi_{1}\left(n\right)\textrm{ y }-\xi_{n,m}\leq-\xi_{3}\left(m\right)\\
\end{eqnarray*}
por lo tanto tenemos las desigualdades 


\begin{eqnarray*}
\begin{array}{ll}
-\tilde{\mu}_{1}\xi_{n,m}\geq-\tilde{\mu}_{1}\xi_{1}\left(n\right),&
-\tilde{\mu}_{2}\xi_{n,m}\geq-\tilde{\mu}_{2}\xi_{1}\left(n\right)
\geq-\tilde{\mu}_{2}\xi_{2}\left(n\right),\\
-\tilde{\mu}_{3}\xi_{n,m}\geq-\tilde{\mu}_{3}\xi_{3}\left(m\right),&
-\tilde{\mu}_{4}\xi_{n,m}\geq-\tilde{\mu}_{4}\xi_{3}\left(m\right)
\geq-\tilde{\mu}_{4}\xi_{4}\left(m\right).
\end{array}
\end{eqnarray*}

Sea $T^{*}\in I\left(n,m\right)$, entonces dado que en particular $T^{*}\in I_{1}\left(n\right)$, se cumple con probabilidad positiva que no hay arribos a las colas $Q_{1}$ y $Q_{2}$, en consecuencia, tampoco hay usuarios de transferencia para $Q_{3}$ y $Q_{4}$, es decir, $\tilde{\mu}_{1}=\mu_{1}$, $\tilde{\mu}_{2}=\mu_{2}$, $\tilde{\mu}_{3}=\mu_{3}$, $\tilde{\mu}_{4}=\mu_{4}$, es decir, los eventos $Q_{1}$ y $Q_{3}$ son condicionalmente independientes en el intervalo $I\left(n,m\right)$; lo mismo ocurre para las colas $Q_{2}$ y $Q_{4}$, por lo tanto tenemos que
%\small{
\begin{eqnarray}
\begin{array}{l}
\prob\left\{A_{1}\left(T^{*}\right),A_{2}\left(T^{*}\right),
A_{3}\left(T^{*}\right),A_{4}\left(T^{*}\right)|T^{*}\in I\left(n,m\right)\right\}\\
=\prod_{j=1}^{4}\prob\left\{A_{j}\left(T^{*}\right)|T^{*}\in I\left(n,m\right)\right\}\\
\geq\prob\left\{A_{1}\left(T^{*}\right)|T^{*}\in I_{1}\left(n\right)\right\}
\prob\left\{A_{2}\left(T^{*}\right)|T^{*}\in I_{2}\left(n\right)\right\}\\
\prob\left\{A_{3}\left(T^{*}\right)|T^{*}\in I_{3}\left(m\right)\right\}
\prob\left\{A_{4}\left(T^{*}\right)|T^{*}\in I_{4}\left(m\right)\right\}\\
=e^{-\mu_{1}\xi_{1}\left(n\right)}
e^{-\mu_{2}\xi_{2}\left(n\right)}
e^{-\mu_{3}\xi_{3}\left(m\right)}
e^{-\mu_{4}\xi_{4}\left(m\right)}\\
=e^{-\left[\tilde{\mu}_{1}\xi_{1}\left(n\right)
+\tilde{\mu}_{2}\xi_{2}\left(n\right)
+\tilde{\mu}_{3}\xi_{3}\left(m\right)
+\tilde{\mu}_{4}\xi_{4}
\left(m\right)\right]}>0.
\end{array}
\end{eqnarray}


Ahora solo resta demostrar que para $n\ge1$, existe $m\geq1$ tal que se cumplen cualquiera de los cuatro casos arriba mencionados: 

\begin{itemize}
\item[a)] $\tau_{3}\left(m\right)<\tau_{2}\left(n\right)<\overline{\tau}_{3}\left(m\right)$

\item[b)] $\overline{\tau}_{3}\left(m\right)<\tau_{2}\left(n\right)
<\tau_{4}\left(m\right)$

\item[c)] $\tau_{4}\left(m\right)<\tau_{2}\left(n\right)<
\overline{\tau}_{4}\left(m\right)$

\item[d)] $\overline{\tau}_{4}\left(m\right)<\tau_{2}\left(n\right)
<\tau_{3}\left(m+1\right)$
\end{itemize}

Consideremos nuevamente el primer caso. Supongamos que no existe $m\geq1$, tal que $I_{1}\left(n\right)\cap I_{3}\left(m\right)\neq\emptyset$, es decir, para toda $m\geq1$, $I_{1}\left(n\right)\cap I_{3}\left(m\right)=\emptyset$, entonces se tiene que ocurren cualquiera de los dos casos

\begin{itemize}
\item[a)] $\tau_{2}\left(n\right)\leq\tau_{3}\left(m\right)$: Recordemos que $\tau_{2}\left(m\right)=\overline{\tau}_{1}+r_{1}\left(m\right)$ donde cada una de las variables aleatorias son tales que $\esp\left[\overline{\tau}_{1}\left(n\right)-\tau_{1}\left(n\right)\right]<\infty$, $\esp\left[R_{1}\right]<\infty$ y $\esp\left[\tau_{3}\left(m\right)\right]<\infty$, lo cual contradice el hecho de que no exista un ciclo $m\geq1$ que satisfaga la condici\'on deseada.

\item[b)] $\tau_{2}\left(n\right)\geq\overline{\tau}_{3}\left(m\right)$: por un argumento similar al anterior se tiene que no es posible que no exista un ciclo $m\geq1$ tal que satisaface la condici\'on deseada.

\end{itemize}

Para el resto de los casos la demostraci\'on es an\'aloga. Por lo tanto, se tiene que efectivamente existe $m\geq1$ tal que $\tau_{3}\left(m\right)<\tau_{2}\left(n\right)<\tau_{4}\left(m\right)$.
\end{proof}
\newpage

En Sigman, Thorison y Wolff \cite{Sigman2} prueban que para la existencia de un una sucesi\'on infinita no decreciente de tiempos de regeneraci\'on $\tau_{1}\leq\tau_{2}\leq\cdots$ en los cuales el proceso se regenera, basta un tiempo de regeneraci\'on $R_{1}$, donde $R_{j}=\tau_{j}-\tau_{j-1}$. Para tal efecto se requiere la existencia de un espacio de probabilidad $\left(\Omega,\mathcal{F},\prob\right)$, y proceso estoc\'astico $\textit{X}=\left\{X\left(t\right):t\geq0\right\}$ con espacio de estados $\left(S,\mathcal{R}\right)$, con $\mathcal{R}$ $\sigma$-\'algebra.

\begin{Prop}
Si existe una variable aleatoria no negativa $R_{1}$ tal que $\theta_{R1}X=_{D}X$, entonces $\left(\Omega,\mathcal{F},\prob\right)$ puede extenderse para soportar una sucesi\'on estacionaria de variables aleatorias $R=\left\{R_{k}:k\geq1\right\}$, tal que para $k\geq1$,
\begin{eqnarray*}
\theta_{k}\left(X,R\right)=_{D}\left(X,R\right).
\end{eqnarray*}

Adem\'as, para $k\geq1$, $\theta_{k}R$ es condicionalmente independiente de $\left(X,R_{1},\ldots,R_{k}\right)$, dado $\theta_{\tau k}X$.

\end{Prop}


\begin{itemize}
\item Doob en 1953 demostr\'o que el estado estacionario de un proceso de partida en un sistema de espera $M/G/\infty$, es Poisson con la misma tasa que el proceso de arribos.

\item Burke en 1968, fue el primero en demostrar que el estado estacionario de un proceso de salida de una cola $M/M/s$ es un proceso Poisson.

\item Disney en 1973 obtuvo el siguiente resultado:

\begin{Teo}
Para el sistema de espera $M/G/1/L$ con disciplina FIFO, el proceso $\textbf{I}$ es un proceso de renovaci\'on si y s\'olo si el proceso denominado longitud de la cola es estacionario y se cumple cualquiera de los siguientes casos:

\begin{itemize}
\item[a)] Los tiempos de servicio son identicamente cero;
\item[b)] $L=0$, para cualquier proceso de servicio $S$;
\item[c)] $L=1$ y $G=D$;
\item[d)] $L=\infty$ y $G=M$.
\end{itemize}
En estos casos, respectivamente, las distribuciones de interpartida $P\left\{T_{n+1}-T_{n}\leq t\right\}$ son


\begin{itemize}
\item[a)] $1-e^{-\lambda t}$, $t\geq0$;
\item[b)] $1-e^{-\lambda t}*F\left(t\right)$, $t\geq0$;
\item[c)] $1-e^{-\lambda t}*\indora_{d}\left(t\right)$, $t\geq0$;
\item[d)] $1-e^{-\lambda t}*F\left(t\right)$, $t\geq0$.
\end{itemize}
\end{Teo}


\item Finch (1959) mostr\'o que para los sistemas $M/G/1/L$, con $1\leq L\leq \infty$ con distribuciones de servicio dos veces diferenciable, solamente el sistema $M/M/1/\infty$ tiene proceso de salida de renovaci\'on estacionario.

\item King (1971) demostro que un sistema de colas estacionario $M/G/1/1$ tiene sus tiempos de interpartida sucesivas $D_{n}$ y $D_{n+1}$ son independientes, si y s\'olo si, $G=D$, en cuyo caso le proceso de salida es de renovaci\'on.

\item Disney (1973) demostr\'o que el \'unico sistema estacionario $M/G/1/L$, que tiene proceso de salida de renovaci\'on  son los sistemas $M/M/1$ y $M/D/1/1$.



\item El siguiente resultado es de Disney y Koning (1985)
\begin{Teo}
En un sistema de espera $M/G/s$, el estado estacionario del proceso de salida es un proceso Poisson para cualquier distribuci\'on de los tiempos de servicio si el sistema tiene cualquiera de las siguientes cuatro propiedades.

\begin{itemize}
\item[a)] $s=\infty$
\item[b)] La disciplina de servicio es de procesador compartido.
\item[c)] La disciplina de servicio es LCFS y preemptive resume, esto se cumple para $L<\infty$
\item[d)] $G=M$.
\end{itemize}

\end{Teo}

\item El siguiente resultado es de Alamatsaz (1983)

\begin{Teo}
En cualquier sistema de colas $GI/G/1/L$ con $1\leq L<\infty$ y distribuci\'on de interarribos $A$ y distribuci\'on de los tiempos de servicio $B$, tal que $A\left(0\right)=0$, $A\left(t\right)\left(1-B\left(t\right)\right)>0$ para alguna $t>0$ y $B\left(t\right)$ para toda $t>0$, es imposible que el proceso de salida estacionario sea de renovaci\'on.
\end{Teo}

\end{itemize}



%________________________________________________________________________
%\subsection{Procesos Regenerativos Sigman, Thorisson y Wolff \cite{Sigman1}}
%________________________________________________________________________


\begin{Def}[Definici\'on Cl\'asica]
Un proceso estoc\'astico $X=\left\{X\left(t\right):t\geq0\right\}$ es llamado regenerativo is existe una variable aleatoria $R_{1}>0$ tal que
\begin{itemize}
\item[i)] $\left\{X\left(t+R_{1}\right):t\geq0\right\}$ es independiente de $\left\{\left\{X\left(t\right):t<R_{1}\right\},\right\}$
\item[ii)] $\left\{X\left(t+R_{1}\right):t\geq0\right\}$ es estoc\'asticamente equivalente a $\left\{X\left(t\right):t>0\right\}$
\end{itemize}

Llamamos a $R_{1}$ tiempo de regeneraci\'on, y decimos que $X$ se regenera en este punto.
\end{Def}

$\left\{X\left(t+R_{1}\right)\right\}$ es regenerativo con tiempo de regeneraci\'on $R_{2}$, independiente de $R_{1}$ pero con la misma distribuci\'on que $R_{1}$. Procediendo de esta manera se obtiene una secuencia de variables aleatorias independientes e id\'enticamente distribuidas $\left\{R_{n}\right\}$ llamados longitudes de ciclo. Si definimos a $Z_{k}\equiv R_{1}+R_{2}+\cdots+R_{k}$, se tiene un proceso de renovaci\'on llamado proceso de renovaci\'on encajado para $X$.


\begin{Note}
La existencia de un primer tiempo de regeneraci\'on, $R_{1}$, implica la existencia de una sucesi\'on completa de estos tiempos $R_{1},R_{2}\ldots,$ que satisfacen la propiedad deseada \cite{Sigman2}.
\end{Note}


\begin{Note} Para la cola $GI/GI/1$ los usuarios arriban con tiempos $t_{n}$ y son atendidos con tiempos de servicio $S_{n}$, los tiempos de arribo forman un proceso de renovaci\'on  con tiempos entre arribos independientes e identicamente distribuidos (\texttt{i.i.d.})$T_{n}=t_{n}-t_{n-1}$, adem\'as los tiempos de servicio son \texttt{i.i.d.} e independientes de los procesos de arribo. Por \textit{estable} se entiende que $\esp S_{n}<\esp T_{n}<\infty$.
\end{Note}
 


\begin{Def}
Para $x$ fijo y para cada $t\geq0$, sea $I_{x}\left(t\right)=1$ si $X\left(t\right)\leq x$,  $I_{x}\left(t\right)=0$ en caso contrario, y def\'inanse los tiempos promedio
\begin{eqnarray*}
\overline{X}&=&lim_{t\rightarrow\infty}\frac{1}{t}\int_{0}^{\infty}X\left(u\right)du\\
\prob\left(X_{\infty}\leq x\right)&=&lim_{t\rightarrow\infty}\frac{1}{t}\int_{0}^{\infty}I_{x}\left(u\right)du,
\end{eqnarray*}
cuando estos l\'imites existan.
\end{Def}

Como consecuencia del teorema de Renovaci\'on-Recompensa, se tiene que el primer l\'imite  existe y es igual a la constante
\begin{eqnarray*}
\overline{X}&=&\frac{\esp\left[\int_{0}^{R_{1}}X\left(t\right)dt\right]}{\esp\left[R_{1}\right]},
\end{eqnarray*}
suponiendo que ambas esperanzas son finitas.
 
\begin{Note}
Funciones de procesos regenerativos son regenerativas, es decir, si $X\left(t\right)$ es regenerativo y se define el proceso $Y\left(t\right)$ por $Y\left(t\right)=f\left(X\left(t\right)\right)$ para alguna funci\'on Borel medible $f\left(\cdot\right)$. Adem\'as $Y$ es regenerativo con los mismos tiempos de renovaci\'on que $X$. 

En general, los tiempos de renovaci\'on, $Z_{k}$ de un proceso regenerativo no requieren ser tiempos de paro con respecto a la evoluci\'on de $X\left(t\right)$.
\end{Note} 

\begin{Note}
Una funci\'on de un proceso de Markov, usualmente no ser\'a un proceso de Markov, sin embargo ser\'a regenerativo si el proceso de Markov lo es.
\end{Note}

 
\begin{Note}
Un proceso regenerativo con media de la longitud de ciclo finita es llamado positivo recurrente.
\end{Note}


\begin{Note}
\begin{itemize}
\item[a)] Si el proceso regenerativo $X$ es positivo recurrente y tiene trayectorias muestrales no negativas, entonces la ecuaci\'on anterior es v\'alida.
\item[b)] Si $X$ es positivo recurrente regenerativo, podemos construir una \'unica versi\'on estacionaria de este proceso, $X_{e}=\left\{X_{e}\left(t\right)\right\}$, donde $X_{e}$ es un proceso estoc\'astico regenerativo y estrictamente estacionario, con distribuci\'on marginal distribuida como $X_{\infty}$
\end{itemize}
\end{Note}


%__________________________________________________________________________________________
%\subsection{Procesos Regenerativos Estacionarios - Stidham \cite{Stidham}}
%__________________________________________________________________________________________


Un proceso estoc\'astico a tiempo continuo $\left\{V\left(t\right),t\geq0\right\}$ es un proceso regenerativo si existe una sucesi\'on de variables aleatorias independientes e id\'enticamente distribuidas $\left\{X_{1},X_{2},\ldots\right\}$, sucesi\'on de renovaci\'on, tal que para cualquier conjunto de Borel $A$, 

\begin{eqnarray*}
\prob\left\{V\left(t\right)\in A|X_{1}+X_{2}+\cdots+X_{R\left(t\right)}=s,\left\{V\left(\tau\right),\tau<s\right\}\right\}=\prob\left\{V\left(t-s\right)\in A|X_{1}>t-s\right\},
\end{eqnarray*}
para todo $0\leq s\leq t$, donde $R\left(t\right)=\max\left\{X_{1}+X_{2}+\cdots+X_{j}\leq t\right\}=$n\'umero de renovaciones ({\emph{puntos de regeneraci\'on}}) que ocurren en $\left[0,t\right]$. El intervalo $\left[0,X_{1}\right)$ es llamado {\emph{primer ciclo de regeneraci\'on}} de $\left\{V\left(t \right),t\geq0\right\}$, $\left[X_{1},X_{1}+X_{2}\right)$ el {\emph{segundo ciclo de regeneraci\'on}}, y as\'i sucesivamente.

Sea $X=X_{1}$ y sea $F$ la funci\'on de distrbuci\'on de $X$


\begin{Def}
Se define el proceso estacionario, $\left\{V^{*}\left(t\right),t\geq0\right\}$, para $\left\{V\left(t\right),t\geq0\right\}$ por

\begin{eqnarray*}
\prob\left\{V\left(t\right)\in A\right\}=\frac{1}{\esp\left[X\right]}\int_{0}^{\infty}\prob\left\{V\left(t+x\right)\in A|X>x\right\}\left(1-F\left(x\right)\right)dx,
\end{eqnarray*} 
para todo $t\geq0$ y todo conjunto de Borel $A$.
\end{Def}

\begin{Def}
Una distribuci\'on se dice que es {\emph{aritm\'etica}} si todos sus puntos de incremento son m\'ultiplos de la forma $0,\lambda, 2\lambda,\ldots$ para alguna $\lambda>0$ entera.
\end{Def}


\begin{Def}
Una modificaci\'on medible de un proceso $\left\{V\left(t\right),t\geq0\right\}$, es una versi\'on de este, $\left\{V\left(t,w\right)\right\}$ conjuntamente medible para $t\geq0$ y para $w\in S$, $S$ espacio de estados para $\left\{V\left(t\right),t\geq0\right\}$.
\end{Def}

\begin{Teo}
Sea $\left\{V\left(t\right),t\geq\right\}$ un proceso regenerativo no negativo con modificaci\'on medible. Sea $\esp\left[X\right]<\infty$. Entonces el proceso estacionario dado por la ecuaci\'on anterior est\'a bien definido y tiene funci\'on de distribuci\'on independiente de $t$, adem\'as
\begin{itemize}
\item[i)] \begin{eqnarray*}
\esp\left[V^{*}\left(0\right)\right]&=&\frac{\esp\left[\int_{0}^{X}V\left(s\right)ds\right]}{\esp\left[X\right]}\end{eqnarray*}
\item[ii)] Si $\esp\left[V^{*}\left(0\right)\right]<\infty$, equivalentemente, si $\esp\left[\int_{0}^{X}V\left(s\right)ds\right]<\infty$,entonces
\begin{eqnarray*}
\frac{\int_{0}^{t}V\left(s\right)ds}{t}\rightarrow\frac{\esp\left[\int_{0}^{X}V\left(s\right)ds\right]}{\esp\left[X\right]}
\end{eqnarray*}
con probabilidad 1 y en media, cuando $t\rightarrow\infty$.
\end{itemize}
\end{Teo}

\begin{Coro}
Sea $\left\{V\left(t\right),t\geq0\right\}$ un proceso regenerativo no negativo, con modificaci\'on medible. Si $\esp <\infty$, $F$ es no-aritm\'etica, y para todo $x\geq0$, $P\left\{V\left(t\right)\leq x,C>x\right\}$ es de variaci\'on acotada como funci\'on de $t$ en cada intervalo finito $\left[0,\tau\right]$, entonces $V\left(t\right)$ converge en distribuci\'on  cuando $t\rightarrow\infty$ y $$\esp V=\frac{\esp \int_{0}^{X}V\left(s\right)ds}{\esp X}$$
Donde $V$ tiene la distribuci\'on l\'imite de $V\left(t\right)$ cuando $t\rightarrow\infty$.

\end{Coro}

Para el caso discreto se tienen resultados similares.



%______________________________________________________________________
%\subsection{Procesos de Renovaci\'on}
%______________________________________________________________________

\begin{Def}%\label{Def.Tn}
Sean $0\leq T_{1}\leq T_{2}\leq \ldots$ son tiempos aleatorios infinitos en los cuales ocurren ciertos eventos. El n\'umero de tiempos $T_{n}$ en el intervalo $\left[0,t\right)$ es

\begin{eqnarray}
N\left(t\right)=\sum_{n=1}^{\infty}\indora\left(T_{n}\leq t\right),
\end{eqnarray}
para $t\geq0$.
\end{Def}

Si se consideran los puntos $T_{n}$ como elementos de $\rea_{+}$, y $N\left(t\right)$ es el n\'umero de puntos en $\rea$. El proceso denotado por $\left\{N\left(t\right):t\geq0\right\}$, denotado por $N\left(t\right)$, es un proceso puntual en $\rea_{+}$. Los $T_{n}$ son los tiempos de ocurrencia, el proceso puntual $N\left(t\right)$ es simple si su n\'umero de ocurrencias son distintas: $0<T_{1}<T_{2}<\ldots$ casi seguramente.

\begin{Def}
Un proceso puntual $N\left(t\right)$ es un proceso de renovaci\'on si los tiempos de interocurrencia $\xi_{n}=T_{n}-T_{n-1}$, para $n\geq1$, son independientes e identicamente distribuidos con distribuci\'on $F$, donde $F\left(0\right)=0$ y $T_{0}=0$. Los $T_{n}$ son llamados tiempos de renovaci\'on, referente a la independencia o renovaci\'on de la informaci\'on estoc\'astica en estos tiempos. Los $\xi_{n}$ son los tiempos de inter-renovaci\'on, y $N\left(t\right)$ es el n\'umero de renovaciones en el intervalo $\left[0,t\right)$
\end{Def}


\begin{Note}
Para definir un proceso de renovaci\'on para cualquier contexto, solamente hay que especificar una distribuci\'on $F$, con $F\left(0\right)=0$, para los tiempos de inter-renovaci\'on. La funci\'on $F$ en turno degune las otra variables aleatorias. De manera formal, existe un espacio de probabilidad y una sucesi\'on de variables aleatorias $\xi_{1},\xi_{2},\ldots$ definidas en este con distribuci\'on $F$. Entonces las otras cantidades son $T_{n}=\sum_{k=1}^{n}\xi_{k}$ y $N\left(t\right)=\sum_{n=1}^{\infty}\indora\left(T_{n}\leq t\right)$, donde $T_{n}\rightarrow\infty$ casi seguramente por la Ley Fuerte de los Grandes Números.
\end{Note}

%___________________________________________________________________________________________
%
%\subsection{Teorema Principal de Renovaci\'on}
%___________________________________________________________________________________________
%

\begin{Note} Una funci\'on $h:\rea_{+}\rightarrow\rea$ es Directamente Riemann Integrable en los siguientes casos:
\begin{itemize}
\item[a)] $h\left(t\right)\geq0$ es decreciente y Riemann Integrable.
\item[b)] $h$ es continua excepto posiblemente en un conjunto de Lebesgue de medida 0, y $|h\left(t\right)|\leq b\left(t\right)$, donde $b$ es DRI.
\end{itemize}
\end{Note}

\begin{Teo}[Teorema Principal de Renovaci\'on]
Si $F$ es no aritm\'etica y $h\left(t\right)$ es Directamente Riemann Integrable (DRI), entonces

\begin{eqnarray*}
lim_{t\rightarrow\infty}U\star h=\frac{1}{\mu}\int_{\rea_{+}}h\left(s\right)ds.
\end{eqnarray*}
\end{Teo}

\begin{Prop}
Cualquier funci\'on $H\left(t\right)$ acotada en intervalos finitos y que es 0 para $t<0$ puede expresarse como
\begin{eqnarray*}
H\left(t\right)=U\star h\left(t\right)\textrm{,  donde }h\left(t\right)=H\left(t\right)-F\star H\left(t\right)
\end{eqnarray*}
\end{Prop}

\begin{Def}
Un proceso estoc\'astico $X\left(t\right)$ es crudamente regenerativo en un tiempo aleatorio positivo $T$ si
\begin{eqnarray*}
\esp\left[X\left(T+t\right)|T\right]=\esp\left[X\left(t\right)\right]\textrm{, para }t\geq0,\end{eqnarray*}
y con las esperanzas anteriores finitas.
\end{Def}

\begin{Prop}
Sup\'ongase que $X\left(t\right)$ es un proceso crudamente regenerativo en $T$, que tiene distribuci\'on $F$. Si $\esp\left[X\left(t\right)\right]$ es acotado en intervalos finitos, entonces
\begin{eqnarray*}
\esp\left[X\left(t\right)\right]=U\star h\left(t\right)\textrm{,  donde }h\left(t\right)=\esp\left[X\left(t\right)\indora\left(T>t\right)\right].
\end{eqnarray*}
\end{Prop}

\begin{Teo}[Regeneraci\'on Cruda]
Sup\'ongase que $X\left(t\right)$ es un proceso con valores positivo crudamente regenerativo en $T$, y def\'inase $M=\sup\left\{|X\left(t\right)|:t\leq T\right\}$. Si $T$ es no aritm\'etico y $M$ y $MT$ tienen media finita, entonces
\begin{eqnarray*}
lim_{t\rightarrow\infty}\esp\left[X\left(t\right)\right]=\frac{1}{\mu}\int_{\rea_{+}}h\left(s\right)ds,
\end{eqnarray*}
donde $h\left(t\right)=\esp\left[X\left(t\right)\indora\left(T>t\right)\right]$.
\end{Teo}

%___________________________________________________________________________________________
%
%\subsection{Propiedades de los Procesos de Renovaci\'on}
%___________________________________________________________________________________________
%

Los tiempos $T_{n}$ est\'an relacionados con los conteos de $N\left(t\right)$ por

\begin{eqnarray*}
\left\{N\left(t\right)\geq n\right\}&=&\left\{T_{n}\leq t\right\}\\
T_{N\left(t\right)}\leq &t&<T_{N\left(t\right)+1},
\end{eqnarray*}

adem\'as $N\left(T_{n}\right)=n$, y 

\begin{eqnarray*}
N\left(t\right)=\max\left\{n:T_{n}\leq t\right\}=\min\left\{n:T_{n+1}>t\right\}
\end{eqnarray*}

Por propiedades de la convoluci\'on se sabe que

\begin{eqnarray*}
P\left\{T_{n}\leq t\right\}=F^{n\star}\left(t\right)
\end{eqnarray*}
que es la $n$-\'esima convoluci\'on de $F$. Entonces 

\begin{eqnarray*}
\left\{N\left(t\right)\geq n\right\}&=&\left\{T_{n}\leq t\right\}\\
P\left\{N\left(t\right)\leq n\right\}&=&1-F^{\left(n+1\right)\star}\left(t\right)
\end{eqnarray*}

Adem\'as usando el hecho de que $\esp\left[N\left(t\right)\right]=\sum_{n=1}^{\infty}P\left\{N\left(t\right)\geq n\right\}$
se tiene que

\begin{eqnarray*}
\esp\left[N\left(t\right)\right]=\sum_{n=1}^{\infty}F^{n\star}\left(t\right)
\end{eqnarray*}

\begin{Prop}
Para cada $t\geq0$, la funci\'on generadora de momentos $\esp\left[e^{\alpha N\left(t\right)}\right]$ existe para alguna $\alpha$ en una vecindad del 0, y de aqu\'i que $\esp\left[N\left(t\right)^{m}\right]<\infty$, para $m\geq1$.
\end{Prop}


\begin{Note}
Si el primer tiempo de renovaci\'on $\xi_{1}$ no tiene la misma distribuci\'on que el resto de las $\xi_{n}$, para $n\geq2$, a $N\left(t\right)$ se le llama Proceso de Renovaci\'on retardado, donde si $\xi$ tiene distribuci\'on $G$, entonces el tiempo $T_{n}$ de la $n$-\'esima renovaci\'on tiene distribuci\'on $G\star F^{\left(n-1\right)\star}\left(t\right)$
\end{Note}


\begin{Teo}
Para una constante $\mu\leq\infty$ ( o variable aleatoria), las siguientes expresiones son equivalentes:

\begin{eqnarray}
lim_{n\rightarrow\infty}n^{-1}T_{n}&=&\mu,\textrm{ c.s.}\\
lim_{t\rightarrow\infty}t^{-1}N\left(t\right)&=&1/\mu,\textrm{ c.s.}
\end{eqnarray}
\end{Teo}


Es decir, $T_{n}$ satisface la Ley Fuerte de los Grandes N\'umeros s\'i y s\'olo s\'i $N\left/t\right)$ la cumple.


\begin{Coro}[Ley Fuerte de los Grandes N\'umeros para Procesos de Renovaci\'on]
Si $N\left(t\right)$ es un proceso de renovaci\'on cuyos tiempos de inter-renovaci\'on tienen media $\mu\leq\infty$, entonces
\begin{eqnarray}
t^{-1}N\left(t\right)\rightarrow 1/\mu,\textrm{ c.s. cuando }t\rightarrow\infty.
\end{eqnarray}

\end{Coro}


Considerar el proceso estoc\'astico de valores reales $\left\{Z\left(t\right):t\geq0\right\}$ en el mismo espacio de probabilidad que $N\left(t\right)$

\begin{Def}
Para el proceso $\left\{Z\left(t\right):t\geq0\right\}$ se define la fluctuaci\'on m\'axima de $Z\left(t\right)$ en el intervalo $\left(T_{n-1},T_{n}\right]$:
\begin{eqnarray*}
M_{n}=\sup_{T_{n-1}<t\leq T_{n}}|Z\left(t\right)-Z\left(T_{n-1}\right)|
\end{eqnarray*}
\end{Def}

\begin{Teo}
Sup\'ongase que $n^{-1}T_{n}\rightarrow\mu$ c.s. cuando $n\rightarrow\infty$, donde $\mu\leq\infty$ es una constante o variable aleatoria. Sea $a$ una constante o variable aleatoria que puede ser infinita cuando $\mu$ es finita, y considere las expresiones l\'imite:
\begin{eqnarray}
lim_{n\rightarrow\infty}n^{-1}Z\left(T_{n}\right)&=&a,\textrm{ c.s.}\\
lim_{t\rightarrow\infty}t^{-1}Z\left(t\right)&=&a/\mu,\textrm{ c.s.}
\end{eqnarray}
La segunda expresi\'on implica la primera. Conversamente, la primera implica la segunda si el proceso $Z\left(t\right)$ es creciente, o si $lim_{n\rightarrow\infty}n^{-1}M_{n}=0$ c.s.
\end{Teo}

\begin{Coro}
Si $N\left(t\right)$ es un proceso de renovaci\'on, y $\left(Z\left(T_{n}\right)-Z\left(T_{n-1}\right),M_{n}\right)$, para $n\geq1$, son variables aleatorias independientes e id\'enticamente distribuidas con media finita, entonces,
\begin{eqnarray}
lim_{t\rightarrow\infty}t^{-1}Z\left(t\right)\rightarrow\frac{\esp\left[Z\left(T_{1}\right)-Z\left(T_{0}\right)\right]}{\esp\left[T_{1}\right]},\textrm{ c.s. cuando  }t\rightarrow\infty.
\end{eqnarray}
\end{Coro}



%___________________________________________________________________________________________
%
%\subsection{Propiedades de los Procesos de Renovaci\'on}
%___________________________________________________________________________________________
%

Los tiempos $T_{n}$ est\'an relacionados con los conteos de $N\left(t\right)$ por

\begin{eqnarray*}
\left\{N\left(t\right)\geq n\right\}&=&\left\{T_{n}\leq t\right\}\\
T_{N\left(t\right)}\leq &t&<T_{N\left(t\right)+1},
\end{eqnarray*}

adem\'as $N\left(T_{n}\right)=n$, y 

\begin{eqnarray*}
N\left(t\right)=\max\left\{n:T_{n}\leq t\right\}=\min\left\{n:T_{n+1}>t\right\}
\end{eqnarray*}

Por propiedades de la convoluci\'on se sabe que

\begin{eqnarray*}
P\left\{T_{n}\leq t\right\}=F^{n\star}\left(t\right)
\end{eqnarray*}
que es la $n$-\'esima convoluci\'on de $F$. Entonces 

\begin{eqnarray*}
\left\{N\left(t\right)\geq n\right\}&=&\left\{T_{n}\leq t\right\}\\
P\left\{N\left(t\right)\leq n\right\}&=&1-F^{\left(n+1\right)\star}\left(t\right)
\end{eqnarray*}

Adem\'as usando el hecho de que $\esp\left[N\left(t\right)\right]=\sum_{n=1}^{\infty}P\left\{N\left(t\right)\geq n\right\}$
se tiene que

\begin{eqnarray*}
\esp\left[N\left(t\right)\right]=\sum_{n=1}^{\infty}F^{n\star}\left(t\right)
\end{eqnarray*}

\begin{Prop}
Para cada $t\geq0$, la funci\'on generadora de momentos $\esp\left[e^{\alpha N\left(t\right)}\right]$ existe para alguna $\alpha$ en una vecindad del 0, y de aqu\'i que $\esp\left[N\left(t\right)^{m}\right]<\infty$, para $m\geq1$.
\end{Prop}


\begin{Note}
Si el primer tiempo de renovaci\'on $\xi_{1}$ no tiene la misma distribuci\'on que el resto de las $\xi_{n}$, para $n\geq2$, a $N\left(t\right)$ se le llama Proceso de Renovaci\'on retardado, donde si $\xi$ tiene distribuci\'on $G$, entonces el tiempo $T_{n}$ de la $n$-\'esima renovaci\'on tiene distribuci\'on $G\star F^{\left(n-1\right)\star}\left(t\right)$
\end{Note}


\begin{Teo}
Para una constante $\mu\leq\infty$ ( o variable aleatoria), las siguientes expresiones son equivalentes:

\begin{eqnarray}
lim_{n\rightarrow\infty}n^{-1}T_{n}&=&\mu,\textrm{ c.s.}\\
lim_{t\rightarrow\infty}t^{-1}N\left(t\right)&=&1/\mu,\textrm{ c.s.}
\end{eqnarray}
\end{Teo}


Es decir, $T_{n}$ satisface la Ley Fuerte de los Grandes N\'umeros s\'i y s\'olo s\'i $N\left/t\right)$ la cumple.


\begin{Coro}[Ley Fuerte de los Grandes N\'umeros para Procesos de Renovaci\'on]
Si $N\left(t\right)$ es un proceso de renovaci\'on cuyos tiempos de inter-renovaci\'on tienen media $\mu\leq\infty$, entonces
\begin{eqnarray}
t^{-1}N\left(t\right)\rightarrow 1/\mu,\textrm{ c.s. cuando }t\rightarrow\infty.
\end{eqnarray}

\end{Coro}


Considerar el proceso estoc\'astico de valores reales $\left\{Z\left(t\right):t\geq0\right\}$ en el mismo espacio de probabilidad que $N\left(t\right)$

\begin{Def}
Para el proceso $\left\{Z\left(t\right):t\geq0\right\}$ se define la fluctuaci\'on m\'axima de $Z\left(t\right)$ en el intervalo $\left(T_{n-1},T_{n}\right]$:
\begin{eqnarray*}
M_{n}=\sup_{T_{n-1}<t\leq T_{n}}|Z\left(t\right)-Z\left(T_{n-1}\right)|
\end{eqnarray*}
\end{Def}

\begin{Teo}
Sup\'ongase que $n^{-1}T_{n}\rightarrow\mu$ c.s. cuando $n\rightarrow\infty$, donde $\mu\leq\infty$ es una constante o variable aleatoria. Sea $a$ una constante o variable aleatoria que puede ser infinita cuando $\mu$ es finita, y considere las expresiones l\'imite:
\begin{eqnarray}
lim_{n\rightarrow\infty}n^{-1}Z\left(T_{n}\right)&=&a,\textrm{ c.s.}\\
lim_{t\rightarrow\infty}t^{-1}Z\left(t\right)&=&a/\mu,\textrm{ c.s.}
\end{eqnarray}
La segunda expresi\'on implica la primera. Conversamente, la primera implica la segunda si el proceso $Z\left(t\right)$ es creciente, o si $lim_{n\rightarrow\infty}n^{-1}M_{n}=0$ c.s.
\end{Teo}

\begin{Coro}
Si $N\left(t\right)$ es un proceso de renovaci\'on, y $\left(Z\left(T_{n}\right)-Z\left(T_{n-1}\right),M_{n}\right)$, para $n\geq1$, son variables aleatorias independientes e id\'enticamente distribuidas con media finita, entonces,
\begin{eqnarray}
lim_{t\rightarrow\infty}t^{-1}Z\left(t\right)\rightarrow\frac{\esp\left[Z\left(T_{1}\right)-Z\left(T_{0}\right)\right]}{\esp\left[T_{1}\right]},\textrm{ c.s. cuando  }t\rightarrow\infty.
\end{eqnarray}
\end{Coro}


%___________________________________________________________________________________________
%
%\subsection{Propiedades de los Procesos de Renovaci\'on}
%___________________________________________________________________________________________
%

Los tiempos $T_{n}$ est\'an relacionados con los conteos de $N\left(t\right)$ por

\begin{eqnarray*}
\left\{N\left(t\right)\geq n\right\}&=&\left\{T_{n}\leq t\right\}\\
T_{N\left(t\right)}\leq &t&<T_{N\left(t\right)+1},
\end{eqnarray*}

adem\'as $N\left(T_{n}\right)=n$, y 

\begin{eqnarray*}
N\left(t\right)=\max\left\{n:T_{n}\leq t\right\}=\min\left\{n:T_{n+1}>t\right\}
\end{eqnarray*}

Por propiedades de la convoluci\'on se sabe que

\begin{eqnarray*}
P\left\{T_{n}\leq t\right\}=F^{n\star}\left(t\right)
\end{eqnarray*}
que es la $n$-\'esima convoluci\'on de $F$. Entonces 

\begin{eqnarray*}
\left\{N\left(t\right)\geq n\right\}&=&\left\{T_{n}\leq t\right\}\\
P\left\{N\left(t\right)\leq n\right\}&=&1-F^{\left(n+1\right)\star}\left(t\right)
\end{eqnarray*}

Adem\'as usando el hecho de que $\esp\left[N\left(t\right)\right]=\sum_{n=1}^{\infty}P\left\{N\left(t\right)\geq n\right\}$
se tiene que

\begin{eqnarray*}
\esp\left[N\left(t\right)\right]=\sum_{n=1}^{\infty}F^{n\star}\left(t\right)
\end{eqnarray*}

\begin{Prop}
Para cada $t\geq0$, la funci\'on generadora de momentos $\esp\left[e^{\alpha N\left(t\right)}\right]$ existe para alguna $\alpha$ en una vecindad del 0, y de aqu\'i que $\esp\left[N\left(t\right)^{m}\right]<\infty$, para $m\geq1$.
\end{Prop}


\begin{Note}
Si el primer tiempo de renovaci\'on $\xi_{1}$ no tiene la misma distribuci\'on que el resto de las $\xi_{n}$, para $n\geq2$, a $N\left(t\right)$ se le llama Proceso de Renovaci\'on retardado, donde si $\xi$ tiene distribuci\'on $G$, entonces el tiempo $T_{n}$ de la $n$-\'esima renovaci\'on tiene distribuci\'on $G\star F^{\left(n-1\right)\star}\left(t\right)$
\end{Note}


\begin{Teo}
Para una constante $\mu\leq\infty$ ( o variable aleatoria), las siguientes expresiones son equivalentes:

\begin{eqnarray}
lim_{n\rightarrow\infty}n^{-1}T_{n}&=&\mu,\textrm{ c.s.}\\
lim_{t\rightarrow\infty}t^{-1}N\left(t\right)&=&1/\mu,\textrm{ c.s.}
\end{eqnarray}
\end{Teo}


Es decir, $T_{n}$ satisface la Ley Fuerte de los Grandes N\'umeros s\'i y s\'olo s\'i $N\left/t\right)$ la cumple.


\begin{Coro}[Ley Fuerte de los Grandes N\'umeros para Procesos de Renovaci\'on]
Si $N\left(t\right)$ es un proceso de renovaci\'on cuyos tiempos de inter-renovaci\'on tienen media $\mu\leq\infty$, entonces
\begin{eqnarray}
t^{-1}N\left(t\right)\rightarrow 1/\mu,\textrm{ c.s. cuando }t\rightarrow\infty.
\end{eqnarray}

\end{Coro}


Considerar el proceso estoc\'astico de valores reales $\left\{Z\left(t\right):t\geq0\right\}$ en el mismo espacio de probabilidad que $N\left(t\right)$

\begin{Def}
Para el proceso $\left\{Z\left(t\right):t\geq0\right\}$ se define la fluctuaci\'on m\'axima de $Z\left(t\right)$ en el intervalo $\left(T_{n-1},T_{n}\right]$:
\begin{eqnarray*}
M_{n}=\sup_{T_{n-1}<t\leq T_{n}}|Z\left(t\right)-Z\left(T_{n-1}\right)|
\end{eqnarray*}
\end{Def}

\begin{Teo}
Sup\'ongase que $n^{-1}T_{n}\rightarrow\mu$ c.s. cuando $n\rightarrow\infty$, donde $\mu\leq\infty$ es una constante o variable aleatoria. Sea $a$ una constante o variable aleatoria que puede ser infinita cuando $\mu$ es finita, y considere las expresiones l\'imite:
\begin{eqnarray}
lim_{n\rightarrow\infty}n^{-1}Z\left(T_{n}\right)&=&a,\textrm{ c.s.}\\
lim_{t\rightarrow\infty}t^{-1}Z\left(t\right)&=&a/\mu,\textrm{ c.s.}
\end{eqnarray}
La segunda expresi\'on implica la primera. Conversamente, la primera implica la segunda si el proceso $Z\left(t\right)$ es creciente, o si $lim_{n\rightarrow\infty}n^{-1}M_{n}=0$ c.s.
\end{Teo}

\begin{Coro}
Si $N\left(t\right)$ es un proceso de renovaci\'on, y $\left(Z\left(T_{n}\right)-Z\left(T_{n-1}\right),M_{n}\right)$, para $n\geq1$, son variables aleatorias independientes e id\'enticamente distribuidas con media finita, entonces,
\begin{eqnarray}
lim_{t\rightarrow\infty}t^{-1}Z\left(t\right)\rightarrow\frac{\esp\left[Z\left(T_{1}\right)-Z\left(T_{0}\right)\right]}{\esp\left[T_{1}\right]},\textrm{ c.s. cuando  }t\rightarrow\infty.
\end{eqnarray}
\end{Coro}

%___________________________________________________________________________________________
%
%\subsection{Propiedades de los Procesos de Renovaci\'on}
%___________________________________________________________________________________________
%

Los tiempos $T_{n}$ est\'an relacionados con los conteos de $N\left(t\right)$ por

\begin{eqnarray*}
\left\{N\left(t\right)\geq n\right\}&=&\left\{T_{n}\leq t\right\}\\
T_{N\left(t\right)}\leq &t&<T_{N\left(t\right)+1},
\end{eqnarray*}

adem\'as $N\left(T_{n}\right)=n$, y 

\begin{eqnarray*}
N\left(t\right)=\max\left\{n:T_{n}\leq t\right\}=\min\left\{n:T_{n+1}>t\right\}
\end{eqnarray*}

Por propiedades de la convoluci\'on se sabe que

\begin{eqnarray*}
P\left\{T_{n}\leq t\right\}=F^{n\star}\left(t\right)
\end{eqnarray*}
que es la $n$-\'esima convoluci\'on de $F$. Entonces 

\begin{eqnarray*}
\left\{N\left(t\right)\geq n\right\}&=&\left\{T_{n}\leq t\right\}\\
P\left\{N\left(t\right)\leq n\right\}&=&1-F^{\left(n+1\right)\star}\left(t\right)
\end{eqnarray*}

Adem\'as usando el hecho de que $\esp\left[N\left(t\right)\right]=\sum_{n=1}^{\infty}P\left\{N\left(t\right)\geq n\right\}$
se tiene que

\begin{eqnarray*}
\esp\left[N\left(t\right)\right]=\sum_{n=1}^{\infty}F^{n\star}\left(t\right)
\end{eqnarray*}

\begin{Prop}
Para cada $t\geq0$, la funci\'on generadora de momentos $\esp\left[e^{\alpha N\left(t\right)}\right]$ existe para alguna $\alpha$ en una vecindad del 0, y de aqu\'i que $\esp\left[N\left(t\right)^{m}\right]<\infty$, para $m\geq1$.
\end{Prop}


\begin{Note}
Si el primer tiempo de renovaci\'on $\xi_{1}$ no tiene la misma distribuci\'on que el resto de las $\xi_{n}$, para $n\geq2$, a $N\left(t\right)$ se le llama Proceso de Renovaci\'on retardado, donde si $\xi$ tiene distribuci\'on $G$, entonces el tiempo $T_{n}$ de la $n$-\'esima renovaci\'on tiene distribuci\'on $G\star F^{\left(n-1\right)\star}\left(t\right)$
\end{Note}


\begin{Teo}
Para una constante $\mu\leq\infty$ ( o variable aleatoria), las siguientes expresiones son equivalentes:

\begin{eqnarray}
lim_{n\rightarrow\infty}n^{-1}T_{n}&=&\mu,\textrm{ c.s.}\\
lim_{t\rightarrow\infty}t^{-1}N\left(t\right)&=&1/\mu,\textrm{ c.s.}
\end{eqnarray}
\end{Teo}


Es decir, $T_{n}$ satisface la Ley Fuerte de los Grandes N\'umeros s\'i y s\'olo s\'i $N\left/t\right)$ la cumple.


\begin{Coro}[Ley Fuerte de los Grandes N\'umeros para Procesos de Renovaci\'on]
Si $N\left(t\right)$ es un proceso de renovaci\'on cuyos tiempos de inter-renovaci\'on tienen media $\mu\leq\infty$, entonces
\begin{eqnarray}
t^{-1}N\left(t\right)\rightarrow 1/\mu,\textrm{ c.s. cuando }t\rightarrow\infty.
\end{eqnarray}

\end{Coro}


Considerar el proceso estoc\'astico de valores reales $\left\{Z\left(t\right):t\geq0\right\}$ en el mismo espacio de probabilidad que $N\left(t\right)$

\begin{Def}
Para el proceso $\left\{Z\left(t\right):t\geq0\right\}$ se define la fluctuaci\'on m\'axima de $Z\left(t\right)$ en el intervalo $\left(T_{n-1},T_{n}\right]$:
\begin{eqnarray*}
M_{n}=\sup_{T_{n-1}<t\leq T_{n}}|Z\left(t\right)-Z\left(T_{n-1}\right)|
\end{eqnarray*}
\end{Def}

\begin{Teo}
Sup\'ongase que $n^{-1}T_{n}\rightarrow\mu$ c.s. cuando $n\rightarrow\infty$, donde $\mu\leq\infty$ es una constante o variable aleatoria. Sea $a$ una constante o variable aleatoria que puede ser infinita cuando $\mu$ es finita, y considere las expresiones l\'imite:
\begin{eqnarray}
lim_{n\rightarrow\infty}n^{-1}Z\left(T_{n}\right)&=&a,\textrm{ c.s.}\\
lim_{t\rightarrow\infty}t^{-1}Z\left(t\right)&=&a/\mu,\textrm{ c.s.}
\end{eqnarray}
La segunda expresi\'on implica la primera. Conversamente, la primera implica la segunda si el proceso $Z\left(t\right)$ es creciente, o si $lim_{n\rightarrow\infty}n^{-1}M_{n}=0$ c.s.
\end{Teo}

\begin{Coro}
Si $N\left(t\right)$ es un proceso de renovaci\'on, y $\left(Z\left(T_{n}\right)-Z\left(T_{n-1}\right),M_{n}\right)$, para $n\geq1$, son variables aleatorias independientes e id\'enticamente distribuidas con media finita, entonces,
\begin{eqnarray}
lim_{t\rightarrow\infty}t^{-1}Z\left(t\right)\rightarrow\frac{\esp\left[Z\left(T_{1}\right)-Z\left(T_{0}\right)\right]}{\esp\left[T_{1}\right]},\textrm{ c.s. cuando  }t\rightarrow\infty.
\end{eqnarray}
\end{Coro}
%___________________________________________________________________________________________
%
%\subsection{Propiedades de los Procesos de Renovaci\'on}
%___________________________________________________________________________________________
%

Los tiempos $T_{n}$ est\'an relacionados con los conteos de $N\left(t\right)$ por

\begin{eqnarray*}
\left\{N\left(t\right)\geq n\right\}&=&\left\{T_{n}\leq t\right\}\\
T_{N\left(t\right)}\leq &t&<T_{N\left(t\right)+1},
\end{eqnarray*}

adem\'as $N\left(T_{n}\right)=n$, y 

\begin{eqnarray*}
N\left(t\right)=\max\left\{n:T_{n}\leq t\right\}=\min\left\{n:T_{n+1}>t\right\}
\end{eqnarray*}

Por propiedades de la convoluci\'on se sabe que

\begin{eqnarray*}
P\left\{T_{n}\leq t\right\}=F^{n\star}\left(t\right)
\end{eqnarray*}
que es la $n$-\'esima convoluci\'on de $F$. Entonces 

\begin{eqnarray*}
\left\{N\left(t\right)\geq n\right\}&=&\left\{T_{n}\leq t\right\}\\
P\left\{N\left(t\right)\leq n\right\}&=&1-F^{\left(n+1\right)\star}\left(t\right)
\end{eqnarray*}

Adem\'as usando el hecho de que $\esp\left[N\left(t\right)\right]=\sum_{n=1}^{\infty}P\left\{N\left(t\right)\geq n\right\}$
se tiene que

\begin{eqnarray*}
\esp\left[N\left(t\right)\right]=\sum_{n=1}^{\infty}F^{n\star}\left(t\right)
\end{eqnarray*}

\begin{Prop}
Para cada $t\geq0$, la funci\'on generadora de momentos $\esp\left[e^{\alpha N\left(t\right)}\right]$ existe para alguna $\alpha$ en una vecindad del 0, y de aqu\'i que $\esp\left[N\left(t\right)^{m}\right]<\infty$, para $m\geq1$.
\end{Prop}


\begin{Note}
Si el primer tiempo de renovaci\'on $\xi_{1}$ no tiene la misma distribuci\'on que el resto de las $\xi_{n}$, para $n\geq2$, a $N\left(t\right)$ se le llama Proceso de Renovaci\'on retardado, donde si $\xi$ tiene distribuci\'on $G$, entonces el tiempo $T_{n}$ de la $n$-\'esima renovaci\'on tiene distribuci\'on $G\star F^{\left(n-1\right)\star}\left(t\right)$
\end{Note}


\begin{Teo}
Para una constante $\mu\leq\infty$ ( o variable aleatoria), las siguientes expresiones son equivalentes:

\begin{eqnarray}
lim_{n\rightarrow\infty}n^{-1}T_{n}&=&\mu,\textrm{ c.s.}\\
lim_{t\rightarrow\infty}t^{-1}N\left(t\right)&=&1/\mu,\textrm{ c.s.}
\end{eqnarray}
\end{Teo}


Es decir, $T_{n}$ satisface la Ley Fuerte de los Grandes N\'umeros s\'i y s\'olo s\'i $N\left/t\right)$ la cumple.


\begin{Coro}[Ley Fuerte de los Grandes N\'umeros para Procesos de Renovaci\'on]
Si $N\left(t\right)$ es un proceso de renovaci\'on cuyos tiempos de inter-renovaci\'on tienen media $\mu\leq\infty$, entonces
\begin{eqnarray}
t^{-1}N\left(t\right)\rightarrow 1/\mu,\textrm{ c.s. cuando }t\rightarrow\infty.
\end{eqnarray}

\end{Coro}


Considerar el proceso estoc\'astico de valores reales $\left\{Z\left(t\right):t\geq0\right\}$ en el mismo espacio de probabilidad que $N\left(t\right)$

\begin{Def}
Para el proceso $\left\{Z\left(t\right):t\geq0\right\}$ se define la fluctuaci\'on m\'axima de $Z\left(t\right)$ en el intervalo $\left(T_{n-1},T_{n}\right]$:
\begin{eqnarray*}
M_{n}=\sup_{T_{n-1}<t\leq T_{n}}|Z\left(t\right)-Z\left(T_{n-1}\right)|
\end{eqnarray*}
\end{Def}

\begin{Teo}
Sup\'ongase que $n^{-1}T_{n}\rightarrow\mu$ c.s. cuando $n\rightarrow\infty$, donde $\mu\leq\infty$ es una constante o variable aleatoria. Sea $a$ una constante o variable aleatoria que puede ser infinita cuando $\mu$ es finita, y considere las expresiones l\'imite:
\begin{eqnarray}
lim_{n\rightarrow\infty}n^{-1}Z\left(T_{n}\right)&=&a,\textrm{ c.s.}\\
lim_{t\rightarrow\infty}t^{-1}Z\left(t\right)&=&a/\mu,\textrm{ c.s.}
\end{eqnarray}
La segunda expresi\'on implica la primera. Conversamente, la primera implica la segunda si el proceso $Z\left(t\right)$ es creciente, o si $lim_{n\rightarrow\infty}n^{-1}M_{n}=0$ c.s.
\end{Teo}

\begin{Coro}
Si $N\left(t\right)$ es un proceso de renovaci\'on, y $\left(Z\left(T_{n}\right)-Z\left(T_{n-1}\right),M_{n}\right)$, para $n\geq1$, son variables aleatorias independientes e id\'enticamente distribuidas con media finita, entonces,
\begin{eqnarray}
lim_{t\rightarrow\infty}t^{-1}Z\left(t\right)\rightarrow\frac{\esp\left[Z\left(T_{1}\right)-Z\left(T_{0}\right)\right]}{\esp\left[T_{1}\right]},\textrm{ c.s. cuando  }t\rightarrow\infty.
\end{eqnarray}
\end{Coro}


%___________________________________________________________________________________________
%
%\subsection{Funci\'on de Renovaci\'on}
%___________________________________________________________________________________________
%


\begin{Def}
Sea $h\left(t\right)$ funci\'on de valores reales en $\rea$ acotada en intervalos finitos e igual a cero para $t<0$ La ecuaci\'on de renovaci\'on para $h\left(t\right)$ y la distribuci\'on $F$ es

\begin{eqnarray}%\label{Ec.Renovacion}
H\left(t\right)=h\left(t\right)+\int_{\left[0,t\right]}H\left(t-s\right)dF\left(s\right)\textrm{,    }t\geq0,
\end{eqnarray}
donde $H\left(t\right)$ es una funci\'on de valores reales. Esto es $H=h+F\star H$. Decimos que $H\left(t\right)$ es soluci\'on de esta ecuaci\'on si satisface la ecuaci\'on, y es acotada en intervalos finitos e iguales a cero para $t<0$.
\end{Def}

\begin{Prop}
La funci\'on $U\star h\left(t\right)$ es la \'unica soluci\'on de la ecuaci\'on de renovaci\'on (\ref{Ec.Renovacion}).
\end{Prop}

\begin{Teo}[Teorema Renovaci\'on Elemental]
\begin{eqnarray*}
t^{-1}U\left(t\right)\rightarrow 1/\mu\textrm{,    cuando }t\rightarrow\infty.
\end{eqnarray*}
\end{Teo}

%___________________________________________________________________________________________
%
%\subsection{Funci\'on de Renovaci\'on}
%___________________________________________________________________________________________
%


Sup\'ongase que $N\left(t\right)$ es un proceso de renovaci\'on con distribuci\'on $F$ con media finita $\mu$.

\begin{Def}
La funci\'on de renovaci\'on asociada con la distribuci\'on $F$, del proceso $N\left(t\right)$, es
\begin{eqnarray*}
U\left(t\right)=\sum_{n=1}^{\infty}F^{n\star}\left(t\right),\textrm{   }t\geq0,
\end{eqnarray*}
donde $F^{0\star}\left(t\right)=\indora\left(t\geq0\right)$.
\end{Def}


\begin{Prop}
Sup\'ongase que la distribuci\'on de inter-renovaci\'on $F$ tiene densidad $f$. Entonces $U\left(t\right)$ tambi\'en tiene densidad, para $t>0$, y es $U^{'}\left(t\right)=\sum_{n=0}^{\infty}f^{n\star}\left(t\right)$. Adem\'as
\begin{eqnarray*}
\prob\left\{N\left(t\right)>N\left(t-\right)\right\}=0\textrm{,   }t\geq0.
\end{eqnarray*}
\end{Prop}

\begin{Def}
La Transformada de Laplace-Stieljes de $F$ est\'a dada por

\begin{eqnarray*}
\hat{F}\left(\alpha\right)=\int_{\rea_{+}}e^{-\alpha t}dF\left(t\right)\textrm{,  }\alpha\geq0.
\end{eqnarray*}
\end{Def}

Entonces

\begin{eqnarray*}
\hat{U}\left(\alpha\right)=\sum_{n=0}^{\infty}\hat{F^{n\star}}\left(\alpha\right)=\sum_{n=0}^{\infty}\hat{F}\left(\alpha\right)^{n}=\frac{1}{1-\hat{F}\left(\alpha\right)}.
\end{eqnarray*}


\begin{Prop}
La Transformada de Laplace $\hat{U}\left(\alpha\right)$ y $\hat{F}\left(\alpha\right)$ determina una a la otra de manera \'unica por la relaci\'on $\hat{U}\left(\alpha\right)=\frac{1}{1-\hat{F}\left(\alpha\right)}$.
\end{Prop}


\begin{Note}
Un proceso de renovaci\'on $N\left(t\right)$ cuyos tiempos de inter-renovaci\'on tienen media finita, es un proceso Poisson con tasa $\lambda$ si y s\'olo s\'i $\esp\left[U\left(t\right)\right]=\lambda t$, para $t\geq0$.
\end{Note}


\begin{Teo}
Sea $N\left(t\right)$ un proceso puntual simple con puntos de localizaci\'on $T_{n}$ tal que $\eta\left(t\right)=\esp\left[N\left(\right)\right]$ es finita para cada $t$. Entonces para cualquier funci\'on $f:\rea_{+}\rightarrow\rea$,
\begin{eqnarray*}
\esp\left[\sum_{n=1}^{N\left(\right)}f\left(T_{n}\right)\right]=\int_{\left(0,t\right]}f\left(s\right)d\eta\left(s\right)\textrm{,  }t\geq0,
\end{eqnarray*}
suponiendo que la integral exista. Adem\'as si $X_{1},X_{2},\ldots$ son variables aleatorias definidas en el mismo espacio de probabilidad que el proceso $N\left(t\right)$ tal que $\esp\left[X_{n}|T_{n}=s\right]=f\left(s\right)$, independiente de $n$. Entonces
\begin{eqnarray*}
\esp\left[\sum_{n=1}^{N\left(t\right)}X_{n}\right]=\int_{\left(0,t\right]}f\left(s\right)d\eta\left(s\right)\textrm{,  }t\geq0,
\end{eqnarray*} 
suponiendo que la integral exista. 
\end{Teo}

\begin{Coro}[Identidad de Wald para Renovaciones]
Para el proceso de renovaci\'on $N\left(t\right)$,
\begin{eqnarray*}
\esp\left[T_{N\left(t\right)+1}\right]=\mu\esp\left[N\left(t\right)+1\right]\textrm{,  }t\geq0,
\end{eqnarray*}  
\end{Coro}

%______________________________________________________________________
%\subsection{Procesos de Renovaci\'on}
%______________________________________________________________________

\begin{Def}%\label{Def.Tn}
Sean $0\leq T_{1}\leq T_{2}\leq \ldots$ son tiempos aleatorios infinitos en los cuales ocurren ciertos eventos. El n\'umero de tiempos $T_{n}$ en el intervalo $\left[0,t\right)$ es

\begin{eqnarray}
N\left(t\right)=\sum_{n=1}^{\infty}\indora\left(T_{n}\leq t\right),
\end{eqnarray}
para $t\geq0$.
\end{Def}

Si se consideran los puntos $T_{n}$ como elementos de $\rea_{+}$, y $N\left(t\right)$ es el n\'umero de puntos en $\rea$. El proceso denotado por $\left\{N\left(t\right):t\geq0\right\}$, denotado por $N\left(t\right)$, es un proceso puntual en $\rea_{+}$. Los $T_{n}$ son los tiempos de ocurrencia, el proceso puntual $N\left(t\right)$ es simple si su n\'umero de ocurrencias son distintas: $0<T_{1}<T_{2}<\ldots$ casi seguramente.

\begin{Def}
Un proceso puntual $N\left(t\right)$ es un proceso de renovaci\'on si los tiempos de interocurrencia $\xi_{n}=T_{n}-T_{n-1}$, para $n\geq1$, son independientes e identicamente distribuidos con distribuci\'on $F$, donde $F\left(0\right)=0$ y $T_{0}=0$. Los $T_{n}$ son llamados tiempos de renovaci\'on, referente a la independencia o renovaci\'on de la informaci\'on estoc\'astica en estos tiempos. Los $\xi_{n}$ son los tiempos de inter-renovaci\'on, y $N\left(t\right)$ es el n\'umero de renovaciones en el intervalo $\left[0,t\right)$
\end{Def}


\begin{Note}
Para definir un proceso de renovaci\'on para cualquier contexto, solamente hay que especificar una distribuci\'on $F$, con $F\left(0\right)=0$, para los tiempos de inter-renovaci\'on. La funci\'on $F$ en turno degune las otra variables aleatorias. De manera formal, existe un espacio de probabilidad y una sucesi\'on de variables aleatorias $\xi_{1},\xi_{2},\ldots$ definidas en este con distribuci\'on $F$. Entonces las otras cantidades son $T_{n}=\sum_{k=1}^{n}\xi_{k}$ y $N\left(t\right)=\sum_{n=1}^{\infty}\indora\left(T_{n}\leq t\right)$, donde $T_{n}\rightarrow\infty$ casi seguramente por la Ley Fuerte de los Grandes Números.
\end{Note}

%___________________________________________________________________________________________
%
%\subsection{Renewal and Regenerative Processes: Serfozo\cite{Serfozo}}
%___________________________________________________________________________________________
%
\begin{Def}%\label{Def.Tn}
Sean $0\leq T_{1}\leq T_{2}\leq \ldots$ son tiempos aleatorios infinitos en los cuales ocurren ciertos eventos. El n\'umero de tiempos $T_{n}$ en el intervalo $\left[0,t\right)$ es

\begin{eqnarray}
N\left(t\right)=\sum_{n=1}^{\infty}\indora\left(T_{n}\leq t\right),
\end{eqnarray}
para $t\geq0$.
\end{Def}

Si se consideran los puntos $T_{n}$ como elementos de $\rea_{+}$, y $N\left(t\right)$ es el n\'umero de puntos en $\rea$. El proceso denotado por $\left\{N\left(t\right):t\geq0\right\}$, denotado por $N\left(t\right)$, es un proceso puntual en $\rea_{+}$. Los $T_{n}$ son los tiempos de ocurrencia, el proceso puntual $N\left(t\right)$ es simple si su n\'umero de ocurrencias son distintas: $0<T_{1}<T_{2}<\ldots$ casi seguramente.

\begin{Def}
Un proceso puntual $N\left(t\right)$ es un proceso de renovaci\'on si los tiempos de interocurrencia $\xi_{n}=T_{n}-T_{n-1}$, para $n\geq1$, son independientes e identicamente distribuidos con distribuci\'on $F$, donde $F\left(0\right)=0$ y $T_{0}=0$. Los $T_{n}$ son llamados tiempos de renovaci\'on, referente a la independencia o renovaci\'on de la informaci\'on estoc\'astica en estos tiempos. Los $\xi_{n}$ son los tiempos de inter-renovaci\'on, y $N\left(t\right)$ es el n\'umero de renovaciones en el intervalo $\left[0,t\right)$
\end{Def}


\begin{Note}
Para definir un proceso de renovaci\'on para cualquier contexto, solamente hay que especificar una distribuci\'on $F$, con $F\left(0\right)=0$, para los tiempos de inter-renovaci\'on. La funci\'on $F$ en turno degune las otra variables aleatorias. De manera formal, existe un espacio de probabilidad y una sucesi\'on de variables aleatorias $\xi_{1},\xi_{2},\ldots$ definidas en este con distribuci\'on $F$. Entonces las otras cantidades son $T_{n}=\sum_{k=1}^{n}\xi_{k}$ y $N\left(t\right)=\sum_{n=1}^{\infty}\indora\left(T_{n}\leq t\right)$, donde $T_{n}\rightarrow\infty$ casi seguramente por la Ley Fuerte de los Grandes N\'umeros.
\end{Note}







Los tiempos $T_{n}$ est\'an relacionados con los conteos de $N\left(t\right)$ por

\begin{eqnarray*}
\left\{N\left(t\right)\geq n\right\}&=&\left\{T_{n}\leq t\right\}\\
T_{N\left(t\right)}\leq &t&<T_{N\left(t\right)+1},
\end{eqnarray*}

adem\'as $N\left(T_{n}\right)=n$, y 

\begin{eqnarray*}
N\left(t\right)=\max\left\{n:T_{n}\leq t\right\}=\min\left\{n:T_{n+1}>t\right\}
\end{eqnarray*}

Por propiedades de la convoluci\'on se sabe que

\begin{eqnarray*}
P\left\{T_{n}\leq t\right\}=F^{n\star}\left(t\right)
\end{eqnarray*}
que es la $n$-\'esima convoluci\'on de $F$. Entonces 

\begin{eqnarray*}
\left\{N\left(t\right)\geq n\right\}&=&\left\{T_{n}\leq t\right\}\\
P\left\{N\left(t\right)\leq n\right\}&=&1-F^{\left(n+1\right)\star}\left(t\right)
\end{eqnarray*}

Adem\'as usando el hecho de que $\esp\left[N\left(t\right)\right]=\sum_{n=1}^{\infty}P\left\{N\left(t\right)\geq n\right\}$
se tiene que

\begin{eqnarray*}
\esp\left[N\left(t\right)\right]=\sum_{n=1}^{\infty}F^{n\star}\left(t\right)
\end{eqnarray*}

\begin{Prop}
Para cada $t\geq0$, la funci\'on generadora de momentos $\esp\left[e^{\alpha N\left(t\right)}\right]$ existe para alguna $\alpha$ en una vecindad del 0, y de aqu\'i que $\esp\left[N\left(t\right)^{m}\right]<\infty$, para $m\geq1$.
\end{Prop}

\begin{Ejem}[\textbf{Proceso Poisson}]

Suponga que se tienen tiempos de inter-renovaci\'on \textit{i.i.d.} del proceso de renovaci\'on $N\left(t\right)$ tienen distribuci\'on exponencial $F\left(t\right)=q-e^{-\lambda t}$ con tasa $\lambda$. Entonces $N\left(t\right)$ es un proceso Poisson con tasa $\lambda$.

\end{Ejem}


\begin{Note}
Si el primer tiempo de renovaci\'on $\xi_{1}$ no tiene la misma distribuci\'on que el resto de las $\xi_{n}$, para $n\geq2$, a $N\left(t\right)$ se le llama Proceso de Renovaci\'on retardado, donde si $\xi$ tiene distribuci\'on $G$, entonces el tiempo $T_{n}$ de la $n$-\'esima renovaci\'on tiene distribuci\'on $G\star F^{\left(n-1\right)\star}\left(t\right)$
\end{Note}


\begin{Teo}
Para una constante $\mu\leq\infty$ ( o variable aleatoria), las siguientes expresiones son equivalentes:

\begin{eqnarray}
lim_{n\rightarrow\infty}n^{-1}T_{n}&=&\mu,\textrm{ c.s.}\\
lim_{t\rightarrow\infty}t^{-1}N\left(t\right)&=&1/\mu,\textrm{ c.s.}
\end{eqnarray}
\end{Teo}


Es decir, $T_{n}$ satisface la Ley Fuerte de los Grandes N\'umeros s\'i y s\'olo s\'i $N\left/t\right)$ la cumple.


\begin{Coro}[Ley Fuerte de los Grandes N\'umeros para Procesos de Renovaci\'on]
Si $N\left(t\right)$ es un proceso de renovaci\'on cuyos tiempos de inter-renovaci\'on tienen media $\mu\leq\infty$, entonces
\begin{eqnarray}
t^{-1}N\left(t\right)\rightarrow 1/\mu,\textrm{ c.s. cuando }t\rightarrow\infty.
\end{eqnarray}

\end{Coro}


Considerar el proceso estoc\'astico de valores reales $\left\{Z\left(t\right):t\geq0\right\}$ en el mismo espacio de probabilidad que $N\left(t\right)$

\begin{Def}
Para el proceso $\left\{Z\left(t\right):t\geq0\right\}$ se define la fluctuaci\'on m\'axima de $Z\left(t\right)$ en el intervalo $\left(T_{n-1},T_{n}\right]$:
\begin{eqnarray*}
M_{n}=\sup_{T_{n-1}<t\leq T_{n}}|Z\left(t\right)-Z\left(T_{n-1}\right)|
\end{eqnarray*}
\end{Def}

\begin{Teo}
Sup\'ongase que $n^{-1}T_{n}\rightarrow\mu$ c.s. cuando $n\rightarrow\infty$, donde $\mu\leq\infty$ es una constante o variable aleatoria. Sea $a$ una constante o variable aleatoria que puede ser infinita cuando $\mu$ es finita, y considere las expresiones l\'imite:
\begin{eqnarray}
lim_{n\rightarrow\infty}n^{-1}Z\left(T_{n}\right)&=&a,\textrm{ c.s.}\\
lim_{t\rightarrow\infty}t^{-1}Z\left(t\right)&=&a/\mu,\textrm{ c.s.}
\end{eqnarray}
La segunda expresi\'on implica la primera. Conversamente, la primera implica la segunda si el proceso $Z\left(t\right)$ es creciente, o si $lim_{n\rightarrow\infty}n^{-1}M_{n}=0$ c.s.
\end{Teo}

\begin{Coro}
Si $N\left(t\right)$ es un proceso de renovaci\'on, y $\left(Z\left(T_{n}\right)-Z\left(T_{n-1}\right),M_{n}\right)$, para $n\geq1$, son variables aleatorias independientes e id\'enticamente distribuidas con media finita, entonces,
\begin{eqnarray}
lim_{t\rightarrow\infty}t^{-1}Z\left(t\right)\rightarrow\frac{\esp\left[Z\left(T_{1}\right)-Z\left(T_{0}\right)\right]}{\esp\left[T_{1}\right]},\textrm{ c.s. cuando  }t\rightarrow\infty.
\end{eqnarray}
\end{Coro}


Sup\'ongase que $N\left(t\right)$ es un proceso de renovaci\'on con distribuci\'on $F$ con media finita $\mu$.

\begin{Def}
La funci\'on de renovaci\'on asociada con la distribuci\'on $F$, del proceso $N\left(t\right)$, es
\begin{eqnarray*}
U\left(t\right)=\sum_{n=1}^{\infty}F^{n\star}\left(t\right),\textrm{   }t\geq0,
\end{eqnarray*}
donde $F^{0\star}\left(t\right)=\indora\left(t\geq0\right)$.
\end{Def}


\begin{Prop}
Sup\'ongase que la distribuci\'on de inter-renovaci\'on $F$ tiene densidad $f$. Entonces $U\left(t\right)$ tambi\'en tiene densidad, para $t>0$, y es $U^{'}\left(t\right)=\sum_{n=0}^{\infty}f^{n\star}\left(t\right)$. Adem\'as
\begin{eqnarray*}
\prob\left\{N\left(t\right)>N\left(t-\right)\right\}=0\textrm{,   }t\geq0.
\end{eqnarray*}
\end{Prop}

\begin{Def}
La Transformada de Laplace-Stieljes de $F$ est\'a dada por

\begin{eqnarray*}
\hat{F}\left(\alpha\right)=\int_{\rea_{+}}e^{-\alpha t}dF\left(t\right)\textrm{,  }\alpha\geq0.
\end{eqnarray*}
\end{Def}

Entonces

\begin{eqnarray*}
\hat{U}\left(\alpha\right)=\sum_{n=0}^{\infty}\hat{F^{n\star}}\left(\alpha\right)=\sum_{n=0}^{\infty}\hat{F}\left(\alpha\right)^{n}=\frac{1}{1-\hat{F}\left(\alpha\right)}.
\end{eqnarray*}


\begin{Prop}
La Transformada de Laplace $\hat{U}\left(\alpha\right)$ y $\hat{F}\left(\alpha\right)$ determina una a la otra de manera \'unica por la relaci\'on $\hat{U}\left(\alpha\right)=\frac{1}{1-\hat{F}\left(\alpha\right)}$.
\end{Prop}


\begin{Note}
Un proceso de renovaci\'on $N\left(t\right)$ cuyos tiempos de inter-renovaci\'on tienen media finita, es un proceso Poisson con tasa $\lambda$ si y s\'olo s\'i $\esp\left[U\left(t\right)\right]=\lambda t$, para $t\geq0$.
\end{Note}


\begin{Teo}
Sea $N\left(t\right)$ un proceso puntual simple con puntos de localizaci\'on $T_{n}$ tal que $\eta\left(t\right)=\esp\left[N\left(\right)\right]$ es finita para cada $t$. Entonces para cualquier funci\'on $f:\rea_{+}\rightarrow\rea$,
\begin{eqnarray*}
\esp\left[\sum_{n=1}^{N\left(\right)}f\left(T_{n}\right)\right]=\int_{\left(0,t\right]}f\left(s\right)d\eta\left(s\right)\textrm{,  }t\geq0,
\end{eqnarray*}
suponiendo que la integral exista. Adem\'as si $X_{1},X_{2},\ldots$ son variables aleatorias definidas en el mismo espacio de probabilidad que el proceso $N\left(t\right)$ tal que $\esp\left[X_{n}|T_{n}=s\right]=f\left(s\right)$, independiente de $n$. Entonces
\begin{eqnarray*}
\esp\left[\sum_{n=1}^{N\left(t\right)}X_{n}\right]=\int_{\left(0,t\right]}f\left(s\right)d\eta\left(s\right)\textrm{,  }t\geq0,
\end{eqnarray*} 
suponiendo que la integral exista. 
\end{Teo}

\begin{Coro}[Identidad de Wald para Renovaciones]
Para el proceso de renovaci\'on $N\left(t\right)$,
\begin{eqnarray*}
\esp\left[T_{N\left(t\right)+1}\right]=\mu\esp\left[N\left(t\right)+1\right]\textrm{,  }t\geq0,
\end{eqnarray*}  
\end{Coro}


\begin{Def}
Sea $h\left(t\right)$ funci\'on de valores reales en $\rea$ acotada en intervalos finitos e igual a cero para $t<0$ La ecuaci\'on de renovaci\'on para $h\left(t\right)$ y la distribuci\'on $F$ es

\begin{eqnarray}%\label{Ec.Renovacion}
H\left(t\right)=h\left(t\right)+\int_{\left[0,t\right]}H\left(t-s\right)dF\left(s\right)\textrm{,    }t\geq0,
\end{eqnarray}
donde $H\left(t\right)$ es una funci\'on de valores reales. Esto es $H=h+F\star H$. Decimos que $H\left(t\right)$ es soluci\'on de esta ecuaci\'on si satisface la ecuaci\'on, y es acotada en intervalos finitos e iguales a cero para $t<0$.
\end{Def}

\begin{Prop}
La funci\'on $U\star h\left(t\right)$ es la \'unica soluci\'on de la ecuaci\'on de renovaci\'on (\ref{Ec.Renovacion}).
\end{Prop}

\begin{Teo}[Teorema Renovaci\'on Elemental]
\begin{eqnarray*}
t^{-1}U\left(t\right)\rightarrow 1/\mu\textrm{,    cuando }t\rightarrow\infty.
\end{eqnarray*}
\end{Teo}



Sup\'ongase que $N\left(t\right)$ es un proceso de renovaci\'on con distribuci\'on $F$ con media finita $\mu$.

\begin{Def}
La funci\'on de renovaci\'on asociada con la distribuci\'on $F$, del proceso $N\left(t\right)$, es
\begin{eqnarray*}
U\left(t\right)=\sum_{n=1}^{\infty}F^{n\star}\left(t\right),\textrm{   }t\geq0,
\end{eqnarray*}
donde $F^{0\star}\left(t\right)=\indora\left(t\geq0\right)$.
\end{Def}


\begin{Prop}
Sup\'ongase que la distribuci\'on de inter-renovaci\'on $F$ tiene densidad $f$. Entonces $U\left(t\right)$ tambi\'en tiene densidad, para $t>0$, y es $U^{'}\left(t\right)=\sum_{n=0}^{\infty}f^{n\star}\left(t\right)$. Adem\'as
\begin{eqnarray*}
\prob\left\{N\left(t\right)>N\left(t-\right)\right\}=0\textrm{,   }t\geq0.
\end{eqnarray*}
\end{Prop}

\begin{Def}
La Transformada de Laplace-Stieljes de $F$ est\'a dada por

\begin{eqnarray*}
\hat{F}\left(\alpha\right)=\int_{\rea_{+}}e^{-\alpha t}dF\left(t\right)\textrm{,  }\alpha\geq0.
\end{eqnarray*}
\end{Def}

Entonces

\begin{eqnarray*}
\hat{U}\left(\alpha\right)=\sum_{n=0}^{\infty}\hat{F^{n\star}}\left(\alpha\right)=\sum_{n=0}^{\infty}\hat{F}\left(\alpha\right)^{n}=\frac{1}{1-\hat{F}\left(\alpha\right)}.
\end{eqnarray*}


\begin{Prop}
La Transformada de Laplace $\hat{U}\left(\alpha\right)$ y $\hat{F}\left(\alpha\right)$ determina una a la otra de manera \'unica por la relaci\'on $\hat{U}\left(\alpha\right)=\frac{1}{1-\hat{F}\left(\alpha\right)}$.
\end{Prop}


\begin{Note}
Un proceso de renovaci\'on $N\left(t\right)$ cuyos tiempos de inter-renovaci\'on tienen media finita, es un proceso Poisson con tasa $\lambda$ si y s\'olo s\'i $\esp\left[U\left(t\right)\right]=\lambda t$, para $t\geq0$.
\end{Note}


\begin{Teo}
Sea $N\left(t\right)$ un proceso puntual simple con puntos de localizaci\'on $T_{n}$ tal que $\eta\left(t\right)=\esp\left[N\left(\right)\right]$ es finita para cada $t$. Entonces para cualquier funci\'on $f:\rea_{+}\rightarrow\rea$,
\begin{eqnarray*}
\esp\left[\sum_{n=1}^{N\left(\right)}f\left(T_{n}\right)\right]=\int_{\left(0,t\right]}f\left(s\right)d\eta\left(s\right)\textrm{,  }t\geq0,
\end{eqnarray*}
suponiendo que la integral exista. Adem\'as si $X_{1},X_{2},\ldots$ son variables aleatorias definidas en el mismo espacio de probabilidad que el proceso $N\left(t\right)$ tal que $\esp\left[X_{n}|T_{n}=s\right]=f\left(s\right)$, independiente de $n$. Entonces
\begin{eqnarray*}
\esp\left[\sum_{n=1}^{N\left(t\right)}X_{n}\right]=\int_{\left(0,t\right]}f\left(s\right)d\eta\left(s\right)\textrm{,  }t\geq0,
\end{eqnarray*} 
suponiendo que la integral exista. 
\end{Teo}

\begin{Coro}[Identidad de Wald para Renovaciones]
Para el proceso de renovaci\'on $N\left(t\right)$,
\begin{eqnarray*}
\esp\left[T_{N\left(t\right)+1}\right]=\mu\esp\left[N\left(t\right)+1\right]\textrm{,  }t\geq0,
\end{eqnarray*}  
\end{Coro}


\begin{Def}
Sea $h\left(t\right)$ funci\'on de valores reales en $\rea$ acotada en intervalos finitos e igual a cero para $t<0$ La ecuaci\'on de renovaci\'on para $h\left(t\right)$ y la distribuci\'on $F$ es

\begin{eqnarray}%\label{Ec.Renovacion}
H\left(t\right)=h\left(t\right)+\int_{\left[0,t\right]}H\left(t-s\right)dF\left(s\right)\textrm{,    }t\geq0,
\end{eqnarray}
donde $H\left(t\right)$ es una funci\'on de valores reales. Esto es $H=h+F\star H$. Decimos que $H\left(t\right)$ es soluci\'on de esta ecuaci\'on si satisface la ecuaci\'on, y es acotada en intervalos finitos e iguales a cero para $t<0$.
\end{Def}

\begin{Prop}
La funci\'on $U\star h\left(t\right)$ es la \'unica soluci\'on de la ecuaci\'on de renovaci\'on (\ref{Ec.Renovacion}).
\end{Prop}

\begin{Teo}[Teorema Renovaci\'on Elemental]
\begin{eqnarray*}
t^{-1}U\left(t\right)\rightarrow 1/\mu\textrm{,    cuando }t\rightarrow\infty.
\end{eqnarray*}
\end{Teo}


\begin{Note} Una funci\'on $h:\rea_{+}\rightarrow\rea$ es Directamente Riemann Integrable en los siguientes casos:
\begin{itemize}
\item[a)] $h\left(t\right)\geq0$ es decreciente y Riemann Integrable.
\item[b)] $h$ es continua excepto posiblemente en un conjunto de Lebesgue de medida 0, y $|h\left(t\right)|\leq b\left(t\right)$, donde $b$ es DRI.
\end{itemize}
\end{Note}

\begin{Teo}[Teorema Principal de Renovaci\'on]
Si $F$ es no aritm\'etica y $h\left(t\right)$ es Directamente Riemann Integrable (DRI), entonces

\begin{eqnarray*}
lim_{t\rightarrow\infty}U\star h=\frac{1}{\mu}\int_{\rea_{+}}h\left(s\right)ds.
\end{eqnarray*}
\end{Teo}

\begin{Prop}
Cualquier funci\'on $H\left(t\right)$ acotada en intervalos finitos y que es 0 para $t<0$ puede expresarse como
\begin{eqnarray*}
H\left(t\right)=U\star h\left(t\right)\textrm{,  donde }h\left(t\right)=H\left(t\right)-F\star H\left(t\right)
\end{eqnarray*}
\end{Prop}

\begin{Def}
Un proceso estoc\'astico $X\left(t\right)$ es crudamente regenerativo en un tiempo aleatorio positivo $T$ si
\begin{eqnarray*}
\esp\left[X\left(T+t\right)|T\right]=\esp\left[X\left(t\right)\right]\textrm{, para }t\geq0,\end{eqnarray*}
y con las esperanzas anteriores finitas.
\end{Def}

\begin{Prop}
Sup\'ongase que $X\left(t\right)$ es un proceso crudamente regenerativo en $T$, que tiene distribuci\'on $F$. Si $\esp\left[X\left(t\right)\right]$ es acotado en intervalos finitos, entonces
\begin{eqnarray*}
\esp\left[X\left(t\right)\right]=U\star h\left(t\right)\textrm{,  donde }h\left(t\right)=\esp\left[X\left(t\right)\indora\left(T>t\right)\right].
\end{eqnarray*}
\end{Prop}

\begin{Teo}[Regeneraci\'on Cruda]
Sup\'ongase que $X\left(t\right)$ es un proceso con valores positivo crudamente regenerativo en $T$, y def\'inase $M=\sup\left\{|X\left(t\right)|:t\leq T\right\}$. Si $T$ es no aritm\'etico y $M$ y $MT$ tienen media finita, entonces
\begin{eqnarray*}
lim_{t\rightarrow\infty}\esp\left[X\left(t\right)\right]=\frac{1}{\mu}\int_{\rea_{+}}h\left(s\right)ds,
\end{eqnarray*}
donde $h\left(t\right)=\esp\left[X\left(t\right)\indora\left(T>t\right)\right]$.
\end{Teo}


\begin{Note} Una funci\'on $h:\rea_{+}\rightarrow\rea$ es Directamente Riemann Integrable en los siguientes casos:
\begin{itemize}
\item[a)] $h\left(t\right)\geq0$ es decreciente y Riemann Integrable.
\item[b)] $h$ es continua excepto posiblemente en un conjunto de Lebesgue de medida 0, y $|h\left(t\right)|\leq b\left(t\right)$, donde $b$ es DRI.
\end{itemize}
\end{Note}

\begin{Teo}[Teorema Principal de Renovaci\'on]
Si $F$ es no aritm\'etica y $h\left(t\right)$ es Directamente Riemann Integrable (DRI), entonces

\begin{eqnarray*}
lim_{t\rightarrow\infty}U\star h=\frac{1}{\mu}\int_{\rea_{+}}h\left(s\right)ds.
\end{eqnarray*}
\end{Teo}

\begin{Prop}
Cualquier funci\'on $H\left(t\right)$ acotada en intervalos finitos y que es 0 para $t<0$ puede expresarse como
\begin{eqnarray*}
H\left(t\right)=U\star h\left(t\right)\textrm{,  donde }h\left(t\right)=H\left(t\right)-F\star H\left(t\right)
\end{eqnarray*}
\end{Prop}

\begin{Def}
Un proceso estoc\'astico $X\left(t\right)$ es crudamente regenerativo en un tiempo aleatorio positivo $T$ si
\begin{eqnarray*}
\esp\left[X\left(T+t\right)|T\right]=\esp\left[X\left(t\right)\right]\textrm{, para }t\geq0,\end{eqnarray*}
y con las esperanzas anteriores finitas.
\end{Def}

\begin{Prop}
Sup\'ongase que $X\left(t\right)$ es un proceso crudamente regenerativo en $T$, que tiene distribuci\'on $F$. Si $\esp\left[X\left(t\right)\right]$ es acotado en intervalos finitos, entonces
\begin{eqnarray*}
\esp\left[X\left(t\right)\right]=U\star h\left(t\right)\textrm{,  donde }h\left(t\right)=\esp\left[X\left(t\right)\indora\left(T>t\right)\right].
\end{eqnarray*}
\end{Prop}

\begin{Teo}[Regeneraci\'on Cruda]
Sup\'ongase que $X\left(t\right)$ es un proceso con valores positivo crudamente regenerativo en $T$, y def\'inase $M=\sup\left\{|X\left(t\right)|:t\leq T\right\}$. Si $T$ es no aritm\'etico y $M$ y $MT$ tienen media finita, entonces
\begin{eqnarray*}
lim_{t\rightarrow\infty}\esp\left[X\left(t\right)\right]=\frac{1}{\mu}\int_{\rea_{+}}h\left(s\right)ds,
\end{eqnarray*}
donde $h\left(t\right)=\esp\left[X\left(t\right)\indora\left(T>t\right)\right]$.
\end{Teo}

\begin{Def}
Para el proceso $\left\{\left(N\left(t\right),X\left(t\right)\right):t\geq0\right\}$, sus trayectoria muestrales en el intervalo de tiempo $\left[T_{n-1},T_{n}\right)$ est\'an descritas por
\begin{eqnarray*}
\zeta_{n}=\left(\xi_{n},\left\{X\left(T_{n-1}+t\right):0\leq t<\xi_{n}\right\}\right)
\end{eqnarray*}
Este $\zeta_{n}$ es el $n$-\'esimo segmento del proceso. El proceso es regenerativo sobre los tiempos $T_{n}$ si sus segmentos $\zeta_{n}$ son independientes e id\'enticamennte distribuidos.
\end{Def}


\begin{Note}
Si $\tilde{X}\left(t\right)$ con espacio de estados $\tilde{S}$ es regenerativo sobre $T_{n}$, entonces $X\left(t\right)=f\left(\tilde{X}\left(t\right)\right)$ tambi\'en es regenerativo sobre $T_{n}$, para cualquier funci\'on $f:\tilde{S}\rightarrow S$.
\end{Note}

\begin{Note}
Los procesos regenerativos son crudamente regenerativos, pero no al rev\'es.
\end{Note}


\begin{Note}
Un proceso estoc\'astico a tiempo continuo o discreto es regenerativo si existe un proceso de renovaci\'on  tal que los segmentos del proceso entre tiempos de renovaci\'on sucesivos son i.i.d., es decir, para $\left\{X\left(t\right):t\geq0\right\}$ proceso estoc\'astico a tiempo continuo con espacio de estados $S$, espacio m\'etrico.
\end{Note}

Para $\left\{X\left(t\right):t\geq0\right\}$ Proceso Estoc\'astico a tiempo continuo con estado de espacios $S$, que es un espacio m\'etrico, con trayectorias continuas por la derecha y con l\'imites por la izquierda c.s. Sea $N\left(t\right)$ un proceso de renovaci\'on en $\rea_{+}$ definido en el mismo espacio de probabilidad que $X\left(t\right)$, con tiempos de renovaci\'on $T$ y tiempos de inter-renovaci\'on $\xi_{n}=T_{n}-T_{n-1}$, con misma distribuci\'on $F$ de media finita $\mu$.



\begin{Def}
Para el proceso $\left\{\left(N\left(t\right),X\left(t\right)\right):t\geq0\right\}$, sus trayectoria muestrales en el intervalo de tiempo $\left[T_{n-1},T_{n}\right)$ est\'an descritas por
\begin{eqnarray*}
\zeta_{n}=\left(\xi_{n},\left\{X\left(T_{n-1}+t\right):0\leq t<\xi_{n}\right\}\right)
\end{eqnarray*}
Este $\zeta_{n}$ es el $n$-\'esimo segmento del proceso. El proceso es regenerativo sobre los tiempos $T_{n}$ si sus segmentos $\zeta_{n}$ son independientes e id\'enticamennte distribuidos.
\end{Def}

\begin{Note}
Un proceso regenerativo con media de la longitud de ciclo finita es llamado positivo recurrente.
\end{Note}

\begin{Teo}[Procesos Regenerativos]
Suponga que el proceso
\end{Teo}


\begin{Def}[Renewal Process Trinity]
Para un proceso de renovaci\'on $N\left(t\right)$, los siguientes procesos proveen de informaci\'on sobre los tiempos de renovaci\'on.
\begin{itemize}
\item $A\left(t\right)=t-T_{N\left(t\right)}$, el tiempo de recurrencia hacia atr\'as al tiempo $t$, que es el tiempo desde la \'ultima renovaci\'on para $t$.

\item $B\left(t\right)=T_{N\left(t\right)+1}-t$, el tiempo de recurrencia hacia adelante al tiempo $t$, residual del tiempo de renovaci\'on, que es el tiempo para la pr\'oxima renovaci\'on despu\'es de $t$.

\item $L\left(t\right)=\xi_{N\left(t\right)+1}=A\left(t\right)+B\left(t\right)$, la longitud del intervalo de renovaci\'on que contiene a $t$.
\end{itemize}
\end{Def}

\begin{Note}
El proceso tridimensional $\left(A\left(t\right),B\left(t\right),L\left(t\right)\right)$ es regenerativo sobre $T_{n}$, y por ende cada proceso lo es. Cada proceso $A\left(t\right)$ y $B\left(t\right)$ son procesos de MArkov a tiempo continuo con trayectorias continuas por partes en el espacio de estados $\rea_{+}$. Una expresi\'on conveniente para su distribuci\'on conjunta es, para $0\leq x<t,y\geq0$
\begin{equation}\label{NoRenovacion}
P\left\{A\left(t\right)>x,B\left(t\right)>y\right\}=
P\left\{N\left(t+y\right)-N\left((t-x)\right)=0\right\}
\end{equation}
\end{Note}

\begin{Ejem}[Tiempos de recurrencia Poisson]
Si $N\left(t\right)$ es un proceso Poisson con tasa $\lambda$, entonces de la expresi\'on (\ref{NoRenovacion}) se tiene que

\begin{eqnarray*}
\begin{array}{lc}
P\left\{A\left(t\right)>x,B\left(t\right)>y\right\}=e^{-\lambda\left(x+y\right)},&0\leq x<t,y\geq0,
\end{array}
\end{eqnarray*}
que es la probabilidad Poisson de no renovaciones en un intervalo de longitud $x+y$.

\end{Ejem}

\begin{Note}
Una cadena de Markov erg\'odica tiene la propiedad de ser estacionaria si la distribuci\'on de su estado al tiempo $0$ es su distribuci\'on estacionaria.
\end{Note}


\begin{Def}
Un proceso estoc\'astico a tiempo continuo $\left\{X\left(t\right):t\geq0\right\}$ en un espacio general es estacionario si sus distribuciones finito dimensionales son invariantes bajo cualquier  traslado: para cada $0\leq s_{1}<s_{2}<\cdots<s_{k}$ y $t\geq0$,
\begin{eqnarray*}
\left(X\left(s_{1}+t\right),\ldots,X\left(s_{k}+t\right)\right)=_{d}\left(X\left(s_{1}\right),\ldots,X\left(s_{k}\right)\right).
\end{eqnarray*}
\end{Def}

\begin{Note}
Un proceso de Markov es estacionario si $X\left(t\right)=_{d}X\left(0\right)$, $t\geq0$.
\end{Note}

Considerese el proceso $N\left(t\right)=\sum_{n}\indora\left(\tau_{n}\leq t\right)$ en $\rea_{+}$, con puntos $0<\tau_{1}<\tau_{2}<\cdots$.

\begin{Prop}
Si $N$ es un proceso puntual estacionario y $\esp\left[N\left(1\right)\right]<\infty$, entonces $\esp\left[N\left(t\right)\right]=t\esp\left[N\left(1\right)\right]$, $t\geq0$

\end{Prop}

\begin{Teo}
Los siguientes enunciados son equivalentes
\begin{itemize}
\item[i)] El proceso retardado de renovaci\'on $N$ es estacionario.

\item[ii)] EL proceso de tiempos de recurrencia hacia adelante $B\left(t\right)$ es estacionario.


\item[iii)] $\esp\left[N\left(t\right)\right]=t/\mu$,


\item[iv)] $G\left(t\right)=F_{e}\left(t\right)=\frac{1}{\mu}\int_{0}^{t}\left[1-F\left(s\right)\right]ds$
\end{itemize}
Cuando estos enunciados son ciertos, $P\left\{B\left(t\right)\leq x\right\}=F_{e}\left(x\right)$, para $t,x\geq0$.

\end{Teo}

\begin{Note}
Una consecuencia del teorema anterior es que el Proceso Poisson es el \'unico proceso sin retardo que es estacionario.
\end{Note}

\begin{Coro}
El proceso de renovaci\'on $N\left(t\right)$ sin retardo, y cuyos tiempos de inter renonaci\'on tienen media finita, es estacionario si y s\'olo si es un proceso Poisson.

\end{Coro}


%________________________________________________________________________
%\subsection{Procesos Regenerativos}
%________________________________________________________________________



\begin{Note}
Si $\tilde{X}\left(t\right)$ con espacio de estados $\tilde{S}$ es regenerativo sobre $T_{n}$, entonces $X\left(t\right)=f\left(\tilde{X}\left(t\right)\right)$ tambi\'en es regenerativo sobre $T_{n}$, para cualquier funci\'on $f:\tilde{S}\rightarrow S$.
\end{Note}

\begin{Note}
Los procesos regenerativos son crudamente regenerativos, pero no al rev\'es.
\end{Note}
%\subsection*{Procesos Regenerativos: Sigman\cite{Sigman1}}
\begin{Def}[Definici\'on Cl\'asica]
Un proceso estoc\'astico $X=\left\{X\left(t\right):t\geq0\right\}$ es llamado regenerativo is existe una variable aleatoria $R_{1}>0$ tal que
\begin{itemize}
\item[i)] $\left\{X\left(t+R_{1}\right):t\geq0\right\}$ es independiente de $\left\{\left\{X\left(t\right):t<R_{1}\right\},\right\}$
\item[ii)] $\left\{X\left(t+R_{1}\right):t\geq0\right\}$ es estoc\'asticamente equivalente a $\left\{X\left(t\right):t>0\right\}$
\end{itemize}

Llamamos a $R_{1}$ tiempo de regeneraci\'on, y decimos que $X$ se regenera en este punto.
\end{Def}

$\left\{X\left(t+R_{1}\right)\right\}$ es regenerativo con tiempo de regeneraci\'on $R_{2}$, independiente de $R_{1}$ pero con la misma distribuci\'on que $R_{1}$. Procediendo de esta manera se obtiene una secuencia de variables aleatorias independientes e id\'enticamente distribuidas $\left\{R_{n}\right\}$ llamados longitudes de ciclo. Si definimos a $Z_{k}\equiv R_{1}+R_{2}+\cdots+R_{k}$, se tiene un proceso de renovaci\'on llamado proceso de renovaci\'on encajado para $X$.




\begin{Def}
Para $x$ fijo y para cada $t\geq0$, sea $I_{x}\left(t\right)=1$ si $X\left(t\right)\leq x$,  $I_{x}\left(t\right)=0$ en caso contrario, y def\'inanse los tiempos promedio
\begin{eqnarray*}
\overline{X}&=&lim_{t\rightarrow\infty}\frac{1}{t}\int_{0}^{\infty}X\left(u\right)du\\
\prob\left(X_{\infty}\leq x\right)&=&lim_{t\rightarrow\infty}\frac{1}{t}\int_{0}^{\infty}I_{x}\left(u\right)du,
\end{eqnarray*}
cuando estos l\'imites existan.
\end{Def}

Como consecuencia del teorema de Renovaci\'on-Recompensa, se tiene que el primer l\'imite  existe y es igual a la constante
\begin{eqnarray*}
\overline{X}&=&\frac{\esp\left[\int_{0}^{R_{1}}X\left(t\right)dt\right]}{\esp\left[R_{1}\right]},
\end{eqnarray*}
suponiendo que ambas esperanzas son finitas.

\begin{Note}
\begin{itemize}
\item[a)] Si el proceso regenerativo $X$ es positivo recurrente y tiene trayectorias muestrales no negativas, entonces la ecuaci\'on anterior es v\'alida.
\item[b)] Si $X$ es positivo recurrente regenerativo, podemos construir una \'unica versi\'on estacionaria de este proceso, $X_{e}=\left\{X_{e}\left(t\right)\right\}$, donde $X_{e}$ es un proceso estoc\'astico regenerativo y estrictamente estacionario, con distribuci\'on marginal distribuida como $X_{\infty}$
\end{itemize}
\end{Note}

%________________________________________________________________________
%\subsection{Procesos Regenerativos}
%________________________________________________________________________

Para $\left\{X\left(t\right):t\geq0\right\}$ Proceso Estoc\'astico a tiempo continuo con estado de espacios $S$, que es un espacio m\'etrico, con trayectorias continuas por la derecha y con l\'imites por la izquierda c.s. Sea $N\left(t\right)$ un proceso de renovaci\'on en $\rea_{+}$ definido en el mismo espacio de probabilidad que $X\left(t\right)$, con tiempos de renovaci\'on $T$ y tiempos de inter-renovaci\'on $\xi_{n}=T_{n}-T_{n-1}$, con misma distribuci\'on $F$ de media finita $\mu$.



\begin{Def}
Para el proceso $\left\{\left(N\left(t\right),X\left(t\right)\right):t\geq0\right\}$, sus trayectoria muestrales en el intervalo de tiempo $\left[T_{n-1},T_{n}\right)$ est\'an descritas por
\begin{eqnarray*}
\zeta_{n}=\left(\xi_{n},\left\{X\left(T_{n-1}+t\right):0\leq t<\xi_{n}\right\}\right)
\end{eqnarray*}
Este $\zeta_{n}$ es el $n$-\'esimo segmento del proceso. El proceso es regenerativo sobre los tiempos $T_{n}$ si sus segmentos $\zeta_{n}$ son independientes e id\'enticamennte distribuidos.
\end{Def}


\begin{Note}
Si $\tilde{X}\left(t\right)$ con espacio de estados $\tilde{S}$ es regenerativo sobre $T_{n}$, entonces $X\left(t\right)=f\left(\tilde{X}\left(t\right)\right)$ tambi\'en es regenerativo sobre $T_{n}$, para cualquier funci\'on $f:\tilde{S}\rightarrow S$.
\end{Note}

\begin{Note}
Los procesos regenerativos son crudamente regenerativos, pero no al rev\'es.
\end{Note}

\begin{Def}[Definici\'on Cl\'asica]
Un proceso estoc\'astico $X=\left\{X\left(t\right):t\geq0\right\}$ es llamado regenerativo is existe una variable aleatoria $R_{1}>0$ tal que
\begin{itemize}
\item[i)] $\left\{X\left(t+R_{1}\right):t\geq0\right\}$ es independiente de $\left\{\left\{X\left(t\right):t<R_{1}\right\},\right\}$
\item[ii)] $\left\{X\left(t+R_{1}\right):t\geq0\right\}$ es estoc\'asticamente equivalente a $\left\{X\left(t\right):t>0\right\}$
\end{itemize}

Llamamos a $R_{1}$ tiempo de regeneraci\'on, y decimos que $X$ se regenera en este punto.
\end{Def}

$\left\{X\left(t+R_{1}\right)\right\}$ es regenerativo con tiempo de regeneraci\'on $R_{2}$, independiente de $R_{1}$ pero con la misma distribuci\'on que $R_{1}$. Procediendo de esta manera se obtiene una secuencia de variables aleatorias independientes e id\'enticamente distribuidas $\left\{R_{n}\right\}$ llamados longitudes de ciclo. Si definimos a $Z_{k}\equiv R_{1}+R_{2}+\cdots+R_{k}$, se tiene un proceso de renovaci\'on llamado proceso de renovaci\'on encajado para $X$.

\begin{Note}
Un proceso regenerativo con media de la longitud de ciclo finita es llamado positivo recurrente.
\end{Note}


\begin{Def}
Para $x$ fijo y para cada $t\geq0$, sea $I_{x}\left(t\right)=1$ si $X\left(t\right)\leq x$,  $I_{x}\left(t\right)=0$ en caso contrario, y def\'inanse los tiempos promedio
\begin{eqnarray*}
\overline{X}&=&lim_{t\rightarrow\infty}\frac{1}{t}\int_{0}^{\infty}X\left(u\right)du\\
\prob\left(X_{\infty}\leq x\right)&=&lim_{t\rightarrow\infty}\frac{1}{t}\int_{0}^{\infty}I_{x}\left(u\right)du,
\end{eqnarray*}
cuando estos l\'imites existan.
\end{Def}

Como consecuencia del teorema de Renovaci\'on-Recompensa, se tiene que el primer l\'imite  existe y es igual a la constante
\begin{eqnarray*}
\overline{X}&=&\frac{\esp\left[\int_{0}^{R_{1}}X\left(t\right)dt\right]}{\esp\left[R_{1}\right]},
\end{eqnarray*}
suponiendo que ambas esperanzas son finitas.

\begin{Note}
\begin{itemize}
\item[a)] Si el proceso regenerativo $X$ es positivo recurrente y tiene trayectorias muestrales no negativas, entonces la ecuaci\'on anterior es v\'alida.
\item[b)] Si $X$ es positivo recurrente regenerativo, podemos construir una \'unica versi\'on estacionaria de este proceso, $X_{e}=\left\{X_{e}\left(t\right)\right\}$, donde $X_{e}$ es un proceso estoc\'astico regenerativo y estrictamente estacionario, con distribuci\'on marginal distribuida como $X_{\infty}$
\end{itemize}
\end{Note}

%__________________________________________________________________________________________
%\subsection{Procesos Regenerativos Estacionarios - Stidham \cite{Stidham}}
%__________________________________________________________________________________________


Un proceso estoc\'astico a tiempo continuo $\left\{V\left(t\right),t\geq0\right\}$ es un proceso regenerativo si existe una sucesi\'on de variables aleatorias independientes e id\'enticamente distribuidas $\left\{X_{1},X_{2},\ldots\right\}$, sucesi\'on de renovaci\'on, tal que para cualquier conjunto de Borel $A$, 

\begin{eqnarray*}
\prob\left\{V\left(t\right)\in A|X_{1}+X_{2}+\cdots+X_{R\left(t\right)}=s,\left\{V\left(\tau\right),\tau<s\right\}\right\}=\prob\left\{V\left(t-s\right)\in A|X_{1}>t-s\right\},
\end{eqnarray*}
para todo $0\leq s\leq t$, donde $R\left(t\right)=\max\left\{X_{1}+X_{2}+\cdots+X_{j}\leq t\right\}=$n\'umero de renovaciones ({\emph{puntos de regeneraci\'on}}) que ocurren en $\left[0,t\right]$. El intervalo $\left[0,X_{1}\right)$ es llamado {\emph{primer ciclo de regeneraci\'on}} de $\left\{V\left(t \right),t\geq0\right\}$, $\left[X_{1},X_{1}+X_{2}\right)$ el {\emph{segundo ciclo de regeneraci\'on}}, y as\'i sucesivamente.

Sea $X=X_{1}$ y sea $F$ la funci\'on de distrbuci\'on de $X$


\begin{Def}
Se define el proceso estacionario, $\left\{V^{*}\left(t\right),t\geq0\right\}$, para $\left\{V\left(t\right),t\geq0\right\}$ por

\begin{eqnarray*}
\prob\left\{V\left(t\right)\in A\right\}=\frac{1}{\esp\left[X\right]}\int_{0}^{\infty}\prob\left\{V\left(t+x\right)\in A|X>x\right\}\left(1-F\left(x\right)\right)dx,
\end{eqnarray*} 
para todo $t\geq0$ y todo conjunto de Borel $A$.
\end{Def}

\begin{Def}
Una distribuci\'on se dice que es {\emph{aritm\'etica}} si todos sus puntos de incremento son m\'ultiplos de la forma $0,\lambda, 2\lambda,\ldots$ para alguna $\lambda>0$ entera.
\end{Def}


\begin{Def}
Una modificaci\'on medible de un proceso $\left\{V\left(t\right),t\geq0\right\}$, es una versi\'on de este, $\left\{V\left(t,w\right)\right\}$ conjuntamente medible para $t\geq0$ y para $w\in S$, $S$ espacio de estados para $\left\{V\left(t\right),t\geq0\right\}$.
\end{Def}

\begin{Teo}
Sea $\left\{V\left(t\right),t\geq\right\}$ un proceso regenerativo no negativo con modificaci\'on medible. Sea $\esp\left[X\right]<\infty$. Entonces el proceso estacionario dado por la ecuaci\'on anterior est\'a bien definido y tiene funci\'on de distribuci\'on independiente de $t$, adem\'as
\begin{itemize}
\item[i)] \begin{eqnarray*}
\esp\left[V^{*}\left(0\right)\right]&=&\frac{\esp\left[\int_{0}^{X}V\left(s\right)ds\right]}{\esp\left[X\right]}\end{eqnarray*}
\item[ii)] Si $\esp\left[V^{*}\left(0\right)\right]<\infty$, equivalentemente, si $\esp\left[\int_{0}^{X}V\left(s\right)ds\right]<\infty$,entonces
\begin{eqnarray*}
\frac{\int_{0}^{t}V\left(s\right)ds}{t}\rightarrow\frac{\esp\left[\int_{0}^{X}V\left(s\right)ds\right]}{\esp\left[X\right]}
\end{eqnarray*}
con probabilidad 1 y en media, cuando $t\rightarrow\infty$.
\end{itemize}
\end{Teo}
%
%___________________________________________________________________________________________
%\vspace{5.5cm}
%\chapter{Cadenas de Markov estacionarias}
%\vspace{-1.0cm}


%__________________________________________________________________________________________
%\subsection{Procesos Regenerativos Estacionarios - Stidham \cite{Stidham}}
%__________________________________________________________________________________________


Un proceso estoc\'astico a tiempo continuo $\left\{V\left(t\right),t\geq0\right\}$ es un proceso regenerativo si existe una sucesi\'on de variables aleatorias independientes e id\'enticamente distribuidas $\left\{X_{1},X_{2},\ldots\right\}$, sucesi\'on de renovaci\'on, tal que para cualquier conjunto de Borel $A$, 

\begin{eqnarray*}
\prob\left\{V\left(t\right)\in A|X_{1}+X_{2}+\cdots+X_{R\left(t\right)}=s,\left\{V\left(\tau\right),\tau<s\right\}\right\}=\prob\left\{V\left(t-s\right)\in A|X_{1}>t-s\right\},
\end{eqnarray*}
para todo $0\leq s\leq t$, donde $R\left(t\right)=\max\left\{X_{1}+X_{2}+\cdots+X_{j}\leq t\right\}=$n\'umero de renovaciones ({\emph{puntos de regeneraci\'on}}) que ocurren en $\left[0,t\right]$. El intervalo $\left[0,X_{1}\right)$ es llamado {\emph{primer ciclo de regeneraci\'on}} de $\left\{V\left(t \right),t\geq0\right\}$, $\left[X_{1},X_{1}+X_{2}\right)$ el {\emph{segundo ciclo de regeneraci\'on}}, y as\'i sucesivamente.

Sea $X=X_{1}$ y sea $F$ la funci\'on de distrbuci\'on de $X$


\begin{Def}
Se define el proceso estacionario, $\left\{V^{*}\left(t\right),t\geq0\right\}$, para $\left\{V\left(t\right),t\geq0\right\}$ por

\begin{eqnarray*}
\prob\left\{V\left(t\right)\in A\right\}=\frac{1}{\esp\left[X\right]}\int_{0}^{\infty}\prob\left\{V\left(t+x\right)\in A|X>x\right\}\left(1-F\left(x\right)\right)dx,
\end{eqnarray*} 
para todo $t\geq0$ y todo conjunto de Borel $A$.
\end{Def}

\begin{Def}
Una distribuci\'on se dice que es {\emph{aritm\'etica}} si todos sus puntos de incremento son m\'ultiplos de la forma $0,\lambda, 2\lambda,\ldots$ para alguna $\lambda>0$ entera.
\end{Def}


\begin{Def}
Una modificaci\'on medible de un proceso $\left\{V\left(t\right),t\geq0\right\}$, es una versi\'on de este, $\left\{V\left(t,w\right)\right\}$ conjuntamente medible para $t\geq0$ y para $w\in S$, $S$ espacio de estados para $\left\{V\left(t\right),t\geq0\right\}$.
\end{Def}

\begin{Teo}
Sea $\left\{V\left(t\right),t\geq\right\}$ un proceso regenerativo no negativo con modificaci\'on medible. Sea $\esp\left[X\right]<\infty$. Entonces el proceso estacionario dado por la ecuaci\'on anterior est\'a bien definido y tiene funci\'on de distribuci\'on independiente de $t$, adem\'as
\begin{itemize}
\item[i)] \begin{eqnarray*}
\esp\left[V^{*}\left(0\right)\right]&=&\frac{\esp\left[\int_{0}^{X}V\left(s\right)ds\right]}{\esp\left[X\right]}\end{eqnarray*}
\item[ii)] Si $\esp\left[V^{*}\left(0\right)\right]<\infty$, equivalentemente, si $\esp\left[\int_{0}^{X}V\left(s\right)ds\right]<\infty$,entonces
\begin{eqnarray*}
\frac{\int_{0}^{t}V\left(s\right)ds}{t}\rightarrow\frac{\esp\left[\int_{0}^{X}V\left(s\right)ds\right]}{\esp\left[X\right]}
\end{eqnarray*}
con probabilidad 1 y en media, cuando $t\rightarrow\infty$.
\end{itemize}
\end{Teo}

Para $\left\{X\left(t\right):t\geq0\right\}$ Proceso Estoc\'astico a tiempo continuo con estado de espacios $S$, que es un espacio m\'etrico, con trayectorias continuas por la derecha y con l\'imites por la izquierda c.s. Sea $N\left(t\right)$ un proceso de renovaci\'on en $\rea_{+}$ definido en el mismo espacio de probabilidad que $X\left(t\right)$, con tiempos de renovaci\'on $T$ y tiempos de inter-renovaci\'on $\xi_{n}=T_{n}-T_{n-1}$, con misma distribuci\'on $F$ de media finita $\mu$.


%______________________________________________________________________
%\subsection{Ejemplos, Notas importantes}


Sean $T_{1},T_{2},\ldots$ los puntos donde las longitudes de las colas de la red de sistemas de visitas c\'iclicas son cero simult\'aneamente, cuando la cola $Q_{j}$ es visitada por el servidor para dar servicio, es decir, $L_{1}\left(T_{i}\right)=0,L_{2}\left(T_{i}\right)=0,\hat{L}_{1}\left(T_{i}\right)=0$ y $\hat{L}_{2}\left(T_{i}\right)=0$, a estos puntos se les denominar\'a puntos regenerativos. Sea la funci\'on generadora de momentos para $L_{i}$, el n\'umero de usuarios en la cola $Q_{i}\left(z\right)$ en cualquier momento, est\'a dada por el tiempo promedio de $z^{L_{i}\left(t\right)}$ sobre el ciclo regenerativo definido anteriormente:

\begin{eqnarray*}
Q_{i}\left(z\right)&=&\esp\left[z^{L_{i}\left(t\right)}\right]=\frac{\esp\left[\sum_{m=1}^{M_{i}}\sum_{t=\tau_{i}\left(m\right)}^{\tau_{i}\left(m+1\right)-1}z^{L_{i}\left(t\right)}\right]}{\esp\left[\sum_{m=1}^{M_{i}}\tau_{i}\left(m+1\right)-\tau_{i}\left(m\right)\right]}
\end{eqnarray*}

$M_{i}$ es un tiempo de paro en el proceso regenerativo con $\esp\left[M_{i}\right]<\infty$\footnote{En Stidham\cite{Stidham} y Heyman  se muestra que una condici\'on suficiente para que el proceso regenerativo 
estacionario sea un procesoo estacionario es que el valor esperado del tiempo del ciclo regenerativo sea finito, es decir: $\esp\left[\sum_{m=1}^{M_{i}}C_{i}^{(m)}\right]<\infty$, como cada $C_{i}^{(m)}$ contiene intervalos de r\'eplica positivos, se tiene que $\esp\left[M_{i}\right]<\infty$, adem\'as, como $M_{i}>0$, se tiene que la condici\'on anterior es equivalente a tener que $\esp\left[C_{i}\right]<\infty$,
por lo tanto una condici\'on suficiente para la existencia del proceso regenerativo est\'a dada por $\sum_{k=1}^{N}\mu_{k}<1.$}, se sigue del lema de Wald que:


\begin{eqnarray*}
\esp\left[\sum_{m=1}^{M_{i}}\sum_{t=\tau_{i}\left(m\right)}^{\tau_{i}\left(m+1\right)-1}z^{L_{i}\left(t\right)}\right]&=&\esp\left[M_{i}\right]\esp\left[\sum_{t=\tau_{i}\left(m\right)}^{\tau_{i}\left(m+1\right)-1}z^{L_{i}\left(t\right)}\right]\\
\esp\left[\sum_{m=1}^{M_{i}}\tau_{i}\left(m+1\right)-\tau_{i}\left(m\right)\right]&=&\esp\left[M_{i}\right]\esp\left[\tau_{i}\left(m+1\right)-\tau_{i}\left(m\right)\right]
\end{eqnarray*}

por tanto se tiene que


\begin{eqnarray*}
Q_{i}\left(z\right)&=&\frac{\esp\left[\sum_{t=\tau_{i}\left(m\right)}^{\tau_{i}\left(m+1\right)-1}z^{L_{i}\left(t\right)}\right]}{\esp\left[\tau_{i}\left(m+1\right)-\tau_{i}\left(m\right)\right]}
\end{eqnarray*}

observar que el denominador es simplemente la duraci\'on promedio del tiempo del ciclo.


Haciendo las siguientes sustituciones en la ecuaci\'on (\ref{Corolario2}): $n\rightarrow t-\tau_{i}\left(m\right)$, $T \rightarrow \overline{\tau}_{i}\left(m\right)-\tau_{i}\left(m\right)$, $L_{n}\rightarrow L_{i}\left(t\right)$ y $F\left(z\right)=\esp\left[z^{L_{0}}\right]\rightarrow F_{i}\left(z\right)=\esp\left[z^{L_{i}\tau_{i}\left(m\right)}\right]$, se puede ver que

\begin{eqnarray}\label{Eq.Arribos.Primera}
\esp\left[\sum_{n=0}^{T-1}z^{L_{n}}\right]=
\esp\left[\sum_{t=\tau_{i}\left(m\right)}^{\overline{\tau}_{i}\left(m\right)-1}z^{L_{i}\left(t\right)}\right]
=z\frac{F_{i}\left(z\right)-1}{z-P_{i}\left(z\right)}
\end{eqnarray}

Por otra parte durante el tiempo de intervisita para la cola $i$, $L_{i}\left(t\right)$ solamente se incrementa de manera que el incremento por intervalo de tiempo est\'a dado por la funci\'on generadora de probabilidades de $P_{i}\left(z\right)$, por tanto la suma sobre el tiempo de intervisita puede evaluarse como:

\begin{eqnarray*}
\esp\left[\sum_{t=\tau_{i}\left(m\right)}^{\tau_{i}\left(m+1\right)-1}z^{L_{i}\left(t\right)}\right]&=&\esp\left[\sum_{t=\tau_{i}\left(m\right)}^{\tau_{i}\left(m+1\right)-1}\left\{P_{i}\left(z\right)\right\}^{t-\overline{\tau}_{i}\left(m\right)}\right]=\frac{1-\esp\left[\left\{P_{i}\left(z\right)\right\}^{\tau_{i}\left(m+1\right)-\overline{\tau}_{i}\left(m\right)}\right]}{1-P_{i}\left(z\right)}\\
&=&\frac{1-I_{i}\left[P_{i}\left(z\right)\right]}{1-P_{i}\left(z\right)}
\end{eqnarray*}
por tanto

\begin{eqnarray*}
\esp\left[\sum_{t=\tau_{i}\left(m\right)}^{\tau_{i}\left(m+1\right)-1}z^{L_{i}\left(t\right)}\right]&=&
\frac{1-F_{i}\left(z\right)}{1-P_{i}\left(z\right)}
\end{eqnarray*}

Por lo tanto

\begin{eqnarray*}
Q_{i}\left(z\right)&=&\frac{\esp\left[\sum_{t=\tau_{i}\left(m\right)}^{\tau_{i}
\left(m+1\right)-1}z^{L_{i}\left(t\right)}\right]}{\esp\left[\tau_{i}\left(m+1\right)-\tau_{i}\left(m\right)\right]}\\
&=&\frac{1}{\esp\left[\tau_{i}\left(m+1\right)-\tau_{i}\left(m\right)\right]}
\left\{
\esp\left[\sum_{t=\tau_{i}\left(m\right)}^{\overline{\tau}_{i}\left(m\right)-1}
z^{L_{i}\left(t\right)}\right]
+\esp\left[\sum_{t=\overline{\tau}_{i}\left(m\right)}^{\tau_{i}\left(m+1\right)-1}
z^{L_{i}\left(t\right)}\right]\right\}\\
&=&\frac{1}{\esp\left[\tau_{i}\left(m+1\right)-\tau_{i}\left(m\right)\right]}
\left\{
z\frac{F_{i}\left(z\right)-1}{z-P_{i}\left(z\right)}+\frac{1-F_{i}\left(z\right)}
{1-P_{i}\left(z\right)}
\right\}
\end{eqnarray*}

es decir

\begin{equation}
Q_{i}\left(z\right)=\frac{1}{\esp\left[C_{i}\right]}\cdot\frac{1-F_{i}\left(z\right)}{P_{i}\left(z\right)-z}\cdot\frac{\left(1-z\right)P_{i}\left(z\right)}{1-P_{i}\left(z\right)}
\end{equation}

\begin{Teo}
Dada una Red de Sistemas de Visitas C\'iclicas (RSVC), conformada por dos Sistemas de Visitas C\'iclicas (SVC), donde cada uno de ellos consta de dos colas tipo $M/M/1$. Los dos sistemas est\'an comunicados entre s\'i por medio de la transferencia de usuarios entre las colas $Q_{1}\leftrightarrow Q_{3}$ y $Q_{2}\leftrightarrow Q_{4}$. Se definen los eventos para los procesos de arribos al tiempo $t$, $A_{j}\left(t\right)=\left\{0 \textrm{ arribos en }Q_{j}\textrm{ al tiempo }t\right\}$ para alg\'un tiempo $t\geq0$ y $Q_{j}$ la cola $j$-\'esima en la RSVC, para $j=1,2,3,4$.  Existe un intervalo $I\neq\emptyset$ tal que para $T^{*}\in I$, tal que $\prob\left\{A_{1}\left(T^{*}\right),A_{2}\left(Tt^{*}\right),
A_{3}\left(T^{*}\right),A_{4}\left(T^{*}\right)|T^{*}\in I\right\}>0$.
\end{Teo}

\begin{proof}
Sin p\'erdida de generalidad podemos considerar como base del an\'alisis a la cola $Q_{1}$ del primer sistema que conforma la RSVC.

Sea $n>0$, ciclo en el primer sistema en el que se sabe que $L_{j}\left(\overline{\tau}_{1}\left(n\right)\right)=0$, pues la pol\'itica de servicio con que atienden los servidores es la exhaustiva. Como es sabido, para trasladarse a la siguiente cola, el servidor incurre en un tiempo de traslado $r_{1}\left(n\right)>0$, entonces tenemos que $\tau_{2}\left(n\right)=\overline{\tau}_{1}\left(n\right)+r_{1}\left(n\right)$.


Definamos el intervalo $I_{1}\equiv\left[\overline{\tau}_{1}\left(n\right),\tau_{2}\left(n\right)\right]$ de longitud $\xi_{1}=r_{1}\left(n\right)$. Dado que los tiempos entre arribo son exponenciales con tasa $\tilde{\mu}_{1}=\mu_{1}+\hat{\mu}_{1}$ ($\mu_{1}$ son los arribos a $Q_{1}$ por primera vez al sistema, mientras que $\hat{\mu}_{1}$ son los arribos de traslado procedentes de $Q_{3}$) se tiene que la probabilidad del evento $A_{1}\left(t\right)$ est\'a dada por 

\begin{equation}
\prob\left\{A_{1}\left(t\right)|t\in I_{1}\left(n\right)\right\}=e^{-\tilde{\mu}_{1}\xi_{1}\left(n\right)}.
\end{equation} 

Por otra parte, para la cola $Q_{2}$, el tiempo $\overline{\tau}_{2}\left(n-1\right)$ es tal que $L_{2}\left(\overline{\tau}_{2}\left(n-1\right)\right)=0$, es decir, es el tiempo en que la cola queda totalmente vac\'ia en el ciclo anterior a $n$. Entonces tenemos un sgundo intervalo $I_{2}\equiv\left[\overline{\tau}_{2}\left(n-1\right),\tau_{2}\left(n\right)\right]$. Por lo tanto la probabilidad del evento $A_{2}\left(t\right)$ tiene probabilidad dada por

\begin{equation}
\prob\left\{A_{2}\left(t\right)|t\in I_{2}\left(n\right)\right\}=e^{-\tilde{\mu}_{2}\xi_{2}\left(n\right)},
\end{equation} 

donde $\xi_{2}\left(n\right)=\tau_{2}\left(n\right)-\overline{\tau}_{2}\left(n-1\right)$.



Entonces, se tiene que

\begin{eqnarray*}
\prob\left\{A_{1}\left(t\right),A_{2}\left(t\right)|t\in I_{1}\left(n\right)\right\}&=&
\prob\left\{A_{1}\left(t\right)|t\in I_{1}\left(n\right)\right\}
\prob\left\{A_{2}\left(t\right)|t\in I_{1}\left(n\right)\right\}\\
&\geq&
\prob\left\{A_{1}\left(t\right)|t\in I_{1}\left(n\right)\right\}
\prob\left\{A_{2}\left(t\right)|t\in I_{2}\left(n\right)\right\}\\
&=&e^{-\tilde{\mu}_{1}\xi_{1}\left(n\right)}e^{-\tilde{\mu}_{2}\xi_{2}\left(n\right)}
=e^{-\left[\tilde{\mu}_{1}\xi_{1}\left(n\right)+\tilde{\mu}_{2}\xi_{2}\left(n\right)\right]}.
\end{eqnarray*}


es decir, 

\begin{equation}
\prob\left\{A_{1}\left(t\right),A_{2}\left(t\right)|t\in I_{1}\left(n\right)\right\}
=e^{-\left[\tilde{\mu}_{1}\xi_{1}\left(n\right)+\tilde{\mu}_{2}\xi_{2}
\left(n\right)\right]}>0.
\end{equation}

En lo que respecta a la relaci\'on entre los dos SVC que conforman la RSVC, se afirma que existe $m>0$ tal que $\overline{\tau}_{3}\left(m\right)<\tau_{2}\left(n\right)<\tau_{4}\left(m\right)$.

Para $Q_{3}$ sea $I_{3}=\left[\overline{\tau}_{3}\left(m\right),\tau_{4}\left(m\right)\right]$ con longitud  $\xi_{3}\left(m\right)=r_{3}\left(m\right)$, entonces 

\begin{equation}
\prob\left\{A_{3}\left(t\right)|t\in I_{3}\left(n\right)\right\}=e^{-\tilde{\mu}_{3}\xi_{3}\left(n\right)}.
\end{equation} 

An\'alogamente que como se hizo para $Q_{2}$, tenemos que para $Q_{4}$ se tiene el intervalo $I_{4}=\left[\overline{\tau}_{4}\left(m-1\right),\tau_{4}\left(m\right)\right]$ con longitud $\xi_{4}\left(m\right)=\tau_{4}\left(m\right)-\overline{\tau}_{4}\left(m-1\right)$, entonces


\begin{equation}
\prob\left\{A_{4}\left(t\right)|t\in I_{4}\left(m\right)\right\}=e^{-\tilde{\mu}_{4}\xi_{4}\left(n\right)}.
\end{equation} 

Al igual que para el primer sistema, dado que $I_{3}\left(m\right)\subset I_{4}\left(m\right)$, se tiene que

\begin{eqnarray*}
\xi_{3}\left(m\right)\leq\xi_{4}\left(m\right)&\Leftrightarrow& -\xi_{3}\left(m\right)\geq-\xi_{4}\left(m\right)
\\
-\tilde{\mu}_{4}\xi_{3}\left(m\right)\geq-\tilde{\mu}_{4}\xi_{4}\left(m\right)&\Leftrightarrow&
e^{-\tilde{\mu}_{4}\xi_{3}\left(m\right)}\geq e^{-\tilde{\mu}_{4}\xi_{4}\left(m\right)}\\
\prob\left\{A_{4}\left(t\right)|t\in I_{3}\left(m\right)\right\}&\geq&
\prob\left\{A_{4}\left(t\right)|t\in I_{4}\left(m\right)\right\}
\end{eqnarray*}

Entonces, dado que los eventos $A_{3}$ y $A_{4}$ son independientes, se tiene que

\begin{eqnarray*}
\prob\left\{A_{3}\left(t\right),A_{4}\left(t\right)|t\in I_{3}\left(m\right)\right\}&=&
\prob\left\{A_{3}\left(t\right)|t\in I_{3}\left(m\right)\right\}
\prob\left\{A_{4}\left(t\right)|t\in I_{3}\left(m\right)\right\}\\
&\geq&
\prob\left\{A_{3}\left(t\right)|t\in I_{3}\left(n\right)\right\}
\prob\left\{A_{4}\left(t\right)|t\in I_{4}\left(n\right)\right\}\\
&=&e^{-\tilde{\mu}_{3}\xi_{3}\left(m\right)}e^{-\tilde{\mu}_{4}\xi_{4}
\left(m\right)}
=e^{-\left[\tilde{\mu}_{3}\xi_{3}\left(m\right)+\tilde{\mu}_{4}\xi_{4}
\left(m\right)\right]}.
\end{eqnarray*}


es decir, 

\begin{equation}
\prob\left\{A_{3}\left(t\right),A_{4}\left(t\right)|t\in I_{3}\left(m\right)\right\}
=e^{-\left[\tilde{\mu}_{3}\xi_{3}\left(m\right)+\tilde{\mu}_{4}\xi_{4}
\left(m\right)\right]}>0.
\end{equation}

Por construcci\'on se tiene que $I\left(n,m\right)\equiv I_{1}\left(n\right)\cap I_{3}\left(m\right)\neq\emptyset$,entonces en particular se tienen las contenciones $I\left(n,m\right)\subseteq I_{1}\left(n\right)$ y $I\left(n,m\right)\subseteq I_{3}\left(m\right)$, por lo tanto si definimos $\xi_{n,m}\equiv\ell\left(I\left(n,m\right)\right)$ tenemos que

\begin{eqnarray*}
\xi_{n,m}\leq\xi_{1}\left(n\right)\textrm{ y }\xi_{n,m}\leq\xi_{3}\left(m\right)\textrm{ entonces }
-\xi_{n,m}\geq-\xi_{1}\left(n\right)\textrm{ y }-\xi_{n,m}\leq-\xi_{3}\left(m\right)\\
\end{eqnarray*}
por lo tanto tenemos las desigualdades 



\begin{eqnarray*}
\begin{array}{ll}
-\tilde{\mu}_{1}\xi_{n,m}\geq-\tilde{\mu}_{1}\xi_{1}\left(n\right),&
-\tilde{\mu}_{2}\xi_{n,m}\geq-\tilde{\mu}_{2}\xi_{1}\left(n\right)
\geq-\tilde{\mu}_{2}\xi_{2}\left(n\right),\\
-\tilde{\mu}_{3}\xi_{n,m}\geq-\tilde{\mu}_{3}\xi_{3}\left(m\right),&
-\tilde{\mu}_{4}\xi_{n,m}\geq-\tilde{\mu}_{4}\xi_{3}\left(m\right)
\geq-\tilde{\mu}_{4}\xi_{4}\left(m\right).
\end{array}
\end{eqnarray*}

Sea $T^{*}\in I_{n,m}$, entonces dado que en particular $T^{*}\in I_{1}\left(n\right)$ se cumple con probabilidad positiva que no hay arribos a las colas $Q_{1}$ y $Q_{2}$, en consecuencia, tampoco hay usuarios de transferencia para $Q_{3}$ y $Q_{4}$, es decir, $\tilde{\mu}_{1}=\mu_{1}$, $\tilde{\mu}_{2}=\mu_{2}$, $\tilde{\mu}_{3}=\mu_{3}$, $\tilde{\mu}_{4}=\mu_{4}$, es decir, los eventos $Q_{1}$ y $Q_{3}$ son condicionalmente independientes en el intervalo $I_{n,m}$; lo mismo ocurre para las colas $Q_{2}$ y $Q_{4}$, por lo tanto tenemos que


\begin{eqnarray}
\begin{array}{l}
\prob\left\{A_{1}\left(T^{*}\right),A_{2}\left(T^{*}\right),
A_{3}\left(T^{*}\right),A_{4}\left(T^{*}\right)|T^{*}\in I_{n,m}\right\}
=\prod_{j=1}^{4}\prob\left\{A_{j}\left(T^{*}\right)|T^{*}\in I_{n,m}\right\}\\
\geq\prob\left\{A_{1}\left(T^{*}\right)|T^{*}\in I_{1}\left(n\right)\right\}
\prob\left\{A_{2}\left(T^{*}\right)|T^{*}\in I_{2}\left(n\right)\right\}
\prob\left\{A_{3}\left(T^{*}\right)|T^{*}\in I_{3}\left(m\right)\right\}
\prob\left\{A_{4}\left(T^{*}\right)|T^{*}\in I_{4}\left(m\right)\right\}\\
=e^{-\mu_{1}\xi_{1}\left(n\right)}
e^{-\mu_{2}\xi_{2}\left(n\right)}
e^{-\mu_{3}\xi_{3}\left(m\right)}
e^{-\mu_{4}\xi_{4}\left(m\right)}
=e^{-\left[\tilde{\mu}_{1}\xi_{1}\left(n\right)
+\tilde{\mu}_{2}\xi_{2}\left(n\right)
+\tilde{\mu}_{3}\xi_{3}\left(m\right)
+\tilde{\mu}_{4}\xi_{4}
\left(m\right)\right]}>0.
\end{array}
\end{eqnarray}
\end{proof}


Estos resultados aparecen en Daley (1968) \cite{Daley68} para $\left\{T_{n}\right\}$ intervalos de inter-arribo, $\left\{D_{n}\right\}$ intervalos de inter-salida y $\left\{S_{n}\right\}$ tiempos de servicio.

\begin{itemize}
\item Si el proceso $\left\{T_{n}\right\}$ es Poisson, el proceso $\left\{D_{n}\right\}$ es no correlacionado si y s\'olo si es un proceso Poisso, lo cual ocurre si y s\'olo si $\left\{S_{n}\right\}$ son exponenciales negativas.

\item Si $\left\{S_{n}\right\}$ son exponenciales negativas, $\left\{D_{n}\right\}$ es un proceso de renovaci\'on  si y s\'olo si es un proceso Poisson, lo cual ocurre si y s\'olo si $\left\{T_{n}\right\}$ es un proceso Poisson.

\item $\esp\left(D_{n}\right)=\esp\left(T_{n}\right)$.

\item Para un sistema de visitas $GI/M/1$ se tiene el siguiente teorema:

\begin{Teo}
En un sistema estacionario $GI/M/1$ los intervalos de interpartida tienen
\begin{eqnarray*}
\esp\left(e^{-\theta D_{n}}\right)&=&\mu\left(\mu+\theta\right)^{-1}\left[\delta\theta
-\mu\left(1-\delta\right)\alpha\left(\theta\right)\right]
\left[\theta-\mu\left(1-\delta\right)^{-1}\right]\\
\alpha\left(\theta\right)&=&\esp\left[e^{-\theta T_{0}}\right]\\
var\left(D_{n}\right)&=&var\left(T_{0}\right)-\left(\tau^{-1}-\delta^{-1}\right)
2\delta\left(\esp\left(S_{0}\right)\right)^{2}\left(1-\delta\right)^{-1}.
\end{eqnarray*}
\end{Teo}



\begin{Teo}
El proceso de salida de un sistema de colas estacionario $GI/M/1$ es un proceso de renovaci\'on si y s\'olo si el proceso de entrada es un proceso Poisson, en cuyo caso el proceso de salida es un proceso Poisson.
\end{Teo}


\begin{Teo}
Los intervalos de interpartida $\left\{D_{n}\right\}$ de un sistema $M/G/1$ estacionario son no correlacionados si y s\'olo si la distribuci\'on de los tiempos de servicio es exponencial negativa, es decir, el sistema es de tipo  $M/M/1$.

\end{Teo}



\end{itemize}


%\section{Resultados para Procesos de Salida}

En Sigman, Thorison y Wolff \cite{Sigman2} prueban que para la existencia de un una sucesi\'on infinita no decreciente de tiempos de regeneraci\'on $\tau_{1}\leq\tau_{2}\leq\cdots$ en los cuales el proceso se regenera, basta un tiempo de regeneraci\'on $R_{1}$, donde $R_{j}=\tau_{j}-\tau_{j-1}$. Para tal efecto se requiere la existencia de un espacio de probabilidad $\left(\Omega,\mathcal{F},\prob\right)$, y proceso estoc\'astico $\textit{X}=\left\{X\left(t\right):t\geq0\right\}$ con espacio de estados $\left(S,\mathcal{R}\right)$, con $\mathcal{R}$ $\sigma$-\'algebra.

\begin{Prop}
Si existe una variable aleatoria no negativa $R_{1}$ tal que $\theta_{R\footnotesize{1}}X=_{D}X$, entonces $\left(\Omega,\mathcal{F},\prob\right)$ puede extenderse para soportar una sucesi\'on estacionaria de variables aleatorias $R=\left\{R_{k}:k\geq1\right\}$, tal que para $k\geq1$,
\begin{eqnarray*}
\theta_{k}\left(X,R\right)=_{D}\left(X,R\right).
\end{eqnarray*}

Adem\'as, para $k\geq1$, $\theta_{k}R$ es condicionalmente independiente de $\left(X,R_{1},\ldots,R_{k}\right)$, dado $\theta_{\tau k}X$.

\end{Prop}


\begin{itemize}
\item Doob en 1953 demostr\'o que el estado estacionario de un proceso de partida en un sistema de espera $M/G/\infty$, es Poisson con la misma tasa que el proceso de arribos.

\item Burke en 1968, fue el primero en demostrar que el estado estacionario de un proceso de salida de una cola $M/M/s$ es un proceso Poisson.

\item Disney en 1973 obtuvo el siguiente resultado:

\begin{Teo}
Para el sistema de espera $M/G/1/L$ con disciplina FIFO, el proceso $\textbf{I}$ es un proceso de renovaci\'on si y s\'olo si el proceso denominado longitud de la cola es estacionario y se cumple cualquiera de los siguientes casos:

\begin{itemize}
\item[a)] Los tiempos de servicio son identicamente cero;
\item[b)] $L=0$, para cualquier proceso de servicio $S$;
\item[c)] $L=1$ y $G=D$;
\item[d)] $L=\infty$ y $G=M$.
\end{itemize}
En estos casos, respectivamente, las distribuciones de interpartida $P\left\{T_{n+1}-T_{n}\leq t\right\}$ son


\begin{itemize}
\item[a)] $1-e^{-\lambda t}$, $t\geq0$;
\item[b)] $1-e^{-\lambda t}*F\left(t\right)$, $t\geq0$;
\item[c)] $1-e^{-\lambda t}*\indora_{d}\left(t\right)$, $t\geq0$;
\item[d)] $1-e^{-\lambda t}*F\left(t\right)$, $t\geq0$.
\end{itemize}
\end{Teo}


\item Finch (1959) mostr\'o que para los sistemas $M/G/1/L$, con $1\leq L\leq \infty$ con distribuciones de servicio dos veces diferenciable, solamente el sistema $M/M/1/\infty$ tiene proceso de salida de renovaci\'on estacionario.

\item King (1971) demostro que un sistema de colas estacionario $M/G/1/1$ tiene sus tiempos de interpartida sucesivas $D_{n}$ y $D_{n+1}$ son independientes, si y s\'olo si, $G=D$, en cuyo caso le proceso de salida es de renovaci\'on.

\item Disney (1973) demostr\'o que el \'unico sistema estacionario $M/G/1/L$, que tiene proceso de salida de renovaci\'on  son los sistemas $M/M/1$ y $M/D/1/1$.



\item El siguiente resultado es de Disney y Koning (1985)
\begin{Teo}
En un sistema de espera $M/G/s$, el estado estacionario del proceso de salida es un proceso Poisson para cualquier distribuci\'on de los tiempos de servicio si el sistema tiene cualquiera de las siguientes cuatro propiedades.

\begin{itemize}
\item[a)] $s=\infty$
\item[b)] La disciplina de servicio es de procesador compartido.
\item[c)] La disciplina de servicio es LCFS y preemptive resume, esto se cumple para $L<\infty$
\item[d)] $G=M$.
\end{itemize}

\end{Teo}

\item El siguiente resultado es de Alamatsaz (1983)

\begin{Teo}
En cualquier sistema de colas $GI/G/1/L$ con $1\leq L<\infty$ y distribuci\'on de interarribos $A$ y distribuci\'on de los tiempos de servicio $B$, tal que $A\left(0\right)=0$, $A\left(t\right)\left(1-B\left(t\right)\right)>0$ para alguna $t>0$ y $B\left(t\right)$ para toda $t>0$, es imposible que el proceso de salida estacionario sea de renovaci\'on.
\end{Teo}

\end{itemize}

Estos resultados aparecen en Daley (1968) \cite{Daley68} para $\left\{T_{n}\right\}$ intervalos de inter-arribo, $\left\{D_{n}\right\}$ intervalos de inter-salida y $\left\{S_{n}\right\}$ tiempos de servicio.

\begin{itemize}
\item Si el proceso $\left\{T_{n}\right\}$ es Poisson, el proceso $\left\{D_{n}\right\}$ es no correlacionado si y s\'olo si es un proceso Poisso, lo cual ocurre si y s\'olo si $\left\{S_{n}\right\}$ son exponenciales negativas.

\item Si $\left\{S_{n}\right\}$ son exponenciales negativas, $\left\{D_{n}\right\}$ es un proceso de renovaci\'on  si y s\'olo si es un proceso Poisson, lo cual ocurre si y s\'olo si $\left\{T_{n}\right\}$ es un proceso Poisson.

\item $\esp\left(D_{n}\right)=\esp\left(T_{n}\right)$.

\item Para un sistema de visitas $GI/M/1$ se tiene el siguiente teorema:

\begin{Teo}
En un sistema estacionario $GI/M/1$ los intervalos de interpartida tienen
\begin{eqnarray*}
\esp\left(e^{-\theta D_{n}}\right)&=&\mu\left(\mu+\theta\right)^{-1}\left[\delta\theta
-\mu\left(1-\delta\right)\alpha\left(\theta\right)\right]
\left[\theta-\mu\left(1-\delta\right)^{-1}\right]\\
\alpha\left(\theta\right)&=&\esp\left[e^{-\theta T_{0}}\right]\\
var\left(D_{n}\right)&=&var\left(T_{0}\right)-\left(\tau^{-1}-\delta^{-1}\right)
2\delta\left(\esp\left(S_{0}\right)\right)^{2}\left(1-\delta\right)^{-1}.
\end{eqnarray*}
\end{Teo}



\begin{Teo}
El proceso de salida de un sistema de colas estacionario $GI/M/1$ es un proceso de renovaci\'on si y s\'olo si el proceso de entrada es un proceso Poisson, en cuyo caso el proceso de salida es un proceso Poisson.
\end{Teo}


\begin{Teo}
Los intervalos de interpartida $\left\{D_{n}\right\}$ de un sistema $M/G/1$ estacionario son no correlacionados si y s\'olo si la distribuci\'on de los tiempos de servicio es exponencial negativa, es decir, el sistema es de tipo  $M/M/1$.

\end{Teo}



\end{itemize}
%\newpage
%________________________________________________________________________
%\section{Redes de Sistemas de Visitas C\'iclicas}
%________________________________________________________________________

Sean $Q_{1},Q_{2},Q_{3}$ y $Q_{4}$ en una Red de Sistemas de Visitas C\'iclicas (RSVC). Supongamos que cada una de las colas es del tipo $M/M/1$ con tasa de arribo $\mu_{i}$ y que la transferencia de usuarios entre los dos sistemas ocurre entre $Q_{1}\leftrightarrow Q_{3}$ y $Q_{2}\leftrightarrow Q_{4}$ con respectiva tasa de arribo igual a la tasa de salida $\hat{\mu}_{i}=\mu_{i}$, esto se sabe por lo desarrollado en la secci\'on anterior.  

Consideremos, sin p\'erdida de generalidad como base del an\'alisis, la cola $Q_{1}$ adem\'as supongamos al servidor lo comenzamos a observar una vez que termina de atender a la misma para desplazarse y llegar a $Q_{2}$, es decir al tiempo $\tau_{2}$.

Sea $n\in\nat$, $n>0$, ciclo del servidor en que regresa a $Q_{1}$ para dar servicio y atender conforme a la pol\'itica exhaustiva, entonces se tiene que $\overline{\tau}_{1}\left(n\right)$ es el tiempo del servidor en el sistema 1 en que termina de dar servicio a todos los usuarios presentes en la cola, por lo tanto se cumple que $L_{1}\left(\overline{\tau}_{1}\left(n\right)\right)=0$, entonces el servidor para llegar a $Q_{2}$ incurre en un tiempo de traslado $r_{1}$ y por tanto se cumple que $\tau_{2}\left(n\right)=\overline{\tau}_{1}\left(n\right)+r_{1}$. Dado que los tiempos entre arribos son exponenciales se cumple que 

\begin{eqnarray*}
\prob\left\{0 \textrm{ arribos en }Q_{1}\textrm{ en el intervalo }\left[\overline{\tau}_{1}\left(n\right),\overline{\tau}_{1}\left(n\right)+r_{1}\right]\right\}=e^{-\tilde{\mu}_{1}r_{1}},\\
\prob\left\{0 \textrm{ arribos en }Q_{2}\textrm{ en el intervalo }\left[\overline{\tau}_{1}\left(n\right),\overline{\tau}_{1}\left(n\right)+r_{1}\right]\right\}=e^{-\tilde{\mu}_{2}r_{1}}.
\end{eqnarray*}

El evento que nos interesa consiste en que no haya arribos desde que el servidor abandon\'o $Q_{2}$ y regresa nuevamente para dar servicio, es decir en el intervalo de tiempo $\left[\overline{\tau}_{2}\left(n-1\right),\tau_{2}\left(n\right)\right]$. Entonces, si hacemos


\begin{eqnarray*}
\varphi_{1}\left(n\right)&\equiv&\overline{\tau}_{1}\left(n\right)+r_{1}=\overline{\tau}_{2}\left(n-1\right)+r_{1}+r_{2}+\overline{\tau}_{1}\left(n\right)-\tau_{1}\left(n\right)\\
&=&\overline{\tau}_{2}\left(n-1\right)+\overline{\tau}_{1}\left(n\right)-\tau_{1}\left(n\right)+r,,
\end{eqnarray*}

y longitud del intervalo

\begin{eqnarray*}
\xi&\equiv&\overline{\tau}_{1}\left(n\right)+r_{1}-\overline{\tau}_{2}\left(n-1\right)
=\overline{\tau}_{2}\left(n-1\right)+\overline{\tau}_{1}\left(n\right)-\tau_{1}\left(n\right)+r-\overline{\tau}_{2}\left(n-1\right)\\
&=&\overline{\tau}_{1}\left(n\right)-\tau_{1}\left(n\right)+r.
\end{eqnarray*}


Entonces, determinemos la probabilidad del evento no arribos a $Q_{2}$ en $\left[\overline{\tau}_{2}\left(n-1\right),\varphi_{1}\left(n\right)\right]$:

\begin{eqnarray}
\prob\left\{0 \textrm{ arribos en }Q_{2}\textrm{ en el intervalo }\left[\overline{\tau}_{2}\left(n-1\right),\varphi_{1}\left(n\right)\right]\right\}
=e^{-\tilde{\mu}_{2}\xi}.
\end{eqnarray}

De manera an\'aloga, tenemos que la probabilidad de no arribos a $Q_{1}$ en $\left[\overline{\tau}_{2}\left(n-1\right),\varphi_{1}\left(n\right)\right]$ esta dada por

\begin{eqnarray}
\prob\left\{0 \textrm{ arribos en }Q_{1}\textrm{ en el intervalo }\left[\overline{\tau}_{2}\left(n-1\right),\varphi_{1}\left(n\right)\right]\right\}
=e^{-\tilde{\mu}_{1}\xi},
\end{eqnarray}

\begin{eqnarray}
\prob\left\{0 \textrm{ arribos en }Q_{2}\textrm{ en el intervalo }\left[\overline{\tau}_{2}\left(n-1\right),\varphi_{1}\left(n\right)\right]\right\}
=e^{-\tilde{\mu}_{2}\xi}.
\end{eqnarray}

Por tanto 

\begin{eqnarray}
\begin{array}{l}
\prob\left\{0 \textrm{ arribos en }Q_{1}\textrm{ y }Q_{2}\textrm{ en el intervalo }\left[\overline{\tau}_{2}\left(n-1\right),\varphi_{1}\left(n\right)\right]\right\}\\
=\prob\left\{0 \textrm{ arribos en }Q_{1}\textrm{ en el intervalo }\left[\overline{\tau}_{2}\left(n-1\right),\varphi_{1}\left(n\right)\right]\right\}\\
\times
\prob\left\{0 \textrm{ arribos en }Q_{2}\textrm{ en el intervalo }\left[\overline{\tau}_{2}\left(n-1\right),\varphi_{1}\left(n\right)\right]\right\}=e^{-\tilde{\mu}_{1}\xi}e^{-\tilde{\mu}_{2}\xi}
=e^{-\tilde{\mu}\xi}.
\end{array}
\end{eqnarray}

Para el segundo sistema, consideremos nuevamente $\overline{\tau}_{1}\left(n\right)+r_{1}$, sin p\'erdida de generalidad podemos suponer que existe $m>0$ tal que $\overline{\tau}_{3}\left(m\right)<\overline{\tau}_{1}+r_{1}<\tau_{4}\left(m\right)$, entonces

\begin{eqnarray}
\prob\left\{0 \textrm{ arribos en }Q_{3}\textrm{ en el intervalo }\left[\overline{\tau}_{3}\left(m\right),\overline{\tau}_{1}\left(n\right)+r_{1}\right]\right\}
=e^{-\tilde{\mu}_{3}\xi_{3}},
\end{eqnarray}
donde 
\begin{eqnarray}
\xi_{3}=\overline{\tau}_{1}\left(n\right)+r_{1}-\overline{\tau}_{3}\left(m\right)=
\overline{\tau}_{1}\left(n\right)-\overline{\tau}_{3}\left(m\right)+r_{1},
\end{eqnarray}

mientras que para $Q_{4}$ al igual que con $Q_{2}$ escribiremos $\tau_{4}\left(m\right)$ en t\'erminos de $\overline{\tau}_{4}\left(m-1\right)$:

$\varphi_{2}\equiv\tau_{4}\left(m\right)=\overline{\tau}_{4}\left(m-1\right)+r_{4}+\overline{\tau}_{3}\left(m\right)
-\tau_{3}\left(m\right)+r_{3}=\overline{\tau}_{4}\left(m-1\right)+\overline{\tau}_{3}\left(m\right)
-\tau_{3}\left(m\right)+\hat{r}$, adem\'as,

$\xi_{2}\equiv\varphi_{2}\left(m\right)-\overline{\tau}_{1}-r_{1}=\overline{\tau}_{4}\left(m-1\right)+\overline{\tau}_{3}\left(m\right)
-\tau_{3}\left(m\right)-\overline{\tau}_{1}\left(n\right)+\hat{r}-r_{1}$. 

Entonces


\begin{eqnarray}
\prob\left\{0 \textrm{ arribos en }Q_{4}\textrm{ en el intervalo }\left[\overline{\tau}_{1}\left(n\right)+r_{1},\varphi_{2}\left(m\right)\right]\right\}
=e^{-\tilde{\mu}_{4}\xi_{2}},
\end{eqnarray}

mientras que para $Q_{3}$ se tiene que 

\begin{eqnarray}
\prob\left\{0 \textrm{ arribos en }Q_{3}\textrm{ en el intervalo }\left[\overline{\tau}_{1}\left(n\right)+r_{1},\varphi_{2}\left(m\right)\right]\right\}
=e^{-\tilde{\mu}_{3}\xi_{2}}
\end{eqnarray}

Por tanto

\begin{eqnarray}
\prob\left\{0 \textrm{ arribos en }Q_{3}\wedge Q_{4}\textrm{ en el intervalo }\left[\overline{\tau}_{1}\left(n\right)+r_{1},\varphi_{2}\left(m\right)\right]\right\}
=e^{-\hat{\mu}\xi_{2}}
\end{eqnarray}
donde $\hat{\mu}=\tilde{\mu}_{3}+\tilde{\mu}_{4}$.

Ahora, definamos los intervalos $\mathcal{I}_{1}=\left[\overline{\tau}_{1}\left(n\right)+r_{1},\varphi_{1}\left(n\right)\right]$  y $\mathcal{I}_{2}=\left[\overline{\tau}_{1}\left(n\right)+r_{1},\varphi_{2}\left(m\right)\right]$, entonces, sea $\mathcal{I}=\mathcal{I}_{1}\cap\mathcal{I}_{2}$ el intervalo donde cada una de las colas se encuentran vac\'ias, es decir, si tomamos $T^{*}\in\mathcal{I}$, entonces  $L_{1}\left(T^{*}\right)=L_{2}\left(T^{*}\right)=L_{3}\left(T^{*}\right)=L_{4}\left(T^{*}\right)=0$.

Ahora, dado que por construcci\'on $\mathcal{I}\neq\emptyset$ y que para $T^{*}\in\mathcal{I}$ en ninguna de las colas han llegado usuarios, se tiene que no hay transferencia entre las colas, por lo tanto, el sistema 1 y el sistema 2 son condicionalmente independientes en $\mathcal{I}$, es decir

\begin{eqnarray}
\prob\left\{L_{1}\left(T^{*}\right),L_{2}\left(T^{*}\right),L_{3}\left(T^{*}\right),L_{4}\left(T^{*}\right)|T^{*}\in\mathcal{I}\right\}=\prod_{j=1}^{4}\prob\left\{L_{j}\left(T^{*}\right)\right\},
\end{eqnarray}

para $T^{*}\in\mathcal{I}$. 

%\newpage























%________________________________________________________________________
%\section{Procesos Regenerativos}
%________________________________________________________________________

%________________________________________________________________________
%\subsection*{Procesos Regenerativos Sigman, Thorisson y Wolff \cite{Sigman1}}
%________________________________________________________________________


\begin{Def}[Definici\'on Cl\'asica]
Un proceso estoc\'astico $X=\left\{X\left(t\right):t\geq0\right\}$ es llamado regenerativo is existe una variable aleatoria $R_{1}>0$ tal que
\begin{itemize}
\item[i)] $\left\{X\left(t+R_{1}\right):t\geq0\right\}$ es independiente de $\left\{\left\{X\left(t\right):t<R_{1}\right\},\right\}$
\item[ii)] $\left\{X\left(t+R_{1}\right):t\geq0\right\}$ es estoc\'asticamente equivalente a $\left\{X\left(t\right):t>0\right\}$
\end{itemize}

Llamamos a $R_{1}$ tiempo de regeneraci\'on, y decimos que $X$ se regenera en este punto.
\end{Def}

$\left\{X\left(t+R_{1}\right)\right\}$ es regenerativo con tiempo de regeneraci\'on $R_{2}$, independiente de $R_{1}$ pero con la misma distribuci\'on que $R_{1}$. Procediendo de esta manera se obtiene una secuencia de variables aleatorias independientes e id\'enticamente distribuidas $\left\{R_{n}\right\}$ llamados longitudes de ciclo. Si definimos a $Z_{k}\equiv R_{1}+R_{2}+\cdots+R_{k}$, se tiene un proceso de renovaci\'on llamado proceso de renovaci\'on encajado para $X$.


\begin{Note}
La existencia de un primer tiempo de regeneraci\'on, $R_{1}$, implica la existencia de una sucesi\'on completa de estos tiempos $R_{1},R_{2}\ldots,$ que satisfacen la propiedad deseada \cite{Sigman2}.
\end{Note}


\begin{Note} Para la cola $GI/GI/1$ los usuarios arriban con tiempos $t_{n}$ y son atendidos con tiempos de servicio $S_{n}$, los tiempos de arribo forman un proceso de renovaci\'on  con tiempos entre arribos independientes e identicamente distribuidos (\texttt{i.i.d.})$T_{n}=t_{n}-t_{n-1}$, adem\'as los tiempos de servicio son \texttt{i.i.d.} e independientes de los procesos de arribo. Por \textit{estable} se entiende que $\esp S_{n}<\esp T_{n}<\infty$.
\end{Note}
 


\begin{Def}
Para $x$ fijo y para cada $t\geq0$, sea $I_{x}\left(t\right)=1$ si $X\left(t\right)\leq x$,  $I_{x}\left(t\right)=0$ en caso contrario, y def\'inanse los tiempos promedio
\begin{eqnarray*}
\overline{X}&=&lim_{t\rightarrow\infty}\frac{1}{t}\int_{0}^{\infty}X\left(u\right)du\\
\prob\left(X_{\infty}\leq x\right)&=&lim_{t\rightarrow\infty}\frac{1}{t}\int_{0}^{\infty}I_{x}\left(u\right)du,
\end{eqnarray*}
cuando estos l\'imites existan.
\end{Def}

Como consecuencia del teorema de Renovaci\'on-Recompensa, se tiene que el primer l\'imite  existe y es igual a la constante
\begin{eqnarray*}
\overline{X}&=&\frac{\esp\left[\int_{0}^{R_{1}}X\left(t\right)dt\right]}{\esp\left[R_{1}\right]},
\end{eqnarray*}
suponiendo que ambas esperanzas son finitas.
 
\begin{Note}
Funciones de procesos regenerativos son regenerativas, es decir, si $X\left(t\right)$ es regenerativo y se define el proceso $Y\left(t\right)$ por $Y\left(t\right)=f\left(X\left(t\right)\right)$ para alguna funci\'on Borel medible $f\left(\cdot\right)$. Adem\'as $Y$ es regenerativo con los mismos tiempos de renovaci\'on que $X$. 

En general, los tiempos de renovaci\'on, $Z_{k}$ de un proceso regenerativo no requieren ser tiempos de paro con respecto a la evoluci\'on de $X\left(t\right)$.
\end{Note} 

\begin{Note}
Una funci\'on de un proceso de Markov, usualmente no ser\'a un proceso de Markov, sin embargo ser\'a regenerativo si el proceso de Markov lo es.
\end{Note}

 
\begin{Note}
Un proceso regenerativo con media de la longitud de ciclo finita es llamado positivo recurrente.
\end{Note}


\begin{Note}
\begin{itemize}
\item[a)] Si el proceso regenerativo $X$ es positivo recurrente y tiene trayectorias muestrales no negativas, entonces la ecuaci\'on anterior es v\'alida.
\item[b)] Si $X$ es positivo recurrente regenerativo, podemos construir una \'unica versi\'on estacionaria de este proceso, $X_{e}=\left\{X_{e}\left(t\right)\right\}$, donde $X_{e}$ es un proceso estoc\'astico regenerativo y estrictamente estacionario, con distribuci\'on marginal distribuida como $X_{\infty}$
\end{itemize}
\end{Note}


%__________________________________________________________________________________________
%\subsection*{Procesos Regenerativos Estacionarios - Stidham \cite{Stidham}}
%__________________________________________________________________________________________


Un proceso estoc\'astico a tiempo continuo $\left\{V\left(t\right),t\geq0\right\}$ es un proceso regenerativo si existe una sucesi\'on de variables aleatorias independientes e id\'enticamente distribuidas $\left\{X_{1},X_{2},\ldots\right\}$, sucesi\'on de renovaci\'on, tal que para cualquier conjunto de Borel $A$, 

\begin{eqnarray*}
\prob\left\{V\left(t\right)\in A|X_{1}+X_{2}+\cdots+X_{R\left(t\right)}=s,\left\{V\left(\tau\right),\tau<s\right\}\right\}=\prob\left\{V\left(t-s\right)\in A|X_{1}>t-s\right\},
\end{eqnarray*}
para todo $0\leq s\leq t$, donde $R\left(t\right)=\max\left\{X_{1}+X_{2}+\cdots+X_{j}\leq t\right\}=$n\'umero de renovaciones ({\emph{puntos de regeneraci\'on}}) que ocurren en $\left[0,t\right]$. El intervalo $\left[0,X_{1}\right)$ es llamado {\emph{primer ciclo de regeneraci\'on}} de $\left\{V\left(t \right),t\geq0\right\}$, $\left[X_{1},X_{1}+X_{2}\right)$ el {\emph{segundo ciclo de regeneraci\'on}}, y as\'i sucesivamente.

Sea $X=X_{1}$ y sea $F$ la funci\'on de distrbuci\'on de $X$


\begin{Def}
Se define el proceso estacionario, $\left\{V^{*}\left(t\right),t\geq0\right\}$, para $\left\{V\left(t\right),t\geq0\right\}$ por

\begin{eqnarray*}
\prob\left\{V\left(t\right)\in A\right\}=\frac{1}{\esp\left[X\right]}\int_{0}^{\infty}\prob\left\{V\left(t+x\right)\in A|X>x\right\}\left(1-F\left(x\right)\right)dx,
\end{eqnarray*} 
para todo $t\geq0$ y todo conjunto de Borel $A$.
\end{Def}

\begin{Def}
Una distribuci\'on se dice que es {\emph{aritm\'etica}} si todos sus puntos de incremento son m\'ultiplos de la forma $0,\lambda, 2\lambda,\ldots$ para alguna $\lambda>0$ entera.
\end{Def}


\begin{Def}
Una modificaci\'on medible de un proceso $\left\{V\left(t\right),t\geq0\right\}$, es una versi\'on de este, $\left\{V\left(t,w\right)\right\}$ conjuntamente medible para $t\geq0$ y para $w\in S$, $S$ espacio de estados para $\left\{V\left(t\right),t\geq0\right\}$.
\end{Def}

\begin{Teo}
Sea $\left\{V\left(t\right),t\geq\right\}$ un proceso regenerativo no negativo con modificaci\'on medible. Sea $\esp\left[X\right]<\infty$. Entonces el proceso estacionario dado por la ecuaci\'on anterior est\'a bien definido y tiene funci\'on de distribuci\'on independiente de $t$, adem\'as
\begin{itemize}
\item[i)] \begin{eqnarray*}
\esp\left[V^{*}\left(0\right)\right]&=&\frac{\esp\left[\int_{0}^{X}V\left(s\right)ds\right]}{\esp\left[X\right]}\end{eqnarray*}
\item[ii)] Si $\esp\left[V^{*}\left(0\right)\right]<\infty$, equivalentemente, si $\esp\left[\int_{0}^{X}V\left(s\right)ds\right]<\infty$,entonces
\begin{eqnarray*}
\frac{\int_{0}^{t}V\left(s\right)ds}{t}\rightarrow\frac{\esp\left[\int_{0}^{X}V\left(s\right)ds\right]}{\esp\left[X\right]}
\end{eqnarray*}
con probabilidad 1 y en media, cuando $t\rightarrow\infty$.
\end{itemize}
\end{Teo}

\begin{Coro}
Sea $\left\{V\left(t\right),t\geq0\right\}$ un proceso regenerativo no negativo, con modificaci\'on medible. Si $\esp <\infty$, $F$ es no-aritm\'etica, y para todo $x\geq0$, $P\left\{V\left(t\right)\leq x,C>x\right\}$ es de variaci\'on acotada como funci\'on de $t$ en cada intervalo finito $\left[0,\tau\right]$, entonces $V\left(t\right)$ converge en distribuci\'on  cuando $t\rightarrow\infty$ y $$\esp V=\frac{\esp \int_{0}^{X}V\left(s\right)ds}{\esp X}$$
Donde $V$ tiene la distribuci\'on l\'imite de $V\left(t\right)$ cuando $t\rightarrow\infty$.

\end{Coro}

Para el caso discreto se tienen resultados similares.



%______________________________________________________________________
%\section{Procesos de Renovaci\'on}
%______________________________________________________________________

\begin{Def}\label{Def.Tn}
Sean $0\leq T_{1}\leq T_{2}\leq \ldots$ son tiempos aleatorios infinitos en los cuales ocurren ciertos eventos. El n\'umero de tiempos $T_{n}$ en el intervalo $\left[0,t\right)$ es

\begin{eqnarray}
N\left(t\right)=\sum_{n=1}^{\infty}\indora\left(T_{n}\leq t\right),
\end{eqnarray}
para $t\geq0$.
\end{Def}

Si se consideran los puntos $T_{n}$ como elementos de $\rea_{+}$, y $N\left(t\right)$ es el n\'umero de puntos en $\rea$. El proceso denotado por $\left\{N\left(t\right):t\geq0\right\}$, denotado por $N\left(t\right)$, es un proceso puntual en $\rea_{+}$. Los $T_{n}$ son los tiempos de ocurrencia, el proceso puntual $N\left(t\right)$ es simple si su n\'umero de ocurrencias son distintas: $0<T_{1}<T_{2}<\ldots$ casi seguramente.

\begin{Def}
Un proceso puntual $N\left(t\right)$ es un proceso de renovaci\'on si los tiempos de interocurrencia $\xi_{n}=T_{n}-T_{n-1}$, para $n\geq1$, son independientes e identicamente distribuidos con distribuci\'on $F$, donde $F\left(0\right)=0$ y $T_{0}=0$. Los $T_{n}$ son llamados tiempos de renovaci\'on, referente a la independencia o renovaci\'on de la informaci\'on estoc\'astica en estos tiempos. Los $\xi_{n}$ son los tiempos de inter-renovaci\'on, y $N\left(t\right)$ es el n\'umero de renovaciones en el intervalo $\left[0,t\right)$
\end{Def}


\begin{Note}
Para definir un proceso de renovaci\'on para cualquier contexto, solamente hay que especificar una distribuci\'on $F$, con $F\left(0\right)=0$, para los tiempos de inter-renovaci\'on. La funci\'on $F$ en turno degune las otra variables aleatorias. De manera formal, existe un espacio de probabilidad y una sucesi\'on de variables aleatorias $\xi_{1},\xi_{2},\ldots$ definidas en este con distribuci\'on $F$. Entonces las otras cantidades son $T_{n}=\sum_{k=1}^{n}\xi_{k}$ y $N\left(t\right)=\sum_{n=1}^{\infty}\indora\left(T_{n}\leq t\right)$, donde $T_{n}\rightarrow\infty$ casi seguramente por la Ley Fuerte de los Grandes Números.
\end{Note}

%___________________________________________________________________________________________
%
%\subsection*{Teorema Principal de Renovaci\'on}
%___________________________________________________________________________________________
%

\begin{Note} Una funci\'on $h:\rea_{+}\rightarrow\rea$ es Directamente Riemann Integrable en los siguientes casos:
\begin{itemize}
\item[a)] $h\left(t\right)\geq0$ es decreciente y Riemann Integrable.
\item[b)] $h$ es continua excepto posiblemente en un conjunto de Lebesgue de medida 0, y $|h\left(t\right)|\leq b\left(t\right)$, donde $b$ es DRI.
\end{itemize}
\end{Note}

\begin{Teo}[Teorema Principal de Renovaci\'on]
Si $F$ es no aritm\'etica y $h\left(t\right)$ es Directamente Riemann Integrable (DRI), entonces

\begin{eqnarray*}
lim_{t\rightarrow\infty}U\star h=\frac{1}{\mu}\int_{\rea_{+}}h\left(s\right)ds.
\end{eqnarray*}
\end{Teo}

\begin{Prop}
Cualquier funci\'on $H\left(t\right)$ acotada en intervalos finitos y que es 0 para $t<0$ puede expresarse como
\begin{eqnarray*}
H\left(t\right)=U\star h\left(t\right)\textrm{,  donde }h\left(t\right)=H\left(t\right)-F\star H\left(t\right)
\end{eqnarray*}
\end{Prop}

\begin{Def}
Un proceso estoc\'astico $X\left(t\right)$ es crudamente regenerativo en un tiempo aleatorio positivo $T$ si
\begin{eqnarray*}
\esp\left[X\left(T+t\right)|T\right]=\esp\left[X\left(t\right)\right]\textrm{, para }t\geq0,\end{eqnarray*}
y con las esperanzas anteriores finitas.
\end{Def}

\begin{Prop}
Sup\'ongase que $X\left(t\right)$ es un proceso crudamente regenerativo en $T$, que tiene distribuci\'on $F$. Si $\esp\left[X\left(t\right)\right]$ es acotado en intervalos finitos, entonces
\begin{eqnarray*}
\esp\left[X\left(t\right)\right]=U\star h\left(t\right)\textrm{,  donde }h\left(t\right)=\esp\left[X\left(t\right)\indora\left(T>t\right)\right].
\end{eqnarray*}
\end{Prop}

\begin{Teo}[Regeneraci\'on Cruda]
Sup\'ongase que $X\left(t\right)$ es un proceso con valores positivo crudamente regenerativo en $T$, y def\'inase $M=\sup\left\{|X\left(t\right)|:t\leq T\right\}$. Si $T$ es no aritm\'etico y $M$ y $MT$ tienen media finita, entonces
\begin{eqnarray*}
lim_{t\rightarrow\infty}\esp\left[X\left(t\right)\right]=\frac{1}{\mu}\int_{\rea_{+}}h\left(s\right)ds,
\end{eqnarray*}
donde $h\left(t\right)=\esp\left[X\left(t\right)\indora\left(T>t\right)\right]$.
\end{Teo}

%___________________________________________________________________________________________
%
%\subsection*{Propiedades de los Procesos de Renovaci\'on}
%___________________________________________________________________________________________
%

Los tiempos $T_{n}$ est\'an relacionados con los conteos de $N\left(t\right)$ por

\begin{eqnarray*}
\left\{N\left(t\right)\geq n\right\}&=&\left\{T_{n}\leq t\right\}\\
T_{N\left(t\right)}\leq &t&<T_{N\left(t\right)+1},
\end{eqnarray*}

adem\'as $N\left(T_{n}\right)=n$, y 

\begin{eqnarray*}
N\left(t\right)=\max\left\{n:T_{n}\leq t\right\}=\min\left\{n:T_{n+1}>t\right\}
\end{eqnarray*}

Por propiedades de la convoluci\'on se sabe que

\begin{eqnarray*}
P\left\{T_{n}\leq t\right\}=F^{n\star}\left(t\right)
\end{eqnarray*}
que es la $n$-\'esima convoluci\'on de $F$. Entonces 

\begin{eqnarray*}
\left\{N\left(t\right)\geq n\right\}&=&\left\{T_{n}\leq t\right\}\\
P\left\{N\left(t\right)\leq n\right\}&=&1-F^{\left(n+1\right)\star}\left(t\right)
\end{eqnarray*}

Adem\'as usando el hecho de que $\esp\left[N\left(t\right)\right]=\sum_{n=1}^{\infty}P\left\{N\left(t\right)\geq n\right\}$
se tiene que

\begin{eqnarray*}
\esp\left[N\left(t\right)\right]=\sum_{n=1}^{\infty}F^{n\star}\left(t\right)
\end{eqnarray*}

\begin{Prop}
Para cada $t\geq0$, la funci\'on generadora de momentos $\esp\left[e^{\alpha N\left(t\right)}\right]$ existe para alguna $\alpha$ en una vecindad del 0, y de aqu\'i que $\esp\left[N\left(t\right)^{m}\right]<\infty$, para $m\geq1$.
\end{Prop}


\begin{Note}
Si el primer tiempo de renovaci\'on $\xi_{1}$ no tiene la misma distribuci\'on que el resto de las $\xi_{n}$, para $n\geq2$, a $N\left(t\right)$ se le llama Proceso de Renovaci\'on retardado, donde si $\xi$ tiene distribuci\'on $G$, entonces el tiempo $T_{n}$ de la $n$-\'esima renovaci\'on tiene distribuci\'on $G\star F^{\left(n-1\right)\star}\left(t\right)$
\end{Note}


\begin{Teo}
Para una constante $\mu\leq\infty$ ( o variable aleatoria), las siguientes expresiones son equivalentes:

\begin{eqnarray}
lim_{n\rightarrow\infty}n^{-1}T_{n}&=&\mu,\textrm{ c.s.}\\
lim_{t\rightarrow\infty}t^{-1}N\left(t\right)&=&1/\mu,\textrm{ c.s.}
\end{eqnarray}
\end{Teo}


Es decir, $T_{n}$ satisface la Ley Fuerte de los Grandes N\'umeros s\'i y s\'olo s\'i $N\left/t\right)$ la cumple.


\begin{Coro}[Ley Fuerte de los Grandes N\'umeros para Procesos de Renovaci\'on]
Si $N\left(t\right)$ es un proceso de renovaci\'on cuyos tiempos de inter-renovaci\'on tienen media $\mu\leq\infty$, entonces
\begin{eqnarray}
t^{-1}N\left(t\right)\rightarrow 1/\mu,\textrm{ c.s. cuando }t\rightarrow\infty.
\end{eqnarray}

\end{Coro}


Considerar el proceso estoc\'astico de valores reales $\left\{Z\left(t\right):t\geq0\right\}$ en el mismo espacio de probabilidad que $N\left(t\right)$

\begin{Def}
Para el proceso $\left\{Z\left(t\right):t\geq0\right\}$ se define la fluctuaci\'on m\'axima de $Z\left(t\right)$ en el intervalo $\left(T_{n-1},T_{n}\right]$:
\begin{eqnarray*}
M_{n}=\sup_{T_{n-1}<t\leq T_{n}}|Z\left(t\right)-Z\left(T_{n-1}\right)|
\end{eqnarray*}
\end{Def}

\begin{Teo}
Sup\'ongase que $n^{-1}T_{n}\rightarrow\mu$ c.s. cuando $n\rightarrow\infty$, donde $\mu\leq\infty$ es una constante o variable aleatoria. Sea $a$ una constante o variable aleatoria que puede ser infinita cuando $\mu$ es finita, y considere las expresiones l\'imite:
\begin{eqnarray}
lim_{n\rightarrow\infty}n^{-1}Z\left(T_{n}\right)&=&a,\textrm{ c.s.}\\
lim_{t\rightarrow\infty}t^{-1}Z\left(t\right)&=&a/\mu,\textrm{ c.s.}
\end{eqnarray}
La segunda expresi\'on implica la primera. Conversamente, la primera implica la segunda si el proceso $Z\left(t\right)$ es creciente, o si $lim_{n\rightarrow\infty}n^{-1}M_{n}=0$ c.s.
\end{Teo}

\begin{Coro}
Si $N\left(t\right)$ es un proceso de renovaci\'on, y $\left(Z\left(T_{n}\right)-Z\left(T_{n-1}\right),M_{n}\right)$, para $n\geq1$, son variables aleatorias independientes e id\'enticamente distribuidas con media finita, entonces,
\begin{eqnarray}
lim_{t\rightarrow\infty}t^{-1}Z\left(t\right)\rightarrow\frac{\esp\left[Z\left(T_{1}\right)-Z\left(T_{0}\right)\right]}{\esp\left[T_{1}\right]},\textrm{ c.s. cuando  }t\rightarrow\infty.
\end{eqnarray}
\end{Coro}



%___________________________________________________________________________________________
%
%\subsection{Propiedades de los Procesos de Renovaci\'on}
%___________________________________________________________________________________________
%

Los tiempos $T_{n}$ est\'an relacionados con los conteos de $N\left(t\right)$ por

\begin{eqnarray*}
\left\{N\left(t\right)\geq n\right\}&=&\left\{T_{n}\leq t\right\}\\
T_{N\left(t\right)}\leq &t&<T_{N\left(t\right)+1},
\end{eqnarray*}

adem\'as $N\left(T_{n}\right)=n$, y 

\begin{eqnarray*}
N\left(t\right)=\max\left\{n:T_{n}\leq t\right\}=\min\left\{n:T_{n+1}>t\right\}
\end{eqnarray*}

Por propiedades de la convoluci\'on se sabe que

\begin{eqnarray*}
P\left\{T_{n}\leq t\right\}=F^{n\star}\left(t\right)
\end{eqnarray*}
que es la $n$-\'esima convoluci\'on de $F$. Entonces 

\begin{eqnarray*}
\left\{N\left(t\right)\geq n\right\}&=&\left\{T_{n}\leq t\right\}\\
P\left\{N\left(t\right)\leq n\right\}&=&1-F^{\left(n+1\right)\star}\left(t\right)
\end{eqnarray*}

Adem\'as usando el hecho de que $\esp\left[N\left(t\right)\right]=\sum_{n=1}^{\infty}P\left\{N\left(t\right)\geq n\right\}$
se tiene que

\begin{eqnarray*}
\esp\left[N\left(t\right)\right]=\sum_{n=1}^{\infty}F^{n\star}\left(t\right)
\end{eqnarray*}

\begin{Prop}
Para cada $t\geq0$, la funci\'on generadora de momentos $\esp\left[e^{\alpha N\left(t\right)}\right]$ existe para alguna $\alpha$ en una vecindad del 0, y de aqu\'i que $\esp\left[N\left(t\right)^{m}\right]<\infty$, para $m\geq1$.
\end{Prop}


\begin{Note}
Si el primer tiempo de renovaci\'on $\xi_{1}$ no tiene la misma distribuci\'on que el resto de las $\xi_{n}$, para $n\geq2$, a $N\left(t\right)$ se le llama Proceso de Renovaci\'on retardado, donde si $\xi$ tiene distribuci\'on $G$, entonces el tiempo $T_{n}$ de la $n$-\'esima renovaci\'on tiene distribuci\'on $G\star F^{\left(n-1\right)\star}\left(t\right)$
\end{Note}


\begin{Teo}
Para una constante $\mu\leq\infty$ ( o variable aleatoria), las siguientes expresiones son equivalentes:

\begin{eqnarray}
lim_{n\rightarrow\infty}n^{-1}T_{n}&=&\mu,\textrm{ c.s.}\\
lim_{t\rightarrow\infty}t^{-1}N\left(t\right)&=&1/\mu,\textrm{ c.s.}
\end{eqnarray}
\end{Teo}


Es decir, $T_{n}$ satisface la Ley Fuerte de los Grandes N\'umeros s\'i y s\'olo s\'i $N\left/t\right)$ la cumple.


\begin{Coro}[Ley Fuerte de los Grandes N\'umeros para Procesos de Renovaci\'on]
Si $N\left(t\right)$ es un proceso de renovaci\'on cuyos tiempos de inter-renovaci\'on tienen media $\mu\leq\infty$, entonces
\begin{eqnarray}
t^{-1}N\left(t\right)\rightarrow 1/\mu,\textrm{ c.s. cuando }t\rightarrow\infty.
\end{eqnarray}

\end{Coro}


Considerar el proceso estoc\'astico de valores reales $\left\{Z\left(t\right):t\geq0\right\}$ en el mismo espacio de probabilidad que $N\left(t\right)$

\begin{Def}
Para el proceso $\left\{Z\left(t\right):t\geq0\right\}$ se define la fluctuaci\'on m\'axima de $Z\left(t\right)$ en el intervalo $\left(T_{n-1},T_{n}\right]$:
\begin{eqnarray*}
M_{n}=\sup_{T_{n-1}<t\leq T_{n}}|Z\left(t\right)-Z\left(T_{n-1}\right)|
\end{eqnarray*}
\end{Def}

\begin{Teo}
Sup\'ongase que $n^{-1}T_{n}\rightarrow\mu$ c.s. cuando $n\rightarrow\infty$, donde $\mu\leq\infty$ es una constante o variable aleatoria. Sea $a$ una constante o variable aleatoria que puede ser infinita cuando $\mu$ es finita, y considere las expresiones l\'imite:
\begin{eqnarray}
lim_{n\rightarrow\infty}n^{-1}Z\left(T_{n}\right)&=&a,\textrm{ c.s.}\\
lim_{t\rightarrow\infty}t^{-1}Z\left(t\right)&=&a/\mu,\textrm{ c.s.}
\end{eqnarray}
La segunda expresi\'on implica la primera. Conversamente, la primera implica la segunda si el proceso $Z\left(t\right)$ es creciente, o si $lim_{n\rightarrow\infty}n^{-1}M_{n}=0$ c.s.
\end{Teo}

\begin{Coro}
Si $N\left(t\right)$ es un proceso de renovaci\'on, y $\left(Z\left(T_{n}\right)-Z\left(T_{n-1}\right),M_{n}\right)$, para $n\geq1$, son variables aleatorias independientes e id\'enticamente distribuidas con media finita, entonces,
\begin{eqnarray}
lim_{t\rightarrow\infty}t^{-1}Z\left(t\right)\rightarrow\frac{\esp\left[Z\left(T_{1}\right)-Z\left(T_{0}\right)\right]}{\esp\left[T_{1}\right]},\textrm{ c.s. cuando  }t\rightarrow\infty.
\end{eqnarray}
\end{Coro}


%___________________________________________________________________________________________
%
%\subsection{Propiedades de los Procesos de Renovaci\'on}
%___________________________________________________________________________________________
%

Los tiempos $T_{n}$ est\'an relacionados con los conteos de $N\left(t\right)$ por

\begin{eqnarray*}
\left\{N\left(t\right)\geq n\right\}&=&\left\{T_{n}\leq t\right\}\\
T_{N\left(t\right)}\leq &t&<T_{N\left(t\right)+1},
\end{eqnarray*}

adem\'as $N\left(T_{n}\right)=n$, y 

\begin{eqnarray*}
N\left(t\right)=\max\left\{n:T_{n}\leq t\right\}=\min\left\{n:T_{n+1}>t\right\}
\end{eqnarray*}

Por propiedades de la convoluci\'on se sabe que

\begin{eqnarray*}
P\left\{T_{n}\leq t\right\}=F^{n\star}\left(t\right)
\end{eqnarray*}
que es la $n$-\'esima convoluci\'on de $F$. Entonces 

\begin{eqnarray*}
\left\{N\left(t\right)\geq n\right\}&=&\left\{T_{n}\leq t\right\}\\
P\left\{N\left(t\right)\leq n\right\}&=&1-F^{\left(n+1\right)\star}\left(t\right)
\end{eqnarray*}

Adem\'as usando el hecho de que $\esp\left[N\left(t\right)\right]=\sum_{n=1}^{\infty}P\left\{N\left(t\right)\geq n\right\}$
se tiene que

\begin{eqnarray*}
\esp\left[N\left(t\right)\right]=\sum_{n=1}^{\infty}F^{n\star}\left(t\right)
\end{eqnarray*}

\begin{Prop}
Para cada $t\geq0$, la funci\'on generadora de momentos $\esp\left[e^{\alpha N\left(t\right)}\right]$ existe para alguna $\alpha$ en una vecindad del 0, y de aqu\'i que $\esp\left[N\left(t\right)^{m}\right]<\infty$, para $m\geq1$.
\end{Prop}


\begin{Note}
Si el primer tiempo de renovaci\'on $\xi_{1}$ no tiene la misma distribuci\'on que el resto de las $\xi_{n}$, para $n\geq2$, a $N\left(t\right)$ se le llama Proceso de Renovaci\'on retardado, donde si $\xi$ tiene distribuci\'on $G$, entonces el tiempo $T_{n}$ de la $n$-\'esima renovaci\'on tiene distribuci\'on $G\star F^{\left(n-1\right)\star}\left(t\right)$
\end{Note}


\begin{Teo}
Para una constante $\mu\leq\infty$ ( o variable aleatoria), las siguientes expresiones son equivalentes:

\begin{eqnarray}
lim_{n\rightarrow\infty}n^{-1}T_{n}&=&\mu,\textrm{ c.s.}\\
lim_{t\rightarrow\infty}t^{-1}N\left(t\right)&=&1/\mu,\textrm{ c.s.}
\end{eqnarray}
\end{Teo}


Es decir, $T_{n}$ satisface la Ley Fuerte de los Grandes N\'umeros s\'i y s\'olo s\'i $N\left/t\right)$ la cumple.


\begin{Coro}[Ley Fuerte de los Grandes N\'umeros para Procesos de Renovaci\'on]
Si $N\left(t\right)$ es un proceso de renovaci\'on cuyos tiempos de inter-renovaci\'on tienen media $\mu\leq\infty$, entonces
\begin{eqnarray}
t^{-1}N\left(t\right)\rightarrow 1/\mu,\textrm{ c.s. cuando }t\rightarrow\infty.
\end{eqnarray}

\end{Coro}


Considerar el proceso estoc\'astico de valores reales $\left\{Z\left(t\right):t\geq0\right\}$ en el mismo espacio de probabilidad que $N\left(t\right)$

\begin{Def}
Para el proceso $\left\{Z\left(t\right):t\geq0\right\}$ se define la fluctuaci\'on m\'axima de $Z\left(t\right)$ en el intervalo $\left(T_{n-1},T_{n}\right]$:
\begin{eqnarray*}
M_{n}=\sup_{T_{n-1}<t\leq T_{n}}|Z\left(t\right)-Z\left(T_{n-1}\right)|
\end{eqnarray*}
\end{Def}

\begin{Teo}
Sup\'ongase que $n^{-1}T_{n}\rightarrow\mu$ c.s. cuando $n\rightarrow\infty$, donde $\mu\leq\infty$ es una constante o variable aleatoria. Sea $a$ una constante o variable aleatoria que puede ser infinita cuando $\mu$ es finita, y considere las expresiones l\'imite:
\begin{eqnarray}
lim_{n\rightarrow\infty}n^{-1}Z\left(T_{n}\right)&=&a,\textrm{ c.s.}\\
lim_{t\rightarrow\infty}t^{-1}Z\left(t\right)&=&a/\mu,\textrm{ c.s.}
\end{eqnarray}
La segunda expresi\'on implica la primera. Conversamente, la primera implica la segunda si el proceso $Z\left(t\right)$ es creciente, o si $lim_{n\rightarrow\infty}n^{-1}M_{n}=0$ c.s.
\end{Teo}

\begin{Coro}
Si $N\left(t\right)$ es un proceso de renovaci\'on, y $\left(Z\left(T_{n}\right)-Z\left(T_{n-1}\right),M_{n}\right)$, para $n\geq1$, son variables aleatorias independientes e id\'enticamente distribuidas con media finita, entonces,
\begin{eqnarray}
lim_{t\rightarrow\infty}t^{-1}Z\left(t\right)\rightarrow\frac{\esp\left[Z\left(T_{1}\right)-Z\left(T_{0}\right)\right]}{\esp\left[T_{1}\right]},\textrm{ c.s. cuando  }t\rightarrow\infty.
\end{eqnarray}
\end{Coro}

%___________________________________________________________________________________________
%
%\subsection{Propiedades de los Procesos de Renovaci\'on}
%___________________________________________________________________________________________
%

Los tiempos $T_{n}$ est\'an relacionados con los conteos de $N\left(t\right)$ por

\begin{eqnarray*}
\left\{N\left(t\right)\geq n\right\}&=&\left\{T_{n}\leq t\right\}\\
T_{N\left(t\right)}\leq &t&<T_{N\left(t\right)+1},
\end{eqnarray*}

adem\'as $N\left(T_{n}\right)=n$, y 

\begin{eqnarray*}
N\left(t\right)=\max\left\{n:T_{n}\leq t\right\}=\min\left\{n:T_{n+1}>t\right\}
\end{eqnarray*}

Por propiedades de la convoluci\'on se sabe que

\begin{eqnarray*}
P\left\{T_{n}\leq t\right\}=F^{n\star}\left(t\right)
\end{eqnarray*}
que es la $n$-\'esima convoluci\'on de $F$. Entonces 

\begin{eqnarray*}
\left\{N\left(t\right)\geq n\right\}&=&\left\{T_{n}\leq t\right\}\\
P\left\{N\left(t\right)\leq n\right\}&=&1-F^{\left(n+1\right)\star}\left(t\right)
\end{eqnarray*}

Adem\'as usando el hecho de que $\esp\left[N\left(t\right)\right]=\sum_{n=1}^{\infty}P\left\{N\left(t\right)\geq n\right\}$
se tiene que

\begin{eqnarray*}
\esp\left[N\left(t\right)\right]=\sum_{n=1}^{\infty}F^{n\star}\left(t\right)
\end{eqnarray*}

\begin{Prop}
Para cada $t\geq0$, la funci\'on generadora de momentos $\esp\left[e^{\alpha N\left(t\right)}\right]$ existe para alguna $\alpha$ en una vecindad del 0, y de aqu\'i que $\esp\left[N\left(t\right)^{m}\right]<\infty$, para $m\geq1$.
\end{Prop}


\begin{Note}
Si el primer tiempo de renovaci\'on $\xi_{1}$ no tiene la misma distribuci\'on que el resto de las $\xi_{n}$, para $n\geq2$, a $N\left(t\right)$ se le llama Proceso de Renovaci\'on retardado, donde si $\xi$ tiene distribuci\'on $G$, entonces el tiempo $T_{n}$ de la $n$-\'esima renovaci\'on tiene distribuci\'on $G\star F^{\left(n-1\right)\star}\left(t\right)$
\end{Note}


\begin{Teo}
Para una constante $\mu\leq\infty$ ( o variable aleatoria), las siguientes expresiones son equivalentes:

\begin{eqnarray}
lim_{n\rightarrow\infty}n^{-1}T_{n}&=&\mu,\textrm{ c.s.}\\
lim_{t\rightarrow\infty}t^{-1}N\left(t\right)&=&1/\mu,\textrm{ c.s.}
\end{eqnarray}
\end{Teo}


Es decir, $T_{n}$ satisface la Ley Fuerte de los Grandes N\'umeros s\'i y s\'olo s\'i $N\left/t\right)$ la cumple.


\begin{Coro}[Ley Fuerte de los Grandes N\'umeros para Procesos de Renovaci\'on]
Si $N\left(t\right)$ es un proceso de renovaci\'on cuyos tiempos de inter-renovaci\'on tienen media $\mu\leq\infty$, entonces
\begin{eqnarray}
t^{-1}N\left(t\right)\rightarrow 1/\mu,\textrm{ c.s. cuando }t\rightarrow\infty.
\end{eqnarray}

\end{Coro}


Considerar el proceso estoc\'astico de valores reales $\left\{Z\left(t\right):t\geq0\right\}$ en el mismo espacio de probabilidad que $N\left(t\right)$

\begin{Def}
Para el proceso $\left\{Z\left(t\right):t\geq0\right\}$ se define la fluctuaci\'on m\'axima de $Z\left(t\right)$ en el intervalo $\left(T_{n-1},T_{n}\right]$:
\begin{eqnarray*}
M_{n}=\sup_{T_{n-1}<t\leq T_{n}}|Z\left(t\right)-Z\left(T_{n-1}\right)|
\end{eqnarray*}
\end{Def}

\begin{Teo}
Sup\'ongase que $n^{-1}T_{n}\rightarrow\mu$ c.s. cuando $n\rightarrow\infty$, donde $\mu\leq\infty$ es una constante o variable aleatoria. Sea $a$ una constante o variable aleatoria que puede ser infinita cuando $\mu$ es finita, y considere las expresiones l\'imite:
\begin{eqnarray}
lim_{n\rightarrow\infty}n^{-1}Z\left(T_{n}\right)&=&a,\textrm{ c.s.}\\
lim_{t\rightarrow\infty}t^{-1}Z\left(t\right)&=&a/\mu,\textrm{ c.s.}
\end{eqnarray}
La segunda expresi\'on implica la primera. Conversamente, la primera implica la segunda si el proceso $Z\left(t\right)$ es creciente, o si $lim_{n\rightarrow\infty}n^{-1}M_{n}=0$ c.s.
\end{Teo}

\begin{Coro}
Si $N\left(t\right)$ es un proceso de renovaci\'on, y $\left(Z\left(T_{n}\right)-Z\left(T_{n-1}\right),M_{n}\right)$, para $n\geq1$, son variables aleatorias independientes e id\'enticamente distribuidas con media finita, entonces,
\begin{eqnarray}
lim_{t\rightarrow\infty}t^{-1}Z\left(t\right)\rightarrow\frac{\esp\left[Z\left(T_{1}\right)-Z\left(T_{0}\right)\right]}{\esp\left[T_{1}\right]},\textrm{ c.s. cuando  }t\rightarrow\infty.
\end{eqnarray}
\end{Coro}
%___________________________________________________________________________________________
%
%\subsection{Propiedades de los Procesos de Renovaci\'on}
%___________________________________________________________________________________________
%

Los tiempos $T_{n}$ est\'an relacionados con los conteos de $N\left(t\right)$ por

\begin{eqnarray*}
\left\{N\left(t\right)\geq n\right\}&=&\left\{T_{n}\leq t\right\}\\
T_{N\left(t\right)}\leq &t&<T_{N\left(t\right)+1},
\end{eqnarray*}

adem\'as $N\left(T_{n}\right)=n$, y 

\begin{eqnarray*}
N\left(t\right)=\max\left\{n:T_{n}\leq t\right\}=\min\left\{n:T_{n+1}>t\right\}
\end{eqnarray*}

Por propiedades de la convoluci\'on se sabe que

\begin{eqnarray*}
P\left\{T_{n}\leq t\right\}=F^{n\star}\left(t\right)
\end{eqnarray*}
que es la $n$-\'esima convoluci\'on de $F$. Entonces 

\begin{eqnarray*}
\left\{N\left(t\right)\geq n\right\}&=&\left\{T_{n}\leq t\right\}\\
P\left\{N\left(t\right)\leq n\right\}&=&1-F^{\left(n+1\right)\star}\left(t\right)
\end{eqnarray*}

Adem\'as usando el hecho de que $\esp\left[N\left(t\right)\right]=\sum_{n=1}^{\infty}P\left\{N\left(t\right)\geq n\right\}$
se tiene que

\begin{eqnarray*}
\esp\left[N\left(t\right)\right]=\sum_{n=1}^{\infty}F^{n\star}\left(t\right)
\end{eqnarray*}

\begin{Prop}
Para cada $t\geq0$, la funci\'on generadora de momentos $\esp\left[e^{\alpha N\left(t\right)}\right]$ existe para alguna $\alpha$ en una vecindad del 0, y de aqu\'i que $\esp\left[N\left(t\right)^{m}\right]<\infty$, para $m\geq1$.
\end{Prop}


\begin{Note}
Si el primer tiempo de renovaci\'on $\xi_{1}$ no tiene la misma distribuci\'on que el resto de las $\xi_{n}$, para $n\geq2$, a $N\left(t\right)$ se le llama Proceso de Renovaci\'on retardado, donde si $\xi$ tiene distribuci\'on $G$, entonces el tiempo $T_{n}$ de la $n$-\'esima renovaci\'on tiene distribuci\'on $G\star F^{\left(n-1\right)\star}\left(t\right)$
\end{Note}


\begin{Teo}
Para una constante $\mu\leq\infty$ ( o variable aleatoria), las siguientes expresiones son equivalentes:

\begin{eqnarray}
lim_{n\rightarrow\infty}n^{-1}T_{n}&=&\mu,\textrm{ c.s.}\\
lim_{t\rightarrow\infty}t^{-1}N\left(t\right)&=&1/\mu,\textrm{ c.s.}
\end{eqnarray}
\end{Teo}


Es decir, $T_{n}$ satisface la Ley Fuerte de los Grandes N\'umeros s\'i y s\'olo s\'i $N\left/t\right)$ la cumple.


\begin{Coro}[Ley Fuerte de los Grandes N\'umeros para Procesos de Renovaci\'on]
Si $N\left(t\right)$ es un proceso de renovaci\'on cuyos tiempos de inter-renovaci\'on tienen media $\mu\leq\infty$, entonces
\begin{eqnarray}
t^{-1}N\left(t\right)\rightarrow 1/\mu,\textrm{ c.s. cuando }t\rightarrow\infty.
\end{eqnarray}

\end{Coro}


Considerar el proceso estoc\'astico de valores reales $\left\{Z\left(t\right):t\geq0\right\}$ en el mismo espacio de probabilidad que $N\left(t\right)$

\begin{Def}
Para el proceso $\left\{Z\left(t\right):t\geq0\right\}$ se define la fluctuaci\'on m\'axima de $Z\left(t\right)$ en el intervalo $\left(T_{n-1},T_{n}\right]$:
\begin{eqnarray*}
M_{n}=\sup_{T_{n-1}<t\leq T_{n}}|Z\left(t\right)-Z\left(T_{n-1}\right)|
\end{eqnarray*}
\end{Def}

\begin{Teo}
Sup\'ongase que $n^{-1}T_{n}\rightarrow\mu$ c.s. cuando $n\rightarrow\infty$, donde $\mu\leq\infty$ es una constante o variable aleatoria. Sea $a$ una constante o variable aleatoria que puede ser infinita cuando $\mu$ es finita, y considere las expresiones l\'imite:
\begin{eqnarray}
lim_{n\rightarrow\infty}n^{-1}Z\left(T_{n}\right)&=&a,\textrm{ c.s.}\\
lim_{t\rightarrow\infty}t^{-1}Z\left(t\right)&=&a/\mu,\textrm{ c.s.}
\end{eqnarray}
La segunda expresi\'on implica la primera. Conversamente, la primera implica la segunda si el proceso $Z\left(t\right)$ es creciente, o si $lim_{n\rightarrow\infty}n^{-1}M_{n}=0$ c.s.
\end{Teo}

\begin{Coro}
Si $N\left(t\right)$ es un proceso de renovaci\'on, y $\left(Z\left(T_{n}\right)-Z\left(T_{n-1}\right),M_{n}\right)$, para $n\geq1$, son variables aleatorias independientes e id\'enticamente distribuidas con media finita, entonces,
\begin{eqnarray}
lim_{t\rightarrow\infty}t^{-1}Z\left(t\right)\rightarrow\frac{\esp\left[Z\left(T_{1}\right)-Z\left(T_{0}\right)\right]}{\esp\left[T_{1}\right]},\textrm{ c.s. cuando  }t\rightarrow\infty.
\end{eqnarray}
\end{Coro}


%___________________________________________________________________________________________
%
%\subsection*{Funci\'on de Renovaci\'on}
%___________________________________________________________________________________________
%


\begin{Def}
Sea $h\left(t\right)$ funci\'on de valores reales en $\rea$ acotada en intervalos finitos e igual a cero para $t<0$ La ecuaci\'on de renovaci\'on para $h\left(t\right)$ y la distribuci\'on $F$ es

\begin{eqnarray}\label{Ec.Renovacion}
H\left(t\right)=h\left(t\right)+\int_{\left[0,t\right]}H\left(t-s\right)dF\left(s\right)\textrm{,    }t\geq0,
\end{eqnarray}
donde $H\left(t\right)$ es una funci\'on de valores reales. Esto es $H=h+F\star H$. Decimos que $H\left(t\right)$ es soluci\'on de esta ecuaci\'on si satisface la ecuaci\'on, y es acotada en intervalos finitos e iguales a cero para $t<0$.
\end{Def}

\begin{Prop}
La funci\'on $U\star h\left(t\right)$ es la \'unica soluci\'on de la ecuaci\'on de renovaci\'on (\ref{Ec.Renovacion}).
\end{Prop}

\begin{Teo}[Teorema Renovaci\'on Elemental]
\begin{eqnarray*}
t^{-1}U\left(t\right)\rightarrow 1/\mu\textrm{,    cuando }t\rightarrow\infty.
\end{eqnarray*}
\end{Teo}

%___________________________________________________________________________________________
%
%\subsection{Funci\'on de Renovaci\'on}
%___________________________________________________________________________________________
%


Sup\'ongase que $N\left(t\right)$ es un proceso de renovaci\'on con distribuci\'on $F$ con media finita $\mu$.

\begin{Def}
La funci\'on de renovaci\'on asociada con la distribuci\'on $F$, del proceso $N\left(t\right)$, es
\begin{eqnarray*}
U\left(t\right)=\sum_{n=1}^{\infty}F^{n\star}\left(t\right),\textrm{   }t\geq0,
\end{eqnarray*}
donde $F^{0\star}\left(t\right)=\indora\left(t\geq0\right)$.
\end{Def}


\begin{Prop}
Sup\'ongase que la distribuci\'on de inter-renovaci\'on $F$ tiene densidad $f$. Entonces $U\left(t\right)$ tambi\'en tiene densidad, para $t>0$, y es $U^{'}\left(t\right)=\sum_{n=0}^{\infty}f^{n\star}\left(t\right)$. Adem\'as
\begin{eqnarray*}
\prob\left\{N\left(t\right)>N\left(t-\right)\right\}=0\textrm{,   }t\geq0.
\end{eqnarray*}
\end{Prop}

\begin{Def}
La Transformada de Laplace-Stieljes de $F$ est\'a dada por

\begin{eqnarray*}
\hat{F}\left(\alpha\right)=\int_{\rea_{+}}e^{-\alpha t}dF\left(t\right)\textrm{,  }\alpha\geq0.
\end{eqnarray*}
\end{Def}

Entonces

\begin{eqnarray*}
\hat{U}\left(\alpha\right)=\sum_{n=0}^{\infty}\hat{F^{n\star}}\left(\alpha\right)=\sum_{n=0}^{\infty}\hat{F}\left(\alpha\right)^{n}=\frac{1}{1-\hat{F}\left(\alpha\right)}.
\end{eqnarray*}


\begin{Prop}
La Transformada de Laplace $\hat{U}\left(\alpha\right)$ y $\hat{F}\left(\alpha\right)$ determina una a la otra de manera \'unica por la relaci\'on $\hat{U}\left(\alpha\right)=\frac{1}{1-\hat{F}\left(\alpha\right)}$.
\end{Prop}


\begin{Note}
Un proceso de renovaci\'on $N\left(t\right)$ cuyos tiempos de inter-renovaci\'on tienen media finita, es un proceso Poisson con tasa $\lambda$ si y s\'olo s\'i $\esp\left[U\left(t\right)\right]=\lambda t$, para $t\geq0$.
\end{Note}


\begin{Teo}
Sea $N\left(t\right)$ un proceso puntual simple con puntos de localizaci\'on $T_{n}$ tal que $\eta\left(t\right)=\esp\left[N\left(\right)\right]$ es finita para cada $t$. Entonces para cualquier funci\'on $f:\rea_{+}\rightarrow\rea$,
\begin{eqnarray*}
\esp\left[\sum_{n=1}^{N\left(\right)}f\left(T_{n}\right)\right]=\int_{\left(0,t\right]}f\left(s\right)d\eta\left(s\right)\textrm{,  }t\geq0,
\end{eqnarray*}
suponiendo que la integral exista. Adem\'as si $X_{1},X_{2},\ldots$ son variables aleatorias definidas en el mismo espacio de probabilidad que el proceso $N\left(t\right)$ tal que $\esp\left[X_{n}|T_{n}=s\right]=f\left(s\right)$, independiente de $n$. Entonces
\begin{eqnarray*}
\esp\left[\sum_{n=1}^{N\left(t\right)}X_{n}\right]=\int_{\left(0,t\right]}f\left(s\right)d\eta\left(s\right)\textrm{,  }t\geq0,
\end{eqnarray*} 
suponiendo que la integral exista. 
\end{Teo}

\begin{Coro}[Identidad de Wald para Renovaciones]
Para el proceso de renovaci\'on $N\left(t\right)$,
\begin{eqnarray*}
\esp\left[T_{N\left(t\right)+1}\right]=\mu\esp\left[N\left(t\right)+1\right]\textrm{,  }t\geq0,
\end{eqnarray*}  
\end{Coro}

%______________________________________________________________________
%\subsection{Procesos de Renovaci\'on}
%______________________________________________________________________

\begin{Def}\label{Def.Tn}
Sean $0\leq T_{1}\leq T_{2}\leq \ldots$ son tiempos aleatorios infinitos en los cuales ocurren ciertos eventos. El n\'umero de tiempos $T_{n}$ en el intervalo $\left[0,t\right)$ es

\begin{eqnarray}
N\left(t\right)=\sum_{n=1}^{\infty}\indora\left(T_{n}\leq t\right),
\end{eqnarray}
para $t\geq0$.
\end{Def}

Si se consideran los puntos $T_{n}$ como elementos de $\rea_{+}$, y $N\left(t\right)$ es el n\'umero de puntos en $\rea$. El proceso denotado por $\left\{N\left(t\right):t\geq0\right\}$, denotado por $N\left(t\right)$, es un proceso puntual en $\rea_{+}$. Los $T_{n}$ son los tiempos de ocurrencia, el proceso puntual $N\left(t\right)$ es simple si su n\'umero de ocurrencias son distintas: $0<T_{1}<T_{2}<\ldots$ casi seguramente.

\begin{Def}
Un proceso puntual $N\left(t\right)$ es un proceso de renovaci\'on si los tiempos de interocurrencia $\xi_{n}=T_{n}-T_{n-1}$, para $n\geq1$, son independientes e identicamente distribuidos con distribuci\'on $F$, donde $F\left(0\right)=0$ y $T_{0}=0$. Los $T_{n}$ son llamados tiempos de renovaci\'on, referente a la independencia o renovaci\'on de la informaci\'on estoc\'astica en estos tiempos. Los $\xi_{n}$ son los tiempos de inter-renovaci\'on, y $N\left(t\right)$ es el n\'umero de renovaciones en el intervalo $\left[0,t\right)$
\end{Def}


\begin{Note}
Para definir un proceso de renovaci\'on para cualquier contexto, solamente hay que especificar una distribuci\'on $F$, con $F\left(0\right)=0$, para los tiempos de inter-renovaci\'on. La funci\'on $F$ en turno degune las otra variables aleatorias. De manera formal, existe un espacio de probabilidad y una sucesi\'on de variables aleatorias $\xi_{1},\xi_{2},\ldots$ definidas en este con distribuci\'on $F$. Entonces las otras cantidades son $T_{n}=\sum_{k=1}^{n}\xi_{k}$ y $N\left(t\right)=\sum_{n=1}^{\infty}\indora\left(T_{n}\leq t\right)$, donde $T_{n}\rightarrow\infty$ casi seguramente por la Ley Fuerte de los Grandes Números.
\end{Note}

\begin{Def}\label{Def.Tn}
Sean $0\leq T_{1}\leq T_{2}\leq \ldots$ son tiempos aleatorios infinitos en los cuales ocurren ciertos eventos. El n\'umero de tiempos $T_{n}$ en el intervalo $\left[0,t\right)$ es

\begin{eqnarray}
N\left(t\right)=\sum_{n=1}^{\infty}\indora\left(T_{n}\leq t\right),
\end{eqnarray}
para $t\geq0$.
\end{Def}

Si se consideran los puntos $T_{n}$ como elementos de $\rea_{+}$, y $N\left(t\right)$ es el n\'umero de puntos en $\rea$. El proceso denotado por $\left\{N\left(t\right):t\geq0\right\}$, denotado por $N\left(t\right)$, es un proceso puntual en $\rea_{+}$. Los $T_{n}$ son los tiempos de ocurrencia, el proceso puntual $N\left(t\right)$ es simple si su n\'umero de ocurrencias son distintas: $0<T_{1}<T_{2}<\ldots$ casi seguramente.

\begin{Def}
Un proceso puntual $N\left(t\right)$ es un proceso de renovaci\'on si los tiempos de interocurrencia $\xi_{n}=T_{n}-T_{n-1}$, para $n\geq1$, son independientes e identicamente distribuidos con distribuci\'on $F$, donde $F\left(0\right)=0$ y $T_{0}=0$. Los $T_{n}$ son llamados tiempos de renovaci\'on, referente a la independencia o renovaci\'on de la informaci\'on estoc\'astica en estos tiempos. Los $\xi_{n}$ son los tiempos de inter-renovaci\'on, y $N\left(t\right)$ es el n\'umero de renovaciones en el intervalo $\left[0,t\right)$
\end{Def}


\begin{Note}
Para definir un proceso de renovaci\'on para cualquier contexto, solamente hay que especificar una distribuci\'on $F$, con $F\left(0\right)=0$, para los tiempos de inter-renovaci\'on. La funci\'on $F$ en turno degune las otra variables aleatorias. De manera formal, existe un espacio de probabilidad y una sucesi\'on de variables aleatorias $\xi_{1},\xi_{2},\ldots$ definidas en este con distribuci\'on $F$. Entonces las otras cantidades son $T_{n}=\sum_{k=1}^{n}\xi_{k}$ y $N\left(t\right)=\sum_{n=1}^{\infty}\indora\left(T_{n}\leq t\right)$, donde $T_{n}\rightarrow\infty$ casi seguramente por la Ley Fuerte de los Grandes N\'umeros.
\end{Note}







Los tiempos $T_{n}$ est\'an relacionados con los conteos de $N\left(t\right)$ por

\begin{eqnarray*}
\left\{N\left(t\right)\geq n\right\}&=&\left\{T_{n}\leq t\right\}\\
T_{N\left(t\right)}\leq &t&<T_{N\left(t\right)+1},
\end{eqnarray*}

adem\'as $N\left(T_{n}\right)=n$, y 

\begin{eqnarray*}
N\left(t\right)=\max\left\{n:T_{n}\leq t\right\}=\min\left\{n:T_{n+1}>t\right\}
\end{eqnarray*}

Por propiedades de la convoluci\'on se sabe que

\begin{eqnarray*}
P\left\{T_{n}\leq t\right\}=F^{n\star}\left(t\right)
\end{eqnarray*}
que es la $n$-\'esima convoluci\'on de $F$. Entonces 

\begin{eqnarray*}
\left\{N\left(t\right)\geq n\right\}&=&\left\{T_{n}\leq t\right\}\\
P\left\{N\left(t\right)\leq n\right\}&=&1-F^{\left(n+1\right)\star}\left(t\right)
\end{eqnarray*}

Adem\'as usando el hecho de que $\esp\left[N\left(t\right)\right]=\sum_{n=1}^{\infty}P\left\{N\left(t\right)\geq n\right\}$
se tiene que

\begin{eqnarray*}
\esp\left[N\left(t\right)\right]=\sum_{n=1}^{\infty}F^{n\star}\left(t\right)
\end{eqnarray*}

\begin{Prop}
Para cada $t\geq0$, la funci\'on generadora de momentos $\esp\left[e^{\alpha N\left(t\right)}\right]$ existe para alguna $\alpha$ en una vecindad del 0, y de aqu\'i que $\esp\left[N\left(t\right)^{m}\right]<\infty$, para $m\geq1$.
\end{Prop}

\begin{Ejem}[\textbf{Proceso Poisson}]

Suponga que se tienen tiempos de inter-renovaci\'on \textit{i.i.d.} del proceso de renovaci\'on $N\left(t\right)$ tienen distribuci\'on exponencial $F\left(t\right)=q-e^{-\lambda t}$ con tasa $\lambda$. Entonces $N\left(t\right)$ es un proceso Poisson con tasa $\lambda$.

\end{Ejem}


\begin{Note}
Si el primer tiempo de renovaci\'on $\xi_{1}$ no tiene la misma distribuci\'on que el resto de las $\xi_{n}$, para $n\geq2$, a $N\left(t\right)$ se le llama Proceso de Renovaci\'on retardado, donde si $\xi$ tiene distribuci\'on $G$, entonces el tiempo $T_{n}$ de la $n$-\'esima renovaci\'on tiene distribuci\'on $G\star F^{\left(n-1\right)\star}\left(t\right)$
\end{Note}


\begin{Teo}
Para una constante $\mu\leq\infty$ ( o variable aleatoria), las siguientes expresiones son equivalentes:

\begin{eqnarray}
lim_{n\rightarrow\infty}n^{-1}T_{n}&=&\mu,\textrm{ c.s.}\\
lim_{t\rightarrow\infty}t^{-1}N\left(t\right)&=&1/\mu,\textrm{ c.s.}
\end{eqnarray}
\end{Teo}


Es decir, $T_{n}$ satisface la Ley Fuerte de los Grandes N\'umeros s\'i y s\'olo s\'i $N\left/t\right)$ la cumple.


\begin{Coro}[Ley Fuerte de los Grandes N\'umeros para Procesos de Renovaci\'on]
Si $N\left(t\right)$ es un proceso de renovaci\'on cuyos tiempos de inter-renovaci\'on tienen media $\mu\leq\infty$, entonces
\begin{eqnarray}
t^{-1}N\left(t\right)\rightarrow 1/\mu,\textrm{ c.s. cuando }t\rightarrow\infty.
\end{eqnarray}

\end{Coro}


Considerar el proceso estoc\'astico de valores reales $\left\{Z\left(t\right):t\geq0\right\}$ en el mismo espacio de probabilidad que $N\left(t\right)$

\begin{Def}
Para el proceso $\left\{Z\left(t\right):t\geq0\right\}$ se define la fluctuaci\'on m\'axima de $Z\left(t\right)$ en el intervalo $\left(T_{n-1},T_{n}\right]$:
\begin{eqnarray*}
M_{n}=\sup_{T_{n-1}<t\leq T_{n}}|Z\left(t\right)-Z\left(T_{n-1}\right)|
\end{eqnarray*}
\end{Def}

\begin{Teo}
Sup\'ongase que $n^{-1}T_{n}\rightarrow\mu$ c.s. cuando $n\rightarrow\infty$, donde $\mu\leq\infty$ es una constante o variable aleatoria. Sea $a$ una constante o variable aleatoria que puede ser infinita cuando $\mu$ es finita, y considere las expresiones l\'imite:
\begin{eqnarray}
lim_{n\rightarrow\infty}n^{-1}Z\left(T_{n}\right)&=&a,\textrm{ c.s.}\\
lim_{t\rightarrow\infty}t^{-1}Z\left(t\right)&=&a/\mu,\textrm{ c.s.}
\end{eqnarray}
La segunda expresi\'on implica la primera. Conversamente, la primera implica la segunda si el proceso $Z\left(t\right)$ es creciente, o si $lim_{n\rightarrow\infty}n^{-1}M_{n}=0$ c.s.
\end{Teo}

\begin{Coro}
Si $N\left(t\right)$ es un proceso de renovaci\'on, y $\left(Z\left(T_{n}\right)-Z\left(T_{n-1}\right),M_{n}\right)$, para $n\geq1$, son variables aleatorias independientes e id\'enticamente distribuidas con media finita, entonces,
\begin{eqnarray}
lim_{t\rightarrow\infty}t^{-1}Z\left(t\right)\rightarrow\frac{\esp\left[Z\left(T_{1}\right)-Z\left(T_{0}\right)\right]}{\esp\left[T_{1}\right]},\textrm{ c.s. cuando  }t\rightarrow\infty.
\end{eqnarray}
\end{Coro}


Sup\'ongase que $N\left(t\right)$ es un proceso de renovaci\'on con distribuci\'on $F$ con media finita $\mu$.

\begin{Def}
La funci\'on de renovaci\'on asociada con la distribuci\'on $F$, del proceso $N\left(t\right)$, es
\begin{eqnarray*}
U\left(t\right)=\sum_{n=1}^{\infty}F^{n\star}\left(t\right),\textrm{   }t\geq0,
\end{eqnarray*}
donde $F^{0\star}\left(t\right)=\indora\left(t\geq0\right)$.
\end{Def}


\begin{Prop}
Sup\'ongase que la distribuci\'on de inter-renovaci\'on $F$ tiene densidad $f$. Entonces $U\left(t\right)$ tambi\'en tiene densidad, para $t>0$, y es $U^{'}\left(t\right)=\sum_{n=0}^{\infty}f^{n\star}\left(t\right)$. Adem\'as
\begin{eqnarray*}
\prob\left\{N\left(t\right)>N\left(t-\right)\right\}=0\textrm{,   }t\geq0.
\end{eqnarray*}
\end{Prop}

\begin{Def}
La Transformada de Laplace-Stieljes de $F$ est\'a dada por

\begin{eqnarray*}
\hat{F}\left(\alpha\right)=\int_{\rea_{+}}e^{-\alpha t}dF\left(t\right)\textrm{,  }\alpha\geq0.
\end{eqnarray*}
\end{Def}

Entonces

\begin{eqnarray*}
\hat{U}\left(\alpha\right)=\sum_{n=0}^{\infty}\hat{F^{n\star}}\left(\alpha\right)=\sum_{n=0}^{\infty}\hat{F}\left(\alpha\right)^{n}=\frac{1}{1-\hat{F}\left(\alpha\right)}.
\end{eqnarray*}


\begin{Prop}
La Transformada de Laplace $\hat{U}\left(\alpha\right)$ y $\hat{F}\left(\alpha\right)$ determina una a la otra de manera \'unica por la relaci\'on $\hat{U}\left(\alpha\right)=\frac{1}{1-\hat{F}\left(\alpha\right)}$.
\end{Prop}


\begin{Note}
Un proceso de renovaci\'on $N\left(t\right)$ cuyos tiempos de inter-renovaci\'on tienen media finita, es un proceso Poisson con tasa $\lambda$ si y s\'olo s\'i $\esp\left[U\left(t\right)\right]=\lambda t$, para $t\geq0$.
\end{Note}


\begin{Teo}
Sea $N\left(t\right)$ un proceso puntual simple con puntos de localizaci\'on $T_{n}$ tal que $\eta\left(t\right)=\esp\left[N\left(\right)\right]$ es finita para cada $t$. Entonces para cualquier funci\'on $f:\rea_{+}\rightarrow\rea$,
\begin{eqnarray*}
\esp\left[\sum_{n=1}^{N\left(\right)}f\left(T_{n}\right)\right]=\int_{\left(0,t\right]}f\left(s\right)d\eta\left(s\right)\textrm{,  }t\geq0,
\end{eqnarray*}
suponiendo que la integral exista. Adem\'as si $X_{1},X_{2},\ldots$ son variables aleatorias definidas en el mismo espacio de probabilidad que el proceso $N\left(t\right)$ tal que $\esp\left[X_{n}|T_{n}=s\right]=f\left(s\right)$, independiente de $n$. Entonces
\begin{eqnarray*}
\esp\left[\sum_{n=1}^{N\left(t\right)}X_{n}\right]=\int_{\left(0,t\right]}f\left(s\right)d\eta\left(s\right)\textrm{,  }t\geq0,
\end{eqnarray*} 
suponiendo que la integral exista. 
\end{Teo}

\begin{Coro}[Identidad de Wald para Renovaciones]
Para el proceso de renovaci\'on $N\left(t\right)$,
\begin{eqnarray*}
\esp\left[T_{N\left(t\right)+1}\right]=\mu\esp\left[N\left(t\right)+1\right]\textrm{,  }t\geq0,
\end{eqnarray*}  
\end{Coro}


\begin{Def}
Sea $h\left(t\right)$ funci\'on de valores reales en $\rea$ acotada en intervalos finitos e igual a cero para $t<0$ La ecuaci\'on de renovaci\'on para $h\left(t\right)$ y la distribuci\'on $F$ es

\begin{eqnarray}\label{Ec.Renovacion}
H\left(t\right)=h\left(t\right)+\int_{\left[0,t\right]}H\left(t-s\right)dF\left(s\right)\textrm{,    }t\geq0,
\end{eqnarray}
donde $H\left(t\right)$ es una funci\'on de valores reales. Esto es $H=h+F\star H$. Decimos que $H\left(t\right)$ es soluci\'on de esta ecuaci\'on si satisface la ecuaci\'on, y es acotada en intervalos finitos e iguales a cero para $t<0$.
\end{Def}

\begin{Prop}
La funci\'on $U\star h\left(t\right)$ es la \'unica soluci\'on de la ecuaci\'on de renovaci\'on (\ref{Ec.Renovacion}).
\end{Prop}

\begin{Teo}[Teorema Renovaci\'on Elemental]
\begin{eqnarray*}
t^{-1}U\left(t\right)\rightarrow 1/\mu\textrm{,    cuando }t\rightarrow\infty.
\end{eqnarray*}
\end{Teo}



Sup\'ongase que $N\left(t\right)$ es un proceso de renovaci\'on con distribuci\'on $F$ con media finita $\mu$.

\begin{Def}
La funci\'on de renovaci\'on asociada con la distribuci\'on $F$, del proceso $N\left(t\right)$, es
\begin{eqnarray*}
U\left(t\right)=\sum_{n=1}^{\infty}F^{n\star}\left(t\right),\textrm{   }t\geq0,
\end{eqnarray*}
donde $F^{0\star}\left(t\right)=\indora\left(t\geq0\right)$.
\end{Def}


\begin{Prop}
Sup\'ongase que la distribuci\'on de inter-renovaci\'on $F$ tiene densidad $f$. Entonces $U\left(t\right)$ tambi\'en tiene densidad, para $t>0$, y es $U^{'}\left(t\right)=\sum_{n=0}^{\infty}f^{n\star}\left(t\right)$. Adem\'as
\begin{eqnarray*}
\prob\left\{N\left(t\right)>N\left(t-\right)\right\}=0\textrm{,   }t\geq0.
\end{eqnarray*}
\end{Prop}

\begin{Def}
La Transformada de Laplace-Stieljes de $F$ est\'a dada por

\begin{eqnarray*}
\hat{F}\left(\alpha\right)=\int_{\rea_{+}}e^{-\alpha t}dF\left(t\right)\textrm{,  }\alpha\geq0.
\end{eqnarray*}
\end{Def}

Entonces

\begin{eqnarray*}
\hat{U}\left(\alpha\right)=\sum_{n=0}^{\infty}\hat{F^{n\star}}\left(\alpha\right)=\sum_{n=0}^{\infty}\hat{F}\left(\alpha\right)^{n}=\frac{1}{1-\hat{F}\left(\alpha\right)}.
\end{eqnarray*}


\begin{Prop}
La Transformada de Laplace $\hat{U}\left(\alpha\right)$ y $\hat{F}\left(\alpha\right)$ determina una a la otra de manera \'unica por la relaci\'on $\hat{U}\left(\alpha\right)=\frac{1}{1-\hat{F}\left(\alpha\right)}$.
\end{Prop}


\begin{Note}
Un proceso de renovaci\'on $N\left(t\right)$ cuyos tiempos de inter-renovaci\'on tienen media finita, es un proceso Poisson con tasa $\lambda$ si y s\'olo s\'i $\esp\left[U\left(t\right)\right]=\lambda t$, para $t\geq0$.
\end{Note}


\begin{Teo}
Sea $N\left(t\right)$ un proceso puntual simple con puntos de localizaci\'on $T_{n}$ tal que $\eta\left(t\right)=\esp\left[N\left(\right)\right]$ es finita para cada $t$. Entonces para cualquier funci\'on $f:\rea_{+}\rightarrow\rea$,
\begin{eqnarray*}
\esp\left[\sum_{n=1}^{N\left(\right)}f\left(T_{n}\right)\right]=\int_{\left(0,t\right]}f\left(s\right)d\eta\left(s\right)\textrm{,  }t\geq0,
\end{eqnarray*}
suponiendo que la integral exista. Adem\'as si $X_{1},X_{2},\ldots$ son variables aleatorias definidas en el mismo espacio de probabilidad que el proceso $N\left(t\right)$ tal que $\esp\left[X_{n}|T_{n}=s\right]=f\left(s\right)$, independiente de $n$. Entonces
\begin{eqnarray*}
\esp\left[\sum_{n=1}^{N\left(t\right)}X_{n}\right]=\int_{\left(0,t\right]}f\left(s\right)d\eta\left(s\right)\textrm{,  }t\geq0,
\end{eqnarray*} 
suponiendo que la integral exista. 
\end{Teo}

\begin{Coro}[Identidad de Wald para Renovaciones]
Para el proceso de renovaci\'on $N\left(t\right)$,
\begin{eqnarray*}
\esp\left[T_{N\left(t\right)+1}\right]=\mu\esp\left[N\left(t\right)+1\right]\textrm{,  }t\geq0,
\end{eqnarray*}  
\end{Coro}


\begin{Def}
Sea $h\left(t\right)$ funci\'on de valores reales en $\rea$ acotada en intervalos finitos e igual a cero para $t<0$ La ecuaci\'on de renovaci\'on para $h\left(t\right)$ y la distribuci\'on $F$ es

\begin{eqnarray}\label{Ec.Renovacion}
H\left(t\right)=h\left(t\right)+\int_{\left[0,t\right]}H\left(t-s\right)dF\left(s\right)\textrm{,    }t\geq0,
\end{eqnarray}
donde $H\left(t\right)$ es una funci\'on de valores reales. Esto es $H=h+F\star H$. Decimos que $H\left(t\right)$ es soluci\'on de esta ecuaci\'on si satisface la ecuaci\'on, y es acotada en intervalos finitos e iguales a cero para $t<0$.
\end{Def}

\begin{Prop}
La funci\'on $U\star h\left(t\right)$ es la \'unica soluci\'on de la ecuaci\'on de renovaci\'on (\ref{Ec.Renovacion}).
\end{Prop}

\begin{Teo}[Teorema Renovaci\'on Elemental]
\begin{eqnarray*}
t^{-1}U\left(t\right)\rightarrow 1/\mu\textrm{,    cuando }t\rightarrow\infty.
\end{eqnarray*}
\end{Teo}


\begin{Note} Una funci\'on $h:\rea_{+}\rightarrow\rea$ es Directamente Riemann Integrable en los siguientes casos:
\begin{itemize}
\item[a)] $h\left(t\right)\geq0$ es decreciente y Riemann Integrable.
\item[b)] $h$ es continua excepto posiblemente en un conjunto de Lebesgue de medida 0, y $|h\left(t\right)|\leq b\left(t\right)$, donde $b$ es DRI.
\end{itemize}
\end{Note}

\begin{Teo}[Teorema Principal de Renovaci\'on]
Si $F$ es no aritm\'etica y $h\left(t\right)$ es Directamente Riemann Integrable (DRI), entonces

\begin{eqnarray*}
lim_{t\rightarrow\infty}U\star h=\frac{1}{\mu}\int_{\rea_{+}}h\left(s\right)ds.
\end{eqnarray*}
\end{Teo}

\begin{Prop}
Cualquier funci\'on $H\left(t\right)$ acotada en intervalos finitos y que es 0 para $t<0$ puede expresarse como
\begin{eqnarray*}
H\left(t\right)=U\star h\left(t\right)\textrm{,  donde }h\left(t\right)=H\left(t\right)-F\star H\left(t\right)
\end{eqnarray*}
\end{Prop}

\begin{Def}
Un proceso estoc\'astico $X\left(t\right)$ es crudamente regenerativo en un tiempo aleatorio positivo $T$ si
\begin{eqnarray*}
\esp\left[X\left(T+t\right)|T\right]=\esp\left[X\left(t\right)\right]\textrm{, para }t\geq0,\end{eqnarray*}
y con las esperanzas anteriores finitas.
\end{Def}

\begin{Prop}
Sup\'ongase que $X\left(t\right)$ es un proceso crudamente regenerativo en $T$, que tiene distribuci\'on $F$. Si $\esp\left[X\left(t\right)\right]$ es acotado en intervalos finitos, entonces
\begin{eqnarray*}
\esp\left[X\left(t\right)\right]=U\star h\left(t\right)\textrm{,  donde }h\left(t\right)=\esp\left[X\left(t\right)\indora\left(T>t\right)\right].
\end{eqnarray*}
\end{Prop}

\begin{Teo}[Regeneraci\'on Cruda]
Sup\'ongase que $X\left(t\right)$ es un proceso con valores positivo crudamente regenerativo en $T$, y def\'inase $M=\sup\left\{|X\left(t\right)|:t\leq T\right\}$. Si $T$ es no aritm\'etico y $M$ y $MT$ tienen media finita, entonces
\begin{eqnarray*}
lim_{t\rightarrow\infty}\esp\left[X\left(t\right)\right]=\frac{1}{\mu}\int_{\rea_{+}}h\left(s\right)ds,
\end{eqnarray*}
donde $h\left(t\right)=\esp\left[X\left(t\right)\indora\left(T>t\right)\right]$.
\end{Teo}


\begin{Note} Una funci\'on $h:\rea_{+}\rightarrow\rea$ es Directamente Riemann Integrable en los siguientes casos:
\begin{itemize}
\item[a)] $h\left(t\right)\geq0$ es decreciente y Riemann Integrable.
\item[b)] $h$ es continua excepto posiblemente en un conjunto de Lebesgue de medida 0, y $|h\left(t\right)|\leq b\left(t\right)$, donde $b$ es DRI.
\end{itemize}
\end{Note}

\begin{Teo}[Teorema Principal de Renovaci\'on]
Si $F$ es no aritm\'etica y $h\left(t\right)$ es Directamente Riemann Integrable (DRI), entonces

\begin{eqnarray*}
lim_{t\rightarrow\infty}U\star h=\frac{1}{\mu}\int_{\rea_{+}}h\left(s\right)ds.
\end{eqnarray*}
\end{Teo}

\begin{Prop}
Cualquier funci\'on $H\left(t\right)$ acotada en intervalos finitos y que es 0 para $t<0$ puede expresarse como
\begin{eqnarray*}
H\left(t\right)=U\star h\left(t\right)\textrm{,  donde }h\left(t\right)=H\left(t\right)-F\star H\left(t\right)
\end{eqnarray*}
\end{Prop}

\begin{Def}
Un proceso estoc\'astico $X\left(t\right)$ es crudamente regenerativo en un tiempo aleatorio positivo $T$ si
\begin{eqnarray*}
\esp\left[X\left(T+t\right)|T\right]=\esp\left[X\left(t\right)\right]\textrm{, para }t\geq0,\end{eqnarray*}
y con las esperanzas anteriores finitas.
\end{Def}

\begin{Prop}
Sup\'ongase que $X\left(t\right)$ es un proceso crudamente regenerativo en $T$, que tiene distribuci\'on $F$. Si $\esp\left[X\left(t\right)\right]$ es acotado en intervalos finitos, entonces
\begin{eqnarray*}
\esp\left[X\left(t\right)\right]=U\star h\left(t\right)\textrm{,  donde }h\left(t\right)=\esp\left[X\left(t\right)\indora\left(T>t\right)\right].
\end{eqnarray*}
\end{Prop}

\begin{Teo}[Regeneraci\'on Cruda]
Sup\'ongase que $X\left(t\right)$ es un proceso con valores positivo crudamente regenerativo en $T$, y def\'inase $M=\sup\left\{|X\left(t\right)|:t\leq T\right\}$. Si $T$ es no aritm\'etico y $M$ y $MT$ tienen media finita, entonces
\begin{eqnarray*}
lim_{t\rightarrow\infty}\esp\left[X\left(t\right)\right]=\frac{1}{\mu}\int_{\rea_{+}}h\left(s\right)ds,
\end{eqnarray*}
donde $h\left(t\right)=\esp\left[X\left(t\right)\indora\left(T>t\right)\right]$.
\end{Teo}

\begin{Def}
Para el proceso $\left\{\left(N\left(t\right),X\left(t\right)\right):t\geq0\right\}$, sus trayectoria muestrales en el intervalo de tiempo $\left[T_{n-1},T_{n}\right)$ est\'an descritas por
\begin{eqnarray*}
\zeta_{n}=\left(\xi_{n},\left\{X\left(T_{n-1}+t\right):0\leq t<\xi_{n}\right\}\right)
\end{eqnarray*}
Este $\zeta_{n}$ es el $n$-\'esimo segmento del proceso. El proceso es regenerativo sobre los tiempos $T_{n}$ si sus segmentos $\zeta_{n}$ son independientes e id\'enticamennte distribuidos.
\end{Def}


\begin{Note}
Si $\tilde{X}\left(t\right)$ con espacio de estados $\tilde{S}$ es regenerativo sobre $T_{n}$, entonces $X\left(t\right)=f\left(\tilde{X}\left(t\right)\right)$ tambi\'en es regenerativo sobre $T_{n}$, para cualquier funci\'on $f:\tilde{S}\rightarrow S$.
\end{Note}

\begin{Note}
Los procesos regenerativos son crudamente regenerativos, pero no al rev\'es.
\end{Note}


\begin{Note}
Un proceso estoc\'astico a tiempo continuo o discreto es regenerativo si existe un proceso de renovaci\'on  tal que los segmentos del proceso entre tiempos de renovaci\'on sucesivos son i.i.d., es decir, para $\left\{X\left(t\right):t\geq0\right\}$ proceso estoc\'astico a tiempo continuo con espacio de estados $S$, espacio m\'etrico.
\end{Note}

Para $\left\{X\left(t\right):t\geq0\right\}$ Proceso Estoc\'astico a tiempo continuo con estado de espacios $S$, que es un espacio m\'etrico, con trayectorias continuas por la derecha y con l\'imites por la izquierda c.s. Sea $N\left(t\right)$ un proceso de renovaci\'on en $\rea_{+}$ definido en el mismo espacio de probabilidad que $X\left(t\right)$, con tiempos de renovaci\'on $T$ y tiempos de inter-renovaci\'on $\xi_{n}=T_{n}-T_{n-1}$, con misma distribuci\'on $F$ de media finita $\mu$.



\begin{Def}
Para el proceso $\left\{\left(N\left(t\right),X\left(t\right)\right):t\geq0\right\}$, sus trayectoria muestrales en el intervalo de tiempo $\left[T_{n-1},T_{n}\right)$ est\'an descritas por
\begin{eqnarray*}
\zeta_{n}=\left(\xi_{n},\left\{X\left(T_{n-1}+t\right):0\leq t<\xi_{n}\right\}\right)
\end{eqnarray*}
Este $\zeta_{n}$ es el $n$-\'esimo segmento del proceso. El proceso es regenerativo sobre los tiempos $T_{n}$ si sus segmentos $\zeta_{n}$ son independientes e id\'enticamennte distribuidos.
\end{Def}

\begin{Note}
Un proceso regenerativo con media de la longitud de ciclo finita es llamado positivo recurrente.
\end{Note}

\begin{Teo}[Procesos Regenerativos]
Suponga que el proceso
\end{Teo}


\begin{Def}[Renewal Process Trinity]
Para un proceso de renovaci\'on $N\left(t\right)$, los siguientes procesos proveen de informaci\'on sobre los tiempos de renovaci\'on.
\begin{itemize}
\item $A\left(t\right)=t-T_{N\left(t\right)}$, el tiempo de recurrencia hacia atr\'as al tiempo $t$, que es el tiempo desde la \'ultima renovaci\'on para $t$.

\item $B\left(t\right)=T_{N\left(t\right)+1}-t$, el tiempo de recurrencia hacia adelante al tiempo $t$, residual del tiempo de renovaci\'on, que es el tiempo para la pr\'oxima renovaci\'on despu\'es de $t$.

\item $L\left(t\right)=\xi_{N\left(t\right)+1}=A\left(t\right)+B\left(t\right)$, la longitud del intervalo de renovaci\'on que contiene a $t$.
\end{itemize}
\end{Def}

\begin{Note}
El proceso tridimensional $\left(A\left(t\right),B\left(t\right),L\left(t\right)\right)$ es regenerativo sobre $T_{n}$, y por ende cada proceso lo es. Cada proceso $A\left(t\right)$ y $B\left(t\right)$ son procesos de MArkov a tiempo continuo con trayectorias continuas por partes en el espacio de estados $\rea_{+}$. Una expresi\'on conveniente para su distribuci\'on conjunta es, para $0\leq x<t,y\geq0$
\begin{equation}\label{NoRenovacion}
P\left\{A\left(t\right)>x,B\left(t\right)>y\right\}=
P\left\{N\left(t+y\right)-N\left((t-x)\right)=0\right\}
\end{equation}
\end{Note}

\begin{Ejem}[Tiempos de recurrencia Poisson]
Si $N\left(t\right)$ es un proceso Poisson con tasa $\lambda$, entonces de la expresi\'on (\ref{NoRenovacion}) se tiene que

\begin{eqnarray*}
\begin{array}{lc}
P\left\{A\left(t\right)>x,B\left(t\right)>y\right\}=e^{-\lambda\left(x+y\right)},&0\leq x<t,y\geq0,
\end{array}
\end{eqnarray*}
que es la probabilidad Poisson de no renovaciones en un intervalo de longitud $x+y$.

\end{Ejem}

\begin{Note}
Una cadena de Markov erg\'odica tiene la propiedad de ser estacionaria si la distribuci\'on de su estado al tiempo $0$ es su distribuci\'on estacionaria.
\end{Note}


\begin{Def}
Un proceso estoc\'astico a tiempo continuo $\left\{X\left(t\right):t\geq0\right\}$ en un espacio general es estacionario si sus distribuciones finito dimensionales son invariantes bajo cualquier  traslado: para cada $0\leq s_{1}<s_{2}<\cdots<s_{k}$ y $t\geq0$,
\begin{eqnarray*}
\left(X\left(s_{1}+t\right),\ldots,X\left(s_{k}+t\right)\right)=_{d}\left(X\left(s_{1}\right),\ldots,X\left(s_{k}\right)\right).
\end{eqnarray*}
\end{Def}

\begin{Note}
Un proceso de Markov es estacionario si $X\left(t\right)=_{d}X\left(0\right)$, $t\geq0$.
\end{Note}

Considerese el proceso $N\left(t\right)=\sum_{n}\indora\left(\tau_{n}\leq t\right)$ en $\rea_{+}$, con puntos $0<\tau_{1}<\tau_{2}<\cdots$.

\begin{Prop}
Si $N$ es un proceso puntual estacionario y $\esp\left[N\left(1\right)\right]<\infty$, entonces $\esp\left[N\left(t\right)\right]=t\esp\left[N\left(1\right)\right]$, $t\geq0$

\end{Prop}

\begin{Teo}
Los siguientes enunciados son equivalentes
\begin{itemize}
\item[i)] El proceso retardado de renovaci\'on $N$ es estacionario.

\item[ii)] EL proceso de tiempos de recurrencia hacia adelante $B\left(t\right)$ es estacionario.


\item[iii)] $\esp\left[N\left(t\right)\right]=t/\mu$,


\item[iv)] $G\left(t\right)=F_{e}\left(t\right)=\frac{1}{\mu}\int_{0}^{t}\left[1-F\left(s\right)\right]ds$
\end{itemize}
Cuando estos enunciados son ciertos, $P\left\{B\left(t\right)\leq x\right\}=F_{e}\left(x\right)$, para $t,x\geq0$.

\end{Teo}

\begin{Note}
Una consecuencia del teorema anterior es que el Proceso Poisson es el \'unico proceso sin retardo que es estacionario.
\end{Note}

\begin{Coro}
El proceso de renovaci\'on $N\left(t\right)$ sin retardo, y cuyos tiempos de inter renonaci\'on tienen media finita, es estacionario si y s\'olo si es un proceso Poisson.

\end{Coro}

%______________________________________________________________________

%\section{Ejemplos, Notas importantes}
%______________________________________________________________________
%\section*{Ap\'endice A}
%__________________________________________________________________

%________________________________________________________________________
%\subsection*{Procesos Regenerativos}
%________________________________________________________________________



\begin{Note}
Si $\tilde{X}\left(t\right)$ con espacio de estados $\tilde{S}$ es regenerativo sobre $T_{n}$, entonces $X\left(t\right)=f\left(\tilde{X}\left(t\right)\right)$ tambi\'en es regenerativo sobre $T_{n}$, para cualquier funci\'on $f:\tilde{S}\rightarrow S$.
\end{Note}

\begin{Note}
Los procesos regenerativos son crudamente regenerativos, pero no al rev\'es.
\end{Note}
%\subsection*{Procesos Regenerativos: Sigman\cite{Sigman1}}
\begin{Def}[Definici\'on Cl\'asica]
Un proceso estoc\'astico $X=\left\{X\left(t\right):t\geq0\right\}$ es llamado regenerativo is existe una variable aleatoria $R_{1}>0$ tal que
\begin{itemize}
\item[i)] $\left\{X\left(t+R_{1}\right):t\geq0\right\}$ es independiente de $\left\{\left\{X\left(t\right):t<R_{1}\right\},\right\}$
\item[ii)] $\left\{X\left(t+R_{1}\right):t\geq0\right\}$ es estoc\'asticamente equivalente a $\left\{X\left(t\right):t>0\right\}$
\end{itemize}

Llamamos a $R_{1}$ tiempo de regeneraci\'on, y decimos que $X$ se regenera en este punto.
\end{Def}

$\left\{X\left(t+R_{1}\right)\right\}$ es regenerativo con tiempo de regeneraci\'on $R_{2}$, independiente de $R_{1}$ pero con la misma distribuci\'on que $R_{1}$. Procediendo de esta manera se obtiene una secuencia de variables aleatorias independientes e id\'enticamente distribuidas $\left\{R_{n}\right\}$ llamados longitudes de ciclo. Si definimos a $Z_{k}\equiv R_{1}+R_{2}+\cdots+R_{k}$, se tiene un proceso de renovaci\'on llamado proceso de renovaci\'on encajado para $X$.




\begin{Def}
Para $x$ fijo y para cada $t\geq0$, sea $I_{x}\left(t\right)=1$ si $X\left(t\right)\leq x$,  $I_{x}\left(t\right)=0$ en caso contrario, y def\'inanse los tiempos promedio
\begin{eqnarray*}
\overline{X}&=&lim_{t\rightarrow\infty}\frac{1}{t}\int_{0}^{\infty}X\left(u\right)du\\
\prob\left(X_{\infty}\leq x\right)&=&lim_{t\rightarrow\infty}\frac{1}{t}\int_{0}^{\infty}I_{x}\left(u\right)du,
\end{eqnarray*}
cuando estos l\'imites existan.
\end{Def}

Como consecuencia del teorema de Renovaci\'on-Recompensa, se tiene que el primer l\'imite  existe y es igual a la constante
\begin{eqnarray*}
\overline{X}&=&\frac{\esp\left[\int_{0}^{R_{1}}X\left(t\right)dt\right]}{\esp\left[R_{1}\right]},
\end{eqnarray*}
suponiendo que ambas esperanzas son finitas.

\begin{Note}
\begin{itemize}
\item[a)] Si el proceso regenerativo $X$ es positivo recurrente y tiene trayectorias muestrales no negativas, entonces la ecuaci\'on anterior es v\'alida.
\item[b)] Si $X$ es positivo recurrente regenerativo, podemos construir una \'unica versi\'on estacionaria de este proceso, $X_{e}=\left\{X_{e}\left(t\right)\right\}$, donde $X_{e}$ es un proceso estoc\'astico regenerativo y estrictamente estacionario, con distribuci\'on marginal distribuida como $X_{\infty}$
\end{itemize}
\end{Note}

Para $\left\{X\left(t\right):t\geq0\right\}$ Proceso Estoc\'astico a tiempo continuo con estado de espacios $S$, que es un espacio m\'etrico, con trayectorias continuas por la derecha y con l\'imites por la izquierda c.s. Sea $N\left(t\right)$ un proceso de renovaci\'on en $\rea_{+}$ definido en el mismo espacio de probabilidad que $X\left(t\right)$, con tiempos de renovaci\'on $T$ y tiempos de inter-renovaci\'on $\xi_{n}=T_{n}-T_{n-1}$, con misma distribuci\'on $F$ de media finita $\mu$.


\begin{Def}
Para el proceso $\left\{\left(N\left(t\right),X\left(t\right)\right):t\geq0\right\}$, sus trayectoria muestrales en el intervalo de tiempo $\left[T_{n-1},T_{n}\right)$ est\'an descritas por
\begin{eqnarray*}
\zeta_{n}=\left(\xi_{n},\left\{X\left(T_{n-1}+t\right):0\leq t<\xi_{n}\right\}\right)
\end{eqnarray*}
Este $\zeta_{n}$ es el $n$-\'esimo segmento del proceso. El proceso es regenerativo sobre los tiempos $T_{n}$ si sus segmentos $\zeta_{n}$ son independientes e id\'enticamennte distribuidos.
\end{Def}


\begin{Note}
Si $\tilde{X}\left(t\right)$ con espacio de estados $\tilde{S}$ es regenerativo sobre $T_{n}$, entonces $X\left(t\right)=f\left(\tilde{X}\left(t\right)\right)$ tambi\'en es regenerativo sobre $T_{n}$, para cualquier funci\'on $f:\tilde{S}\rightarrow S$.
\end{Note}

\begin{Note}
Los procesos regenerativos son crudamente regenerativos, pero no al rev\'es.
\end{Note}

\begin{Def}[Definici\'on Cl\'asica]
Un proceso estoc\'astico $X=\left\{X\left(t\right):t\geq0\right\}$ es llamado regenerativo is existe una variable aleatoria $R_{1}>0$ tal que
\begin{itemize}
\item[i)] $\left\{X\left(t+R_{1}\right):t\geq0\right\}$ es independiente de $\left\{\left\{X\left(t\right):t<R_{1}\right\},\right\}$
\item[ii)] $\left\{X\left(t+R_{1}\right):t\geq0\right\}$ es estoc\'asticamente equivalente a $\left\{X\left(t\right):t>0\right\}$
\end{itemize}

Llamamos a $R_{1}$ tiempo de regeneraci\'on, y decimos que $X$ se regenera en este punto.
\end{Def}

$\left\{X\left(t+R_{1}\right)\right\}$ es regenerativo con tiempo de regeneraci\'on $R_{2}$, independiente de $R_{1}$ pero con la misma distribuci\'on que $R_{1}$. Procediendo de esta manera se obtiene una secuencia de variables aleatorias independientes e id\'enticamente distribuidas $\left\{R_{n}\right\}$ llamados longitudes de ciclo. Si definimos a $Z_{k}\equiv R_{1}+R_{2}+\cdots+R_{k}$, se tiene un proceso de renovaci\'on llamado proceso de renovaci\'on encajado para $X$.

\begin{Note}
Un proceso regenerativo con media de la longitud de ciclo finita es llamado positivo recurrente.
\end{Note}


\begin{Def}
Para $x$ fijo y para cada $t\geq0$, sea $I_{x}\left(t\right)=1$ si $X\left(t\right)\leq x$,  $I_{x}\left(t\right)=0$ en caso contrario, y def\'inanse los tiempos promedio
\begin{eqnarray*}
\overline{X}&=&lim_{t\rightarrow\infty}\frac{1}{t}\int_{0}^{\infty}X\left(u\right)du\\
\prob\left(X_{\infty}\leq x\right)&=&lim_{t\rightarrow\infty}\frac{1}{t}\int_{0}^{\infty}I_{x}\left(u\right)du,
\end{eqnarray*}
cuando estos l\'imites existan.
\end{Def}

Como consecuencia del teorema de Renovaci\'on-Recompensa, se tiene que el primer l\'imite  existe y es igual a la constante
\begin{eqnarray*}
\overline{X}&=&\frac{\esp\left[\int_{0}^{R_{1}}X\left(t\right)dt\right]}{\esp\left[R_{1}\right]},
\end{eqnarray*}
suponiendo que ambas esperanzas son finitas.

\begin{Note}
\begin{itemize}
\item[a)] Si el proceso regenerativo $X$ es positivo recurrente y tiene trayectorias muestrales no negativas, entonces la ecuaci\'on anterior es v\'alida.
\item[b)] Si $X$ es positivo recurrente regenerativo, podemos construir una \'unica versi\'on estacionaria de este proceso, $X_{e}=\left\{X_{e}\left(t\right)\right\}$, donde $X_{e}$ es un proceso estoc\'astico regenerativo y estrictamente estacionario, con distribuci\'on marginal distribuida como $X_{\infty}$
\end{itemize}
\end{Note}

%__________________________________________________________________________________________
%\subsection{Procesos Regenerativos Estacionarios - Stidham \cite{Stidham}}
%__________________________________________________________________________________________


Un proceso estoc\'astico a tiempo continuo $\left\{V\left(t\right),t\geq0\right\}$ es un proceso regenerativo si existe una sucesi\'on de variables aleatorias independientes e id\'enticamente distribuidas $\left\{X_{1},X_{2},\ldots\right\}$, sucesi\'on de renovaci\'on, tal que para cualquier conjunto de Borel $A$, 

\begin{eqnarray*}
\prob\left\{V\left(t\right)\in A|X_{1}+X_{2}+\cdots+X_{R\left(t\right)}=s,\left\{V\left(\tau\right),\tau<s\right\}\right\}=\prob\left\{V\left(t-s\right)\in A|X_{1}>t-s\right\},
\end{eqnarray*}
para todo $0\leq s\leq t$, donde $R\left(t\right)=\max\left\{X_{1}+X_{2}+\cdots+X_{j}\leq t\right\}=$n\'umero de renovaciones ({\emph{puntos de regeneraci\'on}}) que ocurren en $\left[0,t\right]$. El intervalo $\left[0,X_{1}\right)$ es llamado {\emph{primer ciclo de regeneraci\'on}} de $\left\{V\left(t \right),t\geq0\right\}$, $\left[X_{1},X_{1}+X_{2}\right)$ el {\emph{segundo ciclo de regeneraci\'on}}, y as\'i sucesivamente.

Sea $X=X_{1}$ y sea $F$ la funci\'on de distrbuci\'on de $X$


\begin{Def}
Se define el proceso estacionario, $\left\{V^{*}\left(t\right),t\geq0\right\}$, para $\left\{V\left(t\right),t\geq0\right\}$ por

\begin{eqnarray*}
\prob\left\{V\left(t\right)\in A\right\}=\frac{1}{\esp\left[X\right]}\int_{0}^{\infty}\prob\left\{V\left(t+x\right)\in A|X>x\right\}\left(1-F\left(x\right)\right)dx,
\end{eqnarray*} 
para todo $t\geq0$ y todo conjunto de Borel $A$.
\end{Def}

\begin{Def}
Una distribuci\'on se dice que es {\emph{aritm\'etica}} si todos sus puntos de incremento son m\'ultiplos de la forma $0,\lambda, 2\lambda,\ldots$ para alguna $\lambda>0$ entera.
\end{Def}


\begin{Def}
Una modificaci\'on medible de un proceso $\left\{V\left(t\right),t\geq0\right\}$, es una versi\'on de este, $\left\{V\left(t,w\right)\right\}$ conjuntamente medible para $t\geq0$ y para $w\in S$, $S$ espacio de estados para $\left\{V\left(t\right),t\geq0\right\}$.
\end{Def}

\begin{Teo}
Sea $\left\{V\left(t\right),t\geq\right\}$ un proceso regenerativo no negativo con modificaci\'on medible. Sea $\esp\left[X\right]<\infty$. Entonces el proceso estacionario dado por la ecuaci\'on anterior est\'a bien definido y tiene funci\'on de distribuci\'on independiente de $t$, adem\'as
\begin{itemize}
\item[i)] \begin{eqnarray*}
\esp\left[V^{*}\left(0\right)\right]&=&\frac{\esp\left[\int_{0}^{X}V\left(s\right)ds\right]}{\esp\left[X\right]}\end{eqnarray*}
\item[ii)] Si $\esp\left[V^{*}\left(0\right)\right]<\infty$, equivalentemente, si $\esp\left[\int_{0}^{X}V\left(s\right)ds\right]<\infty$,entonces
\begin{eqnarray*}
\frac{\int_{0}^{t}V\left(s\right)ds}{t}\rightarrow\frac{\esp\left[\int_{0}^{X}V\left(s\right)ds\right]}{\esp\left[X\right]}
\end{eqnarray*}
con probabilidad 1 y en media, cuando $t\rightarrow\infty$.
\end{itemize}
\end{Teo}
%
%___________________________________________________________________________________________
%\vspace{5.5cm}
%\chapter{Cadenas de Markov estacionarias}
%\vspace{-1.0cm}


%__________________________________________________________________________________________
%\subsection{Procesos Regenerativos Estacionarios - Stidham \cite{Stidham}}
%__________________________________________________________________________________________


Un proceso estoc\'astico a tiempo continuo $\left\{V\left(t\right),t\geq0\right\}$ es un proceso regenerativo si existe una sucesi\'on de variables aleatorias independientes e id\'enticamente distribuidas $\left\{X_{1},X_{2},\ldots\right\}$, sucesi\'on de renovaci\'on, tal que para cualquier conjunto de Borel $A$, 

\begin{eqnarray*}
\prob\left\{V\left(t\right)\in A|X_{1}+X_{2}+\cdots+X_{R\left(t\right)}=s,\left\{V\left(\tau\right),\tau<s\right\}\right\}=\prob\left\{V\left(t-s\right)\in A|X_{1}>t-s\right\},
\end{eqnarray*}
para todo $0\leq s\leq t$, donde $R\left(t\right)=\max\left\{X_{1}+X_{2}+\cdots+X_{j}\leq t\right\}=$n\'umero de renovaciones ({\emph{puntos de regeneraci\'on}}) que ocurren en $\left[0,t\right]$. El intervalo $\left[0,X_{1}\right)$ es llamado {\emph{primer ciclo de regeneraci\'on}} de $\left\{V\left(t \right),t\geq0\right\}$, $\left[X_{1},X_{1}+X_{2}\right)$ el {\emph{segundo ciclo de regeneraci\'on}}, y as\'i sucesivamente.

Sea $X=X_{1}$ y sea $F$ la funci\'on de distrbuci\'on de $X$


\begin{Def}
Se define el proceso estacionario, $\left\{V^{*}\left(t\right),t\geq0\right\}$, para $\left\{V\left(t\right),t\geq0\right\}$ por

\begin{eqnarray*}
\prob\left\{V\left(t\right)\in A\right\}=\frac{1}{\esp\left[X\right]}\int_{0}^{\infty}\prob\left\{V\left(t+x\right)\in A|X>x\right\}\left(1-F\left(x\right)\right)dx,
\end{eqnarray*} 
para todo $t\geq0$ y todo conjunto de Borel $A$.
\end{Def}

\begin{Def}
Una distribuci\'on se dice que es {\emph{aritm\'etica}} si todos sus puntos de incremento son m\'ultiplos de la forma $0,\lambda, 2\lambda,\ldots$ para alguna $\lambda>0$ entera.
\end{Def}


\begin{Def}
Una modificaci\'on medible de un proceso $\left\{V\left(t\right),t\geq0\right\}$, es una versi\'on de este, $\left\{V\left(t,w\right)\right\}$ conjuntamente medible para $t\geq0$ y para $w\in S$, $S$ espacio de estados para $\left\{V\left(t\right),t\geq0\right\}$.
\end{Def}

\begin{Teo}
Sea $\left\{V\left(t\right),t\geq\right\}$ un proceso regenerativo no negativo con modificaci\'on medible. Sea $\esp\left[X\right]<\infty$. Entonces el proceso estacionario dado por la ecuaci\'on anterior est\'a bien definido y tiene funci\'on de distribuci\'on independiente de $t$, adem\'as
\begin{itemize}
\item[i)] \begin{eqnarray*}
\esp\left[V^{*}\left(0\right)\right]&=&\frac{\esp\left[\int_{0}^{X}V\left(s\right)ds\right]}{\esp\left[X\right]}\end{eqnarray*}
\item[ii)] Si $\esp\left[V^{*}\left(0\right)\right]<\infty$, equivalentemente, si $\esp\left[\int_{0}^{X}V\left(s\right)ds\right]<\infty$,entonces
\begin{eqnarray*}
\frac{\int_{0}^{t}V\left(s\right)ds}{t}\rightarrow\frac{\esp\left[\int_{0}^{X}V\left(s\right)ds\right]}{\esp\left[X\right]}
\end{eqnarray*}
con probabilidad 1 y en media, cuando $t\rightarrow\infty$.
\end{itemize}
\end{Teo}

Sea la funci\'on generadora de momentos para $L_{i}$, el n\'umero de usuarios en la cola $Q_{i}\left(z\right)$ en cualquier momento, est\'a dada por el tiempo promedio de $z^{L_{i}\left(t\right)}$ sobre el ciclo regenerativo definido anteriormente. Entonces 



Es decir, es posible determinar las longitudes de las colas a cualquier tiempo $t$. Entonces, determinando el primer momento es posible ver que


\begin{Def}
El tiempo de Ciclo $C_{i}$ es el periodo de tiempo que comienza cuando la cola $i$ es visitada por primera vez en un ciclo, y termina cuando es visitado nuevamente en el pr\'oximo ciclo. La duraci\'on del mismo est\'a dada por $\tau_{i}\left(m+1\right)-\tau_{i}\left(m\right)$, o equivalentemente $\overline{\tau}_{i}\left(m+1\right)-\overline{\tau}_{i}\left(m\right)$ bajo condiciones de estabilidad.
\end{Def}


\begin{Def}
El tiempo de intervisita $I_{i}$ es el periodo de tiempo que comienza cuando se ha completado el servicio en un ciclo y termina cuando es visitada nuevamente en el pr\'oximo ciclo. Su  duraci\'on del mismo est\'a dada por $\tau_{i}\left(m+1\right)-\overline{\tau}_{i}\left(m\right)$.
\end{Def}

La duraci\'on del tiempo de intervisita es $\tau_{i}\left(m+1\right)-\overline{\tau}\left(m\right)$. Dado que el n\'umero de usuarios presentes en $Q_{i}$ al tiempo $t=\tau_{i}\left(m+1\right)$ es igual al n\'umero de arribos durante el intervalo de tiempo $\left[\overline{\tau}\left(m\right),\tau_{i}\left(m+1\right)\right]$ se tiene que


\begin{eqnarray*}
\esp\left[z_{i}^{L_{i}\left(\tau_{i}\left(m+1\right)\right)}\right]=\esp\left[\left\{P_{i}\left(z_{i}\right)\right\}^{\tau_{i}\left(m+1\right)-\overline{\tau}\left(m\right)}\right]
\end{eqnarray*}

entonces, si $I_{i}\left(z\right)=\esp\left[z^{\tau_{i}\left(m+1\right)-\overline{\tau}\left(m\right)}\right]$
se tiene que $F_{i}\left(z\right)=I_{i}\left[P_{i}\left(z\right)\right]$
para $i=1,2$.

Conforme a la definici\'on dada al principio del cap\'itulo, definici\'on (\ref{Def.Tn}), sean $T_{1},T_{2},\ldots$ los puntos donde las longitudes de las colas de la red de sistemas de visitas c\'iclicas son cero simult\'aneamente, cuando la cola $Q_{j}$ es visitada por el servidor para dar servicio, es decir, $L_{1}\left(T_{i}\right)=0,L_{2}\left(T_{i}\right)=0,\hat{L}_{1}\left(T_{i}\right)=0$ y $\hat{L}_{2}\left(T_{i}\right)=0$, a estos puntos se les denominar\'a puntos regenerativos. Entonces, 

\begin{Def}
Al intervalo de tiempo entre dos puntos regenerativos se le llamar\'a ciclo regenerativo.
\end{Def}

\begin{Def}
Para $T_{i}$ se define, $M_{i}$, el n\'umero de ciclos de visita a la cola $Q_{l}$, durante el ciclo regenerativo, es decir, $M_{i}$ es un proceso de renovaci\'on.
\end{Def}

\begin{Def}
Para cada uno de los $M_{i}$'s, se definen a su vez la duraci\'on de cada uno de estos ciclos de visita en el ciclo regenerativo, $C_{i}^{(m)}$, para $m=1,2,\ldots,M_{i}$, que a su vez, tambi\'en es n proceso de renovaci\'on.
\end{Def}

\footnote{In Stidham and  Heyman \cite{Stidham} shows that is sufficient for the regenerative process to be stationary that the mean regenerative cycle time is finite: $\esp\left[\sum_{m=1}^{M_{i}}C_{i}^{(m)}\right]<\infty$, 


 como cada $C_{i}^{(m)}$ contiene intervalos de r\'eplica positivos, se tiene que $\esp\left[M_{i}\right]<\infty$, adem\'as, como $M_{i}>0$, se tiene que la condici\'on anterior es equivalente a tener que $\esp\left[C_{i}\right]<\infty$,
por lo tanto una condici\'on suficiente para la existencia del proceso regenerativo est\'a dada por $\sum_{k=1}^{N}\mu_{k}<1.$}

Para $\left\{X\left(t\right):t\geq0\right\}$ Proceso Estoc\'astico a tiempo continuo con estado de espacios $S$, que es un espacio m\'etrico, con trayectorias continuas por la derecha y con l\'imites por la izquierda c.s. Sea $N\left(t\right)$ un proceso de renovaci\'on en $\rea_{+}$ definido en el mismo espacio de probabilidad que $X\left(t\right)$, con tiempos de renovaci\'on $T$ y tiempos de inter-renovaci\'on $\xi_{n}=T_{n}-T_{n-1}$, con misma distribuci\'on $F$ de media finita $\mu$.

\begin{Def}
Un elemento aleatorio en un espacio medible $\left(E,\mathcal{E}\right)$ en un espacio de probabilidad $\left(\Omega,\mathcal{F},\prob\right)$ a $\left(E,\mathcal{E}\right)$, es decir,
para $A\in \mathcal{E}$,  se tiene que $\left\{Y\in A\right\}\in\mathcal{F}$, donde $\left\{Y\in A\right\}:=\left\{w\in\Omega:Y\left(w\right)\in A\right\}=:Y^{-1}A$.
\end{Def}

\begin{Note}
Tambi\'en se dice que $Y$ est\'a soportado por el espacio de probabilidad $\left(\Omega,\mathcal{F},\prob\right)$ y que $Y$ es un mapeo medible de $\Omega$ en $E$, es decir, es $\mathcal{F}/\mathcal{E}$ medible.
\end{Note}

\begin{Def}
Para cada $i\in \mathbb{I}$ sea $P_{i}$ una medida de probabilidad en un espacio medible $\left(E_{i},\mathcal{E}_{i}\right)$. Se define el espacio producto
$\otimes_{i\in\mathbb{I}}\left(E_{i},\mathcal{E}_{i}\right):=\left(\prod_{i\in\mathbb{I}}E_{i},\otimes_{i\in\mathbb{I}}\mathcal{E}_{i}\right)$, donde $\prod_{i\in\mathbb{I}}E_{i}$ es el producto cartesiano de los $E_{i}$'s, y $\otimes_{i\in\mathbb{I}}\mathcal{E}_{i}$ es la $\sigma$-\'algebra producto, es decir, es la $\sigma$-\'algebra m\'as peque\~na en $\prod_{i\in\mathbb{I}}E_{i}$ que hace al $i$-\'esimo mapeo proyecci\'on en $E_{i}$ medible para toda $i\in\mathbb{I}$ es la $\sigma$-\'algebra inducida por los mapeos proyecci\'on. $$\otimes_{i\in\mathbb{I}}\mathcal{E}_{i}:=\sigma\left\{\left\{y:y_{i}\in A\right\}:i\in\mathbb{I}\textrm{ y }A\in\mathcal{E}_{i}\right\}.$$
\end{Def}

\begin{Def}
Un espacio de probabilidad $\left(\tilde{\Omega},\tilde{\mathcal{F}},\tilde{\prob}\right)$ es una extensi\'on de otro espacio de probabilidad $\left(\Omega,\mathcal{F},\prob\right)$ si $\left(\tilde{\Omega},\tilde{\mathcal{F}},\tilde{\prob}\right)$ soporta un elemento aleatorio $\xi\in\left(\Omega,\mathcal{F}\right)$ que tienen a $\prob$ como distribuci\'on.
\end{Def}

\begin{Teo}
Sea $\mathbb{I}$ un conjunto de \'indices arbitrario. Para cada $i\in\mathbb{I}$ sea $P_{i}$ una medida de probabilidad en un espacio medible $\left(E_{i},\mathcal{E}_{i}\right)$. Entonces existe una \'unica medida de probabilidad $\otimes_{i\in\mathbb{I}}P_{i}$ en $\otimes_{i\in\mathbb{I}}\left(E_{i},\mathcal{E}_{i}\right)$ tal que 

\begin{eqnarray*}
\otimes_{i\in\mathbb{I}}P_{i}\left(y\in\prod_{i\in\mathbb{I}}E_{i}:y_{i}\in A_{i_{1}},\ldots,y_{n}\in A_{i_{n}}\right)=P_{i_{1}}\left(A_{i_{n}}\right)\cdots P_{i_{n}}\left(A_{i_{n}}\right)
\end{eqnarray*}
para todos los enteros $n>0$, toda $i_{1},\ldots,i_{n}\in\mathbb{I}$ y todo $A_{i_{1}}\in\mathcal{E}_{i_{1}},\ldots,A_{i_{n}}\in\mathcal{E}_{i_{n}}$
\end{Teo}

La medida $\otimes_{i\in\mathbb{I}}P_{i}$ es llamada la medida producto y $\otimes_{i\in\mathbb{I}}\left(E_{i},\mathcal{E}_{i},P_{i}\right):=\left(\prod_{i\in\mathbb{I}},E_{i},\otimes_{i\in\mathbb{I}}\mathcal{E}_{i},\otimes_{i\in\mathbb{I}}P_{i}\right)$, es llamado espacio de probabilidad producto.


\begin{Def}
Un espacio medible $\left(E,\mathcal{E}\right)$ es \textit{Polaco} si existe una m\'etrica en $E$ tal que $E$ es completo, es decir cada sucesi\'on de Cauchy converge a un l\'imite en $E$, y \textit{separable}, $E$ tienen un subconjunto denso numerable, y tal que $\mathcal{E}$ es generado por conjuntos abiertos.
\end{Def}


\begin{Def}
Dos espacios medibles $\left(E,\mathcal{E}\right)$ y $\left(G,\mathcal{G}\right)$ son Borel equivalentes \textit{isomorfos} si existe una biyecci\'on $f:E\rightarrow G$ tal que $f$ es $\mathcal{E}/\mathcal{G}$ medible y su inversa $f^{-1}$ es $\mathcal{G}/\mathcal{E}$ medible. La biyecci\'on es una equivalencia de Borel.
\end{Def}

\begin{Def}
Un espacio medible  $\left(E,\mathcal{E}\right)$ es un \textit{espacio est\'andar} si es Borel equivalente a $\left(G,\mathcal{G}\right)$, donde $G$ es un subconjunto de Borel de $\left[0,1\right]$ y $\mathcal{G}$ son los subconjuntos de Borel de $G$.
\end{Def}

\begin{Note}
Cualquier espacio Polaco es un espacio est\'andar.
\end{Note}


\begin{Def}
Un proceso estoc\'astico con conjunto de \'indices $\mathbb{I}$ y espacio de estados $\left(E,\mathcal{E}\right)$ es una familia $Z=\left(\mathbb{Z}_{s}\right)_{s\in\mathbb{I}}$ donde $\mathbb{Z}_{s}$ son elementos aleatorios definidos en un espacio de probabilidad com\'un $\left(\Omega,\mathcal{F},\prob\right)$ y todos toman valores en $\left(E,\mathcal{E}\right)$.
\end{Def}

\begin{Def}
Un proceso estoc\'astico \textit{one-sided contiuous time} (\textbf{PEOSCT}) es un proceso estoc\'astico con conjunto de \'indices $\mathbb{I}=\left[0,\infty\right)$.
\end{Def}


Sea $\left(E^{\mathbb{I}},\mathcal{E}^{\mathbb{I}}\right)$ denota el espacio producto $\left(E^{\mathbb{I}},\mathcal{E}^{\mathbb{I}}\right):=\otimes_{s\in\mathbb{I}}\left(E,\mathcal{E}\right)$. Vamos a considerar $\mathbb{Z}$ como un mapeo aleatorio, es decir, como un elemento aleatorio en $\left(E^{\mathbb{I}},\mathcal{E}^{\mathbb{I}}\right)$ definido por $Z\left(w\right)=\left(Z_{s}\left(w\right)\right)_{s\in\mathbb{I}}$ y $w\in\Omega$.

\begin{Note}
La distribuci\'on de un proceso estoc\'astico $Z$ es la distribuci\'on de $Z$ como un elemento aleatorio en $\left(E^{\mathbb{I}},\mathcal{E}^{\mathbb{I}}\right)$. La distribuci\'on de $Z$ esta determinada de manera \'unica por las distribuciones finito dimensionales.
\end{Note}

\begin{Note}
En particular cuando $Z$ toma valores reales, es decir, $\left(E,\mathcal{E}\right)=\left(\mathbb{R},\mathcal{B}\right)$ las distribuciones finito dimensionales est\'an determinadas por las funciones de distribuci\'on finito dimensionales

\begin{eqnarray}
\prob\left(Z_{t_{1}}\leq x_{1},\ldots,Z_{t_{n}}\leq x_{n}\right),x_{1},\ldots,x_{n}\in\mathbb{R},t_{1},\ldots,t_{n}\in\mathbb{I},n\geq1.
\end{eqnarray}
\end{Note}

\begin{Note}
Para espacios polacos $\left(E,\mathcal{E}\right)$ el Teorema de Consistencia de Kolmogorov asegura que dada una colecci\'on de distribuciones finito dimensionales consistentes, siempre existe un proceso estoc\'astico que posee tales distribuciones finito dimensionales.
\end{Note}


\begin{Def}
Las trayectorias de $Z$ son las realizaciones $Z\left(w\right)$ para $w\in\Omega$ del mapeo aleatorio $Z$.
\end{Def}

\begin{Note}
Algunas restricciones se imponen sobre las trayectorias, por ejemplo que sean continuas por la derecha, o continuas por la derecha con l\'imites por la izquierda, o de manera m\'as general, se pedir\'a que caigan en alg\'un subconjunto $H$ de $E^{\mathbb{I}}$. En este caso es natural considerar a $Z$ como un elemento aleatorio que no est\'a en $\left(E^{\mathbb{I}},\mathcal{E}^{\mathbb{I}}\right)$ sino en $\left(H,\mathcal{H}\right)$, donde $\mathcal{H}$ es la $\sigma$-\'algebra generada por los mapeos proyecci\'on que toman a $z\in H$ a $z_{t}\in E$ para $t\in\mathbb{I}$. A $\mathcal{H}$ se le conoce como la traza de $H$ en $E^{\mathbb{I}}$, es decir,
\begin{eqnarray}
\mathcal{H}:=E^{\mathbb{I}}\cap H:=\left\{A\cap H:A\in E^{\mathbb{I}}\right\}.
\end{eqnarray}
\end{Note}


\begin{Note}
$Z$ tiene trayectorias con valores en $H$ y cada $Z_{t}$ es un mapeo medible de $\left(\Omega,\mathcal{F}\right)$ a $\left(H,\mathcal{H}\right)$. Cuando se considera un espacio de trayectorias en particular $H$, al espacio $\left(H,\mathcal{H}\right)$ se le llama el espacio de trayectorias de $Z$.
\end{Note}

\begin{Note}
La distribuci\'on del proceso estoc\'astico $Z$ con espacio de trayectorias $\left(H,\mathcal{H}\right)$ es la distribuci\'on de $Z$ como  un elemento aleatorio en $\left(H,\mathcal{H}\right)$. La distribuci\'on, nuevemente, est\'a determinada de manera \'unica por las distribuciones finito dimensionales.
\end{Note}


\begin{Def}
Sea $Z$ un PEOSCT  con espacio de estados $\left(E,\mathcal{E}\right)$ y sea $T$ un tiempo aleatorio en $\left[0,\infty\right)$. Por $Z_{T}$ se entiende el mapeo con valores en $E$ definido en $\Omega$ en la manera obvia:
\begin{eqnarray*}
Z_{T}\left(w\right):=Z_{T\left(w\right)}\left(w\right). w\in\Omega.
\end{eqnarray*}
\end{Def}

\begin{Def}
Un PEOSCT $Z$ es conjuntamente medible (\textbf{CM}) si el mapeo que toma $\left(w,t\right)\in\Omega\times\left[0,\infty\right)$ a $Z_{t}\left(w\right)\in E$ es $\mathcal{F}\otimes\mathcal{B}\left[0,\infty\right)/\mathcal{E}$ medible.
\end{Def}

\begin{Note}
Un PEOSCT-CM implica que el proceso es medible, dado que $Z_{T}$ es una composici\'on  de dos mapeos continuos: el primero que toma $w$ en $\left(w,T\left(w\right)\right)$ es $\mathcal{F}/\mathcal{F}\otimes\mathcal{B}\left[0,\infty\right)$ medible, mientras que el segundo toma $\left(w,T\left(w\right)\right)$ en $Z_{T\left(w\right)}\left(w\right)$ es $\mathcal{F}\otimes\mathcal{B}\left[0,\infty\right)/\mathcal{E}$ medible.
\end{Note}


\begin{Def}
Un PEOSCT con espacio de estados $\left(H,\mathcal{H}\right)$ es can\'onicamente conjuntamente medible (\textbf{CCM}) si el mapeo $\left(z,t\right)\in H\times\left[0,\infty\right)$ en $Z_{t}\in E$ es $\mathcal{H}\otimes\mathcal{B}\left[0,\infty\right)/\mathcal{E}$ medible.
\end{Def}

\begin{Note}
Un PEOSCTCCM implica que el proceso es CM, dado que un PECCM $Z$ es un mapeo de $\Omega\times\left[0,\infty\right)$ a $E$, es la composici\'on de dos mapeos medibles: el primero, toma $\left(w,t\right)$ en $\left(Z\left(w\right),t\right)$ es $\mathcal{F}\otimes\mathcal{B}\left[0,\infty\right)/\mathcal{H}\otimes\mathcal{B}\left[0,\infty\right)$ medible, y el segundo que toma $\left(Z\left(w\right),t\right)$  en $Z_{t}\left(w\right)$ es $\mathcal{H}\otimes\mathcal{B}\left[0,\infty\right)/\mathcal{E}$ medible. Por tanto CCM es una condici\'on m\'as fuerte que CM.
\end{Note}

\begin{Def}
Un conjunto de trayectorias $H$ de un PEOSCT $Z$, es internamente shift-invariante (\textbf{ISI}) si 
\begin{eqnarray*}
\left\{\left(z_{t+s}\right)_{s\in\left[0,\infty\right)}:z\in H\right\}=H\textrm{, }t\in\left[0,\infty\right).
\end{eqnarray*}
\end{Def}


\begin{Def}
Dado un PEOSCTISI, se define el mapeo-shift $\theta_{t}$, $t\in\left[0,\infty\right)$, de $H$ a $H$ por 
\begin{eqnarray*}
\theta_{t}z=\left(z_{t+s}\right)_{s\in\left[0,\infty\right)}\textrm{, }z\in H.
\end{eqnarray*}
\end{Def}

\begin{Def}
Se dice que un proceso $Z$ es shift-medible (\textbf{SM}) si $Z$ tiene un conjunto de trayectorias $H$ que es ISI y adem\'as el mapeo que toma $\left(z,t\right)\in H\times\left[0,\infty\right)$ en $\theta_{t}z\in H$ es $\mathcal{H}\otimes\mathcal{B}\left[0,\infty\right)/\mathcal{H}$ medible.
\end{Def}

\begin{Note}
Un proceso estoc\'astico con conjunto de trayectorias $H$ ISI es shift-medible si y s\'olo si es CCM
\end{Note}

\begin{Note}
\begin{itemize}
\item Dado el espacio polaco $\left(E,\mathcal{E}\right)$ se tiene el  conjunto de trayectorias $D_{E}\left[0,\infty\right)$ que es ISI, entonces cumpe con ser CCM.

\item Si $G$ es abierto, podemos cubrirlo por bolas abiertas cuay cerradura este contenida en $G$, y como $G$ es segundo numerable como subespacio de $E$, lo podemos cubrir por una cantidad numerable de bolas abiertas.

\end{itemize}
\end{Note}


\begin{Note}
Los procesos estoc\'asticos $Z$ a tiempo discreto con espacio de estados polaco, tambi\'en tiene un espacio de trayectorias polaco y por tanto tiene distribuciones condicionales regulares.
\end{Note}

\begin{Teo}
El producto numerable de espacios polacos es polaco.
\end{Teo}


\begin{Def}
Sea $\left(\Omega,\mathcal{F},\prob\right)$ espacio de probabilidad que soporta al proceso $Z=\left(Z_{s}\right)_{s\in\left[0,\infty\right)}$ y $S=\left(S_{k}\right)_{0}^{\infty}$ donde $Z$ es un PEOSCTM con espacio de estados $\left(E,\mathcal{E}\right)$  y espacio de trayectorias $\left(H,\mathcal{H}\right)$  y adem\'as $S$ es una sucesi\'on de tiempos aleatorios one-sided que satisfacen la condici\'on $0\leq S_{0}<S_{1}<\cdots\rightarrow\infty$. Considerando $S$ como un mapeo medible de $\left(\Omega,\mathcal{F}\right)$ al espacio sucesi\'on $\left(L,\mathcal{L}\right)$, donde 
\begin{eqnarray*}
L=\left\{\left(s_{k}\right)_{0}^{\infty}\in\left[0,\infty\right)^{\left\{0,1,\ldots\right\}}:s_{0}<s_{1}<\cdots\rightarrow\infty\right\},
\end{eqnarray*}
donde $\mathcal{L}$ son los subconjuntos de Borel de $L$, es decir, $\mathcal{L}=L\cap\mathcal{B}^{\left\{0,1,\ldots\right\}}$.

As\'i el par $\left(Z,S\right)$ es un mapeo medible de  $\left(\Omega,\mathcal{F}\right)$ en $\left(H\times L,\mathcal{H}\otimes\mathcal{L}\right)$. El par $\mathcal{H}\otimes\mathcal{L}^{+}$ denotar\'a la clase de todas las funciones medibles de $\left(H\times L,\mathcal{H}\otimes\mathcal{L}\right)$ en $\left(\left[0,\infty\right),\mathcal{B}\left[0,\infty\right)\right)$.
\end{Def}


\begin{Def}
Sea $\theta_{t}$ el mapeo-shift conjunto de $H\times L$ en $H\times L$ dado por
\begin{eqnarray*}
\theta_{t}\left(z,\left(s_{k}\right)_{0}^{\infty}\right)=\theta_{t}\left(z,\left(s_{n_{t-}+k}-t\right)_{0}^{\infty}\right)
\end{eqnarray*}
donde 
$n_{t-}=inf\left\{n\geq1:s_{n}\geq t\right\}$.
\end{Def}

\begin{Note}
Con la finalidad de poder realizar los shift's sin complicaciones de medibilidad, se supondr\'a que $Z$ es shit-medible, es decir, el conjunto de trayectorias $H$ es invariante bajo shifts del tiempo y el mapeo que toma $\left(z,t\right)\in H\times\left[0,\infty\right)$ en $z_{t}\in E$ es $\mathcal{H}\otimes\mathcal{B}\left[0,\infty\right)/\mathcal{E}$ medible.
\end{Note}

\begin{Def}
Dado un proceso \textbf{PEOSSM} (Proceso Estoc\'astico One Side Shift Medible) $Z$, se dice regenerativo cl\'asico con tiempos de regeneraci\'on $S$ si 

\begin{eqnarray*}
\theta_{S_{n}}\left(Z,S\right)=\left(Z^{0},S^{0}\right),n\geq0
\end{eqnarray*}
y adem\'as $\theta_{S_{n}}\left(Z,S\right)$ es independiente de $\left(\left(Z_{s}\right)s\in\left[0,S_{n}\right),S_{0},\ldots,S_{n}\right)$
Si lo anterior se cumple, al par $\left(Z,S\right)$ se le llama regenerativo cl\'asico.
\end{Def}

\begin{Note}
Si el par $\left(Z,S\right)$ es regenerativo cl\'asico, entonces las longitudes de los ciclos $X_{1},X_{2},\ldots,$ son i.i.d. e independientes de la longitud del retraso $S_{0}$, es decir, $S$ es un proceso de renovaci\'on. Las longitudes de los ciclos tambi\'en son llamados tiempos de inter-regeneraci\'on y tiempos de ocurrencia.

\end{Note}

\begin{Teo}
Sup\'ongase que el par $\left(Z,S\right)$ es regenerativo cl\'asico con $\esp\left[X_{1}\right]<\infty$. Entonces $\left(Z^{*},S^{*}\right)$ en el teorema 2.1 es una versi\'on estacionaria de $\left(Z,S\right)$. Adem\'as, si $X_{1}$ es lattice con span $d$, entonces $\left(Z^{**},S^{**}\right)$ en el teorema 2.2 es una versi\'on periodicamente estacionaria de $\left(Z,S\right)$ con periodo $d$.

\end{Teo}

\begin{Def}
Una variable aleatoria $X_{1}$ es \textit{spread out} si existe una $n\geq1$ y una  funci\'on $f\in\mathcal{B}^{+}$ tal que $\int_{\rea}f\left(x\right)dx>0$ con $X_{2},X_{3},\ldots,X_{n}$ copias i.i.d  de $X_{1}$, $$\prob\left(X_{1}+\cdots+X_{n}\in B\right)\geq\int_{B}f\left(x\right)dx$$ para $B\in\mathcal{B}$.

\end{Def}



\begin{Def}
Dado un proceso estoc\'astico $Z$ se le llama \textit{wide-sense regenerative} (\textbf{WSR}) con tiempos de regeneraci\'on $S$ si $\theta_{S_{n}}\left(Z,S\right)=\left(Z^{0},S^{0}\right)$ para $n\geq0$ en distribuci\'on y $\theta_{S_{n}}\left(Z,S\right)$ es independiente de $\left(S_{0},S_{1},\ldots,S_{n}\right)$ para $n\geq0$.
Se dice que el par $\left(Z,S\right)$ es WSR si lo anterior se cumple.
\end{Def}


\begin{Note}
\begin{itemize}
\item El proceso de trayectorias $\left(\theta_{s}Z\right)_{s\in\left[0,\infty\right)}$ es WSR con tiempos de regeneraci\'on $S$ pero no es regenerativo cl\'asico.

\item Si $Z$ es cualquier proceso estacionario y $S$ es un proceso de renovaci\'on que es independiente de $Z$, entonces $\left(Z,S\right)$ es WSR pero en general no es regenerativo cl\'asico

\end{itemize}

\end{Note}


\begin{Note}
Para cualquier proceso estoc\'astico $Z$, el proceso de trayectorias $\left(\theta_{s}Z\right)_{s\in\left[0,\infty\right)}$ es siempre un proceso de Markov.
\end{Note}



\begin{Teo}
Supongase que el par $\left(Z,S\right)$ es WSR con $\esp\left[X_{1}\right]<\infty$. Entonces $\left(Z^{*},S^{*}\right)$ en el teorema 2.1 es una versi\'on estacionaria de 
$\left(Z,S\right)$.
\end{Teo}


\begin{Teo}
Supongase que $\left(Z,S\right)$ es cycle-stationary con $\esp\left[X_{1}\right]<\infty$. Sea $U$ distribuida uniformemente en $\left[0,1\right)$ e independiente de $\left(Z^{0},S^{0}\right)$ y sea $\prob^{*}$ la medida de probabilidad en $\left(\Omega,\prob\right)$ definida por $$d\prob^{*}=\frac{X_{1}}{\esp\left[X_{1}\right]}d\prob$$. Sea $\left(Z^{*},S^{*}\right)$ con distribuci\'on $\prob^{*}\left(\theta_{UX_{1}}\left(Z^{0},S^{0}\right)\in\cdot\right)$. Entonces $\left(Z^{}*,S^{*}\right)$ es estacionario,
\begin{eqnarray*}
\esp\left[f\left(Z^{*},S^{*}\right)\right]=\esp\left[\int_{0}^{X_{1}}f\left(\theta_{s}\left(Z^{0},S^{0}\right)\right)ds\right]/\esp\left[X_{1}\right]
\end{eqnarray*}
$f\in\mathcal{H}\otimes\mathcal{L}^{+}$, and $S_{0}^{*}$ es continuo con funci\'on distribuci\'on $G_{\infty}$ definida por $$G_{\infty}\left(x\right):=\frac{\esp\left[X_{1}\right]\wedge x}{\esp\left[X_{1}\right]}$$ para $x\geq0$ y densidad $\prob\left[X_{1}>x\right]/\esp\left[X_{1}\right]$, con $x\geq0$.

\end{Teo}


\begin{Teo}
Sea $Z$ un Proceso Estoc\'astico un lado shift-medible \textit{one-sided shift-measurable stochastic process}, (PEOSSM),
y $S_{0}$ y $S_{1}$ tiempos aleatorios tales que $0\leq S_{0}<S_{1}$ y
\begin{equation}
\theta_{S_{1}}Z=\theta_{S_{0}}Z\textrm{ en distribuci\'on}.
\end{equation}

Entonces el espacio de probabilidad subyacente $\left(\Omega,\mathcal{F},\prob\right)$ puede extenderse para soportar una sucesi\'on de tiempos aleatorios $S$ tales que

\begin{eqnarray}
\theta_{S_{n}}\left(Z,S\right)=\left(Z^{0},S^{0}\right),n\geq0,\textrm{ en distribuci\'on},\\
\left(Z,S_{0},S_{1}\right)\textrm{ depende de }\left(X_{2},X_{3},\ldots\right)\textrm{ solamente a traves de }\theta_{S_{1}}Z.
\end{eqnarray}
\end{Teo}





%_________________________________________________________________________
%
%\subsection{Una vez que se tiene estabilidad}
%_________________________________________________________________________
%

Also the intervisit time $I_{i}$ is defined as the period beginning at the time of its service completion in a cycle and ending at the time when it is polled in the next cycle; its duration is given by $\tau_{i}\left(m+1\right)-\overline{\tau}_{i}\left(m\right)$.

So we the following are still true 

\begin{eqnarray}
\begin{array}{ll}
\esp\left[L_{i}\right]=\mu_{i}\esp\left[I_{i}\right], &
\esp\left[C_{i}\right]=\frac{f_{i}\left(i\right)}{\mu_{i}\left(1-\mu_{i}\right)},\\
\esp\left[S_{i}\right]=\mu_{i}\esp\left[C_{i}\right],&
\esp\left[I_{i}\right]=\left(1-\mu_{i}\right)\esp\left[C_{i}\right],\\
Var\left[L_{i}\right]= \mu_{i}^{2}Var\left[I_{i}\right]+\sigma^{2}\esp\left[I_{i}\right],& 
Var\left[C_{i}\right]=\frac{Var\left[L_{i}^{*}\right]}{\mu_{i}^{2}\left(1-\mu_{i}\right)^{2}},\\
Var\left[S_{i}\right]= \frac{Var\left[L_{i}^{*}\right]}{\left(1-\mu_{i}\right)^{2}}+\frac{\sigma^{2}\esp\left[L_{i}^{*}\right]}{\left(1-\mu_{i}\right)^{3}},&
Var\left[I_{i}\right]= \frac{Var\left[L_{i}^{*}\right]}{\mu_{i}^{2}}-\frac{\sigma_{i}^{2}}{\mu_{i}^{2}}f_{i}\left(i\right).
\end{array}
\end{eqnarray}
\begin{Def}
El tiempo de Ciclo $C_{i}$ es el periodo de tiempo que comienza cuando la cola $i$ es visitada por primera vez en un ciclo, y termina cuando es visitado nuevamente en el pr\'oximo ciclo. La duraci\'on del mismo est\'a dada por $\tau_{i}\left(m+1\right)-\tau_{i}\left(m\right)$, o equivalentemente $\overline{\tau}_{i}\left(m+1\right)-\overline{\tau}_{i}\left(m\right)$ bajo condiciones de estabilidad.
\end{Def}


\begin{Def}
El tiempo de intervisita $I_{i}$ es el periodo de tiempo que comienza cuando se ha completado el servicio en un ciclo y termina cuando es visitada nuevamente en el pr\'oximo ciclo. Su  duraci\'on del mismo est\'a dada por $\tau_{i}\left(m+1\right)-\overline{\tau}_{i}\left(m\right)$.
\end{Def}

La duraci\'on del tiempo de intervisita es $\tau_{i}\left(m+1\right)-\overline{\tau}\left(m\right)$. Dado que el n\'umero de usuarios presentes en $Q_{i}$ al tiempo $t=\tau_{i}\left(m+1\right)$ es igual al n\'umero de arribos durante el intervalo de tiempo $\left[\overline{\tau}\left(m\right),\tau_{i}\left(m+1\right)\right]$ se tiene que


\begin{eqnarray*}
\esp\left[z_{i}^{L_{i}\left(\tau_{i}\left(m+1\right)\right)}\right]=\esp\left[\left\{P_{i}\left(z_{i}\right)\right\}^{\tau_{i}\left(m+1\right)-\overline{\tau}\left(m\right)}\right]
\end{eqnarray*}

entonces, si $I_{i}\left(z\right)=\esp\left[z^{\tau_{i}\left(m+1\right)-\overline{\tau}\left(m\right)}\right]$
se tiene que $F_{i}\left(z\right)=I_{i}\left[P_{i}\left(z\right)\right]$
para $i=1,2$.

Conforme a la definici\'on dada al principio del cap\'itulo, definici\'on (\ref{Def.Tn}), sean $T_{1},T_{2},\ldots$ los puntos donde las longitudes de las colas de la red de sistemas de visitas c\'iclicas son cero simult\'aneamente, cuando la cola $Q_{j}$ es visitada por el servidor para dar servicio, es decir, $L_{1}\left(T_{i}\right)=0,L_{2}\left(T_{i}\right)=0,\hat{L}_{1}\left(T_{i}\right)=0$ y $\hat{L}_{2}\left(T_{i}\right)=0$, a estos puntos se les denominar\'a puntos regenerativos. Entonces, 

\begin{Def}
Al intervalo de tiempo entre dos puntos regenerativos se le llamar\'a ciclo regenerativo.
\end{Def}

\begin{Def}
Para $T_{i}$ se define, $M_{i}$, el n\'umero de ciclos de visita a la cola $Q_{l}$, durante el ciclo regenerativo, es decir, $M_{i}$ es un proceso de renovaci\'on.
\end{Def}

\begin{Def}
Para cada uno de los $M_{i}$'s, se definen a su vez la duraci\'on de cada uno de estos ciclos de visita en el ciclo regenerativo, $C_{i}^{(m)}$, para $m=1,2,\ldots,M_{i}$, que a su vez, tambi\'en es n proceso de renovaci\'on.
\end{Def}


Sea la funci\'on generadora de momentos para $L_{i}$, el n\'umero de usuarios en la cola $Q_{i}\left(z\right)$ en cualquier momento, est\'a dada por el tiempo promedio de $z^{L_{i}\left(t\right)}$ sobre el ciclo regenerativo definido anteriormente:

\begin{eqnarray*}
Q_{i}\left(z\right)&=&\esp\left[z^{L_{i}\left(t\right)}\right]=\frac{\esp\left[\sum_{m=1}^{M_{i}}\sum_{t=\tau_{i}\left(m\right)}^{\tau_{i}\left(m+1\right)-1}z^{L_{i}\left(t\right)}\right]}{\esp\left[\sum_{m=1}^{M_{i}}\tau_{i}\left(m+1\right)-\tau_{i}\left(m\right)\right]}
\end{eqnarray*}

$M_{i}$ es un tiempo de paro en el proceso regenerativo con $\esp\left[M_{i}\right]<\infty$\footnote{En Stidham\cite{Stidham} y Heyman  se muestra que una condici\'on suficiente para que el proceso regenerativo 
estacionario sea un procesoo estacionario es que el valor esperado del tiempo del ciclo regenerativo sea finito, es decir: $\esp\left[\sum_{m=1}^{M_{i}}C_{i}^{(m)}\right]<\infty$, como cada $C_{i}^{(m)}$ contiene intervalos de r\'eplica positivos, se tiene que $\esp\left[M_{i}\right]<\infty$, adem\'as, como $M_{i}>0$, se tiene que la condici\'on anterior es equivalente a tener que $\esp\left[C_{i}\right]<\infty$,
por lo tanto una condici\'on suficiente para la existencia del proceso regenerativo est\'a dada por $\sum_{k=1}^{N}\mu_{k}<1.$}, se sigue del lema de Wald que:


\begin{eqnarray*}
\esp\left[\sum_{m=1}^{M_{i}}\sum_{t=\tau_{i}\left(m\right)}^{\tau_{i}\left(m+1\right)-1}z^{L_{i}\left(t\right)}\right]&=&\esp\left[M_{i}\right]\esp\left[\sum_{t=\tau_{i}\left(m\right)}^{\tau_{i}\left(m+1\right)-1}z^{L_{i}\left(t\right)}\right]\\
\esp\left[\sum_{m=1}^{M_{i}}\tau_{i}\left(m+1\right)-\tau_{i}\left(m\right)\right]&=&\esp\left[M_{i}\right]\esp\left[\tau_{i}\left(m+1\right)-\tau_{i}\left(m\right)\right]
\end{eqnarray*}

por tanto se tiene que


\begin{eqnarray*}
Q_{i}\left(z\right)&=&\frac{\esp\left[\sum_{t=\tau_{i}\left(m\right)}^{\tau_{i}\left(m+1\right)-1}z^{L_{i}\left(t\right)}\right]}{\esp\left[\tau_{i}\left(m+1\right)-\tau_{i}\left(m\right)\right]}
\end{eqnarray*}

observar que el denominador es simplemente la duraci\'on promedio del tiempo del ciclo.


Haciendo las siguientes sustituciones en la ecuaci\'on (\ref{Corolario2}): $n\rightarrow t-\tau_{i}\left(m\right)$, $T \rightarrow \overline{\tau}_{i}\left(m\right)-\tau_{i}\left(m\right)$, $L_{n}\rightarrow L_{i}\left(t\right)$ y $F\left(z\right)=\esp\left[z^{L_{0}}\right]\rightarrow F_{i}\left(z\right)=\esp\left[z^{L_{i}\tau_{i}\left(m\right)}\right]$, se puede ver que

\begin{eqnarray}\label{Eq.Arribos.Primera}
\esp\left[\sum_{n=0}^{T-1}z^{L_{n}}\right]=
\esp\left[\sum_{t=\tau_{i}\left(m\right)}^{\overline{\tau}_{i}\left(m\right)-1}z^{L_{i}\left(t\right)}\right]
=z\frac{F_{i}\left(z\right)-1}{z-P_{i}\left(z\right)}
\end{eqnarray}

Por otra parte durante el tiempo de intervisita para la cola $i$, $L_{i}\left(t\right)$ solamente se incrementa de manera que el incremento por intervalo de tiempo est\'a dado por la funci\'on generadora de probabilidades de $P_{i}\left(z\right)$, por tanto la suma sobre el tiempo de intervisita puede evaluarse como:

\begin{eqnarray*}
\esp\left[\sum_{t=\tau_{i}\left(m\right)}^{\tau_{i}\left(m+1\right)-1}z^{L_{i}\left(t\right)}\right]&=&\esp\left[\sum_{t=\tau_{i}\left(m\right)}^{\tau_{i}\left(m+1\right)-1}\left\{P_{i}\left(z\right)\right\}^{t-\overline{\tau}_{i}\left(m\right)}\right]=\frac{1-\esp\left[\left\{P_{i}\left(z\right)\right\}^{\tau_{i}\left(m+1\right)-\overline{\tau}_{i}\left(m\right)}\right]}{1-P_{i}\left(z\right)}\\
&=&\frac{1-I_{i}\left[P_{i}\left(z\right)\right]}{1-P_{i}\left(z\right)}
\end{eqnarray*}
por tanto

\begin{eqnarray*}
\esp\left[\sum_{t=\tau_{i}\left(m\right)}^{\tau_{i}\left(m+1\right)-1}z^{L_{i}\left(t\right)}\right]&=&
\frac{1-F_{i}\left(z\right)}{1-P_{i}\left(z\right)}
\end{eqnarray*}

Por lo tanto

\begin{eqnarray*}
Q_{i}\left(z\right)&=&\frac{\esp\left[\sum_{t=\tau_{i}\left(m\right)}^{\tau_{i}\left(m+1\right)-1}z^{L_{i}\left(t\right)}\right]}{\esp\left[\tau_{i}\left(m+1\right)-\tau_{i}\left(m\right)\right]}
=\frac{1}{\esp\left[\tau_{i}\left(m+1\right)-\tau_{i}\left(m\right)\right]}
\esp\left[\sum_{t=\tau_{i}\left(m\right)}^{\tau_{i}\left(m+1\right)-1}z^{L_{i}\left(t\right)}\right]\\
&=&\frac{1}{\esp\left[\tau_{i}\left(m+1\right)-\tau_{i}\left(m\right)\right]}
\esp\left[\sum_{t=\tau_{i}\left(m\right)}^{\overline{\tau}_{i}\left(m\right)-1}z^{L_{i}\left(t\right)}
+\sum_{t=\overline{\tau}_{i}\left(m\right)}^{\tau_{i}\left(m+1\right)-1}z^{L_{i}\left(t\right)}\right]\\
&=&\frac{1}{\esp\left[\tau_{i}\left(m+1\right)-\tau_{i}\left(m\right)\right]}\left\{
\esp\left[\sum_{t=\tau_{i}\left(m\right)}^{\overline{\tau}_{i}\left(m\right)-1}z^{L_{i}\left(t\right)}\right]
+\esp\left[\sum_{t=\overline{\tau}_{i}\left(m\right)}^{\tau_{i}\left(m+1\right)-1}z^{L_{i}\left(t\right)}\right]\right\}\\
&=&\frac{1}{\esp\left[\tau_{i}\left(m+1\right)-\tau_{i}\left(m\right)\right]}\left\{
z\frac{F_{i}\left(z\right)-1}{z-P_{i}\left(z\right)}+\frac{1-F_{i}\left(z\right)}{1-P_{i}\left(z\right)}
\right\}\\
&=&\frac{1}{\esp\left[C_{i}\right]}\cdot\frac{1-F_{i}\left(z\right)}{P_{i}\left(z\right)-z}\cdot\frac{\left(1-z\right)P_{i}\left(z\right)}{1-P_{i}\left(z\right)}
\end{eqnarray*}

es decir

\begin{equation}
Q_{i}\left(z\right)=\frac{1}{\esp\left[C_{i}\right]}\cdot\frac{1-F_{i}\left(z\right)}{P_{i}\left(z\right)-z}\cdot\frac{\left(1-z\right)P_{i}\left(z\right)}{1-P_{i}\left(z\right)}
\end{equation}


Si hacemos:

\begin{eqnarray}
S\left(z\right)&=&1-F\left(z\right)\\
T\left(z\right)&=&z-P\left(z\right)\\
U\left(z\right)&=&1-P\left(z\right)
\end{eqnarray}
entonces 

\begin{eqnarray}
\esp\left[C_{i}\right]Q\left(z\right)=\frac{\left(z-1\right)S\left(z\right)P\left(z\right)}{T\left(z\right)U\left(z\right)}
\end{eqnarray}

A saber, si $a_{k}=P\left\{L\left(t\right)=k\right\}$
\begin{eqnarray*}
S\left(z\right)=1-F\left(z\right)=1-\sum_{k=0}^{+\infty}a_{k}z^{k}
\end{eqnarray*}
entonces

%\begin{eqnarray}
%\begin{array}{ll}
%S^{'}\left(z\right)=-\sum_{k=1}^{+\infty}ka_{k}z^{k-1},& %S^{(1)}\left(1\right)=-\sum_{k=1}^{+\infty}ka_{k}=-\esp\left[L\left(t\right)\right],\\
%S^{''}\left(z\right)=-\sum_{k=2}^{+\infty}k(k-1)a_{k}z^{k-2},& S^{(2)}\left(1\right)=-\sum_{k=2}^{+\infty}k(k-1)a_{k}=\esp\left[L\left(L-1\right)\right],\\
%S^{'''}\left(z\right)=-\sum_{k=3}^{+\infty}k(k-1)(k-2)a_{k}z^{k-3},&
%S^{(3)}\left(1\right)=-\sum_{k=3}^{+\infty}k(k-1)(k-2)a_{k}\\
%&=-\esp\left[L\left(L-1\right)\left(L-2\right)\right]\\
%&=-\esp\left[L^{3}\right]+3-\esp\left[L^{2}\right]-2-\esp\left[L\right];
%\end{array}
%\end{eqnarray}

$S^{'}\left(z\right)=-\sum_{k=1}^{+\infty}ka_{k}z^{k-1}$, por tanto $S^{(1)}\left(1\right)=-\sum_{k=1}^{+\infty}ka_{k}=-\esp\left[L\left(t\right)\right]$,
luego $S^{''}\left(z\right)=-\sum_{k=2}^{+\infty}k(k-1)a_{k}z^{k-2}$ y $S^{(2)}\left(1\right)=-\sum_{k=2}^{+\infty}k(k-1)a_{k}=\esp\left[L\left(L-1\right)\right]$;
de la misma manera $S^{'''}\left(z\right)=-\sum_{k=3}^{+\infty}k(k-1)(k-2)a_{k}z^{k-3}$ y $S^{(3)}\left(1\right)=-\sum_{k=3}^{+\infty}k(k-1)(k-2)a_{k}=-\esp\left[L\left(L-1\right)\left(L-2\right)\right]
=-\esp\left[L^{3}\right]+3-\esp\left[L^{2}\right]-2-\esp\left[L\right]$. 

Es decir

\begin{eqnarray*}
S^{(1)}\left(1\right)&=&-\esp\left[L\left(t\right)\right],\\ S^{(2)}\left(1\right)&=&-\esp\left[L\left(L-1\right)\right]
=-\esp\left[L^{2}\right]+\esp\left[L\right],\\
S^{(3)}\left(1\right)&=&-\esp\left[L\left(L-1\right)\left(L-2\right)\right]
=-\esp\left[L^{3}\right]+3\esp\left[L^{2}\right]-2\esp\left[L\right].
\end{eqnarray*}


Expandiendo alrededor de $z=1$

\begin{eqnarray*}
S\left(z\right)&=&S\left(1\right)+\frac{S^{'}\left(1\right)}{1!}\left(z-1\right)+\frac{S^{''}\left(1\right)}{2!}\left(z-1\right)^{2}+\frac{S^{'''}\left(1\right)}{3!}\left(z-1\right)^{3}+\ldots+\\
&=&\left(z-1\right)\left\{S^{'}\left(1\right)+\frac{S^{''}\left(1\right)}{2!}\left(z-1\right)+\frac{S^{'''}\left(1\right)}{3!}\left(z-1\right)^{2}+\ldots+\right\}\\
&=&\left(z-1\right)R_{1}\left(z\right)
\end{eqnarray*}
con $R_{1}\left(z\right)\neq0$, pues

\begin{eqnarray}\label{Eq.R1}
R_{1}\left(z\right)=-\esp\left[L\right]
\end{eqnarray}
entonces

\begin{eqnarray}
R_{1}\left(z\right)&=&S^{'}\left(1\right)+\frac{S^{''}\left(1\right)}{2!}\left(z-1\right)+\frac{S^{'''}\left(1\right)}{3!}\left(z-1\right)^{2}+\frac{S^{iv}\left(1\right)}{4!}\left(z-1\right)^{3}+\ldots+
\end{eqnarray}
Calculando las derivadas y evaluando en $z=1$

\begin{eqnarray}
R_{1}\left(1\right)&=&S^{(1)}\left(1\right)=-\esp\left[L\right]\\
R_{1}^{(1)}\left(1\right)&=&\frac{1}{2}S^{(2)}\left(1\right)=-\frac{1}{2}\esp\left[L^{2}\right]+\frac{1}{2}\esp\left[L\right]\\
R_{1}^{(2)}\left(1\right)&=&\frac{2}{3!}S^{(3)}\left(1\right)
=-\frac{1}{3}\esp\left[L^{3}\right]+\esp\left[L^{2}\right]-\frac{2}{3}\esp\left[L\right]
\end{eqnarray}

De manera an\'aloga se puede ver que para $T\left(z\right)=z-P\left(z\right)$ se puede encontrar una expanci\'on alrededor de $z=1$

Expandiendo alrededor de $z=1$

\begin{eqnarray*}
T\left(z\right)&=&T\left(1\right)+\frac{T^{'}\left(1\right)}{1!}\left(z-1\right)+\frac{T^{''}\left(1\right)}{2!}\left(z-1\right)^{2}+\frac{T^{'''}\left(1\right)}{3!}\left(z-1\right)^{3}+\ldots+\\
&=&\left(z-1\right)\left\{T^{'}\left(1\right)+\frac{T^{''}\left(1\right)}{2!}\left(z-1\right)+\frac{T^{'''}\left(1\right)}{3!}\left(z-1\right)^{2}+\ldots+\right\}\\
&=&\left(z-1\right)R_{2}\left(z\right)
\end{eqnarray*}

donde 
\begin{eqnarray*}
T^{(1)}\left(1\right)&=&-\esp\left[X\left(t\right)\right]=-\mu,\\ T^{(2)}\left(1\right)&=&-\esp\left[X\left(X-1\right)\right]
=-\esp\left[X^{2}\right]+\esp\left[X\right]=-\esp\left[X^{2}\right]+\mu,\\
T^{(3)}\left(1\right)&=&-\esp\left[X\left(X-1\right)\left(X-2\right)\right]
=-\esp\left[X^{3}\right]+3\esp\left[X^{2}\right]-2\esp\left[X\right]\\
&=&-\esp\left[X^{3}\right]+3\esp\left[X^{2}\right]-2\mu.
\end{eqnarray*}

Por lo tanto $R_{2}\left(1\right)\neq0$, pues

\begin{eqnarray}\label{Eq.R2}
R_{2}\left(1\right)=1-\esp\left[X\right]=1-\mu
\end{eqnarray}
entonces

\begin{eqnarray}
R_{2}\left(z\right)&=&T^{'}\left(1\right)+\frac{T^{''}\left(1\right)}{2!}\left(z-1\right)+\frac{T^{'''}\left(1\right)}{3!}\left(z-1\right)^{2}+\frac{T^{(iv)}\left(1\right)}{4!}\left(z-1\right)^{3}+\ldots+
\end{eqnarray}
Calculando las derivadas y evaluando en $z=1$

\begin{eqnarray}
R_{2}\left(1\right)&=&T^{(1)}\left(1\right)=1-\mu\\
R_{2}^{(1)}\left(1\right)&=&\frac{1}{2}T^{(2)}\left(1\right)=-\frac{1}{2}\esp\left[X^{2}\right]+\frac{1}{2}\mu\\
R_{2}^{(2)}\left(1\right)&=&\frac{2}{3!}T^{(3)}\left(1\right)
=-\frac{1}{3}\esp\left[X^{3}\right]+\esp\left[X^{2}\right]-\frac{2}{3}\mu
\end{eqnarray}

Finalmente para de manera an\'aloga se puede ver que para $U\left(z\right)=1-P\left(z\right)$ se puede encontrar una expanci\'on alrededor de $z=1$

\begin{eqnarray*}
U\left(z\right)&=&U\left(1\right)+\frac{U^{'}\left(1\right)}{1!}\left(z-1\right)+\frac{U^{''}\left(1\right)}{2!}\left(z-1\right)^{2}+\frac{U^{'''}\left(1\right)}{3!}\left(z-1\right)^{3}+\ldots+\\
&=&\left(z-1\right)\left\{U^{'}\left(1\right)+\frac{U^{''}\left(1\right)}{2!}\left(z-1\right)+\frac{U^{'''}\left(1\right)}{3!}\left(z-1\right)^{2}+\ldots+\right\}\\
&=&\left(z-1\right)R_{3}\left(z\right)
\end{eqnarray*}

donde 
\begin{eqnarray*}
U^{(1)}\left(1\right)&=&-\esp\left[X\left(t\right)\right]=-\mu,\\ U^{(2)}\left(1\right)&=&-\esp\left[X\left(X-1\right)\right]
=-\esp\left[X^{2}\right]+\esp\left[X\right]=-\esp\left[X^{2}\right]+\mu,\\
U^{(3)}\left(1\right)&=&-\esp\left[X\left(X-1\right)\left(X-2\right)\right]
=-\esp\left[X^{3}\right]+3\esp\left[X^{2}\right]-2\esp\left[X\right]\\
&=&-\esp\left[X^{3}\right]+3\esp\left[X^{2}\right]-2\mu.
\end{eqnarray*}

Por lo tanto $R_{3}\left(1\right)\neq0$, pues

\begin{eqnarray}\label{Eq.R2}
R_{3}\left(1\right)=-\esp\left[X\right]=-\mu
\end{eqnarray}
entonces

\begin{eqnarray}
R_{3}\left(z\right)&=&U^{'}\left(1\right)+\frac{U^{''}\left(1\right)}{2!}\left(z-1\right)+\frac{U^{'''}\left(1\right)}{3!}\left(z-1\right)^{2}+\frac{U^{(iv)}\left(1\right)}{4!}\left(z-1\right)^{3}+\ldots+
\end{eqnarray}

Calculando las derivadas y evaluando en $z=1$

\begin{eqnarray}
R_{3}\left(1\right)&=&U^{(1)}\left(1\right)=-\mu\\
R_{3}^{(1)}\left(1\right)&=&\frac{1}{2}U^{(2)}\left(1\right)=-\frac{1}{2}\esp\left[X^{2}\right]+\frac{1}{2}\mu\\
R_{3}^{(2)}\left(1\right)&=&\frac{2}{3!}U^{(3)}\left(1\right)
=-\frac{1}{3}\esp\left[X^{3}\right]+\esp\left[X^{2}\right]-\frac{2}{3}\mu
\end{eqnarray}

Por lo tanto

\begin{eqnarray}
\esp\left[C_{i}\right]Q\left(z\right)&=&\frac{\left(z-1\right)\left(z-1\right)R_{1}\left(z\right)P\left(z\right)}{\left(z-1\right)R_{2}\left(z\right)\left(z-1\right)R_{3}\left(z\right)}
=\frac{R_{1}\left(z\right)P\left(z\right)}{R_{2}\left(z\right)R_{3}\left(z\right)}\equiv\frac{R_{1}P}{R_{2}R_{3}}
\end{eqnarray}

Entonces

\begin{eqnarray}\label{Eq.Primer.Derivada.Q}
\left[\frac{R_{1}\left(z\right)P\left(z\right)}{R_{2}\left(z\right)R_{3}\left(z\right)}\right]^{'}&=&\frac{PR_{2}R_{3}R_{1}^{'}
+R_{1}R_{2}R_{3}P^{'}-R_{3}R_{1}PR_{2}-R_{2}R_{1}PR_{3}^{'}}{\left(R_{2}R_{3}\right)^{2}}
\end{eqnarray}
Evaluando en $z=1$
\begin{eqnarray*}
&=&\frac{R_{2}(1)R_{3}(1)R_{1}^{(1)}(1)+R_{1}(1)R_{2}(1)R_{3}(1)P^{'}(1)-R_{3}(1)R_{1}(1)R_{2}(1)^{(1)}-R_{2}(1)R_{1}(1)R_{3}^{'}(1)}{\left(R_{2}(1)R_{3}(1)\right)^{2}}\\
&=&\frac{1}{\left(1-\mu\right)^{2}\mu^{2}}\left\{\left(-\frac{1}{2}\esp L^{2}+\frac{1}{2}\esp L\right)\left(1-\mu\right)\left(-\mu\right)+\left(-\esp L\right)\left(1-\mu\right)\left(-\mu\right)\mu\right.\\
&&\left.-\left(-\frac{1}{2}\esp X^{2}+\frac{1}{2}\mu\right)\left(-\mu\right)\left(-\esp L\right)-\left(1-\mu\right)\left(-\esp L\right)\left(-\frac{1}{2}\esp X^{2}+\frac{1}{2}\mu\right)\right\}\\
&=&\frac{1}{\left(1-\mu\right)^{2}\mu^{2}}\left\{\left(-\frac{1}{2}\esp L^{2}+\frac{1}{2}\esp L\right)\left(\mu^{2}-\mu\right)
+\left(\mu^{2}-\mu^{3}\right)\esp L\right.\\
&&\left.-\mu\esp L\left(-\frac{1}{2}\esp X^{2}+\frac{1}{2}\mu\right)
+\left(\esp L-\mu\esp L\right)\left(-\frac{1}{2}\esp X^{2}+\frac{1}{2}\mu\right)\right\}\\
&=&\frac{1}{\left(1-\mu\right)^{2}\mu^{2}}\left\{-\frac{1}{2}\mu^{2}\esp L^{2}
+\frac{1}{2}\mu\esp L^{2}
+\frac{1}{2}\mu^{2}\esp L
-\mu^{3}\esp L
+\mu\esp L\esp X^{2}
-\frac{1}{2}\esp L\esp X^{2}\right\}\\
&=&\frac{1}{\left(1-\mu\right)^{2}\mu^{2}}\left\{
\frac{1}{2}\mu\esp L^{2}\left(1-\mu\right)
+\esp L\left(\frac{1}{2}-\mu\right)\left(\mu^{2}-\esp X^{2}\right)\right\}\\
&=&\frac{1}{2\mu\left(1-\mu\right)}\esp L^{2}-\frac{\frac{1}{2}-\mu}{\left(1-\mu\right)^{2}\mu^{2}}\sigma^{2}\esp L
\end{eqnarray*}

por lo tanto (para Takagi)

\begin{eqnarray*}
Q^{(1)}=\frac{1}{\esp C}\left\{\frac{1}{2\mu\left(1-\mu\right)}\esp L^{2}-\frac{\frac{1}{2}-\mu}{\left(1-\mu\right)^{2}\mu^{2}}\sigma^{2}\esp L\right\}
\end{eqnarray*}
donde 

\begin{eqnarray*}
\esp C = \frac{\esp L}{\mu\left(1-\mu\right)}
\end{eqnarray*}
entonces

\begin{eqnarray*}
Q^{(1)}&=&\frac{1}{2}\frac{\esp L^{2}}{\esp L}-\frac{\frac{1}{2}-\mu}{\left(1-\mu\right)\mu}\sigma^{2}
=\frac{\esp L^{2}}{2\esp L}-\frac{\sigma^{2}}{2}\left\{\frac{2\mu-1}{\left(1-\mu\right)\mu}\right\}\\
&=&\frac{\esp L^{2}}{2\esp L}+\frac{\sigma^{2}}{2}\left\{\frac{1}{1-\mu}+\frac{1}{\mu}\right\}
\end{eqnarray*}

Mientras que para nosotros

\begin{eqnarray*}
Q^{(1)}=\frac{1}{\mu\left(1-\mu\right)}\frac{\esp L^{2}}{2\esp C}
-\sigma^{2}\frac{\esp L}{2\esp C}\cdot\frac{1-2\mu}{\left(1-\mu\right)^{2}\mu^{2}}
\end{eqnarray*}

Retomando la ecuaci\'on (\ref{Eq.Primer.Derivada.Q})

\begin{eqnarray*}
\left[\frac{R_{1}\left(z\right)P\left(z\right)}{R_{2}\left(z\right)R_{3}\left(z\right)}\right]^{'}&=&\frac{PR_{2}R_{3}R_{1}^{'}
+R_{1}R_{2}R_{3}P^{'}-R_{3}R_{1}PR_{2}-R_{2}R_{1}PR_{3}^{'}}{\left(R_{2}R_{3}\right)^{2}}
=\frac{F\left(z\right)}{G\left(z\right)}
\end{eqnarray*}

donde 

\begin{eqnarray*}
F\left(z\right)&=&PR_{2}R_{3}R_{1}^{'}
+R_{1}R_{2}R_{3}P^{'}-R_{3}R_{1}PR_{2}^{'}-R_{2}R_{1}PR_{3}^{'}\\
G\left(z\right)&=&R_{2}^{2}R_{3}^{2}\\
G^{2}\left(z\right)&=&R_{2}^{4}R_{3}^{4}=\left(1-\mu\right)^{4}\mu^{4}
\end{eqnarray*}
y por tanto

\begin{eqnarray*}
G^{'}\left(z\right)&=&2R_{2}R_{3}\left[R_{2}^{'}R_{3}+R_{2}R_{3}^{'}\right]\\
G^{'}\left(1\right)&=&-2\left(1-\mu\right)\mu\left[\left(-\frac{1}{2}\esp\left[X^{2}\right]+\frac{1}{2}\mu\right)\left(-\mu\right)+\left(1-\mu\right)\left(-\frac{1}{2}\esp\left[X^{2}\right]+\frac{1}{2}\mu\right)\right]
\end{eqnarray*}


\begin{eqnarray*}
F^{'}\left(z\right)&=&\left[\left(R_{2}R_{3}\right)R_{1}^{''}
-\left(R_{1}R_{3}\right)R_{2}^{''}
-\left(R_{1}R_{2}\right)R_{3}^{''}
-2\left(R_{2}^{'}R_{3}^{'}\right)R_{1}\right]P
+2\left(R_{2}R_{3}\right)R_{1}^{'}P^{'}
+\left(R_{1}R_{2}R_{3}\right)P^{''}
\end{eqnarray*}

Por lo tanto, encontremos $F^{'}\left(z\right)G\left(z\right)+F\left(z\right)G^{'}\left(z\right)$:

\begin{eqnarray*}
&&F^{'}\left(z\right)G\left(z\right)+F\left(z\right)G^{'}\left(z\right)=
\left\{\left[\left(R_{2}R_{3}\right)R_{1}^{''}
-\left(R_{1}R_{3}\right)R_{2}^{''}
-\left(R_{1}R_{2}\right)R_{3}^{''}
-2\left(R_{2}^{'}R_{3}^{'}\right)R_{1}\right]P\right.\\
&&\left.+2\left(R_{2}R_{3}\right)R_{1}^{'}P^{'}
+\left(R_{1}R_{2}R_{3}\right)P^{''}\right\}R_{2}^{2}R_{3}^{2}
-\left\{\left[PR_{2}R_{3}R_{1}^{'}+R_{1}R_{2}R_{3}P^{'}
-R_{3}R_{1}PR_{2}^{'}\right.\right.\\
&&\left.\left.
-R_{2}R_{1}PR_{3}^{'}\right]\left[2R_{2}R_{3}\left(R_{2}^{'}R_{3}+R_{2}R_{3}^{'}\right)\right]\right\}
\end{eqnarray*}
Evaluando en $z=1$

\begin{eqnarray*}
&=&\left(1+R_{3}\right)^{3}R_{3}^{3}R_{1}^{''}-\left(1+R_{3}\right)^{2}R_{1}R_{3}^{3}R_{3}^{''}
-\left(1+R_{3}\right)^{3}R_{3}^{2}R_{1}R_{3}^{''}-2\left(1+R_{3}\right)^{2}R_{3}^{2}
\left(R_{3}^{'}\right)^{2}\\
&+&2\left(1+R_{3}\right)^{3}R_{3}^{3}R_{1}^{'}P^{'}
+\left(1+R_{3}\right)^{3}R_{3}^{3}R_{1}P^{''}
-2\left(1+R_{3}\right)^{2}R_{3}^{2}\left(1+2R_{3}\right)R_{3}^{'}R_{1}^{'}\\
&-&2\left(1+R_{3}\right)^{2}R_{3}^{2}R_{1}R_{3}^{'}\left(1+2R_{3}\right)P^{'}
+2\left(1+R_{3}\right)\left(1+2R_{3}\right)R_{3}^{3}R_{1}\left(R_{3}^{'}\right)^{2}\\
&+&2\left(1+R_{3}\right)^{2}\left(1+2R_{3}\right)R_{1}R_{3}R_{3}^{'}\\
&=&-\left(1-\mu\right)^{3}\mu^{3}R_{1}^{''}-\left(1-\mu\right)^{2}\mu^{2}R_{1}\left(1-2\mu\right)R_{3}^{''}
-\left(1-\mu\right)^{3}\mu^{3}R_{1}P^{''}\\
&+&2\left(1-\mu\right)\mu^{2}\left[\left(1-2\mu\right)R_{1}-\left(1-\mu\right)\right]\left(R_{3}^{'}\right)^{2}
-2\left(1-\mu\right)^{2}\mu R_{1}\left(1-2\mu\right)R_{3}^{'}\\
&-&2\left(1-\mu\right)^{3}\mu^{4}R_{1}^{'}-2\mu\left(1-\mu\right)\left(1-2\mu\right)R_{3}^{'}R_{1}^{'}
-2\mu^{3}\left(1-\mu\right)^{2}\left(1-2\mu\right)R_{1}R_{1}^{'}
\end{eqnarray*}

por tanto

\begin{eqnarray*}
\left[\frac{F\left(z\right)}{G\left(z\right)}\right]^{'}&=&\frac{1}{\mu^{3}\left(1-\mu\right)^{3}}\left\{
-\left(1-\mu\right)^{2}\mu^{2}R_{1}^{''}-\mu\left(1-\mu\right)\left(1-2\mu\right)R_{1}R_{3}^{''}
-\mu^{2}\left(1-\mu\right)^{2}R_{1}P^{''}\right.\\
&&\left.+2\mu\left[\left(1-2\mu\right)R_{1}-\left(1-\mu\right)\right]\left(R_{3}^{'}\right)^{2}
-2\left(1-\mu\right)\left(1-2\mu\right)R_{1}R_{3}^{'}-2\mu^{3}\left(1-\mu\right)^{2}R_{1}^{'}\right.\\
&&\left.-2\left(1-2\mu\right)R_{3}^{'}R_{1}^{'}-2\mu^{2}\left(1-\mu\right)\left(1-2\mu\right)R_{1}R_{1}^{'}\right\}
\end{eqnarray*}

recordemos que


\begin{eqnarray*}
R_{1}&=&-\esp L\\
R_{3}&=& -\mu\\
R_{1}^{'}&=&-\frac{1}{2}\esp L^{2}+\frac{1}{2}\esp L\\
R_{3}^{'}&=&-\frac{1}{2}\esp X^{2}+\frac{1}{2}\mu\\
R_{1}^{''}&=&-\frac{1}{3}\esp L^{3}+\esp L^{2}-\frac{2}{3}\esp L\\
R_{3}^{''}&=&-\frac{1}{3}\esp X^{3}+\esp X^{2}-\frac{2}{3}\mu\\
R_{1}R_{3}^{'}&=&\frac{1}{2}\esp X^{2}\esp L-\frac{1}{2}\esp X\esp L\\
R_{1}R_{1}^{'}&=&\frac{1}{2}\esp L^{2}\esp L+\frac{1}{2}\esp^{2}L\\
R_{3}^{'}R_{1}^{'}&=&\frac{1}{4}\esp X^{2}\esp L^{2}-\frac{1}{4}\esp X^{2}\esp L-\frac{1}{4}\esp L^{2}\esp X+\frac{1}{4}\esp X\esp L\\
R_{1}R_{3}^{''}&=&\frac{1}{6}\esp X^{3}\esp L^{2}-\frac{1}{6}\esp X^{3}\esp L-\frac{1}{2}\esp L^{2}\esp X^{2}+\frac{1}{2}\esp X^{2}\esp L+\frac{1}{3}\esp X\esp L^{2}-\frac{1}{3}\esp X\esp L\\
R_{1}P^{''}&=&-\esp X^{2}\esp L\\
\left(R_{3}^{'}\right)^{2}&=&\frac{1}{4}\esp^{2}X^{2}-\frac{1}{2}\esp X^{2}\esp X+\frac{1}{4}\esp^{2} X
\end{eqnarray*}




\begin{Def}
Let $L_{i}^{*}$ be the number of users at queue $Q_{i}$ when it is polled, then
\begin{eqnarray}
\begin{array}{cc}
\esp\left[L_{i}^{*}\right]=f_{i}\left(i\right), &
Var\left[L_{i}^{*}\right]=f_{i}\left(i,i\right)+\esp\left[L_{i}^{*}\right]-\esp\left[L_{i}^{*}\right]^{2}.
\end{array}
\end{eqnarray}
\end{Def}

\begin{Def}
The cycle time $C_{i}$ for the queue $Q_{i}$ is the period beginning at the time when it is polled in a cycle and ending at the time when it is polled in the next cycle; it's duration is given by $\tau_{i}\left(m+1\right)-\tau_{i}\left(m\right)$, equivalently $\overline{\tau}_{i}\left(m+1\right)-\overline{\tau}_{i}\left(m\right)$ under steady state assumption.
\end{Def}

\begin{Def}
The intervisit time $I_{i}$ is defined as the period beginning at the time of its service completion in a cycle and ending at the time when it is polled in the next cycle; its duration is given by $\tau_{i}\left(m+1\right)-\overline{\tau}_{i}\left(m\right)$.
\end{Def}

The intervisit time duration $\tau_{i}\left(m+1\right)-\overline{\tau}\left(m\right)$ given the number of users found at queue $Q_{i}$ at time $t=\tau_{i}\left(m+1\right)$ is equal to the number of arrivals during the preceding intervisit time $\left[\overline{\tau}\left(m\right),\tau_{i}\left(m+1\right)\right]$. 

So we have



\begin{eqnarray*}
\esp\left[z_{i}^{L_{i}\left(\tau_{i}\left(m+1\right)\right)}\right]=\esp\left[\left\{P_{i}\left(z_{i}\right)\right\}^{\tau_{i}\left(m+1\right)-\overline{\tau}\left(m\right)}\right]
\end{eqnarray*}

if $I_{i}\left(z\right)=\esp\left[z^{\tau_{i}\left(m+1\right)-\overline{\tau}\left(m\right)}\right]$
we have $F_{i}\left(z\right)=I_{i}\left[P_{i}\left(z\right)\right]$
for $i=1,2$. Futhermore can be proved that

\begin{eqnarray}
\begin{array}{ll}
\esp\left[L_{i}\right]=\mu_{i}\esp\left[I_{i}\right], &
\esp\left[C_{i}\right]=\frac{f_{i}\left(i\right)}{\mu_{i}\left(1-\mu_{i}\right)},\\
\esp\left[S_{i}\right]=\mu_{i}\esp\left[C_{i}\right],&
\esp\left[I_{i}\right]=\left(1-\mu_{i}\right)\esp\left[C_{i}\right],\\
Var\left[L_{i}\right]= \mu_{i}^{2}Var\left[I_{i}\right]+\sigma^{2}\esp\left[I_{i}\right],& 
Var\left[C_{i}\right]=\frac{Var\left[L_{i}^{*}\right]}{\mu_{i}^{2}\left(1-\mu_{i}\right)^{2}},\\
Var\left[S_{i}\right]= \frac{Var\left[L_{i}^{*}\right]}{\left(1-\mu_{i}\right)^{2}}+\frac{\sigma^{2}\esp\left[L_{i}^{*}\right]}{\left(1-\mu_{i}\right)^{3}},&
Var\left[I_{i}\right]= \frac{Var\left[L_{i}^{*}\right]}{\mu_{i}^{2}}-\frac{\sigma_{i}^{2}}{\mu_{i}^{2}}f_{i}\left(i\right).
\end{array}
\end{eqnarray}

Let consider the points when the process $\left[L_{1}\left(1\right),L_{2}\left(1\right),L_{3}\left(1\right),L_{4}\left(1\right)
\right]$ becomes zero at the same time, this points, $T_{1},T_{2},\ldots$ will be denoted as regeneration points, then we have that

\begin{Def}
the interval between two such succesive regeneration points will be called regenerative cycle.
\end{Def}

\begin{Def}
Para $T_{i}$ se define, $M_{i}$, el n\'umero de ciclos de visita a la cola $Q_{l}$, durante el ciclo regenerativo, es decir, $M_{i}$ es un proceso de renovaci\'on.
\end{Def}

\begin{Def}
Para cada uno de los $M_{i}$'s, se definen a su vez la duraci\'on de cada uno de estos ciclos de visita en el ciclo regenerativo, $C_{i}^{(m)}$, para $m=1,2,\ldots,M_{i}$, que a su vez, tambi\'en es n proceso de renovaci\'on.
\end{Def}



Sea la funci\'on generadora de momentos para $L_{i}$, el n\'umero de usuarios en la cola $Q_{i}\left(z\right)$ en cualquier momento, est\'a dada por el tiempo promedio de $z^{L_{i}\left(t\right)}$ sobre el ciclo regenerativo definido anteriormente. Entonces 

\begin{equation}\label{Eq.Longitud.Tiempo.t}
Q_{i}\left(z\right)=\frac{1}{\esp\left[C_{i}\right]}\cdot\frac{1-F_{i}\left(z\right)}{P_{i}\left(z\right)-z}\cdot\frac{\left(1-z\right)P_{i}\left(z\right)}{1-P_{i}\left(z\right)}.
\end{equation}

Es decir, es posible determinar las longitudes de las colas a cualquier tiempo $t$. Entonces, determinando el primer momento es posible ver que


$M_{i}$ is an stopping time for the regenerative process with $\esp\left[M_{i}\right]<\infty$, from Wald's lemma follows that:


\begin{eqnarray*}
\esp\left[\sum_{m=1}^{M_{i}}\sum_{t=\tau_{i}\left(m\right)}^{\tau_{i}\left(m+1\right)-1}z^{L_{i}\left(t\right)}\right]&=&\esp\left[M_{i}\right]\esp\left[\sum_{t=\tau_{i}\left(m\right)}^{\tau_{i}\left(m+1\right)-1}z^{L_{i}\left(t\right)}\right]\\
\esp\left[\sum_{m=1}^{M_{i}}\tau_{i}\left(m+1\right)-\tau_{i}\left(m\right)\right]&=&\esp\left[M_{i}\right]\esp\left[\tau_{i}\left(m+1\right)-\tau_{i}\left(m\right)\right]
\end{eqnarray*}
therefore 

\begin{eqnarray*}
Q_{i}\left(z\right)&=&\frac{\esp\left[\sum_{t=\tau_{i}\left(m\right)}^{\tau_{i}\left(m+1\right)-1}z^{L_{i}\left(t\right)}\right]}{\esp\left[\tau_{i}\left(m+1\right)-\tau_{i}\left(m\right)\right]}
\end{eqnarray*}

Doing the following substitutions en (\ref{Corolario2}): $n\rightarrow t-\tau_{i}\left(m\right)$, $T \rightarrow \overline{\tau}_{i}\left(m\right)-\tau_{i}\left(m\right)$, $L_{n}\rightarrow L_{i}\left(t\right)$ and $F\left(z\right)=\esp\left[z^{L_{0}}\right]\rightarrow F_{i}\left(z\right)=\esp\left[z^{L_{i}\tau_{i}\left(m\right)}\right]$, 
we obtain

\begin{eqnarray}\label{Eq.Arribos.Primera}
\esp\left[\sum_{n=0}^{T-1}z^{L_{n}}\right]=
\esp\left[\sum_{t=\tau_{i}\left(m\right)}^{\overline{\tau}_{i}\left(m\right)-1}z^{L_{i}\left(t\right)}\right]
=z\frac{F_{i}\left(z\right)-1}{z-P_{i}\left(z\right)}
\end{eqnarray}



Por otra parte durante el tiempo de intervisita para la cola $i$, $L_{i}\left(t\right)$ solamente se incrementa de manera que el incremento por intervalo de tiempo est\'a dado por la funci\'on generadora de probabilidades de $P_{i}\left(z\right)$, por tanto la suma sobre el tiempo de intervisita puede evaluarse como:

\begin{eqnarray*}
\esp\left[\sum_{t=\tau_{i}\left(m\right)}^{\tau_{i}\left(m+1\right)-1}z^{L_{i}\left(t\right)}\right]&=&\esp\left[\sum_{t=\tau_{i}\left(m\right)}^{\tau_{i}\left(m+1\right)-1}\left\{P_{i}\left(z\right)\right\}^{t-\overline{\tau}_{i}\left(m\right)}\right]=\frac{1-\esp\left[\left\{P_{i}\left(z\right)\right\}^{\tau_{i}\left(m+1\right)-\overline{\tau}_{i}\left(m\right)}\right]}{1-P_{i}\left(z\right)}\\
&=&\frac{1-I_{i}\left[P_{i}\left(z\right)\right]}{1-P_{i}\left(z\right)}
\end{eqnarray*}
por tanto

\begin{eqnarray*}
\esp\left[\sum_{t=\tau_{i}\left(m\right)}^{\tau_{i}\left(m+1\right)-1}z^{L_{i}\left(t\right)}\right]&=&
\frac{1-F_{i}\left(z\right)}{1-P_{i}\left(z\right)}
\end{eqnarray*}

Por lo tanto

\begin{eqnarray*}
Q_{i}\left(z\right)&=&\frac{\esp\left[\sum_{t=\tau_{i}\left(m\right)}^{\tau_{i}\left(m+1\right)-1}z^{L_{i}\left(t\right)}\right]}{\esp\left[\tau_{i}\left(m+1\right)-\tau_{i}\left(m\right)\right]}
=\frac{1}{\esp\left[\tau_{i}\left(m+1\right)-\tau_{i}\left(m\right)\right]}
\esp\left[\sum_{t=\tau_{i}\left(m\right)}^{\tau_{i}\left(m+1\right)-1}z^{L_{i}\left(t\right)}\right]\\
&=&\frac{1}{\esp\left[\tau_{i}\left(m+1\right)-\tau_{i}\left(m\right)\right]}
\esp\left[\sum_{t=\tau_{i}\left(m\right)}^{\overline{\tau}_{i}\left(m\right)-1}z^{L_{i}\left(t\right)}
+\sum_{t=\overline{\tau}_{i}\left(m\right)}^{\tau_{i}\left(m+1\right)-1}z^{L_{i}\left(t\right)}\right]\\
&=&\frac{1}{\esp\left[\tau_{i}\left(m+1\right)-\tau_{i}\left(m\right)\right]}\left\{
\esp\left[\sum_{t=\tau_{i}\left(m\right)}^{\overline{\tau}_{i}\left(m\right)-1}z^{L_{i}\left(t\right)}\right]
+\esp\left[\sum_{t=\overline{\tau}_{i}\left(m\right)}^{\tau_{i}\left(m+1\right)-1}z^{L_{i}\left(t\right)}\right]\right\}\\
&=&\frac{1}{\esp\left[\tau_{i}\left(m+1\right)-\tau_{i}\left(m\right)\right]}\left\{
z\frac{F_{i}\left(z\right)-1}{z-P_{i}\left(z\right)}+\frac{1-F_{i}\left(z\right)}{1-P_{i}\left(z\right)}
\right\}\\
&=&\frac{1}{\esp\left[C_{i}\right]}\cdot\frac{1-F_{i}\left(z\right)}{P_{i}\left(z\right)-z}\cdot\frac{\left(1-z\right)P_{i}\left(z\right)}{1-P_{i}\left(z\right)}
\end{eqnarray*}

es decir

\begin{eqnarray}
\begin{array}{ll}
S^{'}\left(z\right)=-\sum_{k=1}^{+\infty}ka_{k}z^{k-1},& S^{(1)}\left(1\right)=-\sum_{k=1}^{+\infty}ka_{k}=-\esp\left[L\left(t\right)\right],\\
S^{''}\left(z\right)=-\sum_{k=2}^{+\infty}k(k-1)a_{k}z^{k-2},& S^{(2)}\left(1\right)=-\sum_{k=2}^{+\infty}k(k-1)a_{k}=\esp\left[L\left(L-1\right)\right],\\
S^{'''}\left(z\right)=-\sum_{k=3}^{+\infty}k(k-1)(k-2)a_{k}z^{k-3},&
S^{(3)}\left(1\right)=-\sum_{k=3}^{+\infty}k(k-1)(k-2)a_{k}\\
&=-\esp\left[L\left(L-1\right)\left(L-2\right)\right]\\
&=-\esp\left[L^{3}\right]+3-\esp\left[L^{2}\right]-2-\esp\left[L\right];
\end{array}
\end{eqnarray}






%\section{Existencia de Tiempos de Regeneraci\'on}
%___________________________________________________________
%


\section{Introduction}
%______________________________________________________________________

A cyclic polling system consists of multiple queues that are served by a single server in cyclic order. Users arrive at each queue according to independent processes, which also are independent of the service times. The server attends each queue according to a service policy previously established. The most commonly service policies studied are the exhaustive, gated and the k-limited. The exhaustive policy consists in attending all users until the queue is emptied. When the server finishes, it moves to the next queue incurring in a switchover time that is an independent and identically distributed random variable. An exhaustive analysis have been made in this subject. For an overview of the literature on polling systems, their applications and standard results we refer to surveys such as: \cite{Boxma, Kleinrock, LevySidi, Semenova, TakagiI, Takagi}. 

Bos and Boon \cite{BosBoon} published a report where they studied a Network of Polling Systems applied to a traffic problem, there they analyzed a network of intersections and followed a path in it. Their objective was to predict if the costumers can pass through the network in a finite time or not. The buffer occupancy method was used in this analysis and simulation techniques were also used to verify the results. It is important to remark that the heavy traffic case was studied in this report, as well as the cyclic case was not considered.

In this work, we study a Network of Cyclic Polling Systems (NCPS) that consists of two cyclic polling systems, each of them conformed by two queues attended by a single server. We apply the buffer occupancy method described by Kleinrock and Takagi \cite{TakagiI}. This method is based on the use of the Probability Generating Function (PGF) of the joint distribution function of the queues lengths at the moment the server starts a visit period in each of the queues that conform the system.

We present a theorem that guarantees the stability for the NCPS under specific conditions, also we obtain explicit expressions for the queue lengths at the moment the server arrives. With this results we obtain the queue lengths of the NCPS at any time for the servers.

We believe these results can be generalized for the continuous case and from the point of view of applications, the results are useful because they allow us to obtain analytical expressions for the performance measures, and also give us the keys to determine waiting times and queue lengths for any time during the operation of the network. Initially our main goal was studying the system of public transportation, which can be seen as a network consisting of several cyclic polling systems.

%_________________________________________________________________________
%
\section{Construcci\'on del Modelo e Hip\'otesis}
%_________________________________________________________________________
%
\begin{figure}[H]\caption{Network of Cyclic Polling System with simple transfer}
\centering
\includegraphics[width=7cm]{Grafica3.jpg}
\end{figure}\label{FigureRSVC}

Consider a Network consisting of two cyclic polling systems with two queues each, $Q_{1}, Q_{2}$ for the first system and $\hat{Q}_{1},\hat{Q}_{2}$ for the second one, each of them with infinite-sized buffer. In each system a single server visits the queues in cyclic order, where it applies the exhaustive policy, i.e., when the server polls a queue, it serves all the customers present until the queue becomes empty. This case is illustrated in \texttt{Figure 1}. 

The second system's users at queue 2, can moves to the first system after being attended, also we assume that the network is open; that is, all customers eventually leave the network. As usually in polling systems theory we assume the arrivals in each queue are Poisson processes from with independent identical distributed (i.i.d.) inter arrival exponential times. The service times are exponential independent and identically distributed random variables. Finally upon completion of a visit at any queue, the servers incurs in a random switchover time according to an arbitrary distribution. We define a cycle to be the time interval between two consecutive polling instants, the time period in a cycle during which the server is attending a queue is called a service period. We are considering the case where the server visit the queues in cyclic order.

Time is slotted with slot size equal to the service time of a fixed costumer, we call the time interval $\left[t,t+1\right]$ the $t$-th slot. The arrival processes are denoted by $X_{1}\left(t\right),X_{2}\left(t\right)$ for the first system and $\hat{X}_{1}\left(t\right)$, $\hat{X}_{2}\left(t\right)$ for the second, the arrival rate at $Q_{i}$ and $\hat{Q}_{i}$ is denoted by $\mu_{i}$ and $\hat{\mu}_{i}$ respectively, with the condition $\mu_{i}<1$ and $\hat{\mu}_{i}<1$. The second system's users pass to the first one according to a process $Y_{2}$, with arrival rate $\tilde{\mu}_{2}$. 

Let's denote by $\tau_{i}$ the polling instant at queue $Q_{1}$ and by $\overline{\tau}_{i}$ the instant when the servers leaves to queue and starts a switchover time. Like the rest of the random variables the swithcover period is an i.i.d random variable $R_{i}$ with general distribution. 


To determine the length of the queues, i.e., the number of users in the queue at the moment the server arrives we define the process $L_{i}$ and $\hat{L}_{i}$ for the first and second system, respectively, in the sequel we use the buffer occupancy method to obtain the generating function, first and second moments of queue size distributions at polling instants. At each of the queues in the network the number of users is the number of users at the time the server arrives plus the numbers of users from the other system. 


In order to obtain the joint probability generating function (PGF) for the number or users residing in queue $i$ when the queue is polled in the NCPS, we define for each of the arrival processes $X_{1},X_{2},\hat{X}_{1},\hat{X}_{2},Y_{2}$, and $\tilde{X}_{2}$ with $\tilde{X}_{2}=X_{2}+Y_{2}$, their PGF

\begin{eqnarray*}
\begin{array}{cc}
P_{i}\left(z_{i}\right)=\esp\left[z_{i}^{X_{i}\left(t\right)}\right],&
\hat{P}_{i}\left(w_{i}\right)=\esp\left[w_{i}^{\hat{X}_{i}\left(t\right)}\right]
\end{array}
\end{eqnarray*}
for $i=1,2$, and
\begin{eqnarray*}
\begin{array}{cc}
\check{P}_{2}\left(z_{2}\right)=\esp\left[z_{2}^{Y_{2}\left(t\right)}\right],& \tilde{P}_{2}\left(z_{2}\right)=\esp\left[z_{2}^{\tilde{X}_{2}\left(t\right)}
\right],
\end{array}
\end{eqnarray*}

for $i=1,2$, and
\begin{eqnarray*} 
\begin{array}{cc}
\check{\mu}_{2}=\esp\left[Y_{2}\left(t\right)\right]=\check{P}_{2}^{(1)}
\left(1\right),&
\tilde{\mu}_{2}=\esp\left[\tilde{X}_{2}\left(t\right)\right]
=\tilde{P}_{2}^{(1)}\left(1\right).
\end{array}
\end{eqnarray*} The PGF For the service time is defined by:

\begin{eqnarray*}
\begin{array}{cc}
S_{i}\left(z_{i}\right)=\esp\left[z_{i}^{\overline{\tau}_{i}-\tau_{i}}
\right], &
\hat{S}_{i}\left(w_{i}\right)=\esp\left[w_{i}^{\overline{\zeta}_{i}-\zeta_{i}}\right]
\end{array}
\end{eqnarray*} with first moment 
\begin{eqnarray*}
\begin{array}{cc}
s_{i}=\esp\left[\overline{\tau}_{i}-\tau_{i}\right],&\hat{s}_{i}=\esp\left[\overline{\zeta}_{i}-\zeta_{i}\right]
\end{array}
\end{eqnarray*} for $i=1,2$. In a similar manner the PGF for the switchover time of the server from the moment it ends to attend a queue, to the time of arrival to the next queue is given by 
\begin{eqnarray*}
\begin{array}{cc}
R_{i}\left(z_{i}\right)=\esp\left[z_{1}^{\tau_{i+1}-\overline{\tau}_{i}}\right],&
\hat{R}_{i}\left(w_{i}\right)=\esp\left[w_{i}^{\zeta_{i+1}-\overline{\zeta}_{i}}\right]
\end{array}
\end{eqnarray*} with first moment 

\begin{eqnarray*}
\begin{array}{cc}
r_{i}=\esp\left[\tau_{i+1}-\overline{\tau}_{i}\right],&
\hat{r}_{i}=\esp\left[\zeta_{i+1}-\overline{\zeta}_{i}\right]
\end{array}
\end{eqnarray*} for $i=1,2$. The number of users in the queue at times $\overline{\tau}_{1},\overline{\tau}_{2}, \overline{\zeta}_{1},\overline{\zeta}_{2}$, it's zero, i.e.,
 $L_{i}\left(\overline{\tau_{i}}\right)=0,$ and $\hat{L}_{i}\left(\overline{\zeta_{i}}\right)=0$ for $i=1,2$. Then the number of users in the queue of the second system at the moment the server ends attending in the queue is given by the number of users present at the moment it arrives plus the number of arrivals during the service time, i.e.,
$$\hat{L}_{i}\left(\overline{\tau}_{j}\right)=\hat{L}_{i}\left(\tau_{j}\right)+\hat{X}_{i}\left(\overline{\tau}_{j}-\tau_{j}\right),$$
for $i,j=1,2$, meanwhile for the first system : $$L_{1}\left(\overline{\tau}_{j}\right)=L_{1}\left(\tau_{j}\right)+X_{1}\left(\overline{\tau}_{j}-\tau_{j}\right).$$ Specifically for the second queue of the first system we need to consider the users of transfer becoming from the second queue in the second system while the server its in the other queue attending, it means that this users have been already attended by the server before they can go to the first queue:

\begin{equation}\label{Eq.UsuariosTotalesZ2}
L_{2}\left(\overline{\tau}_{1}\right)=L_{2}\left(\tau_{1}\right)+X_{2}\left(\overline{\tau}_{1}-\tau_{1}\right)+Y_{2}\left(\overline{\tau}_{1}-\tau_{1}\right).
\end{equation}

As is know, the gambler's ruin problem can be used to model the server's busy period in a cyclic polling system, so let $\tilde{L}_{0}\geq0$ be the number of users present at the moment the server arrive to start attending, also let $T$ be the time the server need to attend the users in the queue starting with $\tilde{L}_{0}$ users. Suppose the gambler has two independent and simultaneous moves, such events are independent and identical to each other for each realization. The gain on the $n$-th game is $\tilde{\mathsf{X}}_{n}=\mathsf{X}_{n}+\mathsf{Y}_{n}$ units from which is substracted a playing fee of 1 unit for each move. His PGF is given by $F\left(z\right)=\esp\left[z^{\tilde{L}_{0}}\right]$, futhermore
%$\tilde{\mathrm{X}}$, $\tilde{\mathit{X}}$, $\tilde{\mathcal{X}}$, $\tilde{\mathfrak{X}}$,$\tilde{\mathbb{X}}$,$\tilde{\mathtt{X}}$,$\tilde{\mathsf{X}}$,

$$\tilde{P}\left(z\right)=\esp\left[z^{\tilde{\mathsf{X}}_{n}}\right]=\esp\left[z^{\mathsf{X}_{n}+\mathsf{X}_{n}}\right]=\esp\left[z^{\mathsf{X}_{n}}z^{\mathsf{X}_{n}}\right]=\esp\left[z^{\mathsf{X}_{n}}\right]\esp\left[z^{\mathsf{X}_{n}}\right]=P\left(z\right)\check{P}\left(z\right),$$ with $\tilde{\mu}=\esp\left[\tilde{\mathsf{X}}_{n}\right]=\tilde{P}\left[z\right]<1$. If  $\tilde{L}_{n}$ denotes the capital remaining after the $n$-th game, then $\tilde{L}_{n}=\tilde{L}_{0}+\tilde{\mathsf{X}}_{1}+\tilde{\mathsf{X}}_{2}+\cdots+\tilde{\mathsf{X}}_{n}-2n$. The result that relates the gambler's ruin problem with the busy period of the server it's a generalization of the result given in Takagi \cite{Takagi} chapter 3.

\begin{Prop}
Let's $G_{n}\left(z\right)$ and $G\left(z,w\right)$ defined as in 
(\ref{Eq.3.16.b.2S}), then $G_{n}\left(z\right)=\frac{1}{z}\left[G_{n-1}\left(z\right)-G_{n-1}\left(0\right)\right]\tilde{P}\left(z\right)$. Futhermore $G\left(z,w\right)=\frac{zF\left(z\right)-wP\left(z\right)G\left(0,w\right)}{z-wR\left(z\right)}$, with a unique pole in the unit circle, also the pole is of the form $z=\theta\left(w\right)$ and satisfies 
\begin{multicols}{3}
\begin{itemize}
\item[i)]$\tilde{\theta}\left(1\right)=1$,

\item[ii)] $\tilde{\theta}^{(1)}\left(1\right)=\frac{1}{1-\tilde{\mu}}$,

\item[iii)]
$\tilde{\theta}^{(2)}\left(1\right)=\frac{\tilde{\mu}}{\left(1-\tilde{\mu}\right)^{2}}+\frac{\tilde{\sigma}}{\left(1-\tilde{\mu}\right)^{3}}$.
\end{itemize}
\end{multicols}
\end{Prop}
%_________________________________________________________________________
%
\subsection{Description of the model: Probability Generating Function}
%_________________________________________________________________________
%

In order to model the network of cyclic polling system it's necessary to consider the users arrivals to each queue in one of the system, but on times the other system's server arrival, $\zeta_{i}$. In the case of the first system and the server arrives to a queue in the second one: $$F_{i,j}\left(z_{i};\zeta_{j}\right)=\esp\left[z_{i}^{L_{i}\left(\zeta_{j}\right)}\right]=
\sum_{k=0}^{\infty}\prob\left[L_{i}\left(\zeta_{j}\right)
=k\right]z_{i}^{k},$$ for $i,j=1,2$. Now consider the case of the queues in the second system and the server arrive to a queue in the first system $$\hat{F}_{i,j}\left(w_{i};\tau_{j}\right)=\esp\left[w_{i}^{\hat{L}_{i}\left(\tau_{j}\right)}\right] =\sum_{k=0}^{\infty}\prob\left[\hat{L}_{i}\left(\tau_{j}\right)
=k\right]w_{i}^{k},$$ for $i,j=1,2$. With the developed we can define the joint PGF for the second system:
$$\esp\left[w_{1}^{\hat{L}_{1}\left(\tau_{j}\right)}w_{2}^{\hat{L}_{2}\left(\tau_{j}\right)}\right]
=\esp\left[w_{1}^{\hat{L}_{1}\left(\tau_{j}\right)}\right]
\esp\left[w_{2}^{\hat{L}_{2}\left(\tau_{j}\right)}\right]=\hat{F}_{1,j}\left(w_{1};\tau_{j}\right)\hat{F}_{2,j}\left(w_{2};\tau_{j}\right)\equiv\hat{\mathbf{F}}_{j}\left(w_{1},w_{2};\tau_{j}\right).$$
%\end{eqnarray*}

In a similar manner we define the joint PGF for the first system, and the second system's server:
%\begin{eqnarray*}
$$\esp\left[z_{1}^{L_{1}\left(\zeta_{j}\right)}z_{2}^{L_{2}\left(\zeta_{j}\right)}\right]
=\esp\left[z_{1}^{L_{1}\left(\zeta_{j}\right)}\right]
\esp\left[z_{2}^{L_{2}\left(\zeta_{j}\right)}\right]=F_{1,j}\left(z_{1};\zeta_{j}\right)F_{2,j}\left(z_{2};\zeta_{j}\right)\equiv\mathbf{F}_{j}\left(z_{1},z_{2};\zeta_{j}\right).$$
%\end{eqnarray*}

Now we proceed to determine the joint PGF for the times that the server visit each queue in their corresponding system, i.e., $t=\left\{\tau_{1},\tau_{2},\zeta_{1},\zeta_{2}\right\}$:

\begin{eqnarray}\label{Eq.Conjuntas}
\begin{array}{l}
\mathbf{F}_{j}\left(z_{1},z_{2},w_{1},w_{2}\right)=\esp\left[\prod_{i=1}^{2}z_{i}^{L_{i}\left(\tau_{j}
\right)}\prod_{i=1}^{2}w_{i}^{\hat{L}_{i}\left(\tau_{j}\right)}\right],\\
\hat{\mathbf{F}}_{j}\left(z_{1},z_{2},w_{1},w_{2}\right)=\esp\left[\prod_{i=1}^{2}z_{i}^{L_{i}
\left(\zeta_{j}\right)}\prod_{i=1}^{2}w_{i}^{\hat{L}_{i}\left(\zeta_{j}\right)}\right],
\end{array}
\end{eqnarray} for $j=1,2$. Then with the purpose of find the number of users present in the netwotk when the server ends attending one of the queues in any of the systems we have that

\begin{eqnarray*}
&&\esp\left[z_{1}^{L_{1}\left(\overline{\tau}_{1}\right)}z_{2}^{L_{2}\left(\overline{\tau}_{1}\right)}w_{1}^{\hat{L}_{1}\left(\overline{\tau}_{1}\right)}w_{2}^{\hat{L}_{2}\left(\overline{\tau}_{1}\right)}\right]
=\esp\left[z_{2}^{L_{2}\left(\overline{\tau}_{1}\right)}w_{1}^{\hat{L}_{1}\left(\overline{\tau}_{1}
\right)}w_{2}^{\hat{L}_{2}\left(\overline{\tau}_{1}\right)}\right]\\
&=&\esp\left[z_{2}^{L_{2}\left(\tau_{1}\right)+X_{2}\left(\overline{\tau}_{1}-\tau_{1}\right)+Y_{2}\left(\overline{\tau}_{1}-\tau_{1}\right)}w_{1}^{\hat{L}_{1}\left(\tau_{1}\right)+\hat{X}_{1}\left(\overline{\tau}_{1}-\tau_{1}\right)}w_{2}^{\hat{L}_{2}\left(\tau_{1}\right)+\hat{X}_{2}\left(\overline{\tau}_{1}-\tau_{1}\right)}\right]
\end{eqnarray*}

using the equation (\ref{Eq.UsuariosTotalesZ2}) we have


\begin{eqnarray*}
&=&\esp\left[z_{2}^{L_{2}\left(\tau_{1}\right)}z_{2}^{X_{2}\left(\overline{\tau}_{1}-\tau_{1}\right)}z_{2}^{Y_{2}\left(\overline{\tau}_{1}-\tau_{1}\right)}w_{1}^{\hat{L}_{1}\left(\tau_{1}\right)}w_{1}^{\hat{X}_{1}\left(\overline{\tau}_{1}-\tau_{1}\right)}w_{2}^{\hat{L}_{2}\left(\tau_{1}\right)}w_{2}^{\hat{X}_{2}\left(\overline{\tau}_{1}-\tau_{1}\right)}\right]\\
&=&\esp\left[z_{2}^{L_{2}\left(\tau_{1}\right)}\left\{w_{1}^{\hat{L}_{1}\left(\tau_{1}\right)}w_{2}^{\hat{L}_{2}\left(\tau_{1}\right)}\right\}\left\{z_{2}^{X_{2}\left(\overline{\tau}_{1}-\tau_{1}\right)}
z_{2}^{Y_{2}\left(\overline{\tau}_{1}-\tau_{1}\right)}w_{1}^{\hat{X}_{1}\left(\overline{\tau}_{1}-\tau_{1}\right)}w_{2}^{\hat{X}_{2}\left(\overline{\tau}_{1}-\tau_{1}\right)}\right\}\right]
\end{eqnarray*}

applying the fact that the arrivals processes in the queues in each systems are independent:

$$=\esp\left[z_{2}^{L_{2}\left(\tau_{1}\right)}\left\{z_{2}^{X_{2}\left(\overline{\tau}_{1}-\tau_{1}\right)}z_{2}^{Y_{2}\left(\overline{\tau}_{1}-
\tau_{1}\right)}w_{1}^{\hat{X}_{1}\left(\overline{\tau}_{1}-\tau_{1}\right)}w_{2}^{\hat{X}_{2}\left(\overline{\tau}_{1}-\tau_{1}\right)}\right\}\right]
\esp\left[w_{1}^{\hat{L}_{1}\left(\tau_{1}\right)}w_{2}^{\hat{L}_{2}\left(\tau_{1}\right)}\right]$$ given that the arrival processes in the queues are independent, it's possible to separate the expectation for the arrival processes in $Q_{1}$ and $Q_{2}$ at time $\tau_{1}$, which is the time the server visits $Q_{1}$. Considering
$\tilde{X}_{2}\left(z_{2}\right)=X_{2}\left(z_{2}\right)+Y_{2}\left(z_{2}\right)$ we have


\begin{eqnarray*}
\begin{array}{l}
=\esp\left[z_{2}^{L_{2}\left(\tau_{1}\right)}\left\{z_{2}^{\tilde{X}_{2}\left(\overline{\tau}_{1}-\tau_{1}\right)}w_{1}^{\hat{X}_{1}\left(\overline{\tau}_{1}
-\tau_{1}\right)}
w_{2}^{\hat{X}_{2}\left(\overline{\tau}_{1}-\tau_{1}\right)}\right\}\right]\esp\left[w_{1}^{\hat{L}_{1}\left(\tau_{1}\right)}
w_{2}^{\hat{L}_{2}\left(\tau_{1}\right)}\right]\\
=\esp\left[z_{2}^{L_{2}\left(\tau_{1}\right)}\left\{\tilde{P}_{2}\left(z_{2}\right)
^{\overline{\tau}_{1}-\tau_{1}}\hat{P}_{1}\left(w_{1}\right)^{\overline{\tau}_{1}-
\tau_{1}}\hat{P}_{2}\left(w_{2}\right)^{\overline{\tau}_{1}-\tau_{1}}\right\}\right]
\esp\left[w_{1}^{\hat{L}_{1}\left(\tau_{1}\right)}w_{2}^{\hat{L}_{2}\left(\tau_{1}\right)}\right]\\
=\esp\left[z_{2}^{L_{2}\left(\tau_{1}\right)}\left\{\tilde{P}_{2}\left(z_{2}\right)\hat{P}_{1}\left(w_{1}\right)\hat{P}_{2}\left(w_{2}\right)\right\}^{\overline{\tau}_{1}-\tau_{1}}\right]\esp\left[w_{1}^{\hat{L}_{1}\left(\tau_{1}\right)}w_{2}^{\hat{L}_{2}\left(\tau_{1}\right)}\right]\\
=\esp\left[z_{2}^{L_{2}\left(\tau_{1}\right)}\theta_{1}\left(\tilde{P}_{2}\left(z_{2}\right)\hat{P}_{1}\left(w_{1}\right)\hat{P}_{2}\left(w_{2}\right)\right)
^{L_{1}\left(\tau_{1}\right)}\right]\esp\left[w_{1}^{\hat{L}_{1}\left(\tau_{1}\right)}w_{2}^{\hat{L}_{2}\left(\tau_{1}\right)}\right]\\
=F_{1}\left(\theta_{1}\left(\tilde{P}_{2}\left(z_{2}\right)\hat{P}_{1}\left(w_{1}\right)\hat{P}_{2}\left(w_{2}\right)\right),z_{2}\right)\cdot
\hat{F}_{1}\left(w_{1},w_{2};\tau_{1}\right)\\
\equiv \mathbf{F}_{1}\left(\theta_{1}\left(\tilde{P}_{2}\left(z_{2}\right)\hat{P}_{1}\left(w_{1}\right)\hat{P}_{2}\left(w_{2}\right)\right),z_{2},w_{1},w_{2}\right).
\end{array}
\end{eqnarray*}

The last equalities  are true because the number of arrivals to $\hat{Q}_{2}$ 
during the time interval $\left[\tau_{1},\overline{\tau}_{1}\right]$ still haven't been attended by the server in the system 2, then the users can't pass to the first system through the queue $Q_{2}$. Therefore the number of users switching from $\hat{Q}_{2}$ to $Q_{2}$ during the time interval $\left[\tau_{1},\overline{\tau}_{1}\right]$ depends on the policy of transfer between the two systems, according to the last section
%{\small{
\begin{eqnarray*}\label{Eq.Fs}
\begin{array}{l}
\esp\left[z_{1}^{L_{1}\left(\overline{\tau}_{1}\right)}z_{2}^{L_{2}\left(\overline{\tau}_{1}
\right)}w_{1}^{\hat{L}_{1}\left(\overline{\tau}_{1}\right)}w_{2}^{\hat{L}_{2}\left(
\overline{\tau}_{1}\right)}\right]
=\mathbf{F}_{1}\left(\theta_{1}\left(\tilde{P}_{2}\left(z_{2}\right)
\hat{P}_{1}\left(w_{1}\right)\hat{P}_{2}\left(w_{2}\right)\right),z_{2},w_{1},w_{2}\right)\\
\equiv F_{1}\left(\theta_{1}\left(\tilde{P}_{2}\left(z_{2}\right)\hat{P}_{1}\left(w_{1}\right)\hat{P}_{2}\left(w_{2}\right)\right),z_{2}\right)\hat{F}_{1}\left(w_{1},w_{2};\tau_{1}\right).
\end{array}
\end{eqnarray*}%}}

Using similar reasoning for the rest of the server's arrival times we have that

\begin{eqnarray*}
\esp\left[z_{1}^{L_{1}\left(\overline{\tau}_{2}\right)}z_{2}^{L_{2}\left(\overline{\tau}_{2}\right)}w_{1}^{\hat{L}_{1}\left(\overline{\tau}_{2}\right)}w_{2}^{\hat{L}_{2}\left(\overline{\tau}_{2}\right)}\right]&=&F_{2}\left(z_{1},\tilde{\theta}_{2}\left(P_{1}\left(z_{1}\right)\hat{P}_{1}\left(w_{1}\right)\hat{P}_{2}\left(w_{2}\right)\right)\right)
\hat{F}_{2}\left(w_{1},w_{2};\tau_{2}\right)\\
&\equiv& \mathbf{F}_{2}\left(z_{1},\tilde{\theta}_{2}\left(P_{1}\left(z_{1}\right)\hat{P}_{1}\left(w_{1}\right)\hat{P}_{2}\left(w_{2}\right)\right),w_{1},w_{2}\right),\\
\esp\left[z_{1}^{L_{1}\left(\overline{\zeta}_{1}\right)}z_{2}^{L_{2}\left(\overline{\zeta}_{1}
\right)}w_{1}^{\hat{L}_{1}\left(\overline{\zeta}_{1}\right)}w_{2}^{\hat{L}_{2}\left(\overline{\zeta}_{1}\right)}\right]
&=&F_{1}\left(z_{1},z_{2};\zeta_{1}\right)\hat{F}_{1}\left(\hat{\theta}_{1}\left(P_{1}\left(z_{1}\right)\tilde{P}_{2}\left(z_{2}\right)\hat{P}_{2}\left(w_{2}\right)\right),w_{2}\right)\\
&\equiv&\hat{\mathbf{F}}_{1}\left(z_{1},z_{2},\hat{\theta}_{1}\left(P_{1}\left(z_{1}\right)\tilde{P}_{2}\left(z_{2}\right)\hat{P}_{2}\left(w_{2}\right)\right),w_{2}\right),\\
\esp\left[z_{1}^{L_{1}\left(\overline{\zeta}_{2}\right)}z_{2}^{L_{2}\left(\overline{\zeta}_{2}\right)}w_{1}^{\hat{L}_{1}\left(\overline{\zeta}_{2}\right)}w_{2}^{\hat{L}_{2}\left(\overline{\zeta}_{2}\right)}\right]
&=&F_{2}\left(z_{1},z_{2};\zeta_{2}\right)\hat{F}_{2}\left(w_{1},\hat{\theta}_{2}\left(P_{1}\left(z_{1}\right)\tilde{P}_{2}\left(z_{2}\right)\hat{P}_{1}\left(w_{1}\right)\right)\right)\\
&\equiv&\hat{\mathbf{F}}_{2}\left(z_{1},z_{2},w_{1},\hat{\theta}_{2}\left(P_{1}\left(z_{1}\right)\tilde{P}_{2}\left(z_{2}\right)\hat{P}_{1}\left(w_{1}\right)\right)\right).
\end{eqnarray*}

Now we are in conditions to obtain the recursive equations that model the NCPS. We need to consider the switchover times that the server need to translate from one queue to another and, the number or user presents in the system at the time the server leaves to the queue to start attending the next. Thus far developed, we can find that for the NCPS:

\begin{eqnarray}\label{Recursive.Equations.First.Casse}
\begin{array}{r}
\mathbf{F}_{2}\left(z_{1},z_{2},w_{1},w_{2}\right)=R_{1}\left(P_{1}\left(z_{1}\right)\tilde{P}_{2}
\left(z_{2}\right)\prod_{i=1}^{2}
\hat{P}_{i}\left(w_{i}\right)\right)\mathbf{F}_{1}\left(\theta_{1}\left(\tilde{P}_{2}\left(z_{2}
\right)\hat{P}_{1}\left(w_{1}\right)\hat{P}_{2}\left(w_{2}\right)\right),z_{2},w_{1},w_{2}\right),\\
\mathbf{F}_{1}\left(z_{1},z_{2},w_{1},w_{2}\right)=R_{2}\left(P_{1}\left(z_{1}\right)\tilde{P}_{2}
\left(z_{2}\right)\prod_{i=1}^{2}
\hat{P}_{i}\left(w_{i}\right)\right)\mathbf{F}_{2}\left(z_{1},\tilde{\theta}_{2}\left(P_{1}\left(z_{1}\right)\hat{P}_{1}\left(w_{1}\right)\hat{P}_{2}\left(w_{2}
\right)\right),w_{1},w_{2}\right),\\
\hat{\mathbf{F}}_{2}\left(z_{1},z_{2},w_{1},w_{2}\right)=\hat{R}_{1}\left(P_{1}\left(z_{1}\right)\tilde{P}_{2}\left(z_{2}\right)\prod_{i=1}^{2}
\hat{P}_{i}\left(w_{i}\right)\right)\hat{\mathbf{F}}_{1}\left(z_{1},z_{2},\hat{\theta}_{1}\left(P_{1}\left(z_{1}\right)\tilde{P}_{2}\left(z_{2}\right)\hat{P}_{2}\left(w_{2}
\right)\right),w_{2}\right),\\
\hat{\mathbf{F}}_{1}\left(z_{1},z_{2},w_{1},w_{2}\right)=\hat{R}_{2}\left(P_{1}\left(z_{1}\right)\tilde{P}_{2}\left(z_{2}\right)\prod_{i=1}^{2}
\hat{P}_{i}\left(w_{i}\right)\right)\hat{\mathbf{F}}_{2}\left(z_{1},z_{2},w_{1},\hat{\theta}_{2}\left(P_{1}\left(z_{1}\right)\tilde{P}_{2}\left(z_{2}\right)\hat{P}_{1}\left(w_{1}
\right)\right)\right).
\end{array}
\end{eqnarray}


\begin{figure}[H]\caption{Network of Cyclic Polling System with double bidirectional transfer}
\centering
\includegraphics[width=9cm]{Grafica4.jpg}
\end{figure}\label{FigureRSVC3}


%_____________________________________________________
%\subsubsection{Server Switchover times}
%_____________________________________________________
It's necessary to give an step ahead, considering the case illustrated in \texttt{Figure 2}, where just like before, the server's switchover times are given by the generals equations
$R_{i}\left(\mathbf{z,w}\right)=R_{i}\left(\tilde{P}_{1}\left(z_{1}\right)
\tilde{P}_{2}\left(z_{2}\right)\tilde{P}_{3}\left(z_{3}\right)
\tilde{P}_{4}\left(z_{4}\right)\right)$, with first order derivatives given by $D_{i}R_{i}=r_{i}\tilde{\mu}_{i}$, and second order partial derivatives $D_{j}D_{i}R_{k}=R_{k}^{(2)}\tilde{\mu}_{i}\tilde{\mu}_{j}+\indora_{i=j}r_{k}P_{i}^{(2)}+\indora_{i\neq j}r_{k}\tilde{\mu}_{i}\tilde{\mu}_{j}$ for any $i,j,k$. According to the equations given before and the queue lengths for the other system's server times, we can obtain general expressions

\begin{eqnarray}\label{Ec.Gral.Primer.Momento.Ind.Exh}
\begin{array}{ll}
D_{j}\mathbf{F}_{i}\left(z_{1},z_{2};\tau_{i+2}\right)=\indora_{j\leq2}F_{j,i+2}^{(1)},&
D_{j}\mathbf{F}_{i}\left(z_{3},z_{4};\tau_{i-2}\right)=\indora_{j\geq3}F_{j,i-2}^{(1)},
\end{array}
\end{eqnarray}

for $i,j=1,2,3,4$; with second order derivatives given by

\begin{eqnarray}\label{Ec.Gral.Segundo.Momento.Ind.Exh}
\begin{array}{l}
D_{j}D_{i}\mathbf{F}_{k}\left(z_{1},z_{2};\tau_{k+2}\right)=\indora_{i\geq3}\indora_{j=i}F_{i,k+2}^{(2)}+\indora_{i\geq 3}\indora_{j\neq i}F_{j,k-2}^{(1)}F_{i,k+2}^{(1)},\\
D_{j}D_{i}\mathbf{F}_{k}\left(z_{3},z_{4};\tau_{k-2}\right)=\indora_{i\geq3}\indora_{j=i}F_{i,k-2}^{(2)}+\indora_{i\geq 3}\indora_{j\neq i}F_{j,k-2}^{(1)}F_{i,k-2}^{(1)}.
\end{array}
\end{eqnarray}


 According with the developed at the moment, we can get the recursive equations which are of the following form

\begin{eqnarray}\label{General.System.Double.Transfer}
\begin{array}{l}
\mathbf{F}_{1}\left(z_{1},z_{2},z_{3},z_{4}\right)=R_{2}\left(\prod_{i=1}^{4}\tilde{P}_{i}\left(z_{i}
\right)\right)\mathbf{F}_{2}\left(z_{1},\tilde{\theta}_{2}\left(\tilde{P}_{1}\left(z_{1}\right)\tilde{P}_{3}\left(z_{3}\right)\tilde{P}_{4}
\left(z_{4}\right)\right),z_{3},z_{4}\right),\\
\mathbf{F}_{2}\left(z_{1},z_{2},z_{3},z_{4}\right)=R_{1}\left(\prod_{i=1}^{4}\tilde{P}_{i}\left(z_{i}
\right)\right)
\mathbf{F}_{1}\left(\tilde{\theta}_{1}\left(\tilde{P}_{2}\left(z_{2}\right)\tilde{P}_{3}\left(z_{3}
\right)\tilde{P}_{4}\left(z_{4}\right)\right),z_{2},z_{3},z_{4}\right),\\
\mathbf{F}_{3}\left(z_{1},z_{2},z_{3},z_{4}\right)=R_{4}\left(\prod_{i=1}^{4}\tilde{P}_{i}\left(z_{i}
\right)\right)\mathbf{F}_{4}\left(z_{1},z_{2},z_{3},\tilde{\theta}_{4}\left(\tilde{P}_{1}\left(z_{1}\right)\tilde{P}_{2}\left(z_{2}\right)\tilde{P}_{3}\left(z_{3}\right)
\right)\right),\\
\mathbf{F}_{4}\left(z_{1},z_{2},z_{3},z_{4}\right)=R_{3}\left(\prod_{i=1}^{4}\tilde{P}_{i}\left(z_{i}
\right)\right)
\mathbf{F}_{3}\left(z_{1},z_{2},\tilde{\theta}_{3}\left(\tilde{P}_{1}\left(z_{1}\right)\tilde{P}_{2}\left(z_{2}\right)\tilde{P}_{4}
\left(z_{4}\right)\right),z_{4}\right).
\end{array}
\end{eqnarray}
%_________________________________________________________________________
%
%\subsection{Hipotesis sobre las colas}
%_________________________________________________________________________
%


So we have the first theorem

\begin{Teo}
Suppose  $\tilde{\mu}=\tilde{\mu}_{1}+\tilde{\mu}_{2}<1$, $\hat{\mu}=\tilde{\mu}_{3}+\tilde{\mu}_{4}<1$, then the number of users in the queues conforming the network of cyclic polling system (\ref{General.System.Double.Transfer}), when the server visit a queue can be found solving the linear system given by equations (\ref{Ec.Primer.Orden.General.Impar}) and (\ref{Ec.Primer.Orden.General.Par}):

\begin{eqnarray}\label{Ec.Primer.Orden.General.Impar}
\begin{array}{l}
f_{j}\left(i\right)=r_{j+1}\tilde{\mu}_{i}
+\indora_{i\neq j+1}f_{j+1}\left(j+1\right)\frac{\tilde{\mu}_{i}}{1-\tilde{\mu}_{j+1}}
+\indora_{i=j}f_{j+1}\left(i\right)
+\indora_{j=1}\indora_{i\geq3}F_{i,j+1}^{(1)}
+\indora_{j=3}\indora_{i\leq2}F_{i,j+1}^{(1)}
\end{array}
\end{eqnarray}
$j=1,3$ and $i=1,2,3,4$.

\begin{eqnarray}\label{Ec.Primer.Orden.General.Par}
\begin{array}{l}
f_{j}\left(i\right)=r_{j-1}\tilde{\mu}_{i}
+\indora_{i\neq j-1}f_{j-1}\left(j-1\right)\frac{\tilde{\mu}_{i}}{1-\tilde{\mu}_{j-1}}
+\indora_{i=j}f_{j-1}\left(i\right)
+\indora_{j=2}\indora_{i\geq3}F_{i,j-1}^{(1)}
+\indora_{j=4}\indora_{i\leq2}F_{i,j-1}^{(1)}
\end{array}
\end{eqnarray}
$j=2,4$ and $i=1,2,3,4$, whose solutions are:
%{\footnotesize{


\begin{eqnarray}
\begin{array}{l}
f_{i}\left(j\right)=\left(\indora_{j=i-1}+\indora_{j=i+1}\right)r_{j}\tilde{\mu}_{j}+\indora_{i=j}\left(\indora_{i\leq2}\frac{r\tilde{\mu}_{i}\left(1-\tilde{\mu}_{i}\right)}{1-\tilde{\mu}}+\indora_{i\geq2}\frac{\hat{r}\tilde{\mu}_{i}\left(1-\tilde{\mu}_{i}\right)}{1-\hat{\mu}}\right)\\
+\indora_{i=1}\indora_{j\geq3}\left(\tilde{\mu}_{j}\left(r_{i+1}+\frac{r\tilde{\mu}_{i+1}}{1-\tilde{\mu}}\right)+F_{j,i+1}^{(1)}\right)
+\indora_{i=3}\indora_{j\geq3}\left(\tilde{\mu}_{j}\left(r_{i+1}+\frac{\hat{r}\tilde{\mu}_{i+1}}{1-\hat{\mu}}\right)+F_{j,i+1}^{(1)}\right)\\
+\indora_{i=2}\indora_{j\leq2}\left(\tilde{\mu}_{j}\left(r_{i-1}+\frac{r\tilde{\mu}_{i-1}}{1-\tilde{\mu}}\right)+F_{j,i-1}^{(1)}\right)
+\indora_{i=4}\indora_{j\leq2}\left(\tilde{\mu}_{j}\left(r_{i-1}+\frac{\hat{r}\tilde{\mu}_{i-1}}{1-\hat{\mu}}\right)+F_{j,i-1}^{(1)}\right).
\end{array}
\end{eqnarray}
\end{Teo}
%______________________________________________________________________

\begin{Teo}
For the system given in (\ref{General.System.Double.Transfer}) we have that the second moments are in their general form

%{\small{
\begin{eqnarray}\label{Eq.Gral.Second.Order.Exhaustive}
\begin{array}{r}
f_{1}\left(i,k\right)=D_{k}D_{i}\left(R_{2}+\mathbf{F}_{2}+\indora_{i\geq3}\mathbf{F}_{4}\right)
+D_{i}R_{2}D_{k}\left(\mathbf{F}_{2}+\indora_{k\geq3}\mathbf{F}_{4}\right)
+D_{i}F_{2}D_{k}\left(R_{2}+\indora_{k\geq3}\mathbf{F}_{4}\right)\\
+\indora_{i\geq3}D_{i}\mathbf{F}_{4}D_{k}\left(R_{2}+\mathbf{F}_{2}\right)\\
f_{2}\left(i,k\right)=D_{k}D_{i}\left(R_{1}+\mathbf{F}_{1}+\indora_{i\geq3}\mathbf{F}_{3}\right)+D_{i}R_{1}D_{k}\left(\mathbf{F}_{1}+\indora_{k\geq3}\mathbf{F}_{3}\right)+D_{i}\mathbf{F}_{1}D_{k}\left(R_{1}+\indora_{k\geq3}\mathbf{F}_{3}\right)\\
+\indora_{i\geq3}D_{i}\mathbf{F}_{3}D_{k}\left(R_{1}+\mathbf{F}_{1}\right)\\
f_{3}\left(i,k\right)=D_{k}D_{i}\left(R_{4}+\indora_{i\leq2}\mathbf{F}_{2}+\mathbf{F}_{4}\right)+D_{i}\tilde{R}_{4}D_{k}\left(\indora_{k\leq2}\mathbf{F}_{2}+\mathbf{F}_{4}\right)+D_{i}\mathbf{F}_{4}D_{k}\left(R_{4}+\indora_{k\leq2}\mathbf{F}_{2}\right)\\
+\indora_{i\leq2}D_{i}\mathbf{F}_{2}D_{k}\left(R_{4}+\mathbf{F}_{4}\right)\\
f_{4}\left(i,k\right)=D_{k}D_{i}\left(R_{3}+\indora_{i\leq2}\mathbf{F}_{1}+\mathbf{F}_{3}\right)+D_{i}R_{3}D_{k}\left(\indora_{k\leq2}\mathbf{F}_{1}+\mathbf{F}_{3}\right)+D_{i}\mathbf{F}_{3}D_{k}\left(R_{3}+\indora_{k\leq2}\mathbf{F}_{1}\right)\\
+\indora_{i\leq2}D_{i}\mathbf{F}_{1}D_{k}\left(R_{3}+\mathbf{F}_{3}\right)
\end{array}
\end{eqnarray}%}}

\end{Teo}


\begin{Coro}\label{Coro.Second.Order.Eqs}
Conforming the equations given in (\ref{Eq.Gral.Second.Order.Exhaustive}) the second order moments are obtained solving the linear systems given by  (\ref{System.Second.Order.Moments.uno}). These solutions are 

\begin{eqnarray}\label{Sol.System.Second.Order.Exhaustive}
\begin{array}{ll}
f_{1}\left(1,1\right)=b_{3},&
f_{2}\left(2,2\right)=\frac{b_{2}}{1-b_{1}},\\
f_{1}\left(1,3\right)=a_{4}\left(\frac{b_{2}}{1-b_{1}}\right)+a_{5}K_{12}+K_{3},&
f_{1}\left(1,4\right)=a_{6}\left(\frac{b_{2}}{1-b_{1}}\right)+a_{7}K_{12}+K_{4},\\
f_{1}\left(3,3\right)=a_{8}\left(\frac{b_{2}}{1-b_{1}}\right)+K_{8},&
f_{1}\left(3,4\right)=a_{9}\left(\frac{b_{2}}{1-b_{1}}\right)+K_{9},\\
f_{1}\left(4,4\right)=a_{10}\left(\frac{b_{2}}{1-b_{1}}\right)+a_{5}K_{12}+K_{10},&
f_{2}\left(2,3\right)=a_{14}b_{3}+a_{15}K_{2}+K_{16},\\
f_{2}\left(2,4\right)=a_{16}b_{3}+a_{17}K_{2}+K_{17},&
f_{2}\left(3,3\right)=a_{18}b_{3}+K_{18},\\
f_{2}\left(3,4\right)=a_{19}b_{3}+K_{19},&
f_{2}\left(4,4\right)=a_{20}b_{3}+K_{20},\\
f_{3}\left(3,3\right)=\frac{b_{5}}{1-b_{4}},&
f_{4}\left(4,4\right)=b_{6},\\
f_{3}\left(1,1\right)=a_{21}b_{6}+K_{21},&
f_{3}\left(1,2\right)=a_{22}b_{6}+K_{22},\\
f_{3}\left(1,3\right)=a_{23}b_{6}+a_{24}K_{39}+K_{23},&
f_{3}\left(2,2\right)=a_{25}b_{6}+K_{25},\\
f_{3}\left(2,3\right)=a_{26}b_{6}+a_{27}K_{39}+K_{26},&
f_{4}\left(1,1\right)=a_{31}\left(\frac{b_{5}}{1-b_{4}}\right)+K_{31},\\
f_{4}\left(1,2\right)=a_{32}\left(\frac{b_{5}}{1-b_{4}}\right)+K_{32},&
f_{4}\left(1,4\right)=a_{33}\left(\frac{b_{5}}{1-b_{4}}\right)+a_{34}K_{29}+K_{31},\\
f_{4}\left(2,2\right)=a_{35}\left(\frac{b_{5}}{1-b_{4}}\right)+K_{35},&
f_{4}\left(2,4\right)=a_{36}\left(\frac{b_{5}}{1-b_{4}}\right)+a_{37}K_{29}+K_{37}.
\end{array}
\end{eqnarray}

where
\begin{eqnarray*}
\begin{array}{lll}
N_{1}=a_{2}K_{12}+a_{3}K_{11}+K_{1},&
N_{2}=a_{12}K_{2}+a_{13}K_{5}+K_{15},&
b_{1}=a_{1}a_{11},\\
b_{2}=a_{11}N_{1}+N_{2},&
b_{3}=a_{1}\left(\frac{b_{2}}{1-b_{1}}\right)+N_{1},&
N_{3}=a_{29}K_{39}+a_{30}K_{38}+K_{28},\\
N_{4}=a_{39}K_{29}+a_{40}K_{30}+K_{40},&
b_{4}=a_{28}a_{38},&
b_{5}=a_{28}N_{4}+N_{3},\\
&b_{6}=a_{38}\left(\frac{b_{5}}{1-b_{4}}\right)+N_{4}.&
\end{array}
\end{eqnarray*}

\end{Coro}
The values for the $a_{i}$'s and $K_{i}$ can be found in \textit{Appendix B}. Finally 

\begin{Def}
Let $L_{i}^{*}$ be the number of users at queue $Q_{i}$ when it is polled, then
\begin{eqnarray}
\begin{array}{cc}
\esp\left[L_{i}^{*}\right]=f_{i}\left(i\right), &
Var\left[L_{i}^{*}\right]=f_{i}\left(i,i\right)+\esp\left[L_{i}^{*}\right]-\esp\left[L_{i}^{*}\right]^{2}.
\end{array}
\end{eqnarray}
\end{Def}

%_________________________________________________________________________
%
\subsection{Stability Analysis}
%_________________________________________________________________________
%

We are interested in determine the queue lengths at any time, not just when the server arrives to the queue to start attending according to the exhaustive policy. For this purpose we need to make assumptions over the processes involved in order to guarantee the stability of the Network.



First of all we are going to assume the arrival processes are Poisson, the service time are exponential. In 1973 Disney \cite{Disney} prove that the only stationary system $M/G/1/L$, with renewal departure process are the $M/M/1$ y $M/D/1/1$ systems, also this implies that the output process is Poisson with the same rate of the arrival process. The switchover times has no particular distribution, the only condition they have to satisfy is the first moment finite.

Sigman, Thorison and Wolff \cite{Sigman2} proved that if there is a first regeneration time then exist a non decreasing infinite sequence of regeneration times. With this in consideration we have the following theorem 


\begin{Teo}\label{First.Regeneration.Time.Theorem}
Given a Network of Cyclic Polling Systems (NCPS) conformed by two cyclic polling systems, each of them with $M/M/1$ queues. Both systems are related by users transfer between the queues $Q_{1},Q_{3}$ and $Q_{2},Q_{4}$. Suppose $\tilde{\mu},\hat{\mu}<1$. Let's define the following events for the arrival processes at time $t$: $A_{j}\left(t\right)=\left\{0 \textrm{ arrivals on }Q_{j}\textrm{ at time }t\right\}$, for some $t\geq0$ and queue $Q_{j}$ in the NCPS for $j=1,2,3,4$. Then there exist an non empty interval $I$ such that for $T^{*}\in I$ the $\prob\left\{A_{1}\left(T^{*}\right),A_{2}\left(Tt^{*}\right),
A_{3}\left(T^{*}\right),A_{4}\left(T^{*}\right)|T^{*}\in I\right\}>0$ is satisfied.

\end{Teo}
\begin{proof}

Without of loss of generality we are going to consider as base of the analysis the queue $Q_{1}$ from the first system.

Let's $n\geq1$ cycle for the first system, so let's be $\overline{\tau}_{1}\left(n\right)$ time the server ends attending en queue $Q_{1}$, it means 
$L_{j}\left(\overline{\tau}_{1}\left(n\right)\right)=0$. The server incurrs in a switchover time to traslate to the other queue, which is a random variable whose realitation is $r_{1}\left(n\right)>0$, then we have that $\tau_{2}\left(n\right)=\overline{\tau}_{1}\left(n\right)+r_{1}\left(n\right)$.

Let's be $I_{1}\equiv\left[\overline{\tau}_{1}\left(n\right),\tau_{2}\left(n\right)\right]$ the intreval with length $\xi_{1}=r_{1}\left(n\right)$. Given that the arrival times are exponentials with rate $\tilde{\mu}_{1}=\mu_{1}+\hat{\mu}_{1}$ and the transfer users process from queue $Q_{3}$ are exponentials with rate $\hat{\mu}_{1}$, we have that the event $A_{1}\left(t\right)$ has probability given by 

\begin{equation}
\prob\left\{A_{1}\left(t\right)|t\in I_{1}\left(n\right)\right\}=e^{-\tilde{\mu}_{1}\xi_{1}\left(n\right)}.
\end{equation} 

In the other side, for the queue $Q_{2}$, the time 
$\overline{\tau}_{2}\left(n-1\right)$ is such that 
$L_{2}\left(\overline{\tau}_{2}\left(n-1\right)\right)=0$, it means, it's the time when the queue is emptied by the server en the previous cycle. So we have a second time interval $I_{2}\equiv\left[\overline{\tau}_{2}\left(n-1\right),\tau_{2}\left(n\right)\right]$ so the event $A_{2}\left(t\right)$ has probability

\begin{equation}
\prob\left\{A_{2}\left(t\right)|t\in I_{2}\left(n\right)\right\}=e^{-\tilde{\mu}_{2}\xi_{2}\left(n\right)},
\end{equation} 
with length 
$\xi_{2}\left(n\right)=\tau_{2}\left(n\right)-\overline{\tau}_{2}\left(n-1\right)$. Given the time intervals construction we have that $I_{1}\left(n\right)\subset I_{2}\left(n\right)$, therefore  $\xi_{1}\left(n\right)\leq\xi_{2}\left(n\right)$ so $-\xi_{1}\left(n\right)\geq-\xi_{2}\left(n\right)$ then $-\tilde{\mu}_{2}\xi_{1}\left(n\right)\geq-\tilde{\mu}_{2}\xi_{2}\left(n\right)$ and finally $e^{-\tilde{\mu}_{2}\xi_{1}\left(n\right)}\geq e^{-\tilde{\mu}_{2}\xi_{2}\left(n\right)}$, then

\begin{equation}
\prob\left\{A_{2}\left(t\right)|t\in I_{1}\left(n\right)\right\}\geq
\prob\left\{A_{2}\left(t\right)|t\in I_{2}\left(n\right)\right\}.
\end{equation}

Now we can determine the joint conditional probability on the interval $I_{1}\left(n\right)$
\begin{eqnarray*}
\prob\left\{A_{1}\left(t\right),A_{2}\left(t\right)|t\in I_{1}\left(n\right)\right\}&=&
\prob\left\{A_{1}\left(t\right)|t\in I_{1}\left(n\right)\right\}
\prob\left\{A_{2}\left(t\right)|t\in I_{1}\left(n\right)\right\}\\
&\geq&
\prob\left\{A_{1}\left(t\right)|t\in I_{1}\left(n\right)\right\}
\prob\left\{A_{2}\left(t\right)|t\in I_{2}\left(n\right)\right\}\\
&=&e^{-\tilde{\mu}_{1}\xi_{1}\left(n\right)}e^{-\tilde{\mu}_{2}\xi_{2}\left(n\right)}
=e^{-\left[\tilde{\mu}_{1}\xi_{1}\left(n\right)+\tilde{\mu}_{2}\xi_{2}\left(n\right)\right]}.
\end{eqnarray*}

It means 
\begin{equation}
\prob\left\{A_{1}\left(t\right),A_{2}\left(t\right)|t\in I_{1}\left(n\right)\right\}
=e^{-\left[\tilde{\mu}_{1}\xi_{1}\left(n\right)+\tilde{\mu}_{2}\xi_{2}
\left(n\right)\right]}>0.
\end{equation}

With respect the relation between both systems, there exists some $m\geq1$ such that $\tau_{3}\left(m\right)<\tau_{2}\left(n\right)<\tau_{4}\left(m\right)$ therefore we have the following cases for $\tau_{2}\left(n\right)$:

\begin{multicols}{2}
\begin{itemize}
\item[a)] $\tau_{3}\left(m\right)<\tau_{2}\left(n\right)<\overline{\tau}_{3}\left(m\right)$,

\item[b)] $\overline{\tau}_{3}\left(m\right)<\tau_{2}\left(n\right)
<\tau_{4}\left(m\right)$,

\item[c)] $\tau_{4}\left(m\right)<\tau_{2}\left(n\right)<
\overline{\tau}_{4}\left(m\right)$,

\item[d)] $\overline{\tau}_{4}\left(m\right)<\tau_{2}\left(n\right)
<\tau_{3}\left(m+1\right)$.
\end{itemize}
\end{multicols}

First consider the time interval $I_{3}\left(m\right)\equiv\left[\tau_{3}\left(m\right),\overline{\tau}_{3}\left(m\right)\right]$ such that $\tau_{2}\left(n\right)\in I_{3}\left(m\right)$, with length $\xi_{3}\equiv\overline{\tau}_{3}\left(m\right)-\tau_{3}\left(m\right)$, then we have for the queue $Q_{3}$
\begin{equation}
\prob\left\{A_{3}\left(t\right)|t\in I_{3}\left(m\right)\right\}=e^{-\tilde{\mu}_{3}\xi_{3}\left(m\right)}.
\end{equation} 

whereas for $Q_{4}$ lets consider the time interval $I_{4}\left(m\right)\equiv\left[\tau_{4}\left(m-1\right),\overline{\tau}_{3}\left(m\right)\right]$, then we have that $I_{3}\left(m\right)\subset I_{4}\left(m\right)$, therefore in a similar manner that we have done for $Q_{1}$ and $Q_{2}$ we obtain


\begin{equation}
\prob\left\{A_{4}\left(t\right)|t\in I_{3}\left(m\right)\right\}\geq
\prob\left\{A_{4}\left(t\right)|t\in I_{4}\left(m\right)\right\}
\end{equation}

and

\begin{equation}
\prob\left\{A_{3}\left(t\right),A_{4}\left(t\right)|t\in I_{3}\left(m\right)\right\}\geq
e^{-\left(\tilde{\mu}_{3}\xi_{3}\left(m\right)+\tilde{\mu}_{4}\xi_{4}\left(m\right)\right)}>0.
\end{equation}


For the rest of the cases the demonstration is similar. It means we always can find a time interval where we can guarantee there is no arrivals to the queues in each system with positive probability.  


By construction we have that $I\left(n,m\right)\equiv I_{1}\left(n\right)\cap I_{3}\left(m\right)\neq\emptyset$, then in particular we have the following contentions $I\left(n,m\right)\subseteq I_{1}\left(n\right)$ and $I\left(n,m\right)\subseteq I_{3}\left(m\right)$, therefore if we define $\xi\left(n,m\right)$ as the length of the interval $I\left(n,m\right)$ we have $\xi\left(n,m\right)\leq\xi_{1}\left(n\right)$, $\xi\left(n,m\right)\leq\xi_{3}\left(m\right)$, then $-\xi\left(n,m\right)\geq-\xi_{1}\left(n\right)$ and finally $-\xi\left(n,m\right)\leq-\xi_{3}\left(m\right)$ therefore we have the following
\begin{multicols}{2}
\begin{enumerate}
\item $-\tilde{\mu}_{1}\xi_{n,m}\geq-\tilde{\mu}_{1}\xi_{1}\left(n\right)$,
\item $-\tilde{\mu}_{2}\xi_{n,m}\geq-\tilde{\mu}_{2}\xi_{1}\left(n\right)
\geq-\tilde{\mu}_{2}\xi_{2}\left(n\right)$,
\item $-\tilde{\mu}_{3}\xi_{n,m}\geq-\tilde{\mu}_{3}\xi_{3}\left(m\right)$,
\item $-\tilde{\mu}_{4}\xi_{n,m}\geq-\tilde{\mu}_{4}\xi_{3}\left(m\right)
\geq-\tilde{\mu}_{4}\xi_{4}\left(m\right).$
\end{enumerate}
\end{multicols}

Let's $T^{*}\in I\left(n,m\right)$, then given that in particular $T^{*}\in I_{1}\left(n\right)$, there is no arrivals to the queues $Q_{1}$ and $Q_{2}$, therefore there is no transfer users from $Q_{3}$ and $Q_{4}$, it means, $\tilde{\mu}_{1}=\mu_{1}$, $\tilde{\mu}_{2}=\mu_{2}$, $\tilde{\mu}_{3}=\mu_{3}$, $\tilde{\mu}_{4}=\mu_{4}$, thats it, the events $A_{1}$ and $A_{3}$ are conditionally independent in the interval $I\left(n,m\right)$; the same goes for the events $A_{2}$ and $A_{4}$, therefore we have
%\small{
\begin{eqnarray}
\begin{array}{l}
\prob\left\{A_{1}\left(T^{*}\right),A_{2}\left(T^{*}\right),
A_{3}\left(T^{*}\right),A_{4}\left(T^{*}\right)|T^{*}\in I\left(n,m\right)\right\}
=\prod_{j=1}^{4}\prob\left\{A_{j}\left(T^{*}\right)|T^{*}\in I\left(n,m\right)\right\}\\
\geq\prob\left\{A_{1}\left(T^{*}\right)|T^{*}\in I_{1}\left(n\right)\right\}
\prob\left\{A_{2}\left(T^{*}\right)|T^{*}\in I_{2}\left(n\right)\right\}
\prob\left\{A_{3}\left(T^{*}\right)|T^{*}\in I_{3}\left(m\right)\right\}
\prob\left\{A_{4}\left(T^{*}\right)|T^{*}\in I_{4}\left(m\right)\right\}\\
=e^{-\mu_{1}\xi_{1}\left(n\right)}
e^{-\mu_{2}\xi_{2}\left(n\right)}
e^{-\mu_{3}\xi_{3}\left(m\right)}
e^{-\mu_{4}\xi_{4}\left(m\right)}
=e^{-\left[\tilde{\mu}_{1}\xi_{1}\left(n\right)
+\tilde{\mu}_{2}\xi_{2}\left(n\right)
+\tilde{\mu}_{3}\xi_{3}\left(m\right)
+\tilde{\mu}_{4}\xi_{4}
\left(m\right)\right]}>0.
\end{array}
\end{eqnarray}

Now we only need to prove that for $n\ge1$, there exist an $m\geq1$ such that the cases mentioned before are satisfied: 

\begin{multicols}{2}
\begin{itemize}
\item[a)] $\tau_{3}\left(m\right)<\tau_{2}\left(n\right)<\overline{\tau}_{3}\left(m\right)$,

\item[b)] $\overline{\tau}_{3}\left(m\right)<\tau_{2}\left(n\right)
<\tau_{4}\left(m\right)$,

\item[c)] $\tau_{4}\left(m\right)<\tau_{2}\left(n\right)<
\overline{\tau}_{4}\left(m\right)$,

\item[d)] $\overline{\tau}_{4}\left(m\right)<\tau_{2}\left(n\right)
<\tau_{3}\left(m+1\right)$.
\end{itemize}
\end{multicols}
We only give the proof for the fist case, for the rest the demonstration are similar. Suppose there is no $m\geq1$, with $I_{1}\left(n\right)\cap I_{3}\left(m\right)\neq\emptyset$, it means that for all $m\geq1$, $I_{1}\left(n\right)\cap I_{3}\left(m\right)=\emptyset$, then we have only two cases

\begin{itemize}
\item[a)] $\tau_{2}\left(n\right)\leq\tau_{3}\left(m\right)$: Recall that $\tau_{2}\left(m\right)=\overline{\tau}_{1}+r_{1}\left(m\right)$ 
where each of the random variables are such that $\esp\left[\overline{\tau}_{1}\left(n\right)-\tau_{1}\left(n\right)\right]<\infty$, $\esp\left[R_{1}\right]<\infty$ y $\esp\left[\tau_{3}\left(m\right)\right]<\infty$, which contradicts the fact that there is no such $m\geq1$.

\item[b)] $\tau_{2}\left(n\right)\geq\overline{\tau}_{3}\left(m\right)$: the reasoning is similar to the previous given.

\end{itemize}

\end{proof}


According to the stablished in Sigman, Thorison and Wolff \cite{Sigman2} theorem (\ref{First.Regeneration.Time.Theorem}) allow us to ensure that there is an infinite sequence of regeneration times, let $T_{1},T_{2},\ldots$ considered as the regeneration points, then we have that just like in Takagi \cite{Takagi}, the following definition

\begin{Def}
the interval between two such succesive regeneration points will be called regenerative cycle.
\end{Def}

And for the regeneration points 

\begin{Def}
Let $M_{i}$ be the number of polling cycles in a regenerative cycle.
\end{Def}

\begin{Def}
Considering the $M_{i}$'s, the duration of the $m$-th polling cycle in a regeneration cycle will be denoted by $C_{i}^{(m)}$, for $m=1,2,\ldots,M_{i}$.
\end{Def}

And finally, the mean polling cycle time is defined by

\begin{Def}
\begin{equation}
\esp\left[C_{i}\right]=\frac{\sum_{m=1}^{M_{i}}\esp\left[C_{i}^{(m)}\right]}{\esp\left[M_{i}\right]}
\end{equation}
\end{Def}

\begin{Teo}
The process $\left\{C_{i}:i=1,2,\ldots,M_{i}\right\}$ is a regenerative process. Also there exists a regenerative and stationary process as function of this process.
\end{Teo}

With this in mind let denote by $L_{i}$ the number of users at queue $Q_{i}$ at arbitrary times. Their generating probability function  will be denoted by $Q_{i}\left(z\right)$ which is also given by the time average of $z^{L_{i}\left(t\right)}$ over the regenerative cycled defined before so we have

\begin{eqnarray*}
Q_{i}\left(z\right)&=&\esp\left[z^{L_{i}\left(t\right)}\right]=\frac{\esp\left[\sum_{m=1}^{M_{i}}\sum_{t=\tau_{i}\left(m\right)}^{\tau_{i}\left(m+1\right)-1}z^{L_{i}\left(t\right)}\right]}{\esp\left[\sum_{m=1}^{M_{i}}\tau_{i}\left(m+1\right)-\tau_{i}\left(m\right)\right]}
\end{eqnarray*}

which can be rewritten as

\begin{equation}\label{Eq.Long.Caulquier.Tiempo}
Q_{i}\left(z\right)=\frac{1}{\esp\left[C_{i}\right]}\cdot\frac{1-F_{i}\left(z\right)}{P_{i}\left(z\right)-z}\cdot\frac{\left(1-z\right)P_{i}\left(z\right)}{1-P_{i}\left(z\right)}
\end{equation}

If we define the following
\begin{eqnarray}
\begin{array}{ccc}
S\left(z\right)=1-F\left(z\right),&
T\left(z\right)=z-P\left(z\right),&
U\left(z\right)=1-P\left(z\right).
\end{array}
\end{eqnarray}
then 

\begin{eqnarray}
\esp\left[C_{i}\right]Q\left(z\right)=\frac{\left(z-1\right)S\left(z\right)P\left(z\right)}{T\left(z\right)U\left(z\right)}.
\end{eqnarray}

Where if we define $a_{k}=P\left\{L\left(t\right)=k\right\}$ then 
\begin{eqnarray*}
S\left(z\right)=1-F\left(z\right)=1-\sum_{k=0}^{+\infty}a_{k}z^{k}
\end{eqnarray*}
therefore $S^{'}\left(z\right)=-\sum_{k=1}^{+\infty}ka_{k}z^{k-1}$, with $S^{(1)}\left(1\right)=-\sum_{k=1}^{+\infty}ka_{k}=-\esp\left[L\left(t\right)\right]$,
and $S^{''}\left(z\right)=-\sum_{k=2}^{+\infty}k(k-1)a_{k}z^{k-2}$ so  $S^{(2)}\left(1\right)=-\sum_{k=2}^{+\infty}k(k-1)a_{k}=\esp\left[L\left(L-1\right)\right]$;
in the same way we obtain $S^{'''}\left(z\right)=-\sum_{k=3}^{+\infty}k(k-1)(k-2)a_{k}z^{k-3}$ and $S^{(3)}\left(1\right)=-\sum_{k=3}^{+\infty}k(k-1)(k-2)a_{k}=-\esp\left[L\left(L-1\right)\left(L-2\right)\right]
=-\esp\left[L^{3}\right]+3-\esp\left[L^{2}\right]-2-\esp\left[L\right]$. 

it means

\begin{eqnarray}
\begin{array}{l}
S^{(1)}\left(1\right)=-\esp\left[L\left(t\right)\right],\\ S^{(2)}\left(1\right)=-\esp\left[L\left(L-1\right)\right]
=-\esp\left[L^{2}\right]+\esp\left[L\right],\\
S^{(3)}\left(1\right)=-\esp\left[L\left(L-1\right)\left(L-2\right)\right]
=-\esp\left[L^{3}\right]+3\esp\left[L^{2}\right]-2\esp\left[L\right].
\end{array}
\end{eqnarray}


expanding around $z=1$

\begin{eqnarray*}
S\left(z\right)&=&S\left(1\right)+\frac{S^{'}\left(1\right)}{1!}\left(z-1\right)+\frac{S^{''}\left(1\right)}{2!}\left(z-1\right)^{2}+\frac{S^{'''}\left(1\right)}{3!}\left(z-1\right)^{3}+\ldots+\\
&=&\left(z-1\right)\left\{S^{'}\left(1\right)+\frac{S^{''}\left(1\right)}{2!}\left(z-1\right)+\frac{S^{'''}\left(1\right)}{3!}\left(z-1\right)^{2}+\ldots+\right\}\\
&=&\left(z-1\right)R_{1}\left(z\right)
\end{eqnarray*}
with $R_{1}\left(z\right)\neq0$, given that $R_{1}\left(z\right)=-\esp\left[L\right]$ then

\begin{eqnarray}
R_{1}\left(z\right)&=&S^{'}\left(1\right)+\frac{S^{''}\left(1\right)}{2!}\left(z-1\right)+\frac{S^{'''}\left(1\right)}{3!}\left(z-1\right)^{2}+\frac{S^{iv}\left(1\right)}{4!}\left(z-1\right)^{3}+\ldots+
\end{eqnarray}
Calculating the derivatives and evaluating in $z=1$

\begin{eqnarray}
\begin{array}{l}
R_{1}\left(1\right)=S^{(1)}\left(1\right)=-\esp\left[L\right]\\
R_{1}^{(1)}\left(1\right)=\frac{1}{2}S^{(2)}\left(1\right)=-\frac{1}{2}\esp\left[L^{2}\right]+\frac{1}{2}\esp\left[L\right]\\
R_{1}^{(2)}\left(1\right)=\frac{2}{3!}S^{(3)}\left(1\right)
=-\frac{1}{3}\esp\left[L^{3}\right]+\esp\left[L^{2}\right]-\frac{2}{3}\esp\left[L\right]
\end{array}
\end{eqnarray}

In a similar manner for $T\left(z\right)=z-P\left(z\right)$ can be found an expansion around $z=1$:

\begin{eqnarray*}
T\left(z\right)&=&T\left(1\right)+\frac{T^{'}\left(1\right)}{1!}\left(z-1\right)+\frac{T^{''}\left(1\right)}{2!}\left(z-1\right)^{2}+\frac{T^{'''}\left(1\right)}{3!}\left(z-1\right)^{3}+\ldots+\\
&=&\left(z-1\right)\left\{T^{'}\left(1\right)+\frac{T^{''}\left(1\right)}{2!}\left(z-1\right)+\frac{T^{'''}\left(1\right)}{3!}\left(z-1\right)^{2}+\ldots+\right\}\\
&=&\left(z-1\right)R_{2}\left(z\right)
\end{eqnarray*}

where
\begin{eqnarray}
\begin{array}{l}
T^{(1)}\left(1\right)=-\esp\left[X\left(t\right)\right]=-\mu,\\ T^{(2)}\left(1\right)=-\esp\left[X\left(X-1\right)\right]
=-\esp\left[X^{2}\right]+\esp\left[X\right]=-\esp\left[X^{2}\right]+\mu,\\
T^{(3)}\left(1\right)=-\esp\left[X\left(X-1\right)\left(X-2\right)\right]
=-\esp\left[X^{3}\right]+3\esp\left[X^{2}\right]-2\esp\left[X\right]\\
=-\esp\left[X^{3}\right]+3\esp\left[X^{2}\right]-2\mu.
\end{array}
\end{eqnarray}

therefore $R_{2}\left(1\right)\neq0$, because

\begin{eqnarray}\label{Eq.R2}
R_{2}\left(1\right)=1-\esp\left[X\right]=1-\mu
\end{eqnarray}
then 

\begin{eqnarray}
R_{2}\left(z\right)&=&T^{'}\left(1\right)+\frac{T^{''}\left(1\right)}{2!}\left(z-1\right)+\frac{T^{'''}\left(1\right)}{3!}\left(z-1\right)^{2}+\frac{T^{(iv)}\left(1\right)}{4!}\left(z-1\right)^{3}+\ldots+
\end{eqnarray}
Calculating the derivatives and evaluating $z=1$

\begin{eqnarray}
\begin{array}{l}
R_{2}\left(1\right)=T^{(1)}\left(1\right)=1-\mu\\
R_{2}^{(1)}\left(1\right)=\frac{1}{2}T^{(2)}\left(1\right)=-\frac{1}{2}\esp\left[X^{2}\right]+\frac{1}{2}\mu\\
R_{2}^{(2)}\left(1\right)=\frac{2}{3!}T^{(3)}\left(1\right)
=-\frac{1}{3}\esp\left[X^{3}\right]+\esp\left[X^{2}\right]-\frac{2}{3}\mu
\end{array}
\end{eqnarray}
Finally proceeding in analogous manner for $U\left(z\right)=1-P\left(z\right)$ also can be found an expansion around $z=1$

\begin{eqnarray*}
\begin{array}{l}
U\left(z\right)=U\left(1\right)+\frac{U^{'}\left(1\right)}{1!}\left(z-1\right)+\frac{U^{''}\left(1\right)}{2!}\left(z-1\right)^{2}+\frac{U^{'''}\left(1\right)}{3!}\left(z-1\right)^{3}+\ldots+\\
=\left(z-1\right)\left\{U^{'}\left(1\right)+\frac{U^{''}\left(1\right)}{2!}\left(z-1\right)+\frac{U^{'''}\left(1\right)}{3!}\left(z-1\right)^{2}+\ldots+\right\}
=\left(z-1\right)R_{3}\left(z\right)
\end{array}
\end{eqnarray*}

where

\begin{eqnarray*}
\begin{array}{l}
U^{(1)}\left(1\right)=-\esp\left[X\left(t\right)\right]=-\mu,\\ U^{(2)}\left(1\right)=-\esp\left[X\left(X-1\right)\right]
=-\esp\left[X^{2}\right]+\esp\left[X\right]=-\esp\left[X^{2}\right]+\mu,\\
U^{(3)}\left(1\right)=-\esp\left[X\left(X-1\right)\left(X-2\right)\right]
=-\esp\left[X^{3}\right]+3\esp\left[X^{2}\right]-2\esp\left[X\right]\\
=-\esp\left[X^{3}\right]+3\esp\left[X^{2}\right]-2\mu.
\end{array}
\end{eqnarray*}

Therefore $R_{3}\left(1\right)\neq0$, because

\begin{eqnarray}\label{Eq.R3}
R_{3}\left(1\right)=-\esp\left[X\right]=-\mu
\end{eqnarray}
then

\begin{eqnarray}
R_{3}\left(z\right)&=&U^{'}\left(1\right)+\frac{U^{''}\left(1\right)}{2!}\left(z-1\right)+\frac{U^{'''}\left(1\right)}{3!}\left(z-1\right)^{2}+\frac{U^{(iv)}\left(1\right)}{4!}\left(z-1\right)^{3}+\ldots+
\end{eqnarray}

calculating the derivatives and evaluating in $z=1$

\begin{eqnarray}
\begin{array}{l}
R_{3}\left(1\right)=U^{(1)}\left(1\right)=-\mu\\
R_{3}^{(1)}\left(1\right)=\frac{1}{2}U^{(2)}\left(1\right)=-\frac{1}{2}\esp\left[X^{2}\right]+\frac{1}{2}\mu\\
R_{3}^{(2)}\left(1\right)=\frac{2}{3!}U^{(3)}\left(1\right)
=-\frac{1}{3}\esp\left[X^{3}\right]+\esp\left[X^{2}\right]-\frac{2}{3}\mu
\end{array}
\end{eqnarray}

Then we have that 

\begin{eqnarray}
\esp\left[C_{i}\right]Q\left(z\right)&=&\frac{\left(z-1\right)\left(z-1\right)R_{1}\left(z\right)P\left(z\right)}{\left(z-1\right)R_{2}\left(z\right)\left(z-1\right)R_{3}\left(z\right)}
=\frac{R_{1}\left(z\right)P\left(z\right)}{R_{2}\left(z\right)R_{3}\left(z\right)}\equiv\frac{R_{1}P}{R_{2}R_{3}}
\end{eqnarray}

Calcuating the derivative with respect $z$

\begin{eqnarray}\label{Ec.Primer.Derivada.Q}
\left[\frac{R_{1}\left(z\right)P\left(z\right)}{R_{2}\left(z\right)R_{3}\left(z\right)}\right]^{'}&=&\frac{PR_{2}R_{3}R_{1}^{'}
+R_{1}R_{2}R_{3}P^{'}-R_{3}R_{1}PR_{2}-R_{2}R_{1}PR_{3}^{'}}{\left(R_{2}R_{3}\right)^{2}}
\end{eqnarray}
evauatong in $z=1$
\begin{eqnarray*}
&=&\frac{R_{2}(1)R_{3}(1)R_{1}^{(1)}(1)+R_{1}(1)R_{2}(1)R_{3}(1)P^{'}(1)-R_{3}(1)R_{1}(1)R_{2}(1)^{(1)}-R_{2}(1)R_{1}(1)R_{3}^{'}(1)}{\left(R_{2}(1)R_{3}(1)\right)^{2}}\\
&=&\frac{1}{\left(1-\mu\right)^{2}\mu^{2}}\left\{\left(-\frac{1}{2}\esp L^{2}+\frac{1}{2}\esp L\right)\left(1-\mu\right)\left(-\mu\right)+\left(-\esp L\right)\left(1-\mu\right)\left(-\mu\right)\mu\right.\\
&&\left.-\left(-\frac{1}{2}\esp X^{2}+\frac{1}{2}\mu\right)\left(-\mu\right)\left(-\esp L\right)-\left(1-\mu\right)\left(-\esp L\right)\left(-\frac{1}{2}\esp X^{2}+\frac{1}{2}\mu\right)\right\}\\
&=&\frac{1}{\left(1-\mu\right)^{2}\mu^{2}}\left\{\left(-\frac{1}{2}\esp L^{2}+\frac{1}{2}\esp L\right)\left(\mu^{2}-\mu\right)
+\left(\mu^{2}-\mu^{3}\right)\esp L\right.\\
&&\left.-\mu\esp L\left(-\frac{1}{2}\esp X^{2}+\frac{1}{2}\mu\right)
+\left(\esp L-\mu\esp L\right)\left(-\frac{1}{2}\esp X^{2}+\frac{1}{2}\mu\right)\right\}\\
&=&\frac{1}{\left(1-\mu\right)^{2}\mu^{2}}\left\{-\frac{1}{2}\mu^{2}\esp L^{2}
+\frac{1}{2}\mu\esp L^{2}
+\frac{1}{2}\mu^{2}\esp L
-\mu^{3}\esp L
+\mu\esp L\esp X^{2}
-\frac{1}{2}\esp L\esp X^{2}\right\}\\
&=&\frac{1}{\left(1-\mu\right)^{2}\mu^{2}}\left\{
\frac{1}{2}\mu\esp L^{2}\left(1-\mu\right)
+\esp L\left(\frac{1}{2}-\mu\right)\left(\mu^{2}-\esp X^{2}\right)\right\}\\
&=&\frac{1}{2\mu\left(1-\mu\right)}\esp L^{2}-\frac{\frac{1}{2}-\mu}{\left(1-\mu\right)^{2}\mu^{2}}\sigma^{2}\esp L
\end{eqnarray*}

Then we get (Takagi's formula)

\begin{eqnarray*}
Q^{(1)}=\frac{1}{\esp C}\left\{\frac{1}{2\mu\left(1-\mu\right)}\esp L^{2}-\frac{\frac{1}{2}-\mu}{\left(1-\mu\right)^{2}\mu^{2}}\sigma^{2}\esp L\right\}
\end{eqnarray*}
with

\begin{eqnarray*}
\esp C = \frac{\esp L}{\mu\left(1-\mu\right)}
\end{eqnarray*}
therefore

\begin{eqnarray*}
Q^{(1)}&=&\frac{1}{2}\frac{\esp L^{2}}{\esp L}-\frac{\frac{1}{2}-\mu}{\left(1-\mu\right)\mu}\sigma^{2}
=\frac{\esp L^{2}}{2\esp L}-\frac{\sigma^{2}}{2}\left\{\frac{2\mu-1}{\left(1-\mu\right)\mu}\right\}\\
&=&\frac{\esp L^{2}}{2\esp L}+\frac{\sigma^{2}}{2}\left\{\frac{1}{1-\mu}+\frac{1}{\mu}\right\}
\end{eqnarray*}

while for us

\begin{eqnarray*}
Q^{(1)}=\frac{1}{\mu\left(1-\mu\right)}\frac{\esp L^{2}}{2\esp C}
-\sigma^{2}\frac{\esp L}{2\esp C}\cdot\frac{1-2\mu}{\left(1-\mu\right)^{2}\mu^{2}}.
\end{eqnarray*}

Now, reconsider the equation (\ref{Ec.Primer.Derivada.Q})

\begin{eqnarray*}
\left[\frac{R_{1}\left(z\right)P\left(z\right)}{R_{2}\left(z\right)R_{3}\left(z\right)}\right]^{'}&=&\frac{PR_{2}R_{3}R_{1}^{'}
+R_{1}R_{2}R_{3}P^{'}-R_{3}R_{1}PR_{2}-R_{2}R_{1}PR_{3}^{'}}{\left(R_{2}R_{3}\right)^{2}}
\equiv\frac{F\left(z\right)}{G\left(z\right)}
\end{eqnarray*}

where

\begin{eqnarray}
\begin{array}{l}
F\left(z\right)=PR_{2}R_{3}R_{1}^{'}
+R_{1}R_{2}R_{3}P^{'}-R_{3}R_{1}PR_{2}^{'}-R_{2}R_{1}PR_{3}^{'}\\
G\left(z\right)=R_{2}^{2}R_{3}^{2}\\
G^{2}\left(z\right)=R_{2}^{4}R_{3}^{4}=\left(1-\mu\right)^{4}\mu^{4}
\end{array}
\end{eqnarray}
so 

\begin{eqnarray}
\begin{array}{l}
G^{'}\left(z\right)=2R_{2}R_{3}\left[R_{2}^{'}R_{3}+R_{2}R_{3}^{'}\right]\\
G^{'}\left(1\right)=-2\left(1-\mu\right)\mu\left[\left(-\frac{1}{2}\esp\left[X^{2}\right]+\frac{1}{2}\mu\right)\left(-\mu\right)+\left(1-\mu\right)\left(-\frac{1}{2}\esp\left[X^{2}\right]+\frac{1}{2}\mu\right)\right]\\
F^{'}\left(z\right)=\left[\left(R_{2}R_{3}\right)R_{1}^{''}
-\left(R_{1}R_{3}\right)R_{2}^{''}
-\left(R_{1}R_{2}\right)R_{3}^{''}
-2\left(R_{2}^{'}R_{3}^{'}\right)R_{1}\right]P
+2\left(R_{2}R_{3}\right)R_{1}^{'}P^{'}
+\left(R_{1}R_{2}R_{3}\right)P^{''}.
\end{array}
\end{eqnarray}

Now, let us calculate $F^{'}\left(z\right)G\left(z\right)+F\left(z\right)G^{'}\left(z\right)$:

\begin{eqnarray*}
&&F^{'}\left(z\right)G\left(z\right)+F\left(z\right)G^{'}\left(z\right)=
\left\{\left[\left(R_{2}R_{3}\right)R_{1}^{''}
-\left(R_{1}R_{3}\right)R_{2}^{''}
-\left(R_{1}R_{2}\right)R_{3}^{''}
-2\left(R_{2}^{'}R_{3}^{'}\right)R_{1}\right]P\right.\\
&&\left.+2\left(R_{2}R_{3}\right)R_{1}^{'}P^{'}
+\left(R_{1}R_{2}R_{3}\right)P^{''}\right\}R_{2}^{2}R_{3}^{2}
-\left\{\left[PR_{2}R_{3}R_{1}^{'}+R_{1}R_{2}R_{3}P^{'}
-R_{3}R_{1}PR_{2}^{'}\right.\right.\\
&&\left.\left.
-R_{2}R_{1}PR_{3}^{'}\right]\left[2R_{2}R_{3}\left(R_{2}^{'}R_{3}+R_{2}R_{3}^{'}\right)\right]\right\}
\end{eqnarray*}
now evaluate in $z=1$

\begin{eqnarray*}
F^{'}\left(1\right)G\left(1\right)&+&F\left(1\right)G^{'}\left(1\right)
=\left(1+R_{3}\right)^{3}R_{3}^{3}R_{1}^{''}-\left(1+R_{3}\right)^{2}R_{1}R_{3}^{3}R_{3}^{''}
-\left(1+R_{3}\right)^{3}R_{3}^{2}R_{1}R_{3}^{''}\\
&-&2\left(1+R_{3}\right)^{2}R_{3}^{2}
\left(R_{3}^{'}\right)^{2}\\
&+&2\left(1+R_{3}\right)^{3}R_{3}^{3}R_{1}^{'}P^{'}
+\left(1+R_{3}\right)^{3}R_{3}^{3}R_{1}P^{''}
-2\left(1+R_{3}\right)^{2}R_{3}^{2}\left(1+2R_{3}\right)R_{3}^{'}R_{1}^{'}\\
&-&2\left(1+R_{3}\right)^{2}R_{3}^{2}R_{1}R_{3}^{'}\left(1+2R_{3}\right)P^{'}
+2\left(1+R_{3}\right)\left(1+2R_{3}\right)R_{3}^{3}R_{1}\left(R_{3}^{'}\right)^{2}\\
&+&2\left(1+R_{3}\right)^{2}\left(1+2R_{3}\right)R_{1}R_{3}R_{3}^{'}\\
&=&-\left(1-\mu\right)^{3}\mu^{3}R_{1}^{''}-\left(1-\mu\right)^{2}\mu^{2}R_{1}\left(1-2\mu\right)R_{3}^{''}
-\left(1-\mu\right)^{3}\mu^{3}R_{1}P^{''}\\
&+&2\left(1-\mu\right)\mu^{2}\left[\left(1-2\mu\right)R_{1}-\left(1-\mu\right)\right]\left(R_{3}^{'}\right)^{2}
-2\left(1-\mu\right)^{2}\mu R_{1}\left(1-2\mu\right)R_{3}^{'}\\
&-&2\left(1-\mu\right)^{3}\mu^{4}R_{1}^{'}-2\mu\left(1-\mu\right)\left(1-2\mu\right)R_{3}^{'}R_{1}^{'}
-2\mu^{3}\left(1-\mu\right)^{2}\left(1-2\mu\right)R_{1}R_{1}^{'}
\end{eqnarray*}
therefore
\begin{eqnarray*}
\left[\frac{F\left(z\right)}{G\left(z\right)}\right]^{'}&=&\frac{1}{\mu^{3}\left(1-\mu\right)^{3}}\left\{
-\left(1-\mu\right)^{2}\mu^{2}R_{1}^{''}-\mu\left(1-\mu\right)\left(1-2\mu\right)R_{1}R_{3}^{''}
-\mu^{2}\left(1-\mu\right)^{2}R_{1}P^{''}\right.\\
&&\left.+2\mu\left[\left(1-2\mu\right)R_{1}-\left(1-\mu\right)\right]\left(R_{3}^{'}\right)^{2}
-2\left(1-\mu\right)\left(1-2\mu\right)R_{1}R_{3}^{'}-2\mu^{3}\left(1-\mu\right)^{2}R_{1}^{'}\right.\\
&&\left.-2\left(1-2\mu\right)R_{3}^{'}R_{1}^{'}-2\mu^{2}\left(1-\mu\right)\left(1-2\mu\right)R_{1}R_{1}^{'}\right\}
\end{eqnarray*}
recall that


\begin{eqnarray*}
R_{1}&=&-\esp L\\
R_{3}&=& -\mu\\
R_{1}^{'}&=&-\frac{1}{2}\esp L^{2}+\frac{1}{2}\esp L\\
R_{3}^{'}&=&-\frac{1}{2}\esp X^{2}+\frac{1}{2}\mu\\
R_{1}^{''}&=&-\frac{1}{3}\esp L^{3}+\esp L^{2}-\frac{2}{3}\esp L\\
R_{3}^{''}&=&-\frac{1}{3}\esp X^{3}+\esp X^{2}-\frac{2}{3}\mu\\
R_{1}R_{3}^{'}&=&\frac{1}{2}\esp X^{2}\esp L-\frac{1}{2}\esp X\esp L\\
R_{1}R_{1}^{'}&=&\frac{1}{2}\esp L^{2}\esp L+\frac{1}{2}\esp^{2}L\\
R_{3}^{'}R_{1}^{'}&=&\frac{1}{4}\esp X^{2}\esp L^{2}-\frac{1}{4}\esp X^{2}\esp L-\frac{1}{4}\esp L^{2}\esp X+\frac{1}{4}\esp X\esp L\\
R_{1}R_{3}^{''}&=&\frac{1}{6}\esp X^{3}\esp L^{2}-\frac{1}{6}\esp X^{3}\esp L-\frac{1}{2}\esp L^{2}\esp X^{2}+\frac{1}{2}\esp X^{2}\esp L+\frac{1}{3}\esp X\esp L^{2}-\frac{1}{3}\esp X\esp L\\
R_{1}P^{''}&=&-\esp X^{2}\esp L\\
\left(R_{3}^{'}\right)^{2}&=&\frac{1}{4}\esp^{2}X^{2}-\frac{1}{2}\esp X^{2}\esp X+\frac{1}{4}\esp^{2} X
\end{eqnarray*}









\newpage
%______________________________________________________________________
\section{Appendix A: General Case Calculations Exhaustive Policy}\label{Secc.Append.B}
%______________________________________________________________________

%_______________________________________________________________
%\subsection{Calculations}
%_______________________________________________________________


Remember the equations given in equations (\ref{Ec.Gral.Primer.Momento.Ind.Exh}) and (\ref{Eq.Gral.Second.Order.Exhaustive}) for the first and second order partial derivatives respectively. The first moments equations for the queue lengths as before for the times the server arrives to the queue to start attending are obtained solving the system given by $f_{1}\left(i\right)=D_{i}R_{2}+D_{i}\mathbf{F}_{2}+\indora_{i\geq3}D_{i}\mathbf{F}_{4}$, similar expressions of the queues for the rest give us the linear system



\begin{eqnarray*}
\begin{array}{ll}
f_{1}\left(1\right)=r_{2}\tilde{\mu}_{1}+\frac{\tilde{\mu}_{1}}{1-\tilde{\mu}_{2}}f_{2}\left(2\right)+f_{2}\left(1\right),&
f_{1}\left(2\right)=r_{2}\tilde{\mu}_{2},\\
f_{1}\left(3\right)=r_{2}\tilde{\mu}_{3}+\frac{\tilde{\mu}_{3}}{1-\tilde{\mu}_{2}}f_{2}\left(2\right)+F_{3,2}^{(1)}\left(1\right),&
f_{1}\left(4\right)=r_{2}\tilde{\mu}_{4}+\frac{\tilde{\mu}_{4}}{1-\tilde{\mu}_{2}}f_{2}\left(2\right)+F_{4,2}^{(1)}\left(1\right),\\
f_{2}\left(1\right)=r_{1}\tilde{\mu}_{1},&
f_{2}\left(2\right)=r_{1}\tilde{\mu}_{2}+\frac{\tilde{\mu}_{2}}{1-\tilde{\mu}_{1}}f_{1}\left(1\right)+f_{1}\left(2\right),\\
f_{2}\left(3\right)=r_{1}\tilde{\mu}_{3}+\frac{\tilde{\mu}_{3}}{1-\tilde{\mu}_{1}}f_{1}\left(1\right)+F_{3,1}^{(1)}\left(1\right),&
f_{2}\left(4\right)=r_{1}\tilde{\mu}_{4}+\frac{\tilde{\mu}_{4}}{1-\tilde{\mu}_{1}}f_{1}\left(1\right)+F_{4,1}^{(1)}\left(1\right),\\
f_{3}\left(1\right)=\tilde{r}_{4}\tilde{\mu}_{1}+\frac{\tilde{\mu}_{1}}{1-\tilde{\mu}_{4}}f_{4}\left(4\right)+F_{1,4}^{(1)}\left(1\right),&
f_{3}\left(2\right)=\tilde{r}_{4}\tilde{\mu}_{2}+\frac{\tilde{\mu}_{2}}{1-\tilde{\mu}_{4}}f_{4}\left(4\right)+F_{2,4}^{(1)}\left(1\right),\\
f_{3}\left(3\right)=\tilde{r}_{4}\tilde{\mu}_{3}+\frac{\tilde{\mu}_{3}}{1-\tilde{\mu}_{4}}f_{4}\left(4\right)+f_{4}\left(3\right),&
f_{3}\left(4\right)=\tilde{r}_{4}\tilde{\mu}_{4}\\
f_{4}\left(1\right)=\tilde{r}_{3}\tilde{\mu}_{1}+\frac{\tilde{\mu}_{1}}{1-\tilde{\mu}_{3}}f_{3}\left(3\right)+F_{1,3}^{(1)}\left(1\right),&
f_{4}\left(2\right)=\tilde{r}_{3}\mu_{2}+\frac{\tilde{\mu}_{2}}{1-\tilde{\mu}_{3}}f_{3}\left(3\right)+F_{2,3}^{(1)}\left(1\right),\\
f_{4}\left(3\right)=\tilde{r}_{3}\tilde{\mu}_{3},&
f_{4}\left(4\right)=\tilde{r}_{3}\tilde{\mu}_{4}+\frac{\tilde{\mu}_{4}}{1-\tilde{\mu}_{3}}f_{3}\left(3\right)+f_{3}\left(4\right),\\
\end{array}
\end{eqnarray*}

Then we have that if $\mu=\tilde{\mu}_{1}+\tilde{\mu}_{2}<1$, $\hat{\mu}=\tilde{\mu}_{3}+\tilde{\mu}_{4}<1$, $r=r_{1}+r_{2}$ and $\hat{r}=\tilde{r}_{3}+\tilde{r}_{4}$  the system's solution are obtained by direct calculations:

\begin{eqnarray*}
\begin{array}{ll}
f_{2}\left(1\right)=r_{1}\tilde{\mu}_{1},&
f_{1}\left(2\right)=r_{2}\tilde{\mu}_{2},\\
f_{3}\left(4\right)=r_{4}\tilde{\mu}_{4},&
f_{4}\left(3\right)=r_{3}\tilde{\mu}_{3},\\
f_{1}\left(1\right)=r\frac{\tilde{\mu}_{1}\left(1-\tilde{\mu}_{1}\right)}{1-\mu},&
f_{2}\left(2\right)=r\frac{\tilde{\mu}_{2}\left(1-\tilde{\mu}_{2}\right)}{1-\mu},\\
f_{1}\left(3\right)=\tilde{\mu}_{3}\left(r_{2}+\frac{r\tilde{\mu}_{2}}{1-\mu}\right)+F_{3,2}^{(1)}\left(1\right),&
f_{1}\left(4\right)=\tilde{\mu}_{4}\left(r_{2}+\frac{r\tilde{\mu}_{2}}{1-\mu}\right)+F_{4,2}^{(1)}\left(1\right),\\
f_{2}\left(3\right)=\tilde{\mu}_{3}\left(r_{1}+\frac{r\tilde{\mu}_{1}}{1-\tilde{\mu}}\right)+F_{3,1}^{(1)}\left(1\right),&
f_{2}\left(4\right)=\tilde{\mu}_{4}\left(r_{1}+\frac{r\tilde{\mu}_{1}}{1-\mu}\right)+F_{4,,1}^{(1)}\left(1\right),\\
f_{3}\left(1\right)=\tilde{\mu}_{1}\left(r_{4}+\frac{\hat{r}\tilde{\mu}_{4}}{1-\hat{\mu}}\right)+F_{1,4}^{(1)}\left(1\right),&
f_{3}\left(2\right)=\tilde{\mu}_{2}\left(r_{4}+\frac{\hat{r}\tilde{\mu}_{4}}{1-\hat{\mu}}\right)+F_{2,4}^{(1)}\left(1\right),\\
f_{3}\left(3\right)=\hat{r}\frac{\tilde{\mu}_{3}\left(1-\tilde{\mu}_{3}\right)}{1-\hat{\mu}},&
f_{4}\left(1\right)=\tilde{\mu}_{1}\left(r_{3}+\frac{\hat{r}\tilde{\mu}_{3}}{1-\hat{\mu}}\right)+F_{1,3}^{(1)}\left(1\right),\\
f_{4}\left(2\right)=\tilde{\mu}_{2}\left(r_{3}+\frac{\hat{r}\tilde{\mu}_{3}}{1-\hat{\mu}}\right)+F_{2,3}^{(1)}\left(1\right),&
f_{4}\left(4\right)=\hat{r}\frac{\tilde{\mu}_{4}\left(1-\tilde{\mu}_{4}\right)}{1-\hat{\mu}}.
\end{array}
\end{eqnarray*}

Now, developing the equations given in (\ref{Eq.Gral.Second.Order.Exhaustive}) we obtain for instance $f_{1}\left(1,1\right)=\left(\frac{\tilde{\mu}_{1}}{1-\tilde{\mu}_{2}}\right)^{2}f_{2}\left(2,2\right)
+2\frac{\tilde{\mu}_{1}}{1-\tilde{\mu}_{2}}f_{2}\left(2,1\right)
+f_{2}\left(1,1\right)
+\tilde{\mu}_{1}^{2}\left(R_{2}^{(2)}+f_{2}\left(2\right)\theta_{2}^{(2)}\right)
+\tilde{P}_{1}^{(2)}\left(\frac{f_{2}\left(2\right)}{1-\tilde{\mu}_{2}}+r_{2}\right)+2r_{2}\tilde{\mu}_{2}f_{2}\left(1\right)$; similar reasoning lead us the following general expressions

\begin{eqnarray}\label{Eq.Sdo.Orden.Exh.uno}
\begin{array}{l}
f_{1}\left(i,j\right)=\indora_{i=1}f_{2}\left(1,1\right)
+\left[\left(1-\indora_{i=j=3}\right)\indora_{i+j\leq6}\indora_{i\leq j}\frac{\mu_{j}}{1-\tilde{\mu}_{2}}
+\left(1-\indora_{i=j=3}\right)\indora_{i+j\leq6}\indora_{i>j}\frac{\mu_{i}}{1-\tilde{\mu}_{2}}\right.\\
\left.+\indora_{i=1}\frac{\mu_{i}}{1-\tilde{\mu}_{2}}\right]f_{2}\left(1,2\right)+\indora_{i,j\neq2}\left(\frac{1}{1-\tilde{\mu}_{2}}\right)^{2}\mu_{i}\mu_{j}f_{2}\left(2,2\right)
+\left[\indora_{i,j\neq2}\tilde{\theta}_{2}^{(2)}\tilde{\mu}_{i}\tilde{\mu}_{j}
+\indora_{i,j\neq2}\indora_{i=j}\frac{\tilde{P}_{i}^{(2)}}{1-\tilde{\mu}_{2}}\right.\\
\left.+\indora_{i,j\neq2}\indora_{i\neq j}\frac{\tilde{\mu}_{i}\tilde{\mu}_{j}}{1-\tilde{\mu}_{2}}\right]f_{2}\left(2\right)
+\left[r_{2}\tilde{\mu}_{i}
+\indora_{i\geq3}F_{i,2}^{(1)}\right]f_{2}\left(j\right)
+\left[r_{2}\tilde{\mu}_{j}
+\indora_{j\geq3}F_{j,2}^{(1)}\right]f_{2}\left(i\right)\\
+\left[R_{2}^{(2)}
+\indora_{i=j}r_{2}\right]\tilde{\mu}_{i}\mu_{j}+\indora_{j\geq3}F_{j,2}^{(1)}\left[\indora_{j\neq i}F_{i,2}^{(1)}
+r_{2}\tilde{\mu}_{i}\right]
+r_{2}\left[\indora_{i=j}P_{i}^{(2)}
+\indora_{i\geq3}F_{i,2}^{(1)}\tilde{\mu}_{j}\right]\\
+\indora_{i\geq3}\indora_{j=i}F_{i,2}^{(2)}
\end{array}
\end{eqnarray}

in a similar manner we obtain expressions for $f_{2}\left(i,j\right)$, $f_{3}\left(i,j\right)$ and $f_{4}\left(i,j\right)$

for $i,k=1,2,3,4$; from which we obtain the linear equations system
\begin{eqnarray}\label{System.Second.Order.Moments.uno}
\begin{array}{ll}
f_{1}\left(1,1\right)=a_{1}f_{2}\left(2,2\right)
+a_{2}f_{2}\left(2,1\right)
+a_{3}f_{2}\left(1,1\right)
+K_{1},&
f_{1}\left(1,2\right)=K_{2},\\
f_{1}\left(1,3\right)=a_{4}f_{2}\left(2,2\right)+a_{5}f\left(2,1\right)+K_{3},&
f_{1}\left(1,4\right)=a_{6}f_{2}\left(2,2\right)+a_{7}f_{2}\left(2,1\right)+K_{4},\end{array}
\end{eqnarray}
for the rest equations, similar reasoning lead us to a linear system equations whose solutions are described in corolary (\ref{Coro.Second.Order.Eqs}) with coefficients given by, we just show a few of them


%Which can be reduced to solve the system given in (\ref{System.Second.Order.Moments.uno}) and (\ref{System.Second.Order.Moments.dos}).

with values for $a_{i}$ and $K_{i}$  
%{\small{
\begin{eqnarray}\label{Coefficients.Ais.Exh.uno}
\begin{array}{llll}
a_{1}=\left(\frac{\tilde{\mu}_{1}}{1-\tilde{\mu}_{2}}\right)^{2},&
a_{2}=\frac{2\tilde{\mu}_{1}}{1-\tilde{\mu}_{2}},&
a_{3}=1,&
a_{4}=\left(\frac{1}{1-\tilde{\mu}_{2}}\right)^{2}\tilde{\mu}_{1}\tilde{\mu}_{3},\\
a_{5}=\frac{\tilde{\mu}_{3}}{1-\tilde{\mu}_{2}},&
a_{6}=\left(\frac{1}{1-\tilde{\mu}_{2}}\right)^{2}\tilde{\mu}_{1}\tilde{\mu}_{4},&
a_{7}=\frac{\tilde{\mu}_{4}}{1-\tilde{\mu}_{2}},&\\
\end{array}
\end{eqnarray}%}}





\begin{eqnarray}\label{Coefficients.kis.Exh.uno}
\begin{array}{l}
K_{1}=\tilde{\mu}_{1}^{2}\left(R_{2}^{(2)}+f_{2}\left(2\right)\theta_{2}^{(2)}\right)
+\tilde{P}_{1}^{(2)}\left(\frac{f_{2}\left(2\right)}{1-\tilde{\mu}_{2}}+r_{2}\right)
+2r_{2}\tilde{\mu}_{2}f_{2}\left(1\right),\\
K_{2}=\tilde{\mu}_{1}\tilde{\mu}_{2}\left[R_{2}^{(2)}
+r_{2}\right]
+r_{2}\left[\tilde{\mu}_{1}f_{2}\left(2\right)
+\tilde{\mu}_{2}f_{2}\left(1\right)\right],\\
K_{3}=\tilde{\mu}_{1}\tilde{\mu}_{3}\left[R_{2}^{(2)}+r_{2}+f_{2}\left(2\right)\left(\tilde{\theta}_{2}^{(2)}+\frac{1}{1-\tilde{\mu}_{2}}\right)\right]
+r_{2}\tilde{\mu}_{1}\left[F_{3,2}^{(1)}+f_{2}\left(1\right)\right]
+\left[r_{2}\tilde{\mu}_{3}+F_{3,2}^{(1)}\right]f_{2}\left(1\right),\\
K_{4}=\tilde{\mu}_{1}\tilde{\mu}_{4}\left[R_{2}^{(2)}
+r_{2}+f_{2}\left(2\right)\left(\tilde{\theta}_{2}^{(2)}
+\frac{1}{1-\tilde{\mu}_{2}}\right)\right]
+r_{2}\tilde{\mu}_{1}\left[f_{2}\left(4\right)+F_{4,2}^{(1)}\right]
+f_{2}\left(1\right)\left[r_{2}\tilde{\mu}_{4}+F_{4,2}^{(1)}\right],
\end{array}
\end{eqnarray}

\newpage
%______________________________________________________________________
\section{Appendix B: Stability Analysis for a NCPS}
%__________________________________________________________________
%
\begin{Teo}
Dada una Red de Sistemas de Visitas C\'iclicas (RSVC), conformada por dos Sistemas de Visitas C\'iclicas (SVC), donde cada uno de ellos consta de dos colas tipo $M/M/1$. Los dos sistemas est\'an comunicados entre s\'i por medio de la transferencia de usuarios entre las colas $Q_{1}\leftrightarrow Q_{3}$ y $Q_{2}\leftrightarrow Q_{4}$. Se definen los eventos para los procesos de arribos al tiempo $t$, $A_{j}\left(t\right)=\left\{0 \textrm{ arribos en }Q_{j}\textrm{ al tiempo }t\right\}$ para alg\'un tiempo $t\geq0$ y $Q_{j}$ la cola $j$-\'esima en la RSVC, para $j=1,2,3,4$.  Existe un intervalo $I\neq\emptyset$ tal que para $T^{*}\in I$, tal que $\prob\left\{A_{1}\left(T^{*}\right),A_{2}\left(Tt^{*}\right),
A_{3}\left(T^{*}\right),A_{4}\left(T^{*}\right)|T^{*}\in I\right\}>0$.
\end{Teo}



\begin{proof}
Sin p\'erdida de generalidad podemos considerar como base del an\'alisis a la cola $Q_{1}$ del primer sistema que conforma la RSVC.\medskip 

Sea $n\geq1$, ciclo en el primer sistema en el que se sabe que $L_{j}\left(\overline{\tau}_{1}\left(n\right)\right)=0$, pues la pol\'itica de servicio con que atienden los servidores es la exhaustiva. Como es sabido, para trasladarse a la siguiente cola, el servidor incurre en un tiempo de traslado $r_{1}\left(n\right)>0$, entonces tenemos que $\tau_{2}\left(n\right)=\overline{\tau}_{1}\left(n\right)+r_{1}\left(n\right)$.\medskip 


Definamos el intervalo $I_{1}\equiv\left[\overline{\tau}_{1}\left(n\right),\tau_{2}\left(n\right)\right]$ de longitud $\xi_{1}=r_{1}\left(n\right)$.

Dado que los tiempos entre arribo son exponenciales con tasa $\tilde{\mu}_{1}=\mu_{1}+\hat{\mu}_{1}$ ($\mu_{1}$ son los arribos a $Q_{1}$ por primera vez al sistema, mientras que $\hat{\mu}_{1}$ son los arribos de traslado procedentes de $Q_{3}$) se tiene que la probabilidad del evento $A_{1}\left(t\right)$ est\'a dada por 

\begin{equation}
\prob\left\{A_{1}\left(t\right)|t\in I_{1}\left(n\right)\right\}=e^{-\tilde{\mu}_{1}\xi_{1}\left(n\right)}.
\end{equation} 


Por otra parte, para la cola $Q_{2}$ el tiempo $\overline{\tau}_{2}\left(n-1\right)$ es tal que $L_{2}\left(\overline{\tau}_{2}\left(n-1\right)\right)=0$, es decir, es el tiempo en que la cola queda totalmente vac\'ia en el ciclo anterior a $n$. \medskip 


Entonces tenemos un sgundo intervalo $I_{2}\equiv\left[\overline{\tau}_{2}\left(n-1\right),\tau_{2}\left(n\right)\right]$. Por lo tanto la probabilidad del evento $A_{2}\left(t\right)$ tiene probabilidad dada por

\begin{eqnarray}
\prob\left\{A_{2}\left(t\right)|t\in I_{2}\left(n\right)\right\}=e^{-\tilde{\mu}_{2}\xi_{2}\left(n\right)},\\
\xi_{2}\left(n\right)=\tau_{2}\left(n\right)-\overline{\tau}_{2}\left(n-1\right)
\end{eqnarray}
%\end{equation} 

%donde $$.

Ahora, dado que $I_{1}\left(n\right)\subset I_{2}\left(n\right)$, se tiene que

\begin{eqnarray*}
\xi_{1}\left(n\right)\leq\xi_{2}\left(n\right)&\Leftrightarrow& -\xi_{1}\left(n\right)\geq-\xi_{2}\left(n\right)
\\
-\tilde{\mu}_{2}\xi_{1}\left(n\right)\geq-\tilde{\mu}_{2}\xi_{2}\left(n\right)&\Leftrightarrow&
e^{-\tilde{\mu}_{2}\xi_{1}\left(n\right)}\geq e^{-\tilde{\mu}_{2}\xi_{2}\left(n\right)}\\
\prob\left\{A_{2}\left(t\right)|t\in I_{1}\left(n\right)\right\}&\geq&
\prob\left\{A_{2}\left(t\right)|t\in I_{2}\left(n\right)\right\}.
\end{eqnarray*}


Entonces se tiene que
\small{
\begin{eqnarray*}
\prob\left\{A_{1}\left(t\right),A_{2}\left(t\right)|t\in I_{1}\left(n\right)\right\}&=&
\prob\left\{A_{1}\left(t\right)|t\in I_{1}\left(n\right)\right\}
\prob\left\{A_{2}\left(t\right)|t\in I_{1}\left(n\right)\right\}\\
&\geq&
\prob\left\{A_{1}\left(t\right)|t\in I_{1}\left(n\right)\right\}
\prob\left\{A_{2}\left(t\right)|t\in I_{2}\left(n\right)\right\}\\
&=&e^{-\tilde{\mu}_{1}\xi_{1}\left(n\right)}e^{-\tilde{\mu}_{2}\xi_{2}\left(n\right)}
=e^{-\left[\tilde{\mu}_{1}\xi_{1}\left(n\right)+\tilde{\mu}_{2}\xi_{2}\left(n\right)\right]}.
\end{eqnarray*}}


Es decir, 

\begin{equation}
\prob\left\{A_{1}\left(t\right),A_{2}\left(t\right)|t\in I_{1}\left(n\right)\right\}
=e^{-\left[\tilde{\mu}_{1}\xi_{1}\left(n\right)+\tilde{\mu}_{2}\xi_{2}
\left(n\right)\right]}>0.
\end{equation}
En lo que respecta a la relaci\'on entre los dos SVC que conforman la RSVC para alg\'un $m\geq1$ se tiene que $\tau_{3}\left(m\right)<\tau_{2}\left(n\right)<\tau_{4}\left(m\right)$ por lo tanto se cumple cualquiera de los siguientes cuatro casos
\begin{itemize}
\item[a)] $\tau_{3}\left(m\right)<\tau_{2}\left(n\right)<\overline{\tau}_{3}\left(m\right)$

\item[b)] $\overline{\tau}_{3}\left(m\right)<\tau_{2}\left(n\right)
<\tau_{4}\left(m\right)$

\item[c)] $\tau_{4}\left(m\right)<\tau_{2}\left(n\right)<
\overline{\tau}_{4}\left(m\right)$

\item[d)] $\overline{\tau}_{4}\left(m\right)<\tau_{2}\left(n\right)
<\tau_{3}\left(m+1\right)$
\end{itemize}


Sea el intervalo $I_{3}\left(m\right)\equiv\left[\tau_{3}\left(m\right),\overline{\tau}_{3}\left(m\right)\right]$ tal que $\tau_{2}\left(n\right)\in I_{3}\left(m\right)$, con longitud de intervalo $\xi_{3}\equiv\overline{\tau}_{3}\left(m\right)-\tau_{3}\left(m\right)$, entonces se tiene que para $Q_{3}$
\begin{equation}
\prob\left\{A_{3}\left(t\right)|t\in I_{3}\left(m\right)\right\}=e^{-\tilde{\mu}_{3}\xi_{3}\left(m\right)}.
\end{equation} 

mientras que para $Q_{4}$ consideremos el intervalo $I_{4}\left(m\right)\equiv\left[\tau_{4}\left(m-1\right),\overline{\tau}_{3}\left(m\right)\right]$, entonces por construcci\'on  $I_{3}\left(m\right)\subset I_{4}\left(m\right)$, por lo tanto


\begin{eqnarray*}
\xi_{3}\left(m\right)\leq\xi_{4}\left(m\right)&\Leftrightarrow& -\xi_{3}\left(m\right)\geq-\xi_{4}\left(m\right)
\\
-\tilde{\mu}_{4}\xi_{3}\left(m\right)\geq-\tilde{\mu}_{4}\xi_{4}\left(m\right)&\Leftrightarrow&
e^{-\tilde{\mu}_{4}\xi_{3}\left(m\right)}\geq e^{-\tilde{\mu}_{4}\xi_{4}\left(n\right)}\\
\prob\left\{A_{4}\left(t\right)|t\in I_{3}\left(m\right)\right\}&\geq&
\prob\left\{A_{4}\left(t\right)|t\in I_{4}\left(m\right)\right\}.
\end{eqnarray*}



Entonces se tiene que
\small{
\begin{eqnarray*}
\prob\left\{A_{3}\left(t\right),A_{4}\left(t\right)|t\in I_{3}\left(m\right)\right\}&=&
\prob\left\{A_{3}\left(t\right)|t\in I_{3}\left(m\right)\right\}
\prob\left\{A_{4}\left(t\right)|t\in I_{3}\left(m\right)\right\}\\
&\geq&
\prob\left\{A_{3}\left(t\right)|t\in I_{3}\left(m\right)\right\}
\prob\left\{A_{4}\left(t\right)|t\in I_{4}\left(m\right)\right\}\\
&=&e^{-\tilde{\mu}_{3}\xi_{3}\left(m\right)}e^{-\tilde{\mu}_{4}\xi_{4}
\left(m\right)}
=e^{-\left(\tilde{\mu}_{3}\xi_{3}\left(m\right)+\tilde{\mu}_{4}\xi_{4}\left(m\right)\right)}.
\end{eqnarray*}}

Es decir, 

\begin{equation}
\prob\left\{A_{3}\left(t\right),A_{4}\left(t\right)|t\in I_{3}\left(m\right)\right\}\geq
e^{-\left(\tilde{\mu}_{3}\xi_{3}\left(m\right)+\tilde{\mu}_{4}\xi_{4}\left(m\right)\right)}>0.
\end{equation}


Sea el intervalo $I_{3}\left(m\right)\equiv\left[\overline{\tau}_{3}\left(m\right),\tau_{4}\left(m\right)\right]$ con longitud $\xi_{3}\equiv\tau_{4}\left(m\right)-\overline{\tau}_{3}\left(m\right)$, entonces se tiene que para $Q_{3}$
\begin{equation}
\prob\left\{A_{3}\left(t\right)|t\in I_{3}\left(m\right)\right\}=e^{-\tilde{\mu}_{3}\xi_{3}\left(m\right)}.
\end{equation} 

mientras que para $Q_{4}$ consideremos el intervalo $I_{4}\left(m\right)\equiv\left[\overline{\tau}_{4}\left(m-1\right),\tau_{4}\left(m\right)\right]$, entonces por construcci\'on  $I_{3}\left(m\right)\subset I_{4}\left(m\right)$, y al igual que en el caso anterior se tiene que 

\begin{eqnarray*}
\xi_{3}\left(m\right)\leq\xi_{4}\left(m\right)&\Leftrightarrow& -\xi_{3}\left(m\right)\geq-\xi_{4}\left(m\right)
\\
-\tilde{\mu}_{4}\xi_{3}\left(m\right)\geq-\tilde{\mu}_{4}\xi_{4}\left(m\right)&\Leftrightarrow&
e^{-\tilde{\mu}_{4}\xi_{3}\left(m\right)}\geq e^{-\tilde{\mu}_{4}\xi_{4}\left(n\right)}\\
\prob\left\{A_{4}\left(t\right)|t\in I_{3}\left(m\right)\right\}&\geq&
\prob\left\{A_{4}\left(t\right)|t\in I_{4}\left(m\right)\right\}.
\end{eqnarray*}


Entonces se tiene que
\small{
\begin{eqnarray*}
\prob\left\{A_{3}\left(t\right),A_{4}\left(t\right)|t\in I_{3}\left(m\right)\right\}&=&
\prob\left\{A_{3}\left(t\right)|t\in I_{3}\left(m\right)\right\}
\prob\left\{A_{4}\left(t\right)|t\in I_{3}\left(m\right)\right\}\\
&\geq&
\prob\left\{A_{3}\left(t\right)|t\in I_{3}\left(m\right)\right\}
\prob\left\{A_{4}\left(t\right)|t\in I_{4}\left(m\right)\right\}\\
&=&e^{-\tilde{\mu}_{3}\xi_{3}\left(m\right)}e^{-\tilde{\mu}_{4}\xi_{4}\left(m\right)}
=e^{-\left(\tilde{\mu}_{3}\xi_{3}\left(m\right)+\tilde{\mu}_{4}\xi_{4}\left(m\right)\right)}.
\end{eqnarray*}}

Es decir, 

\begin{equation}
\prob\left\{A_{3}\left(t\right),A_{4}\left(t\right)|t\in I_{4}\left(m\right)\right\}\geq
e^{-\left(\tilde{\mu}_{3}+\tilde{\mu}_{4}\right)\xi_{3}\left(m\right)}>0.
\end{equation}


Para el intervalo $I_{3}\left(m\right)=\left[\tau_{4}\left(m\right),\overline{\tau}_{4}\left(m\right)\right]$, se tiene que este caso es an\'alogo al caso (a).


Para el intevalo $I_{3}\left(m\right)\equiv\left[\overline{\tau}_{4}\left(m\right),\tau_{4}\left(m+1\right)\right]$, se tiene que es an\'alogo al caso (b).


Por construcci\'on se tiene que $I\left(n,m\right)\equiv I_{1}\left(n\right)\cap I_{3}\left(m\right)\neq\emptyset$,entonces en particular se tienen las contenciones $I\left(n,m\right)\subseteq I_{1}\left(n\right)$ y $I\left(n,m\right)\subseteq I_{3}\left(m\right)$, por lo tanto si definimos $\xi_{n,m}\equiv\ell\left(I\left(n,m\right)\right)$ tenemos que

\begin{eqnarray*}
\xi_{n,m}\leq\xi_{1}\left(n\right)\textrm{ y }\xi_{n,m}\leq\xi_{3}\left(m\right)\textrm{ entonces }\\
-\xi_{n,m}\geq-\xi_{1}\left(n\right)\textrm{ y }-\xi_{n,m}\leq-\xi_{3}\left(m\right)\\
\end{eqnarray*}
por lo tanto tenemos las desigualdades 


\begin{eqnarray*}
\begin{array}{ll}
-\tilde{\mu}_{1}\xi_{n,m}\geq-\tilde{\mu}_{1}\xi_{1}\left(n\right),&
-\tilde{\mu}_{2}\xi_{n,m}\geq-\tilde{\mu}_{2}\xi_{1}\left(n\right)
\geq-\tilde{\mu}_{2}\xi_{2}\left(n\right),\\
-\tilde{\mu}_{3}\xi_{n,m}\geq-\tilde{\mu}_{3}\xi_{3}\left(m\right),&
-\tilde{\mu}_{4}\xi_{n,m}\geq-\tilde{\mu}_{4}\xi_{3}\left(m\right)
\geq-\tilde{\mu}_{4}\xi_{4}\left(m\right).
\end{array}
\end{eqnarray*}

Sea $T^{*}\in I\left(n,m\right)$, entonces dado que en particular $T^{*}\in I_{1}\left(n\right)$, se cumple con probabilidad positiva que no hay arribos a las colas $Q_{1}$ y $Q_{2}$, en consecuencia, tampoco hay usuarios de transferencia para $Q_{3}$ y $Q_{4}$, es decir, $\tilde{\mu}_{1}=\mu_{1}$, $\tilde{\mu}_{2}=\mu_{2}$, $\tilde{\mu}_{3}=\mu_{3}$, $\tilde{\mu}_{4}=\mu_{4}$, es decir, los eventos $Q_{1}$ y $Q_{3}$ son condicionalmente independientes en el intervalo $I\left(n,m\right)$; lo mismo ocurre para las colas $Q_{2}$ y $Q_{4}$, por lo tanto tenemos que
%\small{
\begin{eqnarray}
\begin{array}{l}
\prob\left\{A_{1}\left(T^{*}\right),A_{2}\left(T^{*}\right),
A_{3}\left(T^{*}\right),A_{4}\left(T^{*}\right)|T^{*}\in I\left(n,m\right)\right\}\\
=\prod_{j=1}^{4}\prob\left\{A_{j}\left(T^{*}\right)|T^{*}\in I\left(n,m\right)\right\}\\
\geq\prob\left\{A_{1}\left(T^{*}\right)|T^{*}\in I_{1}\left(n\right)\right\}
\prob\left\{A_{2}\left(T^{*}\right)|T^{*}\in I_{2}\left(n\right)\right\}\\
\prob\left\{A_{3}\left(T^{*}\right)|T^{*}\in I_{3}\left(m\right)\right\}
\prob\left\{A_{4}\left(T^{*}\right)|T^{*}\in I_{4}\left(m\right)\right\}\\
=e^{-\mu_{1}\xi_{1}\left(n\right)}
e^{-\mu_{2}\xi_{2}\left(n\right)}
e^{-\mu_{3}\xi_{3}\left(m\right)}
e^{-\mu_{4}\xi_{4}\left(m\right)}\\
=e^{-\left[\tilde{\mu}_{1}\xi_{1}\left(n\right)
+\tilde{\mu}_{2}\xi_{2}\left(n\right)
+\tilde{\mu}_{3}\xi_{3}\left(m\right)
+\tilde{\mu}_{4}\xi_{4}
\left(m\right)\right]}>0.
\end{array}
\end{eqnarray}


Ahora solo resta demostrar que para $n\ge1$, existe $m\geq1$ tal que se cumplen cualquiera de los cuatro casos arriba mencionados: 

\begin{itemize}
\item[a)] $\tau_{3}\left(m\right)<\tau_{2}\left(n\right)<\overline{\tau}_{3}\left(m\right)$

\item[b)] $\overline{\tau}_{3}\left(m\right)<\tau_{2}\left(n\right)
<\tau_{4}\left(m\right)$

\item[c)] $\tau_{4}\left(m\right)<\tau_{2}\left(n\right)<
\overline{\tau}_{4}\left(m\right)$

\item[d)] $\overline{\tau}_{4}\left(m\right)<\tau_{2}\left(n\right)
<\tau_{3}\left(m+1\right)$
\end{itemize}

Consideremos nuevamente el primer caso. Supongamos que no existe $m\geq1$, tal que $I_{1}\left(n\right)\cap I_{3}\left(m\right)\neq\emptyset$, es decir, para toda $m\geq1$, $I_{1}\left(n\right)\cap I_{3}\left(m\right)=\emptyset$, entonces se tiene que ocurren cualquiera de los dos casos

\begin{itemize}
\item[a)] $\tau_{2}\left(n\right)\leq\tau_{3}\left(m\right)$: Recordemos que $\tau_{2}\left(m\right)=\overline{\tau}_{1}+r_{1}\left(m\right)$ donde cada una de las variables aleatorias son tales que $\esp\left[\overline{\tau}_{1}\left(n\right)-\tau_{1}\left(n\right)\right]<\infty$, $\esp\left[R_{1}\right]<\infty$ y $\esp\left[\tau_{3}\left(m\right)\right]<\infty$, lo cual contradice el hecho de que no exista un ciclo $m\geq1$ que satisfaga la condici\'on deseada.

\item[b)] $\tau_{2}\left(n\right)\geq\overline{\tau}_{3}\left(m\right)$: por un argumento similar al anterior se tiene que no es posible que no exista un ciclo $m\geq1$ tal que satisaface la condici\'on deseada.

\end{itemize}

Para el resto de los casos la demostraci\'on es an\'aloga. Por lo tanto, se tiene que efectivamente existe $m\geq1$ tal que $\tau_{3}\left(m\right)<\tau_{2}\left(n\right)<\tau_{4}\left(m\right)$.
\end{proof}
\newpage

%_________________________________________________________________________
%
\section{Appendix C: Output Process and Regenerative Processes}
%_________________________________________________________________________
%
En Sigman, Thorison y Wolff \cite{Sigman2} prueban que para la existencia de un una sucesi\'on infinita no decreciente de tiempos de regeneraci\'on $\tau_{1}\leq\tau_{2}\leq\cdots$ en los cuales el proceso se regenera, basta un tiempo de regeneraci\'on $R_{1}$, donde $R_{j}=\tau_{j}-\tau_{j-1}$. Para tal efecto se requiere la existencia de un espacio de probabilidad $\left(\Omega,\mathcal{F},\prob\right)$, y proceso estoc\'astico $\textit{X}=\left\{X\left(t\right):t\geq0\right\}$ con espacio de estados $\left(S,\mathcal{R}\right)$, con $\mathcal{R}$ $\sigma$-\'algebra.

\begin{Prop}
Si existe una variable aleatoria no negativa $R_{1}$ tal que $\theta_{R1}X=_{D}X$, entonces $\left(\Omega,\mathcal{F},\prob\right)$ puede extenderse para soportar una sucesi\'on estacionaria de variables aleatorias $R=\left\{R_{k}:k\geq1\right\}$, tal que para $k\geq1$,
\begin{eqnarray*}
\theta_{k}\left(X,R\right)=_{D}\left(X,R\right).
\end{eqnarray*}

Adem\'as, para $k\geq1$, $\theta_{k}R$ es condicionalmente independiente de $\left(X,R_{1},\ldots,R_{k}\right)$, dado $\theta_{\tau k}X$.

\end{Prop}


\begin{itemize}
\item Doob en 1953 demostr\'o que el estado estacionario de un proceso de partida en un sistema de espera $M/G/\infty$, es Poisson con la misma tasa que el proceso de arribos.

\item Burke en 1968, fue el primero en demostrar que el estado estacionario de un proceso de salida de una cola $M/M/s$ es un proceso Poisson.

\item Disney en 1973 obtuvo el siguiente resultado:

\begin{Teo}
Para el sistema de espera $M/G/1/L$ con disciplina FIFO, el proceso $\textbf{I}$ es un proceso de renovaci\'on si y s\'olo si el proceso denominado longitud de la cola es estacionario y se cumple cualquiera de los siguientes casos:

\begin{itemize}
\item[a)] Los tiempos de servicio son identicamente cero;
\item[b)] $L=0$, para cualquier proceso de servicio $S$;
\item[c)] $L=1$ y $G=D$;
\item[d)] $L=\infty$ y $G=M$.
\end{itemize}
En estos casos, respectivamente, las distribuciones de interpartida $P\left\{T_{n+1}-T_{n}\leq t\right\}$ son


\begin{itemize}
\item[a)] $1-e^{-\lambda t}$, $t\geq0$;
\item[b)] $1-e^{-\lambda t}*F\left(t\right)$, $t\geq0$;
\item[c)] $1-e^{-\lambda t}*\indora_{d}\left(t\right)$, $t\geq0$;
\item[d)] $1-e^{-\lambda t}*F\left(t\right)$, $t\geq0$.
\end{itemize}
\end{Teo}


\item Finch (1959) mostr\'o que para los sistemas $M/G/1/L$, con $1\leq L\leq \infty$ con distribuciones de servicio dos veces diferenciable, solamente el sistema $M/M/1/\infty$ tiene proceso de salida de renovaci\'on estacionario.

\item King (1971) demostro que un sistema de colas estacionario $M/G/1/1$ tiene sus tiempos de interpartida sucesivas $D_{n}$ y $D_{n+1}$ son independientes, si y s\'olo si, $G=D$, en cuyo caso le proceso de salida es de renovaci\'on.

\item Disney (1973) demostr\'o que el \'unico sistema estacionario $M/G/1/L$, que tiene proceso de salida de renovaci\'on  son los sistemas $M/M/1$ y $M/D/1/1$.



\item El siguiente resultado es de Disney y Koning (1985)
\begin{Teo}
En un sistema de espera $M/G/s$, el estado estacionario del proceso de salida es un proceso Poisson para cualquier distribuci\'on de los tiempos de servicio si el sistema tiene cualquiera de las siguientes cuatro propiedades.

\begin{itemize}
\item[a)] $s=\infty$
\item[b)] La disciplina de servicio es de procesador compartido.
\item[c)] La disciplina de servicio es LCFS y preemptive resume, esto se cumple para $L<\infty$
\item[d)] $G=M$.
\end{itemize}

\end{Teo}

\item El siguiente resultado es de Alamatsaz (1983)

\begin{Teo}
En cualquier sistema de colas $GI/G/1/L$ con $1\leq L<\infty$ y distribuci\'on de interarribos $A$ y distribuci\'on de los tiempos de servicio $B$, tal que $A\left(0\right)=0$, $A\left(t\right)\left(1-B\left(t\right)\right)>0$ para alguna $t>0$ y $B\left(t\right)$ para toda $t>0$, es imposible que el proceso de salida estacionario sea de renovaci\'on.
\end{Teo}

\end{itemize}



%________________________________________________________________________
%\subsection{Procesos Regenerativos Sigman, Thorisson y Wolff \cite{Sigman1}}
%________________________________________________________________________


\begin{Def}[Definici\'on Cl\'asica]
Un proceso estoc\'astico $X=\left\{X\left(t\right):t\geq0\right\}$ es llamado regenerativo is existe una variable aleatoria $R_{1}>0$ tal que
\begin{itemize}
\item[i)] $\left\{X\left(t+R_{1}\right):t\geq0\right\}$ es independiente de $\left\{\left\{X\left(t\right):t<R_{1}\right\},\right\}$
\item[ii)] $\left\{X\left(t+R_{1}\right):t\geq0\right\}$ es estoc\'asticamente equivalente a $\left\{X\left(t\right):t>0\right\}$
\end{itemize}

Llamamos a $R_{1}$ tiempo de regeneraci\'on, y decimos que $X$ se regenera en este punto.
\end{Def}

$\left\{X\left(t+R_{1}\right)\right\}$ es regenerativo con tiempo de regeneraci\'on $R_{2}$, independiente de $R_{1}$ pero con la misma distribuci\'on que $R_{1}$. Procediendo de esta manera se obtiene una secuencia de variables aleatorias independientes e id\'enticamente distribuidas $\left\{R_{n}\right\}$ llamados longitudes de ciclo. Si definimos a $Z_{k}\equiv R_{1}+R_{2}+\cdots+R_{k}$, se tiene un proceso de renovaci\'on llamado proceso de renovaci\'on encajado para $X$.


\begin{Note}
La existencia de un primer tiempo de regeneraci\'on, $R_{1}$, implica la existencia de una sucesi\'on completa de estos tiempos $R_{1},R_{2}\ldots,$ que satisfacen la propiedad deseada \cite{Sigman2}.
\end{Note}


\begin{Note} Para la cola $GI/GI/1$ los usuarios arriban con tiempos $t_{n}$ y son atendidos con tiempos de servicio $S_{n}$, los tiempos de arribo forman un proceso de renovaci\'on  con tiempos entre arribos independientes e identicamente distribuidos (\texttt{i.i.d.})$T_{n}=t_{n}-t_{n-1}$, adem\'as los tiempos de servicio son \texttt{i.i.d.} e independientes de los procesos de arribo. Por \textit{estable} se entiende que $\esp S_{n}<\esp T_{n}<\infty$.
\end{Note}
 


\begin{Def}
Para $x$ fijo y para cada $t\geq0$, sea $I_{x}\left(t\right)=1$ si $X\left(t\right)\leq x$,  $I_{x}\left(t\right)=0$ en caso contrario, y def\'inanse los tiempos promedio
\begin{eqnarray*}
\overline{X}&=&lim_{t\rightarrow\infty}\frac{1}{t}\int_{0}^{\infty}X\left(u\right)du\\
\prob\left(X_{\infty}\leq x\right)&=&lim_{t\rightarrow\infty}\frac{1}{t}\int_{0}^{\infty}I_{x}\left(u\right)du,
\end{eqnarray*}
cuando estos l\'imites existan.
\end{Def}

Como consecuencia del teorema de Renovaci\'on-Recompensa, se tiene que el primer l\'imite  existe y es igual a la constante
\begin{eqnarray*}
\overline{X}&=&\frac{\esp\left[\int_{0}^{R_{1}}X\left(t\right)dt\right]}{\esp\left[R_{1}\right]},
\end{eqnarray*}
suponiendo que ambas esperanzas son finitas.
 
\begin{Note}
Funciones de procesos regenerativos son regenerativas, es decir, si $X\left(t\right)$ es regenerativo y se define el proceso $Y\left(t\right)$ por $Y\left(t\right)=f\left(X\left(t\right)\right)$ para alguna funci\'on Borel medible $f\left(\cdot\right)$. Adem\'as $Y$ es regenerativo con los mismos tiempos de renovaci\'on que $X$. 

En general, los tiempos de renovaci\'on, $Z_{k}$ de un proceso regenerativo no requieren ser tiempos de paro con respecto a la evoluci\'on de $X\left(t\right)$.
\end{Note} 

\begin{Note}
Una funci\'on de un proceso de Markov, usualmente no ser\'a un proceso de Markov, sin embargo ser\'a regenerativo si el proceso de Markov lo es.
\end{Note}

 
\begin{Note}
Un proceso regenerativo con media de la longitud de ciclo finita es llamado positivo recurrente.
\end{Note}


\begin{Note}
\begin{itemize}
\item[a)] Si el proceso regenerativo $X$ es positivo recurrente y tiene trayectorias muestrales no negativas, entonces la ecuaci\'on anterior es v\'alida.
\item[b)] Si $X$ es positivo recurrente regenerativo, podemos construir una \'unica versi\'on estacionaria de este proceso, $X_{e}=\left\{X_{e}\left(t\right)\right\}$, donde $X_{e}$ es un proceso estoc\'astico regenerativo y estrictamente estacionario, con distribuci\'on marginal distribuida como $X_{\infty}$
\end{itemize}
\end{Note}


%__________________________________________________________________________________________
%\subsection{Procesos Regenerativos Estacionarios - Stidham \cite{Stidham}}
%__________________________________________________________________________________________


Un proceso estoc\'astico a tiempo continuo $\left\{V\left(t\right),t\geq0\right\}$ es un proceso regenerativo si existe una sucesi\'on de variables aleatorias independientes e id\'enticamente distribuidas $\left\{X_{1},X_{2},\ldots\right\}$, sucesi\'on de renovaci\'on, tal que para cualquier conjunto de Borel $A$, 

\begin{eqnarray*}
\prob\left\{V\left(t\right)\in A|X_{1}+X_{2}+\cdots+X_{R\left(t\right)}=s,\left\{V\left(\tau\right),\tau<s\right\}\right\}=\prob\left\{V\left(t-s\right)\in A|X_{1}>t-s\right\},
\end{eqnarray*}
para todo $0\leq s\leq t$, donde $R\left(t\right)=\max\left\{X_{1}+X_{2}+\cdots+X_{j}\leq t\right\}=$n\'umero de renovaciones ({\emph{puntos de regeneraci\'on}}) que ocurren en $\left[0,t\right]$. El intervalo $\left[0,X_{1}\right)$ es llamado {\emph{primer ciclo de regeneraci\'on}} de $\left\{V\left(t \right),t\geq0\right\}$, $\left[X_{1},X_{1}+X_{2}\right)$ el {\emph{segundo ciclo de regeneraci\'on}}, y as\'i sucesivamente.

Sea $X=X_{1}$ y sea $F$ la funci\'on de distrbuci\'on de $X$


\begin{Def}
Se define el proceso estacionario, $\left\{V^{*}\left(t\right),t\geq0\right\}$, para $\left\{V\left(t\right),t\geq0\right\}$ por

\begin{eqnarray*}
\prob\left\{V\left(t\right)\in A\right\}=\frac{1}{\esp\left[X\right]}\int_{0}^{\infty}\prob\left\{V\left(t+x\right)\in A|X>x\right\}\left(1-F\left(x\right)\right)dx,
\end{eqnarray*} 
para todo $t\geq0$ y todo conjunto de Borel $A$.
\end{Def}

\begin{Def}
Una distribuci\'on se dice que es {\emph{aritm\'etica}} si todos sus puntos de incremento son m\'ultiplos de la forma $0,\lambda, 2\lambda,\ldots$ para alguna $\lambda>0$ entera.
\end{Def}


\begin{Def}
Una modificaci\'on medible de un proceso $\left\{V\left(t\right),t\geq0\right\}$, es una versi\'on de este, $\left\{V\left(t,w\right)\right\}$ conjuntamente medible para $t\geq0$ y para $w\in S$, $S$ espacio de estados para $\left\{V\left(t\right),t\geq0\right\}$.
\end{Def}

\begin{Teo}
Sea $\left\{V\left(t\right),t\geq\right\}$ un proceso regenerativo no negativo con modificaci\'on medible. Sea $\esp\left[X\right]<\infty$. Entonces el proceso estacionario dado por la ecuaci\'on anterior est\'a bien definido y tiene funci\'on de distribuci\'on independiente de $t$, adem\'as
\begin{itemize}
\item[i)] \begin{eqnarray*}
\esp\left[V^{*}\left(0\right)\right]&=&\frac{\esp\left[\int_{0}^{X}V\left(s\right)ds\right]}{\esp\left[X\right]}\end{eqnarray*}
\item[ii)] Si $\esp\left[V^{*}\left(0\right)\right]<\infty$, equivalentemente, si $\esp\left[\int_{0}^{X}V\left(s\right)ds\right]<\infty$,entonces
\begin{eqnarray*}
\frac{\int_{0}^{t}V\left(s\right)ds}{t}\rightarrow\frac{\esp\left[\int_{0}^{X}V\left(s\right)ds\right]}{\esp\left[X\right]}
\end{eqnarray*}
con probabilidad 1 y en media, cuando $t\rightarrow\infty$.
\end{itemize}
\end{Teo}

\begin{Coro}
Sea $\left\{V\left(t\right),t\geq0\right\}$ un proceso regenerativo no negativo, con modificaci\'on medible. Si $\esp <\infty$, $F$ es no-aritm\'etica, y para todo $x\geq0$, $P\left\{V\left(t\right)\leq x,C>x\right\}$ es de variaci\'on acotada como funci\'on de $t$ en cada intervalo finito $\left[0,\tau\right]$, entonces $V\left(t\right)$ converge en distribuci\'on  cuando $t\rightarrow\infty$ y $$\esp V=\frac{\esp \int_{0}^{X}V\left(s\right)ds}{\esp X}$$
Donde $V$ tiene la distribuci\'on l\'imite de $V\left(t\right)$ cuando $t\rightarrow\infty$.

\end{Coro}

Para el caso discreto se tienen resultados similares.



%______________________________________________________________________
%\subsection{Procesos de Renovaci\'on}
%______________________________________________________________________

\begin{Def}%\label{Def.Tn}
Sean $0\leq T_{1}\leq T_{2}\leq \ldots$ son tiempos aleatorios infinitos en los cuales ocurren ciertos eventos. El n\'umero de tiempos $T_{n}$ en el intervalo $\left[0,t\right)$ es

\begin{eqnarray}
N\left(t\right)=\sum_{n=1}^{\infty}\indora\left(T_{n}\leq t\right),
\end{eqnarray}
para $t\geq0$.
\end{Def}

Si se consideran los puntos $T_{n}$ como elementos de $\rea_{+}$, y $N\left(t\right)$ es el n\'umero de puntos en $\rea$. El proceso denotado por $\left\{N\left(t\right):t\geq0\right\}$, denotado por $N\left(t\right)$, es un proceso puntual en $\rea_{+}$. Los $T_{n}$ son los tiempos de ocurrencia, el proceso puntual $N\left(t\right)$ es simple si su n\'umero de ocurrencias son distintas: $0<T_{1}<T_{2}<\ldots$ casi seguramente.

\begin{Def}
Un proceso puntual $N\left(t\right)$ es un proceso de renovaci\'on si los tiempos de interocurrencia $\xi_{n}=T_{n}-T_{n-1}$, para $n\geq1$, son independientes e identicamente distribuidos con distribuci\'on $F$, donde $F\left(0\right)=0$ y $T_{0}=0$. Los $T_{n}$ son llamados tiempos de renovaci\'on, referente a la independencia o renovaci\'on de la informaci\'on estoc\'astica en estos tiempos. Los $\xi_{n}$ son los tiempos de inter-renovaci\'on, y $N\left(t\right)$ es el n\'umero de renovaciones en el intervalo $\left[0,t\right)$
\end{Def}


\begin{Note}
Para definir un proceso de renovaci\'on para cualquier contexto, solamente hay que especificar una distribuci\'on $F$, con $F\left(0\right)=0$, para los tiempos de inter-renovaci\'on. La funci\'on $F$ en turno degune las otra variables aleatorias. De manera formal, existe un espacio de probabilidad y una sucesi\'on de variables aleatorias $\xi_{1},\xi_{2},\ldots$ definidas en este con distribuci\'on $F$. Entonces las otras cantidades son $T_{n}=\sum_{k=1}^{n}\xi_{k}$ y $N\left(t\right)=\sum_{n=1}^{\infty}\indora\left(T_{n}\leq t\right)$, donde $T_{n}\rightarrow\infty$ casi seguramente por la Ley Fuerte de los Grandes Números.
\end{Note}

%___________________________________________________________________________________________
%
%\subsection{Teorema Principal de Renovaci\'on}
%___________________________________________________________________________________________
%

\begin{Note} Una funci\'on $h:\rea_{+}\rightarrow\rea$ es Directamente Riemann Integrable en los siguientes casos:
\begin{itemize}
\item[a)] $h\left(t\right)\geq0$ es decreciente y Riemann Integrable.
\item[b)] $h$ es continua excepto posiblemente en un conjunto de Lebesgue de medida 0, y $|h\left(t\right)|\leq b\left(t\right)$, donde $b$ es DRI.
\end{itemize}
\end{Note}

\begin{Teo}[Teorema Principal de Renovaci\'on]
Si $F$ es no aritm\'etica y $h\left(t\right)$ es Directamente Riemann Integrable (DRI), entonces

\begin{eqnarray*}
lim_{t\rightarrow\infty}U\star h=\frac{1}{\mu}\int_{\rea_{+}}h\left(s\right)ds.
\end{eqnarray*}
\end{Teo}

\begin{Prop}
Cualquier funci\'on $H\left(t\right)$ acotada en intervalos finitos y que es 0 para $t<0$ puede expresarse como
\begin{eqnarray*}
H\left(t\right)=U\star h\left(t\right)\textrm{,  donde }h\left(t\right)=H\left(t\right)-F\star H\left(t\right)
\end{eqnarray*}
\end{Prop}

\begin{Def}
Un proceso estoc\'astico $X\left(t\right)$ es crudamente regenerativo en un tiempo aleatorio positivo $T$ si
\begin{eqnarray*}
\esp\left[X\left(T+t\right)|T\right]=\esp\left[X\left(t\right)\right]\textrm{, para }t\geq0,\end{eqnarray*}
y con las esperanzas anteriores finitas.
\end{Def}

\begin{Prop}
Sup\'ongase que $X\left(t\right)$ es un proceso crudamente regenerativo en $T$, que tiene distribuci\'on $F$. Si $\esp\left[X\left(t\right)\right]$ es acotado en intervalos finitos, entonces
\begin{eqnarray*}
\esp\left[X\left(t\right)\right]=U\star h\left(t\right)\textrm{,  donde }h\left(t\right)=\esp\left[X\left(t\right)\indora\left(T>t\right)\right].
\end{eqnarray*}
\end{Prop}

\begin{Teo}[Regeneraci\'on Cruda]
Sup\'ongase que $X\left(t\right)$ es un proceso con valores positivo crudamente regenerativo en $T$, y def\'inase $M=\sup\left\{|X\left(t\right)|:t\leq T\right\}$. Si $T$ es no aritm\'etico y $M$ y $MT$ tienen media finita, entonces
\begin{eqnarray*}
lim_{t\rightarrow\infty}\esp\left[X\left(t\right)\right]=\frac{1}{\mu}\int_{\rea_{+}}h\left(s\right)ds,
\end{eqnarray*}
donde $h\left(t\right)=\esp\left[X\left(t\right)\indora\left(T>t\right)\right]$.
\end{Teo}

%___________________________________________________________________________________________
%
%\subsection{Propiedades de los Procesos de Renovaci\'on}
%___________________________________________________________________________________________
%

Los tiempos $T_{n}$ est\'an relacionados con los conteos de $N\left(t\right)$ por

\begin{eqnarray*}
\left\{N\left(t\right)\geq n\right\}&=&\left\{T_{n}\leq t\right\}\\
T_{N\left(t\right)}\leq &t&<T_{N\left(t\right)+1},
\end{eqnarray*}

adem\'as $N\left(T_{n}\right)=n$, y 

\begin{eqnarray*}
N\left(t\right)=\max\left\{n:T_{n}\leq t\right\}=\min\left\{n:T_{n+1}>t\right\}
\end{eqnarray*}

Por propiedades de la convoluci\'on se sabe que

\begin{eqnarray*}
P\left\{T_{n}\leq t\right\}=F^{n\star}\left(t\right)
\end{eqnarray*}
que es la $n$-\'esima convoluci\'on de $F$. Entonces 

\begin{eqnarray*}
\left\{N\left(t\right)\geq n\right\}&=&\left\{T_{n}\leq t\right\}\\
P\left\{N\left(t\right)\leq n\right\}&=&1-F^{\left(n+1\right)\star}\left(t\right)
\end{eqnarray*}

Adem\'as usando el hecho de que $\esp\left[N\left(t\right)\right]=\sum_{n=1}^{\infty}P\left\{N\left(t\right)\geq n\right\}$
se tiene que

\begin{eqnarray*}
\esp\left[N\left(t\right)\right]=\sum_{n=1}^{\infty}F^{n\star}\left(t\right)
\end{eqnarray*}

\begin{Prop}
Para cada $t\geq0$, la funci\'on generadora de momentos $\esp\left[e^{\alpha N\left(t\right)}\right]$ existe para alguna $\alpha$ en una vecindad del 0, y de aqu\'i que $\esp\left[N\left(t\right)^{m}\right]<\infty$, para $m\geq1$.
\end{Prop}


\begin{Note}
Si el primer tiempo de renovaci\'on $\xi_{1}$ no tiene la misma distribuci\'on que el resto de las $\xi_{n}$, para $n\geq2$, a $N\left(t\right)$ se le llama Proceso de Renovaci\'on retardado, donde si $\xi$ tiene distribuci\'on $G$, entonces el tiempo $T_{n}$ de la $n$-\'esima renovaci\'on tiene distribuci\'on $G\star F^{\left(n-1\right)\star}\left(t\right)$
\end{Note}


\begin{Teo}
Para una constante $\mu\leq\infty$ ( o variable aleatoria), las siguientes expresiones son equivalentes:

\begin{eqnarray}
lim_{n\rightarrow\infty}n^{-1}T_{n}&=&\mu,\textrm{ c.s.}\\
lim_{t\rightarrow\infty}t^{-1}N\left(t\right)&=&1/\mu,\textrm{ c.s.}
\end{eqnarray}
\end{Teo}


Es decir, $T_{n}$ satisface la Ley Fuerte de los Grandes N\'umeros s\'i y s\'olo s\'i $N\left/t\right)$ la cumple.


\begin{Coro}[Ley Fuerte de los Grandes N\'umeros para Procesos de Renovaci\'on]
Si $N\left(t\right)$ es un proceso de renovaci\'on cuyos tiempos de inter-renovaci\'on tienen media $\mu\leq\infty$, entonces
\begin{eqnarray}
t^{-1}N\left(t\right)\rightarrow 1/\mu,\textrm{ c.s. cuando }t\rightarrow\infty.
\end{eqnarray}

\end{Coro}


Considerar el proceso estoc\'astico de valores reales $\left\{Z\left(t\right):t\geq0\right\}$ en el mismo espacio de probabilidad que $N\left(t\right)$

\begin{Def}
Para el proceso $\left\{Z\left(t\right):t\geq0\right\}$ se define la fluctuaci\'on m\'axima de $Z\left(t\right)$ en el intervalo $\left(T_{n-1},T_{n}\right]$:
\begin{eqnarray*}
M_{n}=\sup_{T_{n-1}<t\leq T_{n}}|Z\left(t\right)-Z\left(T_{n-1}\right)|
\end{eqnarray*}
\end{Def}

\begin{Teo}
Sup\'ongase que $n^{-1}T_{n}\rightarrow\mu$ c.s. cuando $n\rightarrow\infty$, donde $\mu\leq\infty$ es una constante o variable aleatoria. Sea $a$ una constante o variable aleatoria que puede ser infinita cuando $\mu$ es finita, y considere las expresiones l\'imite:
\begin{eqnarray}
lim_{n\rightarrow\infty}n^{-1}Z\left(T_{n}\right)&=&a,\textrm{ c.s.}\\
lim_{t\rightarrow\infty}t^{-1}Z\left(t\right)&=&a/\mu,\textrm{ c.s.}
\end{eqnarray}
La segunda expresi\'on implica la primera. Conversamente, la primera implica la segunda si el proceso $Z\left(t\right)$ es creciente, o si $lim_{n\rightarrow\infty}n^{-1}M_{n}=0$ c.s.
\end{Teo}

\begin{Coro}
Si $N\left(t\right)$ es un proceso de renovaci\'on, y $\left(Z\left(T_{n}\right)-Z\left(T_{n-1}\right),M_{n}\right)$, para $n\geq1$, son variables aleatorias independientes e id\'enticamente distribuidas con media finita, entonces,
\begin{eqnarray}
lim_{t\rightarrow\infty}t^{-1}Z\left(t\right)\rightarrow\frac{\esp\left[Z\left(T_{1}\right)-Z\left(T_{0}\right)\right]}{\esp\left[T_{1}\right]},\textrm{ c.s. cuando  }t\rightarrow\infty.
\end{eqnarray}
\end{Coro}



%___________________________________________________________________________________________
%
%\subsection{Propiedades de los Procesos de Renovaci\'on}
%___________________________________________________________________________________________
%

Los tiempos $T_{n}$ est\'an relacionados con los conteos de $N\left(t\right)$ por

\begin{eqnarray*}
\left\{N\left(t\right)\geq n\right\}&=&\left\{T_{n}\leq t\right\}\\
T_{N\left(t\right)}\leq &t&<T_{N\left(t\right)+1},
\end{eqnarray*}

adem\'as $N\left(T_{n}\right)=n$, y 

\begin{eqnarray*}
N\left(t\right)=\max\left\{n:T_{n}\leq t\right\}=\min\left\{n:T_{n+1}>t\right\}
\end{eqnarray*}

Por propiedades de la convoluci\'on se sabe que

\begin{eqnarray*}
P\left\{T_{n}\leq t\right\}=F^{n\star}\left(t\right)
\end{eqnarray*}
que es la $n$-\'esima convoluci\'on de $F$. Entonces 

\begin{eqnarray*}
\left\{N\left(t\right)\geq n\right\}&=&\left\{T_{n}\leq t\right\}\\
P\left\{N\left(t\right)\leq n\right\}&=&1-F^{\left(n+1\right)\star}\left(t\right)
\end{eqnarray*}

Adem\'as usando el hecho de que $\esp\left[N\left(t\right)\right]=\sum_{n=1}^{\infty}P\left\{N\left(t\right)\geq n\right\}$
se tiene que

\begin{eqnarray*}
\esp\left[N\left(t\right)\right]=\sum_{n=1}^{\infty}F^{n\star}\left(t\right)
\end{eqnarray*}

\begin{Prop}
Para cada $t\geq0$, la funci\'on generadora de momentos $\esp\left[e^{\alpha N\left(t\right)}\right]$ existe para alguna $\alpha$ en una vecindad del 0, y de aqu\'i que $\esp\left[N\left(t\right)^{m}\right]<\infty$, para $m\geq1$.
\end{Prop}


\begin{Note}
Si el primer tiempo de renovaci\'on $\xi_{1}$ no tiene la misma distribuci\'on que el resto de las $\xi_{n}$, para $n\geq2$, a $N\left(t\right)$ se le llama Proceso de Renovaci\'on retardado, donde si $\xi$ tiene distribuci\'on $G$, entonces el tiempo $T_{n}$ de la $n$-\'esima renovaci\'on tiene distribuci\'on $G\star F^{\left(n-1\right)\star}\left(t\right)$
\end{Note}


\begin{Teo}
Para una constante $\mu\leq\infty$ ( o variable aleatoria), las siguientes expresiones son equivalentes:

\begin{eqnarray}
lim_{n\rightarrow\infty}n^{-1}T_{n}&=&\mu,\textrm{ c.s.}\\
lim_{t\rightarrow\infty}t^{-1}N\left(t\right)&=&1/\mu,\textrm{ c.s.}
\end{eqnarray}
\end{Teo}


Es decir, $T_{n}$ satisface la Ley Fuerte de los Grandes N\'umeros s\'i y s\'olo s\'i $N\left/t\right)$ la cumple.


\begin{Coro}[Ley Fuerte de los Grandes N\'umeros para Procesos de Renovaci\'on]
Si $N\left(t\right)$ es un proceso de renovaci\'on cuyos tiempos de inter-renovaci\'on tienen media $\mu\leq\infty$, entonces
\begin{eqnarray}
t^{-1}N\left(t\right)\rightarrow 1/\mu,\textrm{ c.s. cuando }t\rightarrow\infty.
\end{eqnarray}

\end{Coro}


Considerar el proceso estoc\'astico de valores reales $\left\{Z\left(t\right):t\geq0\right\}$ en el mismo espacio de probabilidad que $N\left(t\right)$

\begin{Def}
Para el proceso $\left\{Z\left(t\right):t\geq0\right\}$ se define la fluctuaci\'on m\'axima de $Z\left(t\right)$ en el intervalo $\left(T_{n-1},T_{n}\right]$:
\begin{eqnarray*}
M_{n}=\sup_{T_{n-1}<t\leq T_{n}}|Z\left(t\right)-Z\left(T_{n-1}\right)|
\end{eqnarray*}
\end{Def}

\begin{Teo}
Sup\'ongase que $n^{-1}T_{n}\rightarrow\mu$ c.s. cuando $n\rightarrow\infty$, donde $\mu\leq\infty$ es una constante o variable aleatoria. Sea $a$ una constante o variable aleatoria que puede ser infinita cuando $\mu$ es finita, y considere las expresiones l\'imite:
\begin{eqnarray}
lim_{n\rightarrow\infty}n^{-1}Z\left(T_{n}\right)&=&a,\textrm{ c.s.}\\
lim_{t\rightarrow\infty}t^{-1}Z\left(t\right)&=&a/\mu,\textrm{ c.s.}
\end{eqnarray}
La segunda expresi\'on implica la primera. Conversamente, la primera implica la segunda si el proceso $Z\left(t\right)$ es creciente, o si $lim_{n\rightarrow\infty}n^{-1}M_{n}=0$ c.s.
\end{Teo}

\begin{Coro}
Si $N\left(t\right)$ es un proceso de renovaci\'on, y $\left(Z\left(T_{n}\right)-Z\left(T_{n-1}\right),M_{n}\right)$, para $n\geq1$, son variables aleatorias independientes e id\'enticamente distribuidas con media finita, entonces,
\begin{eqnarray}
lim_{t\rightarrow\infty}t^{-1}Z\left(t\right)\rightarrow\frac{\esp\left[Z\left(T_{1}\right)-Z\left(T_{0}\right)\right]}{\esp\left[T_{1}\right]},\textrm{ c.s. cuando  }t\rightarrow\infty.
\end{eqnarray}
\end{Coro}


%___________________________________________________________________________________________
%
%\subsection{Propiedades de los Procesos de Renovaci\'on}
%___________________________________________________________________________________________
%

Los tiempos $T_{n}$ est\'an relacionados con los conteos de $N\left(t\right)$ por

\begin{eqnarray*}
\left\{N\left(t\right)\geq n\right\}&=&\left\{T_{n}\leq t\right\}\\
T_{N\left(t\right)}\leq &t&<T_{N\left(t\right)+1},
\end{eqnarray*}

adem\'as $N\left(T_{n}\right)=n$, y 

\begin{eqnarray*}
N\left(t\right)=\max\left\{n:T_{n}\leq t\right\}=\min\left\{n:T_{n+1}>t\right\}
\end{eqnarray*}

Por propiedades de la convoluci\'on se sabe que

\begin{eqnarray*}
P\left\{T_{n}\leq t\right\}=F^{n\star}\left(t\right)
\end{eqnarray*}
que es la $n$-\'esima convoluci\'on de $F$. Entonces 

\begin{eqnarray*}
\left\{N\left(t\right)\geq n\right\}&=&\left\{T_{n}\leq t\right\}\\
P\left\{N\left(t\right)\leq n\right\}&=&1-F^{\left(n+1\right)\star}\left(t\right)
\end{eqnarray*}

Adem\'as usando el hecho de que $\esp\left[N\left(t\right)\right]=\sum_{n=1}^{\infty}P\left\{N\left(t\right)\geq n\right\}$
se tiene que

\begin{eqnarray*}
\esp\left[N\left(t\right)\right]=\sum_{n=1}^{\infty}F^{n\star}\left(t\right)
\end{eqnarray*}

\begin{Prop}
Para cada $t\geq0$, la funci\'on generadora de momentos $\esp\left[e^{\alpha N\left(t\right)}\right]$ existe para alguna $\alpha$ en una vecindad del 0, y de aqu\'i que $\esp\left[N\left(t\right)^{m}\right]<\infty$, para $m\geq1$.
\end{Prop}


\begin{Note}
Si el primer tiempo de renovaci\'on $\xi_{1}$ no tiene la misma distribuci\'on que el resto de las $\xi_{n}$, para $n\geq2$, a $N\left(t\right)$ se le llama Proceso de Renovaci\'on retardado, donde si $\xi$ tiene distribuci\'on $G$, entonces el tiempo $T_{n}$ de la $n$-\'esima renovaci\'on tiene distribuci\'on $G\star F^{\left(n-1\right)\star}\left(t\right)$
\end{Note}


\begin{Teo}
Para una constante $\mu\leq\infty$ ( o variable aleatoria), las siguientes expresiones son equivalentes:

\begin{eqnarray}
lim_{n\rightarrow\infty}n^{-1}T_{n}&=&\mu,\textrm{ c.s.}\\
lim_{t\rightarrow\infty}t^{-1}N\left(t\right)&=&1/\mu,\textrm{ c.s.}
\end{eqnarray}
\end{Teo}


Es decir, $T_{n}$ satisface la Ley Fuerte de los Grandes N\'umeros s\'i y s\'olo s\'i $N\left/t\right)$ la cumple.


\begin{Coro}[Ley Fuerte de los Grandes N\'umeros para Procesos de Renovaci\'on]
Si $N\left(t\right)$ es un proceso de renovaci\'on cuyos tiempos de inter-renovaci\'on tienen media $\mu\leq\infty$, entonces
\begin{eqnarray}
t^{-1}N\left(t\right)\rightarrow 1/\mu,\textrm{ c.s. cuando }t\rightarrow\infty.
\end{eqnarray}

\end{Coro}


Considerar el proceso estoc\'astico de valores reales $\left\{Z\left(t\right):t\geq0\right\}$ en el mismo espacio de probabilidad que $N\left(t\right)$

\begin{Def}
Para el proceso $\left\{Z\left(t\right):t\geq0\right\}$ se define la fluctuaci\'on m\'axima de $Z\left(t\right)$ en el intervalo $\left(T_{n-1},T_{n}\right]$:
\begin{eqnarray*}
M_{n}=\sup_{T_{n-1}<t\leq T_{n}}|Z\left(t\right)-Z\left(T_{n-1}\right)|
\end{eqnarray*}
\end{Def}

\begin{Teo}
Sup\'ongase que $n^{-1}T_{n}\rightarrow\mu$ c.s. cuando $n\rightarrow\infty$, donde $\mu\leq\infty$ es una constante o variable aleatoria. Sea $a$ una constante o variable aleatoria que puede ser infinita cuando $\mu$ es finita, y considere las expresiones l\'imite:
\begin{eqnarray}
lim_{n\rightarrow\infty}n^{-1}Z\left(T_{n}\right)&=&a,\textrm{ c.s.}\\
lim_{t\rightarrow\infty}t^{-1}Z\left(t\right)&=&a/\mu,\textrm{ c.s.}
\end{eqnarray}
La segunda expresi\'on implica la primera. Conversamente, la primera implica la segunda si el proceso $Z\left(t\right)$ es creciente, o si $lim_{n\rightarrow\infty}n^{-1}M_{n}=0$ c.s.
\end{Teo}

\begin{Coro}
Si $N\left(t\right)$ es un proceso de renovaci\'on, y $\left(Z\left(T_{n}\right)-Z\left(T_{n-1}\right),M_{n}\right)$, para $n\geq1$, son variables aleatorias independientes e id\'enticamente distribuidas con media finita, entonces,
\begin{eqnarray}
lim_{t\rightarrow\infty}t^{-1}Z\left(t\right)\rightarrow\frac{\esp\left[Z\left(T_{1}\right)-Z\left(T_{0}\right)\right]}{\esp\left[T_{1}\right]},\textrm{ c.s. cuando  }t\rightarrow\infty.
\end{eqnarray}
\end{Coro}

%___________________________________________________________________________________________
%
%\subsection{Propiedades de los Procesos de Renovaci\'on}
%___________________________________________________________________________________________
%

Los tiempos $T_{n}$ est\'an relacionados con los conteos de $N\left(t\right)$ por

\begin{eqnarray*}
\left\{N\left(t\right)\geq n\right\}&=&\left\{T_{n}\leq t\right\}\\
T_{N\left(t\right)}\leq &t&<T_{N\left(t\right)+1},
\end{eqnarray*}

adem\'as $N\left(T_{n}\right)=n$, y 

\begin{eqnarray*}
N\left(t\right)=\max\left\{n:T_{n}\leq t\right\}=\min\left\{n:T_{n+1}>t\right\}
\end{eqnarray*}

Por propiedades de la convoluci\'on se sabe que

\begin{eqnarray*}
P\left\{T_{n}\leq t\right\}=F^{n\star}\left(t\right)
\end{eqnarray*}
que es la $n$-\'esima convoluci\'on de $F$. Entonces 

\begin{eqnarray*}
\left\{N\left(t\right)\geq n\right\}&=&\left\{T_{n}\leq t\right\}\\
P\left\{N\left(t\right)\leq n\right\}&=&1-F^{\left(n+1\right)\star}\left(t\right)
\end{eqnarray*}

Adem\'as usando el hecho de que $\esp\left[N\left(t\right)\right]=\sum_{n=1}^{\infty}P\left\{N\left(t\right)\geq n\right\}$
se tiene que

\begin{eqnarray*}
\esp\left[N\left(t\right)\right]=\sum_{n=1}^{\infty}F^{n\star}\left(t\right)
\end{eqnarray*}

\begin{Prop}
Para cada $t\geq0$, la funci\'on generadora de momentos $\esp\left[e^{\alpha N\left(t\right)}\right]$ existe para alguna $\alpha$ en una vecindad del 0, y de aqu\'i que $\esp\left[N\left(t\right)^{m}\right]<\infty$, para $m\geq1$.
\end{Prop}


\begin{Note}
Si el primer tiempo de renovaci\'on $\xi_{1}$ no tiene la misma distribuci\'on que el resto de las $\xi_{n}$, para $n\geq2$, a $N\left(t\right)$ se le llama Proceso de Renovaci\'on retardado, donde si $\xi$ tiene distribuci\'on $G$, entonces el tiempo $T_{n}$ de la $n$-\'esima renovaci\'on tiene distribuci\'on $G\star F^{\left(n-1\right)\star}\left(t\right)$
\end{Note}


\begin{Teo}
Para una constante $\mu\leq\infty$ ( o variable aleatoria), las siguientes expresiones son equivalentes:

\begin{eqnarray}
lim_{n\rightarrow\infty}n^{-1}T_{n}&=&\mu,\textrm{ c.s.}\\
lim_{t\rightarrow\infty}t^{-1}N\left(t\right)&=&1/\mu,\textrm{ c.s.}
\end{eqnarray}
\end{Teo}


Es decir, $T_{n}$ satisface la Ley Fuerte de los Grandes N\'umeros s\'i y s\'olo s\'i $N\left/t\right)$ la cumple.


\begin{Coro}[Ley Fuerte de los Grandes N\'umeros para Procesos de Renovaci\'on]
Si $N\left(t\right)$ es un proceso de renovaci\'on cuyos tiempos de inter-renovaci\'on tienen media $\mu\leq\infty$, entonces
\begin{eqnarray}
t^{-1}N\left(t\right)\rightarrow 1/\mu,\textrm{ c.s. cuando }t\rightarrow\infty.
\end{eqnarray}

\end{Coro}


Considerar el proceso estoc\'astico de valores reales $\left\{Z\left(t\right):t\geq0\right\}$ en el mismo espacio de probabilidad que $N\left(t\right)$

\begin{Def}
Para el proceso $\left\{Z\left(t\right):t\geq0\right\}$ se define la fluctuaci\'on m\'axima de $Z\left(t\right)$ en el intervalo $\left(T_{n-1},T_{n}\right]$:
\begin{eqnarray*}
M_{n}=\sup_{T_{n-1}<t\leq T_{n}}|Z\left(t\right)-Z\left(T_{n-1}\right)|
\end{eqnarray*}
\end{Def}

\begin{Teo}
Sup\'ongase que $n^{-1}T_{n}\rightarrow\mu$ c.s. cuando $n\rightarrow\infty$, donde $\mu\leq\infty$ es una constante o variable aleatoria. Sea $a$ una constante o variable aleatoria que puede ser infinita cuando $\mu$ es finita, y considere las expresiones l\'imite:
\begin{eqnarray}
lim_{n\rightarrow\infty}n^{-1}Z\left(T_{n}\right)&=&a,\textrm{ c.s.}\\
lim_{t\rightarrow\infty}t^{-1}Z\left(t\right)&=&a/\mu,\textrm{ c.s.}
\end{eqnarray}
La segunda expresi\'on implica la primera. Conversamente, la primera implica la segunda si el proceso $Z\left(t\right)$ es creciente, o si $lim_{n\rightarrow\infty}n^{-1}M_{n}=0$ c.s.
\end{Teo}

\begin{Coro}
Si $N\left(t\right)$ es un proceso de renovaci\'on, y $\left(Z\left(T_{n}\right)-Z\left(T_{n-1}\right),M_{n}\right)$, para $n\geq1$, son variables aleatorias independientes e id\'enticamente distribuidas con media finita, entonces,
\begin{eqnarray}
lim_{t\rightarrow\infty}t^{-1}Z\left(t\right)\rightarrow\frac{\esp\left[Z\left(T_{1}\right)-Z\left(T_{0}\right)\right]}{\esp\left[T_{1}\right]},\textrm{ c.s. cuando  }t\rightarrow\infty.
\end{eqnarray}
\end{Coro}
%___________________________________________________________________________________________
%
%\subsection{Propiedades de los Procesos de Renovaci\'on}
%___________________________________________________________________________________________
%

Los tiempos $T_{n}$ est\'an relacionados con los conteos de $N\left(t\right)$ por

\begin{eqnarray*}
\left\{N\left(t\right)\geq n\right\}&=&\left\{T_{n}\leq t\right\}\\
T_{N\left(t\right)}\leq &t&<T_{N\left(t\right)+1},
\end{eqnarray*}

adem\'as $N\left(T_{n}\right)=n$, y 

\begin{eqnarray*}
N\left(t\right)=\max\left\{n:T_{n}\leq t\right\}=\min\left\{n:T_{n+1}>t\right\}
\end{eqnarray*}

Por propiedades de la convoluci\'on se sabe que

\begin{eqnarray*}
P\left\{T_{n}\leq t\right\}=F^{n\star}\left(t\right)
\end{eqnarray*}
que es la $n$-\'esima convoluci\'on de $F$. Entonces 

\begin{eqnarray*}
\left\{N\left(t\right)\geq n\right\}&=&\left\{T_{n}\leq t\right\}\\
P\left\{N\left(t\right)\leq n\right\}&=&1-F^{\left(n+1\right)\star}\left(t\right)
\end{eqnarray*}

Adem\'as usando el hecho de que $\esp\left[N\left(t\right)\right]=\sum_{n=1}^{\infty}P\left\{N\left(t\right)\geq n\right\}$
se tiene que

\begin{eqnarray*}
\esp\left[N\left(t\right)\right]=\sum_{n=1}^{\infty}F^{n\star}\left(t\right)
\end{eqnarray*}

\begin{Prop}
Para cada $t\geq0$, la funci\'on generadora de momentos $\esp\left[e^{\alpha N\left(t\right)}\right]$ existe para alguna $\alpha$ en una vecindad del 0, y de aqu\'i que $\esp\left[N\left(t\right)^{m}\right]<\infty$, para $m\geq1$.
\end{Prop}


\begin{Note}
Si el primer tiempo de renovaci\'on $\xi_{1}$ no tiene la misma distribuci\'on que el resto de las $\xi_{n}$, para $n\geq2$, a $N\left(t\right)$ se le llama Proceso de Renovaci\'on retardado, donde si $\xi$ tiene distribuci\'on $G$, entonces el tiempo $T_{n}$ de la $n$-\'esima renovaci\'on tiene distribuci\'on $G\star F^{\left(n-1\right)\star}\left(t\right)$
\end{Note}


\begin{Teo}
Para una constante $\mu\leq\infty$ ( o variable aleatoria), las siguientes expresiones son equivalentes:

\begin{eqnarray}
lim_{n\rightarrow\infty}n^{-1}T_{n}&=&\mu,\textrm{ c.s.}\\
lim_{t\rightarrow\infty}t^{-1}N\left(t\right)&=&1/\mu,\textrm{ c.s.}
\end{eqnarray}
\end{Teo}


Es decir, $T_{n}$ satisface la Ley Fuerte de los Grandes N\'umeros s\'i y s\'olo s\'i $N\left/t\right)$ la cumple.


\begin{Coro}[Ley Fuerte de los Grandes N\'umeros para Procesos de Renovaci\'on]
Si $N\left(t\right)$ es un proceso de renovaci\'on cuyos tiempos de inter-renovaci\'on tienen media $\mu\leq\infty$, entonces
\begin{eqnarray}
t^{-1}N\left(t\right)\rightarrow 1/\mu,\textrm{ c.s. cuando }t\rightarrow\infty.
\end{eqnarray}

\end{Coro}


Considerar el proceso estoc\'astico de valores reales $\left\{Z\left(t\right):t\geq0\right\}$ en el mismo espacio de probabilidad que $N\left(t\right)$

\begin{Def}
Para el proceso $\left\{Z\left(t\right):t\geq0\right\}$ se define la fluctuaci\'on m\'axima de $Z\left(t\right)$ en el intervalo $\left(T_{n-1},T_{n}\right]$:
\begin{eqnarray*}
M_{n}=\sup_{T_{n-1}<t\leq T_{n}}|Z\left(t\right)-Z\left(T_{n-1}\right)|
\end{eqnarray*}
\end{Def}

\begin{Teo}
Sup\'ongase que $n^{-1}T_{n}\rightarrow\mu$ c.s. cuando $n\rightarrow\infty$, donde $\mu\leq\infty$ es una constante o variable aleatoria. Sea $a$ una constante o variable aleatoria que puede ser infinita cuando $\mu$ es finita, y considere las expresiones l\'imite:
\begin{eqnarray}
lim_{n\rightarrow\infty}n^{-1}Z\left(T_{n}\right)&=&a,\textrm{ c.s.}\\
lim_{t\rightarrow\infty}t^{-1}Z\left(t\right)&=&a/\mu,\textrm{ c.s.}
\end{eqnarray}
La segunda expresi\'on implica la primera. Conversamente, la primera implica la segunda si el proceso $Z\left(t\right)$ es creciente, o si $lim_{n\rightarrow\infty}n^{-1}M_{n}=0$ c.s.
\end{Teo}

\begin{Coro}
Si $N\left(t\right)$ es un proceso de renovaci\'on, y $\left(Z\left(T_{n}\right)-Z\left(T_{n-1}\right),M_{n}\right)$, para $n\geq1$, son variables aleatorias independientes e id\'enticamente distribuidas con media finita, entonces,
\begin{eqnarray}
lim_{t\rightarrow\infty}t^{-1}Z\left(t\right)\rightarrow\frac{\esp\left[Z\left(T_{1}\right)-Z\left(T_{0}\right)\right]}{\esp\left[T_{1}\right]},\textrm{ c.s. cuando  }t\rightarrow\infty.
\end{eqnarray}
\end{Coro}


%___________________________________________________________________________________________
%
%\subsection{Funci\'on de Renovaci\'on}
%___________________________________________________________________________________________
%


\begin{Def}
Sea $h\left(t\right)$ funci\'on de valores reales en $\rea$ acotada en intervalos finitos e igual a cero para $t<0$ La ecuaci\'on de renovaci\'on para $h\left(t\right)$ y la distribuci\'on $F$ es

\begin{eqnarray}%\label{Ec.Renovacion}
H\left(t\right)=h\left(t\right)+\int_{\left[0,t\right]}H\left(t-s\right)dF\left(s\right)\textrm{,    }t\geq0,
\end{eqnarray}
donde $H\left(t\right)$ es una funci\'on de valores reales. Esto es $H=h+F\star H$. Decimos que $H\left(t\right)$ es soluci\'on de esta ecuaci\'on si satisface la ecuaci\'on, y es acotada en intervalos finitos e iguales a cero para $t<0$.
\end{Def}

\begin{Prop}
La funci\'on $U\star h\left(t\right)$ es la \'unica soluci\'on de la ecuaci\'on de renovaci\'on (\ref{Ec.Renovacion}).
\end{Prop}

\begin{Teo}[Teorema Renovaci\'on Elemental]
\begin{eqnarray*}
t^{-1}U\left(t\right)\rightarrow 1/\mu\textrm{,    cuando }t\rightarrow\infty.
\end{eqnarray*}
\end{Teo}

%___________________________________________________________________________________________
%
%\subsection{Funci\'on de Renovaci\'on}
%___________________________________________________________________________________________
%


Sup\'ongase que $N\left(t\right)$ es un proceso de renovaci\'on con distribuci\'on $F$ con media finita $\mu$.

\begin{Def}
La funci\'on de renovaci\'on asociada con la distribuci\'on $F$, del proceso $N\left(t\right)$, es
\begin{eqnarray*}
U\left(t\right)=\sum_{n=1}^{\infty}F^{n\star}\left(t\right),\textrm{   }t\geq0,
\end{eqnarray*}
donde $F^{0\star}\left(t\right)=\indora\left(t\geq0\right)$.
\end{Def}


\begin{Prop}
Sup\'ongase que la distribuci\'on de inter-renovaci\'on $F$ tiene densidad $f$. Entonces $U\left(t\right)$ tambi\'en tiene densidad, para $t>0$, y es $U^{'}\left(t\right)=\sum_{n=0}^{\infty}f^{n\star}\left(t\right)$. Adem\'as
\begin{eqnarray*}
\prob\left\{N\left(t\right)>N\left(t-\right)\right\}=0\textrm{,   }t\geq0.
\end{eqnarray*}
\end{Prop}

\begin{Def}
La Transformada de Laplace-Stieljes de $F$ est\'a dada por

\begin{eqnarray*}
\hat{F}\left(\alpha\right)=\int_{\rea_{+}}e^{-\alpha t}dF\left(t\right)\textrm{,  }\alpha\geq0.
\end{eqnarray*}
\end{Def}

Entonces

\begin{eqnarray*}
\hat{U}\left(\alpha\right)=\sum_{n=0}^{\infty}\hat{F^{n\star}}\left(\alpha\right)=\sum_{n=0}^{\infty}\hat{F}\left(\alpha\right)^{n}=\frac{1}{1-\hat{F}\left(\alpha\right)}.
\end{eqnarray*}


\begin{Prop}
La Transformada de Laplace $\hat{U}\left(\alpha\right)$ y $\hat{F}\left(\alpha\right)$ determina una a la otra de manera \'unica por la relaci\'on $\hat{U}\left(\alpha\right)=\frac{1}{1-\hat{F}\left(\alpha\right)}$.
\end{Prop}


\begin{Note}
Un proceso de renovaci\'on $N\left(t\right)$ cuyos tiempos de inter-renovaci\'on tienen media finita, es un proceso Poisson con tasa $\lambda$ si y s\'olo s\'i $\esp\left[U\left(t\right)\right]=\lambda t$, para $t\geq0$.
\end{Note}


\begin{Teo}
Sea $N\left(t\right)$ un proceso puntual simple con puntos de localizaci\'on $T_{n}$ tal que $\eta\left(t\right)=\esp\left[N\left(\right)\right]$ es finita para cada $t$. Entonces para cualquier funci\'on $f:\rea_{+}\rightarrow\rea$,
\begin{eqnarray*}
\esp\left[\sum_{n=1}^{N\left(\right)}f\left(T_{n}\right)\right]=\int_{\left(0,t\right]}f\left(s\right)d\eta\left(s\right)\textrm{,  }t\geq0,
\end{eqnarray*}
suponiendo que la integral exista. Adem\'as si $X_{1},X_{2},\ldots$ son variables aleatorias definidas en el mismo espacio de probabilidad que el proceso $N\left(t\right)$ tal que $\esp\left[X_{n}|T_{n}=s\right]=f\left(s\right)$, independiente de $n$. Entonces
\begin{eqnarray*}
\esp\left[\sum_{n=1}^{N\left(t\right)}X_{n}\right]=\int_{\left(0,t\right]}f\left(s\right)d\eta\left(s\right)\textrm{,  }t\geq0,
\end{eqnarray*} 
suponiendo que la integral exista. 
\end{Teo}

\begin{Coro}[Identidad de Wald para Renovaciones]
Para el proceso de renovaci\'on $N\left(t\right)$,
\begin{eqnarray*}
\esp\left[T_{N\left(t\right)+1}\right]=\mu\esp\left[N\left(t\right)+1\right]\textrm{,  }t\geq0,
\end{eqnarray*}  
\end{Coro}

%______________________________________________________________________
%\subsection{Procesos de Renovaci\'on}
%______________________________________________________________________

\begin{Def}%\label{Def.Tn}
Sean $0\leq T_{1}\leq T_{2}\leq \ldots$ son tiempos aleatorios infinitos en los cuales ocurren ciertos eventos. El n\'umero de tiempos $T_{n}$ en el intervalo $\left[0,t\right)$ es

\begin{eqnarray}
N\left(t\right)=\sum_{n=1}^{\infty}\indora\left(T_{n}\leq t\right),
\end{eqnarray}
para $t\geq0$.
\end{Def}

Si se consideran los puntos $T_{n}$ como elementos de $\rea_{+}$, y $N\left(t\right)$ es el n\'umero de puntos en $\rea$. El proceso denotado por $\left\{N\left(t\right):t\geq0\right\}$, denotado por $N\left(t\right)$, es un proceso puntual en $\rea_{+}$. Los $T_{n}$ son los tiempos de ocurrencia, el proceso puntual $N\left(t\right)$ es simple si su n\'umero de ocurrencias son distintas: $0<T_{1}<T_{2}<\ldots$ casi seguramente.

\begin{Def}
Un proceso puntual $N\left(t\right)$ es un proceso de renovaci\'on si los tiempos de interocurrencia $\xi_{n}=T_{n}-T_{n-1}$, para $n\geq1$, son independientes e identicamente distribuidos con distribuci\'on $F$, donde $F\left(0\right)=0$ y $T_{0}=0$. Los $T_{n}$ son llamados tiempos de renovaci\'on, referente a la independencia o renovaci\'on de la informaci\'on estoc\'astica en estos tiempos. Los $\xi_{n}$ son los tiempos de inter-renovaci\'on, y $N\left(t\right)$ es el n\'umero de renovaciones en el intervalo $\left[0,t\right)$
\end{Def}


\begin{Note}
Para definir un proceso de renovaci\'on para cualquier contexto, solamente hay que especificar una distribuci\'on $F$, con $F\left(0\right)=0$, para los tiempos de inter-renovaci\'on. La funci\'on $F$ en turno degune las otra variables aleatorias. De manera formal, existe un espacio de probabilidad y una sucesi\'on de variables aleatorias $\xi_{1},\xi_{2},\ldots$ definidas en este con distribuci\'on $F$. Entonces las otras cantidades son $T_{n}=\sum_{k=1}^{n}\xi_{k}$ y $N\left(t\right)=\sum_{n=1}^{\infty}\indora\left(T_{n}\leq t\right)$, donde $T_{n}\rightarrow\infty$ casi seguramente por la Ley Fuerte de los Grandes Números.
\end{Note}

%___________________________________________________________________________________________
%
%\subsection{Renewal and Regenerative Processes: Serfozo\cite{Serfozo}}
%___________________________________________________________________________________________
%
\begin{Def}%\label{Def.Tn}
Sean $0\leq T_{1}\leq T_{2}\leq \ldots$ son tiempos aleatorios infinitos en los cuales ocurren ciertos eventos. El n\'umero de tiempos $T_{n}$ en el intervalo $\left[0,t\right)$ es

\begin{eqnarray}
N\left(t\right)=\sum_{n=1}^{\infty}\indora\left(T_{n}\leq t\right),
\end{eqnarray}
para $t\geq0$.
\end{Def}

Si se consideran los puntos $T_{n}$ como elementos de $\rea_{+}$, y $N\left(t\right)$ es el n\'umero de puntos en $\rea$. El proceso denotado por $\left\{N\left(t\right):t\geq0\right\}$, denotado por $N\left(t\right)$, es un proceso puntual en $\rea_{+}$. Los $T_{n}$ son los tiempos de ocurrencia, el proceso puntual $N\left(t\right)$ es simple si su n\'umero de ocurrencias son distintas: $0<T_{1}<T_{2}<\ldots$ casi seguramente.

\begin{Def}
Un proceso puntual $N\left(t\right)$ es un proceso de renovaci\'on si los tiempos de interocurrencia $\xi_{n}=T_{n}-T_{n-1}$, para $n\geq1$, son independientes e identicamente distribuidos con distribuci\'on $F$, donde $F\left(0\right)=0$ y $T_{0}=0$. Los $T_{n}$ son llamados tiempos de renovaci\'on, referente a la independencia o renovaci\'on de la informaci\'on estoc\'astica en estos tiempos. Los $\xi_{n}$ son los tiempos de inter-renovaci\'on, y $N\left(t\right)$ es el n\'umero de renovaciones en el intervalo $\left[0,t\right)$
\end{Def}


\begin{Note}
Para definir un proceso de renovaci\'on para cualquier contexto, solamente hay que especificar una distribuci\'on $F$, con $F\left(0\right)=0$, para los tiempos de inter-renovaci\'on. La funci\'on $F$ en turno degune las otra variables aleatorias. De manera formal, existe un espacio de probabilidad y una sucesi\'on de variables aleatorias $\xi_{1},\xi_{2},\ldots$ definidas en este con distribuci\'on $F$. Entonces las otras cantidades son $T_{n}=\sum_{k=1}^{n}\xi_{k}$ y $N\left(t\right)=\sum_{n=1}^{\infty}\indora\left(T_{n}\leq t\right)$, donde $T_{n}\rightarrow\infty$ casi seguramente por la Ley Fuerte de los Grandes N\'umeros.
\end{Note}







Los tiempos $T_{n}$ est\'an relacionados con los conteos de $N\left(t\right)$ por

\begin{eqnarray*}
\left\{N\left(t\right)\geq n\right\}&=&\left\{T_{n}\leq t\right\}\\
T_{N\left(t\right)}\leq &t&<T_{N\left(t\right)+1},
\end{eqnarray*}

adem\'as $N\left(T_{n}\right)=n$, y 

\begin{eqnarray*}
N\left(t\right)=\max\left\{n:T_{n}\leq t\right\}=\min\left\{n:T_{n+1}>t\right\}
\end{eqnarray*}

Por propiedades de la convoluci\'on se sabe que

\begin{eqnarray*}
P\left\{T_{n}\leq t\right\}=F^{n\star}\left(t\right)
\end{eqnarray*}
que es la $n$-\'esima convoluci\'on de $F$. Entonces 

\begin{eqnarray*}
\left\{N\left(t\right)\geq n\right\}&=&\left\{T_{n}\leq t\right\}\\
P\left\{N\left(t\right)\leq n\right\}&=&1-F^{\left(n+1\right)\star}\left(t\right)
\end{eqnarray*}

Adem\'as usando el hecho de que $\esp\left[N\left(t\right)\right]=\sum_{n=1}^{\infty}P\left\{N\left(t\right)\geq n\right\}$
se tiene que

\begin{eqnarray*}
\esp\left[N\left(t\right)\right]=\sum_{n=1}^{\infty}F^{n\star}\left(t\right)
\end{eqnarray*}

\begin{Prop}
Para cada $t\geq0$, la funci\'on generadora de momentos $\esp\left[e^{\alpha N\left(t\right)}\right]$ existe para alguna $\alpha$ en una vecindad del 0, y de aqu\'i que $\esp\left[N\left(t\right)^{m}\right]<\infty$, para $m\geq1$.
\end{Prop}

\begin{Ejem}[\textbf{Proceso Poisson}]

Suponga que se tienen tiempos de inter-renovaci\'on \textit{i.i.d.} del proceso de renovaci\'on $N\left(t\right)$ tienen distribuci\'on exponencial $F\left(t\right)=q-e^{-\lambda t}$ con tasa $\lambda$. Entonces $N\left(t\right)$ es un proceso Poisson con tasa $\lambda$.

\end{Ejem}


\begin{Note}
Si el primer tiempo de renovaci\'on $\xi_{1}$ no tiene la misma distribuci\'on que el resto de las $\xi_{n}$, para $n\geq2$, a $N\left(t\right)$ se le llama Proceso de Renovaci\'on retardado, donde si $\xi$ tiene distribuci\'on $G$, entonces el tiempo $T_{n}$ de la $n$-\'esima renovaci\'on tiene distribuci\'on $G\star F^{\left(n-1\right)\star}\left(t\right)$
\end{Note}


\begin{Teo}
Para una constante $\mu\leq\infty$ ( o variable aleatoria), las siguientes expresiones son equivalentes:

\begin{eqnarray}
lim_{n\rightarrow\infty}n^{-1}T_{n}&=&\mu,\textrm{ c.s.}\\
lim_{t\rightarrow\infty}t^{-1}N\left(t\right)&=&1/\mu,\textrm{ c.s.}
\end{eqnarray}
\end{Teo}


Es decir, $T_{n}$ satisface la Ley Fuerte de los Grandes N\'umeros s\'i y s\'olo s\'i $N\left/t\right)$ la cumple.


\begin{Coro}[Ley Fuerte de los Grandes N\'umeros para Procesos de Renovaci\'on]
Si $N\left(t\right)$ es un proceso de renovaci\'on cuyos tiempos de inter-renovaci\'on tienen media $\mu\leq\infty$, entonces
\begin{eqnarray}
t^{-1}N\left(t\right)\rightarrow 1/\mu,\textrm{ c.s. cuando }t\rightarrow\infty.
\end{eqnarray}

\end{Coro}


Considerar el proceso estoc\'astico de valores reales $\left\{Z\left(t\right):t\geq0\right\}$ en el mismo espacio de probabilidad que $N\left(t\right)$

\begin{Def}
Para el proceso $\left\{Z\left(t\right):t\geq0\right\}$ se define la fluctuaci\'on m\'axima de $Z\left(t\right)$ en el intervalo $\left(T_{n-1},T_{n}\right]$:
\begin{eqnarray*}
M_{n}=\sup_{T_{n-1}<t\leq T_{n}}|Z\left(t\right)-Z\left(T_{n-1}\right)|
\end{eqnarray*}
\end{Def}

\begin{Teo}
Sup\'ongase que $n^{-1}T_{n}\rightarrow\mu$ c.s. cuando $n\rightarrow\infty$, donde $\mu\leq\infty$ es una constante o variable aleatoria. Sea $a$ una constante o variable aleatoria que puede ser infinita cuando $\mu$ es finita, y considere las expresiones l\'imite:
\begin{eqnarray}
lim_{n\rightarrow\infty}n^{-1}Z\left(T_{n}\right)&=&a,\textrm{ c.s.}\\
lim_{t\rightarrow\infty}t^{-1}Z\left(t\right)&=&a/\mu,\textrm{ c.s.}
\end{eqnarray}
La segunda expresi\'on implica la primera. Conversamente, la primera implica la segunda si el proceso $Z\left(t\right)$ es creciente, o si $lim_{n\rightarrow\infty}n^{-1}M_{n}=0$ c.s.
\end{Teo}

\begin{Coro}
Si $N\left(t\right)$ es un proceso de renovaci\'on, y $\left(Z\left(T_{n}\right)-Z\left(T_{n-1}\right),M_{n}\right)$, para $n\geq1$, son variables aleatorias independientes e id\'enticamente distribuidas con media finita, entonces,
\begin{eqnarray}
lim_{t\rightarrow\infty}t^{-1}Z\left(t\right)\rightarrow\frac{\esp\left[Z\left(T_{1}\right)-Z\left(T_{0}\right)\right]}{\esp\left[T_{1}\right]},\textrm{ c.s. cuando  }t\rightarrow\infty.
\end{eqnarray}
\end{Coro}


Sup\'ongase que $N\left(t\right)$ es un proceso de renovaci\'on con distribuci\'on $F$ con media finita $\mu$.

\begin{Def}
La funci\'on de renovaci\'on asociada con la distribuci\'on $F$, del proceso $N\left(t\right)$, es
\begin{eqnarray*}
U\left(t\right)=\sum_{n=1}^{\infty}F^{n\star}\left(t\right),\textrm{   }t\geq0,
\end{eqnarray*}
donde $F^{0\star}\left(t\right)=\indora\left(t\geq0\right)$.
\end{Def}


\begin{Prop}
Sup\'ongase que la distribuci\'on de inter-renovaci\'on $F$ tiene densidad $f$. Entonces $U\left(t\right)$ tambi\'en tiene densidad, para $t>0$, y es $U^{'}\left(t\right)=\sum_{n=0}^{\infty}f^{n\star}\left(t\right)$. Adem\'as
\begin{eqnarray*}
\prob\left\{N\left(t\right)>N\left(t-\right)\right\}=0\textrm{,   }t\geq0.
\end{eqnarray*}
\end{Prop}

\begin{Def}
La Transformada de Laplace-Stieljes de $F$ est\'a dada por

\begin{eqnarray*}
\hat{F}\left(\alpha\right)=\int_{\rea_{+}}e^{-\alpha t}dF\left(t\right)\textrm{,  }\alpha\geq0.
\end{eqnarray*}
\end{Def}

Entonces

\begin{eqnarray*}
\hat{U}\left(\alpha\right)=\sum_{n=0}^{\infty}\hat{F^{n\star}}\left(\alpha\right)=\sum_{n=0}^{\infty}\hat{F}\left(\alpha\right)^{n}=\frac{1}{1-\hat{F}\left(\alpha\right)}.
\end{eqnarray*}


\begin{Prop}
La Transformada de Laplace $\hat{U}\left(\alpha\right)$ y $\hat{F}\left(\alpha\right)$ determina una a la otra de manera \'unica por la relaci\'on $\hat{U}\left(\alpha\right)=\frac{1}{1-\hat{F}\left(\alpha\right)}$.
\end{Prop}


\begin{Note}
Un proceso de renovaci\'on $N\left(t\right)$ cuyos tiempos de inter-renovaci\'on tienen media finita, es un proceso Poisson con tasa $\lambda$ si y s\'olo s\'i $\esp\left[U\left(t\right)\right]=\lambda t$, para $t\geq0$.
\end{Note}


\begin{Teo}
Sea $N\left(t\right)$ un proceso puntual simple con puntos de localizaci\'on $T_{n}$ tal que $\eta\left(t\right)=\esp\left[N\left(\right)\right]$ es finita para cada $t$. Entonces para cualquier funci\'on $f:\rea_{+}\rightarrow\rea$,
\begin{eqnarray*}
\esp\left[\sum_{n=1}^{N\left(\right)}f\left(T_{n}\right)\right]=\int_{\left(0,t\right]}f\left(s\right)d\eta\left(s\right)\textrm{,  }t\geq0,
\end{eqnarray*}
suponiendo que la integral exista. Adem\'as si $X_{1},X_{2},\ldots$ son variables aleatorias definidas en el mismo espacio de probabilidad que el proceso $N\left(t\right)$ tal que $\esp\left[X_{n}|T_{n}=s\right]=f\left(s\right)$, independiente de $n$. Entonces
\begin{eqnarray*}
\esp\left[\sum_{n=1}^{N\left(t\right)}X_{n}\right]=\int_{\left(0,t\right]}f\left(s\right)d\eta\left(s\right)\textrm{,  }t\geq0,
\end{eqnarray*} 
suponiendo que la integral exista. 
\end{Teo}

\begin{Coro}[Identidad de Wald para Renovaciones]
Para el proceso de renovaci\'on $N\left(t\right)$,
\begin{eqnarray*}
\esp\left[T_{N\left(t\right)+1}\right]=\mu\esp\left[N\left(t\right)+1\right]\textrm{,  }t\geq0,
\end{eqnarray*}  
\end{Coro}


\begin{Def}
Sea $h\left(t\right)$ funci\'on de valores reales en $\rea$ acotada en intervalos finitos e igual a cero para $t<0$ La ecuaci\'on de renovaci\'on para $h\left(t\right)$ y la distribuci\'on $F$ es

\begin{eqnarray}%\label{Ec.Renovacion}
H\left(t\right)=h\left(t\right)+\int_{\left[0,t\right]}H\left(t-s\right)dF\left(s\right)\textrm{,    }t\geq0,
\end{eqnarray}
donde $H\left(t\right)$ es una funci\'on de valores reales. Esto es $H=h+F\star H$. Decimos que $H\left(t\right)$ es soluci\'on de esta ecuaci\'on si satisface la ecuaci\'on, y es acotada en intervalos finitos e iguales a cero para $t<0$.
\end{Def}

\begin{Prop}
La funci\'on $U\star h\left(t\right)$ es la \'unica soluci\'on de la ecuaci\'on de renovaci\'on (\ref{Ec.Renovacion}).
\end{Prop}

\begin{Teo}[Teorema Renovaci\'on Elemental]
\begin{eqnarray*}
t^{-1}U\left(t\right)\rightarrow 1/\mu\textrm{,    cuando }t\rightarrow\infty.
\end{eqnarray*}
\end{Teo}



Sup\'ongase que $N\left(t\right)$ es un proceso de renovaci\'on con distribuci\'on $F$ con media finita $\mu$.

\begin{Def}
La funci\'on de renovaci\'on asociada con la distribuci\'on $F$, del proceso $N\left(t\right)$, es
\begin{eqnarray*}
U\left(t\right)=\sum_{n=1}^{\infty}F^{n\star}\left(t\right),\textrm{   }t\geq0,
\end{eqnarray*}
donde $F^{0\star}\left(t\right)=\indora\left(t\geq0\right)$.
\end{Def}


\begin{Prop}
Sup\'ongase que la distribuci\'on de inter-renovaci\'on $F$ tiene densidad $f$. Entonces $U\left(t\right)$ tambi\'en tiene densidad, para $t>0$, y es $U^{'}\left(t\right)=\sum_{n=0}^{\infty}f^{n\star}\left(t\right)$. Adem\'as
\begin{eqnarray*}
\prob\left\{N\left(t\right)>N\left(t-\right)\right\}=0\textrm{,   }t\geq0.
\end{eqnarray*}
\end{Prop}

\begin{Def}
La Transformada de Laplace-Stieljes de $F$ est\'a dada por

\begin{eqnarray*}
\hat{F}\left(\alpha\right)=\int_{\rea_{+}}e^{-\alpha t}dF\left(t\right)\textrm{,  }\alpha\geq0.
\end{eqnarray*}
\end{Def}

Entonces

\begin{eqnarray*}
\hat{U}\left(\alpha\right)=\sum_{n=0}^{\infty}\hat{F^{n\star}}\left(\alpha\right)=\sum_{n=0}^{\infty}\hat{F}\left(\alpha\right)^{n}=\frac{1}{1-\hat{F}\left(\alpha\right)}.
\end{eqnarray*}


\begin{Prop}
La Transformada de Laplace $\hat{U}\left(\alpha\right)$ y $\hat{F}\left(\alpha\right)$ determina una a la otra de manera \'unica por la relaci\'on $\hat{U}\left(\alpha\right)=\frac{1}{1-\hat{F}\left(\alpha\right)}$.
\end{Prop}


\begin{Note}
Un proceso de renovaci\'on $N\left(t\right)$ cuyos tiempos de inter-renovaci\'on tienen media finita, es un proceso Poisson con tasa $\lambda$ si y s\'olo s\'i $\esp\left[U\left(t\right)\right]=\lambda t$, para $t\geq0$.
\end{Note}


\begin{Teo}
Sea $N\left(t\right)$ un proceso puntual simple con puntos de localizaci\'on $T_{n}$ tal que $\eta\left(t\right)=\esp\left[N\left(\right)\right]$ es finita para cada $t$. Entonces para cualquier funci\'on $f:\rea_{+}\rightarrow\rea$,
\begin{eqnarray*}
\esp\left[\sum_{n=1}^{N\left(\right)}f\left(T_{n}\right)\right]=\int_{\left(0,t\right]}f\left(s\right)d\eta\left(s\right)\textrm{,  }t\geq0,
\end{eqnarray*}
suponiendo que la integral exista. Adem\'as si $X_{1},X_{2},\ldots$ son variables aleatorias definidas en el mismo espacio de probabilidad que el proceso $N\left(t\right)$ tal que $\esp\left[X_{n}|T_{n}=s\right]=f\left(s\right)$, independiente de $n$. Entonces
\begin{eqnarray*}
\esp\left[\sum_{n=1}^{N\left(t\right)}X_{n}\right]=\int_{\left(0,t\right]}f\left(s\right)d\eta\left(s\right)\textrm{,  }t\geq0,
\end{eqnarray*} 
suponiendo que la integral exista. 
\end{Teo}

\begin{Coro}[Identidad de Wald para Renovaciones]
Para el proceso de renovaci\'on $N\left(t\right)$,
\begin{eqnarray*}
\esp\left[T_{N\left(t\right)+1}\right]=\mu\esp\left[N\left(t\right)+1\right]\textrm{,  }t\geq0,
\end{eqnarray*}  
\end{Coro}


\begin{Def}
Sea $h\left(t\right)$ funci\'on de valores reales en $\rea$ acotada en intervalos finitos e igual a cero para $t<0$ La ecuaci\'on de renovaci\'on para $h\left(t\right)$ y la distribuci\'on $F$ es

\begin{eqnarray}%\label{Ec.Renovacion}
H\left(t\right)=h\left(t\right)+\int_{\left[0,t\right]}H\left(t-s\right)dF\left(s\right)\textrm{,    }t\geq0,
\end{eqnarray}
donde $H\left(t\right)$ es una funci\'on de valores reales. Esto es $H=h+F\star H$. Decimos que $H\left(t\right)$ es soluci\'on de esta ecuaci\'on si satisface la ecuaci\'on, y es acotada en intervalos finitos e iguales a cero para $t<0$.
\end{Def}

\begin{Prop}
La funci\'on $U\star h\left(t\right)$ es la \'unica soluci\'on de la ecuaci\'on de renovaci\'on (\ref{Ec.Renovacion}).
\end{Prop}

\begin{Teo}[Teorema Renovaci\'on Elemental]
\begin{eqnarray*}
t^{-1}U\left(t\right)\rightarrow 1/\mu\textrm{,    cuando }t\rightarrow\infty.
\end{eqnarray*}
\end{Teo}


\begin{Note} Una funci\'on $h:\rea_{+}\rightarrow\rea$ es Directamente Riemann Integrable en los siguientes casos:
\begin{itemize}
\item[a)] $h\left(t\right)\geq0$ es decreciente y Riemann Integrable.
\item[b)] $h$ es continua excepto posiblemente en un conjunto de Lebesgue de medida 0, y $|h\left(t\right)|\leq b\left(t\right)$, donde $b$ es DRI.
\end{itemize}
\end{Note}

\begin{Teo}[Teorema Principal de Renovaci\'on]
Si $F$ es no aritm\'etica y $h\left(t\right)$ es Directamente Riemann Integrable (DRI), entonces

\begin{eqnarray*}
lim_{t\rightarrow\infty}U\star h=\frac{1}{\mu}\int_{\rea_{+}}h\left(s\right)ds.
\end{eqnarray*}
\end{Teo}

\begin{Prop}
Cualquier funci\'on $H\left(t\right)$ acotada en intervalos finitos y que es 0 para $t<0$ puede expresarse como
\begin{eqnarray*}
H\left(t\right)=U\star h\left(t\right)\textrm{,  donde }h\left(t\right)=H\left(t\right)-F\star H\left(t\right)
\end{eqnarray*}
\end{Prop}

\begin{Def}
Un proceso estoc\'astico $X\left(t\right)$ es crudamente regenerativo en un tiempo aleatorio positivo $T$ si
\begin{eqnarray*}
\esp\left[X\left(T+t\right)|T\right]=\esp\left[X\left(t\right)\right]\textrm{, para }t\geq0,\end{eqnarray*}
y con las esperanzas anteriores finitas.
\end{Def}

\begin{Prop}
Sup\'ongase que $X\left(t\right)$ es un proceso crudamente regenerativo en $T$, que tiene distribuci\'on $F$. Si $\esp\left[X\left(t\right)\right]$ es acotado en intervalos finitos, entonces
\begin{eqnarray*}
\esp\left[X\left(t\right)\right]=U\star h\left(t\right)\textrm{,  donde }h\left(t\right)=\esp\left[X\left(t\right)\indora\left(T>t\right)\right].
\end{eqnarray*}
\end{Prop}

\begin{Teo}[Regeneraci\'on Cruda]
Sup\'ongase que $X\left(t\right)$ es un proceso con valores positivo crudamente regenerativo en $T$, y def\'inase $M=\sup\left\{|X\left(t\right)|:t\leq T\right\}$. Si $T$ es no aritm\'etico y $M$ y $MT$ tienen media finita, entonces
\begin{eqnarray*}
lim_{t\rightarrow\infty}\esp\left[X\left(t\right)\right]=\frac{1}{\mu}\int_{\rea_{+}}h\left(s\right)ds,
\end{eqnarray*}
donde $h\left(t\right)=\esp\left[X\left(t\right)\indora\left(T>t\right)\right]$.
\end{Teo}


\begin{Note} Una funci\'on $h:\rea_{+}\rightarrow\rea$ es Directamente Riemann Integrable en los siguientes casos:
\begin{itemize}
\item[a)] $h\left(t\right)\geq0$ es decreciente y Riemann Integrable.
\item[b)] $h$ es continua excepto posiblemente en un conjunto de Lebesgue de medida 0, y $|h\left(t\right)|\leq b\left(t\right)$, donde $b$ es DRI.
\end{itemize}
\end{Note}

\begin{Teo}[Teorema Principal de Renovaci\'on]
Si $F$ es no aritm\'etica y $h\left(t\right)$ es Directamente Riemann Integrable (DRI), entonces

\begin{eqnarray*}
lim_{t\rightarrow\infty}U\star h=\frac{1}{\mu}\int_{\rea_{+}}h\left(s\right)ds.
\end{eqnarray*}
\end{Teo}

\begin{Prop}
Cualquier funci\'on $H\left(t\right)$ acotada en intervalos finitos y que es 0 para $t<0$ puede expresarse como
\begin{eqnarray*}
H\left(t\right)=U\star h\left(t\right)\textrm{,  donde }h\left(t\right)=H\left(t\right)-F\star H\left(t\right)
\end{eqnarray*}
\end{Prop}

\begin{Def}
Un proceso estoc\'astico $X\left(t\right)$ es crudamente regenerativo en un tiempo aleatorio positivo $T$ si
\begin{eqnarray*}
\esp\left[X\left(T+t\right)|T\right]=\esp\left[X\left(t\right)\right]\textrm{, para }t\geq0,\end{eqnarray*}
y con las esperanzas anteriores finitas.
\end{Def}

\begin{Prop}
Sup\'ongase que $X\left(t\right)$ es un proceso crudamente regenerativo en $T$, que tiene distribuci\'on $F$. Si $\esp\left[X\left(t\right)\right]$ es acotado en intervalos finitos, entonces
\begin{eqnarray*}
\esp\left[X\left(t\right)\right]=U\star h\left(t\right)\textrm{,  donde }h\left(t\right)=\esp\left[X\left(t\right)\indora\left(T>t\right)\right].
\end{eqnarray*}
\end{Prop}

\begin{Teo}[Regeneraci\'on Cruda]
Sup\'ongase que $X\left(t\right)$ es un proceso con valores positivo crudamente regenerativo en $T$, y def\'inase $M=\sup\left\{|X\left(t\right)|:t\leq T\right\}$. Si $T$ es no aritm\'etico y $M$ y $MT$ tienen media finita, entonces
\begin{eqnarray*}
lim_{t\rightarrow\infty}\esp\left[X\left(t\right)\right]=\frac{1}{\mu}\int_{\rea_{+}}h\left(s\right)ds,
\end{eqnarray*}
donde $h\left(t\right)=\esp\left[X\left(t\right)\indora\left(T>t\right)\right]$.
\end{Teo}

\begin{Def}
Para el proceso $\left\{\left(N\left(t\right),X\left(t\right)\right):t\geq0\right\}$, sus trayectoria muestrales en el intervalo de tiempo $\left[T_{n-1},T_{n}\right)$ est\'an descritas por
\begin{eqnarray*}
\zeta_{n}=\left(\xi_{n},\left\{X\left(T_{n-1}+t\right):0\leq t<\xi_{n}\right\}\right)
\end{eqnarray*}
Este $\zeta_{n}$ es el $n$-\'esimo segmento del proceso. El proceso es regenerativo sobre los tiempos $T_{n}$ si sus segmentos $\zeta_{n}$ son independientes e id\'enticamennte distribuidos.
\end{Def}


\begin{Note}
Si $\tilde{X}\left(t\right)$ con espacio de estados $\tilde{S}$ es regenerativo sobre $T_{n}$, entonces $X\left(t\right)=f\left(\tilde{X}\left(t\right)\right)$ tambi\'en es regenerativo sobre $T_{n}$, para cualquier funci\'on $f:\tilde{S}\rightarrow S$.
\end{Note}

\begin{Note}
Los procesos regenerativos son crudamente regenerativos, pero no al rev\'es.
\end{Note}


\begin{Note}
Un proceso estoc\'astico a tiempo continuo o discreto es regenerativo si existe un proceso de renovaci\'on  tal que los segmentos del proceso entre tiempos de renovaci\'on sucesivos son i.i.d., es decir, para $\left\{X\left(t\right):t\geq0\right\}$ proceso estoc\'astico a tiempo continuo con espacio de estados $S$, espacio m\'etrico.
\end{Note}

Para $\left\{X\left(t\right):t\geq0\right\}$ Proceso Estoc\'astico a tiempo continuo con estado de espacios $S$, que es un espacio m\'etrico, con trayectorias continuas por la derecha y con l\'imites por la izquierda c.s. Sea $N\left(t\right)$ un proceso de renovaci\'on en $\rea_{+}$ definido en el mismo espacio de probabilidad que $X\left(t\right)$, con tiempos de renovaci\'on $T$ y tiempos de inter-renovaci\'on $\xi_{n}=T_{n}-T_{n-1}$, con misma distribuci\'on $F$ de media finita $\mu$.



\begin{Def}
Para el proceso $\left\{\left(N\left(t\right),X\left(t\right)\right):t\geq0\right\}$, sus trayectoria muestrales en el intervalo de tiempo $\left[T_{n-1},T_{n}\right)$ est\'an descritas por
\begin{eqnarray*}
\zeta_{n}=\left(\xi_{n},\left\{X\left(T_{n-1}+t\right):0\leq t<\xi_{n}\right\}\right)
\end{eqnarray*}
Este $\zeta_{n}$ es el $n$-\'esimo segmento del proceso. El proceso es regenerativo sobre los tiempos $T_{n}$ si sus segmentos $\zeta_{n}$ son independientes e id\'enticamennte distribuidos.
\end{Def}

\begin{Note}
Un proceso regenerativo con media de la longitud de ciclo finita es llamado positivo recurrente.
\end{Note}

\begin{Teo}[Procesos Regenerativos]
Suponga que el proceso
\end{Teo}


\begin{Def}[Renewal Process Trinity]
Para un proceso de renovaci\'on $N\left(t\right)$, los siguientes procesos proveen de informaci\'on sobre los tiempos de renovaci\'on.
\begin{itemize}
\item $A\left(t\right)=t-T_{N\left(t\right)}$, el tiempo de recurrencia hacia atr\'as al tiempo $t$, que es el tiempo desde la \'ultima renovaci\'on para $t$.

\item $B\left(t\right)=T_{N\left(t\right)+1}-t$, el tiempo de recurrencia hacia adelante al tiempo $t$, residual del tiempo de renovaci\'on, que es el tiempo para la pr\'oxima renovaci\'on despu\'es de $t$.

\item $L\left(t\right)=\xi_{N\left(t\right)+1}=A\left(t\right)+B\left(t\right)$, la longitud del intervalo de renovaci\'on que contiene a $t$.
\end{itemize}
\end{Def}

\begin{Note}
El proceso tridimensional $\left(A\left(t\right),B\left(t\right),L\left(t\right)\right)$ es regenerativo sobre $T_{n}$, y por ende cada proceso lo es. Cada proceso $A\left(t\right)$ y $B\left(t\right)$ son procesos de MArkov a tiempo continuo con trayectorias continuas por partes en el espacio de estados $\rea_{+}$. Una expresi\'on conveniente para su distribuci\'on conjunta es, para $0\leq x<t,y\geq0$
\begin{equation}\label{NoRenovacion}
P\left\{A\left(t\right)>x,B\left(t\right)>y\right\}=
P\left\{N\left(t+y\right)-N\left((t-x)\right)=0\right\}
\end{equation}
\end{Note}

\begin{Ejem}[Tiempos de recurrencia Poisson]
Si $N\left(t\right)$ es un proceso Poisson con tasa $\lambda$, entonces de la expresi\'on (\ref{NoRenovacion}) se tiene que

\begin{eqnarray*}
\begin{array}{lc}
P\left\{A\left(t\right)>x,B\left(t\right)>y\right\}=e^{-\lambda\left(x+y\right)},&0\leq x<t,y\geq0,
\end{array}
\end{eqnarray*}
que es la probabilidad Poisson de no renovaciones en un intervalo de longitud $x+y$.

\end{Ejem}

\begin{Note}
Una cadena de Markov erg\'odica tiene la propiedad de ser estacionaria si la distribuci\'on de su estado al tiempo $0$ es su distribuci\'on estacionaria.
\end{Note}


\begin{Def}
Un proceso estoc\'astico a tiempo continuo $\left\{X\left(t\right):t\geq0\right\}$ en un espacio general es estacionario si sus distribuciones finito dimensionales son invariantes bajo cualquier  traslado: para cada $0\leq s_{1}<s_{2}<\cdots<s_{k}$ y $t\geq0$,
\begin{eqnarray*}
\left(X\left(s_{1}+t\right),\ldots,X\left(s_{k}+t\right)\right)=_{d}\left(X\left(s_{1}\right),\ldots,X\left(s_{k}\right)\right).
\end{eqnarray*}
\end{Def}

\begin{Note}
Un proceso de Markov es estacionario si $X\left(t\right)=_{d}X\left(0\right)$, $t\geq0$.
\end{Note}

Considerese el proceso $N\left(t\right)=\sum_{n}\indora\left(\tau_{n}\leq t\right)$ en $\rea_{+}$, con puntos $0<\tau_{1}<\tau_{2}<\cdots$.

\begin{Prop}
Si $N$ es un proceso puntual estacionario y $\esp\left[N\left(1\right)\right]<\infty$, entonces $\esp\left[N\left(t\right)\right]=t\esp\left[N\left(1\right)\right]$, $t\geq0$

\end{Prop}

\begin{Teo}
Los siguientes enunciados son equivalentes
\begin{itemize}
\item[i)] El proceso retardado de renovaci\'on $N$ es estacionario.

\item[ii)] EL proceso de tiempos de recurrencia hacia adelante $B\left(t\right)$ es estacionario.


\item[iii)] $\esp\left[N\left(t\right)\right]=t/\mu$,


\item[iv)] $G\left(t\right)=F_{e}\left(t\right)=\frac{1}{\mu}\int_{0}^{t}\left[1-F\left(s\right)\right]ds$
\end{itemize}
Cuando estos enunciados son ciertos, $P\left\{B\left(t\right)\leq x\right\}=F_{e}\left(x\right)$, para $t,x\geq0$.

\end{Teo}

\begin{Note}
Una consecuencia del teorema anterior es que el Proceso Poisson es el \'unico proceso sin retardo que es estacionario.
\end{Note}

\begin{Coro}
El proceso de renovaci\'on $N\left(t\right)$ sin retardo, y cuyos tiempos de inter renonaci\'on tienen media finita, es estacionario si y s\'olo si es un proceso Poisson.

\end{Coro}


%________________________________________________________________________
%\subsection{Procesos Regenerativos}
%________________________________________________________________________



\begin{Note}
Si $\tilde{X}\left(t\right)$ con espacio de estados $\tilde{S}$ es regenerativo sobre $T_{n}$, entonces $X\left(t\right)=f\left(\tilde{X}\left(t\right)\right)$ tambi\'en es regenerativo sobre $T_{n}$, para cualquier funci\'on $f:\tilde{S}\rightarrow S$.
\end{Note}

\begin{Note}
Los procesos regenerativos son crudamente regenerativos, pero no al rev\'es.
\end{Note}
%\subsection*{Procesos Regenerativos: Sigman\cite{Sigman1}}
\begin{Def}[Definici\'on Cl\'asica]
Un proceso estoc\'astico $X=\left\{X\left(t\right):t\geq0\right\}$ es llamado regenerativo is existe una variable aleatoria $R_{1}>0$ tal que
\begin{itemize}
\item[i)] $\left\{X\left(t+R_{1}\right):t\geq0\right\}$ es independiente de $\left\{\left\{X\left(t\right):t<R_{1}\right\},\right\}$
\item[ii)] $\left\{X\left(t+R_{1}\right):t\geq0\right\}$ es estoc\'asticamente equivalente a $\left\{X\left(t\right):t>0\right\}$
\end{itemize}

Llamamos a $R_{1}$ tiempo de regeneraci\'on, y decimos que $X$ se regenera en este punto.
\end{Def}

$\left\{X\left(t+R_{1}\right)\right\}$ es regenerativo con tiempo de regeneraci\'on $R_{2}$, independiente de $R_{1}$ pero con la misma distribuci\'on que $R_{1}$. Procediendo de esta manera se obtiene una secuencia de variables aleatorias independientes e id\'enticamente distribuidas $\left\{R_{n}\right\}$ llamados longitudes de ciclo. Si definimos a $Z_{k}\equiv R_{1}+R_{2}+\cdots+R_{k}$, se tiene un proceso de renovaci\'on llamado proceso de renovaci\'on encajado para $X$.




\begin{Def}
Para $x$ fijo y para cada $t\geq0$, sea $I_{x}\left(t\right)=1$ si $X\left(t\right)\leq x$,  $I_{x}\left(t\right)=0$ en caso contrario, y def\'inanse los tiempos promedio
\begin{eqnarray*}
\overline{X}&=&lim_{t\rightarrow\infty}\frac{1}{t}\int_{0}^{\infty}X\left(u\right)du\\
\prob\left(X_{\infty}\leq x\right)&=&lim_{t\rightarrow\infty}\frac{1}{t}\int_{0}^{\infty}I_{x}\left(u\right)du,
\end{eqnarray*}
cuando estos l\'imites existan.
\end{Def}

Como consecuencia del teorema de Renovaci\'on-Recompensa, se tiene que el primer l\'imite  existe y es igual a la constante
\begin{eqnarray*}
\overline{X}&=&\frac{\esp\left[\int_{0}^{R_{1}}X\left(t\right)dt\right]}{\esp\left[R_{1}\right]},
\end{eqnarray*}
suponiendo que ambas esperanzas son finitas.

\begin{Note}
\begin{itemize}
\item[a)] Si el proceso regenerativo $X$ es positivo recurrente y tiene trayectorias muestrales no negativas, entonces la ecuaci\'on anterior es v\'alida.
\item[b)] Si $X$ es positivo recurrente regenerativo, podemos construir una \'unica versi\'on estacionaria de este proceso, $X_{e}=\left\{X_{e}\left(t\right)\right\}$, donde $X_{e}$ es un proceso estoc\'astico regenerativo y estrictamente estacionario, con distribuci\'on marginal distribuida como $X_{\infty}$
\end{itemize}
\end{Note}

%________________________________________________________________________
%\subsection{Procesos Regenerativos}
%________________________________________________________________________

Para $\left\{X\left(t\right):t\geq0\right\}$ Proceso Estoc\'astico a tiempo continuo con estado de espacios $S$, que es un espacio m\'etrico, con trayectorias continuas por la derecha y con l\'imites por la izquierda c.s. Sea $N\left(t\right)$ un proceso de renovaci\'on en $\rea_{+}$ definido en el mismo espacio de probabilidad que $X\left(t\right)$, con tiempos de renovaci\'on $T$ y tiempos de inter-renovaci\'on $\xi_{n}=T_{n}-T_{n-1}$, con misma distribuci\'on $F$ de media finita $\mu$.



\begin{Def}
Para el proceso $\left\{\left(N\left(t\right),X\left(t\right)\right):t\geq0\right\}$, sus trayectoria muestrales en el intervalo de tiempo $\left[T_{n-1},T_{n}\right)$ est\'an descritas por
\begin{eqnarray*}
\zeta_{n}=\left(\xi_{n},\left\{X\left(T_{n-1}+t\right):0\leq t<\xi_{n}\right\}\right)
\end{eqnarray*}
Este $\zeta_{n}$ es el $n$-\'esimo segmento del proceso. El proceso es regenerativo sobre los tiempos $T_{n}$ si sus segmentos $\zeta_{n}$ son independientes e id\'enticamennte distribuidos.
\end{Def}


\begin{Note}
Si $\tilde{X}\left(t\right)$ con espacio de estados $\tilde{S}$ es regenerativo sobre $T_{n}$, entonces $X\left(t\right)=f\left(\tilde{X}\left(t\right)\right)$ tambi\'en es regenerativo sobre $T_{n}$, para cualquier funci\'on $f:\tilde{S}\rightarrow S$.
\end{Note}

\begin{Note}
Los procesos regenerativos son crudamente regenerativos, pero no al rev\'es.
\end{Note}

\begin{Def}[Definici\'on Cl\'asica]
Un proceso estoc\'astico $X=\left\{X\left(t\right):t\geq0\right\}$ es llamado regenerativo is existe una variable aleatoria $R_{1}>0$ tal que
\begin{itemize}
\item[i)] $\left\{X\left(t+R_{1}\right):t\geq0\right\}$ es independiente de $\left\{\left\{X\left(t\right):t<R_{1}\right\},\right\}$
\item[ii)] $\left\{X\left(t+R_{1}\right):t\geq0\right\}$ es estoc\'asticamente equivalente a $\left\{X\left(t\right):t>0\right\}$
\end{itemize}

Llamamos a $R_{1}$ tiempo de regeneraci\'on, y decimos que $X$ se regenera en este punto.
\end{Def}

$\left\{X\left(t+R_{1}\right)\right\}$ es regenerativo con tiempo de regeneraci\'on $R_{2}$, independiente de $R_{1}$ pero con la misma distribuci\'on que $R_{1}$. Procediendo de esta manera se obtiene una secuencia de variables aleatorias independientes e id\'enticamente distribuidas $\left\{R_{n}\right\}$ llamados longitudes de ciclo. Si definimos a $Z_{k}\equiv R_{1}+R_{2}+\cdots+R_{k}$, se tiene un proceso de renovaci\'on llamado proceso de renovaci\'on encajado para $X$.

\begin{Note}
Un proceso regenerativo con media de la longitud de ciclo finita es llamado positivo recurrente.
\end{Note}


\begin{Def}
Para $x$ fijo y para cada $t\geq0$, sea $I_{x}\left(t\right)=1$ si $X\left(t\right)\leq x$,  $I_{x}\left(t\right)=0$ en caso contrario, y def\'inanse los tiempos promedio
\begin{eqnarray*}
\overline{X}&=&lim_{t\rightarrow\infty}\frac{1}{t}\int_{0}^{\infty}X\left(u\right)du\\
\prob\left(X_{\infty}\leq x\right)&=&lim_{t\rightarrow\infty}\frac{1}{t}\int_{0}^{\infty}I_{x}\left(u\right)du,
\end{eqnarray*}
cuando estos l\'imites existan.
\end{Def}

Como consecuencia del teorema de Renovaci\'on-Recompensa, se tiene que el primer l\'imite  existe y es igual a la constante
\begin{eqnarray*}
\overline{X}&=&\frac{\esp\left[\int_{0}^{R_{1}}X\left(t\right)dt\right]}{\esp\left[R_{1}\right]},
\end{eqnarray*}
suponiendo que ambas esperanzas son finitas.

\begin{Note}
\begin{itemize}
\item[a)] Si el proceso regenerativo $X$ es positivo recurrente y tiene trayectorias muestrales no negativas, entonces la ecuaci\'on anterior es v\'alida.
\item[b)] Si $X$ es positivo recurrente regenerativo, podemos construir una \'unica versi\'on estacionaria de este proceso, $X_{e}=\left\{X_{e}\left(t\right)\right\}$, donde $X_{e}$ es un proceso estoc\'astico regenerativo y estrictamente estacionario, con distribuci\'on marginal distribuida como $X_{\infty}$
\end{itemize}
\end{Note}

%__________________________________________________________________________________________
%\subsection{Procesos Regenerativos Estacionarios - Stidham \cite{Stidham}}
%__________________________________________________________________________________________


Un proceso estoc\'astico a tiempo continuo $\left\{V\left(t\right),t\geq0\right\}$ es un proceso regenerativo si existe una sucesi\'on de variables aleatorias independientes e id\'enticamente distribuidas $\left\{X_{1},X_{2},\ldots\right\}$, sucesi\'on de renovaci\'on, tal que para cualquier conjunto de Borel $A$, 

\begin{eqnarray*}
\prob\left\{V\left(t\right)\in A|X_{1}+X_{2}+\cdots+X_{R\left(t\right)}=s,\left\{V\left(\tau\right),\tau<s\right\}\right\}=\prob\left\{V\left(t-s\right)\in A|X_{1}>t-s\right\},
\end{eqnarray*}
para todo $0\leq s\leq t$, donde $R\left(t\right)=\max\left\{X_{1}+X_{2}+\cdots+X_{j}\leq t\right\}=$n\'umero de renovaciones ({\emph{puntos de regeneraci\'on}}) que ocurren en $\left[0,t\right]$. El intervalo $\left[0,X_{1}\right)$ es llamado {\emph{primer ciclo de regeneraci\'on}} de $\left\{V\left(t \right),t\geq0\right\}$, $\left[X_{1},X_{1}+X_{2}\right)$ el {\emph{segundo ciclo de regeneraci\'on}}, y as\'i sucesivamente.

Sea $X=X_{1}$ y sea $F$ la funci\'on de distrbuci\'on de $X$


\begin{Def}
Se define el proceso estacionario, $\left\{V^{*}\left(t\right),t\geq0\right\}$, para $\left\{V\left(t\right),t\geq0\right\}$ por

\begin{eqnarray*}
\prob\left\{V\left(t\right)\in A\right\}=\frac{1}{\esp\left[X\right]}\int_{0}^{\infty}\prob\left\{V\left(t+x\right)\in A|X>x\right\}\left(1-F\left(x\right)\right)dx,
\end{eqnarray*} 
para todo $t\geq0$ y todo conjunto de Borel $A$.
\end{Def}

\begin{Def}
Una distribuci\'on se dice que es {\emph{aritm\'etica}} si todos sus puntos de incremento son m\'ultiplos de la forma $0,\lambda, 2\lambda,\ldots$ para alguna $\lambda>0$ entera.
\end{Def}


\begin{Def}
Una modificaci\'on medible de un proceso $\left\{V\left(t\right),t\geq0\right\}$, es una versi\'on de este, $\left\{V\left(t,w\right)\right\}$ conjuntamente medible para $t\geq0$ y para $w\in S$, $S$ espacio de estados para $\left\{V\left(t\right),t\geq0\right\}$.
\end{Def}

\begin{Teo}
Sea $\left\{V\left(t\right),t\geq\right\}$ un proceso regenerativo no negativo con modificaci\'on medible. Sea $\esp\left[X\right]<\infty$. Entonces el proceso estacionario dado por la ecuaci\'on anterior est\'a bien definido y tiene funci\'on de distribuci\'on independiente de $t$, adem\'as
\begin{itemize}
\item[i)] \begin{eqnarray*}
\esp\left[V^{*}\left(0\right)\right]&=&\frac{\esp\left[\int_{0}^{X}V\left(s\right)ds\right]}{\esp\left[X\right]}\end{eqnarray*}
\item[ii)] Si $\esp\left[V^{*}\left(0\right)\right]<\infty$, equivalentemente, si $\esp\left[\int_{0}^{X}V\left(s\right)ds\right]<\infty$,entonces
\begin{eqnarray*}
\frac{\int_{0}^{t}V\left(s\right)ds}{t}\rightarrow\frac{\esp\left[\int_{0}^{X}V\left(s\right)ds\right]}{\esp\left[X\right]}
\end{eqnarray*}
con probabilidad 1 y en media, cuando $t\rightarrow\infty$.
\end{itemize}
\end{Teo}
%
%___________________________________________________________________________________________
%\vspace{5.5cm}
%\chapter{Cadenas de Markov estacionarias}
%\vspace{-1.0cm}


%__________________________________________________________________________________________
%\subsection{Procesos Regenerativos Estacionarios - Stidham \cite{Stidham}}
%__________________________________________________________________________________________


Un proceso estoc\'astico a tiempo continuo $\left\{V\left(t\right),t\geq0\right\}$ es un proceso regenerativo si existe una sucesi\'on de variables aleatorias independientes e id\'enticamente distribuidas $\left\{X_{1},X_{2},\ldots\right\}$, sucesi\'on de renovaci\'on, tal que para cualquier conjunto de Borel $A$, 

\begin{eqnarray*}
\prob\left\{V\left(t\right)\in A|X_{1}+X_{2}+\cdots+X_{R\left(t\right)}=s,\left\{V\left(\tau\right),\tau<s\right\}\right\}=\prob\left\{V\left(t-s\right)\in A|X_{1}>t-s\right\},
\end{eqnarray*}
para todo $0\leq s\leq t$, donde $R\left(t\right)=\max\left\{X_{1}+X_{2}+\cdots+X_{j}\leq t\right\}=$n\'umero de renovaciones ({\emph{puntos de regeneraci\'on}}) que ocurren en $\left[0,t\right]$. El intervalo $\left[0,X_{1}\right)$ es llamado {\emph{primer ciclo de regeneraci\'on}} de $\left\{V\left(t \right),t\geq0\right\}$, $\left[X_{1},X_{1}+X_{2}\right)$ el {\emph{segundo ciclo de regeneraci\'on}}, y as\'i sucesivamente.

Sea $X=X_{1}$ y sea $F$ la funci\'on de distrbuci\'on de $X$


\begin{Def}
Se define el proceso estacionario, $\left\{V^{*}\left(t\right),t\geq0\right\}$, para $\left\{V\left(t\right),t\geq0\right\}$ por

\begin{eqnarray*}
\prob\left\{V\left(t\right)\in A\right\}=\frac{1}{\esp\left[X\right]}\int_{0}^{\infty}\prob\left\{V\left(t+x\right)\in A|X>x\right\}\left(1-F\left(x\right)\right)dx,
\end{eqnarray*} 
para todo $t\geq0$ y todo conjunto de Borel $A$.
\end{Def}

\begin{Def}
Una distribuci\'on se dice que es {\emph{aritm\'etica}} si todos sus puntos de incremento son m\'ultiplos de la forma $0,\lambda, 2\lambda,\ldots$ para alguna $\lambda>0$ entera.
\end{Def}


\begin{Def}
Una modificaci\'on medible de un proceso $\left\{V\left(t\right),t\geq0\right\}$, es una versi\'on de este, $\left\{V\left(t,w\right)\right\}$ conjuntamente medible para $t\geq0$ y para $w\in S$, $S$ espacio de estados para $\left\{V\left(t\right),t\geq0\right\}$.
\end{Def}

\begin{Teo}
Sea $\left\{V\left(t\right),t\geq\right\}$ un proceso regenerativo no negativo con modificaci\'on medible. Sea $\esp\left[X\right]<\infty$. Entonces el proceso estacionario dado por la ecuaci\'on anterior est\'a bien definido y tiene funci\'on de distribuci\'on independiente de $t$, adem\'as
\begin{itemize}
\item[i)] \begin{eqnarray*}
\esp\left[V^{*}\left(0\right)\right]&=&\frac{\esp\left[\int_{0}^{X}V\left(s\right)ds\right]}{\esp\left[X\right]}\end{eqnarray*}
\item[ii)] Si $\esp\left[V^{*}\left(0\right)\right]<\infty$, equivalentemente, si $\esp\left[\int_{0}^{X}V\left(s\right)ds\right]<\infty$,entonces
\begin{eqnarray*}
\frac{\int_{0}^{t}V\left(s\right)ds}{t}\rightarrow\frac{\esp\left[\int_{0}^{X}V\left(s\right)ds\right]}{\esp\left[X\right]}
\end{eqnarray*}
con probabilidad 1 y en media, cuando $t\rightarrow\infty$.
\end{itemize}
\end{Teo}

Para $\left\{X\left(t\right):t\geq0\right\}$ Proceso Estoc\'astico a tiempo continuo con estado de espacios $S$, que es un espacio m\'etrico, con trayectorias continuas por la derecha y con l\'imites por la izquierda c.s. Sea $N\left(t\right)$ un proceso de renovaci\'on en $\rea_{+}$ definido en el mismo espacio de probabilidad que $X\left(t\right)$, con tiempos de renovaci\'on $T$ y tiempos de inter-renovaci\'on $\xi_{n}=T_{n}-T_{n-1}$, con misma distribuci\'on $F$ de media finita $\mu$.


%______________________________________________________________________
%\subsection{Ejemplos, Notas importantes}


Sean $T_{1},T_{2},\ldots$ los puntos donde las longitudes de las colas de la red de sistemas de visitas c\'iclicas son cero simult\'aneamente, cuando la cola $Q_{j}$ es visitada por el servidor para dar servicio, es decir, $L_{1}\left(T_{i}\right)=0,L_{2}\left(T_{i}\right)=0,\hat{L}_{1}\left(T_{i}\right)=0$ y $\hat{L}_{2}\left(T_{i}\right)=0$, a estos puntos se les denominar\'a puntos regenerativos. Sea la funci\'on generadora de momentos para $L_{i}$, el n\'umero de usuarios en la cola $Q_{i}\left(z\right)$ en cualquier momento, est\'a dada por el tiempo promedio de $z^{L_{i}\left(t\right)}$ sobre el ciclo regenerativo definido anteriormente:

\begin{eqnarray*}
Q_{i}\left(z\right)&=&\esp\left[z^{L_{i}\left(t\right)}\right]=\frac{\esp\left[\sum_{m=1}^{M_{i}}\sum_{t=\tau_{i}\left(m\right)}^{\tau_{i}\left(m+1\right)-1}z^{L_{i}\left(t\right)}\right]}{\esp\left[\sum_{m=1}^{M_{i}}\tau_{i}\left(m+1\right)-\tau_{i}\left(m\right)\right]}
\end{eqnarray*}

$M_{i}$ es un tiempo de paro en el proceso regenerativo con $\esp\left[M_{i}\right]<\infty$\footnote{En Stidham\cite{Stidham} y Heyman  se muestra que una condici\'on suficiente para que el proceso regenerativo 
estacionario sea un procesoo estacionario es que el valor esperado del tiempo del ciclo regenerativo sea finito, es decir: $\esp\left[\sum_{m=1}^{M_{i}}C_{i}^{(m)}\right]<\infty$, como cada $C_{i}^{(m)}$ contiene intervalos de r\'eplica positivos, se tiene que $\esp\left[M_{i}\right]<\infty$, adem\'as, como $M_{i}>0$, se tiene que la condici\'on anterior es equivalente a tener que $\esp\left[C_{i}\right]<\infty$,
por lo tanto una condici\'on suficiente para la existencia del proceso regenerativo est\'a dada por $\sum_{k=1}^{N}\mu_{k}<1.$}, se sigue del lema de Wald que:


\begin{eqnarray*}
\esp\left[\sum_{m=1}^{M_{i}}\sum_{t=\tau_{i}\left(m\right)}^{\tau_{i}\left(m+1\right)-1}z^{L_{i}\left(t\right)}\right]&=&\esp\left[M_{i}\right]\esp\left[\sum_{t=\tau_{i}\left(m\right)}^{\tau_{i}\left(m+1\right)-1}z^{L_{i}\left(t\right)}\right]\\
\esp\left[\sum_{m=1}^{M_{i}}\tau_{i}\left(m+1\right)-\tau_{i}\left(m\right)\right]&=&\esp\left[M_{i}\right]\esp\left[\tau_{i}\left(m+1\right)-\tau_{i}\left(m\right)\right]
\end{eqnarray*}

por tanto se tiene que


\begin{eqnarray*}
Q_{i}\left(z\right)&=&\frac{\esp\left[\sum_{t=\tau_{i}\left(m\right)}^{\tau_{i}\left(m+1\right)-1}z^{L_{i}\left(t\right)}\right]}{\esp\left[\tau_{i}\left(m+1\right)-\tau_{i}\left(m\right)\right]}
\end{eqnarray*}

observar que el denominador es simplemente la duraci\'on promedio del tiempo del ciclo.


Haciendo las siguientes sustituciones en la ecuaci\'on (\ref{Corolario2}): $n\rightarrow t-\tau_{i}\left(m\right)$, $T \rightarrow \overline{\tau}_{i}\left(m\right)-\tau_{i}\left(m\right)$, $L_{n}\rightarrow L_{i}\left(t\right)$ y $F\left(z\right)=\esp\left[z^{L_{0}}\right]\rightarrow F_{i}\left(z\right)=\esp\left[z^{L_{i}\tau_{i}\left(m\right)}\right]$, se puede ver que

\begin{eqnarray}\label{Eq.Arribos.Primera}
\esp\left[\sum_{n=0}^{T-1}z^{L_{n}}\right]=
\esp\left[\sum_{t=\tau_{i}\left(m\right)}^{\overline{\tau}_{i}\left(m\right)-1}z^{L_{i}\left(t\right)}\right]
=z\frac{F_{i}\left(z\right)-1}{z-P_{i}\left(z\right)}
\end{eqnarray}

Por otra parte durante el tiempo de intervisita para la cola $i$, $L_{i}\left(t\right)$ solamente se incrementa de manera que el incremento por intervalo de tiempo est\'a dado por la funci\'on generadora de probabilidades de $P_{i}\left(z\right)$, por tanto la suma sobre el tiempo de intervisita puede evaluarse como:

\begin{eqnarray*}
\esp\left[\sum_{t=\tau_{i}\left(m\right)}^{\tau_{i}\left(m+1\right)-1}z^{L_{i}\left(t\right)}\right]&=&\esp\left[\sum_{t=\tau_{i}\left(m\right)}^{\tau_{i}\left(m+1\right)-1}\left\{P_{i}\left(z\right)\right\}^{t-\overline{\tau}_{i}\left(m\right)}\right]=\frac{1-\esp\left[\left\{P_{i}\left(z\right)\right\}^{\tau_{i}\left(m+1\right)-\overline{\tau}_{i}\left(m\right)}\right]}{1-P_{i}\left(z\right)}\\
&=&\frac{1-I_{i}\left[P_{i}\left(z\right)\right]}{1-P_{i}\left(z\right)}
\end{eqnarray*}
por tanto

\begin{eqnarray*}
\esp\left[\sum_{t=\tau_{i}\left(m\right)}^{\tau_{i}\left(m+1\right)-1}z^{L_{i}\left(t\right)}\right]&=&
\frac{1-F_{i}\left(z\right)}{1-P_{i}\left(z\right)}
\end{eqnarray*}

Por lo tanto

\begin{eqnarray*}
Q_{i}\left(z\right)&=&\frac{\esp\left[\sum_{t=\tau_{i}\left(m\right)}^{\tau_{i}
\left(m+1\right)-1}z^{L_{i}\left(t\right)}\right]}{\esp\left[\tau_{i}\left(m+1\right)-\tau_{i}\left(m\right)\right]}\\
&=&\frac{1}{\esp\left[\tau_{i}\left(m+1\right)-\tau_{i}\left(m\right)\right]}
\left\{
\esp\left[\sum_{t=\tau_{i}\left(m\right)}^{\overline{\tau}_{i}\left(m\right)-1}
z^{L_{i}\left(t\right)}\right]
+\esp\left[\sum_{t=\overline{\tau}_{i}\left(m\right)}^{\tau_{i}\left(m+1\right)-1}
z^{L_{i}\left(t\right)}\right]\right\}\\
&=&\frac{1}{\esp\left[\tau_{i}\left(m+1\right)-\tau_{i}\left(m\right)\right]}
\left\{
z\frac{F_{i}\left(z\right)-1}{z-P_{i}\left(z\right)}+\frac{1-F_{i}\left(z\right)}
{1-P_{i}\left(z\right)}
\right\}
\end{eqnarray*}

es decir

\begin{equation}
Q_{i}\left(z\right)=\frac{1}{\esp\left[C_{i}\right]}\cdot\frac{1-F_{i}\left(z\right)}{P_{i}\left(z\right)-z}\cdot\frac{\left(1-z\right)P_{i}\left(z\right)}{1-P_{i}\left(z\right)}
\end{equation}

\begin{Teo}
Dada una Red de Sistemas de Visitas C\'iclicas (RSVC), conformada por dos Sistemas de Visitas C\'iclicas (SVC), donde cada uno de ellos consta de dos colas tipo $M/M/1$. Los dos sistemas est\'an comunicados entre s\'i por medio de la transferencia de usuarios entre las colas $Q_{1}\leftrightarrow Q_{3}$ y $Q_{2}\leftrightarrow Q_{4}$. Se definen los eventos para los procesos de arribos al tiempo $t$, $A_{j}\left(t\right)=\left\{0 \textrm{ arribos en }Q_{j}\textrm{ al tiempo }t\right\}$ para alg\'un tiempo $t\geq0$ y $Q_{j}$ la cola $j$-\'esima en la RSVC, para $j=1,2,3,4$.  Existe un intervalo $I\neq\emptyset$ tal que para $T^{*}\in I$, tal que $\prob\left\{A_{1}\left(T^{*}\right),A_{2}\left(Tt^{*}\right),
A_{3}\left(T^{*}\right),A_{4}\left(T^{*}\right)|T^{*}\in I\right\}>0$.
\end{Teo}

\begin{proof}
Sin p\'erdida de generalidad podemos considerar como base del an\'alisis a la cola $Q_{1}$ del primer sistema que conforma la RSVC.

Sea $n>0$, ciclo en el primer sistema en el que se sabe que $L_{j}\left(\overline{\tau}_{1}\left(n\right)\right)=0$, pues la pol\'itica de servicio con que atienden los servidores es la exhaustiva. Como es sabido, para trasladarse a la siguiente cola, el servidor incurre en un tiempo de traslado $r_{1}\left(n\right)>0$, entonces tenemos que $\tau_{2}\left(n\right)=\overline{\tau}_{1}\left(n\right)+r_{1}\left(n\right)$.


Definamos el intervalo $I_{1}\equiv\left[\overline{\tau}_{1}\left(n\right),\tau_{2}\left(n\right)\right]$ de longitud $\xi_{1}=r_{1}\left(n\right)$. Dado que los tiempos entre arribo son exponenciales con tasa $\tilde{\mu}_{1}=\mu_{1}+\hat{\mu}_{1}$ ($\mu_{1}$ son los arribos a $Q_{1}$ por primera vez al sistema, mientras que $\hat{\mu}_{1}$ son los arribos de traslado procedentes de $Q_{3}$) se tiene que la probabilidad del evento $A_{1}\left(t\right)$ est\'a dada por 

\begin{equation}
\prob\left\{A_{1}\left(t\right)|t\in I_{1}\left(n\right)\right\}=e^{-\tilde{\mu}_{1}\xi_{1}\left(n\right)}.
\end{equation} 

Por otra parte, para la cola $Q_{2}$, el tiempo $\overline{\tau}_{2}\left(n-1\right)$ es tal que $L_{2}\left(\overline{\tau}_{2}\left(n-1\right)\right)=0$, es decir, es el tiempo en que la cola queda totalmente vac\'ia en el ciclo anterior a $n$. Entonces tenemos un sgundo intervalo $I_{2}\equiv\left[\overline{\tau}_{2}\left(n-1\right),\tau_{2}\left(n\right)\right]$. Por lo tanto la probabilidad del evento $A_{2}\left(t\right)$ tiene probabilidad dada por

\begin{equation}
\prob\left\{A_{2}\left(t\right)|t\in I_{2}\left(n\right)\right\}=e^{-\tilde{\mu}_{2}\xi_{2}\left(n\right)},
\end{equation} 

donde $\xi_{2}\left(n\right)=\tau_{2}\left(n\right)-\overline{\tau}_{2}\left(n-1\right)$.



Entonces, se tiene que

\begin{eqnarray*}
\prob\left\{A_{1}\left(t\right),A_{2}\left(t\right)|t\in I_{1}\left(n\right)\right\}&=&
\prob\left\{A_{1}\left(t\right)|t\in I_{1}\left(n\right)\right\}
\prob\left\{A_{2}\left(t\right)|t\in I_{1}\left(n\right)\right\}\\
&\geq&
\prob\left\{A_{1}\left(t\right)|t\in I_{1}\left(n\right)\right\}
\prob\left\{A_{2}\left(t\right)|t\in I_{2}\left(n\right)\right\}\\
&=&e^{-\tilde{\mu}_{1}\xi_{1}\left(n\right)}e^{-\tilde{\mu}_{2}\xi_{2}\left(n\right)}
=e^{-\left[\tilde{\mu}_{1}\xi_{1}\left(n\right)+\tilde{\mu}_{2}\xi_{2}\left(n\right)\right]}.
\end{eqnarray*}


es decir, 

\begin{equation}
\prob\left\{A_{1}\left(t\right),A_{2}\left(t\right)|t\in I_{1}\left(n\right)\right\}
=e^{-\left[\tilde{\mu}_{1}\xi_{1}\left(n\right)+\tilde{\mu}_{2}\xi_{2}
\left(n\right)\right]}>0.
\end{equation}

En lo que respecta a la relaci\'on entre los dos SVC que conforman la RSVC, se afirma que existe $m>0$ tal que $\overline{\tau}_{3}\left(m\right)<\tau_{2}\left(n\right)<\tau_{4}\left(m\right)$.

Para $Q_{3}$ sea $I_{3}=\left[\overline{\tau}_{3}\left(m\right),\tau_{4}\left(m\right)\right]$ con longitud  $\xi_{3}\left(m\right)=r_{3}\left(m\right)$, entonces 

\begin{equation}
\prob\left\{A_{3}\left(t\right)|t\in I_{3}\left(n\right)\right\}=e^{-\tilde{\mu}_{3}\xi_{3}\left(n\right)}.
\end{equation} 

An\'alogamente que como se hizo para $Q_{2}$, tenemos que para $Q_{4}$ se tiene el intervalo $I_{4}=\left[\overline{\tau}_{4}\left(m-1\right),\tau_{4}\left(m\right)\right]$ con longitud $\xi_{4}\left(m\right)=\tau_{4}\left(m\right)-\overline{\tau}_{4}\left(m-1\right)$, entonces


\begin{equation}
\prob\left\{A_{4}\left(t\right)|t\in I_{4}\left(m\right)\right\}=e^{-\tilde{\mu}_{4}\xi_{4}\left(n\right)}.
\end{equation} 

Al igual que para el primer sistema, dado que $I_{3}\left(m\right)\subset I_{4}\left(m\right)$, se tiene que

\begin{eqnarray*}
\xi_{3}\left(m\right)\leq\xi_{4}\left(m\right)&\Leftrightarrow& -\xi_{3}\left(m\right)\geq-\xi_{4}\left(m\right)
\\
-\tilde{\mu}_{4}\xi_{3}\left(m\right)\geq-\tilde{\mu}_{4}\xi_{4}\left(m\right)&\Leftrightarrow&
e^{-\tilde{\mu}_{4}\xi_{3}\left(m\right)}\geq e^{-\tilde{\mu}_{4}\xi_{4}\left(m\right)}\\
\prob\left\{A_{4}\left(t\right)|t\in I_{3}\left(m\right)\right\}&\geq&
\prob\left\{A_{4}\left(t\right)|t\in I_{4}\left(m\right)\right\}
\end{eqnarray*}

Entonces, dado que los eventos $A_{3}$ y $A_{4}$ son independientes, se tiene que

\begin{eqnarray*}
\prob\left\{A_{3}\left(t\right),A_{4}\left(t\right)|t\in I_{3}\left(m\right)\right\}&=&
\prob\left\{A_{3}\left(t\right)|t\in I_{3}\left(m\right)\right\}
\prob\left\{A_{4}\left(t\right)|t\in I_{3}\left(m\right)\right\}\\
&\geq&
\prob\left\{A_{3}\left(t\right)|t\in I_{3}\left(n\right)\right\}
\prob\left\{A_{4}\left(t\right)|t\in I_{4}\left(n\right)\right\}\\
&=&e^{-\tilde{\mu}_{3}\xi_{3}\left(m\right)}e^{-\tilde{\mu}_{4}\xi_{4}
\left(m\right)}
=e^{-\left[\tilde{\mu}_{3}\xi_{3}\left(m\right)+\tilde{\mu}_{4}\xi_{4}
\left(m\right)\right]}.
\end{eqnarray*}


es decir, 

\begin{equation}
\prob\left\{A_{3}\left(t\right),A_{4}\left(t\right)|t\in I_{3}\left(m\right)\right\}
=e^{-\left[\tilde{\mu}_{3}\xi_{3}\left(m\right)+\tilde{\mu}_{4}\xi_{4}
\left(m\right)\right]}>0.
\end{equation}

Por construcci\'on se tiene que $I\left(n,m\right)\equiv I_{1}\left(n\right)\cap I_{3}\left(m\right)\neq\emptyset$,entonces en particular se tienen las contenciones $I\left(n,m\right)\subseteq I_{1}\left(n\right)$ y $I\left(n,m\right)\subseteq I_{3}\left(m\right)$, por lo tanto si definimos $\xi_{n,m}\equiv\ell\left(I\left(n,m\right)\right)$ tenemos que

\begin{eqnarray*}
\xi_{n,m}\leq\xi_{1}\left(n\right)\textrm{ y }\xi_{n,m}\leq\xi_{3}\left(m\right)\textrm{ entonces }
-\xi_{n,m}\geq-\xi_{1}\left(n\right)\textrm{ y }-\xi_{n,m}\leq-\xi_{3}\left(m\right)\\
\end{eqnarray*}
por lo tanto tenemos las desigualdades 



\begin{eqnarray*}
\begin{array}{ll}
-\tilde{\mu}_{1}\xi_{n,m}\geq-\tilde{\mu}_{1}\xi_{1}\left(n\right),&
-\tilde{\mu}_{2}\xi_{n,m}\geq-\tilde{\mu}_{2}\xi_{1}\left(n\right)
\geq-\tilde{\mu}_{2}\xi_{2}\left(n\right),\\
-\tilde{\mu}_{3}\xi_{n,m}\geq-\tilde{\mu}_{3}\xi_{3}\left(m\right),&
-\tilde{\mu}_{4}\xi_{n,m}\geq-\tilde{\mu}_{4}\xi_{3}\left(m\right)
\geq-\tilde{\mu}_{4}\xi_{4}\left(m\right).
\end{array}
\end{eqnarray*}

Sea $T^{*}\in I_{n,m}$, entonces dado que en particular $T^{*}\in I_{1}\left(n\right)$ se cumple con probabilidad positiva que no hay arribos a las colas $Q_{1}$ y $Q_{2}$, en consecuencia, tampoco hay usuarios de transferencia para $Q_{3}$ y $Q_{4}$, es decir, $\tilde{\mu}_{1}=\mu_{1}$, $\tilde{\mu}_{2}=\mu_{2}$, $\tilde{\mu}_{3}=\mu_{3}$, $\tilde{\mu}_{4}=\mu_{4}$, es decir, los eventos $Q_{1}$ y $Q_{3}$ son condicionalmente independientes en el intervalo $I_{n,m}$; lo mismo ocurre para las colas $Q_{2}$ y $Q_{4}$, por lo tanto tenemos que


\begin{eqnarray}
\begin{array}{l}
\prob\left\{A_{1}\left(T^{*}\right),A_{2}\left(T^{*}\right),
A_{3}\left(T^{*}\right),A_{4}\left(T^{*}\right)|T^{*}\in I_{n,m}\right\}
=\prod_{j=1}^{4}\prob\left\{A_{j}\left(T^{*}\right)|T^{*}\in I_{n,m}\right\}\\
\geq\prob\left\{A_{1}\left(T^{*}\right)|T^{*}\in I_{1}\left(n\right)\right\}
\prob\left\{A_{2}\left(T^{*}\right)|T^{*}\in I_{2}\left(n\right)\right\}
\prob\left\{A_{3}\left(T^{*}\right)|T^{*}\in I_{3}\left(m\right)\right\}
\prob\left\{A_{4}\left(T^{*}\right)|T^{*}\in I_{4}\left(m\right)\right\}\\
=e^{-\mu_{1}\xi_{1}\left(n\right)}
e^{-\mu_{2}\xi_{2}\left(n\right)}
e^{-\mu_{3}\xi_{3}\left(m\right)}
e^{-\mu_{4}\xi_{4}\left(m\right)}
=e^{-\left[\tilde{\mu}_{1}\xi_{1}\left(n\right)
+\tilde{\mu}_{2}\xi_{2}\left(n\right)
+\tilde{\mu}_{3}\xi_{3}\left(m\right)
+\tilde{\mu}_{4}\xi_{4}
\left(m\right)\right]}>0.
\end{array}
\end{eqnarray}
\end{proof}


Estos resultados aparecen en Daley (1968) \cite{Daley68} para $\left\{T_{n}\right\}$ intervalos de inter-arribo, $\left\{D_{n}\right\}$ intervalos de inter-salida y $\left\{S_{n}\right\}$ tiempos de servicio.

\begin{itemize}
\item Si el proceso $\left\{T_{n}\right\}$ es Poisson, el proceso $\left\{D_{n}\right\}$ es no correlacionado si y s\'olo si es un proceso Poisso, lo cual ocurre si y s\'olo si $\left\{S_{n}\right\}$ son exponenciales negativas.

\item Si $\left\{S_{n}\right\}$ son exponenciales negativas, $\left\{D_{n}\right\}$ es un proceso de renovaci\'on  si y s\'olo si es un proceso Poisson, lo cual ocurre si y s\'olo si $\left\{T_{n}\right\}$ es un proceso Poisson.

\item $\esp\left(D_{n}\right)=\esp\left(T_{n}\right)$.

\item Para un sistema de visitas $GI/M/1$ se tiene el siguiente teorema:

\begin{Teo}
En un sistema estacionario $GI/M/1$ los intervalos de interpartida tienen
\begin{eqnarray*}
\esp\left(e^{-\theta D_{n}}\right)&=&\mu\left(\mu+\theta\right)^{-1}\left[\delta\theta
-\mu\left(1-\delta\right)\alpha\left(\theta\right)\right]
\left[\theta-\mu\left(1-\delta\right)^{-1}\right]\\
\alpha\left(\theta\right)&=&\esp\left[e^{-\theta T_{0}}\right]\\
var\left(D_{n}\right)&=&var\left(T_{0}\right)-\left(\tau^{-1}-\delta^{-1}\right)
2\delta\left(\esp\left(S_{0}\right)\right)^{2}\left(1-\delta\right)^{-1}.
\end{eqnarray*}
\end{Teo}



\begin{Teo}
El proceso de salida de un sistema de colas estacionario $GI/M/1$ es un proceso de renovaci\'on si y s\'olo si el proceso de entrada es un proceso Poisson, en cuyo caso el proceso de salida es un proceso Poisson.
\end{Teo}


\begin{Teo}
Los intervalos de interpartida $\left\{D_{n}\right\}$ de un sistema $M/G/1$ estacionario son no correlacionados si y s\'olo si la distribuci\'on de los tiempos de servicio es exponencial negativa, es decir, el sistema es de tipo  $M/M/1$.

\end{Teo}



\end{itemize}


%\section{Resultados para Procesos de Salida}

En Sigman, Thorison y Wolff \cite{Sigman2} prueban que para la existencia de un una sucesi\'on infinita no decreciente de tiempos de regeneraci\'on $\tau_{1}\leq\tau_{2}\leq\cdots$ en los cuales el proceso se regenera, basta un tiempo de regeneraci\'on $R_{1}$, donde $R_{j}=\tau_{j}-\tau_{j-1}$. Para tal efecto se requiere la existencia de un espacio de probabilidad $\left(\Omega,\mathcal{F},\prob\right)$, y proceso estoc\'astico $\textit{X}=\left\{X\left(t\right):t\geq0\right\}$ con espacio de estados $\left(S,\mathcal{R}\right)$, con $\mathcal{R}$ $\sigma$-\'algebra.

\begin{Prop}
Si existe una variable aleatoria no negativa $R_{1}$ tal que $\theta_{R\footnotesize{1}}X=_{D}X$, entonces $\left(\Omega,\mathcal{F},\prob\right)$ puede extenderse para soportar una sucesi\'on estacionaria de variables aleatorias $R=\left\{R_{k}:k\geq1\right\}$, tal que para $k\geq1$,
\begin{eqnarray*}
\theta_{k}\left(X,R\right)=_{D}\left(X,R\right).
\end{eqnarray*}

Adem\'as, para $k\geq1$, $\theta_{k}R$ es condicionalmente independiente de $\left(X,R_{1},\ldots,R_{k}\right)$, dado $\theta_{\tau k}X$.

\end{Prop}


\begin{itemize}
\item Doob en 1953 demostr\'o que el estado estacionario de un proceso de partida en un sistema de espera $M/G/\infty$, es Poisson con la misma tasa que el proceso de arribos.

\item Burke en 1968, fue el primero en demostrar que el estado estacionario de un proceso de salida de una cola $M/M/s$ es un proceso Poisson.

\item Disney en 1973 obtuvo el siguiente resultado:

\begin{Teo}
Para el sistema de espera $M/G/1/L$ con disciplina FIFO, el proceso $\textbf{I}$ es un proceso de renovaci\'on si y s\'olo si el proceso denominado longitud de la cola es estacionario y se cumple cualquiera de los siguientes casos:

\begin{itemize}
\item[a)] Los tiempos de servicio son identicamente cero;
\item[b)] $L=0$, para cualquier proceso de servicio $S$;
\item[c)] $L=1$ y $G=D$;
\item[d)] $L=\infty$ y $G=M$.
\end{itemize}
En estos casos, respectivamente, las distribuciones de interpartida $P\left\{T_{n+1}-T_{n}\leq t\right\}$ son


\begin{itemize}
\item[a)] $1-e^{-\lambda t}$, $t\geq0$;
\item[b)] $1-e^{-\lambda t}*F\left(t\right)$, $t\geq0$;
\item[c)] $1-e^{-\lambda t}*\indora_{d}\left(t\right)$, $t\geq0$;
\item[d)] $1-e^{-\lambda t}*F\left(t\right)$, $t\geq0$.
\end{itemize}
\end{Teo}


\item Finch (1959) mostr\'o que para los sistemas $M/G/1/L$, con $1\leq L\leq \infty$ con distribuciones de servicio dos veces diferenciable, solamente el sistema $M/M/1/\infty$ tiene proceso de salida de renovaci\'on estacionario.

\item King (1971) demostro que un sistema de colas estacionario $M/G/1/1$ tiene sus tiempos de interpartida sucesivas $D_{n}$ y $D_{n+1}$ son independientes, si y s\'olo si, $G=D$, en cuyo caso le proceso de salida es de renovaci\'on.

\item Disney (1973) demostr\'o que el \'unico sistema estacionario $M/G/1/L$, que tiene proceso de salida de renovaci\'on  son los sistemas $M/M/1$ y $M/D/1/1$.



\item El siguiente resultado es de Disney y Koning (1985)
\begin{Teo}
En un sistema de espera $M/G/s$, el estado estacionario del proceso de salida es un proceso Poisson para cualquier distribuci\'on de los tiempos de servicio si el sistema tiene cualquiera de las siguientes cuatro propiedades.

\begin{itemize}
\item[a)] $s=\infty$
\item[b)] La disciplina de servicio es de procesador compartido.
\item[c)] La disciplina de servicio es LCFS y preemptive resume, esto se cumple para $L<\infty$
\item[d)] $G=M$.
\end{itemize}

\end{Teo}

\item El siguiente resultado es de Alamatsaz (1983)

\begin{Teo}
En cualquier sistema de colas $GI/G/1/L$ con $1\leq L<\infty$ y distribuci\'on de interarribos $A$ y distribuci\'on de los tiempos de servicio $B$, tal que $A\left(0\right)=0$, $A\left(t\right)\left(1-B\left(t\right)\right)>0$ para alguna $t>0$ y $B\left(t\right)$ para toda $t>0$, es imposible que el proceso de salida estacionario sea de renovaci\'on.
\end{Teo}

\end{itemize}

Estos resultados aparecen en Daley (1968) \cite{Daley68} para $\left\{T_{n}\right\}$ intervalos de inter-arribo, $\left\{D_{n}\right\}$ intervalos de inter-salida y $\left\{S_{n}\right\}$ tiempos de servicio.

\begin{itemize}
\item Si el proceso $\left\{T_{n}\right\}$ es Poisson, el proceso $\left\{D_{n}\right\}$ es no correlacionado si y s\'olo si es un proceso Poisso, lo cual ocurre si y s\'olo si $\left\{S_{n}\right\}$ son exponenciales negativas.

\item Si $\left\{S_{n}\right\}$ son exponenciales negativas, $\left\{D_{n}\right\}$ es un proceso de renovaci\'on  si y s\'olo si es un proceso Poisson, lo cual ocurre si y s\'olo si $\left\{T_{n}\right\}$ es un proceso Poisson.

\item $\esp\left(D_{n}\right)=\esp\left(T_{n}\right)$.

\item Para un sistema de visitas $GI/M/1$ se tiene el siguiente teorema:

\begin{Teo}
En un sistema estacionario $GI/M/1$ los intervalos de interpartida tienen
\begin{eqnarray*}
\esp\left(e^{-\theta D_{n}}\right)&=&\mu\left(\mu+\theta\right)^{-1}\left[\delta\theta
-\mu\left(1-\delta\right)\alpha\left(\theta\right)\right]
\left[\theta-\mu\left(1-\delta\right)^{-1}\right]\\
\alpha\left(\theta\right)&=&\esp\left[e^{-\theta T_{0}}\right]\\
var\left(D_{n}\right)&=&var\left(T_{0}\right)-\left(\tau^{-1}-\delta^{-1}\right)
2\delta\left(\esp\left(S_{0}\right)\right)^{2}\left(1-\delta\right)^{-1}.
\end{eqnarray*}
\end{Teo}



\begin{Teo}
El proceso de salida de un sistema de colas estacionario $GI/M/1$ es un proceso de renovaci\'on si y s\'olo si el proceso de entrada es un proceso Poisson, en cuyo caso el proceso de salida es un proceso Poisson.
\end{Teo}


\begin{Teo}
Los intervalos de interpartida $\left\{D_{n}\right\}$ de un sistema $M/G/1$ estacionario son no correlacionados si y s\'olo si la distribuci\'on de los tiempos de servicio es exponencial negativa, es decir, el sistema es de tipo  $M/M/1$.

\end{Teo}



\end{itemize}
%\newpage
%________________________________________________________________________
%\section{Redes de Sistemas de Visitas C\'iclicas}
%________________________________________________________________________

Sean $Q_{1},Q_{2},Q_{3}$ y $Q_{4}$ en una Red de Sistemas de Visitas C\'iclicas (RSVC). Supongamos que cada una de las colas es del tipo $M/M/1$ con tasa de arribo $\mu_{i}$ y que la transferencia de usuarios entre los dos sistemas ocurre entre $Q_{1}\leftrightarrow Q_{3}$ y $Q_{2}\leftrightarrow Q_{4}$ con respectiva tasa de arribo igual a la tasa de salida $\hat{\mu}_{i}=\mu_{i}$, esto se sabe por lo desarrollado en la secci\'on anterior.  

Consideremos, sin p\'erdida de generalidad como base del an\'alisis, la cola $Q_{1}$ adem\'as supongamos al servidor lo comenzamos a observar una vez que termina de atender a la misma para desplazarse y llegar a $Q_{2}$, es decir al tiempo $\tau_{2}$.

Sea $n\in\nat$, $n>0$, ciclo del servidor en que regresa a $Q_{1}$ para dar servicio y atender conforme a la pol\'itica exhaustiva, entonces se tiene que $\overline{\tau}_{1}\left(n\right)$ es el tiempo del servidor en el sistema 1 en que termina de dar servicio a todos los usuarios presentes en la cola, por lo tanto se cumple que $L_{1}\left(\overline{\tau}_{1}\left(n\right)\right)=0$, entonces el servidor para llegar a $Q_{2}$ incurre en un tiempo de traslado $r_{1}$ y por tanto se cumple que $\tau_{2}\left(n\right)=\overline{\tau}_{1}\left(n\right)+r_{1}$. Dado que los tiempos entre arribos son exponenciales se cumple que 

\begin{eqnarray*}
\prob\left\{0 \textrm{ arribos en }Q_{1}\textrm{ en el intervalo }\left[\overline{\tau}_{1}\left(n\right),\overline{\tau}_{1}\left(n\right)+r_{1}\right]\right\}=e^{-\tilde{\mu}_{1}r_{1}},\\
\prob\left\{0 \textrm{ arribos en }Q_{2}\textrm{ en el intervalo }\left[\overline{\tau}_{1}\left(n\right),\overline{\tau}_{1}\left(n\right)+r_{1}\right]\right\}=e^{-\tilde{\mu}_{2}r_{1}}.
\end{eqnarray*}

El evento que nos interesa consiste en que no haya arribos desde que el servidor abandon\'o $Q_{2}$ y regresa nuevamente para dar servicio, es decir en el intervalo de tiempo $\left[\overline{\tau}_{2}\left(n-1\right),\tau_{2}\left(n\right)\right]$. Entonces, si hacemos


\begin{eqnarray*}
\varphi_{1}\left(n\right)&\equiv&\overline{\tau}_{1}\left(n\right)+r_{1}=\overline{\tau}_{2}\left(n-1\right)+r_{1}+r_{2}+\overline{\tau}_{1}\left(n\right)-\tau_{1}\left(n\right)\\
&=&\overline{\tau}_{2}\left(n-1\right)+\overline{\tau}_{1}\left(n\right)-\tau_{1}\left(n\right)+r,,
\end{eqnarray*}

y longitud del intervalo

\begin{eqnarray*}
\xi&\equiv&\overline{\tau}_{1}\left(n\right)+r_{1}-\overline{\tau}_{2}\left(n-1\right)
=\overline{\tau}_{2}\left(n-1\right)+\overline{\tau}_{1}\left(n\right)-\tau_{1}\left(n\right)+r-\overline{\tau}_{2}\left(n-1\right)\\
&=&\overline{\tau}_{1}\left(n\right)-\tau_{1}\left(n\right)+r.
\end{eqnarray*}


Entonces, determinemos la probabilidad del evento no arribos a $Q_{2}$ en $\left[\overline{\tau}_{2}\left(n-1\right),\varphi_{1}\left(n\right)\right]$:

\begin{eqnarray}
\prob\left\{0 \textrm{ arribos en }Q_{2}\textrm{ en el intervalo }\left[\overline{\tau}_{2}\left(n-1\right),\varphi_{1}\left(n\right)\right]\right\}
=e^{-\tilde{\mu}_{2}\xi}.
\end{eqnarray}

De manera an\'aloga, tenemos que la probabilidad de no arribos a $Q_{1}$ en $\left[\overline{\tau}_{2}\left(n-1\right),\varphi_{1}\left(n\right)\right]$ esta dada por

\begin{eqnarray}
\prob\left\{0 \textrm{ arribos en }Q_{1}\textrm{ en el intervalo }\left[\overline{\tau}_{2}\left(n-1\right),\varphi_{1}\left(n\right)\right]\right\}
=e^{-\tilde{\mu}_{1}\xi},
\end{eqnarray}

\begin{eqnarray}
\prob\left\{0 \textrm{ arribos en }Q_{2}\textrm{ en el intervalo }\left[\overline{\tau}_{2}\left(n-1\right),\varphi_{1}\left(n\right)\right]\right\}
=e^{-\tilde{\mu}_{2}\xi}.
\end{eqnarray}

Por tanto 

\begin{eqnarray}
\begin{array}{l}
\prob\left\{0 \textrm{ arribos en }Q_{1}\textrm{ y }Q_{2}\textrm{ en el intervalo }\left[\overline{\tau}_{2}\left(n-1\right),\varphi_{1}\left(n\right)\right]\right\}\\
=\prob\left\{0 \textrm{ arribos en }Q_{1}\textrm{ en el intervalo }\left[\overline{\tau}_{2}\left(n-1\right),\varphi_{1}\left(n\right)\right]\right\}\\
\times
\prob\left\{0 \textrm{ arribos en }Q_{2}\textrm{ en el intervalo }\left[\overline{\tau}_{2}\left(n-1\right),\varphi_{1}\left(n\right)\right]\right\}=e^{-\tilde{\mu}_{1}\xi}e^{-\tilde{\mu}_{2}\xi}
=e^{-\tilde{\mu}\xi}.
\end{array}
\end{eqnarray}

Para el segundo sistema, consideremos nuevamente $\overline{\tau}_{1}\left(n\right)+r_{1}$, sin p\'erdida de generalidad podemos suponer que existe $m>0$ tal que $\overline{\tau}_{3}\left(m\right)<\overline{\tau}_{1}+r_{1}<\tau_{4}\left(m\right)$, entonces

\begin{eqnarray}
\prob\left\{0 \textrm{ arribos en }Q_{3}\textrm{ en el intervalo }\left[\overline{\tau}_{3}\left(m\right),\overline{\tau}_{1}\left(n\right)+r_{1}\right]\right\}
=e^{-\tilde{\mu}_{3}\xi_{3}},
\end{eqnarray}
donde 
\begin{eqnarray}
\xi_{3}=\overline{\tau}_{1}\left(n\right)+r_{1}-\overline{\tau}_{3}\left(m\right)=
\overline{\tau}_{1}\left(n\right)-\overline{\tau}_{3}\left(m\right)+r_{1},
\end{eqnarray}

mientras que para $Q_{4}$ al igual que con $Q_{2}$ escribiremos $\tau_{4}\left(m\right)$ en t\'erminos de $\overline{\tau}_{4}\left(m-1\right)$:

$\varphi_{2}\equiv\tau_{4}\left(m\right)=\overline{\tau}_{4}\left(m-1\right)+r_{4}+\overline{\tau}_{3}\left(m\right)
-\tau_{3}\left(m\right)+r_{3}=\overline{\tau}_{4}\left(m-1\right)+\overline{\tau}_{3}\left(m\right)
-\tau_{3}\left(m\right)+\hat{r}$, adem\'as,

$\xi_{2}\equiv\varphi_{2}\left(m\right)-\overline{\tau}_{1}-r_{1}=\overline{\tau}_{4}\left(m-1\right)+\overline{\tau}_{3}\left(m\right)
-\tau_{3}\left(m\right)-\overline{\tau}_{1}\left(n\right)+\hat{r}-r_{1}$. 

Entonces


\begin{eqnarray}
\prob\left\{0 \textrm{ arribos en }Q_{4}\textrm{ en el intervalo }\left[\overline{\tau}_{1}\left(n\right)+r_{1},\varphi_{2}\left(m\right)\right]\right\}
=e^{-\tilde{\mu}_{4}\xi_{2}},
\end{eqnarray}

mientras que para $Q_{3}$ se tiene que 

\begin{eqnarray}
\prob\left\{0 \textrm{ arribos en }Q_{3}\textrm{ en el intervalo }\left[\overline{\tau}_{1}\left(n\right)+r_{1},\varphi_{2}\left(m\right)\right]\right\}
=e^{-\tilde{\mu}_{3}\xi_{2}}
\end{eqnarray}

Por tanto

\begin{eqnarray}
\prob\left\{0 \textrm{ arribos en }Q_{3}\wedge Q_{4}\textrm{ en el intervalo }\left[\overline{\tau}_{1}\left(n\right)+r_{1},\varphi_{2}\left(m\right)\right]\right\}
=e^{-\hat{\mu}\xi_{2}}
\end{eqnarray}
donde $\hat{\mu}=\tilde{\mu}_{3}+\tilde{\mu}_{4}$.

Ahora, definamos los intervalos $\mathcal{I}_{1}=\left[\overline{\tau}_{1}\left(n\right)+r_{1},\varphi_{1}\left(n\right)\right]$  y $\mathcal{I}_{2}=\left[\overline{\tau}_{1}\left(n\right)+r_{1},\varphi_{2}\left(m\right)\right]$, entonces, sea $\mathcal{I}=\mathcal{I}_{1}\cap\mathcal{I}_{2}$ el intervalo donde cada una de las colas se encuentran vac\'ias, es decir, si tomamos $T^{*}\in\mathcal{I}$, entonces  $L_{1}\left(T^{*}\right)=L_{2}\left(T^{*}\right)=L_{3}\left(T^{*}\right)=L_{4}\left(T^{*}\right)=0$.

Ahora, dado que por construcci\'on $\mathcal{I}\neq\emptyset$ y que para $T^{*}\in\mathcal{I}$ en ninguna de las colas han llegado usuarios, se tiene que no hay transferencia entre las colas, por lo tanto, el sistema 1 y el sistema 2 son condicionalmente independientes en $\mathcal{I}$, es decir

\begin{eqnarray}
\prob\left\{L_{1}\left(T^{*}\right),L_{2}\left(T^{*}\right),L_{3}\left(T^{*}\right),L_{4}\left(T^{*}\right)|T^{*}\in\mathcal{I}\right\}=\prod_{j=1}^{4}\prob\left\{L_{j}\left(T^{*}\right)\right\},
\end{eqnarray}

para $T^{*}\in\mathcal{I}$. 

%\newpage























%________________________________________________________________________
%\section{Procesos Regenerativos}
%________________________________________________________________________

%________________________________________________________________________
%\subsection*{Procesos Regenerativos Sigman, Thorisson y Wolff \cite{Sigman1}}
%________________________________________________________________________


\begin{Def}[Definici\'on Cl\'asica]
Un proceso estoc\'astico $X=\left\{X\left(t\right):t\geq0\right\}$ es llamado regenerativo is existe una variable aleatoria $R_{1}>0$ tal que
\begin{itemize}
\item[i)] $\left\{X\left(t+R_{1}\right):t\geq0\right\}$ es independiente de $\left\{\left\{X\left(t\right):t<R_{1}\right\},\right\}$
\item[ii)] $\left\{X\left(t+R_{1}\right):t\geq0\right\}$ es estoc\'asticamente equivalente a $\left\{X\left(t\right):t>0\right\}$
\end{itemize}

Llamamos a $R_{1}$ tiempo de regeneraci\'on, y decimos que $X$ se regenera en este punto.
\end{Def}

$\left\{X\left(t+R_{1}\right)\right\}$ es regenerativo con tiempo de regeneraci\'on $R_{2}$, independiente de $R_{1}$ pero con la misma distribuci\'on que $R_{1}$. Procediendo de esta manera se obtiene una secuencia de variables aleatorias independientes e id\'enticamente distribuidas $\left\{R_{n}\right\}$ llamados longitudes de ciclo. Si definimos a $Z_{k}\equiv R_{1}+R_{2}+\cdots+R_{k}$, se tiene un proceso de renovaci\'on llamado proceso de renovaci\'on encajado para $X$.


\begin{Note}
La existencia de un primer tiempo de regeneraci\'on, $R_{1}$, implica la existencia de una sucesi\'on completa de estos tiempos $R_{1},R_{2}\ldots,$ que satisfacen la propiedad deseada \cite{Sigman2}.
\end{Note}


\begin{Note} Para la cola $GI/GI/1$ los usuarios arriban con tiempos $t_{n}$ y son atendidos con tiempos de servicio $S_{n}$, los tiempos de arribo forman un proceso de renovaci\'on  con tiempos entre arribos independientes e identicamente distribuidos (\texttt{i.i.d.})$T_{n}=t_{n}-t_{n-1}$, adem\'as los tiempos de servicio son \texttt{i.i.d.} e independientes de los procesos de arribo. Por \textit{estable} se entiende que $\esp S_{n}<\esp T_{n}<\infty$.
\end{Note}
 


\begin{Def}
Para $x$ fijo y para cada $t\geq0$, sea $I_{x}\left(t\right)=1$ si $X\left(t\right)\leq x$,  $I_{x}\left(t\right)=0$ en caso contrario, y def\'inanse los tiempos promedio
\begin{eqnarray*}
\overline{X}&=&lim_{t\rightarrow\infty}\frac{1}{t}\int_{0}^{\infty}X\left(u\right)du\\
\prob\left(X_{\infty}\leq x\right)&=&lim_{t\rightarrow\infty}\frac{1}{t}\int_{0}^{\infty}I_{x}\left(u\right)du,
\end{eqnarray*}
cuando estos l\'imites existan.
\end{Def}

Como consecuencia del teorema de Renovaci\'on-Recompensa, se tiene que el primer l\'imite  existe y es igual a la constante
\begin{eqnarray*}
\overline{X}&=&\frac{\esp\left[\int_{0}^{R_{1}}X\left(t\right)dt\right]}{\esp\left[R_{1}\right]},
\end{eqnarray*}
suponiendo que ambas esperanzas son finitas.
 
\begin{Note}
Funciones de procesos regenerativos son regenerativas, es decir, si $X\left(t\right)$ es regenerativo y se define el proceso $Y\left(t\right)$ por $Y\left(t\right)=f\left(X\left(t\right)\right)$ para alguna funci\'on Borel medible $f\left(\cdot\right)$. Adem\'as $Y$ es regenerativo con los mismos tiempos de renovaci\'on que $X$. 

En general, los tiempos de renovaci\'on, $Z_{k}$ de un proceso regenerativo no requieren ser tiempos de paro con respecto a la evoluci\'on de $X\left(t\right)$.
\end{Note} 

\begin{Note}
Una funci\'on de un proceso de Markov, usualmente no ser\'a un proceso de Markov, sin embargo ser\'a regenerativo si el proceso de Markov lo es.
\end{Note}

 
\begin{Note}
Un proceso regenerativo con media de la longitud de ciclo finita es llamado positivo recurrente.
\end{Note}


\begin{Note}
\begin{itemize}
\item[a)] Si el proceso regenerativo $X$ es positivo recurrente y tiene trayectorias muestrales no negativas, entonces la ecuaci\'on anterior es v\'alida.
\item[b)] Si $X$ es positivo recurrente regenerativo, podemos construir una \'unica versi\'on estacionaria de este proceso, $X_{e}=\left\{X_{e}\left(t\right)\right\}$, donde $X_{e}$ es un proceso estoc\'astico regenerativo y estrictamente estacionario, con distribuci\'on marginal distribuida como $X_{\infty}$
\end{itemize}
\end{Note}


%__________________________________________________________________________________________
%\subsection*{Procesos Regenerativos Estacionarios - Stidham \cite{Stidham}}
%__________________________________________________________________________________________


Un proceso estoc\'astico a tiempo continuo $\left\{V\left(t\right),t\geq0\right\}$ es un proceso regenerativo si existe una sucesi\'on de variables aleatorias independientes e id\'enticamente distribuidas $\left\{X_{1},X_{2},\ldots\right\}$, sucesi\'on de renovaci\'on, tal que para cualquier conjunto de Borel $A$, 

\begin{eqnarray*}
\prob\left\{V\left(t\right)\in A|X_{1}+X_{2}+\cdots+X_{R\left(t\right)}=s,\left\{V\left(\tau\right),\tau<s\right\}\right\}=\prob\left\{V\left(t-s\right)\in A|X_{1}>t-s\right\},
\end{eqnarray*}
para todo $0\leq s\leq t$, donde $R\left(t\right)=\max\left\{X_{1}+X_{2}+\cdots+X_{j}\leq t\right\}=$n\'umero de renovaciones ({\emph{puntos de regeneraci\'on}}) que ocurren en $\left[0,t\right]$. El intervalo $\left[0,X_{1}\right)$ es llamado {\emph{primer ciclo de regeneraci\'on}} de $\left\{V\left(t \right),t\geq0\right\}$, $\left[X_{1},X_{1}+X_{2}\right)$ el {\emph{segundo ciclo de regeneraci\'on}}, y as\'i sucesivamente.

Sea $X=X_{1}$ y sea $F$ la funci\'on de distrbuci\'on de $X$


\begin{Def}
Se define el proceso estacionario, $\left\{V^{*}\left(t\right),t\geq0\right\}$, para $\left\{V\left(t\right),t\geq0\right\}$ por

\begin{eqnarray*}
\prob\left\{V\left(t\right)\in A\right\}=\frac{1}{\esp\left[X\right]}\int_{0}^{\infty}\prob\left\{V\left(t+x\right)\in A|X>x\right\}\left(1-F\left(x\right)\right)dx,
\end{eqnarray*} 
para todo $t\geq0$ y todo conjunto de Borel $A$.
\end{Def}

\begin{Def}
Una distribuci\'on se dice que es {\emph{aritm\'etica}} si todos sus puntos de incremento son m\'ultiplos de la forma $0,\lambda, 2\lambda,\ldots$ para alguna $\lambda>0$ entera.
\end{Def}


\begin{Def}
Una modificaci\'on medible de un proceso $\left\{V\left(t\right),t\geq0\right\}$, es una versi\'on de este, $\left\{V\left(t,w\right)\right\}$ conjuntamente medible para $t\geq0$ y para $w\in S$, $S$ espacio de estados para $\left\{V\left(t\right),t\geq0\right\}$.
\end{Def}

\begin{Teo}
Sea $\left\{V\left(t\right),t\geq\right\}$ un proceso regenerativo no negativo con modificaci\'on medible. Sea $\esp\left[X\right]<\infty$. Entonces el proceso estacionario dado por la ecuaci\'on anterior est\'a bien definido y tiene funci\'on de distribuci\'on independiente de $t$, adem\'as
\begin{itemize}
\item[i)] \begin{eqnarray*}
\esp\left[V^{*}\left(0\right)\right]&=&\frac{\esp\left[\int_{0}^{X}V\left(s\right)ds\right]}{\esp\left[X\right]}\end{eqnarray*}
\item[ii)] Si $\esp\left[V^{*}\left(0\right)\right]<\infty$, equivalentemente, si $\esp\left[\int_{0}^{X}V\left(s\right)ds\right]<\infty$,entonces
\begin{eqnarray*}
\frac{\int_{0}^{t}V\left(s\right)ds}{t}\rightarrow\frac{\esp\left[\int_{0}^{X}V\left(s\right)ds\right]}{\esp\left[X\right]}
\end{eqnarray*}
con probabilidad 1 y en media, cuando $t\rightarrow\infty$.
\end{itemize}
\end{Teo}

\begin{Coro}
Sea $\left\{V\left(t\right),t\geq0\right\}$ un proceso regenerativo no negativo, con modificaci\'on medible. Si $\esp <\infty$, $F$ es no-aritm\'etica, y para todo $x\geq0$, $P\left\{V\left(t\right)\leq x,C>x\right\}$ es de variaci\'on acotada como funci\'on de $t$ en cada intervalo finito $\left[0,\tau\right]$, entonces $V\left(t\right)$ converge en distribuci\'on  cuando $t\rightarrow\infty$ y $$\esp V=\frac{\esp \int_{0}^{X}V\left(s\right)ds}{\esp X}$$
Donde $V$ tiene la distribuci\'on l\'imite de $V\left(t\right)$ cuando $t\rightarrow\infty$.

\end{Coro}

Para el caso discreto se tienen resultados similares.



%______________________________________________________________________
%\section{Procesos de Renovaci\'on}
%______________________________________________________________________

\begin{Def}\label{Def.Tn}
Sean $0\leq T_{1}\leq T_{2}\leq \ldots$ son tiempos aleatorios infinitos en los cuales ocurren ciertos eventos. El n\'umero de tiempos $T_{n}$ en el intervalo $\left[0,t\right)$ es

\begin{eqnarray}
N\left(t\right)=\sum_{n=1}^{\infty}\indora\left(T_{n}\leq t\right),
\end{eqnarray}
para $t\geq0$.
\end{Def}

Si se consideran los puntos $T_{n}$ como elementos de $\rea_{+}$, y $N\left(t\right)$ es el n\'umero de puntos en $\rea$. El proceso denotado por $\left\{N\left(t\right):t\geq0\right\}$, denotado por $N\left(t\right)$, es un proceso puntual en $\rea_{+}$. Los $T_{n}$ son los tiempos de ocurrencia, el proceso puntual $N\left(t\right)$ es simple si su n\'umero de ocurrencias son distintas: $0<T_{1}<T_{2}<\ldots$ casi seguramente.

\begin{Def}
Un proceso puntual $N\left(t\right)$ es un proceso de renovaci\'on si los tiempos de interocurrencia $\xi_{n}=T_{n}-T_{n-1}$, para $n\geq1$, son independientes e identicamente distribuidos con distribuci\'on $F$, donde $F\left(0\right)=0$ y $T_{0}=0$. Los $T_{n}$ son llamados tiempos de renovaci\'on, referente a la independencia o renovaci\'on de la informaci\'on estoc\'astica en estos tiempos. Los $\xi_{n}$ son los tiempos de inter-renovaci\'on, y $N\left(t\right)$ es el n\'umero de renovaciones en el intervalo $\left[0,t\right)$
\end{Def}


\begin{Note}
Para definir un proceso de renovaci\'on para cualquier contexto, solamente hay que especificar una distribuci\'on $F$, con $F\left(0\right)=0$, para los tiempos de inter-renovaci\'on. La funci\'on $F$ en turno degune las otra variables aleatorias. De manera formal, existe un espacio de probabilidad y una sucesi\'on de variables aleatorias $\xi_{1},\xi_{2},\ldots$ definidas en este con distribuci\'on $F$. Entonces las otras cantidades son $T_{n}=\sum_{k=1}^{n}\xi_{k}$ y $N\left(t\right)=\sum_{n=1}^{\infty}\indora\left(T_{n}\leq t\right)$, donde $T_{n}\rightarrow\infty$ casi seguramente por la Ley Fuerte de los Grandes Números.
\end{Note}

%___________________________________________________________________________________________
%
%\subsection*{Teorema Principal de Renovaci\'on}
%___________________________________________________________________________________________
%

\begin{Note} Una funci\'on $h:\rea_{+}\rightarrow\rea$ es Directamente Riemann Integrable en los siguientes casos:
\begin{itemize}
\item[a)] $h\left(t\right)\geq0$ es decreciente y Riemann Integrable.
\item[b)] $h$ es continua excepto posiblemente en un conjunto de Lebesgue de medida 0, y $|h\left(t\right)|\leq b\left(t\right)$, donde $b$ es DRI.
\end{itemize}
\end{Note}

\begin{Teo}[Teorema Principal de Renovaci\'on]
Si $F$ es no aritm\'etica y $h\left(t\right)$ es Directamente Riemann Integrable (DRI), entonces

\begin{eqnarray*}
lim_{t\rightarrow\infty}U\star h=\frac{1}{\mu}\int_{\rea_{+}}h\left(s\right)ds.
\end{eqnarray*}
\end{Teo}

\begin{Prop}
Cualquier funci\'on $H\left(t\right)$ acotada en intervalos finitos y que es 0 para $t<0$ puede expresarse como
\begin{eqnarray*}
H\left(t\right)=U\star h\left(t\right)\textrm{,  donde }h\left(t\right)=H\left(t\right)-F\star H\left(t\right)
\end{eqnarray*}
\end{Prop}

\begin{Def}
Un proceso estoc\'astico $X\left(t\right)$ es crudamente regenerativo en un tiempo aleatorio positivo $T$ si
\begin{eqnarray*}
\esp\left[X\left(T+t\right)|T\right]=\esp\left[X\left(t\right)\right]\textrm{, para }t\geq0,\end{eqnarray*}
y con las esperanzas anteriores finitas.
\end{Def}

\begin{Prop}
Sup\'ongase que $X\left(t\right)$ es un proceso crudamente regenerativo en $T$, que tiene distribuci\'on $F$. Si $\esp\left[X\left(t\right)\right]$ es acotado en intervalos finitos, entonces
\begin{eqnarray*}
\esp\left[X\left(t\right)\right]=U\star h\left(t\right)\textrm{,  donde }h\left(t\right)=\esp\left[X\left(t\right)\indora\left(T>t\right)\right].
\end{eqnarray*}
\end{Prop}

\begin{Teo}[Regeneraci\'on Cruda]
Sup\'ongase que $X\left(t\right)$ es un proceso con valores positivo crudamente regenerativo en $T$, y def\'inase $M=\sup\left\{|X\left(t\right)|:t\leq T\right\}$. Si $T$ es no aritm\'etico y $M$ y $MT$ tienen media finita, entonces
\begin{eqnarray*}
lim_{t\rightarrow\infty}\esp\left[X\left(t\right)\right]=\frac{1}{\mu}\int_{\rea_{+}}h\left(s\right)ds,
\end{eqnarray*}
donde $h\left(t\right)=\esp\left[X\left(t\right)\indora\left(T>t\right)\right]$.
\end{Teo}

%___________________________________________________________________________________________
%
%\subsection*{Propiedades de los Procesos de Renovaci\'on}
%___________________________________________________________________________________________
%

Los tiempos $T_{n}$ est\'an relacionados con los conteos de $N\left(t\right)$ por

\begin{eqnarray*}
\left\{N\left(t\right)\geq n\right\}&=&\left\{T_{n}\leq t\right\}\\
T_{N\left(t\right)}\leq &t&<T_{N\left(t\right)+1},
\end{eqnarray*}

adem\'as $N\left(T_{n}\right)=n$, y 

\begin{eqnarray*}
N\left(t\right)=\max\left\{n:T_{n}\leq t\right\}=\min\left\{n:T_{n+1}>t\right\}
\end{eqnarray*}

Por propiedades de la convoluci\'on se sabe que

\begin{eqnarray*}
P\left\{T_{n}\leq t\right\}=F^{n\star}\left(t\right)
\end{eqnarray*}
que es la $n$-\'esima convoluci\'on de $F$. Entonces 

\begin{eqnarray*}
\left\{N\left(t\right)\geq n\right\}&=&\left\{T_{n}\leq t\right\}\\
P\left\{N\left(t\right)\leq n\right\}&=&1-F^{\left(n+1\right)\star}\left(t\right)
\end{eqnarray*}

Adem\'as usando el hecho de que $\esp\left[N\left(t\right)\right]=\sum_{n=1}^{\infty}P\left\{N\left(t\right)\geq n\right\}$
se tiene que

\begin{eqnarray*}
\esp\left[N\left(t\right)\right]=\sum_{n=1}^{\infty}F^{n\star}\left(t\right)
\end{eqnarray*}

\begin{Prop}
Para cada $t\geq0$, la funci\'on generadora de momentos $\esp\left[e^{\alpha N\left(t\right)}\right]$ existe para alguna $\alpha$ en una vecindad del 0, y de aqu\'i que $\esp\left[N\left(t\right)^{m}\right]<\infty$, para $m\geq1$.
\end{Prop}


\begin{Note}
Si el primer tiempo de renovaci\'on $\xi_{1}$ no tiene la misma distribuci\'on que el resto de las $\xi_{n}$, para $n\geq2$, a $N\left(t\right)$ se le llama Proceso de Renovaci\'on retardado, donde si $\xi$ tiene distribuci\'on $G$, entonces el tiempo $T_{n}$ de la $n$-\'esima renovaci\'on tiene distribuci\'on $G\star F^{\left(n-1\right)\star}\left(t\right)$
\end{Note}


\begin{Teo}
Para una constante $\mu\leq\infty$ ( o variable aleatoria), las siguientes expresiones son equivalentes:

\begin{eqnarray}
lim_{n\rightarrow\infty}n^{-1}T_{n}&=&\mu,\textrm{ c.s.}\\
lim_{t\rightarrow\infty}t^{-1}N\left(t\right)&=&1/\mu,\textrm{ c.s.}
\end{eqnarray}
\end{Teo}


Es decir, $T_{n}$ satisface la Ley Fuerte de los Grandes N\'umeros s\'i y s\'olo s\'i $N\left/t\right)$ la cumple.


\begin{Coro}[Ley Fuerte de los Grandes N\'umeros para Procesos de Renovaci\'on]
Si $N\left(t\right)$ es un proceso de renovaci\'on cuyos tiempos de inter-renovaci\'on tienen media $\mu\leq\infty$, entonces
\begin{eqnarray}
t^{-1}N\left(t\right)\rightarrow 1/\mu,\textrm{ c.s. cuando }t\rightarrow\infty.
\end{eqnarray}

\end{Coro}


Considerar el proceso estoc\'astico de valores reales $\left\{Z\left(t\right):t\geq0\right\}$ en el mismo espacio de probabilidad que $N\left(t\right)$

\begin{Def}
Para el proceso $\left\{Z\left(t\right):t\geq0\right\}$ se define la fluctuaci\'on m\'axima de $Z\left(t\right)$ en el intervalo $\left(T_{n-1},T_{n}\right]$:
\begin{eqnarray*}
M_{n}=\sup_{T_{n-1}<t\leq T_{n}}|Z\left(t\right)-Z\left(T_{n-1}\right)|
\end{eqnarray*}
\end{Def}

\begin{Teo}
Sup\'ongase que $n^{-1}T_{n}\rightarrow\mu$ c.s. cuando $n\rightarrow\infty$, donde $\mu\leq\infty$ es una constante o variable aleatoria. Sea $a$ una constante o variable aleatoria que puede ser infinita cuando $\mu$ es finita, y considere las expresiones l\'imite:
\begin{eqnarray}
lim_{n\rightarrow\infty}n^{-1}Z\left(T_{n}\right)&=&a,\textrm{ c.s.}\\
lim_{t\rightarrow\infty}t^{-1}Z\left(t\right)&=&a/\mu,\textrm{ c.s.}
\end{eqnarray}
La segunda expresi\'on implica la primera. Conversamente, la primera implica la segunda si el proceso $Z\left(t\right)$ es creciente, o si $lim_{n\rightarrow\infty}n^{-1}M_{n}=0$ c.s.
\end{Teo}

\begin{Coro}
Si $N\left(t\right)$ es un proceso de renovaci\'on, y $\left(Z\left(T_{n}\right)-Z\left(T_{n-1}\right),M_{n}\right)$, para $n\geq1$, son variables aleatorias independientes e id\'enticamente distribuidas con media finita, entonces,
\begin{eqnarray}
lim_{t\rightarrow\infty}t^{-1}Z\left(t\right)\rightarrow\frac{\esp\left[Z\left(T_{1}\right)-Z\left(T_{0}\right)\right]}{\esp\left[T_{1}\right]},\textrm{ c.s. cuando  }t\rightarrow\infty.
\end{eqnarray}
\end{Coro}



%___________________________________________________________________________________________
%
%\subsection{Propiedades de los Procesos de Renovaci\'on}
%___________________________________________________________________________________________
%

Los tiempos $T_{n}$ est\'an relacionados con los conteos de $N\left(t\right)$ por

\begin{eqnarray*}
\left\{N\left(t\right)\geq n\right\}&=&\left\{T_{n}\leq t\right\}\\
T_{N\left(t\right)}\leq &t&<T_{N\left(t\right)+1},
\end{eqnarray*}

adem\'as $N\left(T_{n}\right)=n$, y 

\begin{eqnarray*}
N\left(t\right)=\max\left\{n:T_{n}\leq t\right\}=\min\left\{n:T_{n+1}>t\right\}
\end{eqnarray*}

Por propiedades de la convoluci\'on se sabe que

\begin{eqnarray*}
P\left\{T_{n}\leq t\right\}=F^{n\star}\left(t\right)
\end{eqnarray*}
que es la $n$-\'esima convoluci\'on de $F$. Entonces 

\begin{eqnarray*}
\left\{N\left(t\right)\geq n\right\}&=&\left\{T_{n}\leq t\right\}\\
P\left\{N\left(t\right)\leq n\right\}&=&1-F^{\left(n+1\right)\star}\left(t\right)
\end{eqnarray*}

Adem\'as usando el hecho de que $\esp\left[N\left(t\right)\right]=\sum_{n=1}^{\infty}P\left\{N\left(t\right)\geq n\right\}$
se tiene que

\begin{eqnarray*}
\esp\left[N\left(t\right)\right]=\sum_{n=1}^{\infty}F^{n\star}\left(t\right)
\end{eqnarray*}

\begin{Prop}
Para cada $t\geq0$, la funci\'on generadora de momentos $\esp\left[e^{\alpha N\left(t\right)}\right]$ existe para alguna $\alpha$ en una vecindad del 0, y de aqu\'i que $\esp\left[N\left(t\right)^{m}\right]<\infty$, para $m\geq1$.
\end{Prop}


\begin{Note}
Si el primer tiempo de renovaci\'on $\xi_{1}$ no tiene la misma distribuci\'on que el resto de las $\xi_{n}$, para $n\geq2$, a $N\left(t\right)$ se le llama Proceso de Renovaci\'on retardado, donde si $\xi$ tiene distribuci\'on $G$, entonces el tiempo $T_{n}$ de la $n$-\'esima renovaci\'on tiene distribuci\'on $G\star F^{\left(n-1\right)\star}\left(t\right)$
\end{Note}


\begin{Teo}
Para una constante $\mu\leq\infty$ ( o variable aleatoria), las siguientes expresiones son equivalentes:

\begin{eqnarray}
lim_{n\rightarrow\infty}n^{-1}T_{n}&=&\mu,\textrm{ c.s.}\\
lim_{t\rightarrow\infty}t^{-1}N\left(t\right)&=&1/\mu,\textrm{ c.s.}
\end{eqnarray}
\end{Teo}


Es decir, $T_{n}$ satisface la Ley Fuerte de los Grandes N\'umeros s\'i y s\'olo s\'i $N\left/t\right)$ la cumple.


\begin{Coro}[Ley Fuerte de los Grandes N\'umeros para Procesos de Renovaci\'on]
Si $N\left(t\right)$ es un proceso de renovaci\'on cuyos tiempos de inter-renovaci\'on tienen media $\mu\leq\infty$, entonces
\begin{eqnarray}
t^{-1}N\left(t\right)\rightarrow 1/\mu,\textrm{ c.s. cuando }t\rightarrow\infty.
\end{eqnarray}

\end{Coro}


Considerar el proceso estoc\'astico de valores reales $\left\{Z\left(t\right):t\geq0\right\}$ en el mismo espacio de probabilidad que $N\left(t\right)$

\begin{Def}
Para el proceso $\left\{Z\left(t\right):t\geq0\right\}$ se define la fluctuaci\'on m\'axima de $Z\left(t\right)$ en el intervalo $\left(T_{n-1},T_{n}\right]$:
\begin{eqnarray*}
M_{n}=\sup_{T_{n-1}<t\leq T_{n}}|Z\left(t\right)-Z\left(T_{n-1}\right)|
\end{eqnarray*}
\end{Def}

\begin{Teo}
Sup\'ongase que $n^{-1}T_{n}\rightarrow\mu$ c.s. cuando $n\rightarrow\infty$, donde $\mu\leq\infty$ es una constante o variable aleatoria. Sea $a$ una constante o variable aleatoria que puede ser infinita cuando $\mu$ es finita, y considere las expresiones l\'imite:
\begin{eqnarray}
lim_{n\rightarrow\infty}n^{-1}Z\left(T_{n}\right)&=&a,\textrm{ c.s.}\\
lim_{t\rightarrow\infty}t^{-1}Z\left(t\right)&=&a/\mu,\textrm{ c.s.}
\end{eqnarray}
La segunda expresi\'on implica la primera. Conversamente, la primera implica la segunda si el proceso $Z\left(t\right)$ es creciente, o si $lim_{n\rightarrow\infty}n^{-1}M_{n}=0$ c.s.
\end{Teo}

\begin{Coro}
Si $N\left(t\right)$ es un proceso de renovaci\'on, y $\left(Z\left(T_{n}\right)-Z\left(T_{n-1}\right),M_{n}\right)$, para $n\geq1$, son variables aleatorias independientes e id\'enticamente distribuidas con media finita, entonces,
\begin{eqnarray}
lim_{t\rightarrow\infty}t^{-1}Z\left(t\right)\rightarrow\frac{\esp\left[Z\left(T_{1}\right)-Z\left(T_{0}\right)\right]}{\esp\left[T_{1}\right]},\textrm{ c.s. cuando  }t\rightarrow\infty.
\end{eqnarray}
\end{Coro}


%___________________________________________________________________________________________
%
%\subsection{Propiedades de los Procesos de Renovaci\'on}
%___________________________________________________________________________________________
%

Los tiempos $T_{n}$ est\'an relacionados con los conteos de $N\left(t\right)$ por

\begin{eqnarray*}
\left\{N\left(t\right)\geq n\right\}&=&\left\{T_{n}\leq t\right\}\\
T_{N\left(t\right)}\leq &t&<T_{N\left(t\right)+1},
\end{eqnarray*}

adem\'as $N\left(T_{n}\right)=n$, y 

\begin{eqnarray*}
N\left(t\right)=\max\left\{n:T_{n}\leq t\right\}=\min\left\{n:T_{n+1}>t\right\}
\end{eqnarray*}

Por propiedades de la convoluci\'on se sabe que

\begin{eqnarray*}
P\left\{T_{n}\leq t\right\}=F^{n\star}\left(t\right)
\end{eqnarray*}
que es la $n$-\'esima convoluci\'on de $F$. Entonces 

\begin{eqnarray*}
\left\{N\left(t\right)\geq n\right\}&=&\left\{T_{n}\leq t\right\}\\
P\left\{N\left(t\right)\leq n\right\}&=&1-F^{\left(n+1\right)\star}\left(t\right)
\end{eqnarray*}

Adem\'as usando el hecho de que $\esp\left[N\left(t\right)\right]=\sum_{n=1}^{\infty}P\left\{N\left(t\right)\geq n\right\}$
se tiene que

\begin{eqnarray*}
\esp\left[N\left(t\right)\right]=\sum_{n=1}^{\infty}F^{n\star}\left(t\right)
\end{eqnarray*}

\begin{Prop}
Para cada $t\geq0$, la funci\'on generadora de momentos $\esp\left[e^{\alpha N\left(t\right)}\right]$ existe para alguna $\alpha$ en una vecindad del 0, y de aqu\'i que $\esp\left[N\left(t\right)^{m}\right]<\infty$, para $m\geq1$.
\end{Prop}


\begin{Note}
Si el primer tiempo de renovaci\'on $\xi_{1}$ no tiene la misma distribuci\'on que el resto de las $\xi_{n}$, para $n\geq2$, a $N\left(t\right)$ se le llama Proceso de Renovaci\'on retardado, donde si $\xi$ tiene distribuci\'on $G$, entonces el tiempo $T_{n}$ de la $n$-\'esima renovaci\'on tiene distribuci\'on $G\star F^{\left(n-1\right)\star}\left(t\right)$
\end{Note}


\begin{Teo}
Para una constante $\mu\leq\infty$ ( o variable aleatoria), las siguientes expresiones son equivalentes:

\begin{eqnarray}
lim_{n\rightarrow\infty}n^{-1}T_{n}&=&\mu,\textrm{ c.s.}\\
lim_{t\rightarrow\infty}t^{-1}N\left(t\right)&=&1/\mu,\textrm{ c.s.}
\end{eqnarray}
\end{Teo}


Es decir, $T_{n}$ satisface la Ley Fuerte de los Grandes N\'umeros s\'i y s\'olo s\'i $N\left/t\right)$ la cumple.


\begin{Coro}[Ley Fuerte de los Grandes N\'umeros para Procesos de Renovaci\'on]
Si $N\left(t\right)$ es un proceso de renovaci\'on cuyos tiempos de inter-renovaci\'on tienen media $\mu\leq\infty$, entonces
\begin{eqnarray}
t^{-1}N\left(t\right)\rightarrow 1/\mu,\textrm{ c.s. cuando }t\rightarrow\infty.
\end{eqnarray}

\end{Coro}


Considerar el proceso estoc\'astico de valores reales $\left\{Z\left(t\right):t\geq0\right\}$ en el mismo espacio de probabilidad que $N\left(t\right)$

\begin{Def}
Para el proceso $\left\{Z\left(t\right):t\geq0\right\}$ se define la fluctuaci\'on m\'axima de $Z\left(t\right)$ en el intervalo $\left(T_{n-1},T_{n}\right]$:
\begin{eqnarray*}
M_{n}=\sup_{T_{n-1}<t\leq T_{n}}|Z\left(t\right)-Z\left(T_{n-1}\right)|
\end{eqnarray*}
\end{Def}

\begin{Teo}
Sup\'ongase que $n^{-1}T_{n}\rightarrow\mu$ c.s. cuando $n\rightarrow\infty$, donde $\mu\leq\infty$ es una constante o variable aleatoria. Sea $a$ una constante o variable aleatoria que puede ser infinita cuando $\mu$ es finita, y considere las expresiones l\'imite:
\begin{eqnarray}
lim_{n\rightarrow\infty}n^{-1}Z\left(T_{n}\right)&=&a,\textrm{ c.s.}\\
lim_{t\rightarrow\infty}t^{-1}Z\left(t\right)&=&a/\mu,\textrm{ c.s.}
\end{eqnarray}
La segunda expresi\'on implica la primera. Conversamente, la primera implica la segunda si el proceso $Z\left(t\right)$ es creciente, o si $lim_{n\rightarrow\infty}n^{-1}M_{n}=0$ c.s.
\end{Teo}

\begin{Coro}
Si $N\left(t\right)$ es un proceso de renovaci\'on, y $\left(Z\left(T_{n}\right)-Z\left(T_{n-1}\right),M_{n}\right)$, para $n\geq1$, son variables aleatorias independientes e id\'enticamente distribuidas con media finita, entonces,
\begin{eqnarray}
lim_{t\rightarrow\infty}t^{-1}Z\left(t\right)\rightarrow\frac{\esp\left[Z\left(T_{1}\right)-Z\left(T_{0}\right)\right]}{\esp\left[T_{1}\right]},\textrm{ c.s. cuando  }t\rightarrow\infty.
\end{eqnarray}
\end{Coro}

%___________________________________________________________________________________________
%
%\subsection{Propiedades de los Procesos de Renovaci\'on}
%___________________________________________________________________________________________
%

Los tiempos $T_{n}$ est\'an relacionados con los conteos de $N\left(t\right)$ por

\begin{eqnarray*}
\left\{N\left(t\right)\geq n\right\}&=&\left\{T_{n}\leq t\right\}\\
T_{N\left(t\right)}\leq &t&<T_{N\left(t\right)+1},
\end{eqnarray*}

adem\'as $N\left(T_{n}\right)=n$, y 

\begin{eqnarray*}
N\left(t\right)=\max\left\{n:T_{n}\leq t\right\}=\min\left\{n:T_{n+1}>t\right\}
\end{eqnarray*}

Por propiedades de la convoluci\'on se sabe que

\begin{eqnarray*}
P\left\{T_{n}\leq t\right\}=F^{n\star}\left(t\right)
\end{eqnarray*}
que es la $n$-\'esima convoluci\'on de $F$. Entonces 

\begin{eqnarray*}
\left\{N\left(t\right)\geq n\right\}&=&\left\{T_{n}\leq t\right\}\\
P\left\{N\left(t\right)\leq n\right\}&=&1-F^{\left(n+1\right)\star}\left(t\right)
\end{eqnarray*}

Adem\'as usando el hecho de que $\esp\left[N\left(t\right)\right]=\sum_{n=1}^{\infty}P\left\{N\left(t\right)\geq n\right\}$
se tiene que

\begin{eqnarray*}
\esp\left[N\left(t\right)\right]=\sum_{n=1}^{\infty}F^{n\star}\left(t\right)
\end{eqnarray*}

\begin{Prop}
Para cada $t\geq0$, la funci\'on generadora de momentos $\esp\left[e^{\alpha N\left(t\right)}\right]$ existe para alguna $\alpha$ en una vecindad del 0, y de aqu\'i que $\esp\left[N\left(t\right)^{m}\right]<\infty$, para $m\geq1$.
\end{Prop}


\begin{Note}
Si el primer tiempo de renovaci\'on $\xi_{1}$ no tiene la misma distribuci\'on que el resto de las $\xi_{n}$, para $n\geq2$, a $N\left(t\right)$ se le llama Proceso de Renovaci\'on retardado, donde si $\xi$ tiene distribuci\'on $G$, entonces el tiempo $T_{n}$ de la $n$-\'esima renovaci\'on tiene distribuci\'on $G\star F^{\left(n-1\right)\star}\left(t\right)$
\end{Note}


\begin{Teo}
Para una constante $\mu\leq\infty$ ( o variable aleatoria), las siguientes expresiones son equivalentes:

\begin{eqnarray}
lim_{n\rightarrow\infty}n^{-1}T_{n}&=&\mu,\textrm{ c.s.}\\
lim_{t\rightarrow\infty}t^{-1}N\left(t\right)&=&1/\mu,\textrm{ c.s.}
\end{eqnarray}
\end{Teo}


Es decir, $T_{n}$ satisface la Ley Fuerte de los Grandes N\'umeros s\'i y s\'olo s\'i $N\left/t\right)$ la cumple.


\begin{Coro}[Ley Fuerte de los Grandes N\'umeros para Procesos de Renovaci\'on]
Si $N\left(t\right)$ es un proceso de renovaci\'on cuyos tiempos de inter-renovaci\'on tienen media $\mu\leq\infty$, entonces
\begin{eqnarray}
t^{-1}N\left(t\right)\rightarrow 1/\mu,\textrm{ c.s. cuando }t\rightarrow\infty.
\end{eqnarray}

\end{Coro}


Considerar el proceso estoc\'astico de valores reales $\left\{Z\left(t\right):t\geq0\right\}$ en el mismo espacio de probabilidad que $N\left(t\right)$

\begin{Def}
Para el proceso $\left\{Z\left(t\right):t\geq0\right\}$ se define la fluctuaci\'on m\'axima de $Z\left(t\right)$ en el intervalo $\left(T_{n-1},T_{n}\right]$:
\begin{eqnarray*}
M_{n}=\sup_{T_{n-1}<t\leq T_{n}}|Z\left(t\right)-Z\left(T_{n-1}\right)|
\end{eqnarray*}
\end{Def}

\begin{Teo}
Sup\'ongase que $n^{-1}T_{n}\rightarrow\mu$ c.s. cuando $n\rightarrow\infty$, donde $\mu\leq\infty$ es una constante o variable aleatoria. Sea $a$ una constante o variable aleatoria que puede ser infinita cuando $\mu$ es finita, y considere las expresiones l\'imite:
\begin{eqnarray}
lim_{n\rightarrow\infty}n^{-1}Z\left(T_{n}\right)&=&a,\textrm{ c.s.}\\
lim_{t\rightarrow\infty}t^{-1}Z\left(t\right)&=&a/\mu,\textrm{ c.s.}
\end{eqnarray}
La segunda expresi\'on implica la primera. Conversamente, la primera implica la segunda si el proceso $Z\left(t\right)$ es creciente, o si $lim_{n\rightarrow\infty}n^{-1}M_{n}=0$ c.s.
\end{Teo}

\begin{Coro}
Si $N\left(t\right)$ es un proceso de renovaci\'on, y $\left(Z\left(T_{n}\right)-Z\left(T_{n-1}\right),M_{n}\right)$, para $n\geq1$, son variables aleatorias independientes e id\'enticamente distribuidas con media finita, entonces,
\begin{eqnarray}
lim_{t\rightarrow\infty}t^{-1}Z\left(t\right)\rightarrow\frac{\esp\left[Z\left(T_{1}\right)-Z\left(T_{0}\right)\right]}{\esp\left[T_{1}\right]},\textrm{ c.s. cuando  }t\rightarrow\infty.
\end{eqnarray}
\end{Coro}
%___________________________________________________________________________________________
%
%\subsection{Propiedades de los Procesos de Renovaci\'on}
%___________________________________________________________________________________________
%

Los tiempos $T_{n}$ est\'an relacionados con los conteos de $N\left(t\right)$ por

\begin{eqnarray*}
\left\{N\left(t\right)\geq n\right\}&=&\left\{T_{n}\leq t\right\}\\
T_{N\left(t\right)}\leq &t&<T_{N\left(t\right)+1},
\end{eqnarray*}

adem\'as $N\left(T_{n}\right)=n$, y 

\begin{eqnarray*}
N\left(t\right)=\max\left\{n:T_{n}\leq t\right\}=\min\left\{n:T_{n+1}>t\right\}
\end{eqnarray*}

Por propiedades de la convoluci\'on se sabe que

\begin{eqnarray*}
P\left\{T_{n}\leq t\right\}=F^{n\star}\left(t\right)
\end{eqnarray*}
que es la $n$-\'esima convoluci\'on de $F$. Entonces 

\begin{eqnarray*}
\left\{N\left(t\right)\geq n\right\}&=&\left\{T_{n}\leq t\right\}\\
P\left\{N\left(t\right)\leq n\right\}&=&1-F^{\left(n+1\right)\star}\left(t\right)
\end{eqnarray*}

Adem\'as usando el hecho de que $\esp\left[N\left(t\right)\right]=\sum_{n=1}^{\infty}P\left\{N\left(t\right)\geq n\right\}$
se tiene que

\begin{eqnarray*}
\esp\left[N\left(t\right)\right]=\sum_{n=1}^{\infty}F^{n\star}\left(t\right)
\end{eqnarray*}

\begin{Prop}
Para cada $t\geq0$, la funci\'on generadora de momentos $\esp\left[e^{\alpha N\left(t\right)}\right]$ existe para alguna $\alpha$ en una vecindad del 0, y de aqu\'i que $\esp\left[N\left(t\right)^{m}\right]<\infty$, para $m\geq1$.
\end{Prop}


\begin{Note}
Si el primer tiempo de renovaci\'on $\xi_{1}$ no tiene la misma distribuci\'on que el resto de las $\xi_{n}$, para $n\geq2$, a $N\left(t\right)$ se le llama Proceso de Renovaci\'on retardado, donde si $\xi$ tiene distribuci\'on $G$, entonces el tiempo $T_{n}$ de la $n$-\'esima renovaci\'on tiene distribuci\'on $G\star F^{\left(n-1\right)\star}\left(t\right)$
\end{Note}


\begin{Teo}
Para una constante $\mu\leq\infty$ ( o variable aleatoria), las siguientes expresiones son equivalentes:

\begin{eqnarray}
lim_{n\rightarrow\infty}n^{-1}T_{n}&=&\mu,\textrm{ c.s.}\\
lim_{t\rightarrow\infty}t^{-1}N\left(t\right)&=&1/\mu,\textrm{ c.s.}
\end{eqnarray}
\end{Teo}


Es decir, $T_{n}$ satisface la Ley Fuerte de los Grandes N\'umeros s\'i y s\'olo s\'i $N\left/t\right)$ la cumple.


\begin{Coro}[Ley Fuerte de los Grandes N\'umeros para Procesos de Renovaci\'on]
Si $N\left(t\right)$ es un proceso de renovaci\'on cuyos tiempos de inter-renovaci\'on tienen media $\mu\leq\infty$, entonces
\begin{eqnarray}
t^{-1}N\left(t\right)\rightarrow 1/\mu,\textrm{ c.s. cuando }t\rightarrow\infty.
\end{eqnarray}

\end{Coro}


Considerar el proceso estoc\'astico de valores reales $\left\{Z\left(t\right):t\geq0\right\}$ en el mismo espacio de probabilidad que $N\left(t\right)$

\begin{Def}
Para el proceso $\left\{Z\left(t\right):t\geq0\right\}$ se define la fluctuaci\'on m\'axima de $Z\left(t\right)$ en el intervalo $\left(T_{n-1},T_{n}\right]$:
\begin{eqnarray*}
M_{n}=\sup_{T_{n-1}<t\leq T_{n}}|Z\left(t\right)-Z\left(T_{n-1}\right)|
\end{eqnarray*}
\end{Def}

\begin{Teo}
Sup\'ongase que $n^{-1}T_{n}\rightarrow\mu$ c.s. cuando $n\rightarrow\infty$, donde $\mu\leq\infty$ es una constante o variable aleatoria. Sea $a$ una constante o variable aleatoria que puede ser infinita cuando $\mu$ es finita, y considere las expresiones l\'imite:
\begin{eqnarray}
lim_{n\rightarrow\infty}n^{-1}Z\left(T_{n}\right)&=&a,\textrm{ c.s.}\\
lim_{t\rightarrow\infty}t^{-1}Z\left(t\right)&=&a/\mu,\textrm{ c.s.}
\end{eqnarray}
La segunda expresi\'on implica la primera. Conversamente, la primera implica la segunda si el proceso $Z\left(t\right)$ es creciente, o si $lim_{n\rightarrow\infty}n^{-1}M_{n}=0$ c.s.
\end{Teo}

\begin{Coro}
Si $N\left(t\right)$ es un proceso de renovaci\'on, y $\left(Z\left(T_{n}\right)-Z\left(T_{n-1}\right),M_{n}\right)$, para $n\geq1$, son variables aleatorias independientes e id\'enticamente distribuidas con media finita, entonces,
\begin{eqnarray}
lim_{t\rightarrow\infty}t^{-1}Z\left(t\right)\rightarrow\frac{\esp\left[Z\left(T_{1}\right)-Z\left(T_{0}\right)\right]}{\esp\left[T_{1}\right]},\textrm{ c.s. cuando  }t\rightarrow\infty.
\end{eqnarray}
\end{Coro}


%___________________________________________________________________________________________
%
%\subsection*{Funci\'on de Renovaci\'on}
%___________________________________________________________________________________________
%


\begin{Def}
Sea $h\left(t\right)$ funci\'on de valores reales en $\rea$ acotada en intervalos finitos e igual a cero para $t<0$ La ecuaci\'on de renovaci\'on para $h\left(t\right)$ y la distribuci\'on $F$ es

\begin{eqnarray}\label{Ec.Renovacion}
H\left(t\right)=h\left(t\right)+\int_{\left[0,t\right]}H\left(t-s\right)dF\left(s\right)\textrm{,    }t\geq0,
\end{eqnarray}
donde $H\left(t\right)$ es una funci\'on de valores reales. Esto es $H=h+F\star H$. Decimos que $H\left(t\right)$ es soluci\'on de esta ecuaci\'on si satisface la ecuaci\'on, y es acotada en intervalos finitos e iguales a cero para $t<0$.
\end{Def}

\begin{Prop}
La funci\'on $U\star h\left(t\right)$ es la \'unica soluci\'on de la ecuaci\'on de renovaci\'on (\ref{Ec.Renovacion}).
\end{Prop}

\begin{Teo}[Teorema Renovaci\'on Elemental]
\begin{eqnarray*}
t^{-1}U\left(t\right)\rightarrow 1/\mu\textrm{,    cuando }t\rightarrow\infty.
\end{eqnarray*}
\end{Teo}

%___________________________________________________________________________________________
%
%\subsection{Funci\'on de Renovaci\'on}
%___________________________________________________________________________________________
%


Sup\'ongase que $N\left(t\right)$ es un proceso de renovaci\'on con distribuci\'on $F$ con media finita $\mu$.

\begin{Def}
La funci\'on de renovaci\'on asociada con la distribuci\'on $F$, del proceso $N\left(t\right)$, es
\begin{eqnarray*}
U\left(t\right)=\sum_{n=1}^{\infty}F^{n\star}\left(t\right),\textrm{   }t\geq0,
\end{eqnarray*}
donde $F^{0\star}\left(t\right)=\indora\left(t\geq0\right)$.
\end{Def}


\begin{Prop}
Sup\'ongase que la distribuci\'on de inter-renovaci\'on $F$ tiene densidad $f$. Entonces $U\left(t\right)$ tambi\'en tiene densidad, para $t>0$, y es $U^{'}\left(t\right)=\sum_{n=0}^{\infty}f^{n\star}\left(t\right)$. Adem\'as
\begin{eqnarray*}
\prob\left\{N\left(t\right)>N\left(t-\right)\right\}=0\textrm{,   }t\geq0.
\end{eqnarray*}
\end{Prop}

\begin{Def}
La Transformada de Laplace-Stieljes de $F$ est\'a dada por

\begin{eqnarray*}
\hat{F}\left(\alpha\right)=\int_{\rea_{+}}e^{-\alpha t}dF\left(t\right)\textrm{,  }\alpha\geq0.
\end{eqnarray*}
\end{Def}

Entonces

\begin{eqnarray*}
\hat{U}\left(\alpha\right)=\sum_{n=0}^{\infty}\hat{F^{n\star}}\left(\alpha\right)=\sum_{n=0}^{\infty}\hat{F}\left(\alpha\right)^{n}=\frac{1}{1-\hat{F}\left(\alpha\right)}.
\end{eqnarray*}


\begin{Prop}
La Transformada de Laplace $\hat{U}\left(\alpha\right)$ y $\hat{F}\left(\alpha\right)$ determina una a la otra de manera \'unica por la relaci\'on $\hat{U}\left(\alpha\right)=\frac{1}{1-\hat{F}\left(\alpha\right)}$.
\end{Prop}


\begin{Note}
Un proceso de renovaci\'on $N\left(t\right)$ cuyos tiempos de inter-renovaci\'on tienen media finita, es un proceso Poisson con tasa $\lambda$ si y s\'olo s\'i $\esp\left[U\left(t\right)\right]=\lambda t$, para $t\geq0$.
\end{Note}


\begin{Teo}
Sea $N\left(t\right)$ un proceso puntual simple con puntos de localizaci\'on $T_{n}$ tal que $\eta\left(t\right)=\esp\left[N\left(\right)\right]$ es finita para cada $t$. Entonces para cualquier funci\'on $f:\rea_{+}\rightarrow\rea$,
\begin{eqnarray*}
\esp\left[\sum_{n=1}^{N\left(\right)}f\left(T_{n}\right)\right]=\int_{\left(0,t\right]}f\left(s\right)d\eta\left(s\right)\textrm{,  }t\geq0,
\end{eqnarray*}
suponiendo que la integral exista. Adem\'as si $X_{1},X_{2},\ldots$ son variables aleatorias definidas en el mismo espacio de probabilidad que el proceso $N\left(t\right)$ tal que $\esp\left[X_{n}|T_{n}=s\right]=f\left(s\right)$, independiente de $n$. Entonces
\begin{eqnarray*}
\esp\left[\sum_{n=1}^{N\left(t\right)}X_{n}\right]=\int_{\left(0,t\right]}f\left(s\right)d\eta\left(s\right)\textrm{,  }t\geq0,
\end{eqnarray*} 
suponiendo que la integral exista. 
\end{Teo}

\begin{Coro}[Identidad de Wald para Renovaciones]
Para el proceso de renovaci\'on $N\left(t\right)$,
\begin{eqnarray*}
\esp\left[T_{N\left(t\right)+1}\right]=\mu\esp\left[N\left(t\right)+1\right]\textrm{,  }t\geq0,
\end{eqnarray*}  
\end{Coro}

%______________________________________________________________________
%\subsection{Procesos de Renovaci\'on}
%______________________________________________________________________

\begin{Def}\label{Def.Tn}
Sean $0\leq T_{1}\leq T_{2}\leq \ldots$ son tiempos aleatorios infinitos en los cuales ocurren ciertos eventos. El n\'umero de tiempos $T_{n}$ en el intervalo $\left[0,t\right)$ es

\begin{eqnarray}
N\left(t\right)=\sum_{n=1}^{\infty}\indora\left(T_{n}\leq t\right),
\end{eqnarray}
para $t\geq0$.
\end{Def}

Si se consideran los puntos $T_{n}$ como elementos de $\rea_{+}$, y $N\left(t\right)$ es el n\'umero de puntos en $\rea$. El proceso denotado por $\left\{N\left(t\right):t\geq0\right\}$, denotado por $N\left(t\right)$, es un proceso puntual en $\rea_{+}$. Los $T_{n}$ son los tiempos de ocurrencia, el proceso puntual $N\left(t\right)$ es simple si su n\'umero de ocurrencias son distintas: $0<T_{1}<T_{2}<\ldots$ casi seguramente.

\begin{Def}
Un proceso puntual $N\left(t\right)$ es un proceso de renovaci\'on si los tiempos de interocurrencia $\xi_{n}=T_{n}-T_{n-1}$, para $n\geq1$, son independientes e identicamente distribuidos con distribuci\'on $F$, donde $F\left(0\right)=0$ y $T_{0}=0$. Los $T_{n}$ son llamados tiempos de renovaci\'on, referente a la independencia o renovaci\'on de la informaci\'on estoc\'astica en estos tiempos. Los $\xi_{n}$ son los tiempos de inter-renovaci\'on, y $N\left(t\right)$ es el n\'umero de renovaciones en el intervalo $\left[0,t\right)$
\end{Def}


\begin{Note}
Para definir un proceso de renovaci\'on para cualquier contexto, solamente hay que especificar una distribuci\'on $F$, con $F\left(0\right)=0$, para los tiempos de inter-renovaci\'on. La funci\'on $F$ en turno degune las otra variables aleatorias. De manera formal, existe un espacio de probabilidad y una sucesi\'on de variables aleatorias $\xi_{1},\xi_{2},\ldots$ definidas en este con distribuci\'on $F$. Entonces las otras cantidades son $T_{n}=\sum_{k=1}^{n}\xi_{k}$ y $N\left(t\right)=\sum_{n=1}^{\infty}\indora\left(T_{n}\leq t\right)$, donde $T_{n}\rightarrow\infty$ casi seguramente por la Ley Fuerte de los Grandes Números.
\end{Note}

%___________________________________________________________________________________________
%
\section{Renewal and Regenerative Processes: Serfozo\cite{Serfozo}}
%___________________________________________________________________________________________
%
\begin{Def}\label{Def.Tn}
Sean $0\leq T_{1}\leq T_{2}\leq \ldots$ son tiempos aleatorios infinitos en los cuales ocurren ciertos eventos. El n\'umero de tiempos $T_{n}$ en el intervalo $\left[0,t\right)$ es

\begin{eqnarray}
N\left(t\right)=\sum_{n=1}^{\infty}\indora\left(T_{n}\leq t\right),
\end{eqnarray}
para $t\geq0$.
\end{Def}

Si se consideran los puntos $T_{n}$ como elementos de $\rea_{+}$, y $N\left(t\right)$ es el n\'umero de puntos en $\rea$. El proceso denotado por $\left\{N\left(t\right):t\geq0\right\}$, denotado por $N\left(t\right)$, es un proceso puntual en $\rea_{+}$. Los $T_{n}$ son los tiempos de ocurrencia, el proceso puntual $N\left(t\right)$ es simple si su n\'umero de ocurrencias son distintas: $0<T_{1}<T_{2}<\ldots$ casi seguramente.

\begin{Def}
Un proceso puntual $N\left(t\right)$ es un proceso de renovaci\'on si los tiempos de interocurrencia $\xi_{n}=T_{n}-T_{n-1}$, para $n\geq1$, son independientes e identicamente distribuidos con distribuci\'on $F$, donde $F\left(0\right)=0$ y $T_{0}=0$. Los $T_{n}$ son llamados tiempos de renovaci\'on, referente a la independencia o renovaci\'on de la informaci\'on estoc\'astica en estos tiempos. Los $\xi_{n}$ son los tiempos de inter-renovaci\'on, y $N\left(t\right)$ es el n\'umero de renovaciones en el intervalo $\left[0,t\right)$
\end{Def}


\begin{Note}
Para definir un proceso de renovaci\'on para cualquier contexto, solamente hay que especificar una distribuci\'on $F$, con $F\left(0\right)=0$, para los tiempos de inter-renovaci\'on. La funci\'on $F$ en turno degune las otra variables aleatorias. De manera formal, existe un espacio de probabilidad y una sucesi\'on de variables aleatorias $\xi_{1},\xi_{2},\ldots$ definidas en este con distribuci\'on $F$. Entonces las otras cantidades son $T_{n}=\sum_{k=1}^{n}\xi_{k}$ y $N\left(t\right)=\sum_{n=1}^{\infty}\indora\left(T_{n}\leq t\right)$, donde $T_{n}\rightarrow\infty$ casi seguramente por la Ley Fuerte de los Grandes N\'umeros.
\end{Note}







Los tiempos $T_{n}$ est\'an relacionados con los conteos de $N\left(t\right)$ por

\begin{eqnarray*}
\left\{N\left(t\right)\geq n\right\}&=&\left\{T_{n}\leq t\right\}\\
T_{N\left(t\right)}\leq &t&<T_{N\left(t\right)+1},
\end{eqnarray*}

adem\'as $N\left(T_{n}\right)=n$, y 

\begin{eqnarray*}
N\left(t\right)=\max\left\{n:T_{n}\leq t\right\}=\min\left\{n:T_{n+1}>t\right\}
\end{eqnarray*}

Por propiedades de la convoluci\'on se sabe que

\begin{eqnarray*}
P\left\{T_{n}\leq t\right\}=F^{n\star}\left(t\right)
\end{eqnarray*}
que es la $n$-\'esima convoluci\'on de $F$. Entonces 

\begin{eqnarray*}
\left\{N\left(t\right)\geq n\right\}&=&\left\{T_{n}\leq t\right\}\\
P\left\{N\left(t\right)\leq n\right\}&=&1-F^{\left(n+1\right)\star}\left(t\right)
\end{eqnarray*}

Adem\'as usando el hecho de que $\esp\left[N\left(t\right)\right]=\sum_{n=1}^{\infty}P\left\{N\left(t\right)\geq n\right\}$
se tiene que

\begin{eqnarray*}
\esp\left[N\left(t\right)\right]=\sum_{n=1}^{\infty}F^{n\star}\left(t\right)
\end{eqnarray*}

\begin{Prop}
Para cada $t\geq0$, la funci\'on generadora de momentos $\esp\left[e^{\alpha N\left(t\right)}\right]$ existe para alguna $\alpha$ en una vecindad del 0, y de aqu\'i que $\esp\left[N\left(t\right)^{m}\right]<\infty$, para $m\geq1$.
\end{Prop}

\begin{Ejem}[\textbf{Proceso Poisson}]

Suponga que se tienen tiempos de inter-renovaci\'on \textit{i.i.d.} del proceso de renovaci\'on $N\left(t\right)$ tienen distribuci\'on exponencial $F\left(t\right)=q-e^{-\lambda t}$ con tasa $\lambda$. Entonces $N\left(t\right)$ es un proceso Poisson con tasa $\lambda$.

\end{Ejem}


\begin{Note}
Si el primer tiempo de renovaci\'on $\xi_{1}$ no tiene la misma distribuci\'on que el resto de las $\xi_{n}$, para $n\geq2$, a $N\left(t\right)$ se le llama Proceso de Renovaci\'on retardado, donde si $\xi$ tiene distribuci\'on $G$, entonces el tiempo $T_{n}$ de la $n$-\'esima renovaci\'on tiene distribuci\'on $G\star F^{\left(n-1\right)\star}\left(t\right)$
\end{Note}


\begin{Teo}
Para una constante $\mu\leq\infty$ ( o variable aleatoria), las siguientes expresiones son equivalentes:

\begin{eqnarray}
lim_{n\rightarrow\infty}n^{-1}T_{n}&=&\mu,\textrm{ c.s.}\\
lim_{t\rightarrow\infty}t^{-1}N\left(t\right)&=&1/\mu,\textrm{ c.s.}
\end{eqnarray}
\end{Teo}


Es decir, $T_{n}$ satisface la Ley Fuerte de los Grandes N\'umeros s\'i y s\'olo s\'i $N\left/t\right)$ la cumple.


\begin{Coro}[Ley Fuerte de los Grandes N\'umeros para Procesos de Renovaci\'on]
Si $N\left(t\right)$ es un proceso de renovaci\'on cuyos tiempos de inter-renovaci\'on tienen media $\mu\leq\infty$, entonces
\begin{eqnarray}
t^{-1}N\left(t\right)\rightarrow 1/\mu,\textrm{ c.s. cuando }t\rightarrow\infty.
\end{eqnarray}

\end{Coro}


Considerar el proceso estoc\'astico de valores reales $\left\{Z\left(t\right):t\geq0\right\}$ en el mismo espacio de probabilidad que $N\left(t\right)$

\begin{Def}
Para el proceso $\left\{Z\left(t\right):t\geq0\right\}$ se define la fluctuaci\'on m\'axima de $Z\left(t\right)$ en el intervalo $\left(T_{n-1},T_{n}\right]$:
\begin{eqnarray*}
M_{n}=\sup_{T_{n-1}<t\leq T_{n}}|Z\left(t\right)-Z\left(T_{n-1}\right)|
\end{eqnarray*}
\end{Def}

\begin{Teo}
Sup\'ongase que $n^{-1}T_{n}\rightarrow\mu$ c.s. cuando $n\rightarrow\infty$, donde $\mu\leq\infty$ es una constante o variable aleatoria. Sea $a$ una constante o variable aleatoria que puede ser infinita cuando $\mu$ es finita, y considere las expresiones l\'imite:
\begin{eqnarray}
lim_{n\rightarrow\infty}n^{-1}Z\left(T_{n}\right)&=&a,\textrm{ c.s.}\\
lim_{t\rightarrow\infty}t^{-1}Z\left(t\right)&=&a/\mu,\textrm{ c.s.}
\end{eqnarray}
La segunda expresi\'on implica la primera. Conversamente, la primera implica la segunda si el proceso $Z\left(t\right)$ es creciente, o si $lim_{n\rightarrow\infty}n^{-1}M_{n}=0$ c.s.
\end{Teo}

\begin{Coro}
Si $N\left(t\right)$ es un proceso de renovaci\'on, y $\left(Z\left(T_{n}\right)-Z\left(T_{n-1}\right),M_{n}\right)$, para $n\geq1$, son variables aleatorias independientes e id\'enticamente distribuidas con media finita, entonces,
\begin{eqnarray}
lim_{t\rightarrow\infty}t^{-1}Z\left(t\right)\rightarrow\frac{\esp\left[Z\left(T_{1}\right)-Z\left(T_{0}\right)\right]}{\esp\left[T_{1}\right]},\textrm{ c.s. cuando  }t\rightarrow\infty.
\end{eqnarray}
\end{Coro}


Sup\'ongase que $N\left(t\right)$ es un proceso de renovaci\'on con distribuci\'on $F$ con media finita $\mu$.

\begin{Def}
La funci\'on de renovaci\'on asociada con la distribuci\'on $F$, del proceso $N\left(t\right)$, es
\begin{eqnarray*}
U\left(t\right)=\sum_{n=1}^{\infty}F^{n\star}\left(t\right),\textrm{   }t\geq0,
\end{eqnarray*}
donde $F^{0\star}\left(t\right)=\indora\left(t\geq0\right)$.
\end{Def}


\begin{Prop}
Sup\'ongase que la distribuci\'on de inter-renovaci\'on $F$ tiene densidad $f$. Entonces $U\left(t\right)$ tambi\'en tiene densidad, para $t>0$, y es $U^{'}\left(t\right)=\sum_{n=0}^{\infty}f^{n\star}\left(t\right)$. Adem\'as
\begin{eqnarray*}
\prob\left\{N\left(t\right)>N\left(t-\right)\right\}=0\textrm{,   }t\geq0.
\end{eqnarray*}
\end{Prop}

\begin{Def}
La Transformada de Laplace-Stieljes de $F$ est\'a dada por

\begin{eqnarray*}
\hat{F}\left(\alpha\right)=\int_{\rea_{+}}e^{-\alpha t}dF\left(t\right)\textrm{,  }\alpha\geq0.
\end{eqnarray*}
\end{Def}

Entonces

\begin{eqnarray*}
\hat{U}\left(\alpha\right)=\sum_{n=0}^{\infty}\hat{F^{n\star}}\left(\alpha\right)=\sum_{n=0}^{\infty}\hat{F}\left(\alpha\right)^{n}=\frac{1}{1-\hat{F}\left(\alpha\right)}.
\end{eqnarray*}


\begin{Prop}
La Transformada de Laplace $\hat{U}\left(\alpha\right)$ y $\hat{F}\left(\alpha\right)$ determina una a la otra de manera \'unica por la relaci\'on $\hat{U}\left(\alpha\right)=\frac{1}{1-\hat{F}\left(\alpha\right)}$.
\end{Prop}


\begin{Note}
Un proceso de renovaci\'on $N\left(t\right)$ cuyos tiempos de inter-renovaci\'on tienen media finita, es un proceso Poisson con tasa $\lambda$ si y s\'olo s\'i $\esp\left[U\left(t\right)\right]=\lambda t$, para $t\geq0$.
\end{Note}


\begin{Teo}
Sea $N\left(t\right)$ un proceso puntual simple con puntos de localizaci\'on $T_{n}$ tal que $\eta\left(t\right)=\esp\left[N\left(\right)\right]$ es finita para cada $t$. Entonces para cualquier funci\'on $f:\rea_{+}\rightarrow\rea$,
\begin{eqnarray*}
\esp\left[\sum_{n=1}^{N\left(\right)}f\left(T_{n}\right)\right]=\int_{\left(0,t\right]}f\left(s\right)d\eta\left(s\right)\textrm{,  }t\geq0,
\end{eqnarray*}
suponiendo que la integral exista. Adem\'as si $X_{1},X_{2},\ldots$ son variables aleatorias definidas en el mismo espacio de probabilidad que el proceso $N\left(t\right)$ tal que $\esp\left[X_{n}|T_{n}=s\right]=f\left(s\right)$, independiente de $n$. Entonces
\begin{eqnarray*}
\esp\left[\sum_{n=1}^{N\left(t\right)}X_{n}\right]=\int_{\left(0,t\right]}f\left(s\right)d\eta\left(s\right)\textrm{,  }t\geq0,
\end{eqnarray*} 
suponiendo que la integral exista. 
\end{Teo}

\begin{Coro}[Identidad de Wald para Renovaciones]
Para el proceso de renovaci\'on $N\left(t\right)$,
\begin{eqnarray*}
\esp\left[T_{N\left(t\right)+1}\right]=\mu\esp\left[N\left(t\right)+1\right]\textrm{,  }t\geq0,
\end{eqnarray*}  
\end{Coro}


\begin{Def}
Sea $h\left(t\right)$ funci\'on de valores reales en $\rea$ acotada en intervalos finitos e igual a cero para $t<0$ La ecuaci\'on de renovaci\'on para $h\left(t\right)$ y la distribuci\'on $F$ es

\begin{eqnarray}\label{Ec.Renovacion}
H\left(t\right)=h\left(t\right)+\int_{\left[0,t\right]}H\left(t-s\right)dF\left(s\right)\textrm{,    }t\geq0,
\end{eqnarray}
donde $H\left(t\right)$ es una funci\'on de valores reales. Esto es $H=h+F\star H$. Decimos que $H\left(t\right)$ es soluci\'on de esta ecuaci\'on si satisface la ecuaci\'on, y es acotada en intervalos finitos e iguales a cero para $t<0$.
\end{Def}

\begin{Prop}
La funci\'on $U\star h\left(t\right)$ es la \'unica soluci\'on de la ecuaci\'on de renovaci\'on (\ref{Ec.Renovacion}).
\end{Prop}

\begin{Teo}[Teorema Renovaci\'on Elemental]
\begin{eqnarray*}
t^{-1}U\left(t\right)\rightarrow 1/\mu\textrm{,    cuando }t\rightarrow\infty.
\end{eqnarray*}
\end{Teo}



Sup\'ongase que $N\left(t\right)$ es un proceso de renovaci\'on con distribuci\'on $F$ con media finita $\mu$.

\begin{Def}
La funci\'on de renovaci\'on asociada con la distribuci\'on $F$, del proceso $N\left(t\right)$, es
\begin{eqnarray*}
U\left(t\right)=\sum_{n=1}^{\infty}F^{n\star}\left(t\right),\textrm{   }t\geq0,
\end{eqnarray*}
donde $F^{0\star}\left(t\right)=\indora\left(t\geq0\right)$.
\end{Def}


\begin{Prop}
Sup\'ongase que la distribuci\'on de inter-renovaci\'on $F$ tiene densidad $f$. Entonces $U\left(t\right)$ tambi\'en tiene densidad, para $t>0$, y es $U^{'}\left(t\right)=\sum_{n=0}^{\infty}f^{n\star}\left(t\right)$. Adem\'as
\begin{eqnarray*}
\prob\left\{N\left(t\right)>N\left(t-\right)\right\}=0\textrm{,   }t\geq0.
\end{eqnarray*}
\end{Prop}

\begin{Def}
La Transformada de Laplace-Stieljes de $F$ est\'a dada por

\begin{eqnarray*}
\hat{F}\left(\alpha\right)=\int_{\rea_{+}}e^{-\alpha t}dF\left(t\right)\textrm{,  }\alpha\geq0.
\end{eqnarray*}
\end{Def}

Entonces

\begin{eqnarray*}
\hat{U}\left(\alpha\right)=\sum_{n=0}^{\infty}\hat{F^{n\star}}\left(\alpha\right)=\sum_{n=0}^{\infty}\hat{F}\left(\alpha\right)^{n}=\frac{1}{1-\hat{F}\left(\alpha\right)}.
\end{eqnarray*}


\begin{Prop}
La Transformada de Laplace $\hat{U}\left(\alpha\right)$ y $\hat{F}\left(\alpha\right)$ determina una a la otra de manera \'unica por la relaci\'on $\hat{U}\left(\alpha\right)=\frac{1}{1-\hat{F}\left(\alpha\right)}$.
\end{Prop}


\begin{Note}
Un proceso de renovaci\'on $N\left(t\right)$ cuyos tiempos de inter-renovaci\'on tienen media finita, es un proceso Poisson con tasa $\lambda$ si y s\'olo s\'i $\esp\left[U\left(t\right)\right]=\lambda t$, para $t\geq0$.
\end{Note}


\begin{Teo}
Sea $N\left(t\right)$ un proceso puntual simple con puntos de localizaci\'on $T_{n}$ tal que $\eta\left(t\right)=\esp\left[N\left(\right)\right]$ es finita para cada $t$. Entonces para cualquier funci\'on $f:\rea_{+}\rightarrow\rea$,
\begin{eqnarray*}
\esp\left[\sum_{n=1}^{N\left(\right)}f\left(T_{n}\right)\right]=\int_{\left(0,t\right]}f\left(s\right)d\eta\left(s\right)\textrm{,  }t\geq0,
\end{eqnarray*}
suponiendo que la integral exista. Adem\'as si $X_{1},X_{2},\ldots$ son variables aleatorias definidas en el mismo espacio de probabilidad que el proceso $N\left(t\right)$ tal que $\esp\left[X_{n}|T_{n}=s\right]=f\left(s\right)$, independiente de $n$. Entonces
\begin{eqnarray*}
\esp\left[\sum_{n=1}^{N\left(t\right)}X_{n}\right]=\int_{\left(0,t\right]}f\left(s\right)d\eta\left(s\right)\textrm{,  }t\geq0,
\end{eqnarray*} 
suponiendo que la integral exista. 
\end{Teo}

\begin{Coro}[Identidad de Wald para Renovaciones]
Para el proceso de renovaci\'on $N\left(t\right)$,
\begin{eqnarray*}
\esp\left[T_{N\left(t\right)+1}\right]=\mu\esp\left[N\left(t\right)+1\right]\textrm{,  }t\geq0,
\end{eqnarray*}  
\end{Coro}


\begin{Def}
Sea $h\left(t\right)$ funci\'on de valores reales en $\rea$ acotada en intervalos finitos e igual a cero para $t<0$ La ecuaci\'on de renovaci\'on para $h\left(t\right)$ y la distribuci\'on $F$ es

\begin{eqnarray}\label{Ec.Renovacion}
H\left(t\right)=h\left(t\right)+\int_{\left[0,t\right]}H\left(t-s\right)dF\left(s\right)\textrm{,    }t\geq0,
\end{eqnarray}
donde $H\left(t\right)$ es una funci\'on de valores reales. Esto es $H=h+F\star H$. Decimos que $H\left(t\right)$ es soluci\'on de esta ecuaci\'on si satisface la ecuaci\'on, y es acotada en intervalos finitos e iguales a cero para $t<0$.
\end{Def}

\begin{Prop}
La funci\'on $U\star h\left(t\right)$ es la \'unica soluci\'on de la ecuaci\'on de renovaci\'on (\ref{Ec.Renovacion}).
\end{Prop}

\begin{Teo}[Teorema Renovaci\'on Elemental]
\begin{eqnarray*}
t^{-1}U\left(t\right)\rightarrow 1/\mu\textrm{,    cuando }t\rightarrow\infty.
\end{eqnarray*}
\end{Teo}


\begin{Note} Una funci\'on $h:\rea_{+}\rightarrow\rea$ es Directamente Riemann Integrable en los siguientes casos:
\begin{itemize}
\item[a)] $h\left(t\right)\geq0$ es decreciente y Riemann Integrable.
\item[b)] $h$ es continua excepto posiblemente en un conjunto de Lebesgue de medida 0, y $|h\left(t\right)|\leq b\left(t\right)$, donde $b$ es DRI.
\end{itemize}
\end{Note}

\begin{Teo}[Teorema Principal de Renovaci\'on]
Si $F$ es no aritm\'etica y $h\left(t\right)$ es Directamente Riemann Integrable (DRI), entonces

\begin{eqnarray*}
lim_{t\rightarrow\infty}U\star h=\frac{1}{\mu}\int_{\rea_{+}}h\left(s\right)ds.
\end{eqnarray*}
\end{Teo}

\begin{Prop}
Cualquier funci\'on $H\left(t\right)$ acotada en intervalos finitos y que es 0 para $t<0$ puede expresarse como
\begin{eqnarray*}
H\left(t\right)=U\star h\left(t\right)\textrm{,  donde }h\left(t\right)=H\left(t\right)-F\star H\left(t\right)
\end{eqnarray*}
\end{Prop}

\begin{Def}
Un proceso estoc\'astico $X\left(t\right)$ es crudamente regenerativo en un tiempo aleatorio positivo $T$ si
\begin{eqnarray*}
\esp\left[X\left(T+t\right)|T\right]=\esp\left[X\left(t\right)\right]\textrm{, para }t\geq0,\end{eqnarray*}
y con las esperanzas anteriores finitas.
\end{Def}

\begin{Prop}
Sup\'ongase que $X\left(t\right)$ es un proceso crudamente regenerativo en $T$, que tiene distribuci\'on $F$. Si $\esp\left[X\left(t\right)\right]$ es acotado en intervalos finitos, entonces
\begin{eqnarray*}
\esp\left[X\left(t\right)\right]=U\star h\left(t\right)\textrm{,  donde }h\left(t\right)=\esp\left[X\left(t\right)\indora\left(T>t\right)\right].
\end{eqnarray*}
\end{Prop}

\begin{Teo}[Regeneraci\'on Cruda]
Sup\'ongase que $X\left(t\right)$ es un proceso con valores positivo crudamente regenerativo en $T$, y def\'inase $M=\sup\left\{|X\left(t\right)|:t\leq T\right\}$. Si $T$ es no aritm\'etico y $M$ y $MT$ tienen media finita, entonces
\begin{eqnarray*}
lim_{t\rightarrow\infty}\esp\left[X\left(t\right)\right]=\frac{1}{\mu}\int_{\rea_{+}}h\left(s\right)ds,
\end{eqnarray*}
donde $h\left(t\right)=\esp\left[X\left(t\right)\indora\left(T>t\right)\right]$.
\end{Teo}


\begin{Note} Una funci\'on $h:\rea_{+}\rightarrow\rea$ es Directamente Riemann Integrable en los siguientes casos:
\begin{itemize}
\item[a)] $h\left(t\right)\geq0$ es decreciente y Riemann Integrable.
\item[b)] $h$ es continua excepto posiblemente en un conjunto de Lebesgue de medida 0, y $|h\left(t\right)|\leq b\left(t\right)$, donde $b$ es DRI.
\end{itemize}
\end{Note}

\begin{Teo}[Teorema Principal de Renovaci\'on]
Si $F$ es no aritm\'etica y $h\left(t\right)$ es Directamente Riemann Integrable (DRI), entonces

\begin{eqnarray*}
lim_{t\rightarrow\infty}U\star h=\frac{1}{\mu}\int_{\rea_{+}}h\left(s\right)ds.
\end{eqnarray*}
\end{Teo}

\begin{Prop}
Cualquier funci\'on $H\left(t\right)$ acotada en intervalos finitos y que es 0 para $t<0$ puede expresarse como
\begin{eqnarray*}
H\left(t\right)=U\star h\left(t\right)\textrm{,  donde }h\left(t\right)=H\left(t\right)-F\star H\left(t\right)
\end{eqnarray*}
\end{Prop}

\begin{Def}
Un proceso estoc\'astico $X\left(t\right)$ es crudamente regenerativo en un tiempo aleatorio positivo $T$ si
\begin{eqnarray*}
\esp\left[X\left(T+t\right)|T\right]=\esp\left[X\left(t\right)\right]\textrm{, para }t\geq0,\end{eqnarray*}
y con las esperanzas anteriores finitas.
\end{Def}

\begin{Prop}
Sup\'ongase que $X\left(t\right)$ es un proceso crudamente regenerativo en $T$, que tiene distribuci\'on $F$. Si $\esp\left[X\left(t\right)\right]$ es acotado en intervalos finitos, entonces
\begin{eqnarray*}
\esp\left[X\left(t\right)\right]=U\star h\left(t\right)\textrm{,  donde }h\left(t\right)=\esp\left[X\left(t\right)\indora\left(T>t\right)\right].
\end{eqnarray*}
\end{Prop}

\begin{Teo}[Regeneraci\'on Cruda]
Sup\'ongase que $X\left(t\right)$ es un proceso con valores positivo crudamente regenerativo en $T$, y def\'inase $M=\sup\left\{|X\left(t\right)|:t\leq T\right\}$. Si $T$ es no aritm\'etico y $M$ y $MT$ tienen media finita, entonces
\begin{eqnarray*}
lim_{t\rightarrow\infty}\esp\left[X\left(t\right)\right]=\frac{1}{\mu}\int_{\rea_{+}}h\left(s\right)ds,
\end{eqnarray*}
donde $h\left(t\right)=\esp\left[X\left(t\right)\indora\left(T>t\right)\right]$.
\end{Teo}

\begin{Def}
Para el proceso $\left\{\left(N\left(t\right),X\left(t\right)\right):t\geq0\right\}$, sus trayectoria muestrales en el intervalo de tiempo $\left[T_{n-1},T_{n}\right)$ est\'an descritas por
\begin{eqnarray*}
\zeta_{n}=\left(\xi_{n},\left\{X\left(T_{n-1}+t\right):0\leq t<\xi_{n}\right\}\right)
\end{eqnarray*}
Este $\zeta_{n}$ es el $n$-\'esimo segmento del proceso. El proceso es regenerativo sobre los tiempos $T_{n}$ si sus segmentos $\zeta_{n}$ son independientes e id\'enticamennte distribuidos.
\end{Def}


\begin{Note}
Si $\tilde{X}\left(t\right)$ con espacio de estados $\tilde{S}$ es regenerativo sobre $T_{n}$, entonces $X\left(t\right)=f\left(\tilde{X}\left(t\right)\right)$ tambi\'en es regenerativo sobre $T_{n}$, para cualquier funci\'on $f:\tilde{S}\rightarrow S$.
\end{Note}

\begin{Note}
Los procesos regenerativos son crudamente regenerativos, pero no al rev\'es.
\end{Note}


\begin{Note}
Un proceso estoc\'astico a tiempo continuo o discreto es regenerativo si existe un proceso de renovaci\'on  tal que los segmentos del proceso entre tiempos de renovaci\'on sucesivos son i.i.d., es decir, para $\left\{X\left(t\right):t\geq0\right\}$ proceso estoc\'astico a tiempo continuo con espacio de estados $S$, espacio m\'etrico.
\end{Note}

Para $\left\{X\left(t\right):t\geq0\right\}$ Proceso Estoc\'astico a tiempo continuo con estado de espacios $S$, que es un espacio m\'etrico, con trayectorias continuas por la derecha y con l\'imites por la izquierda c.s. Sea $N\left(t\right)$ un proceso de renovaci\'on en $\rea_{+}$ definido en el mismo espacio de probabilidad que $X\left(t\right)$, con tiempos de renovaci\'on $T$ y tiempos de inter-renovaci\'on $\xi_{n}=T_{n}-T_{n-1}$, con misma distribuci\'on $F$ de media finita $\mu$.



\begin{Def}
Para el proceso $\left\{\left(N\left(t\right),X\left(t\right)\right):t\geq0\right\}$, sus trayectoria muestrales en el intervalo de tiempo $\left[T_{n-1},T_{n}\right)$ est\'an descritas por
\begin{eqnarray*}
\zeta_{n}=\left(\xi_{n},\left\{X\left(T_{n-1}+t\right):0\leq t<\xi_{n}\right\}\right)
\end{eqnarray*}
Este $\zeta_{n}$ es el $n$-\'esimo segmento del proceso. El proceso es regenerativo sobre los tiempos $T_{n}$ si sus segmentos $\zeta_{n}$ son independientes e id\'enticamennte distribuidos.
\end{Def}

\begin{Note}
Un proceso regenerativo con media de la longitud de ciclo finita es llamado positivo recurrente.
\end{Note}

\begin{Teo}[Procesos Regenerativos]
Suponga que el proceso
\end{Teo}


\begin{Def}[Renewal Process Trinity]
Para un proceso de renovaci\'on $N\left(t\right)$, los siguientes procesos proveen de informaci\'on sobre los tiempos de renovaci\'on.
\begin{itemize}
\item $A\left(t\right)=t-T_{N\left(t\right)}$, el tiempo de recurrencia hacia atr\'as al tiempo $t$, que es el tiempo desde la \'ultima renovaci\'on para $t$.

\item $B\left(t\right)=T_{N\left(t\right)+1}-t$, el tiempo de recurrencia hacia adelante al tiempo $t$, residual del tiempo de renovaci\'on, que es el tiempo para la pr\'oxima renovaci\'on despu\'es de $t$.

\item $L\left(t\right)=\xi_{N\left(t\right)+1}=A\left(t\right)+B\left(t\right)$, la longitud del intervalo de renovaci\'on que contiene a $t$.
\end{itemize}
\end{Def}

\begin{Note}
El proceso tridimensional $\left(A\left(t\right),B\left(t\right),L\left(t\right)\right)$ es regenerativo sobre $T_{n}$, y por ende cada proceso lo es. Cada proceso $A\left(t\right)$ y $B\left(t\right)$ son procesos de MArkov a tiempo continuo con trayectorias continuas por partes en el espacio de estados $\rea_{+}$. Una expresi\'on conveniente para su distribuci\'on conjunta es, para $0\leq x<t,y\geq0$
\begin{equation}\label{NoRenovacion}
P\left\{A\left(t\right)>x,B\left(t\right)>y\right\}=
P\left\{N\left(t+y\right)-N\left((t-x)\right)=0\right\}
\end{equation}
\end{Note}

\begin{Ejem}[Tiempos de recurrencia Poisson]
Si $N\left(t\right)$ es un proceso Poisson con tasa $\lambda$, entonces de la expresi\'on (\ref{NoRenovacion}) se tiene que

\begin{eqnarray*}
\begin{array}{lc}
P\left\{A\left(t\right)>x,B\left(t\right)>y\right\}=e^{-\lambda\left(x+y\right)},&0\leq x<t,y\geq0,
\end{array}
\end{eqnarray*}
que es la probabilidad Poisson de no renovaciones en un intervalo de longitud $x+y$.

\end{Ejem}

\begin{Note}
Una cadena de Markov erg\'odica tiene la propiedad de ser estacionaria si la distribuci\'on de su estado al tiempo $0$ es su distribuci\'on estacionaria.
\end{Note}


\begin{Def}
Un proceso estoc\'astico a tiempo continuo $\left\{X\left(t\right):t\geq0\right\}$ en un espacio general es estacionario si sus distribuciones finito dimensionales son invariantes bajo cualquier  traslado: para cada $0\leq s_{1}<s_{2}<\cdots<s_{k}$ y $t\geq0$,
\begin{eqnarray*}
\left(X\left(s_{1}+t\right),\ldots,X\left(s_{k}+t\right)\right)=_{d}\left(X\left(s_{1}\right),\ldots,X\left(s_{k}\right)\right).
\end{eqnarray*}
\end{Def}

\begin{Note}
Un proceso de Markov es estacionario si $X\left(t\right)=_{d}X\left(0\right)$, $t\geq0$.
\end{Note}

Considerese el proceso $N\left(t\right)=\sum_{n}\indora\left(\tau_{n}\leq t\right)$ en $\rea_{+}$, con puntos $0<\tau_{1}<\tau_{2}<\cdots$.

\begin{Prop}
Si $N$ es un proceso puntual estacionario y $\esp\left[N\left(1\right)\right]<\infty$, entonces $\esp\left[N\left(t\right)\right]=t\esp\left[N\left(1\right)\right]$, $t\geq0$

\end{Prop}

\begin{Teo}
Los siguientes enunciados son equivalentes
\begin{itemize}
\item[i)] El proceso retardado de renovaci\'on $N$ es estacionario.

\item[ii)] EL proceso de tiempos de recurrencia hacia adelante $B\left(t\right)$ es estacionario.


\item[iii)] $\esp\left[N\left(t\right)\right]=t/\mu$,


\item[iv)] $G\left(t\right)=F_{e}\left(t\right)=\frac{1}{\mu}\int_{0}^{t}\left[1-F\left(s\right)\right]ds$
\end{itemize}
Cuando estos enunciados son ciertos, $P\left\{B\left(t\right)\leq x\right\}=F_{e}\left(x\right)$, para $t,x\geq0$.

\end{Teo}

\begin{Note}
Una consecuencia del teorema anterior es que el Proceso Poisson es el \'unico proceso sin retardo que es estacionario.
\end{Note}

\begin{Coro}
El proceso de renovaci\'on $N\left(t\right)$ sin retardo, y cuyos tiempos de inter renonaci\'on tienen media finita, es estacionario si y s\'olo si es un proceso Poisson.

\end{Coro}

%______________________________________________________________________

%\section{Ejemplos, Notas importantes}
%______________________________________________________________________
%\section*{Ap\'endice A}
%__________________________________________________________________

%________________________________________________________________________
%\subsection*{Procesos Regenerativos}
%________________________________________________________________________



\begin{Note}
Si $\tilde{X}\left(t\right)$ con espacio de estados $\tilde{S}$ es regenerativo sobre $T_{n}$, entonces $X\left(t\right)=f\left(\tilde{X}\left(t\right)\right)$ tambi\'en es regenerativo sobre $T_{n}$, para cualquier funci\'on $f:\tilde{S}\rightarrow S$.
\end{Note}

\begin{Note}
Los procesos regenerativos son crudamente regenerativos, pero no al rev\'es.
\end{Note}
%\subsection*{Procesos Regenerativos: Sigman\cite{Sigman1}}
\begin{Def}[Definici\'on Cl\'asica]
Un proceso estoc\'astico $X=\left\{X\left(t\right):t\geq0\right\}$ es llamado regenerativo is existe una variable aleatoria $R_{1}>0$ tal que
\begin{itemize}
\item[i)] $\left\{X\left(t+R_{1}\right):t\geq0\right\}$ es independiente de $\left\{\left\{X\left(t\right):t<R_{1}\right\},\right\}$
\item[ii)] $\left\{X\left(t+R_{1}\right):t\geq0\right\}$ es estoc\'asticamente equivalente a $\left\{X\left(t\right):t>0\right\}$
\end{itemize}

Llamamos a $R_{1}$ tiempo de regeneraci\'on, y decimos que $X$ se regenera en este punto.
\end{Def}

$\left\{X\left(t+R_{1}\right)\right\}$ es regenerativo con tiempo de regeneraci\'on $R_{2}$, independiente de $R_{1}$ pero con la misma distribuci\'on que $R_{1}$. Procediendo de esta manera se obtiene una secuencia de variables aleatorias independientes e id\'enticamente distribuidas $\left\{R_{n}\right\}$ llamados longitudes de ciclo. Si definimos a $Z_{k}\equiv R_{1}+R_{2}+\cdots+R_{k}$, se tiene un proceso de renovaci\'on llamado proceso de renovaci\'on encajado para $X$.




\begin{Def}
Para $x$ fijo y para cada $t\geq0$, sea $I_{x}\left(t\right)=1$ si $X\left(t\right)\leq x$,  $I_{x}\left(t\right)=0$ en caso contrario, y def\'inanse los tiempos promedio
\begin{eqnarray*}
\overline{X}&=&lim_{t\rightarrow\infty}\frac{1}{t}\int_{0}^{\infty}X\left(u\right)du\\
\prob\left(X_{\infty}\leq x\right)&=&lim_{t\rightarrow\infty}\frac{1}{t}\int_{0}^{\infty}I_{x}\left(u\right)du,
\end{eqnarray*}
cuando estos l\'imites existan.
\end{Def}

Como consecuencia del teorema de Renovaci\'on-Recompensa, se tiene que el primer l\'imite  existe y es igual a la constante
\begin{eqnarray*}
\overline{X}&=&\frac{\esp\left[\int_{0}^{R_{1}}X\left(t\right)dt\right]}{\esp\left[R_{1}\right]},
\end{eqnarray*}
suponiendo que ambas esperanzas son finitas.

\begin{Note}
\begin{itemize}
\item[a)] Si el proceso regenerativo $X$ es positivo recurrente y tiene trayectorias muestrales no negativas, entonces la ecuaci\'on anterior es v\'alida.
\item[b)] Si $X$ es positivo recurrente regenerativo, podemos construir una \'unica versi\'on estacionaria de este proceso, $X_{e}=\left\{X_{e}\left(t\right)\right\}$, donde $X_{e}$ es un proceso estoc\'astico regenerativo y estrictamente estacionario, con distribuci\'on marginal distribuida como $X_{\infty}$
\end{itemize}
\end{Note}

Para $\left\{X\left(t\right):t\geq0\right\}$ Proceso Estoc\'astico a tiempo continuo con estado de espacios $S$, que es un espacio m\'etrico, con trayectorias continuas por la derecha y con l\'imites por la izquierda c.s. Sea $N\left(t\right)$ un proceso de renovaci\'on en $\rea_{+}$ definido en el mismo espacio de probabilidad que $X\left(t\right)$, con tiempos de renovaci\'on $T$ y tiempos de inter-renovaci\'on $\xi_{n}=T_{n}-T_{n-1}$, con misma distribuci\'on $F$ de media finita $\mu$.


\begin{Def}
Para el proceso $\left\{\left(N\left(t\right),X\left(t\right)\right):t\geq0\right\}$, sus trayectoria muestrales en el intervalo de tiempo $\left[T_{n-1},T_{n}\right)$ est\'an descritas por
\begin{eqnarray*}
\zeta_{n}=\left(\xi_{n},\left\{X\left(T_{n-1}+t\right):0\leq t<\xi_{n}\right\}\right)
\end{eqnarray*}
Este $\zeta_{n}$ es el $n$-\'esimo segmento del proceso. El proceso es regenerativo sobre los tiempos $T_{n}$ si sus segmentos $\zeta_{n}$ son independientes e id\'enticamennte distribuidos.
\end{Def}


\begin{Note}
Si $\tilde{X}\left(t\right)$ con espacio de estados $\tilde{S}$ es regenerativo sobre $T_{n}$, entonces $X\left(t\right)=f\left(\tilde{X}\left(t\right)\right)$ tambi\'en es regenerativo sobre $T_{n}$, para cualquier funci\'on $f:\tilde{S}\rightarrow S$.
\end{Note}

\begin{Note}
Los procesos regenerativos son crudamente regenerativos, pero no al rev\'es.
\end{Note}

\begin{Def}[Definici\'on Cl\'asica]
Un proceso estoc\'astico $X=\left\{X\left(t\right):t\geq0\right\}$ es llamado regenerativo is existe una variable aleatoria $R_{1}>0$ tal que
\begin{itemize}
\item[i)] $\left\{X\left(t+R_{1}\right):t\geq0\right\}$ es independiente de $\left\{\left\{X\left(t\right):t<R_{1}\right\},\right\}$
\item[ii)] $\left\{X\left(t+R_{1}\right):t\geq0\right\}$ es estoc\'asticamente equivalente a $\left\{X\left(t\right):t>0\right\}$
\end{itemize}

Llamamos a $R_{1}$ tiempo de regeneraci\'on, y decimos que $X$ se regenera en este punto.
\end{Def}

$\left\{X\left(t+R_{1}\right)\right\}$ es regenerativo con tiempo de regeneraci\'on $R_{2}$, independiente de $R_{1}$ pero con la misma distribuci\'on que $R_{1}$. Procediendo de esta manera se obtiene una secuencia de variables aleatorias independientes e id\'enticamente distribuidas $\left\{R_{n}\right\}$ llamados longitudes de ciclo. Si definimos a $Z_{k}\equiv R_{1}+R_{2}+\cdots+R_{k}$, se tiene un proceso de renovaci\'on llamado proceso de renovaci\'on encajado para $X$.

\begin{Note}
Un proceso regenerativo con media de la longitud de ciclo finita es llamado positivo recurrente.
\end{Note}


\begin{Def}
Para $x$ fijo y para cada $t\geq0$, sea $I_{x}\left(t\right)=1$ si $X\left(t\right)\leq x$,  $I_{x}\left(t\right)=0$ en caso contrario, y def\'inanse los tiempos promedio
\begin{eqnarray*}
\overline{X}&=&lim_{t\rightarrow\infty}\frac{1}{t}\int_{0}^{\infty}X\left(u\right)du\\
\prob\left(X_{\infty}\leq x\right)&=&lim_{t\rightarrow\infty}\frac{1}{t}\int_{0}^{\infty}I_{x}\left(u\right)du,
\end{eqnarray*}
cuando estos l\'imites existan.
\end{Def}

Como consecuencia del teorema de Renovaci\'on-Recompensa, se tiene que el primer l\'imite  existe y es igual a la constante
\begin{eqnarray*}
\overline{X}&=&\frac{\esp\left[\int_{0}^{R_{1}}X\left(t\right)dt\right]}{\esp\left[R_{1}\right]},
\end{eqnarray*}
suponiendo que ambas esperanzas son finitas.

\begin{Note}
\begin{itemize}
\item[a)] Si el proceso regenerativo $X$ es positivo recurrente y tiene trayectorias muestrales no negativas, entonces la ecuaci\'on anterior es v\'alida.
\item[b)] Si $X$ es positivo recurrente regenerativo, podemos construir una \'unica versi\'on estacionaria de este proceso, $X_{e}=\left\{X_{e}\left(t\right)\right\}$, donde $X_{e}$ es un proceso estoc\'astico regenerativo y estrictamente estacionario, con distribuci\'on marginal distribuida como $X_{\infty}$
\end{itemize}
\end{Note}

%__________________________________________________________________________________________
%\subsection{Procesos Regenerativos Estacionarios - Stidham \cite{Stidham}}
%__________________________________________________________________________________________


Un proceso estoc\'astico a tiempo continuo $\left\{V\left(t\right),t\geq0\right\}$ es un proceso regenerativo si existe una sucesi\'on de variables aleatorias independientes e id\'enticamente distribuidas $\left\{X_{1},X_{2},\ldots\right\}$, sucesi\'on de renovaci\'on, tal que para cualquier conjunto de Borel $A$, 

\begin{eqnarray*}
\prob\left\{V\left(t\right)\in A|X_{1}+X_{2}+\cdots+X_{R\left(t\right)}=s,\left\{V\left(\tau\right),\tau<s\right\}\right\}=\prob\left\{V\left(t-s\right)\in A|X_{1}>t-s\right\},
\end{eqnarray*}
para todo $0\leq s\leq t$, donde $R\left(t\right)=\max\left\{X_{1}+X_{2}+\cdots+X_{j}\leq t\right\}=$n\'umero de renovaciones ({\emph{puntos de regeneraci\'on}}) que ocurren en $\left[0,t\right]$. El intervalo $\left[0,X_{1}\right)$ es llamado {\emph{primer ciclo de regeneraci\'on}} de $\left\{V\left(t \right),t\geq0\right\}$, $\left[X_{1},X_{1}+X_{2}\right)$ el {\emph{segundo ciclo de regeneraci\'on}}, y as\'i sucesivamente.

Sea $X=X_{1}$ y sea $F$ la funci\'on de distrbuci\'on de $X$


\begin{Def}
Se define el proceso estacionario, $\left\{V^{*}\left(t\right),t\geq0\right\}$, para $\left\{V\left(t\right),t\geq0\right\}$ por

\begin{eqnarray*}
\prob\left\{V\left(t\right)\in A\right\}=\frac{1}{\esp\left[X\right]}\int_{0}^{\infty}\prob\left\{V\left(t+x\right)\in A|X>x\right\}\left(1-F\left(x\right)\right)dx,
\end{eqnarray*} 
para todo $t\geq0$ y todo conjunto de Borel $A$.
\end{Def}

\begin{Def}
Una distribuci\'on se dice que es {\emph{aritm\'etica}} si todos sus puntos de incremento son m\'ultiplos de la forma $0,\lambda, 2\lambda,\ldots$ para alguna $\lambda>0$ entera.
\end{Def}


\begin{Def}
Una modificaci\'on medible de un proceso $\left\{V\left(t\right),t\geq0\right\}$, es una versi\'on de este, $\left\{V\left(t,w\right)\right\}$ conjuntamente medible para $t\geq0$ y para $w\in S$, $S$ espacio de estados para $\left\{V\left(t\right),t\geq0\right\}$.
\end{Def}

\begin{Teo}
Sea $\left\{V\left(t\right),t\geq\right\}$ un proceso regenerativo no negativo con modificaci\'on medible. Sea $\esp\left[X\right]<\infty$. Entonces el proceso estacionario dado por la ecuaci\'on anterior est\'a bien definido y tiene funci\'on de distribuci\'on independiente de $t$, adem\'as
\begin{itemize}
\item[i)] \begin{eqnarray*}
\esp\left[V^{*}\left(0\right)\right]&=&\frac{\esp\left[\int_{0}^{X}V\left(s\right)ds\right]}{\esp\left[X\right]}\end{eqnarray*}
\item[ii)] Si $\esp\left[V^{*}\left(0\right)\right]<\infty$, equivalentemente, si $\esp\left[\int_{0}^{X}V\left(s\right)ds\right]<\infty$,entonces
\begin{eqnarray*}
\frac{\int_{0}^{t}V\left(s\right)ds}{t}\rightarrow\frac{\esp\left[\int_{0}^{X}V\left(s\right)ds\right]}{\esp\left[X\right]}
\end{eqnarray*}
con probabilidad 1 y en media, cuando $t\rightarrow\infty$.
\end{itemize}
\end{Teo}iudad de M\'exico\\}}

\noindent{\bf{Academia de Matem\'aticas}}

\noindent{\bf{Colegio de Ciencia y Tecnolog\'ia}}

\vspace{1.5in} \noindent{\centerline{\bf{\Large{
Revisi\'on de Procesos Regenerativos Estacionarios}}}}
\vspace{.5in} \noindent{\centerline{{\bf{\large{Carlos Ernesto
Mart\'inez Rodr\'iguez}}}}}

\vspace{.5in} \vfill \hfill {\bf{Agosto, 2016}}
%______________________________________________________________________
%
%\makeindex
\renewcommand{\contentsname}{Index}
\maketitle \tableofcontents
%\newpage
%________________________________________________________________________

%___________________________________________________________
%
\section{Existencia de Tiempos de Regeneraci\'on}
%___________________________________________________________
%

%________________________________________________________________________
%\subsection{Procesos Regenerativos: Thorisson}
%________________________________________________________________________

Un elemento aleatorio en un espacio medible $\left(E,\mathcal{E}\right)$ en un espacio de probabilidad $\left(\Omega,\mathcal{F},\prob\right)$ a $\left(E,\mathcal{E}\right)$, es decir,
para $A\in \mathcal{E}$,  se tiene que $\left\{Y\in A\right\}\in\mathcal{F}$, donde $\left\{Y\in A\right\}:=\left\{w\in\Omega:Y\left(w\right)\in A\right\}=:Y^{-1}A$. Tambi\'en se dice que $Y$ est\'a soportado por el espacio de probabilidad $\left(\Omega,\mathcal{F},\prob\right)$ y que $Y$ es un mapeo medible de $\Omega$ en $E$, es decir, es $\mathcal{F}/\mathcal{E}$ medible. Para cada $i\in \mathbb{I}$ sea $P_{i}$ una medida de probabilidad en un espacio medible $\left(E_{i},\mathcal{E}_{i}\right)$. Se define el espacio producto $\otimes_{i\in\mathbb{I}}\left(E_{i},\mathcal{E}_{i}\right):=\left(\prod_{i\in\mathbb{I}}E_{i},\otimes_{i\in\mathbb{I}}\mathcal{E}_{i}\right)$, donde $\prod_{i\in\mathbb{I}}E_{i}$ es el producto cartesiano de los $E_{i}$'s, y $\otimes_{i\in\mathbb{I}}\mathcal{E}_{i}$ es la $\sigma$-\'algebra producto, es decir, es la $\sigma$-\'algebra m\'as peque\~na en $\prod_{i\in\mathbb{I}}E_{i}$ que hace al $i$-\'esimo mapeo proyecci\'on en $E_{i}$ medible para toda $i\in\mathbb{I}$ es la $\sigma$-\'algebra inducida por los mapeos proyecci\'on. $$\otimes_{i\in\mathbb{I}}\mathcal{E}_{i}:=\sigma\left\{\left\{y:y_{i}\in A\right\}:i\in\mathbb{I}\textrm{ y }A\in\mathcal{E}_{i}\right\}.$$ Un espacio de probabilidad $\left(\tilde{\Omega},\tilde{\mathcal{F}},\tilde{\prob}\right)$ es una extensi\'on de otro espacio de probabilidad $\left(\Omega,\mathcal{F},\prob\right)$ si $\left(\tilde{\Omega},\tilde{\mathcal{F}},\tilde{\prob}\right)$ soporta un elemento aleatorio $\xi\in\left(\Omega,\mathcal{F}\right)$ que tienen a $\prob$ como distribuci\'on.

\begin{Teo}
Sea $\mathbb{I}$ un conjunto de \'indices arbitrario. Para cada $i\in\mathbb{I}$ sea $P_{i}$ una medida de probabilidad en un espacio medible $\left(E_{i},\mathcal{E}_{i}\right)$. Entonces existe una \'unica medida de probabilidad $\otimes_{i\in\mathbb{I}}P_{i}$ en $\otimes_{i\in\mathbb{I}}\left(E_{i},\mathcal{E}_{i}\right)$ tal que 

\begin{eqnarray*}
\otimes_{i\in\mathbb{I}}P_{i}\left(y\in\prod_{i\in\mathbb{I}}E_{i}:y_{i}\in A_{i_{1}},\ldots,y_{n}\in A_{i_{n}}\right)=P_{i_{1}}\left(A_{i_{n}}\right)\cdots P_{i_{n}}\left(A_{i_{n}}\right)
\end{eqnarray*}
para todos los enteros $n>0$, toda $i_{1},\ldots,i_{n}\in\mathbb{I}$ y todo $A_{i_{1}}\in\mathcal{E}_{i_{1}},\ldots,A_{i_{n}}\in\mathcal{E}_{i_{n}}$
\end{Teo}

La medida $\otimes_{i\in\mathbb{I}}P_{i}$ es llamada la medida producto y $\otimes_{i\in\mathbb{I}}\left(E_{i},\mathcal{E}_{i},P_{i}\right):=\left(\prod_{i\in\mathbb{I}},E_{i},\otimes_{i\in\mathbb{I}}\mathcal{E}_{i},\otimes_{i\in\mathbb{I}}P_{i}\right)$, es llamado espacio de probabilidad producto.

\begin{Def}
Un espacio medible $\left(E,\mathcal{E}\right)$ es \textit{Polaco} si existe una m\'etrica en $E$ tal que $E$ es completo. (es decir cada sucesi\'on de Cauchy converge a un l\'imite en $E$, y \textit{separable}, $E$ tienen un subconjunto denso numerable, y tal que $\mathcal{E}$ es generado por conjuntos abiertos.)
\end{Def}

Dos espacios medibles $\left(E,\mathcal{E}\right)$ y $\left(G,\mathcal{G}\right)$ son Borel equivalentes \textit{isomorfos} si existe una biyecci\'on $f:E\rightarrow G$ tal que $f$ es $\mathcal{E}/\mathcal{G}$ medible y su inversa $f^{-1}$ es $\mathcal{G}/\mathcal{E}$ medible. La biyecci\'on es una equivalencia de Borel.  Un espacio medible  $\left(E,\mathcal{E}\right)$ es un \textit{espacio est\'andar} si es Borel equivalente a $\left(G,\mathcal{G}\right)$, donde $G$ es un subconjunto de Borel de $\left[0,1\right]$ y $\mathcal{G}$ son los subconjuntos de Borel de $G$. Cualquier espacio Polaco es un espacio est\'andar. Un proceso estoc\'astico (\textbf{PE}) con conjunto de \'indices $\mathbb{I}$ y espacio de estados $\left(E,\mathcal{E}\right)$ es una familia $Z=\left(\mathbb{Z}_{s}\right)_{s\in\mathbb{I}}$ donde $\mathbb{Z}_{s}$ son elementos aleatorios definidos en un espacio de probabilidad com\'un $\left(\Omega,\mathcal{F},\prob\right)$ y todos toman valores en $\left(E,\mathcal{E}\right)$.

\begin{Def}
Un Proceso Estoc\'astico \textit{One-Sided Contiuous Time} (\textbf{PEOSCT}) es un proceso estoc\'astico con conjunto de \'indices $\mathbb{I}=\left[0,\infty\right)$.
\end{Def}

Sea $\left(E^{\mathbb{I}},\mathcal{E}^{\mathbb{I}}\right)$ denota el espacio producto $\left(E^{\mathbb{I}},\mathcal{E}^{\mathbb{I}}\right):=\otimes_{s\in\mathbb{I}}\left(E,\mathcal{E}\right)$. Vamos a considerar $\mathbb{Z}$ como un mapeo aleatorio, es decir, como un elemento aleatorio en $\left(E^{\mathbb{I}},\mathcal{E}^{\mathbb{I}}\right)$ definido por $Z\left(w\right)=\left(Z_{s}\left(w\right)\right)_{s\in\mathbb{I}}$ y $w\in\Omega$. La distribuci\'on de un proceso estoc\'astico $Z$ es la distribuci\'on de $Z$ como un elemento aleatorio en $\left(E^{\mathbb{I}},\mathcal{E}^{\mathbb{I}}\right)$. La distribuci\'on de $Z$ esta determinada de manera \'unica por las distribuciones finito dimensionales. En particular cuando $Z$ toma valores reales, es decir, $\left(E,\mathcal{E}\right)=\left(\mathbb{R},\mathcal{B}\right)$ las distribuciones finito dimensionales est\'an determinadas por las funciones de distribuci\'on finito dimensionales
\begin{eqnarray}
\prob\left(Z_{t_{1}}\leq x_{1},\ldots,Z_{t_{n}}\leq x_{n}\right),x_{1},\ldots,x_{n}\in\mathbb{R},t_{1},\ldots,t_{n}\in\mathbb{I},n\geq1.
\end{eqnarray}

Para espacios polacos $\left(E,\mathcal{E}\right)$ el Teorema de Consistencia de Kolmogorov asegura que dada una colecci\'on de distribuciones finito dimensionales consistentes, siempre existe un proceso estoc\'astico que posee tales distribuciones finito dimensionales. Las trayectorias de $Z$ son las realizaciones $Z\left(w\right)$ para $w\in\Omega$ del mapeo aleatorio $Z$. Algunas restricciones se imponen sobre las trayectorias, por ejemplo que sean continuas por la derecha, o continuas por la derecha con l\'imites por la izquierda, o de manera m\'as general, se pedir\'a que caigan en alg\'un subconjunto $H$ de $E^{\mathbb{I}}$. En este caso es natural considerar a $Z$ como un elemento aleatorio que no est\'a en $\left(E^{\mathbb{I}},\mathcal{E}^{\mathbb{I}}\right)$ sino en $\left(H,\mathcal{H}\right)$, donde $\mathcal{H}$ es la $\sigma$-\'algebra generada por los mapeos proyecci\'on que toman a $z\in H$ a $z_{t}\in E$ para $t\in\mathbb{I}$. A $\mathcal{H}$ se le conoce como la traza de $H$ en $E^{\mathbb{I}}$, es decir,
\begin{eqnarray}
\mathcal{H}:=E^{\mathbb{I}}\cap H:=\left\{A\cap H:A\in E^{\mathbb{I}}\right\}.
\end{eqnarray}

$Z$ tiene trayectorias con valores en $H$ y cada $Z_{t}$ es un mapeo medible de $\left(\Omega,\mathcal{F}\right)$ a $\left(H,\mathcal{H}\right)$. Cuando se considera un espacio de trayectorias en particular $H$, al espacio $\left(H,\mathcal{H}\right)$ se le llama el espacio de trayectorias de $Z$. La distribuci\'on del proceso estoc\'astico $Z$ con espacio de trayectorias $\left(H,\mathcal{H}\right)$ es la distribuci\'on de $Z$ como  un elemento aleatorio en $\left(H,\mathcal{H}\right)$. La distribuci\'on, nuevemente, est\'a determinada de manera \'unica por las distribuciones finito dimensionales.

\begin{Def}
Sea $Z$ un PEOSCT  con espacio de estados $\left(E,\mathcal{E}\right)$ y sea $T$ un tiempo aleatorio en $\left[0,\infty\right)$. Por $Z_{T}$ se entiende el mapeo con valores en $E$ definido en $\Omega$ en la manera obvia:
\begin{eqnarray*}
Z_{T}\left(w\right):=Z_{T\left(w\right)}\left(w\right). w\in\Omega.
\end{eqnarray*}
\end{Def}

\begin{Def}
Un PEOSCT $Z$ es Conjuntamente Medible (\textbf{CM}) si el mapeo que toma $\left(w,t\right)\in\Omega\times\left[0,\infty\right)$ a $Z_{t}\left(w\right)\in E$ es $\mathcal{F}\otimes\mathcal{B}\left[0,\infty\right)/\mathcal{E}$ medible.
\end{Def}

Un PEOSCT-CM implica que el proceso es medible, dado que $Z_{T}$ es una composici\'on  de dos mapeos continuos: el primero que toma $w$ en $\left(w,T\left(w\right)\right)$ es $\mathcal{F}/\mathcal{F}\otimes\mathcal{B}\left[0,\infty\right)$ medible, mientras que el segundo toma $\left(w,T\left(w\right)\right)$ en $Z_{T\left(w\right)}\left(w\right)$ es $\mathcal{F}\otimes\mathcal{B}\left[0,\infty\right)/\mathcal{E}$ medible.

\begin{Def}
Un PEOSCT con espacio de estados $\left(H,\mathcal{H}\right)$ es Can\'onicamente Conjuntamente Medible (\textbf{CCM}) si el mapeo $\left(z,t\right)\in H\times\left[0,\infty\right)$ en $Z_{t}\in E$ es $\mathcal{H}\otimes\mathcal{B}\left[0,\infty\right)/\mathcal{E}$ medible.
\end{Def}

Un PEOSCT-CCM implica que el proceso es CM, dado que un PECCM $Z$ es un mapeo de $\Omega\times\left[0,\infty\right)$ a $E$, es la composici\'on de dos mapeos medibles: el primero, toma $\left(w,t\right)$ en $\left(Z\left(w\right),t\right)$ es $\mathcal{F}\otimes\mathcal{B}\left[0,\infty\right)/\mathcal{H}\otimes\mathcal{B}\left[0,\infty\right)$ medible, y el segundo que toma $\left(Z\left(w\right),t\right)$  en $Z_{t}\left(w\right)$ es $\mathcal{H}\otimes\mathcal{B}\left[0,\infty\right)/\mathcal{E}$ medible. Por tanto CCM es una condici\'on m\'as fuerte que CM.

\begin{Def}
Un conjunto de trayectorias $H$ de un PEOSCT $Z$, es Internamente Shift-Invariante (\textbf{ISI}) si 
\begin{eqnarray*}
\left\{\left(z_{t+s}\right)_{s\in\left[0,\infty\right)}:z\in H\right\}=H\textrm{, }t\in\left[0,\infty\right).
\end{eqnarray*}
\end{Def}


\begin{Def}
Dado un PEOSCT-ISI, se define el mapeo-shift $\theta_{t}$, $t\in\left[0,\infty\right)$, de $H$ a $H$ por 
\begin{eqnarray*}
\theta_{t}z=\left(z_{t+s}\right)_{s\in\left[0,\infty\right)}\textrm{, }z\in H.
\end{eqnarray*}
\end{Def}

\begin{Def}
Se dice que un proceso $Z$ es Shift-Medible (\textbf{SM}) si $Z$ tiene un conjunto de trayectorias $H$ que es ISI y adem\'as el mapeo que toma $\left(z,t\right)\in H\times\left[0,\infty\right)$ en $\theta_{t}z\in H$ es $\mathcal{H}\otimes\mathcal{B}\left[0,\infty\right)/\mathcal{H}$ medible.
\end{Def}

Un proceso estoc\'astico con conjunto de trayectorias $H$ ISI es SM si y s\'olo si es CCM. Dado el espacio polaco $\left(E,\mathcal{E}\right)$ se tiene el  conjunto de trayectorias $D_{E}\left[0,\infty\right)$ que es ISI, entonces cumpe con ser CCM. Si $G$ es abierto, podemos cubrirlo por bolas abiertas cuay cerradura este contenida en $G$, y como $G$ es segundo numerable como subespacio de $E$, lo podemos cubrir por una cantidad numerable de bolas abiertas. Los procesos estoc\'asticos $Z$ a tiempo discreto con espacio de estados polaco, tambi\'en tiene un espacio de trayectorias polaco y por tanto tiene distribuciones condicionales regulares.

\begin{Teo}
El producto numerable de espacios polacos es polaco.
\end{Teo}


\begin{Def}
Sea $\left(\Omega,\mathcal{F},\prob\right)$ espacio de probabilidad que soporta al proceso $Z=\left(Z_{s}\right)_{s\in\left[0,\infty\right)}$ y $S=\left(S_{k}\right)_{0}^{\infty}$ donde $Z$ es un PEOSCTM con espacio de estados $\left(E,\mathcal{E}\right)$  y espacio de trayectorias $\left(H,\mathcal{H}\right)$  y adem\'as $S$ es una sucesi\'on de tiempos aleatorios one-sided que satisfacen la condici\'on $0\leq S_{0}<S_{1}<\cdots\rightarrow\infty$. Considerando $S$ como un mapeo medible de $\left(\Omega,\mathcal{F}\right)$ al espacio sucesi\'on $\left(L,\mathcal{L}\right)$, donde 
\begin{eqnarray*}
L=\left\{\left(s_{k}\right)_{0}^{\infty}\in\left[0,\infty\right)^{\left\{0,1,\ldots\right\}}:s_{0}<s_{1}<\cdots\rightarrow\infty\right\},
\end{eqnarray*}
donde $\mathcal{L}$ son los subconjuntos de Borel de $L$, es decir, $\mathcal{L}=L\cap\mathcal{B}^{\left\{0,1,\ldots\right\}}$.

As\'i el par $\left(Z,S\right)$ es un mapeo medible de  $\left(\Omega,\mathcal{F}\right)$ en $\left(H\times L,\mathcal{H}\otimes\mathcal{L}\right)$. El par $\mathcal{H}\otimes\mathcal{L}^{+}$ denotar\'a la clase de todas las funciones medibles de $\left(H\times L,\mathcal{H}\otimes\mathcal{L}\right)$ en $\left(\left[0,\infty\right),\mathcal{B}\left[0,\infty\right)\right)$.
\end{Def}


\begin{Def}
Sea $\theta_{t}$ el mapeo-shift conjunto de $H\times L$ en $H\times L$ dado por
\begin{eqnarray*}
\theta_{t}\left(z,\left(s_{k}\right)_{0}^{\infty}\right)=\theta_{t}\left(z,\left(s_{n_{t-}+k}-t\right)_{0}^{\infty}\right)
\end{eqnarray*}
donde 
$n_{t-}=inf\left\{n\geq1:s_{n}\geq t\right\}$.
\end{Def}

Con la finalidad de poder realizar los shift's sin complicaciones de medibilidad, se supondr\'a que $Z$ es shit-medible, es decir, el conjunto de trayectorias $H$ es invariante bajo shifts del tiempo y el mapeo que toma $\left(z,t\right)\in H\times\left[0,\infty\right)$ en $z_{t}\in E$ es $\mathcal{H}\otimes\mathcal{B}\left[0,\infty\right)/\mathcal{E}$ medible.

\begin{Def}
Dado un proceso \textbf{PEOSSM} (Proceso Estoc\'astico One Side Shift Medible) $Z$, se dice regenerativo cl\'asico con tiempos de regeneraci\'on $S$ si 

\begin{eqnarray*}
\theta_{S_{n}}\left(Z,S\right)=\left(Z^{0},S^{0}\right),n\geq0
\end{eqnarray*}
y adem\'as $\theta_{S_{n}}\left(Z,S\right)$ es independiente de $\left(\left(Z_{s}\right)s\in\left[0,S_{n}\right),S_{0},\ldots,S_{n}\right)$
Si lo anterior se cumple, al par $\left(Z,S\right)$ se le llama regenerativo cl\'asico.
\end{Def}

Si el par $\left(Z,S\right)$ es regenerativo cl\'asico, entonces las longitudes de los ciclos $X_{1},X_{2},\ldots,$ son i.i.d. e independientes de la longitud del retraso $S_{0}$, es decir, $S$ es un proceso de renovaci\'on. Las longitudes de los ciclos tambi\'en son llamados tiempos de inter-regeneraci\'on y tiempos de ocurrencia.

\begin{Teo}
Sup\'ongase que el par $\left(Z,S\right)$ es regenerativo cl\'asico con $\esp\left[X_{1}\right]<\infty$. Entonces $\left(Z^{*},S^{*}\right)$ en el teorema 2.1 es una versi\'on estacionaria de $\left(Z,S\right)$. Adem\'as, si $X_{1}$ es lattice con span $d$, entonces $\left(Z^{**},S^{**}\right)$ en el teorema 2.2 es una versi\'on periodicamente estacionaria de $\left(Z,S\right)$ con periodo $d$.
\end{Teo}

\begin{Def}
Una variable aleatoria $X_{1}$ es \textit{spread out} si existe una $n\geq1$ y una  funci\'on $f\in\mathcal{B}^{+}$ tal que $\int_{\rea}f\left(x\right)dx>0$ con $X_{2},X_{3},\ldots,X_{n}$ copias i.i.d  de $X_{1}$, $$\prob\left(X_{1}+\cdots+X_{n}\in B\right)\geq\int_{B}f\left(x\right)dx$$ para $B\in\mathcal{B}$.
\end{Def}

\begin{Def}
Dado un proceso estoc\'astico $Z$ se le llama \textit{Wide-Sense Regenerative} (\textbf{WSR}) con tiempos de regeneraci\'on $S$ si $\theta_{S_{n}}\left(Z,S\right)=\left(Z^{0},S^{0}\right)$ para $n\geq0$ en distribuci\'on y $\theta_{S_{n}}\left(Z,S\right)$ es independiente de $\left(S_{0},S_{1},\ldots,S_{n}\right)$ para $n\geq0$. Se dice que el par $\left(Z,S\right)$ es WSR si lo anterior se cumple.
\end{Def}

El proceso de trayectorias $\left(\theta_{s}Z\right)_{s\in\left[0,\infty\right)}$ es WSR con tiempos de regeneraci\'on $S$ pero no es regenerativo cl\'asico. Si $Z$ es cualquier proceso estacionario y $S$ es un proceso de renovaci\'on que es independiente de $Z$, entonces $\left(Z,S\right)$ es WSR pero en general no es regenerativo cl\'asico. Para cualquier proceso estoc\'astico $Z$, el proceso de trayectorias $\left(\theta_{s}Z\right)_{s\in\left[0,\infty\right)}$ es siempre un proceso de Markov.

\begin{Teo}
Supongase que el par $\left(Z,S\right)$ es WSR con $\esp\left[X_{1}\right]<\infty$. Entonces $\left(Z^{*},S^{*}\right)$ en el teorema 2.1 es una versi\'on estacionaria de 
$\left(Z,S\right)$.
\end{Teo}


\begin{Teo}
Supongase que $\left(Z,S\right)$ es cycle-stationary con $\esp\left[X_{1}\right]<\infty$. Sea $U$ distribuida uniformemente en $\left[0,1\right)$ e independiente de $\left(Z^{0},S^{0}\right)$ y sea $\prob^{*}$ la medida de probabilidad en $\left(\Omega,\prob\right)$ definida por $$d\prob^{*}=\frac{X_{1}}{\esp\left[X_{1}\right]}d\prob$$. Sea $\left(Z^{*},S^{*}\right)$ con distribuci\'on $\prob^{*}\left(\theta_{UX_{1}}\left(Z^{0},S^{0}\right)\in\cdot\right)$. Entonces $\left(Z^{}*,S^{*}\right)$ es estacionario,
\begin{eqnarray*}
\esp\left[f\left(Z^{*},S^{*}\right)\right]=\esp\left[\int_{0}^{X_{1}}f\left(\theta_{s}\left(Z^{0},S^{0}\right)\right)ds\right]/\esp\left[X_{1}\right]
\end{eqnarray*}
$f\in\mathcal{H}\otimes\mathcal{L}^{+}$, and $S_{0}^{*}$ es continuo con funci\'on distribuci\'on $G_{\infty}$ definida por $$G_{\infty}\left(x\right):=\frac{\esp\left[X_{1}\right]\wedge x}{\esp\left[X_{1}\right]}$$ para $x\geq0$ y densidad $\prob\left[X_{1}>x\right]/\esp\left[X_{1}\right]$, con $x\geq0$.

\end{Teo}


\begin{Teo}
Sea $Z$ un Proceso Estoc\'astico un lado shift-medible \textit{one-sided shift-measurable stochastic process}, (PEOSSM),
y $S_{0}$ y $S_{1}$ tiempos aleatorios tales que $0\leq S_{0}<S_{1}$ y
\begin{equation}
\theta_{S_{1}}Z=\theta_{S_{0}}Z\textrm{ en distribuci\'on}.
\end{equation}

Entonces el espacio de probabilidad subyacente $\left(\Omega,\mathcal{F},\prob\right)$ puede extenderse para soportar una sucesi\'on de tiempos aleatorios $S$ tales que

\begin{eqnarray}
\theta_{S_{n}}\left(Z,S\right)=\left(Z^{0},S^{0}\right),n\geq0,\textrm{ en distribuci\'on},\\
\left(Z,S_{0},S_{1}\right)\textrm{ depende de }\left(X_{2},X_{3},\ldots\right)\textrm{ solamente a traves de }\theta_{S_{1}}Z.
\end{eqnarray}
\end{Teo}
%________________________________________________________________________________
\section{Main Theorem: Article}
%__________________________________________________________________________________

The authors are interested in determining the means of the queue lengths at any time, for the exhaustive policy. For this purpose, it is necessary to make assumptions over the processes involved in order to guarantee the stationarity of the queue lengths processes in the NCPS. 
\begin{Sup}\label{Supuestos.Estabilidad}\medskip
%\textbf{Assumption 1}\label{Supuestos.Estabilidad}
\begin{itemize}
\item[\textit{(i)} ] Each of the queues of the NCPS is an $M/M/1$ system, with  $\tilde{\rho}_{i}:=\tilde{\mu}_{i}/\lambda_{i}<1$, for $i=1,2,3,4$ (observe that in the case considered $\tilde{\rho}_{i}=\tilde{\mu}_{i}$, for $i=1,2,3,4$, given that the service time is assumed to be proportional to the length of the slot).
\item[\textit{(ii)} ]  The switchover times have a finite first moment.
\end{itemize}
\end{Sup}
\begin{Remark}\label{Observacion.1}
In \cite{Disney} conditions are given which guarantee that the $M/M/1$ system has a renewal departure process (see Theorem 3.4 (4) in this reference). This will be used in the proofs of Theorems \ref{First.Regeneration.Time.Theorem} and \ref{Tma.Stability.Regenerative.Process}.
\end{Remark}

Consider the processes  $\mathbb{L}\left(t\right)=\left(L_{1}\left(t\right),L_{2}\left(t\right),L_{3}\left(t\right),L_{4}\left(t\right)\right)$, $\mathbb{B}\left(t\right)=\left(B_{1}\left(t\right),B_{2}\left(t\right),\right.$ $\left.B_{3}\left(t\right),B_{4}\left(t\right)\right)$, $\mathbb{K}\left(t\right)=\left(K_{1}\left(t\right),K_{2}\left(t\right),K_{3}\left(t\right),K_{4}\left(t\right)\right)$, and $\mathbb{V}\left(t\right)=\left(V_{1}\left(t\right),V_{2}\left(t\right),V_{3}\left(t\right),\right.$ $\left.V_{4}\left(t\right)\right)$, $t\geq0$, for the queue lengths, switchover times, service times and polling instants, respectively. Associated to  the queue length processes, the state spaces $\mathbb{D}_{i}=\left\{0,1,2,\ldots,\right\}$ and $\mathcal{D}_{i}=\mathcal{P}\left(\mathbb{D}_{i}\right)$, are considered, where $\mathcal{P}\left(\mathbb{D}_{i}\right)$ is the class of all subsets of $\mathbb{D}_{i}$, for $i=1,2,3,4$. For the following processes: switchover times, service times, and polling instants, consider the state space $\mathbb{G}=\left[0,\infty\right)$ with $\mathcal{G}=\mathcal{B}\left(\mathbb{G}\right)$, where $\mathcal{B}\left(\mathbb{G}\right)$ is the Borel $\sigma$-algebra generated by the open subsets of $\mathbb{G}$. Then the general process $$\mathbb{W}\left(t\right)=\left(\mathbb{L}\left(t\right),\mathbb{B}\left(t\right),\mathbb{K}\left(t\right),\mathbb{V}\left(t\right)\right)\textrm{,  }t>0,$$ is obtained, with state spaces $\Xi=\mathbb{D}_{i}^{4}\times \mathbb{G}^{12}$ and $\Im=\mathcal{B}\left(\Xi\right)$. In the sequel, the random variables taken into account are supposed to be defined in the measurable space $\left(\Xi,\Im\right)$. For a random variable $\eta$, $\eta\left[\Xi\right]:=\left\{\eta\left(t\right):t\geq0\right\}$ will be denoted. Besides, the following events for the arrival processes will be defined for some $t\geq0$ and queue $Q_{i}$ in the NCPS:  $A_{i}\left(t\right)=\left\{0 \textrm{ arrivals on }Q_{i}\textrm{ at time }t\right\}$,  for $i=1,2,3,4$.\medskip
\begin{Teo}\label{First.Regeneration.Time.Theorem}
Suppose that Assumption 1 holds. Given an NCPS, there exists a random variable $T^{*}$, such that $$\prob\left\{A_{1}\left(T^{*}\right)\cap A_{2}\left(T^{*}\right)\cap
A_{3}\left(T^{*}\right)\cap A_{4}\left(T^{*}\right)|T^{*}=t^{*}\right\}>0\textrm{, }t^{*}\in T^{*}\left[\Xi\right]$$  is satisfied.
\end{Teo}
\begin{proof}
The proof is divided into four steps given from a) to d).
\begin{itemize}
\item[a) ] 
In this incise it is going to be proved that both queues of system $\Gamma_{1}$ become empty at certain time. \\\\
The index $j=1,2,3,\ldots$, is used to denote the cycle when the server of system $\Gamma_{1}$ visits a queue. Denote its switchover times  by $r_{1}\left(j\right)$ and by $r_{2}\left(j\right)$, and its service times by $K_{1}\left(j\right)$, $K_{2}\left(j\right)$, for the queues $Q_{1}$ and $Q_{2}$,  respectively.\medskip

It is also necessary to define appropriate random variables which are going to help calculating the joint probability for the events $A_{1}\left(t\right)$ and $A_{2}\left(t\right)$ for certain times $\tilde{t}>0$. When the server arrives to the queue $Q_{1}$, at $t=0$, $\tau_{1}\left(1\right)=0$, there are no users in the queue, so that $K_{1}\left(1\right)=0$, therefore $\overline{\tau}_{1}\left(1\right)=\tau_{1}\left(1\right)=0$. Then the server moves to $Q_{2}$ according to the random variable $r_{1}\left(1\right)$ so that the arrival time to queue $Q_{2}$ is given by $\tau_{2}\left(1\right)=r_{1}\left(1\right)$.\medskip

There are two  possible situations when the server arrives to $Q_{2}$: the queue is empty or not. This means that $K_{2}\left(1\right)$ could be equal or greater than $0$. Here the authors just present the case when $K_{2}\left(1\right)>0$; the proof when $ K_{2}\left(1\right)=0$ is similar to the case of $K_{2}\left(1\right)>0$. Suppose $K_{2}\left(1\right)>0$, then $$\overline{\tau}_{2}\left(1\right)=\tau_{2}\left(1\right)+K_{2}\left(1\right),$$ consequently the server moves to $Q_{1}$ incurring in a switchover time $r_{2}\left(1\right)$, so that $$\tau_{1}\left(2\right)=\overline{\tau}_{2}\left(1\right)+r_{2}\left(1\right).$$
For $\tau_{1}\left(2\right)$ there are two possibilities: $Q_{1}$ is still empty or not. This means that $K_{1}\left(2\right)=0$ or $K_{1}\left(2\right)>0$, therefore $\overline{\tau}_{1}\left(2\right)=\tau_{1}\left(2\right)$ or  $\overline{\tau}_{1}\left(2\right)=\tau_{1}\left(2\right)+K_{1}\left(2\right)$. In both cases $$\tau_{2}\left(2\right)=\overline{\tau}_{1}\left(2\right)+r_{1}\left(2\right).$$

If $K_{1}\left(2\right)$ is not equal to zero, it is necessary to consider the period of time when it is possible to guarantee that both queues are empty at the same time. If $K_{1}\left(1\right)=K_{2}\left(1\right)=K_{1}\left(2\right)=K_{2}\left(2\right)=0$,  the calculations are simple, then it is supposed that they are not equal to zero. 
Let $\alpha$ be fixed with $\alpha\in\left[0,1\right]$. Consider the random variables
\begin{eqnarray}\medskip
\begin{array}{ll}
T_{1}=\left(1-\alpha\right)\overline{\tau}_{1}\left(2\right)+\alpha\tau_{2}\left(2\right),&
T_{2}=\left(1-\alpha\right)\overline{\tau}_{2}\left(1\right)+\alpha\tau_{2}\left(2\right).
\end{array}
\end{eqnarray}
For $t_{1}\in T_{1}\left[\Xi\right]$ and $t_{2}\in T_{2}\left[\Xi\right]$, it is obtained that
\begin{eqnarray}
\begin{array}{ll}
\prob\left\{A_{1}\left(T_{1}\right)|T_{1}=t_{1}\right\}=e^{-\tilde{\mu}_{1}t_{1}}\textrm{, and }&
\prob\left\{A_{2}\left(T_{2}\right)|T_{2}=t_{2}\right\}=e^{-\tilde{\mu}_{2}t_{2}}.
\end{array}
\end{eqnarray}
By construction it is gotten that $T_{1}\left[\Xi\right]\subseteq T_{2}\left[\Xi\right]$, therefore  $T_{1}\left[\Xi\right]\cap T_{2}\left[\Xi\right]=T_{1}\left[\Xi\right]\subseteq T_{2}\left[\Xi\right]$, so that, if $\tilde{t}\in T_{1}\left[\Xi\right]\cap T_{2}\left[\Xi\right]$ is considered, then $$\prob\left\{A_{2}\left(T_{1}\right)|T_{1}=\tilde{t}\right\}\geq\prob\left\{A_{2}\left(T_{2}\right)|T_{2}=\tilde{t}\right\}.$$ Therefore the joint probability is given by
\begin{eqnarray*}\medskip
\prob\left\{A_{1}\left(T_{1}\right)\cap A_{2}\left(T_{1}\right)|T_{1}=\tilde{t}\right\}&=&\prob\left\{A_{1}\left(T_{1}\right)|T_{1}=\tilde{t}\right\}\prob\left\{A_{2}\left(T_{1}\right)|T_{1}=\tilde{t}\right\}\\\medskip
&\geq&\prob\left\{A_{1}\left(T_{1}\right)|T_{1}=\tilde{t}\right\}\prob\left\{A_{2}\left(T_{2}\right)|T_{2}=\tilde{t}\right\}\\
&=& e^{-\tilde{\mu}_{1}\tilde{t}}e^{-\tilde{\mu}_{2}\tilde{t}}
=e^{-\left(\tilde{\mu}_{1}+\tilde{\mu}_{2}\right)\tilde{t}}>0.
\end{eqnarray*}
(Note that events $A_{1}\left(t\right)$ and $A_{2}\left(t\right)$ are conditionally independent under $T_{1}$.)  Hence,
\begin{eqnarray}
\prob\left\{A_{1}\left(T_{1}\right)\cap A_{2}\left(T_{1}\right)|T_{1}=\tilde{t}\right\}>0,\textrm{ for }\tilde{t}\in T_{1}\left[\Xi\right]\cap T_{2}\left[\Xi\right].
\end{eqnarray}
\item[b) ] Now it will be demonstrated that for the time of departure of the server in system $\Gamma_{1}$ it is always possible to find a cycle such that the server in system $\Gamma_{2}$ has just finished to attend one of the queues. So let us prove that for $\tau_{2}\left(2\right)$ there exists an $n\geq1$ such that one of the following inequalities is satisfied:
\begin{eqnarray}\label{Eq.Casos.Ciclos}
\begin{array}{ll}\medskip
\textrm{a) }\tau_{3}\left(n\right)<\tau_{2}\left(2\right)
<\overline{\tau}_{3}\left(n\right),&
\textrm{b) }\overline{\tau}_{3}\left(n\right)<\tau_{2}\left(2\right)
<\tau_{4}\left(n\right),\\
\textrm{c) }\tau_{4}\left(n\right)<\tau_{2}\left(2\right)<
\overline{\tau}_{4}\left(n\right), &
\textrm{d) }\overline{\tau}_{4}\left(n\right)<\tau_{2}\left(2\right)
<\tau_{3}\left(n+1\right).
\end{array}
\end{eqnarray}
The authors only give the proof for the case a); the proofs for the rest of the cases are similar. 
Suppose that for all $n\geq1$, $\tau_{2}\left(2\right)\leq\overline{\tau}_{3}\left(n\right)$ or $\overline{\tau}_{3}\left(n\right)\leq\tau_{2}\left(2\right)$. Consider $\tau_{3}\left(n\right)=0$ and $\tau_{2}\left(2\right)>0$, with $\overline{\tau}_{3}\left(n\right)\leq\tau_{2}\left(2\right)$ for all $n\geq1$ (the proof of the case $\tau_{2}\left(2\right)\leq\overline{\tau}_{3}\left(n\right)$ is analogous). It implies that all arrivals take place before $\tau_{2}\left(2\right)$, but this is not possible. Hence, case a) holds.\medskip
\item[c) ] In this part two random variables will be constructed for the system $\Gamma_{2}$ such that it will be possible to determine the joint probability for the events $A_{3}\left(t\right)$ and $A_{4}\left(t\right)$ for certain times $\hat{t}$. Without loss of generality, consider that $\tau_{3}\left(n\right)<\tau_{2}\left(2\right)<\overline{\tau}_{3}\left(n\right)$ for $n$ whose existence is guaranteed in (b), and define the random variables
\begin{eqnarray}
\begin{array}{l}\medskip
T_{3}=\left(1-\alpha\right)\tau_{3}\left(n-1\right)
+\alpha\overline{\tau}_{3}\left(n\right)\textrm{ and}\\
T_{4}=\left(1-\alpha\right)\overline{\tau}_{4}\left(n-1\right)+\alpha
\overline{\tau}_{3}\left(n\right).
\end{array}
\end{eqnarray}
Again, as above, for $\hat{t}\in  T_{3}\left[\Xi\right]\cap T_{4}\left[\Xi\right]$, the joint probability is given by
\begin{eqnarray*}\medskip
\prob\left\{A_{3}\left(T_{4}\right)\cap
A_{4}\left(T_{4}\right)|T_{4}=\hat{t}\right\}&\geq&
\prob\left\{A_{3}\left(T_{3}\right)|T_{3}=\hat{t}\right\}\cdot
\prob\left\{A_{4}\left(T_{4}\right)|T_{4}=\hat{t}\right\}\\
&=&e^{-\left(\tilde{\mu}_{3}+\tilde{\mu}_{4}\right)\hat{t}}>0.
\end{eqnarray*}%}
\item[d)] Finally, with the random variables $T_{1},T_{2},T_{3}$, and $T_{4}$, a new random variable $T^{*}$ is constructed, such that all the queues of both systems become empty for $T^{*}$. For $n$, whose existence is guaranteed in (b), let
\begin{eqnarray}
\begin{array}{lc}
T^{+}:=\overline{\tau}_{3}\left(n\right),\textrm{ and }&
T^{-}:=\min\left\{\overline{\tau}_{2}\left(1\right),\overline{\tau}_{3}\left(n-1\right)\right\}\\
\end{array}
\end{eqnarray}
and observe that  $$T^{-}\left[\Xi\right]\subset T_{i}\left[\Xi\right]\subset T^{+}\left[\Xi\right],\textrm{ for }i=1,2,3,4.$$
Define the random variable
\begin{eqnarray}
T^{*}:=\left(1-\alpha\right)T^{-}+\alpha T^{+},
\end{eqnarray}
such that for all $i=1,2,3,4$, it satisfies that $T^{*}\left[\Xi\right]\subset T_{i}\left[\Xi\right]$, so  $T^{*}\left[\Xi\right]\subset \cap_{i=1}^{4}T_{i}\left[\Xi\right]$. This implies that
\begin{equation}
\prob\left\{A_{i}\left(T^{*}\right)|T^{*}=t^{*}\right\}\geq\prob\left\{A_{i}\left(T_{i}\right)|T_{i}=t^{*}\right\}=e^{-\tilde{\mu}_{i}t^{*}}>0,\textrm{ for }t^{*}\in T^{*}\left[\Xi\right].
\end{equation}
This means that for $t^{*}\in T^{*}\left[\Xi\right]$ there are no arrivals to the queues $Q_{1}$ and $Q_{2}$, it results that there are no transfer users from $Q_{3}$ and $Q_{4}$, so $\tilde{\mu}_{1}=\mu_{1}$, $\tilde{\mu}_{2}=\mu_{2}$, $\tilde{\mu}_{3}=\mu_{3}$, and $\tilde{\mu}_{4}=\mu_{4}$. Consequently the events $A_{1}\left(T^{*}\right)$ and $A_{3}\left(T^{*}\right)$ are conditionally independent for $T^{*}$; the same argument can be applied for the events $A_{2}\left(T^{*}\right)$ and $A_{4}\left(T^{*}\right)$. Therefore, it follows that
\begin{eqnarray}
\begin{array}{c}\medskip
\prob\left\{A_{1}\left(T^{*}\right)\cap A_{2}\left(T^{*}\right)\cap
A_{3}\left(T^{*}\right)\cap A_{4}\left(T^{*}\right)
|T^{*}=t^{*}\right\}\geq\prod_{i=1}^{4}
\prob\left\{A_{i}\left(T_{i}\right)
|T_{i}=t_{i}\right\}\\
=\prod_{i=1}^{4}e^{-\mu_{i}t_{i}}
=e^{-\sum_{i=1}^{4}\mu_{i}t_{i}}>0,\textrm{ for }t^{*}\in T^{*}\left[\Xi\right].
\end{array}
\end{eqnarray}
\end{itemize}%\qed
\end{proof}
\begin{Remark}\label{Obs.Primer.Momento.Finito}
Note that it is easy to prove that $T^{*}$ has a finite first moment. This is a consequence of the fact that each of the variables involved in the definition of $T^{*}$ has a finite first moment, given that each of the queues in the NCPS is an $M/M/1$ system.
\end{Remark}
Now, the stationarity of the stochastic process related to the NCPS will be proved. Definitions very close to the ones given in Chapters 2, 3, 4, and 10 in Thorisson \cite{Thorisson} with respect to the stationarity of regenerative processes will be followed. Consider the processes $\mathbb{L}\left(t\right)=\left(L_{1}\left(t\right),L_{2}\left(t\right),L_{3}\left(t\right),L_{4}\left(t\right)\right)$, $t\geq0$, defined previously. Observe that the process $\mathbb{L}\left(t\right)$, $t\geq0$, takes values in the product space given by $$\left(\mathbf{E},\tilde{\mathcal{E}}\right)=\left(\prod_{i=1}^{4}D_{i},\prod_{i=1}^{4}\mathcal{D}_{i}\right),$$ which also results to be a polish space, and $\tilde{\mathcal{E}}$ is the corresponding product $\sigma$-algebra. 
\begin{Teo}\label{Tma.Stability.Regenerative.Process}
Given the stochastic process $\mathbb{L}\left(t\right)$, $t\geq0$, there is an infinite sequence of regeneration times $\Phi_{n},n\geq0$ defined on $\left(\Xi,\Im\right)$, such that for $\phi_{n}\in \Phi_{n}\left[\Xi\right]$,
\begin{eqnarray}\label{Eq.Regeneracion}
\mathbb{L}\left(\phi_{n}\right)=\left(L_{1}\left(\phi_{n}\right),L_{2}\left(\phi_{n}\right),L_{3}\left(\phi_{n}\right),L_{4}\left(\phi_{n}\right)\right)=\left(0,0,0,0\right),n\geq0.
\end{eqnarray}
Furthermore, there exists a stationary version of the process $\mathbb{L}\left(t\right)$, $t\geq0$.
\end{Teo}
\begin{proof}
Suppose that the process $\mathbf{L}=\left\{\mathbb{L}\left(t\right),t\geq0\right\}$ has been constructed in a canonical way on a probability space $\left(\overline{\Omega},\overline{\mathcal{F}},\overline{\prob}\right)$.
Given that the queue length processes $\mathbb{L}\left(t\right)$, $t\geq0$ are counting processes, it follows that they have right continuous paths with left hand limits. Hence, it is obtained that the process $\mathbf{L}$ is a continuous time one sided process, with path set $H:=D_{\mathbf{E}}\left[0,\infty\right)$, where  $D_{\mathbf{E}}\left[0,\infty\right)$ is the set of right continuous  maps from $\left[0,\infty\right)$ to $\mathbf{E}$ with left hand limits (see Section 2.1,  p. 126 in \cite{Thorisson}).

According to the second paragraph, Section 2.8, p. 131 in \cite{Thorisson}, the path set $H$ is internally shift-invariant and therefore canonically jointly measurable. By Section 2.7, p. 130 in \cite{Thorisson}, the stochastic process $\mathbf{L}$ is shift-measurable. 

Hence, by Theorem \ref{First.Regeneration.Time.Theorem} it is obtained that the hypothesis of Theorem 4.5, p. 362 in \cite{Thorisson} are satisfied, so that the underlying probability space $\left(\overline{\Omega},\overline{\mathcal{F}},\overline{\prob}\right)$ can be extended to support a sequence of random times $\left\{\Phi_{n}:n\geq1\right\}$ such that the stochastic process regenerates in these points. In the case discussed here, each of the queues becomes empty at $\phi_{n}\in\Phi_{n}\left[\Xi\right]$.

Finally, by Remark \ref{Obs.Primer.Momento.Finito} and Definition 3.1, p. 346 in \cite{Thorisson} (in fact, comparing the equations (3.1) and (3.2) with equations (4.6) and (4.7) in Theorem 4.5, p. 362 in \cite{Thorisson}), it is gotten that the stochastic process $\mathbf{L}$ is {\it{classical regenerative}} with finite first moment for the cycle lengths, therefore it is possible to apply Theorem 3.1, p. 348 in \cite{Thorisson}, in order to obtain that there exists a stationary version of the stochastic processes $\mathbf{L}$ and $\left\{\Phi_{n}:n\geq0\right\}$ satisfies (\ref{Eq.Regeneracion}).
\end{proof}%\qed
\medskip
Now, the probability generating functions that model the NCPS will be determined to calculate the expected queue lengths at any time.

In this section it is supposed that Assumption \ref{Supuestos.Estabilidad} holds. Here, the idea given in \cite{Takagi} is followed, in order to find the expected queue lengths at any time for the NCPS. 

Fix $i\in\left\{1,2,3,4\right\}$. Let $L_{i}^{*}$ be the number of users at queue $Q_{i}$ at polling instants, then, following Section \ref{CuerpoTrabajo:Explicits}, it is obtained that
\begin{eqnarray}
\begin{array}{cc}
\esp\left[L_{i}^{*}\right]=f_{i}\left(i\right), &
Var\left[L_{i}^{*}\right]=\mathbf{f}_{i}\left(i,i\right)+\esp\left[L_{i}^{*}\right]-\esp\left[L_{i}^{*}\right]^{2}.
\end{array}
\end{eqnarray}
Consider the cycle time $C_{i}$ for the queue $Q_{i}$ with duration given by $\tau_{i}\left(m+1\right)-\tau_{i}\left(m\right)$ for $m\geq1$. The intervisit time $I_{i}$ of the queue $Q_{i}$ is defined as the period beginning at the time the server leaves $Q_{i}$ in a cycle and ends at the time when the queue $Q_{i}$ is polled in the next cycle; its duration is given by $\tau_{i}\left(m+1\right)-\overline{\tau}_{i}\left(m\right)$. The interval between two successive regeneration points will be called regenerative cycle. Observe that Theorem \ref{Tma.Stability.Regenerative.Process} guarantees its existence. Let $M_{i}$ be the number of polling cycles in a regenerative cycle. The duration of the $m$-th polling cycle in a regeneration cycle will be denoted by $C_{i}^{(m)}$, for $m=1,2,\ldots,M_{i}$. The mean polling cycle time is defined by
\begin{equation}\label{Eq.FGP.Ciclos}
\esp\left[C_{i}\right]=\frac{\sum_{m=1}^{M_{i}}\esp\left[C_{i}^{(m)}\right]}{\esp\left[M_{i}\right]}.
\end{equation}
Again, note that Theorem \ref{Tma.Stability.Regenerative.Process} guarantees that all the terms in the right-hand side of (\ref{Eq.FGP.Ciclos}) are well defined.
%{\Large{
For the process $L_{i}\left(t\right)$, $t\geq0$, their PGF will be denoted by $\mathcal{Q}_{i}\left(z\right)$, $z\in\mathbb{C}$, which is also given by the time average of $z^{L_{i}\left(t\right)}$ over the regenerative cycle defined before, so it is obtained that
\begin{eqnarray}\label{Eq.Q.any.time}
\mathcal{Q}_{i}\left(z\right)&=&\esp\left[z^{L_{i}\left(t\right)}\right]=\frac{\esp\left[\sum_{m=1}^{M_{i}}\sum_{t=\tau_{i}\left(m\right)}^{\tau_{i}\left(m+1\right)-1}z^{L_{i}\left(t\right)}\right]}{\esp\left[\sum_{m=1}^{M_{i}}\left(\tau_{i}\left(m+1\right)-\tau_{i}\left(m\right)\right)\right]},
\end{eqnarray}
which can be rewritten in the form 
\begin{equation}\label{Eq.Long.Caulquier.Tiempo}
\mathcal{Q}_{i}\left(z\right)=\frac{1}{\esp\left[C_{i}\right]}\cdot\frac{1-F_{i}\left(z\right)}{P_{i}\left(z\right)-z}\cdot\frac{\left(1-z\right)P_{i}\left(z\right)}{1-P_{i}\left(z\right)},
\end{equation}%}}
(see Section 3 in \cite{Takagi}). The following proposition provides the expected queue lengths for each of the queues in the NCPS at any time.
\begin{Teo}
For the queue lengths in the NCPS at any time, with PGF given in (\ref{Eq.Long.Caulquier.Tiempo}), the first and second order moments are given by
\begin{eqnarray}
\mathcal{Q}_{i}^{(1)}\left(1\right)=\frac{1}{\tilde{\mu}_{i}\left(1-\tilde{\mu}_{i}\right)}\frac{\esp (L_{i}^{*})^{2}}{2\esp\left[C_{i}\right]}
-\sigma_{i}^{2}\frac{\esp\left[L_{i}^{*}\right]}{2\esp \left[C_{i}\right]}\cdot\frac{1-2\tilde{\mu}_{i}}{\left(1-\tilde{\mu}_{i}\right)^{2}\tilde{\mu}_{i}^{2}},
\end{eqnarray}
where $\sigma_{i}^{2}=\left(Var\left[\tilde{X}_{i}\left(t\right)\right]\right)^{2}$, and 
\begin{eqnarray}
\begin{array}{l}
\esp\left[C_{i}\right]\mathcal{Q}_{i}^{(2)}\left(1\right)=\frac{1}{\tilde{\mu}_{i}^{3}\left(1-\tilde{\mu}_{i}\right)^{3}}\left\{
-\left(1-\tilde{\mu}_{i}\right)^{2}\tilde{\mu}_{i}^{2}O_{1,i}^{(2)}(1)\right.\\
-\left.\tilde{\mu}_{i}\left(1-\tilde{\mu}_{i}\right)\left(1-2\tilde{\mu}_{i}\right)O_{1,i}(1)O_{3,i}^{(2)}(1)
%\right.\\&-&\left.
-\tilde{\mu}_{i}^{2}\left(1-\tilde{\mu}_{i}\right)^{2}O_{1,i}(1)P_{i}^{(2)}(1)\right.\\
+\left.2\tilde{\mu}_{i}\left[\left(1-2\tilde{\mu}_{i}\right)O_{1,i}(1)-\left(1-\tilde{\mu}_{i}\right)\right]\left(O_{3,i}^{(1)}(1)\right)^{2}
\right.\\-\left.
2\left(1-\tilde{\mu}_{i}\right)\left(1-2\tilde{\mu}_{i}\right)O_{1,i}(1)O_{3,i}^{(1)}(1)
%\right.\\&-&\left.
-2\tilde{\mu}_{i}^{3}\left(1-\tilde{\mu}_{i}\right)^{2}O_{1,i}^{(1)}(1)
\right.\\-\left.
2\left(1-2\tilde{\mu}_{i}\right)O_{3,i}^{(1)}(1)O_{1,i}^{(1)}(1)
%\right.\\&-&\left.
-2\tilde{\mu}_{i}^{2}\left(1-\tilde{\mu}_{i}\right)\left(1-2\tilde{\mu}_{i}\right)O_{1,i}(1)O_{1,i}^{(1)}(1)\right\},
\end{array}
\end{eqnarray}
for $i=1,2,3,4$. 
\end{Teo}
\begin{proof}
Fix $i\in\left\{1,2,3,4\right\}$ and $z\in\mathbb{C}$. To remove the singularities in (\ref{Eq.Long.Caulquier.Tiempo}) it is necessary to define the following analytic functions:
\begin{eqnarray}
\begin{array}{ccc}
\varphi_{i}\left(z\right)=1-F_{i}\left(z\right),&
\psi_{i}\left(z\right)=z-P_{i}\left(z\right),&\textrm{ and }
\varsigma_{i}\left(z\right)=1-P_{i}\left(z\right),
\end{array}
\end{eqnarray}
then
\begin{eqnarray}
\esp\left[C_{i}\right]\mathcal{Q}_{i}\left(z\right)=\frac{\left(z-1\right)\varphi_{i}\left(z\right)P_{i}\left(z\right)}{\psi_{i}\left(z\right)\varsigma_{i}\left(z\right)}.
\end{eqnarray}
For $k\geq0$, define $a_{k}=P\left\{L_{i}^{*}\left(t\right)=k\right\}$. It is obtained that
\begin{eqnarray*}
\varphi_{i}\left(z\right)=1-F_{i}\left(z\right)=1-\sum_{k=0}^{+\infty}a_{k}z^{k},
\end{eqnarray*}
therefore 
\begin{eqnarray*}
\varphi_{i}^{(1)}\left(z\right)&=&-\sum_{k=1}^{+\infty}ka_{k}z^{k-1},\textrm{ with }\varphi_{i}^{(1)}\left(1\right)=-\esp\left[L_{i}^{*}\left(t\right)\right],\textrm{ and}\\
\varphi_{i}^{(2)}\left(z\right)&=&-\sum_{k=2}^{+\infty}k(k-1)a_{k}z^{k-2},\textrm{ hence }\varphi_{i}^{(2)}\left(1\right)=\esp\left[L_{i}^{*}\left(L_{i}^{*}-1\right)\right].
\end{eqnarray*}
In the same way it is gotten that
\begin{eqnarray*}
\varphi_{i}^{(3)}\left(z\right)&=&-\sum_{k=3}^{+\infty}k(k-1)(k-2)a_{k}z^{k-3}\textrm{ and }\varphi_{i}^{(3)}\left(1\right)=-\esp\left[L_{i}^{*}\left(L_{i}^{*}-1\right)\left(L_{i}^{*}-2\right)\right].
\end{eqnarray*}
Expanding $\varphi_{i}\left(z\right)$ around $z=1$,
\begin{eqnarray*}
\varphi_{i}\left(z\right)&=&\varphi_{i}\left(1\right)+\frac{\varphi_{i}^{(1)}\left(1\right)}{1!}\left(z-1\right)+\frac{\varphi_{i}^{(2)}\left(1\right)}{2!}\left(z-1\right)^{2}+\frac{\varphi^{(3)}\left(1\right)}{3!}\left(z-1\right)^{3}+\ldots+\\
&=&\left(z-1\right)\left\{\varphi_{i}^{(1)}\left(1\right)+\frac{\varphi^{(2)}\left(1\right)}{2!}\left(z-1\right)+\frac{\varphi_{i}^{(3)}\left(1\right)}{3!}\left(z-1\right)^{2}+\ldots+\right\}=\left(z-1\right)O_{1,i}\left(z\right)
\end{eqnarray*}
with $O_{1,i}\left(z\right)\neq0$, given that $O_{1,i}\left(z\right)=-\esp\left[L^{*}_{i}\right]$, where
\begin{eqnarray}
O_{1,i}\left(z\right)&=&\varphi_{i}^{(1)}\left(1\right)+\frac{\varphi_{i}^{(2)}\left(1\right)}{2!}\left(z-1\right)+\frac{\varphi_{i}^{(3)}\left(1\right)}{3!}\left(z-1\right)^{2}+\ldots+.
\end{eqnarray}
Calculating the derivatives of $O_{1,i}\left(z\right)$, and evaluating in $z=1$, it is obtained that 
\begin{eqnarray*}
O_{1,i}\left(1\right)&=&-\esp\left[L_{i}^{*}\right]\textrm{, }O_{1,i}^{(1)}\left(1\right)=-\frac{1}{2}\esp\left[(L_{i}^{*})^{2}\right]+\frac{1}{2}\esp\left[L_{i}^{*}\right]\\
\textrm{ and }O_{1,i}^{(2)}\left(1\right)&=&-\frac{1}{3}\esp\left[(L_{i}^{*})^{3}\right]+\esp\left[(L_{i}^{*})^{2}\right]-\frac{2}{3}\esp\left[L_{i}^{*}\right].
\end{eqnarray*}
Proceeding in a similar manner for $\psi_{i}\left(z\right)=z-P_{i}\left(z\right)$ and $\varsigma_{i}\left(z\right)=1-P_{i}\left(z\right)$, it is gotten that
\begin{eqnarray}
\esp\left[C_{i}\right]Q_{i}\left(z\right)&=&\frac{O_{1,i}\left(z\right)P_{i}\left(z\right)}{O_{2,i}\left(z\right)O_{3,i}\left(z\right)}.
\end{eqnarray}
Calculating the derivative with respect to $z$, and evaluating in $z=1$,
\begin{eqnarray*}\label{Ec.Primer.Derivada.Q}
\esp\left[C_{i}\right]\mathcal{Q}_{i}^{(1)}\left(1\right)
&=&\frac{1}{\left(1-\tilde{\mu}_{i}\right)^{2}\tilde{\mu}_{i}^{2}}\left\{\left(-\frac{1}{2}\esp \left[(L_{i}^{*})^{2}\right]+\frac{1}{2}\esp\left[L_{i}^{*}\right]\right)\left(1-\tilde{\mu}_{i}\right)\left(-\tilde{\mu}_{i}\right)\right.\\
&+&\left.
\left(-\esp\left[ L_{i}^{*}\right]\right)\left(1-\tilde{\mu}_{i}\right)\left(-\tilde{\mu}_{i}\right)\tilde{\mu}_{i}%\right.\\
%&-&\left.
-\left(-\frac{1}{2}\esp\left[\tilde{X}_{i}^{2}\left(t\right)\right]+\frac{1}{2}\tilde{\mu}_{i}\right)\left(-\tilde{\mu}_{i}\right)\left(-\esp\left[ L_{i}^{*}\right]\right)\right.\\
&-&\left.\left(1-\tilde{\mu}_{i}\right)\left(-\esp\left[ L_{i}^{*}\right]\right)\left(-\frac{1}{2}\esp\left[\tilde{X}_{i}^{2}\left(t\right)\right]+\frac{1}{2}\tilde{\mu}_{i}\right)\right\}\\
&=&\frac{1}{\left(1-\tilde{\mu}_{i}\right)^{2}\tilde{\mu}_{i}^{2}}\left\{-\frac{1}{2}\tilde{\mu}_{i}^{2}\esp\left[ (L_{i}^{*})^{2}\right]
+\frac{1}{2}\tilde{\mu}_{i}\esp\left[(L_{i}^{*})^{2}\right]
+\frac{1}{2}\tilde{\mu}_{i}^{2}\esp\left[ L_{i}^{*}\right]
-\tilde{\mu}_{i}^{3}\esp\left[ L_{i}^{*}\right]\right.\\
&+&\left.\tilde{\mu}_{i}\esp\left[ L_{i}^{*}\right]\esp\left[\tilde{X}_{i}^{2}\left(t\right)\right]%\right.\\
%&-&\left.
-\frac{1}{2}\esp\left[ L_{i}^{*}\right]\esp \left[\tilde{X}_{i}^{2}\left(t\right)\right]\right\}\\
&=&\frac{1}{2\tilde{\mu}_{i}\left(1-\tilde{\mu}_{i}\right)}\esp\left[ (L_{i}^{*})^{2}\right]-\frac{\frac{1}{2}-\tilde{\mu}_{i}}{\left(1-\tilde{\mu}_{i}\right)^{2}\tilde{\mu}_{i}^{2}}\sigma_{i}^{2}\esp\left[ L_{i}^{*}\right].
\end{eqnarray*}
It means that 
\begin{eqnarray*}
\mathcal{Q}_{i}^{(1)}\left(1\right)=\frac{1}{\tilde{\mu}_{i}\left(1-\tilde{\mu}_{i}\right)}\frac{\esp\left[ (L_{i}^{*})^{2}\right]}{2\esp\left[C_{i}\right]}
-\sigma_{i}^{2}\frac{\esp L_{i}^{*}}{2\esp \left[C_{i}\right]}\cdot\frac{1-2\tilde{\mu}_{i}}{\left(1-\tilde{\mu}_{i}\right)^{2}\tilde{\mu}_{i}^{2}}.
\end{eqnarray*}
Deriving again and evaluating in $z=1$, it follows that
\begin{eqnarray*}
\esp\left[C_{i}\right]\mathcal{Q}_{i}^{(2)}\left(1\right)&=&\frac{1}{\tilde{\mu}_{i}^{3}\left(1-\tilde{\mu}_{i}\right)^{3}}\left\{
-\left(1-\tilde{\mu}_{i}\right)^{2}\tilde{\mu}_{i}^{2}O_{1,i}^{(2)}(1)\right.\\
&-&\left.\tilde{\mu}_{i}\left(1-\tilde{\mu}_{i}\right)\left(1-2\tilde{\mu}_{i}\right)O_{1,i}(1)O_{3,i}^{(2)}(1)
%\right.\\&-&\left.
-\tilde{\mu}_{i}^{2}\left(1-\tilde{\mu}_{i}\right)^{2}O_{1,i}(1)P_{i}^{(2)}(1)\right.\\
&+&\left.2\tilde{\mu}_{i}\left[\left(1-2\tilde{\mu}_{i}\right)O_{1,i}(1)-\left(1-\tilde{\mu}_{i}\right)\right]\left(O_{3,i}^{(1)}(1)\right)^{2}
\right.\\&-&\left.
2\left(1-\tilde{\mu}_{i}\right)\left(1-2\tilde{\mu}_{i}\right)O_{1,i}(1)O_{3,i}^{(1)}(1)
%\right.\\&-&\left.
-2\tilde{\mu}_{i}^{3}\left(1-\tilde{\mu}_{i}\right)^{2}O_{1,i}^{(1)}(1)
\right.\\&-&\left.
2\left(1-2\tilde{\mu}_{i}\right)O_{3,i}^{(1)}(1)O_{1,i}^{(1)}(1)
%\right.\\&-&\left.
-2\tilde{\mu}_{i}^{2}\left(1-\tilde{\mu}_{i}\right)\left(1-2\tilde{\mu}_{i}\right)O_{1,i}(1)O_{1,i}^{(1)}(1)\right\},
\end{eqnarray*}
where $O_{1,i}\left(1\right),O_{1,i}^{(1)}\left(1\right),O_{3,i}^{(1)}(1),O_{3,i}^{(2)}(1),P_{i}^{(2)}(1)$ can be obtained using direct operations.
\end{proof}%\qed
\begin{Remark}
To determine the second order moments for the queue lengths, it is necessary to calculate the third derivative of the arrival processes for each of the queues.
\end{Remark}





\chapter{Transformaciones}
%%___________________________________________________________________________________________
%
% Cap 1
%
%___________________________________________________________________________________________
\section{Transformaciones}

%___________________________________________________________________________________________
%
%___________________________________________________________________________________________
\section{Transformaciones de Variables Aleatorias}
%___________________________________________________________________________________________
Dada una variable aleatoria $X$ existen tres tipos de transformaciones b\'asicas.
\begin{enumerate}
\item Dada $X$ variable aleatoria con funcion de densidad $f_{X}\left(x\right)$ y una transformaci\'on $Y$, tal que para $g:\rea\rightarrow\rea$. Sea $Y=g\left(x\right)$. Se quiere encontrar $f_{Y}\left(y\right)$.
\item Para $X$ variable aleatoria con funcion de densidad $f_{XY}\left(x,y\right)$ y una transformaci\'on $Y$, con $g:\rea^2\rightarrow\rea$. Sea $Y=g\left(x,y\right)$. Encontrar $f_{Y}\left(y\right)$.
\item Sea $X_{1}, X_{2}, X_{3},..., X_{n}$ una sucesi\'on de variables aleatorias con funcion de densidad conjunta $f_{X_{1}},...,f_{X_{n}} \left(x_{1},...,x_{n}\right)$ y una transformaci\'on $Y$, tal que para $g_{i}:\rea^n\rightarrow\rea^n$. Sea 
\begin{eqnarray*}
Y_{1}&=&g_{1}\left(x_{1}\ldots x_{n}\right) \\ 
Y_{2}&=&g_{2}\left(x_{1}\ldots x_{n}\right)\\ 
\vdots \\ 
Y_{n}&=&g_{n}\left(x_{1}\ldots x_{n}\right)
\end{eqnarray*}
Encontrar $f_{Y_{1}\ldots Y_{n}}\left(y_{1} \ldots y_{n}\right)$
\end{enumerate}
%_________________________________________________________________-
\subsection{Ejercicios}
%_________________________________________________________________-
\begin{Ejem}
Sea $X$ V.A distribuida uniformemente sobre el intervalo $ \left(0,1 \right)$, es decir,  $X \backsim U_{\left(0,1\right)}$ y sea $Y$ la transformaci\'on de $X$ definida por  $Y=\frac{1}{\lambda}\left( 1-X \right)$, 
para $\lambda > 0$. Encontrar $f_{Y}\left(y\right)$

Dado que $X$ se distribuye uniforme en el intervalo $\left(0,1 \right)$, se tiene que su funci\'on de densidad $ f_{X}\left(x\right)$ est\'a dada por: 
\begin{eqnarray*}
f_{X}\left(x\right) = \left\{
\begin{array}{cc}
1 & 0< x <1, \\
0 & \textrm {e.o.c.} 
\end{array}
\right.
\end{eqnarray*} 
 Entonces $f_{Y}\left( y \right)=\dfrac{\partial}{\partial y} F_{Y}\left(y \right)$, donde $F_{Y}\left(y \right)$ est\'a dada por:
\begin{eqnarray*}
F_{Y}\left( y \right) &=& \prob \left[ Y \leqslant y \right]  = \prob \left[-\dfrac{1}{\lambda}\ln \left( 1-X\right) \leqslant y \right] =\prob \left[ \ln \left( 1-X \right) \leqslant \lambda y \right]\\
&=&\prob \left[ \ln \left( 1-X \right) > - \lambda y \right] = \prob \left[ \left( 1-X \right) > e^ {- \lambda y}\right] = \prob \left[-X > e^ {-\lambda y} - 1 \right] \\
&=& \prob \left[X \leqslant 1- e^{- \lambda y}\right] = F_{X}\left(1-e^{-\lambda y}\right)
\end{eqnarray*}
Por tanto $F_{Y}\left(y \right)=F_{X}\left(1-e^{-\lambda y}\right)$. Sabemos que $0<X<1 \Rightarrow -1< -X < 0 \Rightarrow 0 < 1-X < 1$.  Aplicando el logaritmo natural en ambos lados $-\infty < \ln \left(1-X\right) < 0\Rightarrow  -\infty < \dfrac{1}{\lambda} \ln \left(1-X\right) < 0\Rightarrow0 < - \dfrac{1}{\lambda}  \ln \left(1-X\right) <\infty $ es decir, $0<y<\infty$. Recordemos que $F_{Y} \left(y \right)= \prob \left[Y \leq y \right]$ para $0 < y < \infty$. Como $X \sim U \left(0,1 \right)$ tenemos que 
$F_{Y}\left(y\right) =1-e^{-\lambda y}$ para $ y \geq 0$, entonces $0< y < \infty \Rightarrow0< \lambda y < \infty\Rightarrow- \infty < -\lambda y < 0\Rightarrow0<e ^{-\lambda y} < 1\Rightarrow-1<e^{-\lambda y} < 0\Rightarrow0<1 - e^{-\lambda y} < 1 $, es decir, su rango est\'a contenido en el intervalo  $ \left( 0,1 \right)$. Finalmente podemos determinar $\dfrac{\partial}{\partial Y} F_{Y} \left( y \right)$
\begin{eqnarray*}
\dfrac{\partial}{\partial Y} F_{y} \left( y \right) &=& -e^{-\lambda y} \left(-\lambda \right) = -\lambda e^{-\lambda y}. \, \, \textrm{es decir}\\
f_{Y}\left(y \right) &=& \left\{
\begin{array}{cc} 
\lambda e^{-\lambda y} &y \geq 0 \\
0 &  \textrm{e.o.c}
\end{array}
\right.
\end{eqnarray*} 
 Por lo tanto se tiene que
$$Y \sim exp \left( \lambda \right)$$ .
\end{Ejem}  

\begin{Ejem}
Sea $X$ v.a con distribuci\'on normal est\'andar,es decir $X \sim N\left(0,1 \right)$. Sea $Y$ transformacion definida por  $Y= \sqrt{x}$. Encontrar $f_{Y}\left(y\right)$.

Al igual que antes recordemos que buscamos a $f_{Y}\left(y\right)=\dfrac{\partial}{\partial y}F_{Y}\left(y\right)$. Dado que:
\begin{eqnarray*}
F_{Y}\left(y\right) &=&\prob \left[ Y \leq y\right] = \prob \left[ X^{1/2}\leq y\right]
\end{eqnarray*}
Ahora, como $ X\sim N \left(0,1 \right)$, entonces $ \prob \left[ \left( X^{1/2} \right)^{2} \leq y^{2} \right]=
 \prob \left[ X \leq y^{2} \right] = F_{X}\left(y^{2} \right)$. Por otra parte, $-\infty < X < \infty$, para $0 < X < \infty$. Por lo tanto se tiene que $0 < \sqrt{X} < \infty$, luego $0 < Y < \infty$. Ahora, $f_{X}\left( x \right)=\frac{1}{\sqrt{2 \pi}}exp ^{\frac{-x^{2}}{2}}$, con $f_{Y}\left(y\right)= f_{X}\left(y^{2}\right)\left(2y\right)$. Lo anterior es cierto por el teorema de cambio de variable. Por tanto:
 $$f_{Y}\left(y\right)= \frac{1}{\sqrt{2\pi}} exp^{ \frac{-y^4}{2}} 2y,$$
para $0<y< \infty$ .
\end{Ejem}

\begin{Ejem}
Sea $X \sim Exp  \left( \lambda \right)$. Encontrar la densidad de $Y = \left[ 3X \right] $. Dado que
\begin{eqnarray*}
X \sim exp \left( \lambda \right) \Rightarrow f_{X}\left( x \right) = \left\{\begin{array}{ll} \lambda  e^{-\lambda x} & x > 0 ,\\ 
0 & e.o.c. 
\end{array}
\right.
\end{eqnarray*}
Entonces tenemos
\begin{eqnarray*}
F_{Y}\left( y \right) &=& \prob\left[Y=y\right]=\prob\left[\left[ 3X\right]=y\right]=\prob\left[y\leq3X<y+1\right]=\prob\left[\frac{y}{3}\leq X\leq\frac{y+1}{3}\right]\\
&=&\int_{y/3}^{(y+1)/3} f_{X}\left(x \right)dx 
=\int_{y/3}^{\frac{y+1}{3}}\lambda e^{-\lambda x}dx = \lambda \int_{y/3}^{(y+1)/3 } e^{-\lambda x}dx =\lambda \Bigg[\dfrac{-e^{-\lambda x}}{\lambda}\Bigg]_{y/3}^{(y+1)/3} \\
&=&-e^{-\lambda\left((y+1)/3 \right)}+e^{-\lambda\left(\frac{y}{3}\right)}=-e^{-\lambda\left(\frac{y}{3}\right)} e^{-\lambda\left(\frac{1}{3}\right)}+e^{-\lambda\left(\frac{y}{3}\right)}=e^{-\lambda\left(\frac{y}{3}\right)}\left[1-e^{\frac{-\lambda}{3}}\right].
\end{eqnarray*}

\end{Ejem}

\begin{Ejem}
Sea $X \sim Gamma \left(2,2 \right)$. Encontrar la densidad de $Y=\dfrac{X}{1+X}$
\begin{eqnarray*}
X \sim Gamma \left( 2,2 \right) \Rightarrow f_{X}\left( x \right)= 
\begin{cases}
\begin{array}{lc}
\frac{2^{2}}{\Gamma\left(2\right)}x^{2-1}e^{2x} & x \geq 0, \\
0 & \textrm{ e.o.c}
\end{array}
\end{cases}
=\begin{cases}
\begin{array}{lc}
4xe^{-2x} & x \geq 0, \\
0 & \textrm {e.o.c}
\end{array}
\end{cases}
\end{eqnarray*}
Sabemos que 
$0<x\leq \infty \Rightarrow 1 \leq x+1 < \infty \Rightarrow 0 \leq \frac{1}{X+1}\leq 1 \Rightarrow 0\leq\frac{X}{X+1}\leq X\leq \infty$. Por tanto
\begin{eqnarray*}
F_{Y}\left(y\right)&=&\prob\left[Y\leq y\right]=\prob \left[\frac{X}{1+X}\leq y\right]=\prob\left[X\leq y\left(1+X\right)\right]= \prob\left[X\leq y+yX\right]\\
&=&\prob\left[X-yX\leq y\right]=\prob\left[X\left(1-y\right)\leq y\right]=\prob\left[X\leq \frac{y}{1-y}\right]= F_{X}\left(\dfrac{y}{1-y}\right).
\end{eqnarray*}
Luego se tiene que
\begin{eqnarray*}
f_{Y}\left(y\right)&=&f_{X}\left(\frac{y}{1-y}\right)\dfrac{\left[\left(1-y\right)-y\left(-1\right)\right]}{\left(1-y\right)^{2}}=\frac{4\left(\frac{y}{1-y}\right)e^{-2\left(\frac{y}{1-y}\right)}}{\left(1-y\right)^{2}}\\
&=&\begin{cases}
\begin{array}{cc}
\frac{4\left(\frac{y}{1-y}\right)e^{-2\left(\frac{y}{1-y}\right)}}{\left(1-y\right)^{3}} & 0<y<1, \\
0 & \textrm {e.o.c.}
\end{array}
\end{cases}
\end{eqnarray*}
Entonces
\begin{eqnarray*}
\int_{0}^{1}\frac{4\left(\frac{y}{1-y}\right)e^{-2\left(\frac{y}{1-y}\right)}}{\left(1-y\right)^{3}}\, dy &=& 4\int_{0}^{1}\frac{y}{\left(1-y\right)^3}e^{-2\left(\frac{y}{1-y}\right)}dy = \left(\frac{y}{1-y}\right)\left(-\frac{1}{2}e^{-2\left(\frac{y}{1-y}\right)}\right)\vert_{0}^{1}\\
&+& 2 \int_{0}^{1} e^{-2\left(\frac{y}{1-y}\right)}\left(\frac{1}{\left(1-y\right)^{2}}\right) dy
\end{eqnarray*}
si hacemos el siguiente cambio de variable
$u=\frac{y}{1-y}\Rightarrow du=\frac{1}{\left(1-y\right)^{2}} dy$ y $dv= \frac{y}{\left(1-y\right)^3}e^{-2\left(\frac{y}{1-y}\right)} dy \Rightarrow v=-\frac{1}{2}e^{-2\left(\frac{y}{1-y}\right)}$, si ahora
$u=\frac{1}{\left(1-y\right)^{2}} \Rightarrow 2\left(1-y\right)dy$, con $dv= e^{-2\left(\frac{y}{1-y}\right)} dy \Rightarrow v= \frac{\left(1-y\right)^2 e^{-2\left(\frac{y}{1-y}\right)}}{2}$ entonces se tiene
$$-2\left(\frac{y}{1-y}\right)=\frac{\left(1-y\right)\left(-2\right)-\left(-2y\right)\left(-1\right)}{\left(1-y\right)^2}= \frac{2}{\left(1-y\right)^2}.$$
\end{Ejem}

\begin{Ejem}
Sea $X \sim \mathcal{P}o\left(\lambda\right)$. Hallar la densidad de $Y=4X+3$. Dado que $X$ se distribuye Poisson
\begin{eqnarray*}
f_{X}\left(x\right)=
\begin{cases}
\begin{array}{lc}
\frac{e^{-\lambda}\lambda^{x}}{x!} & x=0,1,2,\\
0 & \textrm{ e.o.c.}
\end{array}
\end{cases}
\end{eqnarray*}
se tiene que
\begin{eqnarray*}
f_{Y}\left(y\right)&=& \prob\left[Y=y\right]= \prob\left[4X+3=y\right]=\prob\left[4X=y-3\right]=\prob\left[X=\frac{y-3}{4}\right]=f_{X}\left(\frac{y-3}{4}\right)\end{eqnarray*}entonces
\begin{eqnarray*}
f_{Y}\left(y\right)=f_{X}\left(\frac{y-3}{4}\right)\left(\dfrac{1}{4}\right)= 
\begin{cases}
\begin{array}{lc}
\dfrac{e^{-\lambda} \lambda^{\left(\frac{y-3}{4}\right)}}{4\left(\frac{y-3}{4}\right)!} & y=3,7,11,15,\ldots\\
0 & \textrm{ e.o.c.}
\end{array}
\end{cases}
\end{eqnarray*}
Lo anterior es cierto dado que 
$x=0,1,2,3,\cdots$, entonces $4X=0,4,8,12, \cdots$, es decir
$4X+3=3,7,11,15, \cdots$ por tanto $Y=3,7,11,15, \cdots$. 
\end{Ejem}

\begin{Ejem} Sea $X \sim  U\left(0,1\right)$. Hallar una funci\'on $g$ t.q $Y\sim U\left(a,t\right)$, con  $Y=g\left(X\right)$. Dado que $X\sim U\left(0,1\right)$, 
\begin{eqnarray*}
f_{X}\left(x\right) =
\begin{cases}
\begin{array}{lc}
1  & 0\leq x \leq 1, \\
0 & \textrm{ e.o.c.}
\end{array}
\end{cases}
\end{eqnarray*}
Adem\'as
\begin{eqnarray*}
F_{X}\left(x\right)=
\begin{cases}
\begin{array}{lc}
0  & x<0 ,\\
x & 0\leq x \leq 1, \\
1 & x \geq 1.
\end{array}
\end{cases}
\end{eqnarray*}
Se busca que $Y\sim U\left(a,b\right)$, entonces
\begin{eqnarray*}
f_{Y}\left(y\right)=
\begin{cases}
\begin{array}{lc}
\frac{1}{b-a}  & 0\leq y \leq 1, \\
0 & \textrm{ e.o.c.}
\end{array}
\end{cases}
\end{eqnarray*}
por tanto
$$\int_{a}^{y}\frac{1}{b-a} \, dv = \dfrac{y-a}{b-a},$$ entonces
\begin{eqnarray*}
F_{Y}\left(y\right)=
\begin{cases}
\begin{array}{lc}
0  & y<0 \\
\frac{y-a}{b-a} & a\leq y \leq b \\
1 & y > b.
\end{array}
\end{cases}
\end{eqnarray*}

Sea $y \in \left[a,b\right]$, supongamos que  $g$ es creciente:
$F_{Y}\left(y\right)=\prob \left[Y\leq y\right]= \prob\left[g\left(X\right)\leq y\right]= \prob\left[X\leq g^{-1}\left(y\right)\right]$, es decir, $F_{X}\left(g^{-1}\left(y\right)\right)=g^{-1}\left(y\right)$, con $g:\left(0,1\right) \rightarrow \left(a,b\right)$, por tanto $g^{-1}:\left(a,b\right) \rightarrow \left(0,1\right)$. Lo anterior es cierto puesto que 
$F_{X}\left(x\right)=x$ para $x \in \left[0,1\right]$, por lo tanto $g^{-1}\left(y\right) = \frac{y-a}{b-a} \Rightarrow y =g\left(\frac{y-a}{b-a}\right)$. 
Si hacemos $t=\frac{y-a}{b-a}\Rightarrow y-a \Rightarrow y=a+t\left(b-a\right)$, por tanto se propone
$Y=g\left(X\right) = a+ \left(b-a\right)X$.
\end{Ejem}

\begin{Ejem}[\textbf{Importante}] Un insecto deposita un n\'umero grande de huevos, el n\'umero de huevos depositado es una v.a que frecuentemente se asocia una distribuci\'on Poisson $\left(\lambda\right)$. La supervivencia de un cierto huevo tiene probabilidad $p$. Encontrar el n\'umero promedio de huevos sobrevivientes.
Sean las variables aleatorias
 $Y \equiv $ N\'umero de huevos depositados, es decir, $Y \sim Poi\left(\lambda\right)$. $X \equiv $ N\'umero de huevos sobrevivientes $X|Y \sim Bin\left(Y,p\right)$. A saber
 
 \begin{eqnarray*}
\prob \left[X=x\right]&=& \sum_{y=0}^{\infty}\prob \left[X=x   Y=y \right]= \sum_{y=0}^{\infty}\prob \left[X=x | Y=y \right]\prob \left[Y=y \right]\\
&=&\sum_{y=x}^{\infty} \left[ \left( 
\begin{array}{cc}
y \\ x
\end{array} \right) p^{x}\left(1-p\right)^{y-x}\right]\left[\frac{\lambda^{y}e^{-\lambda}}{y!}\right]=\sum_{y=x}^{\infty}\left( \frac{y!}{\left(y-x\right)! x!}p^{x}\left(1-p\right)^{y-x}\right)\left(\frac{\lambda^{y}e^{-\lambda}}{y!}\right)
\end{eqnarray*}
\begin{eqnarray*}
\prob \left[X=x\right] &=&e^{-\lambda}p^{x}\sum_{y=x}^{\infty}\frac{\left(1-p\right)^{y-x}\lambda^{y}}{\left(y-x\right)!x!}=\frac{e^{-\lambda}\left(\lambda p\right)^{x}}{x!} \sum_{y=x}^{\infty}\frac{\left[\left(1-p\right)\lambda\right]^{y-x}}{\left(y-x\right)!}
\end{eqnarray*}
si hacemos el cambio de variable $z=y-x$, obtenemos
\begin{eqnarray*}
\prob \left[X=x\right]&=& \frac{e^{-\lambda}\left(\lambda p\right)^{x}}{x!} \sum_{z=0}^{\infty}\frac{\left[\left(1-p\right)\lambda\right]^{z}}{z!}=\frac{e^{-\lambda}\left(\lambda p\right)^{x}}{x!}\sum_{z=0}^{\infty}\frac{\left[\left(1-p\right)\lambda\right]^{t}}{t!}\\
&=&\frac{e^{-\lambda}\left(\lambda p\right)^{x}}{x!} \left(1 + \left(1-p\right)\lambda + \frac{\left(\left(1-p\right)\lambda\right)^{2}}{2!}+ \cdots \right)=\frac{e^{-\lambda}\left(\lambda p\right)^{x}}{x!} e ^{\left(1-p\right)\lambda}\\
&=& \frac{e^{-\lambda p}\left(\lambda p\right)^{x}}{x!}.
\end{eqnarray*}
Por lo tanto 
$\prob\left[X=x\right] = \frac{e^{-\lambda p}\left(\lambda p\right)^{x}}{x!}$, es decir, $X \sim \mathcal{P}o\left(\lambda p \right)$, por lo tanto $E\left[X\right] = \lambda p$.
\end{Ejem}

\begin{Ejem}
\item Sea $X,Y$ variables aleatorias independientes, tales que $X\sim U\left(0,1\right)$ y $Y\sim U\left(0,2\right)$. Def\'inase la nueva variable aleatoria $Z=X+Y$, determinar la densidad de $Z$.
A saber
\begin{eqnarray*}
\begin{array}{ll}
X\sim U_{0,1} \Rightarrow f_{X}\left( x \right)= 
\begin{cases}
1 & 0\leq x \leq 1\\
0 & e.o.c
\end{cases},
&
Y\sim U_{0,2} \Rightarrow f_{Y}\left( y \right)= 
\begin{cases}
\frac{1}{2} & 0\leq y \leq 2\\
0 & e.o.c
\end{cases}
\end{array}
\end{eqnarray*}
Por tanto la densidad conjunta es
\begin{eqnarray*}
f_{XY}\left( x,y \right)= f_{X}\left(x\right)f_{Y}\left(y\right)
\begin{cases}
\frac{1}{2} & 0\leq x \leq 1, 0\leq y \leq 2,\\
0 & e.o.c
\end{cases}
\end{eqnarray*}
con distribuci\'on de probabilidad
\begin{eqnarray*}
F_{Z}\left(z\right)= \mathbb{P}\left[X+Y\leq z \right]= \mathbb{P}\left[Y \leq z-X \right],
\end{eqnarray*}
adem\'as el rango de $Z$ es el intervalo $\left(0,3\right)$.
\end{Ejem}

\begin{Ejem}
\item  Sea $X,Y$ variables aleatorias tales que $X \sim exp \left( \lambda \right)$ y $Y \sim exp \left( \lambda\right)$ independientes. Sea $Z=min\left\{x,y\right\}$. Hallar $f_{z} \left( z \right)$.

A saber
\begin{eqnarray*}
\begin{array}{cc}
X\sim exp \left(\lambda\right) \Rightarrow f_{X}\left( x \right)= 
\begin{cases}
\lambda e^{- \lambda x }  & x > 0\\
0 & e.o.c
\end{cases}
,&
Y\sim exp \left(\lambda\right) \Rightarrow f_{Y}\left( y \right)= 
\begin{cases}
\lambda e^{- \lambda y }  & y > 0\\
0 & e.o.c
\end{cases}
\end{array}
\end{eqnarray*}
Entonces,
\begin{eqnarray*}
f_{XY} \left( x,y\right) = f_{X}\left(x\right)f_{Y}\left(y\right)= \left(\lambda e^{- \lambda x} \right)\left(\lambda e ^{-\lambda y } \right)=
\begin{cases}
\lambda^{2} e^{- \lambda \left(x+y \right) }  & x > 0\\
& y>0\\
0 & e.o.c
\end{cases}
\end{eqnarray*}
Sea $z\geq 0$, entonces
\begin{eqnarray*}
F_{z} \left( z \right)&=& \mathbb{P} \left[Z \leq z \right]= \mathbb{P} \left[ min \{X,Y \} > Z \right]=1- \mathbb{P} \left[ X < z , Y > z\right]= 1- \mathbb{P} \left[ X < z\right]\mathbb{P}\left[ Y > z\right]\\
&=& 1-\mathbb{P}\left[ X > z\right]^{2} =  1 -\left( \mathbb{P} \left[ X \leq z\right] \right) ^{2}=1 - \left( 1- F_{X} \left( z \right) \right)^{2}.
\end{eqnarray*}
Por lo tanto
\begin{eqnarray*}
F_{X} \left( x \right) &=& \int_{0}^{x} \lambda e^{-\lambda u} \, du = \lambda \int_{0}^{x}  e^{-\lambda u} \, du = x\left(-e ^{- \lambda u} /\! x\right)_{0}^{x}= -e ^{- \lambda x} + 1 = 1 -e ^{- \lambda x}\\
&=& 1 - \left(1-1-e ^{- \lambda x} \right)^{2} = 1- e ^{- 2 \lambda z},
\end{eqnarray*}
por lo tanto $z>0$, as\'i
\begin{eqnarray*}
f_{x}\left(z \right) = \left(-e^{-2 \lambda z } \right)\left(-2 \lambda \right)= 
\begin{cases}
2 \lambda e^{-2 \lambda z }  & z > 0\\
0 & e.o.c.
\end{cases},
\end{eqnarray*}
es decir $z=min\{x,y\}\sim exp\left( 2 \lambda \right) $.
\end{Ejem}
%\begin{center}
%_______________________________________________________________________________
\section{Cambio de Variable}
%_______________________________________________________________________________

\begin{Teo}
Sean $X_{1}, X_{2},\ldots,X _{n}$ v.a continuas con densidad conjunta $f_{x_{1}\ldots x_{n}}\left( x_{1} \ldots x_{n}\right)$ y $Y_{1}=g_{1} \left(x_{1}\ldots x_{n}\right),Y_{2}=g_{2} \left(x_{1}\ldots x_{n}  \right)\ldots $ $Y_{n}= g_{n} \left(x_{1}\ldots x_{n}\right)$. Si existe soluci?n ?nica $x_{1}=h_{1}\left(y_{1}\ldots y_{n}\right),x_{2}=h_{2}\left(y_{1}\ldots y_{n}\right),\ldots$,$x_{n}=h_{n} \left(y_{1}\ldots y{n}\right)$, con 
\begin{eqnarray*}
J= \left| 
\begin{array}{cccc}
\frac{\partial h_{1}}{\partial y_{1}} & \frac{\partial h_{1}}{\partial y_{2}} & \ldots & \frac{\partial h_{1}}{\partial y_{n}}\\
\vdots & \vdots & \vdots & \vdots \\
\frac{\partial h_{n}}{\partial y_{1}} & \frac{\partial h_{2}}{\partial y_{2}} & \ldots & \frac{\partial h_{n}}{\partial y_{n}}
\end{array}
\right|
\neq 0.
\end{eqnarray*}
 Entonces
\begin{eqnarray*}
f_{Y1, \ldots Y_{n}}\left( y_{1} \ldots y_{n} \right)=\begin{cases}
\mid J \mid f_{X1 \ldots Xn}\left[h_{1} \ldots h_{n}\right] & \left(y_{1} \ldots y_{n} \right) \in D \\  
0 & e.o.c.
\end{cases}
\end{eqnarray*}
\end{Teo}
%_________________________________________________________________
\subsection{Ejercicios}
%_________________________________________________________________

\begin{Ejem}
Sean $X,Y$ variables aleatorias independientes tales que $X \sim Gamma \left( \alpha _{1},\lambda\right)$, $Y \sim Gamma \left( \alpha_{2}, \lambda \right)$. Hallar $f_{u,v}$ donde $ U= \dfrac{Y}{X}$ y $V = X+Y$ ?Son $U$ y $V$ independientes?.


A saber
\begin{eqnarray*}
X\sim Gamma \left( \alpha _{1},\lambda \right)\Rightarrow f_{X} \left(x \right)=
\begin{cases}
\dfrac{\lambda^{\alpha_{1}} }{\Gamma \left(\alpha_{1}\right)} x^{\alpha_{1}-1}e^{-\lambda x} & x>0, \\ 
0 & e.o.c
\end{cases}.\\
Y\sim Gamma \left( \alpha _{2},\lambda \right)\Rightarrow f_{Y} \left(y \right)=
\begin{cases}
\dfrac{\lambda^{\alpha_{2}} }{\Gamma \left(\alpha_{2}\right)} y^{\alpha_{2}-1}e^{-\lambda y} & y>0, \\ 
0 & e.o.c.
\end{cases}
\end{eqnarray*}

entonces
\begin{eqnarray*}
f_{XY} \left(x,y \right)=
\begin{cases}
\dfrac{\lambda^{\alpha_{1}+\alpha_{2} } }{\Gamma \left(\alpha_{1}\right)\Gamma \left(\alpha_{2}\right)} x^{\alpha_{1}-1} y^{\alpha_{2}-1}e^{-\lambda \left( x+y \right) } & y>0, x>0, \\ 
0 & e.o.c.
\end{cases}
\end{eqnarray*}
Si $U=\frac{y}{x} \Rightarrow y=ux$, entonces $v=x+y \Rightarrow v= x+ux \Rightarrow v= \left( 1+u \right)x \Rightarrow x= \frac{v}{1+u}$, por lo tanto se tiene que $y= u \left( \dfrac{v}{1+u}\right)$, entonces
\begin{eqnarray*}
\begin{vmatrix}
\frac{\partial x}{\partial u} & \frac{\partial x}{\partial v} \\\\
\frac{\partial y}{\partial u} & \frac{\partial y}{\partial v}\\
\end{vmatrix} = \begin{vmatrix}
\frac{-v}{\left(1+u \right)^{2} } & \frac{1}{\left(1+u \right)} \\\\
v\left(1+u \right)^{-1}-uv \left(1+u \right)^{-2} & \frac{u}{1+u}\\
\end{vmatrix}=-\frac{-vu}{\left( 1+u \right)^{3}}-\dfrac{v}{\left(1+u \right)^{2}}+ \frac{-vu}{\left(1+u \right)^{3}}= v \left(1+u \right)^{-2}.
\end{eqnarray*}
Por lo tanto 
\begin{eqnarray*}
f_{uv} \left(u,v \right)&=&|J| f_{XY}\left(x,y \right)=v \left(1+u \right)^{2} \dfrac{\lambda^{\alpha_{1}+\alpha_{2} } }{\Gamma \left(\alpha_{1}\right)\Gamma \left(\alpha_{2}\right)} \left( \frac{v}{1+u}\right)^{ \alpha_{1}-1} \left( \frac{uv}{1+u} \right)^{\alpha_{2}-1}e^{-\lambda \left(\frac{v+uv}{1+u} \right)}\\
&=&\frac{u^{\alpha_{2}-1}v^{\alpha_{1}+\alpha_{2}-2}}{\left(1+u \right)^{ \alpha_{2}+ \alpha_{2}}} \dfrac{\lambda^{\alpha_{1}+\alpha_{2} } }{\Gamma\left(\alpha_{1}\right)\Gamma\left(\alpha_{2}\right)} e^{-\lambda v }.
\end{eqnarray*}
\end{Ejem}

\begin {Ejem}
Sea $X,Y$ variables aleatorias, tales que $X \sim exp \left( 1 \right)$ y $Y \sim U \left( 0,1 \right)$ independientes. Sean $U=X+Y$ y $V=X-Y$, hallar $f_{U,V} \left( u,v \right)$ y averiguar independencia.

A saber
\begin{eqnarray*}
\begin{array}{cc}
f_{X}\left( x \right)=
\begin{cases}
e^{-x} & x>0, \\ 
0 & e.o.c.
\end{cases},&
f_{Y}\left(y \right)=
\begin{cases}
1 & 0 \leq  y \leq 1, \\ 
0 & e.o.c.
\end{cases}
\end{array}
\end{eqnarray*}
Entonces para $u=x+y$, $x=u-y$m adem\'as $v=x-y$ implica  que $y=x-v$, por lo tanto $y=u-y-v \Rightarrow y= \frac{u-v}{2}$. Por lo tanto $x=u-\left(\frac{u-v}{2}\right)=\frac{2u-u+v}{2}=\frac{u+v}{2}$.
Haciendo $T \left(t,0 \right)=\left( t,t \right)$, $T \left(0,t \right)=\left( t,-t \right)$ y $T \left(t,1 \right)=\left( t+1,t-1 \right)$, tenemos que $T \left(u,v \right) = \left( x+y,x-y \right)$. Sean 
\begin{eqnarray*}
\sigma_{1} \left(t \right)&=&\left( t,0 \right)\textrm{ para }t>0, \\
\sigma_{2} \left(t \right)&=&\left( 0,t \right)\textrm{ para }0<t<1 \textrm{ y }v=-u, \\
\sigma_{3} \left(t \right)&=&\left( t,1 \right) \textrm{ para } t>0. 
\end{eqnarray*}
Ahora, si hacemos $u=t \Rightarrow u=v$, $u=t+1$, entonces $v=t$ y $v= t-1$, entonces
$u=t v=t-1 \Rightarrow u=-v \Rightarrow u-1=t v+1=t$ y $u-1=v+1 \Rightarrow v=u-2$. Por tanto
\begin{eqnarray*}
J=
\begin{vmatrix}
\frac{1}{2} & \frac{1}{2}\\
\frac{1}{2} & -\frac{1}{2}\\
\end{vmatrix}=\frac{1}{2}. 
\end{eqnarray*}
Luego
\begin{eqnarray*} 
f_{u,v} \left(u,v \right) &=& |j| f_{XY}\left( u,v \right)=
\begin{cases}
\frac{1}{2} e^{-\left(\frac{u+v}{2} \right)} & x,y\in D,\\
0 & e.o.c.\\
\end{cases}
\end{eqnarray*}
Por tanto
\begin{eqnarray*} 
f_{u}\left(u \right) = \begin{cases}
\int_{-u}^{u}\frac{1}{2}e^{- \frac{u+v}{2} }\,dv & 0 \leq u \leq 1, \\
\int_{u-2}^{u}\frac{1}{2}e^{- \frac{u+v}{2} }\,dv & 0 \geq 1,  \\
0 & e.o.c.\\
\end{cases}
\end{eqnarray*}
Donde
\begin{eqnarray*} 
\int _{-u}^{u}\frac{1}{2}e^{- \frac{u+v}{2} }\,dv &=&\frac{1}{2}e^{- \frac{u}{2} } \int _{-u}^{u} e^{- \frac{v}{2} } \,dv= \left( \frac{1}{2} e^{-\frac{u}{2} } \right)\left(\frac{ e^-{\frac{v}{2} } }{-\frac{1}{2}}\right)_{-u}^{u}=
 \left( \frac{1}{2} e^{-\frac{u}{2} } \right)\left(-2 e^{\frac{-v}{2}}\right)_{-u}^{u}\\
 &=& \left( \frac{1}{2} e^{-\frac{u}{2} } \right)\left(-2 \left(e^{\frac{-u}{2}}- e^{\frac{u}{2}}\right)\right)= -e^{-u}+1=1-e^{-u}.\\
 \int_{u-2}^{u} e^{\frac{v}{2} }\,dv &=& \left(-2e^{-\frac{v}{2} } \right)_{u-2}^{u}=-e^{-\frac{u}{2}}+ e^{-\frac{\left( u-2\right)}{2}}= e^{-\frac{u}{2}}\left(-e^{-\frac{u}{2}}+ e^{-\frac{\left( u-2\right)}{2}} \right)\\
 &=& -e^{-u}+e^{-\frac{u}{2} -\frac{\left( u-2\right)}{2}}=-e^{-u}+e^{-u}e^{+1} = e^{1-u}-e^{-u}.
\end{eqnarray*}
Luego
\begin{eqnarray*} 
f_{u} \left(u \right) = \begin{cases}
1-e^{-u} & 0 \leq u \leq 1, \\
e^{1-u}-e^{-u} & u\geq1,  \\
0 & e.o.c.
\end{cases}
\end{eqnarray*}
De manera an\'aloga para $V$:
\begin{eqnarray*} 
f_{v} \left( v \right) = \begin{cases}
\int_{-v}^{v+2} \frac{1}{2}e^{\left(\frac{u+v}{2}\right)}\,du & -1\leq v \leq 0,\\
\int_{u}^{v+2} \frac{1}{2}e^{\left(\frac{u+v}{2}\right)}\,du & v \geq 0,\\
0 & e.o.c.
\end{cases}
= \begin{cases}
e^{-\frac{v}{2}}\left(e^{\frac{v}{2}}-e^{- \frac{v+2}{2} } \right)= 1-e^{\left(v+1 \right) }  & -1\leq v \leq 0,\\
-e^{-\left(v+1 \right)}+e^{-v} & v \geq 0,\\
0 & e.o.c.
\end{cases} 
\end{eqnarray*}
\end{Ejem}

\begin{Ejem} Sean $X_{1},X_{2}$ variables aleatorias independientes tales que $X_{1}, X_{2} \sim U_{ \left(-3,3 \right) }$. Sean $Y_{1}=X_{2}-X_{1}, Y_{2}= X_{1}-X_{2}$. Entonces
$y_{1}&=&x_{2}-x_{1} \Rightarrow x_{2}=y_{1}+x_{1} \Rightarrow x_{2} = y_{1}+y_{2}-x_{2} \Rightarrow x_{2}=\frac{y_{1}+y_{2}}{2}$, $y_{2}&=& x_{1}+x_{2} \Rightarrow x_{1}= y_{2}-x_{2} \Rightarrow x_{1}=y_{2}-\left( \frac{y_{1}+y_{2}}{2} \right) = \frac{2y_{2}-y_{1}-y_{2}}{2}$, por lo tanto $x_{1}=\frac{y_{2}-y_{1}}{2}$. Entonces
\begin{eqnarray*}
J=
\begin{vmatrix}
-\frac{1}{2} & \frac{1}{2}\\
\frac{1}{2} & \frac{1}{2}
\end{vmatrix} = -\frac{1}{4}-\frac{1}{4}= -\frac{1}{2}, 
\end{eqnarray*}
es decir $|J|=\frac{1}{2}$. Por tanto
\begin{eqnarray*}
f_{Y_{1},Y_{2}}\left( y_{1},y_{2}\right)&=& \frac{1}{2} f_{X_{1},X_{2}}\left( \frac{y_{2}-y_{1}}{2}, \frac{y_{2}+y_{2}}{2}\right)
=\frac{1}{2} \left( \frac{1}{36}\right)= \begin{cases}
\frac{1}{72}  & x,y \in D\\
0 & e.o.c.
\end{cases}
\end{eqnarray*}
Si hacemos $\sigma_{1}\left(t \right) = \left(t,3 \right)$, $\sigma_{2}\left(t \right) = \left(-3,t \right)$, $\sigma_{3}\left(t \right) = \left(t,-3 \right)$, $\sigma_{4}\left(t \right) = \left(3,t \right)$ ,$T\left(t,3 \right)= \left(3-t, t+3 \right)$ ,$T\left(-3,t \right)=\left(t+3, -3+t \right)$ ,$T\left(t,-3 \right)=\left(-\left(3+t \right), -3+t \right)$ y $T\left(3,t \right)=\left(t-3,3+t \right)$. Entonces tenemos que 

\begin{eqnarray*}
y_{1}&=&3-t,\\
y_{2}&=& -3+t \Rightarrow t=y_{1}-3 \Rightarrow t=y_{2}+3 \Rightarrow y_{1}-3=y_{2}+3,\\
y_{2}&=&y_{2}=y_{1}-6,\\
y_{1}&=&-\left(3+t \right),\\
y_{2}&=& -3+t \Rightarrow t=-\left(3+y_{1} \right),\\ 
y_{2}&=&-3+t \Rightarrow t=-\left(3+y_{1} \right)= y_{2}+3\Rightarrow -3 \left(3+y_{1} \right)= y_{2}+3,
 \end{eqnarray*}
 luego
 \begin{eqnarray*}
 y_{2}&=&-6-y_{1}\textrm{ es decir,}\\
 y_{1}&=&t-3,\textrm{ con}\\
 y_{2}&=&3+t.\\
\end{eqnarray*}
Luego 
\begin{eqnarray*}
\begin{array}{ll}
y_{1}= 3-t, & y_{2}=t+3,\textrm{ por tanto}\\
\Rightarrow t=3-y_{1} & t= y_{2}-3.\textrm{ Entonces}\\
3-y_{1}=y_{2}-3 \Rightarrow y_{2}=6-y_{1} \Rightarrow y_{1}+3=t, & t=y_{2}-3.\\
\Rightarrow y_{1}+3= y_{2}-3 \Rightarrow y_{2}=y_{1}+6. & 
\end{array}
\end{eqnarray*}
Entonces
\begin{eqnarray*}
\begin{array}{cc}
y_{2}=6+y_{1}, & \left(0,6 \right)\left(6,0\right),\\
y_{2}=y_{1}-6, & \left(0,-6 \right)\left(6,0\right),\\
y_{2}=-y_{1}-6, & \left(0,-6 \right)\left(-6,0\right)\textrm{ y }\\
y_{2}=y_{1}+6, & \left(0,6 \right)\left(-6,0\right).
\end{array}
\end{eqnarray*}
Por lo tanto obtenemos
\begin{eqnarray*}
f_{Y_{2}} \left(y_{2}\right) &=& \begin{cases}
\int _{-y_{2}-6}^{y_{2}+6} f_{Y_{1}Y_{2}}\left(y_{1}, y_{2} \right)dy_{1} & -6\leq y_{2}\leq 0,\\
\int _{y_{2}-6}^{6-y_{2}} f_{Y_{1}Y_{2}}\left(y_{1}, y_{2} \right)dy_{1} & 0 < y_{2}\leq 6,\\  
0 & e.o.c.
\end{cases}\\
f_{Y_{1}} \left(y_{1}\right)& =& \begin{cases}
\int _{-y_{1}-6}^{y_{1}+6} f_{Y_{1}Y_{2}}\left(y_{1}, y_{2} \right)dy_{2} & -6\leq y_{1}\leq 0,\\
\int _{y_{1}-6}^{6-y_{1}} f_{Y_{1}Y_{2}}\left(y_{1}, y_{2} \right)dy_{2} & 0 < y_{1}\leq 6,\\  
0 & e.o.c.
\end{cases}\\
f_{Y_{1}} \left(y_{1}\right) &=& \begin{cases}
\frac{1}{36} y_{1}+\frac{1}{6}  & -6\leq y_{1}\leq 0,\\
\frac{1}{6}-\frac{1}{36}y_{1} & 0 < y_{1}\leq 6,\\  
0 & e.o.c.
\end{cases}\\
f_{Y_{2}} \left(y_{2}\right) &=& \begin{cases}
\frac{1}{6}+ \frac{1}{36}y_{2} & -6\leq y_{2}\leq 0,\\
\frac{1}{6}-\frac{1}{36}y_{1} & 0 < y_{2}\leq 6,\\  
0 & e.o.c.
\end{cases} 
\end{eqnarray*}
\end{Ejem}

%_____________________________________________________________________
\section{Estad?sticas de Orden}
%_____________________________________________________________________

Sea $X_{1}... X_{n}$ muestra aleatoria de variables continuas. Se definen las estad?sticas de orden como:

\begin{eqnarray*}
 Y_{1}&\equiv & \textrm{min} \left\{X_{1}... X_{n}\right\}, \textrm{estad?stica de orden 1}.\\
 Y_{j}&\equiv & \textrm{La j-?sima m?s chica de la muestra est. de orden j.}\\
 Y_{n}&\equiv & \textrm{m?x}\left\{X_{1}...X_{n}\right\}, \textrm{estad?stica de orden n}.
\end{eqnarray*}
Entonces $Y_{1} < &Y_{2}& <...<Y_{n}$.

%_____________________________________________________________________
\subsection{Ejercicios}
%_____________________________________________________________________
\begin{Ejem}
Sean $X,Y$ variables aleatorias tales que $X,Y \sim Geo \left( p \right)$, es decir $f_{X}\left(x \right)=pq^{x}$, para $x=0,1,\ldots$. 
Sean $U=X V= \textrm {min} \left\{X,Y \right\}$. Hallar la densidad conjunta de $U$ y $V$

\begin{eqnarray*}
\begin{array}{ll}
f_{X}\left(x\right) = \begin{cases}
pq^{x} & x=0,1,...\\  
0 & \textrm {e.o.c.}
\end{cases},  &
f_{Y}\left(y\right) = \begin{cases}
pq^{y} & y=0,1,...\\  
0 & \textrm {e.o.c.}
\end{cases} 
\end{array}
\end{eqnarray*}
Por lo tanto
\begin{eqnarray*}
f_{u,v}\left(u,v \right) &=& \prob \left[U=u , V=v \right]= \prob \left[ X=u, \textrm{min}\left\{X,Y \right\}=v \right]=\prob \left[X=u, X=v,Y>v \right]\\
&+&\prob \left[X=u, Y=v, X>v \right]+\prob \left[X=u, X=v,Y=v \right].
\end{eqnarray*}
 
 Caso 1:$u=v$
\begin{eqnarray*}
f_{u,v}\left(u,v \right)&=&\prob \left[X=u, Y>v \right]+ \prob \left[X=u, Y=u \right]=\prob \left[ X=u\right]\prob \left[Y\geqslant u \right].
\end{eqnarray*}
Recordemos que 
\begin{eqnarray*}
F_{X}\left(x \right) &=& \sum_{v=0}^{x} p \left(1-p \right)^{v} = p \left[1+ \left(1-p \right)+\left(1-p \right)^{2}+...+ \left(1-p\right)^{x} \right]\\
&=& p \left[ \frac{1- \left( 1-p \right)^{x+1}} {1-\left(1-p \right)} \right] = p \left[\frac{1-\left(1-p \right)^{x+1} }{p} \right]= 1-\left(1-p \right)^{x+1}.
\end{eqnarray*}
Entonces
\begin{eqnarray*}
f_{u,v}\left(u,v \right)&=& p \left(1-p \right)^{u} \sum_{y=u}^{\infty} p \left(1-p \right)^{y}= p^{2}\left(1-p  \right)^{u} \sum_{y=u}^{\infty} \left(1-p\right)^{y}\\
&=& p^{2}\left[\left(1-p \right)^{u}\left(1-p \right)^{u}+ \left(1-p \right)^{u+1}+...+ \right]= p^{2}\left(1-p \right)^{2u}\left[1 + \left(1-p \right)+ ...+ \right]\\
&=& p^{2}\left(1-p \right)^{2u} \frac{1}{1-\left(1-p \right) } = p\left(1-p \right)^{2u}.
\end{eqnarray*}  

Caso 2: $u>v$
\begin{eqnarray*}
\prob \left[X=u, Y=v \right]&=&\prob \left[X=u \right]\prob \left[Y=v \right]= p \left(1-p \right)^{u} p \left(1-p \right)^{v}\\
&=& p^{2}\left( 1-p \right) ^{u+v}\\
u>v,f_{u,v} \left(u,v \right)= 0.
\end{eqnarray*}
\end{Ejem}

\begin{Ejem}
Para $f_{XY}\left(x,y \right)= \frac{1}{8}\left(y^{2}- x^{2} \right)e^{-y}$, con $-y\leq x \leq y$, $0<y<\infty $. Hallar la densidad de $Z= max \left\{ X,Y\right\}$.
 
A saber
\begin{eqnarray*}
F_{z}\left( z \right)= \prob\left[ Z\leq z \right] = \prob \left[ \textrm{max} \left\{ X,Y \right\} \leq z \right]=\prob \left[X \leq  z, Y\leq z \right]
\end{eqnarray*}
Dado $y\geq 0 \Rightarrow z\geq 0$, $\prob \left[X\leq  0  Y \leq 0\right]= 0$ para $z=0$. Por lo tanto
\begin{eqnarray*}
\prob  \left[X \leq 1,Y\leq 1 \right]&=&\prob \left[Z\leq z,Y\leq z \right]= \int_{0}^{z}\int_{-y}^{y}\frac{1}{8}\left(y^{2}- x^{2} \right)e^{-y} \,dx\,dy\\
&=&e^{-z}\left(-\frac{1}{6}z^{3}-\frac{1}{2}z^{2}-z-1 \right)+1
\end{eqnarray*}
Por lo tanto
\begin{eqnarray*}
f_{z}\left(z \right)&=&e^{-z}\left(-\frac{3}{6}z^{2}-z-1 \right)- e^{-z}\left(-\frac{1}{6}z^{3}-\frac{1}{2}z^{2}-z-1 \right)\\
&=&e^{-z}\left(-\frac{1}{2}z^{2}-z-1+\frac{1}{6}z^{3}+\frac{1}{2}z^{2}+z+1 \right).
\end{eqnarray*}
Por lo tanto
 
\begin{eqnarray*}
f_{z}\left(z\right) = \begin{cases}
\frac{1}{6}e^{-z}z^{3} & z>0,\\  
0 & \textrm {e.o.c.}
\end{cases}  
\end{eqnarray*}
\end{Ejem}

\begin{Ejem}
Sean $X_{1},X_{2},X_{3}$ m.a. $U\left(0,1\right)$. Hallar $f_{Yj}\left( t \right) j=1,2,3$, $Y_{j}$ es la estad\'istica de orden $j$. Entonces 

\begin{eqnarray*}
\begin{array}{ll}
f_{X_{1}X_{2}X_{3}}\left(x_{1},x_{2},x_{3} \right) = \begin{cases}
1 & x_{1},x_{2},x_{3} \in \left(0,1\right),\\  
0 & e.o.c.
\end{cases} ,& 
f_{Y_{1}Y_{2}Y_{3}}\left(y_{1},y_{2},y_{3} \right) = \begin{cases}
3! & 0<y_{1}<y_{2}<y_{3}<1,\\  
0 & \textrm {e.o.c.}
\end{cases}  
\end{array}
\end{eqnarray*}
Por lo tanto, calculando la marginal con respecto a $Y_{1}$
\begin{eqnarray*}
f_{Y_{1}}\left( y_{1}\right)&=&\int_{y_{1}}^{y_{3}}\int_{y_{1}}^{1}6dy_{3}dy_{2} = 6\int_{y_{1}}^{1} y_{2}\mid _{y_{1}}^{y_{3}}dy_{3}, 0<y_{1}<1\\
&=& 6\int_{y_{1}}^{1}\left( y_{3}-y_{1} \right)dy_{3}= 6 \left[\frac{y_{3}^{2}}{2}+y_{1}y_{3}\right]_{y_{1}}^{1}=6\left[\frac{1}{2}+y_{1}-\left[\frac{y_{1}^{2}}{2}+ y_{1}^{2}\right]\right]\\
&=& 6\left[\frac{1}{2}+y_{1}-\frac{y_{1}^{2}}{2} \right]=12\left[1-\frac{y_{1}}{2}+y_{1}^{2}\right].
\end{eqnarray*}
Por lo tanto
\begin{eqnarray*}
f_{Y_{2}}\left(y_{2}\right)=\int _{y_{2}}^{1} \int_{0}^{y_{2}} \,dy_{1}\,dy_{3}= \begin{cases}
6y_{2}\left(1-y_{2}\right) & 0<y_{2}<1,\\  
0 & \textrm{e.o.c.}
\end{cases}
\end{eqnarray*}
\end{Ejem}

%_____________________________________________________________
\section{Esperanza}
%_____________________________________________________________

Si $ \left(X,Y \right)$ es un vector aleatorio, la esperanza de $g \left(X,Y \right) $ es:
\begin{eqnarray*}
E \left[ g \left(Z,Y \right)\right] = \begin{cases}
\int  \int g \left(X,Y \right)f_{XY}\left(x,y \right)dxdy &\textrm{ caso continuo,} \\  
\sum \sum g \left( x,y\right)f_{XY}\left(x,y \right)&\textrm{ caso discreto.}
\end{cases}
\end{eqnarray*}
Si $g \left(X,Y \right)=X$
\begin{eqnarray*}
E \left[X \right]&=& \int  \int x  f_{XY}\left(x,y \right)dydx=\int x \left( \int f_{XY}\left(x,y \right)dy \right)dx=xf_{X}\left(x \right)dx.
\end{eqnarray*}
Si $X$ y $Y$ son independientes $$E \left[ h_{1}\left( X \right)h_{2}\left(Y \right)  \right]= E \left[ h_{1}\left( X \right) \right] E \left[ h_{2}\left( X \right) \right]$.
%_____________________________________________________________
\subsection{Ejercicios}
%_____________________________________________________________

\begin{Ejem}
Sea $X$ v.a con densidad

\begin{eqnarray*}
f_{X}\left(x \right) = \begin{cases}
\frac{1}{3} & x=-2  \\  
\frac{1}{2} & x=3  \\  
\frac{1}{6} & x=1  \\  
0 & \textrm{e.o.c}
\end{cases}
\end{eqnarray*}
Hallar:
\begin{itemize}
\item[a)] $E \left[X \right] $
\item[b)] $E \left[2X+5 \right] $
\item[c)] $E \left[X^{2} \right] $
\item[d)] $Var \left(X \right) $

\begin{eqnarray*}
E \left(X \right)&=& -2 \left(\frac{1}{3} \right)+3 \left(\frac{1}{2}+ \left(\frac{1}{6} \right) \right)= -\frac{2}{3}+\frac{3}{2}+ \frac{1}{6}= \frac{-4+9+1}{6}= \frac{6}{6}=1\\
E \left[2X+5 \right]&=& 2 \left[X \right]+5=2+5=7\\
E \left[X^{2} \right]&=& \sum_{x}x^{2}f_{X}\left(x\right)=\left(-2 \right)^{2}\left(\frac{1}{3} \right)+ \left(3 \right)^{2}\left(\frac{1}{2}\right)+ \left(\frac{1}{6} \right)= \frac{4}{3}+\frac{9}{2}+\frac{1}{6}\\
&=& \frac{8+27+1}{6}=\frac{36}{6}=6\\
\textrm{Var}\left(X \right)&=& E \left[X^{2} \right]+ E^{2}\left(X \right)=6+1=7
\end{eqnarray*}
\end{Ejem}
\begin{Ejem}
Si $X\sim \textrm{Bin} \left(n,Q \right)$ y $Q\sim \textrm{Beta} \left(4,2 \right)$. Hallar $E \left(X \right) \textrm{Var} \left(X \right)$ 

\smallskip

\begin{eqnarray*}
f_{X|Q}\left(x,q \right)&=& \left(x^{n} \right)q^{x}\left(1-a \right)^{n-x}, x=0,1,2,...\\
G &\sim & \textrm{Beta} \left(4,2 \right) \Rightarrow f_{Q}\left(a \right)= \frac{M \left(G \right) }{M \left(4 \right) \left(2 \right)} q^{3} \left(1-a \right)\\
E \left(X \right)&=& E \left(E \left(X|Q\right) \right)= E \left(nQ \right)=n E \left(Q \right)= n \frac{ \alpha }{ \alpha + \beta }  = n \frac{2}{3}\\
\textrm {Var}\left(X \right)&=& \textrm {Var} \left( E \left(X|Q \right) \right)+ E \left( \textrm{Var} \left(X|Q \right) \right)\\
&=&\textrm {Var} \left(nQ \right) + E \left(nQ \left(1-Q \right)  \right)\\
&=& n^{2}  \textrm{Var} \left(Q \right) + n \left[E \left(Q \right)- E \left(Q^2 \right) \right] \\
&=& n^2 \left(\frac{\alpha \beta }{ \left(\alpha + \beta \right)^{2} \left(\alpha + \beta + 1 \right) } \right)
+ n \left[ \frac{2}{3}- \textrm{Var} \left(Q \right) - E^{2}\left(Q \right) \right]\\
&=& n^2 \left( \frac{8}{ \left(36 \right) \left(7 \right) } \right)
+ n \left[\frac{2}{3}-  \frac{8}{ \left(36 \right) \left(7 \right)} - \frac{4}{9} \right]\\
&=& n^2 \frac{2}{63} + n \left[ \frac{2}{3} - \frac{2}{63}- \frac{4}{9} \right]\\
&=&\frac{2}{63} n^2 + n \left[\frac{42-2-28}{63} \right]\\
&=& \frac{2}{63}
\end{eqnarray*}
\end{Ejem}

\begin{Ejem}
$X\sim U \left(1,T \right) $ y $f_{T} \left( t \right) = ct^2$, $1\leq t \leq 3$. Hallar $E \left(X \right)$  Var $\left(X \right) $
 
 \smallskip
 
 \begin{eqnarray*}
f_{X|T} \left(x|T \right) = \begin{cases}
\frac{1}{T-1} & 1\leq x \leq t\\  
0 & \textrm {e.o.c}
\end{cases}  
\end{eqnarray*}

 \begin{eqnarray*}
f_{T}\left(t \right) = ct^{2}\\
1= \int_{1}^{3} ct^2\, dt = c \int_{1}^{3} t^2\, dt = \left( \frac{t^3}{3} \right)_{1}^{3}= c \left(\frac{27}{3} - \frac{1}{3} \right)= c \left(\frac{26}{3} \right)\\
\therefore c= \frac{3}{26}
\end{eqnarray*}

 \begin{eqnarray*}
f_{T} \left(t \right) = \begin{cases}
\frac{3}{26}t^{2} & 1\leq t \leq 3\\  
0 & \textrm {e.o.c}
\end{cases}  
\end{eqnarray*}
\end{Ejem}
%______________________________________________________________
\section{Ejercicios  de funci?n generadora de momentos.}
%_______________________________________________________________
\begin{enumerate}
\item Encontrar todos los momentos de una v.a. $X$ si:

\smallskip

a) $X\sim N \left(0,1 \right)$ 

\smallskip

b) $X \sim U \left(a,b \right)$

\smallskip

c) $x \sim \textrm{Beta} \left(a,b \right)$

\smallskip

d) $x\sim Gamma \left( \alpha, \lambda \right)$

\smallskip

Para a)

\begin{eqnarray*}
X &\sim & N \left(0,1 \right) \Rightarrow f_{X}\left(x \right)= \frac{1}{\sqrt{2\pi}} \textrm{exp} \left\{- \frac{1}{2}x^2 \right\} x\in \rea \\
E \left[ X^{r} \right] &=& \left.\frac{d^{r} } {dt^{r}} M_{X}\left( t \right) {y=0} \right|_{t=0}\\
M_{X} \left(t \right) &=& E \left[e^{tX} \right] = \int_{-\infty }^{\infty} e^{tX}\left(\frac{1}{\sqrt{2\pi}}e^{\frac{x^{2} }{2}} \right) \, dx\\
&=& \frac{1}{\sqrt{2\pi}}\int_{-\infty }^{ \infty } e^{\frac{x^{2} }{2}}\, dx \\
tx- \frac{x^{2}}{2} & \Rightarrow & - \frac{1}{2}\left(x^{2}- 2tx  \right)= - \frac{1}{2} \left(x-t \right)^{2}+ \frac{t^{2}}{2}\\
&=&\frac{1}{\sqrt{2\pi}} \int_{-\infty}^{\infty} e^{- \frac{1}{2}\left(x-t \right)^2+ \frac{t^2}{2} }\,dx\\
&=&\frac{e^{\frac{t^2}{2}}}{\sqrt{2\pi}} \int_{-\infty}^{\infty} e^{- \frac{1}{2}\left(x-t \right)^2 }\,dt = e^{\frac{t^2}{2}}\\
N \left(t,1 \right)\\
\left. M^{i}_{X}\left(t \right)\right|_{t=0} &=& e^{\frac{1}{2}t^2}\left(t \right)=0\\
\left. M^{ii}_{X}\left(t \right)\right|_{t=0} &=&\left. e^{\frac{1}{2}t^2}+ t^2e^{\frac{1}{2}t^2}\right|_{t=0}\\
\left. M^{iii}_{X}\left(t \right)\right|_{t=0} &=& \left. te^{\frac{1}{2}t^2}+ t^3e^{\frac{1}{2}e^2}+ e^{\frac{1}{2}t^2}\left( 2t \right)\right|_{t=0} =0\\
\left. M^{iv}_{X}\left(t \right)\right|_{t=0} &=& \left. t^2e^{\frac{1}{2}t^2} + e^{ \frac{1}{2}t^2} + t^4 e^{\frac{1}{2}t^2 }+3 t^2 e^{\frac{1}{2}t^2 }+ 2t^2e^{\frac{1}{2}t^2}+ 2e^{\frac{1}{2}t^2 } \right|_{t=0}\\
&=& 1+2=3
\end{eqnarray*} 
\begin{eqnarray*}
E \left[X^r \right] &=&  \int x^r f_{X}\left(x \right)\, dx\\
&=& \frac{1}{\sqrt{2\pi}} \int_{-\infty}^{\infty}x^r e^{- \frac{x^2}{2} }\, dx\\
u &=& x^{r-1} \Rightarrow \frac{du}{dx}= \left(r-1 \right)x^{r-2} \Rightarrow du= \left( r-1 \right)x^{r-2} dx\\
dv &=& xe^{-\frac{-x^2}{2}} dx \Rightarrow v= - e^{-\frac{x^2}{2}}\\
&=& \left. x^{r-1} e^{\frac{x^2}{2}}\right|_{t=0} + \int e^{\frac{x^2 }{2}} \left(r-1 \right)x^{r-2}\, dx\\
u &=& x^{r-3} \Rightarrow du= \left(r-3 \right)x^{r-4} dx\\
dv &=& xe^{-\frac{1}{2}  x^2} dx \Rightarrow v= -e^{-\frac{1}{2}x^2 }\\
&=&\left. -x^{r-3} e^{-\frac{1}{2}x^2} \right|_{-\infty } ^{\infty} + \int _{-\infty}^{\infty} e^{- \frac{1}{2}x^2 } \left(r-3 \right)x^{r-4} \, dx \\
&=& \left(r-1 \right) \left(r-3 \right) \frac{1}{\sqrt{2\pi}} \int _{-\infty}^{\infty} e^{- \frac{1}{2}x^2 }x^{r-4} \, dx 
\end{eqnarray*} 
\begin{eqnarray*}
r &=& 1\\ e \left[X \right] &=& 0 \\
r &=& 2 \\
&=&\frac{1}{\sqrt{2\pi}} \left(r-1 \right) \int_{-\infty }^{\infty}x^{r-2}e^{\frac{1}{2}t^2 } \, dx\\
&=& \frac{1}{\sqrt{2\pi}}\left( 1 \right)  \int_{-\infty }^{\infty}x^{2-2}e^{\frac{1}{2}t^2 } \, dx\\
&=& \frac{1}{\sqrt{2\pi}}\int_{-\infty }^{\infty}e^{\frac{1}{2}t^2 } \, dx = 1\\
r &=& 3\\
&=& \frac{1}{\sqrt{2\pi}} \left(r-1 \right) \left(r-3\right)\int_{-\infty }^{\infty } x^{r-4} e^{-\frac{1}{2}x^2 }\, dx \\
&=& \frac{1}{\sqrt{2\pi}} \left(2 \right) \left(3-3\right))\int_{-\infty }^{\infty } x^{r-4} e^{-\frac{1}{2}x^2 }\, dx = 0
\end{eqnarray*} 
Para b)


$X \sim U \left(a,b \right)$

 \begin{eqnarray*}
f_{X} \left(x \right) = \begin{cases}
\frac{1}{b-a} & x \in\left( a,b \right) \\  
0 & \textrm {e.o.c}
\end{cases}  
\end{eqnarray*}
 
\begin{eqnarray*}
E \left[X^r\right]&=& \int_{a},^{b} x^r f_{X} \left(x \right)\, dx = \frac{1}{b-a} \int_{a}^b x^r \, dx \\
&=& \frac{1}{b-a} \left[\frac{x{r+1}}{r+1} \right]= \frac{1}{b-a}\left[\frac{b^{r+1} }{r+1}- \frac{a^{r+1}}{r+1} \right]\\
M_{X}\left(t \right)&=& E \left[e^{tx}  \right]= \int_{a}^{b} e^{tx} \frac{1}{b-a} \, dx\\
&=& \frac{1}{b-a} \int _{a}^{b} e^{tx} \, dx = \frac{1}{b-a} \left(e^{tx} \right)_{a}^b\\ 
&=& \frac{1}{b-a} e^{t \left(b-a \right) } \\
M^{'}_{X}|_{t=0}&=& \frac{1}{b-a} e^{t \left(b-a \right) } \left(b-a \right)\\
&=& e^{t \left(b-a \right) }
\end{eqnarray*}

Para c) 

\smallskip

$X \sim \textrm{Beta} \left(a,b \right)$


 \begin{eqnarray*}
f_{X} \left(x \right) = \begin{cases}
\frac{M \left(a+b\right)}{M \left(a \right)M \left(b \right)} x^{a-1} \left(1-x \right)^{b-1} & 0<x<1\\
0 &  \textrm{e.o.c}
\end{cases}  
\end{eqnarray*}
 
\begin{eqnarray*}
E \left[X^{r} \right]&=& \int_{0}^{1} x^r \frac{M \left(a+b \right) }{ M \left( a\right) M \left(t \right)  } x^{a-1} \left(1-x \right)^{b-1} \, dx \\
&=& \frac{M \left( a+b \right)}{M \left(a \right) M \left(b \right)} \, \frac{M \left( r+a \right) M \left( b\right) }{r \left( a+b+r \right) } \int_{0}^{1} \frac{M \left(a+r+b \right)}{M \left(r+a \right)r \left(b \right) } x^{r+a+1} \left( a-x\right)^{b-1} \, dx\\
&=& \frac{M \left(a+b \right) M \left(r+a \right) }{r \left(a \right)M \left(a+b+r \right) }
\end{eqnarray*} 
 
Para d) 

\smallskip

$X \sim \textrm{Gamma} \left(\alpha , \lambda \right)$

\begin{eqnarray*}
f_{X} \left(x \right) = \begin{cases}
\frac{1}{M \left(\alpha \right) \lambda^{ \alpha }} x^{\alpha -1 }e^-{\frac{x}{\lambda }}  & x>0\\
0 &  \textrm{e.o.c}
\end{cases}  
\end{eqnarray*}


 \begin{eqnarray*}
E \left[X ^r \right] &=& \int_{0}^{ \infty } x^r \frac{1}{M \left(\alpha \right)\lambda^{ \alpha} } x^{\alpha - 1} e^{- \frac{x}{\lambda } }\, dx\\
&=& \frac{1}{M \left( \alpha \right)  \lambda ^\alpha  } x^{r+ \alpha-1 } e^{- \frac{x}{\lambda} } \, dx\\
&=& \frac{M \left( r+ \alpha \right) \lambda^{r+\alpha}} {M \left(\alpha\right) \lambda ^{ \alpha } M \left(r+\alpha \right) \lambda^{r+\alpha} } \int _{0}^{ \infty } x^{r+ \alpha -1 }   e^{\frac{x}{\lambda } } \, dx\\
&=& \frac{M \left(\alpha + \alpha \right) }{ M \left(\alpha \right) \lambda^{ \alpha } }\\
\textrm{Gamma} \left(r+\alpha, \lambda \right)
\end{eqnarray*}

\item Sean $X,Y \sim U_{1,2,..., m}$ ind. Calcular $E \left[|X-Y| \right]$

\begin{eqnarray*}
f_{X} \left(x \right) = \begin{cases}
\frac{1}{m}  & x=1... m\\
0 &  \textrm{e.o.c}
\end{cases}  
\end{eqnarray*}

\begin{eqnarray*}
f_{X} \left(x \right) = \begin{cases}
\frac{1}{m}  & y=1... m\\
0 &  \textrm{e.o.c}
\end{cases}  
\end{eqnarray*}

\begin{eqnarray*}
f_{XY} \left(x,y \right) = \begin{cases}
\frac{1}{m^2}  & x,y =1... m\\
0 &  \textrm{e.o.c}
\end{cases}  
\end{eqnarray*}

\begin{eqnarray*}
|X-Y| = \begin{cases}
x-y  & x>y \\
y-x  &  y>x
\end{cases}  
\end{eqnarray*}

\begin{eqnarray*}
Z &=&|X-Y|\\
f_{Z} \left( z \right) &=& \prob \left[Z=z \right]= \prob \left[ |X-Y|=z \right]= \prob \left[X-Y=z \right] + \prob \left[ Y-X=z \right]\\
&=& \prob \left[X= Y+z \right] + \prob \left[Y=z+X \right]
\end{eqnarray*}

\begin{eqnarray*}
\prob \left[ X= Y+z \right] &=& \sum _{K} \prob \left[X=K+z \right] \prob \left[Y=K \right]\\
&=& \sum_{K=1}^{m-z} \prob \left[X=K+z \right] \prob \left[ Y= K \right]\\
\prob \left[Y=z+X \right] &=& \sum _{K}  \prob \left[ K+z\right] \left[X=K \right]\\
&=& \sum_{K=1}^{m-z}\prob \left[ Y=K + z\right]  \prob \left[X=K \right]
\end{eqnarray*}

\begin{eqnarray*}
&=& \prob \left[X= Y+z \right] + \prob \left[Y=X+z \right]\\
&=& \sum _{K=1}^{m-z} \left[\prob \left[X= K+z \right] \prob \left[Y=K \right]+ \prob \left[X=K \right] \right]\\
&=& \sum _{K=1}^{m-z}\left( \frac{1}{m^2} + \frac{1}{m^2} \right) = 2 \sum_{K=1}^{m-z} \frac{1}{m^2} \\
&=& \frac{2}{m^2} \sum _{K=1}^{m-z} = \frac{2}{m^2} \left( m-z-1+1 \right) = \frac{2}{m^2} \left( m-z \right) = \frac{2 \left( m-z \right) }{m^2}
\end{eqnarray*}

\begin{eqnarray*}
f_{Z} \left( z \right) = \begin{cases}
\frac{2 \left( m-z \right) }{m^2}  & z=1,...,m-1 \\
0 &  \textrm{e.o.c} 
\end{cases}  
\end{eqnarray*}

\begin{eqnarray*}
&=& E \left[ |X-Y| \right]= E \left[Z \right]= \sum_{z} z f_{z} \left(z \right)) \sum_{z}z \frac{2 \left(m-z \right) }{m^2}\\
&=& \frac{2}{m^2} \sum _{z=1}^{m-1} zm-z^2 = \frac{2}{m^2} \left[ \left( m \left( \frac{ m-1 \left( m \right)}{2}\right) \right) - \sum _{z=1}^{m-1} z^2 \right]\\
&=& \frac{2}{m^2} \left[ \frac{m^2 \left(m-1 \right) }{2} - \left(1+4+9+...+ \left( m-1\right) ^2 \right) \right]\\
&=&\frac{2}{m^2} \left[ \frac{m^2 \left(m-1 \right) }{2} - \frac{ \left(m-1 \right) \left( m \right) \left(2 \left(m-1 \right)+1 \right)   }{6 } \right]\\
&=& \frac{2}{m^2} \left[ \frac{m^2 \left(m-1 \right) }{2} - \frac{m \left( m-1 \right)\left( 2m - 1 \right) }{6 } \right]\\
&=& m-1 \left[1 - \frac{m \left(2m-1 \right) }{3} \right]
\end{eqnarray*}

\item 
$ X_{1} \sim \textrm{Exp} \left( 2 \right)$ y $X_{2} \sim U \left(X_{1}+1, X_{1}+2 \right) $. Hallar $E \left(X_{2} \right) $ y $ \textrm{Var} \left(X_{2} \right) $

\begin{eqnarray*}
X_{1} \sim \textrm{Exp} \left(2 \right)\Rightarrow f_{X_1} \left(x_{1} \right) &=& \begin{cases}
2e^{-2x_{1}}  & x_{1 > 0} \\
0 &  \textrm{e.o.c} 
\end{cases} \\\\
X_{2|X_{1}}&\sim & U \left(X_{1}+1, X_{1}+2 \right) \\
\Rightarrow f_{X_{2}|X_{1}}\left( x_{2}| x_{1}\right) &=& \begin{cases}
2e^{-2x_{1}}  & x_{1 > 0} \\
0 &  \textrm{e.o.c} 
\end{cases}\\\\
f_{X_{1}X_{2}}\left(x_{1},x_{2} \right) &=& f_{X_{2}| X_{1}} \left(x_{2} | x_{1} \right) f_{X_{1}}\left(x_{1} \right)\\
&=& \begin{cases}
2e^{-2x_{1}}  & x_{1}+1 < x_{2} < x_{1}+2, x_1>0 \\
0 &  \textrm{e.o.c} 
\end{cases}
\end{eqnarray*}

\begin{eqnarray*}
x_{2}=x_{1}+ 1\\
\left(0,1 \right)\left(-1,0 \right)\\
x_{2}=x_{1}+ 2\\
\left(0,2 \right)\left(-2,0 \right)
\end{eqnarray*}

\begin{eqnarray*}									
f_{X_{2}}\left(x_{2} \right) &=& \begin{cases}
\int _{0}^{x_{2}-1} 2e^{-2x_{1}} \, dx_{1} & 1\leq x < 2 \\
\int _{x_{2}-2}^{x_{2}-1} 2e^{-2x_{1}} \, dx_{1} &  x \geq 2 \\
0 &  \textrm{e.o.c} 
\end{cases}\\\\
f_{X_{2}} \left( x_{2} \right) &=& \begin{cases}
1-e^{-2 \left(x_{2}-1 \right) } & 1\leq x < 2 \\
e^{-2 \left(x_{2}-2 \right)}- e^{-2 \left(x_{2}-1 \right) } &  x > 2 \\
0 &  \textrm{e.o.c} 
\end{cases}\\\\
 E \left[ X_{2} \right] &=& \int _{1}^{2}x_{2} \left(1- e^{-2 \left(x_{2}-1 \right)}\right) \, dx_{2} + \int _{0}^{ \infty } x_{2} e^{-2 \left(x_{2}-2 \right)}- e ^{-2 \left( x_{2}-1 \right) } \, dx_{2} = 2\\ 
 E \left[ X_{2}^{2} \right] &=& \int _{1}^{2}x_{2}^{2} \left(1- e^{-2 \left(x_{2}-1 \right)}\right) \, dx_{2} + \int _{0}^{ \infty } x_{2}^{2} e^{-2 \left(x_{2}-2 \right)}- e ^{-2 \left( x_{2}-1 \right) } \, dx_{2} = \frac{13}{3}\\
 \textrm{Var} \left(X_{2} \right)&=& E \left[X_{2}^{2} \right]- E^2 \left[X_{2} \right]= \frac{13}{3}-4 = \frac{1}{3}
\end{eqnarray*}

\item $X_{1}, X_{2},X_{3},X_{4} \sim U \left(0,1 \right) $ Ind. Hallar $E \left( Y_{4}- Y_{1} \right)$ donde $Y_{j}$ es la estad?stica de orden $j$.

\begin{eqnarray*}
X_{1} \sim f_{X_{1}} \left(x_{1} \right) &=& \begin{cases}
1  & 0 < x_{1} < 1 \\
0 &  \textrm{e.o.c} 
\end{cases}  \\\\
\Rightarrow  f_{X_{1}X_{2}X_{3}X_{4}} \left(x_{1},X_{2,}X_{3},X_{4} \right) &=& \begin{cases}
1  & 0 < x_{1} < 1 \, 0 < x_{2} < 1 \, 0 < x_{3} < 1 \, 0 < x_{4} < 1 \\
0 &  \textrm{e.o.c} 
\end{cases} \\\\
\Rightarrow  f_{Y_{1}Y_{2}Y_{3}Y_{4}} \left(y_{1},y_{2,}y_{3},y_{4} \right) &=& \begin{cases}
4!  & 0 < y_{1} < 1 \, 0 < y_{1}<y_{2}<y_{3}<y_{4} < 1 \\
0 &  \textrm{e.o.c} 
\end{cases}\\\\
f_{Y_{4}Y_{1}} \left(y_{1}, y_{4} \right) &=& \int _{y_{1}}^{y_{4}} \int _{y_{1}}^{y_{3}} 4! \, dy_{2}\, dy_{3} = \left. 4! \int _{y_{1}}^{y_{4}} y_{2} \right| _{y_{1}}^{y_{3}}\, dy_{3}\\
&=& 4! \int _{y_{1}}^{y_{4}} y_{3}- y_{1} \, dy_{3}= 4! \left(\frac{y_{3}^2}{2}-y_{1} y_{3} \right)_{y_{1}}^{y_{4}}\\
&=& 4! \left(\frac{y_{4}^2}{2} - \frac{y_{1}^2}{2}- y_{1}y_{4} +  y_{1}^2 \right)\\
&=& 4! \left(\frac{y_{4}^2}{2} + \frac{y_{1}^2}{2}- y_{1}y_{4} \right)\\
&=& \frac{4!}{2} \left(y_{4}^2+ y_{1}^2 -2y_{1}y_{4}\right) = 12 \left(y_{4}-y_{1} \right)^2 \\
0<y_{1}< y_{4}<1\\
E \left[Y_{4}- Y_{1} \right] &=& \int_{0}^{1} \int_{0}^{y_{4}} \left(Y_{4}-Y_{1} \right)f_{Y_{1}Y_{4}} \left(y_{1}, y_{4} \right) \, dy_{1} \, dy_{4}\\
&=&\int_{0}^{1} \int_{0}^{y_{4}} \left(Y_{4}-Y_{1} \right)12 \left(y_{4}- y_{1} \right)^2 \, dy_{1} \, dy_{4}\\
&=& 12 \int_{0}^{1} \int_{0}^{y_{4}}  \left(y_{4}- y_{1} \right)^3 \, dy_{1} \, dy_{4}\\
&=& 12 \int_{0}^{1}\left. - \frac{ \left(y_{4}-y_{1} \right)^4 }{ 4} \right| _{0}^{y_{4}} \, dy_{4}\\
&=& 12 \int _{0}^{1}  - \frac{ \left(y_{4}-y_{4} \right)^4 }{ 4} +  \frac{ \left(y_{4}-0 \right)^4 }{ 4} \, dy_{4}\\
&=& \frac{12}{4} \left(\frac{y_{4}^5}{5} \right)_{0} ^{1} = 3 \left(\frac{1}{5} \right) = \frac{3}{5}
\end{eqnarray*}

\item Sea \begin{eqnarray*} f_{X,Y} \left( x,y \right) &=& \begin{cases}
C  & x\geq 0 \, y \geq 0 \, x+y\geq 1\\
0 &  \textrm{e.o.c} 
\end{cases}\end{eqnarray*}
 Hallar $ \textrm{Cov} \left(X,Y \right) $.
 
\begin{eqnarray*} 
y &=& 1-x \, \left( 0,1 \right) \, \left(1,0 \right)\\
1 &=& \int_{0}^{1} \int _{0}^{1-x} c \, dy \, dx= \left. c \int _{0}^{1} y \right|_{0}^{1-x} \, dx = c \int _{0}^{1} \left( 1-x  \right) \, dx \\
&=& \left. -c \frac{\left(1-x \right)^2 }{2} \right| _{0}^{1} = -c \frac{\left(1-1 \right) }{2} + c \frac{\left( 1-0\right)^2 }{2} = \frac{c}{2}\\
\therefore c &=& 2\\
 f_{X,Y} \left( x,y \right) &=& \begin{cases}
2  & x\geq 0 \, y \geq 0 \, x+y\geq 1\\
0 &  \textrm{e.o.c} 
\end{cases}\\
 \textrm{Cov} \left(X,Y \right) &=& E \left(XY \right)- E \left(X \right)E \left(Y \right)\\
 f_{X} \left(x \right) &=& \int _{0}^{1-x} 2 \, dy = \left. 2y  \right| _{0}^{1-x} = 2 \left(1-x \right) \\
  0 &<& x<1\\
 f_{Y} \left(y \right) &=& \int _{0}^{1-y} 2 \, dx =  2 \left(1-y \right) \\
  0 &<& y<1\\
   E \left(X \right) &=& \int _{0} ^{1} 2x \left(1-x \right) \, dx = x^2- \left.\frac{2x ^3}{3} \right| _{0} ^{1} \\
  &=& 1- \frac{2}{3} = \frac{1}{3}\\
  E \left[ Y \right]& =& \int_{0} ^{1} 2y \left( 1-y\right) \, dy = 2 \left(\frac{y^2}{2}- \frac{y^3}{3} \right)_{0} ^{1}\\
  &=& 1- \frac{2}{3} = \frac{1}{3}\\
   E \left[ XY \right] &=& \int _{0} ^{1} \int _{0}^{1-x} 2xy \, dy \, dx = 2 \int _{0}^{1} \left. x \frac{y^2}{2}\right|_{0}^{1-x} \, dx
   \end{eqnarray*}
\begin{eqnarray*}
   &=& \int _{0} ^1 x \left(1-x \right)^{2} \, dx = \int _{0}^{1} x- 2x^2+x^3 \, dx = \left. \frac{x^2}{2}- \frac{2}{3} x^3 + \frac{x^4}{4} \right| _{0}^{1} \\
   &=& \frac{1}{2} - \frac{2}{3}+ \frac{1}{4} = \frac{6-8+3}{12}= \frac{1}{12}\\
   \textrm{Cov} \left(X,Y \right)&=& \frac{1}{12} - \left(\frac{1}{3} \right)\left(\frac{1}{3} \right) = \frac{1}{12} - \frac{1}{9} = \frac{3-4}{36}= - \frac{1}{36}\\
   p \left(X,Y \right)&=& \frac{\textrm{Cov} \left(X,Y \right) }{\sqrt{\textrm{Var}\left(X \right)}\sqrt{\textrm{Var}\left(X \right) }} = \frac{-\frac{1}{36} }{\sqrt{\left( \frac{1}{18} \right)\left( \frac{1}{18} \right) }}= -\frac{\frac{1}{36} }{\frac{1}{18} }= -\frac{18}{36} = - \frac{1}{2}\\
   \textrm{Var} \left(X \right) &=& E \left(X^2 \right)- E^2 \left(X \right)= \frac{1}{6} - \left( \frac{1}{3} \right)^2 = \frac{1}{6} - \frac{1}{9} = \frac{9-6}{54}= \frac{3}{54}= \frac{1}{18}\\
 E \left[X ^2 \right]&=& \int _{0}^{1} 2x ^2 \left(1-x \right) \, dx = 2 \left. \frac{x^3}{3} - \frac{2x ^ 4}{4} \right| _{0}^{1}\\
 &=& \frac{2}{3}- \frac{2}{4}= \frac{8-6}{12}= \frac{2}{12} = \frac{1}{6}
\end{eqnarray*}

 \item Si $X \sim N \left(0,1 \right) $. 
 
 \begin{eqnarray*}
\textrm{Cov}\left(X^2-1 , X + \frac{1}{2} \right) &=& E \left[\left(X^2-1 \right) \left( X+ \frac{1}{2}\right) \right]- E \left[X^2-1 \right] E \left[X+ \frac{1}{2} \right]\\
&=& E   \left[X^3 + \frac{1}{2} X^2-X - \frac{1}{2} \right]- \left[ E \left[X^2 \right]-1 \right] \left[ E \left[ X \right]+ \frac{1}{2}\right]\\
&=& E \left[X^3 \right]+ \frac{1}{2} E \left[X^2 \right]- E \left[ X \right]- \frac{1}{2}- E \left[X^2 \right] E \left[X \right]\\
&-& \frac{1}{2} E \left[X^2 \right]+ E \left[X \right]+ \frac{1}{2}\\
&=& E \left[X^3 \right]- E \left[ X^2\right]E \left[X \right]=0
\end{eqnarray*}

\begin{eqnarray*}
\textrm{Var} \left(X^2 - 1  \right)&=& E \left[ \left(X^2 - 1 \right)^2 \right]- E ^2 \left[X^2 - 1 \right]\\
&=& E \left[ X^4 - 2X^2 + 1 \right] - \left(E \left[X^2 \right]-1 \right)^2\\
&=& E \left[X^4 \right]- 2E \left[X^2 \right]+1 - \left( E^2 \left[ X^2\right]- 2E \left[ X^2 \right]+1 \right)\\
&=& E \left[ X^4 \right]- E^2 \left[X^2 \right]= 3-1=2
\end{eqnarray*}

                                                                                                                                                 \begin{eqnarray*}
\textrm{Var} \left( X + \frac{1}{2}  \right)&=&  E \left[\left(X + \frac{1}{2} \right)^2 \right]- \left( E \left[X + \frac{1}{2} \right]  \right)^2\\
&=& E \left[X^2 + X + \frac{1}{4} \right] - \left( E \left[X \right] + \frac{1}{2} \right)^2\\
&=& E \left[X ^ 2 \right] +  E \left[X \right]+ \frac{1}{4} - \left( E^2 \left[X \right]+ E \left[ X\right]+ \frac{1}{4} \right)\\
&=& E \left[X^2 \right] - E^2\left[X \right] = 1
\end{eqnarray*}

\begin{eqnarray*}
X \sim N \left( 0,1 \right) &\Rightarrow & f_{X} \left( x \right) = \frac{1}{ \sqrt{2 \pi}} \textrm{exp} \left\{- \frac{1}{2} x^2 \right\}\\
M_{X} \left(t \right) &=& E\left[e^{xt} \right] = \int _{-\infty}^{\infty} e^{xt} \frac{1}{\sqrt{2 \pi}} e ^{- \frac{x^2}{2}} \, dx = \frac{1}{\sqrt{2 \pi}} \int _{-\infty}^{\infty}  e^{xt- \frac{x^2}{2}} \, dx\\\\
 \textrm{Nota }-\frac{x^2}{2} + xt &=& \frac{-1}{2} \left(x^2 - 2xt \right) = - \frac{1}{2} \left(x-t \right)^2 + \frac{t^2}{2}\\\\ 
& = &\frac{1}{\sqrt{2\pi}}\int _{-\infty}^{\infty} e^{- \frac{1}{2} \left(x-t \right)^2 + \frac{t^2}{2}}  \, dx = e^\frac{t^2}{2} \frac{1}{\sqrt{2\pi}}\int _{-\infty}^{\infty} e^{- \frac{1}{2} \left(x-t \right)^2} \, dx\\
&=& e^{\frac{t^2}{2}}
\end{eqnarray*}

\item
Hallar el n\'umero de lanzamientos necesarios para obtener al menos 1 vez c/cara de un lado.

\begin{eqnarray*}
i &=& \left\{1,2,3,..., 6 \right\}\\
P_{i} &\equiv & \textrm{probabilidad de que caiga la cara i} \\
X_{i} &\sim & \textrm{Geo} \left(p_{i} \right)\\
E \left[X_{i} \right] &=& \frac{1}{i_{i}}\\
f_{X} \left(x \right) &=& p \left(1-p \right)^{x-1}\, , x=1...\\
X&=&X_{1}+X_{2}+X_{3}+X_{4}+X_{5}+X_{6}\\
 E \left[ X \right] ?
 \end{eqnarray*}
 
 $X_{1} \equiv$  No. de lanzamientos para obtener la cara 1  (La cara que salga al tirar por primera vez un dado).
 $p_{1} = 1 \Rightarrow X_{1} \sim $Geo $\left( 1 \right) \Rightarrow E \left[X_{1} \right] = 1$
 
 \smallskip
 
 $X_{2} \equiv$  No. de lanzamientos necesarios para obtener la cara 2. (La cara 2 es la primera cara diferente a la cara 1, como la cara 1 ya sali\'o por primera vez, entonces s\'olo interesa que salga cualquiera de las 5 restantes).
 
 \smallskip
 
 $p_{2} = \frac{5}{6} \Rightarrow X_{2} \sim $ Geo $\left(\frac{5}{6} \right) \Rightarrow E \left[X_{2} \right] = \frac{1}{\frac{5}{6}}= \frac{6}{5}$
 
 \smallskip
 
 Razonamiento an\'alogico


\begin{eqnarray*}
p_{3} =  \frac{4}{6}= \frac{2}{3} & X_{3} \sim \textrm{Geo} \left(\frac{2}{3}\right) & E \left[ X_{3}\right] = \frac{3}{2} \\
p_{4}=  \frac{3}{6}= \frac{1}{2} & X_{4} \sim \textrm{Geo} \left(\frac{1}{2}\right) & E \left[ X_{4}\right] =2 \\
p_{5}=  \frac{2}{6}= \frac{1}{3} & X_{5} \sim \textrm{Geo} \left(\frac{1}{3}\right) & E \left[ X_{5}\right] = 3 \\
p_{6}=  \frac{1}{6} & X_{6} \sim \textrm{Geo} \left(\frac{1}{6}\right) & E \left[ X_{6}\right] = 6 
\end{eqnarray*}


\begin{eqnarray*}
\therefore E \left[X \right] &=& 1 + \frac{6}{5}+ \frac{3}{2}+2+3+6 = 12 + \frac{12+15}{10}\\
&=& \frac{120+12+15}{10}= \frac{147}{10} = 14.7
\end{eqnarray*}

\item

Si $X \sim$ Geo $\left(\frac{5}{8} \right)$ $f_{X} \left(x \right)= \left(\frac{5}{8} \right) \left(\frac{3}{8} \right)^x$, $x=0,1,....$ 

\smallskip

Calcular $E \left[2X | Z \leq 3, x \right]$ par.

\begin{eqnarray*}
E  \left[2X | X \geq \textrm{par} \right] &=& \sum_{x=0}^{\infty} 2x f_{X|D} \left(x|d \right)\\
\prob \left[X=x | X \geq 3 , X \textrm{par} \right]&=& \frac{\prob\left[X=x , X \geq 3, X \textrm{par} \right]}{\prob \left[X \geq 3 , X \textrm{par} \right]}\\
\prob \left[ X=x , X \geq 3 , X \textrm{par} \right] &=& \prob \left[X=x \textrm{par mayor \'o igual a 4} \right]\\
&=& \left(\frac{5}{8} \right)\left(\frac{3}{8} \right)^{x} , x=4,6,8,...
\end{eqnarray*}


\begin{eqnarray*}
\prob \left[X 0 3 , X \geq 3 , X par \right] &=& 0 \\
\prob \left[X=4, X \geq 3 , X par \right] &=& \prob \left[ X=4 \right]\\
\prob \left[X \geq 3 , X par  \right] & = & \prob \left[ X=4 \right] + \prob \left[ X=6\right]+ \ldots \\
&=& \left(\frac{5}{8} \right) \left( \frac{3}{8}\right)^4 + \left( \frac{5}{8} \right) \left(\frac{3}{8} \right)^6 + \ldots +\\
&=&\left(\frac{5}{8} \right) \left( \frac{3}{8}\right)^4  \left[1 + \left(\frac{3}{8} \right)^2  + \left(\frac{3}{8} \right)^4 + \ldots \right]\\
&=&\frac{5}{8} \left(\frac{3}{8} \right)^4 \left[\frac{1}{1- \left(\frac{3}{8} \right)^2} \right]\\
&=&\frac{5}{8} \left(\frac{3}{8} \right)^4 \left(\frac{1}{\frac{64-9}{64}} \right)= \left(\frac{5}{8} \right) \left(\frac{3}{8} \right)^4 \left(\frac{64}{55} \right)\\
&=& \frac{81}{5632}\\
\prob \left[X=x \mid X \geq 3 , X par \right] &=& \begin{cases}
\frac{\left( \frac{5}{8} \right) \left( \frac{3}{8} \right)^{\alpha}}{\left( \frac{5}{8} \right) \left( \frac{3}{8} \right)^{4} \left(\frac{64}{55} \right)}= \frac{3520}{81}\left(\frac{3}{8} \right)^x & x=4,6,8, \ldots\\
0 &  \textrm{e.o.c} 
\end{cases}
\end{eqnarray*}

\begin{eqnarray*}
 \sum_{x=4}^{\infty} \left( \frac{3520}{81} \right)\left(\frac{3}{8} \right)^{x}&=& \frac{3520}{81} \left[\left( \frac{3}{8} \right)^4 + \left(\frac{3}{8} \right)^6+ \ldots \right]\\
 &=& \frac{3520}{81} \left(\frac{3}{8} \right)^4 \left[1 + \left(\frac{3}{8} \right)^2 + \left( \frac{3}{8} \right)^4 + \ldots \right]\\
 &=& \frac{55}{64} \left[ \frac{1}{1 - \frac{9}{64}} \right] = 1
\end{eqnarray*}

\begin{eqnarray*}
 \therefore \prob \left[2X \mid Z \geq 3, X par \right] &=& \sum _{x=4}^{\infty} 2x \frac{3520}{81} \left( \frac{3}{8} \right)^{x}\\
 &=& 2 \left( \frac{3520}{81} \right) \sum _{x=4} ^{\infty} x \left( \frac{3}{8} \right)^{x}\\
 &=& \frac{7040}{81} \left[4 \left( \frac{3}{8} \right)^4 + 6 \left(\frac{3}{8} \right)^6 + \ldots \right]\\
 &=& \frac{14080}{81}\left[ \frac{\frac{9}{64}}{\left(1 - \frac{9}{64} \right)^2} - \frac{9}{64} \right] = \frac{14080}{81} \left[ \frac{9}{64} \left(\frac{4086}{3025} \right)- \frac{9}{64} \right]\\
&=& \left( \frac{14080}{81} \right) \left(\frac{9}{64} \right) \left[\frac{4096}{3025} - 1\right] = \frac{14080}{81} \left(\frac{9}{64} \right)\left( \frac{1071}{3025} \right)= \frac{476}{55}
\end{eqnarray*}




\item
Sean $X,Y$ Vra. con densidad $f_{XY}$ constante en el tri\'angulo $\left(0,0 \right), \left(2,0 \right), \left(1,2 \right)$. Hallar $E \left[ Y \mid X \right]$

\begin{eqnarray*}
f_{Y \mid X} \left( y \mid x \right) &=& \begin{cases}
\frac{1}{2x}  & 0 \leq y \leq 2x \mid 0\leq x < 1 \\
\frac{1}{-2x+4} & 0 \leq y \leq -2x+4  \mid  1\leq x \leq 2\\
0 &  \textrm{e.o.c} 
\end{cases}\\
\end{eqnarray*}

\begin{eqnarray*}
E \left[Y \mid X \right] &=& \begin{cases}
\int_{0}^{2x} y \frac{1}{2x} \,dy= x  & 0\leq x \leq 1\\
\int y \frac{1}{-2x+4}=-x+2 & 1 \leq x \leq 2\\
0 &  \textrm{e.o.c} 
\end{cases}\\
\end{eqnarray*}

\begin{eqnarray*}
 &=& \begin{cases}
x & 0 <x < 1\\
-x+2 & 1 \leq x < 2\\
0 &  \textrm{e.o.c} 
\end{cases}\\
\end{eqnarray*}
\end{enumerate}
\section{Ejercicios para Evaluaci\'on}
\begin{enumerate}

\item Sea $f_{X,Y} \left(x,y \right) = c \mid x \mid$. Si $0<y<1 - \mid x  \mid \, -1<x<1$

\begin{eqnarray*}
1 &=& \int_{-1}^{1} \int_{0}^{1-\mid x \mid}  c \mid x \mid \, dy \, dx = c \int_{-1}^{1} \int_{0}^{1-\mid x \mid}  \mid x \mid \, dy \, dx\\
&=& c \left[ \int_{-1}^{0} \int_{0}^{1+ x }    -x  \, dy \, dx + \int_{0}^{1} \int_{0}^{1- x }    x  \, dy \, dx \right ]\\
&=& c \left[ \int _{-1}^{0} -xy \mid_{0}^{1+x}  + \int _{0}^{1} xy \mid_{0}^{1-x} \, dx \right]\\
&=& c \left[ \int _{-1}^{0} -x \left( 1+x \right) \, dx  + \int_{0}^{1} x \left(1-x \right) \, dx \right]\\
&=& c \left[ \int _{-1}^{0} -x-x^2  \, dx  + \int_{0}^{1} x-x^2 \, dx \right]\\
&=& c \left(- \frac{x^2}{2} - \frac{x^3}{3} \mid _{-1}^{0} + \frac{x^2}{2} - \frac{x^3}{3}\mid _{0}^{1}  \right)\\
&=& c   \left( +\frac{1}{2} - \frac{1}{3} + \frac{1}{2} - \frac{1}{3}  \right) = c \left( 1-\frac{2}{3} \right) = \left( \frac{3-2}{3} \right)= c \left( \frac{1}{3} \right)\\
\therefore c&=&3\\
f_{XY}\left(x,y \right) &=&  \begin{cases}
3 \mid x \mid & 0<y<1-\mid x \mid \, , -1<x<1 \\
0 &  \textrm{e.o.c} 
\end{cases}\\
f_{X|Y} \left(  x | y\right) &=& \frac{f_{XY}\left(x,y \right)}{f_{Y} \left(y \right) }\\
f_{Y}\left(y \right) &=& \int 3 \mid x \mid \, dx = -3 \int _{y-1}^{0} x \, dx + 3 \int _{0}^{1-y} x \, dx\\
&=& -3 \left(\frac{x^2}{2} \right)_{y-1}^{0} + 3  \left( \frac{x^2}{2} \right)_{0}^{1-y} = 3 \frac{\left(y-1 \right)^2}{2}+ 3 \frac{\left(1-y\right)^2}{2}\\
&=&\frac{3}{2} \left( \left(y-1 \right)^2 + \left(1-y \right)^2 \right)\\
f_{Y} \left(\frac{1}{2} \right) &=& \frac{3}{2} \left( \frac{1}{4} \right) + \frac{3}{2} \left( \frac{1}{4} \right) = \frac{3}{4}\\
f_{X|Y} \left( x \mid 1/2 \right)&=& \begin{cases}
\frac{f_{XY}\left(x, \frac{1}{2}\right)}{f_{Y} \left(\frac{1}{2  } \right)}=  \frac{3\mid x \mid}{\frac{3}{4}}\\
0 &  \textrm{e.o.c} 
\end{cases}
\end{eqnarray*}

\item Hallar $f_{XY} \left( x,y \right) $ con los datos del problema anterior.
\item Calcular $\prob \left[ Y > \mid X \mid \right]$ con los datos del problema 1.

\begin{eqnarray*}
y= \mid x \mid &=& \begin{cases}
x & x \geq 0\\
-x & x<0
\end{cases}\\
\prob \left[Y > \mid X \mid \right] &=& \int_{- \frac{1}{2}}^{0} \int _{-x}^{1+x} 3 \mid x \mid \, dy \, dx\\
&=& \int_{0}^{\frac{1}{2}} \int_{x}^{1-x} 3 \mid x \mid  \, dy \, dx\\
&=& \int _{\frac{-1}{2}}^{0} \int _{-x}^{1+x} -3x  \, dx \, dy + \int _{0}^{\frac{1}{2}} \int _{x}^{1-x} 3x \, dx \, dy\\
&=& -3 \int _{\frac{-1}{2}}^{0} \int _{-x}^{1+x} x  \, dx \, dy + 3 \int _{0}^{\frac{1}{2}} \int _{x}^{1-x} x \, dx \, dy\\
&=& -3 \int _{\frac{-1}{2}}^{0}  xy \mid_{x}^{1+x}  \, dx  + 3 \int _{0}^{\frac{1}{2}} xy \mid_{x}^{1-x}  \, dx \\
&=& -3 \int _{\frac{-1}{2}}^{0}  x \left(1+x+x \right) \, dx  + 3 \int _{0}^{\frac{1}{2}} x \left(1-x-x \right) \, dx \\
&=& -3 \int _{\frac{-1}{2}}^{0}  x+2x^{2} \, dx  + 3 \int _{0}^{\frac{1}{2}} x-2x^2 \, dx \\
&=& -3 \left(\frac{x^2}{2}+\frac{2}{3}x^{3} \right)_{\frac{-1}{2}}^{0}+  3 \left(\frac{x^2}{2}-\frac{2}{3}x^{3} \right)_{0}^{\frac{1}{2}}\\
&=& -3 \left(\frac{\frac{1}{4}}{2}+\frac{2\left(\frac{1}{2} \right)^3}{3} \right)+  3 \left(\frac{\left(\frac{1}{2} \right)^2}{2}-\frac{2\left(\frac{1}{2} \right)^3}{3} \right)\\
&=& -3 \left( - \frac{1}{8}+ \dfrac{\frac{2}{8}}{\frac{3}{1}} \right)+ +3 \left( - \dfrac{\frac{1}{4}}{2}- \dfrac{\frac{2}{8}}{\frac{3}{1}} \right)\\
&=& -3 \left( - \frac{1}{8} + \frac{1}{12} \right) + 3 \left( \frac{1}{2}- \frac{1}{2}- \frac{1}{12} \right)\\
&=& \frac{3}{24}+ \frac{2}{12}\\
&=& \frac{3}{8}
\end{eqnarray*}

\item $X\sim U_{\left\{ 1 \ldots N \right\}}, N \sim U_{\left\{ 1 \ldots m \right\}} $. Calcular $\prob \left[ \mid X-N \mid \right]$

\begin{eqnarray*}
f_{X/N} &=& \begin{cases}
\frac{1}{N} & x= 1 \ldots N\\
0 & \textrm{e.o.c}
\end{cases}\\
f_{N\left(n \right)}&=&\begin{cases}
\frac{1}{m} & N=1 \ldots m\\
0 & \textrm{e.o.c}
\end{cases}\\
f_{X/N}  \left(x/n \right)f_{N}\left(n \right)= f_{XN} \left(x,n \right)&=&  \begin{cases}
\left(\frac{1}{N} \right) \left(\frac{1}{m} \right) & x=1 \ldots N, N= 1 \ldots m \\
0 & \textrm{e.o.c}
\end{cases}\\
\prob \left[X-N \leq 1 \right] &=& \prob \left[-1 \leq X - N \leq 1 \right] = \prob \left[X-N \leq 1 \right] - \prob \left[ X-N \leq -1 \right]\\
\prob \left[ X-N \leq 1 \right] &=& \prob \left[ X \leq 1+N \right] \\
&=& \sum_{k=0} \prob \left[ X \leq 1+k \mid N=k \right]\prob \left[X \leq  1+k \mid N=k \right] \prob \left[N=k \right]\\
&=& \sum_{k=1}^{m-1} \prob \left[X \leq 1+k, N=k \right]\\
&=& \sum_{k=1}^{m-1} \sum_{j=1}^{1+k} \prob \left[X=j, N=k \right]\\
&=&  \sum_{k=1}^{m-1} \sum_{j=1}^{1+k} \frac{1}{N} \frac{1}{m}= \sum_{k=1}^{m-1}\frac{1}{N} \frac{1}{m} \sum_{j=1}^{1+k} 1 = \sum_{k=1}^{m-1} \frac{1}{N} \frac{1}{m} \left(k+1 \right)\\
&=& \frac{1}{N} \frac{1}{m} \sum_{k=1}^{m-1} \left(k+1 \right)= \frac{1}{Nm} \left[\sum_{k=1}^{m-1} k + \sum_{k=1}^{m-1} 1  \right]\\
&=& \frac{1}{Nm}\left[ \frac{\left(m-1 \right)m }{2} + \left(m-1 \right)\right] = \frac{1}{Nm}\left(m-1 \right)\left[\frac{m}{2} +1\right]\\
&=&\frac{1}{Nm} \left(m-1 \right)\left[\frac{m+2}{2} \right] = \frac{\left(m-1 \right)\left(m+2\right)}{2Nm}
\end{eqnarray*}

\item $X \sim U\left(0,1 \right)$. Hallar la densidad de $Y= - \ln \left( \frac{X}{X-1}\right)$

\begin{eqnarray*}
X \sim U\left(0,1 \right) \Rightarrow f_{X}\left(x \right) =  \begin{cases}
1 & 0 < x < 1\\
0 & \textrm{e.o.c}
\end{cases}\\
X< X+1 &\Rightarrow & \frac{X}{X+1} < 1\\
0 &<& \frac{X}{X+1} < 1\\
\ln \left( 0 \right) &>& \ln \left( \frac{X}{X+1} \right) > \ln \left(1 \right)\\
- \infty &<& \ln \left(\frac{X}{X+1}\right)< 0\\
0 &<& -\ln \left(\frac{X}{X+1} \right) < \infty\\
0 &<& Y < \infty
\end{eqnarray*}

\begin{eqnarray*}
f_{Y} \left(y \right)&=& \frac{\partial}{\partial y} F_{Y} \left(y \right)\\
F_{Y}\left(y \right) &=& \prob \left[ Y \leq y \right]= \prob \left[-\ln \left( \frac{X}{X-1} \right) \leq y \right] = \prob \left[ \ln \left( \frac{X}{X-1} \right) \geq -y \right]\\
&=& \prob \left[ \frac{X}{X-1} \geq e^{-y} \right] = \prob \left[ X \geq e^{-y} \left( X+1 \right)\right] = \prob \left[X \geq X e^{-y} + e^{-y}\right]\\
&=& \prob \left[ X - Xe^{-y} \geq e^{-y} \right] = \prob \left[X \left(1-e^{-y} \right) \geq e^{-y} \right]\\
&=& \prob \left[ -X \left( e^{-y}-1 \right) \geq e^{-y} \right] = \prob \left[ \frac{-X \geq e^{-y}}{e^{-y}-1} \right]\\
&=& \prob  \left[ X \leq - \frac{e^{-y}}{e^{-y}-1} \right] = \prob \left[X \leq \frac{e^{-y}}{1-e^{-y}} \right] = F_{X} \left( \frac{e^{-y}}{1-e^{-y}} \right)\\
f_{Y}\left(y \right) &=& f_{X} \left( \frac{e^{-y}}{1-e^{-y}} \right)\\
&=& \frac{\left(1-e^{-y} \right)\left(e^{-y} \right) \left( -1 \right) - e^{-y} \left(-e^{-y} \right) \left( -1\right) }{\left( 1 - e ^{-y} \right)^2}\\
&=& \frac{-e^{-y}\left( 1- e ^{-y} \right)- e ^{-2y}}{\left(1-e^{-y} \right)^2} = \frac{e^{-y}+ e ^{-2y}- e^{-2y}}{\left(1-e^{-y} \right)^2}\\
&=& \frac{e^{-y}}{\left( 1-e^{-y} \right)^2}\\
0&<&y<\infty\\
-\infty &<& y < 0\\
 0&<& e^{-y} < 1\\
 -1&<& -e^{-y}< 0\\
 0&<& 1-e^{-y}< 1\\
 0&<& \left(1-e^{-y} \right)^2 < 1\\
 f_{Y}\left(y \right) &=&  \begin{cases}
\frac{e^{-y}}{\left(1-e^{-y} \right)^2} & \\
0 & \textrm{e.o.c}
\end{cases}\\
\end{eqnarray*}

\item Sea $f_{XY}\left( x,y\right) = cx, 0<x<y<1$. Hallar la densidad de $T=\frac{Y}{X}$

\begin{eqnarray*}
1&=& \int_{0}^{1} \int_{x}^{1} cx \, dy \, dx =  c \int_{0}^{1}  xy \mid _{x}^{1}  \, dx \\
&=&  c \int_{0}^{1} x \left( 1-x \right) \, dx = c \int_{0}^{1} x -x^{2} \, dx\\
&=& c \left( \frac{x^{2}}{2}- \frac{x^{3}}{3}\right)_{0}^{1}= c \left(\frac{1}{2}- \frac{1}{3} \right)= c \left(\frac{3-2}{6} \right)\\
&=& c \left(\frac{1}{6} \right)\\
\therefore c &=& 6\\
F_{T} \left( t 	\right)&=& \prob \left[T \leq t \right] = \prob \left[Y \leq tX \right] , 0<t<1\\
&=& \int  _{0}^{\frac{1}{t}} \int_{x}^{tx} 6x \, dy \, dx + \int  _{\frac{1}{t}}^{1} \int_{x}^{1} 6x \, dy \, dx \\
&=& \int  _{0}^{\frac{1}{t}} 6xy \mid_{x}^{tx}  \, dx + \int _{\frac{1}{t}}^{1}  6xy \mid_{x}^{1}  \, dx = \int _{0}^{\frac{1}{t}}  6x \left( tx-x \right)   \, dx \\
t \int_{\frac{1}{t}}^{1} 6x \left(1-x \right) \, dx &=& \int  _{0}^{\frac{1}{t}} 6tx^{2}- \frac{6x^{2}}{2}   \, dx + \int _{\frac{1}{t}}^{1}  6x- \frac{6x^{2}}{2}   \, dx \\
&=& \frac{6tx^{3}}{3}\mid_{0}^{\frac{1}{t}} - \frac{6x^{3}}{6}\mid_{0}^{\frac{1}{t}} + \frac{6x^{2}}{2}\mid_{\frac{1}{t}}^{1} - \frac{6x^{3}}{6}\mid_{\frac{1}{t}}^{1} \\
&=& 2t \left(\frac{1}{t}\right) ^{3}- \left(\frac{1}{t}\right) ^{3}+3\left( 1 \right)^{2}- 3\left( \frac{1}{t}\right)^{2}-\left(1 \right)^{3}+\left( \frac{1}{t} \right)^{3}\\
&=& \frac{2}{t^{3}}- \frac{1}{t^{3}}+3-3\left(\frac{1}{t^{2}} \right)-1+\frac{1}{t^{3}}= - \frac{1}{t^{2}}+2\\
F_{T}\left(t \right)&=&\begin{cases}
-t^{-2}+2 & t \in \left( 0,1 \right) \\
0 & \textrm{e.o.c}
\end{cases}\\
\end{eqnarray*}

\item $X,Y \sim U\left(0,1\right)$ Ind. Hallar $f_{u,v}\left(u,v \right)$ si $U=X, V= X+Y$

\begin{eqnarray*}
J&=& \left| 
\begin{array}{cc}
0 & 1\\
-1 & 1
\end{array}
\right| 
=\mid 1 \mid=1\\
f_{XY} \left(x,y \right)&=&\begin{cases}
1 & 0<x<1, 0<y<1 \\
0 & \textrm{e.o.c}
\end{cases}\\
f_{u,v} \left(u,v \right) &=& \mid J \mid f_{XY} \left(x,y \right)\\
&=& 1*1=1\\
f_{uv} \left(u,v \right)&=&\begin{cases}
1 & \textrm{Si} 0<u<1 \\
0 & \textrm{e.o.c}
\end{cases}\\
\end{eqnarray*}

\item Con los datos del ejercicio 4. Calcular $E \left(X \right) $ y $Var \left(X \right)$

\smallskip

\begin{eqnarray*}
X &\sim & U_{ \left\{ 1, \ldots N   \right\}  } \, y \, Y\sim U_{ \left\{ 1, \ldots m    \right\}  } \,  \textrm{Con m par}\\
f_{X\nat}&=&\begin{cases}
\frac{1}{N} & x= 1, \ldots N \\
0 & \textrm{e.o.c}
\end{cases}\\
f_{Y}\left(y \right)&=&\begin{cases}
\frac{1}{m} & y= 1, \ldots m \\
0 & \textrm{e.o.c}
\end{cases}\\
E \left(X \right)&=& \sum_{x=1}^{N} X \frac{1}{N}= \frac{1}{N}\sum_{x=1}^{N} X = \frac{1}{N}\left(1+2+\ldots + N \right)\\
&=& \frac{1}{N} \dfrac{\left(N \left(N+1\right)\right)}{2}= \frac{N+1}{2}\\
E \left(X^{2}\right)&=& \sum_{x=1}^{N} x^{2}\frac{1}{N}= \frac{1}{N} \sum_{x=1}^{N} x^{2}= \frac{1}{N} \left( 1^2+2^2+ \ldots+ N^2 \right)\\
&=& \frac{1}{N} \left( \dfrac{N \left(N+1\right)\left(2N+1\right)}{6} \right)= \dfrac{\left(N+1\right)\left(2N+1\right)}{6}\\
Var\left(X\right)&=& E \left(X^2\right)- E^2\left(X\right)= \dfrac{\left(N+1\right)\left(2N+1\right)}{6}- \dfrac{\left(N+1\right)^2}{4}\\
&=&\dfrac{2\left(N+1\right)\left(2N+1\right)-3\left(N+1\right)^2}{12}\\
E\left(X\right)&=& E \left(E\left(X\mid N \right)\right)= E \left(\frac{N+1}{2} \right)= \frac{1}{2} E \left(N \right)+ \frac{1}{2}= \frac{1}{2}\left( \frac{m+1}{2} \right)+\frac{1}{2}\\
&=& \frac{m+1}{4} + \frac{1}{2}= \frac{m+1+2}{4}= \frac{m+3}{4}\\
Var \left(X \right)&=& E\left(Var \left(X \mid N \right) \right)+ var \left(E \left(X \mid N \right) \right)\\
&=& E \left(\dfrac{2\left(N+1\right)\left(2N+1\right)-3\left(N+1\right)^2}{12} \right)  + Var \left(\frac{N+1}{2} \right)
\end{eqnarray*}

\end{enumerate}

\section{Ejercicios para Tarea}

\begin{enumerate}
\item Dar un ejemplo de v.a discretas tales que $Cov\left(X,Y \right)=0$ pero que no sean ind.

\begin{eqnarray*} 
f_{XY}\left(x,y \right)&=&\begin{cases}
\frac{1}{13} & x,y \in D \\
0 & \textrm{e.o.c}
\end{cases}\\
f_{Y}\left(y \right)= f_{X}\left(x\right)&=& \begin{cases}
\frac{1}{13} & x=-2 \\
\frac{3}{13} & x=-1 \\
\frac{5}{13} & x=0 \\
\frac{3}{13} & x=1 \\
\frac{1}{13} & x=2 \\
0 & \textrm{e.o.c}
\end{cases}\\
E\left[X \right]&=& \left(-2 \right)\left(\frac{1}{13}\right)+ \left(-1 \right) \left(\frac{3}{13}\right)+ \left(0 \right)\left(\frac{3}{13}\right)+ \left(1 \right)\left(\frac{3}{13}\right)+ \left(2 \right) \left(\frac{2}{13}\right)\\
&=& 0\\
&=& E \left[Y \right]\\
E \left[XY \right] &=& \sum_{y} \sum_{x} xy \frac{1}{13}= \left( \frac{1}{13} \right)\sum_{y} \sum_{x} xy\\
&=& \left(-2 \right) \left(-2 \right) + \left(-2 \right)\left(-1\right)+ \left(-2 \right)\left(0\right)+ \left(-2\right)\left(2\right)+ \left(-1 \right)\left(-2\right)\\
&+& \left( -1\right)\left(-1 \right)+ \ldots + \left(-1 \right)\left(2\right)+0+ \left(1 \right)\left(-2\right)\left(1 \right)\left(-1\right)\\
&+& \left(1 \right)\left(0\right)+ \left(1 \right)\left(1\right)+ \left(1 \right)\left(2\right)+ \left(2 \right)\left(-2\right)+ \left(2 \right)\left(-1\right)+ \ldots + \left(2 \right)\left(2\right)\\
&=& 0
\end{eqnarray*}

\item Hallar todos los momentos de $X$ si:

\smallskip
\textbf{a)} $X \sim  N \left(0,1\right)$
\begin{eqnarray*} 
M_{N}\left(t \right)&=& e^{\frac{t^2}{2}}\\
M^{'}_{X} \left(0 \right)&=& te^{\frac{t^2}{2}}=0\\
M^{''}_{X} \left(0 \right)&=& t^{2}e^{\frac{t^2}{2}}+ e^{\frac{t^2}{2}}=1\\
\end{eqnarray*}

\textbf{b)} $X \sim  U \left(a,b\right)$
\begin{eqnarray*} 
E\left(X^r\right)&=& \int_{a}^{b} x^{r} \frac{1}{b-a}\, dx = \frac{1}{b-a} \int_{a}^{b} x^{r} \, dx\\
&=& \frac{1}{b-a} \frac{x^{r+1}}{r+1}\mid_{a}^{b}= \frac{1}{b-a}\left( \frac{b^{r+1}}{r+1}- \frac{a^{r+1}}{r+1} \right)\\
&=&\frac{1}{\left(r+1\right)\left(b-a\right)}\left(b^{r+1}- a^{r+1}\right)\\
r&=& 1\\
&=&\frac{1}{2\left(b-a\right)}\left(b^2 - a^2 \right)= \frac{1}{2} \frac{\left(b-a\right)\left(b+a\right)}{\left(b-a\right)}= \frac{b+a}{2}
\end{eqnarray*}

\textbf{c)} $X \sim Gamma \left(2,1 \right)$
\begin{eqnarray*} 
E \left[X^r \right]&=& \frac{M \left(\alpha + r\right)}{\lambda^{r}M\left(r \right)}\\
r&=& 1\\
\frac{M \left(2+1\right)}{2^1 M \left(1\right)}&=& M\left(3\right)= 2\\
r&=&2\\
\frac{M\left(2+2\right)}{1^2 M \left(2\right)})&=& M\left(4 \right)= 3! = 6
\end{eqnarray*}


\item A una fiesta llegan 20 pares de gemelos, se forman equipos de 2 personas al azar. Hallar el n?mero promedio de equipos formados por gemelos.     
\begin{eqnarray*}  
X_{i}&=&  \begin{cases}
1 & \textrm{Si la pareja i es de Gemelos}, \, i= 1... 20 \\
0 & \textrm{e.o.c}
\end{cases}\\   
X &=& X_{i}+ \ldots + X_{20}= \textrm{N?mero de parejas formadas por gemelos} \\
E \left(X \right) &=& \sum_{i=1}^{20} E \left(X_{i} \right) = \sum_{i=1}^{20} \prob \left[ X_{i}= 1 \right]= 20 \prob \left[X_{i}=1 \right]\\
\prob \left[ X_{1} = 1 \right] &=& \prob \left[ \textrm{Primer pareja sea de gemelos} \right] = \dfrac{20 \left(\begin{array}{c} 2 \\2\end{array} \right)\left(\begin{array}{c} 38 \\0\end{array}  \right)}{\left(\begin{array}{c} 40 \\2\end{array} \right)} = \frac{1}{39} \\
\prob \left[ X_{2}=1 \right]&=& \prob \left[ \textrm{Segunda pareja sea de gemelos} \right]= \prob \left[\textrm{2da}\ldots \mid \textrm{1ra. Fue }\right] \prob \left[\textrm{2da. Fue}\mid \textrm{1ra. No fue} \right]\prob \left[ \textrm{1ra. no fue} \right]\\
&=& \dfrac{19 \left( \begin{array}{c} 2 \\2\end{array} \right) \left(\begin{array}{c} 36 \\0 \end{array} \right)}{\left( \begin{array}{c} 38 \\2\end{array} \right)} \left( \frac{1}{39} \right)+ \dfrac{18 \left(\begin{array}{c} 2 \\2\end{array} \right) \left( \begin{array}{c} 34 \\2\end{array} \right)}{\left(\begin{array}{c} 38 \\2\end{array} \right) }= \frac{1}{3739}+ \frac{36}{3739} = \frac{1}{39}
\end{eqnarray*}

\item Hallar el tercer momento factorial de $X$ si:

\textbf{a)} $X \sim Poisson \left(1 \right)$
\begin{eqnarray*} 
f_{x}\left(2\right)&=& \begin{cases}
\frac{e^{-\lambda}\lambda ^{x}}{x!} & x= 0,1,\ldots\\
0 & \textrm{e.o.c}
\end{cases}\\
E\left[X\left(X-1\right)\left(X-2\right)\right] &=& \sum_{x=0}^{\infty} x \left(x-1\right)\left(x-2\right)\frac{e^{-\lambda}\lambda ^{x}}{x!}\\
&=& \sum_{x=0}^{\infty} \frac{e^{-\lambda}\lambda ^{x}}{\left( x-3\right)!} = \sum_{x=3}^{\infty}\frac{e^{-\lambda}\lambda ^{x}}{\left( x-3\right)!} \\
&=& \lambda ^{3} \sum_{x=3}^{\infty} \frac{e^{-\lambda}\lambda ^{x-3}}{\left( x-3\right)!}\\
y&=& x-3\\
&\therefore & E \left[X \left(X-1\right)\left(X-2\right) \right]= \lambda ^{3}
\end{eqnarray*}


\textbf{b)} $X \sim Bin \left(n,p \right)$
\begin{eqnarray*} 
f_{X}\left(x\right)&=& \left(\begin{array}{c}n\\x \end{array}\right)p^{x}\left(1-p\right)^{n-x}, x=0,1,2.\ldots \\
E \left[X \left(X-1 \right)\left(X-2 \right) \right] &=& \sum_{x=0}^{\infty} \left( \begin{array}{c} n\\x \end{array}\right)p^x \left( 1-p\right)^{n-x} x \left( x-1\right) \left(x-2 \right)\\
&=& \sum _{x=0}^{\infty}  x \left( x-1 \right) \left(x-2 \right) \dfrac{n!}{\left( n-x \right)! x! }  p^{x} \left( 1-p\right)^{n-x}\\
&=& \dfrac{n!}{\left( n-x \right)! \left( x-3\right)! }  p^{x} \left( 1-p\right)^{n-x}\\
&=& n \left(n-1 \right)\left(n-2 \right)\sum_{x=3}^{\infty}\dfrac{\left(n-3 \right)! }{\left( n-x \right)! \left( x-3\right)! } p^{x} \left( 1-p\right)^{n-x}
\end{eqnarray*}
 \item Si $E \left[X \right]$ existe mostrar que $E \left[X \right]= \int_{0}^{\infty} \left( 1 - F_{X}\left(x \right) \right)\, dx - \int_{-\infty}^{0} F_{X}\left(x \right)\, dx $

\begin{eqnarray*} 
F_{X}\left(x \right)&=& \int_{-\infty}^{x} f_{X}\left(t \right)\, dt \Rightarrow 1-F_{X}\left( x\right)= 1 -\int_{-\infty}^{x}F_{X}\left( t\right) \, dt\\
&=& \int_{x}^{\infty} f_{X} \left( t\right)\, dt\\
E \left[X \right]&=& \int_{0}^{\infty} \left(1 - \int _{-\infty}^{x} f_{X} \left(t \right)\, dt\right)\, dx - \int_{-\infty}^{0}\int_{-\infty}^{x} f_{X} \left(t \right)\, dt\\
&=& \int_{0}^{\infty} \int_{x}^{\infty} f_{X}\left(t \right)\, dt\,dx - \int_{-\infty}^{0} \int_{-\infty}^{x} f_{X}\left(t \right)\, dt\,dx\\
&=& \int_{0}^{\infty} \int_{0}^{t} f_{X}\left(t \right)\, dx\,dt - \int_{-\infty}^{0} \int_{t}^{0} f_{X}\left(t \right)\, dx\,dt\\
&=&\int_{0}^{\infty}f_{X}\left(t \right) \int_{0}^{t} \, dx\,dt - \int_{-\infty}^{0}f_{X}\left(t \right) \int_{t}^{0} \, dx\,dt\\
&=&\int_{0}^{\infty}f_{X}\left(t \right)t \,dt - \int_{-\infty}^{0}f_{X}\left(t \right)\left(-t \right) \,dt\\
&=& \int_{-\infty}^{\infty} tf_{X}\left(t\right) \, dt = \int_{-\infty}^{\infty} x f_{X} \left(x \right)\, dt
\end{eqnarray*} 
 \item Hallar $E \left(X^2 Y \right)$ si:
 
 \smallskip
 
 \textbf{a)} $X,Y \sim U_{\left\{1 \ldots N \right\}}$ Ind.
 
 \begin{eqnarray*}
 f_{X} \left(X \right)&=& \begin{cases}
 \frac{1}{N} & x=1 \ldots N\\
 0 & \textrm{e.o.c}
 \end{cases}\\
  f_{Y} \left(y \right)&=&
 \begin{cases}
 \frac{1}{N} & y=1 \ldots N\\
 0 & \textrm{e.o.c}
 \end{cases}\\
 f_{XY} \left(x,y \right)&=& f_{X} \left(x \right)f_{Y} \left(y \right)
 \begin{cases}
 \frac{1}{N^2} & x,y \in \left\{ 1 \ldots N \right\} \\
 0 & \textrm{e.o.c}
 \end{cases}\\
 E \left(X^2 Y \right) &=& \sum_{x}\sum_{y} x^{2}y \frac{1}{N^2} = \frac{1}{N^2}\sum_{x=1}^{N} x^2 \sum_{y=1}^{N} y\\
 &=& \frac{1}{N^2}\sum_{x=1}^{N} x^2 \left(\frac{N \left( N+1\right) }{2} \right)= \frac{\left(N+1\right) }{2N} \sum_{x=1}^{N} x^2\\
 &=& \frac{N+1}{2N} \left[1+2^2+ 3^2+ \ldots + N^2 \right]
 \end{eqnarray*}
 
 \textbf{b)} $X, Y \sim U_{ \left(0,1 \right) }$ ind.
 
 \begin{eqnarray*}
 f_{X} \left(x \right) &=&
 \begin{cases}
 1 & 0<x<1 \\
 0 & \textrm{e.o.c}
 \end{cases}\\ 
  f_{Y} \left(y \right) &=&
 \begin{cases}
 1 & 0<y<1 \\
 0 & \textrm{e.o.c}
 \end{cases}\\ 
 f_{XY} \left(x,y \right)&=& f_{X} \left(x \right)f_{Y} \left(y \right)
 \begin{cases}
1 & 0<x<1 , \, 0<y<1 \\
 0 & \textrm{e.o.c}
 \end{cases}\\
 E \left(X^{2} Y \right)&=& \int_{0}^{1}\int_{0}^{1} x^{2} y \, dx \, dy = \left.\int_{0}^{1} y \frac{x^3}{3} \right|_{0}^{1} \, dy\\ &=&\frac{1}{3} \int_{0}^{1} y \, dy = \frac{1}{6} \left(\left.y^{2} \right|_{0}^{1}\right) = \frac{1}{6}
 \end{eqnarray*}
 \item Una Urna tiene 2 bolas negras y 3 rojas. Se extraen bolas sucesivamente sin reemplazo. \\
 Sean $X = $ N?mero de extraccoines donde apareci? la primera roja. 
 $Y=$ N?mero de extracciones donde apareci? la primera negra. 
 Hallar $p \left(X,Y \right)$
\begin{eqnarray*}
X &=& \textrm{N?mero de extracciones donde apareci? la Primer bola roja.}= \left\{1,2,3 \right\}\\
Y &=& \textrm{N?mero de extracciones donde apareci? la Primer bola negra.}= \left\{1,2,3,4 \right\}\\
P \left[X=2, Y=1 \right] &=& \left(\frac{2}{5}\right) \left(\frac{3}{4}\right)
\end{eqnarray*}  

\item 
\textbf{a)}  $X,Y \sim U \left(0,3 \right)$ Ind. Hallar $E \left(X \mid X+Y > 4 \right)$

\begin{eqnarray*}
y&=& 4-x, \left(0,4 \right), \left(4,0 \right)\\
3 &=& 4-x \Rightarrow x= 4-3 \Rightarrow x=1\\
f_{XY} \left(x,y \right) &=& \begin{cases}
\frac{1}{9} & 0<x<3, 0<x<3\\
0 & \textrm{e.o.c}
\end{cases}\\
E \left[X \mid X+Y > 4 \right] &=& \int x f_{X\mid D} \left( x \mid D \right) \, dx\\
f_{X\mid D} \left( x \mid D \right) &=& \dfrac{\prob \left[X \leq x , X+Y > 4\right] }{\prob \left[X+Y > 4 \right]  }\\
\prob \left[X \leq x , X+Y > 4\right] &=& \begin{cases}
\int_{1}^{x} \int_{4-t}^{3} \frac{1}{9} \, dy \, dt & 1<x<3\\
\int_{1}^{3} \int_{4-x}^{3}  \frac{1}{9} \, dy \, dx & x>3\\
0 & x<1
\end{cases}\\
\prob \left[X+Y > 4 \right] &=& \frac{1}{9} \, dy \, dx
\end{eqnarray*}
\textbf{b)} $X,Y \sim Geo \left(p \right)$ ind. $f_{X}\left(x \right)= pq^{x-1}$, $x\ 1,2, \ldots$. \\ Hallar $E \left(X \mid Y = X+2 \right)$

\begin{eqnarray*}
f_{X}\left(x \right) &=& \begin{cases}
pq^{x-1} & x=1,2.\ldots\\
0 & \textrm{e.o.c}
\end{cases}\\
f_{Y}\left( y\right) &=& \begin{cases}
pq^{y-1} & x=1,2.\ldots\\
0 & \textrm{e.o.c}
\end{cases}\\
f_{XY}\left(x,y \right) &=& \begin{cases}
p^{2}q^{x+y-2} & x,y \in \left\{1,2.\ldots \right\} \\
0 & \textrm{e.o.c}
\end{cases}\\
E \left[X \mid Y = X+2 \right]&=& \sum_{1}^{\infty} x f_{X\mid D}\left(x\mid D \right)\\
f_{X\mid D}\left(x\mid D \right)&=& \dfrac{\prob \left[X=x, Y=X+2 \right] }{
\left[Y=X+2 \right] }\\
\prob \left[X=x, Y = X+2\right]&=& \prob \left[Y=X+2 \mid X=x \right]\prob \left[X=x \right]\\
\prob \left[Y=X+2 \right]&=& \sum_{K=1}^{\infty} \prob \left[Y=K+2 \mid X=K \right] \prob \left[X=K \right]\\
&=& \sum_{K=1}^{\infty}\dfrac{ \prob \left[Y=K+2 \mid X=K \right]}{\prob \left[X=K \right]} \prob \left[X=K \right]\\
&=& \prob \left[Y=K+2  \right] \prob \left[X=K \right]
\end{eqnarray*}

\item Hallar $\varphi_{X}\left(t\right)$ si $X \sim Geo \left(p \right)$
\begin{eqnarray*}
f_{X} \left(x \right) &=& \begin{cases}
pq^{x-1} & x=1,2\ldots \\
0 & \textrm{e.o.c}
\end{cases}\\
\varphi_{X}\left(t \right)&=& E \left( e^{iXt} \right)= \sum_{x=1}^{\infty} e^{iXt} pq^{x-1} \\
&=& pe^{it} \sum \left(e^{it}\left( 1-p\right) \right)^{x-1}\\
&=& \dfrac{pe^{it}}{1-e^{it}\left(1-p \right) }
\end{eqnarray*}
\item $X \sim Bin neg \left(r,p \right)$\\
Si 
\begin{eqnarray*}
Y_{1}, Y_{2}, \ldots, Y_{r} &\sim& Geo \left(p \right) \Rightarrow X= Y_{1}, Y_{2}, \ldots, Y_{r} \sim Binneg \left(r,p \right) \Rightarrow \varphi_{X} \left(t \right)\\
&=& \left( \varphi_{Y} \left(t \right) \right)^{r}
 =\left(\dfrac{pe^{it}}{1-e^{it}\left(1-p \right) } \right)^{r}
\end{eqnarray*}
\item Hallar $\varphi_{X} \left(t \right)$ si $\prob \left[X-2 \right]= \frac{1}{4}= \prob \left[X=-2 \right] \prob \left[X=0 \right]= \frac{1}{2}$

\begin{eqnarray*}
\varphi_{X} \left(t \right)&=& E \left(e^{itX} \right) = e^{itX} \prob \left[X=0 \right]+ e^{itx} \prob \left[X=2 \right]+ e {itx} \prob \left[X=-2 \right]\\
&=& \frac{1}{2}+ e^{2it}\left(\frac{1}{4}\right)+ e^{2it} \left(\frac{1}{4}\right)\\
&=& \frac{1}{2} + \frac{1}{4} \left(e^{2it}+ e^{-2it} \right)\\
&=& \frac{1}{2} + \frac{1}{4} \cos \left(2t \right)
\end{eqnarray*}

\item $X$ variabe aleatroria continua con densidad $f_{X} \left(x \right)= \frac{1}{2} e ^{- \mid x \mid}$ con $x \in \rea$ Mostrar que $\varphi_{X}\left(t \right)= \frac{1}{1+t^{2}}$

\begin{eqnarray*}
\varphi_{X}\left(t \right)&=& E \left(e^{iXt}\right)= \int_{-\infty}^{\infty} e^{iXt} \frac{1}{2} e ^{- \mid x \mid}\, dx\\
&=& \frac{1}{2} \left[ \int_{-\infty}^{0} e^{iXt} e^x \, dx + \int _{0}^{\infty} e^{iXt}e^{-x}\, dx \right]\\
&=& \frac{1}{2} \left[ \int_{-\infty}^{0} e^{\left(1+it \right) } \, dx + \int _{0}^{\infty} e^{-x \left(1-it \right)}\, dx \right]\\
&=&\left. \frac{1}{2} \left( \frac{1}{1+it}e^{1+it}\right|_{-\infty}^{0} - \left.\frac{1}{1-it}e^{-x} \left( 1-it \right)\right|_{0}^{\infty}\right)\\
&=& \frac{1}{2} \left( \frac{1}{1+it} - \frac{1}{it-1} \right) \\
&=& \frac{1}{1+t^{2}}
\end{eqnarray*}
 
 
 \item  Mostrar que $E \left[E \left(X \mid Y,Z \right) \right]= E \left(X \right)$
 
\begin{eqnarray*}
E \left(X \right)&=& \int_{-\infty}^{\infty} X f_{X}\left( x\right) \, dx\\
f_{X} \left(x \right)&=& \int_{Y} \int_{Z} f_{XYZ} \left(x,y,z \right) \, dz\, dy  \\
&=& \int_{-\infty}^{\infty} x \int_{y} \int_{z} f_{xyz} \left(x,y,z \right) \, dz \, dy \, dx\\
E \left( E \left(X\mid Y,Z \right)\right)&=& \int \int E \left(X \mid Y=y, Z=z \right)f_{YZ} \left(y,z \right) \, dy \, dz\\
&=& \int \int \int x f_{X\mid Y,Z} \left( x\mid y,z\right)f_{YZ} \left(y,z \right) \, dy\, dz \, dx\\
&=& \int \int \int xf_{XYZ} \left(x,y,z \right)\,dz\,dy\,dx\\
&=& \int xf_{X} \left(x \right) \, dx\\
&=& E \left(X \right)
\end{eqnarray*} 

\end{enumerate}
\section{Ejercicios para  Evaluaci\'on}

\begin{enumerate}
\item Se tienen bolas numeradas. Dos personas I y II eligen cada una 3 de las 6 bolas sin reemplazo y con independencia entre personas. Sea $X$ el n\' umero de bolas que nadie seleccion\' o. Hallar $E \left(X \right)$.

\begin{eqnarray*}
X_{i} &=&\begin{cases}
1 & \textrm{Si la bola i nadie la elige}  \\
0 & \textrm{e.o.c}
\end{cases}\\
i&=&1,\ldots ,6\\
X&=& X_{1}+\ldots+ X_{6}\\
E\left(X \right) &=& \sum_{i=1}^{6} E\left(X_{i} \right)= 6 E\left(X_{i} \right) = 6 \prob \left[X_{i} = 1 \right] = 6 \left(\frac{1}{4} \right) = \frac{3}{2}\\
\prob \left[X_{i} = 1 \right]  &=& \prob \left[ \textrm{ Bola 1 no la elija I} \right] \prob \left[ \textrm{ Bola 1 no la elija II} \right]\\
&=& \left( \prob \left[ \textrm{ Bola 1 no la elija I} \right]\right)^{2}\\
&=& \left(\dfrac{\left(\begin{array}{c} 
1\\
0
\end{array} \right)\left(\begin{array}{c} 
5\\
3
\end{array} \right) }{ \left(\begin{array}{c} 
6\\
3
\end{array} \right)} \right)^{2}= \left( \frac{10}{20}\right)^{2} = \frac{1}{4}
\end{eqnarray*}

\item $X,Y \sim U \left(0,2 \right)$ ind. Hallar $E \left(X \mid X+Y > 2 \right) $

\begin{eqnarray*}
E \left(X \mid X+Y>2\right) &=& \frac{1}{\prob \left[D \right] } E \left(X \parallel _{D} \right)\\
x+y &=& 2 \Rightarrow y=2-x\\
\left(0,2 \right) \left(2,0 \right)\\
f_{XY} \left(x,y \right)&=& \begin{cases}
\frac{1}{4} & 0<x<2 , 0<y<2\\
0 & \textrm{e.o.c}
\end{cases}\\
\prob \left[D \right] &=& \prob \left[X+Y=2 \right] = \prob \left[ Y >2 -X\right]\\
&=& \int _{0}^{2} \int_{2-x}^{2} f_{XY} \left(x,y \right) \, dy\,dx\\
&=&\frac{1}{4} \int _{0}^{2} \left. y \right|_{2-x}^{2} \, dx = \frac{1}{4} \int _{0}^{2} 2-2+x \, dx\\
&=& \frac{1}{4} \int_{0}^{2} x \, dx = \frac{1}{4} \left(\frac{x^2}{2} \right)_{0}^{2}= \frac{1}{4} \left(2 \right) = \frac{1}{2}\\
E \left( X \parallel _{D} \right)&=& \frac{1}{4} \int \int_{\rea}  x \parallel_{D}
\, dx\, dy = \frac{1}{4} \int \int_{D\cap\rea} x \, dx\, dy\\
 &=&\frac{1}{4} \int_{0}^{2} \int_{2-x}^{2}  x  \, dy\, dx = \frac{1}{4}\left.  \int_{0}^{2} xy \right|_{2-x}^{2} \, dx\\
 &=& \frac{8}{3} \left(\frac{1}{4} \right)= \frac{2}{3}\\
 \frac{1}{\prob \left(D \right)}E \left(X\parallel _{D} \right) &=& 2 \left(\frac{2}{3} \right)= \frac{4}{3}
 \end{eqnarray*}

\item $X_{1} \ldots X_{121}$ m.a de una densis?dad $f$ con media $2$ y varianza $1$. Hallar $c \in \rea$ t.q.
$\prob \left[240- \frac{c}{2} \leq S_{121} \leq 240 + \frac{c}{2} \right]= 0.94$
 \begin{eqnarray*}
 S_{121} &=& \sum_{i=1}^{121} X_{i}\\
 E \left(X_{i} \right)&=& 2\\
  Var \left(X_{i} \right)&=& 4 \, \, \, \, \, \, \, \, \, \, \, \, \forall i=1 \ldots 121 \\
  E \left(S_{121} \right)&=& \sum_{i=1}^{121} E \left(X_{i}\right) = \sum_{i=1}^{121} \left(2 \right)= 2 \left(121 \right)= 242\\
  Var \left( S_{121} \right)&=& Var \left( \sum_{i=1}^{121} \right) X_{i})= \sum_{i=1}^{121} Var \left(X_{i} \right)= 4 \left(121 \right)= 484\\
\prob \left[S_{121} \leq 240 + \frac{c}{2} \right]&-& \prob \left[S_{121} \leq 240-\frac{c}{2} \right]\\
\prob \left[S_{121} \leq 240 + \frac{c}{2} \right]&=& \prob \left[  \dfrac{S_{121}-E \left( S_{121} \right) }{ \sqrt{484} } \leq \dfrac{240 + \frac{c}{2}- E \left(S_{121} \right)}{\sqrt{484}  }\right]\\
&=& \prob \left[ \dfrac{S_{121}- 242}{22} \leq \dfrac{240 - 242 + \frac{c}{2} }{22} \right]\\
&=& \prob \left[z \leq \frac{ \frac{-4+c}{2} }{22} \right] = \prob \left[ z \leq \frac{c-4}{44} \right] = \phi \left( \frac{c-4}{44} \right)\\
\prob \left[S_{121} \leq 240-\frac{c}{2} \right] &=& \prob \left[ \dfrac{S_{121} - 242 }{22} \leq \dfrac{240 - \frac{c}{2}-242 }{22} \right] = \prob \left[z \leq \frac{c+4}{22} \right]\\
&=& \phi \left(- \frac{c+4}{44} \right)
 \end{eqnarray*}
 \item Sean $X_{1} \sim N \left( 1,3 \right) $, $X_{2} \sim N \left( 0,5 \right) $ ind. Usar funci?n caracter?stica para encontra la densidad de $Y=2X_{1}+3X_{2}$
 
 \begin{eqnarray*}
 \varphi_{Y} \left(t \right) &=& E \left(e^{itY} \right) = E \left( e^{it \left(2X_{1}+ 3X_{2} \right) } \right) = E \left( e^{it2X_{1}}e^{it3X_2{}} \right)\\
 &=&  E \left( e^{2itX_{1}} \right) E\left(e^{3itX_2} \right)=  E \left( e^{i \left( 2t \right) X_{1}} \right) E\left(e^{i \left( 3t \right) X_2} \right)\\
 &=& \varphi_{X_{1}} \left(2t \right)\varphi_{X_{2}} \left(3t \right) \\
 X &\sim & N \left(\mu, \sigma^{2} \right)\\
 f_{X} \left(x \right) &=& \dfrac{1}{\sqrt{2\pi\sigma^2}}exp \left\{ - \frac{\left( x-\mu \right)^2 }{2\sigma^{2}} \right\}\\
 \varphi_{X} \left(t \right) &=& E \left( e^{itX} \right) = \int e^{itX} \frac{1}{\sqrt{2\pi\sigma^2}}exp \left\{ - \frac{\left( x-\mu \right)^2 }{2\sigma^{2}} \right\}\\
 &=& \frac{1}{\sqrt{2\pi\sigma^2}} \int_{-\infty}^{\infty} exp \left\{ - \frac{\left( x-\mu \right)^2 }{2\sigma^{2}}+ itX \right\}\, dx\\
 &=& - \frac{\left( x-\mu \right)^2 }{2\sigma^{2}}+ itX =- \frac{1}{2\sigma^2} \left[ x^2+2x\mu+ \mu^2+ itx2\sigma^2 \right]\\
 &=& - \frac{1}{2\sigma^2}\left( x^2 - 2x \left( \mu - it\sigma^2 \right)+ \mu^2 \right)\\
 &=& - \frac{1}{2\sigma^2}\left( x^2 - 2x \left(\mu - it \sigma^2 \right) + \left(\mu - it\sigma^2 \right)^2 - \left(\mu - it\sigma^2 \right)^2+ \mu^2 \right)\\
 &=&  - \frac{1}{2\sigma^2} \left[ \left( x- \left(\mu - it\sigma^2 \right) \right)^2 + 2\mu it\sigma^2- \left(it\sigma ^2 \right)^2 \right]\\
 &=& \dfrac{1}{\sqrt{2\pi\sigma^2}} \int_{-\infty}^{\infty} exp \left\{ - \frac{1}{2\sigma^2} \left[ \left( x- \left(\mu - it\sigma^2 \right) \right)^2 + 2\mu it\sigma^2- \left(it\sigma ^2 \right)^2 \right] \right\} \, dx\\
 &=& exp \left\{ - \frac{1}{2\sigma^2} \left( 2\mu it \sigma^2 - \left( it\sigma^2 \right)^2 \right)\right\}\\
 \mu &=& 1\\
 \sigma^2 &=& 3\\
 \varphi_{X_{1}} \left(2t \right) &=& exp \left\{ - \frac{1}{6} \left( 2 \left(1 \right)i \left(2t \right) \left(3 \right)- \left(i \left(2t \right)\left(3 \right) \right)^{2} \right)\right\}\\
 &=& exp \left\{- \frac{1}{6} \left(12it \right) - \left( 6it\right)^2 \right\}\\
 &=& exp \left\{- \left(2it + 6t^2 \right) \right\}\\
 \mu &=& 0\\
 \sigma^2 &=& 5\\
 \varphi _{X_{2}}\left(3t \right) &=& exp \left\{ - \frac{1}{10} \left(- \left( i \left(3t \right)5\right)^2 \right) \right\}\\
 &=& exp \left\{ - \frac{1}{10} \left(225t^2 \right)\right\}
 \end{eqnarray*} 
  \begin{eqnarray*} 
  \varphi_{Y} \left(t \right)&=& \varphi_{X_{1}} \left(2t \right)\varphi_{X_{2}} \left(3t \right)\\
  &=& exp \left\{- \left(2it + 6t^2 \right) \right\} exp \left\{- \frac{1}{10}\left(225t^2 \right) \right\} \\
  &=& exp \left\{ -2it - \frac{285t^2}{10} \right\}\\
  &=& exp \left\{ - \frac{20it - 285t^2}{10} \right\} = exp \left\{ 2it - \frac{57}{2} t^2 \right\}
   \end{eqnarray*} 
   \item $Y \sim U_{0,1,\ldots, N }$, $N \sim Geo$. Hallar $Var \left( Y+1 \right)$ $f_{N} \left(n \right)= pq^{n}$
   \begin{eqnarray*}
   Y\mid N \sim U_{0,1, \ldots,N }&=& f_{Y\mid N} \left( y \mid n \right) = \begin{cases}
   \frac{1}{N+1} & y=0,1 \ldots N  \\   o & \textrm{e.o.c}
   \end{cases}\\
   N \sim Geo \left(p \right) &\Rightarrow& f_{N} \left( n \right) = 
 \begin{cases}
   pq^n & n=0,1,2, \ldots\\
   o & \textrm{e.o.c}
   \end{cases}\\  
   Var \left(Y+1 \right)&=& Var \left(Y \right) = Var \left( E \left(Y \mid N \right) \right)+ E \left( Var \left(Y \mid N \right) \right)\\
   &=& Var \left( \frac{N}{2} \right)+ E \left( \frac{N \left(N+2 \right) }{12 }\right)\\
   &=& \frac{1}{4} Var \left(N \right)+ \frac{1}{12} E \left[ N \left(N+2 \right) \right]\\
   &=& \frac{1}{4} \left(\frac{q}{p^2} \right)+ \frac{1}{12} \left[E \left(N^2 \right)+ 2E \left(N \right) \right]\\
   &=& \frac{1}{4} \frac{q}{p^2} + \frac{1}{12} \left[ \frac{q \left(1+q \right) }{p^2} + 2 \frac{q}{p} \right]\\
   &=& \frac{1}{3} \frac{q}{p^2}+ \frac{1}{12} \frac{q}{p^2}+ \frac{1}{12} \frac{q^2}{p^2}+ \frac{q}{6p}= \frac{- \left(p-4 \right)\left(p-1\right) }{12p^2} 
\end{eqnarray*}
\end{enumerate}


\section{Ejercicios para tarea del tema de transformaciones}



\begin{enumerate}
\item 
\begin{enumerate}
\item Si $X \sim U_{-1,4}$. Hallar la densidad de $Y= \left[5X \right]$
 \begin{eqnarray*}
 X \sim U_{-1,4} \Rightarrow f_{X}\left(x \right)&=& \begin{cases} \frac{1}{5} & -1 \leq x \leq 4\\
 0 & \textrm{e.o.c}
 \end{cases}\\
 F_{Y} \left( y \right) &=& \prob \left[Y=y \right]= \prob \left[\left[5X \right]= y \right] = \prob \left[ y \leq 5X < y+1 \right]\\
 &=& \prob \left[\frac{y}{5}\leq X < \frac{y+1}{5} \right] = \int_{\frac{y}{5}}^{\frac{y+1}{5}} \frac{1}{5} \, dx = \left. \frac{1}{5}x \right|_{\frac{y}{5}}^{\frac{y+1}{5}} = \frac{1}{5} \left[ \frac{y+1}{5}- \frac{y}{5} \right]= \frac{1}{25}\\
 f_{Y}\left(y \right)&=&    \begin{cases} \frac{1}{25} & y= -5 , \ldots, 20\\
 0 & \textrm{e.o.c}
 \end{cases}
\end{eqnarray*}

\item Si $X \sim \textrm{Cauchy}$. Hallar la densidad de $Y= \frac{1}{X}$

\begin{eqnarray*}
 X \sim \textrm{Cauchy} \Rightarrow f_{X}\left(x \right)&=& \frac{1}{\pi \left(1+x^2 \right)} x \in \rea\\
 Y&=& \frac{1}{X}\\
  \textrm{Sea} y&>&0\\
  F_{Y} \left(y \right)&=& \prob \left[Y \leq y \right] = \prob \left[ \frac{1}{X} \leq y \right] 
\end{eqnarray*}
\item Si $X \sim N \left(0,1 \right) $. Hallar la densidad de $Y= \mid2X\mid $

\begin{eqnarray*}
 X \sim N \left(0,1 \right)  &\Rightarrow& f_{X}\left( x \right)= \frac{1}{\sqrt{2\pi}  }exp \left\{ - \frac{- x^2}{2} \right\}\\
 F_{Y} \left(y \right) &=& \prob \left[Y \leq y \right] = \prob \left[ \mid2X \mid \leq y \right] = \prob \left[-y \leq 2X < y \right]\\
 &=& \prob \left[ - \frac{y}{2} \leq X \leq \frac{y}{2} \right]= \prob \left[X \leq \frac{y}{2} \right] - \prob \left[ X \leq - \frac{y}{2}\right]\\
 &=& F_{X} \left(\frac{y}{2} \right)- F_{X} \left(- \frac{y}{2} \right)\\
 f_{Y} \left(y \right) &=& f_{X}\left(\frac{y}{2}  \right) \left(\frac{1}{2} \right) - f_{x} \left(-\frac{y}{2}  \right) \left(-\frac{1}{2} \right)\\
 &=& f_{X} \left(\frac{y}{2}  \right) \left(\frac{1}{2} \right)  + f_{X} \left(-\frac{y}{2}  \right) \left(\frac{1}{2} \right)  = f_{X} \left(\frac{y}{2}  \right) \left(\frac{1}{2} \right)  + f_{X} \left(\frac{y}{2}  \right) \left(\frac{1}{2} \right)\\
 &=& f_{X} \left( \frac{y}{2} \right)\\
 &=& \begin{cases}
 \frac{1}{\sqrt{2\pi}  }exp \left\{ - \frac{\left(\frac{y}{2} \right)^2 }{2} \right\}= \frac{1}{\sqrt{2\pi}  }exp \left\{ - \frac{y^2}{8} \right\} &  y \geq 0\\
 0 & \textrm{e.o.c}
 \end{cases}\\
 -\infty &<& X < \infty\\
  -\infty &<& 2X < \infty\\
 0 &<& \mid 2X \mid < \infty\\
  0 &<& Y < \infty\\
\end{eqnarray*}
\item Si $X \sim Bin \left( n, \frac{1}{3} \right) $. Hallar la densidad de $Y = n-X$
\begin{eqnarray*}
X \sim Bin \left(n, \frac{1}{3} \right) \Rightarrow f_{X} \left(x \right) &=& \begin{cases}
\left(\begin{array}{c} n \\ x
\end{array}\right) \left(\frac{1}{3} \right)^{x} \left(\frac{2}{3} \right)^{n-x} & x= 0,1, \ldots n \\
 0 & \textrm{e.o.c}
 \end{cases}\\
 f_{Y} \left( y \right)&=& \prob \left[Y=y \right] = \prob \left[ n-X = y \right] = \prob \left[n-y=X \right] = f_{X} \left(n-y \right)\\
 &=& \left(\begin{array}{c}
 n \\ n-y
 \end{array} \right)\left( \frac{1}{3} \right)^{n-y} \left(\frac{2}{3} \right)^{n-n+y}\\
&=& \begin{cases}
\left(\begin{array}{c}
 n \\ n-y
 \end{array} \right)\left( \frac{1}{3} \right)^{n-y} \left(\frac{2}{3} \right)^{y} & y=n-x \Rightarrow y=n,n-1, \ldots ,0\\
 0 & \textrm{e.o.c}
 \end{cases}
 \end{eqnarray*}
 \item Si $X \sim U_{ \left(0,2 \right) }$. Hallar la distribuci?n $Y= X^{2} - X +1$
\begin{eqnarray*}
X \sim U_{ \left(0,2 \right) } \Rightarrow f_{X} \left(x \right) &=& \begin{cases}
\frac{1}{2} & 0 \leq x\leq  \\
 0 & \textrm{e.o.c}
 \end{cases}\\
 F_{Y} \left(y \right) &=& \prob \left[Y \leq y \right]= \prob \left[ X^2 - X + 1 \leq y  \right] = \prob \left[ \left(X - \frac{1}{2} \right)^2 - \frac{1}{4}+ 1 \leq y \right]\\
 &=& \prob \left[ \left(X - \frac{1}{2} \right) ^{2} \leq Y- \frac{3}{4}\right]\\
 \textrm{Caso 1:} \, \,
 \frac{3}{4}<y<1\\
 \prob \left[- \sqrt{y- \frac{3}{4} } \leq X- \frac{1}{2} < \sqrt{y- \frac{3}{4} } \right]\\
 &=& \prob \left[- \sqrt{y- \frac{3}{4}} + \frac{1}{2} < X < \sqrt{y- \frac{3}{4} } + \frac{1}{2}  \right]\\
 F_{X} \left( \sqrt{y- \frac{3}{4}} \right)&-& F_{X} \left(- \sqrt{y- \frac{3}{4} } + \frac{1}{2}  \right)\\
 f_{Y} \left(y \right)&=&  f_{X} \left( \sqrt{y- \frac{3}{4}} \right) \left( \frac{1}{2} \left( y - \frac{3}{4} \right)^{- \frac{1}{2} } \right)\\ &+& f_{X} \left(- \sqrt{y- \frac{3}{4} } +
  \frac{1}{2}  \right)\left(4 \right)\\
  &=& \frac{1}{4 \sqrt{y - \frac{3}{4} }} + \frac{1}{4 \sqrt{y - \frac{3}{4} }} = \frac{1}{2 \sqrt{y - \frac{3}{4} }}
\end{eqnarray*} 
\end{enumerate}
\item 
\begin{enumerate}
 \item Si $X,y \sim Exp \left( \lambda \right) $ ind. Hallar la densidad de $T= min \left\{ X,Y \right\}$
 
\begin{eqnarray*}
X \sim Exp  \left(\lambda \right) \Rightarrow f_{X} \left(x \right) &=& \begin{cases}
\lambda e^{-\lambda x } & x>0\\
0 & \textrm{e.o.c}
\end{cases}\\
t  &\geq & 0\\
F_{T} \left(t \right) &=& \prob \left[ T \leq t \right] = \prob \left[ T= min \left\{ X,Y \right\} \leq t \right] = 1 - \prob \left[T= min \left\{ X,Y \right\} > t \right]\\
&=& 1 - \prob \left[ X > t , Y >t \right] = 1- \prob \left[ X > t \right]\prob \left[ Y > t \right]= 1 - \prob \left[X > t \right]^{2}\\
F_{X} \left(x \right) &=& \int_{0}^{x} \lambda e^{- \lambda t } \, dt  = \lambda \int_{0}^{x}  e^{- \lambda t } \, dt = x \left( - \frac{e^ {- \lambda t }}{\lambda } \right)_{0}^{x}\\
&=& -e ^{- \lambda x } + e ^{0} = 1 - e ^{-\lambda x }\\
&=& 1- \left( 1 - \left( 1 - e ^{-\lambda t }\right) \right)^{2} = 1 - e ^{-2 \lambda t }\\
f_{T} \left(t \right) &=& \left( -e ^{- 2\lambda t } \right) \left(-2 \lambda \right) = \begin{cases} 2 \lambda e^{-2 \lambda t } & t>0\\
0 & \textrm{e.o.c}
\end{cases}\\
\therefore T &\sim& \textrm{Exp} \left(2 \lambda  \right)
\end{eqnarray*}
\item Si $X , Y \sim Geo \left( p \right)$ ind. Hallar la densidad de $W = X+Y$

\begin{eqnarray*}
X \sim Geo \left( p \right) &\Rightarrow& f_{X}\left(x \right) \begin{cases} pq^{x} & x= 0,1, \ldots\\
0 & \textrm{e.o.c}
\end{cases}\\
Y \sim Geo \left( p \right) &\Rightarrow& f_{Y}\left(y \right) \begin{cases} pq^{y} & y= 0,1, \ldots\\
0 & \textrm{e.o.c}
\end{cases}\\
 f_{XY}\left(x,y \right)&=& \begin{cases} p^{2}q^{x+y} & x= 0,1, \ldots, y=0,1, \ldots\\
0 & \textrm{e.o.c}
\end{cases}\\
f_{W} \left(w \right) &=& \prob \left[W=w \right]= \prob \left[X+Y=w \right]= \prob \left[Y= w-X \right]\\
&=& \sum_{k=0}^{w} \prob \left[Y= w-k \right] \prob \left[X=k \right]\\
&=& \sum_{k=0}^{w} \left( pq^{w-k} \right) \left( pq^{k} \right)= \sum_{k=0}^{w} p^{2} q^{w} = p^{2}q^{w} \sum_{k=0}^{w} \\
&=& \begin{cases}p^{2} q^{w} \left( w \right) & w=0,1,\ldots\\
0 & \textrm{e.o.c}
\end{cases}
\end{eqnarray*}
\item  Si $X,Y \sim U_{ \left\{ 1, \ldots, N \right\} }$ ind. Hallar la densidad de $Z = max \left\{X,Y \right\} $

\begin{eqnarray*}
z &\in& \left\{ 1 , \ldots, 2N \right\} \\
f_{z} \left(z \right) &=& \prob \left[Z = z \right] = \prob \left[max \left\{ X,Y\right\} = Z \right]\\
f_{X} \left(x \right)&=& \begin{cases}
\frac{1}{N} & x=1,2, \ldots ,N\\
0 & \textrm{e.o.c}
\end{cases}\\
F_{X} \left(x \right)&=& \begin{cases}
\frac{1}{N} & x= \ldots -2, -1, 0\\
\sum_{t=- \infty }^{x} f_{X} \left(t \right)= \sum_{t=-\infty}^{x} \frac{1}{N}=\frac{1}{N} \sum_{t=-\infty}^{x} = \frac{1}{N} \left(x-0 \right) = \frac{x}{N} & x= 1, \ldots ,N\\
1 & x= N+1, N+2, \ldots
\end{cases}\\
\left(F_{X} \left(z \right) \right)^{2} &=& \left( \frac{z}{N} \right)^ {2} = \frac{z^2}{N^2}
\end{eqnarray*}

\item $X,Y \sim N \left( \sigma, \sigma ^{2} \right) $ ind. Hallar la densidad de $T= \frac{Y}{X}$

\begin{eqnarray*}
f_{X} \left(x \right)=  \frac{1}{\sqrt{2\pi \sigma^2}} \textrm{Exp} \left\{ - \frac{x^2}{2\sigma^2} \right\}, x\in \rea\\
f_{Y} \left(x \right)=  \frac{1}{\sqrt{2\pi \sigma^2}} \textrm{Exp} \left\{ - \frac{y^2}{2\sigma^2} \right\}, y\in \rea\\
\Rightarrow  f_{XY} \left(x \right) = \frac{1}{\sqrt{2\pi \sigma^2}} \textrm{Exp} \left\{ - \frac{\left( x^2+y^2\right) }{2\sigma^2} \right\} & x\in \rea\\
\end{eqnarray*}
\begin{eqnarray*}
f_{T} \left(t \right) &=& \int_{-\infty} ^{\infty } \mid x \mid \frac{1}{\sqrt{2\pi \sigma^2}} e^{- \frac{x^2}{2\sigma^2} } \frac{1}{\sqrt{2\pi \sigma^2}} e^{- \frac{t^2 x^2}{2\sigma^2} } \, dx\\
&=& \frac{1}{2\pi \sigma^2} \int_{- \infty}^{\infty} \mid x  \mid e^{- \frac{x^2 \left(1+t^2 \right) }{2\sigma^2} } \, dx\\
&=& \frac{1}{2\pi \sigma^2}\left[ \int_{- \infty}^{0} \left( -x  \right) e^{- \frac{x^2 \left(1+t^2 \right) }{2\sigma^2} } \, dx 
+  \frac{1}{2\pi \sigma^2} \int_{0}^{\infty}  x   e^{- \frac{x^2 \left(1+t^2 \right) }{2\sigma^2} } \, dx\right]\\
&=& - \frac{1}{2\pi \sigma^2} \left[ \dfrac{-2\sigma^2}{2 \left( 1+t^2 \right) } e^{- \frac{1}{2\sigma^2}x^2 \left(1+t^2 \right) } \right]_{-\infty} ^{0} + 
\frac{1}{2\pi \sigma^2} \left[ \dfrac{-2\sigma^2}{2 \left( 1+t^2 \right) } e^{- \frac{1}{2\sigma^2}x^2 \left(1+t^2 \right) } \right]_{0} ^{\infty}\\
&=& - \frac{1}{2\pi} \left[ - \frac{1}{1+t^2} \right] + \frac{1}{2\pi} \left[\frac{1}{ 1+t^2} \right]\\
&=& \frac{1}{\pi \left( 1+z^2 \right) }\\
Z &\sim& Cauchy\\
-\infty &<& z < \infty
\end{eqnarray*}

\end{enumerate}
\item
\begin{enumerate}
\item $X_{1}, X_{2}, X_{3}\sim N \left(\sigma, \sigma^2 \right) $ ind. Entonces $Y= \left( X_{1}^2 + X_{2}^{2}+ X_{3}^2 \right)^{ \frac{1}{2} } $ tiene la densidad de Maxwell. Encon trarla.

\begin{eqnarray*}
f_{Xi} \left(x_{i} \right) &=& \frac{1}{\sqrt{2\pi}} Exp \left\{ - \frac{x^2}{2\sigma^2} \right\}\\
\Rightarrow f_{X_{1}X_{2}X_{3} } \left(x_{1}, x_{2}, x_{3} \right) &=& \frac{1}{\left(2\pi \right)^{ \frac{3}{2} } } exp \left\{- \frac{\sum x_{i}^2}{2\sigma^2} \right\}\\
f_{Y} \left(y \right)&=& \prob \left[Y \leq y \right]= \prob \left[\left( X_{1}^2 + X_{2}^2+ X_{3}^2 \right)^{ \frac{1}{2}} \leq y  \right]\\
&=& \prob \left[ X_{1}^2 + X_{2}^2+ X_{3}^2 \leq y^2  \right]\\
\prob \left[ X_{1}^2 + X_{2}^2+ X_{3}^2 \leq y^2 \right] &=& \int \int \int_{s} \frac{1}{\left( 2\pi \right)^{ \frac{3}{2}}} exp \left\{ - \frac{ x_{1}^2 + x_{2}^2+ x_{3}^2}{2\sigma^2} \right\}\, ds\\
\begin{array}{ll}
\rho \leq y & x_{1}= \rho \cos \theta \sin \phi\\
0 < \theta < 2\pi & x_{2}= \rho \cos \theta \sin \phi\\
0< \rho < \pi & x_{3} = \rho \cos \phi
\end{array}\\
x_{1}^2 + x_{2}^2+ x_{3}^2 &=& \rho^2 \left[ \cos^{2} \theta \sin{2} \phi + \sin^{2}\theta \sin^2 \phi + \cos^2 \phi \right]\\
&=& \rho^2 \left[ \sin ^{2} \phi + \cos^{2} \phi \right]= \rho^2\\
&=& \frac{1}{ \left(2\pi\sigma^2 \right)^{ \frac{3}{2} } } \int_{0}^{y} \int_{0}^{2\pi} \int_{0}^{ \pi} e^{- \frac{\rho^2}{2\sigma^2} } \rho^2 \sin \varphi \,  d\varphi\, d\theta \, d\rho \\
F_{Y} \left(y \right) &=& 4\pi \left( \frac{1}{\sqrt{2\pi} \sigma} \right)^3 \int _{0}^{y} e^{- \frac{\rho^2}{2\sigma^2} }  \, d \rho\\
f_{Y} \left(y \right) &=& 4\pi \frac{1}{\left(2\pi \sigma^2 \right)^{ \frac{3}{2} } } y^2 e^{- \frac{y^2}{ 2\sigma^2} }\\
&=& 2^{\frac{1}{2} } \pi^{-\frac{1}{2}} \sigma^{-3} y^{2}e^{- \frac{y^2}{ 2\sigma^2}}\\
f_{Y} \left( y\right) &=& \begin{cases}
\sqrt{ \frac{2}{\pi} } \frac{1}{\sigma^3} y^2 e^{- \frac{\alpha y^2}{ 2\sigma^2} } & y>0\\
0 & \textrm{e.o.c}
 \end{cases}\\
 \alpha &=& \frac{1}{\sigma^2}\\
 \Rightarrow f_{Y} \left(y \right)&=& \begin{cases} 
 \sqrt{ \frac{2}{\pi} } \alpha^{ \frac{3}{2}} y^2 e^{- \frac{\alpha y^2}{ 2\sigma^2} } & y>0\\
0 & \textrm{e.o.c}
 \end{cases}\\
 Y &\sim & Maxwell\\
 \mid J  \mid &=& \mid - \rho^2 \sin \varphi \mid = \rho^2 \sin \varphi
\end{eqnarray*}
 \item  Sean $R$ y $\Theta$ v.a ind. t.q $\Theta \sim U \left( -\pi , \pi \right)$ y
 $ \begin{cases} 
 \frac{1}{\sigma^2}r e^{- \frac{r^2}{2\sigma^2} } & r\geq 0\\
 0 & \textrm{e.o.c}
 \end{cases} 
 $
 Hallar la densidad conjunta de $X = R \cos \Theta$, $Y=R \sin \Theta$ ? Son ind?.
 
\begin{eqnarray*} 
f_{\Theta } \left( \theta \right) &=&
\begin{cases}
\frac{1}{2\pi} & -\pi < \theta < \pi\\
0 & \textrm{e.o.c}
\end{cases}\\
f_{R,\theta} &=& 
\begin{cases}
\frac{r}{2\pi \sigma^2}e^{-\frac{r^2}{2\sigma 2} } & r \geq 0 , -\pi < \theta < \pi\\
0 & \textrm{e.o.c}
\end{cases}\\
\begin{array}{cc}
\sigma_{1} \left( t \right)= \left(0,t \right) & t\geq 0\\
\sigma_{2} \left( t \right)= \left(t,0 \right) & 0\leq t < \pi\\
\sigma_{3} \left( t \right)= \left(\pi,0 \right) & t\geq 0\\
\end{array}\\
T \left(x,y \right)&=& \left( R \cos \Theta , R \sin \Theta \right)\\
T \left( 0,t \right)&=& \left( t,0 \right)\\
T \left( t, 0 \right)&=& \left( 0,0 \right)\\
T \left( \pi,0 \right)&=& \left( 0,0 \right)\\
x^2 + y^2 &=& R^2 \cos^2 \Theta + R^2 \sin^{2} \Theta = R^2
 \Rightarrow R= \sqrt{x^2 + y^2}\\
 \frac{Y}{X} &=& \frac{R \sin \Theta }{R \cos \Theta } =  \tan  \Theta\\
 \Theta &=& \arctan \left( \frac{y}{x} \right) \\
 J&=& \left|
 \begin{array}{cc}
 \dfrac{\left( -1 \right) \frac{y_{2}}{y_{1} 2} }{1 + \left( \frac{y_{2}}{y_{1}} \right)^2 } & \dfrac{\frac{1}{y_{1}} }{1 +\left( \frac{y_{2}}{y_{1}} \right)^{2} } \\
 \dfrac{y_{1}}{\left( y_{1}^{y}+ y_{2}^{2} \right)^{ \frac{1}{2} } } & \dfrac{y_{2}}{\left( y_{1}^2 + y_{2}^{2} \right)^{ \frac{1}{2} } }
 \end{array} 
  \right|\\
  &=& \dfrac{-1}{\sqrt{y_{1}^2 + y_{2}^2}}\\
  \mid J  \mid &=&  \dfrac{1}{\sqrt{y_{1}^2 + y_{2}^2}}\\
  f_{X,Y} \left( x,y \right)&=& \frac{1}{\sqrt{ x^2 + y^2}} f_{\Theta, R } \left( \arctan \left(\frac{y}{x} \right) , \sqrt{ x^2 + y^2} \right)\\
  \Rightarrow f_{X,Y} \left( x,y \right) &=& \begin{cases}
  \frac{1}{2\pi \sigma^2 } Exp \left\{ - \frac{x^2+y^2}{2\sigma^ 2} \right\} & x\in \rea, y \in \rea\\
  0 & \textrm{e.o.c}
  \end{cases}\\
  f_{X} \left(x \right) &=& \int_{-\infty}^{\infty} \frac{1}{2\pi \sigma^2} Exp \left\{ - \frac{x^2+y^2}{2\sigma^ 2} \right\} \,dy\\
  &=&  \frac{1}{\left(2\pi \sigma^2 \right)^{ \frac{1}{2} } }  Exp \left\{ - \frac{x^2}{2\sigma^ 2}\right\} \int_{- \infty}^{\infty} \frac{1}{  \left( 2\pi \sigma^2 \right)^{ \frac{1}{2} } } e^{- \frac{y^2}{2\sigma^2} } \, dy\\
  X &\sim& N \left( \sigma, \sigma^2\right)\\
  y &\sim& N \left( \sigma, \sigma^2 \right) ind.
\end{eqnarray*}
\end{enumerate}
\end{enumerate}

\section{ Evaluaci\'on de Probabilidad }


\begin{enumerate}
\item Sea $X \sim U \left(0,2 \right) $. Hallar la densidad de $Y= X^2 - 3X + 1$
\begin{eqnarray*}
f_{Y}&=& \left(y \right)= \frac{dF_{Y}\left(y \right) }{dy}\\
F_{Y} \left(y \right)&=& \prob \left[ Y \leq y \right] = \prob \left[ X^2 - 3X + 1 \leq y \right] = \prob \left[ \left( X - \frac{3}{2} \right)^2 + 1 \leq y + \frac{9}{4} \right]\\
&=& \prob \left[ \left( X - \frac{3}{2} \right)^2  \leq y + \frac{9}{4}-1 \right] = \prob \left[ X- \frac{3}{2} \leq \sqrt{y+\frac{5}{4} }\right]\\
&=& \prob \left[ X \leq \sqrt{y+\frac{5}{4} } - \frac{3}{2} \right]= F_{X} \left( \left( y+\frac{5}{4} \right)^{ \frac{1}{2} } + \frac{3}{2} \right)\\
&=& f_{Y} \left(y \right)= f_{X} \left( \left(  y + \frac{5}{4}\right)^{ \frac{1}{2}}+ \frac{3}{2}    \right) \left(\frac{1}{2} \left( y + \frac{5}{4} \right)^{- \frac{1}{2} } \right)\\
f_{Y} \left( y \right) &=& \begin{cases} \frac{1}{2}  \left(\frac{1}{2} \left( y + \frac{5}{4} \right)^{- \frac{1}{2} } \right)= \frac{1}{4} \left( y + \frac{5}{4} \right)^{-\frac{1}{2} } & \left(-1,1 \right)\\
0 & \textrm{e.o.c}
\end{cases}\\
\int_{-1}^{1} \frac{1}{4} \left( y + \frac{5}{4} \right)^{-\frac{1}{2} } \, dy &=& \frac{1}{4} \left(2 \left( y + \frac{5}{4} \right)^{ \frac{1}{2} } \right)_{-1}^{1}\\
&=& \left. \frac{1}{2} \left( y + \frac{5}{4} \right)^{\frac{1}{2}} \right|_{-1}^{1} = \frac{1}{2} \left[ \left( 1 + \frac{5}{4} \right)^{ \frac{1}{2} } - \left(-1 + \frac{5}{4} \right)^{ \frac{1}{2} } \right]\\
&=& \frac{1}{2} \left( \left(\frac{9}{4} \right)^{ \frac{1}{2}} - \left(\frac{1}{4} \right)^{ \frac{1}{2}} \right) = \frac{1}{2} \left( \frac{3}{2} - \frac{1}{2} \right) = \frac{1}{2}
\end{eqnarray*}

\item $X_{1}, X_{2} \sim U \left(0,1 \right)$ ind. $X_{1}= Y_{1} \cos \left(Y_{2} \right)$, $X_{2} = Y_{1} \sin \left(Y_{2} \right)$ Hallar la densidad conjunta de $Y_{1} $ y $Y_{2}$ y averiguar independencia
\begin{eqnarray*}
f_{X_{1}X_{2}} \left(x_{1} x_{2} \right) &=&  
\begin{cases}
1 & x_{1} \in  \left(0,1 \right), x_{2} \in \left(0,1 \right)\\
0 & \textrm{e.o.c}
\end{cases}\\
x_{1}^2+ x_{2}^2 &=& y_{1}^2 \cos^2 \left(y_{2} \right) + y_{1}^2 \sin ^2 \left(y_{2} \right)\\
&=& y_{1}^2 + y_{1}^2 = 2 y_{1}^2\\
\Rightarrow y_{1} &=& \sqrt{\frac{1}{2} \left( x_{1}^2 + x_{2}^2 \right) }\\
\frac{X_{1}}{X_{2}} &=& \frac{Y_{1} \cos Y_{2}}{Y_{1} \sin Y_{2}} = \tan Y_{2} \Rightarrow Y_{2}= \arctan \left( \frac{X_{1}}{X_{2}} \right)\\
J &=& \left| \begin{array}{cc}
\cos \left( Y_{2} \right) & -Y_{1} \sin \left( Y_{2} \right) \\
\sin \left( Y_{2} \right) & Y_{1} \cos \left( Y_{2} \right)
\end{array} \right| = Y_{1} \cos^2 Y_{2} + Y_{1} \sin^2 Y_{2}\\
&=& Y_{1} \Rightarrow \mid J \mid = Y_{1}\\
f_{Y_{1}Y_{2}} \left( y_{1}, y_{2} \right) &=& \mid J \mid f_{X_{1}X_{2}} \left( \sqrt{ \frac{1}{2} \left(x_{1}^2 + x_{2}^2 \right)}, \arctan  \left( \frac{X_{1}}{X_{2}} \right) \right)\\
&=& \begin{cases}
Y_{1} & y_{1}y_{2}\in \rea\\
0 & \textrm{e.o.c}
\end{cases}\\
\left.
\begin{array}{cccc}
\sigma_{1} \left( t \right) = \left(t,1 \right)\\
\sigma_{2} \left( t \right) = \left(0,t \right)\\
\sigma_{3} \left( t \right) = \left(t,0 \right)\\
\sigma_{4} \left( t \right) = \left(1,t \right)
\end{array}
\right\} 0 &\leq& t \leq 1\\
Y_{1} &=& X_{1} \cos X_{2}, Y_{2} = X_{1} \sin \left( X_{2} \right)\\
T \left(t,1 \right) &=& \left( t \cos \left(1 \right), t \sin \left(1 \right) \right) = \left( t \left(0.5403 \right), t \left( 0.8414 \right) \right)\\
T \left(0,t \right) &=& \left(0,0 \right)\\
T \left(t,0 \right) &=& \left( t \cos \left(0 \right), t \sin \left(0 \right) \right)= \left( t,0 \right)\\
T \left(1,t \right) &=& \left( \cos \left(t \right), \sin \left(t \right) \right)\\
y_{1} &=& t \cos \left(1 \right) \Rightarrow x_{1} \cos^{-1} \left(1 \right)= t\\
y_{2} &=& t \sin \left(1 \right) \Rightarrow x_{2} \sin^{-1} \left(1 \right)= t\\
\frac{y_{1}}{\cos \left(1 \right) } &=& t, \frac{y_{2}}{\sin \left(1 \right) }= t \Rightarrow
\frac{y_{1}}{\cos \left(1 \right) } =  \frac{y_{2}}{\sin \left(1 \right) } \\
y_{2}&=& y_{1} \tan \left(1 \right)\\
y_{2}&=&t \Rightarrow
t= \arccos \left(y_{1} \right) \Rightarrow \arccos \left(y_{1} \right) = \arcsin \left(y_{2} \right)\\
y_{2} &=& \sin t \Rightarrow t= \arcsin \left( y_{2} \right) \Rightarrow \sin \left( \arccos \left(y_{1} \right) \right) = y_{2}
\end{eqnarray*}
\begin{eqnarray*}
f_{Y_{1}} &=& \begin{cases}
\int_{0} ^{y_{1} \tan \left(1 \right) } f_{Y_{1}Y_{2}} \left( y_{1}y_{2} \right) \, dy_{2} & y_{1} \in \left(0,0.5403 \right)\\
\int_{0}^{\sin \left(\arccos \left(y_{1} \right) \right) }  f_{Y_{1}Y_{2}} \left( y_{1}y_{2} \right) \, dy_{2} & y_{1} \in \left(0.5403, 1 \right)\\
0 & \textrm{e.o.c}
\end{cases}\\ 
f_{Y_{2}} \left(y_{2} \right) &=&
\begin{cases}
\int_{\cos \left( \arcsin \left( y_{2} \right) \right) } ^{\frac{y_{2}}{\tan \left( 1\right) } }  f_{Y_{1}Y_{2}} \left( y_{1}y_{2} \right) \, dy_{1} & y_{2} \in \left( 0 , 0.8414 \right)\\
0 & \textrm{e.o.c/}
\end{cases}
\end{eqnarray*}
 \item $N$ es una v.a t.q $\prob          \left[ N=1 \right]= \frac{1}{4},  \prob \left[ N = 2\right]= \frac{1}{3},  \prob \left[ N=3 \right]= \frac{5}{12}$ \\
 $X_{i} \sim Poisson \left(1 \right), i=1,2, \ldots$\\
 Hallar la densidad de la suma aleatoria $SN$
 
 \begin{eqnarray*}
 \prob \left[ S_{N} = x \right] &=& \prob \left[ X_{1}+ X_{2}+ \ldots + X_{N} = x\right]\\
 &=& \sum_{n=1}^{3} \prob \left[ X_{1}+ X_{2}+ \ldots + X_{N} = x \mid N=n \right] \prob \left[ N=n \right]\\
 &=&\sum_{n=1}^{3} \dfrac{X_{1}+ X_{2}+ \ldots + X_{N} = x , N=n}{\prob \left[N=n \right] } \prob \left[N=n \right]\\
 Sol \sim Poisson \left(n \lambda \right)\\
 &=& \sum_{n=1}^{3} \prob \left[S_{n}= x \right] \prob \left[ N=n\right]= \prob \left[ S_{1}=x \right] \prob \left[N=1 \right]+ \prob \left[S_{2}= x \right] \prob \left[ N=2 \right]\\ &+& \prob \left[S_{3}= x \right]\prob \left[N=3 \right]\\
 S_{1} &=& X_{1}\\
 &=&\prob \left[S_{1} = x \right] \prob \left[ N=1\right]+ \prob \left[ S_{2}= x \right] \prob \left[N=2 \right] + \prob  \left[S_{3}= x \right]\prob \left[ N=3 \right]\\
 &=& \left( \frac{e^{-1}}{x!}\right)\frac{1}{4}+ \left( \frac{e^{-2}}{x!} \right) \frac{1}{3} + \left( \frac{e^{-3}}{x!} \right) \frac{5}{12}
 \end{eqnarray*}
\end{enumerate}

Hay tres tipos de transformaciones b\'asicas.

\begin{enumerate}
\item Sea $X$ una variable aleatoria con funcion de densidad $f_{X}\left(x\right)$ y una transformaci\'on $T$, tal que $T:\rea\rightarrow\rea$. Sea $Y=g\left(x\right)$. Encontrar $f_{Y}\left(y\right)$

\item Sea $X$ una variable aleatoria con funcion de densidad $f_{XY}\left(x,y\right)$ y una transformaci\'on $T$, tal que $T:\rea^2\rightarrow\rea$. Sea $Y=g\left(x,y\right)$. Encontrar $f_{Z}\left(z\right)$

\item Sea $X_{1}, X_{2}, X_{3},..., X_{n}$ una variable aleatoria con funcion de densidad $f_{X_{1}},...,f_{X_{n}} \left(x_{1},...,x_{n}\right)$ y una transformaci\'on $T$, tal que $T:\rea^n\rightarrow\rea^n$. Sea 

\begin{eqnarray*}
Y_{1}&=&g_{1}\left(x_{1}\ldots x_{n}\right) \\ 
Y_{2}&=&g_{2}\left(x_{1}\ldots x_{n}\right)\\ 
\vdots \\ 
Y_{n}&=&g_{n}\left(x_{1}\ldots x_{n}\right)
\end{eqnarray*}


Encontrar $f_{Y_{1}\ldots Y_{n}}\left(y_{1} \ldots y_{n}\right)$

\end{enumerate}

\bigskip

\begin{Ejem}
\begin{itemize}
\item Sea $X \backsim U_{\left(0,1\right)}$ y $Y=\frac{1}{\lambda}\in \left( 1-X \right)$. 
Para $\lambda > 0$. Encontrar $f_{Y}\left(y\right)$

\textbf{Sol:}

\begin{eqnarray*}
X \sim U_{\left(0,1\right)}\longrightarrow f_{X}\left(x\right)= \left\{
\begin{array}{cc}
1 & 0< x <1 \\
0 & \textrm {e.o.c} 
\end{array}
\right.
\end{eqnarray*} 

\begin{eqnarray*}
f_{Y}\left( y \right) &=& \dfrac{\partial}{\partial y} F_{Y}\left(y \right)\\
F_{Y}\left( y \right) &=& \prob \left[ Y \leqslant y \right]  = \prob \left[-\dfrac{1}{\lambda}\ln \left( 1-X\right) \leqslant y \right] =\prob \left[ \ln \left( 1-X \right) \leqslant \lambda y \right]\\
&=&\prob \left[ \ln \left( 1-X \right) > - \lambda y \right] = \prob \left[ \left( 1-X \right) > e^ {- \lambda y}\right] \\
&=& \prob \left[-X > e^ {-\lambda y} - 1 \right] = \prob \left[X \leqslant 1- e^{- \lambda y}\right] = F_{X}\left(1-e^{-\lambda y}\right).
\end{eqnarray*}  

\begin{eqnarray*}
0<X<1 \Rightarrow -1< -X < 0 \Rightarrow 0 < 1-X < 1&\Rightarrow&
-\infty < \ln \left(1-X\right) < 0\\
\Rightarrow  -\infty < \dfrac{1}{\lambda} \ln \left(1-X\right) < 0 \Rightarrow  \infty > - \dfrac{1}{\lambda}  \ln \left(1-X\right) > 0&\Rightarrow&
0 < - \dfrac{1}{\lambda}  \ln \left(1-X\right) <\infty \Rightarrow 0< Y < \infty
\end{eqnarray*}

\begin{eqnarray*}
F_{Y}\left(y\right) = \left\{ 
\begin{array}{cc} 1-e^{-\lambda y} &  y \geq 0 \\
0 &  e.o.c 
\end{array}
\right.
\end{eqnarray*}

\begin{eqnarray*}
0 < y < \infty \\
0 < \lambda y < \infty \\
- \infty < -\lambda y < 0 \\
0 < e ^{-\lambda y} < 1 \\
-1 < e^{-\lambda y < 0}\\
0 < 1 - e^{-\lambda y} < 1\\
\end{eqnarray*}

\begin{eqnarray*}
\dfrac{\partial}{\partial Y} F_{y} \left( y \right) &=& -e^{-\lambda_{y}} \left(-\lambda \right) = -\lambda e^{-\lambda_{y}}\\
f_{Y}\left(y \right) &=& \left\{
\begin{array}{cc} 
\lambda e^{-\lambda y} &y \geq 0 \\
 0 &  \textrm{e.o.c}\\
 Y \sim exp \left( \lambda \right)
\end{array}
\right.
\end{eqnarray*} 
 
\smallskip

$$Y \sim exp \left( \lambda \right)$$
\end{itemize}
\end{Ejem}

 
  
\begin{Ejem}
\begin{itemize}
\item Sea $X \sim N\left(0,1 \right)$ y $Y= \sqrt{x}$. Encontrar $f_{Y}\left(y\right)$.
{Sol}
 
\bigskip
\begin{eqnarray*}
f_{Y}\left(y\right)&=&\dfrac{\partial}{\partial y}F_{Y}\left(y\right)\\
F_{Y}\left(y\right) &=&\prob \left[ Y \leq y\right] = \prob \left[ X^{1/2}\leq y\right]= \prob \left[ \left( X^{1/2} \right)^{2} \leq y^{2} \right] = \prob \left[ X \leq y^{2} \right] = F_{X}\left(y^{2} \right) 
\end{eqnarray*}

\begin{eqnarray*}
-\infty < X < \infty \Rightarrow 0 < X < \infty \Rightarrow  0 < \sqrt{X} < \infty \Rightarrow 0 < Y < \infty
\end{eqnarray*}

\begin{eqnarray*}
f_{X}\left( x \right) &=& \dfrac{1}{\sqrt{2 \pi}} exp \bigg\{ \dfrac{-x^{2}}{2} \bigg\} \\
f_{Y}\left(y\right)&=& f_{X}\left(y^{2}\right)\left(2y\right)
\end{eqnarray*}

 \begin{eqnarray*}  
 f_{Y}\left(y\right)= \left\{
 \begin{array}{cc} 
 \dfrac{1}{\sqrt{2 \pi}} exp\bigg\{ \dfrac{-y^4}{2}\bigg\} 2y & 0<y< \infty \\ 
 0 & \textrm {e.o.c} 
 \end{array}
 \right.
 \end{eqnarray*}


\end{itemize}
\end{Ejem}


\begin{Ejem}

Sea $X \sim U_{\left(0,1 \right)}$. Con $Y= g\left( X \right)$. Encontrar $g$ t.q $Y\sim U_{\left( 0,b\right)}$.\\

\begin{eqnarray*}  
 f_{Y}\left(y\right)= \left\{
 \begin{array}{cc} 
 0 & x\leq 0\\
 x & 0<x<1\\
 1 & x\geq 1     
 \end{array}
 \right.
 \end{eqnarray*}
 

\end{Ejem}

\begin{Ejem}

Sea $X \sim Exp  \left( \lambda \right)$. Encontrar la densidad de $Y = \left[ 3X \right] $

\begin{eqnarray*}
X \sim Exp \left( \lambda \right) \Rightarrow f_{X}\left( x \right) = F_{Y}\left(y\right) = \left\{\begin{array}{ll} \lambda  e^{-\lambda x} & x > 0 \\ 
0 & eoc 
\end{array}
\right.
\end{eqnarray*}

\begin{eqnarray*}
F_{Y}\left( y \right) &=& \prob \left[ Y= 7y \right] = \prob \left[ \left[ 3X\right] = y \right] = \prob \left[ y \leq X < y+1 \right]\\
&=& \prob \left[ \dfrac{y}{3} \leq \right] X \leq \dfrac{y+1}{3}] = \int_{y/3}^{(y+1)/3 }   f_{X} \left(x \right) \, dx \\
&=& \int_{y/3}^{\frac{y+1}{3} } \lambda e^{-\lambda x}, dx = \lambda \int_{y/3}^{(y+1)/3 } e^{-\lambda x} \, dx = \lambda \Bigg[ \dfrac{-e^{-\lambda x}}{\lambda} \Bigg]_{y/3}^{(y+1)/3} \\
&=& -e^{-\lambda \left((y+1)/3 \right)} + e^{-\lambda \left(\frac{y}{3} \right)} = -e^{-\lambda\left(\frac{y}{3} \right)} e^{-\lambda\left(\frac{1}{3} \right)}+ e^{-\lambda\left(\frac{y}{3}\right)}\\
&=& e^{-\lambda \left(\frac{y}{3} \right)}\left[ 1-e^{\frac{-\lambda}{3}}\right]
\end{eqnarray*}

\end{Ejem}

\begin{Ejem}

Sea $X \sim Gamma \left( 2,2 \right)$. Encontrar la densidad de $Y=\dfrac{X}{1+X}$\\ 

\begin{eqnarray*}
X \sim Gamma \left( 2,2 \right) \Rightarrow f_{X}\left( x \right)= 
\begin{cases}
\begin{array}{lc}
\dfrac{\beta^{\alpha}}{\mu\left(\alpha\right)}x^{\alpha -1}e^{x\beta} & x \geq 0 \\
0 & \textrm{ e.o.c}
\end{array}
\end{cases}
\end{eqnarray*}

\begin{eqnarray*}
\Rightarrow
\begin{cases}
\begin{array}{lc}
4xe^{-2x} & x \geq 0 \\
0 & \textrm {e.o.c}
\end{array}
\end{cases}
\end{eqnarray*}

\begin{eqnarray*}
0\leq \infty \Rightarrow 1 \leq x+1 < \infty \Rightarrow 0 \leq \frac{1}{X+1}\leq 1 \Rightarrow 0\leq\frac{X}{X+1}\leq\leq \infty
\end{eqnarray*}
\begin{eqnarray*}
F_{Y}\left(y\right)&=& \prob \left[Y\leq y\right]=\prob \left[\frac{X}{1+X}\leq y\right]= \left[X\leq y\left(1+X\right)\right]\\
&=& \prob\left[X\leq y+yX\right]=\prob\left[X-yX\leq y\right]=\prob\left[X\left(1-y\right)\leq y\right]\\
&=& \left[X\leq \frac{y}{1-y}\right]= F_{X}\left(\dfrac{y}{1-y}\right)
\end{eqnarray*}

\begin{eqnarray*}
f_{Y}\left(y\right)&=&f_{X}\left(\frac{y}{1-y}\right)\dfrac{\left[\left(1-y\right)-y\left(-1\right)\right]}{\left(1-y\right)^{2}}\\  
&=&f_{X}\left(\frac{y}{1-y}\right)=\frac{4\left(\frac{y}{1-y}\right)e^{-2\left(\frac{y}{1-y}\right)}}{\left(1-y\right)^{2}}\\
&=&\begin{cases}
\begin{array}{cc}
\frac{4\left(\frac{y}{1-y}\right)e^{-2\left(\frac{y}{1-y}\right)}}{\left(1-y\right)^{3}} & 0<y<1 \\
0 & \textrm {e.o.c}
\end{array}
\end{cases}
\end{eqnarray*}

\begin{eqnarray*}
\int_{0}^{1}\frac{4\left(\frac{y}{1-y}\right)e^{-2\left(\frac{y}{1-y}\right)}}{\left(1-y\right)^{3}}\, dy &=& 4\int_{0}^{1}\frac{y}{\left(1-y\right)^3}e^{-2\left(\frac{y}{1-y}\right)}\, dy =\\
&=& \left(\frac{y}{1-y}\right)\left(-\frac{1}{2}e^{-2\left(\frac{y}{1-y}\right)}\right)\left. \right|{0}^{1} + 2 \int_{0}^{1} e^{-2\left(\frac{y}{1-y}\right)}\left(\frac{1}{\left(1-y\right)^{2}}\right) dy
\end{eqnarray*}
\begin{eqnarray*}
\begin{array}{cc}
u=\frac{y}{1-y}\Rightarrow du=\frac{1}{\left(1-y\right)^{2}} dy\\
dv= \frac{y}{\left(1-y\right)^3}e^{-2\left(\frac{y}{1-y}\right)} dy \Rightarrow v=-\frac{1}{2}e^{-2\left(\frac{y}{1-y}\right)}\\
\end{array}
\end{eqnarray*}


\begin{eqnarray*}
u=\frac{1}{\left(1-y\right)^{2}} \Rightarrow 2\left(1-y\right)dy\\
dv= e^{-2\left(\frac{y}{1-y}\right)} dy \Rightarrow v= \frac{\left(1-y\right)^2 e^{-2\left(\frac{y}{1-y}\right)}}{2}\\
-2\left(\frac{y}{1-y}\right)=\frac{\left(1-y\right)\left(-2\right)-\left(-2y\right)\left(-1\right)}{\left(1-y\right)^2}= \frac{2}{\left(1-y\right)^2}\\
\end{eqnarray*}
\end{Ejem}

\begin{Ejem}
Sea $X \sim $ Poisson $\left(\lambda\right)$. Hallar la densidad de $Y=4X+3$

\begin{eqnarray*}
X\sim \textrm{Poisson}\left(\lambda\right) \Rightarrow f_{X}\left(x\right)=
\begin{cases}
\begin{array}{lc}
\frac{e^{-\lambda}\lambda^{x}}{x!} & x=0,1,2\\
0 & \textrm{ e.o.c}
\end{array}
\end{cases}
\end{eqnarray*}

\begin{eqnarray*}
f_{Y}\left(y\right)&=& \prob\left[Y=y\right]= \prob\left[4X+3=y\right]=\prob\left[4X0y-3\right]\\
&=&\prob\left[X=\frac{y-3}{4}\right]=f_{X}\left(\frac{y-3}{4}\right)\\
f_{Y}\left(y\right)=f_{X}\left(\frac{y-3}{4}\right)\left(\dfrac{1}{4}\right) \\
&=& 
\begin{cases}
\begin{array}{lc}
\dfrac{e^{-\lambda} \lambda^{\left(\frac{y-3}{4}\right)}}{4\left(\frac{y-3}{4}\right)!} & y=3,7,11,15\\
0 & \textrm{ e.o.c}
\end{array}
\end{cases}
\end{eqnarray*}

\begin{eqnarray*}
x=0,1,2,3,\cdots \\
4X=0,4,8,12, \cdots \\
4X+303,7,11,15, \cdots \\
Y=3,7,11,15, \cdots \\
\end{eqnarray*}
\end{Ejem}

\begin{Ejem} Sea $X \sim  U_{\left(0,1\right)}$. Hallar $g$ t.q $Y\sim U_{\left(a,t\right) }$. $Y=g\left(X\right)$.

\begin{eqnarray*}
X\sim U_{\left(0,1\right)} \Rightarrow f_{X}\left(x\right) =
\begin{cases}
\begin{array}{lc}
1  & 0\leq x \leq 1 \\
0 & \textrm{ e.o.c}
\end{array}
\end{cases}
\end{eqnarray*}

\begin{eqnarray*}
\Rightarrow  F_{X}\left(x\right)=
\begin{cases}
\begin{array}{lc}
1  & x<0 \\
x & 0\leq x \leq 1 \\
1 & x \geq 1
\end{array}
\end{cases}
\end{eqnarray*}

\begin{eqnarray*}
Y\sim U_{\left(a,b\right)}\Rightarrow f_{Y}\left(y\right)=
\begin{cases}
\begin{array}{lc}
\frac{1}{b-a}  & 0\leq y \leq 1 \\
0 & \textrm{ e.o.c}
\end{array}
\end{cases}
\Rightarrow \int_{a}^{y}\frac{1}{b-a} \, dv = \dfrac{y-a}{b-a} 
\end{eqnarray*}

\begin{eqnarray*}
\Rightarrow  F_{Y}\left(y\right)=
\begin{cases}
\begin{array}{lc}
1  & y<0 \\
\frac{y-a}{b-a} & a\leq y \leq b \\
1 & y > b
\end{array}
\end{cases}
\end{eqnarray*}

Sea $y \in \left[a,b\right]$, sup. $g$ creciente:
\begin{eqnarray*}
F_{Y}= \left(y\right)&=&\prob \left[Y\leq y\right]= \prob\left[g\left(X\right)\leq y\right]= \prob\left[X\leq g^{-1}\left(y\right)\right]\\
F_{X}\left(g^{-1}\left(y\right)\right)&=& g^{-1}\left(y\right)\\
g:\left(0,1\right) \rightarrow \left(a,b\right)\\
g^{-1}:\left(a,b\right) \rightarrow \left(0,1\right)\\
\textrm {Ya que}  F_{X}\left(x\right) &=& x, x \in \left[0,1\right]\\
\therefore g^{-1}\left(y\right) &=& \frac{y-a}{b-a} \Rightarrow y = y\left(\frac{y-a}{b-a}\right)\\
\textrm{Sea}  t &=&\frac{y-a}{b-a}\Rightarrow y-a \Rightarrow y=a+t\left(b-a\right)\\
\therefore Y &=& g\left(X\right) = a+ \left(b-a\right)X
\end{eqnarray*}
\end{Ejem}



\begin{Ejem}[\textbf{Importante}] Un insecto deposita un n\'umero grande de huevos, el n\'umero de huevos depositado es una v.a que frecuentemente se asocia una distribuci\'on Poisson $\left(\lambda\right)$. La supervivencia de un cierto huevo tiene probabilidad $p$. Encontrar el n\'umero promedio de huevos sobrevivientes.

\smallskip

 $Y \equiv $ N\'umero de huevos depositados $Y \sim Poi\left(\lambda\right)$
 
 \smallskip
 
 $X \equiv $ N\'umero de huevos sobrevivientes $X|Y \sim \beta \left(Y,p\right)$
 
\smallskip

$\prob\left[X=x\right] \equiv$ Esperanza de $X$
 
 \begin{eqnarray*}
\prob \left[X=x\right]&=& \sum_{y=0}^{\infty}\prob \left[X=x   Y=y \right]\\ 
&=& \sum_{y=0}^{\infty}\prob \left[X=x | Y=y \right]\prob \left[Y=y \right]\\ 
&=&\sum_{y=x}^{\infty} \left[ \left( 
\begin{array}{cc}
y \\ x
\end{array} \right) p^{x}\left(1-p\right)^{y-x}\right]\left[\frac{\lambda^{y}e^{-\lambda}}{y!}\right]\\
 &=& \sum_{y=x}^{\infty}\left( \frac{y!}{\left(y-x\right)! x!}p^{x}\left(1-p\right)^{y-x}\right)\left(\frac{\lambda^{y}e^{-\lambda}}{y!}\right)\\
 &=& e^{-\lambda}p^{x}\sum_{y=x}^{\infty}\left(1-p\right)^{y-x}\lambda^{y} /\! \left(y-x\right)!x!\\
 &=& \frac{e^{-\lambda}\left(\lambda p\right)^{x}}{x!} \sum_{y=x}^{\infty} \left[\left(1-p\right)\lambda\right]^{y-x}/\! \left(y-x\right)!\\
z &=& y-x\\
&=&  \frac{e^{-\lambda}\left(\lambda p\right)^{x}}{x!} \sum_{y=x}^{\infty} \left[\left(1-p\right)\lambda\right]^{z}/\! z! \\
&=&  \frac{e^{-\lambda}\left(\lambda p\right)^{x}}{x!} \sum_{y=0}^{\infty} \left[\left(1-p\right)\lambda\right]^{t}/\! t! \\
&=&  \frac{e^{-\lambda}\left(\lambda p\right)^{x}}{x!} \left(1 + \left(1-p\right)\lambda + \frac{\left(\left(1-p\right)\lambda\right)^{2}}{2!}+ \cdots \right) \\
&=& \begin{array}{cc} 
\frac{e^{-\lambda}\left(\lambda p\right)^{x}}{x!} e ^{\left(1-p\right)\lambda} & \textrm {Series de potencias} 
\end{array}\\
\begin{array}{cc} 
=\frac{e^{-\lambda p}\left(\lambda p\right)^{x}}{x!}   & \therefore \prob\left[X=x\right] = \frac{e^{-\lambda p}\left(\lambda p\right)^{x}}{x!}\\
X \sim Poi \left(\lambda p \right) & \therefore E\left[X\right] = \lambda p
\end{array}
 \end{eqnarray*}
\end{Ejem}

\begin{Ejem}
Sea $X\sim U_{0,1}$ y $Y \sim$ independientes.  $Z=X+Y$


\begin{eqnarray*}
X\sim U_{0,1} \Rightarrow f_{X}\left( x \right)= 
\begin{cases}
1 & 0\leq x \leq 1\\
0 & e.o.c
\end{cases}
\end{eqnarray*}

\begin{eqnarray*}
Y\sim U_{0,2} \Rightarrow f_{Y}\left( y \right)= 
\begin{cases}
\frac{1}{2} & 0\leq y \leq 2\\
0 & e.o.c
\end{cases}
\end{eqnarray*}

\begin{eqnarray*}
 f_{XY}\left( x,y \right)= f_{X}\left(x\right)f_{Y}\left(y\right)
\begin{cases}
\frac{1}{2} & 0\leq x \leq 1\\
& 0\leq y \leq 2\\
0 & e.o.c
\end{cases}
\end{eqnarray*}
\begin{eqnarray*}
F_{Z}\left(z\right)= \mathbb{P}\left[X+Y\leq z \right]= \mathbb{P}\left[Y \leq z-X \right]
\end{eqnarray*}
\end{Ejem}

\begin{Ejem}
 Sea $X \sim Exp \left( \lambda \right)$ y $Y \sim Exp \left( \lambda\right)$ independientes. $Z = min \{ x,y \}$. Hallar $f_{z} \left( z \right)$.

\begin{eqnarray*}
X\sim Exp \left(\lambda\right) \Rightarrow f_{X}\left( x \right)= 
\begin{cases}
\lambda e^{- \lambda x }  & x > 0\\
0 & e.o.c
\end{cases}
\end{eqnarray*}

\begin{eqnarray*}
Y\sim Exp \left(\lambda\right) \Rightarrow f_{Y}\left( y \right)= 
\begin{cases}
\lambda e^{- \lambda y }  & y > 0\\
0 & e.o.c
\end{cases}
\end{eqnarray*}

\begin{eqnarray*}
\Rightarrow f_{XY} \left( x,y\right) = f_{X}\left(x\right)f_{Y}\left(y\right)= \left(\lambda e^{- \lambda x} \right)\left(\lambda e ^{-\lambda y } \right)=
\begin{cases}
\lambda^{2} e^{- \lambda \left(x+y \right) }  & x > 0\\
& y>0\\
0 & e.o.c
\end{cases}
\end{eqnarray*}
Sea $z \geq 0$

\begin{eqnarray*}
F_{z} \left( z \right)&=& \mathbb{P} \left[Z \leq z \right]= \mathbb{P} \left[ min \{X,Y \} > Z \right] \\
&=& 1- \mathbb{P} \left[ X < z , Y > z\right]= 1- \mathbb{P} \left[ X < z\right]\mathbb{P}\left[ Y > z\right]\\
&=& 1-\mathbb{P}\left[ X > z\right]^{2} =  1 -\left( \mathbb{P} \left[ X \leq z\right] \right) ^{2}\\
&=& 1 - \left( 1- F_{X} \left( z \right) \right) ^{2}\\
F_{X} \left( x \right) &=& \int_{0}^{x} \lambda e^{-\lambda u} \, du = \lambda \int_{0}^{x}  e^{-\lambda u} \, du = x\left(-e ^{- \lambda u} /\! x\right)_{0}^{x}\\
&=& -e ^{- \lambda x} + 1 = 1 -e ^{- \lambda x}\\
&=& 1 - \left(1-1-e ^{- \lambda x} \right)^{2} = 1- e ^{- 2 \lambda z}  \Rightarrow z>0\\
\end{eqnarray*}
\end{Ejem}

\begin{center}
\textbf{Cambio de Variables}
\end{center}
\textbf{Teorema:} Sean $X_{1}, X_{2},\ldots,X _{n}$ v.a continuos con densidad.

\smallskip

$f_{x_{1}\ldots x_{n}}\left( x_{1} \ldots x_{n}\right)$   y
\begin{eqnarray*}
Y_{1} &=& g_{1} \left(x_{1}\ldots x_{n}  \right)\\
\vdots  \\
Y_{n} &=& g_{n} \left(x_{1}\ldots x_{n}  \right)
 \end{eqnarray*}
Si existe soluci?n ?nica 

\begin{eqnarray*}
x_{1} &=& h_{1} \left(y_{1}\ldots y_{n}  \right)\\
\vdots  \\
x_{n} &=& h_{n} \left(y_{1}\ldots y{n}  \right)
\end{eqnarray*}


 
\begin{eqnarray*}
J= \left| 
\begin{array}{cccc}
\frac{\partial h_{1}}{\partial y_{1}} & \frac{\partial h_{1}}{\partial y_{2}} & \ldots & \frac{\partial h_{1}}{\partial y_{n}}\\
\vdots & \vdots & \vdots & \vdots \\
\frac{\partial h_{n}}{\partial y_{1}} & \frac{\partial h_{2}}{\partial y_{2}} & \ldots & \frac{\partial h_{n}}{\partial y_{n}}
\end{array}
\right|
\neq 0
\end{eqnarray*}
 Entonces


$f_{Y1, \ldots Y_{n}}\left( y_{1} \ldots y_{n} \right) =$\begin{eqnarray*}
\begin{cases}
\mid J \mid f_{X1 \ldots Xn}\left[h_{1} \ldots h_{n}\right] & \left(y_{1} \ldots y_{n} \right) \in D \\  
0 & e.o.c
\end{cases}
\end{eqnarray*}

\textbf{Ejercicios:}

\smallskip

 Sean $X,Y$ ind. t.q. $X \sim Gamma \left( \alpha _{1}, \lambda\right) Y \sim Gamma \left( \alpha_{2}, \lambda \right)$. Hallar $f_{u,v}$ donde $ U= \dfrac{Y}{X}$ $V = X+Y$ ? Son $U$ y $V$ ind?
 
\[
\sim Gamma \left( \alpha _{1},\lambda \right)\Rightarrow f_{X} \left(x \right)=
\begin{cases}
\dfrac{\lambda^{\alpha_{1}} }{M \left(\alpha_{1}\right)} x^{\alpha_{1}-1}e^{-\lambda x} & x>0 \\ 
0 & e.o.c
\end{cases}
\]

\[
\sim Gamma \left( \alpha _{2},\lambda \right)\Rightarrow f_{Y} \left(y \right)=
\begin{cases}
\dfrac{\lambda^{\alpha_{2}} }{M \left(\alpha_{2}\right)} y^{\alpha_{2}-1}e^{-\lambda y} & y>0 \\ 
0 & e.o.c
\end{cases}
\]

\[
 f_{XY} \left(x,y \right)=
\begin{cases}
\dfrac{\lambda^{\alpha_{1}+\alpha_{2} } }{M \left(\alpha_{1}\right)M \left(\alpha_{2}\right)} x^{\alpha_{1}-1} y^{\alpha_{2}-1}e^{-\lambda \left( x+y \right) } & y>0, x>0 \\ 
0 & e.o.c
\end{cases}
\]

$U=\frac{y}{x} \Rightarrow y=ux$

\smallskip

$v=x+y \Rightarrow v= x+ux \Rightarrow v= \left( 1+u \right) \Rightarrow x= \frac{v}{1+u}$

\smallskip

$y= u \left( \dfrac{v}{1+u}\right)$

\[
\begin{vmatrix}
\frac{\partial x}{\partial u} & \frac{\partial x}{\partial v} \\\\
\frac{\partial y}{\partial u} & \frac{\partial y}{\partial v}\\
\end{vmatrix} = \begin{vmatrix}
\frac{-v}{\left(1+u \right)^{2} } & \frac{1}{\left(1+u \right)} \\\\
v\left(1+u \right)^{-1}-uv \left(1+u \right)^{-2} & \frac{u}{1+u}\\
\end{vmatrix}
\]


$-\frac{-vu}{\left( 1+u \right)^{3}}-\dfrac{v}{\left(1+u \right)^{2}}+ \frac{-vu}{\left(1+u \right)^{3}}= v \left(1+u \right)^{-2}$

 \smallskip
 
$f_{uv} \left(u,v \right) = |J| f_{XY}\left(x,y \right)$

\smallskip

$ = v \left(1+u \right)^{2} \dfrac{\lambda^{\alpha_{1}+\alpha_{2} } }{M \left(\alpha_{1}\right)M \left(\alpha_{2}\right)} \left( \frac{v}{1+u}\right)^{ \alpha_{1}-1} \left( \frac{uv}{1+u} \right)^{\alpha_{2}-1}e^{-\lambda \left(\frac{v+uv}{1+u} \right)}$

\smallskip

$= \frac{u^{\alpha_{2}-1}v^{\alpha_{1}+\alpha_{2}-2}}{\left(1+u \right)^{ \alpha_{2}+ \alpha_{2}}} \dfrac{\lambda^{\alpha_{1}+\alpha_{2} } }{M \left(\alpha_{1}\right)M \left(\alpha_{2}\right)} e^{-\lambda v } $

\smallskip

\textbf{Ejercicios para clase:}

\smallskip

\noindent \textbf{1.}Sea $X \sim Exp \left( 1 \right)$ y $Y \sim U \left( 0,1 \right)$ ind. $U=X+Y$ y $V=X-Y$. Hallar $f_{U,V} \left( u,v \right)$ y averiguar independencia.

\[
X \sim \text{Exp} \left( 1 \right) \Rightarrow f_{X}\left( x \right)=
\begin{cases}
\lambda e^{-\lambda x }= e^{-x} & x>0 \\ 
0 & e.o.c
\end{cases}
\]

\[
Y \sim U_{ \left( 0,1 \right) } \Rightarrow f_{Y}\left(y \right)=
\begin{cases}
1 & 0 \leq  y \leq 1 \\ 
0 & e.o.c
\end{cases}
\]

\begin{eqnarray*}
u&=& x+y \Rightarrow x=u-y  \\
v&=& x-y \Rightarrow y=x-v \Rightarrow y=u-y-v \Rightarrow y= \frac{u-v}{2}\\
x&=&u-\left(\frac{u-v}{2}\right)=\frac{2u-u+v}{2}=\frac{u+v}{2}
\end{eqnarray*}

\begin{eqnarray*}
T \left(t,0 \right) &=& \left( t,t \right)\\
T \left(0,t \right) &=& \left( t,-t \right)\\
T \left(t,1 \right) &=& \left( t+1,t-1 \right)\\
\end{eqnarray*}

\bigskip

\begin{eqnarray*}
T \left(0,v \right) = \left( x+y,x-y \right)\\
\sigma_{1} \left(t \right) = \left( t,0 \right) & t>0 \\
\sigma_{2} \left(t \right)= \left( 0,t \right) & 0<t<1 & v=-u \\
\sigma_{3} \left(t \right) = \left( t,1 \right) & t>0 
\end{eqnarray*}

\begin{eqnarray*}
u=t \Rightarrow u=v & u=t+1 \\
v=t & v= t-1 \\
u=t v=t-1 \Rightarrow u=-v \Rightarrow u-1=t v+1=t \\
u-1=v+1 \Rightarrow v=u-2
\end{eqnarray*}
\[J=
\begin{vmatrix}
\frac{1}{2} & \frac{1}{2}\\
\frac{1}{2} & -\frac{1}{2}\\
\end{vmatrix} 
\]

\[ 
\begin{vmatrix} J \end{vmatrix}=\frac{1}{2}
\]
 
\begin{eqnarray*} 
f_{u,v} \left(u,v \right) &=& |j| f_{XY}\left( u,v \right)\\
&=& 
\begin{cases}
\frac{1}{2} e^{-\left(\frac{u+v}{2} \right)} & x,y\in D \\
0 & e.o.c\\
\end{cases}
\end{eqnarray*}

\begin{eqnarray*} 
f_{u} \left(u \right) = \begin{cases}
\int _{-u}^{u}\frac{1}{2}e^{- \frac{u+v}{2} }\,dv & 0 \leq u \leq 1  \\

\int _{-u-2}^{u}\frac{1}{2}e^{- \frac{u+v}{2} }\,dv & 0 \geq 1  \\

0 & e.o.c\\
\end{cases}
\end{eqnarray*}

\begin{eqnarray*} 
\int _{-u}^{u}\frac{1}{2}e^{- \frac{u+v}{2} }\,dv &=&\frac{1}{2}e^{- \frac{u}{2} } \int _{-u}^{u} e^{- \frac{v}{2} } \,dv\\
&=& \left( \frac{1}{2} e^{-\frac{u}{2} } \right)\left(\frac{ e^-{\frac{v}{2} } }{-\frac{1}{2}}\right)_{-u}^{u}=
 \left( \frac{1}{2} e^{-\frac{u}{2} } \right)\left(-2 e^{\frac{-v}{2}}\right)_{-u}^{u}\\
 &=& \left( \frac{1}{2} e^{-\frac{u}{2} } \right)\left(-2 \left(e^{\frac{-u}{2}}- e^{\frac{u}{2}}\right)\right)= -e^{-u}+1=1-e^{-u}\\
 \int_{u-2}^{u} e^{\frac{v}{2} }\,dv &=& \left(-2e^{-\frac{v}{2} } \right)_{u-2}^{u}=-e^{-\frac{u}{2}}+ e^{-\frac{\left( u-2\right)}{2}}\\
 &=& e^{-\frac{u}{2}}\left(-e^{-\frac{u}{2}}+ e^{-\frac{\left( u-2\right)}{2}} \right)= -e^{-u}+e^{-\frac{u}{2} -\frac{\left( u-2\right)}{2}}\\
 &=& -e^{-u}+e^{-u}e^{+1} = e^{1-u}-e^{-u}
\end{eqnarray*}

\begin{eqnarray*} 
f_{u} \left(u \right) = \begin{cases}
1-e^{-u} & 0 \leq u \leq 1 \\
e^{1-u}-e^{-u} & u\geq1  \\
0 & e.o.c
\end{cases}
\end{eqnarray*}

\begin{eqnarray*} 
f_{v} \left( v \right) = \begin{cases}
\int_{-v}^{v+2} \frac{1}{2}e^{\left(\frac{u+v}{2}\right)}\,du & -1\leq v \leq 0\\
\int_{u}^{v+2} \frac{1}{2}e^{\left(\frac{u+v}{2}\right)}\,du & v \geq 0\\
0 & e.o.c
\end{cases} 
\end{eqnarray*}

\begin{eqnarray*} 
f_{v} \left( v \right) = \begin{cases}
e^{-\frac{v}{2}}\left(e^{\frac{v}{2}}-e^{- \frac{v+2}{2} } \right)= 1-e^{\left(v+1 \right) }  & -1\leq v \leq 0\\
-e^{-\left(v+1 \right)}+e^{-v} & v \geq 0\\
0 & e.o.c
\end{cases} 
\end{eqnarray*}

\smallskip

\noindent \textbf{2.} Sean $X_{1}, X_{2} \sim U_{ \left(-3,3 \right) }$ ind. $Y_{1}=X_{2}-X_{1}, Y_{2}= X_{1}-X_{2}$

\begin{eqnarray*}
y_{1}&=&x_{2}-x_{1} \Rightarrow x_{2}=y_{1}+x_{1} \Rightarrow x_{2} = y_{1}+y_{2}-x_{2} \Rightarrow x_{2}=\frac{y_{1}+y_{2}}{2}\\
y_{2}&=& x_{1}+x_{2} \Rightarrow x_{1}= y_{2}-x_{2} \Rightarrow x_{1}=y_{2}-\left( \frac{y_{1}+y_{2}}{2} \right) = \frac{2y_{2}-y_{1}-y_{2}}{2}\\
x_{1}&=& \frac{y_{2}-y_{1}}{2}
\end{eqnarray*}

\[
J=
\begin{vmatrix}
-\frac{1}{2} & \frac{1}{2}\\
\frac{1}{2} & \frac{1}{2}
\end{vmatrix} = -\frac{1}{4}-\frac{1}{4}= -\frac{1}{2}, |J|=\frac{1}{2}
\]

\begin{eqnarray*}
f_{Y_{1},Y_{2}}\left( y_{1},y_{2}\right)&=& \frac{1}{2} f_{X_{1},X_{2}}\left( \frac{y_{2}-y_{1}}{2}, \frac{y_{2}+y_{2}}{2}\right)\\ 
&=&\frac{1}{2} \left( \frac{1}{36}\right)= \begin{cases}
\frac{1}{72}  & x,y \in D\\
0 & e.o.c
\end{cases} 
\end{eqnarray*}
%_________________________________________________________________
\begin{Ejem} Sea $X \sim U \left(0,2 \right) $. Hallar la densidad de $Y= X^2 - 3X + 1$
\begin{eqnarray*}
f_{Y}&=& \left(y \right)= \frac{dF_{Y}\left(y \right) }{dy}\\
F_{Y} \left(y \right)&=& \prob \left[ Y \leq y \right] = \prob \left[ X^2 - 3X + 1 \leq y \right] = \prob \left[ \left( X - \frac{3}{2} \right)^2 + 1 \leq y + \frac{9}{4} \right]\\
&=& \prob \left[ \left( X - \frac{3}{2} \right)^2  \leq y + \frac{9}{4}-1 \right] = \prob \left[ X- \frac{3}{2} \leq \sqrt{y+\frac{5}{4} }\right]\\
&=& \prob \left[ X \leq \sqrt{y+\frac{5}{4} } - \frac{3}{2} \right]= F_{X} \left( \left( y+\frac{5}{4} \right)^{ \frac{1}{2} } + \frac{3}{2} \right)\\
&=& f_{Y} \left(y \right)= f_{X} \left( \left(  y + \frac{5}{4}\right)^{ \frac{1}{2}}+ \frac{3}{2}    \right) \left(\frac{1}{2} \left( y + \frac{5}{4} \right)^{- \frac{1}{2} } \right)\\
f_{Y} \left( y \right) &=& \begin{cases} \frac{1}{2}  \left(\frac{1}{2} \left( y + \frac{5}{4} \right)^{- \frac{1}{2} } \right)= \frac{1}{4} \left( y + \frac{5}{4} \right)^{-\frac{1}{2} } & \left(-1,1 \right)\\
0 & \textrm{e.o.c}
\end{cases}\\
\int_{-1}^{1} \frac{1}{4} \left( y + \frac{5}{4} \right)^{-\frac{1}{2} } \, dy &=& \frac{1}{4} \left(2 \left( y + \frac{5}{4} \right)^{ \frac{1}{2} } \right)_{-1}^{1}\\
&=& \left. \frac{1}{2} \left( y + \frac{5}{4} \right)^{\frac{1}{2}} \right|_{-1}^{1} = \frac{1}{2} \left[ \left( 1 + \frac{5}{4} \right)^{ \frac{1}{2} } - \left(-1 + \frac{5}{4} \right)^{ \frac{1}{2} } \right]\\
&=& \frac{1}{2} \left( \left(\frac{9}{4} \right)^{ \frac{1}{2}} - \left(\frac{1}{4} \right)^{ \frac{1}{2}} \right) = \frac{1}{2} \left( \frac{3}{2} - \frac{1}{2} \right) = \frac{1}{2}
\end{eqnarray*}

\end{Ejem}
\begin{Ejem}$X_{1}, X_{2} \sim U \left(0,1 \right)$ ind. $X_{1}= Y_{1} \cos \left(Y_{2} \right)$, $X_{2} = Y_{1} \sin \left(Y_{2} \right)$ Hallar la densidad conjunta de $Y_{1} $ y $Y_{2}$ y averiguar independencia
\begin{eqnarray*}
f_{X_{1}X_{2}} \left(x_{1} x_{2} \right) &=&  
\begin{cases}
1 & x_{1} \in  \left(0,1 \right), x_{2} \in \left(0,1 \right)\\
0 & \textrm{e.o.c}
\end{cases}\\
x_{1}^2+ x_{2}^2 &=& y_{1}^2 \cos^2 \left(y_{2} \right) + y_{1}^2 \sin ^2 \left(y_{2} \right)\\
&=& y_{1}^2 + y_{1}^2 = 2 y_{1}^2\\
\Rightarrow y_{1} &=& \sqrt{\frac{1}{2} \left( x_{1}^2 + x_{2}^2 \right) }\\
\frac{X_{1}}{X_{2}} &=& \frac{Y_{1} \cos Y_{2}}{Y_{1} \sin Y_{2}} = \tan Y_{2} \Rightarrow Y_{2}= \arctan \left( \frac{X_{1}}{X_{2}} \right)\\
J &=& \left| \begin{array}{cc}
\cos \left( Y_{2} \right) & -Y_{1} \sin \left( Y_{2} \right) \\
\sin \left( Y_{2} \right) & Y_{1} \cos \left( Y_{2} \right)
\end{array} \right| = Y_{1} \cos^2 Y_{2} + Y_{1} \sin^2 Y_{2}\\
&=& Y_{1} \Rightarrow \mid J \mid = Y_{1}\\
f_{Y_{1}Y_{2}} \left( y_{1}, y_{2} \right) &=& \mid J \mid f_{X_{1}X_{2}} \left( \sqrt{ \frac{1}{2} \left(x_{1}^2 + x_{2}^2 \right)}, \arctan  \left( \frac{X_{1}}{X_{2}} \right) \right)\\
&=& \begin{cases}
Y_{1} & y_{1}y_{2}\in \rea\\
0 & \textrm{e.o.c}
\end{cases}\\
\left.
\begin{array}{cccc}
\sigma_{1} \left( t \right) = \left(t,1 \right)\\
\sigma_{2} \left( t \right) = \left(0,t \right)\\
\sigma_{3} \left( t \right) = \left(t,0 \right)\\
\sigma_{4} \left( t \right) = \left(1,t \right)
\end{array}
\right\} 0 &\leq& t \leq 1\\
Y_{1} &=& X_{1} \cos X_{2}, Y_{2} = X_{1} \sin \left( X_{2} \right)\\
T \left(t,1 \right) &=& \left( t \cos \left(1 \right), t \sin \left(1 \right) \right) = \left( t \left(0.5403 \right), t \left( 0.8414 \right) \right)\\
T \left(0,t \right) &=& \left(0,0 \right)\\
T \left(t,0 \right) &=& \left( t \cos \left(0 \right), t \sin \left(0 \right) \right)= \left( t,0 \right)\\
T \left(1,t \right) &=& \left( \cos \left(t \right), \sin \left(t \right) \right)\\
y_{1} &=& t \cos \left(1 \right) \Rightarrow x_{1} \cos^{-1} \left(1 \right)= t\\
y_{2} &=& t \sin \left(1 \right) \Rightarrow x_{2} \sin^{-1} \left(1 \right)= t\\
\frac{y_{1}}{\cos \left(1 \right) } &=& t, \frac{y_{2}}{\sin \left(1 \right) }= t \Rightarrow
\frac{y_{1}}{\cos \left(1 \right) } =  \frac{y_{2}}{\sin \left(1 \right) } \\
y_{2}&=& y_{1} \tan \left(1 \right)\\
y_{2}&=&t \Rightarrow
t= \arccos \left(y_{1} \right) \Rightarrow \arccos \left(y_{1} \right) = \arcsin \left(y_{2} \right)\\
y_{2} &=& \sin t \Rightarrow t= \arcsin \left( y_{2} \right) \Rightarrow \sin \left( \arccos \left(y_{1} \right) \right) = y_{2}
\end{eqnarray*}
\begin{eqnarray*}
f_{Y_{1}} &=& \begin{cases}
\int_{0} ^{y_{1} \tan \left(1 \right) } f_{Y_{1}Y_{2}} \left( y_{1}y_{2} \right) \, dy_{2} & y_{1} \in \left(0,0.5403 \right)\\
\int_{0}^{\sin \left(\arccos \left(y_{1} \right) \right) }  f_{Y_{1}Y_{2}} \left( y_{1}y_{2} \right) \, dy_{2} & y_{1} \in \left(0.5403, 1 \right)\\
0 & \textrm{e.o.c}
\end{cases}\\ 
f_{Y_{2}} \left(y_{2} \right) &=&
\begin{cases}
\int_{\cos \left( \arcsin \left( y_{2} \right) \right) } ^{\frac{y_{2}}{\tan \left( 1\right) } }  f_{Y_{1}Y_{2}} \left( y_{1}y_{2} \right) \, dy_{1} & y_{2} \in \left( 0 , 0.8414 \right)\\
0 & \textrm{e.o.c/}
\end{cases}
\end{eqnarray*}
\end{Ejem}
\begin{Ejem} $N$ es una v.a t.q $\prob          \left[ N=1 \right]= \frac{1}{4},  \prob \left[ N = 2\right]= \frac{1}{3},  \prob \left[ N=3 \right]= \frac{5}{12}$ \\
 $X_{i} \sim Poisson \left(1 \right), i=1,2, \ldots$\\
 Hallar la densidad de la suma aleatoria $SN$
 
 \begin{eqnarray*}
 \prob \left[ S_{N} = x \right] &=& \prob \left[ X_{1}+ X_{2}+ \ldots + X_{N} = x\right]\\
 &=& \sum_{n=1}^{3} \prob \left[ X_{1}+ X_{2}+ \ldots + X_{N} = x \mid N=n \right] \prob \left[ N=n \right]\\
 &=&\sum_{n=1}^{3} \dfrac{X_{1}+ X_{2}+ \ldots + X_{N} = x , N=n}{\prob \left[N=n \right] } \prob \left[N=n \right]\\
 Sol \sim Poisson \left(n \lambda \right)\\
 &=& \sum_{n=1}^{3} \prob \left[S_{n}= x \right] \prob \left[ N=n\right]= \prob \left[ S_{1}=x \right] \prob \left[N=1 \right]+ \prob \left[S_{2}= x \right] \prob \left[ N=2 \right]\\ &+& \prob \left[S_{3}= x \right]\prob \left[N=3 \right]\\
 S_{1} &=& X_{1}\\
 &=&\prob \left[S_{1} = x \right] \prob \left[ N=1\right]+ \prob \left[ S_{2}= x \right] \prob \left[N=2 \right] + \prob  \left[S_{3}= x \right]\prob \left[ N=3 \right]\\
 &=& \left( \frac{e^{-1}}{x!}\right)\frac{1}{4}+ \left( \frac{e^{-2}}{x!} \right) \frac{1}{3} + \left( \frac{e^{-3}}{x!} \right) \frac{5}{12}
 \end{eqnarray*}
\end{Ejem}

\chapter{Simulacion de Variables Aleatorias}
%%_________________________________________________-
\section{Generadores de Variables Aleatorias}


Realizar

\begin{eqnarray*}
\left(aX_{i}+C\right)mod M\Leftrightarrow
X_{n+1}=Res\left(\frac{aX_{i}+C}{M}\right)
\end{eqnarray*}
Entonces para $a,C\in\ent^{+}$, se tiene que $x_{n}\in\left\{0,1,2,\ldots,M\right\}$.

$X_{i+1}=\left(aX_{i}+C\right)mod M$ para
$M,a,c,\in\ent^{+}\Rightarrow X_{i}\in{0,1,2,\ldots,M-1}$

Si hacemos $J_{i}=\left(\frac{X_{i}}{M}\right)$
una sucesi\'on de numeros pseudoaleatorios que se puede considerar como una aproximaci\'on a una sucesi\'on de V.A. uniformes.


Observaciones:La sucesi\'on $\left\{ X_{O},X_{1},X_{2},X_{3},\ldots\right\}$

Se repetira despues de $M$ pasos y por tanto ser\'a peri\'odica, con periodo $M$

\begin{eqnarray*}
a&=&c=X_{0}=3\textrm{ y }M=5\\
X_{1}&=&\left(3\left(4\right)+3\right)mod 5=12mod 5=2\\
X_{2}&=&\left(3\left(2\right)+3\right)mod 5=9mod 5=4\\
X_{3}&=&\left(3\left(4\right)+3\right)mod 5=15mod 5=0\\
X_{4}&=&3 mod 5=3\textrm{ periodo es igual a }4.
\end{eqnarray*}

El periodo de un generador\\
$=M_{min} \lbrace T \mid X_{i}+T=X_{i}\rbrace$\\
Se quisiera que T fuese suficientemente grande para una maquina de 32 bits se recomienda:\\
1) M n\'umero  primo grande apropiado, del termino de la palabra\\
$M=2^{32}-1$\\
$a=7^{5}$\\

Otro generador:

\begin{eqnarray*}
1)Ix=, Iy, Iz\in \jmath_{29999}\\
2)Ix=171\ast Ix MOD 177-\left(\frac{2Ix}{177}\right)\\
Iy=17\ast Iy MOD 176-\left(\frac{35Iy}{176}\right)\\
Iz=170\ast Iz MOD 178-\left(\frac{63Iz}{178}\right)\\
3)\textrm{Si } Ix\leq 0\Rightarrow Ix= Ix+30269\\
Iy\leq 0\Rightarrow Iy= Iy+30307\\
Iz\leq 0\Rightarrow Iz= Iz+30323\\
4)u=\left(\frac{Ix}{30269}+\frac{Iy}{30307}+\frac{Iz}{30325}\right)\\
5)\textrm{Usar} Ix, Iy, Iz en 2\\
6)Por si las dudas\\
\textrm{Si} u=\left\{
\begin{array}{cc}
1 & u=u-EPS\\
0 & u=u+EPS\\
\end{array}
\right.
\end{eqnarray*}

$EPS\rightarrow \textrm{precisi\'on de la maquina}\\
EPS\Rightarrow 1+\frac{EPS}{2}=1$\\

Supongamos que tenemos $\lbrace Y_{1}, Y_{2}\ldots Y_{n}\rbrace$ muestra aleatoria de la V.A Y con distintas $F$. Sea $F_{e}$ la distribuci\'on empirica afectada como:\\

$F_{e}=\frac{\lbrace i: Y_{i}\leq n\rbrace}{n}$\\

Es decir $F_{e}=$ proporci\'on de n\'umeros $\leq n$ (observados)\\
Si la hipotesis nula: F es la distribuci\'on subyacente, es cierta, entonces:\\

$F_{e} \approx F$ entonces se propone:\\
$D\equiv \max_{x}\left\{
F_{e}\left(x\right)
-F\left(x\right)\right\}$\\

El estadistico se prueba 

Sea $Y_(i)=y_(i), i= 1,2,\ldots n$\\

${Y_(1), Y_(2),\ldots Y _(n)}$ M.A 


\begin{eqnarray*}
F_{e}(x)\left(y\right)=
\left\{
\begin{array}{lc}
\prob\left[Y\leq y\right] & 0<y<\infty\\
0 & \textrm{e.o.c.}
\end{array}
\right.
\end{eqnarray*}

Caso discreto:\\
Si $X\leq x \Rightarrow F^{-1} (u) \leq x$ 

\begin{eqnarray*}
u\leq F (x+ xi)\\
u\leq F (x) \\
\Rightarrow P{X\leq x} = P{U\leq F(x)} =F (x)  
\end{eqnarray*}

$\xi por$ definici\'on de $F^{-}$ 
por ser continua por la derecha\\
 
es decir sean $ {u_(1), u_(2)\ldots u(n)} VA \sim U (0,1)$  

\begin{eqnarray*}
\Rightarrow{Y_(1), Y_(2)\ldots Y_(n)}\\ \sim F(Y:\theta)\\
y_(i) =F^{-1} (u_(i)\\
\end{eqnarray*}
Caso cerrado:\\
Si $F(x\vert\theta)$ es la densidad asociada a F\\

\begin{eqnarray*}
\Rightarrow F(x)=\int_(\infty)^{x} f_(x)(t\vert\theta)\delta t\\
\int_(\infty)^{x_(i)} f_(x)(t\vert\theta)\delta t= U _(i)\\
\end{eqnarray*}

\begin{eqnarray*}
Y\backsim U(0,1)\\
a=b=1\\
1 caso a\neq b\\
a=2, b=1\\
Z=a+ (b-a) Y\backsim U(z(a, b))\\
\lbrace X_(n)\ldots ,X_(n)\rbrace u.a.i.i.d. M_(x)=E[x] y \\
\end{eqnarray*}

Qu\'e tan buena es la aproximaci\'on?\\
De d\'onde sacamos $\lbrace X_(1),\ldots ,X_(n)\rbrace$?\\

\begin{eqnarray*}
a&=&1=b\\
1.n&=&m=10\\
2.n&=&20, m=10\\
3.n&=&50, m=60\\
4.n&=&100, m=100\\
5.n&=&100, n=10\\
\end{eqnarray*}
%_____________________________________________________________
\section{Generaci\'on de Variables Aleatorias}
%_____________________________________________________________


Generar variables aleatorias\\
De todas las atribciones?\\
Teoricamente NO\\

1.-Basta saber generar $X_(1),\ldots X_(n) \backsim U (X\vert 0,1)$ para tener $y_(1), y_(2), \ldots, y_(n)\backsim F(Y\vert\theta)$\\
Con tener $y_(1)= T_(i)(X_(1), \ldots, X_(n))$\\

2.-Tiene sentido la frase \textit{Sea $\lbrace x_(1), \ldots, x_(n)\rbrace$ muestra aleatoria} de $U (x\vert0, 1)?$\\

3.- Si nos dan $\lbrace X_(1),\ldots, x_(n)\rbrace $ C\'omo checamos que son una sola muestra de $U(X\vert 0,1)$?\\

1)Supongamos:\\
\begin{eqnarray*}
F(X,\theta)=1-e^{x\theta}, \theta, 
x \in\rea^{+},\theta=3
\end{eqnarray*}


$F_(\theta): (0,\infty)\rightarrow(0,1)$de manera biyectiva\\ 
$u=1- e^{-x\theta} \Rightarrow x=-\frac{1}{\theta}$\\ 

aseguro\\

Si $u\backsim(u\vert0,1) \Rightarrow x=-\frac{1}{\theta}log (i-u) \backsim exp(x\vert\theta)$\\

$p(x\vert\theta)=\frac{e^{-\theta}\theta^x}{x^1}$\\

$P\swarrow= P[x=k]$\\

$F:(0,\infty)\rightarrow (0,1)$ pero no de manera inyectiva\\ 
$F(2,3)= F(2,\theta)=u$\\
Sea $F^-:(0,1)\rightarrow\pi$\\
$F^{-}(t)= \infty \lbrace x \epsilon\pi: F(x)\geq t\rbrace Vt\epsilon(0,1)$\\

La llamada iversa generalizada de F\\

O sea $\rightarrow$ Si $F$ continua entonces $F^{-1}=F^{-}$\\

Si $u\backsim U(u\vert0,1)$ y F es una funci\'on de distribuci\'on, entonces $X=F^{-}(u)$ tiene como funci\'on de distribuci\'on a $F$\\

P.D $P[X\leq x]= F(x)$\\

Caso continuo\\
$P[X\leq x]= P[F^{-1}(u)\leq x]=P[U\leq F(x)]=F(x)$\\

Caso discreto\\
$\delta X\leq x\Longrightarrow F^-(u)\leq x$\\


$u\leq F(x+\varepsilon) \varepsilon\epsilon\pi^{+}$ por defecto de $F^{-}$
$u\leq F(x)$ por ser contado por la derecha
$\Longrightarrow P[X\leq x]= P[U\leq F(x)]= F(x)$

O sea\\
Si $\lbrace u_(1), \ldots, U_(n)\rbrace$ son $u\cdot a\cdot i\backsim U(u\vert0,1)$\\

Entiende $\lbrace Y_(1),\ldots, Y_(n)\rbrace \backsim F(Y; \theta)$ VAIID

donde $Y_(i)=F^{-1}(u_(1)) V i \epsilon Jn$

Continuo\\
Si $p(x\vert\theta)$ es la densidad asociada a F, entiende\\

$F(x)= \int^{x}_{-\infty} p(\epsilon\vert\theta)dt$\\

Nos podemos concentrar en aprender a generar $\lbrace X_(1), \ldots, X_(n)\rbrace\backsim U(x\vert0,1)$\\

TRENT $\longleftarrow$ Funci\'on\\



$1.-D(x)= \lbrace \frac{2x  x\epsilon[0,\frac{1}{2}]}{2(1-x) x\epsilon [\frac{1}{2}, 1]}$\\
$x_i= D(x_i-1)=\ldots =D^i (x_1)$\\

La funci\'on queda as\'i\\
2.-$G(x)= 4(x)(1-x)$ $x\epsilon(0,1)$\\
Y sea $X_{n}=G^{n}(x_{1})$\\
$U_n=\frac{2}{\pi}sen^{-1}\sqrt{x_n}$\\

$u\epsilon 80,1), P(a,b]= b-a=\lambda (a,b]$\\
$45=101101\longleftarrow$ binaria\\
$u=\sum _{i=1}^{\infty}\frac{d_i(u)}{z^{i}}= 0,d_1(w)d_2(w)\ldots$\\

quiero construir 
I)$X_1\ldots, X_n,\ldots \epsilon(0,1)$\\
II)Independientes\\
III)$P[X_1\leq x]= x$\\

Lancemos una moneda al aire\\
águila=1\\
sol=0\\

Lanzo n veces $u_1, u_2, \ldots, u_n$\\

\begin{eqnarray*}
P\lbrace\epsilon (0,1]: d_i(w)=u_i, i\epsilon J_n\rbrace =P \lbrace w\epsilon (0,1]: \sum _{i=1}^{\infty}\frac{d_i(w)}{2_{i}}\epsilon (\sum _{i=1}^{n}\frac{u_i}{z_i},\sum _{i=1}^{n}\frac{u_i}{2_i}+\frac{1}{2^n}]\rbrace =\frac{1}{2^{n}}\forall n\epsilon IN\\
P\lbrace W\epsilon(O,1):d_i(w)=u_i\rbrace]=\frac{1}{2}\\
P[d_i=u_i :i\epsilon J_n]=\pi^{n} P[d_i=u_i]=\frac{1}{2^{n}}	\forall n\in IN\\
\Rightarrow \left\{d_{i}:i\in \mathbb{N}\right\} 
\end{eqnarray*}
son u.a independientes\\ 
							
												
Debida al Kolmogoroff\\

\begin{eqnarray*}						
(0,1), \lambda (a,b)=b-a\\						
w=\sum_{n=1}^{\infty}\frac{d_{n}(w)}{2^{n}}\epsilon (0,1]\\
d_{n}(w)=\left\{0,1\right\},\forall n\in N
\end{eqnarray*}
$u_{1},u_{2},\ldots,u_{n},u_{i}\in \left\{0,1\right\}$\\

\begin{eqnarray*}
P[d_i=u_i, i\epsilon J_n]= P[d_i(w)= u_i]= 2^{n}\\
P[d _(n)(w)=u _n\\
\zeta (w)= \\
P[w\epsilon\Omega:\xi_n(w)=1]=\frac{1}{2} \forall n\in N
P[W\epsilon\Omega:\xi_1 (w)=u_1,\ldots ,\xi_n(w)=u _n=\frac{1}{2^{n}}\\
\end{eqnarray*}

%alaide 

u.e $\lbrace \xi _{n}: n\in {N}\rbrace$ sucesi\'on de u.a.i\\

Sean\ $\lbrace \eta_{i,j} :j  \in \nat \rbrace\\  
\forall  i \in \nat$  

Subsecciones distintas de $\lbrace{\xi _n}\rbrace$\\
Entonces\\

%_______________________________
\begin{eqnarray*}	
\eta_i(w)=\sum_{j=1}^{\infty} \frac{\eta_{ij}(w)}{2^{j}}
\end{eqnarray*}
%________________________________
\begin{eqnarray*}	
P[\eta_i\leq w]=w =\sum_{n=1}^{\infty} \frac{d_n(w)}{2^{n}}
\end{eqnarray*}
%________________________________

\begin{proof}

\begin{eqnarray*}	
P[\eta_i < w]=P\lbrace U_{j=1}^{\infty}(\eta_{i1}=d_1,\eta_{i2}=d_2, ... ,\eta_{ik}<d_k)\rbrace
\\
=\sum_{k=1}^{\infty} P[\eta_{i1}=d_1\eta_{i2}=d_2, ... ,\eta_{ik}<d_k]
\\
=\sum_{k=1}^{\infty}\dfrac{P[\eta_{ik}<d_k]}{2^{k-1}}
\\
=\sum_{k=1}^{\infty}\dfrac{d_k(w)}{2^{k}}=w
\end{eqnarray*}

\end{proof}
%____________________________________
\begin{eqnarray*}
demostremos:
\\
P[\eta_{ik}<d_k]=\dfrac{d_k}{2}
\\
P[\eta_{i}\leq w]=w
\end{eqnarray*}
%demostracion pendiente falta entendimiento en ecuaciones 
\\
%____________________________________
$\lbrace\eta_n: n \in \nat\rbrace $ donde $ u.a.i $ son constantes distintas $
\\U(\eta(0,1)
 $ por lo tanto, tiene sentido decir$\\
 $sea $ \lbrace x_1,..., x_n\rbrace m.a $ de $F(x;\theta)$
\\
Es $\lbrace x_1,...,X_n \rbrace m.a$ de $F(x;\theta)$?
\\
Sea $\ent_n=\lbrace x_1,x_2,...,x_n\rbrace v.a.i.i.d.F$
\\
i. Independientes
\\
ii. $X \sim F$
%____________________________________

\section{Teorema de Glivenico-Cantelli}

\begin{eqnarray*}
\textrm{Sea }
 \ent_n \backsim F 
 \textrm{ y } F_n(x)=\dfrac{1}{n}=\sum_{i=1}^{n}I_{(-\infty,x)} (x_i)
 \\\textrm{(Funcion de Distribucion Empirica) }\\
 %faltan los acentos 
%_____________________________________
 \end{eqnarray*} 
 
entonces:\\
$P\lbrace sup_{x\in R} \parallel F_n(x)-F(x)\parallel> \epsilon \rbrace_{n\rightarrow\infty}\rightarrow 0    \\ \forall \epsilon \in R^{+}$
\\
$F_n(x)_{n\longrightarrow\infty}\longrightarrow F(x)$
\\
$H_{0}: Z_{n}\sim U(X\vert 0,1)\\
H_{1}:$ No 
\\
(u.a continuas)
%_____________________________________
\begin{eqnarray*}
\textrm{Sea  }t\in(0,1)  \textrm{y} G_n(t)=\frac{1}{n}\sum_{i=1}^{n}I_{(-\infty,t)}(x_i)
\\\\
\delta\lbrace g_n(t)-F(t) \rbrace
\\
\textrm{existen muchas puebas libres de distribu..}
\\\textrm{Anderson-Darling}
\\\\
A_n^{2}=n\int_{0}^{1}\dfrac{(G_n(t)-t)^{2}}{t(1-t)}dt=\sum_{i=0}^{n}\int_{x_1}^{x_i+1}\dfrac{(G_n(t)-t^{2})}{t(1-t)}dt
 \end{eqnarray*} 
%____________________________________________  
 Si  $x_1\leq x_2\leq ... \leq X_n$
 \\
tenemos
\\ 
 $A_n^{2}=-n-\frac{1}{n}\sum_{i=1}^{n} \lbrace(Z_i-1)(\ln_e x_i) + (\ln(1-x_{n_{i+1}}))\rbrace$
 \\\\
 o bien
\\ \\
 $A_n^{2}=-n-\frac{1}{n}\sum_{i=1}^{n} \lbrace(Z_i-1)(\ln_e x_i) + (2(n-i)+1)\rbrace$
 \\\\
 %___________________________________________
\\\\ 
 Ejemplo:
 \\
 $A_{n}^{2}(n\geq 5)$
 \\\\
 Cuantiles
 \\

\begin{table}[h!]
\centering
 \begin{tabular}{||c| c c c c c c c c ||} 
 \hline
 $\alpha=1-\infty$ & 0.25 & 0.15 & 0.10 & 0.05 & 0.025 & 0.01 & 0.005 & 0.001\\ [0.5ex] 
 \hline\hline
 $\Delta^{2}$ & 1.248 & 1.610 & 1.933 & 2.492 & 3.070 & 3.88 & 4.5 & 6\\ [1ex] 
 \hline
 \end{tabular}
\end{table}

%___________________________________________________

D\'\  Agostino, R, Stephens, M.A (1986)(eds)
\\
Goodness-of-fit techniques
\\
NY: Marcel Decker
\\
Ref
\\
CAp 4, M.Stephens
\\
Edf Statistics
\\
Regla (suponiendo que $x_1, x_2, ... ,x_n$ son independientes)
\\
1.- Ordenar las obsiones
\\
2.- Calcular $An^{2}$
\\
3.- Si $An^{2} > w_{1-\infty} \Longrightarrow H:u(0,1)$ es falsa $\propto$
%______________________________________________
\\\\
Ejemplo:
\\\\
Mathematica
\\
Gen $100$ obs $\sim U (x\vert0,1)$ 
\\
$A_{100}^{2}=0.446177$ si son unif, por que son independientes.
\\
(Tarcita): Checar el generadior de $U(0,1)$
\\
$n=10$, de $10$ en $10$ hasta $100$ 
%____________________________________________
\\\\

\section{Independencia}
\begin{eqnarray*}
COV(x_i,x_j)=E\lbrace (x_i-E(x_i)(x_j-E(x_j))\rbrace
\\\\ 
 \textrm{Nota: es coolineal, si la rel no es lineal, no la detecta }
 \\\\
r(s)=\sum_{i=1}^{n}
 \end{eqnarray*} 
%_____________________________________________
Bibliografia:
\\
Randles, Wolfe (1978-1980) -  Intruduction techniques to  the theory of Non parametric statistic
\\
N.Y: Wiley 
\\
Conover, J (197*) 2° Non parametric statisctics N.Y. Wiley
\\
Prueba de signos, rachas, etc 
\\
Ripley, B (1987) Stochastic Simulation N.Y: Wiley
%______________________________________________

Encontrar algoritmo que genere "numeros aleatorios" en el (0,1)
\\
Qu\'e significa que $ \lbrace x_1, ..., x_n \rbrace $  sean numeros aleatorios en el (0,1)? 
\\

Ripley (1987):
\\
Una sucesi\' on de n\' umeros en el (0,1) generada por un proceso determinista que tiene las mismas propiedades relevantes de una sucesi\' on de $ v^{s}, a^{s}, c^{s} $ en el (0,1) 
\\
\\
Qu\'e  es un proceso determinista?
\\
Algoritmo matematico de la forma $x_{n+1}=D(\lbrace x_n \rbrace)$
\\
Donde sus propiedades relevantes son:
\\
%____________________________________
\begin{eqnarray*}
\textrm{De independencia}
\\
\lbrace u_1,..., u_n\rbrace u.a.i.i.d    \ U(u\vert 0,1)
\\ P\lbrace u_1 \leq u_1, ..., U_n\leq u_1\rbrace = \prod_{i=1}^{n} U_{i}
\\ P\lbrace u_1 \leq u_1, ..., U_n\leq u_1\rbrace = \prod_{j=i}^{k} U_{ij}
\rightarrow   \forall k \in J_n
\\
\textrm{prueba de indep Wald-Wolfowite}
\\
u_1,...,u_n \ \ \tilde{u}= \textrm{mediana} \lbrace u_1,...,u_n\rbrace =U_{\dfrac{n+1}{2}} 
\\
u_{(1)} \leq u_{(2)} \leq ... \leq u_{(n)}
\\
y_{i}= \left\{
\begin{array}{lc}
0 & \textrm{ si } u_{i} \leq \tilde{u}\\
1 & \textrm{ si } u_{i} > \tilde{u}
\end{array}\right.         
\\
%______________________________________
T= \textrm{numero de rachas}
\\
\textrm{T} < w_{\dfrac{\alpha}{2}} \ \ \ \textrm{\'o \ \ T}  \geqslant w_{1-\dfrac{\alpha}{2}}
\\
\Longrightarrow \textrm{se rechaza} \ H_{0} : u.a.i.
 \end{eqnarray*}
%____________________________________________
 \\
Los generadores m\'as famosos y utilizados son los llamados generadores congruenciales (mixtos, lineales).
\\
\begin{eqnarray*}  
U_{n+1}=(a u_{n}+b)\ \textrm{mod}1 \ \ a,b \in \mathbf{R^{+}} \longleftarrow sepierden decimales
\\
x_{n+1}=(a x_{n}+b)\ \textrm{mod M} \ \ a,b \in  \nat \longleftarrow para trabajar con enteros
\\
x_{n} \in \lbrace  0,1,..., m-1\rbrace \ \ \forall n \in \nat
 \end{eqnarray*}
 %_______________________________________-
 
 El periodo de un generador es el m\'inimo entero T tal que $X_{i+T}=X_i$  
\\
$D^{t}(x_{i})=x{i}\longleftarrow $ regreso al inicio 
\\
Mientras no llegue al periodo todos los n\'umeros son diferentes a s
\\
En un generador 
\\
$T$ fuese grandote \ \ \ $T=T(a,b,M)$
\\
con mod M
\\
$ \lbrace u_{1},..., u_{k}\rbrace$ $k$ hiperplanos del $[0,1]$
\\
periodos razonables son del orden $2^{37}-1$
\\
No hay forma de demostrar que existen  $a,b \in \nat \cup \lbrace 0 \rbrace$ \ y $\ M \in \nat $ tales que $D(x_{n}= a x_{n}+b)$ mod M sea optimo.
\\
\\
%_____________________________________________
Otro generador 
\\
\begin{enumerate}
\item Dar tres semillas iniciales $IX, IY, It \in J_{29999}$
\\\\
\item $IX=171 * IX$ (mod 177)$-\frac{2IX}{177}$ 
\\\\
$IY=172*IY$ (mod 176) $-\frac{35IY}{176}$
\\\\
$IZ=170*IZ$ (mod 178) $-\frac{63IZ}{178}$
\\\\
\item $IX\leq 0 \Longrightarrow IX=IX+30269$
\\
$I\leq 0 \Longrightarrow IY=IY+30307$
\\
$IZ\leq 0 \Longrightarrow IZ=IZ+30323$
\\
\item
$u=\lbrace \frac{IX}{30269}+\frac{IY}{30307}+ \frac{IZ}{30325} \rbrace$ mod 1
\\
\item 
Usen $IX, JY, IZ$ en $2$
\\
\item
(por si las dudas)
\\
Si $u=\frac{1}{0}$
\\
$u=\frac{- eps}{+ eps}$ \\
Donde:
\\
eps= presici\'on de la maquina $1+\frac{eps}{2}=1$
\\
\end{enumerate} 
%____________________________________
Supongamos que tenemos un generador de numeros aleatorios en $(0,1)$ 
Este, KISS, Ripley; G.Marsaglia
\\
Y Ahora? C\'omo generamos n\'umeros aleatrorios de una dist F? 
\\
(Que sea r\'apido, eficiente, portable) 
%_________________________________
\\
Hab\'abiamos visto que si una $F_{n}$ de distintas entidades.
\\
$u=F(x)\sim U(u\vert0,1)
\\ 
x=F^{-}(u)\backsim F(x)$
\\
Ejemplo 1:
%_______________________________________________
\begin{eqnarray*} 
X\backsim exp ( 0 \ \vert \ \theta )
\\
F(x;\theta)=1-e^{-x\theta}
\\
\Longrightarrow \ \ u=1-e^{-x\theta} \backsim U(u\vert0,1)
\\
\Longrightarrow  \ \ x=\frac{-1}{\theta} \log(1-u) \ \backsim  \ \exp (x\vert\theta)
\\
n=-\frac{1}{\theta} \log (a)  \exp ( x \vert \theta )
\\
\textrm{si} \  \  U\backsim \ (u\vert 0,1)
\\
\textrm{tambien} 1-u \backsim U(u\vert 0,1)
\end{eqnarray*}
%________________________________________________
Ejemplo 2:
%__________________________________________________
 \begin{eqnarray*}
 X\backsim Gamma (X \ \vert \alpha, \beta )
 \\
 F(x; \alpha, \beta )= \int_{0}^{x}\frac{\beta^{\alpha} t^{\alpha-1}}{\pi (\alpha)}e^{-t\beta}dt=u
 \\\\
 \textrm{No tiene forma anal\'itica, es una forma de generar Gamma}
 \end{eqnarray*}
 %___________________________________________________
 Si $ \ \ X \backsim \exp (X \vert \theta)$
 \begin{eqnarray*}
\\
F(x;\theta)=e^{-x\theta}
\\
\rightarrow x=-\frac{1}{\theta} \log u \backsim \exp (x\vert\theta) \longleftarrow \ \ F^{-1}(u)=x
 \\
 \end{eqnarray*}
 %_____________________________________________________
\\\\
Si $ \ \ u \backsim U(u\vert0,1)$
 \\
  \begin{eqnarray*}
 p(x \vert \alpha, \beta)=\frac{\beta^{\alpha} n^{\alpha-1}}{\pi(\alpha)}e^{-x\beta}
 \\
 \textrm{tal que} \ \alpha, \beta \in \mathbf{R^{+}} \textrm{al igual} \ \ x \in \mathbf{R^{+}}
 \\
 Ga(x\vert \alpha \beta)
\\
Ga(x\vert1, \beta)=\exp (x\vert \beta)
\\\\
Ga(x\vert \frac{n}{2}, \frac{1}{2})=X^{2}(x\vert n) 
\\\\
\textrm{Si} \ \ \lbrace x_{1},...,x_{n}\rbrace \ \ \textrm{son} \ \  u.a.i \ \  Ga(x_{i}\vert \alpha_{i}, \beta) 
\\
\rightarrow \sum_{i=1}^{n}x_{i} \backsim Ga \lbrace \Sigma x_{i}\vert \Sigma \alpha_{i}, \beta\rbrace
\\
\phi(t)=E [e^{itx}]
 \end{eqnarray*}
 %__________________________________________________________________________
 Si $\ \ \ X^{2} \ (x\vert2_{n})$
\begin{enumerate}
\item $ U_{1},...,u_{n} \backsim U(u\vert 0,1)$
\\
$(x-{1}=-2 \log U_{i})$
\item $Y=-2 \Sigma_{i=1}^n \log u_{i} \backsim X^{2}(y\vert 2n)$
\\ genero un valor de $X^{2}$
\\ ya no necesita la inversa de la $X^{2}$, pero generamos solo las que tienen g del pares.
\item $y_{1},...,y_{m}$ hay que repetir 1 y 2 
 \end{enumerate} 
 
Para generar variables aleatorias $Ga$, $Ga(x \vert n, \beta)$
\begin{enumerate}
\item $u_{1},..., u_{n} \backsim U(u \vert 0,1)$
\item $Y= -\frac{1}{\beta} \Sigma_{i=1}^{n} \log u_{1} \backsim Ga(Y \vert n, \beta)$
\item igual
\end{enumerate}
%_____________________________________________________________________________

  \begin{eqnarray*}
\textrm{Sean} \ \ X_{1} \backsim Ga (X_{1} \vert \alpha, 1), \ X_{2} \backsim Ga(X_{2} \vert \beta, 1)  \ \ \textrm{ independientes }
\\
Y_{1} =\dfrac{X_{1}}{X_{1}+ X_{2}} \ \ , Y_{2}= X_{1} + X_{2} \ \ ,  \ \ Y_{i} = Y_{i} (X_{1}, X_{2})
\\
X_{1}=Y_{1} Y_{2} = x_{1}(Y_{1}, Y_{2})
\\
X_{2}= Y_{2}(1-Y_{1}) = x_{2} (Y_{1}, Y_{2})
\\ \\
\textrm{TEOREMA}
\\
\textrm{Si} X_{1},..., X_{n}  \ \textrm{son} \ \  u.a.i \ \textrm{con} \ \ p(X_{1},...,X_{n} \vert \theta) \  \textrm{y} \ \  Y_{1},...,Y_{n}
\\ \textrm{es una transferencia uno a uno de } \lbrace X_{1}, ..., X_{n} \rbrace \Longrightarrow \textrm{la densidad de } \ Y
\\
\\
p(Y_{1},...,y_{n} \vert  \theta) = \frac{p(X_{1}(Y_{1},...,y_{n}),..., X_{n}(Y_{1},..., Y_{n}) )}{\vert J \vert}
\\
\\
\vert J \vert =  \left \vert 
\begin{array}{lc}
\ Y_{2} & \ \ Y_{1}\\
-Y_{2} &  \ \ 1-Y_{1}
\end{array}\right. 
\\
\\
p(Y_{1},Y_{2})= \frac{1}{\Gamma(\alpha)} \ (Y_{1}Y_{2})^{\alpha - 1}  \ e^{-Y_{1} Y_{2}}\frac{1}{\Gamma (\beta)} \ [Y_{2}(1-Y_{1})]^{\beta - 1} \ e^{-Y_{2}(1-Y_{1})} \ Y_{2}\\
\\
p(Y_{1},Y_{2} \vert \alpha , \beta)= \frac{1}{\Gamma(\alpha) \Gamma(\beta)} Y_{1}^{\alpha - 1} (1 - Y_{1})^{\beta - 1} Y_{2}^{\alpha + \beta - 1} \ e^{-Y_{2}} \\ \\
p(Y_{1} \vert \alpha , \beta) = \frac{\Gamma (\alpha + \beta )}{\Gamma (\alpha) \Gamma (\beta)} \ \ Y_{1}^{\alpha - 1} 
\\ \textrm{tal que}  \ \ \ 
\\
\alpha , \beta \in \mathbf{R^{+}} 
\\ Y_{1} \in  (0 , 1)
 \end{eqnarray*}
%_____________________________________________________________________________________________________
     
Supongamos que se quiere generar un $Be(x \vert \alpha, \beta)$
 \begin{enumerate}
\item  $X \backsim$ Gamma $(x \vert \alpha, 1)$ ,  $Y \backsim  Ga (Y \vert \beta, 1)$ indepen
\item $Z = \frac{X}{X + Y} \backsim Beta (Z \vert \alpha, \beta)$
\end{enumerate}
%________________________________________________________
$X \backsim B_{er} (X \vert \theta) \ \ p(x \vert \theta)= \theta^{x} (1 - \theta)^{1-x}; \ x \in \lbrace 0,1 \rbrace, \ \ \theta \in (0,1) $
\\ \\
$F(X \vert \theta) =  \left \lbrace 
\begin{array}{lc}
\ 0 & si \ \ x<0\\
\ i-1 & \ \  \ si  \ \ 0<= x <1 \\
\ 1 & si \ \ x<=1
\end{array}\right.$ 
\\ \\
$F^{-}(u \vert \theta) =  \left\lbrace 
\begin{array}{lc}
\ 0 & si \ \ u< = 0 - \theta \\
\ 1 & si \ \ u> 1 - \theta
\end{array}\right. $

 \begin{enumerate}
\item  $u_{1},...,u_{n}$
\item  Si $u  1 - \theta$, hacer $X=0$, \ en otro caso, \ hacer $x=1$ \\
  $\longrightarrow X \backsim B_{er}(X \vert \theta )$ \\
  y de aqui a generar $X \longrightarrow bin (X \vert n)$ est\'a f\'acil \\
  $Y= \sum X_{i}$ ;  $X_{i} \backsim Bernoulli (X \vert \theta)$ \ indep
\end{enumerate}
%____________________________________________________________
En general Si
$X \backsim F$ cuya densidad es $p_{k} = P[X=k], \ \ k \in J_{m}$
 \begin{enumerate}
\item $u \backsim U(u \vert 0,1) $
\item Si $u \in [p_{k}, p_{k+1}] \longrightarrow x=k$
\end{enumerate}
%________________________________________________________
Ahora, dadas 2 va 
$X_{1} \backsim N(x_{1} \vert 0,1)$ \ y \ $X_{2} \backsim N(x_{2} \vert 0,1 )$ indeped
$
\textrm{Sea} \theta \in [0, 2\pi]  \  \ \textrm{y} \ \ r \in \mathbf{R^{+}} u\lbrace 0 \rbrace
\\
X_{1} = r cos \theta \ \ \ \ \ p(x_{1}, x_{2})=\frac{1}{2 \pi} \ \ e^{-\frac{1}{2}(X_{1}^{2})+X_{2}^{2}}
\\
X_{2}=r sen \theta
\\
p(r, \theta)= \frac{1}{2 \pi} \ e^{-\frac{1}{2}} \left |  \begin{array}{lc}
\ Cos \theta - r Sen \theta \\
\ Sen \theta + r Cos \theta
\end{array}\right.\left | = \frac{r}{2 \pi} \ e^{-\frac{1}{2} r^{2}}=g_{1}(\theta) g_{2} (r)\right.$  \\ 
\\ $ \longrightarrow  r$ \ y \ $ \theta$ \ son independientes y obviamente $ \theta \backsim U (\theta \vert 0, 2 \pi)$
%___________________________________________________________
Entonces podemos generar V A $X^{2}(n)$
  \begin{eqnarray*}
  \textrm{Si} X_{1}, ..., X_{n} \backsim N(X \vert 0,1) \longrightarrow \sum_{i=1}^{n} Xi^{2} \backsim  X^{2}(. \vert n)\\ \\
 \end{eqnarray*} 
 Sean $\ \  s=r^{2} \longrightarrow \ \ ds=2rdr \\ $
  \begin{eqnarray*}
  p(\theta, s)= \frac{1}{2 \pi} \ \frac{1}{2} e^{-\frac{s}{2}}
 \end{eqnarray*}
%____________________________________
$X^{2} (x \vert n)= Ga (x \vert \frac{n}{2}, \frac{1}{2})\ = \ \frac{x^{\frac{n}{2}-1} \ e^{-\frac{1}{2}x}}{2^{\frac{n}{2}} \Gamma (\frac{n}{2})}$
  \begin{eqnarray*}
  X^{2}(x \vert 2)= \frac{e^{-\frac{x}{2}}}{2}
  \\
  \longrightarrow s \backsim X^{2} (s \vert 2)
  \\ \\
  \int_{0}^{t} \frac{1}{2} \ e^{-\frac{s}{2}} ds = -e^{-\frac{s}{2}} \int_{0}^{t} = 1-e^{-\frac{t}{2}} \\ \\
  \longrightarrow 1-F(t)= e^{-\frac{t}{2}} = U \\
  P[U \leq u]= P[e^{-\frac{1}{2}} \leq u] = P[t \geq -2 lim u] = u
 \end{eqnarray*}
 Normales 
 
 \begin{enumerate}
\item Sea $u_{1}, u_{2} $
\item  $\theta =2 \pi u_{1} r= \sqrt{-2 lim u_{2}}$
\item $X_{1} = \ \ rCos \theta, \ \ X_{2} = \ \ rSen \theta$
\end{enumerate}
$\longrightarrow X_{1}, X_{2} \ \ \backsim \ \ N(X \vert 0, 1)$
%_____________________________________________
\section{M\'etodo Box, Muller (1958)}

$X \ \ \backsim N (x \vert 0,1) \\ \longrightarrow \ \ Y= M + \varphi X \ \ \ \backsim N (y \vert \mu , \varphi)
\\
\underline{X} \backsim N_{k} ( \underline{X}\ \ \vert \ \  \underline{\mu}  , \underline{\sum_{k}}) $
\\ \\ \\
Descomposici\'on expectral y anexas. 
\\
Si una matrix $\sum_{k*k}$ \ es def positiva, ent. $ \exists n \ \Lambda$ \ y \ $ \underline{P}$ \\
donde $\Lambda = diag \{ \lambda_{1},..., \lambda_{k} \} $ \ \ y \ \ $ \underline{P}$ ortogonal tales que 
\begin{eqnarray*}
\Sigma = \underline{P}' \Lambda P  \ \ \ \ \ \ \ ( P^{1} = P^{-1} ) \\ 
= \underline{P^{1}} \ \ \Lambda^{-\frac{1}{2}} \ \Lambda^{\frac{1}{2}} \ \underline{P} \ \ \  \ \ \Lambda^{\frac{1}{2}}= \{ \sqrt{\lambda_{1}}, ... , \sqrt{\lambda_{k}} \} \\
= A' A \\ \\ 
A^{-1} \Sigma A^{-1} = I \\
(A \Sigma ^{-1} A^{1})^{-1} = I \\
Y = A (X - \mu) \backsim N_{k} (Y \vert \ \underline{o} , I_{k})
\end{eqnarray*}
%_________________________________________________________________

 \section{Transformaciones (Mezclas)}
 
 \begin{eqnarray*}
X \backsim \ N (X \vert 0 , \tau h) \ \ \ \ \ \tau h = \frac{1}{v(x)} \equiv \textrm{precisi\'on} \\
 h \backsim Ga (h \vert \  \alpha \ ,  \ \beta) \\ \\
 p(x \vert \ \tau \ , \ \alpha \ , \ \beta ) = \int_{o}^{\infty} \ p (x , h \vert \ \tau \ , \ \alpha \ , \beta)dh \ = \int_{o}^{\infty} \ p (X \vert \ 0 \ , \ h \tau) \ p (h \vert \ \alpha \ , \ \beta)dh \\
 OC \int_{0}^{\infty}h^{\frac{1}{2}} e^{-\frac{\tau h}{2}x^{2}} h^{\alpha - 1} e^{-h \beta} dh \\ \\
 OC \int_{0}^{\infty} h^{\alpha + \frac{1}{2} - 1} \ e^{-h (\beta + \frac{\tau x^{2}}{2})} dh \\ \\
 OC \left( \beta + \frac{\tau x^{2}}{2} \right)^{- \frac{2 \alpha + 1}{2}}\\ \\
 OC \left( 1 + \frac{\tau \alpha}{\beta} \frac{x^{2}}{2 \alpha} \right)^{-\frac{2 \alpha + 1}{2}}
\end{eqnarray*}

$p(X \vert \tau , \alpha , \beta) = STU (X \vert 2 \alpha , 0 , \frac{\tau \alpha }{\beta})$
\\  
$\Longrightarrow \sqrt{\frac{\beta}{\tau \alpha}} \ X \ \sim STU (X \vert 2 \alpha, 0 , 1) = t(x \vert 2 \alpha) $ \\
\\ \\ \\ 
Algoritmo para generar (students) \\
 
 \begin{enumerate}
\item $h \sim Gamma (h \vert \alpha , \beta)$ 
\item $X \sim N (X \vert  \mu , \frac{1}{\sqrt{\tau h}})$
\item $Y = X \sim STU (Y \vert  2 \alpha ,  \mu , \frac{\tau \alpha}{\beta})$
\end{enumerate}
$X \sim bin (X \vert n , \theta ) \ \ \ \ \ \ \theta \sim Beta (\theta \vert \alpha , \beta)\\ \\
p(X \vert n, \alpha , \beta) = \displaystyle \int_{0}^{1} bin (X \vert n , \theta) Beta (\theta \vert  \alpha . \beta) d \theta \\ \\
= \dfrac{\binom{n}{x} \Gamma (\alpha + \beta)}{\Gamma(\alpha) \Gamma (\beta)} \displaystyle \int_{0}^{1} \theta^{x} (1 - \theta )^{n - x} \theta^{\alpha - 1} (1 - \theta)^{\beta - 1} d \theta \\ \\
= \dfrac{\Gamma (n + 1) \Gamma (\alpha + \beta)}{\Gamma (\alpha) \Gamma (x + 1) \Gamma (n-x+1) \Gamma (\alpha)}
\displaystyle \int_{0}^{1} \theta^{\alpha + x - 1} (1 - \theta)^{\beta + n - x -1} d \theta \\ \\
= \dfrac{\Gamma (n + 1) \Gamma (\alpha + \beta)}{\Gamma (x + 1)\Gamma (n - x + 1) \Gamma (\alpha) \Gamma (\beta)}
\ \ \ \ \dfrac{\Gamma (\alpha + x ) \Gamma (\beta + n - x)}{\Gamma (\alpha + \beta + n)} \\
\\ = BB (X \vert n , \alpha , \beta) \ \ \ \ \ x \in J_{n} U {0} \\
n \in \nat \\ \alpha , \beta \in \mathrm{R^{+}}$
%_____________________________
$X \sim N (X \vert \mu , 1) \\
X^{2} \sim X^{2} (X^{2} \vert 1 , \mu^{2})\\
= \displaystyle \sum_{i=1}^{\infty} P(n \vert \Gamma_{i}) X^{2} (x \vert n)$\\
\\
M\'etodos generales. \\
$f(x)= (x + 2)^{3}(1 -x)\ \ , \ \ x \in (0,1) \\
{X_{1}, ... , X_{n}} \ \ $ \ \ generar $u.a$ que se dist asi
 \begin{eqnarray*}
 \log f(x)= 3 \log (x + 2) + \log (1 - x)\\
 \dfrac{d}{dx} \log f(x) =\dfrac{3}{x + 2} - \dfrac{1}{1 - x} = 0 \\
 3(1 - \sim x) - (\sim x + 2)= 0\\
 3 - 3 \sim x - \sim x - 2 = -4 \sim x + 1 = 0\\
  \sim x = \frac{1}{4} \ \  \textrm{m\'aximo global}\\
  f \left(\frac{1}{4}\right)= \left(\frac{9}{4}\right)^{3} \left(\frac{3}{4} \right)=\frac{2187}{256} \approx 8.54 
\end{eqnarray*}
$\dfrac{d^{2}}{dx^{2}} \log f(x) = - \dfrac{3}{(x+2)^{2}} - \dfrac{1}{(1 - x)^{2}} < 0$
 fraf
 \begin{eqnarray*}
\int_{0}^{1} F(x)dx = \int_{0}^{1}(x^{3}+6 x^{2} + 12 x + 8)(1 - x)dx \\
= \int_{0}^{1} (x^{3} + 6 x^{2} + 12x + 8 - x^{4} - 6 x^{3} - 12 x^{2} - 8x) dx \\
= \int_{0}^{1} (-x^{4} - 5 x^{3} - 6 x^{2} + 4x + 8)dx \\
=  \left [ - \frac{x^{5}}{5} - \frac{5 x^{4}}{4} - \frac{6 x^{3}}{3} + \frac{4 x^{2}}{2} + 8x\right ]_{0}^{1}\\
= -\frac{1}{5} - \frac{5}{4} - 2 + 2 + 8 = \frac{131}{20} \approx 6.55
\end{eqnarray*}
$g(x)= \frac{20}{130} f(x) \ \ \ \ x \in (0,1)$ \ \ \ \ es una densidad
\\ Si $M = \frac{2187}{256} \ \ \Longrightarrow f(x) \leq M U (X \vert 0, 1)$ \\
\\ En gral, si $\exists M \in \mathrm{R^{+}}$ y $h(x)$ una funcion de dens tal q \\
$0 \leq f(x) \leq Mh (x)$

 \begin{enumerate}
\item generar $ x \sim h(x)$ \ \ y $\ \ u \sim U (u \vert 0 , 1)$
\item Hagan $Y = X$ \ \ si \ \ $ u \leq \frac{f(x)}{M h(x)}$
\item en caso contrario ir a 1 
\end{enumerate}

Algoritmo de aceptacion y rechazo \\
 \begin{enumerate}
\item ? $Y \sim f $?
\item ? el proceso termina alg\'un d\'ia?
\item En promedio ?cu\'antas iteraciones necesitamos?
\end{enumerate}

 \begin{enumerate}
\item $P[ Y \leq y] = \dfrac{ \displaystyle \int_{- \infty}^{y} F(x) dx}{\displaystyle \int_{- \infty}^{\infty} F(x)dx} = F_{X}(y)$ \\ \\
$P[Y \leq y] = P \left[ X \leq y \vert U \leq \dfrac{f(x)}{Mh(x)} \right] = \dfrac{P\left[ X \leq y, U \leq \frac{f(x)}{Mh(x)}\right]}{P \left[ U \leq \frac{f(x)}{Mh(x)}\right]}$ \\ \\ \\
$= \dfrac{\displaystyle \int_{\infty}^{y} \left\{  \displaystyle \int_{\infty}^{\frac{f(x)}{Mh(x)}} du\right\} h(x) dx}{\displaystyle \int_{- \infty}^{\infty} \left\{  \displaystyle \int_{- \infty}^{\frac{f(x)}{Mh(x)}} du\right\} h(x) dx} = \dfrac{\frac{1}{M} \displaystyle \int_{- \infty}^{y} f(x) dx}{\frac{1}{M} \displaystyle \int_{- \infty}^{\infty} f(x)dx} = F_{X} (y)$ \\ \\ \\ por lo tanto $Y \ \sim oc \ f(x)$
\item $\theta = P \left[ U \leq \frac{f(x)}{Mh(x)} \right] \\ \\
= E_{X} \left[ \underline{P} \left[ u \leq \frac{f(x)}{Mh(x)} \vert X \right]\right] \\ \\
= \frac{1}{M} E_{X} \left[ \frac{f(x)}{Mh(x)}\right] > 0$ \\ \\
abusando de notacion \\ \\
 \\ $P(-\infty , x) =F(x) = \displaystyle \int_{- \infty}^{x} f(t) dt \\\\
\underline{P} (A) = \displaystyle \int_{a} d \underline{P} = \displaystyle \int_{\Omega}^{- \infty} I_{A} d \underline{P} = E [I_{A}] \\ \\
E \left[  X \vert Y\right] = \displaystyle \int x p(X  \vert Y) dx \\ \\
E_{Y} \ E_{X \vert Y} \left[ X \vert Y \right] = \displaystyle \int  \left\{ x p(X \vert Y) dx \right\} p(y) dy \\ \\
= \displaystyle \int \displaystyle \int  x p(x, y)dy dx \\ \\
= \displaystyle \int  x p(x)dx = E [X]
$\\ \\
$T$
\\ $\Longrightarrow P_{i} = P$ [aceptar $Y$ en el i-esimo intento]$= (1 - \theta)^{i-1} \theta$ \\ \\
$\displaystyle \sum_{i=1}^{\infty} P_{i} = \theta \displaystyle \sum_{i=1}^{\infty} (1 - \theta)^{i - 1} = \frac{\theta}{1 - (1 - \theta)} = 1$
eventualmente termina 
\item E (intentos) $\displaystyle \sum_{i=1}^{\infty} i \theta (1 - \theta)^{i - 1} = \theta \displaystyle \sum_{i=1}^{\infty} i(1 - \theta)^{i - 1} = \\ \\
= \theta  \begin{bmatrix}
  \  & 1+(1- \theta) + (1 - \theta)^{2} + & .... \\
  \ \ \ & (1 - \theta) + (1 - \theta)^{2} + &  ... \\
  \ \ \ \ & (1 - \theta)^{2} + &  ...
 \end{bmatrix}
 \\ \\ \\ \\
\dfrac{1}{\theta} + \dfrac{(1 - \theta)}{\theta} + \dfrac{(1 - \theta)^{2}}{\theta} = \dfrac{1}{\theta} $
\\ \\ \\ de 2 $\theta = P \left[ U \leq \frac{f(x)}{Mh(x)}\right]$ \\ \\
$\frac{1}{\theta} = \dfrac{M}{E_{X} \left( \frac{f(x)}{h(x)} \right)} = 1.3$ \\ 
num de prom de intentos q neces para gen una $u.a Y \sim f(x)$
\end{enumerate}
T \\ En bayesiana \\ \\
$p( \theta \vert Zn) = \dfrac{p(x) P(Zn \vert \theta)}{\int p(\theta) p(zn \vert \theta)d \theta} $ \\ \\
T \\ Generar $v.a$ de la funci\'on $f(x)$ del ejemplo \\ \\
$f(x) = (x + 2)^{3} (1 - x) \ \ \ \ x \in (0,1) \\
\{ X_{1},..., X_{n}\} \approx g(x)$ \\
?$M$? \ \ contar el numero de intentos en prom. \\ \\
$M = 10 ; \ \ \ \frac{1}{\theta} \approx 1.6 \\ 
M = 13 ; \ \ \ \frac{1}{\theta} \approx 2$ \\ \\
Mientras mas grande es $M$ mas desecho \\ 
Si $ u \sim U (u \vert 0,1) \\
\Longrightarrow X = -\frac{1}{\theta} \log u \sim exp (X \vert \theta)$\\ \\
Si $\{ X_{1},...,X_{n} \} \sim exp (X \vert \theta )$ \\
$\Longrightarrow Y = \displaystyle \sum_{i=1}^{n} X_{1} \sim Ga (X \vert n, \theta)$ \ \ para valores enteros de $\alpha$ \\ \\ \\
$Ga (X \vert \alpha , 1) = \dfrac{x^{\alpha - 1}}{\Gamma (\alpha)} e^{-x} \\ \\
Z=\dfrac{X}{\beta} \Longrightarrow \ \ X= Z \beta \ \ \ \ dx= \beta dz \\ \\
\Longrightarrow p(Z \vert \alpha , \beta) = \dfrac{(Z \beta)^{\alpha - 1}}{\Gamma (\alpha)} e^{- Z \beta } \beta \ \ = \dfrac{\beta^{\alpha} Z^{\alpha - 1}}{\Gamma (\alpha)} e^{-Z \beta}$ \\ \\ \\
Generar \\ 
$\{ X_{1}, ..., X_{n}\} \sim Ga (X \vert \alpha , 1) \ \ \ \ \ \alpha \geq 1 $\\
Sea $a =[\alpha]$ y quiero usar $Ga (X \vert a, b)$ para generar $Ga (X \vert \alpha , 1)$ \\ 
$f(x) \ \ \ \ Ga(\alpha , 1) \\
g(x) \ \ \ \ Ga(a,b) \\ \\
\dfrac{f(x)}{g(x)} = \dfrac{Ga(X \vert \alpha , 1)}{Ga(X \vert a,b)} OC \dfrac{x^{\alpha - 1} e^{-x}}{ b^{a} x^{a - 1} e^{-bx}} = x^{\alpha - a} b^{-a} e^{-x (1-b)}$ \\ \\
voy a encontrar el max de $X$
\\ $\log \left( \frac{f(x)}{g(x)}\right) \approx (\alpha - a) \log x - a \log b - x (1 - b) \\ \\
\frac{d}{dx} \log \left( \frac{f(x)}{g(x)}\right) = \dfrac{\alpha - a}{x} - (1 -b) = \dfrac{\alpha -a}{x} - 1 + b = 0 \\ \\
\dfrac{\alpha - a}{x} = 1 - b \ \ \ \ \ \ \dfrac{- (\alpha - a)}{x^{2}} < 0 \\ \\
\sim x = \dfrac{\alpha - a}{1 - b}$ --- maximo global \\ \\
$\Longrightarrow \dfrac{f(x)}{g(x)} \leq \left( \dfrac{\alpha - a}{1 - b}\right)^{\alpha - a} b^{-a} e^{-(\alpha - a)} = \left( \dfrac{x - a}{e (1 - b)} \right)^{\alpha - a} b^{-a} \ \ \ \ \ \forall x \in \mathcal{R^{+}} \\ \\
(\alpha - a) \log (x -a) - (\alpha - a) - (\alpha - a) \log (1 - b) - a \log b + \dfrac{(\alpha - a)}{1 - b} - \dfrac{a}{b} = 0 \\ \\
\alpha \sim b - a \sim b - a + a \sim b = 0 \\
\sim b = \frac{a}{\alpha}$ \ \ hace la cot min \\ 
\\ $M = \left[ \dfrac{\alpha - a}{e (1 - \frac{a}{\alpha})}\right]^{\alpha - a} \left( \dfrac{a}{\alpha}\right)^{-a} = \left( \dfrac{\alpha}{e} \right)^{\alpha - a} \left( \dfrac{a}{\alpha}\right)^{-a}$ \\ \\
O sea pa generar $Ga (x \vert \alpha, 1 )$
 \begin{enumerate}
\item $g(x) = Ga (X \vert [\alpha], \frac{[\alpha]}{\alpha })$
\item  Generar $u \sim U (u \sim 0, 1) \\ 
X \sim Ga \left( X \vert [\alpha] , \frac{[\alpha]}{\alpha}\right)$
\item Si \\ $u \leq  \dfrac{x^{\alpha - 1} e^{-1}}{Mg(x)}$ \\ \\ $Y = X$ \\ \\ si no pos de nuevo \\ \\
$\dfrac{1}{\theta}  = M \Gamma (\alpha)$
\end{enumerate}
Ultimo ejemplo
 \\ Grafica
\\ $N(x \vert 0,1)$
 \begin{eqnarray*}
STU (X \vert 1 , \mu , \tau ) \\ \\
Ca (X \vert \mu , \tau) = \dfrac{1}{\pi \tau (1 + (\frac{x - \mu}{\tau})^{2})} \\ \\
Ca (X \vert a , \tau) = \dfrac{1}{\pi \tau (1 + \frac{x^{2}}{\tau^{2}})} \\ \\
\dfrac{N (x \vert 0, 1)}{Ca (x \vert 0 , \tau)} = \dfrac{\frac{1}{\sqrt{2 \pi}} e^{-\frac{x^{2}}{2}}}{\frac{1}{\pi (1 + x^{2})}} = \sqrt{\frac{\pi}{2}} = (1 + x^{2}) e^{-\frac{x^{2}}{2}} \\ \\
\textrm{la mejor densidad cauchy para generar } N(0,1) \ \textrm{es la} \ 0, 1 \\ \\
Ca (X \vert 0 , 1) =\frac{1}{\pi (1 + x^{2})} \ \ \ \textrm{algorit acept y rechazo} \\ \\
Cota \sqrt{2 - \tau^{2}} 
\end{eqnarray*}
Box Muller \ $\Longrightarrow$  \ generar muchisimas \ \ $10,000$

 \begin{enumerate}
\item $g(x) = Ca (X \vert 0,1) =  \dfrac{1}{\pi (1 + x^{2})}$ 
\item Generar $u \sim U (0 , 1) \ \ \ \ \ x \sim Ca (X \vert 0,1)$ \\ 
$\Longrightarrow$  $ X \sim No (x \vert 0,1)$ \\ \\
si $u \leq \frac{f(x)}{Mg(x)} = \frac{\sqrt{\pi}}{2} (1 + x^{2}) e^{-\frac{x^{2}}{2}}$ \\ \\
$f(x) OC e^{-\frac{x^{2}}{2}} \\ \\
g(x) = Ca (X \vert 0,1) = \frac{1}{\pi (1 + x^{2})} \\ \\
f(x) \leq Mg(x) \ \ \ \ \ \ \ \ \ \ h(1) = \pi (2) e^{- \frac{1}{2}} = M    \\ \\
\Longrightarrow \ \ \dfrac{f(x)}{g(x)} \leq M \\ \\
\dfrac{f(x)}{g(x)} = \pi (1 + x^{2}) e^{- \frac{x^{2}}{2}}  \\ \\
h= \log \dfrac{f(x)}{g(x)} = \log \pi  + \log (1 + x^{2}) - \frac{x^{2}}{2} \\ \\
h' = \dfrac{1}{1 + x^{2}} (2x) - x = 0 \ \ \ \ \Longrightarrow \ x \left[ \frac{2}{1 + x^{2}} - 1\right] = 0$
\end{enumerate}
%_________________________________________________________________

\section{Algoritmo de aceptaci\'on y rechazo}

Sea $g$ densidad,  $N \in \mathcal{R}$  tales que 
$f(x) \leq Mg(x)$ \\
 \begin{enumerate}
\item Generar $X \sim g$ \ \ y \ \ $u \sim U$
\item Si $u \leq \dfrac{f(x)}{Mg(x)}$
\end{enumerate}
Quedarse con $X$ , \ si no , ir a $1$ \\
grafica
\\ \\
$X \sim Ga (X \vert \theta )$
 \begin{eqnarray*}
P(X \vert \theta) = \dfrac{1}{\pi (1 + (X - \theta)^{2})} \\ \\
Z = \{ X_{1}, ... , X_{n}\} \\ \\ \textrm{ y} \ \ \ \ 
p(\theta) = N (\theta \vert 0 , \tau) \ \ \ \tau \ \ \  \textrm{conocida} \\ \\
p(\theta \vert Zn) \ \ OC \ \ p(\theta) \ \ p(Zn \vert \theta) \\ \\
p(\theta \vert Zn) \ \ OC \ \ e^{-\frac{\theta^{2}}{2 \tau^{2}}} \ \displaystyle  \Pi_{i=1}^{n} (1 + (X_{i} - \theta)^{2})^{-1} \\ \\
 \textrm{Intervalo de probabilidad } \ \ p  \ \ \textrm{para} \ \ \ \theta \\ \\
 P \left[ \theta \in I \vert Zn \right] = \int_{I} p(\theta \vert Zn) d\theta = p
\end{eqnarray*}

Preguntas  
$p(X \vert \theta)$ \ \ \ ? C\'omo es ? \\ grafica \\ \\
$Z = \{ X_{1}, X_{2}\} \\ \\
p(X_{1} , X-{2} \vert \theta ) = l (\theta ) = \dfrac{1}{\pi^{2}} \left\{ \frac{1}{1 + (x_{1} - \theta)^{2}} - \frac{1}{1 + (x_{2} - \theta)^{2}} \right\} \\ \\
\log p(X_{1} , X_{2} \vert \theta) = - \log \Pi - \log (1 + (X_{1} - \theta)^{2} ) - \log (1 + (X_{2} - \theta)^{2}) \\ \\
\dfrac{d}{d \theta} \log p(X_{1}, X_{2} \vert  \theta) = \frac{2 (X_{1} - \theta)}{1 + (X_{1} - \theta)^{2}} + \frac{2 (X_{2} - \theta)}{1 + (X_{2} - \theta)^{2}} \\ \\
X_{1} = 3 \  \ \ \ \ X_{2} = 7 \\ \\
l(-2) = 0.000047 \ \ \ \ \ l(o)= 0.00020 \\ 
l(-1) = 0.000092 \ \ \ \ \ l(1) = 0.00055 \ \ \ \ l(2) = 0.0019$ \\
grafica
$
\\ p(\theta \vert Zn)\ \ OC \ \ e^{-\frac{\theta^{2}}{2  \tau^{2}}} \ \  \Pi_{i=1}^{n} \frac{1}{(1+ (x_{i} - \theta)^{2})} \\ \\
\displaystyle \int_{a}^{b} p(\theta \vert Zn) d \theta =p$
\\ \\ T 
\\ \\
 \begin{enumerate}
\item condiciones para que $ l (\theta \vert X_{1}, X_{2})$ \ \ sea unimodal
\item Intentar generar $l (\theta x_{1}, ... , x_{n})$ 
\end{enumerate} 

Marsagha (1977) \\
sea $g$ densidad $g_{l}$ una funcion y $M \in \mathcal{R}$  tales que  $g_{l} \leq f \ \leq Mg$ \\
grafica 
 \begin{enumerate}
\item Generar $X \sim g \ \ \ \ \ u \sim U$
\item Quedarse con $X$ si $u \leq \dfrac{g_{l} (x)}{Mg (x)}$ \\ \\ en caso contrario ir a $1$
\end{enumerate}

La familia exponencial de distribuciones \\ \\
$p(x \vert \theta ) = h(x) exp \{ \theta x - \Psi (\theta)\}$ \\ 
una funcion $f$ \ es log-c\'oncava si log f es c\'oncava \\ grafica \\ \\
$bin (X \vert \theta) = \displaystyle\binom{n}{x} \theta^{x} (1 - \theta)^{n - x} \\ \\ 
= \displaystyle\binom{n}{x} \left( \frac{\theta}{1 - \theta }\right)^{x} (1 - \theta)^{n} \\ \\ 
=  \displaystyle\binom{n}{x} exp \left\{ x \log \frac{\theta}{1 - \theta} + n \log (1 - \theta )\right\} \\ \\
P_{o} (X \vert \lambda) = \dfrac{e^{- \lambda} \lambda^{x}}{x !} \ \  =  \ \ \frac{1}{x !} \ exp \{ x \log \lambda - \lambda \} \\ \\ 
N \ \, \ \ Ga, \ \ Beta, \ \ exp, \ \ geo$ \\ \\
$\log p(x \vert \theta) = \log h(x) + \theta x - \Psi (\theta ) \\ \\
\dfrac{\varrho }{\varrho x} \log p(x \vert \theta) = \frac{h' (x)}{h(x)} + \theta \\ \\
\dfrac{\varrho^{2}}{\varrho x^{2}} \ \log p(x \vert \theta ) = \dfrac{h(x) h''(x) - [h'(x)]^{2}}{h^{2}(x)} < 0 \ \ \ \ \ \ \forall x \in \mathcal{R}$
\\ \\
$\Longrightarrow p (x \vert \theta)$ \ es log c\'oncaba \\ \\ grafica \\ 
$sn = \{ X_{0}, X_{1}, ..., X_{n}\}$ \ \ tales que $p (x \vert \theta) \in \mathrm{R^{+}}$ \\
$ L_{i, i+1} \longrightarrow (x_{i},f(x_{i})) \ , \ (x_{i + 1} , f(x_{i+1}) ) \\ \\ $
Sea $x \in [x_{i}, x_{i+1}]$,
$f_{n} = L_{i, i+1} (x)$ y $\overline f_{n} (x) = min \{ L_{i-1 , i}^{(x)} , L_{i+1, i+2}^{(x)} \}$ \\ \\
$\Longrightarrow f_{-n}(x) \leq f(x) \leq \overline{f}_{n}(x)$ \ \ \ \ \ $\forall x  \in sup (f)$ \\ \\

$\{f_{n} (x)\}_{x \to -\infty}= 0 \ \ \ $ y $ \ \ \ \{\overline{f}_{x \to \infty}n(x)\} = min \{   L_{0,1}(x), L_{n, n+1} (x)\} \\ \\
f(x)= \log p (X \vert \theta)\\ \\
\underline{h}_{n} (x) = exp \underline{f}_{n}(x)\ \ \ \  $ y $ \ \ \ \ \overline{h}_{n}(x)= exp \overline{f}_{n}(x) \\ \\
\Longrightarrow \underline{h}_{n} (x) \leq p (X \vert \theta) \leq \overline{h}_{n}(x) = Cn gn (x)$ \ \ con $gn$ \ \ densidad \\ \\ \\ \\ \\
Sea $f$ log-c\'oncava \\ Si $f$ es c\'oncava \\ 
 \begin{eqnarray*}
 \dfrac{\varrho^{2} f}{\varrho x^{2}} < 0 \\
 \\
 \dfrac{\varrho^{2} \log f}{\varrho x^{2}} = \dfrac{f f'' - [f']^{2}}{f^{2}} < 0 \\ \\
 \textrm{grafica} \\ \\
 Sn = \{ X_{0}, X_{1}, ... , X_{n+1} \} \ C \  sop \ f \\
 \textrm{Para} \ \ \ x \in [x_{i}, x_{i+1}] \\ 
 \underline{h}_{n} (x) = L_{i, i+1} (x) \\ 
 \overline{h}_{n} (x) = min \{ L_{i-1 , i}(x), L_{i+1 , i}(x)\} \\
 \underline{h}_{n}(x) \leq h(x) \leq \overline{h}_{n}(x) \ \ \ \ \ \ \ \ \ \forall \ x \in [x_{0},x_{n+1}] \\ \\
 \textrm{En} \ \ \  [x_{0} , x_{n+1}]^{c} \ \ \ \underline{h}_{n}(x)= 0 \ \ \ \ \overline{h}_{n+1} = min \{ L_{0, 1} (x) L_{n, n+1} (x)\} \\ 
 \Longrightarrow  \underline{f_{n}} \leq f(x) \leq \overline{f}_{n} (x) \ \ \ \ \forall \ \ x \in Df \\ 
 \textrm{Con } \ \ \underline{f}_{n} = exp \{ \underline{h}_{n}\} \ \ \textrm{y} \ \ \ \overline{f}_{n} = exp \{ \overline{h}_{n}\} \\ 
 g \ell \leq f \leq Mg \\ 
 u, x \sim g \\ \\
\end{eqnarray*}
 \begin{enumerate}
\item $u \leq  \dfrac{g \ell}{Mg}$
\item $u \leq \dfrac{f}{Mg}$
\end{enumerate}
Gr\'afica 

\section{M\'etodo Adaptativo de rechazo u aceptaci\'on} 
Sea   $ Cn = \displaystyle \int_{x_{0}}^{x_{n+1}} \overline{f}_{n} (x) dx \\
\Longrightarrow \underline{f}_n (x) \leq f(x) \leq \overline{f}_{n} (x) =C_{n} g_{n} (x)$ \ \ \\
con $g_{n}$ densidad
\\ Algoritmo 
\\
 \begin{enumerate}
\item Fijar $n \in \nat $ \ y \ $S_{n}$
\item Generar $u \sim  U (u \vert 0,1) \ \ $ y \ \ $X \sim  g_{n}$
\item Si $u \leq \dfrac{\underline{f}_{n} (x)}{C_{n} g_{n} (x)}$ \ \ nos quedamos con \ \ $X (X \sim f)$
\item Si no \\ Si $u \leq \dfrac{f(x)}{C_{n} g_{n}(x)} \ \ \ \Longrightarrow \ \ \ \ X \sim f$ \ \ y actualizo  \ \ $S_{n+1} = S_{n} U \{ x \}$
\item Si no 
\end{enumerate}

$g_{n } (x) = C^{-1}_{n}$
\\ grafica 
\\ 
Sea $\alpha_{i} x + \beta_{i} = L_{i , i+1} (x)$ \ la ec de la recta en $[x_{i}, x_{i+1}]$ \ y tenemos  $\ell_{n}$ \ rectas \\ Grafica \\ \\
$
\sum_{i=0}^{n} e^{\alpha_{i}x + \beta_{i}} I_{x_{i}, x_{i+1}} \\ 
$ 
Grafica \\ \\ 
Vou a intentar en ... $a$ ...
\\  \\ \\ 
$[- \infty , x_{0}] \ \ L_{0, 1} (x) \ \ \ \ \ \ [x_{n+1}, \infty ] L_{x_{n}, x_{n+1}} \\
\\ L_{-1 , 0} \rightsquigarrow (x_{0}, y_{0}) \ \ \ \ \ L_{0,1} \rightsquigarrow (x_{1},y_{1}) \\ 
L_{1,2} \rightsquigarrow (y_{0}, x_{1}) \ \ \ \ \ \ L_{2, 3} \rightsquigarrow (y_{1}, x_{2})$ \\ \\ \\ \\ 
Sean $Z_{0},..., Z_{r_{n}}$ los puntos que definen a las $r_{n}$ rectas que cunran a $f$
\begin{eqnarray*}
g_{n} (x) = C_{n}^{-1} \left\lbrace \sum_{i=0}^{r_{n}} e^{\alpha_{i}x + \beta_{i}} \ \ I_{[Z_{i}, Z_{i+1}]} (x) \ \ + \ \ e^{\alpha_{-1} x + \beta_{-1}} I_{[- \infty , Z_{0}]} (x) \ \ + \ \ e^{\alpha_{r_{n} + 1}x + \beta_{r_{n} + 1}} I_{Z_{r_{n}}, \infty} (x)\right\rbrace \\ 
\\ C_{n} =  \int_{- \infty}^{Z-{0}} e^{\alpha_{-1} x + \beta_{-1}} dx +  \sum_{i=0}^{r_{n}} \int_{Z_{i}}^{Z_{i} + 1} dx  + \int_{Z_{r_{n+1}}}^{\infty} e^{\alpha_{r_{n+1}x + \beta_{r_{n+1}}}} dx \\ \\
C_{n} = \dfrac{e^{\alpha_{-1} Z_{0} + \beta_{-1}} }{\alpha_{-1}} + \sum_{i=0}^{r_{n}} \left( \dfrac{e^{Z_{i+1} \alpha_{i} + \beta_{i}} - e^{Z_{i} \alpha_{i} + \beta_{i}}}{\alpha_{i}}\right) - \dfrac{e^{\alpha_{r_{n+1} Z_{r_{n+1}} + \beta_{r_{n}}} }}{\alpha_{r_{n+1}}} \\ \\
g_{n}(x) = C_{n}^{-1} \{ \} = \dfrac{1}{C_{n}}  \sum_{i} e^{\alpha_{i} x + \beta_{i}} I_{\Delta_{i}}(x) \\ \\
= \sum \Pi_{i} f_{i} \ \ \ \ \ \ \Pi_{i}=\dfrac{e^{\alpha_{i}x + \beta_{i}}}{C_{n}^{\alpha_{i}}}
\end{eqnarray*}
Para generar $X \sim g_{n}(x)$
 \begin{enumerate}
\item Eleir $[Z_{i}, Z_{i +1}]$ \ con probabilidad \ \\
$U(X)= \upsilon_{1} (x) I_{1} + \upsilon_{2} (x) I_{2} \\ \\
C = \int U(x)dx = \int_{I_{1}} \upsilon_{1} + \int_{I_{2}} \upsilon_{2} \\ \\
\dfrac{u(\alpha)}{C} = \dfrac{\upsilon_{1}}{C} = \dfrac{\upsilon_{2}}{C} $ \ \ \ \ es densidad 
\item Generar $u \sim (u \vert 0 , 1)$ \ \ y \ \ hacer  $X =$ tarcita
\end{enumerate}

El algoritmo adaptativo de aceptaci\'on y rechazo es un m\'etodo para generar $u.a$ de funciones continuas $log$-c\'oncavas.
\\ ?Qu\'e pasa con las distribuciones discretas? Esto es, sea $X u.a$ cuya funcion de densidad est\'a dada $X$
\\ 
\\
$P [X = k] = P_{k} \ \ \ \ \ \ \forall k \in I \subseteq  \nat U \{ 0 \}$ \ \ \ \ y tales que \ \ \ \ $\displaystyle \sum_{k \in I} P_{k} = 1$ \\ 
Graficas 

$F^{-} (u) = inf \{ t : F(t) \geq u \}$ \\
Sea $P_{k}^{*} = \displaystyle \sum_{i=1}^{k} P_{i} = F(k)$\\
$F^{-} (u) = k $ \ \ si \ \ $P^{*}_{k-1} < u \leq P_{k}^{*}$
 \begin{enumerate}
\item $k = 0 (k = 1)$ \ \ segun donde empiece.
\item Generar $u \sim U (0 , 1)$
\item Hacer $X = k $ \ \ si \ \ $u \ \leq \ P_{k}^{*}$
\item En caso contrario $k = k + 1$ \ \ y regresamos  a 3  
\end{enumerate}
El inicio es importante por que si es asi me voy a tardar mucho en generar los valores grandes.
\\ \\
$E [X] = E $ [numero de comp para obtener $k$] = $\sum_{k \in I} k P_{k}$
\\ \\
Ejemplo
\\ 
bin $\left( x \vert 3 , \dfrac{1}{3}   \right) = \left( \dfrac{3}{x} \right)^{x} \left( \dfrac{2}{3} \right)^{3-x} \ \ \ \ \ \ \ \ \ x \in \{ 0, 1, 2, 3 \}$ \\ \\ \\ \\ 
 $\begin{array}{ccc}
   k & p_{k} &  \ P_{k} \\ \\ 
   0 & 0.296 & \  \frac{8}{27} \\  
 1 & 0.445 & \ \frac{12}{27} \\  
 2 & 0.222 &  \ \frac{6}{27} \\ 
 3 & 0.037 & \ \frac{1}{27}
  \end{array}$
  \\ \\
  $P_{0} = 0.296 = 0.8 * \frac{2}{8} + 0.18 * \frac{9}{18} + 0.02 * \frac{6}{20} \\
  P_{1} = 0.445 = 0.8 * \frac{4}{8} + 0.18 * \frac{4}{18} + 0.02 * \frac{5}{20} \\
  P_{2} = 0.222 = 0.8 * \frac{2}{8} + 0.18 * \frac{2}{18} + 0.02 * \frac{2}{20} \\
  P_{3} = 0.037 = 0.8 * \frac{0}{8} + 0.18 * \frac{3}{18} + 0.02 * \frac{7}{20} \\
  P_{k} = 0.8 f_{1K} + 0.18 f_{2k} + 0.02 f_{3k} = \displaystyle \sum_{j=1}^{3} \Pi_{j} f_{jk} \\
\underline{f_{1}} = \dfrac{1}{8}  
  \begin{bmatrix}
  2\\ 4 \\ 2 \\ 0
 \end{bmatrix} $ \ \ \ \ 
$ 
 \underline{f_{2}} = \dfrac{1}{18}  
  \begin{bmatrix}
  9\\ 4 \\ 2 \\ 3
 \end{bmatrix}
$ \ \ \ \ 
$ 
 \underline{f_{3}} =   
  \begin{bmatrix}
  6\\ 5 \\ 2 \\ 7
 \end{bmatrix}
$ \\
 \begin{enumerate}
\item Generar $i \in \{ 1, 2, 3\}$ \  deacuerdo a $\Pi_{i}$
\item Generar $k$ \ dado \ $i$
\end{enumerate}
Grafica 
\\
$P[k=2] = P ........$
% no se entiende 
\\
$\dfrac{8}{27} = \dfrac{1}{4} \left( \dfrac{9}{27} + 0 + 0 + \dfrac{23}{27} \right) \\
\dfrac{12}{27} = \dfrac{1}{4} \left( \dfrac{1}{18} + \dfrac{9}{27} + \dfrac{21}{27} + 0\right) \\
\dfrac{6}{27} = \dfrac{1}{4} \left( 0 + \dfrac{18}{27} + \dfrac{6}{27} + 0  \right) \\
\dfrac{1}{7} = \dfrac{1}{4} \left( 0 + 0 + 0 + \dfrac{4}{27}\right)$ \\ \\

$\Pi_{i} = \dfrac{1}{4} \ \ \ \forall \ \ i \in J_{i}$ \\
$\underline{P}_{1}^{*} = \dfrac{1}{27}$
$\begin{bmatrix}
  8\\ 18 \\ 0 \\ 0
 \end{bmatrix}$,  $\underline{P}_{2}^{*} = \dfrac{1}{27}$\ \ $\begin{bmatrix}
  0\\ 9 \\ 18 \\ 0
 \end{bmatrix}$  \ \ \ $\underline{P}_{3}^{*} = \dfrac{1}{27}$ $\begin{bmatrix}
  0\\ 18 \\ 6 \\ 0
 \end{bmatrix}$ \ \ \ $\underline{P}_{4}^{*} = \dfrac{1}{27}$ $\begin{bmatrix}
  2\\ 0 \\ 0 \\ 4
 \end{bmatrix}$
\begin{enumerate}
\item Genero $i \in \{ 1, 2, 3, 4\}$
\item Generar $k$ dado $i$
\end{enumerate}
?Existe siemore la descomposici\'on? \\
$P_{k} = \displaystyle \sum_{j} \Pi_{j} f_{jk}$ ? \\
NO SE SABE \\
Si existiera este algoritmo es m\'as eficiente que el gral. \\
?El algoritmo de aceptaci\'on y rechazo funciona para $v.a$ dise? \\
Si, ero no sirve para nada \\
En resumen: \\
- No existe el algoritmo 'optimo para generar $\{ X_{1},...,X_{n}\} \sim F(x; \theta)$ \\
-S\'i existen algoritmos generales  para generar $\{ X_{1},..., X_{n} \} \sim F(x; \theta)$ \\
$F^{-} A y R $ Disc, adaptivo, etc EXPERIENCIA \\
Tiene que ver : \\
\begin{enumerate}
\item[-] n
\item[-] Lenguaje de programaci\'on.
\item[-] Velocidad de la maquina. \ \ (menos imp)
\item[-] Precisi\'on 
\end{enumerate}
Asumimos que tenemos "excelentes" generadores b\'asicos de $u.a$ de distribuciones distintas y un programa general del algoritmo de aceptaci\'on y rechazo. \\
?D\'onde vamos a usar "simulaci\'on"? \\
Sea $Z_{n} \{ X_{1}, X_{2}, ..., X_{n}\} \sim U(X \vert \theta_{1}, \theta_{2}) \\
H . \theta_{1} = \theta_{0} \ \ \ \ us \ \ \ H' : \theta_{1} \neq  \theta_{0}$ \\
Neyman - Pearson \\
Cociente de verosimilitudes (generalizadas) \\ \\
$\Lambda = \dfrac{\ell (\theta_{1}, \theta_{2} \vert Z_{n}, H)}{\ell (\theta_{1}, \theta_{2} \vert Z_{n})} = \dfrac{\theta_{1}, \theta_{2} \in H \ell (\theta_{1}, \theta_{2} \vert Z_{n})}{H \ell (\theta_{1}, \theta_{2} \vert Z_{n})}  \\ \\
\ell (\theta_{1}, \theta_{2} \vert Z_{n}) OC p (Z_{n} \vert \theta_{1}, \theta_{2}) = \Pi_{i=1}^{n} p (x_{i} \vert \theta_{1}, \theta_{2}) \\ \\
\Longrightarrow  \ell (\theta_{1}, \theta_{2} \vert Z_{n}) = \Pi_{i=1}^{n} \frac{1}{\theta_{2} - \theta_{1}} I_{(\theta_{1}, \theta_{2})} (x_{i}) \\ \\
=  \ell (\theta_{1}, \theta_{2} \vert y_{1}, Y_{n}) = (\theta_{2} - \theta_{1})^{-n} I_{[- \infty , \theta_{2}]} (Y_{n}) I_{[\theta, \infty]} (Y_{1}) \\ \\ $ con $Y_{1} = min Z_{n} \ \ \ Y_{n} = max Z_{n} $ \\
Grafica \\ \\
sup $\ell (\theta_{1}, \theta_{2} \vert Z_{n}) = \ell (\theta_{0}, Y_{n} \vert Z_{n}) \\ $
sup $\ell (\theta_{1}, \theta_{2} \vert Z_{n}) = \ell (Y_{1} , Y_{n} \vert Z_{n}) \\
\Longrightarrow \Lambda = \left( \dfrac{Y_{n} - Y_{1}}{Y_{n}- \theta_{0}}\right)^{n} \\ 
\wp = \{ Z_{n} . \Lambda (Z_{n})\leq k \Lambda Y_{i} < \theta_{0}\} \\
P [\Lambda (Z_{n}) \leq k \vert H] \leq \alpha \\
\\ \\ \\ 
p(X_{1}, ... , X_{n} \vert \theta_{1}, \theta_{2}) \rightsquigarrow p (Y_{1}, Y_{n} \vert \theta_{1}, \theta_{2}) \rightsquigarrow p(\Lambda (Z_{n}) \vert \theta_{1}, \theta_{2}) \\
Y_{1} = min \{ x_{1} , ..., x_{n}\} \\
Y_{2} = min \{ \{ x_{1}, ..., x_{n}\} \backslash \{ Y_{1}\}\}
\\ . \\ . \\ . \\ \\
Y_{n} = max \{ X_{1}, ... , X_{n}\} \\ \\ 
\theta_{1} \leq Y_{1} \leq Y_{2} \leq ... \leq Y_{n} \leq \theta_{2}$
\begin{eqnarray*}
p(Y_{1}, ..., Y_{n}) = n! \ \ p_{x} (Y_{1}, ... ,Y_{n}) \\ \\
p(Y_{1}, Y_{3}, ... , Y_{n}) = n! \ \ \displaystyle \int_{y_{1}}^{Y_{3}} p_{x} (Y_{1},... Y_{n})dY_{z} = n! \ \  p_{x} (Y_{1}, Y_{3}, ..., Y_{n}) \displaystyle \int_{Y_{1}}^{Y_{3}} p_{x} (Y_{2}) d Y_{2} \\ \\
= n! \ \ p_{x} (Y_{1}, Y_{3}, ..., Y_{n}) \left\{  F_{x} (Y_{3}) - F_{x} (Y_{1})\right\} \\ \\
p(Y_{1}, Y_{4}, ..., Y_{n} ) = n! \ \ p_{x} (Y_{1}, Y_{4}, ..., Y_{n}) \displaystyle\int_{Y_{1}}^{Y_{4}} p_{x} (Y_{3}) \{ F_{x} (Y_{3}) - F_{x} (Y_{1})\} d Y_{3} \\ \\
= n! \ \ p_{x} () \dfrac{\{ F_{x} (Y_{3}) - F_{x} (Y_{1})\}^{2}}{2} \displaystyle]_{Y_{1}}^{Y_{4}} = n! \ \ p_{x} () \dfrac{\{ F_{x} (Y_{4}) - F_{x} (Y_{1})\}^{2}}{2}
\\
. \\ . \\ . \\
p(Y_{1}, Y_{n}) = n! p_{x} (Y_{1}) p_{x} (Y_{n}) \dfrac{\{ F_{x} (Y_{n}) - F_{x} (Y_{1})\}^{n-2}}{(n-2)!} \\
\\
Z_{1} = \frac{Y_{n} - Y_{1}}{ Y_{n} - \theta_{0}} \ \ \ \ , \ \ \ Z_{2} = Y_{n} - \theta_{0} \ \ \ , \ Y_{n} = \theta_{0} + Z_{2} \\ \\
Z_{1} = \frac{\theta + Z_{2} - Y_{1} }{Z_{2}} \ \ \ \ = \ \ \ \ - Z_{1} Z_{2} + \theta + Z_{2} = Y_{1} \\ \\
|J| = 
\begin{vmatrix}
 -Z_{2} & 1 - Z_{1} \\  
  0 & 1 
  \end{vmatrix} 
  = Z_{2} \ \ \ \ \ \ F_{x} (x) = \dfrac{x - \theta_{1}}{\theta_{2} - \theta_{1}}
\end{eqnarray*}
\begin{eqnarray*}
p(Z_{1}, Z_{2}) = \dfrac{n (n-1)}{(\theta_{2} - \theta_1)^{n}} \left\{ \dfrac{\theta_{0} + Z - \theta_{1}}{\theta_{2} - \theta_{1}} - \dfrac{\theta_{0} + Z_{2} - Z_{1} Z_{2} - \theta_{1}}{\theta_{2} - \theta_{1}} \right\}^{n-2} \\ \\
p(Z_{1}, Z_{2}) = \dfrac{n (n - 1)}{(\theta_{2} - \theta_{1})^{n}} \{ Z_{1} Z_{2} \}^{n-2} \\ 
p (Z_{1} \vert \theta_{1}, \theta_{2}) = \dfrac{n (n - 1)}{(\theta_{2} - \theta_{1})^{n}} Z_{1}^{n-2} \displaystyle\int_{0}^{\theta_{2} - \theta_{0}} Z_{2}^{n-2} dZ_{2}\\
= \dfrac{n (n-1)}{(\theta_{2} - \theta_{1})^{2}} Z_{1}^{n-2} \left( \dfrac{Z_{2}^{n-1}}{n-1}\right) \left|_{0}^{\theta_{2} - \theta_{1}}\right. = \dfrac{n (\theta_{2} - \theta_{0})^{n-1}}{(\theta_{2} - \theta_{1})^{2}} Z_{1}^{n-2} I_{0,1} (Z_{1})
\end{eqnarray*}
Bajo H \\
$p (Z_{1} \vert \theta_{1}, \theta_{2}, H) = p (Z_{1}) = \dfrac{n Z_{1}^{n-2}}{(\theta_{2} - \theta_{0})} I_{ (0,1) } (Z_{1})$ \\
Otro ejemplo:
\begin{eqnarray*}
X_{1}, ..., X_{n} \sim Ca (X \vert \theta) \\ \\
\ell (Z_{n} \vert \theta) = \Pi_{i=1}^{n} \dfrac{1}{\Pi (1 + (x_{1} - \theta)^{2})} \\
\log \ell (Z_{n} \vert \theta) =  -\sum_{i=0}^{n} \log (1 + (x_{i} - \theta)^{2}) \\ \\
\dfrac{d}{d \theta} m (Z_{n} \vert \theta) = + \displaystyle \sum_{i=1}^{n} \dfrac{2 (x_{i} - \theta)}{(1 + (x_{i} - \theta)^{2})} \\ \\
\displaystyle \sum_{i=1}^{n} \dfrac{(x_{i} - \hat{\theta} )}{1 + (x_{i} - \hat{\theta})^{2}} = 0
\end{eqnarray*}

Las prob que surgen en estad\'istica son integraci\'on obtener m\'aximos o minimos.
\\
\section{Integraci\'on}
Sea $X \sim N (X \vert 0 , 1)$ \\
$F_{x} (x) = \int_{-\infty}^{x} \dfrac{1}{\sqrt(2 \pi)} dt = E_{N(X \vert 0,1)} I_{(- \infty , x)} (t)$ \\
En general, nos interesa resolver.
\\ \\
$E_{x}\left[ g (x)\right] = \displaystyle \int_{-\infty}^{x} g(x) f(x) dx$ \ \ $\rightsquigarrow$ \ \ lo primero que hay que checar es que $E_{x} [g(x)] < \infty$ \\ \\
Leyes de los grandes n\'umeros \\ \\
$\{ X_{n} : n \int \nat \} u.a.i.id$ \ con \ $E [X_{n}] = \theta < \infty \\ \\ 
\dfrac{1}{n} \displaystyle \sum_{i=1}^{n} \longrightarrow \theta $ \\
\\ Desigua?dad de Cebychev
\\ \\
$
\Lambda \xi \in \mathcal{R^{+}}
\\ \\ P \left[ \vert \overline{X}_{n} - \theta \vert \geq \xi \right] \leq \dfrac{V (\overline{X}_{n})}{\xi^{2}}  
 $ \\ \\
 La idea es terminar $\theta$ \ con \ \ $\hat{\theta} = \dfrac{1}{n} \displaystyle \sum_{i=1}^{n} g(X_{i}) \\
 \{ X_{1} , ... , X_{n}\} \sim f(X) \\ \\ 
 \in [\hat{\theta}] = \theta \\ \\
 V (\hat{\theta}) = \dfrac{1}{n} \displaystyle \int_{- \infty}^{\infty} ( g(x) - \theta)^{2} f(x) dx = \dfrac{c (\theta)}{n} \\ \\  $ 
 Si \ \ $\xi = \sqrt{\dfrac{V[\hat{\theta_{n}}]}{\delta}} $ \ \ \ , \ \ \ $ \delta \in \mathcal{R^{+}} \\ \\
 P \left[ \vert \hat{\theta} - \theta \vert \geq \sqrt{\dfrac{V(\hat{\theta_{n}})}{\delta}}\right] \leq \dfrac{V(\hat{\theta_{n}} \delta)}{V[\hat{\theta_{n}}]} = \delta $ \\ \\
 Es decir \\ \\
 $\vert \hat{\theta} - \theta \vert \leq \sqrt{\dfrac{V (\hat{\theta_{n}})}{\delta}} = \sqrt{\dfrac{c(\theta)}{n \delta}} = O (n^{- \frac{1}{2}})$ \\ \\
 Con prob al menos $1-\delta$
 \\ La precisi\'on del estimador $\hat{\theta}$ depende fuertemente de su varianza \\
 \\ $\vert \hat{\theta_{n}} - \theta \vert \approx O (n^{-\frac{1}{2}}) \\ \\
 E_{n} [ I_{(-\infty,1)}] \\ 
 \hat{\theta} = \dfrac{1}{n} \displaystyle\sum_{i=1}^{n} I_{[-\infty, t]}  (x_{i}) \\ \\
 \hat{\theta_{n}} = \frac{j}{n} \\ \\
 I_{- \infty , t}(x_{i}) \sim B(\centerdot \vert  \Phi (t)) \\ \\
 n \hat{\theta_{n}} \sim bin (n \hat{\theta_{n}} \vert n , \Phi (t)) \\ \\
 Var (n \hat{\theta_{n}}) = n \Phi (t) [1 - \Phi (t)] = \frac{1}{n} \Phi (t) (1 - \Phi (t)) \\ \\
 t= 0 , \ \ \ \Phi(t) = 0.5 \\ \\
 V(\hat{\theta_{n}^{?}}) = \sqrt{\frac{1}{4n}}
 $ \\ \\ 
 tarea \\ \\
 $\Phi (t) \ \ \ t \in \{ 0, 1.64, 1.96\}$ \\
 
exactas en 4 cifras significativas \\
$X_{1}, ... , X_{n} \sim N (X \vert 0,1)$ \\ \\
Tarea \\ \\
$X \sim Cauchy (X \vert 0, 1 ) \\ \\
p(x) = \dfrac{1}{\pi (1 + x^{2})} ;  \theta = P [x > 2] = \displaystyle \int_{2}^{\infty} \dfrac{1}{\pi (1 + x^{2})} dx$
\\ \\
Estimar $\Phi (x) = \dfrac{1}{\sqrt{2 \pi}} \displaystyle\int_{- \infty}^{x} e^{- \frac{1}{2} t^{2}} dt $
\begin{eqnarray*}
= \displaystyle \int_{- \infty}^{\infty} I_{[- \infty, x ]} (t) N (t \vert 0, 1) dt \\
\\ = E [I_{(- \infty , x)} (t)] \\ \\
= \dfrac{1}{n} \displaystyle\int_{1}^{n} I_{(- \infty , x)} (t_{i})
\end{eqnarray*}
donde $\{ t_{1}, ..., t_{n}\} \sim N (t \vert 0 , 1)$ \\
\begin{enumerate}
\item Dadas $x$ \ y \ $n$ \ generar ${t_{1},...,t_{n}} \sim N (t \vert 0 , 1)$
\item Cortar cuantas $t's \leq x$ \ sean m
\item $\Phi (x) = \dfrac{m}{n} $
\end{enumerate}
De que tama?o tiene que ser "$n$" para tener exactas cuatro cifras significativas.
\\
En general
\\ \\
$\displaystyle \int_{-\infty}^{\infty} g(x) dx  = \int_{-\infty}^{\infty} h(x) p (x) dx$ \\ \\
con $p(x)$ \ funcion de desnsidad, de manera que \\
$\theta = E_{p} [h(x)] = \displaystyle\int_{-\infty}^{\infty} h(x) p(x) dx$ \\ Puede ser estimado por \\ \\
$\hat{\theta_{n}} = \dfrac{1}{n} \displaystyle\sum_{1}^{n} h (x_{i})$ \\ \\
con $\{ x_{1}, ..., x_{n} \} \sim p (x) $ \\ \\
$\hat{\theta_{n}}$ \ \ cumple con \\ \\
$E[\hat{\theta}] = \theta$ \ \ y \ \ $V [\hat{\theta_{n}}] = O (n^{-\frac{1}{2}}) = \vert \theta - \hat{\theta_{n}} \vert $ \\ \\ 
Ejemplo :
\\ 
\begin{eqnarray*}
\textrm{Si} \ \ X \sim Ca(X \vert 0, 1) \\ 
p(x) = \dfrac{1}{\pi (1 + x^{2})} \ \ , \ \ x \in \mathrm{R} \\ \\
\theta = p [X \geq 2] = \displaystyle\int_{2}^{\infty} \dfrac{dx}{\pi (1 + x^{2})} \\
\theta = \dfrac{acr Cot 2}{\pi} \approx 0.147534 \\
\\ \theta = \displaystyle\int_{- \infty}^{\infty} I_{(2, \infty)} (x) Ca (x \vert 0,1) dx = E [I_{(2, \infty)} (x)] \\ 
\\ \hat{\theta_{n}} = \dfrac{1}{n} \displaystyle\sum_{i=1}^{n} F_{(2, \infty)}(x_{i}),\\
\{x_{1},..., x_{n}\} \sim Ca(X \vert 0,1)\\
\Longrightarrow n \hat{\theta_{n}} \sim Bm (n \hat{\theta} \vert n , \theta) \\ 
\Longrightarrow \not V \not\vert \not\hat{\theta} V[n \hat{\theta_{n}}] = n \theta (1 - \theta) \\
V[]\hat{\theta_{n}} = \dfrac{\theta (1 - \theta)}{n} = \dfrac{0.1258}{n}
\end{eqnarray*}
Grafica
\begin{eqnarray*}
2 \theta = 1 - \displaystyle\int_{-2}^{2} \dfrac{dx}{\pi (1 + x^{2})} = P [\vert X \vert \geq 2] \\
\theta = \dfrac{1}{2} - \dfrac{1}{2} \displaystyle\int_{-2}^{2} \dfrac{dx}{\pi (1 + x^{2})} = \dfrac{1}{2} - 2 \displaystyle\int_{-2}^{2} \dfrac{dx}{4 \pi (1 + x^{2})} \\
U (X \vert -2 , 2) = \dfrac{1}{4} \\
\theta = \dfrac{1}{2} - \dfrac{2}{\pi} E_{u} \left[ \dfrac{1}{1 + x^{2}} \right] \\
\hat{\theta_{1}} = \dfrac{1}{2n} \displaystyle\sum_{i=1}^{n} I_{(-2 , 2)^{c}} (x_{i}) \ \ \ \ \{ x_{1}, ... , x_{n}\} \sim Ca (X \vert 0,1) \\
n2 \hat{\theta_{1}} \sim Bin (2n \hat{\theta} \vert n , 2 \theta) \\ \\ 
Var (\hat{\theta_{1}}) = \dfrac{2n \theta (1 - 2 \theta)}{4 n^{2}} = \dfrac{\theta(1 - 2 \theta)}{2 n} = \dfrac{0.0520}{n} \\ \\
\theta = \dfrac{1}{2} - \dfrac{1}{2} \displaystyle\int_{-2}^{2} \dfrac{dx}{\pi (1 + x^{2})} = \dfrac{1}{2} - \int_{0}^{2} \dfrac{dx}{\pi (1 + x^{2})} = \dfrac{1}{2} - 2 \int_{0}^{2} \dfrac{dx}{2 \pi (1 + x^{2})} \\ \\
\theta = \dfrac{1}{2} - \dfrac{2}{\pi} E_{p*} [(1 + x^{2})^{-1}] \\ \\
p^{*} (x) = U (x \vert 0 ,2) \\ \\
\hat{\theta_{2}} = \dfrac{1}{2} - \dfrac{2}{\pi} \dfrac{1}{n} \sum_{i=1}^{n} \dfrac{1}{1 + x_{i}^{2}} \ \ \ \ \ \ \ \{ x_{1}, ... , x_{n}\} \sim U (X \vert 0,2) \\ \\
Var (\hat{\theta_{2}}) = \dfrac{4n}{n^{2} \pi^{2}} Var \left( \dfrac{1}{1 + x^{2}}\right) = \dfrac{4}{n \pi^{2}} \left\{ \dfrac{1}{2} \int_{0}^{2} \dfrac{dx}{(1 + x^{2})^{2}} - \left[ \dfrac{1}{2} \int_{0}^{2} \dfrac{dx}{(1 + x^{2})} \right]^{2}\right\} \approx \dfrac{0.0285}{n}
\end{eqnarray*}
\\ \\ \\ Sea $Y = \dfrac{1}{X} \Longrightarrow \ \ dx = - \dfrac{dy}{y^{2}}
\\ \\ \Longrightarrow \theta = \displaystyle\int_{0}^{\frac{1}{2}} \dfrac{1}{\Pi} \dfrac{y^{-2} dy}{(1 - y^{-2})} = \dfrac{1}{\Pi} \int_{0}^{\frac{1}{2}} \dfrac{dy}{1 + y^{2}} = \dfrac{1}{2 \pi} \int_{0}^{\frac{1}{2}} \dfrac{2 dy}{1 + y^{2}}$ \\ \\
$\hat{\theta_{3}} = \dfrac{1}{2 n \Pi} \displaystyle\sum_{i=1}^{n} \dfrac{1}{1 + x_{i}^{2}} \\ \\
Var (\hat{\theta_{3}}) = \dfrac{1}{4 n \Pi^{2}} \left\{ 2 \ \int_{0}^{\frac{1}{2}} \dfrac{dx}{(1 + x^{2})^{2}} - 4 \left[ \int_{0}^{\frac{1}{2}} \dfrac{dx}{(1 + x^{2})}\right]^{2} \right\} \approx \dfrac{9.55 * 10^{-5}}{n}$
\\ ?C\'omo reducir varianza en gral? \\ \\
$\theta = \displaystyle\int_{- \infty}^{\infty} h(x) dx = \int_{-\infty}^{\infty} = \dfrac{h(x)}{p(x)} p(x) dx $ \ \ \ \ \ Muestreo por importancia \\
$ \theta = E_{p} \left[\dfrac{h (x)}{p (x)} \right] \\
\\ \hat{\theta} = \dfrac{1}{n} = \displaystyle\sum_{i=1}^{n} \dfrac{h(x_{i})}{p (x_{i})}$ \ \ \ \ \ donde $\{ x_{1}, ..., x_{n}\} \sim p(x)$ \\ \\
$V[\hat{\theta_{n}}] = \displaystyle\int_{-\infty}^{\infty}( \hat{\theta_{n}} - \theta)^{2} p(x) dx$ \\ \\
$V[\hat{\theta_{n}}] = \dfrac{1}{n} V \left[ \dfrac{h(x)}{p (x)}\right] = \dfrac{1}{n} \int \left( \dfrac{h(x)}{p (x)} - \theta \right)^{2} p (x) dx$ \\ \\
$= \dfrac{1}{n} \left\{  \displaystyle\int \dfrac{h^{2} (x)}{p (2)} dx - \theta^{2} \right\}$ \\ \\
$= \dfrac{1}{n} \left\{ \int \left[ \dfrac{h(x)}{p(x)}\right]^{2} p(x) dx - \theta^{2} \right\} \geq 0$ \\ \\ 
$\Longrightarrow \int \left[  \dfrac{h(x)}{p (x)}\right]^{2} p (x) dx \geq \theta^{2}$ \ \ \ \ \ se minimiza cuando $h$ se parece a $p$ \\ \\
 $g(x) = \dfrac{1}{4 x^{2}} \\ \int_{2}^{\infty} g(x)dx = \dfrac{1}{4} \left\{ - x^{-1} \right\}_{2}^{\infty
 } = \dfrac{1}{8}$ \\ 
 Sea $p(x)= \dfrac{2}{x^{2}}$ \ \ \ , \ \ \ $x \in [2, \infty) \\ \\
 \theta = \displaystyle\int_{2}^{\infty}
\dfrac{x^{2}}{2 \Pi (1 + x^{2})} \ \  \dfrac{2}{x^{2}} dx \\ \\
$








\chapter{Procesos de Renovacion}
%


\chapter{Revision Procesos Regenerativos}
%
%________________________________________________________________________
\section{Procesos Regenerativos: Thorisson}
%________________________________________________________________________
%________________________________________________________________________
\subsection{Tiempos de Regeneraci\'on para Redes de Sistemas de Visitas C\'iclicas}
%________________________________________________________________________
\begin{Teo}
Dada una Red de Sistemas de Visitas C\'iclicas (RSVC), conformada por dos Sistemas de Visitas C\'iclicas (SVC), donde cada uno de ellos consta de dos colas tipo $M/M/1$. Los dos sistemas est\'an comunicados entre s\'i por medio de la transferencia de usuarios entre las colas $Q_{1}\leftrightarrow Q_{3}$ y $Q_{2}\leftrightarrow Q_{4}$.

\end{Teo}

\begin{proof}

Para cada cola $Q_{j}$, $j=1,\ldots,4$, se tienen los siguientes procesos $L_{j}\left(t\right)$ el n\'umero de usuarios presentes en la cola al tiempo $t$, $A_{j}\left(t\right)$ el residual del tiempo de arribo del siguiente usuario. $B_{j}\left(t\right)$ el residual del tiempo de servicio del usuario que est\'a siendo atendido. $C_{j}\left(t\right)$ el residual del tiempo de traslado del servidor entre una cola y otra, en caso de que se encuentre dando servicio se considera $C_{j}\left(t\right)=0$, para $j=1,\ldots,4$. Con base en lo anterior se tienen los procesos
\begin{eqnarray}\label{Procesos.RSVC}
L\left(t\right)=\left(L_{j}\left(t\right)\right)_{j=1}^{4},
A\left(t\right)=\left(A_{j}\left(t\right)\right)_{j=1}^{4}, B\left(t\right)=\left(B_{j}\left(t\right)\right)_{j=1}^{4}
\textrm{ y } C\left(t\right)=\left(C_{j}\left(t\right)\right)_{j=1}^{4}.
\end{eqnarray}
Por lo tanto se tiene el proceso estoc\'astico
\begin{eqnarray}\label{Proceso.Estocastico.Z}
\mathbb{Z}=\left(L\left(t\right),A\left(t\right),
B\left(t\right),C\left(t\right)\right)
\end{eqnarray}
Para los procesos residuales de los tiempos de traslado, servicio y de arribos, su espacio de estados es un subconjunto de $\rea_{+}=\left[0,\infty\right)$, es decir, $E\subset\left[0,\infty\right)$ y por tanto $\mathcal{E}\subset\mathcal{B}\left[0,\infty\right)$, luego el espacio $\left(E,\mathcal{E}\right)$ es un espacio polaco.
Para cada proceso de residuales se tienen los siguientes espacios producto: Para $A\left(t\right)=\left(A_{j}\left(t\right)\right)_{j=1}^{4}$ se tiene el espacio producto $\left(E_{2},\mathcal{E}_{2}\right)=\otimes_{j=1}^{4}\left(E_{j},\mathcal{E}_{j}\right)$, para $B\left(t\right)=\left(B_{j}\left(t\right)\right)_{j=1}^{4}$ se tiene el espacio producto $\left(E_{3},\mathcal{E}_{3}\right)=\otimes_{j=1}^{4}\left(E_{j},\mathcal{E}_{j}\right)$,
para $C\left(t\right)=\left(C_{j}\left(t\right)\right)_{j=1}^{4}$ se tiene el espacio producto $\left(E_{4},\mathcal{E}_{4}\right)=\otimes_{j=1}^{4}\left(E_{j},\mathcal{E}_{j}\right)$.

En lo que respecta al proceso $L\left(t\right)=\left(L_{j}\left(t\right)\right)_{j=1}^{4}$
 el proceso de estados $E_{j}\subset\mathbb{N}$ y $\mathcal{E}_{j}\subset\sigma\left(E\right)$, por lo tanto el espacio producto $\left(E_{1},\mathcal{E}_{1}\right)=\otimes_{j=1}^{4}\left(E_{j},\mathcal{E}_{j}\right)$ que adem\'as tambi\'en resulta ser polaco. Entonces con los espacios productos $\left(E_{i},\mathcal{E}_{i}\right)_{i=1}^{4}$, se define el espacio producto $\left(E,\mathcal{E}\right)=\otimes_{i=1}^{4} \left(E_{i},\mathcal{E}_{i}\right)$ que nuevamente resulta ser un espacio polaco. De acuerdo con Thorisson existe un espacio de probabilidad $\left(\Omega,\mathcal{F},\prob\right)$ en el que est\'a definido el proceso estoc\'astico definido en  (\ref{Proceso.Estocastico.Z}) que toma valores en $\left(E,\mathcal{E}\right)$.
  
Con la finalidad de analizar las propiedades del proceso $\mathbb{Z}$ consideremos el conjunto de \'indices $\mathbb{I}=\left[0,\infty\right)$, entonces tenemos el elemento aleatorio $\mathbb{Z}=\left(Z\right)_{s\in\mathbb{I}}$ que est\'a definido en el espacio de probabilidad $\left(\Omega,\mathcal{F},\prob\right)$ y con valores en $\left(E,\mathcal{E}\right)$. El proceso $Z$ as\'i definido es un PEOSCT conforme a la definici\'on dada en (\ref{PEOSCT}). Ahora consideremos al espacio de trayectorias de $Z$ conforme a la definici\'on (\ref{Conjunto.Trayectorias}); por construcci\'on el espacio de trayectorias $H:=D_{E}\left[0,\infty\right)$ que por la nota (\ref{Conjunto.Trayectorias}) resulta ser que el Proceso es Canonicamente Conjuntamente Medible (CCM) y por la nota (\ref{Nota.ISI.sii.CCM}) adem\'as es Internamente Shift Invariante (ISI), es decir, resulta ser un proceso estoc\'astico one-side a tiempo continuo shift medible, y por lo tanto satisface la primera parte de las hip\'otesis del Teorema (\ref{Tma.Existencia.Tiempos.Regeneracion}). 

Conforme a la construcci\'on dada en la secci\'on 1, se tiene que los dos tiempos $S_{0}=0$ y $S_{1}=T^{*}$ satisfacen la segunda parte de las hip\'otesis del Teorema (\ref{Tma.Existencia.Tiempos.Regeneracion}) y por tanto se puede asegurar que existe un espacio de probabilidad $\left(\Omega,\mathcal{F},\prob\right)$ en el cu\'al existe una sucesi\'on de tiempos aleatorios en los cuales el proceso se regenera, es decir, se garantiza que existe una sucesi\'on de tiempos de regeneraci\'on $T_{0}, T_{1},\ldots$ en los cuales el proceso $L\left(T_{k}\right)=\left(0,0,0,0\right)$.

Adem\'as por el Corolario (\ref{Tma.Estacionariedad}) se garantiza que existe una versi\'on estacionaria del proceso $\left(Z,S\right)$.
\end{proof}

\newpage

%_________________________________________________________________________
\subsection{Introduction to Stochastic Processes}
%_________________________________________________________________________

\begin{Def}
Un elemento aleatorio con valores en un espacio medible $\left(E,\mathcal{E}\right)$, es un mapeo definido en un espacio de probabilidad $\left(\Omega,\mathcal{F},\prob\right)$ a $\left(E,\mathcal{E}\right)$, es decir,
para $A\in \mathcal{E}$,  se tiene que $\left\{Y\in A\right\}\in\mathcal{F}$, donde $\left\{Y\in A\right\}:=\left\{w\in\Omega:Y\left(w\right)\in A\right\}=:Y^{-1}A$.
\end{Def}

\begin{Note}
Tambi\'en se dice que $Y$ est\'a soportado por el espacio de probabilidad $\left(\Omega,\mathcal{F},\prob\right)$ y que $Y$ es un mapeo medible de $\Omega$ en $E$, es decir, es \textbf{$\mathcal{F}/\mathcal{E}$ medible}.
\end{Note}

\begin{Def}
Para cada $i\in \mathbb{I}$, sea $P_{i}$ una medida de probabilidad en un espacio medible $\left(E_{i},\mathcal{E}_{i}\right)$. Se define el espacio producto
$\otimes_{i\in\mathbb{I}}\left(E_{i},\mathcal{E}_{i}\right):=\left(\prod_{i\in\mathbb{I}}E_{i},\otimes_{i\in\mathbb{I}}\mathcal{E}_{i}\right)$, donde $\prod_{i\in\mathbb{I}}E_{i}$ es el producto cartesiano de los $E_{i}$'s, y $\otimes_{i\in\mathbb{I}}\mathcal{E}_{i}$ es la \textbf{$\sigma$-\'algebra producto}, es decir, es la $\sigma$-\'algebra m\'as peque\~na en $\prod_{i\in\mathbb{I}}E_{i}$ que hace al $i$-\'esimo mapeo proyecci\'on en $E_{i}$ medible para toda $i\in\mathbb{I}$, es la $\sigma$-\'algebra inducida por los mapeos proyecci\'on, es decir
$$\otimes_{i\in\mathbb{I}}\mathcal{E}_{i}:=\sigma\left\{\left\{y:y_{i}\in A\right\}:i\in\mathbb{I}\textrm{ y }A\in\mathcal{E}_{i}\right\}.$$
\end{Def}

\begin{Def}
Un espacio de probabilidad $\left(\tilde{\Omega},\tilde{\mathcal{F}},\tilde{\prob}\right)$ es una \textbf{extensi\'on de otro espacio de probabilidad $\left(\Omega,\mathcal{F},\prob\right)$} si $\left(\tilde{\Omega},\tilde{\mathcal{F}},\tilde{\prob}\right)$ soporta un elemento aleatorio $\xi\in\left(\Omega,\mathcal{F}\right)$ que tienen a $\prob$ como distribuci\'on.
\end{Def}

\begin{Teo}
Sea $\mathbb{I}$ un conjunto de \'indices arbitrario. Para cada $i\in\mathbb{I}$ sea $P_{i}$ una medida de probabilidad en un espacio medible $\left(E_{i},\mathcal{E}_{i}\right)$. Entonces existe una \'unica medida de probabilidad $\otimes_{i\in\mathbb{I}}P_{i}$ en $\otimes_{i\in\mathbb{I}}\left(E_{i},\mathcal{E}_{i}\right)$ tal que 

\begin{eqnarray*}
\otimes_{i\in\mathbb{I}}P_{i}\left(y\in\prod_{i\in\mathbb{I}}E_{i}:y_{i}\in A_{i_{1}},\ldots,y_{n}\in A_{i_{n}}\right)=P_{i_{1}}\left(A_{i_{n}}\right)\cdots P_{i_{n}}\left(A_{i_{n}}\right)
\end{eqnarray*}
para todos los enteros $n>0$, toda $i_{1},\ldots,i_{n}\in\mathbb{I}$ y todo $A_{i_{1}}\in\mathcal{E}_{i_{1}},\ldots,A_{i_{n}}\in\mathcal{E}_{i_{n}}$
\end{Teo}

La medida $\otimes_{i\in\mathbb{I}}P_{i}$ es llamada la \textbf{medida producto} y $\otimes_{i\in\mathbb{I}}\left(E_{i},\mathcal{E}_{i},P_{i}\right):=\left(\prod_{i\in\mathbb{I}},E_{i},\otimes_{i\in\mathbb{I}}\mathcal{E}_{i},\otimes_{i\in\mathbb{I}}P_{i}\right)$, es llamado \textbf{espacio de probabilidad producto}.


\begin{Def}
Un espacio medible $\left(E,\mathcal{E}\right)$ es \textbf{\textit{Polaco}} si existe una m\'etrica en $E$ tal que $E$ es completo, es decir cada sucesi\'on de Cauchy converge a un l\'imite en $E$, y \textit{separable}, $E$ tienen un subconjunto denso numerable, y tal que $\mathcal{E}$ es generado por conjuntos abiertos.
\end{Def}


\begin{Def}
Dos espacios medibles $\left(E,\mathcal{E}\right)$ y $\left(G,\mathcal{G}\right)$ son Borel equivalentes (\textit{isomorfos}) si existe una biyecci\'on $f:E\rightarrow G$ tal que $f$ es $\mathcal{E}/\mathcal{G}$ medible y su inversa $f^{-1}$ es $\mathcal{G}/\mathcal{E}$ medible. La biyecci\'on es una equivalencia de Borel.
\end{Def}

\begin{Def}
Un espacio medible  $\left(E,\mathcal{E}\right)$ es un \textbf{espacio est\'andar} si es Borel equivalente a $\left(G,\mathcal{G}\right)$, donde $G$ es un subconjunto de Borel de $\left[0,1\right]$ y $\mathcal{G}$ son los subconjuntos de Borel de $G$.
\end{Def}

\begin{Note}
Cualquier espacio polaco es un espacio est\'andar.
\end{Note}


\begin{Def}
Un proceso estoc\'astico con conjunto de \'indices $\mathbb{I}$ y espacio de estados $\left(E,\mathcal{E}\right)$ es una familia $Z=\left(\mathbb{Z}_{s}\right)_{s\in\mathbb{I}}$ donde $\mathbb{Z}_{s}$ son elementos aleatorios definidos en un espacio de probabilidad com\'un $\left(\Omega,\mathcal{F},\prob\right)$ y todos toman valores en $\left(E,\mathcal{E}\right)$.
\end{Def}

\begin{Def}\label{PEOSCT}
Un proceso estoc\'astico \textit{one-sided contiuous time} (\textbf{PEOSCT}) es un proceso estoc\'astico con conjunto de \'indices $\mathbb{I}=\left[0,\infty\right)$.
\end{Def}


El espacio $\left(E^{\mathbb{I}},\mathcal{E}^{\mathbb{I}}\right)$ denota el espacio producto $\left(E^{\mathbb{I}},\mathcal{E}^{\mathbb{I}}\right):=\otimes_{s\in\mathbb{I}}\left(E,\mathcal{E}\right)$. Vamos a considerar $\mathbb{Z}$ como un mapeo aleatorio, es decir, como un elemento aleatorio en $\left(E^{\mathbb{I}},\mathcal{E}^{\mathbb{I}}\right)$ definido por $Z\left(w\right)=\left(Z_{s}\left(w\right)\right)_{s\in\mathbb{I}}$ y $w\in\Omega$.

\begin{Note}
La distribuci\'on de un proceso estoc\'astico $Z$ es la distribuci\'on de $Z$ como un elemento aleatorio en $\left(E^{\mathbb{I}},\mathcal{E}^{\mathbb{I}}\right)$. La distribuci\'on de $Z$ esta determinada de manera \'unica por las distribuciones finito dimensionales.
\end{Note}

\begin{Note}
En particular cuando $Z$ toma valores reales, es decir, $\left(E,\mathcal{E}\right)=\left(\mathbb{R},\mathcal{B}\right)$ las distribuciones finito dimensionales est\'an determinadas por las funciones de distribuci\'on finito dimensionales

\begin{eqnarray}
\prob\left(Z_{t_{1}}\leq x_{1},\ldots,Z_{t_{n}}\leq x_{n}\right),x_{1},\ldots,x_{n}\in\mathbb{R},t_{1},\ldots,t_{n}\in\mathbb{I},n\geq1.
\end{eqnarray}
\end{Note}

\begin{Note}
Para espacios polacos $\left(E,\mathcal{E}\right)$ el \textbf{Teorema de Consistencia de Kolmogorov} asegura que dada una colecci\'on de distribuciones finito dimensionales consistentes, siempre existe un proceso estoc\'astico que posee tales distribuciones finito dimensionales.
\end{Note}


\begin{Def}\label{Conjunto.Trayectorias}
Las trayectorias de $Z$ son las realizaciones $Z\left(w\right)$ para $w\in\Omega$ del mapeo aleatorio $Z$.
\end{Def}

\begin{Note}
Algunas restricciones se imponen sobre las trayectorias, por ejemplo que sean continuas por la derecha, o continuas por la derecha con l\'imites por la izquierda, o de manera m\'as general, se pedir\'a que caigan en alg\'un subconjunto $H$ de $E^{\mathbb{I}}$. En este caso es natural considerar a $Z$ como un elemento aleatorio que no est\'a en $\left(E^{\mathbb{I}},\mathcal{E}^{\mathbb{I}}\right)$ sino en $\left(H,\mathcal{H}\right)$, donde $\mathcal{H}$ es la $\sigma$-\'algebra generada por los mapeos proyecci\'on que toman a $z\in H$ en $z_{t}\in E$ para $t\in\mathbb{I}$. A $\mathcal{H}$ se le conoce como la traza de $H$ en $E^{\mathbb{I}}$, es decir,
\begin{eqnarray}
\mathcal{H}:=E^{\mathbb{I}}\cap H&:=&\left\{A\cap H:A\in E^{\mathbb{I}}\right\}.\\
Z_{t}:\left(\Omega.\mathcal{F}\right)&\rightarrow&\left(H,\mathcal{H}\right)
\end{eqnarray}
\end{Note}


\begin{Note}
$Z$ tiene \textbf{trayectorias con valores en $H$} y cada $Z_{t}$ es un mapeo medible de $\left(\Omega,\mathcal{F}\right)$ a $\left(H,\mathcal{H}\right)$. Cuando se considera un espacio de trayectorias en particular $H$, al espacio $\left(H,\mathcal{H}\right)$ se le llama \textbf{el espacio de trayectorias de $Z$}.
\end{Note}

\begin{Note}
La distribuci\'on del proceso estoc\'astico $Z$ con espacio de trayectorias $\left(H,\mathcal{H}\right)$ es la distribuci\'on de $Z$ como  un elemento aleatorio en $\left(H,\mathcal{H}\right)$. La distribuci\'on, nuevemente, est\'a determinada de manera \'unica por las distribuciones finito dimensionales.
\end{Note}


\begin{Def}
Sea $Z$ un PEOSCT (ver definici\'on \ref{PEOSCT}) con espacio de estados $\left(E,\mathcal{E}\right)$ y sea $T$ un tiempo aleatorio en $\left[0,\infty\right)$. Por $Z_{T}$ se entiende el mapeo con valores en $E$ definido en $\Omega$ por:
\begin{eqnarray*}
Z_{T}\left(w\right)&:=&Z_{T\left(w\right)}\left(w\right), w\in\Omega.\\
Z_{t}:\left(\Omega,\mathcal{F}\right)&\rightarrow&\left(E,\mathcal{E}\right).
\end{eqnarray*}
\end{Def}

\begin{Def}
Un PEOSCT $Z$ es conjuntamente medible (\textbf{CM}), es decir un \textbf{PEOSCTCM}, si el mapeo que toma $\left(w,t\right)\in\Omega\times\left[0,\infty\right)$ a $Z_{t}\left(w\right)\in E$ es $\mathcal{F}\otimes\mathcal{B}\left[0,\infty\right)/\mathcal{E}$ medible.
\begin{eqnarray*}
\left(\Omega,\left[0,\infty\right)\right)&\rightarrow&\left(E,\mathcal{E}\right)\\
\left(w,t\right)&\rightarrow& Z_{t}\left(w\right).
\end{eqnarray*}
\end{Def}

\begin{Note}
Un PEOSCT-CM implica que el proceso es medible, dado que $Z_{T}$ es una composici\'on  de dos mapeos continuos: el primero que toma $w$ en $\left(w,T\left(w\right)\right)$ es $\mathcal{F}/\mathcal{F}\otimes\mathcal{B}\left[0,\infty\right)$ medible, mientras que el segundo toma $\left(w,T\left(w\right)\right)$ en $Z_{T\left(w\right)}\left(w\right)$ es $\mathcal{F}\otimes\mathcal{B}\left[0,\infty\right)/\mathcal{E}$ medible.
\end{Note}


\begin{Def}
Un PEOSCT con espacio de estados $\left(H,\mathcal{H}\right)$ es can\'onicamente conjuntamente medible (\textbf{CCM}) si el mapeo $\left(z,t\right)\in H\times\left[0,\infty\right)$ en $Z_{t}\in E$ es $\mathcal{H}\otimes\mathcal{B}\left[0,\infty\right)/\mathcal{E}$ medible.
\begin{eqnarray*}
\left(H\times\left[0,\infty\right),\mathcal{H}\times\mathcal{B}\left[0,\infty\right)\right)&\rightarrow& \left(E,\mathcal{E}\right)\\
\left(z,t\right)&\rightarrow& Z_{t}
\end{eqnarray*}
\end{Def}

\begin{Note}
Un PEOSCTCCM implica que el proceso es CM, dado que un PEOSCTCCM $Z$ es un mapeo de $\Omega\times\left[0,\infty\right)$ a $E$, es la composici\'on de dos mapeos medibles: el primero, toma $\left(w,t\right)$ en $\left(Z\left(w\right),t\right)$ es $\mathcal{F}\otimes\mathcal{B}\left[0,\infty\right)/\mathcal{H}\otimes\mathcal{B}\left[0,\infty\right)$ medible, y el segundo que toma $\left(Z\left(w\right),t\right)$  en $Z_{t}\left(w\right)$ es $\mathcal{H}\otimes\mathcal{B}\left[0,\infty\right)/\mathcal{E}$ medible. Por tanto CCM es una condici\'on m\'as fuerte que CM.
\begin{eqnarray*}
\left(\Omega\times\left[0,\infty\right),\mathcal{F}\times\mathcal{B}\left[0,\infty\right)\right)
&\rightarrow& 
\left(H\times\left[0,\infty\right),\mathcal{H}\times\mathcal{B}\left[0,\infty\right)\right)
\rightarrow\left(E,\mathcal{E}\right)\\
\left(w,t\right)&\rightarrow& 
\left(Z\left(w\right),t\right])\rightarrow Z_{t}\left(w\right)
\end{eqnarray*}

\end{Note}

\begin{Def}
Un conjunto de trayectorias $H$ de un PEOSCT $Z$, es internamente shift-invariante (\textbf{ISI}) si 
\begin{eqnarray*}
\left\{\left(z_{t+s}\right)_{s\in\left[0,\infty\right)}:z\in H\right\}=H\textrm{, }t\in\left[0,\infty\right).
\end{eqnarray*}
\end{Def}


\begin{Def}
Dado un PEOSCTISI, se define el mapeo-shift $\theta_{t}$, $t\in\left[0,\infty\right)$, de $H$ a $H$ por 
\begin{eqnarray*}
\theta_{t}z=\left(z_{t+s}\right)_{s\in\left[0,\infty\right)}\textrm{, }z\in H.
\end{eqnarray*}
\end{Def}

\begin{Def}
Se dice que un proceso $Z$ es shift-medible (\textbf{SM}) si $Z$ tiene un conjunto de trayectorias $H$ que es ISI y adem\'as el mapeo que toma $\left(z,t\right)\in H\times\left[0,\infty\right)$ en $\theta_{t}z\in H$ es $\mathcal{H}\otimes\mathcal{B}\left[0,\infty\right)/\mathcal{H}$ medible.
\begin{eqnarray*}
\left(H\times\left[0,\infty\right),\mathcal{H}\times\mathcal{B}\left[0,\infty\right)\right)
&\rightarrow& 
\left(H,\mathcal{H}\right)\\
\left(z,t\right)&\rightarrow& 
\theta_{t}\left(z\right)
\end{eqnarray*}

\end{Def}

\begin{Note}\label{Nota.ISI.sii.CCM}
Un proceso estoc\'astico (PEOSCT) con conjunto de trayectorias $H$ ISI es shift-medible si y s\'olo si es PEOSCTCCM.
\end{Note}

\begin{Note}\label{Nota.ISI.CCM}
\begin{itemize}
\item Por la nota (\ref{Nota.ISI.sii.CCM}) dado el espacio polaco $\left(E,\mathcal{E}\right)$ si se tiene el  conjunto de trayectorias $D_{E}\left[0,\infty\right)$, que es ISI, entonces cumple con ser CCM.

\item Si $G$ es abierto, podemos cubrirlo por bolas abiertas cuya cerradura este contenida en $G$, y como $G$ es segundo numerable como subespacio de $E$, lo podemos cubrir por una cantidad numerable de bolas abiertas.

\end{itemize}
\end{Note}


\begin{Note}
Los procesos estoc\'asticos $Z$ a tiempo discreto con espacio de estados polaco, tambi\'en tiene un espacio de trayectorias polaco y por tanto tiene distribuciones condicionales regulares.
\end{Note}

\begin{Teo}
El producto numerable de espacios polacos es polaco.
\end{Teo}

%__________________________________________________________
\subsection{One Sided Process}
%___________________________________________________________

%\begin{Def}
Sea $\left(\Omega,\mathcal{F},\prob\right)$ espacio de probabilidad que soporta al proceso $Z=\left(Z_{s}\right)_{s\in\left[0,\infty\right)}$ y $S=\left(S_{k}\right)_{0}^{\infty}$ donde $Z$ es un PEOSCTM con espacio de estados $\left(E,\mathcal{E}\right)$  y espacio de trayectorias $\left(H,\mathcal{H}\right)$  y adem\'as $S$ es una sucesi\'on de tiempos aleatorios one-sided que satisfacen la condici\'on $0\leq S_{0}<S_{1}<\cdots\rightarrow\infty$. Considerando $S$ como un mapeo medible de $\left(\Omega,\mathcal{F}\right)$ al espacio sucesi\'on $\left(L,\mathcal{L}\right)$, $S:\left(\Omega,\mathcal{F}\right)\rightarrow\left(L,\mathcal{L}\right)$, donde 
\begin{eqnarray*}
L=\left\{\left(s_{k}\right)_{0}^{\infty}\in\left[0,\infty\right)^{\left\{0,1,\ldots\right\}}:s_{0}<s_{1}<\cdots\rightarrow\infty\right\},
\end{eqnarray*}
donde $\mathcal{L}$ son los subconjuntos de Borel de $L$, es decir, $\mathcal{L}=L\cap\mathcal{B}^{\left\{0,1,\ldots\right\}}$.

As\'i el par $\left(Z,S\right)$ es un mapeo medible de  $\left(\Omega,\mathcal{F}\right)$ en $\left(H\times L,\mathcal{H}\otimes\mathcal{L}\right)$. El par $\mathcal{H}\otimes\mathcal{L}^{+}$ denotar\'a la clase de todas las funciones medibles de $\left(H\times L,\mathcal{H}\otimes\mathcal{L}\right)$ en $\left(\left[0,\infty\right),\mathcal{B}\left[0,\infty\right)\right)$.
%\end{Def}

\begin{eqnarray*}
\left(Z,S\right):\left(\Omega,\mathcal{F}\right)&\rightarrow& \left(H\times L,\mathcal{H}\times\mathcal{L}\right)\\
\mathcal{H}\times\mathcal{L}^{*}:\left(H\times L,\mathcal{H}\times\mathcal{L}\right)
&\rightarrow& 
\left(\left[0,\infty\right),\mathcal{B}\left[0,\infty\right)\right).
\end{eqnarray*}



%_________________________________________________________
\subsection{Regeneration: Shift-Measurability}
%__________________________________________________________

\begin{Def}
Sea $\theta_{t}$ el mapeo-shift conjunto de $H\times L$ en $H\times L$ dado por
\begin{eqnarray*}
\theta_{t}\left(z,\left(s_{k}\right)_{0}^{\infty}\right)=\theta_{t}\left(z,\left(s_{n_{t-}+k}-t\right)_{0}^{\infty}\right)
\end{eqnarray*}
donde 
$n_{t-}=inf\left\{n\geq1:s_{n}\geq t\right\}$.
\end{Def}


\begin{Note}
Con la finalidad de poder realizar los shift's sin complicaciones de medibilidad, se supondr\'a que $Z$ es shit-medible, es decir, el conjunto de trayectorias $H$ es invariante bajo shifts del tiempo y el mapeo que toma $\left(z,t\right)\in H\times\left[0,\infty\right)$ en $z_{t}\in E$ es $\mathcal{H}\otimes\mathcal{B}\left[0,\infty\right)/\mathcal{E}$ medible.
\end{Note}




%_________________________________________________________
\subsection{Cycle-Stationarity}
%_________________________________________________________
%\textit{\textbf{Faltan definiciones}}
\begin{Def}
Los tiempos aleatorios $S_{n}$ dividen $Z$ en 

\begin{itemize}
\item[a)] un retraso $D=\left(Z_{s}\right)_{s\in\left[0,\infty\right)}$,
\item[b)] una sucesi\'on de ciclos $C_{n}=\left(Z_{S_{n-1}+s}\right)_{
s\in\left[0,X_{n}\right)}$, $n\geq1$,
\item[c)] las longitudes de los ciclos $X_{n}=S_{n}-S_{n-1}$, $n\neq1$.
\end{itemize}
\end{Def}

\begin{Note}
\begin{itemize}
\item[a)] El retraso $D$ y los ciclos $C_{n}$ son procesos estoc\'asticos que se desvanecen en los tiempos aleatorios $S_{0}$ y $X_{n}$ respectivamente.
\item[b)] Las longitudes de los ciclos $X_{1},X_{2},\ldots$ y el retraso de la longitud (\textit{delay-length}) $S_{0}$ son obtenidos por el mismo mapeo medible de sus respectivos ciclos $C_{1},C_{2},\ldots$ y el retraso $D$. 
\item[c)] El par $\left(Z,S\right)$ es un mapeo medible del retraso y de los ciclos y viceversa.
\end{itemize}
\end{Note}

\begin{Def}
$\left(Z,S\right)$ es \textit{zero-delayed} si $S_{0}\equiv0$. Se define el par \textit{zero-delayed} por
\begin{eqnarray*}
\left(Z^{0},S^{0}\right):=\theta_{S_{0}}\left(Z,S\right)
\end{eqnarray*}
Entonces $S_{0}^{0}\equiv0$ y $S_{0}^{0}\equiv X_{1}^{0}$, mientras que para $n\geq1$ se tiene que $X_{n}^{0}\equiv X_{n}$ y $C_{n}^{0}\equiv C_{n}$.
\end{Def}

\begin{Def}
Se le llama al par $\left(Z,S\right)$ \textbf{ciclo-stacionario} si los ciclos forman una sucesi\'on estacionaria, es decir, con $=^{D}$ denota iguales en distribuci\'on:
\begin{eqnarray*}
\left(C_{n+1},C_{n+2},\ldots\right)=^{D}\left(C_{1},C_{2},\ldots\right),\geq0
\end{eqnarray*}
Ciclo-estacionareidad es equivalente a 
\begin{eqnarray*}
\theta_{S_{n}}\left(Z,S\right)=^{D}
\left(Z^{0},S^{0}\right),\geq0,
\end{eqnarray*}
donde $\left(C_{n+1},C_{n+2},\ldots\right)$ y $\theta_{S_{n}}\left(Z,S\right)$ son mapeos medibles de cada uno y que no dependen de $n$.
\end{Def}


\begin{Def}
Un par $\left(Z^{*},S^{*}\right)$ es \textbf{estacionario} si $\theta\left(Z^{*},S^{*}\right)=^{D}
\left(Z^{*},S^{*}\right)$, para $t\geq0$.
\end{Def}


\begin{Teo}\label{Teorema.2.1}
Supongase que $\left(Z,S\right)$ es cycle-stationary con $\esp\left[X_{1}\right]<\infty$. Sea $U$ distribuida uniformemente en $\left[0,1\right)$ e independiente de $\left(Z^{0},S^{0}\right)$ y sea $\prob^{*}$ la medida de probabilidad en $\left(\Omega,\prob\right)$ definida por $$d\prob^{*}=\frac{X_{1}}{\esp\left[X_{1}\right]}d\prob$$. Sea $\left(Z^{*},S^{*}\right)$ con distribuci\'on $\prob^{*}\left(\theta_{UX_{1}}\left(Z^{0},S^{0}\right)\in\cdot\right)$. Entonces $\left(Z^{*},S^{*}\right)$ es estacionario,
\begin{eqnarray*}
\esp\left[f\left(Z^{*},S^{*}\right)\right]=\esp\left[\int_{0}^{X_{1}}f\left(\theta_{s}\left(Z^{0},S^{0}\right)\right)ds\right]/\esp\left[X_{1}\right]
\end{eqnarray*}
$f\in\mathcal{H}\otimes\mathcal{L}^{+}$, and $S_{0}^{*}$ es continuo con funci\'on distribuci\'on $G_{\infty}$ definida por $$G_{\infty}\left(x\right):=\frac{\esp\left[X_{1}\right]\wedge x}{\esp\left[X_{1}\right]}$$ para $x\geq0$ y densidad $\prob\left[X_{1}>x\right]/\esp\left[X_{1}\right]$, con $x\geq0$.

\end{Teo}

%___________________________________________________________
\subsection{Classical Regeneration}
%___________________________________________________________

\begin{Def}
Dado un proceso \textbf{PEOSSM} (Proceso Estoc\'astico One Side Shift Medible) $Z$, se dice \textbf{regenerativo cl\'asico} con tiempos de regeneraci\'on $S$ si 

\begin{eqnarray*}
\theta_{S_{n}}\left(Z,S\right)=\left(Z^{0},S^{0}\right),n\geq0
\end{eqnarray*}
y adem\'as $\theta_{S_{n}}\left(Z,S\right)$ es independiente de $\left(\left(Z_{s}\right)s\in\left[0,S_{n}\right),S_{0},\ldots,S_{n}\right)$
Si lo anterior se cumple, al par $\left(Z,S\right)$ se le llama regenerativo cl\'asico.
\end{Def}

\begin{Note}
Si el par $\left(Z,S\right)$ es regenerativo cl\'asico, entonces las longitudes de los ciclos $X_{1},X_{2},\ldots,$ son i.i.d. e independientes de la longitud del retraso $S_{0}$, es decir, $S$ es un \textbf{proceso de renovaci\'on}. Las longitudes de los ciclos tambi\'en son llamados tiempos de inter-regeneraci\'on y tiempos de ocurrencia.

\end{Note}

%___________________________________________________________
\subsection{Stationary Version}
%___________________________________________________________



\begin{Teo}\label{Teo.3.1}
Sup\'ongase que el par $\left(Z,S\right)$ es regenerativo cl\'asico con $\esp\left[X_{1}\right]<\infty$. Entonces $\left(Z^{*},S^{*}\right)$ en el teorema \ref{Teorema.2.1} es una versi\'on estacionaria de $\left(Z,S\right)$.
\end{Teo}

%___________________________________________________________
\subsection{Spread Out}
%___________________________________________________________


\begin{Def}
Una variable aleatoria $X_{1}$ es \textbf{spread out} si existe una $n\geq1$ y una  funci\'on $f\in\mathcal{B}^{+}$ tal que $\int_{\rea}f\left(x\right)dx>0$ con $X_{2},X_{3},\ldots,X_{n}$ copias i.i.d  de $X_{1}$, $$\prob\left(X_{1}+\cdots+X_{n}\in B\right)\geq\int_{B}f\left(x\right)dx$$ para $B\in\mathcal{B}$.
\end{Def}

%___________________________________________________________
\subsection{Wide Sense Regeneration}
%___________________________________________________________


\begin{Def}
Dado un proceso estoc\'astico $Z$ se le llama \textit{wide-sense regenerative} (\textbf{WSR}) con tiempos de regeneraci\'on $S$ si $\theta_{S_{n}}\left(Z,S\right)=\left(Z^{0},S^{0}\right)$ para $n\geq0$ en distribuci\'on y $\theta_{S_{n}}\left(Z,S\right)$ es independiente de $\left(S_{0},S_{1},\ldots,S_{n}\right)$ para $n\geq0$.
Se dice que el par $\left(Z,S\right)$ es WSR si lo anterior se cumple.
\end{Def}


\begin{Note}
\begin{itemize}
\item El proceso de trayectorias $\left(\theta_{s}Z\right)_{s\in\left[0,\infty\right)}$ es WSR con tiempos de regeneraci\'on $S$ pero no es regenerativo cl\'asico.

\item Si $Z$ es cualquier proceso estacionario y $S$ es un proceso de renovaci\'on que es independiente de $Z$, entonces $\left(Z,S\right)$ es WSR pero en general no es regenerativo cl\'asico

\end{itemize}

\end{Note}


\begin{Note}
Para cualquier proceso estoc\'astico $Z$, el proceso de trayectorias $\left(\theta_{s}Z\right)_{s\in\left[0,\infty\right)}$ es siempre un proceso de Markov.
\end{Note}


\begin{Teo}\label{Teo.4.1}
Supongase que el par $\left(Z,S\right)$ es WSR con $\esp\left[X_{1}\right]<\infty$. Entonces $\left(Z^{*},S^{*}\right)$ en el teorema (\ref{Teorema.2.1}) es una versi\'on estacionaria de 
$\left(Z,S\right)$.
\end{Teo}


%___________________________________________________________
\subsection{Existence of Regeneration Times}
%___________________________________________________________


\begin{Teo}\label{Tma.Existencia.Tiempos.Regeneracion}
Sea $Z$ un Proceso Estoc\'astico un lado shift-medible \textit{one-sided shift-measurable stochastic process}, (PEOSCTSM),
y $S_{0}$ y $S_{1}$ tiempos aleatorios tales que $0\leq S_{0}<S_{1}$ y
\begin{equation}
\theta_{S_{1}}Z=\theta_{S_{0}}Z\textrm{ en distribuci\'on}.
\end{equation}

Entonces el espacio de probabilidad subyacente $\left(\Omega,\mathcal{F},\prob\right)$ puede extenderse para soportar una sucesi\'on de tiempos aleatorios $S$ tales que

\begin{eqnarray}
\theta_{S_{n}}\left(Z,S\right)=\left(Z^{0},S^{0}\right),n\geq0,\textrm{ en distribuci\'on},\\
\left(Z,S_{0},S_{1}\right)\textrm{ depende de }\left(X_{2},X_{3},\ldots\right)\textrm{ solamente a traves de }\theta_{S_{1}}Z.
\end{eqnarray}
\end{Teo}

\begin{Coro}\label{Tma.Estacionariedad}
Bajo las condiciones del Teorema anterior (\ref{Tma.Existencia.Tiempos.Regeneracion}), el par $\left(Z,S\right)$ es regenerativo cl\'asico. Si adem\'as se tiene que $\esp\left[X_{1}\right]<\infty$ por el Teorema (\ref{Teo.3.1}) existe un par $\left(Z^{*},S^{*}\right)$ que es una vesi\'on estacionaria de $\left(Z,S\right)$.
\end{Coro}

\newpage

%________________________________________________________________________
\section{Procesos Regenerativos}
%________________________________________________________________________
%______________________________________________________________________
%\subsection*{Procesos Regenerativos}
%________________________________________________________________________



\begin{Note}
Si $\tilde{X}\left(t\right)$ con espacio de estados $\tilde{S}$ es regenerativo sobre $T_{n}$, entonces $X\left(t\right)=f\left(\tilde{X}\left(t\right)\right)$ tambi\'en es regenerativo sobre $T_{n}$, para cualquier funci\'on $f:\tilde{S}\rightarrow S$.
\end{Note}

\begin{Note}
Los procesos regenerativos son crudamente regenerativos, pero no al rev\'es.
\end{Note}
%\subsection*{Procesos Regenerativos: Sigman\cite{Sigman1}}
\begin{Def}[Definici\'on Cl\'asica]
Un proceso estoc\'astico $X=\left\{X\left(t\right):t\geq0\right\}$ es llamado regenerativo is existe una variable aleatoria $R_{1}>0$ tal que
\begin{itemize}
\item[i)] $\left\{X\left(t+R_{1}\right):t\geq0\right\}$ es independiente de $\left\{\left\{X\left(t\right):t<R_{1}\right\},\right\}$
\item[ii)] $\left\{X\left(t+R_{1}\right):t\geq0\right\}$ es estoc\'asticamente equivalente a $\left\{X\left(t\right):t>0\right\}$
\end{itemize}

Llamamos a $R_{1}$ tiempo de regeneraci\'on, y decimos que $X$ se regenera en este punto.
\end{Def}

$\left\{X\left(t+R_{1}\right)\right\}$ es regenerativo con tiempo de regeneraci\'on $R_{2}$, independiente de $R_{1}$ pero con la misma distribuci\'on que $R_{1}$. Procediendo de esta manera se obtiene una secuencia de variables aleatorias independientes e id\'enticamente distribuidas $\left\{R_{n}\right\}$ llamados longitudes de ciclo. Si definimos a $Z_{k}\equiv R_{1}+R_{2}+\cdots+R_{k}$, se tiene un proceso de renovaci\'on llamado proceso de renovaci\'on encajado para $X$.




\begin{Def}
Para $x$ fijo y para cada $t\geq0$, sea $I_{x}\left(t\right)=1$ si $X\left(t\right)\leq x$,  $I_{x}\left(t\right)=0$ en caso contrario, y def\'inanse los tiempos promedio
\begin{eqnarray*}
\overline{X}&=&lim_{t\rightarrow\infty}\frac{1}{t}\int_{0}^{\infty}X\left(u\right)du\\
\prob\left(X_{\infty}\leq x\right)&=&lim_{t\rightarrow\infty}\frac{1}{t}\int_{0}^{\infty}I_{x}\left(u\right)du,
\end{eqnarray*}
cuando estos l\'imites existan.
\end{Def}

Como consecuencia del teorema de Renovaci\'on-Recompensa, se tiene que el primer l\'imite  existe y es igual a la constante
\begin{eqnarray*}
\overline{X}&=&\frac{\esp\left[\int_{0}^{R_{1}}X\left(t\right)dt\right]}{\esp\left[R_{1}\right]},
\end{eqnarray*}
suponiendo que ambas esperanzas son finitas.

\begin{Note}
\begin{itemize}
\item[a)] Si el proceso regenerativo $X$ es positivo recurrente y tiene trayectorias muestrales no negativas, entonces la ecuaci\'on anterior es v\'alida.
\item[b)] Si $X$ es positivo recurrente regenerativo, podemos construir una \'unica versi\'on estacionaria de este proceso, $X_{e}=\left\{X_{e}\left(t\right)\right\}$, donde $X_{e}$ es un proceso estoc\'astico regenerativo y estrictamente estacionario, con distribuci\'on marginal distribuida como $X_{\infty}$
\end{itemize}
\end{Note}

Para $\left\{X\left(t\right):t\geq0\right\}$ Proceso Estoc\'astico a tiempo continuo con estado de espacios $S$, que es un espacio m\'etrico, con trayectorias continuas por la derecha y con l\'imites por la izquierda c.s. Sea $N\left(t\right)$ un proceso de renovaci\'on en $\rea_{+}$ definido en el mismo espacio de probabilidad que $X\left(t\right)$, con tiempos de renovaci\'on $T$ y tiempos de inter-renovaci\'on $\xi_{n}=T_{n}-T_{n-1}$, con misma distribuci\'on $F$ de media finita $\mu$.


\begin{Def}
Para el proceso $\left\{\left(N\left(t\right),X\left(t\right)\right):t\geq0\right\}$, sus trayectoria muestrales en el intervalo de tiempo $\left[T_{n-1},T_{n}\right)$ est\'an descritas por
\begin{eqnarray*}
\zeta_{n}=\left(\xi_{n},\left\{X\left(T_{n-1}+t\right):0\leq t<\xi_{n}\right\}\right)
\end{eqnarray*}
Este $\zeta_{n}$ es el $n$-\'esimo segmento del proceso. El proceso es regenerativo sobre los tiempos $T_{n}$ si sus segmentos $\zeta_{n}$ son independientes e id\'enticamennte distribuidos.
\end{Def}


\begin{Note}
Si $\tilde{X}\left(t\right)$ con espacio de estados $\tilde{S}$ es regenerativo sobre $T_{n}$, entonces $X\left(t\right)=f\left(\tilde{X}\left(t\right)\right)$ tambi\'en es regenerativo sobre $T_{n}$, para cualquier funci\'on $f:\tilde{S}\rightarrow S$.
\end{Note}

\begin{Note}
Los procesos regenerativos son crudamente regenerativos, pero no al rev\'es.
\end{Note}

\begin{Def}[Definici\'on Cl\'asica]
Un proceso estoc\'astico $X=\left\{X\left(t\right):t\geq0\right\}$ es llamado regenerativo is existe una variable aleatoria $R_{1}>0$ tal que
\begin{itemize}
\item[i)] $\left\{X\left(t+R_{1}\right):t\geq0\right\}$ es independiente de $\left\{\left\{X\left(t\right):t<R_{1}\right\},\right\}$
\item[ii)] $\left\{X\left(t+R_{1}\right):t\geq0\right\}$ es estoc\'asticamente equivalente a $\left\{X\left(t\right):t>0\right\}$
\end{itemize}

Llamamos a $R_{1}$ tiempo de regeneraci\'on, y decimos que $X$ se regenera en este punto.
\end{Def}

$\left\{X\left(t+R_{1}\right)\right\}$ es regenerativo con tiempo de regeneraci\'on $R_{2}$, independiente de $R_{1}$ pero con la misma distribuci\'on que $R_{1}$. Procediendo de esta manera se obtiene una secuencia de variables aleatorias independientes e id\'enticamente distribuidas $\left\{R_{n}\right\}$ llamados longitudes de ciclo. Si definimos a $Z_{k}\equiv R_{1}+R_{2}+\cdots+R_{k}$, se tiene un proceso de renovaci\'on llamado proceso de renovaci\'on encajado para $X$.

\begin{Note}
Un proceso regenerativo con media de la longitud de ciclo finita es llamado positivo recurrente.
\end{Note}


%_________________________________________________________________________
%
%\section{Appendix F: Output Process and Regenerative Processes}
%_________________________________________________________________________
%
En Sigman, Thorison y Wolff \cite{Sigman2} prueban que para la existencia de un una sucesi\'on infinita no decreciente de tiempos de regeneraci\'on $\tau_{1}\leq\tau_{2}\leq\cdots$ en los cuales el proceso se regenera, basta un tiempo de regeneraci\'on $R_{1}$, donde $R_{j}=\tau_{j}-\tau_{j-1}$. Para tal efecto se requiere la existencia de un espacio de probabilidad $\left(\Omega,\mathcal{F},\prob\right)$, y proceso estoc\'astico $\textit{X}=\left\{X\left(t\right):t\geq0\right\}$ con espacio de estados $\left(S,\mathcal{R}\right)$, con $\mathcal{R}$ $\sigma$-\'algebra.

\begin{Prop}
Si existe una variable aleatoria no negativa $R_{1}$ tal que $\theta_{R1}X=_{D}X$, entonces $\left(\Omega,\mathcal{F},\prob\right)$ puede extenderse para soportar una sucesi\'on estacionaria de variables aleatorias $R=\left\{R_{k}:k\geq1\right\}$, tal que para $k\geq1$,
\begin{eqnarray*}
\theta_{k}\left(X,R\right)=_{D}\left(X,R\right).
\end{eqnarray*}

Adem\'as, para $k\geq1$, $\theta_{k}R$ es condicionalmente independiente de $\left(X,R_{1},\ldots,R_{k}\right)$, dado $\theta_{\tau k}X$.

\end{Prop}


\begin{itemize}
\item Doob en 1953 demostr\'o que el estado estacionario de un proceso de partida en un sistema de espera $M/G/\infty$, es Poisson con la misma tasa que el proceso de arribos.

\item Burke en 1968, fue el primero en demostrar que el estado estacionario de un proceso de salida de una cola $M/M/s$ es un proceso Poisson.

\item Disney en 1973 obtuvo el siguiente resultado:

\begin{Teo}
Para el sistema de espera $M/G/1/L$ con disciplina FIFO, el proceso $\textbf{I}$ es un proceso de renovaci\'on si y s\'olo si el proceso denominado longitud de la cola es estacionario y se cumple cualquiera de los siguientes casos:

\begin{itemize}
\item[a)] Los tiempos de servicio son identicamente cero;
\item[b)] $L=0$, para cualquier proceso de servicio $S$;
\item[c)] $L=1$ y $G=D$;
\item[d)] $L=\infty$ y $G=M$.
\end{itemize}
En estos casos, respectivamente, las distribuciones de interpartida $P\left\{T_{n+1}-T_{n}\leq t\right\}$ son


\begin{itemize}
\item[a)] $1-e^{-\lambda t}$, $t\geq0$;
\item[b)] $1-e^{-\lambda t}*F\left(t\right)$, $t\geq0$;
\item[c)] $1-e^{-\lambda t}*\indora_{d}\left(t\right)$, $t\geq0$;
\item[d)] $1-e^{-\lambda t}*F\left(t\right)$, $t\geq0$.
\end{itemize}
\end{Teo}


\item Finch (1959) mostr\'o que para los sistemas $M/G/1/L$, con $1\leq L\leq \infty$ con distribuciones de servicio dos veces diferenciable, solamente el sistema $M/M/1/\infty$ tiene proceso de salida de renovaci\'on estacionario.

\item King (1971) demostro que un sistema de colas estacionario $M/G/1/1$ tiene sus tiempos de interpartida sucesivas $D_{n}$ y $D_{n+1}$ son independientes, si y s\'olo si, $G=D$, en cuyo caso le proceso de salida es de renovaci\'on.

\item Disney (1973) demostr\'o que el \'unico sistema estacionario $M/G/1/L$, que tiene proceso de salida de renovaci\'on  son los sistemas $M/M/1$ y $M/D/1/1$.



\item El siguiente resultado es de Disney y Koning (1985)
\begin{Teo}
En un sistema de espera $M/G/s$, el estado estacionario del proceso de salida es un proceso Poisson para cualquier distribuci\'on de los tiempos de servicio si el sistema tiene cualquiera de las siguientes cuatro propiedades.

\begin{itemize}
\item[a)] $s=\infty$
\item[b)] La disciplina de servicio es de procesador compartido.
\item[c)] La disciplina de servicio es LCFS y preemptive resume, esto se cumple para $L<\infty$
\item[d)] $G=M$.
\end{itemize}

\end{Teo}

\item El siguiente resultado es de Alamatsaz (1983)

\begin{Teo}
En cualquier sistema de colas $GI/G/1/L$ con $1\leq L<\infty$ y distribuci\'on de interarribos $A$ y distribuci\'on de los tiempos de servicio $B$, tal que $A\left(0\right)=0$, $A\left(t\right)\left(1-B\left(t\right)\right)>0$ para alguna $t>0$ y $B\left(t\right)$ para toda $t>0$, es imposible que el proceso de salida estacionario sea de renovaci\'on.
\end{Teo}

\end{itemize}

%________________________________________________________________________
%\subsection*{Procesos Regenerativos}
%________________________________________________________________________



\begin{Note}
Si $\tilde{X}\left(t\right)$ con espacio de estados $\tilde{S}$ es regenerativo sobre $T_{n}$, entonces $X\left(t\right)=f\left(\tilde{X}\left(t\right)\right)$ tambi\'en es regenerativo sobre $T_{n}$, para cualquier funci\'on $f:\tilde{S}\rightarrow S$.
\end{Note}

\begin{Note}
Los procesos regenerativos son crudamente regenerativos, pero no al rev\'es.
\end{Note}
%\subsection*{Procesos Regenerativos: Sigman\cite{Sigman1}}
\begin{Def}[Definici\'on Cl\'asica]
Un proceso estoc\'astico $X=\left\{X\left(t\right):t\geq0\right\}$ es llamado regenerativo is existe una variable aleatoria $R_{1}>0$ tal que
\begin{itemize}
\item[i)] $\left\{X\left(t+R_{1}\right):t\geq0\right\}$ es independiente de $\left\{\left\{X\left(t\right):t<R_{1}\right\},\right\}$
\item[ii)] $\left\{X\left(t+R_{1}\right):t\geq0\right\}$ es estoc\'asticamente equivalente a $\left\{X\left(t\right):t>0\right\}$
\end{itemize}

Llamamos a $R_{1}$ tiempo de regeneraci\'on, y decimos que $X$ se regenera en este punto.
\end{Def}

$\left\{X\left(t+R_{1}\right)\right\}$ es regenerativo con tiempo de regeneraci\'on $R_{2}$, independiente de $R_{1}$ pero con la misma distribuci\'on que $R_{1}$. Procediendo de esta manera se obtiene una secuencia de variables aleatorias independientes e id\'enticamente distribuidas $\left\{R_{n}\right\}$ llamados longitudes de ciclo. Si definimos a $Z_{k}\equiv R_{1}+R_{2}+\cdots+R_{k}$, se tiene un proceso de renovaci\'on llamado proceso de renovaci\'on encajado para $X$.




\begin{Def}
Para $x$ fijo y para cada $t\geq0$, sea $I_{x}\left(t\right)=1$ si $X\left(t\right)\leq x$,  $I_{x}\left(t\right)=0$ en caso contrario, y def\'inanse los tiempos promedio
\begin{eqnarray*}
\overline{X}&=&lim_{t\rightarrow\infty}\frac{1}{t}\int_{0}^{\infty}X\left(u\right)du\\
\prob\left(X_{\infty}\leq x\right)&=&lim_{t\rightarrow\infty}\frac{1}{t}\int_{0}^{\infty}I_{x}\left(u\right)du,
\end{eqnarray*}
cuando estos l\'imites existan.
\end{Def}

Como consecuencia del teorema de Renovaci\'on-Recompensa, se tiene que el primer l\'imite  existe y es igual a la constante
\begin{eqnarray*}
\overline{X}&=&\frac{\esp\left[\int_{0}^{R_{1}}X\left(t\right)dt\right]}{\esp\left[R_{1}\right]},
\end{eqnarray*}
suponiendo que ambas esperanzas son finitas.

\begin{Note}
\begin{itemize}
\item[a)] Si el proceso regenerativo $X$ es positivo recurrente y tiene trayectorias muestrales no negativas, entonces la ecuaci\'on anterior es v\'alida.
\item[b)] Si $X$ es positivo recurrente regenerativo, podemos construir una \'unica versi\'on estacionaria de este proceso, $X_{e}=\left\{X_{e}\left(t\right)\right\}$, donde $X_{e}$ es un proceso estoc\'astico regenerativo y estrictamente estacionario, con distribuci\'on marginal distribuida como $X_{\infty}$
\end{itemize}
\end{Note}

Para $\left\{X\left(t\right):t\geq0\right\}$ Proceso Estoc\'astico a tiempo continuo con estado de espacios $S$, que es un espacio m\'etrico, con trayectorias continuas por la derecha y con l\'imites por la izquierda c.s. Sea $N\left(t\right)$ un proceso de renovaci\'on en $\rea_{+}$ definido en el mismo espacio de probabilidad que $X\left(t\right)$, con tiempos de renovaci\'on $T$ y tiempos de inter-renovaci\'on $\xi_{n}=T_{n}-T_{n-1}$, con misma distribuci\'on $F$ de media finita $\mu$.


\begin{Def}
Para el proceso $\left\{\left(N\left(t\right),X\left(t\right)\right):t\geq0\right\}$, sus trayectoria muestrales en el intervalo de tiempo $\left[T_{n-1},T_{n}\right)$ est\'an descritas por
\begin{eqnarray*}
\zeta_{n}=\left(\xi_{n},\left\{X\left(T_{n-1}+t\right):0\leq t<\xi_{n}\right\}\right)
\end{eqnarray*}
Este $\zeta_{n}$ es el $n$-\'esimo segmento del proceso. El proceso es regenerativo sobre los tiempos $T_{n}$ si sus segmentos $\zeta_{n}$ son independientes e id\'enticamennte distribuidos.
\end{Def}


\begin{Note}
Si $\tilde{X}\left(t\right)$ con espacio de estados $\tilde{S}$ es regenerativo sobre $T_{n}$, entonces $X\left(t\right)=f\left(\tilde{X}\left(t\right)\right)$ tambi\'en es regenerativo sobre $T_{n}$, para cualquier funci\'on $f:\tilde{S}\rightarrow S$.
\end{Note}

\begin{Note}
Los procesos regenerativos son crudamente regenerativos, pero no al rev\'es.
\end{Note}

\begin{Def}[Definici\'on Cl\'asica]
Un proceso estoc\'astico $X=\left\{X\left(t\right):t\geq0\right\}$ es llamado regenerativo is existe una variable aleatoria $R_{1}>0$ tal que
\begin{itemize}
\item[i)] $\left\{X\left(t+R_{1}\right):t\geq0\right\}$ es independiente de $\left\{\left\{X\left(t\right):t<R_{1}\right\},\right\}$
\item[ii)] $\left\{X\left(t+R_{1}\right):t\geq0\right\}$ es estoc\'asticamente equivalente a $\left\{X\left(t\right):t>0\right\}$
\end{itemize}

Llamamos a $R_{1}$ tiempo de regeneraci\'on, y decimos que $X$ se regenera en este punto.
\end{Def}

$\left\{X\left(t+R_{1}\right)\right\}$ es regenerativo con tiempo de regeneraci\'on $R_{2}$, independiente de $R_{1}$ pero con la misma distribuci\'on que $R_{1}$. Procediendo de esta manera se obtiene una secuencia de variables aleatorias independientes e id\'enticamente distribuidas $\left\{R_{n}\right\}$ llamados longitudes de ciclo. Si definimos a $Z_{k}\equiv R_{1}+R_{2}+\cdots+R_{k}$, se tiene un proceso de renovaci\'on llamado proceso de renovaci\'on encajado para $X$.

\begin{Note}
Un proceso regenerativo con media de la longitud de ciclo finita es llamado positivo recurrente.
\end{Note}


\begin{Def}
Para $x$ fijo y para cada $t\geq0$, sea $I_{x}\left(t\right)=1$ si $X\left(t\right)\leq x$,  $I_{x}\left(t\right)=0$ en caso contrario, y def\'inanse los tiempos promedio
\begin{eqnarray*}
\overline{X}&=&lim_{t\rightarrow\infty}\frac{1}{t}\int_{0}^{\infty}X\left(u\right)du\\
\prob\left(X_{\infty}\leq x\right)&=&lim_{t\rightarrow\infty}\frac{1}{t}\int_{0}^{\infty}I_{x}\left(u\right)du,
\end{eqnarray*}
cuando estos l\'imites existan.
\end{Def}

Como consecuencia del teorema de Renovaci\'on-Recompensa, se tiene que el primer l\'imite  existe y es igual a la constante
\begin{eqnarray*}
\overline{X}&=&\frac{\esp\left[\int_{0}^{R_{1}}X\left(t\right)dt\right]}{\esp\left[R_{1}\right]},
\end{eqnarray*}
suponiendo que ambas esperanzas son finitas.

\begin{Note}
\begin{itemize}
\item[a)] Si el proceso regenerativo $X$ es positivo recurrente y tiene trayectorias muestrales no negativas, entonces la ecuaci\'on anterior es v\'alida.
\item[b)] Si $X$ es positivo recurrente regenerativo, podemos construir una \'unica versi\'on estacionaria de este proceso, $X_{e}=\left\{X_{e}\left(t\right)\right\}$, donde $X_{e}$ es un proceso estoc\'astico regenerativo y estrictamente estacionario, con distribuci\'on marginal distribuida como $X_{\infty}$
\end{itemize}
\end{Note}

%________________________________________________________________________
%\subsection*{Procesos Regenerativos}
%________________________________________________________________________



\begin{Note}
Si $\tilde{X}\left(t\right)$ con espacio de estados $\tilde{S}$ es regenerativo sobre $T_{n}$, entonces $X\left(t\right)=f\left(\tilde{X}\left(t\right)\right)$ tambi\'en es regenerativo sobre $T_{n}$, para cualquier funci\'on $f:\tilde{S}\rightarrow S$.
\end{Note}

\begin{Note}
Los procesos regenerativos son crudamente regenerativos, pero no al rev\'es.
\end{Note}
%\subsection*{Procesos Regenerativos: Sigman\cite{Sigman1}}
\begin{Def}[Definici\'on Cl\'asica]
Un proceso estoc\'astico $X=\left\{X\left(t\right):t\geq0\right\}$ es llamado regenerativo is existe una variable aleatoria $R_{1}>0$ tal que
\begin{itemize}
\item[i)] $\left\{X\left(t+R_{1}\right):t\geq0\right\}$ es independiente de $\left\{\left\{X\left(t\right):t<R_{1}\right\},\right\}$
\item[ii)] $\left\{X\left(t+R_{1}\right):t\geq0\right\}$ es estoc\'asticamente equivalente a $\left\{X\left(t\right):t>0\right\}$
\end{itemize}

Llamamos a $R_{1}$ tiempo de regeneraci\'on, y decimos que $X$ se regenera en este punto.
\end{Def}

$\left\{X\left(t+R_{1}\right)\right\}$ es regenerativo con tiempo de regeneraci\'on $R_{2}$, independiente de $R_{1}$ pero con la misma distribuci\'on que $R_{1}$. Procediendo de esta manera se obtiene una secuencia de variables aleatorias independientes e id\'enticamente distribuidas $\left\{R_{n}\right\}$ llamados longitudes de ciclo. Si definimos a $Z_{k}\equiv R_{1}+R_{2}+\cdots+R_{k}$, se tiene un proceso de renovaci\'on llamado proceso de renovaci\'on encajado para $X$.




\begin{Def}
Para $x$ fijo y para cada $t\geq0$, sea $I_{x}\left(t\right)=1$ si $X\left(t\right)\leq x$,  $I_{x}\left(t\right)=0$ en caso contrario, y def\'inanse los tiempos promedio
\begin{eqnarray*}
\overline{X}&=&lim_{t\rightarrow\infty}\frac{1}{t}\int_{0}^{\infty}X\left(u\right)du\\
\prob\left(X_{\infty}\leq x\right)&=&lim_{t\rightarrow\infty}\frac{1}{t}\int_{0}^{\infty}I_{x}\left(u\right)du,
\end{eqnarray*}
cuando estos l\'imites existan.
\end{Def}

Como consecuencia del teorema de Renovaci\'on-Recompensa, se tiene que el primer l\'imite  existe y es igual a la constante
\begin{eqnarray*}
\overline{X}&=&\frac{\esp\left[\int_{0}^{R_{1}}X\left(t\right)dt\right]}{\esp\left[R_{1}\right]},
\end{eqnarray*}
suponiendo que ambas esperanzas son finitas.

\begin{Note}
\begin{itemize}
\item[a)] Si el proceso regenerativo $X$ es positivo recurrente y tiene trayectorias muestrales no negativas, entonces la ecuaci\'on anterior es v\'alida.
\item[b)] Si $X$ es positivo recurrente regenerativo, podemos construir una \'unica versi\'on estacionaria de este proceso, $X_{e}=\left\{X_{e}\left(t\right)\right\}$, donde $X_{e}$ es un proceso estoc\'astico regenerativo y estrictamente estacionario, con distribuci\'on marginal distribuida como $X_{\infty}$
\end{itemize}
\end{Note}

%________________________________________________________________________
%\subsection*{Procesos Regenerativos}
%________________________________________________________________________

Para $\left\{X\left(t\right):t\geq0\right\}$ Proceso Estoc\'astico a tiempo continuo con estado de espacios $S$, que es un espacio m\'etrico, con trayectorias continuas por la derecha y con l\'imites por la izquierda c.s. Sea $N\left(t\right)$ un proceso de renovaci\'on en $\rea_{+}$ definido en el mismo espacio de probabilidad que $X\left(t\right)$, con tiempos de renovaci\'on $T$ y tiempos de inter-renovaci\'on $\xi_{n}=T_{n}-T_{n-1}$, con misma distribuci\'on $F$ de media finita $\mu$.



\begin{Def}
Para el proceso $\left\{\left(N\left(t\right),X\left(t\right)\right):t\geq0\right\}$, sus trayectoria muestrales en el intervalo de tiempo $\left[T_{n-1},T_{n}\right)$ est\'an descritas por
\begin{eqnarray*}
\zeta_{n}=\left(\xi_{n},\left\{X\left(T_{n-1}+t\right):0\leq t<\xi_{n}\right\}\right)
\end{eqnarray*}
Este $\zeta_{n}$ es el $n$-\'esimo segmento del proceso. El proceso es regenerativo sobre los tiempos $T_{n}$ si sus segmentos $\zeta_{n}$ son independientes e id\'enticamennte distribuidos.
\end{Def}


\begin{Note}
Si $\tilde{X}\left(t\right)$ con espacio de estados $\tilde{S}$ es regenerativo sobre $T_{n}$, entonces $X\left(t\right)=f\left(\tilde{X}\left(t\right)\right)$ tambi\'en es regenerativo sobre $T_{n}$, para cualquier funci\'on $f:\tilde{S}\rightarrow S$.
\end{Note}

\begin{Note}
Los procesos regenerativos son crudamente regenerativos, pero no al rev\'es.
\end{Note}

\begin{Def}[Definici\'on Cl\'asica]
Un proceso estoc\'astico $X=\left\{X\left(t\right):t\geq0\right\}$ es llamado regenerativo is existe una variable aleatoria $R_{1}>0$ tal que
\begin{itemize}
\item[i)] $\left\{X\left(t+R_{1}\right):t\geq0\right\}$ es independiente de $\left\{\left\{X\left(t\right):t<R_{1}\right\},\right\}$
\item[ii)] $\left\{X\left(t+R_{1}\right):t\geq0\right\}$ es estoc\'asticamente equivalente a $\left\{X\left(t\right):t>0\right\}$
\end{itemize}

Llamamos a $R_{1}$ tiempo de regeneraci\'on, y decimos que $X$ se regenera en este punto.
\end{Def}

$\left\{X\left(t+R_{1}\right)\right\}$ es regenerativo con tiempo de regeneraci\'on $R_{2}$, independiente de $R_{1}$ pero con la misma distribuci\'on que $R_{1}$. Procediendo de esta manera se obtiene una secuencia de variables aleatorias independientes e id\'enticamente distribuidas $\left\{R_{n}\right\}$ llamados longitudes de ciclo. Si definimos a $Z_{k}\equiv R_{1}+R_{2}+\cdots+R_{k}$, se tiene un proceso de renovaci\'on llamado proceso de renovaci\'on encajado para $X$.

\begin{Note}
Un proceso regenerativo con media de la longitud de ciclo finita es llamado positivo recurrente.
\end{Note}


\begin{Def}
Para $x$ fijo y para cada $t\geq0$, sea $I_{x}\left(t\right)=1$ si $X\left(t\right)\leq x$,  $I_{x}\left(t\right)=0$ en caso contrario, y def\'inanse los tiempos promedio
\begin{eqnarray*}
\overline{X}&=&lim_{t\rightarrow\infty}\frac{1}{t}\int_{0}^{\infty}X\left(u\right)du\\
\prob\left(X_{\infty}\leq x\right)&=&lim_{t\rightarrow\infty}\frac{1}{t}\int_{0}^{\infty}I_{x}\left(u\right)du,
\end{eqnarray*}
cuando estos l\'imites existan.
\end{Def}

Como consecuencia del teorema de Renovaci\'on-Recompensa, se tiene que el primer l\'imite  existe y es igual a la constante
\begin{eqnarray*}
\overline{X}&=&\frac{\esp\left[\int_{0}^{R_{1}}X\left(t\right)dt\right]}{\esp\left[R_{1}\right]},
\end{eqnarray*}
suponiendo que ambas esperanzas son finitas.

\begin{Note}
\begin{itemize}
\item[a)] Si el proceso regenerativo $X$ es positivo recurrente y tiene trayectorias muestrales no negativas, entonces la ecuaci\'on anterior es v\'alida.
\item[b)] Si $X$ es positivo recurrente regenerativo, podemos construir una \'unica versi\'on estacionaria de este proceso, $X_{e}=\left\{X_{e}\left(t\right)\right\}$, donde $X_{e}$ es un proceso estoc\'astico regenerativo y estrictamente estacionario, con distribuci\'on marginal distribuida como $X_{\infty}$
\end{itemize}
\end{Note}

%________________________________________________________________________
\section{Procesos Regenerativos Sigman, Thorisson y Wolff \cite{Sigman1}}
%________________________________________________________________________


\begin{Def}[Definici\'on Cl\'asica]
Un proceso estoc\'astico $X=\left\{X\left(t\right):t\geq0\right\}$ es llamado regenerativo is existe una variable aleatoria $R_{1}>0$ tal que
\begin{itemize}
\item[i)] $\left\{X\left(t+R_{1}\right):t\geq0\right\}$ es independiente de $\left\{\left\{X\left(t\right):t<R_{1}\right\},\right\}$
\item[ii)] $\left\{X\left(t+R_{1}\right):t\geq0\right\}$ es estoc\'asticamente equivalente a $\left\{X\left(t\right):t>0\right\}$
\end{itemize}

Llamamos a $R_{1}$ tiempo de regeneraci\'on, y decimos que $X$ se regenera en este punto.
\end{Def}

$\left\{X\left(t+R_{1}\right)\right\}$ es regenerativo con tiempo de regeneraci\'on $R_{2}$, independiente de $R_{1}$ pero con la misma distribuci\'on que $R_{1}$. Procediendo de esta manera se obtiene una secuencia de variables aleatorias independientes e id\'enticamente distribuidas $\left\{R_{n}\right\}$ llamados longitudes de ciclo. Si definimos a $Z_{k}\equiv R_{1}+R_{2}+\cdots+R_{k}$, se tiene un proceso de renovaci\'on llamado proceso de renovaci\'on encajado para $X$.


\begin{Note}
La existencia de un primer tiempo de regeneraci\'on, $R_{1}$, implica la existencia de una sucesi\'on completa de estos tiempos $R_{1},R_{2}\ldots,$ que satisfacen la propiedad deseada \cite{Sigman2}.
\end{Note}


\begin{Note} Para la cola $GI/GI/1$ los usuarios arriban con tiempos $t_{n}$ y son atendidos con tiempos de servicio $S_{n}$, los tiempos de arribo forman un proceso de renovaci\'on  con tiempos entre arribos independientes e identicamente distribuidos (\texttt{i.i.d.})$T_{n}=t_{n}-t_{n-1}$, adem\'as los tiempos de servicio son \texttt{i.i.d.} e independientes de los procesos de arribo. Por \textit{estable} se entiende que $\esp S_{n}<\esp T_{n}<\infty$.
\end{Note}
 


\begin{Def}
Para $x$ fijo y para cada $t\geq0$, sea $I_{x}\left(t\right)=1$ si $X\left(t\right)\leq x$,  $I_{x}\left(t\right)=0$ en caso contrario, y def\'inanse los tiempos promedio
\begin{eqnarray*}
\overline{X}&=&lim_{t\rightarrow\infty}\frac{1}{t}\int_{0}^{\infty}X\left(u\right)du\\
\prob\left(X_{\infty}\leq x\right)&=&lim_{t\rightarrow\infty}\frac{1}{t}\int_{0}^{\infty}I_{x}\left(u\right)du,
\end{eqnarray*}
cuando estos l\'imites existan.
\end{Def}

Como consecuencia del teorema de Renovaci\'on-Recompensa, se tiene que el primer l\'imite  existe y es igual a la constante
\begin{eqnarray*}
\overline{X}&=&\frac{\esp\left[\int_{0}^{R_{1}}X\left(t\right)dt\right]}{\esp\left[R_{1}\right]},
\end{eqnarray*}
suponiendo que ambas esperanzas son finitas.
 
\begin{Note}
Funciones de procesos regenerativos son regenerativas, es decir, si $X\left(t\right)$ es regenerativo y se define el proceso $Y\left(t\right)$ por $Y\left(t\right)=f\left(X\left(t\right)\right)$ para alguna funci\'on Borel medible $f\left(\cdot\right)$. Adem\'as $Y$ es regenerativo con los mismos tiempos de renovaci\'on que $X$. 

En general, los tiempos de renovaci\'on, $Z_{k}$ de un proceso regenerativo no requieren ser tiempos de paro con respecto a la evoluci\'on de $X\left(t\right)$.
\end{Note} 

\begin{Note}
Una funci\'on de un proceso de Markov, usualmente no ser\'a un proceso de Markov, sin embargo ser\'a regenerativo si el proceso de Markov lo es.
\end{Note}

 
\begin{Note}
Un proceso regenerativo con media de la longitud de ciclo finita es llamado positivo recurrente.
\end{Note}


\begin{Note}
\begin{itemize}
\item[a)] Si el proceso regenerativo $X$ es positivo recurrente y tiene trayectorias muestrales no negativas, entonces la ecuaci\'on anterior es v\'alida.
\item[b)] Si $X$ es positivo recurrente regenerativo, podemos construir una \'unica versi\'on estacionaria de este proceso, $X_{e}=\left\{X_{e}\left(t\right)\right\}$, donde $X_{e}$ es un proceso estoc\'astico regenerativo y estrictamente estacionario, con distribuci\'on marginal distribuida como $X_{\infty}$
\end{itemize}
\end{Note}


%________________________________________________________________________
%\subsection*{Procesos Regenerativos Sigman, Thorisson y Wolff \cite{Sigman1}}
%________________________________________________________________________


\begin{Def}[Definici\'on Cl\'asica]
Un proceso estoc\'astico $X=\left\{X\left(t\right):t\geq0\right\}$ es llamado regenerativo is existe una variable aleatoria $R_{1}>0$ tal que
\begin{itemize}
\item[i)] $\left\{X\left(t+R_{1}\right):t\geq0\right\}$ es independiente de $\left\{\left\{X\left(t\right):t<R_{1}\right\},\right\}$
\item[ii)] $\left\{X\left(t+R_{1}\right):t\geq0\right\}$ es estoc\'asticamente equivalente a $\left\{X\left(t\right):t>0\right\}$
\end{itemize}

Llamamos a $R_{1}$ tiempo de regeneraci\'on, y decimos que $X$ se regenera en este punto.
\end{Def}

$\left\{X\left(t+R_{1}\right)\right\}$ es regenerativo con tiempo de regeneraci\'on $R_{2}$, independiente de $R_{1}$ pero con la misma distribuci\'on que $R_{1}$. Procediendo de esta manera se obtiene una secuencia de variables aleatorias independientes e id\'enticamente distribuidas $\left\{R_{n}\right\}$ llamados longitudes de ciclo. Si definimos a $Z_{k}\equiv R_{1}+R_{2}+\cdots+R_{k}$, se tiene un proceso de renovaci\'on llamado proceso de renovaci\'on encajado para $X$.


\begin{Note}
La existencia de un primer tiempo de regeneraci\'on, $R_{1}$, implica la existencia de una sucesi\'on completa de estos tiempos $R_{1},R_{2}\ldots,$ que satisfacen la propiedad deseada \cite{Sigman2}.
\end{Note}


\begin{Note} Para la cola $GI/GI/1$ los usuarios arriban con tiempos $t_{n}$ y son atendidos con tiempos de servicio $S_{n}$, los tiempos de arribo forman un proceso de renovaci\'on  con tiempos entre arribos independientes e identicamente distribuidos (\texttt{i.i.d.})$T_{n}=t_{n}-t_{n-1}$, adem\'as los tiempos de servicio son \texttt{i.i.d.} e independientes de los procesos de arribo. Por \textit{estable} se entiende que $\esp S_{n}<\esp T_{n}<\infty$.
\end{Note}
 


\begin{Def}
Para $x$ fijo y para cada $t\geq0$, sea $I_{x}\left(t\right)=1$ si $X\left(t\right)\leq x$,  $I_{x}\left(t\right)=0$ en caso contrario, y def\'inanse los tiempos promedio
\begin{eqnarray*}
\overline{X}&=&lim_{t\rightarrow\infty}\frac{1}{t}\int_{0}^{\infty}X\left(u\right)du\\
\prob\left(X_{\infty}\leq x\right)&=&lim_{t\rightarrow\infty}\frac{1}{t}\int_{0}^{\infty}I_{x}\left(u\right)du,
\end{eqnarray*}
cuando estos l\'imites existan.
\end{Def}

Como consecuencia del teorema de Renovaci\'on-Recompensa, se tiene que el primer l\'imite  existe y es igual a la constante
\begin{eqnarray*}
\overline{X}&=&\frac{\esp\left[\int_{0}^{R_{1}}X\left(t\right)dt\right]}{\esp\left[R_{1}\right]},
\end{eqnarray*}
suponiendo que ambas esperanzas son finitas.
 
\begin{Note}
Funciones de procesos regenerativos son regenerativas, es decir, si $X\left(t\right)$ es regenerativo y se define el proceso $Y\left(t\right)$ por $Y\left(t\right)=f\left(X\left(t\right)\right)$ para alguna funci\'on Borel medible $f\left(\cdot\right)$. Adem\'as $Y$ es regenerativo con los mismos tiempos de renovaci\'on que $X$. 

En general, los tiempos de renovaci\'on, $Z_{k}$ de un proceso regenerativo no requieren ser tiempos de paro con respecto a la evoluci\'on de $X\left(t\right)$.
\end{Note} 

\begin{Note}
Una funci\'on de un proceso de Markov, usualmente no ser\'a un proceso de Markov, sin embargo ser\'a regenerativo si el proceso de Markov lo es.
\end{Note}

 
\begin{Note}
Un proceso regenerativo con media de la longitud de ciclo finita es llamado positivo recurrente.
\end{Note}


\begin{Note}
\begin{itemize}
\item[a)] Si el proceso regenerativo $X$ es positivo recurrente y tiene trayectorias muestrales no negativas, entonces la ecuaci\'on anterior es v\'alida.
\item[b)] Si $X$ es positivo recurrente regenerativo, podemos construir una \'unica versi\'on estacionaria de este proceso, $X_{e}=\left\{X_{e}\left(t\right)\right\}$, donde $X_{e}$ es un proceso estoc\'astico regenerativo y estrictamente estacionario, con distribuci\'on marginal distribuida como $X_{\infty}$
\end{itemize}
\end{Note}


%________________________________________________________________________
%\subsection*{Procesos Regenerativos Sigman, Thorisson y Wolff \cite{Sigman1}}
%________________________________________________________________________


\begin{Def}[Definici\'on Cl\'asica]
Un proceso estoc\'astico $X=\left\{X\left(t\right):t\geq0\right\}$ es llamado regenerativo is existe una variable aleatoria $R_{1}>0$ tal que
\begin{itemize}
\item[i)] $\left\{X\left(t+R_{1}\right):t\geq0\right\}$ es independiente de $\left\{\left\{X\left(t\right):t<R_{1}\right\},\right\}$
\item[ii)] $\left\{X\left(t+R_{1}\right):t\geq0\right\}$ es estoc\'asticamente equivalente a $\left\{X\left(t\right):t>0\right\}$
\end{itemize}

Llamamos a $R_{1}$ tiempo de regeneraci\'on, y decimos que $X$ se regenera en este punto.
\end{Def}

$\left\{X\left(t+R_{1}\right)\right\}$ es regenerativo con tiempo de regeneraci\'on $R_{2}$, independiente de $R_{1}$ pero con la misma distribuci\'on que $R_{1}$. Procediendo de esta manera se obtiene una secuencia de variables aleatorias independientes e id\'enticamente distribuidas $\left\{R_{n}\right\}$ llamados longitudes de ciclo. Si definimos a $Z_{k}\equiv R_{1}+R_{2}+\cdots+R_{k}$, se tiene un proceso de renovaci\'on llamado proceso de renovaci\'on encajado para $X$.


\begin{Note}
La existencia de un primer tiempo de regeneraci\'on, $R_{1}$, implica la existencia de una sucesi\'on completa de estos tiempos $R_{1},R_{2}\ldots,$ que satisfacen la propiedad deseada \cite{Sigman2}.
\end{Note}


\begin{Note} Para la cola $GI/GI/1$ los usuarios arriban con tiempos $t_{n}$ y son atendidos con tiempos de servicio $S_{n}$, los tiempos de arribo forman un proceso de renovaci\'on  con tiempos entre arribos independientes e identicamente distribuidos (\texttt{i.i.d.})$T_{n}=t_{n}-t_{n-1}$, adem\'as los tiempos de servicio son \texttt{i.i.d.} e independientes de los procesos de arribo. Por \textit{estable} se entiende que $\esp S_{n}<\esp T_{n}<\infty$.
\end{Note}
 


\begin{Def}
Para $x$ fijo y para cada $t\geq0$, sea $I_{x}\left(t\right)=1$ si $X\left(t\right)\leq x$,  $I_{x}\left(t\right)=0$ en caso contrario, y def\'inanse los tiempos promedio
\begin{eqnarray*}
\overline{X}&=&lim_{t\rightarrow\infty}\frac{1}{t}\int_{0}^{\infty}X\left(u\right)du\\
\prob\left(X_{\infty}\leq x\right)&=&lim_{t\rightarrow\infty}\frac{1}{t}\int_{0}^{\infty}I_{x}\left(u\right)du,
\end{eqnarray*}
cuando estos l\'imites existan.
\end{Def}

Como consecuencia del teorema de Renovaci\'on-Recompensa, se tiene que el primer l\'imite  existe y es igual a la constante
\begin{eqnarray*}
\overline{X}&=&\frac{\esp\left[\int_{0}^{R_{1}}X\left(t\right)dt\right]}{\esp\left[R_{1}\right]},
\end{eqnarray*}
suponiendo que ambas esperanzas son finitas.
 
\begin{Note}
Funciones de procesos regenerativos son regenerativas, es decir, si $X\left(t\right)$ es regenerativo y se define el proceso $Y\left(t\right)$ por $Y\left(t\right)=f\left(X\left(t\right)\right)$ para alguna funci\'on Borel medible $f\left(\cdot\right)$. Adem\'as $Y$ es regenerativo con los mismos tiempos de renovaci\'on que $X$. 

En general, los tiempos de renovaci\'on, $Z_{k}$ de un proceso regenerativo no requieren ser tiempos de paro con respecto a la evoluci\'on de $X\left(t\right)$.
\end{Note} 

\begin{Note}
Una funci\'on de un proceso de Markov, usualmente no ser\'a un proceso de Markov, sin embargo ser\'a regenerativo si el proceso de Markov lo es.
\end{Note}

 
\begin{Note}
Un proceso regenerativo con media de la longitud de ciclo finita es llamado positivo recurrente.
\end{Note}


\begin{Note}
\begin{itemize}
\item[a)] Si el proceso regenerativo $X$ es positivo recurrente y tiene trayectorias muestrales no negativas, entonces la ecuaci\'on anterior es v\'alida.
\item[b)] Si $X$ es positivo recurrente regenerativo, podemos construir una \'unica versi\'on estacionaria de este proceso, $X_{e}=\left\{X_{e}\left(t\right)\right\}$, donde $X_{e}$ es un proceso estoc\'astico regenerativo y estrictamente estacionario, con distribuci\'on marginal distribuida como $X_{\infty}$
\end{itemize}
\end{Note}


%__________________________________________________________________________________________
\section{Procesos Regenerativos Estacionarios - Stidham \cite{Stidham}}
%__________________________________________________________________________________________


Un proceso estoc\'astico a tiempo continuo $\left\{V\left(t\right),t\geq0\right\}$ es un proceso regenerativo si existe una sucesi\'on de variables aleatorias independientes e id\'enticamente distribuidas $\left\{X_{1},X_{2},\ldots\right\}$, sucesi\'on de renovaci\'on, tal que para cualquier conjunto de Borel $A$, 

\begin{eqnarray*}
\prob\left\{V\left(t\right)\in A|X_{1}+X_{2}+\cdots+X_{R\left(t\right)}=s,\left\{V\left(\tau\right),\tau<s\right\}\right\}=\prob\left\{V\left(t-s\right)\in A|X_{1}>t-s\right\},
\end{eqnarray*}
para todo $0\leq s\leq t$, donde $R\left(t\right)=\max\left\{X_{1}+X_{2}+\cdots+X_{j}\leq t\right\}=$n\'umero de renovaciones ({\emph{puntos de regeneraci\'on}}) que ocurren en $\left[0,t\right]$. El intervalo $\left[0,X_{1}\right)$ es llamado {\emph{primer ciclo de regeneraci\'on}} de $\left\{V\left(t \right),t\geq0\right\}$, $\left[X_{1},X_{1}+X_{2}\right)$ el {\emph{segundo ciclo de regeneraci\'on}}, y as\'i sucesivamente.

Sea $X=X_{1}$ y sea $F$ la funci\'on de distrbuci\'on de $X$


\begin{Def}
Se define el proceso estacionario, $\left\{V^{*}\left(t\right),t\geq0\right\}$, para $\left\{V\left(t\right),t\geq0\right\}$ por

\begin{eqnarray*}
\prob\left\{V\left(t\right)\in A\right\}=\frac{1}{\esp\left[X\right]}\int_{0}^{\infty}\prob\left\{V\left(t+x\right)\in A|X>x\right\}\left(1-F\left(x\right)\right)dx,
\end{eqnarray*} 
para todo $t\geq0$ y todo conjunto de Borel $A$.
\end{Def}

\begin{Def}
Una distribuci\'on se dice que es {\emph{aritm\'etica}} si todos sus puntos de incremento son m\'ultiplos de la forma $0,\lambda, 2\lambda,\ldots$ para alguna $\lambda>0$ entera.
\end{Def}


\begin{Def}
Una modificaci\'on medible de un proceso $\left\{V\left(t\right),t\geq0\right\}$, es una versi\'on de este, $\left\{V\left(t,w\right)\right\}$ conjuntamente medible para $t\geq0$ y para $w\in S$, $S$ espacio de estados para $\left\{V\left(t\right),t\geq0\right\}$.
\end{Def}

\begin{Teo}
Sea $\left\{V\left(t\right),t\geq\right\}$ un proceso regenerativo no negativo con modificaci\'on medible. Sea $\esp\left[X\right]<\infty$. Entonces el proceso estacionario dado por la ecuaci\'on anterior est\'a bien definido y tiene funci\'on de distribuci\'on independiente de $t$, adem\'as
\begin{itemize}
\item[i)] \begin{eqnarray*}
\esp\left[V^{*}\left(0\right)\right]&=&\frac{\esp\left[\int_{0}^{X}V\left(s\right)ds\right]}{\esp\left[X\right]}\end{eqnarray*}
\item[ii)] Si $\esp\left[V^{*}\left(0\right)\right]<\infty$, equivalentemente, si $\esp\left[\int_{0}^{X}V\left(s\right)ds\right]<\infty$,entonces
\begin{eqnarray*}
\frac{\int_{0}^{t}V\left(s\right)ds}{t}\rightarrow\frac{\esp\left[\int_{0}^{X}V\left(s\right)ds\right]}{\esp\left[X\right]}
\end{eqnarray*}
con probabilidad 1 y en media, cuando $t\rightarrow\infty$.
\end{itemize}
\end{Teo}

\begin{Coro}
Sea $\left\{V\left(t\right),t\geq0\right\}$ un proceso regenerativo no negativo, con modificaci\'on medible. Si $\esp <\infty$, $F$ es no-aritm\'etica, y para todo $x\geq0$, $P\left\{V\left(t\right)\leq x,C>x\right\}$ es de variaci\'on acotada como funci\'on de $t$ en cada intervalo finito $\left[0,\tau\right]$, entonces $V\left(t\right)$ converge en distribuci\'on  cuando $t\rightarrow\infty$ y $$\esp V=\frac{\esp \int_{0}^{X}V\left(s\right)ds}{\esp X}$$
Donde $V$ tiene la distribuci\'on l\'imite de $V\left(t\right)$ cuando $t\rightarrow\infty$.

\end{Coro}

Para el caso discreto se tienen resultados similares.


%__________________________________________________________________________________________
%\subsection*{Procesos Regenerativos Estacionarios - Stidham \cite{Stidham}}
%__________________________________________________________________________________________


Un proceso estoc\'astico a tiempo continuo $\left\{V\left(t\right),t\geq0\right\}$ es un proceso regenerativo si existe una sucesi\'on de variables aleatorias independientes e id\'enticamente distribuidas $\left\{X_{1},X_{2},\ldots\right\}$, sucesi\'on de renovaci\'on, tal que para cualquier conjunto de Borel $A$, 

\begin{eqnarray*}
\prob\left\{V\left(t\right)\in A|X_{1}+X_{2}+\cdots+X_{R\left(t\right)}=s,\left\{V\left(\tau\right),\tau<s\right\}\right\}=\prob\left\{V\left(t-s\right)\in A|X_{1}>t-s\right\},
\end{eqnarray*}
para todo $0\leq s\leq t$, donde $R\left(t\right)=\max\left\{X_{1}+X_{2}+\cdots+X_{j}\leq t\right\}=$n\'umero de renovaciones ({\emph{puntos de regeneraci\'on}}) que ocurren en $\left[0,t\right]$. El intervalo $\left[0,X_{1}\right)$ es llamado {\emph{primer ciclo de regeneraci\'on}} de $\left\{V\left(t \right),t\geq0\right\}$, $\left[X_{1},X_{1}+X_{2}\right)$ el {\emph{segundo ciclo de regeneraci\'on}}, y as\'i sucesivamente.

Sea $X=X_{1}$ y sea $F$ la funci\'on de distrbuci\'on de $X$


\begin{Def}
Se define el proceso estacionario, $\left\{V^{*}\left(t\right),t\geq0\right\}$, para $\left\{V\left(t\right),t\geq0\right\}$ por

\begin{eqnarray*}
\prob\left\{V\left(t\right)\in A\right\}=\frac{1}{\esp\left[X\right]}\int_{0}^{\infty}\prob\left\{V\left(t+x\right)\in A|X>x\right\}\left(1-F\left(x\right)\right)dx,
\end{eqnarray*} 
para todo $t\geq0$ y todo conjunto de Borel $A$.
\end{Def}

\begin{Def}
Una distribuci\'on se dice que es {\emph{aritm\'etica}} si todos sus puntos de incremento son m\'ultiplos de la forma $0,\lambda, 2\lambda,\ldots$ para alguna $\lambda>0$ entera.
\end{Def}


\begin{Def}
Una modificaci\'on medible de un proceso $\left\{V\left(t\right),t\geq0\right\}$, es una versi\'on de este, $\left\{V\left(t,w\right)\right\}$ conjuntamente medible para $t\geq0$ y para $w\in S$, $S$ espacio de estados para $\left\{V\left(t\right),t\geq0\right\}$.
\end{Def}

\begin{Teo}
Sea $\left\{V\left(t\right),t\geq\right\}$ un proceso regenerativo no negativo con modificaci\'on medible. Sea $\esp\left[X\right]<\infty$. Entonces el proceso estacionario dado por la ecuaci\'on anterior est\'a bien definido y tiene funci\'on de distribuci\'on independiente de $t$, adem\'as
\begin{itemize}
\item[i)] \begin{eqnarray*}
\esp\left[V^{*}\left(0\right)\right]&=&\frac{\esp\left[\int_{0}^{X}V\left(s\right)ds\right]}{\esp\left[X\right]}\end{eqnarray*}
\item[ii)] Si $\esp\left[V^{*}\left(0\right)\right]<\infty$, equivalentemente, si $\esp\left[\int_{0}^{X}V\left(s\right)ds\right]<\infty$,entonces
\begin{eqnarray*}
\frac{\int_{0}^{t}V\left(s\right)ds}{t}\rightarrow\frac{\esp\left[\int_{0}^{X}V\left(s\right)ds\right]}{\esp\left[X\right]}
\end{eqnarray*}
con probabilidad 1 y en media, cuando $t\rightarrow\infty$.
\end{itemize}
\end{Teo}


%__________________________________________________________________________________________
%\subsection*{Procesos Regenerativos Estacionarios - Stidham \cite{Stidham}}
%__________________________________________________________________________________________


Un proceso estoc\'astico a tiempo continuo $\left\{V\left(t\right),t\geq0\right\}$ es un proceso regenerativo si existe una sucesi\'on de variables aleatorias independientes e id\'enticamente distribuidas $\left\{X_{1},X_{2},\ldots\right\}$, sucesi\'on de renovaci\'on, tal que para cualquier conjunto de Borel $A$, 

\begin{eqnarray*}
\prob\left\{V\left(t\right)\in A|X_{1}+X_{2}+\cdots+X_{R\left(t\right)}=s,\left\{V\left(\tau\right),\tau<s\right\}\right\}=\prob\left\{V\left(t-s\right)\in A|X_{1}>t-s\right\},
\end{eqnarray*}
para todo $0\leq s\leq t$, donde $R\left(t\right)=\max\left\{X_{1}+X_{2}+\cdots+X_{j}\leq t\right\}=$n\'umero de renovaciones ({\emph{puntos de regeneraci\'on}}) que ocurren en $\left[0,t\right]$. El intervalo $\left[0,X_{1}\right)$ es llamado {\emph{primer ciclo de regeneraci\'on}} de $\left\{V\left(t \right),t\geq0\right\}$, $\left[X_{1},X_{1}+X_{2}\right)$ el {\emph{segundo ciclo de regeneraci\'on}}, y as\'i sucesivamente.

Sea $X=X_{1}$ y sea $F$ la funci\'on de distrbuci\'on de $X$


\begin{Def}
Se define el proceso estacionario, $\left\{V^{*}\left(t\right),t\geq0\right\}$, para $\left\{V\left(t\right),t\geq0\right\}$ por

\begin{eqnarray*}
\prob\left\{V\left(t\right)\in A\right\}=\frac{1}{\esp\left[X\right]}\int_{0}^{\infty}\prob\left\{V\left(t+x\right)\in A|X>x\right\}\left(1-F\left(x\right)\right)dx,
\end{eqnarray*} 
para todo $t\geq0$ y todo conjunto de Borel $A$.
\end{Def}

\begin{Def}
Una distribuci\'on se dice que es {\emph{aritm\'etica}} si todos sus puntos de incremento son m\'ultiplos de la forma $0,\lambda, 2\lambda,\ldots$ para alguna $\lambda>0$ entera.
\end{Def}


\begin{Def}
Una modificaci\'on medible de un proceso $\left\{V\left(t\right),t\geq0\right\}$, es una versi\'on de este, $\left\{V\left(t,w\right)\right\}$ conjuntamente medible para $t\geq0$ y para $w\in S$, $S$ espacio de estados para $\left\{V\left(t\right),t\geq0\right\}$.
\end{Def}

\begin{Teo}
Sea $\left\{V\left(t\right),t\geq\right\}$ un proceso regenerativo no negativo con modificaci\'on medible. Sea $\esp\left[X\right]<\infty$. Entonces el proceso estacionario dado por la ecuaci\'on anterior est\'a bien definido y tiene funci\'on de distribuci\'on independiente de $t$, adem\'as
\begin{itemize}
\item[i)] \begin{eqnarray*}
\esp\left[V^{*}\left(0\right)\right]&=&\frac{\esp\left[\int_{0}^{X}V\left(s\right)ds\right]}{\esp\left[X\right]}\end{eqnarray*}
\item[ii)] Si $\esp\left[V^{*}\left(0\right)\right]<\infty$, equivalentemente, si $\esp\left[\int_{0}^{X}V\left(s\right)ds\right]<\infty$,entonces
\begin{eqnarray*}
\frac{\int_{0}^{t}V\left(s\right)ds}{t}\rightarrow\frac{\esp\left[\int_{0}^{X}V\left(s\right)ds\right]}{\esp\left[X\right]}
\end{eqnarray*}
con probabilidad 1 y en media, cuando $t\rightarrow\infty$.
\end{itemize}
\end{Teo}

Sea la funci\'on generadora de momentos para $L_{i}$, el n\'umero de usuarios en la cola $Q_{i}\left(z\right)$ en cualquier momento, est\'a dada por el tiempo promedio de $z^{L_{i}\left(t\right)}$ sobre el ciclo regenerativo definido anteriormente. Entonces 



Es decir, es posible determinar las longitudes de las colas a cualquier tiempo $t$. Entonces, determinando el primer momento es posible ver que


\begin{Def}
El tiempo de Ciclo $C_{i}$ es el periodo de tiempo que comienza cuando la cola $i$ es visitada por primera vez en un ciclo, y termina cuando es visitado nuevamente en el pr\'oximo ciclo. La duraci\'on del mismo est\'a dada por $\tau_{i}\left(m+1\right)-\tau_{i}\left(m\right)$, o equivalentemente $\overline{\tau}_{i}\left(m+1\right)-\overline{\tau}_{i}\left(m\right)$ bajo condiciones de estabilidad.
\end{Def}


\begin{Def}
El tiempo de intervisita $I_{i}$ es el periodo de tiempo que comienza cuando se ha completado el servicio en un ciclo y termina cuando es visitada nuevamente en el pr\'oximo ciclo. Su  duraci\'on del mismo est\'a dada por $\tau_{i}\left(m+1\right)-\overline{\tau}_{i}\left(m\right)$.
\end{Def}

La duraci\'on del tiempo de intervisita es $\tau_{i}\left(m+1\right)-\overline{\tau}\left(m\right)$. Dado que el n\'umero de usuarios presentes en $Q_{i}$ al tiempo $t=\tau_{i}\left(m+1\right)$ es igual al n\'umero de arribos durante el intervalo de tiempo $\left[\overline{\tau}\left(m\right),\tau_{i}\left(m+1\right)\right]$ se tiene que


\begin{eqnarray*}
\esp\left[z_{i}^{L_{i}\left(\tau_{i}\left(m+1\right)\right)}\right]=\esp\left[\left\{P_{i}\left(z_{i}\right)\right\}^{\tau_{i}\left(m+1\right)-\overline{\tau}\left(m\right)}\right]
\end{eqnarray*}

entonces, si $I_{i}\left(z\right)=\esp\left[z^{\tau_{i}\left(m+1\right)-\overline{\tau}\left(m\right)}\right]$
se tiene que $F_{i}\left(z\right)=I_{i}\left[P_{i}\left(z\right)\right]$
para $i=1,2$.

Conforme a la definici\'on dada al principio del cap\'itulo, definici\'on (\ref{Def.Tn}), sean $T_{1},T_{2},\ldots$ los puntos donde las longitudes de las colas de la red de sistemas de visitas c\'iclicas son cero simult\'aneamente, cuando la cola $Q_{j}$ es visitada por el servidor para dar servicio, es decir, $L_{1}\left(T_{i}\right)=0,L_{2}\left(T_{i}\right)=0,\hat{L}_{1}\left(T_{i}\right)=0$ y $\hat{L}_{2}\left(T_{i}\right)=0$, a estos puntos se les denominar\'a puntos regenerativos. Entonces, 

\begin{Def}
Al intervalo de tiempo entre dos puntos regenerativos se le llamar\'a ciclo regenerativo.
\end{Def}

\begin{Def}
Para $T_{i}$ se define, $M_{i}$, el n\'umero de ciclos de visita a la cola $Q_{l}$, durante el ciclo regenerativo, es decir, $M_{i}$ es un proceso de renovaci\'on.
\end{Def}

\begin{Def}
Para cada uno de los $M_{i}$'s, se definen a su vez la duraci\'on de cada uno de estos ciclos de visita en el ciclo regenerativo, $C_{i}^{(m)}$, para $m=1,2,\ldots,M_{i}$, que a su vez, tambi\'en es n proceso de renovaci\'on.
\end{Def}

\footnote{In Stidham and  Heyman \cite{Stidham} shows that is sufficient for the regenerative process to be stationary that the mean regenerative cycle time is finite: $\esp\left[\sum_{m=1}^{M_{i}}C_{i}^{(m)}\right]<\infty$, 


 como cada $C_{i}^{(m)}$ contiene intervalos de r\'eplica positivos, se tiene que $\esp\left[M_{i}\right]<\infty$, adem\'as, como $M_{i}>0$, se tiene que la condici\'on anterior es equivalente a tener que $\esp\left[C_{i}\right]<\infty$,
por lo tanto una condici\'on suficiente para la existencia del proceso regenerativo est\'a dada por $\sum_{k=1}^{N}\mu_{k}<1.$}

%__________________________________________________________________________________________
%\subsection*{Procesos Regenerativos Estacionarios - Stidham \cite{Stidham}}
%__________________________________________________________________________________________


Un proceso estoc\'astico a tiempo continuo $\left\{V\left(t\right),t\geq0\right\}$ es un proceso regenerativo si existe una sucesi\'on de variables aleatorias independientes e id\'enticamente distribuidas $\left\{X_{1},X_{2},\ldots\right\}$, sucesi\'on de renovaci\'on, tal que para cualquier conjunto de Borel $A$, 

\begin{eqnarray*}
\prob\left\{V\left(t\right)\in A|X_{1}+X_{2}+\cdots+X_{R\left(t\right)}=s,\left\{V\left(\tau\right),\tau<s\right\}\right\}=\prob\left\{V\left(t-s\right)\in A|X_{1}>t-s\right\},
\end{eqnarray*}
para todo $0\leq s\leq t$, donde $R\left(t\right)=\max\left\{X_{1}+X_{2}+\cdots+X_{j}\leq t\right\}=$n\'umero de renovaciones ({\emph{puntos de regeneraci\'on}}) que ocurren en $\left[0,t\right]$. El intervalo $\left[0,X_{1}\right)$ es llamado {\emph{primer ciclo de regeneraci\'on}} de $\left\{V\left(t \right),t\geq0\right\}$, $\left[X_{1},X_{1}+X_{2}\right)$ el {\emph{segundo ciclo de regeneraci\'on}}, y as\'i sucesivamente.

Sea $X=X_{1}$ y sea $F$ la funci\'on de distrbuci\'on de $X$


\begin{Def}
Se define el proceso estacionario, $\left\{V^{*}\left(t\right),t\geq0\right\}$, para $\left\{V\left(t\right),t\geq0\right\}$ por

\begin{eqnarray*}
\prob\left\{V\left(t\right)\in A\right\}=\frac{1}{\esp\left[X\right]}\int_{0}^{\infty}\prob\left\{V\left(t+x\right)\in A|X>x\right\}\left(1-F\left(x\right)\right)dx,
\end{eqnarray*} 
para todo $t\geq0$ y todo conjunto de Borel $A$.
\end{Def}

\begin{Def}
Una distribuci\'on se dice que es {\emph{aritm\'etica}} si todos sus puntos de incremento son m\'ultiplos de la forma $0,\lambda, 2\lambda,\ldots$ para alguna $\lambda>0$ entera.
\end{Def}


\begin{Def}
Una modificaci\'on medible de un proceso $\left\{V\left(t\right),t\geq0\right\}$, es una versi\'on de este, $\left\{V\left(t,w\right)\right\}$ conjuntamente medible para $t\geq0$ y para $w\in S$, $S$ espacio de estados para $\left\{V\left(t\right),t\geq0\right\}$.
\end{Def}

\begin{Teo}
Sea $\left\{V\left(t\right),t\geq\right\}$ un proceso regenerativo no negativo con modificaci\'on medible. Sea $\esp\left[X\right]<\infty$. Entonces el proceso estacionario dado por la ecuaci\'on anterior est\'a bien definido y tiene funci\'on de distribuci\'on independiente de $t$, adem\'as
\begin{itemize}
\item[i)] \begin{eqnarray*}
\esp\left[V^{*}\left(0\right)\right]&=&\frac{\esp\left[\int_{0}^{X}V\left(s\right)ds\right]}{\esp\left[X\right]}\end{eqnarray*}
\item[ii)] Si $\esp\left[V^{*}\left(0\right)\right]<\infty$, equivalentemente, si $\esp\left[\int_{0}^{X}V\left(s\right)ds\right]<\infty$,entonces
\begin{eqnarray*}
\frac{\int_{0}^{t}V\left(s\right)ds}{t}\rightarrow\frac{\esp\left[\int_{0}^{X}V\left(s\right)ds\right]}{\esp\left[X\right]}
\end{eqnarray*}
con probabilidad 1 y en media, cuando $t\rightarrow\infty$.
\end{itemize}
\end{Teo}

\begin{Coro}
Sea $\left\{V\left(t\right),t\geq0\right\}$ un proceso regenerativo no negativo, con modificaci\'on medible. Si $\esp <\infty$, $F$ es no-aritm\'etica, y para todo $x\geq0$, $P\left\{V\left(t\right)\leq x,C>x\right\}$ es de variaci\'on acotada como funci\'on de $t$ en cada intervalo finito $\left[0,\tau\right]$, entonces $V\left(t\right)$ converge en distribuci\'on  cuando $t\rightarrow\infty$ y $$\esp V=\frac{\esp \int_{0}^{X}V\left(s\right)ds}{\esp X}$$
Donde $V$ tiene la distribuci\'on l\'imite de $V\left(t\right)$ cuando $t\rightarrow\infty$.

\end{Coro}

Para el caso discreto se tienen resultados similares.



%__________________________________________________________________________________________
%\subsection*{Procesos Regenerativos Estacionarios - Stidham \cite{Stidham}}
%__________________________________________________________________________________________


Un proceso estoc\'astico a tiempo continuo $\left\{V\left(t\right),t\geq0\right\}$ es un proceso regenerativo si existe una sucesi\'on de variables aleatorias independientes e id\'enticamente distribuidas $\left\{X_{1},X_{2},\ldots\right\}$, sucesi\'on de renovaci\'on, tal que para cualquier conjunto de Borel $A$, 

\begin{eqnarray*}
\prob\left\{V\left(t\right)\in A|X_{1}+X_{2}+\cdots+X_{R\left(t\right)}=s,\left\{V\left(\tau\right),\tau<s\right\}\right\}=\prob\left\{V\left(t-s\right)\in A|X_{1}>t-s\right\},
\end{eqnarray*}
para todo $0\leq s\leq t$, donde $R\left(t\right)=\max\left\{X_{1}+X_{2}+\cdots+X_{j}\leq t\right\}=$n\'umero de renovaciones ({\emph{puntos de regeneraci\'on}}) que ocurren en $\left[0,t\right]$. El intervalo $\left[0,X_{1}\right)$ es llamado {\emph{primer ciclo de regeneraci\'on}} de $\left\{V\left(t \right),t\geq0\right\}$, $\left[X_{1},X_{1}+X_{2}\right)$ el {\emph{segundo ciclo de regeneraci\'on}}, y as\'i sucesivamente.

Sea $X=X_{1}$ y sea $F$ la funci\'on de distrbuci\'on de $X$


\begin{Def}
Se define el proceso estacionario, $\left\{V^{*}\left(t\right),t\geq0\right\}$, para $\left\{V\left(t\right),t\geq0\right\}$ por

\begin{eqnarray*}
\prob\left\{V\left(t\right)\in A\right\}=\frac{1}{\esp\left[X\right]}\int_{0}^{\infty}\prob\left\{V\left(t+x\right)\in A|X>x\right\}\left(1-F\left(x\right)\right)dx,
\end{eqnarray*} 
para todo $t\geq0$ y todo conjunto de Borel $A$.
\end{Def}

\begin{Def}
Una distribuci\'on se dice que es {\emph{aritm\'etica}} si todos sus puntos de incremento son m\'ultiplos de la forma $0,\lambda, 2\lambda,\ldots$ para alguna $\lambda>0$ entera.
\end{Def}


\begin{Def}
Una modificaci\'on medible de un proceso $\left\{V\left(t\right),t\geq0\right\}$, es una versi\'on de este, $\left\{V\left(t,w\right)\right\}$ conjuntamente medible para $t\geq0$ y para $w\in S$, $S$ espacio de estados para $\left\{V\left(t\right),t\geq0\right\}$.
\end{Def}

\begin{Teo}
Sea $\left\{V\left(t\right),t\geq\right\}$ un proceso regenerativo no negativo con modificaci\'on medible. Sea $\esp\left[X\right]<\infty$. Entonces el proceso estacionario dado por la ecuaci\'on anterior est\'a bien definido y tiene funci\'on de distribuci\'on independiente de $t$, adem\'as
\begin{itemize}
\item[i)] \begin{eqnarray*}
\esp\left[V^{*}\left(0\right)\right]&=&\frac{\esp\left[\int_{0}^{X}V\left(s\right)ds\right]}{\esp\left[X\right]}\end{eqnarray*}
\item[ii)] Si $\esp\left[V^{*}\left(0\right)\right]<\infty$, equivalentemente, si $\esp\left[\int_{0}^{X}V\left(s\right)ds\right]<\infty$,entonces
\begin{eqnarray*}
\frac{\int_{0}^{t}V\left(s\right)ds}{t}\rightarrow\frac{\esp\left[\int_{0}^{X}V\left(s\right)ds\right]}{\esp\left[X\right]}
\end{eqnarray*}
con probabilidad 1 y en media, cuando $t\rightarrow\infty$.
\end{itemize}
\end{Teo}

%__________________________________________________________________________________________
%\subsection*{Procesos Regenerativos Estacionarios - Stidham \cite{Stidham}}
%__________________________________________________________________________________________


Un proceso estoc\'astico a tiempo continuo $\left\{V\left(t\right),t\geq0\right\}$ es un proceso regenerativo si existe una sucesi\'on de variables aleatorias independientes e id\'enticamente distribuidas $\left\{X_{1},X_{2},\ldots\right\}$, sucesi\'on de renovaci\'on, tal que para cualquier conjunto de Borel $A$, 

\begin{eqnarray*}
\prob\left\{V\left(t\right)\in A|X_{1}+X_{2}+\cdots+X_{R\left(t\right)}=s,\left\{V\left(\tau\right),\tau<s\right\}\right\}=\prob\left\{V\left(t-s\right)\in A|X_{1}>t-s\right\},
\end{eqnarray*}
para todo $0\leq s\leq t$, donde $R\left(t\right)=\max\left\{X_{1}+X_{2}+\cdots+X_{j}\leq t\right\}=$n\'umero de renovaciones ({\emph{puntos de regeneraci\'on}}) que ocurren en $\left[0,t\right]$. El intervalo $\left[0,X_{1}\right)$ es llamado {\emph{primer ciclo de regeneraci\'on}} de $\left\{V\left(t \right),t\geq0\right\}$, $\left[X_{1},X_{1}+X_{2}\right)$ el {\emph{segundo ciclo de regeneraci\'on}}, y as\'i sucesivamente.

Sea $X=X_{1}$ y sea $F$ la funci\'on de distrbuci\'on de $X$


\begin{Def}
Se define el proceso estacionario, $\left\{V^{*}\left(t\right),t\geq0\right\}$, para $\left\{V\left(t\right),t\geq0\right\}$ por

\begin{eqnarray*}
\prob\left\{V\left(t\right)\in A\right\}=\frac{1}{\esp\left[X\right]}\int_{0}^{\infty}\prob\left\{V\left(t+x\right)\in A|X>x\right\}\left(1-F\left(x\right)\right)dx,
\end{eqnarray*} 
para todo $t\geq0$ y todo conjunto de Borel $A$.
\end{Def}

\begin{Def}
Una distribuci\'on se dice que es {\emph{aritm\'etica}} si todos sus puntos de incremento son m\'ultiplos de la forma $0,\lambda, 2\lambda,\ldots$ para alguna $\lambda>0$ entera.
\end{Def}


\begin{Def}
Una modificaci\'on medible de un proceso $\left\{V\left(t\right),t\geq0\right\}$, es una versi\'on de este, $\left\{V\left(t,w\right)\right\}$ conjuntamente medible para $t\geq0$ y para $w\in S$, $S$ espacio de estados para $\left\{V\left(t\right),t\geq0\right\}$.
\end{Def}

\begin{Teo}
Sea $\left\{V\left(t\right),t\geq\right\}$ un proceso regenerativo no negativo con modificaci\'on medible. Sea $\esp\left[X\right]<\infty$. Entonces el proceso estacionario dado por la ecuaci\'on anterior est\'a bien definido y tiene funci\'on de distribuci\'on independiente de $t$, adem\'as
\begin{itemize}
\item[i)] \begin{eqnarray*}
\esp\left[V^{*}\left(0\right)\right]&=&\frac{\esp\left[\int_{0}^{X}V\left(s\right)ds\right]}{\esp\left[X\right]}\end{eqnarray*}
\item[ii)] Si $\esp\left[V^{*}\left(0\right)\right]<\infty$, equivalentemente, si $\esp\left[\int_{0}^{X}V\left(s\right)ds\right]<\infty$,entonces
\begin{eqnarray*}
\frac{\int_{0}^{t}V\left(s\right)ds}{t}\rightarrow\frac{\esp\left[\int_{0}^{X}V\left(s\right)ds\right]}{\esp\left[X\right]}
\end{eqnarray*}
con probabilidad 1 y en media, cuando $t\rightarrow\infty$.
\end{itemize}
\end{Teo}

Para $\left\{X\left(t\right):t\geq0\right\}$ Proceso Estoc\'astico a tiempo continuo con estado de espacios $S$, que es un espacio m\'etrico, con trayectorias continuas por la derecha y con l\'imites por la izquierda c.s. Sea $N\left(t\right)$ un proceso de renovaci\'on en $\rea_{+}$ definido en el mismo espacio de probabilidad que $X\left(t\right)$, con tiempos de renovaci\'on $T$ y tiempos de inter-renovaci\'on $\xi_{n}=T_{n}-T_{n-1}$, con misma distribuci\'on $F$ de media finita $\mu$.

%__________________________________________________________________________________________
%\subsection*{Procesos Regenerativos Estacionarios - Stidham \cite{Stidham}}
%__________________________________________________________________________________________


Un proceso estoc\'astico a tiempo continuo $\left\{V\left(t\right),t\geq0\right\}$ es un proceso regenerativo si existe una sucesi\'on de variables aleatorias independientes e id\'enticamente distribuidas $\left\{X_{1},X_{2},\ldots\right\}$, sucesi\'on de renovaci\'on, tal que para cualquier conjunto de Borel $A$, 

\begin{eqnarray*}
\prob\left\{V\left(t\right)\in A|X_{1}+X_{2}+\cdots+X_{R\left(t\right)}=s,\left\{V\left(\tau\right),\tau<s\right\}\right\}=\prob\left\{V\left(t-s\right)\in A|X_{1}>t-s\right\},
\end{eqnarray*}
para todo $0\leq s\leq t$, donde $R\left(t\right)=\max\left\{X_{1}+X_{2}+\cdots+X_{j}\leq t\right\}=$n\'umero de renovaciones ({\emph{puntos de regeneraci\'on}}) que ocurren en $\left[0,t\right]$. El intervalo $\left[0,X_{1}\right)$ es llamado {\emph{primer ciclo de regeneraci\'on}} de $\left\{V\left(t \right),t\geq0\right\}$, $\left[X_{1},X_{1}+X_{2}\right)$ el {\emph{segundo ciclo de regeneraci\'on}}, y as\'i sucesivamente.

Sea $X=X_{1}$ y sea $F$ la funci\'on de distrbuci\'on de $X$


\begin{Def}
Se define el proceso estacionario, $\left\{V^{*}\left(t\right),t\geq0\right\}$, para $\left\{V\left(t\right),t\geq0\right\}$ por

\begin{eqnarray*}
\prob\left\{V\left(t\right)\in A\right\}=\frac{1}{\esp\left[X\right]}\int_{0}^{\infty}\prob\left\{V\left(t+x\right)\in A|X>x\right\}\left(1-F\left(x\right)\right)dx,
\end{eqnarray*} 
para todo $t\geq0$ y todo conjunto de Borel $A$.
\end{Def}

\begin{Def}
Una distribuci\'on se dice que es {\emph{aritm\'etica}} si todos sus puntos de incremento son m\'ultiplos de la forma $0,\lambda, 2\lambda,\ldots$ para alguna $\lambda>0$ entera.
\end{Def}


\begin{Def}
Una modificaci\'on medible de un proceso $\left\{V\left(t\right),t\geq0\right\}$, es una versi\'on de este, $\left\{V\left(t,w\right)\right\}$ conjuntamente medible para $t\geq0$ y para $w\in S$, $S$ espacio de estados para $\left\{V\left(t\right),t\geq0\right\}$.
\end{Def}

\begin{Teo}
Sea $\left\{V\left(t\right),t\geq\right\}$ un proceso regenerativo no negativo con modificaci\'on medible. Sea $\esp\left[X\right]<\infty$. Entonces el proceso estacionario dado por la ecuaci\'on anterior est\'a bien definido y tiene funci\'on de distribuci\'on independiente de $t$, adem\'as
\begin{itemize}
\item[i)] \begin{eqnarray*}
\esp\left[V^{*}\left(0\right)\right]&=&\frac{\esp\left[\int_{0}^{X}V\left(s\right)ds\right]}{\esp\left[X\right]}\end{eqnarray*}
\item[ii)] Si $\esp\left[V^{*}\left(0\right)\right]<\infty$, equivalentemente, si $\esp\left[\int_{0}^{X}V\left(s\right)ds\right]<\infty$,entonces
\begin{eqnarray*}
\frac{\int_{0}^{t}V\left(s\right)ds}{t}\rightarrow\frac{\esp\left[\int_{0}^{X}V\left(s\right)ds\right]}{\esp\left[X\right]}
\end{eqnarray*}
con probabilidad 1 y en media, cuando $t\rightarrow\infty$.
\end{itemize}
\end{Teo}

\begin{Coro}
Sea $\left\{V\left(t\right),t\geq0\right\}$ un proceso regenerativo no negativo, con modificaci\'on medible. Si $\esp <\infty$, $F$ es no-aritm\'etica, y para todo $x\geq0$, $P\left\{V\left(t\right)\leq x,C>x\right\}$ es de variaci\'on acotada como funci\'on de $t$ en cada intervalo finito $\left[0,\tau\right]$, entonces $V\left(t\right)$ converge en distribuci\'on  cuando $t\rightarrow\infty$ y $$\esp V=\frac{\esp \int_{0}^{X}V\left(s\right)ds}{\esp X}$$
Donde $V$ tiene la distribuci\'on l\'imite de $V\left(t\right)$ cuando $t\rightarrow\infty$.

\end{Coro}

Para el caso discreto se tienen resultados similares.


%___________________________________________________________________________________________
%
\section{Teorema Principal de Renovaci\'on}
%___________________________________________________________________________________________
%

\begin{Note} Una funci\'on $h:\rea_{+}\rightarrow\rea$ es Directamente Riemann Integrable en los siguientes casos:
\begin{itemize}
\item[a)] $h\left(t\right)\geq0$ es decreciente y Riemann Integrable.
\item[b)] $h$ es continua excepto posiblemente en un conjunto de Lebesgue de medida 0, y $|h\left(t\right)|\leq b\left(t\right)$, donde $b$ es DRI.
\end{itemize}
\end{Note}

\begin{Teo}[Teorema Principal de Renovaci\'on]
Si $F$ es no aritm\'etica y $h\left(t\right)$ es Directamente Riemann Integrable (DRI), entonces

\begin{eqnarray*}
lim_{t\rightarrow\infty}U\star h=\frac{1}{\mu}\int_{\rea_{+}}h\left(s\right)ds.
\end{eqnarray*}
\end{Teo}

\begin{Prop}
Cualquier funci\'on $H\left(t\right)$ acotada en intervalos finitos y que es 0 para $t<0$ puede expresarse como
\begin{eqnarray*}
H\left(t\right)=U\star h\left(t\right)\textrm{,  donde }h\left(t\right)=H\left(t\right)-F\star H\left(t\right)
\end{eqnarray*}
\end{Prop}

\begin{Def}
Un proceso estoc\'astico $X\left(t\right)$ es crudamente regenerativo en un tiempo aleatorio positivo $T$ si
\begin{eqnarray*}
\esp\left[X\left(T+t\right)|T\right]=\esp\left[X\left(t\right)\right]\textrm{, para }t\geq0,\end{eqnarray*}
y con las esperanzas anteriores finitas.
\end{Def}

\begin{Prop}
Sup\'ongase que $X\left(t\right)$ es un proceso crudamente regenerativo en $T$, que tiene distribuci\'on $F$. Si $\esp\left[X\left(t\right)\right]$ es acotado en intervalos finitos, entonces
\begin{eqnarray*}
\esp\left[X\left(t\right)\right]=U\star h\left(t\right)\textrm{,  donde }h\left(t\right)=\esp\left[X\left(t\right)\indora\left(T>t\right)\right].
\end{eqnarray*}
\end{Prop}

\begin{Teo}[Regeneraci\'on Cruda]
Sup\'ongase que $X\left(t\right)$ es un proceso con valores positivo crudamente regenerativo en $T$, y def\'inase $M=\sup\left\{|X\left(t\right)|:t\leq T\right\}$. Si $T$ es no aritm\'etico y $M$ y $MT$ tienen media finita, entonces
\begin{eqnarray*}
lim_{t\rightarrow\infty}\esp\left[X\left(t\right)\right]=\frac{1}{\mu}\int_{\rea_{+}}h\left(s\right)ds,
\end{eqnarray*}
donde $h\left(t\right)=\esp\left[X\left(t\right)\indora\left(T>t\right)\right]$.
\end{Teo}

%___________________________________________________________________________________________
%
%\subsection*{Teorema Principal de Renovaci\'on}
%___________________________________________________________________________________________
%

\begin{Note} Una funci\'on $h:\rea_{+}\rightarrow\rea$ es Directamente Riemann Integrable en los siguientes casos:
\begin{itemize}
\item[a)] $h\left(t\right)\geq0$ es decreciente y Riemann Integrable.
\item[b)] $h$ es continua excepto posiblemente en un conjunto de Lebesgue de medida 0, y $|h\left(t\right)|\leq b\left(t\right)$, donde $b$ es DRI.
\end{itemize}
\end{Note}

\begin{Teo}[Teorema Principal de Renovaci\'on]
Si $F$ es no aritm\'etica y $h\left(t\right)$ es Directamente Riemann Integrable (DRI), entonces

\begin{eqnarray*}
lim_{t\rightarrow\infty}U\star h=\frac{1}{\mu}\int_{\rea_{+}}h\left(s\right)ds.
\end{eqnarray*}
\end{Teo}

\begin{Prop}
Cualquier funci\'on $H\left(t\right)$ acotada en intervalos finitos y que es 0 para $t<0$ puede expresarse como
\begin{eqnarray*}
H\left(t\right)=U\star h\left(t\right)\textrm{,  donde }h\left(t\right)=H\left(t\right)-F\star H\left(t\right)
\end{eqnarray*}
\end{Prop}

\begin{Def}
Un proceso estoc\'astico $X\left(t\right)$ es crudamente regenerativo en un tiempo aleatorio positivo $T$ si
\begin{eqnarray*}
\esp\left[X\left(T+t\right)|T\right]=\esp\left[X\left(t\right)\right]\textrm{, para }t\geq0,\end{eqnarray*}
y con las esperanzas anteriores finitas.
\end{Def}

\begin{Prop}
Sup\'ongase que $X\left(t\right)$ es un proceso crudamente regenerativo en $T$, que tiene distribuci\'on $F$. Si $\esp\left[X\left(t\right)\right]$ es acotado en intervalos finitos, entonces
\begin{eqnarray*}
\esp\left[X\left(t\right)\right]=U\star h\left(t\right)\textrm{,  donde }h\left(t\right)=\esp\left[X\left(t\right)\indora\left(T>t\right)\right].
\end{eqnarray*}
\end{Prop}

\begin{Teo}[Regeneraci\'on Cruda]
Sup\'ongase que $X\left(t\right)$ es un proceso con valores positivo crudamente regenerativo en $T$, y def\'inase $M=\sup\left\{|X\left(t\right)|:t\leq T\right\}$. Si $T$ es no aritm\'etico y $M$ y $MT$ tienen media finita, entonces
\begin{eqnarray*}
lim_{t\rightarrow\infty}\esp\left[X\left(t\right)\right]=\frac{1}{\mu}\int_{\rea_{+}}h\left(s\right)ds,
\end{eqnarray*}
donde $h\left(t\right)=\esp\left[X\left(t\right)\indora\left(T>t\right)\right]$.
\end{Teo}


%___________________________________________________________________________________________
%
%\subsection*{Teorema Principal de Renovaci\'on}
%___________________________________________________________________________________________
%

\begin{Note} Una funci\'on $h:\rea_{+}\rightarrow\rea$ es Directamente Riemann Integrable en los siguientes casos:
\begin{itemize}
\item[a)] $h\left(t\right)\geq0$ es decreciente y Riemann Integrable.
\item[b)] $h$ es continua excepto posiblemente en un conjunto de Lebesgue de medida 0, y $|h\left(t\right)|\leq b\left(t\right)$, donde $b$ es DRI.
\end{itemize}
\end{Note}

\begin{Teo}[Teorema Principal de Renovaci\'on]
Si $F$ es no aritm\'etica y $h\left(t\right)$ es Directamente Riemann Integrable (DRI), entonces

\begin{eqnarray*}
lim_{t\rightarrow\infty}U\star h=\frac{1}{\mu}\int_{\rea_{+}}h\left(s\right)ds.
\end{eqnarray*}
\end{Teo}

\begin{Prop}
Cualquier funci\'on $H\left(t\right)$ acotada en intervalos finitos y que es 0 para $t<0$ puede expresarse como
\begin{eqnarray*}
H\left(t\right)=U\star h\left(t\right)\textrm{,  donde }h\left(t\right)=H\left(t\right)-F\star H\left(t\right)
\end{eqnarray*}
\end{Prop}

\begin{Def}
Un proceso estoc\'astico $X\left(t\right)$ es crudamente regenerativo en un tiempo aleatorio positivo $T$ si
\begin{eqnarray*}
\esp\left[X\left(T+t\right)|T\right]=\esp\left[X\left(t\right)\right]\textrm{, para }t\geq0,\end{eqnarray*}
y con las esperanzas anteriores finitas.
\end{Def}

\begin{Prop}
Sup\'ongase que $X\left(t\right)$ es un proceso crudamente regenerativo en $T$, que tiene distribuci\'on $F$. Si $\esp\left[X\left(t\right)\right]$ es acotado en intervalos finitos, entonces
\begin{eqnarray*}
\esp\left[X\left(t\right)\right]=U\star h\left(t\right)\textrm{,  donde }h\left(t\right)=\esp\left[X\left(t\right)\indora\left(T>t\right)\right].
\end{eqnarray*}
\end{Prop}

\begin{Teo}[Regeneraci\'on Cruda]
Sup\'ongase que $X\left(t\right)$ es un proceso con valores positivo crudamente regenerativo en $T$, y def\'inase $M=\sup\left\{|X\left(t\right)|:t\leq T\right\}$. Si $T$ es no aritm\'etico y $M$ y $MT$ tienen media finita, entonces
\begin{eqnarray*}
lim_{t\rightarrow\infty}\esp\left[X\left(t\right)\right]=\frac{1}{\mu}\int_{\rea_{+}}h\left(s\right)ds,
\end{eqnarray*}
donde $h\left(t\right)=\esp\left[X\left(t\right)\indora\left(T>t\right)\right]$.
\end{Teo}

%___________________________________________________________________________________________
%
\section{Propiedades de los Procesos de Renovaci\'on}
%___________________________________________________________________________________________
%

Los tiempos $T_{n}$ est\'an relacionados con los conteos de $N\left(t\right)$ por

\begin{eqnarray*}
\left\{N\left(t\right)\geq n\right\}&=&\left\{T_{n}\leq t\right\}\\
T_{N\left(t\right)}\leq &t&<T_{N\left(t\right)+1},
\end{eqnarray*}

adem\'as $N\left(T_{n}\right)=n$, y 

\begin{eqnarray*}
N\left(t\right)=\max\left\{n:T_{n}\leq t\right\}=\min\left\{n:T_{n+1}>t\right\}
\end{eqnarray*}

Por propiedades de la convoluci\'on se sabe que

\begin{eqnarray*}
P\left\{T_{n}\leq t\right\}=F^{n\star}\left(t\right)
\end{eqnarray*}
que es la $n$-\'esima convoluci\'on de $F$. Entonces 

\begin{eqnarray*}
\left\{N\left(t\right)\geq n\right\}&=&\left\{T_{n}\leq t\right\}\\
P\left\{N\left(t\right)\leq n\right\}&=&1-F^{\left(n+1\right)\star}\left(t\right)
\end{eqnarray*}

Adem\'as usando el hecho de que $\esp\left[N\left(t\right)\right]=\sum_{n=1}^{\infty}P\left\{N\left(t\right)\geq n\right\}$
se tiene que

\begin{eqnarray*}
\esp\left[N\left(t\right)\right]=\sum_{n=1}^{\infty}F^{n\star}\left(t\right)
\end{eqnarray*}

\begin{Prop}
Para cada $t\geq0$, la funci\'on generadora de momentos $\esp\left[e^{\alpha N\left(t\right)}\right]$ existe para alguna $\alpha$ en una vecindad del 0, y de aqu\'i que $\esp\left[N\left(t\right)^{m}\right]<\infty$, para $m\geq1$.
\end{Prop}


\begin{Note}
Si el primer tiempo de renovaci\'on $\xi_{1}$ no tiene la misma distribuci\'on que el resto de las $\xi_{n}$, para $n\geq2$, a $N\left(t\right)$ se le llama Proceso de Renovaci\'on retardado, donde si $\xi$ tiene distribuci\'on $G$, entonces el tiempo $T_{n}$ de la $n$-\'esima renovaci\'on tiene distribuci\'on $G\star F^{\left(n-1\right)\star}\left(t\right)$
\end{Note}


\begin{Teo}
Para una constante $\mu\leq\infty$ ( o variable aleatoria), las siguientes expresiones son equivalentes:

\begin{eqnarray}
lim_{n\rightarrow\infty}n^{-1}T_{n}&=&\mu,\textrm{ c.s.}\\
lim_{t\rightarrow\infty}t^{-1}N\left(t\right)&=&1/\mu,\textrm{ c.s.}
\end{eqnarray}
\end{Teo}


Es decir, $T_{n}$ satisface la Ley Fuerte de los Grandes N\'umeros s\'i y s\'olo s\'i $N\left/t\right)$ la cumple.


\begin{Coro}[Ley Fuerte de los Grandes N\'umeros para Procesos de Renovaci\'on]
Si $N\left(t\right)$ es un proceso de renovaci\'on cuyos tiempos de inter-renovaci\'on tienen media $\mu\leq\infty$, entonces
\begin{eqnarray}
t^{-1}N\left(t\right)\rightarrow 1/\mu,\textrm{ c.s. cuando }t\rightarrow\infty.
\end{eqnarray}

\end{Coro}


Considerar el proceso estoc\'astico de valores reales $\left\{Z\left(t\right):t\geq0\right\}$ en el mismo espacio de probabilidad que $N\left(t\right)$

\begin{Def}
Para el proceso $\left\{Z\left(t\right):t\geq0\right\}$ se define la fluctuaci\'on m\'axima de $Z\left(t\right)$ en el intervalo $\left(T_{n-1},T_{n}\right]$:
\begin{eqnarray*}
M_{n}=\sup_{T_{n-1}<t\leq T_{n}}|Z\left(t\right)-Z\left(T_{n-1}\right)|
\end{eqnarray*}
\end{Def}

\begin{Teo}
Sup\'ongase que $n^{-1}T_{n}\rightarrow\mu$ c.s. cuando $n\rightarrow\infty$, donde $\mu\leq\infty$ es una constante o variable aleatoria. Sea $a$ una constante o variable aleatoria que puede ser infinita cuando $\mu$ es finita, y considere las expresiones l\'imite:
\begin{eqnarray}
lim_{n\rightarrow\infty}n^{-1}Z\left(T_{n}\right)&=&a,\textrm{ c.s.}\\
lim_{t\rightarrow\infty}t^{-1}Z\left(t\right)&=&a/\mu,\textrm{ c.s.}
\end{eqnarray}
La segunda expresi\'on implica la primera. Conversamente, la primera implica la segunda si el proceso $Z\left(t\right)$ es creciente, o si $lim_{n\rightarrow\infty}n^{-1}M_{n}=0$ c.s.
\end{Teo}

\begin{Coro}
Si $N\left(t\right)$ es un proceso de renovaci\'on, y $\left(Z\left(T_{n}\right)-Z\left(T_{n-1}\right),M_{n}\right)$, para $n\geq1$, son variables aleatorias independientes e id\'enticamente distribuidas con media finita, entonces,
\begin{eqnarray}
lim_{t\rightarrow\infty}t^{-1}Z\left(t\right)\rightarrow\frac{\esp\left[Z\left(T_{1}\right)-Z\left(T_{0}\right)\right]}{\esp\left[T_{1}\right]},\textrm{ c.s. cuando  }t\rightarrow\infty.
\end{eqnarray}
\end{Coro}


%___________________________________________________________________________________________
%
%\subsection*{Propiedades de los Procesos de Renovaci\'on}
%___________________________________________________________________________________________
%

Los tiempos $T_{n}$ est\'an relacionados con los conteos de $N\left(t\right)$ por

\begin{eqnarray*}
\left\{N\left(t\right)\geq n\right\}&=&\left\{T_{n}\leq t\right\}\\
T_{N\left(t\right)}\leq &t&<T_{N\left(t\right)+1},
\end{eqnarray*}

adem\'as $N\left(T_{n}\right)=n$, y 

\begin{eqnarray*}
N\left(t\right)=\max\left\{n:T_{n}\leq t\right\}=\min\left\{n:T_{n+1}>t\right\}
\end{eqnarray*}

Por propiedades de la convoluci\'on se sabe que

\begin{eqnarray*}
P\left\{T_{n}\leq t\right\}=F^{n\star}\left(t\right)
\end{eqnarray*}
que es la $n$-\'esima convoluci\'on de $F$. Entonces 

\begin{eqnarray*}
\left\{N\left(t\right)\geq n\right\}&=&\left\{T_{n}\leq t\right\}\\
P\left\{N\left(t\right)\leq n\right\}&=&1-F^{\left(n+1\right)\star}\left(t\right)
\end{eqnarray*}

Adem\'as usando el hecho de que $\esp\left[N\left(t\right)\right]=\sum_{n=1}^{\infty}P\left\{N\left(t\right)\geq n\right\}$
se tiene que

\begin{eqnarray*}
\esp\left[N\left(t\right)\right]=\sum_{n=1}^{\infty}F^{n\star}\left(t\right)
\end{eqnarray*}

\begin{Prop}
Para cada $t\geq0$, la funci\'on generadora de momentos $\esp\left[e^{\alpha N\left(t\right)}\right]$ existe para alguna $\alpha$ en una vecindad del 0, y de aqu\'i que $\esp\left[N\left(t\right)^{m}\right]<\infty$, para $m\geq1$.
\end{Prop}


\begin{Note}
Si el primer tiempo de renovaci\'on $\xi_{1}$ no tiene la misma distribuci\'on que el resto de las $\xi_{n}$, para $n\geq2$, a $N\left(t\right)$ se le llama Proceso de Renovaci\'on retardado, donde si $\xi$ tiene distribuci\'on $G$, entonces el tiempo $T_{n}$ de la $n$-\'esima renovaci\'on tiene distribuci\'on $G\star F^{\left(n-1\right)\star}\left(t\right)$
\end{Note}


\begin{Teo}
Para una constante $\mu\leq\infty$ ( o variable aleatoria), las siguientes expresiones son equivalentes:

\begin{eqnarray}
lim_{n\rightarrow\infty}n^{-1}T_{n}&=&\mu,\textrm{ c.s.}\\
lim_{t\rightarrow\infty}t^{-1}N\left(t\right)&=&1/\mu,\textrm{ c.s.}
\end{eqnarray}
\end{Teo}


Es decir, $T_{n}$ satisface la Ley Fuerte de los Grandes N\'umeros s\'i y s\'olo s\'i $N\left/t\right)$ la cumple.


\begin{Coro}[Ley Fuerte de los Grandes N\'umeros para Procesos de Renovaci\'on]
Si $N\left(t\right)$ es un proceso de renovaci\'on cuyos tiempos de inter-renovaci\'on tienen media $\mu\leq\infty$, entonces
\begin{eqnarray}
t^{-1}N\left(t\right)\rightarrow 1/\mu,\textrm{ c.s. cuando }t\rightarrow\infty.
\end{eqnarray}

\end{Coro}


Considerar el proceso estoc\'astico de valores reales $\left\{Z\left(t\right):t\geq0\right\}$ en el mismo espacio de probabilidad que $N\left(t\right)$

\begin{Def}
Para el proceso $\left\{Z\left(t\right):t\geq0\right\}$ se define la fluctuaci\'on m\'axima de $Z\left(t\right)$ en el intervalo $\left(T_{n-1},T_{n}\right]$:
\begin{eqnarray*}
M_{n}=\sup_{T_{n-1}<t\leq T_{n}}|Z\left(t\right)-Z\left(T_{n-1}\right)|
\end{eqnarray*}
\end{Def}

\begin{Teo}
Sup\'ongase que $n^{-1}T_{n}\rightarrow\mu$ c.s. cuando $n\rightarrow\infty$, donde $\mu\leq\infty$ es una constante o variable aleatoria. Sea $a$ una constante o variable aleatoria que puede ser infinita cuando $\mu$ es finita, y considere las expresiones l\'imite:
\begin{eqnarray}
lim_{n\rightarrow\infty}n^{-1}Z\left(T_{n}\right)&=&a,\textrm{ c.s.}\\
lim_{t\rightarrow\infty}t^{-1}Z\left(t\right)&=&a/\mu,\textrm{ c.s.}
\end{eqnarray}
La segunda expresi\'on implica la primera. Conversamente, la primera implica la segunda si el proceso $Z\left(t\right)$ es creciente, o si $lim_{n\rightarrow\infty}n^{-1}M_{n}=0$ c.s.
\end{Teo}

\begin{Coro}
Si $N\left(t\right)$ es un proceso de renovaci\'on, y $\left(Z\left(T_{n}\right)-Z\left(T_{n-1}\right),M_{n}\right)$, para $n\geq1$, son variables aleatorias independientes e id\'enticamente distribuidas con media finita, entonces,
\begin{eqnarray}
lim_{t\rightarrow\infty}t^{-1}Z\left(t\right)\rightarrow\frac{\esp\left[Z\left(T_{1}\right)-Z\left(T_{0}\right)\right]}{\esp\left[T_{1}\right]},\textrm{ c.s. cuando  }t\rightarrow\infty.
\end{eqnarray}
\end{Coro}



%___________________________________________________________________________________________
%
%\subsection*{Propiedades de los Procesos de Renovaci\'on}
%___________________________________________________________________________________________
%

Los tiempos $T_{n}$ est\'an relacionados con los conteos de $N\left(t\right)$ por

\begin{eqnarray*}
\left\{N\left(t\right)\geq n\right\}&=&\left\{T_{n}\leq t\right\}\\
T_{N\left(t\right)}\leq &t&<T_{N\left(t\right)+1},
\end{eqnarray*}

adem\'as $N\left(T_{n}\right)=n$, y 

\begin{eqnarray*}
N\left(t\right)=\max\left\{n:T_{n}\leq t\right\}=\min\left\{n:T_{n+1}>t\right\}
\end{eqnarray*}

Por propiedades de la convoluci\'on se sabe que

\begin{eqnarray*}
P\left\{T_{n}\leq t\right\}=F^{n\star}\left(t\right)
\end{eqnarray*}
que es la $n$-\'esima convoluci\'on de $F$. Entonces 

\begin{eqnarray*}
\left\{N\left(t\right)\geq n\right\}&=&\left\{T_{n}\leq t\right\}\\
P\left\{N\left(t\right)\leq n\right\}&=&1-F^{\left(n+1\right)\star}\left(t\right)
\end{eqnarray*}

Adem\'as usando el hecho de que $\esp\left[N\left(t\right)\right]=\sum_{n=1}^{\infty}P\left\{N\left(t\right)\geq n\right\}$
se tiene que

\begin{eqnarray*}
\esp\left[N\left(t\right)\right]=\sum_{n=1}^{\infty}F^{n\star}\left(t\right)
\end{eqnarray*}

\begin{Prop}
Para cada $t\geq0$, la funci\'on generadora de momentos $\esp\left[e^{\alpha N\left(t\right)}\right]$ existe para alguna $\alpha$ en una vecindad del 0, y de aqu\'i que $\esp\left[N\left(t\right)^{m}\right]<\infty$, para $m\geq1$.
\end{Prop}


\begin{Note}
Si el primer tiempo de renovaci\'on $\xi_{1}$ no tiene la misma distribuci\'on que el resto de las $\xi_{n}$, para $n\geq2$, a $N\left(t\right)$ se le llama Proceso de Renovaci\'on retardado, donde si $\xi$ tiene distribuci\'on $G$, entonces el tiempo $T_{n}$ de la $n$-\'esima renovaci\'on tiene distribuci\'on $G\star F^{\left(n-1\right)\star}\left(t\right)$
\end{Note}


\begin{Teo}
Para una constante $\mu\leq\infty$ ( o variable aleatoria), las siguientes expresiones son equivalentes:

\begin{eqnarray}
lim_{n\rightarrow\infty}n^{-1}T_{n}&=&\mu,\textrm{ c.s.}\\
lim_{t\rightarrow\infty}t^{-1}N\left(t\right)&=&1/\mu,\textrm{ c.s.}
\end{eqnarray}
\end{Teo}


Es decir, $T_{n}$ satisface la Ley Fuerte de los Grandes N\'umeros s\'i y s\'olo s\'i $N\left/t\right)$ la cumple.


\begin{Coro}[Ley Fuerte de los Grandes N\'umeros para Procesos de Renovaci\'on]
Si $N\left(t\right)$ es un proceso de renovaci\'on cuyos tiempos de inter-renovaci\'on tienen media $\mu\leq\infty$, entonces
\begin{eqnarray}
t^{-1}N\left(t\right)\rightarrow 1/\mu,\textrm{ c.s. cuando }t\rightarrow\infty.
\end{eqnarray}

\end{Coro}


Considerar el proceso estoc\'astico de valores reales $\left\{Z\left(t\right):t\geq0\right\}$ en el mismo espacio de probabilidad que $N\left(t\right)$

\begin{Def}
Para el proceso $\left\{Z\left(t\right):t\geq0\right\}$ se define la fluctuaci\'on m\'axima de $Z\left(t\right)$ en el intervalo $\left(T_{n-1},T_{n}\right]$:
\begin{eqnarray*}
M_{n}=\sup_{T_{n-1}<t\leq T_{n}}|Z\left(t\right)-Z\left(T_{n-1}\right)|
\end{eqnarray*}
\end{Def}

\begin{Teo}
Sup\'ongase que $n^{-1}T_{n}\rightarrow\mu$ c.s. cuando $n\rightarrow\infty$, donde $\mu\leq\infty$ es una constante o variable aleatoria. Sea $a$ una constante o variable aleatoria que puede ser infinita cuando $\mu$ es finita, y considere las expresiones l\'imite:
\begin{eqnarray}
lim_{n\rightarrow\infty}n^{-1}Z\left(T_{n}\right)&=&a,\textrm{ c.s.}\\
lim_{t\rightarrow\infty}t^{-1}Z\left(t\right)&=&a/\mu,\textrm{ c.s.}
\end{eqnarray}
La segunda expresi\'on implica la primera. Conversamente, la primera implica la segunda si el proceso $Z\left(t\right)$ es creciente, o si $lim_{n\rightarrow\infty}n^{-1}M_{n}=0$ c.s.
\end{Teo}

\begin{Coro}
Si $N\left(t\right)$ es un proceso de renovaci\'on, y $\left(Z\left(T_{n}\right)-Z\left(T_{n-1}\right),M_{n}\right)$, para $n\geq1$, son variables aleatorias independientes e id\'enticamente distribuidas con media finita, entonces,
\begin{eqnarray}
lim_{t\rightarrow\infty}t^{-1}Z\left(t\right)\rightarrow\frac{\esp\left[Z\left(T_{1}\right)-Z\left(T_{0}\right)\right]}{\esp\left[T_{1}\right]},\textrm{ c.s. cuando  }t\rightarrow\infty.
\end{eqnarray}
\end{Coro}


%___________________________________________________________________________________________
%
%\subsection*{Propiedades de los Procesos de Renovaci\'on}
%___________________________________________________________________________________________
%

Los tiempos $T_{n}$ est\'an relacionados con los conteos de $N\left(t\right)$ por

\begin{eqnarray*}
\left\{N\left(t\right)\geq n\right\}&=&\left\{T_{n}\leq t\right\}\\
T_{N\left(t\right)}\leq &t&<T_{N\left(t\right)+1},
\end{eqnarray*}

adem\'as $N\left(T_{n}\right)=n$, y 

\begin{eqnarray*}
N\left(t\right)=\max\left\{n:T_{n}\leq t\right\}=\min\left\{n:T_{n+1}>t\right\}
\end{eqnarray*}

Por propiedades de la convoluci\'on se sabe que

\begin{eqnarray*}
P\left\{T_{n}\leq t\right\}=F^{n\star}\left(t\right)
\end{eqnarray*}
que es la $n$-\'esima convoluci\'on de $F$. Entonces 

\begin{eqnarray*}
\left\{N\left(t\right)\geq n\right\}&=&\left\{T_{n}\leq t\right\}\\
P\left\{N\left(t\right)\leq n\right\}&=&1-F^{\left(n+1\right)\star}\left(t\right)
\end{eqnarray*}

Adem\'as usando el hecho de que $\esp\left[N\left(t\right)\right]=\sum_{n=1}^{\infty}P\left\{N\left(t\right)\geq n\right\}$
se tiene que

\begin{eqnarray*}
\esp\left[N\left(t\right)\right]=\sum_{n=1}^{\infty}F^{n\star}\left(t\right)
\end{eqnarray*}

\begin{Prop}
Para cada $t\geq0$, la funci\'on generadora de momentos $\esp\left[e^{\alpha N\left(t\right)}\right]$ existe para alguna $\alpha$ en una vecindad del 0, y de aqu\'i que $\esp\left[N\left(t\right)^{m}\right]<\infty$, para $m\geq1$.
\end{Prop}


\begin{Note}
Si el primer tiempo de renovaci\'on $\xi_{1}$ no tiene la misma distribuci\'on que el resto de las $\xi_{n}$, para $n\geq2$, a $N\left(t\right)$ se le llama Proceso de Renovaci\'on retardado, donde si $\xi$ tiene distribuci\'on $G$, entonces el tiempo $T_{n}$ de la $n$-\'esima renovaci\'on tiene distribuci\'on $G\star F^{\left(n-1\right)\star}\left(t\right)$
\end{Note}


\begin{Teo}
Para una constante $\mu\leq\infty$ ( o variable aleatoria), las siguientes expresiones son equivalentes:

\begin{eqnarray}
lim_{n\rightarrow\infty}n^{-1}T_{n}&=&\mu,\textrm{ c.s.}\\
lim_{t\rightarrow\infty}t^{-1}N\left(t\right)&=&1/\mu,\textrm{ c.s.}
\end{eqnarray}
\end{Teo}


Es decir, $T_{n}$ satisface la Ley Fuerte de los Grandes N\'umeros s\'i y s\'olo s\'i $N\left/t\right)$ la cumple.


\begin{Coro}[Ley Fuerte de los Grandes N\'umeros para Procesos de Renovaci\'on]
Si $N\left(t\right)$ es un proceso de renovaci\'on cuyos tiempos de inter-renovaci\'on tienen media $\mu\leq\infty$, entonces
\begin{eqnarray}
t^{-1}N\left(t\right)\rightarrow 1/\mu,\textrm{ c.s. cuando }t\rightarrow\infty.
\end{eqnarray}

\end{Coro}


Considerar el proceso estoc\'astico de valores reales $\left\{Z\left(t\right):t\geq0\right\}$ en el mismo espacio de probabilidad que $N\left(t\right)$

\begin{Def}
Para el proceso $\left\{Z\left(t\right):t\geq0\right\}$ se define la fluctuaci\'on m\'axima de $Z\left(t\right)$ en el intervalo $\left(T_{n-1},T_{n}\right]$:
\begin{eqnarray*}
M_{n}=\sup_{T_{n-1}<t\leq T_{n}}|Z\left(t\right)-Z\left(T_{n-1}\right)|
\end{eqnarray*}
\end{Def}

\begin{Teo}
Sup\'ongase que $n^{-1}T_{n}\rightarrow\mu$ c.s. cuando $n\rightarrow\infty$, donde $\mu\leq\infty$ es una constante o variable aleatoria. Sea $a$ una constante o variable aleatoria que puede ser infinita cuando $\mu$ es finita, y considere las expresiones l\'imite:
\begin{eqnarray}
lim_{n\rightarrow\infty}n^{-1}Z\left(T_{n}\right)&=&a,\textrm{ c.s.}\\
lim_{t\rightarrow\infty}t^{-1}Z\left(t\right)&=&a/\mu,\textrm{ c.s.}
\end{eqnarray}
La segunda expresi\'on implica la primera. Conversamente, la primera implica la segunda si el proceso $Z\left(t\right)$ es creciente, o si $lim_{n\rightarrow\infty}n^{-1}M_{n}=0$ c.s.
\end{Teo}

\begin{Coro}
Si $N\left(t\right)$ es un proceso de renovaci\'on, y $\left(Z\left(T_{n}\right)-Z\left(T_{n-1}\right),M_{n}\right)$, para $n\geq1$, son variables aleatorias independientes e id\'enticamente distribuidas con media finita, entonces,
\begin{eqnarray}
lim_{t\rightarrow\infty}t^{-1}Z\left(t\right)\rightarrow\frac{\esp\left[Z\left(T_{1}\right)-Z\left(T_{0}\right)\right]}{\esp\left[T_{1}\right]},\textrm{ c.s. cuando  }t\rightarrow\infty.
\end{eqnarray}
\end{Coro}

%___________________________________________________________________________________________
%
%\subsection*{Propiedades de los Procesos de Renovaci\'on}
%___________________________________________________________________________________________
%

Los tiempos $T_{n}$ est\'an relacionados con los conteos de $N\left(t\right)$ por

\begin{eqnarray*}
\left\{N\left(t\right)\geq n\right\}&=&\left\{T_{n}\leq t\right\}\\
T_{N\left(t\right)}\leq &t&<T_{N\left(t\right)+1},
\end{eqnarray*}

adem\'as $N\left(T_{n}\right)=n$, y 

\begin{eqnarray*}
N\left(t\right)=\max\left\{n:T_{n}\leq t\right\}=\min\left\{n:T_{n+1}>t\right\}
\end{eqnarray*}

Por propiedades de la convoluci\'on se sabe que

\begin{eqnarray*}
P\left\{T_{n}\leq t\right\}=F^{n\star}\left(t\right)
\end{eqnarray*}
que es la $n$-\'esima convoluci\'on de $F$. Entonces 

\begin{eqnarray*}
\left\{N\left(t\right)\geq n\right\}&=&\left\{T_{n}\leq t\right\}\\
P\left\{N\left(t\right)\leq n\right\}&=&1-F^{\left(n+1\right)\star}\left(t\right)
\end{eqnarray*}

Adem\'as usando el hecho de que $\esp\left[N\left(t\right)\right]=\sum_{n=1}^{\infty}P\left\{N\left(t\right)\geq n\right\}$
se tiene que

\begin{eqnarray*}
\esp\left[N\left(t\right)\right]=\sum_{n=1}^{\infty}F^{n\star}\left(t\right)
\end{eqnarray*}

\begin{Prop}
Para cada $t\geq0$, la funci\'on generadora de momentos $\esp\left[e^{\alpha N\left(t\right)}\right]$ existe para alguna $\alpha$ en una vecindad del 0, y de aqu\'i que $\esp\left[N\left(t\right)^{m}\right]<\infty$, para $m\geq1$.
\end{Prop}


\begin{Note}
Si el primer tiempo de renovaci\'on $\xi_{1}$ no tiene la misma distribuci\'on que el resto de las $\xi_{n}$, para $n\geq2$, a $N\left(t\right)$ se le llama Proceso de Renovaci\'on retardado, donde si $\xi$ tiene distribuci\'on $G$, entonces el tiempo $T_{n}$ de la $n$-\'esima renovaci\'on tiene distribuci\'on $G\star F^{\left(n-1\right)\star}\left(t\right)$
\end{Note}


\begin{Teo}
Para una constante $\mu\leq\infty$ ( o variable aleatoria), las siguientes expresiones son equivalentes:

\begin{eqnarray}
lim_{n\rightarrow\infty}n^{-1}T_{n}&=&\mu,\textrm{ c.s.}\\
lim_{t\rightarrow\infty}t^{-1}N\left(t\right)&=&1/\mu,\textrm{ c.s.}
\end{eqnarray}
\end{Teo}


Es decir, $T_{n}$ satisface la Ley Fuerte de los Grandes N\'umeros s\'i y s\'olo s\'i $N\left/t\right)$ la cumple.


\begin{Coro}[Ley Fuerte de los Grandes N\'umeros para Procesos de Renovaci\'on]
Si $N\left(t\right)$ es un proceso de renovaci\'on cuyos tiempos de inter-renovaci\'on tienen media $\mu\leq\infty$, entonces
\begin{eqnarray}
t^{-1}N\left(t\right)\rightarrow 1/\mu,\textrm{ c.s. cuando }t\rightarrow\infty.
\end{eqnarray}

\end{Coro}


Considerar el proceso estoc\'astico de valores reales $\left\{Z\left(t\right):t\geq0\right\}$ en el mismo espacio de probabilidad que $N\left(t\right)$

\begin{Def}
Para el proceso $\left\{Z\left(t\right):t\geq0\right\}$ se define la fluctuaci\'on m\'axima de $Z\left(t\right)$ en el intervalo $\left(T_{n-1},T_{n}\right]$:
\begin{eqnarray*}
M_{n}=\sup_{T_{n-1}<t\leq T_{n}}|Z\left(t\right)-Z\left(T_{n-1}\right)|
\end{eqnarray*}
\end{Def}

\begin{Teo}
Sup\'ongase que $n^{-1}T_{n}\rightarrow\mu$ c.s. cuando $n\rightarrow\infty$, donde $\mu\leq\infty$ es una constante o variable aleatoria. Sea $a$ una constante o variable aleatoria que puede ser infinita cuando $\mu$ es finita, y considere las expresiones l\'imite:
\begin{eqnarray}
lim_{n\rightarrow\infty}n^{-1}Z\left(T_{n}\right)&=&a,\textrm{ c.s.}\\
lim_{t\rightarrow\infty}t^{-1}Z\left(t\right)&=&a/\mu,\textrm{ c.s.}
\end{eqnarray}
La segunda expresi\'on implica la primera. Conversamente, la primera implica la segunda si el proceso $Z\left(t\right)$ es creciente, o si $lim_{n\rightarrow\infty}n^{-1}M_{n}=0$ c.s.
\end{Teo}

\begin{Coro}
Si $N\left(t\right)$ es un proceso de renovaci\'on, y $\left(Z\left(T_{n}\right)-Z\left(T_{n-1}\right),M_{n}\right)$, para $n\geq1$, son variables aleatorias independientes e id\'enticamente distribuidas con media finita, entonces,
\begin{eqnarray}
lim_{t\rightarrow\infty}t^{-1}Z\left(t\right)\rightarrow\frac{\esp\left[Z\left(T_{1}\right)-Z\left(T_{0}\right)\right]}{\esp\left[T_{1}\right]},\textrm{ c.s. cuando  }t\rightarrow\infty.
\end{eqnarray}
\end{Coro}
%___________________________________________________________________________________________
%
%\subsection*{Propiedades de los Procesos de Renovaci\'on}
%___________________________________________________________________________________________
%

Los tiempos $T_{n}$ est\'an relacionados con los conteos de $N\left(t\right)$ por

\begin{eqnarray*}
\left\{N\left(t\right)\geq n\right\}&=&\left\{T_{n}\leq t\right\}\\
T_{N\left(t\right)}\leq &t&<T_{N\left(t\right)+1},
\end{eqnarray*}

adem\'as $N\left(T_{n}\right)=n$, y 

\begin{eqnarray*}
N\left(t\right)=\max\left\{n:T_{n}\leq t\right\}=\min\left\{n:T_{n+1}>t\right\}
\end{eqnarray*}

Por propiedades de la convoluci\'on se sabe que

\begin{eqnarray*}
P\left\{T_{n}\leq t\right\}=F^{n\star}\left(t\right)
\end{eqnarray*}
que es la $n$-\'esima convoluci\'on de $F$. Entonces 

\begin{eqnarray*}
\left\{N\left(t\right)\geq n\right\}&=&\left\{T_{n}\leq t\right\}\\
P\left\{N\left(t\right)\leq n\right\}&=&1-F^{\left(n+1\right)\star}\left(t\right)
\end{eqnarray*}

Adem\'as usando el hecho de que $\esp\left[N\left(t\right)\right]=\sum_{n=1}^{\infty}P\left\{N\left(t\right)\geq n\right\}$
se tiene que

\begin{eqnarray*}
\esp\left[N\left(t\right)\right]=\sum_{n=1}^{\infty}F^{n\star}\left(t\right)
\end{eqnarray*}

\begin{Prop}
Para cada $t\geq0$, la funci\'on generadora de momentos $\esp\left[e^{\alpha N\left(t\right)}\right]$ existe para alguna $\alpha$ en una vecindad del 0, y de aqu\'i que $\esp\left[N\left(t\right)^{m}\right]<\infty$, para $m\geq1$.
\end{Prop}


\begin{Note}
Si el primer tiempo de renovaci\'on $\xi_{1}$ no tiene la misma distribuci\'on que el resto de las $\xi_{n}$, para $n\geq2$, a $N\left(t\right)$ se le llama Proceso de Renovaci\'on retardado, donde si $\xi$ tiene distribuci\'on $G$, entonces el tiempo $T_{n}$ de la $n$-\'esima renovaci\'on tiene distribuci\'on $G\star F^{\left(n-1\right)\star}\left(t\right)$
\end{Note}


\begin{Teo}
Para una constante $\mu\leq\infty$ ( o variable aleatoria), las siguientes expresiones son equivalentes:

\begin{eqnarray}
lim_{n\rightarrow\infty}n^{-1}T_{n}&=&\mu,\textrm{ c.s.}\\
lim_{t\rightarrow\infty}t^{-1}N\left(t\right)&=&1/\mu,\textrm{ c.s.}
\end{eqnarray}
\end{Teo}


Es decir, $T_{n}$ satisface la Ley Fuerte de los Grandes N\'umeros s\'i y s\'olo s\'i $N\left/t\right)$ la cumple.


\begin{Coro}[Ley Fuerte de los Grandes N\'umeros para Procesos de Renovaci\'on]
Si $N\left(t\right)$ es un proceso de renovaci\'on cuyos tiempos de inter-renovaci\'on tienen media $\mu\leq\infty$, entonces
\begin{eqnarray}
t^{-1}N\left(t\right)\rightarrow 1/\mu,\textrm{ c.s. cuando }t\rightarrow\infty.
\end{eqnarray}

\end{Coro}


Considerar el proceso estoc\'astico de valores reales $\left\{Z\left(t\right):t\geq0\right\}$ en el mismo espacio de probabilidad que $N\left(t\right)$

\begin{Def}
Para el proceso $\left\{Z\left(t\right):t\geq0\right\}$ se define la fluctuaci\'on m\'axima de $Z\left(t\right)$ en el intervalo $\left(T_{n-1},T_{n}\right]$:
\begin{eqnarray*}
M_{n}=\sup_{T_{n-1}<t\leq T_{n}}|Z\left(t\right)-Z\left(T_{n-1}\right)|
\end{eqnarray*}
\end{Def}

\begin{Teo}
Sup\'ongase que $n^{-1}T_{n}\rightarrow\mu$ c.s. cuando $n\rightarrow\infty$, donde $\mu\leq\infty$ es una constante o variable aleatoria. Sea $a$ una constante o variable aleatoria que puede ser infinita cuando $\mu$ es finita, y considere las expresiones l\'imite:
\begin{eqnarray}
lim_{n\rightarrow\infty}n^{-1}Z\left(T_{n}\right)&=&a,\textrm{ c.s.}\\
lim_{t\rightarrow\infty}t^{-1}Z\left(t\right)&=&a/\mu,\textrm{ c.s.}
\end{eqnarray}
La segunda expresi\'on implica la primera. Conversamente, la primera implica la segunda si el proceso $Z\left(t\right)$ es creciente, o si $lim_{n\rightarrow\infty}n^{-1}M_{n}=0$ c.s.
\end{Teo}

\begin{Coro}
Si $N\left(t\right)$ es un proceso de renovaci\'on, y $\left(Z\left(T_{n}\right)-Z\left(T_{n-1}\right),M_{n}\right)$, para $n\geq1$, son variables aleatorias independientes e id\'enticamente distribuidas con media finita, entonces,
\begin{eqnarray}
lim_{t\rightarrow\infty}t^{-1}Z\left(t\right)\rightarrow\frac{\esp\left[Z\left(T_{1}\right)-Z\left(T_{0}\right)\right]}{\esp\left[T_{1}\right]},\textrm{ c.s. cuando  }t\rightarrow\infty.
\end{eqnarray}
\end{Coro}


%___________________________________________________________________________________________
%
%\subsection*{Propiedades de los Procesos de Renovaci\'on}
%___________________________________________________________________________________________
%

Los tiempos $T_{n}$ est\'an relacionados con los conteos de $N\left(t\right)$ por

\begin{eqnarray*}
\left\{N\left(t\right)\geq n\right\}&=&\left\{T_{n}\leq t\right\}\\
T_{N\left(t\right)}\leq &t&<T_{N\left(t\right)+1},
\end{eqnarray*}

adem\'as $N\left(T_{n}\right)=n$, y 

\begin{eqnarray*}
N\left(t\right)=\max\left\{n:T_{n}\leq t\right\}=\min\left\{n:T_{n+1}>t\right\}
\end{eqnarray*}

Por propiedades de la convoluci\'on se sabe que

\begin{eqnarray*}
P\left\{T_{n}\leq t\right\}=F^{n\star}\left(t\right)
\end{eqnarray*}
que es la $n$-\'esima convoluci\'on de $F$. Entonces 

\begin{eqnarray*}
\left\{N\left(t\right)\geq n\right\}&=&\left\{T_{n}\leq t\right\}\\
P\left\{N\left(t\right)\leq n\right\}&=&1-F^{\left(n+1\right)\star}\left(t\right)
\end{eqnarray*}

Adem\'as usando el hecho de que $\esp\left[N\left(t\right)\right]=\sum_{n=1}^{\infty}P\left\{N\left(t\right)\geq n\right\}$
se tiene que

\begin{eqnarray*}
\esp\left[N\left(t\right)\right]=\sum_{n=1}^{\infty}F^{n\star}\left(t\right)
\end{eqnarray*}

\begin{Prop}
Para cada $t\geq0$, la funci\'on generadora de momentos $\esp\left[e^{\alpha N\left(t\right)}\right]$ existe para alguna $\alpha$ en una vecindad del 0, y de aqu\'i que $\esp\left[N\left(t\right)^{m}\right]<\infty$, para $m\geq1$.
\end{Prop}


\begin{Note}
Si el primer tiempo de renovaci\'on $\xi_{1}$ no tiene la misma distribuci\'on que el resto de las $\xi_{n}$, para $n\geq2$, a $N\left(t\right)$ se le llama Proceso de Renovaci\'on retardado, donde si $\xi$ tiene distribuci\'on $G$, entonces el tiempo $T_{n}$ de la $n$-\'esima renovaci\'on tiene distribuci\'on $G\star F^{\left(n-1\right)\star}\left(t\right)$
\end{Note}


\begin{Teo}
Para una constante $\mu\leq\infty$ ( o variable aleatoria), las siguientes expresiones son equivalentes:

\begin{eqnarray}
lim_{n\rightarrow\infty}n^{-1}T_{n}&=&\mu,\textrm{ c.s.}\\
lim_{t\rightarrow\infty}t^{-1}N\left(t\right)&=&1/\mu,\textrm{ c.s.}
\end{eqnarray}
\end{Teo}


Es decir, $T_{n}$ satisface la Ley Fuerte de los Grandes N\'umeros s\'i y s\'olo s\'i $N\left/t\right)$ la cumple.


\begin{Coro}[Ley Fuerte de los Grandes N\'umeros para Procesos de Renovaci\'on]
Si $N\left(t\right)$ es un proceso de renovaci\'on cuyos tiempos de inter-renovaci\'on tienen media $\mu\leq\infty$, entonces
\begin{eqnarray}
t^{-1}N\left(t\right)\rightarrow 1/\mu,\textrm{ c.s. cuando }t\rightarrow\infty.
\end{eqnarray}

\end{Coro}


Considerar el proceso estoc\'astico de valores reales $\left\{Z\left(t\right):t\geq0\right\}$ en el mismo espacio de probabilidad que $N\left(t\right)$

\begin{Def}
Para el proceso $\left\{Z\left(t\right):t\geq0\right\}$ se define la fluctuaci\'on m\'axima de $Z\left(t\right)$ en el intervalo $\left(T_{n-1},T_{n}\right]$:
\begin{eqnarray*}
M_{n}=\sup_{T_{n-1}<t\leq T_{n}}|Z\left(t\right)-Z\left(T_{n-1}\right)|
\end{eqnarray*}
\end{Def}

\begin{Teo}
Sup\'ongase que $n^{-1}T_{n}\rightarrow\mu$ c.s. cuando $n\rightarrow\infty$, donde $\mu\leq\infty$ es una constante o variable aleatoria. Sea $a$ una constante o variable aleatoria que puede ser infinita cuando $\mu$ es finita, y considere las expresiones l\'imite:
\begin{eqnarray}
lim_{n\rightarrow\infty}n^{-1}Z\left(T_{n}\right)&=&a,\textrm{ c.s.}\\
lim_{t\rightarrow\infty}t^{-1}Z\left(t\right)&=&a/\mu,\textrm{ c.s.}
\end{eqnarray}
La segunda expresi\'on implica la primera. Conversamente, la primera implica la segunda si el proceso $Z\left(t\right)$ es creciente, o si $lim_{n\rightarrow\infty}n^{-1}M_{n}=0$ c.s.
\end{Teo}

\begin{Coro}
Si $N\left(t\right)$ es un proceso de renovaci\'on, y $\left(Z\left(T_{n}\right)-Z\left(T_{n-1}\right),M_{n}\right)$, para $n\geq1$, son variables aleatorias independientes e id\'enticamente distribuidas con media finita, entonces,
\begin{eqnarray}
lim_{t\rightarrow\infty}t^{-1}Z\left(t\right)\rightarrow\frac{\esp\left[Z\left(T_{1}\right)-Z\left(T_{0}\right)\right]}{\esp\left[T_{1}\right]},\textrm{ c.s. cuando  }t\rightarrow\infty.
\end{eqnarray}
\end{Coro}



%___________________________________________________________________________________________
%
%\subsection*{Propiedades de los Procesos de Renovaci\'on}
%___________________________________________________________________________________________
%

Los tiempos $T_{n}$ est\'an relacionados con los conteos de $N\left(t\right)$ por

\begin{eqnarray*}
\left\{N\left(t\right)\geq n\right\}&=&\left\{T_{n}\leq t\right\}\\
T_{N\left(t\right)}\leq &t&<T_{N\left(t\right)+1},
\end{eqnarray*}

adem\'as $N\left(T_{n}\right)=n$, y 

\begin{eqnarray*}
N\left(t\right)=\max\left\{n:T_{n}\leq t\right\}=\min\left\{n:T_{n+1}>t\right\}
\end{eqnarray*}

Por propiedades de la convoluci\'on se sabe que

\begin{eqnarray*}
P\left\{T_{n}\leq t\right\}=F^{n\star}\left(t\right)
\end{eqnarray*}
que es la $n$-\'esima convoluci\'on de $F$. Entonces 

\begin{eqnarray*}
\left\{N\left(t\right)\geq n\right\}&=&\left\{T_{n}\leq t\right\}\\
P\left\{N\left(t\right)\leq n\right\}&=&1-F^{\left(n+1\right)\star}\left(t\right)
\end{eqnarray*}

Adem\'as usando el hecho de que $\esp\left[N\left(t\right)\right]=\sum_{n=1}^{\infty}P\left\{N\left(t\right)\geq n\right\}$
se tiene que

\begin{eqnarray*}
\esp\left[N\left(t\right)\right]=\sum_{n=1}^{\infty}F^{n\star}\left(t\right)
\end{eqnarray*}

\begin{Prop}
Para cada $t\geq0$, la funci\'on generadora de momentos $\esp\left[e^{\alpha N\left(t\right)}\right]$ existe para alguna $\alpha$ en una vecindad del 0, y de aqu\'i que $\esp\left[N\left(t\right)^{m}\right]<\infty$, para $m\geq1$.
\end{Prop}


\begin{Note}
Si el primer tiempo de renovaci\'on $\xi_{1}$ no tiene la misma distribuci\'on que el resto de las $\xi_{n}$, para $n\geq2$, a $N\left(t\right)$ se le llama Proceso de Renovaci\'on retardado, donde si $\xi$ tiene distribuci\'on $G$, entonces el tiempo $T_{n}$ de la $n$-\'esima renovaci\'on tiene distribuci\'on $G\star F^{\left(n-1\right)\star}\left(t\right)$
\end{Note}


\begin{Teo}
Para una constante $\mu\leq\infty$ ( o variable aleatoria), las siguientes expresiones son equivalentes:

\begin{eqnarray}
lim_{n\rightarrow\infty}n^{-1}T_{n}&=&\mu,\textrm{ c.s.}\\
lim_{t\rightarrow\infty}t^{-1}N\left(t\right)&=&1/\mu,\textrm{ c.s.}
\end{eqnarray}
\end{Teo}


Es decir, $T_{n}$ satisface la Ley Fuerte de los Grandes N\'umeros s\'i y s\'olo s\'i $N\left/t\right)$ la cumple.


\begin{Coro}[Ley Fuerte de los Grandes N\'umeros para Procesos de Renovaci\'on]
Si $N\left(t\right)$ es un proceso de renovaci\'on cuyos tiempos de inter-renovaci\'on tienen media $\mu\leq\infty$, entonces
\begin{eqnarray}
t^{-1}N\left(t\right)\rightarrow 1/\mu,\textrm{ c.s. cuando }t\rightarrow\infty.
\end{eqnarray}

\end{Coro}


Considerar el proceso estoc\'astico de valores reales $\left\{Z\left(t\right):t\geq0\right\}$ en el mismo espacio de probabilidad que $N\left(t\right)$

\begin{Def}
Para el proceso $\left\{Z\left(t\right):t\geq0\right\}$ se define la fluctuaci\'on m\'axima de $Z\left(t\right)$ en el intervalo $\left(T_{n-1},T_{n}\right]$:
\begin{eqnarray*}
M_{n}=\sup_{T_{n-1}<t\leq T_{n}}|Z\left(t\right)-Z\left(T_{n-1}\right)|
\end{eqnarray*}
\end{Def}

\begin{Teo}
Sup\'ongase que $n^{-1}T_{n}\rightarrow\mu$ c.s. cuando $n\rightarrow\infty$, donde $\mu\leq\infty$ es una constante o variable aleatoria. Sea $a$ una constante o variable aleatoria que puede ser infinita cuando $\mu$ es finita, y considere las expresiones l\'imite:
\begin{eqnarray}
lim_{n\rightarrow\infty}n^{-1}Z\left(T_{n}\right)&=&a,\textrm{ c.s.}\\
lim_{t\rightarrow\infty}t^{-1}Z\left(t\right)&=&a/\mu,\textrm{ c.s.}
\end{eqnarray}
La segunda expresi\'on implica la primera. Conversamente, la primera implica la segunda si el proceso $Z\left(t\right)$ es creciente, o si $lim_{n\rightarrow\infty}n^{-1}M_{n}=0$ c.s.
\end{Teo}

\begin{Coro}
Si $N\left(t\right)$ es un proceso de renovaci\'on, y $\left(Z\left(T_{n}\right)-Z\left(T_{n-1}\right),M_{n}\right)$, para $n\geq1$, son variables aleatorias independientes e id\'enticamente distribuidas con media finita, entonces,
\begin{eqnarray}
lim_{t\rightarrow\infty}t^{-1}Z\left(t\right)\rightarrow\frac{\esp\left[Z\left(T_{1}\right)-Z\left(T_{0}\right)\right]}{\esp\left[T_{1}\right]},\textrm{ c.s. cuando  }t\rightarrow\infty.
\end{eqnarray}
\end{Coro}


%___________________________________________________________________________________________
%
%\subsection*{Propiedades de los Procesos de Renovaci\'on}
%___________________________________________________________________________________________
%

Los tiempos $T_{n}$ est\'an relacionados con los conteos de $N\left(t\right)$ por

\begin{eqnarray*}
\left\{N\left(t\right)\geq n\right\}&=&\left\{T_{n}\leq t\right\}\\
T_{N\left(t\right)}\leq &t&<T_{N\left(t\right)+1},
\end{eqnarray*}

adem\'as $N\left(T_{n}\right)=n$, y 

\begin{eqnarray*}
N\left(t\right)=\max\left\{n:T_{n}\leq t\right\}=\min\left\{n:T_{n+1}>t\right\}
\end{eqnarray*}

Por propiedades de la convoluci\'on se sabe que

\begin{eqnarray*}
P\left\{T_{n}\leq t\right\}=F^{n\star}\left(t\right)
\end{eqnarray*}
que es la $n$-\'esima convoluci\'on de $F$. Entonces 

\begin{eqnarray*}
\left\{N\left(t\right)\geq n\right\}&=&\left\{T_{n}\leq t\right\}\\
P\left\{N\left(t\right)\leq n\right\}&=&1-F^{\left(n+1\right)\star}\left(t\right)
\end{eqnarray*}

Adem\'as usando el hecho de que $\esp\left[N\left(t\right)\right]=\sum_{n=1}^{\infty}P\left\{N\left(t\right)\geq n\right\}$
se tiene que

\begin{eqnarray*}
\esp\left[N\left(t\right)\right]=\sum_{n=1}^{\infty}F^{n\star}\left(t\right)
\end{eqnarray*}

\begin{Prop}
Para cada $t\geq0$, la funci\'on generadora de momentos $\esp\left[e^{\alpha N\left(t\right)}\right]$ existe para alguna $\alpha$ en una vecindad del 0, y de aqu\'i que $\esp\left[N\left(t\right)^{m}\right]<\infty$, para $m\geq1$.
\end{Prop}


\begin{Note}
Si el primer tiempo de renovaci\'on $\xi_{1}$ no tiene la misma distribuci\'on que el resto de las $\xi_{n}$, para $n\geq2$, a $N\left(t\right)$ se le llama Proceso de Renovaci\'on retardado, donde si $\xi$ tiene distribuci\'on $G$, entonces el tiempo $T_{n}$ de la $n$-\'esima renovaci\'on tiene distribuci\'on $G\star F^{\left(n-1\right)\star}\left(t\right)$
\end{Note}


\begin{Teo}
Para una constante $\mu\leq\infty$ ( o variable aleatoria), las siguientes expresiones son equivalentes:

\begin{eqnarray}
lim_{n\rightarrow\infty}n^{-1}T_{n}&=&\mu,\textrm{ c.s.}\\
lim_{t\rightarrow\infty}t^{-1}N\left(t\right)&=&1/\mu,\textrm{ c.s.}
\end{eqnarray}
\end{Teo}


Es decir, $T_{n}$ satisface la Ley Fuerte de los Grandes N\'umeros s\'i y s\'olo s\'i $N\left/t\right)$ la cumple.


\begin{Coro}[Ley Fuerte de los Grandes N\'umeros para Procesos de Renovaci\'on]
Si $N\left(t\right)$ es un proceso de renovaci\'on cuyos tiempos de inter-renovaci\'on tienen media $\mu\leq\infty$, entonces
\begin{eqnarray}
t^{-1}N\left(t\right)\rightarrow 1/\mu,\textrm{ c.s. cuando }t\rightarrow\infty.
\end{eqnarray}

\end{Coro}


Considerar el proceso estoc\'astico de valores reales $\left\{Z\left(t\right):t\geq0\right\}$ en el mismo espacio de probabilidad que $N\left(t\right)$

\begin{Def}
Para el proceso $\left\{Z\left(t\right):t\geq0\right\}$ se define la fluctuaci\'on m\'axima de $Z\left(t\right)$ en el intervalo $\left(T_{n-1},T_{n}\right]$:
\begin{eqnarray*}
M_{n}=\sup_{T_{n-1}<t\leq T_{n}}|Z\left(t\right)-Z\left(T_{n-1}\right)|
\end{eqnarray*}
\end{Def}

\begin{Teo}
Sup\'ongase que $n^{-1}T_{n}\rightarrow\mu$ c.s. cuando $n\rightarrow\infty$, donde $\mu\leq\infty$ es una constante o variable aleatoria. Sea $a$ una constante o variable aleatoria que puede ser infinita cuando $\mu$ es finita, y considere las expresiones l\'imite:
\begin{eqnarray}
lim_{n\rightarrow\infty}n^{-1}Z\left(T_{n}\right)&=&a,\textrm{ c.s.}\\
lim_{t\rightarrow\infty}t^{-1}Z\left(t\right)&=&a/\mu,\textrm{ c.s.}
\end{eqnarray}
La segunda expresi\'on implica la primera. Conversamente, la primera implica la segunda si el proceso $Z\left(t\right)$ es creciente, o si $lim_{n\rightarrow\infty}n^{-1}M_{n}=0$ c.s.
\end{Teo}

\begin{Coro}
Si $N\left(t\right)$ es un proceso de renovaci\'on, y $\left(Z\left(T_{n}\right)-Z\left(T_{n-1}\right),M_{n}\right)$, para $n\geq1$, son variables aleatorias independientes e id\'enticamente distribuidas con media finita, entonces,
\begin{eqnarray}
lim_{t\rightarrow\infty}t^{-1}Z\left(t\right)\rightarrow\frac{\esp\left[Z\left(T_{1}\right)-Z\left(T_{0}\right)\right]}{\esp\left[T_{1}\right]},\textrm{ c.s. cuando  }t\rightarrow\infty.
\end{eqnarray}
\end{Coro}

%___________________________________________________________________________________________
%
%\subsection*{Propiedades de los Procesos de Renovaci\'on}
%___________________________________________________________________________________________
%

Los tiempos $T_{n}$ est\'an relacionados con los conteos de $N\left(t\right)$ por

\begin{eqnarray*}
\left\{N\left(t\right)\geq n\right\}&=&\left\{T_{n}\leq t\right\}\\
T_{N\left(t\right)}\leq &t&<T_{N\left(t\right)+1},
\end{eqnarray*}

adem\'as $N\left(T_{n}\right)=n$, y 

\begin{eqnarray*}
N\left(t\right)=\max\left\{n:T_{n}\leq t\right\}=\min\left\{n:T_{n+1}>t\right\}
\end{eqnarray*}

Por propiedades de la convoluci\'on se sabe que

\begin{eqnarray*}
P\left\{T_{n}\leq t\right\}=F^{n\star}\left(t\right)
\end{eqnarray*}
que es la $n$-\'esima convoluci\'on de $F$. Entonces 

\begin{eqnarray*}
\left\{N\left(t\right)\geq n\right\}&=&\left\{T_{n}\leq t\right\}\\
P\left\{N\left(t\right)\leq n\right\}&=&1-F^{\left(n+1\right)\star}\left(t\right)
\end{eqnarray*}

Adem\'as usando el hecho de que $\esp\left[N\left(t\right)\right]=\sum_{n=1}^{\infty}P\left\{N\left(t\right)\geq n\right\}$
se tiene que

\begin{eqnarray*}
\esp\left[N\left(t\right)\right]=\sum_{n=1}^{\infty}F^{n\star}\left(t\right)
\end{eqnarray*}

\begin{Prop}
Para cada $t\geq0$, la funci\'on generadora de momentos $\esp\left[e^{\alpha N\left(t\right)}\right]$ existe para alguna $\alpha$ en una vecindad del 0, y de aqu\'i que $\esp\left[N\left(t\right)^{m}\right]<\infty$, para $m\geq1$.
\end{Prop}


\begin{Note}
Si el primer tiempo de renovaci\'on $\xi_{1}$ no tiene la misma distribuci\'on que el resto de las $\xi_{n}$, para $n\geq2$, a $N\left(t\right)$ se le llama Proceso de Renovaci\'on retardado, donde si $\xi$ tiene distribuci\'on $G$, entonces el tiempo $T_{n}$ de la $n$-\'esima renovaci\'on tiene distribuci\'on $G\star F^{\left(n-1\right)\star}\left(t\right)$
\end{Note}


\begin{Teo}
Para una constante $\mu\leq\infty$ ( o variable aleatoria), las siguientes expresiones son equivalentes:

\begin{eqnarray}
lim_{n\rightarrow\infty}n^{-1}T_{n}&=&\mu,\textrm{ c.s.}\\
lim_{t\rightarrow\infty}t^{-1}N\left(t\right)&=&1/\mu,\textrm{ c.s.}
\end{eqnarray}
\end{Teo}


Es decir, $T_{n}$ satisface la Ley Fuerte de los Grandes N\'umeros s\'i y s\'olo s\'i $N\left/t\right)$ la cumple.


\begin{Coro}[Ley Fuerte de los Grandes N\'umeros para Procesos de Renovaci\'on]
Si $N\left(t\right)$ es un proceso de renovaci\'on cuyos tiempos de inter-renovaci\'on tienen media $\mu\leq\infty$, entonces
\begin{eqnarray}
t^{-1}N\left(t\right)\rightarrow 1/\mu,\textrm{ c.s. cuando }t\rightarrow\infty.
\end{eqnarray}

\end{Coro}


Considerar el proceso estoc\'astico de valores reales $\left\{Z\left(t\right):t\geq0\right\}$ en el mismo espacio de probabilidad que $N\left(t\right)$

\begin{Def}
Para el proceso $\left\{Z\left(t\right):t\geq0\right\}$ se define la fluctuaci\'on m\'axima de $Z\left(t\right)$ en el intervalo $\left(T_{n-1},T_{n}\right]$:
\begin{eqnarray*}
M_{n}=\sup_{T_{n-1}<t\leq T_{n}}|Z\left(t\right)-Z\left(T_{n-1}\right)|
\end{eqnarray*}
\end{Def}

\begin{Teo}
Sup\'ongase que $n^{-1}T_{n}\rightarrow\mu$ c.s. cuando $n\rightarrow\infty$, donde $\mu\leq\infty$ es una constante o variable aleatoria. Sea $a$ una constante o variable aleatoria que puede ser infinita cuando $\mu$ es finita, y considere las expresiones l\'imite:
\begin{eqnarray}
lim_{n\rightarrow\infty}n^{-1}Z\left(T_{n}\right)&=&a,\textrm{ c.s.}\\
lim_{t\rightarrow\infty}t^{-1}Z\left(t\right)&=&a/\mu,\textrm{ c.s.}
\end{eqnarray}
La segunda expresi\'on implica la primera. Conversamente, la primera implica la segunda si el proceso $Z\left(t\right)$ es creciente, o si $lim_{n\rightarrow\infty}n^{-1}M_{n}=0$ c.s.
\end{Teo}

\begin{Coro}
Si $N\left(t\right)$ es un proceso de renovaci\'on, y $\left(Z\left(T_{n}\right)-Z\left(T_{n-1}\right),M_{n}\right)$, para $n\geq1$, son variables aleatorias independientes e id\'enticamente distribuidas con media finita, entonces,
\begin{eqnarray}
lim_{t\rightarrow\infty}t^{-1}Z\left(t\right)\rightarrow\frac{\esp\left[Z\left(T_{1}\right)-Z\left(T_{0}\right)\right]}{\esp\left[T_{1}\right]},\textrm{ c.s. cuando  }t\rightarrow\infty.
\end{eqnarray}
\end{Coro}
%___________________________________________________________________________________________
%
%\subsection*{Propiedades de los Procesos de Renovaci\'on}
%___________________________________________________________________________________________
%

Los tiempos $T_{n}$ est\'an relacionados con los conteos de $N\left(t\right)$ por

\begin{eqnarray*}
\left\{N\left(t\right)\geq n\right\}&=&\left\{T_{n}\leq t\right\}\\
T_{N\left(t\right)}\leq &t&<T_{N\left(t\right)+1},
\end{eqnarray*}

adem\'as $N\left(T_{n}\right)=n$, y 

\begin{eqnarray*}
N\left(t\right)=\max\left\{n:T_{n}\leq t\right\}=\min\left\{n:T_{n+1}>t\right\}
\end{eqnarray*}

Por propiedades de la convoluci\'on se sabe que

\begin{eqnarray*}
P\left\{T_{n}\leq t\right\}=F^{n\star}\left(t\right)
\end{eqnarray*}
que es la $n$-\'esima convoluci\'on de $F$. Entonces 

\begin{eqnarray*}
\left\{N\left(t\right)\geq n\right\}&=&\left\{T_{n}\leq t\right\}\\
P\left\{N\left(t\right)\leq n\right\}&=&1-F^{\left(n+1\right)\star}\left(t\right)
\end{eqnarray*}

Adem\'as usando el hecho de que $\esp\left[N\left(t\right)\right]=\sum_{n=1}^{\infty}P\left\{N\left(t\right)\geq n\right\}$
se tiene que

\begin{eqnarray*}
\esp\left[N\left(t\right)\right]=\sum_{n=1}^{\infty}F^{n\star}\left(t\right)
\end{eqnarray*}

\begin{Prop}
Para cada $t\geq0$, la funci\'on generadora de momentos $\esp\left[e^{\alpha N\left(t\right)}\right]$ existe para alguna $\alpha$ en una vecindad del 0, y de aqu\'i que $\esp\left[N\left(t\right)^{m}\right]<\infty$, para $m\geq1$.
\end{Prop}


\begin{Note}
Si el primer tiempo de renovaci\'on $\xi_{1}$ no tiene la misma distribuci\'on que el resto de las $\xi_{n}$, para $n\geq2$, a $N\left(t\right)$ se le llama Proceso de Renovaci\'on retardado, donde si $\xi$ tiene distribuci\'on $G$, entonces el tiempo $T_{n}$ de la $n$-\'esima renovaci\'on tiene distribuci\'on $G\star F^{\left(n-1\right)\star}\left(t\right)$
\end{Note}


\begin{Teo}
Para una constante $\mu\leq\infty$ ( o variable aleatoria), las siguientes expresiones son equivalentes:

\begin{eqnarray}
lim_{n\rightarrow\infty}n^{-1}T_{n}&=&\mu,\textrm{ c.s.}\\
lim_{t\rightarrow\infty}t^{-1}N\left(t\right)&=&1/\mu,\textrm{ c.s.}
\end{eqnarray}
\end{Teo}


Es decir, $T_{n}$ satisface la Ley Fuerte de los Grandes N\'umeros s\'i y s\'olo s\'i $N\left/t\right)$ la cumple.


\begin{Coro}[Ley Fuerte de los Grandes N\'umeros para Procesos de Renovaci\'on]
Si $N\left(t\right)$ es un proceso de renovaci\'on cuyos tiempos de inter-renovaci\'on tienen media $\mu\leq\infty$, entonces
\begin{eqnarray}
t^{-1}N\left(t\right)\rightarrow 1/\mu,\textrm{ c.s. cuando }t\rightarrow\infty.
\end{eqnarray}

\end{Coro}


Considerar el proceso estoc\'astico de valores reales $\left\{Z\left(t\right):t\geq0\right\}$ en el mismo espacio de probabilidad que $N\left(t\right)$

\begin{Def}
Para el proceso $\left\{Z\left(t\right):t\geq0\right\}$ se define la fluctuaci\'on m\'axima de $Z\left(t\right)$ en el intervalo $\left(T_{n-1},T_{n}\right]$:
\begin{eqnarray*}
M_{n}=\sup_{T_{n-1}<t\leq T_{n}}|Z\left(t\right)-Z\left(T_{n-1}\right)|
\end{eqnarray*}
\end{Def}

\begin{Teo}
Sup\'ongase que $n^{-1}T_{n}\rightarrow\mu$ c.s. cuando $n\rightarrow\infty$, donde $\mu\leq\infty$ es una constante o variable aleatoria. Sea $a$ una constante o variable aleatoria que puede ser infinita cuando $\mu$ es finita, y considere las expresiones l\'imite:
\begin{eqnarray}
lim_{n\rightarrow\infty}n^{-1}Z\left(T_{n}\right)&=&a,\textrm{ c.s.}\\
lim_{t\rightarrow\infty}t^{-1}Z\left(t\right)&=&a/\mu,\textrm{ c.s.}
\end{eqnarray}
La segunda expresi\'on implica la primera. Conversamente, la primera implica la segunda si el proceso $Z\left(t\right)$ es creciente, o si $lim_{n\rightarrow\infty}n^{-1}M_{n}=0$ c.s.
\end{Teo}

\begin{Coro}
Si $N\left(t\right)$ es un proceso de renovaci\'on, y $\left(Z\left(T_{n}\right)-Z\left(T_{n-1}\right),M_{n}\right)$, para $n\geq1$, son variables aleatorias independientes e id\'enticamente distribuidas con media finita, entonces,
\begin{eqnarray}
lim_{t\rightarrow\infty}t^{-1}Z\left(t\right)\rightarrow\frac{\esp\left[Z\left(T_{1}\right)-Z\left(T_{0}\right)\right]}{\esp\left[T_{1}\right]},\textrm{ c.s. cuando  }t\rightarrow\infty.
\end{eqnarray}
\end{Coro}


%___________________________________________________________________________________________
%
\section{Funci\'on de Renovaci\'on}
%___________________________________________________________________________________________
%


\begin{Def}
Sea $h\left(t\right)$ funci\'on de valores reales en $\rea$ acotada en intervalos finitos e igual a cero para $t<0$ La ecuaci\'on de renovaci\'on para $h\left(t\right)$ y la distribuci\'on $F$ es

\begin{eqnarray}\label{Ec.Renovacion}
H\left(t\right)=h\left(t\right)+\int_{\left[0,t\right]}H\left(t-s\right)dF\left(s\right)\textrm{,    }t\geq0,
\end{eqnarray}
donde $H\left(t\right)$ es una funci\'on de valores reales. Esto es $H=h+F\star H$. Decimos que $H\left(t\right)$ es soluci\'on de esta ecuaci\'on si satisface la ecuaci\'on, y es acotada en intervalos finitos e iguales a cero para $t<0$.
\end{Def}

\begin{Prop}
La funci\'on $U\star h\left(t\right)$ es la \'unica soluci\'on de la ecuaci\'on de renovaci\'on (\ref{Ec.Renovacion}).
\end{Prop}

\begin{Teo}[Teorema Renovaci\'on Elemental]
\begin{eqnarray*}
t^{-1}U\left(t\right)\rightarrow 1/\mu\textrm{,    cuando }t\rightarrow\infty.
\end{eqnarray*}
\end{Teo}

%___________________________________________________________________________________________
%
%\subsection*{Funci\'on de Renovaci\'on}
%___________________________________________________________________________________________
%


Sup\'ongase que $N\left(t\right)$ es un proceso de renovaci\'on con distribuci\'on $F$ con media finita $\mu$.

\begin{Def}
La funci\'on de renovaci\'on asociada con la distribuci\'on $F$, del proceso $N\left(t\right)$, es
\begin{eqnarray*}
U\left(t\right)=\sum_{n=1}^{\infty}F^{n\star}\left(t\right),\textrm{   }t\geq0,
\end{eqnarray*}
donde $F^{0\star}\left(t\right)=\indora\left(t\geq0\right)$.
\end{Def}


\begin{Prop}
Sup\'ongase que la distribuci\'on de inter-renovaci\'on $F$ tiene densidad $f$. Entonces $U\left(t\right)$ tambi\'en tiene densidad, para $t>0$, y es $U^{'}\left(t\right)=\sum_{n=0}^{\infty}f^{n\star}\left(t\right)$. Adem\'as
\begin{eqnarray*}
\prob\left\{N\left(t\right)>N\left(t-\right)\right\}=0\textrm{,   }t\geq0.
\end{eqnarray*}
\end{Prop}

\begin{Def}
La Transformada de Laplace-Stieljes de $F$ est\'a dada por

\begin{eqnarray*}
\hat{F}\left(\alpha\right)=\int_{\rea_{+}}e^{-\alpha t}dF\left(t\right)\textrm{,  }\alpha\geq0.
\end{eqnarray*}
\end{Def}

Entonces

\begin{eqnarray*}
\hat{U}\left(\alpha\right)=\sum_{n=0}^{\infty}\hat{F^{n\star}}\left(\alpha\right)=\sum_{n=0}^{\infty}\hat{F}\left(\alpha\right)^{n}=\frac{1}{1-\hat{F}\left(\alpha\right)}.
\end{eqnarray*}


\begin{Prop}
La Transformada de Laplace $\hat{U}\left(\alpha\right)$ y $\hat{F}\left(\alpha\right)$ determina una a la otra de manera \'unica por la relaci\'on $\hat{U}\left(\alpha\right)=\frac{1}{1-\hat{F}\left(\alpha\right)}$.
\end{Prop}


\begin{Note}
Un proceso de renovaci\'on $N\left(t\right)$ cuyos tiempos de inter-renovaci\'on tienen media finita, es un proceso Poisson con tasa $\lambda$ si y s\'olo s\'i $\esp\left[U\left(t\right)\right]=\lambda t$, para $t\geq0$.
\end{Note}


\begin{Teo}
Sea $N\left(t\right)$ un proceso puntual simple con puntos de localizaci\'on $T_{n}$ tal que $\eta\left(t\right)=\esp\left[N\left(\right)\right]$ es finita para cada $t$. Entonces para cualquier funci\'on $f:\rea_{+}\rightarrow\rea$,
\begin{eqnarray*}
\esp\left[\sum_{n=1}^{N\left(\right)}f\left(T_{n}\right)\right]=\int_{\left(0,t\right]}f\left(s\right)d\eta\left(s\right)\textrm{,  }t\geq0,
\end{eqnarray*}
suponiendo que la integral exista. Adem\'as si $X_{1},X_{2},\ldots$ son variables aleatorias definidas en el mismo espacio de probabilidad que el proceso $N\left(t\right)$ tal que $\esp\left[X_{n}|T_{n}=s\right]=f\left(s\right)$, independiente de $n$. Entonces
\begin{eqnarray*}
\esp\left[\sum_{n=1}^{N\left(t\right)}X_{n}\right]=\int_{\left(0,t\right]}f\left(s\right)d\eta\left(s\right)\textrm{,  }t\geq0,
\end{eqnarray*} 
suponiendo que la integral exista. 
\end{Teo}

\begin{Coro}[Identidad de Wald para Renovaciones]
Para el proceso de renovaci\'on $N\left(t\right)$,
\begin{eqnarray*}
\esp\left[T_{N\left(t\right)+1}\right]=\mu\esp\left[N\left(t\right)+1\right]\textrm{,  }t\geq0,
\end{eqnarray*}  
\end{Coro}



%___________________________________________________________________________________________
%
%\subsection*{Funci\'on de Renovaci\'on}
%___________________________________________________________________________________________
%


\begin{Def}
Sea $h\left(t\right)$ funci\'on de valores reales en $\rea$ acotada en intervalos finitos e igual a cero para $t<0$ La ecuaci\'on de renovaci\'on para $h\left(t\right)$ y la distribuci\'on $F$ es

\begin{eqnarray}%\label{Ec.Renovacion}
H\left(t\right)=h\left(t\right)+\int_{\left[0,t\right]}H\left(t-s\right)dF\left(s\right)\textrm{,    }t\geq0,
\end{eqnarray}
donde $H\left(t\right)$ es una funci\'on de valores reales. Esto es $H=h+F\star H$. Decimos que $H\left(t\right)$ es soluci\'on de esta ecuaci\'on si satisface la ecuaci\'on, y es acotada en intervalos finitos e iguales a cero para $t<0$.
\end{Def}

\begin{Prop}
La funci\'on $U\star h\left(t\right)$ es la \'unica soluci\'on de la ecuaci\'on de renovaci\'on (\ref{Ec.Renovacion}).
\end{Prop}

\begin{Teo}[Teorema Renovaci\'on Elemental]
\begin{eqnarray*}
t^{-1}U\left(t\right)\rightarrow 1/\mu\textrm{,    cuando }t\rightarrow\infty.
\end{eqnarray*}
\end{Teo}

%___________________________________________________________________________________________
%
%\subsection*{Funci\'on de Renovaci\'on}
%___________________________________________________________________________________________
%


Sup\'ongase que $N\left(t\right)$ es un proceso de renovaci\'on con distribuci\'on $F$ con media finita $\mu$.

\begin{Def}
La funci\'on de renovaci\'on asociada con la distribuci\'on $F$, del proceso $N\left(t\right)$, es
\begin{eqnarray*}
U\left(t\right)=\sum_{n=1}^{\infty}F^{n\star}\left(t\right),\textrm{   }t\geq0,
\end{eqnarray*}
donde $F^{0\star}\left(t\right)=\indora\left(t\geq0\right)$.
\end{Def}


\begin{Prop}
Sup\'ongase que la distribuci\'on de inter-renovaci\'on $F$ tiene densidad $f$. Entonces $U\left(t\right)$ tambi\'en tiene densidad, para $t>0$, y es $U^{'}\left(t\right)=\sum_{n=0}^{\infty}f^{n\star}\left(t\right)$. Adem\'as
\begin{eqnarray*}
\prob\left\{N\left(t\right)>N\left(t-\right)\right\}=0\textrm{,   }t\geq0.
\end{eqnarray*}
\end{Prop}

\begin{Def}
La Transformada de Laplace-Stieljes de $F$ est\'a dada por

\begin{eqnarray*}
\hat{F}\left(\alpha\right)=\int_{\rea_{+}}e^{-\alpha t}dF\left(t\right)\textrm{,  }\alpha\geq0.
\end{eqnarray*}
\end{Def}

Entonces

\begin{eqnarray*}
\hat{U}\left(\alpha\right)=\sum_{n=0}^{\infty}\hat{F^{n\star}}\left(\alpha\right)=\sum_{n=0}^{\infty}\hat{F}\left(\alpha\right)^{n}=\frac{1}{1-\hat{F}\left(\alpha\right)}.
\end{eqnarray*}


\begin{Prop}
La Transformada de Laplace $\hat{U}\left(\alpha\right)$ y $\hat{F}\left(\alpha\right)$ determina una a la otra de manera \'unica por la relaci\'on $\hat{U}\left(\alpha\right)=\frac{1}{1-\hat{F}\left(\alpha\right)}$.
\end{Prop}


\begin{Note}
Un proceso de renovaci\'on $N\left(t\right)$ cuyos tiempos de inter-renovaci\'on tienen media finita, es un proceso Poisson con tasa $\lambda$ si y s\'olo s\'i $\esp\left[U\left(t\right)\right]=\lambda t$, para $t\geq0$.
\end{Note}


\begin{Teo}
Sea $N\left(t\right)$ un proceso puntual simple con puntos de localizaci\'on $T_{n}$ tal que $\eta\left(t\right)=\esp\left[N\left(\right)\right]$ es finita para cada $t$. Entonces para cualquier funci\'on $f:\rea_{+}\rightarrow\rea$,
\begin{eqnarray*}
\esp\left[\sum_{n=1}^{N\left(\right)}f\left(T_{n}\right)\right]=\int_{\left(0,t\right]}f\left(s\right)d\eta\left(s\right)\textrm{,  }t\geq0,
\end{eqnarray*}
suponiendo que la integral exista. Adem\'as si $X_{1},X_{2},\ldots$ son variables aleatorias definidas en el mismo espacio de probabilidad que el proceso $N\left(t\right)$ tal que $\esp\left[X_{n}|T_{n}=s\right]=f\left(s\right)$, independiente de $n$. Entonces
\begin{eqnarray*}
\esp\left[\sum_{n=1}^{N\left(t\right)}X_{n}\right]=\int_{\left(0,t\right]}f\left(s\right)d\eta\left(s\right)\textrm{,  }t\geq0,
\end{eqnarray*} 
suponiendo que la integral exista. 
\end{Teo}

\begin{Coro}[Identidad de Wald para Renovaciones]
Para el proceso de renovaci\'on $N\left(t\right)$,
\begin{eqnarray*}
\esp\left[T_{N\left(t\right)+1}\right]=\mu\esp\left[N\left(t\right)+1\right]\textrm{,  }t\geq0,
\end{eqnarray*}  
\end{Coro}


%___________________________________________________________________________________________
%
%\subsection*{Funci\'on de Renovaci\'on}
%___________________________________________________________________________________________
%


\begin{Def}
Sea $h\left(t\right)$ funci\'on de valores reales en $\rea$ acotada en intervalos finitos e igual a cero para $t<0$ La ecuaci\'on de renovaci\'on para $h\left(t\right)$ y la distribuci\'on $F$ es

\begin{eqnarray}\label{Ec.Renovacion}
H\left(t\right)=h\left(t\right)+\int_{\left[0,t\right]}H\left(t-s\right)dF\left(s\right)\textrm{,    }t\geq0,
\end{eqnarray}
donde $H\left(t\right)$ es una funci\'on de valores reales. Esto es $H=h+F\star H$. Decimos que $H\left(t\right)$ es soluci\'on de esta ecuaci\'on si satisface la ecuaci\'on, y es acotada en intervalos finitos e iguales a cero para $t<0$.
\end{Def}

\begin{Prop}
La funci\'on $U\star h\left(t\right)$ es la \'unica soluci\'on de la ecuaci\'on de renovaci\'on (\ref{Ec.Renovacion}).
\end{Prop}

\begin{Teo}[Teorema Renovaci\'on Elemental]
\begin{eqnarray*}
t^{-1}U\left(t\right)\rightarrow 1/\mu\textrm{,    cuando }t\rightarrow\infty.
\end{eqnarray*}
\end{Teo}

%___________________________________________________________________________________________
%
%\subsection*{Funci\'on de Renovaci\'on}
%___________________________________________________________________________________________
%


Sup\'ongase que $N\left(t\right)$ es un proceso de renovaci\'on con distribuci\'on $F$ con media finita $\mu$.

\begin{Def}
La funci\'on de renovaci\'on asociada con la distribuci\'on $F$, del proceso $N\left(t\right)$, es
\begin{eqnarray*}
U\left(t\right)=\sum_{n=1}^{\infty}F^{n\star}\left(t\right),\textrm{   }t\geq0,
\end{eqnarray*}
donde $F^{0\star}\left(t\right)=\indora\left(t\geq0\right)$.
\end{Def}


\begin{Prop}
Sup\'ongase que la distribuci\'on de inter-renovaci\'on $F$ tiene densidad $f$. Entonces $U\left(t\right)$ tambi\'en tiene densidad, para $t>0$, y es $U^{'}\left(t\right)=\sum_{n=0}^{\infty}f^{n\star}\left(t\right)$. Adem\'as
\begin{eqnarray*}
\prob\left\{N\left(t\right)>N\left(t-\right)\right\}=0\textrm{,   }t\geq0.
\end{eqnarray*}
\end{Prop}

\begin{Def}
La Transformada de Laplace-Stieljes de $F$ est\'a dada por

\begin{eqnarray*}
\hat{F}\left(\alpha\right)=\int_{\rea_{+}}e^{-\alpha t}dF\left(t\right)\textrm{,  }\alpha\geq0.
\end{eqnarray*}
\end{Def}

Entonces

\begin{eqnarray*}
\hat{U}\left(\alpha\right)=\sum_{n=0}^{\infty}\hat{F^{n\star}}\left(\alpha\right)=\sum_{n=0}^{\infty}\hat{F}\left(\alpha\right)^{n}=\frac{1}{1-\hat{F}\left(\alpha\right)}.
\end{eqnarray*}


\begin{Prop}
La Transformada de Laplace $\hat{U}\left(\alpha\right)$ y $\hat{F}\left(\alpha\right)$ determina una a la otra de manera \'unica por la relaci\'on $\hat{U}\left(\alpha\right)=\frac{1}{1-\hat{F}\left(\alpha\right)}$.
\end{Prop}


\begin{Note}
Un proceso de renovaci\'on $N\left(t\right)$ cuyos tiempos de inter-renovaci\'on tienen media finita, es un proceso Poisson con tasa $\lambda$ si y s\'olo s\'i $\esp\left[U\left(t\right)\right]=\lambda t$, para $t\geq0$.
\end{Note}


\begin{Teo}
Sea $N\left(t\right)$ un proceso puntual simple con puntos de localizaci\'on $T_{n}$ tal que $\eta\left(t\right)=\esp\left[N\left(\right)\right]$ es finita para cada $t$. Entonces para cualquier funci\'on $f:\rea_{+}\rightarrow\rea$,
\begin{eqnarray*}
\esp\left[\sum_{n=1}^{N\left(\right)}f\left(T_{n}\right)\right]=\int_{\left(0,t\right]}f\left(s\right)d\eta\left(s\right)\textrm{,  }t\geq0,
\end{eqnarray*}
suponiendo que la integral exista. Adem\'as si $X_{1},X_{2},\ldots$ son variables aleatorias definidas en el mismo espacio de probabilidad que el proceso $N\left(t\right)$ tal que $\esp\left[X_{n}|T_{n}=s\right]=f\left(s\right)$, independiente de $n$. Entonces
\begin{eqnarray*}
\esp\left[\sum_{n=1}^{N\left(t\right)}X_{n}\right]=\int_{\left(0,t\right]}f\left(s\right)d\eta\left(s\right)\textrm{,  }t\geq0,
\end{eqnarray*} 
suponiendo que la integral exista. 
\end{Teo}

\begin{Coro}[Identidad de Wald para Renovaciones]
Para el proceso de renovaci\'on $N\left(t\right)$,
\begin{eqnarray*}
\esp\left[T_{N\left(t\right)+1}\right]=\mu\esp\left[N\left(t\right)+1\right]\textrm{,  }t\geq0,
\end{eqnarray*}  
\end{Coro}

%______________________________________________________________________
\section{Procesos de Renovaci\'on}
%______________________________________________________________________

\begin{Def}\label{Def.Tn}
Sean $0\leq T_{1}\leq T_{2}\leq \ldots$ son tiempos aleatorios infinitos en los cuales ocurren ciertos eventos. El n\'umero de tiempos $T_{n}$ en el intervalo $\left[0,t\right)$ es

\begin{eqnarray}
N\left(t\right)=\sum_{n=1}^{\infty}\indora\left(T_{n}\leq t\right),
\end{eqnarray}
para $t\geq0$.
\end{Def}

Si se consideran los puntos $T_{n}$ como elementos de $\rea_{+}$, y $N\left(t\right)$ es el n\'umero de puntos en $\rea$. El proceso denotado por $\left\{N\left(t\right):t\geq0\right\}$, denotado por $N\left(t\right)$, es un proceso puntual en $\rea_{+}$. Los $T_{n}$ son los tiempos de ocurrencia, el proceso puntual $N\left(t\right)$ es simple si su n\'umero de ocurrencias son distintas: $0<T_{1}<T_{2}<\ldots$ casi seguramente.

\begin{Def}
Un proceso puntual $N\left(t\right)$ es un proceso de renovaci\'on si los tiempos de interocurrencia $\xi_{n}=T_{n}-T_{n-1}$, para $n\geq1$, son independientes e identicamente distribuidos con distribuci\'on $F$, donde $F\left(0\right)=0$ y $T_{0}=0$. Los $T_{n}$ son llamados tiempos de renovaci\'on, referente a la independencia o renovaci\'on de la informaci\'on estoc\'astica en estos tiempos. Los $\xi_{n}$ son los tiempos de inter-renovaci\'on, y $N\left(t\right)$ es el n\'umero de renovaciones en el intervalo $\left[0,t\right)$
\end{Def}


\begin{Note}
Para definir un proceso de renovaci\'on para cualquier contexto, solamente hay que especificar una distribuci\'on $F$, con $F\left(0\right)=0$, para los tiempos de inter-renovaci\'on. La funci\'on $F$ en turno degune las otra variables aleatorias. De manera formal, existe un espacio de probabilidad y una sucesi\'on de variables aleatorias $\xi_{1},\xi_{2},\ldots$ definidas en este con distribuci\'on $F$. Entonces las otras cantidades son $T_{n}=\sum_{k=1}^{n}\xi_{k}$ y $N\left(t\right)=\sum_{n=1}^{\infty}\indora\left(T_{n}\leq t\right)$, donde $T_{n}\rightarrow\infty$ casi seguramente por la Ley Fuerte de los Grandes Números.
\end{Note}


\begin{Ejem}[\textbf{Proceso Poisson}]

Suponga que se tienen tiempos de inter-renovaci\'on \textit{i.i.d.} del proceso de renovaci\'on $N\left(t\right)$ tienen distribuci\'on exponencial $F\left(t\right)=q-e^{-\lambda t}$ con tasa $\lambda$. Entonces $N\left(t\right)$ es un proceso Poisson con tasa $\lambda$.

\end{Ejem}


\begin{Note}
Si el primer tiempo de renovaci\'on $\xi_{1}$ no tiene la misma distribuci\'on que el resto de las $\xi_{n}$, para $n\geq2$, a $N\left(t\right)$ se le llama Proceso de Renovaci\'on retardado, donde si $\xi$ tiene distribuci\'on $G$, entonces el tiempo $T_{n}$ de la $n$-\'esima renovaci\'on tiene distribuci\'on $G\star F^{\left(n-1\right)\star}\left(t\right)$
\end{Note}

\begin{Note} Una funci\'on $h:\rea_{+}\rightarrow\rea$ es Directamente Riemann Integrable en los siguientes casos:
\begin{itemize}
\item[a)] $h\left(t\right)\geq0$ es decreciente y Riemann Integrable.
\item[b)] $h$ es continua excepto posiblemente en un conjunto de Lebesgue de medida 0, y $|h\left(t\right)|\leq b\left(t\right)$, donde $b$ es DRI.
\end{itemize}
\end{Note}

\begin{Teo}[Teorema Principal de Renovaci\'on]
Si $F$ es no aritm\'etica y $h\left(t\right)$ es Directamente Riemann Integrable (DRI), entonces

\begin{eqnarray*}
lim_{t\rightarrow\infty}U\star h=\frac{1}{\mu}\int_{\rea_{+}}h\left(s\right)ds.
\end{eqnarray*}
\end{Teo}

\begin{Prop}
Cualquier funci\'on $H\left(t\right)$ acotada en intervalos finitos y que es 0 para $t<0$ puede expresarse como
\begin{eqnarray*}
H\left(t\right)=U\star h\left(t\right)\textrm{,  donde }h\left(t\right)=H\left(t\right)-F\star H\left(t\right)
\end{eqnarray*}
\end{Prop}

\begin{Def}
Un proceso estoc\'astico $X\left(t\right)$ es crudamente regenerativo en un tiempo aleatorio positivo $T$ si
\begin{eqnarray*}
\esp\left[X\left(T+t\right)|T\right]=\esp\left[X\left(t\right)\right]\textrm{, para }t\geq0,\end{eqnarray*}
y con las esperanzas anteriores finitas.
\end{Def}

\begin{Prop}
Sup\'ongase que $X\left(t\right)$ es un proceso crudamente regenerativo en $T$, que tiene distribuci\'on $F$. Si $\esp\left[X\left(t\right)\right]$ es acotado en intervalos finitos, entonces
\begin{eqnarray*}
\esp\left[X\left(t\right)\right]=U\star h\left(t\right)\textrm{,  donde }h\left(t\right)=\esp\left[X\left(t\right)\indora\left(T>t\right)\right].
\end{eqnarray*}
\end{Prop}

\begin{Teo}[Regeneraci\'on Cruda]
Sup\'ongase que $X\left(t\right)$ es un proceso con valores positivo crudamente regenerativo en $T$, y def\'inase $M=\sup\left\{|X\left(t\right)|:t\leq T\right\}$. Si $T$ es no aritm\'etico y $M$ y $MT$ tienen media finita, entonces
\begin{eqnarray*}
lim_{t\rightarrow\infty}\esp\left[X\left(t\right)\right]=\frac{1}{\mu}\int_{\rea_{+}}h\left(s\right)ds,
\end{eqnarray*}
donde $h\left(t\right)=\esp\left[X\left(t\right)\indora\left(T>t\right)\right]$.
\end{Teo}


\begin{Note} Una funci\'on $h:\rea_{+}\rightarrow\rea$ es Directamente Riemann Integrable en los siguientes casos:
\begin{itemize}
\item[a)] $h\left(t\right)\geq0$ es decreciente y Riemann Integrable.
\item[b)] $h$ es continua excepto posiblemente en un conjunto de Lebesgue de medida 0, y $|h\left(t\right)|\leq b\left(t\right)$, donde $b$ es DRI.
\end{itemize}
\end{Note}

\begin{Teo}[Teorema Principal de Renovaci\'on]
Si $F$ es no aritm\'etica y $h\left(t\right)$ es Directamente Riemann Integrable (DRI), entonces

\begin{eqnarray*}
lim_{t\rightarrow\infty}U\star h=\frac{1}{\mu}\int_{\rea_{+}}h\left(s\right)ds.
\end{eqnarray*}
\end{Teo}

\begin{Prop}
Cualquier funci\'on $H\left(t\right)$ acotada en intervalos finitos y que es 0 para $t<0$ puede expresarse como
\begin{eqnarray*}
H\left(t\right)=U\star h\left(t\right)\textrm{,  donde }h\left(t\right)=H\left(t\right)-F\star H\left(t\right)
\end{eqnarray*}
\end{Prop}

\begin{Def}
Un proceso estoc\'astico $X\left(t\right)$ es crudamente regenerativo en un tiempo aleatorio positivo $T$ si
\begin{eqnarray*}
\esp\left[X\left(T+t\right)|T\right]=\esp\left[X\left(t\right)\right]\textrm{, para }t\geq0,\end{eqnarray*}
y con las esperanzas anteriores finitas.
\end{Def}

\begin{Prop}
Sup\'ongase que $X\left(t\right)$ es un proceso crudamente regenerativo en $T$, que tiene distribuci\'on $F$. Si $\esp\left[X\left(t\right)\right]$ es acotado en intervalos finitos, entonces
\begin{eqnarray*}
\esp\left[X\left(t\right)\right]=U\star h\left(t\right)\textrm{,  donde }h\left(t\right)=\esp\left[X\left(t\right)\indora\left(T>t\right)\right].
\end{eqnarray*}
\end{Prop}

\begin{Teo}[Regeneraci\'on Cruda]
Sup\'ongase que $X\left(t\right)$ es un proceso con valores positivo crudamente regenerativo en $T$, y def\'inase $M=\sup\left\{|X\left(t\right)|:t\leq T\right\}$. Si $T$ es no aritm\'etico y $M$ y $MT$ tienen media finita, entonces
\begin{eqnarray*}
lim_{t\rightarrow\infty}\esp\left[X\left(t\right)\right]=\frac{1}{\mu}\int_{\rea_{+}}h\left(s\right)ds,
\end{eqnarray*}
donde $h\left(t\right)=\esp\left[X\left(t\right)\indora\left(T>t\right)\right]$.
\end{Teo}

\begin{Def}
Para el proceso $\left\{\left(N\left(t\right),X\left(t\right)\right):t\geq0\right\}$, sus trayectoria muestrales en el intervalo de tiempo $\left[T_{n-1},T_{n}\right)$ est\'an descritas por
\begin{eqnarray*}
\zeta_{n}=\left(\xi_{n},\left\{X\left(T_{n-1}+t\right):0\leq t<\xi_{n}\right\}\right)
\end{eqnarray*}
Este $\zeta_{n}$ es el $n$-\'esimo segmento del proceso. El proceso es regenerativo sobre los tiempos $T_{n}$ si sus segmentos $\zeta_{n}$ son independientes e id\'enticamennte distribuidos.
\end{Def}


\begin{Note}
Si $\tilde{X}\left(t\right)$ con espacio de estados $\tilde{S}$ es regenerativo sobre $T_{n}$, entonces $X\left(t\right)=f\left(\tilde{X}\left(t\right)\right)$ tambi\'en es regenerativo sobre $T_{n}$, para cualquier funci\'on $f:\tilde{S}\rightarrow S$.
\end{Note}

\begin{Note}
Los procesos regenerativos son crudamente regenerativos, pero no al rev\'es.
\end{Note}


\begin{Note}
Un proceso estoc\'astico a tiempo continuo o discreto es regenerativo si existe un proceso de renovaci\'on  tal que los segmentos del proceso entre tiempos de renovaci\'on sucesivos son i.i.d., es decir, para $\left\{X\left(t\right):t\geq0\right\}$ proceso estoc\'astico a tiempo continuo con espacio de estados $S$, espacio m\'etrico.
\end{Note}

Para $\left\{X\left(t\right):t\geq0\right\}$ Proceso Estoc\'astico a tiempo continuo con estado de espacios $S$, que es un espacio m\'etrico, con trayectorias continuas por la derecha y con l\'imites por la izquierda c.s. Sea $N\left(t\right)$ un proceso de renovaci\'on en $\rea_{+}$ definido en el mismo espacio de probabilidad que $X\left(t\right)$, con tiempos de renovaci\'on $T$ y tiempos de inter-renovaci\'on $\xi_{n}=T_{n}-T_{n-1}$, con misma distribuci\'on $F$ de media finita $\mu$.



\begin{Def}
Para el proceso $\left\{\left(N\left(t\right),X\left(t\right)\right):t\geq0\right\}$, sus trayectoria muestrales en el intervalo de tiempo $\left[T_{n-1},T_{n}\right)$ est\'an descritas por
\begin{eqnarray*}
\zeta_{n}=\left(\xi_{n},\left\{X\left(T_{n-1}+t\right):0\leq t<\xi_{n}\right\}\right)
\end{eqnarray*}
Este $\zeta_{n}$ es el $n$-\'esimo segmento del proceso. El proceso es regenerativo sobre los tiempos $T_{n}$ si sus segmentos $\zeta_{n}$ son independientes e id\'enticamennte distribuidos.
\end{Def}

\begin{Note}
Un proceso regenerativo con media de la longitud de ciclo finita es llamado positivo recurrente.
\end{Note}

\begin{Teo}[Procesos Regenerativos]
Suponga que el proceso
\end{Teo}


\begin{Def}[Renewal Process Trinity]
Para un proceso de renovaci\'on $N\left(t\right)$, los siguientes procesos proveen de informaci\'on sobre los tiempos de renovaci\'on.
\begin{itemize}
\item $A\left(t\right)=t-T_{N\left(t\right)}$, el tiempo de recurrencia hacia atr\'as al tiempo $t$, que es el tiempo desde la \'ultima renovaci\'on para $t$.

\item $B\left(t\right)=T_{N\left(t\right)+1}-t$, el tiempo de recurrencia hacia adelante al tiempo $t$, residual del tiempo de renovaci\'on, que es el tiempo para la pr\'oxima renovaci\'on despu\'es de $t$.

\item $L\left(t\right)=\xi_{N\left(t\right)+1}=A\left(t\right)+B\left(t\right)$, la longitud del intervalo de renovaci\'on que contiene a $t$.
\end{itemize}
\end{Def}

\begin{Note}
El proceso tridimensional $\left(A\left(t\right),B\left(t\right),L\left(t\right)\right)$ es regenerativo sobre $T_{n}$, y por ende cada proceso lo es. Cada proceso $A\left(t\right)$ y $B\left(t\right)$ son procesos de MArkov a tiempo continuo con trayectorias continuas por partes en el espacio de estados $\rea_{+}$. Una expresi\'on conveniente para su distribuci\'on conjunta es, para $0\leq x<t,y\geq0$
\begin{equation}\label{NoRenovacion}
P\left\{A\left(t\right)>x,B\left(t\right)>y\right\}=
P\left\{N\left(t+y\right)-N\left((t-x)\right)=0\right\}
\end{equation}
\end{Note}

\begin{Ejem}[Tiempos de recurrencia Poisson]
Si $N\left(t\right)$ es un proceso Poisson con tasa $\lambda$, entonces de la expresi\'on (\ref{NoRenovacion}) se tiene que

\begin{eqnarray*}
\begin{array}{lc}
P\left\{A\left(t\right)>x,B\left(t\right)>y\right\}=e^{-\lambda\left(x+y\right)},&0\leq x<t,y\geq0,
\end{array}
\end{eqnarray*}
que es la probabilidad Poisson de no renovaciones en un intervalo de longitud $x+y$.

\end{Ejem}

\begin{Note}
Una cadena de Markov erg\'odica tiene la propiedad de ser estacionaria si la distribución de su estado al tiempo $0$ es su distribuci\'on estacionaria.
\end{Note}


\begin{Def}
Un proceso estoc\'astico a tiempo continuo $\left\{X\left(t\right):t\geq0\right\}$ en un espacio general es estacionario si sus distribuciones finito dimensionales son invariantes bajo cualquier  traslado: para cada $0\leq s_{1}<s_{2}<\cdots<s_{k}$ y $t\geq0$,
\begin{eqnarray*}
\left(X\left(s_{1}+t\right),\ldots,X\left(s_{k}+t\right)\right)=_{d}\left(X\left(s_{1}\right),\ldots,X\left(s_{k}\right)\right).
\end{eqnarray*}
\end{Def}

\begin{Note}
Un proceso de Markov es estacionario si $X\left(t\right)=_{d}X\left(0\right)$, $t\geq0$.
\end{Note}

Considerese el proceso $N\left(t\right)=\sum_{n}\indora\left(\tau_{n}\leq t\right)$ en $\rea_{+}$, con puntos $0<\tau_{1}<\tau_{2}<\cdots$.

\begin{Prop}
Si $N$ es un proceso puntual estacionario y $\esp\left[N\left(1\right)\right]<\infty$, entonces $\esp\left[N\left(t\right)\right]=t\esp\left[N\left(1\right)\right]$, $t\geq0$

\end{Prop}

\begin{Teo}
Los siguientes enunciados son equivalentes
\begin{itemize}
\item[i)] El proceso retardado de renovaci\'on $N$ es estacionario.

\item[ii)] EL proceso de tiempos de recurrencia hacia adelante $B\left(t\right)$ es estacionario.


\item[iii)] $\esp\left[N\left(t\right)\right]=t/\mu$,


\item[iv)] $G\left(t\right)=F_{e}\left(t\right)=\frac{1}{\mu}\int_{0}^{t}\left[1-F\left(s\right)\right]ds$
\end{itemize}
Cuando estos enunciados son ciertos, $P\left\{B\left(t\right)\leq x\right\}=F_{e}\left(x\right)$, para $t,x\geq0$.

\end{Teo}

\begin{Note}
Una consecuencia del teorema anterior es que el Proceso Poisson es el \'unico proceso sin retardo que es estacionario.
\end{Note}

\begin{Coro}
El proceso de renovaci\'on $N\left(t\right)$ sin retardo, y cuyos tiempos de inter renonaci\'on tienen media finita, es estacionario si y s\'olo si es un proceso Poisson.

\end{Coro}

%______________________________________________________________________
%\subsection*{Procesos de Renovaci\'on}
%______________________________________________________________________

\begin{Def}\label{Def.Tn}
Sean $0\leq T_{1}\leq T_{2}\leq \ldots$ son tiempos aleatorios infinitos en los cuales ocurren ciertos eventos. El n\'umero de tiempos $T_{n}$ en el intervalo $\left[0,t\right)$ es

\begin{eqnarray}
N\left(t\right)=\sum_{n=1}^{\infty}\indora\left(T_{n}\leq t\right),
\end{eqnarray}
para $t\geq0$.
\end{Def}

Si se consideran los puntos $T_{n}$ como elementos de $\rea_{+}$, y $N\left(t\right)$ es el n\'umero de puntos en $\rea$. El proceso denotado por $\left\{N\left(t\right):t\geq0\right\}$, denotado por $N\left(t\right)$, es un proceso puntual en $\rea_{+}$. Los $T_{n}$ son los tiempos de ocurrencia, el proceso puntual $N\left(t\right)$ es simple si su n\'umero de ocurrencias son distintas: $0<T_{1}<T_{2}<\ldots$ casi seguramente.

\begin{Def}
Un proceso puntual $N\left(t\right)$ es un proceso de renovaci\'on si los tiempos de interocurrencia $\xi_{n}=T_{n}-T_{n-1}$, para $n\geq1$, son independientes e identicamente distribuidos con distribuci\'on $F$, donde $F\left(0\right)=0$ y $T_{0}=0$. Los $T_{n}$ son llamados tiempos de renovaci\'on, referente a la independencia o renovaci\'on de la informaci\'on estoc\'astica en estos tiempos. Los $\xi_{n}$ son los tiempos de inter-renovaci\'on, y $N\left(t\right)$ es el n\'umero de renovaciones en el intervalo $\left[0,t\right)$
\end{Def}


\begin{Note}
Para definir un proceso de renovaci\'on para cualquier contexto, solamente hay que especificar una distribuci\'on $F$, con $F\left(0\right)=0$, para los tiempos de inter-renovaci\'on. La funci\'on $F$ en turno degune las otra variables aleatorias. De manera formal, existe un espacio de probabilidad y una sucesi\'on de variables aleatorias $\xi_{1},\xi_{2},\ldots$ definidas en este con distribuci\'on $F$. Entonces las otras cantidades son $T_{n}=\sum_{k=1}^{n}\xi_{k}$ y $N\left(t\right)=\sum_{n=1}^{\infty}\indora\left(T_{n}\leq t\right)$, donde $T_{n}\rightarrow\infty$ casi seguramente por la Ley Fuerte de los Grandes Números.
\end{Note}





%______________________________________________________________________
%\subsection*{Procesos de Renovaci\'on}
%______________________________________________________________________

\begin{Def}%\label{Def.Tn}
Sean $0\leq T_{1}\leq T_{2}\leq \ldots$ son tiempos aleatorios infinitos en los cuales ocurren ciertos eventos. El n\'umero de tiempos $T_{n}$ en el intervalo $\left[0,t\right)$ es

\begin{eqnarray}
N\left(t\right)=\sum_{n=1}^{\infty}\indora\left(T_{n}\leq t\right),
\end{eqnarray}
para $t\geq0$.
\end{Def}

Si se consideran los puntos $T_{n}$ como elementos de $\rea_{+}$, y $N\left(t\right)$ es el n\'umero de puntos en $\rea$. El proceso denotado por $\left\{N\left(t\right):t\geq0\right\}$, denotado por $N\left(t\right)$, es un proceso puntual en $\rea_{+}$. Los $T_{n}$ son los tiempos de ocurrencia, el proceso puntual $N\left(t\right)$ es simple si su n\'umero de ocurrencias son distintas: $0<T_{1}<T_{2}<\ldots$ casi seguramente.

\begin{Def}
Un proceso puntual $N\left(t\right)$ es un proceso de renovaci\'on si los tiempos de interocurrencia $\xi_{n}=T_{n}-T_{n-1}$, para $n\geq1$, son independientes e identicamente distribuidos con distribuci\'on $F$, donde $F\left(0\right)=0$ y $T_{0}=0$. Los $T_{n}$ son llamados tiempos de renovaci\'on, referente a la independencia o renovaci\'on de la informaci\'on estoc\'astica en estos tiempos. Los $\xi_{n}$ son los tiempos de inter-renovaci\'on, y $N\left(t\right)$ es el n\'umero de renovaciones en el intervalo $\left[0,t\right)$
\end{Def}


\begin{Note}
Para definir un proceso de renovaci\'on para cualquier contexto, solamente hay que especificar una distribuci\'on $F$, con $F\left(0\right)=0$, para los tiempos de inter-renovaci\'on. La funci\'on $F$ en turno degune las otra variables aleatorias. De manera formal, existe un espacio de probabilidad y una sucesi\'on de variables aleatorias $\xi_{1},\xi_{2},\ldots$ definidas en este con distribuci\'on $F$. Entonces las otras cantidades son $T_{n}=\sum_{k=1}^{n}\xi_{k}$ y $N\left(t\right)=\sum_{n=1}^{\infty}\indora\left(T_{n}\leq t\right)$, donde $T_{n}\rightarrow\infty$ casi seguramente por la Ley Fuerte de los Grandes Números.
\end{Note}


%______________________________________________________________________
%\subsection*{Procesos de Renovaci\'on}
%______________________________________________________________________

\begin{Def}%\label{Def.Tn}
Sean $0\leq T_{1}\leq T_{2}\leq \ldots$ son tiempos aleatorios infinitos en los cuales ocurren ciertos eventos. El n\'umero de tiempos $T_{n}$ en el intervalo $\left[0,t\right)$ es

\begin{eqnarray}
N\left(t\right)=\sum_{n=1}^{\infty}\indora\left(T_{n}\leq t\right),
\end{eqnarray}
para $t\geq0$.
\end{Def}

Si se consideran los puntos $T_{n}$ como elementos de $\rea_{+}$, y $N\left(t\right)$ es el n\'umero de puntos en $\rea$. El proceso denotado por $\left\{N\left(t\right):t\geq0\right\}$, denotado por $N\left(t\right)$, es un proceso puntual en $\rea_{+}$. Los $T_{n}$ son los tiempos de ocurrencia, el proceso puntual $N\left(t\right)$ es simple si su n\'umero de ocurrencias son distintas: $0<T_{1}<T_{2}<\ldots$ casi seguramente.

\begin{Def}
Un proceso puntual $N\left(t\right)$ es un proceso de renovaci\'on si los tiempos de interocurrencia $\xi_{n}=T_{n}-T_{n-1}$, para $n\geq1$, son independientes e identicamente distribuidos con distribuci\'on $F$, donde $F\left(0\right)=0$ y $T_{0}=0$. Los $T_{n}$ son llamados tiempos de renovaci\'on, referente a la independencia o renovaci\'on de la informaci\'on estoc\'astica en estos tiempos. Los $\xi_{n}$ son los tiempos de inter-renovaci\'on, y $N\left(t\right)$ es el n\'umero de renovaciones en el intervalo $\left[0,t\right)$
\end{Def}


\begin{Note}
Para definir un proceso de renovaci\'on para cualquier contexto, solamente hay que especificar una distribuci\'on $F$, con $F\left(0\right)=0$, para los tiempos de inter-renovaci\'on. La funci\'on $F$ en turno degune las otra variables aleatorias. De manera formal, existe un espacio de probabilidad y una sucesi\'on de variables aleatorias $\xi_{1},\xi_{2},\ldots$ definidas en este con distribuci\'on $F$. Entonces las otras cantidades son $T_{n}=\sum_{k=1}^{n}\xi_{k}$ y $N\left(t\right)=\sum_{n=1}^{\infty}\indora\left(T_{n}\leq t\right)$, donde $T_{n}\rightarrow\infty$ casi seguramente por la Ley Fuerte de los Grandes Números.
\end{Note}


%______________________________________________________________________
%\subsection*{Procesos de Renovaci\'on}
%______________________________________________________________________

\begin{Def}\label{Def.Tn}
Sean $0\leq T_{1}\leq T_{2}\leq \ldots$ son tiempos aleatorios infinitos en los cuales ocurren ciertos eventos. El n\'umero de tiempos $T_{n}$ en el intervalo $\left[0,t\right)$ es

\begin{eqnarray}
N\left(t\right)=\sum_{n=1}^{\infty}\indora\left(T_{n}\leq t\right),
\end{eqnarray}
para $t\geq0$.
\end{Def}

Si se consideran los puntos $T_{n}$ como elementos de $\rea_{+}$, y $N\left(t\right)$ es el n\'umero de puntos en $\rea$. El proceso denotado por $\left\{N\left(t\right):t\geq0\right\}$, denotado por $N\left(t\right)$, es un proceso puntual en $\rea_{+}$. Los $T_{n}$ son los tiempos de ocurrencia, el proceso puntual $N\left(t\right)$ es simple si su n\'umero de ocurrencias son distintas: $0<T_{1}<T_{2}<\ldots$ casi seguramente.

\begin{Def}
Un proceso puntual $N\left(t\right)$ es un proceso de renovaci\'on si los tiempos de interocurrencia $\xi_{n}=T_{n}-T_{n-1}$, para $n\geq1$, son independientes e identicamente distribuidos con distribuci\'on $F$, donde $F\left(0\right)=0$ y $T_{0}=0$. Los $T_{n}$ son llamados tiempos de renovaci\'on, referente a la independencia o renovaci\'on de la informaci\'on estoc\'astica en estos tiempos. Los $\xi_{n}$ son los tiempos de inter-renovaci\'on, y $N\left(t\right)$ es el n\'umero de renovaciones en el intervalo $\left[0,t\right)$
\end{Def}


\begin{Note}
Para definir un proceso de renovaci\'on para cualquier contexto, solamente hay que especificar una distribuci\'on $F$, con $F\left(0\right)=0$, para los tiempos de inter-renovaci\'on. La funci\'on $F$ en turno degune las otra variables aleatorias. De manera formal, existe un espacio de probabilidad y una sucesi\'on de variables aleatorias $\xi_{1},\xi_{2},\ldots$ definidas en este con distribuci\'on $F$. Entonces las otras cantidades son $T_{n}=\sum_{k=1}^{n}\xi_{k}$ y $N\left(t\right)=\sum_{n=1}^{\infty}\indora\left(T_{n}\leq t\right)$, donde $T_{n}\rightarrow\infty$ casi seguramente por la Ley Fuerte de los Grandes Números.
\end{Note}


%______________________________________________________________________
%\subsection*{Procesos de Renovaci\'on}
%______________________________________________________________________

\begin{Def}\label{Def.Tn}
Sean $0\leq T_{1}\leq T_{2}\leq \ldots$ son tiempos aleatorios infinitos en los cuales ocurren ciertos eventos. El n\'umero de tiempos $T_{n}$ en el intervalo $\left[0,t\right)$ es

\begin{eqnarray}
N\left(t\right)=\sum_{n=1}^{\infty}\indora\left(T_{n}\leq t\right),
\end{eqnarray}
para $t\geq0$.
\end{Def}

Si se consideran los puntos $T_{n}$ como elementos de $\rea_{+}$, y $N\left(t\right)$ es el n\'umero de puntos en $\rea$. El proceso denotado por $\left\{N\left(t\right):t\geq0\right\}$, denotado por $N\left(t\right)$, es un proceso puntual en $\rea_{+}$. Los $T_{n}$ son los tiempos de ocurrencia, el proceso puntual $N\left(t\right)$ es simple si su n\'umero de ocurrencias son distintas: $0<T_{1}<T_{2}<\ldots$ casi seguramente.

\begin{Def}
Un proceso puntual $N\left(t\right)$ es un proceso de renovaci\'on si los tiempos de interocurrencia $\xi_{n}=T_{n}-T_{n-1}$, para $n\geq1$, son independientes e identicamente distribuidos con distribuci\'on $F$, donde $F\left(0\right)=0$ y $T_{0}=0$. Los $T_{n}$ son llamados tiempos de renovaci\'on, referente a la independencia o renovaci\'on de la informaci\'on estoc\'astica en estos tiempos. Los $\xi_{n}$ son los tiempos de inter-renovaci\'on, y $N\left(t\right)$ es el n\'umero de renovaciones en el intervalo $\left[0,t\right)$
\end{Def}


\begin{Note}
Para definir un proceso de renovaci\'on para cualquier contexto, solamente hay que especificar una distribuci\'on $F$, con $F\left(0\right)=0$, para los tiempos de inter-renovaci\'on. La funci\'on $F$ en turno degune las otra variables aleatorias. De manera formal, existe un espacio de probabilidad y una sucesi\'on de variables aleatorias $\xi_{1},\xi_{2},\ldots$ definidas en este con distribuci\'on $F$. Entonces las otras cantidades son $T_{n}=\sum_{k=1}^{n}\xi_{k}$ y $N\left(t\right)=\sum_{n=1}^{\infty}\indora\left(T_{n}\leq t\right)$, donde $T_{n}\rightarrow\infty$ casi seguramente por la Ley Fuerte de los Grandes Números.
\end{Note}
%___________________________________________________________________________________________
%
\section{Renewal and Regenerative Processes: Serfozo\cite{Serfozo}}\label{Appendix.E}
%___________________________________________________________________________________________
%
\begin{Def}\label{Def.Tn}
Sean $0\leq T_{1}\leq T_{2}\leq \ldots$ son tiempos aleatorios infinitos en los cuales ocurren ciertos eventos. El n\'umero de tiempos $T_{n}$ en el intervalo $\left[0,t\right)$ es

\begin{eqnarray}
N\left(t\right)=\sum_{n=1}^{\infty}\indora\left(T_{n}\leq t\right),
\end{eqnarray}
para $t\geq0$.
\end{Def}

Si se consideran los puntos $T_{n}$ como elementos de $\rea_{+}$, y $N\left(t\right)$ es el n\'umero de puntos en $\rea$. El proceso denotado por $\left\{N\left(t\right):t\geq0\right\}$, denotado por $N\left(t\right)$, es un proceso puntual en $\rea_{+}$. Los $T_{n}$ son los tiempos de ocurrencia, el proceso puntual $N\left(t\right)$ es simple si su n\'umero de ocurrencias son distintas: $0<T_{1}<T_{2}<\ldots$ casi seguramente.

\begin{Def}
Un proceso puntual $N\left(t\right)$ es un proceso de renovaci\'on si los tiempos de interocurrencia $\xi_{n}=T_{n}-T_{n-1}$, para $n\geq1$, son independientes e identicamente distribuidos con distribuci\'on $F$, donde $F\left(0\right)=0$ y $T_{0}=0$. Los $T_{n}$ son llamados tiempos de renovaci\'on, referente a la independencia o renovaci\'on de la informaci\'on estoc\'astica en estos tiempos. Los $\xi_{n}$ son los tiempos de inter-renovaci\'on, y $N\left(t\right)$ es el n\'umero de renovaciones en el intervalo $\left[0,t\right)$
\end{Def}


\begin{Note}
Para definir un proceso de renovaci\'on para cualquier contexto, solamente hay que especificar una distribuci\'on $F$, con $F\left(0\right)=0$, para los tiempos de inter-renovaci\'on. La funci\'on $F$ en turno degune las otra variables aleatorias. De manera formal, existe un espacio de probabilidad y una sucesi\'on de variables aleatorias $\xi_{1},\xi_{2},\ldots$ definidas en este con distribuci\'on $F$. Entonces las otras cantidades son $T_{n}=\sum_{k=1}^{n}\xi_{k}$ y $N\left(t\right)=\sum_{n=1}^{\infty}\indora\left(T_{n}\leq t\right)$, donde $T_{n}\rightarrow\infty$ casi seguramente por la Ley Fuerte de los Grandes N\'umeros.
\end{Note}







Los tiempos $T_{n}$ est\'an relacionados con los conteos de $N\left(t\right)$ por

\begin{eqnarray*}
\left\{N\left(t\right)\geq n\right\}&=&\left\{T_{n}\leq t\right\}\\
T_{N\left(t\right)}\leq &t&<T_{N\left(t\right)+1},
\end{eqnarray*}

adem\'as $N\left(T_{n}\right)=n$, y 

\begin{eqnarray*}
N\left(t\right)=\max\left\{n:T_{n}\leq t\right\}=\min\left\{n:T_{n+1}>t\right\}
\end{eqnarray*}

Por propiedades de la convoluci\'on se sabe que

\begin{eqnarray*}
P\left\{T_{n}\leq t\right\}=F^{n\star}\left(t\right)
\end{eqnarray*}
que es la $n$-\'esima convoluci\'on de $F$. Entonces 

\begin{eqnarray*}
\left\{N\left(t\right)\geq n\right\}&=&\left\{T_{n}\leq t\right\}\\
P\left\{N\left(t\right)\leq n\right\}&=&1-F^{\left(n+1\right)\star}\left(t\right)
\end{eqnarray*}

Adem\'as usando el hecho de que $\esp\left[N\left(t\right)\right]=\sum_{n=1}^{\infty}P\left\{N\left(t\right)\geq n\right\}$
se tiene que

\begin{eqnarray*}
\esp\left[N\left(t\right)\right]=\sum_{n=1}^{\infty}F^{n\star}\left(t\right)
\end{eqnarray*}

\begin{Prop}
Para cada $t\geq0$, la funci\'on generadora de momentos $\esp\left[e^{\alpha N\left(t\right)}\right]$ existe para alguna $\alpha$ en una vecindad del 0, y de aqu\'i que $\esp\left[N\left(t\right)^{m}\right]<\infty$, para $m\geq1$.
\end{Prop}

\begin{Ejem}[\textbf{Proceso Poisson}]

Suponga que se tienen tiempos de inter-renovaci\'on \textit{i.i.d.} del proceso de renovaci\'on $N\left(t\right)$ tienen distribuci\'on exponencial $F\left(t\right)=q-e^{-\lambda t}$ con tasa $\lambda$. Entonces $N\left(t\right)$ es un proceso Poisson con tasa $\lambda$.

\end{Ejem}


\begin{Note}
Si el primer tiempo de renovaci\'on $\xi_{1}$ no tiene la misma distribuci\'on que el resto de las $\xi_{n}$, para $n\geq2$, a $N\left(t\right)$ se le llama Proceso de Renovaci\'on retardado, donde si $\xi$ tiene distribuci\'on $G$, entonces el tiempo $T_{n}$ de la $n$-\'esima renovaci\'on tiene distribuci\'on $G\star F^{\left(n-1\right)\star}\left(t\right)$
\end{Note}


\begin{Teo}
Para una constante $\mu\leq\infty$ ( o variable aleatoria), las siguientes expresiones son equivalentes:

\begin{eqnarray}
lim_{n\rightarrow\infty}n^{-1}T_{n}&=&\mu,\textrm{ c.s.}\\
lim_{t\rightarrow\infty}t^{-1}N\left(t\right)&=&1/\mu,\textrm{ c.s.}
\end{eqnarray}
\end{Teo}


Es decir, $T_{n}$ satisface la Ley Fuerte de los Grandes N\'umeros s\'i y s\'olo s\'i $N\left/t\right)$ la cumple.


\begin{Coro}[Ley Fuerte de los Grandes N\'umeros para Procesos de Renovaci\'on]
Si $N\left(t\right)$ es un proceso de renovaci\'on cuyos tiempos de inter-renovaci\'on tienen media $\mu\leq\infty$, entonces
\begin{eqnarray}
t^{-1}N\left(t\right)\rightarrow 1/\mu,\textrm{ c.s. cuando }t\rightarrow\infty.
\end{eqnarray}

\end{Coro}


Considerar el proceso estoc\'astico de valores reales $\left\{Z\left(t\right):t\geq0\right\}$ en el mismo espacio de probabilidad que $N\left(t\right)$

\begin{Def}
Para el proceso $\left\{Z\left(t\right):t\geq0\right\}$ se define la fluctuaci\'on m\'axima de $Z\left(t\right)$ en el intervalo $\left(T_{n-1},T_{n}\right]$:
\begin{eqnarray*}
M_{n}=\sup_{T_{n-1}<t\leq T_{n}}|Z\left(t\right)-Z\left(T_{n-1}\right)|
\end{eqnarray*}
\end{Def}

\begin{Teo}
Sup\'ongase que $n^{-1}T_{n}\rightarrow\mu$ c.s. cuando $n\rightarrow\infty$, donde $\mu\leq\infty$ es una constante o variable aleatoria. Sea $a$ una constante o variable aleatoria que puede ser infinita cuando $\mu$ es finita, y considere las expresiones l\'imite:
\begin{eqnarray}
lim_{n\rightarrow\infty}n^{-1}Z\left(T_{n}\right)&=&a,\textrm{ c.s.}\\
lim_{t\rightarrow\infty}t^{-1}Z\left(t\right)&=&a/\mu,\textrm{ c.s.}
\end{eqnarray}
La segunda expresi\'on implica la primera. Conversamente, la primera implica la segunda si el proceso $Z\left(t\right)$ es creciente, o si $lim_{n\rightarrow\infty}n^{-1}M_{n}=0$ c.s.
\end{Teo}

\begin{Coro}
Si $N\left(t\right)$ es un proceso de renovaci\'on, y $\left(Z\left(T_{n}\right)-Z\left(T_{n-1}\right),M_{n}\right)$, para $n\geq1$, son variables aleatorias independientes e id\'enticamente distribuidas con media finita, entonces,
\begin{eqnarray}
lim_{t\rightarrow\infty}t^{-1}Z\left(t\right)\rightarrow\frac{\esp\left[Z\left(T_{1}\right)-Z\left(T_{0}\right)\right]}{\esp\left[T_{1}\right]},\textrm{ c.s. cuando  }t\rightarrow\infty.
\end{eqnarray}
\end{Coro}


Sup\'ongase que $N\left(t\right)$ es un proceso de renovaci\'on con distribuci\'on $F$ con media finita $\mu$.

\begin{Def}
La funci\'on de renovaci\'on asociada con la distribuci\'on $F$, del proceso $N\left(t\right)$, es
\begin{eqnarray*}
U\left(t\right)=\sum_{n=1}^{\infty}F^{n\star}\left(t\right),\textrm{   }t\geq0,
\end{eqnarray*}
donde $F^{0\star}\left(t\right)=\indora\left(t\geq0\right)$.
\end{Def}


\begin{Prop}
Sup\'ongase que la distribuci\'on de inter-renovaci\'on $F$ tiene densidad $f$. Entonces $U\left(t\right)$ tambi\'en tiene densidad, para $t>0$, y es $U^{'}\left(t\right)=\sum_{n=0}^{\infty}f^{n\star}\left(t\right)$. Adem\'as
\begin{eqnarray*}
\prob\left\{N\left(t\right)>N\left(t-\right)\right\}=0\textrm{,   }t\geq0.
\end{eqnarray*}
\end{Prop}

\begin{Def}
La Transformada de Laplace-Stieljes de $F$ est\'a dada por

\begin{eqnarray*}
\hat{F}\left(\alpha\right)=\int_{\rea_{+}}e^{-\alpha t}dF\left(t\right)\textrm{,  }\alpha\geq0.
\end{eqnarray*}
\end{Def}

Entonces

\begin{eqnarray*}
\hat{U}\left(\alpha\right)=\sum_{n=0}^{\infty}\hat{F^{n\star}}\left(\alpha\right)=\sum_{n=0}^{\infty}\hat{F}\left(\alpha\right)^{n}=\frac{1}{1-\hat{F}\left(\alpha\right)}.
\end{eqnarray*}


\begin{Prop}
La Transformada de Laplace $\hat{U}\left(\alpha\right)$ y $\hat{F}\left(\alpha\right)$ determina una a la otra de manera \'unica por la relaci\'on $\hat{U}\left(\alpha\right)=\frac{1}{1-\hat{F}\left(\alpha\right)}$.
\end{Prop}


\begin{Note}
Un proceso de renovaci\'on $N\left(t\right)$ cuyos tiempos de inter-renovaci\'on tienen media finita, es un proceso Poisson con tasa $\lambda$ si y s\'olo s\'i $\esp\left[U\left(t\right)\right]=\lambda t$, para $t\geq0$.
\end{Note}


\begin{Teo}
Sea $N\left(t\right)$ un proceso puntual simple con puntos de localizaci\'on $T_{n}$ tal que $\eta\left(t\right)=\esp\left[N\left(\right)\right]$ es finita para cada $t$. Entonces para cualquier funci\'on $f:\rea_{+}\rightarrow\rea$,
\begin{eqnarray*}
\esp\left[\sum_{n=1}^{N\left(\right)}f\left(T_{n}\right)\right]=\int_{\left(0,t\right]}f\left(s\right)d\eta\left(s\right)\textrm{,  }t\geq0,
\end{eqnarray*}
suponiendo que la integral exista. Adem\'as si $X_{1},X_{2},\ldots$ son variables aleatorias definidas en el mismo espacio de probabilidad que el proceso $N\left(t\right)$ tal que $\esp\left[X_{n}|T_{n}=s\right]=f\left(s\right)$, independiente de $n$. Entonces
\begin{eqnarray*}
\esp\left[\sum_{n=1}^{N\left(t\right)}X_{n}\right]=\int_{\left(0,t\right]}f\left(s\right)d\eta\left(s\right)\textrm{,  }t\geq0,
\end{eqnarray*} 
suponiendo que la integral exista. 
\end{Teo}

\begin{Coro}[Identidad de Wald para Renovaciones]
Para el proceso de renovaci\'on $N\left(t\right)$,
\begin{eqnarray*}
\esp\left[T_{N\left(t\right)+1}\right]=\mu\esp\left[N\left(t\right)+1\right]\textrm{,  }t\geq0,
\end{eqnarray*}  
\end{Coro}


\begin{Def}
Sea $h\left(t\right)$ funci\'on de valores reales en $\rea$ acotada en intervalos finitos e igual a cero para $t<0$ La ecuaci\'on de renovaci\'on para $h\left(t\right)$ y la distribuci\'on $F$ es

\begin{eqnarray}\label{Ec.Renovacion}
H\left(t\right)=h\left(t\right)+\int_{\left[0,t\right]}H\left(t-s\right)dF\left(s\right)\textrm{,    }t\geq0,
\end{eqnarray}
donde $H\left(t\right)$ es una funci\'on de valores reales. Esto es $H=h+F\star H$. Decimos que $H\left(t\right)$ es soluci\'on de esta ecuaci\'on si satisface la ecuaci\'on, y es acotada en intervalos finitos e iguales a cero para $t<0$.
\end{Def}

\begin{Prop}
La funci\'on $U\star h\left(t\right)$ es la \'unica soluci\'on de la ecuaci\'on de renovaci\'on (\ref{Ec.Renovacion}).
\end{Prop}

\begin{Teo}[Teorema Renovaci\'on Elemental]
\begin{eqnarray*}
t^{-1}U\left(t\right)\rightarrow 1/\mu\textrm{,    cuando }t\rightarrow\infty.
\end{eqnarray*}
\end{Teo}



Sup\'ongase que $N\left(t\right)$ es un proceso de renovaci\'on con distribuci\'on $F$ con media finita $\mu$.

\begin{Def}
La funci\'on de renovaci\'on asociada con la distribuci\'on $F$, del proceso $N\left(t\right)$, es
\begin{eqnarray*}
U\left(t\right)=\sum_{n=1}^{\infty}F^{n\star}\left(t\right),\textrm{   }t\geq0,
\end{eqnarray*}
donde $F^{0\star}\left(t\right)=\indora\left(t\geq0\right)$.
\end{Def}


\begin{Prop}
Sup\'ongase que la distribuci\'on de inter-renovaci\'on $F$ tiene densidad $f$. Entonces $U\left(t\right)$ tambi\'en tiene densidad, para $t>0$, y es $U^{'}\left(t\right)=\sum_{n=0}^{\infty}f^{n\star}\left(t\right)$. Adem\'as
\begin{eqnarray*}
\prob\left\{N\left(t\right)>N\left(t-\right)\right\}=0\textrm{,   }t\geq0.
\end{eqnarray*}
\end{Prop}

\begin{Def}
La Transformada de Laplace-Stieljes de $F$ est\'a dada por

\begin{eqnarray*}
\hat{F}\left(\alpha\right)=\int_{\rea_{+}}e^{-\alpha t}dF\left(t\right)\textrm{,  }\alpha\geq0.
\end{eqnarray*}
\end{Def}

Entonces

\begin{eqnarray*}
\hat{U}\left(\alpha\right)=\sum_{n=0}^{\infty}\hat{F^{n\star}}\left(\alpha\right)=\sum_{n=0}^{\infty}\hat{F}\left(\alpha\right)^{n}=\frac{1}{1-\hat{F}\left(\alpha\right)}.
\end{eqnarray*}


\begin{Prop}
La Transformada de Laplace $\hat{U}\left(\alpha\right)$ y $\hat{F}\left(\alpha\right)$ determina una a la otra de manera \'unica por la relaci\'on $\hat{U}\left(\alpha\right)=\frac{1}{1-\hat{F}\left(\alpha\right)}$.
\end{Prop}


\begin{Note}
Un proceso de renovaci\'on $N\left(t\right)$ cuyos tiempos de inter-renovaci\'on tienen media finita, es un proceso Poisson con tasa $\lambda$ si y s\'olo s\'i $\esp\left[U\left(t\right)\right]=\lambda t$, para $t\geq0$.
\end{Note}


\begin{Teo}
Sea $N\left(t\right)$ un proceso puntual simple con puntos de localizaci\'on $T_{n}$ tal que $\eta\left(t\right)=\esp\left[N\left(\right)\right]$ es finita para cada $t$. Entonces para cualquier funci\'on $f:\rea_{+}\rightarrow\rea$,
\begin{eqnarray*}
\esp\left[\sum_{n=1}^{N\left(\right)}f\left(T_{n}\right)\right]=\int_{\left(0,t\right]}f\left(s\right)d\eta\left(s\right)\textrm{,  }t\geq0,
\end{eqnarray*}
suponiendo que la integral exista. Adem\'as si $X_{1},X_{2},\ldots$ son variables aleatorias definidas en el mismo espacio de probabilidad que el proceso $N\left(t\right)$ tal que $\esp\left[X_{n}|T_{n}=s\right]=f\left(s\right)$, independiente de $n$. Entonces
\begin{eqnarray*}
\esp\left[\sum_{n=1}^{N\left(t\right)}X_{n}\right]=\int_{\left(0,t\right]}f\left(s\right)d\eta\left(s\right)\textrm{,  }t\geq0,
\end{eqnarray*} 
suponiendo que la integral exista. 
\end{Teo}

\begin{Coro}[Identidad de Wald para Renovaciones]
Para el proceso de renovaci\'on $N\left(t\right)$,
\begin{eqnarray*}
\esp\left[T_{N\left(t\right)+1}\right]=\mu\esp\left[N\left(t\right)+1\right]\textrm{,  }t\geq0,
\end{eqnarray*}  
\end{Coro}


\begin{Def}
Sea $h\left(t\right)$ funci\'on de valores reales en $\rea$ acotada en intervalos finitos e igual a cero para $t<0$ La ecuaci\'on de renovaci\'on para $h\left(t\right)$ y la distribuci\'on $F$ es

\begin{eqnarray}\label{Ec.Renovacion}
H\left(t\right)=h\left(t\right)+\int_{\left[0,t\right]}H\left(t-s\right)dF\left(s\right)\textrm{,    }t\geq0,
\end{eqnarray}
donde $H\left(t\right)$ es una funci\'on de valores reales. Esto es $H=h+F\star H$. Decimos que $H\left(t\right)$ es soluci\'on de esta ecuaci\'on si satisface la ecuaci\'on, y es acotada en intervalos finitos e iguales a cero para $t<0$.
\end{Def}

\begin{Prop}
La funci\'on $U\star h\left(t\right)$ es la \'unica soluci\'on de la ecuaci\'on de renovaci\'on (\ref{Ec.Renovacion}).
\end{Prop}

\begin{Teo}[Teorema Renovaci\'on Elemental]
\begin{eqnarray*}
t^{-1}U\left(t\right)\rightarrow 1/\mu\textrm{,    cuando }t\rightarrow\infty.
\end{eqnarray*}
\end{Teo}


\begin{Note} Una funci\'on $h:\rea_{+}\rightarrow\rea$ es Directamente Riemann Integrable en los siguientes casos:
\begin{itemize}
\item[a)] $h\left(t\right)\geq0$ es decreciente y Riemann Integrable.
\item[b)] $h$ es continua excepto posiblemente en un conjunto de Lebesgue de medida 0, y $|h\left(t\right)|\leq b\left(t\right)$, donde $b$ es DRI.
\end{itemize}
\end{Note}

\begin{Teo}[Teorema Principal de Renovaci\'on]
Si $F$ es no aritm\'etica y $h\left(t\right)$ es Directamente Riemann Integrable (DRI), entonces

\begin{eqnarray*}
lim_{t\rightarrow\infty}U\star h=\frac{1}{\mu}\int_{\rea_{+}}h\left(s\right)ds.
\end{eqnarray*}
\end{Teo}

\begin{Prop}
Cualquier funci\'on $H\left(t\right)$ acotada en intervalos finitos y que es 0 para $t<0$ puede expresarse como
\begin{eqnarray*}
H\left(t\right)=U\star h\left(t\right)\textrm{,  donde }h\left(t\right)=H\left(t\right)-F\star H\left(t\right)
\end{eqnarray*}
\end{Prop}

\begin{Def}
Un proceso estoc\'astico $X\left(t\right)$ es crudamente regenerativo en un tiempo aleatorio positivo $T$ si
\begin{eqnarray*}
\esp\left[X\left(T+t\right)|T\right]=\esp\left[X\left(t\right)\right]\textrm{, para }t\geq0,\end{eqnarray*}
y con las esperanzas anteriores finitas.
\end{Def}

\begin{Prop}
Sup\'ongase que $X\left(t\right)$ es un proceso crudamente regenerativo en $T$, que tiene distribuci\'on $F$. Si $\esp\left[X\left(t\right)\right]$ es acotado en intervalos finitos, entonces
\begin{eqnarray*}
\esp\left[X\left(t\right)\right]=U\star h\left(t\right)\textrm{,  donde }h\left(t\right)=\esp\left[X\left(t\right)\indora\left(T>t\right)\right].
\end{eqnarray*}
\end{Prop}

\begin{Teo}[Regeneraci\'on Cruda]
Sup\'ongase que $X\left(t\right)$ es un proceso con valores positivo crudamente regenerativo en $T$, y def\'inase $M=\sup\left\{|X\left(t\right)|:t\leq T\right\}$. Si $T$ es no aritm\'etico y $M$ y $MT$ tienen media finita, entonces
\begin{eqnarray*}
lim_{t\rightarrow\infty}\esp\left[X\left(t\right)\right]=\frac{1}{\mu}\int_{\rea_{+}}h\left(s\right)ds,
\end{eqnarray*}
donde $h\left(t\right)=\esp\left[X\left(t\right)\indora\left(T>t\right)\right]$.
\end{Teo}


\begin{Note} Una funci\'on $h:\rea_{+}\rightarrow\rea$ es Directamente Riemann Integrable en los siguientes casos:
\begin{itemize}
\item[a)] $h\left(t\right)\geq0$ es decreciente y Riemann Integrable.
\item[b)] $h$ es continua excepto posiblemente en un conjunto de Lebesgue de medida 0, y $|h\left(t\right)|\leq b\left(t\right)$, donde $b$ es DRI.
\end{itemize}
\end{Note}

\begin{Teo}[Teorema Principal de Renovaci\'on]
Si $F$ es no aritm\'etica y $h\left(t\right)$ es Directamente Riemann Integrable (DRI), entonces

\begin{eqnarray*}
lim_{t\rightarrow\infty}U\star h=\frac{1}{\mu}\int_{\rea_{+}}h\left(s\right)ds.
\end{eqnarray*}
\end{Teo}

\begin{Prop}
Cualquier funci\'on $H\left(t\right)$ acotada en intervalos finitos y que es 0 para $t<0$ puede expresarse como
\begin{eqnarray*}
H\left(t\right)=U\star h\left(t\right)\textrm{,  donde }h\left(t\right)=H\left(t\right)-F\star H\left(t\right)
\end{eqnarray*}
\end{Prop}

\begin{Def}
Un proceso estoc\'astico $X\left(t\right)$ es crudamente regenerativo en un tiempo aleatorio positivo $T$ si
\begin{eqnarray*}
\esp\left[X\left(T+t\right)|T\right]=\esp\left[X\left(t\right)\right]\textrm{, para }t\geq0,\end{eqnarray*}
y con las esperanzas anteriores finitas.
\end{Def}

\begin{Prop}
Sup\'ongase que $X\left(t\right)$ es un proceso crudamente regenerativo en $T$, que tiene distribuci\'on $F$. Si $\esp\left[X\left(t\right)\right]$ es acotado en intervalos finitos, entonces
\begin{eqnarray*}
\esp\left[X\left(t\right)\right]=U\star h\left(t\right)\textrm{,  donde }h\left(t\right)=\esp\left[X\left(t\right)\indora\left(T>t\right)\right].
\end{eqnarray*}
\end{Prop}

\begin{Teo}[Regeneraci\'on Cruda]
Sup\'ongase que $X\left(t\right)$ es un proceso con valores positivo crudamente regenerativo en $T$, y def\'inase $M=\sup\left\{|X\left(t\right)|:t\leq T\right\}$. Si $T$ es no aritm\'etico y $M$ y $MT$ tienen media finita, entonces
\begin{eqnarray*}
lim_{t\rightarrow\infty}\esp\left[X\left(t\right)\right]=\frac{1}{\mu}\int_{\rea_{+}}h\left(s\right)ds,
\end{eqnarray*}
donde $h\left(t\right)=\esp\left[X\left(t\right)\indora\left(T>t\right)\right]$.
\end{Teo}

\begin{Def}
Para el proceso $\left\{\left(N\left(t\right),X\left(t\right)\right):t\geq0\right\}$, sus trayectoria muestrales en el intervalo de tiempo $\left[T_{n-1},T_{n}\right)$ est\'an descritas por
\begin{eqnarray*}
\zeta_{n}=\left(\xi_{n},\left\{X\left(T_{n-1}+t\right):0\leq t<\xi_{n}\right\}\right)
\end{eqnarray*}
Este $\zeta_{n}$ es el $n$-\'esimo segmento del proceso. El proceso es regenerativo sobre los tiempos $T_{n}$ si sus segmentos $\zeta_{n}$ son independientes e id\'enticamennte distribuidos.
\end{Def}


\begin{Note}
Si $\tilde{X}\left(t\right)$ con espacio de estados $\tilde{S}$ es regenerativo sobre $T_{n}$, entonces $X\left(t\right)=f\left(\tilde{X}\left(t\right)\right)$ tambi\'en es regenerativo sobre $T_{n}$, para cualquier funci\'on $f:\tilde{S}\rightarrow S$.
\end{Note}

\begin{Note}
Los procesos regenerativos son crudamente regenerativos, pero no al rev\'es.
\end{Note}


\begin{Note}
Un proceso estoc\'astico a tiempo continuo o discreto es regenerativo si existe un proceso de renovaci\'on  tal que los segmentos del proceso entre tiempos de renovaci\'on sucesivos son i.i.d., es decir, para $\left\{X\left(t\right):t\geq0\right\}$ proceso estoc\'astico a tiempo continuo con espacio de estados $S$, espacio m\'etrico.
\end{Note}

Para $\left\{X\left(t\right):t\geq0\right\}$ Proceso Estoc\'astico a tiempo continuo con estado de espacios $S$, que es un espacio m\'etrico, con trayectorias continuas por la derecha y con l\'imites por la izquierda c.s. Sea $N\left(t\right)$ un proceso de renovaci\'on en $\rea_{+}$ definido en el mismo espacio de probabilidad que $X\left(t\right)$, con tiempos de renovaci\'on $T$ y tiempos de inter-renovaci\'on $\xi_{n}=T_{n}-T_{n-1}$, con misma distribuci\'on $F$ de media finita $\mu$.



\begin{Def}
Para el proceso $\left\{\left(N\left(t\right),X\left(t\right)\right):t\geq0\right\}$, sus trayectoria muestrales en el intervalo de tiempo $\left[T_{n-1},T_{n}\right)$ est\'an descritas por
\begin{eqnarray*}
\zeta_{n}=\left(\xi_{n},\left\{X\left(T_{n-1}+t\right):0\leq t<\xi_{n}\right\}\right)
\end{eqnarray*}
Este $\zeta_{n}$ es el $n$-\'esimo segmento del proceso. El proceso es regenerativo sobre los tiempos $T_{n}$ si sus segmentos $\zeta_{n}$ son independientes e id\'enticamennte distribuidos.
\end{Def}

\begin{Note}
Un proceso regenerativo con media de la longitud de ciclo finita es llamado positivo recurrente.
\end{Note}

\begin{Teo}[Procesos Regenerativos]
Suponga que el proceso
\end{Teo}


\begin{Def}[Renewal Process Trinity]
Para un proceso de renovaci\'on $N\left(t\right)$, los siguientes procesos proveen de informaci\'on sobre los tiempos de renovaci\'on.
\begin{itemize}
\item $A\left(t\right)=t-T_{N\left(t\right)}$, el tiempo de recurrencia hacia atr\'as al tiempo $t$, que es el tiempo desde la \'ultima renovaci\'on para $t$.

\item $B\left(t\right)=T_{N\left(t\right)+1}-t$, el tiempo de recurrencia hacia adelante al tiempo $t$, residual del tiempo de renovaci\'on, que es el tiempo para la pr\'oxima renovaci\'on despu\'es de $t$.

\item $L\left(t\right)=\xi_{N\left(t\right)+1}=A\left(t\right)+B\left(t\right)$, la longitud del intervalo de renovaci\'on que contiene a $t$.
\end{itemize}
\end{Def}

\begin{Note}
El proceso tridimensional $\left(A\left(t\right),B\left(t\right),L\left(t\right)\right)$ es regenerativo sobre $T_{n}$, y por ende cada proceso lo es. Cada proceso $A\left(t\right)$ y $B\left(t\right)$ son procesos de MArkov a tiempo continuo con trayectorias continuas por partes en el espacio de estados $\rea_{+}$. Una expresi\'on conveniente para su distribuci\'on conjunta es, para $0\leq x<t,y\geq0$
\begin{equation}\label{NoRenovacion}
P\left\{A\left(t\right)>x,B\left(t\right)>y\right\}=
P\left\{N\left(t+y\right)-N\left((t-x)\right)=0\right\}
\end{equation}
\end{Note}

\begin{Ejem}[Tiempos de recurrencia Poisson]
Si $N\left(t\right)$ es un proceso Poisson con tasa $\lambda$, entonces de la expresi\'on (\ref{NoRenovacion}) se tiene que

\begin{eqnarray*}
\begin{array}{lc}
P\left\{A\left(t\right)>x,B\left(t\right)>y\right\}=e^{-\lambda\left(x+y\right)},&0\leq x<t,y\geq0,
\end{array}
\end{eqnarray*}
que es la probabilidad Poisson de no renovaciones en un intervalo de longitud $x+y$.

\end{Ejem}

\begin{Note}
Una cadena de Markov erg\'odica tiene la propiedad de ser estacionaria si la distribución de su estado al tiempo $0$ es su distribuci\'on estacionaria.
\end{Note}


\begin{Def}
Un proceso estoc\'astico a tiempo continuo $\left\{X\left(t\right):t\geq0\right\}$ en un espacio general es estacionario si sus distribuciones finito dimensionales son invariantes bajo cualquier  traslado: para cada $0\leq s_{1}<s_{2}<\cdots<s_{k}$ y $t\geq0$,
\begin{eqnarray*}
\left(X\left(s_{1}+t\right),\ldots,X\left(s_{k}+t\right)\right)=_{d}\left(X\left(s_{1}\right),\ldots,X\left(s_{k}\right)\right).
\end{eqnarray*}
\end{Def}

\begin{Note}
Un proceso de Markov es estacionario si $X\left(t\right)=_{d}X\left(0\right)$, $t\geq0$.
\end{Note}

Considerese el proceso $N\left(t\right)=\sum_{n}\indora\left(\tau_{n}\leq t\right)$ en $\rea_{+}$, con puntos $0<\tau_{1}<\tau_{2}<\cdots$.

\begin{Prop}
Si $N$ es un proceso puntual estacionario y $\esp\left[N\left(1\right)\right]<\infty$, entonces $\esp\left[N\left(t\right)\right]=t\esp\left[N\left(1\right)\right]$, $t\geq0$

\end{Prop}

\begin{Teo}
Los siguientes enunciados son equivalentes
\begin{itemize}
\item[i)] El proceso retardado de renovaci\'on $N$ es estacionario.

\item[ii)] EL proceso de tiempos de recurrencia hacia adelante $B\left(t\right)$ es estacionario.


\item[iii)] $\esp\left[N\left(t\right)\right]=t/\mu$,


\item[iv)] $G\left(t\right)=F_{e}\left(t\right)=\frac{1}{\mu}\int_{0}^{t}\left[1-F\left(s\right)\right]ds$
\end{itemize}
Cuando estos enunciados son ciertos, $P\left\{B\left(t\right)\leq x\right\}=F_{e}\left(x\right)$, para $t,x\geq0$.

\end{Teo}

\begin{Note}
Una consecuencia del teorema anterior es que el Proceso Poisson es el \'unico proceso sin retardo que es estacionario.
\end{Note}

\begin{Coro}
El proceso de renovaci\'on $N\left(t\right)$ sin retardo, y cuyos tiempos de inter renonaci\'on tienen media finita, es estacionario si y s\'olo si es un proceso Poisson.

\end{Coro}





%___________________________________________________________________________________________
%
%\subsection*{Renewal and Regenerative Processes: Serfozo\cite{Serfozo}}
%___________________________________________________________________________________________
%
\begin{Def}%\label{Def.Tn}
Sean $0\leq T_{1}\leq T_{2}\leq \ldots$ son tiempos aleatorios infinitos en los cuales ocurren ciertos eventos. El n\'umero de tiempos $T_{n}$ en el intervalo $\left[0,t\right)$ es

\begin{eqnarray}
N\left(t\right)=\sum_{n=1}^{\infty}\indora\left(T_{n}\leq t\right),
\end{eqnarray}
para $t\geq0$.
\end{Def}

Si se consideran los puntos $T_{n}$ como elementos de $\rea_{+}$, y $N\left(t\right)$ es el n\'umero de puntos en $\rea$. El proceso denotado por $\left\{N\left(t\right):t\geq0\right\}$, denotado por $N\left(t\right)$, es un proceso puntual en $\rea_{+}$. Los $T_{n}$ son los tiempos de ocurrencia, el proceso puntual $N\left(t\right)$ es simple si su n\'umero de ocurrencias son distintas: $0<T_{1}<T_{2}<\ldots$ casi seguramente.

\begin{Def}
Un proceso puntual $N\left(t\right)$ es un proceso de renovaci\'on si los tiempos de interocurrencia $\xi_{n}=T_{n}-T_{n-1}$, para $n\geq1$, son independientes e identicamente distribuidos con distribuci\'on $F$, donde $F\left(0\right)=0$ y $T_{0}=0$. Los $T_{n}$ son llamados tiempos de renovaci\'on, referente a la independencia o renovaci\'on de la informaci\'on estoc\'astica en estos tiempos. Los $\xi_{n}$ son los tiempos de inter-renovaci\'on, y $N\left(t\right)$ es el n\'umero de renovaciones en el intervalo $\left[0,t\right)$
\end{Def}


\begin{Note}
Para definir un proceso de renovaci\'on para cualquier contexto, solamente hay que especificar una distribuci\'on $F$, con $F\left(0\right)=0$, para los tiempos de inter-renovaci\'on. La funci\'on $F$ en turno degune las otra variables aleatorias. De manera formal, existe un espacio de probabilidad y una sucesi\'on de variables aleatorias $\xi_{1},\xi_{2},\ldots$ definidas en este con distribuci\'on $F$. Entonces las otras cantidades son $T_{n}=\sum_{k=1}^{n}\xi_{k}$ y $N\left(t\right)=\sum_{n=1}^{\infty}\indora\left(T_{n}\leq t\right)$, donde $T_{n}\rightarrow\infty$ casi seguramente por la Ley Fuerte de los Grandes N\'umeros.
\end{Note}







Los tiempos $T_{n}$ est\'an relacionados con los conteos de $N\left(t\right)$ por

\begin{eqnarray*}
\left\{N\left(t\right)\geq n\right\}&=&\left\{T_{n}\leq t\right\}\\
T_{N\left(t\right)}\leq &t&<T_{N\left(t\right)+1},
\end{eqnarray*}

adem\'as $N\left(T_{n}\right)=n$, y 

\begin{eqnarray*}
N\left(t\right)=\max\left\{n:T_{n}\leq t\right\}=\min\left\{n:T_{n+1}>t\right\}
\end{eqnarray*}

Por propiedades de la convoluci\'on se sabe que

\begin{eqnarray*}
P\left\{T_{n}\leq t\right\}=F^{n\star}\left(t\right)
\end{eqnarray*}
que es la $n$-\'esima convoluci\'on de $F$. Entonces 

\begin{eqnarray*}
\left\{N\left(t\right)\geq n\right\}&=&\left\{T_{n}\leq t\right\}\\
P\left\{N\left(t\right)\leq n\right\}&=&1-F^{\left(n+1\right)\star}\left(t\right)
\end{eqnarray*}

Adem\'as usando el hecho de que $\esp\left[N\left(t\right)\right]=\sum_{n=1}^{\infty}P\left\{N\left(t\right)\geq n\right\}$
se tiene que

\begin{eqnarray*}
\esp\left[N\left(t\right)\right]=\sum_{n=1}^{\infty}F^{n\star}\left(t\right)
\end{eqnarray*}

\begin{Prop}
Para cada $t\geq0$, la funci\'on generadora de momentos $\esp\left[e^{\alpha N\left(t\right)}\right]$ existe para alguna $\alpha$ en una vecindad del 0, y de aqu\'i que $\esp\left[N\left(t\right)^{m}\right]<\infty$, para $m\geq1$.
\end{Prop}

\begin{Ejem}[\textbf{Proceso Poisson}]

Suponga que se tienen tiempos de inter-renovaci\'on \textit{i.i.d.} del proceso de renovaci\'on $N\left(t\right)$ tienen distribuci\'on exponencial $F\left(t\right)=q-e^{-\lambda t}$ con tasa $\lambda$. Entonces $N\left(t\right)$ es un proceso Poisson con tasa $\lambda$.

\end{Ejem}


\begin{Note}
Si el primer tiempo de renovaci\'on $\xi_{1}$ no tiene la misma distribuci\'on que el resto de las $\xi_{n}$, para $n\geq2$, a $N\left(t\right)$ se le llama Proceso de Renovaci\'on retardado, donde si $\xi$ tiene distribuci\'on $G$, entonces el tiempo $T_{n}$ de la $n$-\'esima renovaci\'on tiene distribuci\'on $G\star F^{\left(n-1\right)\star}\left(t\right)$
\end{Note}


\begin{Teo}
Para una constante $\mu\leq\infty$ ( o variable aleatoria), las siguientes expresiones son equivalentes:

\begin{eqnarray}
lim_{n\rightarrow\infty}n^{-1}T_{n}&=&\mu,\textrm{ c.s.}\\
lim_{t\rightarrow\infty}t^{-1}N\left(t\right)&=&1/\mu,\textrm{ c.s.}
\end{eqnarray}
\end{Teo}


Es decir, $T_{n}$ satisface la Ley Fuerte de los Grandes N\'umeros s\'i y s\'olo s\'i $N\left/t\right)$ la cumple.


\begin{Coro}[Ley Fuerte de los Grandes N\'umeros para Procesos de Renovaci\'on]
Si $N\left(t\right)$ es un proceso de renovaci\'on cuyos tiempos de inter-renovaci\'on tienen media $\mu\leq\infty$, entonces
\begin{eqnarray}
t^{-1}N\left(t\right)\rightarrow 1/\mu,\textrm{ c.s. cuando }t\rightarrow\infty.
\end{eqnarray}

\end{Coro}


Considerar el proceso estoc\'astico de valores reales $\left\{Z\left(t\right):t\geq0\right\}$ en el mismo espacio de probabilidad que $N\left(t\right)$

\begin{Def}
Para el proceso $\left\{Z\left(t\right):t\geq0\right\}$ se define la fluctuaci\'on m\'axima de $Z\left(t\right)$ en el intervalo $\left(T_{n-1},T_{n}\right]$:
\begin{eqnarray*}
M_{n}=\sup_{T_{n-1}<t\leq T_{n}}|Z\left(t\right)-Z\left(T_{n-1}\right)|
\end{eqnarray*}
\end{Def}

\begin{Teo}
Sup\'ongase que $n^{-1}T_{n}\rightarrow\mu$ c.s. cuando $n\rightarrow\infty$, donde $\mu\leq\infty$ es una constante o variable aleatoria. Sea $a$ una constante o variable aleatoria que puede ser infinita cuando $\mu$ es finita, y considere las expresiones l\'imite:
\begin{eqnarray}
lim_{n\rightarrow\infty}n^{-1}Z\left(T_{n}\right)&=&a,\textrm{ c.s.}\\
lim_{t\rightarrow\infty}t^{-1}Z\left(t\right)&=&a/\mu,\textrm{ c.s.}
\end{eqnarray}
La segunda expresi\'on implica la primera. Conversamente, la primera implica la segunda si el proceso $Z\left(t\right)$ es creciente, o si $lim_{n\rightarrow\infty}n^{-1}M_{n}=0$ c.s.
\end{Teo}

\begin{Coro}
Si $N\left(t\right)$ es un proceso de renovaci\'on, y $\left(Z\left(T_{n}\right)-Z\left(T_{n-1}\right),M_{n}\right)$, para $n\geq1$, son variables aleatorias independientes e id\'enticamente distribuidas con media finita, entonces,
\begin{eqnarray}
lim_{t\rightarrow\infty}t^{-1}Z\left(t\right)\rightarrow\frac{\esp\left[Z\left(T_{1}\right)-Z\left(T_{0}\right)\right]}{\esp\left[T_{1}\right]},\textrm{ c.s. cuando  }t\rightarrow\infty.
\end{eqnarray}
\end{Coro}


Sup\'ongase que $N\left(t\right)$ es un proceso de renovaci\'on con distribuci\'on $F$ con media finita $\mu$.

\begin{Def}
La funci\'on de renovaci\'on asociada con la distribuci\'on $F$, del proceso $N\left(t\right)$, es
\begin{eqnarray*}
U\left(t\right)=\sum_{n=1}^{\infty}F^{n\star}\left(t\right),\textrm{   }t\geq0,
\end{eqnarray*}
donde $F^{0\star}\left(t\right)=\indora\left(t\geq0\right)$.
\end{Def}


\begin{Prop}
Sup\'ongase que la distribuci\'on de inter-renovaci\'on $F$ tiene densidad $f$. Entonces $U\left(t\right)$ tambi\'en tiene densidad, para $t>0$, y es $U^{'}\left(t\right)=\sum_{n=0}^{\infty}f^{n\star}\left(t\right)$. Adem\'as
\begin{eqnarray*}
\prob\left\{N\left(t\right)>N\left(t-\right)\right\}=0\textrm{,   }t\geq0.
\end{eqnarray*}
\end{Prop}

\begin{Def}
La Transformada de Laplace-Stieljes de $F$ est\'a dada por

\begin{eqnarray*}
\hat{F}\left(\alpha\right)=\int_{\rea_{+}}e^{-\alpha t}dF\left(t\right)\textrm{,  }\alpha\geq0.
\end{eqnarray*}
\end{Def}

Entonces

\begin{eqnarray*}
\hat{U}\left(\alpha\right)=\sum_{n=0}^{\infty}\hat{F^{n\star}}\left(\alpha\right)=\sum_{n=0}^{\infty}\hat{F}\left(\alpha\right)^{n}=\frac{1}{1-\hat{F}\left(\alpha\right)}.
\end{eqnarray*}


\begin{Prop}
La Transformada de Laplace $\hat{U}\left(\alpha\right)$ y $\hat{F}\left(\alpha\right)$ determina una a la otra de manera \'unica por la relaci\'on $\hat{U}\left(\alpha\right)=\frac{1}{1-\hat{F}\left(\alpha\right)}$.
\end{Prop}


\begin{Note}
Un proceso de renovaci\'on $N\left(t\right)$ cuyos tiempos de inter-renovaci\'on tienen media finita, es un proceso Poisson con tasa $\lambda$ si y s\'olo s\'i $\esp\left[U\left(t\right)\right]=\lambda t$, para $t\geq0$.
\end{Note}


\begin{Teo}
Sea $N\left(t\right)$ un proceso puntual simple con puntos de localizaci\'on $T_{n}$ tal que $\eta\left(t\right)=\esp\left[N\left(\right)\right]$ es finita para cada $t$. Entonces para cualquier funci\'on $f:\rea_{+}\rightarrow\rea$,
\begin{eqnarray*}
\esp\left[\sum_{n=1}^{N\left(\right)}f\left(T_{n}\right)\right]=\int_{\left(0,t\right]}f\left(s\right)d\eta\left(s\right)\textrm{,  }t\geq0,
\end{eqnarray*}
suponiendo que la integral exista. Adem\'as si $X_{1},X_{2},\ldots$ son variables aleatorias definidas en el mismo espacio de probabilidad que el proceso $N\left(t\right)$ tal que $\esp\left[X_{n}|T_{n}=s\right]=f\left(s\right)$, independiente de $n$. Entonces
\begin{eqnarray*}
\esp\left[\sum_{n=1}^{N\left(t\right)}X_{n}\right]=\int_{\left(0,t\right]}f\left(s\right)d\eta\left(s\right)\textrm{,  }t\geq0,
\end{eqnarray*} 
suponiendo que la integral exista. 
\end{Teo}

\begin{Coro}[Identidad de Wald para Renovaciones]
Para el proceso de renovaci\'on $N\left(t\right)$,
\begin{eqnarray*}
\esp\left[T_{N\left(t\right)+1}\right]=\mu\esp\left[N\left(t\right)+1\right]\textrm{,  }t\geq0,
\end{eqnarray*}  
\end{Coro}


\begin{Def}
Sea $h\left(t\right)$ funci\'on de valores reales en $\rea$ acotada en intervalos finitos e igual a cero para $t<0$ La ecuaci\'on de renovaci\'on para $h\left(t\right)$ y la distribuci\'on $F$ es

\begin{eqnarray}%\label{Ec.Renovacion}
H\left(t\right)=h\left(t\right)+\int_{\left[0,t\right]}H\left(t-s\right)dF\left(s\right)\textrm{,    }t\geq0,
\end{eqnarray}
donde $H\left(t\right)$ es una funci\'on de valores reales. Esto es $H=h+F\star H$. Decimos que $H\left(t\right)$ es soluci\'on de esta ecuaci\'on si satisface la ecuaci\'on, y es acotada en intervalos finitos e iguales a cero para $t<0$.
\end{Def}

\begin{Prop}
La funci\'on $U\star h\left(t\right)$ es la \'unica soluci\'on de la ecuaci\'on de renovaci\'on (\ref{Ec.Renovacion}).
\end{Prop}

\begin{Teo}[Teorema Renovaci\'on Elemental]
\begin{eqnarray*}
t^{-1}U\left(t\right)\rightarrow 1/\mu\textrm{,    cuando }t\rightarrow\infty.
\end{eqnarray*}
\end{Teo}



Sup\'ongase que $N\left(t\right)$ es un proceso de renovaci\'on con distribuci\'on $F$ con media finita $\mu$.

\begin{Def}
La funci\'on de renovaci\'on asociada con la distribuci\'on $F$, del proceso $N\left(t\right)$, es
\begin{eqnarray*}
U\left(t\right)=\sum_{n=1}^{\infty}F^{n\star}\left(t\right),\textrm{   }t\geq0,
\end{eqnarray*}
donde $F^{0\star}\left(t\right)=\indora\left(t\geq0\right)$.
\end{Def}


\begin{Prop}
Sup\'ongase que la distribuci\'on de inter-renovaci\'on $F$ tiene densidad $f$. Entonces $U\left(t\right)$ tambi\'en tiene densidad, para $t>0$, y es $U^{'}\left(t\right)=\sum_{n=0}^{\infty}f^{n\star}\left(t\right)$. Adem\'as
\begin{eqnarray*}
\prob\left\{N\left(t\right)>N\left(t-\right)\right\}=0\textrm{,   }t\geq0.
\end{eqnarray*}
\end{Prop}

\begin{Def}
La Transformada de Laplace-Stieljes de $F$ est\'a dada por

\begin{eqnarray*}
\hat{F}\left(\alpha\right)=\int_{\rea_{+}}e^{-\alpha t}dF\left(t\right)\textrm{,  }\alpha\geq0.
\end{eqnarray*}
\end{Def}

Entonces

\begin{eqnarray*}
\hat{U}\left(\alpha\right)=\sum_{n=0}^{\infty}\hat{F^{n\star}}\left(\alpha\right)=\sum_{n=0}^{\infty}\hat{F}\left(\alpha\right)^{n}=\frac{1}{1-\hat{F}\left(\alpha\right)}.
\end{eqnarray*}


\begin{Prop}
La Transformada de Laplace $\hat{U}\left(\alpha\right)$ y $\hat{F}\left(\alpha\right)$ determina una a la otra de manera \'unica por la relaci\'on $\hat{U}\left(\alpha\right)=\frac{1}{1-\hat{F}\left(\alpha\right)}$.
\end{Prop}


\begin{Note}
Un proceso de renovaci\'on $N\left(t\right)$ cuyos tiempos de inter-renovaci\'on tienen media finita, es un proceso Poisson con tasa $\lambda$ si y s\'olo s\'i $\esp\left[U\left(t\right)\right]=\lambda t$, para $t\geq0$.
\end{Note}


\begin{Teo}
Sea $N\left(t\right)$ un proceso puntual simple con puntos de localizaci\'on $T_{n}$ tal que $\eta\left(t\right)=\esp\left[N\left(\right)\right]$ es finita para cada $t$. Entonces para cualquier funci\'on $f:\rea_{+}\rightarrow\rea$,
\begin{eqnarray*}
\esp\left[\sum_{n=1}^{N\left(\right)}f\left(T_{n}\right)\right]=\int_{\left(0,t\right]}f\left(s\right)d\eta\left(s\right)\textrm{,  }t\geq0,
\end{eqnarray*}
suponiendo que la integral exista. Adem\'as si $X_{1},X_{2},\ldots$ son variables aleatorias definidas en el mismo espacio de probabilidad que el proceso $N\left(t\right)$ tal que $\esp\left[X_{n}|T_{n}=s\right]=f\left(s\right)$, independiente de $n$. Entonces
\begin{eqnarray*}
\esp\left[\sum_{n=1}^{N\left(t\right)}X_{n}\right]=\int_{\left(0,t\right]}f\left(s\right)d\eta\left(s\right)\textrm{,  }t\geq0,
\end{eqnarray*} 
suponiendo que la integral exista. 
\end{Teo}

\begin{Coro}[Identidad de Wald para Renovaciones]
Para el proceso de renovaci\'on $N\left(t\right)$,
\begin{eqnarray*}
\esp\left[T_{N\left(t\right)+1}\right]=\mu\esp\left[N\left(t\right)+1\right]\textrm{,  }t\geq0,
\end{eqnarray*}  
\end{Coro}


\begin{Def}
Sea $h\left(t\right)$ funci\'on de valores reales en $\rea$ acotada en intervalos finitos e igual a cero para $t<0$ La ecuaci\'on de renovaci\'on para $h\left(t\right)$ y la distribuci\'on $F$ es

\begin{eqnarray}%\label{Ec.Renovacion}
H\left(t\right)=h\left(t\right)+\int_{\left[0,t\right]}H\left(t-s\right)dF\left(s\right)\textrm{,    }t\geq0,
\end{eqnarray}
donde $H\left(t\right)$ es una funci\'on de valores reales. Esto es $H=h+F\star H$. Decimos que $H\left(t\right)$ es soluci\'on de esta ecuaci\'on si satisface la ecuaci\'on, y es acotada en intervalos finitos e iguales a cero para $t<0$.
\end{Def}

\begin{Prop}
La funci\'on $U\star h\left(t\right)$ es la \'unica soluci\'on de la ecuaci\'on de renovaci\'on (\ref{Ec.Renovacion}).
\end{Prop}

\begin{Teo}[Teorema Renovaci\'on Elemental]
\begin{eqnarray*}
t^{-1}U\left(t\right)\rightarrow 1/\mu\textrm{,    cuando }t\rightarrow\infty.
\end{eqnarray*}
\end{Teo}


\begin{Note} Una funci\'on $h:\rea_{+}\rightarrow\rea$ es Directamente Riemann Integrable en los siguientes casos:
\begin{itemize}
\item[a)] $h\left(t\right)\geq0$ es decreciente y Riemann Integrable.
\item[b)] $h$ es continua excepto posiblemente en un conjunto de Lebesgue de medida 0, y $|h\left(t\right)|\leq b\left(t\right)$, donde $b$ es DRI.
\end{itemize}
\end{Note}

\begin{Teo}[Teorema Principal de Renovaci\'on]
Si $F$ es no aritm\'etica y $h\left(t\right)$ es Directamente Riemann Integrable (DRI), entonces

\begin{eqnarray*}
lim_{t\rightarrow\infty}U\star h=\frac{1}{\mu}\int_{\rea_{+}}h\left(s\right)ds.
\end{eqnarray*}
\end{Teo}

\begin{Prop}
Cualquier funci\'on $H\left(t\right)$ acotada en intervalos finitos y que es 0 para $t<0$ puede expresarse como
\begin{eqnarray*}
H\left(t\right)=U\star h\left(t\right)\textrm{,  donde }h\left(t\right)=H\left(t\right)-F\star H\left(t\right)
\end{eqnarray*}
\end{Prop}

\begin{Def}
Un proceso estoc\'astico $X\left(t\right)$ es crudamente regenerativo en un tiempo aleatorio positivo $T$ si
\begin{eqnarray*}
\esp\left[X\left(T+t\right)|T\right]=\esp\left[X\left(t\right)\right]\textrm{, para }t\geq0,\end{eqnarray*}
y con las esperanzas anteriores finitas.
\end{Def}

\begin{Prop}
Sup\'ongase que $X\left(t\right)$ es un proceso crudamente regenerativo en $T$, que tiene distribuci\'on $F$. Si $\esp\left[X\left(t\right)\right]$ es acotado en intervalos finitos, entonces
\begin{eqnarray*}
\esp\left[X\left(t\right)\right]=U\star h\left(t\right)\textrm{,  donde }h\left(t\right)=\esp\left[X\left(t\right)\indora\left(T>t\right)\right].
\end{eqnarray*}
\end{Prop}

\begin{Teo}[Regeneraci\'on Cruda]
Sup\'ongase que $X\left(t\right)$ es un proceso con valores positivo crudamente regenerativo en $T$, y def\'inase $M=\sup\left\{|X\left(t\right)|:t\leq T\right\}$. Si $T$ es no aritm\'etico y $M$ y $MT$ tienen media finita, entonces
\begin{eqnarray*}
lim_{t\rightarrow\infty}\esp\left[X\left(t\right)\right]=\frac{1}{\mu}\int_{\rea_{+}}h\left(s\right)ds,
\end{eqnarray*}
donde $h\left(t\right)=\esp\left[X\left(t\right)\indora\left(T>t\right)\right]$.
\end{Teo}


\begin{Note} Una funci\'on $h:\rea_{+}\rightarrow\rea$ es Directamente Riemann Integrable en los siguientes casos:
\begin{itemize}
\item[a)] $h\left(t\right)\geq0$ es decreciente y Riemann Integrable.
\item[b)] $h$ es continua excepto posiblemente en un conjunto de Lebesgue de medida 0, y $|h\left(t\right)|\leq b\left(t\right)$, donde $b$ es DRI.
\end{itemize}
\end{Note}

\begin{Teo}[Teorema Principal de Renovaci\'on]
Si $F$ es no aritm\'etica y $h\left(t\right)$ es Directamente Riemann Integrable (DRI), entonces

\begin{eqnarray*}
lim_{t\rightarrow\infty}U\star h=\frac{1}{\mu}\int_{\rea_{+}}h\left(s\right)ds.
\end{eqnarray*}
\end{Teo}

\begin{Prop}
Cualquier funci\'on $H\left(t\right)$ acotada en intervalos finitos y que es 0 para $t<0$ puede expresarse como
\begin{eqnarray*}
H\left(t\right)=U\star h\left(t\right)\textrm{,  donde }h\left(t\right)=H\left(t\right)-F\star H\left(t\right)
\end{eqnarray*}
\end{Prop}

\begin{Def}
Un proceso estoc\'astico $X\left(t\right)$ es crudamente regenerativo en un tiempo aleatorio positivo $T$ si
\begin{eqnarray*}
\esp\left[X\left(T+t\right)|T\right]=\esp\left[X\left(t\right)\right]\textrm{, para }t\geq0,\end{eqnarray*}
y con las esperanzas anteriores finitas.
\end{Def}

\begin{Prop}
Sup\'ongase que $X\left(t\right)$ es un proceso crudamente regenerativo en $T$, que tiene distribuci\'on $F$. Si $\esp\left[X\left(t\right)\right]$ es acotado en intervalos finitos, entonces
\begin{eqnarray*}
\esp\left[X\left(t\right)\right]=U\star h\left(t\right)\textrm{,  donde }h\left(t\right)=\esp\left[X\left(t\right)\indora\left(T>t\right)\right].
\end{eqnarray*}
\end{Prop}

\begin{Teo}[Regeneraci\'on Cruda]
Sup\'ongase que $X\left(t\right)$ es un proceso con valores positivo crudamente regenerativo en $T$, y def\'inase $M=\sup\left\{|X\left(t\right)|:t\leq T\right\}$. Si $T$ es no aritm\'etico y $M$ y $MT$ tienen media finita, entonces
\begin{eqnarray*}
lim_{t\rightarrow\infty}\esp\left[X\left(t\right)\right]=\frac{1}{\mu}\int_{\rea_{+}}h\left(s\right)ds,
\end{eqnarray*}
donde $h\left(t\right)=\esp\left[X\left(t\right)\indora\left(T>t\right)\right]$.
\end{Teo}

\begin{Def}
Para el proceso $\left\{\left(N\left(t\right),X\left(t\right)\right):t\geq0\right\}$, sus trayectoria muestrales en el intervalo de tiempo $\left[T_{n-1},T_{n}\right)$ est\'an descritas por
\begin{eqnarray*}
\zeta_{n}=\left(\xi_{n},\left\{X\left(T_{n-1}+t\right):0\leq t<\xi_{n}\right\}\right)
\end{eqnarray*}
Este $\zeta_{n}$ es el $n$-\'esimo segmento del proceso. El proceso es regenerativo sobre los tiempos $T_{n}$ si sus segmentos $\zeta_{n}$ son independientes e id\'enticamennte distribuidos.
\end{Def}


\begin{Note}
Si $\tilde{X}\left(t\right)$ con espacio de estados $\tilde{S}$ es regenerativo sobre $T_{n}$, entonces $X\left(t\right)=f\left(\tilde{X}\left(t\right)\right)$ tambi\'en es regenerativo sobre $T_{n}$, para cualquier funci\'on $f:\tilde{S}\rightarrow S$.
\end{Note}

\begin{Note}
Los procesos regenerativos son crudamente regenerativos, pero no al rev\'es.
\end{Note}


\begin{Note}
Un proceso estoc\'astico a tiempo continuo o discreto es regenerativo si existe un proceso de renovaci\'on  tal que los segmentos del proceso entre tiempos de renovaci\'on sucesivos son i.i.d., es decir, para $\left\{X\left(t\right):t\geq0\right\}$ proceso estoc\'astico a tiempo continuo con espacio de estados $S$, espacio m\'etrico.
\end{Note}

Para $\left\{X\left(t\right):t\geq0\right\}$ Proceso Estoc\'astico a tiempo continuo con estado de espacios $S$, que es un espacio m\'etrico, con trayectorias continuas por la derecha y con l\'imites por la izquierda c.s. Sea $N\left(t\right)$ un proceso de renovaci\'on en $\rea_{+}$ definido en el mismo espacio de probabilidad que $X\left(t\right)$, con tiempos de renovaci\'on $T$ y tiempos de inter-renovaci\'on $\xi_{n}=T_{n}-T_{n-1}$, con misma distribuci\'on $F$ de media finita $\mu$.



\begin{Def}
Para el proceso $\left\{\left(N\left(t\right),X\left(t\right)\right):t\geq0\right\}$, sus trayectoria muestrales en el intervalo de tiempo $\left[T_{n-1},T_{n}\right)$ est\'an descritas por
\begin{eqnarray*}
\zeta_{n}=\left(\xi_{n},\left\{X\left(T_{n-1}+t\right):0\leq t<\xi_{n}\right\}\right)
\end{eqnarray*}
Este $\zeta_{n}$ es el $n$-\'esimo segmento del proceso. El proceso es regenerativo sobre los tiempos $T_{n}$ si sus segmentos $\zeta_{n}$ son independientes e id\'enticamennte distribuidos.
\end{Def}

\begin{Note}
Un proceso regenerativo con media de la longitud de ciclo finita es llamado positivo recurrente.
\end{Note}

\begin{Teo}[Procesos Regenerativos]
Suponga que el proceso
\end{Teo}


\begin{Def}[Renewal Process Trinity]
Para un proceso de renovaci\'on $N\left(t\right)$, los siguientes procesos proveen de informaci\'on sobre los tiempos de renovaci\'on.
\begin{itemize}
\item $A\left(t\right)=t-T_{N\left(t\right)}$, el tiempo de recurrencia hacia atr\'as al tiempo $t$, que es el tiempo desde la \'ultima renovaci\'on para $t$.

\item $B\left(t\right)=T_{N\left(t\right)+1}-t$, el tiempo de recurrencia hacia adelante al tiempo $t$, residual del tiempo de renovaci\'on, que es el tiempo para la pr\'oxima renovaci\'on despu\'es de $t$.

\item $L\left(t\right)=\xi_{N\left(t\right)+1}=A\left(t\right)+B\left(t\right)$, la longitud del intervalo de renovaci\'on que contiene a $t$.
\end{itemize}
\end{Def}

\begin{Note}
El proceso tridimensional $\left(A\left(t\right),B\left(t\right),L\left(t\right)\right)$ es regenerativo sobre $T_{n}$, y por ende cada proceso lo es. Cada proceso $A\left(t\right)$ y $B\left(t\right)$ son procesos de MArkov a tiempo continuo con trayectorias continuas por partes en el espacio de estados $\rea_{+}$. Una expresi\'on conveniente para su distribuci\'on conjunta es, para $0\leq x<t,y\geq0$
\begin{equation}\label{NoRenovacion}
P\left\{A\left(t\right)>x,B\left(t\right)>y\right\}=
P\left\{N\left(t+y\right)-N\left((t-x)\right)=0\right\}
\end{equation}
\end{Note}

\begin{Ejem}[Tiempos de recurrencia Poisson]
Si $N\left(t\right)$ es un proceso Poisson con tasa $\lambda$, entonces de la expresi\'on (\ref{NoRenovacion}) se tiene que

\begin{eqnarray*}
\begin{array}{lc}
P\left\{A\left(t\right)>x,B\left(t\right)>y\right\}=e^{-\lambda\left(x+y\right)},&0\leq x<t,y\geq0,
\end{array}
\end{eqnarray*}
que es la probabilidad Poisson de no renovaciones en un intervalo de longitud $x+y$.

\end{Ejem}

\begin{Note}
Una cadena de Markov erg\'odica tiene la propiedad de ser estacionaria si la distribución de su estado al tiempo $0$ es su distribuci\'on estacionaria.
\end{Note}


\begin{Def}
Un proceso estoc\'astico a tiempo continuo $\left\{X\left(t\right):t\geq0\right\}$ en un espacio general es estacionario si sus distribuciones finito dimensionales son invariantes bajo cualquier  traslado: para cada $0\leq s_{1}<s_{2}<\cdots<s_{k}$ y $t\geq0$,
\begin{eqnarray*}
\left(X\left(s_{1}+t\right),\ldots,X\left(s_{k}+t\right)\right)=_{d}\left(X\left(s_{1}\right),\ldots,X\left(s_{k}\right)\right).
\end{eqnarray*}
\end{Def}

\begin{Note}
Un proceso de Markov es estacionario si $X\left(t\right)=_{d}X\left(0\right)$, $t\geq0$.
\end{Note}

Considerese el proceso $N\left(t\right)=\sum_{n}\indora\left(\tau_{n}\leq t\right)$ en $\rea_{+}$, con puntos $0<\tau_{1}<\tau_{2}<\cdots$.

\begin{Prop}
Si $N$ es un proceso puntual estacionario y $\esp\left[N\left(1\right)\right]<\infty$, entonces $\esp\left[N\left(t\right)\right]=t\esp\left[N\left(1\right)\right]$, $t\geq0$

\end{Prop}

\begin{Teo}
Los siguientes enunciados son equivalentes
\begin{itemize}
\item[i)] El proceso retardado de renovaci\'on $N$ es estacionario.

\item[ii)] EL proceso de tiempos de recurrencia hacia adelante $B\left(t\right)$ es estacionario.


\item[iii)] $\esp\left[N\left(t\right)\right]=t/\mu$,


\item[iv)] $G\left(t\right)=F_{e}\left(t\right)=\frac{1}{\mu}\int_{0}^{t}\left[1-F\left(s\right)\right]ds$
\end{itemize}
Cuando estos enunciados son ciertos, $P\left\{B\left(t\right)\leq x\right\}=F_{e}\left(x\right)$, para $t,x\geq0$.

\end{Teo}

\begin{Note}
Una consecuencia del teorema anterior es que el Proceso Poisson es el \'unico proceso sin retardo que es estacionario.
\end{Note}

\begin{Coro}
El proceso de renovaci\'on $N\left(t\right)$ sin retardo, y cuyos tiempos de inter renonaci\'on tienen media finita, es estacionario si y s\'olo si es un proceso Poisson.

\end{Coro}




%______________________________________________________________________
\section{Resultados para Procesos de Salida}
%______________________________________________________________________
En Sigman, Thorison y Wolff \cite{Sigman2} prueban que para la existencia de un una sucesi\'on infinita no decreciente de tiempos de regeneraci\'on $\tau_{1}\leq\tau_{2}\leq\cdots$ en los cuales el proceso se regenera, basta un tiempo de regeneraci\'on $R_{1}$, donde $R_{j}=\tau_{j}-\tau_{j-1}$. Para tal efecto se requiere la existencia de un espacio de probabilidad $\left(\Omega,\mathcal{F},\prob\right)$, y proceso estoc\'astico $\textit{X}=\left\{X\left(t\right):t\geq0\right\}$ con espacio de estados $\left(S,\mathcal{R}\right)$, con $\mathcal{R}$ $\sigma$-\'algebra.

\begin{Prop}
Si existe una variable aleatoria no negativa $R_{1}$ tal que $\theta_{R\footnotesize{1}}X=_{D}X$, entonces $\left(\Omega,\mathcal{F},\prob\right)$ puede extenderse para soportar una sucesi\'on estacionaria de variables aleatorias $R=\left\{R_{k}:k\geq1\right\}$, tal que para $k\geq1$,
\begin{eqnarray*}
\theta_{k}\left(X,R\right)=_{D}\left(X,R\right).
\end{eqnarray*}

Adem\'as, para $k\geq1$, $\theta_{k}R$ es condicionalmente independiente de $\left(X,R_{1},\ldots,R_{k}\right)$, dado $\theta_{\tau k}X$.

\end{Prop}


\begin{itemize}
\item Doob en 1953 demostr\'o que el estado estacionario de un proceso de partida en un sistema de espera $M/G/\infty$, es Poisson con la misma tasa que el proceso de arribos.

\item Burke en 1968, fue el primero en demostrar que el estado estacionario de un proceso de salida de una cola $M/M/s$ es un proceso Poisson.

\item Disney en 1973 obtuvo el siguiente resultado:

\begin{Teo}
Para el sistema de espera $M/G/1/L$ con disciplina FIFO, el proceso $\textbf{I}$ es un proceso de renovaci\'on si y s\'olo si el proceso denominado longitud de la cola es estacionario y se cumple cualquiera de los siguientes casos:

\begin{itemize}
\item[a)] Los tiempos de servicio son identicamente cero;
\item[b)] $L=0$, para cualquier proceso de servicio $S$;
\item[c)] $L=1$ y $G=D$;
\item[d)] $L=\infty$ y $G=M$.
\end{itemize}
En estos casos, respectivamente, las distribuciones de interpartida $P\left\{T_{n+1}-T_{n}\leq t\right\}$ son


\begin{itemize}
\item[a)] $1-e^{-\lambda t}$, $t\geq0$;
\item[b)] $1-e^{-\lambda t}*F\left(t\right)$, $t\geq0$;
\item[c)] $1-e^{-\lambda t}*\indora_{d}\left(t\right)$, $t\geq0$;
\item[d)] $1-e^{-\lambda t}*F\left(t\right)$, $t\geq0$.
\end{itemize}
\end{Teo}


\item Finch (1959) mostr\'o que para los sistemas $M/G/1/L$, con $1\leq L\leq \infty$ con distribuciones de servicio dos veces diferenciable, solamente el sistema $M/M/1/\infty$ tiene proceso de salida de renovaci\'on estacionario.

\item King (1971) demostro que un sistema de colas estacionario $M/G/1/1$ tiene sus tiempos de interpartida sucesivas $D_{n}$ y $D_{n+1}$ son independientes, si y s\'olo si, $G=D$, en cuyo caso le proceso de salida es de renovaci\'on.

\item Disney (1973) demostr\'o que el \'unico sistema estacionario $M/G/1/L$, que tiene proceso de salida de renovaci\'on  son los sistemas $M/M/1$ y $M/D/1/1$.



\item El siguiente resultado es de Disney y Koning (1985)
\begin{Teo}
En un sistema de espera $M/G/s$, el estado estacionario del proceso de salida es un proceso Poisson para cualquier distribuci\'on de los tiempos de servicio si el sistema tiene cualquiera de las siguientes cuatro propiedades.

\begin{itemize}
\item[a)] $s=\infty$
\item[b)] La disciplina de servicio es de procesador compartido.
\item[c)] La disciplina de servicio es LCFS y preemptive resume, esto se cumple para $L<\infty$
\item[d)] $G=M$.
\end{itemize}

\end{Teo}

\item El siguiente resultado es de Alamatsaz (1983)

\begin{Teo}
En cualquier sistema de colas $GI/G/1/L$ con $1\leq L<\infty$ y distribuci\'on de interarribos $A$ y distribuci\'on de los tiempos de servicio $B$, tal que $A\left(0\right)=0$, $A\left(t\right)\left(1-B\left(t\right)\right)>0$ para alguna $t>0$ y $B\left(t\right)$ para toda $t>0$, es imposible que el proceso de salida estacionario sea de renovaci\'on.
\end{Teo}

\end{itemize}

Estos resultados aparecen en Daley (1968) \cite{Daley68} para $\left\{T_{n}\right\}$ intervalos de inter-arribo, $\left\{D_{n}\right\}$ intervalos de inter-salida y $\left\{S_{n}\right\}$ tiempos de servicio.

\begin{itemize}
\item Si el proceso $\left\{T_{n}\right\}$ es Poisson, el proceso $\left\{D_{n}\right\}$ es no correlacionado si y s\'olo si es un proceso Poisso, lo cual ocurre si y s\'olo si $\left\{S_{n}\right\}$ son exponenciales negativas.

\item Si $\left\{S_{n}\right\}$ son exponenciales negativas, $\left\{D_{n}\right\}$ es un proceso de renovaci\'on  si y s\'olo si es un proceso Poisson, lo cual ocurre si y s\'olo si $\left\{T_{n}\right\}$ es un proceso Poisson.

\item $\esp\left(D_{n}\right)=\esp\left(T_{n}\right)$.

\item Para un sistema de visitas $GI/M/1$ se tiene el siguiente teorema:

\begin{Teo}
En un sistema estacionario $GI/M/1$ los intervalos de interpartida tienen
\begin{eqnarray*}
\esp\left(e^{-\theta D_{n}}\right)&=&\mu\left(\mu+\theta\right)^{-1}\left[\delta\theta
-\mu\left(1-\delta\right)\alpha\left(\theta\right)\right]
\left[\theta-\mu\left(1-\delta\right)^{-1}\right]\\
\alpha\left(\theta\right)&=&\esp\left[e^{-\theta T_{0}}\right]\\
var\left(D_{n}\right)&=&var\left(T_{0}\right)-\left(\tau^{-1}-\delta^{-1}\right)
2\delta\left(\esp\left(S_{0}\right)\right)^{2}\left(1-\delta\right)^{-1}.
\end{eqnarray*}
\end{Teo}



\begin{Teo}
El proceso de salida de un sistema de colas estacionario $GI/M/1$ es un proceso de renovaci\'on si y s\'olo si el proceso de entrada es un proceso Poisson, en cuyo caso el proceso de salida es un proceso Poisson.
\end{Teo}


\begin{Teo}
Los intervalos de interpartida $\left\{D_{n}\right\}$ de un sistema $M/G/1$ estacionario son no correlacionados si y s\'olo si la distribuci\'on de los tiempos de servicio es exponencial negativa, es decir, el sistema es de tipo  $M/M/1$.

\end{Teo}



\end{itemize}



%==<>====<>====<>====<>====<>====<>====<>====<>====<>====<>====
\part{BIBLIOGRAFIA}
%==<>====<>====<>====<>====<>====<>====<>====<>====<>====<>====

\chapter{Bibliografía}
\begin{thebibliography}{99}

\bibitem{ISL}
James, G., Witten, D., Hastie, T., and Tibshirani, R. (2013). \textit{An Introduction to Statistical Learning: with Applications in R}. Springer.

\bibitem{Logistic}
Hosmer, D. W., Lemeshow, S., and Sturdivant, R. X. (2013). \textit{Applied Logistic Regression} (3rd ed.). Wiley.

\bibitem{PatternRecognition}
Bishop, C. M. (2006). \textit{Pattern Recognition and Machine Learning}. Springer.

\bibitem{Harrell}
Harrell, F. E. (2015). \textit{Regression Modeling Strategies: With Applications to Linear Models, Logistic and Ordinal Regression, and Survival Analysis}. Springer.

\bibitem{RDocumentation}
R Documentation and Tutorials: \url{https://cran.r-project.org/manuals.html}

\bibitem{RBlogger}
Tutorials on R-bloggers: \url{https://www.r-bloggers.com/}

\bibitem{CourseraML}
Coursera: \textit{Machine Learning} by Andrew Ng.

\bibitem{edXDS}
edX: \textit{Data Science and Machine Learning Essentials} by Microsoft.

% Libros adicionales
\bibitem{Ross}
Ross, S. M. (2014). \textit{Introduction to Probability and Statistics for Engineers and Scientists}. Academic Press.

\bibitem{DeGroot}
DeGroot, M. H., and Schervish, M. J. (2012). \textit{Probability and Statistics} (4th ed.). Pearson.

\bibitem{Hogg}
Hogg, R. V., McKean, J., and Craig, A. T. (2019). \textit{Introduction to Mathematical Statistics} (8th ed.). Pearson.

\bibitem{Kleinbaum}
Kleinbaum, D. G., and Klein, M. (2010). \textit{Logistic Regression: A Self-Learning Text} (3rd ed.). Springer.

% Artículos y tutoriales adicionales
\bibitem{Wasserman}
Wasserman, L. (2004). \textit{All of Statistics: A Concise Course in Statistical Inference}. Springer.

\bibitem{KhanAcademy}
Probability and Statistics Tutorials on Khan Academy: \url{https://www.khanacademy.org/math/statistics-probability}

\bibitem{OnlineStatBook}
Online Statistics Education: \url{http://onlinestatbook.com/}

\bibitem{Peng}
Peng, C. Y. J., Lee, K. L., and Ingersoll, G. M. (2002). \textit{An Introduction to Logistic Regression Analysis and Reporting}. The Journal of Educational Research.

\bibitem{Agresti}
Agresti, A. (2007). \textit{An Introduction to Categorical Data Analysis} (2nd ed.). Wiley.

\bibitem{Han}
Han, J., Pei, J., and Kamber, M. (2011). \textit{Data Mining: Concepts and Techniques}. Morgan Kaufmann.

\bibitem{TowardsDataScience}
Data Cleaning and Preprocessing on Towards Data Science: \url{https://towardsdatascience.com/data-cleaning-and-preprocessing}

\bibitem{Molinaro}
Molinaro, A. M., Simon, R., and Pfeiffer, R. M. (2005). \textit{Prediction error estimation: a comparison of resampling methods}. Bioinformatics.

\bibitem{EvaluatingModels}
Evaluating Machine Learning Models on Towards Data Science: \url{https://towardsdatascience.com/evaluating-machine-learning-models}

\bibitem{LogisticRegressionGuide}
Practical Guide to Logistic Regression in R on Towards Data Science: \url{https://towardsdatascience.com/practical-guide-to-logistic-regression-in-r}

% Cursos en línea adicionales
\bibitem{CourseraStatistics}
Coursera: \textit{Statistics with R} by Duke University.

\bibitem{edXProbability}
edX: \textit{Data Science: Probability} by Harvard University.

\bibitem{CourseraLogistic}
Coursera: \textit{Logistic Regression} by Stanford University.

\bibitem{edXInference}
edX: \textit{Data Science: Inference and Modeling} by Harvard University.

\bibitem{CourseraWrangling}
Coursera: \textit{Data Science: Wrangling and Cleaning} by Johns Hopkins University.

\bibitem{edXRBasics}
edX: \textit{Data Science: R Basics} by Harvard University.

\bibitem{CourseraRegression}
Coursera: \textit{Regression Models} by Johns Hopkins University.

\bibitem{edXStatInference}
edX: \textit{Data Science: Statistical Inference} by Harvard University.

\bibitem{SurvivalAnalysis}
An Introduction to Survival Analysis on Towards Data Science: \url{https://towardsdatascience.com/an-introduction-to-survival-analysis}

\bibitem{MultinomialLogistic}
Multinomial Logistic Regression on DataCamp: \url{https://www.datacamp.com/community/tutorials/multinomial-logistic-regression-R}

\bibitem{CourseraSurvival}
Coursera: \textit{Survival Analysis} by Johns Hopkins University.

\bibitem{edXHighthroughput}
edX: \textit{Data Science: Statistical Inference and Modeling for High-throughput Experiments} by Harvard University.

\end{thebibliography}


\end{document}
