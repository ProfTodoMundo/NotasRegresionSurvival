
\section{2. Pruebas de Hipótesis}

\subsection{2.1 Tipos de errores}

\begin{itemize}
    \item Una hipótesis estadística es una afirmación acerca de la distribución de probabilidad de una variable aleatoria, a menudo involucran uno o más parámetros de la distribución.
    \item Las hipótesis son afirmaciones respecto a la población o distribución bajo estudio, no en torno a la muestra.
    \item La mayoría de las veces, la prueba de hipótesis consiste en determinar si la situación experimental ha cambiado.
    \item El interés principal es decidir sobre la veracidad o falsedad de una hipótesis, a este procedimiento se le llama \textit{prueba de hipótesis}.
    \item Si la información es consistente con la hipótesis, se concluye que esta es verdadera, de lo contrario que con base en la información, es falsa.
\end{itemize}

Una prueba de hipótesis está formada por cinco partes:
\begin{itemize}
    \item La hipótesis nula, denotada por $H_{0}$.
    \item La hipótesis alternativa, denotada por $H_{1}$.
    \item El estadístico de prueba y su valor $p$.
    \item La región de rechazo.
    \item La conclusión.
\end{itemize}

\begin{Def}
Las dos hipótesis en competencia son la \textbf{hipótesis alternativa $H_{1}$}, usualmente la que se desea apoyar, y la \textbf{hipótesis nula $H_{0}$}, opuesta a $H_{1}$.
\end{Def}
En general, es más fácil presentar evidencia de que $H_{1}$ es cierta, que demostrar que $H_{0}$ es falsa, es por eso que por lo regular se comienza suponiendo que $H_{0}$ es cierta, luego se utilizan los datos de la muestra para decidir si existe evidencia a favor de $H_{1}$, más que a favor de $H_{0}$, así se tienen dos conclusiones:
\begin{itemize}
    \item Rechazar $H_{0}$ y concluir que $H_{1}$ es verdadera.
    \item Aceptar, no rechazar, $H_{0}$ como verdadera.
\end{itemize}

\begin{Ejem}
Se desea demostrar que el salario promedio por hora en cierto lugar es distinto de $19$ usd, que es el promedio nacional. Entonces $H_{1}:\mu \neq 19$, y $H_{0}:\mu = 19$.
\end{Ejem}
A esta se le denomina \textbf{Prueba de hipótesis de dos colas}.

\begin{Ejem}
Un determinado proceso produce un promedio de $5\%$ de piezas defectuosas. Se está interesado en demostrar que un simple ajuste en una máquina reducirá $p$, la proporción de piezas defectuosas producidas en este proceso. Entonces se tiene $H_{0}: p < 0.3$ y $H_{1}: p = 0.03$. Si se puede rechazar $H_{0}$, se concluye que el proceso ajustado produce menos del $5\%$ de piezas defectuosas.
\end{Ejem}
A esta se le denomina \textbf{Prueba de hipótesis de una cola}.

La decisión de rechazar o aceptar la hipótesis nula está basada en la información contenida en una muestra proveniente de la población de interés. Esta información tiene estas formas:
\begin{itemize}
    \item \textbf{Estadístico de prueba:} un sólo número calculado a partir de la muestra.
    \item \textbf{$p$-value:} probabilidad calculada a partir del estadístico de prueba.
\end{itemize}

\begin{Def}
El $p$-value es la probabilidad de observar un estadístico de prueba tanto o más alejado del valor observado, si en realidad $H_{0}$ es verdadera.\medskip
Valores grandes del estadístico de prueba y valores pequeños de $p$ significan que se ha observado un evento muy poco probable, si $H_{0}$ en realidad es verdadera.
\end{Def}
Todo el conjunto de valores que puede tomar el estadístico de prueba se divide en dos regiones. Un conjunto, formado de valores que apoyan la hipótesis alternativa y llevan a rechazar $H_{0}$, se denomina \textbf{región de rechazo}. El otro, conformado por los valores que sustentan la hipótesis nula, se le denomina \textbf{región de aceptación}.\medskip

Cuando la región de rechazo está en la cola izquierda de la distribución, la prueba se denomina \textbf{prueba lateral izquierda}. Una prueba con región de rechazo en la cola derecha se le llama \textbf{prueba lateral derecha}.\medskip

Si el estadístico de prueba cae en la región de rechazo, entonces se rechaza $H_{0}$. Si el estadístico de prueba cae en la región de aceptación, entonces la hipótesis nula se acepta o la prueba se juzga como no concluyente.\medskip

Dependiendo del nivel de confianza que se desea agregar a las conclusiones de la prueba, y el \textbf{nivel de significancia $\alpha$}, el riesgo que está dispuesto a correr si se toma una decisión incorrecta.

\begin{Def}
Un \textbf{error de tipo I} para una prueba estadística es el error que se tiene al rechazar la hipótesis nula cuando es verdadera. El \textbf{nivel de significancia} para una prueba estadística de hipótesis es
\begin{eqnarray*}
\alpha &=& P\left\{\textrm{error tipo I}\right\} = P\left\{\textrm{rechazar equivocadamente } H_{0}\right\} \\
&=& P\left\{\textrm{rechazar } H_{0} \textrm{ cuando } H_{0} \textrm{ es verdadera}\right\}
\end{eqnarray*}
\end{Def}
Este valor $\alpha$ representa el valor máximo de riesgo tolerable de rechazar incorrectamente $H_{0}$. Una vez establecido el nivel de significancia, la región de rechazo se define para poder determinar si se rechaza $H_{0}$ con un cierto nivel de confianza.


\section{2.2 Muestras grandes: una media poblacional}
\subsection{2.2.1 Cálculo de valor $p$}


\begin{Def}
El \textbf{valor de $p$} (\textbf{$p$-value}) o nivel de significancia observado de un estadístico de prueba es el valor más pequeño de $\alpha$ para el cual $H_{0}$ se puede rechazar. El riesgo de cometer un error tipo $I$, si $H_{0}$ es rechazada con base en la información que proporciona la muestra.
\end{Def}

\begin{Note}
Valores pequeños de $p$ indican que el valor observado del estadístico de prueba se encuentra alejado del valor hipotético de $\mu$, es decir se tiene evidencia de que $H_{0}$ es falsa y por tanto debe de rechazarse.
\end{Note}

\begin{Note}
Valores grandes de $p$ indican que el estadístico de prueba observado no está alejado de la media hipotética y no apoya el rechazo de $H_{0}$.
\end{Note}

\begin{Def}
Si el valor de $p$ es menor o igual que el nivel de significancia $\alpha$, determinado previamente, entonces $H_{0}$ es rechazada y se puede concluir que los resultados son estadísticamente significativos con un nivel de confianza del $100 (1-\alpha)\%$.
\end{Def}
Es usual utilizar la siguiente clasificación de resultados:

\begin{tabular}{|c||c|l|}\hline
$p$ & $H_{0}$ & Significativa \\ \hline
$\leq 0.01$ & rechazada & \begin{tabular}[c]{@{}l@{}}Result. altamente significativos \\ y en contra de $H_{0}$\end{tabular} \\ \hline
$\leq 0.05$ & rechazada & \begin{tabular}[c]{@{}l@{}}Result. significativos \\ y en contra de $H_{0}$\end{tabular} \\ \hline
$\leq 0.10$ & rechazada & \begin{tabular}[c]{@{}l@{}}Result. posiblemente \\ significativos \\ y en contra de $H_{0}$\end{tabular} \\ \hline
$> 0.10$ & no rechazada & \begin{tabular}[c]{@{}l@{}}Result. no significativos \\ y no rechazar $H_{0}$\end{tabular} \\ \hline
\end{tabular}

