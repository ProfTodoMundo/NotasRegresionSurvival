
\section{Introducci\'on}

La regresi\'on log\'istica multinomial y el an\'alisis de supervivencia son extensiones de la regresi\'on log\'istica binaria. Este cap\'itulo se enfoca en las t\'ecnicas y aplicaciones de estos m\'etodos avanzados.

\section{Regresi\'on Log\'istica Multinomial}

La regresi\'on log\'istica multinomial se utiliza cuando la variable dependiente tiene m\'as de dos categor\'ias.

\subsection{Modelo Multinomial}

El modelo de regresi\'on log\'istica multinomial generaliza el modelo binario para manejar m\'ultiples categor\'ias. La probabilidad de que una observaci\'on pertenezca a la categor\'ia $k$ se expresa como:

\begin{eqnarray*}
P(Y = k) = \frac{e^{\beta_{0k} + \beta_{1k} X_1 + \ldots + \beta_{nk} X_n}}{\sum_{j=1}^{K} e^{\beta_{0j} + \beta_{1j} X_1 + \ldots + \beta_{nj} X_n}}
\end{eqnarray*}

\subsection{Estimaci\'on de Par\'ametros}

Los coeficientes del modelo multinomial se estiman utilizando m\'axima verosimilitud, similar a la regresi\'on log\'istica binaria.

\section{An\'alisis de Supervivencia}

El an\'alisis de supervivencia se utiliza para modelar el tiempo hasta que ocurre un evento de inter\'es, como la muerte o la falla de un componente.

\subsection{Funci\'on de Supervivencia}

La funci\'on de supervivencia $S(t)$ describe la probabilidad de que una observaci\'on sobreviva m\'as all\'a del tiempo $t$:

\begin{eqnarray*}
S(t) = P(T > t)
\end{eqnarray*}

\subsection{Modelo de Riesgos Proporcionales de Cox}

El modelo de Cox es un modelo de regresi\'on semiparam\'etrico utilizado para analizar datos de supervivencia:

\begin{eqnarray*}
h(t|X) = h_0(t) e^{\beta_1 X_1 + \ldots + \beta_p X_p}
\end{eqnarray*}
donde $h(t|X)$ es la tasa de riesgo en el tiempo $t$ dado el vector de covariables $X$ y $h_0(t)$ es la tasa de riesgo basal.

\section{Implementaci\'on en R}

\subsection{Regresi\'on Log\'istica Multinomial}

\begin{verbatim}
# Cargar el paquete necesario
library(nnet)

# Entrenar el modelo de regresi\'on log\'istica multinomial
model_multinom <- multinom(var1 ~ ., data = dataTrain)

# Resumen del modelo
summary(model_multinom)
\end{verbatim}

\subsection{An\'alisis de Supervivencia}

\begin{verbatim}
# Cargar el paquete necesario
library(survival)

# Crear el objeto de supervivencia
surv_object <- Surv(time = data$time, event = data$status)

# Ajustar el modelo de Cox
model_cox <- coxph(surv_object ~ var1 + var2, data = data)

# Resumen del modelo
summary(model_cox)
\end{verbatim}

