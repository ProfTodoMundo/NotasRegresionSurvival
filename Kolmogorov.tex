
%____________________________________________________________________________________________
%
\section{Teorema de Kolmogorov}
%____________________________________________________________________________________________
%
\begin{Teo}{\bf (Teorema de Kolmogorov)}\label{Tma Kolmogorov}
Las funciones de distribuci{\'o}n de probabilidad \[\left\{F_{t}\left(\cdot\right):t\in{\it F}\right\}\]
son funciones de distribuci\'on de un proceso estoc\'astico, \Xt\ si y s\'olo si se cumple que para toda $n\in\nat$, ${\bf t}=\left(t_{1},t_{2},\ldots,t_{n}\right)^{'}\in{\it F}$ con
$1\leq i\leq n$ se tiene que
\begin{eqnarray*}
lim_{x_{i}\rightarrow\infty}F_{t}\left(\Xv\right)=F_{t\left(i\right)}\left(\Xv\left(i\right)\right)
\end{eqnarray*}
donde $\tv\left(i\right)$ y $\Xv\left(i\right)$ son las $n-1$ componentes del vector, obtenidas al eliminar la $i$-\'e}sima componente de $\tv$ y $\Xv$ respectivamente; con
\[{\it F}=\left\{\left(t_{1},t_{2},\ldots, t_{n}\right)'\in T^{n}:t_{1}\leq t_{2}\leq\ldots\leq t_{n}, n = 1,2,\ldots\right\}.\] Entonces las funciones de distribuci\'on finito dimensionales de \Xt\ son las funciones $\left\{F_{t}\left(\cdot\right),t\in{\it F}\right\}$ definidas para $\tv$ por \[F_{t}\left(\Xv\right)=\Prob\left[X_{t_{1}}<x_{1},X_{t_{2}}<x_{2},\ldots,X_{t_{n}}<x_{n}\right].\]
\end{Teo}
%
%____________________________________________________________________________________________
%
\vspace{-1.cm}
\section{Teoremas de Helly y Helly-Bray}
\vspace{-.5cm}
%____________________________________________________________________________________________
%

Una funci\'on de distribuci\'on generalizada es una funci\'on definida sobre $\rea\cup\left\{-\infty,\infty\right\}$ creciente, tal que $F\left(-\infty\right)=0$ y
$F\left(\infty\right)=1$.
\begin{Lema}\label{Lema1}
Sea $\left\{F_{n}\right\}$ sucesi\'on de funciones de distribuci\'on
generalizadas tales que $\left|F_{n}\right|\leq M$ para toda $n$,

supongamos que existe $G:\rea\rightarrow\rea$ creciente y acotada,
tal que $\left\{F_{n}\right\}$ converge puntualmente a $G$ sobre
$\DD\subseteq\rea$ denso. Entonces
$\left\{F_{n}\left(x\right)\right\}$ converge a $G\left(x\right)$
para toda $x\in \DD$ punto de continuidad de $G$.
\end{Lema}
\begin{proof}
%____________________________________________________________________________________________
%con al menos un punto $x$ donde $G$ tiene l{\'\i}mite por la izquierda
%y por la derecha,
%____________________________________________________________________________________________
Sea $x$ punto de continuidad de $G$, sean $w_{k},y_{k}\in \DD$
tales que $w_{k}<x<y_{k}$ y $y_{k}\downarrow x$,\ $w_{k}\uparrow
x$, entonces \[F_{n}\left(w_{k}\right)\leq F_{n}\left(x\right)\leq
F_{n}\left(y_{k}\right)\] dado que $\left\{F_{n}\right\}$ converge
puntualmente a $G$ sobre \DD, tenemos que
\begin{eqnarray*}
G\left(w_{k}\right)\leq \liminf_{n\rightarrow\infty}F_{n}\left(x\right)\leq \limsup_{n\rightarrow\infty}F_{n}\left(x\right)\leq G\left(y_{k}\right)
\end{eqnarray*} si hacemos $k\rightarrow\infty$, entonces
\begin{eqnarray*}
G\left(x\right)=G\left(x^{-}\right)\leq \liminf_{n\rightarrow\infty}F_{n}\left(x\right)\leq \limsup_{n\rightarrow\infty}F_{n}\left(x\right)\leq G\left(x^{+}\right)=G\left(x\right)
\end{eqnarray*} por lo tanto \[\lim_{n\rightarrow\infty}f_{n}\left(x\right)=G\left(x\right)\] para todo $x$ punto de continuidad de $G$.
%____________________________________________________________________________________________
\end{proof}
\begin{Lema}\label{Lema2}
Sea $\left\{F_{n}\right\}$ sucesi\'on de funciones de distribuci\'on
generalizadas tales que $\left|F_{n}\right|\leq M$ para toda $n$ y
sea $D=\left\{x_{k}|k\geq 1\right\}$ conjunto numerable de n\'umeros
reales, entonces existe una subsucesi\'on $\left\{F_{n_{j}}\right\}$
de $\left\{F_{n}\right\}$ tal que
$\left\{F_{n_{j}}\left(x\right)\right\}$ converge para toda $K$
cuando $j\rightarrow\infty$.
\end{Lema}
\begin{proof}
%____________________________________________________________________________________________
Sean $\left\{F_{n}\left(x_{1}\right)\right\}_{n\geq 1}\subseteq
\left[-M-M\right]$ entonces  existe una subsucesi\'on
$\left\{F_{n}^{1}\left(x_{1}\right)\right\}_{n\geq 1}$
convergente, es decir, existe $a_{1}\in\rea$ tal que
$\left\{F_{n}^{1}\left(x_{1}\right)\right\}$ converge a $a_{1}$.
Para $\left\{F_{n}^{1}\left(x_{2}\right)\right\}$ una subsucesi\'on
convergente $\left\{F_{n}^{2}\left(x_{2}\right)\right\}$ a
$a_{2}\in\rea$ y como
$\left\{F_{n}^{2}\left(x_{2}\right)\right\}\subseteq
\left\{F_{n}^{1}\left(x_{1}\right)\right\}$ entonces se tiene que
$\left\{F_{n}^{2}\left(x_{1}\right)\right\}$ converge a $a_{1}$,
procediendo sucesivamente tenemos una sucesi\'on
$\left\{F_{n}^{1}\left(x_{1}\right)\right\}$,
$\left\{F_{n}^{2}\left(x_{2}\right)\right\}$,$\ldots,$,
$\left\{F_{n}^{k}\left(x_{k}\right)\right\}$ tal que
$\left\{F_{n}^{k}\left(x_{k}\right)\right\}$ converge a $a_{k}$
para toda $k=1,2,\ldots$. Consideremos la diagonal
$\left\{F_{n}^{n}\left(\cdot\right):n\geq 1\right\}$ y veamos que
para toda $x\in\DD$, se tiene que
$\left\{F_{n}^{n}\left(x\right)\right\}_{n\geq 1}$ converge. \\
Sea $k\geq 1$, $x_{k}\in\DD$, entonces
$\left\{F_{n}^{n}\left(x_{k}\right)\right\}_{n\geq k}$ es una
subsucesi\'on de $\left\{F_{n}^{k}\left(x\right)\right\}_{n\geq 1}$,
la cual ya se vio, que converge a $a_{k}$, por lo tanto
$\left\{F_{n}^{n}\left(x_{k}\right)\right\}_{n\geq k}$ converge a
$a_{k}$ cuando $n\rightarrow\infty$.
%____________________________________________________________________________________________
\end{proof}
\begin{Teo}{\bf (Teorema de Helly)}\label{Tma Helly}
Sean $\left\{F_{n}\right\}$ sucesi\'on de funciones de Distribuci\'on
Generalizadas tales que $\left|F_{n}\right|\leq M$ para toda $n$,
entonces $\left\{F_{n}\right\}$ contiene una subsucesi\'on
$\left\{F_{n_{k}}\right\}$ que converge d\'e}bilmente a $F$ funci\'on
de distribuci\'on generalizada.
\end{Teo}
\begin{proof}
%____________________________________________________________________________________________
Sea $\DD=\left\{x_{k}:k\geq 1\right\}$ subconjunto denso de
$\rea$, por el lema (\ref{Lema2}) existe una subsucesi\'on
$\left\{F_{n_{j}}\right\}_{j\geq 1}$ de
$\left\{F_{n}\right\}_{n\geq 1}$, tal que, para toda
$x_{k}\in\DD$,\[\lim_{j\rightarrow\infty}F_{n_{j}}\left(x_{k}\right)=a_{k}=G\left(x_{k}\right)\]
por el lema (\ref{Lema1}). Sea
$G\left(x\right)=\sup_{x_{k}<x}G\left(x\right)$ con $x\in\DD^{c}$
y $x_{k}\in\DD$, entonces se tiene que $G\left(x\right)$ es
acotada puesto que para toda $k$ se cumple
\[|\lim_{j\rightarrow\infty}F_{n_{j}}\left(x_{k}\right)|=|G\left(x_{k}\right)|\leq
M.\] Ahora veamos que $G$ es creciente:
\begin{description}
    \item{Caso 1:}\ Sean $x_{k},x_{l}\in\DD$ tales que
    $x_{k}<x_{l}$, entonces
    \[F_{n_{j}}\left(x_{k}\right)\leq F_{n_{j}}\left(x_{l}\right)\] para toda $j$, por lo tanto \[G\left(x_{k}\right)=\lim_{j\rightarrow\infty}F_{n_{j}}\left(x_{k}\right)\leq \lim_{j\rightarrow\infty}F_{n_{j}}\left(x_{l}\right)=G\left(x_{l}\right)\]
    \item{Caso 2:} Supongamos que $x\notin\DD$ y $x<x_{k}$, donde $x_{k}\in\DD$, sea $x_{l}\in\DD$ tal que $x_{l}<x$, entonces $x_{l}<x_{k}$ y por el {\em caso 1} se tiene que $G\left(x_{l}\right)\leq G\left(x_{k}\right)$ por lo tanto \[G\left(x\right)=\sup_{x_{l}<x}G\left(x_{l}\right)\leq G\left(x_{k}\right)\]
    \item{Caso 3:} Ahora supongamos $x_{k}\in\DD$, $x_{k}<x$, y $x\notin\DD$, entonces se tiene \[G\left(x\right)\geq \sup_{x_{l}<x,x_{l}\in\DD}G\left(x_{l}\right)\geq G\left(x_{k}\right)\] para $k\in\left\{l:l\geq 1\right\}$
    \item{Caso 4:} \ $x<y, x,y\notin\DD$, entonces \[\left\{x_{l}\in\DD|x_{l}<x\right\} \subset \left\{x_{l}\in\DD|x_{l}<y\right\} \] por lo tanto se tiene que \[G\left(x\right)= \sup_{x_{l}<x,x_{l}\in\DD}G\left(x_{l}\right) \leq \sup_{x_{l}<y,x_{l}\in\DD}G\left(x_{l}\right)=G\left(y\right)\]
\end{description}
por lo tanto $G$ es creciente.\\

Ahora, como \[\lim_{n\rightarrow\infty}F_{n_{j}}\left(x\right)=G\left(x\right)\]
para $x\in\DD$ conjunto Denso, se tiene que por el lema (\ref{Lema1}) $\left\{F_{n_{j}}\right\}$ converge a $G\left(x\right)$ para todo $x$ punto de continuidad de $G$. \\ Sea \[F\left(x\right)=\Bigg\{_{G\left(x^{+}\right)\ \ \ \ \ \ \ \ \ \ \ \ \ \ \ x\notin C_{G}}^{G\left(x\right),\ \ \ \ x \in C_{G}=\left\{x\in\rea:\ G {\it es\ continua\ en\ }x\right\}}\] entonces se tiene que $|F|\leq M$, es decir, $F$ es acotada.\\

Adem\'as, $F$ es creciente, puesto que si $x<y$, entonces dado que $G$ es creciente:
\begin{description}
\item{(i)} Si $x,y \in C_{G}$, entonces $F\left(x\right)=G\left(x\right)\leq G\left(y\right)\leq F\left(y\right)$
\item{(ii)} Si $x\in C_{G}, y\notin C_{G}$, entonces, $F\left(x\right)=G\left(x\right)\leq G\left(y\right)\leq G\left(y^{+}\right)=F\left(y\right)$
\item{(iii)} Si $x\notin C_{G}, y\in C_{G}$, entonces $F\left(x\right)=G\left(x^{+}\right)\leq G\left(y^{+}\right)=G\left(y\right)=F\left(y\right)$
\item{(iv)} Si $x\notin C_{G}, y\notin C_{G}$, entonces, $F\left(x\right)=G\left(x^{+}\right)\leq G\left(y^{+}\right)\leq F\left(y\right)$
\end{description}
por lo tanto se tiene que $F$ es creciente, adem\'as de ser continua por la derecha por como est\'a definida. Finalmente, $F_{n_{j}}$ converge d\'e}bilmente a $F$, puesto que $x$ es punto de continuidad de $F$ si y s\'olo si $x$ es punto de continuidad de $G$, pero $F_{n_{j}}$ converge a $G\left(x\right)=F\left(x\right)$, para todo $x$ punto de continuidad de $G$ y por lo tanto de $F$.
%____________________________________________________________________________________________
\end{proof}
\begin{Teo}{\bf (Teorema de Helly-Bray)} \label{Tma Helly-Bray}
Sea $g:\left[a,b\right]\rightarrow\rea$ funci\'on continua y acotada
y sea $\left\{F_{n}\right\}$ sucesi\'on de funciones de distribuci\'on
generalizadas tales que $\left|F_{n}\right|\leq M$ y tales que
$\left\{F_{n}\right\}$ converge d\'e}bilmente a $F$ donde $a,b\in
C_{F}=\left\{x\in\rea|F\ es\ continua \ en \ x\right\}$; entonces
\begin{eqnarray*}
\lim_{n\rightarrow\infty}\int_{a}^{b}g dF_{n}=\int_{a}^{b}g dF.
\end{eqnarray*}
\end{Teo}
\begin{proof}
%____________________________________________________________________________________________
A saber, $F$ es acotada y creciente con l{\'\i}mite por la derecha e izquierda. Sea $\mu_{n}$ medida asociada a $F_{n}$, definida por la relaci\'on:
\[\mu_{n}\left(\left(-\infty,x\right]\right)=F_{n}\left(x\right)\] y sea $\mu$ medida asociada a $F$, definida \[\mu\left(\left(-\infty,x\right]\right)=F\left(x\right)\]
para toda $x \in \rea$, y $\left\{\left(-\infty,x\right]\right\}$
clase determinante. Sea $a=x_{0,N}<x_{1,N}<\ldots<x_{N,N}=b$
partici\'on de $\left[a,b\right]$ tal que
\[\lim_{N\rightarrow \infty}A_{N}=\lim_{N\rightarrow \infty}\max_{1\leq j\leq k_{N}}{\left(x_{j,N}-x_{j-1,N}\right)}=0,\] entonces funci\'on $g$ puede aproximarse uniformemente sobre el intervalo$\left[a,b\right]$ por una sucesi\'on de funciones simples $\left\{g_{n}\right\}$ tales que
\[\lim_{N\rightarrow \infty}\sup_{a\leq x\leq b}|g_{N}\left(x\right)-g\left(x\right)|=0,\] es decir, \[g_{N}\left(x\right)=\sum_{k=1}^{K_{N}}g\left(x_{K,N}\right)\indora_{\left(x_{K-1,N},x_{K,N}\right]}\] y tal que  $|g_{N}|\leq |g||$ para toda $N$. La sucesi\'on $\left\{g_{n}\right\}$ es uniformemente acotada y por el {\em Teorema de Convergencia Dominada} se tiene
\[\int_{a}^{b}gdF_{n}=\int_{\left(a,b\right]gd\mu_{n}}=\lim_{N\rightarrow\infty}\int_{\left(a,b\right]}g_{N}d\mu_{n}\] para toda $n \geq 1$ y
\[\int_{a}^{b}gdF=\int_{\left(a,b\right]}gd\mu=\lim_{N\rightarrow\infty}\int_{\left(a,b\right]}g_{N}d\mu.\] Ahora veamos que para toda $n \geq 1$ se cumple que
\[\lim_{n\rightarrow \infty}\int_{\left(a,b\right]}g_{N}d\mu_{n}=\int_{\left(a,b\right]}g_{N}d\mu.\] Sean $a=x_{0,N}<x_{1,N}<\ldots<x_{N,N}=b$ puntos de continuidad de $F$; dado que la sucesi\'on $\left\{F_{n}\right\}$ converge d\'ebilmente a $F$
\[\lim_{n\rightarrow \infty}\mu_{n}\left(\left(x_{k-1,N},x_{k,N}\right]\right)=\lim_{n \rightarrow \infty} \left[F_{n}\left(x_{k,n}\right)-F_{n}\left(x_{k-1,n}\right)\right] \\
= F\left(x_{k,n}\right)-F\left(x_{k-1,n}\right)
= \mu\left(\left(x_{k-1,N},x_{k,N}\right]\right),\] para toda $k=1,2,\ldots,k_{N}$, por lo tanto
\[\lim_{n\rightarrow\infty}\int_{\left(a,b\right]}g_{N}d\mu_{n}
=\lim_{n\rightarrow\infty}\sum_{k=1}^{k_{N}}g\left(x_{k,N}\right)\mu_{n}\left(\left(x_{k-1,N},x_{k,N}\right]\right)=\sum_{k=1}^{k_{N}}g\left(x_{k,N}\right)\mu\left(\left(x_{k-1,N},x_{k,N}\right]\right)=\int_{\left(a,b\right]}g_{N}d\mu.\] Sea
$M_{N}=\sup_{a\leq x\leq b}|g_{N}\left(x\right)-g\left(x\right)|$, tal que $\lim_{N\rightarrow\infty}M_{N}=0$, entonces
\begin{eqnarray*}
\Bigg|\int_{a}^{b}gdF_{n}-\int_{a}^{b}gdF\Bigg|
=\Bigg|\int_{\left(a,b\right]}gd\mu_{n}-\int_{\left(a,b\right]}gd\mu\Bigg|\\ =\Bigg|\int_{\left(a,b\right]}gd\mu_{n}-\int_{\left(a,b\right]}g_{N}d\mu_{n}+ \int_{\left(a,b\right]}g_{N}d\mu_{n}+\int_{\left(a,b\right]}g_{N}d\mu
-\int_{\left(a,b\right]}g_{N}d\mu-\int_{\left(a,b\right]}gd\mu\Bigg|\\
=\Bigg| \left(\int_{\left(a,b\right]}\mid g-g_{N}\mid d\mu_{n}\right)+
\left(\int_{\left(a,b\right]}g_{N}d\mu_{n}-\int_{\left(a,b\right]}g_{N}d\mu\right)
+ \int_{\left(a,b\right]}\left( g-g_{N}\right) d\mu\Bigg|\\
 \leq \int_{\left(a,b\right]}\mid g-g_{N}\mid d\mu_{n}
+ \Bigg|\int_{\left(a,b\right]}g_{N}d\mu_{n}-\int_{\left(a,b\right]}g_{N}d\mu\Bigg|
+ \int_{\left(a,b\right]}\mid g-g_{N}\mid d\mu\\
\leq M_{n}\mu_{n}\left(\left(a,b\right]\right) + \Bigg|\int_{\left(a,b\right]}g_{N}d\mu_{n}-\int_{\left(a,b\right]}g_{N}d\mu\Bigg|
+ M_{n}\mu\left(\left(a,b\right]\right)
\end{eqnarray*} como $\lim_{N\rightarrow\infty}M_{N}=0$ y si hacemos $n\rightarrow\infty$, entonces,
\[\lim_{N,n\rightarrow\infty}\Bigg|\int_{a}^{b}gdF_{n}-\int_{a}^{b}gdF\Bigg|=0\]
%____________________________________________________________________________________________
\end{proof}
