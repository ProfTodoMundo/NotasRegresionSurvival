%____________________________________________________________________________
%%%\begin{frame}
\section{Queueing Theory at Markovian Level}


\subsection{General Death Birth Processes}




\subsection{General Death Birth Processes}


Consideremos un estado que comienza en el estado $x_{0}$ al tiempo
$0$, supongamos que el sistema permanece en $x_{0}$ hasta alg\'un
tiempo positivo $\tau_{1}$, tiempo en el que el sistema salta a un
nuevo estado $x_{1}\neq x_{0}$. Puede ocurrir que el sistema
permanezca en $x_{0}$ de manera indefinida, en este caso hacemos
$\tau_{1}=\infty$. Si $\tau_{1}$ es finito, el sistema
permanecer\'a en $x_{1}$ hasta $\tau_{2}$, y as\'i sucesivamente.
Sea
\begin{equation}
X\left(t\right)=\left\{\begin{array}{cc}
x_{0} & 0\leq t<\tau_{1}\\
x_{1} & \tau_{1}\leq t<\tau_{2}\\
x_{2} & \tau_{2}\leq t<\tau_{3}\\
\vdots &\\
\end{array}\right.
\end{equation}

A este proceso  se le llama {\em proceso de salto}. Si
\begin{equation}
lim_{n\rightarrow\infty}\tau_{n}=\left\{\begin{array}{cc}
<\infty & X_{t}\textrm{ explota}\\
=\infty & X_{t}\textrm{ no explota}\\
\end{array}\right.
\end{equation}

Un proceso puro de saltos es un proceso de saltos que satisface la
propiedad de Markov.
%\end{frame}
%____________________________________________________________________________
%%%\begin{frame}
\begin{Prop}
Un proceso de saltos es Markoviano si y s\'olo si todos los
estados no absorbentes $x$ son tales que
\begin{eqnarray*}
P_{x}\left(\tau_{1}>t+s|\tau_{1}>s\right)=P_{x}\left(\tau_{1}>t\right)
\end{eqnarray*}
para $s,t\geq0$, equivalentemente

\begin{equation}\label{Eq.5}
\frac{1-F_{x}\left(t+s\right)}{1-F_{x}\left(s\right)}=1-F_{x}\left(t\right).
\end{equation}

\end{Prop}

\begin{Note}
Una distribuci\'on $F_{x}$ satisface la ecuaci\'on (\ref{Eq.5}) si
y s\'olo si es una funci\'on de distribuci\'on exponencial para
todos los estados no absorbentes $x$.
\end{Note}

%\end{frame}
%____________________________________________________________________________
%%%\begin{frame}
Por un proceso de nacimiento y muerte se entiende un proceso de
Markov de Saltos, $\left\{X_{t}\right\}_{t\geq0}$ en $E=\nat$ tal
que del estado $n$ s\'olo se puede mover a $n-1$ o $n+1$, es
decir, la matriz intensidad es de la forma:

\begin{equation}
\Lambda=\left(\begin{array}{ccccc}
-\beta_{0}&\beta_{0} & 0 & 0 & \ldots\\
\delta_{1}&-\beta_{1}-\delta_{1} & \beta_{1}&0&\ldots\\
0&\delta_{2}&-\beta_{2}-\delta_{2} & \beta_{2}&\ldots\\
\vdots & & & \ddots &
\end{array}\right)
\end{equation}

donde $\beta_{n}$ son las probabilidades de nacimiento y
$\delta_{n}$ las probabilidades de muerte.

La matriz de transici\'on es
\begin{equation}
Q=\left(\begin{array}{ccccc}
0& 1 & 0 & 0 & \ldots\\
q_{1}&0 & p_{1}&0&\ldots\\
0&q_{2}&0& p_{2}&\ldots\\
\vdots & & & \ddots &
\end{array}\right)
\end{equation}
con $p_{n}=\frac{\beta_{n}}{\beta_{n}+\delta_{n}}$ y
$q_{n}=\frac{\delta_{n}}{\beta_{n}+\delta_{n}}$
%\end{frame}
%____________________________________________________________________________
%%%\begin{frame}
\begin{Prop}
La recurrencia de un Proceso Markoviano de Saltos
$\left\{X_{t}\right\}_{t\geq0}$ con espacio de estados numerable,
o equivalentemente de la cadena encajada $\left\{Y_{n}\right\}$ es
equivalente a
\begin{equation}\label{Eq.2.1}
\sum_{n=1}^{\infty}\frac{\delta_{1}\cdots\delta_{n}}{\beta_{1}\cdots\beta_{n}}=\sum_{n=1}^{\infty}\frac{q_{1}\cdots
q_{n}}{p_{1}\cdots p_{n}}=\infty
\end{equation}
\end{Prop}

\begin{Lem}
Independientemente de la recurrencia o transitoriedad de la
cadena, hay una y s\'olo una, salvo m\'ultiplos, soluci\'on $\nu$
a $\nu\Lambda=0$, dada por
\begin{equation}\label{Eq.2.2}
\nu_{n}=\frac{\beta_{0}\cdots\beta_{n-1}}{\delta_{1}\cdots\delta_{n}}\nu_{0}
\end{equation}
\end{Lem}

%\end{frame}
%____________________________________________________________________________
%%%\begin{frame}
\begin{Col}\label{Corolario2.3}
En el caso recurrente, la medida estacionaria $\mu$ para
$\left\{Y_{n}\right\}$ est\'a dada por
\begin{equation}\label{Eq.2.3}
\mu_{n}=\frac{p_{1}\cdots p_{n-1}}{q_{1}\cdots q_{n}}\mu_{0}
\end{equation}
para $n=1,2,\ldots$
\end{Col}

\begin{Def}
Una medida $\nu$ es estacionaria si $0\leq\nu_{j}<\infty$ y para
toda $t$ se cumple que $\nu P^{t}=nu$.
\end{Def}


\begin{Def}
Un proceso irreducible recurrente con medida estacionaria con masa
finita es llamado erg\'odico.
\end{Def}

\begin{Teo}\label{Teo4.3}
Un Proceso de Saltos de Markov irreducible no explosivo es
erg\'odico si y s\'olo si uno puede encontrar una soluci\'on
$\pi$ de probabilidad, $|\pi|=1$, $0\leq\pi_{j}\leq1$ para
$\nu\Lambda=0$. En este caso $\pi$ es la distribuci\'on
estacionaria.
\end{Teo}
%\end{frame}
%____________________________________________________________________________
%%%\begin{frame}

\begin{Col}\label{Corolario2.4}
$\left\{X_{t}\right\}_{t\geq0}$ es erg\'odica si y s\'olo si
(\ref{Eq.2.1}) se cumple y $S<\infty$, en cuyo caso la
distribuci\'on estacionaria $\pi$ est\'a dada por

\begin{equation}\label{Eq.2.4}
\pi_{0}=\frac{1}{S}\textrm{,
}\pi_{n}=\frac{1}{S}\frac{\beta_{0}\cdots\beta_{n-1}}{\delta_{1}\cdots\delta_{n}}\textrm{,
}n=1,2,\ldots
\end{equation}
\end{Col}


