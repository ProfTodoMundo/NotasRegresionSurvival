%_____________________________________________________________________________________
%
\section{Procesos de Markov de Saltos}
%_____________________________________________________________________________________
%


Consideremos un estado que comienza en el estado $x_{0}$ al tiempo $0$, supongamos que el sistema permanece en $x_{0}$ hasta alg\'un tiempo positivo $\tau_{1}$, tiempo en el que el sistema salta a un nuevo estado $x_{1}\neq x_{0}$. Puede ocurrir que el sistema permanezca en $x_{0}$ de manera indefinida, en este caso hacemos $\tau_{1}=\infty$. Si $\tau_{1}$ es finito, el sistema permanecer\'a en $x_{1}$ hasta $\tau_{2}$, y as\'i sucesivamente.
Sea
\begin{equation}
X\left(t\right)=\left\{\begin{array}{cc}
x_{0} & 0\leq t<\tau_{1}\\
x_{1} & \tau_{1}\leq t<\tau_{2}\\
x_{2} & \tau_{2}\leq t<\tau_{3}\\
\vdots &\\
\end{array}\right.
\end{equation}

A este proceso  se le llama {\em proceso de salto}. Si
\begin{equation}
lim_{n\rightarrow\infty}\tau_{n}=\left\{\begin{array}{cc}
<\infty & X_{t}\textrm{ explota}\\
=\infty & X_{t}\textrm{ no explota}\\
\end{array}\right.
\end{equation}

Un proceso puro de saltos es un proceso de saltos que satisface la propiedad de Markov.

\begin{Prop}
Un proceso de saltos es Markoviano si y s\'olo si todos los estados no absorbentes $x$ son tales que
\begin{eqnarray*}
P_{x}\left(\tau_{1}>t+s|\tau_{1}>s\right)=P_{x}\left(\tau_{1}>t\right)
\end{eqnarray*}
para $s,t\geq0$, equivalentemente

\begin{equation}\label{Eq.5}
\frac{1-F_{x}\left(t+s\right)}{1-F_{x}\left(s\right)}=1-F_{x}\left(t\right).
\end{equation}
\end{Prop}

\begin{Note}
Una distribuci\'on $F_{x}$ satisface la ecuaci\'on (\ref{Eq.5}) si y s\'olo si es una funci\'on de distribuci\'on exponencial para todos los estados no absorbentes $x$.
\end{Note}

Por un proceso de nacimiento y muerte se entiende un proceso de Markov de Saltos, $\left\{X_{t}\right\}_{t\geq0}$ en $E=\nat$ tal que del estado $n$ s\'olo se puede mover a $n-1$ o $n+1$, es decir, la matriz intensidad es de la forma:

\begin{equation}
\Lambda=\left(\begin{array}{ccccc}
-\beta_{0}&\beta_{0} & 0 & 0 & \ldots\\
\delta_{1}&-\beta_{1}-\delta_{1} & \beta_{1}&0&\ldots\\
0&\delta_{2}&-\beta_{2}-\delta_{2} & \beta_{2}&\ldots\\
\vdots & & & \ddots &
\end{array}\right)
\end{equation}

donde $\beta_{n}$ son las probabilidades de nacimiento y
$\delta_{n}$ las probabilidades de muerte.

La matriz de transici\'on es
\begin{equation}
Q=\left(\begin{array}{ccccc}
0& 1 & 0 & 0 & \ldots\\
q_{1}&0 & p_{1}&0&\ldots\\
0&q_{2}&0& p_{2}&\ldots\\
\vdots & & & \ddots &
\end{array}\right)
\end{equation}
con $p_{n}=\frac{\beta_{n}}{\beta_{n}+\delta_{n}}$ y
$q_{n}=\frac{\delta_{n}}{\beta_{n}+\delta_{n}}$

\begin{Prop}
La recurrencia de un Proceso Markoviano de Saltos
$\left\{X_{t}\right\}_{t\geq0}$ con espacio de estados numerable, o equivalentemente de la cadena encajada $\left\{Y_{n}\right\}$ es equivalente a
\begin{equation}\label{Eq.2.1}
\sum_{n=1}^{\infty}\frac{\delta_{1}\cdots\delta_{n}}{\beta_{1}\cdots\beta_{n}}=\sum_{n=1}^{\infty}\frac{q_{1}\cdots
q_{n}}{p_{1}\cdots p_{n}}=\infty
\end{equation}
\end{Prop}

\begin{Lem}
Independientemente de la recurrencia o transitoriedad de la cadena, hay una y s\'olo una, salvo m\'ultiplos, soluci\'on $\nu$
a $\nu\Lambda=0$, dada por
\begin{equation}\label{Eq.2.2}
\nu_{n}=\frac{\beta_{0}\cdots\beta_{n-1}}{\delta_{1}\cdots\delta_{n}}\nu_{0}
\end{equation}
\end{Lem}

\begin{Cor}\label{Corolario2.3}
En el caso recurrente, la medida estacionaria $\mu$ para
$\left\{Y_{n}\right\}$ est\'a dada por
\begin{equation}\label{Eq.2.3}
\mu_{n}=\frac{p_{1}\cdots p_{n-1}}{q_{1}\cdots q_{n}}\mu_{0}
\end{equation}
para $n=1,2,\ldots$
\end{Cor}

\begin{Def}
Una medida $\nu$ es estacionaria si $0\leq\nu_{j}<\infty$ y para toda $t$ se cumple que $\nu P^{t}=nu$.
\end{Def}


\begin{Def}
Un proceso irreducible recurrente con medida estacionaria con masa finita es llamado erg\'odico.
\end{Def}

\begin{Teo}\label{Teo4.3}
Un Proceso de Saltos de Markov irreducible no explosivo es erg\'odico si y s\'olo si uno puede encontrar una soluci\'on $\pi$ de probabilidad, $|\pi|=1$, $0\leq\pi_{j}\leq1$ para $\nu\Lambda=0$. En este caso $\pi$ es la distribuci\'on estacionaria.
\end{Teo}
\begin{Cor}\label{Corolario2.4}
$\left\{X_{t}\right\}_{t\geq0}$ es erg\'odica si y s\'olo si (\ref{Eq.2.1}) se cumple y $S<\infty$, en cuyo caso la distribuci\'on estacionaria $\pi$ est\'a dada por

\begin{equation}\label{Eq.2.4}
\pi_{0}=\frac{1}{S}\textrm{,
}\pi_{n}=\frac{1}{S}\frac{\beta_{0}\cdots\beta_{n-1}}{\delta_{1}\cdots\delta_{n}}\textrm{,
}n=1,2,\ldots
\end{equation}
\end{Cor}


Sea $E$ espacio discreto de estados, finito o numerable, y sea $\left\{X_{t}\right\}$ un proceso de Markov con espacio de estados $E$. Una medida $\mu$ en $E$ definida por sus probabilidades puntuales $\mu_{i}$, escribimos $p_{ij}^{t}=P^{t}\left(i,\left\{j\right\}\right)=P_{i}\left(X_{t}=j\right)$.

El monto del tiempo gastado en cada estado es positivo, de modo tal que las trayectorias muestrales son constantes por partes. Para un proceso de saltos denotamos por los tiempos de saltos a $S_{0}=0<S_{1}<S_{2}\cdots$, los tiempos entre saltos consecutivos $T_{n}=S_{n+1}-S_{n}$ y la secuencia de estados visitados por $Y_{0},Y_{1},\ldots$, as\'i las trayectorias muestrales son constantes entre $S_{n}$ consecutivos, continua por la derecha, es decir, $X_{S_{n}}=Y_{n}$. 

La descripci\'on de un modelo pr\'actico est\'a dado usualmente en t\'erminos de las intensidades $\lambda\left(i\right)$ y las probabilidades de salto $q_{ij}$ m\'as que en t\'erminos de la matriz de transici\'on $P^{t}$. Sup\'ongase de ahora en adelante que $q_{ii}=0$ cuando $\lambda\left(i\right)>0$

\begin{Teo}
Cualquier Proceso de Markov de Saltos satisface la Propiedad
Fuerte de Markov
\end{Teo}

\begin{Def}
Una medida $v\neq0$ es estacionaria si $0\leq v_{j}<\infty$, $vP^{t}=v$ para toda $t$.
\end{Def}

\begin{Teo}\label{Teo.4.2}
Supongamos que $\left\{X_{t}\right\}$ es irreducible recurrente en $E$. Entonces existe una y s\'olo una, salvo m\'ultiplos, medida estacionaria $v$. Esta $v$ tiene la propiedad de que $0<v_{j}<\infty$ para todo $j$ y puede encontrarse en cualquiera de las siguientes formas

\begin{itemize}
\item[i)] Para alg\'un estado $i$, fijo pero arbitrario, $v_{j}$ es el tiempo esperado utilizado en $j$ entre dos llegadas consecutivas al estado $i$;
\begin{equation}\label{Eq.4.2}
v_{j}=\esp_{i}\int_{0}^{w\left(i\right)}\indora\left(X_{t}=j\right)dt
\end{equation}
con $w\left(i\right)=\inf\left\{t>0:X_{t}=i,X_{t^{-}}=\lim_{s\uparrow t}X_{s}\neq i\right\}$. 
\item[ii)]
$v_{j}=\frac{\mu_{j}}{\lambda\left(j\right)}$, donde $\mu$ es estacionaria para $\left\{Y_{n}\right\}$. \item[iii)] como
soluci\'on de $v\Lambda=0$.
\end{itemize}
\end{Teo}

\begin{Def}
Un proceso irreducible recurrente con medida estacionaria de masa
finita es llamado erg\'odico.
\end{Def}

\begin{Teo}\label{Teo.4.3}
Un proceso de Markov de saltos irreducible no explosivo es erg\'odico si y s\'olo si se puede encontrar una soluci\'on, de probabilidad, $\pi$, con $|\pi|=1$ y $0\leq\pi_{j}\leq1$, a $\pi\Lambda=0$. En este caso $\pi$ es la distribuci\'on estacionaria.
\end{Teo}

\begin{Cor}\label{Cor.4.4}
Una condici\'on suficiente para la ergodicidad de un proceso irreducible es la existencia de una probabilidad $\pi$ que resuelva el sistema $\pi\Lambda=0$ y que adem\'as tenga la propiedad de que $\sum\pi_{j}\lambda\left(j\right)$.
\end{Cor}

%_____________________________________________________________________________________
%
\section{Matriz Intensidad}
%_____________________________________________________________________________________
%


\begin{Def}
La matriz intensidad
$\Lambda=\left(\lambda\left(i,j\right)\right)_{i,j\in E}$ del proceso de saltos $\left\{X_{t}\right\}_{t\geq0}$ est\'a dada por
\begin{eqnarray*}
\lambda\left(i,j\right)&=&\lambda\left(i\right)q_{i,j}\textrm{,    }j\neq i\\
\lambda\left(i,i\right)&=&-\lambda\left(i\right)
\end{eqnarray*}
\end{Def}


\begin{Prop}\label{Prop.3.1}
Una matriz $E\times E$, $\Lambda$ es la matriz de intensidad de un proceso markoviano de saltos $\left\{X_{t}\right\}_{t\geq0}$ si y s\'olo si
\begin{eqnarray*}
\lambda\left(i,i\right)\leq0\textrm{, }\lambda\left(i,j\right)\textrm{,   }i\neq j\textrm{,  }\sum_{j\in E}\lambda\left(i,j\right)=0.
\end{eqnarray*}
Adem\'as, $\Lambda$ est\'a en correspondencia uno a uno con la
distribuci\'on del proceso minimal dado por el teorema 3.1.
\end{Prop}


Para el caso particular de la Cola $M/M/1$, la matr\'iz de itensidad est\'a dada por
\begin{eqnarray*}
\Lambda=\left[\begin{array}{cccccc}
-\beta & \beta & 0 &0 &0& \cdots\\
\delta & -\beta-\delta & \beta & 0 & 0 &\cdots\\
0 & \delta & -\beta-\delta & \beta & 0 &\cdots\\
\vdots & & & & & \ddots\\
\end{array}\right]
\end{eqnarray*}


%____________________________________________________________________________
\section{Medidas Estacionarias}
%____________________________________________________________________________
%


\begin{Def}
Una medida $v\neq0$ es estacionaria si $0\leq v_{j}<\infty$, $vP^{t}=v$ para toda $t$.
\end{Def}

\begin{Teo}\label{Teo.4.2}
Supongamos que $\left\{X_{t}\right\}$ es irreducible recurrente en $E$. Entonces existe una y s\'olo una, salvo m\'ultiplos, medida estacionaria $v$. Esta $v$ tiene la propiedad de que $0<v_{j}<\infty$ para todo $j$ y puede encontrarse en cualquiera de las siguientes formas

\begin{itemize}
\item[i)] Para alg\'un estado $i$, fijo pero arbitrario, $v_{j}$ es el tiempo esperado utilizado en $j$ entre dos llegadas consecutivas al estado $i$;
\begin{equation}\label{Eq.4.2}
v_{j}=\esp_{i}\int_{0}^{w\left(i\right)}\indora\left(X_{t}=j\right)dt
\end{equation}
con $w\left(i\right)=\inf\left\{t>0:X_{t}=i,X_{t^{-}}=\lim_{s\uparrow t}X_{s}\neq i\right\}$. 
\item[ii)]
$v_{j}=\frac{\mu_{j}}{\lambda\left(j\right)}$, donde $\mu$ es estacionaria para $\left\{Y_{n}\right\}$. 
\item[iii)] como soluci\'on de $v\Lambda=0$.
\end{itemize}
\end{Teo}


%____________________________________________________________________________
\section{Criterios de Ergodicidad}
%____________________________________________________________________________
%

\begin{Def}
Un proceso irreducible recurrente con medida estacionaria de masa finita es llamado erg\'odico.
\end{Def}

\begin{Teo}\label{Teo.4.3}
Un proceso de Markov de saltos irreducible no explosivo es erg\'odico si y s\'olo si se puede encontrar una soluci\'on, de probabilidad, $\pi$, con $|\pi|=1$ y $0\leq\pi_{j}\leq1$, a $\pi\Lambda=0$. En este caso $\pi$ es la distribuci\'on estacionaria.
\end{Teo}

\begin{Cor}\label{Cor.4.4}
Una condici\'on suficiente para la ergodicidad de un proceso irreducible es la existencia de una probabilidad $\pi$ que resuelva el sistema $\pi\Lambda=0$ y que adem\'as tenga la propiedad de que $\sum\pi_{j}\lambda\left(j\right)<\infty$.
\end{Cor}

\begin{Prop}
Si el proceso es erg\'odico, entonces existe una versi\'on estrictamente estacionaria
$\left\{X_{t}\right\}_{-\infty<t<\infty}$con doble tiempo
infinito.
\end{Prop}

\begin{Teo}
Si $\left\{X_{t}\right\}$ es erg\'odico y $\pi$ es la distribuci\'on estacionaria, entonces para todo $i,j$, $p_{ij}^{t}\rightarrow\pi_{j}$ cuando $t\rightarrow\infty$.
\end{Teo}

\begin{Cor}
Si $\left\{X_{t}\right\}$ es irreducible recurente pero no erg\'odica, es decir $|v|=\infty$, entonces $p_{ij}^{t}\rightarrow0$ para todo $i,j\in E$.
\end{Cor}

\begin{Cor}
Para cualquier proceso Markoviano de Saltos minimal, irreducible o
no, los l\'imites $li_{t\rightarrow\infty}p_{ij}^{t}$ existe.
\end{Cor}



