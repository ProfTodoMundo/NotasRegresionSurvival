



%_____________________________________________________________________________________
%
\section{Notaci\'on Kendall-Lee}
%_____________________________________________________________________________________
%

A partir de este momento se har\'an las siguientes consideraciones:
\begin{itemize}
\item[a) ]Si $t_{n}$ es el tiempo aleatorio en el que llega al sistema el $n$-\'esimo cliente, para $n=1,2,\ldots$, $t_{0}=0$ y $t_{0}<t_{1}<\cdots$ se definen los tiempos entre arribos $\tau_{n}=t_{n}-t_{n-1}$ para $n=1,2,\ldots$, variables aleatorias independientes e id\'enticamente distribuidas.

\item[b) ]Los tiempos entre arribos tienen un valor medio $E\left(\tau\right)$ finito y positivo $\frac{1}{\beta}$, es decir, $\beta$ se puede ver como la tasa o intensidad promedio de arribos al sistema por unidad de tiempo.
\item[c) ]  Adem\'as se supondr\'a que los servidores son identicos y si $s$ denota la variable aleatoria que describe el tiempo de servicio, entonces $E\left(s\right)=\frac{1}{\delta}$, $\delta$ es la tasa promedio de servicio por servidor.
\end{itemize}


La notaci\'on de Kendall-Lee es una forma abreviada de describir un sistema de espera con las siguientes componentes:
\begin{itemize}
\item[a)] {\em\bf Fuente}: Poblaci\'on de clientes potenciales del sistema, esta puede ser finita o infinita. 
\item[b)] {\em\bf Proceso de Arribos}: Proceso determinado por la funci\'on de distribuci\'on $A\left(t\right)=P\left\{\tau\leq t\right\}$ de los tiempos entre arribos.
\end{itemize}

Adem\'as tenemos las siguientes igualdades
\begin{equation}\label{Eq.0.1}
N\left(t\right)=N_{q}\left(t\right)+N_{s}\left(s\right)
\end{equation}
donde
\begin{itemize}
\item $N\left(t\right)$ es el n\'umero de clientes en el sistema al tiempo $t$. 
\item $N_{q}\left(t\right)$ es el n\'umero de cliente en la cola al tiempo $t$.
\item $N_{s}\left(t\right)$ es el n\'umero de clientes recibiendo servicio en el tiempo $t$.
\end{itemize}

Bajo la hip\'otesis de estacionareidad, es decir, las caracter\'isticas de funcionamiento del sistema se han estabilizado en valores independientes del tiempo, entonces
\begin{equation}
N=N_{q}+N_{s}.
\end{equation}

Los valores medios de las cantidades anteriores se escriben como $L=E\left(N\right)$, $L_{q}=E\left(N_{q}\right)$ y $L_{s}=E\left(N_{s}\right)$, entonces de la ecuaci\'on \ref{Eq.0.1} se obtiene

\begin{equation}
L=L_{q}+L_{s}
\end{equation}
Si $q$ es el tiempo que pasa un cliente en la cola antes de recibir servicio, y W es el tiempo total que un cliente pasa en el sistema, entonces \[w=q+s\] por lo tanto \[W=W_{q}+W_{s},\] donde $W=E\left(w\right)$, $W_{q}=E\left(q\right)$ y $W_{s}=E\left(s\right)=\frac{1}{\delta}$.

La intensidad de tr\'afico se define como
\begin{equation}
\rho=\frac{E\left(s\right)}{E\left(\tau\right)}=\frac{\beta}{\delta}.
\end{equation}

La utilizaci\'on por servidor es
\begin{equation}
u=\frac{\rho}{c}=\frac{\beta}{c\delta}.
\end{equation}
donde $c$ es el n\'umero de servidores.

Esta notaci\'on es una forma abreviada de describir un sistema de espera con componentes dados a continuaci\'on, la notaci\'on es

\begin{equation}\label{Notacion.K.L.}
A/S/c/K/F/d
\end{equation}

Cada una de las letras describe:

\begin{itemize}
\item $A$ es la distribuci\'on de los tiempos entre arribos.
\item $S$ es la distribuci\'on del tiempo de servicio.
\item $c$ es el n\'umero de servidores.
\item $K$ es la capacidad del sistema.
\item $F$ es el n\'umero de individuos en la fuente.
\item $d$ es la disciplina del servicio
\end{itemize}

Usualmente se acostumbra suponer que $K=\infty$, $F=\infty$ y $d=FIFO$, es decir, First In First Out. Las distribuciones usuales para $A$ y $B$ son:

\begin{itemize}
\item $GI$ para la distribuci\'on general de los tiempos entre arribos.
\item $G$ distribuci\'on general del tiempo de servicio.
\item $M$ Distribuci\'on exponencial para $A$ o $S$.
\item $E_{K}$ Distribuci\'on Erlang-$K$, para $A$ o $S$.
\item $D$ tiempos entre arribos o de servicio constantes, es decir, deterministicos.
\end{itemize}

